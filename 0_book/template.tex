% 中文书籍LaTeX模板
% 使用xelatex编译

\documentclass[12pt,a4paper,twoside]{ctexbook}

% 页面设置
% 纸张设置配置文件
% 用于定义书籍的页面尺寸和边距

\usepackage[a4paper,twoside]{geometry}
\geometry{
	left=25mm,
	right=20mm,
	top=25mm,
	bottom=25.4mm,
	headsep=1cm, 
    footskip=1cm,
	bindingoffset=10mm
}

% 字体设置
\usepackage{xeCJK}
\usepackage{fontspec}
\usepackage{microtype}

% 设置中文字体
\setCJKmainfont{SimSun}[  % 正文宋体
    BoldFont=SimHei,        % 粗体黑体
    ItalicFont=KaiTi        % 斜体楷体
]
\setCJKsansfont{SimHei}    % 无衬线字体黑体
\setCJKmonofont{SimSun}    % 等宽字体宋体
\setCJKfamilyfont{kai}[    % 楷体
    BoldFont=KaiTi
]{KaiTi}
\setCJKfamilyfont{fs}[     % 仿宋
    BoldFont=FangSong
]{FangSong}

% 常用字体命令
\newcommand{\song}{\CJKfamily{zhsong}}
\newcommand{\hei}{\CJKfamily{zhhei}}
\newcommand{\kai}{\CJKfamily{kai}}
\newcommand{\fs}{\CJKfamily{fs}}

% 标题格式设置
\ctexset{
    part/name={第,卷},
    part/number={\chinese{part}},
    chapter/name={第,章},
    chapter/number={\chinese{chapter}},
    section/name={第,节},
    section/number={\arabic{section}},
    subsection/number={\arabic{section}.\arabic{subsection}},
    chapter/format={\centering\hei\zihao{2}},
    section/format={\hei\zihao{4}},
    subsection/format={\hei\zihao{5}}
}

% 目录设置
\usepackage{titletoc}
\titlecontents{chapter}[0pt]{\vspace{10pt}\bfseries\zihao{-4}}{\contentspush{\thecontentslabel\hspace{1em}}}{}{\titlerule*[8pt]{.}\contentspage}
\titlecontents{section}[2.5em]{\vspace{5pt}\zihao{5}}{\contentspush{\thecontentslabel\hspace{1em}}}{}{\titlerule*[8pt]{.}\contentspage}
\titlecontents{subsection}[5em]{\vspace{3pt}\zihao{5}}{\contentspush{\thecontentslabel\hspace{1em}}}{}{\titlerule*[8pt]{.}\contentspage}

% 页眉页脚设置
\usepackage{fancyhdr}
\pagestyle{fancy}
\fancyhf{}
\fancyhead[LE,RO]{\zihao{5}\thepage}
\fancyhead[LO]{\zihao{5}\leftmark}
\fancyhead[RE]{\zihao{5}\rightmark}
\renewcommand{\chaptermark}[1]{\markboth{\chaptername\ \thechapter\ #1}{}}
\renewcommand{\sectionmark}[1]{\markright{\thesection\ #1}}
\fancyfoot[C]{\zihao{5} \thepage}
\renewcommand{\headrulewidth}{0.4pt}
\renewcommand{\footrulewidth}{0pt}

% 插图设置
\usepackage{graphicx}
\usepackage{float}
\usepackage{subfigure}
\graphicspath{{images/}}
\floatstyle{plaintop}
\restylefloat{figure}

% 表格设置
\usepackage{tabularx}
\usepackage{booktabs}
\usepackage{longtable}

% 数学公式设置
\usepackage{amsmath, amssymb, amsthm}
\usepackage{mathrsfs}

% 定理环境
\newtheorem{theorem}{定理}[chapter]
\newtheorem{definition}{定义}[chapter]
\newtheorem{lemma}{引理}[chapter]
\newtheorem{corollary}{推论}[chapter]
\newtheorem{example}{例}[chapter]

% 目录、摘要等设置
\usepackage{makeidx}
\makeindex

% 摘要设置
\newenvironment{abstract}{
    \cleardoublepage
    \thispagestyle{empty}
    \begin{center}
        \textbf{\zihao{1} 摘要}
    \end{center}
    \vspace{1cm}
    \itshape
}{
    \normalfont
}

% 关键词设置
\newcommand{\keywords}[1]{
    \vspace{1cm}
    \noindent\textbf{关键词:} #1
}

% 引用设置
\usepackage{hyperref}
\hypersetup{
    colorlinks=true,
    linkcolor=blue,
    citecolor=blue,
    urlcolor=blue,
    pdftitle={中文书籍模板},
    pdfauthor={作者名},
    pdfsubject={主题},
    pdfkeywords={关键词1, 关键词2}
}

% 目录深度
\setcounter{tocdepth}{3}
\setcounter{secnumdepth}{3}

% 标题页设置
\usepackage{titling}

% 封面信息
\title{\hei\zihao{0} 中文书籍模板}
\author{\song\zihao{2} 作者名}
\date{\song\zihao{4} \today}

\begin{document}

% 封面
\begin{titlepage}
    \begin{center}
        \vspace*{6cm}
        \hei\zihao{0} 中文书籍模板
        \vspace*{3cm}
        \song\zihao{2} 作者名
        \vspace*{3cm}
        \song\zihao{4} \today
    \end{center}
\end{titlepage}

% 版权页
\newpage
\thispagestyle{empty}
\begin{center}
    \vspace*{8cm}
    \song\zihao{5} 版权所有 \textcopyright\ 2026 作者名
    \vspace*{1cm}
    \song\zihao{5} 出版社名称
\end{center}

% 摘要
\begin{abstract}
    这是一本中文书籍的LaTeX模板,使用xelatex编译,支持完整的中文排版功能,包括章节设置、插图、表格、数学公式、引用等。
    
    \keywords{中文书籍 \quad LaTeX模板 \quad xelatex}
\end{abstract}

% 目录
\newpage
\tableofcontents

% 正文开始
\mainmatter

% 示例章节
\chapter{第一章 引言}
这是第一章的内容,使用宋体排版。\par
\hei 这是黑体文字。\par
\kai 这是楷体文字。\par
\fs 这是仿宋文字。

\section{第一节 研究背景}
这是第一节的内容。

\subsection{第一小节 问题提出}
这是第一小节的内容。

\begin{figure}[H]
    \centering
    \includegraphics[width=0.8\linewidth]{example-image}
    \caption{示例图片}
    \label{fig:example}
\end{figure}

\begin{table}[H]
    \centering
    \caption{示例表格}
    \label{tab:example}
    \begin{tabular}{ccc}
        \toprule
        列1 & 列2 & 列3 \\ 
        \midrule
        数据1 & 数据2 & 数据3 \\ 
        数据4 & 数据5 & 数据6 \\ 
        \bottomrule
    \end{tabular}
\end{table}

\begin{theorem}[勾股定理]
    直角三角形的两条直角边的平方和等于斜边的平方。
    \[a^2 + b^2 = c^2\]
\end{theorem}

\chapter{第二章 相关工作}
这是第二章的内容。

% 参考文献
\backmatter

\begin{thebibliography}{99}
    \bibitem{ref1} 作者1, 作者2. 书名[M]. 出版社, 出版年份.
    \bibitem{ref2} 作者3. 文章标题[J]. 期刊名称, 卷号(期号): 页码范围, 出版年份.
    \bibitem{ref3} 作者4. 网页标题[EB/OL]. URL, 访问日期.
\end{thebibliography}

% 索引
\printindex

\end{document}
