% Hacker文化简史
% Hacker文化简史.tex

\documentclass[a4paper,12pt,UTF8,twoside]{ctexbook}

% 设置纸张信息。
\RequirePackage[a4paper]{geometry}
\geometry{
	%textwidth=138mm,
	%textheight=215mm,
	%left=27mm,
	%right=27mm,
	%top=25.4mm, 
	%bottom=25.4mm,
	%headheight=2.17cm,
	%headsep=4mm,
	%footskip=12mm,
	%heightrounded,
	inner=1in,
	outer=1.25in
}

% 设置字体,并解决显示难检字问题。
\xeCJKsetup{AutoFallBack=true}
\setCJKmainfont{SimSun}[BoldFont=SimHei, ItalicFont=KaiTi, FallBack=SimSun-ExtB]

% 目录 chapter 级别加点(.)。
\usepackage{titletoc}
\titlecontents{chapter}[0pt]{\vspace{3mm}\bf\addvspace{2pt}\filright}{\contentspush{\thecontentslabel\hspace{0.8em}}}{}{\titlerule*[8pt]{.}\contentspage}

% 设置 part 和 chapter 标题格式。
\ctexset{
	chapter/name={},
	chapter/number={}
}

% 设置古文原文格式。
\newenvironment{yuanwen}{\bfseries\zihao{4}}

% 设置署名格式。
\newenvironment{shuming}{\hfill\bfseries\zihao{4}}

\title{\heiti\zihao{0} Hacker文化简史(A Brief History of  Hackerdom)}
\author{Eric S. Raymond}
\date{}

\begin{document}

\maketitle
\tableofcontents

\frontmatter
\chapter{前言}

英文原文请看

http://www.tuxedo.org/~esr/writings/cathedral-bazaar/hacker-history/

本篇原作者为Eric S. Raymond,他是一位大哥级的 Hacker,写了很多自由软件,知名著作有Jargon File等,近年来发表“大教堂与集市”论文为Opensource software努力,Netscape 愿意公开Navigator的原始码,与这篇文章有很大的关系。

\mainmatter

\chapter{序曲: Real Programmer}

故事一开始,我要介绍的是所谓的 Real Programmer 。

他们从不自称是 Real Programmer 、 Hacker 或任何特殊的称号;Real Programmer 这个名词是在1980年代才出现,但早自1945年起,电脑科学便不断地吸引世界上头脑最顶尖、想像力最丰富的人投入其中。从Eckert \& Mauchly发明ENIAC 后,便不断有狂热的 programmer 投入其中,他们以撰写软件与玩弄各种程式设计技巧为乐,逐渐形成具有自我意识的一套科技文化。当时这批 Real Programmers 主要来自工程界与物理界,他们戴著厚厚的眼镜, 穿聚酯纤维T恤与纯白袜子,用机器语言、汇编语言、FORTRAN及很多古老的 语言写程序。他们是Hacker时代的先驱者,默默贡献,却鲜为人知。

从二次大战结束后到1970早期,是打卡计算机与所谓"大铁块"的mainframes 流行的年代,由Real Programmer主宰
电脑文化。Hacker传奇故事如有名的 Mel (收录在Jargon File\footnote{Jargon File亦是本文原作者所编写的,里面收录了很多Hacker用语、缩写意义、传奇故事等等。Jargon File有出版成一本书:The New Hacker's Dictionary,MIT PRESS出版。也有Online版本: http://www.ccil.org/jargon}中)、Murphy's Law\footnote{莫非定律是:当有两条路让你抉择,若其中一条会导致失败,你一定会选到它。 它有很多衍生说法: 比如一个程式在demo前测试几千几万次都正确无误,但demo 那一天偏偏就会出bug。}的各种版本、mock- German`Blinkenlight' 文章都是流传久远的老掉牙笑话了。

一些Real Programmer仍在世且十分活跃 (本文写在1996年)。超级电脑Cray 的设计者Seymour Cray, 据说亲手设计 Cray 全部的硬体与其操作系统,作业系统是他用机器码硬干出来的,没有出过任何 bug 或 error 。Real Programmer 真是超强!

举个比较不那么夸张的例子:Stan Kelly-Bootle,The Devil's DP Dictionary 一书的作者(McGraw-Hill, 1981年初版,ISBN 0-07-034022-6)与Hacker传奇专家,当年在一台Manchester Mark I开发程式。 他现在是电脑杂志的专
栏作家,写一些科学幽默小品,文笔生动有趣投今日hackers所好,所以很受欢迎。 其他人像David E. Lundstorm,写了许多关於Real Programmer的小故事, 收录在A few Good Men From UNIVAC这本书,1987年出版,ISBN-0- 262-62075-8。\footnote{译:看到这里,大家应该能了解,所谓Real Programmer指的就是用组合语 言或甚至机器码,把程式用打卡机punch出一片片纸卡片,由主机读卡机输入电脑的那种石器时代Programmer。}

Real Programmer的时代步入尾声,取而代之的是逐渐盛行的Interactive computing,大学成立电算相关科系及电脑网络。它们催生了另一个持续的工程传统,并最终演化为今天的开放代码黑客文化。

\chapter{早期的黑客}

Hacker时代的滥觞始於1961年MIT出现第一台电脑DEC PDP-1。MIT的Tech Model Railroad Club(简称TMRC)的Power and Signals Group买了这台机器后,把它当成最时髦的科技玩具,各种程式工具与电脑术语开始出现,整个环境与文化一直发展下去至今日。 这在Steven Levy的书`Hackers' 前段有详细的记载(Anchor/Doubleday 公司,1984年出版,

※译:Interactive computing并非指Windows、GUI、WYSIWYG等介面, 当时有terminal、有shell可以下指令就算是Interactive computing了。 最先使用Hacker这个字应该是MIT。1980年代早期学术界人工智慧的权威:MIT 的Artificial Intelligence Laboratory,其核心人物皆来自TMRC。从1969年 起,正好是ARPANET建置的第一年, 这群人在电脑科学界便不断有重大突破与贡献。

ARPANET是第一个横跨美国的高速网络。由美国国防部所出资兴建,一个实验性 质的数位通讯网络,逐渐成长成联系各大学、国防部承包商及研究机构的大网络。 各地研究人员能以史无前例的速度与弹性交流资讯, 超高效率的合作模式导致科技 的突飞猛进。

ARPANET另一项好处是,资讯高速公路使得全世界的hackers能聚在一起,不再像以前孤立在各地形成一股股的短命文化,网络把他们汇流成一股强大力量。 开始有人感受到Hacker文化的存在,动手整理术语放上网络, 在网上发表讽刺文学与讨论Hacker所应有的道德规范。(Jargon File的第一版出现在1973年,就是一个好例子), Hacker文化在有接上ARPANET的各大学间快速发展,特别是(但不全是)在信息相关科系。

一开始,整个Hacker文化的发展以MIT的AI Lab为中心,但Stanford University 的Artificial Intelligence Laboratory(简称SAIL)与稍後的Carnegie-Mellon University(简称CMU)正快速崛起中。三个都是大型的资讯科学研究中心及人工智慧的权威,聚集著世界各地的精英,不论在技术上或精神层次上,对Hacker文化都有极高的贡献。

为能了解後来的故事,我们得先看看电脑本身的变化;随著科技的进步,主角MIT AI Lab也从红极一时到最後淡出舞台。

从MIT那台PDP-1开始,Hacker们主要程式开发平台都是Digital Equipment Corporation 的PDP迷你电脑序列。DEC率先发展出商业用途为主的interactive computing及time-sharing操作系统,当时许多的大学都是买DEC的机器, 因为它兼具弹性与速度,还很便宜(相对於较快的大型电脑mainframe)。 便宜的分时系统是Hacker文化能快速成长因素之一,在PDP流行的时代,ARPANET上是DEC机器的天下,其中最重要的便属PDP-10,PDP-10受到Hacker们的青睐达十五年;

TOPS-10(DEC的操作系统)与MACRO-10(它的组译器),许多怀旧的术语及Hacker传奇中仍常出现这两个字。

MIT像大家一样用PDP-10,但他们不屑用DEC的操作系统。他们偏要自己写一个:传说中赫赫有名的ITS。

ITS全名是`Incompatible Timesharing System',取这个怪名果然符合MIT的搞怪作风 -- 就是要与众不同, 他们很臭屁但够本事自己去写一套操作系统。ITS始终不稳,设计古怪,bug也不少,但仍有许多独到的创见,似乎还是分时系统中开机时间最久的纪录保持者。

ITS本身是用汇编语言写的,其他部分由LISP写成。LISP在当时是一个威力强大与极具弹性的程式语言;事实上,二十五年後的今天,它的设计仍优於目前大多数的程式语言。LISP让ITS的Hacker得以尽情发挥想像力与搞怪能力。LISP是MIT AI Lab成功的最大功臣,现在它仍是Hacker们的最爱之一。很多ITS的产物到现在仍活著;EMACS大概是最有名的一个,而ITS的稗官野史仍为今日的Hacker们所津津乐道, 就如同你在Jargon File中所读到的一般。在MIT红得发紫之际,SAIL与CMU也没闲著。SAIL的中坚份子後来成为PC 界或图形使用者介面研发的要角。CMU的Hacker则开发出第一个实用的大型专 家系统与工业用机器人。

另一个Hacker重镇是XEROX PARC公司的Palo Alto Research Center。从1970初期到1980中期这十几年间,PARC不断出现惊人的突破与发明,不论质或量,软件或硬体方面。如现今的视窗滑鼠介面,雷射印表机与区域网络; 其D系列的机器,催生了能与迷你电脑一较长短的强力个人电脑。不幸这群先知先觉者并不受到公司高层的赏识;PARC是家专门提供好点子帮别人赚钱的公司成为众所皆知的大笑话。即使如此,PARC这群人对Hacker文化仍有不可抹灭的贡献。1970年代与PDP-10文化迅速成长茁壮。Mailing list的出现使世界各地的人得以组成许多SIG(Special-interest group),不只在电脑方面,也有社会与娱乐方面的。DARPA对这些非`正当性'活动睁一只眼闭一只眼, 因为靠这些活动会吸引更多的聪明小夥子们投入电脑领域呢。

有名的非电脑技术相关的ARPANET mailing list首推科幻小说迷的,时至今日ARPANET变成Internet, 愈来愈多的读者参与讨论。Mailing list逐渐成为一种公众讨论的媒介,导致许多商业化上网服务如CompuServe、Genie与Prodigy的成立。

\chapter{Unix 的兴起}

此时在新泽西州的郊外,另一股神秘力量积极入侵Hacker社会,终於席卷整个PDP-10的传统。它诞生在1969年,也就是ARPANET成立的那一年,有个在AT\&T Bell Labs的年轻小夥子Ken Thompson发明了Unix。

Thomspon曾经参与Multics的开发,Multics是源自ITS的操作系统,用来实做当时一些较新的OS理论, 如把操作系统较复杂的内部结构隐藏起来,提供一个介面,使的programmer能不用深入了解操作系统与硬体设备,也能快速开发程式。

译:那时的programmer写个程式必须彻底了解操作系统内部,或硬体设备。比方说写有IO的程式,对於硬碟的转速,磁轨与磁头数量等等都要搞的一清二楚才行。

在发现继续开发Multics是做白工时,Bell Labs很快的退出了(後来有一家公司Honeywell出售Multics,赔的很惨)。

Ken Thompson很喜欢Multics上的作业环境,於是他在实验室里一台报废的DEC PDP-7上胡乱写了一个操作系统,  该系统在设计上有从Multics抄来的也有他自己的构想。他将这个操作系统命名Unix,用来反讽Multics。

译:其实是Ken Thompson写了一个游戏`Star Travel' 没地方跑,就去找一台的报废机器PDP-7来玩。他同事Brian Kernighan嘲笑Ken Thompson说:「你写的系统好逊哦,乾脆叫Unics算了。」(Unics发音与太监的英文eunuches 一样),後来才改为Unix。

他的同事Dennis Ritchie,发明了一个新的程式语言C,於是他与Thompson用C把原来用汇编语言写的Unix重写一遍。

C的设计原则就是好用,自由与弹性,C与Unix很快地在Bell Labs得到欢迎。1971年Thompson与Ritchie争取到一个办公室自动化系统的专案,Unix开始在Bell Labs中流行。不过Thompson与Ritchie的雄心壮志还不止於此。

那时的传统是,一个操作系统必须完全用汇编语言写成,始能让机器发挥最高的效能。Thompson与Ritchie, 是头几位领悟硬体与编译器的技术,已经进步到作业系统可以完全用高阶语言如C来写,仍保有不错的效能。五年後, Unix已经成功地移植到数种机器上。

译:Ken Thompson与Dennis Ritchie是唯一两位获得Turing Award(电脑界的诺贝尔奖)的工程师(其他都是学者)。

这当时是一件不可思议的事!它意味著,如果Unix可以在各种平台上跑的话,Unix 软件就能移植到各种机器上。

再也用不著为特定的机器写软件了,能在Unix上跑最重要,重新发明轮子已经成为过去式了。

除了跨平台的优点外,Unix与C还有许多显著的优势。Unix与C的设计哲学是Keep It Simple, Stupid'。programmer可以轻易掌握整个C的逻辑结构(不像其他之前或以後的程式语言)而不用一天到晚翻手册写程式。 而Unix提供许多有用的小工具程式,经过适当的组合(写成Shell script或Perl script),可以发挥强大的威力。

※注:The C Programming Language是所有程式语言书最薄的一本,只有两百多页哦。作者是Brian Kernighan 与Dennis Ritchie,所以这本C语言的圣经又称`K\&R'。

※注:`Keep It Simple, Stupid' 简称KISS,今日Unix已不follow这个原则,几乎所有Unix 都是要灌一堆有的没

的utilities,唯一例外是MINIX。

C与Unix的应用范围之广,出乎原设计者之意料,很多领域的研究要用到电脑时,他们是最佳拍档。 尽管缺乏一个正式支援的机构,它们仍在AT\&T内部中疯狂的散播。到了1980年,已蔓延到大学与研究机构,还有数以千计的hacker想把Unix装在家里的机器上。

当时跑Unix的主力机器是PDP-11、VAX系列的机器。不过由於UNIX的高移植性,它几乎可安装在所有的电脑机型上。

一旦新型机器上的UNIX安装好,把软件的C原始码抓来重新编译就一切OK了,谁还要用汇编语言来开发软件? 有一套专为UNIX设计的网络 --- UUCP:一种低速、不稳但很成本低廉的网络。 两台UNIX机器用条电话线连起来,就可以使用互

传电子邮件。UUCP是内建在UNIX系统中的,不用另外安装。於是UNIX站台连成了专属的一套网络,形成其Hacker文化。

在1980第一个USENET站台成立之後,组成了一个特大号的分散式布告栏系统,吸引而来的人数很快地超过了ARPANET。

少数UNIX站台有连上ARPANET。PDP-10与UNIX的Hacker文化开始交流,不过一开始不怎么愉快就是了。PDP-10的Hacker们觉得UNIX的拥护者都是些什么也不懂的新手,比起他们那复杂华丽,令人爱不释手的LISP与ITS,C与 UNIX简直

原始的令人好笑。『一群穿兽皮拿石斧的野蛮人』他们咕哝著。

在这当时,又有另一股新潮流风行起来。第一部PC出现在1975年;苹果电脑在1977年成立,以飞快的速度成长。微电脑的潜力,立刻吸引了另一批年轻的Hackers。他们最爱的程式语言是BASIC,由於它过於简陋,PDP-10 的死忠派与UNIX迷们根本不屑用它,更看不起使用它的人。

译:这群Hacker中有一位大家一定认识,他的名字叫Bill Gates,最初就是他在8080上发展BASIC compiler的。

\chapter{古老时代的终结}

1980年同时有三个Hacker文化在发展,尽管彼此偶有接触与交流,但还是各玩各的。ARPANET/PDP-10文化, 玩的是LISP、MACRO、TOPS-10与ITS。UNIX与C的拥护者用电话线把他们的PDP-11与VAX机器串起来玩。 还有另一群散乱无秩序的微电脑迷,致力於将电脑科技平民化。

三者中ITS文化(也就是以MIT AI LAB为中心的Hacker文化)可说在此时达到全盛时期, 但乌云逐渐笼罩这个实验室。ITS赖以维生的PDP-10逐渐过时,开始有人离开实验室去外面开公司,将人工智慧的科技商业化。MIT AI Lab 的高手挡不住新公司的高薪挖角而纷纷出走,SAIL与CMU也遭遇到同样的问题。

译:这个情况在GNU宣言中有详细的描述,请参阅:(特别感谢由AKA的chuhaibo翻成中文)

http://www.aka.citf.net/Magazine/Gnu/manifesto.html

致命一击终於来临,1983年DEC宣布:为了要集中在PDP-11与VAX生产线, 将停止生产PDP-10;ITS没搞头了,因为它无法移植到其他机器上,或说根本没人办的到。而Berkeley Univeristy修改过的UNIX在新型的VAX跑得很顺,是 ITS理想的取代品。有远见的人都看得出,在快速成长的微电脑科技下,Unix一统江湖是迟早的事。

差不多在此时Steven Levy完成``Hackers'' 这本书,主要的资料来源是Richard M. Stallman(RMS)的故事,他是MIT AI Lab领袖人物,坚决反对实验室的研 究成果商业化。

Stallman接著创办了Free Software Foundation,全力投入写出高品质的自由软件。Levy以哀悼的笔调描述他是the last true hacker',还好事实证明Levy完全错了。

译:Richard M. Stallman的相关事迹请参考: http://www.aka.citf.net/Magazine/Gnu/cover.htm

Stallman的宏大计划可说是80年代早期Hacker文化的缩影 -- 在1982年他 开始建构一个与UNIX 相容但全新的操作系统,以C来写并完全免费。整个ITS的精神与传统,经由RMS的努力,被整合在一个新的,UNIX与VAX机器上的Hacker文化。 微电脑与区域网络的科技,开始对Hacker文化产生影响。Motorola 68000 CPU 加Ethernet是个有力的组合,也有几家公司相继成立生产第一代的工作站。 1982年,一群Berkeley出来的UNIX Hacker成立了Sun Microsystems,他们的算盘打的是:把UNIX架在以68000为CPU的机器,物美价廉又符合多数应用程式的要求。他们的高瞻远嘱为整个工业界树立了新的里程碑。虽然对个人而言,工作站仍太昂贵,不过在公司与学校眼中,工作站真是比迷你电脑便宜太多了。在这些机构里,工作站(几乎是一人一台)很快地取代了老旧庞大的VAX等timesharing机器。

译:Sun一开始生产的工作站CPU是用Motorola 68000系列,到1989才推出自行研发的以SPARC系列为CPU的SPARCstation。

\chapter{私有Unix时代}

1984年AT\&T解散了,UNIX正式成为一个商品。当时的Hacker文化分成两大类,一类集中在Internet与USENET上(主要是跑UNIX的迷你电脑或工作站连上网络),以及另一类PC迷,他们绝大多数没有连上Internet。

※译:台湾在1992年左右连上Internet前,玩家们主要以电话拨接BBS交换资讯,但是有区域性的限制, 发展性也大不如USENET。Sun与其他厂商制造的工作站为Hacker们开启了另一个美丽新世界。 工作站诉求的是高效能的绘图与网络,1980年代Hacker们致力为工作站撰写软件,不断挑战及突破以求将这些功能发挥到百分之一百零一。Berkeley发展出一套内建支援ARPANET protocols的UNIX,让UNIX能轻松连上网络,Internet也成长的更加迅速。

除了Berkeley让UNIX网络功能大幅提升外,尝试为工作站开发一套图形界面也不少。最有名的要算MIT开发的Xwindow了。Xwindow成功的关键在完全公开原始码,展现出Hacker一贯作风,并散播到Internet上。X 成功的干掉其他商业化的图形界面的例子,对数年後UNIX的发展有著深远的启发与影响。少数ITS死忠派仍在顽抗著,到1990年最後一台ITS也永远关机长眠了;那些死忠派在穷途末路下只有悻悻地投向UNIX的怀抱。

UNIX们此时也分裂为BerkeleyUNIX与AT\&T两大阵营,也许你看过一些当时的海报,上面画著一台钛翼战机全速飞离一个爆炸中、上面印著AT\&T的商标的死星。Berkeley UNIX的拥护者自喻为冷酷无情的公司帝国的反抗军。 就销售量来说,AT\&TUNIX始终赶不上BSD/Sun,但它赢了标准制订的战争。到1990年,AT\&T与BSD版本已难明显区分,因为彼此都有采用对方的新发明。随著90年代的来到,工作站的地位逐渐受到新型廉价的高档PC的威胁,他们主要是用Intel80386系列CPU。第一次Hacker能买一台威力等同於十年前的迷你电脑的机器,上面跑著一个完整的UNIX,且能轻易的连上网络。

沈浸在MS-DOS世界的井底蛙对这些巨变仍一无所知,从早期只有少数人对微电脑有兴趣,到此时玩DOS与Mac的人数已超过所谓的"网络民族"的文化,但他们始终没成什么气候或搞出什么飞机,虽然聊有佳作光芒乍现,却没有稳定发展出统一的文化传统,术语字典,传奇故事与神话般的历史。它们没有真正的网络,只能聚在小型的BBS 站或一些失败的网络如FIDONET。提供上网服务的公司如CompuServe或Genie生意日益兴隆,事实显示non-UNIX的操作系统因为并没有内附如compiler等程式发展工具,很少有source 在网络上流传,也因此无法形成合作开发软件的风气。 Hacker文化的主力,是散布在Internet各地,几乎可说是玩UNIX的文化。他们玩电脑才不在乎什么售後服务之类,他们要的是更好的工具、更多的上网时间、还有一台便宜32-bitPC。机器有了,可以上网了,但软件去哪找?商业的UNIX贵的要命,一套要好几千大洋。90年代早期, 开始有公司将AT\&T与BSDUNIX移植到PC上出售。成功与否不论,价格并没有降下来,更要紧的是没有附原始码, 你根本不能也不准修改它,以符合自己的需要或拿去分享给别人。传统的商业软件并没有给Hacker们真正想要的。即使是FreeSoftwareFoundation(FSF)也没有写出Hacker想要的操作系统,RMS承诺的GNU操作系统--HURD 说了好久了,到1996年都没看到影子(虽然1990年开始,FSF的软件已经可以在所有的UNIX平台执行)。

\chapter{早期的免费 Unix}

在这空窗期中,1992年一位芬兰HelsinkiUniversity的学生--LinusTorvalds开始在一台386PC上发展一个自由软件的UNIX kernel,使用FSF的程式开发工具。

他很快的写好简单的版本,丢到网络上分享给大家,吸引了非常多的Hacker来帮忙一起发展Linux-一个功能完整的UNIX,完全免费且附上全部的原始码。 Linux最大的特色,不是功能上的先进而是全新的软件开发模式。直到Linux 的成功前,人人都认为像操作系统这么复杂的软件,非得要靠一个开发团队密切合作,互相协调与分工才有可能写的出来。

商业软件公司与80年代的FreeSoftwareFoundation所采用都是这种发展模式。

Linux则迥异於前者。一开始它就是一大群Hacker在网络上一起涂涂抹抹出来的。 没有严格品质控制与高层决策发展方针,靠的是每周发表新版供大家下载测试,测试者再把bug与patch贴到网络上改进下一版。一种全新的物竞天择、去芜存菁的快速发展模式。令大夥傻眼的是,东修西改出来的Linux,跑的顺极了。

1993年底,Linux发展趋於成熟稳定,能与商业的UNIX一分高下,渐渐有商业应用软件移植到Linux上。不过小型UNIX厂商也因为Linux的出现而关门大吉 -因为再没有人要买他们的东西。幸存者都是靠提供BSD为基础的UNIX 的完整原始码,有Hacker加入发展才能继续生存。

Hacker文化,一次次被人预测即将毁灭,却在商业软件充斥的世界中,披荆斩棘,筚路蓝缕,开创出另一番自己的天地。

\chapter{网络大爆炸时代}

Linux能快速成长的来自令一个事实:Internet大受欢迎,90年代早期ISP如雨後春笋般的冒出来,World-WideWeb的出现,使得Internet成长的速度,快到有令人窒息的感觉。

BSD专案在1994正式宣布结束,Hacker们用的主要是免费的UNIX(Linux与一些4.4BSD的衍生版本)。而 LinuxCD-ROM销路非常好(好到像卖煎饼般)。近几年来Hacker们主要活跃在Linux与Internet发展上。WorldWideWeb让 Internet成为世界最大的传输媒体,很多80年代与90年代早期的Hacker们现在都在经营ISP。

Internet的盛行,Hacker文化受到重视并发挥其政治影响力。94、95年美国政府打算把一些较安全、难解的编码学加以监控,不容许外流与使用。这个称为Clipper proposal的专案引起了Hacker们的群起反对与强烈抗议而半途夭折。

96年Hacker又发起了另一项抗议运动对付那取名不当的"Communications DecencyAct",誓言维护 Internet上的言论自由。

电脑与Internet在21世纪将是大家不可或缺的生活用品,现代孩子在使用Internet科技迟早会接触到Hacker文化。

它的故事传奇与哲学,将吸引更多人投入。未来对Hacker们是充满光明的。

\backmatter

\end{document}