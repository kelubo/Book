% 北洋水师的最后一百天
% 北洋水师的最后一百天.tex

\documentclass[12pt,UTF8]{ctexbook}

% 设置纸张信息。
\usepackage[a4paper,twoside]{geometry}
\geometry{
	left=25mm,
	right=25mm,
	bottom=25.4mm,
	bindingoffset=10mm
}

% 设置字体,并解决显示难检字问题。
\xeCJKsetup{AutoFallBack=true}
\setCJKmainfont{SimSun}[BoldFont=SimHei, ItalicFont=KaiTi, FallBack=SimSun-ExtB]

% 目录 chapter 级别加点(.)。
\usepackage{titletoc}
\titlecontents{chapter}[0pt]{\vspace{3mm}\bf\addvspace{2pt}\filright}{\contentspush{\thecontentslabel\hspace{0.8em}}}{}{\titlerule*[8pt]{.}\contentspage}

% 设置 part 和 chapter 标题格式。
\ctexset{
	part/name= {第,卷},
	part/number={\chinese{part}},
	chapter/name={第,篇},
	chapter/number={\chinese{chapter}}
}

% 设置古文原文格式。
\newenvironment{yuanwen}{\bfseries\zihao{4}}

% 设置署名格式。
\newenvironment{shuming}{\hfill\bfseries\zihao{4}}

\title{\heiti\zihao{0} 北洋水师的最后一百天}
\author{金满楼}
\date{2011.4}

\begin{document}

\maketitle
\tableofcontents

\frontmatter

\chapter{前言}

100多年前,大清帝国耗费巨资打造了一支前所未有的铁甲舰队,它曾经辉煌一时,但由此也承载了时人乃至后人过高的期望。在很多年后,当人们提起这支舰队的时候,大多数人只记取他们的惨痛覆灭,而对他们曾经的英勇与牺牲只字不提。

这是历史的不幸和势利,也是历史的残酷与不公道。国人有个由来已久的坏传统,那就是对失败过于不宽容,这种以“成败论英雄”的历史观,在北洋舰队的身上得到了充分的宣泄。在经历了近百年来的屈辱后,在如今这样一个呼唤富贵与英雄的时代里,大多数国人已经不再需要也不想面对更多的悲情。

回首北洋舰队正式成军的时候,计有“定远”、“镇远”两艘主力铁甲舰(排水量达7000吨)和“济远”、“致远”、“靖远”、“经远”、“来远”5艘次轻量级的快速巡洋舰(排水量在2000吨~3000吨之间),加上早前购买的“超勇”和“扬威”2艘千吨级的老式巡洋舰和1艘国产近海防御性战舰“平远”舰,还有6艘炮舰及鱼雷艇队、练习舰、运输船等,全舰队的排水量接近40000吨,官兵近4000人。

据当时国外的《军事年鉴》统计,成军后的北洋舰队排名位列世界海军的前八,而当时日本海军全部吨位及海战潜力在世界仅排名第16位。遥想当年,每次北洋舰队出海操练的时候,那也是“樯橹如云、旌旗蔽空”,各国海军都为之侧目。
但是,北洋舰队在1888年初具规模后,由于当权者的战略误判,权贵们把持下的海军衙门竟未给北洋舰队添购过一舰一炮。更有甚者,主管拨款的户部在1892年中日局势已经相当紧张的时候,宣布为慈禧太后万寿筹款而停购舰艇两年。很不幸,北洋舰队成了当时政治斗争和错误战略的牺牲品。

与之形成鲜明对比的是,北洋舰队止步不前的这几年,却正是日本舰队高速发展的时期。日本为了尽快赶超北洋舰队,不惜勒紧国民的裤带并拿出大部分的国防预算用于购买军舰。在海军军费仍旧不足的情况下,天皇甚至发动全民捐款,并令公务员捐出相当比例的俸禄以补助日本舰队,在这种狂热的“海军梦”热潮中,就连明治天皇的老妈也捐出了她仅有的几件首饰。

1889年后,日本舰队飞速发展,在短短的几年间即购买了9艘配备了大量速射炮的新型快速战舰,后来在黄海大战中横行一时的“吉野”、“浪速”等舰,便是此时所购。19世纪末是弱肉强食的强权时代,是海军技术飞速发展的非常时期,所谓“逆水行舟,不进则退”,这一静一动之间,日本舰队的实力实际上已经大大超越了原来号称“亚洲第一”的北洋舰队。

作为北洋舰队的当家人李鸿章,他何尝不知北洋舰队的致命弱点在于舰体老化,整体速度不够快又极度缺乏速射炮?但李鸿章为北洋舰队添购快船快炮的数次请求,都由于朝廷官员的短视和倾轧而毫无下文。政治的斗争,战略的失误,由此带来了看似偶然、其实必然的灾难,体制之误,夫庸何言?

鸦片战争后,中国步入了三千年未有之大变局,在洋人的坚船利炮打击下,清廷的士大夫们着实见识了西方列强海军的强大实力,促使了清廷建立新式海军的决心。因此,清廷海军的转型,其实比陆军要早20年。在两次鸦片战争中,大清帝国苦于没有自己的海军,单纯地依靠陆上防御,只有被动挨打的份儿。有鉴于此,在左宗棠和沈葆桢等人的努力下,福州船政学堂得以成立并培养了中国第一批海军人才,这其中就包括后来成为北洋舰队骨干力量的刘步蟾、林泰曾等人。这批人在国内学习后又被派往英、德等欧美国家深造,接受了当时世界一流的海军军事训练。这些“海归派”回国后,大都当上北洋舰队的各舰管带(相当于今天的舰长),而北洋舰队也成为当时整个大清帝国最有技术含量的兵种。

内行看门道,外行看热闹。甲午战争爆发后,令世人大跌眼镜的是,北洋舰队在不到半年的时间里,竟然一败涂地,全军尽殁。这是何等惨痛的失败!不但令时人无法原谅,就是百年后,也足以令国人耿耿于怀。

1895年2月17日,在坚持了近100天后,威海保卫战以北洋舰队的投降而宣告结束。根据协议,幸存的官兵们和丁汝昌等将领的灵柩\footnote{ji\`u}一起挤在被解除了武装的“康济”练习舰上,于当天下午的凄风苦雨中黯然离开刘公岛。留在威海湾内的“镇远”、“济远”、“平远”、“广丙”、“镇东”、“镇西”、“镇北”、“镇南”、“镇中”、“镇边”等10艘残余的舰艇,则作为战利品被日军俘获。由此,曾经煊赫一时的北洋舰队最终全军覆没。

惨痛的失败,足以让人忘掉那些海军军人们曾经战斗的英勇。在残酷的战争中,北洋舰队的将士们曾勇往直前,义无返顾,以至于战死或者自杀殉国的将领竟占到高级指挥官的半数以上,这在整个世界海军史上也是极为罕见的。然而,北洋舰队这一切的奋战和牺牲,在世人的眼中,他们仍旧没有资格说“虽败犹荣”这四个字。

历史学家茅海建先生曾经说过:“历史学最基本的价值,就在于提供错误,即失败的教训。”同理,一个不能面对自己曾经失败的大国,不能算是真正的崛起;真正的大国国民,应该具有审视和反省过去历史失败的气度和胸襟。历史不能忘记,更不应蔑视那些英雄、那些曾经为国捐躯的勇士们。可这一切,国人并未做到。

这里要说的,是100多年前曾经为了这个国家、这片土地而英勇奋战至死的海军将士,这批海军精英在黄海大战和威海之役中,或战火中阵亡,或兵败后自杀殉国,他们是一群失败的英雄。残酷的战争结束后,这些活着的和逝去的败军之将们,他们中的逝者,只能默默承受时人和后人无尽的指责和羞辱;而海战后的生者,亦难免毕生蒙受忍辱苟存的责难,独自面对失败的苦果。这些耻辱,深深地压在他们乃至于后人的心头,以至每当提起这场不幸的战争时,总有些羞于启齿,甚至言不由衷。

100多年前的“康济”号,飘着孤独的龙旗离开威海卫,后人已经记不起那些曾经英勇的事迹与那些已经远去的英灵,但是历史,仍旧听得见他们的怒吼。

\mainmatter

\part{金旅保卫战}

\chapter{突如其来的丰岛海战}

1894年是中国农历的甲午年,这一年原本平淡无奇,只因为赶上了慈禧太后的六十寿辰,空气中似乎添了不少的喜庆色彩。古代没有国庆节、劳动节、六一儿童节,因而皇帝、皇太后的生日便成为举国同庆的好日子,这不,清廷的上上下下,包括24岁的光绪皇帝在内,都在为这个十年一遇的万寿庆典而忙个不停。

按原定的计划,慈禧太后的六十寿诞本该在欢欢喜喜、热热闹闹的气氛中进行,毕竟,人生七十古来稀,即便是寻常的老太太,六十也是个大寿,值得好好的庆祝一下。但事与愿违的是,这一年从年初便不太平,当年的二三月间,朝鲜的“东学党”突然起兵造反,兵焚三千里,一片乱象,朝鲜国王在焦头烂额之下,只得厚着脸皮请清廷出兵帮助镇压。
当时的朝鲜是清廷的藩属,清军应邀前去帮助平定内乱也是应有之义,但令清廷没有想到的是,日本对朝鲜觊觎已久,他们随后也以“保护侨民”为借口派兵进入了朝鲜,而且派出的兵力比清军多出数倍,狼子野心,不待智者而知之。
在外来兵力的威慑和朝鲜方面的招抚下,“东学党”起义渐渐平息,朝鲜国王希望中日双方同时退兵,恢复之前的状况。但是,“请神容易送神难”,清军应邀而来、应求而去,这固然是不成问题的,但日本人心怀叵测,非但不肯退兵,反而想要借机挤走清军,独霸朝鲜。

眼见日本的野心已经暴露,清廷一来要保护自己数百年来的藩属,二来也要维护自己在远东的传统地位,随后便决定向朝鲜增兵,以稳定日益恶化的朝鲜局势。由于从陆路进兵过于缓慢,清廷随后派出北洋舰队由海路护送运兵船前往朝鲜半岛。

1894年7月,在接到李鸿章的指令后,提督丁汝昌派出“济远”舰管带方伯谦指挥本舰及“广乙”“威远”两舰,护送运兵船“爱仁”、“飞鲸”及“高升”号前往朝鲜牙山,以增援驻扎在当地的清军。“爱仁”、“飞鲸”及“高升”三船系李鸿章租赁的英国商船,由于出发港口条件限制,登船麻烦,所以2500多名天津练军士兵只能分批登上这3艘轮船,依次出发。

最开始的护送过程尚且平安无事。护送任务的指挥官、“济远”舰管带方伯谦带着3艘军舰将“爱仁”号和“飞鲸”号安全护送到朝鲜牙山之后,由于护送舰“威远”号系国产的老旧舰艇,航速较慢,于是方伯谦便令其先期返航,等“爱仁”号和“飞鲸”号上的清军士兵大体登岸后,这才率领“济远”舰和“广乙”舰沿途返回,准备去接第三艘运兵船“高升”号。

令人料想不到的是,当“济远”和“广乙”两舰回驶到黄海的丰岛海面时,正好与前来搜索偷袭的日本舰队第一游击队不期而遇。日本舰队第一游击队共编有4艘军舰,都是1889年前后购买的新式舰艇,这次来了3艘,即“吉野”、“浪速”和“秋津洲”三舰。事实上,日本舰队蓄谋已久,这次见北洋舰队只有2艘军舰,于是当即决定发动突然袭击,并首先炮击“济远”舰。
和日本军舰相比,“济远”和“广乙”是北洋舰队的两艘老舰、弱舰,速度比不上日本三舰,火力上也远逊于对手,甫一接战,两舰便很快被打散。激战伊始,“广乙”舰便中了日舰的两发炮弹,舰身倾斜,形势危急。所幸的是,当时海面硝烟弥漫,“广乙”舰在烟雾的掩护下才得以脱离战场,后来被迫搁浅于朝鲜的西海岸。

管带林国祥虽然带领“广乙”舰侥幸成功撤退,免于被日舰击沉,但此时的军舰伤痕累累,早已丧失了战斗力与返航逃生之能力。为防止军舰落入日本人之手,林国祥下令纵火烧毁军舰,以免被日本舰队掳获。尔后,他带着幸存的水兵们赶往牙山前去会合清军叶志超部,但由于此时叶部已经北撤,林国祥等人最终由朝鲜人和英国人将他们送回国,这是后话。

在“广乙”舰脱离战场后,“济远”舰以一敌三,更是支持不住。管带方伯谦见势不妙,慌忙下令西撤,想要赶回旅顺。在占据绝对优势的情况下,日本三舰哪肯轻易罢休,它们仍旧在后面穷追不舍,一定要把“济远”舰击沉。

在紧张的追逐间,前方海面又来了两艘船。不幸的是,这正是“济远”舰要护送的第三艘运兵船“高升”号。另外一艘则是北洋舰队的小型运输舰“操江”号,它此行的目的是给驻朝鲜的清军运送军饷,两舰在航行途中正好遇上,于是结伴而行。

“高升”号是商船,“操江”号是运输舰,两艘船都没有战斗力,这下形势更危急了。“济远”舰本有护卫之责,但在三艘日本军舰追赶下,自己也是“泥菩萨过河,自身难保”,哪有办法去保护“高升”号和“操江”号呢?

无奈之下,“济远”舰只好一边继续撤退、一边打信号旗告诉“高升”号和“操江”号,情况危急,让它们赶紧自行逃避。这时,日本的三艘军舰也很快发现了“高升”号和“操江”号,日舰指挥官在经旗语商议后,决定分开追击:“吉野”舰继续追击“济远”舰,“浪速”舰拦截“高升”号,“秋津洲”则追赶“操江”号。

“浪速”的舰长名叫东乡平八郎,此人曾留学英国,回国后历任日本舰队的多艘军舰舰长,作风一向凶狠毒辣。航速飞快的“浪速”号追赶商船“高升”号,如同老鹰抓小鸡一般轻而易举。很快,“浪速”号便迫近了“高升”号,东乡平八郎立刻命令舰上的21门大炮都摆出战斗态势,并用右舷炮对准“高升”号船身,意在将“高升”号俘虏。“高升”号本不是战斗船只,于是很快落入“浪速”舰的完全控制之中。

随后,日本人放下一只小艇驶近“高升”号,由几名携带枪械的日本海军军官登船,要求检查执照。“高升”号的英国船长高惠悌赶紧出示执照,并提醒日本军官,“高升”号乃是英国商船,请勿轻举妄动。日本人对此毫不理会,他们掷回执照,当即宣布“高升”号已经被俘,必须跟“浪速”舰而去。

在日本人离船后,船上的清军士兵们也隐隐感觉到了即将来临的危险。当“高升”号准备随“浪速”舰而去之后,船上顿时陷入了一片骚动。在仁字军营务处帮带高善继的带领下,清军将士们冲进船长室,并对高惠悌拔刀怒喝道:“敢有降日本者,当污我刀!”
英国船长高惠悌无奈地摊开手,叽里咕噜地说了几句洋文,大概意思是“局势危险,不照日本人的要求去做的话,整个船都会被击毁……”因为语言不通,帮带高善继派人把当时受雇于北洋舰队的外国专家、德国退役军官汉纳根找来,让他告诉船长高惠悌:“中国士兵宁愿死,也决不服从日本人的命令!”

由于高惠悌不肯合作,清军士兵便将他扣押,不准任何人离船。“浪速”号见“高升”号半天没有动静,便再次派艇过来,要求它跟随“浪速”号而去。在日本人的威逼之下,汉纳根前去交涉说:“船上的士兵不许船长服从你们的命令,他们要坚持回原来出发的海港去,船长没办法说服他们。”

这时,高惠悌也过来说:“请告诉你们的舰长,说中国人拒绝‘高升’号当俘虏,他们要求退回大沽口。我们‘高升’号是一艘英国船,离开中国海港时中日并未宣战,我们出发的时候还是和平时期……就算已经宣战,让这些中国人返回也是个公平合理的要求。”

日本人哼哼了两声,便驾艇回舰。随后,“浪速”舰便迫近“高升”号,并突然用6门右舷炮对准“高升”号猛轰。在日舰炮火的猛烈轰击下,可怜的清军将士们只能用步枪还击……用步枪还击全副武装的敌方巡洋舰!这是何等的凄惨与不人道!不到半小时,“高升”号便全部沉没。

让人震惊的是,那些已经落水的中国士兵,日本人竟然也不放过,他们非但不施救,反而一直用机关炮射击,刻意制造了一场海上的野蛮屠杀!惨剧之下,“高升”号上的中国官兵,除245人被附近路过的其他中立国船只施救获生外,其余的871名江淮子弟,全部壮烈殉国。

至于“操江”号,也很快被日舰“秋津洲”号追上。“操江”号原本是上海江南制造总局所造的小型炮舰,船龄已超过25年,因为过于老旧而改为北洋舰队的运输舰。在被“秋津洲”号追赶的过程中,管带王永发只得先把机密文件烧毁,正当他指挥手下士兵将20多万两军饷丢进大海时,一群凶神恶煞的日本兵已经持械登船,将他们俘虏。“操江”号被俘虏之后,包括王永发在内的80多人被押送到日本关押,备受屈辱,一直到战争结束后,他们这些人才被释放。

在“浪速”和“秋津洲”二舰离开后,“吉野”舰则继续追击“济远”舰。由于“吉野”舰是日本刚刚下水服役的最新军舰,其速度远在早已有10年舰龄的“济远”舰之上。当时眼看就要追上了,方伯谦不知是为了逃命还是诈降,他下令水手将白旗挂上,以示投降。在部下不肯挂的情况下,方伯谦甚至亲手将白旗挂上。

“吉野”舰见“济远”舰挂上白旗后,随后挂出旗语,令“济远”舰立即停航,准备派兵接收。在日舰放松警惕之时,“济远”舰的水兵突然用尾炮连发四炮,其中三炮命中,特别是第四炮,更是狠狠地击中了“吉野”舰的要害,将其打得舰首下沉,这下把“吉野”舰长河原要一给打蒙了,他慌忙下令转舵返航,不敢继续追击。“济远”舰这才逃过一劫。

\chapter{黄海血战波涛涌}

“济远”舰带伤返回旅顺后,虽然有传闻说管带方伯谦挂了白旗,但提督丁汝昌考虑到双方力量悬殊,“济远”舰尚能全身而退,便对此事未予追究,后来还向朝廷为方伯谦请了功。
丰岛海战后,中日双方都向对方宣战,甲午战争正式爆发。北洋舰队的将士们得知“高升”号被日本军舰击沉、船上的800多名弟兄葬身大海后,一个个都怒火中烧,愤慨异常,水兵们都摩拳擦掌,要和日本人决一死战,为陆军弟兄们报仇。
据当时北洋舰队雇用的外国专家回忆,北洋舰队的下层官兵大多求战心切,而高级将领包括提督丁汝昌在内,却十分谨慎,并不敢轻易言战。
这个细节颇值得玩味,想必是对彼此知根知底的缘故。
令人扼腕叹息的是,中方的陆军在朝鲜战场上连吃败仗,一退再退,陷入了极端被动的局势。就在日本陆军逼近平壤的同时,日本舰队也集中了“吉野”“高千穗”“秋津洲”“浪速”“松岛”“千代田”“严岛”“桥立”等12艘主力舰艇,在海面上极力寻求与北洋舰队进行决战的机会。
由于陆军受挫,指挥作战的北洋大臣李鸿章随后又令北洋舰队再次护送军队前往大同江登陆,以增援守卫在平壤的清军。由于上次吃了日本人以多打少的亏,这一次丁汝昌决定舰队全体出动,执行护航任务。
这次护送运兵船的过程还算顺利,海面平静无事。1894年9月17日,在清军士兵登岸的过程中,北洋舰队留下“平远”“广丙”两艘舰艇和鱼雷艇在岸边执行护卫任务,随后在丁汝昌的率领下,舰队主力开始沿海面进行常规的巡航训练。当舰队行进到大东沟海面附近时,西南方向突然出现了几簇黑烟,来袭的正是日本舰队。
在得到报告后,丁汝昌立即下令,各舰准备战斗。为了发挥各舰舰首主炮的威力,丁汝昌命北洋舰队以犄角雁行阵迎敌,其中以旗舰“定远”号和“镇远”号两艘巨舰居中,“致远”“靖远”“济远”“广甲”四舰布于左翼,“经远”“来远”“超勇”“扬威”4舰列于右翼,舰队形成一个扁V字形,向一字阵形的日本舰队拦腰冲去。
中午时分,双方舰队接近至5300米时,北洋舰队旗舰“定远”号为震慑敌舰,首开巨炮,炮弹落在日舰“吉野”号舰舷左100米处,海水四溅,声势惊人。10秒钟后,“镇远”舰也向敌舰开炮。在行进了3分钟后,日本舰队也开始发炮还击,刹时间,双方各舰百炮轰鸣,大东沟海面顿时硝烟弥漫,海水都为之沸腾。
至此,这场规模巨大、时间持久的海上鏖战,终于正式打响了。值得一提的是,这是近代史上以蒸汽战舰为主的第一次大规模海战,在世界海军史上记下了浓重的一笔。
开战后,日本舰队速度最快的第一游击队四舰本是攻击北洋舰队的“定远”“镇远”两舰,但它们见北洋舰队的这两艘主力舰来势凶猛,便突然向左大转弯,以斜线加速从“定远”“镇远”两舰之前夺路而进,直扑北洋舰队的右翼“超勇”“扬威”两艘弱舰。
“超勇”“扬威”两舰的舰体规模小,船龄长,速度慢,火力与防御能力都很差,虽然奋勇抵抗,并击中“吉野”和“高千穗”两舰数炮,但最终不能与号称“帝国精锐”的日本第一游击队四艘强舰相抗衡。半小时后,“超勇”“扬威”两舰都已中弹累累,船上燃起大火,危在旦夕。
激战中,“超勇”舰最终不支并逐渐沉入海中,管带黄建勋坠水后,北洋舰队的一鱼雷艇驶近抛长绳相救,但黄建勋推绳不就,自沉于海,与坐舰共生死。而另一边,“扬威”舰也在敌舰的轮番轰击下,全船尽毁,管带林履中奋然蹈海,以身殉国。
与此同时,北洋舰队的其他诸舰也和日本各舰展开了殊死的搏斗。
开战不久,在飞桥上督战的提督丁汝昌不幸腿部受伤,动弹不得,但他仍旧坚拒不退,并坐在甲板过道之侧督战到底,以鼓舞士气。在丁汝昌受伤后,“定远”管带刘步蟾随即代为指挥,其表现极为英勇出色。没多久,“定远”舰便击中日本旗舰“松岛”号的炮塔,令其不敢继续正面对峙而是急转舵向左逃避。
此时,日舰“比睿”号也因为船龄老、速度慢而脱离了日本本队,正好与“定远”“靖远”舰正面相遇,逃无可逃,差点就被俘虏。令人诧异的是,“比睿”号的舰长在必死的勇气支撑下,用速射炮连射作掩护,并极其大胆地从北洋舰队的舰艇中穿过,而“定远”等舰担心误伤友舰,无法对其猛击,致使“比睿”号侥幸逃脱。不过,“比睿”号虽然逃过一劫,但已经丧失了战斗能力,随后挂出“本舰退出战列”的信号,逃离战场。
小号的日舰“赤城”号也遭到同样命运,在北洋舰队左翼诸舰猛击下,“赤城”舰中弹累累,死伤甚众,舰长坂元八郎太被一炮击中脑袋,当场毙命。在“赤城”舰转舵驶逃的时候,又被尾追的“来远”舰屡屡击中,代理舰长佐藤铁太郎也被击成重伤。最后,“赤城”舰勉强逃出海战区域,已经丧失了战斗力(注:在后来北洋舰队上报的战报中,这两艘日本战舰被误记为“击沉”,但事实上,这几艘日本军舰在经过修理后,仍旧参加了之后的战斗)。
此时,在清军登陆处执行护卫任务的“平远”“广丙”两舰听到大东沟海面的炮声后,随即召集港内的鱼雷艇也赶来参战。进入作战海域后,“平远”舰恰好经过日本旗舰“松岛”号的左侧,管带李和抓住战机,一炮击中了“松岛”的中央水雷室,当场击毙敌发射手4名。“松岛”舰作为日本舰队的旗舰,当然也不是吃素的,它很快发炮还击并引起了“平远”
舰上的火灾。为扑灭大火,“平远”舰只得暂时驰离战场,结队而行的友舰“广丙”号也只好随之驶避。
随后,北洋舰队又击中日舰“西京丸”,致使其甲板渗水,舵机损坏,“西京丸”用人力舵才勉强得以航行。就在这时,北洋舰队的“福龙”号鱼雷艇突然赶上前去,并向其发射鱼雷,“西京丸”躲避不及,无路可逃,在船上督战的日本海军中将桦山资纪双眼一闭,惊呼道:“我事毕矣!”但阴差阳错的是,大概是因为发射距离过近,鱼雷竟然从舰下穿水而过,“西京丸”这才侥幸逃脱。
但是,日舰第一游击队的四艘新军舰依旧十分凶猛。激战中,“定远”舰忽中一炮并引发大火,火势极为猛烈,甲板上的各炮台一时间无法发射,日舰见机便群起向“定远”舰扑来。为给“定远”舰争取扑灭大火的时间,“镇远”“致远”舰急忙上前掩护,当时“致远”舰以一舰对四舰,舰上炮弹几乎用尽,舰体也身受重伤。
“致远”舰管带邓世昌见日本第一游击队的“吉野”舰在海面上横行无忌,危害巨大,加上自己军舰上炮弹即将用尽,于是果断地下令用舰首犄角直冲“吉野”,准备将之撞沉,挫灭敌焰。不料正当两舰迫近、敌人惊呼逃避时,“致远”舰却不幸被敌方鱼雷击中,全舰沉没。邓世昌落海后,其爱犬凫到身边,咬着他的衣服不让其下沉,但邓世昌用力按爱犬入水,与舰同沉,壮烈殉国。
“致远”舰沉没后,左翼的“济远”“广甲”两舰处境孤危,伤亡惨重,“济远”舰管带方伯谦见“致远”舰被击沉,惊慌失措,随后竟挂“本舰已受重伤之旗”,逃离战场,“广甲”舰见势不妙,也随之而逃。
“济远”“广甲”两舰逃走后,“吉野”等4艘日舰死死咬住“经远”
舰不放,意在击沉而后快。由于敌众我寡,“经远”舰死伤惨重,大副陈荣、二副陈京莹先后阵亡,管带林永升也在炮火下中弹脑裂而亡。令人感慨的是,“经远”舰在无人指挥的情况下,剩下的水兵们仍旧坚守岗位并继续开炮击敌,最后在熊熊烈焰中浩然沉没。“经远”舰全舰200多人,除16人获救外,其余全部壮烈殉国。
残酷的海战进行到这时,北洋舰队只剩下“定远”“镇远”“靖远”“来远”4舰在最初的海面继续作战,而日本舰队尚有9艘战舰,在数量上占据了绝对优势。尽管如此,双方舰队仍旧分为两拨厮杀,日舰本队“松岛”“千代田”“严岛”“桥立”和“扶桑”5舰死死缠住北洋舰队的主力舰“定远”号和“镇远”号;日本第一游击队“吉野”“高千穗”“秋津洲”、“浪速”4舰则专攻“靖远”和“来远”两舰,想将两舰击沉后全军合力围攻“定远”和“镇远”舰,意图全歼北洋舰队。
此时,北洋舰队已经到了最为危急的时候!
面对日本4舰的围攻,“靖远”“来远”两舰随机应变,它们临时结成姊妹舰,彼此互相依持,虽然两舰都已经中弹数百,但仍旧顽强作战。战至后来,为了修补中弹后的漏洞和扑灭船上的烈火,两舰协同冲出包围并驶至大鹿岛附近,以背靠浅滩的有利地势,用舰首重炮对准尾追而至的日本4舰,敌舰也无可奈何。
在最初的作战海域,北洋舰队只剩下“定远”和“镇远”两艘主力舰还在同日舰本队五舰激战。由于“定远”和“镇远”的舰体规模大、装甲厚,虽然身中敌方上千弹,但两舰仍旧岿然不动,在血战中未落下风(敌方多用速射炮,只在“定远”“镇远”舰吃水线上的的装甲留下弹痕,尚无大碍)。
在管带刘步蟾和林泰曾的英勇指挥下,北洋舰队的两艘主力舰配合默契,官兵们发挥了大无畏的精神,终于顶住了日本5舰的疯狂进攻。战至下午3点半,“镇远”一炮击中日本旗舰“松岛”右舷下甲板,打得“松岛”号舰体倾斜,烈焰冲天,当场杀死杀伤敌兵84人。“松岛”舰被击中后,已经丧失了指挥和战斗能力,击沉“定远”和“镇远”舰的图谋也基本落空。
下午5时,“靖远”和“来远”舰在扑灭大火后,又召集“广丙”诸舰及鱼雷艇归队,北洋舰队的声势益振。战至下午5点半,海面上浓烟四起,落日将沉,日本舰队司令伊东佑亨见本队各舰多已受伤,加上害怕北洋舰队的鱼雷艇袭击,于是便下令向南驶逃。北洋舰队尾追数海里后,因日舰开足马力逃走,一时难以追赶,便也转舵回航。
在鏖战了5个多小时之后,这场惊心动魄的大海战,终于以日本舰队首先撤出战场而宣告结束。此时,海面上已是暮色苍茫,残阳如血,海风吹来的,尽是刺鼻的硝烟和死亡的气息。

\chapter{方伯谦军前正法}

黄海大战后,北洋舰队有4艘军舰被毁,分别是邓世昌管带的“致远”舰、林永升管带的“经远”舰、黄建勋管带的“超勇”舰、林履中管带的“扬威”舰,这4艘军舰的管带全部以身殉国。
最让人憋屈的是,日本舰队虽然也有5艘军舰遭到重创,但由于北洋舰队速射炮少、炮弹威力不大,加上作战经验不足,携带过多的实心弹而爆破弹较少,竟然在这场残酷的海战中未能击沉一艘敌舰。受伤的日本军舰开回军港后,经过一段时间的修理后仍旧可以使用,但北洋舰队被击沉的舰艇则只能永沉大海了。因此,日本舰队的胜利,既有一丝运气的成分在内,也不能不说是比北洋舰队的战术略胜一筹。
失败的耻辱像毒蛇一样吞噬着水兵们的心,但更令人气愤的是,北洋舰队中竟然出了逃兵,这就是上次挂了白旗的“济远”舰。也许是丰岛之战打怕了,海战尚未结束,方伯谦便借口军舰受伤无法再战,竟然带着“济远”舰仓皇逃回了旅顺,这实在是让整个北洋舰队蒙羞。
回到旅顺后,方伯谦赶紧向丁汝昌请罪,丁冷笑道:“岂敢,方大人,这个罪我可受不起!”言罢,丁汝昌便命送客。但在第二天,丁汝昌又派方伯谦率“济远”舰和“金龙”号拖轮前往大连湾将重伤搁浅的“广甲”舰拖回,看能否修理,也算是再给方伯谦一次机会。
当“济远”舰开出旅顺港的时候,军港中的水兵们冷冷地看着,在这冷漠的空气中,突然有水兵大声骂道:“狗日的逃兵!”这句话立刻点燃了众人的满腔怒火,水兵们纷纷拿起身边的东西砸向“济远”舰,有骂“黄鼠狼”的、也有骂“胆小鬼、王八蛋”的,群情激涌,恨不能将方伯谦生吞活剥。
在愤怒的骂声中,“济远”和“金龙”号缓缓离开旅顺港。方伯谦大概是心中有愧,他一直躲在船舱中,始终没有露面。
“济远”舰和“金龙”号并没有将“广甲”舰带回,只是把管带吴敬荣及剩余的水兵带了回来。原来,“广甲”舰受伤太重,搁浅的地方又极其复杂,几经努力都无法将其拖出。在作业过程中,方伯谦害怕日本军舰会发现他们,为避免被日本人掳获,于是让吴敬荣和他的水兵离开军舰,随后用炸药将“广甲”舰炸毁了。
在失败的阴影纠缠下,在忐忑不安的气氛中,朝廷的命令在黄海大战后第七天下发了:逃将方伯谦问斩。
接到这个命令后,丁汝昌的心里也颇不是滋味,他也知道,培养一个海军将领在当时实在是太不容易了,但军纪是残酷无情的,逃将若是不杀,士卒何以用命?再说了,这对那些殉国死难的将士们也是一种不公。
沉吟再三后,丁汝昌将贴身卫士杨发和夏景春两人找来,让他们去执行这个任务。
杨发是天津人,夏景春是山东荣城县人,两人都已跟随丁汝昌多年,他们既是丁汝昌的亲信卫士,也是北洋舰队的老兵,这些年来共同见证了北洋舰队的成长与兴衰。次日清晨,杨发和夏景春带人前去营务处弹药库提人,方伯谦前几天便被关在这里。
在幽暗的灯光指引下,杨发等人顺着弯弯曲曲的石阶来到一个作为临时监狱的仓房,当时只见“来远”舰的管带邱宝仁也在那里,而方伯谦则穿着青色的丝绸睡衣,不知道是没睡醒还是其他原因,他正对着桌上的一壶酒和一盘菜发愣。邱宝仁与方伯谦是昔日福州船政学堂的同窗,这次想必是已经知道了消息,特为方伯谦送行来了。邱见杨发等人进来,叹了口气,随后便起身离去,方伯谦则目光呆滞地望着邱宝仁的身影,坐在那里一动也不动。
等邱宝仁走后,杨发从衣袖里掏出一纸电报念道:“本日奉旨:李鸿章电奏,查明海军接仗详细情形,本月十八日开战时,自‘致远’冲锋后,‘济远’管带副将方伯谦首先逃走,致将船伍牵乱,实属临阵退缩,着即行正法。钦此。”
这时的方伯谦,虽然早已从邱宝仁处得知了要将自己处斩的消息,但听完这个命令后依旧脸色苍白,两腿战栗不止,头上冒出了豆大的汗珠。
杨发见后,颇为不屑,冷冷地说:“方大人,请吧!”随即令人架起方伯谦,押赴到旅顺黄金山炮台下行刑。
旅顺的清晨颇有些寒意,拂晓前的一丝霞光,掠过威武雄壮的黄金山炮台,映在水兵们的脸上,让人感到眩晕。黄海大战虽然已经过去七天了,但战争的火药味依旧浓烈,一阵萧瑟的秋风刮过,黄金山下的刑场上一片荒草更萧杀了,似乎正洒落前夜的一层露气。到刑场立定后,杨发看了看即将喷薄而出的朝日,闷声喝道:“时辰差不多了,行刑吧!”
这时,半个血红的太阳浮出了海面,刽子手举起了手中的大刀。
刀光闪处,方伯谦人头落地。

\chapter{日军花园口登陆}

黄海大战后,中日陆军又在平壤展开大战,但这一次,清军败得更惨。
按李鸿章在战前制订的计划,以陆军为主,先调派陆军至平壤,再南下驱逐日军;同时,以北洋海军扼守渤海湾口,掩护陆军进驻朝鲜。日本大本营制订的作战计划则要复杂得多,其战略重点却是以海战为基础,再决定下一步的战略目标,可谓是反其道而行之。
日本人的计划是,在中日两国海军实力未明、胜负难测的前提下,分别有三个选择:
1.假如海战失败,中国取得制海权的话,则增派陆军固守已占朝鲜地区,尽可能地击退清军的进攻。
2.如果海战胜败未分,则由陆军将清军驱逐出朝鲜,以实现完全控制朝鲜的目的。
3.如果日本舰队在海战中获得胜利,则由海军护送陆军至中国本土,寻求与清军主力决战,最终迫使清廷投降。
从黄海大战的结果来看,日本可以采取第二策,也可以采取第三策,因为北洋舰队虽然损失较大,但主力舰“定远”“镇远”基本完好无损,加上其他舰艇,对日本舰队仍旧是一个潜在的威胁,特别是日本舰队在执行护送运兵任务的时候,更不能掉以轻心。
但是,清朝陆军的疲弱不堪和连战连败,大大激发了日本的勃勃野心,日本最终决定采取第三策:彻底消灭北洋舰队,派陆军进攻中国本土,迫使清廷投降。
1894年10月24日,日军在联合舰队的护卫下,在辽东半岛的花园口登陆。
这一天的清晨,浓雾尚未散去,花园口海滩上除了几声鸟叫,四周静悄悄的。这里本是荒凉无人之地,附近几乎没有人烟,日军从这里登陆很难被发现。从地理位置上看,花园口距离金州160里,距大连湾200里,正好可以切断清军的退路。
但是,花园口也有不利的条件,因为它的海岸很浅,大船无法就近停泊,如果日军在这里登陆,需要跋涉七八里长的泥泞滩涂,难度非常之大。正因为如此,当时防守金州和大连湾的清军,谁也没有想到日军会在这里登陆。
日本人恰好抓住了这个空子,所以决定宁可冒风险,也不能让清军过早地发现他们登陆。在朝鲜战场上取得了最终胜利后,日军休整了近一个月,随后兵分两路,一路由陆路突破鸭绿江进攻九连城,另一路则是从花园口登陆辽东半岛,两路同时开战。
这次前来花园口登陆的,系日本陆军大将大山岩指挥的第二军(进攻九连城的为第一军),由陆军中将山地元治的第一师团和谷川好道少将的混成第十二旅团组成,其中第一师团又包括陆军少将乃木希典的第一旅团和陆军少将西宽二郎的第二旅团,总兵力在25000人左右,战马2700匹。日军第二军的计划,是先攻占旅顺、大连,随后北上与第一军会合,以控制整个辽东地区。
在日本联合舰队的护送下,日本第二军于登陆前一天的上午搭乘40多艘运兵船,从朝鲜的渔隐洞出发,随后便悄悄地向辽东半岛前进,目标直扑花园口。进入中国海域时,已经是夜半时分,船灯全熄,船队靠着微弱的指示灯,在一片黑暗的海面上航行。
而先期出发的“千代田”等5艘日本军舰此时已经到达花园口,随后派舢板载着一个海军陆战队先行登陆,按计划在高处树立了一面日本军旗,为后期到达的运兵船指示登陆地点。
24日早上7点左右,残夜刚刚褪去,海面上的雾气尚未消散,日本的第一批运兵船在舰队的护送下到达了花园口。执行护卫任务的日本联合舰队司令官伊东佑亨命令本队及第一、第二游击队停泊在远海严密监视,防止北洋舰队突然来袭;第三、第四游击队停泊在靠近花园口的海面,掩护陆军登陆;“八重山”“筑紫”“大岛”“鸟海”“西京丸”和“相模丸”6舰则协助陆军登陆。另外,伊东佑亨又派航度快的“秋津洲”和“浪速”
两舰分别驶往威海卫和旅顺口,以监视北洋舰队的行动。
布置妥当后,大群如蚂蚁般的日本兵从运兵船中鱼贯而出,他们颤颤巍巍地顺着舷梯登上摇晃不定的舢板。所幸的是,天气很好,太阳出来后不久海上的雾气便很快消散了,可以清楚地看到几里外海岸边的悬崖和礁石。尽管岸边高处已经树立了登陆的指示旗帜,但日本兵还是十分谨慎不安,特别是看到那些像是城墙的礁石后,就赶紧猫下身,生怕岸边有伏击清军。
每艘日本运兵船都备有数条舢板,每条舢板可以坐40~50名士兵。由军舰拖运至海滩。登陆命令下达后,花园口海面异常混乱,大小汽艇喷着黑烟,到处都是横七竖八的舢板,指挥官不断地催骂士兵下船上岸,士兵们则抱着枪支,胆战心惊地坐在舢板上等待汽艇将他们拉走。
汽艇将舢板拖到离岸边几百米的地方后,便因为吃水太浅而不能再前进了,日本兵只能跳下舢板,涉水登陆。当时已是秋天,水温不算太冷,日本兵挽起裤腿,扛着步枪和背包踏着滩涂上的淤泥向岸边艰难地跋涉,一路上狼狈不堪。由于登陆部队人数过多,加上受到涨落潮的影响,日本第二军总共花了3天时间才全部登陆完毕。
日本人占据花园口后,放眼望去,只见高山连绵,丘陵起伏,山河雄伟,旷野空寂,不觉雀跃心喜,他们终于来到了这个觊觎已久的邻国土地上。很快,登陆的日军便在岸边不远处找到一个村落并霸占了其中一栋最好的房子作为第二军司令部的所在地,以等待后续部队完成登陆。
稍作休整后,日军便兵分两路直扑金州:第一师团主力由山地元治率领,从复州大道包抄金州;第一旅团则由乃木希典率领从金州大道正面进攻。数日后,两路日军像一群蝗虫一样来到了金州城下。
金州城是个古城,也是辽东半岛的咽喉所在,由徐邦道的拱卫军和连顺的捷胜营防守,兵力远比日军单薄。好在金州城墙高大稳固,居高临下,向日军射击很有优势,此外城门上还有一门克虏伯重炮,威力很大,一炮放去,声音有如万雷齐鸣,山河都为之震动。很显然,这门炮对攻城的日军构成了很大的威胁。这时,金州城内的旗民们也都在隆隆的炮声中登上城墙,为士兵们运送炮弹,协助防守。
在硝烟弥漫中,进攻的日军虽然在人数和火力上远占优势,但一时也束手无策。日军指挥官山地元治见久攻不下,心中十分焦躁,于是下令总攻,让日军从东、北两面向金州城冲击,一定要将金州拿下。
在日军潮水般的进攻下,清军的兵力虽然有限,但城墙有3丈多高,日军一时也无法攻进城中,而清军士兵却居高临下,不断给予日军杀伤。情急之下,日军连派数十名工兵,冒死接近城门,以图实行爆破。日本工兵背着炸药试图接近城门时,清军士兵纷纷向其射击,日本工兵很快便被打死打伤了好几个。
可惜的是,由于当时的枪大都是单发,换弹药的时间比较长,最后还是有几个不怕死的日本工兵潜到了城门之下并成功实施了爆破,不久,永安门(北门)被炸开了一个口子,日军便从此门蜂拥而入,攻入城内。守军统领徐邦道和连顺见城门已破,只得率领余部从西门和南门突围而出。令人为之扼腕的是,当时还有一哨士兵没来得及接到撤退命令,他们一直在城内和日军展开顽强的巷战,最终全部战死。
徐邦道和连顺率部在撤出的路上,正好遇到从大连湾赶来救援的赵怀业部。金州已经不保,3个统领聚起来一合计,觉得剩余的部队无法守住大连湾,倒不如直接撤往旅顺,会合那里的兵力再作打算。于是,日军在占领金州后,很快又兵不血刃地占领了大连湾。
说来惭愧,大连湾本有多座德国专家汉纳根设计建造的坚固炮台,清军撤退后,大量的德制枪炮被日军轻易获得,实在有点可惜和可悲。

\chapter{天津保旅会议}

在日军登陆的时候,李鸿章在天津的衙署也是一片忙乱,滴滴答答的电报收发声彻夜响个不停。李鸿章手里紧紧地攥着一纸译好的电报,电报上说,日军已经突破鸭绿江并攻占了九连城。
正当李鸿章眉头紧锁、烦躁地来回踱步的时候,驻守大连湾的总兵赵怀业又发来电报,称日本兵已在花园口登陆,请求速派援兵增援。日军兵分两路的态势已经很明确了,但这时李鸿章手里哪里有兵可调?他唯一的办法,也只能令诸将领各守营盘,以静待动,仅此而已。
至于北洋舰队这边,由于在黄海大战中“定远”“镇远”“靖远”和“来远”等各舰的损伤都十分严重,舰队正停泊在旅顺船坞中日夜赶班,抢修受伤舰只。提督丁汝昌虽然有伤在身,但也不时地进入船坞督工,催促赶紧修理,尽早出海。
“来远”舰是损伤最严重的,甲板烧得几乎只剩下个架子,因而修复的工期拉得很长(就这点而言,当时日本船坞的建造和修理能力远在旅顺之上)。但是,朝廷对北洋舰队毫不了解,而且还十分不满,他们不时地发电报责问:北洋舰队为何不出海巡游,是否胆怯避战?
因此,丁汝昌只得令船坞工匠尽快抢修,并在尚未完全完成修理的情况下带领舰队出海巡游。在黄海巡航一圈后,丁汝昌带领舰队返回威海卫,以补充弹药和给养。在日军登陆花园口的第二天,曾有两艘日本军舰前来威海口挑衅,丁汝昌接报后,立刻令各船升火,随后亲率“定远”“镇远”“济远”“靖远”“平远”“广丙”6舰和两艘鱼雷艇出击。
日本军舰见北洋舰队大队出战,调转方向逃走了。
由于当时通讯和侦察水平的落后,北洋舰队对日军在辽东半岛花园口的登陆毫不知情。直到10月28日,李鸿章发电报给丁汝昌,命令他亲率北洋舰队前往大连湾大孤山一带巡查,以探明日军的登陆情况,丁汝昌才得知日军已经在辽东登陆。接到命令的当天晚上,丁汝昌便率领前两天出战的六舰和两艘鱼雷艇离开威海,前往旅顺一带探察。
当天晚上,凉风习习,星光不见,除了机器的轰鸣声外,海面上一片平静。在海风的吹拂下,北洋舰队各舰劈开层层波浪,在茫茫大海上快速前进。晚饭后,提督丁汝昌来到甲板上,看着前方这片黑沉沉的海域,不禁唏嘘不已,感慨万千。
一个多月前,北洋舰队还有13艘战舰,但在丰岛海战和黄海大战后,舰队损失了“广乙”“致远”“经远”“超勇”“扬威”“广甲”6艘军舰,邓世昌、林永升等优秀的将领和英勇的水兵们,都在不久前的战火和硝烟弥漫中永沉大海,壮烈殉国,他们战死的地方,离这里并不远。想到这里,丁汝昌心里不禁有点酸楚,对自己和北洋舰队的前途命运感到担忧。
“定远”舰的管带刘步蟾不久也来到甲板上,他走到丁汝昌的身边,手扶栏杆,欲言又止。两人看起来都是心事重重,默默地看着前方的海面。过了一会儿,丁汝昌突然发问:“刘大人,你说我们这一去,倘若遇到日本舰队,我们可有胜算?”刘步蟾摇了摇头,沉吟了半晌才说:“如今我们舰艇大都有伤,且舰少力薄,如果日本大队出击,恐怕凶多吉少。”
丁汝昌又问:“你看这旅顺能否守住?”刘步蟾说:“日本陆军登陆的都是万人大队,兵力集中,有如风云席卷而来,而我防军都是分兵把守,人数不过千,一旦接战,岂不是以卵击石?”丁汝昌听后,遥望远方,又陷入了长久的沉默。
这时,海面上开始下雨了。风吹舰旗,雨洒大洋,淅淅沥沥,如诉如泣。
第二天的一大清早,昨夜大片的乌云已经不见踪影,大海依旧是碧水粼粼,海面上一无所有,还是和往常一样空旷和寂寞。天亮后没多久,北洋舰队便到达了旅顺。令人吃惊的是,日军在花园口登陆的消息传开后,旅顺已经是风声鹤唳,人心惶惶,丁汝昌本想抓紧时间对受创的舰艇再进行一次维修,但船坞里的工匠们很多都已经逃散,一时间连人都找不到了。
无奈之下,北洋舰队匆忙补充了燃料和炮弹后,丁汝昌便率舰队再次出海,前往大连湾巡查。走到一半,丁汝昌和刘步蟾一商议,觉得舰队的兵力过于单薄,舰伤又尚未完全修复,“定远”和“镇远”两舰的锚机也需要修理,随后又下令舰队返航回到了旅顺。
11月6日,也就是日军攻陷金州的当天,远在天津的李鸿章同时接到了两份电报。
一份电报是朝廷发来的。在电报中,光绪皇帝心急火燎地指示李鸿章:“日军进逼金州,旅顺万分危急。日军在中国登陆,必有贼舰停泊并来往接济,你速令海军各舰前往游弋截击,阻其后路,不得有误!”
李鸿章听身边的幕僚念完电报后不免苦笑,心里暗叹道:“日本舰队全队出动,北洋舰队现在这几艘军舰,哪里还有能力主动截击?这岂不是用鸡蛋碰石头?”
幕僚又拿起第二封电报报告说:“中堂大人,这是北洋舰队提督丁汝昌发来的。”
李鸿章听后急切地道:“快念!”
幕僚说:“丁提督说昨日有日本军舰窥测旅顺,当即被炮台击退。大连湾形势吃紧,丁提督想将舰队撤回威海卫。”
李鸿章听后十分不悦,他气呼呼地从幕僚手中接过电报,细细研读。
丁汝昌报告说,北洋舰队有三个担忧:一是大连湾一旦失手,日军必定从后路进攻旅顺,北洋舰队在港内,起不到防守的作用;二是旅顺港的口门窄小,一旦日本舰队从海面进攻,北洋舰队不能整队出击,特别是“定远”和“镇远”两艘主力铁甲舰,出港必须等涨潮时,万一情况紧急,想冲出港内也不是易事;三是倘若日本舰队晚上来袭,特别是鱼雷艇夜间偷袭,北洋舰队缺少快炮,恐怕难以防备。
李鸿章看后大怒,将电报掷于桌上,厉声道:“丁汝昌等人尚未接战,便如此慌张,着实可恨!”
恼怒归恼怒,冷静下来想想,李鸿章觉得丁汝昌说的也不是完全没有道理,于是令幕僚复电:“旅顺乃是水师口岸,一旦船坞有失,舰队断不可全毁。旅顺口外目前是否确有日舰,须先探明再定进止,见机行事,且勿慌张!”
11月7日,丁汝昌接到李鸿章的电报后,随即下令北洋舰队离开旅顺,驶回威海卫。当天晚上,日军攻占了大连湾。
说来讽刺的是,大连湾失陷的这天,正好是慈禧太后的六十寿辰。令慈禧太后万万没有想到的是,这场筹备了数年并耗费上千万两白银的万寿庆典,原本是喜庆的高潮,但最后还是被小日本的隆隆炮声给扫了兴致。
这一天,慈禧太后在前往皇极殿接受光绪皇帝和文武百官朝贺的路上,一直脸色阴沉,十分不高兴。
因为战争,原本在颐和园举行的万寿庆典最后只好简化到紫禁城里举行。虽然紫禁城里四处张灯结彩,宁寿宫的大戏锣鼓也已经敲响,但进进出出的大小官员们却个个神色紧张,交头接耳,前方战场的小道消息把刻意营造出来的喜庆色彩给冲得所剩无几。
这一切,慈禧太后都看在眼里,她心里这个气啊,真是无法用言语来表达。
想当年,慈禧太后四十寿庆的时候遇到亲生儿子同治皇帝病危,没有心思去搞什么庆典;五十寿庆的时候本想热闹一下,可偏偏又遇到和法国人开仗;现在六十寿庆了,也没有安生日子,连这小日本都欺负到头上来了。
慈禧太后心想,我为大清辛辛苦苦操持了这么多年,没有功劳也有苦劳;如今皇帝也亲政了,我也该好好享享福了,可怎么每次大寿都碰到这样或是那样的倒霉事呢?我的命怎么这么苦啊!我这都招谁惹谁了?
慈禧太后越想越气,恨不能杀几个人来泄心头之愤。在光绪皇帝、各王公大臣和奉旨请安的各省总督巡抚面前,她尽管勉强挤出一点笑容,但心里却依旧十分不痛快。在贺寿的朝拜仪式草草结束后,各王公大臣退了朝,慈禧太后便把光绪皇帝叫到一边,狠狠一顿痛骂,骂皇帝无能,连个日本人的事情都搞不定;随后,慈禧太后又骂前方将士作战不力,搅了自己的兴头。总而言之,没一个好东西。
挨了慈禧太后一顿痛斥后,光绪皇帝也十分窝火,随后便给李鸿章发电报,也是对他一顿痛骂。此外,他让李鸿章赶紧想办法,怎么对付日本人。李鸿章接到电报,也是一筹莫展,只得电令丁汝昌率北洋舰队前来天津商议战事,同时特令北洋舰队的顾问、德国人汉纳根也前来参会。
11月10日,丁汝昌率北洋舰队来到天津,商议前往旅顺运兵和保战的事宜。在会议上,李鸿章满脸乌云,脸色阴沉,他尚未开口,便拿出一张刚刚收到的朝廷谕旨递给丁汝昌,让他先读。
在谕旨中,光绪皇帝大发雷霆:“旅顺防务紧要,电饬李鸿章激烈守御,至今不见一字回奏。‘定远’各船,前奏已经修好,今日来电又称尚未修妥。‘来远’亦只修一半,不知丁汝昌两月以来,所司何事,殊堪痛恨!若‘定远’‘来远’两船有失,即将丁汝昌军前正法!李鸿章当遵旨办理,不得再为捏饰。旅顺援兵仍着设法运送,不得因来往冒险而漠视不救!”
丁汝昌看后脸色煞白,他长叹一声道:“中堂大人,卑职现在真是有苦难言,都快成千古罪人了!”李鸿章摆摆手,闷声道:“现在埋怨这些又有什么用?朝廷现在是要我们赶紧往旅顺运兵,你们说说,怎么办吧?”
这时,一起参会的德国军事顾问汉纳根起身道:“中堂大人,现在金州和大连已失,日本舰队聚集于大连湾,北洋舰队现在也只有‘定远’和‘镇远’两舰还有威慑力,假如要护送运兵船的话,北洋舰队一身难担两任。万一日本舰队来袭,届时运兵船被毁且不说,就算是北洋舰队,恐怕也是自身难保,还请中堂大人三思哪。”
李鸿章听后不免有些不快,于是便转向丁汝昌,想听听他的意见。丁汝昌迟疑了一下,道:“中堂大人,汉纳根先生的谏议也不无道理,如果一定要北洋舰队护送运兵船,的确多有不便。与其这样,倒不如让卑职率舰队赴旅顺口巡弋,遇敌即击,相撞即攻,也好随机应变。”
李鸿章心里也清楚,北洋舰队现在实力不济,又见丁汝昌等人不愿冒险护送运兵船,只得同意他们的意见。会议到最后,李鸿章道:“既如此,运兵一事改由商船由内海运往营口登陆,免得遭遇日本舰队,反为所害。北洋舰队应速速前往旅顺,遥为牵制。”
说罢,李鸿章起身来到丁汝昌身边,低声道:“禹廷(丁汝昌的字)啊,如今国事艰难,你切不可泄气,我们这些作臣子的,还有什么可分辩的呢?只能听天命而尽人事了!”
丁汝昌听后百感交集,他迎着李鸿章复杂的眼神,似乎也已经感觉到,此行前途未卜,这可能是自己和老上司的最后一次见面了。
是的,这是丁汝昌与李鸿章见的最后一面,但令李鸿章与丁汝昌没有想到的,他们苦心经营的北洋舰队,也即将要沉没了!


\chapter{“镇远”舰遭遇重创}

天津会议结束后,丁汝昌未做任何停留,便于当晚率舰队前往旅顺。
13日清晨,舰队安全抵达旅顺。
此时的旅顺口,还没有等到开战,就已经完全乱了套,陷入了兵荒马乱之中。当时驻守旅顺的兵力大概有14000多人,却有7个总兵分别统领,彼此间互不相属,难以统一指挥。当时负责协调指挥的是前敌营务处道台龚照玙,但龚道台是个文官,不懂军事,于是各将领又公推年龄最长的毅军统领姜桂题为首领,可惜这老姜是个赳赳武夫,大字不识一箩筐,对他也不能有太高的指望。
身为旅顺最高行政长官的龚道台,急得像是热锅上的蚂蚁,他见丁汝昌率北洋舰队来后,就像见了救命稻草一样,一路上就知道诉苦:“丁军门啊,你可算是来了!日本兵从3个方向朝旅顺进逼,旅顺已成瓮中捉鳖之势,如何是好,如何是好啊?”
丁汝昌见龚道台急成这样,只好故作镇定地安慰他:“龚大人不要太着急,中堂大人已经调遣了嵩武军8个营前来增援,不日即可赶到。”
这时,姜桂题和其他几个总兵听说北洋舰队来旅顺了,也急忙赶来相见。他们都是些军中老粗,见到丁汝昌后,劈头便问:“丁军门,援兵啥时到啊,我们都快顶不住了!”
援兵什么时候到,丁汝昌哪里敢打包票?!他只得敷衍道:“快了快了,中堂大人已经在调派了。”
姜桂题急了:“调派调派,这旅顺都已成绝境了!丁军门你看,日本兵比我们多一倍,敌众我寡,咱们要是没有后援,这仗可怎么个打法?”
其他几个总兵听后,也跺脚的跺脚,拍桌子的拍桌子,连连喊着要援兵。丁汝昌和龚照玙两人你看看我,我看看你,毫无办法……总不能给他们撒豆成兵吧?
这时,“定远”舰管带刘步蟾快步走了进来,向丁汝昌秘密报告了一事。丁汝昌听后十分焦躁,和众人敷衍了几句后便拱手告辞,随后同刘步蟾起身快步离去。各总兵见后,面面相觑,不知丁汝昌意欲何为。
刘步蟾给丁汝昌带来了一个不妙的消息:在旅顺口外不远的海面上,北洋舰队的哨兵已经发现了日本鱼雷艇的踪迹。丁汝昌听后顿时变得十分紧张,因为旅顺口十分窄小,一旦日本舰队封锁港口,北洋舰队将成瓮中之鳖,既行动不便,又难以冲出口外,必然要吃大亏,于是他决定:当晚即返航威海卫。
龚照玙听说丁汝昌当天来、当天就走,十分惊讶,快步赶到码头去找丁汝昌。他们赶到码头时,北洋舰队已经升火起锚,马上就要出发了,所幸的是,丁汝昌还没有上船。龚照玙小步快跑、满头是汗地追了上去,他一把扯住丁汝昌,上气不接下气地问:“丁军门,你,这就走?!”
丁汝昌含糊其辞地说:“嗯啊,我们先走,过几天还要来的。”
龚照玙急得要命,他可怜巴巴地抓住丁汝昌的手说:“丁军门,你们可一定要回来啊!”
丁汝昌面无表情,说:“龚大人,旅顺要塞,防务要紧,你可要多保重啊!”
随后,丁汝昌便登上“定远”舰,下令起航。龚照玙愣愣地站在码头上,呆呆地看着北洋舰队在“定远”舰的率领下,一艘接着一艘,缓缓驶离了旅顺口,最后消失在茫茫大海中。
夜色深沉,海天弥阔,连一丁点星光都看不到。在微弱的航行灯指示下,“定远”舰在最前面领航,其他各舰都紧随其后,向威海方向驶去。
海面上,一个又一个的涌浪咆哮着扑来,“定远”舰用巨大厚重的舰首劈开海浪,在一声声的“哗哗”巨响中,舰船的周围卷起了如雪的海沫。
整个北洋舰队都在沉默的黑暗中行进,水兵们静静地坐在船舱里,听到战舰桅杆上的军旗被强劲的西北风吹得嘶嘶作响。漆黑的海面上,不时闪现一丝白沫,似乎有海的精灵要跳跃而出。负责了望的水兵紧张地盯着海面四周,以提防日本舰队突然出现。这一夜,就像是风暴即将来临之时。
度过紧张的一夜后,北洋舰队终于在第二天凌晨平安无事地来到了威海港外。面对自己熟悉的母港,水兵们总算是心定了不少,丁汝昌的眉头也稍微松弛了些,他随即给信号兵下达命令:“各舰按原编队进港!”
由于这时正值枯水期间,进港又遇到落潮的时候,海水显得比平时要浅很多,很多原本看不见的礁石也露出了黑色的脊背,它们一直在无声无息地守望着这个军港。
按老规矩,“定远”舰在前,“镇远”舰其次,其他军舰从东口相继进入威海湾。这时,“镇远”舰的管带林泰曾也来到甲板,指挥着“镇远”舰行进。旁边的大副杨用霖聚精会神地看着不远处的浮标,突然,他脸上有点疑惑,便对林泰曾说:“咦,这浮标好像不对劲吧?”
林泰曾听后,拿起望远镜查看,说:“对啊,昨日风大,浮标好像是被风吹得偏移了。”
两人正说着,“镇远”舰底部突然一阵剧震,同时海底传来一阵极其刺耳的“喀哧”声,似乎军舰与海底的礁石相碰了。林泰曾顿时脸色大变,把望远镜一扔,高声喊道:“快,下各舱室检查!”
一片忙乱中,水兵们沮丧地报告说,“镇远”舰不幸触礁,多处受伤,弹药舱、锅炉舱和帆舱都被礁石划了好几条大口子,长的有十七八尺,短的也有五六尺,海水已经涌入了“镇远”舰的底舱。
尽管林泰曾和杨用霖指挥水兵们极力抢救,并最终堵住了这些口子,但“镇远”舰在未经完全修复前,已经不能再度出海作战了。此时的中国,缺乏一个修理大型舰艇的军用船坞,之前北洋舰队的舰艇也是要到日本的长崎去修理!这大概也是日本舰队能够很快恢复战斗力而北洋舰队却做不到的原因之一吧!
丁汝昌得知这个消息后,当下便脸色发黑:这真是“屋漏偏逢连夜雨,船迟又遇打头风!”北洋舰队原本就只有“定远”和“镇远”两艘主力铁甲舰对日本舰队尚有威胁,如今“镇远”舰受伤不能出海,“定远”
舰势单力薄,威力大减,怎不让人着急?
“定远”与“镇远”为同级姊妹舰,皆由德国伏尔铿船厂建造,1885年10月交付中国,两舰耗费了白银340万两,舰名也是李鸿章亲自拟定。两舰的母型是德国的“萨克森”级铁甲舰,炮台布置参考了英国的“英弗莱息白”号铁甲舰,由伏尔铿船厂总工程师鲁道夫·哈克设计,集中了当时世界上这两艘最先进的铁甲舰之优点,号称“全球第一等铁甲舰”和亚洲第一巨舰。
“定远”和“镇远”二舰长94.5米、宽18米、吃水6米,正常排水量7220吨,航速14.5节,装甲总重为1461吨,两舰的主要武器为4门克虏伯305毫米后膛主炮、两门克虏伯150毫米后膛副炮,这在当时的世界海军中,堪称威力巨大。回国后,“定远”舰便成了北洋舰队的旗舰。
按《北洋海军章程》规定,“定远”和“镇远”二舰的管带均为总兵衔(仅次于提督),分别为刘步蟾和林泰曾。刘、林两人都是福州船政学堂的毕业生,之后又留学英国,并在英国地中海舰队中实习,回国后便被李鸿章委以重任。
在这个不幸的事故发生后,“镇远”舰上一片忙乱,工匠们还在尽可能地进行抢修,“叮叮当当”的声音响彻铁码头一带。管带林泰曾和大副杨用霖都在现场指挥,两个人脸色很是憔悴,看起来已经是数夜未眠了。
更令人震惊的是,由于愧疚难当,“镇远”舰管带林泰曾在巨大的心理压力之下,竟然在“镇远”舰受伤后的第三天服毒自尽。消息传开后,水兵们一片骚动,纷纷赶到刘公岛的铁码头为林管带送行。林泰曾在水兵中很有人缘,他虽然是留洋出身,但平时为人和气,很受水兵爱戴,可谁也没有想到,这位海军骄子竟然会走上这样一条不归路。
丁汝昌闻讯后,也急忙来到铁码头为林泰曾送行。在人群中,“定远”舰的刘步蟾、“来远”舰的邱宝仁、“靖远”舰的叶祖珪等人,这些原福州船政学堂的老同学也都来了。当林泰曾冰冷的遗体从“镇远”舰抬下来的时候,他们心里也在悲痛地翻腾:“泰曾啊,你这又是何苦呢?难道这就是北洋舰队的命运吗?”
据说,林泰曾自杀前,曾对大副杨用霖说:“朝廷一再严令我们出海作战,可如今又出了这事!朝廷里那些当官的,会怎么说我们呢?!弄不好,他们得说我们是有意触礁,为惧不出战找理由呢!这真是跳到黄河也洗不清啊!”
舰在人在,舰亡人亡。林泰曾本就是一个内向的人,他的压力实在太大了。
在这些海军将领的心目中,舰艇的存亡是比天还大的事情。国家花费了巨资购买这些舰艇,这是多少老百姓的血汗钱啊!而且,朝廷也有极不符人情和实际的规定,那就是,舰艇一旦有失,将领必须负全责,这也是黄海大战中一些管带宁可以身殉国的原因之一……他们宁可在战场上光荣地死去,也不愿意以戴罪之身虚度自己的后半生!
这,也应该算是甲午海战失败的原因之一吧!清廷不能爱惜自己费尽千辛万苦而培养出来的人才,如何能让将士们在残酷的战争中放下包袱呢?
45岁的林泰曾就这样走了,他怀着辜负国恩的内疚,本着与战舰共存亡的信念和压力,永远地走了。当夜,被灯火管制的刘公岛上一片漆黑,唯有岛东的礁盘上,火星点点,在黑夜中显得格外凄迷,这是“镇远”舰的水兵们在为他们的林管带焚香送行……


\chapter{战云笼罩下的旅顺}

北洋舰队撤离后,旅顺局势便更加紧张了。
11月18日上午,在旅顺外围十几里外,在一片已经收获了的玉米地里,突然传来了一阵窸窸窣窣的声音。不久,从高高的尚未颓败的玉米秆子中,钻出了一队日本的骑兵。原来,这是日军骑兵第一大队长秋山好古少佐所带的骑兵搜索队,其奉命前来侦察旅顺的防御情况。
由于在行进途中经常遭到当地人的袭扰,令日本兵疲惫不堪,有时候一个不小心,还会掉到猎人有意设置的陷阱中,比如之前就多次发生过连人带马都跌进插满荆棘的坑中之事,当场死伤多人。因此,日本兵在前进的过程中十分谨慎,并不敢横冲直撞。
当日本兵小心翼翼地来到一个名叫“土城子村”的地方时,突然听到前面的树林里似乎有动静,日军指挥官秋山好古见势不妙,便立刻命令搜索队准备战斗。不料未等日军准备好,前方突然传来一阵嘹亮的军号和喊杀声。原来,徐邦道和卫汝成所率的部分军队经过这里,正好与日本的骑兵搜索队不期而遇。
惊慌之下,秋山好古急令各士兵迅速下马,寻找有利位置进行防守。
走在前面的骑兵中队长浅川敏靖大尉探知前方清军有步兵300多人后,慌忙向大队长秋山好古请求撤退。秋山好古认为这是有利地形,不准其撤退,浅川敏靖无奈之下,只好带领手下士兵拼死反击。
双方正激战中,清军的一队骑兵突然从日军的侧后方发起冲击,日军这下阵脚大乱,浅川敏靖见势不妙,慌忙上马想逃走。正当他跳上军马准备逃窜的时候,远处的清军中飞过来一颗子弹击中其肩膀,浅川敏靖“嗷”的一声大叫,立刻翻落马下。在这危急时刻,他手下的一名日本兵抢过来将浅川敏靖扶上自己的战马,浅川敏靖这才得以逃走。
就在这时,清军的骑兵队也已经从正面冲了过来,他们挥舞着马刀,向负隅顽抗的日军头上狠狠砍去。在第一波冲击后,被包围的日本兵害怕被俘受辱,纷纷自刎的自刎,剖腹的剖腹,尸体狼藉一地。
由于徐邦道的拱卫军和卫汝成的成字军陆续投入战斗,清军在数量上占据了绝对优势,而且士气极为旺盛。一时间,战场上硝烟迷漫,笼罩原野,炮声如雷,弹如雨下,日军在清军的进攻下极为狼狈。秋山好古见清军势不可当,只得下令全体撤退,不料此时日军骑兵已经被三面包围,进退维谷,死伤惨重。秋山好古这下才知道大事不好,情急之下,只得喝令手下士兵拼死突围,向双台沟方向奔逃。
好在日军前卫第三联队听到枪炮声后及时赶到双台沟,这才接应了正在溃逃的秋山好古残部。这时,清军又从土城子方向攻来,而日军在双台沟高地早已选好防御阵地,双方你来我往,又展开了激烈的对攻。
由于日军随后又来了几个中队增援,双方连战了近6个小时,清军见久攻不下,又见日落西山,统领徐邦道和卫汝成商议后,便决定徐徐退兵,暂时撤出战场。日军见清军枪炮渐稀,便也缓缓地退走。在步兵的救护下,日本骑兵第一大队虽然勉强突围,但已经是损兵折将,在回撤的路上,马蹄声残,伤兵哀号,大队长秋山好古却长舒了一口气,为自己这次能够逃命感到十分的庆幸。
土城子迎击战是保卫旅顺的前哨战,清军在优势兵力下,总共打死打伤日军近百人。经此一战,日军不敢再掉以轻心,于是开始纠集兵力并准备大举反扑旅顺。
旅顺与威海卫隔海相望,共扼渤海的门户,一向是海防重镇。在李鸿章多年的经营下,旅顺成为了北洋舰队的船坞和补给所在地,在十多年间,旅顺军港总共耗费了上千万两白银修建、加固,其易守难攻,号称是“远东第一要塞”,与威海一起拱卫渤海和京津。
旅顺港口的入口窄小,最狭处宽仅9丈,但内港水深两丈以上,周长大约14里,可停泊大型的铁甲兵舰,是中国北方海域极为难得的天然军港。
在旅顺军港修建完成后,北洋舰队在出海操练时,一般都会到此修理舰只,或者补给燃料和舰炮武器等。
为了增加防守能力,旅顺周围的要害位置还修有大量炮台,并装备了一百五十多门从德国进口的各式克虏伯大炮,德国退役军官汉纳根便在设计建造中起了很大的作用。甲午战争爆发前,李鸿章曾信心满满地夸口说:“旅顺有充足的弹药、军粮,有优良的火炮和北洋舰队的声援,坚守三年没有问题。”
但是,随着战事的发展,李鸿章越来越乐观不起来了。由于甲午战争的突然爆发,加上朝鲜战事的需要,炮台守军被无序调度,比如旅顺后路的各炮台,原本是由提督宋庆的毅军驻守,后来毅军被调往鸭绿江防线,导致旅顺后路极为空虚。
拆东墙补西墙终究不是办法,为了应急,李鸿章只能命总兵姜桂题和记名提督程允和临时各招募了四营兵力,分别驻守椅子山、案子山、望台北、松树山、二龙山、鸡冠山及蟠桃山等炮台。如此一来,姜桂题和程允和的这8营虽然号称有4000兵力,但大都是些未经战阵的新兵,训练和经验都明显不足,因而战斗力也就无从谈起,难以依靠。
甲午海战后,李鸿章也意识到了这个问题,于是在11月初的时候再次调派记名提督卫汝成率成字军五营及一小队马队乘轮渡增援旅顺,以加强其后路的防御。新来的成字军大约有3000兵力,他们到旅顺后主要驻守在白玉山东麓一带,作为旅顺后路的总预备队。
在日军登陆花园口并进攻金州后,其攻占旅顺的战略意图已经十分明显,这时,从金州和大连湾撤回的连顺、徐邦道及赵怀业等部也都齐聚旅顺,这样一来,虽然看似兵力稍壮,但实际上仍旧是岌岌可危。
连顺所部兵力原本就不多,加上在金州保卫战中损失惨重,撤到旅顺时已经是所剩无几,因而连顺在旅顺保卫战打响之前便率领残部乘船离开旅顺前往复州,实质上并未参战;徐邦道的拱卫军在金州激战中也损失颇大,撤至旅顺时大概还剩下一千四百多人;赵怀业的怀字营在撤离大连湾时有所减员,大概还剩1800人。如此算来,防守旅顺后路的兵力有一万多人,加上防守海岸炮台的黄仕林和张光前部4000人,清军在旅顺的总兵力有14000多人。
从兵力上来看,旅顺守军和日军相比并不占优势,从战斗力来看,就更处于下风了。更糟糕的是,驻守旅顺的各统领互不隶属,无法统一指挥。旅顺前敌营务处兼船坞工程总办龚照玙被李鸿章令其以道员身份代北洋大臣节度,但龚本人系文官出身,他既缺乏军事才能和经验,为人又贪鄙庸劣,胆小怕事,难以服众。
就在金州失守的当晚,龚照玙在既没有请示李鸿章,又没有和驻守旅顺的诸将商议的情况下,便私自以“商运粮米”的名义乘旅顺港的鱼雷艇跑到烟台,差点被山东巡抚以“临阵脱逃”之罪就地正法。后来,龚照玙又跑到天津,结果又被李鸿章逮住一顿臭骂,并严令其即刻返回旅顺,否则,“离旅顺一步则为汝死所!”在李鸿章的盛怒之下,龚照玙吓得飞也似的又逃回了旅顺。
龚照玙离开旅顺的时候,旅顺破天荒地出现了7个统领,即姜桂题、张光前、黄仕林、程允和、卫汝成、赵怀业和徐邦道,这七统领不相统属,在龚照玙走后更是各自为政,混乱不堪。后来统领张光前觉得大敌当前,如果各军再各行其事的话,旅顺必然难保,于是便与黄仕林、程允和、卫妆成等人商量,公推了年长的姜桂题为总统,各军约定,一切听姜桂题的调度,同心协力,支持大局。可惜的是,姜桂题虽然出身行伍,但也不是什么大将之才,不过徒长几岁而已。
面对困境,姜桂题唯一的办法,便是一味地告援,可李鸿章决定调派的嵩武军八营却因战争形势极为紧张而无船敢冒险运送,最后不了了之。
姜桂题等人“盼星星、盼月亮”般地等待援军,但最后还是一场空,只能依靠自己。
尽管守卫旅顺的清军实力不济,但日军对旅顺的险要地形及众多的炮台还是颇为忌惮。在进攻旅顺前,第一师团长山地元治曾召集各将佐开会商讨进攻旅顺事宜,会前,其副官预先编制了一份500人的敢死队名册,山地元治看后将之退回,说:“旅顺要塞,难以攻打,这点人数不够。”副官听后便又增加500人,山地元治看后,仍摇头说“不够,不够”。直到将敢死队人数增至1500人,山地元治这才点头认可。
进攻的前夜,山地元治向各参战部队分配任务,令全军于11月21日进军旅顺。当日凌晨2时,日军各部队点燃篝火,做好各项进攻的准备后,便悄悄地向指定阵地前进了。


\chapter{日军在旅顺的暴行}

21日凌晨6时,天色微亮,第二旅团长西宽二郎率领所部逼近椅子山炮台的西北面,准备发动进攻。椅子山炮台高出地面50多丈,地势险要,可以俯视周围,因此也成为日军进攻的第一个目标。由于烟雾茫茫,日军难以辨别目标,因而进攻反而由海上的日本联合舰队首先发动。在旅顺附近的海面上,日本舰队一字排开,向炮台群轮番放炮,试图从海面上吸引清军的注意力,以帮助日本陆军进攻。
日本军舰开炮后,日军第二旅团也拉来40多门攻城炮、野炮和山炮,随后向椅子山炮台猛烈地发炮轰击,意图在气势上压倒清军。战斗打响后,椅子山炮台和周围的松树山、黄金山等炮台上的清军也毫不示弱,他们分别用岸防大炮还击日军,只听两军阵地上炮声隆隆,响彻云天,似有天崩地裂之感。
在炮火的掩护和指挥官的逼迫之下,大队日军逐渐逼近了椅子山炮台,并沿着崎岖的山路陆续蚁附而上。防守清军在日军的围攻下竭力抵抗,他们从各个角度向日军不断射击,但由于日军人数众多,清军的防线最终还是被突破了。
在日军即将攻上炮台的危急时刻,清军士兵在营官的鼓舞下,纷纷拔出大刀和日军展开了生死白刃战,炮台上顿时刀光剑影,兵刃相接,战斗极为惨烈。日军指挥官见炮台久攻不下,随后又调派预备队投入战斗。日军不断增援,坚守炮台的清军终于支持不住了。最后,清军只得放弃椅子山炮台并向周围炮台撤退。日军此战势在必得,攻势非常凶猛,最后旅顺后路西炮台群的椅子山、案子山及望台北诸炮台都先后被日军攻陷,清军残部战败后被迫撤向旅顺西海岸。
日军攻占椅子山炮垒群后,又迅速转攻二龙山炮台和松树山炮台。当时对二龙山炮台发动进攻的是日军混成第十二旅团,在人数上占有绝对优势。面对黑压压一大片日军,守台清军奋起反击,炮台阵地前的炮声如万雷齐鸣,山上山下,硝烟弥漫,战斗极为激烈。在战斗进行中,由于日军逐渐接近了炮台,大炮失去有效射程,防守炮台的清军只得用步枪扫射蜂拥而上的日军。所幸,周围的炮台见二龙山炮台形势危急,便从侧面不断发炮支援,将进攻的日军轰下去不少。
由于侧面松树山炮台的威胁很大,日军第十二旅团只得暂时停止进攻二龙山炮台,而改派一个分队向松树山前进,佯攻松树山炮台,以诱使其停止对二龙山方面的炮击。而在这时,日军第一师团也已赶到并开始了对松树山炮台的进攻,他们摆出数十门野炮向松树山猛烈发射,其炮兵联队也用攻城炮不断发炮,日军的攻势变得极为凶猛。更加不幸的是,松树山炮台的火药库没过多久便被日军的一枚炮弹击中,炮台内顿时爆炸连连并燃起熊熊大火,在此情况下,守台清军只好放弃松树山炮台而撤向二龙山。
松树山炮台沦陷后,日军合兵两路向二龙山炮台发起冲锋。防守二龙山炮台的是姜桂题指挥的桂字营,他们用新买进的克虏伯大炮和格林炮不断猛轰日军阵地,并以步枪俯射冲到炮台下的日本兵,给予敌人大量杀伤。但由于日军人多势众,在指挥官军刀的逼迫之下,他们仍旧不要命地向炮台不断攀登冲击。有一次,当日军攀至炮台前沿时,清军引爆了之前预埋的地雷,只听“轰”的一声,方圆几十米的日军都被炸上了西天。
日军的多次冲锋受阻后,简直杀红了眼,继续跨过累累尸体,逼近炮台。当时的场面非常惊人,日本兵狂呼乱叫,几乎布满山野,他们从四面向炮台不断攀登,蚁附而上。姜桂题见自己的士兵伤亡太重,炮台又是难以防御,只得率部撤下炮台,突围而出。这时,有一名勇敢的士兵志愿留下,他在大队日军攀上炮台的刹那间,在弹药库里引爆了地雷,只听天崩地裂的一声巨响,首先登上炮台的日军全部命丧黄泉。
部分日军进攻二龙山炮台的同时,日军第十四联队也向鸡冠山炮台发起了进攻。在激战中,联队第一大队的大队长花冈正贞少佐的臀部被击穿,因失血过多而毙命。在付出沉重的代价后,日军在当天晚间最终攻占了鸡冠山炮台及附近临时搭建的一些炮台。
陆路炮台相继失守后,各统领相继率残部突围而出,卫妆成和赵怀业向金州方向而去;徐邦道所部在与日军激战教场沟后冲出重围;程允和与姜桂题残部也都先后突围而出。可怜的道台龚照玙则乘渔船逃往烟台。日军占领旅顺后路炮台后,便转而向黄金山海岸炮台发动进攻。黄金山炮台备有大口径火炮,并且可以旋转360度,可以八面射击,可惜守军军心已乱,炮台很快便失陷了。
在战争之前,时人评价威海与旅顺时说,“铁打的旅顺,纸糊的刘公”(即刘公岛,指威海卫),但出乎人们意料的是,不到一个礼拜,旅顺便全部失陷。这一点,甚至连日本人也没有料到,事实上,日军的伤亡人数远远不及开始制定的敢死队人数。
究其原因,恐怕还是因为守卫清军训练不足,对现代炮台的使用远远不够熟练,未能发挥其真正的作用。在10年之后的日俄战争中,俄军就利用旅顺要塞(当时旅顺租借给了俄国人)给予了日军沉重的打击,令日本人付出了沉重的代价。当然,俄军在后来十年中通过添设炮台等措施加强了旅顺要塞,但其训练水平、战斗力等方面显然是远在清军之上的,何况当时的清军中有很大一部分是招募不久的新兵,这就令旅顺的守卫能力更是大打折扣。实事求是地说,清军士兵在战斗中不能说不勇敢,但这毕竟是一场现代战争,光靠勇敢是远远不够的。
日军攻占旅顺后,很快便暴露出了疯狂的兽性,旅顺随即陷入了一场空前的浩劫之中。日本兵冲入旅顺后,不分青红皂白,在街上见人就杀,见东西就抢,杀人放火,无恶不作,甚至连无辜的老人、妇女、儿童都不放过。后来,为了掩盖他们的罪行,日本兵甚至故意纵火,将杀人抢劫的现场毁尸灭迹,其暴行令人发指。
对于日军在旅顺的暴行,当时世界上有很多媒体进行了报道和严厉谴责,比如英国《泰晤士报》就根据其本国武官的叙述而刊载了这样的新闻:“日本攻取旅顺时,杀戮百姓达四日之久,这种非理性的杀伐,甚为惨伤。据记者所见,日本绑缚数群清兵,先开枪打死,然后还残忍地用刀肢解,日本士卒如此残暴的行径,而那些指挥官也熟视无睹。”
另一目击者英国人艾伦,当时随美国货轮“哥伦布”号赴华为正在同日本作战的清军运送军火。在旅顺大屠杀期间,他被困于旅顺口,他在日记中记载说:“在我周围都是狂奔的难民。我第一次亲眼看见日本兵追逐逃难的百姓,用枪杆和刺刀对付所有的人,对跌倒的人更是凶狠地乱刺。
日军击毙所有遇见的人,在街道上行走,脚下到处踩着死尸。天已经黑了,屠杀还在继续进行着,丝亳没有停息的迹象。枪声、呼喊声、尖厉的叫声和呻吟的声音,四处回荡。街道上呈现出一幅可怕的景象:地上浸透了血水,遍地躺卧着肢体残缺的尸体,有些小胡同简直被死尸堵住了,死者大都是城里人。”
艾伦又写道:“我站在一处高地,离我不远处有一个池塘,池塘边站着好多日本兵,拼命将一群难民往池塘里赶,不一会儿池塘里便塞满了人。只见难民在水里乱成一片,池塘边的日本兵,有的拿枪射击,有的用枪上的刺刀刺。池塘里断头的、斩腰的、穿胸的、破腹的,搅成一团,水变成通红一片。日本兵在一旁欢笑狂喊,快活得不得了。池塘里少数活人,在死尸上爬来爬去,满身血污。其中一个女人,抱着一个小孩子,浮出水面,朝日本兵发出凄婉的哀求。岸边的日本兵竟拿刺刀来捅,当胸捅了个对穿。第二下又捅那个孩子,只见刺刀一捅,小孩子被捅到刺刀上,他高高地挑起枪来,摇了几摇,当作玩耍的东西。那女人伏在地上,尚未被捅死,她用奄奄一息的力气,想要起来看看那个孩子,刚挣扎了一下,又趴下了。日本兵就照屠杀别人的方法,也将这个女人斩成几段。”
美国驻日本武官海军上尉欧伯连当时也正在旅顺,他在写给该国驻日公使谭恩的报告中说:“我曾亲眼看见一些人被屠杀的情形。有一些尸体,双手是绑在背后的。我也看见一些被大加屠割的尸体上有伤,从创伤可以知道他们是被刺刀杀死的,从尸体的所在地去看,可以确定地知道这些死的人未曾抵抗。我看到了这些事情,并不是我专为到各处看可怖的情况才发现的,而是我观察战事的途中所看到的。”
美国《纽约世界》的记者克里曼在旅顺大屠杀的第四天,将自己的目击记录发回国内:“我亲眼看见旅顺难民并没有抗拒日军。日本人所谓有人从窗口向日军射击之言,全是谎话。日本人并不想抓俘虏。我见到一个人跪于日本兵面前,磕头饶命。这个日本兵用刺刀将他的头钉在了地上,然后拿刀将他的头砍了下来。有一个老百姓吓得缩在墙角,一队日本兵发现了,一人一枪将其击碎。有一个老人跪于街中,日本兵将他斩成两段……”
在这次惨绝人寰的旅顺大屠杀中,日军共杀害国人20000多人。当时能够幸免于难的中国人,只有36人,这还是日军为驱使他们掩埋尸体而留下的,在这些人的帽子上,粘有“勿杀此人”的标记,才得以免死。
世界自有公道。在战争爆发之前,欧美各国都对日本学习西方文明而表示称赞,而对中国的保守落后颇有微辞。但在日军制造旅顺大屠杀后,美国的《世界》杂志就曾愤怒地谴责道:“日本是披着文明的外衣,实际是长着野蛮筋骨的怪兽!日本于今摘下了文明的假面具,暴露了野蛮的真面目。”
这无疑是当时世界舆论的公论。


\part{威海外围枪炮急}

\chapter{言官弹劾丁汝昌}
\chapter{外援军舰在哪里?}
\chapter{将士齐心,威海布防}
\chapter{挟奇技投效的洋人}
\chapter{打仗难,议和更难}
\chapter{兵来将挡,威海卫加紧防守}
\chapter{误判敌情,戴宗骞轻言分兵}
\chapter{声东击西,日军诈攻登州府}
\chapter{白浪滔天,日舰夜袭荣成湾}
\chapter{大雪茫茫,日军荣成湾登陆}
\chapter{大军压境,日军轻取荣成县}
\chapter{白马河之战}


\part{炮台大血战}

\chapter{将帅不和,战前风波}
\chapter{日军进逼南帮炮台}
\chapter{周家恩坚守摩天岭}
\chapter{炸死日军旅团长}
\chapter{皂埠嘴炮台失守}
\chapter{敢死队摧毁皂埠嘴炮台}
\chapter{炮击鹿角嘴}
\chapter{风雪漫天,日舰退兵}
\chapter{阎得胜军前正法}
\chapter{士兵哗变,自毁北帮炮台}


\part{激战刘公岛}

\chapter{刘公岛上的年三十}
\chapter{日军剑指刘公岛}
\chapter{日军一战刘公岛,英舰司令代日劝降}
\chapter{日本鱼雷艇深夜来袭裴}
\chapter{日军二战刘公岛,萨镇冰坚守日岛}
\chapter{日本鱼雷艇再次偷袭}
\chapter{日军三战刘公岛,无功而返}
\chapter{日军四战刘公岛,鱼雷艇队突然出逃}
\chapter{萨镇冰放弃日岛炮台}
\chapter{日舰破坏拦坝,五战刘公岛}

\part{龙旗飘零}

\chapter{刘公岛人心渐乱}
\chapter{烟台求援}
\chapter{援兵在哪里?}
\chapter{自毁“定远”舰,刘步蟾悲愤自杀}
\chapter{炮火纷飞,日军六战刘公岛}
\chapter{夜渡威海湾,回报援军无望}
\chapter{绝处难逢生,丁汝昌自杀身亡}
\chapter{屈辱的投降}
\chapter{日军入港接收残余舰艇}
\chapter{泪别威海}

\backmatter

\chapter{尾声 难以下葬的丁汝昌}

\end{document}