\documentclass[UTF8]{ctexbook}
\usepackage{graphicx}
\usepackage{grffile}
\usepackage{url}
\usepackage{float}
\usepackage{booktabs}
\usepackage{array}
\usepackage{hyperref}
\usepackage{geometry}
\geometry{a4paper, left=2.5cm, right=2.5cm, top=3cm, bottom=3cm}
\usepackage{titlesec}
\usepackage{enumitem}

\begin{document}

\chapter{无线相关技术介绍}

\section{WIFI6技术概述}

Wi-Fi 6(802.11ax)是第六代无线局域网技术,提供更高的带宽、更低的延迟和更好的多设备连接能力。

\section{Wi-Fi 7 下一代无线标准}

Wi-Fi 7(802.11be)是最新的无线局域网技术标准,进一步提升了传输速度和网络容量。

\chapter{AC + AP 组网方案}

\section{AC + AP方案基础概念}

AC + AP方案一直是企业无线网络的首选方案,也是目前唯一能满足企业级稳定和性能需求的方案。家用路由器无法满足大型网络的需求。

\subsection{AC:无线控制器}

负责统一控制整个无线网络,进行网络功能设置,有独立AC和AC一体机两种形式:

\begin{itemize}
  \item \textbf{独立AC}:一般作为旁路部署,只进行网络配置和无线功能的计算,不参与数据转发,负载较低。
  \item \textbf{AC网关一体机}:需要进行数据转发,属于网络的主要设备,复杂度高,运算量大。一旦出现问题,整个网络都无法正常运行,但优势是组网成本低,很适合家用。
\end{itemize}

\subsection{AP:无线访问点}

只负责网络连接,没有其他多余功能,硬件设置主要围绕无线射频进行:

\begin{itemize}
  \item 负载根据连接的终端数决定
  \item 使用多AP方案,单个AP覆盖不会很高,依靠数量解决覆盖不足和负载过高问题
  \item 单个AP出现故障时,终端可以移动到其他AP,影响较小
\end{itemize}

\subsection{POE交换机}

带有POE供电功能的交换机,为AP提供数据传输和电力供应,是网络中的核心设备。一旦出现故障,上面连接的所有AP都会离线。

家庭使用中,一般可以使用AC + 网关 + POE的一体机,节省成本。如果一体机上网口不足,再增加独立的POE交换机。

\section{AC + AP方案的优势}

在AC+AP网络中,可以实现以下复杂网络功能:

\begin{itemize}
  \item 调节AP的无线功率,控制AP的无线覆盖范围
  \item 调整无线终端的漫游体验
  \item 设置多个SSID,设置多个VLAN
  \item 设置访客专用无线,设置访客的Portal认证
  \item VPN设置等
\end{itemize}

AP的漫游体验更好,这是用户体验最直观的优势。AP只需要网线,不需要电源线,网线同时提供数据传输和电力供应,在部署上更简单和安全,只有弱电工程,没有强电工程。

AP更美观,企业一般使用吸顶AP,家庭一般使用面板AP,没有外露的天线,外形美观小巧。

\section{AC + AP组网拓扑}

\subsection{独立AC组网拓扑}

AC作为独立设备部署,负责AP管理和数据转发。

这种是普遍的组网方式,AC是独立的,即使在企业环境中也是如此。绝大多数AC负责AP管理和集中的数据转发,所以AC在很大程度上决定整个网络的性能。

\subsection{AC与路由一体机拓扑}

将AC功能与路由集成,节省硬件成本。

部分品牌将AC功能与路由做在一起,节省一个硬件AC的费用。也有的品牌使用软件AC,可以安装在软路由或者PC中,只用来管理AP,不进行数据转发,不是网络中的性能瓶颈。

\subsection{路由+ AC + POE一体化设备拓扑}

将三种功能集成在一个设备中,简化网络结构。

有的品牌将AC、路由、POE功能做在一起,节省POE交换机费用,简化网络拓扑,适合小型网络环境。大部分家用AC+AP产品都会采用这种硬件一体机方案。

\section{AP的分类}

\subsection{按工作方式分类}

\begin{itemize}
  \item \textbf{胖AP(FAT AP)}:独立工作,相当于一个独立的无线路由器,不需要AC。
  \item \textbf{瘦AP(FIT AP)}:需要搭配AC才能工作,通过AC进行统一管理。一般都使用瘦AP模式。
\end{itemize}

\subsection{按安装位置分类}

\begin{itemize}
  \item \textbf{吸顶/壁挂AP}:可以吸顶安装或者壁挂安装,AP周围很少有遮挡,体积较大,性能更强,信号覆盖范围更好。
  \item \textbf{面板AP}:大小是86面板,可以安装在墙上暗盒中,预留网线和设计点位时需要注意。受限于86面板,性能稍弱,信号覆盖范围较小。
\end{itemize}


吸顶AP:安装在天花板上,无外露天线,覆盖范围广。


面板AP:86面板尺寸,安装在墙面上,美观小巧。

\section{AC+AP组网所需设备清单}

在家用环境下,可以选择AC+POE+网关的一体机,这样就可以3个功能合并为1个设备,节省空间和费用。

\begin{table}[htbp]
\centering
\begin{tabular}{|l|l|}
\hline
\textbf{设备类型} & \textbf{说明} \\
\hline
AC控制器 & 独立AC或AC一体机 \\
\hline
POE交换机 & 为AP提供数据和电力 \\
\hline
AP设备 & 吸顶AP或面板AP \\
\hline
网线 & 超五类及以上 \\
\hline
其他配件 & 水晶头、扎带、标签等 \\
\hline
\end{tabular}
\caption{AC+AP设备清单}
\end{table}

\end{document}