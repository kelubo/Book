% book模板
% book模板.tex

\documentclass[a4paper,12pt,UTF8,twoside]{ctexbook}

% 设置纸张信息。
\RequirePackage[a4paper]{geometry}
\geometry{
	%textwidth=138mm,
	%textheight=215mm,
	%left=27mm,
	%right=27mm,
	%top=25.4mm, 
	%bottom=25.4mm,
	%headheight=2.17cm,
	%headsep=4mm,
	%footskip=12mm,
	%heightrounded,
	inner=1in,
	outer=1.25in
}

% 设置字体,并解决显示难检字问题。
\xeCJKsetup{AutoFallBack=true}
\setCJKmainfont{SimSun}[BoldFont=SimHei, ItalicFont=KaiTi, FallBack=SimSun-ExtB]

% 目录 chapter 级别加点(.)。
\usepackage{titletoc}
\titlecontents{chapter}[0pt]{\vspace{3mm}\bf\addvspace{2pt}\filright}{\contentspush{\thecontentslabel\hspace{0.8em}}}{}{\titlerule*[8pt]{.}\contentspage}

% 设置 part 和 chapter 标题格式。
\ctexset{
	part/name= {第,卷},
	part/number={\chinese{part}},
	chapter/name={第,篇},
	chapter/number={\chinese{chapter}}
}

% 设置古文原文格式。
\newenvironment{yuanwen}{\bfseries\zihao{4}}

% 设置署名格式。
\newenvironment{shuming}{\hfill\bfseries\zihao{4}}

\title{\heiti\zihao{0} 题目}
\author{WangFei}
\date{\today}

\begin{document}

\maketitle
\tableofcontents

\frontmatter
\chapter{前言、序言}

\mainmatter

% 增加空行
~\\

% 增加字间间隔,适用于三字经、诗文等。
 \qquad  

\chapter{1}

开拓智域

Eric S. Raymond五部曲之Homesteading the Noosphere
开拓智域
by Eric S. Raymond
April 1998

--------------------------------------------------------------------------------
在观察由开放原始码版权所定义的"官方"意识形态与真正玩家的行为後, 发现到一些矛盾之处.  重新检视真正掌控开放原始码拥有权的文化.  我们发现暗藏著一个源自於Locke的土地产权理论的现象. 我们将此与玩家文化跟`礼物文化'关联起来, 意即参与者透过投入时间, 精力, 及创造力所生产的礼物来竞争并获取名望. 接下来检视在文化中, 迁涉到对冲突解决的分析, 并发展一些常规.
--------------------------------------------------------------------------------

1. 矛盾的现象
2. 玩家意识形态的多样性
3. 杂乱的理论, 清教徒实践
4. 开放原始码及拥有权
5. Locke及土地头衔(12p)
6. 玩家文化即礼物经济
7. 驾御之乐无穷
8. 名望的多面性
9. 拥有权及名望诱因
10. 自我的问题
11. 人性的价值
12. 名望游戏模型对整体的密切关系
13. 智域特质及领土於动物行为学的影响
14. 冲突的起因
15. 计划结构及拥有权
16. 冲突与冲突解决
17. 薪传机制及与学界的关联
18. 结论: 由文化到文化规范
19. 对进一步研究的一些问题
20. 参考文件, 附注, 及感谢(不翻译)

[目录]

矛盾的现象

1. 矛盾的现象
任何人观察忙碌, 巨大生产力的网际网路开放原始码软件世界一阵子, 一定都会注意到一个很有趣的矛盾, 开放原始码玩家们, 所说及所做之间的矛盾 -- 即开放原始码官方的意识形态及其实际的实践.
文化是善於应变的机器. 开放原始码文化是对应到一系统可指认出的驱力及压力. 一如往常, 文化对环境的适应力显示出清析的意识形态, 及隐含地, 潜意识或半意识的意识形态. 而, 并非不寻常地, 半意识的适应力扮演与清析的意识形态同样重要的地位.

在本文, 我们将深掘此一矛盾的根, 并用来发掘这些驱力及压力. 我们将演绎一些关於玩家文化习俗的有趣的事.  并透过建议一些方式, 以升华这一玩家文化的知识层次.


[目录]

玩家意识形态的多样性

2.玩家意识形态的多样性
网际网路开放原始码文化的意识形态(玩家们说他们所相信的)本身是相当复杂的话题. 所有成员都同意开放原始码(也就是, 软件可以免费散播及能够毫无困难地演化及修改成适合自己所需)是件好事, 并值得投入大量群体的力量. 这样的一致很有效地定义了文化的成员资格.  不过, 有许多理由值得我们考量, 尤其是许多个体及次文化信念的多样化.

一种是狂热; 一种不论开放原始码开发仅仅是个到达终点的便利工具(好的工具, 有趣的玩具, 及有意思的游戏)或者本身就是终点的热情.

狂热之辈会说``自由自由软件是我的生命! 我生命的意义在於生产有用, 优美程式及资讯资源, 然後送给人家.'' 中等热诚者会说``开放原始码是好东西, 我愿意花下大量时间协助''. 低度热诚之流则会说``对, 开放原码有时还可以. 我玩弄它, 并且尊重建造它的人们''.

另一种是对商业软件或企图统治商业软件市场的企业的敌意.

非常反对商业的人会说``商业软件是偷窃及聚财. 我写自由软件来结束这个恶魔.'' 中等反商业的人会说``商业软件一般还好啦.  程式设计师需要有收入, 但是那些推行劣质产品并且胡搞一通的则是恶魔'' 不反商业的人会说``商业软件还好啦.  我用或写开放原始码软件是因为我比较喜欢它''.

这样的态度组合表现出九种开放原始码次文化. 值得将这分别指出的原因是它们暗示著不同的议题, 及不同的适应, 及合作的行为.

从历史上来看, 最引人注目及最有组织的玩家文化是非常狂热及非常反商业的. 由Richard M. Stallman(RMS)所设立的自由软件基金会, 自1980年代早期, 大力支持著开放原始码的发展, 包含了像Emacs及GCC到目前为止, 在网际网路开放原始码世界依然是最基本的, 而且看来依然会在未来继续下去.

许多年来, FSF是唯一最重要的开放原始码的焦点, 产生大量至今依然对该文化非常重要的工具. FSF同时也是对外界来说, 唯一对玩家文化的赞助者.  他们有效地定义该辞`free software', 慎重地定义其重要性(新一点的 ` open source' 则慎重地避免一些误解).

因此, 内内外外对玩家文化的认知都倾向指出该文化是FSF的热情态度及反商业目标(RMS自己否认他是反商业的, 但他的程式是被许多人这样的解读, 包含他许多的口头游击战). FSF的强力而明显的口号``赶走软件聚财员外!''变成玩家意识形态的中心, 而RMS是最接近玩家文化的领导者的地位.

FSF的版权条文, ``一般大众版权'' (GPL), 表达了FSF的态度. 它被开放原始码界广泛地使用. 北卡的Sunsite是Linux界最大及最知名的软件库. 在1997七月, 大约一半的Sunsite软件套件都是用GPL做版权声明.

但FSF并不是游戏中的唯一成员. 还有一些比较安静的, 比较不那麽激烈, 而且在玩家文化中, 对市场比较友善. 实用主义者比较不像早期FSF那样的传统上的意识形态. 这些传统包含了, 更重要的, 纠缠著UNIX技术文化及前商业的网际网路.

典型的实用主义者态度是适度地反商业, 而其主要对商业的抱怨并非"聚宝库". 而是该世界对较好的技术UNIX, 开放标准及开放原始码软件的顽强拒用. 如果实用主义者痛恨任何东西, 那大概不是"聚宝", 而是在软件市场的龙头; 过去是IBM, 现在Microsoft.

对实用主义者来说, GPL的重要性是个工具而非终点目标. 它主要的价值不是用来对付`聚宝', 而是用来鼓励分享软件及成长市集模式发展团体. 实用主义者以拥有好工具及玩具来衡量价值, 而非不喜欢商业, 并且会使用高品质商业软件而不会有意识形态上的不舒适.  同时, 其开放原始码经验已经教导他技术品质的标准, 很少会有封闭软件能够达到.

许多年来, 实用主义者的观点在玩家文化中, 常顽固地透过拒绝完全使用GPL版权及FSF章程的方式来表达. 从1980年代到1990年代早期, 这样的态度倾向与Berkeley Unix的风迷有关, BSD版权的使用者, 及早期以BSD原始码为基础来建立开放原始码UNIXes的付出. 这些付出, 无法建立起大规模的市集团体, 因而变得片断及无效率.

一直到Linux在1993-1994年间开始爆发, 实用主义者找到了很有力的基础. 虽然Linus Torvalds从未反对过RMS, 他设下了商业Linux工业成长的良好典范, 透过保证商业用途软件上的高品质, 及轻微地嘲弄清教徒及狂热文化中的元素.

Linux快速成长的副作用引介了许多新的玩家进入, 而Linux是他们主要的兴趣, FSF的信念则是历史的兴趣.  虽然新一波的Linux玩家会描述Linux为``GNU一代的选择'', 大部份会仿效Torvalds而不是Stallman.

许多的反商业清教徒都发现他们自己处於少数地位. 在Netscape於1998年二月宣布要公开其Navigator 5.0原始码之前, 许许多多的事情都已经改变了. 这激起商业世界对`自由软件'的更大兴趣. 其後果是唤起玩家文化探索空前的机会及重贴其产品标签, 由`自由软件'成为`开放原始码', 以符合大众参与的认证许可.

在一个增援开发中, 文化中的实用主义者在1990年代有许多个中心. 其它半独立的团体及其半意识及有魔力的领导者开始由UNIX/Internet的根发芽. 在这些当中, 在Linux之後, 最重要的是Larry Wall带领的Perl文化. 比较小, 但依然很重要, 的是由John Osterhout创建的Tcl及Guido Van Rossum的Python语言. 所有这三个团体都透过发布其非GPL版权方案来表达其意识形态.

--------------------------------------------------------------------------------

[目录]

杂乱的理论, 清教徒实践

3. 杂乱的理论, 清教徒实践
然而, 在历经了所有的这些改变, 依然存在广泛的舆论在什麽是`自由软件'或`开放原码'. 对共同理论的最清楚的表述可在各式开放原始码版权中找到, 都有一些重要的共同元素.
在1997年, 这些共同元素被结合到Debian Free Software Guidelines, 後来变成开放原始码定义. 在开放原始码定义的指导下, 一份开放原始码版权必须要保护无条件的由任何人或团体来修改开放原始码软件的权利.

因此, 与OSD相关的理论(及OSD-conformant版权, 诸如GPL, BSD, 及Perl's Artistic License)就是任何人可以玩弄任何东西. 没有人可以防止一大票人来玩弄任何开放原始码产品(诸如, 以Free Software Foundations的gcc C compiler), 复制原始码, 将它们朝各种不同的方向演化, 但都可以宣称依然是该产品.

但在实际上, 这样的`分歧'几乎没有发生过. 大计划分歧的状况很少, 而且几乎都是由重新贴个标签及大规模的自我正当化来完成. 很清楚的, 在GNU Emacs/XEmacs分支, 或gcc/egcs分支, 或各式各样的BSD分支群, 这些分歧者都感觉到他们是在对抗一个相当巨大力量的社会规范.

事实上(与anyone-can-hack-anything的大众理论相矛盾之处)开放原始码文化有个很复杂但大体自我正当化的拥有权习俗. 这些习俗规范了谁能修改软件, 这些状况决定了谁能够修改,  谁有权力再发行修改过的版本到团体中.

这些文化的禁条明确地规范基准. 因此, 我们在此摘要一些重要的部份会对未来有点用处.

对计划分歧来说, 社会的压力很大. 除非实在是非这样做不可, 透过大众的自我审判.
发行一个未经原创者同意对计划的改变是受到阻止的, 除非在特殊状况下, 诸如一些一般的修正码.
将一个人的名字从计划历史, 或维护者中抹去是绝对不可行的, 除非当事者同意.
在本文, 我们将细细检验这些禁条及拥有权习俗. 我们将探索它们是如何运作的, 并且揭露在其内中的社会动力及开放原始码团体的诱因结构.

[目录]

开放原码及拥有权

4. 开放原码及拥有权
当产权可无限复制时, `拥有权'的意义为何, 或者延申来说, 整个文化不具有强制高压力量关系或稀有物质经济?
事实上, 在开放原始码文化的例子中这是个很容易回答的问题. 软件计划的拥有者是那些拥有独有权力, 被大规模团体认定, 可再发行修改过的版本.

(在讨论`拥有权'这一节中, 我使用单数, 因为所有计划都是由某个个体所拥有. 有时候, 应该要了解有些计划是由一群人所拥有的. 我们会在本文後段检验这样族群的内部的动力.)

根据标准的开放原始码版权, 所有参与者在演化游戏中都平等. 但在实际上, 大众对`正式'版, 即由大众认可的维护者所整合并认可的版本, 及支援厂商的`游离'修补版, 有很明显的差别待遇. 游离修补版是不寻常的, 而且一般都不受信任 [RP].

大众发行版是基础的是很容易了解的. 习俗鼓励个人需要时可以进行对软件的修补. 习俗对那些再将修改版发行的人或发展团队的待遇是不同的. 当修改版在开放原始码团体中被发行时, 用以与原有版本来竞争, 拥有权就变成是个问题.

一般来说有三种方法来看一个开放原始码计划的拥有权. 第一种, 最明显的, 是计划的创建者. 当计划只有一个维护者/创建者, 而创建者还在活动时, 习俗不允许质疑谁是计划的拥有者.

第二个方式是计划的拥有权是由上一个拥有者所转移过来的(有时可称为`转交指挥棒'). 在这个团体中, 一般认可, 计划拥有者不再感兴趣或无能再进行维护的时候, 会将责任交给下一个继任者.

在大计划中这就很重要, 控制权的转移往往伴随著华丽的吟咏. 对大多数开放原始码团体中, 大家所不知道的, 它实际影响著拥有者对继承者的选择, 习俗明白说明正统嫡系的重要性.

对小一点的计划, 在历史记录上记录一下计划拥有者的改变即可. 如果前一位拥有者并非志愿的转移控制权, 他可以在一段时间内在团体透过向大众公开来取回控制权.

第三种获取计划拥有权的方式是观察该计划需要工作, 而拥有者却失去兴趣或消失了. 如果您希望做这件事, 您需要去找到拥有者. 如果您找不到, 那麽可以在相关的地方宣告(像在该领域的新闻讨论网)该计划似乎变成孤儿, 而您正在考虑负起责任来认养.

习俗要求您在您宣告您是新的拥有者之前, 要等待一段时间. 在这段时间内, 如果有人宣告他们已经在这方面开始工作, 那麽他们的宣告胜过您的. 多让大众知道您对这方面的兴趣是很好的方式. 最好是你在许多相关的讨论区中宣告(相关新闻讨论网, mailing lists); 而如果您有耐心等待回应. 一般, 您制造越多的注意, 让上一个拥有者或其它的宣告者来反应, 当没有人反应时, 您的宣告会更有效.

如果您已经通过在计划使用者中这样的过程, 而没有人异议, 那麽您可以宣告该孤儿计划的拥有权, 并且在历史档中记下一笔. 不过, 这比正式交棒来的不安全, 您并不被视为嫡系, 除非您在使用者族群中做了许多的改善以後.

我已经观察了这样的习俗二十年了, 可以回朔到前FSF的古早开放原始码软件历史中. 他们有许多非常有趣的特色. 最有趣的事,  大多数的玩家都不需要人家告知便知道要这样做. 的确, 以上所述是第一次完整写下的摘要.

值得一提的事, 潜意识的习俗, 实在是令人惊人的协调. 我已经不夸张地观察到数百个开放原始码计划的演化, 而我还可以用手指来算出这些违反传统的例子.

第三个有趣的特色是这些传统在时间下的演化, 他们在协调的方向下如此运作. 这样的方向鼓舞更多的大众责任, 大众注意, 及更关心於现有拥有者保留原有拥有人的成就及历史记录.

这些特色建议了这些习俗并非意外发生的, 而是某种暗示性的条文, 或是开放原始码文化中的生产形式, 在运作中这些是完全基本的条件.

早先一位回响者指出, 相对於网际网路玩家文化的骇客/海盗文化(``warez d00dz''主要集中於破解游戏及海盗BBS)也有类同的两种生产形式. 我们将会在文中稍後回到d00dz的比较.


[目录]

Locke及土地头衔

5. Locke及土地头衔
要将整个一般的型态了解, 历史上, 在玩家之外, 有些类同的传统, 与玩家的习俗是相同的. 一些历史或政治哲学的学生或者会认出, 这个产权的理论暗示著与英美一般惯例法理论中的土地产权相同的观念!
在这个理论中, 有三种方法来确认土地的拥有权.
在边疆, 领土可以存在而没有拥有者, 一个人可以透过开垦来获取拥有权, 在未有人拥有土地时, 混以血汗, 架起栏干, 并且抵御个人的名衔.

一般转移土地的方法是头衔转移, 也就是从上一位拥有者的手中收到契据. 在这个理论中, `头衔传承'观念是很重要的. 拥有权理想证据是整个契据传承及转移可用於追朔土地在最早期被开垦时的规模.

最後, 一般法理论了解到土地头衔可能会遗失或被抛弃(例如, 如果拥有者死去而无继承者, 或者要建立头衔传承的文件遗失了). 无主的土地可透过adverse possession的方式所认领 -- 透过迁入, 改善它, 并为其抵御, 就由如是开垦它一般.

这个理论, 正如玩家习俗一般, 会在中央权威薄弱或不存在的背景下有机地演化. 它在挪威及日耳曼部族演化了数千年. 因为它被早期英国政治哲学家John Locke理性而系统化, 有时它被称为`Locke'产权理论.

类似逻辑的理论倾向在高度经济或生存价值, 而没有任何权威有足够的力量来强制分配稀有物品. 在以打猎聚货的文化中, 大家认为没有产权观念的想法下, 这一点甚至也有. 例如, 在Kalahari沙漠的!Kung San布西曼族, 并没有猎地的拥有权. 但水洞或泉却有类同於Locke的产权存在.

!Kung San的范例是具有教育性的, 因为它显示了Locke产权习俗适用的时机 - 当从该资源所能获取的回报大过於需要抵御的代价时. 猎地并非产权, 因为打猎所能获得的回报很难预测而可变的, 而且并非每日生存的必须品. 水洞, 另一方面来说, 对生存来说非常重要, 并且小到足够抵御.

在本文标题中, `智域'(`noosphere')是所有思想的领土, 所有可能想法的空间[N]. 我们在玩家拥有权习俗中看见了Locke产权理论在智域次集合 - 所有程式的空间 - 上的暗示. 即此`开拓智域', 正如所有新的开放原始码计划的建立者所做的事一般.

Fare Rideau <rideau@ens.fr> 正确地指出玩家们并不是在单纯的思想领土中运作. 他断定玩家们所拥有的是程式计划 -- 有目标地专注在物质劳力上(发展, 服务, 等等), 而与名望, 信赖度等等连上关系. 他因此断定由玩家计划所延展的空间, 并非智域, 而是一种双重的智域性质, 由智域计划拓展的空间. (有位天体物理学家在此同意, 在词语学上, 正确地可称这个重复的空间为`ergosphere'或`工作的领域'.)

在实际上, 智域及工域的分别在本文目地中并不重要. 在纯净的思想中, `智域'要存在实在也很难; 大概要有柏拉图式的哲学家才会相信了. 而将智域及工域分开, 也只有在某人希望断定想法(智域的元素)不能被拥有, 但如计划之类的可以被拥有的时候, 才会有实际作用. 这个问题导致一个智慧财产权理论的问题, 远远超过本文探讨的范围.

为了避免困惑, 要注意到智域或工域跟整个虚拟的电子媒体被称为`cyberspace'(大部份玩家伪装的地方)是绝不相同的. 产权规范法则在此是完全不同於物质阶层的 -- 基本上, 拥有媒体或机器属於`cyberspace'便拥有`cyberspace'的一部份.

Locke的结构强烈建议开放原始码玩家观察这些所为的习俗以持有对付出的代价所得的回报.  这些回报必然要比开拓计划所付出的更重要 - 花在维护`头衔传承'的版本历史上的代价, 花在引起大众注意的时间代价, 及等待领养孤儿计划的时间上.

再者, 由开放原始码所获得的`成果'必然是远超过简单的使用软件, 是一些会被分歧所连累或稀释的其它东西. 如果就是使用软件这麽简单, 应该不会有对分歧的禁条存在, 而开放原始码拥有权不会像土地产权一样的类同. 事实上, 这样的世界确实(即使用为唯一成果)在现有开放原始码版权中存在.

我们可以现在就将一些可能的成果候选者扫除. 因为您不能够强制网路连线, 找寻在那里的存在的力量. 同样地, 开放原始码文化不能构有任何像钱或内部稀有经济的类同品, 因此玩家不能够追求与物质富裕相关的任何事物.

在开放原始码活动中, 有个方法可以协助人们变得更富裕. 偶而, 某人在玩家文化中所获得的名望可在现实生活中获得经济上的重大好处. 它可以使您获的更好的职业收入, 或者顾问合约, 或者书约.

这样的副作用对大多数玩家来说是很少见的; 这可独立做为一个解释, 既使我们经常见到玩家们抗议说, 他们所为是出发於理想或爱, 而不是为了钱.

不论如何, 调停这样的经济副作用很值得加以检验. 以下我们会看到了解在开放原始码文化自身的名望动力, 很可以自我解释.

[目录]

玩家文化即礼物经济

6. 玩家文化即礼物经济
要了解名望在开放原始码文化中的角色, 我们需要从历史移到进一步的人类学及经济, 并检验交换文化及礼物文化之间的不同.
人类对社会地位的竞争有天生的驱动力; 它与我们的演化史息息相关. 在农业发展之前, 90%的历史, 我们的祖先生活在游牧打猎的生活形态中. 地位高个体获取较建康的伴侣并取用最好的食物. 这个透过地位来表达自我的驱力表现在多方面, 大致上是由於生存货物的缺乏所致.
大部份方式中, 人类采行组织的方式来获取稀有货品及所需. 每种方式都有其获取社会地位的方式.

最简单的是命令阶层. 在命令阶层中, 稀有品的分配是由中心权威来完成, 并以武力做为後盾.  命令阶层所达程度很有限[Mal]; 他们在组织成长时变得越来越兽性而无效率. 基於这个理由, 在大家族中的命令阶层往往在不同大型经济形态中变成寄生虫.  在命令阶层中, 社会地位主要是透过取得高压力量来达到.

占我们社会主导地位的是交换经济. 这是对稀有品的复杂采用形式, 不像命令阶层模式, 它成就很高. 稀有品的分配是透过分散的交易及志愿合作(事实上, 竞争野心是产生合作行为的主要效应). 在交换经济中, 社会地位主要是透过控制用以交易的东西来决定(不一定需要是物质的).

大多数人在精神上都有受到以上两种模式的影响, 并决定如何与他人互动. 政府, 军队, 及组织罪犯(举例而言)皆为在我们称为"自由市场"的广泛交换经济下的命令阶层寄生虫. 不过, 其实还有第三种模式, 在根本上完全与两者不同, 而且除了人类学家以外, 一般人并不知晓; 即礼物文化.

礼物文化不是因为稀有而采用而是丰富才采用. 它们是在没有物质稀有问题, 而生存必须品丰富的族群中掘起. 我们可以观察到礼物文化在气候温和及食物丰足的原始生态系中发展. 我们可以看见他们是在我们社会阶层中的确定地位, 特别是那些商业中富裕的族群.

丰富使得命令关系结构变得很难维系, 而交换关系则变程式无意义的游戏. 在礼物文化中, 社会地位不是由您能控制多少而决定, 而是由您给出多少来决定.

因此这是瓜基乌图酋长的冬季赠礼舞会. 这是百万富翁精心的大众慈善行为. 这是玩家长时间付出的生产高品质开放原始码.

透过这样的检验, 开放原始码玩家社会很明显地是个礼物文化. 在其中, 没有严重的`生存必须品'的短缺 -- 硬碟空间, 网路频宽, 电脑速度. 软件是免费地分享的. 这样的丰富产生一种状况, 即唯一的竞争是同跻间的名望.

不过, 这样的观察并不足以完全解释整个玩家文化的特性. cracker d00dz也有相同於礼物文化的特质, 但他们的行为是大不相同的. 其族群文化的智力在玩家中算是很强及独有的. 他们聚集秘密, 而非分享; 一个人需要投靠某个cracker组织来获取破解的软件, 而非取得如何破解的技巧.

(译注: 很多人认为Hacker及Cracker之间没有明显的界线. 但实际上, 这是错误的观点. Hacker及Cracker不但可以很容易的分开, 而且可以分出第三群 - "海盗"Internet Pirate出来, 一般大众认定的"破坏份子", 事实上是这第三种. Hacker及Cracker都有明确的定义, 要发表有关Hacker及Cracker之间的评议之前, 最好要详细调查一番, 否则招惹这两群技术高明的族群都不是好受的事. 比较容易判断的方式, "Hacker从来不自称Hacker; Cracker会自称Cracker; 自称Hacker的不是Hacker; 自称Cracker的不见得是Cracker; 被确认为Hacker称为Hacker的, 是Hacker; 而Richard M. Stallman是Hacker圣者;" 相信大家应该可以看出来, 为何大众对Hacker及Cracker会有错误观点的由来, Pirate利用这样的漏洞来污染整个玩家文化在大众的观点. )

在此所展示的, 看来不是很明显, 是礼物文化运作方式不只是一种. 历史及价值是很重要的. 我已经将整个玩家文化做了大体的摘要[HH]; 现有行为状态并非神秘的. 玩家透过选择他们竞争的形式来定义他们的文化. 而本文的接下来的部份将会检验这些形式.


[目录]

驾御之乐无穷

7. 驾御之乐无穷
在做这个`名望游戏'分析的同时, 顺便说一下, 我并非有意诋毁或忽略单纯的设计美妙软件并使其工作的艺术上的满足. 我们都经历过这样的满足并爱上它. 这些重大动机, 对那些并非是那麽执著的人们来说, 从一开始就不会成为玩家, 正如不爱音乐的人不会成为作曲家一样.
因此, 我们应该要考虑到另一种玩家行为模式, 即以技艺为单纯的主要动机. 这个`技艺'模式应可解释玩家传统在技艺的机会上及结果的品质上的最大效果. 这是否建议与`名望游戏'模式相冲突或不同的结果呢?
(译注: 在电脑科学中, 有个很大的争议, 即"软件设计是艺术"及"软件设计是工业"的争执. 最著名的"软件艺术家"可说是Knuth这位大师. 我们在此所见到的, 便是"软件艺术". "软件工业"在电脑科学上, 可由软件工程来代表.)

不见得. 在检视`技艺'模式中, 我们回到同样的问题上, 玩家圈子运作像礼物文化. 如果品质没有尺度来衡量, 要如何将品质最佳化呢? 如果稀有品经济不运作, 除了同跻评估外有什麽度量呢? 这显示出任何技艺文化最终要架构在名望游戏中 -- 而事实上, 我们观察到正是这个动力, 在中世纪同业公会中, 推动许多历史上的技艺文化.

从一个重要的观点来看, `技艺'模式比起`礼物文化'来得弱; 其自身, 并没有帮助我们解释在本文开始所说的矛盾.

最後, `技艺'动机本身可能如我们所假设的, 在心理上不会离名望游戏太远. 想想看您美妙的程式被锁在抽屉中, 而不再被使用. 现在想像它被许多人很满意的有效使用. 那一个梦想比较满足您呢?

不过, 我们将会注意这个技匠模式. 它直觉地揭露了许多玩家, 并相当完美地解释了许多的个体的行为.

在我发表了本文的第一个版本後, 许多匿名的回响者建议: ``您没办法用获取名望的动机来工作, 不过名望是当您把工作做好, 而获取的真实回报.'' 这是个微妙而重要的观点. 名望诱因可继续运作, 不论工匠是否关心到它们; 因此, 最终, 不管玩家是否了解其自身的行为是名望游戏的一部份, 他的行为都会被这个游戏所修饰.


[目录]

名望的多面性

8. 名望的多面性
在每个礼物文化中, 有许多理由表明同跻名望是值得付出的:
第一, 最明显的, 同跻之间的好名望是基本的回报. 我们是为此而活, 这是个我们在前面已触及的革命性动力. (许多人试著将驱策他们的动力, 由名望换成其它升华的形式, 而与同跻之间没有明显的关连, 诸如``荣耀'', ``建全伦理'', ``信仰''等等.; 这些并没有改变其中的机制.)
其次, 名望是(在单纯的礼物经济中, 是唯一的方法)吸引注意力及与他人合作的好办法. 如果一个人有雅量, 智慧, 公平处置, 领导能力, 及其它的好品德, 这将会非常有力地说服其他人来一起共同合作.

第三, 如果您的礼物经济与交换经济或命令结构纠缠关联, 您的名望将会溢满并使您获得更高的地位.

在这些一般理由之外, 玩家文化独特的状况造成名望比"真实世界"礼物文化更有价值.

主要的`特例'是某人送出的手工艺品(或者, 换另一种方式来解译, 是明显地看得出是一个人下足精力及时间所制的礼物)是非常复杂的. 其价值很明显地跟物质礼物或交换经济金钱完全不同的. 要客观地区别出好礼物及坏礼物是更加困难的. 所以, 赠与者企图的成功是微妙地由同跻间的风评来决定.

另一种特质是相对纯正的开放原始码文化.  大部份礼物文化是由这些元素所组成的 -- 由交换经济关系, 诸如交换奢侈品, 或是由命令经济关系, 诸如家族或部落族群. 在开放原始码文化中, 并没有类同的方式存在; 因而, 除了透过争取同跻名望之外别无它法.

[目录]

拥有权及名望诱因

9. 拥有权及名望诱因
我们现在准备将之前的分析放在一起, 成为一个连续而完整的玩家拥有权传统. 我们了解到这是由进驻智域而来的; 它是在玩家礼物文化中的同辈名望, 及其所附带的副作用.
由此这样的了解, 我们可以分析Locke在玩家圈子的产权习俗, 是一种名望诱因的最大化; 这可确保同跻名望可由该获得的人取得, 而不会漏到其它地方去.
我们以上所观察到的三个禁条, 在这个分析下, 就很有点道理了. 在某些状况下, 有些人胡搞, 有些人的名望会遭到不公平的处理; 这些禁条(及相关习俗)试图避免以下这些发生.

分歧计划是不好的, 因为它将过去的贡献者置於牺牲名望的危险中, 而他们只好在分歧後, 继续在两个案子中都活动. (一般来说这会变得很混乱而且难以实践.)
发行一个游离修补版会使拥有者受到不公平的名望危机. 就算正式版是完美的, 拥有者也会被这些游离版中的臭虫发出的高射炮射中(但请见 [RP]).
偷偷摸摸地将某人的名字将计划中移除, 在文化条文中, 是大恶不赦的事. 他偷取受害者的礼物并且变成小偷的礼物.
这三项禁条的违反者伤害整个开放原始码团体及受害者本身. 这还迁连他们会伤害到整个团体, 导致降低潜在贡献者对礼物/产品会收到回报感受的可能性.
很重要的对其他两个禁条有其它可能的解释.

第一, 玩家会对分歧计划表示厌恶, 并且悲叹重复的工作, 在未来, 需要在所有几个子计划中付出代价.  他们也会观察到分歧会将共同开发者团体分开, 导致两个子计划有较少於原有计划的智群在工作.

有位回响者指出, 在分歧的工作中, 很少有见到能够有多於一个分歧能够存活, 并且在长期享有`市场占有率'. 这加强了所有参与者参与的诱因, 互相合作并避免分歧, 因为很难事先知道, 谁将会站在输的一边, 而看到他们辛苦的工作不是整个消失就是逐渐漠落.

不喜欢游离修补通常可被解释为造成错误追踪的复杂度增高, 而且会造成维护者增加他们的工作量, 而他们自己通常就已经有很多自己需要做的事了.

这些解释都可以考虑为正确的, 而他们确实在Locke逻辑下的拥有权理论下行得通. 但同时理性吸引力, 他们无法解释为何当这些偶而发生的违反禁条事件, 会导致如此在情绪及领土上的反感 -- 不只是在受伤的一方, 而且旁观者及观察者也会反应严厉. 冷血地关心工作量及维护量的重复并不足以解释这些观察者的行为.

对於第三个禁条. 除了名望游戏的分析以外很难解释. 事实上这项禁条很少被分析为``它不公平''的结论, 则揭露的其本身的问题, 而我们会在下一节看到.


[目录]

自我的问题

10. 自我的问题
在本文的一开始, 我提到了文化的潜意识适应知识通常是起源於其意识形态. 一个很大的事实范例即Locke的拥有权传统被广泛地使用, 虽然事实上, 它们违反标准版权所陈述的条文.
我在与玩家们讨论到名望游戏的分析时, 观察到另一个有趣现象的范例. 即许多玩家拒绝这个分析, 并且强烈反对承认他们的行为是起源於渴望同跻名望的动机, 或者我在此不太精确地标为, `自我满足'.
这展现了玩家文化有趣的一面. 很清悉的大家都不信任并鄙视自我中心及自我出发的动机; 自我奖励被残酷的批评, 既使该团体事实上对此可获取许多有利之处. 很大一部份, 事实上, 该文化的`大大'及部族长者被要求要谦逊并幽默地自我贬低, 以便维护自己的地位.  这样的态度几乎使自尊呼之欲出用以编织解释整个诱因结构.

在大体下, 可以确认的, 这一般起源於欧裔美国人对`自我'的负面态度. 整个玩家圈子都告诉自己, 渴望自我满足是很坏的动机(至少是不成熟的); 自我只是对追求女性时的可容忍的怪僻, 而且通常被认为是个精神病状. 只有在升华的形式或伪装的形式如``同跻名望'', ``自尊'', ``内行气派''或``成就的自豪''才是可被接受的.

我可以写一整篇论文来讨论这个文化传承中的不良部份的根,  我们可以发见,  相信我们有真正的`无私'动机(违背心理学及行为的证据),  将会造成自我迷惘的巨大伤害. 或者我可以, 如果Friedrich Wilhelm Nietzsche及Ayn Rand没有已经将整个解构`利他主义'成为许多不同的个别兴趣的工作完成.

我并非在此做道德哲学或是心理学, 因此我在此仅观察一个以认为自我是邪恶的观点所造成的一些伤害, 就是: 它在情绪上造成许多玩家很难以清析地了解自己文化的社会动力!

但我们并没有在此线调查中完成. 在玩家文化中对於自我出发的行为的禁忌是如此的深,玩家要怀疑是否该采用其它的可行方式. 当然了这样的禁条在其它礼物文化中并没有这麽强, 像剧场或巨富之间的同跻名望!


[目录]

人性的价值

11. 人性的价值
在建立了名望是玩家文化回馈机制的中心之後, 我们现在需要了解为何它看来如此的重要, 然而确依然是半隐密地并且不受承认.
对比的是海盗文化的指令式. 在那个文化, 追求地位的行为是露骨的甚或炫耀的. 这些crackers追求宣称释放"zero-day warez" (在原版软件释放的当天破解该软件并再发行)但却对如何做到的闭口不提. 这些魔术师并不喜欢把密技公开. 因此, 这导致了cracker文化的知识库在整体上进步缓慢.
(译注: 社会大众及Eric S. Raymond对骇客文化有所误解, 在整体知识库的发展上, 骇客文化透过结帮来进化, 因此骇客文化有点像帮派文化. 其整体知识库的发展上, 步调并不慢.)

在玩家团体中, 对比来说, 一个人的工作代表一个人的表达方式. 这是非常严格的英才制度(最好的工匠胜利)而且有很强的意识, 即品质会说话. 最大的吹牛就是程式``能跑'', 而任何有能力的程式设计师会看到的是好东西. 因此, 玩家文化的知识库增加快速.

对抗自大驱使的装模作样的禁条导致增加生产力. 但那是个次级效应; 在此受到保护的是该团体同跻评估系统的资讯品质. 即, 自吹自擂或自我放大是被镇压的, 因为在创造及合作的行为者中, 它就像杂音一样搞乱实验室中的美妙.

玩家文化传播礼物的媒介是无形的, 其通讯频道很难表现情绪的细微差别, 而面对面的成员接触是唯一的例外. 这使得该文化比其它礼物文化对杂音有更低的容忍力, 而需要花不少时间来解释大众谦逊需要部族长者.

谈吐谦逊对一个有志成为成功计划维护者的人也是有用; 他必须要使该团体相信他有很好的判断能力, 因为一个维护者的大部份工作是判断其他人的工作. 谁愿意贡献给那些无法判断其工作品质的人呢, 或者谁愿意贡献给那些会吞食计划成果的人呢? 潜在的贡献者希望计划领导者有足够的谦逊及品味, 当客观时机到来时, 有能力说, ``对, 这比我的版本工作来得好, 我会用它'' -- 并且将成就让该收到的人收取.

另一个在开放原始码世界需要谦逊行为的理由, 您很少希望给大众认为计划已经`完成'的映象. 这可能会导致潜在的贡献者觉得不须要贡献. 要将您的工作成效最大化, 需要对您程式的状态谦逊. 如果您吹牛自己的程式, 然後说``没什麽价值, 它对x, y, 及z无效, 因此它并不好'', 然後很快的自己偷偷补上x, y, 及z.

最後, 我自己亲眼见到有些玩家领导者自鄙的行为反映了害怕成为崇拜的教首. Linus Torvalds及Larry Wall两者的行为都很明显地想要避免成为这样的对象. 有一次跟Larry Wall一起吃晚餐, 我开玩笑``您是在场的头号玩家 -- 您选餐厅''. 他畏缩了. 这样做是对的; 无法分辨他们领导的分享价值会搞砸一个很好的团体, 这是个他及Linus都无法忽略的一点. 另一方面来说, 大部份玩家喜欢有Larry的问题, 如果他们有办法自我承认这一点.


[目录]

名望游戏模型对整体的密切关系

12. 名望游戏模型对整体的密切关系
名望游戏的分析有些不是很明显的其它关连. 许多都是从建立一个成功的计划可获得比与现有计划合作获得更大的名望而来. 一个计划如果有许多创新, 也会获得许多的名望, 相反於`me, too'的对现有计划的持续改善. 另一方面来说, 只有作者了解的软件或是需要非入门者的, 在名望游戏中, 通常贡献一个现有计划比自己建立一个来得受人注目. 最後, 要与现有成功的计划相竞争比填补一个空缺的壁灶来的困难的多.
因此, 有个与邻居的最佳距离(类同计划间的竞争). 太接近就会有其中一个产品会变成``me, too!''的有线价值, 一个贫乏的礼物(其中一个可能最好放弃). 离得太远, 没有人有能力使用, 了解, 或察觉到另一位的付出的关系(同样, 贫乏的礼物). 这产生一个在进驻智域中的型态, 如拓荒者散播在实体边疆上 -- 并非散乱的, 但很像是散乱的碎形波. 计划倾向开始於在边疆的范围填补空缺.
有些非常成功的计划变成`目录杀手'; 没有人愿意再去与那些已经建立起来的计划竞争, 因为对玩家来说实在太难. 大家最多是发现自己所须, 而在这些成功的计划中, 新增附加功能. 典型的`目录杀手'范例是GNU Emacs; 由1980年代开始, 它的多样化功能就已经填补整个程式设计编辑器的目录生态系, 没有人再去试图写一个新的出来. 因而, 人们只写Emacs modes.

整体来说, 这两种倾向(缝隙填补及目录杀手)在时序上的发展可用来预测未来的倾向. 在1970年代, 大部份开放原始码都是玩具或范例. 在1980年代, 则是在开发工具及网际网路工具.  在1990年代, 这项行动移向作业系统. 在每个案例中, 当前一个问题接近被处理掉时, 一个新而更加复杂的问题层次被开始攻下.

这种倾向在不远的未来有点有趣的关连. 在1998早期, Linux看来很像是`开放原始码作业系统'目录的杀手 -- 所有为其它竞争的作业系统, 现在都已经开始为Linux device drivers及extensions写作. 而大部份这个文化所想要的开放原始码的低阶的工具都已经存在了. 还有什麽吗?

应用软件. 当2000年接近, 看来预测开放原始码发展的能力会逐步迈向最後的处女地带 - 为非技术人员所设计的程式 - 是不过份的. 稍早的指标是GIMP的发展, Photoshop-like的影像处理软件, 这是个开放原始码第一个主要对end-user-friendly GUI的界面软件, 并可被考虑为在过去十年中足勘与商业软件相比的应用软件. 另一种则为流言四传的应用软件工具KDE及GNOME.

最後, 名望游戏分析解释了屡次引用的格言, 即您无法用自称玩家来变成玩家 -- 当其它玩家称您为玩家时, 您才是玩家. `玩家', 以这种观点来考量, 是某人显示出了(透过贡献礼物)他或她有技术能力及了解名望游戏的运作.  这样的评断大半是一种知觉及传承, 并且只有身在文化中之辈才会意会.

[目录]

智域特质及领土於动物行为学的影响

13. 智域特质及领土於动物行为学的影响
了解产权习俗的结果会帮助我们从另一个角度来看它; 即动物行为学, 特别是领土动物行为学.
产权是一种动物领土观念的抽象化, 是演化来用以降低同物种之间暴力发生. 透过划出界线, 并尊重其他的界线, 一苹狼如果与其它发生战斗, 可能会导致它受伤或致死, 并降低生存繁演的机会.
类同的, 在人类社会中的产权功能是避免人际间的冲突, 透过设立疆域清楚地分别合平的行为及侵略的行为. 有时大家喜欢将人类产权看成是个抽象的社会传统, 但这完全是错的. 任何人有苹狗就会知道, 当狗对陌生人接近时的吠叫, 是一种介於动物领土及人类产权之间的连续性. 我们本地的狼兄地就本能地比许多人类政治理论家知道的更清楚.

宣告领土(就像制造领土)是个表示实现愿望的行为, 一种表示在什麽样的界线下将会防御. 社会支持产权宣告是一种减少阻力及促进合作的行为. 这些比栏杆或狗吠更抽象的``宣告产权''依然是有效的, 甚至既使它只是在README档案中简单的描述计划维护者的名字, 也是有效的. 它依然是个抽象的领土, 而且(就像其它的产权形式)我们本能式的产权模式是由领土演化而用来解决冲突解决.

这个动物行为学分析, 第一眼看来很抽象, 很难以跟真正玩家的行为关连起来. 但它有一些重要的结果.  其中之一解释了全球资讯网的大众化, 特别是解释了为何开放原始码计划有个公开网站看来会比没有网站的来得重要.

客观地想, 这看来很难解释. 与原来在原创及维护上所付出的力量来说, 一个网页很简单, 因此很难想像一篇网页是个重要或非凡的付出.

网站的本身功能并不足以解释一切. 网页的通讯功能可以混合FTP站, mailing list, 及Usenet. 事实上一个计划的平日通讯很少是完全透过网站来通讯的, 反而mailing list或newsgroup更重要. 那麽, 为什麽公开网站会变成是计划的中心呢?

`home page'这个隐喻提供了一个重要的线索. 当在建立开放原始码计划时, 它是个在智域中宣告领土的行为(而传统也认知为此), 它在心理层次上并非十足强制的. 毕竟, 软件并没有位置的本质, 而且可以立即复制的. 它在我们本能的`领土'及`产权'记号上是可同化的, 不过, 要在付出一点代价後.

一个计划的网页表现了抽象的进驻可能的计划领域空间, 透过全球资讯网表达其王国的`本土'领域. 从智域降到`cyberspace'并没有让我们通过真实世界的栏干及狗吠, 但它确实保障我们宣告产权的力量, 并使我们对领土更加感到安全. 而这也是有网页的计划看起来更加地`真实'.

这个动物行为学的分析, 也鼓励我们更加详细的检验, 在开放原始码文化中, 这个处理冲突的机制. 他带领我们预期, 除了将名望诱因最大化以外, 拥有权传统将会在避免冲突及解决冲突上扮演一个角色.


[目录]

冲突的起因

14. 冲突的起因
在开放原始码软件的冲突大致有以下主要议题:
由谁来做下计划的决策?
由谁来接受荣耀或谴责, 承受什麽样的?
要如何减低负担, 特别是在复杂的错误追踪中避免劣质版本?
技术上来说, 什麽是正确的事?
如果我们看看``什麽是正确的事''议题第二眼, 它将应该要消失掉.  对任何这样的问题, 要看是否有个客观的方式来让大家都接受. 如果有, 那游戏结束大夥都赢. 如果没有, 那麽就变成``谁来决定?''.
所以, 这三个问题的冲突解决理论将可解决一个计划的三大问题: (A) 决定设计时的闲扯要何时而止, (B) 要如何决定那些贡献者接受荣耀及如何授与, 及 (C) 如何保持一个团队不会变成多重分歧.
拥有权的角色在解决(A)及(C)的问题上是很清楚的. 惯例确保计划拥有者可做下决策. 我们在以前曾见过, 惯例会对分歧者施加压力并稀释其拥有权价值.

注意这些惯例是合理的, 甚至对某些不关心名望游戏的人来说都很有帮助.  我们可从检验纯正的"工匠"模型的玩家文化来看出. 在此观点下, 这些惯例很少与稀释名望诱因有关, 而比较保护工匠在选择眼光上的权益.

技匠模型并不足以解释关於(B)的玩家惯例, 谁做了什麽而接受什麽荣耀(因为对一个纯正的工匠来说, 并不关心名望游戏, 因此没有其它动机可循). 要分析这一点, 我们需要将Locke理论带到另一个新高点, 并检验冲突及运作的权力在计画中及计划之间.


[目录]

计划结构及拥有权

15. 计划结构及拥有权
一般的个案来说是一个计划有一个单一个拥有者/维护者. 在这种状况下没有可能会有冲突. 拥有者可做下所有的决定, 拥有所有的益处, 及受到所有的谴责. 唯一会冲突的问题可能是继承者 -- 当旧的拥有者失去兴趣或不见了, 由谁来当新的拥有者. 该团体亦对避免分歧很感兴趣. 这些兴趣都由文化规范所表达, 即当一个拥有者/维护者对该计划不再有兴趣, 应该公开地将名衔交给下一位.
非一般的最简单的例子, 是许多位共同维护者在一位`仁慈的独裁者'下工作. 在共同计划中习惯於这个模式; 我们在大计划像Linux kernel或Emacs中看到, 及解决``由谁来决定''的问题, 这看起来不会比其它的方式来得糟糕.
典型来说, 一个仁慈的独裁者组织由一个拥有者-维护者组织, 建立者吸引贡献者演化而来. 即使拥有者依然在位, 它也有计划部份的功劳由谁来获取的争执存在.

在这种状况下, 习俗有义务由拥有者/独裁者来公平地处理贡献者的贡献(例如透过在README或历史档中提及). 在Locke的产权模型术语来说, 这意味透过贡献一个计划来获取部份的名望(正面或负面的).

追朔这个逻辑, 我们见到`仁慈的领导者'并非真正拥有整个计划. 虽然他有权力做下决策, 他事实上是透过交易名望来交换其他人的工作. 这类比就像佃农耕种, 是很难抗拒的, 除了有时候当某个贡献者不再活动, 他的名字依然继续获得好处.

一个仁慈的独裁者计划加上许多参与者, 他们倾向於发展出两族贡献者; 一般贡献者及共同发展者. 一个典型的途径, 变成共同开发者, 可获得计划较大部份的责任. 另一种是以`lord high fixer'的角色, 专长在修正许多臭虫. 以这种方式或其它种类的, 共同开发者在整个计划中, 是那些以透过投资时间并有重大贡献的贡献者.

次系统拥有者角色在我们的分析中特别重要并且要求更进一步的检验. 玩家喜欢说`权威要负责'. 接受维护责任的共同作者可获得一个次系统的掌控权及所有相关部份的计划, 只受到计划领导者的修正(可说是建筑师). 我们观察到这项法则有效地圈起Locke模型的计划产权, 并且同样地有其它产权界线避免冲突的效用.

传统上, `独裁者'或计划中的领导者及共同作者, 是预期要与共同作者在关键决定上做协商. 特别是该决定与共同作者所拥有的次系统有关(也就是说, 有投资时间并负责任). 一个有智慧的领导者, 认识出计划内部产权界线的作用, 是不会轻率地影响或反转由次系统拥有者的决定.

有些很大的计划完全不吃`仁慈的独裁者'这一套. 一个方法是转共同作者成为投票委员(像Apache). 另一种为轮换独裁者, 即控制权由一位成员轮换到另一位资深成员(为Perl组织所采用).

这样复杂的安排通常被考虑为不太稳定及复杂. 很明显地这些困难的认知, 为这些委员及设计者本身所了解; 这些问题是玩家文化中, 大家清悉地了解的.  不论如何, 我认为有些玩家对委员会或轮换独裁者组织心中不舒服, 因为它与过去Locke模型的思考大不相同. 在这些复杂的组织中, 不论是做拥有权或是名望回报的计算是有问题的. 很难看出其内部界线在那里, 除非高度的协调及信任, 否则很难避免冲突.

--------------------------------------------------------------------------------

[目录]

冲突与冲突解决

16. 冲突与冲突解决
我们曾经见过在计划中, 角色复杂度的增加是由设计权威及部份产权的分布来表现. 这是个有效的方式来分散诱因, 同时也稀释了计划领导者的权威性 -- 更重要地, 它稀释了领导对潜在冲突镇压的权威性.
当设计上的技术争执看来会导致互相残杀的冲突时,  它们很少是起因於吵架的. 这些通常都是由个别的领域的权威来负责解决.
另一种解决冲突的方式是"资深" -- 如果两个贡献者或贡献群有所争执, 而争执无法客观地解决, 并且都不拥有该争执的领域, 曾经为这个计划付出较多的一方胜利(即在计划中, 拥有较多产权的一方胜利).

这些法则通常都足以解决大部份计划的争议. 当这些不生效时, 计划领导者的命令通常可解决. 争议在通过这两关之後还存在的很罕见.

冲突除非在两个关键上都指向不同的方向的时候("当权者要负责"及"长者胜利")是不会变得严重的, 而计划领导者的权威是很微弱或不发生作用的. 这种状况最明显地是在继承权争执时, 而领导者不在时. 我曾见过这样的战斗一次. 非常地丑陋, 痛苦, 折磨, 只有当所有参与者都变得精疲力竭时, 将它交给外人来处理, 我锺心地期望不要再看见任何这一类的争执.

最终, 当所有冲突解决机制, 在所有玩家社团中都失效时. 唯一剩下的机制只剩吵架及躲避 -- 对那些拒绝合作, 破坏传统的人们的公众审判.


[目录]

薪传机制及与学界的关联

17. 薪传机制及与学界的关联
在这篇文章的先前版本曾提出一个研究问题: 这个团体究竟是如何告知并指导成员符合习俗?  这些习俗是否是在半知觉状态下, 不言自明的或自我组织起来的?  是否是由范例所导引的?  是否由明白的教条所授?
由教条所授明显的很罕见, 因为至今为止只有少数几条该文化的规范曾经有人提出过..
大部份的规范都是由范例所导引的. 用个很简单的例子, 在所有软件发行中的规范中, 应该都会有个README或READ.ME档案, 用以说明这个软件的简略说明. 这个传统至少从1980年早期就已经建立起来了, 但至今尚未有人将本条写下. 这是由观察许多软件发行而来的.

另一方面, 有些玩家习俗是自我组织起来的, 尤其是一旦有其中的份子了解到这个基本的(或者不自觉地)名望游戏. 大部份玩家不需要被传授以我在第三节所列的三个禁忌, 或者可说这些是不言自明的. 这个现象导出更进一步的分析 -- 而我们可能可以在探讨玩家在探索该文化智库的过程中找到解释.

大部份文化透过暗示(更精确一点`神秘', 透过宗教式或神秘性的方法)做为薪传机制. 这些秘密对外来者是不揭露的, 但可预期到可被热诚的新手所发现或理解. 要被内部所接受, 他需要展示他对该文化的神秘的了解及学习程度, 以接受认可.

玩家文化对此是不凡地警觉, 并且大量使用这些暗示及测试. 我们可在至少三种层次的过程运作中找到:

犹如密码式的神秘. 拿个例子, 在USENET新闻讨论群中有个alt.sysadmin.recovery有个很明显的秘密; 您不可能在不知道之前就提出问题, 而知道该秘密您才会被准许进入. 其中的老手对揭露这个秘密有很严格的禁忌.
对特定技术秘密的基本要求. 一个人必须要在送出礼物之前先吸收大量的技术知识(例如, 必须要至少了解一种主要的电脑语言). 这个隐藏线索由大处至小节都有作用, 犹如做为品质过滤(诸如抽象思考力, 坚持, 及精神力量)的功能用以发挥该文化的力量.
神秘社会内容. 要与该文化迁连上关系必须要参加特定计划. 每个计划都是活生生的玩家社会文化范例, 即贡献者必须要调查并了解该团体的社会及技术以便能有效参与. (具体来说, 一般的方式是透过阅读该计划的网页或邮件) 透过这些计划团体, 新手体验到有经验的老玩家的社会行为.
在透过探索这些秘密的过程中, 这位未来的玩家学到丰富的知识, 而使得这些禁条及其它习俗不言自明.
有些人可能会偶然议论玩家礼物文化结构是其中心秘密. 一个人在掏心吐胆的展现其对名望游戏及其暗藏的习俗, 禁忌及使用的了解之前, 是不被考虑为受传承的. 但这不重要; 所有文化都对其未来的参与者有这样的要求. 更进一步地说, 玩家文化表明对其参与者没有野心 -- 或者, 至少, 没有人因为我揭露这些而来跟我吵架!

有大量的人对本文回应指出玩家拥有权习俗看来十分接近(而且很可能直接源於)学术界的业务, 特别是科学研究团体. 研究团体有很类似的问题, 特别是在开采潜在可能的思考领域, 及展现非常类似的对问题可行的解决途径上, 使用同跻检视及名望.

既然许多玩家曾经在学界打滚过(通常都是在大学时学会玩电脑), 在了解过玩家文化後, 将玩家文化与学界做会类比自然是不稀奇的.

玩家`礼物文化'有明显地与学界平行类同的特质. 一旦研究员取得终身职, 他不再需要担心生存问题(的确, 终身职的观念可回溯至早期的礼物文化, 即``自然哲学家''基本上是富裕的绅士, 时间满满可奉献於研究上.) 在生存的危机解除後, 名望成了驱动的目标, 也就是在期刊或媒体上, 鼓励分享新点子及研究成果. 这造成客观而正面的功能, 因为科学研究, 就如玩家文化一般, 非常依赖`站在巨人的肩膀上', 而不需要一再地重新发现一些非常基本的原理.

有些人则进一步推论玩家文化只是研究团体风气的一种反射, 而且已经几乎到达相当程度. 这可能讲得过头了, 因为玩家文化似乎不过是以高中学历的聪明人们所架构起来的!

这里还有一些有趣的可能性存在. 我怀疑学界与玩家文化的类同型式, 并不止是因为其起源类同, 还因为它们所做的事, 在自然法则及人类本能的连系之下, 而演化出最理想的社会组织.  历史的裁决似乎断定自由市场资本主义是整体来说对经济效益最佳的方式; 或者, 以类同的方式来说, 名望游戏礼物文化是在生产(及检验)高品质创造力的工作上最佳的合作方式.

这一点如果是真的话, 那就比学术兴趣更加有意义了.  因为这提出了一点与教堂观与市集观中稍微不同的观点; 即, 最终, 软件产品工业资本家模式的末日到来, 而将被排出竞争, 从资本主义开始产生大量的剩馀财富, 而导致大量程式设计师生活下饥荒後的礼物文化下的那一刻开始. (译注: 换言之, 也就是大软件公司自掘坟场. 任意垄断操纵软件市场的後果, 造成软件设计师无法生存, 引发大规模投向以名望为基础的礼物文化中的活动, 增强了玩家文化的後盾. 我们可以在科技史的发展上发现许多类同现象.)

--------------------------------------------------------------------------------

[目录]

结论_ 由文化到文化规范

18. 结论: 由文化到文化规范
我们已经检视了用以控制及规范开放原始码软件产权的习俗. 我们已经见到这是如何揭露出在其下的权益特质及与Locke的土地产权理论的关系. 我们已经将玩家文化与`礼物文化'关连起来, 即参与者透过投入时间, 精力, 及创造力来竞争名望. 我们已经检验过在文化中的有相关的冲突解决分析.
下一个合理的问题应该是"为何这些这麽重要?" 玩家们并无意识分析而发展出这些习俗, 而且直到今日, 下意识地遵守这些习俗. 这些有意识地分析并没有立即明显地有任何的实用性 -- 除非, 或许, 我们可以进一步推展这些描述成为处方, 并演绎出一些改善这些习俗功能的方法.
在英式美国一般法传统下, 在玩家文化及土地产权理论之间, 我们已经发现一个相当合理的类比. 历史上来说 [Miller], 欧洲种族文化发明了这个传统来解决他们的争执解决系统, 即由未书写的, 下意识的习俗系统到明白地由种族中智者所记下的惯例法规 -- 然後最後白纸黑字写下.

或者, 既然我们的人口逐步上扬, 而对所有成员的薪传越来越困难, 是为玩家文化做点类同的事的时候了 -- 发展一套对解决各种争议的实用"程式码"(written code), 可增加在开放原始码计划的实力, 及一套仲裁传统, 即团体中的资深成员可做一些争执调解.

本文中的分析已经将这样的"程式码"大纲划出, 将过去暗示性的变成明白书写的. 不会出现在以上没有出现过的; 他们需要由各计划的建立者或拥有者志愿采用. 也不会完全地苛刻, 因为在该文化的压力会随时间而改变. 最後, 要将这样的"程式"付诸实现, 他们必须要反射出该玩家部落的广泛接受.

我已经开始进行这样的"程式码", 可能会叫做"Malvern Protocol", 以我所住的小镇名字来命名. 如果在本文中的这些分析逐渐受到广泛接受, 我会让大众都可取得Malvern Protocol, 用以解决争执范例"程式". 有兴趣批评及发展这套"程式"的团体, 或者希望提供一些他们的想法的, 都很欢迎与我连络.


[目录]

对进一步研究的一些问题

19. 对进一步研究的一些问题
该文化(也是我自己的文化)了解到, 不跟著一位仁慈的独裁者的模式是脆弱的. 这样的计划大多都失败了. 有些则惊人地成功而重要(Perl, Apache, KDE). 没有人真正了解其中的差别在何处. (每个计划的生命力, 与其参与者的组织动力生息相关, 是个相当暧 的观点, 是否真得存在一个组织可重复实行无碍的策略呢?)
就观察到的事实而论, 我们确实发现到, 成功的计划, 获取比需要相同工作量的除错及协助成功计划的工作来得更高的名望. 这是否是个对相同付出的理性价值评断呢?  或者它是个我们在此所演绎出来潜意识的领土模型所造成的次级效应?

[目录]

参考文件, 附注, 及感谢

20. Bibliography, Notes, and Acknowledgements
[Miller] Miller, William Ian; Bloodtaking and Peacemaking: Feud, Law, and Society in Saga Iceland; University of Chicago Press 1990, ISBN 0-226-52680-1. A fascinating study of Icelandic folkmoot law, which both illuminates the ancestry of the Lockean theory of property and describes the later stages of a historical process by which custom passed into customary law and thence to written law.
[Mal] Malaclypse the Younger; Principia Discordia, or How I Found Goddess and What I Did To Her When I Found Her; Loompanics, ISBN 1-55950-040-9. Amidst much enlightening silliness, the `SNAFU principle' provides a rather trenchant analysis of why command hierarchies don't scale well. There's a browseable HTML version.

[BCT] J. Barkow, L. Cosmides, and J. Tooby (Eds.); The adapted mind: Evolutionary psychology and the generation of culture. New York: Oxford University Press 1992. An excellent introduction to evolutionary psychology. Some of the papers bear directly on the three cultural types I discuss (command/exchange/gift), suggesting that these patterns are wired into the human psyche fairly deep.

[MHG] Goldhaber, Michael K.; The Attention Economy and the Net. I discovered this paper after my version 1.7. It has obvious flaws (Goldhaber's argument for the inapplicability of economic reasoning to attention does not bear close examination), but Goldhaber nevertheless has funny and perceptive things to say about the role of attention-seeking in organizing behavior. The prestige or peer repute I have discussed can fruitfully be viewed as a particular case of attention in his sense.

[HH] I have summarized the history of hackerdom at http://www.tuxedo.org/~esr/faqs/hacker-hist.html. The book that will explain it really well remains to be written, probably not by me.

[N] The term `noosphere' is an obscure term of art in philosophy derived from the Greek `nous' meaning `mind', `spirit', or `breath'. It is pronounced KNOW-uh-sfeer (two o-sounds, one long and stressed, one short and unstressed tending towards schwa). If one is being excruciatingly correct about one's orthography, it is properly spelled with a diaresis over one `o' -- just don't ask me which one.

[RP] There are some subtleties about rogue patches. One can divide them into `friendly' and `unfriendly' types. A `friendly' patch is designed to be merged back into the project's main-line sources under the maintainer's control (whether or not that merge actually happens); an `unfriendly' one is intended to yank the project in a direction the maintainer doesn't approve. Some projects (notably the Linux kernel itself) are pretty relaxed about friendly patches and even encourage independent distribution of them as part of their beta-test phase. An unfriendly patch, on the other hand, represents a decision to compete with the original and is a serious matter. Maintaining a whole raft of unfriendly patches tends to lead to forking.

I am indebted to Michael Funk <mwfunk@uncc.campus.mci.net> for pointing out how instructive a contrast with hackers the pirate culture are. Robert Lanphier <robla@real.com> contributed much to the discussion of egoless behavior. Eric Kidd <eric.kidd@pobox.com> highlighted the role of valuing humility in preventing cults of personality. The section on global effects was inspired by comments from Daniel Burn <daniel@tsathoggua.lab.usyd.edu.au>. Mike Whitaker <mrw@entropic.co.uk> inspired the main thread in the section on acculturation.


\section{1}
\section{2}

\backmatter

\end{document}