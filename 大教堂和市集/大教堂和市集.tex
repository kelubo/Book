% 大教堂和市集
% 大教堂和市集.tex

\documentclass[a4paper,12pt,UTF8,twoside]{ctexbook}

% 设置纸张信息。
\RequirePackage[a4paper]{geometry}
\geometry{
	%textwidth=138mm,
	%textheight=215mm,
	%left=27mm,
	%right=27mm,
	%top=25.4mm, 
	%bottom=25.4mm,
	%headheight=2.17cm,
	%headsep=4mm,
	%footskip=12mm,
	%heightrounded,
	inner=1in,
	outer=1.25in
}

% 设置字体,并解决显示难检字问题。
\xeCJKsetup{AutoFallBack=true}
\setCJKmainfont{SimSun}[BoldFont=SimHei, ItalicFont=KaiTi, FallBack=SimSun-ExtB]

% 目录 chapter 级别加点(.)。
\usepackage{titletoc}
\titlecontents{chapter}[0pt]{\vspace{3mm}\bf\addvspace{2pt}\filright}{\contentspush{\thecontentslabel\hspace{0.8em}}}{}{\titlerule*[8pt]{.}\contentspage}

% 设置 part 和 chapter 标题格式。
\ctexset{
	chapter/name={},
	chapter/number={}
}

% 设置古文原文格式。
\newenvironment{yuanwen}{\bfseries\zihao{4}}

% 设置署名格式。
\newenvironment{shuming}{\hfill\bfseries\zihao{4}}

\title{\heiti\zihao{0} The Cathedral and the Bazaar}
\author{Eric Steven Raymond}
\date{1997 - 5 - 21}

\begin{document}

\maketitle
\tableofcontents

\chapter{Abstract}

I anatomize a successful open-source project, fetchmail, that was run as a deliberate test of the surprising theories about software engineering suggested by the history of Linux. I discuss these theories in terms of two fundamentally different development styles, the ``cathedral'' model of most of the commercial world versus the ``bazaar'' model of the Linux world. I show that these models derive from opposing assumptions about the nature of the software-debugging task. I then make a sustained argument from the Linux experience for the proposition that ``Given enough eyeballs, all bugs are shallow'', suggest productive analogies with other self-correcting systems of selfish agents, and conclude with some exploration of the implications of this insight for the future of software.

\chapter{摘要}
我剖析了一个成功的开源项目fetchmail, 作为一个关于软件的令人惊讶的理论的故意测试运行 Linux的历史所暗示的工程学。 我讨论这些 两种根本不同的发展模式, 大多数商业世界的"大教堂“模式与"集市”模式 Linux世界的典范。 我证明这些模型来自于 关于软件调试任务的性质的相反假设。 然后,我从Linux的经验中提出了一个持续的论点, “只要有足够的眼球,所有的虫子都是肤浅的”, 建议生产类比与其他自我纠正系统, 自私的代理人,并得出结论,一些探索的影响 对软件未来的洞察力。

http://www.catb.org/esr/writings/cathedral-bazaar/cathedral-bazaar/index.html

\chapter{The Cathedral and the Bazaar}

Linux is subversive. Who would have thought even five years ago (1991) that a world-class operating system could coalesce as if by magic out of part-time hacking by several thousand developers scattered all over the planet, connected only by the tenuous strands of the Internet?

Certainly not I. By the time Linux swam onto my radar screen in early 1993, I had already been involved in Unix and open-source development for ten years. I was one of the first GNU contributors in the mid-1980s. I had released a good deal of open-source software onto the net, developing or co-developing several programs (nethack, Emacs's VC and GUD modes, xlife, and others) that are still in wide use today. I thought I knew how it was done.

Linux overturned much of what I thought I knew. I had been preaching the Unix gospel of small tools, rapid prototyping and evolutionary programming for years. But I also believed there was a certain critical complexity above which a more centralized, a priori approach was required. I believed that the most important software (operating systems and really large tools like the Emacs programming editor) needed to be built like cathedrals, carefully crafted by individual wizards or small bands of mages working in splendid isolation, with no beta to be released before its time.

Linus Torvalds's style of development—release early and often, delegate everything you can, be open to the point of promiscuity—came as a surprise. No quiet, reverent cathedral-building here—rather, the Linux community seemed to resemble a great babbling bazaar of differing agendas and approaches (aptly symbolized by the Linux archive sites, who'd take submissions from anyone) out of which a coherent and stable system could seemingly emerge only by a succession of miracles.

The fact that this bazaar style seemed to work, and work well, came as a distinct shock. As I learned my way around, I worked hard not just at individual projects, but also at trying to understand why the Linux world not only didn't fly apart in confusion but seemed to go from strength to strength at a speed barely imaginable to cathedral-builders.

By mid-1996 I thought I was beginning to understand. Chance handed me a perfect way to test my theory, in the form of an open-source project that I could consciously try to run in the bazaar style. So I did—and it was a significant success.

This is the story of that project. I'll use it to propose some aphorisms about effective open-source development. Not all of these are things I first learned in the Linux world, but we'll see how the Linux world gives them particular point. If I'm correct, they'll help you understand exactly what it is that makes the Linux community such a fountain of good software—and, perhaps, they will help you become more productive yourself.


\chapter{大教堂和集市}


Linux的影响是非常巨大的。甚至在5年以前,有谁能够想象一个世界级的操作系统能够仅仅用细细的Internet连接起来的散布在全球的几千个开发人员有以业余时间来创造呢?

我当然不会这么想。在1993年早期我开始注意Linux时,我已经参与Unix和自由软件开发达十年之久了。我是八十年代中期GNU最早的几个参与者之一。我已经在网上发布了大量的自由软件,开发和协助开发了几个至今仍在广泛使用的程序(Nethack,Emacs VC和GND模式,xlife等等)。我想我知道该怎样做。


Linux推翻了许多我认为自己明白的事情。我已经宣扬小工具、快速原型和演进式开发的Unix福音多年了。但是我也相信某些重要的复杂的事情需要更集中化的,严密的方法。我相信多数重要的软件(操作系统和象Emacs一样的真正大型的工具)需要向建造大教堂一样来开发,需要一群于世隔绝的奇才的细心工作,在成功之前没有beta版的发布。

Linus Torvalds的开发风格(尽早尽多的发布,委托所有可以委托的事,对所有的改动和融合开放)令人惊奇的降临了。这里没有安静的、虔诚的大教堂的建造工作——相反,Linux团体看起来像一个巨大的有各种不同议程和方法的乱哄哄的集市(Linux归档站点接受任何人的建议和作品,并聪明的加以管理),一个一致而稳定的系统就象奇迹一般从这个集市中产生了。


这种设计风格确实能工作,并且工作得很好,这个事实确实是一个冲击。在我的研究过程中,我不仅在单个工程中努力工作,而且试图理解为什么Linux世界不仅没有在一片混乱中分崩离析,反而以大教堂建造者们不可想象的速度变得越来越强大。


到了 1996 年中,我想我开始理解了。我有一个极好的测试我的理论的机会,以一个自由软件计划的形式,我有意识的是用了市集风格。我这样做了,并取得了很大的成功。


在本文的余下部分,我将讲述这个计划的故事,我用它来明确一些自由软件高效开发的格言。并不是所有这些都是从Linux世界中学到的,但我们将看到Linux世界给予了它们一个什么样的位置。如果我是正确的,它们将使你理解是什么使Linux团体成为好软件的源泉,帮助你变得更加高效。

\chapter{The Mail Must Get Through}

Since 1993 I'd been running the technical side of a small free-access Internet service provider called Chester County InterLink (CCIL) in West Chester, Pennsylvania. I co-founded CCIL and wrote our unique multiuser bulletin-board software—you can check it out by telnetting to locke.ccil.org. Today it supports almost three thousand users on thirty lines. The job allowed me 24-hour-a-day access to the net through CCIL's 56K line—in fact, the job practically demanded it!

I had gotten quite used to instant Internet email. I found having to periodically telnet over to locke to check my mail annoying. What I wanted was for my mail to be delivered on snark (my home system) so that I would be notified when it arrived and could handle it using all my local tools.

The Internet's native mail forwarding protocol, SMTP (Simple Mail Transfer Protocol), wouldn't suit, because it works best when machines are connected full-time, while my personal machine isn't always on the Internet, and doesn't have a static IP address. What I needed was a program that would reach out over my intermittent dialup connection and pull across my mail to be delivered locally. I knew such things existed, and that most of them used a simple application protocol called POP (Post Office Protocol). POP is now widely supported by most common mail clients, but at the time, it wasn't built in to the mail reader I was using.

I needed a POP3 client. So I went out on the Internet and found one. Actually, I found three or four. I used one of them for a while, but it was missing what seemed an obvious feature, the ability to hack the addresses on fetched mail so replies would work properly.

The problem was this: suppose someone named `joe' on locke sent me mail. If I fetched the mail to snark and then tried to reply to it, my mailer would cheerfully try to ship it to a nonexistent `joe' on snark. Hand-editing reply addresses to tack on <@ccil.org> quickly got to be a serious pain.

This was clearly something the computer ought to be doing for me. But none of the existing POP clients knew how! And this brings us to the first lesson:

1. Every good work of software starts by scratching a developer's personal itch.

Perhaps this should have been obvious (it's long been proverbial that ``Necessity is the mother of invention'') but too often software developers spend their days grinding away for pay at programs they neither need nor love. But not in the Linux world—which may explain why the average quality of software originated in the Linux community is so high.

So, did I immediately launch into a furious whirl of coding up a brand-new POP3 client to compete with the existing ones? Not on your life! I looked carefully at the POP utilities I had in hand, asking myself ``Which one is closest to what I want?'' Because:

2. Good programmers know what to write. Great ones know what to rewrite (and reuse).

While I don't claim to be a great programmer, I try to imitate one. An important trait of the great ones is constructive laziness. They know that you get an A not for effort but for results, and that it's almost always easier to start from a good partial solution than from nothing at all.

Linus Torvalds, for example, didn't actually try to write Linux from scratch. Instead, he started by reusing code and ideas from Minix, a tiny Unix-like operating system for PC clones. Eventually all the Minix code went away or was completely rewritten—but while it was there, it provided scaffolding for the infant that would eventually become Linux.

In the same spirit, I went looking for an existing POP utility that was reasonably well coded, to use as a development base.

The source-sharing tradition of the Unix world has always been friendly to code reuse (this is why the GNU project chose Unix as a base OS, in spite of serious reservations about the OS itself). The Linux world has taken this tradition nearly to its technological limit; it has terabytes of open sources generally available. So spending time looking for some else's almost-good-enough is more likely to give you good results in the Linux world than anywhere else.

And it did for me. With those I'd found earlier, my second search made up a total of nine candidates—fetchpop, PopTart, get-mail, gwpop, pimp, pop-perl, popc, popmail and upop. The one I first settled on was `fetchpop' by Seung-Hong Oh. I put my header-rewrite feature in it, and made various other improvements which the author accepted into his 1.9 release.

A few weeks later, though, I stumbled across the code for popclient by Carl Harris, and found I had a problem. Though fetchpop had some good original ideas in it (such as its background-daemon mode), it could only handle POP3 and was rather amateurishly coded (Seung-Hong was at that time a bright but inexperienced programmer, and both traits showed). Carl's code was better, quite professional and solid, but his program lacked several important and rather tricky-to-implement fetchpop features (including those I'd coded myself).

Stay or switch? If I switched, I'd be throwing away the coding I'd already done in exchange for a better development base.

A practical motive to switch was the presence of multiple-protocol support. POP3 is the most commonly used of the post-office server protocols, but not the only one. Fetchpop and the other competition didn't do POP2, RPOP, or APOP, and I was already having vague thoughts of perhaps adding IMAP (Internet Message Access Protocol, the most recently designed and most powerful post-office protocol) just for fun.

But I had a more theoretical reason to think switching might be as good an idea as well, something I learned long before Linux.

3. ``Plan to throw one away; you will, anyhow.'' (Fred Brooks, The Mythical Man-Month, Chapter 11)

Or, to put it another way, you often don't really understand the problem until after the first time you implement a solution. The second time, maybe you know enough to do it right. So if you want to get it right, be ready to start over at least once \footnote{In Programing Pearls, the noted computer-science aphorist Jon Bentley comments on Brooks's observation with ``If you plan to throw one away, you will throw away two.''. He is almost certainly right. The point of Brooks's observation, and Bentley's, isn't merely that you should expect first attempt to be wrong, it's that starting over with the right idea is usually more effective than trying to salvage a mess.}.

Well (I told myself) the changes to fetchpop had been my first try. So I switched.

After I sent my first set of popclient patches to Carl Harris on 25 June 1996, I found out that he had basically lost interest in popclient some time before. The code was a bit dusty, with minor bugs hanging out. I had many changes to make, and we quickly agreed that the logical thing for me to do was take over the program.

Without my actually noticing, the project had escalated. No longer was I just contemplating minor patches to an existing POP client. I took on maintaining an entire one, and there were ideas bubbling in my head that I knew would probably lead to major changes.

In a software culture that encourages code-sharing, this is a natural way for a project to evolve. I was acting out this principle:

4. If you have the right attitude, interesting problems will find you.

But Carl Harris's attitude was even more important. He understood that

5. When you lose interest in a program, your last duty to it is to hand it off to a competent successor.

Without ever having to discuss it, Carl and I knew we had a common goal of having the best solution out there. The only question for either of us was whether I could establish that I was a safe pair of hands. Once I did that, he acted with grace and dispatch. I hope I will do as well when it comes my turn.

\chapter{邮件必须得通过}

1993年以前我在一个小的免费访问的名为Chester County InterLink的ISP的做技术工作,它位于Pennsylvania的West Chester。(我协助建立了CCIL,并写了我们独特的多用户BBS系统——你可以telnet到locke.ccil.org来检测一下。今天它在十九条线上支持三千的用户)。这个工作使我可以一天二十四小时通过CCIL的56K专线连在网上,实际上,它要求我怎么做!


所以,我对Internet email很熟悉。因为复杂的原因,很难在我家里的机器(snark.thyrsus.com)和CCIL之间用SLIP工作。最后我终于成功了,但我发现不得不时常telnet到locke来检查我的邮件,这真是太烦了。我所需要的是我的邮件发送到snark,这样biff(1)会在它到达时通知我。


简单地sendmail的转送功能是不够的,因为snark并不是总在网上而且没有一个静态地址。我需要一个程序通过我的SLIP连接把我的本地发送的邮件拉过来。我知道这种东西是存在的,它们大多使用一个简单的协议POP(Post Office Protocol)。而且,locke的BSD/OS操作系统已经自带了一个POP3服务器。


我需要一个POP3客户。所以我到网上去找到了一个。实际上,我发现了三、四个。我用了一会pop-perl,但它却少一个明显的特征:抽取收到的邮件的地址以便正确回复。


问题是这样的:假设locke上一个叫“joe”的人向我发了一封邮件。如果我把它取到snark上准备回复时,我的邮件程序会很高兴地把它发送给一个不存在的snark上的“joe”。手工的在地址上加上“@ccil.org”变成了一个严酷的痛苦。


这显然应是计算机替我做的事。(实际上,依据RFC1123的5.2.18节,sendmail应该做这件事)。但是没有一个现存的POP客户知道怎样做!于是这就给我们上了第一课:

1.每个好的软件工作都开始于搔到了开发者本人的痒处。

也许这应该是显而易见的(“需要是发明之母”长久以来就被证明是正确的),但是软件开发人员常常把他们的精力放在它们既不需要也不喜欢的程序,但在Linux世界中却不是这样——这解释了为什么从Linux团体中产生的软件质量都如此之高。


那么,我是否立即投入疯狂的工作中,要编出一个新的POP3客户与现存的那些竞争呢?才不是哪!我仔细考察了手头上的POP工具,问自己“那一个最接近我的需要?”因为:
2.好程序员知道该写什么,伟大的程序员知道该重写(和重用)什么。


我并没有声称自己是一个伟大的程序员,可是我试着效仿他们。伟大程序员的一个重要特点是建设性的懒惰。他们知道你是因为成绩而不是努力得到奖赏,而且从一个好的实际的解决方案开始总是要比从头干起容易。


例如,Linux并不是从头开始写Linux的。相反的它从重用Minix(一个386机型上的类似Unix的微型操作系统)的代码和思想入手。最后所有的Minix代码都消失或被彻底的重写了,但是当它们在的时候它为最终成为Linux的雏形做了铺垫。


秉承同样的精神,我去寻找良好编码的现成的POP工具,用来作为基础。


Unix世界中的代码共享传统一直对代码重用很友好(这正是为什么GNU计划不管Unix本身有多么保守而选取它作为基础操作系统的原因)。Linux世界把这个传统推向技术极限:它有几个T字节的源代码可以用。所以在Linux世界中花时间寻找其他几乎足够好的东西,会比在别处带来更好的结果。


这也适合我。加上我先前发现的,第二次寻找找到了9个候选者——fetchPOP,PopTart,get-mail,gwpop,pimp,pop-perl,popc,popmail 和 upop)。我首先选定的是“fetchpop”。我加入了头标重写功能,并且做了一些被作者加入他的1.9版中的改进。


但是几个星期之后,我偶然发现了Carl Harris写的“popclient”的代码,然后发现有个问题,虽然fetchpop有一些好的原始思想(比如它的守护进程模式),它只能处理pop3,而且编码的水平相当业余(Seung-Hong是个很聪明但是经验不足的程序员),Carl的代码更好一些,相当专业和稳固,但他的程序缺少几个重要的相当容易实现的fetchpop的特征(包括我自己写的那些)。


继续呢还是换一个? 如果换一个的话,作为得到一个更好开发基础的代价,我就要扔掉我已经有的那些代码。


换一个的一个实际的动机是支持多协议,pop3是用的最广的邮局协议,但并非唯一一个,Fetchpop和其余几个没有实现POP2.RPOP,或者APOP,而且我还有一个为了兴趣加入IMAP(Internet Message Access Protocol,最近设计的最强大的邮局协议)的模糊想法。


但是我有一个更加理论化的原因认为换一下会是一个好主意,这是我在Linux很久以前学到的:

3.“计划好抛弃,无论如何,你会的”(Fred Brooks,《神秘的人月》第11章)


或者换句话说,你常常在第一次实现一个解决方案之后才能理解问题所在,第二次你也许才足够清楚怎样做好它,因此如果你想做好,准备好推翻重来至少一次。


好吧(我告诉自己),对fetchpop的尝试是我第一次的尝试,因此我换了一下。


当我在1996年6月25日把我第一套popclient的补丁程序寄给Carl Harris之后,我发现一段时间以前他已经对popclient基本上失去了兴趣,这些代码有些陈旧,有一些次要的错误,我有许多修改要做,我们很快达成一致,我来接手这个程序。不知不觉的,这个计划扩大了,再也不是我原先打算的在已有的pop客户上加几个次要的补丁而已了,我得维护整个的工程,而且我脑袋里涌动着一些念头要引起一个大的变化。


在一个鼓励代码共享的软件文化里,这是一个工程进化的自然道路,我要指出:

4. 如果你有正确的态度,有趣的问题会找上你的,但是Carl Harris的态度甚至更加重要,他理解:

5.当你对一个程序失去兴趣时,你最后的责任就是把它传给一个能干的后继者。


甚至没有商量,Carl和我知道我们有一个共同目标就是找到最好的解决方案,对我们来说唯一的问题是我能否证明我有一双坚强的手,他优雅而快速的写出了程序,我希望轮到我时我也能做到。

\chapter{The Importance of Having Users}

And so I inherited popclient. Just as importantly, I inherited popclient's user base. Users are wonderful things to have, and not just because they demonstrate that you're serving a need, that you've done something right. Properly cultivated, they can become co-developers.

Another strength of the Unix tradition, one that Linux pushes to a happy extreme, is that a lot of users are hackers too. Because source code is available, they can be effective hackers. This can be tremendously useful for shortening debugging time. Given a bit of encouragement, your users will diagnose problems, suggest fixes, and help improve the code far more quickly than you could unaided.

6. Treating your users as co-developers is your least-hassle route to rapid code improvement and effective debugging.

The power of this effect is easy to underestimate. In fact, pretty well all of us in the open-source world drastically underestimated how well it would scale up with number of users and against system complexity, until Linus Torvalds showed us differently.

In fact, I think Linus's cleverest and most consequential hack was not the construction of the Linux kernel itself, but rather his invention of the Linux development model. When I expressed this opinion in his presence once, he smiled and quietly repeated something he has often said: ``I'm basically a very lazy person who likes to get credit for things other people actually do.'' Lazy like a fox. Or, as Robert Heinlein famously wrote of one of his characters, too lazy to fail.

In retrospect, one precedent for the methods and success of Linux can be seen in the development of the GNU Emacs Lisp library and Lisp code archives. In contrast to the cathedral-building style of the Emacs C core and most other GNU tools, the evolution of the Lisp code pool was fluid and very user-driven. Ideas and prototype modes were often rewritten three or four times before reaching a stable final form. And loosely-coupled collaborations enabled by the Internet, a la Linux, were frequent.

Indeed, my own most successful single hack previous to fetchmail was probably Emacs VC (version control) mode, a Linux-like collaboration by email with three other people, only one of whom (Richard Stallman, the author of Emacs and founder of the Free Software Foundation) I have met to this day. It was a front-end for SCCS, RCS and later CVS from within Emacs that offered ``one-touch'' version control operations. It evolved from a tiny, crude sccs.el mode somebody else had written. And the development of VC succeeded because, unlike Emacs itself, Emacs Lisp code could go through release/test/improve generations very quickly.

The Emacs story is not unique. There have been other software products with a two-level architecture and a two-tier user community that combined a cathedral-mode core and a bazaar-mode toolbox. One such is MATLAB, a commercial data-analysis and visualization tool. Users of MATLAB and other products with a similar structure invariably report that the action, the ferment, the innovation mostly takes place in the open part of the tool where a large and varied community can tinker with it.

三. 拥有用户的重要性


于是我继承了popclient,同样重要的是,我继承了popclient的用户基础,用户是你所拥有的极好的东西,不仅仅是因为他们显示了你正在满足需要,你做了正确的事情,如果加以适当的培养,他们可以成为合作开发者。


Unix传统另一有力之处是许多用户都是黑客,因为源优码是公开的,他们可以成为高效的黑客,这一点在Linux世界中也被推向了令人高兴的极致,这对缩短调试时间是极端重要的,在一点鼓励之下,你的用户会诊断问题,提出修订建议,帮你以远比你期望快得多的速度的改进代码。


6. 把用户当做协作开发者是快速改进代码和高效调试的无可争辩的方式。


这种效果的力量很容易被低估,实际上,几乎所有我们自由软件世界中的人都强烈低估了用户可以多么有效地对付系统复杂性,直到Linus让我们看到了这一点。


实际上,我认为Linus最聪明最了不起的工作不是创建了Linux内核本身,而是发明了Linux开发模式,当我有一次当着他的面表达这种观点时,他微笑了一下,重复了一句他经常说的话:“我基本上是一个懒惰的人,依靠他人的工作来获取成绩。”象狐狸一样懒惰,或者如Robert Heinlein所说,太懒了而不会失败。


回顾起来,在GNU Emacs Lisp库和Lisp代码集中可以看到Linux方法的成功,与Emacs的C内核和许多其他FSF的工具相比,Lisp代码库的演化是流动性的和用户驱动的,思想和原型在达到最终的稳定形式之前往往要重写三或四次,而且经常利用Internet的松散合作。


实际上,我自己在fetchmail之前最成功的作品要算Emacs VC模式,它是三个其他的人通过电子邮件进行的类似Linux的合作,至今我只见过其中一个人(Richard Stallman),它是SCCS、RCS和后来的CVS的前端,为Emacs提供“one-touch”版本控制操作,它是从一个微型的、粗糙的别人写好的sccs.el模式开始演化的,VC开发的成功不像Emacs本身,而是因为Emacs Lisp代码可以很快的通过发布/测试/改进的过程。


(FSF的试图把代码放入GPL之下的策略有一个未曾预料到的副作用,它让FSF难以采取市集模式,因为他们认为每个想贡献二十行以上代码的人都必须得到一个授权,以使受到GPL的代码免受版权法的侵扰,具有BSD和MITX协会的授权的用户不会有这个问题,因为他们并不试图保留那些会使人可能受到质询的权力)。

\chapter{Release Early, Release Often}

Early and frequent releases are a critical part of the Linux development model. Most developers (including me) used to believe this was bad policy for larger than trivial projects, because early versions are almost by definition buggy versions and you don't want to wear out the patience of your users.

This belief reinforced the general commitment to a cathedral-building style of development. If the overriding objective was for users to see as few bugs as possible, why then you'd only release a version every six months (or less often), and work like a dog on debugging between releases. The Emacs C core was developed this way. The Lisp library, in effect, was not—because there were active Lisp archives outside the FSF's control, where you could go to find new and development code versions independently of Emacs's release cycle [QR].

The most important of these, the Ohio State Emacs Lisp archive, anticipated the spirit and many of the features of today's big Linux archives. But few of us really thought very hard about what we were doing, or about what the very existence of that archive suggested about problems in the FSF's cathedral-building development model. I made one serious attempt around 1992 to get a lot of the Ohio code formally merged into the official Emacs Lisp library. I ran into political trouble and was largely unsuccessful.

But by a year later, as Linux became widely visible, it was clear that something different and much healthier was going on there. Linus's open development policy was the very opposite of cathedral-building. Linux's Internet archives were burgeoning, multiple distributions were being floated. And all of this was driven by an unheard-of frequency of core system releases.

Linus was treating his users as co-developers in the most effective possible way:

7. Release early. Release often. And listen to your customers.

Linus's innovation wasn't so much in doing quick-turnaround releases incorporating lots of user feedback (something like this had been Unix-world tradition for a long time), but in scaling it up to a level of intensity that matched the complexity of what he was developing. In those early times (around 1991) it wasn't unknown for him to release a new kernel more than once a day! Because he cultivated his base of co-developers and leveraged the Internet for collaboration harder than anyone else, this worked.

But how did it work? And was it something I could duplicate, or did it rely on some unique genius of Linus Torvalds?

I didn't think so. Granted, Linus is a damn fine hacker. How many of us could engineer an entire production-quality operating system kernel from scratch? But Linux didn't represent any awesome conceptual leap forward. Linus is not (or at least, not yet) an innovative genius of design in the way that, say, Richard Stallman or James Gosling (of NeWS and Java) are. Rather, Linus seems to me to be a genius of engineering and implementation, with a sixth sense for avoiding bugs and development dead-ends and a true knack for finding the minimum-effort path from point A to point B. Indeed, the whole design of Linux breathes this quality and mirrors Linus's essentially conservative and simplifying design approach.

So, if rapid releases and leveraging the Internet medium to the hilt were not accidents but integral parts of Linus's engineering-genius insight into the minimum-effort path, what was he maximizing? What was he cranking out of the machinery?

Put that way, the question answers itself. Linus was keeping his hacker/users constantly stimulated and rewarded—stimulated by the prospect of having an ego-satisfying piece of the action, rewarded by the sight of constant (even daily) improvement in their work.

Linus was directly aiming to maximize the number of person-hours thrown at debugging and development, even at the possible cost of instability in the code and user-base burnout if any serious bug proved intractable. Linus was behaving as though he believed something like this:

8. Given a large enough beta-tester and co-developer base, almost every problem will be characterized quickly and the fix obvious to someone.

Or, less formally, ``Given enough eyeballs, all bugs are shallow.'' I dub this: ``Linus's Law''.

My original formulation was that every problem ``will be transparent to somebody''. Linus demurred that the person who understands and fixes the problem is not necessarily or even usually the person who first characterizes it. ``Somebody finds the problem,'' he says, ``and somebody else understands it. And I'll go on record as saying that finding it is the bigger challenge.'' That correction is important; we'll see how in the next section, when we examine the practice of debugging in more detail. But the key point is that both parts of the process (finding and fixing) tend to happen rapidly.

In Linus's Law, I think, lies the core difference underlying the cathedral-builder and bazaar styles. In the cathedral-builder view of programming, bugs and development problems are tricky, insidious, deep phenomena. It takes months of scrutiny by a dedicated few to develop confidence that you've winkled them all out. Thus the long release intervals, and the inevitable disappointment when long-awaited releases are not perfect.

In the bazaar view, on the other hand, you assume that bugs are generally shallow phenomena—or, at least, that they turn shallow pretty quickly when exposed to a thousand eager co-developers pounding on every single new release. Accordingly you release often in order to get more corrections, and as a beneficial side effect you have less to lose if an occasional botch gets out the door.

And that's it. That's enough. If ``Linus's Law'' is false, then any system as complex as the Linux kernel, being hacked over by as many hands as the that kernel was, should at some point have collapsed under the weight of unforseen bad interactions and undiscovered ``deep'' bugs. If it's true, on the other hand, it is sufficient to explain Linux's relative lack of bugginess and its continuous uptimes spanning months or even years.

Maybe it shouldn't have been such a surprise, at that. Sociologists years ago discovered that the averaged opinion of a mass of equally expert (or equally ignorant) observers is quite a bit more reliable a predictor than the opinion of a single randomly-chosen one of the observers. They called this the Delphi effect. It appears that what Linus has shown is that this applies even to debugging an operating system—that the Delphi effect can tame development complexity even at the complexity level of an OS kernel. [CV]

One special feature of the Linux situation that clearly helps along the Delphi effect is the fact that the contributors for any given project are self-selected. An early respondent pointed out that contributions are received not from a random sample, but from people who are interested enough to use the software, learn about how it works, attempt to find solutions to problems they encounter, and actually produce an apparently reasonable fix. Anyone who passes all these filters is highly likely to have something useful to contribute.

Linus's Law can be rephrased as ``Debugging is parallelizable''. Although debugging requires debuggers to communicate with some coordinating developer, it doesn't require significant coordination between debuggers. Thus it doesn't fall prey to the same quadratic complexity and management costs that make adding developers problematic.

In practice, the theoretical loss of efficiency due to duplication of work by debuggers almost never seems to be an issue in the Linux world. One effect of a ``release early and often'' policy is to minimize such duplication by propagating fed-back fixes quickly [JH].

Brooks (the author of The Mythical Man-Month) even made an off-hand observation related to this: ``The total cost of maintaining a widely used program is typically 40 percent or more of the cost of developing it. Surprisingly this cost is strongly affected by the number of users. More users find more bugs.'' [emphasis added].

More users find more bugs because adding more users adds more different ways of stressing the program. This effect is amplified when the users are co-developers. Each one approaches the task of bug characterization with a slightly different perceptual set and analytical toolkit, a different angle on the problem. The ``Delphi effect'' seems to work precisely because of this variation. In the specific context of debugging, the variation also tends to reduce duplication of effort.

So adding more beta-testers may not reduce the complexity of the current ``deepest'' bug from the developer's point of view, but it increases the probability that someone's toolkit will be matched to the problem in such a way that the bug is shallow to that person.

Linus coppers his bets, too. In case there are serious bugs, Linux kernel version are numbered in such a way that potential users can make a choice either to run the last version designated ``stable'' or to ride the cutting edge and risk bugs in order to get new features. This tactic is not yet systematically imitated by most Linux hackers, but perhaps it should be; the fact that either choice is available makes both more attractive. \footnote{The split between Linux's experimental and stable versions has another function related to, but distinct from, hedging risk. The split attacks another problem: the deadliness of deadlines. When programmers are held both to an immutable feature list and a fixed drop-dead date, quality goes out the window and there is likely a colossal mess in the making. I am indebted to Marco Iansiti and Alan MacCormack of the Harvard Business School for showing me me evidence that relaxing either one of these constraints can make scheduling workable.}

\chapter{早发布、常发布}

尽量早尽量频繁的发布是Linux开发模式的一个重要部分,多数开发人员(包括我)过去都相信这对大型工程来说是个不好的策略,因为早期版本都是些充满错误的版本,而你不想耗光用户的耐心。

这种信仰强化了建造大教堂开发方式的必要性,如果目标是让用户尽可能少的见到错误,那你怎能不会仅仅每六个月发布一次(或更不经常),而且在发布之间象一只狗一样辛勤“捉虫”呢? Emacs C内核就是以这种方式开发的,Lisp库,实际上却相反,因为有一些有FSF控制之外的Lisp库,在那里你可以独立于Emacs发布周期地找寻新的和开发代码版本。


这其中最重要的是Ohio州的elisp库,预示了今天的巨大的Linux库的许多特征的精神,但是我们很少真正仔细考虑我们在做什么,或者这个库的存在指出了FSF建造教堂式开发模式的什么问题,1992年我曾经做了一次严肃的尝试,想把Ohio的大量代码正式合并到Emacs的官方Lisp库中,结果我陷入了政治斗争中,彻底失败了。


但是一年之后,在Linux广泛应用之后,很清楚,一些不同的更加健康的东西诞生了,Linus的开发模式正好与建造教堂方式相反,Sunsite和tsx-11的库开始成长,推动了许多发布。所有这些都是闻所未闻的频繁的内核系统的发布所推动的。


Linus以所有实际可能的方式把它的用户作为协作开发人员。


7. 早发布、常发布、听取客户的建议

Linus的创新并不是这个(这在Unix世界中是一个长期传统),而是把它扩展到和他所开发的东西的复杂程度相匹配的地步,在早期一天一次发布对他来说都不是罕见的!而且因为他培育了他的协作开发者基础,比其他任何人更努力地充分利用了Internet进行合作,所以这确实能行。


但是它是怎样进行的呢?它是我能模仿的吗?还是这依赖于Linus的独特天才?


我不这样想,我承认Linus是一个极好的黑客(我们有多少人能够做出一个完整的高质量的操作系统内核?),但是Linux并不是一个令人敬畏的概念上的飞跃,Linus不是(至少还不曾是)象Richard stallman或James Gosling一样的创新天才,在我看来,Linus更象一个工程天才,具有避免错误和开发失败的第六感觉,掌握了发现从A点到B点代价最小的路径的决窍,确实,Linux的整个设计受益于这个特质,并反映出Linus的本质上保守和简化设计的方法。


如果快速的发布和充分利用Internet不是偶然而是Linus的对代价最小的路径的洞察力的工程天才的内在部分,那么他极大增强了什么?他创建了什么样的方法?


问题回答了它自己,Linus保持他的黑客用户经常受到激励和奖赏:被行动的自我满足的希望所激励,而奖赏则是经常(甚至每天)都看到工作在进步。


Linus直接瞄准了争取最多的投入调试和开发的人时,甚至冒代码不稳定和一旦有非常棘手的错误而失去用户基础的险,Linus似乎相信下面这个:

8. 如果有一个足够大的beta测试人员和协作开发人员的基础,几乎所有的问题都可以被快速的找出并被一些人纠正。


或者更不正式的讲:“如果有足够多的眼睛,所有的错误都是浅显的”(群众的眼睛是雪亮的),我把这称为“Linus定律”。


我最初的表述是每个问题“对某些人是透明的”,Linus反对说,理解和修订问题的那个人不一定非是甚至往往不是首先发现它的人,“某个人发现了问题”,他说,“另一个理解它,我认为发现它是个更大的挑战”,但是要点是所有事都趋向于迅速发生。


我认为这是建造教堂和集市模式的核心区别,在建造教堂模式的编程模式看来,错误和编程问题是狡猾的、阴险的、隐藏很深的现象,花费几个月的仔细检查,也不能给你多大确保把它们都挑出来的信心,因此很长的发布周期,和在长期等待之后并没有得到完美的版本发布所引起的失望都是不可避免的。


以市集模式观点来看,在另一方面,我们认为错误是浅显的现象,或者至少当暴露给上千个热切的协作开发人员,让他们来对每个新发布进行测试的时候,它们很快变得浅显了,所以我们经常发布来获得更多的更正,作为一个有益的副作用,如果你偶尔做了一个笨拙的修改,也不会损失太多。也许我们本不应该这样的惊奇,社会学家在几年前已经发现一群相同专业的(或相同无知的)观察者的平均观点比在其中随机挑选一个来得更加可靠,他们称此为“Delhpi效应”,Linus所显示的证明在调试一个操作系统时它也适用——Delphi效应甚至可以战胜操作系统内核一级的复杂度。


我受Jeff Dutky (dutky @ wam.umd.edu)的启发指出Linus定律可以重新表述为“调试可以并行”,Jeff观察到虽然调试工作需要调试人员和对应的开发人员相交流,但它不需要在调试人员之间进行大量的协调,于是它就没有陷入开发时遇到的平方复杂度和管理开销。

在实际中,由于重复劳动而导致的理论上的丧失效率的现象在Linux世界中并不是一个大问题,“早发布、常发布策略”的一个效果就是利用快速的传播反馈修订来使重复劳动达到最小。

Brooks甚至做了一个与Jeff相关的更精确的观察:“维护一个广泛使用的程序的成本一般是其开发成本的 40\%,奇怪的是这个成本受到用户个数的强烈影响,更多的用户发现更多的错误”(我的强调)。


更多的用户发现更多的错误是因为更多的用户提供了更多测试程序的方法,当用户是协作开发人员时这个效果被放大了,每个找寻错误的人都有自己稍微不同的感觉和分析工具,从不同角度来看待问题。“Delphi效应”似乎因为这个变体工作变得更加精确,在调试的情况下,这个变体同时减小了重复劳动。


所以加入更多的beta测试人员虽不能从开发人员的P.O.V中减小“最深”的错误的复杂度,但是它增加了这样一种可能性,即某个人的工具和问题正好匹配,而这个错误对这个人来说是浅显的。


Linus也做了一些改进,如果有一些严重的错误,Linux内核的版本在编号上做了些处理,让用户可以自己选择是运行上一个“稳定”的版本,还是冒遇到错误的险而得到新特征,这个战略还没被大多数Linux黑客所仿效,但它应该被仿效,存在两个选择的事实让二者都很吸引 人。

\chapter{How Many Eyeballs Tame Complexity}

It's one thing to observe in the large that the bazaar style greatly accelerates debugging and code evolution. It's another to understand exactly how and why it does so at the micro-level of day-to-day developer and tester behavior. In this section (written three years after the original paper, using insights by developers who read it and re-examined their own behavior) we'll take a hard look at the actual mechanisms. Non-technically inclined readers can safely skip to the next section.

One key to understanding is to realize exactly why it is that the kind of bug report non–source-aware users normally turn in tends not to be very useful. Non–source-aware users tend to report only surface symptoms; they take their environment for granted, so they (a) omit critical background data, and (b) seldom include a reliable recipe for reproducing the bug.

The underlying problem here is a mismatch between the tester's and the developer's mental models of the program; the tester, on the outside looking in, and the developer on the inside looking out. In closed-source development they're both stuck in these roles, and tend to talk past each other and find each other deeply frustrating.

Open-source development breaks this bind, making it far easier for tester and developer to develop a shared representation grounded in the actual source code and to communicate effectively about it. Practically, there is a huge difference in leverage for the developer between the kind of bug report that just reports externally-visible symptoms and the kind that hooks directly to the developer's source-code–based mental representation of the program.

Most bugs, most of the time, are easily nailed given even an incomplete but suggestive characterization of their error conditions at source-code level. When someone among your beta-testers can point out, "there's a boundary problem in line nnn", or even just "under conditions X, Y, and Z, this variable rolls over", a quick look at the offending code often suffices to pin down the exact mode of failure and generate a fix.

Thus, source-code awareness by both parties greatly enhances both good communication and the synergy between what a beta-tester reports and what the core developer(s) know. In turn, this means that the core developers' time tends to be well conserved, even with many collaborators.

Another characteristic of the open-source method that conserves developer time is the communication structure of typical open-source projects. Above I used the term "core developer"; this reflects a distinction between the project core (typically quite small; a single core developer is common, and one to three is typical) and the project halo of beta-testers and available contributors (which often numbers in the hundreds).

The fundamental problem that traditional software-development organization addresses is Brook's Law: ``Adding more programmers to a late project makes it later.'' More generally, Brooks's Law predicts that the complexity and communication costs of a project rise with the square of the number of developers, while work done only rises linearly.

Brooks's Law is founded on experience that bugs tend strongly to cluster at the interfaces between code written by different people, and that communications/coordination overhead on a project tends to rise with the number of interfaces between human beings. Thus, problems scale with the number of communications paths between developers, which scales as the square of the humber of developers (more precisely, according to the formula N*(N - 1)/2 where N is the number of developers).

The Brooks's Law analysis (and the resulting fear of large numbers in development groups) rests on a hidden assummption: that the communications structure of the project is necessarily a complete graph, that everybody talks to everybody else. But on open-source projects, the halo developers work on what are in effect separable parallel subtasks and interact with each other very little; code changes and bug reports stream through the core group, and only within that small core group do we pay the full Brooksian overhead. [SU]

There are are still more reasons that source-code–level bug reporting tends to be very efficient. They center around the fact that a single error can often have multiple possible symptoms, manifesting differently depending on details of the user's usage pattern and environment. Such errors tend to be exactly the sort of complex and subtle bugs (such as dynamic-memory-management errors or nondeterministic interrupt-window artifacts) that are hardest to reproduce at will or to pin down by static analysis, and which do the most to create long-term problems in software.

A tester who sends in a tentative source-code–level characterization of such a multi-symptom bug (e.g. "It looks to me like there's a window in the signal handling near line 1250" or "Where are you zeroing that buffer?") may give a developer, otherwise too close to the code to see it, the critical clue to a half-dozen disparate symptoms. In cases like this, it may be hard or even impossible to know which externally-visible misbehaviour was caused by precisely which bug—but with frequent releases, it's unnecessary to know. Other collaborators will be likely to find out quickly whether their bug has been fixed or not. In many cases, source-level bug reports will cause misbehaviours to drop out without ever having been attributed to any specific fix.

Complex multi-symptom errors also tend to have multiple trace paths from surface symptoms back to the actual bug. Which of the trace paths a given developer or tester can chase may depend on subtleties of that person's environment, and may well change in a not obviously deterministic way over time. In effect, each developer and tester samples a semi-random set of the program's state space when looking for the etiology of a symptom. The more subtle and complex the bug, the less likely that skill will be able to guarantee the relevance of that sample.

For simple and easily reproducible bugs, then, the accent will be on the "semi" rather than the "random"; debugging skill and intimacy with the code and its architecture will matter a lot. But for complex bugs, the accent will be on the "random". Under these circumstances many people running traces will be much more effective than a few people running traces sequentially—even if the few have a much higher average skill level.

This effect will be greatly amplified if the difficulty of following trace paths from different surface symptoms back to a bug varies significantly in a way that can't be predicted by looking at the symptoms. A single developer sampling those paths sequentially will be as likely to pick a difficult trace path on the first try as an easy one. On the other hand, suppose many people are trying trace paths in parallel while doing rapid releases. Then it is likely one of them will find the easiest path immediately, and nail the bug in a much shorter time. The project maintainer will see that, ship a new release, and the other people running traces on the same bug will be able to stop before having spent too much time on their more difficult traces [RJ].

\chapter{When Is a Rose Not a Rose?}

Having studied Linus's behavior and formed a theory about why it was successful, I made a conscious decision to test this theory on my new (admittedly much less complex and ambitious) project.

But the first thing I did was reorganize and simplify popclient a lot. Carl Harris's implementation was very sound, but exhibited a kind of unnecessary complexity common to many C programmers. He treated the code as central and the data structures as support for the code. As a result, the code was beautiful but the data structure design ad-hoc and rather ugly (at least by the high standards of this veteran LISP hacker).

I had another purpose for rewriting besides improving the code and the data structure design, however. That was to evolve it into something I understood completely. It's no fun to be responsible for fixing bugs in a program you don't understand.

For the first month or so, then, I was simply following out the implications of Carl's basic design. The first serious change I made was to add IMAP support. I did this by reorganizing the protocol machines into a generic driver and three method tables (for POP2, POP3, and IMAP). This and the previous changes illustrate a general principle that's good for programmers to keep in mind, especially in languages like C that don't naturally do dynamic typing:

9. Smart data structures and dumb code works a lot better than the other way around.

Brooks, Chapter 9: ``Show me your flowchart and conceal your tables, and I shall continue to be mystified. Show me your tables, and I won't usually need your flowchart; it'll be obvious.'' Allowing for thirty years of terminological/cultural shift, it's the same point.

At this point (early September 1996, about six weeks from zero) I started thinking that a name change might be in order—after all, it wasn't just a POP client any more. But I hesitated, because there was as yet nothing genuinely new in the design. My version of popclient had yet to develop an identity of its own.

That changed, radically, when popclient learned how to forward fetched mail to the SMTP port. I'll get to that in a moment. But first: I said earlier that I'd decided to use this project to test my theory about what Linus Torvalds had done right. How (you may well ask) did I do that? In these ways:

I released early and often (almost never less often than every ten days; during periods of intense development, once a day).

I grew my beta list by adding to it everyone who contacted me about fetchmail.

I sent chatty announcements to the beta list whenever I released, encouraging people to participate.

And I listened to my beta-testers, polling them about design decisions and stroking them whenever they sent in patches and feedback.

The payoff from these simple measures was immediate. From the beginning of the project, I got bug reports of a quality most developers would kill for, often with good fixes attached. I got thoughtful criticism, I got fan mail, I got intelligent feature suggestions. Which leads to:

10. If you treat your beta-testers as if they're your most valuable resource, they will respond by becoming your most valuable resource.

One interesting measure of fetchmail's success is the sheer size of the project beta list, fetchmail-friends. At the time of latest revision of this paper (November 2000) it has 287 members and is adding two or three a week.

Actually, when I revised in late May 1997 I found the list was beginning to lose members from its high of close to 300 for an interesting reason. Several people have asked me to unsubscribe them because fetchmail is working so well for them that they no longer need to see the list traffic! Perhaps this is part of the normal life-cycle of a mature bazaar-style project.

五. 什么时候玫瑰不是玫瑰?


在研究了Linus的行为和形成了为什么它成功的理论之后,我决定在我的工程(显然没有那么复杂和雄心勃勃)里有意识的测试这个理论。
但我首先做的事是熟悉和简化Popclient。 Carl Harris的实现非常好,但是有一种对许多C程序来说没有必要的复杂性。他把代码当作核心而把数据结构当作对代码的支持,结果是代码非常漂亮但是数据结构设计得很特别,相当丑陋(至少对以这个老LISP黑客的标准来看),然而除了提高代码和数据结构设计之外,重写它还有一个目的,就是要把它演化为我彻底理解的东西,对修改你不理解的程序中的错误负责可不是一件有趣的事。


第一个月我只是在领会Carl's的基本设计的含义,我所做的第一个重大修改是加入了IMAP支持,我把协议机重新组织为一个通用驱动程序和三个方法表(对应POP2、POP3和IMAP),这个前面的修改指出一个需要程序员(特别是象C这种没有自然的动态类型支持的语言)记在脑中的一般原理:


9. 聪明的数据结构和笨拙的代码要比相反的搭配工作的更好


Fred Brooks也在他第11章中讲道:“让我看你的[代码],把你的[数据结构]隐藏起来,我还是会迷惑;让我看看你的[数据结构],那我就不需要你的[代码]了,它是显而易见的”。


实际上,他说的是“流程图”和“表”,但是在三十年的术语/文化演进之后,事情还是一样的。


此时(1996年9月初,在从零开始六个月后),我开始想接下来修改名字——毕竟,它已不仅仅是一个POP客户,但我犹豫了,因为还没有什么新的漂亮设计呢,我的popclient版本需要有自己的特色。


当fetehmail学会怎样把取到的邮件转送到SMTP端口时,事情就完全改变了,但是首先:上面我说过我决定使用这个工程来测试我关于Linus Torualds所做的行为的理论,(你可能会问)我怎样做到这点呢? 以下面的方式:
1. 我尽早尽量频繁的发布(几乎从未少于每十天发布一次;在密集开发的时候是每天一次)。
2. 我把每一个和我讨论fetchmail的人加入一个beta表中。
3. 每当我发布我都向beta表中的人发出通告,鼓励人们参与。
4. 我听取beta测试员的意见,向他们询问设计决策,对他们寄来的补丁和反馈表示感谢。


这些简单的手段立即收到的回报,在工程的开始,我收到了一些错误报告,其质量足以使开发者因此被杀掉,而且经常还附有补丁、我得到了理智的批评,有趣的邮件,和聪明的特征建议,这导致了:


10. 如果你象对待最宝贵的资源一样对待你的beta测试员,他们就会成为你最宝贵的资源。

\chapter{Popclient becomes Fetchmail}

The real turning point in the project was when Harry Hochheiser sent me his scratch code for forwarding mail to the client machine's SMTP port. I realized almost immediately that a reliable implementation of this feature would make all the other mail delivery modes next to obsolete.

For many weeks I had been tweaking fetchmail rather incrementally while feeling like the interface design was serviceable but grubby—inelegant and with too many exiguous options hanging out all over. The options to dump fetched mail to a mailbox file or standard output particularly bothered me, but I couldn't figure out why.

(If you don't care about the technicalia of Internet mail, the next two paragraphs can be safely skipped.)

What I saw when I thought about SMTP forwarding was that popclient had been trying to do too many things. It had been designed to be both a mail transport agent (MTA) and a local delivery agent (MDA). With SMTP forwarding, it could get out of the MDA business and be a pure MTA, handing off mail to other programs for local delivery just as sendmail does.

Why mess with all the complexity of configuring a mail delivery agent or setting up lock-and-append on a mailbox when port 25 is almost guaranteed to be there on any platform with TCP/IP support in the first place? Especially when this means retrieved mail is guaranteed to look like normal sender-initiated SMTP mail, which is really what we want anyway.

(Back to a higher level....)

Even if you didn't follow the preceding technical jargon, there are several important lessons here. First, this SMTP-forwarding concept was the biggest single payoff I got from consciously trying to emulate Linus's methods. A user gave me this terrific idea—all I had to do was understand the implications.

11. The next best thing to having good ideas is recognizing good ideas from your users. Sometimes the latter is better.

Interestingly enough, you will quickly find that if you are completely and self-deprecatingly truthful about how much you owe other people, the world at large will treat you as though you did every bit of the invention yourself and are just being becomingly modest about your innate genius. We can all see how well this worked for Linus!

(When I gave my talk at the first Perl Conference in August 1997, hacker extraordinaire Larry Wall was in the front row. As I got to the last line above he called out, religious-revival style, ``Tell it, tell it, brother!''. The whole audience laughed, because they knew this had worked for the inventor of Perl, too.)

After a very few weeks of running the project in the same spirit, I began to get similar praise not just from my users but from other people to whom the word leaked out. I stashed away some of that email; I'll look at it again sometime if I ever start wondering whether my life has been worthwhile :-).

But there are two more fundamental, non-political lessons here that are general to all kinds of design.

12. Often, the most striking and innovative solutions come from realizing that your concept of the problem was wrong.

I had been trying to solve the wrong problem by continuing to develop popclient as a combined MTA/MDA with all kinds of funky local delivery modes. Fetchmail's design needed to be rethought from the ground up as a pure MTA, a part of the normal SMTP-speaking Internet mail path.

When you hit a wall in development—when you find yourself hard put to think past the next patch—it's often time to ask not whether you've got the right answer, but whether you're asking the right question. Perhaps the problem needs to be reframed.

Well, I had reframed my problem. Clearly, the right thing to do was (1) hack SMTP forwarding support into the generic driver, (2) make it the default mode, and (3) eventually throw out all the other delivery modes, especially the deliver-to-file and deliver-to-standard-output options.

I hesitated over step 3 for some time, fearing to upset long-time popclient users dependent on the alternate delivery mechanisms. In theory, they could immediately switch to .forward files or their non-sendmail equivalents to get the same effects. In practice the transition might have been messy.

But when I did it, the benefits proved huge. The cruftiest parts of the driver code vanished. Configuration got radically simpler—no more grovelling around for the system MDA and user's mailbox, no more worries about whether the underlying OS supports file locking.

Also, the only way to lose mail vanished. If you specified delivery to a file and the disk got full, your mail got lost. This can't happen with SMTP forwarding because your SMTP listener won't return OK unless the message can be delivered or at least spooled for later delivery.

Also, performance improved (though not so you'd notice it in a single run). Another not insignificant benefit of this change was that the manual page got a lot simpler.

Later, I had to bring delivery via a user-specified local MDA back in order to allow handling of some obscure situations involving dynamic SLIP. But I found a much simpler way to do it.

The moral? Don't hesitate to throw away superannuated features when you can do it without loss of effectiveness. Antoine de Saint-Exupéry (who was an aviator and aircraft designer when he wasn't authoring classic children's books) said:

13. ``Perfection (in design) is achieved not when there is nothing more to add, but rather when there is nothing more to take away.''

When your code is getting both better and simpler, that is when you know it's right. And in the process, the fetchmail design acquired an identity of its own, different from the ancestral popclient.

It was time for the name change. The new design looked much more like a dual of sendmail than the old popclient had; both are MTAs, but where sendmail pushes then delivers, the new popclient pulls then delivers. So, two months off the blocks, I renamed it fetchmail.

There is a more general lesson in this story about how SMTP delivery came to fetchmail. It is not only debugging that is parallelizable; development and (to a perhaps surprising extent) exploration of design space is, too. When your development mode is rapidly iterative, development and enhancement may become special cases of debugging—fixing `bugs of omission' in the original capabilities or concept of the software.

Even at a higher level of design, it can be very valuable to have lots of co-developers random-walking through the design space near your product. Consider the way a puddle of water finds a drain, or better yet how ants find food: exploration essentially by diffusion, followed by exploitation mediated by a scalable communication mechanism. This works very well; as with Harry Hochheiser and me, one of your outriders may well find a huge win nearby that you were just a little too close-focused to see.

六. popclient变成了Fetchmail


这个工程的真正转折点是Harry Hochleiser寄给我他写的代码草稿,他把邮件转发到客户端机器的SMTP端口,我立即意识到这个特征的可靠实现将淘汰所有其他的递送模式。


几个星期以来我一直在修改而不是改进fetchmail,因为我觉得界面设计虽然有用但是太笨拙琐碎了,到处充满了太多的粗陋的细小选项。


当我思考SMTP转发时我发现popclient试图做的事太多了,它被设计成既是一个邮件传输代理(MTA)也是一个本地递送代理(MDA)。使用SMTP转发,它就可以从MDA的事务中解脱出来而成为一个纯MTA,而象sendmail一样把邮件交给本地递送程序来处理。


既然端口25在所有支撑TCP/IP的平台上早已被预留,为什么还要为一个邮件传输代理的配置或为一个邮箱设置加锁的附加功能而操心呢?尤其是当这意味着抽取的邮件就象一个正常的发送者发出的SMTP邮件一样,而这就是我们需要的。


这里有几个教益:第一,SMTP转发的想法是我有意识地模拟Linus的方法以来的最大的单个回报,一个用户告诉我这个非同寻常的想法——我所需做的只是理解它的含义。


11. 想出好主意是好事,从你的用户那里发现好主意也是好事,有时候后者更好。


很有趣的是,你很快将发现,如果你完全承认你从其他人那里得到多少教益的话,整个世界将会认为所有的发明都是你做出的,而你会对你的天才变得谦虚。我们可以看到这在Linus身上体现得多明显!(当我在1997年8月的Perl会议上发表这个论文时,Larry Wall坐在前排,当我讲到上面的观点时,他激动的叫了出来:“对了!说对了!哥们!”所有的听众都哄堂大笑起来,因为他们知道同样的事情也发生在Perl的发明者身上)。


于是在同样精神指导下工程进行了几个星期,我开始不光从我的用户那儿也从听说我的系统的人那儿得到类似的赞扬,我把一些这种邮件收藏起来,我将在我开始怀疑自己的生命是否有价值时重新读读这些信。:)


但是有两个更基本的,非政治性的对所有设计都有普遍意义的教益。


12. 最重要和最有创新的解决方案常常来自于你认识到你对问题的概念是错误的。

一个衡量fetchmail成功的有趣方式是工程的beta测试人员表(fegtchmail的朋友们)的长度,在创立它的时候已经有249个成员了,而且每个星期增加两到三个。


实际上,当我在1997年5月校订它时,这张表开始因为一个有趣的原因而缩短了,有几个人请求我把他们从表中去掉,因为fetchmail已经工作的如此之好,他们不需要看到这些邮件了!也许这是一个成熟的市集风格工程的生命周期的一部分。

我以前一直在解决错误的问题,把popclient当作MTA和具有许多本地递送模式的MDA的结合物,Fetchmail的设计需要从头考虑为一个纯的MTA,做为一个普通Internet邮件路径的一部分。


当你在开发中碰了壁时(当你发现自己很难想通下一步时),那通常不是要问自己是否找到正确答案,而是要问是否问了正确问题,也许需要重新构造问题。


于是,我重新构造了我的问题,很清楚,要做的正确的事是(1)把SMTP转发支持放在通用驱动程序中,(2)把它做为缺省模式,(3)最终分离所有其他的递送模式,尤其是递送到文件和标准输出的选项。


我在第三步上犹豫了一下,担心会让popdiant的长期用户对新的递送方法感到烦心,在理论上,他们可以立即转而转发文件或者他们的非sendmail等价物来得到同样的效果,在实际中这种转换可能会很麻烦。
但是当我这么做之后,证明好处是巨大的,驱动程序代码的冗余的部分消失了,配置完全变得简单了——不用屈从于系统MDA和用户的邮箱,也不用为下层OS是否支持文件锁定而担心了。


而且,丢失邮件的唯一漏洞也被堵死了,如果你选择了递送到一个文件而磁盘已满,你的邮件就会丢失,这在SMTP转发中不会发生,因为SMTP侦听器不会返回OK的,除非邮件可以递送成功或至少被缓冲留待以后递送。


还有,性能也改善了(虽然在单次执行中你不会注意到),这个修改的另一个不可忽视的好处是手册变得大大简单了。


后来,为了允许处理一些罕见的情况,包括动态SLIP,我必须回到让用户定义本地MDA递送上来,但是我发现了一个更加简单的方法。


所有这些给了我们什么启发呢?如果可以不损失效率,就要毫不犹豫抛弃陈旧的特性,Antonine de Saint-Exupery(在他成为经典儿童书籍作家之前是一个飞行员和飞机设计师)曾说过:


13. “最好的设计不是再也没有什么东西可以添加了,而是再也没有什么东西可以去掉。”


当你的代码变得更好和更简单时,这就是你知道它是正确的时候了,而且在这个过程中,fetehmail的设计具有了自己的特点,而区别于其前身popclient。


现在是改名的时候了,这个新的设计看起来比老popclient更象一个sendmail的复制品,它们都是MTA,但是Senmail是推然后递送,而新的popclient是拉然后递送。于是,在两个月之后,我把它重新命名为fetehmail。

\chapter{Fetchmail Grows Up}

There I was with a neat and innovative design, code that I knew worked well because I used it every day, and a burgeoning beta list. It gradually dawned on me that I was no longer engaged in a trivial personal hack that might happen to be useful to few other people. I had my hands on a program that every hacker with a Unix box and a SLIP/PPP mail connection really needs.

With the SMTP forwarding feature, it pulled far enough in front of the competition to potentially become a ``category killer'', one of those classic programs that fills its niche so competently that the alternatives are not just discarded but almost forgotten.

I think you can't really aim or plan for a result like this. You have to get pulled into it by design ideas so powerful that afterward the results just seem inevitable, natural, even foreordained. The only way to try for ideas like that is by having lots of ideas—or by having the engineering judgment to take other peoples' good ideas beyond where the originators thought they could go.

Andy Tanenbaum had the original idea to build a simple native Unix for IBM PCs, for use as a teaching tool (he called it Minix). Linus Torvalds pushed the Minix concept further than Andrew probably thought it could go—and it grew into something wonderful. In the same way (though on a smaller scale), I took some ideas by Carl Harris and Harry Hochheiser and pushed them hard. Neither of us was `original' in the romantic way people think is genius. But then, most science and engineering and software development isn't done by original genius, hacker mythology to the contrary.

The results were pretty heady stuff all the same—in fact, just the kind of success every hacker lives for! And they meant I would have to set my standards even higher. To make fetchmail as good as I now saw it could be, I'd have to write not just for my own needs, but also include and support features necessary to others but outside my orbit. And do that while keeping the program simple and robust.

The first and overwhelmingly most important feature I wrote after realizing this was multidrop support—the ability to fetch mail from mailboxes that had accumulated all mail for a group of users, and then route each piece of mail to its individual recipients.

I decided to add the multidrop support partly because some users were clamoring for it, but mostly because I thought it would shake bugs out of the single-drop code by forcing me to deal with addressing in full generality. And so it proved. Getting RFC 822 address parsing right took me a remarkably long time, not because any individual piece of it is hard but because it involved a pile of interdependent and fussy details.

But multidrop addressing turned out to be an excellent design decision as well. Here's how I knew:

14. Any tool should be useful in the expected way, but a truly great tool lends itself to uses you never expected.

The unexpected use for multidrop fetchmail is to run mailing lists with the list kept, and alias expansion done, on the client side of the Internet connection. This means someone running a personal machine through an ISP account can manage a mailing list without continuing access to the ISP's alias files.

Another important change demanded by my beta-testers was support for 8-bit MIME (Multipurpose Internet Mail Extensions) operation. This was pretty easy to do, because I had been careful to keep the code 8-bit clean (that is, to not press the 8th bit, unused in the ASCII character set, into service to carry information within the program). Not because I anticipated the demand for this feature, but rather in obedience to another rule:

15. When writing gateway software of any kind, take pains to disturb the data stream as little as possible—and never throw away information unless the recipient forces you to!

Had I not obeyed this rule, 8-bit MIME support would have been difficult and buggy. As it was, all I had to do is read the MIME standard (RFC 1652) and add a trivial bit of header-generation logic.

Some European users bugged me into adding an option to limit the number of messages retrieved per session (so they can control costs from their expensive phone networks). I resisted this for a long time, and I'm still not entirely happy about it. But if you're writing for the world, you have to listen to your customers—this doesn't change just because they're not paying you in money.

七. Fetchmail成长起来


现在我有了一个简洁和富有创意的设计,工作得很好的代码,因为我每天都用它,和一直在增长的beta表,它让我渐渐明白我已经不是在从事只能对少数其他人有用的工作中,我写了一个所有有一个Unix邮箱和SLIP/PPP邮件连接的人都真正需要的程序。


通过SMTP转发功能,它成为一个潜在的“目录杀手”,远远领先于它的竞争者,这个程序如此能干以至于其他的程序不但被放弃简直被忘记了。


我知道你不可以真得瞄准或计划出这样的结果,你只能努力去设计这些强大的思想,以后这些结果就好象是不可避免的、自然的、注定了的,得到这种思想的唯一办法是获取许多思想,或者用工程化的思考其他人的好主意而超过原来想到它的人的设想。


Andrew Tanenbanm原来设想建造一个适合386的简单的Unix用做教学,Linus Torvalels把Andrew的可能想到的Minix可以做什么的概念推进了一步,成长为一个极好的东西,同样的(虽然规模较小),我接受了Card Harris和Harry Hochheiser的想法,把它们变得更强大,我们都不是人们所浪漫幻想的天才的创始人,但是大多数科学和工程和软件开发不是被天才的创始人完成的,这和流传的神话恰恰相反。


结果总是执着的原因——实际上,它是每个黑客为之生存的成功!而且它们意味着我必须把自己的标准定高一点,为了把fetchmail变得和我所能设想的那样好,我必须不仅为我自己的需要写代码,而且也要包括对在我生活围主页外的人们的需求的支持,而且同时也要保证程序的简单和健壮。


在实现它之后我首先写的最重要的特征是支持多投——从集中一组用户的邮件的邮箱中取出邮件,然后把它路由到每个人手中。


我之所以加上多投功能部分是因为有些用户一直在闹着要它,更是因为我想它可以从单投的代码中揭露出错误来,让我完全一般地处理寻址,而且这被证明了。正确解释RFC822花了我相当长的时间,不仅因为它的每个单独部分都很难,而且因为它有一大堆相互依赖的苛刻的细节。


但是多投寻址也成为一个极好的设计决策,由此我知道:


14. 任何工具都应该能以预想的方式使用,但是一个伟大的工具提供你没料到的功能。


Fetchmant多投功能的一个没有料到的用途是在SLIP/PPP的客户端提供邮件列表、别名扩展。这意味着一个使用个人机器的人不必持续访问ISP的别名文件就能通过一个ISP帐户管理一个邮件列表。我的beta测试员提出的另一个重要的改变是支持8位MIME操作,这很容易做,因为我已经仔细的保证了8位代码的清晰,不仅因为我预见到了这个特性的需求,而且因为我忠实于另一准则:


15. 当写任何种类的网关型程序时,多费点力,尽量少干扰数据流,永远不要抛弃信息,除非接收方强迫这么作!


如果我不遵从这个准则,那么8位MIME支持将会变得困难和笨拙,现在我所需要做的,是只读一下RFC 1652,在产生信头的逻辑加上一点而已。


一些欧洲用户要求我加上一个选项来限制每次会话取得消息数(这样他们就可以从昂贵的电话网中控制花费了),我很长一段时间拒绝这样做,而且我仍然对它不很高兴,但是如果你是为了世界而写代码,你必须听取顾客的意见——这并不随他们不付给你钱而改变。

\chapter{A Few More Lessons from Fetchmail}

Before we go back to general software-engineering issues, there are a couple more specific lessons from the fetchmail experience to ponder. Nontechnical readers can safely skip this section.

The rc (control) file syntax includes optional `noise' keywords that are entirely ignored by the parser. The English-like syntax they allow is considerably more readable than the traditional terse keyword-value pairs you get when you strip them all out.

These started out as a late-night experiment when I noticed how much the rc file declarations were beginning to resemble an imperative minilanguage. (This is also why I changed the original popclient ``server'' keyword to ``poll'').

It seemed to me that trying to make that imperative minilanguage more like English might make it easier to use. Now, although I'm a convinced partisan of the ``make it a language'' school of design as exemplified by Emacs and HTML and many database engines, I am not normally a big fan of ``English-like'' syntaxes.

Traditionally programmers have tended to favor control syntaxes that are very precise and compact and have no redundancy at all. This is a cultural legacy from when computing resources were expensive, so parsing stages had to be as cheap and simple as possible. English, with about 50% redundancy, looked like a very inappropriate model then.

This is not my reason for normally avoiding English-like syntaxes; I mention it here only to demolish it. With cheap cycles and core, terseness should not be an end in itself. Nowadays it's more important for a language to be convenient for humans than to be cheap for the computer.

There remain, however, good reasons to be wary. One is the complexity cost of the parsing stage—you don't want to raise that to the point where it's a significant source of bugs and user confusion in itself. Another is that trying to make a language syntax English-like often demands that the ``English'' it speaks be bent seriously out of shape, so much so that the superficial resemblance to natural language is as confusing as a traditional syntax would have been. (You see this bad effect in a lot of so-called ``fourth generation'' and commercial database-query languages.)

The fetchmail control syntax seems to avoid these problems because the language domain is extremely restricted. It's nowhere near a general-purpose language; the things it says simply are not very complicated, so there's little potential for confusion in moving mentally between a tiny subset of English and the actual control language. I think there may be a broader lesson here:

16. When your language is nowhere near Turing-complete, syntactic sugar can be your friend.

Another lesson is about security by obscurity. Some fetchmail users asked me to change the software to store passwords encrypted in the rc file, so snoopers wouldn't be able to casually see them.

I didn't do it, because this doesn't actually add protection. Anyone who's acquired permissions to read your rc file will be able to run fetchmail as you anyway—and if it's your password they're after, they'd be able to rip the necessary decoder out of the fetchmail code itself to get it.

All .fetchmailrc password encryption would have done is give a false sense of security to people who don't think very hard. The general rule here is:

17. A security system is only as secure as its secret. Beware of pseudo-secrets.

八. 从Fetchmail得来的另一些教益


在他们回到一般的软件工程问题以前,还有几个从fetchmail得到的教益需要思考。


rc文件语法包括可选的“noise”关键字,它被扫描器完全忽略了,当你把它们全抽取出的时候,关键字/值对更具可读性。


当我注意到rc文件的声明在多大程度上开始象一个微型命令语言时,这是一个Late-night的体验(这也是我为什么把popclient原来的“server”关键字改成了“poll”)。


对我来说似乎把这个微型命令语言变得更象英语可能会使它更容易使用。现在,虽然我对经过Emacs和HTML及许多数据库引擎所证实的“把它做成一个语言”的设计方式确信不疑,但是我并不是一个通常的“类英语”语法的狂热拥护者。


传统程序员容易控制语法使它尽量精确和紧凑,完全没有冗余,这是计算机资源还很昂贵时遗留下的一种文化传统,所以扫描策略需要尽可能的廉价和简单,而具有50%冗余度的英语,看来好象是一个非常不合适的模型。


这并不是我不用类英语语法的原因,我提到这一点是为了推翻它,在更廉价的时钟周期与核心的时代,简洁并没有走到尽头,今天对一个语言来说,对人更方便比对机器更廉价来的更加重要。


然而,有几个原因提醒我们小心一点,一个是扫描策略的复杂度开销——你并不想把它变成一个巨大的错误来源和让用户困惑,另一个是试图使语言表面上的类似可以和传统语言一样令人困惑(你可以在许多4GL和商业数据库查询语言上看到这一点)。


Fetchmail的控制语法避免了这些问题,因为语言的领域是极其有限的。它一点也不象一个一般性的语言,它很简单地描述的东西并不复杂,所以很少可能在英语的一个小子集与实际的控制语言之间发生混淆,我想这有一个更广泛的教益:


16. 如果你的语言一点也不象是图灵完备的,严格的语法会有好处。


另一个教益是关于安全的,一些fetchmail用户要求我修改软件把口令加密存贮在rc文件里,这样觑探者就不能看到它们了。


我没有这样做,因为这实际上起不到任何保护作用,任何有权读取你的rc文件的人都可以以你的名义运行fetchmail——如果他们要破你的口令,它们可以从fetchmail的代码中找到制作解码器的方法。


所以fetchmail口令的加密都会给那些不慎重思考的人一种安全的错觉,这里一般性的准则是:


17. 一个安全系统只能和它的秘密一样安全,当心伪安全。

\chapter{Necessary Preconditions for the Bazaar Style}

Early reviewers and test audiences for this essay consistently raised questions about the preconditions for successful bazaar-style development, including both the qualifications of the project leader and the state of code at the time one goes public and starts to try to build a co-developer community.

It's fairly clear that one cannot code from the ground up in bazaar style [IN]. One can test, debug and improve in bazaar style, but it would be very hard to originate a project in bazaar mode. Linus didn't try it. I didn't either. Your nascent developer community needs to have something runnable and testable to play with.

When you start community-building, what you need to be able to present is a plausible promise. Your program doesn't have to work particularly well. It can be crude, buggy, incomplete, and poorly documented. What it must not fail to do is (a) run, and (b) convince potential co-developers that it can be evolved into something really neat in the foreseeable future.

Linux and fetchmail both went public with strong, attractive basic designs. Many people thinking about the bazaar model as I have presented it have correctly considered this critical, then jumped from that to the conclusion that a high degree of design intuition and cleverness in the project leader is indispensable.

But Linus got his design from Unix. I got mine initially from the ancestral popclient (though it would later change a great deal, much more proportionately speaking than has Linux). So does the leader/coordinator for a bazaar-style effort really have to have exceptional design talent, or can he get by through leveraging the design talent of others?

I think it is not critical that the coordinator be able to originate designs of exceptional brilliance, but it is absolutely critical that the coordinator be able to recognize good design ideas from others.

Both the Linux and fetchmail projects show evidence of this. Linus, while not (as previously discussed) a spectacularly original designer, has displayed a powerful knack for recognizing good design and integrating it into the Linux kernel. And I have already described how the single most powerful design idea in fetchmail (SMTP forwarding) came from somebody else.

Early audiences of this essay complimented me by suggesting that I am prone to undervalue design originality in bazaar projects because I have a lot of it myself, and therefore take it for granted. There may be some truth to this; design (as opposed to coding or debugging) is certainly my strongest skill.

But the problem with being clever and original in software design is that it gets to be a habit—you start reflexively making things cute and complicated when you should be keeping them robust and simple. I have had projects crash on me because I made this mistake, but I managed to avoid this with fetchmail.

So I believe the fetchmail project succeeded partly because I restrained my tendency to be clever; this argues (at least) against design originality being essential for successful bazaar projects. And consider Linux. Suppose Linus Torvalds had been trying to pull off fundamental innovations in operating system design during the development; does it seem at all likely that the resulting kernel would be as stable and successful as what we have?

A certain base level of design and coding skill is required, of course, but I expect almost anybody seriously thinking of launching a bazaar effort will already be above that minimum. The open-source community's internal market in reputation exerts subtle pressure on people not to launch development efforts they're not competent to follow through on. So far this seems to have worked pretty well.

There is another kind of skill not normally associated with software development which I think is as important as design cleverness to bazaar projects—and it may be more important. A bazaar project coordinator or leader must have good people and communications skills.

This should be obvious. In order to build a development community, you need to attract people, interest them in what you're doing, and keep them happy about the amount of work they're doing. Technical sizzle will go a long way towards accomplishing this, but it's far from the whole story. The personality you project matters, too.

It is not a coincidence that Linus is a nice guy who makes people like him and want to help him. It's not a coincidence that I'm an energetic extrovert who enjoys working a crowd and has some of the delivery and instincts of a stand-up comic. To make the bazaar model work, it helps enormously if you have at least a little skill at charming people.

九. 集市风格的必要的先决条件


本文的早期评审人员和测试人员坚持提出成功的市集模式开发的先决条件,包括工程领导人的资格问题和在把项目公开和开始建造一个协作开发人员的社团的时候代码的状态。


相当清楚,不能以一个市集模式从头开发一个软件,我们可以以市集模式、测试、调试和改进,但是以市集模式从头开始一个项目将是非常困难的,Linus没有这样做,我也没有,初期的开发人员的社团应该有一此可以运行和测试的东西来玩。


当你开始创建社团时,你需要演示的是一个诺言,你的程序不需要工作的很好,它可以很粗糙、很笨拙、不完整和缺少文档、它不能忽略的东西是要吸引哪些人卷入一个整洁的项目。


Linux和fetchmail都是以一个吸引人的基本设计进入公共领域的,许多和我一样在思考市集模式的人已经正确的认为这是非常关键的,然后得出了一个结论,工程领导者的高度的设计直觉和聪颖是必不可少的。


但是Linus是从Unix得到他的设计的,我最初是从先前的popmail得到启发的(虽然相对Linux而言,它最后改变巨大),所以市集风格的领导人/协调人需要有出众的设计才能,或者他可以利用别人的设计才能?


我认为能够提出卓越的原始设计思想对协调人来说不是最关键的,但是对他/她来说绝对关键的是要能把从他人那里得到的好的设计重新组织起来。


Linux和fetchmail项目都显示了这些证据,Linus(如同前面所说)并不是惊人的原始设计者,但他显示了发现好的设计并把它集成到Linux内核中的强大决窍。还有我也描述了怎样从别人那里得到了fetchmail中最强大的设计思想(SMTP转发)。


本文的早期读者称赞我,说因为我做了许多关于原始设计的事,所以倾向于低估原始设计在市集项目中的价值,也许有些是对的吧,但是设计(而不是编码或调试)本来就是我最强的能力。


变得聪明和软件设计的原始创作的问题是它会变成一个习惯,当需要保持事物健壮和简洁的时候,你却开始把事情变得漂亮但却复杂。我曾经犯过错误,使得一些项目因我而崩溃了,但我努力不让它发生在fetchmail身上。


所以我相信fetchmail项目的成功部分是因为我抑制自己不要变得太聪明,这说明(至少)对市集模式而言原始设计并不是本质的,请考察一下Linux假设Linus Torvalds在开发时试图彻底革新操作系统设计,它还会象今天我们所拥有的内核那样稳定和成功吗?


当然基本的设计和编码技巧还是必需的,但我希望每个严肃考虑发起一个市集计划的人都已至少具备这些能力,自由软件社团的内部市场对人们有某些微妙的压力,让他们不要发起自由不能搞定的开发,目前为止,这工作得仍然相当好。


对市集项目来说,我认为还有另一种通常与软件开发无关的技能和设计能力同样重要——或者更加重要,市集项目的协调人或领导人必须有良好的人际和交流能力。


这是很显然的,为了建造一个开发社团,你需要吸引人,你所做的东西要让他们感到有趣,而且要保持他们对他们正在做的工作感到有趣,而且要保持他们对他们正在做的工作感到高兴,技术方面对达成这些目标有一定帮助,但这远远不是全部,你的个人素质也有关系。


并不是说Linus是一个好小伙子,让人们喜爱并乐于帮助他,也并不是说我是个积极外向的,喜欢扎堆儿工作,有出众的幽默感的人,对市集模式的工作而言,至少有一点吸引人的技巧是非常有帮助的。

\chapter{The Social Context of Open-Source Software}

It is truly written: the best hacks start out as personal solutions to the author's everyday problems, and spread because the problem turns out to be typical for a large class of users. This takes us back to the matter of rule 1, restated in a perhaps more useful way:

18. To solve an interesting problem, start by finding a problem that is interesting to you.

So it was with Carl Harris and the ancestral popclient, and so with me and fetchmail. But this has been understood for a long time. The interesting point, the point that the histories of Linux and fetchmail seem to demand we focus on, is the next stage—the evolution of software in the presence of a large and active community of users and co-developers.

In The Mythical Man-Month, Fred Brooks observed that programmer time is not fungible; adding developers to a late software project makes it later. As we've seen previously, he argued that the complexity and communication costs of a project rise with the square of the number of developers, while work done only rises linearly. Brooks's Law has been widely regarded as a truism. But we've examined in this essay an number of ways in which the process of open-source development falsifies the assumptionms behind it—and, empirically, if Brooks's Law were the whole picture Linux would be impossible.

Gerald Weinberg's classic The Psychology of Computer Programming supplied what, in hindsight, we can see as a vital correction to Brooks. In his discussion of ``egoless programming'', Weinberg observed that in shops where developers are not territorial about their code, and encourage other people to look for bugs and potential improvements in it, improvement happens dramatically faster than elsewhere. (Recently, Kent Beck's `extreme programming' technique of deploying coders in pairs looking over one anothers' shoulders might be seen as an attempt to force this effect.)

Weinberg's choice of terminology has perhaps prevented his analysis from gaining the acceptance it deserved—one has to smile at the thought of describing Internet hackers as ``egoless''. But I think his argument looks more compelling today than ever.

The bazaar method, by harnessing the full power of the ``egoless programming'' effect, strongly mitigates the effect of Brooks's Law. The principle behind Brooks's Law is not repealed, but given a large developer population and cheap communications its effects can be swamped by competing nonlinearities that are not otherwise visible. This resembles the relationship between Newtonian and Einsteinian physics—the older system is still valid at low energies, but if you push mass and velocity high enough you get surprises like nuclear explosions or Linux.

The history of Unix should have prepared us for what we're learning from Linux (and what I've verified experimentally on a smaller scale by deliberately copying Linus's methods [EGCS]). That is, while coding remains an essentially solitary activity, the really great hacks come from harnessing the attention and brainpower of entire communities. The developer who uses only his or her own brain in a closed project is going to fall behind the developer who knows how to create an open, evolutionary context in which feedback exploring the design space, code contributions, bug-spotting, and other improvements come from from hundreds (perhaps thousands) of people.

But the traditional Unix world was prevented from pushing this approach to the ultimate by several factors. One was the legal contraints of various licenses, trade secrets, and commercial interests. Another (in hindsight) was that the Internet wasn't yet good enough.

Before cheap Internet, there were some geographically compact communities where the culture encouraged Weinberg's ``egoless'' programming, and a developer could easily attract a lot of skilled kibitzers and co-developers. Bell Labs, the MIT AI and LCS labs, UC Berkeley—these became the home of innovations that are legendary and still potent.

Linux was the first project for which a conscious and successful effort to use the entire world as its talent pool was made. I don't think it's a coincidence that the gestation period of Linux coincided with the birth of the World Wide Web, and that Linux left its infancy during the same period in 1993–1994 that saw the takeoff of the ISP industry and the explosion of mainstream interest in the Internet. Linus was the first person who learned how to play by the new rules that pervasive Internet access made possible.

While cheap Internet was a necessary condition for the Linux model to evolve, I think it was not by itself a sufficient condition. Another vital factor was the development of a leadership style and set of cooperative customs that could allow developers to attract co-developers and get maximum leverage out of the medium.

But what is this leadership style and what are these customs? They cannot be based on power relationships—and even if they could be, leadership by coercion would not produce the results we see. Weinberg quotes the autobiography of the 19th-century Russian anarchist Pyotr Alexeyvich Kropotkin's Memoirs of a Revolutionist to good effect on this subject:

Having been brought up in a serf-owner's family, I entered active life, like all young men of my time, with a great deal of confidence in the necessity of commanding, ordering, scolding, punishing and the like. But when, at an early stage, I had to manage serious enterprises and to deal with [free] men, and when each mistake would lead at once to heavy consequences, I began to appreciate the difference between acting on the principle of command and discipline and acting on the principle of common understanding. The former works admirably in a military parade, but it is worth nothing where real life is concerned, and the aim can be achieved only through the severe effort of many converging wills.

The ``severe effort of many converging wills'' is precisely what a project like Linux requires—and the ``principle of command'' is effectively impossible to apply among volunteers in the anarchist's paradise we call the Internet. To operate and compete effectively, hackers who want to lead collaborative projects have to learn how to recruit and energize effective communities of interest in the mode vaguely suggested by Kropotkin's ``principle of understanding''. They must learn to use Linus's Law.[SP]

Earlier I referred to the ``Delphi effect'' as a possible explanation for Linus's Law. But more powerful analogies to adaptive systems in biology and economics also irresistably suggest themselves. The Linux world behaves in many respects like a free market or an ecology, a collection of selfish agents attempting to maximize utility which in the process produces a self-correcting spontaneous order more elaborate and efficient than any amount of central planning could have achieved. Here, then, is the place to seek the ``principle of understanding''.

The ``utility function'' Linux hackers are maximizing is not classically economic, but is the intangible of their own ego satisfaction and reputation among other hackers. (One may call their motivation ``altruistic'', but this ignores the fact that altruism is itself a form of ego satisfaction for the altruist). Voluntary cultures that work this way are not actually uncommon; one other in which I have long participated is science fiction fandom, which unlike hackerdom has long explicitly recognized ``egoboo'' (ego-boosting, or the enhancement of one's reputation among other fans) as the basic drive behind volunteer activity.

Linus, by successfully positioning himself as the gatekeeper of a project in which the development is mostly done by others, and nurturing interest in the project until it became self-sustaining, has shown an acute grasp of Kropotkin's ``principle of shared understanding''. This quasi-economic view of the Linux world enables us to see how that understanding is applied.

We may view Linus's method as a way to create an efficient market in ``egoboo''—to connect the selfishness of individual hackers as firmly as possible to difficult ends that can only be achieved by sustained cooperation. With the fetchmail project I have shown (albeit on a smaller scale) that his methods can be duplicated with good results. Perhaps I have even done it a bit more consciously and systematically than he.

Many people (especially those who politically distrust free markets) would expect a culture of self-directed egoists to be fragmented, territorial, wasteful, secretive, and hostile. But this expectation is clearly falsified by (to give just one example) the stunning variety, quality, and depth of Linux documentation. It is a hallowed given that programmers hate documenting; how is it, then, that Linux hackers generate so much documentation? Evidently Linux's free market in egoboo works better to produce virtuous, other-directed behavior than the massively-funded documentation shops of commercial software producers.

Both the fetchmail and Linux kernel projects show that by properly rewarding the egos of many other hackers, a strong developer/coordinator can use the Internet to capture the benefits of having lots of co-developers without having a project collapse into a chaotic mess. So to Brooks's Law I counter-propose the following:

19: Provided the development coordinator has a communications medium at least as good as the Internet, and knows how to lead without coercion, many heads are inevitably better than one.

I think the future of open-source software will increasingly belong to people who know how to play Linus's game, people who leave behind the cathedral and embrace the bazaar. This is not to say that individual vision and brilliance will no longer matter; rather, I think that the cutting edge of open-source software will belong to people who start from individual vision and brilliance, then amplify it through the effective construction of voluntary communities of interest.

Perhaps this is not only the future of open-source software. No closed-source developer can match the pool of talent the Linux community can bring to bear on a problem. Very few could afford even to hire the more than 200 (1999: 600, 2000: 800) people who have contributed to fetchmail!

Perhaps in the end the open-source culture will triumph not because cooperation is morally right or software ``hoarding'' is morally wrong (assuming you believe the latter, which neither Linus nor I do), but simply because the closed-source world cannot win an evolutionary arms race with open-source communities that can put orders of magnitude more skilled time into a problem.

十. 自由软件的社会学语境


下述如实:最好的开发是从作者解决每天工作中的个人问题开始的,因为它对一大类用户来说是一个典型问题,所以它就推广开来了,这把我们带回到准则1,也许是用一个更有用的方式来描述:


18. 要解决一个有趣的问题,请从发现让你感兴趣的问题开始。


这是Carl Harris和原先的popclient的情形,也是我和fetchmail的情形,但这已在很长一段时间被大家知晓了,Linux和fetchmail的历史要求我们注意的有趣之处是下一个阶段——软件在一个庞大的活跃的用户和协作开发人员的社团中的进化。


在《神秘的人月》一书中,Fred Brooks观察到程序员的工作时间是不可替代的:在一个误了工期的软件项目中增加开发人员只会让它拖得更久,他声称项目的复杂度和通讯开销以开发人员的平方增长,而工作成绩只是以线性增长,这个说法被称为“Brooks定律”,被普遍当作真理,但如果Brooks定律就是全部,那Linux就不可能成功。


几年之后,Gerald Weinbeng的经典之作“The Psychology Of Computer Progromming”为我们更正了Brooks的看法,在他的“忘我(egoless)的编程”中,Weinberg观察到在开发人员不顽固保守自己的代码,鼓励其他人寻找错误和发展潜力的地方,软件的改进的速度会比其他地方有戏剧性的提高。


Weinberg的用词可阻止了他的分析得到应有的接受,人们对把Internet黑客称为“忘我”的想法微笑,但是我想今天他的想法比以往任何时候都要引人注目。


Unix的历史已经为我们准备好了我们正在从Linux学到的(和我在更小规模上模仿Linus的方法所验证的)东西,这就是,虽然编码仍是一个人干的活,真正伟大的工作来自于利用整个社团的注意和脑力,在一个封闭的项目中只利用他自己的脑力的人会落在知道怎样创建一个开放的、进化的,成百上千的人在其中查找错误和进行修改的环境的开发人员之后。


但是Unix的传统中有几个因素阻止把这种方法推到极致。一个是各种授权的法律约束、商业机密和商业利益,另一个(事后来看)是Internet还不够好。


在Internet变得便宜之前,有一些在地理上紧密的社团,它们的文化鼓励Weingberg的“忘我”编程,一个开发人员很容易吸引许多熟练的人和协作开发人员,贝尔实验室,MIT A1实验室,UC Berkeley,都成为传统的、今天仍然是革新的源泉。


Linux是第一个有意识的成功的利用整个世界做为它的头脑库的项目,我不认为Linux的孕育和万维网的诞生相一致是一个巧合,而且Linux在1993-1994的一段ISP工业大发展和对Internet的兴趣爆炸式增长的时期中成长起来,Linus是第一个学会怎样利用Internet的新规的人。


廉价的Internet对Linux模式的演化来说是一个必要条件,但它并不充分,另一个关键因素是领导风格的开发和一套协作的氛围使开发人员可以吸引协作开发人员和最大限度地利用媒体。


但是这种领导风格与氛围到底是什么呢?它不能建立在权力关系之上——甚至如果它们可以,高压的领导权力不能产生我们所看到的结果,Weinberg引用了19世纪俄国的无政府主义者Kropotkin的“Memoris of a Revolutionist”来证明这个观点:


“我从小生活在一个农奴主的家庭中,我有一个活跃的生活,象我们时代的所有年轻人一样,我深信命令、强制、责骂、惩罚等等的必要性。但是当我(在早期)必须管理一个企业,和(自由)人打交道时,当每一个错误都会产生严重后果时,我开始接受以命令和纪律为准则来行动和以普通理解为准则来行动的区别。前者在军事阅兵中工作的很好,但是它在现实生活中一文不值,目标达成只是靠许多愿望的聚合的简单后果。”“许多聚合在一起的愿望的直接后果”精确地指出了象Linux的项目所需要的东西。“命令的准则”在Internet这种无政府主义的天堂中一群自愿者之中是没有市场的,为了更有效的操作和竞争,想领导协作项目的黑客们必须学会怎样以Kropotkins含糊指出的“理解的准则”模式来恢复和激活社团的力量,他们必须学会使用Linus定律。


前面我引用“Delhpi效应”来作为Linus定律的一个可能的解释,但是来自生物学和经常学的自适应系统的更强大的分析也提出了自己的解释,Linus世界的行为更象一个自由市场或生态系统,由一大群自私的个体组成,它们试图取得(自己)最大的实效,在这个过程中产生了比任何一种中央计划都细致和高效的自发的改进的结果,所以,这里就是寻找“理解的准则”的地方。


Linux黑客取得的最大化的“实际利益”不是经典的经济利益,而是无形的他们的自我满足和在其他黑客中的声望,(有人会说他们的动机是“利他的”,但这忽略了这样的事实:利他主义本身是利他主义者的一种自我满足的形式),自愿的文化以这种方式工作的实际上并非不寻常,我已参与一个科幻迷团体很长时间了,它不象黑客团体一样,显式地识别出“egoboo”(一个人在其他爱好者之中的声望的增长)作为自愿者活动背后的基础驱动力)。


Linus成功地把自己置于项目的守门人的位置,在项目中开发大部分是别人做的,他只是在项目中培养兴趣直到它可以自己发展下去,这为我们展示了对Kropokin的“共同理解原则”的敏锐把握,对Linux这种类似经济学的观点让我们看到这种理解是怎样应用的。


我们可以把Linus的方法视为创建一个高效的关于“egoboo”(而不是钱)的市场,来把自私的黑客个体尽可能紧密的联系起来,达成只能通过高度协作才能得到的困难的结果,在fetchmail项目中我展示了(在较小规模上)这种模式可以复制,得到良好的结果,也许我比他更有意识一点、更加系统一点。


许多人(尤其是哪些由于政治原因不信任自由市场的人)会盼望自我导向的自我主义者的文化破碎、报废、秘密和敌对,但这种盼望很明显地被Linux的文档的多样性、质量和深度打破了,程序员讨厌写文档似乎已是圣训,但Linux的黑客们怎么产生了这么多?显然Linux的egoboo自由市场比有大量资金的商业软件产品的文档部在产生有品德的、他人导向的行为方面工作的更好。


Fetchmail和Linux内核项目都表明,通过恰当的表彰许多其他黑客,一个强大的开发者/协调者可以用Internet得到许多协同开发人员而不是让项目分崩离析为一片混乱,所以关于Brooks定律我得到了下面的想法:


19. 如果开发协调人员有至少和Internet一样好的媒介,而且知道怎样不通过强迫来领导,许多头脑将不可避免地比一个好。


我认为自由软件的将来将属于那些知道怎样玩Linus的游戏的人,把大教堂抛之脑后拥抱市集的人,这并不是说个人的观点与才气不再重要,而是,我认为自由软件的前沿将属于从个人观点和才气出发的人,然后通过共同兴趣自愿社团的高效建造来扩展。


可能不只是自由软件的将来,在解决问题方面,没有任何商业性开发者可以与Linux社团的头脑库相匹敌,很少有人能负担起雇佣200多个为fetchmail出过力的人!


也许最终自由软件文化将胜利,不是因为协作在道德上是正确的或软件“囤积居奇”在道德上是错的(假设你相信后者,Linus和我都不),而仅仅是因为商业世界在进化的军备竞赛中不能战胜自由软件社团,因为后者可以把更大更好的开发资源放在解决问题上。

**** 网友写给作者的感想: ****

你好,Eric:
我刚读了你的大教堂/市集的文章,因为你的主页指出你还要继续关于这个问题的思考,我提供一些个人的观察。 首先介绍一些背景:当1990年出现BSD Net/2的时候,Brad Grantham和我把它移植到了MacⅡ平台上,它在几个月之后以Mac BSD发布(当然是以市集风格),后来成为Net BSD/Mac。 我作为一个市集协调人学到了一些东西:


1. 人们很快地自愿提供帮助,但是常常很慢,我们收到上百封信说:“我很想帮助,请告诉我需要什么?” 这些人没提供什么帮助,不管他们有多么积极,真正有帮助的人那些给我们的第一封信便说:“嘿,我修改了这个,这儿有一个补丁。”最后我们忽略了所有第一种类型的邮件(只是把他们引向工作列表),培养与第二种人的关系,这种情况所有协调人都应知道,来克服看到这么多“志愿者”时的盲目高兴。
(注意:他们的动机是好的,他们只是没有认识到他们正在志愿做什么)。


2. 你已经提到了这一点,但我认为它是极端重要的:甚至在你宣布产品以前你必须有一个可工作的系统:例如,我们一直等到有了一个可引导的内核和一个单用户根shell之后才把它贴到Usenet,曾有过(据我所知)四个不同的Mac Linux项目,每一个都在Linux新闻组中有一大批拥护者,都创建了邮件列表,每个人都很热情,写了FAQ,还有许多诸如MacOS的图标应是什么样的讨论。所有这些项目没有发布一行代码或者一个内核、我挑选了MkLinux(Apple开发的)作为一个可工作的Mac版Linux(在一个项目中,MacLinux假设运转在68K Mac上,而邮件列表中所有的讨论都是关于怎样把它移植到Power Mac上。68K版本甚至不能远程工作!),这些项目吸引了上述的第一种“帮助者”,热情高涨但是实际上却没做什么事,杀掉一个项目最快的方法是在你什么都还没有之前就宣布它,我已经见的太多了,尤其是在Linux世界里。

我知道这两点看起来相当悲观,但我知道当我们想到“啊,我们做了这么多事了,肯定搞定了不少问题了吧!”的时候,我们太容易失去理智。而那实际上只不过是一些善良的动机罢了(谁说过:“不要把动机和行动混淆在一起?” 本·弗兰克林?)协调人需要解散所有那些诸如图标应该是什么样的、FAQ用HTML格式还是SGML模式的热情讨论,而把注意力放在取得产品的一个可工作的版本,一旦得到了,人们就真正开始帮助了。

(从正面来看,MacBSD极大地得益于从它的开发风格,我们得到了代码、设备驱动程序、钱和一些捐赠和借到的测试和开发的硬件设备)。
我期望看到对我上述观点的任何评论和你关于这个主题写的任何东西。

\chapter{On Management and the Maginot Line}

The original Cathedral and Bazaar paper of 1997 ended with the vision above—that of happy networked hordes of programmer/anarchists outcompeting and overwhelming the hierarchical world of conventional closed software.

A good many skeptics weren't convinced, however; and the questions they raise deserve a fair engagement. Most of the objections to the bazaar argument come down to the claim that its proponents have underestimated the productivity-multiplying effect of conventional management.

Traditionally-minded software-development managers often object that the casualness with which project groups form and change and dissolve in the open-source world negates a significant part of the apparent advantage of numbers that the open-source community has over any single closed-source developer. They would observe that in software development it is really sustained effort over time and the degree to which customers can expect continuing investment in the product that matters, not just how many people have thrown a bone in the pot and left it to simmer.

There is something to this argument, to be sure; in fact, I have developed the idea that expected future service value is the key to the economics of software production in the essay The Magic Cauldron.

But this argument also has a major hidden problem; its implicit assumption that open-source development cannot deliver such sustained effort. In fact, there have been open-source projects that maintained a coherent direction and an effective maintainer community over quite long periods of time without the kinds of incentive structures or institutional controls that conventional management finds essential. The development of the GNU Emacs editor is an extreme and instructive example; it has absorbed the efforts of hundreds of contributors over 15 years into a unified architectural vision, despite high turnover and the fact that only one person (its author) has been continuously active during all that time. No closed-source editor has ever matched this longevity record.

This suggests a reason for questioning the advantages of conventionally-managed software development that is independent of the rest of the arguments over cathedral vs. bazaar mode. If it's possible for GNU Emacs to express a consistent architectural vision over 15 years, or for an operating system like Linux to do the same over 8 years of rapidly changing hardware and platform technology; and if (as is indeed the case) there have been many well-architected open-source projects of more than 5 years duration -- then we are entitled to wonder what, if anything, the tremendous overhead of conventionally-managed development is actually buying us.

Whatever it is certainly doesn't include reliable execution by deadline, or on budget, or to all features of the specification; it's a rare `managed' project that meets even one of these goals, let alone all three. It also does not appear to be ability to adapt to changes in technology and economic context during the project lifetime, either; the open-source community has proven far more effective on that score (as one can readily verify, for example, by comparing the 30-year history of the Internet with the short half-lives of proprietary networking technologies—or the cost of the 16-bit to 32-bit transition in Microsoft Windows with the nearly effortless upward migration of Linux during the same period, not only along the Intel line of development but to more than a dozen other hardware platforms, including the 64-bit Alpha as well).

One thing many people think the traditional mode buys you is somebody to hold legally liable and potentially recover compensation from if the project goes wrong. But this is an illusion; most software licenses are written to disclaim even warranty of merchantability, let alone performance—and cases of successful recovery for software nonperformance are vanishingly rare. Even if they were common, feeling comforted by having somebody to sue would be missing the point. You didn't want to be in a lawsuit; you wanted working software.

So what is all that management overhead buying?

In order to understand that, we need to understand what software development managers believe they do. A woman I know who seems to be very good at this job says software project management has five functions:

To define goals and keep everybody pointed in the same direction

To monitor and make sure crucial details don't get skipped

To motivate people to do boring but necessary drudgework

To organize the deployment of people for best productivity

To marshal resources needed to sustain the project

Apparently worthy goals, all of these; but under the open-source model, and in its surrounding social context, they can begin to seem strangely irrelevant. We'll take them in reverse order.

My friend reports that a lot of resource marshalling is basically defensive; once you have your people and machines and office space, you have to defend them from peer managers competing for the same resources, and from higher-ups trying to allocate the most efficient use of a limited pool.

But open-source developers are volunteers, self-selected for both interest and ability to contribute to the projects they work on (and this remains generally true even when they are being paid a salary to hack open source.) The volunteer ethos tends to take care of the `attack' side of resource-marshalling automatically; people bring their own resources to the table. And there is little or no need for a manager to `play defense' in the conventional sense.

Anyway, in a world of cheap PCs and fast Internet links, we find pretty consistently that the only really limiting resource is skilled attention. Open-source projects, when they founder, essentially never do so for want of machines or links or office space; they die only when the developers themselves lose interest.

That being the case, it's doubly important that open-source hackers organize themselves for maximum productivity by self-selection—and the social milieu selects ruthlessly for competence. My friend, familiar with both the open-source world and large closed projects, believes that open source has been successful partly because its culture only accepts the most talented 5% or so of the programming population. She spends most of her time organizing the deployment of the other 95%, and has thus observed first-hand the well-known variance of a factor of one hundred in productivity between the most able programmers and the merely competent.

The size of that variance has always raised an awkward question: would individual projects, and the field as a whole, be better off without more than 50% of the least able in it? Thoughtful managers have understood for a long time that if conventional software management's only function were to convert the least able from a net loss to a marginal win, the game might not be worth the candle.

The success of the open-source community sharpens this question considerably, by providing hard evidence that it is often cheaper and more effective to recruit self-selected volunteers from the Internet than it is to manage buildings full of people who would rather be doing something else.

Which brings us neatly to the question of motivation. An equivalent and often-heard way to state my friend's point is that traditional development management is a necessary compensation for poorly motivated programmers who would not otherwise turn out good work.

This answer usually travels with a claim that the open-source community can only be relied on only to do work that is `sexy' or technically sweet; anything else will be left undone (or done only poorly) unless it's churned out by money-motivated cubicle peons with managers cracking whips over them. I address the psychological and social reasons for being skeptical of this claim in Homesteading the Noosphere. For present purposes, however, I think it's more interesting to point out the implications of accepting it as true.

If the conventional, closed-source, heavily-managed style of software development is really defended only by a sort of Maginot Line of problems conducive to boredom, then it's going to remain viable in each individual application area for only so long as nobody finds those problems really interesting and nobody else finds any way to route around them. Because the moment there is open-source competition for a `boring' piece of software, customers are going to know that it was finally tackled by someone who chose that problem to solve because of a fascination with the problem itself—which, in software as in other kinds of creative work, is a far more effective motivator than money alone.

Having a conventional management structure solely in order to motivate, then, is probably good tactics but bad strategy; a short-term win, but in the longer term a surer loss.

So far, conventional development management looks like a bad bet now against open source on two points (resource marshalling, organization), and like it's living on borrowed time with respect to a third (motivation). And the poor beleaguered conventional manager is not going to get any succour from the monitoring issue; the strongest argument the open-source community has is that decentralized peer review trumps all the conventional methods for trying to ensure that details don't get slipped.

Can we save defining goals as a justification for the overhead of conventional software project management? Perhaps; but to do so, we'll need good reason to believe that management committees and corporate roadmaps are more successful at defining worthy and widely shared goals than the project leaders and tribal elders who fill the analogous role in the open-source world.

That is on the face of it a pretty hard case to make. And it's not so much the open-source side of the balance (the longevity of Emacs, or Linus Torvalds's ability to mobilize hordes of developers with talk of ``world domination'') that makes it tough. Rather, it's the demonstrated awfulness of conventional mechanisms for defining the goals of software projects.

One of the best-known folk theorems of software engineering is that 60% to 75% of conventional software projects either are never completed or are rejected by their intended users. If that range is anywhere near true (and I've never met a manager of any experience who disputes it) then more projects than not are being aimed at goals that are either (a) not realistically attainable, or (b) just plain wrong.

This, more than any other problem, is the reason that in today's software engineering world the very phrase ``management committee'' is likely to send chills down the hearer's spine—even (or perhaps especially) if the hearer is a manager. The days when only programmers griped about this pattern are long past; Dilbert cartoons hang over executives' desks now.

Our reply, then, to the traditional software development manager, is simple—if the open-source community has really underestimated the value of conventional management, why do so many of you display contempt for your own process?

Once again the example of the open-source community sharpens this question considerably—because we have fun doing what we do. Our creative play has been racking up technical, market-share, and mind-share successes at an astounding rate. We're proving not only that we can do better software, but that joy is an asset.

Two and a half years after the first version of this essay, the most radical thought I can offer to close with is no longer a vision of an open-source–dominated software world; that, after all, looks plausible to a lot of sober people in suits these days.

Rather, I want to suggest what may be a wider lesson about software, (and probably about every kind of creative or professional work). Human beings generally take pleasure in a task when it falls in a sort of optimal-challenge zone; not so easy as to be boring, not too hard to achieve. A happy programmer is one who is neither underutilized nor weighed down with ill-formulated goals and stressful process friction. Enjoyment predicts efficiency.

Relating to your own work process with fear and loathing (even in the displaced, ironic way suggested by hanging up Dilbert cartoons) should therefore be regarded in itself as a sign that the process has failed. Joy, humor, and playfulness are indeed assets; it was not mainly for the alliteration that I wrote of "happy hordes" above, and it is no mere joke that the Linux mascot is a cuddly, neotenous penguin.

It may well turn out that one of the most important effects of open source's success will be to teach us that play is the most economically efficient mode of creative work.

\chapter{Epilog: Netscape Embraces the Bazaar}

It's a strange feeling to realize you're helping make history....

On January 22 1998, approximately seven months after I first published The Cathedral and the Bazaar, Netscape Communications, Inc. announced plans to give away the source for Netscape Communicator. I had had no clue this was going to happen before the day of the announcement.

Eric Hahn, executive vice president and chief technology officer at Netscape, emailed me shortly afterwards as follows: ``On behalf of everyone at Netscape, I want to thank you for helping us get to this point in the first place. Your thinking and writings were fundamental inspirations to our decision.''

The following week I flew out to Silicon Valley at Netscape's invitation for a day-long strategy conference (on 4 Feb 1998) with some of their top executives and technical people. We designed Netscape's source-release strategy and license together.

A few days later I wrote the following:

Netscape is about to provide us with a large-scale, real-world test of the bazaar model in the commercial world. The open-source culture now faces a danger; if Netscape's execution doesn't work, the open-source concept may be so discredited that the commercial world won't touch it again for another decade.

On the other hand, this is also a spectacular opportunity. Initial reaction to the move on Wall Street and elsewhere has been cautiously positive. We're being given a chance to prove ourselves, too. If Netscape regains substantial market share through this move, it just may set off a long-overdue revolution in the software industry.

The next year should be a very instructive and interesting time.

And indeed it was. As I write in mid-2000, the development of what was later named Mozilla has been only a qualified success. It achieved Netscape's original goal, which was to deny Microsoft a monopoly lock on the browser market. It has also achieved some dramatic successes (notably the release of the next-generation Gecko rendering engine).

However, it has not yet garnered the massive development effort from outside Netscape that the Mozilla founders had originally hoped for. The problem here seems to be that for a long time the Mozilla distribution actually broke one of the basic rules of the bazaar model; it didn't ship with something potential contributors could easily run and see working. (Until more than a year after release, building Mozilla from source required a license for the proprietary Motif library.)

Most negatively (from the point of view of the outside world) the Mozilla group didn't ship a production-quality browser for two and a half years after the project launch—and in 1999 one of the project's principals caused a bit of a sensation by resigning, complaining of poor management and missed opportunities. ``Open source,'' he correctly observed, ``is not magic pixie dust.''

And indeed it is not. The long-term prognosis for Mozilla looks dramatically better now (in November 2000) than it did at the time of Jamie Zawinski's resignation letter—in the last few weeks the nightly releases have finally passed the critical threshold to production usability. But Jamie was right to point out that going open will not necessarily save an existing project that suffers from ill-defined goals or spaghetti code or any of the software engineering's other chronic ills. Mozilla has managed to provide an example simultaneously of how open source can succeed and how it could fail.

In the mean time, however, the open-source idea has scored successes and found backers elsewhere. Since the Netscape release we've seen a tremendous explosion of interest in the open-source development model, a trend both driven by and driving the continuing success of the Linux operating system. The trend Mozilla touched off is continuing at an accelerating rate.

\chapter{Notes}

[QR] Examples of successful open-source, bazaar development predating the Internet explosion and unrelated to the Unix and Internet traditions have existed. The development of the info-Zip compression utility during 1990–x1992, primarily for DOS machines, was one such example. Another was the RBBS bulletin board system (again for DOS), which began in 1983 and developed a sufficiently strong community that there have been fairly regular releases up to the present (mid-1999) despite the huge technical advantages of Internet mail and file-sharing over local BBSs. While the info-Zip community relied to some extent on Internet mail, the RBBS developer culture was actually able to base a substantial on-line community on RBBS that was completely independent of the TCP/IP infrastructure.

[CV] That transparency and peer review are valuable for taming the complexity of OS development turns out, after all, not to be a new concept. In 1965, very early in the history of time-sharing operating systems, Corbató and Vyssotsky, co-designers of the Multics operating system, wrote

It is expected that the Multics system will be published when it is operating substantially... Such publication is desirable for two reasons: First, the system should withstand public scrutiny and criticism volunteered by interested readers; second, in an age of increasing complexity, it is an obligation to present and future system designers to make the inner operating system as lucid as possible so as to reveal the basic system issues.

[JH] John Hasler has suggested an interesting explanation for the fact that duplication of effort doesn't seem to be a net drag on open-source development. He proposes what I'll dub ``Hasler's Law'': the costs of duplicated work tend to scale sub-qadratically with team size—that is, more slowly than the planning and management overhead that would be needed to eliminate them.

This claim actually does not contradict Brooks's Law. It may be the case that total complexity overhead and vulnerability to bugs scales with the square of team size, but that the costs from duplicated work are nevertheless a special case that scales more slowly. It's not hard to develop plausible reasons for this, starting with the undoubted fact that it is much easier to agree on functional boundaries between different developers' code that will prevent duplication of effort than it is to prevent the kinds of unplanned bad interactions across the whole system that underly most bugs.

The combination of Linus's Law and Hasler's Law suggests that there are actually three critical size regimes in software projects. On small projects (I would say one to at most three developers) no management structure more elaborate than picking a lead programmer is needed. And there is some intermediate range above that in which the cost of traditional management is relatively low, so its benefits from avoiding duplication of effort, bug-tracking, and pushing to see that details are not overlooked actually net out positive.

Above that, however, the combination of Linus's Law and Hasler's Law suggests there is a large-project range in which the costs and problems of traditional management rise much faster than the expected cost from duplication of effort. Not the least of these costs is a structural inability to harness the many-eyeballs effect, which (as we've seen) seems to do a much better job than traditional management at making sure bugs and details are not overlooked. Thus, in the large-project case, the combination of these laws effectively drives the net payoff of traditional management to zero.

One way to do this is to fix the deadline but leave the feature list flexible, allowing features to drop off if not completed by deadline. This is essentially the strategy of the "stable" kernel branch; Alan Cox (the stable-kernel maintainer) puts out releases at fairly regular intervals, but makes no guarantees about when particular bugs will be fixed or what features will beback-ported from the experimental branch.

The other way to do this is to set a desired feature list and deliver only when it is done. This is essentially the strategy of the "experimental" kernel branch. De Marco and Lister cited research showing that this scheduling policy ("wake me up when it's done") produces not only the highest quality but, on average, shorter delivery times than either "realistic" or "aggressive" scheduling.

I have come to suspect (as of early 2000) that in earlier versions of this essay I severely underestimated the importance of the "wake me up when it's done" anti-deadline policy to the open-source community's productivity and quality. General experience with the rushed GNOME 1.0 release in 1999 suggests that pressure for a premature release can neutralize many of the quality benefits open source normally confers.

It may well turn out to be that the process transparency of open source is one of three co-equal drivers of its quality, along with "wake me up when it's done" scheduling and developer self-selection.

[SU] It's tempting, and not entirely inaccurate, to see the core-plus-halo organization characteristic of open-source projects as an Internet-enabled spin on Brooks's own recommendation for solving the N-squared complexity problem, the "surgical-team" organization—but the differences are significant. The constellation of specialist roles such as "code librarian" that Brooks envisioned around the team leader doesn't really exist; those roles are executed instead by generalists aided by toolsets quite a bit more powerful than those of Brooks's day. Also, the open-source culture leans heavily on strong Unix traditions of modularity, APIs, and information hiding—none of which were elements of Brooks's prescription.

[RJ] The respondent who pointed out to me the effect of widely varying trace path lengths on the difficulty of characterizing a bug speculated that trace-path difficulty for multiple symptoms of the same bug varies "exponentially" (which I take to mean on a Gaussian or Poisson distribution, and agree seems very plausible). If it is experimentally possible to get a handle on the shape of this distribution, that would be extremely valuable data. Large departures from a flat equal-probability distribution of trace difficulty would suggest that even solo developers should emulate the bazaar strategy by bounding the time they spend on tracing a given symptom before they switch to another. Persistence may not always be a virtue...

[IN] An issue related to whether one can start projects from zero in the bazaar style is whether the bazaar style is capable of supporting truly innovative work. Some claim that, lacking strong leadership, the bazaar can only handle the cloning and improvement of ideas already present at the engineering state of the art, but is unable to push the state of the art. This argument was perhaps most infamously made by the Halloween Documents, two embarrassing internal Microsoft memoranda written about the open-source phenomenon. The authors compared Linux's development of a Unix-like operating system to ``chasing taillights'', and opined ``(once a project has achieved "parity" with the state-of-the-art), the level of management necessary to push towards new frontiers becomes massive.''

There are serious errors of fact implied in this argument. One is exposed when the Halloween authors themseselves later observe that ``often [...] new research ideas are first implemented and available on Linux before they are available / incorporated into other platforms.''

If we read ``open source'' for ``Linux'', we see that this is far from a new phenomenon. Historically, the open-source community did not invent Emacs or the World Wide Web or the Internet itself by chasing taillights or being massively managed—and in the present, there is so much innovative work going on in open source that one is spoiled for choice. The GNOME project (to pick one of many) is pushing the state of the art in GUIs and object technology hard enough to have attracted considerable notice in the computer trade press well outside the Linux community. Other examples are legion, as a visit to Freshmeat on any given day will quickly prove.

But there is a more fundamental error in the implicit assumption that the cathedral model (or the bazaar model, or any other kind of management structure) can somehow make innovation happen reliably. This is nonsense. Gangs don't have breakthrough insights—even volunteer groups of bazaar anarchists are usually incapable of genuine originality, let alone corporate committees of people with a survival stake in some status quo ante. Insight comes from individuals. The most their surrounding social machinery can ever hope to do is to be responsive to breakthrough insights—to nourish and reward and rigorously test them instead of squashing them.

Some will characterize this as a romantic view, a reversion to outmoded lone-inventor stereotypes. Not so; I am not asserting that groups are incapable of developing breakthrough insights once they have been hatched; indeed, we learn from the peer-review process that such development groups are essential to producing a high-quality result. Rather I am pointing out that every such group development starts from—is necessarily sparked by—one good idea in one person's head. Cathedrals and bazaars and other social structures can catch that lightning and refine it, but they cannot make it on demand.

Therefore the root problem of innovation (in software, or anywhere else) is indeed how not to squash it—but, even more fundamentally, it is how to grow lots of people who can have insights in the first place.

To suppose that cathedral-style development could manage this trick but the low entry barriers and process fluidity of the bazaar cannot would be absurd. If what it takes is one person with one good idea, then a social milieu in which one person can rapidly attract the cooperation of hundreds or thousands of others with that good idea is going inevitably to out-innovate any in which the person has to do a political sales job to a hierarchy before he can work on his idea without risk of getting fired.

And, indeed, if we look at the history of software innovation by organizations using the cathedral model, we quickly find it is rather rare. Large corporations rely on university research for new ideas (thus the Halloween Documents authors' unease about Linux's facility at coopting that research more rapidly). Or they buy out small companies built around some innovator's brain. In neither case is the innovation native to the cathedral culture; indeed, many innovations so imported end up being quietly suffocated under the "massive level of management" the Halloween Documents' authors so extol.

That, however, is a negative point. The reader would be better served by a positive one. I suggest, as an experiment, the following:

Pick a criterion for originality that you believe you can apply consistently. If your definition is ``I know it when I see it'', that's not a problem for purposes of this test.

Pick any closed-source operating system competing with Linux, and a best source for accounts of current development work on it.

Watch that source and Freshmeat for one month. Every day, count the number of release announcements on Freshmeat that you consider `original' work. Apply the same definition of `original' to announcements for that other OS and count them.

Thirty days later, total up both figures.

The day I wrote this, Freshmeat carried twenty-two release announcements, of which three appear they might push state of the art in some respect, This was a slow day for Freshmeat, but I will be astonished if any reader reports as many as three likely innovations a month in any closed-source channel.

[EGCS] We now have history on a project that, in several ways, may provide a more indicative test of the bazaar premise than fetchmail; EGCS, the Experimental GNU Compiler System.

This project was announced in mid-August of 1997 as a conscious attempt to apply the ideas in the early public versions of The Cathedral and the Bazaar. The project founders felt that the development of GCC, the Gnu C Compiler, had been stagnating. For about twenty months afterwards, GCC and EGCS continued as parallel products—both drawing from the same Internet developer population, both starting from the same GCC source base, both using pretty much the same Unix toolsets and development environment. The projects differed only in that EGCS consciously tried to apply the bazaar tactics I have previously described, while GCC retained a more cathedral-like organization with a closed developer group and infrequent releases.

This was about as close to a controlled experiment as one could ask for, and the results were dramatic. Within months, the EGCS versions had pulled substantially ahead in features; better optimization, better support for FORTRAN and C++. Many people found the EGCS development snapshots to be more reliable than the most recent stable version of GCC, and major Linux distributions began to switch to EGCS.

In April of 1999, the Free Software Foundation (the official sponsors of GCC) dissolved the original GCC development group and officially handed control of the project to the the EGCS steering team.

[SP] Of course, Kropotkin's critique and Linus's Law raise some wider issues about the cybernetics of social organizations. Another folk theorem of software engineering suggests one of them; Conway's Law—commonly stated as ``If you have four groups working on a compiler, you'll get a 4-pass compiler''. The original statement was more general: ``Organizations which design systems are constrained to produce designs which are copies of the communication structures of these organizations.'' We might put it more succinctly as ``The means determine the ends'', or even ``Process becomes product''.

It is accordingly worth noting that in the open-source community organizational form and function match on many levels. The network is everything and everywhere: not just the Internet, but the people doing the work form a distributed, loosely coupled, peer-to-peer network that provides multiple redundancy and degrades very gracefully. In both networks, each node is important only to the extent that other nodes want to cooperate with it.

The peer-to-peer part is essential to the community's astonishing productivity. The point Kropotkin was trying to make about power relationships is developed further by the `SNAFU Principle': ``True communication is possible only between equals, because inferiors are more consistently rewarded for telling their superiors pleasant lies than for telling the truth.'' Creative teamwork utterly depends on true communication and is thus very seriously hindered by the presence of power relationships. The open-source community, effectively free of such power relationships, is teaching us by contrast how dreadfully much they cost in bugs, in lowered productivity, and in lost opportunities.

Further, the SNAFU principle predicts in authoritarian organizations a progressive disconnect between decision-makers and reality, as more and more of the input to those who decide tends to become pleasant lies. The way this plays out in conventional software development is easy to see; there are strong incentives for the inferiors to hide, ignore, and minimize problems. When this process becomes product, software is a disaster.

\chapter{Bibliography}

I quoted several bits from Frederick P. Brooks's classic The Mythical Man-Month because, in many respects, his insights have yet to be improved upon. I heartily recommend the 25th Anniversary edition from Addison-Wesley (ISBN 0-201-83595-9), which adds his 1986 ``No Silver Bullet'' paper.

The new edition is wrapped up by an invaluable 20-years-later retrospective in which Brooks forthrightly admits to the few judgements in the original text which have not stood the test of time. I first read the retrospective after the first public version of this essay was substantially complete, and was surprised to discover that Brooks attributed bazaar-like practices to Microsoft! (In fact, however, this attribution turned out to be mistaken. In 1998 we learned from the Halloween Documents that Microsoft's internal developer community is heavily balkanized, with the kind of general source access needed to support a bazaar not even truly possible.)

Gerald M. Weinberg's The Psychology Of Computer Programming (New York, Van Nostrand Reinhold 1971) introduced the rather unfortunately-labeled concept of ``egoless programming''. While he was nowhere near the first person to realize the futility of the ``principle of command'', he was probably the first to recognize and argue the point in particular connection with software development.

Richard P. Gabriel, contemplating the Unix culture of the pre-Linux era, reluctantly argued for the superiority of a primitive bazaar-like model in his 1989 paper ``LISP: Good News, Bad News, and How To Win Big''. Though dated in some respects, this essay is still rightly celebrated among LISP fans (including me). A correspondent reminded me that the section titled ``Worse Is Better'' reads almost as an anticipation of Linux. The paper is accessible on the World Wide Web at http://www.naggum.no/worse-is-better.html.

De Marco and Lister's Peopleware: Productive Projects and Teams (New York; Dorset House, 1987; ISBN 0-932633-05-6) is an underappreciated gem which I was delighted to see Fred Brooks cite in his retrospective. While little of what the authors have to say is directly applicable to the Linux or open-source communities, the authors' insight into the conditions necessary for creative work is acute and worthwhile for anyone attempting to import some of the bazaar model's virtues into a commercial context.

Finally, I must admit that I very nearly called this essay ``The Cathedral and the Agora'', the latter term being the Greek for an open market or public meeting place. The seminal ``agoric systems'' papers by Mark Miller and Eric Drexler, by describing the emergent properties of market-like computational ecologies, helped prepare me to think clearly about analogous phenomena in the open-source culture when Linux rubbed my nose in them five years later. These papers are available on the Web at http://www.agorics.com/agorpapers.html.

\chapter{Acknowledgements}

This essay was improved by conversations with a large number of people who helped debug it. Particular thanks to Jeff Dutky <dutky@wam.umd.edu>, who suggested the ``debugging is parallelizable'' formulation, and helped develop the analysis that proceeds from it. Also to Nancy Lebovitz <nancyl@universe.digex.net> for her suggestion that I emulate Weinberg by quoting Kropotkin. Perceptive criticisms also came from Joan Eslinger <wombat@kilimanjaro.engr.sgi.com> and Marty Franz <marty@net-link.net> of the General Technics list. Glen Vandenburg <glv@vanderburg.org> pointeed out the importance of self-selection in contributor populations and suggested the fruitful idea that much development rectifies `bugs of omission'; Daniel Upper <upper@peak.org> suggested the natural analogies for this. I'm grateful to the members of PLUG, the Philadelphia Linux User's group, for providing the first test audience for the first public version of this essay. Paula Matuszek <matusp00@mh.us.sbphrd.com> enlightened me about the practice of software management. Phil Hudson <phil.hudson@iname.com> reminded me that the social organization of the hacker culture mirrors the organization of its software, and vice-versa. John Buck <johnbuck@sea.ece.umassd.edu> pointed out that MATLAB makes an instructive parallel to Emacs. Russell Johnston <russjj@mail.com> brought me to consciousness about some of the mechanisms discussed in ``How Many Eyeballs Tame Complexity.'' Finally, Linus Torvalds's comments were helpful and his early endorsement very encouraging.

\end{document}