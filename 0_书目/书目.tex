% 书目
% 书目.tex

\documentclass[a4paper,12pt,UTF8,twoside]{ctexbook}

% 设置纸张信息。
\RequirePackage[a4paper]{geometry}
\geometry{
	%textwidth=138mm,
	%textheight=215mm,
	%left=27mm,
	%right=27mm,
	%top=25.4mm, 
	%bottom=25.4mm,
	%headheight=2.17cm,
	%headsep=4mm,
	%footskip=12mm,
	%heightrounded,
	inner=1in,
	outer=1.25in
}

% 设置字体,并解决显示难检字问题。
\xeCJKsetup{AutoFallBack=true}
\setCJKmainfont{SimSun}[BoldFont=SimHei, ItalicFont=KaiTi, FallBack=SimSun-ExtB]

% 目录 chapter 级别加点(.)。
\usepackage{titletoc}
\titlecontents{chapter}[0pt]{\vspace{3mm}\bf\addvspace{2pt}\filright}{\contentspush{\thecontentslabel\hspace{0.8em}}}{}{\titlerule*[8pt]{.}\contentspage}

% 设置 part 和 chapter 标题格式。
\ctexset{
	chapter/name={第,篇},
	chapter/number={\chinese{chapter}},
	section/name={},
	section/number={}
}

% 设置古文原文格式。
\newenvironment{yuanwen}{\bfseries\zihao{4}}

\title{\heiti\zihao{0} 书目}
\author{王飞}
\date{\today}

\begin{document}

\maketitle
\tableofcontents

\frontmatter
\chapter{前言}


\mainmatter

\chapter{论语}

《论语》是春秋时期思想家、教育家孔子的弟子及再传弟子记录孔子及其弟子言行而编成的语录体散文集,成书于战国前期。全书共20篇492章,以语录体为主,叙事体为辅,较为集中地体现了孔子及儒家学派的政治主张、伦理思想、道德观念、教育原则等。

作品多为语录,但辞约义富,有些语句、篇章形象生动,其主要特点是语言简练,浅近易懂,而用意深远,有一种雍容和顺、纡徐含蓄的风格,能在简单的对话和行动中展示人物形象。

《论语》自宋代以后,被列为“四书”之一,成为古代学校官定教科书和科举考试必读书。

《论语》内容涉及政治、教育、文学、哲学以及立身处世的道理等多方面。现存《论语》20篇,492章,其中记录孔子与弟子及时人谈论之语约444章,记孔门弟子相互谈论之语48章。

\section{作品目录}

《论语》的篇名通常取开篇前两个字作为篇名;若开篇前两个字是“子曰”,则跳过取句中的前两个字;若开篇三个字是一个词,则取前三个字。篇名与其中的各章没有意义上的逻辑关系,仅可当作页码看待。这种类型的篇题在先秦时代的典籍中比较常见。

\section{成书过程}

《论语》是孔门弟子集体智慧的结晶。早在春秋后期孔子设坛讲学时期,《论语》主体内容就已初始创成;孔子去世以后,他的弟子和再传弟子代代传授他的言论,并逐渐将这些口头记诵的语录言行记录下来,因此称为“论”;《论语》主要记载孔子及其弟子的言行,因此称为“语”。清朝赵翼解释说:“语者,圣人之语言,论者,诸儒之讨论也。”其实,“论”又有纂的意思,所谓《论语》,是指将孔子及其弟子的言行记载下来编纂成书。其编纂者主要是仲弓、子游、子夏、子贡,他们忧虑师道失传,首先商量起草以纪念老师。然后和少数留在鲁国的弟子及再传弟子完成。

清代学者崔述注意到今本《论语》前后十篇在文体和称谓上存在差异,前十篇记孔子答定公、哀公之问,皆变文称“孔子对曰”,以表示尊君。答大夫之问则称“子曰”,表示有别于君,“以辨上下而定民志”。而后十篇中的《先进》《颜渊》等篇,答大夫之问也皆作“孔子对曰”,故怀疑“前十篇皆有子、曾子门人所记,去圣未远,礼制方明;后十篇则后人所续记,其时卿位益尊,卿权益重,盖有习于当世所称而未尝详考其体例者,故不能无异同也”。又如,前十篇中孔子一般称“子”不称“孔子”,门人问学也不作“问于孔子”。而后十篇中的《季氏》《微子》多称孔子,《阳货》篇子张问仁,《尧曰》篇子张问政,皆称“问于孔子”,与《论语》其他篇不同,“其非孔氏遗书明甚,盖皆后人采之他书者”。受崔述的影响,以后学者继续从《论语》前后十篇用语、称谓的差异对其成书作出判断,有学者甚至认为《论语》最初只有单独的篇,其编定成书,要在汉代以后。

唐代陆德明《经典释文》转引郑玄注云:《论语》“仲弓、子游、子夏等撰。”这一说法在郭店简中得到旁证。郭店一号墓不晚于公元前300年。郭店简“《语丛·三》简引述《论语》,更确证该书之早”。《语丛·一》引用子思子《坊记》内容,而《坊记》还引用过《论语》的内容。“《语丛》摘录《坊记》,证明《坊记》早于战国中期之末,而《坊记》又引述《论语》,看来《论语》为孔子门人仲弓、子夏等撰定之说还是可信的。”孔子弟子中,有子代孔子,以所事孔子事之,称子并不奇怪,子指老师,对儒家学子除了师承之外亦有掌门人之意。除了孔子,有子、曾子、冉子、闵子亦称子,冉子、闵子早逝,故师承的儒家掌门,只能是曾参,故代有子者只剩曾子有可能。颜回,虽然被尊奉,但由于早死,没来得及收徒,不是弟子记载,故不称子,可能是家人所记。原宪、漆雕开,虽然收徒,世称子思子、漆雕子,但《论语》所记,亦不是弟子记载,故亦不称子,多半是师兄弟偶而提及。

《论语》既是语录体又是若干断片的篇章集合体。这些篇章的排列不一定有什么道理;就是前后两章间,也不一定有什么关联。而且这些断片的篇章绝不是一个人的手笔。《论语》一书,篇幅不多,却出现了不少次的重复的章节。其中有字句完全相同的,如“巧言令色鲜矣仁”一章,先见于《学而篇第一》,又重出于《阳货篇第十七》;“博学于文”一章,先见于《雍也篇第六》,又重出于《颜渊篇第十二》。又有基本上是重复只是详略不同的,如“君子不重”章,《学而篇第一》多出十一字,《子罕篇第九》只载“主忠信”以下的十四个字;“父在观其志”章,《学而篇第一》多出十字,《里仁篇第四》只载“三年”以下的十二字。还有一个意思,却有各种记载的,如《里仁篇第四》说:“不患莫己知,求可为也。”《宪问篇第十四》又说:“不患人之不己知,患不知人也。”《卫灵公篇第十五》又说:“君子病无能焉,不病人之不己知也。”如果加上《学而篇第一》的“人不知而不愠,不亦君子乎”,便是重复四次。这种现象只能作一个合理的推论:孔子的言论,当时弟子各有记载,后来才汇集成书。

《论语》的作者中当然有孔子的学生。《论语》的篇章不但出自孔子不同学生之手,而且还出自他不同的再传弟子之手。这里面不少是曾参的学生的记载。如《泰伯篇第八》第一章:“曾子有疾,召门弟子曰:‘启予足!启予手!《诗》云,战战兢兢,如临深渊,如履薄冰。而今而后,吾知免夫!小子!’”又如《子张篇第十九》:“子夏之门人问交于子张。子张曰:‘子夏云何?’对曰:‘子夏曰:可者与之,其不可者拒之。’子张曰:‘异乎吾所闻:君子尊贤而容众,嘉善而矜不能。我之大贤欤,于人何所不容?我之不贤欤,人将拒我,如之何其拒人也?’”这一段又像子张或者子夏的学生的记载。又如《先进篇第十一》的第五章和第十三章:“子曰:‘孝哉闵子骞,人不间于其父母昆弟之言。’”“闵子侍侧,訚訚如也;子路,行行如也;冉有、子贡,侃侃如也。子乐。”孔子称学生从来直呼其名,独独这里对闵损称字。有人说,这是“孔子述时人之言”,崔述在《论语余说》中对这一解释加以驳斥。这一章可能就是闵损的学生所追记的,因而有这一不经意的失实。至于《闵子侍侧》一章,不但闵子骞称“子”,而且列在子路、冉有、子贡三人之前,都是难以理解的,以年龄而论,子路最长;以仕宦而论,闵子更赶不上这三人。他凭什么能在这一段记载上居于首位而且得着“子”的尊称呢?合理的推论是,这也是闵子骞的学生把平日闻于老师之言追记下来而成的。
《论语》一书有孔子弟子的笔墨,也有孔子再传弟子的笔墨,其著作年代有先有后。崔述《洙泗信录》断定《论语》的少数篇章的“驳杂”。从词义的运用上可反映出《论语》的著笔先后间相距或者不止于三、五十年。

思想内容
《论语》作为儒家经典,其内容博大精深,包罗万象,《论语》的思想主要有三个既各自独立又紧密相依的范畴:伦理道德范畴——仁,社会政治范畴——礼,认识方法论范畴——中庸。仁,首先是人内心深处的一种真实的状态,这种真的极致必然是善的,这种真和善的全体状态就是“仁”。孔子确立的仁的范畴,进而将礼阐述为适应仁、表达仁的一种合理的社会关系与待人接物的规范,进而明确“中庸”的系统方法论原则。“仁”是《论语》的思想核心。

《论语》反映了孔子的教育原则。孔子因材施教,对于不同的对象,考虑其不同的素质、优点和缺点、进德修业的具体情况,给予不同的教诲,表现了诲人不倦的可贵精神。据《颜渊》记载,同是弟子问仁,孔子有不同的回答,答颜渊“克己复礼为仁”(为仁的表现之一为克己复礼,有所不为);答仲弓“己所不欲,勿施于人”(就己与人之间的关系,以欲施做答,欲是个人的主观能动性之取舍,施是个人主观能动性的实践,用好心坏心来说,要防止好心办坏事,就要慎施);答司马牛“仁者其言也讱”。颜渊学养高深,故答以“仁”学纲领,对仲弓和司马牛则答以细目。又如,孔子回答子路和冉有的同一个问题,内容完全不同。答子路的是:“又父兄在,如之何其闻斯行之。”因为“由也兼人,故退之”。答冉有的是:“闻斯行之。”因为“求也退,故进之”。这不仅是因材施教教育方法的问题,其中还饱含孔子对弟子的高度的责任心。

艺术特色
《论语》多为语录,但都辞约义富,有些语句、篇章形象生动。如《子路曾皙冉有公西华侍坐》不仅篇幅较长,而且注重记述,算得上一篇结构完整的记叙文,人物形象鲜明,思想倾向通过人物表情、动作、对话自然地显露出来,具有较强的艺术性。
孔子是《论语》描述的中心,“夫子风采,溢于格言”(《文心雕龙·征圣》);书中不仅有关于他的仪态举止的静态描写,而且有关于他的个性气质的传神刻画。此外,围绕孔子这一中心,《论语》还成功地刻画了一些孔门弟子的形象。如子路的率直鲁莽,颜回的温雅贤良,子贡的聪颖善辩,曾皙的潇洒脱俗等等,都称得上个性鲜明,能给人留下深刻印象。
《论语》的主要特点是语言简练,用意深远,有一种雍容和顺、纡徐含蓄的风格;还有就是在简单的对话和行动中展示人物形象;同时语言浅近易懂,接近口语,也是一个特点。 

作品评价

西汉刘向《别录》:“《鲁论语》二十篇,皆孔子弟子记诸善言也。”
东汉班固《汉书·艺文志》:“《论语》者,孔子应答弟子、时人,及弟子相与言而接闻于夫子之语也。当时弟子各有所记,夫子既卒,门人相与辑而论纂,故谓之《论语》。”
东汉王充《论衡·正说篇》:“初,孔子孙孔安国以教鲁人扶卿,官至荆州刺史,始曰《论语》。”
东汉刘熙《释名·释典艺》:“《论语》,记孔子与弟子所语之言也。论,伦也,有伦理也。语,叙也,叙己所欲说也。”
西晋傅玄《傅子》:“昔仲尼既没,仲弓之徒追论夫子之言,谓之《论语》。” (南梁萧统 《文选·辨命论注》引)
北宋赵普:“臣平生所知,诚不出此,昔以其半辅太祖定天下,今欲以其半辅陛下致太平。”(罗大经《鹤林玉露》卷七)
北宋邢昺《邢疏》:“直言曰言,答述曰语,散则言语可通,故此论夫子之语而谓之善言也。”
南宋朱熹《朱子语类》卷一〇五:“‘四子’,‘六经’之阶梯;《近思录》,‘四子’之阶梯。”
南宋何异孙《十一经问对》:“《论语》有弟子记夫子之言者,有夫子答弟子问,有弟子自相答者,又有时人相言者,有臣对君问者,有师弟子对大夫之问者,皆所以讨论文义,故谓之《论语》。”
清代邵懿辰 《仪宋堂后记》:“明太祖既一海内,其佐刘基 ,以‘四子书’章义试士。行之五百年不改,以至于今。”
清代俞樾《春在堂随笔》卷九:“余撰《文勤神道碑》,即据其子儒卿等所撰行状,言公年十有一,‘四子书’、‘十三经’皆卒读。”
清代薛福成 《选举论中》:“常科以待天下占毕之士,试策论;论仍以‘四子’、‘ 五经’命题,特易其体格而已;策则参问古今事。” 

后世影响

《论语》是儒家经典之一。自汉武帝“罢黜百家,独尊儒术”之后,《论语》被尊为“五经之輨辖,六艺之喉衿”,是研究孔子及儒家思想尤其是原始儒家思想的第一手资料。南宋时朱熹将《大学》《中庸》《论语》《孟子》合为“四书”,使之在儒家经典中的地位日益提高。元代延祐年间,科举开始以“四书”开科取士。此后一直到清朝末年推行洋务运动,废除科举之前,《论语》一直是学子士人推施奉行的金科玉律。
《论语》进入经书之列是在唐代。“到唐代,礼有《周礼》《仪礼》《礼记》,春秋有《左传》《公羊》《谷梁》,加上《论语》《尔雅》《孝经》,这样是十三经。”北宋政治家赵普曾有“半部《论语》治天下”之说。这从一个侧面反映出此书在中国古代社会所发挥的作用与影响之大。
《论语》中保留了一些人们对孔子师徒的批评讽刺,有的作了辩驳,有的没有回答。其驳议辩难部分对后世很有影响,如《答客难》等设为主客问答进行辩难的小赋,都从《论语》受到启发;其自我解嘲部分,表现了儒家对自我价值的肯定,对“知其不可为而为之”的积极奋进精神的赞扬。

版本流传

各种版本

《论语》成书于战国初期。因秦始皇焚书坑儒(古时称为方术士,擅长祭祀,算命等),到西汉时期仅有口头传授及从孔子住宅夹壁中所得的本子,计有三种不同的本子:鲁人口头传授的《鲁论语》二十篇;齐人口头传授的《齐论语》二十二篇,其中二十篇的章句很多和《鲁论语》相同,但是多出《问王》和《知道》两篇;从孔子住宅夹壁中发现的《古文论语》(即《古论语》)二十一篇,也没有《问王》和《知道》两篇,但是把《尧曰篇》的“子张问”另分为一篇,于是有了两个《子张篇》,篇次也和《齐论》《鲁论》不一样,文字不同的计四百多字。

《鲁论语》和《齐论语》最初各有师传,到西汉末年,安昌侯张禹先学习了《鲁论语》,后来又讲习《齐论语》,于是把两个本子融合为一,但是篇目以《鲁论语》为根据,“采获所安”,另成一论,称为《张侯论》。张禹是汉成帝的师傅,其时极为尊贵,所以他的这一个本子便为当时一般儒生所尊奉,后汉灵帝时所刻的《熹平石经》就是用的《张侯论》。此本成为当时的权威读本,据《汉书·张禹传》记载:“诸儒为之语曰:‘欲为《论》,念张文。’由是学者多从张氏,馀家寝微。”《齐论语》《古论语》不久亡佚。东汉末郑玄又以“张侯论”为底本,参照《齐论》《古论》作《论语注》,遂为《论语》定本。

孔壁中书本《论语》由孔安国定。当孔安国向汉武帝献书时,正值“巫蛊事件”,朝廷将这批书退还给孔氏,“其学于是在孔家流传”。

《古文论语》是在汉景帝时由鲁恭王刘余在孔子旧宅壁中发现的,当时并没有传授。何晏《论语集解·序》说:“《古论》,唯博士孔安国为之训解,而世不传。”《论语集解》并经常引用了孔安国的《注》。但孔安国是否曾为《论语》作训解,《集解》中的孔安国说是否伪作,陈鳣的《论语古训·自序》已有怀疑,沈涛的《论语孔注辨伪》认为就是何晏自己的伪造品,丁晏的《论语孔注证伪》又认为出于王肃之手。

东汉末年,大学者郑玄以《张侯论》为依据,参照《齐论》《古论》,作了《论语注》。在残存的郑玄《论语注》中还可以窥见鲁、齐、古三种《论语》本子的异同,然而,今天所用的《论语》本子,基本上就是《张侯论》。张禹这个人实际上够不上说是一位“经师”,只是一个无耻的政客,附会王氏,保全富贵,当时便被斥为“佞臣”,所以崔述在《论语源流附考》中竟说:“《公山》《佛肸》两章安知非其有意采之以入《鲁论》为己解嘲乎?”但是,崔述的话纵然不为无理,而《论语》的篇章仍然不能说有后人所杜撰的东西在内,顶多只是说有掺杂着孔门弟子以及再传弟子之中的不同传说而已。

《论语》的版本之争实际上就是真伪之辩。1973年河北定县八角廊出土有竹简《论语》。2016年江西南昌西汉海昏侯刘贺墓出土了约五千枚竹简,考古人员在这些竹简中发现了失传已久的《论语·知道》篇,并初步断定属《论语》的《齐论》版本。

历代注本

《论语》与《孝经》都是汉初学习者必读之书,是汉人启蒙书的一种。自汉代以来,便有不少人注解《论语》。汉朝人所注《论语》,已亡佚殆尽,今日所残存的,以郑玄注为较多,因为敦煌和日本发现了一些唐写本的残卷,估计十存六七;其他各家,在何晏《论语集解》以后,就多半只存于《论语集解》中。《十三经注疏·论语注疏》就是用三国何晏的《集解》和宋人邢昺的《疏》。至于何晏、邢昺前后还有不少专注《论语》的书,可以参看清人朱彝尊《经义考》、纪昀等《四库全书总目提要》以及唐陆德明《经典释文序录》和吴检斋《疏证》。

两千年来,为《论语》作注释的书籍不胜枚举。据统计,历代研治《论语》的专著不下三千余种。可惜的是,这些古籍亡佚者居多。流传有序且影响较大的《论语》注释性著作有:一、汉郑玄《论语注》;二、魏何晏《论语集解》;三、梁皇侃《论语义疏》;四、宋朱熹《论语集注》;五、清刘宝楠《论语正义》;六、民国程树德《论语集释》。从文献学的角度来看,其中重要的有四部:一是《论语集解》,它是两汉、三国时期经学家研究《论语》的结晶;二是《论语义疏》,它囊括了魏晋南北朝时期玄学家对《论语》的发挥;三是《论语集注》,它是两宋时期理学家《论语》精义的荟粹;四是《论语正义》,集清代考据学《论语》研究成果之大成。这四部《论语》注释代表了《论语》研究的四个阶段,同时也代表了四种研究方法,是现代研究《论语》基本资料。

《论语》是孔子及其弟子的语录结集,由孔子弟子及再传弟子编写而成,并不是某一个人的著作。孔子开创了私人讲学的风气,相传他有弟子三千,贤弟子七十二人。孔子去世后,其弟子及再传弟子把孔子及其弟子的言行语录和思想记录下来,整理编成了儒家经典《论语》。

\part{人物}

\chapter{孔子}



\end{document}