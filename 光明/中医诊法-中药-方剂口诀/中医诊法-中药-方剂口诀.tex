% 中医诊法·中药·方剂口诀
% 中医诊法-中药-方剂口诀.tex

\documentclass[a4paper,12pt,UTF8,twoside]{ctexbook}

% 设置纸张信息。
\RequirePackage[a4paper]{geometry}
\geometry{
	%textwidth=138mm,
	%textheight=215mm,
	%left=27mm,
	%right=27mm,
	%top=25.4mm, 
	%bottom=25.4mm,
	%headheight=2.17cm,
	%headsep=4mm,
	%footskip=12mm,
	%heightrounded,
	inner=1in,
	outer=1.25in
}

% 设置字体,并解决显示难检字问题。
\xeCJKsetup{AutoFallBack=true}
\setCJKmainfont{SimSun}[BoldFont=SimHei, ItalicFont=KaiTi, FallBack=SimSun-ExtB]

% 目录 chapter 级别加点(.)。
\usepackage{titletoc}
\titlecontents{chapter}[0pt]{\vspace{3mm}\bf\addvspace{2pt}\filright}{\contentspush{\thecontentslabel\hspace{0.8em}}}{}{\titlerule*[8pt]{.}\contentspage}

% 设置 part 和 chapter 标题格式。
\ctexset{
	part/name= {第,卷},
	part/number={\chinese{part}},
	chapter/name={第,篇},
	chapter/number={\chinese{chapter}}
}

% 设置古文原文格式。
\newenvironment{yuanwen}{\bfseries\zihao{4}}

% 设置署名格式。
\newenvironment{shuming}{\hfill\bfseries\zihao{4}}

\title{\heiti\zihao{0} 中医诊法·中药·方剂口诀}
\author{}
\date{}

\begin{document}

\maketitle
\tableofcontents

\frontmatter

\chapter{编者}

光明中医函授大学 主编

蓝静海 高 铎 编

贾维诚 审

\chapter{编者的话}

《中医诊法•中药•方剂口诀》(简称《口诀》)是髙等中医函授教材之一,被列为光明中医函授大学前期必修的课程。在前期课程中有三门课部分内容与《口诀》内容重复,即,《中医药学概论》中的诊法部分;《本草备要讲解》中的常用中药部分;《方剂讲解》中方剂组成、功用部分。

《口诀》作为必修课的内容,要求学员熟记,是反映中医教育自身规律的一种做法,以引导初学者重视中医临床基本功的训练为宗旨。诊察与辨证论治的方法,药性与方剂的主治等知识和技能,时刻伴随着中医师的医疗生涯,是中医临床基本功。只有打好基本功,并在实践中逐步提高运用基本功的技能,才能把自己造就成为合格的临床中医师。所以,无论是过去或现在,掌握和善于运用临床基本功,的确是学好中医学的一个关键。

学习《口诀》,第一步,要逐字逐句的能读、能写、能背诵,是掌握临床基本功的起码要求。第二步,要逐字逐句的加以理解,结合《中医药学概论》,理解诊法口诀的含义;结合《本草备要讲解》,理解方药口诀(药性歌括四百味或药性赋)的含义;结合《方剂讲解》,理解方剂口诀的含义。这个学习方法,不但可以促使在理解的基础上加强记忆,同时能补充口诀内容之不足。第三步,要在实践中坚持运用《口诀》中认证、组方、遣药的内容,达到熟练运用的程度。

编者认为,背诵《口诀》并不难,难在认识其重要性,难在能不能持之以恒。希望学习和从事中医临床工作的同志们,都能重视中医临床基本功的训练。《口诀》是供学员随身携带、随时诵记之用,不必安排面授辅导,必要时可由老师解惑答疑。

​ 编者

​ 一九八八年七月二十八日

\chapter{导言}

中医教育学,是一门古老而崭新的科学。中医教育的历史,若从师徒授受和医籍编纂算起,已有两千余年。近代史上的中医教育,首推一八八五年浙江陈虬创立的利济医学堂。新中国诞生不久,创办了北京、上海、广州和成都四所中医学院,从而揭开了当代中医教育的序幕,至现在,全国已发展到二十三所。但是,如果把我国中医教育的实践经验加以分析、研究、总结和提炼,升华,揭示它的规律,使之成为一门专门的学科——中医教育学的话,那么,它还处在再创阶段。这就是说,中医教育及其规律存在的历史是悠久的,但论述中医教育及其规律的学科却是崭新的。因此,中医教育工作需要进行探索和研究。

在探索和创建适合我国国情的中医教育的时候,我们必须植根于我们民族文化的肥沃土壤之中,充分重视中医典籍在培育和造就历代医家中的伟大作用。事实上,在长期的历史发展中,逐渐形成了具有中华民族特色的中医药理论体系,它既有丰富临床经验,又有高深的理论基础。历代医学家就是把这些道理传授给他们的弟子,其中部分人经过刻苦自学和临床实践,成为医术高超的医学家,这是我国历代医学家成才之路,亦是中医教育史上培养人才的宝贵经验。这就是我们民族中医教育事业的光辉历史。

在新的历史时期,作为中医教育工作来说,既要给学生打好传统医学的基本功,又要使他们掌握一些新兴的科学知识,使继承与发展得到统一。根据这种认识,我们十分认真地研究和设计了光明中医函授大学的教学计划、教材内容、教学方法与教学手段。归结起来即是:注重打好中医基本功,注意提高中医基本理论水平和培养临床诊治技能,着力培养辨证论治的思维方法,竭诚发挥中医在防病治病中的特长。并在这个基础上,扩大学员知识面。我们把这些要求与思想,全面体现在本校的教材建设中。其目的是使中医人才的知识结构更加合理,以便能担负起继承和发扬祖国医药学防病治病的光荣任务。

在回顾中华医学教育历史,展望现代医学教育的发展趋势以及总结三十多年正反两方面经验的基础上,我们认为,要培养出适合四化需要的合格中医人才,对中医教育的课程设置和教材内容,就要进行必要的改革,建立起为新形势所需要的中医教材。我们正在朝这一方向努力。在认真研究高等中医院校教材和广泛征询中医专家、学者和医务人员意见的基础上,新编了这套较为完整的中医教材,定名为《高等中医函授教材》(包括了二十八门课程)教材的编写人员,由本校选聘知名教授、学者和学有专长者担任,编写时,我们力求各门教材要有鲜明的针对性,在内容上富有实用性,在文字表达上深入浅出、简明易懂,以利便于自学或函授,此外,我们还将根据需要,选编一些辅导材料,以帮助学员(读者)理解教材内容,更好地学取中医知识。

由于教材编写时间仓促,又竭力于继承与创新,不足之处在所难免,敬希学员和广大读者惠赐宝贵意见,以便在再版时修订。

光明中医函授大学教育研究室

一九八五年十月四日

\mainmatter

% 增加空行
~\\

% 增加字间间隔,适用于三字经、诗文等。
 \qquad  

\chapter{四诊心法要诀}

\section{望诊}

\begin{yuanwen}
望以目察,闻以耳占\footnote{占,根据观察,以辨吉凶为占。耳占,通过医生的耳来辨别病人声音的正常与病变,是古代的闻诊方法。},问以言审,切以指参。明斯诊道,识病根源,能合色脉\footnote{色,指病人的神色形态;脉,指诊得的脉象。能合色脉,是说诊断疾病,必须从整体出发,审察内外,四诊合参,互相参照,才能作出正确的诊断。},可以万全。
\end{yuanwen}

本段总说望、闻、问、切四诊是诊断疾病的主要诊法。

\begin{yuanwen}
五行五色\footnote{五行:木、火、土、金、水;五色:青、赤、黄、白、黑。},青赤黄白,黑复生青,如环常德\footnote{福的意思。}。
\end{yuanwen}

本段是说运用五行学说,说明望诊中的色诊。

\begin{yuanwen}
变色大要,生克顺逆。青赤兼化,赤黄合一,黄白淡黄,黑青深碧,白黑淡黑。白青浅碧,赤白化红,青黄变绿,黑赤成紫,黑黄黧立。
\end{yuanwen}

本段,前部分为五色相生之合化;后部分为五色相克之兼化。

\begin{yuanwen}
天有五气\footnote{风、暑、湿、燥、寒。},食人入鼻\footnote{食:sì,音饲。给人吃东西。食人入鼻,是说天地正常的五气,自鼻而入。},藏于五脏,上华面颐\footnote{yí,音移。口角后腮之下的部位。本段是说五脏无病的人,应该具有正常的色泽。}。肝青心赤,脾脏色黄,肺白肾黑,五脏之常。
\end{yuanwen}

\begin{yuanwen}
脏色为主,时色为客,春青夏赤,秋白冬黑,长夏四季,色黄常则。客胜主善,主胜客恶。
\end{yuanwen}

本段是说主色(又叫脏色〉,即人的正常面色;客色(又叫时色),即由于气候不同,而出现的四时面色。诊断疾病,必先注意主客气正常与否,如果出现异常,而又不是其他如饮食、劳倦等引起的,就是病色。

\begin{yuanwen}
色脉相合,青弦赤洪,黄缓白浮,黑沉乃平。已见其色,不得其脉,得克则死,得生则生。
\end{yuanwen}

本段是说望色与诊脉有相合相应和相反不相应的状况。

\begin{yuanwen}
新病脉夺\footnote{脉象微小。}、其色不夺。久病色夺\footnote{色无光泽。},其脉不夺。新病易已,色脉不夺。久病难治,色脉俱夺。
\end{yuanwen}

本段是说诊脉与望色的密切关系。

\begin{yuanwen}
色见皮外,气含皮中,内光外泽,气色相融。有色无气,不病命倾。有气无色,虽困不凶。
\end{yuanwen}

本段是说五色配五气的望色诊法。

\begin{yuanwen}
缟\footnote{gǎo,音稿。古时的白色罗绢。}裹雄黄,脾状并臻\footnote{zhēn,音珍。到达的意思。}。缟裹红肺,缟裹朱心。缟裹黑赤,紫艳肾缘\footnote{原因。}。缟裹蓝赤,石青属肝。
\end{yuanwen}

本段是以缟、雄黄、硃砂等以喻五脏反映在颜面上的色泽,应该是气色并至,是谓得神。

\begin{yuanwen}
青如苍璧\footnote{苍,青色;璧,美玉。},不欲如蓝。赤白裹朱,衃赭\footnote{衃,pī,音丕。败恶凝集的血色,即赤黑色。}死原。黑重漆炲\footnote{tái,音台。煤烟黑色。},白羽枯盐。雄黄罗裹,黄土修难。
\end{yuanwen}

本段是说从颜面反映的五色去辨别四时五脏、五部五官等疾病善恶的望诊方法。

\begin{yuanwen}
舌赤卷短,心官病常。肺鼻白喘,胸满喘张。肝目眦\footnote{zì,音恣。指内、外眼角。}青,脾病唇黄。耳黑肾病,深浅分彰\footnote{明的意思。}。
\end{yuanwen}

本段是以望得面部五官的色泽形态,诊断五脏病的虚实。

\begin{yuanwen}
左颊部肝,右颊部肺,额心頦肾,鼻脾部位。部见本色,深浅病累,若见他色,按法推类。
\end{yuanwen}

本段是以五色的浅淡深浓配合面部五个部位,以诊断病邪的虚实。

\begin{yuanwen}
天庭\footnote{阙上至前发际的部分。}面首,阙上\footnote{印堂之上部。}喉咽,阙中\footnote{即两眉之间的印堂。}印堂,候肺之原。山根\footnote{两眼之间。}候心,年寿\footnote{即鼻樑部。}侯肝,两傍候胆,脾胃鼻端。颊肾腰脐,颧下大肠,颧内小府\footnote{即小肠。},面王子膀\footnote{从鼻尖下至颊,为面王。子:子处,即子宫;膀:胱。}。当颧候肩,颧外候臂,颧外之下,乃候手位。根傍乳膺\footnote{yīng,音英。胸前两旁肌肉隆起处。},绳\footnote{绳骨。从两颊外侧向上,相当额部的转角处。}上候背,牙车\footnote{牙车骨,即牙床。}下股。膝胫足位。
\end{yuanwen}

本段是根据面部的色泽进行望诊的方法。

\begin{yuanwen}
庭阙鼻端,高起直平,颧颊蕃蔽\footnote{耳屏与面颊之间的部位。},大广丰隆,骨骼明显,寿享遐\footnote{xiá,音霞。长久的意思。}龄,骨胳陷弱,易受邪攻。
\end{yuanwen}

本段是根据人体面部五官的外观情况,推测身体强弱和健康与否。

\begin{yuanwen}
黄赤风热,青白主寒,青黑为痛,甚则痹挛,恍白\footnote{即㿠白,形容面色枯白。}脱血,微黑水寒,痿黄诸虚,颧赤劳缠。
\end{yuanwen}

本段是说五色出现者在头面部,可反映出一定的疾病。

\begin{yuanwen}
视色之锐\footnote{即锐处。为一片颜色的光形突出处。},所向部官。内走外易,外走内难。官部\footnote{面部五官部。}色脉,五病交参。上逆下顺,左右反阽\footnote{diàn,音店。危险的意思。}。
\end{yuanwen}

本段说明五色主病应根据其色部尖锐之处的指向,来推断疾病的起始、传变、生克和顺逆的方法。

\begin{yuanwen}
沉浊晦暗,内久而重,浮泽明显,外新而轻。其病不甚\footnote{指疾病不严重。},半泽半明。云散易治,摶\footnote{tuán,音团。结聚的意思。}聚难攻。
\end{yuanwen}

本段是按照面部五色浅深、明晦、聚散等不同情況来分辨疾病的新久、轻重、难治易治的方法。

\begin{yuanwen}
黑庭\footnote{指天庭。}赤颧,出如拇指,病虽小愈,亦必卒死;唇面黑青,五官黑起,擦残汗粉\footnote{五官忽然出现黑中夹白,好像擦汗时将残粉揩去后的面色,黑白明显的样子。},白色皆死。
\end{yuanwen}

本段是以面部出现异常色泽,来诊断病人将要出现险恶症候的方法。

\begin{yuanwen}
善色不病,于义诚当\footnote{按理当不病。};恶色不病,必主凶殃。五官陷弱,庭阙不张\footnote{这里指不壮满丰硕。},蕃蔽卑小\footnote{这里指槁弱瘦小。},不病神强\footnote{见这种外表的人而没有病,那他一定是神气强壮的。}。
\end{yuanwen}

本段说明善色与恶色的诊断方法。

\begin{yuanwen}
肝病善怒,面色当青,左有动气\footnote{古人有“肝左脾右”的说法。而肝的实体居于右。“左有动气”是指肝的气化行于左,并非指肝的实体。动气,一种跳动的动态、气势,即气化现象。},转筋胁疼。诸风掉眩\footnote{掉,指动摇抽搐的症状:眩,指头晕眼黑的症状。},疝病耳聋,目视䀮䀮\footnote{huáng,音黄。目视不明。},如将捕惊。
\end{yuanwen}

木段是说肝病配合五色以分辨虚实的诊断方法。

\begin{yuanwen}
心赤善喜,舌红口干,脐上动气,心胸痛烦,健忘惊悸,怔忡不安、实狂昏冒,虚悲悽然。
\end{yuanwen}

本段是说心病配合五色以分辨虚实的诊断方法。

\begin{yuanwen}
脾黄善忧,当脐动气,善思食少,倦怠乏力,腹满肠鸣,痛而下利\footnote{通痢。这里指泄泻及下痢。},实则身重,胀满便闭。
\end{yuanwen}

本段是说脾病配合五色以分辨虚实的诊断方法。

\begin{yuanwen}
肺白善悲,脐右动气,洒淅寒热\footnote{毛发悚然怕冷、身热的样子。},咳唾喷嚏,喘呼气促,肤痛胸痹,虚则气短,不能续息。
\end{yuanwen}

本段是说肺病配合五色以分辨虚实的诊断方法。

\begin{yuanwen}
肾黑善恐,脐下动气,腹胀肿喘,溲便不利,腰背少腹,骨痛欠气,心悬如饥,足寒厥逆。
\end{yuanwen}

本段是说肾病配合五色以分辨虚实的诊断方法。

\begin{yuanwen}
正病正色,为病多顺,病色交错,为病多逆,母乘子顺,子乘母逆,相克逆凶,相生顺吉。
\end{yuanwen}

本段是五色合五胜病根据五行生克来推断病情顺逆吉凶。

\begin{yuanwen}
色生于脏,各命其部,神藏于心,外候在目,光晦神短\footnote{光,指目光;目光晦暗,神气短,是将要患病或病势严重的现象。},了了\footnote{形容目光清莹、清爽分明。}神足,单失\footnote{指色与神,二者失其一。}久病,双失即故\footnote{既失色,又失神,即是险恶的死症。}。
\end{yuanwen}

本段是说以面色结合限神来预测疾病的善恶。

\begin{yuanwen}
面目之色,各有相当,交互错见,皆主身亡,面黄有救,眦红疹疡,眦黄病愈,睛黄发黄。
\end{yuanwen}

本段是说以面色结合眼的颜色来诊断疾病。

\begin{yuanwen}
闭目阴病,开目病阳,朦胧热盛,时瞑衄常\footnote{瞑,闭目。病人时常闭目,一会儿又开目,这是要发生衄血的征兆。},阳绝戴眼\footnote{指病人眼睛上视,不能转动。},阴脱目盲,气脱眶陷,睛定神亡。
\end{yuanwen}

本段是说根据眼的神情动静,以辨疾病的阴阳、生死。

以上各段,讲的都是望诊法。另外,还有一种舌诊,也是望诊的组成部分。兹介绍于后:

\subsection{舌诊新法}

\begin{yuanwen}
苔有六色,白黄赤黑,绛紫传变,皆为病色。
\end{yuanwen}

正常舌象,舌体柔软,活动自如,舌质淡红,舌苔薄润。若白不薄润,或黄或赤,或黑或绛,逐渐传变、皆为病色。

\begin{yuanwen}
白属湿聚,黄在胃经,赤伤津液,黑为寒侵,传黑重热,绛多血分,紫多热毒,死胎全黑。
\end{yuanwen}

\begin{yuanwen}
光润曰泽,涩厚白腻,无苔曰虚,肿满曰胀,中边曰地,淤积曰浊。
\end{yuanwen}

\begin{yuanwen}
黄浊可下,惟须有地。不厚而滑,清热透表。黄白相间,灰白不浊,或白不燥,不可苦泄。白厚干燥,胃燥气伤,滋肾药中,甘守津还。白薄外邪,只宜疏散,干薄肺伤,麦露芦根。白苔绛底,湿热遏伏,泄湿透热,黑透则润。
\end{yuanwen}

\begin{yuanwen}
色绛传营,绛兼黄白,气分之邪,泄卫透营。纯绛鲜泽,胞络受病,犀角翘郁,陷用牛黄。绛中心干,黄连石膏,色绛粘腻,芳香逐恶。抵齿难伸,内风已炽。绛而光华,胃阴已竭,急用甘凉,濡润救药。绛燥刼营,凉血清火。绛白黄点,病已生疳。绛大红点,热正攻心。绛干而痿,肾阴涸亡,阿胶鸡子,天冬地黄。中心绛干,清胃清心。舌尖绛干,心火上尖。导赤泻火,金汁黄连。
\end{yuanwen}

\begin{yuanwen}
紫暗散血,搏热琥珀,黄白边红,须用凉膈;紫肿冲心,半系酒毒,紫干难治,救宜从速。
\end{yuanwen}

\begin{yuanwen}
初病舌干,神志不昏,养正透邪,免至内陷,神昏不治,此意宜识。
\end{yuanwen}

\begin{yuanwen}
舌生芒刺,上焦热极,青布薄荷,拭之便去。旋生者险,用药勿滞。舌润闷极,脾热必盛,伤痕血枯,若搔以湿。
\end{yuanwen}

\begin{yuanwen}
神清舌胀,郁热化风,脾湿胃热,大黄见功。
\end{yuanwen}

\begin{yuanwen}
白腻吐涎,口甜滋味,湿热气象,脾湿瘅病,盈满上泛,芳香辛散。
\end{yuanwen}

\begin{yuanwen}
无苔似煤,不渴肢寒,知挟阴病,慎不可攻;燥者甘寒,润甘扶中。舌黑而滑、水来克火,阴病当温,肾竭短缩,五味人参,舌黑而干,津枯火炽,泻南补北\footnote{泻心火,补肾水。}。
\end{yuanwen}

\begin{yuanwen}
淡红胃伤,当甘勿凉。
\end{yuanwen}

\begin{yuanwen}
色如粉滑,四边紫绛,温疫初起,急急透解。见此舌者,病必见凶。
\end{yuanwen}

\begin{yuanwen}
小儿温疹,一般温毒,舌起小点,绛地点白,记切托法,勿使闭邪。
\end{yuanwen}

\begin{yuanwen}
伤寒初起,无苔而泽,温病苔先,慎之应知。
\end{yuanwen}

\section{闻诊}

\begin{yuanwen}
五色既审,五音当明。声为音本,音以声生,声之余韵,音遂以名。角征\footnote{zhī,音只。角征宫商羽为五音,以上各段说的是五色诊断方法,是为望诊。从本段开始以下共七段说辨别声音的闻诊方法。}宫商,并羽五声。
\end{yuanwen}

本段是说根据五色、五音的诊断方法。

\begin{yuanwen}
中空有窍,故肺主声。喉为声路,会厌\footnote{会厌,位于舌部及舌骨之后,形如一树叶,能张能收,呼吸发音时会厌开启,吞咽食物或呕吐时则会厌关闭,为发音之户。}门户,舌为声机,唇齿搧动,宽隘锐钝,厚薄之故。
\end{yuanwen}

本段是说声音由喉、会厌、舌、唇、齿等器官而发生的。

\begin{yuanwen}
舌后中发,喉音正宫,极长下浊\footnote{形容宫音的赀音极长、极低下、极重浊。},沉厚雄洪。开口张㗁,口音商成,次长下浊,铿锵\footnote{kēng qiāng,音坑、枪。金属撞击的清脆声。}肃清。撮口唇音,极短高清,柔细透彻,尖利羽声。舌点齿音,次短高清,抑扬咏越,征声始通。角缩舌音,条畅正中,长遂高下,清浊和平。
\end{yuanwen}

本段是说正常情况下所发的宫、商、羽、征、角五音。

\begin{yuanwen}
喜心所感,忻\footnote{xīn,音欣。高兴的样子。}散之声。怒心所感,忿厉之声。哀心所感,悲嘶\footnote{sī,音斯。亩、声音发哑。}之声。乐心所感,舒缓之声。敬心所感,正肃之声。爱心所感,温和之声。
\end{yuanwen}

本段是说由于人的情绪变化,会使五音有所不同,这是正常现象。

\begin{yuanwen}
五声之变,变则病生。肝呼而急,心笑而雄,脾歌以漫,肺哭促声,肾呻低微,色克则凶。
\end{yuanwen}

本段是说五脏有病,因而呼、笑、歌、哭、呻五种声音变化失常。

\begin{yuanwen}
好言者热,懒言者寒。言壮为实,言轻为虚。言微难复,夺气可知。讝妄无伦\footnote{谵言妄语,语无伦次,是为危候。},神明已失。
\end{yuanwen}

本段是以病人的语言状态,来辨别疾病的寒热虚实以及预后。

\begin{yuanwen}
失音声重,内火外寒。疮痛已久,劳哑使然。哑风\footnote{小儿抽风不语、大人中风不语,称为哑风。}不语,虽治命难。讴\footnote{加,音欧。讴歌就是唱歌。}歌失音,不治亦痊。
\end{yuanwen}

本段是讲几种失音、不语症的辨诊。

\section{问诊}

声色既详,问亦当知。视其五入\footnote{五味所入。即酸入肝,苦入心,甘入脾,辛入肺,咸入肾。},以知起止。心主五臭\footnote{即焦、臊、香、腥、腐。},自入为焦,脾香肾腐,肺腥肝臊。脾主五味\footnote{即酸、苦、甘、辛、咸。},自入为甘,肝酸心苦,肺辛肾咸。肾主五液\footnote{即汗、泣、涎、涕、唾。},心汗肝泣,自入为唾,脾涎肺涕。

以上各段讲的是闻诊。自本段开始以下共五段讲问诊。本段是说根据病人的五入及五液的排出情况,以作诊断的参考。

\begin{yuanwen}
五病之常,昼安朝慧\footnote{在这里作清爽解释。},夕加夜甚,正邪进退。潮作\footnote{指病邪来潮发作的时候。}之时,精神为贵,不衰者实,困弱虚累。
\end{yuanwen}

本段是说病人在一天一夜中疾病消长、精神盛衰的情况,以诊断正气的虚实。

昼剧而热,阳旺于阳;夜剧而寒,阴旺于阴;昼剧而寒,阴上乘阳;夜剧而热,阳下陷阴。昼夜寒厥,重阴无阳;昼夜烦热,重阳无阴。昼寒夜热,阴阳交错。饮食不入,死终难却。

本段是讲病人日夜寒热发作的情况,以诊断疾病属阴、属阳、属寒、属热和病势轻重。

食多气少,火化新痊\footnote{}。食少气多,胃、肺两愆\footnote{}。喜冷有热,喜热有寒,寒热虚实,多少之间\footnote{}。

注:

\footnote{}火化新痊:火化,指胃火,消谷善饥。新痊,指病后新愈而贪食。

\footnote{}愆:过错。这里引伸为疾病。

\footnote{}多少之间:辨别喜吃冷热饮食多少的不同。

本段是说病人饮食多少和喜热、喜冷等,以诊断疾病属气、属火、属寒、属热。

大便通闭,关乎虚实,无热阴结,无寒阳利。小便红白,主乎热寒,阴虚红浅\footnote{},湿热白泔\footnote{}。

注:

\footnote{}红浅:指小便颜色浅红或浅黄色。

\footnote{}白泔:指小便混浊象米泔水一样。

本段是说病人大小便排泄情况,以诊断疾病的阴阳、寒热、虚实。

望以观色,问以测情,召医至榻,不盼不惊,或告之痛,并无苦容,色脉皆和,诈医欺蒙\footnote{}。

注:

\footnote{}诈医欺蒙:病人主诉病痛,但医生看其无痛苦表情,色脉均无异常,这是诈病欺骗医生。

本段是说医生逋过望诊、问诊等,以辨疾病的真假。

脉之呻吟,病者常情。摇头而言,护处\footnote{}必疼。三言三止\footnote{},言蹇\footnote{}为风,咽唾呵欠,皆非病征。

注:

\footnote{}护处:指病人以手按覆。如手按腹部,是腹痛的表现。

\footnote{}三言三止:病人想说不说,不说又想说,如此者三。

\footnote{}蹇:jiǎn,音剪。迟纯的意思。这里引伸为言语蹇滞。

本段是说通过望诊、问诊等,以辨疾病的真假。

黑色无痛,女疸肾伤,非疸蓄血,衄下后黄。面微黄黑,纹绕口角,饥瘦之容,询\footnote{}必噎膈。

注:

\footnote{}询:问的意思,如询问。

本段是说望诊结合问诊的诊断。

白不脱血\footnote{},脉如乱丝,问因恐怖,气下神失。乍白乍赤\footnote{},脉浮气怯,羞愧神荡,有此气色。

注:

\footnote{}白不脱血:面色白多为失血之貌。现面色白而无失血之证,这是受恐吓所致。

\footnote{}乍白乍赤:指面色一阵白、一阵红,不是病态,乃羞愧的缘故,

本段是讲望色、切脉和问诊结合起来,以分析面色白的原因。

眉起五色,其病在皮。营变蠕动\footnote{},血脉可知。眦目筋病,唇口主肌,耳主骨病,焦枯垢泥。

注:

\footnote{}营变蠕动:营变,指见到脉管起异常色泽,而且蠕蠕搏动。

本段是从观察两眉间的印堂、脉管、眼目、唇口、耳等色泽的变化,以诊断疾病。本段和以下二段,是古人诊病的“杂法”。

发上属火,鬚下属水,皮毛属金,眉横属木。属土之毫\footnote{},腋阴脐腹。发直如麻,毛焦死故。

注:

\footnote{}毫:指下句腋毛、阴毛及脐部、腹部的毫毛。

本段是按毛发的部位和形状来诊断疾病所属脏腑。

阴络从经,而有常色,阳络无常,随时变色。寒多则凝,凝则黑青;热多则淖\footnote{},淖则黄红。

注:

\footnote{}淖:nào,音闹。形容雨后路上的烂泥。这里指脉络沾濡的状态。

本段是讲有关阴阳脉络的诊断方法,

胃之大络,名曰虚里,动在乳下,有过不及。其动应衣\footnote{},宗气\footnote{}外泄,促结积聚\footnote{},不至则死。

注:

\footnote{}其动应衣:这里是说虚里搏动太大,甚则衣外可见。

\footnote{}宗气:也称大气,积于胸中,也即维持生命的根本元气。

\footnote{}促结积聚:虚里跳动三、四次停一下,或五、六次停一下,则为促结,是有积聚的病证。

本段是古代医生诊视胸腹部的方法。

脉尺\footnote{}相应,尺寒虚泻,尺热病温,阴虚寒热,风病尺滑,痹病尺涩,尺大丰盛,尺小亏竭。

注:

\footnote{}尺:从掌后高骨至尺泽穴之间的皮肤,称尺肤。

本段是讲诊察尺肤的方法。

肘候腰腹,手股足端。尺\footnote{}外肩背,尺内膺前。掌中腹中,鱼\footnote{}青胃寒。寒热所在,病生热寒。

注:

\footnote{}尺:这里的尺,指小臂内侧。

\footnote{}鱼:手拇指(或足拇趾)后方的掌(或跖)骨处有明显肌肉隆起,状如鱼腹的部位。

本段是讲诊察肘臂的方法。

诊脐上下,上胃下肠,腹皮寒热,肠胃相当。胃喜冷饮,肠喜热汤,热无灼灼\footnote{},寒无沧沧\footnote{}。

注:

\footnote{}灼灼:火热的意思,

\footnote{}沧沧:寒冷的意思。

本段是讲按诊脐腹的方法。

胃热口糜,悬心善饥。肠热利\footnote{}热,土黄如糜。胃寒清厥,腹胀而疼。肠寒尿白,飧泻\footnote{}肠鸣。

注:

\footnote{}利:这里指大便或痢便。

\footnote{}飧泻:飧,sūn,音孙。泻下清稀,完谷不化叫飧泻。

本段是讲诊断胃肠寒热的方法。

木形之人,其色必苍。身直五小,五瘦五长,多才劳心,多忧劳事。软弱曲短,一有非良。

火形赤明,小面五锐。反露偏陋,神清主贵。重气轻财,少信多虑。好动心急,最忌不配。

土形之状,黄亮五圆。五实五厚,五短贵全。面圆头大,厚腹股肩。容人有信,行缓心安。

金形洁白,五正五方。五朝五润,偏削败亡。居处静悍,行廉性刚。为吏威肃,兼小无伤。

水形紫润,面肥不平。五肥五嫩,五秀五清。流动摇身,常不敬畏。内欺外恭,粗浊主废。

贵乎相得,最忌相胜。形胜色微,色胜形重。至胜时年,加感则病。年忌七九,犹宜慎恐。

形有强弱,肉有脆坚。强者难犯,弱者易干。肥食少痰,最怕如绵。痩食多火,著骨难全。

以上七段,系按阴阳五行学说,把禀赋不同的各种体形归纳为木、火、土、金、水五种类型的人,以及说明形与色的配合和形与肥瘦强弱的关系。但有些内容,尚需进一步探讨,只供参考。

形气已脱,脉调犹死;形气不足,脉调可医。形盛脉小,少气休治;形衰脉大,多气死期。

本段是讲形色与脉和疾病预后好坏的关系。

颈痛喘疾,目裹肿水,面肿风水,足肿石水。手肿至腕,足肿至踝,面肿至项,阳虚可嗟\footnote{}。

注:

\footnote{}嗟:juē,音撅。叹息,感叹。

本段是讲几种肿病的诊断。

头倾视深,背曲肩随,坐则腰痿,转摇迟回,行则偻俯,立则振掉\footnote{},形神将夺,筋骨虺颓\footnote{}。

注:

\footnote{}掉:diào,音吊。振摇的意思。

\footnote{}虺颓:虺,huī,音会。古书上说的一种毒蛇。虺颓,衰败惫颓的现象,属于病情危重。

本段是讲观察形体外表的有关情状,可知病情预后不良。

\section{切诊}

脉为血府,百体贯通。寸口\footnote{}动脉,大会朝宗\footnote{}。

注:

\footnote{}寸口:在两手桡骨动脉应手处,约长一寸,所以叫寸口,又称脉口或气口。

\footnote{}大会朝宗:寸口属手太阴肺经的动脉,肺主气而朝百脉,所以称大会朝宗。

自本段以下讲的是切诊。本段是讲诊脉要诊寸口的道理。

诊人之脉,高骨\footnote{}上取,因何名关\footnote{},界乎寸尺\footnote{}。

注:

\footnote{}高骨:手掌后面即手腕部有高骨隆起(桡骨茎突〉。

\footnote{}关:在高骨的内侧凹陷中。

\footnote{}界乎寸尺:关部界于寸、尺两部之中间。

寸口脉分寸、关、尺三部。本段是说切寸口脉的方法和关部的位置。

至鱼一寸\footnote{},至泽一尺\footnote{},因此命名,阳寸阴尺\footnote{}。

注:

\footnote{}至鱼一寸,鱼,这里指掌后鱼际处。自高骨至掌后鱼际一寸。

\footnote{}至泽一尺:泽,指尺泽。至尺泽一尺。

⑧阳寸阴尺:寸部候身体上部属阳,尺部候身体下部属阴。

本段是说寸、尺命名的由来和阴阳的区分。

右寸肺胸,左寸心膻。右关脾胃,左肝膈胆。三部三焦,两尺两肾,左小膀胱,右大肠认。

本段是讲左右两手寸、关、尺所候脏腑,右手寸部候肺与胸中;左手寸部候心与膻中;右手关部候脾与胃,左手关部候肝、胆与膈;两手尺部均候肾,左手尺部候小肠、膀胱,右手尺部候大肠。两手寸、关、尺部候上、中、下三焦。

命门\footnote{}属肾,生气之原,人无两尺,必死不痊。

注:

\footnote{}命门:在两肾之间。

本段是讲两尺脉候肾,以及命门的重要性。

关脉一分,右食\footnote{}左风\footnote{}。右为气口,左为人迎\footnote{}。

注:

\footnote{}右食:右关脉候脾胃主食。

\footnote{}左风:左关脉候肝胆主风。

\footnote{}右为气口,左为人迎:根据《灵枢》各篇记载,人迎在颈部的动脉,属足阳明胃经;气口在手高骨侧的动脉,属手太阴肺经。因此“右为气口、左为人迎”的说法欠妥。

本段是讲寸口诊风症与食症,并辨气口、人迎的说法。

脉有七诊\footnote{},曰浮中沉,上竟下竟\footnote{}左右推寻。

注:

\footnote{}七诊:脉有七种诊法,即:浮、中、沉、上、下、左、右。

\footnote{}上竟下竟:竟,从始至终。上竟指寸以上的部位,下竟指尺以下部位。详参后附订正《素问•脉要精微论》原文。

本段是说诊脉中的七诊。

男左大顺,女右大宜\footnote{},男尺恒\footnote{}虚,女尺恒实。

注:

\footnote{}男左大顺,女右大宜:指男子左脉大和女子右脉大均为正常。

\footnote{}恒:经常的。

本段是说男女在脉象上的不同。

又有三部\footnote{},曰天地人。部各有三,九候\footnote{}名焉。额颊耳前,寸口岐锐。下足三阴,肝肾脾胃。

注:

\footnote{}三部:指人体三个切脉部位。上部是头面部称为天;中部是手寸口部,称为人;下部是足部,称为地。

\footnote{}九候:在各三部中,各又分天、地、人而为九候。

本段是讲三部九候。

寸口大会,五十合经\footnote{}。不满其动,无气必凶,更加疏数\footnote{},止还不能。短死岁内,期定难生。

注:

\footnote{}五十合经:诊脉必须诊至五十次搏动以上,在五十动之内,脉动均勻而无间歇为正常人之脉。

\footnote{}疏数,数,shuò,音朔。频的意思。疏数是脉在五十动之内,忽快忽慢不规律,是病危现象。

本段是讲古人诊脉必满五十动的常法。

五脏本脉\footnote{},各有所管。心浮大散,肺浮涩短、肝沉弦长,肾沉滑软。从容而和,脾中迟缓。

注:

\footnote{}本脉:即是平脉,各脏的正常脉。

本段是说五脏各有不属于病脉的本脉。

四时平脉\footnote{},缓而和匀。春弦夏洪,秋毛\footnote{}冬沉。

注:

\footnote{}四时平脉:指春,夏、秋、冬四季的正常脉。

\footnote{}毛:秋季的脉应指轻如毛。

本段是说四时都有不属于病脉的平脉。

太过实强,病生于外。不及虚微,病生于内。

本段是说四时五脏脉有太过和不及的病态。

饮食劳倦,诊在右关,有力为实,无力虚看。

本段是说由于各种因素致病所出现的脉象,以有力、无力辨别虚实。

凡诊病脉,平旦\footnote{}为准,虚静宁神,调息细审。

注:

\footnote{}平且:清晨之时。

本段是说诊脉应在适当时间,并注意平心静气。

一呼一吸,合为一息。脉来四至,平和之则。五至无疴,闰\footnote{}以太息。三至为迟,迟则为冷。六至为数,数则热证。转迟转冷,转数转热。

注:

\footnote{}闰:选里引伸为加的意思。

本段是说一息的正常脉至、和、迟、数脉的至数,以及主病的大体辨别。

迟数既明,浮沉须别。浮沉迟数,辨内外因。外因于天,内因于人。天有阴阳,风雨晦\footnote{}明。人喜忧怒,思悲恐惊。

注:

\footnote{}晦:huì音会。昏暗之意。

本段是讲浮脉候外因和沉脉候内因。

浮沉已辨,滑涩当明。涩为血滞,滑为气壅。

本段是讲滑、涩两脉的主病。

浮脉皮脉\footnote{},沉脉筋骨。肌肉候中,部位统属。

注:

\footnote{}皮脉:轻下手指,脉浮在皮与脉之间。

本段是讲浮、中、沉三种脉候所统属的部位。

浮无力濡,沉无力弱,沉极力牢,浮极力革。

本段是讲浮、沉两脉有力无力,又可分为濡、弱、牢、革四种脉。

三部有力,其名曰实。三部无力,其名曰虚。

本段是讲虚、实两脉。

三部无力,按之且小,似有似无,微脉可考。

本段讲的是微脉。

三部无力,按之且大,涣漫\footnote{}不收,散脉可察。

注:

\footnote{}涣漫,浮散不聚,漫无根蒂。

本段讲的是散脉。

惟中无力,其名曰芤\footnote{}。推筋着骨\footnote{},伏脉可求。

注,

\footnote{}芤,kōu,音相。芤脉,指中间无两边有之脉,有“如按葱管”的感觉。

\footnote{}推筋着骨:诊脉重按至骨。

本段讲的是芤脉、伏脉。

三至为迟,六至为数。四至为缓,七至疾脉。缓止曰结\footnote{},数止曰促®,凡此之诊,皆统至数。动而中止,不能自还,至数不乖\footnote{},代则难痊。

注:

\footnote{}缓止曰结:在一息四至的缓脉中有时而一止称为结脉。

\footnote{}数止曰促:在一息六至的数脉中有时而一止称为促脉。

\footnote{}乖,quai,不和谐,不正常,至数不乖,谓至数不谐调。这里指脉动不整齐。

本段讲的是迟、数、缓、急、结、促、代七脉的至数。

形状如珠,滑溜不定。往来涩滞,涩脉可证。

弦细端直,且劲曰弦。紧比弦粗,劲左右弹。

来盛去衰,洪脉名显。大则宽阔,小则细减。

如豆乱动,不移约约\footnote{}。长则迢迢\footnote{},短则缩缩。

注:

\footnote{}不移约约:动脉是在一个位置不移的,

\footnote{}迢:tiáo ,音条。长远之意。这里引伸为脉长之意。

本段讲的是滑、涩、弦、紧、洪、大、小、动、长,短十脉。

浮阳主表,风淫六气,有力表实,无力表虚。浮迟表冷,浮缓风湿。浮濡伤暑,浮散虚极。浮洪阳盛,浮大阳实。浮细气少,浮涩血虚。浮数风热,浮紧风寒。浮弦风饮,浮滑风痰。

本段是讲浮脉的兼脉和所主病证。

沉阴主里,七情气食。沉大里实,沉小里虚。沉迟里冷,沉缓里湿。沉紧冷痛,沉数热极。沉涩痹气,沉滑痰食。沉伏闭郁,沉弦饮疾。

本段是讲沆脉的兼脉和所主病证。

濡阳虚病,弱阴虚疾,微主诸虚,散为虚剧。革伤精血\footnote{},半产带崩\footnote{},牢疝(疒征)瘕,心腹寒疼。

注:

\footnote{}革伤精血:革脉主男子亡血,伤精之病。

\footnote{}半产带崩:革脉主女子半产、带下、血崩之病。

本段是讲濡、弱、微、散、革、牢六脉所主病证。

虚主诸虚,实主诸实,芤主失血,随见可知。

迟寒主脏,阴冷相干,有力寒痛,无力虚寒。

数热主腑,数细阴伤,有力实热,无力虚疮。

缓湿脾胃,坚大湿壅。促为阳郁\footnote{}结为阴凝\footnote{}。

注:

\footnote{}促为阳郁:促为阳盛而郁之脉。

\footnote{}结为阴凝:结为阴盛而凝之脉。

本段是讲虚、实、芤、迟、数和数脉兼细、缓、促、结八脉所主的病证。

代则气乏,跌打闷绝,夺气痛疮,女胎三月。

滑司痰病,关主食风。寸候吐逆\footnote{},尺便血脓\footnote{}。

注:

\footnote{}寸候吐逆:寸候上焦,寸脉滑主吐逆。

\footnote{}尺便血脓:尺候下焦,尺脉滑主便下血脓。

本段是讲代、滑二脉所主的病证。

涩虚湿痹,尺精血伤,寸汗津竭,关膈液亡。

弦关主饮,木侮脾经\footnote{}。寸弦头痛,尺弦腹疼。

注:

\footnote{}木侮脾经:肝木克脾土。

本段是讲涩、弦二脉所主的病证。

紧主寒痛,洪是火伤,动主痛热,崩汗惊狂。

长则气治,短则气病,细则气衰,大则病进。

本段是讲紧、洪、动、长、短、细、大七脉所主的病证。

脉之主病,有宜不宜,阴阳顺逆,吉凶可推。

本段是讲各种病证都有一定的脉象,有相宜属顺,不相宜属逆之别。

中风之脉,却喜浮迟,坚大急疾\footnote{},其凶可知。

注:

\footnote{}坚大急疾:中风脉见坚、大、急、疾,是脉证不合,预后不佳。

本段是讲中风脉象的顺逆。

伤寒热病,脉喜浮洪,沉微涩小,证反必凶,汗后脉静,身凉则安,汗后脉躁,热甚必难。阳证见阴\footnote{},命必危殆\footnote{},阴证见阳,虽困无害。

注:

\footnote{}阳证见阴:阳证出现阴脉则病危。

\footnote{}殆:dài音怠。危险。

本段是讲伤寒脉象的顺逆。

劳倦伤脾,脉当虚弱,自汗脉躁,死不可却。

本段是讲内伤劳倦脉象的顺逆。

疟脉自弦,弦迟多寒,弦数多热,代散则难。

本段是讲疟疾脉象的顺逆。

泄泻下利,沉小滑弱,实大浮数,发热则恶。

本段是讲泄泻脉象的顺逆。

呕吐反胃,浮滑者昌。沉数细涩,结肠者亡\footnote{}。

注:

\footnote{}结肠者亡:呕吐脉见沉、数、细、涩,乃气血少津液枯而见结肠之证。

本段是讲呕吐脉象的顺逆。

霍乱之候,脉代勿讶,舌卷囊缩,厥伏可嗟。

本段是讲霍乱脉象的顺逆。

嗽脉多浮,浮濡易治,沉伏而紧,死期将至。

本段是讲嗽咳脉象的顺逆。

喘息抬肩,浮滑是顺,沉涩肢寒,切为逆证。

本段是讲喘症脉象的顺逆。

火热之证,洪数为宜,微弱无神,根本脱离。

本段是讲火症脉象的顺逆。

骨蒸发热,脉数而虚,热而涩小,必殒\footnote{}其躯。

注:

\footnote{}殒:yǔn音允。死亡。

本段是讲骨蒸脉象的顺逆。

劳极诸虚,浮软微弱,土败双弦,火炎细数。

本段是讲虚劳脉象的顺逆。

失血诸证,脉必见芤,缓小可喜,数大堪忧。

本段是讲失血诸证脉象的顺逆。

蓄血在中,牢大却宜,沉涩而微,速愈者稀。

本段是讲蓄血症脉象的顺逆。

三消\footnote{}之脉,数大者生,细微短涩,应手堪惊。

注:

\footnote{}三消:上,中、下三消,又称消渴病。

本段是讲三消症脉象的顺逆。

小便淋闭,鼻色必黄,实大可疗,涩小知亡。

本段是讲小便淋闭症脉象的顺逆。

癫乃重阴\footnote{},狂乃重阳\footnote{},浮洪吉象,沉急凶殃。

注:

\footnote{}重阴:癫证以静为主故曰阴。

\footnote{}重阳:狂证以动为主故曰阳。

本段是讲癫狂症脉象的顺逆。

痫宜浮缓,沉小急实,但弦无胃,必死不失。

本段是讲痫症脉象的顺逆。

心腹之痛,其类有九\footnote{},细迟速愈,浮大延久。

注:

\footnote{}其类有九:古人将心(腹)痛分为虫痛、注痛、悸痛、食痛、饮痛、冷痛、热痛、风痛、去来痛(一说无风痛、去来痛,有气痛、血痛)九种。

本段是讲心腹痛脉象的顺逆。

疝属肝病,脉必弦急,牢急者生,弱急者死。

本段是讲疝症脉象的顺逆。

黄疸湿热,洪数便宜,不妨浮大,微涩难医。

本段是讲黄疸脉象的顺逆。

肿胀之脉,浮大洪实,细而沉微,岐黄\footnote{}无术。

注:

\footnote{}歧黄:歧,指歧伯,古代名医。黄,指黄帝,即轩辕黄帝。

本段是讲肿胀病脉象的顺逆。

五脏为积,六腑为聚,实强可生,沉细难愈。

本段是讲积聚症脉象的顺逆。

中恶\footnote{}腹胀,紧细乃生,浮大为何?邪气已深。

注:

\footnote{}中恶:中不正之气。

本段是讲中恶症脉象的顺逆。

鬼祟之脉\footnote{},左右不齐,乍大乍小、乍数乍迟。

注:

\footnote{}鬼祟之脉,是指感受四时不正之邪气的脉象。

本段是讲感受四时不正的邪气所出现的脉象。

痈疽未溃,洪大脉宜,及其已溃,洪大最忌。

本段是讲痈疽脉象的顺逆。

肺痈已成,寸数而实。肺痿之证,数而无力。痈痿色白,脉宜短涩,数大相逢,气损血失。肠痈实热,滑数相宜,沉细无根,其死可期。

本段是讲肺痈、肺痿、肠痈脉象的顺逆。

妇人有子,阴搏阳别\footnote{},少阴动甚\footnote{},其胎已结。滑疾而散,胎必三月,按之不散,五月可别。左男右女,孕乳是主。女腹如箕\footnote{},男腹如釜\footnote{}。

注:

\footnote{}阴搏阳别:阴指尺脉,阳指寸脉,尺脉搏动应指,异于寸脉。

\footnote{}少阴动甚:手少阴指心脉,左寸脉动,这是结胎的反应。

\footnote{}女腹如箕:女胎腹形如簸箕成圆形。

\footnote{}男腹如釜:釜,古时无脚的锅。男胎腹形如釜之上小而下大。

\footnote{}\footnote{}二说无考价值,只是一种古说。

本段是讲孕妇的脉象。

欲产离经,新产小缓,实弦牢大,其凶不免。

本段是讲妇女临产和产后的脉象。

经脉病脉,业已昭详,将绝之形,更当度量。

本段是说下面要讲死绝之脉。

心绝之脉,如操带钩,转豆\footnote{}躁疾,一日可忧。

注:

\footnote{}转豆,脉来象一连串豆子,颗颗累累的感觉。

本段是讲心绝脉。

肝绝之脉,循刀责责\footnote{},新张弓弦,死在八日。

注:

\footnote{}循刀责责:如按在刀刃上,细硬毫无从容、生气。

本段是讲肝绝脉。

脾绝雀啄\footnote{},又同屋漏,复杯水流,四日无救。

注:

\footnote{}啄:zhuó,音镯。鸟吃东西叫啄。雀啄,鸟雀啄食状。

本段是讲脾绝脉,

肺绝维何?如风吹毛,毛羽中肤\footnote{},三日而号。

注:

\footnote{}毛羽中肤:形容鸟的羽毛碰到人的皮肤那样,轻浮无根。

本段是讲肺绝脉。

肾绝伊何?发如夺索\footnote{},辟辟弹石\footnote{},四日而作。

注:

\footnote{}夺索,按脉好像摸在长而散乱无序的绳索上。

\footnote{}弹石:如摸在弹石上很硬很沉。

本段是讲肾绝脉。

命脉将绝,鱼翔\footnote{}虾游\footnote{}。至如涌泉,莫可挽留。

注:

\footnote{}鱼翔:是怪脉的一种,如同鱼在水中游翔的状态。

\footnote{}虾游:也是怪脉之一,好像虾游一样,静而不动,忽而一跃即去。

本段是讲鱼翔、虾游等败脉。

脉有反关,动在臂后,别由列缺,不干证候。

本段是讲反关脉的部位和性质。

\section{附:订正《素问•脉要精微论》一则备考}

尺内两傍,则季胁也,尺外以候肾,尺里以候腹中。中附上\footnote{},左外以候肝,内以候膈;右外以候胃,内以候脾。上附上\footnote{},右外以候肺,内以候胸中;左外以候心,内以候膻中。前以候前\footnote{},后以候后\footnote{}。上竟上者⑤,胸喉中事也;下竟下者⑥,少腹、腰、股、膝、胫、足中事也。

注:

\footnote{}中附上:指关部左右之内外。

\footnote{}上附上:指寸部左右之内外。

\footnote{}前以候前:指关前即寸部。

\footnote{}后以候后:指关后即尺部。

⑤上竟上者:指脉来上尽于鱼际。

⑥下竟下者:指脉来下尽于尺泽。

这是《素问•脉要精微论》所讲的脉的部位配候脏腑和人体部位的原文。

电子版注:

1、问诊:原文错写"夜剧而热,阴下陷阴",应为"夜剧而热,阳下陷阴",已改

2、问诊:原文错写"尽寒夜热,阴阳交错",应为"昼寒夜热,阴阳交错",已改

3、问诊:原文为“喉为生路,会厌\footnote{}门户”,应为“喉为声路,会厌\footnote{}门户”,已改

4、原文的“藏主五脏,上华面颐”,应为“藏于五脏,上华面颐”,已改

\chapter{药性歌括四百味}

人参味甘,大补元气,止渴生津,调荣养卫。

黄芪性温,收汗固表,托疮生肌,气虚莫少。

白术甘温,健脾强胃,止泻除湿,兼祛痰痞。

茯苓味淡,渗湿利窍,白\footnote{}化痰涎,赤\footnote{}通水道。

甘草甘温,调和诸药,炙\footnote{}则温中,生\footnote{}则泻火。

当归甘温,生血补心,扶虚益损,逐瘀生新。

白芍酸寒,能收能补,泻痢腹痛,虚寒勿与。

赤芍酸寒,能泻能散,破血通经,产后勿犯⑤。

注:

\footnote{}白:指白茯苓。

\footnote{}赤:指赤茯苓。

\footnote{}炙:指炙甘草。

\footnote{}生:指生甘草。

⑤勿犯:即勿用。

生地微寒,能消温热,骨蒸烦劳,养阴凉血。

熟地微温,滋肾补血,益髓填精,乌须黑发。

麦门甘寒,解渴祛烦,补心清肺,虚热自安。

天门甘寒,肺痿肺痈,消痰止嗽,喘热有功。

黄连味苦,泻心除痞,清热明眸\footnote{},厚肠止痢。

黄芩苦寒,枯\footnote{}泻肺火,子\footnote{}清大肠,湿热皆可。

黄柏苦寒,降火滋阴,骨蒸湿热,下血堪任。

栀子性寒,解郁除烦,吐衄胃痛,火降小便。

注:

\footnote{}眸:mou,音牟。眼中瞳仁。

\footnote{}枯:指枯黄芩。

\footnote{}子:指子黄芩或称条芩。

连翘苦寒,能消痈毒,气聚血凝,温热堪逐。

石膏大寒,能泻胃火,发渴头疼,解肌立妥。

滑石沉寒,滑能利窍,解渴除烦,湿热可疗。

贝母微寒,止嗽化痰,肺痈肺痿,开郁除烦。

大黄苦寒,实热积聚,蠲\footnote{}痰逐水,疏通便闭。

柴胡味苦,能泻肝火,寒热往来,疟疾均可。

前胡微寒,宁嗽化痰,寒热头痛,痞闷能安。

升麻性寒,清胃解毒,升提下陷,牙痛可逐。

注:

\footnote{}蠲:juan音捐。免除、除去。

桔梗味苦,疗咽肿痛,载药上升,开胸利壅。

紫苏味辛,风寒发表,梗下诸气、消除胀满。

麻黄味辛,解表出汗,平喘消肿,风寒发散。

葛根味甘,祛风发散,温疟往来,止渴解酒。

薄荷味辛,最清头目,祛风散热,骨蒸宜服。

防风甘温,能除头晕,骨节痹疼,诸风口噤。

荆介味辛,能清头目,表汗祛风,治疮消瘀。

细辛辛温,少阴头痛,利窍通关,风湿皆用。

羌活微温,祛风除湿,身头疼痛,舒筋活络。

独活辛苦,颈项难舒,两足湿痹,诸风能除。

知母味苦,热渴能除,骨蒸有汗,痰咳皆舒。

白芷辛温,阳明头痛,风热搔痒,排脓通用。

藁本气温,除头巅顶,寒湿可祛,风邪可屏。

香附味甘,快气开郁,止痛调经,更消宿食。

乌药辛温,心腹胀痛,小便滑数,顺气通用。

枳实味苦,消食除痞,破积化痰,冲墙倒璧\footnote{}。

注:

\footnote{}冲墙倒壁,形容破气之力很强。

枳壳微寒,快气宽肠,胸中气结,胀满堪尝。

白蔻辛温,能祛瘴翳,温中行气,止呕和胃。

青皮苦温,能攻气滞,削坚平肝,安胃下食。

陈皮辛温,顺气宽膈,留白和胃,消痰去白。

苍术苦温,健脾燥湿,发汗宽中,更祛瘴疫。

厚朴苦温,消胀泄满,痰气泻痢,其功不缓。

南星性热,能治风痰,破伤\footnote{}强直,风搐自安。

半夏味辛,健脾燥湿,痰厥头痛,嗽呕堪入。

注:

\footnote{}破伤:指破伤风。

藿香辛温,能止呕吐,发散风寒,霍乱为主。

槟榔辛温,破气杀虫,祛痰逐水,专除后重\footnote{}。

腹皮微温,能下膈气,安胃健脾,浮肿消去。

香薷味辛,伤暑便涩,霍乱水肿,除烦解热。

扁豆微温,转筋吐泻,下气和中,酒毒能化。

猪苓味淡,利水通淋,消肿除湿,多服损肾。

泽泻甘寒,消肿止渴,除湿通淋,阴汗自遏\footnote{}。

木通性寒,小肠热闭,利窍通经,最能导滞。

注:

\footnote{}后重:指里急后重。

\footnote{}遏:e,音饿。阻止的意思。

车前子寒,溺涩眼赤,小便能通,大便能实。

地骨皮寒,解肌退热,有汗骨蒸,强阴凉血。

木瓜味酸,湿肿脚气,霍乱转筋,足膝无力。

威灵苦温,腰膝冷痛,消痰痃癖\footnote{},风湿皆用。

牡丹苦寒,破血通经,血分有热,无汗骨蒸。

玄参苦寒,清无根火,消肿骨蒸,补肾亦可。

沙参味甘,消肿排脓,补肝益肺,退热除风。

丹参味苦,破积调经,生新去恶,祛除带崩。

注:

\footnote{}痃癖:痃,xuan,音玄。痃癖是脐腹或胁肋部患有癖块的泛称。

苦参味苦,痈肺疮疥,下血肠风,眉脱赤癞\footnote{}。

龙胆苦寒,疗眼赤疼,下焦湿肺,肝经热烦。

五加皮温,祛痛风痹,健步坚筋,益精止沥。

防己气寒,风湿脚痛,热积膀胱,消痈散肺。

地榆沉寒,血热堪用,血痢带崩,金疮\footnote{}止痛。

茯神补心,善镇惊悸,恍惚健忘,善除怒恚\footnote{}。

远志气温,能驱惊悸,安神镇心,令人多记。

酸枣味酸,敛汗驱烦,多眠用生\footnote{},不眠用炒®。

注:

\footnote{}癞:lai,音赖。即麻风病。

\footnote{}金疮:即刀伤斧砍的伤口。

\footnote{}恚:hui,音卉。恨、怒的意思。

\footnote{}生:指生酸枣仁。

⑤炒:指炒酸枣仁。

菖蒲性温,开心利窍,去痹除风,出声至妙。

柏子味甘,补心益气,敛汗润肠,更疗惊悸。

益智辛温,安神益气,遗溺遗精,呕逆皆治。

甘松味香,善除恶气,治体香肌,心腹痛已。

小茴性温,能除疝气,腹痛腰疼,调中暖胃。

大茴味辛,疝气脚气,肿痛膀胱,止呕开胃。

干姜味辛,表解风寒,炮\footnote{}苦逐冷,虚寒尤堪。

附子辛热,性走不守,四肢厥冷,回阳功有。

注:

\footnote{}炮:指炮姜。

川乌大热,搜风入骨,湿痹寒疼,破积之物。

木香微温,散滞和胃,诸风能调,行肝泻肺。

沉香降气,暖胃追邪,通天彻地,气逆为佳。

丁香辛热,能除寒呕,心腹疼痛,温胃可晓。

砂仁性温,养胃进食,止痛安胎,行气破滞。

荜澄茄辛,除胀化食,消痰止哕,能逐寒气。

肉桂辛热,善通血脉,腹痛虚寒,温补可得。

桂枝小梗,横行手臂,止汗舒筋,治手足痹。

吴萸辛热,能调疝气,脐腹寒疼,酸水能治。

延胡气温,心腹卒痛,通经活血,跌扑血崩。

薏苡味甘,专除湿痹,筋节拘挛,肺痈肺痿。

肉蔻辛温,脾胃虚冷,泻痢不休,功可立等。

草蔻辛温,治寒犯胃,作痛呕吐,不食能食。

诃子味苦,涩肠止痢,痰嗽喘急,降火敛肺。

草果味辛,消食除胀,截疟逐痰,解瘟辟瘴。

常山苦寒,截疟除痰,解伤寒热,水胀能宽。

良姜性热,下气温中,转筋霍乱,酒食能攻。

山楂味甘,磨消肉食,疗疝催疮,消膨健胃。

神曲味甘,开胃进食,破结逐痰,调中下气。

麦芽甘温,能消宿食,心腹膨胀,行血散滞。

苏子味辛,驱痰降气,止咳定喘,更润心肺。

白芥子辛,专化胁痰,疟蒸痞块,服之能安。

甘遂苦寒,破癥消痰,面浮蛊胀,利水能安。

大戟甘寒,消水利便,腹胀癥坚,其功暝眩\footnote{}。

注:

\footnote{}其功瞑眩:是说服大戟后病人可能会发生一时性的瞑眩(眩晕),这是药物对证,将蠲除疾患的反映。所谓:“药不瞑眩、厥疾弗瘳。”即是此意。

芫花寒苦,能消胀蛊,利水泻湿,止咳痰吐。

商陆苦寒,赤白各异,赤者消风,白利水气。

海藻咸寒,消癭散疬,除胀破癥,利水通闭。

牵牛苦寒,利水消肿,蛊胀痃癖,散滞除壅。

葶苈辛苦,利水消肿,痰咳癥瘕,治喘肺痈。

瞿麦苦寒,专治淋病,且能堕胎,通经立应。

三棱味苦,利血消癖,气滞作痛,虚者当忌。

五灵味甘,血滞腹痛,止血用炒,行血用生。

莪术温苦,善破痃癖,止痛消瘀,通经最宜。

干漆辛温,通经破瘕,追积杀虫,效如奔马。

蒲黄味甘,逐瘀止崩,止血须炒,破血用生。

苏木甘咸,能行积血,产后血经,兼医扑跌。

桃仁甘平,能润大肠,通经破瘀,血瘕堪尝。

姜黄味辛,消痈破血,心腹结痛,下气最捷,

郁金味苦,破血行气,血淋溺血,郁结能舒。

金银花甘,疗痈无对\footnote{},未成则散,已成则溃。

注:

\footnote{}无对:是说没有胜过的。

漏芦性寒,祛恶疮毒,补血排脓,生肌长肉。

蒺藜味苦,疗疮搔痒,白癜头疮,翳除目朗。

白芨味苦,功专收敛,肿毒疮疡,外科最善。

蛇床辛苦,下气温中,恶疮疥癞,逐瘀祛风。

天麻味甘,能驱头眩,小儿惊痫,拘挛瘫痪。

白附辛温,治面百病,血痹风疮,中风痰症。

全蝎味辛,却风痰毒,口眼㖞斜,风痫发搐。

蝉蜕甘寒,消风定惊,杀疳除热,退翳侵睛。

僵蚕味咸,诸风惊痫,湿痰喉痹,疮毒瘢痕。

蜈蚣味辛,蛇虺\footnote{}恶毒,镇惊止痉,堕胎逐瘀。

木鳖甘寒,能追疮毒,乳痈腰疼,消肿最速。

蜂房咸苦,惊痫瘛疭,\footnote{}牙疼肿毒,瘰疬乳痈。

花蛇温毒,瘫痪喁斜,大风疥癞,诸毒称佳。

蛇蛻咸平,能除翳膜,肠痔蛊毒,惊痫搐搦。

槐花味苦,痔漏肠风,大肠热痢,更杀蛔虫。

鼠粘子辛,能除疮毒,瘾疹风热,咽疼可逐。

注:

\footnote{}虺:hui,音悔。古书记载的一种毒蛇。蜈蚣能解其毒。

\footnote{}瘛疭:qi音气综。即搐搦、抽风。

茵陈味苦,退疸除黄,泻湿利水,清热为凉。

红花辛温,最消瘀热,多则通经,少则养血。

蔓荆子苦,头疼能医,拘挛湿痹,泪眼堪除。

兜铃苦寒,能熏痔漏,定喘消痰,肺热久嗽。

百合味甘,安心定胆,止嗽消浮,痈疽可啖。

秦艽微寒,除湿荣筋,肢节风痛,下血骨蒸。

紫菀苦辛,痰喘咳通,肺痈吐脓,寒热并济。

款花甘温,理肺消痰,肺痈喘咳,补劳除烦。

金沸草温,消痰止嗽,明目祛风,逐水尤妙。

桑皮甘辛,止嗽定喘,泻肺火邪,其功不浅。

杏仁温苦,风寒喘嗽,大肠气闭,便难切要。

乌梅酸温,收敛肺气,止渴生津,能安泻痢。

天花粉寒,止渴祛烦,排脓消毒。善除痰热。

瓜蒌仁寒,宁嗽化痰,伤寒结胸,解渴止烦。

密蒙花甘,主能明目,虚翳青盲,服之效速。

菊花味甘,除热祛风,头晕目赤,收泪殊功。

木贼味甘,祛风退翳,能止月经,更消积聚。

决明子甘,能祛肝热,目疼收泪,仍止鼻血。

犀角酸寒,化毒辟邪,解热止血,消肿毒蛇。

羚羊角寒,明目清肝,祛惊解毒,神志能安。

龟甲味甘,滋阴补肾,止血续筋,更医颅囟。

鳖甲咸平,劳嗽骨蒸,散瘀消肿,去痞除症。

桑上寄生,风湿腰痛,止漏安胎,疮疡亦用。

火麻味甘,下乳催生,润肠通结,小水能行。

山豆根苦,疗咽肿痛,敷蛇虫伤,可救急用。

益母草苦,女科为主,产后胎前,生新去瘀。

紫草咸寒,能通九窍,利水消膨,痘疹最要。

紫葳味酸,调经止痛,崩中带下,癥瘕通用。

地肤子寒,去膀胱热,皮肤搔痒,除热甚捷。

楝根性寒,能追诸虫,疼痛立止,积聚立通。

樗\footnote{}根味苦,泻痢带崩,肠风痔漏,燥湿涩精。

泽兰甘苦,痈肿能消,打仆伤损,肢体虚浮。

牙皂味辛,通关利窍,敷肿痛消,吐风痰妙。

注:

\footnote{}樗:chu,音初。即臭椿树。

芜荑味辛,驱邪杀虫,痔瘘癣疥,化食除风。

雷丸味苦,善杀诸虫,癫痫蛊毒,治儿有功。

胡麻仁甘,疔肿恶疮,熟补虚损,筋壮力强。

苍耳子苦,疥癣细疮,驱风湿痹,搔痒堪尝。

蕤仁味甘,风肿烂弦,热胀胬肉,眼泪立痊。

青箱子苦,肝脏热毒,暴发赤障,青盲可服。

谷精草辛,牙齿风痛,口疮咽痹,眼翳通用。

白薇大寒,疗风治疟,人事不知,昏厥堪却。

白蔹微寒,儿疟惊痫,女阴肿痛,痈疔可啖\footnote{}。

注:

\footnote{}啖:dan,音淡。吃,或给人吃的意思。

青蒿气寒,童便熬膏,虚热盗汗,除骨蒸劳。

茅根味甘,通关逐瘀,止吐衄血,客热可去。

大小蓟苦,消肿破血,吐衄咯唾,崩漏可啜。

枇杷叶苦,偏理肺脏,吐哕不止,解酒清上。

射干味苦,逐瘀通经,喉痹口臭,痈毒堪凭。

鬼箭羽苦,通经堕胎,杀虫祛结,驱邪除乖。

夏枯草苦,瘰疬瘿瘤,破癥散结,湿痹能瘳。

卷柏味辛,症瘕血闭,风眩痿璧,更驱鬼疰®。

注:

\footnote{}疰:zhù ,音注。这里指具有传染和病程长的慢性病,主要指劳瘵,亦称鬼疰(参劳瘵注)。

马鞭味苦,破血通经,症瘕痞块,服之最灵。

鹤虱味苦,杀虫追毒,心腹卒痛,蛔虫堪逐。

白头翁寒,散癥逐血,癭疬疟疝,止痛百节。

旱莲草甘,生须黑发,赤痢堪止,血流可截。

慈菰辛苦,疗肿痈疽,恶疮瘾疹,蛇虺并施。

榆皮味甘,通水除淋,能利关节,敷肿痛定。

钩藤微寒,疗儿惊痫,手足瘛疭,抽搐口眼。

豨莶草苦,追风除湿,聪耳明目,乌须黑发。

辛夷味辛,鼻塞流涕,香臭不闻,通窍之剂。

续随子辛,恶疮蛊毒,通经消积,不可过服。

海桐皮苦,霍乱久痢,疳䘌\footnote{}疥癖,牙疼亦治。

石楠味辛,肾衰脚弱,风淫湿痹,堪为妙药。

大青气寒,伤寒热毒,黄汗黄疸,时疫宜服。

侧柏叶苦,吐衄崩痢,能生须眉,除湿之剂。

槐实味苦,阴疮湿痒,五痔肿痛,止血极莽。

瓦楞子咸,妇人血块,男人痰癖,癥瘕可瘥。

棕榈子苦,禁泄涩痢,带下崩中,肠风堪治。

注:

\footnote{}䘌:ni,音匿。妇人阴户生疮,称蜃疮。

冬葵子寒,滑胎易产,癃利小便,善通乳难。

淫羊藿辛,阴起阳兴,坚筋益骨,志强力增。

松脂味甘,滋阴补阳,驱风安脏,膏可贴疮。

覆盆子甘,肾损精竭,黑须胡眸,补虚续绝。

合欢味甘,利人心志,安脏明目,快乐无虑。

金樱子涩,梦遗精滑,禁止遗尿,寸白杀虫。

楮实味甘,壮筋明目,益气补虚,阳痿当服。

郁李仁酸,破血润燥,消肿利便,关格\footnote{}通辱。

注:

\footnote{}关格:指小便不通与呕吐不止并见的病证。小便不通名关,呕吐不止名格。

密陀僧咸,止痢医痔,能除白癜,诸疮可治。

伏龙肝温,治疫安胎,吐血咳逆,心烦妙哉。

石灰味辛,性烈有毒,辟虫立死,堕胎甚速。

穿山甲毒,痔癖恶疮,吹奶肿痛,通络散风,

蚯蚓气寒,伤寒温病,大热狂言,投之立应。

蟾蜍气凉,杀疳蚀癖,瘟疫能辟\footnote{},疮毒可祛。

刺猥皮苦,主医五痔,阴肿疝痛,能开胃气。

蛤蚧味咸,肺痿血咯,传尸劳疰,服之可知。

注:

\footnote{}辟:pi,音譬。排除的意思。

蝼蛄味咸,治十水肿,上下左右,效不旋踵。

桑螵蛸咸,淋浊精泄,除疝腰疼,虚损莫失。

田螺性冷,利大小便,消肿除热,醒酒立见。

水蛭味咸,除积瘀坚,通经堕产,折伤可痊。

贝子味咸,解肌散结,利水消肿,目翳清洁。

海螵蛸咸,漏下赤白,癥瘕疝气,阴肿可得。

青檬石寒,硝煅金色\footnote{},坠痰消食,疗效莫测。

磁石味咸,专杀铁毒,若误吞针,系线即出。

注:

\footnote{}硝煅金色:是说礞石有青礞石,煅礞石之分,青礞石用燄硝煅过,因有金星光泽,称金礞石。

花蕊石寒,善止诸血,金疮血流,产后血涌。

代赭石寒,下胎崩带,儿疳泻痢,惊痫呕嗳。

黑铅味甘,止呕反胃,瘰疬外敷,安神定志。

狗脊味甘,酒蒸入剂,腰背膝疼,风寒湿痹。

骨碎补温,折伤骨节,风血积疼,最能破血。

茜草味苦,便衄吐血,经带崩漏,损伤虚热。

王不留行,调经催产,除风痹痛,乳痈当啖。

狼毒味辛,破积瘕症,恶疮鼠瘘,止心腹疼。

藜芦味辛,最能发吐,肠避泻痢,杀虫消蛊。

蓖麻子辛,吸出滞物,涂顶肠收,涂足胎出。

荜拨味辛,温中下气,痃癖阴疝,霍乱泻痢。

百部味甘,骨蒸劳瘵\footnote{},杀瘠蛔虫,久嗽功大。

京墨味辛,吐衄下血,产后崩中,止血甚捷。

女贞子苦,黑发乌须,强筋壮力,去风补虚。

瓜蒂苦寒,善能吐痰,消身肿胀,并治黄疸。

粟壳性涩,泄痢嗽怯,劫病如神,杀人如剑。

注:

\footnote{}劳瘵:瘵,zhai,音债。劳瘵,即肺痨,也称传尸劳、尸注、鬼注等。

巴豆辛热,除胃寒积,破癥消痰,大能通利。

夜明砂粪,能下死胎,小儿无辜,瘰疬堪裁。

斑蝥有毒,破血通经,诸疮瘰疬,水道能行。

蚕砂性温,湿痹瘾疹,瘫风肠鸣,消渴可饮。

胡黄连苦,治劳骨蒸,小儿疳痢,盗汗虚惊。

使君日温,消疳消浊,泻痢诸虫,总能除却。

赤石脂温,保固肠胃,溃疡生肌,涩精泻痢。

青黛咸寒,能平肝木,惊痫疳痢,兼除热毒。

阿胶甘平,止咳脓血,吐血胎崩,虚羸可啜。

白矶味酸,化痰解毒,治症多能,难以尽述。

五倍苦酸,疗齿疳䘌,痔痈疮脓,兼除风热。

玄明粉辛,能蠲宿垢,化积消痰,诸热可瘳。

通草味甘,善治膀胱,消痈散肿,能医乳房。

枸杞甘平,添精补髓、明目祛风,阴兴阳起。

黄精味甘,能安脏腑,五劳七伤,此药大补。

何首乌甘。添精种子,黑发悦颜,强身延纪\footnote{}。

注:

\footnote{}延纪:纪,纪年。延纪,即延年益寿。

五味酸温,生津止渴,久嗽虚劳,肺肾枯竭。

山茱性温,涩精益髓,肾虚耳鸣,腰膝痛止。

石斛味甘,却惊定志,壮骨补虚,善驱冷痹。

破故纸温,腰膝酸痛,兴阳固精,盐酒炒用。

薯蓣甘温,理脾止泻,益肾补中,诸虚可治。

苁蓉味甘,峻补精血,若骤用之,更动便滑。

菟丝甘平,梦遗滑精,腰痛膝冷,添髓壮筋。

牛膝味苦,除湿痹痿,腰膝酸痛,小便淋沥。

巴戟辛甘,大补虚损,精滑梦遗,强筋固本。

仙茅味辛,腰足挛痹,虚损劳伤,阳道兴起。

牡蛎微寒,涩精止汗,崩带胁痛,老痰祛散。

楝子苦寒,膀胱疝气,中湿伤寒,利水之剂。

萆薢甘苦,风寒湿痹,腰背冷痛,添精益气。

续断味辛,接骨续筋,跌扑折损,且固遗精。

龙骨味甘,梦遗精泄,崩带肠痈,惊痫风热。

人之头发,补阴甚捷,吐衄血晕,风惊痫热。

鹿茸甘温,益气补阳,泄精尿血,崩带堪尝。

鹿角胶温,吐衄虚贏,跌扑伤损,崩带安胎。

腽肭脐热,补益元阳,固精起痿,痃癖劳伤。

紫河车甘,疗诸虚损,劳瘵骨蒸,滋培根本。

枫香味辛,外科要药,搔疹瘾疹,齿痛亦可。

檀香味辛,升胃进食,霍乱腹痛,中恶秽气。

安息香辛,驱除秽恶,开窍通关,死胎能落。

苏合香甘,祛痰辟秽,蛊毒痫痓,梦魔能去。

熊胆味苦,热蒸黄疸,恶疮虫痔,五疳惊痫。

硇砂有毒,溃痈烂肉,除翳生肌,破症消毒。

硼砂味辛,疗喉肿痛,膈上热痰,噙化立中。

朱砂味甘,镇心养神,祛邪解毒,定魄安魂。

硫黄性热,扫除疥疮,壮阳逐冷,寒邪敢当。

龙脑味辛,目痛窍闭,狂躁妄语,真为良剂。

芦荟气寒,杀虫消疳,癲痫惊搐,服之立安。

天竺黄甘,急慢惊风,镇心解热,化痰有功。

麝香辛温,善通关窍,辟秽安惊,解毒其妙。

乳香辛苦,疗诸恶疮,生肌止痛,心腹尤良。

没药苦平,治疮止痛,跌打损伤,破血通用。

阿魏性温,除症破结,止痛杀虫,传尸可灭。

水银性寒,治疥杀虫,断绝胎孕,催生立通。

轻粉性燥,外科要药,杨梅诸疮,杀虫可托。

砒霜大毒,风痰可吐,截疟除哮,能消沉痼。

雄黄苦辛,辟邪解毒,更治蛇虺,喉风瘜肉。

珍珠气寒,镇惊除痫,开聋磨翳,止渴坠痰。

牛黄味苦,大治风痰,定魄安魂,惊痫灵丹。

琥珀味甘,安魂定魄,破瘀消症,利水通涩。

血竭味咸,跌扑伤损,恶毒疮痈,破血有准。

石钟乳甘,气乃慓悍,益气固精,治目昏暗。

阳起石甘,肾气乏绝,阳痿不起,其效甚捷。

桑椹子甘,解金石燥,清除热渴,染须发皓\footnote{}。

蒲公英苦,溃坚消肿,结核能除,食毒堪用。

注:

\footnote{}皓: hao, 音浩。洁白的意思。

石韦味苦,通利膀胱,遗尿或淋,发背疮疡。

萹蓄味苦,疥搔疽痔,小儿蛔虫,女人阴蚀。

鸡内金寒,溺遗精泄,禁痢漏崩,更除烦热。

鲤鱼味甘,消水肿满,下气安治,其功不缓。

芡实味甘,能益精气,腰膝酸疼,皆主湿痹。

石莲子苦,疗噤口痢,白浊遗精,清心良剂。

藕味甘寒,解酒清热,消烦逐瘀,止吐衄血。

龙眼味甘,归脾益智,健忘怔忡,聪明广记\footnote{}。

注:

\footnote{}广记:这里指增强记忆力。

莲须味甘,益肾乌须,涩精固髓,悦颜补虚。

石榴皮酸,能禁精漏,止痢涩肠,染须尤妙。

陈仓谷米,调和脾胃,解渴除烦,能止泻痢。

莱菔子辛,喘咳下气,倒壁冲墙,胀满消去。

砂糖味甘,润肺利中,多食损齿,湿热生虫。

饴糖味甘,和脾润肺,止渴消痰,中满休食。

麻油性冷,善解诸毒,百病能治,功难悉述。

白果甘苦,喘嗽白浊,点茶压酒,不可多嚼\footnote{}。

注:

\footnote{}不可多嚼:《现代实用中药学》载;白果肉有毒,不可生吃,或过量吃。

胡桃肉甘,补肾黑发,多食生痰,动气之物。

梨味甘酸,解酒除渴,止嗽消痰,善驱烦热。

榧实味甘,主疗五痔,蛊毒三虫,不可多食。

竹茹止呕,能除寒热,胃热咳哕,不寐安歇。

竹叶味甘,退热安眠,化痰定喘,止渴消烦。

竹沥味甘,阴虚痰火,汗热烦渴,效如开锁。

莱菔子甘,下气消谷,痰癖咳嗽,兼解面毒。

灯草味甘,运利小便,癃闭成淋,湿肿为最。

艾叶温平,温经散寒,漏血安胎,心痛即安。

绿豆气寒,能解百毒,止渴除烦,诸热可服。

川椒辛热,祛邪逐寒,明目杀虫,温而不猛。

胡椒味辛,心腹冷痛,下气温中,跌扑堪用。

石蜜\footnote{}甘平,入药炼熟,益气补中,润燥解毒。

马齿苋寒,青盲白翳,利便杀虫,症痈咸治\footnote{}。

葱白辛温,发表出汗,伤寒头疼,肿痈皆散。

胡荽味甘,上止头痛,内消谷食,痘疹发生。

注:

\footnote{}石蜜:即蜂蜜。一说蜂蜜生于岩石上的称石蜜。

\footnote{}咸:都、全能的意思。

韭味辛温,祛除胃寒,汁清血瘀,子\footnote{}医梦泄。

大蒜辛温,化肉消谷,解毒散痈,多用伤目。

食盐味咸,能吐中痰,心腹卒痛,过多损颜。

茶茗性苦,热渴能济,上清头目,下消食气。

酒通血脉,消愁遣兴,少饮壮神,过多损命。

醋消肿毒,积瘕可去,产后金疮,血晕皆治。

淡豆豉寒,能除懊憹,伤寒头痛,兼理瘴气。

莲子味甘,健脾理胃,止泻涩精,清心养气。

注:

\footnote{}子:指韭子。

大枣味甘,调和百药,益气养脾,中满休嚼。

生姜性温,通畅神明,痰嗽呕吐,开胃极灵。

桑叶性寒,善散风热,明目清肝,又兼凉血。

浮萍辛寒,发汗利尿,透疹散邪,退肿有效。

柽柳甘咸,透疹解毒,熏洗最宜,亦可内服。

胆矾酸寒,涌吐风痰,癫痫喉痹,烂眼牙疳。

番泻叶寒,食积可攻,肿胀皆逐,便秘能通。

寒水石咸,能清大热,兼利小便,又能凉血。

芦根甘寒,清热生津,烦渴呕吐,肺痈尿频。

银柴胡寒,虚热能清,又兼凉血,善治骨蒸。

丝瓜络甘,通络行经,解毒凉血,疮肿可平。

秦皮苦寒,明目涩肠,清火燥湿,热痢功良。

紫花地丁,性寒解毒,痈肿疔疮,外敷内服。

败酱微寒,善治肠痈,解毒行瘀,止痛排脓。

红藤苦平,消肿解毒,肠痈乳痈,疗效迅速。

鸦胆子苦,治痢杀虫,疟疾能止,赘疣有功。

白鲜皮寒,疥癣疮毒,痹痛发黄,湿热可逐。

土茯苓平,梅毒宜服,即能利湿,又可解毒。

马勃味辛,散热清金,咽痛咳嗽,吐衄失音。

橄榄甘平,清肺生津,解河豚毒,治咽喉痛。

蕺菜微寒,肺痈宜服,熏洗痔疮,消肿解毒。

板兰根寒,清热解毒,凉血利咽,大头瘟毒。

西瓜甘寒,解渴利尿,天生白虎\footnote{},清暑最好。

荷叶苦平,暑热能除,升清治泻,止血散瘀。

注:

\footnote{}天生白虎:是说西瓜性甘寒,有天然白虎汤之誉。

豆卷甘平,内清湿热,外解表邪,湿热最宜。

佩兰辛平,芳香辟秽,祛暑和中,化湿开胃。

冬瓜子寒,利湿清热,排脓消肿,化痰亦良。

海金砂寒,淋病宜用,湿热可除,又善止痛。

金钱草咸,利尿软坚,通淋消肿,结石可痊。

赤小豆平,活血排脓,又能利水,退肿有功。

泽漆微寒,逐水捷效,退肿祛痰,兼消瘰疬。

葫芦甘平,通利小便,兼治心烦,退肿最善。

半边莲辛,能解蛇毒,痰喘能平,腹水可逐。

海风藤辛,痹证宜用,除湿祛风,通络止痛。

络石微寒,经络能通,祛风止痛,凉血消痈。

桑枝苦平,通络祛风,痹痛拘挛,脚气有功。

千年健温,除湿祛风,强筋健骨,痹痛能攻。

松节苦温,燥湿祛风,筋骨酸痛,用之有功。

伸筋草温,祛风止痛,通络舒筋,痹通宜用。

虎骨味辛,健骨强筋,散风止痛,镇惊安神。

乌梢蛇平,无毒性善,功同白花\footnote{},作用较缓。

夜交藤平,失眠宜用,皮肤痒疮,肢体酸痛。

玳瑁甘寒,平肝镇心,神昏痉厥,热毒能清。

石决明咸,眩晕目昏,惊风抽搐,劳热骨蒸。

香橼性温,理气疏肝,化痰止呕,胀痛皆安。

佛手性温,理气宽胸,疏肝解郁,胀痛宜用。

薤白苦温,辛滑通阳,下气散结,胸痹宜尝。

荔枝核温,理气解寒,疝瘕腹痛,服之俱安。

注:

\footnote{}白花:指白花蛇。

柿蒂苦涩,呃逆能医,柿霜甘凉,燥咳可治。

刀豆甘温,味甘补中,气温暖肾,止呃有功。

九香虫温,胃寒宜用,助阳温中,理气止痛。

玫瑰花温,疏肝解郁,理气调中,行瘀活血。

紫石英温,镇心养肝,惊悸怔忡,子宫虚寒。

仙鹤草涩,收敛补虚,出血可止,劳伤能愈。

三七性温,止血行瘀,消肿定痛,内服外敷。

百草霜温,止血功良,化积止泻,外用疗疮。

玉竹微寒,养阴生津,燥热咳嗽,烦渴皆平。

鸡子黄甘,善补阴虚,除烦止呕,疗疮熬涂。

谷芽甘平,养胃健脾,饮食停滞,并治不饥。

白前微温,降气下痰,咳嗽喘满,服之皆安。

胖大海淡,清热开肺,咳嗽咽疼,音哑便秘。

海浮石咸,清肺软坚,痰热喘咳,瘰疬能痊。

海蛤壳咸,软坚散结,清肺化痰,利尿止血。

昆布咸寒,软坚清热,瘿瘤癥瘕,瘰疬痰核。

海蜇味咸,化痰散结,痰热咳嗽,并消瘰疬。

荸荠微寒,痰热宜服,止渴生津,滑肠明目。

禹余粮平,止泻止血,固涩下焦,泻痢最宜。

小麦甘凉,除烦养心,浮麦止汗,兼治骨蒸。

贯众微寒,解毒清热,止血杀虫,预防瘟疫。

南瓜子温,杀虫无毒,血吸绦蛔\footnote{},大剂吞服。

铅丹微寒,解毒生肌,疮疡溃烂,外敷颇宜。

樟脑辛热,开窍杀虫,理气辟浊,除痒止疼。

注:

\footnote{}血吸绦蛔:指血吸虫、绦虫,蛔虫。

降香性温,止血行瘀,辟恶降气,胀痛皆除。

川芎辛温,活血通经,除寒行气,散风止痛。

月季花温,调经宜服,瘰疬可治,又消肿毒。

刘寄奴苦,温通行瘀,消胀定痛,止血外敷。

自然铜辛,接骨续筋,即散瘀血,又善止疼。

皂角刺温,消肿排脓,疮癣瘙痒,乳汁不通。

虻虫微寒,逐瘀散结,癥瘕蓄血,药性猛烈。

䗪虫咸寒,行瘀通经,破癥消瘕,接骨续筋。

党参甘平,补中益气,止渴生津,邪实者忌。

太子参凉,补而能清,益气养胃,又可生津。

鸡血藤温,血虚宜用,月经不调,麻木酸痛。

冬虫夏草,味甘性温,虚劳咳血,阳痿遗精。

锁阳甘温,壮阳补精,润燥通便,强骨养筋。

葫芦巴温,逐冷壮阳,寒疝腹痛,脚气宜尝。

杜仲甘温,腰痛脚弱,阳痿尿频,安胎良药。

沙苑子温,补肾固精,养肝明目,并治尿频。

炉甘石平,去翳明目,生肌敛疮。燥湿解毒。

大枫子热,善治麻风,疥疮梅毒,燥湿杀虫。

孩儿茶凉,收湿清热,生肌敛疮,定痛止血。

木槿皮凉,疥癣能愈,杀虫止痒,浸汁外涂。

蚤休微寒,清热解毒,痈疽蛇伤,惊痫发搐。

番木鳖寒,消肿通络,喉痹痈疡,瘫痪麻木。

\chapter{药性赋}

〔寒性〕

诸药赋性,此类最寒。犀角解乎心热;羚羊清乎肺肝。泽泻利水通淋而补阴不足;海藻散瘿破气而治疝何难。闻之菊花能明目而清头风;射干疗咽闭而消痈毒,薏苡理脚气而除风湿;藕节消瘀血而止吐衄。瓜蒌子下气润肺喘兮、又且宽中,车前子止泻利小便兮,尤能明目。是以黄柏疮用,兜铃嗽医。地骨皮有退热除蒸之效;薄荷叶宜消风清肿之施。宽中下气,枳壳缓而枳实速也;疗肌解表,干葛先而柴胡次之。百部治肺热,咳嗽可止;栀子凉心肾,鼻衄最宜。玄参治结热毒痈,清利咽膈;升麻消风热肿毒,发散疮痍。尝闻腻粉抑肺而敛肛门;金箔镇心而安魂魄。茵陈主黄疸而利水;瞿麦治热淋之有血。朴硝通大肠,破血而止痰癖;石膏治头痛,解肌而消烦渴。前胡除内外之痰实;滑石利六腑之涩结。天门冬止嗽,补血涸而润肝心;麦门冬清心,解烦渴而除肺热。又闻治虚烦、除哕呕,须用竹茹;通秘结、导瘀血,必资大黄。宣黄连治冷热之痢,又厚肠胃而止泻;淫羊藿疗风寒之痹,且补阴虚而助阳。茅根止血与吐衄;石苇通淋于小肠。熟地黄补血且疗虚损;生地黄宣血更医眼疮。赤芍药破血而疗腹痛,烦热亦解;白芍药补虚而生新血,退热尤良。若乃消肿满逐水于牵牛;除毒热杀虫于贯仲。金铃子治疝气而补精血;萱草根治五淋而消乳肿。侧柏叶治血山崩漏之疾;香附子理气血妇人之用。地肤子利膀胱,可洗皮肤之风;山豆根解热毒,能止咽喉之痛。白藓皮去风治筋弱,而疗足顽痹;旋复花明目治头风,而消痰嗽壅。又况荆芥穗清头目便血,疏风散疮之用;瓜蒌根疗黄疸毒痈,消渴解痰之忧。地榆疗崩漏,止血止痢;昆布破疝气,散瘿散瘤。疗伤寒、解虚烦,淡竹叶之功倍;除结气、破瘀血,牡丹皮之用同。知母止嗽而骨蒸退;牡蛎涩精而虚汗收。贝母清痰止咳嗽而利心肺;桔梗开肺利胸膈而治咽喉。若夫黄芩治诸热,兼主五淋;槐花治肠风,亦医痔痢。常山理痰结而治温疟;葶苈泻肺喘而通水气。此六十六种药性之寒者也。
〔热性〕

药有温热,又当审详。欲温中以荜拨;用发散以生姜。五味子止嗽痰,且滋肾水;腽肭脐疗劳瘵,更壮元阳。原夫川芎祛风湿、补血清头;续断治崩漏、益筋强脚。麻黄表汗以疗咳逆;韭子壮阳而医白浊。川乌破积,有消痰治风痹之功;天雄散寒,为去湿助精阳之药。观夫川椒达下,干姜暖中。胡芦巴治虚冷之疝气;生卷柏破症瘕而血通。白术消痰壅、温胃,兼止吐泻;菖蒲开心气、散冷,更治耳聋。丁香快脾胃而止吐逆;良姜止心气痛之攻冲。肉苁蓉填精益肾;石硫黄暖胃驱虫。胡椒主去痰而除冷;秦椒主攻痛而去风。吴茱萸疗心腹之冷气;灵砂定心脏之怔忡。盖夫散肾冷、助脾胃,须荜澄茄;疗心痛、破积聚,用蓬莪术。缩砂止吐泻安胎、化酒食之剂;附子疗虚寒反胃、壮元阳之方。白豆蔻治冷泻,疗痈止痛于乳香;红豆蔻止吐酸,消血杀虫于干漆。岂知鹿茸生精血,腰脊崩漏之均补;虎骨壮筋骨,寒湿毒风之并祛。檀香定霍乱,而心气之痛愈;鹿角秘精髓,而腰脊之痛除。消肿益血于米醋;下气散寒于紫苏。扁豆助脾,则酒有行药破结之用;麝香开窍,则葱为通中发汗之需。尝观五灵脂治崩漏,理血气之刺痛;麒麟竭止血出,疗金疮之伤折。麋茸壮阳以助肾;当归补虚而养血。乌贼骨止带下,且除崩漏目翳;鹿角胶住血崩,能补虚羸劳绝。白花蛇治瘫痪,疗风痒之癣疹;乌梢蛇疗不仁,去疮疡之风热。乌药有治冷气之理;禹余粮乃疗崩漏之因。巴豆利痰水,能破寒积;独活疗诸风,不论新久。山茱萸治头晕遗精之药;白石英医咳嗽吐脓之人。厚朴温胃而去呕胀,消痰亦验;肉桂行血而疗心痛,止汗如神。是则鲫鱼有温胃之功;代赭乃镇肝之剂。沉香下气补肾,定霍乱之心痛;橘皮开胃去痰,导壅滞之逆气。此六十种药性之热者也。
〔温性〕

温药总括,医家素谙\footnote{}。木香理乎气滞;半夏主乎湿痰。苍术治目盲,燥脾去湿宜用;萝卜去膨胀,下气制面尤堪。况夫锺乳粉补肺气,兼疗肺虚;青盐治腹痛,且滋肾水。山药而腰湿能医;阿胶而痢嗽皆止。赤石脂治精浊而止泄,兼补崩中;阳起石暖子宫以壮阳,更疗阴痿。诚以紫菀治嗽,防风祛风,苍耳子透脑止涕,威灵仙宣风通气。细辛去头风,止嗽而疗齿痛;艾叶治崩漏、安胎而医痢红。羌活明目驱风,除湿毒肿痛;白芷止崩治肿,疗痔漏疮痈。若乃红蓝花通经,治产后恶血之余;刘寄奴散血,疗烫火金疮之苦。减风湿之痛则茵芋叶;疗折伤之症则骨碎补。藿香叶辟恶气而定霍乱;草果仁温脾胃而止呕吐。巴戟天治阴疝白浊,补肾尤滋;元胡索理气痛血凝,调经有助。尝闻款冬花润肺,去痰嗽以定喘;肉豆蔻温中,止霍乱而助脾。抚芎走经络之痛;何首乌治疮疥之资。姜黄能下气、破恶血之积;防己宜消肿、去风湿之施。藁本除风,主妇人阴痛之用;仙茅益肾,扶元气虚弱之衰。

乃曰破故纸温肾,补精髓与劳伤;宣木瓜入肝,疗脚气并水肿。杏仁润肺燥止嗽之剂;茴香治疝气肾病之用。诃子生精止渴,兼疗滑泄之疴\footnote{};秦艽攻风逐水,又除肢节之痛。槟榔豁痰而逐水,杀寸白虫;杜仲益肾而添精,去腰膝重。当知紫石英疗惊悸崩中之疾;橘核仁治腰痛疝气之瘨。金樱子兮涩遗精;紫苏子兮下气涎。淡豆豉发伤寒之表;大小蓟除诸血之鲜。益智安神,治小便之频数;麻仁润肺,利六腑之燥坚。抑又闻补虚弱、排疮脓,莫若黄芪;强腰脚、壮筋骨,无如狗脊。菟丝子补肾以明目;马蔺花治疝而有益。此五十四种药性之温者也。

注:

\footnote{}谙:An音安。熟悉的意思。

\footnote{}疴:ke,音柯,病的意思。
〔平性〕

详论药性,平和惟在。以硇砂而去积;用龙齿以安魂。青皮快膈除膨胀,且利脾胃;芡实益精治白浊,兼补真元。原夫木贼草去目翳,崩漏亦医;花蕊石治金疮,血行则却。决明和肝气,治眼之剂;天麻主头眩,祛风之药。甘草和诸药而解百毒,盖以性平;石斛平胃气而补肾虚,更医脚弱。观乎商陆治肿,复盆益精。琥珀安神而散血;朱砂镇心而有灵。牛膝强足补精,兼疗腰痛;龙骨止汗住泄,更治血崩。甘松理风气而痛止;蒺藜疗风疮而目明。人参润肺宁心,开脾助胃;蒲黄止崩治衄,消瘀调经。岂不以南星醒脾。去惊风痰吐之忧;三棱破积,除血块气滞之症。没石主泄泻而神效;皂角治风痰而响应;桑螵蛸疗遗精之泄,鸭头血医水肿之盛。蛤蚧治劳嗽,牛蒡子疏风壅之痰;全蝎主风瘫,酸枣仁去怔忡之病。尝闻桑寄生益血安胎,且止腰痛;大腹子去膨下气,亦令胃和。小草、远志,俱有宁心之妙;木通、猪苓尤为利水之多。莲肉有清心醒脾之用;没药乃治疮散血之科。郁李仁润肠宣水,去浮肿之疾;茯神宁心益智,除惊悸之疴。白茯苓补虚劳,多在心脾之有眚\footnote{};赤茯苓破结血,独利水道以无毒。因知麦芽有助脾化食之功;小麦有止汗养心之力。白附子去面风之游走;大腹皮治水肿之泛溢。椿根白皮主泻血;桑根白皮主喘息。桃仁破瘀血兼治腰痛;神曲健脾胃而进饮食。五加皮坚筋骨以立行;柏子仁养心神而有益。抑又闻安息香辟恶,且止心腹之痛;冬瓜仁醒脾,实为饮食之资。僵蚕治诸风之喉闭;百合敛肺劳之嗽萎。赤小豆解热毒,疮肿宜用;枇杷叶下逆气,哕呕可医。连翘排疮脓与肿毒;石南叶利筋骨与毛皮。谷芽养脾,阿魏除邪气而破积;紫河车补血,大枣和药性以开脾。然而鳖甲治劳疟,兼破症瘕;龟甲坚筋骨,更疗崩疾。乌梅主便血疟痢之用;竹沥治中风声音之失。此六十八种药性之平者也。

注:

\footnote{}眚:shěng,音省。过错。这里指病。

\chapter{十八反歌}

\begin{yuanwen}
本草明言十八反,半蒌贝蔹芨攻\footnote{攻、战、叛三字,都作“反”字理解。全歌是:乌头反半夏、栝蒌、贝母、白芨、白蔹。甘草反大戟、芫花、甘遂、海藻。藜芦反细辛、芍药、人参、沙参、苦参、丹参、元参。}乌。藻戟遂芫俱战草,诸参辛芍叛藜芦。
\end{yuanwen}

\chapter{十九畏歌}

硫黄原是火中精,朴硝一见便相争。水银莫与砒霜见,狼毒最怕密陀僧。巴豆性烈最为上,偏与牵牛不顺情。丁香莫与郁金见,牙硝难合京三棱。川乌草乌不顺犀,人参最怕五灵脂。官桂善能调冷气,若逢石脂便相欺。大凡修合看顺逆,炮烂炙煿莫相依。

\chapter{六陈歌}

枳壳陈皮半夏齐,麻黄狼毒及茱萸;六般之药宜陈久,入药方知奏效奇。

\chapter{妊娠服药禁歌}

芫斑水蛭及虻虫,乌头附子配天雄,野葛水银并巴豆,牛膝薏苡与蜈蚣,三棱芫花代赭麝,大戟蝉蜕黄雌雄,牙硝芒硝牡丹桂,槐花牵牛皂角同,半夏南星与通草,瞿麦干姜桃仁通,硇砂干漆蟹爪甲,地胆茅根与䗪虫。

\chapter{方剂口诀}

解表剂
〔辛温解表〕

麻黄汤(附:麻黄加术汤、麻杏苡甘汤)

麻黄汤中用桂枝,杏仁甘草四般施,发热恶寒头项痛,伤寒无汗服之宜;方中加术治湿痹,去桂加苡亦能医。

桂枝汤(附:桂枝加桂汤、桂枝加芍药汤、桂枝加大黄汤、桂枝加葛根汤)

桂枝汤治太阳风,芍药甘草枣姜同;加重桂枝治奔豚,倍用芍药腹痛松;或加葛根医项强,或加大黄表里通。

大青龙汤(附:越婢汤)

大青龙用桂麻黄,杏草石膏姜枣藏,太阳无汗兼烦躁,解表清热此为良;若于方中除杏桂,主治风水身肿强。

九味羌活汤(附:大羌活汤)

九味羌活用防风,细辛苍芷与川芎,黄芩生地同甘草,发汗祛风可建功;大羌活汤除白芷,白术连知己独充。

葱豉汤(附:活人葱豉汤、葱豉桔梗汤)

葱豉汤原肘后方,同煎葱豉代麻黄;或加麻黄葛根用,伤寒无汗项背强;或添栀翘与甘桔,薄荷竹叶效均良。
〔辛凉解表〕

麻黄杏仁甘草石膏汤

麻杏石甘汤法良,四药组合有擅长,肺热壅盛气喘急,辛凉疏泄效能彰。

桑菊饮

桑菊饮中桔梗翘,杏仁甘草薄荷饶,芦根为饮轻清剂,风温咳嗽服之消。

银翘散(附:银翘汤)

银翘散主上焦疾,竹叶荆牛薄荷豉;甘桔芦根凉解法,风温初感此方医;银翘竹草与麦地,下后无汗脉浮宜。

柴葛解肌汤(附:程氏柴葛解肌汤)

陶氏柴葛解肌汤,邪在三阳热势张,芩芍桔草羌活芷,石膏大枣与生姜;程氏除桔羌石芷,加入丹地二母良。

加减葳蕤汤

加减葳蕤用白薇,豆豉生葱桔梗随,枣草薄荷共八味,滋阴发汗最相宜。

葱白七味饮

葱白七味外台方,新豉葛根与生姜;麦冬生地千扬水,血虚外感效验彰。
〔助阳解表〕

麻黄附子细辛汤(附:麻附甘草汤)

麻黄附子细辛汤,发表温经两法彰;除去细辛加炙草,少阴反热亦能康。

再造散

再造散用参芪甘,桂附羌防芎芍参,细辛加枣煨姜煎,阳虚无汗法当谙。

参苏饮

参苏饮内用陈皮,枳壳前胡半夏宜,甘葛木香桔梗茯,内伤外感此方推。

败毒散(附:荆防败毒散、连翘败毒散、仓廪散)

人参败毒茯苓草,枳桔柴前羌独芎,薄荷少许姜三片,时行寒温有奇功;加入荆防免生姜,温毒发斑肿痛松;仓廪散加陈仓米,痢疾作吐毒气冲;连翘败毒羌栀芎,玄薄升柴归桔风,芩芍红花牛蒡子,伤寒邪结耳下痛。

香苏散(附:正气天香散)

香苏散内用陈皮,香附紫苏二药随,甘草和中兼补正,风寒气滞此方宜;正气天香除甘草,增入乌姜妇女施。
〔化饮解表〕

小青龙汤(附:射干麻黄汤)

小青龙汤治水气,喘咳呕哕渴利慰,姜桂麻黄芍药甘,细辛半夏兼五味;射干麻黄亦治水,不在发表在宣肺,紫菀冬花味枣姜,辛夏平逆喘症贵。
〔透疹解表〕

升麻葛根汤(附:宣毒发表汤)

阎氏升麻葛根汤,芍药甘草合成方,麻疹初期出不透,升阳透表效彰彰;宣毒发表去芍药,再加薄荷与荆防,前胡枳桔淡竹叶,木通连翘共牛蒡。

竹叶柳蒡汤

竹叶柳蒡葛根知,蝉衣荆芥薄荷司,石膏粳米参甘麦,初起风痧此方施。
〔解毒发表〕

清咽汤(《疫喉浅论》、《温病学》)

疫喉初起清咽汤,枳桔杏草前荆防,泄卫透表薄浮萍,解毒僵蚕榄牛蒡。
涌吐剂
〔实证吐〕

瓜蒂散

瓜蒂散中赤小豆,香豉和调酸苦凑,宿食痰涎填上脘,逐邪涌吐功能奏。

三圣散

三圣散中瓜防芦,齑汁同煎进一盂,实热风痰皆可吐,诸般中毒亦能除。

急救稀涎散

稀涎皂角白矾班,直中痰潮此斩关,中风暴仆口不语,闭证能开气自还。
〔虚证吐法〕

参芦饮

参芦饮是丹溪方,竹沥新加效更良,气虚体弱痰壅盛,服此得吐自然康。
泻下剂
〔寒下〕

大承气汤(附:小承气汤、调胃承气汤、三化汤)

大承气汤用芒硝,枳实大黄厚朴标,救阴泻热功偏擅,急下阳明有数条;去硝名为小承气,调胃只有硝黄草;小承气内加羌活,中风便秘三化超。

凉膈散

凉膈硝黄栀子翘,黄芩甘草薄荷饶,竹叶蜜煎疗膈热,中焦燥实服之消。

大陷胸汤(附:大陷胸丸)

大陷胸汤治结胸,芒硝甘遂大黄供;再加白蜜杏葶入,项强如痉有奇功。

十枣汤(附:控涎丹、葶苈大枣泻肺汤)

十枣汤用遂戟花,强人伏饮效堪夸;控涎丹用遂戟芥,葶苈大枣亦可嘉。

舟车丸(附:疏凿饮子)

舟车牵牛及大黄,遂戟芫花槟木香,青皮橘皮加轻粉,水肿腹胀力能当;疏凿饮子亦泻水,木通泽泻与槟榔,秦艽苓腹椒商陆,赤豆姜皮退肿良。
〔温下〕

大黄附子汤

大黄附子仲景方,胁下寒凝痛莫当,共合细辛三种药,功专温下妙非常。

温脾汤

温脾附子与干姜,人参甘草及大黄,寒热并行兼补泻,温通寒积最相当。
〔润下〕

麻子仁丸(附:润肠丸)

麻子仁丸治脾约,枳朴大黄麻杏芍,土燥津枯便难出,通幽养液蜜丸嚼;润肠丸用当归尾,大黄桃仁与羌活,麻仁共捣蜂蜜丸,同属润下功效捷。

五仁丸

五仁柏子杏仁桃,松肉陈皮郁李饶,蜜水为丸米饮下,血结气滞可通调。

更衣丸

更衣丸治大便难,芦荟硃砂滴酒丸,肝经火旺肠道结,泻热通幽仗苦寒。

济川煎

济川归膝肉苁蓉,泽泻升麻枳壳从,便结体虚难下夺,寓通于补法堪宗。
〔攻补兼施〕

黄龙汤

黄龙汤即大承气,加入参归甘桔比,生姜红枣同煎服,攻补兼施通便秘。

增液承气汤(附:承气养营汤)

增液承气参地冬,硝黄加入五药供,热结津枯大便秘,增水行舟润下功;承气养营生地黄,当归白芍知母从,结合攻下小承气,肠润津回秘结松。

宣清导浊汤

湿闭宣清导浊汤,蚕砂化浊通清阳,皂角辛走上下窍,化气二苓寒水尝。
和解剂
〔和解少阳〕

小柴胡汤

小柴胡汤和解供,半夏人参甘草从,更用黄芩加姜枣,少阳为病此方宗。

蒿芩清胆汤

蒿芩清胆碧玉须,苓夏陈皮枳竹茹,少阳热重寒轻证,胸痞呕恶总能除。
〔调和肝脾〕

四逆散(附:柴胡疏肝散)

四逆散里用柴胡,芍药枳实甘草倶,此是阳邪成厥逆,清热散郁平剂扶;柴胡疏肝加芎香,枳实易壳功效殊。

逍遥散(附:加味逍遥散、黑逍遥散)

逍遥散用当归芍,柴苓术草加姜薄,散郁除蒸功最捷,调经八味丹栀着;若加生地或熟地,黑逍遥散名亦高。

痛泻要方

痛泻要方陈皮芍,防风白术煎丸酌,补土泻木理肝脾,若作食伤医便错。
〔调和肠胃〕

半夏泻心汤

半夏泻心黄连芩,干姜甘草与人参,大枣和中治虚痞,法在调阳而和阴。

黄连汤

黄连汤内用干姜,半夏人参甘草藏,更用桂枝兼大枣,寒热平调呕痛忘。
〔其他〕

达原饮(附:柴胡达原饮、清脾饮)

达原饮用朴槟芩,白芍知甘草果仁,邪伏膜原瘟疫发,疏邪宣壅急先行;除去知芍加柴胡,枳桔青荷痰疟灵;清脾青皮与草果,朴术柴苓半夏芩,甘草炙用姜五片,疟疾热多寒少进。
表里双解剂
〔解表攻里〕

厚朴七物汤

厚朴七物是复方,甘桂枳朴枣黄姜,腹满发热脉浮数,表里交攻效力彰。

大柴胡汤

大柴胡汤用大黄,枳实芩夏芍枣姜,往来寒热心下满,柴胡承气加减方。

防风通圣散

防风通圣大黄硝,荆芥麻黄栀芍翘,甘桔芎归膏滑石,薄荷芩术力偏饶。
〔解表清里〕

葛根黄芩黄连汤

葛根黄芩黄连汤,甘草四般治二阳,解表清里兼和胃,喘汗自利保安康。

石膏汤

石膏汤用芩柏连,麻黄豆豉山栀全,清热发汗兼解毒,姜枣细茶一同煎。
〔解表温里〕

五积散

五积散治五般积,麻苍归芍芎芷桔,枳朴姜桂茯苓草,陈皮半夏功最捷。
清热泻火剂
〔清气分热〕

白虎汤(附:白虎加人参汤、白虎加桂枝汤、加苍术汤)

白虎汤清气分热,膏知甘草粳米入,津伤口渴宜急投,气虚加参最相得,加桂加术用不同,温疟湿温宜辨别。

竹叶石膏汤

竹叶石膏汤人参,麦冬半夏竹叶增,甘草粳米水煎服,暑热烦渴脉虚寻。
〔清营凉血〕

清营汤(附:清宫汤)

清营汤是鞠通方,暑入心包营血伤,犀角丹元连地麦,银翘竹叶煎服康;去银连地与丹参,加莲用心清宫汤。

犀角地黄汤

犀角地黄芍药丹,血升胃热火邪干,斑黄阳毒皆堪治,热在营血服之安。
〔气血两清〕

清瘟败毒散(附:化斑汤)

清瘟败毒地连芩,丹石栀甘竹叶灵,犀角玄翘知芍桔,清邪泻毒亦滋阴;化斑白虎为基础,加入犀玄除神昏。
〔泻火解毒〕

黄连解毒汤

黄连解毒汤四味,黄柏黄芩栀子备,躁狂大热呕不眠,吐衄斑黄均可贵。

普济消毒饮

普济消毒芩连鼠,玄参甘桔蓝根侣,升柴马勃连翘陈,僵蚕薄荷为末咀;或加人参及大黄,大头天行效力普。
〔清脏腑热〕

泻心汤

泻心汤是仲师方,并用连芩及大黄,热迫血行成吐衄,火平血静自然康。

导赤散

导赤生地与木通,草梢竹叶四般攻,口糜淋痛小肠火,引热同归小便中。

清心莲子饮

清心莲子石莲参,地骨黄芩赤茯苓,芪草麦冬车前子,烦渴梦遗及淋崩。

龙胆泻肝汤(附:当归龙荟丸)

龙胆泻肝栀芩柴,生地车前泽泻偕,木通甘草当归合,肝经湿热力可排;当归龙荟用四黄,龙胆芦荟木麝香,黑栀青黛姜汤下,一切肝火尽能攘\footnote{}。

注:

\footnote{}攘:rǎng,音壤。排斥、除掉的意思。

泻青丸

泻青丸用龙脑栀,下行泻火大黄资,羌防上升芎归润,火郁肝经用此宜。

左金丸(附:戊己丸、香连丸、连附六一汤)

左金茱连六一丸,肝经火郁吐吞酸,再加芍药名戊己,香连去萸热痢安,连附六一治腹痛,寒因热用理一般。

泻白散

泻白桑皮地骨皮,甘草粳米四般宜,参茯知芩均可入,肺热喘咳此方施。

泻黄散

泻黄甘草与防风,石膏栀子藿香充,炒香蜜酒调和服,胃热口疮并见功。

清胃散

清胃散用升麻连,当归生地牡丹全,或益石膏平胃热,口疮吐衄与牙宣。

玉女煎

玉女煎中熟地黄,滋阴清胃效力强,石膏知母麦冬膝,烦热失血牙疼良。

黄芩汤(附:芍药汤)

黄芩汤用甘芍并,二阳合利枣加烹;洁古芍药去大枣,归黄香草连桂槟。

白头翁汤(附:白头翁加甘草阿胶汤)

白头翁汤治热痢,黄连黄柏秦皮备;若加阿胶与甘草,产后虚痢称良剂。
〔清虚热〕

青蒿鳖甲汤

青蒿鳖甲知地丹,阴分伏热此剂攀,夜热早凉无汗者,从里泄外服之安。

秦艽鳖甲散

秦艽鳖甲治风劳,地骨柴胡及青蒿,当归知母乌梅合,止嗽除蒸敛汗高。

当归六黄汤

当归六黄盗汗良,芪柏芩连二地黄,泻火固表复滋阴,麻黄根加效更强。
祛暑剂
〔清暑〕

清络饮

清络饮用荷叶边,竹丝银扁翠衣添,暑伤气分轻清剂,平时饮服预防先。

清宣金脏法(《时病论》)

清宣金脏法牛蒡,马贝杷桔杏蒌桑,暑伤肺气清宣降,热咳胸闷自如常。

香薷散(附:新加香薷饮)

三物香薷豆朴先,祛暑解表功效坚;再加银翘豆易花,太阴暑温须当辨。

六一散

六一散中滑石草,解肌行水兼清燥,统治表里及三焦,暑热烦渴泻痢保。

桂苓甘露散

桂苓甘露散河间\footnote{},暑湿伤中治不难,滑石石膏寒水石,五苓甘草共煎餐。

注:

\footnote{}河间:指桂苓甘露散出刘河间《宣明论方》。

清凉涤暑法(《时病论》)

清凉涤暑用六一,佐以蒿扁翘瓜衣,通苓渗湿支河利,暑泻冒暑并可医。

清暑益气汤

清暑益气用洋参,竹叶知母荷梗增,麦冬甘斛连瓜翠,气阴耗伤此方能。
开窍通关剂
〔凉开〕

牛黄清心丸(附:安宫牛黄丸)

万氏牛黄丸最精,芩连栀子郁砂并,金箔雄犀珠冰麝,安宫开窍效最灵。

至宝丹

至宝朱砂麝息香,雄黄犀角与牛黄,金银二箔兼龙脑,琥珀还同玳瑁襄\footnote{}。

注:

\footnote{}襄:帮助的意思。

紫雪丹

紫雪犀羚朱朴硝,硝磁寒水滑石膏,丁沉木麝升玄草,更用赤金法亦超。

菖蒲郁金汤(《温病学讲义》)

菖蒲郁金扼翘菊,丹皮银花沥姜滴,竹叶滑石牛蒡入,玉枢丹入启心机。

神犀丹

神犀丹内用犀君,金汁参蒲芩地群,豉粉银翘蓝紫草,暑邪瘟疫有奇勋。

行军散

诸葛行军痧胀方,珍珠牛麝冰雄黄,硼硝金箔共研末,窍闭神昏服之康。

小儿回春丹

回春丹里二连并,礞石珠黄菖半星,贝母辰砂天竺麝,小儿惊搐赖斯\footnote{}平。

注:

\footnote{}斯:这、这个的意思。

抱龙丸(附:牛黄抱龙丸、琥珀抱龙丸)

抱龙星麝竺雄黄,加入辰砂痰热尝;牛黄抱龙牛黄入,清心解毒效尤强。琥珀抱龙加参草,茯苓山药共檀香,枳实枳壳与金箔,体弱急惊去麝良。
〔温开〕

苏合香丸

苏合香丸麝息香,木丁熏陆荜檀香,犀冰白术沉诃附,应用朱砂中恶尝。

通关散

通关散里细辛皂,吹鼻取嚏有奇效,气机开利一身畅,风痰口噤用之妙。
温里回阳剂
〔温中祛寒〕

理中丸(附:桂枝人参汤、附子理中丸、枳实理中丸、连理汤、理中化痰丸)

理中丸主理中乡,甘草人参术干姜,呕利腹痛阴寒盛,或加附子是扶阳。桂枝加入理中内,表里双解记端详。伤寒结胸实满痛,枳实理中茯蜜丸。黄连加入理中汤,主治呕吐酸水浆。如于理中加苓夏,咳唾稀痰效力彰。

大建中汤

大建中汤建中阳,蜀椒干姜参饴糖,寒疝冲起有头足,痛不能按用此攘。

小建中汤(附:黄芪建中汤、当归建中汤)

小建中汤芍药多,桂姜甘草大枣和,更加饴糖补中气,阳虚劳损起沉疴。黄芪建中补不足,表虚身痛效无过。又有当归建中法,产后诸虚属妇科。

吴茱萸汤

吴茱萸汤人参枣,重用生姜温胃好,中寒呕吐阴寒利,阴盛头痛皆能保。
〔回阳救逆〕

四逆汤(附:四逆加人参汤、茯苓四逆汤、白通汤、白通加人尿猪胆汁汤、浆水散)

四逆汤中附草姜,四肢厥冷急煎尝,腹痛吐利脉沉微,教逆回阳赖此方。人参加入四逆内,益气固脱救阴伤,病仍不解烦躁者,参苓增来有奇长。四逆汤中除甘草,葱白加入散寒良。阴盛格阳面见赤,尿胆加进白通汤。四逆加入良桂夏,浆水入煎挽亡阳。

回阳救急汤(附:正阳散,大顺散)

回阳救急用六君,桂附干姜五味群,加麝三厘或胆汁,三阴寒厥建奇勋。正阳散内仅附姜,甘麝皂荚阴寒伤。大顺散中甘杏桂,霍乱呕吐是妙方。

参附汤(附:芪附汤、术附汤)

参附汤是救急方,补气回阳效力彰,正气大亏阳暴脱,喘汗肢冷可煎尝。除参易芪或易术,芪附术附均高强。

黑锡丹

黑锡丹能镇浮阳,金铃桂附蔻硫黄,芦巴起纸茴沉木,定喘痰逆有专长。

真武汤

真武汤壮肾中阳,茯苓术芍附生姜,少阴腹痛寒水气,悸眩瞤惕保安康。

附子汤(附:桂枝附子汤、白术附子汤、甘草附子汤)

少阴阳虚附子汤,人参苓芍术堪尝,体痛背寒肢逆冷,脉微无力服之康。桂枝附子草枣姜,体痛脉虚风湿方。去桂加术余减半,风湿伤气效亦良。又有甘草附子汤,汗出气短宜煎尝,甘草附子桂白术,身寒微肿辨端详。
〔其他〕

暖培卑监法(《时病论》)

暖培卑监\footnote{}四君强,喜升葛根喜燥苍,温以益智缓粳米,土寒泄利炭炮姜。

注:

\footnote{}卑监:土运不及之年称卑监,这里指脾土不足。

厚朴温中汤

厚朴温中陈草苓,干姜草蔻木香并,煎服加姜治腹痛,虚寒胀满用皆灵。

当归四逆汤(附:当归四逆加吴茱萸生姜汤)

当归四逆芍桂枝,细辛甘枣通草施,内有久寒姜茱入,散寒温里此方宜。

四神丸

四神故纸吴茱萸,肉蔻除油五味须,大枣百枚姜八两,五更肾泻火衰扶。

半硫丸

半硫丸内药无多,阳虚气寒便闭疴,或是便溏经日久,此方即可令调和。
消导化积剂
〔消食导滞〕

保和丸(附:大安丸、小保和丸)

保和神曲与山楂,苓夏陈翘菔子加,炊饼\footnote{}为丸白汤下,方中亦可用麦芽。大安丸内加白术,消中兼补效堪夸。去夏菔翘加术芍,小保和丸助脾佳。

注:

\footnote{}炊饼:古时烧饼一类的面饼。

枳实导滞丸(附:木香导滞丸)

枳实导滞首大黄,芩连曲术茯苓襄,泽泻蒸饼糊丸服,湿热积滞力能攘。若还后重兼气滞,木香导滞加槟榔。

木香槟榔丸

木香槟榔青陈皮,枳柏黄连莪术随,大黄黑丑兼香附,泻痢食疟用咸宜。

枳术丸(附:曲麦枳术丸、橘半汤术丸、香砂祝术丸)

枳术丸是消补方,荷叶裹饭为丸良。若于本方加曲麦,食多胀满效尤强。或加陈皮与半夏。脾虚停痰力能匡\footnote{}。更有方中加香砂,行气消食开胃长。

注:

\footnote{}匡:纠正的意思

健脾丸(附:启脾散)

健脾参术苓草陈,肉蔻香连合砂仁,楂肉山药曲麦炒,消补兼施建奇勋。启脾散中用参术,砂陈莲楂五谷虫,脾虚胃弱正可服,疳疾等证妙通神。
〔消痞化积〕

枳实消痞丸

枳实消痞四君全,麦夏曲与朴姜连,消积除满兼清热,消中有补两相兼。

葛花解酲汤

葛花解酲\footnote{}香砂仁,参术蔻青陈二苓,神曲干姜兼泽泻,温中化湿酒醉灵。

注:

\footnote{}酲:chéng,音呈。酒醉,神志不清。

蟾砂散

蟾砂散用大蟾蜍,砂仁纳腹泥封涂,煅红研末桔汤下,气膨得此数服除。

鳖甲煎丸(附:化癥回生丹)

鳖甲煎丸疟母方,䗪虫鼠妇及蜣螂,蜂窠石苇硝黄射,桂朴凌霄丹芍姜,瞿麦柴芩阿半夏,桃仁葶苈和参尝。化癥回生苏子霜,艾蒲椒炭桂参香,杏桃归芍萸芎索,苏木红花片子姜,虻蛭良脂棱乳没,降香茴麝与丁香,两头尖漆真阿魏,益母鳖胶地大黄,寒邪侵袭深延入,血分成癥证可详,炼蜜为丸钱五重,蜡皮封护慎毋忘。

胆道石汤(天津南开医院验方)

胆道排石郁金香,茵陈枳壳与大黄,金钱草同木香配,或制丸剂缓缓尝。

海藻玉壶汤(《医宗金鉴·外科心法要诀》)

海藻玉壶藻带昆,连翘半贝与青陈,川芎独活当归合,化痰散结善消瘿。
补益剂
〔补气〕

四君子汤(附:六君子汤、异功散、香砂六君子汤、六神散)

四君子汤中和义,参术茯苓甘草比。益以夏陈名六君,祛痰补气阳虚饵。除却半夏名异功,或加香砂胃寒使。加入豆芪名六神,脾虚食少虚热宜。

参苓白术散(附:资生丸)

参苓白术扁豆陈,山药甘莲砂苡仁,桔梗上浮兼保肺,枣汤调服益脾神。资生丸亦参术苓,去枣加藿白蔻仁,桔连泽黄山楂麦,调理脾胃力能行。

补中益气汤

补中益气芪术陈,升柴参草当归身,劳倦内伤功独擅,亦治阳虚外感因。

玉屏风散

玉屏风散最有灵,芪术防风鼎足形,表虚汗多易感冒,药虽相畏效相成。

保元汤

保元补益总偏温,桂草参芪四味珍,男妇虚劳幼科痘,阳虚气弱力能振。

生脉散

生脉麦味与参施,补气生津保肺机,少气汗多口干渴,病危脉绝效力奇。

人参蛤蚧散(附:人参胡桃汤)

宝鉴人参蛤蚧散,喘咳痰血与胸烦,桑皮二母杏苓草,若非虚热慎毋餐。动则喘息不能卧,人参胡桃生姜揽。

升阳益胃汤(《脾胃论》)

内伤升阳益胃汤,湿多热少抑清阳,倦怠懒食身重痛,口苦舌干便不常,六君白芍连泽泻,羌独黄芪柴与防。

参麦汤(《湿热病篇》)

气液两亏宜清补,当推薛氏参麦汤,参麦瓜斛草连谷,肺气胃津两复方。
〔补血〕

四物汤

四物地芍与归芎,血家百病此方通,凡属血虚之病证,加减运用在胸中。

当归补血汤

当归补血东垣笺,黄芪一两归二钱,证象白虎烦渴甚,脉大而虚宜此煎。

归脾汤(附:妙香散)

归脾汤用术参芪,归草茯神远志随,酸枣木香龙眼肉,煎加姜枣益心脾。妙香散内参芪草,二茯山药与远志,辰砂桔麝术香入,治疗惊怖意不定。

当归生姜羊肉汤(附:当归羊肉汤)

当归生姜羊肉汤,产后腹痛蓐劳匡,虚劳寒疝皆堪治,痛呕陈皮白术襄。更有当归羊肉汤,参芪加入是其方,产后发热自汗出,肢体疼痛服之良。

复脉汤(附:加减复脉汤)

复脉甘草参桂姜,麦冬生地麻仁襄,大枣阿胶加酒服,虚劳肺痿效力强。除去参桂大枣姜,加入白芍治阴伤,温邪久羁阳明证,身热口干服之康。
〔气血两补〕

八珍汤(附:十全大补汤、人参养营汤、泰山磐石散)

四君四物八珍汤,气血两补是名方。再加黄芪与肉桂,十全大补效无双。若益志陈五味子,去芎辛窜养营良。十全汤内去茯桂,加续苓砂磐石强。

薯蓣丸

薯蓣丸中用八珍,桔防豆枣阿杏仁,桂姜麦蔹同柴曲,风气虚劳总可珍。
〔补阴〕

六味地黄丸(附:知柏地黄丸、杞菊地黄丸、都气丸、八仙长寿丸)

六味地黄益肾肝,茱薯丹泽地苓专。更加知柏成八味,阴虚火旺可煎餐。养阴明目加杞菊,滋阴都气五味搬,肺肾两调金水生,麦冬加入长寿丸。

左归饮(附:左归丸)

左归萸地药苓从,杞草齐成壮水功。若要为丸除苓草,更加龟鹿二胶从,菟丝牛膝均采用,精血能充效无穷。虚火上炎阴失守,去除柏鹿益门冬。

大补阴丸(附:通关丸)

大补阴亏治水亏,虚火劳热致虚羸,地黄龟板兼知柏,猪脊髓共蜜和为。滋肾通关桂柏知,溺癃不渴下焦医,丸号通关能利水,又名滋肾补阴虚。

虎潜丸

虎潜足痿是神方,虎骨陈皮共锁阳,龟板干姜知母芍,再加柏地制丸尝。

大造丸

河车大造地天冬,黄柏人参牛膝从,龟板麦冬川杜仲,阴虚劳伤此方宗。

天王补心丹

天王补心柏枣仁,二冬生地与茯苓,三参桔梗朱砂味,远志归身共养神。

甘麦大枣汤

甘草小麦大枣汤,妇人脏躁服之康,精神恍惚悲欲哭,和肝解郁效力彰。

一贯煎

一贯煎中用地黄,沙参杞子麦冬襄,当归川楝水煎服,肝肾阴虚是妙方。

酸枣仁汤

酸枣二升先煮汤,茯知二两佐之良,芎二甘一相调剂,服后安然入睡乡。

拯阴理劳汤(《医宗必读》)

阴虚火动用拯阴,皮寒骨蒸咳嗽侵,食少痰多烦少气,生脉归芍地版贞,薏苡橘丹连合草,汗多不寐加枣仁,燥痰桑贝湿苓半,阿胶咳血骨热深。

补肺阿胶汤(附:月华丸)

补肺阿胶马兜铃,鼠粘甘草杏糯呈,肺虚火盛人当服,顺气生津咳嗽宁。月华丸最能滋阴,二冬二地沙贝苓,药部阿胶广三七,獭肝桑菊保肺金。

石斛夜光丸

石斛夜光枳膝芎,二地二冬杞丝苁,青箱草决犀羚药,参味连芩蒺草风。
〔补阳〕

肾气丸(附:济生肾气丸、十补丸)

金匮肾气治肾虚,熟地淮药及山萸,丹皮苓泽加附桂,引火归元热下趋。济生加入车牛膝,二便通调肿胀除。若加鹿茸五味子,十补丸子肾阳扶。

右归饮(附:右归丸)

右归桂附兼山药,杞子地黄甘草炙,杜仲山萸煎汤饮,益火之源不可缺。除却甘草鹿胶入,菟丝当归改丸作。

拯阳理劳汤(《医宗必读》)

阳虚气弱用拯阳,倦怠恶烦劳则张,表热自汗身酸痛,减去升柴补中方,更添桂味寒加附,泄入升柴诃蔻香,夏咳减桂加麦味,冬咳不减味干姜。

升陷汤(《医学衷中参西录》)

张氏巧思升陷汤,黄芪为君知母旁,升柴桔梗为舟揖,气难接续试此方。
重镇安神剂

安神丸

东垣朱砂安神丸,地草归连配合全,烦乱懊憹神不静,怔忡失寐奏凯旋。
固涩剂
〔敛汗固表〕

牡蛎散

牡蛎散内用黄芪,浮麦麻黄根最宜,自汗阳虚或盗汗,此方取后显神奇。
〔敛肺止咳〕

九仙散

九仙散用乌梅参,桔梗桑皮贝母呈,栗壳阿胶冬五味,敛肺止咳气能增。
〔涩肠固脱〕

桃花散(附:赤石脂禹余粮汤)

桃花散中赤石脂,粳米干姜共用之。石脂又与余粮合,久痢滑脱正可施。

真人养脏汤

真人养脏木香诃,粟壳当归肉蔻和,术芍桂参甘草共,脱肛久痢服之瘥。

驻车丸(附:地榆丸)

驻车丸治冷热剂,湿热久郁大肠气,黄连干姜当归胶,米醋为丸如豆许。地榆丸内当归胶,黄连去须诃子烧,乌梅取肉木香晒,炼蜜为丸米饮调。

益黄散

益黄散内用青陈,诃子丁香炙草伦,脾土虚衰或泄泻,和脾调气效如神。
〔涩精止遗〕

金锁固精丸(附:水陆二仙丹)

金锁固精芡莲须,龙骨蒺藜牡蛎需,莲粉糊丸盐下汤,夜寐无梦滑遗除。水陆二仙金樱芡,精遗带下两能祛。

桑螵蛸散

桑螵蛸散治便数,参苓龙骨同龟壳,菖蒲远志及当归,补肾宁心健忘却。

缩泉丸

缩泉丸治小便数,脬气虚寒失约束,山药台乌益智仁,糊丸服后收功速。
〔固崩止带〕

固经丸

固经丸用龟版君,黄柏椿皮香附群,黄芩芍药酒丸服,漏下崩中色黑殷。

樗树银丸(附:愈带丸)

樗树根丸良姜灰,黄柏樗皮白芍随,湿热阻胞成带下,淋漓腥秽此方推。愈带丸中熟地黄,归芍芎柏椿良姜,妇人冲任带失司,经浊淋漓宜此方。

易黄汤(《傅青主女科》)

易黄带下功效奇,山药白果并芡实,黄柏车前清湿热,一切湿饮并能医。
理气剂
〔行气〕

越鞠丸

越鞠丸治六郁侵,气血痰火湿食因,芎苍香附兼栀曲,气畅郁舒痛闷伸。

半夏厚朴汤

半夏厚朴气滞疏,茯苓生姜共紫苏,加枣同煎名四七,痰涎凝聚尽能祛。

加味乌药汤(附:乌药汤)

加味乌药汤砂仁,香附木香乌草伦,配入胡索共六味,月经胀痛效堪珍。加入当归去砂索,专治血海痛不宁。

瓜蒌薤白白酒汤

瓜蒌薤白治胸痹,配以白酒最相宜,加夏加朴枳桂枝,治法稍殊名亦异。

暖肝煎

暖肝煎中杞茯归,茴沉乌药合肉桂,下焦虚寒疝气痛,温补肝肾此方推。

天台乌药散(附:三层茴香丸)

天台乌药木茴香,青楝槟榔巴豆姜,三层茴香亦相当,寒疝腹痛是验方,除去巴豆青姜乌,增茯沙参荜附尝。

橘核丸

橘核丸是济生方,㿗疝痛顽正堪尝,朴实延胡藻带昆,楝桂桃香木通匡。

良附丸

良附丸中沉木香,青皮须共当归姜,单用良附名亦尔,逐寒止痛效验彰。

金铃子散(附:延胡索汤)

金铃子散止痛方,须辨作痛多种因,胡索金铃调酒下,制方原是远温辛。延胡索汤七情伤,血气刺痛服之良,归芍乳没香甘桂,胡索蒲黄生姜黄。

丹参饮

丹参饮用大砂仁,更入檀香妙绝伦,温胃化瘀兼理气,胃脘疼痛即舒伸。

启膈散

启膈散中用郁金,沙丹二参贝荷苓,杵头糠与砂仁壳,噎膈津枯法可饮。

化肝煎(《景岳全书》)

肝郁化火化肝煎,青陈栀泽贝芍丹,气舒郁火自然解,烦热满痛并能安。
〔降气〕

旋复代赭汤

旋复代赭用人参,半夏姜甘大枣增,噫气不除心下痞,虚中实证此方灵。

橘皮竹茹汤(附:济生橘皮竹茹汤、新制橘皮竹茹汤)

橘皮竹茹治呕呃,参甘姜枣效力捷、严氏济生方名同,加苓夏麦枇杷叶。新制橘皮竹茹汤,亦止吐逆与呕呃,去参枣草加柿蒂,鞠通立法疗胃热。

丁香柿蒂汤(附:柿蒂汤、柿钱散)

丁香柿蒂人参姜,呃逆因寒中气戕\footnote{},济生参去仅三味,呃逆不止胸满尝。柿钱散去生姜用,专治呃逆名亦彰。

注:

\footnote{}戕:qiāng,音腔。仿害、损害的意思。

大半夏汤(附:小半夏汤、干姜人参半夏丸)

大半夏汤治胃反,朝食薯吐急须挽,人参白蜜共三味,降逆培中法当赞。除去参蜜加生姜,支饮呕吐不渴鉴。仅去白蜜易干姜,妊娠呕吐治不难。

四磨饮(附:五磨饮子)

四磨饮治七情侵,人参乌药及槟沉,四味浓磨煎温服,实邪枳売易人参。去参加入木香枳,五磨饮子白酒斟。

定喘汤

定喘白果与麻黄,款冬半夏白皮桑,杏苏黄芩兼甘草,肺寒膈热喘哮尝。

苏子降气汤

苏子降气橘半归,前胡桂朴草姜依,下虚上盛痰嗽喘,或入沉香贵合机。
理血剂
〔活血祛瘀〕

桃仁承气汤(附:桃仁散)

桃仁承气五般施,甘草硝黄并桂枝,热结膀胱小腹胀,如狂蓄血最相宜。桃仁散用䗪虫奇,桂茯苡膝赭黄围,主治经来脐腹痛,寒热如疟勿迟疑。

抵当汤(附:抵当丸,下瘀血汤)

抵当直抵当攻处,大黄虻蛭桃仁聚,专治蓄血且发狂,痰血下行病自去。改汤为丸属缓剂,病勿急逐正可取。仅用䗪虫配桃黄,主治妇人血着脐。

大黄䗪虫丸

大黄䗪虫芩芍桃,地黄杏草漆蛴螬,虻虫水蛭和丸服,瘀去新生此剂豪\footnote{}

注:

\footnote{}豪:这里是出色的意思。

桂枝茯苓丸

仲景桂枝茯苓丸,丹芍桃仁共五般,等分炼蜜和丸服,活血化瘀癥块蠲。

温经汤(附:艾附暧官丸)

温经归芍桂萸芎,姜夏丹皮又麦冬,参草扶脾胶益血,调经亦可治崩中。艾附暖宫用四物,吴萸续断与桂芪,米醋糊丸醋汤下,妇人子官虚冷宜。

生化汤(附:黑神散)

生化汤宜产后尝,归芎桃草炮姜良,消瘀活血功偏擅,止痛温经效亦强。黑神散中有熟地,归芍甘草桂炮姜,蒲黄黑豆童便酒,消瘀下胎痛逆忘。

失笑散

失笑五灵与蒲黄,晕平痛止效非常,活血行瘀散结痛,产后血滞此方尝。

通窍生血汤(附:血府逐瘀汤、膈下逐瘀汤、少腹逐瘀汤)

通窍全凭好麝香,桃红大枣老葱姜,川芎黄酒赤芍药,表里通经第一方。血府当归生地桃,紅花甘草壳赤芍,柴胡芎桔牛膝等,血化下行不作劳。膈下逐瘀桃牡丹,赤芍乌药元胡甘,归芎灵脂红花壳,香附开郁血亦安。少腹茴香与炒姜,元胡灵脂没芎当,蒲黄官桂赤芍药,调经种子第一方。

复元活血汤

复元活血汤柴胡,花粉当归山甲同,桃仁红花大黄草,损伤瘀血酒煎除。

七厘散

七厘散是伤科方,血竭红花冰麝香,乳没儿茶朱共末,酒调内服外敷良。

三甲散(《瘟疫论》)

三甲散治主客交,龟鳖山甲草归芍,牡蛎䗪虫僵蝉蜕,正虚血结此方超。
〔止血〕

十灰散

十灰散用十般灰,柏茅茜荷丹榈煨,二蓟栀黄各炒黑,用此止血热势微。

四生丸

四生丸用三般叶,侧柏艾荷生地协,等分生捣如泥煎,血热妄行止衄惬\footnote{}。

注:

\footnote{}惬:qiè,音妾。满足的意思。

咳血方

咳血方中诃子收,蒌海山栀力最优,青黛蜜丸口噙化,咳嗽痰血服之瘳。

小蓟饮子

小蓟饮子藕蒲黄,木通滑石生地襄,归草黑栀淡竹叶,血淋热结服之良。

槐花散(附:槐角丸、脏连丸)

槐花散用治肠风,侧柏黑荆枳壳充,为末等分米饮下,宽肠凉血逐风功。槐角丸亦治肠风,防风归榆枳芩供,更有脏连疗下血,猪肠黄连法当宗。

黄土汤

黄土汤中术附芩,阿胶甘草地黄并,便后下血功专擅,吐衄崩中效亦灵。

胶艾汤

胶艾汤中四物先,阿胶艾叶草同煎,妇人良方单胶艾,胎动能安腹痛全。
治风剂
〔疏散外风〕

小续命汤(附:续命汤)

小续命汤桂附芎,麻黄参芍杏伤风,黄芩防己兼姜草,风中诸经以此通。除去风己附芩芍,生姜改作干姜充,再加当归与石膏,风痱肢废有殊功。

大秦艽汤

大秦艽用羌独防,芎芷辛芩二地黄,石膏归芍苓甘术,疏散风邪可通尝。

三生饮(附:星香散、大省风汤、青州白丸子)

三生饮用乌附星,三药生用木香听,加参扶正宗薛氏,风痰卒中效颇灵。若加生姜除乌附,专治咽喉痰上升。大省风汤即三生,防蝎独草去香煎,中风痰厥人昏倒,手足抽搐口潮涎。青州白丸亦三生,再加半夏威力添,晒露糊丸姜薄引,风痰瘫痪服之验。

牵正散

牵正散是杨家方,全蝎僵蚕白附襄,服用少量热酒下,风中络脉疗效彰。

独活寄生汤(附:三痹汤、羌活续断汤)

独活寄生艽防辛,芎归地芍桂苓均,杜仲牛膝人参草,冷风顽痹屈能伸。若去寄生加芪续,汤名三痹古方珍。或去独寄与甘草,加羌续芷痿痹行。

蠲痹汤(附:舒筋汤、程氏蠲痹汤)

蠲痹汤医风气痹,羌防归芍共黄芪,姜黄甘草姜煎服,体痛筋挛手足痱,除去芪防加桐术,肩腰作痛使筋舒。程氏蠲痹用桂羌,独艽芎归乳木香,甘草桑枝海风藤,风寒湿邪均能匡。

玉真散

玉真散治破伤风,牙关紧急反张弓,星麻白附羌防芷,外敷内服一方通。

川芎茶调散

川芎茶调散荆防,辛芷薄荷甘草羌,目昏鼻塞风攻上,偏正头痛悉平康。

大活络丹(附:小活络丹)

大活络丹五十味,三物四君为首位,羌防辛葛桂黄麻,麻安丁木沉藿附,附蔻青乌血竭研,乌连龙星骨碎补,二蛇蚕蝎乳没尖,牛黄冰麝犀龟虎,贯芩玄参何首乌,灵仙松脂足其数。小活络丹用胆星,二乌乳没地龙寻,酒丸酒下能通络,风血痰涎闭在经。

着痹验方(《中医方剂临床手册》)

验方著痹蝎蜈吞,地鳖蛇蜂虎骨伸,钻地老鹳蜣螂草,当归寻骨鹿衔斟,蛇虫搜剔能镇痛,畸形僵硬肾阳温。
〔平熄内风〕

羚角钩藤汤(附:钩藤饮、撮风散)

俞氏羚角钩藤汤,桑菊茯神鲜地黄,贝草竹茹同芍药,肝风内动急煎尝。钩藤饮用羚羊角,全蝎参麻炙草酌,小儿急惊牙关紧,手足抽搐效验卓。撮风散内直僵蚕,赤脚蜈蚣朱麝全,钩藤蝎尾共为末,口撮如囊服之痊。

阿胶鸡子黄汤

阿胶鸡子黄汤好,地芍钩藤牡炙草,石决茯神络石藤,阴虚风动此方保。

大定风珠(附:小定风珠)

大定风珠复脉商,再加三甲味鸡黄,脉虚瘛疭时欲脱,邪少虚多是妙方。小定风珠鸡子黄,阿龟童便淡菜尝,风病阴虚成厥逆,脉细筋劲奴弓弦。

镇肝熄风汤

镇肝熄风芍天冬,玄参牡蛎赭茵供,麦龟膝草龙川楝,肝风内动有奇功。

地黄饮子

地黄饮子山茱斛,麦味菖蒲远志茯,苁蓉桂附巴戟天,少入薄荷姜枣服。
祛湿剂
〔燥湿化浊〕

平胃散(附:不换金正气散、柴平汤)

平胃散是苍术朴,陈皮甘草四般药,除湿散满驱瘴疠,调胃化湿此方速。平胃散内功夏藿,金不换来正气复。小柴胡汤合平胃,寒多热少属湿疟。

六和汤

六和藿朴杏砂呈,半夏木瓜赤茯苓,参术扁豆同甘草,姜枣煎之六气平。

藿香正气散

藿香正气腹皮苏,甘桔陈苓术朴倶,夏曲白芷加姜枣,风寒暑湿并能驱。
〔清热利湿〕

甘露消毒丹

甘露消毒蔻藿香,茵陈滑石木通菖,芩翘贝母射干薄,暑疫湿温为末尝。

三仁汤(附:藿补夏苓汤)

三仁杏蔻薏苡仁,朴夏白通滑竹伦,水用甘澜扬百遍,湿温初起法堪遵。藿朴夏苓有三仁,猪苓泽泻豆豉群,主治湿温口不渴,胸闷口腻最有灵。

连朴饮

连朴饮内用豆豉,菖蒲半夏炒山栀,芦根厚朴黄连入,湿热霍乱此方施。

二妙散(附:三妙丸)

二妙散中苍柏兼,若云三妙牛膝添,痿痹足疾堪多服,湿热全除病自痊。

茵陈蒿汤(附:栀子柏皮汤、茵陈四逆汤)

茵陈蒿汤治黄疸,栀子大黄组成方。栀子柏皮加甘草,二方同属治阳黄。

八正散(附:石苇散、五淋散)

八正木通与车前,扁蓄大黄滑石研,草梢瞿麦兼栀子,煎加灯草效立见。石苇散用通车前,榆皮甘草赤茯苓,葵子瞿麦与滑石,主治茎痛砂石淋。五淋散用栀子仁,赤芍当归与茯苓,甘草生用为细末,热结能除水道澄。

中满分消丸(附:中满分消汤)

中满分消砂朴姜,芩连夏陈知泽襄,二苓参术姜黄草,枳实为丸效力彰。又有中满分消汤,乌泽连参青归姜,陈柴干姜茄益智,夏苓升麻与木香,吴萸厚朴草豆蔻,芪柏共煎中满匡。

燃照汤(《霍乱论》)

燃照汤中省头草,朴蔻香豉肠胃保,滑夏栀芩并清热,骤作吐泻凉服好。

蚕矢汤(《霍乱论》)

蚕矢汤中用蚕砂,豆卷栀苡芩连夏,湿热壅郁经络拘,还入吴萸通木瓜。

薏苡竹叶散(《温病条辨》)

薏苡竹叶治白痦,通苓滑蔻翘草随,汗多自利身热痛,辛凉淡泄法堪垂。
〔利水化湿〕

五苓散(附:四苓散、茵陈五苓散、胃苓汤)

五苓散治太阳腑,白术泽泻猪苓茯,桂枝化气兼解表,小便通行水饮逐。除却桂枝名四苓,五苓散加茵陈助。平胃五苓二散合,行气燥湿法兼顾。

猪苓汤

猪苓汤内二苓全,泽泻阿胶滑石添,利水育阴兼泻热,溺秘心烦呕渴痊。

五皮散

五皮散用五般皮,陈茯姜桑大腹奇,或以五加易桑白,脾虚腹胀此方宜。

防己黄芪汤(附:防己茯苓汤)

防己黄芪金匮方,白术甘草枣生姜,此疗风水与诸湿,汗出恶风身重尝。方中去枣白术姜,加入茯苓桂枝良,水在皮肤聂聂动,可用防己茯苓汤。
〔温化水湿〕

苓桂术甘汤

苓桂术甘蠲饮剂,崇脾以利膀胱气,饮邪上逆气冲胸,胸满能愈眩晕弃。

甘草干姜茯苓白术汤

肾着汤内用干姜,茯苓甘草白术襄,伤湿身痛腰冷重,亦名干姜苓术汤。

实脾散

实脾苓术与木瓜,甘草木香大腹加,草蔻姜附兼厚朴,虚寒阴水效堪夸。

萆薢分清饮(附:程氏萆解分清饮)

萆薢分清石菖蒲,草梢乌药智仁俱,或益茯苓盐煎服,淋浊留连数剂除。心悟萆薢分清饮,清利湿热功效准,萆柏菖苓丹白术,车前子与莲子心。
〔宣散湿邪〕

羌活胜湿汤

羌活胜湿羌独芎、蔓苷藁本与防风,湿邪在表头腰重,发汗升阳有异功。

鸡鸣散

鸡鸣散是绝奇方,苏叶吴萸桔梗姜,瓜橘槟榔煎冷服,肿浮脚气效验彰。

五叶芦根汤(《湿热病篇》)

五叶芦根化湿方,藿佩枇杷二荷汤,芦尖瓜瓣轻芳化,湿热蒙蔽透清阳。

瓜蒌瞿麦丸(《金匮要略》)

金匮瓜蒌瞿麦丸,山药茯苓水气宣,苦渴仍以附子化,阴阳补泻此方全。
润燥剂
〔轻宣润燥〕

杏苏散

杏苏散内夏陈前,甘桔枳苓姜枣研,轻宣温润治凉燥,服后微汗病自痊。

桑杏汤(附:翘荷汤)

桑杏汤用象贝宜,沙参扼豉与梨皮。翘荷甘桔焦栀豆,清宣凉润燥能医。

清燥救肺汤

清燥救肺参草杷,石膏胶杏麦胡麻,经霜收下冬桑叶,清肺润燥效堪夸。

沙参麦冬汤

沙参麦冬扁豆桑,甘草玉粉合成方,秋燥耗津伤肺胃,苔光干咳最堪尝。

清咽栀豉汤(《疫喉浅论》、《温病学》)

热盛清咽扼豉汤,银翘蚕豉桔牛蒡,马勃薄荷栀犀草,透邪清里两解方。

清咽养营汤(《疫喉浅论》、《温病学》)

余毒伤阴养营汤,甘苦合法增液汤,天冬知芍茯神草,洋参花粉共煎尝。
〔甘寒滋润〕

琼玉膏(附:臞仙琼玉膏)

琼玉膏中生地黄,参苓白蜜炼膏尝,肺枯干咳虚劳证,金水相生效倍彰。又有臞仙琼玉膏,原方加沉琥半两,喉中血腥肠隐痛,嗜酒久咳效尤强。

百合固金汤

百合固金二地黄,玄参贝母甘桔藏,麦冬芍药当归配,喘咳痰血肺家伤。

养阴清肺汤

郑氏养阴清肺汤,玄参甘芍冬地黄,薄荷贝母丹皮入,时疫白喉是妙方。

五汁饮

五汁饮用五般汁,梨麦藕荠苇根给,或以蔗浆来易藕,甘寒救液合医律\footnote{}。

注:

\footnote{}律:法则、规章。

麦门冬汤(附:加减麦门冬汤)

麦门冬汤半夏参,枣甘粳米共合成,咽喉不利因虚火,养胃除烦逆气平。若于方中除粳米,山药丹芍桃仁增。

增液汤

增液汤用参地冬,滋阴生津大有功,阳明温热肠燥结,水津四布大便通。

玉液汤(《医学衷中参西录》)

玉液消渴方中珍,山药芪葛鸡内金,花粉知味滋兼敛,气升液至立意新。
祛痰剂
〔燥湿化痰〕

二陈汤(附:温胆汤、导痰汤、金水六君煎)

二陈汤用夏和陈,益以茯苓甘草臣,利气调中兼去湿,一切痰饮此为珍。若加竹茹与枳实,汤名温胆可宁神。导痰汤内加星枳,顽痰胶固力能驯。涤痰汤用半夏星,甘草橘红参茯苓,竹茹菖蒲兼枳实,痰迷舌强服之醒。金水六君用二陈,再加熟地与归身,肺肾阴虚兼有痰,滋阴化痰效颇灵。

指迷茯苓丸

指迷茯苓丸半夏,风硝枳壳姜汤下,中脘停痰肩臂疼,脉来沉细留心把。
〔润燥化痰〕

贝母瓜蒌散(附:程氏贝母瓜蒌散)

贝母瓜蒌散茯苓,橘红花桔共煎成,呛咳哽痛咽喉燥,润燥化痰奏效频。程氏散中除茯苓,花粉桔梗加连芩,山栀胆星甘草组,主治类风痰热盛。
〔清热化痰〕

清气化痰丸

清气化痰星夏橘,杏仁枳实瓜蒌实,苓苓姜汁为糊丸,气顺火消痰自失。

小陷胸汤(附:柴胡陷胸汤)

小陷胸汤连夏蒌,宽胸开结涤痰优,脉浮滑兮按之痛,不按自疼大陷求。柴胡陷胸汤合成,清化痰热功颇灵,除却人参甘大枣,加实桔连瓜蒌仁。

滚痰丸(附:竹沥达痰丸)

滚痰丸用青礞石,大黄黄芩沉香辑,诸证多因痰作怪,顽痰固结可消失。竹沥达痰滚痰丸,沉礞黄芩及大黄,茯苓桔夏人参草,姜汁竹沥虚痰匡。

冷哮丸

冷哮丸用麻乌辛,蜀椒白矾陈胆星,皂杏半曲甘紫款,痰结冷喘此方珍。
〔治风化痰〕

止嗽散

止嗽散中用桔梗,紫菀荆芥百部前,陈皮甘草共为末,姜汤调下约三钱。

半夏白术天麻汤

半夏白术天麻汤,苓草橘红大枣姜,痰饮头痛兼眩晕,热盛阴亏切莫尝。
〔其他〕

三子养亲汤

三子养亲祛痰方,芥苏莱菔共煎汤,大便素实加熟蜜,冬寒可更加生姜。
驱虫剂

乌梅丸

乌梅丸用细辛桂,人参附子椒姜继,黄连黄柏及当归,温脏安蛔寒厥剂。

理中安蛔汤

理中方内有安蛔,姜术参苓椒与梅,呕吐胃寒胸膈痛,温脾扶土去虫灾。

肥儿丸

肥儿丸内用使君,豆蔻香连曲麦槟,猪胆为丸热水下,疳虫积食尽离身。
痈疡剂
〔痈疡阳证〕

仙方活命饮(附:冲和汤)

仙方活命金银花,防芷归陈草节加,贝母花粉兼乳没,穿山皂刺酒煎嘉,一切痈疽能溃散,溃后忌服用毋差。冲和汤用术归芪,参陈苓芎皂白芷,乳没银花甘草节,酒水各半煎服宜,半阴半阳疮疡证,似溃非溃效力奇。

银花解毒汤(附:五味消毒饮)

银花解毒地丁翘,犀角丹皮赤苓饶,更用川连夏枯草,痈疽疔毒服之消。五味消毒疗诸疔,银花野菊蒲公英,紫花地丁天葵子,煎加酒服勿看轻。

犀黄丸(附:醒消丸)

犀黄丸内用麝香,乳香没药与牛黄,乳岩流注肠痈等,正气未虚均可尝。醒消丸治红肿毒,乳没麝香雄精逐,米饭为丸菔子大,服用酒送效力速。

蟾酥丸

蟾酥丸用寒水石,麝朱乳没胆矾枯,轻粉铜绿雄蜗入,疔毒内服又外敷。

牛蒡解肌汤

牛蒡解肌石斛丹,夏枯荆芥薄荷餐,连翘栀子玄参入,风热疮疡勿等闲。

五神汤

五神汤用紫地丁,牛膝车前白茯苓,银花三两水煎服,湿热痈疮功可钦。

消瘰丸

消瘰丸用贝母参,解郁化痰阴气增,瘰疬津虚痰火结,加减临时细酌斟。

透脓散(附:程氏透脓散、代刀散、托里透脓汤)

透脓散治毒成脓,服此能成速溃功,川芎归芪甲片皂,加芷蒡银效方雄。去芎归甲加乳草,方名代刀力无穷。托里透脓参术芷,当归甘草与皂刺,黄芪青麻穿山甲,痈疽发背用最宜。
〔痈疡阴证〕

阳和汤

阳和汤法解寒凝,外证虚寒色属阴,熟地鹿胶姜炭桂,麻黄白芥草相承。

小金丹

小金丹内白胶香,木鳖地龙乳麝香,归没五灵草乌墨,流注瘰疬服之良。
〔内痈〕

苇茎汤

苇茎汤方出千金,桃仁薏苡冬瓜仁,瘀热肺脏成痈毒,甘寒清热上焦宁。

大黄牡丹汤(附:清肠饮、红藤煎)

金匮大黄牡丹汤,桃仁瓜子芒硝襄,肠痈初起腹按痛,尚未成脓服之康。清肠饮用金银花,玄芩麦榆归草加,苡仁五钱为煎剂,肠痈腹痛效不差。红藤煎用金银花,红藤乳没翘索佳,大黄丹皮地丁草,肠痈脓肿可代茶。

薏苡附子败酱散

薏苡附子败酱散,主治肠痈毒可斩,肌肤甲错腹皮急,排脓消肿方堪赞。

四妙勇安汤(《验方新编》)

四妙勇安治脱疽,银玄归草合用宜。四药清热且解毒,活血散,奏效奇。


\backmatter

\end{document}