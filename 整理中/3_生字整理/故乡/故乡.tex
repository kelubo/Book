% 故乡
% 故乡.tex

\documentclass[12pt,UTF8]{ctexbook}

% 设置纸张信息。
\usepackage[a4paper,twoside]{geometry}
\geometry{
	left=25mm,
	right=25mm,
	bottom=25.4mm,
	bindingoffset=10mm
}

% 设置字体,并解决显示难检字问题。
\xeCJKsetup{AutoFallBack=true}
\setCJKmainfont{SimSun}[BoldFont=SimHei, ItalicFont=KaiTi, FallBack=SimSun-ExtB]

% 目录 chapter 级别加点(.)。
\usepackage{titletoc}
\titlecontents{chapter}[0pt]{\vspace{3mm}\bf\addvspace{2pt}\filright}{\contentspush{\thecontentslabel\hspace{0.8em}}}{}{\titlerule*[8pt]{.}\contentspage}

% 设置 part 和 chapter 标题格式。
\ctexset{
	part/name= {第,卷},
	part/number={\chinese{part}},
	chapter/name={第,篇},
	chapter/number={\chinese{chapter}}
}

% 设置署名格式。
\newenvironment{shuming}{\hfill}

% 注脚每页重新编号,避免编号过大。
\usepackage[perpage]{footmisc}

\title{\heiti\zihao{0} 故乡}
\author{鲁迅}
\date{}

\begin{document}

\maketitle
\tableofcontents

\frontmatter

我冒了严寒,回到相隔二千余里,别了二十余年的故乡去。

时候既然是深冬;渐近故乡时,天气又阴晦了,冷风吹进船舱中,呜呜的响,从篷隙向外一望,苍黄的天底下,远近横着几个萧索的荒村,没有一些活气。我的心禁不住悲凉起来了。

阿!这不是我二十年来时记得的故乡?

我所记得的故乡全不如此。我的故乡好得多了。但要我记起他的美丽,说出他的佳处来,却又没有影像,没有言辞了。仿佛也就如此。于是我自己解释说:故乡本也如此,——虽然没有进步,也未必有如我所感的悲凉,这只是我自己心情的改变罢了,因为我这次回乡,本没有什么好心绪。

我这次是专为了别他而来的。我们多年聚族而居的老屋,已经公同卖给别姓了,交屋的期限,只在本年,所以必须赶在正月初一以前,永别了熟识的老屋,而且远离了熟识的故乡,搬家到我在谋食的异地去。

第二日清早晨我到了我家的门口了。瓦楞上许多枯草的断茎当风抖着,正在说明这老屋难免易主的原因。几房的本家大约已经搬走了,所以很寂静。我到了自家的房外,我的母亲早已迎着出来了,接着便飞出了八岁的侄儿宏儿。

我的母亲很高兴,但也藏着许多凄凉的神情,教我坐下,歇息,喝茶,且不谈搬家的事。宏儿没有见过我,远远的对面站着只是看。

但我们终于谈到搬家的事。我说外间的寓所已经租定了,又买了几件家具,此外须将家里所有的木器卖去,再去增添。母亲也说好,而且行李也略已集齐,木器不便搬运的,也小半卖去了,只是收不起钱来。

“你休息一两天,去拜望亲戚本家一回,我们便可以走了。”母亲说。

“是的。”

“还有闰土,他每到我家来时,总问起你,很想见你一回面。我已经将你到家的大约日期通知他,他也许就要来了。”

这时候,我的脑里忽然闪出一幅神异的图画来:深蓝的天空中挂着一轮金黄的圆月,下面是海边的沙地,都种着一望无际的碧绿的西瓜,其间有一个十一二岁的少年,项带银圈,手捏一柄钢叉,向一匹猹\footnote{作者在 1929 年 5 月 4 日致舒新城的信中说:“‘猹’字是我据乡下人所说的声音,生造出来的,读如‘查’。但我自己也不知道究竟是怎样的动物,因为这乃是闰土所说,别人不知其详。现在想起来,也许是獾罢。”}尽力的刺去,那猹却将身一扭,反从他的胯下逃走了。

这少年便是闰土。我认识他时,也不过十多岁,离现在将有三十年了;那时我的父亲还在世,家景也好,我正是一个少爷。那一年,我家是一件大祭祀的值年\footnote{封建社会中的大家族,每年都有祭祀祖先的活动,费用从族中“祭产”收入支取,由各房按年轮流主持,轮到的称为“值年”。}。这祭祀,说是三十多年才能轮到一回,所以很郑重;正月里供祖像,供品很多,祭器很讲究,拜的人也很多,祭器也很要防偷去。我家只有一个忙月(我们这里给人做工的分三种:整年给一定人家做工的叫长工;按日给人家做工的叫短工;自己也种地,只在过年过节以及收租时候来给一定的人家做工的称忙月),忙不过来,他便对父亲说,可以叫他的儿子闰土来管祭器的。

我的父亲允许了;我也很高兴,因为我早听到闰土这名字,而且知道他和我仿佛年纪,闰月生的,五行缺土\footnote{旧时所谓算“八字”的迷信说法。即用天干(甲乙丙丁戊己庚辛壬癸)和地支(子丑寅卯辰巳午未申酉戌亥)相配,来记一个人出生的年、月、日、时,各得两字,合为“八字”;又认为它们在五行(金、木、水、火、土)中各有所属,如甲乙寅卯属木,丙丁巳午属火等等,如八个字能包括五者,就是五行俱全。“五行缺土”,就是这八个字中没有属土的字,需用土或土作偏旁的字取名等办法来弥补。},所以他的父亲叫他闰土。他是能装弶捉小鸟雀的。

我于是日日盼望新年,新年到,闰土也就到了。好容易到了年末,有一日,母亲告诉我,闰土来了,我便飞跑的去看。他正在厨房里,紫色的圆脸,头戴一顶小毡帽,颈上套一个明晃晃的银项圈,这可见他的父亲十分爱他,怕他死去,所以在神佛面前许下愿心,用圈子将他套住了。他见人很怕羞,只是不怕我,没有旁人的时候,便和我说话,于是不到半日,我们便熟识了。

我们那时候不知道谈些什么,只记得闰土很高兴,说是上城之后,见了许多没有见过的东西。

第二日,我便要他捕鸟。他说:

“这不能。须大雪下了才好。我们沙地上,下了雪,我扫出一块空地来,用短棒支起一个大竹匾,撒下秕谷,看鸟雀来吃时,我远远地将缚在棒上的绳子只一拉,那鸟雀就罩在竹匾下了。什么都有:稻鸡,角鸡,鹁鸪,蓝背......”

我于是又很盼望下雪。

闰土又对我说:

“现在太冷,你夏天到我们这里来。我们日里到海边检贝壳去,红的绿的都有,鬼见怕也有,观音手\footnote{都是小贝壳的名称。旧时浙江沿海的人把这种小贝壳用线串在一起,戴在孩子的手腕或脚踝上,认为可以“避邪”。这类名称多是根据“避邪”的意思取的。}也有。晚上我和爹管西瓜去,你也去。”

“管贼么?”

“不是。走路的人口渴了摘一个瓜吃,我们这里是不算偷的。要管的是獾猪,刺猬,猹。月亮地下,你听,啦啦的响了,猹在咬瓜了。你便捏了胡叉,轻轻地走去......”

我那时并不知道这所谓猹的是怎么一件东西——便是现在也没有知道——只是无端的觉得状如小狗而很凶猛。

“他不咬人么?”

“有胡叉呢。走到了,看见猹了,你便刺。这畜生很伶俐,倒向你奔来,反从胯下窜了。他的皮毛是油一般的滑......”

我素不知道天下有这许多新鲜事:海边有如许五色的贝壳;西瓜有这样危险的经历,我先前单知道他在水果店里出卖罢了。

“我们沙地里,潮汛要来的时候,就有许多跳鱼儿只是跳,都有青蛙似的两个脚......”

阿!闰土的心里有无穷无尽的希奇的事,都是我往常的朋友所不知道的。他们不知道一些事,闰土在海边时,他们都和我一样只看见院子里高墙上的四角的天空。

可惜正月过去了,闰土须回家里去,我急得大哭,他也躲到厨房里,哭着不肯出门,但终于被他父亲带走了。他后来还托他的父亲带给我一包贝壳和几支很好看的鸟毛,我也曾送他一两次东西,但从此没有再见面。

现在我的母亲提起了他,我这儿时的记忆,忽而全都闪电似的苏生过来,似乎看到了我的美丽的故乡了。我应声说:

“这好极!他,——怎样?……”

“他?......他景况也很不如意......”母亲说着,便向房外看,“这些人又来了。说是买木器,顺手也就随便拿走的,我得去看看。”

母亲站起身,出去了。门外有几个女人的声音。我便招宏儿走近面前,和他闲话:问他可会写字,可愿意出门。

“我们坐火车去么?”

“我们坐火车去。”

“船呢?”

“先坐船,......”

“哈!这模样了!胡子这么长了!”一种尖利的怪声突然大叫起来。

我吃了一吓,赶忙抬起头,却见一个凸颧骨,薄嘴唇,五十岁上下的女人站在我面前,两手搭在髀间,没有系裙,张着两脚,正像一个画图仪器里细脚伶仃的圆规。

我愕然了。

“不认识了么?我还抱过你咧!”

我愈加愕然了。幸而我的母亲也就进来,从旁说:

“他多年出门,统忘却了。你该记得罢,”便向着我说,“这是斜对门的杨二嫂,......开豆腐店的。”

哦,我记得了。我孩子时候,在斜对门的豆腐店里确乎终日坐着一个杨二嫂,人都叫伊“豆腐西施\footnote{春秋时苎罗(今浙江诸暨南)人,越国的美女。后借以泛称一般美女。}”。但是擦着白粉,颧骨没有这么高,嘴唇也没有这么薄,而且终日坐着,我也从没有见过这圆规式的姿势。那时人说:因为伊,这豆腐店的买卖非常好。但这大约因为年龄的关系,我却并未蒙着一毫感化,所以竟完全忘却了。然而圆规很不平,显出鄙夷的神色,仿佛嗤笑法国人不知道拿破仑,美国人不知道华盛顿似的,冷笑说:

“忘了?这真是贵人眼高......”

“那有这事......我......”我惶恐着,站起来说。

“那么,我对你说。迅哥儿,你阔了,搬动又笨重,你还要什么这些破烂木器,让我拿去罢。我们小户人家,用得着。”

“我并没有阔哩。我须卖了这些,再去......”

“阿呀呀,你放了道台\footnote{清朝官职道员的俗称,分总管一个区域行政职务的道员和专掌某一特定职务的道员。前者是省以下、府州以上的行政长官;后者掌管一省特定事务,如督粮道、兵备道等。辛亥革命后,北洋政府也曾沿用此制,改称道尹。}了,还说不阔?你现在有三房姨太太;出门便是八抬的大轿,还说不阔?吓,什么都瞒不过我。”

我知道无话可说了,便闭了口,默默的站着。

“阿呀阿呀,真是愈有钱,便愈是一毫不肯放松,愈是一毫不肯放松,便愈有钱......”圆规一面愤愤的回转身,一面絮絮的说,慢慢向外走,顺便将我母亲的一副手套塞在裤腰里,出去了。

此后又有近处的本家和亲戚来访问我。我一面应酬,偷空便收拾些行李,这样的过了三四天。

一日是天气很冷的午后,我吃过午饭,坐着喝茶,觉得外面有人进来了,便回头去看。我看时,不由的非常出惊,慌忙站起身,迎着走去。

这来的便是闰土。虽然我一见便知道是闰土,但又不是我这记忆上的闰土了。他身材增加了一倍;先前的紫色的圆脸,已经变作灰黄,而且加上了很深的皱纹;眼睛也像他父亲一样,周围都肿得通红,这我知道,在海边种地的人,终日吹着海风,大抵是这样的。他头上是一顶破毡帽,身上只一件极薄的棉衣,浑身瑟索着;手里提着一个纸包和一支长烟管,那手也不是我所记得的红活圆实的手,却又粗又笨而且开裂,像是松树皮了。

我这时很兴奋,但不知道怎么说才好,只是说:

“阿!闰土哥,———你来了?......”

我接着便有许多话,想要连珠一般涌出:角鸡,跳鱼儿,贝壳,猹,......但又总觉得被什么挡着似的,单在脑里面回旋,吐不出口外去。

他站住了,脸上现出欢喜和凄凉的神情;动着嘴唇,却没有作声。他的态度终于恭敬起来了,分明的叫道:

“老爷!......”

我似乎打了一个寒噤;我就知道,我们之间已经隔了一层可悲的厚障壁了。我也说不出话。

他回过头去说,“水生,给老爷磕头。”便拖出躲在背后的孩子来,这正是一个廿年前的闰土,只是黄瘦些,颈子上没有银圈罢了。“这是第五个孩子,没有见过世面,躲躲闪闪......”

母亲和宏儿下楼来了,他们大约也听到了声音。

“老太太。信是早收到了。我实在喜欢的了不得,知道老爷回来......”闰土说。

“阿,你怎的这样客气起来。你们先前不是哥弟称呼么?还是照旧:迅哥儿。”母亲高兴的说。

“阿呀,老太太真是......这成什么规矩。那时是孩子,不懂事......”闰土说着,又叫水生上来打拱,那孩子害羞,紧紧的只贴在他背后。

“他就是水生?第五个?都是生人,怕生也难怪的;还是宏儿和他去走走。”母亲说。

宏儿听得这话,便来招水生,水生却松松爽爽同他一路出去了。母亲叫闰土坐,他迟疑了一回,终于就了坐,将长烟管靠在桌旁,递过纸包来,说:

“冬天没有什么东西了。这一点干青豆倒是自家晒在那里的,请老爷......”

我问问他的景况。他只是摇头。

“非常难。第六个孩子也会帮忙了,却总是吃不够......又不太平......什么地方都要钱,没有定规......收成又坏。种出东西来,挑去卖,总要捐几回钱,折了本;不去卖,又只能烂掉......”

他只是摇头;脸上虽然刻着许多皱纹,却全然不动,仿佛石像一般。他大约只是觉得苦,却又形容不出,沉默了片时,便拿起烟管来默默的吸烟了。

母亲问他,知道他的家里事务忙,明天便得回去;又没有吃过午饭,便叫他自己到厨下炒饭吃去。

他出去了;母亲和我都叹息他的景况:多子,饥荒,苛税,兵,匪,官,绅,都苦得他像一个木偶人了。母亲对我说,凡是不必搬走的东西,尽可以送他,可以听他自己去拣择。

下午,他拣好了几件东西:两条长桌,四个椅子,一副香炉和烛台,一杆抬秤。他又要所有的草灰(我们这里煮饭是烧稻草的,那灰,可以做沙地的肥料),待我们启程的时候,他用船来载去。

夜间,我们又谈些闲天,都是无关紧要的话;第二天早晨,他就领了水生回去了。

又过了九日,是我们启程的日期。闰土早晨便到了,水生没有同来,却只带着一个五岁的女儿管船只。我们终日很忙碌,再没有谈天的工夫。来客也不少,有送行的,有拿东西的,有送行兼拿东西的。待到傍晚我们上船的时候,这老屋里的所有破旧大小粗细东西,已经一扫而空了。

我们的船向前走,两岸的青山在黄昏中,都装成了深黛颜色,连着退向船后梢去。

宏儿和我靠着船窗,同看外面模糊的风景,他忽然问道:

“大伯!我们什么时候回来?”

“回来?你怎么还没有走就想回来了。”

“可是,水生约我到他家玩去咧......”他睁着大的黑眼睛,痴痴的想。

我和母亲也都有些惘然,于是又提起闰土来。母亲说,那豆腐西施的杨二嫂,自从我家收拾行李以来,本是每日必到的,前天伊在灰堆里,掏出十多个碗碟来,议论之后,便定说是闰土埋着的,他可以在运灰的时候,一齐搬回家里去;杨二嫂发见了这件事,自已很以为功,便拿了那狗气杀(这是我们这里养鸡的器具,木盘上面有着栅栏,内盛食料,鸡可以伸进颈子去啄,狗却不能,只能看着气死),飞也似的跑了,亏伊装着这么高底的小脚,竟跑得这样快。

老屋离我愈远了;故乡的山水也都渐渐远离了我,但我却并不感到怎样的留恋。我只觉得我四面有看不见的高墙,将我隔成孤身,使我非常气闷;那西瓜地上的银项圈的小英雄的影像,我本来十分清楚,现在却忽地模糊了,又使我非常的悲哀。

母亲和宏儿都睡着了。

我躺着,听船底潺潺的水声,知道我在走我的路。我想:我竟与闰土隔绝到这地步了,但我们的后辈还是一气,宏儿不是正在想念水生么。我希望他们不再像我,又大家隔膜起来......然而我又不愿意他们因为要一气,都如我的辛苦展转而生活,也不愿意他们都如闰土的辛苦麻木而生活,也不愿意都如别人的辛苦恣睢而生活。他们应该有新的生活,为我们所未经生活过的。

我想到希望,忽然害怕起来了。闰土要香炉和烛台的时候,我还暗地里笑他,以为他总是崇拜偶像,什么时候都不忘却。现在我所谓希望,不也是我自己手制的偶像么?只是他的愿望切近,我的愿望茫远罢了。

我在朦胧中,眼前展开一片海边碧绿的沙地来,上面深蓝的天空中挂着一轮金黄的圆月。我想:希望是本无所谓有,无所谓无的。这正如地上的路;其实地上本没有路,走的人多了,也便成了路。

\begin{shuming}
一九二一年一月\footnote{本篇最初发表于 1921 年 5 月《新青年》第九卷第一号。}。
\end{shuming}

\backmatter

\end{document}