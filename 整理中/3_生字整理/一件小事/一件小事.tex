% 一件小事
% 一件小事.tex

\documentclass[12pt,UTF8]{ctexbook}

% 设置纸张信息。
\usepackage[a4paper,twoside]{geometry}
\geometry{
	left=25mm,
	right=25mm,
	bottom=25.4mm,
	bindingoffset=10mm
}

% 设置字体,并解决显示难检字问题。
\xeCJKsetup{AutoFallBack=true}
\setCJKmainfont{SimSun}[BoldFont=SimHei, ItalicFont=KaiTi, FallBack=SimSun-ExtB]

% 目录 chapter 级别加点(.)。
\usepackage{titletoc}
\titlecontents{chapter}[0pt]{\vspace{3mm}\bf\addvspace{2pt}\filright}{\contentspush{\thecontentslabel\hspace{0.8em}}}{}{\titlerule*[8pt]{.}\contentspage}

% 设置 part 和 chapter 标题格式。
\ctexset{
	chapter/name={},
	chapter/number={}
}

% 设置署名格式。
\newenvironment{shuming}{\hfill}

% 注脚每页重新编号,避免编号过大。
\usepackage[perpage]{footmisc}

\title{\heiti\zihao{0} 一件小事}
\author{鲁迅}
\date{}

\begin{document}

\maketitle
\tableofcontents

\frontmatter
\chapter{前言、序言}

\mainmatter

我从乡下跑到京城里,一转眼已经六年了。其间耳闻目睹的所谓国家大事,算起来也很不少;但在我心里,都不留什么痕迹,倘要我寻出这些事的影响来说,便只是增长了我的坏脾气,——老实说,便是教我一天比一天的看不起人。

但有一件小事,却于我有意义,将我从坏脾气里拖开,使我至今忘记不得。

这是民国六年的冬天,大北风刮得正猛,我因为生计关系,不得不一早在路上走。一路几乎遇不见人,好容易才雇定了一辆人力车,教他拉到S门去。不一会,北风小了,路上浮尘早已刮净,剩下一条洁白的大道来,车夫也跑得更快。刚近S门,忽而车把上带着一个人,慢慢地倒了。

跌倒的是一个女人,花白头发,衣服都很破烂。伊从马路边上突然向车前横截过来;车夫已经让开道,但伊的破棉背心没有上扣,微风吹着,向外展开,所以终于兜着车把。幸而车夫早有点停步,否则伊定要栽一个大觔\footnote{j\`in}斗,跌到头破血出了。

伊伏在地上;车夫便也立住脚。我料定这老女人并没有伤,又没有别人看见,便很怪他多事,要自己惹出是非,也误了我的路。

我便对他说,“没有什么的。走你的罢!”

车夫毫不理会,——或者并没有听到,——却放下车子,扶那老女人慢慢起来,搀着臂膊立定,问伊说:

“您怎么啦?”

“我摔坏了。”

我想,我眼见你慢慢倒地,怎么会摔坏呢,装腔作势罢了,这真可憎恶。车夫多事,也正是自讨苦吃,现在你自己想法去。

车夫听了这老女人的话,却毫不踌躇,仍然搀着伊的臂膊,便一步一步的向前走。我有些诧异,忙看前面,是一所巡警分驻所,大风之后,外面也不见人。这车夫扶着那老女人,便正是向那大门走去。

我这时突然感到一种异样的感觉,觉得他满身灰尘的后影,刹时高大了,而且愈走愈大,须仰视才见。而且他对于我,渐渐的又几乎变成一种威压,甚而至于要榨出皮袍下面藏着的“小”来。

我的活力这时大约有些凝滞了,坐着没有动,也没有想,直到看见分驻所里走出一个巡警,才下了车。

巡警走近我说:“你自己雇车罢,他不能拉你了。”

我没有思索的从外套袋里抓出一大把铜元,交给巡警,说,“请你给他……”

风全住了,路上还很静。我走着,一面想,几乎怕敢想到我自己。以前的事姑且搁起,这一大把铜元又是什么意思,奖他么?我还能裁判车夫么?我不能回答自己。

这事到了现在,还是时时记起。我因此也时时煞了苦痛,努力的要想到我自己。几年来的文治武力,在我早如幼小时候所读过的“子曰诗云”一般,背不上半句了。独有这一件小事,却总是浮在我眼前,有时反更分明,教我惭愧,催我自新,并且增长我的勇气和希望。

\begin{shuming}
一九二〇年七月\footnote{本篇最初发表于1919年12月1日北京《晨报·周年纪念增刊》。据报刊发表的年月及鲁迅日记,本篇写作时间当在1919年11月。}
\end{shuming}
  
\end{document}