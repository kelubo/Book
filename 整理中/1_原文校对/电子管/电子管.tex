% 电子管
% 电子管.tex

\documentclass[12pt,UTF8]{ctexbook}

% 设置纸张信息。
% 纸张设置配置文件
% 用于定义书籍的页面尺寸和边距

\usepackage[a4paper,twoside]{geometry}
\geometry{
	left=25mm,
	right=20mm,
	top=25mm,
	bottom=25.4mm,
	headsep=1cm, 
    footskip=1cm,
	bindingoffset=10mm
}

% 设置字体,并解决显示难检字问题。
\xeCJKsetup{AutoFallBack=true}
\setCJKmainfont{SimSun}[BoldFont=SimHei, ItalicFont=KaiTi, FallBack=SimSun-ExtB]

% 目录 chapter 级别加点(.)。
\usepackage{titletoc}
\titlecontents{chapter}[0pt]{\vspace{3mm}\bf\addvspace{2pt}\filright}{\contentspush{\thecontentslabel\hspace{0.8em}}}{}{\titlerule*[8pt]{.}\contentspage}

% 设置 part 和 chapter 标题格式。
\ctexset{
	part/name= {第,卷},
	part/number={\arabic{part}},
	chapter/name={第,篇},
	chapter/number={\arabic{chapter}}
}

% 图片相关设置。
\usepackage{graphicx}
\graphicspath{{Images/}}

% 设置署名格式。
\newenvironment{shuming}{\hfill\zihao{4}}

% 注脚每页重新编号,避免编号过大。
\usepackage[perpage]{footmisc}

\title{\heiti\zihao{0} 电子管}
\author{}
\date{}

\begin{document}

\maketitle
\tableofcontents

\frontmatter

\chapter{前言}



\mainmatter

\part{电子管基础}

电子管的种类非常之多,如按电极的数量来区分,就有二极管、三极管、五极管、复合管等多种。

\chapter{原理}

热电子发射

金属是由许多相同的原子集合而成的,这些原子中的自由电子都在时刻不停地作着剧烈的杂乱运动。但在常温下,自由电子受到金属原子内部带正电荷的原子核对它们的吸引力,而不能从金属表面发射出来。因此,要使电子离开金属表面,就必须克服金属原子内部原子核对它们的吸引力,克服这种吸引力所需作的功称为逸出功。

如果把金属的温度加以提高,金属原子内部自由电子的动能将大大增加,当它们的动能超过逸出功时,就将有大量的电子逸出金属表面,即有大量的自由电子发射到空间去。这种现象称为热电子发射。

常用的电子管是依靠热电子发射产生的自由电子在真空中流动而工作的。

\chapter{二极管}

二极管是最简单的一种电子管。它本身是一个密封的玻璃管(或金属管),管内被抽成高度真空。其中装有两个电极,一个是阴极,另一个是屏极。

阴极是用来发射电子的。通常发射电子的方法是采用热电子发射。这就是通过电流将金属加热,使金属内部的自由电子获得足够的动能,使之能克服金属表面原子的原子核的吸引力而离开金属。

阴极按加热的方式又分为直热式和旁热式两种。直热式阴极通常叫做灯丝,它是用钨丝或敷钍钨丝扭成各种形状支撑起来的,它的构造和符号如图2.1所示。这种阴极的加热电流是直接流入灯丝,而热电子也是从灯丝发射出来的。钨阴极的工作温度一般在2600~2800K之间,敷钍阴极的工作温度约2000K。旁热式阴极是表面涂上一层钡,锶等氧化物,内部装有热丝的镍质套管,依靠加热电流通过热丝使阴极间接加热而发射电子。热丝有时也称为灯丝,它用钨丝或钨钼合金丝制成,表面涂覆氧化铝绝缘层与套管绝缘,图2.2是旁热式阴极的构造和符号。在通信设备中的电子管,绝大部分是用旁热式阴极的。

\begin{figure}[htbp]
	\centering
	\includegraphics[width=0.7\linewidth]{1}
	\caption{直热式阴极和直热式二极管的符号}
	\label{fig:1}
\end{figure}

\begin{figure}[htbp]
	\centering
	\includegraphics[width=0.7\linewidth]{2}
	\caption{旁热式二极管和它的符号}
	\label{fig:1}
\end{figure}

屏极是用来吸收从阴极发射出来的电子的。它一般是用镍、钼或钽等金属制成圆筒形或椭圆形围绕在阴极之外,见图2.2。

在电子管电路中,字母G表示电子管,阴极则用K来表示,屏极用a来表示,而灯丝或热丝用f来表示。

\section{导电性能}

现在来看看二极管屏极和阴极之间的导电性能。当接上灯丝电源以后,阴极因受热而温度升高,从而把电子发射到阴极周围的空问。如果在屏极与阴极之间加上一个直流电压,当屏极接到电源的正极而阴极接到负极时,如图2.3(a),由阴极发射出来的电子在屏极正电场的吸引下便向屏极移动,使外电路通过电流,因而电流表有读数指示。当外加直流电源的极性反过来,如图2.3(b),即它的正极接到阴极,负极接到屏极,这时屏极的负电场将排斥从阴极发射出来的电子,外电路没有电流通过,电流表没有读数。

\begin{figure}[htbp]
	\centering
	\includegraphics[width=0.7\linewidth]{3}
	\caption{二极管的导电性能}
	\label{fig:1}
\end{figure}

这说明二极管只有当屏极电位高于阴极电位时才有电流通过,所以二极管具有单向导电的性能。

\section{整流电路}

二极管的主要用途之一是整流,将结合实际要求来介绍二极管整流器的基本原理。

在通信设备中,常常需要直流电源,而交流电源是比较容易得到的,这就需要把交流电变成直流电,这种把交流电变成直流电的方法就叫整流,把交流电变成直流电的设备,就称为整流器。利用二极管的单向导电性,可以把交流电变为直流电。

图2.4(a)是一个最简单的整流器电路,其中B是电源变压器,用来改变交流电压,以配合获得所要求的直流电压数值,一般3-4线圈是升压线圈,称为高压线圈,它接到电子屏极上;5-6线圈是降压线圈,称为灯丝线圈,它用来加热灯丝使产生热电子发射。Rfg是负载电阻,代表使用直流电源的用电设备。

\begin{figure}[htbp]
	\centering
	\includegraphics[width=0.7\linewidth]{4}
	\caption{半波整流电路和各部分电流电压的波形图}
	\label{fig:1}
\end{figure}

在电源变压器的初级线圈加上交流电压e1时,经变压器升压在高压线圈得到的电压为e2,它以一定的频率(例如50赫)改变着大小和方向,如图1-4(b)所示。当交流电正半周时3端为正、4端为负,这时二极管导电,即屏极吸引电子形成屏流ia,它的方向和途径如图1-4(a)所示。在交流电的负半周,3端为负,4端为正,如图中有圈的符号,这时二极管不导电,电路中没有电流通过。这样,原来方向随时间变化的交流电压,因二极管的单向导电作用,使电流只能向一个方向流动如图1-4(c)。这时屏流ia流过负载电阻Rfg产生电压降ufg,它的极性与大小决定于ia的方向与大小,因为ia在电子管内部总是从屏极流向阴极,所以ufg的极性是一直不变的,但大小是随着ia在不断地变化着,如图1-4(d),这种方向不变而大小随时间而变的电压称为脉动电压。单靠二极管的单向导电作用,只能将交流电压变为脉动电压,它与我们所要求的直流电压还有很大的差别,要想得到平稳的直流电压还需要进行滤波,这将在以后再来说明。

上述的整流电路因为只在半个周期内让电流通过,所以称为半波整流电路。半波整流电路只利用了输入交流电压的半个周期,而另半个周期二极管不导电,所以效率很低,且负载上得到的电压与直流电压差别很大。

常用的整流电路是全波整流电路,它在效率和效果上都比半波整流好些,电路如图1-5(4)示,它由一个次级线圈具有中心抽头的电源变压器B(即3-4线图和4-5线圈的圈数相等)、两个特性相同的二极管G1和G2及负载电阻Rfg联接而成。

\begin{figure}[htbp]
	\centering
	\includegraphics[width=0.7\linewidth]{5}
	\caption{全波整流电路和各部分电流电压的波形}
	\label{fig:1}
\end{figure}

G1、G,的屏极分别接在高压线圈的两端,它们的阴极连在一起并通过负载电阻R:与中心抽头联接。这样,加在G,屏极的交流电压e,与加在G,屏极的交流电压e,大小相等而相位相反。如在交流电某一半周时,e的极性是3端为正,4端为负,而e,的极性是4端为正、5端为负,如图1-6(a),这时G,的屏极电压为正而导电,G,的屏极电压为负处于截止状态由G,供给负载的电流是i,如图中实线所示;经过半个周期电压的极性反过来,ez的极性是3端为负、4端为正,而e是4端为负,5端为正,如图中有圈的符号,这时G,截止而G,导电,G,供给负载的电流为i。,如图中虚线所示。由此可见,交流电源的两个半周,电于管G,和G,轮流地以相同的方向自上而下供给负载电流,波形如图1-5(e)所示,
从负载电流的波形可以看到,全波整流电路的输出电流(或电压)虽然也是脉动电流(或电压),但与半波整流比较,它更接近于直流,而且交流电源的正、负半周都利用上,因此效率也提高了。
为了适应全波整流电路的需要,还专门生产了一种双二极整流管,如6Z4等,在管内装有两个独立的极和一个公用的阴极,同样能起到图1-5(a)中两个二极管的作用,它用作全波整流时的接法如图1-5(8)。
全波整流的输出电压虽然比半波整流更接近直流电压,但它的大小还是要随时间而变的,为了要得到稳定的直流电压,还需要进行滤波。最简单的滤波电路是在负载上并联一个电容器C,如图1-6。接入电容器以后,流过负载的电流就发生显著



\chapter{三极管}

\chapter{多极管}

\chapter{复合管}

\part{放大器}

\part{振荡器}

\part{稳压电路}

\part{脉冲电路}

\backmatter



\end{document}