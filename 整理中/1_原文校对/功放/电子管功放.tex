% 电子管功放
% 使用xelatex编译

\documentclass[12pt,a4paper,twoside]{ctexbook}

% 页面设置
\usepackage{geometry}
\geometry{a4paper, top=2.5cm, bottom=2.5cm, left=3cm, right=3cm}

% 字体设置
\usepackage{xeCJK}
\usepackage{fontspec}
\usepackage{microtype}

% 设置中文字体
\setCJKmainfont{SimSun}[  % 正文宋体
    BoldFont=SimHei,        % 粗体黑体
    ItalicFont=KaiTi        % 斜体楷体
]
\setCJKsansfont{SimHei}    % 无衬线字体黑体
\setCJKmonofont{SimSun}    % 等宽字体宋体

% 标题格式设置
\ctexset{
    part/name={\hei 第},        % 卷名
    part/number={\hei \chinese{part}卷}, % 卷号
    chapter/name={\hei 第},     % 章名
    chapter/number={\hei \chinese{chapter}章}, % 章号
    section/name={\hei 第},     % 节名
    section/number={\hei \arabic{section}节},  % 节号
    subsection/name={\hei },    % 小节名
    subsection/number={\hei \arabic{section}.\arabic{subsection} }, % 小节号
    chapter/format={\centering\hei\zihao{1}},% 章标题格式
    section/format={\hei\zihao{3}},% 节标题格式
    subsection/format={\hei\zihao{4}}% 小节标题格式
}

% 页眉页脚设置
\usepackage{fancyhdr}
\pagestyle{fancy}
\fancyhf{}
\fancyhead[LE,RO]{\zihao{5}\thepage}
\fancyhead[LO]{\zihao{5}\leftmark}
\fancyhead[RE]{\zihao{5}\rightmark}
\renewcommand{\chaptermark}[1]{\markboth{\chaptername\ \thechapter\ #1}{}}
\renewcommand{\sectionmark}[1]{\markright{\thesection\ #1}}
\fancyfoot[C]{\zihao{5} 电子管功放}
\renewcommand{\headrulewidth}{0.4pt}
\renewcommand{\footrulewidth}{0pt}

% 目录设置
\usepackage{titletoc}
\titlecontents{chapter}[0pt]{\vspace{10pt}\bfseries\zihao{-4}}{\contentspush{\thecontentslabel\hspace{1em}}}{}{\titlerule*[8pt]{.}\contentspage}
\titlecontents{section}[2.5em]{\vspace{5pt}\zihao{5}}{\contentspush{\thecontentslabel\hspace{1em}}}{}{\titlerule*[8pt]{.}\contentspage}
\titlecontents{subsection}[5em]{\vspace{3pt}\zihao{5}}{\contentspush{\thecontentslabel\hspace{1em}}}{}{\titlerule*[8pt]{.}\contentspage}

% 目录深度
\setcounter{tocdepth}{2}
\setcounter{secnumdepth}{2}

% 封面信息
\title{\hei\zihao{0} 电子管功放}
\author{\song\zihao{2} 电子管功放编写组}
\date{\song\zihao{4} \today}

\begin{document}

% 封面
\begin{titlepage}
    \begin{center}
        \vspace*{6cm}
        \hei\zihao{0} 电子管功放
        \vspace*{3cm}
        \song\zihao{2} 电子管功放编写组
        \vspace*{3cm}
        \song\zihao{4} \today
    \end{center}
\end{titlepage}

% 版权页
\newpage
\thispagestyle{empty}
\begin{center}
    \vspace*{8cm}
    \song\zihao{5} 版权所有\textcopyright\ 2026 电子管功放编写组
    \vspace*{1cm}
    \song\zihao{5} 仅供学习和参考使用
\end{center}

% 目录
\newpage
\tableofcontents

% 正文开始
\mainmatter

\chapter{电子管功放概述}

\section{电子管功放的定义与发展历程}
\subsection{电子管功放的基本概念}
\subsection{电子管功放的发展历史}
\subsection{电子管功放的现状与趋势}

\section{电子管功放的特点与优势}
\subsection{电子管与晶体管的比较}
\subsection{电子管功放的音质特点}
\subsection{电子管功放的应用领域}

\chapter{电子管基础知识}

\section{电子管的基本结构与工作原理}
\subsection{电子管的基本组成}
\subsection{电子管的工作原理}
\subsection{电子管的主要参数}

\section{常见电子管类型与特性}
\subsection{二极管}
\subsection{三极管}
\subsection{四极管与五极管}
\subsection{束射管与束射四极管}
\subsection{功率电子管}

\section{电子管的选择与使用}
\subsection{电子管的型号标识}
\subsection{电子管的选择原则}
\subsection{电子管的安装与更换}
\subsection{电子管的使用寿命}

\chapter{电子管功放电路原理}

\section{放大电路基础}
\subsection{放大电路的基本概念}
\subsection{放大电路的主要性能指标}
\subsection{负反馈的原理与应用}

\section{电子管功放的基本电路结构}
\subsection{前置放大电路}
\subsection{电压放大电路}
\subsection{功率放大电路}
\subsection{电源电路}

\section{常见电子管功放电路拓扑}
\subsection{A类放大电路}
\subsection{AB类放大电路}
\subsection{B类放大电路}
\subsection{单端放大电路}
\subsection{推挽放大电路}
\subsection{超线性放大电路}

\chapter{电子管功放的分类}

\section{按功率输出分类}
\subsection{小功率电子管功放}
\subsection{中功率电子管功放}
\subsection{大功率电子管功放}

\section{按电路结构分类}
\subsection{单声道电子管功放}
\subsection{立体声电子管功放}
\subsection{合并式电子管功放}
\subsection{前后级分离式电子管功放}

\section{按应用场景分类}
\subsection{家用电子管功放}
\subsection{专业电子管功放}
\subsection{发烧级电子管功放}
\subsection{便携式电子管功放}

\chapter{电子管功放的设计与制作}

\section{设计基础与原则}
\subsection{设计目标与需求分析}
\subsection{电路设计的基本原则}
\subsection{元器件的选择}

\section{具体电路设计}
\subsection{前置放大电路设计}
\subsection{功率放大电路设计}
\subsection{电源电路设计}
\subsection{保护电路设计}

\section{制作工艺与调试}
\subsection{PCB设计与制作}
\subsection{机箱设计与加工}
\subsection{元器件的安装与焊接}
\subsection{电路的调试与测试}

\chapter{电子管功放的调试与维护}

\section{电子管功放的调试}
\subsection{调试前的准备工作}
\subsection{静态工作点的调整}
\subsection{动态性能的测试}
\subsection{音质的调整与优化}

\section{电子管功放的维护}
\subsection{日常维护与保养}
\subsection{常见故障的排查与修复}
\subsection{电子管的老化与更换}
\subsection{安全注意事项}

\chapter{电子管功放的音质特点}

\section{电子管功放的音质评价}
\subsection{音质评价的基本要素}
\subsection{电子管功放的音质特点分析}
\subsection{影响音质的主要因素}

\section{音质调整与优化}
\subsection{电路参数的调整}
\subsection{元器件的搭配与选择}
\subsection{听音环境的影响与优化}

\chapter{电子管功放的应用与发展}

\section{电子管功放的应用场景}
\subsection{家用音响系统}
\subsection{专业音频领域}
\subsection{音乐制作与录音}

\section{电子管功放的发展趋势}
\subsection{技术创新与发展方向}
\subsection{市场需求与发展前景}
\subsection{电子管功放与现代音频技术的融合}

\end{document}