% 男性生殖系统指南。
% 使用xelatex编译

\documentclass[12pt,a4paper,twoside]{ctexbook}

% 页面设置
% 纸张设置配置文件
% 用于定义书籍的页面尺寸和边距

\usepackage[a4paper,twoside]{geometry}
\geometry{
	left=25mm,
	right=20mm,
	top=25mm,
	bottom=25.4mm,
	headsep=1cm, 
    footskip=1cm,
	bindingoffset=10mm
}

% 字体设置
\usepackage{xeCJK}
\usepackage{fontspec}
\usepackage{microtype}

% 设置中文字体
\setCJKmainfont{SimSun}[  % 正文宋体
    BoldFont=SimHei,        % 粗体黑体
    ItalicFont=KaiTi        % 斜体楷体
]
\setCJKsansfont{SimHei}    % 无衬线字体黑。
\setCJKmonofont{SimSun}    % 等宽字体宋体
\setCJKfamilyfont{kai}[    % 楷体
    BoldFont=KaiTi
]{KaiTi}
\setCJKfamilyfont{fs}[     % 仿宋
    BoldFont=FangSong
]{FangSong}

% 常用字体命令
\newcommand{\song}{\CJKfamily{zhsong}}
\newcommand{\hei}{\CJKfamily{zhhei}}
\newcommand{\kai}{\CJKfamily{kai}}
\newcommand{\fs}{\CJKfamily{fs}}

% 标题格式设置
\ctexset{
    part/name={。卷},
    part/number={\chinese{part}},
    chapter/name={。章},
    chapter/number={\chinese{chapter}},
    section/name={。节},
    section/number={\arabic{section}},
    subsection/number={\arabic{section}.\arabic{subsection}},
    chapter/format={\centering\hei\zihao{2}},
    section/format={\hei\zihao{4}},
    subsection/format={\hei\zihao{5}}
}

% 目录设置
\usepackage{titletoc}
\titlecontents{chapter}[0pt]{\vspace{10pt}\bfseries\zihao{-4}}{\contentspush{\thecontentslabel\hspace{1em}}}{}{\titlerule*[8pt]{.}\contentspage}
\titlecontents{section}[2.3em]{\vspace{5pt}\zihao{5}}{\contentspush{\thecontentslabel\hspace{1em}}}{}{\titlerule*[8pt]{.}\contentspage}
\titlecontents{subsection}[5em]{\zihao{5}}{\contentspush{\thecontentslabel\hspace{1em}}}{}{\titlerule*[8pt]{.}\contentspage}

% 图表目录设置
\usepackage[titles]{tocloft}
\renewcommand{\cftfigpresnum}{图}
\renewcommand{\cftfigaftersnum}{、}
\settowidth{\cftfignumwidth}{图100、}
\renewcommand{\cfttabpresnum}{表}
\renewcommand{\cfttabaftersnum}{、}
\settowidth{\cfttabnumwidth}{表100、}

% 章节标题页
\usepackage{titlesec}
\titleformat{\chapter}[display]
    {\bfseries\centering\zihao{-2}}{\chaptertitlename\ \thechapter}{20pt}{\zihao{-2}}
\titlespacing*{\chapter}{0pt}{-50pt}{30pt}

% 段落设置
\usepackage{indentfirst}    % 首段缩进
\setlength{\parindent}{2em} % 首行缩进2字符
\setlength{\parskip}{0pt}   % 段落间距为0

% 间距设置
\usepackage{setspace}
\onehalfspacing             % 1.5倍行距

% 列表设置
\usepackage{enumitem}
\setlist{leftmargin=2em, labelsep=0.5em} % 列表缩进

% 图表设置
\usepackage{graphicx}
\usepackage{float}
\usepackage{subfig}
\graphicspath{{../../../common/images/}}
\floatstyle{plaintop}
\restylefloat{table}
\floatstyle{plain}
\restylefloat{figure}

% 表格设置
\usepackage{array}
\usepackage{longtable}
\usepackage{booktabs}
\usepackage{multirow}
\usepackage{makecell}

% 数学公式
\usepackage{amsmath}
\usepackage{amssymb}
\usepackage{mathtools}

% 引用设置
\usepackage{hyperref}
\hypersetup{
    colorlinks=true,
    linkcolor=blue,
    citecolor=blue,
    urlcolor=blue,
    bookmarks=true,
    bookmarksnumbered=true,
    bookmarksopen=false,
    pdftitle={夫妻性生活1000问 - 男性篇},
    pdfauthor={陈亦洋},
    pdfsubject={夫妻性生活},
    pdfkeywords={夫妻性生活, 男性}
}
\usepackage{bookmark}

% 页眉页脚设置
\usepackage{fancyhdr}
\pagestyle{fancy}
\fancyhf{}
\fancyhead[LE,RO]{\thepage}
\fancyhead[LO]{\song\leftmark}
\fancyhead[RE]{\song\rightmark}
\renewcommand{\headrulewidth}{0.4pt}
\renewcommand{\footrulewidth}{0pt}

% 特殊符号
\usepackage{wasysym}

% 日期时间
\usepackage{datetime}

% 其他设置
\usepackage{calc}
\usepackage{ifthen}
\usepackage{afterpage}
\usepackage{pdfpages}

% 开始文档
\title{\hei\zihao{0} 夫妻性生活1000问 - 男性篇}
\author{陈亦洋}
\date{\today}

\begin{document}

% 标题页
\maketitle

% 版权页
\newpage
\thispagestyle{empty}
\begin{center}
    \vspace*{8cm}
    \song\zihao{5} 图书在版编目(CIP)数据\par
    \vspace{0.5em}
    \song\zihao{5} 夫妻性生活1000问/陈亦洋编著.—长春:吉林科学技术出版社,2011.10\par
    \vspace{0.5em}
    \song\zihao{5} ISBN 978-7-5384-5494-9\par
    \vspace{0.5em}
    \song\zihao{5} Ⅰ.①夫… Ⅱ.①陈… Ⅲ.①性知识—问题解答 Ⅳ.①R167-44\par
    \vspace{0.5em}
    \song\zihao{5} 中国版本图书馆CIP数据核字(2011)第204046号\par
    \vspace{2em}
    \song\zihao{5} 版权所有。\textcopyright\ 2011 陈亦洋。\par
    \vspace{0.5em}
    \song\zihao{5} 出版社名:吉林科学技术出版社\par
\end{center}

% 作者简介
\newpage
\thispagestyle{empty}
\begin{center}
    \vspace*{5cm}
    \hei\zihao{2} ——作者简介——
\end{center}

\begin{center}
    \hei\zihao{3} 陈亦洋
\end{center}

\vspace{1cm}
\begin{center}
    \song\zihao{4} 毕业于长春中医学院,主任医师。长期从事临床诊疗工作,擅长结合中西医治疗男科、妇科疾病,尤其对性传播疾病,梅毒、尖锐湿疣、生殖器疱疹等诊治有独特之处。
\end{center}

\vspace{2cm}

% 目录
\tableofcontents
\listoffigures
\listoftables

% 正文内容

\chapter{男性生殖系统}

\section{男性生殖系统解剖与生理}

男性生殖系统包括外生殖器和内生殖器两部分。外生殖器包括阴阜、阴茎和阴囊,内生殖器包括睾丸、附睾、输精管、精囊腺、前列腺、尿道球腺等。这些器官协同工作,完成精子的产生、储存、运输以及精液的分泌和排出。

\begin{figure}[htbp]
    \centering
    \includegraphics[width=0.7\linewidth]{male_reproductive_system.jpg}
    \caption{男性生殖系统解剖图}
    \label{fig:male_reproductive_system}
\end{figure}

\subsection{阴阜}

阴阜为耻骨前方的皮肤和丰富的皮下脂肪组织。成年人皮肤上有阴毛,皮下组织有皮脂腺和汗腺。青壮年时阴阜显著隆起,中年以后脂肪组织减少下陷,老年则萎缩变平。青春期的阴毛发生是男性第二性征的标志之一,雄性激素的缺乏表现为阴毛稀少或不发育。阴阜的主要功能是保护耻骨联合和阴茎根部,在性生活中也可以起到缓冲作用。

\subsection{阴茎}

阴茎是男性最重要的外生殖器官,具有性交、排尿和射精三大功能。阴茎位于耻骨联合前下方,阴囊的上方。阴茎由3条海绵体外包筋膜和皮肤构成。其中阴茎海绵体有两条,尿道海绵体有一条。分根部、体部及头部。根部固定于会阴部,阴茎前端膨大部分形成阴茎头,头部与体部交接部较细,为颈部,是一环形沟,又称冠状沟。尿道海绵体内有尿道通过,开口于尿道外口。阴茎未勃起时呈圆柱状,长约7~9厘米。勃起时,呈三棱形圆柱状,长度增加一倍以上,主要功能是完成性交。阴茎外面包有皮肤,包盖着阴茎头,称为阴茎包皮。阴茎海绵体内的特殊结构是阴茎勃起功能的重要组织结构,而阴茎勃起又是完成性交的先决条件。

\paragraph{解剖结构}

\subparagraph{海绵体}
阴茎由三个平行的海绵体组成,它们是阴茎的主要结构:

1. \textbf{阴茎海绵体}:
   - 位于阴茎背侧,左右各一,呈圆柱形,前端变细嵌入阴茎头底面的凹陷内,后端分离形成阴茎脚,附着于耻骨下支和坐骨支。
   - 每个阴茎海绵体内有许多螺旋状的血管窦(血窦),血窦之间有结缔组织隔(小梁)。当受到性刺激时,血窦扩张,血液大量流入,使阴茎海绵体充血膨胀,形成勃起。

2. \textbf{尿道海绵体}:
   - 位于阴茎腹侧中央,呈圆柱形,前端膨大形成阴茎头(龟头),后端膨大形成尿道球,附着于尿生殖膈。
   - 尿道海绵体内有尿道通过,贯穿整个阴茎,前端开口于阴茎头的尿道外口,后端与前列腺部尿道相连。
   - 尿道海绵体的血窦也参与勃起过程,但膨胀程度不如阴茎海绵体明显。

\subparagraph{白膜}
- 白膜是包裹在海绵体外面的一层致密结缔组织膜,质地坚韧,具有很强的弹性和韧性。
- 白膜在阴茎勃起时起到限制海绵体过度膨胀的作用,维持阴茎的硬度和形态。
- 三个海绵体的白膜在阴茎背侧融合形成阴茎隔,将两个阴茎海绵体分开。

\subparagraph{阴茎头和包皮}
- \textbf{阴茎头}:又称龟头,是尿道海绵体前端的膨大结构,呈圆锥状,表面光滑,富含神经末梢,对性刺激非常敏感。
- \textbf{尿道外口}:位于阴茎头顶端,是尿液和精液的共同出口。
- \textbf{包皮}:是覆盖在阴茎头表面的一层皱襞皮肤,具有保护阴茎头的作用。
  - 包皮与阴茎头之间的间隙称为包皮腔,容易积存包皮垢,需要经常清洗。
  - 包皮口狭窄或包皮过长可能导致包皮炎、龟头炎等疾病,需要进行包皮环切手术。

\subparagraph{皮肤和筋膜}
- \textbf{皮肤}:阴茎的皮肤薄而柔软,富有弹性,容易伸展,适应阴茎勃起时的体积变化。
  - 阴茎皮肤在阴茎颈处游离,向前反折形成包皮。
  - 阴茎皮肤无皮下脂肪,与深层组织连接紧密。

- \textbf{浅筋膜}:位于皮肤下方,由疏松结缔组织组成,内含少量平滑肌纤维(肉膜)。
- \textbf{深筋膜}:又称Buck筋膜,包裹在三个海绵体外面,与白膜紧密相连。

\subparagraph{血管和神经供应}

- \textbf{血液供应}:
  - 主要来自阴茎背动脉和阴茎深动脉,它们是阴部内动脉的分支。
  - 阴茎背动脉:沿阴茎背侧行走,供应阴茎头和包皮的血液。
  - 阴茎深动脉:进入阴茎海绵体,分支形成螺旋动脉,供应血窦的血液,是阴茎勃起的主要血管。

- \textbf{静脉回流}:
  - 阴茎背浅静脉:收集阴茎皮肤和浅筋膜的静脉血,汇入大隐静脉。
  - 阴茎背深静脉:收集阴茎海绵体和尿道海绵体的静脉血,汇入前列腺静脉丛。

- \textbf{神经支配}:
  - 感觉神经:主要来自阴茎背神经,是阴部神经的分支,分布于阴茎头、包皮和阴茎皮肤,传递性感觉。
  - 自主神经:包括交感神经和副交感神经,控制阴茎的勃起和疲软。
    - 交感神经:来自腹主动脉丛,控制阴茎的疲软,使阴茎恢复非勃起状态。
    - 副交感神经:来自盆神经丛,控制阴茎的勃起,使阴茎充血膨胀。

\paragraph{发育与变化}

阴茎的发育和变化贯穿男性的整个生命周期:

- \textbf{胎儿期}:
  - 阴茎在胎儿第7周开始发育,起源于生殖结节。
  - 胎儿第9周,生殖结节迅速生长,形成阴茎的雏形。
  - 胎儿第12周,阴茎已具雏形,可区分阴茎头、阴茎体和阴茎根。
  - 胎儿第20周,阴茎长度约为2-3厘米,包皮已覆盖阴茎头。

- \textbf{新生儿期}:
  - 新生儿阴茎长度约为2.5-3.5厘米,包皮与阴茎头粘连,无法上翻。
  - 阴茎外观短小,但比例与成人相似。

- \textbf{儿童期}:
  - 阴茎生长缓慢,长度和直径变化不大。
  - 3-4岁后,包皮与阴茎头逐渐分离,包皮可部分上翻。
  - 儿童期阴茎处于相对静止状态,无明显的性反应。

- \textbf{青春期}:
  - 青春期开始,在雄激素的作用下,阴茎迅速发育,长度和直径显著增加。
  - 12-14岁:阴茎开始加速生长,长度增加1-2厘米。
  - 14-16岁:阴茎生长最快,长度增加2-3厘米,直径也明显增加。
  - 16-18岁:阴茎发育基本完成,接近成人大小。
  - 青春期后,阴茎头完全暴露,或包皮仍部分覆盖,但可自由上翻。

- \textbf{性成熟期}:
  - 性成熟期是阴茎功能最活跃的时期,也是男性的生育期。
  - 阴茎大小和形态达到成人水平,能够完成勃起、性交和射精等功能。
  - 阴茎的勃起频率和硬度处于最佳状态。

- \textbf{中年期}:
  - 40岁以后,阴茎的勃起功能逐渐下降,勃起频率和硬度有所降低。
  - 阴茎的长度和直径可能略有减小,皮肤弹性下降。

- \textbf{老年期}:
  - 60岁以后,阴茎进一步萎缩,长度和直径明显减小。
  - 勃起功能显著下降,勃起需要更长的时间,硬度也明显降低。
  - 阴茎皮肤松弛,皱纹增加,颜色变深。
\paragraph{生理功能}

阴茎具有三种主要生理功能:

\subparagraph{排尿功能}
- 阴茎是尿液排出体外的通道,尿道贯穿整个阴茎,连接膀胱和尿道外口。
- 排尿时,膀胱逼尿肌收缩,尿道括约肌松弛,尿液通过尿道从尿道外口排出。
- 阴茎的位置和形态有利于尿液的排出,避免尿液污染身体。

\subparagraph{性交功能}
- 阴茎是男性的性交器官,通过勃起插入阴道,进行抽送动作,完成性交过程。
- 勃起时阴茎的硬度和角度适合插入阴道,阴茎头的刺激有助于女性达到性高潮。
- 性交过程中,阴茎的抽送动作可以刺激女性的阴道和阴蒂,促进性快感的产生。

\subparagraph{射精功能}
- 射精是男性性高潮的表现,也是将精子排出体外的过程。
- 性高潮时,输精管、精囊腺、前列腺等器官的平滑肌收缩,将精液排入尿道。
- 同时,尿道海绵体和阴茎海绵体的肌肉收缩,将精液从尿道外口射出。
- 射精过程分为两个阶段:泄精(精液排入尿道)和射精(精液射出体外)。

\paragraph{性反应机制}

阴茎血管本身和血流的供应在阴茎勃起机制中占有极其重要的地位。因为阴茎勃起的本质是血管的充血反应,这些血运系统出了毛病,当然会造成阴茎勃起不良。阴茎内动脉的分支有:位于阴茎背面白膜外的阴茎背动脉,沿阴茎海绵体走向的阴茎海绵体动脉,以及沿尿道海绵体走向的尿道腹侧的两条尿道球动脉。这些动脉的末端分支,即螺旋动脉终止于小毛细血管,后者又直接开口于海绵体腔。静脉回流有两条通路,浅表背静脉引流整个尿道海绵体(包括龟头和尿道球)的血液,深部背静脉引流阴茎海绵体血液。

3个海绵体腔行使阴茎勃起组织的功能,龟头和尿道海绵体提供体积,而一对阴茎海绵体提供硬度。由称为小梁的纤维和平滑肌索带将组织分隔成许多不规则的间隙,又称血窦,窦间隙表面覆盖有内皮。位于海绵体中央的窦隙较大,在阴茎高度充血时,直径可达1~9毫米,松弛时则为不明显的间隙。每个血窦都有深动脉和输出静脉与其直接相通,深动脉与输出静脉之间还有直接的交通支脉,称为动静脉分流系统。在深动脉、输出静脉和动静脉短路的管壁上都存在瓣膜状平滑肌皱襞,受勃起神经调节。

阴茎的性反应主要包括勃起、射精和疲软三个阶段:

1. \textbf{勃起阶段}:
   - 当受到性刺激(视觉、听觉、触觉等)时,副交感神经兴奋,释放乙酰胆碱等神经递质。
   
   - 这些神经递质作用于阴茎海绵体的血管内皮细胞,释放一氧化氮(NO)。
   - 一氧化氮激活鸟苷酸环化酶,使环磷酸鸟苷(cGMP)水平升高。
   - cGMP使阴茎海绵体的平滑肌松弛,血管窦扩张,血液大量流入海绵体。
   - 同时,白膜下的静脉受压关闭,血液无法流出,使阴茎体积增大、硬度增加,形成勃起。
   
   - 当神经冲动作用于瓣膜状平滑肌皱襞时,窦的深动脉完全开放,而输出静脉和动静脉交通支的管壁部分关闭,故入窦血量增多,导致海绵体的充盈及勃起。
   - 血液由深静脉流出减少,并有选择地积聚在海绵窦内,海绵组织内的平滑肌松弛,血窦因充满血液而膨大,因而阴茎体积也明显增大,但白膜不会无限膨大,最后使勃起的阴茎达到所需要的硬度。
   - 在勃起中起主要作用的是动脉的作用,但阴茎内静脉血管瓣膜部分关闭并限制静脉血液回流也很重要。若血液排放系统关闭不全,静脉排放流量过大,勃起可以减退而形成阳痿。此外,动静脉交通支在勃起时不关闭也可造成阳痿。
   - 这种皱襞瓣膜组织于婴儿出生后2个月即已开始出现,自3岁起大量增加,这说明男性儿童可能有勃起现象。新生儿阴茎勃起,纯属神经生理反应,不存在瓣膜问题。

\subparagraph{贴心小叮咛}

年轻人勃起只需要5秒钟,而年长者可能需要6~7分钟之久。这说明人的体力、精力、心理与精神因素在年龄增长过程中,对性欲与阴茎勃起也有一定影响。

2. \textbf{射精阶段}:
   - 当性刺激达到阈值时,脊髓的射精中枢兴奋,发出神经冲动。
   - 交感神经兴奋,导致输精管、精囊腺、前列腺等器官的平滑肌收缩,将精液排入尿道。
   - 随后,尿道海绵体和阴茎海绵体的肌肉发生节律性收缩,将精液从尿道外口射出。
   - 射精时会产生强烈的性快感,称为性高潮。

3. \textbf{疲软阶段}:
   
   - 射精后,交感神经兴奋,释放去甲肾上腺素等神经递质。
   - 这些神经递质使阴茎海绵体的平滑肌收缩,血管窦关闭,血液流出海绵体。
   - 阴茎体积减小、硬度降低,恢复到非勃起状态。
   - 疲软阶段通常持续几分钟到几小时,期间阴茎对性刺激的反应减弱。
   - 射精后,由于腹下神经中的交感神经纤维兴奋,阴茎内动脉收缩,使窦深动脉壁平滑肌皱襞增厚,形成瓣膜样部分关闭,窦血流减少,输出静脉交通支完全开放,静脉回流增加,阴茎很快疲软。

阴茎像一个令人吃惊的天然液压机械装置,勃起与消退的生理反应,表现为一个器官在一定容量下呈现出的流入与流出的血流动力学变化。根据阴茎的大小,勃起时血容量的增加为80~200毫升。阴茎勃起时,阴茎内动脉扩张,进入阴茎的血流量比松弛时大8倍之多。当达到120毫升/分时(海绵体内压增高至10千帕左右)引起勃起,而一旦勃起之后维持勃起的血流量只为原先灌注率的60%左右,或70毫升/分即可,这说明静脉没有完全关闭。若静脉完全关闭,只能引起阴茎水肿和发绀,而不是真正的勃起。

\paragraph{阴茎勃起弯曲}

有些男性发现,阴茎在勃起时会呈弯曲状态,这是不是病态呢?这要具体问题具体分析。构成阴茎的3条海绵体,里面充满空隙,当充血勃起时,由于这些充血海绵体不是一样大小的,因此,阴茎就会出现向充血少的海绵体方向弯曲。由此可见,阴茎弯曲或向某侧稍偏,是正常现象;大多数男性阴茎勃起时偏向左侧,不会影响性功能。只有那些阴茎系带过短,在勃起时将阴茎折向下方,或因包皮环切术时的包皮切除过多、外伤形成瘢痕,以及先天性的筋膜海绵体发育不平衡等,使勃起的阴茎明显弯曲的男性,并且影响性交的顺利进行,才属于病态而需要手术矫治。

\subparagraph{贴心小叮咛}

阴茎是不容易断裂的,如果断裂的话,有如下表现:
1.突然的外力作用;
2.有断裂的响声;
3.剧烈疼痛,肿胀,青紫。

\paragraph{为什么阴茎会无端勃起}

阴茎勃起是人类的一种本能,有时并非由人的意志所决定,而是由一系列的反射活动所引起的。在通常情况下,当男子出现性兴奋时,阴茎便会勃起;性冲动消退以后,阴茎便恢复到疲软状态。在生理情况下,从一两岁的小孩到老年人,都会出现阴茎勃起现象。青少年正处在青春发育期,阴茎的勃起现象就更为频繁。

青少年进入青春发育期以后,由于性意识的觉醒和性心理的萌生,会产生一定的性冲动和性欲望。如果在生活中受到一些有关性的刺激,例如对异性的爱慕、看了影视片或书刊上关于爱情情节的描写、睡眠时做有关性的色情梦,以及紧身裤对生殖器的摩擦等,都会导致阴茎的勃起。这一切,都不是病态,而是正常现象。有些人在睡眠中也会发生阴茎勃起,不过,阴茎勃起的次数和时间,却是随着年龄的不同而有异:青春期少年,平均每晚可勃起6次左右,每次勃起持续时间20~30分钟:青壮年平均每1~1.5个小时勃起一次;中年以后,阴茎勃起的次数便减少;65岁的健康老人,每晚仍可有数次勃起。只是由于这些勃起都发生在睡眠中,当事人并不知晓而已。

熟睡后阴茎勃起的生理机制:我们的内脏受交感神经和副交感神经的支配,性器官也不例外。阴茎勃起是受到骶髓副交感神经的支配的,副交感神经兴奋时,阴茎就会勃起。

交感神经和副交感神经都有一定的紧张性(即微弱而持续的兴奋),而且像跷跷板那样互相对抗。当交感神经兴奋时,副交感神经就相对抑制;反过来,副交感神经兴奋时则交感神经相对抑制。白天受环境的各种刺激,交感神经占相对优势;晚上安静和睡眠时,尤其是在异相睡眠时,副交感神经占优势,就会有男性阴茎勃起、女性阴部充血的现象。民间有两句顺口溜:“男儿三更竹竿起,女子半夜莲花开。”说明前人早就观察到这种生理现象。因此,夜间睡着后,阴茎勃起不是病态,不需治疗。

清晨勃起是由于夜间有邪念吗?许多男子都有这样的经验:早晨醒来,发现阴茎无意中勃起,民间认为那是贪色之故,其实这是误解。

正常男子的阴茎,除了在性刺激和某种外界刺激时会勃起外,通常处于松弛状态。但是有时内脏器官的反射作用也会导致阴茎勃起。

最明显的是早晨清醒前时常会出现阴茎勃起,由于膀胱内压力增加而产生刺激作用,可以导致阴茎发生一种潜意识的反射性勃起,也就是常常说的让尿给“憋”硬了,这是一种正常的生理现象,医学上称之为清晨勃起。但这种勃起现象的强度或大或小,一般达不到性交所特有的水平。

据美国一位学者的研究资料报道,男子在成年后,20~30岁时,清晨勃起次数增多,中年以后逐渐减少。一位德国医生也研究了这个生理现象:他发现男子在疾病期间,清晨阴茎勃起的现象会消失,当身体康复后,清晨阴茎勃起的现象又重新出现。于是他提出,清晨阴茎勃起现象可以作为观察男子精力和健康状况的参考指标之一。

但是,关于清晨阴茎勃起的确切机制,至今尚未研究清楚,无论如何,清晨阴茎勃起是男子的一种正常的生理反应,是肯定无疑的,而且每天男子的个体差异所产生的变化也不尽一致,千万不能只凭这一点来判断男子性功能的好坏。诚然,医生在鉴别诊断性功能变化时,需要依此作为参考,但那完全是限定病态情况时而做的。而对于一个正常男子则不应盲目地以此来识别好坏。

\subparagraph{贴心小叮咛}

在睡眠中有无阴茎勃起,是衡量在清醒状态下阴茎能否勃起的可靠标志。在睡眠中阴茎能勃起,说明性功能正常,即使阳痿,也是心因性(精神性)的;睡眠中阴茎不能勃起,那就是器质性阳痿,可见于糖尿病、神经损伤或血管阻塞等原因。

如何使用简单的方法来知道有无夜间勃起呢?方法是:使用一宽度如邮票大小的长纸条,用针刺一排如邮票上的针孔,睡前环绕固定在阴茎上。如果第二天纸条裂开、断裂,那就是夜间勃起引起的,这在医学上称之为“邮票试验”。

有些男子,一旦早晨勃起现象暂停或松弛时,就怀疑自己患了阳痿症,于是忧心忡忡,越是担心自己有阳痿,早晨勃起现象就越少,甚至软弱无力。这是一种可怕的恶性循环,因为心理上的不安,会导致生理上的不健全;而生理方面的毛病又诱发心理上的进一步不安,如此往复,越来越坏。此时要赶快求医诊治。

\paragraph{健康护理}

阴茎的健康护理对于男性的生殖健康和性健康至关重要:

- \textbf{保持清洁卫生}:
  - 每天用温水清洗阴茎和阴囊,清除包皮垢,避免细菌滋生和感染。
  - 清洗时应将包皮上翻,彻底清洗包皮腔,然后将包皮复位,避免发生包皮嵌顿。
  - 避免使用刺激性的肥皂或清洁剂,以免损伤阴茎皮肤。

- \textbf{注意性生活卫生}:
  - 在性生活前后注意清洗阴茎和外阴,使用安全套,避免性传播疾病的感染。
  - 避免多个性伴侣,减少性传播疾病的风险。
  - 性生活不宜过于频繁,避免阴茎过度疲劳和损伤。

- \textbf{避免损伤}:
  - 避免过度手淫或粗暴的性生活,以免损伤阴茎海绵体或包皮。
  - 避免长时间骑自行车或骑马,减少对阴茎的压迫和摩擦。
  - 避免使用刺激性的药物或润滑剂,以免引起过敏反应。

- \textbf{定期检查}:
  - 定期检查阴茎的大小、形态和功能,注意是否有异常肿块、溃疡、分泌物等。
  - 如果发现阴茎有异常情况,应及时就医,进行诊断和治疗。

- \textbf{保持健康的生活方式}:
  - 戒烟限酒,避免滥用药物。
  - 保持充足的睡眠,避免熬夜。
  - 适当运动,增强体质,提高免疫力。
  - 均衡饮食,多吃富含维生素和矿物质的食物,避免过度节食或暴饮暴食。

\subparagraph{贴心小叮咛}

男性也要经常清洗会阴部。清洗时应注意先洗前部的阴茎、阴囊,尤其注意清除包皮内的污物,然后清洗后部的肛门,使这些部位保持清洁。有包皮过长和包茎的青少年应请医生检查一下,看看是否需要进行手术。

\paragraph{常见问题及处理}

\subparagraph{包皮过长和包茎}
- \textbf{定义}:
  - 包皮过长:包皮覆盖阴茎头,但能自由上翻露出阴茎头。
  - 包茎:包皮口狭窄,包皮不能上翻露出阴茎头。

- \textbf{症状}:
  - 包皮垢积存,容易引起包皮炎、龟头炎等感染。
  - 可能导致排尿困难、尿流缓慢等症状。
  - 长期刺激可能增加阴茎癌的风险。

- \textbf{处理}:
  - 包皮过长:注意清洁卫生,定期清洗包皮腔;如果经常发生感染,可考虑行包皮环切手术。
  - 包茎:必须行包皮环切手术,以避免并发症的发生。

\subparagraph{勃起功能障碍}
- \textbf{定义}:又称阳痿,是指男性持续或反复不能达到或维持足够的勃起硬度,以完成满意的性生活。

- \textbf{原因}:
  - 心理因素:焦虑、抑郁、压力、夫妻关系不和谐等。
  - 生理因素:血管疾病(如高血压、动脉硬化)、神经疾病(如糖尿病神经病变)、内分泌疾病(如睾酮缺乏)、药物影响(如抗高血压药、抗抑郁药)等。

- \textbf{处理}:
  - 心理治疗:如性心理治疗、认知行为治疗等,帮助患者消除焦虑和压力。
  - 药物治疗:口服磷酸二酯酶-5(PDE-5)抑制剂,如西地那非(伟哥)、他达拉非(希爱力)等,改善勃起功能。
  - 物理治疗:如真空勃起装置、阴茎海绵体注射等。
  - 手术治疗:如阴茎假体植入手术,适用于严重的勃起功能障碍患者。

\subparagraph{早泄}
- \textbf{定义}:是指男性在性交时射精过快,无法控制射精时间,导致性伴侣不能获得满意的性生活。

- \textbf{原因}:
  - 心理因素:焦虑、紧张、压力等。
  - 生理因素:龟头敏感度高、神经反射过快、内分泌失调等。

- \textbf{处理}:
  - 心理治疗:帮助患者消除焦虑和紧张情绪。
  - 行为治疗:如挤压法、停-动法等,训练患者控制射精的能力。
  - 药物治疗:口服选择性5-羟色胺再摄取抑制剂(SSRIs),如达泊西汀,或局部使用麻醉剂,降低龟头敏感度。

\subparagraph{阴茎异常勃起}
- \textbf{定义}:是指阴茎持续勃起超过4小时,且与性刺激无关,是一种急症。

- \textbf{原因}:
  - 血液疾病(如镰状细胞贫血)、神经系统疾病、药物影响(如壮阳药、抗抑郁药)、肿瘤等。

- \textbf{处理}:
  - 立即就医,进行紧急处理,如阴茎海绵体穿刺放血、药物治疗(如注射去甲肾上腺素)等。
  - 延误治疗可能导致阴茎海绵体纤维化,永久性勃起功能障碍。

\subparagraph{阴茎癌}
- \textbf{定义}:是发生在阴茎头、包皮或阴茎体的恶性肿瘤,多与包茎、包皮过长、HPV感染等因素有关。

- \textbf{症状}:
  - 阴茎头或包皮出现溃疡、肿块、分泌物等,久治不愈。
  - 可能伴有疼痛、出血、腹股沟淋巴结肿大等症状。

- \textbf{处理}:
  - 手术治疗:如阴茎部分切除术、阴茎全切术等,是主要的治疗方法。
  - 放射治疗:适用于早期阴茎癌患者。
  - 化学治疗:作为辅助治疗,用于晚期阴茎癌患者。

\subparagraph{龟头炎}
- \textbf{定义}:是龟头和包皮的炎症,多由细菌、真菌等感染引起。

- \textbf{症状}:
  - 龟头和包皮红肿、瘙痒、疼痛。
  - 可能伴有分泌物增多、异味等。

- \textbf{处理}:
  - 保持清洁卫生,用温水清洗。
  - 根据病因使用抗生素或抗真菌药物治疗。
  - 如果是包皮过长或包茎引起的,可考虑行包皮环切手术。

\subsection{阴囊}

阴囊为一皮肤囊袋,位于阴茎的后下方。阴囊的皮肤薄而柔软,有少量阴毛,色素沉着明显。阴囊壁由皮肤和肉膜组成。肉膜含有平滑肌纤维。平滑肌随外界温度呈反射性的舒缩,以调节阴囊内的温度,有利于精子的发育。如外界温度高时,平滑肌舒张;而外界温度低时则收缩。肉膜在正中线向深部形成阴囊中隔,将阴囊腔分为左、右两部,分别容纳两侧的睾丸和附睾。阴囊内含有睾丸、附睾和输精管的起始段,其独特的结构和功能对于精子的生成和发育至关重要。

\subsection{男性内生殖器官}

发育正常的成熟男性性器官分为内外两部分:外生殖器和内生殖器。内生殖器官由睾丸、附睾、输精管、射精管、尿道以及精囊腺、前列腺、尿道球腺等组成。

\paragraph{睾丸}

睾丸是男性生殖腺,为卵圆形腺体,左右各一,表面光滑,分别由精索悬吊于阴囊之中。每个睾丸重10~15克,平均长3.34厘米,宽2.32厘米,厚1.74厘米。右侧略大。若睾丸体积过小,会引起生精不良,影响生育。睾丸共有200多个小叶,每个小叶有2~4条曲细精管,管壁内有生精上皮,其中的精原细胞可发育生成精子。曲细精管之间为间质细胞组成的间叶,分泌雄激素。睾丸的功能包括产生精子、分泌雄激素等。

\subparagraph{贴心小叮咛}

男性在10岁左右睾丸开始逐渐增大,到性成熟时一般左侧较右侧的睾丸略低一些、大一些,极少完全一样大的;一个大点、一个小点是正常的,不必大惊小怪。每个成人的睾丸平均长4~6厘米,宽2.5~3.5厘米,前后直径约3厘米。若睾丸体积过小,会引起生精不良,影响生育。(跟上面有冲突,需确认。)

\paragraph{附睾}

附睾紧贴睾丸的上端和后缘,为半月形小体,左右各一,分头、体、尾三部分。附睾头较膨大,体和尾逐渐变细。附睾的主要功能是储存精子。精子的成熟过程需在附睾内环境中进行,精子在附睾内通常停留10~25天,以获得运动和受精能力,才能成熟。附睾尾部是精子的储存场所。排精时,由于附睾及输精管的收缩,精子随同精液通过射精管和尿道排出体外。

\paragraph{输精管}

输精管为平滑肌组成的成对管道,呈白色,长约40厘米,宽约3毫米,分为睾丸段、精索段和盆段。

\paragraph{射精管}

输精管沿睾丸后缘上升,进入盆腔与精囊腺排泄管会合,形成射精管。射精管长约2厘米,被前列腺包围,开口于尿道前列腺部后壁的精阜,射精管只在性兴奋达到一定阈值时才开放,使精液射出。

\paragraph{尿道}

男性尿道既有排尿功能,又有排精的功能。长12~20厘米,分为前列腺部、膜部和海绵体部三段。尿道前列腺部是尿道穿过前列腺的部分,开口于精阜,是射精管和前列腺排泄管的开口处。尿道膜部是尿道穿过尿生殖膈的部分,是尿道最狭窄的部分。尿道海绵体部是尿道穿过尿道海绵体的部分,终止于尿道外口。

\paragraph{精囊腺}

精囊腺是输精管末端发出的盲囊,为成对梭形体,附着于膀胱底部。管道一端封闭,另一端开放,与输精管联合构成射精管。其功能是分泌精囊液,参与精液的合成。精囊液中含有果糖、前列腺素等物质,为精子提供能量和营养。

\paragraph{前列腺}

前列腺位于盆腔内的膀胱底部、膀胱颈附近,围绕尿道,呈扁平栗子状。前列腺宽约4厘米,长约3厘米,厚约2厘米,重约20克。前列腺是最大的男性附属性腺,主要功能是分泌前列腺液。前列腺液含有大量的钠、钾、锌等元素,对精子的生理功能有重要的影响。这种液体由腺体细胞不停地制造和分泌,在性兴奋时可以大量产生,平时有部分渗入润滑尿道。

\paragraph{尿道球腺}

尿道球腺在前列腺下方,尿道两旁,为一对豌豆大小的圆形小体。在性兴奋和射精之前有少量分泌液进入尿道,分泌液呈碱性,可中和酸性的尿液,有利于精子生存。尿道球腺的分泌液还有润滑作用。

\subsection{男性性感区}

性感区是指某些部位的皮肤对异性的性刺激很敏感,其敏感性与性兴奋保持反向联系,在青春期后特别显著,性感区在接受异性性刺激后,可使得性兴奋性逐步增强,促使性行为的发生。男女在性解剖、性反应方面均有明显不同,其性感区也有很大差异。男性的性行为较为主动,其性感区较为狭窄和集中,由强至弱可大致分为三部分。

\subsubsection{第一部分:阴茎区域}

该区域对性刺激最为敏感,如龟头富含感觉神经末梢,对性刺激十分敏感,是男子的主要性感区。阴茎颈部、阴茎系带、阴茎体部的皮肤对刺激很敏感,均为男子的性感区。

\subsubsection{第二部分:阴茎周围区域}

这一部分对性刺激也相当敏感,如阴囊、阴阜、大腿内侧及会阴部皮肤等其敏感性均高,均为男性性感区。

\subsubsection{第三部分:其他区域}

口唇、舌头以及胸部和手指、手掌。口唇、舌头对性刺激也很敏感,在接吻时,异性对口唇、舌头的刺激可明显地激发男性的情感与性欲。手指、手掌及胸部的某些部位对异性的刺激也较为敏感,为男性性感区之一。如男性在与女性接触时,通过手指、手掌抚摸女性的身体可以产生强烈的性兴奋。

\subparagraph{贴心小叮咛}

适当有度地刺激性敏感区,不但可以唤起春情,并且可以造成性欲的亢进,所以说性敏感区实在是正常性生活中十分重要的部分。要讲求性生活的美满,就不能抹杀或简化利用性敏感区的作用。

% 结束文档
\end{document}