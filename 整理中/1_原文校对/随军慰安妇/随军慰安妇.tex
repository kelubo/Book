% 随军慰安妇
% 随军慰安妇.tex

\documentclass[12pt,UTF8]{ctexbook}

% 设置纸张信息。
\usepackage[a4paper,twoside]{geometry}
\geometry{
	left=25mm,
	right=25mm,
	bottom=25.4mm,
	bindingoffset=10mm
}

% 设置字体,并解决显示难检字问题。
\xeCJKsetup{AutoFallBack=true}
\setCJKmainfont{SimSun}[BoldFont=SimHei, ItalicFont=KaiTi, FallBack=SimSun-ExtB]

% 目录 chapter 级别加点(.)。
\usepackage{titletoc}
\titlecontents{chapter}[0pt]{\vspace{3mm}\bf\addvspace{2pt}\filright}{\contentspush{\thecontentslabel\hspace{0.8em}}}{}{\titlerule*[8pt]{.}\contentspage}

% 设置 part 和 chapter 标题格式。
\ctexset{
	chapter/name= {第,章},
	chapter/number={\chinese{chapter}},
	section/name={},
	section/number={}
}

% 图片相关设置。
\usepackage{graphicx}
\graphicspath{{Images/}}

% 设置署名格式。
\newenvironment{shuming}{\hfill}

% 注脚每页重新编号,避免编号过大。
\usepackage[perpage]{footmisc}

\title{\heiti\zihao{0} 随军慰安妇}
\author{(日)千里夏光}
\date{}

\begin{document}

\maketitle
\tableofcontents

\frontmatter

\chapter{安魂之碑——一个新女性史的课题(代序)}

在日本军的占领地上,有叫作慰安所的房屋。在这所房屋的前边,士兵们排着长队,等待着轮到自己。在慰安所中的简陋的房间里,迎接他们的是慰安妇。本书《随军慰安妇》中提到,士兵们等待的队伍也有排得长达三公里的时候。

在那长长的队伍中,站着三千多人。当时慰安妇只有十个人,这就意味着一名慰安妇,一天须接待370名以上的男人。

这是异乎寻常的情形。但这种情形却是在号称皇军的日本军的管理之下进行的。

这样的设施和女人们的存在,防卫厅编纂出版的战史《大东亚战争》上不会记载,其实际情况的记录也没有留存下来。仅有马赖义、伊藤桂一及其他作家,将其写进了他们的小说里。

把它作为纪实战记整理出来,本书的作者千田夏光,既是头一个,也是唯一的一个吧。

纪实文学的采访和调查,困难很多,特别是本书,有许多的障碍。那是因为那些当过慰安妇的女人们,试图隐瞒她们从前的身份,不想公开自己的经历。

然而由于本书作者遇见了原军医麻生彻男先生,因而得以了解慰安妇的诞生、军队的管理情况、营运方式等具体问题,作者以此为线索,走访了大量证人,把它秘密的实况给明朗起来了。

慰安妇组织,始建于中日战争开仗的同时,太平洋战争一打响,慰安妇便从亚洲大陆各地,被送到了太平洋上的各个岛屿。没有去的,据说只有新几内亚。

但是,不知道陆续被送到整个战争区域的慰安妇总人数。据说1938年,打下中国武昌、汉口时,分散在中国各地的慰安妇,就有3万至4万名。

1941年6月,日本军以“关特演”(关东军特别大演习)的名义,把兵力集中在苏联边境,准备对苏作战。此时距日本后来和美、英开战,袭击珍珠港只有半年之久。

“关特演”集中了70万兵力。为此,日军计划搜集两万名慰安妇送往攻打苏联的战场。而实际上,只搜集了1万名。

在这些人数众多的慰安妇中,最多的是朝鲜女性。特别是“关特演”中搜集的,几乎全都是朝鲜女性。

日本军队之所以能搜集这么多朝鲜女性,是因为朝鲜是日本的殖民地,朝鲜民族被置于歧视和压迫之下。但是,朝鲜的女性不管怎么受压迫,或者贫困,也不会自己主动去当慰安妇。年轻的女性们以“挺身队”的美名,被强制性地征用了去。所谓挺身队,就是在军需工厂劳动的临时工,所以年轻女性都前往应征。结果却是在军队和行政的欺骗之下被带走,被迫成为慰安妇。

公开军队这样的秘密,不仅仅是困难多。作者有时还受到威胁和非难。遇到这种情况的,不只是本书作者一个人。1984年8月战争纪念日前后,报纸上展开了一轮到底有无“南京大屠杀”这一事实的讨论。其中一些谈了经历的旧军人,遭到电话和信件的中伤、威胁和谩骂,因而向报社求救。

直到如今,仍有不少相信天皇军队之正义且想保卫其名誉的狂信忠君爱国之徒。本书作者出版《随军慰安妇》一书,是需要不屈不挠的勇气的。

作者在致力于写作《随军慰安妇》以前,曾写过《随军护士》一书。战场上护士的行动纪律严谨,秩序井然。但是,随着日本军的连续惨败,医疗用品缺乏起来,收容的患者猛增,粮食也被吃光了。通常护士一个人,要负责照料700名甚至1000名患者。为此,成了过于残酷的重体力劳动。在退却、败走时,由于饥饿、疾病和枪炮弹,有很多护士遇难了。或者还有的被英军俘虏;被带走的一些人中,也出现了用自杀来保卫自己身子的悲剧。

在战后,发放给随军护士的津贴和抚恤金没有多少,因而生活陷入穷困的人也不少。军人、遗属各自都领取了抚恤金和养老金,但却没有旧军人为护士们奔走的。遣送随军护士的日本红十字会,也说些“护士必须以博爱的精神多做无偿的奉献”的大道理,对她们不理不睬。竟然到了这种程度,所以,也就更没有人为随军慰安妇开辟救济、取得报偿的道路了。

作家伊藤桂一,在其所著的《士兵们的陆军史》中,写下了如下的充满义愤的文字:“我觉得在靖国神社的院内建立随军护士和慰安妇的忠灵塔之类也是可以的。特别是慰安妇,她们和士兵们一样消耗生命。应该说她们在上实践了诏敕中‘博爱及于众’的教导。”

千田夏光在《随军慰安妇》和《随军护士》里,也表达了作者痛切的心情。或者也可以说,作者强烈的愤怒结晶成为这两部著作。这种愤怒,就是对国家、政治、军队以及对战争、对残暴和邪恶的极大的激愤。

千田夏光以充满痛切的心情所写下的这两部著作,可以说是为那些不为人知而埋骨异国或者虽然回归了故国但不得不隐姓埋名的女性树立的真正的安魂之碑!

提到朝鲜人慰安妇,不由使人想起在战争中日本强加给朝鲜的种种强权政治。日本政府在朝鲜强行推行“皇民化”的政策,硬要他们使用日语;逼迫他们更名改姓,改用日本式的姓名;对青少年实行了“志愿”兵制度;强制征用壮年、老年男子送往日本本土,让他们在矿山上从事过于繁重的劳动。其人数,据说有60万人以上,而且其中大部分人死于矿山,或者遭到了杀害。以“皇民化”政策之名所强行推行的,是对朝鲜人人权的剥夺,即无视人性的暴力镇压。

与此同时,5万至7万名朝鲜女性,在各种美名的欺骗下,被强制到战场上去卖春。战争与性欲,自古以来就有着密切的关联。斩断这种关联,才能开辟真正和平的道路。由于遭到歧视和蔑视,年少纯洁的女性,被当成了性奴隶,这是现代女性史上的重大事件,这也是在考虑21世纪女性问题时,所必须正视的现实。

为此,《随军慰安妇》和《随军护士》两部著作,成了珍贵的资料,必须为更多的女性所阅读才行。

1984年10月我通读着本书,当时韩国的金斗焕总统作为韩国的国家元首,第一次正式访问日本。在6日晚上的宫中宴会上,天皇所致的欢迎词中有以下一段话:“在本世纪的一个时期里,两国之间存在着不幸的过去,这实在令人遗憾,我感到不能再使之重演。”

如今还悄悄地活着的当过慰安妇的朝鲜女性们,是以什么样的心情听这样的话的呢?我想她们或许没能听到吧。因为我曾经在照片上看到,在这样一些女性的狭窄而摇摇欲坠的斗室里,连一台电视机都放不下!

~\\

\begin{shuming}
高术俊朗
\end{shuming}

\mainmatter

\chapter{慰安妇这主意是怎么想出来的}

《广辞苑》上可以找到“慰安妇”这一词条,其解释为:“随军到战地部队,安慰过官兵的女人。”这一称谓,包含着她们的悲哀。自那以后已经过去28年了,却没有人谈起她们。然而,如果有愿意谈起过去的慰安妇,她一定会这样说:“我们的悲哀,决不会永远变成化石的。”

\section{吃人肉而活下来的士兵}

我想先从这个故事讲起。

同我谈话的,是我花了几年的工夫才找到的西山幸吉。为什么说花了好几年才找到呢?说来话长。以前曾经有个番号为步兵第一四四团(团长楠濑正雄上校)的部队。这是一支因为太平洋战争(第二次世界大战)的爆发而临时在四国组建的部队,这支部队名义是一个团,但实际上却拥有四千多名官兵。1942年1月22日深夜,这个团奉命强攻腊包尔。当时新加坡还没有打下来,南方战线还在继续混战。尽管如此,这支部队却受命去攻打远离日本本土五千余公里的作为敌人心脏的这一据点。

同年7月2O日,这支部队撤出腊包尔,侵入新几内亚,受命攻打莫尔兹比港。在没有粮食补给的密林中作战80余日之后,据说这支部队中出现了靠吃人肉活命的悲惨一幕。吃人肉事件在瓜达尔卡纳尔岛,以及在战争末期的吕宋岛上也有过,然而就我们所知,是这支部队开的头。我想进一步了解当时的悲惨状况。

但是,由四国的多度津港出发的,由四千多名官兵组成的这个团,活着回到日本的仅仅只有两个人。而西山幸吉,就是其中之一。他就是能就我想知道的情况提供证言的两名活证人中的一个。

西山幸吉在东京都大田区盖了一座小小的工厂。他和儿子一起花了几年的时间,设计出了能自动处理养猪场排泄物的机械装置,他本人就是生产这种装置的技术人员。我去时,正赶上他在30年前在新几内亚感染的疟疾复发。他在病床上指挥儿子工作,身上有一条从肩膀起纵贯脊背,由腰部穿出的枪伤。这是澳大利亚军的机枪子弹打的。

当我说明来意后,一开始,西山欲言又止。但过了一会儿,他便淡淡地跟我讲起了在断绝给养40多天以后,士兵们开始吃敌人尸肉的情形:开始吃人肉时说也奇怪,个个都从臀部的肉开始吃;有一个把一整个肝全都吃下去的人,就像发了疯,从战壕里一跃而出,他的身子被相隔数十米的敌人打成了蜂窝似的。正因为淡淡地谈,所以才可悲。但我关心的,不是这件事,而是在攻克腊包尔之后发生的事。

攻克腊包尔时,该团作为南海支队(堀井富太郎少将)的主力,于1942年1月22日11时40分登陆,到22日晚占领了腊包尔市。西山做出证言说:“有一个营长,他在三个营长中是最勇敢的。他让人把被澳军守备队撤退时丢下的腊包尔市长一家人带来,然后把他的女儿强奸了。在营长完事之后,有几名军官也照样行事。我记得那是23日或24日下午的事。在亲眼目睹这一情形的士兵当中,开始出现一种异样的气氛。被奸污的姑娘脚步沉重地回到双亲膝下,深夜便悬梁自尽了。”营长的名字他说得很清楚,但西山和我约定不予公布。因为这个人还活着。

这一强奸事件当然应该提交军事法庭,但不知何故,部队却没有认定这是违反军纪。想来其理由是,自1937年中日战争以来,这样的事件便是家常便饭。在战场上越是勇猛的官兵,越是频繁地侵犯占领地的妇女。反过来说,也普遍存在这样的认识,即把侵犯女性,看做是官兵勇猛的表现。这与不管是从哪儿弄来的,只要是能盗窃来羊羹和美酒的士兵,就是能打仗的士兵的想法有共通之处。话虽如此,长此以往,占领地的占领行政工作会搞不好,也是事实。何况受害人又是市长的小姐,情况有多么糟糕,是无法预料的。更成问题的是,亲眼看着上级长官干这事的士兵的眼睛,开始出现异样了。

西山苦笑着说:“于是,部队的领导匆忙决定从内地往这里送慰安妇。大概南海支队是第一个在南方战线得到慰安妇配备的吧。”

这已经是几年前的事了。在这之前,我已经知道了存在慰安妇这一事实,也听到过一些有关的事情。然而就在这时,我才下了决心,想尽可能地把有关慰安妇的资料加以整理。这是因为,慰安妇的处境太可怜了。前线的营长当着士兵的面强奸妇女,因为士兵们看得眼热,为了安抚士兵,就在敌机飞来飞去的情况下把慰安妇送来了。

接着西山说道:“她们到了腊包尔开始做生意的第一天,士兵们的队伍便排了3公里长,她们一整天都以这些士兵为对手。说到3公里长,就等于说有3000名以上的士兵在排队。当然啰,女子大约只有十来个人……”

计算一下的话,一个慰安妇,这一天平均是以370个至380名士兵为对手的。这如果不是凄惨的悲剧,又是什么呢?我计划在整理关于她们的材料时,尽可能地把慰安妇的悲惨历史挖掘出来。如果说,士兵们为了国家,一张纸(征召令)就可以让他们奔赴死地;那么她们则是连一张纸都没有,便绝别家园,无声地使自己的肉体遭到奸灭。那些死亡的士兵还被授予勋章和给予即便是少得可怜的慰问金,而她们,却连一块墓碑都没人给建过。

\section{在天皇的军队中设置慰安妇}

在福冈市西K茶屋町,有位叫麻生彻男的妇产科医生。地点在东公园一带。经福冈民间广播报道部的一位朋友介绍,我才得以认识麻生先生。朋友是听说我在收集有关慰安妇的材料,才告诉我的。那位朋友说:“确切的情况,没见到麻生先生本人不知道,但他作为军医,好像给头一批陆军慰安妇检查过身体。”于是我乘坐飞机飞往福冈,去拜访这位麻生彻男先生。

麻生彻男是一位剪着小平头,初上年纪的人。他开了一所妇产科医院。据说在业余时间他也曾担任过助产士学校的老师。1937年中日战争开始时,他应召作为军医少尉被送往上海战线,在战败之前经历了许多战争。

寒暄之后,他一边喝着美丽的胭脂色的红茶,一边对我说:

“如您所说,我和另一位妇产科医生一道,给第一批慰安妇做了身体检查。她们在军队的管理之下,虽然不是军人和军中文职人员,但身在军队机构之中。我还记得当时自己很纳闷,在战场上为什么需要这样的女人,军队中为什么有这样的女性?地点是上海战线,时间是1938年早春。详细日期已经记不清了。”

麻生原军医的这番话,我还是头次听说,出乎意料。按照他的这番话,在1938年早春的时候,那堂堂的陆军就以军队的名义保有这样的慰安妇了。军方有卖春的女人,绝不是件光荣的事。不,毋宁说作为天皇的军队,如能避免,更应该极力避免才是。这里,我们不妨对中日战争本身做一回顾。中日战争众所周知,于1937年7月7日夜间爆发于北京郊外,一开始被称为"华北事变",但在同年8月13日夜,如同癌的转移一样,扩散到了上海。

更加详细点说,7月7日日本在北京挑起战端的情形,更是众所周知。1937年7月7日傍晚,日本陆军“华北驻军”第一团第三营清水连在龙王庙附近演习时,遭到了数十发子弹的射击(见该团战斗详报)。连长当即命令部下中止演习,在点名时,发现少了一个人。团长判断这个人有可能被击毙了,于是出动了一个营的兵力。团部要求宛平县城的中国县政府对射击一事进行认错和调查,但在得到答复之前,营部(一木清直少校,后来在瓜达尔卡纳尔岛战死)就下令向该城发动炮击。

这时,那个去向不明的士兵,向连长、营长以至团长都报告情况说,由于解手而没有赶上点名。但在这些指挥官的头脑中只有侵略中国这一念头。他们已经在1931年发动了“满洲事变”,让殖民地傀儡建立了“满洲国”。染指华北,也仅仅是时间的问题。下令炮击的营长、团长,在他们心里起作用的,也许只是作为军人的功利之心。军人被功名心弄得发狂的时候,就失去了道义。这次攻击,一石激起千层浪。去向不明的士兵原来是在解手这件如同“相声”般的事实,长时期以来被作为军事机密秘而不宣,公开的只推说夜间枪声四起。

上海,这一知识阶层聚集的城市,也出离愤怒了。人们回忆起1931年9月18日夜间发生的“满洲事变”,关东军也即日本军一面用小型炸药爆破自己的铁路(南满铁路),一面宣称是“中国军队的行为”而开始了炮击,不久就占领了满洲全土,难道这次的“华北事变”与上一事件就没有关系吗?何况这次发生在北京郊外的事件,也与上一事件太相似了!然而就在8月9日,上海战线的导火索也出现了。日本上海海军特别陆战队西部派遣队长大山勇夫中尉,在乘坐斋藤一等水兵驾驶的汽车视察时,在虹桥飞机场附近遭到了射击。终于在8月13日傍晚,中日两军开始了交战。

在“华北事件”发生仅仅第四天头上,日本政府就派出3个师的陆军部队,不久便增至5个师;紧接着,上海战线也于8月15日,增派第三师和第十一师。然而中国张治中将军的3个师以及陈诚等率领的蒋介石直系军的抵抗也相当激烈,以至出现了日本军的战斗部队有的损失了八分之七的局面。于是,日本政府又增派了第六师、第十六师、第一一四师,还命令以第十八师为主力的柳川兵团,由侧面的杭州湾登陆。

麻生原军医应征的第十一军就包括这个第十八师。该军在付出了战死9115名官兵、负伤31259名官兵的代价之后,总算是打下了上海地区。之后就像要报这个仇似的逼近了南京。12月10日冲进光华门,13日占领了南京。电力函授大学教授臼井胜美就此写道:“自此以后历时一周,城内城外不折不扣地化作了恐怖和死亡的街。”

该教授还写道:“农村地带也是如此,军队通过地或者驻扎地的周围,方圆10公里至40公里的地方,所有的农户都蒙受了火灾、掠夺、征用之祸。农民失去了刚刚收获的粮食,有很多人成为难民四处流浪,其状惨不忍睹。”于是我想,日本政府考虑让华中派遣军抱上“慰安妇”,其秘密仿佛就在这儿。西山所谈的刚刚打下腊包尔时某营长的强奸事件,意外地与此重叠在一起了。在掠夺和征用中,不会不附带那种事的。

于是,军部首脑也不得不考虑,就是那些行为使得占领地的治安变得极不安定了。据说在腊包尔,那个受害者的父亲气得眼睛充血,母亲冲到厨房去拿菜刀。这些事,即便发生在普通农民的身上,其愤怒的情绪也该是一样的。关东军在纪念“满洲事变”五周年时,曾起草过这样的文章:“大和民族内藏优秀的资质和卓越的实力,外具宽仁,指导诱掖其他民族,补其不足,鞭策其不争,有着天赋的用以共同完成道义世界的使命。”但是,这样一来,“宽仁”不见了。“宽仁”一消失,“指导诱掖”也好,“鞭策”也好,不是也办不到了吗?这就是军部首脑的考虑。

我想找出从1937年8月14日起到同年12月2O日止,也即上海战事开始到南京战争结束为止的华中派遣军的军事法庭记录。可是令人吃惊的是,在这4个月之间因掠夺而被起诉的人只有20名,因强奸而被起诉的只有两个人。这是厚生省(卫生福利部)保存下来的军部记录。然而这些只是冰山的一角,据预测,至少也有数百件甚至上千件以上的强奸事件,而军事法庭有记录的却仅仅只有两件,这又是怎么回事呢?是我想象错了吗?然而,一位不愿意透露姓名的人苦笑着给我讲述了这样的事实:

“如您所知,军法法庭就是军队内部的法庭,在那儿被判一个月以上徒刑的人,就由军士降级为一等兵。不管是上士也好,中士也好,都降为一等兵。战场上只有一审,不允许上诉。而如果犯了强奸罪,大致是判一个月以上的徒刑。昨天还过着坐享其成的生活呢,一下子就得当牛当马供人使役,这太可怜了,队长首先就有这种想法。再说,自己的部下如果做了军事法庭上的被告,关系到队长的成绩,影响自己以后的晋级,因此,他也就当作没看见。在军队,如果没有人看见,就构不成犯罪。还是到在中日战争中从过军的高级军官的家里去看看吧,从中国掠夺来的艺术品至今还到处都是。我也能够指出他们的名字来。总之,知道高级军官这些底细的老军士,如果耍赖要他们包庇伙伴和部下的话,这些高级军官当然就不会把犯了强奸罪的士兵交给宪兵队的。再说……”

说到这里,他喝了一口水,又说道:

“搞女人的士兵即所谓"强兵",勇敢的士兵。身为下级干部的连、排长,对这样的士兵不愿意撒手。因为一撒手,就会影响到自己的战功。”

谈话又与腊包尔的话头联系到一起了。他还说:“从杭州湾到南京,有一天之内两次强奸妇女的军士。他还作为‘勇猛的班长’得到了勋章。”他是步兵一二四团(在福冈编制而成)的成员,自杭州湾登陆以来打下南京,后来又参加了攻打新加坡的一位原军士。即使如此,因掠夺被起诉的只有20名官兵、因强奸而被起诉的只有两名官兵这个极小的数字的秘密,仅仅是由于这些原因吗?

一位在第十一军兵站部担任后方工作的军官向我做了这样一番解释:军人喜欢总结战争的经验教训。撇开“满洲事变”那样的短期战争不说,在当时高级军官的脑子里,一直回响的是从大正七年七月到大正十一年六月之间出兵西伯利亚的战争教训。当时他们作为下级干部,亲自见证了日本军的可悲状态。那么,这个出兵西伯利亚的可悲战争教训又是什么呢?那就是“性病”。在那场战争中,出现了远远超过了战死者人数的性病患者,虽经过军医的救治,但仍然明显地削弱了军队的战斗力。哪怕是打下南京之后,在采取彻底抗战政策的中国面前,他们所担心的,是同样事态的再次出现。

\section{大正六年的出兵西伯利亚}

大正六年十一月七日,推翻了克伦斯基政府的布尔什维克党宣告“十月革命”成功和建立了苏维埃政权。此事对英、美等列强的冲击极大,闹得像地球起了火。之所以如此,是因为在同年三月的“三月革命”中,罗曼诺夫王朝的皇帝尼古拉二世被迫退位。第一次世界大战,明显地暴露出君主专制制度的弊端。君王作为这一制度的代表相继被赶下台。当时在日本,正是寺内正毅(陆军上将,后为元帅)内阁的时代。为避免波及自身,日本政府把“多数派的党”这一意思的“布尔什维克党”故意歪曲地翻译成为“过激派”,妄图使之妖魔化,使国民对它产生恐惧。与此同时,抵制布尔什维克革命的浪潮从西伯利亚往东波及,这也代表了作为统治阶级的寺内内阁的意愿。

俄国“十月革命”动摇了日俄同盟,日本决定全面控制中国以便干涉苏俄,并与段祺瑞政府就共同防御办法缔结协定。翌年,大正七年一月,日本海军派出两只军舰前往海参崴港。理由是协助受布尔什维克势力威胁的俄国临时政府。

“十月革命”后俄国退出战争,其领土上由捷克和斯洛伐克族奥匈战俘和俄国籍捷克和斯洛伐克人组成的捷克斯洛伐克军团有5万人之多。其领导人希望继续与德奥作战。在乘火车前往海参崴途中,传来德国人要求苏俄拘禁捷克斯洛伐克军团,并将其囚禁在战俘营的要求。于是,该军团发起暴动。为了在“过激派”手里保卫这些转移的人们,就有了让日美军出兵西伯利亚的借口了。

被视为这次出兵的主力的日、美两军出兵的兵力,提议为“各自以约七千名的军队登陆”。那意思是不让日本独尝甜头。

在这种情况下,经过双方对出兵兵力的反复交涉,决定日本陆军先派遣1200名官兵(第十二师)去西伯利亚,同月19日美国也派遣与此匹敌的兵力。然而想控制西伯利亚的日本陆军,不满足于所投入的这点兵力。日本一面强调“捷克军有覆没的危机”,一面于8月9日将满洲(现在中国东北)的第七师派往满洲里,又从贝加尔地方派去了第三师,10日又把第十九师派往乌苏里方面,终于派去了7.2万人的兵力。

结果,在国内反对出兵派和列强的压力下,野心遭到挫败的日本政府,于大正十一年六月二十四日声明“到十月末撤兵”,而那个可悲的战争教训,就是在这期间发生的。

那么,那次“战争教训”有具体的数字吗?

那份资料保存在陆军自卫队卫生学校里。在出兵西伯利亚期间,有关战死、战伤、战病的详情,有多达八卷的报告书,而在封面变黄的报告书里,有一项“性病”。记载的内容从第三师、第五师、第十一师、第十二师、第十三师、第十四师的战斗部队,到派遣军司令部、宪兵队、临时电信队、铁道团、兵站司令部、运输队,极其详细。在担任监督军队风纪之责的宪兵队员中,也出现了26名性病患者,这有点可笑。其详情按月份做了统计,现供参考转载于此:(略)。

\section{军队患性病后就写悔过书}

在此要加以说明的是,这儿记载的数字,仅只是经过军医之手写了病历的。这是必须注意的。这种事哪怕是在旧陆军中,也被看做是可耻的事。首先战伤、工伤,为一等症;内科疾患为二等症;性病被列为最低的三等症,患上性病的人洗澡也被安排在最后,手巾也让他们使用染成红色的。

此外,在大正时代,一旦得了性病,就必须向连长递交悔过书,而且根本就别指望晋级了,所以申报自己得了性病的为数极少。听说上级的私下制裁(殴打)对他们也格外狠。

由此可以很容易地推断,凡经军医之手的,都是重症患者。因此,包括轻症患者在内,实际数字应为以上数字的五至七倍。如果是五倍,就有一万人以上,七倍的话就有1.4万多人了。再说,申报后在接受军医治疗期间,这些患者既不参加演习,也不执行勤务,更不参加战斗。在战斗形势之下,这就相应地削弱了战斗力。

话有点离题,少年时代我在大连念中学。那是昭和十年间的事。在当时的中学里,有预备役军官和现役军官做军事教官。其中有一个人曾参加过出兵西伯利亚的军事行动,是位从士兵、军士升为上尉的人。他在走路时有点罗圈腿,样子有点可笑。有一天,有个同学悄悄地告诉我说:“那小子在西伯利亚得了花柳病,没有完全治好。”因为这位同学的父母是经营妓院的,所以他的话是十分可信的。他说:“因为他忍耐着,得了花柳病还参加战斗,所以才有出息,当上了军官。”我看着西伯利亚派遣军中患性病官兵的人数,不知为什么竟想起了这件事。

不管怎样,对于当时西伯利亚战斗部队的指挥官和作战参谋们来说,这个数字是一种威胁。得性病的人比战死者还多,其数目大约与战伤者数目相等,更成为问题的是其严重地削弱了战斗力。这部分人接受军医的治疗而入院,不能上战场,而潜在患者其战斗力只有通常作战状态的几分之一。就这样,尽管日本在整个西伯利亚有七个师的兵力,但据判断,由于性病的暴发,其实力如同被消灭了一个师。

照这样,就会影响到作战的结果。比方说本打算派遣180个人的连队发动攻击,而实际上只有120个人能参加战斗。而且在当时国力的情况下出兵西伯利亚,如果没有美国的援助,肯定是太勉强了。在那种情况下日本冒然出兵,投入军费十亿元,终于导致国内米价上涨,发生了谷米暴动。战死战伤还有情可原,由那些不是因为直接战斗而使部队战斗力受到损害的个人行为,即得了性病,而使那贫乏的军费又被损耗掉,实在是令人难以忍受的。

那么,为什么在出兵西伯利亚的官兵中,出现这么多的性病患者呢?首先应该提到的就是所谓“西伯利亚卖春姐”。流落到西伯利亚各地的日本卖春妇相当多,但日本兵买不起。在征兵制度下的日本兵的薪水,在昭和初期每月是三元八角一分(二等兵),照此,买不起是当然的。但是,一起出兵西伯利亚的美国士兵,薪水却有日本兵的十倍。那些日本卖春妇便都被他们包下来了。日本兵只好望洋兴叹。这种精神压抑,便以对俄国女性的暴行的形式爆发出来。尽管没有确切的证据,据说失控的暴行带来了性病蔓延的结果,已成为定论。

话扯得太远了,但这位在第十一军兵站部担任后方工作的军官谈的战争教训,就是这个。概括起来说,只要军部对士兵的性欲问题不加以严格管制,任由日军在中国的暴行(1937年8月至12月)持续,日本军方首先就害怕出现像西伯利亚那样因性病而导致日军战斗力严重削弱的状况,其次是害怕民心的背离。

当然啰,非正义的战争会扰乱军队内部的风纪,这也是侵略军的宿命。军部是不是有这种反省,尚属疑问。但基于以上考虑,军部便开始着手建立卖春制度,计划着征集由军队直辖管理的卖春妇。

但是,关于这些事实,尚无当时身为华中战线军官的证言。我曾经找到了几个人,但他们都不愿多谈。

“我当时没有在场,没有担任后方工作参谋的经验。作为作战参谋,在作战开始以前,我也参加过是否性病会导致兵力减少一成的排查工作。由于日本国力的问题,常常要求军队以少数的兵力,与比自己多两至三成的敌人作战。比方说对苏作战时,苏军狙击兵有15个师,日军以这11个师就转为攻击。从战斗力来看,这15比11,是个很勉强的数字。所以在这种情况下,如果在作战开始前就消失了一成,也就是消失了一个师还多的兵力,首先宣战就没有把握,发动攻势也是没有把握的。日本陆军是以攻击为主,在防御的情况下也是以攻为守的。但在防御的情况下,减少一个师就意味着绝对数的不足。在中日战争时,日军更是计划以四成或五成的兵力对付中国军。如果由于性病的蔓延有一成的患病士兵这一可能性的话,那么军队考虑采取相应的对策,我认为也是当然的。即便是我,也会这样做,总之,作为预防措施,我也会考虑搜集些从医学上来说相对安全的女人来。”

能够作为依据的,只有这位在战争末期从陆军大学毕业,然后担任本土决战部队参谋的人的说法。他接着说道:“当时派遣军的长官们,是否考虑到了出兵西伯利亚时的战争教训呢?即使校级军官没有这个意识,但我想在更高级军官的头脑里是有的。比方说作为满洲事变的战争教训,在战场的饮用水上出了问题。饮用陌生土地的生水,腹泻的人超过了一成,确实影响了战斗力。因而在中国战线的校级以下的军官,首先在脑子里就想到这条战争教训。从此以后,在中日战争中,设立了防疫给水部这一饮用水补给部队。所以,从以往经验来看,我觉得校级以上的军官,是会考虑这些战争教训的。”

他还说:“有些在摔跤界被认定很有希望的大力士垮了,听说就是因为得了花柳病。一旦得上那种病,好像膝盖就不行了。在旧日本军的作战思想中,有个包围歼灭的作战方针,对于战备不足的当时的日本陆军来说,为了弥补这一不足,就激励士兵用脚力猛走,让士兵靠两只脚包围敌人才行。因此,部队首长考虑到性病对脚力的削弱,因此才想到慰安妇。特别是在对中国军的战斗中,包围歼灭是第一重要的事。在徐州作战时,日军曾企图包围三倍于己的中国军。徐州战役,是刚刚打下南京之后计划的一场作战。因而日军考虑设立慰安妇,仔细一想觉得好像能够理解似的。不管怎样,这同美军具有的首先对敌来个毁灭性的炮击,然后用步兵部队慢慢地侵入的作战思想,成为鲜明的对照。这多半是由于日、美国力悬殊吧。”

因此说,出兵西伯利亚的战争教训直接导致中日战争中慰安妇的设立,这种说法是没有错的。

总之,就这样,经麻生彻男等原军医做了体检的日本陆军第一批“慰安妇”,于1938年初,匆匆忙忙地被集中于华中战线,并在那里活跃起来了。理所当然,麻生军医对于她们是从哪儿征集来的,是怎样被带到异国来的,一无所知。

还有一件事是他所不知道的。他所看到的设立慰安妇的只有华中派遣军,那么是不是在华北派遣军中也有同样的举措呢。当然啰,在华北派遣军中,当时的慰安妇制度还没有表面化。这是不是说明,从上海战线向南京进攻的部队的恶行,比华北部队更为严重呢?

\chapter{性病的威胁}

确实,当时的日本,到处都挥舞着太阳旗,飘荡着“胜利归来,勇敢战斗”的歌声,在身穿白罩衣、斜挎着肩带的妇女的欢送之下,脸上泛着红潮、剃着光头的出征士兵在街上来来往往。那是军国主义政府的宣传和洗脑的结果。这种观念竟然深入到了娼妇当中去了,实在令人吃惊。

\section{怎样来征集慰安妇}

那么,日本陆军是以什么样的手段和方法来征集慰安妇的呢?是由军人本身来充当人贩子吗?其募集地点又是在哪里呢?

花了好几个月的时间,我终于找到了一位名叫田口荣造的62岁的男子。他住在福冈市的东部,是个围着博多港近海诸岛兜圈子、在小型客船里做买卖的人。他个子矮小、脖子短粗。这个田口,正是1938年华中派遣军开始搜集军队直辖慰安妇时的人贩子之一。他曾经跟着福冈步兵第一二四团,准备带慰安妇去瓜达尔卡纳尔岛。

“你接到命令时是哪年哪月?”

“月日记不清了,多半是1937年年底。”

“那时候你是军人吗?如果是军人,你是什么军衔,属于哪一个部队?”

“既不是军人,也不是军队文职人员。”

“那么说,当时你住在国内,做花柳界的生意,是当地军队委托你征集慰安妇的啰?”

“无意中跟着在福冈编成的团到了中国。也还不是御用商人,如果士兵们说想要酒喝,我就设法弄来……干那些事。”

“薪水是团部给你发吗?”

“饭嘛,团部的士兵们总算是给我吃……”

“不拿薪水?”

“是的。”

“那么,军队说让你去征集慰安妇吗?”

“不光是我一个人。”

“此外还有什么人?还委托给那种团体了吗?”

“这个嘛……可就不知道了。再说已经是三十几年前的事了,不记得了。详细情况,自那以后发生了各种各样的事,都忘了。好像还委托过别人。”

这是一开始和他的对话,从此军队募集了第一批慰安妇,但我们弄明白了并不是由军人直接出面募集的。尽管如此,却不能说军部的手脚就是干净的。他们让既非军人又非军队文职人员,更不是御用商人的人坐上军队运输船,由陆军把他们带上了战场。这多半是一种杂役吧。

和他的谈话还在继续:

“征集是在当地,也就是在上海啰。”

“不,在内地。”

“那么说,你是坐上军队的运输船回日本去搜集女性去的?”

“可以说是那样的。从上海到长崎二十五、六个小时就到,不是什么了不得的事情。”

“你是用什么方法募集的?比方说吧,是委托给了内地专管妓院的人吗?”

“不,零零星星地募集的。在达磨房之类的地方转悠。”

“你所说的达磨房,指的是花酒馆或者是私娼窑吧?地点在哪儿呢?”

“因为我所在的那个部队北九州的士兵多,毕竟还是同乡的女人好吧,所以是在北九州募集的。我记得一开始是在远贺川的河边上寻找来着。开头人数不是那么多……”

“总之你募集的是有卖淫经验的啰。条件是什么呢?对这种女性一般是由老板借给她预支款来拴住她的身子。什么手纸费啦,伙食费啦,衣裳费啦,眼看着借支越来越多,让她动不了身。军队当时是以什么条件要她们去呢?和这相似吗?”

“没有这么厉害。给了她们每个人1000元预支金。把这个全都返还之后就可以自由了。”

“可是手纸费、伙食费、衣裳费什么的呢?”“伙食由军队免费供给,其他什么都不需要,因为是上战场。跟她们明说了。军队在当时有信用,人们都相信。”

“实际上有返还1000元钱,而得到自由的女性吗?”

“有,有。多得很。第一批去的一伙人,最迟的几个月也能还清借支,变成自由之身。可是她们当中的许多人,不想停止这种营生。”

“为什么?”

“应募的当时,有的就说像我这样的身子,还能为士兵们做事,为国家尽力,因此很高兴。因为她们知道,就是获得了自由回到内地,仍然也只能出卖肉体,所以宁愿为士兵服务。当然啰,她们也想挣钱。”

这种把自己的肉体出卖给士兵是为了国家,也就是尽忠义的想法到底是从哪来的?她们对“忠义”两个字,有她们独特的理解吗?我感到吃惊。确实,当时的日本,到处都挥舞着太阳旗,飘荡着“胜利归来,勇敢战斗”的歌声,在身穿白罩衣、斜挎着肩带的妇女的欢送之下,脸上泛着红潮、剃着光头的出征士兵在街上来来往往。那是军国主义政府的宣传和洗脑的结果。这种观念竟然深入到了娼妇当中去了,实在令人吃惊。然而田口说:

“这种募集是军事机密。是的,因为是军事机密,所以应募的女人们都是悄悄地聚拢来的。”

据说有的娼妇的伙伴拿来了一壶酒和干板栗(日本称为"胜栗",象征吉祥)来送她们上阵。她们穿的虽然是丝绸,但只有这一身衣服。下面还是和田口的对话:

“从日本内地去中国时,坐的仍然还是军队的运输船吗?”

“这个成了问题。当时陆军的运输规定中,可以运输士兵、军马、军犬、军鸽,却没有妇女这一项。歌曲中不是有这样的词儿吗?‘不载女人的运输船’。因为这是‘军规’,不能破坏。担任运输指挥的军官对此感到为难。”

“于是,怎么办了呢?”

“决定当物资运输。当成既非武器,又非弹药,更不是粮秣的物资。”

“物资?指的是军需物资吧,那就是把人当成物资了?”

“作为军队,不这样就上不了军队运输船,没有办法。这事没有告诉她们。当然,她们就是知道了,也不会介意的。因为她们受惯了虐待。”

“可是,她们不是按军队的需要募集的吗?哪怕是给她们文职人员的待遇也好啊。”

对此,田口默不作声。军队给予慰安妇待遇的不合情理,田口也无话可说。还有一个问题,也是田口不做回答的。那就是:

“你募集来的女性有多少人?”

“记不很清了,一百多个吧。当然并非全都是我一个人募集的。经我谈妥的,充其量只有二十来个人。”

“全都是日本女性吗?”

“有少量的朝鲜人。”

“她们是从哪儿募集的呢?也是北九州吗?还是让你到朝鲜半岛去了?”

“我记得也是北九州。”

“那么说,在远贺川河边达磨房里,也有朝鲜人女性啰。”

沉默就是对这个问题的回答。被募集来的或者说集中来的朝鲜人慰安妇没有“营生”经验,据此我推断,这些女性是达磨房以外的女性。竟至让外行的女性来当慰安妇,我觉得他才不得不沉默的。据说日本军动员和使用的慰安妇的总数,从1938年到1945年有8万或者10万之多,而其大部分是朝鲜女性,这已经是公开的秘密了。《强行带走的朝鲜人的记录》(未来社),这本书是朝鲜大学教授朴庆植先生根据大量的资料写下的报告。在这本书中,有隶属于一六七九八部队的军队文职人员玉致守(船员,庆尚南道统营郡出身)的证言。书中写道:“仅乘坐玉致守的船被带到南方去的朝鲜女性,就超过了2000人。这些女性在故乡时,即被强迫前往支持战争,说是到军需工厂、被服厂去劳动,被强迫去的多半是17岁至20岁左右的年轻姑娘。她们就是这样作为”物资“被装上运输船送往南方各地的战线,当上了军队的慰安妇。玉氏在冲绳也好,在下关和博多站的候车室、候船室也好,都目击了相当多这样一些女性同胞……”

然而,这些都放在后面再加以叙述。总之,她们这些渡过中国东海的“物资”,被收容进了上海的其美路沙泾小学。日本陆军常常使用小学或体育馆作为临时住处,在这儿也采取了同样的方式。与此同时,在远离市中心的军工路附近的杨家宅,日军建成了一个个小单间组成的几排平房,这就是所说的杨家宅“慰安所”。

\section{定名为“陆军娱乐所”}

麻生彻男军医的上场,就是在这个时候。麻生彻男先生,是九州大学医学部毕业后当上妇产科医生的。当时,作为应召的军医少尉他隶属于第十一军兵站医院。当然啰,在清一色由男性组成的军队里,本不需要妇产科医生。麻生少尉被分配到外科,奉命给上海至南京之间作战负伤后送来的官兵做外科手术。他在埋头于手术时,接到了一项特别奇怪的命令:“鉴于近来陆军娱乐所的建立,麻生军医务必前往其美路沙泾小学为那里的百余名妇女做身体检查。”发布命令的,是兵站司令部。

“当时,您立刻就明白了您所执行的任务的内容了么?”

“不,一开始看到陆军娱乐所这几个文字,还以为是演戏或什么的地方呢。这也是当然的。所以说到待检的妇女,以为是从内地来的弹三弦、唱歌的艺人呢。所以感到奇怪,这样一些妇女,有什么必要为她们做身体检查呢,还以为是生了病呢。与此同时,另一名军医也接到了同样的命令,一问,他也是专学妇产科的。心想一定是到中国来的艺人,得了妇产科的疾病了吧……如此说来,一百多名也未免太多了。”

“那么说,叫做陆军娱乐所,话有点难以开口,还没有告诉您这就是卖春所或做慰安妇买卖的地方吧?”

“是的。当时的陆军是称为皇军的。在杭州湾登陆之际,还充满着威压对方的意思,在登陆前夕,甚至放出了‘皇军百万登陆’的气球,皇军意识极为强烈。如此宣称自己为皇军的日本兵,把这场战争当作圣战的日本兵,为什么需要这种类型的女性呢?而且从文件来判断,觉得这是由军队本身管理的。真是难以理解。”

“从上海到南京的战斗中,日本兵干尽了掠夺和强暴的勾当。可以说是暴行吧,强奸事件极多,您听说了吗?”

“作为传说,听到过两三件,但没听说有那么多。我所工作的兵站医院,和第一线的野战医院不同,在后方能看见随军护士的身影,处于比较安静的气氛当中,也许那些血腥的故事传不过来。但是,也许是战场上的士兵,他们作为人来说,精神状态是异常的,所以,这样一些事件是有的吧。在大的战争之后,好不容易生存下来的人,精神特别亢奋,就容易干出不理智的事情。”

听说在上海战线上,有时一个晚上会受到中国军队的上十次反击,从扔手榴弹到拼刺刀,甚至出现军官、上级军士全部伤亡,最后不得不由下级军士的下士担任全连指挥的连队。据记载(臼井胜美著《中日战争》,中央公论社)有许多连队伤亡了八分之七。在上海战役中,战死9115名,负伤31257人(见上书)。听说在各处战场笼罩着焚烧尸体的烟雾。这就是异常兴奋的根源。特别是白刃格斗,把人类逼迫到了异状的极限。

但是,因此遭受侵害的中国百姓,其心底的愤恨是可想而知的。虽然他们一看见日本军的影子,就被迫挥舞着赶制的太阳旗(旗子中央的太阳不是太小就是太大,颜色也是各式各样)无言出迎。但在中国人的心底里,反日、反日本军的思想(还谈不上思想)翻滚着,这是可以充分理解的。可以说,军队的高级干部所担心的问题之一,也正是这个。

“去做体检的,是麻生军医您一个人吗?”

“我刚才说过了,还有一位军医,此外,还有在兵站医院工作的7名卫生兵,还让做过这种体检的福民医院派去了两名护士,总共去了11个人。这事写在当时的笔记本上了,没有错儿。”

“怎么也不理解,因此一问兵站司令部,才明白是怎么回事。虽说是‘必须进行检查’,但因内容的不同,准备的器具和药品也不一样……一问才知道是这个,心想哈哈……”

“检查的日期您还记得吗?”

“这一点笔记本上没有。记忆也淡漠了。因为已经是35年前的事了。多半是早春吧,确实是微寒的季节,我还穿着大衣呢。地点是在沙泾小学的医务室。检查器材在头一天由卫生兵搬去的。她们也习惯了,我记得一切顺利。一说让她们上‘检诊台’,便满不在乎地上去,一看她们那样子,就知道是老手。可是作为外行的女性就很害臊,轻易不愿意上去。只是难以开口的是,朝鲜女性近于一至两成,准确数字忘了,她们似乎不习惯,只记得做这种检查时,她们扭扭捏捏的。”

“朝鲜女性,是外行人吧?”

“因为没有和她们交谈,说不清楚,其中纯洁的女性,也就是处女很多。年纪轻的女性多数都是的。日本女性,从年轻人到中年人都有,年纪大的略多一些。听方言,立刻就知道是从北九州募集来的。一看就知道她们全都有卖春的经验。令人吃惊的是,其中有些人在腹股沟部有很大的刀伤疤,即患过重花柳病既往症的人。有的女性性器官使用得过于残酷了。”

“问得有些失礼。面对这种情况,您有什么想法呢?”

“她们不是带着慰问皇军的使命,被派遣来的女性吗?可是给我的印象好像是她们在内地无法糊口,才来到战地似的。我认为这样对皇军官兵也是件极大的麻烦事。”

“当然啰,这种情况只在日本女性中存在。朝鲜女性是纯洁的。这且不提,检查的结果,有不合格的吗?”

“我记得没有。多半从内地出发前,做过一定的检查。即使如此,这些日本女性,作为慰安妇质量不高这事,深深地留在了我的脑海里。为此,我在后来写了意见书。”

有关麻生军医的意见书,后面再谈。总之,在这次检查之后,在日本陆军中便第一次正式地诞生了慰安妇。营业地点就在军工路附近的杨家宅。慰问所是一些木造简易房,十间一栋,共约十栋,还有一栋管理处,用围墙围着,据说远远望去很像仓库。各个房间有板门,写着房号。一进门就摆着床,门上有扇30公分乘50公分的小窗。窗玻璃的下边三分之二是毛玻璃,上边三分之一是普通透明玻璃。设计得从小窗上部分能看见室内,设计者的苦心,好像可以理解似的。说是屋顶一下雨就发出很大的声音,也许是白铁板的。那么士兵们是怎样利用这里的呢?还是看看“陆军娱乐所规则”明白得更快一些:

一、本慰安所除陆军军人、军中文职人员(军中民夫在外)之外不许入内。入场者须持来慰安所的外出证。

二、入场者必须在传达室付款,付款后领取入场券及避孕套一个。

三、入场券价,军士、兵、文职人员,每人两元。

四、入场券只限当日有效。如当日不能入内,可凭券换回现金。但一旦将券交给女招待,就不能退还。

五、购券后,按指定房间号入室。时间限30分钟。

六、在入室的同时,将券交给女招待。

七、在室内禁止饮酒。

八、完毕后,立即退出房间。

九、不遵守规定及扰乱军风纪者责令退场。

十、不使用避孕套者,禁止接触女人。

读这些规则,我们发现虽然军队创造了“慰安所”这个词儿,但还是没有加以使用,而是仍然按照日本的习惯,称呼她们为女招待。令人担心的是,“购券后,按指定房间号入室”和“时间限30分钟”这两项,这表明剥夺了入场人选择女性的权利。与其说军队,不如说日本陆军对性的问题的根本想法就在这儿。也就是说,他们的理解,所谓性欲问题,只有腰部以下的性器官。确实,在妓院里(如今的土耳其澡堂也可以)男人买女人的行为,是把性欲的处理放在第一位的。但对于男人来说,并非对手是谁都行。哪怕是一个晚上,也愿意尽可能地选择自己喜欢的女性,并和选中的女性在当时产生点儿恋情。

总之性欲器官,性欲的处理并不是一切。因为我们都是人。这也是游妓院最低限度的伦理、人性问题。你们这些当兵的,是个女人就行,在这儿有着没有人性的军队组织的想法。当然啰,这样毫无选择权,女性那方面更有问题,这也是知道的。但这个问题且让我暂时放下。

再者,“入场者须持来慰安所的外出证”,这条也有问题。两年兵、三年兵这些老兵还好说,新兵和性情懦弱的兵,能那么轻松地去领去慰安所的外出证吗?这似乎是不理解士兵的军官的考虑。但从结果来看,毕竟说明这个娱乐所办得不太成功。因为该是客人的兵,没怎么来。只要看看规则就知道,尝过一次这种方式的味道,就不再想来第二次了。

军队干部们,也慢慢地察觉了吧,大约过了一个来月,在上海叫做江湾镇的地方,让御用商人又开办了几家民营娱乐所。最为重要的也许是,军队害怕由军队自己管理这种女性的事传出去。细想想如果被外国新闻机构给报道出去,确实有失体面。就是让本土的民间知道了,也会给皇军的脸上抹黑。也许是有这种考虑的吧。不愧是商人,他们很理解士兵这些无名小卒们的心情。他们给了当兵的选择对方——女性的权利,对“时间限30分钟”这一条规则,也改为“只要有钱,多长时间都行”。巡察军官前来巡视也一样,军医来做定期检查,也一视同仁。因此从这时起,买卖盛况空前。在门前贴着“欢迎圣战中打了大胜仗的勇士”,“身心全部奉献的大和抚子服务”的条幅。

\section{麻生军医有关慰安妇的意见书}

自从给第一批慰安妇检诊以来,麻生彻男军医不断地接到为娱乐所检诊的命令。他在做这些检诊的过程中,思考了一些问题,并将它们写成了一份意见书,交给了上级领导。由于这份《有关花柳病与慰安妇的意见书》后来成为军方用以制定慰安妇政策的重要资料,因此本书决定全文登载。这份意见书涉及战场上的性问题、军队最为担心的花柳病的预防问题,以及如何提高慰安妇的质量等方面的问题。这份报告由九章组成,尽管当时的麻生军医有皇军意识,但这也是无可奈何的事,其报告还是有一读的价值。

题名为《有关花柳病与慰安妇的意见书》,作者为第十一军第十四兵站医院陆军军医少尉麻生彻男。

这份意见书,是从1939年6月26日杨家宅陆军娱乐所的设立开始,麻生军医为其慰安妇以及江湾镇民营慰安所慰安妇检诊的一年间检诊的结果形成的意见,是一份调查报告。目录如下:

一、绪言

二、娼妇

三、检梅

四、酒精饮料

五、禁欲

六、对花柳病的认识

七、狭义的预防法

八、患者的处置

九、结语

一 绪言

一种疾病对患者及其周围的人影响越大,其疾病的社会影响可以说也就越大。在这种意义上,花柳病不论平时还是战时,与急性传染病不同,有不亚于结核的危险性。因此,为实现将其消灭的目的,至今为止制定的各种对策,实施者皆稍稍缺乏决心,均不可期望得到全面落实。

其目的即对患者进行治疗,且不使健康者受传染。同时,对其社会原因的调查研究尤为必要。进行传染源、病种、数量的统计性的观察,亦有助于将来的决策,不可或缺。然如本症这样有着重大社会性之疾病,一切对策均应有平等之发言权。

二 娼妇

去年卑职在工作中,根据命令,对新来娼妇进行了检梅工作。当时被检者半岛(朝鲜)妇女80名,内地妇女20名。半岛人中疑为花柳病者极少,但大部分内地人眼下虽无急性症状,但尽是甚为可疑者,年龄也几乎都已过了20岁,甚至有将近40岁者,尽是些多年来从事卖淫者。与半岛人之年轻且多是生手恰成鲜明的对照。后者中,有此次事变之际应募者,虽说是未经教育就补充进来了,但可供使用。

一般来说娼妇的质量,以年轻者为好。在慕尼黑市检查时,在2686名娼妇中,患花柳症者达26.5%,按年龄划分则:

16岁以下者19人

16岁至18岁者104名

18岁至21岁者239名

21岁至30岁者281名

且该市未成年之年轻娼妇中,得过花柳病者是:

15岁55%

16岁61.5%

17岁68.6%

照此,愈是年轻,患病率愈低。再如斯图加特的娼妇565人中:

14岁至20岁55%

在慕尼黑市1906年占23.5%,在巴黎市占58%。

再如1932年福冈县至年龄40岁的调查,20岁以下人数为:

艺妓56.3%

娼妓29.1%

女招待44.6%

女佣46.5%

即可以说娼妇中约半数,系年龄在20岁以下者。故对年轻的娼妇加以保护,事关重要且有意义。因此,送往战地的娼妇,须是年轻者。卑职在某地检梅中,常常发现在两腹股沟部有鱼口手术之瘢痕,这明显是打下了既往有花柳病的烙印,类似这样的案例,敢请重新予以考虑。以此作为送给皇军官兵的礼物,实在需谨慎。

送往战地的娼妇,有在内地最后一个港口予以充分淘汰的必要。何况在内地混不下去的女人,投身到战地来,可以说荒谬之极。

在上文中“作为送给皇军官兵的礼物”的句子,在表明军队的慰安妇观的同时,显得非常刺眼。其次,文中强调作为军队慰安妇“须是年轻者”,以及要把半岛人放在焦点上。

杨家宅的第一批慰安妇中有朝鲜人,已如上述。她们是被从南朝鲜带来的。当时的朝鲜被日本掠夺,处于极端贫穷的情况下。一想到此,当时日本募集慰安妇的方法便自然地浮现出来。但作为这个报道的结果,军队的注意力便转向了朝鲜。与此同时,也是朝鲜女性陷入恐慌的开始。

麻生的报告书还在继续。除了军队管理或者其影响下的慰安所之外,一些见钱眼开的中国人和御用商人,也办起了同样的妓院。这就是中国的卖春妇。

与此类似,中国当地卖春妇及难民中卖淫者患梅毒性疾病的情况亦如此惊人。对此,作为军队,如有必要,应把它纳入军用慰安所,置于我方监督之下。对不同意者,须采取断然措施处置。

在德国科隆市的守备队中,为应对花柳病的蔓延,即便采取了严厉的检梅措施也无效,患病者据说高达22%。此即由于私娼跋扈所致。

为此在该市,学习美、英先例,设女警官,令其肃清,据说有了效果。在这儿必须注意的是,在中国娼妓中,一些人对预防性病的方法,特别是避孕套的使用加以抵制,更有甚者予以毁坏。这一点,与敌人用计谋消耗我军战斗力,有异曲同工之效。

说是侵略者对被侵略者怀有恐惧,这也许是恰当的例子。容我往下介绍:

三 检梅

花柳病的蔓延,以其容易传染及扑灭之困难为重大原因,淋病之根治很难。而且,当它一旦侵入妇女的下腹部各内脏器官时,更将事倍功半。因此,检梅也将无效。根据至今为止的文献,所谓有检梅制度的公娼和秘密卖淫所受的感染率不相上下,使人感到这恰如检梅无用论一般。因此,卑职现在在此就检梅加以考察。

其历史久远,对1162年伦敦市外“南部事业”的妓院,就已经有“温茄斯塔僧正”的命令书。1413年及1469年有“求里希”及“鲁泽伦”市议会对此有关的规定。其后,到了1828年,在巴黎市进行了娼妇的登记,并据此受到医师的监督、统制。于是,有了今天检梅制度的确立。

然而,1908年海希特提出卖淫者乃一小部分,即没有达到5%或10%,检梅完全无用。不仅是他,今天仍倡导无用论者也为数不少。他们说登记娼妇数比起全体卖淫妇数来是少数。在已判明数字的柏林、科隆、巴黎市等地,此等秘密卖淫者达公娼的七倍以至十倍。恐怕今天的日本内地也是如此。而且在现在的战地,也承认有与其类似的当地秘密卖淫者出没。从大局来看,军队设置了在其管理下的特殊慰安所,因此,这种检梅无用论全然不适用,且据最近的报告,纽伦贝尔希及波赫米亚,据说频繁的检梅已见成效。检梅虽说有了成效,如不得当,也会功亏一篑。即如欲从卫生上全面提高检梅效果,那么患病者的隔离和治疗则成为必需。不伴随对患病的隔离和治疗的检梅,完全是有名无实。卑职在某地工作中,痛感此点。近来对此无一定之方针,所能做到者,只限于依赖大城市的地方医院治疗。其后过了一年有半,卑职也长时间地远离该项检查工作,因而对目前的情况不详,但如无检梅之后随之而来的彻底治疗,那么检梅又是为了什么呢?军队对此必须适当地加以切实管理。

以上可以看出,自从将慰安所移交民间经营以后,麻生对其管理不善的反省,以及在这种情况下,军队已经不能直接经营这一事实的进退维谷。与此同时,也包含着他作为医生的愤怒。在当时,军方也是在某些方面滑进了慰安妇这一泥塘。

麻生的报告书还在继续:

下面有件事,作为检梅的弊病不得忽视。即该检查者与经营者乃至被检梅者之间的私情问题。即身处检查、监督地位的人的个人登楼,或与娼妇接触之事,是需要考虑一下的问题。如果其滥用职权,就万分遗憾了。让这样的人进行检查,可以说是岂有此理!

还有,在检查娼妇的同时,对妓楼的检查也是必要的。

卑职在某地工作时,曾检查了两所妓楼,其一是作为军用特殊慰安所新建的简易房屋,每个房间有洗涤处,售票处、进出口及其他各项设备,几乎近于理想。另一所使用的是中国住房,其室内的间壁、洗涤处等各项设备均不如意。果然不出所料,在开设后,两处地方均出现了患者,虽人数不多,但值得注意。由此可见,检梅在统制下的军用妓楼还是取得了成绩。

然而不可过于相信其成绩,在一般士兵中有轻视卖淫的危险。

在这儿叙述的是对慰安所设施的批判。自从这份报告书提出以后,便由“后勤参谋”来管理慰安妇问题了。所谓后勤参谋,大致是配合作战计划和收集并分发炮弹、炸药、粮秣、被服等作战必要物资补给计划的。本来是把考虑作战当成任务的参谋,现在要着手管理腰部以下的问题了。这是战场上性问题的不可思议之处。

四 酒精饮料

自古以来,可以说酒和女人是分不开的。尽管酒为百药之长,一般说来酒精的药理性的作用在影响人体的第一期上,使人失去了自制,从而道德的批判能力下降。于是平素哪怕是在未醉状态下所不能干的事,酒醉后便发生了。诸如会若无其事地步入花柳之巷。还由于酒精的原因,性行为的时间也变长了。于是自然招来罹患花柳病危险的事态。且在饮酒后,万事都大咧咧的,放弃了预防措施及预防药的使用。劳姆赫尔特在报告中说,酒精入肚后,人们通常不喜欢使用保护剂,说得实在恰当。酒精与花柳病之间的关系,近来在各方面都进行了调查。据福莱尔说,在182名男子中,在酒精状态时得花柳病者占76.6%;据朗斯塔因的调查,在179名女子中,占43.8%,还有梅拉报告说1225人中占17.7%,海希特说1000人中占43%。



更为有趣的是,据外斯对700名患性病男子的调查,独身者占30%,有妻子者占51%。这就说明有了妻子且通情达理的人,没喝酒到底不会那样去冒险。



英国军队的统计显示,军队的娱乐所如果禁酒,则花柳病会显著地减少。下面是对被收容的1000名患者的统计:



据此可以看出,只要减少酒精的摄入,花柳病也会随之减少。又据斯特达的报告,在马萨尤塞州自发布禁酒令以来,患花柳病患者的数字下降了。据依库特曼的报告,列宁格勒花柳病传染的25%病例是饮酒的结果。然而另一方面在美国军队中,由于1900年以来在服务部完全严禁酒精饮料的结果,士兵便处于不得已而去酒馆和酒吧间的状态,据说花柳病反而日见增加。此点如不深思熟虑,势必变成虎头蛇尾。不管怎么说,在花柳病的传播上,酒精起着重大的作用这一事实,是谁也不能否认的。



卑职据此,切望在军队内将酒的消费限制在最小限度之内。何况至今为止,军队内所发生的大部分事故,都是因为饮酒,我对此之确信愈益强烈。军用特殊慰安所并非享乐场所,因为它是卫生性的公共厕所,所以军方在慰安所中禁止酒类,不用说也是当然的事。然而卑职在慰安所中常常见到酒类饮用的迹象,甚感遗憾。为此也有必要监视营业,监督娼妇,进而对之加以教育指导。



在这儿有"公共厕所"这一刺人眼目的词汇。"享乐"也烙印在人的眼底。一开始下令在杨家宅禁酒的是谁,虽无从得知,但以后巡视慰安所的巡查军官的眼睛开始集中于酒上,实行了饮酒处和慰安所的隔离。





五 禁欲



有人说禁欲有害。他们甚至罗列性欲禁忌现象,陈说其害。卑职认为性欲和精力在个人之间有着很大的差异。有的人性欲虽强,但对不当色欲的抑制,不需借助那样大的意志即可,可是有的人,其性欲强烈,却怎样也抑制不住。



禁欲有害,其结果会发生生殖神经衰弱症,引起前列腺肿大以及附睾炎等。



总之,禁欲目的贯彻情况如何,是以卫生的生活方式为前提的。勤勤恳恳的工作之余,讲求适当的慰安之道,成为不给色欲以可乘之隙的首要条件。这是个为每个人的品质、性格、信念所左右的问题,有些现象不再论述。据卑职看来,只要考虑到东亚百年大计,可以说这一两年的禁欲不在话下。



在这儿,麻生军医提出了清教徒主义的主张。这与前边的观点似乎相反,但这是太没有人性的战争环境,和把她们不得不规定为"公共厕所"的军医的矛盾,在无奈之下的主张吧。本来是作为军医来到战地的,却让他去干检诊花柳病的工作,在这一过程中,如不疾呼自己的主张,也许也会陷入自我嫌恶的境地去的。



无论如何,战场上的日本兵,如果有麻生军医这样的想法的话,这样的慰安所也好,慰安妇也好,也就不会被需要了。他的想法扩及到了不需要慰安妇的日清战争、日俄战争去了,也许那叫做正义战争吧,因而他提出了侵略战争所带来的风纪问题。但是,毕竟还是没有冲破作为"皇军军医"的这个框框。



六 对花柳病的认识



凡欲歼灭敌人者,不可不深知敌人。在对花柳病的战斗中亦然。对敌人的兵力、毒力不可无知。不仅对军队内,对娼妇也必须有充分的认识。



雷塞尔努力改善娼妇监督工作,提议建立一项法律,把它命名为卖淫纪律,认为有了它就会使娼妇的花柳病减少。想来它不仅只是对娼妇,对于利用她们的人们也大为有利。即常常用冷静的目光看待性交的人,就会把她搁在一边,决不向她提出要求。



在这种意义上,对军用慰安所的娼妇经常进行监督指导也是必要的。



再者,对于利用她们的男子,也需要进一步提高他们对性病的认识。俗话说,"不赌博、不好色者,不是好男儿",在欧洲各国十六七世纪时不以梅毒为耻,觉得谈论自己的性病是件光荣的事情。



慢性淋病的治疗是何等之难,一旦侵入脑神经细胞,螺旋体是如何影响其生命力,这不仅是个人的问题,对家庭、子孙,进而甚至对民族素质的下降有因果关系。想到这里,我们学习医学的人,能不战栗?!



为此之故,在军队内贯彻性教育,可以说是重大的问题。最近在美国军队中,为此常常散发传单、小册子以及照片。据说特别是在电影中进行性教育取得显著效果。这一对花柳病的启蒙运动,才可以说是委以部队卫生人员的重大任务。

他说的是性病教育,与今天所说的性教育略异其趣,但至于散发传单,这是军队下不了决心的。再说其他军医对这种性病教育也可以说不那么热心。



七 狭义的预防法



就狭义的预防法在进一言之前,先打算谈谈军队中常常见到的包茎问题。即至今为止还没有人为了预防花柳病,而对包茎过长的人实行过手术。从前有人对宗教性的切除者和非切除者,进行过患病率的比较。



据布莱廷斯塔因在印尼的军队中的调查,在1500名切除了包皮的士兵中,患花柳病者占16%,其中有梅毒者0.8%,但在1800名未切除包皮的歇洲士兵中,患花柳病者占41%,其中有梅毒者达4.1%。



又据罗埃布的调查,在2468名患者中,没切除者的39.1%的患者患有下疳和梅毒,而切除者不过占15%。确实可以肯定,有包茎的阴茎容易不洁,成为病菌的良好温床。



所以那种情况如有可能,于壮丁检查之后,接着在刚入伍之前就加以治疗,以成为对花柳病预防远大计划的一点帮助。



这是麻生军医作为医师的一种主张吧。那是后来南方战线上的话了;有一位军医说:"给士兵们有包茎的人做手术,用割下来的包茎肉片当饵食钓鱼,随随便便地就可以钓上大鱼来。"一部分军医也许对这份报告书感到佩服。当然军队里做这种手术的军医,也许感到麻烦了吧,用手指头抓起包茎皮的部分,用剪刀"咔嚓"一下就给剪掉。听说用这么粗暴的方法,剪完之后涂上碘酒,说句"好了"、"辛苦了",就完事。即使如此,麻生军医在战场上搜集国内外有关医学书,进行研究所做的努力,还是了不起的。



其次谈谈有关在军队内的狭义的预防法。说起来花柳病的发生率,在平时的军队和战时的军队中,无论在质的方面和量的方面都不相同。例如在美国军队里,大战前为止只占16%,但刚刚进行战争动员,就上升为40%。何况在戎马倥偬之间呢。至于其预防法,到今天为止,都大致相同,有的用药物保护剂,有的用避孕套,有的两者并用,还有的用交接后的洗涤、消毒等办法。卑职以全部利用这些为原则。



用于这方面的具体药品的组成及其他,实在是多种多样,所以在此不能尽述,总之,在于选定可靠的物品。



从卫生材料厂领取且不说它,有时有补给民间制品的事,所以此点需要特别注意。因为最近偶尔听说用了某种药品对局部刺激太大,使用时受不了。



再如避孕套近来也几乎变为使用硫化橡胶制品,质量提高,效果良好。但有的因保存太久,变得脆弱了,需要注意。在内地听说橡胶制品统制起来了。不仅用于制造汽车轮胎,为此目的也须选用优质原料,要杜绝生产粗制滥造的产品。

话有点离题,从东京大学副教授当上某私立医科大学教授的N博士,也是作为当时的军医,从军来到了华中。听说他提议:"光有避孕套不济事。在根部须缠上手绢。"于是人们取他的姓氏,将其提议命名为"N方式",这是苦笑着谈往事的麻生军医在回顾时说的。当然啰,这种方式似乎没有严格进行的迹象。同时从文中也可以看出当时送交军队的避孕套,质量很差。在当时来说,橡胶是贵重的军需品。试图抢占橡胶和石油资源,也是日本决心打太平洋战争的原因之一。



在交接后的洗涤、消毒中,须使用消毒肥皂及其他液体。然而进行这种重要的处置,须要求有适当的场所,如在该娼妓家里或在外面。



总之,交接后消毒的时间越快,罹病率越少,这是当然的事。请参看荷兰海军所得到的结果:



从以上可以看出,交接后的预防消毒是何等必要。在欧美的军队内,在兵站所在地等处的公共厕所设置这种消毒所,也有在兵营内设置的例子,基于上述理由,卑职希望将其设在娼妓家中,而且设在交接的各个室内。在这一点上,1938年春开业的上海旧军工路附近的杨家宅陆军慰安所的设备,设置得很理想。



当时的消毒药,使用的是"变色龙"水。只放在娼妓家内的一个地方,在卑职看来是不适当的。



……



通过以上我们了解到上海杨家宅的慰安所,即娱乐所的设备很理想。同时也得知欧美的军队确实采取了针对性病的对策。当然在欧美的军队中,没有"军队直接管理的卖春妇"。他们让士兵们利用街上的女人,比方说在士兵参加了三个月的战斗之后,就让他们撤回后方休养。没有条件仿效这些欧美的富裕军队,这是贫穷的日本军队的悲哀。太平洋战争中,在瘦得只剩了皮包骨的日本军的面前,美军飞快地换上了新兵,老兵送到檀香山和悉尼的疗养地去休养,这已是众所周知的事实。



八 患者的处置



一旦得上花柳病之后,必须尽可能地对患者实施早期治疗。卑职在南京负责兵站医院门诊病人治疗时,常常发现一些患者或卫生军士在领取治疗本人花柳病的药物之余,有多要药品的倾向。究其原因,一是他们害怕让自己本队医生知道自己的疾病,二是军医中也有不喜欢自己监督的队里有人这样消耗药物的,进而让患者本人自费购入药物,才出现了上述的倾向。



花柳病治疗方案的确定,是很困难的事情。然而眼下只好在本地域内一如既往,照华中医示第四号文件所规定的那样办,另无办法。另据华中医示第五号、华中兵监医第八十五号文件,关于花柳病情况的调查今后更须加强,有必要按各个作战时期,有组织地进行统计观察。





此时成为问题的,是所谓"花柳病士兵的特典",即作为花柳病患者被收容后,在战斗期间,生命的安全得到了保证。实在令人可恨极了。他们的战友平素认真而没有被花柳病感染的人曝露在枪林弹雨之间,他们却与"六○六"和"性病灵"为友,过着安逸的医院生活。而且他们在医院内的起居坐卧决非良好。因此海希特说无论患梅毒也好,患淋病也好,只要症状消失,这些人应立即被送往前线,而且,在前线期间进行注射和完成治疗也不困难。奈塞尔之所提倡在堑壕内注射"六○六"和实行水银疗法,那也不是办不到的。



如在平时,虽只数千发炮弹。即便只有一发炮弹,也必须防止其在炮身内破裂,重炮演习时也要在扳机上拴上绳子进行。但在实际战场上绳子之类的东西根本不能用。"六○六"也并非没有千分之一的危险,然而在战场上,不管事态如何,也有不成问题的时候。军队在防止战斗力下降问题上,有无可奈何之处。



在这儿,使人不由地想起那个"出兵西伯利亚的战争教训"。"即作为花柳病患者被收容……而且他们在医院内的起居坐卧……"军医的愤怒,如在眼前。与此同时,军干部关于出兵西伯利亚时的回忆,也如在眼前。当然没有详细的数字,但由于有了慰安所,据说比起出兵西伯利亚时,得以防止出现花柳病士兵的数字超过战死者的数字那一情况。至少可以说是战争教训结下了果实吧。



九 结语



至今为止,世界各国的军队试行了形形色色的对策。陆军军医团发行的《皮肤及花柳病讲义录》中,详细记载了芬格尔等14个人的意见。他们大致上议论的是同一件事。卑职也得出和他们大致相同的结论。因此,对这一卑职认为正当的各种对策的切实贯彻,无论在何时何地也切望完成,决不落于人后。根据以上各项看法,卑职以下列各条作为结语。即:



一、在军队内进行有关花柳病的教育。必须认识花柳病为何物。



二、严格实行个人的预防方法。



三、进行局部精密的身体检查。每月一次的身体检查时需要注意。



四、酒精饮料的限制。



即作为代替它的,需要较为高尚的娱乐设施。音乐、电影、图书或运动即可。



照想用16厘米的放映机放电影,只要稍加研究,在前线附近各地放映并不那么困难。也希望设立非娼楼军用娱乐所。这样,兵员可养成不以禁欲为意的良好风气。



五、检梅必须确切严格地进行。



再则,有必要加强对娼家、楼主的监督。罹病娼妇的治疗、隔离必须实行。



为此在兵站地区内,需要建立特殊医院,希望全机关统一,使她们也有病后治疗的可能。





六、提高娼妇的质量及加以选择。



同时,也有必要提高对私娼的警惕和对传染源进行彻底的追究。



七、严守防疫军纪。



实现以上几点,实在是关系到军纪的振作、防疫军纪的高涨。作为花柳病的对策,将来兵站司令部为此或者是更进一步地提高和充实自己的医疗能力,或者是必须把工作的一部分转让给其他有能力的卫生、防疫机关。



八、对于上述各方面诸因素的整体性、统计性的研究,对于将来、对一些社会问题,将起到重要的作用。



1939年6月26日


\section{强制朝鲜妇女当慰安妇}



总之,在中日战争初期出现的慰安妇的身影,以及军队对她们的姿态和想法,可以说以上意见书全都收录进来了。



谁知这份麻生军医的意见书,使当时的军干部们明确了理想的慰安妇首选朝鲜女性,于是把他们的割草场指向了朝鲜半岛。



那么,其搜集的规模到底有多大呢?我好不容易见到了在第十一军兵站部供职的原财会军官。他现在在关西开办着一个会计师事务所,是位比麻生军医白发还多的绅士。



"首先那些军干部,是些只知道打仗的人,一开始他们对这份意见书好像怀着极大的兴趣。我们这些担任兵站工作的人,一看也觉得写得很好,做了很多调查研究。听说这份意见书也引起了柳川军司令官的注意。说句老实话,当时对她们这些人,大部分人都认为不过是"公共厕所"似的东西。但一些高级干部对这份意见书极力赞成,认为就像兵站部似的慰安所非有不可,在中国战线上慰安妇非有不可,而且得有更优秀的才行。出兵西伯利亚的教训嘛!这么说,在高级干部中,确实有过这种思想。在昭和初期日本陆军的中坚干部中,精神主义非常多,从而产生了"求粮于敌"的作战思想。也就是拿着枪支、弹药、刺刀跃入敌阵,夺取敌人的粮食。比如参谋等人就是这种思想的代表人物。这种作战思想到了太平洋战争,把军队本身陷于极其严酷的惨状。但当时的战场是广大的中国,多为农村地带,只要是进攻,粮食要多少有多少也是事实。在中国农民看来,那是一种掠夺。但边夺取粮食边作战,日本军早已成为习惯。与此异曲同工,可以认为中坚干部在心里有了这种"求女(粮)于敌"的思想,和高级干部的想法有了分歧。所以后来高级干部对麻生军医的意见书也就没有表示什么兴趣。"这个人淡淡地说,"但是,所求之"粮"如果是中国军队的东西,就没有什么问题,而如果是民间的东西,就会涉及治安问题。从战史来看,他们也明白。再说如果是短时间作战的话,光靠"求粮于敌"还能凑合,但一到长期作战,如果一再使用这种办法,他们也知道会导致民众的反抗。另一方面,他们认为只要打下当时的中国首都南京,中日战争就会结束,然而蒋介石在南京陷落前就把首都迁到了汉口,主张彻底抗战、百年抗战。这时,日本军队的中坚干部也开始明白了得准备长期战争才行。在这种情况下,关于慰安妇的问题也开始重新考虑了。他们开始考虑如果让士兵两三年都留在战场上,慰安妇也是必需品。于是他们决定从日本招来第一批慰安妇,这是1937年底刚刚打下南京之后,发动徐州战役之前,这也是开始考虑战争长期化的时候。对高级干部的这一想法,作为军队实力人物的中坚干部总算是赞成了。我觉得这就是原因。"



据说,设置慰安妇,募集慰安妇,也就是意味着陆军放弃短期作战的想法,转变为从长期战到持久战的作战方针。这一点是重要的。他又告诉我说:"作为军部,开始考虑从内地大量动员慰安妇了。事实上我也记得,从1938年早春到初夏,仅仅华中就来了1000名慰安妇。"



据说一开始对麻生军医的意见书之类丝毫不感兴趣的华北派遣军,也逐渐开始考虑整顿体制了。该华北地区有多少慰安妇,虽然没有数字,但在整个中国战线的慰安妇,与日、与月、与年俱增却是事实。虽属推定,但1938年10月28日攻下武昌、汉口时,在整个中国战线上,集中了3到4万名慰安妇。



另一方面,如此众多的慰安妇,仅从日本内地募集,近乎不可能。如果从日本国内募集,从常识来看,就是从有卖淫经验的人中来募集,而当时的日本国内的妓女人数,公娼、私娼合起来认定为25万到30万人。从这里边抽出3到4万人也太勉强了。话虽如此,从一般妇女中募集问题极大,保不住会成为社会问题。再说日本内地的妓女的多数,像麻生报告书中写的那样,都是些疲惫不堪的患过性病的人,作为慰安妇是不合适的。



前面的说明太长了,在第一批"理想的慰安妇"中压倒多数是朝鲜女性这一结论得出后,军干部们的眼睛便必然地转向了她们。



因为计划着打一场长期战争,为了控制全中国,日本军方算出至少需要50至70万兵力,需要3至5万名慰安妇,而部队对慰安妇的需求程度也逐渐强烈起来。当时的朝鲜半岛是殖民地,因为是殖民地,所以多少用些强制的手段来募集,有问题也压制得住。不可否认,不仅在当时的军干部中,就是在日本人当中,也存在着把她们当作"公共厕所"的想法,她们也无力抵抗这类歧视。



这一朝鲜慰安妇问题,在后面将予以详述。以上就是日本陆军从开始设置慰安妇到大量配置的过程。


\chapter{“冲锋第一号”}

当然,这“和平”只限于士兵数和慰安妇数保持平衡的时候。前面提到的在孙吴军队和慰安妇的比例是2万人对50人,在这种情况下就发生对慰安妇的争夺战。据说挥动刺刀伤人的事情时而发生。由于这样,听说也出现了演习时悄悄溜掉,让当地人当向导,犯禁找当地妓女花钱求欢的。

\section{随军慰安妇,在陆军组织中固定下来}

前面也曾写过,从1938年暮春起,慰安所在中国的各军驻扎地扩展开来。比方在华中战线,由上海开始,向杭州、常州、南京展开。这是因为被动员到中国去的士兵都是年纪较大的召集兵,也就是结束了两年的征兵训练,一度回到了市民社会,过上了婚后生活时召集来的。这样一些士兵在战场上的性问题频频发作,这个问题前面已经谈过了。



可悲的是,对于一个结了婚或者是和其他异性交游而知道了女子身体的男性来说,对女人身体的要求是遏止不住的。何况在战场这种异常情况下,更有强烈的反应了。



当时的征兵年龄是21岁,也许当时这种年龄的男子纯情一些吧,一多半是没有性体验的。而对这种没有性体验,且被课以现役那样的繁重紧张训练的男性来说,慰安妇未必是必需品。两者恰成一种对照。



在这儿有个事例可以说明当时召集兵们的性欲要求是何等强烈。



这是1938年春的事情。有个出入于小仓步兵第一一四团的名叫岛田俊夫的御用商人,在北九州募集了二十多名慰安妇渡海到了上海。到达后,他奉命在杭州营业。插曲就是从这儿开始的。岛田现在在小仓火车站右手开了个酒馆。他苦笑着说:



"我不是奉部队之命去募集慰安妇的。正想为士兵们做点什么事情时,副官部的人对我说,你办个慰安所如何?"



到这儿为止,与前面记载的田口所说的情况不同。这且不管它。还是听岛田说下去:



"于是我回到了与其说是部队士兵的出生地,不如说是我的出生故乡北九州,募集了二十多名女子。准确的数字不记得了。带着她们在l938年4月到了上海。那时节就是顺便搭载军队运输船也不那么困难,看情面顺利地坐进去了。我本想在上海营业,到了上海之后让我到杭州去。说是杭州还没有慰安所,士兵们正在"饥饿"。"



他也不是没有发现自己谈话的语气太轻松了。我所见到的了解慰安妇情况的人,没有谁能像上岛田说话那么有节奏。"可是使我吃惊的是从上海往杭州出发的半路上发生的事。上海和杭州之间有180公里,在一般情况下有五到六个小时就可以到的吧。但是周围中国兵的游击队多得很。我们一行人乘坐的军用列车,每到一个站就停车,慢腾腾地一边警戒着周围,一边走。而且一到晚上就停下来,在车站过夜。那是离开上海后的第三还是第四站,车照样"咕咚"地一声停了下来。这时,车站上正在警备的士兵走了过来。当时我们坐的是一节没有窗子的闷罐车,他"哗啦"一下把门打开了。我们吃了一惊,那个兵也吃了一惊。不管怎样,车厢里有女人。于是他问道:



"这些女的是什么人?"



"是去杭州营业的陆军慰安妇。"



"慰安妇?"



"专门慰问士兵们的女人。是从日本来的。"



"士兵专用吗?既然这样,何必去杭州营业呢?在这儿营业嘛。多少钱?"



"军士和士兵两元。但一次30分钟,这是规矩。"



"一次30分钟吗?好啦,我把钱放在这里。"



"做了这样的谈话之后,官兵们就强行让我们就地营业。每个车站的警备队一个班,大约十二、三个人,多的有20个人的。就这样在路上走走停停,花了两天半的时间才到达杭州,敢情在路上慰安妇就把借支全赚出来了。这样一来,慰安妇和军队之间就可以马上解除雇佣关系,获得自由了。"







"慰安妇的预支款在1938年是1000元。在从上海去杭州的路途中每人就挣了1000多元。挣了1000元就可以自由了,到哪儿去都行。总之一个人两元,1000元嘛,就是500人次啰。在货车里,用一张草席隔开,作临时慰安所,3分钟一次。有的连续6次,就是说,有18分钟做了6次花掉12元的猛人。一般的都在3到4次。慰安妇连睡觉的时间也没有,有的身上还趴着士兵就打起瞌睡来。都尽是一些召集兵,他们饥渴得红了眼。是召集兵还是现役兵,立刻就见分晓。因为召集兵动作大。"



不管怎样,这就是慰安所这种场所儿从1938年春天起,在日本陆军组织中固定下来的过程。它们成了军队的机构之一。然而作为军队机构的慰安所,不久就换上了民营的招牌。理由是军队既然号称为"皇军",带着"慰安妇"上战场会遭到国民的抵制,这事在前面也曾稍稍提到过。在军队干部中,怕有损军队名誉的意见相当多。战败当时担任陆军大臣的阿南惟几上将也是其中之一,他作为第二方面军司令官去苏拉威西赴任时,轻易不允许在所属部队中设置慰安所。在职业军人中,这种类型的人似乎很多。尽管如此,只要不设置这种机构,部队就出事。在这种情况下,大约于1938年年中,慰安所完全改成了民营。



但是,虽说是民营,其实质内容与过去却没有一点变化。



一位在玉兵团(第一师)作为军医从军,现在在川崎市的某公立医院当院长的人(希望匿名),就慰安所管理状况,做了如下的说明:



"军的战略单位是师,师的中枢是师司令部,司令部下设参谋部、经理部、军医部、兽医部、兵器部、管理部。参谋部担任作战计划、作战补给计划的制订、执行和收集情报。经理部和一般公司的经理部一样,担任物品的采购和分配。军医部分为卫生队和后方医院。卫生队的军医主要担任野战中的医务,而后方医院的军医担任部队的卫生管理和防疫工作。兽医部、兵器部的职责如其名字所示。管理部类似管理师司令部机关的总务科。



"且说,我们在战争初期驻扎在北满的孙吴。这个孙吴本来是个无名的农村。是1939年诺门汉事件之后,作为对苏作战的兵团基地,由日本人建立起来的镇子。日本人的艺妓也在那儿营业。当然了,虽说叫艺妓,却是兼做妓女和卖艺的生意,劝酒、表演歌舞然后陪你睡觉。但是这种营业,不是以军人和军队为对象,而是以一般的日本人为对象的。军队的兵营设在离镇子5公里远的野地里。



"军队的慰安所在兵营的附近,专供军队使用。四面围着砖墙,里面有日本人和朝鲜人慰安妇,大约有50名。这儿由民间人士管理。就营业问题,军队不介入。






"但在卫生方面,由军队后方医院的军医负责。定期进行检诊,一发现花柳病患者,就由各团的值班司令通知各部队,命令该慰安妇停止营业。就是说由军队掌握着管理权,是一种军队间接管理方式。作为军人,反正是最害怕性病的发生。军医部在各个慰安所的房间里放置高锰酸钾水溶液,命令在完事之后,士兵们必须给自己的性器官消毒。



"这种水溶液被称为"变色龙"水。所谓高锰酸钾,是紫色的、闪闪发光的、有点像鱼鳞似的结晶的药品。这种"变色龙"水的洗涤装置全军都是统一的。多半是根据全军的命令。此外,官兵外出时也一定要带上写着"冲锋第一号"的袋装避孕套,这也是全军统一的。"冲锋第一号"是军需品,用配给的生橡胶优先制造的。



"不管怎样,用它可以大体上防止性病。当然了,没有血液检查的设备。所以检诊从医学上来说,还不完善。慰安妇中也有的得了淋病。经调查,是由外部传染的。她们在检诊就要开始之前,把局部的脓洗掉,用从街上药店买来磺化剂来抑制化脓。但逃不过由很多内、外科专家组成的军医们的眼睛。



"另一方面,就带来的慰安妇,军医部有她们贴了头像的名簿,不如说是"名册簿"。在宪兵队里也有抄本。我原是属于卫生队的,后方医院的军医跟我说:"一看这些照片,就能想起她们身体的一切部位来。"后方医院如有妇产科医生,就让他们担任这项检诊工作。当然了,妇产科医生在召集的军医中为数很少,因此检诊的工作大体上由内、外科医生担任。在就要检诊之前洗涤啦或者用磺化剂啦,这种成药名叫做台拉坡尔,一喝下这个就看不出来了。"



总之,由此可知慰安所注意了防治性病这一事实。同时,也可以得知慰安所当时虽说委托民间经营,但实行了与军队直接管理同样的方式。这个玉兵团第一师,虽然驻扎在北满,却处于战时体制之下,是一支1944年直接转用于莱特攻防作战的典型的战斗师。多半对其他部队也采用了同样的运营方法吧。



那么,从军队方面来说,是怎样利用这些慰安妇的呢?和我谈起这个的,是一位原《读卖新闻》的社会部记者,写了《血淋淋的事件报道》的三田和夫。他原来是陆军少尉,现在在亲自主持《正论新闻》。



"我在中国代号为"弘"的一九六七部队,正式番号是第一百一十七师独立混成第八十七旅独立第二百○五营第五连当过排长。在战争结束前夕开到满洲,而在那以前驻扎在黄河南岸霸王城附近的考城县。任务是警备京汉线的铁桥。把高射炮摆在桥边防备着美军的空袭。那是典型的中国驻屯军的姿态。



"霸王城有座很大的慰安所,在考城县有它的办事处。经过霸王城的部队多,是个交通要道,所以才为他们开办了那么大的慰安所。考城县的慰安所,供当地驻屯部队用。因为慰安妇人数相对少,决定各队分别定下日子加以利用。当然,警备驻屯,特别像我们那样,在美军飞机来袭的时间加以警备,过了那个时间,就闲暇无事了。老兵们会找窍门出去。队长也装作没有看见。在长期驻屯生活里,还是不那么絮絮叨叨地碎嘴才能搞得好关系。







"再说在长期驻屯生活中,和同一个慰安妇一起过日子,觉得就像自己的老婆一样。士兵们也不再那么贪馋了。他们感到随时可以解决问题。她们因此也好像成了驻屯部队的一员。还可以比作装饰品吧,没有慰安妇的驻屯部队,就像没有点心的孩子似的不像样子,因此士兵们都很珍视她们。



"慰安妇方面也报答他们,又是休息的日子给士兵带来礼品,又是帮助士兵们洗衣服,或者坐在阵地旁边以手托腮眺望士兵们保养机枪,或者摘花,天空中鸟儿鸣啭,悠闲得很。士兵们也让她们一道吃午饭。驻屯地的士兵和慰安妇的关系,使人觉得到处都是这样。当然,洗衣服对于士兵来说,当新兵的时候已经出了徒,比接待客人为行业的女人洗得还要好一些。只是她们的心意让人喜欢。当然,朝鲜人慰安妇从小就受过训练吧,记得她们洗得干净极了。



"说起乐趣来,有点儿那个,但检诊也是士兵的乐趣。我们高射机枪队在高地上,配有高倍率的望远镜。用它能在阵地上看清眼下面驻屯地军医检诊的情形。张开胯股的样子,就像在用手能够触到的地方。检诊于每周星期五的下午进行。"啊,在检诊呢。今天谁会被发现问题呢?"士兵们吵吵嚷嚷,但到了晚上就会送来检诊报告。具体地写着"花子,不合格。理由,梅毒第二期"。当然,没有发现什么的日子居多。所谓检诊报告,就是把军医检诊结果正式通知士兵们。看这个,也可以理解军队和慰安妇的关系。即使如此,连她们的检诊,对于悲叹无聊的士兵来说,也成了除忧解闷的材料了。



从这儿了解到,士兵们扎下根来担任警备任务时,好像就不光是把她们看作处理性欲的对象了。因此,也有一些好笑的事发生。






\section{“稍息,继续工作!”}



"当军官的,有值周勤务。他们的任务,主要是监督所属部队的军风纪。考城县驻屯部队,大约三个月能轮到一次。这时便在肩上斜挎红白条纹的值周带子到处巡视,当然也得到慰安所去巡视才行……



"那是任务,按规定也得到慰安所里边去。考城县的慰安所就在中国的民宅,用席子隔开,进门的土地间就是候班室。候班室和房间之间也是用一张席勉强遮住。进去就是进候班室,有趣的事还在后头。帝国陆军礼节第五十九条写着:"军官来到士兵的室内时,最先看到的喊"敬礼",在室内的人听到口令后必须全体当场敬礼,经军官许可后,再继续从事作业。"根据这一条,不知是谁最先看到值周军官三田和夫少尉的人大声喊叫:"敬礼!"



"响应这声音,在慰安所的人都当场起立敬礼。在候班室等着的人穿着衣服倒还好,但在草席那边的士兵裸着体怎么"立正"呢?让他们裸体中止下来起立敬礼太可怜了,据说习惯上只是让他们原地"立正"就不必起立敬礼了。总之,得中止运动,"立正"。



"受礼者的值周军官便答礼说:



"稍息!继续工作!"



"于是,运动又开始了。如果军官不发话,就得那么站一二十分钟。把性行为理解为工作,倒也有趣,但如果刚脱裤子,值周军官就出现了,就成了悲剧。如果敬礼,军裤就会掉下来。话虽如此,如果不敬礼,那就成了违反第五十九条了。



"也有的差一点哭出声来,但那是年轻的新兵。成了老兵的时候,奇怪的是能感觉到值周军官什么时候到来,会巧妙地错开时间。"



这个故事是三田和夫讲的。"慰安妇们看着这些觉得很好笑。有的时候慰安妇故意吓着士兵们玩,搞恶作剧,在行为中军官没来也喊"立正!"但是,对答礼"稍息,继续工作!"喊得慢的军官,在士兵中人缘儿不好。在长期的战阵生活中,他们都把这种恶作剧当成了很大的乐趣。前面是士兵把慰安妇当玩物,后面是慰安妇把士兵当玩物了。"



只是在1944年以后的中国战线上,因避孕套不足,在考城县,一到早晨,卫生兵就回收慰安所使用过的避孕套,用消毒液洗过之后再用。"洗过后在日阴中使之干燥,撒上面粉,使用起来很滑,士兵反映很好。"士兵和她们均很满意。



当然,这"和平"只限于士兵数量和慰安妇数量保持平衡的时候。前面提到的在孙吴军队和慰安妇的比例是2万人对50人,在这种情况下就发生对慰安妇的争夺战。据说挥动刺刀伤人的事情时而发生。由于这样,听说也出现了在演习时悄悄溜掉,让当地人当向导,犯禁找当地妓女花钱求欢的。



在中国像三田所在的排那样在边境警备驻屯的倒还好,但在城市里驻屯的部队,从1940年以后,除了使用军队管理的慰安妇以外,也有的官兵去寻求当地妓女,这使上边感到很不好办,因为由此感染性病的士兵不断地出现。这样一来,建立慰安所就失去了意义。这也是出兵西伯利亚时代悲剧的再现。





\section{机密文件中所记录下来的实况}

这儿有一份资料。是一份以“武昌地区防疫卫生委员会”的名义向驻屯武昌地区各部队发出的“机密文件”。该文件很好地表现出为当地妓女问题所困扰的当地军方干部的苦恼,因此全文照录。一开头是前言:

近来武昌地区中国方面(汪伪--译者注)私娼甚是猖獗,中国政府对此尚未研讨实施对策。军人及文职人员被此等私娼感染入院者为数不少。现状令人不胜寒心。尽管眼下省政府正在考虑应对措施,但至实施还需相当时间,因此作为军方,有必要采取手段不准官兵接近私娼。另册系安村少校调查概要,聊作中国方面对私娼取缔之参考予以散发。

武昌地区防疫卫生委员会

这份文件中提到的被“私娼感染入院者为数不少”的数字,因资料不足,具体究竟有多少人不得而知。既然以文件的形式向各部队秘密散发,因此只能推定为病情属于恶性,至少是三位数。

另一方面,文中所说的另册,其笔者是小原兵站司令部军医少校安村光享。比起麻生彻男军医少尉的意见书来,此另册略有风流韵味,题名为《武昌之柳——私娼之由来及现状》。

下面就介绍正文:

这本小册子的出发点系出于忠心,愿为武昌花柳病预防及对策助一臂之力。感谢湖北省政府警务厅托田上明先生的协助,为我梗概地叙述武昌市中国私娼的散布及交易情况。因私娼出没极其巧妙,尽管实际数目不详,但只要根据四周状况加以观察,盖不难想象,因此而受害者也有相当大的数字。由此看来,相信采取措施加以取缔,已成当务之急。

至于如何加以取缔,当然省政府已预先知道了在公共卫生方面无知的民国方面私娼的存在,各所属部门的长官,预防此种受害之事,也是目前急务。

中国自古以多柳之国而闻名,这从古代诗歌中歌咏柳树一事即可明白。

每当春风骀荡时,望见被风摇曳的湖岸边的柳树时,便使人联想起古代中国的达官贵人们,与身着姹紫嫣红绚丽服装的窈窕美人同船游兴之事。

是的,中国是多柳之国,由此柳而染病扩展到全国国民,加之此次之中国事变中据说几乎处女极少,发展到如此程度,因而病的性质之恶,出于想象之外。

可以说中国花柳病之所以如此之多,最大原因是个人卫生与医学的不进步,以及公娼制度之没有确立。

然而自古以来,武昌就是湖北省政府的所在地,各种官衙、学校极多,俗称文教之都。由于这种关系,因害怕有伤风化的坏影响,不准像其他地方那样允许公娼,尽管作了相当于取缔的样子,据说仍有蒙过官宪的眼睛,利用旅馆饭店、普通民家进行私娼行为者,约有三百余人。

事变前有旅舍数十家,其中大的有五到六家,此外还有许多饮食店,其主要集中于斗级营街及大朝街,旅客往来频繁,极其殷盛,其取缔不甚严,因此私娼的出没也相当多。

文中“此次之中国事变中,据说几乎处女极少”这句话,有点使人感到不打自招。是谁把处女弄少的呢?只要略加思考,不,不用思考就看得出,这不正是向天吐痰,反而掉到自己的脸上吗?还有,渗透进来“花柳病”即性病的究竟是谁呢,也是个问题。总之,正因为安村军医少校在兵站司令部工作,也就是在后方医院做事,让往街区居住的中国人私娼那里跑的士兵给弄得对付不了啦。文章还在继续:

武昌于1926年9月被蒋介石攻打下来,自那以后由中央政治委员会系统的人当了省长进行统治,随着新生活运动的强化,私娼的取缔更加严厉了。1927年8月12日在《违警惩处法》第四十三条中,制定了如下法令,即实行秘密卖淫者、其嫖客、拉皮条者及止宿者,处以85元以下罚金。当时的省长就是陈诚。



为此,私娼出入于旅舍,几乎看不到影子了,但据说他们避开官宪的眼睛,煞费苦心地来到街头在各自的住宅接客。



干这些给私娼拉皮条的行为,俗语叫做"台基",又称"堤调行",秘密地应客人的请求,给私娼撮合。事变前武昌与汉口之间有25个轮渡码头,每一个小时就开出一班船,交通至为便利。所以想干这种勾当者,当时就渡到汉口,因此据说自那以后武昌市内就见不到私娼公然拉客了。



武昌于1938年被日本军攻占,同年11月建立了治安维持会,翌年1939年4月,接受了特别市政府的统治,更于同年11月成立了湖北省政府。自那以后,居民的迁回也显著增加,以至于今。



进行花柳交易的地点,自去年春天起到处开始出现,现在多集中于粮道街,其他出现于12家旅馆,上黑名单的人32名,流动的及由台基达成交易的私娼加在一起,可以说恐怕呈现以其数倍来计算的盛况。



此类私娼的年龄,大多在十四五岁到三十岁上下。其身份如出一辙,大部分为未婚者,其中有相当多的人离了婚孀居。也有丈夫出外谋生或有丈夫而生计不如意者。偶尔为了享乐而奉献者,盖亦不少。



从这儿可以了解到,士兵们去的是挂着"旅馆"招牌的地方。多半儿像现在日本的有温泉标记的旅馆,或者是为汽车旅行者准备的那种地方吧。



目前的省政府,对这种花柳的取缔毫不关心,因为这个呢,还是因为制度不健全呢,至使花柳交易呈不断猖獗之势。据说目前武昌花柳交易正在用下述三种方法进行:



一、由关照娼妇的台基或堤调行撮合,在民家进行交易。用此种方法被撮合者,多为有夫或有家庭者。她们瞒着丈夫进行交易,系短时间,多为一至两个小时。据说其费用为六至十元。



二、有在饭店或旅馆行业的娼妇,经起台基作用的"侍者"的中介,与客人面对面地进行交易,其费用为一个小时两元,过夜20元以上。这类娼妇是无配偶的妇女,即所谓专司花柳业的妇女,因卫生设施差及未检梅,说她们是严重的病毒传播中介者也无不可。(也有一说,警察及车夫的妻子也混迹其中)



三、为维持生计,妇女在自己家里公然进行交易,表面上一副普通人的样子,因此难知其详。属于这一类者多为女仆、办事员等人,其价格也纷纷不一。





在这儿使人感到,仅管日本方面提出让被占领的中国方面取缔私娼,事情总不见有着落。尽管如此,却找不到士兵们为什么仍然支付超出"士兵两元"的慰安妇价格的三倍到五倍的钱去逛私娼的原因。这使人联想到刚刚战败时日本的私娼中也混杂着一般的家庭主妇和女办事员。异国的军队,以违反道义的理由占领他国所导致的军队本身的风纪败坏,虽然也是不幸,但在这儿也可看出被他们冲垮、吞没下去的被占领国的悲哀。"多为……办事员"这一词句刺人眼目。文章还在继续:



自古以来中国的女性,在处女时代贞操观念虽强,可是一旦脱离处女之域,则转而在处世上也好,生活上也好,都有享乐化的倾向。据闻湖南、湖北两省,特别是洞庭湖畔,自古以来连续经过数次革命战乱,人心极度颓废,在中国18个省中也是贞操观念最薄弱的地方。听说茶叶产地江西的修水特甚。加之因事变的长期性,在秩序及制度的松弛和生活艰难的情况下,黑市交易女子现象的滋生盖亦当然。



现在武昌旅馆住在店内的佣工妇女,同旅馆主人之间没有借支及其他金钱上的借贷关系,享有人身自由,她们为生活所迫而出现于人肉市场上,也易于流动。



她们在旅馆的待遇及利益的分配上,大致情况如下:



一、玩乐费旅馆主与娼妇各半。



二、娼妇的伙食费自理,据说一天平均需七角钱。



三、床上用具及其他室内用品,由旅馆免费提供。



四、娼妇不需要给台基和跑腿人谢礼及手续费,由旅馆主直接适当付给。



五、娼妇与旅馆主之间无其他金钱上的关系。



六、没有来客时,除伙食费自理之外,不需要其他费用。



她们营业的旅馆多在繁华街道上,其主要顾客似乎是日本人,特别以军人、文职人员为主。中国人在玩乐方面的费用与其他消费相比,居于次要地位,不必说系按其他方法行事的。



偏僻的旅馆及饭店,则与繁华地之旅馆、饭店在若干方面异其情趣,除以上行为外,尚用于打麻将及其赌博、洽谈开会之用。对此等猖獗的私娼之现状,民国人一般的观念是接受的,与为了生活不得已而为之的说法一致。



据传闻,有关花柳交易,有如下之迷信需要注意。自古以来中国的妓女相信,一旦得上花柳病时,必须使对手感染,自己才能治愈。即使戴避孕套,妓女也会用指甲及其他方法弄破橡皮套,致使对方感染。



以上是预防对策的梗概。尽管个人还有种种对策,此处所寻求的乃是社会的预防对策,即预防上之必须推行法令的制定甚至取缔,是当前的急务。武昌中国方面私娼的现状如上,都是自由散娼,未检梅者,所以是严重的病毒携带者,说感染率是"百分之百"也不过分。下面将花柳病预防对策之有关项目列出,作为来日的研究课题,求教于识者。






第一条,娼妇的年龄限制和保护人。



第二条,确立公娼,驱逐私娼。



第三条,妓楼的卫生设施及娼妇的卫生教育。



第四条,检梅与治疗。



第五条,有关卖娼行为营业法令的制定并监督。



第六条,失业救济与社会政策。



第七条,关于卖娼行为的公共卫生及警察管理。



主要有以上几条,照日下省政府的现状,全部急速实施虽有困难,但吾人大声疾呼要求实施第二、三、四条,盖以上为民族之保护及国际传染病预防之不可或缺,所以敢于促使当局的猛醒。



提到对私娼的检诊、检梅,只有战败后在日本的由MP(美国宪兵)实施的街娼"围捕"方式。但是对于爱面子的国家,那样干就成了寒碜中国方面了。结局只能由日本军队在内部整顿军规军纪。同时只能派宪兵和风纪卫兵到军人出入的中国人旅馆(私娼窑)监视。在这儿把有问题的"旅馆"名单附上:



武昌市内旅馆业一览表(下略)



旅馆及私娼姓名年龄一览表(下略)



我把这张表给几个曾在武昌地区驻屯过的人看,他们苦笑着指着这表上的几家说,"我还背着宪兵的眼睛,常常出入过呢。"还说:"有时候巡视的军官带着几名武装士兵,突然就闯进来了。让他们抓住可不得了。那时大家没有充分注意,得了恶性性病也是事实,光靠从卫生兵手里悄悄要来的那点台拉波尔剂是难以处理的。在兵站医院里不住上两个月的院,是治不好的。军队在医院里挨个地寻问,彻底地追查感染源。痊愈出院后的人就被送到了最危险地区的部队去了。"一览表上不但有旅馆名,还记载着所属的私娼姓名。这也是一张被军方盯上的武昌地区私娼名单。



当被发现"武汉旅馆"里有十五六岁的少女时,"武汉旅馆"主人秦晋荣搓着手,对日本军士兵们赔着笑脸说:"我们这里弄来的全是些年轻水灵的。"尽管武昌地区的军干部经常为她们所烦恼,但这类情况决不单是武昌才有,似乎是城市驻屯部队的共同烦恼。






\section{提高慰安妇的战斗热情}

另一方面,在军队管理下的慰安妇这边,随着时间的拖长,慰安妇们也产生了倦怠。慰安妇的倦怠,与军队的战斗热情的丧失是一码事。失掉了战斗热情的军队,被认为还不如没有的好。只要军队存在,就得把他们计算在内来制定作战计划,这一来就等于在制定导致整个战线崩溃的计划。慰安妇也是如此,她们的倦怠,导致士兵对她们失去了兴趣。对她们失去兴趣,就会像武昌那样出现往当地妓女那儿跑的局面。

于是管理慰安妇的各兵团军医部(也有由管理部管的,在经过部队的基地由兵站部管),制定了鼓舞她们“士气”的计划。第一师就是其中之一,听说他们定期召开运动会。当然是单独给慰安妇开。这是因为如果有士兵参加,怕发生意外。前面提到的川崎某公立医院院长顼军医谈到运动会的情况时说:

“她们的学历,大部分是初小毕业,其中高小毕业的有两个人,这也是属于‘知识分子’了。所以一提到运动会,她们中的大部分只有念初小时候的经验。在那以后尽是在社会最底层挣扎了,没有任何一点乐趣。所以后方医院的军医刚一提到‘开个运动会吧’,她们立刻就兴奋起来。特别是朝鲜人慰安妇眼里闪闪发光,有的甚至流出了眼泪。她们多半是忽然想起了欢乐的少女时代的事了吧。”

“那天像女子学校开运动会似的。她们发出嘻嘻哈哈的娇笑声,高兴得在草地上直打滚儿。她们最爱看的是百米赛跑。朝鲜人慰安妇们年纪轻,有体力,因此一等奖都成了她们的。她们挺着胸脯前去领奖,激动得热泪盈眶。”

更为有趣的是运动会的头一天和开过的次日,来到慰安所的士兵们流露出:“恰如看到了另外一个人。她们的身体那么有弹力,真是惊人。”直至运动会开完一个月以后,还残留着余波或者说是影响。士兵们也很高兴。运动会无论是对她们,还是对士兵们都有好处。军医部也试着搞了文艺演出会,但听说效果没有运动会那么好。总之,开罢运动会的一个星期以内,一边让士兵们搂着,一边述说自己是如何顽强地跑来着,吃面包赛跑时,面包是多么香啊。

在腊包尔,军医和司令部人员不会带她们外出游玩,据说她们有时让报道班的人员和派遣来的新闻记者们带着她们去郊游。

当然啰,这样一些做法,只有在战场上带着若干人性的军官所在的部队和驻屯地存在,确实是为数不多。开展这些活动的,不是大部分部队和驻屯地。在一般情况下慰安妇们没有任何乐趣,是被当作“公共厕所”看待的。在这样的一些慰安所里,听说女人们只有一整天脱下三角裤衩往那里一丢,无表情地数着数,口里机械地说着“好了,下一个”,“好了,下一个!”士兵们也极其粗野。写到这儿,使人想起了一妇多夫的夫妇来。军队这种东西,嘴里说着“连队就是家庭”,但实际上始终还是战斗集体。战斗是第一位的,参谋也好,军医也好,经理部也好,所考虑的只是这个。加之,没把性欲的问题当作新的问题,干脆不穿三角裤衩的慰安妇多,也是无可奈何的事。

在三河岛的私人铁路站附近开酒店的斋藤雾,现年58岁,她是在中国中部辗转于九江、安庆、汉口等地慰安所的原慰安妇。听说她原先在四国的松山当过私娼,1939年被劝诱当了四国兵团的慰安妇。有一天,她就战场上的慰安妇的心理,跟我说:“当上慰安妇的时候,刚刚到达战场时,心想‘像我这样身体的人,还能为国家效劳’啊。可是,在第一线的慰安所时倒还好。然而到了后方的兵站基地慰安所之后,也可能是渐渐习惯了吧,有些感到疲劳起来。这是因为在第一线,和士兵们一块儿吃饭,觉得士兵们也许明天就会死。我们也出于真心来安慰他们。军官们也是如此,一见面就说:‘你们辛苦了!’谁知到了后方,真是被当作‘公共厕所’一样啊。军官和士兵们当中,有的甚至当面也那样说。这是在安庆时候的事,我被一位路过部队的军官叫了去,只在行事的时候让上床,完了之后他倨傲地说:‘睡在地板上吧!’‘作为人来说,太没有礼貌了!’我跟他这样说,挨了一顿踢。越是年轻人越纯情,完了之后大声说完‘你辛苦了’之后才回去。

“让人高兴的是,到底还是在那种情况下见到了四国的士兵。越是离故乡爱媛或松山近越高兴。士兵方面也恰如见到了骨肉亲人,不去性交,反倒一起谈起了故乡的节日、山川。士兵们也以此为满足了。‘运动会’我们没有举行过。”

“中日战争初期慰安所虽不是属于文职人员,却让文职人员样子的男子当经营人。然而从1939年前后起还清了借支的慰安妇,成了自由之身,开始当上了慰安所的老鸨子。干这一行,男人到底有关照不到之处,如接待士兵啦,女人之间发生了纠葛的时候,如果没有善于应付男人的女的,是不行的。后来我也当上了老鸨子。可是在军队中品质不好的管理军官也多,他们来到慰安所之后死气白赖地白喝酒,让拿出好饭菜招待也不出钱,还想抱女人。我们真是敢怒而不敢言,如不照办,就会说‘哎呀,士兵让这儿的女人给传染上了疾病’,‘卫生管理不好’,让停止营业,不由你不照今天的话说来‘行贿’才了事。被白白地玩了一顿,女人们还是面现嫌恶之色。”

“一开始的时候,一提到女人,只有慰安妇。但不久,以为有利可图,也有的中国人找了些中国女人来开业的。于是一开始慰安所也有着强烈的竞争意识想搞好服务,但不知不觉随着前线的不断推进,开始我们还以为是身处第一线呢,慢慢地竟然成了后方,很多人闹开了情绪,反正是被当成了‘公共厕所’。”

\section{起劲——心醉}

她还谈到了这样的事:

“照现在的话来说,叫做心醉吧。我们叫做起劲,那是慰安妇所难以体会的。在开始做生意时,老鸨子对她们说:“对每一个人都那么认真,你的身体就保不住了,做戏吧。”虽然如此,一开始到慰安妇这儿来的士兵都已经是憋得劲儿足足的人了,刚一进门就爆发出来。在这种情况下,也用不着做戏就完了。在这种情况下,少的时候一天十来个人,多的时候一天几十个,因此三个月之后,就得了不感症。我也变得那样了。当然了,一天如以四五十个人为对手时,就仰卧在床上叉开两腿,随便让士兵们一个接一个地来好啦,那也是当然的。”

“可是,尽管是那样的身体,大约三个月左右一次,在梦中也有起劲的时候。一问,其他女人也好像是一样的。即使如此,大约三个月左右有一次,在士兵的怀抱里猛烈地燃烧,细想想,那个士兵不是和自己昔日初恋时的男子相似,就是体臭一样。在这件事上其他女人也是一样的。这说明慰安妇的身上也残留着女人的部分吧。月经的时候,和在内地的妓院时一样,猛喝食盐水,过上半年就发生月经不调,接着就进入停经阶段。一停经,意外地鼻子下边就会生汗毛,士兵们就说:‘你是男的吗?’这样一来不管怎样,就变成了不会生孩子的人了。这时,作为女人的希望就消失了,成为了一名顶用的慰安妇。这就是无法反悔的悲惨人生的开始。”

这确实是悲惨的故事。她同我说,为什么活着回到内地的慰安妇们互不交往。这也是可以理解的。

“因为那过去如果是能用橡皮擦掉的东西的话,真想擦掉。”那时她说这番话的脸上沉静而阴暗起来,“和曾经当过对手的士兵也不想见面,想静静地在社会的一个角落里悄悄地度过残生。总之,是件令人厌烦的回忆。”

她和我说,战后自己的这些往事,跟谁都没谈过。还说今后仍会保密。据说她从给黑市的烤全墨鱼店当帮工开始,后来又当酒店的女佣,省吃俭用地攒钱,十几年之后才开了现在这个店子。她是我经过一年半的时间,好不容易才打听到的。“很失礼了,请问你现在还继续患不感症吗?”我大胆地问道。

“您问得真怪。”她笑着说,“仍在继续。心想为了国家,也许像残废军人那样,由政府给发养老金呢。半真半假地这样想。我虽然没作调查,是不是寿命短一些呢?”

她说罢又笑了。

慰安所,一开始的时候是在兵站部简单地盖了一些房子。随着战线的扩大,后来就利用了中国人的民房。到了太平洋战争,战线南移时日本国内设计了“简易慰安所”,器材作为军需品装进了运输船。

第一批慰安所开设在腊包尔。1942年,军部一决定在腊包尔建立慰安所,建慰安所的器材使用进入特拉克岛的运输船运来了。那些预制件器材迅速就装配好了。把3尺和6尺的柱子,固定在混凝土的骨架上,屋顶用镀锌白铁板一盖,两天就盖完了。当然是简易建筑,是长屋式的平房,给人以间壁成若干间房屋的感觉。一栋长屋有10个房间。一个房间3铺席左右大。一个慰安妇占一间接待士兵。地点在科科坡的中国人街和日本人基地之间,共有3栋。当然,这种设施,是在补给有条件的情况下才会运来,到了太平洋战争中期,即便是好容易制成的设施,也失去了运往当地的手段。

这种设施,也没运到中国去过。在中国,自1940年以后,就没有增加兵员了,多半是他们判断以现状就能够应付吧。

“在中国大陆,无论走到哪里,家家都有臭虫。可是臭虫这玩意儿慢慢的对它有了耐性,不久之后,就是挨了咬也不肿了。”斋藤雾说。

据说这种慰安所房间的分配,通常是由老资格的慰安妇占好的。经调查,最大年纪的42岁(战败当时),她1938年年底开始当慰安妇,一直做到战败。借支早在1939年就还清了,但不知为什么她没有当老鸨子,一直当慰安妇。

\section{对朝鲜慰安妇的歧视}

日本人慰安妇和朝鲜人慰安妇之间的差别,表面上没有,但为了一些小事,比方说有的士兵没给日本人慰安妇带来礼物,而把羊羹送给了朝鲜人慰安妇时,她们就会打起架来。在这种时候就吐露出找碴的话来道:“一个‘窑包’,拿什么大!”对方说:“朝鲜怎么啦?别把朝鲜人当活宝!”据说就这样扭打在一起。所谓“窑包”,是当时的日本人轻蔑朝鲜人时常常使用的称呼。在这种情况下,朝鲜慰安妇就得在条件差的、日照不好的房间里忍耐。尽管从数目来看,朝鲜人占多数。

朝鲜人慰安妇,如同麻生军医的报告书中所说,因为她们年轻,刚来时多数是处女,事故发生得也多。虽说是事故,也只是慰安妇的事故。前面提到的在孙吴担任军医的人说:“这是在孙吴发生的事。有一次,一个朝鲜人慰安妇让我‘来一下’。一开始我纳闷她是不是有什么企图,但看样子没有这类迹象。在后方医院的军医中,她该是有熟人的,但可能是觉得我这个偶尔被抓去支援一下检诊的人好说话吧。不管怎样,我还是去了。来到房间门口时,她让我‘进来’。因为是上午,慰安所里的人多半已经睡熟了吧。什么人都没碰见。我刚想脱鞋,她说‘不必了’。”

到此还没有什么,但进了房间之后,她刷地一下把壁橱打开说道:

“请看这个。”

“啊!”一瞬间他叫了一声。

他说他看见在壁橱中安安稳稳地睡着一个生下七天左右的婴儿。

“是士兵的孩子吗?”

“……”

“检诊的时候,不知道是妊娠了吗?”

“……”

“你一个人接生的吗?”

“……”

不管他怎么问,她再也不回答了。但这与其说是骗过了军医的眼睛,不如说她自身没有妊娠的知识,等到发现时已经到了出生的时候了。似乎也没有妊娠反应和痛苦,这是因为她年轻且身体健康。即使如此,他对她没有一点性知识而感到吃惊。在从朝鲜农村领来的人当中,这样的人很多。

“军医心想怎么会出了这样的事呢,也许是因为一个固定的人长时期负责检诊,就感到单调而看漏了。尽管如此,一旦有六个月的身孕,哪怕是看上一眼也会知道。”他这样追怀往事说。

那时候,问题是该如何处置这个婴孩和这个朝鲜慰安妇。没想到尽管那样严格命令“必须使用避孕套”,还是带来了这样的结果,发怒去追查那个士兵,也没有用了。在士兵当中有人搞恶作剧,一了解到对手是个无知的年轻朝鲜人慰安妇,就不用避孕套,或者故意把避孕套弄破。因为她们不太懂得日语,在好多情况下是糊里糊涂地被捉弄了。可怜的是被传染上性病的,也多为这样一些年轻的刚刚当上慰安妇的人。

现实是有人在慰安所里生下了婴儿。她只会说:“我受了骗,受了骗。”是说受了士兵的欺骗吗?不是的,是说“从朝鲜被领来受了骗”,受骗被领来的结果是生了孩子。她是要求骗人的一方,也就是要求军队帮她设法解决问题。在这种情况下,母亲让嫖客抱着期间,婴儿却没有哭,这位军医也感到吃惊。据说是由日本人慰安妇帮她做米汤来代替牛奶给孩子吃的。

不管怎样,得处置这件事才行。他回到师司令部报告了情况,得到的回答只有一句话:“这是军队管辖以外的事情,没有处置的必要。”

也确实是这样,检诊是军队的责任,更多的事情在军队管辖之外。说这儿不是幼儿园也就完了。这位军医也没有能力做进一步的处置。不觉就这样过了好几天。他说:“过了半个来月再去一看,那个慰安妇已经不在那里了。上次离开时,她那依赖求救的眼神,让人永远不能忘记。”多半不是把孩子给了谁,一个人又去当慰安妇,就是抱着孩子流落到哪里去了。

据说年轻的朝鲜人慰安妇当中,也有人逃跑。日本人慰安妇是下了决心自愿来的,而那些朝鲜人是被骗、被逼了来的,她们来了一看,很多人大吃一惊,逃跑也不奇怪。这些逃亡者,多半是由雇佣来当翻译的朝鲜男人给引的路。与其说是出于反日意识,不如说对这些作为同胞的本民族女性当牺牲品看不下去吧。虽然如此,在异国他乡,不知道路,语言又不通,结局是好多都去向不明。再说,这种想逃跑的念头,也只是一开始才有,两三个月一过一般也就死了心。这样一些朝鲜人慰安妇,战争结束以后也不回祖国;不,是有家难奔,有国难投。她们来到了日本,其中的大半据说住到了以横滨为中心的地方。

但不知为什么,几乎没有自杀的。在这方面日本人慰安妇也一样,但悲剧还是以朝鲜人为多也是事实。

\section{慰安妇也遭到敌人袭击}

攻打汉口和部队强行军中慰安妇们也遭到了敌人的袭击,也有过她们一个通晚蜷着身子屏住气息待在交战的士兵身边的例子。这样一来就豁出命来了。前面提到的1938年在由上海去往杭州的路上,在货车上就挣了1000元的慰安妇们在行进时,岛田俊夫说:“有两次在附近听到了枪声。”

岛田俊夫还说:

“一开始的时候她们也害怕,但在战场上待惯了的士兵跟她们说:‘没关系,没关系,子弹就是打过来,也打不中。’听他们这么说着,在搂抱当中也就习惯了。再说慰安妇这种人,嘴里虽然说着‘为了国家’,其实已沦落到了社会的最底层,叫做已经豁出去了呢,还是破罐子破摔呢,所以对于敌人来袭根本不当回事。并非拼着命也要挣钱,而是要打死就打死好了。”

加之从战场到战场撑着身子辗转当中,这种事情也就成为家常便饭了。仔细想来,即使有人用鱿鱼干和板栗干为她们送行,她们也发誓为皇军效忠,但毕竟还是前来出卖肉体的,能说是她们特别坚强吗?要求她们绝对“忠”不行吧。说矛盾也是矛盾,但这也许叫做不是矛盾的矛盾。

如果上面说的是日本人慰安妇,那么朝鲜人慰安妇,赴死的心情,也许更为强烈。她们来的时候既没有送行的鱿鱼干,也没有板栗干。一旦来到这里,就连可回的故乡都没有。在自己藏身的堑壕上头,士兵们对打起来,即便流弹飞来,也没有什么可怕的。

“在那种情况下,奇怪的是没有慰安妇被敌弹打死的。莫非人豁出去,子弹就躲着人飞吗?”岛田俊夫说。

但这种评论家式的发言,事到如今才能说,因为是第三者,才能说的吧。豁出去的人,只有本人才知道自己非人的处境。就这一点我问原慰安妇斋藤雾。

“我在转移中也受到过一次袭击……是啊,细想想毕竟还是要死就死好了……那时候下了决心……”她想了一阵子,咕哝似的回答说,“到底是那么回事。”她还说,“那种生活,现在想起来,真是不死不活似的。”

这话使人感到了分量。

据说她们的大敌是胸部疾病,特别是肺结核。我手头没有准确的数字,据说死于肺结核的慰安妇占相当的比例。虽说有军医的检诊,但那只是就性病,不作内科的诊察。就是跟军医诉说胸部的疾患,作为检诊对象之外,也不给予治疗。当然,就是想治疗,当时也没有特效药链霉素。摄取营养,安静地休息是治疗肺结核的方法,而这是慰安妇所无法奢求的。如果是士兵,就被送往后方陆军医院,给予营养,让他静养,但慰安妇既不是军人,又不是文职人员,因此不会被送往陆军医院的。

话虽如此,就是想坐上船回归日本故乡静养,她们也知道如今不是能够回去的身份。发低烧时她们只是感到奇怪,就是咳嗽起来也悄悄地忍耐,只能一天天地一面欺骗着自己,一面继续出卖自己的肉体。咯血就是她们死的宣告。得了肺病的慰安妇,在慰安所的一个角落里蜷着身子躺着,脸色铁青地挣命。据说有10个慰安妇的慰安所里,就有一个结核病患者。其他的疾病,例如传染病之类,由军队的卫生队严加管理,感冒和腹泻这种小病,据说由部队的卫生兵给药。虽然如此,像这样的稍医即愈的小病姑且不说,得了致命的疾病时没有救,对她们这些身在异乡的人来说,是悲惨的。当她们痛哭的时候,哀诉的时候,也没有一个骨肉至亲用手摸摸她们的额头,她们会怎么想呢?

“有一个人在临终的时候,从手提箱里拿出好衣服,一面说着‘把它帮我穿上’,一面偎依着我死了。我想,到底她还是个女人哪。因为没有药,一个朝鲜人慰安妇跟我说:‘把挤的大蒜汁。帮我煮煮喝就好了。’我给她弄着喝下去了。那时的结核真可怕呀。”

这番话是前面提到的斋藤雾跟我说的。总之,战场上的慰安妇,过的就是这样的生活。

\chapter{痛哭,“挺身队”}

\section{朝鲜人未婚女性的悲剧}

\section{开始大量捕捉朝鲜女性}

\chapter{饥饿,然后“玉碎”}

\section{为了国家,我们也死}
\section{阵亡了的慰安妇既无勋章,又无抚恤金}
\section{去瓜岛的慰安妇碰上了空袭}
\section{在菲律宾战线上丢弃慰安妇}

\chapter{随军慰安妇的战斗记录}

\section{华中}
\section{苏门答腊}
\section{缅甸}
\section{腊包尔、新几内亚}
\section{苏拉威西、印度尼西亚}
\section{南婆罗洲}
\section{哈马黑拉岛}
\section{华北、西南诸岛}
\section{马绍尔群岛}

\chapter{跟踪到底}

\section{战败后她们的逸话}
\section{“他们”中的一个人与“她们”中的一个人}

\chapter{“玉碎”中所看到的男人和女人的心理}

\section{火焰喷射器使人变成炭}

\section{是地狱,不,地狱还要好一些}

\section{她们喊道:“军人,你杀了我吧!”}

\section{逃跑,当了俘虏}

\section{士兵为谁而死?}

\section{朝鲜人慰安妇回归故乡}

\chapter{追随慰安妇的虚幻的将军}

\section{让艺伎逃跑晋升为上将}

\section{前来召唤的女密使的本来面目}

\section{当了“面子”牺牲品的救援队}

\section{当地女性的半数当慰安妇不及格}

\section{被遗弃的“女人”如今}

\section{假扮的护士是慰安妇}

\chapter{败走!当时的她们}

\section{服现役六年仍是上等兵}

\section{谈强奸当地妇女的体会}

\section{士兵们的黑话“保重身体”}

\section{用慰安妇交换物资}

\section{以“处置”之名用药毒杀}

\section{被盯上,要吃朝鲜慰安妇人肉的恐怖}

\section{围绕着一块木薯发了狂的士兵}

\section{如今还有不少人流落在异国他乡}

\chapter{歧视到了执拗的程度}

\section{军队教育本身就贯穿着歧视}

\section{警察对杀死慰安妇事件的态度}

\section{对随地便溺也不当回事的士兵们}

\section{穷困者回国吧}

\chapter{“求粮于敌”的思想}

\section{“日本兵那么爱女人”}

\section{掠夺战争的下场}

\backmatter

\section{译后记}

\end{document}