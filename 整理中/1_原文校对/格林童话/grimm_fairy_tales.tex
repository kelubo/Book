% 格林童话 - 原文与中译对照
% 使用xelatex编译

\documentclass[12pt,a4paper,twoside]{ctexbook}

% 页面设置
\usepackage[a4paper,
            left=2.5cm, right=2.5cm,
            top=2.5cm, bottom=2.5cm,
            headsep=1cm, footskip=1cm,
            bindingoffset=0.5cm]{geometry}

% 字体设置 - 移除babel包,使用ctex内置的中文支持
\usepackage{xeCJK}
\usepackage{fontspec}
\usepackage{microtype}
\usepackage{tikz} % 用于封面装饰
\usepackage{xcolor} % 用于颜色设置

% 设置中文字体
\setCJKmainfont{SimSun}[BoldFont=SimHei, ItalicFont=KaiTi]
\setCJKsansfont{SimHei}
\setCJKmonofont{SimSun}

% 设置英文字体
\setmainfont{Times New Roman}[BoldFont=Times New Roman Bold, ItalicFont=Times New Roman Italic]

% 定义常用字体命令
\newcommand{\hei}{\heiti}
\newcommand{\song}{\songti}

% 章节标题设置
\ctexset{
    part/name={第,卷},
    part/number={\chinese{part}},
    chapter/name={第,章},
    chapter/number={\chinese{chapter}},
    section/name={第,节},
    section/number={\arabic{section}},
    chapter/format={\centering\hei\zihao{2}},
    section/format={\hei\zihao{4}}
}

% 目录设置
\usepackage{titletoc}
\titlecontents{chapter}[0pt]{\vspace{10pt}\bfseries\zihao{-3}\filcenter}{}{}{\titlerule*[8pt]{.}\contentspage}
\titlecontents{section}[2.5em]{\vspace{5pt}\zihao{4}\filcenter}{}{}{\titlerule*[8pt]{.}\contentspage}

% 页眉页脚设置
\usepackage{fancyhdr}
\pagestyle{fancy}
\fancyhf{}
\fancyhead[LE,RO]{\zihao{5}\thepage}
\fancyhead[LO]{\zihao{5}\leftmark}
\fancyhead[RE]{\zihao{5}\rightmark}
\renewcommand{\chaptermark}[1]{\markboth{\chaptername\ \thechapter\ #1}{}}
\renewcommand{\sectionmark}[1]{\markright{\thesection\ #1}}
\fancyfoot[C]{\zihao{5} \thepage}
\renewcommand{\headrulewidth}{0.4pt}
\renewcommand{\footrulewidth}{0pt}

% 段落设置
\usepackage{indentfirst}
\setlength{\parindent}{2em}
\setlength{\parskip}{0.5em}

% 原文与译文环境 - 简化版本
\newcommand{\original}[1]{
    \vspace{1em}
    \noindent\textbf{【原文】}\\
    \begin{quote}
        #1
    \end{quote}
    \vspace{1em}
}

\newcommand{\translation}[1]{
    \vspace{1em}
    \noindent\textbf{【译文】}\\
    \begin{quote}
        #1
    \end{quote}
    \vspace{1em}
}

% 标题页信息
\title{\hei\zihao{0} 格林童话}
\author{\song\zihao{2} 雅各布·格林 / 威廉·格林}
\date{\song\zihao{4} \today}

\begin{document}

% 封面
\begin{titlepage}
    \begin{center}
        % 顶部装饰
        \vspace*{5cm}
        
        % 书名
        \begin{minipage}{0.8\textwidth}
            \centering
            \Huge\bfseries\hei 格林童话
        \end{minipage}
        
        \vspace*{1cm}
        
        % 副标题
        \begin{minipage}{0.7\textwidth}
            \centering
            \Large\hei 原文与中译对照
        \end{minipage}
        
        \vspace*{2.5cm}
        
        % 装饰线条
        \rule{0.5\textwidth}{0.8pt}
        
        \vspace*{1.5cm}
        
        % 作者
        \begin{minipage}{0.8\textwidth}
            \centering
            \Large\song 雅各布·格林 / 威廉·格林
        \end{minipage}
        
        \vspace*{1.5cm}
        
        % 装饰线条
        \rule{0.5\textwidth}{0.8pt}
        
        \vspace*{2.5cm}
        
        % 日期
        \begin{minipage}{0.8\textwidth}
            \centering
            \large\song \today
        \end{minipage}
        
        \vspace*{3cm}
    \end{center}
\end{titlepage}

% 版权页
\newpage
\thispagestyle{empty}
\begin{center}
    \vspace*{8cm}
    \song\zihao{5} 版权所有 \textcopyright\ 2026 格林童话编译组
    \vspace*{1cm}
    \song\zihao{5} 仅供学习交流使用
\end{center}

% 前言
\newpage
\chapter{前言}

格林童话是德国民间文学的代表作品,由雅各布·格林和威廉·格林兄弟收集、整理、加工完成。本书收录了多篇经典童话,并提供原文与中文译文对照,以便读者学习和欣赏。

格林童话自问世以来,已被翻译成世界上100多种语言,在全球范围内广泛传播。其中许多故事,如《小红帽》《白雪公主》《灰姑娘》等,早已成为世界文学宝库中的经典之作。

\vspace{1cm}
\song\zihao{4} \today

% 目录
\newpage
\tableofcontents

% 正文开始
\mainmatter

% 第一篇:小红帽
\chapter{小红帽}
\section{故事内容}

\original{
Rotkäppchen

Es war einmal ein kleines Mädchen, das hatte eine Großmutter, die ihr alles, was sie wollte, schenkte. Da schenkte sie ihr einmal ein Käppchen aus roter Seide, das passte ihr so gut, dass sie nichts anderes mehr trug, und darum hießen alle sie Rotkäppchen.

Eines Tages sprach die Mutter zu ihr: "Komm, Rotkäppchen, da hast du ein Stück Kuchen und eine Flasche Wein. Bring das der Großmutter; sie ist krank und schwach, und das wird ihr wohl tun. Mach dich auf, bevor es heiß wird, und wenn du hinausgehst, so geh ganz lieb und sonder dich nicht vom Weg, sonst fällst du und zerbrichst das Glas, und die Großmutter bekommt nichts. Und wenn du in ihr Haus kommst, so sage erst Guten Tag, und geh nicht gleich in die Kammer hinein."

"Ich werde schon alles genau beachten", sagte Rotkäppchen zur Mutter, und gab ihr die Hand darauf.

Die Großmutter wohnte draußen im Wald, eine halbe Stunde vom Dorf entfernt. Als Rotkäppchen nun in den Wald kam, begegnete ihr der Wolf. Rotkäppchen aber wusste nicht, was ein böses Tier das ist, und fürchtete es gar nicht.

"Guten Tag, Rotkäppchen", sagte der Wolf.

"Guten Tag, Wolf", antwortete Rotkäppchen.

"Wo gehst du hin so früh, Rotkäppchen?"

"Zum Großmutterhaus."

"Was trägst du unter dem Schärpe?"

"Kuchen und Wein. Die Großmutter ist krank und schwach, da soll es ihr besser machen."

"Wo wohnt deine Großmutter, Rotkäppchen?"

"Einen kleinen Weg weiter im Wald, unter den drei großen Eichen, dort steht ihr Haus, hinter dem Haselbusch. Du kennst ihn ja wohl."

Der Wolf dachte bei sich: "So jung und frisch, das ist besser als die alte Großmutter. Ich will beide fressen, das ist einen Tag wert."

Er ging ein Weilchen mit Rotkäppchen weiter, dann sagte er: "Rotkäppchen, sieh mal die schönen Blumen, die rings um dich stehen. Warum guckst du nicht hin? Du gehst ja so schnell, als ob du zur Schule ginge."

Rotkäppchen sah sich um und fand, dass die Blumen im Wald sehr schön waren. Sie dachte: "Wenn ich der Großmutter eine Handvoll frischer Blumen mitbringe, die wird sich sehr freuen. Es ist noch so früh, ich komme doch noch rechtzeitig an."

Da lief sie von dem Wege ab und suchte Blumen. Jedes Mal, wenn sie eine schöne fand, merkte sie, dass da noch eine schönere weiter drüben war, und so ging sie immer tiefer in den Wald hinein.

Währenddessen lief der Wolf geradeswegs zum Haus der Großmutter und klopfte an die Tür.

"Wer ist da?", rief die Großmutter.

"Ich bin Rotkäppchen," antwortete der Wolf, "ich bringe dir Kuchen und Wein. Mach die Tür auf."

"Drück nur den Knauf," rief die Großmutter, "ich bin schwach und kann nicht aufstehen."

Der Wolf drückte den Knauf, die Tür sprang auf, und er ging ohne ein Wort weiter zur Bettstatt und friss die arme Großmutter. Dann zog er ihre Kleider an, setzte ihre Mütze auf und legte sich in ihr Bett.

Rotkäppchen aber hatte viele Blumen gesammelt und war nun zu spät gekommen. Als sie kam und anklopfte, rief der Wolf mit verstellter Stimme: "Drück nur den Knauf, ich bin schwach und kann nicht aufstehen."

Rotkäppchen drückte den Knauf, die Tür sprang auf. Als sie hereinkam, merkte sie etwas Seltsames, und sagte: "Ach, Großmutter, was hast du für große Ohren!"

"Damit ich dich besser hören kann, mein Kind," antwortete der Wolf.

"Ach, Großmutter, was hast du für große Augen!"

"Damit ich dich besser sehen kann, mein Kind."

"Ach, Großmutter, was hast du für große Hände!"

"Damit ich dich besser fassen kann, mein Kind."

"Und Großmutter, was hast du für ein großes Maul mit solchen scharfen Zähnen!"

"Damit ich dich besser fressen kann!"

Und mit diesen Worten sprang der Wolf aus dem Bett und fraß Rotkäppchen auf.

Und nun war der Wolf satt und legte sich wieder ins Bett und schlief ein und schnarchte sehr laut.

Da kam ein Jäger vorbei, der dachte: "Warum schnarcht die alte Frau so laut? Ich will mal sehen, ob es ihr gut geht."

Er ging in die Kammer und sah den Wolf im Bett liegen. "Du alter Schurke, ich hab dich endlich erwischt!", sagte er, und machte sich bereit, den Wolf zu erschießen.

Als er aber den Wolf genau ansah, dachte er: "Er hat vielleicht die Großmutter gefressen, ich muss sehen, ob ich sie noch retten kann."

Also nahm er sein Messer und schnitt den Wolf mitten im Bauch auf.

Kaum hatte er zwei Schnitte gemacht, da guckte Rotkäppchen heraus und rief: "Ach, wie war ich erschrocken! Der Wolf hat mich so fest geschluckt!"

Und dann sprang die Großmutter auch heraus, noch lebendig, aber sehr erschrocken.

Rotkäppchen holte große Steine, die füllten sie in den Bauch des Wolfes. Als der Wolf aufwachte, wollte er wegrennen, aber die Steine waren so schwer, dass er gleich zusammenstürzte und starb.

Da waren alle drei froh: der Jäger nahm den Wolf pelz und Felle mit, die Großmutter aß den Kuchen und trank den Wein, den Rotkäppchen mitgebracht hatte, und Rotkäppchen dachte: "Ich werde künftig nie wieder von dem Wege ablaufen, wenn meine Mutter mir das verbietet."

Und damit war das Märchen zu Ende.
}

\tr\t\translation{
小红帽

从前有个小女孩,她的外婆非常疼爱她,总是给她想要的一切。有一次,外婆送给她一顶红色的丝绒帽子,她戴起来非常合适,从此以后她就只戴这顶帽子,因此大家都叫她“小红帽”。

一天,妈妈对她说:“来,小红帽,这里有一块蛋糕和一瓶葡萄酒,你把它们带给外婆吧。外婆生病了,身体很虚弱,吃了这些东西会好一些。趁天气还没热起来,你就出发吧。出门后要走大路,不要离开路径,否则你会跌倒,把玻璃瓶子摔碎,这样外婆就什么也得不到了。到了外婆家,要先问声好,不要直接就进房间。”

“我会小心的。”小红帽对妈妈说,并且拉着妈妈的手保证。

外婆住在森林里,离村子有半小时的路程。当小红帽走进森林时,遇到了一只狼。但小红帽不知道狼是一种坏动物,所以一点也不害怕它。

“你好,小红帽。”狼说。

“你好,狼。”小红帽回答。

“这么早你要去哪里呀,小红帽?”

“去外婆家。”

“你围裙下带着什么?”

“蛋糕和葡萄酒。外婆生病了,身体很虚弱,吃了这些东西会好一些。”

“你的外婆住在哪里,小红帽?”

“在森林里的小路尽头,三棵大橡树下,她的房子就在那里,榛树丛后面。你一定知道吧。”

狼心里想:“这个小东西又年轻又新鲜,比那个老太婆好吃多了。我要把她们两个都吃掉,这值得我花一天的时间。”

他和小红帽一起走了一会儿,然后说:“小红帽,你看看周围这些美丽的花。你为什么不看看呢?你走得这么快,好像要去上学似的。”

小红帽环顾四周,发现森林里的花确实非常美丽。她想:“如果我给外婆带一把新鲜的花,她一定会很高兴的。现在还早,我肯定能准时到达。”

于是她离开小路,去采摘鲜花。每次她发现一朵美丽的花,就会看到远处还有更美丽的一朵,就这样她一直走到了森林深处。

与此同时,狼直接跑到了外婆家,敲了敲门。

“谁呀?”外婆喊道。

“我是小红帽,”狼回答,“我给你带来了蛋糕和葡萄酒。请开门。”

“你只需要按一下门把手,”外婆喊道,“我身体虚弱,起不来。”

狼按了门把手,门开了,他一言不发地走到床边,把可怜的外婆吃掉了。然后他穿上她的衣服,戴上她的帽子,躺到她的床上。

小红帽摘了很多花,然后才出发。当她到达外婆家时,敲了敲门。狼用变了调的声音喊道:“你只需要按一下门把手,我身体虚弱,起不来。”

小红帽按了门把手,门开了。当她走进房间时,她觉得有些奇怪,说:“哦,外婆,你的耳朵怎么这么大!”

“这样我才能更好地听到你的声音,我的孩子。”狼回答。

“哦,外婆,你的眼睛怎么这么大!”

“这样我才能更好地看到你,我的孩子。”

“哦,外婆,你的手怎么这么大!”

“这样我才能更好地抓住你,我的孩子。”

“还有外婆,你的嘴巴怎么这么大,牙齿这么锋利!”

“这样我才能更好地吃掉你!”

说完这些话,狼从床上跳起来,把小红帽吃掉了。

狼吃得饱饱的,重新躺回床上,睡着了,还打着响亮的呼噜。

这时,一个猎人经过,他想:“这个老太婆怎么打这么响的呼噜?我要去看看她是否安好。”

他走进房间,看到狼躺在床上。“你这个老坏蛋,我终于抓住你了!”他说,准备开枪打死狼。

但当他仔细看狼时,他想:“他可能已经吃掉了外婆,我必须看看是否能救她。”

于是他拿出刀,在狼的肚子上划了两刀。

他刚划了两刀,小红帽就探出头来,喊道:“哦,我吓死了!狼把我吃得这么紧!”

然后外婆也跳了出来,还活着,但非常害怕。

小红帽拿来大石头,把狼的肚子填满。当狼醒来时,想逃跑,但石头太重了,他立刻倒在地上死了。

这时三个人都很高兴:猎人拿走了狼的皮和毛,外婆吃了小红帽带来的蛋糕,喝了葡萄酒,小红帽想:“我以后再也不会违背妈妈的话,离开大路了。”

故事就这样结束了。
}

% 第二篇:白雪公主
\chapter{白雪公主}
\section{故事内容}

\original{
Schneewittchen

Es war einmal eine Königin, die saß an ihrem Fenster, das hatte einen Rahmen aus schwarzem Ebenholz. Der Schnee fiel draußen, und die Königin nähte und blickte zum Schnee hinaus. Plötzlich stach sie sich mit der Nadel in den Finger, und drei Tropfen Blut fielen ins Schnee. Der rote Blut auf dem weißen Schnee, das sah so schön aus, dass sie dachte: "Ach, wenn ich nur ein Kind hätte, so weiß wie Schnee, so rot wie Blut und so schwarzhaarig wie dieser Ebenrahmen!"

Bald darauf bekam sie ein Mädchen, das war so weiß wie Schnee, so rot wie Blut und hatte schwarze Haare, also rief sie es Schneewittchen. Aber als das Kind geboren war, starb die Königin.

Der König heiratete bald eine andere Frau, die war sehr schön, aber eine eitel und boshaft Frau. Sie hatte einen Spiegel, den sie jeden Tag fragte: "Spieglein, Spieglein an der Wand, wer ist die Schönste im ganzen Land?"

Und der Spiegel antwortete immer: "Frau Königin, Ihr seid die Schönste im ganzen Land."

Da war die Königin zufrieden, denn sie wusste, dass der Spiegel immer die Wahrheit sagte.

Schneewittchen aber wuchs heran und wurde immer schöner. Als es sieben Jahre alt war, war es so schön, dass es selbst die Königin an Schönheit übertraf. Eines Tages fragte die Königin den Spiegel wieder: "Spieglein, Spieglein an der Wand, wer ist die Schönste im ganzen Land?"

Und der Spiegel antwortete: "Frau Königin, Ihr seid schön, es ist wahr, aber Schneewittchen ist tausendmal schöner als Ihr."

Die Königin wurde böse und voller Neid. Sie konnte nicht ruhig schlafen, weil sie das denken musste, dass Schneewittchen schöner war als sie.

Endlich rief sie einen Jäger und sagte zu ihm: "Bring Schneewittchen hinaus in den Wald, ich will es nie wieder sehen. Töt es und bring mir sein Lunge und Leber als Beweis."

Der Jäger nahm Schneewittchen mit in den Wald. Als er sein Schwert ziehen wollte, um es zu töten, fing das arme Kind an zu weinen und bat ihn: "Ach, guter Jäger, lass mich leben! Ich will in den Wald gehen und nie wieder nach Hause kommen."

Der Jäger hatte Mitleid mit dem Kind und ließ es laufen. Er tötete stattdessen ein Wild und nahm seine Lunge und Leber, um der Königin zu zeigen.

Schneewittchen aber war allein im großen Wald. Es begann zu weinen, aber es lief weiter und weiter, bis es ein kleines Häuschen fand. Das Häuschen war so klein, dass alles darin sehr klein war: kleine Stühle, ein kleines Tisch, sieben kleine Teller, sieben kleine Messer und Gabeln, sieben kleine Gläser und sieben kleine Betten nebeneinander.

Schneewittchen war hungrig und durstig, also aß es von jedem Tellern ein bisschen Brot und Butter und trank aus jedem Glas einen kleinen Schluck Wein. Dann war es müde und legte sich in eines der Betten, aber keines passte. Endlich fand es eins, das passte, und schlief ein.

Abends kamen sieben Zwerge nach Hause. Sie waren klein, aber stark, und arbeiteten im Berg als Bergleute. Als sie ins Haus kamen, merkten sie, dass jemand da gewesen war. "Wer hat meinen Stuhl berührt?", fragte der erste. "Wer hat von meinem Teller gegessen?", fragte der zweite. "Wer hat von meinem Brot gegessen?", fragte der dritte. "Wer hat von meiner Butter gegessen?", fragte der vierte. "Wer hat mein Messer berührt?", fragte der fünfte. "Wer hat meine Gabel benutzt?", fragte der sechste. "Wer hat aus meinem Glas getrunken?", fragte der siebte.

Dann gingen sie ins Schlafzimmer und sahen Schneewittchen im Bett schlafen. "Wie schön ist das Kind!", riefen sie alle auf einmal. Sie ließen es schlafen und wachten über es.

Am nächsten Morgen erwachte Schneewittchen. Die Zwerge fragten sie, wer sie sei und wie sie dazu gekommen sei. Schneewittchen erzählte ihnen alles, und die Zwerge sagten: "Wenn du bei uns bleiben willst und unser Haus für uns ordnest, kochst und waschst, machen und strickst, so kannst du bei uns bleiben. Wir werden dich behüten und dir nichts zuleiden lassen."

Schneewittchen sagte ja, und so lebte sie bei den sieben Zwergen.

Die Königin aber nahm an, dass Schneewittchen tot war. Eines Tages fragte sie wieder den Spiegel: "Spieglein, Spieglein an der Wand, wer ist die Schönste im ganzen Land?"

Und der Spiegel antwortete: "Frau Königin, Ihr seid schön, es ist wahr, aber Schneewittchen lebt bei den sieben Zwergen, und sie ist tausendmal schöner als Ihr."

Die Königin war wütend, denn sie wusste nun, dass der Jäger sie belogen hatte. Sie verkleidete sich als eine alte Krämerin und ging in den Wald zu den Zwergen.

Als die Zwerge ausgingen, klopfte sie an die Tür und rief: "Schöne Ware, schöne Ware! Spangen und Schnallen!"

Schneewittchen fragte: "Was habt Ihr zu verkaufen?"

"Schöne Spangen", antwortete die Krämerin und nahm eine Spange aus ihrem Korb. "Lass mich dir eine Spange anlegen, mein Kind. Das wird dich sehr schön machen."

Schneewittchen ließ sich nicht aufhalten, und die Krämerin band die Spange so fest, dass Schneewittchen ohnmächtig wurde. Die Königin ging weg, und dachte: "Jetzt ist sie tot!"

Aber abends kamen die Zwerge nach Hause und fanden Schneewittchen ohnmächtig. Sie banden die Spange los, und Schneewittchen erwachte wieder.

Die Königin aber ging wieder zu ihrem Spiegel und fragte: "Spieglein, Spieglein an der Wand, wer ist die Schönste im ganzen Land?"

Und der Spiegel antwortete wieder: "Frau Königin, Ihr seid schön, es ist wahr, aber Schneewittchen lebt bei den sieben Zwergen, und sie ist tausendmal schöner als Ihr."

Diese Zeit verkleidete die Königin sich als eine alte Alte und brachte einen Giftkamm mit. Als die Zwerge weg waren, klopfte sie an die Tür und rief: "Schöne Ware, schöne Ware!"

Schneewittchen sagte: "Ich darf nicht mehr etwas öffnen."

"Dann schau dir wenigstens diesen schönen Kamm an", antwortete die Alte und hielt ihn in die Höhe. Schneewittchen nahm den Kamm, und die Alte kämmte ihr die Haare. Aber sobald sie den Kamm in das Haar steckte, wurde Schneewittchen ohnmächtig.

Die Königin ging weg, aber die Zwerge kamen nach Hause und fanden Schneewittchen ohnmächtig. Sie zogen den Kamm heraus, und Schneewittchen erwachte wieder.

Die Königin fragte den Spiegel ein drittes Mal: "Spieglein, Spieglein an der Wand, wer ist die Schönste im ganzen Land?"

Und der Spiegel antwortete wieder: "Frau Königin, Ihr seid schön, es ist wahr, aber Schneewittchen lebt bei den sieben Zwergen, und sie ist tausendmal schöner als Ihr."

Die Königin war sehr wütend und dachte: "Jetzt werde ich sie wirklich töten."

Sie kochte einen roten Apfel, der auf der einen Seite rot war und gut schmeckte, auf der anderen Seite aber vergiftet war. Dann verkleidete sie sich als eine Bäuerin und ging zu den Zwergen.

Als die Zwerge weg waren, klopfte sie an die Tür. Schneewittchen sagte: "Ich darf nicht mehr etwas öffnen und nicht mehr etwas nehmen."

"Das macht nichts", antwortete die Bäuerin, "ich will den Apfel teilen. Der rote Teil ist für dich, der weiße für mich."

Schneewittchen wollte nicht, aber die Bäuerin hielt den Apfel in die Höhe und biss von der weißen Seite. Dann wollte Schneewittchen auch probieren. Als sie einen Bissen von der roten Seite aß, fiel sie tot zu Boden.

Die Königin ging weg, und sagte: "Jetzt ist sie endlich tot!"

Als die Zwerge nach Hause kamen, fanden sie Schneewittchen tot. Sie versuchten alles, um sie zu retten, aber es half nichts. Sie weinten sehr und legten sie in einen Sarg aus Glas. Er stand auf einem Berg, und die Zwerge wachten abwechselnd über ihn.

Ein Jahr verging, und ein junger König kam durch den Wald. Er sah den Sarg mit Schneewittchen darin und war so betroffen von ihrer Schönheit, dass er sagte: "Ich will den Sarg mitnehmen. Ich will ihm alles geben, was er will, nur nicht ihn verlassen."

Die Zwerge ließen sich überzeugen und gaben ihm den Sarg. Als die Diener den Sarg trugen, stolperten sie über einen Stein, und der Apfelkern, der im Hals von Schneewittchen steckte, sprang heraus. Plötzlich öffnete Schneewittchen die Augen, reckte sich und sagte: "Wo bin ich?"

Der König war sehr glücklich und sagte: "Du bist bei mir. Willst du meine Frau werden?"

Schneewittchen sagte ja, und sie heirateten in großer Pracht.

Die böse Königin wurde eingeladen, und als sie kam, erkannte sie Schneewittchen. Aber die Königin wurde sehr aufgeregt und starb vor Neid und Zorn.

Und Schneewittchen und der König lebten glücklich bis ans Ende ihrer Tage.
}

\tr\t\translation{
白雪公主

从前有一位王后,她坐在黑檀木做的窗户旁。外面正下着雪,王后一边做着针线活,一边望着窗外的雪景。突然,她的手指被针扎了一下,三滴鲜血滴在了雪地上。白雪上的红血看起来非常美丽,王后心想:“啊,如果我能有一个孩子,皮肤像雪一样白,嘴唇像血一样红,头发像黑檀木一样黑,那该多好啊!”

不久之后,王后生下了一个小女孩,她的皮肤像雪一样白,嘴唇像血一样红,头发像黑檀木一样黑,于是王后给她取名叫“白雪公主”。但是白雪公主出生后不久,王后就去世了。

国王很快又娶了一位妻子,她非常美丽,但却是一个虚荣又邪恶的女人。她有一面镜子,每天都要问它:“镜子,镜子,墙上的镜子,谁是全国最美丽的人?”

镜子总是回答:“王后陛下,您是全国最美丽的人。”

王后对此很满意,因为她知道镜子总是说实话。

白雪公主渐渐长大了,变得越来越美丽。当她七岁时,她已经美得连王后都比不上了。有一天,王后又问镜子:“镜子,镜子,墙上的镜子,谁是全国最美丽的人?”

镜子回答说:“王后陛下,您很美,这是真的,但白雪公主比您美丽一千倍。”

王后变得非常生气和嫉妒。她无法平静地入睡,因为她总是想着白雪公主比她美丽这件事。

最后,她叫来了一个猎人,对他说:“把白雪公主带到森林里去,我不想再见到她。杀了她,把她的肺和肝带给我作为证据。”

猎人带着白雪公主走进了森林。当他拔出剑想要杀死她时,可怜的孩子开始哭泣,哀求他:“啊,好心的猎人,放我一条生路吧!我会走进森林,再也不回家了。”

猎人同情这个孩子,让她跑了。他杀死了一只野兽,取出了它的肺和肝,拿给王后看。

白雪公主独自一人在大森林里。她开始哭泣,但她继续向前走,直到她发现了一座小房子。这座房子很小,里面的一切都很小:小椅子、小桌子、七个小盘子、七把小刀和叉子、七个小杯子和七张并排的小床。

白雪公主又饿又渴,于是她从每个盘子里吃了一点面包和黄油,从每个杯子里喝了一小口葡萄酒。然后她累了,躺到了一张床上,但没有一张合适。最后,她找到了一张合适的床,就睡着了。

晚上,七个小矮人回家了。他们个子很小,但很强壮,在山里当矿工。当他们走进房子时,他们发现有人来过。“谁动了我的椅子?”第一个问道。“谁吃了我的盘子里的东西?”第二个问道。“谁吃了我的面包?”第三个问道。“谁吃了我的黄油?”第四个问道。“谁碰了我的刀?”第五个问道。“谁用了我的叉子?”第六个问道。“谁喝了我的杯子里的酒?”第七个问道。

然后他们走进卧室,看到白雪公主在床上睡觉。“这个孩子多么美丽啊!”他们异口同声地喊道。他们让她睡觉,并守着她。

第二天早上,白雪公主醒来了。小矮人们问她是谁,她是怎么到那里的。白雪公主把一切都告诉了他们,小矮人们说:“如果你愿意和我们住在一起,为我们整理房子,做饭、洗衣、缝补和编织,你就可以和我们住在一起。我们会保护你,不让你受到伤害。”

白雪公主同意了,于是她和七个小矮人一起生活。

王后以为白雪公主已经死了。有一天,她又问镜子:“镜子,镜子,墙上的镜子,谁是全国最美丽的人?”

镜子回答说:“王后陛下,您很美,这是真的,但白雪公主和七个小矮人住在一起,她比您美丽一千倍。”

王后很生气,因为她现在知道猎人骗了她。她装扮成一个老妇人,走进森林去找小矮人。

当小矮人们外出时,她敲了敲门,喊道:“漂亮的东西,漂亮的东西!发夹和搭扣!”

白雪公主问道:“你卖什么?”

“漂亮的发夹,”老妇人回答,从她的篮子里拿出一个发夹。“让我给你戴一个发夹,我的孩子。这会让你变得非常美丽。”

白雪公主没有拒绝,老妇人把发夹紧紧地系在她的头上,让她昏了过去。王后离开了,心想:“现在她死了!”

但晚上小矮人们回家了,发现白雪公主昏迷不醒。他们解开了发夹,白雪公主又醒了过来。

王后又去问镜子:“镜子,镜子,墙上的镜子,谁是全国最美丽的人?”

镜子又回答说:“王后陛下,您很美,这是真的,但白雪公主和七个小矮人住在一起,她比您美丽一千倍。”

这次王后装扮成一个老太婆,带来了一把有毒的梳子。当小矮人们外出时,她敲了敲门,喊道:“漂亮的东西,漂亮的东西!”

白雪公主说:“我不能再开门了。”

“那你至少看看这把漂亮的梳子吧,”老太婆回答,把梳子举了起来。白雪公主接过了梳子,老太婆给她梳了头发。但她一把梳子插在头发里,白雪公主就昏了过去。

王后离开了,但小矮人们回家后发现白雪公主昏迷不醒。他们拔出了梳子,白雪公主又醒了过来。

王后第三次问镜子:“镜子,镜子,墙上的镜子,谁是全国最美丽的人?”

镜子又回答说:“王后陛下,您很美,这是真的,但白雪公主和七个小矮人住在一起,她比您美丽一千倍。”

王后非常生气,心想:“现在我要真的杀了她。”

她做了一个红苹果,一边是红色的,味道很好,另一边是有毒的。然后她装扮成一个农妇,去找小矮人。

当小矮人们外出时,她敲了敲门。白雪公主说:“我不能再开门,也不能再拿任何东西了。”

“没关系,”农妇回答说,“我会把苹果分开。红色的部分给你,白色的部分给我。”

白雪公主不想吃,但农妇把苹果举了起来,咬了一口白色的部分。然后白雪公主也想尝一尝。当她咬了一口红色的部分时,她倒在地上死了。

王后离开了,说:“现在她终于死了!”

当小矮人们回家时,他们发现白雪公主死了。他们尝试了一切来救她,但都无济于事。他们非常伤心,把她放在一个玻璃棺材里。棺材放在一座山上,小矮人们轮流守着它。

一年过去了,一位年轻的国王穿过森林。他看到了装有白雪公主的棺材,被她的美丽所打动,说:“我要带走这个棺材。我会给他他想要的一切,只是不要离开他。”

小矮人们被说服了,把棺材给了他。当侍从们抬着棺材时,他们被一块石头绊倒了,卡在白雪公主喉咙里的苹果核掉了出来。突然,白雪公主睁开了眼睛,坐了起来,说:“我在哪里?”

国王非常高兴,说:“你在我身边。你愿意成为我的妻子吗?”

白雪公主答应了,他们举行了盛大的婚礼。

邪恶的王后被邀请了,当她来的时候,她认出了白雪公主。但是王后非常激动,嫉妒和愤怒而死。

白雪公主和国王幸福地生活在一起,直到他们生命的尽头。
}

% 第三篇:灰姑娘
\chapter{灰姑娘}
\section{故事内容}

\original{
Aschenputtel

Es war einmal ein reicher Mann, der hatte eine schöne Frau mit einem treuen Herzen. Die Frau erkrankte und starb. Vor ihrem Tod sagte sie zu ihrer einzigen Tochter: "Sei immer gut und fromm, so wird dir der Himmel behüten."

Nach dem Tod seiner Frau heiratete der Mann eine schöne Frau, die aber eine schlechte und heimtückische Stiefmutter war. Sie hatte zwei Töchter, die nach ihrem Aussehen schön waren, aber ihre Seelen waren boshaft und hässlich.

Die arme Stieftochter musste nun alle Arbeiten im Haus tun: Kochen, waschen, putzen und stricken. Sie trug ein altes graues Kittel und Schuhe voll Löcher, und die Stiefschwestern nannten sie "Aschenputtel", weil sie immer im Aschehof saß.

Eines Tages gab der König ein Fest, das drei Tage dauern sollte, um seine einzige Tochter, die Prinzessin, zu verheiraten. Jede junge Frau im Land wurde eingeladen, auch die Stiefschwestern von Aschenputtel.

Als die Stiefschwestern von dieser Einladung hörten, waren sie sehr froh und baten Aschenputtel, ihre Kleider auszupressen und zu bügeln. "Gut," sagte Aschenputtel, "wenn ich das getan habe, darf ich auch gehen?"

"Aschenputtel gehen zum Fest?" riefen die Stiefschwestern. "Du hast keine Kleider und keine Schuhe, und außerdem weißt du nicht, tanzen!"

Aber Aschenputtel bat weiter, bis die Stiefmutter sagte: "Ich habe eine Schüssel Erbsen auf den Aschenhof geworfen. Wenn du sie in zwei Stunden aufsortieren kannst, darfst du mitkommen."

Aschenputtel ging hinaus auf den Aschenhof und rief: "Kleine Vögel, kleine Vögel, alle die da sind, Spatz und Rabe, Krähe und Kuckuck, Kommt und helft mir, Erbsen aufzusortieren!"

Sofort kamen tausende Vögel, alle kleinen Vögel, und sortierten die Erbsen: die guten in die Schüssel, die schlechten auf den Aschenhof. In einer Stunde hatten sie die Arbeit beendet und flogen wieder davon.

Aschenputtel ging zurück zu ihrer Stiefmutter und zeigte ihr die sortierten Erbsen. Aber die Stiefmutter sagte: "Es ist zu spät. Du hast keine Kleider, und du darfst nicht mitkommen."

Die Stiefmutter und die Stiefschwestern gingen ohne Aschenputtel zum Fest. Aschenputtel weinte sehr, aber dann ging ihr Vater herein und fragte: "Was fehlt dir, mein Kind?"

"Ich möchte auch zum Fest gehen, aber die Stiefmutter lässt mich nicht."

"Du hast keine Kleider, mein Kind, aber wenn du willst, kann ich dir etwas geben."

Aber die Stiefmutter kam herein und sagte: "Sie hat keine Kleider, und sie darf nicht gehen."

Aschenputtel weinte sehr, aber dann ging sie in den Garten und rief: "Vater, Vater, mein Vater, gib mir ein Stämmchen von der Nussbaum, die vor dem Haus steht."

Ihr Vater gab ihr ein Stämmchen, und Aschenputtel ging mit ihm in den Wald, pflanzte es in das Grab ihrer Mutter und weinte so viel, dass ihre Tränen es bewässerten.

Das Stämmchen wuchs schnell zu einem großen Baum, und ein kleiner weißer Vögel saß darauf. Jeden Tag kam Aschenputtel zu dem Baum und fragte: "Nussbaum, Nussbaum, schau mich an, Nussbaum, Nussbaum, wachsen schnell, Gib mir Gold und Silber, dass ich zum Fest gehen kann!"

Und der kleine weiße Vogel warf ihr ein Kleid aus Gold und Silber und Schuhe aus Glas herab.

Eines Tages, als das Fest begann, ging Aschenputtel zu dem Baum, fragte ihn und bekam ein wunderschönes Kleid. Sie zog es an und ging zum Fest. Niemand erkannte sie, nicht einmal ihre Stiefmutter und ihre Stiefschwestern. Sie tanzten mit dem Prinzen, der sie nicht losließ.

Als es spät wurde, wollte Aschenputtel nach Hause gehen. Der Prinz sagte: "Ich will dich begleiten."

Aber Aschenputtel lief weg, und als der Prinz ihr nachlief, verlor sie ein Glas-Schuh. Der Prinz nahm ihn auf und sagte: "Ich werde die Frau finden, die in diesen Schuh passt, und sie wird meine Frau werden."

Am nächsten Tag ging der Prinz mit dem Schuh durch das ganze Land. Jede junge Frau versuchte, den Schuh anzuziehen, aber er passte niemandem. Schließlich kam er zu dem Haus von Aschenputtel.

Die Stiefschwestern waren sehr froh und versuchten, den Schuh anzuziehen. Die erste schnitt ihre Zehen ab, damit er passte, aber der Prinz sah das Blut und wusste, dass es nicht die Richtige war. Die zweite schnitt ihre Ferse ab, aber auch das Blut kam hervor.

Dann sagte Aschenputtel: "Ich möchte es auch versuchen."

Die Stiefmutter und die Stiefschwestern lachten, aber der Prinz sagte: "Jede junge Frau muss es versuchen."

Aschenputtel zog den Schuh an, und er passte perfekt. Dann zog sie den anderen Schuh aus ihrer Tasche, und der Prinz erkannte sie.

"Du bist die Frau, die ich suchte," sagte der Prinz. "Willst du meine Frau werden?"

Aschenputtel sagte ja, und der Prinz nahm sie mit nach Hause.

Am Tag der Hochzeit kamen die Stiefmutter und die Stiefschwestern, um zu feiern. Aber als sie die Treppe hinaufgingen, packten zwei Raben sie an den Haaren und zogen sie runter. Am Ende starben sie an ihrem Leid.

Aschenputtel aber heiratete den Prinzen und lebte glücklich bis ans Ende ihrer Tage.
}

\tr\t\translation{
灰姑娘

从前有一个有钱的男人,他有一个美丽善良的妻子。妻子生病了,不久就去世了。临终前,她对唯一的女儿说:“你要永远善良虔诚,这样上帝会保佑你的。”

妻子死后,男人娶了一个美丽的女人,但这个女人是个坏心肠的继母。她有两个女儿,外表看起来很漂亮,但内心却邪恶丑陋。

可怜的继女现在不得不做家里所有的工作:做饭、洗衣、打扫和缝纫。她穿着一件旧的灰色连衣裙和满是破洞的鞋子,继姐妹们叫她“灰姑娘”,因为她总是坐在灰堆里。

有一天,国王举办了一场为期三天的盛宴,为他唯一的女儿——公主找丈夫。全国所有的年轻女孩都被邀请了,包括灰姑娘的继姐妹们。

当继姐妹们听到这个邀请时,她们非常高兴,请求灰姑娘帮她们压平并熨烫衣服。“好的,”灰姑娘说,“如果我做完了,我也可以去吗?”

“灰姑娘去参加盛宴?”继姐妹们喊道。“你没有衣服和鞋子,而且你也不会跳舞!”

但灰姑娘继续请求,直到继母说:“我把一碗豌豆扔到了灰堆上。如果你能在两个小时内把它们分类出来,你就可以一起去。”

灰姑娘走出家门,来到灰堆,喊道:“小鸟,小鸟,所有的小鸟,麻雀和乌鸦,渡鸦和布谷鸟,来帮助我,把豌豆分类!”

立刻来了成千上万只小鸟,所有的小鸟,它们把豌豆分类:好的放进碗里,坏的放回灰堆。一个小时后,它们完成了工作,又飞走了。

灰姑娘回到继母身边,给她看分类好的豌豆。但是继母说:“太晚了。你没有衣服,不能去。”

继母和继姐妹们不带灰姑娘去参加盛宴。灰姑娘哭得很厉害,但这时她的父亲进来了,问道:“我的孩子,你怎么了?”

“我也想去参加盛宴,但继母不让我去。”

“我的孩子,你没有衣服,但如果你愿意,我可以给你一些东西。”

但继母进来了,说:“她没有衣服,不能去。”

灰姑娘哭得很厉害,但后来她走到花园里,喊道:“父亲,父亲,我的父亲,给我一根前院核桃树的树枝。”

她的父亲给了她一根树枝,灰姑娘带着它来到森林里,把它种在母亲的坟墓上,哭得很厉害,眼泪把它淋湿了。

树枝很快长成了一棵大树,一只白色的小鸟坐在上面。灰姑娘每天都来到这棵树前,问道:“核桃树,核桃树,看看我,核桃树,核桃树,快快生长,给我金和银,让我能去参加盛宴!”

白色的小鸟就会扔给她一件金和银的衣服和一双玻璃鞋。

有一天,当盛宴开始时,灰姑娘来到这棵树前,问它,得到了一件美丽的衣服。她穿上它,去参加盛宴。没有人认出她,甚至她的继母和继姐妹们。她和王子跳舞,王子一刻也不放她走。

当夜深了,灰姑娘想回家。王子说:“我送你回去。”

但灰姑娘跑开了,当王子追她时,她掉了一只玻璃鞋。王子捡起它,说:“我会找到适合这只鞋的女人,她将成为我的妻子。”

第二天,王子带着这只鞋走遍了全国。每个年轻女孩都试图穿上这只鞋,但没有人适合。最后,他来到了灰姑娘的家。

继姐妹们非常高兴,试图穿上这只鞋。第一个剪掉了她的脚趾,以便它适合,但王子看到了血,知道她不是正确的人。第二个剪掉了她的脚跟,但血也流了出来。

然后灰姑娘说:“我也想试试。”

继母和继姐妹们笑了,但王子说:“每个年轻女孩都必须试试。”

灰姑娘穿上这只鞋,它完美地适合。然后她从口袋里拿出另一只鞋,王子认出了她。

“你就是我要找的女人,”王子说。“你愿意成为我的妻子吗?”

灰姑娘答应了,王子带她回家。

在婚礼那天,继母和继姐妹们来庆祝。但当她们走上楼梯时,两只乌鸦抓住她们的头发,把她们拉下来。最后,她们在痛苦中死去。

灰姑娘嫁给了王子,幸福地生活到生命的尽头。
}

% 第四篇:青蛙王子
\chapter{青蛙王子}
\section{故事内容}

\original{
Der Froschkönig oder der eiserne Heinrich

Es war einmal ein König, der hatte mehrere schöne Töchter. Die jüngste war so schön, dass die Sonne sich jedes Mal freute, wenn sie sie ansah.

Nahe dem Königsschloss war ein großer, dunkler Wald, und in der Mitte des Waldes war ein tiefes Wasserloch. Eines heißen Sommertags ging das Königskind hinaus und spielte am Wasserloch.

Es war so heiß, dass das Mädchen sich nach dem Wasser sehnte. Plötzlich fiel ihr goldener Ball ins Wasser und sank bis zum Grund.

Das Mädchen weinte sehr, als es den Ball verloren hatte. Da kam ein Frosch aus dem Wasser und sagte: "Warum weinst du so, schönes Mädchen?"

"Mein goldener Ball ist ins Wasser gefallen", sagte das Mädchen.

"Ich will ihn dir holen", sagte der Frosch, "aber was gibst du mir dafür?"

"Alles, was du willst", sagte das Mädchen, "meine Kleider, meine Perlen und Edelsteine, ja selbst den goldenen Krone, die ich trage."

"Ich will deine Kleider nicht, deine Perlen und Edelsteine nicht, noch weniger deine goldene Krone", sagte der Frosch. "Aber wenn du mich liebst und mich als Freund annimmst, mich sitzen lässt an deinem Tisch, in deinem Bett schlafen lässt und mir alles geben willst, was du hast, dann will ich dir deinen Ball holen."

"Ja, alles, was du willst", sagte das Mädchen, "ich will dir alles geben, wenn du mir meinen Ball holst."

Aber sie dachte: "Was will der dumme Frosch mit allem das? Er sitzt im Wasser und quakt, und kann nicht zu mir kommen."

Der Frosch sagte: "Versprichst du es mir?"

"Ja, ich verspreche es dir", sagte das Mädchen.

Da sank der Frosch ins Wasser und schwamm hinunter. Bald kam er wieder mit dem goldenen Ball im Mund. Er warf den Ball auf das Ufer.

Das Mädchen war sehr froh, als sie ihren Ball wieder hatte. Sie nahm ihn und lief fort.

"Warte, warte!", rief der Frosch, "nehme mich mit!"

Aber das Mädchen hörte nicht auf ihn, sondern lief so schnell wie sie konnte ins Schloss.

Am nächsten Tag, als das Mädchen mit dem König und den anderen Prinzessinnen beim Tisch saß und aß, klopfte es an die Tür.

"Wer ist da?", rief das Mädchen.

"Ich bin der Frosch, den du gestern am Wasserloch versprochen hast, mich an deinem Tisch sitzen zu lassen. Öffne mir die Tür!"

Das Mädchen erschrak, als sie den Frosch hörte. Sie erzählte dem König, was passiert war.

"Du musst halten, was du versprochen hast", sagte der König. "Öffne ihm die Tür!"

Das Mädchen öffnete die Tür, und der Frosch hüpfte herein. Er sagte: "Lass mich auf dein Stuhl sitzen!"

Das Mädchen zögerte, aber der König sagte: "Du musst halten, was du versprochen hast."

Also hob das Mädchen den Frosch auf und setzte ihn auf den Stuhl neben sich. Aber der Frosch sagte: "Nun lass mich auf deinen Tellern essen!"

Das Mädchen tat es, aber sie war sehr widerwillig.

"Nun lass mich von deinem Löffel essen!", sagte der Frosch.

Das Mädchen tat es, aber sie hasste den Frosch.

"Nun lass mich auf dein Knie sitzen!", sagte der Frosch.

Das Mädchen tat es, aber sie war sehr unglücklich.

"Nun lass mich in dein Bett schlafen!", sagte der Frosch.

Das Mädchen begann zu weinen, denn sie fürchtete den kalten, nassen Frosch.

Aber der König sagte: "Du musst halten, was du versprochen hast. Wenn du den Frosch geholfen hast, musst du ihm auch das geben, was du versprochen hast."

Also nahm das Mädchen den Frosch in ihre Kammer und legte ihn auf den Fußboden. Aber der Frosch sagte: "Ich will nicht auf dem Fußboden schlafen! Lass mich in dein Bett!"

Das Mädchen war so wütend, dass sie den Frosch aufhob und gegen die Wand warf. "Nun bist du zufrieden!", schrie sie.

Aber als der Frosch gegen die Wand fiel, verwandelte er sich in einen schönen Prinzen mit goldenem Haar. Er war der Prinz von einem fernen Land, der von einer bösen Hexe in einen Frosch verwandelt worden war.

"Danke dir", sagte der Prinz, "dass du mich von dem Fluch befreit hast. Wenn du willst, kannst du meine Frau werden und mit mir in mein Königreich gehen."

Das Mädchen war sehr glücklich, und sie versprach es ihm.

Am nächsten Morgen kam ein Wagen mit sechs weißen Pferden, um sie abzuholen. Der Prinz und das Mädchen setzten sich hinein, und sie fuhren fort.

Auf dem Weg sahen sie einen Mann, der hinter ihnen herlief. Er schrie: "Warte, warte!"

Der Prinz hieß den Wagen anhalten, und der Mann kam näher. Er war Heinrich, der treue Diener des Prinzen. Als der Prinz in einen Frosch verwandelt worden war, hatte Heinrich drei eiserne Bänder um sein Herz geschlungen, damit es nicht vor Trauer brechen würde.

"Ich bin so froh, dass mein Herr wieder frei ist!", sagte Heinrich. "Meine eisernen Bänder sind nun zerbrochen!"

Und so fuhren sie weiter, und alle waren sehr glücklich.

Der Prinz und das Mädchen heirateten, und sie lebten glücklich bis ans Ende ihrer Tage.
}

\tr\t\translation{
青蛙王子

从前有一位国王,他有几个美丽的女儿。最小的女儿非常美丽,以至于太阳每次看到她都会高兴。

国王的宫殿附近有一片大而黑的森林,森林中央有一个深水池。一个炎热的夏天,小公主出去在水池边玩耍。

天气非常炎热,女孩渴望喝水。突然,她的金球掉进了水里,沉到了底部。

女孩失去了球,哭得很厉害。这时,一只青蛙从水里出来,说:“你为什么哭得这么伤心,美丽的女孩?”

“我的金球掉进水里了,”女孩说。

“我会帮你拿回来,”青蛙说,“但你会给我什么作为回报?”

“你想要什么都可以,”女孩说,“我的衣服、我的珍珠和宝石,甚至我戴的金冠。”

“我不想要你的衣服,不想要你的珍珠和宝石,更不想要你的金冠,”青蛙说,“但如果你爱我,把我当作朋友,让我坐在你的桌子旁,让我在你的床上睡觉,愿意给我你拥有的一切,那么我就会帮你拿回你的球。”

“是的,你想要什么都可以,”女孩说,“如果你帮我拿回我的球,我会给你一切。”

但她想:“这个愚蠢的青蛙想要这一切做什么?他坐在水里呱呱叫,不能到我这里来。”

青蛙说:“你答应我吗?”

“是的,我答应你,”女孩说。

然后青蛙沉入水中,游了下去。很快他又回来了,嘴里叼着金球。他把球扔到岸上。

女孩找回了球,非常高兴。她拿起球就跑了。

“等等,等等!”青蛙喊道,“带上我!”

但女孩没有听他的话,而是尽可能快地跑回了城堡。

第二天,当女孩和国王以及其他公主们坐在桌旁吃饭时,有人敲门。

“是谁?”女孩喊道。

“我是你昨天在水池边答应让我坐在你桌子旁的青蛙。给我开门!”

女孩听到青蛙的声音,吓了一跳。她告诉国王发生了什么事。

“你必须遵守你的承诺,”国王说,“给他开门!”

女孩打开门,青蛙跳了进来。他说:“让我坐在你的椅子上!”

女孩犹豫了,但国王说:“你必须遵守你的承诺。”

于是女孩把青蛙抱起来,放在她旁边的椅子上。但青蛙说:“现在让我在你的盘子里吃饭!”

女孩照做了,但她非常不情愿。

“现在让我用你的勺子吃饭!”青蛙说。

女孩照做了,但她讨厌青蛙。

“现在让我坐在你的膝盖上!”青蛙说。

女孩照做了,但她非常不高兴。

“现在让我在你的床上睡觉!”青蛙说。

女孩开始哭泣,因为她害怕冰冷潮湿的青蛙。

但国王说:“你必须遵守你的承诺。如果你帮助了青蛙,你也必须给他你承诺的东西。”

于是女孩把青蛙带进她的房间,放在地板上。但青蛙说:“我不想睡在地板上!让我到你的床上!”

女孩非常生气,她把青蛙捡起来,扔到墙上。“现在你满意了!”她喊道。

但当青蛙撞到墙上时,他变成了一个有着金色头发的美丽王子。他是一个遥远国家的王子,被一个邪恶的女巫变成了青蛙。

“谢谢你,”王子说,“谢谢你把我从诅咒中解放出来。如果你愿意,你可以成为我的妻子,和我一起去我的王国。”

女孩非常高兴,她答应了他。

第二天早上,一辆由六匹白马拉着的马车来接她们。王子和女孩上了车,他们出发了。

在路上,他们看到一个男人在他们后面跑。他喊道:“等等,等等!”

王子让马车停下来,男人走近了。他是海因里希,王子忠实的仆人。当王子被变成青蛙时,海因里希在他的心上缠了三条铁带,以免它因悲伤而破碎。

“我很高兴我的主人重获自由!”海因里希说,“我的铁带现在断了!”

于是他们继续前进,所有人都非常高兴。

王子和女孩结婚了,他们幸福地生活在一起,直到生命的尽头。
}

% 第五篇:睡美人
\chapter{睡美人}
\section{故事内容}

\original{
Dornröschen

Es war einmal ein König und eine Königin, die waren sehr glücklich, aber sie hatten kein Kind. Eines Tages bat die Königin einen Stuhlkreis an, um sich ihrer Sehnsucht nach einem Kind zu befreien.

Eines Tages, als die Königin im Garten spazierte, trafen sie auf eine alte Hexe, die sagte: "Dein Wunsch wird bald erfüllt. Du wirst ein Mädchen bekommen, das so schön sein wird, dass alle es bewundern werden."

Und tatsächlich bekam die Königin bald ein Mädchen. Das Königspaar war so glücklich, dass sie eine große Feier veranstalteten. Sie luden sieben gute Feeen ein, um dem Kind ihre Segnungen zu geben.

Aber es gab acht Feeen im ganzen Land, und sie hatten nur sieben Goldteller mit, also konnte eine Fee nicht eingeladen werden. Sie war sehr böse.

Am Tag der Taufe kamen die sieben guten Feeen. Jede gab dem Mädchen eine Segnung: die erste Segnete es mit Schönheit, die zweite mit Weisheit, die dritte mit Güte, die vierte mit Tanzkunst, die fünfte mit Gesang, die sechste mit Musik und die siebte mit Gesundheit.

Just als die siebte Fee ihre Segnung gegeben hatte, klopfte es an die Tür. Es war die böse Fee. Sie war sehr wütend, weil sie nicht eingeladen worden war. Sie trat herein und rief: "Das Mädchen wird im 15. Jahre an einer Spindel stechen und sterben!"

Die Königstochter weinte, aber die siebte Fee, die ihre Segnung noch nicht gegeben hatte, sagte: "Ich kann den Fluch nicht aufheben, aber ich kann ihn mildern. Das Mädchen wird nicht sterben, sondern nur 100 Jahre schlafen."

Der König war sehr erschrocken. Er befahl, alle Spindeln im ganzen Land zu verbrennen.

Das Mädchen wuchs heran und erhielt alle Segnungen der sieben guten Feeen. Es war sehr schön, weise, gut, konnte tanzen, singen und spielen auf allen Instrumenten.

Am Tag ihrer 15. Geburt war der König und die Königin ausgegangen. Das Mädchen war allein im Schloss. Sie wanderte durch die Räume und kam endlich in einen Turm. Dort saß eine alte Frau, die an einer Spindel spann.

"Was machst du da?" fragte das Mädchen.

"Ich spinne, mein Kind", antwortete die alte Frau.

"Das ist ein schönes Werkzeug", sagte das Mädchen. "Kann ich auch versuchen?"

Die alte Frau gab ihr die Spindel, aber als das Mädchen sie ergriff, stach sie sich an der Spitze. Sofort fiel sie tot zu Boden.

Aber die alte Frau war die böse Fee, die den Fluch verhängt hatte. Und nun war der Fluch in Erfüllung gegangen.

Der König und die Königin kamen zurück und fanden das Mädchen tot. Sie legten es in ein Bett in dem Turm und weinten sehr.

Dann befahl der König, dass alle im Schloss schlafen sollten, bis die 100 Jahre vorbei waren. So schliefen alle: der König, die Königin, die Hofleute, die Tiere, selbst die Fliegen auf der Wand.

Und um das Schloss wuchsen große Dornensträucher, die es ganz umschlossen. Man konnte das Schloss nicht mehr sehen, nur die Spitze des Turms rauchte noch darüber hinaus.

Die Geschichte von dem schlafenden Mädchen verbreitete sich im ganzen Land. Viele Prinzen kamen, um das Mädchen zu retten, aber die Dornensträucher zerrissen ihre Kleider und wundeten sie. Einige starben gar daran.

100 Jahre vergingen. Eines Tages kam ein junger Prinz durch den Wald. Er hörte von der Geschichte und wollte das Mädchen retten.

Als er zum Schloss kam, wuchsen die Dornensträucher zu schönen Rosenblüten auf. Er ging durch sie hindurch, und sie verletzten ihn nicht.

Im Schloss war alles still. Die Hofleute schliefen an ihren Plätzen, die Tiere schliefen im Stall, die Fliegen schliefen auf der Wand.

Der Prinz ging zum Turm und fand das Mädchen im Bett. Es war so schön, dass er es küsste.

Sofort erwachte das Mädchen und der ganze Hof. Der König und die Königin erwachten, die Hofleute erwachten, die Tiere erwachten und die Fliegen flogen davon.

Der Prinz und das Mädchen verliebten sich sofort ineinander und heirateten. Sie lebten glücklich bis ans Ende ihrer Tage.
}

\tr\t\translation{
睡美人

从前有一位国王和王后,他们非常幸福,但没有孩子。有一天,王后举行了一个圆桌会议,以缓解她对孩子的渴望。

一天,当王后在花园里散步时,她遇到了一位老女巫,女巫说:“你的愿望很快就会实现。你将有一个女孩,她会非常美丽,所有人都会钦佩她。”

果然,王后不久就有了一个女孩。国王和王后非常高兴,他们举行了一个盛大的庆祝活动。他们邀请了七位好仙女,为孩子送上她们的祝福。

但全国共有八位仙女,她们只有七个金盘子,所以有一位仙女没有被邀请。她非常生气。

在洗礼当天,七位好仙女来了。每位仙女都给了孩子一个祝福:第一位祝福她美丽,第二位祝福她智慧,第三位祝福她善良,第四位祝福她舞蹈艺术,第五位祝福她歌唱,第六位祝福她音乐,第七位祝福她健康。

就在第七位仙女送上她的祝福时,有人敲门。是那个邪恶的仙女。她非常生气,因为她没有被邀请。她走进来,喊道:“这个女孩在15岁时会被纺锤刺伤而死!”

公主哭了,但第七位仙女还没有送上她的祝福,她说:“我无法解除诅咒,但我可以减轻它。女孩不会死,只会睡100年。”

国王非常震惊。他下令烧毁全国所有的纺锤。

女孩长大了,得到了七位好仙女的所有祝福。她非常美丽、聪明、善良,会跳舞、唱歌,还能演奏所有乐器。

在她15岁生日那天,国王和王后出去了。女孩独自一人在城堡里。她在房间里走来走去,最后来到了一座塔。那里坐着一位老妇人,正在用纺锤纺纱。

“你在做什么?”女孩问。

“我在纺纱,我的孩子,”老妇人回答。

“这是一个美丽的工具,”女孩说。“我也可以试试吗?”

老妇人把纺锤给了她,但当女孩抓住它时,她被尖端刺伤了。她立刻倒在地上死了。

但老妇人是那个施加诅咒的邪恶仙女。现在诅咒已经应验了。

国王和王后回来,发现女孩死了。他们把她放在塔上的床上,哭得很厉害。

然后国王下令,城堡里的所有人都应该睡觉,直到100年过去。于是所有人都睡着了:国王、王后、宫廷侍从、动物,甚至墙上的苍蝇。

城堡周围长出了巨大的荆棘丛,把它完全包围起来。人们再也看不见城堡了,只有塔顶还露在外面。

关于睡美人的故事传遍了全国。许多王子前来拯救女孩,但荆棘丛撕碎了他们的衣服,伤害了他们。有些人甚至因此而死。

100年过去了。有一天,一位年轻的王子穿过森林。他听说了这个故事,想拯救女孩。

当他来到城堡时,荆棘丛变成了美丽的玫瑰花。他穿过它们,它们没有伤害他。

城堡里一片寂静。宫廷侍从在他们的位置上睡着了,动物在马厩里睡着了,苍蝇在墙上睡着了。

王子走上塔楼,发现女孩躺在床上。她非常美丽,王子吻了她。

女孩和整个宫廷立刻醒了过来。国王和王后醒了,宫廷侍从醒了,动物醒了,苍蝇飞走了。

王子和女孩立刻坠入爱河,结婚了。他们幸福地生活在一起,直到生命的尽头。
}

% 第六篇:长发公主
\chapter{长发公主}
\section{故事内容}

\original{
Rapunzel

Es war einmal ein Mann und eine Frau, die wohnten neben einem großen Zaubergarten. Der Zaubergarten gehörte einer bösen Hexe, die sehr stark war. Niemand wagte, in den Garten zu gehen, denn wer darin erwischt wurde, musste sterben.

In dem Garten wuchsen viele schöne Blumen und Kräuter, aber vor allem wuchsen da Rampion, das auch Rapunzel genannt wird. Die Frau hatte einen großen Hunger nach Rampion, so großen, dass sie fast krank wurde.

Einmal sagte sie zu ihrem Mann: "Ich sterbe vor Hunger nach dem Rampion im Zaubergarten. Wenn ich nicht bald davon esse, werde ich sterben."

Der Mann liebte seine Frau sehr, also sagte er: "Ich werde in den Zaubergarten gehen und dir etwas Rampion holen."

Nachts stieg er über den Zaun in den Garten und pflückte einen großen Strauß Rampion. Seine Frau aß ihn, und sie fühlte sich sofort besser.

Aber der Hunger nach Rampion wurde nur größer. Also stieg der Mann noch einmal nachts in den Garten. Aber diesmal wartete die Hexe auf ihn.

"Was tust du hier?" rief die Hexe. "Du weißt, dass wer in meinen Garten kommt, sterben muss!"

Der Mann bat um Vergebung und erzählte der Hexe von seinem Problem.

Die Hexe sagte: "Ich werde dir vergeben, aber du musst mir das Kind geben, das deine Frau bald bekommen wird. Ich werde es wie meine eigene Tochter aufziehen."

Der Mann hatte keine andere Wahl, als zuzustimmen.

Bald darauf bekam die Frau ein Mädchen, das war so schön wie der Sonnenschein. Die Hexe kam und nahm das Mädchen weg. Sie nannte es Rapunzel, nach dem Kräuter, das die Mutter so sehr geliebt hatte.

Die Hexe baute für Rapunzel ein Schloss in einem hohen Turm. Der Turm hatte keine Treppe und keine Türe, nur ein kleines Fenster ganz oben.

Rapunzel wuchs zu einer sehr schönen Jungfrau mit langen, goldenen Haaren, die bis zum Boden reichten.

Jeden Tag rief die Hexe: "Rapunzel, Rapunzel, lass dein Haar herunter!"

Dann warf Rapunzel ihre langen Haare aus dem Fenster, und die Hexe kletterte daran hinauf.

Eines Tages kam ein junger Prinz durch den Wald. Er hörte eine schöne Stimme singen. Es war Rapunzel, die im Turm sang.

Der Prinz suchte den Turm, aber er fand keine Türe. Er saß jeden Tag unter dem Turm und hörte Rapunzel singen.

Eines Tages sah er die Hexe, die rief: "Rapunzel, Rapunzel, lass dein Haar herunter!"

Und Rapunzel warf ihre Haare aus dem Fenster, und die Hexe kletterte daran hinauf.

Der Prinz dachte: "Ich werde das auch tun."

Am nächsten Tag rief er: "Rapunzel, Rapunzel, lass dein Haar herunter!"

Rapunzel warf ihre Haare aus dem Fenster, und der Prinz kletterte daran hinauf.

Als Rapunzel den Prinzen sah, war sie erschrocken, aber der Prinz sprach zu ihr mit sanfter Stimme. Er erzählte ihr, wie er zu ihr gefunden hatte, und dass er sich in sie verliebt hatte.

Rapunzel verliebte sich auch in den Prinzen, und sie versprach, ihn zu heiraten.

Aber eines Tages verriet Rapunzel der Hexe, dass sie den Prinzen kannte. Die Hexe war sehr wütend. Sie schnitt Rapunzels lange Haare ab und warf sie in die Wüste.

Dann führte sie Rapunzel in ein wildes Land, wo sie elendig leben musste.

Am nächsten Tag rief der Prinz wieder: "Rapunzel, Rapunzel, lass dein Haar herunter!"

Die Hexe warf die abgeschnittenen Haare aus dem Fenster, und der Prinz kletterte daran hinauf.

Als er die Hexe sah, war er erschrocken. Die Hexe sagte zu ihm: "Rapunzel ist nicht mehr hier! Du wirst sie nie wieder sehen!"

Der Prinz war so traurig, dass er aus dem Fenster sprang. Er überlebte den Sturz, aber er war blind geworden.

Er wanderte für Jahre durch die Wüsten und die Wälder, bis er eines Tages in das wilde Land kam, wo Rapunzel lebte.

Rapunzel hörte seinen Schritt und rief ihn an. Als sie sich sahen, weinte Rapunzel. Ihre Tränen fielen in seine Augen, und er konnte wieder sehen.

Der Prinz führte Rapunzel zurück zu seinem Königreich, und sie heirateten in großer Pracht.

Die böse Hexe wurde nie wieder gesehen. Rapunzel und der Prinz lebten glücklich bis ans Ende ihrer Tage.
}

\tr\t\translation{
长发公主

从前有一个男人和一个女人,他们住在一个大魔法花园旁边。魔法花园属于一个邪恶的女巫,她非常强大。没有人敢走进花园,因为任何被抓住的人都必须死。

花园里长着许多美丽的花朵和草药,但最主要的是长着一种叫做Rampion的植物,也叫长发公主。女人非常想吃这种植物,以至于她几乎生病了。

有一次,她对丈夫说:“我快饿死了,想吃魔法花园里的Rampion。如果我不尽快吃一些,我会死的。”

男人非常爱他的妻子,所以他说:“我会去魔法花园给你摘一些Rampion。”

晚上,他翻过墙进入花园,摘了一大束Rampion。他的妻子吃了它,立刻感觉好多了。

但是对Rampion的渴望变得更加强烈。于是男人再次在晚上进入花园。但这次女巫在等他。

“你在这里做什么?”女巫喊道。“你知道谁进入我的花园,谁就必须死!”

男人请求原谅,并向女巫讲述了他的问题。

女巫说:“我会原谅你,但你必须把你妻子即将生下的孩子给我。我会像抚养自己的女儿一样抚养它。”

男人别无选择,只能同意。

不久之后,女人生了一个女孩,她像阳光一样美丽。女巫来了,把女孩带走了。她给她取名叫长发公主,以纪念母亲非常喜欢的那种植物。

女巫为长发公主在一座高塔里建造了一座城堡。塔没有楼梯,没有门,只有一个很小的窗户在最上面。

长发公主长成了一个非常美丽的少女,有着长长的金色头发,一直垂到地上。

每天女巫都会喊:“长发公主,长发公主,放下你的头发!”

然后长发公主把她的长发从窗户扔出去,女巫顺着头发爬上去。

有一天,一位年轻的王子穿过森林。他听到一个美丽的声音在唱歌。是长发公主在塔里唱歌。

王子寻找塔,但他找不到门。他每天坐在塔下,听长发公主唱歌。

有一天,他看到女巫喊:“长发公主,长发公主,放下你的头发!”

长发公主把她的头发从窗户扔出去,女巫顺着头发爬上去。

王子想:“我也会这样做。”

第二天,他喊:“长发公主,长发公主,放下你的头发!”

长发公主把她的头发从窗户扔出去,王子顺着头发爬上去。

当长发公主看到王子时,她很害怕,但王子用温柔的声音对她说话。他告诉她他是如何找到她的,以及他是如何爱上她的。

长发公主也爱上了王子,她答应嫁给他。

但是有一天,长发公主向女巫透露她认识王子。女巫非常生气。她剪掉了长发公主的长发,把她扔进了沙漠。

然后她把长发公主带到一片荒野,让她在那里过着悲惨的生活。

第二天,王子又喊:“长发公主,长发公主,放下你的头发!”

女巫把剪下来的头发从窗户扔出去,王子顺着头发爬上去。

当他看到女巫时,他很震惊。女巫对他说:“长发公主不在这里了!你永远不会再见到她了!”

王子非常伤心,从窗户跳了下去。他活了下来,但失明了。

他在沙漠和森林中游荡了多年,直到有一天他来到了长发公主居住的荒野。

长发公主听到了他的脚步声,叫他。当他们见面时,长发公主哭了。她的眼泪落在他的眼睛里,他又能看见了。

王子带着长发公主回到了他的王国,他们举行了盛大的婚礼。

邪恶的女巫再也没有被见过。长发公主和王子幸福地生活在一起,直到生命的尽头。
}

% 第七篇:汉塞尔与格莱特
\chapter{汉塞尔与格莱特}
\section{故事内容}

\original{
Hänsel und Gretel

Es war einmal ein armer Holzhacker, der hatte eine Frau und zwei Kinder, Hänsel und Gretel. Die Kinder waren sehr arm, und es gab nicht genug zu essen.

Eines Abends sagte die Frau zum Holzhacker: "Wir haben zu wenig zu essen. Die Kinder müssen weg. Morgen früh gehen wir mit ihnen in den Wald und lassen sie dort allein."

Der Holzhacker sagte: "Nein, ich kann nicht. Die Kinder sind meine eigenen."

Aber die Frau wurde wütend und sagte: "Wenn du nicht willst, dann müssen wir alle sterben."

Der Holzhacker hatte keine andere Wahl, als zuzustimmen.

Hänsel und Gretel hörten die Unterredung. Hänsel sagte zu Gretel: "Keine Sorge, ich habe einen Plan."

Nachts, als alle schliefen, stand Hänsel auf und sammelte viele weiße Steine. Er legte sie in seine Tasche.

Am nächsten Morgen gingen sie in den Wald. Hänsel legte dabei immer wieder Steine auf den Weg.

Als sie tief im Wald waren, baute der Holzhacker ein Feuer und sagte: "Wartet hier, wir gehen weiter und holen Holz. Wir kommen bald zurück."

Aber sie kamen nie zurück. Hänsel und Gretel warten bis zum Abend, aber es wurde dunkel.

Gretel weinte, aber Hänsel sagte: "Keine Angst, folge mir."

Er folgte den Steinen zurück zum Haus. Die Eltern waren sehr erstaunt, als sie die Kinder wieder sahen.

Aber die Not wurde immer größer. Einige Wochen später sagte die Frau wieder: "Wir müssen die Kinder wegbringen. Dieses Mal werden wir sie tiefer im Wald lassen."

Der Holzhacker sagte wieder: "Nein, ich kann nicht."

Aber die Frau wurde wütend und schrie: "Wenn du nicht willst, dann sterben wir alle."

Wieder musste der Holzhacker zustimmen.

Hänsel und Gretel hörten die Unterredung wieder. Hänsel sagte zu Gretel: "Ich werde wieder einen Plan machen."

Aber dieses Mal sperrte die Frau die Vorräte, so dass Hänsel keine Steine sammeln konnte. Er musste sich mit Brot zufrieden geben.

Nachts, als alle schliefen, stand Hänsel auf und zerkrümelte sein Brot. Er legte die Krümel in seine Tasche.

Am nächsten Morgen gingen sie wieder in den Wald. Hänsel legte dabei immer wieder Brotkrümel auf den Weg.

Als sie tief im Wald waren, baute der Holzhacker wieder ein Feuer und sagte: "Wartet hier, wir kommen bald zurück."

Aber sie kamen nie zurück. Hänsel und Gretel warten bis zum Abend, aber es wurde dunkel.

Hänsel sagte: "Keine Angst, folge den Brotkrümeln."

Aber als sie den Weg suchten, waren die Brotkrümel verschwunden. Die Vögel hatten sie gefressen.

Hänsel und Gretel wanderten durch den Wald, bis sie einen Haus aus Brot und Zucker fanden. Das Dach war aus Zimt, die Fenster aus Zuckerwatte.

"Oh, wie schön!" rief Gretel. "Wir können davon essen!"

Sie begannen, am Haus zu knabbern. Plötzlich kam eine alte Frau aus dem Haus. Sie sagte: "Kinder, komm herein. Ich habe für euch etwas zu essen."

Hänsel und Gretel gingen herein. Die alte Frau war eine böse Hexe. Sie wollte die Kinder füttern und dann essen.

Sie sperrte Hänsel in einen Käfig und sagte zu Gretel: "Du musst kochen und backen, damit Hänsel dick wird. Dann werde ich ihn essen."

Jeden Tag fragte die Hexe: "Ist Hänsel schon dick genug?"

Aber Hänsel hielt immer ein Knochen aus dem Käfig heraus, so dass die Hexe dachte, er sei noch zu dünn.

Nach einiger Zeit wurde die Hexe ungeduldig. Sie sagte zu Gretel: "Heute werde ich Hänsel essen. Mach das Feuer heiß."

Gretel musste tun, was die Hexe sagte. Als das Feuer heiß war, sagte die Hexe: "Komm, ich will sehen, ob das Feuer heiß genug ist."

Gretel sagte: "Ich kann es nicht sehen. Komm her und schau selbst."

Die Hexe stellte sich vor das Feuer. Da schob Gretel sie ins Feuer und schloss die Tür.

Die Hexe brannte zu Asche.

Gretel öffnete den Käfig und befreite Hänsel. Zusammen suchten sie im Haus nach Schätzen. Sie fanden Gold und Silber und Edelsteine.

Sie trugen die Schätze und gingen weiter durch den Wald. Plötzlich traten sie auf einen Weg, der sie nach Hause führte.

Die Eltern waren sehr glücklich, als sie die Kinder wieder sahen. Der Holzhacker sagte: "Ich werde nie wieder zulassen, dass ihr weggeht."

Die Familie lebte nun glücklich und reich, denn die Schätze reichten für immer.

Und damit war das Märchen zu Ende.
}

\tr\t\translation{
汉塞尔与格莱特

从前有一个贫穷的樵夫,他有一个妻子和两个孩子,汉塞尔和格莱特。孩子们非常贫穷,没有足够的食物吃。

一天晚上,妻子对樵夫说:“我们没有足够的食物。孩子们必须离开。明天早上我们带他们去森林,把他们留在那里。”

樵夫说:“不,我不能。孩子们是我自己的。”

但是妻子生气了,说:“如果你不愿意,我们都得死。”

樵夫别无选择,只能同意。

汉塞尔和格莱特听到了谈话。汉塞尔对格莱特说:“别担心,我有一个计划。”

晚上,当所有人都睡着了,汉塞尔起床收集了许多白色的石头。他把它们放在他的口袋里。

第二天早上,他们去了森林。汉塞尔在路上不断地放下石头。

当他们深入森林时,樵夫生了一堆火,说:“在这里等着,我们继续去砍柴。我们很快就回来。”

但他们再也没有回来。汉塞尔和格莱特一直等到晚上,但天黑了。

格莱特哭了,但汉塞尔说:“别害怕,跟着我。”

他跟着石头回到了家。当父母再次看到孩子们时,他们非常惊讶。

但困境越来越严重。几周后,妻子再次说:“我们必须把孩子们送走。这次我们会把他们留在森林深处。”

樵夫再次说:“不,我不能。”

但是妻子生气了,喊道:“如果你不愿意,我们都得死。”

樵夫又不得不同意。

汉塞尔和格莱特又听到了谈话。汉塞尔对格莱特说:“我会再制定一个计划。”

但这次妻子锁上了食物,所以汉塞尔无法收集石头。他只能用面包来凑合。

晚上,当所有人都睡着了,汉塞尔起床,把他的面包弄碎。他把面包屑放在他的口袋里。

第二天早上,他们再次去了森林。汉塞尔在路上不断地放下面包屑。

当他们深入森林时,樵夫又生了一堆火,说:“在这里等着,我们很快就回来。”

但他们再也没有回来。汉塞尔和格莱特一直等到晚上,但天黑了。

汉塞尔说:“别害怕,跟着面包屑走。”

但当他们寻找路时,面包屑已经消失了。鸟儿们已经把它们吃掉了。

汉塞尔和格莱特在森林里走着,直到他们发现了一座用面包和糖做的房子。屋顶是肉桂做的,窗户是棉花糖做的。

“哦,多漂亮啊!”格莱特喊道。“我们可以吃它!”

他们开始啃房子。突然,一个老妇人从房子里出来。她说:“孩子们,进来吧。我有东西给你们吃。”

汉塞尔和格莱特走了进去。老妇人是一个邪恶的女巫。她想把孩子们喂胖然后吃掉。

她把汉塞尔锁在一个笼子里,对格莱特说:“你必须做饭和烤面包,让汉塞尔变胖。然后我会吃掉他。”

每天女巫都会问:“汉塞尔已经够胖了吗?”

但汉塞尔总是从笼子里拿出一根骨头,这样女巫就以为他还太瘦了。

过了一段时间,女巫变得不耐烦了。她对格莱特说:“今天我要吃汉塞尔。把火弄热。”

格莱特不得不按照女巫说的去做。当火变热时,女巫说:“过来,我想看看火是否足够热。”

格莱特说:“我看不见。过来自己看看。”

女巫站在火前。这时格莱特把她推到火里,关上了门。

女巫被烧成了灰烬。

格莱特打开笼子,释放了汉塞尔。他们一起在房子里寻找宝藏。他们找到了黄金、白银和宝石。

他们带着宝藏继续穿过森林。突然,他们走上了一条通往家的路。

当父母再次看到孩子们时,他们非常高兴。樵夫说:“我再也不会让你们离开了。”

这个家庭现在幸福而富有地生活着,因为宝藏永远用不完。

故事就这样结束了。
}

% 第八篇:不莱梅的城市乐手
\chapter{不莱梅的城市乐手}
\section{故事内容}

\original{
Die Bremer Stadtmusikanten

Es war einmal ein alter Hund, der war so alt, dass er nicht mehr jagten konnte. Sein Herr sagte zu ihm: "Du bist zu alt. Ich werde dich wegschicken. Geh weg!"

Der Hund war sehr traurig. Er dachte: "Ich muss einen neuen Beruf finden. Ich werde nach Bremen gehen und Stadtmusikant werden."

Er machte sich auf den Weg nach Bremen. Auf dem Weg traf er einen alten Hund. Der Hund sagte: "Warum bist du so traurig?"

Der erste Hund erzählte seine Geschichte. Der zweite Hund sagte: "Ich bin auch alt. Mein Herr will mich wegschicken. Ich werde mit dir gehen und Stadtmusikant werden."

Sie gingen weiter zusammen. Bald trafen sie einen alten Kater. Der Kater sagte: "Warum seid ihr so traurig?"

Die Hunde erzählten ihre Geschichte. Der Kater sagte: "Ich bin auch alt. Ich kann nicht mehr maus fangen. Mein Herr will mich wegschicken. Ich werde mit euch gehen und Stadtmusikant werden."

Sie gingen weiter zusammen. Bald trafen sie eine alte Henne. Die Henne sagte: "Warum seid ihr so traurig?"

Die anderen Tiere erzählten ihre Geschichte. Die Henne sagte: "Ich bin auch alt. Ich kann nicht mehr Eier legen. Mein Herr will mich schlachten. Ich werde mit euch gehen und Stadtmusikant werden."

Sie gingen weiter zusammen. Am Abend kamen sie zu einem Haus. Das Haus gehörte einem Räuber.

Sie sahen, dass das Haus beleuchtet war und dass es Geräusche gab.

Der Hund sagte: "Ich habe einen Plan. Wir wollen die Räuber vertreiben."

Der Hund stand auf die Hinterbeine, der Bube auf den Hund, der Kater auf den Bube und die Henne auf den Kater.

Dann begannen sie zu singen: Der Hund bellte, der Bube heulte, der Kater miaute und die Henne krähte.

Es war ein schreckliches Geräusch. Die Räuber dachten, es wäre ein Ungeheuer, und liefen weg.

Die Tiere gingen ins Haus und aßen das Essen, das die Räuber vorbereitet hatten.

Später schliefen sie ein: Der Hund auf dem Ofen, der Bube hinter der Tür, der Kater auf dem Herd und die Henne auf den Eierkisten.

In der Nacht kam ein Räuber zurück. Er wollte prüfen, ob das Haus sicher war.

Er hielt eine Kerze an und sah den Hund auf dem Ofen. Der Hund bellte und biss ihn in die Hand.

Der Räuber rannte zur Tür. Der Bube sprang hinter die Tür und biss ihn in die Beine.

Der Räuber rannte zum Herd. Der Kater sprang auf den Herd und kratzte sein Gesicht.

Der Räuber rannte zum Fenster. Die Henne krahte und flatterte über seinen Kopf.

Der Räuber war so erschrocken, dass er weglief und sagte: "Im Haus wohnt ein Ungeheuer, das mit den Krallen kratzt, mit den Zähnen beißt, mit dem Mund heult und mit der Stimme kraht!"

Die Räuber kamen nie wieder zurück. Die Tiere blieben im Haus und lebten glücklich zusammen.

Und so wurden sie die berühmten Bremer Stadtmusikanten.
}

\tr\t\translation{
不莱梅的城市乐手

从前有一只老狗,它已经很老了,不能再打猎了。它的主人对它说:“你太老了。我要把你送走。走吧!”

狗非常伤心。它想:“我必须找到一个新的职业。我要去不莱梅,当城市音乐家。”

它出发去不莱梅。在路上,它遇到了一只老狗。这只狗说:“你为什么这么伤心?”

第一只狗讲述了它的故事。第二只狗说:“我也老了。我的主人想把我送走。我要和你一起去,当城市音乐家。”

它们一起继续走。不久,它们遇到了一只老猫。猫说:“你们为什么这么伤心?”

狗们讲述了它们的故事。猫说:“我也老了。我不能再抓老鼠了。我的主人想把我送走。我要和你们一起去,当城市音乐家。”

它们一起继续走。不久,它们遇到了一只老母鸡。母鸡说:“你们为什么这么伤心?”

其他动物讲述了它们的故事。母鸡说:“我也老了。我不能再下蛋了。我的主人想杀我。我要和你们一起去,当城市音乐家。”

它们一起继续走。晚上,它们来到了一所房子前。这所房子属于一个强盗。

它们看到房子里亮着灯,还有声音。

狗说:“我有一个计划。我们要把强盗赶走。”

狗用后腿站起来,驴站在狗身上,猫站在驴身上,母鸡站在猫身上。

然后它们开始唱歌:狗汪汪叫,驴嘶叫,猫喵喵叫,母鸡咯咯叫。

这是一种可怕的声音。强盗们以为是怪物,就跑开了。

动物们走进房子,吃了强盗们准备的食物。

后来它们睡着了:狗在炉子上,驴在门后,猫在炉灶上,母鸡在蛋箱上。

晚上,一个强盗回来了。他想检查房子是否安全。

他点燃一支蜡烛,看到狗在炉子上。狗汪汪叫,咬了他的手。

强盗跑到门口。驴从门后跳出来,咬了他的腿。

强盗跑到炉灶前。猫跳到炉灶上,抓了他的脸。

强盗跑到窗户前。母鸡咯咯叫,从他头顶飞过。

强盗非常害怕,跑了出去,说:“房子里住着一个怪物,它用爪子抓,用牙齿咬,用嘴巴叫,用声音咯咯叫!”

强盗们再也没有回来。动物们留在房子里,幸福地生活在一起。

就这样,它们成了著名的不莱梅城市乐手。
}

% 第九篇:侏儒妖
\chapter{侏儒妖}
\section{故事内容}

\original{
Rumpelstilzchen

Es war einmal ein Müller, der hatte eine schöne Tochter. Er war sehr arm, aber er gab sich vor, reich zu sein.

Eines Tages traf er den König und sagte: "Meine Tochter kann Stroh zu Gold spinnen."

Der König war sehr erstaunt und sagte: "Bring deine Tochter morgen zu mir. Ich werde ihr ein Zimmer voll Stroh geben, und wenn sie es in einer Nacht zu Gold spinnt, soll sie meine Königin werden."

Die Müllerstochter weinte sehr, als sie das hörte, denn sie konnte Stroh nicht zu Gold spinnen.

Aber als sie allein im Zimmer war, kam ein kleiner Mann und fragte: "Warum weinst du, Mädel?"

"Ich muss Stroh zu Gold spinnen," sagte sie, "sonst sterbe ich."

"Ich werde dir helfen," sagte der Mann, "wenn du mir etwas gibst."

"Was willst du?" fragte die Mädchen.

"Dein kleines Ring," sagte der Mann.

Die Mädchen gab ihm den Ring, und der Mann spann das Stroh in einer Nacht zu Gold.

Der König war sehr erstaunt, als er das Gold sah. Aber er war gierig und wollte mehr. Er gab der Mädchen ein größeres Zimmer voll Stroh und sagte: "Wenn du dies in einer Nacht zu Gold spinnst, sollst du meine Königin werden."

Die Mädchen weinte wieder, und wieder kam der kleine Mann. "Was willst du diesmal?" fragte er.

"Meine Kette," sagte die Mädchen.

Die Mädchen gab ihm die Kette, und der Mann spann das Stroh wieder zu Gold.

Der König war noch gieriger und sagte: "Heute Nacht sollst du in einem größeren Zimmer spinnen, das bis zum Rand mit Stroh gefüllt ist. Wenn du dies zu Gold spinnst, wirst du meine Königin, sonst wirst du hingerichtet."

Die Mädchen weinte sehr, und wieder kam der kleine Mann. "Was willst du diesmal?" fragte er.

"Ich habe nichts mehr zu geben," sagte die Mädchen.

"Dann sollst du mir dein erstes Kind geben, wenn du Königin bist," sagte der Mann.

Die Mädchen versprach es ihm, denn sie dachte, er werde es nicht einfordern, und der Mann spann das Stroh zu Gold.

Der König heiratete die Mädchen, und sie wurde Königin. Nach einiger Zeit hatte sie ein Kind.

Da kam der kleine Mann und verlangte das Kind. Die Königin weinte sehr und bat ihn, es zu lassen.

"Ich will dir ein Jahr Zeit geben," sagte der Mann, "um meinen Namen zu erraten. Wenn du ihn in dieser Zeit errätst, sollst du das Kind behalten."

Die Königin schickte Boten aus, um den Namen des Mannes herauszufinden. Sie fragten alle Menschen, die sie trafen, nach ihrem Namen.

Am letzten Tag des Jahres kam der kleine Mann wieder. "Hast du meinen Namen erraten?" fragte er.

"Ist dein Name Hans?" fragte die Königin.

"Nein," sagte der Mann.

"Ist dein Name Fritz?" fragte die Königin.

"Nein," sagte der Mann.

"Ist dein Name Rumpelstilzchen?" fragte die Königin.

Da wurde der kleine Mann sehr wütend und schrie: "Du hast meinen Namen erraten!" Und dann stieß er mit dem Füß auf den Boden, dass er bis zum Leib hinein sank. Und das war das Ende von Rumpelstilzchen.
}

\tr\t\translation{
侏儒妖

从前有一个磨坊主,他有一个美丽的女儿。他很穷,但他假装自己很富有。

有一天,他遇到了国王,说:“我的女儿可以把稻草纺成金子。”

国王非常惊讶,说:“明天把你的女儿带来见我。我会给她一个装满稻草的房间,如果她能在一夜之间把它纺成金子,她将成为我的王后。”

磨坊主的女儿听到这话后哭得很厉害,因为她不能把稻草纺成金子。

但当她一个人在房间里时,来了一个小矮人,问:“你为什么哭,小姑娘?”

“我必须把稻草纺成金子,”她说,“否则我就会死。”

“我会帮你,”小矮人说,“如果你给我一些东西。”

“你想要什么?”女孩问。

“你的小戒指,”小矮人说。

女孩给了他戒指,小矮人在一夜之间把稻草纺成了金子。

国王看到金子时非常惊讶。但他很贪婪,想要更多。他给了女孩一个更大的装满稻草的房间,说:“如果你能在一夜之间把它纺成金子,你将成为我的王后。”

女孩又哭了,小矮人又出现了。“这次你想要什么?”他问。

“我的项链,”女孩说。

女孩给了他项链,小矮人又把稻草纺成了金子。

国王更加贪婪,说:“今晚你要在一个更大的房间里纺,这个房间装满了稻草,直到边缘。如果你把它纺成金子,你将成为我的王后,否则你将被处决。”

女孩哭得很厉害,小矮人又出现了。“这次你想要什么?”他问。

“我没有什么可给的了,”女孩说。

“那么当你成为王后时,你要把你的第一个孩子给我,”小矮人说。

女孩答应了他,因为她认为他不会要求的,小矮人把稻草纺成了金子。

国王娶了女孩,她成为了王后。过了一段时间,她有了一个孩子。

这时小矮人来了,要求得到孩子。王后哭得很厉害,恳求他不要拿走。

“我给你一年的时间,”小矮人说,“来猜我的名字。如果你在这段时间内猜对了,你可以留下孩子。”

王后派出使者去找出小矮人的名字。他们问所有遇到的人,他们的名字是什么。

在一年的最后一天,小矮人又来了。“你猜出我的名字了吗?”他问。

“你的名字是汉斯吗?”王后问。

“不,”小矮人说。

“你的名字是弗里茨吗?”王后问。

“不,”小矮人说。

“你的名字是侏儒妖吗?”王后问。

小矮人非常生气,喊道:“你猜对了我的名字!”然后他用脚踩在地上,一直陷到腰部。侏儒妖就这样结束了。
}

% 第十篇:勇敢的小裁缝
\chapter{勇敢的小裁缝}
\section{故事内容}

\original{
Der tapfere Schneiderlein

Es war einmal ein Schneiderlein, der war klein aber tapfer. Eines Morgens früh machte er sich auf den Weg, um sein Glück zu suchen.

Vor dem Haus hatte er einen Weizenkornkuchen gebacken. Er steckte ihn in seine Tasche und machte sich auf den Weg.

Auf dem Weg traf er eine Riesin. Die Riesin fragte: "Was hast du in deiner Tasche?"

Der Schneiderlein sagte: "Weizenkornkuchen."

Die Riesin sagte: "Ich will ihn haben."

Der Schneiderlein sagte: "Komm mit mir, ich gebe dir etwas."

Er führte die Riesin zu einem Baum und sagte: "Kletter auf den Baum, da ist mehr Kuchen."

Die Riesin kletterte auf den Baum, aber der Schneiderlein schnitt die Ranken ab, so dass die Riesin stürzte und starb.

Dann ging der Schneiderlein weiter und kam zu einem König. Der König hatte eine große Aufgabe für ihn: Er musste einen wilden Drachen töten.

Der Schneiderlein sagte: "Ich will es tun."

Er ging zum Drachen und sagte: "Komm mit mir, ich habe etwas für dich."

Er führte den Drachen zu einem See und sagte: "Trink aus dem See, da ist Wein."

Der Drachen trank so viel, dass er schlief. Dann tötete der Schneiderlein den Drachen.

Der König war sehr erstaunt und machte den Schneiderlein zu seinem Helden.

Aber der König hatte Angst vor dem kleinen Schneiderlein und wollte ihn loswerden.

Er sagte: "Du musst in einem Turm schlafen, der ist verhext."

Der Schneiderlein sagte: "Ich will es tun."

Im Turm gab es viele Geister, aber der Schneiderlein war so tapfer, dass er sie alle vertrieb.

Am nächsten Morgen kam der König und war sehr erstaunt, dass der Schneiderlein am Leben war.

Er sagte: "Du bist ein tapferer Mann. Du sollst meine Tochter heiraten und nach meinem Tod das Königreich erben."

Der Schneiderlein heiratete die Königstochter und lebte glücklich bis ans Ende seiner Tage.
}

\translation{
勇敢的小裁缝

从前有一个小裁缝,他个子很小,但很勇敢。一天清晨,他出发去寻找自己的幸福。

出门前,他烤了一个小麦面包。他把面包放进袋子里,就出发了。

在路上,他遇到了一个巨人。巨人问:“你的袋子里有什么?”

小裁缝说:“小麦面包。”

巨人说:“我想要它。”

小裁缝说:“跟我来,我给你一些。”

他把巨人带到一棵树上,说:“爬上树,上面有更多的面包。”

巨人爬上了树,但小裁缝砍断了树枝,巨人掉下来摔死了。

然后小裁缝继续走,来到了一个国王的宫殿。国王给他布置了一个大任务:他必须杀死一条凶猛的龙。

小裁缝说:“我愿意去做。”

他走到龙的面前,说:“跟我来,我有东西给你。”

他把龙带到一个湖边,说:“从湖里喝水,那里有酒。”

龙喝了很多,然后睡着了。然后小裁缝杀死了龙。

国王非常惊讶,把小裁缝当作英雄。

但国王害怕小裁缝,想摆脱他。

他说:“你必须在一个被诅咒的塔里睡觉。”

小裁缝说:“我愿意去做。”

塔里有很多幽灵,但小裁缝非常勇敢,把它们都赶走了。

第二天早上,国王来了,非常惊讶小裁缝还活着。

他说:“你是一个勇敢的人。你应该娶我的女儿,在我死后继承王国。”

小裁缝娶了公主,幸福地生活到生命的尽头。
}

% 第十一篇:大拇指汤姆
\chapter{大拇指汤姆}
\section{故事内容}

\original{
Tom Thumb

Es war einmal ein armer Fischer, der hatte keine Kinder. Eines Tages bat er die Göttin: "Gebt mir ein Kind, auch wenn es nur so groß wie mein Daumen ist."

Die Göttin hörte ihn und bald bekam seine Frau ein kleines Kind, das wirklich nur so groß wie ein Daumen war. Sie nannten es Tom Thumb.

Tom Thumb war sehr klug und tapfer, obwohl er so klein war. Eines Tages sagte der Vater: "Ich muss in den Wald gehen, um Holz zu holen."

Tom Thumb sagte: "Ich will mit dir gehen."

"Aber du bist zu klein," sagte der Vater. "Du wirst verloren gehen."

"Keine Sorge," sagte Tom Thumb. "Ich werde in deiner Tasche sitzen."

Der Vater nahm Tom Thumb mit und ging in den Wald. Tom Thumb saß in der Tasche und sah sich um.

Plötzlich rief Tom Thumb: "Halt! Da ist ein großer Fuchs!"

Der Vater erschrak und lief weg. Aber der Fuchs hatte Tom Thumb gesehen und wollte ihn fressen.

Tom Thumb sprang aus der Tasche und versteckte sich unter einem Blatt.

Der Fuchs suchte nach ihm, aber er konnte ihn nicht finden.

Später kam Tom Thumb zu einem König. Der König war sehr erstaunt über das kleine Kind.

Er sagte: "Ich will dich haben. Du sollst mein Kind sein."

Tom Thumb sagte: "Ich will, aber ich muss zuerst meinen Vater besuchen."

Der König gab ihm Geld und einen Wagen, damit er zu seinem Vater gehen konnte.

Tom Thumb fuhr mit dem Wagen zu seinem Vater und sagte: "Ich bin reich geworden. Komm mit mir zum König."

Der Vater ging mit ihm zum König und der König machte ihn zu seinem Hofmann.

Tom Thumb lebte glücklich bei dem König und seiner Familie bis ans Ende seiner Tage.
}

\translation{
大拇指汤姆

从前有一个贫穷的渔夫,他没有孩子。一天,他向女神祈祷:“请给我一个孩子,哪怕他只有我的大拇指那么大。”

女神听到了他的祈祷,不久他的妻子生了一个小男孩,真的只有大拇指那么大。他们给他取名叫汤姆。

尽管汤姆很小,但他非常聪明和勇敢。有一天,父亲说:“我必须去森林里砍柴。”

汤姆说:“我要和你一起去。”

“但你太小了,”父亲说。“你会迷路的。”

“别担心,”汤姆说。“我会坐在你的口袋里。”

父亲带着汤姆去了森林。汤姆坐在口袋里,四处张望。

突然,汤姆喊道:“停下!那里有一只大狐狸!”

父亲吓了一跳,跑开了。但狐狸看到了汤姆,想吃掉他。

汤姆从口袋里跳出来,躲在一片叶子下面。

狐狸找他,但找不到。

后来,汤姆来到了一个国王的宫殿。国王对这个小不点非常惊讶。

他说:“我想要你。你将成为我的孩子。”

汤姆说:“我愿意,但我必须先去看望我的父亲。”

国王给了他钱和一辆马车,让他去看望父亲。

汤姆坐着马车去了父亲那里,说:“我变得富有了。跟我去见国王吧。”

父亲跟着他去见国王,国王让他做了宫廷侍从。

汤姆在国王和他的家人身边幸福地生活到生命的尽头。
}

% 第十二篇:狐狸和猫
\chapter{狐狸和猫}
\section{故事内容}

\original{
Der Fuchs und die Katze

Es war einmal ein Fuchs und eine Katze, die unterwegs waren. Der Fuchs sagte zur Katze: "Ich habe hundert Tricks, um aus Gefahr zu kommen. Wie viele hast du?"

Die Katze sagte: "Nur einen, aber es ist ein guter."

Der Fuchs lachte und sagte: "Hundert Tricks sind besser als einer."

Plötzlich sah der Fuchs einen Jäger mit Hunden. Er sagte zur Katze: "Jetzt zeige ich dir meine Tricks."

Er suchte nach einem Versteck, aber er konnte keins finden. Er kletterte auf einen Baum, aber die Hunde konnten ihn dort finden.

Der Fuchs sprang von dem Baum und rannte weiter, aber die Hunde verfolgten ihn.

Plötzlich fiel der Fuchs in ein Loch. Die Hunde kamen und bissen ihn.

Die Katze sah das und sagte zu sich: "Mein Trick ist besser. Ich klettere auf einen hohen Baum und bleibe dort, bis die Gefahr vorbei ist."

Und so taten sie. Die Katze kletterte auf einen hohen Baum und wartete, bis die Hunde und der Jäger weg waren.

Später kam die Katze zu dem Fuchs, der verletzt lag. Der Fuchs sagte: "Ich habe hundert Tricks, aber keiner hat mir geholfen. Dein einziger Trick war besser als alle meine."

Der Fuchs starb bald danach, aber die Katze lebte lange und glücklich.
}

\translation{
狐狸和猫

从前有一只狐狸和一只猫,它们一起旅行。狐狸对猫说:“我有一百种办法从危险中逃脱。你有多少种?”

猫说:“只有一种,但它很管用。”

狐狸笑了,说:“一百种办法比一种好。”

突然,狐狸看到一个带着狗的猎人。他对猫说:“现在我给你看看我的办法。”

他寻找藏身之处,但找不到。他爬上一棵树,但狗可以在那里找到他。

狐狸从树上跳下来,继续跑,但狗追着他。

突然,狐狸掉进了一个洞里。狗来了,咬了他。

猫看到了,自言自语地说:“我的办法更好。我爬上一棵高高的树,待在那里,直到危险过去。”

她就这样做了。猫爬上一棵高高的树,等待着,直到狗和猎人离开。

后来,猫来到受伤的狐狸身边。狐狸说:“我有一百种办法,但没有一种帮助我。你唯一的办法比我所有的都好。”

狐狸不久后就死了,但猫活得很长很幸福。
}

% 第十三篇:猫和老鼠做朋友
\chapter{猫和老鼠做朋友}
\section{故事内容}

\original{
Die Katze und die Maus in Gesellschaft

Es war einmal eine Katze, die eine Maus besuchte und sagte: "Du bist meine beste Freundin. Lass uns zusammen wohnen und unsere Güter gemeinsam nutzen."

Die Maus sagte: "Gut, ich will."

Sie kauften einen Topf Butter und stellten ihn in den Keller. Die Katze sagte: "Wir müssen ihn gut aufbewahren. Niemand soll davon wissen."

Einige Tage später sagte die Katze: "Ich muss meine Cousine besuchen, die gerade Kinder bekommen hat. Ich werde etwas Butter mitnehmen."

Die Maus sagte: "Gut, geh nur."

Aber die Katze ging nicht zu ihrer Cousine, sondern zum Keller und aß einen Teil des Butters. Sie kehrte erst am Abend zurück.

Die Maus fragte: "Wie heißen die Kinder?"

Die Katze sagte: "Einer heißt 'Der erste Biss'."

Einige Tage später sagte die Katze wieder: "Ich muss meine Cousine besuchen, die wieder Kinder bekommen hat."

Sie ging wieder zum Keller und aß einen größeren Teil des Butters. Am Abend kehrte sie zurück.

Die Maus fragte: "Wie heißen die Kinder diesmal?"

Die Katze sagte: "Einer heißt 'Der zweite Biss'."

Einige Tage später sagte die Katze noch einmal: "Ich muss meine Cousine besuchen."

Sie ging zum Keller und aß den ganzen Rest des Butters. Am Abend kehrte sie zurück, sehr dick und satt.

Die Maus fragte: "Wie heißen die Kinder diesmal?"

Die Katze sagte: "Einer heißt 'Der ganze Rest'."

Im Winter kam es, dass sie kein Essen mehr hatten. Die Maus sagte: "Lass uns den Buttertopf holen."

Sie gingen zum Keller und fanden den Topf leer. Die Maus sagte: "Ach du Schurke! Du hast den ganzen Butter gegessen!"

Die Katze sagte: "Wenn du noch mehr sagst, dann fresse ich dich auch."

Und so taten sie. Die Katze fraß die Maus.
}

\translation{
猫和老鼠做朋友

从前有一只猫,它去拜访一只老鼠,说:“你是我最好的朋友。让我们一起住,共同使用我们的财产。”

老鼠说:“好的,我愿意。”

它们买了一个黄油罐,把它放在地窖里。猫说:“我们必须好好保管它。没有人应该知道这件事。”

几天后,猫说:“我必须去看望我的表姐,她刚生了孩子。我会带一些黄油去。”

老鼠说:“好的,去吧。”

但猫没有去它的表姐那里,而是去了地窖,吃了一部分黄油。它直到晚上才回来。

老鼠问:“孩子们叫什么名字?”

猫说:“一个叫‘第一口’。”

几天后,猫又说:“我必须去看望我的表姐,她又生了孩子。”

它又去了地窖,吃了更大一部分黄油。晚上它回来了。

老鼠问:“这次孩子们叫什么名字?”

猫说:“一个叫‘第二口’。”

几天后,猫又说:“我必须去看望我的表姐。”

它去了地窖,吃了剩下的所有黄油。晚上它回来了,非常胖,吃饱了。

老鼠问:“这次孩子们叫什么名字?”

猫说:“一个叫‘全部吃完’。”

冬天来了,它们没有食物了。老鼠说:“让我们把黄油罐拿出来。”

它们去了地窖,发现罐子是空的。老鼠说:“你这个骗子!你把所有的黄油都吃了!”

猫说:“如果你再说,我也会吃掉你。”

它就这样做了。猫吃掉了老鼠。
}

% 第十四篇:渔夫和他的妻子
\chapter{渔夫和他的妻子}
\section{故事内容}

\original{
Der Fischer und seine Frau

Es war einmal ein armer Fischer, der lebte mit seiner Frau in einem kleinen Hütte an der See.

Eines Tages fing er einen großen Fisch. Der Fisch sagte: "Ich bin ein verwandelter König. Lass mich frei, und ich werde dir Gutes tun."

Der Fischer war ein gutes Herz und ließ den Fisch frei.

Als er zu seiner Frau zurückkam, erzählte er ihr von dem Fisch.

Die Frau sagte: "Du bist ein Dummkopf! Du solltest ihn bitten, uns ein besseres Haus zu geben."

Der Fischer ging zum See und rief: "Fisch, Fisch, komm her! Meine Frau will ein besseres Haus."

Der Fisch sagte: "Geh heim, dein Wunsch ist erfüllt."

Als der Fischer nach Hause kam, fand er ein schönes Haus. Seine Frau war sehr glücklich.

Aber bald wurde sie unzufrieden. Sie sagte: "Ich will ein Schloss sein Königin."

Der Fischer ging wieder zum See und rief: "Fisch, Fisch, komm her! Meine Frau will ein Schloss und Königin sein."

Der Fisch sagte: "Geh heim, dein Wunsch ist erfüllt."

Als der Fischer nach Hause kam, fand er ein großes Schloss. Seine Frau war jetzt Königin.

Aber bald wurde sie noch unzufriedener. Sie sagte: "Ich will Kaiserin sein."

Der Fischer ging wieder zum See. Das Wasser war jetzt schwarz und stürmisch.

Er rief: "Fisch, Fisch, komm her! Meine Frau will Kaiserin sein."

Der Fisch sagte: "Geh heim, dein Wunsch ist erfüllt."

Als der Fischer nach Hause kam, war seine Frau Kaiserin.

Aber bald wurde sie noch unzufriedener. Sie sagte: "Ich will Gott sein."

Der Fischer ging wieder zum See. Das Wasser war jetzt wild und stürmisch.

Er rief: "Fisch, Fisch, komm her! Meine Frau will Gott sein."

Der Fisch sagte: "Geh heim, dein Wunsch ist nicht erfüllt."

Als der Fischer nach Hause kam, stand wieder die alte Hütte. Seine Frau war wieder arm. Sie lebten dort bis ans Ende ihrer Tage.
}

\translation{
渔夫和他的妻子

从前有一个贫穷的渔夫,他和妻子住在海边的一个小茅屋里。

一天,他捕到了一条大鱼。鱼说:“我是一个被施了魔法的国王。放我自由,我会报答你的。”

渔夫心地善良,放了这条鱼。

当他回到妻子身边时,他把这件事告诉了她。

妻子说:“你是个傻瓜!你应该让他给我们一个更好的房子。”

渔夫走到海边,喊道:“鱼,鱼,过来!我妻子想要一个更好的房子。”

鱼说:“回家吧,你的愿望已经实现了。”

当渔夫回到家时,他发现了一座漂亮的房子。他的妻子非常高兴。

但很快她就不满足了。她说:“我想要一座城堡,成为王后。”

渔夫又走到海边,喊道:“鱼,鱼,过来!我妻子想要一座城堡,成为王后。”

鱼说:“回家吧,你的愿望已经实现了。”

当渔夫回到家时,他发现了一座大城堡。他的妻子现在是王后了。

但很快她就更加不满足了。她说:“我想成为女皇。”

渔夫又走到海边。海水现在是黑色的,暴风雨肆虐。

他喊道:“鱼,鱼,过来!我妻子想成为女皇。”

鱼说:“回家吧,你的愿望已经实现了。”

当渔夫回到家时,他的妻子已经是女皇了。

但很快她就更加不满足了。她说:“我想成为上帝。”

渔夫又走到海边。海水现在汹涌澎湃,暴风雨肆虐。

他喊道:“鱼,鱼,过来!我妻子想成为上帝。”

鱼说:“回家吧,你的愿望不会实现了。”

当渔夫回到家时,旧茅屋又出现了。他的妻子又变得贫穷了。他们在那里一直生活到生命的尽头。
}

% 第十五篇:聪明的爱尔莎
\chapter{聪明的爱尔莎}
\section{故事内容}

\original{
Kluge Else

Es war einmal ein Mädchen namens Else, das war sehr klug. Eines Tages sagte ihre Mutter: "Else, geh zum Markt und kauf Butter und Ei."

Else nahm das Geld und ging zum Markt. Sie kaufte Butter und Ei und legte sie in ihren Korb.

Auf dem Weg nach Hause kam sie zu einem Bach. Der Bach war sehr flach, aber Else war Angst, dass das Ei fallen könnte.

Sie sagte zu sich: "Ich werde den Korb auf meinen Kopf legen, dann fällt nichts aus."

Aber als sie den Korb auf den Kopf legte, fiel das Ei herab und zerbrach. Die Butter fiel auch herab und wurde dreckig.

Else ging weinend nach Hause. Ihre Mutter sagte: "Du bist nicht klug. Du solltest den Korb in der Hand halten."

Einige Tage später sagte die Mutter wieder: "Else, geh zum Markt und kauf Zucker."

Else nahm das Geld und ging zum Markt. Sie kaufte Zucker und legte ihn in ihren Korb.

Auf dem Weg nach Hause kam sie wieder zum Bach. Diesmal sagte sie zu sich: "Ich werde den Korb in der Hand halten, wie Mutter gesagt hat."

Aber als sie über den Bach ging, stieß sie mit dem Korb an einen Stein. Der Zucker fiel in den Bach und wurde nass und zerschmolz.

Else ging weinend nach Hause. Ihre Mutter sagte: "Du bist nicht klug. Du solltest den Korb vorsichtig tragen."

Einige Tage später sagte die Mutter noch einmal: "Else, geh zu deiner Tante und bring ihr einen Kuchen."

Else nahm den Kuchen und ging zu ihrer Tante. Auf dem Weg kam sie zu einem Friedhof. Es war dort sehr still.

Else sagte zu sich: "Ich werde mich auf ein Grab legen und ausruhen."

Aber als sie auf das Grab legte, schlief sie ein. Während sie schlief, kam ein Wolf. Der Wolf sah den Kuchen und aß ihn.

Als Else erwachte, war der Kuchen verschwunden. Sie ging weinend zu ihrer Tante und erzählte ihr was passiert war.

Die Tante sagte: "Du bist nicht klug. Du solltest den Kuchen behalten und nicht schlafen."

Else kam nach Hause und erzählte ihrer Mutter alles. Die Mutter sagte: "Du bist eine dumme Else. Aber ich liebe dich trotzdem."

Und so lebten sie glücklich zusammen.
}

\translation{
聪明的爱尔莎

从前有一个名叫爱尔莎的女孩,她非常聪明。一天,她的母亲说:“爱尔莎,去市场买黄油和鸡蛋。”

爱尔莎拿着钱去了市场。她买了黄油和鸡蛋,把它们放在篮子里。

在回家的路上,她来到了一条小溪边。小溪很浅,但爱尔莎害怕鸡蛋会掉下来。

她自言自语地说:“我会把篮子放在头上,这样就不会掉东西了。”

但当她把篮子放在头上时,鸡蛋掉了下来,摔碎了。黄油也掉了下来,变得很脏。

爱尔莎哭着回到家。她的母亲说:“你不聪明。你应该用手拿着篮子。”

几天后,母亲又说:“爱尔莎,去市场买糖。”

爱尔莎拿着钱去了市场。她买了糖,把它放在篮子里。

在回家的路上,她又来到了那条小溪边。这次她自言自语地说:“我会像妈妈说的那样用手拿着篮子。”

但当她过小溪时,篮子撞到了一块石头。糖掉进了小溪里,变得潮湿并融化了。

爱尔莎哭着回到家。她的母亲说:“你不聪明。你应该小心地提篮子。”

几天后,母亲又说:“爱尔莎,去你阿姨家,给她带一个蛋糕。”

爱尔莎拿着蛋糕去了阿姨家。在路上,她来到了一个墓地。那里非常安静。

爱尔莎自言自语地说:“我会躺在一块墓碑上休息。”

但当她躺在墓碑上时,她睡着了。她睡着的时候,来了一只狼。狼看到蛋糕,就把它吃了。

当爱尔莎醒来时,蛋糕不见了。她哭着去了阿姨家,告诉她发生了什么事。

阿姨说:“你不聪明。你应该保管好蛋糕,不要睡觉。”

爱尔莎回到家,把一切都告诉了她的母亲。母亲说:“你是个笨爱尔莎。但我还是爱你。”

就这样,她们幸福地生活在一起。
}

% 第十六篇:画眉嘴国王
\chapter{画眉嘴国王}
\section{故事内容}

\original{
König Drosselbart

Es war einmal ein König, der hatte eine sehr schöne Tochter. Aber sie war sehr stolz und verachtete alle Bewerber.

Eines Tages rief der König alle Prinzen und Fürsten zusammen, damit seine Tochter einen Gemahl wählen konnte.

Aber die Prinzessin verachtete alle. Sie sagte zu einem Prinzen, der einen großen Mund hatte: "Du siehst aus wie ein Drosselbart!"

Der König war sehr wütend. Er sagte: "Ich werde dich an den ersten Mann geben, der vor unserer Tür vorbeikommt."

Am nächsten Tag kam ein armer Bettelmann vorbei. Der König sagte: "Du sollst meine Tochter heiraten."

Die Prinzessin weinte sehr, aber sie musste dem Willen ihres Vaters folgen.

Der Bettelmann nahm sie mit in sein armes Haus. Er sagte: "Jetzt musst du arbeiten, um zu überleben."

Er gab ihr ein Spinnrad und sagte: "Spinn Garn, damit wir Geld haben."

Aber die Prinzessin konnte nicht spinnen. Der Bettelmann gab ihr dann ein Weberschiffchen und sagte: "Webe Stoff."

Aber die Prinzessin konnte auch nicht weben. Der Bettelmann gab ihr dann einen Besen und sagte: "Kehr den Hof."

Aber die Prinzessin konnte auch nicht kehren.

Schließlich sagte der Bettelmann: "Ich muss dich zum Markt schicken, um zu verkaufen."

Er machte ein Schild mit den Worten "Ich verkaufe mich" und hängte es um ihren Hals.

An dem Markt kaufte sie ein reicher Mann. Aber als sie zu ihm kam, entdeckte sie, dass er der Prinz war, den sie "Drosselbart" genannt hatte.

Der Prinz sagte: "Ich habe dich getestet. Jetzt weiß ich, dass du nicht mehr stolz bist. Willst du mich heiraten?"

Die Prinzessin war sehr froh und sagte: "Ja, ich will."

Der Prinz führte sie in sein Schloss und heiratete sie. Die Prinzessin war nun eine gute Königin und lebte glücklich mit ihrem Mann.
}

\translation{
画眉嘴国王

从前有一个国王,他有一个非常美丽的女儿。但她非常骄傲,看不起所有的求婚者。

一天,国王召集了所有的王子和贵族,让他的女儿选择一个丈夫。

但公主看不起所有人。她对一个有大嘴巴的王子说:“你看起来像个画眉嘴!”

国王非常生气。他说:“我会把你嫁给第一个经过我们家门前的人。”

第二天,一个贫穷的乞丐经过。国王说:“你将娶我的女儿。”

公主哭了,但她必须听从父亲的意愿。

乞丐带着她去了他的贫穷的家。他说:“现在你必须工作才能生存。”

他给了她一个纺车,说:“纺线,这样我们就有钱了。”

但公主不会纺线。乞丐又给了她一个织布机,说:“织布。”

但公主也不会织布。乞丐又给了她一把扫帚,说:“打扫院子。”

但公主也不会打扫。

最后,乞丐说:“我必须把你送到市场去卖。”

他做了一个写着“我卖自己”的牌子,挂在她的脖子上。

在市场上,一个富人买了她。但当她来到他身边时,她发现他就是她称为“画眉嘴”的王子。

王子说:“我考验了你。现在我知道你不再骄傲了。你愿意嫁给我吗?”

公主非常高兴,说:“是的,我愿意。”

王子带她进入他的城堡,并娶了她。公主现在是一个好王后,和她的丈夫幸福地生活在一起。
}

% 第十七篇:十二兄弟
\chapter{十二兄弟}
\section{故事内容}

\original{
Die zwölf Brüder

Es war einmal ein König und eine Königin, die hatten zwölf Söhne. Eines Tages sagte die Königin: "Ich werde bald ein Kind bekommen. Wenn es eine Tochter ist, müssen die zwölf Brüder sterben, damit sie das Königreich erben kann."

Der König war sehr wütend, aber er konnte nichts tun. Die Königin machte zwölf Kissen und tötete jedes Mal ein Kind, wenn es schrie.

Aber die Mutter der zwölf Brüder hörte davon. Sie schickte die zwölf Brüder in den Wald und sagte: "Versteckt euch dort, bis ich euch rufe."

Die zwölf Brüder versteckten sich im Wald. Eines Tages kam ihre Schwester zu ihnen. Sie hatte ein goldenes Locken und war sehr schön.

Die Brüder waren sehr erstaunt und fragten: "Wer bist du?"

Die Schwester sagte: "Ich bin eure Schwester. Ich bin hier, um euch zu retten."

Sie erzählte ihnen von der Königin und ihrem Plan.

Die Brüder waren sehr wütend und sagten: "Wir werden nie wieder in das Königreich zurückkehren."

Aber die Schwester sagte: "Ich werde euch helfen. Ich werde die Königin überzeugen, euch zu schonen."

Sie kehrte ins Königreich zurück und sprach mit der Königin. Die Königin war sehr gerührt und sagte: "Ich werde die Brüder schonen."

Die Schwester ging zurück in den Wald und rief die Brüder. Die Brüder kamen zurück ins Königreich und lebten glücklich mit ihrer Schwester und den Eltern.
}

\translation{
十二兄弟

从前有一个国王和王后,他们有十二个儿子。一天,王后说:“我很快就要生孩子了。如果是女孩,十二个兄弟必须死,这样她才能继承王国。”

国王非常生气,但他无能为力。王后做了十二个枕头,每次孩子哭的时候就杀死一个。

但是十二个兄弟的母亲听说了这件事。她把十二个兄弟送到森林里,说:“藏在那里,直到我叫你们。”

十二个兄弟躲在森林里。有一天,他们的妹妹来找他们。她有金色的卷发,非常漂亮。

兄弟们非常惊讶,问道:“你是谁?”

妹妹说:“我是你们的妹妹。我来救你们。”

她告诉他们王后和她的计划。

兄弟们非常生气,说:“我们永远不会回到王国。”

但妹妹说:“我会帮助你们。我会说服王后饶了你们。”

她回到王国,和王后说话。王后非常感动,说:“我会饶了兄弟们。”

妹妹回到森林,叫兄弟们。兄弟们回到王国,和他们的妹妹、父母幸福地生活在一起。
}

% 第十八篇:六个仆人
\chapter{六个仆人}
\section{故事内容}

\original{
Die sechs Diener

Es war einmal ein König, der hatte eine sehr schöne Tochter. Er sagte zu einem Prinzen: "Wenn du sechs Diener mit besonderen Fähigkeiten finden kannst, sollst du meine Tochter heiraten."

Der Prinz machte sich auf den Weg, um sechs Diener zu finden. Er traf einen Mann, der konnte so weit sehen, dass er die Welt übersehen konnte. Er nannte ihn "Der Weitsichtige".

Dann traf er einen Mann, der konnte so stark sein, dass er einen Berg umwerfen konnte. Er nannte ihn "Der Starke".

Dann traf er einen Mann, der konnte so schnell laufen, dass er den Wind einholen konnte. Er nannte ihn "Der Schnelle".

Dann traf er einen Mann, der konnte so viel essen, dass er einen Ochsen in einem Bissen fressen konnte. Er nannte ihn "Der Fressende".

Dann traf er einen Mann, der konnte so viel trinken, dass er einen See leer trinken konnte. Er nannte ihn "Der Trinkende".

Dann traf er einen Mann, der konnte so laut schreien, dass er die Wälder erschüttern konnte. Er nannte ihn "Der Schreier".

Mit diesen sechs Dienern kehrte der Prinz zum König zurück. Der König hatte eine Aufgabe für sie: Sie mussten einen Drachen töten, der die Tochter gefangen hielt.

Der Weitsichtige sah den Drachen in einem Berg. Der Starke warf den Berg um und der Drachen kam heraus.

Der Schnelle rannte dem Drachen nach und der Fressende aß ihn auf.

Der König war sehr erstaunt und gab den Prinzen seine Tochter. Der Prinz heiratete die Prinzessin und lebte mit ihr und den sechs Dienern glücklich bis ans Ende seiner Tage.
}

\translation{
六个仆人

从前有一个国王,他有一个非常美丽的女儿。他对一位王子说:“如果你能找到六个有特殊能力的仆人,你就可以娶我的女儿。”

王子出发去寻找六个仆人。他遇到了一个人,他能看得很远,能看到整个世界。他称他为“千里眼”。

然后他遇到了一个人,他非常强壮,能把一座山推翻。他称他为“大力士”。

然后他遇到了一个人,他跑得非常快,能追上风。他称他为“飞毛腿”。

然后他遇到了一个人,他能吃很多,一口就能吃掉一头牛。他称他为“大胃王”。

然后他遇到了一个人,他能喝很多,能喝干一个湖。他称他为“海量”。

然后他遇到了一个人,他能喊得很大声,能震动森林。他称他为“大嗓门”。

王子带着这六个仆人回到了国王那里。国王给他们布置了一个任务:他们必须杀死一条把女儿囚禁起来的龙。

千里眼在一座山上看到了龙。大力士把山推翻,龙出来了。

飞毛腿追着龙跑,大胃王把它吃掉了。

国王非常惊讶,把女儿嫁给了王子。王子娶了公主,和她以及六个仆人幸福地生活到生命的尽头。
}

% 第十九篇:铁汉斯
\chapter{铁汉斯}
\section{故事内容}

\original{
Iron Hans

Es war einmal ein König, der hatte einen großen Wald. In diesem Wald lebte ein wilder Mann, den alle Iron Hans nannten, weil er in einem eisernen Anzug lebte.

Eines Tages verlor der König seinen Jungen im Wald. Er schickte Jäger und Diener aus, um den Jungen zu finden, aber sie konnten ihn nicht finden.

Der Junge war bei Iron Hans angekommen. Iron Hans sagte: "Ich werde dich schonen, wenn du für mich arbeitest."

Der Junge arbeitete für Iron Hans und lernte viele Dinge von ihm. Eines Tages sagte Iron Hans: "Du hast lange genug für mich gearbeitet. Du kannst gehen, aber du musst ein Versprechen geben: Du sollst nie über mich sprechen."

Der Junge versprach es und ging ins Königreich zurück. Seine Eltern waren sehr glücklich, als sie ihn wieder sahen.

Eines Tages raste der Junge mit einem Pferd und traf eine schöne Prinzessin. Er verliebte sich in sie und wollte sie heiraten.

Aber der König der Prinzessin sagte: "Du sollst eine Aufgabe lösen, um meine Tochter zu heiraten. Du musst einen Drachen töten."

Der Junge erinnerte sich an Iron Hans und bat ihn um Hilfe. Iron Hans sagte: "Ich werde dir helfen, aber du musst dein Versprechen brechen."

Der Junge sagte: "Ich werde es tun."

Iron Hans half dem Jungen, den Drachen zu töten. Der König gab ihm seine Tochter, und der Junge heiratete sie.

Iron Hans sagte: "Ich bin jetzt frei. Ich war ein verwandelter König. Vielen Dank, dass du mich geholfen hast."

Der Junge und die Prinzessin lebten glücklich bis ans Ende ihrer Tage.
}

\translation{
铁汉斯

从前有一个国王,他有一片大森林。在这片森林里住着一个野人,大家都叫他铁汉斯,因为他穿着一件铁盔甲。

一天,国王的儿子在森林里迷路了。他派出猎人和仆人去找儿子,但他们找不到他。

王子来到了铁汉斯的身边。铁汉斯说:“如果你为我工作,我会饶了你。”

王子为铁汉斯工作,从他那里学到了很多东西。有一天,铁汉斯说:“你为我工作的时间够长了。你可以走了,但你必须保证:永远不要谈论我。”

王子答应了,回到了王国。他的父母看到他回来,非常高兴。

有一天,王子骑马飞驰,遇到了一位美丽的公主。他爱上了她,想娶她为妻。

但公主的父亲说:“你必须完成一项任务才能娶我的女儿。你必须杀死一条龙。”

王子想起了铁汉斯,向他求助。铁汉斯说:“我会帮助你,但你必须打破你的誓言。”

王子说:“我会这么做的。”

铁汉斯帮助王子杀死了龙。国王把女儿嫁给了他,王子娶了公主。

铁汉斯说:“我现在自由了。我是一个被施了魔法的国王。谢谢你帮助我。”

王子和公主幸福地生活在一起,直到生命的尽头。
}

\end{document}