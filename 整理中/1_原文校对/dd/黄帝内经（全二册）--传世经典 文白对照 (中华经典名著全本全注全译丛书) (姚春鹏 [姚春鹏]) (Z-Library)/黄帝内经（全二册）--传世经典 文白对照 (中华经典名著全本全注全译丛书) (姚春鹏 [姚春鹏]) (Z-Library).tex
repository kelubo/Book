\documentclass{article}
\usepackage{ctex}
\begin{document}
\title{黄帝内经(全二册)--传世经典 文白对照 (中华经典名著全本全注全译丛书) (姚春鹏 [姚春鹏]) (Z-Library)}
\maketitle
\section{内容}
目录\\
目录\\
出版说明\\
黄帝内经上·素问\\
重广补注黄帝内经素问序一\\
重广补注黄帝内经素问序二\\
卷一\\
上古天真论篇第一\\
四气调神大论篇第二\\
生气通天论篇第三\\
金匮真言论篇第四\\
卷二\\
阴阳应象大论篇第五\\
阴阳离合论篇第六\\
阴阳别论篇第七\\
卷三\\
灵兰秘典论篇第八\\
六节脏象论篇第九\\
五脏生成篇第十\\
五脏别论篇第十一\\
卷四\\
异法方宜论篇第十二\\
移精变气论篇第十三\\
汤液醪醴论篇第十四\\
玉版论要篇第十五\\
诊要经终论篇第十六\\
卷五\\
脉要精微论篇第十七\\
平人气象论篇第十八\\
卷六\\
玉机真藏论篇第十九\\
三部九候论篇第二十\\
卷七\\
经脉别论篇第二十一\\
脏气法时论篇第二十二\\
宣明五气篇第二十三\\
血气形志篇第二十四\\
卷八\\
宝命全形论篇第二十五\\
八正神明论篇第二十六\\
离合真邪论篇第二十七\\
通评虚实论篇第二十八\\
太阴阳明论篇第二十九\\
阳明脉解篇第三十\\
卷九\\
热论篇第三十一\\
刺热篇第三十二\\
评热病论篇第三十三\\
逆调论篇第三十四\\
卷十\\
疟论篇第三十五\\
刺疟篇第三十六\\
气厥论篇第三十七\\
咳论篇第三十八\\
卷十一\\
举痛论篇第三十九\\
腹中论篇第四十\\
刺腰痛篇第四十一\\
卷十二\\
风论篇第四十二\\
痹论篇第四十三\\
痿论篇第四十四\\
厥论篇第四十五\\
卷十三\\
病能论篇第四十六\\
奇病论篇第四十七\\
大奇论篇第四十八\\
脉解篇第四十九\\
卷十四\\
刺要论篇第五十\\
刺齐论篇第五十一\\
刺禁论篇第五十二\\
刺志论篇第五十三\\
针解篇第五十四\\
长刺节论篇第五十五\\
卷十五\\
皮部论篇第五十六\\
经络论篇第五十七\\
气穴论篇第五十八\\
气府论篇第五十九\\
卷十六\\
骨空论篇第六十\\
水热穴论篇第六十一\\
卷十七\\
调经论篇第六十二\\
卷十八\\
缪刺论篇第六十三\\
四时刺逆从论篇第六十四\\
标本病传论篇第六十五\\
卷十九\\
天元纪大论篇第六十六\\
五运行大论篇第六十七\\
六微旨大论篇第六十八\\
卷二十\\
气交变大论篇第六十九\\
五常政大论篇第七十\\
卷二十一\\
六元正纪大论篇第七十一\\
卷二十二\\
至真要大论篇第七十四\\
卷二十三\\
著至教论篇第七十五\\
示从容论篇第七十六\\
疏五过论篇第七十七\\
徵四失论篇第七十八\\
卷二十四\\
阴阳类论篇第七十九\\
方盛衰论篇第八十\\
解精微论篇第八十一\\
附录\\
刺法论篇第七十二\\
本病论篇第七十三\\
黄帝内经下·灵枢\\
叙\\
卷一\\
九针十二原第一 法天\\
本输第二 法地\\
小针解第三 法人\\
邪气脏腑病形第四 法时\\
卷二\\
根结第五 法音\\
寿夭刚柔第六 法律\\
官针第七 法星\\
本神第八 法风\\
终始第九 法野\\
卷三\\
经脉第十\\
经别第十一\\
经水第十二\\
卷四\\
经筋第十三\\
骨度第十四\\
五十营第十五\\
营气第十六\\
脉度第十七\\
营卫生会第十八\\
四时气第十九\\
卷五\\
五邪第二十\\
寒热病第二十一\\
癫狂第二十二\\
热病第二十三\\
厥病第二十四\\
病本第二十五\\
杂病第二十六\\
周痹第二十七\\
口问第二十八\\
卷六\\
师传第二十九\\
决气第三十\\
肠胃第三十一\\
平人绝谷第三十二\\
海论第三十三\\
五乱第三十四\\
胀论第三十五\\
五癃津液别第三十六\\
五阅五使第三十七\\
逆顺肥瘦第三十八\\
血络论第三十九\\
阴阳清浊第四十\\
卷七\\
阴阳系日月第四十一\\
病传第四十二\\
淫邪发梦第四十三\\
顺气一日分为四时第四十四\\
外揣第四十五\\
五变第四十六\\
本脏第四十七\\
卷八\\
禁服第四十八\\
五色第四十九\\
论勇第五十\\
背腧第五十一\\
卫气第五十二\\
论痛第五十三\\
天年第五十四\\
逆顺第五十五\\
五味第五十六\\
卷九\\
水胀第五十七\\
贼风第五十八\\
卫气失常第五十九\\
玉版第六十\\
五禁第六十一\\
动输第六十二\\
五味论第六十三\\
阴阳二十五人第六十四\\
卷十\\
五音五味第六十五\\
百病始生第六十六\\
行针第六十七\\
上膈第六十八\\
忧恚无言第六十九\\
寒热第七十\\
邪客第七十一\\
通天第七十二\\
卷十一\\
官能第七十三\\
论疾诊尺第七十四\\
刺节真邪第七十五\\
卫气行第七十六\\
九宫八风第七十七\\
卷十二\\
九针论第七十八\\
岁露论第七十九\\
大惑论第八十\\
痈疽第八十一\\
黄帝内经(全二册)——传世经典 文白对照\\
[传世经典 文白对照]\\
黄帝内经\\
姚春鹏  译\\
中华书局\\
黄帝内经(全二册)——传世经典 文白对照\\
出版说明\\
《黄帝内经》简称《内经》,包括《素问》和《灵枢》两部分,各十八卷、各八十一篇。《黄帝内经》之名最早见于《汉书·艺文志·方技略》。该书以黄帝和岐伯等人对话的形式写成,作者似乎就是黄帝和岐伯等人。但正如《淮南子·修务训》所云:“世俗之人,多尊古而贱今,故为道者必托之于神农、黄帝而后能入说。”所以黄帝、岐伯等显系托名。现在多数学者认为《内经》非一人所作而是集体、多人长期努力的结晶。它也不是初创之作,而是经过编纂的作品。成编时间大约从春秋战国至两汉之间。\\
《内经》成编后,《素问》和《灵枢》既有同时传世者,也曾分别流传。张仲景写作《伤寒杂病论》时曾利用过《素问》和《九卷》。《九卷》即《灵枢》。晋人皇甫谧撰《针灸甲乙经》则几乎辑录了《素问》和《九卷》的全部文字。现在可见到的最早注本就是唐代王冰的《重广补注黄帝内经素问》,但其原书也已亡佚。现在见到的是经宋人林亿和高保衡整理过的,称为《次注》。到明清时期为《素问》作注的就较多了。著名的有马莳的《黄帝内经素问注证发微》,吴昆的《吴注黄帝内经素问》,张志聪的《黄帝内经素问集注》,高世栻的《素问直解》等。《灵枢》历史上一直以《九卷》之名流传。至宋史崧始以“家藏旧本《灵枢》九卷”“参对诸书”整理成《灵枢》的定本,称为《黄帝内经灵枢经》,流传至今。马莳的《黄帝内经灵枢发微》是《灵枢》最早的注释本。把《素问》和《灵枢》合编注释的有明代张景岳的《类经》。\\
《黄帝内经》是我国现存最早的医学典籍,但其内容又不仅限于医学,而与中国古代的哲学、天文、地理等学科密切相关,是一部关于哲学和自然科学的综合著作。我国古代并不把医学看成是孤立的为医学专家所垄断的专门学问,而是把它放在天地自然和社会文化的大视野中来思考的。所谓“道者,上知天文,下知地理,中知人事,可以长久”。《内经》医学著述写作于诸子百家学术争鸣的年代,与诸子之学相互倡和,对诸子学多有吸收,并深受其影响。从《内经》文本看,黄老道家,《周易》与《内经》关系最紧密。如老子的无为思想,庄子的真人、至人、圣人、贤人人格,在《内经》的很多篇章中出现,《内经》多处引用《老子》、《庄子》中的语言。可以说,在价值观上,《内经》与黄老道家是一致的。这也是《内经》托名黄帝的内在根源。《周易》的“象数”思维是《内经》理论体系的核心方法。脏象学说、十二经脉理论与《周易》有着渊源关系。《周易》的观象论、制器尚象论导出了医学上的藏象学说。《周易》对阴阳的太少划分、八卦的三爻论及天地人三才论,成为医学三阴三阳、十二经脉理论的依据。另外,儒家的中庸、中和,“有诸内必形诸外”以及重“本”的观念等也都是《内经》医学的重要观念。\\
中华民族是有着悠久历史传统,创造了光辉灿烂的文化,富有伟大智慧的民族。我们祖先所创造的文化是与以西方文化为代表的现代文化不同的另一种文化。这种文化虽然在强势的西方文化面前有其劣势,但另一方面,又是富有相当智慧的文化。在一定意义上代表了人类未来的发展方向。当然,随着中国近百年来的现代化运动,我们祖先所创造的文化已经越来越远去了,现代的中国人已经不太理解我们传统的文化和思维方式了。这影响了我们阅读和理解古人的作品,阻碍了我们与先人的心灵交通。\\
阅读中国文化的经典,首先要排除现代思维定势的干扰,进入古人的思维之中,才可能理解经典的本来意蕴。\\
天人合一的天人相应观。天人问题是中国古代哲学的基本问题,《内经》持天人相应的观点。天人相应的基本内涵是人由天地之气所化生,人的生命活动取决于天地自然的变化规律,人也应该主动地去顺应天地自然的变化规律。顺应天地自然对养生和治病有着特别重要的意义。\\
天地万物由一气所化。中国古人认为气是宇宙和生命的本源,人与天地万物都由气所化生。天与人之间之所以存在着相应的关系,源于天人一气。气是沟通天人万物的中介。气是人与万物生死存亡的根据,是生命的本质。在气论自然观的宇宙图景中,整个宇宙是一个大生命体,是由气所推动的大化流行过程。就人来说,生命取决于气,宝气、养气、调气是养生和治病的根本要求。\\
阴阳五行是中医学认识世界的基本框架。古人认为作为天地万物本源的气或称元气,具有运动化生的本性。气的运动展开为阴阳五行,阴阳五行之气是世界的基本结构。整个世界就是以气为内在本质,以阴阳五行为外在形态表现的动态统一系统。万事万物通过阴阳五行联系为一个统一的整体。\\
形神统一,重神轻形,是中医区别于西医的基本特征。古人认为,天地万物由气所化生,具体说来,是由在天之气(阳气),和在地之形(阴气)合和而成。就人来说则是形神合一。神是气之功能的极致表现,神本质上也是气。人的生命活动虽然要以形体为依托,但终究以气为本质,气在生命存,气去生命亡。所以中医在生命观上重气轻形。最佳的生理状态应该是形气相得,在病理状态下则是气胜形则生,形胜气则死。\\
阴阳和平是中医学最高的价值追求。追求宇宙万物的和谐是中华民族的永恒价值观。《内经》认为阴平阳秘是生命存在的前提。在养生上,调和阴阳,达到和同筋脉,气血皆从,内外调和是养生的最终目标。人之所以生病,根本原因就是气血阴阳的逆乱失调,所以中医的具体治疗原则虽有很多,但都以平调阴阳气血为最后目的。\\
取象运数比类是中医思维的基本方法。这一思维方法肇始于上古,形成于《周易》。在中医学和其他传统学科中得到了运用和发展。象是物象,事物显现于外的形象,观是对物象的观摩、研究。古人认为万物皆由阴阳五行之气所化生,相同的气所生之物具有相同或类似的作用功能和形象,彼此之间具有特别的亲和力,所谓“同气相求”、“同类相动”。古人就是以此为根据归类划分事物,作为认知基础的。《内经》根据五行把万物归为五大类。运数就是运用天地之数作为认知世界的纲领。《内经》所重视的是一至九这九个数。“天地之至数,始于一而终于九焉。”运数思维使《内经》能够运用一个简单的框架来认知复杂的世界及人体生理病理现象;取象思维使《内经》根据同象归并的原则类分事物,并认知事物之间的相互作用和联系。取象运数思维是《内经》建立医学理论体系的核心思维方法,也是其后中医学家运用中医理论认识疾病、认识药物的基本思维模式。\\
《内经》理论体系包含着丰富的思想内容,古今学者对《内经》理论体系所包含的内容从不同角度有不同的划分,明代医家张景岳在《类经》中则将《内经》理论体系划分为:摄生类、阴阳类、藏象类、脉色类、经络类、标本类、论治类、疾病类、针刺类、运气类和会通类,共计十一类。最后的会通类属综合类,不构成理论体系的内容,实际上是十类。现代中医学者从不同的分类角度,运用不同的方法,得出的结论就更多了。这里不再一一说明。这里要说明的一点是从唐代的王冰开始到明代的张景岳都把养生内容列为《内经》医学理论之首,是十分有见地的。在中国古代先哲看来,只有对天地宇宙有一个正确的认识,养成高尚的道德人格,建立一种合理的生活方式,才是保持身体健康,免除疾病缠绕的关键所在,才是“跻斯民于仁寿”的恒久之道。至于得病之后的治疗则是不得已而为之的下策。《老子》曾说:“夫唯病病,是以不病。”《内经》倡导“治未病”的积极养生思想。认为“病已成而后药之,譬犹渴而穿井,斗而铸兵,不亦晚乎?!”\\
本书根据丛书出版要求,对《素问》和《灵枢》的全文进行了现代语译,希望读者朋友通过阅读本书能够对《黄帝内经》及中医学有个比较全面的概要了解,为进一步学习打下基础。\\
姚春鹏\\
2012年1月1日\\
于曲阜师范大学日照校区静远斋\\
黄帝内经(全二册)——传世经典 文白对照\\
黄帝内经 素问\\
重广补注黄帝内经素问序一\\
重广补注黄帝内经素问序二\\
卷一 上古天真论篇第一\\
卷一 四气调神大论篇第二\\
卷一 生气通天论篇第三\\
卷一 金匮真言论篇第四\\
······\\
重广补注黄帝内经素问序一\\
启玄子 王冰 撰\\
夫释缚脱艰,全真导气,拯黎元于仁寿,济羸劣以获安者,非三圣道,则不能致之矣。孔安国序《尚书》曰:“伏羲、神农、黄帝之书,谓之三坟,言大道也。”班固《汉书·艺文志》曰:“《黄帝内经》十八卷。”《素问》即其经之九卷也,兼《灵枢》九卷,乃其数焉。\\
虽复年移代革,而授学犹存,惧非其人,而时有所隐,故第七一卷,师氏藏之,今之奉行,惟八卷尔。然而其文简,其意博,其理奥,其趣深。天地之象分,阴阳之候列,变化之由表,死生之兆彰。不谋而遐迩自同,勿约而幽明斯契。稽其言有征,验之事不忒。诚可谓至道之宗,奉生之始矣。\\
假若天机迅发,妙识玄通,蒇谋虽属乎生知,标格亦资于诂训,未尝有行不由径,出不由户者也。然刻意研精,探微索隐,或识契真要,则目牛无全。故动则有成,犹鬼神幽赞,而命世奇杰,时时间出焉。则周有秦公,汉有淳于公,魏有张公、华公,皆得斯妙道者也。咸日新其用,大济蒸人,华叶递荣,声实相副,盖教之著矣,亦天之假也。\\
冰弱龄慕道,夙好养生,幸遇真经,式为龟镜。而世本纰缪,篇目重叠,前后不伦,文义悬隔,施行不易,披会亦难。岁月既淹,袭以成弊。或一篇重出,而别立二名;或两论并吞,而都为一目;或问答未已,别树篇题,或脱简不书,而云世阙;重《经合》而冠《针服》,并《方宜》而为《咳篇》,隔《虚实》而为《逆从》,合《经络》而为《论要》,节《皮部》为《经络》,退《至教》以先《针》。诸如此流,不可胜数。且将升岱岳,非径奚为!欲诣扶桑,无舟莫适!乃精勤博访,而并有其人,历十二年,方臻理要,询谋得失,深遂夙心。\\
时于先生郭子斋堂,受得先师张公秘本,文字昭晰,义理环周,一以参详,群疑冰释。恐散于末学,绝彼师资,因而撰注,用传不朽。兼旧藏之卷,合八十一篇,二十四卷,勒成一部。冀乎究尾明首,寻注会经,开发童蒙,宣扬至理而已。其中简脱文断,义不相接者,搜求经论所有,迁移以补其处;篇目坠缺,指事不明者,量其意趣,加字以昭其义;篇论吞并,义不相涉,阙漏名目者,区分事类,别目以冠篇首;君臣请问,礼仪乖失者,考校尊卑,增益以光其意,错简碎文,前后重叠者,详其指趣,削去繁杂,以存其要,辞理秘密,难粗论述者,别撰《玄珠》,以陈其道。凡所加字,皆朱书其文,使今古必分,字不杂糅。庶厥昭彰圣旨,敷畅玄言,有如列宿高悬,奎张不乱,深泉净莹,鳞介咸分。君臣无夭枉之期,夷夏有延龄之望。俾工徒勿误,学者惟明。至道流行,徽音累属,千载之后,方知大圣之慈惠无穷。\\
时大唐宝应元年岁次壬寅序\\
解除疾病的缠绕,脱离病患的艰难,保全真精,通导元气,拯救黎民达到长寿之境,帮助体弱多病的人获得平安,如果没有三圣之道,就不能达到目的。孔安国在《尚书》序中说:“伏羲、神农、黄帝的书,叫作三坟,是讲大道的。” 班固在《汉书·艺文志》中说:“《黄帝内经》十八卷。”《素问》就是其中的九卷,加上《灵枢》九卷,就是它的总数。\\
虽然年代一再推移,朝代一再更替,但通过对它的传授和学习却依然保存着,只因担心不是适当的人选而时有隐藏,秘而不传,所以第七这一卷,前辈的先生把它隐藏起来,现在世上流传的只有八卷了。虽然如此,可是《内经》的文字简练,含义广博,旨趣深远。分清了天地的形象,序列了阴阳四时的节候,表明了变化的缘由,彰明了生死的征兆。不用谋划而远近自然相同,不用约定而无论隐微还是明显的都能相合。稽查其中的言论有证据,用事实验证也没有差错。实在是至高医学道理的源泉,是养生的根本。\\
假若天资迅敏萌动,可通晓玄妙深奥的道理,完整准确地理解经文虽然属于资质聪明的人,但正确理解经文也要凭借对经文的训诂阐释,这正如从来没有行走不沿着道路,出入不经过门户的。然而专心致志深入研究,探讨微妙求索隐奥,如果认识能够符合真理要道,就能达到目无全牛的高深境界。所以常常会有成就,就像鬼神在暗中帮助,而闻名于世的杰出人才,就不断地涌现出来了。如东周有秦越人,汉代有淳于意,魏国有张仲景、华佗,这些都是掌握了医道奥妙的人。他们都能够一天天地更新扩大医学的功用,大大地救助民众,他们的事业就像鲜花和绿叶一样,递相繁荣,名声与实际相符合,这大概是教学的显著成果,也是上天的帮助吧。\\
我年轻时就仰慕医道,平素喜好养生之学,有幸得遇真经,把它作为养生治病的指导。可是世上流传的本子有错误,篇目重复,前后不伦不类,文字义理差别很大,不容易运用,阅读理解也困难。年月久远,相互沿袭成为顽弊。有的同一篇文章重复出现,却分立两个题目;有的两篇文章混在一起,却用同一个篇名;有的没有回答完,就设定了另外的篇题;有的文字有脱落不能指明,却说历来就缺少。如在重出的《经合》篇之前冠以《八正神明论》之名;把《异法方宜论》并入《咳论》篇;把《四时刺逆从论》中论“虚实”的内容割裂出来,放在远隔的第六卷;把《诊要经终论》合并于《玉版论要》;把《经络论》混于《皮部论》中;把论“至教”的《上古天真论》退后,而把论针灸的《调经论》、《四时刺逆从论》等篇置前。诸如此类,不胜枚举。要登泰山,没有道路怎么上!要去日出的扶桑之地,没有舟船也不能到达!于是精心研究,广泛寻访,而且有志同道合之人,经历十二年,才达到理会要领,思考其中的得与失,深感实现了夙愿。\\
当时在郭先生的书房里,得到了先师张公的秘本,其文字明白清楚,义理周详,无矛盾不通之处,用这本书一参对,原来的众多疑问就像冰雪融化一样得到解决了。我担心被后学所散失,断绝了他们学习的资本,因而对此撰写注释,以使之流传不休。加上原来收藏的第七卷,合计八十一篇二十四卷,汇总成一部书。希望能够研究了后面就能明了前面,寻找注文就能够领会经文,开启后学,宣扬医道的至理而已。其中因古书中竹简脱落,文字缺断,义理不相连续的,寻找经论中的相关内容,迁移过来以补其空缺之处;篇名丢失,所指事类不明白的,考量其意旨所在,增加文字以使其义理明白;篇论混杂,义理无关,缺少名称的,区别不同的事类,另立题目置于篇章的开始;君臣之间的相互请问,礼仪错乱的,考核校订尊卑高下,增加文字以使其意义清楚;错乱的段落零碎的文字,前后重复的,详究其意趣,删去繁杂重复之处,而保存其要点;文辞义理奥秘,难以粗略说清楚的,另外撰写了《玄珠密语》,来陈说其中的道理。凡所增加的文字,都用红色书写,使今文古文一定分开,文字不混杂在一起。这样或许能够彰明岐、黄之圣意,讲明深奥的道理,就像二十八宿高悬天际,奎宿、张宿各居其位,毫不紊乱,又像深深的泉水晶莹透明,鳞类、甲壳类的水生物都能分开。君臣上下没有夭折的时候,华夏蛮夷都有延年益寿的希望。使医工不发生错误,学习的人能够详明医理。医学的至道流传四方,德音不断,千年之后,才知道大圣先贤对后人的慈爱恩惠是无穷无尽的。\\
时间在大唐宝应元年壬寅年作序\\
重广补注黄帝内经素问序二\\
臣闻安不忘危,存不忘亡者,往圣之先务;求民之瘼,恤民之隐者,上主之深仁。在昔黄帝之御极也,以理身绪余治天下,坐于明堂之上,临观八极,考建五常。以谓人之生也,负阴而抱阳,食味而被色,外有寒暑之相荡,内有喜怒之交侵,夭昏札瘥,国家代有。将欲敛时五福,以敷锡厥庶民,乃与岐伯上穷天纪,下极地理,远取诸物,近取诸身,更相问难,垂法以福万世。于是雷公之伦,授业传之,而《内经》作矣。历代宝之,未有失坠。苍周之兴,秦和述六气之论,具明于《左史》。厥后越人得其一二,演而述《难经》。西汉仓公传其旧学,东汉张仲景撰其遗论,晋皇甫谧刺而为《甲乙》,及隋杨上善纂而为《太素》。时则有全元起者,始为之训解,阙第七一通。\\
迄唐宝应中,太仆王冰笃好之,得先师所藏之卷,大为次注,犹是三皇遗文,烂然可观。惜唐令列之医学,付之执技之流,而荐绅先生罕言之,去圣已远,其术晻昧,是以文注纷错,义理混淆。殊不知三坟之余,帝王之高致,圣贤之能事,唐尧之授四时,虞舜之齐七政,神禹修六府以兴帝功,文王推六子以叙卦气,伊尹调五味以致君,箕子陈五行以佐世,其致一也。奈何以至精致微之道,传之以至下至浅之人,其不废绝,为已幸矣。\\
顷在嘉祐中,仁宗念圣祖之遗事,将坠于地,乃诏通知其学者,俾之是正。臣等承乏典校,伏念旬岁。遂乃搜访中外,裒集众本,浸寻其义,正其讹舛,十得其三四,余不能具。窃谓未足以称明诏,副圣意,而又采汉唐书录古医经之存于世者,得数十家,叙而考正焉。贯穿错综,磅礴会通,或端本以寻支,或溯流而讨源,定其可知,次以旧目,正谬误者六千余字,增注义者二千余条,一言去取,必有稽考,舛文疑义,于是详明。以之治身,可以消患于未兆,施于有政,可以广生于无穷。恭惟皇帝抚大同之运,拥无疆之修,述先志以奉成,兴微学而永正,则和气可召,灾害不生,陶一世之民,同跻于寿域矣。\\
国子博士臣高保衡、光禄卿直秘阁臣林亿等谨上\\
臣等听说平安之时不忘危机,存在之时不忘灭亡,这是从前圣人首先要努力的;探求人民的疾病,体恤人民的隐痛,这是上古君王的深厚仁德。从前黄帝临御八极,在养身修性之余治理天下,坐在明堂上,远观八方,考察建立五行气运之常。而认为人的生命由阴阳二气和合而成,本身贪图美味、迷恋美色,在外有风寒暑湿的影响,在内有喜怒哀乐交相干扰,病困夭死,无论任何地方、任何时候总是存在的。黄帝要按天时建立福善民众的医道,以广泛传赐给百姓,于是和岐伯上究天文,下穷地理,在远处取于外物,在近处取于自身,交替提问辩难,留下医学大法以造福千秋万代。于是雷公这些人,接受岐、黄之业,并传播它,因而创作了《内经》。历代以之为珍宝,没有坠地失传。东周兴起后,秦国名医和阐述六气致病理论,这件事具体明确地记载于左丘明的《春秋左氏传》中。其后秦越人抽取《内经》的十分之一二,推演发挥为《难经》。西汉仓公传播《内经》之学,东汉张仲景据《内经》理论写成《伤寒杂病论》,晋朝皇甫谧拣选其内容写成《针灸甲乙经》,到了隋朝的杨上善编纂而成《黄帝内经太素》。齐梁时有全元起,开始训释解说《内经》,但已经缺第七这一卷了。\\
到了唐代宝应时期,太仆令王冰特别喜好《内经》,并且得到了先师珍藏的第七这一卷,大幅地重新调整篇卷次序,并进行了大量的注释,这样一来三皇遗留下来的文献才文理灿烂,可以观读了。可惜唐朝的法令把《内经》列于医学,看成是小技之类,因而士大夫读书人很少有讨论的,距离圣人已经久远了,医术湮灭,因此文字注释纷纭错乱,义理混淆矛盾。完全不懂得三坟之遗文,帝王之崇高目标,圣贤所能之事,唐尧观象授时,虞舜齐同七政,神禹修六府而完成帝王的功业,文王推演八卦六十甲子而叙述卦象与节气的关系,伊尹调配饮食五味来保养君王的健康,箕子陈述五行来辅助武王治世,其目的都是一样的。为什么把医学这极精极微的事业,传给最卑下最浅陋的人呢,医学不衰败灭绝,就已经是幸事了。\\
最近在我朝嘉祐年间,仁宗皇帝恤念担忧圣祖岐、黄的遗业,将坠地失传,于是诏令通晓了解《内经》之学的人,让他们加以厘正。臣等忝列其中,参与典校,我等沉思潜索十余年。于是就搜求询访官府和民间所藏,聚集了众多的版本,渐渐地寻求其义理,订正其中的讹误舛错,十分之三四,其余的不能一一具述。我们觉得还不足以达到明诏的要求,还不能符合圣君的心意,因而又采集汉唐书录中尚存于世的古医经,得到几十家,对之进行叙述而考核订正。对于各家之间文字上的错综复杂关系进行核对寻找其可以贯穿一致者,进行了大量的会通比较工作,有的是端正其根本,而寻求其支脉,有的是沿其支流而寻求其本源,确定其可以了解的,按原来的目录排序,订正了谬误六千多字,增加了注释两千多条,每个字的取舍,必须以稽查考订为据,错误的文字,疑惑的道理,于是就清楚明白了,用它来养生,可以把疾患消除在萌芽之中,运用于行政可以对民众的养生健康发挥无穷无尽的作用。我等认为皇帝安抚着运化不息的大同世界,拥有美好的无限疆域,继承先王先圣的遗志而持守既成的事业,复兴隐微的医学而使之永远沿着正确的道路发展,这样祥和之气可以到来,灾害就不会发生,陶养全社会的民众,共同达到长寿的境域了。\\
国子博士臣高保、衡光禄卿直秘阁臣林亿等谨上\\
卷一\\
上古天真论篇第一\\
昔在黄帝,生而神灵,弱而能言,幼而徇齐,长而敦敏,成而登天。\\
乃问于天师曰:余闻上古之人,春秋皆度百岁,而动作不衰;今时之人,年半百而动作皆衰者,时世异耶?人将失之耶?\\
岐伯对曰:上古之人,其知道者,法于阴阳,知于术数,食饮有节,起居有常,不妄作劳,故能形与神俱,而尽终其天年,度百岁乃去。今时之人不然也,以酒为浆,以妄为常,醉以入房,以欲竭其精,以耗散其真。不知持满,不时御神,务快其心,逆于生乐,起居无节,故半百而衰也。\\
夫上古圣人之教也,下皆为之。虚邪贼风,避之有时,恬惔虚无,真气从之,精神内守,病安从来?是以志闲而少欲,心安而不惧,形劳而不倦。气从以顺,各从其欲,皆得所愿。故美其食,任其服,乐其俗,高下不相慕,其民故自朴。是以嗜欲不能劳其目,淫邪不能惑其心。愚智贤不肖,不惧于物,故合于道。所以能年皆度百岁而动作不衰者,以其德全不危故也。\\
帝曰:人年老而无子者,材力尽邪?将天数然也?\\
岐伯曰:女子七岁,肾气实,齿更发长。二七而天癸至,任脉通,太冲脉盛,月事以时下,故有子。三七,肾气平均,故真牙生而长极。四七,筋骨坚,发长极,身体盛壮。五七,阳明脉衰,面始焦,发始堕。六七,三阳脉衰于上,面皆焦,发始白。七七,任脉虚,太冲脉衰少,天癸竭,地道不通,故形坏而无子也。\\
丈夫八岁,肾气实,发长齿更。二八,肾气盛,天癸至,精气溢泻,阴阳和,故能有子。三八,肾气平均,筋骨劲强,故真牙生而长极。四八,筋骨隆盛,肌肉满壮。五八,肾气衰,发堕齿槁。六八,阳气衰竭于上,面焦,发鬓颁白。七八,肝气衰,筋不能动。八八,天癸竭,精少,肾脏衰,则齿发去,形体皆极。肾者主水,受五脏六腑之精而藏之,故脏腑盛,乃能泻。今五脏皆衰,筋骨解堕,天癸尽矣,故发鬓白,身体重,行步不正,而无子耳。\\
帝曰:有其年已老而有子者,何也?\\
岐伯曰:此其天寿过度,气脉常通,而肾气有余也。此虽有子,男不过尽八八,女不过尽七七,而天地之精气皆竭矣。\\
帝曰:夫道者,年皆百数,能有子乎?\\
岐伯曰:夫道者,能却老而全形,身年虽寿,能生子也。\\
黄帝曰:余闻上古有真人者,提挈天地,把握阴阳。呼吸精气,独立守神,肌肉若一。故能寿敝天地,无有终时。此其道生。\\
中古之时,有至人者,淳德全道,和于阴阳。调于四时,去世离俗。积精全神,游行天地之间,视听八达之外。此盖益其寿命而强者也。亦归于真人。\\
其次有圣人者,处天地之和,从八风之理,适嗜欲于世俗之间,无恚嗔之心。行不欲离于世,举不欲观于俗。外不劳形于事,内无思想之患。以恬愉为务,以自得为功。形体不敝,精神不散,亦可以百数。\\
其次有贤人者,法则天地,象似日月。辩列星辰,逆从阴阳。分别四时,将从上古。合同于道,亦可使益寿而有极时。\\
古代的轩辕黄帝,生来就异常聪明,小时候就善于言辞,很小的时候就对事物有着敏锐的洞察力,长大后,敦厚朴实而又勤勉努力,到了成年就登上了天子位。\\
黄帝问岐伯道:我听说上古时代的人,年龄都超过了百岁,但行动没有衰老的迹象;现在的人,年龄到五十岁,动作就显得衰老了,这是时代的不同呢?还是人们违背了养生之道的缘故呢?\\
岐伯回答说:上古时代的人,大都懂得养生之道,取法天地阴阳的变化规律,用保养精气的方法来调和,饮食有节制,起居有规律,不过分劳作,所以形体和精神能够协调统一,享尽自然的寿命,度过百岁才离开世间。现在的人就不同了,把浓酒当作甘泉无节制地贪饮,把任意妄为当作生活的常态,醉后还勉强行房,纵情声色,以致精气衰竭,真气耗散。不懂得保持精气的盈满,不明白节省精神,一味追求感官快乐,违背了生命的真正乐趣,起居没有规律,所以五十岁左右就衰老了。\\
上古时期,对通晓养生之道的圣人的教诲,人们都能遵守。对于四时不正之气,能够及时回避,思想上清静安闲,无欲无求,真气深藏顺从,精神持守于内而不耗散,这样,疾病怎么会发生呢?所以他们心志闲淑,私欲很少,心情安宁,没有恐惧,形体虽然劳动,但不过分疲倦。真气从容和顺,每个人的希望和要求,都能满足。无论吃什么都觉得甜美,穿什么都觉得漂亮,喜欢社会习俗,互相之间也不羡慕地位的高低,人们日渐变得自然朴实。所以过度的嗜好,不会干扰他的视听,淫乱邪说也不会惑乱他的心志。无论愚笨聪明有能力无能力的,都不追求酒色等身外之物,所以合于养生之道。因而他们都能够度过百岁而动作不衰老,这是因为他们的养生之道完备而无偏颇的缘故。\\
黄帝问道:人年老了,就不能再生育子女,是筋力不足呢?还是自然的生理变化规律就是这样的呢?\\
岐伯回答说:女子到了七岁,肾气开始充实,牙齿更换,头发生长。到了十四岁时,天癸发育成熟,任脉畅通,冲脉旺盛,月经按时而来,所以能够孕育子女。到了二十一岁,肾气平和,智齿生长,身高长到最高点。到了二十八岁,筋骨坚强,毛发长到了极点,身体非常强壮。到了三十五岁,阳明经脉开始衰微,面部开始枯槁,头发也开始脱落。到了四十二岁,三阳经脉之气从头部开始都衰退了,面部枯槁,头发变白。到了四十九岁,任脉空虚,太冲脉衰微,天癸枯竭,月经断绝,所以形体衰老,不能再生育儿女。\\
男子八岁时,肾气开始充实,头发生长,牙齿更换。到了十六岁时,肾气盛满,天癸发育成熟,精气充满,如男女交合,就能生育子女了。到了二十四岁,肾气平和,筋骨强劲,智齿生长,身高也长到最高了。到了三十二岁,筋骨粗壮,肌肉充实。到了四十岁,肾气开始衰退,头发开始脱落,牙齿干枯。到了四十八岁,人体上部阳明经气衰竭了,面色憔悴,发鬓斑白。到了五十六岁,肝气衰,筋脉迟滞,手足运动不灵活了。到了六十四岁,天癸枯竭,精气少,肾脏衰,牙齿头发脱落,身体感到为病所苦。人体的肾脏主水,它接受五脏六腑的精华以后贮存在里面,所以脏腑旺盛,肾脏才有精气排泄。现在年龄大了,五脏皆衰,筋骨无力,天癸竭尽,所以发鬓斑白,身体沉重,走路不稳,不能再生育子女。\\
黄帝问道:有人年纪已很大,还能生育子女,是什么道理?\\
岐伯说:这是因为他的先天禀赋超过了常人,气血经脉还畅通,而肾气有余。虽然能够生育,但在一般情况下,男子不超过六十四岁,女子不超过四十九岁,到这个岁数男女的精气都穷尽了。\\
黄帝问:养生有成的人,年纪都达百岁,能不能生育呢?\\
岐伯说:善于养生的人,能够推迟衰老,保全身体如壮年,所以即使年寿很高,仍然能生育。\\
黄帝说:我听说上古时代有真人,他能与天地阴阳自然消长变化的规律同步,自由地呼吸天地之间的精气,来保守精神,身体与精神合而为一。所以寿命就与天地相当,没有终了之时。这就是因得道而长生。\\
中古时代有至人,他道德淳朴完美,符合天地阴阳的变化。适应四时气候的变迁,避开世俗的喧闹。聚精会神,悠游于天地之间,所见所闻,能够广及八方荒远之外。这是能够延长寿命,身体强健的人。这种人也属于真人。\\
其次有圣人,能够安居平和的天地之间,顺从八风的变化规律,调整自己的爱好以适合世俗习惯,从来不生气。行为不脱离世俗,但举动又不仿效世俗而保有自己独特的风格。在外不使身体为事务所劳,在内不使思想有过重负担。以清静愉悦为本务,以悠然自得为目的。所以形体毫不衰老,精神也不耗散,年寿也可以达到百岁。\\
其次有贤人,能效法天地的变化,取象日月的升降。分辨星辰的运行,顺从阴阳的消长。根据四时气候的变化来调养身体,追随上古真人,以求合于养生之道,这样,也可以延长寿命而接近自然的天寿。\\
四气调神大论篇第二\\
春三月,此谓发陈。天地俱生,万物以荣。夜卧早起,广步于庭。被发缓形,以使志生。生而勿杀,予而勿夺,赏而勿罚。此春气之应,养生之道也。逆之则伤肝,夏为寒变。奉长者少。\\
夏三月,此谓蕃秀。天地气交,万物华实。夜卧早起,无厌于日。使志无怒,使华英成秀。使气得泄,若所爱在外。此夏气之应,养长之道也。逆之则伤心,秋为痎疟。奉收者少。\\
秋三月,此谓容平。天气以急,地气以明。早卧早起,与鸡俱兴。使志安宁,以缓秋刑。收敛神气,使秋气平。无外其志,使肺气清。此秋气之应,养收之道也。逆之则伤肺,冬为飧泄。奉藏者少。\\
冬三月,此谓闭藏。水冰地坼,无扰乎阳。早卧晚起,必待日光。使志若伏若匿,若有私意。若已有得,去寒就温。无泄皮肤,使气亟夺。此冬气之应,养藏之道也。逆之则伤肾,春为痿厥。奉生者少。\\
天气,清净光明者也,藏德不止,故不下也。天明则日月不明,邪害空窍。阳气者闭塞,地气者冒明。云雾不精,则上应白露不下。交通不表,万物命故不施,不施则名木多死。恶气不发,风雨不节,白露不下,则菀槁不荣。贼风数至,暴雨数起,天地四时不相保,与道相失,则未央绝灭。唯圣人从之,故身无奇病。万物不失,生气不竭。\\
逆春气,则少阳不生,肝气内变。逆夏气,则太阳不长,心气内洞。逆秋气,则少阴不收,肺气焦满。逆冬气,则太阴不藏,肾气独沉。夫四时阴阳者,万物之根本也。所以圣人春夏养阳,秋冬养阴,以从其根。逆其根,则伐其本,坏其真矣。故阴阳四时者,万物之终始也,死生之本也。逆之则灾害生,从之则苛疾不起。是谓得道。道者,圣人行之,愚者背之。从阴阳则生,逆之则死,从之则治,逆之则乱。反顺为逆,是谓内格。\\
是故圣人不治已病治未病,不治已乱治未乱,此之谓也。夫病已成而后药之,乱已成而后治之,譬犹渴而穿井,斗而铸兵,不亦晚乎?\\
春季三个月,是万物复苏的季节。大自然生机勃发,草木欣欣向荣。适应这种环境,应当夜卧早起,在庭院里散步。披开束发,舒缓身体,以使神志随着生发之气而舒畅。神志活动要顺应春生之气,而不要违逆它。这就与春生之气相应,是养生的方法。违背了这个方法,会伤肝,到了夏天就要发生寒变。这是因为春天生养的基础差,供给夏天成长的条件也就差了。\\
夏季三个月,是草木繁茂秀美的季节。天地阴阳之气上下交通,各种草木开花结果。适应这种环境,应该夜卧早起,不要厌恶白天太长。心中没有郁怒,使容色秀美。并使腠理宣通,如有为所爱之物吸引一样,使阳气疏泄于外。这就是与夏长之气相应,是养长的办法。如果违背了这个道理,会损伤心气,到了秋天就会患疟疾。这是因为夏天长养的基础差,供给秋天收敛的能力也就差了。\\
秋季三个月,是草木自然成熟的季节。天气劲急,地气清明。适应这种环境,应当早卧早起,和鸡同时活动。保持意志安定,从而舒缓秋天劲急之气对身体的影响。精神内守,不急不躁,使秋天肃杀之气得以平和。不使意志外驰,使肺气清和均匀。这就是与秋收之气相应,是养收的方法。如果违背了这个方法,会损伤肺气,到了冬天就要生飧泄病。这是因为秋天收敛的基础差,供给冬天潜藏之气的能力也就差了。\\
冬季三个月,是万物生机潜伏闭藏的季节。寒冷的天气,使河水结冰,大地冻裂。这时不能扰动阳气。适应这种环境,应该早睡晚起,一定等到太阳出来时再起床。使意志如伏似藏,好像心里很充实。好像已经得到满足,还要避开寒凉,保持温暖。不要让皮肤开张出汗,而频繁耗伤阳气。这就是与冬藏之气相应,是养藏的方法。如果违背了这个道理,会损伤肾气,到了春天,就要得痿厥病。这是因为冬天闭藏的基础差,供给春季生养的能力也就差了。\\
天气是清净光明的,天气潜藏着清净光明的生生之德,永远无尽,所以万物能长久生存而不会消亡。如果天德不藏,显露他的光明,日月就没有了光辉,如同外邪乘虚侵入孔窍,酿成灾害一样。流畅的阳气,就会闭塞不通,沉浊的地气,反而遮蔽光明。云雾弥漫不晴,那么,地气不得上应天气,甘露也就不能下降了。天地之气不能交流,万物的生命不得成长,这样名果珍木多亡。草木就枯槁,而不会茂盛了。邪气潜藏而不得散发,风雨失节,白露不降,草木枯槁不荣。邪风时时侵袭,暴雨不断袭击,春、夏、秋、冬不能保持相互间的平衡,与正常的规律相违背,这样的话,万物在生长的中途便都夭折了。只有圣人能够顺应自然变化,注意养生,所以身体没有重病。如果万物都不失保养之道,那么它的生命之气是不会衰竭的。\\
如果违背了春天之气,那么少阳之气就不能生发,会使肝气内郁而发生病变。如果违背了夏天之气,那么太阳之气就不能生长,会使心气内虚。如果违背了秋天之气,那么少阴之气就不能收敛,会使肺热叶焦而胀满。如果违背了冬天之气,那么太阴之气就不能潜藏,会使肾气衰弱。四时阴阳的变化,是万物生长收藏的根本。所以圣人顺应这个规律,在春夏保养心肝,在秋冬保养肺肾,以适应养生的根本原则。假如违背了这一根本原则,便会摧残本元,损坏身体。所以四时阴阳的变化,是万物生长收藏的由来,死生的本源。违背它,就要发生灾害;顺从它,就不会得重病。这样才可以说掌握了养生规律。不过这个养生规律,只有圣人能够奉行,愚昧的人却会违背。如果顺从阴阳变化的规律,就会生存,违背阴阳变化的规律,就会死亡;顺从这个规律就会安定,违背了,就要发生祸乱。如果不顺从阴阳四时的变化而违逆,就会生病,病名叫关格。\\
所以圣人不治已发生的病而倡导未病先防;不治理已形成的动乱而注重在未乱之前的疏导。假如疾病形成以后再去治疗,动乱形成以后再去治理,这就好像口渴才去挖井,发生战斗才去铸造兵器,那不是太晚了吗?\\
生气通天论篇第三\\
黄帝曰:夫自古通天者,生之本,本于阴阳。天地之间,六合之内,其气九州、九窍、五藏、十二节,皆通乎天气。其生五,其气三。数犯此者,则邪气伤人。此寿命之本也。\\
苍天之气,清净则志意治,顺之则阳气固。虽有贼邪,弗能害也。故圣人传精神,服天气而通神明。失之则内闭九窍,外壅肌肉,卫气散解,此谓自伤,气之削也。\\
阳气者,若天与日,失其所则折寿而不彰。故天运当以日光明,是故阳因而上,卫外者也。\\
因于寒,欲如运枢,起居如惊,神气乃浮。因于暑,汗,烦则喘喝,静则多言,体若燔炭,汗出乃散。因于湿,首如裹,湿热不攘,大筋仩短,小筋弛长,仩短为拘,弛长为痿。因于气,为肿,四维相代,阳气乃竭。\\
阳气者,烦劳则张,精绝,辟积于夏,使人煎厥。目盲不可以视,耳闭不可以听,溃溃乎若坏都,汩汩乎不可止。阳气者,大怒则形气绝,而血菀于上,使人薄厥。有伤于筋,纵,其若不容。汗出偏沮,使人偏枯。汗出见湿,乃生痤疿。高梁之变,足生大疽,受如持虚。劳汗当风,寒薄为皶,郁乃痤。\\
阳气者,精则养神,柔则养筋。开阖不得,寒气从之,乃生大偻。营气不从,逆于肉理,乃生痈肿。陷脉为瘘,留连肉腠。俞气化薄,传为善畏,及为惊骇。魄汗未尽,形弱而气烁,穴俞以闭,发为风疟。\\
故风者,百病之始也,清静则肉腠闭,阳气拒,虽有大风苛毒,弗之能害。此因时之序也。\\
故病久则传化,上下不并,良医弗为。故阳畜积病死,而阳气当隔,隔者当泻,不亟正治,粗乃败亡。故阳气者,一日而主外,平旦阳气生,日中而阳气隆,日西而阳气已虚,气门乃闭。是故暮而收拒,无扰筋骨,无见雾露。反此三时,形乃困薄。\\
岐伯曰:阴者,藏精而起亟也;阳者,卫外而为固也。阴不胜其阳,则脉流薄疾,并乃狂;阳不胜其阴,则五脏气争,九窍不通。是以圣人陈阴阳,筋脉和同,骨髓坚固,气血皆从。如是则内外调和,邪不能害,耳目聪明,气立如故。\\
风客淫气,精乃亡,邪伤肝也。因而饱食,筋脉横解,肠澼为痔。因而大饮,则气逆。因而强力,肾气乃伤,高骨乃坏。\\
凡阴阳之要,阳密乃固。两者不和,若春无秋,若冬无夏。因而和之,是谓圣度。故阳强不能密,阴气乃绝;阴平阳秘,精神乃治;阴阳离决,精气乃绝。\\
因于露风,乃生寒热。是以春伤于风,邪气留连,乃为洞泄;夏伤于暑,秋为痎疟;秋伤于湿,冬逆而咳,发为痿厥;冬伤于寒,春必病温。四时之气,更伤五脏。\\
阴之所生,本在五味,阴之五宫,伤在五味。是故味过于酸,肝气以津,脾气乃绝;味过于咸,大骨气劳,短肌,心气抑;味过于甘,心气喘满,肾气不衡;味过于苦,脾气濡,胃气乃厚;味过于辛,筋脉沮弛,精神乃央。是故谨和五味,骨正筋柔,气血以流,腠理以密,如是则骨气以精。谨道如法,长有天命。\\
黄帝说:自古以来人的生命活动与自然界的变化就是息息相通的,这是生命的根本,生命的根本就是阴阳。在天地之间,四方上下之内,无论是地之九州,还是人的九窍、五脏、十二节,都与自然之气相通。天之阴阳化生地之五行之气,地之五行又上应天之三阴三阳。如果经常违反阴阳变化的规律,那么邪气就会伤害人体。所以说阴阳是寿命的根本。\\
自然界的天气清净,人的意志就平和,顺应这个道理,阳气就固密。即使有贼风邪气,也不能侵害人体。所以善于养生的圣人,能够聚集精神,呼吸天地精气,而与天地阴阳的神明变化相统一。如果违背这个道理,在内会使九窍不通,在外会使肌肉壅阻,卫阳之气耗散,这是自己造成的伤害,而使阳气受到削弱。\\
人体的阳气,就像天上的太阳一样,太阳不能在其轨道上正常运行,万物就不能生存;人体的阳气不能正常运行于人体,就会缩短寿命而不能使生命成长壮大。所以天体运行不息,是借着太阳的光明,同理,人体健康无病,是依赖阳气的轻清上浮保卫于体表。\\
人感受了寒邪,阳气就会像门户的开阖一样相应抗拒,起居不宁;如果起居妄动,神气浮越,阳气就不能固密了。如果感受暑邪,就会多汗,烦躁,甚至喘促,喝喝有声;及至暑邪伤气,即使不烦喘时,也会多言多语,身体发热如炭烧,必须出汗,热才能退。如果伤于湿邪,头部就会沉重,如同裹着东西,如果湿热不能及时排除,就会出现大筋收缩不伸,小筋弛缓无力。大筋收缩不伸叫拘,小筋弛缓无力叫痿。如果气被风邪所缚,发为气肿,四肢交替肿痛不休,这是阳气已衰竭了。\\
人体的阳气,由于过度烦劳,就会亢盛外越,导致阴精耗竭,病拖延到了夏天,就容易使人发生煎厥病。主要症状是眼睛昏蒙看不清东西,耳朵闭塞听不见声音,病势危急,就像湖水溃决,流速迅急,不可遏止。人体的阳气,大怒时会造成形与气隔绝,血郁积头部,使人发生暴厥。大怒之后不发暴厥之证的,那就会伤筋。筋受伤,会弛缓不收,肢体行动不自由。半身汗出的,会发生偏枯病。汗出以后感受湿邪,会发生小疖和汗疹。多吃肥甘厚味,能够使人生大疽,发病就像拿着空器皿盛东西一样容易。劳动之后,汗出当风,寒气阻遏于皮肤,会成为粉刺,郁积不解,可成为疮疖。\\
人体的阳气,养神则精微,养筋则柔软。如果腠理开阖失调,寒邪乘机侵入,就会发生背部屈曲的大偻病。如果寒气入于经脉,营气不能顺着经脉走,阻滞在肌肉之中,会发生臃肿。邪气留滞在肌肉纹理,日久深入血脉,可以形成瘘疮。外邪从背部腧穴侵及脏腑,会出现善畏和惊骇之证。汗出不透,形体衰弱,阳气消耗,腧穴闭塞,就会发生风疟。\\
风是引发各种疾病的始因,但是,只要精神安静,意志安定,腠理就能闭密,阳气就能卫外,即使有大风苛毒,也不能造成伤害。这是顺应四时气候变化规律来养生的结果。\\
所以病的时间长了,就会传导变化,发生其他症候;如果病人上下之气不能交通,再高明的医生,也无能为力了。人的阳气过分蓄积,也会致死,因为阳气蓄积,隔塞不通,应该用泻法。如果不赶紧治疗,水平低下的医工就会败亡人体正气而致病人死亡。人身的阳气,白天都运行于人体外部,日出时人体的阳气开始生发,中午阳气最旺盛,到日落时阳气衰退,汗孔也就关闭了。这时,就应当休息,阳气收藏于内而拒邪气于外,不要扰动筋骨,不要冒犯雾露,如果违反了平旦、日中、日暮阳气的活动规律,形体就会为邪气所困,而日趋衰弱。\\
岐伯说:阴是把精气蓄藏于体内,而不断充养阳气;阳是保卫人体外部而坚固腠理的。如果阴不胜阳,那么经脉往来流动就会急迫快速,而发为狂病;如果阳不胜阴,那么五脏之气就会不调,以致九窍不通。所以圣人调整阴阳,使之各安其位,才能筋脉舒和,骨髓坚固,气血畅通。这样内外阴阳之气调和,邪气不能侵害,耳聪目明,真气运行正常。\\
风邪侵入人体,渐渐损害元气,精血就要消亡,这是由于邪气伤害了肝脏。这时,如果再过饱,会使胃的筋脉横逆弛缓,而形成下泻脓血的痢疾,进而引发痔疮。如果饮酒过度,肺气就会上逆。如果勉强入房,就要损伤肾气,使脊椎骨损坏。\\
大凡阴阳的关键,在于阳气固密于外,阴气才能持守于内。如果阴阳失去平衡和谐,就像一年当中,只有春天没有秋天,只有冬天没有夏天一样。因此,调和阴阳,是最好的养生方法。如果阳气过于亢盛,不能固密,阴气就要亏耗而衰竭;阴气和平,阳气固密,精神就会旺盛;如果阴阳分离而不相交,那精气也就随之而耗竭了。\\
如果触冒风邪,就会发生寒热。所以,春天被风邪所伤,邪气留滞不去,到了夏天,就会生洞泄病;夏天被暑邪所伤,潜伏于内,到了秋天,就会发生疟疾;秋天被湿邪所伤,到了冬天,就会气逆而痰咳,进而发展为痿厥病;冬天被寒邪所伤害,到了春天,必然发生温热病。风寒暑湿这些四时邪气,会交替伤害五脏。\\
阴精的产生,来源于饮食五味的营养,但是,贮藏精血的五脏,又因为过食五味而受伤害。所以过食酸味,会使肝气集聚,脾气就会衰弱;过食咸味,会使骨气受伤,肌肉枯槁,心气也就郁滞了;过食甜味,会使心气喘闷,肾气就衰弱了;过食苦味,会使脾气濡滞,胃气也就薄弱了;过食辛味,会使筋脉渐渐衰败,精神也就颓废了。所以谨慎地调和五味,使得骨骼正直,筋脉柔和,气血流通,腠理固密,这样,就会气精骨强了。谨慎地按照养生之道的法则去做,就可以享受自然的寿命。\\
金匮真言论篇第四\\
黄帝问曰:天有八风,经有五风,何谓?\\
岐伯对曰:八风发邪,以为经风,触五脏,邪气发病。所谓得四时之胜者,春胜长夏,长夏胜冬,冬胜夏,夏胜秋,秋胜春。所谓四时之胜也。\\
东风生于春,病在肝,俞在颈项;南风生于夏,病在心,俞在胸胁;西风生于秋,病在肺,俞在肩背;北风生于冬,病在肾,俞在腰股;中央为土,病在脾,俞在脊。\\
故春气者病在头,夏气者病在脏,秋气者病在肩背,冬气者病在四支。\\
故春善病鼽衄,仲夏善病胸胁,长夏善病洞泄寒中,秋善病风疟,冬善病痹厥。\\
故冬不按\\
,春不鼽衄,春不病颈项,仲夏不病胸胁,长夏不病洞泄寒中,秋不病风疟,冬不病痹厥、飧泄而汗出也。\\
夫精者,身之本也。故藏于精者,春不病温。夏暑汗不出者,秋成风疟。\\
故曰:阴中有阴,阳中有阳。平旦至日中,天之阳,阳中之阳也;日中至黄昏,天之阳,阳中之阴也;合夜至鸡鸣,天之阴,阴中之阴也;鸡鸣至平旦,天之阴,阴中之阳也。故人亦应之。\\
夫言人之阴阳,则外为阳,内为阴。言人身之阴阳,则背为阳,腹为阴。言人身之脏腑中阴阳,则脏者为阴,腑者为阳。肝心脾肺肾五脏皆为阴,胆胃大肠小肠膀胱三焦六腑皆为阳。所以欲知阴中之阴、阳中之阳者,何也?为冬病在阴,夏病在阳;春病在阴,秋病在阳。皆视其所在,为施针石也。故背为阳,阳中之阳,心也;背为阳,阳中之阴,肺也;腹为阴,阴中之阴,肾也;腹为阴,阴中之阳,肝也;腹为阴,阴中之至阴,脾也。此皆阴阳、表里、内外、雌雄相输应也。故以应天之阴阳也。\\
帝曰:五脏应四时,各有攸受乎?\\
岐伯曰:有。东方青色,入通于肝。开窍于目,藏精于肝,故病在头。其味酸,其类草木,其畜鸡,其谷麦。其应四时,上为岁星,是以知病之在筋也。其音角,其数八,其臭臊。\\
南方赤色,入通于心。开窍于舌,藏精于心,故病在五脏。其味苦,其类火,其畜羊,其谷黍。其应四时,上为荧惑星,是以知病之在脉也。其音徵,其数七,其臭焦。\\
中央黄色,入通于脾。开窍于口,藏精于脾,故病在脊。其味甘,其类土,其畜牛,其谷稷。其应四时,上为镇星,是以知病之在肉也。其音宫,其数五,其臭香。\\
西方白色,入通于肺。开窍于鼻,藏精于肺,故病在背。其味辛,其类金,其畜马,其谷稻。其应四时,上为太白星,是以知病之在皮毛也。其音商,其数九,其臭腥。\\
北方黑色,入通于肾。开窍于二阴,藏精于肾,故病在谿。其味咸,其类水,其畜彘,其谷豆。其应四时,上为辰星,是以知病之在骨也。其音羽,其数六,其臭腐。\\
故善为脉者,谨察五脏六腑,逆从、阴阳、表里、雌雄之纪,藏之心意,合心于精。非其人勿教,非其真勿授,是谓得道。\\
黄帝问道:天有八方之风,人的经脉有五脏之风,是指什么呢?\\
岐伯回答说:八风会产生致病的邪气,侵犯经脉的风邪,触动人的五脏,因而发病。所说的感受四时季节相克的情况是指,春胜长夏,长夏胜冬,冬胜夏,夏胜秋,秋胜春。这就是所说的四时季节相克。\\
东风生于春季,病变多发生在肝经,而表现于颈项;南风生于夏季,病变常发生在心经,而表现于胸胁;西风生于秋季,病变常发生在肺经,而表现于肩背;北风生于冬季,病变常发生在肾经,而表现于腰股;中央属土,病变常发生在脾经,而表现于脊背。\\
所以春气为病,多在头部;夏气为病,多在心;秋气为病,多在肩背;冬气为病,多在四肢。\\
所以春天多生鼻流清涕和鼻出血的病,夏仲多生胸胁病,长夏多生里寒洞泄病,秋天多生风疟病,冬天多生痹病。\\
所以冬天不做剧烈运动而扰动潜伏的阳气,春天就不会发生鼽衄,不发生颈项病,夏仲也不会发生胸胁病,长夏不会发生里寒洞泄病,秋天不会发生风疟病,冬天也不会发生痹证、飧泄、汗出过多的病。\\
精对人体就如同树木的根,是生命的源泉。所以冬季善于保养精气的,春天就不易得温病。夏天暑热之时,应该出汗而不出汗,到了秋天就会得风疟病。\\
所以说:阴中有阴,阳中有阳。从清晨至中午,自然界的阳气是阳中之阳;从中午至黄昏,自然界的阳气是阳中之阴;从日落到半夜,自然界的阴气是阴中之阴;从半夜到清晨,自然界的阴气是阴中之阳。所以人的阴阳之气也是如此。\\
就人体阴阳来说,外部为阳,内部为阴。单就身体部位来说,背为阳,腹为阴。就脏腑来说,脏属阴,腑属阳。肝、心、脾、肺、肾五脏都属阴;胆、胃、大肠、小肠、膀胱、三焦、六腑都属阳。为什么要知道阴中有阴、阳中有阳的道理呢?这因为冬病发生在阴,夏病发生在阳;春病发生在阴,秋病发生在阳。都要根据疾病所在部位来进行针刺或砭石治疗。所以说,背部为阳,阳中之阳为心;背部为阳,阳中之阴为肺;腹部为阴,阴中之阴为肾;腹部为阴,阴中之阳为肝;腹部为阴,阴中之至阴为脾。这些都是人体阴阳、表里、内外、雌雄的相应关系。它们合于自然界的阴阳变化。\\
黄帝说:五脏与四时相对应,都各有所用吗?\\
岐伯答:有。东方青色,和肝相应。肝开窍于目,精华藏于肝脏,发病多在头部。比象来说,在五味中为酸味,在五行中为木,在五畜中为鸡,在五谷中为麦。在四时中上应于岁星,所以肝病多发生在筋。在五音中为角音,在五行生成数中为八,在五气中为腥臊。\\
南方赤色,和心相应。心开窍于舌,精华藏在心,发病多在五脏。比象来说,在五味中为苦味,在五行中为火,在五畜中为羊,在五谷中为黍。在四时中上应于荧惑星,所以心病多发生在血脉。在五音中为徵音,在五行生成数中为七,在五气中为焦。\\
中央黄色,和脾相应。脾开窍于口,精华藏在脾脏,发病多在脊部。比象来说,在五味中为甘味,在五行中为土,在五畜中为牛,在五谷中为稷。在四时中上应于土星,所以脾病多发生在肌肉。在五音中为宫音,在五行生成数中为五,在五气中为香。\\
西方白色,与肺相应。肺开窍于鼻,精华藏在肺脏,发病多在背部。比象来说,在五味中为辛味,在五行中为金,在五畜中为马,在五谷中为稻。在四时中上应金星,所以病多发生在皮毛。在五音中为商音,在五行生成数中为九,在五气中为腥。\\
北方黑色,与肾相应。肾开窍于二阴,精华藏在肾脏,发病多在四肢。比象来说,在五味中为咸味,在五行中为水,在五畜中为猪,在五谷中为豆。在四时中上应于水星,所以肾有病会发生在骨骼。在五音中为羽音,在五行生成数中为六,在五气中为腐。\\
所以善于诊脉的医生,小心地审察五脏六腑的气血逆顺,以及阴阳、表里、雌雄的所以然,把这些道理牢记于心中,用心精思以知常处变,灵活运用。这样的脉学是宝贵的,但不要传授给不适当的人,不是真正的医学理论也不要向人传授,这才是医学传授之道。\\
卷二\\
阴阳应象大论篇第五\\
黄帝曰:阴阳者,天地之道也,万物之纲纪,变化之父母,生杀之本始,神明之府也,治病必求于本。故积阳为天,积阴为地。阴静阳躁,阳生阴长,阳杀阴藏。阳化气,阴成形,寒极生热,热极生寒。寒气生浊,热气生清。清气在下,则生飧泄。浊气在上,则生\\
胀。此阴阳反作,病之逆从也。\\
故清阳为天,浊阴为地。地气上为云,天气下为雨。雨出地气,云出天气。故清阳出上窍,浊阴出下窍。清阳发腠理,浊阴走五脏。清阳实四支,浊阴归六腑。\\
水为阴,火为阳。阳为气,阴为味。味归形,形归气。气归精,精归化。精食气,形食味。化生精,气生形。味伤形,气伤精。精化为气,气伤于味。\\
阴味出下窍,阳气出上窍。味厚者为阴,薄为阴之阳。气厚者为阳,薄为阳之阴。味厚则泄,薄则通。气薄则发泄,厚则发热。壮火之气衰,少火之气壮。壮火食气,气食少火。壮火散气,少火生气。气味,辛、甘发散为阳,酸、苦涌泄为阴。\\
阴胜则阳病,阳胜则阴病。阳胜则热,阴胜则寒。重寒则热,重热则寒。寒伤形,热伤气。气伤痛,形伤肿。故先痛而后肿者,气伤形也;先肿而后痛者,形伤气也。风胜则动,热胜则肿,燥胜则干,寒胜则浮,湿胜则濡泻。\\
天有四时五行,以生长收藏,以生寒暑燥湿风。人有五脏化五气,以生喜怒悲忧恐。故喜怒伤气,寒暑伤形;暴怒伤阴,暴喜伤阳。厥气上行,满脉去形。喜怒不节,寒暑过度,生乃不固。故重阴必阳,重阳必阴。故曰:冬伤于寒,春必温病;春伤于风,夏生飧泄;夏伤于暑,秋必痎疟;秋伤于湿,冬生咳嗽。\\
帝曰:余闻上古圣人,论理人形,列别脏腑;端络经脉,会通六合,各从其经;气穴所发,各有处名;谿谷属骨,皆有所起;分部逆从,各有条理;四时阴阳,尽有经纪。外内之应,皆有表里。其信然乎?\\
岐伯对曰:东方生风,风生木,木生酸,酸生肝,肝生筋,筋生心。肝主目。其在天为风,在地为木,在体为筋,在藏为肝,在色为苍,在音为角,在声为呼,在变动为握,在窍为目,在味为酸,在志为怒。怒伤肝,悲胜怒;风伤筋,燥胜风;酸伤筋,辛胜酸。\\
南方生热,热生火,火生苦,苦生心,心生血,血生脾。心主舌。其在天为热,在地为火,在体为脉,在脏为心,在色为赤,在音为徵,在声为笑,在变动为忧,在窍为舌,在味为苦,在志为喜。喜伤心,恐胜喜;热伤气,寒胜热;苦伤气,咸胜苦。\\
中央生湿,湿生土,土生甘,甘生脾,脾生肉,肉生肺。脾主口。其在天为湿,在地为土,在体为肉,在藏为脾,在色为黄,在音为宫,在声为歌,在变动为哕,在窍为口,在味为甘,在志为思。思伤脾,怒胜思;湿伤肉,风胜湿;甘伤肉,酸胜甘。\\
西方生燥,燥生金,金生辛,辛生肺,肺生皮毛,皮毛生肾。肺主鼻。其在天为燥,在地为金,在体为皮毛,在藏为肺,在色为白,在音为商,在声为哭,在变动为咳,在窍为鼻,在味为辛,在志为忧。忧伤肺,喜胜忧;热伤皮毛,寒胜热;辛伤皮毛,苦胜辛。\\
北方生寒,寒生水,水生咸,咸生肾,肾生骨髓,髓生肝。肾主耳。其在天为寒,在地为水,在体为骨,在藏为肾,在色为黑,在音为羽,在声为呻,在变动为栗,在窍为耳,在味为咸,在志为恐。恐伤肾,思胜恐;寒伤血,燥胜寒;咸伤血,甘胜咸。\\
故曰:天地者,万物之上下也;阴阳者,血气之男女也;左右者,阴阳之道路也;水火者,阴阳之征兆也;阴阳者,万物之能始也。故曰:阴在内,阳之守也;阳在外,阴之使也。\\
帝曰:法阴阳奈何?\\
岐伯曰:阳胜则身热,腠理闭,喘粗为之俯仰。汗不出而热,齿干以烦冤,腹满死。能冬不能夏。阴胜则身寒,汗出,身常清,数栗而寒,寒则厥,厥则腹满死。能夏不能冬。此阴阳更胜之变,病之形能也。\\
帝曰:调此二者,奈何?\\
岐伯曰:能知七损八益,则二者可调;不知用此,则早衰也。年四十,而阴气自半也,起居衰矣;年五十,体重,耳目不聪明矣;年六十,阴痿,气大衰,九窍不利,下虚上实,涕泣俱出矣。故曰:知之则强,不知则老,故同出而名异耳。智者察同,愚者察异。愚者不足,智者有余。有余则耳目聪明,身体轻强,老者复壮,壮者益治。是以圣人为无为之事,乐恬惔之能,从欲快志于虚无之守,故寿命无穷,与天地终。此圣人之治身也。\\
天不足西北,故西北方阴也,而人右耳目不如左明也。地不满东南,故东南方阳也,而人左手足不如右强也。\\
帝曰:何以然?\\
岐伯曰:东方阳也,阳者其精并于上,并于上则上明而下虚,故使耳目聪明而手足不便也。西方阴也,阴者其精并于下,并于下则下盛而上虚,故其耳目不聪明而手足便也。故俱感于邪,其在上则右甚,在下则左甚,此天地阴阳所不能全也,故邪居之。\\
故天有精,地有形。天有八纪,地有五里。故能为万物之父母。清阳上天,浊阴归地。是故天地之动静,神明为之纲纪。故能以生长收藏,终而复始。惟贤人上配天以养头,下象地以养足,中傍人事以养五脏。天气通于肺,地气通于嗌,风气通于肝,雷气通于心,谷气通于脾,雨气通于肾。六经为川,肠胃为海,九窍为水注之气。以天地为之阴阳,人之汗,以天地之雨名之;人之气,以天地之疾风名之。暴气象雷,逆气象阳。故治不法天之纪,不用地之理,则灾害至矣。\\
故邪风之至,疾如风雨,故善治者治皮毛,其次治肌肤,其次治筋脉,其次治六腑,其次治五脏。治五脏者,半死半生也。故天之邪气,感则害人五脏;水谷之寒热,感则害于六腑;地之湿气,感则害皮肉筋脉。\\
故善用针者,从阴引阳,从阳引阴。以右治左,以左治右。以我知彼,以表知里,以观过与不及之理。见微得过,用之不殆。\\
善诊者,察色按脉,先别阴阳。审清浊,而知部分;视喘息,听音声,而知所苦;观权衡规矩,而知病所主;按尺寸,观浮沉滑涩,而知病所生。以治无过,以诊则不失矣。\\
故曰:病之始起也,可刺而已;其盛,可待衰而已。故因其轻而扬之,因其重而减之,因其衰而彰之。形不足者,温之以气;精不足者,补之以味。其高者,因而越之;其下者,引而竭之;中满者,泻之于内;其有邪者,渍形以为汗;其在皮者,汗而发之;其慓悍者,按而收之;其实者,散而泻之。审其阴阳,以别柔刚。阳病治阴,阴病治阳。定其血气,各守其乡,血实宜决之,气虚宜掣引之。\\
黄帝说:阴阳,是天地间的普遍规律,是一切事物的纲领,是万物发展变化的起源,是生长毁灭的根本,是万物发生发展变化的动力源泉,因此,治病必须寻求治本的方法。清阳之气,积聚上升,就成为天;浊阴之气,凝聚下降,就成为地。阴主静,阳主动,阳主发生,阴主成长,阳主杀伐,阴主收藏。阳能化生力量,阴能构成形体。寒到极点会转化生热,热到极点会转化生寒。寒气的凝聚,能产生浊阴,热气的升腾可产生清阳。清阳之气在下,如不得上升,就会发生飧泄。浊阴之气在上,如不得下降,就会发生胀满。这是违反了阴阳运行规律,因此疾病也有顺证和逆证的不同。\\
在自然界中,清阳之气变为天,浊阴之气变成地。地气上升就成为云,天气下降就变成雨。雨虽从天气下降,却是地气所化;云虽形成于地气,却赖天气的蒸发。这些都是由于阴阳相互转化造成的。同样,在人体的变化中,清阳出于上窍,浊阴出于下窍。清阳发散于腠理,浊阴注入于五脏。清阳使四肢得以充实,浊阴使六腑能够相安。\\
水属于阴,火属于阳。阳是无形的气,而阴则是有形的味。饮食五味进入身体中的胃腑,胃能够腐熟蒸化出水谷中的清气。清气进入五脏与五脏中的精气结合,而化生人体生命的营养物质。精仰赖水谷清气的补养,形体仰赖饮食五味的补给。饮食经过生化生成精,精气化后来充养形体。饮食不节,也能伤害形体,气偏盛,也能损伤精。精血充足,又能够化而为气,气也能被五味太过所伤害。\\
味属阴,所以趋向下窍;气属阳,所以趋向上窍。五味之中,味厚的属于纯阴,味薄的属于阴中之阳。气厚的属于纯阳,气薄的属于阳中之阴。味厚的有泄下作用,味薄的有疏通作用。气薄的能够向外发泄邪气,气厚的能助阳发热。亢阳能使元气衰弱,微阳能使元气旺盛。因为亢阳会侵蚀元气,而元气有赖于微阳的煦养。亢阳耗散元气,微阳却使元气增强。气味之中,辛甘而有发散作用的属阳,酸苦而有涌泄作用的属阴。\\
阴气偏胜,阳气就会受病;阳气偏胜,阴气也会受病。阳气偏胜会生热,阴气偏胜会生寒。寒到极点,会出现热象;热到极点,又会出现寒象。寒邪会损伤人的形体,热邪会损伤人的真气。真气受伤会产生疼痛,形体受伤会发生肿胀。凡是先疼后肿的,是因为真气先伤而影响到形体;先肿后痛的,则是形体先伤而影响真气。风邪太过,就会发生痉挛动摇;热邪太过,肌肉就会发生红肿;燥邪太过,津液就会干涸;寒邪太过,就会发生浮肿;湿邪太过,就会发生泄泻。\\
自然界有春夏秋冬四时的推移、五行的变化,形成了生长收藏的规律,产生了寒暑燥湿风的气候。人有五脏,五脏化生五气,产生喜怒悲忧恐五种情志。所以过喜过怒可以伤气,寒暑外侵,会损伤形体;大怒会伤阴气,大喜会伤阳气。如果逆气上冲,血脉阻塞,也会神气浮越,离形体而去。因此,不节制喜怒,不调适寒暑,生命就不会稳固。阴气过盛会转化为阳,阳气过盛也会转变为阴。所以说:冬天感受寒气过多,到了春天就容易发生热性病;春天感受风气过多,到了夏天就容易发生飧泄;夏天感受暑气过多,到了秋天就容易发生疟疾;秋天感受湿气过多,到了冬天就容易发生咳嗽。\\
黄帝问道:我听说古代圣人,谈论人体的形态,排列辨别脏腑的阴阳;联系会通四方上下六合,来审察十二经脉阴阳六合的起止循行与络属关系;气穴各有它所发的部位和名称;连属于骨骼的“谿谷”,都有它们的起止点;皮部浮络的属阴属阳,为顺为逆,也各有条理;四时阴阳变化,有一定规律;外在环境与人体内部的对应关系也都有表有里。真是这样吗?\\
岐伯回答说:东方属春,阳气上升而生风,风能滋养木气,木气能生酸味,酸味能养肝,肝血又能养筋,筋又能养心。肝气上通于目。它的变化,在天为六气中的风,在地为五行中的木,在人体中为筋,在五脏中为肝,在五色中为苍,在五音中为角,在五声中为呼,在人体的变动中为握,在七窍中为目,在五味中为酸,在情志中为怒。怒能伤肝,但悲伤可以抑制怒;风气能伤筋,但燥可以抑制风;过食酸味能伤筋,但辛味可以抑制酸味。\\
南方属夏,阳气大盛而生热,热能生火,火气能产生苦味,苦味能养心,心又能生血,血又能养脾。心气上通于舌。它的变化,在天为六气中的热,在地为五行中的火,在人体中为血脉,在五脏中为心,在五色中为赤,在五音中为徵,在五声中为笑,在人体的变动中为忧,在七窍中为舌,在五味中为苦,在情志中为喜。过喜能伤心,但恐可以抑制喜;热能伤气,但寒气可以抑制热;过食苦味能伤气,但咸味可以抑制苦味。\\
中央属长夏,蒸发而生湿,湿能使土气生长,土能产生甘味,甘味能滋养脾气,脾气又能够滋养肌肉,肌肉健壮又能使肺气充实。脾气上通于口。它的变化,在天为六气中的湿,在地为五行中的土,在人体中为肌肉,在五脏中为脾,在五色中为黄,在五音中为宫,在五声中为歌,在人体的变动中为干哕,在七窍中为口,在五味中为甘,在情志中为思。思虑能伤脾,但怒可以抑制思虑;湿气能伤肌肉,但风气可以抑制湿气;过食甘味能伤肌肉,但酸味可以抑制甘味。\\
西方属秋,天气劲急而生燥,燥能使金气旺盛,金能产生辛味,辛味能直通肺气,肺气又能滋养皮毛,皮毛润泽又能滋生肾水。肺气上通于鼻。它的变化,在天为六气中的燥,在地为五行中的金,在人体中为皮毛,在五脏中为肺,在五色中为白,在五音中为商,在五声中为哭,在人体的变动中为咳,在七窍中为鼻,在五味中为辛,在情志中为忧。忧能伤肺,但喜可以抑制忧;热能伤皮毛,但寒可以抑制热;过食辛味能伤皮毛,但苦味可以抑制辛味。\\
北方属冬,阴凝而生寒,寒气能使水气旺,水能产生咸味,咸味能滋养肾气,肾气又能滋养骨髓,骨髓充实又能养肝。肾气上通于耳。它的变化,在天为六气中的寒,在地为五行中的水,在人体中为骨髓,在五脏中为肾,在五色中为黑,在五音中为羽,在五声中为呻吟,在人体的变动中为战栗,在七窍中为耳,在五味中为咸,在情志中为恐。恐能伤肾,但思可以抑制恐;寒能伤骨,但燥可以抑制寒;过食咸味能伤骨,但甘味可以抑制咸。\\
所以说:天地上下是负载万物的区宇;阴阳是化生气血,形成雌雄生命体的动源;左右是阴阳运行的道路;而水火则是阴阳的表现;总之,阴阳的变化,是一切事物生成的原始。再进一步说:阴阳是相互为用的。阴在内,有阳作为它的卫外;阳在外,有阴作为它的辅助。\\
黄帝问:人怎样取法阴阳呢?\\
岐伯回答说:阳气太过,身体就会发热,腠理紧闭,喘息急迫,呼吸困难,身体俯仰摆动。手脚厥冷汗出不来并且发热,牙齿干燥,并且心里烦闷,再有腹部胀满,就是死证。患者耐受得冬天,而耐受不得夏天。阴气太过,身体就会恶寒,出汗,身上时常觉冷,甚或时常打寒战,寒重就会出现手足厥冷,手足厥冷之后再有腹部胀满,就是死证。患者耐受得夏天,而耐受不得冬天。这就是阴阳偏胜,所引起疾病的症状。\\
黄帝问:怎样调和阴阳呢?\\
岐伯回答说:能够知道七损八益的道理,就可以调和阴阳;不知道这个道理,就会早衰。人到四十岁,阴气已经减损了一半,起居动作显得衰退了;到五十岁,身体笨重,耳不聪,目不明;到六十岁,阴痿,气大衰,九窍功能减退,下虚上实,流鼻涕、淌眼泪等衰老现象都出现了。所以说:懂得养生的人,身体就强健,不懂得养生的人,身体就容易衰老,因此,同时出生,来到世上生活,最后的结果名称却不相同。聪明人,在没病时,就注意养生;愚蠢的人,在发病时,才知道调养。愚蠢的人,常感到体力不足,聪明的人却感到精力有余。精力有余,就会耳聪目明,身体轻捷强健,即使年老了,还显得健壮,强壮的人就更加强健了。所以明达事理的人,顺乎自然而不做无益于养生的事,以恬静的心情为快乐,持守虚无之道,追寻心志的快乐与自由,因此,他的寿命无穷尽,与天地长存。这就是圣人的养生方法。\\
天气在西北方不足,所以西北方属阴,而人与天气相应,右边的耳目也就不如左边的聪明。地气在东南方是不满的,所以东南方属阳,人左边的手足也就不如右边灵活。\\
黄帝问道:这是什么道理?\\
岐伯回答说:东方属阳,阳气的精华聚合在上部,聚合在上部,上部就旺盛了,而下部就必然虚弱了,所以会出现耳聪目明,而手足不便利的情况。西方属阴,阴气的精华聚合在下部,聚合在下部,下部就旺盛,上部就必然虚弱了。所以就会出现耳不聪目不明,而手足却便利的情况。所以,同样感受外邪,如果在上部,那么身体右侧严重,如果在下部,那么身体左侧严重,这是由于天地阴阳之气的分布不均衡,而在人身也是如此,身体阴阳之气偏虚的地方,就是邪气滞留的所在。\\
所以天有精气,地有形质。天有八节的气序,地有五方的布局。因此,天地能成为万物生长的根本。清阳上起升于天,浊阴下降归于地。所以天地的运动和静止,是由阴阳的神妙变化而决定的。因而能使万物春生、夏长、秋收、冬藏,循环往复,永不休止。只有圣贤之人,对上与天气相配合来养护头;对下与地气相顺来养护足;居中,则依傍人事来养护五脏。天气与肺相通,地气与咽相通,风气与肝相通,雷气与心相通,谷气与脾相通,雨气与肾相通。六经好像大河,肠胃好像大海,九窍好像水流。如果以天地的阴阳比喻人身的阴阳,那么,人的汗,就好像天地间的雨;人的气,就好像天地间的疾风。人的暴怒之气,就好像雷霆;人的逆气,就好像久晴不雨。所以养生不取法于天地之理,那么疾病灾害就要发生了。\\
外界邪风到来,迅猛如急风暴雨,所以善于治病的医生,能在病邪刚侵入皮毛时,就给以治疗;医术稍差的,在病邪侵入到肌肤时才治疗;更差的,在病邪侵入到筋脉时才治疗;再差的,在病邪侵入到六腑时才治疗;最差的,在病邪侵入到五脏时才治疗。病邪侵入到五脏,治愈的希望与死亡的可能各占一半。如果感受了天的邪气,就会伤害五脏;如果感受了饮食的或寒或热,就会伤害六腑;如果感受了地的湿气,就会伤害皮肉筋脉。\\
所以善于运用针刺的人,有时要从阴引阳,有时要从阳引阴。取右边穴以治左边的病,取左边穴以治右边的病。用自己的正常状态比较病人的异常状态;从在表的症状去了解在里的病变,这是为了观察病人的太过和不及的原因。发现病人的细微变化,就能够诊断疾病,用来指导治疗实践就不会有危险了。\\
善于治病的医生,看病人的面色,按病人脉象,首先要辨别疾病属阴还是属阳。审察浮络的五色清浊,从而知道何经发病;看病人喘息的情况,听病人发出的声音,从而知道病人的痛苦所在;看四时不同的脉象,从而知道疾病在哪一脏腑;切按尺肤和寸口,了解脉象浮沉滑涩,从而知道疾病所在部位。这样,在治疗上,就可以没有过失;在诊断上就不会有什么失误了。\\
所以说:病刚发生时,用针刺就可治愈;若邪气盛时,必须等到邪气稍退时再去治疗。所以治病要根据病情来采取相应的措施:在它轻的时候,要加以宣泄;在它重的时候,要加以攻泻;在病邪衰退正气也虚的时候,要以补益正气为主。病人形体羸弱的,应用气厚之品补之;精不足的,应用味厚之品补之。如病在膈上,可用吐法;病在下焦,可用通便之法;胸腹胀满的,可用攻泻之法;如感受风邪的,可用辛凉发汗法;如邪在皮毛的,可用辛温发汗法;病情发越太过的,可用抑收法;病实证,可用散法和泻法。观察疾病属阴属阳,来决定应当用柔剂还是用刚剂。病在阳的,也可治其阴;病在阴的,也可治其阳。辨明气分和血分,使它互不紊乱,血实的就用泻血法,气虚的就用升补法。\\
阴阳离合论篇第六\\
黄帝问曰:余闻天为阳,地为阴,日为阳,月为阴,大小月三百六十日成一岁,人亦应之。今三阴三阳,不应阴阳,其故何也?\\
岐伯对曰:阴阳者,数之可十,推之可百,数之可千,推之可万,万之大不可胜数,然其要一也。\\
天覆地载,万物方生,未出地者,命曰阴处,名曰阴中之阴;则出地者,命曰阴中之阳。阳予之正,阴为之主。故生因春,长因夏,收因秋,藏因冬,失常则天地四塞。阴阳之变,其在人者,亦数之可数。\\
帝曰:愿闻三阴三阳之离合也。\\
岐伯曰:圣人南面而立,前曰广明,后曰太冲,太冲之地,名曰少阴,少阴之上,名曰太阳,太阳根起于至阴,结于命门,名曰阴中之阳。中身而上,名曰广明,广明之下,名曰太阴,太阴之前,名曰阳明,阳明根起于厉兑,名曰阴中之阳。厥阴之表,名曰少阳,少阳根起于窍阴,名曰阴中之少阳。是故三阳之离合也,太阳为开,阳明为阖,少阳为枢。三经者,不得相失也,搏而勿浮,命曰一阳。\\
帝曰:愿闻三阴。\\
岐伯曰:外者为阳,内者为阴,然则中为阴,其冲在下,名曰太阴,太阴根起于隐白,名曰阴中之阴。太阴之后,名曰少阴,少阴根起于涌泉,名曰阴中之少阴。少阴之前,名曰厥阴,厥阴根起于大敦,阴之绝阳,名曰阴之绝阴。是故三阴之离合也,太阴为开,厥阴为阖,少阴为枢。三经者,不得相失也,搏而勿沉,名曰一阴。阴阳犮犮,积传为一周,气里形表而为相成也。\\
黄帝问:我听说天属阳,地属阴,日属阳,月属阴,大月和小月合计三百六十天而成为一年,人身也与此相应。如今听说人身的三阴三阳,和天地阴阳之数不符合,是什么道理呢?\\
岐伯回答说:天地间的阴阳计数有十,推广开来就有百,计数有千,推演开来就有万,万数之大不能一一查数,但其道理归结起来还是一个,总是阴阳的对立统一。\\
天覆盖于上,地承载于下,万物初生,还未长出地面时,叫做潜伏阴中,称之为阴中之阴;若已长出地面的,就叫做阴中之阳。阳气主宰万物的发生,阴气主宰万物的成形。所以万物的发生,依赖于春气的温暖,万物的生长,依赖于夏气的炎热,万物的收成,依赖于秋气的清凉,万物的闭藏,依赖于冬气的寒冷。生长收藏的变化失常,万物就不能发生成长。这阴阳的变化道理,在人身来说,也是有一定的规律,并且可以推测而知的。\\
黄帝说:希望听听三阴三阳的离合情况。\\
岐伯说:圣人面对南方站立,前方名叫广明,后方名叫太冲,太冲所起的地方,叫做少阴,在少阴经上面的经脉,名叫太阳,太阳经的下端起于足小趾外侧的至阴穴,其上端结于睛明穴,因太阳合于少阴,又为少阴之表,故称为阴中之阳。再从人身上下来说,上半身属阳,称为广明,广明之下称为太阴,太阴的前面,名叫阳明,阳明经的下端起于足大趾侧次趾之端的厉兑穴,因阳明是太阴之表,故称为阴中之阳。厥阴之表,名叫少阳,少阳经下端起于窍阴穴,因少阳居厥阴之表,又是阳气始生,故称为阴中之少阳。因此,三阳经的离合情况是,太阳主表为关,阳明主里为阖,少阳介于表里之间为枢。这三者之间,不能失去联系。脉象波动有力而不虚浮,所以合起来称为一阳。\\
黄帝说:希望听听三阴的离合情况。\\
岐伯说:在外的为阳,在内的为阴,所以在里的经脉称为阴经,行于少阴经前面的称为太阴,太阴经的根起于足大趾之端的隐白穴,称为阴中之阴。太阴的后面,称为少阴,少阴经的根起于足心的涌泉穴,称为阴中之少阴。少阴的前面,称为厥阴,厥阴经的根起于足大趾之端的大敦穴,称为阴之绝阴。所以,三阴经的离合,太阴为三阴之表为关,厥阴为三阴之里为阖,少阴位于表里之间为枢。但三者之间不能失去联系,脉象搏动有力而不过沉,所以合起来称为一阴。阴阳之气,运行不息,递相传注全身,气运于里,形立于表,这就是阴阳离合、表里配合,共同完成人体的生命活动。\\
阴阳别论篇第七\\
黄帝问曰:人有四经十二从,何谓?\\
岐伯对曰:四经应四时,十二从应十二月,十二月应十二脉。\\
脉有阴阳,知阳者知阴,知阴者知阳。凡阳有五,五五二十五阳。所谓阴者,真脏也。见则为败,败必死也。所谓阳者,胃脘之阳也。别于阳者,知病处也;别于阴者,知死生之期。三阳在头,三阴在手,所谓一也。别于阳者,知病忌时;别于阴者,知死生之期。谨熟阴阳,无与众谋。\\
所谓阴阳者,去者为阴,至者为阳;静者为阴,动者为阳;迟者为阴,数者为阳。凡持真脉之脏脉者,肝至悬绝急,十八日死;心至悬绝,九日死;肺至悬绝,十二日死;肾至悬绝,七日死;脾至悬绝,四日死。\\
曰:二阳之病,发心脾,有不得隐曲,女子不月;其传为风消,其传为息贲者,死不治。\\
曰:三阳为病,发寒热,下为痈肿,及为痿厥腨\\
。其传为索泽,其传为颓疝。\\
曰:一阳发病,少气善咳善泄。其传为心掣,其传为隔。\\
二阳一阴发病,主惊骇背痛,善噫善欠,名曰风厥。二阴一阳发病,善胀心满善气。三阳三阴发病,为偏枯、痿易、四支不举。\\
鼓一阳曰钩,鼓一阴曰毛,鼓阳胜急曰弦,鼓阳至而绝曰石,阴阳相过曰溜。\\
阴争于内,阳扰于外,魄汗未藏,四逆而起,起则熏肺,使人喘鸣。阴之所生,和本曰和。是故刚与刚,阳气破散,阴气乃消亡。淖则刚柔不和,经气乃绝。\\
死阴之属,不过三日而死;生阳之属,不过四日而已。所谓生阳死阴者,肝之心谓之生阳,心之肺谓之死阴,肺之肾谓之重阴,肾之脾谓之辟阴,死不治。\\
结阳者,肿四支。结阴者,便血一升,再结二升,三结三升。阴阳结斜,多阴少阳,曰石水,少腹肿;二阳结,谓之消;三阳结,谓之隔;三阴结,谓之水;一阴一阳结,谓之喉痹。\\
阴搏阳别,谓之有子。阴阳虚,肠澼死。阳加于阴,谓之汗;阴虚阳搏,谓之崩。三阴俱搏,二十日夜半,死。二阴俱搏,十三日夕时,死。一阴俱搏,十日,死。三阳俱搏且鼓,三日,死。三阴三阳俱搏,心腹满,发尽,不得隐曲,五日,死。二阳俱搏,其病温,死不治,不过十日,死。\\
黄帝问说:人有四经十二从,这是什么意思?\\
岐伯回答说:四经,是指肝心肺肾及其与四时相应的正常脉象,十二从,是指与十二月相应顺次运行的十二经脉。\\
脉有阴脉和阳脉,知道什么是阳脉,就能知道什么是阴脉,知道什么是阴脉,也就能知道什么是阳脉。阳脉有五种,五时各有五脏的阳脉,所以五时配合五脏,共计二十五种阳脉。所谓阴脉,就是五脏真精暴露,没有胃气的真脏脉。真脏脉出现证明胃气已经败坏,败象已见,必死。所谓阳脉,就是有胃气之脉。能辨清阳脉,就能知道病变部位;能辨别真脏脉,就可以断定死期。诊察三阳经脉在颈部的人迎穴,诊察三阴经脉在手鱼际之后的寸口,一般情况下,人迎与寸口的脉象是相应的。能辨别阳脉,就能知道时令气候和疾病的宜忌;能辨别真脏脉,就能知道死期。临证时能谨慎而熟练地辨别阴脉与阳脉,就不需与众人商议而疑惑不决。\\
脉象的所谓阴阳,脉去为阴,脉来为阳;脉静为阴,脉动为阳;脉迟为阴,脉数为阳。凡持诊所见真脏脉,肝脉的形象与其他各脏之脉悬殊极大,或者来得弦急而硬,十八日当死;心的形象与其他各脏之脉悬殊极大,九日当死;肺脉的形象与其他各脏之脉悬殊极大,十二日当死;肾脉的形象与其他各脏之脉悬殊极大,七日当死;脾脉的形象与其他各脏之脉悬殊极大,四日当死。\\
说:胃肠有病,可影响心脾,病人往往有隐情难告,女子会经闭;病久传变,或者是形体逐渐消瘦的风消证,或者是呼吸短促、气息上逆的息贲证,是不治的死证。\\
说:太阳经发病,发热恶寒,或下部发生痈肿,或两足痿弱无力而逆冷,腿肚酸痛。病久传化,或为皮肤干燥不润泽,或变为阴囊肿大的颓疝。\\
说:少阳经发病,气虚不足,容易咳嗽,及泄泻。病久传变,或为心虚掣痛,或为饮食不下,隔塞不通的隔证。\\
阳明与厥阴发病,其主要病状是惊骇,背痛,常常嗳气、呵欠,名叫风厥。少阴和少阳发病,腹部易胀,心下满闷,时常长出气。太阳和太阴发病,其表现为半身不遂的偏枯证,或者痿弱无力而变易常用,或者四肢不能举动。\\
脉象鼓动指下,紧张而有力,如按弓弦,叫弦脉;稍无力,来时轻虚而浮,叫毛脉;来时有力,去时力衰,叫钩脉;有力而必须重按,轻按不足,叫石脉;既非无力,又不过于有力,一来一去,脉象和缓,叫滑脉。\\
阴阳失和,阴气争胜于内,阳气扰乱于外,以致汗出不止,四肢厥冷,下厥上逆,浮阳熏肺,发生喘鸣。阴气之所以能生化,是以阴阳之平和为根本。如果以刚与刚合,有阳无阴,不能生化,则阳气破散,阴气亦必随之消亡。阴阳紊乱,刚柔不和,十二经气就会衰绝。\\
属于死阴的病,不过三日就要死;属于生阳的病,不过四天就会痊愈。所谓“生阳、死阴”是指:肝病传心,为木生火,母病传子,得其生气,叫生阳;心病传肺,为火克金,金被火烁,叫死阴;肺病传肾,少阴而传太阴,属无阳之候,叫重阴;肾病传脾,水反侮土,叫做辟阴,是不治的死证。\\
邪气郁结于阳经,四肢就会浮肿。邪气郁结于阴经,就会大便下血,初结一升,再结二升,三结三升。阴经阳经俱有邪气郁结,如果阴经郁结得多,阳经郁结得少,会发生石水,而现少腹肿胀;邪气郁结于足阳明和手阳明这二阳经,会有消渴证;邪气郁结于足太阳膀胱和手太阳小肠这三阳经,会发生上下不通的隔证;邪气郁结于足太阴脾、手太阴肺这三阴经,多发水肿病;邪气郁结于厥阴和少阳这一阴一阳经,多患喉痹。\\
阴脉搏动有力,与阳脉有明显的不同,是怀孕有子之兆。阴阳尺寸之脉俱虚而且患痢疾的,是死证。阳脉倍于阴脉,主汗出;阴脉虚而阳脉搏指,在妇人为血崩。肺脾三阴之脉,俱搏击于指下,大约到二十天半夜时死亡。心肾二阴之脉俱搏击于指下,大约到十三天傍晚时死亡。心包肝一阴之脉俱搏击于指下,大约十天就要死亡。膀胱小肠三阳之脉俱搏击于指下,而鼓动过甚的,三天就要死亡。三阴三阳之脉俱搏,心腹胀满,阴阳之气败泄已尽,二便不通,则五日死。胃大肠二阳之脉俱搏击于指下,如为温病,是不治之证,不过十日必死。\\
卷三\\
灵兰秘典论篇第八\\
黄帝问曰:愿闻十二脏之相使,贵贱何如?\\
岐伯对曰:悉乎哉问也!请遂言之。心者,君主之官也,神明出焉。肺者,相傅之官,治节出焉。肝者,将军之官,谋虑出焉。胆者,中正之官,决断出焉。膻中者,臣使之官,喜乐出焉。脾胃者,仓廪之官,五味出焉。大肠者,传道之官,变化出焉。小肠者,受盛之官,化物出焉。肾者,作强之官,伎巧出焉。三焦者,决渎之官,水道出焉。膀胱者,州都之官,津液藏焉,气化则能出矣。凡此十二官者,不得相失也。故主明则下安,以此养生则寿,殁世不殆,以为天下则大昌。主不明则十二官危,使道闭塞而不通,形乃大伤,以此养生则殃,以为天下者,其宗大危,戒之戒之!\\
至道在微,变化无穷,孰知其原?窘乎哉!消者瞿瞿,孰知其要?闵闵之当,孰者为良?恍惚之数,生于毫氂,毫氂之数,起于度量,千之万之,可以益大,推之大之,其形乃制。\\
黄帝曰:善哉!余闻精光之道,大圣之业。而宣明大道,非斋戒择吉日,不敢受也。\\
黄帝乃择吉日良兆,而藏灵兰之室,以传保焉。\\
黄帝说:我希望听听十二脏器在体内的相互作用,有无主从的区别?\\
岐伯回答说:问得真详细啊!让我说说吧。心就像君主,智慧是从心产生的。肺好像宰相,主一身之气,治理调节人体内外上下的活动由它完成。肝好比将军,谋虑是从它那来的。胆是清虚的脏器,具有决断的能力。膻中像内臣,心的喜乐,都由它传达。脾胃受纳水谷,好像仓库,五味转化为营养,由它那产生。大肠主管输送,食物的消化、吸收、排泄过程在那里最后完成。小肠接受脾胃已消化的食物后,进一步分清别浊。肾是精力的源泉,能产生技巧。三焦主疏通水道,周身行水的道路由它管理。膀胱是水液聚会的地方,经过气化作用,才能把尿排出体外。以上十二脏器的作用,不能失去协调。当然,君主是最主要的。心的功能正常,下边就能相安。依据这个道理来养生,就能长寿,终身不致有严重的疾病;根据这个道理来治理天下,国家就会繁荣昌盛。反之,如果君主昏庸,功能失常,那么十二官就出问题了。而各个脏器的活动一旦闭塞不通,失去联系,形体就会受到伤害,对于养生来说,这是最大的祸殃。这样治国,国家就有败亡的危险,要千万警惕啊!\\
医学的道理极其微妙,变化没有穷尽,谁能了解它的本源呢?困难得很哪!形体日渐消瘦的人虽然很惊疑,谁能明白其中的原因呢?纵然对自己的身体非常担心,谁能知道如何才好?事物发展的一般规律都是从似有似无极其微小开始的,虽然极其微小,也是可以度量的,千倍万倍地增加,事物就一步步地增大,扩大到一定程度它的形状就明显了。疾病的发生发展也是这个道理,由极其隐微逐渐发展而成。\\
黄帝说:说得好!我听到了一番精纯明白的道理和圣人的事业。这些通达光明的道理,如不诚心诚意选择吉日,是不敢接受的。\\
黄帝就选择了吉日良辰,把这些道理,保存在灵台兰室,如同宝物一般,让它传流下去。\\
六节脏象论篇第九\\
黄帝问曰:余闻天以六六之节,以成一岁,地以九九制会,计人亦有三百六十五节以为天地,久矣。不知其所谓也?\\
岐伯对曰:昭乎哉问也!请遂言之。夫六六之节,九九制会者,所以正天之度,气之数也。天度者,所以制日月之行也,气数者,所以纪化生之用也。天为阳,地为阴;日为阳,月为阴。行有分纪,周有道理。日行一度,月行十三度而有奇焉。故大小月三百六十五日而成岁,积气余而盈闰矣。立端于始,表正于中,推余于终,而天度毕矣。\\
帝曰:余已闻天度矣,愿闻气数,何以合之?\\
岐伯曰:天以六六为节,地以九九制会。天有十日,日六竟而周甲,甲六复而终岁,三百六十日法也。夫自古通天者,生之本,本于阴阳。其气九州、九窍,皆通乎天气。故其生五,其气三。三而成天,三而成地,三而成人,三而三之,合则为九,九分为九野,九野为九脏,故形脏四,神脏五,合为九脏以应之也。\\
帝曰:余已闻六六九九之会也,夫子言积气盈闰,愿闻何谓气?请夫子发蒙解惑焉!\\
岐伯曰:此上帝所秘,先师传之也。\\
帝曰:请遂闻之。\\
岐伯曰:五日谓之候,三候谓之气;六气谓之时,四时谓之岁。而各从其主治焉。五运相袭,而皆治之;终期之日,周而复始。时立气布,如环无端,候亦同法。故曰:不知年之所加,气之盛衰,虚实之所起,不可以为工矣。\\
帝曰:五运终始,如环无端,其太过不及何如?\\
岐伯曰:五气更立,各有所胜,盛虚之变,此其常也。\\
帝曰:平气何如?\\
岐伯曰:无过者也。\\
帝曰:太过不及奈何?\\
岐伯曰:在经有也。\\
帝曰:何谓所胜?\\
岐伯曰:春胜长夏,长夏胜冬,冬胜夏,夏胜秋,秋胜春。所谓得五行时之胜,各以其气命其脏。\\
帝曰:何以知其胜?\\
岐伯曰:求其至也,皆归始春。未至而至,此谓太过。则薄所不胜,而乘所胜也,命曰气淫。至而不至,此谓不及。则所胜妄行,而所生受病,所不胜薄之也,命曰气迫。所谓求其至者,气至之时也,谨候其时,气可与期。失时反候,五治不分,邪僻内生,工不能禁也。\\
帝曰:有不袭乎?\\
岐伯曰:苍天之气,不得无常也。气之不袭,是谓非常,非常则变矣。\\
帝曰:非常而变,奈何?\\
岐伯曰:变至则病。所胜则微,所不胜则甚。因而重感于邪则死矣。故非其时则微,当其时则甚也。\\
帝曰:善!余闻气合而有形,因变以正名,天地之运,阴阳之化,其于万物,孰少孰多,可得闻乎?\\
岐伯曰:悉乎哉问也!天至广不可度,地至大不可量,大神灵问,请陈其方。草生五色,五色之变,不可胜视;草生五味,五味之美,不可胜极。嗜欲不同,各有所通。天食人以五气,地食人以五味。五气入鼻,藏于心肺,上使五色修明,音声能彰;五味入口,藏于肠胃,味有所藏,以养五气。气和而生,津液相成,神乃自生。\\
帝曰:脏象何如?\\
岐伯曰:心者,生之本,神之处也;其华在面,其充在血脉,为阳中之太阳,通于夏气。肺者,气之本,魄之处也;其华在毛,其充在皮,为阳中之太阴,通于秋气。肾者,主蛰,封藏之本,精之处也;其华在发,其充在骨,为阴中之太阴,通于冬气。肝者,罢极之本,魂之居也;其华在爪,其充在筋,以生血气,其味酸,其色苍,此为阴中之少阳,通于春气。脾者,仓廪之本,营之居也;其华在唇四白,其充在肌,此至阴之类,通于土气。胃、大肠、小肠、三焦,膀胱,名曰器,能化糟粕,转味而出入者也。凡十一脏取决于胆也。\\
故人迎一盛,病在少阳,二盛病在太阳,三盛病在阳明,四盛已上为格阳。寸口一盛,病在厥阴,二盛病在少阴,三盛病在太阴,四盛已上为关阴。人迎与寸口俱盛四倍已上为关格,关格之脉赢,不能极于天地之精气,则死矣。\\
黄帝问道:我听说天是以六个甲子日合成一年,地气是以九九之法与天相会通的,而人也有三百六十五节,与天地之数相合,这种说法已经很长期间了。但不知是什么道理。\\
岐伯回答说:问得真高明啊!我就说说吧。六六之节和九九之法,是确定天度和气数的。天度,是用来确定日月行程、迟速的标准;气数,是用来标明万物化生的循环周期的。天是阳,地是阴;日是阳,月是阴。日月运行有一定部位,万物化生的循环也有一定的规律。每昼夜日行周天一度,而月行十三度有余。所以有大月小月,合三百六十五天为一年,而余气积累,则产生了闰月。那么怎样计算呢?首先确定一年节气的开始,用圭表测量日影的长短变化,校正一年里的时令节气,然后再推算余闰,这样,天度就可全部计算出来了。\\
黄帝道:我已听到关于天度的道理了,希望再听听气数是怎样与天度相配合的?\\
岐伯说:天是以六六之数为节度,地是以九九之法与天相会通的。天有十个日干,代表十天,六个十干,叫做一个周甲,六个周甲成为一年,这是三百六十日的计算方法。从古以来,懂得天道的,都认为天是生命的本源,生命是本于阴阳的。无论地之九州还是人之九窍,都与天气相通。因为它们的生长禀受了自然界的五行和三阴三阳之气。天有三气,地有三气,人有三气,三三合而为九,在地分为九野,在人分为九脏,即四个形脏五个神脏,合为九脏,以与天的六六之数相应。\\
黄帝说:我已知道了六六与九九相会通的道理,但夫子说积累余气成为闰月,那什么叫做气呢?请夫子启发我的愚昧,解除我的疑惑!\\
岐伯说:这是上帝所隐秘,而由先师传给我的。\\
黄帝道:希望讲给我听听。\\
岐伯说:五天叫一候,三候成为一个节气;六个节气叫一时,四时叫一年。治病就应顺从其当旺之气。五行气运相互承袭,都有主治之时;到了年终之日,再从头开始循环。一年分立四时,四时分布节气,如圆环一样没有开端,五日一候的推移,也是如此。所以说:不知道一年中当王之气的加临,节气的盛衰,虚实产生的原因,就不能当医生。\\
黄帝道:五运终而复始,循环往复,像圆环一样没有开端,那么它的太过和不及如何呢?\\
岐伯说:五行运气,更迭主时,各有其所胜,所以实虚的变化,这是正常的事情。\\
黄帝问:平气是怎样的?\\
岐伯说:没有太过,也没有不及。\\
黄帝道:太过和不及的情况怎样?\\
岐伯说:经书里有记载。\\
黄帝问:什么叫做所胜?\\
岐伯说:春胜长夏,长夏胜冬,冬胜夏,夏胜秋,秋胜春。这是五行之气以时相胜的情况,而人的五脏就是根据这五行之气来命名的。\\
黄帝说:怎样可以知道它们的所胜呢?\\
岐伯说:推求脏气到来的时间,都以立春前为标准。如果时令未到而相应的脏气先到,就称为太过。太过就侵犯原来自己所不胜的气,而凌侮它所能胜的气,这叫“气淫”。如果时令已到而相应的脏气不到,就称为不及。不及则己所胜之气因无制约就要妄行,所生之气也因无所养而要受病,所不胜之气也来相迫,这叫“气迫”。所谓求其至,就是在脏气来到的时候,谨慎地观察与其相应的时令,看脏气是否与时令相合。假如脏气与时令不合,并且与五行之间的对应关系无从分辨,那就表明内里邪僻之气已经生成,这样,就连医生也无能为力了。\\
黄帝问道:五行气运有不相承袭的情况吗?\\
岐伯回答说:自然界的气运不能没有规律。气运失其承袭,就是反常,反常就要变而为害。\\
黄帝道:反常变而为害又怎样呢?\\
岐伯说:这会使人发生疾病。如属所胜,患病就轻;如属所不胜,患病就重。假若这个时候再感受了邪气,就会死亡。也就是说,五行气运的反常,在不当克我的时候,病比较轻,而在正值克我的时候,病就重了。\\
黄帝道:说得好!我听说天地之气化合而成形体,又根据不同的形态变化来确定万物的名称,那么天地的气运和阴阳的变化,对于万物所起的作用,哪个大哪个小,可以听听吗?\\
岐伯说:你问得很详细啊!天很广阔,不容易测度,地很博大,也难以测量,不过既然你提出了这样的问题,那么我就说说其中的道理吧。草有五种不同的颜色,这五色的变化,是看不尽的;草有五种不同的气味,这五味的美妙也是不能穷尽的。人的嗜欲不同,对于色味,是各有其不同嗜好的。天供给人们五气,地供给人们五味。五气由鼻吸入,贮藏在心肺,能使面色明润,声音洪亮:五味由口进入,藏在肠胃里,所藏的五味,来供养五脏之气。五气和化,就有生机,再加上津液的作用,神气就会旺盛起来。\\
黄帝问道:人体内脏与其外在表现的关系如何?\\
岐伯说:心是生命的根本,智慧的所在;其荣华表现在面部,其功用是充实血脉,是阳中之太阳,与夏气相应。肺是气的根本,是藏魄的所在;其荣华表现在毫毛,其功用是充实肌表,是阳中之太阴,与秋气相应。肾是真阴真阳蛰藏的地方,是封藏的根本,精气储藏的所在;其荣华表现于头发,其功用是充实骨髓,是阴中之太阴,与冬气相应。肝是四肢的根本,藏魂的所在;其荣华表现在爪甲,其功用是充实筋力,可以生养血气,其味酸,其色苍青,是阴中之少阳,与春气相应。脾是水谷所藏的根本,是营气所生的地方;其荣华表现在口唇四周,其功用是充实肌肉,属于至阴一类,与长夏土气相应。胃、大肠、小肠、三焦、膀胱,叫做器,能排泄水谷的糟粕,转化五味而主吸收、排泄。以上十一脏功能的发挥,都取决于胆的功能正常。\\
人迎脉搏大一倍,病在少阳;大两倍,病在太阳;大三倍,病在阳明;大四倍以上为格阳于外。寸口脉搏大一倍,病在厥阴;大两倍,病在少阴;大三倍,病在太阴;大四倍以上称为关阴。若人迎脉与寸口脉都比常人大四倍,便为关格,关格之脉为阴阳亢极而不和,病人不再能汲取天之清气和地之水谷气了,必定死亡。\\
五脏生成篇第十\\
心之合脉也,其荣色也,其主肾也。肺之合皮也,其荣毛也,其主心也。肝之合筋也,其荣爪也,其主肺也。脾之合肉也,其荣唇也,其主肝也。肾之合骨也,其荣发也,其主脾也。\\
是故多食咸,则脉凝泣而变色;多食苦,则皮槁而毛拔;多食辛,则筋急而爪枯;多食酸,则肉胝禸而唇揭;多食甘,则骨痛而发落。此五味之所伤也。故心欲苦,肺欲辛,肝欲酸,脾欲甘,肾欲咸。此五味之所合也。\\
五脏之气,故色见青如草兹者死,黄如枳实者死,黑如炲者死,赤如衃血者死,白如枯骨者死。此五色之见死也。\\
青如翠羽者生,赤如鸡冠者生,黄如蟹腹者生,白如豕膏者生,黑如乌羽者生。此五色之见生也。生于心,如以缟裹朱;生于肺,如以缟裹红;生于肝,如以缟裹绀;生于脾,如以缟裹栝楼实;生于肾,如以缟裹紫。此五脏所生之外荣也。\\
色味当五脏。白当肺、辛,赤当心、苦,青当肝、酸,黄当脾、甘,黑当肾、咸。故白当皮,赤当脉,青当筋,黄当肉,黑当骨。\\
诸脉者皆属于目,诸髓者皆属于脑,诸筋者皆属于节,诸血者皆属于心,诸气者皆属于肺。此四支八谿之朝夕也。\\
故人卧血归于肝。目受血而能视,足受血而能步,掌受血而能握,指受血而能摄。卧出而风吹之,血凝于肤者为痹,凝于脉者为泣,凝于足者为厥。此三者,血行而不得反其空,故为痹厥也。人有大谷十二分,小谿三百五十四名,少十二俞。此皆卫气之所留止,邪气之所客也,针石缘而去之。\\
诊病之始,五决为纪。欲知其始,先建其母。所谓五决者,五脉也。\\
是以头痛巅疾,下虚上实,过在足少阴、巨阳,甚则入肾。徇蒙招尤,目瞑耳聋,下实上虚,过在足少阳、厥阴,甚则入肝。腹满尒胀,支鬲胠胁,下厥上冒,过在足太阴、阳明。咳嗽上气,厥在胸中,过在手阳明、太阴,甚则入肺。心烦头痛,病在鬲中,过在手巨阳、少阴,甚则入心。\\
夫脉之小大滑涩浮沉,可以指别;五脏之象,可以类推;五脏相音,可以意识;五色微诊,可以目察。能合脉色,可以万全。赤,脉之至也,喘而坚,诊曰有积气在中,时害于食,名曰心痹,得之外疾,思虑而心虚,故邪从之。白,脉之至也,喘而浮,上虚下实,惊,有积气在胸中,喘而虚,名曰肺痹,寒热,得之醉而使内也。青,脉之至也,长而左右弹,有积气在心下支胠,名曰肝痹,得之寒湿,与疝同法,腰痛足清头痛。黄,脉之至也,大而虚,有积气在腹中,有厥气,名曰厥疝,女子同法,得之疾使四支,汗出当风。黑,脉之至也,下坚而大,有积气在小腹与阴,名曰肾痹,得之沐浴清水而卧。\\
凡相五色,面黄目青,面黄目赤,面黄目白,面黄目黑者,皆不死也。面青目赤,面赤目白,面青目黑,面黑目白,面赤目青,皆死也。\\
心脏的外合是血脉,它的外荣表现于面部的色泽,制约心脏的是肾。肺脏的外合是皮,它的外荣表现于毛,制约肺脏的是心。肝脏的外合是筋,它的外荣表现于爪甲,制约肝脏的是肺。脾脏的外合是肉,它的外荣表现于唇,制约脾脏的是肝。肾脏的外合是骨,它的外荣表现于发,制约肾脏的是脾。\\
所以多吃咸的东西,会使血脉凝滞,而面色失去光泽;多吃苦的东西,会使皮肤干燥而毫毛脱落;多吃辣的东西,会使筋脉拘挛而爪甲枯槁;多吃酸的东西,会使肉坚厚而唇缩;多吃甜的东西,会使骨骼疼痛而头发脱落。这些是饮食五味的偏嗜造成的伤害。所以心喜苦味,肺喜辛味,肝喜酸味,脾喜甘味,肾喜咸味。这就是五味和五脏的对应关系。\\
五脏外荣于面上的气色,表现出青黑,颜色像死草一样,是死征;表现出黄色,像枳实一样,是死征;表现出黑色,像黑煤一样,是死征;表现出赤色,像败血凝结一样,是死征;表现出白色,像枯骨一样,是死征。这是从五种色泽来判断死证的情况。\\
脸上的气色,如果青得像翠鸟的羽毛,是生色;红得像鸡冠,是生色;黄得像蟹腹,是生色;白得像猪油,是生色;黑得像乌鸦的羽毛,是生色。这是体现还有生气的五种色泽。凡是心脏有生气的色泽,就像白绢裹着硃砂一样;肺脏有生气的色泽,就像白绢裹着红色的东西一样;肝脏有生气的色泽,就像白绢裹着绀色的东西一样;脾脏有生气的色泽,就像白绢裹着栝楼实一样;肾脏有生气的色泽,就像白绢裹着紫色的东西一样。这些是五脏有生气的表现。\\
五色、五味与五脏是相合的。白色合于肺脏和辛味,赤色合于心脏和苦味,青色合于肝脏和酸味,黄色合于脾脏和甜味,黑色合于肾脏和咸味。另外,白色合于皮,赤色合于脉,青色合于筋,黄色合于肉,黑色合于骨。\\
人身的经脉,都上注于目;所有的精髓,都上注于脑;所有的筋,都注于骨节;所有的血液,都注于心;所有的气,都注于肺。气血经脉向四肢八谿灌注就像潮水周而复始。\\
人在躺卧的时候,血就归于肝脏,血是营养四肢百骸的。所以目得了血就能看东西;足得了血就能行走;手掌得了血就能握物;手指得了血就能拿物。刚睡起走到屋外,被风吹着,如果血凝结在肤表,就要发生痹证;如果凝涩在经脉里,就会血行迟滞;如果凝涩在足部上,就会发生下肢厥冷。这三种疾患,都是由于血液不能流回到孔窍,所以,发生痹厥等病。在人身上,有大谷十二处,小谿三百五十四处,那十二关还不在其内。这些都是卫气所留止的地方,也是邪气容易留止的处所,如果受了邪气的侵袭,就赶紧用针刺或砭石去除。\\
在开始诊病时,应当把五决作为纲纪。要想知道某病从哪脏发生,先要考察那一脏脉的胃气怎样。所说的五决,就是五脏之脉。\\
所以巅顶头痛,属于下虚上实,病在足少阴、太阳两经,如病势加剧,就会传入肾脏。眼花摇头,发病急骤的,或者目暗耳聋,病程较长的,属于下实上虚,病在足少阳、厥阴两经,如病势加剧,就会传入肝脏。腹满胀起,胸膈胁胠间像撑拄一样,下体厥冷,上体眩晕,病在足太阴、阳明两经,咳嗽逆喘,胸中有病,病在手阳明、太阴两经,如病势加剧,就会传入肺脏。心烦头痛,胸中不适,病在手太阳、少阴两经,如病势加剧,就会传入心脏。\\
脉象的小大滑涩浮沉,可以用手指分别出来;五脏的气象,可以从比类中去推求;察听从五脏反应出的音声,可以意会而分析;五色虽然精微,可以用眼来观察。在诊断中如果能够参合色、脉,就万无一失。如果脸上现出赤色,脉象躁数而又坚实,就是病气积聚在腹中,常常妨碍饮食,这种病叫做心痹,它致病的原因,是过于思虑伤了心气,所以病邪乘虚而入。如果脸上出现白色,同时脉象躁数而又浮大,上虚下实,这是病气积聚在胸中,喘而且虚惊,这种病叫肺痹,它致病的原因,是感受寒热,并在醉后入房。如果脸上出现青色,同时脉象长,并且左右弹指,这是病气积在心下,撑拄两胠,这种病叫肝痹,它致病的原因,是感受了寒湿,所以病理和疝气一样,并有腰痛、足冷、头痛等症状。如果脸上出现黄色,同时脉象大而虚,这是病气积在腹中,自觉有逆气,这种病叫厥疝,女子同样有这种情况,它致病的原因,是由于四肢过劳,出汗后被风侵袭。如果脸上出现黑色,同时下部脉坚而大,这是病气积在小腹和前阴,这种病叫肾痹,它致病的原因,是由凉水沐浴后就睡觉而得的。\\
大凡观察五色,面黄目青,面黄目赤,面黄目白,面黄目黑的,都不是死的征像。面青目赤,面赤目白,面青目黑,面黑目白,面赤目青的,都是死的征象。\\
五脏别论篇第十一\\
黄帝问曰:余闻方士,或以脑髓为脏,或以肠胃为脏,或以为腑。敢问更相反,皆自谓是,不知其道,愿闻其说。\\
岐伯对曰:脑、髓、骨、脉、胆、女子胞,此六者,地气之所生也,皆藏于阴而象于地,故藏而不泻,名曰奇恒之腑。夫胃、大肠、小肠、三焦、膀胱,此五者,天气之所生也,其气象天,故泻而不藏,此受五脏浊气,名曰传化之腑。此不能久留,输泻者也。魄门亦为六腑,使水谷不得久藏。所谓五脏者,藏精气而不泻也,故满而不能实。六腑者,传化物而不藏,故实而不能满也。水谷入口,则胃实而肠虚;食下,则肠实而胃虚,故曰实而不满。\\
帝曰:气口何以独为五脏主?\\
岐伯曰:胃者,水谷之海,六腑之大源也。五味入口,藏于胃,以养五脏气。气口亦太阴也,是以五脏六腑之气味,皆出于胃,变见于气口。故五气入鼻,藏于肺,肺有病,而鼻为之不利也。凡治病,必察其下,适其脉,观其志意,与其病也。\\
拘于鬼神者,不可与言至德;恶于针石者,不可与言至巧;病不许治者,病必不治,治之无功矣。\\
黄帝问道:我从方士那儿听说,有的把脑髓叫做脏,有的把肠和胃叫做脏,但又有把肠胃叫做腑的。他们的意见不同,却都自以为是,我不知到底谁说得正确,希望听你讲一下。\\
岐伯回答说:脑、髓、骨、脉、胆和女子胞,这六者,是感受地气而生的,都能藏精血,像地之厚能盛载万物那样,它们的作用,是藏精气以濡养机体而不泄于体外,这叫做“奇恒之腑”。像胃、大肠、小肠、三焦、膀胱,这五者,是感受天气而生的,它们的作用,像天之健运行不息一样,所以是泻而不藏,它们受纳五脏的浊气,叫做“传化之腑”。就是说它们受纳水谷浊气以后,不能久停体内,经过分化,要把精华和糟粕分别输送和排出的。加上“魄门”,算是“六腑”,它的作用,同样是使糟粕不能长久留存在体内。五脏是藏精而不泻的,所以虽然常常充满,却不像肠胃那样,要由水谷充实它。六腑是要把食物消化、吸收、输泻出去,所以虽然常常是充实的,却不能像五脏那样被充满。水谷入口以后,胃里虽实,肠子却是空的;等到食物下去,肠中就会充实,而胃里又空了,所以说六腑是“实而不满”的。\\
黄帝问道:诊察气口之脉,为什么能够知道五脏六腑十二经脉之气呢?\\
岐伯说:胃是水谷之海,六腑的源泉。凡是五味入口后,都存留在胃里,经过脾的运化,来营养脏腑血气。气口属于手太阴肺经,而肺经主朝百脉,所以五脏六腑之气,都来源于胃,而其变化则表现在气口脉上。五气入鼻,进入肺里,而肺一有了病,鼻的功能也就差了。凡是在治疗疾病时,首先要问明病人的二便,辨清脉象,观察他的情志以及病证如何。\\
如果病人为鬼神迷信所束缚,就无须向他说明医学理论;如果病人厌恶针石,就无须向他说明针石技巧;如果病人不同意治疗,病一定治不好,即使治疗也不会有效果。\\
卷四\\
异法方宜论篇第十二\\
黄帝问曰:医之治病也,一病而治各不同,皆愈,何也?\\
岐伯对曰:地势使然也。故东方之域,天地之所始生也,鱼盐之地。海滨傍水,其民食鱼而嗜咸,皆安其处,美其食。鱼者使人热中,盐者胜血。故其民皆黑色疏理,其病皆为痈疡。其治宜砭石,故砭石者,亦从东方来。\\
西方者,金玉之域,沙石之处,天地之所收引也。其民陵居而多风,水土刚强。其民不衣而褐荐,华食而脂肥,故邪不能伤其形体,其病生于内。其治宜毒药,故毒药者,亦从西方来。\\
北方者,天地所闭藏之域也。其地高陵居,风寒冰冽。其民乐野处而乳食,脏寒生满病。其治宜灸焫,故灸焫者,亦从北方来。\\
南方者,天地之所长养,阳之所盛处也。其地下,水土弱,雾露之所聚也。其民嗜酸而食胕,故其民皆致理而赤色,其病挛痹。其治宜微针,故九针者,亦从南方来。\\
中央者,其地平以湿,天地所以生万物也众。其民食杂而不劳,故其病多痿厥寒热。其治宜导引按\\
,故导引按\\
者,亦从中央出也。\\
故圣人杂合以治,各得其所宜,故治所以异而病皆愈者,得病之情,知治之大体也。\\
黄帝问道:医生治病,一样的病,而治法不同,但都痊愈了,这是什么道理?\\
岐伯答说:这是地理因素造成的。东方地区,气候像生发的春季,是出产鱼盐的地方。由于靠近海边,当地居民,喜欢吃鱼盐一类东西,习惯于他们居住的地方,觉得吃得好。但是鱼性热,吃多了,使人肠胃内热;盐吃多了,会伤血。所以当地的百姓,大都皮肤色黑,肌理疏松,多生痈疡一类的病。在治疗上,适合用砭石,所以砭石疗法,来自东方。\\
西方地区,出产金玉,是沙漠地带,气候像收敛的秋季。那里的百姓都是依山而居,多风沙,水土性质刚强。当地居民不穿丝绵,多使用毛布和草席;喜欢吃肥美,容易使人发胖的食物,所以外邪不易犯害他们的躯体,他们的疾病是由饮食、情志内因造成的,容易生内脏疾病。治疗上,就需用药物,所以药物疗法,来自西方。\\
北方地区,气候像闭藏的冬季。地势高,人们住在山上,周围环境是寒风席卷冰冻的大地。当地居民,习惯于住在野地里,吃牛羊乳汁。这样,内脏就会受寒,容易生发胀满病。治疗上,应该使用灸焫,所以灸焫疗法,来自北方。\\
南方地区,气候像长养万物的夏季,是阳气盛大的地方。地势低洼,水土卑湿,雾露聚集多。当地百姓,喜欢吃酸类和腐臭的食品,所以当地人的皮肤致密色红,容易发生拘挛湿痹等病。治疗上,应该使用微针,所以微针疗法,来自南方。\\
中央地区,地势平坦多湿,是自然界中物种和数量最为丰富的地方。那里食物的种类很多,人们不感觉烦劳,多生发痿厥寒热等病。在治疗上,应该使用导引按刉的方法,所以导引按刉疗法,来自中央地区。\\
高明的医生综合各种疗法,针对病情,采取恰当的治疗,所以疗法尽管不同,疾病却都能痊愈,这是由于了解病情,掌握了治病大法的原因啊!\\
移精变气论篇第十三\\
黄帝问曰:余闻古之治病,惟其移精变气,可祝由而已。今世治病,毒药治其内,针石治其外,或愈或不愈,何也?\\
岐伯对曰:往古人居禽兽之间,动作以避寒,阴居以避暑。内无眷慕之累,外无伸宦之形。此恬惔之世,邪不能深入也。故毒药不能治其内,针石不能治其外,故可移精变气,祝由而已。当今之世不然。忧患缘其内,苦形伤其外,又失四时之从,逆寒暑之宜,贼风数至,虚邪朝夕,内至五藏骨髓,外伤空窍肌肤,所以小病必甚,大病必死,故祝由不能已也。\\
帝曰:善。余欲临病人,观死生,决嫌疑,欲知其要,如日月光,可得闻乎?\\
岐伯曰:色脉者,上帝之所贵也,先师之所传也。上古使僦贷季,理色脉而通神明,合之金木水火土,四时、八风、六合,不离其常,变化相移,以观其妙,以知其要。欲知其要,则色脉是矣。色以应日,脉以应月,常求其要,则其要也。夫色之变化,以应四时之脉。此上帝之所贵,以合于神明也。所以远死而近生,生道以长,命曰圣王。中古之治病,至而治之。汤液十日,以去八风五痹之病,十日不已,治以草苏草荄之枝。本末为助,标本已得,邪气乃服。暮世之治病也则不然。治不本四时,不知日月,不审逆从,病形已成,乃欲微针治其外,汤液治其内,粗工兇兇,以为可攻,故病未已,新病复起。\\
帝曰:愿闻要道。\\
岐伯曰:治之要极,无失色脉。用之不惑,治之大则。逆从倒行,标本不得,亡神失身。去故就新,乃得真人。\\
帝曰:余闻其要于夫子矣。夫子言不离色脉,此余之所知也。\\
岐伯曰:治之极于一。\\
帝曰:何谓一?\\
岐伯曰:一者因问而得之。\\
帝曰:奈何?\\
岐伯曰:闭户塞牖,系之病者,数问其情,以从其意。得神者昌,失神者亡。\\
帝曰:善。\\
黄帝问道:我听说古时治病,只是转变病人的思想精神,用“祝由”的方法就可以治愈。现在治病,用药物从内治,用针石从外治,结果还是有好有不好的,这是什么道理呢?\\
岐伯答说:古时候,人们穴居野外,周围都是禽兽,靠活动来驱寒,住在阴凉地方来避暑。在内心没有爱慕的累赘,在外没有奔走求取官宦的形役。这是恬惔的时代,外邪不易侵犯人体。因此既不需要“毒药治其内”,也不需要“针石治其外”,所以只是改变精神状态,断绝病根就够了。现在就不同了。人们心里经常为忧虑所苦,形体经常被劳累所伤,再加上违背四时的气候和寒热的变化,这样,贼风虚邪早晚不断侵袭,就会内犯五脏骨髓,外伤孔窍肌肤,所以小病会发展成为重病,而大病就会病危或死亡,因此,仅依靠祝由是不能把病治好的。\\
黄帝说:很好!我希望遇到病人,能够观察疾病的轻重,决断疾病的疑似,掌握其要领时,心中就像有日月一样光明,可以让我听听吗?\\
岐伯回答说:对色和脉的诊察,是上帝所重视,先师所传授的。上古时候,有位名医叫僦贷季,他研究色和脉的道理,通达神明,能联系金木水火土,四时八风六合,不脱离色脉诊法的正常规律,并能从相互变化当中,观察它的奥妙,了解它的要领。所以要想了解诊病的要领,那就是察色与脉。气色就像太阳一样有阴有晴,而脉息像月亮一样有盈有亏,经常注意气色明晦,脉息虚实的差异,这就是诊法的要领。总之,气色的变化跟四时的脉息是相应的。这一道理,上帝是极重视的,因为它合于神明。掌握了这样的诊法,就可以避免死亡而生命安全,生命延长了,人们要称颂为圣王啊!中古时候的医生治病,疾病发生了才加以治疗。先用汤液十天,祛除风痹病邪,如果十天病还没好,再用草药治疗。另外,医生和病人也要相互配合,这样,病邪才会被驱除。后世医生治病就不这样了。治病不根据四时的变化,不了解色、脉的重要,不辨别色、脉的顺逆,等到疾病已经形成了,才想起用汤液治内,微针治外,还大肆吹嘘,自以为能够治愈,结果,原来的疾病没好,又添上了新病。\\
黄帝说:我希望听到有关治疗的根本道理。\\
岐伯说:治病最重要的,在于不误用色诊脉诊。使用色脉诊法,没有疑虑,是诊治的最大原则。如果把病情的顺逆搞颠倒了,处理疾病时又不能取得病人的配合,这样,就会使病人的神气消亡,身体受到损害。所以医生一定要去掉旧习的简陋知识,钻研崭新的色脉学问,努力进取,就可以达到上古真人的水平。\\
黄帝说:我从您那儿听说了治疗的根本法则。您这番话的要领是,治疗不能丢弃气色和脉象的诊察,这我已经知道了。\\
岐伯说:诊治的极要关键,还有一个。\\
黄帝问:是什么?\\
岐伯说:这个关键就是问诊。\\
黄帝说:怎么去做呢?\\
岐伯说:关好门窗,向病人详细地询问病情,使他愿意如实地主诉病情。经过问诊并参考色脉以后,即可作出判断:如果病人面色光华,脉息和平,这叫“得神”,预后良好,如果病人面色无华,脉不应时,这叫“失神”,预后不佳。\\
黄帝说:说得好。\\
汤液醪醴论篇第十四\\
黄帝问曰:为五谷汤液及醪醴奈何?\\
岐伯对曰:必以稻米,炊之稻薪。稻米者完,稻薪者坚。\\
帝曰:何以然?\\
岐伯曰:此得天地之和,高下之宜,故能至完,伐取得时,故能至坚也。\\
帝曰:上古圣人作汤液醪醴,为而不用,何也?\\
岐伯曰:自古圣人之作汤液醪醴者,以为备耳,夫上古作汤液,故为而弗服也。中古之世,道德稍衰,邪气时至,服之万全。\\
帝曰:今之世不必已,何也?\\
岐伯曰:当今之世,必齐毒药攻其中,镵石针艾治其外也。\\
帝曰:形弊血尽而功不立者何?\\
岐伯曰:神不使也。\\
帝曰:何谓神不使?\\
岐伯曰:针石,道也。精神不进,志意不治,故病不可愈。今精坏神去,荣卫不可复收。何者?嗜欲无穷,而忧患不止,精气弛坏,荣泣卫除,故神去之而病不愈也。\\
帝曰:夫病之始生也,极微极精,必先入结于皮肤。今良工皆称曰,病成名曰逆,则针石不能治,良药不能及也。今良工皆得其法,守其数,亲戚兄弟远近,音声日闻于耳,五色日见于目,而病不愈者,亦何暇不早乎?\\
岐伯曰:病为本,工为标;标本不得,邪气不服。此之谓也。\\
帝曰:其有不从毫毛而生,五脏阳以竭也。津液充郭,其魄独居,孤精于内,气耗于外,形不可与衣相保,此四极急而动中。是气拒于内,而形施于外。治之奈何?\\
岐伯曰:平治于权衡。去宛陈莝,微动四极,温衣,缪刺其处,以复其形。开鬼门,洁净府,精以时服。五阳已布,疏涤五脏,故精自生,形自盛,骨肉相保,巨气乃平。\\
帝曰:善。\\
黄帝问道:怎样用五谷来制作汤液和醪醴呢?\\
岐伯答说:用稻米来酝酿,用稻秆做燃料。因为稻米之气完备,而稻秆则很坚硬。\\
黄帝说:这是什么道理?\\
岐伯说:稻谷得天地和气,生长在高低适宜的地方,所以得气最完备,又在适当的季节收割,所以稻秆最坚实。\\
黄帝说:上古时代的医生,制成了汤液醪醴,只是供给祭祀宾客之用,而不用它煎药,这是什么道理?\\
岐伯说:上古医生制成了汤液醪醴,是以备万一的,所以制成了,并不急于用。到了中古时代,社会上讲究养生的少了,外邪乘虚经常侵害人体,但只要吃些汤液醪醴,病也就会好的。\\
黄帝说:现在人有了病,虽然也吃些汤液醪醴,而病不一定都好,这是什么道理呢?\\
岐伯说:现在有病,必定要内服药物,外用镵石针艾,然后病才能治好。\\
黄帝说:病人形体衰败,气血竭尽,治疗不见功效,这是什么原因?\\
岐伯说:这是因为病人的精神,已经不能发挥应有作用了。\\
黄帝说:什么叫做精神不能发挥应有作用呢?\\
岐伯说:针石治病,只是引导血气而已,主要还在于病人的精神志意。如果病人的神气已经衰微,病人的志意已经散乱,那病是不会好的。而现在病人正是到了精神败坏、神气涣散,荣卫不能恢复的地步了。为什么病会发展得这样重呢?主要是由于情欲太过,又让忧患萦心,不能停止,以致精气衰败,荣血枯涩,卫气消失,所以神气就离开人体,而疾病也就不能痊愈了。\\
黄帝说:病在初起的时候,是极其轻浅而隐蔽的,病邪只是潜留在皮肤里。现在,医生一看,说是病情严重,结果针石不能奏效,汤药也不管用了。现在的医生都能掌握医道的法度,遵守医道的具体技术,与病人的关系像父母兄弟一样近,每天都能听到病人声音的变化,每天都能看到病人五色的改变,可是病却没有治好,是不是没有提早治疗的缘故呢?\\
岐伯说:病人是本,医生是标,二者必须相得;病人和医生不能相互配合,病邪就不能驱除。说的就是这种情况啊!\\
黄帝说:有的病并不先从体表发生,而是五脏的阳气衰竭。以致水气充满于皮肤,而阴气独盛,阴气独居于内,则阳气更消耗于外,形体浮肿,原来的衣服不能穿了,四肢肿急,影响内脏。这是阴气格拒于内,而水气弛张于外。对这种病怎么治疗呢?\\
岐伯说:要平复水气。根据病情衡量轻重,去淤血,消积水,叫病人轻微地活动四肢,穿温暖的衣服,使阳气渐渐传布,然后用缪刺方法,使他的形体恢复起来。再使汗液畅达,小便通利,使阴精归于平复。待五脏阳气输布了,五脏郁积荡涤了,那么精气自然会产生,形体自然会强盛,骨骼和肌肉也就会相辅相成,正气自然就恢复了。\\
黄帝说:讲得很好。\\
玉版论要篇第十五\\
黄帝问曰:余闻揆度、奇恒,所指不同,用之奈何?\\
岐伯对曰:揆度者,度病之浅深也;奇恒者,言奇病也。请言道之至数,五色脉变,揆度奇恒,道在于一。神转不回,回则不转,乃失其机。至数之要,迫近以微,著之玉版,命曰合《玉机》。\\
容色见上下左右,各在其要。其色见浅者,汤液主治,十日已。其见深者,必齐主治,二十一日已。其见大深者,醪酒主治,百日已。色夭面脱,不治,百日尽已。脉短气绝,死;病温虚甚,死。\\
色见上下左右,各在其要。上为逆,下为从。女子右为逆,左为从;男子左为逆,右为从。易,重阳死,重阴死。阴阳反他。治在权衡相夺,奇恒事也,揆度事也。\\
搏脉,痹躄,寒热之交。脉孤为消气,虚泄为夺血。孤为逆,虚为从。行奇恒之法,以太阴始。行所不胜曰逆,逆则死;行所胜曰从,从则活。八风四时之胜,终而复始,逆行一过,不复可数。论要毕矣。\\
黄帝问:我听说揆度和奇恒,这两种方法各有所指,怎样联系起来运用呢?\\
岐伯回答说:揆度是度量疾病的深浅;奇恒是辨别那些异乎寻常的疾病。请让我说说诊病的至理,就是要注意五色和脉象的变化,至于揆度和奇恒,它们的要点都在于把握决定人体生命活动的气血神机的运转。人体的气血神机,是永远运转而不回折的,如果回折不运转了,就会失去生机。这个道理虽然浅近,却关乎微妙的神机,应该刻录在玉版上,可以与《玉机真脏论》合参。\\
面容颜色变化,表现在上下左右不同的部位,应分别察看它的深浅和顺逆的要领。色见浅的,病轻,可用五谷汤液调理,十天痊愈。色见深的,病重,必须服用药剂治疗,二十一天才可恢复。如果其色过深,病情更重,必须用药酒治疗,一百天左右才能痊愈。如神色夭晦,面容瘦削,就不能治愈,到百天就会死。此外,脉气短促而阳气虚脱的,必死;温热病而阴血虚极的,也必死。\\
面色的变化表见于上下左右,必须观察其要点。病色向上移的为逆,向下移的为顺。女子病色在右侧的为逆,在左侧的为顺;男子病色在左侧的为逆,在右侧的为顺。如果病色颠倒,就是重阳、重阴,重阳、重阴主死。如果出现阴阳相反,应衡量病情,采用适当治法,使阴阳恢复平和,这就在于比较正常与异常,揣度疾病的浅深。\\
脉象搏击于指下,或为痹证,或为躄证,或为寒热之气相交为病。脉见孤绝,是阳气损耗;脉见虚弱,而又有泄泻,为阴血损伤。脉见孤绝,为逆,预后不良;脉见虚弱,为顺,预后较好。诊脉时运用奇恒之法,从手太阴经之寸口脉来研究。如出现为主令的四时、五行所克制的脉象,为逆,预后不良;如出现克制主令的四时五行的脉象,为从,预后良好。至于八风、四时之间的相互胜复,则循环无端,终而复始,假如四时气候失常,就不能用常理来推断了。这就是揆度、奇恒的全部要点。\\
诊要经终论篇第十六\\
黄帝问曰:诊要何如?\\
岐伯对曰:正月二月,天气始方,地气始发,人气在肝。三月四月,天气正方,地气定发,人气在脾。五月六月,天气盛、地气高,人气在头。七月八月,阴气始杀,人气在肺。九月十月,阴气始冰,地气始闭,人气在心。十一月十二月,冰复,地气合,人气在肾。\\
故春刺散俞,及与分理,血出而止,甚者传气,间者环也。夏刺络俞,见血而止,尽气闭环,痛病必下。秋刺皮肤,循理,上下同法,神变而止。冬刺俞窍于分理,甚者直下,间者散下。\\
春夏秋冬,各有所刺,法其所在。\\
春刺夏分,脉乱气微,入淫骨髓,病不能愈,令人不嗜食,又且少气。春刺秋分,筋挛逆气,环为咳嗽,病不愈,令人时惊,又且哭。春刺冬分,邪气著脏,令人胀,病不愈,又且欲言语。\\
夏刺春分,病不愈,令人解堕。夏刺秋分,病不愈,令人心中欲无言,惕惕如人将捕之。夏刺冬分,病不愈,令人少气,时欲怒。\\
秋刺春分,病不已,令人惕然欲有所为,起而忘之。秋刺夏分,病不已,令人益嗜卧,又且善梦。秋刺冬分,病不已,令人洒洒时寒。\\
冬刺春分,病不已,令人欲卧不能眠,眠而有见。冬刺夏分,病不愈,气上,发为诸痹。冬刺秋分,病不已,令人善渴。\\
凡刺胸腹者,必避五脏。中心者,环死;中脾者,五日死;中肾者,七日死;中肺者,五日死;中鬲者,皆为伤中,其病虽愈,不过一岁必死。刺避五脏者,知逆从也。所谓从者,鬲与脾肾之处,不知者反之。刺胸腹者,必以布憿著之,乃从单布上刺,刺之不愈复刺。刺针必肃,刺肿摇针,经刺勿摇。此刺之道也。\\
帝曰:愿闻十二经脉之终奈何?\\
岐伯曰:太阳之脉,其终也,戴眼、反折、瘛疭,其色白,绝汗乃出,出则死矣。少阳终者,耳聋,百节皆纵,目睘绝系,绝系一日半,死。其死也,色先青白,乃死矣。阳明终者,口目动作,善惊,妄言,色黄,其上下经盛,不仁,则终矣。少阴终者,面黑,齿长而垢,腹胀闭,上下不通而终矣。太阴终者,腹胀闭,不得息,善噫善呕,呕则逆,逆则面赤,不逆则上下不通,不通则面黑,皮毛焦而终矣。厥阴终者,中热嗌干,善溺心烦,甚则舌卷,卵上缩而终矣。此十二经之所败也。\\
黄帝问:诊病的要领是什么?\\
岐伯回答说:正月、二月,天气开始升发,地气开始萌动,这时人气在肝。三月、四月,天气正当方盛,地气正在发育而万物华茂而欲结实,这时人气在脾。五月、六月,天气盛极,地气升高,这时人气在头部。七月、八月,阴气开始肃杀,这时人气在肺。九月、十月,阴气渐盛,开始结冰,地气开始闭藏,这时人气在心。十一月、十二月,冰冻增厚,地气闭密,这时人气在肾。\\
所以春天刺经脉分散的腧穴,刺及分肉腠理,出血而止,病重的针刺后,其气传布才能痊愈,较轻的,候经气循环一周,就可以出针。夏天刺孙络的腧穴,出血而止,邪气除去,用手指按闭针孔,痛病必消除。秋天刺皮肤,针刺时顺着肌肉之分理,不论上下,用同样的方法,观察病人神色,转变而止。冬天深刺腧窍,到达分理之间,病重的,直刺深入;较轻的,可上下左右分散而刺。\\
春夏秋冬,各有适宜的刺法,须根据气之所在,来确定针刺部位。\\
如果春天误刺了夏天的部位,可使脉乱而气微弱,邪气就会深入,浸淫于骨髓之间,病就不能治愈,使人不思饮食,而且少气。春天误刺了秋天的部位,发为筋挛,因误刺而邪气环周于肺,又发为咳嗽,病不能愈,使人时惊惕,且悲哭。春天误刺了冬天的部位,邪气深入内脏,使人胀满,病不能愈,而且使人多言多语。\\
夏天误刺了春天的部位,病不能愈,使人筋力倦怠。夏天误刺了秋天的部位,病不能愈,使人心中不欲言,惕惕然好像有人来抓捕自己似的。夏天误刺了冬天的部位,病不能愈,使人气虚,时常发怒。\\
秋天误刺了春天的部位,病不能愈,使人惕然不宁,想要做事,一会儿又忘了。秋天误刺了夏天的部位,病不能愈,使人嗜睡,而且多梦。秋天误刺了冬天的部位,病不能愈,使人时时发冷。\\
冬天误刺了春天的部位,病不能愈,使人欲睡又不能安眠,即便得眠,又梦境纷纭。冬天误刺了夏天的部位,病不能愈,脉气上逆,邪气闭阻于脉,发为诸痹。冬天误刺了秋天的部位,病不能愈,使人常常口渴。\\
凡针刺胸腹部,必须避开五脏。误中心脏,经气环身一周就会死;误中脾脏,五日就会死;误中肾脏,七日就会死;误中肺脏,五日就会死;误中膈膜的,皆为伤中,虽然当时表面上看疾病痊愈了,但不过一年其人必死。针刺时避开五脏的关键,在于要知道下针的逆从。所谓从,就是要明白膈和脾肾等的具体部位,不知道的人不能避开,就会刺伤五脏,那就是逆。凡针刺胸腹时,应先用布巾像裹腿一样缠绕胸腹,然后从单布上针刺,如果刺之而不愈,可以再刺。在针刺时,必须安静严肃,如刺脓肿病,可以用摇针手法以出脓血,如刺经脉病,就不要摇针。这是刺法的一般准则。\\
黄帝说:希望听听十二经气绝的临终表现是怎样的?\\
岐伯回答说:太阳经脉气绝的临终表现是,两目上视,身背反张,手足抽搐,面色发白,出绝汗,绝汗一出,就要死了。少阳经脉气绝的临终表现是,耳聋,周身骨节松懈,两目直视如惊,眼睛不转动,一日半就要死了,临死时,面色先现青白色,就要死了。阳明经脉气绝的临终表现是,口眼歪斜圴动,时发惊惕,言语失常,面色发黄,其经脉上下所过之处,出现盛躁之证,由盛躁而麻木不仁,就要死了。少阴经脉气绝的临终表现是,面色灰黑,牙龈收缩而牙齿似乎变长,并满积污垢,腹部胀闭,上下不相通,就要死了。太阴经脉气绝的临终表现是,腹胀闭塞,不得喘息,常欲嗳气,并且呕吐,呕则气上逆,气上逆则面红赤,如气不上逆,则上下不通,不通则面色灰黑,皮毛憔悴而死了。厥阴经脉气绝的临终表现是,病人胸中发热,咽喉干燥,小便失禁,心胸烦躁,渐至舌卷,睾丸上缩,便要死了。这就是十二经脉气败绝的表现。\\
卷五\\
脉要精微论篇第十七\\
黄帝问曰:诊法何如?\\
岐伯对曰:诊法常以平旦,阳气未动,阴气未散,饮食未进,经脉未盛,络脉调匀,气血未乱,故乃可诊有过之脉。\\
切脉动静而视精明,察五色,观五脏有余不足,六腑强弱,形之盛衰,以此参伍,决死生之分。\\
夫脉者,血之府也。长则气治,短则气病,数则烦心,大则病进。上盛则气高,下盛则气胀。代则气衰,细则气少,涩则心痛。浑浑革至如涌泉,病进而危弊;绵绵其去如弦绝,死。\\
夫精明五色者,气之华也。赤欲如白裹朱,不欲如赭;白欲如鹅羽,不欲如盐;青欲如苍璧之泽,不欲如蓝;黄欲如罗裹雄黄,不欲如黄土;黑欲如重漆色,不欲如地苍。五色精微象见矣,其寿不久也。夫精明者,所以视万物,别白黑,审短长。以长为短,以白为黑,如是则精衰矣。\\
五脏者,中之守也。中盛藏满,声如从室中言,是中气之湿也。言而微,终日乃复言者,此夺气也。衣被不敛,言语善恶,不避亲疏者,此神明之乱也。仓廪不藏者,是门户不要也。水泉不止者,是膀胱不藏也。得守者生,失守者死。\\
夫五府者,身之强也。头者,精明之府,头倾视深,精神将夺矣。背者,胸中之府,背曲肩随,府将坏矣。腰者,肾之府,转摇不能,肾将惫矣。膝者,筋之府,屈伸不能,行则偻附,筋将惫矣。骨者,髓之府,不能久立,行则振掉,骨将惫矣。得强则生,失强则死。\\
岐伯曰:反四时者,有余为精,不足为消。应太过,不足为精;应不足,有余为消。阴阳不相应,病名曰关格。\\
帝曰:脉其四时动奈何?知病之所在奈何?知病之所变奈何?知病乍在内奈何?知病乍在外奈何?请问此五者,可得闻乎?\\
岐伯曰:请言其与天运转也。万物之外,六合之内。天地之变,阴阳之应,彼春之暖,为夏之暑;彼秋之忿,为冬之怒;四变之动,脉与之上下。以春应中规,夏应中矩,秋应中衡,冬应中权。是故冬至四十五日,阳气微上,阴气微下;夏至四十五日,阴气微上,阳气微下。\\
阴阳有时,与脉为期。期而相失,知脉所分;分之有期,故知死时。微妙在脉,不可不察;察之有纪,从阴阳始。始之有经,从五行生;生之有度,四时为宜。补泻勿失,与天地如一。得一之情,以知死生。是故声合五音,色合五行,脉合阴阳。\\
是知阴盛则梦涉大水恐惧,阳盛则梦大火燔灼,阴阳俱盛则梦相杀毁伤。上盛则梦飞,下盛则梦堕,甚饱则梦予,甚饥则梦取。肝气盛则梦怒,肺气盛则梦哭。短虫多则梦聚众,长虫多则梦相击毁伤。\\
是故持脉有道,虚静为保。春日浮,如鱼之游在波;夏日在肤,泛泛乎万物有余;秋日下肤,蛰虫将去;冬日在骨,蛰虫周密,君子居室。故曰:知内者按而纪之,知外者终而始之。此六者,持脉之大法。\\
心脉搏坚而长,当病舌卷不能言;其耎而散者,当消环自已。肺脉搏坚而长,当病唾血;其耎而散者,当病灌汗,至令不复散发也。肝脉搏坚而长,色不青,当病坠若搏,因血在胁下,令人喘逆;其耎而散,色泽者,当病溢饮,溢饮者,渴暴多饮,而易入肌皮肠胃之外也。胃脉搏坚而长,其色赤,当病折髀;其耎而散者,当病食痹。脾脉搏坚而长,其色黄,当病少气;其耎而散,色不泽者,当病足气肿,若水状也。肾脉搏坚而长,其色黄而赤者,当病折腰;其耎而散者,当病少血,至令不复也。\\
帝曰:诊得心脉而急,此为何病?病形何如?\\
岐伯曰:病名心疝,少腹当有形也。\\
帝曰:何以言之?\\
岐伯曰:心为牡脏,小肠为之使,故曰少腹当有形也。\\
帝曰:诊得胃脉,病形何如?\\
岐伯曰:胃脉实则胀,虚则泄。\\
帝曰:病成而变,何谓?\\
岐伯曰:风成为寒热,瘅成为消中,厥成为巅疾,久风为飧泄,脉风成为疠。病之变化,不可胜数。\\
帝曰:诸痈肿筋挛骨痛,此皆安生?\\
岐伯曰:此寒气之钟,八风之变也。\\
帝曰:治之奈何?\\
岐伯曰:此四时之病,以其胜治之,愈也。\\
帝曰:有故病,五脏发动,因伤脉色,各何以知其久暴之病乎?\\
岐伯曰:悉乎哉问也!征其脉小,色不夺者,新病也;征其脉不夺,其色夺者,此久病也;征其脉与五色俱夺者,此久病也;征其脉与五色俱不夺者,新病也。肝与肾脉并至,其色苍赤,当病毁伤,不见血,已见血,湿若中水也。\\
尺内两傍,则季胁也。尺外以候肾,尺里以候腹。中附上,左外以候肝,内以候鬲;右外以候胃,内以候脾。上附上,右外以候肺,内以候胸中,左外以候心,内以候膻中。前以候前,后以候后。上竟上者,胸喉中事也,下竟下者,少腹、腰、股、膝、胫、足中事也。\\
粗大者,阴不足阳有余,为热中也。来疾去徐,上实下虚,为厥巅疾。来徐去疾,上虚下实,为恶风也。故中恶风者,阳气受也。有脉俱沉细数者,少阴厥也。沉细数散者,寒热也。浮而散者,为眴仆。\\
诸浮不躁者,皆在阳,则为热,其有躁者在手。诸细而沉者,皆在阴,则为骨痛,其有静者在足。数动一代者,病在阳之脉也,泄及便脓血。诸过者切之,涩者,阳气有余也;滑者,阴气有余也。阳气有余,为身热无汗;阴气有余,为多汗身寒;阴阳有余,则无汗而寒。推而外之,内而不外,有心腹积也。推而内之,外而不内,身有热也。推而上之,上而不下,腰足清也;推而下之,下而不上,头项痛也。按之至骨,脉气少者,腰脊痛而身有痹也。\\
黄帝问道:诊脉的方法如何?\\
岐伯回答说:诊脉常在清晨,因为这时阳气未曾扰动,阴气还未散尽,又未用过饮食,经脉之气不亢盛,络脉之气也调和,气血未扰乱,所以容易诊出有病的脉象。\\
在诊察病人脉象动静变化的同时,还要看他的两目瞳神,面部色泽,从而分辨五脏是有余还是不足,六腑是强还是弱,形体是盛还是衰,将这几个方面加以综合考察,来判别病人的死、生。\\
脉是血液聚会的地方,而血的循行,要依赖气的统率。脉长说明气机顺达,脉短说明气分有病,脉数说明心里烦热;脉大表示病势进增。若见上部脉盛,是病气塞于胸;若见下部脉盛,是病气胀于腹。代脉是病气衰,细脉是病气少,涩脉是病气痛。脉来刚硬混乱,势如涌泉,这是病情加重,到了危险地步;若脉来似有似无,其去如弓弦断绝,那是必死的。\\
眼目、面部五色,是精气的外在表现。赤色应该像白绸里裹着硃砂一样,隐现着红润,不应像赭石那样,赤而带紫;白色应该像鹅的羽毛,白而光洁,不应像盐那样,白而晦暗;青色应该像苍璧,青而润泽,不应像青靛那样,青而沉暗;黄色应该像罗裹雄黄,黄中透红,不应像土那样,黄而沉滞;黑色应该像重漆,黑而明润,不应像地苍色那样,黑而枯暗。假如五脏真脏之色显露于外,那么寿命也就不能长了。人的眼睛,是用来观察万物,辨别黑白,审察长短的。如果长短不分,黑白颠倒,就证明精气衰败了。\\
五脏的作用是藏精守内的。如果腹气盛,脏气虚满,说话声音重浊,像从内室中发出的一样,这是中气被湿邪阻滞的缘故。如果讲话时声音低微,好半天才说下句话,这表明正气衰败了。如果病人不知收拾衣被,言语错乱,不分亲疏远近,这是精神错乱了。如果肠胃不能纳藏水谷,大便失禁,这是肾虚不能固摄造成的;如果小便失禁,这是膀胱不能闭藏造成的。总之,如果五脏能够内守,病人的健康就能恢复;否则,五脏失守,病人就会死亡。\\
五府是人体强健的基础。头是精明之府,如果头部下垂,眼胞内陷,说明精神要衰败了。背是胸之府,如果背弯曲而肩下垂,那是胸要坏了。腰是肾之府,如果腰部不能转动,那是肾气要衰竭了。膝是筋之府,如果屈伸困难,走路时曲背低头,那是筋要疲惫了。骨是髓之府,如果不能久立,行走动摇不定,那是骨要衰颓了。总之,如五府能够由弱转强,就可复生;否则,就会死亡。\\
岐伯说:脉气有时会与四时之气相反,如相反的形象为有余,这是邪气胜了精气;相反的形象为不足,这是由于血气先已消损。按照时令来讲,脏气当旺,脉气应有余,却反见不足的,这是邪气胜了精气;脉气应不足,却反见有余的,这是正不胜邪,血气消损而邪气猖獗。这种阴阳气血不相顺从、邪正不相适应的情况,发生的疾病名叫关格。\\
黄帝问道:脉有四时的变化是怎样的?从诊脉知道疾病的所在是怎样的?从诊脉知道疾病的变化是怎样的?从诊脉知道疾病忽然在内是怎样的?从诊脉知道疾病忽然在外是怎样的?请问这五个问题,可以讲给我听吗?\\
岐伯回答说:让我说说这五者的变化与天地运转的关系吧。世间万物之外,四方上下六合之内。天地的变化,阴阳的反应,如春天的舒缓,发展成为夏天的酷热;如秋天的劲急,发展成为冬天的严寒;脉象的往来上下与这四时的变迁是相应的。春脉之应象中规,夏脉之应象中矩,秋脉之应象中衡,冬脉之应象中权。所以四时阴阳的情况,冬至一阳生,到四十五天,阳气微升,阴气微降;夏至一阴生,到四十五天,阴气微升,阳气微降。\\
这阴阳升降,有一定时间性,与脉象的变化相一致。假如脉象和四时不相应,就可从脉象里知道病是属于何脏;再根据脏气的盛衰,就可以推究出病人的死期。这里的微妙都在脉象上,不可不细心地体察;而体察是有一定要领的,必须从阴阳开始。阴阳亦有开端,它是借着五行产生的;而它的产生又是按一定的法则,即以四时的变化为其规律。看病时就要遵循着这个规律而不能偏离,将脉象与天地阴阳的变化联系起来考虑。如果真正掌握了这种联系起来看问题的诀窍,就可以预知死生了。总起来说,人的声音是与五音相适应的,人的气色是与五行相适应的,而人的脉象则是与天地四时的阴阳变化相适应的。\\
于是知道阴气盛则梦渡大水而恐惧,阳气盛则梦见大火烧灼,阴阳俱盛,则梦相互残杀毁伤。上部盛则梦上飞,下部盛则梦下堕,如过饱的时候,则梦送物于人,如过饥的时候,则梦欲取物。肝气盛,则梦发怒,肺气盛,则梦哭泣。如腹中短虫多,则梦见众人集聚;长虫多,则梦见打架损伤。\\
所以持脉有一定的要诀,虚心静气是宝贵的。脉象随着季节的不同而不同。春天脉上浮,像鱼游波中一样;夏天脉充皮肤,浮泛然像万物充盛似的;秋天脉见微沉,似在肤下,就像蛰虫将要入穴一样;冬天脉沉在骨,像蛰虫密藏洞穴,人们深居室内似的。所以说:要知道脉之在里怎样,必须深按才能得其要领;而要知道脉之在表怎样,则要着重根据病情来推究致病的本源。以上春、夏、秋、冬、内、外这六点,就是持脉的大法。\\
心脉搏坚而长的,当病舌卷而不能言语;其脉软而散的,则属刚脉渐转柔和,当营卫之气循环一周的时间,病自痊愈。肺脉搏坚而长的,当病唾血;其脉软而散的,当病盗汗,汗出如漏,不能再用发散法治疗了。肝脉搏坚而长,面部不见青色的,当为跌坠受伤或搏击致伤等病,因而淤血积聚胁下,使人喘逆;其脉软而散,而面色鲜泽的,当病溢饮,溢饮病是由于口渴时暴饮多饮,以致水气流入肌肉皮肤之间、肠胃之外而致。胃脉搏坚而长,面色红赤,当病髀痛如折;其脉软而散,则胃气不足,当病食痹。脾脉搏坚而长,面色发黄的,当病少气;其脉软而散,面无光泽的,当病足胫浮肿,像水肿之状。肾脉搏坚而长,面黄而赤的,当病腰痛如折;其脉软而散,当病精血虚少,使身体不能恢复健康。\\
黄帝问:诊得心脉绷急,这是什么病?病的症状怎样?\\
岐伯说:病名叫做心疝,少腹部位当有形征出现。\\
黄帝说:为什么这么说?\\
岐伯说:心为阳脏,与小肠相表里,小肠位在少腹中,所以说少腹当有形征出现。\\
黄帝说:诊得胃脉有病,它的症状怎样?\\
岐伯说:如果胃脉盛实,其病为脘腹胀满;胃脉虚弱,其病为泄泻。\\
黄帝问:疾病的成因及其变化怎样?\\
岐伯说:因于风邪,则变为寒热;因于热邪,则变为消中;因于气逆不已,则变为癫疾;因于久风入中,则变为飧泄;因于风寒客于脉而不去,则变为疠风。疾病变化多端,难以说完。\\
黄帝说:各种痈肿、筋挛、骨痛,都是怎样产生的?\\
岐伯说:这是由于寒气聚集,风邪侵袭而变成的。\\
黄帝说:怎样治疗?\\
岐伯说:这是四时邪气所致之疾病,用五行相胜法治疗,可以痊愈。\\
黄帝说:有旧病从五脏发动,因而影响到脉色,怎样辨别是久病还是新病呢?\\
岐伯说:问得真详细啊!只要验看脉色,就可以辨别出来。如脉虽小而气色不失常的,是新病;如脉不失常而面色失常的,是久病;如脉与色都失常的,是久病;如脉与色都不失常的,是新病。如肝脉与肾脉并至,它的皮色见苍赤色,这是因为暴病跌打损伤,不见血或已见血,形体必肿,如同因于湿邪或水气中伤的症状,这是淤血肿胀。\\
尺部脉两旁以候察季胁。尺外以候察肾,尺内以候察腹。关部脉,左外以候察肝,内以候察膈,右外以候察胃,内以候察脾。寸部脉,右外以候察肺,内以候察胸中,左外以候察心,内以候察膻中。前以候察前,后以候察后。上竟上,以候察胸喉之疾病;下竟下,以候察少腹、腰、股、膝、胫、足的疾病。\\
脉象洪大的,是阴不足而阳有余,见于热中病。脉象来急疾而去徐缓的,是上部实而下部虚,见于厥逆和癫仆等病。脉象来徐缓而去疾急的,是上部虚而下部实,见于疠风病。中了恶风,是阳气先受病。有脉象都沉细数的,是足少阴经厥逆病。如脉见沉细数散的,是寒热病。脉浮而散的,是眩仆病。\\
各种浮脉而不躁急的,病在表阳,可有发热;如浮而躁,则病在手三阳经。各种细脉而沉的,病在里阴,发为骨节疼痛,如果细沉而静,病在足三阴经。数脉而有间歇的,其病在阳脉,可见泄泻及大便脓血。诊察各种有病之脉,见涩象的,是阳气有余;滑象的,是阴气有余。阳气有余,则身发热而无汗;阴气有余,则身多汗而发冷;阴气阳气都有余,则无汗而身寒。推脉向外,而脉气内而不外的,是心腹有积聚在内。推脉向内,而脉气外而不内,是身体有热。如推而向上,而脉上而不下的,是腰足之间寒冷;如推而下之,而脉下而不上的,是头项疼痛。重按至骨,而脉气少的,是腰脊疼痛而身有痹病。\\
平人气象论篇第十八\\
黄帝问曰:平人何如?\\
岐伯对曰:人一呼脉再动,一吸脉亦再动。呼吸定息脉五动,闰以太息,命曰平人。平人者不病也。常以不病调病人,医不病,故为病人平息,以调之为法。\\
人一呼脉一动,一吸脉一动,曰少气。人一呼脉三动,一吸脉三动而躁,尺热曰病温;尺不热脉滑曰病风。人一呼脉四动以上曰死,脉绝不至曰死,乍疏乍数曰死。\\
平人之常气禀于胃,胃者,平人之常气也。人无胃气曰逆,逆者死。\\
春胃微弦曰平,弦多胃少曰肝病。但弦无胃曰死;胃而有毛曰秋病;毛甚曰今病。脏真散于肝,肝藏筋膜之气也。\\
夏胃微钩曰平,钩多胃少曰心病。但钩无胃曰死;胃而有石曰冬病;石甚曰今病。脏真通于心,心藏血脉之气也。\\
长夏胃微软弱曰平,弱多胃少曰脾病。但弱无胃曰死;软弱有石曰冬病;石甚曰今病。脏真濡于脾,脾藏肌肉之气也。\\
秋胃微毛曰平,毛多胃少曰肺病。但毛无胃曰死;毛而有弦曰春病;弦甚曰今病。脏真高于肺,肺藏皮毛之气也。\\
冬胃微石曰平,石多胃少曰肾病。但石无胃曰死;石而有钩曰夏病;钩甚曰今病。脏真下于肾,肾藏骨髓之气也。\\
胃之大络,名曰虚里。贯鬲络肺,出于左乳下,其动应手,脉宗气也。盛喘数绝者,则病在中;结而横,有积矣;绝不至曰死。乳之下,其动应衣,宗气泄也。\\
欲知寸口太过与不及。寸口之脉中手短者,曰头痛。寸口脉中手长者,曰足胫痛。寸口脉中手促上击者,曰肩背痛。寸口脉沉而坚者,曰病在中。寸口脉浮而盛者,曰病在外。寸口脉沉而弱,曰寒热及疝瘕少腹痛。寸口脉沉而横,曰胁下有积,腹中有横积痛。寸口脉沉而喘,曰寒热。脉盛滑坚者,曰病在外。脉小实而坚者,曰病在内。脉小弱以涩,谓之久病。脉滑浮而疾者,谓之新病。脉急者,曰疝瘕少腹痛。脉滑曰风,脉涩曰痹。缓而滑曰热中。盛而紧曰胀。脉从阴阳,病易已;脉逆阴阳,病难已。脉得四时之顺,曰病无他;脉反四时及不间脏,曰难已。\\
臂多青脉,曰脱血。尺缓脉涩,谓之解汃安卧。尺热脉盛,谓之脱血。尺涩脉滑,谓之多汗。尺寒脉细,谓之后泄。脉尺粗常热者,谓之热中。\\
肝见庚辛死;心见壬癸死;脾见甲乙死;肺见丙丁死;肾见戊己死。是谓真脏见,皆死。\\
颈脉动喘疾咳,曰水。目裹微肿,如卧蚕起之状,曰水。溺黄赤安卧者,黄疸。已食如饥者,胃疸。面肿曰风,足胫肿曰水,目黄者曰黄疸。妇人手少阴脉动甚者,妊子也。\\
脉有逆从四时,未有脏形,春夏而脉瘦,秋冬而脉浮大,命曰逆四时也。风热而脉静;泄而脱血脉实;病在中,脉虚;病在外,脉涩坚者。皆难治,命曰反四时也。\\
人以水谷为本,故人绝水谷则死。脉无胃气亦死。所谓无胃气者,但得真脏脉,不得胃气也。所谓脉不得胃气者,肝不弦,肾不石也。\\
少阳脉至,乍数乍疏,乍短乍长;阳明脉至,浮大而短;太阳脉至,洪大以长。\\
夫平心脉来,累累如连珠,如循琅玕,曰心平,夏以胃气为本。病心脉来,喘喘连属,其中微曲,曰心病。死心脉来,前曲后居,如操带钩,曰心死。\\
平肺脉来,厌厌聂聂,如落榆荚,曰肺平,秋以胃气为本。病肺脉来,不上不下,如循鸡羽,曰肺病。死肺脉来,如物之浮,如风吹毛,曰肺死。\\
平肝脉来,软弱招招,如揭长竿末梢,曰肝平,春以胃气为本。病肝脉来,盈实而滑,如循长竿,曰肝病。死肝脉来,急益劲,如新张弓弦,曰肝死。\\
平脾脉来,和柔相离,如鸡践地,曰脾平。长夏以胃气为本。病脾脉来,实而盈数,如鸡举足,曰脾病。死脾脉来,锐坚如乌之喙,如鸟之距,如屋之漏,如水之流,曰脾死。\\
平肾脉来,喘喘累累如钩,按之而坚,曰肾平,冬以胃气为本。病肾脉来,形如引葛,按之益坚,曰肾病。死肾脉来,发如夺索,辟辟如弹石,曰肾死。\\
黄帝问道:平人的脉象怎样呢?\\
岐伯答说:平人的脉象,一呼脉跳动两次,一吸脉也跳动两次,一呼一吸叫做一息。另外,一吸终了到一呼开始的交换时间,共有五次搏动,这是闰以太息,叫做平人。平人也就是无病的人。诊脉的法则,一般以不病的平人为标准来诊察病人,医生就是无病的平人,所以平调自己的呼吸来为病人诊脉,这是一般的法则。\\
人一呼,脉跳动一次;一吸,脉也跳动一次,这是气虚的现象。若人一呼,脉跳动三次;一吸,脉也跳动三次并且躁急,尺部皮肤发热,这是温病;尺肤不热,脉象往来流利的,这是风病。若人一呼,脉跳动在四次以上的必死,脉象中断不复至的必死,脉象忽慢忽快的也必死。\\
人的正常脉气来源于胃,胃气就是平人脉息的正常之气。人的脉息如无胃气,是逆象,逆象主死。\\
春时的脉象,弦中稍带有冲和的胃气是平脉,如果弦多而冲和的胃气少,就是肝病。假如只见弦脉而无冲和的胃气,就要死亡;若虽有胃气,而兼见毛脉,这是春见毛脉,至秋天就要生病;倘若毛脉太甚,就会立即生病。春天脏真之气散发于肝,肝脏藏筋膜之气。\\
夏时的脉象,钩中稍带有冲和的胃气是平脉,如果钩多而冲和的胃气少,就是心脏有病。假如只见钩脉而无冲和的胃气,就要死亡;若虽有胃气,而兼见石脉,这是夏见冬脉,至冬天就要生病;倘若石脉太甚,就会立即生病。夏天脏真之气通于心,心藏血脉之气。\\
长夏的脉象,微软弱而有冲和的胃气是平脉,如果弱多而冲和的胃气少,就是脾脏有病。假如但见弱脉而无冲和的胃气,就要死亡;若软弱脉中,兼见石脉,到了冬天就要生病;倘若石脉太甚,就会立即生病。长夏脏真之气濡润于脾,脾脏主肌肉之气。\\
秋时的脉象,微毛而有冲和之象的是平脉,如果毛多而冲和的胃气少,就是肺脏有病。假如但见毛脉而无胃气,就要死亡;若毛脉中兼见弦脉,至春天就要生病;倘若弦脉太甚,就会立即生病。秋时脏真之气高藏于肺,肺脏主藏皮毛之气。\\
冬时的脉象,沉石而有冲和之象的是平脉,如果石多而冲和的胃气少,就是肾脏有病。假如但见石脉而无胃气,就要死亡;若沉石脉中兼见钩象,至夏天就要生病;倘若钩脉太甚了,就会立即生病。冬时脏真之气下藏于肾,肾脏主藏骨髓之气。\\
胃经的大络,叫做虚里。贯膈而上络于肺,出于左乳下,其脉搏动应手,这是脉的宗气。倘若跳动极剧,而时兼断绝,这是病在膻中的症候;若见跳动时止,位置横移的,主病有积块;倘若脉绝不至,就要死亡。如果乳下虚里处脉搏动剧烈振衣,是宗气外泄的现象。\\
要了解寸口的太过与不及。寸口脉应指而短,其病头痛。寸口脉应指而长,其病足胫痛。寸口脉应指短促迫疾,有上无下,主肩背痛。寸口脉应指沉紧的,其病在中。寸口脉应指浮盛的,其病在表。寸口脉应指沉弱的,主寒热及疝瘕积聚少腹痛。寸口脉应指沉紧并有横斜的形状,主胁下,腹中有横积作痛。寸口脉应指沉而喘,病发寒热。脉象盛滑而紧的,是外在的六腑病,脉象小实而坚的,是内在的五脏病。脉来小弱而涩的,主久病。脉来浮滑而疾的,主新病。脉来绷急的,主病疝瘕少腹作痛。脉来滑利的,主病风。脉来涩滞的,主病痹。脉来缓滑的,其病热中。脉来盛紧的,主病腹胀。总之,脉象与四时阴阳相应,病易痊愈;脉象与四时阴阳相反,病就不易好了。脉与四时相应为顺,即使患病,也无其他危险;如脉与四时相反,病是难以痊愈的。\\
手臂多见青脉,是由于失血。尺肤缓而脉见涩象,主倦怠无力,喜卧。尺肤热而脉来盛,主大失血。尺肤涩,脉来滑,主多汗。尺肤寒,脉来细,主大便泄泻。尺肤粗,脉气常显热者,主热在里。\\
肝之真脏脉出现,至庚辛日死;心之真脏脉出现,至壬癸日死;脾之真脏脉出现,至甲乙日死;肺之真脏脉出现,至丙丁日死;肾之真脏脉出现,至戊己日死。这就是真脏脉出现死亡的日期。\\
颈部脉搏动异常,并见喘咳症状,主水病。眼胞浮肿如蚕眠后之状,也是水病。小便颜色黄赤,喜卧,是黄疸病。食后仍觉得饥饿,是胃疸病。面部浮肿为风,足胫肿为水,眼珠发黄的,是黄疸。妇人手少阴脉动甚的,是怀孕的现象。\\
脉有逆四时的,就是当其时不出现主时之脏的脉象,却反见它脏的脉象,如春夏的脉反见瘦小,秋冬的脉反见浮大,这就叫做逆四时。风热的脉应该躁,反见沉静;泄泻脱血的病,脉应该虚,反见实脉;病在内的,脉应实,反见虚;病在外的,脉应浮滑,反见涩坚。这样的病都难治,是因为违反了四时。\\
人以水谷为生命的根本,所以断绝了水谷,就要死。脉没有胃气,也要死。什么是无胃气,就是仅见真脏脉,而没有冲和胃气的脉。所说的脉无冲和胃气,就是肝脉不见弦象,肾脉不见石象。\\
少阳主正月二月,这时的脉来,是乍密乍疏,乍短乍长的;阳明主三月四月,这时的脉来,是浮大而短的;太阳主五月六月,这时的脉来,是洪大而长的。\\
正常心脉来时,像一颗颗连珠,连续不断地流转,如抚摩琅玕的圆滑,这是平脉,夏时是以胃气为本的。如果心脏有了病,脉就显出非常急数,带有微曲之象,这是病脉。如果脉来前曲后居,如执带钩一样,全无和缓之意,这是死脉。\\
正常肺脉来时,轻浮虚软,像吹落榆荚一样轻缓,这是平脉,秋季是以胃气为本。如果脉来不上不下,如摩鸡的羽毛一样,毛中含有坚劲之意,这是病脉。如果脉来如草浮在水上,如风吹毛动般轻浮不定,这是死脉。\\
正常肝脉来时,像举着竿子,那竿子末梢显得长而软,这是平脉,春季是以胃气为本。如果脉来满指滑实,像抚摩长竿一样,这是病脉。如果脉来急而有劲,像新张弓弦似的,这是死脉。\\
正常脾脉来时,和柔相附有神,像鸡爪落地一样和缓,这是平脉。长夏季节是以胃气为本。如果脉来充实而数,像鸡往来急走,就是病脉。如果脉来如雀嘴啄物一样坚硬,如鸟跃跳一样快速,如屋漏水一样点滴无伦,如水流之速,去而不返,这是死脉。\\
正常肾脉来时,连绵小坚圆滑,有如心之钩脉,按之坚如石,这是平脉,冬时是以胃气为本。如果脉来形如牵引葛藤,按之更坚,这是病脉。如果脉来像解索一般,数而散乱,又像弹石一样,促而坚硬,这是死脉。\\
卷六\\
玉机真藏论篇第十九\\
黄帝问曰:春脉如弦,何如而弦?\\
岐伯对曰:春脉者肝也,东方木也,万物之所以始生也。故其气来,软弱轻虚而滑,端直以长,故曰弦。反此者病。\\
帝曰:何如而反?\\
岐伯曰:其气来实而强,此谓太过,病在外;其气来不实而微,此谓不及,病在中。\\
帝曰:春脉太过与不及,其病皆何如?\\
岐伯曰:太过则令人善忘,忽忽眩冒而巅疾;其不及,则令人胸痛引背,下则两胁胠满。\\
帝曰:善。\\
帝曰:夏脉如钩,何如而钩?\\
岐伯曰:夏脉者心也,南方火也,万物之所以盛长也。故其气来盛去衰,故曰钩。反此者病。\\
帝曰:何如而反?\\
岐伯曰:其气来盛去亦盛,此谓太过,病在外;其气来不盛去反盛,此谓不及,病在中。\\
帝曰:夏脉太过与不及,其病皆何如?\\
岐伯曰:太过则令人身热而骨痛,为浸淫;其不及则令人烦心,上见咳唾,下为气泄。\\
帝曰:善。\\
帝曰:秋脉如浮,何如而浮?\\
岐伯曰:秋脉者肺也,西方金也,万物之所以收成也。故其气来,轻虚以浮,来急去散,故曰浮。反此者病。\\
帝曰:何如而反?\\
岐伯:其气来,毛而中央坚,两傍虚,此谓太过,病在外;其气来,毛而微,此谓不及,病在中。\\
帝曰:秋脉太过与不及,其病皆何如?\\
岐伯曰:太过则令人逆气而背痛,愠愠然;其不及,则令人喘,呼吸少气而咳,上气见血,下闻病音。\\
帝曰:善。\\
帝曰:冬脉如营,何如而营?\\
岐伯曰:冬脉者肾也,北方水也,万物之所以合藏也。故其气来沉以濡,故曰营。反此者病。\\
帝曰:何如而反?\\
岐伯曰:其气来如弹石者,此谓太过,病在外;其去如数者,此谓不及,病在中。\\
帝曰:冬脉太过与不及,其病皆何如?\\
岐伯曰:太过则令人解汃,脊脉痛,而少气,不欲言;其不及则令人心悬如病饥,尐中清,脊中痛,少腹满,小便变。\\
帝曰:善。\\
帝曰:四时之序,逆从之变异也,然脾脉独何主?\\
岐伯曰:脾脉者土也,孤脏以灌四傍者也。\\
帝曰:然则脾善恶,可得见之乎?\\
岐伯曰:善者不可得见,恶者可见。\\
帝曰:恶者何如可见?\\
岐伯曰:其来如水之流者,此谓太过,病在外;如鸟之喙者,此谓不及,病在中。\\
帝曰:夫子言脾为孤脏,中央土以灌四傍,其太过与不及,其病皆何如?\\
岐伯曰:太过则令人四支不举;其不及则令人九窍不通,名曰重强。\\
帝瞿然而起,再拜稽首曰:善。吾得脉之大要,天下至数。五色脉变,揆度奇恒,道在于一。神转不迴,迴则不转,乃失其机。至数之要,迫近以微,著之玉版,藏之脏腑,每旦读之,名曰《玉机》。\\
五脏受气于其所生,传之于其所胜,气舍于其所生,死于其所不胜。病之且死,必先传行至其所不胜,病乃死,此言气之逆行也。肝受气于心,传之于脾,气舍于肾,至肺而死。心受气于脾,传之于肺,气舍于肝,至肾而死。脾受气于肺,传之于肾,气舍于心,至肝而死。肺受气于肾,传之于肝,气舍于脾,至心而死。肾受气于肝,传之于心,气舍于肺,至脾而死。此皆逆死也。一日一夜五分之,此所以占死者之早暮也。\\
黄帝曰:五脏相通,移皆有次。五脏有病,则各传其所胜。不治,法三月若六月,若三日若六日,传五脏而当死,是顺传所胜之次。故曰:别于阳者,知病从来;别于阴者,知死生之期,言至其所困而死。\\
是故风者百病之长也。今风寒客于人,使人毫毛毕直,皮肤闭而为热,当是之时,可汗而发也;或痹不仁肿痛,当是之时,可汤熨及火灸刺而去之。弗治,病入舍于肺,名曰肺痹,发咳上气。弗治,肺传之肝,病名曰肝痹,一名曰厥,胁痛出食,当是之时,可按若刺耳。弗治,肝传之脾,病名曰脾风,发瘅,腹中热,烦心出黄,当此之时,可按可药可浴。弗治,脾传之肾,病名曰疝瘕,少腹冤热而痛,出白,一名曰蛊,当此之时,可按可药。弗治,肾传之心,筋脉相引而急,病名曰瘛,当此之时,可灸可药。弗治,满十日,法当死。肾因传之心,心即复反传而行之肺,发寒热,法当三日死,此病之次也。\\
然其卒发者,不必治于传,或其传化有不以次,不以次入者。忧恐悲喜怒,令不得以其次,故令人有卒病矣。因而喜则肾气乘矣,怒则肺气乘矣,思则肝气乘矣,恐则脾气乘矣,忧则心气乘矣。此其道也。故病有五,五五二十五变,反其传化。传,乘之名也。\\
大骨枯槁,大肉陷下,胸中气满,喘息不便,其气动形,期六月死,真脏脉见,乃予之期日。\\
大骨枯槁,大肉陷下,胸中气满,喘息不便,内痛引肩项,期一月死,真脏见,乃予之期日。\\
大骨枯槁,大肉陷下,胸中气满,喘息不便,内痛引肩项,身热,脱肉破夬,真脏见,十月之内死。\\
大骨枯槁,大肉陷下,肩髓内消,动作益衰,真脏来见,期一岁死,见其真脏,乃予之期日。\\
大骨枯槁,大肉陷下,胸中气满,腹内痛,心中不便,肩项身热,破夬脱肉,目匡陷,真脏见,目不见人,立死;其见人者,至其所不胜之时则死。\\
急虚身中卒至,五脏绝闭,脉道不通,气不往来,譬于堕溺,不可为期。其脉绝不来,若人一息五六至,其形肉不脱,真脏虽不见,犹死也。\\
真肝脉至,中外急,如循刀刃责责然,如新张弓弦,色青白不泽,毛折,乃死。真心脉至,坚而搏,如循薏苡子累累然,色赤黑不泽,毛折,乃死。真肺脉至,大而虚,如以毛羽中人肤,色白赤不泽,毛折,乃死。真肾脉至,搏而绝,如指弹石辟辟然,色黑黄不泽,毛折,乃死。真脾脉至,弱而乍数乍疏,色黄青不泽,毛折,乃死。诸真脏脉见者,皆死不治也。\\
黄帝曰:见真脏曰死,何也?\\
岐伯曰:五脏者,皆禀气于胃,胃者五脏之本也。脏气者,不能自致于手太阴,必因于胃气,乃至于手太阴也。故五脏各以其时,自为而至于手太阴也。故邪气胜者,精气衰也。故病甚者,胃气不能与之俱至于手太阴,故真脏之气独见。独见者病胜脏也,故曰死。\\
帝曰:善。\\
黄帝曰:凡治病,察其形气色泽,脉之盛衰,病之新故,乃治之,无后其时。形气相得,谓之可治;色泽以浮,谓之易已;脉从四时,谓之可治。脉弱以滑,是有胃气,命曰易治。取之以时。形气相失,谓之难治;色夭不泽,谓之难已;脉实以坚,谓之益甚;脉逆四时,为不可治。必察四难而明告之。\\
所谓逆四时者,春得肺脉,夏得肾脉,秋得心脉,冬得脾脉,其至皆悬绝沉涩者,命曰逆。四时未有脏形,于春夏而脉沉涩,秋冬而脉浮大,名曰逆四时也。\\
病热脉静,泄而脉大,脱血而脉实,病在中脉实坚,病在外脉不实坚者,皆难治。\\
黄帝曰:余闻虚实,以决死生,愿闻其情。\\
岐伯曰:五实死,五虚死。\\
帝曰:愿闻五实五虚。\\
岐伯曰:脉盛、皮热、腹胀、前后不通、闷瞀,此谓五实。脉细、皮寒、气少、泄利前后、饮食不入,此谓五虚。\\
帝曰:其时有生者,何也?\\
岐伯曰:浆粥入胃,泄注止,则虚者活;身汗得后利,则实者活。此其候也。\\
黄帝问道:春天的脉象如弦,那么怎样才算弦呢?\\
岐伯答说:春脉是肝脉,属东方的木,具有万物生长的气象,因此它的脉气弱软轻虚而滑,正直而长,所以叫做弦脉。与此相反,就是病脉。\\
黄帝问:什么是与此相反呢?\\
岐伯答说:脉气来时,实而且强,这叫做太过,主病在外;脉气来时不实而且微弱,这叫做不及,主病在内。\\
帝曰:春脉太过与不及,都能够发生什么病变呢?\\
岐伯回答说:太过了,会使人善忘,发生目眩冒闷头痛;如果不及,会使胸部疼痛,牵引背部,向下两胁胀满。\\
黄帝说:说得好。\\
帝曰:夏天的脉象如钩,那么怎样才算钩呢?\\
岐伯答说:夏脉就是心脉,属南方的火,具有万物盛长的气象。因此脉气来时充盛,去时反衰,犹如钩的形象,所以叫做钩脉。与此相反,是病脉。\\
黄帝说:什么是与此相反呢?\\
岐伯说:其脉气来时盛去时也盛,这叫太过,主病在外;脉气来时不盛,去时反而充盛,这叫不及,主病在内。\\
黄帝说:夏脉太过与不及,都会发生什么病变呢?\\
岐伯说:太过会使人发热、骨痛,发浸淫疮;不及会使人心烦,在上部会发生咳唾,在下部会发生失气。\\
黄帝说:说得好。\\
黄帝问:秋天的脉象如浮,那么怎样才算浮呢?\\
岐伯答说:秋脉是肺脉,属西方的金,具有万物收成的气象。因此脉气来时,轻虚而且浮,来急去散,所以叫做浮脉。与此相反,就是病脉。\\
黄帝问:什么是与此相反呢?\\
岐伯回答说:其脉气来时浮软而中央坚实,两旁虚空,这叫太过,主病在外;其脉气来时浮软而微,这叫不及,主病在里。\\
黄帝说:秋脉太过和不及,都会发生什么病变呢?\\
岐伯说:太过会使人气逆,背部作痛,郁闷而不舒畅;如果不及,会使人喘促,呼吸气短、咳嗽,在上部会发生气逆出血,在下的胸部则可以听到喘息的声音。\\
黄帝说:说得好。\\
黄帝问:冬天的脉象如营,那么怎样才算营呢?\\
岐伯说:冬脉是肾脉,属北方的水,具有万物闭藏的气象。因此脉气来时沉而濡润,所以叫做营脉。与此相反,就是病脉。\\
黄帝问:什么是与此相反呢?\\
岐伯说:其脉气来时如弹石击手,这叫太过,主病在外;如果脉象浮软,这叫不及,主病在里。\\
黄帝说:冬脉太过与不及,发生什么病变?\\
岐伯说:太过会使人身体倦怠,腹痛、气短,不愿说话;不及会使人的心像饥饿时一样感到虚悬,季胁下空软部位清冷,脊骨痛,小腹胀满,小便变色。\\
黄帝说:说得好。\\
黄帝问:四时的顺序,是导致脉象逆顺变化的根源,但是脾脉主哪个时令呢?\\
岐伯说:脾属土,是个独尊之脏,它的作用是用来滋润四旁其他的脏腑的。\\
黄帝问:那么脾的正常与否,可以看出来吗?\\
岐伯说:正常的脾脉看不出来,但病脉是可以看出来的。\\
黄帝问:那么脾的病脉是怎样的呢?\\
岐伯说:其脉来时,如水流动,这叫太过,主病在外;其脉来时,如鸟啄食,这叫不及,主病在里。\\
黄帝问:您说脾是孤脏,位居中央属土,滋润四旁之脏,那么它的太过与不及,都会发生什么病变呢?\\
岐伯说:太过会使人四肢不能举动,不及会使人九窍不通,身重而不自如。\\
黄帝惊异地站了起来,跪拜后说:好!我已懂得了诊脉的根本要领和天下的至理。考察五色和四时脉象的变化,诊察脉的正常与异常,它的精要,归结在于一个“神”字。神的功用运转不息,向前不回,倘若回而不运转,就失去了生机。这是最重要的真理,是非常切近微妙的,把它记录在玉版上,藏在脏腑里,每天早上诵读,就把它叫做《玉机》吧。\\
五脏所受的病气来源于它所生之脏,传给它所克之脏,留止在生己之脏,死于克己之脏。当病到了要死的时候,必先传到克己之脏,病人才死,这所说的就是病气逆行的情况。肝受病气于心,传行到脾,病气留止于肾,传到肺就死了。心受病气于脾,传行到肺,病气留止于肝,传到肾就死了。脾受病气于肺,传行到肾,病气留止于心,传到肝就死了。肺受病气于肾,传行到肝,病气留止于脾,传到心就死了。肾受病气于肝,传行到心,病气留止于肺,传到脾就死了。这都是病气逆行的情况,以一昼夜的时辰来归属五脏,就可推测出死亡的大体时间。\\
黄帝说:五脏是相通的,病气的转移,都有它的次序。五脏如果有病,就会传给各自所克之脏。若不及时治疗,那么多则三个月、六个月,少则三天、六天,只要传遍五脏就必死。这是指顺所克次序的传变。所以说:能够辨别外证,就可知病在何经;能够辨别里证,就可知危在何日,就是说某脏到了它受困的时候,就死了。\\
风为六淫之首,所以说它是百病之长。风寒侵入了人体,就会使人的毫毛都立起来,皮肤闭塞,内里发热,这时,可以用发汗的方法治愈;有的会出现麻痹不仁、肿痛等症状,此时可用热敷、火、灸或针刺等方法治愈。如果耽误了,病气就会传行并留止于肺部,这就是肺痹,发为咳嗽上气。如果还不治疗,就会从肺传到肝,这叫肝痹,也叫肝厥,会发生胁痛、不欲食等症状,这时,可用按摩或针刺等方法治疗。如果仍不及时治疗,病气从肝传到脾,这时的病叫做脾风,会发生黄疸、腹中热、烦心、小便黄色等症状,这时,可用按摩、药物和汤浴等方法治疗。如再不及时治疗,病气从脾传到肾,这时的病叫疝瘕,会出现小腹蓄热作痛、小便白浊等症状,又叫做蛊病,这时,可用按摩、药物等方法治疗。如继续耽误下去,病气从肾传到心,就会出现筋脉相引拘挛的症状,叫做瘛病,这时,可用艾灸、药物来治疗。如仍治不好,十天以后,就会死亡。倘病邪由肾传到心,心又反传到肺脏,又发寒热,三天就会死亡,这是疾病传递的次序。\\
但假如是猝然发病,就不必根据这个传变的次序治疗;而有的传变也不一定完全依着这个次序。忧、恐、悲、喜、怒这五种情志就会使病气不按着这个次第传变,而突然发病。如过喜伤心,克它的肾气就因而乘之;过怒伤肝,克它的肺气就因而乘之;过思伤脾,克它的肝气就因而乘之;过恐伤肾,克它的脾气就因而乘之;过忧伤肺,克它的心气就因而乘之。这就是疾病不依次序传变的规律。所以病虽有五变,但能够发为五五二十五变,这和正常的传化是相反的。传,是“乘”的别名。\\
大骨软弱无力,大肉瘦削,胸中气满,喘息困难,喘气时身体振动,死期在六个月内,见了真脏脉,就可以预知死日。\\
大骨软弱无力,大肉瘦削,胸中气满,喘息困难,胸中疼痛,牵引肩项,死期在一个月内,见了真脏脉,就可以预知死日。\\
大骨软弱无力,大肉瘦削,胸中气满,喘息困难,胸中疼痛,上引肩项,周身发热,脱肉破刐,真脏脉现,死期在十日之内。\\
大骨软弱,大肉瘦削,两肩下垂,骨髓内消,动作衰颓,真脏脉未出现,死期在一年内,若见到真脏脉,就可以预知死日。\\
大骨软弱无力,大肉瘦削,胸中气满,腹中痛,心中气郁不舒,肩项周身俱热,破刐脱肉,目眶下陷,真脏脉出现,目不见人,立即死亡;如尚能见人,是精气尚未全脱,到了它所不胜之时,就会死亡。\\
正气一时暴虚,外邪突然侵入人体,五脏隔塞,脉道不通,大气已不往来,就好像跌坠或溺水一样,这样的突然病变,是不能预测死期的。如果其脉绝而不至,或一吸五六至,形肉不脱,就是不见真脏脉,也要死亡。\\
肝脏的真脏脉来的时候,内外劲急如同循着刀刃震震作响,好像新张开的弓弦,面色显著青白而不润泽,毫毛也枯损不堪,是要死亡的。心脏的真脏脉来的时候,坚而搏指,像循摩意苡仁那样小而坚实,面色显著赤黑而不润泽,毫毛也枯损不堪,是要死亡的。肺脏的真脏脉来的时候,洪大而又非常虚弱,像毛羽触人皮肤,面色显著白赤而不润泽,毫毛也枯损不堪,是要死亡的。肾脏的真脏脉来的时候,既坚而沉,像用指弹石那样硬得很,面色显著黑黄而不润泽,毫毛也枯损不堪,是要死亡的。脾脏的真脏脉来的时候,软弱并且忽数忽散,面色显著黄青而不润泽,毫毛也枯损不堪,是要死亡的。总而言之,凡是见了真脏脉,都是不治的死证。\\
黄帝问:见了真脏脉象,就要死亡,这是什么道理呢?\\
岐伯说:五脏之气,都依赖胃腑的水谷精微来营养,所以胃是五脏的根本。五脏之气,不能直接到达手太阴的寸口,必须借助于胃气,才能到达手太阴寸口。所以五脏才能各自在一定的时候,以不同的脉象出现于手太阴寸口。如果邪气盛了,精气必然衰败,所以病气严重时,胃气就不能同脏气一起到达手太阴,那真脏脉就单独出现了。独见就是病气胜了脏气,那是要死亡的。\\
黄帝说:说得好。\\
黄帝说:治病的一般规律,是要先诊察病人的形气怎样,色泽如何,以及脉的虚实,病的新旧,然后再治疗,而千万不能错过时机。病人形气相称,是可治之证;气色浮润,病易治愈;脉象和四时相适应,是可治之证。脉来弱而流利,是有胃气的现象,属易治的病。以上都算可治、易治之证,但要及时地进行治疗才行。形气不相称,是难治之证;气色枯燥而不润泽,病不易治愈;脉实并且坚,是更加沉重的病证;如果脉象和四时不相适应,就是不可治之证了。一定要察明这四种困难,清楚地告诉病人。\\
所谓脉与四时相逆,就是春得肺脉,夏得肾脉,秋得心脉,冬得脾脉,而且脉来的时候都是独见而沉涩,这就叫逆。五脏脉气未能随四时变化显现于外,在春夏季节里,反见沉涩的脉象;在秋冬季节里,反见浮大的脉象,这都叫做逆四时。\\
病属热而脉象反见平静,发生泄利而脉象反倒洪大,出现脱血而反见实脉,病在里而脉象反倒不坚实,这些都是脉证相反的情况,不易治愈。\\
黄帝说:我听说根据虚实可以预先判断死生,希望听听这其中的道理。\\
岐伯说:凡有五实的死,凡有五虚的也得死。\\
黄帝问:那什么叫做五实五虚呢?\\
岐伯说:脉来势盛,皮肤发热,肚腹胀满,大小便不通,心里烦乱,这就叫做五实。脉象极细,皮肤发冷,气短不足,大便泄泻,不欲饮食,这就叫做五虚。\\
黄帝说:就是得了五实五虚之证,也有痊愈的,这是为什么呢?\\
岐伯说:如果病人能够吃些浆粥,胃气渐渐恢复,泄泻停止,那么得五虚之证的人就可以痊愈;而患五实之证的人如果能汗出大便又通畅了,表里和了,也可以痊愈。这就是根据虚实而决断死生的道理。\\
三部九候论篇第二十\\
黄帝问曰:余闻九针于夫子,众多博大,不可胜数。余愿闻要道,以属子孙,传之后世,著之骨髓,藏之肝肺,歃血而受,不敢妄泄,令合天道,必有终始,上应天光星辰历纪,下副四时五行。贵贱更立,冬阴夏阳,以人应之奈何?愿闻其方。\\
岐伯对曰:妙乎哉问也!此天地之至数。\\
帝曰:愿闻天地之至数,合于人形血气,通决死生,为之奈何?\\
岐伯曰:天地之至数,始于一,终于九焉。一者天,二者地,三者人,因而三之,三三者九,以应九野。故人有三部,部有三候,以决死生,以处百病,以调虚实,而除邪疾。\\
帝曰:何谓三部?\\
岐伯曰:有下部,有中部,有上部,部各有三候,三候者,有天有地有人也,必指而导之,乃以为真。故下部之天以候肝,地以候肾,人以候脾胃之气。\\
帝曰:中部之候奈何?\\
岐伯曰:亦有天,亦有地,亦有人。天以候肺,地以候胸中之气,人以候心。\\
帝曰:上部以何候之?\\
岐伯曰:亦有天,亦有地,亦有人。天以候头角之气,地以候口齿之气,人以候耳目之气。三部者,各有天,各有地,各有人。三而成天,三而成地,三而成人,三而三之,合则为九。九分为九野,九野为九脏。故神脏五,形脏四,合为九脏。五脏已败,其色必夭,夭必死矣。\\
帝曰:以候奈何?\\
岐伯曰:必先度其形之肥瘦,以调其气之虚实,实则泻之,虚则补之。必先去其血脉,而后调之,无问其病,以平为期。\\
帝曰:决死生奈何?\\
岐伯曰:形盛脉细,少气不足以息者危。形瘦脉大,胸中多气者死。形气相得者生,参伍不调者病。三部九候皆相失者死。上下左右之脉相应如参舂者病甚。上下左右相失不可数者死。中部之候虽独调,与众脏相失者死,中部之候相减者死。目内陷者死。\\
帝曰:何以知病之所在?\\
岐伯曰:察九候独小者病,独大者病,独疾者病,独迟者病,独热者病,独寒者病,独陷下者病。以左手足上,上去踝五寸按之,庶右手足当踝而弹之,其应过五寸以上蠕蠕然者,不病;其应疾,中手浑浑然者,病;中手徐徐然者,病;其应上不能至五寸,弹之不应者,死。是以脱肉身不去者,死。中部乍疏乍数者,死。其脉代而钩者,病在络脉。九候之相应也,上下若一,不得相失。一候后则病,二候后则病甚,三候后则病危。所谓后者,应不俱也。察其腑脏,以知死生之期。必先知经脉,然后知病脉。真脏脉见者,胜死。足太阳气绝者,其足不可屈伸,死必戴眼。\\
帝曰:冬阴夏阳,奈何?\\
岐伯曰:九候之脉,皆沉细悬绝者为阴,主冬,故以夜半死。盛躁喘数者为阳,主夏,故以日中死。是故寒热病者,以平旦死。热中及热病者,以日中死。病风者,以日夕死。病水者,以夜半死。其脉乍疏乍数、乍迟乍疾者,日乘四季死。形肉已脱,九候虽调,犹死。七诊虽见,九候皆从者,不死。所言不死者,风气之病,及经月之病,似七诊之病而非也,故言不死。若有七诊之病,其脉候亦败者死矣。必发哕噫。必审问其所始病,与今之所方病,而后各切循其脉,视其经络浮沉,以上下逆从循之。其脉疾者不病,其脉迟者病,脉不往来者死。皮肤著者死。\\
帝曰:其可治者奈何?\\
岐伯曰:经病者,治其经;孙络病者,治其孙络血;血病身有痛者,治其经络。其病者在奇邪,奇邪之脉则缪刺之。留瘦不移,节而刺之。上实下虚,切而从之,索其结络脉,刺出其血,以见通之。瞳子高者,太阳不足;戴眼者,太阳已绝。此决死生之要,不可不察也。\\
黄帝问说:我听了九候的道理,内容众多而广博,难以尽述。希望再听些主要的道理,以传给子孙,流传后世。我一定会把那些话铭刻在心,藏于肺腑。我发誓接受所学,不敢随便泄漏,使它合于天道,有始有终,上应日月星辰节气之数,下合四时五行之变。就五行来说有盛有衰,就四时来说冬阴夏阳,那么人怎样才能够和这些自然规律相适应呢?希望听听具体的方法。\\
岐伯说:问得好!这是天地间的至理啊!\\
黄帝说:希望听听这天地间的至理,从而使它合于人的形体,通利血气,并决定死生。怎样才能做到呢?\\
岐伯说:天地的至数,是从一开始,至九终止,一为阳,代表天,二为阴,代表地,人生天地之间,所以用三代表人。而天地人又合而为三,三三为九,与九野之数对应。所以人有三部脉,每部各有三候,根据它去决定死生,诊断百病,调和虚实,祛除疾病。\\
黄帝问道:什么叫做三部?\\
岐伯说:有下部,有中部,有上部,而每部又各有三候,三候是以天地人来代表的,必须有人指导,才能得到真传。下部的天可以用来诊察肝脏之气,下部的地可以用来诊察肾脏之气,下部的人可以用来诊察脾胃之气。\\
黄帝问:那么中部的情况怎样呢?\\
岐伯说:中部也有天地人三部。中部之天可以用来诊察肺脏之气,中部之地可以用来诊察胸中之气,中部之人可以用来诊察心脏之气。\\
黄帝问:上部的情况又怎样呢?\\
岐伯说:上部也有天地人三部。上部之天可以用来诊察头角之气,上部之地可以用来诊察口齿之气,上部之人可以用来诊察耳目之气。总之,三部之中,各有天,各有地,各有人。三候为天,三候为地,三候为人,三三相乘,合为九候。脉有九候,以应地之九野;地之九野,以应人之九脏。肝、肺、心、脾、肾五神脏,胃、大肠、小肠、膀胱四形脏,合为九脏。如果五脏败坏,气色必见晦暗,而气色晦暗必然要死亡。\\
黄帝问:诊察的方法怎样?\\
岐伯说:一定得先估量病人形体的肥瘦程度,来调和其气的虚实。气实就泻其有余,气虚就补其不足。首先要想法去掉血脉里的淤滞,然后再调和气的虚实,不管治什么病,达到五脏的平和是最终目的。\\
黄帝问:怎样决断死生呢?\\
岐伯说:形体盛,脉反细,气短,呼吸不连续,主危。形体瘦,脉反大,胸中多气胀满,也主死。形体和脉息相称的主生,脉象错杂不调的主病。三部九候都失其常度的主死。上下左右之脉相应,一上一下像舂杵一样,大数而鼓,说明病情很严重。上下左右之脉失去了协调,以至于不可计其至数的,是死候。中部的脉,虽然独自调和,而上部下部众脏之脉已失其常的,也是死候,中部的脉较上下两部偏少的,也是死候。眶内陷的,是精气衰竭的现象,也会死亡。\\
黄帝问:怎样知道疾病的部位呢?\\
岐伯说:九候之中,有一部独小,或独大,或独疾,或独迟,或独热,或独寒,或独陷下的,都会发病。用左手在病人的左足上,距离内踝五寸处触按,用右手在病人足内踝上弹之,医生感到脉中振动,其范围超过五寸以上,蠕蠕而动,为正常现象;如振动剧烈,快速而浑乱的,为病态;若振动微弱迟缓,为病态;若振动不能达到五寸,即使用力弹之,仍没有反应,为死候。所以身体消瘦至极,体弱不能行动,是死亡之证。中部之脉或快或慢,也是死征。如脉代而钩,病在络脉。九候之脉,应相互应和,上下如一,不应参差不齐。如九候之中有一候落后,就是病态;二候落后,则病重;三候落后,则病必危险。所谓落后,就是九候之间,脉动的节律不一致。诊察病邪所在之脏腑,就可以预知死生的时间。必先知道正常之脉,然后才能知道有病之脉。若见到真脏脉象,到胜己的时间,就会死。足太阳经脉气绝,则两足不能屈伸,死亡之时,必两目上视。\\
黄帝问:冬为阴,夏为阳,是什么意思?\\
岐伯说:九候的脉象,都是沉细悬绝的,为阴,好比冬令,这种病死在阴气极盛之夜半。脉象盛大躁动急数的,为阳,好比夏令,这种病死在阳气旺盛之日中。寒热往来之病,死在阴阳交会的平旦之时。热中及热病,死在日中阳极之时。患风病,死在傍晚阳衰之时。患水病,死在夜半阴极之时。其脉象忽疏忽密,忽迟忽疾,死在辰戌丑未之时,也就是平旦、日中、日夕、夜半、日乘四季的时候。若形坏肉脱,九候虽尚协调,还是死证。七诊之脉虽然出现,而九候还都顺于四时的,就不一定是死证。所说的不死,是指新感风病,或月经之病,虽然出现类似七诊之病脉,而实质不同,所以说不是死证。若七诊之脉出现,其脉候有败坏现象的,是死证。死的时候,必发呃逆。治病之时,必须详细询问开始发的病情和现在的症状,然后分别切循其脉,观察其经络的浮沉,根据上下逆顺来诊脉。其脉来流利的,不病;脉来迟缓的,有病;脉不往来的,是死证。久病肉脱,皮肤干枯着于筋骨的,也是死证。\\
黄帝问:那些可治的病,怎样治疗呢?\\
岐伯说:病在经的,刺其经;病在孙络的,刺其孙络出血;血病而身体疼痛的,则治其经与络。若病邪留在大络,则用缪刺法治之。若邪气久留不移,应斟酌刺之。上实下虚,当切循其脉,而寻找其脉络郁结所在,刺出其血,以通其气。如目上视的,是太阳经气不足。目上视而固定不动的,是太阳经气已绝。这是判断死生的要领,不可不认真研究。\\
卷七\\
经脉别论篇第二十一\\
黄帝问曰:人之居处、动静、勇怯,脉亦为之变乎?\\
岐伯对曰:凡人之惊恐恚劳动静,皆为变也。是以夜行则喘出于肾,淫气病肺。有所堕恐,喘出于肝,淫气害脾。有所惊恐,喘出于肺,淫气伤心。度水跌仆,喘出于肾与骨。当是之时,勇者气行则已,怯者则着而为病也。故曰:诊病之道,观人勇怯骨肉皮肤,能知其情,以为诊法也。\\
故饮食饱甚,汗出于胃;惊而夺精,汗出于心;持重远行,汗出于肾;疾走恐惧,汗出于肝;摇体劳苦,汗出于脾。故春秋冬夏,四时阴阳,生病起于过用,此为常也。\\
食气入胃,散精于肝,淫气于筋。食气入胃,浊气归心,淫精于脉。脉气流经,经气归于肺,肺朝百脉,输精于皮毛。脉合精,行气于腑。腑精神明,留于四脏。气归于权衡,权衡以平,气口成寸,以决死生。\\
饮入于胃,游溢精气,上输于脾;脾气散精,上归于肺,通调水道,下输膀胱。水精四布,五经并行,合于四时五脏阴阳,揆度以为常也。\\
太阳脏独至,厥喘虚气逆,是阴不足阳有余也,表里当俱泻,取之下俞。阳明脏独至,是阳气重并也,当泻阳补阴,取之下俞。少阳脏独至,是厥气也,\\
前卒大,取之下俞。少阳独至者,一阳之过也。太阴脏搏者,用心省真,五脉气少,胃气不平,三阴也,宜治其下俞,补阳泻阴。一阳独啸,少阳厥也,阳并于上,四脉争张,气归于肾,宜治其经络,泻阳补阴。一阴至,厥阴之治也,真虚\\
心,厥气留薄,发为白汗,调食和药,治在下俞。\\
帝曰:太阳脏何象?\\
岐伯曰:象三阳而浮也。\\
帝曰:少阳脏何象?\\
岐伯曰:象一阳也,一阳脏者,滑而不实也。\\
帝曰:阳明脏何象?\\
岐伯曰:象大浮也。太阴脏搏,言伏鼓也;二阴搏至,肾沉不浮也。\\
黄帝问:人的居住环境、劳逸和性情的勇怯强弱不同,其经脉血气也会随之发生变化吗?\\
岐伯回答说:大凡人在惊恐、忿怒、劳累、活动或安静的情况下,经脉血气都会因之而发生变化。所以夜晚远行,则恐惧之气出于肾脏,气逆妄行,就要伤害肺脏。或因堕坠而惊恐,则逆气出于肝脏,气逆妄行,就要伤害脾脏。或者由于惊恐,则逆气出于肺脏,气逆妄行,就要伤害心脏。或因渡水跌仆,则逆气出于肾与骨。在这种情况下,勇敢的人,气血畅行,病就自愈,怯懦的人,气血滞留,则邪气留着而为病了。所以说:诊病的方法,必须观察人的勇敢与怯懦,骨肉和皮肤,从而了解病情,这是诊断上的大法。\\
所以饮食过饱的时候,由于食气蒸发而汗出于胃;受惊而影响精神的时候,由于心气受伤而汗出于心;带着重东西远行,骨劳气越而汗出于肾;走得快并且害怕,肝气受伤而汗出于肝;肢体摇动劳累过度的时候,脾气受伤而汗出于脾。所以春秋冬夏四时阴阳变化之中,生病的原因,多是由于体力、饮食、劳累、精神等过度而来,这是一定的。\\
食物入胃,经过消化把一部分精微输散到肝脏,经过肝的疏泄,将浸淫满溢的精气滋养于筋。食物入胃,化生的另一部分浓厚的精气,注入于心,再由心输入血脉。血气流行在经脉之中,上达于肺,肺又将血气送到全身百脉,直至皮毛。脉与精气相合,运行精气到六腑。六腑的精气化生神明,输入留于四脏。这些正常的生理活动,取决于阴阳气血平衡,其平衡的变化,就能从气口的脉象上表现出来,气口脉象变化,可以判断疾病的预后。\\
水液进入胃里,分离出精气,上行输送到脾脏;脾脏散布精华,又向上输送到肺;肺气通调水道,又下行输入膀胱。这样,气化水行,散布于周身皮毛,流行在五脏经脉里,符合于四时五脏阴阳动静的变化,这是可以测度的经脉的正常现象。\\
太阳经脉偏盛,就要发生厥逆、喘息、虚气上逆等症状,这是阴不足、阳有余的缘故,治疗表里都用泻法,取足太阳经的束骨穴和足少阴经的太溪穴。阳明经脉偏盛,是太阳、少阳之气俱趋于阳明,当泻足阳明经的陷谷穴,补足太阴经的太白穴。少阳经脉偏盛,就要发生厥气上逆,所以阳刉脉前的少阳脉,猝然而大,当取足少阳本经的临泣穴。少阳经脉的偏盛,就是少阳的太过。太阴经脉鼓搏有力,应该细心省察真脏脉,若非真脏外泄,是五脏脉气减少,胃气不能平和,这是太阴太过的缘故,应补足阳明之陷谷穴,泻足太阴之太白穴,用补阳泻阴法。二阴经脉的偏盛,是为少阴热厥,虚阳并越于上部,心、肺、肝、脾四脉都受到影响,其病根源于肾脏,应该治其经络,泻足太阳的经穴昆仑、络穴飞扬,补足少阴的经穴复溜、络穴大钟。一阴经脉的偏盛,是厥阴经脉所主,真气虚弱,心中酸疼不适,厥气留于经脉与正气相搏而大汗出,应该注意饮食调节和药物治疗,并针刺厥阴的太冲穴。\\
黄帝问:太阳经脉的脉象怎样?\\
岐伯说:太阳经脉像三阳经脉那样极盛,同时它还轻浮。\\
黄帝问:少阳经脉的脉象怎样?\\
岐伯说:少阳经脉与一阳经脉一样,脉象是滑而不实的。\\
黄帝问:阳明经脉之象怎样?\\
岐伯说:脉象大而且浮。太阴经脉搏动,其脉象沉伏而实鼓指;二阴经脉搏动,是肾脉沉而不浮的现象。\\
脏气法时论篇第二十二\\
黄帝问曰:合人形以法四时五行而治,何如而从?何如而逆?得失之意,愿闻其事。\\
岐伯对曰:五行者,金木水火土也,更贵更贱,以知死生,以决成败,而定五脏之气,间甚之时,死生之期也。\\
帝曰:愿卒闻之。\\
岐伯曰:肝主春,足厥阴少阳主治,其日甲乙;肝苦急,急食甘以缓之。心主夏,手少阴太阳主治,其日丙丁;心苦缓,急食酸以收之。脾主长夏,足太阴阳明主治,其日戊己;脾苦湿,急食苦以燥之。肺主秋,手太阴阳明主治,其日庚辛,肺苦气上逆,急食苦以泄之。肾主冬,足少阴太阳主治,其日壬癸,肾苦燥,急食辛以润之。开腠理,致津液,通气也。\\
病在肝,愈于夏;夏不愈,甚于秋;秋不死,持于冬;起于春,禁当风。肝病者,愈在丙丁;丙丁不愈,加于庚辛;庚辛不死,持于壬癸,起于甲乙。肝病者,平旦慧,下晡甚,夜半静。肝欲散,急食辛以散之,用辛补之,酸泻之。\\
病在心,愈在长夏;长夏不愈,甚于冬;冬不死,持于春,起于夏;禁温食热衣。心病者,愈在戊己;戊己不愈,加于壬癸;壬癸不死,持于甲乙,起于丙丁。心病者,日中慧,夜半甚,平旦静。心欲耎,急食咸以软之,用咸补之,甘泻之。\\
病在脾,愈在秋;秋不愈,甚于春;春不死,持于夏,起于长夏;禁温食饱食,湿地濡衣。脾病者,愈在庚辛;庚辛不愈,加于甲乙;甲乙不死,持于丙丁,起于戊己。脾病者,日昳慧,日出甚,下晡静。脾欲缓,急食甘以缓之,用苦泻之,甘补之。\\
病在肺,愈在冬;冬不愈,甚于夏;夏不死,持于长夏,起于秋;禁寒饮食、寒衣。肺病者,愈在壬癸;壬癸不愈,加于丙丁;丙丁不死,持于戊己,起于庚辛。肺病者,下晡慧,日中甚,夜半静。肺欲收,急食酸以收之,用酸补之,辛泻之。\\
病在肾,愈在春;春不愈,甚于长夏;长夏不死,持于秋,起于冬;禁犯焠劦热食温炙衣。肾病者,愈在甲乙;甲乙不愈,甚于戊己;戊己不死,持于庚辛,起于壬癸。肾病者,夜半慧,四季甚,下晡静。肾欲坚,急食苦以坚之,用苦补之,咸泻之。\\
夫邪气之客于身也,以胜相加,至其所生而愈,至其所不胜而甚,至于所生而持,自得其位而起。必先定五脏之脉,乃可言间甚之时,死生之期也。\\
肝病者,两胁下痛引少腹,令人善怒;虚则目丼丼无所见,耳无所闻,善恐,如人将捕之。取其经,厥阴与少阳。气逆,则头痛,耳聋不聪,颊肿,取血者。\\
心病者,胸中痛,胁支满,胁下痛,膺背肩甲间痛,两臂内痛;虚则胸腹大,胁下与腰相引而痛。取其经,少阴太阳,舌下血者。其变病,刺郄中血者。\\
脾病者,身重善肌,肉痿,足不收,行善瘈,脚下痛;虚则腹满肠鸣,飧泄食不化。取其经,太阴阳明,少阴血者。\\
肺病者,喘咳逆气,肩背痛,汗出,尻、阴股、膝、髀、腨、胻、足皆痛;虚则少气,不能报息,耳聋嗌干。取其经,太阴足太阳之外,厥阴内,血者。\\
肾病者,腹大胫肿,喘咳身重,寝汗出,憎风;虚则胸中痛,大腹小腹痛,清厥,意不乐。取其经,少阴太阳血者。\\
肝色青,宜食甘,粳米、牛肉、枣、葵,皆甘。心色赤,宜食酸,小豆、犬肉、李、韭,皆酸。肺色白,宜食苦,麦、羊肉、杏、薤,皆苦。脾色黄,宜食咸,大豆、豕肉、栗、藿,皆咸。肾色黑,宜食辛,黄黍、鸡肉、桃、葱,皆辛。辛散,酸收,甘缓,苦坚,咸软。\\
毒药攻邪,五谷为养,五果为助,五畜为益,五菜为充,气味合而服之,以补精益气。此五者,有辛酸甘苦咸,各有所利,或散或收,或缓或急,或坚或软,四时五脏病,随五味所宜也。\\
黄帝问:结合人的形体,取法四时五行的规律而进行治疗,怎样是从?怎样是逆?逆从得失的意义,希望听听是怎么一回事?\\
岐伯回答说:五行,就是金、木、水、火、土,与时令气候配合,有衰旺的变化,由此而判断病人的生死,分析医事的成败,从而确定五脏之气的盛衰,疾病缓解和加重的时间,以及死生的日期。\\
黄帝说:希望详尽地听听。\\
岐伯说:肝主春木之气,春天是足厥阴和足少阳主治的时间,肝胆旺日为甲乙;肝苦拘急,急宜食甜味药以缓和。心主夏火之气,夏天为手少阴和手太阳主治的时间,心与小肠旺日为丙丁;心苦弛缓,急宜用酸味药以收敛。脾主长夏土气,长夏为足太阴和足阳明主治的时间,脾与胃旺日为戊己;脾苦湿,急宜用苦味药以燥湿。肺主秋金之气,秋天为手太阴和手阳明主治的时间,肺与大肠旺日为庚辛;肺苦于气上逆,急宜用苦泄之药以宣泄其气。肾主冬水之气,冬天为足少阴和足太阳主治的时间,肾与膀胱旺日为壬癸;肾苦干燥,急宜用辛润之药以润燥。这样可以开发腠理,运行津液,而通畅五脏之气。\\
病在肝脏,在夏天痊愈;若至夏天不愈,到秋天病情就要加重;如秋天不死,到冬天病情相对稳定;到明年春天才能好转,禁止吹风。肝病患者,痊愈当在丙丁日;丙丁日不好,到庚辛日就要加重;庚辛日不死,到壬癸日病情稳定,到甲乙日才能好转。肝病患者,每天清晨神志比较清爽,傍晚时分,病情比较重,半夜时便安静了。肝喜条达疏散,宜急用辛味药来发散,以辛味补之,酸味泻之。\\
病在心脏,在长夏痊愈;若至长夏不愈,到了冬季病情就要加重;如冬季不死,到明年春天病情相对稳定,到了夏天才能好转;禁忌温热食物,衣服不能穿得太暖。心病患者,病愈当在戊己日;戊己日不好,到壬癸日就要加重;壬癸日不死,到甲乙日病情稳定,到丙丁日才能好转。心病患者,每天中午神智比较清爽,到半夜时病情加重,到了天亮时又安静了。心脏病需要软,宜急用咸味药来软减,以咸味补之,以甘味泻之。\\
病在脾脏,在秋天痊愈;若至秋天不愈,到了春天病情就要加重;如春季不死,至夏季病情相对稳定,到了长夏才能好转;应禁食温热性食物及吃得过饱,禁居湿地、禁穿湿衣。脾病患者,其病愈在庚辛日;庚辛日不好,到甲乙日就要加重;甲乙日不死,至丙丁日病情就相对稳定,到了戊己日才能好转。脾病患者,每天午后神志比较清爽,到了日出时,病情就会加重,到了傍晚时,便安静了。脾病缓和,宜急食甘味以缓之,用苦泻之,用甘味补之。\\
病在肺脏,在冬天痊愈;若至冬天不愈,到了夏天病情就要加重;如夏天不死,到长夏时病情会相对稳定,到了秋天才能好转;禁止寒凉饮食及衣服穿得太少。肺病患者,其病愈在壬癸日;壬癸日不好,到丙丁日就要加重;如果丙丁日不死,到戊己日病情就会相对稳定,到了庚辛日才能好转。肺病患者,在每天傍晚时神志比较清爽,到了中午时,病情会加重,午后便安静了。肺脏病需要收敛,宜急食酸味药以收之,用酸味补之,辛味泻之。\\
病在肾脏,在春天痊愈;若至春天不愈,到了长夏病情就要加重;若长夏不死,至秋天病情相对稳定,到了冬季才能好转;禁食过热的食物和穿烘烤过的衣服。肾病患者,其病愈当在甲乙日;甲乙日不好,在戊己日就要加重;戊己日不死,到庚辛日就会相对稳定,到了壬癸日才能好转。肾病患者,在半夜时神志比较清爽,当辰、戌、丑、未四个时辰病情加重,在傍晚时便安静了。肾需要坚,宜急食苦味药以坚之,用苦味补之,咸味泻之。\\
大凡邪气侵袭于人身,都是以强凌弱,病至其所生之时而愈,至其所不胜之时而甚,至于生己之时而持,自得当旺之时而起。但必须先确定五脏的平脉,才可以推论病证轻重的时间,以及预决死生的日期。\\
肝病可见,两胁下疼痛,牵引少腹,使人易怒;如果肝虚,则两眼昏花,视物不清,两耳听不清声音,易恐惧,好像有人来抓他。治疗取刺厥阴和少阳两经穴位。肝气上逆,则有头痛,耳聋不聪,颊肿,在其经血盛处放血。\\
心病可见,胸中疼痛,胁部支满,胁下疼痛,膺背肩胛间痛,两臂内侧疼痛;如果心虚,则见胸腹胀大,胁下和腰部牵引作痛。治疗取刺少阴和太阳两经穴位,并在舌下廉泉穴刺出血。如果疾病有变化,则刺阴郄穴出血。\\
脾病可见,身体沉重,容易饥饿,肌肉痿软,足不能举步,或筋脉牵掣,脚下疼痛;如脾虚,则见腹满肠鸣,泄泻完谷不化。治疗取刺太阴、阳明、少阴经穴,刺出其血。\\
肺病可见,咳喘气逆,肩背疼痛,出汗,尻、大腿内侧、膝、髋、小腿肚、小腿下半部、脚等处都疼痛;如果肺虚,可见短气,呼吸不连续,耳聋,咽喉干燥。治疗取刺太阴、足太阳经脉的外侧,厥阴经脉的内侧,刺出其血。\\
肾病可见,腹大胫肿,喘咳,身体沉重,盗汗出,恶风;如果肾虚,可见胸中疼痛,大腹、小腹疼痛,四肢厥冷,心中不乐。治疗取刺少阴和太阳经穴,刺出其血。\\
肝脏主青色,肝病宜食甜味,粳米、牛肉、大枣、葵菜都是甜味。心脏主赤色,心病宜食酸味,小豆、犬肉、李子、韭菜都是酸味。肺脏主白色,肺病宜食苦味,麦、羊肉、杏、薤都是苦味。脾脏主黄色,脾病宜食咸味,大豆、猪肉、栗子、藿都是咸味。肾脏主黑色,肾病宜食辛味,黄黍、鸡肉、桃、大葱都是辛味。所有食物,辛能发散,酸能收敛,甘能缓急,苦能坚燥,咸能软坚。\\
凡药物用来攻邪,五谷用来营养,五果作为辅助,五畜用来补益,五菜用来充养,气味配合调和而服食,用来补益精气。这五类东西,各有辛、酸、甘、苦、咸的味道,对某一脏之气各有利,或散、或收,或缓、或急,或坚、或软等作用,配合四时五脏,治病要根据五味所宜。\\
宣明五气篇第二十三\\
五味所入:酸入肝,辛入肺,苦入心,咸入肾,甘入脾,是谓五入。\\
五气所病:心为噫,肺为咳,肝为语,脾为吞,肾为欠、为嚏。胃为气逆、为哕、为恐,大肠、小肠为泄,下焦溢为水,膀胱不利为癃、不约为遗溺,胆为怒。是谓五病。\\
五精所并:精气并于心则喜;并于肺则悲;并于肝则忧;并于脾则畏;并于肾则恐。是谓五并,虚而相并者也。\\
五脏所恶:心恶热,肺恶寒,肝恶风,脾恶湿,肾恶燥。是谓五恶。\\
五脏化液:心主汗,肺主涕,肝主泪,脾主涎,肾主唾。是谓五液。\\
五味所禁:辛走气,气病,无多食辛;咸走血,血病,无多食咸;苦走骨,骨病,无多食苦;甘走肉,肉病,无多食甘;酸走筋,筋病,无多食酸。是谓五禁,无令多食。\\
五病所发:阴病发于骨,阳病发于血,阴病发于肉,阳病发于冬,阴病发于夏。是谓五发。\\
五邪所乱:邪入于阳则狂,邪入于阴则痹,搏阳则为巅疾,搏阴则为瘖,阳入之阴则静,阴出之阳则怒。是谓五乱。\\
五邪所见:春得秋脉,夏得冬脉,长夏得春脉,秋得夏脉,冬得长夏脉,名曰阴出之阳,病善怒,不治。是谓五邪,皆同命,死不治。\\
五脏所藏:心藏神,肺藏魄,肝藏魂,脾藏意,肾藏志。是谓五脏所藏。\\
五脏所主:心主脉,肺主皮,肝主筋,脾主肉,肾主骨。是谓五主。\\
五劳所伤:久视伤血,久卧伤气,久坐伤肉,久立伤骨,久行伤筋。是谓五劳所伤。\\
五脉应象:肝脉弦,心脉钩,脾脉代,肺脉毛,肾脉石。是谓五脏之脉。\\
五味各有所入:酸味入肝,辛味入肺,苦味入心,咸味入肾,甘味入脾。这叫五味所入。\\
五脏之气发病:心为噫气,肺为咳嗽,肝为多语,脾为吞酸,肾为呵欠、喷嚏。六腑之气失调,胃为哕逆、恐惧,大肠、小肠为泄泻,下焦泛滥,为水肿,膀胱不通为癃闭、不能约束为遗尿,胆为发怒。这就是五病。\\
五脏精气相并之证:精气并于心而生喜笑;并于肺则生悲哀;并于肝则生忧虑;并于脾则胆怯生畏;并于肾则心悸善恐。这是五脏精气相并之证,因虚而气乱相并。\\
五脏各有所厌恶:心厌恶热,肺厌恶寒,肝厌恶风,脾厌恶湿,肾厌恶燥。这是五脏所恶。\\
五脏化生五液:心主汗液,肺主涕液,肝主泪液,脾主涎液,肾主唾液。这是五脏主五液。\\
疾病所禁食的五味:辛味走气分,气病,不能多食辛味;咸味走血分,血病,不能多食咸味;苦味走骨骼,骨病,不能多食苦味;甘味走肌肉,肉病,不能多食甘味;酸味走筋膜,筋病,不能多食酸味。这就是疾病的五禁,要自我节制,不能多食。\\
五脏发病的部位和季节各不相同:肾为阴脏而主骨,发病多在骨骼;心为阳脏而主血脉,发病多在血脉;饮食五味伤脾,发病多为肌肉痿弱不用;阳虚而病,多发于冬季;阴虚而病,往往发于夏季。这叫五发。\\
五脏为邪所扰的病变:病邪入阳分,则为狂;病邪入阴分,血脉凝涩,发生痹证;病邪入于阳,邪气搏结于上,发生头部疾患;五脏阴经通于喉舌之间,病邪入于阴,搏结不去,伤阴而瘖哑;病邪由阳入阴,病多平静;病邪由阴出阳,病多怒。这叫五乱。\\
五邪所见的脉象是:春天见秋天的毛脉,夏天见冬天的石脉,长夏见春天的弦脉,秋天见夏天的钩脉,冬天见长夏的濡脉。这就是五邪脉,预后相同,都是死证。\\
五脏所藏精神活动:心脏藏神,肺脏藏魄,肝脏藏魂,脾脏藏意,肾脏藏精。这就是五脏所藏。\\
五脏各有所主:心主宰血脉,肺主宰皮毛,肝主宰筋膜,脾主宰肌肉,肾主宰骨胳。这就叫五主。\\
五种劳逸过度所致的损伤:久视伤心血,久卧伤肺气,久坐伤肌肉,久立则伤骨,久行则伤筋。这是五种久劳所伤。\\
五脉与外界事物相应的物象:肝脉如弓弦,心脉如带钩,脾脉如代止,肺脉如秋毛,肾脉如沉石。这就是五脏的脉象。\\
血气形志篇第二十四\\
夫人之常数:太阳常多血少气,少阳常少血多气,阳明常多气多血,少阴常少血多气,厥阴常多血少气,太阴常多气少血。此天之常数。\\
足太阳与少阴为表里,少阳与厥阴为表里,阳明与太阴为表里,是为足阴阳也。手太阳与少阴为表里,少阳与心主为表里,阳明与太阴为表里,是为手之阴阳也。今知手足阴阳所苦。凡治病必先去其血,乃去其所苦,伺之所欲,然后泻有余,补不足。\\
欲知背俞,先度其两乳间,中折之,更以他草度去半已,即以两隅相拄也。乃举以度其背,令其一隅居上,齐脊大椎,两隅在下,当其下隅者,肺之俞也。复下一度,心之俞也。复下一度,左角肝之俞也。右角脾之俞也。复下一度,肾之俞也。是谓五脏之俞,灸刺之度也。\\
形乐志苦,病生于脉,治之以灸刺。形乐志乐,病生于肉,治之以针石。形苦志乐,病生于筋,治之以熨引。形苦志苦,病生于咽嗌,治之以百药。形数惊恐,经络不通,病生于不仁,治之以按摩醪药。是谓五形志也。\\
刺阳明出血气,刺太阳出血恶气,刺少阳出气恶血,刺太阴出气恶血,刺少阴出气恶血,刺厥阴出血恶气也。\\
人身气血多少有一定之数:太阳经常多血少气,少阳经常少血多气,阳明经常多气多血,少阴经常少血多气,厥阴经常多血少气,太阴经常多气少血。这是先天禀赋的一定之数。\\
足太阳膀胱经和足少阴肾经为表里,足少阳胆经和足厥阴肝经为表里,足阳明胃经和足太阴脾经为表里,这是足三阳经和足三阴经之间的关系。手太阳小肠经和手少阴心经为表里,手少阳三焦经和手厥阴心包经为表里,手阳明大肠经和手太阴肺经为表里,这是手三阳经和手三阴经之间的关系。这样就能够知道手足阴阳十二经脉的病苦。大凡治病,血脉充盛的,必须先刺出其血,以除去痛苦,观察病人的意愿,根据病情的虚实,泻其有余,补其不足。\\
要确定背部五腧穴的部位,先用一根草度量病人两乳间的距离,然后从正中对折,再用另一根同样长的草,量到对折后草的正中,即四分之一处;折掉这四分之一,然后使草两端相支撑,成为三角形。叫病人举起臂来,用它量病人的背部,使一只角朝上,和脊背大椎穴相平,两只角在下,在下左右两角所指的部位,就是肺俞穴。再把上角下移一度到两肺俞穴的中点,左右两角是心俞。再移下一度,左角是肝俞,右角是脾俞。再移下一度,左右两角是肾俞。这就是五腧穴的部位,也就是针灸取穴的法度。\\
形体安乐,精神苦闷,发病易在经脉,用针刺治疗。形体安乐,精神愉快,发病易在肌肉,用针刺和砭石治疗。形体劳苦,精神愉快,发病易在筋骨,用熨引治疗。形体劳苦,精神苦闷,发病易在咽嗌,用甘药治疗。形体屡次遭受惊恐,筋脉运行不畅,发病易出现肢体不仁的症状,用按摩、药酒治疗。这就是所谓五种形志病。\\
刺阳明经,可以出血出气;刺太阳经,只可出血,不宜伤气;刺少阳经,只可出气,不宜出血;刺太阴经,只可出气,不宜出血;刺少阴经,只可出气,不宜出血;刺厥阴经,只可出血,不宜伤气。\\
卷八\\
宝命全形论篇第二十五\\
黄帝问曰:天覆地载,万物悉备,莫贵于人。人以天地之气生,四时之法成。君王众庶,尽欲全形,形之疾病,莫知其情,留淫日深,著于骨髓。心私虑之,余欲针除其疾病,为之奈何?\\
岐伯对曰:夫盐之味咸者,其气令器津泄;弦绝者,其音嘶败;木敷者,其叶发;病深者,其声哕。人有此三者,是谓坏腑,毒药无治,短针无取,此皆绝皮伤肉,血气争矣。\\
帝曰:余念其痛,心为之乱惑,反甚其病,不可更代。百姓闻之,以为残贼,为之奈何?\\
岐伯曰:夫人生于地,悬命于天,天地合气,命之曰人。人能应四时者,天地为之父母;知万物者,谓之天子。天有阴阳,人有十二节;天有寒暑,人有虚实。能经天地阴阳之化者,不失四时;知十二节之理者,圣智不能欺也;能存八动之变,五胜更立;能达虚实之数者,独出独入,呿吟至微,秋毫在目。\\
帝曰:人生有形,不离阴阳;天地合气,别为九野,分为四时。月有大小,日有短长,万物并至,不可胜量,虚实呿吟,敢问其方?\\
岐伯曰:木得金而伐,火得水而灭,土得木而达,金得火而缺,水得土而绝。万物尽然,不可胜竭。故针有悬布天下者五,黔首共余食,莫知之也。一曰治神,二曰知养身,三曰知毒药为真,四曰制砭石小大,五曰知腑脏血气之诊。五法俱立,各有所先。今末世之刺也,虚者实之,满者泄之,此皆众工所共知也。若夫法天则地,随应而动,和之者若响,随之者若影。道无鬼神,独来独往。\\
帝曰:愿闻其道。\\
岐伯曰:凡刺之真,必先治神,五脏已定,九候已备,后乃存针。众脉不见,众凶弗闻。外内相得,无以形先,可玩往来,乃施于人。人有虚实,五虚勿近,五实勿远,至其当发,间不容瞚。手动若务,针耀而匀。静意视息,观适之变,是谓冥冥,莫知其形,见其乌乌,见其稷稷,徒见其飞,不知其谁,伏如横弩,起如发机。\\
帝曰:何如而虚?何如而实?\\
岐伯曰:刺虚者须其实,刺实者须其虚。经气已至,慎守勿失。深浅在志,远近若一。如临深渊,手如握虎,神无营于众物。\\
黄帝问道:天地之间,万物俱全,但没有什么比人更为宝贵的。人禀受天地之气而生存,随着四时规律成长的。无论是君王,还是平民,都愿意保持形体的健康,但往往身体有了疾病,自己也不知其所以然,因此病邪就积累日深,潜藏骨髓之内,不易去掉了。这是我心中所担忧的,我想用针刺来解除他们的疾病痛苦,怎样办呢?\\
岐伯回答说:诊断疾病,应该注意观察它所表现的症候:比如盐贮藏在器具中,能够使器具渗出水来;琴弦快断的时候,会发出嘶破的声音;树木弊坏,叶子就要落下来;疾病到了严重阶段,人就要打嗝。人有了这样四种现象,说明脏腑已有严重破坏,药物和针刺都不起作用,这都是皮肉血气各不相得,病不容易治了。\\
黄帝道:我很感伤病人的痛苦,心里惶惑不安,治疗疾病,搞不好,反使病情加重,我又不能替代他们。百姓听了,都会认为我是残忍的人,怎么办好呢?\\
岐伯说:人虽然是生活在地上,但片刻也离不开天,天地之气相合,才产生了人。人如果能适应四时的变化,那么自然界的一切,都会成为他生命的泉源;如果能够了解万物的话,那就是天子了。人与自然是相应的,天有阴阳,人有十二骨节;天有寒暑,人有虚实。所以能效法天地阴阳的变化,就不会违背四时的规律;了解十二骨节的道理,就是所谓圣智也不能超过他;能够观察八风的变动和五行的衰旺,又能够通达虚实的变化规律,就能洞晓病情,即使像病人呼吸那样的细微不易察觉的变化,也如秋毫在目,逃不过他的眼睛。\\
黄帝道:人生而有形体,离不开阴阳;天地之气相合以后,生成了世界上的万物,从地理上,可以分为九野;从气候上,可以分为四时。月份有大有小,白天有短有长,万物同时来到世界,实在是度量不尽的,我只希望解除病人的痛苦,请问应该用什么方法呢?\\
岐伯说:治疗的方法,可根据五行变化的道理分析。如木遇到金,就被折断;火遇到水,就会熄灭;土遇到木,就要松软;金遇到火,就要熔化;水遇到土,就要遏绝。这种种变化,万物都是这样,不胜枚举。所以有五种针法已向天下公布了,但人们只知饱食,而不去了解它们。那五种治法是什么呢?第一要精神专一,第二要修养形体,第三要了解药物的真假性能,第四要制定大小砭石以适应不同的疾病,第五要懂得脏腑血气的诊断方法。这五种治法,各有所长,先用哪个,要视具体情况而定。现在针刺的疗法,用补治虚,用泻治实,而这是普通医生所共知的。至于能够取法天地阴阳的道理,随其变化而施针法,就能取得如响应声,如影随形的疗效。这并没有什么神秘,只要功力积久,就有这样的高超技术。\\
黄帝道:我希望听一下其中的道理。\\
岐伯说:针刺的正法,要先集中精神,待五脏虚实已定,脉象九候已备知,然后再下针。在针刺的时候,必须精神贯注,即使有人旁观,也像看不见一样,有人喧嚣,也像听不到一样。同时还要色脉相参,不能仅看外形,必须将发病的机理揣摩清楚,才能给人治病。病人有虚有实,见到五虚的症状,不能随意去泻;见到五实的症状,也不可远而不泻,在应该进针时,就是一瞬间也不能耽搁。在手捻针时,什么事也不想,针要光净匀称。针者要平心静气,观察病人的呼吸。那血气的变化无形无象,虽不可见,而气至之时,好像群乌一样集合,气盛之时,好像稷一样繁茂。气之往来,正如见鸟之飞翔,而无从捉摸它形迹的起落。所以用针之法,当气未至的时候,应该留针候气,正如横弩之待发,气应的时候,则当迅速起针,正如弩箭之疾出。\\
黄帝道:怎样刺虚?又怎样刺实?\\
岐伯说:刺虚证,须用补法;刺实证,须用泻法。经气已经到了,应慎重掌握,不失时机。无论针刺深浅,无论取穴远近,得气是一样的。在捻针的时候,像面临深渊时那样的谨慎;又像手中捉着老虎那样坚定有力,集中神志,不为其他事物所干扰。\\
八正神明论篇第二十六\\
黄帝问曰:用针之服,必有法则焉,今何法何则?\\
岐伯对曰:法天则地,合以天光。\\
帝曰:愿卒闻之。\\
岐伯曰:凡刺之法,必候日月星辰,四时八正之气,气定乃刺之。是故天温日明,则人血淖液而卫气浮;天寒日阴,则人血凝泣而卫气沉。月始生,则血气始精,卫气始行;月郭满,则血气实,肌肉坚;月郭空,则肌肉减,经络虚,卫气去,形独居,是以因天时而调血气也。是以天寒无刺,天温无疑;月生无泻,月满无补;月郭空无治。是谓得时而调之。因天之序,盛虚之时,移光定位,正立而待之。故曰月生而泻,是谓重虚;月满而补,血气盈溢,络有留血,命曰重实;月郭空而治,是谓乱经。阴阳相错,真邪不别,沉以留止,外虚内乱,淫邪乃起。\\
帝曰:星辰八正四时何候?\\
岐伯曰:星辰者,所以制日月之行也。八正者,所以候八风之虚邪,以时至者也;四时者,所以分春秋冬夏之气所在,以时调之也。八正之虚邪,而遇之勿犯也。以身之虚,而逢天之虚,两虚相感,其气至骨,入则伤五脏,工候救之,弗能伤也。故曰:天忌不可不知也。\\
帝曰:善。其法星辰者,余闻之矣,愿闻法往古者。\\
岐伯曰:法往古者,先知《针经》也。验于来今者,先知日之寒温,月之虚盛,以候气之浮沉,而调之于身,观其立有验也。观于冥冥者,言形气荣卫之不形于外,而工独知之。以日之寒温,月之虚盛,四时气之浮沉,参伍相合而调之。工常先见之,然而不形于外,故曰观于冥冥焉。通于无穷者,可以传于后世也,是故工之所以异也。然而不形见于外,故俱不能见也。视之无形,尝之无味,故谓冥冥,若神仿佛。虚邪者,八正之虚邪气也。正邪者,身形若用力,汗出,腠理开,逢虚风,其中人也微,故莫知其情,莫见其形。上工救其萌芽,必先见三部九候之气,尽调不败而救之,故曰上工。下工救其已成,救其已败。救其已成者,言不知三部九候之相失,因病而败之也。知其所在者,知诊三部九候之病脉处而治之。故曰守其门户焉,莫知其情而见邪形也。\\
帝曰:余闻补泻,未得其意。\\
岐伯曰:泻必用方。方者,以气方盛也,以月方满也,以日方温也,以身方定也。以息方吸而内针,乃复候其方吸而转针,乃复候其方呼而徐引针。故曰泻必用方,其气乃行焉。补必用员。员者行也,行者移也,刺必中其荣,复以吸排针也。故员与方,排针也。故养神者,必知形之肥瘦,荣卫血气之盛衰。血气者,人之神,不可不谨养。\\
帝曰:妙乎哉论也!合人形于阴阳四时,虚实之应,冥冥之期,其非夫子孰能通之?然夫子数言形与神,何谓形?何谓神?愿卒闻之。\\
岐伯曰:请言形,形乎形,目冥冥。问其所病,索之于经,慧然在前。按之不得,不知其情,故曰形。\\
帝曰:何谓神?\\
岐伯曰:请言神。神乎神,耳不闻,目明心开而志先,慧然独悟,口弗能言。俱视独见,适若昏,昭然独明,若风吹云,故曰神。三部九候为之原,九针之论不必存也。\\
黄帝问道:用针的技术,必然有一定法则,那么究竟取法于什么呢?\\
岐伯回答说:要取法于天地阴阳,并结合日月星辰之光来研究。\\
黄帝道:希望详细听听。\\
岐伯说:大凡针刺之法,必须察验日月星辰四时八正之气,气定了,才能进行针刺。如果气候温和,日光明亮,那么人体血液就濡润而卫气上浮;如果气候寒冷,日光晦暗,那么人体血液就滞涩而卫气沉伏。月亮初生的时候,人的血气随月新生,卫气亦随之畅行;月亮正圆的时候,人的血气强盛,肌肉坚实;月黑无光的时候,人的肌肉消瘦,经络空虚,卫气不足,形体独居,所以要顺着天气而调和血气。因此说,气候寒冷,不要行针刺;气候温暖,不要迟疑;月初生的时候,不要用泻法;月正圆的时候,不要用补法;月黑无光的时候,不要进行治疗。这叫顺应天时而调养血气。按照天时推移的次序,结合人身血气的盛衰,来确定气的所在,并聚精会神地等待治疗的最好时机。所以说,月初生时用泻法,这叫做重虚;月正圆时用补法,使血气充溢,经脉中血液留滞,这叫做重实;月黑无光的时候而用针刺,就会扰乱经气,这叫做乱经。这些都是阴阳相错,正气邪气分不清楚,邪气沉伏留而不去,致使络脉外虚,经脉内乱,所以病邪就乘之而起。\\
黄帝问:星辰、八正、四时怎么候察呢?\\
岐伯说:星辰的方位,可以用来测定日月循行的规律。八节常气的交替,可以用来测出八风病邪什么时候到来;四时,可以用来分别春秋冬夏之气的所在;按照时序来调整气血,避免八正病邪的侵犯。假如身体虚弱,又遭遇自然界的虚邪,两虚相感,邪气就会侵犯至骨,进而深入五脏。医生能候察气候变化的道理而及时挽救,病邪就不能伤人。所以说:天时的宜忌,不可不了解。\\
黄帝道:说得好。取法星辰的道理,我已经听到了,希望再听听效法往古的道理。\\
岐伯说:效法往古,要先懂得《针经》。想把前人的针术在现在加以验证,先要知道太阳的寒温,月亮的盈虚,来候察气的浮沉,来给病人进行调整,就会看到它是立有效验的。所谓“观于冥冥”,是说血气荣卫的变化并不显露于外,而医生却能懂得。这就是把太阳的寒温,月亮的盈虚,四时气候的浮沉等情况,综合起来候察以调整病人。这样,医生就常能预见病情,然而疾病尚未显露于外,所以叫“观于冥冥”。所谓“通于无穷”,是说医生的高超技术可以流传后世,这就是医生与一般人不同的地方。不过是病情还没有显露出来,大家都不能发现罢了。看不见形象,尝不到味道,所以叫做“冥冥”,就像神灵一样若隐若现,难以捉摸。虚邪,就是四时八节的病邪。正邪,就是身体因劳累出汗,腠理开张,而为虚风侵袭,正邪伤人轻微,所以一般人不了解它的病情,看不到它的病象。高明的医生,在疾病刚开始就救治,先去候查三部九候的脉气,及时调治,不使脉气衰败,所以疾病容易痊愈,所以叫高明的医生。而低劣的医生,却等疾病已形成,或疾病已经败坏时才治疗。等到病已形成后才治疗,就是不懂得三部九候的脉气混乱是由疾病发展所导致的。他所谓知道疾病的所在,只不过是知道三部九候病脉的所在部位罢了。所以这就像把守门户一样,已经陷入了被动地位。其原因就是不了解病理,而只看到病症的表面现象。\\
黄帝道:我听说针法有补有泻,但不懂它的涵义。\\
岐伯说:泻法必须掌握一个“方”字。因为“方”就是病人邪气正盛,月亮正圆,天气正温和,身体尚安定的时候。要在病人正吸气的时候进针,再等到他正吸气的时候转针。还要等他正呼气的时候慢慢地拔出针来,所以说“泻必用方”,这样,邪气排出,正气流畅,病就会好了。补法必须掌握一个“圆”字。“圆”就是使气运行的意思,行气就是导移血气以至病所,针刺时必须达到荣分,还要在病人吸气时推移其针。所以说圆与方的行针,都要用排针之法。所以善用针术养神的人,必须观察病人形体的肥瘦和荣卫血气的盛衰。因为血气是人的神气寄存之处,不可不谨慎调养。\\
黄帝说:讲得妙极了!把人的形体与阴阳四时结合起来,虚实的感应,无形的病况,要不是夫子您谁能明白呢?然而夫子多次说到形和神,究竟什么叫形神?希望详细听听。\\
岐伯说:请让我先讲形。所谓形,就是说还没有对疾病看得很清楚。问病人的病痛,再从经脉的变化去探索,病情才突然出现在眼前。要是按寻而不可得,便不知道病情了。因为靠诊察形体,才能知道病情,所以叫做形。\\
黄帝道:那什么叫神呢?\\
岐伯说:请让我讲讲神。所谓神,就是耳不闻杂声,目不见异物,心志开朗,非常清醒地领悟其中的道理,但这不是用言语所能表达的。有如观察一种东西,大家都在看,但只是自己看得真,刚才还好像很模糊的东西,突然明显起来,好像风吹云散,这就叫做神。对神的领会,是以三部九候脉法为本源的,真能达到这种地步,九针之论,就不必太拘泥了。\\
离合真邪论篇第二十七\\
黄帝问曰:余闻九针九篇,夫子乃因而九之,九九八十一篇,余尽通其意矣。经言气之盛衰,左右倾移,以上调下,以左调右,有余不足,补泻于荥输,余知之矣。此皆荣卫之倾移,虚实之所生,非邪气从外入于经也。余愿闻邪气之在经也,其病人何如?取之奈何?\\
岐伯对曰:夫圣人之起度数,必应于天地,故天有宿度,地有经水,人有经脉。天地温和,则经水安静;天寒地冻,则经水凝泣;天暑地热,则经水沸溢;卒风暴起,则经水波涌而陇起。夫邪之入于脉也,寒则血凝泣,暑则气淖泽,虚邪因而入客,亦如经水之得风也,经之动脉,其至也亦时陇起,其行于脉中循循然,其至寸口中手也,时大时小,大则邪至,小则平,其行无常处,在阴与阳,不可为度,从而察之,三部九候,卒然逢之,早遏其路。吸则内针,无令气忤;静以久留,无令邪布;吸则转针,以得气为故;候呼引针,呼尽乃去;大气皆出,故命曰泻。\\
帝曰:不足者补之,奈何?\\
岐伯曰:必先扪而循之,切而散之,推而按之,弹而怒之,抓而下之,通而取之,外引其门,以闭其神。呼尽内针,静以久留,以气至为故。如待所贵,不知日暮,其气以至,适而自护,候吸引针,气不得出。各在其处,推阖其门,令神气存,大气留止,故命曰补。\\
帝曰:候气奈何?\\
岐伯曰:夫邪去络入于经也,舍于血脉之中,其寒温未相得,如涌波之起也,时来时去,故不常在。故曰方其来也,必按而止之,止而取之,无逢其冲而泻之。真气者,经气也,经气太虚,故曰其来不可逢,此之谓也。故曰候邪不审,大气已过,泻之则真气脱,脱则不复,邪气复至,而病益蓄,故曰其往不可追,此之谓也。不可挂以发者,待邪之至时而发针泻矣,若先若后者,血气已尽,其病不可下,故曰知其可取如发机,不知其取如扣椎。故曰知机道者不可挂以发,不知机者扣之不发,此之谓也。\\
帝曰:补泻奈何?\\
岐伯曰:此攻邪也,疾出以去盛血,而复其真气。此邪新客,溶溶未有定处也,推之则前,引之则止,逆而刺之,温血也,刺出其血,其病立已。\\
帝曰:善!然真邪以合,波陇不起,候之奈何?\\
岐伯曰:审扪循三部九候之盛虚而调之。察其左右上下相失及相减者,审其病脏以期之。不知三部者,阴阳不别,天地不分,地以候地,天以候天,人以候人,调之中府,以定三部。故曰刺不知三部九候,病脉之处,虽有大过且至,工不能禁也。诛罚无过,命曰大惑,反乱大经,真不可复。用实为虚,以邪为真,用针无义,反为气贼,夺人正气,以从为逆,荣卫散乱,真气已失,邪独内著,绝人长命,予人夭殃。不知三部九候,故不能久长。因不知合之四时五行,因加相胜,释邪攻正,绝人长命。邪之新客来也,未有定处,推之则前,引之则止,逢而泻之,其病立已。\\
黄帝问:我听说九针有九篇,而夫子又从九篇基础上发挥,演绎成为九九八十一篇,我已经完全领会它的精神了。针经上说的气之盛衰,左右偏胜,取上以治下,取左以治右,有余不足,在荥输之间补泻,我也知道了。这些都是由于荣卫偏胜、气血虚实而生成的,并不是邪气从外侵入经脉而发生的。我希望听听邪气侵入经脉之时,使人发病的情况怎样呢?又怎样取穴治疗呢?\\
岐伯回答说:圣人制定治疗法则,必定应于天地自然的变化,所以天有二十八宿,地有十二经水,人有十二经脉。如天地之气温和,则十二经水安静平稳;天寒地冻,则经水凝涩不流;天暑地热,则经水沸腾扬溢;暴风骤起,则经水波涛汹涌。同样,病邪侵入经脉,寒邪则使血行滞涩,热邪则使血气滑润流利,要是虚邪侵入,也像经水遇到暴风一样,经脉的搏动也会波涌隆起,虽然血气在经脉中还是有规则地流动,但其行至寸口处,指下的感觉,则时大时小,大即说明病邪盛,小即说明病邪退,邪气运行,没有一定规律,或在阴经或在阳经,应该进一步用三部九候的方法检查,一旦察知邪气所在,就应早治,阻止它进一步发展。在吸气时进针,勿使气逆;要留针静候其气,不让病邪扩散;吸气时捻针,以得气为目的;等病人呼气时,慢慢地起针,呼气尽时,将针取出;这样,大邪之气都随针外泄,所以叫做泻法。\\
黄帝问:不足之虚证,怎样用针刺补益呢?\\
岐伯说:首先用手抚摸寻找到穴位;然后用手指按压穴位,使邪气消散;再用手指推循揉按穴位周围肌肤;进而用手指弹其穴位,使脉络怒张;左手掐正穴位,右手下针;待气脉流通而出针,出针时,右手拔针,左手按闭孔穴,不让正气外泄。在病人呼气将尽时进针,静候其气,稍久留针,以得气为目的。就像等待贵客一样,忘记早晚,当得气时,要好好守护,等病人吸气的时候,拔出其针,那么正气就不能外泄了。出针以后,应分别在孔穴上揉按,使针孔关闭,真气存内,大经之气留于营卫而不泄,这叫补法。\\
黄帝问:怎样候气呢?\\
岐伯说:当邪气离开络脉而进入经脉,留舍于血脉之中,或寒或温,真邪尚未相合,所以脉气如江水之波动,忽起忽伏,时来时去,无有定处。所以说:邪气方来,必须按而止之,阻止后要用针刺克服它,但不要在邪气冲盛时用针泻之。真气,就是经脉之气,正邪冲突,真气大虚,这时用泻法,反使经气大虚,所以说:在邪气方来正盛的时候不可用泻法,说的就是这个意思。因此,诊候邪气不能审慎全面,当大邪之气已经过去,而用泻法,则反使真气虚脱,真气虚脱,则不能恢复,而邪气又回来,那病就更重了,所以说:邪气已经离去,不可再用泻法追之,说的就是这个意思。阻止邪气,使用泻法,是间不容发的事,须待邪气初至之时,即下针去泻,在邪至之前,或在邪去之后用泻法,都是不适时的,非但不能去邪,反使血气受伤,病邪就不容易消退了,所以说:懂得用针的,像扳动弩机一样,机智灵活;不善于用针的,就像敲击木椎,顽钝不灵。所以说:识得机微之道的,毫不迟疑;不知机微之道的,纵然时机已到,亦不会下针,说的就是这个意思。\\
黄帝问:怎样补泻呢?\\
岐伯说:这就是攻邪,应该抓紧时间刺出盛血,恢复正气。因为病邪刚刚侵入,流动未有定处,推之则前进,引之则留止,迎其气而泻之,以出其毒血,血出之后,病立即就会好。\\
黄帝说:讲得好!如果病邪和真气并合以后,脉气不现波动,那怎样诊察呢?\\
岐伯说:仔细扪按循摸,根据三部九候的盛衰虚实而调治。诊察三部九候看他的左右上下各部是否协调,及是否有特别衰弱之处,就可以知道病在哪一脏腑,待其气至而刺了。不懂三部九候,则阴阳不能辨别,上下也不能分清,更不知道从下部脉(地)以诊察下,从上部脉(天)以诊察上,从中部脉(人)以诊察中,结合胃气多少有无来判定疾病在三部的哪一部。所以说:针刺而不知三部九候,以了解病脉之处,虽然有大病为害,医生也不能事先防止。如果诛罚无过,不当泻而泻之,这就叫“大惑”,反而会扰乱脏腑经脉,使真气不能恢复。把实证当作虚证,邪气当作真气,用针毫无道理,反助邪气为害,耗夺正气,使顺证变成逆证,使荣卫散乱,真气散失,邪气独存于内,断送病人的性命,给人带来最大的祸殃。不知三部九候的医生,是不能长久的。不懂得配合四时五行因加相胜的道理,不攻邪气,反伤正气,就断绝了病人性命。病邪刚侵入人体,没有固定部位,推它就向前,引它就停止,迎其气而泻之,其病马上就能好。\\
通评虚实论篇第二十八\\
黄帝问曰:何谓虚实?\\
岐伯对曰:邪气盛则实,精气夺则虚。\\
帝曰:虚实何如?\\
岐伯曰:气虚者,肺虚也,气逆者,足寒也。非其时则生,当其时则死。余脏皆如此。\\
帝曰:何谓重实?\\
岐伯曰:所谓重实者,言大热病,气热脉满,是谓重实。\\
帝曰:经络俱实何如?何以治之?\\
岐伯曰:经络皆实,是寸脉急而尺缓也,皆当治之。故曰:滑则从,涩则逆也。夫虚实者,皆从其物类始,故五脏骨肉滑利,可以长久也。\\
帝曰:络气不足,经气有余,何如?\\
岐伯曰:络气不足,经气有余者,脉口热而尺寒也。秋冬为逆,春夏为从,治主病者。\\
帝曰:经虚络满,何如?\\
岐伯曰:经虚络满者,尺热满脉口寒涩也。此春夏死,秋冬生也。\\
帝曰:治此者奈何?\\
岐伯曰:络满经虚,灸阴刺阳;经满络虚,刺阴灸阳。\\
帝曰:何谓重虚?\\
岐伯曰:脉虚、气虚、尺虚,是谓重虚。\\
帝曰:何以治之?\\
岐伯曰:所谓气虚者,言无常也;尺虚者,行步恇然;脉虚者,不象阴也。如此者,滑则生,涩则死也。\\
帝曰:寒气暴上,脉满而实,何如?\\
岐伯曰:实而滑则生,实而逆则死。\\
帝曰:脉实满,手足寒,头热,何如?\\
岐伯曰:春秋则生,冬夏则死。脉浮而涩,涩而身有热者死。\\
帝曰:其形尽满,何如?\\
岐伯曰:其形尽满者,脉急大坚,尺涩而不应也。如是者,故从则生,逆则死。\\
帝曰:何谓从则生,逆则死?\\
岐伯曰:所谓从者,手足温也;所谓逆者,手足寒也。\\
帝曰:乳子而病热,脉悬小者,何如?\\
岐伯曰:手足温则生,寒则死。\\
帝曰:乳子中风热,喘鸣肩息者,脉何如?\\
岐伯曰:喘鸣肩息者,脉实大也。缓则生,急则死。\\
帝曰:肠澼便血,何如?\\
岐伯曰:身热则死,寒则生。\\
帝曰:肠澼下白沫,何如?\\
岐伯曰:脉沉则生,脉浮则死。\\
帝曰:肠澼下脓血,何如?\\
岐伯曰:脉悬绝则死,滑大则生。\\
帝曰:肠澼之属,身不热,脉不悬绝,何如?\\
岐伯曰:滑大者曰生,悬涩者曰死,以脏期之。\\
帝曰:癫疾何如?\\
岐伯曰:脉搏大滑,久自已;脉小坚急,死不治。\\
帝曰:癫疾之脉,虚实何如?\\
岐伯曰:虚则可治,实则死。\\
帝曰:消瘅虚实,何如?\\
岐伯曰:脉实大,病久可治;脉悬小坚,病久不可治。\\
帝曰:形度、骨度、脉度、筋度,何以知其度也?\\
帝曰:春亟治经络,夏亟治经俞,秋亟治六腑,冬则闭塞。闭塞者,用药而少针石也。所谓少针石者,非痈疽之谓也,痈疽不得顷时回。痈不知所,按之不应,乍来乍已,刺手太阴傍三痏与缨脉各二。掖痈大热,刺足少阳五;刺而热不止,刺手心主三,刺手太阴经络者大骨之会各三。暴痈筋仩,随分而痛,魄汗不尽,胞气不足,治在经俞。\\
腹暴满,按之不下,取手太阳经络者,胃之募也,少阴俞去脊椎三寸傍五,用员利针。霍乱,刺俞傍五,足阳明及上傍三。刺痫惊脉五,针手太阴各五,刺经太阳五,刺于少阴经络傍者一,足阳明一,上踝五寸刺三针。\\
凡治消瘅仆击,偏枯痿厥,气满发逆,肥贵人,则高梁之疾也。隔塞闭绝,上下不通,则暴忧之病也。暴厥而聋,偏塞闭不通,内气暴薄也。不从内,外中风之病,故瘦留著也。蹠跛,寒风湿之病也。\\
黄帝曰:黄疸,暴痛,癫疾厥狂,久逆之所生也。五脏不平,六腑闭塞之所生也。头痛耳鸣,九窍不利,肠胃之所生也。\\
黄帝问:什么叫虚实?\\
岐伯回答说:邪气正盛,就是实证;精气不足,就是虚证。\\
黄帝问:虚实的情况怎样?\\
岐伯说:肺主气,气虚,就是肺脏先虚;气上逆,两足必寒。肺虚若不在相克的时令,其人可生存;若遇相克之时,病人就会死亡。其他各脏的虚实情况也是如此。\\
黄帝问:什么叫重实?\\
岐伯说:所谓的重实,是说大热病人,邪气甚热,而脉象又盛满,这就叫重实。\\
黄帝问:经络俱实,情况怎样?怎么治疗?\\
岐伯说:所谓经络俱实,是指寸口脉急而尺肤弛缓,经和络都应该治疗。所以说:脉象滑利的象征气血旺盛,为顺;脉象涩滞的象征气血虚衰,为逆。一般所谓虚实,都是从事物表现于外的不同生命状态的种类开始判断的,如万物有生气则滑利,万物欲死则枯涩,所以一个人的五脏骨肉滑利,证明精气充足,生气旺盛,可以长寿。\\
黄帝问:络气不足,经气有余,情况怎样?\\
岐伯说:络气不足,经气有余,是指寸口脉滑利而尺肤寒凉。秋冬季见这样现象,为逆;在春夏之时见这样现象,为顺,治疗必须结合时令。\\
黄帝问:经虚络满的情况怎样?\\
岐伯说:经虚络满,是指尺肤热而盛满,而寸口脉象迟缓而涩滞。这种现象,在春夏则死,在秋冬则生。\\
黄帝问:怎样治疗这种病呢?\\
岐伯说:络满经虚的,灸阴经刺阳经;经满络虚的,刺阴经灸阳经。\\
黄帝问:什么叫重虚?\\
岐伯说:脉虚、气虚、尺虚,称为重虚。\\
黄帝问:怎样辨别呢?\\
岐伯说:所谓气虚,是由于精气虚夺,而语言低微,不能接续;所谓尺虚,是尺肤脆弱,而行动怯弱无力;所谓脉虚,是阴血虚少,不似有阴的脉象。像这种情况,脉象滑利的,虽病可生;脉象涩滞的,就要死亡了。\\
黄帝问:寒气突然上逆,脉象盛满而实,结果会怎样呢?\\
岐伯说:脉象实而滑利的,可生;脉象实而逆涩的,会死。\\
黄帝问:脉象实满,手足寒冷,头部发热,结果会怎样呢?\\
岐伯说:这种病,在春秋之时可生,若在冬夏就会死。有一种脉象浮而涩,脉涩而身体发热的,也会死。\\
黄帝问:身体肿满的,会怎样呢?\\
岐伯说:所谓身形肿满是指脉象急而大坚,而尺肤却涩滞,与脉不相适应。像这样,顺则可生,逆则主死。\\
黄帝问:什么叫顺则可生,逆则主死?\\
岐伯说:所谓顺,就是手足温暖;所谓逆,就是手足厥冷。\\
黄帝问:妇人新产而患热病,脉象悬小,结果怎样?\\
岐伯说:手足温暖的可生;若手足厥冷的,就要死亡。\\
黄帝问:妇人新产,感受风热,喘息有声,张口抬肩,它的脉象怎样?\\
岐伯说:喘息有声,张口抬肩的,脉象应实大。如脉象见缓和的,可生;如实大而弦急的,就要死亡。\\
黄帝问:肠澼中赤痢的,变化怎样?\\
岐伯说:下痢发热的,主死;身寒不发热的,主生。\\
黄帝问:痢疾而下白沫的,变化怎样?\\
岐伯说:脉沉则生,脉浮则死。\\
黄帝问:痢疾下脓血的,变化怎样?\\
岐伯说:脉悬绝者主死;滑大者主生。\\
黄帝问:痢疾,不发热,脉象也不悬绝,结果如何?\\
岐伯说:脉象滑大者主生,脉象悬涩者主死,根据五脏相克的时日来预测死期。\\
黄帝问:癫疾的情况怎样?\\
岐伯说:脉象搏击而大滑的,经过一段时间会自己痊愈;要是脉象小而坚急的,是不治的死证。\\
黄帝问:癫疾之脉,虚实变化怎样?\\
岐伯说:脉虚的可治,脉实的主死。\\
黄帝问:消渴病脉象的虚实怎样?\\
岐伯说:脉象见实大的,病程虽长久,但可以治愈;如脉象悬小而坚的,病程久了,就不可治。\\
黄帝说:形度、骨度、脉度、筋度,怎样才测量得出来呢?\\
黄帝说:春天治病多取各经的络穴,夏天治病多取各经的输穴,秋天治病多取六腑的合穴,冬天主闭藏。闭藏的,治病应多用药物,少用针刺砭石。但所谓少用针石,不包括痈疽等病在内,痈疽等病,是片刻也不能徘徊犹豫的。痈毒初起,不知它发在何处,按又按不到,疼痛忽作忽止,这时可在手太阴经穴针刺三次,和颈部缨脉左右各二次。腋痈的病人,周身高热,应该针刺足少阳经穴五次;针后,热仍然不下,可针手厥阴心包经穴三次,针手太阴经的络穴和大骨之会各三次。急性痈疽,筋肉挛缩,随着痈疽的分肉而痛剧,汗出不止,这是由于膀胱经气不足,应该刺其经的腧穴。\\
腹部突然胀满,按之不减,取手太阳经的络穴,即胃的募穴和脊椎两旁三寸的少阴肾俞穴各刺五次,用员利针。霍乱,刺肾俞旁志室穴五次,和足阳明胃俞及胃仓穴各三次。刺治癫痫惊风,要针五条经脉的穴位,刺手太阴的经穴各五次,刺太阳的经穴各五次,刺手少阴通里穴旁的手太阳经支正穴一次,足阳明经之解溪穴一次,足踝上五寸的少阴经筑宾穴三次。\\
大凡诊治消渴、中风仆倒、半身不遂、痿厥、气粗急、喘逆等病,如是甘食美味的肥胖贵人患这种病,则是由于偏嗜肉食厚味造成的疾病。噎膈则气闭不行,上下不通,都是暴怒或忧郁所引起的疾病。突然厥逆,不知人事,耳聋,闭塞不通,都是因为情绪突然激动,逆气上迫所致。有的病不从内发,外中风邪,风邪留恋不去,伏而化热,消烁肌肉,极为明显。两脚偏跛,是风寒湿侵袭而形成的疾病。\\
黄帝说:黄疸、骤然剧痛、癫疾、厥狂等证,是经脉之气久逆于上,不能下行所致。五脏不和,是六腑闭塞不通所致。头痛耳鸣,九窍不利,是肠胃的病变所致。\\
太阴阳明论篇第二十九\\
黄帝问曰:太阴阳明为表里,脾胃脉也,生病而异者何也?\\
岐伯对曰:阴阳异位,更虚更实,更逆更从,或从内,或从外,所从不同,故病异名也。\\
帝曰:愿闻其异状也。\\
岐伯曰:阳者,天气也,主外;阴者,地气也,主内。故阳道实,阴道虚。故犯贼风虚邪者,阳受之;食饮不节,起居不时者,阴受之。阳受之则入六腑,阴受之则入五脏。入六腑则身热不时卧,上为喘呼;入五脏则尒满闭塞,下为飧泄,久为肠澼。故喉主天气,咽主地气。故阳受风气,阴受湿气。故阴气从足上行至头,而下行循臂至指端;阳气从手上行至头,而下行至足。故曰阳病者上行极而下,阴病者下行极而上。故伤于风者,上先受之;伤于湿者,下先受之。\\
帝曰:脾病而四支不用何也?\\
岐伯曰:四支皆禀气于胃,而不得至经,必因于脾,乃得禀也。今脾病不能为胃行其津液,四支不得禀水谷气,气日以衰,脉道不利,筋骨肌肉,皆无气以生,故不用焉。\\
帝曰:脾不主时何也?\\
岐伯曰:脾者土也,治中央,常以四时长四脏,各十八日寄治,不得独主于时也。脾脏者常著胃土之精也,土者生万物而法天地,故上下至头足,不得主时也。\\
帝曰:脾与胃以膜相连耳,而能为之行其津液何也?\\
岐伯曰:足太阴者,三阴也。其脉贯胃属脾络嗌,故太阴为之行气于三阴。阳明者表也,五脏六腑之海也,亦为之行气于三阳。脏腑各因其经而受气于阳明,故为胃行其津液。四支不得禀水谷气,日以益衰,阴道不利,筋骨肌肉无气以生,故不用焉。\\
黄帝问道:太阴、阳明两经,互为表里,即脾胃二脉,而所生的疾病不同,这是什么道理呢?\\
岐伯回答说:脾属阴经,胃属阳经,二经循行的路线不同;或虚、或实,或顺、或逆,也各不相同;或者从内,或者从外,发病的原因又不同,所以病名也就不同了。\\
黄帝说:希望听听不同情况。\\
岐伯说:阳属天气,为人体的外在护卫;阴属地气,为人体的内在营养。外邪有余多犯阳经,所以阳道常实;内伤不足多伤阴经,所以阴道常虚。所以贼风虚邪伤人时,阳分首当其冲;而饮食不慎,起居失调,阴分独受其害。外表受病,就传入六腑;内里受病,就传入五脏。邪入六腑,就会发热,不能安眠,喘促;病在五脏,就会胀满发闷,飧泄,经过一段时间,会成为痢疾。喉是管呼吸的,所以主天气;咽是管进食的,所以主地气。阳经易感风邪,阴经益感湿邪。三阴之经脉,是由足上行至头,由头而下循臂至手指尖端;三阳之经脉,是由手上行至头,再下至足。所以阳经的病邪,先上行到极点,再向下行;阴经的病邪,先向下行到极点,再向上行。因此外感风邪,多在上部;外中湿气,多在下部。\\
黄帝问:脾一有病四肢就不能正常活动,是什么道理?\\
岐伯说:四肢都从胃接受营养之气,但是胃气不能直达到四肢,必须通过脾的运化,水谷津液才能布达于四肢。如今脾有病了,不能把胃的水谷津液输送出去,四肢得不到水谷精气,一天一天地衰弱,经脉不利,筋骨肌肉也因无营养之气来充实,所以四肢就不能活动了。\\
黄帝问:脾脏不能单独主一个时季,是什么原因?\\
岐伯说:脾属土而位居中央,它经常从四时里分旺于四脏,就是在四季之末各十八日里,不能单独主一个季节。脾脏经常为胃土转输水谷精气,土生养万物而取法天地自然的规律,脾为人体的土,它布散的精微,从头至足,无处不到,所以不单主一个季节。\\
黄帝问:脾和胃通过一膜相连,为什么能够给胃运行津液呢?\\
岐伯说:足太阴脾经,就是三阴经。它的经脉环绕于胃,连属于脾,联络咽喉,所以太阴经脉能够运输阳明之气,进入手足三阴经。足阳明胃经,是足太阴脾经之表,是五脏六腑的营养之海,所以胃经也能运输太阴之气,进入手足三阳经。五脏六腑都能借助脾经而接受阳明的水谷精气,因此说脾能为胃运送津液。四肢不能接受水谷之气的滋养,一天天地衰弱,阴经脉道不通利,筋骨肌肉无气滋生,所以就痿废不用了。\\
阳明脉解篇第三十\\
黄帝问曰:足阳明之脉病,恶人与火,闻木音则惕然而惊,钟鼓不为动。闻木音而惊,何也?愿闻其故。\\
岐伯对曰:阳明者,胃脉也。胃者,土也。故闻木音而惊者,土恶木也。\\
帝曰:善。其恶火何也?\\
岐伯曰:阳明主肉,其脉血气盛,邪客之则热,热甚则恶火。\\
帝曰:其恶人,何也?\\
岐伯曰:阳明厥则喘而惋,惋则恶人。\\
帝曰:或喘而死者,或喘而生者,何也?\\
岐伯曰:厥逆连脏则死,连经则生。\\
帝曰:善!病甚则弃衣而走,登高而歌,或至不食数日,逾垣上屋,所上之处,皆非其素所能也,病反能者,何也?\\
岐伯曰:四支者,诸阳之本也。阳盛则四支实,实则能登高也。\\
帝曰:其弃衣而走者,何也?\\
岐伯曰:热盛于身,故弃衣欲走也。\\
帝曰:其妄言骂詈,不避亲疏而歌者,何也?\\
岐伯曰:阳盛则使人妄言骂詈,不避亲疏,而不欲食,不欲食,故妄走也。\\
黄帝问:足阳明的经脉发生病变,厌恶人声与火光,听到木器响动的声音就受惊,却不为钟鼓的声音惊动。听到木音就惊惕,是为什么?希望听听其中的道理。\\
岐伯回答说:足阳明是胃的经脉。胃属土。所以听到木音而惊惕,是因为土恶木克。\\
黄帝问:好!那么恶火是为什么呢?\\
岐伯说:足阳明经主肌肉,其经脉多血多气,外邪侵袭就发热,热重所以恶火。\\
黄帝问:他厌恶见人是什么道理?\\
岐伯说:足阳明经气上逆,则喘促,心中郁闷,心中郁闷所以不喜欢见人。\\
黄帝说:有的阳明厥逆喘促而死,有的虽喘促而不死,为什么呢?\\
岐伯说:经气厥逆如果累及于五脏,就会死,若仅连及经脉,则可生。\\
黄帝问:好!病情严重的,脱掉衣服,乱跑乱跳,登上高处,狂喊唱歌,有的几天不吃不喝,还能翻墙上屋,所上之处,都是其平素所不能的,发病时反而能上去,为什么呢?\\
岐伯说:四肢是阳气的根本。阳气盛则四肢充实,四肢充实所以能够登高。\\
黄帝问:病人脱掉衣服而乱跑,为什么?\\
岐伯说:身体热盛,所以脱掉衣服,到处乱跑。\\
黄帝问:病人胡言乱语骂人,不分亲疏远近而胡乱唱歌,为什么?\\
岐伯说:阳热亢盛,使病人神志失常,胡言乱语,辱骂别人,不分亲疏,不想吃饭,不想吃饭,所以到处乱跑。\\
卷九\\
热论篇第三十一\\
黄帝问曰:今夫热病者,皆伤寒之类也。或愈或死,其死皆以六七日之间,其愈皆以十日以上者,何也?不知其解,愿闻其故。\\
岐伯对曰:巨阳者,诸阳之属也。其脉连于风府,故为诸阳主气也。人之伤于寒也,则为病热,热虽甚不死。其两感于寒而病者,必不免于死。\\
帝曰:愿闻其状。\\
岐伯曰:伤寒一日,巨阳受之,故头项痛,腰脊强。二日,阳明受之,阳明主肉,其脉挟鼻络于目,故身热,目疼而鼻干,不得卧也。三日,少阳受之,少阳主胆,其脉循胁络于耳,故胸胁痛而耳聋。三阳经络皆受其病,而未入于脏者,故可汗而已;四日,太阴受之,太阴脉布胃中,络于嗌,故腹满而嗌干。五日,少阴受之,少阴脉贯肾,络于肺,系舌本,故口燥舌干而渴。六日,厥阴受之,厥阴脉循阴器而络于肝,故烦满而囊缩。三阴三阳,五脏六腑皆受病,荣卫不行,五脏不通,则死矣。\\
其不两感于寒者,七日,巨阳病衰,头痛少愈。八日,阳明病衰,身热少愈。九日,少阳病衰,耳聋微闻。十日,太阴病衰,腹减如故,则思饮食。十一日,少阴病衰,渴止不满,舌干已而嚏。十二日,厥阴病衰,囊纵,少腹微下,大气皆去,病日已矣。\\
帝曰:治之奈何?\\
岐伯曰:治之各通其脏脉,病日衰已矣。其未满三日者,可汗而已;其满三日者,可泄而已。\\
帝曰:热病已愈,时有所遗者,何也?\\
岐伯曰:诸遗者,热甚而强食之,故有所遗也。若此者,皆病已衰而热有所藏,因其谷气相薄,两热相合,故有所遗也。\\
帝曰:善。治遗奈何?\\
岐伯曰:视其虚实,调其逆从,可使必已矣。\\
帝曰:病热当何禁之?\\
岐伯曰:病热少愈,食肉则复,多食则遗,此其禁也。\\
帝曰:其病两感于寒者,其脉应与其病形何如?\\
岐伯曰:两感于寒者,病一日,则巨阳与少阴俱病,则头痛,口干而烦满;二日,则阳明与太阴俱病,则腹满,身热,不欲食,谵言;三日,则少阳与厥阴俱病,则耳聋,囊缩而厥。水浆不入,不知人,六日死。\\
帝曰:五脏已伤,六腑不通,荣卫不行,如是之后,三日乃死,何也?\\
岐伯曰:阳明者,十二经脉之长也。其血气盛,故不知人,三日其气乃尽,故死矣。\\
凡病伤寒而成温者,先夏至日者为病温,后夏至日者为病暑。暑当与汗皆出,勿止。\\
黄帝问道:一般所谓热病,都属于伤寒一类。有的痊愈了,有的死亡了,死亡的都在六七日之间,痊愈的大约在十日以上,这是什么道理?我不知其中的缘故,希望听听其中的道理。\\
岐伯答道:足太阳经,是诸阳联属会合之处。它的经脉上连风府,所以能够为诸阳主气。人为寒邪所伤,就要发热,如果单是发热,即便热得很厉害,也不会死。但假如阳经、阴经同时感受寒邪为病,就必然死亡。\\
黄帝道:希望听听伤寒的症状。\\
岐伯说:伤寒第一天,太阳经感受寒邪,所以头项疼痛,腰脊僵硬。第二天,病邪传到阳明,阳明经主肌肉,它的经脉挟鼻,络于目,所以身热、目疼、鼻干,不能安卧。第三天,病邪传到少阳,少阳主胆,它的经脉循行于两胁,络于两耳,所以胸胁痛,耳聋。如果三阳经络都已受病,但还没有传入到脏腑里的,可以用发汗来治愈。第四天,病邪传到太阴,太阴经脉分布于胃,络于咽嗌,所以腹胀满,咽嗌发干。第五天,病邪传入少阴,少阴经脉通肾、络肺,连系舌根,所以口燥,舌干而渴。第六天,病邪传入厥阴,厥阴经脉环绕阴器,络于肝,所以烦闷、阴囊紧缩。如果三阴三阳经、五脏六腑都受了病害,营卫不运行,腑脏不通畅,那就要死了。\\
如果不是两感于寒邪,到第七天,太阳病就会减轻,头痛也就会稍好一些。到第八天,阳明病会减轻,身热也会渐渐消退。到第九天,少阳病会减轻,耳聋也会好转而能听到点声音。到第十天,太阴病会减轻,胀起的腹部也会平软得和往常一样,就想吃东西了。到第十一天,少阴病会减轻,口不渴了,也不胀满了,舌也不干了,还会打喷嚏。到第十二天,厥阴病减轻了,阴囊也松缓下来,少腹部也觉得舒服,邪气全退了,病也就好了。\\
黄帝又问:怎样治疗呢?\\
岐伯回答说:治疗的方法,应根据脏腑经脉的症状,分别施治,疾病就会日渐衰退。受病未满三天的,可以通过发汗治愈;病已超过三天的,可以通过泻下治愈。\\
黄帝道:热病已经好了,常常遗有余热,为什么?\\
岐伯说:凡是余热,都是因为发热重的时候,还勉强吃东西造成的。像这样,病虽然已经减轻,可是余热未尽,于是谷气与余热搏结在一起,所以就有余热现象。\\
黄帝说:说得好。那么怎样治疗余热呢?\\
岐伯说:只要根据病的或虚或实,而分别给以正治和反治,病就会好的。\\
黄帝道:患了热病有什么禁忌呢?\\
岐伯说:患热病的,如果稍好些,马上吃肉类食物,就会复发,如果多吃谷食,也会有余热,这就是热病的禁忌。\\
黄帝道:假如两感于寒的病人,它的脉象和症状怎样呢?\\
岐伯说:两感于寒的病人,第一天太阳和少阴二经都患病,就有头痛、口干、烦闷而渴的症状;第二天阳明与太阴二经都患病,就有腹满、发烧、不想吃东西,语无伦次的症状;第三天少阳与厥阴二经都患病,就有耳聋、阴囊紧缩、厥逆的症状。如果再发展到水浆不入口,昏迷不醒,第六天就得死。\\
黄帝说:病情发展到五脏都已损伤,六腑不通,营卫不和的地步以后,三天之后就死亡了,这是为什么?\\
岐伯说:阳明经是十二经脉中最重要的。这一经血气与邪气都盛,正邪相搏病人容易神志昏迷,三天以后阳明经气已尽,所以就死亡了。\\
凡伤于寒邪而变成温病的,在夏至以前发病的叫做温病;在夏至以后发病的叫做暑病。暑病应当发汗,使热从汗出,而不能止汗。\\
刺热篇第三十二\\
肝热病者,小便先黄,腹痛多卧,身热。热争,则狂言及惊,胁满痛,手足躁,不得安卧;庚辛甚,甲乙大汗,气逆则庚辛死。刺足厥阴少阳。其逆则头痛员员,脉引冲头也。\\
心热病者,先不乐,数日乃热。热争,则卒心痛,烦闷善呕,头痛面赤无汗;壬癸甚,丙丁大汗,气逆则壬癸死。刺手少阴太阳。\\
脾热病者,先头重,颊痛,烦心颜青,欲呕,身热。热争,则腰痛不可用俯仰,腹满泄,两颔痛,甲乙甚,戊己大汗,气逆则甲乙死。刺足太阴阳明。\\
肺热病者,先淅然厥,起毫毛,恶风寒,舌上黄,身热。热争,则喘欬,痛走胸膺背,不得太息,头痛不堪,汗出而寒;丙丁甚,庚辛大汗,气逆则丙丁死。刺手太阴阳明,出血如大豆,立已。\\
肾热病者,先腰痛气痠,苦渴数饮,身热。热争,则项痛而强,气寒且痠,足下热,不欲言,其逆则项痛员员澹澹然,戊己甚,壬癸大汗,气逆则戊己死。刺足少阴太阳。诸汗者,至其所胜日,汗出也。\\
肝热病者,左颊先赤;心热病者,颜先赤;脾热病者,鼻先赤;肺热病者,右颊先赤;肾热病者,颐先赤。病虽未发,见赤色者刺之,名曰治未病。热病从部所起者,至期而已;其刺之反者,三周而已,重逆则死。诸当汗者,至其所胜日,汗大出也。\\
诸治热病,以饮之寒水,乃刺之,必寒衣之,居止寒处,身寒而止也。\\
热病先胸胁痛,手足躁,刺足少阳,补足太阴。病甚者为五十九刺。热病始手臂痛者,刺手阳明太阴而汗出止。热病始于头首者,刺项太阳而汗出止。热病始于足胫者,刺足阳明而汗出止。热病先身重骨痛,耳聋好瞑,刺足少阴,病甚为之五十九刺。热病先眩冒而热,胸胁满,刺足少阴少阳。\\
太阳之脉,色荣颧骨,热病也,荣未交,曰今且得汗,待时而已。与厥阴脉争见者,死期不过三日,其热病内连肾,少阳之脉色也。少阳之脉,色荣颊前,热病也,荣未交,曰今且得汗,待时而已,与少阴脉争见者,死期不过三日。\\
热病气穴:三椎下间主胸中热,四椎下间主鬲中热,五椎下间主肝热,六椎下间主脾热,七椎下间主肾热。荣在骶也。项上三椎陷者中也。颊下逆颧为大瘕,下牙车为腹满,颧后为胁痛。颊上者,鬲上也。\\
肝脏发生热病,小便先发黄,腹痛,喜卧,身体发热。热邪和正气相争,就会狂言惊骇,胁肋胀满疼痛,手足躁动,不能安卧;遇庚辛日病情加重,遇甲乙日则大汗出而热退,如果邪气上逆,庚辛日会死。刺治足厥阴和足少阴两经。如果肝气上逆,则头痛昏晕,这是热邪由肝脉上冲牵引头部所致。\\
心脏发生热病,病人先感到不高兴,过几天才发热。热邪与正气相争,出现突然心痛,烦闷,时时作呕,头痛、面赤、无汗;遇壬癸日,病情加重,逢丙丁日则大汗出而热退,如果邪气上逆,壬癸日当死。刺治手少阴和手太阳两经。\\
脾脏发生热病,病人先感到头重,面颊疼痛,心烦,额部发青,想呕吐,身体发热。热邪与正气相争,则腰痛不可以俯仰,腹部胀满而泄泻,两颔疼痛,遇甲乙日,病情加重,遇戊己日则大汗出而热退,如果邪气上逆,甲乙日当死。刺治足太阴和足阳明两经。\\
肺脏发生热病,病人先突然感到寒冷,汗毛竖起,恶风寒,舌苔发黄,身体发热。热邪与正气相争,则气喘咳嗽,疼痛走窜胸膺背部,不能长呼吸,头痛难以忍受,汗出怕冷;遇丙丁日,病情加重,遇庚辛日则大汗出而热退,如果邪气上逆,丙丁日当死。刺治手太阴和手阳明两经,使其出血如黄豆大,立即痊愈。\\
肾脏发生热病,病人先感到腰痛,小腿发痠,口渴多饮,身体发热。热邪与正气相争,则项痛而强,小腿发冷而痠软,足心发热,不欲言语,如果肾气上逆,则颈项疼痛,头晕掉摇不定,遇戊己日,病情加重,遇壬癸日则大汗出而热退,如邪气上逆,戊己日当死。刺治足少阴和足太阳两经。以上所说的诸脏大汗出,都是在五脏各自当旺之日,正胜邪退,故可汗出而愈。\\
肝热病人,左颊先见红色;心热病人,额部先见红色;脾热病人,鼻部先见赤色;肺热病人,右颊先见红色;肾热病人,颐部先见红色。疾病尚未发生,见到面部的红色,就针刺治疗,这叫治未病。热病刚开始,只表现在五脏部所,疾病尚轻浅,及时治疗,到所胜之日,就可痊愈;如果刺治反了,势必延长病程,以致过“三周”才能病愈,如果治疗一再失误,就会死亡。热病应当汗出,到所胜之日,才能够汗大出而痊愈。\\
凡治疗热病,应让病人饮用凉饮,然后针刺,必须使病人少穿衣服,居处清凉,这样才能使热退身凉而病愈。\\
热病先出现胸胁痛的,手足躁动不安的,应刺足少阳经,补足太阴经。病情较重的,用五十九刺的方法。热病开始时手臂疼痛的,应刺手阳明、手太阴两经,汗出而热止。热病起始于头部的,刺足太阳经,汗出而热止。热病起始于足胫的,可刺足阳明经,汗出而热止。热病先感觉身体沉重,骨节疼痛,耳聋、嗜睡,当刺足少阴经,病重的,用五十九刺的方法。热病先出现头昏眩冒而后发热,胸胁胀闷的,应刺足少阴和足少阳两经。\\
太阳经脉的病,红色显现于两颧骨,是热病的象征,如果荣色未恶,只要得汗,至其当旺之时,就可痊愈。如果同时又见到厥阴经的脉证,那么死期不过三日,因为热病已内连肾脏,兼见少阳脉色。少阳经脉的病,红色显现于面颊前方,是热病之征,如果荣色未恶,只要得汗,至其当旺之时,就可痊愈,如果同时兼见少阴经的脉证,那死期就不过三日。\\
治疗热病的腧穴:第三脊椎下,主治胸中的热病;第四脊椎下,主治膈中的热病;第五脊椎下,主治肝热病;第六脊椎下,主治脾热病;第七脊椎下,主治肾热病。治疗热病,既取上部腧穴以泻阳邪,又当取下部腧穴以补阴气,所以在下取尾骶骨处的长强穴。颈项第三椎以下的凹陷中央,是大椎穴。如面颊红色由下上逆到颧骨的,为大瘕泄;红色自颊下行至颊车的,为腹部胀满;赤色见于颧骨后部的,为胁肋痛。红色见于颊上的,病在膈上。\\
评热病论篇第三十三\\
黄帝问曰:有病温者,汗出辄复热,而脉躁疾不为汗衰,狂言不能食,病名为何?\\
岐伯对曰:病名阴阳交,交者死也。\\
帝曰:愿闻其说。\\
岐伯曰:人所以汗出者,皆生于谷,谷生于精,今邪气交争于骨肉而得汗者,是邪却而精胜也。精胜,则当能食而不复热。复热者,邪气也;汗者,精气也。今汗出而辄复热者,是邪胜也,不能食者,精无禆也,病而留者,其寿可立而倾也。且夫《热论》曰:汗出而脉尚躁盛者死。今脉不与汗相应,此不胜其病也,其死明矣。狂言者,是失志,失志者死。今见三死,不见一生,虽愈必死也。\\
帝曰:有病身热,汗出烦满,烦满不为汗解,此为何病?\\
岐伯曰:汗出而身热者,风也:汗出而烦满不解者,厥也。病名曰风厥。\\
帝曰:愿卒闻之。\\
岐伯曰:巨阳主气,故先受邪,少阴与其为表里也,得热则上从之,从之则厥也。\\
帝曰:治之奈何?\\
岐伯曰:表里刺之,饮之服汤。\\
帝曰:劳风为病,何如?\\
岐伯曰:劳风法在肺下。其为病也,使人强上冥视,唾出若涕,恶风而振寒,此为劳风之病。\\
帝曰:治之奈何?\\
岐伯曰:以救俯仰,巨阳引。精者三日,中年者五日,不精者七日。咳出青黄涕,其状如脓,大如弹丸,从口中若鼻中出,不出则伤肺,伤肺则死也。\\
帝曰:有病肾风者,面胕痝然壅,害于言,可刺不?\\
岐伯曰:虚不当刺。不当刺而刺,后五日,其气必至。\\
帝曰:其至何如?\\
岐伯曰:至必少气时热,时热从胸背上至头,汗出手热,口干苦渴,小便黄,目下肿,腹中鸣,身重难以行,月事不来,烦而不能食,不能正偃,正偃则咳,病名曰风水,论在《刺法》中。\\
帝曰:愿闻其说。\\
岐伯曰:邪之所凑,其气必虚。阴虚者,阳必凑之,故少气时热而汗出也。小便黄者,少腹中有热也。不能正偃者,胃中不和也。正偃则咳甚,上迫肺也。诸有水气者,微肿先见于目下也。\\
帝曰:何以言?\\
岐伯曰:水者,阴也;目下,亦阴也;腹者,至阴之所居,故水在腹者,必使目下肿也。真气上逆,故口苦舌干,卧不得正偃,正偃则咳出清水也。诸水病者,故不得卧,卧则惊,惊则咳甚也。腹中鸣者,病本于胃也。薄脾则烦不能食。食不下者,胃脘膈也。身重难以行者,胃脉在足也。月事不来者,胞脉闭也。胞脉者,属心而络于胞中。今气上迫肺,心气不得下通,故月事不来也。\\
帝曰:善。\\
黄帝问:有患温热病的,汗出之后,随即又发热,而且脉搏躁动疾速,不因汗出而衰减,甚至出现胡言乱语,不能饮食,这是什么病?\\
岐伯回答说:这种病叫阴阳交,阴阳交是死证。\\
黄帝说:希望听听其中的道理。\\
岐伯说:人体所以能够出汗,是由于水谷入胃,化生精微。现在邪正在骨肉之间相争而能够出汗,是邪气退而精气胜。精气胜,就应该能进食而不再发热。再发热的,是邪气还在;汗是由精气所化。现在汗出而又发热,是邪气胜过正气。不能饮食,则精气没有补充,病邪再滞留不退,生命就会危在旦夕了。《热论》中曾说过:汗出而脉仍躁盛的,是死证。现在脉象与出汗不适应,可见精气不能战胜其病邪,死亡的征象已经很明显了。胡言乱语,是神志失常,神志失常的也是死证。现在出现了三种死候,却不见一线生机,虽然有时病情稍有减轻,但必死无疑。\\
黄帝问:有的病人身体发热,出汗,烦闷,烦闷不因汗出而解,这是什么病?\\
岐伯说:汗出而身体发热的,是感受了风邪;汗出而烦闷不解的,是由于下气上逆。这种病,名叫做风厥。\\
黄帝说:希望详细听听。\\
岐伯说:太阳经主宰一身诸阳之气,为一身之表,所以首先感受外邪,而少阴和太阳为表里关系,少阴经气受到太阳经热邪的影响,随之上逆,随之上逆便成为厥。\\
黄帝问:怎样治疗?\\
岐伯说:刺太阳、少阴表里两经的穴位,并内服汤药。\\
黄帝问:劳风病是怎样的?\\
岐伯说:劳风发病部位常在肺下。这种病的症状是头项强直,眼目昏眩,唾出粘痰,恶风而身体寒战,这就是劳风病。\\
黄帝问:怎样治疗?\\
岐伯说:首先要引导太阳经气,疏通郁闭,以通利肺气,使其呼吸调畅,俯仰自如。青年人,三日可以病愈;中年人,五日可以病愈;而老年人或精气衰减的人,须七日才能痊愈。这种病人,咳出的青黄色痰液,颜色如脓,凝结成块,大的如弹丸,这种痰应使之从口中或鼻中排出,如不能排出,就要伤肺,肺脏受伤,就会死亡了。\\
黄帝问:有患肾风的病人,面部浮肿,目下臃起如卧蚕,影响言语,这种病可以针刺吗?\\
岐伯说:如果是虚证不能用刺法。不应针刺法而误刺,五天后,邪气必然传导,甚而加重病情。\\
黄帝问:邪气到来的情况怎样?\\
岐伯说:邪气到来时一定出现气短,时常发热,热从胸背上至头部,汗出,手热,口干口渴,小便色黄,目下浮肿,腹中鸣响,身体沉重,行走困难,若是妇女则停经,胸中烦闷,不能进食,也不能仰卧,仰卧则咳嗽,这种病又叫风水,在《刺法》篇中有详细的论述。\\
黄帝说:希望听听其中的道理。\\
岐伯说:邪气侵犯人体,是因为他的正气必定虚弱。肾阴不足,风阳之邪就乘虚侵入,所以气短,时常发热而汗出。小便色黄,是小腹中有热。不能仰卧,是胃中不和。仰卧后咳嗽加剧,是水气上迫肺脏。一般有水气病的,目下必先出现微肿。\\
黄帝问:为什么这样说?\\
岐伯说:水属于阴,目下也属阴,腹部为至阴脾脏所居之处,所以腹中有水,必然目下浮肿。心气上逆,所以口苦舌干,不得仰卧,仰卧就咳吐清水。一般水气病,不能仰卧,仰卧后会惊悸不安,惊悸会使咳嗽加剧。腹中鸣响,病因是胃肠中有水气。如果波及到脾脏,就烦闷而不能饮食。食物不进,是胃脘阻隔。身体沉重行走困难,是因为胃的经脉循行于足胫部。妇女月经不来,是因为胞脉闭塞不通。胞脉属于心脏,而下络胞中。现在水气上逆,逼迫肺脏,心气不能下通,所以月经就不来了。\\
黄帝说:讲得好!\\
逆调论篇第三十四\\
黄帝问曰:人身非常温也,非常热也,为之热而烦满者,何也?\\
岐伯对曰:阴气少而阳气胜,故热而烦满也。\\
帝曰:人身非衣寒也,中非有寒气也,寒从中生者,何?\\
岐伯曰:是人多痹气也,阳气少,阴气多,故身寒如从水中出。\\
帝曰:人有四支热,逢风寒如炙如火者,何也?\\
岐伯曰:是人者,阴气虚,阳气盛。四支者阳也。两阳相得而阴气虚少,少水不能灭盛火,而阳独治。独治者,不能生长也,独胜而止耳。逢风而如炙如火者,是人当肉烁也。\\
帝曰:人有身寒,汤火不能热,厚衣不能温,然不冻栗,是为何病?\\
岐伯曰:是人者,素肾气胜,以水为事,太阳气衰,肾脂枯不长,一水不能胜两火。肾者水也,而生于骨,肾不生则髓不能满,故寒甚至骨也。所以不能冻栗者,肝一阳也,心二阳也,肾孤脏也,一水不能胜二火,故不能冻栗,病名曰骨痹,是人当挛节也。\\
帝曰:人之肉苛者,虽近衣絮,犹尚苛也,是谓何疾?\\
岐伯曰:荣气虚则不仁,卫气虚则不用,荣卫俱虚,则不仁且不用,肉如故也。人身与志不相有,曰死。\\
帝曰:人有逆气不得卧而息有音者;有不得卧而息无音者;有起居如故而息有音者;有得卧,行而喘者;有不得卧,不能行而喘者。有不得卧,卧而喘者。皆何脏使然?愿闻其故。\\
岐伯曰:不得卧而息有音者,是阳明之逆也。足三阳者下行,今逆而上行,故息有音也。阳明者胃脉也,胃者六腑之海,其气亦下行。阳明逆,不得从其道,故不得卧也。《下经》曰:“胃不和,则卧不安。”此之谓也。夫起居如故而息有音者,此肺之络脉逆也,络脉不得随经上下,故留经而不行。络脉之病人也微,故起居如故而息有音也。夫不得卧,卧而喘者,是水气之客也。夫水者,循津液而流也,肾者水脏,主津液,主卧与喘也。\\
帝曰:善。\\
黄帝问:人体不因为衣服穿得过多而温热,然而出现发热、烦闷,是为什么?\\
岐伯回答说:这是由于阴气虚少,阳气偏胜,所以发热而烦闷。\\
黄帝问:有的人不是由于衣服单薄而受寒,也不是体内素有寒气,却感到寒冷从体内产生,这是为什么?\\
岐伯说:这种人多痹气,阳气虚弱,阴气偏盛,所以感到身体寒冷像从冷水中出来一样。\\
黄帝说:有的人四肢发热,遇到风寒,热得更厉害,如同火烤一般,这是为什么?\\
岐伯说:这种人阴气虚弱,阳气偏盛。四肢属阳。属阳的四肢感受属阳的风邪,以致阴气更虚少,阳气更亢盛,衰弱的阴气不能熄灭旺盛的阳火,以致阳气独旺。阳气独旺,便不能生化成长,阳气独胜则生机停息。遇风如同火烤的人,肌肉会逐渐消瘦干枯。\\
黄帝问:有的人身体寒冷,即使用热水温熨或烤火,仍不觉热;虽多穿衣服,也不能温暖,但并不寒战,这是什么病呢?\\
岐伯说:这种人,肾气平素偏胜,从事水中作业,致使太阳经气虚衰,肾中脂液得不到阳气的温煦而枯耗不长。肾是水脏,而生长骨髓,如果肾的脂膏不生,骨髓就不能充满,以致感到寒冷至骨。之所以不发生寒战,是因为肝是一阳,心是二阳,肾是孤脏,一个肾水不能制胜肝心二阳之火的缘故,所以不发生战栗,病名叫骨痹,这种人必然骨节拘挛。\\
黄帝问:有的人,他的皮肉麻木沉重,虽然穿棉衣,仍然麻木不减,这是什么病?\\
岐伯说:荣气虚弱,会使皮肉麻木不仁;卫气虚弱,则肢体不能举动;荣卫俱虚,则既麻木不仁,又举动不便,肌肉就更加麻木沉重了。如果人的形体与神志活动不相适应时,就必然死亡。\\
黄帝说:患气逆的人,有的不能平卧,而且呼吸有喘鸣音;有的虽然不能平卧,但呼吸却没有喘鸣音;有的起居如常,然而呼吸喘促有声;有的能够平卧,行动则气喘;有的不能卧,也不能行动,却气喘;有的不能卧,卧则气喘。这些情况,是什么脏腑病变而发生的?希望听听其中的缘故。\\
岐伯说:不能卧而喘息有声的,是阳明经脉之气上逆。足三阳经脉之气,本来是下行的,现在逆而上行,所以就喘息有音了。阳明是胃脉,胃是六腑之海,胃气也下行。如果阳明经气上逆,胃气就不能循常道而下行,所以就不能平卧。《下经》中说:“胃不和,则卧不安。”说的就是这个意思。起居如常而喘息有音的,是肺的络脉气逆,络脉之气不能随经脉之气上下循行,所以其气留滞经脉而不行于络脉。但络脉使人发病比较轻,所以虽然喘息有声,但起居如常。不能平卧,卧则呼吸喘促的,是水气犯肺。水气是循着津液运行的道路而流动的。肾是水脏,主司津液,现肾病不能主水,水气上泛而侵肺,所以气喘而不能平卧。\\
黄帝说:讲得好!\\
卷十\\
疟论篇第三十五\\
黄帝问曰:夫痎疟皆生于风,其蓄作有时者,何也?\\
岐伯对曰:疟之始发也,先起于毫毛,伸欠乃作,寒栗鼓颔,腰脊俱痛,寒去则内外皆热,头痛如破,渴欲冷饮。\\
帝曰:何气使然?愿闻其道。\\
岐伯曰:阴阳上下交争,虚实更作,阴阳相移也。阳并于阴,则阴实而阳虚,阳明虚,则寒栗鼓颔也;巨阳虚,则腰背头项痛;三阳俱虚,则阴气胜,阴气胜则骨寒而痛;寒生于内,故中外皆寒;阳盛则外热,阴虚则内热,外内皆热则喘而渴,故欲冷饮也。此皆得之夏伤于暑,热气盛,藏于皮肤之内,肠胃之外,此荣气之所舍也。此令人汗空疏,腠理开,因得秋气,汗出遇风,及得之以浴,水气舍于皮肤之内,与卫气并居。卫气者,昼日行于阳,夜行于阴,此气得阳而外出,得阴而内薄,内外相薄,是以日作。\\
帝曰:其间日而作者,何也?\\
岐伯曰:其气之舍深,内薄于阴,阳气独发,阴邪内著,阴与阳争不得出,是以间日而作也。\\
帝曰:善。其作日晏与其日早者,何气使然?\\
岐伯曰:邪气客于风府,循膂而下,卫气一日一夜大会于风府,其明日日下一节,故其作也晏,此先客于脊背也。每至于风府,则腠理开,腠理开则邪气入,邪气入则病作,以此日作稍益晏也。其出于风府,日下一节,二十五日下至骶骨,二十六日入于脊内,注于伏膂之脉;其气上行,九日出于缺盆之中,其气日高,故作日益早也。其间日发者,由邪气内薄于五藏,横连募原也。其道远,其气深,其行迟,不能与卫气俱行,不得皆出,故间日乃作也。\\
帝曰:夫子言卫气每至于风府,腠理乃发,发则邪气入,入则病作。今卫气日下一节,其气之发也,不当风府,其日作者,奈何?\\
岐伯曰:此邪气客于头项,循膂而下者也,故虚实不同,邪中异所,则不得当其风府也。故邪中于头项者,气至头项而病;中于背者,气至背而病;中于腰脊者,气至腰脊而病;中于手足者,气至手足而病。卫气之所在,与邪气相合,则病作。故风无常府,卫气之所应,必开其腠理,邪气之所合,则其府也。\\
帝曰:善!夫风之与疟也,相似同类,而风独常在,疟得有时而休者,何也?\\
岐伯曰:风气留其处,故常在,疟气随经络,沉以内薄,故卫气应乃作。\\
帝曰:疟先寒而后热者,何也?\\
岐伯曰:夏伤于大暑,其汗大出,腠理开发,因遇夏气凄沧之水寒,藏于腠理皮肤之中,秋伤于风,则病成矣。夫寒者,阴气也;风者,阳气也。先伤于寒而后伤于风,故先寒而后热也,病以时作,名曰寒疟。\\
帝曰:先热而后寒者,何也?\\
岐伯曰:此先伤于风,而后伤于寒,故先热而后寒也,亦以时作,名曰温疟。其但热而不寒者,阴气先绝,阳气独发,则少气烦冤,手足热而欲呕,名曰瘅疟。\\
帝曰:夫经言有余者泻之,不足者补之。今热为有余,寒为不足。夫疟者之寒,汤火不能温也;及其热,冰水不能寒也,此皆有余不足之类。当此之时,良工不能止,必须其自衰乃刺之,其故何也?愿闻其说。\\
岐伯曰:经言无刺熇熇之热,无刺浑浑之脉,无刺漉漉之汗,故为其病逆,未可治也。夫疟之始发也,阳气并于阴,当是之时,阳虚而阴盛,外无气,故先寒栗也;阴气逆极,则复出之阳,阳与阴复并于外,则阴虚而阳实,故先热而渴。夫疟气者,并于阳则阳胜,并于阴则阴胜,阴胜则寒,阳胜则热。疟者,风寒之气不常也,病极则复,至病之发也,如火之热,如风雨不可当也。故经言曰:方其盛时必毁,因其衰也,事必大昌,此之谓也。夫疟之未发也,阴未并阳,阳未并阴,因而调之,真气得安,邪气乃亡。故工不能治其已发,为其气逆也。\\
帝曰:善。攻之奈何?早晏何如?\\
岐伯曰:疟之且发也,阴阳之且移也,必从四末始也。阳已伤,阴从之,故先其时,坚束其处,令邪气不得入,阴气不得出。审候见之,在孙络盛坚而血者皆取之,此真往而未得并者也。\\
帝曰:疟不发,其应何如?\\
岐伯曰:疟气者,必更盛更虚,当气之所在也,病在阳,则热而脉躁;在阴,则寒而脉静;极则阴阳俱衰,卫气相离,故病得休,卫气集,则复病也。\\
帝曰:时有间二日或至数日发,或渴或不渴,其故何也?\\
岐伯曰:其间日者,邪气与卫气客于六府,而有时相失,不能相得,故休数日乃作也。疟者,阴阳更胜也,或甚或不甚,故或渴或不渴。\\
帝曰:论言:“夏伤于暑,秋必病疟。”今疟不必应者,何也?\\
岐伯曰:此应四时者也。其病异形者,反四时也。其以秋病者寒甚,以冬病者寒不甚,以春病者恶风,以夏病者多汗。\\
帝曰:夫病温疟与寒疟而皆安舍,舍于何脏?\\
岐伯曰:温疟者,得之冬中于风,寒气藏于骨髓之中,至春则阳气大发,邪气不能自出,因遇大暑,脑髓烁,肌肉消,腠理发泄,或有所用力,邪气与汗皆出。此病藏于肾,其气先从内出之于外也。如是者,阴虚而阳盛,阳盛则热矣,衰则气复反入,入则阳虚,阳虚则寒矣,故先热而后寒,名曰温疟。\\
帝曰:瘅疟何如?\\
岐伯曰:瘅疟者,肺素有热,气盛于身,厥逆上冲,中气实而不外泄,因有所用力,腠理开,风寒舍于皮肤之内、分肉之间而发,发则阳气盛,阳气盛而不衰则病矣,其气不及于阴,故但热而不寒。气内藏于心,而外舍于分肉之间,令人消烁脱肉,故命曰瘅疟。\\
帝曰:善。\\
黄帝问:疟疾都是感受风邪而发,它的休作有一定时间,这是什么道理?\\
岐伯回答说:疟疾开始发作的时候,先使毫毛竖立,继而四体欲得引伸,呵欠连连,乃至寒冷发抖,下颏鼓动,腰脊疼痛,寒冷过去,接着全身发热,头痛欲裂,口渴喜欢冷饮。\\
黄帝说:是什么邪气引起的?希望听听其中的道理。\\
岐伯说:这是阴阳上下相争,虚实交替而作,阴阳相互移易转化所致。阳气并入于阴分,使阴气实而阳气虚,阳明经气虚,就寒冷发抖乃至两颔鼓动;太阳经气虚,就腰背头项疼痛;三阳经气都虚,则阴气胜,阴气胜则骨节寒冷疼痛;寒从内生,所以内外都感觉寒冷;阳气盛则发生外热,阴气盛则发生内热,外内都发热,就气喘口渴,所以喜欢冷饮。这都是由于夏天伤于暑气,热气过盛,潜藏在皮肤之内,肠胃之外,也即是邪气居留在荣气的处所。由于暑热内伏,使人汗孔疏松,腠理开泄,一遇秋凉之气,汗出感受风邪,以及洗澡受凉,风寒之气停留在皮肤之内,与卫气相合。卫气白天行于阳分,夜里行于阴分,邪气也随之循行于阳分时则外出,循行于阴分时则内搏,阴阳内外相搏,所以每日发作。\\
黄帝问:疟疾有隔日发作的,为什么?\\
岐伯说:这是因为邪气滞留较深,向内迫近于阴分,使阳气独行于外,而阴分之邪滞留于里,阴邪与阳气相争而不能即出,所以隔日发作。\\
黄帝说:讲得好。疟疾发作的时间,有逐日推后,或逐日提前的,是什么邪气使然?\\
岐伯说:邪气从风府侵入后,循脊骨下移,卫气是一昼夜大会于风府,而邪气却每日向下移行一节,所以其发作时间也就一天比一天迟,这是由于邪气先侵袭脊骨的缘故。卫气聚会于风府时,则腠理开张,腠理开张则邪气侵入,邪气侵入则发病,因邪气日下一节,所以发病时间就日益推迟了。邪气侵袭风府,逐日下移一节而经二十五日,邪气下行至骶骨,二十六日进入脊内,流注于伏冲脉;邪气再沿冲脉上行,至九日上至于天突穴,因为邪气一天天上升,所以发病的时间也就一天比一天早。隔日发病的,是因为邪气内迫五脏,横连膜原。它所运行的道路较远,邪气深藏,循行迟缓,不能和卫气并行,并与卫气同时外出,所以隔天发作一次。\\
黄帝说:夫子说卫气每运行至风府时,腠理开张,腠理开张则邪气乘机而入,邪气入则病发作。现在又说卫气与邪气相遇的部位每日下行一节,那么发病时,邪气就不一定正好在风府,而每日发作,是什么道理?\\
岐伯说:这是指邪气侵入头项,沿着脊骨而下者说的,但人体虚实不同,而邪气侵犯不同部位,所以邪气不一定都在风府穴处。邪中头项的,卫气循行至头项而病发。邪中于背部的,卫气循行至背部而病发;邪中于腰脊的,卫气循行至腰脊而病发;邪中于手足的,卫气循行至手足而病发。凡卫气循行之处,与邪气相合,病就发作。所以风邪侵袭人体没有固定部位,只要卫气与之相应,腠理开张,邪气得以凑合,就是邪气袭入的地方,就是风府。\\
黄帝说:讲得好!风病和疟疾相似,属同一类,风病持续常在,而疟疾却发作有时,为什么呢?\\
岐伯说:风邪为病滞留在所中之处,所以症状持续常在,疟病则是随着经络循行,深入体内,必须与卫气相遇,病才发作。\\
黄帝问:疟疾发作,先寒而后热的,是什么道理?\\
岐伯说:夏天感受了大暑之气,而汗大出,腠理开泄,再遇着寒凉水湿之气,就潜藏在腠理皮肤之中,到秋天又伤于风邪,就成为疟疾了。水寒,是阴气;风邪,是阳气。先伤于水寒之气,后伤于风邪,所以先寒而后热,疟病按时发作,这名叫寒疟。\\
黄帝问:有一种先热而后寒的,是什么道理?\\
岐伯说:这是先伤于风邪,后伤于水寒之气,所以先热而后寒,也是按时发作,名叫温疟。还有一种只发热而不恶寒的,是因为阴气先亏损,因此阳气独旺,发作时,少气烦闷,手足发热,要想呕吐,这名叫瘅疟。\\
黄帝说:医经上说有余的用泻法,不足的用补法。今发热是有余,发冷是不足。而疟疾的寒冷,用热水或烤火,也不能温暖;等到发热,即使用冰水,也不能凉爽,这些都是有余不足之类。在这个时候,良医也无法制止,必须待其病势自行衰退,才可以刺治,这是什么道理?希望听听。\\
岐伯说:医经上说过,高热时不能针刺,脉象纷乱时不能针刺,大汗不止时不能针刺,因为这时邪气正盛,不可立即治疗。疟疾刚发作,阳气并于阴分,此时是阳虚而阴盛,外表阳气虚,所以先寒冷发抖;等到阴气逆乱已极,势必复出于阳分,于是阳气与阴气相并于外,此时阴分虚而阳分盛,所以先热而口渴。因为疟疾并于阳分,则阳气胜;并于阴分,则阴气胜,阴气胜则发寒,阳气胜则发热。疟疾感受的风寒之气变化无常,热到极点,则阴寒之气至;寒到极点,则阳热之气至,发作时,其热如烈火,其寒如狂风暴雨不可阻挡。所以医经上说:当邪气盛极之时,不可攻邪,攻之则正气也必然受伤,应该乘邪气衰退的时候而攻之,必然获得成功,说的就是这个意思。因此在疟疾未发,阴气尚未并于阳分,阳气尚未并于阴分的时候,进行调治,则正气能够安宁,而邪气可以消灭。所以医生不在疟疾发作时治疗,因为此时正邪交争,邪气正盛。\\
黄帝说:讲得好!怎样攻治呢?时间的早晚,如何掌握?\\
岐伯说:疟疾将发,阴阳也将要相移,它必从四肢开始。阳气已被邪伤,则阴分也必将受到邪气的影响,所以只有在未发病之先,以绳索牢缚其四肢末端,使邪气不得入,阴气不得出,两者不能相移。牢缚之后,详细审察络脉,发现孙络充实而淤血的,都要刺出其血,这是当真气尚未与邪气相并之前的治法。\\
黄帝问:疟疾在未发作的时候,情况怎样?\\
岐伯说:疟气必然使阴阳虚实更替,当邪气在阳分之时,则发热而脉象躁急;邪气在阴分之时,则发冷而脉象安静;病到极期,则阴阳二气都已虚衰,卫气和邪气互相分离,病就暂时停止,若卫气和邪气再相遇,则病情发作。\\
黄帝问:有的疟疾间隔二日,或隔数日才发作一次,发作时有的口渴,有的不渴,是什么缘故?\\
岐伯说:隔日发作的,是因为邪气与卫气相会于风府的时间不一致,有时不能相遇,不得皆出,所以停止几天才发作。疟疾发病,是阴阳更替相胜,阳胜于阴则热甚,阴胜于阳则寒甚,所以有的口渴,有的不渴。\\
黄帝问:医经上说:“夏伤于暑,秋必病疟。”而有的疟疾,并不如此,是什么道理?\\
岐伯说:“夏伤于暑,秋必病疟。”这是指和四时发病规律相应的而言。有些疟疾症状不同,与四时发病规律相反。如在秋天发病的,却寒冷较重;在冬天发病的,却寒冷较轻;在春天发病的,多恶风;在夏天发病的,汗出得很多。\\
黄帝问:患温疟和寒疟的,邪气滞留在哪里?停留在哪一脏?\\
岐伯说:温疟是冬天感受风寒,寒气潜藏在骨髓之中,到春天阳气生发之时,邪气仍不能自行外出,因遇到暑热炽盛,使人脑髓消烁,肌肉消瘦,腠理发泄,或劳力过甚,邪气与汗一齐外出。这种病邪原来潜伏在肾,所以邪气从内而出于外。这种病,阴气虚弱,而阳气偏盛,阳盛就发热,热极而衰,则邪气又重新入阴,邪入阴则阳气又虚,阳气虚便出现寒冷,所以这种病是先热而后寒,名叫温疟。\\
黄帝问:瘅疟的情况怎样?\\
岐伯说:瘅疟是因为肺脏素有热气,肺气壅盛,气逆而上冲,以致胸中气实,不能发泄,再加上劳力之后,腠理开泄,风寒之邪侵入皮肤之内、肌肉之间而发病,发病则阳气偏盛,阳气盛而不见衰减,于是就发病了。邪气没有进入阴分,所以只热不寒,病邪内伏于心脏,而外出则滞留在肌肉之间,能使人肌肉瘦削,所以名叫瘅疟。\\
黄帝说:讲得好!\\
刺疟篇第三十六\\
足太阳之疟,令人腰痛头重,寒从背起,先寒后热,熇熇暍暍然,热止汗出,难已,刺郄中出血。\\
足少阳之疟,令人身体解汃,寒不甚,热不甚,恶见人,见人心惕惕然,热多汗出甚,刺足少阳。\\
足阳明之疟,令人先寒,洒淅洒淅,寒甚久乃热,热去汗出,喜见日月光火气,乃快然,刺足阳明跗上。\\
足太阴之疟,令人不乐,好太息,不嗜食,多寒热汗出,病至则善呕,呕已乃衰,即取之。\\
足少阴之疟,令人呕吐甚,多寒热,热多寒少,欲闭户牖而处,其病难已。\\
足厥阴之疟,令人腰痛,少腹满,小便不利,如癃状,非癃也,数便,意恐惧,气不足,腹中悒悒,刺足厥阴。\\
肺疟者,令人心寒,寒甚热,热间善惊,如有所见者,刺手太阴阳明。\\
心疟者,令人烦心甚,欲得清水,反寒多,不甚热,刺手少阴。\\
肝疟者,令人色苍苍然,太息,其状若死者,刺足厥阴见血。\\
脾疟者,令人寒,腹中痛,热则肠中鸣,鸣已汗出,刺足太阴。\\
肾疟者,令人洒洒然,腰脊痛,不能宛转,大便难,目眴眴然,手足寒,刺足太阳少阴。\\
胃疟者,令人且病也,善饥而不能食,食而支满腹大,刺足阳明太阴横脉出血。\\
疟发,身方热,刺跗上动脉,开其空,出其血,立寒;疟方欲寒,刺手阳明太阴,足阳明太阴。\\
疟脉满大急,刺背俞,用中针,傍五胠俞各一,适肥瘦,出其血也。\\
疟脉小实急,灸胫少阴,刺指井。\\
疟脉满大急,刺背俞,用五胠俞、背俞各一,适行至于血也。\\
疟脉缓大虚,便宜用药,不宜用针。凡治疟,先发如食顷乃可以治,过之则失时也。诸疟而脉不见,刺十指间出血,血去必已;先视身之赤如小豆者,尽取之。十二疟者,其发各不同时,察其病形,以知其何脉之病也。先其发时如食顷而刺之,一刺则衰,二刺则知,三刺则已;不已,刺舌下两脉出血;不已,刺郄中盛经出血,又刺项已下侠脊者,必已。舌下两脉者,廉泉也。\\
刺疟者,必先问其病之所先发者,先刺之。先头痛及重者,先刺头上及两额两眉间出血。先项背痛者,先刺之。先腰脊痛者,先刺郄中出血。先手臂痛者,先刺手少阴阳明十指间。先足胫痠痛者,先刺足阳明十指间出血。风疟,疟发则汗出恶风,刺三阳经背俞之血者。气痠痛甚,按之不可,名曰胕髓病,以镵针针绝骨出血,立已。身体小痛,刺至阴。诸阴之井无出血,间日一刺。疟不渴,间日而作,刺足太阳;渴而间日作,刺足少阳;温疟汗不出,为五十九刺。\\
足太阳经的疟疾,使人腰痛头重,寒冷从脊背发起,先寒后热,热势很盛,热止汗出,不易痊愈,刺取委中穴出血。\\
足少阳经的疟疾,使人身体倦怠无力,恶寒发热都不重,讨厌见人,看见人心中就有恐惧感,发热时间比较长,汗出也多,刺取足少阳经。\\
足阳明经的疟疾,使人先感觉发冷,逐渐恶寒加剧,很久才发热,热退去便汗出,这种病人,喜欢日月亮光及火焰,看见它感到愉快,刺取足阳明经足背上的冲阳穴。\\
足太阴经的疟疾,使人郁郁不乐,时常叹息,不想吃东西,多寒多热,汗出,病发时易呕吐,吐后病势衰减,取治足太阴经。\\
足少阴经的疟疾,使人剧烈呕吐,寒热多发,热多寒少,常常喜欢紧闭门窗而居,这种病不易痊愈。\\
足厥阴经的疟疾,使人腰痛,小腹胀满,小便不利,似乎是癃闭病,实际不是癃闭,只是小便频数不爽,心中恐惧,内气不足,腹中郁滞不畅,刺足厥阴经。\\
肺疟,使人心里发冷,冷极则发热,热时易发惊惧,好像见到了什么可怕的事物,刺取手太阴、手阳明两经。\\
心疟,使人心烦得厉害,想喝凉水,反觉得寒多,不太热,刺取手少阴经。\\
肝疟,使人面色苍青,时常太息,形状如同死人,刺取足厥阴经出血。\\
脾疟,使人发冷,腹中痛,发热则肠中鸣响,肠鸣停止而汗出,刺取足太阴经。\\
肾疟,使人洒淅寒冷,腰脊疼痛,难以转侧,大便困难,目视眩动不明,手足冷,刺取足太阳、足少阴两经。\\
胃疟,使人发生黄疸,易饥饿,但又不能进食,进食就感到脘腹胀满膨大,取治足阳明、足太阴两经横行的络脉,刺出其血。\\
疟疾发作,身体正热时,刺足背上的动脉,刺开孔穴,出其血,马上热退身凉;疟疾刚要发冷时,针刺手阳明、太阴和足阳明、太阴的腧穴。\\
疟病脉象满大而急,刺背部的腧穴,用中等针,靠近五胠俞各取一穴,根据病人的胖瘦,确定出血量。\\
疟病脉象小实而急的,灸足胫部的少阴经穴,刺足趾端的井穴。\\
疟病脉象满大而急,刺背部腧穴,取五胠俞、背俞各一穴,根据病人情况,刺之出血。\\
疟病脉象缓大而虚的,就应该用药治疗,不宜针刺。大凡治疗疟疾,应在病发前约一顿饭的时候,予以治疗,过了这个时候,就会失去时机。凡疟病脉象沉伏不见的,急刺十指间出血,血出病必愈;若先见皮肤上出现像赤小豆样的出血点,应都用针刺去。上述十二种疟疾,其发作时间各不相同,观察病人的症状,从而了解病属哪一经脉。在发作前约一顿饭的时候给以针刺,刺一次病势衰减,刺两次就会显著好转,刺三次病即痊愈;如不痊愈,刺舌下两脉出血;再不痊愈,取委中血盛的经络,刺出其血,并刺项部以下挟脊两旁的经穴,一定会痊愈。舌下两脉,就是廉泉穴。\\
刺治疟疾,必先问明病人发作时最先感觉的部位,给以先刺。先发头痛头重的,就先刺头上及两额、两眉间出血。先发颈项脊背痛的,就先刺颈项和背部。先发腰脊痛的,就先刺委中出血。先发手臂痛的,就先刺手少阴、手阳明十指间的孔穴。先发足胫痠痛的,就先刺足阳明十趾间出血。风疟,发作时汗出怕风,可刺三阳经背部的腧穴出血。小腿痠痛剧烈而按摩无效的,名叫胕髓病,可用镵针刺绝骨穴出血,其痛立止。身体稍感疼痛,刺至阴穴。凡刺诸阴经的井穴,皆不可出血,并应隔日一刺。疟疾口不渴而隔日发作的,刺足太阳经;口渴而隔日发作的,刺足少阳经;温疟而汗不出的,用五十九刺的方法。\\
气厥论篇第三十七\\
黄帝问曰:五脏六腑,寒热相移者,何?\\
岐伯曰:肾移寒于脾,痈肿少气。脾移寒于肝,痈肿筋挛。肝移寒于心,狂隔中。心移寒于肺,肺消,肺消者,饮一溲二,死不治。肺移寒于肾,为涌水,涌水者,按腹不坚,水气客于大肠,疾行则鸣濯濯,如囊裹浆,水之病也。脾移热于肝,则为惊衄。肝移热于心,则死。心移热于肺,传为鬲消。肺移热于肾,传为柔痓。肾移热于脾,传为肠澼,死不可治。胞移热于膀胱,则癃溺血。膀胱移热于小肠,鬲肠不便,上为口麋。小肠移热于大肠,为虙瘕,为沉。大肠移热于胃,善食而瘦,谓之食亦。胃移热于胆,亦曰食亦。胆移热于脑,则辛伔鼻渊,鼻渊者,浊涕下不止也,传为衄蔑瞑目,故得之气厥也。\\
黄帝问:五脏六腑的寒热互相转移的情况,是怎样的?\\
岐伯说:肾移寒于脾,则病浮肿和少气。脾移寒于肝,则病痈肿和筋挛。肝移寒于心,则病发狂和胸中隔塞。心移寒于肺,则为肺消,肺消的症状是饮一份水,排两份小便,是无法治疗的死证。肺移寒于肾,则为涌水,涌水的症状是腹部按之不太坚硬,但因水气滞留大肠,所以快走时肠中濯濯鸣响,如皮囊装水一样,这是水气之病。脾移热于肝,则病惊骇和鼻衄。肝移热于心,则是死证。心移热于肺,日久则为膈消。肺移热于肾,日久则为柔痓。肾移热于脾,日久传变痢疾,是无法治疗的死证。胞移热于膀胱,则病小便不利和尿血。膀胱移热于小肠,使肠道隔塞,大便不通,热气上行,以致口舌糜烂。小肠移热于大肠,则为伏瘕或痔疮。大肠移热于胃,则消谷善食而使人消瘦无力,病为食刓。胃移热于胆,也病食刓。胆移热于脑,则鼻梁内感觉辛辣,成为鼻渊,鼻渊的症状,常流浊涕不止,日久可致鼻中流血,两目不明。以上诸证,皆是寒热之气厥逆引起的。\\
咳论篇第三十八\\
黄帝问曰:肺之令人咳,何也?\\
岐伯对曰:五脏六腑皆令人咳,非独肺也。\\
帝曰:愿闻其状。\\
岐伯曰:皮毛者,肺之合也。皮毛先受邪气,邪气以从其合也。其寒饮食入胃,从肺脉上至于肺则肺寒,肺寒则外内合邪,因而客之,则为肺咳。五脏各以其时受病,非其时,各传以与之。人与天地相参,故五脏各以治时感于寒则受病。微则为咳,甚者为泄为痛。乘秋则肺先受邪,乘春则肝先受之,乘夏则心先受之,乘至阴则脾先受之,乘冬则肾先受之。\\
帝曰:何以异之?\\
岐伯曰:肺咳之状,咳而喘,息有音,甚则唾血。心咳之状,咳则心痛,喉中介介如梗状,甚则咽肿喉痹。肝咳之状,咳则两胁下痛,甚则不可以转,转则两胠下满。脾咳之状,咳则右胁下痛,阴阴引肩背,甚则不可以动,动则咳剧。肾咳之状,咳则腰背相引而痛,甚则咳涎。\\
帝曰:六腑之咳奈何?安所受病?\\
岐伯曰:五脏之久咳,乃移于六腑。脾咳不已,则胃受之;胃咳之状,咳而呕,呕甚则长虫出。肝咳不已,则胆受之;胆咳之状,咳呕胆汁。肺咳不已,则大肠受之;大肠咳状,咳而遗矢。心咳不已,则小肠受之;小肠咳状,咳而失气,气与咳俱失。肾咳不已,则膀胱受之;膀胱咳状,咳而遗溺。久咳不已,则三焦受之,三焦咳状,咳而腹满,不欲食饮。此皆聚于胃,关于肺,使人多涕唾而面浮肿气逆也。\\
帝曰:治之奈何?\\
岐伯曰:治脏者,治其俞;治腑者,治其合;浮肿者,治其经。\\
帝曰:善。\\
黄帝问道:肺脏能使人咳嗽,为什么?\\
岐伯回答说:五脏六腑都能使人咳嗽,不只是肺脏能使人咳嗽。\\
黄帝道:希望听听具体情况。\\
岐伯说:皮毛主表,和肺是相配合的。皮毛受了寒气,寒气就会侵入肺脏。假若喝了冷水或者吃了冷东西,寒气入胃,从肺脉上注于肺,肺也会因此受寒。这样,内外的寒邪互相结合,留止在肺脏,就成为肺咳。至于五脏六腑的咳嗽,是五脏各在所主的时令受病,并不是肺在它所主之时受病,各自传给它的。人与天地相参应,五脏各在它所主的时令中受了寒邪,便能得病。若轻微的,就是咳嗽;严重的,寒气入里,就成为泻泄、腹痛。一般情况是在秋天肺先受邪,在春天肝先受邪,在夏天心先受邪,在季夏脾先受邪,在冬天肾先受邪。\\
黄帝问道:怎样来区别这些咳嗽呢?\\
岐伯说:肺咳的症状,咳嗽的时候,喘息有声音,严重的,还会唾血。心咳的症状,咳嗽的时候,感到心痛,喉中像有东西堵塞,严重的,咽喉肿痛闭塞。肝咳的症状,咳嗽的时候,两胁疼痛,严重的,不能行走,如果行走,两脚就会浮肿。脾咳的症状,咳嗽的时候,右胁痛,隐隐然痛牵肩背,严重的,不能活动,一活动,咳嗽就加重。肾咳的症状,咳嗽的时候,腰背互相牵扯作痛,严重的,就要咳出粘沫来。\\
黄帝问道:六腑咳嗽的症状怎样?又是怎么得病的呢?\\
岐伯说:五脏咳嗽,日久不愈,就要转移到六腑。脾咳不好,胃就要受病;胃咳的症状,咳而呕吐,厉害的时候,可呕出蛔虫。肝咳不好,胆就要受病;胆咳的症状,咳嗽起来,可吐出胆汁。肺咳不好,大肠就要受病;大肠咳的症状,咳嗽的时候,大便失禁。心咳不好,小肠就要受病;小肠咳的症状,咳嗽时要放屁,经常是咳嗽和放屁并作。肾咳不好,膀胱就要受病;膀胱咳的症状,咳嗽的时候,小便失禁。以上各种咳嗽,如果经久不愈,那么三焦就要受病;三焦咳的症状,是咳嗽的时候,肚肠胀满,不想吃东西。这些咳嗽,无论是哪一脏腑的病变,其寒邪都是聚合于胃,联属于肺,使人多吐稠痰,面目浮肿,气逆。\\
黄帝问道:治疗的方法怎样?\\
岐伯说:治疗五脏的咳嗽,要取腧穴;治疗六腑的咳嗽,要取合穴;凡是由于咳嗽而致浮肿的,要取经穴。\\
黄帝说:好。\\
卷十一\\
举痛论篇第三十九\\
黄帝问曰:余闻善言天者,必有验于人;善言古者,必有合于今;善言人者,必有厌于己。如此,则道不惑而要数极,所谓明也。今余问于夫子,令言而可知,视而可见,扪而可得,令验于己,而发蒙解惑,可得而闻乎?\\
岐伯再拜稽首对曰:何道之问也?\\
帝曰:愿闻人之五脏卒痛,何气使然?\\
岐伯对曰:经脉流行不止,环周不休。寒气入经而稽迟,泣而不行。客于脉外则血少,客于脉中则气不通,故卒然而痛。\\
帝曰:其痛或卒然而止者,或痛甚不休者,或痛甚不可按者,或按之而痛止者,或按之无益者,或喘动应手者,或心与背相引而痛者,或胁肋与少腹相引而痛者,或腹痛引阴股者,或痛宿昔而成积者,或卒然痛死不知人,有少间复生者,或痛而呕者,或腹痛而后泄者,或痛而闭不通者。凡此诸痛,各不同形,别之奈何?\\
岐伯曰:寒气客于脉外则脉寒,脉寒则缩踡,缩踡则脉绌急,绌急则外引小络,故卒然而痛。得炅则痛立止;因重中于寒,则痛久矣。\\
寒气客于经脉之中,与炅气相薄则脉满,满则痛而不可按也。寒气稽留,炅气从上,则脉充大而血气乱,故痛甚不可按也。\\
寒气客于肠胃之间,膜原之下,而不得散,小络急引故痛,按之则血气散,故按之痛止。\\
寒气客于侠脊之脉则深,按之不能及,故按之无益也。\\
寒气客于冲脉,冲脉起于关元,随腹直上。寒气客则脉不通,脉不通则气因之,故喘动应手矣。\\
寒气客于背俞之脉则脉泣,脉泣则血虚,血虚则痛。其俞注于心,故相引而痛,按之则热气至,热气至则痛止矣。\\
寒气客于厥阴之脉,厥阴之脉者,络阴器系于肝。寒气客于脉中,则血泣脉急,故胁肋与少腹相引痛矣。\\
厥气客于阴股,寒气上及少腹,血泣在下相引,故腹痛引阴股。\\
寒气客于小肠膜原之间,络血之中,血泣不得注于大经,血气稽留不得行,故宿昔而成积矣。\\
寒气客于五脏,厥逆上泄,阴气竭,阳气未入,故卒然痛死不知人,气复反则生矣。\\
寒气客于肠胃,厥逆上出,故痛而呕也。\\
寒气客于小肠,小肠不得成聚,故后泄腹痛矣。\\
热气留于小肠,肠中痛,瘅热焦渴,则坚干不得出,故痛而闭不通矣。\\
帝曰:所谓言而可知者也,视而可见奈何?\\
岐伯曰:五脏六腑固尽有部,视其五色,黄赤为热,白为寒,青黑为痛,此所谓视而可见者也。\\
帝曰:扪而可得,奈何?\\
岐伯曰:视其主病之脉,坚而血及陷下者,皆可扪而得也。帝曰:善。余知百病生于气也。怒则气上,喜则气缓,悲则气消,恐则气下,寒则气收,炅则气泄,惊则气乱,劳则气耗,思则气结。九气不同,何病之生?\\
岐伯曰:怒则气逆,甚则呕血及飧泄,故气上矣。喜则气和志达,荣卫通利,故气缓矣。悲则心系急,肺布叶举而上焦不通,荣卫不散,热气在中,故气消矣。恐则精却,却则上焦闭,闭则气还,还则下焦胀,故气不行矣。寒则腠理闭,气不行,故气收矣。炅则腠理开,荣卫通,汗大泄,故气泄。惊则心无所倚,神无所归,虑无所定,故气乱矣。劳则喘息汗出,外内皆越,故气耗矣。思则心有所存,神有所归,正气留而不行,故气结矣。\\
黄帝问道:我听说善于谈论天道的,必能从人事上验证天道;善于谈论往古的,必能把过去与现在结合起来;善于谈论他人的,必能结合自己。这样,对于医学道理,才可无所疑惑,而得其真理,才算真正明白。现在我要问您的是那言而可知,视而可见,扪而可得的诊法,使自己有所体验,启发蒙昧,解除疑惑,能够听听吗?\\
岐伯回答说:你要问哪些道理?\\
黄帝问:我希望听听五脏突然作痛,是什么邪气造成的呢?\\
岐伯回答说:人身经脉中的气血周流全身,循环不息。寒气侵入经脉,经血就会留滞,凝涩而不畅通。假如寒邪侵袭在经脉之外,血液就会减少;若侵入脉中,则脉气不通,就会突然作痛。\\
黄帝说:有的疼痛忽然自止;有的剧痛而不能止;有的痛得厉害,不可揉按;有的揉按痛就可止住;有的虽然揉按,也没有效果;有的痛处跳动应手;有的痛时心与背牵引作痛;有的胁肋和小腹牵引作痛;有的腹痛牵引大腿内侧;有的疼痛日久不愈而成小肠气;有的忽然痛得昏死不知人事,过一会儿才苏醒;有的疼痛而且呕吐;有的腹痛而且泄泻;有的疼痛而且胸闷不舒。所有这些疼痛,表现各不相同,怎样区别呢?\\
岐伯说:寒气侵犯于脉外,则脉受寒,脉受寒就收缩,收缩则脉屈曲拘急不舒,屈曲拘急不舒因而牵引在外的细小脉络,就会忽然发生疼痛。但只要得热,疼痛就会停止。因而再感受寒气,疼痛就会很久不好了。\\
寒气侵犯到经脉之中,与经脉中的热气相互搏结,就会经脉满盛,满盛则实,所以痛得不能按。寒气停留,热气跟随而来,冷热相搏,则经脉充溢满大,气血混乱,就会痛得厉害不能触按。\\
寒气侵入肠胃之间,膜原之下,不能散开,细小的脉络因之绷急牵引而痛,以手揉按,则血气可以散行,所以按之疼痛就停止。\\
寒气侵入了督脉,因病位较深,即使重按也不能达到病所,所以按之也无作用。\\
寒气侵入到冲脉,冲脉从关元穴起,循腹上行。所以冲脉的血气不得流通,那么邪气就聚集此处而不通畅,不通畅所以触诊腹部就会应手而痛。\\
寒气侵入到背腧脉,则血脉流行凝涩,血脉凝涩则血虚,血虚则疼痛。因为背腧上注于心,所以互相牵引作痛,用手按之则热气积聚,热气到达病所,疼痛就可停止。\\
寒气侵入到厥阴脉,厥阴脉连络阴器,并系于肝。寒气侵入脉中,血涩不得流畅,脉道紧急,所以胁肋与少腹互相牵引而作痛。\\
寒气逆行侵入到阴股,气血不和累及少腹,阴股之血凝涩,在下相引,所以腹痛连于阴股。\\
寒气侵入到小肠膜原之间,络血之中,血脉凝涩,不能贯注到大经脉里去,因而血气停留,不得畅通,这样日久就成小肠气了。\\
寒气侵入到五脏,则厥逆之气上壅,阴气竭绝,阳气郁遏不通,所以忽然痛死,不知人事;如果阳气恢复,仍然可以苏醒。\\
寒气侵入肠胃,厥逆之气上行,所以发生腹痛并且呕吐。\\
寒气侵入到小肠,小肠失其受盛作用,水谷不得停留,所以就后泄而腹痛了。\\
热气蓄留于小肠,肠中要发生疼痛,并且发热干渴,大便坚硬不得出,所以就疼痛而大便不通了。\\
黄帝说:以上病情,是从问诊中可以了解的。那么通过望诊可以了解病情又如何?\\
岐伯说:五脏六腑,在面部各有自己所属的部位,观察面部的五色,黄色和赤色为热,白色为寒,青色和黑色为痛,这就是所谓的视而可见的。\\
黄帝说:通过触诊了解病情是怎样的?\\
岐伯说:要看他主病的脉象。坚实的、有淤血以及经脉陷下,都可用手触切而得知。\\
黄帝说:说得好!我听说各种疾病是由于气的逆乱而发生的。如暴怒则气上逆,大喜则气涣散,悲哀则气消散,恐惧则气下陷,遇寒则气收聚,受热则气外泄,过惊则气混乱,过劳则气耗损,思虑则气郁结。这九种气的变化各不相同,都能导致什么病呢?\\
岐伯说:大怒则气上逆,严重的,可以引起呕血和飧泄,所以说是“气上逆”。高兴气就和顺,情志畅达,营卫之气通利,所以说是“气缓”。悲哀过甚则心系绷急,肺叶胀起,上焦不通,营卫之气不得布散,热气在内不散,所以说是“气消”。恐惧就会使精气衰退,精气下衰就要使上焦闭塞,上焦不通,还于下焦,气郁下焦,就会胀满,所以说是“气下”。寒冷之气,能使腠理密闭,营卫之气不得流行,所以说是“气收”。热则腠理开泄,营卫之气大通,汗大出,所以说是“气泄”。过惊则心无依靠,神气不能归心,心中疑虑不定,所以说是“气乱”。过劳则喘息汗出,里外都发越消耗,所以说是“气耗”。忧思过多那么心气凝滞,精神偏滞,不能畅行周身,气就会留滞而不能运行,所以说是“气结”。\\
腹中论篇第四十\\
黄帝问曰:有病心腹满,旦食则不能暮食,此为何病?\\
岐伯对曰:名为鼓胀。\\
帝曰:治之奈何?\\
岐伯曰:治之以鸡矢醴。一剂知,二剂已。\\
帝曰:其时有复发者,何也?\\
岐伯曰:此饮食不节,故时有病也。虽然其病且已,时故当病,气聚于腹也。\\
帝曰:有病胸胁支满者,妨于食,病至则先闻腥臊臭,出清液,先唾血,四支清,目眩,时时前后血,病名为何?何以得之?\\
岐伯曰:病名血枯,此得之年少时,有所大脱血;若醉入房中,气竭肝伤,故月事衰少不来也。\\
帝曰:治之奈何?复以何术?\\
岐伯曰:以四乌仱骨一藘茹,二物并合之,丸以雀卵,大如小豆。以五丸为后饭,饮以鲍鱼汁,利胁中及伤肝也。\\
帝曰:病有少腹盛,上下左右皆有根,此为何病?可治不?\\
岐伯曰:病名曰伏梁。\\
帝曰:伏梁何因而得之?\\
岐伯曰:裹大脓血,居肠胃之外,不可治;治之,每切按之,致死。\\
帝曰:何以然?\\
岐伯曰:此下则因阴,必下脓血,上则迫胃脘,生鬲,侠胃脘内痈。此久病也,难治。居齐上为逆,居齐下为从,勿动亟夺。论在《刺法》中。\\
帝曰:人有身体髀股气皆肿,环齐而痛,是为何病?\\
岐伯曰:病名伏梁,此风根也。其气溢于大肠而著于肓,肓之原在齐下,故环齐而痛也,不可动之,动之为水溺涩之病。\\
帝曰:夫子数言热中、消中,不可服高梁芳草石药,石药发瘨,芳草发狂。夫热中、消中者,皆富贵人也,今禁高梁,是不合其心,禁芳草石药,是病不愈,愿闻其说。\\
岐伯曰:夫芳草之气美,石药之气悍,二者其气急疾坚劲,故非缓心和人,不可以服此二者。\\
帝曰:不可以服此二者,何以然?\\
岐伯曰:夫热气慓悍,药气亦然,二者相遇,恐内伤脾。脾者土也而恶木,服此药者,至甲乙日更论。\\
帝曰:善。有病膺肿、颈痛、胸满、腹胀,此为何病?何以得之?\\
岐伯曰:名厥逆。\\
帝曰:治之奈何?\\
岐伯曰:灸之则瘖,石之则狂。须其气并,乃可治也。\\
帝曰:何以然?\\
岐伯曰:阳气重上,有余于上,灸之则阳气入阴,入则瘖;石之则阳气虚,虚则狂。须其气并而治之,可使全也。\\
黄帝曰:善。何以知怀子之且生也?\\
岐伯曰:身有病而无邪脉也。\\
帝曰:病热而有所痛者,何也?\\
岐伯曰:病热者,阳脉也,以三阳之动也。人迎一盛少阳,二盛太阳,三盛阳明,入阴也。夫阳入于阴,故病在头与腹,乃尒胀而头痛也。\\
帝曰:善。\\
黄帝问:有人患心腹胀满,早上进食,晚上便不能再进食,这是什么病?\\
岐伯回答说:病名为鼓胀。\\
黄帝问:怎么治疗呢?\\
岐伯说:用鸡矢醴治疗,一剂就可见效,两剂就能治愈。\\
黄帝说:这种病有时会复发,是什么原因?\\
岐伯说:这是由于饮食没有节制,所以时时发病。这种病虽在表面上看要好了,但又饮食不节,病气就会再次聚集腹中而复发。\\
黄帝说:有人患胸胁支撑胀满,妨碍饮食,发作之前,先闻到腥臊气味,口泛清水,先见吐血,逐渐四肢寒冷,眼花,时常大小便出血;这是什么病?怎么得的?\\
岐伯说:病名为血枯。是由于年轻时曾大出血;或醉酒后肆行房事,使精气耗竭,肝脏损伤,所以月经量少,甚至停经。\\
黄帝说:怎么治疗呢?用什么方法使其恢复健康?\\
岐伯说:用四份乌贼骨,一份藘茹,二药混合,用麻雀卵和制成丸,如小豆大小。于饭前服五丸,用鲍鱼汤送服,以有益于胁肋和补益受伤的肝脏。\\
黄帝说:有一种病小腹盛满,上下左右四周按之有明显的根底,这是什么病?能否治疗?\\
岐伯说:病名叫伏梁。\\
黄帝说:因为什么引起的?\\
岐伯说:小腹肿包裹着大量脓血,停聚在肠胃之外,这种病不易治疗;诊治时每因切按过重而引起死亡。\\
黄帝说:为什么会这样呢?\\
岐伯说:这种病如果生在下腹部,则靠近阴部,重按必然使脓血从下部穿溃排出,向上则靠近胃脘,会使脓血穿出横膈,使胃脘生内痈。这是慢性病,难于治愈。病位在脐以上的属危重的逆证,在脐以下的属预后较好的顺证,总之不可按摩以求疾病立即消除。具体的论述记载在《刺法》中。\\
黄帝说:有的人髀、股、胫部位都肿,而且绕脐疼痛,这是什么病?\\
岐伯说:病名叫伏梁,是平素感受风寒之邪引起的。邪气充溢大肠,滞留附着于肓膜,肓的原穴位于脐下,所以绕脐疼痛。不可用攻下法,误用会发生小便涩滞的病变。\\
黄帝说:夫子多次说热中、消中的病人,不能吃肥甘厚味,也不能服用芳香的草药和矿石类药物,因为矿石类药物能使人发癫,芳香类草药会使人发狂。患热中、消中病的,多是富贵之人,现在不准吃肥甘厚味,这不合乎他们的心愿,禁忌芳香和金石药物,这病又不能治愈,希望听听其中的道理。\\
岐伯回答说:芳香草药之性多辛窜,金石药物之性多峻猛,这两类药物之气都急猛、刚劲,所以不是性情和缓的人,不能服用这两类药物。\\
黄帝问:不可以服用这两类药的原因是什么?\\
岐伯说:内热的性质慓悍,药物的性质也是这样,内热遇药热,恐怕要损伤脾气。脾属土而恶木乘,服用这类药物的病人,到肝木主令的甲日和乙日时,病情就更加严重了。\\
黄帝说:讲得对。有人患膺肿、颈痛、胸满、腹胀,这是什么病?怎么引起的?\\
岐伯说:病名叫厥逆。\\
黄帝问:怎样治疗?\\
岐伯说:用灸法则能引起失音,用针刺则能引起发狂。须待阴阳之气交合,才能治疗。\\
黄帝说:为什么这样呢?\\
岐伯说:阳气逆于上,上部阳有余,再用灸法,是以火济火,阳盛入阴则为失音;若用针刺,则阳气随刺而外泄致虚,阳气虚则发狂。必须待阴阳之气交合,而后治疗,才可以痊愈。\\
黄帝说:讲得对!怎样诊知妇女怀孕并且将要分娩呢?\\
岐伯说:身体不适,似乎有病,却切不到有病脉。\\
黄帝说:发热兼有疼痛,是什么原因?\\
岐伯说:发热病都见阳脉,因而三阳脉盛大而搏动较甚。人迎脉比寸口脉大一倍为病在少阳,大两倍为病在太阳,大三倍为病在阳明,传入三阴。病邪由阳入阴,则病在头与腹部,就会出现腹胀而头痛。\\
黄帝说:讲得对!\\
刺腰痛篇第四十一\\
足太阳脉,令人腰痛,引项脊尻背如重状,刺其郄中,太阳正经出血,春无见血。\\
少阳,令人腰痛,如以针刺其皮中,循循然不可以俯仰,不可以顾,刺少阳成骨之端出血,成骨在膝外廉之骨独起者,夏无见血。\\
阳明,令人腰痛,不可以顾,顾如有见者,善悲,刺阳明于气前三痏,上下和之出血,秋无见血。\\
足少阴,令人腰痛,痛引脊内廉,刺少阴于内踝上二痏,春无见血,出血太多,不可复也。\\
厥阴之脉,令人腰痛,腰中如张弓弩弦,刺厥阴之脉,在腨踵鱼腹之外,循之累累然,乃刺之,其病令人言默默然,不慧,刺之三痏。\\
解脉,令人腰痛,痛引肩,目丼丼然,时遗溲,刺解脉,在膝筋肉分间郄外廉之横脉,出血,血变而止。\\
解脉,令人腰痛如引带,常如折腰状,善恐,刺解脉,在郄中结络如黍米,刺之血射以黑,见赤血而已。\\
同阴之脉,令人腰痛,痛如小锤居其中,怫然肿,刺同阴之脉,在外踝上绝骨之端,为三痏。\\
阳维之脉,令人腰痛,痛上怫然肿,刺阳维之脉,脉与太阳合腨下间,去地一尺所。\\
衡络之脉,令人腰痛,不可以俯仰,仰则恐仆,得之举重伤腰,衡络绝,恶血归之,刺之在郄阳筋之间,上郄数寸,衡居为二痏出血。\\
会阴之脉,令人腰痛,痛上漯漯然汗出,汗干令人欲饮,饮已欲溲,刺直阳之脉上三痏,在\\
上郄下五寸横居,视其盛者出血。\\
飞阳之脉,令人腰痛,痛上拂拂然,甚则悲以恐,刺飞阳之脉,在内踝上五寸,少阴之前,与阴维之会。\\
昌阳之脉,令人腰痛,痛引膺,目丼丼然,甚则反折,舌卷不能言,刺内筋为二痏,在内踝上大筋前,太阴后,上踝二寸所。\\
散脉,令人腰痛而热,热甚生烦,腰下如有横木居其中,甚则遗溲,刺散脉,在膝前骨肉分间,络外廉束脉,为三痏。\\
肉里之脉,令人腰痛,不可以咳,咳则筋缩急,刺肉里之脉为二痏,在太阳之外,少阳绝骨之后。\\
腰痛侠脊而痛至头,几几然,目丼丼欲僵仆,刺足太阳郄中出血。腰痛上寒,刺足太阳阳明;上热,刺足厥阴;不可以俯仰,刺足少阳;中热而喘,刺足少阴,刺郄中出血。\\
腰痛上寒,不可顾,刺足阳明;上热,刺足太阴;中热而喘,刺足少阴。大便难,刺足少阴。少腹满,刺足厥阴。如折,不可以俛仰,不可举,刺足太阳;引脊内廉,刺足少阴。\\
腰痛引少腹控尐,不可仰。刺腰尻交者,两髁胂上,以月生死为痏数,发针立已。左取右,右取左。\\
足太阳经脉发生病变使人腰痛证见,疼痛牵引到颈项脊背和臀部,背部如负重物,刺取太阳经正经上的委中穴,使其出血,但春天勿刺出血。\\
足少阳经脉病变使人腰痛时证见,痛如用针刺入皮肤一样,痛感顺脉下行,不能俯仰和转腰顾盼,刺取少阳经循行经过的胫骨上端出血,即刺成骨在膝外侧的突起部位,但夏季勿刺出血,出血太多,就不能痊愈了。\\
足阳明经脉病变使人腰痛时证见,不能转腰回头,勉强回头好像看到什么,容易悲伤,刺取阳明经在小腿前的穴位三次,并配合上下的穴位,要刺出血,但秋天勿刺出血,出血太多,就不能痊愈了。\\
足少阴经脉病变使人腰痛时证见,疼痛牵引脊骨内侧,刺取少阴经在足踝之上的复溜穴两次,但春天勿刺出血,出血太多,就不能痊愈了。\\
足厥阴经脉病变使人腰痛时证见,腰部似拉开的弓弩般地拘急,刺取厥阴经,取腿肚与足跟之间鱼腹状突出处外侧,摸上去有串珠样硬结,在此针刺,这种病常使人沉默寡言而精神不爽慧,应针刺三次。\\
解脉病变使人腰痛时证见,疼痛牵引到肩部,两眼视物不清,有时遗尿,刺取解脉在膝弯筋肉分界处委中穴外侧横见的血脉,使其出血,待血色由紫黑变红才停止。\\
解脉病变使人腰痛时证见,痛如牵挽腰带,常有折断腰的感觉,易恐惧,刺取解脉,在委中穴寻找有黍米样结滞的血络,刺后有黑色血液射出,待血色变红为止。\\
同阴之脉使人腰痛时证见,痛如小锤梗塞腰中,经脉怒胀肿起,在外踝上绝骨部位刺同阴之脉三次。\\
阳维脉使人腰痛时证见,痛处经脉怒胀肿起,刺阳维脉,取阳维脉和太阳经在腿肚下端会合处离地一尺左右的承山穴。\\
横络之脉使人腰痛时证见,不能俯仰,后仰则恐怕跌倒,得病原因是用力举重伤及腰部,使横络之脉阻绝不通,淤血留滞其中,在委阳穴处和位于委中穴上数寸两筋之间的殷门穴处刺横络之脉,视血络横居盛满者刺两次,使之出血。\\
会阴之脉使人腰痛证见,疼痛发时汗出淋漓,汗止则口渴欲饮水,饮水后又要小便,刺取直阳之脉上的穴位三次,在刉上郄下五寸的承筋穴,视血络充盈者,刺之使出血。\\
飞阳之脉使人腰痛时证见,痛处经脉突然怒张肿胀,痛甚则悲伤恐惧,刺取飞扬之脉,在内踝上五寸,足少阴经之前与阴维脉交会之处。\\
昌阳之脉使人腰痛时证见,疼痛牵引胸膺,两眼视物昏花,严重的腰背向后反折,不能前屈,舌头卷缩,不能言语,刺取筋内侧的复溜穴两次,在内踝之上,大筋之前,足太阴经之后,踝上二寸左右处。\\
散脉使人腰痛,证见发热,热重则烦躁不安,腰下好像有根横木阻塞其中,严重的会遗尿,针刺散脉,在膝前缘骨与肉之间部位,横络外侧的束状脉,刺三次。\\
肉里之脉使人腰痛时证见,痛时不敢咳嗽,咳嗽则筋脉挛缩拘急,刺治肉里之脉两次,在足太阳经的外侧,足少阳经绝骨穴的后面。\\
腰痛沿脊柱两侧作痛上及头部,颈项拘急不舒,两目昏花,好像要跌倒,刺治足太阳经的委中穴出血。腰痛时疼处怕冷的,刺治足太阳、阳明两经;腰痛时痛处发热的,刺治足厥阴经;不能俯仰的,刺治足少阳经;肺内有热而喘促的,刺治足少阴经,并刺委中穴处的血络出血。\\
腰痛时痛处寒冷,不能转侧顾盼的,刺治足阳明经;痛处发热的,刺治足太阴经;肺内有热而喘促的,刺治足少阴经。兼有大便困难的,刺治足少阴经。兼少腹胀满的,刺治足厥阴经。腰痛如折,不能俯仰,不能举动的,刺治足太阳经;疼痛牵掣脊柱内缘的,刺治足少阴经。\\
腰痛牵掣小腹和胁下,不能后仰的,可刺腰尻的下髎穴,穴在两髁骨肌肉群的上方。根据月亮的盈亏计算针刺的次数,针出病愈。左病取右侧腧穴,右病取左侧腧穴。\\
卷十二\\
风论篇第四十二\\
黄帝问曰:风之伤人也,或为寒热,或为热中,或为寒中,或为疠风,或为偏枯,或为风也。其病各异,其名不同。或内至五脏六腑。不知其解,愿闻其说。\\
岐伯对曰:风气藏于皮肤之间,内不得通,外不得泄。风者善行而数变。腠理开则洒然寒,闭则热而闷。其寒也则衰食饮,其热也则消肌肉。故使人怢慄而不能食,名曰寒热。\\
风气与阳明入胃,循脉而上至目内眦。其人肥,则风气不得外泄,则为热中而目黄;人瘦则外泄而寒,则为寒中而泣出。\\
风气与太阳俱入,行诸脉俞,散于分肉之间,与卫气相干。其道不利,故使肌肉愤尒而有疡。卫气有所凝而不行,故其肉有不仁也。疠者,有荣气热胕,其气不清,故使其鼻柱坏而色败,皮肤疡溃。风寒客于脉而不去,名曰疠风,或名曰寒热。\\
以春甲乙伤于风者为肝风,以夏丙丁伤于风者为心风,以季夏戊己伤于邪者为脾风,以秋庚辛中于邪者为肺风,以冬壬癸中于邪者为肾风。\\
风中五脏六腑之俞,亦为脏腑之风,各入其门户,所中则为偏风。风气循风府而上,则为脑风;风入系头,则为目风;眠寒,饮酒中风,则为漏风;入房汗出中风,则为内风;新沐中风,则为首风;久风入中,则为肠风、飧泄;外在腠理,则为泄风。故风者,百病之长也。至其变化,乃为他病也,无常方,然致有风气也。\\
帝曰:五脏风之形状不同者何?愿闻其诊及其病能。\\
岐伯曰:肺风之状,多汗恶风,色皏然白,时咳短气。昼日则差,暮则甚。诊在眉上,其色白。\\
心风之状,多汗恶风,焦绝,善怒嚇,赤色。病甚则言不可快。诊在口,其色赤。\\
肝风之状,多汗恶风,善悲。色微苍,嗌干善怒,时憎女子。诊在目下,其色青。\\
脾风之状,多汗恶风,身体怠惰,四肢不欲动。色薄微黄,不嗜食。诊在鼻上,其色黄。\\
肾风之状,多汗恶风,面痝然浮肿,脊痛不能正立。其色炲,隐曲不利。诊在肌上,其色黑。\\
胃风之状,颈多汗,恶风,食饮不下,鬲塞不通,腹善满。失衣则尒胀,食寒则泄。诊形瘦而腹大。\\
首风之状,头面多汗恶风,当先风一日则病甚,头痛不可以出内。至其风日,则病少愈。\\
漏风之状,或多汗,常不可单衣。食则汗出,甚则身汗,喘息恶风。衣常濡,口干善渴,不能劳事。\\
泄风之状,多汗,汗出泄衣上。口中干,不能劳事,身体尽痛则寒。\\
帝曰:善!\\
黄帝问道:风邪伤害人体,有的发为寒热,有的发为内热,有的发为内寒,有的成为疠风,有的成为偏枯,全由风邪引起。但病情不一样,病名也不相同。有的侵入内部,达到五脏六腑之间。我不了解这其中的道理,希望听您谈谈。\\
岐伯回答说:风气侵入了皮肤里面,既不能在内部流通,又不能向外部疏泄。风行动最快,变化多端。腠理开张的时候,会使人觉得寒冷;腠理关闭的时候,会使人觉得发热烦闷。寒冷时就会饮食减退,发热时就会肌肉消瘦。所以使人突然寒冷而不想吃东西,病名叫做寒热。\\
风气从阳明经入胃,循着经脉上行一直到目内眦。如果是胖人,风邪就不易向外发散,稽留体内,成为内热,出现眼珠发黄;如果是瘦人,阳气容易向外发泄而寒冷,就会成为内寒,而不时流泪。\\
风气从太阳经脉侵入人体,流行于各经腧穴,散布在分肉之间,和卫气纠缠在一起。这样,气道不通畅,肌肉就会肿起而成为疮疡。如因卫气有所凝滞,运行不畅,那么肌肉就会麻木不仁。疠风,是营气有热,血气不清,所以致使鼻柱损伤,面色变坏,皮肤溃烂。因为风寒久留在经脉里而不能去,所以叫做疠风,有的又称寒热。\\
在春季甲乙日伤风的,是肝风;在夏季丙丁日伤风的,是心风;在季夏戊己日伤风的,是脾风;在秋季庚辛日中风的,是肺风;在冬季壬癸日中风的,是肾风。\\
风邪侵入到五脏六腑的腧穴,就变成了五脏六腑的风,无论是络、经、脏、腑,只要风邪从其门户入侵,就成为偏风。风邪侵入后,从风府沿经脉上行至脑,就成为脑风;风入头中的目系,就成为目风;睡觉着凉,并且醉后感受风邪,就成为漏风;入房时汗出,感受风邪,就成为内风;刚洗完头,感受风邪,就成为首风;风邪久留肌腠,伤及脾胃,就成为肠风飧泄;外在腠理之间的,就成为泄风。所以风是引起各种疾病的主要因素。它的变化极多,而且变生其他疾病时,没有一定的规律,但是致病的原因,归根到底来自风气的侵入。\\
黄帝说:五脏风的症状,都有哪些不同?希望听听诊察的要点和病态表现。\\
岐伯说:肺风的症状是多汗怕风,面色白,时而咳嗽气短。白天较轻,傍晚较重。诊察时要注意眉上部位,色白即是。\\
心风的症状是多汗怕风,形体干瘦,经常发怒,面色红。病重时,说话不爽快。诊察要注意口舌,当见赤色。\\
肝风的症状是多汗怕风,易悲伤。面色微青,咽喉干燥,容易发怒,不时厌恶女人。诊察时要注意目下,当见青色。\\
脾风的症状是多汗怕风,身体疲倦,四肢不愿意活动。面色微黄,不想吃东西。诊察时要注意鼻上,当见黄色。\\
肾风的症状是多汗怕风,面部浮肿,腰脊疼痛,不能长时间站立。面色黑得像烟煤,小便不通畅。诊察时要注意面颊,当见黑色。\\
胃风的症状是颈部多汗怕风,食饮不下,膈部痞塞不通,腹满闷。衣服穿少了,腹部就容易胀满,吃了凉东西,就要泄泻。诊察时要注意病人形瘦腹大的特点。\\
头风的症状是头面部多汗怕风,在风气将发的前一天,就感到很痛苦,头痛得厉害,不愿到外面去。到了风胜那天,头痛的情况,反而会减轻。\\
漏风的症状是有的汗出得多,不能穿单薄的衣服。一吃饭就出汗,甚至全身汗出喘息、怕风。衣裳总是湿漉漉的,口干易渴,受不了劳累。\\
内风的症状是多汗,汗出多了,沾湿衣裳。口中干燥,禁受不了劳累,周身疼痛并且怕冷。\\
黄帝说:说得好!\\
痹论篇第四十三\\
黄帝问曰:痹之安生?\\
岐伯对曰:风寒湿三气杂至合而为痹也。其风气胜者为行痹,寒气胜者为痛痹,湿气胜者为著痹也。\\
帝曰:其有五者何也?\\
岐伯曰:以冬遇此者为骨痹;以春遇此者为筋痹;以夏遇此者为脉痹;以至阴遇此者为肌痹;以秋遇此者为皮痹。\\
帝曰:内舍五脏六腑,何气使然?\\
岐伯曰:五脏皆有合,病久而不去者,内舍其合也。故骨痹不已,复感于邪,内舍于肾;筋痹不已,复感于邪,内舍于肝;脉痹不已,复感于邪,内舍于心;肌痹不已,复感于邪,内舍于脾;皮痹不已,复感于邪,内舍于肺。所谓痹者,各以其时重感于风寒湿之气也。\\
凡痹之客五脏者,肺痹者,烦满喘而呕。心痹者,脉不通,烦则心下鼓,暴上气而喘,嗌干善噫,厥气上则恐。肝痹者,夜卧则惊,多饮数小便,上为引如怀。肾痹者,善胀,尻以代踵,脊以代头。脾痹者,四支解堕,发咳呕汁,上为大塞。肠痹者,数饮而出不得,中气喘争,时发飧泄。胞痹者,少腹膀胱按之内痛,若沃以汤,涩于小便,上为清涕。\\
阴气者,静则神藏,躁则消亡。饮食自倍,肠胃乃伤。淫气喘息,痹聚在肺;淫气忧思,痹聚在心;淫气遗溺,痹聚在肾;淫气乏竭,痹聚在肝;淫气肌绝,痹聚在脾。诸痹不已,亦益内也。其风气胜者,其人易已也。\\
帝曰:痹,其时有死者,或疼久者,或易已者,其故何也?\\
岐伯曰:其入脏者死,其留连筋骨者疼久,其留皮肤间者易已。\\
帝曰:其客于六腑者,何也?\\
岐伯曰:此亦其食饮居处,为其病本也。六腑亦各有俞,风寒湿气中其俞,而食饮应之,循俞而入,各舍其府也。\\
帝曰:以针治之奈何?\\
岐伯曰:五脏有俞,六腑有合,循脉之分,各有所发,各随其过,则病瘳也。\\
帝曰:荣卫之气,亦令人痹乎?\\
岐伯曰:荣者,水谷之精气也。和调于五脏,洒陈于六腑,乃能入于脉也,故循脉上下,贯五脏络六腑也。卫者,水谷之悍气也,其气慓疾滑利,不能入于脉也,故循皮肤之中,分肉之间,熏于肓膜,散于胸腹。逆其气则病,从其气则愈。不与风寒湿气合,故不为痹。\\
帝曰:善。痹,或痛,或不仁,或寒,或热,或燥,或湿,其故何也?\\
岐伯曰:痛者,寒气多也,有寒故痛也。其不痛不仁者,病久入深,荣卫之行涩,经络时疏,故不痛;皮肤不营,故为不仁。其寒者,阳气少,阴气多,与病相益,故寒也。其热者,阳气多,阴气少,病气胜,阳遭阴,故为痹热。其多汗而濡者,此其逢湿甚也。阳气少,阴气盛,两气相感,故汗出而濡也。\\
帝曰:夫痹之为病,不痛何也?\\
岐伯曰:痹在于骨则重,在于脉则血凝而不流,在于筋则屈不伸,在于肉则不仁,在于皮则寒。故具此五者,则不痛也。凡痹之类,逢寒则急,逢热则纵。\\
帝曰:善。\\
黄帝问道:痹病是怎样发生的?\\
岐伯回答说:风、寒、湿三气混杂在一起入侵人体而形成痹证。风偏重的,叫行痹;寒偏重的,叫痛痹;湿偏重的,叫著痹。\\
黄帝道:痹病分为五种,都是什么?\\
岐伯说:在冬天得病的叫骨痹;在春天得病的叫筋痹;在夏天得病的叫脉痹;在季夏得病的叫肌痹;在秋天得病的叫皮痹。\\
黄帝道:痹病的病邪有内藏于五脏六腑的,这是什么气使它这样的呢?\\
岐伯说:五脏都有外合的筋、脉、肉、皮、骨,病邪久留在体表不去,就会侵入它所相应的内脏。所以骨痹不愈,又感受了邪气,就内藏于肾;筋痹不愈,又感受了邪气,就内藏于肝;脉痹不愈,又感受了邪气,就内藏于心;肌痹不愈,又感受了邪气,就内藏于脾;皮痹不愈,又感受了邪气,就内藏于肺。所谓的痹病,是在五脏所主季节里感受风、寒、湿三气所形成的。\\
凡痹病侵入到五脏,肺痹的症状,是烦闷,喘息而呕。心痹的症状,是血脉不通,心烦而且心跳,暴气上冲而喘,咽喉干燥,经常嗳气。逆气上乘于心,就令人惊恐。肝痹的症状,是夜间睡眠多惊,好饮水,小便次数多,上引少腹,膨满像怀孕时一样。肾痹的症状,是浑身肿胀,胀得能坐而不能行,能低头而不能仰头,好像用尾骨着地,又好像颈骨下倾、脊骨上耸一样。脾痹的症状,是四肢倦怠无力,咳嗽,呕吐清汁,胸部痞塞。肠痹的症状,是常常喝水而小便困难,中气上逆,喘而急迫,有时要发生飧泄。胞痹的症状,是手按小腹、膀胱,内有痛感,且腹中觉热,好像浇了热水一样,小便涩滞,上部鼻流清涕。\\
五脏的阴气,安静时就精神内藏,躁动时就易于耗散。假如饮食过多了,肠胃就要受伤。气失其平和而喘息迫促,那么风寒湿的痹气就容易凝聚在肺;气失其平和而忧愁思虑,那么风寒湿的痹气就容易凝聚在心;气失其平和而遗尿,那么风寒湿的痹气就容易凝聚在肾;气失其平和而疲乏口渴,那么风寒湿的痹气就容易凝聚在肝;气失其平和而过饥伤胃,那么风寒湿的痹气就容易凝聚在脾。各种痹病日久不愈,会越来越往人体的内部发展。如果属于风气较胜的,那么病人就比较容易痊愈。\\
黄帝问:痹病时有会死的,有疼痛很久不好的,有很快就好的,这是什么缘故?\\
岐伯说:痹病侵入五脏的,就会死亡;缠绵在筋骨里的,疼痛就会长久不好;如果邪气只留在皮肤里的,那就容易好。\\
黄帝道:痹病有的侵入到六腑,是什么情况?\\
岐伯说:这是由于饮食不节,居处失宜,成为腑痹的根本原因。六腑各有腧穴,风、寒、湿三气从外侵袭了一定的腧穴,而又内伤饮食,外内相应,病邪就循着腧穴而入,各自潜留在本腑。\\
黄帝道:用针刺治疗痹证应怎样?\\
岐伯说:五脏有输穴,六腑有合穴,循着经脉所属的部分,各有发生疾病的部位,只要在各发生疾病的地方进行治疗,病就会痊愈的。\\
黄帝道:营气、卫气也与风、寒、湿三气相合而成痹病吗?\\
岐伯说:营气是水谷所化成的精气。它调和于五脏,散布在六腑,然后进入脉中,循着经脉的道路上下,贯通五脏、联络六腑。卫气是水谷所化生的悍气,悍气急滑,不能进入脉中,所以只循行皮肤之中,分肉之间,上熏蒸于肓膜,下散布于胸腹。如果卫气不顺着脉外循行,就会生病,但只要其气顺行,病就会好。总之,卫气是不与风、寒、湿三气相合的,所以不能发生痹病。\\
黄帝道:说得好!痹病有痛的,有麻木的,并有寒、热、燥、湿等不同情况,是什么原因?\\
岐伯说:痛的是寒气偏多,有寒气就疼痛。麻木不痛的,那是病程日久,病邪深入,营卫运行迟滞,但经络有时还能疏通,所以不痛;皮肤得不到营养,所以麻木不仁。寒多的,是阳气少,阴气多,阴气加剧了风寒湿的痹气,所以寒多;热多的,是阳气多,阴气少,病气过强,阳为阴迫,所以是痹热。多汗出而沾湿的,是感受湿气太甚。阳气不足,阴气有余,阴气和湿气相感,所以出汗多而沾湿了。\\
黄帝道:痹病有不痛的,这是什么缘故?\\
岐伯说:痹在骨的则身重,痹在脉的则血凝滞而不流畅,痹在筋的则屈而不伸,痹在肌肉的则麻木不仁,痹在皮肤的则寒凉。所以有这五种症状的,就不会有疼痛。大凡痹病之类,遇到寒气就挛急,遇到热气就弛缓。\\
黄帝说:说得好!\\
痿论篇第四十四\\
黄帝问曰:五脏使人痿,何也?\\
岐伯对曰:肺主身之皮毛,心主身之血脉,肝主身之筋膜,脾主身之肌肉,肾主身之骨髓。故肺热叶焦,则皮毛虚弱急薄,著则生痿躄也。心气热,则下脉厥而上,上则下脉虚,虚则生脉痿,枢折挈,胫纵而不任地也。肝气热,则胆泄,口苦,筋膜干,筋膜干则筋急而挛,发为筋痿。脾气热,则胃干而渴,肌肉不仁,发为肉痿。肾气热,则腰脊不举,骨枯而髓减,发为骨痿。\\
帝曰:何以得之?\\
岐伯曰:肺者,脏之长也,为心之盖也。有所失亡,所求不得,则发为肺鸣,鸣则肺热叶焦。故曰:五脏因肺热叶焦,发为痿躄,此之谓也。悲哀太甚,则胞络绝,胞络绝则阳气内动,发则心下崩,数溲血也。故《本病》曰:大经空虚,发为肌痹,传为脉痿。思想无穷,所愿不得,意淫于外,入房太甚,宗筋弛纵,发为筋痿,及为白淫。故《下经》曰:筋痿者,生于肝,使内也。有渐于湿,以水为事,若有所留,居处相湿,肌肉濡渍,痹而不仁,发为肉痿。故《下经》曰:肉痿者,得之湿地也。有所远行劳倦,逢大热而渴,渴则阳气内伐,内伐则热舍于肾。肾者水脏也,今水不胜火,则骨枯而髓虚,故足不任身,发为骨痿。故《下经》曰:骨痿者,生于大热也。\\
帝曰:何以别之?\\
岐伯曰:肺热者,色白而毛败;心热者,色赤而络脉溢;肝热者,色苍而爪枯;脾热者,色黄而肉蠕动;肾热者,色黑而齿槁。\\
帝曰:如夫子言可矣。论言治痿者,独取阳明,何也?\\
岐伯曰:阳明者,五脏六腑之海,主润宗筋,宗筋主束骨而利机关也。冲脉者,经脉之海也,主渗灌谿谷,与阳明合于宗筋。阴阳摠宗筋之会,会于气街,而阳明为之长,皆属于带脉,而络于督脉。故阳明虚,则宗筋纵,带脉不引,故足痿不用也。\\
帝曰:治之奈何?\\
岐伯曰:各补其荥而通其俞,调其虚实,和其逆顺。筋脉骨肉,各以其时受月,则病已矣。\\
帝曰:善。\\
黄帝问道:五脏都能使人发生痿弱的病,是什么原因?\\
岐伯说:肺主管全身的皮毛,心主管全身的血脉,肝主管全身的筋膜,脾主管全身的肌肉,肾主管全身的骨髓。所以肺脏有热,肺叶就会枯萎,皮毛也呈现虚弱急薄的状态,严重的,就发生痿躄的病。心脏有热,下行之脉就会逆而上行,以致上盛下虚,下虚就形成脉痿,关节像折断了一样,不能互相联系,足胫弛缓不能走路。肝脏有热,可使胆汁上泛而见口苦,筋膜失去营养而干枯,筋膜一干枯,筋就会拘急而挛缩,发生筋痿。脾脏有热,可使胃内津液干燥,口渴,肌肉麻痹不仁,发为肉痿。肾脏有热,则精液耗竭,腰脊不能活动,骨枯髓减,发为骨痿。\\
黄帝问:痿证是怎样发生的呢?\\
岐伯说:肺是五脏之长,又是心脏的华盖。遇到不如意的事,或欲望不能满足,心火烁肺,肺伤后喘喝有声,因此肺热液涸,肺叶焦枯。所以说五脏是由于肺热叶焦,得不到充养,发为痿躄病,说的就是这个道理。悲哀太过,就会损伤心包络,包络受阻,心包络受阻则阳气乘机在内里扰动,致使心下崩损,常常尿血。所以《本病》说:大的经脉空虚,发为脉痹,最后变为脉痿。思虑无穷,愿望又不能实现,意志总驰游在外,或房劳过伤,致使众筋弛缓,就发为筋痿,以致导致遗精、白带等病。所以《下经》说:筋痿病生于肝,是由于入房过度引起的。感受湿邪,在水中劳作,内有湿热留连,外居潮湿之地,肌肉为湿所困,以致麻木不仁,就成为肉痿。所以《下经》说:肉痿病是久居湿地引起的。有的因为远行劳累,又遇到大热天气,感到口渴,渴就是内部的阳气亏乏,阳气亏乏于是虚热就侵入到肾脏。肾属水脏,现在水不能胜火热,就会骨枯髓空,以致两足不能支持身体,发为骨痿。所以《下经》说:骨痿病,是由于大热所引起的。\\
黄帝问道:怎样分别五痿证呢?\\
岐伯回答说:肺脏有热的,面色白而毛发败坏;心脏有热的,面色红而孙络浮见;肝脏有热的,面色青而爪甲干燥;脾脏有热的,面色黄而肌肉痿软;肾脏有热的,面色黑而牙齿枯槁。\\
黄帝问:你以上所说是可取的。但医书上说,治疗痿证,应该独取阳明,是什么道理?\\
岐伯说:阳明是五脏六腑的源泉,能够润养众筋,众筋的功能,是约束骨肉,使关节滑利。冲脉是经脉的源泉,它能渗透灌溉分肉肌腠,与阳明合于众筋。阴经阳经都在众筋处相聚,再复合于气街,阳明是它们的首领,都连属于带脉,而系络于督脉。所以阳明经脉不足,那么众筋就要弛缓,带脉不能收引,所以足部痿弱不堪运用了。\\
黄帝问:那么怎样治疗呢?\\
岐伯回答说:用补荥气和通输气的办法,调虚实,和逆顺。无论筋、脉、骨、肉,各在其当旺的月份,进行治疗,病就会好的。\\
黄帝说:说得好!\\
厥论篇第四十五\\
黄帝问曰:厥之寒热者,何也?\\
岐伯对曰:阳气衰于下,则为寒厥;阴气衰于下,则为热厥。\\
帝曰:热厥之为热也,必起于足下者,何也?\\
岐伯曰:阳气起于足五指之表,阴脉者,集于足下,而聚于足心;故阳气胜则足下热也。\\
帝曰:寒厥之为寒也,必从五指而上于膝者,何也?\\
岐伯曰:阴气起于五指之里,集于膝下而聚于膝上,故阴气胜,则从五指至膝上寒。其寒也,不从外,皆从内也。\\
帝曰:寒厥何失而然也?\\
岐伯曰:前阴者,宗筋之所聚,太阴阳明之所合也。春夏则阳气多而阴气少,秋冬则阴气盛而阳气衰。此人者质壮,以秋冬夺于所用,下气上争不能复,精气溢下,邪气因从之而上也。气因于中,阳气衰,不能渗营其经络。阳气日损,阴气独在,故手足为之寒也。\\
帝曰:热厥何如而然也?\\
岐伯曰:酒入于胃,则络脉满而经脉虚。脾主为胃行其津液者也。阴气虚则阳气入,阳气入则胃不和,胃不和则精气竭,精气竭则不营其四支也。此人必数醉,若饱以入房,气聚于脾中不得散,酒气与谷气相薄,热盛于中,故热遍于身,内热而溺赤也。夫酒气盛而慓悍,肾气有衰,阳气独胜,故手足为之热也。\\
帝曰:厥或令人腹满,或令人暴不知人,或至半日远至一日乃知人者,何也?\\
岐伯曰:阴气盛于上则下虚,下虚则腹胀满;阳气盛于上,则下气重上,而邪气逆,逆则阳气乱,阳气乱则不知人也。\\
帝曰:善。愿闻六经脉之厥状病能也。\\
岐伯曰:巨阳之厥,则肿首头重,足不能行,发为眴仆。阳明之厥,则癫疾欲走呼,腹满不得卧,面赤而热,妄见而妄言。少阳之厥,则暴聋颊肿而热,胁痛,气不可以运。太阴之厥,则腹满尒胀,后不利,不欲食,食则呕,不得卧。少阴之厥,则口干溺赤,腹满心痛。厥阴之厥,则少腹肿痛,腹胀,泾溲不利,好卧屈膝,阴缩,气内热。盛则泻之,虚则补之,不盛不虚,以经取之。\\
太阴厥逆,气急挛,心痛引腹,治主病者。少阴厥逆,虚满、呕变、下泄清,治主病者。厥阴厥逆,挛、腰痛,虚满,前闭,谵言,治主病者。三阴俱逆,不得前后,使人手足寒,三日死。太阳厥逆,僵仆,呕血善衄,治主病者。少阳厥逆,机关不利。机关不利者,腰不可以行,项不可以顾,发肠痈不可治,惊者死。阳明厥逆,喘咳身热,善惊,衄呕血。\\
手太阴厥逆,虚满而咳,善呕沫,治主病者。手心主、少阴厥逆,心痛引喉,身热,死,不可治。手太阳厥逆,耳聋泣出,项不可以顾,腰不可以俯仰,治主病者。手阳明、少阳厥逆,发喉痹、嗌肿,痓,治主病者。\\
黄帝问:厥证有属寒属热的情况,是怎样形成的?\\
岐伯回答说:阳气衰竭于下,成为寒厥;阴气衰竭于下,成为热厥。\\
黄帝问:热厥证发热,一定从足底开始,这是什么道理?\\
岐伯回答说:阳经之气循行于足五趾的外侧端,汇集于足底而聚汇到足心;所以如果阳经之气偏胜,就会足底发热。\\
黄帝问:寒厥证的厥冷,一定从足五趾渐至膝部,这是什么道理?\\
岐伯回答说:阴经之气起于足五趾的内侧端,汇集于膝下后,上聚于膝部,所以阴经之气偏胜,就会从足五趾至膝部寒冷。这种寒冷,不是外寒侵入,而是内部阳虚所致。\\
黄帝问:寒厥是怎样形成的?\\
岐伯回答说:前阴是众筋聚集的地方,也是足太阴脾经和足阳明胃经的会合之处。春夏阳气多而阴气少,秋冬阴气盛而阳气衰。患寒厥的人,自恃身体壮实,在秋冬阳气已衰的季节,房事不节,在下的阴气,向上浮越,与阳气相争,使阳气不能内藏,精气漏泄,阴寒之气从而上逆。寒邪之气,潜居在里,阳气随之虚衰,不能渗透营运于经络之中。阳气一天天地受到损害,只有阴气存在,所以手足寒冷。\\
黄帝问:热厥是怎样形成的?\\
岐伯回答说:酒进入胃中,由于酒性慓悍直接循行于皮肤络脉,所以使络脉中血液充满,经脉反而空虚。脾是主管输送胃中津液的。若饮酒过度,脾无所输送则阴液亏损,阴液亏损则酒热的阳气乘虚而入,酒阳之气侵入,导致胃气不和,胃气不和则阴精化生无源而枯竭,阴精枯竭就不能营养四肢。这种人必然是经常酒醉或饱食之后行房,酒食之气郁聚于脾中,不能宣散,酒气与谷气相搏结,热盛于中焦,波及周身,因有内热而小便色赤。酒性慓悍浓烈,肾气虚衰,而阳气独盛于内,所以手足发热。\\
黄帝问:有的厥证使人腹部胀满,有的使人突然不知人事,或者半天,甚至长达一天时间才能苏醒,这是为什么?\\
岐伯回答说:下部的阴气充盛于上,下部就空虚,下部气虚则水谷不化而致腹部胀满;阳气偏盛于上,则下部的阴气也并聚于上,而致邪气逆乱,逆乱则扰乱阳气,阳气逆乱就不省人事了。\\
黄帝说:说得好!希望听听六经厥证的症状表现。\\
岐伯说:太阳经的厥证,可见头面肿、头沉重,足不能行走,发作时眼花跌倒。阳明经的厥证,可见疯癫,奔跑呼叫,腹部胀满不得安卧,面部赤热,精神障碍,出现幻视,胡言乱语。少阳经的厥证,可见突然耳聋,面颊肿而发热,胁肋疼痛,小腿不能运动。太阴经的厥证,可见腹部胀满,大便不爽,不思饮食,食则呕吐,不能安卧。少阴经的厥证,可见口干,小便色赤,腹胀满,心疼痛。厥阴经的厥证,可见少腹肿痛,腹胀满,二便不利,喜欢屈膝而睡,前阴萎缩,小腿内侧发热。治疗厥证:实证用泻法,虚证用补法,虚实不明显的,从本经取穴治疗。\\
足太阴经的经气厥逆,小腿拘急痉挛,心痛牵引腹部,取本经主病的腧穴治疗。足少阴经的经气厥逆,腹部虚满,呕逆,大便清稀,取本经主病的腧穴治疗。足厥阴经的经气厥逆,腰部拘挛疼痛,腹部虚满,小便不通,胡言乱语,取本经主病的腧穴治疗。足三阴经都发生厥逆,则二便不通,病人手足寒冷,三天死亡。足太阳经的经气厥逆,身体僵直跌倒,呕血,鼻易出血,取本经主病的腧穴治疗。足少阳经的经气厥逆,关节活动不灵。关节不利则腰部不能活动,颈项不能回顾,如果伴发肠痈,就为不可治的危证,如发惊厥,就会死亡。足阳明经的经气厥逆,喘促咳嗽,身体发热,易惊骇,鼻出血,呕血。\\
手太阴经的经气厥逆,胸中虚满而咳嗽,常常呕吐涎沫,取本经主病的腧穴治疗。手厥阴和手少阴经的经气厥逆,心痛牵引咽喉,身体发热,是不可治的死证。手太阳经的经气厥逆,耳聋流泪,颈项不能回顾,腰不能前后俯仰,取本经主病的腧穴治疗。手阳明经和手少阳经的经气厥逆,发为喉痹,咽部肿痛,颈项强直,取本经主病的腧穴治疗。\\
卷十三\\
病能论篇第四十六\\
黄帝问曰:人病胃脘痈者,诊当何如?\\
岐伯对曰:诊此者,当候胃脉,其脉当沉细,沉细者气逆,逆者人迎甚盛,甚盛则热;人迎者胃脉也,逆而盛,则热聚于胃口而不行,故胃脘为痈也。\\
帝曰:善。人有卧而有所不安者,何也?\\
岐伯曰:脏有所伤,及精有所之寄则安,故人不能悬其病也。\\
帝曰:人之不得偃卧者,何也?\\
岐伯曰:肺者,脏之盖也。肺气盛则脉大,脉大则不得偃卧。论在《奇恒阴阳》中。\\
帝曰:有病厥者,诊右脉沉而紧,左脉浮而迟,不然病主安在?\\
岐伯曰:冬诊之,右脉固当沉紧,此应四时;左脉浮而迟,此逆四时。在左当主病在肾,颇关在肺,当腰痛也。\\
帝曰:何以言之?\\
岐伯曰:少阴脉贯肾络肺,今得肺脉。肾为之病,故肾为腰痛之病也。\\
帝曰:善!有病颈痈者,或石治之,或针灸治之,而皆已,其真安在?\\
岐伯曰:此同名异等者也。夫痈气之息者,宜以针开除去之,夫气盛血聚者,宜石而写之。此所谓同病异治也。\\
帝曰:有病怒狂者,此病安生?\\
岐伯曰:生于阳也。\\
帝曰:阳何以使人狂?\\
岐伯曰:阳气者,因暴折而难决,故善怒也,病名曰阳厥。\\
帝曰:何以知之?\\
岐伯曰:阳明者常动,巨阳少阳不动,不动而动大疾,此其候也。\\
帝曰:治之奈何?\\
岐伯曰:夺其食即已。夫食入于阴,长气于阳,故夺其食即已。使之服以生铁洛为饮。夫生铁洛者,下气疾也。\\
帝曰:善!有病身热解堕,汗出如浴,恶风少气。此为何病?\\
岐伯曰:病名曰酒风。\\
帝曰:治之奈何?\\
岐伯曰:以泽泻、术各十分,麋衔五分,合,以三指撮,为后饭。\\
所谓深之细者,其中手如针也;摩之切之,聚者坚也,博者大也。《上经》者,言气之通天也;《下经》者,言病之变化也;《金匮》者,决死生也;《揆度》者,切度之也;《奇恒》者,言奇病也。所谓奇者,使奇病不得以四时死也;恒者,得以四时死也。所谓揆者,方切求之也,言切求其脉理也;度者,得其病处,以四时度之也。\\
黄帝问:病人患胃脘痈的,怎样诊断呢?\\
岐伯回答说:诊断这种病,应当诊察他的胃脉,其脉象应当沉而细,沉细是因为胃气上逆,上逆则人迎脉过盛,过盛则有热。人迎脉属于胃经之脉,气逆脉盛,是热气聚结胃口而不得疏散,所以胃脘发生痈肿。\\
黄帝说:很对。有人睡眠不安宁,这是什么缘故?\\
岐伯说:这是因为五脏损伤,或情志过于偏颇,所以睡眠不安宁。人若不能消除这两种原因,便不能断绝卧不安的病。\\
黄帝问:人不能仰卧,是什么缘故?\\
岐伯说:肺脏为五脏的华盖。肺内邪气充盛,则脉络胀大,肺的脉络胀大,就不能仰卧。在《奇恒阴阳》中有这方面的论述。\\
黄帝问:有患气逆的,诊得右手脉象沉而紧,左手浮而迟,不知主要病变在何处?\\
岐伯说:在冬天诊脉,右脉本来应当沉紧,这与四时相应;左手脉象浮而迟,这与四时相违背。此脉出现在左手,主要病变在肾脏,与肺也颇有关系,当见腰痛。\\
黄帝说:为什么这样说呢?\\
岐伯说:足少阴脉贯串肾脏,连络肺脏。如今反诊得浮而迟的肺脉,说明肾有病变,腰为肾之府,所以出现腰痛。\\
黄帝说:很对!患颈部痈肿的病人,有的用砭石治疗,有的用针灸治疗,而都能痊愈,道理何在?\\
岐伯回答说:这是由于病名相同,而病变的机理不同的缘故。由于气郁停滞而成的痈肿,宜采用针刺的方法开导消除它,由于气滞血淤而成的痈肿,宜用砭石来泻其淤血。这就是所谓“同病异治”。\\
黄帝问:有患发怒狂躁的,这种病是怎么产生的?\\
岐伯说:由于阳气逆乱。\\
黄帝问:阳气逆乱为何使人发狂?\\
岐伯说:阳气因突然受到精神挫折,内心苦闷一时难解,则气郁化火而上逆,所以容易发怒,病名叫做“阳厥”。\\
黄帝问:怎么知道的呢?\\
岐伯说:正常人阳明经脉搏动明显,而太阳、少阳经脉搏动不明显,本不当搏动明显的脉,反而搏动得盛大急疾,这就是阳厥的证候。\\
黄帝问:如何治疗?\\
岐伯说:禁止饮食,即可痊愈。因为饮食物经过脾阴的运化,能够助长阳气,所以禁止饮食,便会痊愈。再用生铁落煎水给他服,因为生铁落有降气开结的作用。\\
黄帝说:很对!有人患周身发热,四肢倦怠,汗出如洗浴状,怕风,呼吸气短,这是什么病?\\
岐伯回答说:病名叫酒风。\\
黄帝问:如何治疗?\\
岐伯说:用泽泻和白术各十份,糜衔五份,合研为末,每次服三指撮的量,在饭前服下。\\
所谓沉伏而细小的脉,其应指细小如针;推之按之,脉气聚而不散的是坚脉,阴阳搏击于指下的是大脉。《上经》是论述人体生命之气与自然的统一关系的;《下经》是论述疾病变化的;《金匮》是论述诊断疾病的预测生死的;《揆度》是阐述切脉方法的;《奇恒》是论述特殊疾病的。所谓奇病,就是患者死亡与四时不相应;所谓常病,就是患者死亡与四时相应。所谓揆,就是说通过切脉以推求疾病的所在及其病机;所谓度,就是指把切脉获得的病理资料,结合四时逆顺,分析治法、死生。\\
奇病论篇第四十七\\
黄帝问曰:人有重身,九月而瘖,此为何也?\\
岐伯对曰:胞之络脉绝也。\\
帝曰:何以言之?\\
岐伯曰:胞络者系于肾,少阴之脉,贯肾系舌本,故不能言。\\
帝曰:治之奈何?\\
岐伯曰:无治也,当十月复。《刺法》曰:无损不足,益有余,以成其疹。所谓无损不足者,身羸瘦,无用鑱石也;无益其有余者,腹中有形而泄之,泄之则精出而病独擅中。故曰疹成也。\\
帝曰:病胁下满、气逆,二、三岁不已,是为何病?\\
岐伯曰:病名曰息积。此不妨于食,不可灸刺,积为导引服药,药不能独治也。\\
帝曰:人有身体髀股气皆肿,环脐而痛,是为何病?\\
岐伯曰:病名曰伏梁,此风根也。其气溢于大肠而著于肓,肓之原在脐下,故环脐而痛也。不可动之,动之为水溺涩之病也。\\
帝曰:人有尺脉数甚,筋急而见,此为何病?\\
岐伯曰:此所谓疹筋。是人腹必急,白色黑色见则病甚。\\
帝曰:人有病头痛,以数岁不已,此安得之?名为何病?\\
岐伯曰:当有所犯大寒,内至骨髓,髓者以脑为主,脑逆故令头痛,齿亦痛,病名曰厥逆。\\
帝曰:善。\\
帝曰:有病口甘者,病名为何?何以得之?\\
岐伯曰:此五气之溢也,名曰脾瘅。夫五味入口,藏于胃,脾为之行其精气,津液在脾,故令人口甘也。此肥美之所发也,此人必数食甘美而多肥也。肥者令人内热,甘者令人中满,故其气上溢,转为消渴。治之以兰,除陈气也。\\
帝曰:有病口苦,取阳陵泉,口苦者病名为何?何以得之?\\
岐伯曰:病名曰胆瘅。夫肝者,中之将也,取决于胆,咽为之使。此人者,数谋而不决,故胆虚气上溢,而口为之苦。治之以胆募、俞,治在《阴阳十二官相使》中。\\
帝曰:有癃者,一日数十溲,此不足也。身热如炭,颈膺如格,人迎躁盛,喘息,气逆,此有余也。太阴脉微细如发者,此不足也。其病安在?名为何病?\\
岐伯曰:病在太阴,其盛在胃,颇在肺,病名曰厥,死不治。此所谓得五有余、二不足也。\\
帝曰:何谓五有余、二不足?\\
岐伯曰:所谓五有余者,五病之气有余也;二不足者,亦病气之不足也。今外得五有余,内得二不足,此其身不表不里,亦正死明矣。\\
帝曰:人生而有病巅疾者,病名曰何?安所得之?\\
岐伯曰:病名为胎病。此得之在母腹中时,其母有所大惊,气上而不下,精气并居,故令子发为巅疾也。\\
帝曰:有病痝然,如有水状,切其脉大紧,身无痛者,形不瘦,不能食,食少,名为何病?\\
岐伯曰:病生在肾,名为肾风。肾风而不能食,善惊不已,心气痿者,死。\\
帝曰:善!\\
黄帝问:有的妇女怀孕九月,不能说话,这是什么原因?\\
岐伯回答说:这是因为胞宫的络脉阻塞不通所致。\\
黄帝问:为什么这样说?\\
岐伯说:胞宫的络脉连系肾脏,而足少阴肾脉贯串肾而上系舌根,所以胞宫络脉被阻,就不能说话了。\\
黄帝问:怎么治疗呢?\\
岐伯说:不用治疗,等到十月分娩后,自然就能恢复。《刺法》上说:不要损伤不足的正气、补益有余的邪气,而造成新的病变。所谓不要损伤不足的正气,就是身体羸弱消瘦的,不要用针石治疗;所谓不能补益有余,是指腹中有孕而妄用攻下,结果只会使精气耗散而反增疾病。所以说错误治疗会造成新的病变。\\
黄帝问:患有胁下胀满,气上逆,二三年不好的,是什么病?\\
岐伯说:病名叫息积。这种病不妨碍饮食,不可用艾灸、针刺治疗,应长期用导引服用药物,单纯用药物是不能治愈的。\\
黄帝问:有的人髀、股、胫部都肿胀,而且绕脐疼痛,是什么病?\\
岐伯说:病名叫做伏梁,这是平素感受风寒而致。风寒之气充溢大肠,滞留附着于肓膜,肓之原穴在脐下,所以绕脐疼痛。不可用攻下方法,误用攻下,会发生小便涩滞的病变。\\
黄帝问:有的人尺脉急数,筋脉拘急而显露,这是什么病?\\
岐伯说:这就是所谓的疹筋。此人腹部必然拘急,如果面部显现白色或黑色,则病情严重。\\
黄帝问:有人患头痛几年不愈,这病是怎么得的?叫什么病?\\
岐伯说:病人应曾感受大寒,寒邪内侵骨髓,髓主要集中于脑,寒邪上逆于脑,所以使人头痛,牙齿也痛,病名叫厥逆。\\
黄帝说:对!\\
黄帝问:有的人口中有甜味,病名是什么?怎样得的?\\
岐伯说:这是由于五味的精气向上泛溢所致,病名叫脾瘅。饮食入口,贮藏在胃中,经脾转输其精气,今脾运失健,津液滞留在脾,迫使胃中的五味精气上泛,所以使人口中有甜味。这种病大都是过食肥甘厚味所致的,这种病人,一定是常吃甘美而肥腻的食品。肥厚食物可使人内热,甜食可使人中焦胀满,所以精气上泛,日久会转为消渴病。用兰草进行治疗,以祛除郁积日久的邪热之气。\\
黄帝问:有的人口发苦,病名叫什么?怎样得的病?\\
岐伯说:病名叫胆瘅。肝为将军之官主谋虑,胆为中正之官主决断,肝谋虑后还取决于胆之决断,咽部又受肝胆支配。这种病人,因经常谋虑而不决,导致胆气不足,胆汁向上泛溢,于是口中发苦。治疗应针刺胆募穴和胆腧穴,治疗方法记载在《阴阳十二官相使》之中。\\
黄帝说:有患小便不利的,一天小便几十次,这是正气不足。如果身热如火炭,颈部和胸膺之间如有物阻隔不通,人迎脉躁动急大,呼吸喘促,肺气上逆,这是邪气有余。如果寸口脉微细如发丝,这也是正气不足。病在哪里?叫什么病?\\
岐伯说:病在太阴脾脏,热邪炽盛于胃,影响到肺,病名叫厥,是不治的死证。这就是所谓五有余、二不足的病证。\\
黄帝问:什么叫五有余、二不足呢?\\
岐伯说:所谓五有余就是五种病气有余;二不足就是两种正气不足。现在外表有五种有余,内里有两种不足,这种病既不能从表治,又不能从里治,所以必死无疑。\\
黄帝问:有的婴儿生下来就患癫痫,病名叫什么?怎样得的?\\
岐伯说:病名叫做胎病。这种病是胎儿在母腹中时,由于其母受到大的惊吓,气逆上而不下行,精随气逆,所以使婴儿生下来就患有癫痫。\\
黄帝问:有人浮肿像有水状,切其脉大而紧,身无痛处,形体不消瘦,不能饮食,或吃得很少,这叫什么病?\\
岐伯说:这种病发于肾脏,名叫肾风。肾风病,如果不能饮食,经常惊悸不已,心气衰竭,就要死了。\\
黄帝说:对!\\
大奇论篇第四十八\\
肝满、肾满、肺满皆实,即为肿。肺之雍,喘而两胠满;肝雍,两胠满,卧则惊,不得小便;肾雍,脚下至少腹满,胫有大小,髀气大跛,易偏枯。\\
心脉满大,痫瘛筋挛;肝脉小急,痫瘛筋挛;肝脉骛暴,有所惊骇,脉不至若瘖,不治自已。肾脉小急,肝脉小急,心脉小急,不鼓皆为瘕。\\
肾肝并沉,为石水;并浮,为风水;并虚,为死;并小弦,欲惊。肾脉大急沉,肝脉大急沉,皆为疝。心脉搏滑急,为心疝;肺脉沉搏,为肺疝。三阳急,为瘕;三阴急,为疝;二阴急,为痫厥;二阳急,为惊。\\
脾脉外鼓,沉为肠澼,久自已。肝脉小缓为肠澼,易治。肾脉小搏沉,为肠澼下血。血温身热者,死。心肝澼亦下血,二脏同病者可治。其脉小沉涩为肠澼,其身热者死,热见七日死。\\
胃脉沉鼓涩,胃外鼓大,心脉小坚急,皆鬲偏枯。男子发左,女子发右,不瘖舌转,可治,三十日起。其从者,瘖,三岁起;年不满二十者,三岁死。脉至而搏,血衄身热者死,脉来悬钩浮为常脉。脉至如喘,名曰暴厥。暴厥者,不知与人言。脉至如数,使人暴惊,三四日自已。\\
脉至浮合,浮合如数,一息十至以上,是经气予不足也,微见九十日死。脉至如火薪然,是心精之予夺也,草干而死。脉至如散叶,是肝气予虚也,木叶落而死。脉至如省客,省客者,脉塞而鼓,是肾气予不足也,悬去枣华而死。脉至如丸泥,是胃精予不足也,榆荚落而死。脉至如横格,是胆气予不足也,禾熟而死。脉至如弦缕,是胞精予不足也,病善言,下霜而死;不言,可治。脉至如交漆,交漆者,左右傍至也,微见三十日死。脉至如涌泉,浮鼓肌中,太阳气予不足也,少气,味韭英而死。\\
脉至如颓土之状,按之不得,是肌气予不足也,五色先见黑,白壘发死。脉至如悬雍,悬雍者,浮揣切之益大,是十二俞之气予不足也,水凝而死。脉至如偃刀,偃刀者,浮之小急,按之坚大急,五藏菀熟,寒热独并于肾也,如此其人不得坐,立春而死。脉至如丸,滑不著手,不著手者,按之不可得也,是大肠气予不足也,枣叶生而死。脉至如华者,令人善恐,不欲坐卧,行立常听,是小肠气予不足也,季秋而死。\\
肝脉盛满、肾脉盛满、肺脉盛满的都为实证,即发生痈肿的征象。肺脉壅塞,喘息则两胠部胀满;肝脉壅塞,则两胠部胀满,眠中易惊骇,小便不通;肾脉壅塞,则从胫下到小腹部胀满,两胫肿胀不同,大小不一,髀部和胫部肿大,跛行,易发半身不遂。\\
心脉满大,是体内热盛,会出现癫痫抽搐及筋脉拘挛;肝脉小而急,是肝脏虚寒,也会出现癫痫抽搐和筋脉拘挛;如果肝脉急疾而乱,或受到惊吓后脉搏一时摸不到,好像失音一样,这是受惊气逆的现象,不治就会痊愈。肾脉小而紧、肝小而紧、心小而紧,不能鼓击于指下,是气血凝聚,都能发为瘕病。\\
肾脉、肝脉并见沉脉的是石水证;并见浮脉的是风水证;并见虚象的是死证,并见小弦之象的是将要发惊。肾脉大而沉紧,或肝脉大而沉紧的,都是疝气病。心脉搏动滑利急疾的是心疝;肺脉搏动见沉象的是肺疝。膀胱和小肠脉紧的是瘕病;肺和脾脉紧的是疝病;心和肾脉紧的是痫厥;胃和大肠脉紧的是惊骇。\\
脾脉外浮但又见沉象的是痢疾,日久能自愈。肝脉见小而缓脉象的是痢疾,易治。肾脉见小搏而沉的是痢疾便血。血蓄于外,身体发热的是死证。心、肝二脏的痢疾也见便血,如二脏同病,木火相生,可以治愈。如果脉见小而沉涩的痢疾,兼有身热不退的,为死证,高热七天就会死亡。\\
胃脉沉涩,或者外浮而大,心脉小而坚紧,皆为气血阻隔不通,半身不遂的征象。若男子偏瘫在左侧,女子偏瘫在右侧,而说话不受影响、舌转灵活的,可以治疗,约经三十天就能痊愈。如果男子偏瘫在右,女子偏瘫在左,说话发不出声音的,三年才能恢复;如果是不满二十岁的患者,三年即会死。脉来搏指有力,并见衄血、身热的,是死证,如脉见浮如悬钩之象,才是衄血病应有的脉象。身体发热的是死证。脉象虽然浮钩而未失中和之气的是常脉,不死。脉来似水流般湍急的,病名叫暴厥。暴厥病人不省人事,不能言语。脉来似有数象,是突然受惊所致,三四日即可自愈。\\
脉来如水波,浮荡分合不定,这种浮合脉如同数脉,一呼一吸之间搏动十次以上,这是经脉之气不足的现象,从微微显现这种脉象,大约九十日就要死亡。脉来如燃薪之火,或明或灭,其形不定,这是心脏精气脱失的现象,在秋尽冬初草枯之时就要死亡。脉来如散落的树叶,这是肝脏精气亏虚的现象,到秋天树木落叶时就会死亡。脉来去不定,如省亲的客人一样往返不居,省客脉,时而闭塞不至,时而应指有力,这是肾脏精气不足的现象,到初夏枣花开落的时节就会死亡。脉来如泥丸滚动,虽有圆象,但不柔软,这是胃腑精气不足的现象,到春季榆树上结挂榆英的时节就会死亡。脉来长而坚硬,如长枝条横于指下,这是胆腑精气不足的现象,到秋天稻禾成熟的时节就会死亡。脉来如弦线紧而细,这是胞的精气不足的现象,如患者胡言乱语,到下霜时节就会死亡;若没有多言之证,尚可治疗。脉来如绞滤漆汁般四面流散,这种绞漆脉,左右旁流,从微微显现这种脉象,经过三十日就要死亡。脉来如泉水外涌,浮而有力,鼓动于肌肉之中,这是太阳经脉的精气不足的现象,可见呼吸气短,在尝到韭菜花的时节就会死亡。\\
脉来如垮塌的松土虚软无力,重按即无,这是肌肉的精气不足的现象,如面部先出现五色中的黑色,到春天白藤发芽的时节就会死亡。脉来如悬雍垂一样,上大下小,这种悬雍脉,轻按浮取感觉虚大,这是十二腧穴的精气不足的现象,到水凝成冰的时节就会死亡。脉来如仰放着的刀,浮取脉小而急,重按脉大而坚,此为五脏郁热,寒热交并于肾,这样的病人不能坐起,到立春时节就会死亡。脉来如弹丸,滑不著手,滑不著手即按之不得,这是大肠精气不足的现象,在枣树生叶之时就会死亡。脉来如草木之花般轻浮软弱,病人易惊恐,不喜坐卧,行走和站立时常听见异常声音,这是小肠的精气不足的现象,在深秋时节就会死亡。\\
脉解篇第四十九\\
太阳所谓肿腰脽痛者,正月太阳寅,寅太阳也。正月阳气出在上,而阴气盛,阳未得自次也,故肿腰脽痛也。病偏虚为跛者,正月阳气冻解,地气而出也。所谓偏虚者,冬寒颇有不足者,故偏虚为跛也。所谓彊上引背者,阳气大上而争,故强上也。所谓耳鸣者,阳气万物盛上而跃,故耳鸣也。所谓甚则狂巅疾者,阳尽在上,而阴气从下,下虚上实,故狂巅疾也。所谓浮为聋者,皆在气也。所谓入中为瘖者,阳盛已衰,故为瘖也。内夺而厥,则为瘖俳,此肾虚也,少阴不至者,厥也。\\
少阳所谓心胁痛者,言少阳戌也,戌者,心之所表也,九月阳气尽而阴气盛,故心胁痛也。所谓不可反侧者,阴气藏物也,物藏则不动,故不可反侧也。所谓甚则跃者,九月万物尽衰,草木毕落而堕,则气去阳而之阴,气盛而阳之下长,故谓跃。\\
阳明所谓洒洒振寒者,阳明者午也,五月盛阳之阴也,阳盛而阴气加之,故洒洒振寒也。所谓胫肿而股不收者,是五月盛阳之阴也,阳者衰于五月,而一阴气上,与阳始争,故胫肿而股不收也。所谓上喘而为水者,阴气下而复上,上则邪客于脏腑间,故为水也。所谓胸痛少气者,水气在脏腑也,水者阴气也,阴气在中,故胸痛少气也。所谓甚则厥,恶人与火,闻木音则惕然而惊者,阳气与阴气相薄,水火相恶,故惕然而惊也。所谓欲独闭户牖而处者,阴阳相薄也,阳尽而阴盛,故欲独闭户牖而居。所谓病至则欲乘高而歌,弃衣而走者,阴阳复争,而外并于阳,故使之弃衣而走也。所谓客孙脉则头痛鼻鼽腹肿者,阳明并于上,上者则其孙络太阴也,故头痛鼻鼽腹肿也。\\
太阴所谓病胀者,太阴者子也,十一月万物气皆藏于中,故曰病胀。所谓上走心为噫者,阴盛而上走于阳明,阳明络属心,故曰上走心为噫也。所谓食则呕者,物盛满而上溢,故呕也。所谓得后与气,则快然如衰者,十一月阴气下衰,而阳气且出,故曰得后与气则快然如衰也。\\
少阴所谓腰痛者,少阴者申也,七月万物阳气皆伤,故腰痛也。所谓呕咳上气喘者,阴气在下,阳气在上,诸阳气浮,无所依从,故呕咳上气喘也。所谓邑邑不能久立久坐,起则目丼丼无所见者,万物阴阳不定未有主也,秋气始至,微霜始下,而方杀万物,阴阳内夺,故目丼丼无所见也。所谓少气善怒者,阳气不治,阳气不治,则阳气不得出,肝气当治而未得,故善怒,善怒者,名曰煎厥。所谓恐如人将捕之者,秋气万物未有毕去,阴气少,阳气入,阴阳相薄,故恐也。所谓恶闻食臭者,胃无气,故恶闻食臭也。所谓面黑如地色者,秋气内夺,故变于色也。所谓咳则有血者,阳脉伤也,阳气未盛于上而脉满,满则咳,故血见于鼻也。\\
厥阴所谓庂疝、妇人少腹肿者,厥阴者辰也,三月阳中之阴,邪在中,故曰庂疝少腹肿也。所谓腰脊痛不可以俯仰者,三月一振荣华,万物一俯而不仰也。所谓庂癃疝肤胀者,曰阴亦盛而脉胀不通,故曰庂癃疝也。所谓甚则嗌干热中者,阴阳相薄而热,故嗌干也。\\
太阳经有所谓的腰臀肿胀疼痛病,是因为正月属于太阳,而月建在寅,所以说正月太阳寅。正月阳气生发在上,但此时阴寒之气尚盛,阳气还不能按照自己应有的位次而逐渐旺盛,所以发生了腰臀肿胀疼痛。有病阳气偏虚而为跛足的,是因为正月阳气使冰冻消散,地气随之而出。所谓的偏虚,是因为冬寒之气使体内阳气颇感不足,所以阳气偏虚一侧,发生跛足。所谓头项强硬牵引及背之病,是因为阳气骤然上升,互相争扰,所以强硬牵引向上。所谓的耳鸣,是因为阳气使自然界万物生长旺盛,向上而活跃,人体阳气也是如此,阳气盛于上,所以发生了耳鸣。所谓阳气亢盛则发生癫狂之病,是因为阳气集中于上部,阴气滞留在下部,下虚而上实,所以发生癫狂之病。所谓气逆上浮而耳聋,是因为阳气亢盛于上所致。所谓阳气入内而音哑不能言语,是因为阳气由盛转衰,故致音哑不能言。色欲过度,精气内耗而厥逆,就会发生不能说话、四肢瘫痪的瘖痱病,这是肾虚,少阴经气不能布散而导致厥逆。\\
少阳经所谓的心胁疼痛,是因为少阳旺于九月,月建在戌,少阳之脉散络心包,故为心之表,九月阳气将尽,而阴气始盛,所以心胁疼痛。所谓的“不可反侧”,是因为此时阴气渐盛,万物开始收藏,万物收藏则静而不动,少阳经气应之,所以不能转侧。所谓的“甚则跃”,是因九月万物都衰退,草木凋零,人身之阳气也由表入里,阴气盛而阳气趋于下部活动,所以容易跌倒。\\
阳明经有所谓的恶寒战栗,是因为阳明旺于五月而月建在午,五月是阳气极盛而阴气始生,阳气旺盛并有阴气相加,所以恶寒战栗。所谓的胫肿而股不收,是因为五月阳盛极而阴始生,五月阳气由盛转衰,初生的阴气上升,与阳气相争,使阳明经气不和,所以小腿肿而大腿弛缓无力。所谓的气逆喘息而为水肿,是因为阴气自下而又上升,阴气上则水邪也随之上行,停聚在脏腑之间,所以形成水肿。所谓的胸痛少气,是因为水气停聚在脏腑之间,水属阴气,阴气在内,所以发生胸痛呼吸气短。所谓的病甚则厥逆,厌恶见人与火光,听到木击音就惊惕不安,是因为阳气与阴气相搏,水火不相协调,所以惊惕不安。所谓的欲独闭户牖而处,是因为阴阳之气相搏靠近,结果阳气衰而阴气盛,阴主静,所以喜欢关闭门窗而独居。所谓的病至则欲乘高而歌,弃衣而走,是因为阴阳二气反复交争,向外并于阳经,使阳气盛,致使病人登高而歌,弃衣而走。所谓的客孙脉则头痛鼻鼽腹肿,是因为阳明经的邪气上逆,邪气上入于本经的细小络脉及太阴之脉,则头痛鼻塞、腹部肿胀。\\
太阴经所谓的病胀,是因为太阴旺于十一月,月建在子,十一月为万物收藏之季,若阴邪循脾经入腹潜藏,就会腹部胀满。所谓的上走心为噫,是因为阴邪盛,邪气向上侵入足阳明经,而足阳明经之经别络属于心,阴气上犯心脏,所以嗳气。所谓的食则呕,是因为进食过多,胃中盛满,不能消化而向上泛溢,所以呕吐。所谓的得后与气则快然如衰,是因为十一月阴气由盛转衰,阳气将要发动,腹中阴邪随大便与矢气下行,所以病人得后与气,感到舒服,好像病情大大衰减了一样。\\
少阴经所谓的腰痛,是因为少阴旺于七月,月建在申,七月为万物阳气开始下降,人体与时令之气相应,阳气减损,所以腰痛。所谓的呕咳上气喘,是因为阴气盛于下,阳气浮越在上而无所依附,所以呕吐、咳嗽、气逆喘息。所谓的身体不适而不能久立久坐,起身则视物不清,目无所见,是因为天地万物阴阳二气交替不定,尚未稳定,此时秋天肃杀之气已经降临,微霜开始下降而克伐万物,人体阴阳之气被伐而衰减,所以视物不清、目无所见。所谓的少气善怒,是因为少阳经脉之气不能正常疏泄,阳气不能正常疏泄,阳气不得外出,肝气郁结不疏,所以容易发怒,易怒这种病名叫煎厥。所谓的恐如人将捕之,是因为秋天万物尚未尽衰,阴气初生而阳气入内,阴阳相搏,所以恐惧不安。所谓的恶闻食臭,是因为胃气衰败,所以厌闻食物气味。所谓的面黑如地色,是因为秋气肃杀,耗夺内脏精气,肾精不足,所以面色变黑。所谓的咳则有血,是因为上部的阳络损伤,虽然阳气还没有充满于上,但血液充盈以致脉满,脉满所以咳嗽,鼻中出血。\\
厥阴经的所谓伒疝、妇人少腹肿,是因为厥阴旺于三月,月建在辰,三月为阳气方长,阴气未尽的阳中之阴,阴邪积聚在内,循厥阴肝经逆行致病,所以伒疝、少腹肿胀。所谓的腰脊痛不可以俯仰,是因为三月阳气开始鼓动振发,万物荣华茂盛,但由于阴气未尽,阳气尚被阴气压抑不能温养,所以腰脊疼痛而不能俯仰。所谓的伒癃疝肤胀,是因为阴邪尚盛而致厥阴经脉闭塞不通,所以前阴肿痛、不得小便、肌肤肿胀。所谓的嗌干热中,是因为阴阳相争而致内热,所以咽喉发干。\\
卷十四\\
刺要论篇第五十\\
黄帝问曰:愿闻刺要。\\
岐伯对曰:病有浮沉,刺有浅深,各至其理,无过其道。过之则内伤,不及则生外壅,壅则邪从之。浅深不得,反为大贼,内动五脏,后生大病。故曰:病有在毫毛腠理者,有在皮肤者,有在肌肉者,有在脉者,有在筋者,有在骨者,有在髓者。\\
是故刺毫毛腠理无伤皮,皮伤则内动肺,肺动则秋病温疟,泝泝然寒栗。刺皮无伤肉,肉伤则内动脾,脾动则七十二日四季之月,病腹胀烦,不嗜食。刺肉无伤脉,脉伤则内动心,心动则夏病心痛。刺脉无伤筋,筋伤则内动肝,肝动则春病热而筋弛。刺筋无伤骨,骨伤则内动肾,肾动则冬病胀,腰痛。刺骨无伤髓,髓伤则销铄气酸。体解汃然不去矣。\\
黄帝问:希望听听针刺的要领。\\
岐伯回答说:疾病有在表在里,刺法有浅刺深刺,各应到达一定的部位,而不能违背这一准则。刺得太深,就会损伤内脏;刺得太浅,反而会使在表的气血壅滞,壅滞则病邪乘机深入。因此,针刺深浅不适合,反而会带来更大的危害,扰乱五脏功能,继而发生大病。所以说:疾病有在毫毛腠理的,有在皮肤的,有在肌肉的,有在脉的,有在筋的,有在骨的,有在髓的。\\
因此,刺毫毛腠理,不要伤及皮肤,皮肤受伤,就会向内影响肺脏功能,肺伤到秋天,易患温疟病,发生恶寒战栗。刺皮肤,不要伤及肌肉,肌肉受伤,就会向内影响脾脏功能,脾伤在每一季节的最后十八天中,会腹胀烦满,不思饮食。刺肌肉,不要伤及血脉,血脉受伤,就会向内影响心脏功能,心伤到夏天,易患心痛。刺血脉,不要伤及筋脉,筋脉受伤,就会向内影响肝脏功能,肝伤到春天,易患热性病,发生筋脉弛缓。刺筋,不要伤及骨,骨受伤,就会向内影响肾脏功能,肾伤到冬天,易患肿胀、腰痛。刺骨,不要伤髓,髓伤则消减,就会导致身体枯瘦,足胫发酸,身体懈怠,无力举动。\\
刺齐论篇第五十一\\
黄帝问曰:愿闻刺浅深之分。\\
岐伯对曰:刺骨者,无伤筋;刺筋者,无伤肉;刺肉者,无伤脉;刺脉者,无伤皮;刺皮者,无伤肉;刺肉者,无伤筋;刺筋者,无伤骨。\\
帝曰:余未知其所谓,愿闻其解。\\
岐伯曰:刺骨无伤筋者,针至筋而去,不及骨也;刺筋无伤肉者,至肉而去,不及筋也;刺肉无伤脉者,至脉而去,不及肉也;刺脉无伤皮者,至皮而去,不及脉也。所谓刺皮无伤肉者,病在皮中,针入皮中,无伤肉也;刺肉无伤筋者,过肉中筋也;刺筋无伤骨者,过筋中骨也。此之谓反也。\\
黄帝说:希望听听针刺浅深的不同要求。\\
岐伯回答说:刺骨,不要损伤筋;刺筋,不要损伤肌肉;刺肌肉,不要损伤脉;刺脉,不要损伤皮肤;刺皮肤,不要损伤肌肉;刺肌肉,不要损伤筋;刺筋,不要损伤骨。\\
黄帝说:我不懂其中的道理,希望能听听解释。\\
岐伯说:所谓刺骨不要伤筋,是说要刺至骨的,不可在仅刺到筋而未达骨的深度时,就停针拔出;刺筋不要伤肌肉,是说要刺至筋的,不可在仅刺到肌肉而未达筋的深度时,就停针拔出;刺肌肉不要伤脉,是说要刺至肌肉深部的,不可在仅刺到脉而未达肌肉深部时,就停针拔去;刺脉不要伤皮肤,是说要刺至脉的,不可在仅刺到皮肤而未达脉的深度时,就停针拔去。所谓针刺皮肤不要伤肌肉,是说病在皮肤之中,针就刺至皮肤,不要刺伤肌肉;刺肌肉不要伤筋,是说针只能刺至肌肉,太过就会伤筋;刺筋不要伤骨,是说针只能刺至筋,太过就会伤骨。这些是说若针刺深浅不当,造成的反常情况。\\
刺禁论篇第五十二\\
黄帝问曰:愿闻禁数。\\
岐伯对曰:脏有要害,不可不察。肝生于左,肺藏于右。心部于表,肾治于里。脾为之使,胃为之市。鬲肓之上,中有父母。七节之傍,中有小心。从之有福,逆之有咎。\\
刺中心,一日死,其动为噫。刺中肝,五日死,其动为语。刺中肾,六日死,其动为嚏。刺中肺,三日死,其动为咳。刺中脾,十日死,其动为吞。刺中胆,一日半死,其动为呕。\\
刺跗上,中大脉,血出不止死。刺面,中溜脉,不幸为盲。刺头,中脑户,入脑立死。刺舌下中脉太过,血出不止为瘖。刺足下布络中脉,血不出为肿。刺郄中,中大脉,令人仆,脱色。刺气街,中脉,血不出,为肿、鼠仆。刺脊间,中髓,为伛。刺乳上,中乳房,为肿,根蚀。刺缺盆中,内陷,气泄,令人喘咳逆。刺手鱼腹,内陷,为肿。\\
无刺大醉,令人气乱。无刺大怒,令人气逆。无刺大劳人,无刺新饱人,无刺大饥人,无刺大渴人,无刺大惊人。\\
刺阴股,中大脉,血出不止,死。刺客主人,内陷中脉,为内漏、为聋。刺膝髌,出液,为跛。刺臂太阴脉,出血多,立死。刺足少阴脉,重虚出血,为舌难以言。\\
刺膺,中陷,中肺,为喘逆仰息。刺肘中,内陷,气归之,为不屈伸。刺阴股下三寸,内陷,令人遗溺。\\
刺掖下胁间,内陷,令人咳。刺少腹,中膀胱,溺出,令人少腹满。刺腨肠,内陷,为肿。刺匡上,陷骨中脉,为漏、为盲。刺关节,中液出,不得屈伸。\\
黄帝问说:希望听听人体禁刺的部位有多少。\\
岐伯回答说:内脏的要害之处,不能不详细审察。肝气生发于左,肺气肃降于右。心脏调节在表的阳气,肾脏管理在里的阴气。脾主运化,水谷精微赖以转输,胃主受纳,饮食水谷汇聚于此。膈肓的上面,有维持生命活动的心、肺两脏。第七椎旁的里面有心包络。上述部位都应该禁刺,遵循这个刺禁,就有利于治疗,违背了,则会造成祸害。\\
刺中心脏,大约一日即死,其变态为嗳气。刺中肝脏,大约五日即死,其变态为多言多语。刺中肾脏,大约六日即死,其变态为打喷嚏。刺中肺脏,大约三日即死,其变态为咳嗽。刺中脾脏,大约十日即死,其变态为频频吞咽。误刺中胆,大约一日半即死,其变态为呕吐。\\
针刺足背,误伤了大脉,若出血不止,就会死亡。针刺面部,误伤了溜脉,会有两目失明的不幸。针刺头部,误伤了脑户穴,若刺至脑髓,就会立即死亡。针刺舌下的廉泉穴,中经脉太深,若出血不止,可使喉哑失音。误刺足下布散的络脉,淤血内留而不出,可致局部肿胀。针刺委中穴太深,误伤了大经脉,可令人跌仆,面色苍白。针刺气街穴,误伤经脉,若淤血留着不去,就会肿胀,牵扯鼠蹊作痛。针刺脊椎间隙,误伤了脊髓,会背曲不伸。针刺乳中穴,伤及乳房,可使乳房肿胀,内部腐烂溃脓。针刺缺盆中央太深,会使肺气外泄,可令人喘咳气逆。针刺手鱼际穴太深,可使局部发生肿胀。\\
不要针刺大醉的人,否则会使气血逆乱。不要针刺正在大怒的人,否则会使气机上逆。不要针刺过度疲劳的人,不要针刺刚刚饱食的人,不要针刺过分饥饿的人,不要针刺极度口渴的人,不要针刺正受到极大惊吓的人。\\
刺大腿内侧,误伤了大脉,若出血不止,便会死亡。刺上关穴太深,误伤经脉,可使耳内化脓或耳聋。刺膝髌部,误伤以致流出液体,会使人发生跛足。刺手太阴经脉,误伤出血过多,则立即死亡。刺足少阴经脉,误伤出血,可使肾气更虚,以致舌体失养,语言困难。\\
针刺胸膺部太深,伤及肺脏,会致气喘上逆、仰面呼吸。针刺肘弯处太深,气结聚于局部而不行,以致手臂不能屈伸。针刺大腿内侧下三寸处太深,使人遗尿。\\
针刺腋下胁肋间太深,使人咳嗽。针刺小腹,误伤膀胱,使小便漏出流入腹腔,以致少腹胀满。针刺小腿肚太深,会使局部肿胀。针刺眼眶而深陷骨间,伤及脉络,就会流泪不止,甚至失明。针刺关节,如液体外流,则关节不能屈伸。\\
刺志论篇第五十三\\
黄帝问曰:愿闻虚实之要。\\
岐伯对曰:气实形实,气虚形虚。此其常也,反此者病。谷盛气盛,谷虚气虚。此其常也,反此者病。脉实血实,脉虚血虚。此其常也,反此者病。\\
帝曰:如何而反?\\
岐伯曰:气盛身寒,此谓反也;气虚身热,此谓反也;谷入多而气少,此谓反也;谷入少而气多,此谓反也;脉盛血少,此谓反也;脉少血多,此谓反也。\\
气盛身寒,得之伤寒。气虚身热,得之伤暑。谷入多而气少者,得之有所脱血,湿居下也。谷入少而气多者,邪在胃及与肺也。脉小血多者,饮中热也。脉大血少者,脉有风气,水浆不入。\\
夫实者,气入也。虚者,气出也。气实者,热也。气虚者,寒也。入实者,左手开针空也。入虚者,左手闭针空也。\\
黄帝说:希望听听有关虚实的要领。\\
岐伯回答说:气充实的,形体也壮实;气不足的,形体也虚弱。这是正常现象,与此相反的,就是病态。纳谷多的气盛,纳谷少的气虚。这是正常现象,与此相反的,就是病态。脉象搏大而有力的,是血液充盛;脉象搏小而细弱的,是血液不足。这是正常现象,与此相反的,就是病态。\\
黄帝问:反常现象是怎样的?\\
岐伯说:气盛而身体反觉寒冷的,这是反常现象;气虚而身体反感发热的,这是反常现象;饮食虽多而气不足的,这是反常现象;饮食少而气反盛的,这是反常现象;脉象搏大而血反少的,这是反常现象;脉象细小而血反多的,这是反常现象。\\
气旺盛而身寒冷,是感受了寒邪。气不足而身发热,是感受了暑热。饮食虽多而气反少的,是失血或湿邪聚集下部。饮食虽少而气反盛的,是邪气在胃和肺。脉象细小而血多,是饮酒过度而中焦有热。脉象盛大而血少,是由于风邪侵入脉中,而且不进汤水。\\
实证,是由于邪气侵入人体。虚证,是由于正气外泄。气实的多见热象。气虚的多见寒象。针刺治疗实证,出针后左手不要按闭针孔,使邪气外泄。治疗虚证,出针后左手随即闭合针孔,使正气不得外散。\\
针解篇第五十四\\
黄帝问曰:愿闻九针之解,虚实之道。\\
岐伯对曰:刺虚则实之者,针下热也,气实乃热也。满而泄之者,针下寒也,气虚乃寒也。菀陈则除之者,出恶血也。邪胜则虚之者,出针勿按。徐而疾则实者,徐出针而疾按之;疾而徐则虚者,疾出针而徐按之。言实与虚者,寒温气多少也。若无若有者,疾不可知也。察后与先者,知病先后也。为虚与实者,工勿失其法。若得若失者,离其法也。虚实之要,九针最妙者,为其各有所宜也。补泻之时者,与气开阖相合也。九针之名,各不同形者,针穷其所当补泻也。\\
刺实须其虚者,留针,阴气隆至,针下寒,乃去针也;刺虚须其实者,阳气隆至,针下热,乃去针也。经气已至,慎守勿失者,勿变更也。深浅在志者,知病之内外也。近远如一者,深浅其候等也。如临深渊者,不敢堕也。手如握虎者,欲其壮也。神无营于众物者,静志观病人,无左右视也。义无邪下者,欲端以正也。必正其神者,欲瞻病人目,制其神,令气易行也。\\
所谓三里者,下膝三寸也,所谓跗之者,举膝分易见也。巨虚者,\\
足气独陷者。下廉者,陷下者也。\\
帝曰:余闻九针,上应天地,四时阴阳,愿闻其方,令可传于后世,以为常也。\\
岐伯曰:夫一天、二地、三人、四时、五音、六律、七星、八风、九野,身形亦应之,针各有所宜,故曰九针。人皮应天,人肉应地,人脉应人,人筋应时,人声应音,人阴阳合气应律,人齿面目应星,人出入气应风,人九窍三百六十五络应野。故一针皮,二针肉,三针脉,四针筋,五针骨,六针调阴阳,七针益精,八针除风,九针通九窍,应三百六十五节气。此之谓各有所主也。人心意应八风,人气应天,人发齿耳目五声应五音六律,人阴阳脉血气应地,人肝目应之九。\\
黄帝说:希望听到关于九针的解释和对虚实的不同治疗方法。\\
岐伯说:针治虚证要用补法,是指针下出现热感,正气实才有热感。针刺实证要用泻法,是指针下出现凉感,邪气虚才有凉感。“菀陈则除之”是指放出恶血。“邪胜则虚之”,是指出针以后,不要按闭针孔而应使邪气外泄。所谓“徐而疾则实”,就是说慢慢地出针,出针后,迅速按闭针孔,这样正气就不致外泄;所谓“疾而徐则虚”,就是说迅速地出针,出针后,缓慢按闭针孔,这样就可使邪气得以外散。这里所说的虚实,是指气至时凉感和热感的多少而言。如果凉感或热感似有似无,那么疾病的虚实就难以断定了。审察疾病的先后,是要认识疾病的标与本。掌握疾病的虚实,医工应该恪守针法,不发生错误。假如似得似失无定,那就是背离了治疗法则。九针能够巧妙地解决疾病的虚实,因为九针能适应各种不同的病证。掌握补泻的时机,是指用针应该与气的开阖相配合。所谓“九针之名,各不同形”,是指根据九针的不同形制就能够完全发挥或补或泻的作用。\\
所谓“刺实须其虚”,是说留针以待阴气大来,针下有凉的感觉,然后去针;“刺虚须其实”,是说应该待阳气大来,针下有热的感觉,然后去针。所谓“经气已至,慎守勿失”,是说不要轻率地改变手法。所谓“深浅在志”,是要求搞清楚疾病的或内或外。所谓“近远如一”,是说不论病变深浅,候气之法是相同的。所谓“如临深渊”,是说不能懈怠大意。所谓“手如握虎”,是说行针需要坚定有力。所谓“神无营于众物”,是说应平心静气地观察病人,不左右张望。所谓“义无邪下”,是说一定要使针保持端正直下。所谓“必正其神”,是说需要注视病人的眼睛,来控制其精神活动,使经气容易运行。\\
所谓的三里穴在膝下外侧三寸处,所谓的“跗之”是说重按三里穴则足跗上动脉停止跳动,所以说“举膝分易见”。上巨虚穴,当举足取之,在胫骨外侧独自下陷处。下巨虚穴,则在陷中的下部。\\
黄帝说:我听说九针,在上与天地四时阴阳相应,希望听听其中的道理,使其能流传后世,作为治病的准则。\\
岐伯说:天地之至数,一至九的配属是,一配天、二配地、三配人、四配时、五配音、六配律、七配星、八配风、九配野,人的形体的各部分与这些是相对应的,而针各有与其相适应的疾病,所以有九针之名。具体地说来,人的皮肤与天相应,如同覆盖万物的天;人的肌肉与地相应,如同敦厚的地;人的血脉与人相应,其盛衰如同人的壮老;人的筋与时相应,其在各部分不同的功用如同四时气候各异;人的声音与自然界的五音相应合,如同五音清浊各异;人的充满阴阳之气的脏腑与六律相应,如同六律各有调节的情况;人的牙齿面目与星相应,其排列如天上的星辰一样;人的呼吸之气与风相应,像自然界的风一样往复流动;人的九窍、三百六十五络与野相应,像九野分布大地一样。所以第一种针法刺皮,第二种针法刺肌肉,第三种针法刺脉,第四种针法刺筋,第五种针法刺骨,第六种针法调和阴阳,第七种针法补益精气,第八种针法驱除风邪,第九种针法疏通九窍,以应三百六十五节之气。这就是说九针各有它独特的功能。人的心意,像八风一样变化无常;人的正气,像天一样运行不息;人的发齿耳目,像五音六律一样有条不紊;人的血气阴阳经脉,如同生化万物的大地;人的肝气通目,与九之数相应。\\
长刺节论篇第五十五\\
刺家不诊,听病者言。在头,头疾痛,为针之,刺至骨,病已,止。无伤骨肉及皮,皮者道也。\\
阳刺,入一傍四处,治寒热。深专者,刺大脏,迫脏刺背,背俞也。刺之迫脏,脏会。腹中寒热去而止。与刺之要,发针而浅出血。\\
治痈肿者刺痈上,视痈小大深浅刺,刺大者多血,小者深之,必端内针为故止。\\
病在少腹有积,刺皮厊以下,至少腹而止;刺侠脊两傍四椎间,刺两髂髎季胁肋间,导腹中气热下已。\\
病在少腹,腹痛不得大小便,病名曰疝,得之寒。刺少腹两股间,刺腰髁骨间,刺而多之,尽炅病已。\\
病在筋,筋挛节痛,不可以行,名曰筋痹。刺筋上为故,刺分肉间,不可中骨也,病起筋炅,病已止。\\
病在肌肤,肌肤尽痛,名曰肌痹,伤于寒湿。刺大分、小分,多发针而深之,以热为故;无伤筋骨,伤筋骨,痈发若变。诸分尽热,病已止。\\
病在骨,骨重不可举,骨髓酸痛,寒气至,名曰骨痹。深者刺,无伤脉肉为故,其道大分小分,骨热,病已止。\\
病在诸阳脉,且寒且热,诸分且寒且热,名曰狂。刺之虚脉,视分尽热,病已止。病初发,岁一发;不治,月一发;不治,月四五发,名曰癫病。刺诸分诸脉,其无寒者,以针调之,病已止。\\
病风,且寒且热,炅汗出,一日数过,先刺诸分理络脉;汗出且寒且热,三日一刺,百日而已。\\
病大风,骨节重,须眉堕,名曰大风。刺肌肉为故,汗出百日,刺骨髓,汗出百日,凡二百日,须眉生而止针。\\
精通针术的医家在未诊脉之前,首先听取病人的自诉。病头部,且头痛剧烈,可以给他用针治疗,刺至骨,病愈停针。针刺不要损伤骨肉与皮肤,皮肤为针刺出入必经之路,更要注意勿使受损。\\
阳刺法是正中刺一针,周围刺四针,以治疗寒热病。如病邪深入抟聚内脏,应刺五脏的募穴,邪气进逼五脏,应刺背部的五脏俞穴,邪气逼脏所以针刺背俞,是因为背俞是脏气聚会的地方。待腹中寒热退去,就应停针。针刺的要领,是出针时稍微出一点血。\\
治疗痈肿,应刺痈肿的部位,并根据其大小,决定针刺的深浅。刺大的痈肿,宜多出血,小的痈肿,要深刺,一定要端直进针,以到达病所为止。\\
病在少腹有积聚,针刺腹部皮肉丰厚以下的部位,直到少腹为止;再针夹脊第四椎间两旁的穴位和髂骨两侧的居髎穴,以及季胁肋间的穴位,以引导腹中热气下行,而病愈。\\
病在少腹,腹痛而且大小便不通,病名叫疝,是感受寒邪所致。应针刺少腹到两大腿内侧间以及腰部和髁骨间的穴位,针刺穴位要多,到少腹部都有热感,病就痊愈了。\\
病在筋,筋脉拘挛、关节疼痛,不能行走,病名为筋痹。应针刺在患病的筋上,针从分肉间刺入,注意不能刺中骨。待有病的筋脉出现热感,病已痊愈,可以停针。\\
病在肌肤,周身肌肤疼痛,病名为肌痹,这是被寒湿之邪所伤。应针刺大小肌肉会合之处,多下针而且深刺,以有热感为度;不要伤及筋骨,若损伤了筋骨,就会引起痈肿这类的病变。待各分肉都出现热感,说明病已痊愈,可以停针。\\
病在骨,骨重肢体不能抬举,骨髓深处酸痛,感到有寒气,病名为骨痹。治疗时应深刺,以不伤血脉肌肉为度,针刺的道路在大小分肉之间,待骨部有热感,病已痊愈,可以停针。\\
病在手足三阳经脉,或寒或热,同时各分肉之间也有或寒或热的感觉,这叫狂病。用针刺泻除脉中的邪气,观察各处分肉,若全部出现热感,病已痊愈,可以停针。有一种病,初起每年发作一次;不及时治疗,则每月发作一次,仍不治疗,则每月发作三、四次,这叫癫病。应针刺各大小分肉以及各部经脉,如不发冷的,用针刺调治,病愈停针。\\
因风患病,出现或寒或热,热则汗出,一日发作数次,首先针刺各分肉腠理及络脉;若仍然汗出且或寒或热,可以三天针刺一次,治疗一百天,就能痊愈。\\
病因大风侵袭,出现骨节沉重,胡须眉毛脱落,病名为大风。应针刺肌肉,使之出汗,连续治疗一百天,再针刺骨髓,仍使之出汗,也一百天,总计二百天,直到胡须眉毛重新生长,才停针。\\
卷十五\\
皮部论篇第五十六\\
黄帝问曰:余闻皮有分部,脉有经纪,筋有结络,骨有度量。其所生病各异,别其分部,左右上下,阴阳所在,病之始终,愿闻其道。\\
岐伯对曰:欲知皮部,以经脉为纪者,诸经皆然。\\
阳明之阳,名曰害蜚,上下同法。视其部中,有浮络者,皆阳明之络也。其色,多青则痛,多黑则痹,黄赤则热,多白则寒,五色皆见,则寒热也。络盛,则入客于经。阳主外,阴主内。\\
少阳之阳,名曰枢持,上下同法。视其部中,有浮络者,皆少阳之络也。络盛,则入客于经。故在阳者主内,在阴者主出,以渗于内,诸经皆然。\\
太阳之阳,名曰关枢,上下同法。视其部中,有浮络者,皆太阳之络也。络盛,则入客于经。\\
少阴之阴,名曰枢儒,上下同法。视其部中,有浮络者,皆少阴之络也。络盛,则入客于经。其入经也,从阳部注于经;其出者,从阴内注于骨。\\
心主之阴,名曰害肩,上下同法。视其部中,有浮络者,皆心主之络也。络盛,则入客于经。\\
太阴之阴,名曰关蛰,上下同法。视其部中,有浮络者,皆太阴之络也。络盛,则入客于经。凡十二经络脉者,皮之部也。\\
是故百病之始生也,必先于皮毛。邪中之则腠理开,开则入客于络脉,留而不去,传入于经,留而不去,传入于腑,廪于肠胃。邪之始入于皮也,泝然起毫毛,开腠理;其入于络也,则络脉盛、色变;其入客于经也,则感虚乃陷下。其留于筋骨之间,寒多则筋挛骨痛;热多则筋弛骨消,肉烁夬破,毛直而败。\\
帝曰:夫子言皮之十二部,其生病,皆何如?\\
岐伯曰:皮者,脉之部也。邪客于皮,则腠理开;开,则邪入客于络脉,络脉满,则注于经脉;经脉满,则入舍于腑脏也。故皮者有分部,不与而生大病也。\\
帝曰:善!\\
黄帝问:我听说皮肤上有十二经脉分属的部位,经络的分布有纵有横,筋脉的分布有结有络,骨胳也各有长短大小。它们所发生的疾病各不相同,这就要从皮肤的分部上来区别病变的左右上下,属阴属阳,疾病的开始和预后,希望听听其中的道理。\\
岐伯回答说:要知道皮肤的分属部位,是以经脉循行于皮肤的部位为依据的,各经都是如此。\\
阳明经的阳络,名叫“害蜚”,手足阳明经都是一样。观察其所属皮部中的浮络,都是阳明经的络脉。若络脉中多见青色,为痛证;多见黑色,为痹证;多见黄赤色,为热证;多见白色,为寒证;五色同时出现,为属寒热错杂之证。络脉邪气盛,就会向内传入本经。络脉属阳主外,经脉属阴主内。\\
少阳经的阳络,名叫“枢持”,手足少阳经都是一样。观察其所属皮部中的浮络,都是少阳经的络脉。络脉的邪气盛,就会向内传入本经。络脉为阳,邪气由络脉内入经脉,所以说“在阳者主内”,经脉属阴,邪气由经脉出而传入内脏,所以说“在阴者主出,以渗于内”,各经都是如此。\\
太阳经的阳络,名叫“关枢”,手足太阳经都是一样。观察其所属皮部中的浮络,都是太阳经的络脉。络脉的邪气盛,就会向内传入本经。\\
少阴经的阴络,名叫“枢儒”,手足少阴经都是一样。观察其所属皮部中的浮络,都是少阴经的络脉。络脉的邪气盛,就会向内传入本经。邪气传入经脉,是从属阳的络脉传到经脉的;然后从属阴的经脉出而向内传入骨。\\
厥阴经的阴络,名叫“害肩”,手足厥阴经都是一样。观察其所属皮部中的浮络,都是厥阴经的络脉。络脉的邪气盛,就会向内传入本经。\\
太阴经的阴络,名叫“关蛰”,手足太阴经都是一样。观察其所属皮部中的浮络,都属于太阴经的络脉。络脉的邪气盛,就会向内传入本经。大凡十二经的络脉在皮肤上的分布部位,就是十二皮部。\\
所以,诸多疾病的发生,都是先从皮毛开始的。病邪侵袭皮毛则腠理开张,腠理开张则邪气进入络脉,滞留而不去则向内传入经脉,仍滞留而不去,则传入六腑,聚集在肠胃。病邪开始从皮毛侵入时,病人感到恶寒,毫毛竖起,腠理开张;病邪侵入络脉,则络脉盛满而色泽改变;病邪侵入经脉时,则经气已虚,邪气内陷;若病邪留滞于筋骨之间,寒邪盛则筋脉挛急,骨节疼痛;热邪盛则筋脉弛纵,骨软无力,肌肉消瘦败坏,毛发枯瘁脱落。\\
黄帝问:夫子所说的十二皮部,它们发病的情况是怎样的?\\
岐伯说:皮肤有十二经脉分属的部位。邪气侵犯皮肤,则腠理开张;腠理开张,邪气因而侵入络脉,络脉为邪气充满则传入经脉;经脉的邪气盛满,则内传滞留在腑脏。所以皮肤上有十二经脉分属的部位,如有病变却不予治疗,邪气就会沿经络内传脏腑,以致发生大病。\\
黄帝说:讲得好!\\
经络论篇第五十七\\
黄帝问曰:夫络脉之见也,其五色各异,青黄赤白黑不同,其故何也?\\
岐伯对曰:经有常色,而络无常变也。\\
帝曰:经之常色,何如?\\
岐伯曰:心赤、肺白、肝青、脾黄、肾黑,皆亦应其经脉之色也。\\
帝曰:络之阴阳,亦应其经乎?\\
岐伯曰:阴络之色应其经,阳络之色变无常,随四时而行也。寒多,则凝泣;凝泣,则青黑;热多,则淖泽;淖泽,则黄赤。此皆常色,谓之无病。五色具见者,谓之寒热。\\
帝曰:善。\\
黄帝问说:络脉显现于外,它的五色各不相同,有青、有黄、有赤、有白、有黑之异,这是什么道理呢?\\
岐伯回答说:经脉的颜色恒常不变,而络脉没有常色,容易变化。\\
黄帝问说:经脉的常色是怎样的?\\
岐伯说:心主赤、肺主白、肝主青、脾主黄、肾主黑,都与其所属经脉的颜色相应。\\
黄帝问:阴络与阳络也和其经脉的颜色相应吗?\\
岐伯说:阴络的颜色与其经脉相应,阳络的颜色则变化无常,随着四时的转移而变化。寒气多则气血运行凝涩迟滞;气血运行凝涩迟滞,因而多见青黑之色;热气多则气血运行滑利急速;气血运行滑利急速,因而多见黄赤之色。这些都是正常的色泽变化,称为无病。如果五色全部显现,是寒热错杂之证。\\
黄帝说:讲得好!\\
气穴论篇第五十八\\
黄帝问曰:余闻气穴三百六十五,以应一岁,未知其所,愿卒闻之。\\
岐伯稽首,再拜对曰:窘乎哉问也!其非圣帝,孰能穷其道焉!因请溢意尽言其处。\\
帝捧手逡巡而却,曰:夫子之开余道也,目未见其处,耳未闻其数,而目以明,耳以聪矣。\\
岐伯曰:此所谓圣人易语,良马易御也。\\
帝曰:余非圣人之易语也。世言真数开人意。今余所访问者真数,发蒙解惑,未足以论也。然余愿闻夫子溢志,尽言其处,令解其意。请藏之金匮,不敢复出。\\
岐伯再拜而起,曰:臣请言之。背与心相控而痛,所治天突与十椎及上纪。上纪者,胃脘也;下纪者,关元也。背胸邪系阴阳左右,如此其病,前后痛涩,胸胁痛,而不得息,不得卧,上气、短气、偏痛,脉满起,斜出尻脉,络胸胁、支心、贯鬲,上肩,加天突;斜下肩,交十椎下。\\
脏俞,五十穴;腑俞,七十二穴;热俞,五十九穴;水俞,五十七穴;头上五行行五,五五二十五穴;中伓两傍各五,凡十穴;大椎上两傍各一,凡二穴;目瞳子浮白,二穴;两髀厌分中,二穴;犊鼻,二穴;耳中多所闻,二穴;眉本,二穴;完骨,二穴;项中央,一穴;枕骨,二穴;上关,二穴;大迎,二穴;下关,二穴;天柱,二穴;巨虚上下廉,四穴;曲牙,二穴;天突,一穴;天府,二穴;天牖,二穴;扶突,二穴;天窗,二穴;肩解,二穴;关元,一穴;委阳,二穴;肩贞,二穴;瘖门,一穴;齐,一穴;胸俞,十二穴;背俞,二穴;膺俞,十二穴;分肉,二穴;踝上横,二穴;阴阳\\
,四穴;水俞,在诸分;热俞,在气穴;寒热俞,在两骸厌中,二穴;大禁,二十五,在天府下五寸。凡三百六十五穴,针之所由行也。\\
帝曰:余已知气穴之处,游针之居,愿闻孙络谿谷,亦有所应乎?\\
岐伯曰:孙络三百六十五穴会,亦以应一岁。以溢奇邪,以通荣卫。荣卫稽留,卫散荣溢,气竭血著,外为发热,内为少气。疾泻无怠,以通荣卫,见而泻之,无问所会。\\
帝曰:善!愿闻谿谷之会也。\\
岐伯曰:肉之大会为谷,肉之小会为谿。肉分之间,谿谷之会,以行荣卫,以会大气。邪溢气壅,脉热肉败,荣卫不行,必将为脓,内销骨髓,外破大夬,留于节凑,必将为败。积寒留舍,荣卫不居,卷肉缩筋,肋肘不得伸,内为骨痹,外为不仁,命曰不足。大寒留于谿谷也。谿谷三百六十五穴会,亦应一岁,其小痹淫溢,循脉往来,微针所及,与法相同。\\
帝乃辟左右而起,再拜曰:今日发蒙解惑,藏之金匮,不敢复出,乃藏之金兰之室,署曰《气穴所在》。\\
岐伯曰:孙络之脉别经者,其血盛而当泻者,亦三百六十五脉,并注于络,传注十二络脉,非独十四络脉也,内解泻于中者十脉。\\
黄帝问:我听说人身有三百六十五个腧穴,与一年的日数相应,但不知具体所在,希望详尽地听听。\\
岐伯稽首再拜回答说:这是非常深奥的问题啊!如果不是圣帝,谁肯深究这些道理呢!因而请让我详尽地讲讲气穴的部位所在。\\
黄帝恭敬而谦逊地说:夫子以大道开导我,虽然眼睛尚未看见具体部位,耳朵尚未听到具体数目,却已耳聪目明,心领神会了。\\
岐伯说:这就是所谓的“圣人易语,良马易御”啊!\\
黄帝说:我并不是易语的圣人。世人常言,真数能开启人的思路,现在我所询问的就是气穴的真数,主要是为了启发蒙昧,解除疑惑,还谈不上讨论更精深的道理。不过我希望先生能详尽全面地说明气穴的部位,使我了解它的道理。请让我把它收藏在金匮之中,不轻易拿出妄传他人。\\
岐伯再拜后回答说:臣请谈谈这个问题。背与心胸部相互牵引而疼痛,其治法是取天突穴与中枢穴,以及上纪穴。上纪就是中脘穴,下纪就是关元穴。背与胸部的经脉斜系着前后左右,所以其一病就表现为前胸与后背牵引疼痛而涩滞,胸胁疼痛,呼吸不利,不能平卧,上气喘急,呼吸短促,或一侧偏痛,经脉满起,这是因为其脉斜出于尻部,连络胸胁,散布于心而贯穿膈,上肩与天突相会;又向下斜行到肩,交会于背部十椎之下的原因。\\
脏俞,有五十个穴位;腑俞,有七十二个穴位;热俞,有五十九个穴位;水俞,有五十七个穴位;在头部有五行,每行五穴,五五共二十五穴;脊椎两侧各有五脏俞五穴,左右共有十穴;大椎之上两侧各有大杼穴一个,共二穴;瞳子髎、浮白二穴,左右共四穴;环跳二穴;犊鼻二穴;听宫二穴;攒竹二穴;完骨二穴;风府一穴;窍阴二穴;上关二穴;大迎二穴;下关二穴;天柱二穴;巨虚上下廉四穴;颊车二穴;天突一穴;天府二穴;天牖二穴;扶突二穴;天窗二穴;肩井二穴;关元一穴;委阳二穴;肩贞二穴;哑门一穴;神阙一穴;胸俞十二穴;背俞二穴;膺俞十二穴;阳辅二穴;解溪二穴;照海、申脉共四穴;治水之俞在诸经分肉之间;治热之俞在经气聚会之处;治寒热之俞在两骸厌中有二穴;大禁之穴五里,禁二十五刺,位置在天府穴下五寸处。以上三百六十五穴,就是针刺时所选取的穴位。\\
黄帝说:我已经知道气穴的部位,就是行针的处所,还想了解孙络、谿谷也各有相应吗?\\
岐伯说:孙络与三百六十五穴相会,也与一岁相应。孙络可以疏散邪气,通畅营卫。如果邪气侵入人体,造成营卫稽留,卫气外散,营血内溢,使卫气耗竭而营血淤留,则在外发热,在内少气。应迅速用针刺泻去邪气,不要耽误,以通达营卫,只要见到有营血稽留,就应施针刺泻,不必问其是否为穴会所在。\\
黄帝说:讲得很对!希望听听谿谷的会合。\\
岐伯说:肌肉的大会合处是谷,肌肉的小会合处是谿。分肉之间,谿谷会合之处,可以通行营卫,聚会宗气。如果邪气盛溢而正气壅塞,脉络发热而肌肉腐败,营卫运行不畅,必将形成痈脓,在内消烁骨髓,在外破溃刐肉,如果邪留关节,必将使筋骨败坏。如果寒邪蓄积留滞,营卫不能循行其所,使筋脉肌肉卷缩,肋肘不能伸展,在内发为骨痹,在外表现为麻木不仁,这是正气不足,大寒留滞于谿谷所致。谿谷与三百六十五穴相会,也与一岁相应。如果从小痹之证发展传变,邪气随络脉往来不定,微针可以达到,方法与刺孙络之法相同。\\
黄帝于是避开左右,起身再拜说:今日承蒙夫子你启发,消除了我的蒙昧疑惑,我将把这些理论藏在金匮之中,不轻易拿出来。于是储藏在金兰之室,署名为《气穴所在》。\\
岐伯说:孙络之脉从经脉别出,其血盛而应当用泻法的,亦从三百六十五脉并注于络脉,进而传注到十二络脉,而不限于十四络脉的范围,若要从内驱散病邪,可取五脏的经脉泻之。\\
气府论篇第五十九\\
足太阳脉气所发者,七十八穴:两眉头各一,入发至顶三寸半,傍五,相去三寸,其浮气在皮中者,凡五行,行五,五五二十五,项中大筋两傍各一,风府两傍各一,侠背以下至尻尾二十一节,十五间各一,五脏之俞各五,六腑之俞各六,委中以下,至足小指傍,各六俞。\\
足少阳脉气所发者六十二穴:两角上各二,直目上发际内各五,耳前角上各一,耳前角下各一,锐发下各一,客主人各一,耳后陷中各一,下关各一,耳下牙车之后各一,缺盆各一,掖下三寸,胁下至胠,八间各一,髀枢中傍各一,膝以下,至足小指次指,各六俞。\\
足阳明脉气所发者六十八穴:额颅发际傍各三,面鼽骨空各一,大迎之骨空各一,人迎各一,缺盆外骨空各一,膺中骨间各一,侠鸠尾之外,当乳下三寸,侠胃脘各五,侠脐广三寸各三,下脐二寸侠之各三,气街动脉各一,伏菟上各一,三里以下至足中指各八俞,分之所在穴空。\\
手太阳脉气所发者三十六穴:目内眦各一,目外各一,鼽骨下各一,耳郭上各一,耳中各一,巨骨穴各一,曲掖上骨穴各一,柱骨上陷者各一,上天窗四寸各一,肩解各一,肩解下三寸各一,肘以下至手小指本各六俞。\\
手阳明脉气所发者二十二穴:鼻空外廉,项上各二,大迎骨空各一,柱骨之会各一,髃骨之会各一,肘以下至手大指、次指本各六俞。\\
手少阳脉气所发者三十二穴:鼽骨下各一,眉后各一,角上各一,下完骨后各一,项中足太阳之前各一,侠扶突各一,肩贞各一,肩贞下三寸分间各一,肘以下至手小指次指本各六俞。\\
督脉气所发者二十八穴:项中央二,发际后中八,面中三,大椎以下至尻尾及傍十五穴,至骶下凡二十一节,脊椎法也。\\
任脉之气所发者二十八穴:喉中央二,膺中骨陷中各一,鸠尾下三寸,胃脘五寸,胃脘以下至横骨六寸半一,腹脉法也。下阴别一,目下各一,下唇一,龂交一。\\
冲脉气所发者二十二穴:侠鸠尾外各半寸至脐寸一,侠脐下傍各五分至横骨寸一,腹脉法也。\\
足少阴舌下,厥阴毛中急脉各一,手少阴各一,阴阳\\
各一。手足诸鱼际脉气所发者。凡三百六十五穴也。\\
足太阳经脉之气通达的有七十八个腧穴:两眉头陷中各一穴,自眉头上行入发至前顶穴,其中有神庭、上星、囟会三穴,共长三寸半,前顶居中央一行,两旁各分两行,共五行,中行与外行相距三寸,浮于头部的脉气,运行在头皮间的共五行,每行五穴,五五二十五穴,在颈项大筋两旁各有一穴,两侧风府穴旁边各有一穴,从大椎循脊柱下行至尾骶,有二十一节,其中的十五个椎间,左右各有一穴,五脏的腧穴左右各有五个,六腑的腧穴左右各有六个,从委中穴以下到足小趾旁,左右各有六个腧穴。\\
足少阳经脉之气通达的有六十二个腧穴:两头角上各有二穴;从眼睛直上发际内,左右各有五穴;耳前角上左右各有一穴;耳前角下左右各有一穴;鬓发下左右各有一穴;客主人穴左右各一穴;耳后陷中各有一穴;下关穴左右各一穴;耳下牙车之后左右各有一穴;缺盆穴左右各一;腋下三寸,从胁下至胠,八肋之间,各有一穴;髀枢中左右各有一穴;从膝下到足小趾侧的次趾,左右足各有六个腧穴。\\
足阳明经脉之气通达的有六十八个腧穴:额颅发际旁左右各有三穴;颧骨骨空中左右各有一穴;大迎穴在下颔骨骨空陷中左右各一穴;人迎穴左右各一穴;缺盆外骨空陷中左右各有一穴;胸膺部每肋间左右各有一穴;夹鸠尾穴之外,正当乳下三寸,夹胃脘左右各有五穴;夹脐旁开三寸,左右各有三穴;夹脐,下二寸,左右各有三穴;气街穴在脉动处左右各一穴;左右伏莵穴上各有一穴;左右足三里穴以下到足中趾,各有八个腧穴,分布于一定的孔穴之中。\\
手太阳经脉之气通达的有三十六个腧穴:目内眦左右各有一穴,目外眦左右各有一穴,颧骨下左右各有一穴,耳廓上左右各有一穴,耳中左右各有一穴,巨骨穴左右各一,曲腋上左右各有一穴,柱骨穴的上陷中左右各有一穴,天窗穴上四寸处,左右各有一穴,肩解部左右各有一穴,肩解下三寸处左右各有一穴,肘部以下到手小指端,左右手各有六个腧穴。\\
手阳明经脉之气通达的有二十二个腧穴:鼻孔外侧及项部左右各有二穴;大迎穴在下颌骨空中左右各一;项肩相会之处,左右各有一穴;肩臂相会之处,左右各有一穴;肘部以下到手大指侧的次指间,左右手各有六个腧穴。\\
手少阳经脉之气通达的有三十二个腧穴:鼽骨之下左右各有一穴,眉后左右各有一穴,头角上左右各有一穴,耳后完骨下左右各有一穴,项中足太阳经之前左右各有一穴,夹扶突穴左右各有一穴,肩贞穴左右各有一穴,肩贞穴下三寸,其间左右各有一穴,肘部以下到手小指侧的次指端,左右各有六个腧穴。\\
督脉之气通达的有二十八个腧穴:项部中央有二穴,前发际向后,中行有八穴,面部中央有三穴,大椎以下到尻尾及尻尾两旁有十五穴。从大椎到尾骶共二十一节,这是计算脊椎骨的方法。\\
任脉之气通达的有二十八个腧穴:喉中央有二穴,胸膺骨陷中每陷各有一穴,鸠尾下三寸处是上脘穴,上脘穴至脐中央相距五寸,脐中央至横骨毛际相距六寸半,每寸各有一穴,共十四穴,这是腹部取穴的方法。下部前后二阴之间有一穴,两目下各有一穴,下唇下有一穴,龈交穴一个。\\
冲脉之气通达的有二十二个腧穴:夹鸠尾两旁各横开半寸,向下到脐有六穴,每穴相距一寸,夹脐两旁各横开五分,向下到横骨有五穴,每穴相距一寸,这是腹部经脉取穴的方法。\\
足少阴经脉之气通达于舌以下的有二穴,厥阴经脉在毛际中左右各有一急脉穴,手少阴经脉左右各有一阴郄穴,阴刉、阳刉脉各有一穴。手足鱼际皆为经脉之气通达的部位。以上共计三百六十五穴。\\
卷十六\\
骨空论篇第六十\\
黄帝问曰:余闻风者百病之始也,以针治之,奈何?\\
岐伯对曰:风从外入,令人振寒,汗出头痛,身重恶寒,治在风府,调其阴阳。不足则补,有余则泻。\\
大风颈项痛,刺风府,风府在上椎。大风汗出,灸伈伝,伈伝在背下侠脊傍三寸所,厌之,令病者呼伈伝,伈伝应手。\\
从风憎风,刺眉头。失枕,在肩上横骨间。折,使揄臂,齐肘正,灸脊中。\\
尐络季胁,引少腹而痛胀,刺伈伝。\\
腰痛不可以转摇,急引阴卵,刺八髎与痛上,八髎在腰尻分间。\\
鼠瘘,寒热还,刺寒府,寒府在附膝外解营。取膝上外者,使之拜;取足心者,使之跪。\\
任脉者,起于中极之下,以上毛际,循腹里,上关元,至咽喉,上颐循面入目。冲脉者,起于气街,并少阴之经,侠脐上行,至胸中而散。任脉为病,男子内结七疝,女子带下瘕聚。冲脉为病,逆气里急。\\
督脉为病,脊强反折。督脉者,起于少腹,以下骨中央。女子入系廷孔,其孔,溺孔之端也。其络循阴器,合篡间,绕篡后,别绕臀,至少阴与巨阳中络者合。少阴上股内后廉,贯脊属肾,与太阳起于目内眦,上额交巅,上入络脑,还出别下项,循肩髆内,侠脊抵腰中,入循膂络肾。其男子循茎下至篡,与女子等。其少腹直上者,贯脐中央,上贯心入喉,上颐环唇,上系两目之下中央。此生病,从少腹上冲心而痛,不得前后,为冲疝;其女子不孕,癃痔、遗溺、嗌干。督脉生病治督脉,治在骨上,甚者在脐下营。\\
其上气有音者,治其喉中央,在缺盆中者。其病上冲喉者治其渐,渐者,上侠颐也。\\
蹇,膝伸不屈,治其楗。坐而膝痛,治其机。立而骨解,治其骸关。膝痛,痛及拇指,治其腘。坐而膝痛,如物隐者,治其关。膝痛不可屈伸,治其背内。连气若折,治阳明中俞髎,若别,治巨阳少阴荥。淫泺胫痠,不能久立,治少阳之维,在外踝上五寸。\\
辅骨上,横骨下为楗,侠髋为机,膝解为骸关,侠膝之骨为连骸,骸下为辅,辅上为腘,腘上为关,头横骨为枕。\\
水俞五十七穴者;尻上五行,行五;伏菟上两行,行五,左右各一行,行五;踝上各一行,行六穴。髓空在脑后三分,在颅际锐骨之下,一在龂基下,一在项后中复骨下,一在脊骨上空在风府上;脊骨下空,在尻骨下,数髓空在面侠鼻,或骨空在口下当两肩;两髆骨空,在髆中之阳,臂骨空在臂阳,去踝四寸两骨空之间;股骨上空在股阳,出上膝四寸;气骨空在辅骨之上端,股际骨空在毛中动脉下;尻骨空在髀骨之后,相去四寸。扁骨有渗理凑,无髓孔,易髓无空。\\
灸寒热之法,先灸项大椎,以年为壮数,次灸橛骨,以年为壮数。视背俞陷者灸之,举臂肩上陷者灸之,两季胁之间灸之,外踝上绝骨之端灸之,足小指次指间灸之,腨下陷脉灸之,外踝后灸之,缺盆骨上切之坚痛如筋者灸之,膺中陷骨间灸之,掌束骨下灸之,脐下关元三寸灸之,毛际动脉灸之,膝下三寸分间灸之,足阳明跗上动脉灸之,巅上一灸之。犬所啮之处,灸之三壮,即以犬伤病法灸之。凡当灸二十九处。伤食灸之,不已者,必视其经之过于阳者,数刺其俞而药之。\\
黄帝问:我听说风邪是诸多疾病的始因,怎么用针法来治疗?\\
岐伯回答说:风邪从外侵入,使人寒战,出汗,头痛,身体沉重,恶寒,治疗取风府穴,以调和其阴阳。正气不足用补法,邪气有余用泻法。\\
如果大风之邪侵入颈项而致疼痛,刺风府穴,风府穴在椎骨第一节的上面。大风之邪而致汗出,灸匢匟穴,匢匟穴在背部第六椎下两旁距脊各三寸之处,用手按压,使病人感觉疼痛而呼出“匢匟”之声,匢匟穴就在指下痛处。\\
迎风怕风的,刺眉头攒竹穴。失枕的,取肩上横骨之间穴位治疗。臂痛如折的,使病人曲臂,取两肘尖相合在一处的姿势,然后在肩胛骨上端引一直线,正当脊部中央的部位,给以灸治。\\
从卐络季胁,牵引到少腹痛胀的,刺匢匟穴。\\
腰痛而不能转侧动摇,痛引筋脉挛急,下连睾丸,刺八髎穴与疼痛处,八髎穴在腰尻骨间孔隙中。\\
瘰疬寒热往来,刺寒府穴,寒府在膝上外侧骨与骨之间的孔穴中。凡取膝上外侧的孔穴,使患者弯腰,成敬拜的体位;取足心涌泉穴时,使患者成跪的姿势。\\
任脉经发源于中极穴的下面,而上行经过毛际到腹部,再上行通过关元穴到咽喉,又上行至颐,循行于面部而进入两目中。冲脉经发源于气街穴,与足少阴经相并,夹脐左右上行,到胸中而散。任脉经的病变,在男子是腹内结为七疝,在女子是带下和积聚之病。冲脉经的病变,是气逆上冲,腹中拘急疼痛。\\
督脉的病变,可见脊柱强硬反折。督脉起于小腹之下的横骨中央。女子入内系于廷孔,廷孔就是尿道的外端。其络脉循着阴户会合于会阴部,再分绕于肛门的后面,再分别行绕臀部,到足少阴经与足太阳经中的络脉会合。与足少阴经相合上行经股内后面,贯穿脊柱,连属于肾脏,与足太阳经共起于目内眦,上行至额部,左右交会于巅顶,向上入内联络于脑,从脑还出,分别左右,经项下行,循行于脊髆内,夹脊抵达腰中,入内循膂,络于肾。其在男子,则循阴茎,下至会阴,与女子相同。其从小腹直上的,穿过脐中央,向上贯穿心脏,进入喉中,上行到颐并环绕口唇,上行系于两目中央之下。督脉的病变,证见气从小腹上冲心而痛,大小便不通,称为冲疝;其在女子则为不孕,或小便不利、痔疾、遗尿、咽喉干燥等症。督脉生病治疗督脉,轻者取治横骨上的曲骨穴,重者则取治在脐下的阴交穴。\\
病人气逆喘鸣有声的,治疗取其喉部中央的天突穴,此穴在两缺盆的中间。病人气逆上冲咽喉的,治疗取其大迎穴,大迎穴在面部两旁夹颐之处。\\
跛足,膝关节能伸不能屈,治疗取股部的经穴。坐下而膝痛,治疗取环跳穴。站立时骨散如解,治疗取膝关节处的经穴。膝痛,牵引拇趾,治疗取膝弯处的委中穴。坐下而膝痛,像有东西潜伏其中的,治疗取承扶穴。膝痛不能屈伸活动,治疗取背部的经穴。如疼痛连及胫骨如同折断似的,治疗取阳明经中的俞髎三里穴,或者别取太阳经和少阴经的荥穴。膝胫部酸痛无力,不能久立,治取少阳经的别络光明穴,穴在外踝上五寸。\\
辅骨之上,横骨之下叫“楗”;侠髋骨两侧环跳穴处叫“机”;膝部的关节叫“骸关”;夹膝两旁的高骨叫“连骸”;连骸下面叫“辅骨”;辅骨上面的膝弯叫“腘”;腘上面就是“骸关”;头后项部的横骨叫“枕骨”。\\
治水之腧穴有五十七个:尻骨上有五行,每行各五穴;伏兔上方有两行,每行各五穴;其左右又各有一行,每行各五穴;足内踝上各一行,每行各六穴。髓空在脑后分为三处,都在颅骨边际锐骨的下面,一在龈基的下面,一在项后正中的复骨下面,一在脊骨上孔的风府穴上面;脊骨下孔在尻骨下面,还有几个髓孔在面部侠鼻两旁,有的骨孔在口唇下方与两肩相平的部位;两肩膊骨孔在肩膊中的外侧,臂骨的骨孔在臂骨的外侧,距离手腕四寸两骨的空隙之间;股骨上面的骨孔在股骨外侧,膝上四寸处;伄骨的骨孔在辅骨的上端,股际的骨孔在阴毛中的动脉下面;尻骨的骨孔在髀骨之后,距离四寸处。扁骨有渗灌的纹理聚合,没有直通骨髓的孔穴,骨髓有血脉渗灌的纹理,没有骨孔。\\
灸寒热证的方法是,先灸项后的大椎穴,根据病人年龄决定艾灸的壮数,其次灸尾骶骨的尾闾穴,也是以年龄决定艾灸的壮数。观察背部有凹陷的地方用灸法,上举手臂在肩上有凹陷的地方即肩髃穴用灸法,两侧的季胁之间的京门穴用灸法,足外踝上正取绝骨穴处用灸法,足小趾与次趾之间的侠谿穴用灸法,腨下凹陷处的承山穴用灸法,外踝后方的昆仑穴用灸法,缺盆骨上方按之坚硬如筋处用灸法,胸膺中陷骨间的天突穴用灸法,手腕部的横骨之下的大陵穴用灸法,脐下三寸的关元穴用灸法,阴毛边缘动脉处的气冲穴用灸法,膝下三寸的两筋间的三里穴用灸法,足阳明经所行足跗上动脉的冲阳穴用灸法,头顶上的百会穴也可用灸法。犬咬伤的地方先灸三壮,再按治犬伤病法灸治。以上灸治寒热证的部位共二十九处。因伤食而用灸法,病仍不愈的,必须仔细观察其经脉移行到络脉的地方,多刺其腧穴,同时再用药物调治。\\
水热穴论篇第六十一\\
黄帝问曰:少阴何以主肾?肾何以主水?\\
岐伯对曰:肾者至阴也,至阴者盛水也,肺者太阴也,少阴者冬脉也,故其本在肾,其末在肺,皆积水也。\\
帝曰:肾何以能聚水而生病?\\
岐伯曰:肾者胃之关也,关门不利,故聚水而从其类也。上下溢于皮肤,故为胕肿。胕肿者,聚水而生病也。\\
帝曰:诸水皆生于肾乎?\\
岐伯曰:肾者牝脏也,地气上者属于肾,而生水液也,故曰至阴。勇而劳甚则肾汗出,肾汗出逢于风,内不得入于脏腑,外不得越于皮肤,客于玄府,行于皮里,传为胕肿。本之于肾,名曰风水。所谓玄府者,汗空也。\\
帝曰:水俞五十七处者,是何主也?\\
岐伯曰:肾俞五十七穴,积阴之所聚也,水所从出入也。尻上五行行五者,此肾俞。故水病下为胕肿、大腹,上为喘呼、不得卧者,标本俱病。故肺为喘呼,肾为水肿,肺为逆不得卧,分为相输。俱受者,水气之所留也。伏兔上各二行行五者,此肾之街也。三阴之所交结于脚也。踝上各一行行六者,此肾脉之下行也,名曰太冲。凡五十七穴者,皆藏之阴络,水之所客也。\\
帝曰:春取络脉分肉,何也?\\
岐伯曰:春者木始治,肝气始生,肝气急,其风疾,经脉常深,其气少,不能深入,故取络脉分肉间。\\
帝曰:夏取盛经分腠,何也?\\
岐伯曰:夏者火始治,心气始长,脉瘦气弱,阳气留溢,热熏分腠,内至于经,故取盛经分腠。绝肤而病去者,邪居浅也。所谓盛经者,阳脉也。\\
帝曰:秋取经、俞,何也?\\
岐伯曰:秋者金始治,肺将收杀,金将胜火,阳气在合,阴气初胜,湿气及体,阴气未盛,未能深入,故取俞以泻阴邪,取合以虚阳邪。阳气始衰,故取于合。\\
帝曰:冬取井荥,何也?\\
岐伯曰:冬者水始治,肾方闭,阳气衰少,阴气坚盛,巨阳伏沉,阳脉乃去,故取井以下阴逆,取荥以实阳气。故曰:“冬取井荥,春不鼽衄。”此之谓也。\\
帝曰:夫子言治热病五十九俞,余论其意,未能领别其处,愿闻其处,因闻其意。\\
岐伯曰:头上五行行五者,以越诸阳之热逆也。大杼、膺俞、缺盆、背俞,此八者,以泻胸中之热也。气街、三里、巨虚上下廉,此八者,以泻胃中之热也。云门、髃骨、委中、髓空,此八者,以泻四肢之热也。五脏俞傍五,此十者,以泻五脏之热也。凡此五十九穴,皆热之左右也。\\
帝曰:人伤于寒而传为热,何也?\\
岐伯曰:夫寒盛则生热也。\\
黄帝问:少阴为什么主肾?肾又为什么主水?\\
岐伯回答说:肾为至阴之脏,至阴之脏水气最盛,肺属太阴之脏,少阴属肾脉而旺于冬季,所以人体水液代谢的根本在肾,其标末在肺,肺肾两脏都能积聚水液而为病。\\
黄帝问:肾为什么能积聚水液而生病呢?\\
岐伯说:肾是胃的关门,关门不通畅,水液就要聚集而生病了。水液上下泛滥于皮肤,所以形成浮肿。浮肿是水液积聚而生的病。\\
黄帝问:各种水病都是由肾而生成的吗?\\
岐伯说:肾属阴脏,水气由下而向上蔓延的属于肾病而生成的水液,所以叫至阴。自逞勇力而劳动或房事太过,则汗出于肾,出汗时又遇到风邪,风邪从开张之腠理侵入,向内不能进入脏腑,向外也不能从皮肤泄越,于是滞留在玄府之中,窜行皮肤之内,传变成为浮肿病。此病之本在于肾,病名叫风水。所谓玄府,就是汗孔。\\
黄帝问:治疗水病的腧穴有五十七个,它们属哪脏所主?\\
岐伯说:肾腧五十七个穴位,是阴气积聚之处,也是水液从此出入之处。尻骨之上有五行,每行五个穴位,这些是肾的腧穴。所以水病表现在下部则为浮肿、腹部胀大,表现在上部则为呼吸急促、不能平卧,这是肺肾标本同病。所以肺病表现为呼吸急促,肾病表现为水肿,肺病还表现为气逆、不得平卧,肺病与肾病的表现虽不相同,但二者之间相互输应、相互影响。这是因为水气停留于两脏的缘故。伏兔上方各有两行,每行五个穴位,这里是肾气循行的重要道路。肾和肝、脾三条阴经所交结的地方在小腿上。足内踝上方各有一行,每行六个穴位,这是肾的经脉下行于脚的部分,名叫太冲。以上共五十七个穴位,都隐藏在人体下部或较深部的络脉之中,也是水液容易停聚的地方。\\
黄帝问:春天针刺,取络脉分肉之间,为什么呢?\\
岐伯说:春天木气开始当令,肝气开始发生,肝气性质急躁,其变动像风一样迅疾,但是肝的经脉往往藏于深部,而春天风邪之气尚不太剧烈,不能深入经脉,所以只要浅刺络脉分肉之间就行了。\\
黄帝问:夏天针刺,取盛经分腠之间,为什么呢?\\
岐伯说:夏天火气开始当令,心气开始壮大,如果脉形瘦小而搏动气势较弱,是阳气流溢体表,热邪熏蒸分肉腠理,向内影响到经脉,所以针刺应当取盛经分腠。针刺只透过皮肤而病就退去,因为邪气居于浅表部位的缘故。所谓盛经,是指充盛、充足的阳脉。\\
黄帝问:秋天要取经穴和输穴,是什么道理?\\
岐伯说:秋天金气开始当令,肺气开始收敛肃杀,金气渐旺逐步胜过火气,阳气处于合闭过程中,阴气刚胜过阳气,湿邪侵犯人体,阴气未至太盛,不能助湿邪深入,所以取阴经的“输穴”以泻阴湿之邪,取阳经的“合穴”以泻阳热之邪。由于阳气开始衰退而阴气未至太盛,所以不取“经穴”而取“合穴”。\\
黄帝问:冬天要取“井穴”和“荥穴”,是什么道理?\\
岐伯说:冬天水气开始当令,肾气开始闭藏,阳气已经衰少,阴气更加坚盛,太阳之气伏沉于下,阳脉消退,所以取阳经的“井穴”以降其阴逆之气,取阴经的“荥穴”以充实阳气。所以说:“冬取井荥,春不鼽衄。”说的就是这个道理。\\
黄帝问:夫子说过治疗热病有五十九个腧穴,我已明其大概,但还不清楚这些腧穴的部位,希望听听它们的部位和治疗作用。\\
岐伯说:头上有五行,每行五个穴位,能发越诸阳经上逆的热邪。大杼、膺俞、缺盆、背俞这八个穴位,可以泻除胸中的热邪。气街、三里、上巨虚和下巨虚这八个穴位,可以泻除胃中的热邪。云门、肩髃、委中、髓空这八个穴位,可以泻除四肢的热邪。五脏的腧穴两傍各有五穴,这十个穴位,可以泻除五脏的热邪。共五十九个穴位,都在热邪所在部位的附近。\\
黄帝问:人伤于寒邪反而会变为热病,这是为什么呢?\\
岐伯说:寒气盛极,就会郁而发热。\\
卷十七\\
调经论篇第六十二\\
黄帝问曰:余闻刺法言,有余泻之,不足补之,何谓有余?何谓不足?\\
岐伯对曰:有余有五,不足亦有五,帝欲何问?\\
帝曰:愿尽闻之。\\
岐伯曰:神有余有不足,气有余有不足,血有余有不足,形有余有不足,志有余有不足。凡此十者,其气不等也。\\
帝曰:人有精气津液,四支九窍,五脏十六部,三百六十五节,乃生百病,百病之生,皆有虚实。今夫子乃言有余有五,不足亦有五,何以生之乎?\\
岐伯曰:皆生于五脏也。夫心藏神,肺藏气,肝藏血,脾藏肉,肾藏志,而此成形。志意通,内连骨髓,而成身形五脏。五脏之道,皆出于经隧,以行血气。血气不和,百病乃变化而生。是故守经隧焉。\\
帝曰:神有余不足何如?\\
岐伯曰:神有余则笑不休,神不足则悲。血气未并,五脏安定,邪客于形,洒淅起于毫毛,未入于经络也,故命曰神之微。\\
帝曰:补泻奈何?\\
岐伯曰:神有余,则泻其小络之血,出血勿之深斥,无中其大经,神气乃平。神不足者,视其虚络,按而致之,刺而利之,无出其血,无泄其气,以通其经,神气乃平。\\
帝曰:刺微奈何?\\
岐伯曰:按摩勿释,著针勿斥,移气于不足,神气乃得复。\\
帝曰:善。气有余不足奈何?\\
岐伯曰:气有余则喘咳上气,不足则息不利少气。血气未并,五脏安定,皮肤微病,命曰白气微泄。\\
帝曰:补泻奈何?\\
岐伯曰:气有余,则泻其经隧,无伤其经,无出其血,无泄其气。不足,则补其经隧,无出其气。\\
帝曰:刺微奈何?\\
岐伯曰:按摩勿释,出针视之,曰故将深之。适入必革,精气自伏,邪气散乱,无所休息,气泄腠理,真气乃相得。\\
帝曰:善。血有余不足奈何?\\
岐伯曰:血有余则怒,不足则恐。血气未并,五脏安定,孙络外溢,则络有留血。\\
帝曰:补泻奈何?\\
岐伯曰:血有余,则泻其盛经出其血;不足,则视其虚经,内针其脉中。久留而视,脉大,疾出其针,无令血泄。\\
帝曰:刺留血奈何?\\
岐伯曰:视其血络,刺出其血,无令恶血得入于经,以成其疾。\\
帝曰:善。形有余不足奈何?\\
岐伯曰:形有余则腹胀,泾溲不利;不足则四支不用。血气未并,五脏安定,肌肉蠕动,命曰微风。\\
帝曰:补泻奈何?\\
岐伯曰:形有余则泻其阳经,不足则补其阳络。\\
帝曰:刺微奈何?\\
岐伯曰:取分肉间,无中其经,无伤其络,卫气得复,邪气乃索。\\
帝曰:善。志有余不足奈何?\\
岐伯曰:志有余则腹胀飧泄,不足则厥。血气未并,五脏安定,骨节有动。\\
帝曰:补泻奈何?\\
岐伯曰:志有余则泻然筋血者,不足则补其复溜。\\
帝曰:刺未并奈何?\\
岐伯曰:即取之,无中其经,邪所乃能立虚。\\
帝曰:善。余已闻虚实之形,不知其何以生。\\
岐伯曰:气血以并,阴阳相倾。气乱于卫,血逆于经,血气离居,一实一虚。血并于阴,气并于阳,故为惊狂。血并于阳,气并于阴,乃为炅中。血并于上,气并于下,心烦惋善怒。血并于下,气并于上,乱而喜忘。\\
帝曰:血并于阴,气并于阳,如是血气离居,何者为实?何者为虚?\\
岐伯曰:血气者,喜温而恶寒。寒则泣不能流,温则消而去之,是故气之所并为血虚,血之所并为气虚。\\
帝曰:人之所有者,血与气耳。今夫子乃言血并为虚,气并为虚,是无实乎?\\
岐伯曰:有者为实,无者为虚,故气并则无血,血并则无气,今血与气相失,故为虚焉。络之与孙脉俱输于经,血与气并,则为实焉。血之与气并走于上,则为大厥,厥则暴死,气复反则生,不反则死。\\
帝曰:实者何道从来?虚者何道从去?虚实之要,愿闻其故。\\
岐伯曰:夫阴与阳皆有俞会。阳注于阴,阴满之外,阴阳匀平,以充其形,九候若一,命曰平人。夫邪之生也,或生于阴,或生于阳。其生于阳者,得之风雨寒暑;其生于阴者,得之饮食居处,阴阳喜怒。\\
帝曰:风雨之伤人奈何?\\
岐伯曰:风雨之伤人也,先客于皮肤,传入于孙脉,孙脉满则传入于络脉,络脉满则输于大经脉,血气与邪并客于分腠之间,其脉坚大,故曰实。实者外坚充满,不可按之,按之则痛。\\
帝曰:寒湿之伤人奈何?\\
岐伯曰:寒湿之中人也,皮肤收,肌肉坚紧,荣血泣,卫气去,故曰虚。虚者,聂辟气不足,按之则气足以温之,故快然而不痛。\\
帝曰:善!阴之生实奈何?\\
岐伯曰:喜怒不节则阴气上逆,上逆则下虚,下虚则阳气走之,故曰实矣。\\
帝曰:阴之生虚奈何?\\
岐伯曰:喜则气下,悲则气消。消则脉虚空。因寒饮食,寒气熏满,则血泣气去,故曰虚矣。\\
帝曰:经言阳虚则外寒,阴虚则内热,阳盛则外热,阴盛则内寒。余已闻之矣,不知其所由然也。\\
岐伯曰:阳受气于上焦,以温皮肤分肉之间。今寒气在外,则上焦不通,上焦不通,则寒气独留于外,故寒慄。\\
帝曰:阴虚生内热奈何?\\
岐伯曰:有所劳倦,形气衰少,谷气不盛,上焦不行,下脘不通。胃气热,热气熏胸中,故内热。\\
帝曰:阳盛生外热奈何?\\
岐伯曰:上焦不通利,皮肤致密,腠理闭塞,玄府不通,卫气不得泄越,故外热。\\
帝曰:阴盛生内寒奈何?\\
岐伯曰:厥气上逆,寒气积于胸中而不泻,不泻则温气去,寒独留,则血凝泣,凝则脉不通,其脉盛大以涩,故中寒。\\
帝曰:阴与阳并,血气以并,病形以成,刺之奈何?\\
岐伯曰:刺此者取之经隧,取血于营,取气于卫,用形哉,因四时多少高下。\\
帝曰:血气以并,病形以成,阴阳相倾,补泻奈何?\\
岐伯曰:泻实者气盛乃内针,针与气俱内,以开其门,如利其户。针与气俱出,精气不伤,邪气乃下。外门不闭,以出其疾,摇大其道,如利其路,是谓大泻。必切而出,大气乃屈。\\
帝曰:补虚奈何?\\
岐伯曰:持针勿置,以定其意。候呼内针,气出针入。针空四塞,精无从去。方实而疾出针,气入针出,热不得还。闭塞其门,邪气布散,精气乃得存。动气候时,近气不失,远气乃来,是谓追之。\\
帝曰:夫子言虚实者有十,生于五脏,五脏五脉耳。夫十二经脉皆生其病,今夫子独言五脏。夫十二经脉者,皆络三百六十五节,节有病必被经脉,经脉之病皆有虚实,何以合之?\\
岐伯曰:五脏者,故得六腑与为表里,经络支节,各生虚实。其病所居,随而调之。病在脉,调之血;病在血,调之络;病在气,调之卫;病在肉,调之分肉;病在筋,调之筋;病在骨,调之骨。燔针劫刺其下及与急者。病在骨,焠针药熨;病不知所痛,两\\
为上;身形有痛,九候莫病,则缪刺之;痛在于左而右脉病者,巨刺之。必谨察其九候,针道备矣。\\
黄帝问道:我听刺法上说,病属有余的用泻法,病属不足的用补法。什么是有余,什么是不足呢?\\
岐伯回答说:有余有五种,不足也有五种,你要问哪一种呢?\\
黄帝道:希望都听听。\\
岐伯说:神有有余和不足,气有有余和不足,血有有余和不足,形有有余和不足,志有有余和不足。这十种情况,随气流变,变化无穷。\\
黄帝问道:人有精气津液,四肢、九窍,五脏、十六部,三百六十五节,能够发生各种疾病,而各种疾病发生,各有虚实的不同。现在,夫子您只说有余的有五种,不足的也有五种,究竟是怎样发生的呢?\\
岐伯说:都是从五脏发生的。心藏神,肺藏气,肝藏血,脾藏肉,肾藏志,因而生成人的形体。而志意通达,与内部骨髓互相连系,而形成了人的身体五脏。五脏之间相互联系的通道,都是出自经脉之间,从而运行血气。如果血气不调和,就会变化发生各种疾病。所以诊断治疗,要以经脉作为根据。\\
黄帝问:神有余和不足的情况如何?\\
岐伯说:神有余就大笑不止,神不足就悲忧。如果病邪还未与血气混杂,那么,五脏还是安定的,这时病邪只是滞留在身体表面,只是肌肤毫毛恶寒,尚未进入经络,这叫做心经的微邪。\\
黄帝又问:治疗时怎样使用补泻之法呢?\\
岐伯说:神有余的,就刺它的小络之脉,使之出血,使之出血但不要推针深刺,更不要刺伤大的经脉,这样,神气就自然平调了。神不足的要用补法,看准那虚络,按摩以达病所,再配合针刺通利经气,不令出血,也不使其气外泄,只是疏通它的经脉,神气就平调了。\\
黄帝又问:针刺微邪应该怎样?\\
岐伯说:按摩病处,不要停止,针刺时不向深推针,只是引导转移病人之气,使之充足,神气就能恢复。\\
黄帝道:很好!气有余和不足的情况是怎样的?\\
岐伯说:气有余就喘咳、上逆,气不足就呼吸不利、气短。如果邪气尚未与气血混杂,那么五脏还是安定的,这时皮肤只是微病,病势尚轻,这叫做肺气微虚。\\
黄帝又问道:补泻的方法怎样?\\
岐伯说:气有余就泻经隧,但不要伤了经脉,不能出血,不能泄气。如气不足的,就要补经隧,不能出气。\\
黄帝又问道:针刺微病时应怎样?\\
岐伯说:应按摩病处,不要停止,同时拿出针让病人看,并佯说,准备深刺。但是刚进针还是改为浅刺,这样病人的精气自然贯注于内,而邪气就散乱于浅表,无处留止,邪气从腠理发泄了,真气自然就能恢复正常。\\
黄帝问说:很好。血有余和不足的情况是怎样的?\\
岐伯说:血有余就易发怒,血不足就易悲忧。如果邪气尚未与血气混杂,五脏还安定,只是孙络邪盛外溢,络内就会有淤血现象。\\
黄帝又问道:补泻的方法怎样?\\
岐伯说:血有余,泻他的盛经,刺之出血;血不足,补益虚弱的经脉,把针扎在经脉上。在进针后,若病人脉象正常,留针时间就要稍长;若脉见洪大,就要立刻拔针,不使出血。\\
黄帝又问道:刺留血的方法怎样?\\
岐伯说:看准哪有留血的络脉,刺出其血,但注意不要让恶血回流入经脉,而引起其他疾病。\\
黄帝道:很好。形有余和不足的情况是怎样的?\\
岐伯说:形有余就腹胀,小便不利;形不足则手足不灵活。如果邪气尚未与血气混杂,五脏还安定,只是肌肉有些微微蠕动的感觉,这叫“微风”。\\
黄帝又问道:补泻的方法怎样?\\
岐伯说:形有余就泻足阳明胃经的经脉之气,形不足就补足阳明胃经的络脉之气。\\
黄帝又问道:针刺微风之病应怎样?\\
岐伯说:刺其分肉间以散其邪,不要刺中经脉,也不要伤及络脉,卫气能够恢复,邪气就消散了。\\
黄帝道:很好。志有余和不足的情形是怎样的?\\
岐伯说:志有余就要腹胀飧泄,志不足就手足厥冷。如果邪气尚未与气血混杂,那么五脏还是安定的,只是骨节间有微动感。\\
黄帝又道:补泻的方法是怎样的?\\
岐伯说:志有余就刺泻然谷穴出血,志不足就在复溜穴采取补法。\\
黄帝又问道:在邪气与血气尚未相混的时候,怎样刺治呢?\\
岐伯说:就刺骨节微动的地方,不要伤及经脉,只刺邪所留止处,病邪马上就能除去。\\
黄帝道:很好。我已经听到关于虚实的各种情况,但还不知道是怎样产生的?\\
岐伯说:虚实的发生,是由于邪气与血气混杂,阴阳混乱,失去平衡。这样,气窜乱于卫分,血逆行于经络,血气都离了本位,就形成了一虚一实的情况。如果血与阴邪相混,气与阳邪相混,就会发生惊狂的病证。如果血与阳邪相混,气与阴邪相混,就会发生内热的病证。如果血与邪气在人体上部相混杂,气与邪气在人体下部相混杂,就会心中烦闷,多怒。如果血与邪气在人体下部相混杂,气与邪气在人体上部相混杂,就会使人气乱健忘。\\
黄帝道:血与阴邪相混,气与阳邪相混,像这样血气离了本位,怎样才算实,怎样才算虚呢?\\
岐伯说:血和气都喜欢温暖而厌恶寒冷。寒冷会使血气涩滞不能畅通,温暖就能使血气消散而易于运行,所以气若偏胜,就有血虚的现象;而血若偏胜,就有气虚的现象。\\
黄帝问说:人体最宝贵的,就是血和气了。现在夫子您说血偏胜,气偏胜都是虚,那么就没有实了吗?\\
岐伯说:多余的就叫做实,不足的就叫做虚。因为,气偏胜,血就显得不足;血偏胜,气就显得不足。加之血和气失去了正常联系,所以就成为虚了。大络和孙络里的血气都流注到经脉,如果血与气混杂,那就成为实了。若血和气混杂后,循着经络上逆,就会发生大厥证,得了大厥证,就会突然昏死过去,气能恢复就能活,否则就会死去。\\
黄帝道:实是从什么渠道来的?虚又是从什么渠道去的?虚实的关键,我希望听听其中的缘故。\\
岐伯说:阴经和阳经,都有输入和会合的腧穴。阳经的气血,灌注到阴经,阴经气血充满了,就流走到其他地方,这样阴阳平衡,来充实人的形体,九候的脉象一致,就是正常人。凡邪气的发生,有生于阴分,有生于阳分。生于阳分,是感受了风雨寒暑;生于阴分,是由于饮食不节,起居失常,房事过度,喜怒无常。\\
黄帝问:风雨伤人的情况如何?\\
岐伯说:风雨伤人是先侵入皮肤,然后传入孙脉,孙脉充满再传到络脉,络脉充满就注入到大经脉,血气和邪气混杂于分肉腠理之间,其脉象坚大,所以说是实证。实证外表坚实充满,肌肤不能够按触,按触就会疼痛。\\
黄帝又问:寒湿伤人的情况如何?\\
岐伯说:寒湿伤人,会使皮肤拘急,肌肉坚紧,营血凝涩,卫气耗散,所以说是虚证。病虚的人,多是皮肤松弛而有皱纹,卫气不足。按摩就会血脉流畅,则气足而温暖了,所以感觉舒服不痛了。\\
黄帝道:很好!阴分发生的实证是怎样的?\\
岐伯说:多怒不节制,就会使阴气上逆。如果阴气上逆,下部的阴气就要不足,下部的阴气不足,阳气就来凑合,所以说是实证。\\
黄帝又道:阴分发生的虚证是怎样的?\\
岐伯说:喜乐太过,其气下陷;悲哀太过,其气消散。气消耗,血脉就虚了。若再吃寒冷的饮食,寒气趁虚而充满于经脉,就会使血涩滞而气耗散,所以说是虚证。\\
黄帝道:古经上所说的阳虚产生外寒,阴虚产生内热,阳盛产生外热,阴盛产生内寒。我已听到了这种说法,但不知其所以然。\\
岐伯说:诸阳都是受气于上焦的,来温养腠理之间。现在寒气侵袭于外,就会使上焦之气不能达于肤腠之间,上焦之气不能达于肤腠之间,以致寒气独留在外表,所以恶寒战慄。\\
黄帝又问:阴虚产生内热是怎么回事?\\
岐伯说:劳倦过度,形体气力衰疲,谷气不足,上焦不能宣发五谷之味,下脘不能布化五谷之精,胃气郁遏生热,上熏胸中,所以阴虚生内热。\\
黄帝又问:阳盛产生外热是怎么回事?\\
岐伯说:上焦之气不畅通顺利,皮肤紧密,腠理闭塞,汗孔不通,卫气不能发泄外越,所以就发生外热。\\
黄帝又问道:阴盛产生内寒是怎么回事?\\
岐伯说:由于厥逆之气上冲,寒气积在胸中而不得下泄,寒气不泻,使阳气消散,而寒气独留,因而血液疑涩,血液凝涩则脉不通畅,其脉虽盛大却兼涩象,所以成为寒中。\\
黄帝道:阴与阳相混杂,同时又与血气相混杂,病已经形成,刺治的方法应怎样?\\
岐伯说:刺治这样的病证,取其经隧刺之,并刺脉中营血和脉外卫气,同时还要观察病人形体的长短肥瘦和四时气候的不同,而采取或多或少或高或下的刺法。\\
黄帝又道:邪气已经和血气混杂,病形已成,阴阳失去平衡,这时补法和泻法怎样运用呢?\\
岐伯说:泻实的方法是在邪气盛时进针,使针与气一起入内,从而开放邪气外泄的门户。拔针时,要使气和针一同出来,精气不受伤,邪气就会消退。不闭塞针孔,让邪气出尽,这就要摇大针孔,从而通利邪气外出的道路,这就叫大泻。拔针时一定要急出其针,邪气才会退。\\
黄帝又问:补虚的方法又是怎样的?\\
岐伯说:拿着针先不要忙着针刺,必须定神定志。等待病人呼气时下针,呼气出而针入。这样,针孔四围紧密,使精气没有地方外泄。待气正实的时候迅速把针拔出,气入而针出。这样,针下的热气不能随针而出。堵住其散失之路,而邪气散去,人的精气就能保存了。总而言之,在针刺时,不论入针还是出针都要不失时机,使已得之气不致从针孔外泄散失,使未至之气能够引导而来,这就叫做补法。\\
黄帝道:你说虚实有十种,都产生于五脏,具体说是与五脏相联系的五脉。可是人身有十二经脉,能够产生各种病变,现在夫子您只是谈了五脏。那十二经脉,联络人体的三百六十五个气穴,每个气穴有病,必定波及经脉,经脉的病又都有虚实,它们与五脏的虚实关系如何呢?\\
岐伯说:五脏本来和六腑有表里的关系,其经络和支节,各有虚实的病证。根据病变的所在,随时调治。病在脉,可以调治其血;病在血,可以调治其络;病在气,可以调治其卫气;病在肌肉,可以调治其分肉;病在筋,可以调治其筋;病在骨,可以调治其骨。用火针劫刺病处和拘急的地方。如病在骨,可用火针深刺,并用药温熨病处;如病人不知疼痛,最好针刺阳刉阴刉二脉;如人身的形体有疼痛,而九候的脉象没有变化,就用缪刺法治疗;如疼痛在左侧,而右脉出现病象,用巨刺法治疗。必须谨慎审察病人九候的脉象,然后进行针治,这样,针刺的道理就算完备了。\\
卷十八\\
缪刺论篇第六十三\\
黄帝问曰:余闻缪刺,未得其意,何谓缪刺?\\
岐伯对曰:夫邪之客于形也,必先舍于皮毛,留而不去入舍于孙脉,留而不去入舍于络脉,留而不去,入舍于经脉,内连五脏,散于肠胃,阴阳俱感,五脏乃伤。此邪之从皮毛而入,极于五脏之次也,如此则治其经焉。今邪客于皮毛,入舍于孙络,留而不去,闭塞不通,不得入于经,流溢于大络,而生奇病也。夫邪客大络者,左注右,右注左,上下左右,与经相干,而布于四末。其气无常处,不入于经俞,命曰缪刺。\\
帝曰:愿闻缪刺,以左取右,以右取左,奈何?其与巨刺,何以别之?\\
岐伯曰:邪客于经,左盛则右病,右盛则左病,亦有移易者,左痛未已,而右脉先病,如此者,必巨刺之,必中其经,非络脉也。故络病者,其痛与经脉缪处,故命曰缪刺。\\
帝曰:愿闻缪刺奈何?取之何如?\\
岐伯曰:邪客于足少阴之络,令人卒心痛,暴胀,胸胁支满,无积者,刺然骨之前出血,如食顷而已。不已,左取右,右取左。病新发者,取五日,已。\\
邪客于手少阳之络,令人喉痹舌卷,口干心烦,臂外廉痛,手不及头,刺手中指次指爪甲上,去端如韭叶,各一痏,壮者立已,老者有顷已。左取右,右取左。此新病,数日已。\\
邪客于足厥阴之络,令人卒疝暴痛,刺足大指爪甲上与肉交者,各一痏,男子立已,女子有顷已。左取右,右取左。\\
邪客于足太阳之络,令人头项肩痛,刺足小指爪甲上与肉交者,各一痏,立已。不已,刺外踝下三痏。左取右,右取左,如食顷已。\\
邪客于手阳明之络,令人气满胸中,喘息而支胠,胸中热,刺手大指次指爪甲上,去端如韭叶,各一痏。左取右,右取左,如食顷已。\\
邪客于臂掌之间,不可得屈,刺其踝后。先以指按之,痛,乃刺之。以月死生为数,月生一日一痏,二日二痏,十五日十五痏,十六日十四痏。\\
邪气客于足阳\\
之脉,令人目痛,从内眦始,刺外踝之下半寸所,各二痏。左刺右,右刺左,如行十里顷而已。\\
人有所堕坠,恶血留内,腹中满胀,不得前后,先饮利药。此上伤厥阴之脉,下伤少阴之络。刺足内踝之下,然骨之前血脉,出血,刺足跗上动脉。不已,刺三毛上,各一痏,见血立已。左刺右,右刺左。善悲惊不乐,刺如右方。\\
邪客于手阳明之络,令人耳聋,时不闻音,刺手大指次指爪甲上,去端如韭叶,各一痏,立闻。不已,刺中指爪甲上与肉交者,立闻。其不时闻者,不可刺也。耳中生风者,亦刺之如此数。左刺右,右刺左。\\
凡痹往来,行无常处者,在分肉间,痛而刺之。以月死生为数。用针者随气盛衰,以为痏数,针过其日数,则脱气;不及日数,则气不泻。左刺右,右刺左。病已,止。不已,复刺之如法。月生一日一痏,二日二痏,渐多之;十五日十五痏,十六日十四痏,渐少之。\\
邪客于足阳明之络,令人鼽衄、上齿寒,刺足中指次指爪甲上与肉交者,各一痏。左刺右,右刺左。\\
邪客于足少阳之络,令人胁痛不得息,咳而汗出,刺足小指次指爪甲上与肉交者,各一痏。不得息,立已,汗出,立止。咳者,温衣饮食,一日已。左刺右,右刺左,病立已。不已,复刺如法。\\
邪客于足少阴之络,令人嗌痛,不可内食,无故善怒,气上走贲上,刺足下中央之脉,各三痏。凡六刺,立已。嗌中肿,不能内唾,时不能出唾者,缪刺然骨之前,出血立已。左刺右,右刺左。\\
邪客于足太阴之络,令人腰痛,引少腹控尐,不可以仰息,刺腰尻之解,两胂之上,是腰俞。以月死生为痏数,发针立已。左刺右,右刺左。\\
邪客于足太阳之络,令人拘挛背急,引胁而痛,刺之从项始,数脊椎侠脊,疾按之,应手如痛,刺之傍三痏,立已。\\
邪客于足少阳之络,令人留于枢中痛,髀不可举,刺枢中以毫针,寒则久留针。以月死生为数,立已。\\
治诸经刺之,所过者不病,则缪刺之。耳聋,刺手阳明;不已,刺其通脉出耳前者。齿龋,刺手阳明;不已,刺其脉入齿中,立已。\\
邪客于五脏之间,其病也,脉引而痛,时来时止;视其病,缪刺之于手足爪甲上;视其脉,出其血。间日一刺;一刺不已,五刺已。缪传引上齿,齿唇寒痛,视其手背脉血者去之,足阳明中指爪甲上一痏,手大指次指爪甲上各一痏,立已。左取右,右取左。\\
邪客于手足少阴太阴足阳明之络,此五络皆会于耳中,上络左角,五络俱竭,令人身脉皆动,而形无知也,其状若尸,或曰尸厥。刺其足大指内侧爪甲上,去端如韭叶;后刺足心;后刺足中指爪甲上,各一痏;后刺手大指内侧,去端如韭叶;后刺手少阴锐骨之端,各一痏,立已。不已,以竹管吹其两耳立已;不已,剃其左角之发,方一寸,燔治,饮以美酒一杯,不能饮者灌之,立已。\\
凡刺之数,先视其经脉。切而从之,审其虚实而调之。不调者,经刺之;有痛而经不病者,缪刺之。因视其皮部有血络者,尽取之。此缪刺之数也。\\
黄帝问道:我听说有缪刺法,但不知道它的意义,什么叫缪刺?\\
岐伯回答说:邪气侵入人体,必须首先侵入皮毛,滞留不去,就进入孙脉,再滞留不去,就进入络脉,还是滞留不去,就进入经脉,向内牵连五脏,流散到肠胃,这时阴阳表里都受到邪气侵袭,五脏受伤。这是邪气从皮毛而侵入,最终影响到五脏的次序,这样,就要治疗经脉了。现在邪气从皮毛侵入,进入孙脉,滞留不去,内外闭塞不通,邪气不能进入经脉,只流溢到大络之中,从而形成一些异常的疾病。邪气侵入大络后,从左边流窜到右边,从右边流窜到左边,或上或下,或左或右,与经脉相关联,流散到四肢。邪气流窜无一定之处,也不能进入经脉腧穴,这时候就应采取缪刺法。\\
黄帝说:希望听听缪刺法为什么左病右取、右病左取呢?它和巨刺法怎么区别?\\
岐伯说:邪气侵入经脉,左边经气较盛则右边经脉先病,右边经气较盛则左边经脉先病,也有左右转移变易的,左边疼痛未止,而右边经脉已开始有病,像这样,就必须用巨刺法,一定要刺中其经脉,因为它不是络脉的病变。所以络病的病痛部位与经脉所在部位不同,因此称为缪刺。\\
黄帝说:希望听听怎样缪刺?怎样取穴?\\
岐伯说:邪气侵入足少阴经的络脉,使人突然心痛,腹胀大,胸胁部胀满,但并无积聚,针刺小指然谷穴出些血,大约过一顿饭的工夫,病情就可缓解。若尚未好,左病则刺右边,右病则刺左边。新近发生的病,针刺五天就可痊愈。\\
邪气侵入手少阳经的络脉,使人咽喉疼痛痹塞,舌卷,口干,心中烦闷,手臂外侧疼痛,举手不能至头,针刺无名指指甲上方,距离指甲如韭菜叶宽的关冲穴,各刺一针,壮年人马上就见缓解,老年人待一会儿也会好。左病则刺右边,右病则刺左边,这是新近发生的病,几天就可痊愈。\\
邪气侵袭足厥阴经的络脉,使人突然发生疝气,剧烈疼痛,针刺足大趾爪甲上与皮肉交接处的大敦穴,各刺一针,男子立刻缓解,女子待一会儿也会好。左病则刺右边,右病则刺左边。\\
邪气侵袭足太阳经的络脉,使人头项肩部疼痛,针刺足小趾爪甲上与皮肉交接处的至阴穴,各刺一针,立刻就缓解。若不缓解,再刺外踝下的金门穴三针。左病则刺右边,右病则刺左边,大约一顿饭的工夫也就好了。\\
邪气侵袭手阳明经的络脉,使人胸中气满,喘息而胁肋撑胀,胸中发热,针刺手大指侧的次指指甲上方,距离指甲如韭菜叶宽的商阳穴,各刺一针。左病则刺右边,右病则刺左边,大约一顿饭的工夫病就好了。\\
邪气侵入手厥阴经的络脉,使人臂掌之间疼痛,不能弯曲,针刺手腕后方。先用手指按压,找到痛处,再用针刺。根据月亮的圆缺确定针刺的次数,月亮初生,初一刺一针,初二刺二针,以后逐日加一针,直到十五日加到十五针,十六日又减为十四针,以后逐日减一针。\\
邪气侵入足部的阳刉脉,使人眼睛疼痛,从内眦开始,针刺外踝下面约半寸处的申脉穴,各刺二针。左病则刺右边,右病则刺左边,大约需一般人步行十里路的工夫就可以好了。\\
人由于堕坠跌伤,淤血停留体内,使人腹部胀满,大小便不通,要先服通便导痰的药物。这种病上面伤了厥阴的经脉,下面伤了少阴经的络脉。针刺足内踝之下、然骨之前的血脉,刺出其血,再刺足背上的动脉。若病不缓解,再刺足大趾三毛处的大敦穴各一针,出血后立即就会缓解。左病则刺右边,右病则刺左边。如有悲伤或惊恐不乐的,刺法同上。\\
邪气侵入手阳明经的络脉,使人耳聋,有时听不到声音,针刺手大指侧的次指指甲上方,距离指甲如韭叶宽的商阳穴各一针,立刻就可听到声音。若不见效,再刺中指爪甲上与皮肉交接处的中冲穴,马上就可听到声音。如果是完全听不到声音的,就不可针刺了。耳中鸣响,若有风声,也可以用上述方法针刺。左病则刺右边,右病则刺左边。\\
凡是痹证疼痛往来,无固定地方的,就随疼痛所在而刺其分肉之间,根据月亮盈亏变化确定针刺的次数。用针刺治疗要随着人体在月周期中气血的盛衰情况来确定用针的次数,如果用针次数超过其相应的日数,会损耗人的正气;如果达不到相应的日数,邪气又不能泻除。左病则刺右边,右病则刺左边。病愈停针,若还没有痊愈,按上述方法再刺。月亮新生的初一刺一针,初二刺两针,逐日加多;十五日加至十五针,十六日又减至十四针,逐日减少。\\
邪气侵入足阳明经的络脉,使人鼻塞,衄血,上齿寒冷,针刺足中趾侧的次趾爪甲上方与皮肉交接处的厉兑穴,各刺一针。左病则刺右边,右病则刺左边。\\
邪气侵入足少阳经的络脉,使人胁痛而不能喘息,咳嗽而汗出,针刺足小趾侧的次趾爪甲上方与皮肉交接处的窍阴穴,各刺一针。不能喘息,马上就缓解,出汗也就马上停止,若咳嗽,要注意衣服饮食饱暖,一天可愈。左病则刺右边,右病则刺左边,疾病马上可痊愈。如果仍未痊愈,按上述方法再刺。\\
邪气侵入足少阴经的络脉,使人咽喉疼痛,不能进食,无故发怒,气上逆直至贲门之上,针刺足心的涌泉穴,左右各三针。共六针,可立刻缓解。咽喉肿痛,不能咽唾,有时唾沫不能吐出,针刺然骨之前,使之出血,很快就好。左病则刺右边,右病则刺左边。\\
邪气侵入足太阴经的络脉,使人腰痛牵连少腹,牵引至胁下,不能挺胸呼吸,针刺腰骶骨节和夹脊肌肉之上方的腰俞穴。根据月亮圆缺确定用针的次数,出针后马上就好了。左病则刺右边,右病则刺左边。\\
邪气侵入足太阳经的络脉,使人背部拘急,牵引胁肋部疼痛,针刺应从项部开始沿着脊骨两旁向下按压,如果按压较重即应手而痛的,就在痛处周围针刺三针,病立刻就好。\\
邪气侵入足少阳经的络脉,使人环跳部疼痛,大腿不能举动,用毫针刺环跳穴,有寒的可长时间留针。根据月亮盈亏确定针刺的次数,很快就好。\\
治疗各经疾病可用针刺,如果经脉所过之处没有病变,就用缪刺法。耳聋针刺手阳明经的商阳穴;如果不好,再刺其经脉走向耳前的听宫穴。蛀牙病刺手阳明经的商阳穴;如果不好,再刺其走入齿中的经络,很快就见效。\\
邪气侵入到五脏之间,其病变为经脉牵引作痛,时痛时止;根据病情,在手足爪甲上缪刺;发现有血液郁滞的络脉,刺出其血。隔日刺一次,一次不见好,连刺五次就可好了。阳明经脉有病气交叉感传而牵引上齿,出现唇齿寒冷疼痛,可诊视其手背上经脉有淤血的地方针刺出血,再在足阳明中趾爪甲上刺一针,在手大指侧的次指爪甲上的商阳穴各刺一针,很快就好了。左病则刺右边,右病则刺左边。\\
邪气侵入到手少阴、手太阴、足少阴、足太阴和足阳明的络脉,这五经的络脉都聚会于耳中,并上绕左耳上面的额角,如果邪气侵袭而致此五络的真气全部衰竭,就会使全身经脉都振动,而形体没有知觉,像死尸一样,有人把它叫做尸厥。这时应当针刺其足大趾内侧爪甲上距离爪甲有韭叶宽的隐白穴;然后再刺足心的涌泉穴;再刺足中趾爪甲上的厉兑穴,各刺一针;然后再刺手大指内侧距离爪甲有韭叶宽的少商穴;再刺手少阴经在掌后锐骨端的神门穴,各刺一针,当立刻清醒。如仍不好,就用竹管吹病人两耳之中,就立刻会好;如果不好,把病人左边头角上的头发剃下来,取一方寸左右,烧制为末,用美酒一杯冲服,不能自己饮服的,就把药酒灌下去,很快就可恢复过来。\\
大凡刺治的方法;先诊视所病的经脉。切按推寻,详审虚实而调治。如果经络不调,先采用经刺的方法;如果有疼痛而经脉没有病变,再采用缪刺的方法,要看其皮部是否有淤血的络脉,若有应全部把淤血刺出。这就是缪刺的方法。\\
四时刺逆从论篇第六十四\\
厥阴有余,病阴痹;不足,病生热痹;滑则病狐疝风;涩则病少腹积气。\\
少阴有余,病皮痹隐轸;不足,病肺痹;滑则病肺风疝,涩则病积溲血。\\
太阴有余,病肉痹寒中;不足,病脾痹;滑则病脾风疝,涩则病积,心腹时满。\\
阳明有余,病脉痹,身时热;不足,病心痹;滑则病心风疝,涩则病积,时善惊。\\
太阳有余,病骨痹身重;不足病肾痹;滑则病肾风疝,涩则病积,善时巅疾。\\
少阳有余,病筋痹胁满;不足,病肝痹;滑则病肝风疝;涩则病积,时筋急目痛。\\
是故春,气在经脉;夏,气在孙络;长夏,气在肌肉;秋,气在皮肤;冬,气在骨髓中。\\
帝曰:余愿闻其故。\\
岐伯曰:春者,天气始开,地气始泄,冻解冰释,水行经通,故人气在脉。夏者,经满气溢,入孙络受血,皮肤充实。长夏者,经络皆盛,内溢肌中。秋者,天气始收,腠理闭塞,皮肤引急。冬者盖藏,血气在中,内著骨髓,通于五脏。是故邪气者,常随四时之气血而入客也,至其变化,不可为度,然必从其经气,辟除其邪。除其邪,则乱气不生。\\
帝曰:逆四时而生乱气,奈何?\\
岐伯曰:春刺络脉,血气外溢,令人少气;春刺肌肉,血气环逆,令人上气;春刺筋骨,血气内著,令人腹胀。夏刺经脉,血气乃竭,令人解汃;夏刺肌肉,血气内却,令人善恐;夏刺筋骨,血气上逆,令人善怒。秋刺经脉,血气上逆,令人善忘;秋刺络脉,气不外行,令人卧、不欲动;秋刺筋骨,血气内散,令人寒栗。冬刺经脉,血气皆脱,令人目不明;冬刺络脉,内气外泄,留为大痹;冬刺肌肉,阳气竭绝,令人善忘。凡此四时刺者,大逆之病,不可不从也;反之,则生乱气,相淫病焉。故刺不知四时之经,病之所生,以从为逆,正气内乱,与精相薄。必审九候,正气不乱,精气不转。\\
帝曰:善!\\
刺五脏,中心,一日死,其动为噫;中肝,五日死,其动为语;中肺,三日死,其动为咳;中肾,六日死,其动为嚏欠;中脾,十日死,其动为吞。刺伤人五脏,必死。其动则依其脏之所变,候知其死也。\\
厥阴之气有余,可以发生阴痹;不足,则发生热痹;气血过于滑利则患狐风疝;气血运行涩滞则形成少腹中有积气。\\
少阴之气有余,可以发生皮痹和瘾疹;不足,则发生肺痹;气血过于滑利则患肺风疝,气血运行涩滞则病积聚和尿血。\\
太阴之气有余,可以发生肉痹和寒中;不足,则发生脾痹;气血过于滑利则患脾风疝,气血运行涩滞则病积聚和心腹胀满。\\
阳明之气有余,可以发生脉痹,身体时时发热;不足,则发生心痹;气血过于滑利则患心风疝,气血运行涩滞则病积聚和不时惊恐。\\
太阳之气有余,可以发生骨痹、身体沉重;不足,则发生肾痹;气血过于滑利则患肾风疝,气血运行涩滞则病积聚,且时时发生巅顶部疾病。\\
少阳之气有余,可以发生筋痹和胁肋满闷;不足,则发生肝痹;气血过于滑利则患肝风疝,气血涩滞则病积聚,有时发生筋脉拘急和眼目疼痛等。\\
所以春天,人的气血在经脉;夏天,人的气血在孙络;长夏,人的气血在肌肉;秋天,人的气血在皮肤;冬天,人的气血在骨髓中。\\
黄帝说:我希望听听其中的道理。\\
岐伯说:春季,天之阳气开始萌动,地之阴气也开始发泄,冬天的冰冻逐渐融化消释,水道通行,所以人的气血也集中在经脉中流行。夏季,经脉中气血充满而流溢到孙络,孙络接受了气血,皮肤也充实了。长夏季,经脉和络脉都旺盛,能充分地灌溉到肌肉中。秋季,天气开始收敛,腠理闭塞,皮肤也收缩紧密起来。冬季主闭藏,气血储藏在内,聚集于骨髓,通达五脏。所以邪气总是随着四时气血的变化而侵入人体,至于邪气的变化,就难以预测了,但必须顺应四时经气的变化,而驱除邪气。驱除了邪气,气血就不会逆乱了。\\
黄帝问:针刺违背了四时而致气血逆乱,情况是怎样的?\\
岐伯说:春天刺络脉,会使血气向外散溢,使人少气无力;春天刺肌肉,会使血气循环逆乱,使人上气咳喘;春天刺筋骨,会使血气留着在内,使人腹胀。夏天刺经脉,会使血气衰竭,使人疲倦懈惰;夏天刺肌肉,会使血气却弱于内,使人易于恐惧;夏天刺筋骨,会使血气上逆,使人易于发怒。秋天刺经脉,会使血气上逆,使人易于忘事;秋天刺络脉,使人阳气不足不能卫外而嗜卧懒动;秋天刺筋骨,会使血气耗散于内,使人寒战。冬天刺经脉,会使血气虚脱,使人目视不明;冬天刺络脉,会使血气外泄而成大痹;冬天刺肌肉,会使阳气竭绝于外,使人易于忘事。以上这些四时的刺法,如悖逆四时变化都能致病,所以不能不顺应四时变化而施刺;否则就会产生逆乱之气,在体内蔓延而生病。所以针刺不懂得四时经气的盛衰和疾病之所以产生的道理,不顺应四时而违背四时变化,致使正气乱于内,邪气便与精气搏结为病。一定要仔细审察九候的脉象,使正气不逆乱,邪气也不会与精气相搏结了。\\
黄帝说:讲得好!\\
针刺五脏,误中心脏一天就会死亡,其变动为嗳气;误中肝脏五天就会死亡,其变动为多语;误中肺脏三天就会死亡,其变动为咳嗽;误中肾脏六天就会死亡,其变动为喷嚏和呵欠;误中脾脏十天就会死亡,其变动为吞咽。刺伤了人的五脏,必致死亡,其变动的征象则随所伤之脏而有不同变化,可以据此来测知死亡的日期。\\
标本病传论篇第六十五\\
黄帝问曰:病有标本,刺有逆从,奈何?\\
岐伯对曰:凡刺之方,必别阴阳,前后相应,逆从得施,标本相移。故曰:有其在标而求之于标,有其在本而求之于本,有其在本而求之于标,有其在标而求之于本。故治有取标而得者,有取本而得者,有逆取得者,有从取而得者。故知逆与从,正行无问,知标本者,万举万当;不知标本,是谓妄行。\\
夫阴阳、逆从、标本之为道也,小而大,言一而知百病之害;少而多,浅而博,可以言一而知百也。以浅而知深,察近而知远,言标与本,易而勿及。\\
治反为逆,治得为从。先病而后逆者治其本,先逆而后病者治其本,先寒而后生病者治其本,先病而后生寒者治其本,先热而后生病者治其本,先热而后生中满者治其标,先病而后泄者治其本,先泄而后生他病者治其本。必且调之,乃治其他病。先病而后生中满者治其标,先中满而后烦心者治其本。人有客气,有同气。小大不利治其标,小大利治其本。病发而有余,本而标之,先治其本,后治其标。病发而不足,标而本之,先治其标,后治其本。谨察间甚,以意调之,间者并行,甚者独行。先小大不利而后生病者治其本。\\
夫病传者,心病,先心痛,一日而咳,三日胁支痛;五日,闭塞不通,身痛体重,三日不已,死。冬夜半,夏日中。\\
肺病,喘咳,三日而胁支满痛;一日身重体痛,五日而胀,十日不已,死。冬日入,夏日出。\\
肝病,头目眩,胁支满,三日体重身痛,五日而胀,三日腰脊少腹痛,胫痠,三日不已,死。冬日入,夏早食。\\
脾病,身痛体重,一日而胀,二日少腹腰脊痛,胫痠,三日背伓筋痛,小便闭;十日不已,死。冬人定,夏晏食。\\
肾病,少腹腰脊痛,气痠,三日背伓筋痛,小便闭;三日腹胀;三日两胁支痛;三日不己,死。冬大晨,夏晏晡。\\
胃病,胀满,五日少腹腰脊痛,气痠,三日背伓筋痛,小便闭;五日身体重;六日不已,死。冬夜半后,夏日昳。\\
膀胱病,小便闭,五日少腹胀,腰脊痛,气痠;一日腹胀;一日身体重;二日不已,死。冬鸡鸣,夏下晡。\\
诸病以次相传,如是者皆有死期,不可刺;间一脏止,及至三四脏者,乃可刺也。\\
黄帝问道:病有标病本病,刺法有逆治从治,是怎么回事?\\
岐伯回答说:大凡针刺的原则,必定要先辨别疾病的阴阳属性,把病情的前期和后期联系起来研究,然后确定是用逆治还是从治,治标还是治本。所以说:有的病在标而治标,有的病在本而治本,有的病在本而治标,有的病在标而治本。所以在治疗上,有治标而取效的,有治本而取效的,有反治而取效的,有正治而取效的。所以懂得了治疗的逆从法则,那么就可以放手治疗而无所疑虑;懂得了治标治本的法则,就能屡试不爽,万无一失;如果不懂得标本,这叫胡乱施治。\\
阴与阳、逆与从、标与本,作为一种原则,可以使人由小到大地认识疾病,从某一点,就能知道各种疾病的害处;还能由少到多,由浅到博,从一种疾病而推知各种疾病。从浅就能知深,察近就能知远,谈论标与本的道理,这两个字容易理解,但真正掌握与熟练运用却不容易做到。\\
背逆病情而治的为逆治,顺从病情而治的为从治。先患某病,然后发生气血逆乱的,治疗它的本病;若先气血不和,然后才患病的,也应先治其本病;先感受寒邪而后发生其他病变的,应当先治其本;先患病而后生寒变的,也当先治其本病;先患热病而后发生其他病变的,应当治其本病;先患热病而后生胸腹胀满的,应先治它的标病;先患病而后发生泄泻的,应先治其本病;先患泄泻而后又生其他病的,应先治它的本病。一定得先把泄泻治好,才可治疗其他病证。先患病而后发生中满的,应当先治它的标病;先患胸腹胀满证,而后又增加了心烦不舒的,应当治其本病。人体有邪气,也有真气。大小便不利的,应当先治其标病;大小便通利的,应当先治其本病。若发病表现为有余的实证,应当用本而标之的治法,即先治其本,后治其标;如发病表现为不足的虚证,应当用标而本之的治法,即先治其标,后治其本。要谨慎地观察病情的轻重,根据具体病情而进行治疗,病轻的可以标本兼治,病重的就要根据病情,或治本或治标。先大小便不通利,而后并发其他疾病的,应当先治其本病。\\
疾病的传变规律,心病先发心痛,过一天病传入肺而咳嗽,再过三天病传入肝而胁肋胀痛;再过五天病传入脾而大便闭塞不通,身体疼痛沉重,再过三天不愈,就要死亡。冬天死于半夜,夏天死于中午。\\
肺病先见喘咳,三天不好病传入肝,则胁肋胀满疼痛;再过一天病邪传入脾,则身体沉重疼痛,再过五天病邪传入胃,则腹胀;再过十天不愈,就要死亡。冬天死于日落之时,夏天死于日出之时。\\
肝病先见头痛目眩,胁肋胀满,三天后病传入脾而身体沉重疼痛,再过五天病传入胃,产生腹胀,再过三天腰脊少腹疼痛,膝胫痠软,再过三天不愈,就要死亡。冬天死于日落的时候,夏天死于吃早饭的时候。\\
脾病先见身体沉重疼痛,一天后病邪传入于胃而腹胀,再过两天病邪传入肾,则少腹腰脊疼痛,膝胫痠软,再过三天病邪传入膀胱,见背脊筋骨间疼痛,小便不通;再过十天不愈,就要死亡。冬天死于申时之后,夏天死于寅时之后。\\
肾病,先见少腹腰脊疼痛,膝胫痠软,三天后病邪传入膀胱,则背脊筋骨疼痛,小便不通;再过三日病邪传入胃,产生腹胀;再过三天病邪传入肝,见两胁胀痛;再过三天不愈,就要死亡。冬天死于天亮,夏天死于黄昏。\\
胃病,先见腹部胀满,五天后病邪传入肾,则少腹腰脊疼痛,膝胫痠软,再过三天病邪传入膀胱,见背脊筋骨疼痛,小便不通;再过五天病邪传入脾,则身体沉重;再过六天不愈,就要死亡。冬天死于夜半后,夏天死于午后。\\
膀胱发病,先见小便不通,五日后病邪传入肾,则少腹胀满,腰脊疼痛,膝胫痠软;再过一天病邪传入胃,则腹胀;再过一天病邪传入脾,见身体沉重;再过两天不愈,就要死亡。冬天死于半夜后,夏天死于下午。\\
各种疾病按次序相传,像上面所说的这些,都有一定的死期,不可以用针刺治疗;如果是间脏相传或间三脏、四脏,还是可以用针刺治疗的。\\
卷十九\\
天元纪大论篇第六十六\\
黄帝问曰:天有五行,御五位,以生寒、暑、燥、湿、风。人有五脏,化五气,以生喜、怒、思、忧、恐。《论》言:五运相袭而皆治之,终期之日,周而复始。余已知之矣,愿闻其与三阴三阳之候奈何合之?\\
鬼臾区稽首再拜对曰:昭乎哉问也!夫五运阴阳者,天地之道也,万物之纲纪,变化之父母,生杀之本始,神明之府也,可不通乎!故物生谓之化,物极谓之变,阴阳不测谓之神,神用无方谓之圣。夫变化之为用也,在天为玄,在人为道,在地为化。化生五味,道生智,玄生神。神在天为风,在地为木;在天为热,在地为火,在天为湿,在地为土;在天为燥,在地为金;在天为寒,在地为水。故在天为气,在地成形,形气相感而化生万物矣。然天地者,万物之上下也;左右者,阴阳之道路也;水火者,阴阳之征兆也;金木者,生成之终始也。气有多少,形有盛衰,上下相召,而损益彰矣。\\
帝曰:愿闻五运之主时也何如?\\
鬼臾区曰:五气运行,各终期日,非独主时也。\\
帝曰:请闻其所谓也。\\
鬼臾区曰:臣积考《太始天元册》文曰:太虚寥廓,肇基化元,万物资始,五运终天,布气真灵,摠统坤元。九星悬朗,七曜周旋,曰阴曰阳,曰柔曰刚。幽显既位,寒暑弛张。生生化化,品物咸章。臣斯十世,此之谓也。\\
帝曰:善。何谓气有多少,形有盛衰?\\
鬼臾区曰:阴阳之气,各有多少,故曰三阴三阳也。形有盛衰,谓五行之治,各有太过不及也。故其始也,有余而往,不足随之;不足而往,有余从之。知迎知随,气可与期。应天为天符,承岁为岁直,三合为治。\\
帝曰:上下相召,奈何?\\
鬼臾区曰:寒暑燥湿风火,天之阴阳也,三阴三阳上奉之。木火土金水火,地之阴阳也,生长化收藏下应之。天以阳生阴长,地以阳杀阴藏。天有阴阳,地亦有阴阳。故阳中有阴,阴中有阳。所以欲知天地之阴阳者。应天之气,动而不息,故五岁而右迁;应地之气,静而守位,故六期而环会。动静相召,上下相临,阴阳相错,而变由生也。\\
帝曰:上下周纪,其有数乎?\\
鬼臾区曰:天以六为节,地以五为制。周天气者,六期为一备;终地纪者,五岁为一周。君火以明,相火以位。五六相合,而七百二十气为一纪,凡三十岁;千四百四十气,凡六十岁而为一周。不及太过,斯皆见矣。\\
帝曰:夫子之言,上终天气,下毕地纪,可谓悉矣。余愿闻而藏之,上以治民,下以治身,使百姓昭著,上下和亲,德泽下流,子孙无忧,传之后世,无有终时。可得闻乎?\\
鬼臾区曰:至数之机,迫迮以微,其来可见,其往可追,敬之者昌,慢之者亡,无道行私,必得夭殃,谨奉天道,请言真要。\\
帝曰:善言始者,必会于终。善言近者,必知其远。是则至数极,而道不惑,所谓明矣。愿夫子推而次之,令有条理,简而不匮,久而不绝,易用难忘,为之纲纪。至数之要,愿尽闻之。\\
鬼臾区曰:昭乎哉问!明乎哉道!如鼓之应桴,响之应声也。臣闻之,甲己之岁,土运统之;乙庚之岁,金运统之;丙辛之岁,水运统之;丁壬之岁,木运统之;戊癸之岁,火运统之。\\
帝曰:其于三阴三阳,合之奈何?\\
鬼臾区曰:子午之岁,上见少阴;丑未之岁,上见太阴;寅申之岁,上见少阳;卯酉之岁,上见阳明;辰戌之岁,上见太阳;巳亥之岁,上见厥阴。少阴所谓标也,厥阴所谓终也。厥阴之上,风气主之;少阴之上,热气主之;太阴之上,湿气主之;少阳之上,相火主之;阳明之上,燥气主之;太阳之上,寒气主之。所谓本也,是谓六元。\\
帝曰:光乎哉道!明乎哉论!请著之玉版,藏之金匮,署曰《天元纪》。\\
黄帝问道:天有五行,统率东、南、西、北、中五方之位,产生寒、暑、燥、湿、风的气候变化。人有五脏,化生五气,产生喜、怒、思、忧、恐。《六节脏象论》说道:五运之气相承袭,都有其固定的顺序,到岁终的那一天是一个周期,然后开始新的循环。这些道理我已经了解了,希望再听听五运与三阴三阳这六气是怎样结合的?\\
鬼臾区恭敬地行了两次礼回答说:你问得很明确啊!五运阴阳是天地自然的根本规律,是一切事物的纲领,是千变万化的起源,是生长、毁灭的根本,是天地万物神奇变化的内在动力,怎能不通晓它!所以万物的生长称为“化”,生长发展到极端叫“变”,阴阳的变化不可测度叫“神”,神的作用变化没有方所叫“圣”。神明变化的作用,在天就是深奥不测的宇宙,在人就是社会人事的道理,在地就是万物的化生。地能够化生,就产生了万物的五味;人明白了道理,就产生了智慧;天的深奥不测,就产生了神明。而神明变化,在天为风,在地为木;在天为热,在地为火;在天为湿,在地为土;在天为燥,在地为金;在天为寒,在地为水。总之在天为无形的六气,在地为有形的五行,形气相互交感,就能化生万物了。然而,天地是万物的上下范围,左右是阴阳升降的道路,水火是阴阳的表现,秋春是生长收成的终结与开始。气有多少的不同,形有盛衰的分别,形气相互交感,或者衰弱或者强盛的现象,也就很明显了。\\
黄帝道:我想听听五运主四时的情况如何?\\
鬼臾区说:五气运行,每气各尽一年的三百六十五日,并不是仅仅主四时的。\\
黄帝又问道:希望听听其中的缘由。\\
鬼臾区说:我查考了《太始天元册》,上面说:广阔无垠的天空,是化生万物的基础,万物依靠它开始成长,五运终而复始地运行于宇宙之中,敷布真灵之气,统摄着作为万物生长之根本的坤元。九星悬挂辉耀,七曜环绕旋转,在天产生了阴与阳的变化,在地有了柔与刚的分别。昼夜的明暗有了固定的规律,四时寒暑更替有常。这样生化不息,万物自然就繁荣昌盛了。我家已经十世相传,就是前面所讲的这些道理。\\
黄帝说:讲得好。什么叫做气有多少、形有盛衰呢?\\
鬼臾区说:阴气与阳气,各有多少不同,所以才分别为三阴三阳。形有盛衰,是说五行主治岁运,各有太过与不及的情况。如果开始为太过,接下来的一运就是不及,开始不及,下一运就是太过。明白了有余不足的道理,也就可以推到运气的循环往复。凡中运与司天之气相符的叫“天符”,与该岁的年支之气相同的叫“岁直”,若既是天符又为岁直的就是“三合”,就算是“治”了。\\
黄帝问:天地阴阳上下相召,是怎么回事?\\
鬼臾区说:寒、暑、燥、湿、风、火,是天的阴阳,三阴三阳与之相应,木、火、土、金、水、火,是地的阴阳,生长化收藏与之相应。天是以阳生阴长的,地是以阳杀阴藏的。天有阴阳,地也有阴阳。天地相合,则阳中有阴,阴中有阳。这就是我们要知道天地之阴阳的原因。与天之六气相应的五运是运动不息的,因此经过五年就右迁一步;与地之五运相应的六气是比较静止的,所以经过六年才循环一周。天地动静相互影响,上下相合,阴阳交错,运气的变化就产生了。\\
黄帝问:天地上下往复周旋,有没有一定的规律呢?\\
鬼臾区说:天以六气为节,地以五行为制。六气司天,需要六年才能循环一周;五运制地,需要五年才能循环一周。因为君火主宰神明,只有相火主运。五与六相合,计三十年中共有七百二十个节气,称为一纪,经过一千四百四十个节气,计六十年就成为甲子一周。这样不及与太过,都可以显现出来了。\\
黄帝说:夫子所讲的,上说完了天气,下说完了地理,可以说是极其详细了。我想把这些听到的道理牢记在心里,上以治疗百姓的疾病,下以保养自己的身心健康,使百姓明白这个道理,上下相和相亲,使恩德泽及广大民众,使子孙后代无忧无虑,把它传到千秋万代,永无终止,能不能让我听听呢?\\
鬼臾区说:五运六气相合的规律,切近深细而且微妙,它的到来可以察见,它的过去可以追寻。敬重这种规律的人,就能保持健康,忽视了它,就会身受灾害,甚至死亡,违逆了五运六气的规律,任由私心而妄为,必然会有祸殃,所以必须要谨慎地适应五运六气的自然天道。请让我来讲一讲它的真理要道吧。\\
黄帝说:善于谈论事物起源的人,必然也能够知道它的结果。善于谈论近处事情的人,必定也知道推及远处的事理。只有这样,对五运六气的道理才不会感到困惑,对其具体方术才能深刻地把握,这就是所谓的彻底明了的境界。希望夫子推论排比,使之有条理,简单而无遗漏,长久流传而不会断绝,容易掌握运用而不易忘记,成为医道的纲领。五运六气的至理要道,希望全都听听。\\
鬼臾区说:问得真高明啊!五运六气的道理真是明明白白啊!好像鼓槌敲在鼓上一样,又好像回声对声音的回应。我听说,甲年和己年由土运统领;乙年和庚年,由金运统领;丙年和辛年,由水运统领;丁年和壬年,由木运统领;戊年和癸年,由火运统领。\\
黄帝说:五运与三阴三阳是怎样配合的?\\
鬼臾区说:子年午年都是少阴司天;丑年未年都是太阴司天;寅年申年都是少阳司天;卯年酉年都是阳明司天;辰年戌年都是太阳司天;巳年亥年都是厥阴司天。年支阴阳的次序以子年为始,亥年为终,所以少阴为首,厥阴为终。厥阴以风气为主;少阴以热气为主;太阴以湿气为主;少阳以相火为主;阳明以燥气为主;太阳以寒气为主。因为风热湿火燥寒是三阴三阳的本气,它是天元一气化之为六,所以称为六元。\\
黄帝又说:道理真是光明伟大啊!论述得真是明明白白啊!请让我把它刻在玉版上,保存在金匮里,署名叫《天元纪》。\\
五运行大论篇第六十七\\
黄帝坐明堂,始正天纲,临观八极,考建五常,请天师而问之曰:论言天地之动静,神明为之纪。阴阳之升降,寒暑彰其兆。余闻五运之数于夫子,夫子之所言,正五气之各主岁尔,首甲定运,余因论之。\\
鬼臾区曰:土主甲己,金主乙庚,水主丙辛,木主丁壬,火主戊癸。子午之上,少阴主之;丑未之上,太阴主之;寅申之上,少阳主之;卯酉之上,阳明主之;辰戌之上,太阳主之;巳亥之上,厥阴主之。不合阴阳,其故何也?\\
岐伯曰:是明道也,此天地之阴阳也。夫数之可数者,人中之阴阳也,然所合,数之可得者也。夫阴阳者,数之可十,推之可百,数之可千,推之可万。天地阴阳者,不以数推,以象之谓也。\\
帝曰:愿闻其所始也。\\
岐伯曰:昭乎哉问也!臣览《太始天元册》文,丹天之气,经于牛女戊分;黅天之气,经于心尾己分;苍天之气,经于危室柳鬼;素天之气,经于亢氐昴毕;玄天之气,经于张翼娄胃。所谓戊己分者,奎壁角轸,则天地之门户也。夫候之所始,道之所生,不可不通也。\\
帝曰:善。《论》言天地者,万物之上下;左右者,阴阳之道路,未知其所谓也。\\
岐伯曰:所谓上下者,岁上下见阴阳之所在也。左右者,诸上见厥阴,左少阴,右太阳;见少阴,左太阴,右厥阴;见太阴,左少阳,右少阴;见少阳,左阳明,右太阴;见阳明,左太阳,右少阳;见太阳,左厥阴,右阳明。所谓面北而命其位,言其见也。\\
帝曰:何谓下?\\
岐伯曰:厥阴在上,则少阳在下,左阳明,右太阴;少阴在上,则阳明在下,左太阳,右少阳;太阴在上,则太阳在下,左厥阴,右阳明,少阳在上,则厥阴在下,左少阴,右太阳;阳明在上,则少阴在下,左太阴,右厥阴;太阳在上,则太阴在下,左少阳,右少阴。所谓面南而命其位,言其见也。上下相遘,寒暑相临,气相得则和,不相得则病。\\
帝曰:气相得而病者,何也?\\
岐伯曰:以下临上,不当位也。\\
帝曰:动静何如?\\
岐伯曰:上者右行,下者左行,左右周天,余而复会也。\\
帝曰:余闻鬼臾区曰:应地者静。今夫子乃言下者左行,不知其所谓也,愿闻何以生之乎?\\
岐伯曰:天地动静,五行迁复,虽鬼臾区其上候而已,犹不能遍明。夫变化之用,天垂象,地成形,七曜纬虚,五行丽地。地者,所以载生成之形类也;虚者,所以列应天之精气也。形精之动,犹根本之与枝叶也,仰观其象,虽远可知也。\\
帝曰:地之为下,否乎?\\
岐伯曰:地为人之下,太虚之中者也。\\
帝曰:冯乎?\\
岐伯曰:大气举之也。燥以干之,暑以蒸之,风以动之,湿以润之,寒以坚之,火以温之。故风寒在下,燥热在上,湿气在中,火游行其间。寒暑六入,故令虚而生化也。故燥胜则地干,暑胜则地热,风胜则地动,湿胜则地泥,寒胜则地裂,火胜则地固矣。\\
帝曰:天地之气,何以候之?\\
岐伯曰:天地之气,胜复之作,不形于诊也。《脉法》曰:天地之变,无以脉诊。此之谓也。\\
帝曰:间气何如?\\
岐伯曰:随气所在,期于左右。\\
帝曰:期之奈何?\\
岐伯曰:从其气则和,违其气则病,不当其位者病,迭移其位者病,失守其位者危,尺寸反者死,阴阳交者死。先立其年,以知其气,左右应见,然后乃可以言死生之逆顺。\\
帝曰:寒暑燥湿风火,在人合之,奈何?其于万物,何以生化?\\
岐伯曰:东方生风,风生木,木生酸,酸生肝,肝生筋,筋生心。其在天为玄,在人为道,在地为化。化生五味,道生智,玄生神,化生气。神在天为风,在地为木,在体为筋,在气为柔,在脏为肝。其性为暄,其德为和,其用为动,其色为苍,其化为荣,其虫毛,其政为散,其令宣发,其变摧拉,其眚为陨,其味为酸,其志为怒。怒伤肝,悲胜怒;风伤肝,燥胜风;酸伤筋,辛胜酸。\\
南方生热,热生火,火生苦,苦生心,心生血,血生脾。其在天为热,在地为火,在体为脉,在气为息,在脏为心。其性为暑,其德为显,其用为躁,其色为赤,其化为茂,其虫羽,其政为明,其令郁蒸,其变炎烁,其眚燔焫,其味为苦,其志为喜。喜伤心,恐胜喜;热伤气,寒胜热;苦伤气,咸胜苦。\\
中央生湿,湿生土,土生甘,甘生脾,脾生肉,肉生肺。其在天为湿,在地为土,在体为肉,在气为充,在脏为脾。其性静兼,其德为濡,其用为化,其色为黄,其化为盈,其虫倮,其政为谧,其令云雨,其变动注,其眚淫溃,其味为甘,其志为思。思伤脾,怒胜思;湿伤肉,风胜湿;甘伤脾,酸胜甘。\\
西方生燥,燥生金,金生辛,辛生肺,肺生皮毛,皮毛生肾。其在天为燥,在地为金,在体为皮毛,在气为成,在脏为肺。其性为凉,其德为清,其用为固,其色为白,其化为敛,其虫介,其政为劲,其令雾露,其变肃杀,其眚苍落,其味为辛,其志为忧。忧伤肺,喜胜忧;热伤皮毛,寒胜热;辛伤皮毛,苦胜辛。\\
北方生寒,寒生水,水生咸,咸生肾,肾生骨髓,髓生肝。其在天为寒,在地为水,在体为骨,在气为坚,在脏为肾。其性为凛,其德为寒,其用为藏,其色为黑,其化为肃,其虫麟,其政为静,其令霰雪,其变凝冽,其眚冰雹,其味为咸,其志为恐。恐伤肾,思胜恐;寒伤血,燥胜寒;咸伤血,甘胜咸。\\
五气更立,各有所先,非其位则邪,当其位则正。\\
帝曰:病生之变,何如?\\
岐伯曰:气相得则微,不相得则甚。\\
帝曰:主岁何如?\\
岐伯曰:气有余,则制己所胜,而侮所不胜;其不及,则己所不胜侮而乘之,己所胜轻而侮之。侮反受邪,侮而受邪,寡于畏也。\\
帝曰:善。\\
黄帝坐在明堂里,开始校正天文,观看八方地理,研究五行运气的阴阳变化,请岐伯来,向他问道:经论中说天地的动静,是由自然的内在动力所控制,而具有一定的规律性。阴阳的升降,可以由天气的寒暑,彰显它的兆端。我也听夫子讲过五运的规律,先生所讲的仅仅是五运主岁,应以甲为首,我与鬼臾区曾经讨论这个问题。\\
鬼臾区认为:土运统领甲、己,金运统领乙、庚,水运统领丙、辛,木运统领丁、壬,火运统领戊、癸。子、午两年是少阴司天;丑、未两年是太阴司天;寅、申两年是少阳司天;卯、酉两年是阳明司天;辰、戌两年是太阳司天;巳、亥两年是厥阴司天。与夫子所讲的阴阳之例不符,是什么缘故呢?\\
岐伯说:这是明显的道理,因为五运六气是天地的阴阳。大凡能够查数的是人身中的阴阳,它与天地阴阳相合,是可以用类推的方法求得的。阴阳,可以由十推演到百,由千推演到万的。但是天地的阴阳,不能用数来类推,而应该用观察自然之象的方法来推求。\\
黄帝说:希望听听它是怎样开始的。\\
岐伯说:这是一个很有意思的问题!我曾阅览《太始天元册》看到其中写道:古人观测天象时,看见天空当中有赤色的气,横亘在牛、女二宿与西北方戊位之间;黄色的气横亘在心、尾二宿与东南方己位之间;青色的气横亘在危、室二宿与柳、鬼二宿之间;白色的气横亘在亢、氐二宿与昴、毕二宿之间;黑色的气横亘在张、翼二宿与娄、胃二宿之间。所谓戊位、己位,分别是奎、壁二宿和角、轸二宿的所在,奎、壁是在立春到立夏的节气之间,所以称为地户;角、轸是在立秋到立冬的节气之间,所以称为天门。时令的开始,也就是推算气候时令的方法产生根据的第一步,是自然规律所产生的,不可不通晓。\\
黄帝说:对。《天元纪大论》中说过,天地是万物的上下;左右是阴阳运行的道路。我还不明白它的意义。\\
岐伯说:所谓上下,是指该年的司天在泉位置上的阴阳。所谓左右,是指司天的左右,凡是司天的位置见到厥阴时,左面是少阴,右面是太阳;见到少阴时,左面是太阴,右面是厥阴;见到太阴时,左面是少阳,右面是少阴;见到少阳时,左面是阳明,右面是太阴;见到阳明时,左面是太阳,右面是少阳;见到太阳时,左面是厥阴,右面是阳明。这里所说的左右是指面向北方时所见的位置。\\
黄帝问:什么是下(在泉)呢?\\
岐伯说:厥阴司天,则少阳在泉,在泉之左是阳明,右是太阴;少阴司天,则阳明在泉,左是太阳,右是少阳;太阴司天,则太阳在泉,左是厥阴,右是阳明;少阳司天,则厥阴在泉,左是少阴,右是太阳;阳明司天,则少阴在泉,左是太阴,右是厥阴;太阳司天,则太阴在泉,左是少阳,右是少阴。这里所说的面向南方确定阴阳的位置,是阴阳在在泉位置上的不同显现。上下相互交合,寒暑相互加临,其气相生的就是和平,相克的就会使人生病。\\
黄帝问:有气相生而使人生病的,这又是什么缘故?\\
岐伯说:以下位加临于上位,虽似相得,但位置不当,也属于克贼之类。\\
黄帝问:司天在泉运转的动静怎样?\\
岐伯说:在上的司天向右行,在下的在泉向左行,左右旋转一周为一年,才回归到原来的位置。\\
黄帝说:我听到鬼臾区说,应地之气多静而不动。现在夫子却说在下面的在泉地气向左行,不知是怎样一回事,希望听听怎样会动的?\\
岐伯说:天地的运动和静止,五行的变换反复,鬼臾区虽然上察天运之候,但还没有了解全面。大凡天地的变化作用,在天显出星象,在地生成有形的万物,日月五星循行天空,五行之气附着大地。所以大地是盛载由五行之气化生的有形物类的;天空是布列日月五星这应天之精气的。地上的有形万物与天上的无形精气之间,好像根本与枝叶一样,抬起头来观看天象,虽然是遥远的天体,也是可以了解的。\\
黄帝问:大地是不是在宇宙的最下面?\\
岐伯说:大地位于人的下面,在宇宙之中。\\
黄帝又问:它依凭着什么东西吗?\\
岐伯说:大气托举着它。燥气使它干燥,暑气使它蒸发,风气使它运动,湿气使它润泽,寒气使它坚实,火气使它温暖。所以风寒在下,燥热在上,湿气位于中间,火气游行于左右上下。一年之中,寒暑往来,六气进入地面,地面受其影响而生化万物。所以燥气太过地面就干燥,暑气太过地面就发热,风气太过地面上万物皆动,湿气太过地面就会湿润,寒气太过地面就会冻裂,火气太过地面就会坚实固密了。\\
黄帝问:司天在泉之气,怎样用脉诊诊察呢?\\
岐伯说:天气与地气胜复的变化,用脉诊是诊察不到的。《脉法》上说:天地的变化,是不能在脉搏上诊察到的。说的就是这个意思。\\
黄帝问:左右间气怎样在脉象上诊察?\\
岐伯说:根据间气的位置,可以诊察左右手的脉象。\\
黄帝问:怎样诊察呢?\\
岐伯说:脉与气相顺应的为和平,脉与气相违逆的就会生病,见于其他部位的会生病,左右相反的会生病,见到相克之脉的病就危险,尺寸俱反的就会死亡,阴阳交错而见的也会死亡。首先要确定该年的司天、在泉,从而知道它的左右间气,然后可以推测或死或生,或逆或顺。\\
黄帝问:天之寒暑燥湿风火六气,在人体怎样与之相配合呢?它对万物是怎样使之生化的呢?\\
岐伯说:东方生风,风气能使木气生长,木气能产生酸味,酸味能滋养肝脏,肝血能养筋,肝与筋膜和调,则木气充旺能使心气旺盛。六气的变化,在天为玄冥之象,在人为适应变化之道,在地为生化万物。地有生化,就能化生五味,人能适应变化之道,就能产生智慧,天的玄冥之象能够产生神明,使天地万物运动不息,从而化生五行六气。天的神明,在天是风,在地是木,在人体是筋,在物体生化是柔软,在内脏是肝。它的性质是温暖的,它的德性是平和的,它的作用是运动,它的颜色是青苍,它的变化是荣美,它在动物上是有毛的兽类,它在作用上是疏散,它的时令是宣发布散阳和之气,它的变动是摧折,它的灾害是陨落,它在五味是酸味,它在情志是发怒。忿怒会损伤肝,悲哀能抑制忿怒;风气能损害肝,燥气能克制风气;酸味太过会伤害筋,辛味能克制酸味。\\
南方生热,热气能使火气旺盛,火气能生苦味,苦味能滋养心脏,心能生血脉,心血和调则滋养脾气。所以天的神明,在天是热,在地是火,在人体是血脉,在气化为使万物生长,在内脏是心。它的性质是炎热,它的德性是显露光华,它的作用是躁急,它的颜色是红赤,它的变化是使万物茂盛,在动物是有羽毛的禽类,它在作用上是日照光明,它的时令是地气上蒸,它的变动是使万物焦烁枯槁,它的灾害是焚焫,它在五味为苦味,在情志为喜乐。喜乐过度会损害心,恐惧能克制喜乐;过热也会损害气,寒气能克制热气;苦味太过能损害心气,咸味能克制苦味。\\
中央生湿,湿气能使土气生长,土气能滋生甘味,甘味能滋养脾脏,脾能滋养肌肉,脾与肉盛则土气充盈,使肺气旺盛。所以天的神明,在天是湿,在地是土,在人体是肌肉,在气化能使形体充实,在内脏是脾。它的性质是安静而包容,它的德性是潮湿润泽,它的功用是化生万物,它的颜色为黄,它的变化是使形体充盛丰满,在动物是无毛羽的裸体动物,它的作用是使天气平静,地气上升,它的时令是云雨及时布行,它的变动是骤雨暴注或淫雨连绵,它的灾变是大水泛滥,在五味为甘甜,在情志为忧思;忧思过度会伤脾,忿怒能克制忧思;湿气会伤害肌肉,风气能克制湿气;甘味过度会伤害脾,酸味能克制甘味。\\
西方生燥,燥气能使金气生长,金气能产生辛味,辛味能滋养肺脏,肺能滋养皮毛,肺与皮毛充旺则金盛而使肾气旺盛。所以天的神明,在天是燥,在地为金,在人体是皮毛,在气化能使万物成就,在内脏是肺。它的性质是清凉,它的德性是清静,它的功用是固卫,它的颜色为白,它的变化为收敛,在动物为甲壳类动物,它的作用是坚劲有力,它的时令是雾露下降,它的变动是使万物生机收杀,它的灾变是枝叶枯萎凋谢,在五味为辛味的物质,在情志为忧愁。忧愁过度会伤害肺,喜乐能克制忧愁;热气过度会伤害皮毛,寒气能克制热气;辛味过度能伤害皮毛,苦味能克制辛味。\\
北方生寒,寒气能使水气生长,水气能生咸味,咸味能滋养肾脏,肾精能滋生骨髓,肾精骨髓充盈水盛而使肝脏充实。所以天的神明,在天是寒,在地是水,在人体是骨,在气化是使物体坚固,在内脏是肾。它的性质是凛冽,它的德性是寒凉,它的作用是贮藏,它的颜色为黑色,它的变化是使万物肃静,在动物为有鳞片的动物,它的作用是静止,它的时令是寒冷冰雪,它的变动是冰冻凛冽,它的灾变是冰雹霜雪非时而下,在五味为咸味,在情志为恐惧。恐惧过度会伤害肾,思虑能克制恐惧;寒气过度会伤害血脉,燥气能克制寒气;咸味能伤害血脉,甘味能克制咸味。\\
五方之气,交替主时,各有先期而至的气候,若与时令相反就是邪气,与时令相合就是四时正气。\\
黄帝问:产生的病变怎样?\\
岐伯说:气与时令相合的虽病亦轻,不相合的其病必重。\\
黄帝问:五气主岁怎样?\\
岐伯说:气太过,一方面能克制自己所克的气,另一方面也能欺侮克制自己的气;如气不及,一方面克制自己的气乘机欺侮,另一方面本来受自己克制的气,也轻视自己而来侵犯。凡是欺侮人的而自己也会受到邪气侵犯,是因为它无所忌惮而招致的。\\
黄帝说:讲得很对。\\
六微旨大论篇第六十八\\
黄帝问曰:呜乎远哉!天之道也,如迎浮云,若视深渊。视深渊尚可测,迎浮云莫知其极。夫子数言谨奉天道,余闻而藏之。心私异之,不知其所谓也。愿夫子溢志尽言其事,令终不灭,久而不绝。天之道可得闻乎?\\
岐伯稽首再拜对曰:明乎哉问!天之道也,此因天之序,盛衰之时也。\\
帝曰:愿闻天道六六之节,盛衰何也?\\
岐伯曰:上下有位,左右有纪。故少阳之右,阳明治之;阳明之右,太阳治之;太阳之右,厥阴治之;厥阴之右,少阴治之;少阴之右,太阴治之;太阴之右,少阳治之。此所谓气之标,盖南面而待也。故曰:因天之序,盛衰之时,移光定位,正立而待之,此之谓也。\\
少阳之上,火气治之,中见厥阴;阳明之上,燥气治之,中见太阴;太阳之上,寒气治之,中见少阴;厥阴之上,风气治之,中见少阳;少阴之上,热气治之,中见太阳;太阴之上,湿气治之,中见阳明。所谓本也,本之下,中之见也,见之下,气之标也。本标不同,气应异象。\\
帝曰:其有至而至,有至而不至,有至而太过,何也?\\
岐伯曰:至而至者和;至而不至,来气不及也;未至而至,来气有余也。\\
帝曰:至而不至,未至而至,如何?\\
岐伯曰:应则顺,否则逆,逆则变生,变则病。\\
帝曰:善。请言其应。\\
岐伯曰:物生其应也,气脉其应也。\\
帝曰:善。愿闻地理之应六节气位,何如?\\
岐伯曰:显明之右,君火之位也;君火之右,退行一步,相火治之;复行一步,土气治之;复行一步,金气治之;复行一步,水气治之;复行一步,木气治之;复行一步,君火治之。\\
相火之下,水气承之;水位之下,土气承之;土位之下,风气承之;风位之下,金气承之;金位之下,火气承之;君火之下,阴精承之。\\
帝曰:何也?\\
岐伯曰:亢则害,承乃制,制则生化。外列盛衰,害则败乱,生化大病。\\
帝曰:盛衰何如?\\
岐伯曰:非其位则邪,当其位则正。邪则变甚,正则微。\\
帝曰:何谓当位?\\
岐伯曰:木运临卯,火运临午,土运临四季,金运临酉,水运临子,所谓岁会,气之平也。\\
帝曰:非位何如?\\
岐伯曰:岁不与会也。\\
帝曰:土运之岁,上见太阴;火运之岁,上见少阳、少阴;金运之岁,上见阳明;木运之岁,上见厥阴;水运之岁,上见太阳,奈何?\\
岐伯曰:天之与会也,故《天元册》曰天符。\\
帝曰:天符岁会,何如?\\
岐伯曰:太一天符之会也。\\
帝曰:其贵贱何如?\\
岐伯曰:天符为执法,岁会为行令,太一天符为贵人。\\
帝曰:邪之中也,奈何?\\
岐伯曰:中执法者,其病速而危;中行令者,其病徐而持;中贵人者,其病暴而死。\\
帝曰:位之易也,何如?\\
岐伯曰:君位臣则顺,臣位君则逆。逆则其病近,其害速;顺则其病远,其害微。所谓二火也。\\
帝曰:善。愿闻其步,何如?\\
岐伯曰:所谓步者,六十度而有奇,故二十四步,积盈百刻而成日也。\\
帝曰:六气应五行之变,何如?\\
岐伯曰:位有终始,气有初中,上下不同,求之亦异也。\\
帝曰:求之奈何?\\
岐伯曰:天气始于甲,地气始于子,子甲相合,命曰岁立。谨候其时,气可与期。\\
帝曰:愿闻其岁,六气始终,早晏何如?\\
岐伯曰:明乎哉问也!甲子之岁,初之气,天数始于水下一刻,终于八十七刻半;二之气,始于八十七刻六分,终于七十五刻;三之气,始于七十六刻,终于六十二刻半;四之气,始于六十二刻六分,终于五十刻;五之气,始于五十一刻,终于三十七刻半;六之气,始于三十七刻六分,终于二十五刻。所谓初六,天之数也。\\
乙丑岁,初之气,天数始于二十六刻,终于一十二刻半;二之气,始于一十二刻六分,终于水下百刻;三之气,始于一刻,终于八十七刻半;四之气,始于八十七刻六分,终于七十五刻;五之气,始于七十六刻,终于六十二刻半;六之气,始于六十二刻六分,终于五十刻。所谓六二,天之数也。\\
丙寅岁,初之气,天数始于五十一刻,终于三十七刻半;二之气,始于三十七刻六分,终于二十五刻;三之气,始于二十六刻,终于一十二刻半;四之气,始于一十二刻六分,终于水下百刻;五之气,始于一刻,终于八十七刻半;六之气,始于八十七刻六分,终于七十五刻。所谓六三,天之数也。\\
丁卯岁,初之气,天数始于七十六刻,终于六十二刻半;二之气,始于六十二刻六分,终于五十刻;三之气,始于五十一刻,终于三十七刻半;四之气,始于三十七刻六分,终于二十五刻;五之气,始于二十六刻,终于一十二刻半;六之气,始于一十二刻六分,终于水下百刻。所谓六四,天之数也。次戊辰岁,初之气复始于一刻。常如是无已,周而复始。\\
帝曰:愿闻其岁候何如?\\
岐伯曰:悉乎哉问也!日行一周,天气始于一刻;日行再周,天气始于二十六刻;日行三周,天气始于五十一刻;日行四周,天气始于七十六刻;日行五周,天气复始于一刻,所谓一纪也。是故寅午戌岁气会同,卯未亥岁气会同,辰申子岁气会同,巳酉丑岁气会同。终而复始。\\
帝曰:愿闻其用也。\\
岐伯曰:言天者求之本,言地者求之位,言人者求之气交。\\
帝曰:何谓气交?\\
岐伯曰:上下之位,气交之中,人之居也。故曰:天枢之上,天气主之;天枢之下,地气主之;气交之分,人气从之,万物由之。此之谓也。\\
帝曰:何谓初中?\\
岐伯曰:初凡三十度而有奇,中气同法。\\
帝曰:初中何也?\\
岐伯曰:所以分天地也。\\
帝曰:愿卒闻之。\\
岐伯曰:初者地气也,中者天气也。\\
帝曰:其升降何如?\\
岐伯曰:气之升降,天地之更用也。\\
帝曰:愿闻其用何如?\\
岐伯曰:升已而降,降者谓天;降已而升,升者谓地。天气下降,气流于地;地气上升,气腾于天。故高下相召,升降相因,而变作矣。\\
帝曰:善。寒湿相遘,燥热相临,风火相值,其有闻乎?\\
岐伯曰:气有胜复,胜复之作,有德有化,有用有变,变则邪气居之。\\
帝曰:何谓邪乎?\\
岐伯曰:夫物之生从于化,物之极由乎变。变化之相薄,成败之所由也。故气有往复,用有迟速,四者之有,而化而变,风之来也。\\
帝曰:迟速往复,风所由生,而化而变,故因盛衰之变耳。成败倚伏游乎中,何也?\\
岐伯曰:成败倚伏生乎动,动而不已,则变作矣。\\
帝曰:有期乎?\\
岐伯曰:不生不化,静之期也。\\
帝曰:不生化乎?\\
岐伯曰:出入废则神机化灭,升降息则气立孤危。故非出入,则无以生长壮老已;非升降,则无以生长化收藏。是以升降出入,无器不有。故器者生化之宇。器散则分之,生化息矣。故无不出入,无不升降,化有小大,期有近远。四者之有,而贵常守,反常则灾害至矣。故曰:无形无患。此之谓也。\\
帝曰:善。有不生不化乎。\\
岐伯曰:悉乎哉问也!与道合同,惟真人也。\\
帝曰:善。\\
黄帝问道:哎呀!真是太深远了!天道运行的规律,就像仰望浮云,又像俯视深渊。视深渊还可以测量,而迎浮云却不能知道它的极点。夫子您常说要谨慎奉行天道,我听了以后,记在心里。但又有疑问,不知其所以然。希望您详细地讲一讲,使它永不泯灭,长久流传而不断绝。天道运行的规律,可以讲给我听吗?\\
岐伯恭敬地行了两次礼回答说:你问得很高明啊!天道运行的规律就是自然的变化所显示出来的时序和盛衰。\\
黄帝说:希望听听天道六六之节和时序盛衰的变化是怎样的?\\
岐伯说:上下六步有一定的位置,左右升降有一定的次序。所以少阳的右面,由阳明掌管;阳明的右面,由太阳掌管;太阳的右面,由厥阴掌管;厥阴的右面,由少阴掌管;少阴的右面,由太阴掌管;太阴的右面,由少阳掌管。这是六气之标,是面向南方而确定的位置。所以说:根据天气变化的一定次序,时令有盛衰的不同,在日中之时,观看日光移影所确定的位置,说的就是这个道理。\\
少阳的上面由火气掌管,它的中气是厥阴;阳明的上面由燥气掌管,它的中气是太阴;太阳的上面由寒气掌管,它的中气是少阴;厥阴的上面由风气掌管,它的中气是少阳;少阴的上面由热气掌管,它的中气是太阳;太阴的上面由湿气掌管,它的中气是阳明。以上所说的“上面”是三阴三阳的本气,本气的下面是中见之气,中气的下面是六气的标。因为六气的本标不同,所以它反映的现象也不是一致的。\\
黄帝问:六气有按时而至的,有时至而气不至的,有先时而至的,这是什么道理?\\
岐伯说:按时而至的是和平之气;时至而气不至的是气之不及;时未至而气先至的是气之有余。\\
黄帝问:时至而气不至的,时未至而气先至的,情况怎样呢?\\
岐伯说:时与气相应而来的则顺,否则为逆,逆则产生异常变化,异常变化能导致疾病。\\
黄帝说:讲得好。请谈谈相应的情况。\\
岐伯说:万物与生长变化是相适应的,大气与脉象变化是相适应的。\\
黄帝问:对。希望听听关于六气主时的位置是怎样的?\\
岐伯说:春分节之后是少阴君火的位置;君火之右,后退一步,是少阳相火主治的位置;再退一步,是太阴土气主治的位置;再退一步,是阳明金气主治的位置;再退一步,是太阳水气主治的位置;再退一步,是厥阴木气主治的位置;再退一步,是少阴君火主治的位置。\\
相火主治之位的下面,有水气承袭制约;水气主治之位的下面,有土气承袭制约;土气主治之位的下面,有风气承袭制约;风气主治之位的下面,有金气承袭制约;金气主治之位的下面,有火气承袭制约;君火主治之位的下面,有阴精承袭制约。\\
黄帝又问:这是为什么?\\
岐伯说:六气亢盛就会产生损害作用,所以要有承袭之气来制约它,有制约然后才能生化。如果亢盛无制,就会使生化之机败坏紊乱,产生病变。\\
黄帝问:自然界的盛衰怎样?\\
岐伯说:不当其位的是邪气,恰当其位的是正常之气。邪气致病,病重多变化,正气致病,病多轻微。\\
黄帝问:什么叫当位?\\
岐伯说:例如木运遇卯年,火运遇午年,土运遇辰、戌、丑、未年,金运遇酉年,水运遇子年,这就称为“岁会”,也属于“平气”。\\
黄帝问:不当其位的怎样?\\
岐伯说:那就不是岁会。\\
黄帝问:土运主岁,司天是太阴;火运主岁,司天是少阳或少阴;金运主岁,司天是阳明;木运主岁,司天是厥阴;水运主岁,司天是太阳,这些是怎么分的?\\
岐伯说:这是司天与五运相会,所以《天元册》里称为“天符”。\\
黄帝问:既是天符又是岁会的怎样呢?\\
岐伯说:这叫太一天符的会合。\\
黄帝问:它们之间有什么贵贱的分别吗?\\
岐伯说:天符好像执法,岁会好像行令,太一天符好像贵人。\\
黄帝问:邪气侵入而发病,三者有何区别?\\
岐伯说:中执法之邪,发病急而比较危险;中行令之邪,病势缓慢而病程较长;中贵人之邪,则发病急剧而很快会死亡。\\
黄帝问:六气的位置相互转移是怎样的?\\
岐伯说:君居臣位是顺的,臣居君位是逆的。逆则发病急,危害大;顺则发病慢,危害小。所谓六气位置移易,是指君火与相火说的。\\
黄帝问说:很对。希望听听六气的步位是什么?\\
岐伯说:所谓一步,就是六十日有零,所以二十四步之后,其奇零之数积满一百刻,就成为一日。\\
黄帝问:六气与五行相应的变化怎样?\\
岐伯说:因主时之六气的每一气位有始有终,每一气又有初气和中气的分别,又有天气和地气的分别,所以推求也就不一样了。\\
黄帝问:怎样推求呢?\\
岐伯说:天气以甲为开始,地气以子为开始,子与甲相互组合,称为岁立。谨慎地候察四时的变化,就可以推求六气始终早晚的时刻了。\\
黄帝说:希望听听每年六气始终的早晚怎样?\\
岐伯说:问得高明啊!甲子的年份,初气开始于水下一刻,终止于八十七刻半;第二气开始于八十七刻六分,终止于七十五刻;第三气开始于七十六刻,终止于六十二刻半;第四气开始于六十二刻六分,终止于五十刻;第五气开始于五十一刻,终止于三十七刻半;第六气开始于三十七刻六分,终止于二十五刻。这就是六气第一周的始终的刻分数。\\
乙丑的年份,第一气开始于二十六刻,终止于十二刻半;第二气开始于十二刻六分,终止于水下百刻;第三气开始于一刻,终止于八十七刻半;第四气开始于八十七刻六分,终止于七十五刻;第五气开始于七十六刻,终止于六十二刻半;第六气开始于六十二刻六分,终止于五十刻。这是六气第二周的始终的刻分数。\\
丙寅的年份,第一气开始于五十一刻,终止于三十七刻半;第二气开始于三十七刻六分,终止于二十五刻;第三气开始于二十六刻,终止于十二刻半;第四气开始于十二刻六分,终止于水下百刻;第五气开始于一刻,终止于八十七刻半;第六气开始于八十七刻六分,终止于七十五刻。这是六气第三周的始终的刻分数。\\
丁卯的年份,初气开始于七十六刻,终止于六十二刻半;第二气开始于六十二刻六分,终止于五十刻;第三气开始于五十一刻,终止于三十七刻半;第四气开始于三十七刻六分,终止于二十五刻;第五气开始于二十六刻,终止于十二刻半;第六气开始于十二刻六分,终止于水下百刻。这是六气第四周的始终的刻分数。再次是戊辰年的初气,重新从水下一刻开始。总是循上述次序,周而复始地循环下去。\\
黄帝问道:希望听听一年六气终始变化的情况是怎样的?\\
岐伯说:问得真详细啊!太阳循行第一周,六气开始于一刻;太阳循行第二周,六气开始于二十六刻;太阳循行第三周,六气开始于五十一刻;太阳循行第四周,六气开始于七十六刻;太阳循行第五周,六气又从一刻开始,这是六气四周的循环,叫做一纪。所以寅年、午年、戌年,六气始终的时刻相同;卯年、未年、亥年,六气始终的时刻相同;辰年、申年、子年,六气始终的时刻相同;巳年、酉年、丑年,六气始终的时刻相同。总之,六气是循环不已,终而复始的。\\
黄帝说:我希望听您讲一讲六气的作用。\\
岐伯说:说到天,当推求于六气,说到地,当推求于主时之六位,说到人体,当推求于天地气交之中。\\
黄帝问:什么叫做气交?\\
岐伯说:天气下降,地气上升,天地气交之处,就是人类生活的地方。所以说:天枢的上面,是属于天气所主;天枢的下面,是属于地气所主;而气交的部分,人气随之而来,万物也由之化生。说的就是这个事。\\
黄帝又问:什么叫做初气、中气呢?\\
岐伯说:初气三十度有零,中气也是这样。\\
黄帝又问:初气、中气,是什么?\\
岐伯说:这是用来分别天气与地气的根据。\\
黄帝说:我希望听个究竟。\\
岐伯说:初就是地气,中就是天气。\\
黄帝问:气的升降是怎样的?\\
岐伯说:地气上升,天气下降,这是天地之气的相互作用。\\
黄帝又问:希望听听它的作用如何?\\
岐伯说:升后而降,这是天的作用;降后又升,这是地的作用。天气下降,气就下流于大地;地气上升,气就蒸腾于天空。所以上下交相呼应,升降互为因果,变化就发生了。\\
黄帝说:讲得好!寒与湿相遇,燥与热相守,风与火相当,这些道理可以说说吗?\\
岐伯说:六气有胜有复,胜复的变化中,有根本与生化,有原因与变异,一旦有了变异,就会招致邪气滞留。\\
黄帝问:什么是邪呢?\\
岐伯说:万物的生长都由于化,万物的终结都由于变。变与化相争是成长与毁败的根源。所以气有往有复,作用有慢有快,从往复快慢里,就会出现化与变的过程,这就是风气的由来。\\
黄帝问:慢快往复是风气产生的原因,由化至变的过程,是随着盛衰的变化而进行的。但是无论成败,其潜伏的因素都是从变化中来,这是为什么?\\
岐伯说:成败因素相互蕴伏是由于六气的运动,运动不止,就会发生变化。\\
黄帝问:变化有停止的时候吗?\\
岐伯说:没有生,没有化,就是停止的时候。\\
黄帝问:有不生不化的时候吗?\\
岐伯说:凡动物类的呼吸停止,那么其生命也就会立即消灭;凡植物类的阴阳升降停止,那么则其活力也就立即萎颓。因此说没有出入,就不可能由生而长、而壮、而老、而死亡;没有升降,也就不能由生而长、而开花、而结实、而收藏。所以升降出入之气,凡是有形之物都具有。因此事物的形器,是气机生化的场所。如果形器瓦解分散,生化也就息灭了。因此任何具有形体的东西,没有不出不入、不升不降的,其间仅仅有生化的大小和时间早晚的分别而已。任何事物都存在升降出入,重要的是要保持正常,如果失常,就有灾害。所以说:除非是无形体的东西,才能免于灾患。\\
黄帝说:讲得好。那么有没有不生不化的人呢?\\
岐伯说:问得真详细啊!能与自然规律相融合,而同其变化的,只有真人。\\
黄帝说:说得好。\\
卷二十\\
气交变大论篇第六十九\\
黄帝问曰:五运更治,上应天期,阴阳往复,寒暑迎随,真邪相薄,内外分离,六经波荡,五气倾移,太过不及,专胜兼并,愿言其始,而有常名,可得闻乎?\\
岐伯稽首再拜对曰:昭乎哉问也!是明道也。此上帝所贵,先师传之,臣虽不敏,往闻其旨。\\
帝曰:余闻得其人不教,是谓失道;传非其人,慢泄天宝。余诚菲德,未足以受至道,然而众子哀其不终。愿夫子保于无穷,流于无极,余司其事,则而行之,奈何?\\
岐伯曰:请遂言之也。《上经》曰:夫道者,上知天文,下知地理,中知人事,可以长久。此之谓也。\\
帝曰:何谓也?\\
岐伯曰:本气位也。位天者,天文也。位地者,地理也。通于人气之变化者,人事也。故太过者,先天;不及者,后天。所谓治化而人应之也。\\
帝曰:五运之化,太过何如?\\
岐伯曰:岁木太过,风气流行,脾土受邪。民病飧泄,食减,体重,烦冤,肠鸣腹支满,上应岁星。甚则忽忽善怒,眩冒巅疾。化气不政,生气独治,云物飞动,草木不宁,甚而摇落,反胁痛而吐甚。冲阳绝者,死不治。上应太白星。\\
岁火太过,炎暑流行,肺金受邪。民病疟,少气咳喘,血溢、血泄、注下,嗌燥耳聋,中热肩背热。上应荧惑星。甚则胸中痛,胁支满,胁痛,膺背肩胛间痛,两臂内痛,身热肤痛,而为浸淫。收气不行,长气独明,雨冰霜寒。上应辰星。上临少阴少阳,火燔焫,冰泉涸,物焦槁。病反谵妄狂越,咳喘息鸣,下甚血溢泄不已。太渊绝者,死不治。上应荧惑星。\\
岁土太过,雨湿流行,肾水受邪。民病腹痛,清厥、意不乐,体重烦冤。上应镇星。甚则肌肉萎,足痿不收,行善瘛,脚下痛,饮发中满食减,四支不举。变生得位,脏气伏,化气独治之,泉涌河衍,涸泽生鱼,风雨大至,土崩溃,鳞见于陆。病腹满溏泄肠鸣,反下甚。而太谿绝者,死不治。上应岁星。\\
岁金太过,燥气流行,肝木受邪。民病两胁下少腹痛,目赤痛、眦疡,耳无所闻。肃杀而甚,则体重烦冤,胸痛引背,两胁满、且痛引少腹。上应太白星。甚则喘咳逆气,肩背痛,尻阴股膝髀腨气足皆病。上应荧惑星。收气峻,生气下,草木敛,苍干凋陨。病反暴痛,胠胁不可反侧,咳逆甚而血溢。太冲绝者,死不治。上应太白星。\\
岁水太过,寒气流行,邪害心火。民病身热烦心,躁悸,阴厥上下中寒,谵妄心痛。寒气早至,上应辰星。甚则腹大胫肿,喘咳,寝汗出、憎风。大雨至,埃雾朦郁,上应镇星。上临太阳,则雨冰雪,霜不时降,湿气变物。病反腹满肠鸣,溏泄食不化,渴而妄冒。神门绝者,死不治。上应荧惑辰星。\\
帝曰:善。其不及如何?\\
岐伯曰:悉乎哉问也!岁木不及,燥乃大行,生气失应,草木晚荣。肃杀而甚,则刚木辟著,柔萎苍干。上应太白星。民病中清,胠胁痛,少腹痛,肠鸣溏泄。凉雨时至,上应太白星,其谷苍。上临阳明,生气失政,草木再荣,化气乃急。上应太白、镇星,其主苍早。复则炎暑流火,湿性燥,柔脆草木焦槁,下体再生,华实齐化。病寒热疮疡疿胗痈痤。上应荧惑、太白,其谷白坚。白露早降,收杀气行,寒雨害物,虫食甘黄。脾土受邪,赤气垢化,心气晚治,上胜肺金,白气乃屈。其谷不成,咳而鼽。上应荧惑、太白星。\\
岁火不及,寒乃大行,长政不用,物荣而下。凝惨而甚,则阳气不化,乃折荣美,上应辰星。民病胸中痛,胁支满,两胁痛,膺背肩胛间及两臂内痛,郁冒朦昧,心痛暴瘖,胸腹大,胁下与腰背相引而痛,甚则屈不能伸,髋髀如别。上应荧惑,辰星,其谷丹。复则埃郁,大雨且至,黑气乃辱,病鹜溏、腹满,食饮不下,寒中肠鸣,泄注腹痛,暴挛痿痹,足不任身。上应镇星、辰星,玄谷不成。\\
岁土不及,风乃大行,化气不令,草木茂荣。飘扬而甚,秀而不实,上应岁星。民病飧泄霍乱,体重腹痛,筋骨繇复,肌肉丱酸,善怒。藏气举事,蛰虫早附,咸病寒中,上应岁星,镇星,其谷黅。复则收政严峻,名木苍凋,胸胁暴痛,下引少腹,善太息。虫食甘黄,气客于脾,黅谷乃减,民食少失味。苍谷乃损。上应太白,岁星。上临厥阴,流水不冰,蛰虫来见,藏气不用,白乃不复,上应岁星,民乃康。\\
岁金不及,炎火乃行,生气乃用,长气专胜,庶物以茂,燥烁以行。上应荧惑星。民病肩背瞀重,鼽嚏血便注下。收气乃后,上应太白星,其谷坚芒。复则寒雨暴至,乃零冰雹霜雪杀物,阴厥且格,阳反上行,头脑户痛,延及囟顶发热。上应辰星,丹谷不成。民病口疮,甚则心痛。\\
岁水不及,湿乃大行,长气反用,其化乃速,暑雨数至。上应镇星。民病腹满身重,濡泄,寒疡流水,腰股痛发,腘腨股膝不便,烦冤,足痿清厥,脚下痛,甚则跗肿。藏气不政,肾气不衡,上应辰星,其谷秬。上临太阴,则大寒数举,蛰虫早藏,地积坚冰,阳光不治。民病寒疾于下,甚则腹满浮肿,上应镇星,其主黅谷。复则大风暴发,草偃木零,生长不鲜。面色时变,筋骨并辟,肉丱瘛,目视丼丼,物疏璺,肌肉胗发,气并鬲中,痛于心腹。黄气乃损,其谷不登,上应岁星。\\
帝曰:善。愿闻其时也。\\
岐伯曰:悉乎哉问也!木不及,春有鸣条律畅之化,则秋有雾露清凉之政;春有惨凄残贼之胜,则夏有炎暑燔烁之复。其眚东,其脏肝,其病内舍胠胁,外在关节。\\
火不及,夏有炳明光显之化,则冬有严肃霜寒之政;夏有惨凄凝冽之胜,则不时有埃昏大雨之复。其眚南,其脏心,其病内舍膺胁,外在经络。\\
土不及,四维有埃云润泽之化,则春有鸣条鼓拆之政;四维发振拉飘腾之变,则秋有肃杀霖霪之复。其眚四维,其脏脾,其病内舍心腹,外在肌肉四支。\\
金不及,夏有光显郁蒸之令,则冬有严凝整肃之应;夏有炎烁燔燎之变,则秋有冰雹霜雪之复。其眚西,其脏肺,其病内舍膺胁肩背,外在皮毛。\\
水不及,四维有湍润埃云之化,则不时有和风生发之应;四维发埃昏骤注之变,则不时有飘荡振拉之复。其眚北,其脏肾,其病内舍腰脊骨髓,外在谿谷踹膝。\\
夫五运之政,犹权衡也,高者抑之,下者举之,化者应之,变者复之。此生长化收藏之理,气之常也。失常则天地四塞矣。故曰:天地之动静,神明为之纪;阴阳之往复,寒暑彰其兆。此之谓也。\\
帝曰:夫子之言五气之变,四时之应,可谓悉矣。夫气之动乱,触遇而作,发无常会,卒然灾合,何以期之?\\
岐伯曰:夫气之动变,固不常在,而德化政令灾变,不同其候也。\\
帝曰:何谓也?\\
岐伯曰:东方生风,风生木。其德敷和,其化生荣,其政舒启,其令风,其变振发,其灾散落。\\
南方生热,热生火。其德彰显,其化蕃茂,其政明曜,其令热,其变销烁,其灾燔焫。\\
中央生湿,湿生土。其德溽蒸,其化丰备,其政安静,其令湿,其变骤注,其灾霖溃。\\
西方生燥,燥生金。其德清洁,其化紧敛,其政劲切,其令燥,其变肃杀,其灾苍陨。\\
北方生寒,寒生水。其德凄沧,其化清谧,其政凝肃,其令寒,其变凓冽,其灾冰雪霜雹。是以察其动也,有德有化,有政有令,有变有灾,而物由之,而人应之也。\\
帝曰:夫子之言岁候,其不及太过而上应五星。今夫德化政令,灾眚变易,非常而有也,卒然而动,其亦为之变乎。\\
岐伯曰:承天而行之,故无妄动,无不应也。卒然而动者,气之交变也,其不应焉。故曰:应常不应卒。此之谓也。\\
帝曰:其应奈何?\\
岐伯曰:各从其气化也。\\
帝曰:其行之徐疾,逆顺何如?\\
岐伯曰:以道留久,逆守而小,是谓省下;以道而去,去而速来,曲而过之,是谓省遗过也;久留而环,或离或附,是谓议灾与其德也。应近则小,应远则大。芒而大倍常之一,其化甚;大常之二,其眚即发也;小常之一,其化减;小常之二,是谓临视,省下之过与其德也。德者福之,过者伐之。是以象之见也,高而远则小,下而近则大。故大则喜怒迩,小则祸福远。岁运太过,则运星北越;运气相得,则各行以道。故岁运太过,畏星失色而兼其母,不及则色兼其所不胜。肖者瞿瞿,莫知其妙,闵闵之当,孰者为良?妄行无征,示畏侯王。\\
帝曰:其灾应,何如?\\
岐伯曰:亦各从其化也。故时至有盛衰,凌犯有逆顺,留守有多少,形见有善恶,宿属有胜负,征应有吉凶矣。\\
帝曰:其善恶,何谓也?\\
岐伯曰:有喜有怒,有忧有丧,有泽有燥。此象之常也,必谨察之。\\
帝曰:六者高下,异乎?\\
岐伯曰:象见高下,其应一也,故人亦应之。\\
帝曰:善。其德化政令之动静损益,皆何如?\\
岐伯曰:夫德化政令灾变,不能相加也。胜复盛衰,不能相多也。往来小大,不能相过也。用之升降,不能相无也。各从其动而复之耳。\\
帝曰:其病生,何如?\\
岐伯曰:德化者,气之祥;政令者,气之章;变易者,复之纪;灾眚者,伤之始。气相胜者,和;不相胜者,病;重感于邪,则甚也。\\
帝曰:善。所谓精光之论,大圣之业,宣明大道,通于无穷,究于无极也。余闻之,善言天者,必应于人;善言古者,必验于今;善言气者,必彰于物;善言应者,同天地之化;善言化言变者,通神明之理。非夫子孰能言至道欤?\\
乃择良兆而藏之灵室,每旦读之,命曰《气交变》。非斋戒不敢发,慎传也。\\
黄帝问:五运交替,与在天之六气相应,阴阳往来,寒暑变化相随,真气与邪气斗争,内外分离,六经的血气波动不安,五脏之气相互倾轧而转移,出现了太过或不及,一气独胜或二气相并,我希望知道它起始的原理和使人发病的一般常规,可否让我听听呢?\\
岐伯稽首拜了两拜说:问得很高明啊!这是应该明白的道理。这是上古帝王所珍贵的,也是前代医师传授下来的,我虽然不聪敏,但过去曾听老师讲过它的道理。\\
黄帝说:我听说遇到适当的人而不教,就会使学术失传,称为“失道”;传授给不适当的人,是轻视随便泄漏宝贵的大道学术。我虽然德行不高,不足以接受宝贵的大道,但是我为百姓因疾病而夭亡,而不得终其天年而悲伤。希望夫子为了保全百姓的健康和学术的永远留传,传给我大道,我来主管其事,一定按规矩来做,怎么样?\\
岐伯说:让我详细地讲给你听吧。《上经》中说:研究医道的人,要上知天文,下知地理,中知人事,他的学说才能保持长久。就是这个道理。\\
黄帝问:这是什么意思?\\
岐伯说:这里的根本是为了推求天、地、人三气的位置啊。求天位的,是天文;求地位的,是地理。通晓人气变化的,是人事。因而太过的气先天时而至;不及的气后天时而至。所以说,天地运动的变化有常有变,而人体也随之发生相应的变化。\\
黄帝问:五运的气化,太过的情况怎样?\\
岐伯说:岁木之运太过,则风气流行,脾土受到侵害。人们多患泄泻,饮食减少,肢体沉重,烦闷抑郁,肠中鸣响,肚腹胀满,由于木气太过,在天上的木星显得光明。木气过于亢盛,甚至会突然发怒,头昏眼花,及头部病证。这是土气无权,不能行其政令,木气独胜的现象,使天上云物飞跑,地上的草木动摇不定,甚至枝叶坠落,病人会胁部疼痛,呕吐不止。若冲阳脉绝,是不治的死证。在天上相应的金星分外光明,这显示木胜则金气制之。\\
岁火之运太过,则暑热流行,肺金受到伤害。人们多患疟疾,呼吸少气,咳嗽气喘,吐血衄血,二便下血,水泻如注,咽喉干燥,耳聋,胸中热,肩背热。在天上应火星显得格外光明。火热之气过于亢盛,在人体会有胸中疼痛,胁下胀满,胁痛,胸背肩胛间部疼痛,两臂内侧疼痛,身热肤痛,而发生浸淫疮。这是金气不行,火气独旺的现象,火气过旺,出现雨冰霜寒。在天上的水星格外光明。如果遇到少阴或少阳司天的年份,火热之气更加亢盛,有如火烧火烤,以致水源干涸,植物焦枯。人们发病,多见谵语妄动,发狂越常,咳嗽气喘,呼吸有声,火气甚于下部,则血从二便下泻不止。若太渊脉绝,是不治的死证。在天上相应的火星格外光明,这是火盛的标志。\\
岁土之运太过,则雨湿之气流行,肾水受到伤害。人们多病腹痛,四肢厥冷,情绪忧郁,身体沉重而烦闷。在天上相应的土星格外光明。重的见肌肉枯萎,两足痿弱不能行动,抽掣挛痛,脚跟痛,水饮之邪积于体内而脘腹胀满,饮食减少,四肢无力,不能举动。若遇土旺之时,水气无权行令,土气独旺,则湿令大行,因此泉水喷涌,河水高涨,本来干涸的池沼也会滋生鱼类了,若木气来复,风雨暴至,使堤岸崩溃,河水泛滥,陆地可出现鱼类。人们就会病肚腹胀满,大便溏泄,肠鸣,泄泻不止。若太谿脉绝,是不治的死证。水气受伤,木气来复,在天上应木星格外光明。\\
岁金之运太过,则燥气流行,肝木受到伤害。人们多病两胁下及少腹疼痛,目赤而痛,眼角溃烂,两耳听不到声音。燥金之气过于亢盛,就会身体沉重烦闷,胸部疼痛牵引背部,两胁胀满,而痛连少腹。在天上相应的金星格外光明。重的则发生喘息咳嗽,呼吸困难,肩背疼痛,尻、阴、股、膝、髀、腨、伄、足等处疼痛的病症。在天上相应的火星格外光明。金气突然亢盛,木气被克,草木呈现收敛之象,枝叶干枯凋落。疾病多见胁肋急剧疼痛,胠胁痛不能转身,咳嗽气逆,甚至吐血衄血。太冲脉绝,是不治的死证。在天上相应的金星格外光明。\\
岁水之运太过,则寒气流行,心火受到伤害。人们多患发热,心悸,烦躁,四肢逆冷,全身发冷,谵语妄动,心痛。寒气先天时早至,在天上相应的水星格外光明。水邪过度亢盛则有腹水,足胫浮肿,气喘咳嗽,眠中汗出,怕风。水气盛,则大雨下降,尘雾迷蒙不清,在天上相应的土星格外光明。如遇太阳寒水司天,则冰雹雪霜不时下降,湿气大盛,物变其形。人们多患腹胀肠鸣,溏泄,食不化,渴而眩晕。如神门脉绝,是不治的死证。在天上相应的火星失明而水星光亮。\\
黄帝说:很好。五运不及怎样?\\
岐伯说:问得真详细啊!岁木之运不及,燥气就会旺盛流行,生气不能及时而来,草木不能当时生荣。肃杀之气亢盛,使刚硬的树木受刑而碎裂如劈,柔嫩的枝叶就会萎弱干枯。在天上相应的金星显得光明。人们多患中气虚寒,胠胁疼痛,少腹痛,腹中鸣响,大便溏泄。在气候方面,冷雨不时下降,在天上相应的金星光明,五谷中青色的谷物不能成熟。如遇阳明司天,金气抑木,木气不能行其政令,失去应有的生气,草木在夏秋再度繁荣,生化之气就显得峻急,因为燥土二气俱盛,在天上相应金、土二星光明,所以草木开花结实的过程非常急促,很早就凋谢。木气受克制,其子气火气来复,就会炎热如火,湿润的变为干燥,柔嫩脆弱的变为干枯焦槁,枝叶从根部重新生长,而致开花结实并见。在人体则热气郁于皮毛,多病寒热、疮疡、疿疹、痈痤。在天上相应金、火二星光明,在五谷则外强中干,秀而不实。白霜提前下降,肃杀之气流行,寒雨非时,损害万物,味甘色黄之物多为虫蛀。在人则脾土受邪,火气后起,心气亢盛较晚,火气克金,金气被抑制。谷物不能成熟,疾病多是咳嗽鼻流清涕。在天上相应金星与火星光明。\\
岁火运之不及,寒气就旺盛流行,夏天生长之气不能行其政令,万物就会由茂盛走向零落。阴寒凝滞之气过盛,则阳气不能生化,万物的荣美就受到摧折,在天上相应水星光明。人们多病,胸中疼痛,胁部胀满,两胁疼痛,胸膺部、背部、肩胛之间及两臂内侧疼痛,抑郁眩晕,视物不清,心痛,突然失音,胸腹肿大,胁下与腰背相互牵引疼痛,重的则屈不能伸,髋骨与大腿之间不能活动自如。在天上相应的火星失明、水星光明,赤色的谷类不能成熟。火被水抑,其子气土气来复,于是土湿之气,上蒸为云,大雨将至,水气受到抑制,病见大便溏泄,腹中胀满,饮食不下,腹中寒冷鸣响,大便泄泻如注,腹中疼痛,突然拘挛、萎缩麻木、两足不能支撑身体。在天上相应的土星光明、水星失明,黑色谷物不能成熟。\\
岁土之运不及,风气因而大规模流行,土气失去生化能力,风气旺盛,则草木茂盛繁荣。但因过分飘扬,虽外秀而不能结实,在天上相应的木星光明。百姓的疾病多见泄泻,霍乱,身体沉重,腹中疼痛,筋骨动摇,肌肉跳动酸疼,易怒。寒水的藏气失制而旺盛,虫类提早伏藏在土中,在人都病寒泄中满,在天上相应木星光明、土星失色,黄色谷类不能成熟。土受木克,其子气金气来复,秋收之气当令,呈现严肃峻烈之气,大树也枝叶凋谢,人体则胸胁突然疼痛,连及少腹,长出气。凡味甘色黄的五谷被虫蛀食,邪气客于脾土,黄色谷物减产,百姓食物减少,饮食失养。金气胜木,所以青色谷物受到损害,在天上相应的金星光亮、土星失明。如遇厥阴司天相火在泉,则流水不能结冰,本已冬眠的虫类,重新出现,寒水之气不能主事,金气也不能复盛,在天上相应木星光明,人们也就健康了。\\
岁金之运不及,火气就会流行,木气得行政令,生长之气专胜,万物因而茂盛,气候干燥炎热。在天上相应的火星光明。人们多患肩背闷重,鼻流清涕,喷嚏,大便下血,泄泻如注。秋收之气后于天时而至,在天上相应的金星失明、火星光明,白色的谷类不能及时成熟。金气被抑制,其子气水气来复,寒雨突然降下,以致降落冰雹霜雪,杀伐万物,人体阴寒之气厥逆而格拒,阳气反而上行,后头部疼,连及头顶,发热。在天上相应水星光明、火星失明,红色谷物不能成熟。人们多病口中生疮,甚至心痛。\\
岁水之运不及,湿气大规模流行,水不制火,火气反行其令,暑雨多降。在天上相应的土星光明。人们多患腹部胀满,身体困重,大便溏泄,阴性疮疡脓水稀薄,腰股疼痛,下肢关节活动不利,烦闷抑郁,两脚痿弱厥冷,脚底疼痛,甚至足背浮肿。这是由于冬藏之气不能行其政令,肾气不平衡,在天上相应的土星光明,水星失明,黑黍不能成熟。如遇太阴司天,寒水在泉,则大寒之气时时侵袭,虫类很早就伏藏冬眠,地上结成厚冰,阳气伏藏,不能发挥温暖作用。人们多患下半身的寒性病,甚至腹满浮肿,在天上相应的土星光明、火星失明,谷类黄色之稻成熟。水气被抑制,其子气木气来复,因而大风暴发,草类偃伏,树木凋零,生长之气不能发挥作用。人的面色时时改变,筋骨拘急疼痛,肌肉跳动抽掣,两眼昏花,视物不清,物体看上去像裂开的样子,肌肉发出风疹,若邪气侵入胸膈之中,则心腹疼痛。这是木气太过,土气受伤,黄色的谷类没有收获,在天上相应的木星光明,土星失明。\\
黄帝说:很对。希望听听五气与四时相应的关系。\\
岐伯说:问得真详细啊!木运不及的,如果春天有和风使草木萌芽抽条的正常时令,那秋天也就有雾露润泽而凉爽的正常气候;如果春天反见寒冷惨凄霜冻残贼的秋天气候,那夏天就有炎热燔烧的气候。它的灾害发生在东方,在人体应在肝脏,其发病部位内在胠胁部,外在关节。\\
火运不及的,如果夏天有显明的正常气候,那冬天也就有严肃霜寒的正常时令;如果夏天反见萧条惨凄寒冷的冬天气候,那就时常会有尘埃昏蒙和倾盆大雨的情况。它的灾害发生在南方,在人体应在心脏,其发病部位内在胸胁部,外在经络。\\
土运不及的,如果辰、戌、丑、未月有尘土飘扬云雾润泽的正常时令,那春天就有风和鸟鸣,草木萌芽的正常气候;如果辰、戌、丑、未月有狂风拔倒树木的变化,那秋天也就有久雨不止的肃杀气象。它的灾害发生在四隅,在人体应在脾脏,其发病部位内在心腹,外在肌肉四肢。\\
金运不及的,如果夏天有显明湿气郁蒸的正常时令,那冬天也就有冰冻寒冷的正常气候;如果夏天出现有如烈火烧灼的炎热气候,那秋天就会有冰雹霜雪的反应。它的灾害发生在西方,在人体应在肺脏,其发病部位内在胸胁肩背,外在皮毛。\\
水运不及的,辰、戌、丑、未月有湿润埃云的正常气候,则时常有和风生发的正常反应;如果辰、戌、丑、未月出现尘埃迷暗,狂风暴雨的变化,则时时会有暴风骤起、吹断树木的反应。它的灾害发生在北方,在人体应在肾脏,其发病部位内在腰脊骨髓,外在谿谷踹膝。\\
五运之气的作用,好比权衡之器,太过的加以抑制,不及的加以辅助,正常的气化则和平地反应,反常的气化则必使其复原,这是万物生长化收藏的道理,是四时气候应有的规律。如果失去了这些规律,天地之气就闭塞不通了。所以说:天地的动静,受自然内在规律的控制;阴阳往来的变化,从四时寒暑来显示出征兆。说的就是这个意思。\\
黄帝说:夫子所讲的五气变化与四时气候的相应,可以说很详尽了。既然气的动乱是互相遇合而发生的,发作又没有一定规律,突然相遇而发生灾害,怎样预知呢?\\
岐伯说:五气的变动,固然没有一定常规,然而它们的德、化、政、令和灾变,都是各不相同的。\\
黄帝问:这是什么道理?\\
岐伯说:东方生风,风能使木气旺盛。木的特性是敷布和气,它的生化是滋生繁荣万物,它的职权是舒展阳气,宣通筋络,它的表现是风,它的异常变化是狂风怒号,它的灾害是摧残散落。\\
南方生热,热能使火气旺盛。火的特性是光明显著,它的生化作用是使万物繁荣茂盛,它的职权是明亮光耀,它的表现是热,它的异常变化是销烁煎熬,它的灾害作用是焚烧。\\
中央生湿,湿能使土气旺盛。土的特性是湿热滋润,它的生化作用是使万物充实丰满,它的职权是使万物安静,它的表现是湿,它的异常变化是暴雨骤降,它的灾害是久雨不止,土溃泥烂。\\
西方生燥,燥能使金气旺盛。金的特性是清洁,它的生化是使万物紧缩收敛,它的职权是使万物锐急,它的表现是干燥,它的异常变化是肃杀,它的灾害是青干凋落。\\
北方生寒,寒能使水气旺盛。水的特性是寒冷,它的生化是使万物清静而安谧,它的职权是使万物凝固严整,它的表现是寒冷,它的异常变化是严寒冰冻,它的灾害是冰雹霜雪。所以观察它的运动,有特性、生化、权力、表现、变异、灾害,而万物与之相随,人也与之相应。\\
黄帝问:夫子讲过五运的不及太过,与天上的五星相应。现在五运的德、化、政、令、灾害、变异,并不是按常规发生,而是突然的变化,五运是否也随之变动呢?\\
岐伯说:五星是随天道的运动而运动的,所以它不会妄动,不存在不应的问题。突然而来的变动,是气候相交合所发生的偶然变化,与天道运行无关,所以五星不受影响。因此说:五星应常规,不应突变。说的就是这个意思。\\
黄帝问:五星与天运正常相应的规律怎样呢?\\
岐伯说:各从其天运之气而变化。\\
黄帝问:五星运行的缓慢迅速、逆行顺行,情况是怎样的?\\
岐伯说:五星在它的轨道上运行,如久延而不进,或逆行留守,而光芒变小,这叫“省下”,好像察看所属分野的情况;若在其轨道上去而速回,或迂回而行的,称为“省遗过”,好像察看所属分野中的情况是否有遗漏和过错;若久延不进而回环旋转,似去似来的,称为“议灾”或“议德”,好像建议在其所属的分野中降灾和降福。气候的变化近则小,变化远则大。光芒大于正常一倍的,气化亢盛;大二倍的,灾害即发。小于正常一倍的,气化衰减,小二倍的,称为“临视”,省察在下之过与德。有德的降福,有过的降灾。所以五星之象的显现,高而远的就小,低而近的就大。大则喜怒灾变的感应近,小则祸福降临的日期远。岁运太过的,主运之星就向北越出常道;运气相和,则五星各运行在正常的轨道上。所以岁运太过,被制之星就暗淡而兼见母星的颜色;岁运不及,那运星就兼见所不胜的颜色。顺从天地的人,看见了天道的微妙变化,而不知其理,心里非常忧惧,道理深远而适宜,但谁能明白它的好处呢?那些无知之人,妄行猜测,毫无征验,徒然用来恐吓侯王。\\
黄帝问:五星在灾害方面的应验,情况怎样?\\
岐伯说:也是各从岁运的变化而变化的。所以岁时的更至有盛衰,运星的侵犯有逆顺,留守日期有长短,五星所见的形象有好坏,星宿所属有胜负,征验所应就有吉有凶了。\\
黄帝问:星象的好坏怎样?\\
岐伯说:五星的形象有喜、怒、忧、丧、泽、燥的不同。这是星象变化所常见的,必须小心观察。\\
黄帝问:星象的喜、怒、忧、丧、泽、燥六种现象,与五星的高低有无关系?\\
岐伯说:五星的形象虽有高下的不同,但其应验是一致的,所以人体也与之相应。\\
黄帝说:讲得对。它们的德、化、政、令的动静损益,都是怎样的?\\
岐伯说:德、化、政、令与灾变都有一定规律,不能彼此相加。胜盛则复盛,胜衰则复衰,不能随意增多。胜复往来的日数,多少相同,不能随便超越。五行阴阳的升降只有互相结合才能发挥作用,不会在没有对方的情况下单独存在。这些都是从运动中所产生而往复循环的。\\
黄帝问:它们对疾病发生有什么影响?\\
岐伯说:德、化是五气正常的吉祥征兆;政、令是五气的职权和表现形式;变易是产生胜气与复气的纲纪;灾祸是万物损伤的开始。大凡人气与岁气相应的,就和平无病;不相应的,就会生病;重复感受邪气病就严重了。\\
黄帝说:讲得好。这些都是所谓的精深高明的理论,圣人的伟大事业,阐扬医学的大道理,达到了无穷无尽的境界。我听说:善于谈论天道的,必定能应验于人身;善于谈论古代的,必定能验证于现在;善于谈论气化的,必定能通晓万物;善于谈论感应的,就会与天地造化相统一;善于谈论化与变的,就会通达自然界变化莫测的道理。除非夫子,还有谁能够说清楚这些精深的道理呢?\\
于是选了一个好日子,把它藏在灵兰书室里,每天早晨取出来诵读,命名为《气交变》。不进行斋戒,就不敢随便打开,非常慎重地传于后世。\\
五常政大论篇第七十\\
黄帝问曰:太虚寥廓,五运回薄,衰盛不同,损益相从,愿闻平气,何如而名?何如而纪也?\\
岐伯对曰:昭乎哉问也!木曰敷和,火曰升明,土曰备化,金曰审平,水曰静顺。\\
帝曰:其不及奈何?\\
岐伯曰:木曰委和,火曰伏明,土曰卑监,金曰从革,水曰涸流。\\
帝曰:太过何谓?\\
岐伯曰:木曰发生,火曰赫曦,土曰敦阜,金曰坚成,水曰流衍。\\
帝曰:三气之纪,愿闻其候。\\
岐伯曰:悉乎哉问也!敷和之纪,木德周行,阳舒阴布,五化宣平。其气端,其性随,其用曲直,其化生荣,其类草木,其政发散,其候温和,其令风,其脏肝,肝其畏清,其主目,其谷麻,其果李,其实核,其应春,其虫毛,其畜犬,其色苍,其养筋,其病里急支满,其味酸,其音角,其物中坚,其数八。\\
升明之纪,正阳而治,德施周普,五化均衡。其气高,其性速,其用燔灼,其化蕃茂,其类火,其政明曜,其候炎暑,其令热,其脏心,心其畏寒,其主舌,其谷麦,其果杏,其实络,其应夏,其虫羽,其畜马,其色赤,其养血,其病丱瘛,其味苦,其音徵,其物脉,其数七。\\
备化之纪,气协天休,德流四政,五化齐修。其气平,其性顺,其用高下,其化丰满,其类土,其政安静,其候溽蒸,其令湿,其脏脾,脾其畏风,其主口,其谷稷,其果枣,其实肉,其应长夏,其虫倮,其畜牛,其色黄,其养肉,其病否,其味甘,其音宫,其物肤,其数五。\\
审平之纪,收而不争,杀而无犯,五化宣明。其气洁,其性刚,其用散落,其化坚敛,其类金,其政劲肃,其候清切,其令燥,其脏肺,肺其畏热,其主鼻,其谷稻,其果桃,其实壳,其应秋,其虫介,其畜鸡,其色白,其养皮毛,其病咳,其味辛,其音商,其物外坚,其数九。\\
静顺之纪,藏而勿害,治而善下,五化咸整。其气明,其性下,其用沃衍,其化凝坚,其类水,其政流演,其候凝肃,其令寒,其脏肾,肾其畏湿,其主二阴,其谷豆,其果栗,其实濡,其应冬,其虫麟,其畜彘,其色黑,其养骨髓,其病厥,其味咸,其音羽,其物濡,其数六。\\
故生而勿杀,长而勿罚,化而勿制,收而勿害,藏而勿抑。是谓平气。\\
委和之纪,是谓胜生。生气不政,化气乃扬,长气自平,收令乃早。凉雨时降,风云并兴,草木晚荣,苍干凋落,物秀而实,肤肉内充。其气敛,其用聚,其动仩戾拘缓,其发惊骇,其脏肝,其果枣李,其实核壳,其谷稷稻,其味酸辛,其色白苍,其畜犬鸡,其虫毛介,其主雾露凄沧,其声角商,其病摇动注恐,从金化也。少角与判商同,上角与正角同,上商与正商同,其病支废,痈肿疮疡,其甘虫,邪伤肝也。上宫与正宫同。萧仚肃杀,则炎赫沸腾,眚于三,所谓复也。其主飞蠹蛆雉,乃为雷霆。\\
伏明之纪,是谓胜长。长气不宣,脏气反布,收气自政,化令乃衡。寒清数举,暑令乃薄,承化物生,生而不长,成实而稚,遇化已老。阳气屈伏,蛰虫早藏。其气郁,其用暴,其动彰伏变易。其发痛,其脏心,其果栗桃,其实络濡,其谷豆稻,其味苦咸,其色玄丹,其畜马彘,其虫羽鳞,其主冰雪霜寒,其声徵羽,其病昏惑悲忘,从水化也。少徵与少羽同,上商与正商同,邪伤心也。凝惨凛冽,则暴雨霖霪,眚于九。其主骤注雷霆震惊,沉霒淫雨。\\
卑监之纪,是谓减化。化气不令,生政独彰,长气整,雨乃愆,收气平,风寒并兴,草木荣美,秀而不实,成而粃也。其气散,其用静定,其动疡涌分溃痈肿,其发濡滞,其脏脾,其果李栗,其实濡核,其谷豆麻,其味酸甘,其色苍黄,其畜牛犬,其虫倮毛,其主飘怒振发,其声宫角,其病留满否塞,从木化也。少宫与少角同,上宫与正宫同,上角与正角同,其病飧泄,邪伤脾也。振拉飘扬,则苍干散落,其眚四维。其主败折虎狼,清气乃用,生政乃辱。\\
从革之纪,是谓折收。收气乃后,生气乃扬,长化合德,火政乃宣,庶类以蕃。其气扬,其用躁切,其动铿禁瞀厥,其发咳喘,其脏肺,其果李杏,其实壳络,其谷麻麦,其味苦辛,其色白丹,其畜鸡羊,其虫介羽,其主明曜炎烁,其声商徵,其病嚏咳鼽衄,从火化也。少商与少徵同,上商与正商同,上角与正角同,邪伤肺也。炎光赫烈,则冰雪霜雹,眚于七。其主鳞伏彘鼠。岁气早至,乃生大寒。\\
涸流之纪,是谓反阳。藏令不举,化气乃昌,长气宣布,蛰虫不藏,土润水泉减,草木条茂,荣秀满盛。其气滞,其用渗泄,其动坚止,其发燥槁,其脏肾,其果枣杏;其实濡肉,其谷黍稷,其味甘咸,其色黅玄,其畜彘牛,其虫鳞倮,其主埃郁昏翳,其声羽宫,其病痿厥坚下,从土化也。少羽与少宫同,上宫与正宫同,其病癃闭,邪伤肾也。埃昏骤雨,则振拉摧拔,眚于一。其主毛显狐狢,变化不藏。\\
故乘危而行,不速而至,暴虐无德,灾反及之。微者复微,甚者复甚,气之常也。\\
发生之纪,是谓启刌。土疏泄,苍气达,阳和布化,阴气乃随,生气淳化,万物以荣。其化生,其气美,其政散,其令条舒。其动掉眩巅疾,其德鸣靡启坼,其变振拉摧拔,其谷麻稻,其畜鸡犬,其果李桃,其色青黄白,其味酸甘辛,其象春,其经足厥阴少阳,其脏肝脾,其虫毛介,其物中坚外坚,其病怒。太角与上商同。上徵则其气逆,其病吐利。不务其德,则收气复,秋气劲切,甚则肃杀,清气大至,草木凋零,邪乃伤肝。\\
赫曦之纪,是谓蕃茂。阴气内化,阳气外荣,炎暑施化,物得以昌。其化长,其气高,其政动,其令鸣显,其动炎灼妄扰,其德暄暑郁蒸,其变炎烈沸腾,其谷麦豆,其畜羊彘,其果杏栗,其色赤白玄,其味苦辛咸,其象夏,其经手少阴太阳,手厥阴少阳,其脏心肺,其虫羽鳞,其物脉濡。其病笑疟疮疡血流狂妄目赤。上羽与正徵同,其收齐,其病痓,上徵而收气后也。暴烈其政,藏气乃复,时见凝惨,甚则雨水霜雹切寒,邪伤心也。\\
敦阜之纪,是谓广化。厚德清静,顺长以盈,至阴内实,物化充成,烟埃朦郁,见于厚土,大雨时行,湿气乃用,燥政乃辟,其化圆,其气丰,其政静,其令周备。其动濡积并稸,其德柔润重淖,其变震惊飘骤崩溃。其谷稷麻,其畜牛犬,其果枣李,其色黅玄苍,其味甘咸酸,其象长夏,其经足太阴阳明,其脏脾肾,其虫倮毛,其物肌核。其病腹满,四支不举,大风迅至,邪伤脾也。\\
坚成之纪,是谓收引。天气洁,地气明,阳气随阴治化,燥行其政,物以司成,收气繁布,化洽不终。其化成,其气削,其政肃,其令锐切,其动暴折疡疰,其德雾露萧仚,其变肃杀凋零,其谷稻黍,其畜鸡马,其果桃杏,其色白青丹,其味辛酸苦,其象秋,其经手太阴阳明,其脏肺肝,其虫介羽,其物壳络,其病喘喝,胸凭仰息。上徵与正商同。其生齐,其病咳。政暴变,则名木不荣,柔脆焦首,长气斯救,大火流,炎烁且至,蔓将槁,邪伤肺也。\\
流衍之纪,是谓封藏。寒司物化,天地严凝,藏政以布,长令不扬。其化凛,其气坚,其政谧,其令流注,其动漂泄沃涌,其德凝惨寒雰,其变冰雪霜雹,其谷豆稷,其畜彘牛,其果栗枣,其色黑丹黅,其味咸苦甘,其象冬,其经足少阴太阳,其脏肾心,其虫鳞倮,其物濡满,其病胀。上羽而长气不化也。政过则化气大举,而埃昏气交,大雨时降,邪伤肾也。\\
故曰:不恒其德,则所胜来复,政恒其理,则所胜同化。此之谓也。\\
帝曰:天不足西北,左寒而右凉;地不满东南,右热而左温。其故何也?\\
岐伯曰:阴阳之气,高下之理,太少之异也。东南方,阳也,阳者其精降于下,故右热而左温;西北方,阴也,阴者其精奉于上,故左寒而右凉。是以地有高下,气有温凉,高者气寒,下者气热,故适寒凉者胀,之温热者疮。下之则胀已,汗之则疮已。此腠理开闭之常,太少之异耳。\\
帝曰:其于寿夭何如?\\
岐伯曰:阴精所奉其人寿,阳精所降其人夭。\\
帝曰:善。其病也,治之奈何?\\
岐伯曰:西北之气,散而寒之;东南之气,收而温之。所谓同病异治也。故曰:气寒气凉,治以寒凉,行水渍之;气温气热,治以温热,强其内守,必同其气,可使平也。假者反之。\\
帝曰:善。一州之气,生化寿夭不同,其故何也?\\
岐伯曰:高下之理,地势使然也。崇高则阴气治之,洿下则阳气治之。阳胜者先天,阴胜者后天。此地理之常,生化之道也。\\
帝曰:其有寿夭乎?\\
岐伯曰:高者其气寿,下者其气夭。地之小大异也,小者小异,大者大异。故治病者,必明天道地理,阴阳更胜,气之先后,人之寿夭,生化之期,乃可以知人之形气矣。\\
帝曰:善。其岁有不病,而脏气不应不用者,何也?\\
岐伯曰:天气制之,气有所从也。\\
帝曰:愿卒闻之。\\
岐伯曰:少阳司天,火气下临,肺气上从,白起金用,草木眚,火见燔焫,革金且耗,大暑以行。咳嚏鼽衄鼻窒,口疡,寒热胕肿。风行于地,尘沙飞扬。心痛胃脘痛,厥逆鬲不通,其主暴速。\\
阳明司天,燥气下临,肝气上从,苍起木用而立,土乃眚,凄沧数至,木伐草萎。胁痛目赤,掉振鼓慄,筋痿不能久立。暴热至,土乃暑,阳气郁发,小便变,寒热如疟,甚则心痛。火行于槁,流水不冰,蛰虫乃见。\\
太阳司天,寒气下临,心气上从,而火且明,丹起,金乃眚,寒清时举,胜则水冰,火气高明。心热烦,嗌干善渴,鼽嚏,喜悲数欠。热气妄行,寒乃复,霜不时降,善忘,甚则心痛。土乃润,水丰衍,寒客至,沉阴化,湿气变物,水饮内稸,中满不食,皮吇,肉苛,筋脉不利,甚则胕肿,身后痈。\\
厥阴司天,风气下临,脾气上从,而土且隆,黄起,水乃眚,土用革。体重,肌肉萎,食减口爽。风行太虚,云物摇动,目转耳鸣。火纵其暴,地乃暑,大热消烁,赤沃下。蛰虫数见,流水不冰,其发机速。\\
少阴司天,热气下临,肺气上从,白起金用,草木眚。喘呕寒热,嚏鼽衄鼻窒。大暑流行,甚则疮疡燔灼,金烁石流。地乃燥清,凄沧数至。胁痛,善太息。肃杀行,草木变。\\
太阴司天,湿气下临,肾气上从,黑起水变,火乃眚,埃冒云雨。胸中不利,阴痿气大衰,而不起不用。当其时,反腰脽痛,动转不便也,厥逆。地乃藏阴,大寒且至,蛰虫早附,心下否痛。地裂冰坚。少腹痛,时害于食。乘金则止水增,味乃咸,行水减也。\\
帝曰:岁有胎孕不育,治之不全,何气使然?\\
岐伯曰:六气五类,有相胜制也。同者盛之,异者衰之。此天地之道,生化之常也。故厥阴司天,毛虫静,羽虫育,介虫不成;在泉,毛虫育,倮虫耗,羽虫不育。少阴司天,羽虫静,介虫育,毛虫不成;在泉,羽虫育,介虫耗不育。太阴司天,倮虫静,鳞虫育,羽虫不成;在泉,倮虫育,鳞虫不成。少阳司天,羽虫静,毛虫育,倮虫不成;在泉,羽虫育,介虫耗,毛虫不育。阳明司天,介虫静,羽虫育,介虫不成;在泉,介虫育,毛虫耗,羽虫不成。太阳司天,鳞虫静,倮虫育;在泉,鳞虫耗,倮虫不育。诸乘所不成之运,则甚也。故气主有所制,岁立有所生。地气制己胜,天气制胜己;天制色,地制形。五类衰盛,各随其气之所宜也,故有胎孕不育,治之不全,此气之常也,所谓中根也。根于外者亦五,故生化之别,有五气、五味、五色、五类、五宜也。\\
帝曰:何谓也?\\
岐伯曰:根于中者,命曰神机,神去则机息。根于外者,命曰气立,气止则化绝。故各有制,各有胜,各有生,各有成。故曰:不知年之所加,气之同异,不足以言生化。此之谓也。\\
帝曰:气始而生化,气散而有形,气布而蕃育,气终而象变,其政一也。然而五味所资,生化有薄厚,成熟有少多,终始不同,其故何也?\\
岐伯曰:地气制之也,非天不生,地不长也。\\
帝曰:愿闻其道。\\
岐伯曰:寒热燥湿,不同其化也。故少阳在泉,寒毒不生,其味辛,其治苦酸,其谷苍丹。阳明在泉,湿毒不生,其味酸,其气湿,其治辛苦甘,其谷丹素。太阳在泉,热毒不生,其味苦,其治淡咸,其谷黅秬。厥阴在泉,清毒不生,其味甘,其治酸苦,其谷苍赤。其气专,其味正。少阴在泉,寒毒不生,其味辛,其治辛苦甘,其谷白丹。太阴在泉,燥毒不生,其味咸,其气热,其治甘咸,其谷黅秬。化淳则咸守,气专则辛化而俱治。\\
故曰:补上下者从之,治上下者逆之,以所在寒热盛衰而调之。故曰:上取下取,内取外取,以求其过。能毒者以厚药,不胜毒者以薄药,此之谓也。气反者,病在上,取之下;病在下,取之上;病在中,傍取之。治热以寒,温而行之;治寒以热,凉而行之,治温以清,冷而行之,治清以温,热而行之。故消之削之,吐之下之,补之泻之,久新同法。\\
帝曰:病在中而不实不坚,且聚且散,奈何?\\
岐伯曰:悉乎哉问也!无积者求其藏,虚则补之,药以祛之,食以随之,行水渍之,和其中外,可使毕已。\\
帝曰:有毒无毒,服有约乎?\\
岐伯曰:病有久新,方有大小,有毒无毒,固宜常制矣。大毒治病,十去其六;常毒治病,十去其七;小毒治病,十去其八;无毒治病,十去其九。谷肉果菜,食养尽之,无使过之,伤其正也。不尽,行复如法,必先岁气,无伐天和。无盛盛,无虚虚而遗人夭殃;无致邪,无失正,绝人长命。\\
帝曰:其久病者,有气从不康,病去而瘠,奈何?\\
岐伯曰:昭乎哉圣人之问也!化不可代,时不可违。夫经络以通,血气以从,复其不足,与众齐同,养之和之,静以待时,谨守其气,无使倾移,其形乃彰,生气以长,命曰圣王。故《大要》曰:无代化,无违时,必养必和,待其来复。此之谓也。\\
帝曰:善。\\
黄帝问:宇宙太虚深远广阔无边,五运循环不息。其中有盛衰的不同,随之而人体也有损益的差别,希望听听五运中的平气,是怎样命名的?怎样确定其标志的?\\
岐伯回答说:问得真有高明啊!木的平气,是敷布和柔,称为“敷和”;火的平气,是上升而光明,称为“升明”;土的平气,是广布生化,称为“备化”;金的平气,是清静平和,称为“审平”;水的平气,是静穆顺达,称为“静顺”。\\
黄帝问:不及怎样?\\
岐伯说:如果不及,木就委曲无阳和之气,称为“委和”;火就伏藏而无光明,称为“伏明”;土就低下而缺乏生化之气,称为“卑监”;金就可因可革而无坚硬之气,称为“从革”;水就干涸而无湿润之气,称为“涸流”。\\
黄帝问:太过怎样?\\
岐伯说:如果太过,木就会发生过早,称为“发生”;火就会炎势太盛,称为“赫曦”;土就会过于高厚,称为“敦阜”;金就会过于刚硬,称为“坚成”;水就会满溢外流,称为“流衍”。\\
黄帝说:以上平气、太过和不及三气的标志,希望听听怎样候察。\\
岐伯说:问得全面啊!敷和的标志,木的德性敷布畅达于四方上下,阳气舒畅,阴气散布,五行的气化都能发挥其正常的功能。其气正直,其性顺从万物,其作用如树木枝干的曲直自由伸展,其生化能使万物繁荣,其属类是草木,其职能是发散,其气候是温和,其职能的表现是风,应于人的内脏是肝,肝畏惧清凉的肺金之气,肝关联于眼目,在谷类是麻,果类是李,其果实是核仁,所应的时令是春,在虫类是毛虫,在畜类是犬,在颜色是苍,其所充养的是筋,其发病则为里急而胀满,其在五味是酸,在五音是角,在物体来说是属于中坚,其在河图成数是八。\\
升明的标志,南方火运正常行令,其德性普及四方,五行气化平衡。其气上升,其性急速,其作用是燃烧,其生化能使万物繁荣茂盛,其属类是火,其职能是光明显耀,其气候炎暑,其职能的表现是热,应于人体内脏是心,心畏惧寒冷的水气,心关联着舌,其在谷类是麦,果类是杏,其在果实是丝络,所应的时令是夏,在虫类是羽虫,在畜类是马,其在颜色是赤,其所充养的是血,其发病则为肌肉跳动,身体抽搐,其在五味是苦,在五音是徵,在物体属于络脉一类,其在河图成数是七。\\
备化的标志,天地气化协调和平,其德性流布于四方,五行气化都能均衡完善地进行。其气和平,其性和顺,其作用能高能下,其生化能使万物成熟丰满,其属类是土,其职能是使万物安静,其气候是湿热交蒸,其职能的表现是湿,应于人体内脏是脾,脾畏惧风,脾关联着口,其在谷类是稷,果类是枣,其在果实是果肉,其所应的时令是长夏,在虫类是倮虫,在畜类是牛,在颜色是黄,其充养的是肉,其发病为痞塞,在五味是甘,在五音是宫,在物体是属于肌肤一类,在河图成数是五。\\
审平的标志,金的气化收敛而无剥夺,肃杀而无残害,五行的气化都能宣畅清明。审平之气洁净,其性刚强,其作用是使万物成熟散落,其生化能使万物结实收敛,其属类是金,其职能为清劲严肃,其气候清凉急切,其职能的表现是燥,应于人体的内脏是肺;肺畏火热,肺关联着鼻,其在谷类是稻,果类是桃,所充实的是外壳,其所应的时令是秋,在虫类是介虫,在畜类是鸡,在颜色是白,其充养的是皮毛,其发病为咳嗽,在五味是辛,在五音是商,在物体是属于外壳坚硬的一类,在河图成数是九。\\
静顺的标志,藏气能藏纳而无害于万物,其生化平顺而下行,五行的气化都能完整。其气明静,其性向下,其作用为水流灌溉,其生化为凝固坚硬,其属类为水,其职能是流动不息,其气候严寒肃静,其职能的表现是寒,应于人体的内脏是肾,肾怕湿土,肾关联着二阴,在谷类是豆,果类是栗,在果实是液汁,其所应的时令是冬,在虫类是鳞虫,在畜类是猪,其颜色是黑,其充养的是骨髓,其发病则为厥逆,在五味是咸,在五音是羽,在物体是属于液体一类,在河图成数是六。\\
所以万物发生时而不杀伤,成长时而不惩罚,化育时而不制止,收敛时而不残害,藏储时而不抑制。这就叫做平气。\\
委和的标志,称为胜生。生气不能发挥作用,土之化气于是发扬播散,火之长气自然平静,收令于是提早到来。凉雨不时下降,风云交相变换,草木繁荣晚于时令,并且易于干枯凋落,万物早秀早熟,皮肉充实。其气收敛,其作用聚集,在人体的变动是筋络拘挛或软弱无力,或者易于惊骇,其应于内脏为肝,在果类是枣、李,在果实中是核和壳,在谷类是稷和稻,在五味是酸和辛,在颜色是白而苍,在畜类是犬和鸡,在虫类是毛虫介虫,所主的气候是雾露寒冷,在声音为角和商,其病变为摇动和恐惧,这是木运不及而从金化的缘故。所以少角等同于半商,上角与正角相同,上商与正商相同,其发病为四肢痿弱、痈肿、疮疡、生虫等病,这是因为邪气伤肝。这时上宫与正宫相同。木受金克,起初是一片萧瑟肃杀的景象,但随之则为火势炎炎,其灾害应于东方,这是由于金气克木,火气前来报复。当火气来复,属火的飞虫、蠹虫、蛆虫和雉鸡应之而出,木郁至极,就会震发而为雷霆。\\
伏明的标志,称为胜长。火的长气不得发扬,水的藏气反而乘机布散,金的收气也自行职权,土的化气平定而不能发展,寒冷之气常现,暑热之气衰减,万物虽承土的化气而生,但因火运不足,生而不能成长,虽能结实,然而幼小,及至长夏生化的时候,已经衰老了。由于阳气伏藏,所以蛰虫很早就蛰藏起来了。火气郁结,所以当其发用时,必然横暴,其变动每隐现多变,无一定之规。其发病为疼痛,其应于内脏为心,其在果类为栗和桃,其在果实是丝络和液汁,在谷类为豆和稻,在五味为苦和咸,在颜色为玄和丹,在畜类为马和猪,在虫类是羽虫鳞虫,在气候主冰、雪、霜、寒,在声音为徵、羽,其病变为昏乱糊涂,悲哀易忘,这是火运不及而从水化的缘故。所以少徵和少羽相同,上商与正商相同,这是邪气伤心所致。火运衰弱,所以阴凝惨淡、寒风凛冽,随之而暴雨淋漓不止,其灾害应于南方。所以伏明主暴雨下注,雷霆震惊,乌云蔽日,阴雨连绵。\\
卑监的标志,称为减化。土的化气不得行其政令,而木的生气独旺,长气自能完整如常,雨水过期不降,收气平定,风寒并起,草木虽繁荣美丽,但秀而不能成实,所成的只是粃子一类不饱满的东西。其气散漫,其作用不足而过于静定,其变动为病发疮疡溃烂、痈肿,并发展为水气不行的水肿,其所应的内脏是脾,在果类是李和栗,所充实的是液汁和核仁,在谷类是豆和麻,在五味是酸、甘,在颜色是苍、黄,在畜类是牛和犬,在虫类是倮虫毛虫,其所主的气候是大风刮起,树木动摇,有摧折之势,在声音为宫、角,其发病为胀满否塞不通,这是土运不及而从木化的缘故。所以少宫和少角相同,上宫和正宫相同,上角和正角相同,其发病为泄泻,这是邪气伤脾所致。土衰木胜,所以见风势振动,树木摧折飘扬,随之而草木干枯凋落,其灾害应于中宫而通于四方。其所主败坏折伤,有如虎狼之势,清冷之气发生作用,生气被抑制而不能行使职能。\\
从革的标志,称为折收。金之收气后于天时而至,生气得以张扬,火之长气和土之化气合而相得,火的职能得以施行,万物繁茂。其气发扬,其作用是急躁,其变动发病为咳嗽失音、胸闷气逆,发展为咳嗽气喘,其所应的内脏是肺,在果类为李和杏,在果实是外壳和丝络,在谷类是麻和麦,在五味是苦与辛,在颜色为白和丹,在畜类为鸡和羊,在虫类是介虫羽虫,其所主的气候是晴朗炎热,在声音为商、徵,其病变为喷嚏、咳嗽、鼻塞流涕、衄血,这是金运不及而从火化的缘故。所以少商和少徵相同,上商和正商相同,上角和正角相同,其病变是由于邪气伤肺。金衰火旺,所以火势炎热,火气过盛则水气来复,随之见冰雪霜雹,其灾害应于西方。其所主的鳞虫之伏藏,猪、鼠之阴沉,冬藏之气提早而至,于是发生大寒。\\
涸流的标志,称为反阳。水之藏气衰弱,不能行使其封藏的职能,土之化气因而昌盛,火之长气乘机宣行而布于四方,蛰虫不按时伏藏,土润泽而泉水减少,草木条达繁茂,万物繁荣秀丽而丰满。藏气不得流畅,其作用为暗中渗泄,其变动为癥结不动,发病为干燥枯槁,其应内脏为肾,在果类为枣、杏,在果实是汁液和肉,在谷类是黍和稷,在五味是甘、咸,在颜色是黄、黑,在畜类是猪、牛,在虫类是鳞虫和倮虫,其所主的气候,是尘土飞扬,天空昏暗,在声音为羽、宫,其病变为痿厥和下部癥结,这是水运不及而从土化的缘故。所以少羽和少宫相同,上宫与正宫相同,其病见大小便不畅或闭塞不通,这是邪气伤肾所致。水运不及,所以尘埃昏蔽,或骤然下雨,但木气来复,随之反见大风振动,树木倒拔,其灾害应于北方,毛虫像狐狢之类应之而出,变化而不潜藏。\\
所以五运有不及之时,所胜与所不胜之气,就乘其衰弱而行令,好像不速之客,不招自来,暴虐而毫无道德,灾害必然反加到自己身上,这是子来报复的关系。暴虐轻微的受到的报复就轻,严重的受到的报复也严重,这是运气中的常规。\\
发生的标志,称为启陈。土气因木气太过而疏松发泄,草木之青气条达,阳气温和布化于四方,阴气随从阳气,生气浮厚,化生万物,万物因之欣欣向荣。其运化为生发,其气秀美,其职能为散布,其表现为舒展畅达,其在人体的变动是颤摇、眩晕和巅顶部疾病,其性能是和风四布,推陈出新,若变化则为狂风振摇,树木摧折,其在谷类为麻、稻,在畜类是鸡、犬,在果实为李、桃,在颜色为青、黄、白,在五味为酸、甘、辛,其象征为春天,其在人体的经脉是足厥阴、足少阳,在内脏为肝、脾,在虫类为毛虫介虫,在物体属内外坚硬,其病变则为怒。这时太角与上商同。若遇上徵少阴君火司天,火性上逆,木旺克土,故病发气逆、吐泻。若木气太过,不守自己的品行而去侮土,则金之收气来复,以致发生秋令劲急的景象,甚则有肃杀之气,突然气候清凉,草木凋零,邪气会损伤肝脏。\\
赫曦的标志,称为蕃茂。阴气从内而化,阳气发扬在外,炎暑的气候施行,万物得以昌盛。其生化之气为成长,火气上升,其职能是活动,其表现为显露声色,其变动能使人身烧灼发热,神志撩乱不宁,其性能是暑热郁蒸,其变化则为炎热如烈火喧腾,其在谷类为麦、豆,在畜类为羊、猪,在果类为杏、栗,在颜色为赤、白、黑,在五味为苦、辛、咸,其象征为夏天,在人体的经脉是手少阴、手太阳和手厥阴、手少阳,在内脏为心、肺,在虫类为羽虫鳞虫,在物体属脉络和津液,在人体的病变主笑、疟疾、疮疡、失血、发狂、目赤。赫曦与上羽和正徵相同,收令得以正常施行,在人发病为痓,逢上徵则收气不能及时施行。由于火运暴烈,水之藏气来复,以致时见阴凝惨淡的景象,甚至雨水霜雹,极为寒冷,病变多是邪气伤心。\\
敦阜的标志,称为广化。其德性敦厚而清静,使万物顺时令生长乃至充盈,土的至阴之气内实,万物就能生化而成形,土运太过,则见土气蒸腾如烟,朦胧笼罩于山陵之上,大雨时常降下,湿气主事,燥气隐退。其运化圆满,其气丰盛,其职能主安静,其表现是周密详备,其变动在人体则湿气积聚,其性能使万物柔润光泽,其变化则为雷霆震动、暴雨骤至、山崩土溃,在谷类为稷、麻,在畜类为牛、犬,在果类为枣、李,在颜色为黄、黑、青,在五味是咸、酸,其象征为长夏,在人体的经脉是足太阴、足阳明,在内脏是脾、肾,在虫类是倮虫毛虫,在物体属于肌肉和果核一类。其病变为腹满,四肢举动不便,土运太过,木气来复,所以大风迅速而来,疾病多为邪气伤脾。\\
坚成的标志,称为收引。天气洁净,地气明朗,阳气随着阴气的职能而生化,燥金之气行使职权,因而万物都成熟,但秋收之气频繁四布,长夏的化气未能尽终其职能。其生化是提早收成,其气是削伐,其职权是肃杀凋零,其表现是尖锐急切,其在人体之变动为突然折伤和疮疡、皮肤病,其性能是散布雾露秋风,其变化则为肃杀凋零的景象,在谷类是稻、黍,在畜类是鸡、马,在果类是桃、杏,它的颜色是白、青、丹,它化生的五味是辛、酸、苦,其象征为秋天,在人体上相应的经脉是手太阴、手阳明,在内脏是肺与肝,其在虫类是介虫羽虫,生成物体是属于皮壳和丝络一类,其病变,大都为气喘有声,呼吸困难。这时上徵与正商相同。金气被抑制,则木气不受克制,生气就能与长、化、收、藏之气齐同,而正常行令,其病变为咳嗽。金运太过,行使职权过分暴虐,各种树木枯槁不荣,草类柔软脆弱,焦首干死,但继火之长气来复,所以炎热的天气又流行,蔓草被炙烤,渐至枯槁,病变多为邪气伤肺。\\
流衍的标志,称为封藏。水寒之气掌管万物的变化,天地间严寒阴凝,闭藏之气行使职权,火的生长之气不能发扬。其生化为凛冽,其气为坚凝,其职权为安静,它的表现是流动灌注,其在人体的变动则或为吐涎沫,或为下泻,其性能是阴凝惨淡的寒冷雾气,其气候的变化为冰雪霜雹,在谷类为豆、稷,在畜类是猪、牛,在果类为栗、枣,其颜色是黑、丹与黄,化生的五味是咸、苦、甘,其象征为冬天,在人体相应的经脉是足少阴、足太阳,在内脏是肾和心,在虫类为鳞虫倮虫,生成物体属充满液汁的一类,其病变是胀。流衍逢上羽,火的生长之气更不能布化。如果水运太过,则土之化气来复,而水土交争,大雨不时下降,病变多为邪气伤肾。\\
所以说:不能保持正常的性能,横施暴虐,而欺侮被我所胜者,但结果必有胜我者前来报复,若行使政令平和,即使所胜之气来侵,也能同化。说的就是这个意思。\\
黄帝问:天气不足于西北,北方寒,西方凉;地气不满于东南,南方热,东方温。这是什么缘故?\\
岐伯说:天气的阴阳,地理的高下,都随着四方疆域的大小而有不同。东南方属阳,阳的精气自上而下降,则南方热而东方温;西北方属阴,阴的精气自下而上承,则西方凉而北方寒。所以地势有高低,气候有温凉,地势高的气候就寒,地势低的气候就热,所以往西北寒凉地方去就容易有胀病,往东南温热的地方去就容易有疮疡。胀满,用通利药可治愈,疮疡,用发汗药可治愈。这是气候和地理影响人体腠理开闭的一般情况,在治疗上根据病情大小的不同而变化就可以了。\\
黄帝问:它与人的寿命长短有什么关系?\\
岐伯说:阴精上承的地方,腠理致密,其人多长寿;阳精下降的地方,腠理开发,其人多夭折。\\
黄帝说:说得好。但人有了病,应该怎样治疗呢?\\
岐伯说:西北方气候寒冷,应该散外寒清里热;东南方气候温热,应该收敛阳气温内寒。这就是同样的病证而治法不同的道理。所以说:气候寒凉的地方,多内热,可以用寒凉药治疗,并可用汤水浸渍;气候温热的地方,多内寒,可用温热药治疗,又必加强内守,不使真阳外泄,治法必须与当地的气候统一,这样可使气机平和。如果有真假寒热之病,又该用相反的方法治疗。\\
黄帝说:说得好。但同是一个地区的气候,而生化寿夭,各有不同,这是什么原因?\\
岐伯说:这是地势的高下不同导致的。地势高的地方多寒,属于阴气所治;地势低下的地方多热,属于阳气所治。阳气太过,四时气候就来得早;阴气太过,四时气候就来得晚。这就是地理高下与生化迟早关系的一般规律。\\
黄帝又说:那么它与寿夭也有关系吗?\\
岐伯说:地势高的地方,因为寒收则元气内守而多寿;地势低的地方,因为热散则元气外泄而多夭。地域差异的大小跟这种差别成正比关系,地域差异小寿夭的差别就小,地域差异大寿夭的差别就大。所以治病必须懂得天道和地理,阴阳的交胜,气候的先后,人的寿命长短,生化的时期,然后才可以了解人的形体和气机啊。\\
黄帝说:说得好。一岁之中,当病而不病,脏气当相应而不相应,当发用而不发用,是什么道理呢?\\
岐伯道:这是由于司天之气制约着,人体五脏之气顺从天气的缘故。\\
黄帝说:希望详细听听。\\
岐伯说:少阳相火司天之年,火气下临,弥漫于大地,人身肺气上从天气,燥金之气起而用事,草木受灾,炎热如烧灼,金气因之变革,而且消耗,火气太过,暑热流行,病变有咳嗽,喷嚏,鼻涕,衄血,鼻塞,口疮,寒热,浮肿。厥阴在泉,风气流行于地,尘土飞扬。病变为心痛,胃脘痛,厥逆,胸膈不通,病变急暴快速。\\
阳明司天之年,燥气下临于地,人身肝气上从天气,风木之气起而用事,土气必受灾害,凄沧清冷之气常常到来,草木被克伐枯萎。人体发病为胁痛,目赤,眩晕,动摇,战栗,筋脉痿弱,不能久立。阳明司天则少阴君火在泉,故突然热至,地上暑热蒸腾,阳气郁结于内而发病,尿色改变,寒热往来如疟疾,重则发生心痛。火气流行于草木枯槁之时,流水不能结冰,蛰虫反而外见不藏。\\
太阳司天之年,寒水之气下临于地,人身心气上从天气。火气显明,火热之气起而用事,金气必然受伤,寒冷之气时时出现,寒气太过则水结成冰,由于火气被迫而上从天气,所以发病为心热烦闷,咽喉干燥,常口渴,鼻涕,喷嚏,易于悲哀,时常打呵欠。热气妄行于上,所以寒气报复于下,寒霜不时下降,水气凌心,则神气伤,发病为善忘,重的至心痛。太阳司天则太阴湿土在泉,土能制水,所以土气滋润,水流丰沛,寒水之客气加临,火为沉阴所化,万物因寒湿而发生变化,在人体的病变则为水饮内停,腹满,不能饮食,皮肤麻痹,肌肉不仁,筋脉不利,甚至浮肿,背部生痈。\\
厥阴司天之年,风木之气下临于大地,人身脾气上从天气,土气隆盛,湿土之气起而用事,水气受损,土从木化,受其克制,其功用发生变易。人体发病为身体沉重,肌肉枯萎,饮食减少,口爽无味。风气运行宇宙之间,云气与草木为之动摇,在人体之病变为目眩,耳鸣。厥阴司天,少阳相火在泉,火气横行,大地一片暑热,在人体则见大热,消烁阴液,多见赤痢赤带。本该蛰居的虫类不藏而常见于外,流水不能成冰,其所发之病急速。\\
少阴君火司天之年,火热之气下临于大地,人身肺气上从天气,燥金之气起而用事,则草木受灾。发病多为气喘,呕吐,寒热,喷嚏,鼻涕,衄血,鼻塞不通。暑热流行,甚至病发疮疡,高热,暑热酷热,大有金熔石化之势。少阴司天则阳明燥气在泉,清燥之气行地,则寒凉之气常至,在病变多为胁痛,好出长气,肃杀之气行令,草木性质就要变化了。\\
太阴司天之年,湿气下临于大地,人身肾气上从天气,寒水之气起而用事,火气受损,寒水畏湿土而从土化,土气上冒而为云雨。人体发病为胸中不通利,阴痿,阳气大衰,阳不能举而失去作用。在土旺之时则感腰臀部疼痛,转动不便,厥逆。太阴司天,太阳寒水在泉,故地气阴凝闭藏,大寒将至,蛰虫很早就蛰伏,发病多为心下痞塞疼痛。若寒气太过则土地冻裂,冰冻坚硬,病发为小腹痛,常常影响进食。水气上乘金气,水得金生,寒凝更著,所以井水增加,水味变咸,这是由于河中流水减少的缘故。\\
黄帝问:每年有的虫类能够胎孕繁殖,有的不能生育,这生化的不同情况,究竟是什么气导致的呢?\\
岐伯说:六气和五行所化的五种虫类,是相胜相克的。若六气与运气相同,则生物就会繁盛;若六气与运气不相同,则生物就会减衰。这是天地孕育的道理,生化的自然规律。所以厥阴司天的时候,毛虫不受影响而安静,羽虫可以生育,介虫不能生成;若厥阴在泉,毛虫可以生育,倮虫遭到损耗,羽虫也就不育。少阴司天的时候,羽虫不受影响而安静,介虫可以生育,毛虫不能生成;若少阴在泉,倮虫可以生育,介虫遭到耗损,不得生育。太阴司天的时候,倮虫不受影响而安静,鳞虫可以生育,羽虫不能生成;太阴在泉,倮虫可以生育,鳞虫虽育而不能生成。少阳司天的时候,羽虫不受影响而安静,毛虫可以生育,倮虫不能生成;少阳在泉,羽虫可以生育,介虫遭到耗损,毛虫不能生育。阳明司天的时候,介虫不受影响而安静,羽虫可以生育,介虫不能生成;阳明在泉,介虫可以生育,毛虫遭到耗损,羽虫不能生成。太阳司天的时候,鳞虫不受影响而安静,倮虫可以生育;太阳在泉,鳞虫遭到耗损,倮虫不能生育。凡是遭到克制而不能成长的气运,就更严重了。所以六气所主各有所胜,而岁运所立,各有其生化的作用。在泉之气,制约己所胜者;司天之气,制约胜己者;司天之气制色,在泉之气制形。五种虫类的繁衍和衰微,都是适应着六气而产生的,所以有胎孕和不育的分别,这不是治化的不全,而是运气的一种正常现象,因此叫做“中根”。中根以外的六气,也是根据五行而施化,所以生化之气不齐,而有臊、焦、香、腥、腐五气,酸、苦、辛、咸、甘五味,青、黄、赤、白、黑五色,毛、羽、倮、鳞、介五类分别,它们在万物之中各得其所宜。\\
黄帝问:这是什么道理呢?\\
岐伯说:生物的生命,其根源藏于内的,叫做神机,如果神离去了,则生化的机能也就停止。凡生命根源于外的,叫做气立,假如在外的六气歇止,那么生化也就随之断绝了。所以说运各有制约,各有相胜,各有所生,各有所成。所以说:设若不知道岁运和六气的加临,以及六气的同异,就不能晓得生化。就是这个道理。\\
黄帝问:气形成就能生化,气分散就能造就物体的形质,气敷布就可繁殖,气终了万物之象便发生变化,一切物质都是如此。然而五味所禀受之气,在生化上有厚有薄,在成熟上有少有多,其结果与开始也不同,这是什么缘故呢?\\
岐伯说:这是由于在泉之气所控制,所以生化上有厚薄多少的差异,所以其生化,非天气则不生,非地气则不长。\\
黄帝说:希望听听其中的道理。\\
岐伯说:寒、热、燥、湿的气化各不相同。所以少阳相火在泉的时候,寒毒之物不能生长,金从火化,所以味辛,其主治之味是苦、酸,谷类颜色是苍色和丹色。阳明燥金在泉的时候,湿毒之物不能生长,木从金化,所以味酸,其气温,其主治之味是辛、苦、甘,谷类颜色是丹色和素色。太阳寒水在泉的时候,热毒之物不能生长,火从水化,所以味苦,其主治之味是淡、咸,谷类颜色是黄色和黑色。厥阴风木在泉的时候,清毒之物不能生长,土从木化,所以味甘,其主治之味是酸、苦,谷类颜色是青色和红色。厥阴司天则少阳在泉,木火相生,则气化专一,其味纯正。少阴君火在泉的时候,寒毒之物不能生长,金从火化,所以味辛,其主治之味是辛、苦、甘,谷类颜色是白色和红色。太阴湿土在泉的时候,燥毒之物不能生长,水从土化,所以味咸,其气热,其主治之味是甘、咸,谷类颜色是黄色和黑色。太阴在泉,而其气化淳厚,土能制水,所以咸味得以内守,土居土味,而能生金,其气专精,所以辛味也得以生化,能与湿土同治。\\
所以说:因司天、在泉之气不及而引起的疾病应该用补法,补就要顺其气而补;因司天在泉之气太过而引起的疾病应该用逆治法,逆治就要逆其气而治,都要根据病情的寒热盛衰来调治。所以说:无论用上取、下取、内取、外取之法,总要先找着病气的所在,再治疗。身体强壮能耐受毒药的就给以性味厚的药,身体柔弱而不能耐受毒药的,就给以性味薄的药,说的就是这个道理。若病气反其常候,病在上而治其下;病在下而治其上;病在中而治其左右。治热证用寒药,应该温服;治寒证用热药,应该凉服;治温证用凉药,应该冷服;治清证用温药,应该热服。病人虚实不同,制方就不同,所以或用消法,或用削法,或用吐法,或用下法,或用补法,或用泻法,无论久病新病,都得遵从这一点。\\
黄帝问:若病在里面,不实也不坚硬,有时聚而有形,有时散而无形,这种病怎样治疗呢?\\
岐伯说:你问得真详尽啊!这种病如果没有积滞的话,就从内脏里寻求病因,如虚就用补法,用药以祛邪,随用饮食加以滋养,用热汤以浴渍肌表,使其内外调和,这样可以使病完全治愈。\\
黄帝问:有毒的药和无毒的药,服法也有什么规定吗?\\
岐伯说:病有新久,处方有大小,药物有毒无毒,固然有它的常规。凡用大毒之药,病去十分之六,不可再服;用平常的毒药,病去十分之七,不可再服;用小毒之药,病去十分之八,不可再服;用无毒的药,病去十分之九,也不必再服。以后用谷肉果菜,饮食调养,就可使病气都去掉了,但不可吃得过多而损伤了正气。如果邪气未尽,还可再按上法服药,一定得先知道岁气的偏胜,千万不能攻伐天真的冲和之气。不要使实者更实,不要使虚者更虚,而给患者留下后患。总之,一方面要注意不能使邪气更盛,另一方面要注意不能使正气丧失,以免断送人的生命。\\
黄帝问:那久病的人,有时气顺,而身体并不健康;病虽去了,而身体仍然瘦弱,又怎么办呢?\\
岐伯说:你问得真高明啊!天地万物的生化,人是不能代替的,四时的气序,人是不可违反的。因此只有顺应天地四时的气化,使经络畅通,气血和顺,慢慢来恢复它的不足,使与正常人一样,或补养,或调和,要静待时机,谨慎地守护正气,不要使它耗损,这样,病人的形体才会强壮,生气也会一天一天地增长起来,这才是圣王之道。所以《大要》说:不要以人力来代替天地的气化,不要违反四时的运行,必须静养,必须安和,等待正气的恢复。说的就是这个意思。\\
黄帝说:说得好。\\
卷二十一\\
六元正纪大论篇第七十一\\
黄帝问曰:六化六变,胜复淫治,甘苦辛咸,酸淡先后,余知之矣。夫五运之化,或从天气,或逆天气;或从天气,而逆地气;或从地气,而逆天气;或相得,或不相得,余未能明其事。欲通天之纪,从地之理,和其运,调其化,使上下合德,无相夺伦,天地升降,不失其宜,五运宣行,勿乖其政,调之正味,从逆奈何?\\
岐伯稽首再拜,对曰:昭乎哉问也!此天地之纲纪,变化之渊源,非圣帝孰能穷其至理欤!臣虽不敏,请陈其道,令终不灭,久而不易。\\
帝曰:愿夫子推而次之,从其类序,分其部主,别其宗司,昭其气数,明其正化,可得闻乎?\\
岐伯曰:先立其年,以明其气,金木水火土,运行之数,寒暑燥湿风火,临御之化,则天道可见,民气可调,阴阳卷舒,近而无惑。数之可数者,请遂言之。\\
帝曰:太阳之政,奈何?\\
岐伯曰:辰戌之纪也。\\
太阳 太角 太阴 壬辰 壬戌  其运风,其化鸣紊启拆,其变振拉摧拔,其病眩掉目瞑。\\
太角(初正) 少徵 太宫 少商 太羽(终)\\
太阳 太徵 太阴 戊辰 戊戌 同正徵。其运热,其化暄暑郁燠,其变炎烈沸腾,其病热郁。\\
太徵 少宫 太商 少羽(终) 少角(初)\\
太阳 太宫 太阴 甲辰(岁会同天符) 甲戌(岁会同天符)  其运阴埃,其化柔润重泽,其变震惊飘骤,其病湿下重。\\
太宫 少商 太羽(终) 太角(初) 少徵\\
太阳 太商 太阴 庚辰 庚戌。其运凉,其化雾露萧仚,其变肃杀凋零,其病燥、背瞀、胸满。\\
太商 少羽(终) 少角(初) 太徵 少宫\\
太阳 太羽 太阴 丙辰(天符) 丙戌(天符)。其运寒,其化凝惨凛冽,其变冰雪霜雹,其病大寒留于谿谷。\\
太羽(终) 太角(初) 少徵 太宫 少商\\
凡此太阳司天之政,气化运行先天。天气肃,地气静,寒临太虚,阳气不令,水土合德,上应辰星、镇星。其谷玄黅,其政肃,其令徐。寒政大举,泽无阳焰,则火发待时。少阳中治,时雨乃涯,止极雨散,还于太阴,云朝北极,湿化乃布,泽流万物。寒敷于上,雷动于下,寒湿之气,持于气交。民病寒湿,发肌肉萎,足痿不收,濡泻血溢。\\
初之气,地气迁,气乃大温,草乃早荣,民乃厉,温病乃作,身热头痛呕吐,肌腠疮疡。\\
二之气,大凉反至,民乃惨,草乃遇寒,火气遂抑,民病气郁中滞,寒乃始。\\
三之气,天政布,寒气行,雨乃降。民病寒、反热中,痈疽注下,心热瞀闷,不治者,死。\\
四之气,风湿交争,风化为雨,乃长乃化乃成。民病大热少气,肌肉萎,足痿,注下赤白。\\
五之气,阳复化,草乃长,乃化乃成,民乃舒。\\
终之气,地气正,湿令行,阴凝太虚,埃昏郊野,民乃惨凄,寒风以至,反者孕乃死。\\
故岁宜苦以燥之温之,必折其郁气,先资其化源,抑其运气,扶其不胜,无使暴过,而生其疾。食岁谷,以全其真;避虚邪,以安其正。适气同异,多少制之。同寒湿者燥热化,异寒湿者燥湿化。故同者多之,异者少之。用寒远寒,用凉远凉,用温远温,用热远热。食宜同法。有假者反常,反是者病。所谓时也。\\
帝曰:善。阳明之政,奈何?\\
岐伯曰:卯酉之纪也。\\
阳明 少角 少阴 清热胜复同,同正商。丁卯(岁会) 丁酉,其运风清热。\\
少角(初正) 太徵 少宫 太商 少羽(终)\\
阳明 少徵 少阴 寒雨胜复同,同正商。癸卯(同岁会) 癸酉(同岁会) 其运热寒雨。\\
少徵 太宫 少商 太羽(终) 太角(初)\\
阳明 少宫 少阴 风凉胜复同。己卯 己酉 其运雨风凉\\
少宫 太商 少羽(终) 少角(初) 太徵\\
阳明 少商 少阴 热寒胜复同,同正商。乙卯(天符) 乙酉(岁会,太一天符)。其运凉热寒。\\
少商 太羽(终) 太角(初) 少徵 太宫\\
阳明 少羽 少阴 雨风胜复同,同少宫。辛酉辛卯 其运寒雨风。\\
少羽(终) 少角(初) 太徵 太宫 太商\\
凡此阳明司天之政,气化运行后天,天气急,地气明,阳专其令,炎暑大行。物燥以坚,淳风乃治。风燥横运,流于气交。多阳少阴,云趋雨府,湿化乃敷,燥极而泽。其谷白丹,间谷命太者,其耗白甲品羽,金火合德,上应太白荧惑。其政切,其令暴。蛰虫乃见,流水不冰。民病咳嗌塞,寒热发,暴振慄癃夗。清先而劲,毛虫乃死;热后而暴,介虫乃殃。其发躁,胜复之作,扰而大乱,清热之气,持于气交。\\
初之气,地气迁,阴始凝,气始肃,水乃冰,寒雨化。其病中热胀,面目浮肿,善眠,鼽衄,嚏欠,呕,小便黄赤,甚则淋。\\
二之气,阳乃布,民乃舒,物乃生荣。厉大至,民善暴死。\\
三之气,天政布,凉乃行,燥热交合,燥极而泽,民病寒热。\\
四之气,寒雨降,病暴仆,振栗谵妄,少气,嗌干引饮,及为心痛,痈肿疮疡,疟寒之疾,骨痿血便。\\
五之气,春令反行,草乃生荣,民气和。\\
终之气,阳气布,候反温,蛰虫来见,流水不冰,民乃康平,其病温。\\
故食岁谷,以安其气;食间谷,以去其邪。岁宜以咸以苦以辛,汗之、清之、散之,安其运气,无使受邪;折其郁气,资其化源。以寒热轻重,少多其制。同热者,多天化;同清者,多地化。用凉远凉,用热远热,用寒远寒,用温远温。食宜同法。有假者反之,此其道也。反是者,乱天地之经,扰阴阳之纪也。\\
帝曰:善。少阳之政,奈何?\\
岐伯曰:寅申之纪也。\\
少阳 太角 厥阴 壬寅(同天符) 壬申(同天符) 其运风鼓,其化鸣紊启坼,其变振拉摧拔,其病掉眩、支胁、惊骇。\\
太角(初正) 少徵 太宫 少商 太羽(终)\\
少阳 太徵 厥阴 戊寅(天符) 戊申(天符)。其运暑,其化暄嚣郁燠,其变炎烈沸腾,其病上热郁,血溢、血泄、心痛。\\
太徵 少宫 太商 少羽(终) 少角(初)\\
少阳 太宫 厥阴 甲寅 甲申 其运阴雨,其化柔润重泽,其变震惊飘骤,其病体重、胕肿、痞饮。\\
太宫 少商 太羽(终) 太角(初) 少徵\\
少阳 太商 厥阴 庚寅 庚申 同正商 其运凉,其化雾露清切,其变肃杀凋零,其病肩背胸中。\\
太商 少羽(终) 少角(初) 太徵 少宫\\
少阳 太羽 厥阴 丙寅 丙申 其运寒肃,其化凝惨凛冽,其变冰雪霜雹,其病寒浮肿。\\
太羽(终) 太角(初) 少徵 太宫 少商\\
凡此少阳司天之政,气化运行先天。天气正,地气扰,风乃暴举,木偃沙飞。炎火乃流,阴行阳化,雨乃时应,火木同德,上应荧惑岁星。其谷丹苍,其政严,其令扰。故风热参布,云物沸腾,太阴横流,寒乃时至,凉雨并起。民病寒中,外发疮疡,内为泄满。故圣人遇之,和而不争。往复之作,民病寒热疟泄,聋瞑呕吐,上怫肿色变。\\
初之气,地气迁,风胜乃摇,寒乃去,候乃大温,草木早荣。寒来不杀,温病乃起。其病气怫于上,血溢目赤,咳逆头痛,血崩胁满,肤腠中疮。\\
二之气,火反郁,白埃四起,云趋雨府,风不胜湿,雨乃零,民乃康。其病热郁于上,咳逆呕吐,疮发于中,胸嗌不利,头痛身热,昏愦脓疮。\\
三之气,天政布,炎暑至,少阳临上,雨乃涯。民病热中,聋瞑血溢,脓疮咳呕,鼽衄渴嚏欠,喉痹目赤,善暴死。\\
四之气,凉乃至,炎暑间化,白露降,民气和平,其病满身重。\\
五之气,阳乃去,寒乃来,雨乃降,气门乃闭,刚木早凋,民避寒邪,君子周密。\\
终之气,地气正,风乃至,万物反生,霿雾以行。其病关闭不禁,心痛,阳气不藏而咳。\\
抑其运气,赞所不胜,必折其郁气,先取化源,暴过不生,苛疾不起。故岁宜咸宜辛宜酸,渗之泄之,渍之发之,观气寒温,以调其过。同风热者,多寒化;异风热者,少寒化。用热远热,用温远温,用寒远寒,用凉远凉。食宜同法,此其道也。有假者反之,反是者,病之阶也。\\
帝曰:善。太阴之政,奈何?\\
岐伯曰:丑未之纪也。\\
太阴 少角 太阳 清热胜复同,同正宫。丁丑 丁未 其运风清热。\\
少角(初正) 太徵 少宫 太商 少羽(终)\\
太阴 少徵 太阳 寒雨胜复同 癸丑 癸未 其运热寒雨。\\
少徵 太宫 少商 太羽(终) 太角(初)\\
太阴 少宫 太阳 风清胜复同,同正宫。己丑(太一天符) 己未(太一天符) 其运雨风清。\\
少宫 太商 少羽(终) 少角(初) 太徵\\
太阴 少商 太阳 热寒胜复同。乙丑 乙未 其运凉热寒。\\
少商 太羽(终) 太角(初) 少徵 太宫\\
太阴 少羽 太阳 雨风胜复同,同正宫。辛丑(同岁会) 辛未(同岁会) 其运寒雨风。\\
少羽(终) 少角(初) 太徵 少宫 太商\\
凡此太阴司天之政,气化运行后天,阴专其政,阳气退避,大风时起,天气下降。地气上腾,原野昏霿,白埃四起,云奔南极,寒雨数至,物成于差夏。民病寒湿,腹满,身尒愤,胕肿痞逆,寒厥拘急。湿寒合德,黄黑埃昏,流行气交,上应镇星辰星。其政肃,其令寂,其谷黅玄。故阴凝于上,寒积于下。寒水胜火,则为冰雹。阳光不治,杀气乃行。故有余宜高,不及宜下;有余宜晚,不及宜早。土之利,气之化也,民气亦从之,间谷命其太也。\\
初之气,地气迁,寒乃去,春气正,风乃来。生布万物以荣,民气条舒,风湿相薄,雨乃后。民病血溢,筋络拘强,关节不利,身重筋痿。\\
二之气,大火正,物承化,民乃和。其病温厉大行,远近咸若。湿蒸相薄,雨乃时降。\\
三之气,天政布,湿气降,地气腾,雨乃时降,寒乃随之。感于寒湿,则民病身重胕肿,胸腹满。\\
四之气,畏火临,溽蒸化,地气腾,天气否隔,寒风晓暮,蒸热相薄,草木凝烟,湿化不流,则白露阴布,以成秋令。民病腠理热,血暴溢疟,心腹满热,胪胀,甚则胕肿。\\
五之气,惨令已行,寒露下,霜乃早降,草木黄落,寒气及体,君子周密,民病皮腠。\\
终之气,寒大举,湿大化,霜乃积,阴乃凝,水坚冰,阳光不治。感于寒,则病人关节禁固,腰脽痛,寒湿持于气交而为疾也。\\
必折其郁气,而取化源。益其岁气,无使邪胜,食岁谷,以全其真;食间谷,以保其精。故岁宜以苦燥之温之,甚者发之泄之。不发不泄,则湿气外溢,肉溃皮拆而水血交流。必赞其阳火,令御甚寒,从气异同,少多其判也。同寒者,以热化;同湿者,以燥化。异者少之,同者多之。用凉远凉,用寒远寒,用温远温,用热远热。食宜同法。假者反之,此其道也。反是者,病也。\\
帝曰:善。少阴之政,奈何?\\
岐伯曰:子午之纪也。\\
少阴 太角 阳明 壬子 壬午 其运风鼓,其化鸣紊启拆,其变振拉摧拔,其病支满。\\
太角(初正) 少徵 太宫 少商 太羽(终)\\
少阴 太徵 阳明 戊子(天符) 戊午(太一天符) 其运炎暑,其化暄曜郁燠,其变炎烈沸腾,其病上热血溢。\\
太徵 少宫 太商 少羽(终) 少角(初)\\
少阴 太宫 阳明 甲子 甲午 其运阴雨,其化柔润时雨,其变震惊飘骤,其病中满身重。\\
太宫 少商 太羽(终) 太角(初) 少徵\\
少阴 太商 阳明 庚子(同天符) 庚午(同天符) 同正商 其运凉劲,其化雾露萧仚,其变肃杀凋零,其病下清。\\
太商 少羽(终) 少角(初) 太徵 少宫\\
少阴 太羽 阳明 丙子(岁会) 丙午 其运寒,其化凝惨凛冽,其变冰雪霜雹,其病寒下。\\
太羽(终) 太角(初) 少徵 太宫 少商\\
凡此少阴司天之政,气化运行先天,地气肃,天气明,寒交暑,热加燥,云驰雨府,湿化乃行,时雨乃降,金火合德,上应荧惑、太白。其政明,其令切,其谷彤白。水火寒热,持于气交,而为病始也。热病生于上,清病生于下,寒热凌犯而争于中。民病咳喘,血溢血泄,鼽嚏,目赤,眦疡,寒厥入胃,心痛,腰痛,腹大,嗌干肿上。\\
初之气,地气迁,暑将去,寒乃始,蛰复藏,水乃冰,霜复降,风乃至,阳气郁。民反周密,关节禁固,腰脽痛,炎暑将起,中外疮疡。\\
二之气,阳气布,风乃行,春气以正,万物应荣,寒气时至,民乃和。其病淋,目瞑目赤,气郁于上而热。\\
三之气,天政布,大火行,庶类蕃鲜,寒气时至。民病气厥心痛,寒热更作,咳喘目赤。\\
四之气,溽暑至,大雨时行,寒热互至。民病寒热,嗌干,黄瘅,鼽衄,饮发。\\
五之气,畏火临,暑反至,阳乃化,万物乃生乃长荣,民乃康,其病温。\\
终之气,燥令行,余火内格,肿于上,咳喘,甚则血溢。寒气数举,则霿雾翳,病生皮腠,内舍于胁,下连少腹,而作寒中,地将易也。\\
必抑其运气,资其岁胜,折其郁发,先取化源。无使暴过,而生其病也。食岁谷,以全真气;食间谷,以辟虚邪。岁宜咸以软之,而调其上,甚则以苦发之,以酸收之;而安其下,甚则以苦泄之。适气同异,而多少之。同天气者,以寒清化;同地气者,以温热化。用热远热,用凉远凉,用温远温,用寒远寒。食宜同法。有假则反,此其道也。反是者,病作矣。\\
帝曰:善。厥阴之政,奈何?\\
岐伯曰:巳亥之纪也。\\
厥阴 少角 少阳 清热胜复同,同正角。丁巳(天符)丁亥(天符) 其运风清热。\\
少角(初正) 太徵 少宫 太商 少羽(终)\\
厥阴 少徵 少阳 寒雨胜复同。癸巳(同岁会) 癸亥(同岁会) 其运热寒雨。\\
少徵 太宫 少商 太羽(终) 太角(初)\\
厥阴 少宫 少阳 风清胜复同,同正角。己巳 己亥 其运雨风清。\\
少宫 太商 少羽(终) 少角(初) 太徵\\
厥阴 少商 少阳 热寒胜复同,同正角。乙巳 乙亥 其运凉热寒。\\
少商 太羽(终) 太角(初) 少徵 太宫\\
厥阴 少羽 少阳 雨风胜复同。辛巳 辛亥 其运寒雨风。\\
少羽(终) 少角(初) 太徵 少宫 太商\\
凡此厥阴司天之政,气化运行后天。诸同正岁,气化运行同天。天气扰,地气正。风生高远,炎热从之。云趋雨府,湿化乃行。风火同德,上应岁星荧惑。其政挠,其令速,其谷苍丹,间谷言太者,其耗文角品羽,风燥火热,胜复更作,蛰虫来见,流水不冰。热病行于下,风病行于上,风燥胜复形于中。\\
初之气,寒始肃,杀气方至。民病寒于右之下。\\
二之气,寒不去,华雪水冰,杀气施化,霜乃降,名草上焦,寒雨数至,阳复化。民病热中。\\
三之气,天政布,风乃时举。民病泣出,耳鸣掉眩。\\
四之气,溽暑湿热相薄,争于左之上,民病黄瘅,而为胕肿。\\
五之气,燥湿更胜,沉阴乃布,寒气及体,风雨乃行。\\
终之气,畏火司令,阳乃大化,蛰虫出见,流水不冰,地气大发,草乃生,人乃舒,其病温厉。\\
必折其郁气,资其化源,赞其运气,无使邪胜。岁宜以辛调上,以咸调下。畏火之气,无妄犯之。用温远温,用热远热,用凉远凉,用寒远寒。食宜同法。有假反常,此之道也。反是者病。\\
帝曰:善。夫子言可谓悉矣,然何以明其应乎?\\
岐伯曰:昭乎哉问也!夫六气者,行有次,止有位。故常以正月朔日,平旦视之。睹其位,而知其所在矣。运有余,其至先;运不及,其至后。此天之道,气之常也。运非有余非不足,是谓正岁,其至当其时也。\\
帝曰:胜复之气,其常在也。灾眚时至,候也奈何?\\
岐伯曰:非气化者,是谓灾也。\\
帝曰:天地之数,终始奈何?\\
岐伯曰:悉乎哉问也!是明道也。数之始,起于上而终于下;岁半之前,天气主之;岁半之后,地气主之;上下交互,气交主之。岁纪毕矣。故曰:位明,气月可知乎,所谓气也。\\
帝曰:余司其事,则而行之,不合其数,何也?\\
岐伯曰:气用有多少,化洽有盛衰。衰盛多少,同其化也。\\
帝曰:愿闻同化,何如?\\
岐伯曰:风温,春化同;热曛昏火,夏化同;胜与复同,燥清烟露,秋化同;云雨昏瞑埃,长夏化同;寒气霜雪冰,冬化同。此天地五运六气之化,更用盛衰之常也。\\
帝曰:五运行同天化者,命曰天符,余知之矣。愿闻同地化者,何谓也?\\
岐伯曰:太过而同天化者,三;不及而同天化者,亦三;太过而同地化者,三;不及而同地化者,亦三。此凡二十四岁也。\\
帝曰:愿闻其所谓也。\\
岐伯曰:甲辰、甲戌、太宫下加太阴,壬寅、壬申、太角下加厥阴,庚子、庚午、太商下加阳明,如是者三。癸巳、癸亥,少徵下加少阳,辛丑、辛未、少羽下加太阳,癸卯、癸酉、少徵下加少阴,如是者三。戊子、戊午、太徵上临少阴,戊寅、戊申、太徵上临少阳,丙辰、丙戌、太羽上临太阳,如是者三。丁巳、丁亥、少角上临厥阴,乙卯、乙酉、少商上临阳明,己丑、己未、少宫上临太阴,如是者三。除此二十四岁,则不加不临也。\\
帝曰:加者何谓?\\
岐伯曰:太过而加同天符,不及而加同岁会也。\\
帝曰:临者何谓?\\
岐伯曰:太过不及,皆曰天符,而变行有多少,病形有微甚,生死有早晏耳。\\
帝曰:夫子言用寒远寒,用热远热。余未知其然也,愿闻何谓远?\\
岐伯曰:热无犯热,寒无犯寒。从者和,逆者病,不可不敬畏而远之,所谓时与六位也。\\
帝曰:温凉何如?\\
岐伯曰:司气以热,用热无犯;司气以寒,用寒无犯;司气以凉,用凉无犯;司气以温,用温无犯。间气同其主无犯,异其主则小犯之。是谓四畏,必谨察之。\\
帝曰:善!其犯者何如?\\
岐伯曰:天气反时,则可依时;及胜其主,则可犯。以平为期,而不可过,是谓邪气反胜者。故曰:无失天信,无逆气宜,无翼其胜,无赞其复。是谓至治。\\
帝曰:善。五运气行主岁之纪,其有常数乎?\\
岐伯曰:臣请次之。\\
甲子 甲午岁\\
上少阴火,中太宫土运,下阳明金 热化二,雨化五,燥化四,所谓正化日也。其化上咸寒,中苦热,下酸热,所谓药食宜也。\\
乙丑 乙未岁\\
上太阴土,中少商金运,下太阳水 热化寒化胜复同,所谓邪气化日也。灾七宫。湿化五,清化四,寒化六,所谓正化日也。其化上苦热,中酸和,下甘热,所谓药食宜也。\\
丙寅 丙申岁\\
上少阳相火,中太羽水运,下厥阴木,火化二,寒化六,风化三,所谓正化日也。其化上咸寒,中咸温,下辛温,所谓药食宜也。\\
丁卯(岁会) 丁酉岁\\
上阳明金,中少角木运,下少阴火,清化热化胜复同,所谓邪气化日也。灾三宫。燥化九,风化三,热化七,所谓正化日也。其化上苦小温,中辛和,下咸寒,所谓药食宜也。\\
戊辰 戊戌岁\\
上太阳水,中太徵火运,下太阴土。寒化六,热化七,湿化五,所谓正化日也。其化上苦温,中甘和,下甘温,所谓药食宜也。\\
己巳 己亥岁\\
上厥阴木,中少宫土运,下少阳相火。风化清化胜复同,所谓邪气化日也。灾五宫。风化三,湿化五,火化七,所谓正化日也。其化上辛凉,中甘和,下咸寒,所谓药食宜也。\\
庚午(同天符) 庚子岁(同天符)\\
上少阴火,中太商金运,下阳明金。热化七,清化九,燥化九,所谓正化日也。其化上咸寒,中辛温,下酸温,所谓药食宜也。\\
辛未(同岁会) 辛丑岁(同岁会)\\
上太阴土,中少羽水运,下太阳水。雨化风化胜复同,所谓邪气化日也。灾一宫。雨化五,寒化一,所谓正化日也。其化上苦热,中苦和,下苦热,所谓药食宜也。\\
壬申(同天符) 壬寅岁(同天符)\\
上少阳相火,中太角木运,下厥阴木。火化二,风化八,所谓正化日也。其化上咸寒,中酸和,下辛凉,所谓药食宜也。\\
癸酉(同岁会) 癸卯岁(同岁会)\\
上阳明金,中少徵火运,下少阴火。寒化雨化胜复同,所谓邪气化日也。灾九宫。燥化九,热化二,所谓正化日也。其化上苦小温,中咸温,下咸寒,所谓药食宜也。\\
甲戌(岁会同天符) 甲辰岁(岁会同天符)\\
上太阳水,中太宫土运,下太阴土。寒化六,湿化五,正化日也。其化上苦热,中苦温,下苦温,药食宜也。\\
乙亥 乙巳岁\\
上厥阴木,中少商金运,下少阳相火。热化寒化胜复同,邪气化日也。灾七宫。风化八,清化四,火化二,正化度也。其化上辛凉,中酸和,下咸寒,药食宜也。\\
丙子(岁会) 丙午岁\\
上少阴火,中太羽水运,下阳明金。热化二,寒化六,清化四,正化度也。其化上咸寒,中咸热,下酸温,药食宜也。\\
丁丑 丁未岁\\
上太阴土,中少角木运,下太阳水。清化热化胜复同,邪气化度也。灾三宫。雨化五,风化三,寒化一,正化度也。其化上苦温,中辛温,下甘热,药食宜也。\\
戊寅 戊申岁(天符)\\
上少阳相火,中太徵火运,下厥阴木。火化七,风化三,正化度也。其化上咸寒,中甘和,下辛凉,药食宜也。\\
己卯(太一天符) 己酉岁(天符)\\
上阳明金,中少宫土运,下少阴火,风化清化胜复同,邪气化度也。灾五宫。清化九,雨化五,热化七,正化度也。其化上苦小温,中甘和,下咸寒,药食宜也。\\
庚辰 庚戌岁\\
上太阳水,中太商金运,下太阴土。寒化一,清化九,雨化五,正化度也。其化上苦热,中辛温,下甘热,药食宜也。\\
辛巳 辛亥岁\\
上厥阴木,中少羽水运,下少阳相火。雨化风化胜复同,邪气化度也。灾一宫。风化三,寒化一,火化七,正化度也。其化上辛凉,中苦和,下咸寒,药食宜也。\\
壬午 壬子岁\\
上少阴火,中太角木运,下阳明金。热化二,风化八,清化四,正化度也。其化上咸寒,中酸凉,下酸温,药食宜也。\\
癸未 癸丑岁\\
上太阴土,中少徵火运,下太阳水。寒化雨化胜复同,邪气化度也。灾九宫。雨化五,火化二,寒化一,正化度也。其化上苦温,中咸温,下甘热,药食宜也。\\
甲申 甲寅岁\\
上少阳相火,中太宫土运,下厥阴木,火化二,雨化五,风化八,正化度也。其化上咸寒,中咸和,下辛凉,药食宜也。\\
乙酉(太一天符) 乙卯岁(天符)\\
上阳明金,中少商金运,下少阴火,热化寒化胜复同,邪气化度也。灾七宫。燥化四,清化四,热化二,正化度也。其化上苦小温,中苦和,下咸寒,药食宜也。\\
丙戌(天符) 丙辰岁(天符)\\
上太阳水,中太羽水运,下太阴土,寒化六,雨化五,正化度也。其化上苦热,中咸温,下甘热,药食宜也。\\
丁亥(天符) 丁巳岁(天符)\\
上厥阴木,中少角木运,下少阳相火,清化热化胜复同,邪气化度也。灾三宫。风化三,火化七,正化度也。其化上辛凉,中辛和,下咸寒,药食宜也。\\
戊子(天符) 戊午岁(太一天符)\\
上少阴火,中太徵火运,下阳明金,热化七,清化九,正化度也。其化上咸寒,中甘寒,下酸温,药食宜也。\\
己丑(太一天符) 己未岁(太一天符)\\
上太阴土,中少宫土运,下太阳水,风化清化胜复同,邪气化度也。灾五宫。雨化五,寒化一,正化度也。其化上苦热,中甘和,下甘热,药食宜也。\\
庚寅 庚申岁\\
上少阳相火,中太商金运,下厥阴木。火化七,清化九,风化三,正化度也。其化上咸寒,中辛温,下辛凉,药食宜也。\\
辛卯 辛酉岁\\
上阳明金,中少羽水运,下少阴火。雨化风化胜复同,邪气化度也。灾一宫。清化九,寒化一,热化七,正化度也。其化上苦小温,中苦和,下咸寒,药食宜也。\\
壬辰 壬戌岁\\
上太阳水,中太角木运,下太阴土,寒化六,风化八,雨化五,正化度也。其化上苦温,中酸和,下甘温,药食宜也。\\
癸巳(同岁会) 癸亥岁(同岁会)\\
上厥阴木,中少徵火运,下少阳相火,寒化雨化胜复同,邪气化度也。灾九宫。风化八,火化二,正化度也。其化上辛凉,中咸和,下咸寒,药食宜也。\\
凡此定期之纪,胜复正化,皆有常数,不可不察。故知其要者,一言而终,不知其要,流散无穷。此之谓也。\\
帝曰:善。五运之气,亦复岁乎?\\
岐伯曰:郁极乃发,待时而作也。\\
帝曰:请问其所谓也?\\
岐伯曰:五常之气,太过不及,其发异也。\\
帝曰:愿卒闻之。\\
岐伯曰:太过者暴,不及者徐。暴者为病甚,徐者为病持。\\
帝曰:太过不及,其数何如?\\
岐伯曰:太过者其数成,不及者其数生,土常以生也。\\
帝曰:其发也何如?\\
岐伯曰:土郁之发,岩谷震惊,雷殷气交,埃昏黄黑,化为白气,飘骤高深,击石飞空,洪水乃从,川流漫衍,田牧土驹。化气乃敷,善为时雨,始生始长,始化始成。故民病心腹胀,肠鸣而为数后,甚则心痛胁尒,呕吐、霍乱,饮发注下,胕肿身重。云奔雨府,霞拥朝阳,山泽埃昏,其乃发也。以其四气,云横天山,浮游生灭,怫之先兆也。\\
金郁之发,天洁地明,风清气切,大凉乃举,草树浮烟,燥气以行,霿雾数起,杀气来至,草木苍干,金乃有声。故民病咳逆,心胁满引少腹,善暴病,不可反侧,嗌干,面尘色恶。山泽焦枯,土凝霜卤,怫乃发也。其气五,夜零白露,林莽声凄,怫之兆也。\\
水郁之发,阳气乃辟,阴气暴举,大寒乃至,川泽严凝,寒雰结为霜雪,甚则黄黑昏翳,流行气交,乃为霜杀,水乃见祥。故民病,寒客心痛,腰脽痛,大关节不利,屈伸不便,善厥逆,痞坚腹满。阳光不治,空积沉阴,白埃昏瞑,而乃发也。其气二火前后,太虚深玄,气犹麻散,微见而隐,色黑微黄,怫之先兆也。\\
木郁之发,太虚埃昏,云物以扰,大风乃至,屋发折木,木有变。故民病,胃脘当心而痛,上支两胁,鬲咽不通,食饮不下,甚则耳鸣眩转,目不识人,善暴僵仆。太虚苍埃,天山一色,或气浊色,黄黑郁若,横云不起,雨而乃发也,其气无常。长川草偃,柔叶呈阴,松吟高山,虎啸岩岫,怫之先兆也。\\
火郁之发,太虚肿翳,大明不彰,炎火行,大暑至,山泽燔燎,材木流津,广厦腾烟,土浮霜卤,止水乃减,蔓草焦黄,风行惑言,湿化乃后。故民病,少气,疮疡痈肿,胁腹胸背,面首四肢尒愤,胪胀,疡痱,呕逆,瘛疭骨痛,节乃有动,注下温疟,腹中暴痛,血溢流注,精液乃少,目赤心热,甚则瞀闷懊屴,善暴死。刻终大温,汗濡玄府,其乃发也。其气四,动复则静,阳极反阴,湿令乃化乃成,华发水凝,山川冰雪,焰阳午泽,怫之先兆也。有怫之应而后报也,皆观其极而乃发也。木发无时,水随火也。谨候其时,病可与期,失时反岁,五气不行,生化收藏,政无恒也。\\
帝曰:水发而雹雪,土发而飘骤,木发而毁折,金发而清明,火发而曛昧,何气使然?\\
岐伯曰:气有多少,发有微甚,微者当其气,甚者兼其下,征其下气,而见可知也。\\
帝曰:善。五气之发,不当位者,何也?\\
岐伯曰:命其差。\\
帝曰:差有数乎?\\
岐伯曰:后皆三十度而有奇也。\\
帝曰:气至而先后者,何?\\
岐伯曰:运太过,则其至先;运不及,则其至后。此候之常也。\\
帝曰:当时而至者,何也?\\
岐伯曰:非太过,非不及,则至当时,非是者,眚也。\\
帝曰:善。气有非时而化者,何也?\\
岐伯曰:太过者,当其时;不及者,归其己胜也。\\
帝曰:四时之气,至有早晏,高下左右,其候何如?\\
岐伯曰:行有逆顺,至有迟速。故太过者,化先天;不及者,化后天。\\
帝曰:愿闻其行,何谓也?\\
岐伯曰:春气西行,夏气北行,秋气东行,冬气南行。故春气始于下,秋气始于上,夏气始于中,冬气始于标。春气始于左,秋气始于右,冬气始于后,夏气始于前。此四时正化之常。故至高之地,冬气常在;至下之地,春气常在。必谨察之。\\
帝曰:善。\\
黄帝问曰:五运六气之应见,六化之正,六变之纪,何如?\\
岐伯对曰:夫六气正纪,有化有变,有胜有复,有用有病。不同其候,帝欲何乎?\\
帝曰:愿尽闻之。\\
岐伯曰:请遂言之。\\
夫气之所至也,厥阴所至为和平,少阴所至为暄,太阴所至为埃溽,少阳所至为炎暑,阳明所至为清劲,太阳所至为寒雰。时化之常也。\\
厥阴所至为风府,为璺启;少阴所至为火府,为舒荣;太阴所至为雨府,为员盈;少阳所至为热府,为行出;阳明所至为司杀府,为庚苍;太阳所至为寒府,为归藏。司化之常也。\\
厥阴所至为生,为风摇;少阴所至为荣,为形见;太阴所至为化,为云雨;少阳所至为长,为番鲜;阳明所至为收,为雾露;太阳所至为藏,为周密。气化之常也。\\
厥阴所至为风生,终为肃;少阴所至为热生,中为寒;太阴所至为湿生,终为注雨;少阳所至为火生,终为蒸溽;阳明所至为燥生,终为凉;太阳所至为寒生,中为温。德化之常也。\\
厥阴所至为毛化,少阴所至为羽化,太阴所至为倮化,少阳所至为羽化,阳明所至为介化,太阳所至为鳞化。德化之常也。\\
厥阴所至为生化,少阴所至为荣化,太阴所至为濡化,少阳所至为茂化,阳明所至为坚化,太阳所至为藏化。布政之常也。\\
厥阴所至,为飘怒,大凉;少阴所至,为大暄,寒;太阴所至,为雷霆骤注,烈风;少阳所至,为飘风燔燎,霜凝;阳明所至,为散落,温;太阳所至,为寒雪冰雹,白埃。气变之常也。\\
厥阴所至为挠动,为迎随;少阴所至为高明,焰为曛;太阴所至为沉阴,为白埃,为晦暝;少阳所至为光显,为彤云,为曛;阳明所至为烟埃,为霜,为劲切,为凄鸣;太阳所至为刚固,为坚芒,为立。令行之常也。\\
厥阴所至为里急;少阴所至为疡胗身热;太阴所至为积饮否隔;少阳所至为嚏呕,为疮疡;阳明所至为浮虚;太阳所至为屈伸不利。病之常也。\\
厥阴所至为支痛;少阴所至为惊惑,恶寒,战栗,谵妄;太阴所至为稸满;少阳所至为惊躁,瞀昧,暴病;阳明所至为鼽,尻阴股膝髀腨气足病;太阳所至为腰痛。病之常也。\\
厥阴所至为仩戾;少阴所至为悲妄衄衊;太阴所至为中满,霍乱、吐下;少阳所至为喉痹,耳鸣呕涌;阳明所至为皴揭;太阳所至为寝汗,痉。病之常也。\\
厥阴所至为胁痛呕泄;少阴所至为语笑;太阴所至为重胕肿;少阳所至为暴注,丱瘛,暴死;阳明所至为鼽嚏;太阳所至为流泄,禁止。病之常也。\\
凡此十二变者,报德以德,报化以化,报政以政,报令以令。气高则高,气下则下,气后则后,气前则前,气中则中,气外则外。位之常也。故风胜则动,热胜则肿,燥胜则干,寒胜则浮,湿胜则濡泄,甚则水闭胕肿。随气所在,以言其变耳。\\
帝曰:愿闻其用也。\\
岐伯曰:夫六气之用,各归不胜而为化。故太阴雨化,施于太阳;太阳寒化,施于少阴;少阴热化,施于阳明;阳明燥化,施于厥阴;厥阴风化,施于太阴。各命其所在,以征之也。\\
帝曰:自得其位,何如?\\
岐伯曰:自得其位,常化也。\\
帝曰:愿闻所在也。\\
岐伯曰:命其位,而方月可知也。\\
帝曰:六位之气,盈虚何如?\\
岐伯曰:太少异也。太者之至,徐而常;少者,暴而亡。\\
帝曰:天地之气,盈虚何如?\\
岐伯曰:天气不足,地气随之;地气不足,天气从之;运居其中,而常先也。恶所不胜,归所同和,随运归从,而生其病也。故上胜,则天气降而下;下胜,则地气迁而上。胜多少而差其分。微者小差,甚者大差,甚则位易气交,易则大变生,而病作矣。《大要》曰:甚纪五分,微纪七分,其差可见。此之谓也。\\
帝曰:善。论言热无犯热,寒无犯寒。余欲不远寒,不远热,奈何?\\
岐伯曰:悉乎哉问也!发表不远热,攻里不远寒。\\
帝曰:不发不攻,而犯寒犯热,何如?\\
岐伯曰:寒热内贼,其病益甚。\\
帝曰:愿闻无病者,何如?\\
岐伯曰:无者生之,有者甚之。\\
帝曰:生者何如?\\
岐伯曰:不远热则热至,不远寒则寒至。寒至则坚否腹满,痛急下利之病生矣。热至则身热,吐下霍乱,痈疽疮疡,瞀郁注下,丱瘛肿胀,呕,鼽衄,头痛,骨节变,肉痛,血溢血泄,淋闭之病生矣。\\
帝曰:治之奈何?\\
岐伯曰:时必顺之,犯者治以胜也。\\
黄帝问曰:妇人重身,毒之何如?\\
岐伯曰:有故无殒,亦无殒也。\\
帝曰:愿闻其故,何谓也?\\
岐伯曰:大积大聚,其可犯也,衰其太半而止,过者死。\\
帝曰:善。郁之甚者,治之奈何?\\
岐伯曰:木郁达之,火郁发之,土郁夺之,金郁泄之,水郁折之。然调其气。过者折之,以其畏也,所谓泻之。\\
帝曰:假者何如?\\
岐伯曰:有假其气,则无禁也。所谓主气不足,客气胜也。\\
帝曰:至哉,圣人之道!天地大化,运行之节,临御之纪,阴阳之政,寒暑之令,非夫子孰能通之!请藏之灵兰之室,署曰《六元正纪》。非斋戒不敢示,慎传也。\\
刺法论篇第七十二(亡)\\
本病论篇第七十三(亡)\\
黄帝问:六气的正常生化和异常变化,胜气、复气、邪气、平治的关系,与甘苦辛咸酸淡等味的先后生化道理,我已经知道了。至于五行主岁的运化,有时与司天之气相从,有时与司天之气相逆,有时与司天之气相从而与地气相逆,有时与地气相从而与司天之气相逆,或者互相适应,或者不相适应,我还不明白其中的道理。要通晓司天、在泉之气的规律和原理,调和五运之气化,使之上下功能协作,不互相冲突,天地之气的升降不致相失,五运之气能宣畅运行,不背离其职权,用适宜的五味来调和气化的从与逆,应该怎样?\\
岐伯行礼再拜,回答说:问题真高明啊!这是天地之气生化的纲领,是万物变化的根源,如果不是圣帝,谁能穷究这些高深的道理呢!我虽然不聪敏,请让我讲述它的道理,使它永久不致湮灭,永存而不改变。\\
黄帝说:请夫子进一步推求排列,依从它的分类和次序,分别六气司天在泉及左右间气的部位和所主之气,详细阐明五行气化之数和法则,这些问题可以听听吗?\\
岐伯说:首先要确定年岁的干支,以明确主岁之气,金木水火土五行运行之数,风寒暑湿燥火六气司天在泉加临的气化,如此则自然天道就可以了解,人们的病气就可以调和,阴阳卷舒平衡,由近及远而不致迷惑。这是可以推算的气运之数,请让我尽可能谈谈。\\
黄帝说:太阳司天之年,运气情况如何?\\
岐伯说:是以辰戌来标志的年份。\\
辰戌年是太阳寒水司天,太阴湿土在泉,若逢岁运是木运太过,便是壬辰、壬戌两个年份。其运主风,风运正常则风声和缓,万物萌动,草木繁荣;若风运异常,则狂风大作,摧折树木,其发疾病是两目昏花,眩晕振掉。\\
木运主岁,主运与客运都起于太角,终于太羽。\\
若逢火运太过,便是戊辰、戊戌两个年份。这两年虽火运太过,但正值太阳寒水司天,太过之火运受司天寒水之气的制约,所以其气运相当于火运平气之年。其运主热,如火运正常则气候温和,渐至暑热熏蒸;如火运异常,则火热炎炽,如沸水蒸腾,其发病,多属热郁。\\
岁运为火运太过,客运起于太徵,终于太角,而主运起于少角,终于少羽。\\
若逢土运太过之年,便是甲辰、甲戌两个年份。天干甲己属土,地支辰戌亦属土,故此两年均为岁会。其运主阴雨。土运正常则风调雨顺,万物润泽;如土运异常,则雷声大作,狂风暴雨,其发病多为湿气甚于下部而肢体重坠。\\
岁运为土运太过,客运起于太宫,终于太徵,而主运起于太角,终于太羽。\\
若逢金运太过之年,便是庚辰、庚戌两个年份。岁运是金,其运为凉,金运正常,是雾露降下,秋风萧瑟;异常是肃杀流行,草木凋零,其发病多为燥病,背部烦闷,胸中胀满。\\
岁运为金运太过,客运起于太商,终于太宫,而主运起于少角,终于少羽。\\
若逢水运太过之年,便是丙辰、丙戌两个年份。因司天与中运相同,故均为天符。岁运是水,故其运是寒,水运正常,则水凉凝惨,气候严寒;若水运异常,则为冰雪霜雹,其发病则为大寒之气留滞在肢节谿谷。\\
岁运为水运太过,客运起于太羽,终于太商,而主运起于太角,终于太羽。\\
凡是太阳司天之年,气化运行先于正常天时,天气清肃,地气安静,寒气充满整个宇宙,阳气不能发挥作用,寒水和湿土相配合,在天上相应水星和土星光明。生长的谷物应为黑色和黄色,司天之政严肃,在泉之令徐缓。寒水之政作用扩张,阳气被抑制,所以湖泽之中没有升腾之阳气,火气只有待时而发。到少阳相火主治的时候,应时之雨水下降,到极点则雨水稀少,又回到太阴湿土行令,云层朝向北极,湿土之气布化四方,润泽流注万物,太阳寒水敷布在上,少阴君火振动在下,寒气湿气相持于气交。人们多病寒湿,发生肌肉萎缩,两足痿软无力,不能收摄,大便泄泻和失血。\\
初之气,由于地气迁易,气候非常温暖,所以百草繁荣得很早,人们多受疫病侵袭,温病发作,证见发热,头痛,呕吐,肌肤疮疡。\\
二之气,阳明燥金之气当令,大凉的气候反而到来,人们感到寒冷凄惨,草木遇到寒气,火气遂被抑制,人们多病气郁于内,而胸腹胀满。司天之寒气开始发生作用。\\
三之气,太阳寒水司天之气当令起用,寒气流行,雨水下降。人们多病伤寒而内热,痈疽,下利,心中烦热,神志昏蒙,胸闷。若不及时治疗,就会死亡。\\
四之气,厥阴风木当令,太阴湿土主运,风湿交争,风气转化为雨,万物因而成长,变化而成熟。人们多病高热,气虚,肌肉萎弱,两足痿软,赤白痢疾。\\
五之气,少阴君火行令,阳气又生化,草木因此成长、变化而成熟,人们也舒畅无病。\\
终之气,太阴湿土当令,地气正胜,湿气运行,阴气凝聚于宇宙之中,尘土飞扬,蒙蔽郊野,人们感到凄惨,寒风到来,风木胜湿土,胎孕往往受伤而损落。\\
本年多发湿寒之病,宜用苦燥以去湿,苦温以去寒。要消除太过致郁的胜气,必须资助不胜之气的生化之源,抑制其太过的运气,扶植其不胜的运气,不要使运气太过而导致疾患。食用与岁气相应的谷物以保全真气;避免虚邪侵袭以安宁正气。根据运气的异同,选择药食气味的多少。岁运与六气同是寒湿,则多用燥热之品以化之;岁运与六气的寒湿之气不同的,则酌用燥湿之品以化之。所以气与运相同的应多用相宜的气味,气与运不同的应酌情少用。总之,用寒性药应避免寒冷的天时,用凉性药应避免清凉的天时,用温性药应避免温暖的天时,用热性药应避免炎热的天时,饮食上也是同一法则。若天气反常,邪气反胜,就不必依照上面的常规,否则会引起新的病变,这就叫因时制宜。\\
黄帝说:好。阳明司天之年,运气情况如何?\\
岐伯说:是以卯酉来标志的年份。\\
卯酉年是阳明燥金司天,少阴君火在泉,丁主少角,木运不及,故金的清气胜之。有胜必有复,火气来复,胜气盛,复气也盛,胜气微,复气也微。金气主清,火气主热,胜复程度大致相同。岁木不及,而上临阳明燥金,形成金的平气。岁运木运不及,是丁卯、丁酉两个年份,其运为风气、清气和热气。\\
因木运主岁,所以主运与客运都起于少角,终于少羽。\\
若逢火运不及,便是癸卯、癸酉两个年份。癸卯、癸酉少徵下加少阴,故此二年为同岁会之年。这两年虽火运不及,由于太阳寒水之气和太阴湿土之气胜复相同,所以其气运相当于金运平气之年。其运气主热,胜气为寒,复气为雨。\\
岁运为火运不及,客运起于少徵,终于少角,而主运起于太角,终于太羽。\\
若逢土运不及之年,便是己卯、己酉两个年份。土运不及,风为胜气,凉为复气,胜复程度大致相同。其运气为雨,胜气为风,复气为凉。\\
岁运为土运不及,客运起于少宫,终于少徵,而主运起于少角,终于少羽。\\
若逢金运不及之年,便是乙卯、乙酉两个年份。乙卯年是天符之年,乙酉年是岁会太一天符之年。金运不及,热为胜气,寒为复气,胜复程度大致相同。由于少阴君火之气和太阳寒水之气胜复相同,所以其气运相当于金运平气之年。其运气为凉,胜气为热,复气为寒。\\
岁运为金运不及,客运起于少商,终于少宫,而主运起于太角,终于太羽。\\
若逢水运不及之年,便是辛酉、辛卯两个年份。水运不及,雨为胜气,风为复气,胜复程度大致相同。辛卯年水运不及,土气来侮,其气化约略同于少宫土运不及的年份。其运气为寒,胜气为雨,复气为风。\\
岁运为水运不及,客运起于少羽,终于少商,而主运起于少角,终于少羽。\\
凡是阳明司天之年,气化运行后于正常天时,天气劲急,地气清明,阳气独自主宰时令,在天地间充满着炎热之气。万物干燥而坚硬,和淳之风行使职权。风气与燥气横行于气运,流布于气交之中。阳气多而阴气少,到太阴土气当令之时,土湿之气上蒸,云趋向雨府之时,湿土之气才能敷布,干燥至极点的气候才变为润泽。正气所化的岁谷为红白二色,其间谷为感受太过的间气而成熟的,白色的甲虫和众多的羽虫易被耗损,金火配合发挥作用,相应在上的为金火二星明亮。天气之政急切,地气的发令急暴。蛰虫不再伏藏,水流动不能结冰。人们多病为咳嗽,咽喉肿塞,恶寒发热,突然寒栗战抖,大小便不通。上半年清金之气劲急有力,毛虫死亡;下半年火热之气急暴,甲壳类昆虫类遭受祸殃。金气和火气的发作都是急躁的,在胜复的发作,常常是纷扰而大乱,清气和热气相持于气交。\\
初之气,地气迁移,阴气开始凝集,而天气肃杀,河水结冰,寒雨运化。发病多为内热胀满,面目浮肿,嗜睡,鼻流清涕,鼻血,喷嚏,呵欠,呕吐,小便色黄赤,甚至淋沥不畅。\\
二之气,阳气敷布,人们很舒服,万物生长欣欣向荣。会有温疫流行,人们常常突然死亡。\\
三之气,燥金司天主令,凉气运行,燥气与热气互相交合,干燥至极,就会化为润泽,人们多寒热病。\\
四之气,寒雨下降,病发为突然仆倒,寒冷战抖,神昏谵语,气息低微,咽喉干燥,口渴引饮,以及心痛,痈肿疮疡,寒疟,骨软无力,二便出血。\\
五之气,厥阴风木之气加临主事,秋天反见春天的时令,草又生长繁荣,人们身体调和无病。\\
终之气,阳气四布,气候反见温暖,蛰虫仍然在外活动,河水流动,不能结冰,人们身体安康,只是容易犯温病。\\
所以在这样的年份,应服食白色或红色的岁谷,以安定真气;服食间谷,以驱除邪气。宜用咸味清热,苦味去火,辛味润燥,治疗宜用汗法,清法,散法,以安定其运气,避免感受邪气;以削减郁遏之气,资助化生的泉源。根据寒热的轻重,决定方药的多少。若运和气同热的,应多以清凉之品;运与气同清的,应多以温热之品。用凉药时应避免清凉的天气。用热药时应避免炎热的天气,用寒药时应避免寒冷的天气,用温药时应避免温暖的天气。饮食上,宜遵循同一法则。若天气反常,可以灵活应用,这是适应自然的法则。如果违反了它,就会扰乱天地阴阳变化的法度和规律。\\
黄帝说:好。少阳司天之年,运气情况如何?\\
岐伯说:是以寅申来标志的年份。\\
寅申年是少阳相火司天,厥阴风木在泉,若逢木运太过,就是壬寅、壬申之年。壬寅壬申太角下加厥阴又为同天符之年。其运如风鼓动,其正常生化是风鸣地坼,万物萌芽,其异常变化是狂风震撼,树木摧折。其病变是头目昏花,视物动摇不定,胸胁支满,惊骇。\\
因木运主岁,所以主运与客运都起于太角,终于太羽。\\
若逢火运太过,便是戊寅、戊申两个年份。戊寅、戊申为天符之年。其运气为暑热,其正常的生化为郁热蕴蒸。其异常变化是炎热异常,如沸腾之水,其病变是上部郁热皮肤溢血,二便下血,心胸疼痛。\\
岁运为火运太过,客运起于太徵,终于太角,而主运起于少角,终于少羽。\\
若逢土运太过,便是甲寅、甲申两个年份。其运气为阴雨连绵,其生化为润泽柔软,其异常变化是飘飞骤雨,惊天动地,其病变是身体沉重,浮肿痞满,水饮内停。\\
岁运为土运太过,客运起于太宫,终于太徵,而主运起于太角,终于太羽。\\
若逢金运太过,便是庚寅、庚申两个年份。这两年虽然金运太过,但少阳相火司天,厥阴风木在泉,所以其气运相当于金运平气之年。其运气为凉,其生化为多雾露清凉劲切,其异常变化是肃杀凋零,其病变发生于肩背胸中。\\
岁运为金运太过,客运起于太商,终于太宫,而主运起于少角,终于少羽。\\
若逢水运太过,便是丙寅、丙申两个年份。其运气为寒冷肃杀,其生化为凝凛惨烈,其变化是冰雪霜雹,其病变是伤寒浮肿。\\
岁运为水运太过,客运起于太羽,终于少商,而主运起于太角,终于太羽。\\
凡是少阳司天之年,气化运行早于正常天时,天气正常,地气扰动,狂风突然发作,吹倒树木,扬起尘沙。炎热的气候流行,厥阴风木之气随从少阳之相火而变化,雨水应时下降,火木合同发挥作用,其上应火星和木星。其应谷物为红色和深青色,其职权是严肃的,其命令是扰动的,所以风热之气互相参合散布,天空中的云物景色变换涌现不息,太阴湿土之气横行逆流,寒气时时到来,凉雨时时降落。人们多病内寒,外生疮疡,内生腹满泄泻。所以圣人遇到这些情况,调和寒热之气,不使交争。若反复发作,人们就会发寒热,疟疾,大便泄泻,耳聋,目盲,呕吐,气血怫郁于上而肿胀,皮肤变色。\\
初之气,地气迁移,风气亢盛而动摇,寒气退去,气候变得异常温暖,草木很早就繁荣。虽有寒气袭来,但草木并不为其杀伐,温热病容易发生。其病见气郁于上,出血,眼睛发红,咳嗽气逆,头痛,血崩,胁肋胀满,肌肤生疮。\\
二之气,太阴湿土加临主事,主时的少阴君火反被郁遏,白色云气四起,云奔雨府,风气不能胜过雨湿之气,雨水下降,人们身体安康。其发病则为热气郁结于上,咳嗽,气逆,呕吐,疮疡发于体内,胸胁咽喉不利,头痛发热,神识昏聩不清,脓疮。\\
三之气,司天之气发挥作用,炎暑到来,因为主气客气都是少阳相火行使职能,雨水不降。人们多病内热,耳聋,目盲,出血,脓疮,咳嗽,呕吐,鼻塞流涕,鼻出血,口渴,喷嚏,呵欠,喉痹,眼睛红赤,容易突然死亡。\\
四之气,阳明清凉之客气加于主时之太阴湿土之上,有时清凉,有时炎热,白露下降,人们体气和平,其发病为胀满,身体沉重。\\
五之气,阳气退去,寒气到来,雨水下降,人体孔窍收敛,刚硬的树木早早凋零,人们应避免寒邪,君子居处周密,以避寒气。\\
终之气,地气回迁正位而行令,风气到来,万物反有生长之势,常见有雾露飘行。其发病为闭密不禁而反发泄,心痛,阳气不能闭藏而咳嗽。\\
治疗应抑制太过的运气,资助所不胜之气,必须削弱郁遏之气,首先补益其化生的泉源,如果没有突然的太过之气发生,人们的重病也就不会发生了。所以本年适宜用咸味、辛味和酸味的药物,并用渗法,泄法,水渍,汗法,观察运气的寒温,来调节不使其有偏失。若岁运与司天在泉的风热是相同的,应多用寒凉之品,不相同的可以少用寒凉之品。用热药时应避免炎热的气候。用温药时应避免温暖的气候,用寒药时应避免寒冷的气候,用凉药时应避免清凉的气候,饮食也应该遵守同一法则,这是一般规律。若遇到反常的气候,就应当用相反的方法,否则就是疾病发生的根由。\\
黄帝说:好。太阴司天之年,运气情况如何?\\
岐伯说:是以丑未来标志的年份。\\
丑未年是太阴湿土司天,太阳寒水在泉。胜气为清,复气为热,胜复程度大致相同,所以其气化与土运的平年相当。若岁运木运不及,是丁丑、丁未两个年份。其运为风气、清气和热气。\\
因木运主岁,所以主运与客运都起于少角,终于少羽。\\
若逢火运不及,便是癸丑、癸未两个年份。寒为胜气,雨为复气,其程度大致相同。其运气主热,胜气为寒,复气为雨。\\
岁运为火运不及,客运起于少徵,终于少角,而主运起于太角,终于太羽。\\
若逢土运不及,便是己丑、己未两个年份。虽然土运不及,风为胜气,凉为复气,胜复程度大致相同,所以其气化与土运平年相当。己丑、己未又是太一天符之年。其运气为寒,胜气为雨,复气为风。\\
岁运为土运不及,客运起于少宫,终于少徵,而主运起于少角,终于少羽。\\
若逢金运不及,便是乙丑、乙未两个年份。其运气为凉,胜气为热,复气为寒。\\
岁运为金运不及,客运起于少商,终于少宫,而主运起于太角,终于太羽。\\
若逢水运不及之年,便是辛丑、辛未两个年份。水运不及,雨为胜气,风为复气,胜复程度大致相同。其气化约略同于正宫土运平年的水平。辛丑、辛未又为同岁会之年。其运气为凉,胜气为热,复气为寒。\\
岁运为水运不及,客运起于少羽,终于少商,而主运起于少角,终于少羽。\\
凡是太阴司天之年,气化运行后于正常天时,阴气专行职权,阳气退隐,大风时时刮起,天气下降。地气蒸腾,原野上昏暗不清,白云四起,云奔南方,寒雨不断下降,万物在立秋后才成熟。人们多病寒湿腹胀,身体胀满,浮肿,痞塞气逆,寒厥,手足拘急。寒湿相合发挥作用,黑黄色的埃雾迷漫,流行于气交之中,其上相应于土星和水星。其职权是严肃的,其命令是寂静的,其应于谷物是黄色和黑色。因为太阴湿气凝结于上,太阳寒气积聚于下,寒水胜火,就成为冰雹,阳气不能发挥作用,肃杀之气流行。在运气太过的年份,应在高地种植谷物;运气不及的年份,应在低下的土地种植谷物;有余的年份要种得晚,不及的年份要种得早。农业生产必须根据地利和天时的情况决定,人体之气也必须适应天时,间谷是感受太过的间气而成熟的。\\
初之气,地气迁移,寒气退去,春气到来,和风吹拂。生气布散,万物繁荣,人体之气条达舒畅,风湿之气互相纠结,不能及时降雨。人们多病出血,筋络拘挛强直,关节活动不利,身体沉重,筋脉痿软。\\
二之气,少阴君火行令用事,万物得以化育,人们身心安和。发病为温疫大流行,病状远近各地完全一样。湿气上蒸与热气相合,雨水才能及时下降。\\
三之气,司天之气行使职权,湿气下降,地气上腾,雨水及时下降,寒气随之到来。感受寒湿,人们发病多为身体沉重,浮肿,胸腹胀满。\\
四之气,少阳相火加临,湿气熏蒸,地气上腾,与火气隔拒而互不相合,早晚仍有寒风,蒸腾的湿气与热气互相纠结,草木之间如有薄烟笼罩,湿气运化既不流动,则白露暗中布散,而形成秋收之令。人们多病肌肤郁热,突然出血,疟疾,心腹胀满发热,腹部发胀,甚至发生浮肿。\\
五之气,主客都是阳明清凉之气,行使凄惨肃杀之令,寒露既下,冷霜早降,草木枯黄凋落,寒气侵入身体,君子起居谨慎周密,人们的疾病多在皮肤肌腠。\\
终之气,寒气大举袭来,湿气运化,冷霜积聚,阴气凝滞,水结成坚冰,阳气不能行使职权。人们感受寒邪,发为关节僵硬,活动不利,腰臀疼痛,这是寒湿之气持于气交而致。\\
要削弱其郁遏之气,取治其化生之泉源。岁运不及的给以补益,不使邪胜为害,服食岁谷以保全真气,服食间谷以保全阴精。本年份应用苦味之品,以燥法、温法,甚至用发法、泄法。如果不发散宣泄,则湿气充滋溢于外,会引起皮肤和肌肉溃烂,血水淋漓。必须助益阳火,使之能抵御严寒,根据运气的同异及多少,然后作出决定。岁运与司天之气同寒的应以热化调和;同湿的以燥化调和。运气不同的宜少用,相同的宜多用。用凉药时应避免清凉的气候,用寒药时应避免寒冷的气候,用温药时应避免温暖的气候,用热药时应避免炎热的气候。饮食也应该遵守同一法则。若遇到反常的气候,就应用相反的方法,这是一般规律。违背这个规律,就容易导致疾病。\\
黄帝说:好。少阴司天之年,运气情况如何?\\
岐伯说:是以子午来标志的年份。\\
子午年是少阴君火司天,阳明燥金在泉,若逢木运太过,就是壬子、壬午之年。其运如风鼓动,其正常生化是风鸣地坼,万物萌芽,其异常变化是狂风震撼,树木摧折,其病变是胸胁支满。\\
因木运主岁,所以主运与客运都起于太角,终于太羽。\\
若逢火运太过,便是戊子、戊午两个年份。戊子为天符之年、戊午为太一天符之年。其运气为暑热,其正常的生化为郁热蕴蒸。其异常变化是炎热异常,如沸腾之水,其病变是上部郁热,血液外溢。\\
岁运为火运太过,客运起于太徵,终于太角,而主运起于少角,终于少羽。\\
若逢土运太过,便是甲子、甲午两个年份。其运气为阴雨连绵,其生化为润泽,其异常变化是飘风骤雨,惊天动地,其病变是脘腹胀满,身体沉重。\\
岁运为土运太过,客运起于太宫,终于太徵,而主运起于太角,终于太羽。\\
若逢金运太过,便是庚子、庚午两个年份。这两年气运相当于金运平气之年。庚子、庚午又为同天符之年。其运气为凉爽劲切,其生化为多雾露萧瑟,其异常变化是肃杀凋零,其病变是二便清泄,下肢清凉。\\
岁运为金运太过,客运起于太商,终于太宫,而主运起于少角,终于少羽。\\
若逢水运太过,便是丙子、丙午两个年份。丙子又为岁会之年。其运气为寒冷,其生化为凝凛惨烈,其变化是冰雪霜雹,其病变是中寒下利。\\
岁运为水运太过,客运起于太羽,终于太商,而主运起于太角,终于太羽。\\
凡是少阴司天之年,气化运行早于正常天时,地气收肃,天气明朗,寒气与暑气相交融,热气和燥气相叠加,云奔于雨府,湿气的化令能行使作用,雨水及时下降,金火配合发挥作用,其相应于上的是火星、金星。司天之政布化光明,在泉地气之令急迫,其应于谷物为红色和白色。水火寒热相持于气交之中,而成为疾病的始因。热病发于上部,寒病发于下部,寒热之气互相侵犯而扰乱于中部。人们多病咳嗽,喘息,血液外溢,大便出血,鼻塞流涕,喷嚏,眼睛红赤,眼角生疡,寒气厥逆侵入胃中,心痛,腰痛,腹部胀大,咽喉干燥,头面上部肿胀。\\
初之气,地气迁移,燥气即将散去,寒气开始散布,蛰虫又伏藏,河水结冰,寒霜又下降,凉风到来,阳气被寒气郁遏,人们生活应注意起居周密。否则就会发生关节僵硬,腰臀疼痛,炎热即将到来时,内部和外部都易生疮疡。\\
二之气,阳气布散,风气流动,春天的气候正常到来,应之而万物繁荣,寒气时常到来,而人们体气安和。其发病则为小便淋沥不尽,两目视物模糊,或两目红赤,气郁于上部而发热。\\
三之气,司天和主气行使权力,君相二火当令,火气旺盛,万物茂盛鲜明,时常有寒气到来。人们发病为气逆,心痛,寒热往来,咳嗽气喘,两目红赤。\\
四之气,潮湿而炎热的气候到来,大雨时常下降,寒热交相并至。人们多病寒热,咽喉干燥,黄疸,鼻塞流涕,鼻出血,水饮病发作。\\
五之气,少阳相火加临,暑热反至,阳气运化,万物生长繁荣,人们身体安康,其病多为温病。\\
终之气,阳明燥气当令,体内余热郁留,隔拒而不能散越,上部肿胀,咳嗽气喘,甚则血液外溢。寒气时常到来,天地间出现烟雾迷漫的景象,此时疾病在外则发于皮肤腠理,在内则停留于胁肋,向下牵连到少腹而寒冷,地气又要转换了。\\
必须抑制有余的运气,资助其岁气之所胜,削弱郁结将发之气,首先要取益其化生的泉源。不使因太过而发生病变。服食岁谷以保全其真气;服食间谷以避免虚邪侵袭。本年份应该用咸寒之品以软坚,而调和其上部,甚至用苦味来发泄它,用酸味来收摄它;调和其下部,甚至用苦味来发泄它。根据运气的同异,决定用药多少。若岁运与司天的热气相同的,应以寒药清火;与在泉的清凉之气相同,则以热药来温里。用热药时要避免炎热的气候。用凉药时要避免清凉的气候,用温药时要避免温暖的气候,用寒药时要避免寒冷的气候。饮食也应该遵守同一法则。若遇到反常的气候,就应用相反的方法,这是一般法则。与此相反,容易导致疾病。\\
黄帝说:好。厥阴司天之年,运气情况如何?\\
岐伯说:是以巳亥来标志的年份。\\
巳亥年是厥阴风木司天,少阳相火在泉。胜气为清,复气为热,胜复程度大致相同,所以其气化与木运的平年相当。若岁运木运不及,是丁巳、丁亥两个年份。丁巳、丁亥又为天符之年。其运为风气、清气和热气。\\
因木运主岁,所以主运与客运都起于少角,终于少羽。\\
若逢火运不及,便是癸巳、癸亥两个年份。寒为胜气,雨为复气,其程度大致相同。癸巳、癸亥又为同岁会之年。其运气主热,胜气为寒,复气为雨。\\
岁运为火运不及,客运起于少徵,终于少角,而主运起于太角,终于太羽。\\
若逢土运不及,便是己巳、己亥两个年份。风为胜气,凉为复气,胜复程度大致相同,所以其气化与木运平年相当。其运气为雨,胜气为风,复气为清。\\
岁运为土运不及,客运起于少宫,终于少徵,而主运起于少角,终于少羽。\\
若逢金运不及,便是乙巳、乙亥两个年份。胜气为热,复气为寒,胜复程度大致相同,所以其气化与木运平年相当。其运气为凉,胜气为热,复气为寒。\\
岁运为金运不及,客运起于少商,终于少宫,而主运起于太角,终于太羽。\\
若逢水运不及,便是辛巳、辛亥两个年份。水运不及,雨为胜气,风为复气,胜复程度大致相同。其运气为寒,胜气为雨,复气为风。\\
岁运为水运不及,客运起于少羽,终于少商,而主运起于少角,终于少羽。\\
凡是厥阴司天之年,气化运行后于正常天时。如遇平气,则气化运行同于天时。风木司天,天气扰乱,少阳在泉,地气正常。风气发生于司天之气,所以高远;在泉之气在下,所以炎热之气从之。云奔雨府,湿土之气流行化育,风火配合发挥作用,其在上相应的为木星和火星。风的职权是扰乱的,火的命令是急速的,其相应的谷物是深青色和红色,间谷是感受太过的间气而成熟的,角虫和羽虫被耗损,风燥火热,交相争胜,蛰虫类反而外出活动,水流动不能结冰。热病多发于下部,风病多发于上部,风燥之气胜复交争于中部。\\
初之气,寒气收肃,杀气正来。人们右下部多生寒病。\\
二之气,寒气不去,大雪纷飞,河水结冰,肃杀之气施行作用,寒霜降下,芳草尖梢焦枯,寒雨屡次下降,阳气又散发。发病多为内热。\\
三之气,司天之气行使职权,大风时常刮起。发病为流泪,耳鸣,掉摇,头昏目眩。\\
四之气,气候炎热而潮湿,湿热互相纠结,扰乱于左上部,发病为黄疸,以致身体浮肿。\\
五之气,燥气湿气交相胜复,阴沉之气散布,寒气侵袭人体,风雨流行。\\
终之气,客气少阳相火当令,阳气旺盛发用,蛰虫出来活动,流水不能结冰,地中阳气发扬,百草重新生长,人体舒畅,发病则为温病疫病。\\
必须削弱其郁遏之气,资助其化生的泉源,赞助其不及的运气,不要使邪气偏胜。因此本年份应用辛味调和上部,以咸味来调和下部,少阳相火之气不要轻易地触犯它。用温药时要避免温暖的气候,用热药时要避免炎热的气候,用凉药时要避免清凉的气候,用寒药时要避免寒冷的气候,饮食也应该遵循同一法则。若遇到反常的天气,就应用相反的方法,这是一般准则。与此相反,就容易导致疾病。\\
黄帝说:好。夫子所讲的,可以说很全面了,然而怎样才能知道其应与不应呢?\\
岐伯说:问得真明白啊!因为六气的运行,各有一定的次序和方位。所以一般以正月初一早晨的气候为标准,来看它所在的气位。看它所在的气位,就可以知道应与不应了。凡是中运有余的,气至先于节候;中运不及的,气至后于节候。这是天道,也是六气的常态。中运既非有余,亦非不及,这就是叫“正岁”,气至与节候同时到来。\\
黄帝说:六气的胜气与复气是经常存在的。灾害也时常到来,怎样候察呢?\\
岐伯说:不属正常的气化,就可称为灾害了。\\
黄帝问:六气司天在泉之数,从开始到终止是怎样的?\\
岐伯说:问得真详细啊!这是要搞明白的医学道理呀。天地之数,开始于司天,终止于在泉;上半年是天气所主;下半年是地气所主;天地之气相交之处是气交所主。一年中的气化规律尽在其中了。所以说:主气和客气所在的位置明确了,则每气所当的月份就可以知道了,这就是六气分主六步的气数。\\
黄帝问:我主管这项工作,按照夫子所讲的为原则来做,结果运气之数与岁候并不相符,这是什么缘故?\\
岐伯说:六气的作用有太过不及,五运与六气相合的变化有盛有衰,因为有多少和盛衰的不同,所以就有同化的问题。\\
黄帝说:希望听听同化怎样?\\
岐伯说:风温之气与春天的木气同化;炎烈酷热之气与夏天的火气同化;胜气与复气也有同化,燥清烟露之气与秋天的金气同化;云雨昏埃之气与长夏的土气同化,寒霜冰雪之气与冬天的水气同化。这是天地五运六气的化洽,盛衰互用的常规。\\
黄帝说:岁运与司天之气相同的称为天符,我已经知道了。希望听听岁运与在泉之气相同的怎样?\\
岐伯说:岁运太过而与司天之气相同的有三,岁运不及而与司天之气相同的也有三,岁运太过而与在泉之气相同的有三,岁运不及而与在泉之气相同的也有三。这共计有二十四年。\\
黄帝说:希望听听其具体内容。\\
岐伯说:甲辰、甲戌年是土运太过,下加太阴在泉;壬寅、壬申年是木运太过,下加厥阴在泉;庚子、庚午年是金运太过,下加阳明在泉;这就是太过而与在泉相同的有三。癸巳、癸亥年是火运不及,下加少阳在泉;辛丑、辛未年是水运不及,下加太阳在泉;癸卯、癸酉年是火运不及,下加少阴在泉;这是不及而与在泉相同的有三。戊子、戊午年是火运太过,上临少阴司天;戊寅、戊申年是火运太过,上临少阳司天;丙辰、丙戌年是水运太过,上临太阳司天;这是太过而与司天相同的有三。丁巳、丁亥年是木运不及,上临厥阴司天;乙卯、乙酉是金运不及,上临阳明司天;己丑、己未年是土运不及,上临太阴司天;这是不及而与司天相同的有三。除了以上二十四年以外,就没有岁运与司天在泉相同的加临了。\\
黄帝问:岁运与在泉相同叫什么?\\
岐伯说:运太过而与在泉相同的称为同天符,运不及而与在泉相同的称为同岁会。\\
黄帝问:岁运与司天相同的怎样讲?\\
岐伯说:不论太过不及,都称为天符,只是其中变化有多少、病证有轻重、生死有早晚的不同罢了。\\
黄帝说:夫子说用寒药应当避免寒冷,用热药应当避免炎热。我还不知道它的具体做法,希望听听什么叫“远”?\\
岐伯说:天热不要用热药,天寒不要用寒药。顺从这一规律则身体和平,违逆这一律则则身体生病,不可不敬畏谨慎而远离它,这就是所说的主气与客气。\\
黄帝问:温凉的情况怎样呢?\\
岐伯说:当旺之气是热,不可用热药;当旺之气是寒,不可用寒药;当旺之气是凉,不可用凉药;当旺之气是温,不可用温药。间气与主气相同的也应避免,与主气不符的可稍稍违逆之。这就是所谓的寒热温凉四畏,必须谨慎考察它。\\
黄帝说:讲得对!如果触犯了会怎样呢?\\
岐伯说:客气与主气不相合的,可以依照主气之时令;客气反胜主气的就可违逆之。以达到平和为目的,不可过度,这是由于邪气反而胜过主时之气的缘故。所以说:不要违背天气应时,不要违背六气的宜忌,不助胜气,不助复气,这是最好的治法。\\
黄帝说:对。五运之气化流行,主岁的纲纪有没有常数呢?\\
岐伯说:让我依次讲解。\\
甲子 甲午年\\
上临少阴君火司天,甲为阳年属太宫,所以中值土运太过,下加阳明燥金在泉。司天热化之数二,中运雨化之数五,在泉燥化之数四,本年无胜复之气,所以叫“正化日”。其气化所致之病,司天热气所致的该用咸寒,中运雨湿之气所致的该用苦热,在泉燥气所致的该用酸热,这是这两年适宜的药食气味。\\
乙丑 乙未年\\
上临太阴湿土司天,中值金运不及,下加太阳寒水在泉。乙为阴年属少商,乙丑、乙未都是金运不及,都有热化的胜气和寒化的复气,胜复之气非本年正常之气,所以称“邪气化日”。胜复之气所致的灾害在西方。司天湿化之数五,中运清化之数四,在泉寒化之数六,这是正气所化,所以称“正化日”。其气化所致之病,司天湿土之气所致的该用苦热,中运清气所致的该用酸和,在泉寒气所致的该用甘热,这是这两年适宜的药食气味。\\
丙寅 丙申年\\
上临少阳相火司天,丙为阳年属太羽,所以中值水运太过,下加厥阴风木在泉。司天火化之数二,中运寒化之数六,在泉风化之数四,本年无胜复之气,所以叫“正化日”。其气化所致之病,司天热气所致的该用咸寒,中运寒水之气所致的该用咸温,在泉风气所致的该用辛温,这是这两年适宜的药食气味。\\
丁卯(岁会) 丁酉年\\
上临阳明燥金司天,中值少角木运不及,下加少阴君火在泉。丁为阴年属少徵,丁卯、丁酉都是火运不及,都有清化的胜气和热化的复气,胜复之气非本年正常之气,所以称“邪气化日”。胜复之气所致的灾害在东方。司天燥化之数九,中运风化之数三,在泉热化之数七,这是正气所化,所以称“正化日”。其气化所致之病,司天燥金之气所致的该用苦小温,中运风气所致的该用辛和,在泉热气所致的该用咸寒,这是这两年适宜的药食气味。\\
戊辰 戊戌年\\
上临太阳寒水司天,戊为阳年属太徵,所以中值火运太过,下加太阴湿土在泉。司天寒化之数六,中运热化之数七,在泉湿化之数五,本年无胜复之气,所以叫“正化日”。其气化所致之病,司天寒气所致的该用苦温,中运火气所致的该用甘和,在泉湿气所致的该用甘温,这是这两年适宜的药食气味。\\
己巳 己亥年\\
上临厥阴风木司天,中属少宫土运不及,下加少阳相火在泉。己为阴年属少宫,己巳、己亥都是土运不及,都有风化的胜气和清化的复气,胜复之气非本年正常之气,所以称“邪气化日”。胜复之气所致的灾害在中央。司天风化之数三,中运湿化之数五,在泉火化之数七,这是正气所化,所以称“正化日”。其气化所致之病,司天风木之气所致的该用辛凉,中运湿气所致的该用甘和,在泉热气所致的该用咸寒,这是这两年适宜的药食气味。\\
庚午(同天符) 庚子年(同天符)\\
上临少阴君火司天,庚为阳年属太商,所以中值金运太过,下加阳明燥金在泉。司天热化之数七,中运清化之数九,在泉燥化之数九,本年无胜复之气,所以叫“正化日”。其气化所致之病,司天热气所致的该用咸寒,中运清气所致的该用辛温,在泉燥气所致的该用酸温,这是这两年适宜的药食气味。\\
辛未(同岁会) 辛丑年(同岁会)\\
上临太阴湿土司天,中值少羽水运不及,下加太阳寒水在泉。辛为阴年属少羽,辛未、辛丑都是水运不及,都有雨化的胜气和风化的复气,胜复之气非本年正常之气,所以称“邪气化日”。胜复之气所致的灾害在北方。司天雨化之数五,在泉寒化之数一,这是正气所化,所以称“正化日”。其气化所致之病,司天湿土之气所致的该用苦热,中运水气所致的该用苦和,在泉寒气所致的该用苦热,这是这两年适宜的药食气味。\\
壬申(同天符) 壬寅年(同天符)\\
上临少阳相火司天,壬为阳年属太角,所以中值木运太过,下加厥阴风木在泉。司天火化之数二,在泉风化之数八,本年无胜复之气,所以叫“正化日”。其气化所致之病,司天火气所致的该用咸寒,中运风气所致的该用酸和,在泉风气所致的该用辛凉,这是这两年适宜的药食气味。\\
癸酉(同岁会) 癸卯年(同岁会)\\
上临阳明燥金司天,中值少徵火运不及,下加少阴君火在泉。癸为阴年属少徵,癸酉、癸卯都是火运不及,都有寒化的胜气和雨化的复气,胜复之气非本年正常之气,所以称“邪气化日”。胜复之气所致的灾害在南方。司天燥化之数九,在泉热化之数二,这是正气所化,所以称“正化日”。其气化所致之病,司天燥金之气所致的该用苦小温,中运火气所致的该用咸温,在泉热气所致的该用咸寒,这是这两年适宜的药食气味。\\
甲戌(岁会同天符) 甲辰年(岁会同天符)\\
上临太阳寒水司天,甲为阳年属太宫,所以中值土运太过,下加太阴湿土在泉。司天寒化之数六,在泉湿化之数五,本年无胜复之气,所以叫“正化日”。其气化所致之病,司天寒气所致的该用苦热,中运土气所致的该用苦温,在泉湿气所致的该用苦温,这是这两年适宜的药食气味。\\
乙亥 乙巳年\\
上临厥阴风木司天,中值少商金运不及,下加少阳相火在泉。乙为阴年属少商,乙亥、乙巳都是金运不及,都有热化的胜气和寒化的复气,胜复之气非本年正常之气,所以称“邪气化日”。胜复之气所致的灾害在西方。司天风化之数八,中运清化之数四,在泉火化之数二,这是正气所化,所以称“正化日”。其气化所致之病,司天风木之气所致的该用辛凉,中运燥金所致的该用酸和,在泉热气所致的该用咸寒,这是这两年适宜的药食气味。\\
丙子(岁会) 丙午年\\
上临少阴君火司天,丙为阳年属太羽,所以中值水运太过,下加阳明燥金在泉。司天热化之数二,中运寒化之数六,在泉清化之数四,本年无胜复之气,所以叫“正化日”。其气化所致之病,司天热气所致的该用咸寒,中运寒水之气所致的该用咸热,在泉燥气所致的该用酸温,这是这两年适宜的药食气味。\\
丁丑 丁未年\\
上临太阴湿土司天,中值少角木运不及,下加太阳寒水在泉。丁为阴年属少角,丁丑、丁未都是木运不及,都有清化的胜气和热化的复气,胜复之气非本年正常之气,所以称“邪气化日”。胜复之气所致的灾害在东方。司天雨化之数五,中运风化之数三,在泉寒化之数一,这是正气所化,所以称“正化日”。其气化所致之病,司天湿土之气所致的该用苦温,中运风气所致的该用辛温,在泉寒气所致的该用甘热,这是这两年适宜的药食气味。\\
戊寅 戊申年(天符)\\
上临少阳相火司天,戊为阳年属太徵,所以中值火运太过,下加厥阴风木在泉。司天火化之数七,在泉风化之数三,本年无胜复之气,所以叫“正化日”。其气化所致之病,司天火气所致的该用咸寒,中运火气所致的该用甘和,在泉风气所致的该用辛凉,这是这两年适宜的药食气味。\\
己卯(太一天符) 己酉年(天符)\\
上临阳明燥金司天,中值少宫土运不及,下加少阴君火在泉。己为阴年属少宫,己卯、己酉都是土运不及,都有风化的胜气和清化的复气,胜复之气非本年正常之气,所以称“邪气化日”。胜复之气所致的灾害在中央。司天清化之数九,中运雨化之数五,在泉热化之数七,这是正气所化,所以称“正化日”。其气化所致之病,司天燥金之气所致的该用苦小温,中运土气所致的该用甘和,在泉热气所致的该用咸寒,这是这两年适宜的药食气味。\\
庚辰 庚戌年\\
上临太阳寒水司天,庚为阳年属太商,所以中值金运太过,下加太阴湿土在泉。司天寒化之数一,中运清化之数九,在泉雨化之数五,本年无胜复之气,所以叫“正化日”。其气化所致之病,司天寒气所致的该用苦热,中运清气所致的该用辛温,在泉湿气所致的该用甘热,这是这两年适宜的药食气味。\\
辛巳 辛亥年\\
上临厥阴风木司天,中值少羽水运不及,下加少阳相火在泉。辛为阴年属少羽,辛巳、辛亥都是水运不及,都有雨化的胜气和风化的复气,胜复之气非本年正常之气,所以称“邪气化日”。胜复之气所致的灾害在北方。司天风化之数三,中运寒化之数一,在泉火化之数七,这是正气所化,所以称“正化日”。其气化所致之病,司天风木之气所致的该用辛凉,中运水气所致的该用苦和,在泉热气所致的该用咸寒,这是这两年适宜的药食气味。\\
壬午 壬子年\\
上临少阴君火司天,壬为阳年属太角,所以中值木运太过,下加阳明燥金在泉。司天热化之数二,中运风化之数八,在泉清化之数四,本年无胜复之气,所以叫“正化日”。其气化所致之病,司天火气所致的该用咸寒,中运风木之气所致的该用酸凉,在泉燥气所致的该用酸温,这是这两年适宜的药食气味。\\
癸未 癸丑年\\
上临太阴湿土司天,中值少徵火运不及,下加太阳寒水在泉。癸为阴年属少徵,癸未、癸丑都是火运不及,都有寒化的胜气和雨化的复气,胜复之气非本年正常之气,所以称“邪气化日”。胜复之气所致的灾害在南方。司天雨化之数五,中运火化之数二,在泉寒化之数一,这是正气所化,所以称“正化日”。其气化所致之病,司天湿土之气所致的该用苦温,中运火气所致的该用咸温,在泉寒气所致的该用甘热,这是这两年适宜的药食气味。\\
甲申 甲寅年\\
上临少阳相火司天,甲为阳年属太宫,所以中值土运太过,下加厥阴风木在泉。司天火化之数二,中运雨化之数五,在泉风化之数八,本年无胜复之气,所以叫“正化日”。其气化所致之病,司天火气所致的该用咸寒,中运湿土之气所致的该用咸和,在泉风气所致的该用辛凉,这是这两年适宜的药食气味。\\
乙酉(太一天符) 乙卯年(天符)\\
上临阳明燥金司天,中值少商金运不及,下加少阴君火在泉。乙为阴年属少商,乙酉、乙卯都是金运不及,都有热化的胜气和寒化的复气,胜复之气非本年正常之气,所以称“邪气化日”。胜复之气所致的灾害在西方。司天燥化之数四,中运清化之数四,在泉热化之数二,这是正气所化,所以称“正化日”。其气化所致之病,司天燥金之气所致的该用苦小温,中运燥气所致的该用苦和,在泉热气所致的该用咸寒,这是这两年适宜的药食气味。\\
丙戌(天符) 丙辰年(天符)\\
上临太阳寒水司天,丙为阳年属太羽,所以中值水运太过,下加太阴湿土在泉。司天寒化之数六,在泉雨化之数五,本年无胜复之气,所以叫“正化日”。其气化所致之病,司天寒气所致的该用苦热,中运寒水之气所致的该用咸温,在泉雨气所致的该用甘热,这是这两年适宜的药食气味。\\
丁亥(天符) 丁巳年(天符)\\
上临厥阴风木司天,中值少角木运不及,下加少阳君火在泉。丁为阴年属少角,丁亥、丁巳都是木运不及,都有清化的胜气和热化的复气,胜复之气非本年正常之气,所以称“邪气化日”。胜复之气所致的灾害在东方。司天风化之数三,在泉火化之数七,这是正气所化,所以称“正化日”。其气化所致之病,司天风木之气所致的该用辛凉,中运风气所致的该用辛和,在泉火气所致的该用咸寒,这是这两年适宜的药食气味。\\
戊子(天符) 戊午年(太一天符)\\
上临少阴君火司天,戊为阳年属太徵,所以中值水运太过,下加阳明燥金在泉。司天热化之数七,在泉清化之数九,本年无胜复之气,所以叫“正化日”。其气化所致之病,司天热气所致的该用咸寒,中运寒水之气所致的该用甘寒,在泉风气所致的该用酸温,这是这两年适宜的药食气味。\\
己丑(太一天符) 己未年(太一天符)\\
上临太阴湿土司天,中值少宫土运不及,下加太阳寒水在泉。己为阴年属少宫,己丑、己未都是土运不及,都有风化的胜气和清化的复气,胜复之气非本年正常之气,所以称“邪气化日”。胜复之气所致的灾害在中央。司天雨化之数五,在泉寒化之数一,这是正气所化,所以称“正化日”。其气化所致之病,司天湿土之气所致的该用苦热,中运雨气所致的该用甘和,在泉寒气所致的该用甘热,这是这两年适宜的药食气味。\\
庚寅 庚申年\\
上临少阳相火司天,庚为阳年属太羽,所以中值金运太过,下加厥阴风木在泉。司天火化之数七,中运清化之数九,在泉风化之数三,本年无胜复之气,所以叫“正化日”。其气化所致之病,司天火气所致的该用咸寒,中运燥金之气所致的该用辛温,在泉风气所致的该用辛凉,这是这两年适宜的药食气味。\\
辛卯 辛酉年\\
上临阳明燥金司天,中值少羽水运不及,下加少阴君火在泉。辛为阴年属少羽,辛卯、辛酉都是水运不及,都有雨化的胜气和风化的复气,胜复之气非本年正常之气,所以称“邪气化日”。胜复之气所致的灾害在北方。司天清化之数九,中运寒化之数一,在泉热化之数七,这是正气所化,所以称“正化日”。其气化所致之病,司天燥金之气所致的该用苦小温,中运水气所致的该用苦和,在泉热气所致的该用咸寒,这是这两年适宜的药食气味。\\
壬辰 壬戌年\\
上临太阳寒水司天,壬为阳年属太角,所以中值木运太过,下加太阴湿土在泉。司天寒化之数六,中运风化之数八,在泉雨化之数五,本年无胜复之气,所以叫“正化日”。其气化所致之病,司天寒气所致的该用苦温,中运风气所致的该用酸和,在泉湿气所致的该用甘温,这是这两年适宜的药食气味。\\
癸巳(同岁会) 癸亥年(同岁会)\\
上临厥阴风木司天,中值少徵火运不及,下加少阳相火在泉。癸为阴年属少徵,癸巳、癸亥都是火运不及,都有寒化的胜气和雨化的复气,胜复之气非本年正常之气,所以称“邪气化日”。胜复之气所致的灾害在南方。司天风化之数八,在泉火化之数二,这是正气所化,所以称“正化日”。其气化所致之病,司天风木之气所致的该用辛凉,中运火气所致的该用咸和,在泉热气所致的该用咸寒,这是这两年适宜的药食气味。\\
以上定期的纪年,胜气、复气和正化,都有定数,不可不察验。所以懂得要领的,一句话就可以明白,如果不懂得要领,就会茫然无绪。说的就是这个道理。\\
黄帝说:对。五运之气,每年也有胜复的岁气吗?\\
岐伯说:抑郁至极就会发生复气,等到一定的时候就会发作。\\
黄帝说:请问它的道理如何?\\
岐伯说:五行之气因为有太过不及之不同,所以复气的发作也不同。\\
黄帝说:希望详尽地听听。\\
岐伯说:太过的发作起来急暴,不及的徐缓。急暴的致病严重,徐缓的病情持久。\\
黄帝说:其太过不及之数怎样?\\
岐伯说:太过的是成数,不及的是生数,只有土不用成数而只用生数。\\
黄帝问:郁极而发,情况怎样?\\
岐伯说:土气郁极发作之时,山岩深谷为之震动,雷声轰隆于气交之间,尘埃黄黑昏暗,湿气蒸发,化为白气,疾风骤雨飘动于高山深谷,冲击岩石,向天空飞溅,洪水从而泛滥,河水四处漫延,水退之后,田野之间土石嵬然,好像一群放牧的马。土气报复之后,化气开始敷布,时雨降下,万物才得以生、长、化、成。这时人们多患心腹胀满,肠鸣而频频如厕,甚至心痛胁肋胀满,呕吐霍乱,痰饮,水泻,肌肤浮肿,身体沉重。乌云奔向雨府,云霞环绕着朝阳,山泽间隐现昏蒙之气,这是土郁开始发作的征兆。其发作的时间是在四气当令之时,云气横于天山,或浮或游或出或没,是郁积将发之先兆。\\
金气郁极而发作之时,天气洁净,地气清明,风气清凉急切,秋凉于是到来,草木之间雾如浮烟,燥气流行,浓雾时现,肃杀之气到来,草木苍黄干枯,金气发出历历西风。人们多患咳嗽气逆,心胁胀满连及小腹,常常突然疼痛,不可转侧翻身,咽喉干燥,面有蒙尘,颜色难看。山泽干涸,地上凝结着像霜一样的白盐碱,这是金郁将发的征兆。其发作的时间是在五之气时,夜降白露,树丛深处风声凄切,是其将发之先兆。\\
水气郁极而将发之时,阳气退去,阴气突起,极寒之气到来,河泽结冰,冷气结成霜雪,甚至水气昏暗黄黑,流行于气交之中,霜降而杀伐草木,河水出现一些征兆。人们多病寒邪入侵,发心痛,腰臀疼痛,大关节活动困难,屈伸不便,容易厥逆,腹中胀满痞硬。阳气失去作用,天空中阴气沉沉,白色尘埃之气蒙蔽天空,这是水郁将发的征象。发作的时间,是君火与相火当令之前后,天色深远,微黄黑色之气,犹如散麻一样,隐约可见,色黑微黄,这是郁积将发的先兆。\\
木气郁极而将发之时,天空中尘埃昏暗,云层扰动,大风到来,掀掉房顶,折断树木,这是木气之暴发。人们多犯胃脘当心疼痛,向上支撑两胁胀满,咽喉胸膈阻塞不通,饮食不能下咽,甚至耳鸣头眩,目不识人,易发突然仆倒。天空苍茫如尘,天山一色,不能分别,或天气浑浊,黄黑之气郁结不散,又如云横天际而没有雨水下降,这是木郁将发的征象。风气之发没有定期,如果发现长河边的野草被风吹而倒伏,柔软的叶子反转而呈现出背面,高山上有松吟之声,岩洞中发出虎啸之声,这是木郁将发的先兆。\\
火气郁极将发之时,天空曛翳昏暗,日光不明,炎火流行,暑热之气到来,山泽之间热如火烤,树木被烤出汁液,大厦之上烟雾升腾,大地上浮现一层霜卤之色,井水日渐减少,蔓草变为焦黄,热极风生,以致言语不清而惑乱,湿气后于天时而至。人们多病呼吸气短,疮疡痈肿,胁腹、胸、背、头面、四肢胀满不舒,皮肤发胀,疮疡痱疹,呕逆,四肢抽搐,骨痛,骨节中如有物游动,泄泻,温疟,腹中突发疼痛,血热妄行,出血如流,津液减少,两目红赤,心中烦热,甚至昏蒙烦闷,心中懊屼不宁,容易猝死。翌日刻数终了的寅时应该凉爽,反而大热,汗水从汗孔流出,濡湿皮肤,这是火郁将发的征象。发作的时间是在四之气。动后必静,阳极反阴,热极生湿,湿土之气敷布,则万物因之而化成,百花盛开之时,又见河水结冰,雪霜降地,则火气正被郁抑,如果看到南面的池塘上,有阳气升腾,就是郁积将发的先兆。有将发的先兆,而后才有报复之气,而报复之气都是抑郁至极,然后才发作的。木的复气发无定时,水的复气,则发于二火前后。只要仔细候察其时令,就可以预期疾病的产生,违反时令岁候,五运之气不得施行,生长收藏,失去了常规,就不能够知道胜复的异常变化了。\\
黄帝问:水郁之发而见雹雪,土郁之发而见飘骤,木郁之发而见毁折,金郁之发而见清明,火郁之发而见曛昧,是什么气使它们这样的?\\
岐伯说:五运之气有太过有不及,其发作就有微有甚。轻微的只见其本气的变化,严重的兼见其下承之气的变化,只要察明下承之气的变化,就知道它发作得微甚了。\\
黄帝问说:对。五运的发作,有的不应其时,为什么?\\
岐伯说:气有盛衰,至时有先后,所以有差数。\\
黄帝问:差数有一定的日数吗?\\
岐伯说:其先后的差数都是三十日有零。\\
黄帝问:主时之气到来时有先后不同,为什么?\\
岐伯说:岁运太过的,气到来得就早;岁运不及的就迟。这是正常的气候。\\
黄帝问:气应时而到的,情况怎样?\\
岐伯说:这既不太过,也不不及,气至适当其时,否则就会有灾害。\\
黄帝说:讲得好。气有不在其所主之时而化的,是什么原因?\\
岐伯说:气太过的,应时而发生作用;而不及之气,则表现为胜己之气的作用。\\
黄帝问:四时之气,到来有早晚、高下、左右的不同,怎样候察呢?\\
岐伯说:气行有逆有顺,气至有迟有速,所以太过的其化先于天时;不及的其化后于天时。\\
黄帝问:希望听听气运行的具体情况是怎样的?\\
岐伯说:春天气由东向西而行,夏天气由南向北而行,秋天气由西向东而行,冬天气由北向南而行。所以春气发生自下而上升,秋气收敛由上而下降,夏气长成旺盛于中,冬气伏藏由表入里。春气生于东方,秋气生于西方,冬气生于北方,夏气生于南方。这是四时正常的气化。所以极高的地区,经常有冬气存在;极低洼的地区,经常有春气存在。必须仔细考察。\\
黄帝说:讲得好。\\
黄帝问道:五运六气变化相应出见的物象,其正常气化与反常变化的规律是怎样的?\\
岐伯回答说:六气变化,有正常之化,有异常之变,有胜气,有复气,有作用,有致病。它们的现象各不相同,您要问什么?\\
黄帝说:我希望全都听听。\\
岐伯说:让我详细地讲吧。\\
六气到来之时,厥阴之气为气候和平,少阴之气为气候温暖,太阴之气为湿润,少阳之气为气候炎热,阳明之气为气候清凉劲急,太阳之气为气候寒冷。这是四时气化的正常现象。\\
厥阴之气到来是风气之聚会处,使万物萌芽发生;少阴之气到来是火气之聚会处,使万物欣欣向荣;太阴之气到来是雨湿之聚会处,使万物肥满丰盛;少阳之气到来是热气之聚会处,使万物的阳气由中达外;阳明之气到来是肃杀之气聚会处,使草木更替苍老;太阳之气到来是寒气之聚会处,使万物生机潜藏。这是六气当令万物的正常变化现象。\\
厥阴之气到来,使万物发生,为和风飘荡;少阴之气到来,使万物荣盛,为形态显露;太阴之气到来,使万物化生,为湿化云雨;少阳之气到来,使万物长极,为繁茂鲜艳;阳明之气到来,使万物阳气收敛,为雾露下降;太阳之气到来,使万物生机潜藏,为阳气固密。这是六气正常变化的现象。\\
厥阴之气到来,有风气发生,终了则为肃杀;少阴之气到来,有热气发生,其中气则为寒冷;太阴之气到来,有湿气发生,其终了为暴雨;少阳之气到来,有火气发生,终了是蒸发湿气;阳明之气到来,有燥气发生,终了则为凉爽;太阳之气到来,有寒气发生,其中气则为温暖。这就是六气自然变化的常规。\\
厥阴之气到来,有毛的动物化育;少阴之气到来,有翅的动物化育;太阴之气到来,倮体的动物化育;少阳之气到来,有翼的虫类化育;阳明之气到来,有甲的动物化育;太阳之气到来,有鳞的动物化育。这是六气化育万物的常规。\\
厥阴之气到来,万物开始生化;少阴之气到来,万物欣欣向荣;太阴之气到来,万物滋润;少阳之气到来,万物茂盛;阳明之气到来,万物坚成收敛;太阳之气到来,万物闭藏。这是六气敷布,万物顺从其变化的常规。\\
厥阴之气到来,狂风怒吼,天气大凉;少阴之气到来,大热大寒;太阴之气到来,雷声大作,狂风暴雨;少阳之气到来,热风拂面,有如熏烤,晚上露水凝结成霜;阳明之气到来,草木凋落,气候反见温暖;太阳之气到来,大雪纷飞,冰雹时下,大地上有白埃之气上升。这是六气异变的常规。\\
厥阴之气到来,为万物扰动,为随风飘摇;少阴之气到来,为高明,为炎热的赤黄色火焰;太阴之气到来,为天气阴沉,有白色尘埃,为晦暗不明;少阳之气到来,为光显,为红云,为炎热;阳明之气到来,为烟尘,为霜降,为西风劲急,为秋虫凄鸣;太阳之气到来,为万物坚硬,为北风刺骨,为万物已成。这是六气行令的常规。\\
厥阴致病,为筋脉拘急;少阴致病,为疡疹,身热;太阴致病,为水饮停积,脘腹痞塞;少阳致病,为喷嚏,呕吐,疮疡;阳明致病,为皮肤浮肿;太阳致病,为关节屈伸不利。这是六气致病的常规。\\
厥阴致病,为胁肋支撑疼痛;少阴致病,为惊骇疑惑,恶寒战栗,谵语躁动;太阴致病,腹中胀满;少阳致病,为惊骇躁动,烦闷昏昧,突然发病;阳明致病,为鼻塞流涕,尻、阴股、膝、髀、腨、伄、足部发病;太阳致病,为腰痛。这也是六气致病的常规。\\
厥阴致病,为筋脉挛急短缩,肢体屈曲不伸;少阴致病,为悲哀狂妄,鼻出血;太阴致病,为腹中胀满,霍乱,呕吐泻下;少阳致病,为喉痹,耳鸣,呕逆;阳明致病,为皮肤糙裂而揭起;太阳致病,为眠中出汗,痉病。这也是六气致病的常规。\\
厥阴致病,为胁痛,呕吐,泄泻;少阴致病,为言笑不止;太阴致病,为身重浮肿;少阳致病,为突然泻下,肌肉跳动,筋脉抽掣,突然死亡;阳明致病,为鼻塞流涕,喷嚏;太阳致病,为二便失禁,或闭塞不通。这也是六气致病的常规。\\
以上十二种变化,说明万物与六气的密切关系,六气的作用是德、化、政、令,万物回答的表现也相应地是德、化、政、令。六气所至的位置,有高下、前后、中外之异,那么万物的变化也随之而有高下、前后、中外的不同。这是正常的时位。所以风气胜则动摇,热气胜则浮肿,燥气胜则皴干,寒气胜则虚浮,湿气胜则水泻,甚至小便不通、浮肿。总之,根据邪气之所在,就可以谈论其病变情况。\\
黄帝说:希望听听气化的作用。\\
岐伯说:六气的气化作用,都是加于不胜之气而产生的。太阴加于太阳而为湿化;太阳加于少阴而为寒化;少阴加于阳明而为热化;阳明加于厥阴而为燥化;厥阴加于太阴而为风化。各根据其所在的方位而预测。\\
黄帝问道:六气加于本位,情况怎样?\\
岐伯说:加于自己的位置,是正常之化。\\
黄帝说:希望听听六气所在。\\
岐伯说:确定了六气的位置,就可以知道它所主的方隅与月令了。\\
黄帝问道:六个位置的气,虚实怎样?\\
岐伯说:太过与不及是不同的。太过的气到来缓慢而作用持久;不及的气到来急暴而作用迅速消失。\\
黄帝问:司天在泉之气的虚实怎样?\\
岐伯说:司天之气不足,在泉之气就随之上升;在泉之气不足,司天之气就随之下降;岁运居司天在泉之中,它的升降常在天气地气之先。它厌恶所不胜之气,而归属同和之气,但同和则助其气,不胜则受其制,都会产生病变。因为司天之气胜,则天气下降;在泉之气胜,则地气上升。由于胜气的多少,就决定了下降与上升的差分。胜气微的差别小,胜气甚的差别大,相差太大则气交之位置移易,移易气交的位置则发生大变,因而疾病就产生了。《大要》中说:胜气甚的差别为十分之五,胜气微的差别为十分之七,其间的差分可见了。说的就是这个道理。\\
黄帝说:讲得好。经论中说过,用热药不要触犯热时,用寒药不要触犯寒时。我想要不避寒,不避热,怎样做呢?\\
岐伯说:问得真全面啊!发表不避热,攻里不避寒。\\
黄帝问:既不是发表也不是攻里,触犯了主时的寒与热,会怎么样呢?\\
岐伯说:寒与热会伤脏腑,病情就严重了。\\
黄帝说:希望听听无病的人,怎么样呢?\\
岐伯说:无病的人会生病,有病的人会加重。\\
黄帝问:生病的情况,怎么样?\\
岐伯说:不避热者热病就到来,不避寒者寒病就到来。寒病到来,则腹部胀满,坚硬痞塞,急剧疼痛,下利等病就发生了。热病到来,则发热,呕吐,泄泻,霍乱,痈疽疮疡,烦闷郁冒,泄泻,身体抽动,肿胀,呕吐鼻塞流涕,鼻出血,头痛,骨节改变,肌肉疼痛,血外溢,便血,小便淋沥不畅或癃闭不通等病就此产生了。\\
黄帝问:怎样治疗呢?\\
岐伯说:必须顺从四时寒热温凉的时序,违反四时禁忌而生病的,治以相克制的药物。\\
黄帝问道:妇人怀孕,如果用峻猛的药物,会怎样呢?\\
岐伯说:有病而应用,既不会损伤胎儿,亦不会伤害母体。\\
黄帝说:希望听听其中的道理,是怎样的?\\
岐伯说:对大积大聚的病,可以使用峻猛的药物,但必须在病邪衰减大半时即停药,如用过分了就会死亡。\\
黄帝说:讲得好。五气抑郁过甚的,怎样治疗?\\
岐伯说:木气抑郁的应该疏泄条达,火气抑郁的应该散去火热,土气抑郁的应该夺去壅滞之邪,金气抑郁的应该宣泄疏利,水气抑郁的应该驱逐水邪。这就是调畅气机。凡太过的应折其势,用其畏惧的相制之品来折之,这也就是所谓的泻法。\\
黄帝问:对假借的气怎样?\\
岐伯说:若有假借之气,则不必依照热无犯热、寒无犯寒的禁忌。所谓假借之气,就是主气不足,客气胜的异常情况。\\
黄帝说:圣人的学问真是高深至极啊!天地的伟大变化,六气运行的规律,相互加临的纲纪,阴阳的作用,寒暑的影响,如果不是夫子您,还有谁能够通晓呢!让我把它藏在灵兰书室里,署名为《六元正纪》。不是经斋戒沐浴,不敢示人,要谨慎地传授。\\
卷二十二\\
至真要大论篇第七十四\\
黄帝问曰:五气交合,盈虚更作,余知之矣。六气分治,司天地者,其至何如?\\
岐伯再拜对曰:明乎哉问也!天地之大纪,人神之通应也。\\
帝曰:愿闻上合昭昭,下合冥冥,奈何?\\
岐伯曰:此道之所主,工之所疑也。\\
帝曰:愿闻其道也。\\
岐伯曰:厥阴司天,其化以风;少阴司天,其化以热;太阴司天,其化以湿;少阳司天,其化以火;阳明司天,其化以燥;太阳司天,其化以寒。以所临脏位,命其病者也。\\
帝曰:地化奈何?\\
岐伯曰:司天同候,间气皆然。\\
帝曰:间气何谓?\\
岐伯曰:司左右者,是谓间气也。\\
帝曰:何以异之?\\
岐伯曰:主岁者纪岁,间气者纪步也。\\
帝曰:善。岁主奈何?\\
岐伯曰:厥阴司天为风化,在泉为酸化,司气为苍化,间气为动化。少阴司天为热化,在泉为苦化,不司气化,居气为灼化。太阴司天为湿化,在泉为甘化,司气为黅化,间气为柔化。少阳司天为火化,在泉为苦化,司气为丹化,间气为明化。阳明司天为燥化,在泉为辛化,司气为素化,间气为清化。太阳司天为寒化,在泉为咸化,司气为玄化,间气为藏化。故治病者,必明六化分治,五味五色所生,五脏所宜,乃可以言盈虚,病生之绪也。\\
帝曰:厥阴在泉而酸化先,余知之矣。风化之行也,何如?\\
岐伯曰:风行于地,所谓本也,余气同法。本乎天者,天之气也;本乎地者,地之气也。天地合气,六节分,而万物化生矣。故曰:谨候气宜,无失病机。此之谓也。\\
帝曰:其主病,何如?\\
岐伯曰:司岁备物,则无遗主矣。\\
帝曰:先岁物,何也?\\
岐伯曰:天地之专精也。\\
帝曰:司气者,何如?\\
岐伯曰:司气者主岁同,然有余不足也。\\
帝曰:非司岁物,何谓也?\\
岐伯曰:散也,故质同而异等也。气味有薄厚,性用有躁静,治保有多少,力化有浅深。此之谓也。\\
帝曰:岁主脏害,何谓?\\
岐伯曰:以所不胜命之,则其要也。\\
帝曰:治之奈何?\\
岐伯曰:上淫于下,所胜平之;外淫于内,所胜治之。\\
帝曰:善。平气何如?\\
岐伯曰:谨察阴阳所在,而调之,以平为期。正者正治,反者反治。\\
帝曰:夫子言察阴阳所在,而调之,论言人迎与寸口相应,若引绳小大齐等,命曰平。阴之所在寸口,何如?\\
岐伯曰:视岁南北,可知之矣。\\
帝曰:愿卒闻之。\\
岐伯曰:北政之岁,少阴在泉,则寸口不应;厥阴在泉,则右不应;太阴在泉,则左不应。南政之岁,少阴司天,则寸口不应:厥阴司天,则右不应;太阴司天,则左不应。诸不应者,反其诊,则见矣。\\
帝曰:尺候如何?\\
岐伯曰:北政之岁,三阴在下,则寸不应;三阴在上,则尺不应。南政之岁,三阴在天,则寸不应;三阴在泉,则尺不应。左右同。故曰:知其要者,一言而终,不知其要,流散无穷。此之谓也。\\
帝曰:善。天地之气,内淫而病,何如?\\
岐伯曰:岁厥阴在泉,风淫所胜,则地气不明,平野昧,草乃早秀。民病洒洒振寒,善伸数欠,心痛支满,两胁里急,饮食不下,鬲咽不通,食则呕,腹胀善噫,得后与气,则快然如衰,身体皆重。\\
岁少阴在泉,热淫所胜,则焰浮川泽,阴处反明。民病腹中肠鸣,气上冲胸,喘,不能久立,寒热,皮肤痛,目瞑,齿痛,帄肿,恶寒发热如疟,少腹中痛,腹大。蛰虫不藏。\\
岁太阴在泉,草乃早荣,湿淫所胜,则埃昏岩谷,黄反见黑,至阴之交,民病饮积,心痛,耳聋,浑浑焞焞,嗌肿喉痹,阴病血见,少腹痛肿,不得小便,病冲头痛,目似脱,项似拔,腰似折,髀不可以回,腘如结,腨如别。\\
岁少阳在泉,火淫所胜,则焰明郊野,寒热更至。民病注泄赤白,少腹痛,溺赤,甚则便血。少阴同候。\\
岁阳明在泉,燥淫所胜,则霿雾清瞑。民病喜呕,呕有苦,善太息,心胁痛,不能反侧,甚则嗌干面尘,身无膏泽,足外反热。\\
岁太阳在泉,寒淫所胜,则凝肃惨慄。民病少腹控睾、引腰脊,上冲心痛,血见,嗌痛颔肿。\\
帝曰:善。治之奈何?\\
岐伯曰:诸气在泉,风淫于内,治以辛凉,佐以苦甘,以甘缓之,以辛散之。热淫于内,治以咸寒,佐以甘苦,以酸收之,以苦发之。湿淫于内,治以苦热,佐以酸淡,以苦燥之,以淡泄之。火淫于内,治以咸冷,佐以苦辛,以酸收之,以苦发之。燥淫于内,治以苦温,佐以甘辛,以苦下之。寒淫于内,治以甘热,佐以苦辛,以咸泻之,以辛润之,以苦坚之。\\
帝曰:善。天气之变,何如?\\
岐伯曰:厥阴司天,风淫所胜,则太虚埃昏,云物以扰,寒生春气,流水不冰,蛰虫不去。民病胃脘当心而痛,上支两胁,鬲咽不通,饮食不下,舌本强,食则呕,冷泄腹胀,溏泄,瘕,水闭,病本于脾。冲阳绝,死不治。\\
少阴司天,热淫所胜,怫热,大雨且至,火行其政。民病胸中烦热,嗌干,右胠满,皮肤痛,寒热咳喘,唾血血泄,鼽衄嚏呕,溺色变,甚则疮疡胕肿,肩背臂臑,及缺盆中痛,心痛,肺尒,腹大满,膨膨而咳喘,病本于肺。尺泽绝,死不治。\\
太阴司天,湿淫所胜,则沉阴且布,雨变枯槁。胕肿,骨痛,阴痹。阴痹者,按之不得,腰脊头项痛,时眩,大便难,阴气不用,饥不欲食,咳唾则有血,心如悬,病本于肾。太谿绝,死不治。\\
少阳司天,火淫所胜,则温气流行,金政不平。民病头痛,发热恶寒而疟,热上,皮肤痛,色变黄赤,传而为水,身面肘肿,腹满仰息,泄注赤白,疮疡,咳唾血,烦心,胸中热,甚则鼽衄,病本于肺。天府绝,死不治。\\
阳明司天,燥淫所胜,则木乃晚荣,草乃晚生。筋骨内变,大凉革候,名木敛,生菀于下,草焦上首,蛰虫来见。民病左胠胁痛,寒清于中,感而疟,咳,腹中鸣,注泄鹜溏,心胁暴痛,不可反侧,嗌干,面尘,腰痛,丈夫庂疝,妇人少腹痛,目眛眦疡,疮痤痈,病本于肝。太冲绝,死不治。\\
太阳司天,寒淫所胜,则寒气反至,水且冰,运火炎烈,雨暴乃雹。民病血变于中,发为痈疡,厥心痛,呕血,血泄,鼽衄,善悲,时眩仆,胸腹满,手热,肘挛,掖肿,心澹澹大动,胸胁胃脘不安,面赤目黄,善噫,嗌干,甚则色炲,渴而欲饮,病本于心。神门绝,死不治。所谓动气,知其脏也。\\
帝曰:善。治之奈何?\\
岐伯曰:司天之气,风淫所胜,平以辛凉,佐以苦甘,以甘缓之,以酸泄之。热淫所胜,平以咸寒,佐以苦甘,以酸收之。湿淫所胜,平以苦热,佐以酸辛,以苦燥之,以淡泄之。湿上甚而热,治以苦温,佐以甘辛,以汗为故而止。火淫所胜,平以咸冷,佐以苦甘,以酸收之,以苦发之,以酸复之,热淫同。燥淫所胜,平以苦温,佐以酸辛,以苦下之。寒淫所胜,平以辛热,佐以甘苦,以咸泻之。\\
帝曰:善。邪气反胜,治之奈何?\\
岐伯曰:风司于地,清反胜之,治以酸温,佐以苦甘,以辛平之。热司于地,寒反胜之,治以甘热,佐以苦辛,以咸平之。湿司于地,热反胜之,治以苦冷,佐以咸甘,以苦平之。火司于地,寒反胜之,治以甘热,佐以苦辛,以咸平之。燥司于地,热反胜之,治以平寒,佐以苦甘,以酸平之,以和为利。寒司于地,热反胜之,治以咸冷,佐以甘辛,以苦平之。\\
帝曰:其司天邪胜,何如?\\
岐伯曰:风化于天,清反胜之,治以酸温,佐以甘苦。热化于天,寒反胜之,治以甘温,佐以苦酸辛。湿化于天,热反胜之,治以苦寒,佐以苦酸。火化于天,寒反胜之,治以甘热,佐以苦辛。燥化于天,热反胜之,治以辛寒,佐以苦甘。寒化于天,热反胜之,治以咸冷,佐以苦辛。\\
帝曰:六气相胜,奈何?\\
岐伯曰:厥阴之胜,耳鸣头眩,愦愦欲吐,胃鬲如寒,大风数举,倮虫不滋,胠胁气并,化而为热,小便黄赤,胃脘当心而痛,上支两胁,肠鸣,飧泄,少腹痛,注下赤白,甚则呕吐,鬲咽不通。\\
少阴之胜,心下热,善饥,脐下反动,气游三焦。炎暑至,木乃津,草乃萎。呕逆烦躁,腹满痛,溏泄,传为赤沃。\\
太阴之胜,火气内郁,疮疡于中,流散于外,病在胠胁,甚则心痛,热格,头痛,喉痹,项强,独胜则湿气内郁,寒迫下焦,痛留顶,互引眉间,胃满。雨数至,湿化乃见,少腹满,腰脽重强,内不便,善注泄,足下温,头重,足胫胕肿,饮发于中,胕肿于上。\\
少阳之胜,热客于胃,烦心心痛,目赤欲呕,呕酸善饥,耳痛溺赤,善惊谵妄,暴热消烁,草萎水涸,介虫乃屈,少腹痛,下沃赤白。\\
阳明之胜,清发于中,左胠胁痛,溏泄,内为嗌塞,外发庂疝。大凉肃杀,华英改容,毛虫乃殃,胸中不便,嗌塞而咳。\\
太阳之胜,凝凓且至,非时水冰,羽乃后化。痔疟发,寒厥入胃,则内生心痛,阴中乃疡,隐曲不利,互引阴股,筋肉拘苛,血脉凝泣,络满色变,或为血泄,皮肤否肿,腹满食减,热反上行,头项囱顶,脑户中痛,目如脱,寒入下焦,传为濡泻。\\
帝曰:治之奈何?\\
岐伯曰:厥阴之胜,治以甘清,佐以苦辛,以酸泻之。少阴之胜,治以辛寒,佐以苦咸,以甘泻之。太阴之胜,治以咸热,佐以辛甘,以苦泻之。少阳之胜,治以辛寒,佐以甘咸,以甘泻之。阳明之胜,治以酸温,佐以辛甘,以苦泻之。太阳之胜,治以甘热,佐以辛酸,以咸泻之。\\
帝曰:六气之复,何如?\\
岐伯曰:悉乎哉问也!厥阴之复,少腹坚满,里急暴痛。偃木飞沙,倮虫不荣。厥心痛,汗发呕吐,饮食不入,入而复出,筋骨掉眩,清厥,甚则入脾,食痹而吐。冲阳绝,死不治。\\
少阴之复,燠热内作,烦躁鼽嚏,少腹绞痛,火见燔焫,嗌燥,分注时止,气动于左,上行于右,咳,皮肤痛,暴瘖心痛,郁冒不知人,乃洒淅恶寒,振慄谵妄,寒已而热,渴而欲饮,少气骨痿,隔肠不便,外为浮肿,哕噫。赤气后化,流水不冰,热气大行,介虫不复。病疿胗疮疡,痈疽痤痔。甚则入肺,咳而鼻渊。天府绝,死不治。\\
太阴之复,湿变乃举,体重中满,食饮不化,阴气上厥,胸中不便,饮发于中,咳喘有声。大雨时行,鳞见于陆。头顶痛重,而掉瘛尤甚,呕而密默,唾吐清液,甚则入肾,窍泻无度。太谿绝,死不治。\\
少阳之复,大热将至,枯燥燔爇,介虫乃耗,惊瘛咳衄,心热烦躁,便数憎风,厥气上行,面如浮埃,目乃丱瘛,火气内发,上为口糜呕逆,血溢血泄,发而为疟,恶寒鼓慄,寒极反热,嗌络焦槁,渴引水浆,色变黄赤,少气脉萎,化而为水,传为胕肿,甚则入肺,咳而血泄。尺泽绝,死不冶。\\
阳明之复,清气大举,森木苍干,毛虫乃厉。病生胠胁,气归于左,善太息,甚则心痛否满,腹胀而泄,呕苦,咳哕,烦心,病在鬲中,头痛,甚则入肝,惊骇筋挛。太冲绝,死不治。\\
太阳之复,厥气上行,水凝雨冰,羽虫乃死,心胃生寒,胸膈不利,心痛否满,头痛善悲,时眩仆,食减,腰脽反痛,屈伸不便,地裂冰坚,阳光不治,少腹控睾,引腰脊,上冲心,唾出清水,及为哕噫,甚则入心,善忘善悲。神门绝,死不治。\\
帝曰:善。治之奈何?\\
岐伯曰:厥阴之复,治以酸寒,佐以甘辛,以酸泻之,以甘缓之。少阴之复,治以咸寒,佐以苦辛,以甘泻之,以酸收之,辛苦发之,以咸软之。太阴之复,治以苦热,佐以酸辛,以苦泻之,燥之,泄之。少阳之复,治以咸冷,佐以苦辛,以咸软之,以酸收之,辛苦发之。发不远热,无犯温凉。少阴同法。阳明之复,治以辛温,佐以苦甘,以苦泄之,以苦下之,以酸补之。太阳之复,治以咸热,佐以甘辛,以苦坚之。治诸胜复,寒者热之,热者寒之,温者清之;清者温之,散者收之,抑者散之,燥者润之,急者缓之,坚者软之,脆者坚之,衰者补之,强者泻之。各安其气,必清必静,则病气衰去,归其所宗。此治之大体也。\\
帝曰:善。气之上下,何谓也?\\
岐伯曰:身半以上,其气三矣,天之分也,天气主之;身半以下,其气三矣,地之分也,地气主之。以名命气,以气命处,而言其病。半,所谓天枢也。故上胜而下俱病者,以地名之;下胜而上俱病者,以天名之。所谓胜至,报气屈伏而未发也。复至则不以天地异名,皆如复气为法也。\\
帝曰:胜复之动,时有常乎?气有必乎?\\
岐伯曰:时有常位,而气无必也。\\
帝曰:愿闻其道也。\\
岐伯曰:初气终三气,天气主之,胜之常也。四气尽终气,地气主之,复之常也。有胜则复,无胜则否。\\
帝曰:善。复已而胜,何如?\\
岐伯曰:胜至则复,无常数也,衰乃止耳。复已而胜,不复则害,此伤生也。\\
帝曰:复而反病,何也?\\
岐伯曰:居非其位,不相得也。大复其胜,则主胜之,故反病也。所谓火燥热也。\\
帝曰:治之何如?\\
岐伯曰:夫气之胜也,微者随之,甚则制之。气之复也,和者平之,暴者夺之。皆随胜气,安其屈伏,无问其数,以平为期。此其道也。\\
帝曰:善。客主之胜复,奈何?\\
岐伯曰:客主之气,胜而无复也。\\
帝曰:其逆从,何如?\\
岐伯曰:主胜逆,客胜从,天之道也。\\
帝曰:其生病,何如?\\
岐伯曰:厥阴司天,客胜则耳鸣掉眩,甚则咳;主胜则胸胁痛,舌难以言。少阴司天,客胜则鼽嚏,颈项强,肩背瞀热,头痛少气,发热,耳聋目瞑,甚则胕肿血溢,疮疡咳喘;主胜则心热烦躁,甚则胁痛支满。\\
太阴司天,客胜则首面胕肿,呼吸气喘;主胜则胸腹满,食已而瞀。\\
少阳司天,客胜则丹胗外发,及为丹熛疮疡,呕逆喉痹,头痛嗌肿,耳聋血溢,内为瘛疭;主胜则胸满,咳,仰息,甚而有血,手热。\\
阳明司天,清复内余,则咳衄嗌塞,心鬲中热,咳不止,面白血出者死。\\
太阳司天,客胜则胸中不利,出清涕,感寒则咳;主胜则喉嗌中鸣。\\
厥阴在泉,客胜则大关节不利,内为痉强拘瘛,外为不便;主胜则筋骨繇并,腰腹时痛。\\
少阴在泉,客胜则腰痛,尻股膝髀腨气足病,瞀热以酸,胕肿不能久立,溲便变;主胜则厥气上行,心痛发热,鬲中众痹皆作,发于胠胁,魄汗不藏,四逆而起。\\
太阴在泉,客胜则足痿下重,便溲不时,湿客下焦,发而濡泻,及为肿,隐曲之疾;主胜则寒气逆满,食饮不下,甚则为疝。\\
少阳在泉,客胜则腰腹痛,而反恶寒,甚则下白、溺白;主胜则热反上行,而客于心,心痛发热,格中而呕。少阴同候。\\
阳明在泉,客胜则清气动下,少腹坚满,而数便泻,主胜则腰重腹痛,少腹生寒,下为鹜溏,则寒厥于肠,上冲胸中,甚则喘,不能久立。\\
太阳在泉,寒复内余,则腰尻痛,屈伸不利,股胫足膝中痛。\\
帝曰:善。治之奈何?\\
岐伯曰:高者抑之,下者举之,有余折之,不足补之。佐以所利,和以所宜。必安其主客,适其寒温。同者逆之,异者从之。\\
帝曰:治寒以热,治热以寒。气相得者逆之,不相得者从之。余以知之矣。其于正味,何如?\\
岐伯曰:木位之主,其泻以酸,其补以辛。火位之主,其泻以甘,其补以咸。土位之主,其泻以苦,其补以甘。金位之主,其泻以辛,其补以酸。水位之主,其泻以咸,其补以苦。\\
厥阴之客,以辛补之,以酸泻之,以甘缓之。少阴之客,以咸补之,以甘泻之,以酸收之。太阴之客,以甘补之,以苦泻之,以甘缓之。少阳之客,以咸补之,以甘泻之,以咸软之。阳明之客,以酸补之,以辛泻之,以苦泄之。太阳之客,以苦补之,以咸泻之,以苦坚之,以辛润之。开发腠理,致津液,通气也。\\
帝曰:善。愿闻阴阳之三也,何谓?\\
岐伯曰:气有多少,异用也。\\
帝曰:阳明,何谓也?\\
岐伯曰:两阳合明也。\\
帝曰:厥阴,何也?\\
岐伯曰:两阴交尽也。\\
帝曰:气有多少,病有盛衰,治有缓急,方有大小,愿闻其约奈何?\\
岐伯曰:气有高下,病有远近,证有中外,治有轻重,适其至所为故也。《大要》曰:君一臣二,奇之制也;君二臣四,偶之制也;君二臣三,奇之制也;君二臣六,偶之制也。故曰,近者奇之,远者偶之;汗者不以奇,下者不以偶;补上治上制以缓,补下治下制以急;急则气味厚,缓则气味薄。适其至所,此之谓也。病所远,而中道气味乏者,食而过之,无越其制度也。是故平气之道,近而奇偶,制小其服也;远而奇偶,制大其服也。大则数少,小则数多。多则九之,少则二之。奇之不去则偶之,是谓重方。偶之不去,则反佐以取之,所谓寒热温凉,反从其病也。\\
帝曰:善。病生于本,余知之矣。生于标者,治之奈何?\\
岐伯曰:病反其本,得标之病;治反其本,得标之方。\\
帝曰:善。六气之胜,何以候之?\\
岐伯曰:乘其至也。清气大来,燥之胜也,风木受邪,肝病生焉。热气大来,火之胜也,金燥受邪,肺病生焉。寒气大来,水之胜也,火热受邪,心病生焉。湿气大来,土之胜也,寒水受邪,肾病生焉。风气大来,木之胜也,土湿受邪,脾病生焉。所谓感邪而生病也。乘年之虚,则邪甚也。失时之和,亦邪甚也。遇月之空,亦邪甚也。重感于邪,则病危矣。有胜之气,其必来复也。\\
帝曰:其脉至,何如?\\
岐伯曰:厥阴之至其脉弦;少阴之至其脉钩;太阴之至其脉沉;少阳之至大而浮;阳明之至短而涩;太阳之至大而长。至而和则平,至而甚则病,至而反者病,至而不至者病,未至而至者病,阴阳易者危。\\
帝曰:六气标本,所从不同,奈何?\\
岐伯曰:气有从本者,有从标本者,有不从标本者也。\\
帝曰:愿卒闻之。\\
岐伯曰:少阳、太阴从本,少阴、太阳从本从标,阳明、厥阴,不从标本,从乎中也。故从本者,化生于本;从标本者,有标本之化;从中者,以中气为化也。\\
帝曰:脉从而病反者,其诊何如?\\
岐伯曰:脉至而从,按之不鼓,诸阳皆然。\\
帝曰:诸阴之反,其脉何如?\\
岐伯曰:脉至而从,按之鼓甚而盛也。\\
是故百病之起,有生于本者,有生于标者,有生于中气者。有取本而得者,有取标而得者,有取中气而得者,有取标本而得者,有逆取而得者,有从取而得者。逆,正顺也;若顺,逆也。故曰:知标与本,用之不殆,明知顺逆,正行无问。此之谓也。不知是者,不足以言诊,足以乱经。故《大要》曰:粗工嘻嘻,以为可知,言热未已,寒病复始。同气异形,迷诊乱经。此之谓也。夫标本之道,要而博,小而大,可以言一,而知百病之害。言标与本,易而勿损;察本与标,气可令调。明知胜复,为万民式。天之道,毕矣。\\
帝曰:胜复之变,早晏何如?\\
岐伯曰:夫所胜者,胜至已病,病已愠愠,而复已萌也。夫所复者,胜尽而起,得位而甚。胜有微甚,复有少多。胜和而和,胜虚而虚。天之常也。\\
帝曰:胜复之作,动不当位,或后时而至,其故何也?\\
岐伯曰:夫气之生,与其化,衰盛异也。寒暑温凉,盛衰之用,其在四维。故阳之动,始于温,盛于暑;阴之动,始于清,盛于寒。春夏秋冬,各差其分。故《大要》曰:彼春之暖,为夏之暑,彼秋之忿,为冬之怒。谨按四维,斥候皆归,其终可见,其始可知。此之谓也。\\
帝曰:差有数乎?\\
岐伯曰:又凡三十度也。\\
帝曰:其脉应,皆何如?\\
岐伯曰:差同正法,待时而去也。《脉要》曰:春不沉,夏不弦,冬不涩,秋不数,是谓四塞。沉甚曰病,弦甚曰病,涩甚曰病,数甚曰病;参见曰病,复见曰病;未去而去曰病,去而不去曰病,反者死。故曰:气之相守司也,如权衡之不得相失也。夫阴阳之气,清静则生化治,动则苛疾起。此之谓也。\\
帝曰:幽明何如?\\
岐伯曰:两阴交尽,故曰幽;两阳合明,故曰明。幽明之配,寒暑之异也。\\
帝曰:分至何如?\\
岐伯曰:气至之谓至,气分之谓分;至则气同,分则气异。所谓天地之正纪也。\\
帝曰:夫子言春秋气始于前,冬夏气始于后,余已知之矣。然六气往复,主岁不常也,其补泻奈何?\\
岐伯曰:上下所主,随其攸利,正其味,则其要也。左右同法。《大要》曰:少阳之主,先甘后咸;阳明之主,先辛后酸;太阳之主,先咸后苦;厥阴之主,先酸后辛;少阴之主,先甘后咸;太阴之主,先苦后甘。佐以所利,资以所生,是谓得气。\\
帝曰:善。夫百病之生也,皆生于风寒暑湿燥火,以之化之变也。经言盛者泻之,虚则补之。余锡以方士,而方士用之,尚未能十全,余欲令要道必行,桴鼓相应,犹拔刺雪污,工巧神圣,可得闻乎?\\
岐伯曰:审察病机,无失气宜,此之谓也。\\
帝曰:愿闻病机如何?\\
岐伯曰:诸风掉眩,皆属于肝。诸寒收引,皆属于肾。诸气刉郁,皆属于肺。诸湿肿满,皆属于脾。诸热瞀瘈,皆属于火。诸痛痒疮,皆属于心。诸厥固泄,皆属于下。诸痿喘呕,皆属于上。诸禁鼓慄,如丧神守,皆属于火。诸痉项强,皆属于湿。诸逆冲上,皆属于火。诸腹胀大,皆属于热。诸躁狂越,皆属于火。诸暴强直,皆属于风。诸病有声,鼓之如鼓,皆属于热。诸病胕肿,疼酸惊骇,皆属于火。诸转反戾,水液浑浊,皆属于热。诸病水液,澄澈清冷,皆属于寒。诸呕吐酸,暴注下迫,皆属于热。故《大要》曰:谨守病机,各司其属,有者求之,无者求之,盛者责之,虚者责之,必先五胜,疏其血气,令其调达,而致和平。此之谓也。\\
帝曰:善。五味阴阳之用,何如?\\
岐伯曰:辛甘发散为阳,酸苦涌泄为阴,咸味涌泄为阴,淡味渗泄为阳。六者,或收或散,或缓或急,或燥或润,或软或坚,以所利而行之,调其气,使其平也。\\
帝曰:非调气而得者,治之奈何?有毒无毒,何先何后?愿闻其道。\\
岐伯曰:有毒无毒,所治为主,适大小为制也。\\
帝曰:请言其制。\\
岐伯曰:君一臣二,制之小也;君一臣三佐五,制之中也;君一臣三佐九,制之大也。寒者热之,热者寒之,微者逆之,甚者从之,坚者削之,客者除之,劳者温之,结者散之,留者攻之,燥者濡之,急者缓之,散者收之,损者温之,逸者行之,惊者平之,上之下之,摩之浴之,薄之劫之,开之发之,适事为故。\\
帝曰:何谓逆从?\\
岐伯曰:逆者正治,从者反治,从少从多,观其事也。\\
帝曰:反治何谓?\\
岐伯曰:热因热用,寒因寒用,塞因塞用,通因通用。必伏其所主,而先其所因。其始则同,其终则异。可使破积,可使溃坚,可使气和,可使必已。\\
帝曰:善。气调而得者,何如?\\
岐伯曰:逆之,从之,逆而从之,从而逆之,疏气令调,则其道也。\\
帝曰:善。病之中外何如?\\
岐伯曰:从内之外者调其内;从外之内者治其外;从内之外而盛于外者,先调其内而后治其外;从外之内而盛于内者,先治其外而后调其内;中外不相及则治主病。\\
帝曰:善。火热复,恶寒发热,有如疟状,或一日发,或间数日发,其故何也?\\
岐伯曰:胜复之气,会遇之时,有多少也。阴气多而阳气少,则其发日远;阳气多而阴气少,则其发日近。此胜复相薄,盛衰之节。疟亦同法。\\
帝曰:论言治寒以热,治热以寒,而方士不能废绳墨而更其道也。有病热者寒之而热,有病寒者热之而寒,二者皆在,新病复起,奈何治?\\
岐伯曰:诸寒之而热者取之阴,热之而寒者取之阳,所谓求其属也。\\
帝曰:善。服寒而反热,服热而反寒,其故何也?\\
岐伯曰:治其王气,是以反也。\\
帝曰:不治王而然者何也?\\
岐伯曰:悉乎哉问也!不治五味属也。夫五味入胃,各归所喜,故酸先入肝,苦先入心,甘先入脾,辛先入肺,咸先入肾。久而增气,物化之常也;气增而久,夭之由也。\\
帝曰:善。方制君臣何谓也?\\
岐伯曰:主病之谓君,佐君之谓臣,应臣之谓使,非上中下三品之谓也。\\
帝曰:三品何谓?\\
岐伯曰:所以明善恶之殊贯也。\\
帝曰:善。病之中外何如?\\
岐伯曰:调气之方,必别阴阳,定其中外,各守其乡,内者内治,外者外治,微者调之,其次平之,盛者夺之。汗之下之,寒热温凉,衰之以属,随其攸利。谨道如法,万举万全,气血正平,长有天命。\\
帝曰:善。\\
黄帝问道:五运之气交相配合,太过不及互相更替,这些道理我已经知道了。那么六气分时主治,其司天、在泉之气到来时所起的变化又怎样?\\
岐伯拜了两拜说:问得多么清楚啊!这是天地变化的基本规律,也是人体与天地变化相适应的规律。\\
黄帝问道:我希望听听它怎样能上合于昭明的天道,下合于玄远的地气?\\
岐伯说:这是医学理论中的主要部分,也是一般医生所不太理解的。\\
黄帝说:希望听听它的道理。\\
岐伯说:厥阴司天,气从风化;少阴司天,气从热化;太阴司天,气从湿化;少阳司天,气从火化;阳明司天,气从燥化;太阳司天,气从寒化。根据客气所临的脏位,来确定其疾病名称。\\
黄帝问:在泉的气化是怎样的?\\
岐伯说:与司天是同一规律,间气也都是如此。\\
黄帝问:什么叫间气呢?\\
岐伯说:分司在司天和在泉之左右的,就叫间气。\\
黄帝问:与司天在泉有何分别?\\
岐伯说:司天在泉主岁之气,主管一年的气化,间气,主一步(六十日)的气化。\\
黄帝说:很对。一岁的主气是怎样的呢?\\
岐伯说:厥阴司天为风化,在泉为酸化,岁运为苍化,间气为动化。少阴司天为热化,在泉为苦化,岁运不司为气化,间气为灼化。太阴司天为湿化,在泉为甘化,岁运为黅化,间气为柔化。少阳司天为火化,在泉为苦化,岁运为丹化,间气为明化。阳明司天为燥化,在泉为辛化,岁运为素化,间气为清化。太阳司天为寒化,在泉为咸化,岁运为玄化,间气为藏化。所以治病的医生,必须明白六气所司的气化,五味、五色的产生与五脏的宜忌,然后才可以对气化的虚实和疾病发生的关系找到头绪。\\
黄帝说:厥阴在泉而从酸化,我以前就知道了。风的气化运行又怎样呢?\\
岐伯说:风气行于地,这是本于地之气而为风化,其他五气也是这样。因为本属于天的,是天之气;本属于地的,是地之气。天地之气相互化合,就有六节之气的划分而后万物才能化生。所以说:要谨慎地察候六气适宜的时令,不可违反病机。说的就是这个意思。\\
黄帝问:那些主治疾病的药物怎样?\\
岐伯说:根据司岁之气来采备药物,就不会有遗漏了。\\
黄帝问:采备岁气所生化的药物,这是为什么?\\
岐伯说:因其能得到天地精专之气,有好的效果。\\
黄帝问:司岁运的药物怎样?\\
岐伯说:司岁运的药物与主岁的药物相同,然而有有余和不足的区别。\\
黄帝问:不属司岁之气生化的药物,是什么情况呢?\\
岐伯说:其气分散而不精专,所以形质虽然相同,却有等级品质的差别。具体说来,气味有厚与薄的分别,性能有动与静的差异,用量有多与少的不同,药力也有浅与深的区别。说的就是这个道理。\\
黄帝问:主岁之气,伤害五脏,怎样理解?\\
岐伯说:以脏气所不胜之气来说明,就是它的要领。\\
黄帝问:怎样治疗?\\
岐伯说:司天之气偏胜而淫于下的,以其所胜之气来治疗;在泉之气偏胜而淫于内的,以其所胜之气来治疗。\\
黄帝说:对。岁气平和之年而患病的,怎样治疗呢?\\
岐伯说:仔细观察三阴三阳司天在泉的病变所在,来调整,以恢复平和为目的。正病用正治法,反病用反治法,\\
黄帝说:夫子说观察阴阳之所在来调治,医经中说人迎和寸口脉相应,像牵引绳索一样大小相等的,叫做平脉。那么阴之所在,在寸口的脉象应该怎样?\\
岐伯说:看主岁的是南政还是北政,就可以推知了。\\
黄帝说:希望详尽听听。\\
岐伯说:北政之年,少阴在泉,则寸口脉沉伏而不应于指;厥阴在泉,则右寸脉沉伏,不应于指;太阴在泉,则左寸脉沉伏,不应于指。南政之年,少阴司天,则寸口脉沉伏,不应于指;厥阴司天,则右寸口脉沉伏,不应于指;太阴司天,则左寸口脉沉伏,不应于指。凡是寸口脉不应的,尺寸倒候或复其手就可以诊见了。\\
黄帝说:尺部的脉候怎样?\\
岐伯说:北政之年,三阴在泉,则寸口不应;三阴司天,则尺部不应。南政之年,三阴司天,则寸口不应;三阴在泉,则尺部不应。左右脉是相同的。所以说:能掌握其要领的,一句话就可以说完了,如果不知其要领,就会茫无头绪。说的就是这个道理。\\
黄帝说:很对。司天在泉之气,向内侵入人体而发病,是怎样的?\\
岐伯说:厥阴在泉之年,风气淫盛,则地气不明,原野昏暗,草类提早秀穗。人们多病洒洒然恶寒发冷,常常伸展肢体呵欠,心痛,感觉撑满,两胁里拘急不舒,饮食不下,胸膈咽喉不利,食入即吐,腹胀,嗳气,得大便或排气后好像病情退去一样轻快,全身沉重。\\
少阴在泉之年,热气淫盛,河川湖泽中阳气蒸腾,阴处反觉光明。人们多病腹中时时鸣响,逆气上冲胸脘,气喘,不能久立,恶寒发热,皮肤疼痛,视物模糊,齿痛,项肿,寒热往来如疟,少腹疼痛,腹部胀大。气候温热,蛰虫不伏藏。\\
太阴在泉之年,草类提早开花,湿气淫盛,则岩谷中昏暗浑浊,黄色见于水位,这是水湿与至阴土气相交合。人们多病痰饮积聚,心痛,耳聋,头目不清,咽喉肿胀,喉痹,阴病出血,少腹肿痛,小便不通,气逆上冲头痛,疼得眼如脱出,项部似拔,腰像拆断,髀骨不能转动,膝弯好似凝结了,小腿肚好像要裂开一样。\\
少阳在泉之年,火气淫盛,则郊野光焰明照,寒热交替而至。人们多病泄泻如注,下痢赤白,少腹疼痛,小便赤色,甚则血便。其余症候与少阴在泉之年相同。\\
阳明在泉之年,燥气淫盛,则雾气清冷,迷蒙昏暗。人们多病呕吐,呕吐苦水,经常长叹息,心胁部疼痛,不能转侧,甚至咽喉干燥,面如蒙尘,身体干枯,无光泽,足外侧反热。\\
太阳在泉之年,寒气淫盛,则天地间有凝肃惨栗之象。人们多病少腹疼痛,牵引睾丸、腰脊,向上冲心而痛,出血,咽喉疼痛,下巴肿。\\
黄帝说:对。怎样治疗呢?\\
岐伯说:凡是在泉之气,风气太过而侵入体内的,主治用辛凉药,辅佐用苦甘味药,用甘味来缓和肝木,用辛味来疏散风邪。热气太过而侵入体内的,主治用咸寒药,辅佐用甘苦药,以酸味收敛阴气,用苦药来发泄热邪。湿气太过而侵入体内的,主治用苦热药,辅佐用酸淡药,用苦味药来燥湿,用淡味药来渗泄湿邪。火气太过而侵入体内的,主治用咸冷药,辅佐用苦辛药,以酸味药收敛阴气,以苦味药发泄火邪。燥气太过而侵入体内的,主治用苦温药,辅助用甘辛药,以苦味泄下。寒气太过而侵入体内的,主治用甘热药,辅佐用苦辛药,用咸味来泻水寒,用辛味来温润,以苦味来巩固阳气。\\
黄帝说:对。司天之气的变化又怎样呢?\\
岐伯说:厥阴司天,风气淫胜,则天空尘埃满布,昏暗不清,云物为风鼓荡而扰动不宁,寒天行春令,流水不能结冰,蛰虫不潜伏。人们多病胃脘心口疼痛,上撑两胁,咽痛不通利,饮食不下,舌根强硬,食则呕吐,冷泻,腹胀,大便溏泄,气聚成瘕,小便不通,病根在脾脏。若冲阳脉绝,是不治的死证。\\
少阴司天,热气淫胜,则天气郁热,热极则大雨将至,君火行其政令。人们多病胸中烦热,咽喉干燥,右胁胠胀满,皮肤疼痛,恶寒发热,咳嗽喘促,唾血,便血,衄血,鼻塞流涕,喷嚏,呕吐,小便变色,甚则疮疡,浮肿,肩、背、臂、臑以及缺盆内疼痛,心痛,肺胀,腹部膨膨胀满,气喘咳嗽,病根在肺脏。若尺泽脉绝,是不治的死证。\\
太阴司天,湿气淫胜,则天空阴沉之气满布,雨水过多,反使草木枯槁。人们多病浮肿,骨痛阴痹。阴痹证见,按之不知痛处,腰脊头项疼痛,时时眩晕,大便困难,阳痿,饥饿却不欲进食,咳唾则有血,心悸不安,有如悬空,病根在肾脏。若太谿脉绝,是不治的死证。\\
少阳司天,火气淫胜,则温热之气流行,秋令失其清肃。人们多病头痛,发热恶寒如疟,热气在上,皮肤疼痛,尿色变为黄赤,传变为水病,身面浮肿,腹胀满,仰面呼吸,泄泻暴注,赤白下痢,疮疡,咳嗽吐血,心烦,胸中热,甚至鼻流涕、出血,病根在肺脏。若天府脉绝,是不治的死证。\\
阳明司天,燥气淫胜,则树木繁荣延迟,百草萌生较晚。人的筋骨发生病变,大凉之气改变了气候,树木枝叶枯敛,生发之气被抑于下,草的花叶焦枯,蛰虫反而外出活动。人们多病左胠胁疼痛,寒凉之气侵入,感而为疟,咳嗽,腹中鸣响,暴注泄泻,大便稀溏,心胁突然疼痛,不能转侧,咽喉干燥,面如蒙尘,腰痛,男子伒疝,妇女少腹疼痛,两目昏昧不明,眼角生疡,疮疡痈痤,病根在肝脏。若太冲脉绝,是不治的死证。\\
太阳司天,寒气淫胜,寒气非时而至,水结成冰,如遇戊癸火运炎烈,就有暴雨冰雹。人们多病血液变化于内,发生痈疡,厥逆心痛,呕血,便血,衄血,鼻塞流涕,易悲伤,时常眩晕仆倒,胸腹胀满,手热,肘臂挛急,腋部肿,心悸不安,胸胁胃脘不舒,面赤目黄,善嗳气,咽喉干燥,甚至面黑如炲,口渴欲饮,病根在心脏。若神门脉绝,是不治的死证。所以说,由脉气的搏动,可以测知其脏气的存亡。\\
黄帝说:对。怎样治疗呢?\\
岐伯说:司天之气,风气淫胜,治以辛凉药,佐以苦甘药,以甘味缓其急,以酸味泻其邪。热气淫胜,治以咸寒药,佐以苦甘药,以酸味药收敛阴气。湿气淫胜,治以苦热药,佐以酸辛药,以苦味药燥湿,以淡味药泄湿邪。湿邪甚于上部而有热,治以苦味温性之药,佐以甘辛药,以汗出复常而止。火气淫胜,治以咸冷药,佐以苦甘药,以酸味药收敛阴气,以苦味药发泄火邪,以酸味药复其真气。热淫所胜,与此相同。燥气淫胜,治以苦温药,佐以酸辛药,以苦味下其燥结。寒气淫胜,治以辛热药,佐以甘苦药,以咸味药泻其寒邪。\\
黄帝:对。邪气反胜所致之病,怎么治疗?\\
岐伯说:风气在泉,而清金之气反胜的,用酸温药治疗,辅佐以苦甘药,以辛味药平调之。热气在泉,而水寒之气反胜的,用甘热药治疗,辅佐以苦辛药,以咸味药平调之。湿气在泉,而火热之气反胜的,用苦冷药治疗,辅佐以咸甘药,以苦味药平调之。火气在泉,而水寒之气反胜的,用甘热药治疗,辅佐以苦辛药,以咸味之药平调之。燥气在泉,而火热之气反胜的,用平寒药治疗,辅佐以苦甘药,以酸味之药平调之,以药性平和为方制准则。寒气在泉,而火热之气反胜的,用咸冷药治疗,辅佐以甘辛药,以苦味药平调之。\\
黄帝问:司天之气被邪气反胜的病,怎么治疗?\\
岐伯说:风气司天而清凉之气反胜的,用酸温药治疗,辅佐以甘苦药。热气司天而寒水之气反胜的,用甘温药治疗,辅佐以苦酸辛药。湿气司天而热气反胜的,用苦寒药治疗,辅佐以苦酸药。火气司天而寒气反胜的,用甘热药治疗,辅佐以苦辛药。燥气司天而热气反胜的,用辛寒药治疗,辅佐以苦甘药。寒气司天而热气反胜的,用咸冷药治疗,辅佐以苦辛药。\\
黄帝问:六气相胜,情况怎样?\\
岐伯说:厥阴风气偏胜,证见耳鸣头眩,胃中烦乱欲吐,胃脘膈膜处寒冷,大风时起,倮虫不能孳生,人们多病胠胁气聚,化而成热,小便黄赤,胃脘当心处疼痛,向上支撑两胁,肠鸣飨泄,少腹疼痛,下痢赤白,病甚则呕吐,咽膈之间隔塞不通。\\
少阴热气偏胜,证见心下热,易饥饿,脐下有气动感,热气游走三焦。炎暑到来,树木流津,草类枯萎。人们病呕逆,烦躁,腹部胀满疼痛,大便溏泄,传变成为血痢。\\
太阴湿气偏胜,火气郁结于内,则成为疮疡,流散在外,则病生胠胁,甚则心痛,热气阻格在上部,发生头痛,喉痹,颈项强硬,单纯由于湿气偏胜而内郁,寒迫下焦,疼痛集中于头顶,牵引至眉间,胃中满闷。多次下雨之后,湿化开始出现,少腹胀满,腰臀沉重而强直,影响房事,时常泄泻如注,足下温暖,头部沉重,足胫浮肿,水饮发于内而上部出现浮肿。\\
少阳火气偏胜,热气侵入胃中,证见烦心,心痛,两目红赤,欲呕,呕酸,易饥,耳痛,小便红赤,易惊,谵妄,暴热之气消烁万物,草木萎枯,河水干涸,介虫屈伏不动,人们病少腹疼痛,下痢赤白。\\
阳明燥气偏胜,则清凉之气发于内,左胠胁疼痛,大便溏泄,内则咽喉窒塞,外为疝病。大凉之气肃杀,草木之花枯萎,毛虫类死亡,人们多病胸中不适,咽喉阻塞而咳嗽。\\
太阳寒气偏胜,凝肃凛冽之气就要到来,不该结冰而结冻,羽虫生化延迟。发为痔疮,疟疾,寒气入胃则生心痛,阴部生疮疡,房事不利,连及两股内侧,筋肉拘急麻木,血脉凝滞,络脉盈满变色,或便血,皮肤由气血痞塞而肿胀,腹中胀满,饮食减少,热气上逆,而头项巅顶脑户中疼痛,眼珠疼如脱出,寒气侵入下焦,传变为水泻。\\
黄帝问:怎么治疗?\\
岐伯说:厥阴风气偏胜致病,用甘清药治疗,辅佐以苦辛药,用酸味药泻其胜气。少阴热气偏胜致病,用辛寒药治疗,辅佐以苦咸药,用甘味药泻其胜气。太阴湿气偏胜致病,用咸热药治疗,辅佐以辛甘药,用苦味药泻其胜气。少阳火气偏胜致病,用辛寒药治疗,辅佐以甘咸药,用甘味药泻其胜气。阳明燥气偏胜致病,用酸温药治疗,辅佐以辛甘药,用苦味药泻其胜气。太阳寒气偏胜致病,用苦热药治疗,辅佐以辛酸药,用咸味药泻其胜气。\\
黄帝问:六气报复,发病情况怎样?\\
岐伯说:问得真详细啊!厥阴之复,病见少腹坚满,腹胁之内拘急暴痛。树木吹倒,沙尘飞扬,倮虫不能繁荣。发生厥心痛,汗出,呕吐,饮食不下,或食入即吐,筋骨抽痛,眩晕,手足厥冷,甚至风邪入脾,食入痹阻而吐出。如果冲阳脉绝,是不治的死证。\\
少阴火气之复,则烦热从内部发生,烦躁,鼻塞流涕,喷嚏,少腹绞痛,火现于外,身热如焚,咽喉干燥,大小便时泄时止,左腹部有动气感而且向上逆行到右侧,咳嗽,皮肤疼痛,突然失音,心痛,昏迷不省人事,继则洒淅恶寒,振栗战抖,谵语妄动,寒退而发热,渴欲饮水,少气,骨软痿弱,肠道梗阻,大便不通,肌肤浮肿,呃逆,嗳气。少阴火热之气后化,流水不能结冰,热气流行过度,介虫不潜伏,病多痱疹,疮疡,痈疽,痤,痔等外证,甚至热邪入肺,咳嗽,鼻渊。如果天府脉绝,是不治的死证。\\
太阴湿气之复,则湿气病变于是发生,身体沉重,胸腹满闷,饮食不消化,阴气上逆,胸中不快,水饮生于内,咳喘有声。大雨时常下降,鱼类游行于陆地。人们多病头顶疼痛而沉重,头部掉摇抽掣尤其严重,呕吐,神情默默,口吐清水,甚则湿邪入肾,泄泻不止。如果太谿脉绝,是不治的死证。\\
少阳热气之复,则大热将要到来,干燥灼热,介虫被消耗,病多惊恐瘈疭,咳嗽,衄血,心热烦躁,小便频数,恶风,逆气上行,面如蒙尘,眼睛因而圴动不宁,火气发于内,则上为口腔糜烂,呕逆,吐血,便血,发为疟疾,证见恶寒鼓慄,寒极转热,咽喉干槁,口渴欲饮,小便黄赤,少气,筋脉萎弱,气蒸热化形成水病,传变成为浮肿,甚则邪气入肺,咳嗽,便血。如果尺泽脉绝,是不治的死证。\\
阳明燥气之复,清肃之气流行,林中树木苍老干枯,兽类多发生疫病。人体的疾病多发生在胠胁,燥气偏聚左侧,喜长出气,甚则心痛痞满,腹胀而泄泻,呕吐苦水,咳嗽,呃逆,心烦,病在膈中,头痛,甚则邪气入肝,惊骇,筋挛。如果太冲脉绝,是不治的死证。\\
太阳寒气之复,则寒气上行,雨水凝结成冰雹,禽类因此死亡。人们多患胃生寒气,胸膈不舒,心痛痞满,头痛,易悲伤,时常眩晕仆倒,饮食减少,腰臀疼痛,屈伸不利,大地冻裂,冰厚而坚,阳光不能发挥温暖的作用,少腹痛牵引睾丸并连腰脊,逆气上冲于心口,唾出清水,及呃逆嗳气,甚则邪气入心,易忘易悲。如果神门脉绝,是不治的死证。\\
黄帝说:对。怎样治疗呢?\\
岐伯说:厥阴复气所致的病,用酸寒药治疗,辅佐以甘辛药,以酸味药泻其邪,以甘味药缓其急。少阴复气所致的病,用咸寒药治疗,辅佐以苦辛药,以甘味药泻其邪,以酸味药收敛,辛苦味药发散,以咸味药软坚。太阴复气所致的病,用苦热药治疗,辅佐以酸辛药,以苦味药泻其邪、燥湿、渗湿。少阳复气所致的病,用咸冷药治疗,辅佐以苦辛药,以咸味药软坚,以酸味药收敛,以辛苦味药发汗。发汗之药不必避讳热天,但不要用温凉之药。少阴复气所致的病,用发汗药时与此法相同。阳明复气所致的病,用辛温药治疗,辅佐以苦甘药,以苦味药渗泄,以苦味药攻下,以酸味药补虚。太阳复气所致的病,用咸热药治疗,辅佐以甘辛药,以苦味药坚其脆弱。治疗各种胜气复气所致之病,寒病用热药,热病用寒药,温病用凉药;凉病用温药,元气耗散的用收敛药,气机抑郁的用发散药,干燥的用滋润药,气急的用缓和药,坚硬的用柔软药,脆弱的用坚固药,衰弱的补虚,亢盛的泻邪。使用各种方法安定正气,使其清静安宁,这样病气就能衰退,余气各归其类属,没有偏胜之害。这是治疗上的基本方法。\\
黄帝说:对。人体之气有上下之分,是指什么说的?\\
岐伯说:身半以上,其气有三,是人身应天的部分,是由司天之气所主持的;身半以下,其气也有三,是人身应地的部分,是由在泉之气所主持的。用上下来指明它的胜气和复气,用六气来指明人身部位而说明疾病。“半”是指天枢而言。所以上部三气胜而下部三气都病的,以地气之名来命名人身受病的脏气;下部三气胜而上部三气都病的,以天气之名来命名人身受病的脏气。以上所说的是指胜气已经到来,而报复之气还处于屈伏未发的状态说的。如果报复之气已经到来,就不能以司天在泉之名来区别了,都应以复气的情况为准则。\\
黄帝问:胜复之气的变动,有一定的时间吗?气的来与不来有一定规律吗?\\
岐伯说:四时有一定的常位,而胜复之气的来与不来,却不是必然的。\\
黄帝说:希望听听其中的道理。\\
岐伯说:初之气至三之气,是司天之气所主,是胜气常见的时位。四之气到终之气,是在泉气之所主,是复气常见的时位。有胜气才有复气,没有胜气就没有复气。\\
黄帝说:对。复气已退去而胜气又到来的情况,是怎样的?\\
岐伯说:有胜气就会有复气,没有一定的规律,直到气衰才停止。复气衰退后又有胜气到来,如果没有复气发生,就会有灾害,这是万物生机被伤害的缘故。\\
黄帝问:复气至反而致病,是什么道理呢?\\
岐伯说:复气到来的时候,不是它时令的正位,与主时之气不相合。复气如果大复其胜气,则复气本身就虚,而反被主时之气所胜,因此反而致病。这是指火、燥、热三气来说的。\\
黄帝问:怎样治疗呢?\\
岐伯说:六气之胜所致之疾病,轻微的随顺之,严重的制止之。复气所致之疾病,和缓的平调之,暴烈的削弱之。都应根据胜气,来使其抑伏之气安定,不论用药次数多少,以达到和平为目的。这是治疗的法则。\\
黄帝说:对。客气与主气的胜复,怎祥?\\
岐伯说:客气与主气二者之间,只有胜没有复。\\
黄帝问:其逆与顺,怎样区别?\\
岐伯说:主气胜是逆,客气胜是顺,这是天道自然的常规。\\
黄帝问:其导致的疾病是怎样的?\\
岐伯说:厥阴司天,客气胜则耳鸣,振掉,眩晕,甚至咳嗽;主气胜则胸胁疼痛,舌强难以说话。少阴司天,客气胜则鼻塞流涕,喷嚏,颈项强硬,肩背部发热,头痛,少气,发热,耳聋,目昏,甚至浮肿,溢血,疮疡,咳嗽气喘;主气胜则心热烦躁,甚则胁痛,支撑胀满。\\
太阴司天,客气胜则头面浮肿,呼吸气喘;主气胜则胸腹满,食后精神昏乱。\\
少阳司天,客气胜则皮肤红疹外发,丹毒,疮疡,呕吐气逆,喉痹,头痛,咽喉肿,耳聋,血溢,内为瘈疭;主气胜则胸满,咳嗽仰息,甚至咳血,两手发热。\\
阳明司天,清气复胜而有余于内,则咳嗽,衄血,咽喉窒塞,胸膈内热,咳嗽不止,面白血出的就会死亡。\\
太阳司天,客气胜则胸闷不舒,流清涕,受寒就咳嗽;主气胜则咽喉鸣响。\\
厥阴在泉,客气胜则大关节不利,在内为痉挛强直拘急瘈疭,在外为活动不便;主气胜则病筋骨动摇强直,腰腹时常作痛。\\
少阴在泉,客气胜则腰痛,尻、股、膝、髀、腨、伄、足等部位发病,闷热而酸,浮肿不能久立,大小便改变;主气胜则逆气上行,心痛发热,膈内诸痹发作,病发于胠胁,汗多不止,四肢厥冷因之而起。\\
太阴在泉,客气胜则病足痿,下肢沉重,大小便不时排泄,湿入下焦,则发为濡泻以及浮肿、前阴病变;主气胜则寒气上逆而痞满,饮食不下,甚至发为疝病。\\
少阳在泉,客气胜则腰腹疼痛而反恶寒,甚至大小便色白;主气胜则热反上行而侵犯心胸,心痛,发热,格拒于中而呕吐。其他各种症候与少阴在泉所致者相同。\\
阳明在泉,客气胜则清寒之气扰动于下部,少腹坚满而频频腹泻;主气胜则腰重,腹痛,少腹生寒,大便溏泄,寒气逆于肠,上冲胸中,甚则气喘不能久立。\\
太阳在泉,寒气复胜而有余于内,则腰、尻疼痛,屈伸不便,股、胫、足、膝中疼痛。\\
黄帝说:对。怎样治疗呢?\\
岐伯说:上冲的抑之使下降,陷下的举之使上升,有余的折其盛势,不足的补其虚弱。用有利的药物辅助,用适宜的药食来调和。必须使主客之气安定,调适其寒温。客主之气相同的用反治法,相反的用从治法。\\
黄帝说:治寒病用热药,治热病用寒药。主客之气相同的用反治,相反的用从治。我已经知道了。五行补泻的正味是怎样的呢?\\
岐伯说:厥阴风木主位之时,其泻用酸味药,其补用辛味药。少阴君火与少阳相火主位之时,其泻用甘味药,其补用咸味药。太阴湿土主位之时,其泻用苦味药,其补用甘味药。阳明燥金主位之时,其泻用辛味药,其补用酸味药。太阳寒水主位之时,其泻用咸味药,其补用苦味药。\\
厥阴客气为病,补用辛味药,泻用酸味药,缓用甘味药。少阴客气为病,补用咸味药,泻用甘味药,收用酸味药。太阴客气为病,补用甘味药,泻用苦味药,缓用甘味药。少阳客气为病,补用咸味药,泻用甘味药,软坚用咸味药。阳明客气为病,补用酸味药,泻用辛味药,泄用苦味药。太阳客气为病,补用苦味药,泻用咸味药,坚用苦味药,润用辛味药。开发腠理,使津液和利,阳气通畅。\\
黄帝说:对。希望听听阴阳各分之为三,具体是指什么?\\
岐伯说:阴阳之气各有多少,其作用也各有不同。\\
黄帝问:为何称为阳明?\\
岐伯说:两阳相合而明,故称阳明。\\
黄帝问:为何称为厥阴?\\
岐伯说:两阴交尽,故称厥阴。\\
黄帝问:阴阳之气有多有少,疾病有盛有衰,治法有缓有急,处方有大有小,希望听听划分标准是什么?\\
岐伯说:邪气有高下之别,疾病有远近之分,症状有表里之异,治法有轻有重,总以药力达到病所为准则。《大要》说:君药一味,臣药二味,是奇方之制;君药二味,臣药四味,是偶方之制;君药二味,臣药三味,是奇方之制;君药二味,臣药六味,是偶方之制。所以说,病在近处用奇方,病在远处用偶方;发汗不用奇方,攻下不用偶方;补上部、治上部的方制宜缓,补下部、治下部的方制宜急;气味迅急的药物其味多厚,性缓的药物其味多薄。方制用药要恰到病处,说的就是这种情况。如果病所远,而在中途药力就已不足,就当考虑饭前或饭后服药,以使药力达到病所,不要违反这个规定。所以平调病气的方法是,病所近,不论用奇方或偶方,其制方服量要小;病所远,不论用奇方或偶方,其制方服量要大。方制大的,是药的味数少而量重;方制小的,是药的味数多而量轻。味数最多可至九味,味数最少仅用二味。用奇方而病不去,就用偶方,这叫做重方。用偶方而病仍不去,就用反佐之药来治疗,这就属于反用寒、热、温、凉的药来治疗了。\\
黄帝说:对。病生于本的,我已经知道了。生于标的怎样治疗?\\
岐伯说:懂得了与本病相反,就明白了病生于标;与治疗本病相反的方法,就是治疗标病的方法。\\
黄帝说:对。六气的胜气,怎样候察呢?\\
岐伯说:在胜气到来时候察。清气大来,是燥气之胜,风木受邪,肝病就发生了。热气大来,是火气之胜,燥金受邪,肺病就发生了。寒气大来,是水气之胜,火热受邪,心病就发生了。湿气大来,是土气之胜,寒水受邪,肾病就发生了。风气大来,是木气之胜,土湿受邪,脾病就发生了。这些都是所说的感受邪气而生病的。如果遇到运气不足之年,则邪气更甚。如主时之气不和,也是邪气更甚。遇月廓中空的时候,其邪也甚。重复感受邪气,其病就危险了。有胜气之后,必然会有复气。\\
黄帝问:六气到来时的脉象是怎样的?\\
岐伯说:厥阴之气到来,其脉为弦;少阴之气到来,其脉为钩;太阴之气到来,其脉为沉;少阳之气到来,其脉为大而浮;阳明之气到来,其脉为短而涩;太阳之气到来,其脉为大而长。气至而脉和缓的为正常,气至而脉应过甚的是病态,气至而脉相反的是病态,气至而脉不至的是病态,气未至而脉已至的是病态,阴阳更易而脉位交错的其病危重。\\
黄帝问:六气的标本,变化所从不同,情况怎样?\\
岐伯说:六气有从本化的,有从标本的,有不从标本的。\\
黄帝说:希望详细地听听。\\
岐伯说:少阳、太阴从本化,少阴、太阳既从本又从标,阳明、厥阴不从标本而从其中气。所以从本的病化生于本,从标本的病或化生于本,或化生于标,从中气的病化生于中气。\\
黄帝问:脉与病似相同而实相反的,怎样诊察呢?\\
岐伯说:脉至与证相一致,但按之鼓动无力,这不是真正的阳病,诸阳证都这样。\\
黄帝问:凡是阴证而相反的,其脉象怎样?\\
岐伯说:脉至与证相从,但鼓指却强盛有力,这不是真正的阴病。\\
所以各种疾病的发生,有生于本的,有生于标的,有生于中气的。治疗有治其本而痊愈的,有治其标而痊愈的,有治其中气而痊愈的,有治其标本而痊愈的,有逆治而痊愈的,有从治而痊愈的。逆,是逆病气而治,其实是顺治;顺,表面上是顺其病气而治,其实是逆治。所以说:知道了标与本,用之于临证就不会有困难,明白了逆与顺,就能正确治疗而不需询问他人。说的就是这个意思。不知道这些理论,就不足以谈论诊治,却足以扰乱经旨。所以《大要》说:技术低劣的医生,沾沾自喜,以为对什么病都搞明白了,认为是热证的,言语未了,而寒病又开始显现出来。不了解同是一气,而所生的病变不同,诊断迷惑,经旨错乱。就是这个道理。标本的理论,简要而广博,从小可及大,说明一个道理可以了解许多病的变化。所以懂得了标与本,就易于掌握而不会有损害;察明属本与属标,病气就可调和。明确胜复之气,就可以成为民众养生治病的准则。天地自然之道,就算彻底明白了。\\
黄帝问:胜气复气的变化,早晚是怎样?\\
岐伯说:大凡所胜之气,胜气到来就发病,等到病气积聚时,而复气已经萌动了。复气,是胜气结束时开始的,得其时位则加剧。胜气有轻重,复气就有多少。胜气和缓,复气也和缓,胜气虚,复气也虚。这是天道自然的常规。\\
黄帝问:胜复之气的发作,有时不当其时位,或后于时位,是什么原因?\\
岐伯说:因为气的发生和变化,有盛和衰的不同。寒暑温凉盛衰的作用,表现在辰戌丑未四季月之时。所以阳气发动,始于温而盛于暑;阴气发动,始于凉而盛于寒。春夏秋冬四季之间,有一定的时差。所以《大要》说:那春天的温暖,成为夏天的暑热,那秋天的肃杀,成为冬天的凛冽。谨慎体察四季月的变化,伺望气候的回归,这样,可以见到气的终结,又可以知道气的开始。说的就是这个意思。\\
黄帝问:四时之气候的时差有常数吗?\\
岐伯说:大多三十天。\\
黄帝问:其在脉象上的反应,怎样?\\
岐伯说:时差与正常相同时,等到其时过去而脉也就退去了。《脉要》说:春脉无沉象,夏脉无弦象,冬脉无涩象,秋脉无数象,是四时生气闭塞。沉而太过的是病脉,弦而太过的是病脉,涩而太过的是病脉,数而太过的是病脉;参差而见的是病脉,去而复见的是病脉;气未去而脉先去的是病脉,气去而脉不去的是病脉,脉与气相反的是死脉。所以说:气与脉之间相互持守,像权衡之器一样不可有所差失。阴阳之气,清静则生化就正常,扰动则疾病就发生。说的就是这个道理。\\
黄帝问:幽和明是什么意思?\\
岐伯说:太阴、少阴两阴交尽,叫做幽;太阳、少阳两阳合明,叫做明。幽和明配合阴阳,就有寒暑的不同。\\
黄帝问:分和至是什么意思?\\
岐伯说:气来叫做至,气分叫做分;气至之时其气相同,气分之时其气不同。所以春分秋分二分和夏至冬至二至,是天地正常气化纪时的纲纪。\\
黄帝说:夫子所说的春秋之气开始在前,冬夏之气开始于后,我已知道了。然而六气往复循环,主岁之时又不是固定不变的,补泻方法应该怎样?\\
岐伯说:根据司天、在泉之气所主之时,随其所宜,正确选用药味,这是治疗的关键。左右间气的治法也是这样。《大要》说:少阳主岁,先用甘药后用咸药;阳明主岁,先用辛药后用酸药;太阳主岁,先用咸药后用苦药;厥阴主岁,先用酸药后用辛药;少阴主岁,先用甘药后用咸药;太阴主岁,先用苦药后用甘药。辅佐以适宜的药物,资助其生化的源泉,这样就掌握了治疗六气之病的规律。\\
黄帝说:讲得好!大凡各种疾病,都由风、寒、暑、湿、燥、火六气的化与变而产生。医经中说,实证用泻法,虚证用补法。我把这些方法,教给医生,而医生使用后还不能达到十全的效果,我想使这些重要的理论得到普遍的运用,达到像桴鼓相应的效果,好像拔除芒刺、洗雪污浊一样,使医生能够达到工、巧、神、圣的程度,可以讲给我听吗?\\
岐伯说:仔细观察疾病的机理,不违背六气平和的原则,说的就是这种情况。\\
黄帝说:希望听听病机是什么?\\
岐伯说:凡是风病而发生的颤动眩晕,都属于肝。凡是寒病而发生的筋脉拘急,都属于肾。凡是气病而发生的烦满郁闷,都属于肺。凡是湿病而发生的浮肿胀满,都属于脾。凡是热病而发生的视物昏花,肢体抽搐,都属于火。凡是疼痛、搔痒、疮疡,都属于心。凡是厥逆、二便不通或失禁,都属于下焦。凡是痿弱患喘逆呕吐,都属于上焦。凡是口噤不开、寒战、口齿叩击,心神烦乱不安,都属于火。凡是痉病颈项强急,都属于湿。凡是气逆上冲,都属于火。凡是胀满腹大,都属于热。凡是躁动不安,发狂而举动失常的,都属于火。凡是突然发生强直的症状,都是属于风邪。凡是病而有声,在触诊时,发现如鼓音的,都属于热。凡是浮肿,疼痛、酸楚,惊骇不安,都属于火。凡是转筋挛急,排出的尿液浑浊,都属于热。凡是排出的尿液感觉清亮、寒冷,都属于寒。凡是呕吐酸水,或者突然急泄而有窘迫感的,都属于热。所以《大要》说:要谨慎地观察病机,了解各种症状的所属,有邪气要加以推求,没有邪气也要加以推求,如果是实证要看为什么实,如果是虚证要看为什么虚。一定得先分析五气中何气所胜,五脏中何脏受病,疏通其血气,使其调和畅达,而回归平和,说得就是这些道理。\\
黄帝说:说得好!药物五味阴阳的作用是怎样的?\\
岐伯说:辛、甘味的药物,其性发散,属于阳;酸、苦味的药物其性涌泄,属于阴;咸味的药物其性也是涌泄的,属于阴;淡味的药物其性是渗泄,也属于阳。这六种性味的药物有的收敛,有的发散,有的缓和,有的迅急,有的干燥,有的濡润,有的柔软,有的坚实,要根据它们的不同作用来使用,从而调和其气,归于平和。\\
黄帝问:病有不是调气所能治好的,应该怎样治疗?有毒的药和无毒的药,哪种先用,哪种后用?希望听听这里的规则。\\
岐伯说:用有毒或用无毒的药,以能治病为准则,根据病情来制定剂量的大小。\\
黄帝说:请您讲讲方制。\\
岐伯说:君药一味,臣药二味,这是小剂的组成;君药一味,臣药三味,佐药五味,这是中剂的组成;君药一味,臣药三味,佐药九味,这是大剂的组成。寒证,要用热药;热证,要用寒药。轻证,逆着病情来治疗;重证,顺着病情来治疗;病邪坚实的,就削弱它;病邪停留在体内的,就驱除它;病属劳倦所致的,就温养它;病属气血郁结的,就加以疏散;病邪滞留的,就加以攻逐;病属枯燥的,就加以滋润;病属急剧的,就加以缓解;病属气血耗散的,就加以收敛;病属虚损的,就加以补益;病属安逸停滞的,要使其畅通;病属惊怯的,要使其平静;或升或降,或用按摩,或用洗浴,或迫邪外出,或截邪发作,或用开泄,或用发散,都以适合病情为好。\\
黄帝问:什么叫做逆从?\\
岐伯说:“逆”就是正治法,“从”就是反治法,应用从治药,应多应少,要观察病情来确定。\\
黄帝问:反治是什么意思呢?\\
岐伯说:就是热因热用,寒因寒用,塞因塞用,通因通用。要制伏其主病,必先找出致病的原因。反治之法,开始时药性与病情之寒热似乎相同,但是它所得的结果却并不相同。可以用来破除积滞,可以用来消散坚块,可以用来调和气血,可以使疾病得到痊愈。\\
黄帝说:说得好!有六气调和而得病的,应怎样治?\\
岐伯说:或用逆治,或用从治,或主药逆治而佐药从治,或主药从治而佐药逆治,疏通气机,使之调和,这是治疗的正法。\\
黄帝说:说得好。病有内外相互影响的,怎样治疗?\\
岐伯说:病从内生而后发展于外的,应先调治其内;病从外生而后发展于内的,应先调治其外;病从内生,影响到外部而偏重于外部的,先调治它的内部,而后治其外部;病从外生,影响到内部而偏重于内部的,先调治它的外部,然后调治它的内部;既不从内,又不从外,内外没有联系的,就治疗它的主要病证。\\
黄帝说:说得好!火热之气来复,使人恶寒发热,好像疟疾的症状,有的一天一发,有的隔几天一发,这是什么缘故?\\
岐伯说:这是胜复之气相遇的时候有多有少的缘故。阴气多而阳气少,那么发作的间隔日数就长;阳气多而阴气少,那么发作的间隔日数就少。这是胜气与复气相互搏击,表现出的或盛或衰的规律。疟疾的道理也是这样。\\
黄帝说:前人的经论中曾说,治寒病用热药,治热病用寒药,医生不能废除这个准则而变更治则。但是有些热病服寒药而更热,有些寒病服热药而更寒,原来的寒热二证还在,又发生新病,应该怎样治呢?\\
岐伯说:各种用寒药而反热的,应该滋阴,用热药而反寒的,应该补阳,这就是求其属类的治法。\\
黄帝说:说得好。服寒药而反热,服热药而反寒,道理何在?\\
岐伯说:只治其偏亢之气,所以有相反的结果。\\
黄帝问:有的没有治偏亢之气也出现这种情况,是什么原因?\\
岐伯说:问得真详尽啊!这是不治偏嗜五味一类。五味入胃以后,各归其所喜的脏器,所以酸味先入肝,苦味先入心,甘味先入脾,辛味先入肺,咸味先入肾,积累日久,便能增加各脏之气,这是五味入胃后所起气化作用的一般规律。脏气增长日久而形成过胜这是导致病夭的原因。\\
黄帝说:说得好。制方有君臣的分别,是什么道理呢?\\
岐伯说:主治疾病的药味就是君,辅佐君药的就是臣,附应臣药的就是使,不是上中下三品的意思。\\
黄帝道:三品是什么意思?\\
岐伯说:所谓三品,是用来说明药性有毒无毒的。\\
黄帝说:说得好。疾病的内在外在都怎样治疗?\\
岐伯说:调治病气的方法,必须分别阴阳,确定在内在外,各依其病之所在,在内的治其内,在外的治其外,病轻的调理,较重的平治,病势盛的就攻夺。或用汗法,或用下法,要分辨病邪的寒、热、温、凉,根据病气的属性,使之消退,要随其所宜。谨慎地遵守如上的法则,就会万治万全,使气血平和,确保天年。\\
黄帝说:好。\\
卷二十三\\
著至教论篇第七十五\\
黄帝坐明堂,召雷公而问之曰:子知医之道乎?\\
雷公对曰:诵而颇能解,解而未能别,别而未能明,明而未能彰。足以治群僚,不足治侯王。愿得树天之度,四时阴阳合之,别星辰与日月光,以彰经术,后世益明。上通神农,著至教,疑于二皇。\\
帝曰:善!无失之,此皆阴阳、表里、上下、雌雄相输应也。而道,上知天文,下知地理,中知人事,可以长久。以教众庶,亦不疑殆。医道论篇,可传后世,可以为宝。\\
雷公曰:请受道,讽诵用解。\\
帝曰:子不闻《阴阳传》乎?\\
曰:不知。\\
曰:夫三阳天为业,上下无常,合而病至,偏害阴阳。\\
雷公曰:三阳莫当,请闻其解。\\
帝曰:三阳独至者,是三阳并至,并至如风雨,上为巅疾,下为漏病。外无期,内无正,不中经纪,诊无上下,以书别。\\
雷公曰:臣治疏愈,说意而已。\\
帝曰:三阳者,至阳也,积并则为惊,病起疾风,至如礔砺,九窍皆塞,阳气滂溢,干嗌喉塞,并于阴,则上下无常,薄为肠澼。此谓三阳直心,坐不得起,卧者便身全。三阳之病,且以知天下,何以别阴阳,应四时,合之五行。\\
雷公曰:阳言不别,阴言不理。请起受解,以为至道。\\
帝曰:子若受传,不知合至道,以惑师教,语子至道之要。病伤五脏,筋骨以消。子言不明不别,是世主学尽矣。肾且绝,惋惋日暮,从容不出,人事不殷。\\
黄帝坐在明堂上,召来雷公,问他说:你通晓医学道理吗?\\
雷公回答说:我诵读医书,但不能完全理解,有的理解了,还不能分析辨别,有的能够分析辨别,却不能够明白它的道理;有的明白了一些,然而在临证时也还不能自由运用。所以,我的医术,足以治疗一般同僚的疾病,却不能治疗王侯的病患。希望能够得到观察天运的尺度,结合四时阴阳,以辨清星辰日月的奥妙,从而彰明医道经术,使后世愈加发扬光大。能够与远古的神农潜通默契,彰明圣人的伟大教化,可以与伏羲、女娲二皇比美。\\
黄帝说:讲得好!不要忘记了,这些都是阴阳、表里、上下、雌雄相互联系,相互感应的道理。得道者,应该上通天文,下通地理,中通人事,医学才可以长久存在。用它来教化百姓,也不会产生疑惑和危险。把这些医学道理著于书籍,传于后世,是极其宝贵的。\\
雷公说:请把医学道理传给我,以便诵读和理解。\\
黄帝说:你听说过《阴阳传》这部书吗?\\
雷公说:不知道。\\
黄帝说:三阳之气护卫于人体之表,使人体能够适应天气变化,如果上下经脉之气运行失常,就会合而发病,伤害人体的阴阳。\\
雷公说:三阳莫当,是什么意思?请让我听听您的解释。\\
黄帝说:三阳独至,就是三阳之气合并而至,既然是合并而至,则其来时疾如风雨,逆上则形成头部疾病,陷下则为二便失禁。在外没有明显的征象可以预期,在内也没有确定的病机可据,不合乎诊断的纲领,无法确定其病在上在下,应根据《阴阳传》加以识别。\\
雷公说:我治疗这类病,很少能治愈的,对其道理也只是略知大意而已。\\
黄帝说:三阳是至盛之阳,三阳之气积聚在一起,就会发生惊骇,病起时快如疾风,病势猛烈如霹雳,九窍都闭塞不通,阳邪之气又盈溢泛滥,证见咽干喉塞,如果传入内脏,就会上下失常,下迫于肠,则生肠癖。这是所谓三阳之邪积并,直冲心膈,其病坐下不能起立,睡卧才觉得身体舒适。以上虽然说的是三阳之病,但从而可以了解天人相应的道理,以及如何来区别阴阳,顺应四时,符合五行的规律。\\
雷公说:上述这些道理,直截了当地讲,我还不能分别;隐约委婉地讲,就更不能领会了。请让我站立聆听您的讲解,以便领会这至深的道理。\\
黄帝说:你虽然接受了老师的传授,但是,如果不知道把师说与至道结合起来,就会对老师所教的产生疑惑,告诉你至道的要领。若病邪伤及五脏,筋骨就会日渐消损。像你所说的那样不能理解,不能辨别,这个世界上主治疾病的医学至道就要失传了。例如肾脉将绝,就会出现心中烦闷不安,日落时更严重,喜欢静处,不想出门,没精神应酬人事。\\
示从容论篇第七十六\\
黄帝燕坐,召雷公而问之曰:汝受术诵书者,若能览观杂学,及于比类,通合道理,为余言子所长。五脏六府,胆胃大小肠脾胞膀胱,脑髓涕唾,哭泣悲哀,水所从行,此皆人之所生。治之过失,子务明之。可以十全,即不能知,为世所怨。\\
雷公曰:臣请诵《脉经》上下篇,甚众多矣,别异比类,犹未能以十全,又安足以明之。\\
帝曰:子别试通五脏之过,六腑之所不和,针石之败,毒药所宜,汤液滋味,具言其状,悉言以对,请问不知。\\
雷公曰:肝虚肾虚脾虚,皆令人体重烦冤,当投毒药刺灸,砭石汤液,或已或不已,愿闻其解。\\
帝曰:公何年之长而问之少!余真问以自谬也。吾问子窈冥,子言《上下篇》以对,何也?夫脾虚浮似肺,肾小浮似脾,肝急沉散似肾,此皆工之所时乱也,然从容得之。若夫三脏土木水参居,此童子之所知,问之何也?\\
雷公曰:于此有人,头痛筋挛骨重,怯然少气,哕噫腹满,时惊,不嗜卧,此何脏之发也?脉浮而弦,切之石坚,不知其解,复问所以三脏者,以知其比类也。\\
帝曰:夫从容之谓也。夫年长则求之于腑;年少则求之于经;年壮则求之于脏。今子所言皆失。八风菀热,五脏消烁,传邪相受。夫浮而弦者,是肾不足也。沉而石者,是肾气内著也。怯然少气者,是水道不行,形气消索也。咳嗽烦冤者,是肾气之逆也。一人之气,病在一脏也,若言三脏俱行,不在法也。\\
雷公曰:于此有人,四肢解墯,喘咳血泄。而愚诊之,以为伤肺。切脉浮大而虚,愚不敢治。粗工下砭石,病愈,多出血,血止身轻。此何物也?\\
帝曰:子所能治,知亦众多,与此病失矣。譬以鸿飞,亦冲于天。夫圣人之治病,循法守度,援物比类,化之冥冥,循上及下,何必守经?今夫脉浮大虚者,是脾气之外绝,去胃,外归阳明也。夫二火不胜三水,是以脉乱而无常也。四肢解墯,此脾精之不行也。喘咳者,是水气并阳明也。血泄者,脉急,血无所行也。\\
若夫以为伤肺者,由失以狂也,不引比类,是知不明也。夫伤肺者,脾气不守,胃气不清,经气不为使,真脏坏决,经脉傍绝,五脏漏泄,不衄则呕,此二者不相类也。譬如天之无形,地之无理,白与黑相去远矣。是失,吾过矣,以子知之,故不告子。明引比类、从容,是以名曰诊经,是谓至道也。\\
黄帝安闲地坐着,召来雷公,问他说:你接受医术,诵读医书,好像已能博览群书,懂得取象比类,把医学道理融汇贯通了,请给我讲讲你的专长。如五脏六腑,胆、胃、大小肠、脾、胞、膀胱,脑、髓、涕、唾,哭泣、悲哀,水液的运行,这些都是人体之所赖以生存的。治疗时容易产生失误,你必须明白这些道理,才能有十全的疗效,如不能明白,就会为世人所抱怨。\\
雷公说:我读了《脉经》上、下篇很多内容,但对于区别异同,取象比类,掌握得还不十分透彻,又怎么能完全明白呢?\\
黄帝说:除《脉经》上下篇之外,根据你所通晓的,论述一下五脏的病变,六腑的不和,针石的主治,毒药的适宜,汤液的滋味,都详尽地叙述它们的情况,详尽地回答,对不知道的,请提出来。\\
雷公说:肝虚、肾虚、脾虚,都能使人身体沉重而郁闷烦乱,应该给予药物、刺灸、砭石、汤液,有的治好了,有的无效,希望听听这个问题的解释。\\
黄帝说:你这么大的年纪,怎么问如此幼稚的问题!也可能我提的问题不太适当。我问的是深奥的医理,而你只用“《脉经》上下篇”的话来回答,这是为什么?脾脉虚浮如肺脉,肾脉小浮像脾脉,肝脉急沉而散似肾脉,这些都是一般医工时常搞乱的,然而如能从容沉着,细致分析,还是可以辨别的。脾土肝木肾水三脏,部位相近,都在膈下,这些问题,小孩子都知道,你为什么还要问呢?\\
雷公说:这里有位病人,证见头痛,筋脉拘挛,骨节沉重,怯弱少气,呃逆嗳气,腹部胀满,时常惊恐,不易入睡,这是什么脏的病变?其脉浮取则弦,重按则坚硬如石,我不理解其中的道理,因而再问三脏,借以知道应怎样比类。\\
黄帝说:比类就是说诊病时要从容不迫。年老的人,应从六腑探求;年轻的人,应从经络探求;年壮的人,应从五脏探求。现在你仅从三脏之脉来言,那就错了。八风蕴结化热,五脏就会消烁,外邪内传,脏腑相受。脉浮取而弦,是肾气不足。沉取而坚,是肾气内著不行。怯弱少气,是水液不能输布,以致形气消散。咳嗽、郁闷烦乱,是肾气上逆所致。这位病人的病状,其病变在于肾脏,如果认为肝脾肾三脏俱病,是不合医经法度的。\\
雷公问:这里有一位病人,四肢懈怠无力,喘息咳嗽,便血,我去诊断,以为是伤肺。切其脉浮大而虚,我不敢治疗。有个粗率的医生用砭石治疗,病治好了,病人出了很多血,血止后全身轻快,这是什么病呢?\\
黄帝说:你所能治疗的和知道的病,已经很多了,但是就此病来说,你错了。譬如鸿雁,有时亦能飞上天空,那个粗率的医生不过是偶然幸中而已。圣人治病,要遵循法度,引物比类,并将规矩法度与变化多端的病情结合起来,察上可以及下,何必拘于经脉呢?病人脉浮大而虚,是脾气外绝,不能为胃行其津液,以致津液独归于阳明经。二火不能制胜三水,所以脉乱无常。四肢懈惰无力,是脾精不能输布的缘故。喘息咳嗽,是水气并走阳明所致。便血,是经脉拘急,血不行于经的缘故。\\
如果认为是伤肺,失误在于太随意了,是错误的诊断。不引物比类,所以认识不明确。如果是伤肺,则脾气不能内守,胃气不清,肺经失去功能,肺脏虚损败坏,经脉不能布散精气,五脏精气漏泄,不是衄血,便是呕血,这是伤脾与伤肺两病不同之处。这就好比天是无形的,地是无际的,又好比白与黑,相差太大了。你在诊断上的错误,也是我的过错,我以为你已经知道了,所以没有告诉你。以后要懂得引物比类、从容分析的法则,这是诊治的根据,是最高明的道理。\\
疏五过论篇第七十七\\
黄帝曰:呜乎远哉!闵闵乎若视深渊,若迎浮云。视深渊尚可测,迎浮云莫知其际。圣人之术,为万民式,论裁志意,必有法则。循经守数,按循医事,为万民副。故事有五过,汝知之乎?\\
雷公避席再拜曰:臣年幼小,蒙愚以惑,不闻五过,比类形名,虚引其经,心无所对。\\
帝曰:凡诊病者,必问尝贵后贱,虽不中邪,病从内生,名曰脱营。尝富后贫,名曰失精。五气留连,病有所并。医工诊之,不在脏腑,不变躯形,诊之而疑,不知病名。身体日减,气虚无精,病深无气,洒洒然时惊。病深者,以其外耗于卫,内夺于荣。良工所失,不知病情。此亦治之一过也。\\
凡欲诊病者,必问饮食居处。暴乐暴苦,始乐后苦,皆伤精气,精气竭绝,形体毁沮。暴怒伤阴,暴喜伤阳,厥气上行,满脉去形。愚医治之,不知补泻,不知病情,精华日脱,邪气乃并。此治之二过也。\\
善为脉者,必以比类、奇恒、从容知之。为工而不知道,此诊之不足贵,此治之三过也。\\
诊有三常,必问贵贱。封君败伤,及欲侯王。故贵脱势,虽不中邪,精神内伤,身必败亡。始富后贫,虽不伤邪,皮焦筋屈,痿躄为挛。医不能严,不能动神,外为柔弱,乱至失常,病不能移,则医事不行。此治之四过也。\\
凡诊者,必知终始,有知余绪。切脉问名,当合男女,离绝菀结,忧恐喜怒。五脏空虚,血气离守。工不能知,何术之语。尝富大伤,斩筋绝脉,身体复行,令泽不息,故伤败结,留薄归阳,脓积寒炅。粗工治之,亟刺阴阳,身体解散,四肢转筋,死日有期。医不能明,不问所发,唯言死日,亦为粗工。此治之五过也。\\
凡此五者,皆受术不通,人事不明也。故曰:圣人之治病也,必知天地阴阳,四时经纪,五脏六腑,雌雄表里,刺灸砭石,毒药所主。从容人事,以明经道,贵贱贫富,各异品理,问年少长,勇怯之理,审于分部,知病本始,八正九候,诊必副矣。\\
治病之道,气内为宝,循求其理。求之不得,过在表里。守数据治,无失俞理。能行此术,终身不殆。不知俞理,五脏菀热,痈发六腑。诊病不审,是谓失常。谨守此治,与经相明。《上经》《下经》,揆度阴阳,奇恒五中,决以明堂,审于终始,可以横行。\\
黄帝道:哎呀,真是太深远了!深远得好像探视深渊,又好像面对空中浮云。深渊还可以测量,而浮云就很难知道它的尽头了。圣人的医术,是众人的典范,他讨论决定医学上的认识,必然有一定的法则。遵守常规和法则,依循医学的原则治疗疾病,才能给众人谋福利。所以在医事上面有五过的说法,你知道吗?\\
雷公离开座位再拜说:我年岁幼小,愚笨而又糊涂,不曾听到五过的说法,只能在疾病的表象和名称上进行比类,空洞地引用经文,而心里却无法对答。\\
黄帝道:凡是在诊病的时候,必须询问病人是否以前高贵而后来卑贱,那么虽然不中外邪,疾病也会从内而生,这种病叫“脱营”。如果是以前富裕而后来贫困而发病,这种病叫“失精”。这两种病都是由于情志不舒,五脏气血郁结,渐渐积累而成的。医生诊察时,疾病的部位不在脏腑,身躯也没有变化,所以诊断上发生疑惑,不知道是什么病。但病人身体却一天天消瘦,气虚精耗,等到病势加深,就会毫无气力,时时怕冷,时时惊恐。这种病会日渐加深,就是因为情志抑郁,在外耗损了卫气,在内劫夺了营血的关系。医生的失误,是不懂得病情,随便处理。这在诊治上是第一种过失。\\
凡是诊察病人,一定得问他饮食起居的情况。精神上有没有突然的欢乐,突然的痛苦,原生活安逸后来生活艰难,这些都能伤害精气,精气衰竭,形体毁坏。暴怒会损伤阴气,暴喜会损伤阳气。阴阳受伤,厥逆之气就会上行而经脉张满,形体羸瘦。愚笨的医生诊治时,不知道该补还是该泻,也不了解病情,以致病人脏腑精华一天天损耗,而邪气愈加盛实。这是诊治上的第二种过失。\\
善于诊脉的医生,必然能够别异比类,分析奇恒,从容细致地掌握疾病的变化规律。作为医生而不懂医道,那他的诊治就没有什么值得称许的了。这是诊治上的第三种过失。\\
诊病时,对于病人的贵贱、贫富、苦乐三种情况,必须先问清楚。比如原来的封君公侯,丧失原来的封土,以及想封侯称王而未成功。过去高贵后来失势,虽然不中外邪,而精神上先已受伤,身体一定要败坏,甚至死亡。如先是富有的人,一旦贫穷,虽没有外邪的伤害,也会发生皮毛枯焦,筋脉拘挛,成为痿躄的病。这种病人,医生如不能认真对待,去转变患者的精神状态,而仅是顺从病人之意,敷衍诊治,以致在治疗上丢掉法度,那么病患就不能去除,当然也就没有什么疗效了。这是诊治上的第四种过失。\\
凡是诊治疾病,必须了解疾病的全部过程,同时还要察本而能知末。在切脉问证的时候,应注意到男女性别的不同,以及生离死别,情怀郁结,忧愁恐惧喜怒等因素。这些都能使五脏空虚,血气难以持守。如果医生不知道这些,还谈什么治疗技术。比如有人曾经富有,一旦失去财势,身心备受打击,以致筋脉的营养断绝,虽然身体还能行动,但津液不能滋生,过去形体的旧伤疼被引发,血气内结,迫于阳分,日久成脓,发生寒热。粗率的医生治疗时,多次刺其阴阳经脉,使病人的身体日见消瘦,难于行动,四肢拘挛转筋,死期已经不远了。而医生不能明辨,不问发病原因,只能说出哪一天会死,这也是粗率的医生。这是诊治上的第五种过失。\\
以上所说的五种过失,都是由于所学医术不精深,又不懂得贵贱、贫富、苦乐人事的缘故啊!所以说:高明的医生治病,必须知道天地阴阳,四时经络,五脏六腑的相互关系,经脉的阴阳表里,刺灸、砭石、毒药所治疗的主要病证。联系人事的变迁,掌握诊治的常规。贵贱贫富及各自不同的体质,询问年龄的少长,分析个性的勇怯,再审察疾病的所属部分,就可以知道疾病的根本原因;然后参考八正的时节,九候的脉象,那么诊治就一定精确了。\\
治病的关键,在于深察病人元气的强弱,来寻求邪正变化的机理。假如不能切中,那么过失就在于对表里关系的认识了。治疗时,应该守数据治,不要搞错取穴的理法。能这样进行治疗,可以一生不发生医疗过错。若不知取穴的理法,妄施刺灸,就会使五脏郁热,六腑发生痈疡。诊病不能审慎,叫做失去常规。谨守常规来治疗,自然就与经旨相合了。《上经》、《下经》二书,都是研究揆度阴阳奇恒之道的,五脏之病,表现于气色,取决于颜色,能从望诊上了解病的终始,可以无往而不胜。\\
徵四失论篇第七十八\\
黄帝在明堂,雷公侍坐。黄帝曰:夫子所通书受事,众多矣。试言得失之意,所以得之,所以失之。\\
雷公对曰:循经受业,皆言十全,其时有过失者,请闻其事解也。\\
帝曰:子年少智未及邪?将言以杂合耶?夫经脉十二,络脉三百六十五,此皆人之所明知,工之所循用也。所以不十全者,精神不专,志意不理,外内相失,故时疑殆。诊不知阴阳逆从之理。此治之一失也。\\
受师不卒,妄作杂术,谬言为道,更名自功,妄用砭石,后遗身咎。此治之二失也。\\
不适贫富贵贱之居,坐之薄厚,形之寒温,不适饮食之宜,不别人之勇怯,不知比类,足以自乱,不足以自明。此治之三失也。\\
诊病不问其始,忧患饮食之失节,起居之过度,或伤于毒?不先言此,卒持寸口,何病能中?妄言作名,为粗所穷。此治之四失也。\\
是以世人之语者,驰千里之外,不明尺寸之论,诊无人事。治数之道,从容之葆,坐持寸口,诊不中五脉,百病所起,始以自怨,遗师其咎。是故治不能循理,弃术于市,妄治时愈,愚心自得。呜呼!窈窈冥冥,孰知其道?道之大者,拟于天地,配于四海,汝不知道之谕,受以明为晦。\\
黄帝坐在明堂里,雷公在一旁侍坐。黄帝说:你研读医书接受医业已经很多了,试谈谈对治病成功失败的看法,治愈的原因,没有治愈的原因。\\
雷公回答说:我在研习医经接受医业当中,听说可以得到十全的疗效,但常常还是没有治好的,希望听听其中的说法。\\
黄帝道:你是因为年轻智力不够呢,还是由于杂合各家学说,缺乏一以贯之的独立见解呢?十二经脉,三百六十五络脉,这是人人都明白了解的,也是医工们所遵循使用的。之所以不能得到十全的疗效,是由于精神不能集中,思想上不加分析,又不能把外在的症状和内在的病机结合起来,因此时常产生疑问和困难。在诊治上,不懂得阴阳逆从的道理。这是治疗工作中的第一个失败原因。\\
从师学习尚未毕业,就胡乱地搞起庞杂的疗法,还荒谬地说是真理,或窃取别人成果而冠以己名,乱用砭石,结果给自己造成了罪过。这是治疗工作中第二个失败原因。\\
不理解贫富贵贱的状况,居处环境的好坏,形体的寒温,不理解适宜的饮食,不能区别性格的勇怯,不知道取象比类的分析方法。像这样,足以搞乱自己的头脑,而不能有清楚的认识。这是治疗工作中第三个失败原因。\\
诊断疾病,不问发病的原因,是由于精神刺激,饮食不节制,生活起居违背常规,还是由于中毒?不先把这些问题搞清楚,就贸然诊察病人的脉息,怎能诊断出什么病呢?信口胡说,编造病名,就会因技术低劣,而陷于困境。这是治疗工作中第四个失败原因。\\
有些医生说起话来,夸大到千里之外,却不明白尺寸诊法,论治疾病,也不考虑人事。诊病技术的原则,医生的从容和缓是最宝贵的,仅知诊察寸口,不能精确地诊察五脏之脉,就不知道百病发生的原因。医疗上出了问题,开始自怨所学不精,继则归罪于老师教得不好。所以治病如果不能遵循医学道理,就不会为人所信任,任意乱治,偶尔有治好的,就夸耀己功。唉!医学的道理是微妙高深的,有谁能够了解其中的道理?医学理论的远大,能和天地相比,能和四海相配,你不了解医理,即使名师传授明白的道理,也依然糊涂。\\
卷二十四\\
阴阳类论篇第七十九\\
孟春始至,黄帝燕坐,临观八极,正八风之气,而问雷公,曰:阴阳之类,经脉之道,五中所主,何脏最贵?\\
雷公对曰:春,甲乙青,中主肝,治七十二日,是脉之主时,臣以其脏最贵。\\
帝曰:却念《上下经》,阴阳从容,子所言贵,最其下也。\\
雷公致斋七日,旦复侍坐。\\
帝曰:三阳为经,二阳为维,一阳为游部,此知五脏终始。三阴为表,二阴为里,一阴至绝作朔晦,却具合以正其理。\\
雷公曰:受业未能明。\\
帝曰:所谓三阳者,太阳为经,三阳脉至手太阴,弦浮而不沉,决以度,察以心,合之阴阳之论。所谓二阳者,阳明也,至手太阴,弦而沉急,不鼓炅至,以病皆死。一阳者,少阳也,至手太阴,上连人迎,弦急悬不绝,此少阳之病也,专阴则死。\\
三阴者,六经之所主也,交于太阴,伏鼓不浮,上空志心。二阴至肺,其气归膀胱,外连脾胃。一阴独至,经绝,气浮不鼓,钩而滑。\\
此六脉者,乍阴乍阳,交属相并,缪通五脏,合于阴阳。先至为主,后至为客。\\
雷公曰:臣悉尽意,受传经脉,颂得《从容》之道,以合《从容》,不知阴阳,不知雌雄。\\
帝曰:三阳为父,二阳为卫,一阳为纪;三阴为母,二阴为雌,一阴为独使。\\
二阳一阴,阳明主病,不胜一阴,脉软而动,九窍皆沉。三阳一阴,太阳脉胜,一阴不能止,内乱五脏,外为惊骇。二阴二阳,病在肺,少阴脉沉,胜肺伤脾,外伤四肢。二阴二阳皆交至,病在肾,骂詈妄行,巅疾为狂。二阴一阳,病出于肾,阴气客游于心脘,下空窍堤,闭塞不通,四支别离。一阴一阳代绝,此阴气至心,上下无常,出入不知,喉咽干燥,病在土脾。二阳三阴,至阴皆在,阴不过阳,阳气不能止阴,阴阳并绝,浮为血瘕,沉为脓胕。阴阳皆壮,下至阴阳。上合昭昭,下合冥冥,诊决死生之期,遂合岁首。\\
雷公曰:请问短期。\\
黄帝不应。\\
雷公复问。\\
黄帝曰:在经论中。\\
雷公曰:请闻短期。\\
黄帝曰:冬三月之病,病合于阳者,至春正月,脉有死征,皆归出春。冬三月之病,在理已尽,草与柳叶皆杀,春阴阳皆绝,期在孟春。春三月之病,曰阳杀,阴阳皆绝,期在草干。夏三月之病,至阴不过十日。阴阳交,期在溓水。秋三月之病,三阳俱起,不治自已。阴阳交合者,立不能坐,坐不能起。三阳独至,期在石水。二阴独至,期在盛水。\\
在立春这天,黄帝安闲地坐着,观看八方的远景,候察八风的方向,问雷公说:根据阴阳的分类方法和经脉理论,配合五脏主时,你认为哪一脏最贵?\\
雷公回答说:春季属甲乙木,其色青,五脏中主肝,肝旺于春季七十二日,也是肝脉主令之时,我认为肝脏为最贵。\\
黄帝说:我依据《上下经》,阴阳比类分析的理论,你认为最宝贵的,却是其中最贱下的。\\
雷公斋戒了七天,早晨又侍坐于黄帝旁边。\\
黄帝说:三阳为经纶,二阳为维系,一阳为游部,由此就可以知道五脏之气运行的终始。三阴为表,二阴为里,一阴为阴气的终结,又是阳气的开始,有如朔晦的交界,都符合于天地阴阳终始的道理。\\
雷公说:我接受了这一学说,但还不明白其道理。\\
黄帝说:所谓三阳是指太阳,其脉至于手太阴寸口,见弦浮不沉的脉象,应当根据常度来判断,用心体察,并参合阴阳理论,来确定预后。所谓二阳就是阳明,其脉至于手太阴寸口,见弦而沉急,鼓指无力,身大热时而有此病脉,都是死证。一阳就是少阳,其脉至于手太阴寸口,上连人迎,见弦急悬而不绝,这是少阳经的病脉,如见有阴而无阳的脉象,是死证。\\
三阴手太阴肺经为六经之主,其气交于太阴寸口,脉象沉伏鼓动不浮,为心志空虚之征。二阴是少阴,其脉至于肺,其气归于膀胱,外与脾胃相连。一阴是厥阴,其脉独至于太阴寸口,经气已绝,故脉气浮而不鼓,脉象如钩而滑。\\
以上六种脉象,忽阴忽阳,相互交错,会聚于寸口,都和五脏相通,与阴阳相合。这种脉象,先见于寸口的为主,后见于寸口的为客。\\
雷公说:我已经完全理解了您的意思,把您传授的经脉理论,自己诵读的《从容》之道,合乎您的《从容》之论,但还不明白其中阴阳雌雄的意义。\\
黄帝说:三阳如高尊之父,二阳如外卫,一阳如枢纽;三阴如养育之母,二阴如雌性之内守,一阴如外交使者,能交通阴阳。\\
二阳一阴是阳明主病,二阳不胜一阴,则阳明脉软而动,九窍之气都沉滞不利。三阳一阴为病,则太阳脉胜,一阴肝气不能制止太阳寒水之气,故内乱五脏,外发惊骇。二阴二阳则病在肺,少阴脉沉,少阴之气胜肺伤脾,在外伤及四肢。二阴与二阳交互为患,其病在肾,骂詈妄行,癫疾狂乱。二阴一阳,其病出于肾,阴气上逆于心胸胃脘,并使下部孔窍如被堤坝阻隔一样闭塞不通,四肢之间好像分离一样不能为用。一阴一阳为病,其脉代绝,这是厥阴之气上至于心,病所或在上或在下,而无定处,饮食无味,二便失司,咽喉干燥,病在脾土。二阳三阴为病,包括至阴脾土在内,阴气不能至于阳,阳气不能达于阴,阴阳隔绝,阳浮于外则内成血瘕,阴沉于里则外成脓肿。若阴阳之气都亢盛,则病变趋向于下,病在男女生殖器。上合天道,下合地理,决断病者死生日期,同时还要结合参考一年之中何气为首。\\
雷公说:请问如何判定疾病的死亡日期。\\
黄帝没有回答。\\
雷公又问。\\
黄帝说:在医经论中有说明。\\
雷公又说:请问如何判定疾病的死亡日期。\\
黄帝说:冬季三月的病,如病性属阳的,则春季正月见脉有死征,死期在春夏之交。冬季三月的病,根据天人之理本应痊愈了,可是草和柳叶都枯死了,到春天阴阳之气都竭绝了,那么其死期就在初春。春季三月的病,名为阳杀。阴阳之气都竭绝,死期在秋天草木枯干之时。夏季三月的病,若不痊愈,死期在至阴后不超过十日。若脉见阴阳交错,死期在初冬结薄冰之时。秋季三月的病,如三阳都有起色,不治也会自愈。如是阴阳交错合而为病,则立而不能坐,坐而不能起。如三阳脉独至,死期在冰结如石的严冬之时。二阴脉独至,则独阴无阳,死期在正月雨水节。\\
方盛衰论篇第八十\\
雷公请问:气之多少,何者为逆?何者为从。\\
黄帝答曰:阳从左,阴从右,老从上,少从下。是以春夏归阳为生,归秋冬为死。反之,则归秋冬为生。是以气多少,逆皆为厥。\\
问曰:有余者厥耶?\\
答曰:一上不下,寒厥到膝,少者秋冬死,老者秋冬生。气上不下,头痛巅疾,求阳不得,求阴不审,五部隔无征,若居旷野,若伏空室,绵绵乎属不满日。\\
是以少气之厥,令人妄梦,其极至迷。三阳绝,三阴微,是为少气。是以肺气虚,则使人梦见白物,见人斩血藉藉;得其时,则梦见兵战。肾气虚,则使人梦见舟舩溺人;得其时,则梦伏水中,若有畏恐。肝气虚,则梦见菌香生草;得其时,则梦伏树下,不敢起。心气虚,则梦救火阳物;得其时,则梦燔灼。脾气虚,则梦饮食不足;得其时,则梦筑垣盖屋。此皆五脏气虚,阳气有余,阴气不足,合之五诊,调之阴阳,以在经脉。\\
诊有十度,度人脉度、脏度、肉度、筋度、俞度。阴阳气尽,人病自具。脉动无常,散阴颇阳。脉脱不具,诊无常行。诊必上下,度民君卿。受师不卒,使术不明。不察逆从,是为妄行。持雌失雄,弃阴附阳。不知并合,诊故不明。传之后世,反论自章。\\
至阴虚,天气绝;至阳盛,地气不足。阴阳并交,至人之所行。阴阳并交者,阳气先至,阴气后至。是以圣人持诊之道,先后阴阳而持之,《奇恒之势》乃六十首,诊合微之事,追阴阳之变,章五中之情。其中之论,取虚实之要,定五度之事,知此乃足以诊。是以切阴不得阳,诊消亡。得阳不得阴,守学不湛,知左不知右,知右不知左,知上不知下,知先不知后,故治不久。知丑知善,知病知不病,知高知下,知坐知起,知行知止,用之有纪,诊道乃具,万世不殆。\\
起所有余,知所不足。度事上下,脉事因格。是以形弱气虚,死。形气有余,脉气不足,死。脉气有余,形气不足,生。是以诊有大方,坐起有常,出入有行,以转神明。必清必净,上观下观,司八正邪,别五中部,按脉动静,循尺滑涩,寒温之意,视其大小,合之病能,逆从以得,复知病名,诊可十全,不失人情。故诊之,或视息视意,故不失条理,道甚明察,故能长久。不知此道,失经绝理,亡言妄期,此谓失道。\\
雷公请问:气的盛衰,什么是逆?什么是顺?\\
黄帝回答说:阳气从左而右,阴气从右而左,老年之气从上而下,少年之气从下而上。因此春夏之病见阳证阳脉,归属于阳,为顺为生;若见阴证阴脉,归属秋冬之阴,为逆为死。反过来,秋冬之病见阴证阴脉,归属于阴,为顺为生。所以不论气盛或气衰,逆则都属厥。\\
雷公又问:气有余也能成厥吗?\\
黄帝回答说:阳气一上而不下,则足部厥冷之气至膝,少年在秋冬出现此证则死,而老年在秋冬见此证却可生。阳气上而不下,会发生头痛巅顶疾患,这种厥病,说它属阳,不见阳证,说它属阴,也不见阴证,五脏之气隔绝,没有显著征象可察,好像置身旷野,伏居空室,无所见闻,而病势绵绵一息,生命已不满一天了。\\
所以,气虚之厥,使人乱梦纷纭,厥逆至极,则梦多离奇迷乱。三阳之脉悬绝,三阴之脉细微,就是少气之候。所以肺气虚则梦见白色的东西,或梦见人被杀流血,尸体狼籍;当金旺之时,则梦见交兵作战。肾气虚则梦见舟船淹死人;当水旺之时,则梦见自己潜伏水中,好像有恐惧的事。肝气虚则梦见菌香草木;当木旺之时,则梦见自己潜伏树下不敢起来。心气虚则梦见救火和阳性事物;当火旺之时,则梦见大火燔烧。脾气虚则梦见饮食不足;在土旺之时,则梦见筑垣盖屋。这些都是五脏气虚,阳气有余,阴气不足所致。参合五脏见证,调其阴阳,审察十二经脉。\\
诊法有十度,可用来衡量病人。具体是脉度、脏度、肉度、筋度、腧度。诊察其阴阳虚实,对病情就可以有全面地了解。脉息之动失去常规,或偏阴,或偏阳,或搏动并不明显,所以诊法也没有固定的常规。诊时必须兼取上部的人迎和下部的趺阳,又必须考虑病人地位的高低,是普通百姓还是达官贵人。如果从师不能毕业,不明了医术,临证不能辨别顺逆,不是补阴伐阳,就是补阳耗阴。不知道综合上下内外,诊断就不会明确。这样的诊断方法,传给后人,错误的论断就会暴露出来。\\
至阴虚,则阳气绝而不降;至阳盛,则阴气微而不升。能使阴阳融合交通,这是高明医生的本事。阴阳之气融合交通,是阳气先至,阴气后至。所以高明医生治病,诊脉要掌握阴阳的先后,参考《奇恒之势》六十首,综合从各种细微诊察所得的情况,推究阴阳的变化,搞清楚五脏的病情,参合其中的原则和虚实的纲要,再用五度加以判断。知道了这些,才可以诊病。所以只诊察其阴分而不能了解其阳分,这没有达到诊察目的;只了解其阳分而不能了解其阴分,说明所学的医道还不精深。知其左而不知其右,知其右而不知其左,知其上而不知其下,知其先而不知其后,这种治疗就不能长久。既了解不好的,也要了解好的;既了解有病的,也要了解无病的;既了解高,也要了解下;既了解坐,也要了解起;既了解行,也要了解止。运用起来有纲纪,有条不紊,诊法才算全备,而永远不会有危险。\\
举其有余,就知道其不足。考虑到病人的上下各部,诊脉的原理就可因此而穷究。因此,形弱气虚的,主死;形气太盛,脉气不足的,也主死;脉气太盛,形气不足的,主生。所以诊病有一定的大法,医生应该坐起有准则,举动有规律,头脑灵活。而且一定要内心虚静地观察上下,分别四时八节,观察邪气中于五脏的何部;诊察脉象动静,循摸尺肤滑涩寒温的概况;视其大小便的变化,参合病态,从而知道是逆是顺,又知道了病名,这样诊察疾病,就可以十不失一,也不会违背人情。所以诊病的时候,察其呼吸、看其精神,都能不失去条理。医理极高明了,所以能长久取得疗效。不知道这些道理,违反医学原理,乱谈病情,乱下结论,这叫违反医道。\\
解精微论篇第八十一\\
黄帝在明堂,雷公请曰:臣受业传之,行教以经论,从容形法,阴阳刺灸,汤药所滋。行治有贤不肖,未必能十全。若先言悲哀喜怒,燥湿寒暑,阴阳妇女,请问其所以然者,卑贱富贵,人之形体所从,群下通使,临事以适道术,谨闻命矣。请问有毚愚仆漏之问,不在经者,欲闻其状。\\
帝曰:大矣。\\
公请问:哭泣而涕泪皆出者,若出而少涕,其故何也?\\
帝曰:在经有也。\\
复问:不知水所从生?涕所从出也?\\
帝曰:若问此者,无益于治也,工之所知,道之所生也。夫心者,五脏之专精也,目者,其窍也,华色者,其荣也。是以人有德也,则气和于目;有亡,忧知于色。是以悲哀则泣下,泣下水所由生。水宗者,积水也。积水者,至阴也。至阴者,肾之精也。宗精之水,所以不出者,是精持之也。辅之裹之,故水不行也。\\
夫水之精为志,火之精为神,水火相感,神志俱悲,是以目之水生也。故谚言曰:心悲名曰志悲。志与心精,共凑于目也,是以俱悲则神气传于心精,上不传于志而志独悲,故泣出也。泣涕者,脑也,脑者,阴也,髓者,骨之充也。故脑渗为涕。志者,骨之主也,是以水流而涕从之者,其行类也。夫涕之与泣者,譬如人之兄弟,急则俱死,生则俱生,其志以早悲,是以涕泣俱出而横行也。夫人涕泣俱出而相从者,所属之类也。\\
雷公曰:大矣。\\
请问:人哭泣而泪不出者,若出而少,涕不从之,何也?\\
帝曰:夫泣不出者,哭不悲也。不泣者,神不慈也。神不慈,则志不悲,阴阳相持,泣安能独来。夫志悲者,惋,惋则冲阴,冲阴则志去目,志去,则神不守精,精神去目,涕泣出也。且子独不诵不念夫经言乎,厥则目无所见。夫人厥则阳气并于上,阴气并于下。阳并于上,则火独光也;阴并于下,则足寒,足寒则胀也。夫一水不胜五火,故目眦盲。\\
是以冲风,泣下而不止。夫风之中目也,阳气内守于精,是火气燔目,故见风则泣下也。有以比之,夫火疾风生,乃能雨,此之类也。\\
黄帝在明堂里,雷公请问说:我接受了您传授的医道,再教给我的学生,教授的内容是经典理论,包括从容形法,阴阳刺灸,汤药所滋等内容。然而他们在临证时,因智有贤愚之别,所以未必能有十全之效。先告诉给他们,在临证时要注意病人的悲哀喜怒,气候的燥湿寒暑,以及阴阳妇女等方面的问题,等他们提问所以然的道理时,再给他们讲述卑贱富贵及人之形体的适从等,使他们通晓这些理论,把学到的道术运用到临证中,这些在过去我已经听您讲过了。现在我还有一些很浅陋的问题,在经典中没有答案,希望听您解释。\\
黄帝说:你谈的问题真是重要啊。\\
雷公请问说:有哭泣而泪涕皆出,或泪出而很少有鼻涕的,是什么道理?\\
黄帝说:在医经中有记载。\\
雷公又问:眼泪是怎样产生的?鼻涕是从哪里出来的?\\
黄帝说:你这些问题,对治疗没有什么意义,但也是医生应该知道的,因为它也是建立医学理论的基础。心为五脏之专精,两目是它的外窍,光华色泽是它的外荣。所以人有得意之事,则和悦之情显露于两目;而有所失意,则忧愁之情表现于面色。因此悲哀落泪,流下的泪是由水产生的。水的来源,是体内积聚的水液。积聚的水液,是至阴。所谓至阴,就是肾藏之精。来源于肾精的水液,平时所以不出,是由于肾精的持守。精能辅助、裹藏水,所以泪水不至外流。\\
水的精气是志,火的精气是神,水火相互交感,神志都感到悲哀,因而泪水就出来了。所以谚语说:心悲叫做志悲。因为肾志与心精,同时上凑于目,所以心肾俱悲,则神气传于心精,而不传于肾志,肾志独悲,水失去了精的约制,故而泪水就出来了。泣涕源于出脑,脑属阴,髓充于骨孔并且藏于脑,所以脑髓渗漏而成涕。肾志是骨之主,所以泪水出而鼻涕也随之而出,是因为涕泪是同类的关系。涕之与泪,好比兄弟,危难则同死,安乐则共生,若肾志先悲,则涕泪同出而横流。涕泪所以俱出而相随,是由于涕泪同属水类的缘故。\\
雷公说:道理真是博大!\\
雷公请问:有人哭泣而没有眼泪流出,或虽出而少,而且涕不随之而出的,是什么道理?\\
黄帝说:哭而没有眼泪,是内心上并不悲伤。不出眼泪,是心神没有被感动,神不感动,则志亦不悲,心神与肾志相持守而不能相互交感,眼泪怎么会独自流出呢?志悲,都是由于内心凄惨,凄惨之意冲动于脑,则肾志去目,肾志去目,则神不能保守精,精和神都离开了眼目,眼泪和鼻涕就出来了。再说你没有读过或没有记住医经中的话吗?气厥则眼目一无所见。当人在气厥之时,阳气积并于上部,阴气积并于下部。阳气积并于上,则上部亢热;阴气积并于下,则足冷,足冷则发胀。因为一水不胜五火,所以眼目就看不见了。\\
所以迎风就会流泪不止。因为风邪中于目,阳气内守于精,也就是火气燔目,所以遇到风吹就会流泪了。打比方说:火热急甚而风生,风生而有雨,就是这一类。\\
附 录\\
刺法论篇第七十二\\
黄帝问曰:升降不前,气交有变,即成暴郁,余已知之。如何预救生灵,可得却乎?\\
岐伯稽首再拜对曰:昭乎哉问!臣闻夫子言,既明天元,须穷刺法,可以折郁扶运,补弱全真,泻盛蠲余,令除斯苦。\\
帝曰:愿卒闻之。\\
岐伯曰:升之不前,即有甚凶也。木欲升而天柱窒抑之,木欲发郁,亦须待时,当刺足厥阴之井。火欲升而天蓬窒抑之,火欲发郁,亦须待时,君火相火同刺包络之荥。土欲升而天冲窒抑之,土欲发郁,亦须待时,当刺足太阴之俞。金欲升而天英窒抑之,金欲发郁,亦须待时,当刺手太阴之经。水欲升而天芮窒抑之,水欲发郁,亦须待时,当刺足少阴之合。\\
帝曰:升之不前,可以预备,愿闻其降,可以先防。\\
岐伯曰:既明其升,必达其降也。升降之道,皆可先治也。木欲降而地皛窒抑之,降而不入,抑之郁发,散而可得位,降而郁发,暴如天间之待时也。降而不下,郁可速矣,降可折其所胜也。当刺手太阴之所出,刺手阳明之所入。火欲降,而地玄窒抑之,降而不入,抑之郁发,散而可入,当折其所胜,可散其郁,当刺足少阴之所出,刺足太阳之所入。土欲降而地苍窒抑之,降而不下,抑之郁发,散而可入,当折其胜,可散其郁,当刺足厥阴之所出,刺足少阳之所入。金欲降而地彤窒抑之,降而不下,抑之郁发,散而可入,当折其胜,可散其郁,当刺心包络所出,刺手少阳所入也。水欲降而地阜窒抑之,降而不下,抑之郁发,散而可入,当折其胜,可散其郁,当刺足太阴之所出,刺足阳明之所入。\\
帝曰:五运之至有前后,与升降往来,有所承抑之,可得闻乎刺法?\\
岐伯曰:当取其化源也。是故太过取之,不及资之。太过取之,次抑其郁,取其运之化源,令折郁气;不及资之,以扶运气,以避虚邪也。资取之法,令出《密语》。\\
黄帝问曰:升降之刺,以知其要。愿闻司天未得迁正,使司化之失其常政,即万化之或其皆妄,然与民为病,可得先除,欲济群生,愿闻其说。\\
岐伯稽首再拜曰:悉乎哉问!言其至理,圣念慈悯,欲济群生,臣乃尽陈斯道,可申洞微。太阳复布,即厥阴不迁正,不迁正,气塞于上,当泻足厥阴之所流。厥阴复布,少阴不迁正,不迁正,即气塞于上,当刺心包络脉之所流。少阴复布,太阴不迁正,不迁正,即气留于上,当刺足太阴之所流。太阴复布,少阳不迁正,不迁正,则气塞未通,当刺手少阳之所流。少阳复布,则阳明不迁正,不迁正,则气未通上,当刺手太阴之所流。阳明复布,太阳不迁正,不迁正,则复塞其气,当刺足少阴之所流。\\
帝曰:迁正不前,以通其要。愿闻不退,欲折其余,无令过失,可得明乎?\\
岐伯曰:气过有余,复作布正,是名不退位也。使地气不得后化。新司天未可迁正,故复布化令如故也。巳亥之岁,天数有余,故厥阴不退位也,风行于上,木化布天,当刺足厥阴之所入。子午之岁,天数有余,故少阴不退位也,热行于上,火余化布天,当刺手厥阴之所入。丑未之岁,天数有余,故太阴不退位也,湿行于上,雨化布天,当刺足太阴之所入。寅申之岁,天数有余,故少阳不退位也,热行于上,火化布天,当刺手少阳之所入。卯酉之岁,天数有余,故阳明不退位也,金行于上,燥化布天,当刺手太阴之所入。辰戌之岁,天数有余,故太阳不退位也,寒行于上,凛水化布天,当刺足少阴之所入。故天地气逆,化成民病,以法刺之,预可平疴。\\
黄帝问曰:刚柔二干,失守其位,使天运之气皆虚乎?与民为病,可得平乎?\\
岐伯曰:深乎哉问!明其奥旨,天地迭移,三年化疫,是谓根之可见,必有逃门。\\
假令甲子,刚柔失守,刚未正,柔孤而有亏,时序不令,即音律非从,如此三年,变大疫也。详其微甚,察其浅深,欲至而可刺,刺之当先补肾俞,次三日,可刺足太阴之所注。又有下位己卯不至,而甲子孤立者,次三年作土疠,其法补泻,一如甲子同法也。其刺以毕,又不须夜行及远行,令七日洁,清净斋戒,所有自来。肾有久病者,可以寅时面向南,净神不乱思,闭气不息七遍,以引颈咽气顺之,如咽甚硬物,如此七遍后,饵舌下津令无数。\\
假令丙寅,刚柔失守,上刚干失守,下柔不可独主之,中水运非太过,不可执法而定之。布天有余,而失守上正,天地不合,即律吕音异,如此即天运失序,后三年变疫。详其微甚,差有大小,徐至即后三年,至甚即首三年,当先补心俞,次五日,可刺肾之所入。又有下位地甲子,辛巳柔不附刚,亦名失守,即地运皆虚,后三年变水疠,即刺法皆如此矣。其刺如毕,慎其大喜欲情于中,如不忌,即其气复散也,令静七日,心欲实,令少思。\\
假令庚辰,刚柔失守,上位失守,下位无合,乙庚金运,故非相招,布天未退,中运胜来,上下相错,谓之失守,姑洗林钟,商音不应也,如此则天运化易,三年变大疫。详其天数,差有微甚,微即微,三年至,甚即甚,三年至,当先补肝俞,次三日,可刺肺之所行。刺毕,可静神七日,慎勿大怒,怒必真气却散之。又或在下地甲子乙未失守者,即乙柔干,即上庚独治之,亦名失守者,即天运孤主之,三年变疠,名曰金疠,其至待时也。详其地数之等差,亦推其微甚,可知迟速耳。诸位乙庚失守,刺法同。肝欲平,即勿怒。\\
假令壬午,刚柔失守,上壬未迁正,下丁独然,即虽阳年,亏及不同,上下失守,相招其有期,差之微甚,各有其数也,律吕二角,失而不和,同音有日,微甚如见,三年大疫,当刺脾之俞,次三日:可刺肝之所出也。刺毕,静神七日,勿大醉歌乐,其气复散,又勿饱食,勿食生物,欲令脾实,气无滞饱,无久坐,食无太酸,无食一切生物,宜甘宜淡。又或地下甲子,丁酉失守其位,未得中司,即气不当位,下不与壬奉合者,亦名失守,非名合德,故柔不附刚,即地运不合,三年变疠。其刺法亦如木疫之法。\\
假令戊申,刚柔失守,戊癸虽火运,阳年不太过也,上失其刚,柔地独主,其气不正,故有邪干,迭移其位,差有浅深,欲至将合,音律先同,如此天运失时,三年之中,火疫至矣,当刺肺之俞。刺毕,静神七日,勿大悲伤也,悲伤即肺动,而真气复散也。人欲实肺者,要在息气也。又或地下甲子,癸亥失守者,即柔失守位也,即上失其刚也。即亦名戊癸不相合德者也,即运与地虚,后三年变疠,即名火疠。\\
是故立地五年,以明失守,以穷法刺,于是疫之与疠,即是上下刚柔之名也,穷归一体也。即刺疫法,只有五法,即总其诸位失守,故只归五行而统之也。\\
黄帝曰:余闻五疫之至,皆相染易,无问大小,病状相似,不施救疗,如何可得不相移易者?\\
岐伯曰:不相染者,正气存内,邪不可干;避其毒气,天牝从来,复得其往,气出于脑,即不邪干。气出于脑,即室先想心如日。欲将入于疫室,先想青气自肝而出,左行于东,化作林木;次想白气自肺而出,右行于西,化作戈甲;次想赤气自心而出,南行于上,化作焰明;次想黑气自肾而出,北行于下,化作水;次想黄气自脾而出,存于中央,化作土。五气护身之毕,以想头上如北斗之煌煌,然后可入于疫室。\\
又一法,于春分之日,日未出而吐之。又一法,于雨水日后,三浴以药泄汗。又一法,小金丹方:辰砂二两,水磨雄黄一两,叶子雌黄一两,紫金半两,同入合中,外固了,地一尺筑地实,不用炉,不须药制,用火二十斤煅之也,七日终,候冷七日取,次日出合子,埋药地中,七日取出,顺日研之三日,炼白沙蜜为丸,如梧桐子大。每日望东吸日华气一口,冰水下一丸,和气咽之。服十粒,无疫干也。\\
黄帝问曰:人虚即神游失守位,使鬼神外干,是致夭亡,何以全真?愿闻刺法。\\
岐伯稽首再拜曰:昭乎哉问!谓神移失守,虽在其体,然不致死,或有邪干,故令夭寿。只如厥阴失守,天以虚,人气肝虚,感天重虚,即魂游于上,邪干,厥大气,身温犹可刺之,刺其足少阳之所过,次刺肝之俞。人病心虚,又遇君相二火司天失守,感而三虚,遇火不及,黑尸鬼犯之,令人暴亡,可刺手少阳之所过,复刺心俞。人脾病,又遇太阴司天失守,感而三虚,又遇土不及,青尸鬼邪犯之于人,令人暴亡,可刺足阳明之所过,复刺脾之俞。人肺病,遇阳明司天失守,感而三虚,又遇金不及,有赤尸鬼犯人,令人暴亡,可刺手阳明之所过,复刺肺俞。人肾病,又遇太阳司天失守,感而三虚,又遇水运不及之年,有黄尸鬼干犯人正气,吸人神魂,致暴亡,可刺足太阳之所过,复刺肾俞。\\
黄帝问曰:十二脏之相使,神失位,使神彩之不圆,恐邪干犯,治之可刺?愿闻其要。\\
岐伯稽首再拜曰:悉乎哉,问至理,道真宗,此非圣帝,焉究斯源!是谓气神合道,契符上天。心者,君主之官,神明出焉,可刺手少阴之源。肺者,相傅之官,治节出焉,可刺手太阴之源。肝者,将军之官,谋虑出焉,可刺足厥阴之源。胆者,中正之官,决断出焉,可刺足少阳之源。膻中者,臣使之官,喜乐出焉,可刺心包络所流。脾为谏议之官,知周出焉,可刺脾之源。胃为仓廪之官,五味出焉,可刺胃之源。大肠者,传道之官,变化出焉,可刺大肠之源。小肠者,受盛之官,化物出焉,可刺小肠之源。肾者,作强之官,伎巧出焉,刺其肾之源。三焦者,决渎之官,水道出焉,刺三焦之源。膀胱者,州都之官,精液藏焉,气化则能出矣,刺膀胱之源。凡此十二官者,不得相失也。是故刺法有全神养真之旨,亦法有修真之道,非治疾也。故要修养和神也,道贵常存,补神固根,精气不散,神守不分,然即神守而虽不去,亦能全真,人神不守,非达至真。至真之要,在乎天玄,神守天息,复入本元,命曰归宗。\\
黄帝问说:应升而不能升,应降而不能降,气之升降交通异常,就要成为暴烈的郁气,这我已经知道了。怎样能预先拯救百姓,能使郁气退却呢?\\
岐伯行礼拜了两拜,回答说:问得真高明啊!我听老师说过,既懂得天元之道,还要穷究刺法,就可以折服郁气,使气运升降正常,补助虚弱而保全真气,泻其盛气以祛除余邪,便能消除这疾病之苦。\\
黄帝说:希望详细听听升降的道理。\\
岐伯说:应该升而不能升,就会有巨大的凶险。厥阴风木,应该从在泉右间,上升为司天左间,而在天的金气过胜阻滞压抑它,木要疏发被阻郁之气,但要等到它当位之时才能发病,可以刺足厥阴经的井穴大敦以疏泻木气之郁。少阴君火应该上升,而在天的水气过胜阻抑它,火要疏发被阻郁之气,也要等到它当位之时才能发病,君火相火都可以刺手厥阴心包络经的荥穴劳宫以疏泻火气之郁。太阴湿土应该上升,而在天的木气过胜阻抑它,土要疏发被阻郁之气,也要等到它当位之时才能发病,可以刺足太阴脾经的输穴太白以疏泻土气之郁。阳明燥金应该上升,而在天的火气过胜阻抑它,金要疏发被阻郁之气,也要等到它当位之时才能发病,可以刺手太阴经的经穴经渠以疏泻金气之郁。太阳寒水应该上升,而在天的土气过胜抑阻它,水要疏发被阻郁之气,也要等到它当位之时才能发病,可以刺足少阴经的合穴阴谷以疏泻水气之郁。\\
黄帝说:应升而不能升,既然可以预先防备,希望听听应降而不能降,事先防止的方法。\\
岐伯说:既然已经知道了升的道理,必然也可以通达其降的情况。升和降中出现的问题,都可以预先防治。厥阴风木应该从司天之右间下降到在泉之左间,而在地之金气阻窒压抑它,使它欲降而不得入,木受阻抑,必使郁滞之气疏发消散,才得降入在泉之左间的位置,应降而不能降所产生的郁滞,为害也和司天之间气应升而不能升需要等到复位之时才能消除。应该降而不能降,郁滞将很快形成,要使它下降,可以折服胜它的金气。当刺手太阴经的所出少商穴,刺手阳明经之所入曲池穴。少阴君火、少阳相火应该从司天右间下降为在泉左间,而在地的水气阻窒压抑它,使它欲降而不得入,火受阻抑,必使郁滞之气疏发消散,才得降入在泉之左间的位置,当折服胜它的水气,可以散火气之郁,当刺足少阴之所出涌泉穴,刺足太阳之所入委中穴。太阴湿土,应该从司天右间降入在泉左间,而在地的木气阻窒压抑,使它欲降而不得入,土受阻抑,必使郁滞之气疏发消散,才可降入在泉左间的位置,当折服胜它的木气,可散土气之郁,当刺足厥阴经之所出大敦穴,刺足少阳经之所入阳陵泉穴。阳明燥金应该从司天右间降入在泉左间,而在地的火气阻窒压抑,使它欲降而不得入,金被阻抑,必使郁滞之气疏发消散,才可降入在泉左间的位置,应该折服胜它的火气,以散金气之郁,当刺心包络手厥阴经之所出中冲穴,刺手少阳经之所入天井穴。太阳寒水应该从司天右间降入在泉左间,而在地的土气阻窒压抑,使它欲降而不得入,水受阻抑,必使郁滞之气疏发消散,才可降入在泉左间的位置,应该折服土气,就可以散水气之郁,当刺足太阴经之所出隐白穴,刺足阳明经之所入三里穴。\\
黄帝说:五运之气的到来,有先有后,它与天气的升降往来,必有承接或抑阻的关系,可以让我听听对此的刺法吗?\\
岐伯说:应当取治它气化的本源。所以气太过的要疏泻,气不及的要资补。太过用泻法,就是按照升降的次序,抑制其郁滞的发作,取法于五运气化的本源,来折服郁滞之气,不及的要资补,就是扶植运气,以避免虚邪。以上资补和疏泻的方法,出于《玄珠密语》一书。\\
黄帝问说:升降的刺法,已经知其大要。希望再听听关于司天未能迁正,而使气化政令失常,万物的生化失去正常规律而妄行,这样人们就要发病,能否预先消除它,以帮助人们,希望听听这个问题。\\
岐伯行礼拜了两拜,回答说:问得真详尽啊!说得很有道理,足见圣君胸怀仁慈怜悯之心,要拯救人们,我一定详尽地陈述这些道理,申明其深奥微妙的意义。如果上年司天的太阳寒水之气继续布政司令,厥阴风木就不能迁正,厥阴不能迁正,则木气郁塞于上部,应当泻足厥阴经的荥穴行间。上年厥阴风木继续布政司令,少阴君火就不能迁正,少阴不能迁正,则火气滞留于上部,应当刺手厥阴心包经的荥穴劳宫。上年少阴君火继续布政司令,太阴湿土就不能迁正,太阴不能迁正,则土气稽留于上部,应当刺足太阴脾经的荥穴大都。上年太阴湿土继续布政司令,少阳相火就不能迁正,少阳不能迁正,则火气阻塞不通,应当刺手少阳经的荥穴液门。上年少阳相火继续布政司令,阳明燥金就不能迁正,阳明不能迁正,则金气不能通于上部,应该刺手太阴经的荥穴鱼际。上年阳明燥金继续布政司令,太阳寒水就不能迁正,太阳不能迁正,则和上年主气的终气太阳寒水双重郁塞,应当刺足少阴经的荥穴然谷。\\
黄帝说:应该迁正而不能迁正,已经通晓它的要点了。希望再听听不退位的问题,要折服它的有余之气,不使它产生过失,能否说明呢?\\
岐伯说:上年司天之气太过有余,继续布政司令,这就叫不退位。使在泉的地气,也不能退居右间。新司天不能迁正,所以上年司天仍旧布政司令。如巳年亥年司天气数有余,所以到子年午年,厥阴风木不退位,风气仍行于上,木的生化之气仍然布散于天,应当刺足厥阴经的合穴曲泉。子年午年司天气数有余,所以到丑年未年,少阴君火不退位,热气仍行于上,火的余气仍然布散于天,应当刺手厥阴心包络经的合穴曲泽。丑年未年司天气数有余,所以到寅年申年,太阴湿土不退位,湿气仍行于上,雨湿之气仍然布散于天,应当刺足太阴脾经的合穴阴陵泉。寅年申年司天气数有余,所以到卯年酉年,少阳相火不退位,热气仍行于上,火热之气仍然布散于天,应当刺手少阳三焦经的合穴天井。卯年酉年司天气数有余,所以到辰年戌年,阳明燥金不退位,金气仍行于上,燥金之气仍然布散于天,应当刺手太阴肺经的合穴尺泽。辰年戌年司天气数有余,所以到巳年亥年,太阳寒水不退位,寒气仍行于上,凛冽的寒水之气仍然布散于天,应当刺足少阴肾经的合穴阴谷。所以司天在泉之气出现逆象,就会形成人们的疾病,按以上方法针刺,可以预先平定将发之病。\\
黄帝问:刚干和柔干失守,能使司天、在泉和中运之气都虚吗?给百姓造成的疾病,能否平定呢?\\
岐伯说:提的问题真深啊!必须明白它的奥妙含义,司天在泉之气逐年更迭迁移,三年左右可造成时疫流行,如果能够找到变化的根源,就必定有避免的方法门路。\\
假如甲子司天之年,刚柔失守,司天之气未能迁正,在泉之气就孤立而空虚,四时气候顺序不按节令到来,就像音律不能相应一样,这样三年之后,就要变成大疫。详尽审察其程度的微与甚、浅与深,在它将发之前,可以针刺预防,针刺时,应先补足太阳膀胱经的肾俞穴,过三天,再刺足太阴经之所注太白穴。又有在泉之气己卯不能迁正,而司天甲子孤立的,过三年之后,可发生土疠,补泻的方法,完全与甲子司天失守一样。针刺结束,不能夜行及远行,七日之内,命受针者洁净,精神清净,斋戒素食,使有所由来的疫邪不能乘虚袭人。肾有慢性久病的人,可以在寅时,面对南方,精神清静,排除乱念,以意念屏住气息,连续七次,伸颈用力咽气,使之顺下,像咽很硬的东西一样,这样七遍之后,再把舌下的津液咽下去,不拘其数。\\
假如丙寅司天之年,刚柔失守,司天之气未能迁正,在泉之气也不能独主其令,丙年虽属水运太过,但上下失守,就不是太过了,不能拘执定法,以太过论治。阳年司天虽属有余,但刚柔失守而不能迁正,天地上下就不能相合,正如律吕不协调而发音各异,这样自然界气候就失去正常时序,三年之后,就会变为疫疠。详细审察它程度的微与甚及差异的大与小,徐缓到来的就在三年后发生疾疠,急骤到来的头三年就发生疫疠,应当先补足太阳膀胱经的心俞穴,过五天,再刺肾经之所入阴谷穴。又有在泉之气,辛巳不能随着司天而迁正,也叫失守,就使在泉之气与运气都虚,三年之后变成水疠,刺法也同丙寅失守一样。针刺结束,当避免内心过喜极欲,如果不注意,就会使气再被耗散,让受刺者静养七天,内心要充实,减少思虑。\\
假如庚辰司天之年,刚柔失守,司天之气失守,在泉之气不能相合,乙庚是金运,刚柔失守,所以上下不相呼应,上年司天的阳明燥金未退,在泉之火胜今年中运之金,上下胜复相错,叫做失守,使太商阳律之姑洗与少商阴吕之林钟不能相应,这样天运变化异常,三年之后变为大疫。审察其天运的变化规律和相差的微与甚,凡相差轻微的疫情也轻微,三年之后发生,相差严重的疫情也严重,也是三年之后发生,应当先补足太阳膀胱经的肝俞穴,过三天,可刺肺经的所行经渠穴。针刺结束,要宁静精神七天,慎勿大怒,若发怒真气必然耗散。又或在泉之气,乙未不能迁正,就是说乙未失守,即下乙柔干不至,而上位庚辰独自司天,也叫失守,即司天与中运单独主治之年,三年之后,变生疫疠,叫做金疠,它的发生要等到一定时候。审察在泉之气变化的差等,也能推断病气的微与甚,可以知道发病的迟与速。凡是乙庚之年上下失守的,刺法都相同。肝喜欢平和,勿要发怒。\\
假如壬午司天之年,刚柔失守,属壬之司天不能迁正,属丁之在泉单独迁正,虽然是阳年,而阳年太过阴年不及的规律就不适用了,上位下位失守,总会有相应的时候,因为差异的微甚,各有一定之数,太角的阳律和少角的阴吕,相失而不和,待上下得位之时,则律吕之音相同有日,如见其微甚,三年之后,要有大疫流行,应当先刺足太阳膀胱经的脾俞穴补之,过三天,可再刺足厥阴肝经之所出大敦穴。针刺结束,宁静精神七天,不能大醉或歌舞取乐,否则正气又被耗散,又不能吃得太饱,不能吃生东西,要想使脾气充实,不致气机郁滞饱满,不能久坐,不要吃过酸之物,不要吃一切生东西,要吃甘淡的食物。又或在泉之气甲子,丁酉失守,未能迁正,就是运气不当位,在泉之气不能同司天之气相合,也叫做失守,不能称为合德,因柔不依附于刚,二者不相应,就是在泉之气与中运不合,三年之后,变为疫疠。刺法完全和壬午司天失守预防木疫一样。\\
假如戊申司天之年,刚柔失守,虽然戊癸年是火运阳年,如刚柔失守,那么阳年也不属太过了,司天刚干失守,在泉柔干独主,气候不正常,所以有邪气干扰,司天在泉之气位置更迭变移,相差的程度有浅与深,等到刚柔将合之时,阳律与阴吕必先应而同,如此天运失去正常时位,三年之内火疫就要发生,应先刺足太阳膀胱经的肺俞穴。针刺结束,宁静精神七天,不要太悲伤,悲伤肺气就会被扰动,而真气又被耗散。要想使肺充实,关键在于调息养气。又或在泉之气甲子,癸亥失守,就是柔干失守不能迁正,就是在泉之气不能上合司天之气。也就称为戊癸不相合德,使运气与在泉之气空虚,三年之后变为疫疠,就叫火疠。\\
因此运用五行来分立五年,以说明刚柔失守,以穷究针刺之法,于是可以知道疫和疠,就是从上下刚干柔干失守来定名的,疫与疠本性是一样的。就是针刺防疫法,只有五种方法,也就是汇总了诸刚柔之位失守的治法,所以只把这些归入五行来统摄治疗。\\
黄帝问说:我听说五疫到来,都相互传染,不论大人小孩,病状都是一样的,未病之前给予治疗,有什么方法能使人不相传染呢?\\
岐伯说:要使人们不相传染,一方面要正气充实于内,邪气就不能侵犯;另一方面要避免疫毒,使它从鼻孔而来,仍从鼻孔而去,所以只要正气出于脑,外邪就不能侵犯了。所谓正气出于脑,就是在进入病室之前,先想象心像太阳一样。将要进入病室时,先想象有一股青气从肝脏发出,向左运行到东方,化作繁茂的树林;其次想象有一股白气从肺脏发出,向右运行到西方,化作兵戈金甲;其次想象有一股红气从心脏发出,向上运行到南方,化作火焰光明;其次想象有一股黑气从肾脏发出,向下运行到北方,化作寒冷之水;再其次想象有一股黄气从脾脏发出,存留于中央,化作生化万物之土。有了五脏护身之气想毕,再想象头上如北斗星一样的煌煌之光,然后进入疫病之室。\\
又有一种方法,是在春分这一天,太阳未出之时,用吐法。又有一法,在雨水节后,用药汤沐浴三次,以药力出汗。又有一法,用小金丹方:辰砂二两,水磨雄黄一两,叶子雌黄一两,紫金半两,一同放在盒中,外面封固,在地上挖一尺深筑成地穴,不用炉子,也没有制法上的限定,只要燃料二十斤锻炼,到七天,等冷却,七天后拿出地穴,第二天从盒子中拿出来,把药埋在地中,再过七天拿出来,天天研,研三天,用熬过的白沙蜜做成梧桐子大的丸药。每天早晨面向东方,吸日华之气一口,再用冰水送服一丸,连气一同咽下去。服十粒,就没有疫邪侵袭了。\\
黄帝问说:人体虚弱精神运行就失其本位,使邪气自外侵袭,这样致人夭亡,怎么能保全真气?希望听听救治的刺法。\\
岐伯行礼,拜了两拜回答说:问得真高明啊!精神游离失守,虽然身体上有所表现,然而不致于死亡,如果再有外邪侵袭,便能使它夭折寿命。例如厥阴风木司天失守,而天运空虚,若人的肝气也虚,人体之虚,再感受天之虚,就成重虚,使魂不藏而游于上,再有外邪侵犯,发生大气厥逆,身体温暖的,还可针刺救治,先刺足少阳经所过的原穴丘墟,再刺足太阳膀胱经的肝俞穴。人平素心气虚弱,又遇到君火或相火司天失守,再感受外邪,便成三虚,遇到火运不及的年份,水邪侵犯,使人猝死,先刺手少阳经所过的原穴阳池,再刺足太阳膀胱经的心俞穴。人平素脾气虚弱,又遇到太阴湿土司天失守,再感受外邪,便成三虚,又遇土运不及的年份,风邪侵犯,使人猝死,先刺足阳明经所过的原穴冲阳,再刺足太阳膀胱经的脾俞穴。人平素肺气虚弱,遇到阳明燥金司天失守,再感受外邪,便成三虚,又遇金运不及的年份,火邪侵犯,使人猝死,先刺手阳明经所过的原穴合谷,再刺足太阳膀胱经的肺俞穴。人平素肾气虚弱,又遇到太阳寒水司天失守,再感受外邪,便成三虚,又遇水运不及的年份,有湿邪侵犯,损伤正气,吸食人的神魂,致使突然死亡,先刺足太阳经所过的原穴京骨,再刺足太阳膀胱经的肾俞穴。\\
黄帝问说:人体十二脏腑之间相互为用,任何一脏神气失位,则使神彩不能丰满,容易受外邪侵犯,可否用刺法调治呢?希望听听其要点。\\
岐伯行礼拜了两拜,回答说:真全面啊!求问至理,陈说真道,如果不是圣明君主,怎能深究这些根源!这是说精气神的活动,符合天道自然的规律。心犹如君主一样的器官,精神活动从此发出,可刺手少阴经的原穴神门。肺犹如宰相一样的器官,治理调节周身,可刺手太阴经的原穴太渊。肝犹如将军一样的器官,深谋远虑从此发出,可刺足厥阴经的原穴太冲。胆犹如中正之官,决定判断从此发出,可刺足少阳经的原穴丘墟。膻中犹如臣使之官,喜乐从此发出,可刺心包络经的荥穴劳宫。脾犹如谏议之官,智慧周密从此发出,可刺足太阴经的原穴太白。胃犹如仓库一样的器官,饮食五味从此发出,可刺胃经的原穴冲阳。大肠犹如传导之官,变化糟粕从此发出,可刺大肠经的原穴合谷。小肠犹如受盛之官,化生精微从此发出,可刺小肠经的原穴腕骨。肾犹如作用强力之官,技巧从此发出,可刺肾经的原穴太溪。三焦犹如疏通隧道之官,水道从此而出,可刺三焦经的原穴阳池。膀胱犹如州都之官,能够存储水液,气化则小便从此而出,可刺膀胱经的原穴京骨。以上十二个脏器,相互之间不能失调。所以刺法有保全精神、调养真气的作用,也就是有修养真气的道理,并不只是用来治病的。所以说要修养真气、调和精神,贵在持之以恒,才能补神固本,精气不耗散,神气内守而不分离,只有神守不离,也才能保全真气,若人的神气失守,就达不到至真之道了。至真之道的关键,在于保养人身之精,神气内守,天息常存,回复本元,就叫归宗。\\
本病论篇第七十三\\
黄帝问曰:天元九窒,余已知之,愿闻气交,何名失守?\\
岐伯曰:谓其上下升降,迁正退位,各有经论,上下各有不前,故名失守也。是故气交失易位,气交乃变,变易非常,即四时失序,万化不安,变民病也。\\
帝曰:升降不前,愿闻其故,气交有变,何以明知?\\
岐伯曰:昭乎问哉!明乎道矣。气交有变,是为天地机,但欲降而不得降者,地窒刑之。又有五运太过,而先天而至者,即交不前,但欲升而不得其升,中运抑之,但欲降而不得其降,中运抑之。于是有升之不前,降之不下者,有降之不下,升而至天者,有升降俱不前,作如此之分别,即气交之变。变之有异,常各各不同,灾有微甚者也。\\
帝曰:愿闻气交遇会胜抑之由,变成民病,轻重何如?\\
岐伯曰:胜相会,抑伏使然。是故辰戌之岁,木气升之,主逢天柱,胜而不前,又遇庚戌,金运先天,中运胜之,忽然不前。木运升天,金乃抑之,升而不前,即清生风少,肃杀于春,露霜复降,草木乃萎。民病温疫早发,咽嗌乃干,四肢满,肢节皆痛,久而化郁,即大风摧拉,折陨鸣紊。民病卒中偏痹,手足不仁。\\
是故巳亥之岁,君火升天,主窒天蓬,胜之不前。又厥阴未迁正,则少阴未得升天,水运以至其中者,君火欲升,而中水运抑之,升之不前,即清寒复作,冷生旦暮。民病伏阳,而内生烦热,心神惊悸,寒热间作,日久成郁,即暴热乃至,赤风肿翳,化疫。温疠暖作,赤气彰而化火疫。皆烦而躁渴,渴甚,治之以泄之可止。\\
是故子午之岁,太阴升天,主窒天冲,胜之不前;又或遇壬子,木运先天而至者,中木运抑之也。升天不前,即风埃四起,时举埃昏,雨湿不化。民病风厥涎潮,偏痹不随,胀满。久而伏郁,即黄埃化疫也。民病夭亡,脸肢府黄疸满闭。湿令弗布,雨化乃微。\\
是故丑未之年,少阳升天,主窒天蓬,胜之不前;又或遇太阴未迁正者,即少阳未升天也,水运以至者,升天不前,即寒雰反布,凛冽如冬,水复涸,冰再结,暄暖乍作,冷复布之,寒暄不时。民病伏阳在内,烦热生中,心神惊骇,寒热间争。以久成郁,即暴热乃生,赤风肿翳,化成疫疠,乃化作伏热内烦,痹而生厥,甚则血溢。\\
是故寅申之年,阳明升天,主窒天英,胜之不前;又或遇戊申戊寅,火运先天而至。金欲升天,火运抑之,升之不前,即时雨不降,西风数举,咸卤燥生。民病上热,喘嗽,血溢,久而化郁,即白埃翳雾,清生杀气。民病胁满,悲伤,寒鼽嚏,嗌干,手拆皮肤燥。\\
是故卯酉之年,太阳升天,主窒天芮,胜之不前;又遇阳明未迁正者,即太阳未升天也,土运以至,水欲升天,土运抑之,升之不前,即湿而热蒸,寒生两间。民病注下,食不及化。久而成郁,冷来客热,冰雹卒至。民病厥逆而哕,热生于内,气痹于外,足胫痠疼,反生心悸,懊热,暴烦而复厥。\\
黄帝曰:升之不前,余已尽知其旨,愿闻降之不下,可得明乎?\\
岐伯曰:悉乎哉问也!是之谓天地微旨,可以尽陈斯道。所谓升已必降也。至天三年,次岁必降,降而入地,始为左间也。如此升降往来,命之六纪也。\\
是故丑未之岁,厥阴降地,主窒地皛,胜而不前;又或遇少阴未退位,即厥阴未降下,金运以至中,金运承之,降之未下,抑之变郁,木欲降下,金承之,降而不下,苍埃远见,白气承之,风举埃昏,清燥行杀,霜露复下,肃杀布令。久而不降,抑之化郁,即作风燥相伏,暄而反清,草木萌动,杀霜乃下,蛰虫未见,惧清伤脏。\\
是故寅申之岁,少阴降地,主窒地玄,胜之不入;又或遇丙申丙寅,水运太过,先天而至,君火欲降,水运承之,降而不下,即彤云才见,黑气反生,暄暖如舒,寒常布雪,凛冽复作,天云惨凄。久而不降,伏之化郁,寒胜复热,赤风化疫。民病面赤、心烦、头痛、目眩也。赤气彰而温病欲作也。\\
是故卯酉之岁,太阴降地,主窒地苍,胜之不入;又或少阳未退位者,即太阴未得降也,或木运以至,木运承之,降而不下,即黄云见而青霞彰,郁蒸作而大风,雾翳埃胜,折损乃作。久而不降也,伏之化郁,天埃黄气,地布湿蒸。民病四肢不举,昏眩,肢节痛,腹满填臆。\\
是故辰戌之岁,少阳降地,主窒地玄,胜之不入;又或遇水运太过,先天而至也,水运承之,降而不下,即彤云才见,黑气反生,暄暖欲生,冷气卒至,甚即冰雹也。久而不降,伏之化郁,冷气复热,赤风化疫。民病面赤、心烦、头痛、目眩也。赤气彰而热病欲作也。\\
是故巳亥之岁,阳明降地,主窒地彤,胜而不入,又或遇太阳未退位,即阳明未得降,即火运以至之,火运承之,降而不下,即天清而肃,赤气乃彰,暄热反作。民皆昏倦,夜卧不安,咽干引饮,懊热内烦。大清朝暮,暄还复作;久而不降,伏之化郁,天清薄寒,远生白气。民病掉眩,手足直而不仁,两胁作痛,满目丼丼。\\
是故子午之年,太阳降地,主窒地阜胜之,降而不入;又或遇土运太过,先天而至,土运承之,降而不下,即天彰黑气,暝暗凄惨,才施黄埃而布湿,寒化令气,蒸湿复令。久而不降,伏之化郁。民病大厥,四肢重怠,阴萎少力。天布沉阴,蒸湿间作。\\
帝曰:升降不前,晰知其宗,愿闻迁正,可得明乎?\\
岐伯曰:正司中位,是谓迁正位,司天不得其迁正者,即前司天,以过交司之日,即遇司天太过有余日也,即仍旧治天数,新司天未得迁正也。\\
厥阴不迁正,即风暄不时,花卉萎瘁。民病淋溲,目系转,转筋,喜怒,小便赤。风欲令而寒由不去,温暄不正,春正失时。\\
少阴不迁正,即冷气不退,春冷后寒,暄暖不时。民病寒热,四肢烦痛,腰脊强直。木气虽有余,位不过于君火也。\\
太阴不迁正,即云雨失令,万物枯焦,当生不发。民病手足肢节肿满,大腹水肿,填臆不食,飨泄胁满,四肢不举。雨化欲令,热犹治之,温煦于气,亢而不泽。\\
少阳不迁正,即炎灼弗令,苗莠不荣,酷暑于秋,肃杀晚至,霜露不时。民病痎疟,骨热,心悸,惊骇,甚时血溢。\\
阳明不迁正,则暑化于前,肃杀于后,草木反荣。民病寒热,鼽嚏,皮毛折,爪甲枯焦,甚则喘嗽息高,悲伤不乐。热化乃布,燥化未令,即清劲未行,肺金复病。\\
太阳不迁正,即冬清反寒,易令于春,杀霜在前,寒冰于后,阳光复治,凛冽不作,雰云待时。民病温疠至,喉闭嗌干,烦躁而渴,喘息而有音也。寒化待燥,犹治天气,过失序,与民作灾。\\
帝曰:迁正早晚,以命其旨,愿闻退位,可得明哉?\\
岐伯曰:所谓不退者,即天数未终,即天数有余,名曰复布政,故名曰再治天也,即天令如故,而不退位也。\\
厥阴不退位,即大风早举,时雨不降,湿令不化。民病温疫,疵废,风生,皆肢节痛,头目痛,伏热内烦,咽喉干引饮。\\
少阴不退位,即温生春冬,蛰虫早至,草木发生。民病膈热,咽干,血溢,惊骇,小便赤涩,丹瘤疹疮疡留毒。\\
太阴不退位,而取寒暑不时,埃昏布作,湿令不去。民病四肢少力,食饮不下,泄注,淋满,足胫寒,阴萎,闭塞,失溺,小便数。\\
少阳不退位,即热生于春,暑乃后化,冬温不冻,流水不冰,蛰虫出见。民病少气,寒热更作,便血,上热,小腹坚满,小便赤沃,甚则血溢。\\
阳明不退位,即春生清冷,草木晚荣,寒热间作。民病呕吐,暴注,食饮不下,大便干燥,四肢不举,目瞑掉眩。\\
太阳不退位,即春寒复作,冷雹乃降,沉阴昏翳,二之气寒犹不去。民病痹厥,阴痿,失溺,腰膝皆痛,温疠晚发。\\
帝曰:天岁早晚,余以知之,愿闻地数,可得闻乎?\\
岐伯曰:地下迁正、升天及退位不前之法,即地土产化,万物失时之化也。\\
帝曰:余闻天地二甲子,十干十二支,上下经纬天地,数有迭移,失守其位,可得昭乎?\\
岐伯曰:失之迭位者,谓虽得岁正,未得正位之司,即四时不节,即生大疫。注《玄珠密语》云:阳年三十年,除六年天刑,计有太过二十四年,除此六年,皆作太过之用。令不然之旨,今言迭支迭位,皆可作其不及也。\\
假令甲子阳年,土运太窒,如癸亥天数有余者,年虽交得甲子,厥阴犹尚治天,地已迁正,阳明在泉,去岁少阳以作右间,即厥阴之地阳明,故不相和奉者也。癸巳相会,土运太过,虚反受木胜,故非太过也,何以言土运太过?况黄钟不应太窒,木既胜而金还复,金既复而少阴如至,即木胜如火而金复微,如此则甲己失守,后三年化成土疫,晚至丁卯,早至丙寅,土疫至也。大小善恶,推其天地,详乎太乙。又只如甲子年,如甲至子而合,应交司而治天,即下己卯未迁正,而戊寅少阳未退位者,亦甲己下有合也,即土运非太过,而木乃乘虚而胜土也,金次又行复胜之,即反邪化也。阴阳天地殊异尔,故其大小善恶,一如天地之法旨也。\\
假令丙寅阳年太过,如乙丑天数有余者,虽交得丙寅,太阴尚治天也。地已迁正,厥阴司地,去岁太阳以作右间,即天太阴而地厥阴,故地不奉天化也。乙辛相会,水运太虚,反受土胜,故非太过,即太簇之管,太羽不应,土胜而雨化,木复即风,此者丙辛失守其会,后三年化成水疫,晚至己巳,早至戊辰,甚即速,微即徐,水疫至也。大小善恶,推其天地数及太乙游宫。又只如丙寅年,丙至寅且合,应交司而治天,即辛巳未得迁正,而庚辰太阳未退位者,亦丙辛不合德也,即水运亦小虚而小胜,或有复,后三年化疠,名曰水疠,其状如水疫。治法如前。\\
假令庚辰阳年太过,如己卯天数有余者,虽交得庚辰年也,阳明犹尚治天,地已迁正,太阴司地,去岁少阴以作右间,即天阳明而地太阴也,故地不奉天也。乙巳相会,金运太虚,反受火胜,故非太过也,即姑洗之管,太商不应,火胜热化,水复寒刑,此乙庚失守,其后三年化成金疫也,速至壬午,徐至癸未,金疫至也。大小善恶,推本年天数及太乙也。又只如庚辰,如庚至辰,且应交司而治天,即下乙未未得迁正者,即地甲午少阴未退位者,且乙庚不合德也,即下乙未柔干失刚,亦金运小虚也,有小胜或无复,后三年化疠,名曰金疠,其状如金疫也。治法如前。\\
假令壬午阳年太过,如辛巳天数有余者,虽交得壬午年也,厥阴犹尚治天,地已迁正,阳明在泉,去岁丙申少阳以作右间,即天厥阴而地阳明,故地不奉天者也。丁辛相合会,木运太虚,反受金胜,故非太过也,即蕤宾之管,太角不应,金行燥胜,火化热复,甚即速,微即徐。疫至大小善恶,推疫至之年天数及太乙。又只如壬午,如壬至午,且应交司而治之,即下丁酉未得迁正者,即地下丙申少阳未得退位者,见丁壬不合德也,即丁柔干失刚,亦木运小虚也,有小胜小复,后三年化疠,名曰木疠,其状如风疫也。治法如前。\\
假令戊申阳年太过,如丁未天数太过者,虽交得戊申年也,太阴犹尚治天,地已迁正,厥阴在泉,去岁壬戌太阳以退位作右间,即天丁未,地癸亥,故地不奉天化也。丁癸相会,火运太虚,反受水胜,故非太过也,即夷则之管,上太徵不应,此戊癸失守其会,后三年化疫也,速至庚戌。大小善恶,推疫至之年天数及太乙。又只如戊申,如戊至申,且应交司而治天,即下癸亥未得迁正者,即地下壬戌太阳未退位者,见戊癸未合德也,即下癸柔干失刚,见火运小虚,有小胜或无复也,后三年化疠,名曰火疠也。治法如前,治之法可寒之泄之。\\
黄帝曰:人气不足,天气如虚,人神失守,神光不聚,邪鬼干人,致有夭亡,可得闻乎?\\
岐伯曰:人之五脏,一脏不足,又会天虚,感邪之至也。人忧愁思虑即伤心,又或遇少阴司天,天数不及,太阴作接间至,即谓天虚也,此即人气天气同虚也。又遇惊而夺精,汗出于心,因而三虚,神明失守。心为君主之官,神明出焉,神失守位,即神游上丹田,在帝太一帝君泥丸宫下。神既失守,神光不聚,却遇火不及之岁,有黑尸鬼见之,令人暴亡。\\
人饮食、劳倦即伤脾,又或遇太阴司天,天数不及,即少阳作接间至,即谓之虚也,此即人气虚而天气虚也。又遇饮食饱甚,汗出于胃,醉饱行房,汗出于脾,因而三虚,脾神失守。脾为谏议之官,智周出焉。神既失守,神光失位而不聚也,却遇土不及之年,或己年或甲年失守,或太阴天虚,青尸鬼见之,令人卒亡。\\
人久坐湿地,强力入水即伤肾。肾为作强之官,伎巧出焉。因而三虚,肾神失守,神志失位,神光不聚,却遇水不及之年,或辛不会符,或丙年失守,或太阳司天虚,有黄尸鬼至,见之令人暴亡。\\
人或恚怒,气逆上而不下,即伤肝也。又遇厥阴司天,天数不及,即少阴作接间至,是谓天虚也,此谓天虚人虚也。又遇疾走恐惧,汗出于肝。肝为将军之官,谋虑出焉。神位失守,神光不聚,又遇木不及年,或丁年不符,或壬年失守,或厥阴司天虚也,有白尸鬼见之,令人暴亡也。\\
已上五失守者,天虚而人虚也,神游失守其位,即有五尸鬼干人,令人暴亡也,谓之曰尸厥。人犯五神易位,即神光不圆也。非但尸鬼,即一切邪犯者,皆是神失守位故也。此谓得守者生,失守者死。得神者昌,失神者亡。\\
黄帝问:天元之气窒抑,我已经知道了,希望听听天地的气交变化,什么叫失守?\\
岐伯说:凡是上下升降,迁正退位,各有固定规律,上下不得前进,升降失常,这就叫失守。所以天地之气的交替,失去正常的易位规律,气交就发生异常变化,气交有异常变化,就是四时失去正常秩序,万物的生化不能平安,人们就要生病了。\\
黄帝问:升和降不能前进,希望听听其中的缘故,天地气交发生变化,怎样能明白知道呢?\\
岐伯说:问得真高明啊!能够明白大道了。天地之气的交替有变化,这是天地运行的机理,只是要降而不得降的,是地窒克伐。又有五运太过,而其气候先于天时而至的,气交就不能前进,要升的不能升,是因为中运抑制它,或者要降的不能降,也是因为中运抑制它。于是有的升之不前,有的降之不下,有的降之不下,升而至天的,有的升降都不前的,能作出这样的区别,就可以了解气交的变化。异常的变化,常各个不相同,灾害也就有轻有重了。\\
黄帝说:希望听听天地气交相遇相会相克相抑的原因,灾变导致人们疾病,病情轻重怎样?\\
岐伯说:气交遇到胜气相会时,就要折伏成郁了。所以辰戌之岁,厥阴木气当从在泉之右间,上升为司天之左间,如遇到在天的金气过胜,木气不能前进。又遇到庚戌之年,金运之气先于天时而至,金胜克木,木气忽然不能前进。木气本来是要上升的,却碰到在天的金气和中见金运的抑制,木气不能上升和前进,就发生清凉之气,风反而减小,春天出现秋令肃杀之气,又降下霜露,使草木枯萎。人们易患温疫早发,咽喉干燥,四肢胀满,肢节尽痛,但木郁既久,必定化郁为通,出现大风劲吹,拔倒树木,声音紊乱。人们多病猝中,半身麻痹,手足不仁。\\
所以巳亥之年,少阴君火应从在泉之右间,上升为司天之左间,若遇到在天的水气过胜,火气不能前进。又厥阴木气未得迁正中位,那么少阴更不能上升,再因水运在中,君火要升,受到在中水运的抑制,而升之不前,因此气候仍然清冷,早晚更甚。人们多病阳气遏伏,内热烦闷,心神惊悸,寒热交作,日久成郁,一旦开通,气候陡然变热,风火之气聚积覆盖于大地之上,容易化成疫疠。大凡温病疫疠,都是因暖而作,赤色之气显著,就化成火疫。病者都出现心烦,躁动口渴,渴得厉害,用清泄的方法治疗,诸证可止。\\
所以子午之年,太阴湿土当从在泉之右间,上升为司天之左间,如遇到在天木气过胜,土气不能前进;又遇到壬子木运太过,其气先于天时而至,则土被木克制,阻窒不前,就要风尘四起,天昏地暗,雨湿之气不能布散。人们要生风厥、涎潮、半身不遂、胀满等病。土气久郁,郁极则发,黄埃之气就要化成疫疠。人们多病夭折,头脸四肢发黄,成为黄疸,六腑胀满闭塞。湿令不能布化,雨水很少下降。\\
所以丑未之年,少阳相火当由在泉之右间,上升为司天之左间,若遇到在天水气过胜,火气不能前进;又若遇到去年未能退位的少阴,以致太阴不能迁正就位,少阳也就无从升天,若逢水运抑制,也不能升天而前进,这时反见寒霜冷雨,凛冽如冬,河水干涸,或再次结冰,有时忽然天暑地热,马上又寒冷布散,寒暖不时。人们多病伏火在内,心中烦热,惊骇不安,寒热交作。郁久必复,气候转为暴热,风火之气聚积覆盖于上,变生疫疠,于是变成伏热心中烦躁,四肢麻痹而厥冷,严重则出血。\\
所以寅申之年,阳明燥金应从在泉之右间,上升为司天之左间,若遇到在天火气过胜,金气不能前进;又若遇到戊申戊寅年,火运太过,其气先于天时而至。金被火克,虽欲上升,仍然不能前进,时雨就不能下降,西风时时到来,大地出现咸卤白霜,燥气也产生了。人们多病上焦有热,气喘咳嗽,甚至出血,当久郁忽然开通之时,白埃之气飞扬,如烟如雾,清凉肃杀之气产生。人们多病胸胁满闷,易于悲伤,鼻塞流涕,喷嚏,咽喉干燥,两手皴裂,皮肤干燥。\\
所以卯酉之年,太阳寒水应从在泉之右间,上升为司天之左间,若遇到在天土气过胜,水气不能前进;又若遇到阳明未得迁正主司中位,就使太阳无从上升,若土运已到,寒水要升,受了土运的抑制,也就不能前进,于是湿气与热气互相蒸郁,寒气生于左右间气之位。人们多病暴泄如注,饮食来不及消化。但久郁忽然开通之时,冷气胜过热气,冰雹突然下降。人们多生厥气上逆而呃逆,热气生于内,阳气痹于外,足胫酸疼,反见心悸懊热,暴烦而又厥逆。\\
黄帝说:升之不前,我已经完全懂得它的道理了,希望听听降之不下,能明白地告诉我吗?\\
岐伯说:问得真详细啊!这是天地间极精微的道理,可以把这道理全部陈说。上升以后,必然要下降,到升天三年以后,下一年必定下降,下降入地,开始成为地之左间。又在地三年,这样升降往来,共为六年,所以就命名为“六气之纪”。\\
所以丑未之年,厥阴风木当从司天右间,下降为在泉左间,若遇到在地的金气,木气受制而不能前进;又或遇到少阴未退位,厥阴就无从下降,而在中的金运已至,因金运下承,导致降而不下,阻抑于中,久之变而成郁,由于木欲下降,金运相承,使它不能下降,远处见到青色的尘埃,白色之气相承接,风吹尘埃而天昏地暗,清凉秋燥行肃杀之令,霜和露又下降,肃杀气候行令。木气久而不降,郁抑日久,就要化成燥气,伏于风内,该温暖而反见清冷,草木该发芽,可是严霜又至,蛰虫也未见出现,人们也惧怕清冷之气伤害内脏。\\
所以寅申之年,少阴君火当从司天右间,下降为在泉左间,若遇到在地的水气,使火受水制而不能前进;又或遇到丙申丙寅,水运之气太过,先于天时而至,少阴君火要下降,遇水运相克,不能下降,火气刚刚出现,水气反而到来,本来气候温暖,可是却冷得时常下雪,凛冽寒风又起,天空阴云惨淡凄凉。少阴君火久伏而不降,则化为郁气,郁伏已久,一旦开通,寒极生热,风火化成疫疠。人们多病面赤、心烦、头痛、目眩。火气过分彰显,是温热病将发的征兆。\\
所以卯酉之年,太阴湿土当从司天右间,下降为在泉左间,若遇到在地的木气,使土受木制而不能前进;又或少阳相火未能退位,太阴不能下降,或者木运到来,抑制太阴湿土,欲降不下,于是黄云出现,青霞彰显,郁滞熏蒸而成大风,尘埃飞扬如雾,甚至折损草木。如果久不得入地,郁伏既久,则天上有黄色埃气,地下湿气熏蒸。人们多患四肢不能抬举,头目昏眩,肢节疼痛,胸腹胀满。\\
所以辰戌之年,少阳相火当从司天右间,下降为在泉左间,若遇到在地的水气,火受水制而不能前进;又或遇到水运太过,先于天时而至,水运相承,相火便不能入地,这样,彤云刚刚出现,水气反而到来,本来是要温暖的,可是冷气突然到来,甚至结成冰雹。久而不能下降,伏久必定化郁为通,冷气变为热气,火气化成疫疠。人们易患面赤、心烦、头痛、目眩等病。火气彰显,热病就要发生了。\\
所以巳亥之年,阳明燥金当从司天右间,下降为在泉左间,若遇到在地的火气,金受火制而不能前进,又或遇到太阳未退位,就使阳明无从下降,或遇火运已至,因火运相乘,金气不能下降入地,这时本应天清气爽,可是反而火气昭彰,炎热异常。人们皆感到昏倦,夜卧不宁,咽喉干燥,口渴引饮,闷热内心烦躁。本来应朝暮清凉,却出现暄暖;如果久不得降,则伏久将要化郁为通,那时天气清凉,凉风阵阵,远处出现白气。人们多患掉眩,手足强直,麻木无知,两胁作痛,两眼视物不清。\\
所以子午之年,太阳寒水应从司天右间,降为在泉左间,若遇到在地的土气,水受土制,太阳不能降而入地;又或遇到土运太过,先于天时而至,土运承制,太阳不能入地,天地之间布满寒水之气,昏暗惨淡,忽然黄埃飞扬,湿气弥漫,本来是寒化之气当令,现在却是蒸湿当令。久而不降,伏久郁化为通。人们多患大厥,四肢沉重倦怠,阴痿少力。天气阴沉,时常湿气蒸发。\\
黄帝说:升与降不能前进,已经清楚地知道它的宗旨了,希望听听迁正的道理,可以明白告诉我吗?\\
岐伯说:正司天地的中位,叫做迁正位,司天不得迁于正位的,是因为前年的司天已过了新旧之交的大寒日,就是司天太过的余日,仍旧治理着天气,这样新司天就不能迁正了。\\
厥阴不能迁居正位,就是风木温暖之气不能按时行令,花草枯槁。人们易患小便淋痛不利,目系转,转筋,易发怒,小便赤。风木要行令而寒气不去,所以气候温暖不正常,失去春天正常的时令。\\
少阴君火不能迁居正位,则冷气不退,春天先冷后寒,温暖气候不能按时到来。人们易患发热恶寒,四肢烦痛,腰脊强直。厥阴风木之气虽然太过,留恋在位不退,但不会超过君火当令之时。\\
太阴不能迁入正位,云雨就不能及时布散,万物干枯憔悴,当生长却不能发育。人们易患手足肢节肿满,大腹水肿,心胸妅胀,不欲饮食,泄泻,胁满,四肢不能抬举。本当太阴湿土行令,而少阴君火仍旧主司热令,所以气候温暖,干旱无雨。\\
少阳不能迁入正位,则炎热的气候不能行令,苗秀出,却不能繁荣,秋天出现酷暑,肃杀之气晚至,霜露不能及时下降。人们易患疟疾,骨热,心悸,惊骇,甚则时见出血。\\
阳明不能迁入正位,炎热气候行于前,肃杀之气后至,草木反见繁荣。人们易患寒热往来,鼻流清涕,喷嚏,毫毛枯折,爪甲枯悴,甚则气喘咳嗽,呼吸抬肩,悲伤不乐。炎热气候仍旧布散,燥气未能行令,就是清肃的气候未来,肺金受病。\\
太阳不能迁入正位,则冬时寒冷的气候反见于春天,肃杀的霜露下降于前,坚冰凝结于后,如果阳气重新行令,则凛冽的寒气不会发生,雰云待时而出现。人们发生温病疫疠,喉闭嗌干,烦躁而渴,喘息有音。太阳寒水之气,要等到燥金之气退去,才能司其气化之令,若燥气过期不去,则时序失常,就会带给人们灾害。\\
黄帝说:迁正早晚的道理,我已明白,希望听听退位的问题,可以说明吗?\\
岐伯说:所谓不退位,就是司天之数未终,也就是天数有余,名叫复布政,也叫再司天,就是天令一如过去,而不退位。\\
厥阴不退位,就会大风早起,时雨不能下降,湿令不能施化发用。人们易患温疫,黑斑,肢体偏废,风病,都有肢节痛,头目痛,热伏于内烦躁,咽喉发干口渴引饮。\\
少阴不退位,则春冬气候温暖异常,蛰虫早出活动,草木提前发芽生长。人们易患胸膈郁热,咽干,出血,惊骇,小便赤涩,丹瘤疹疮疡瘤毒。\\
太阴不退位,寒冷与暑热不时发生,尘埃昏蒙弥漫天空,太阴湿土之令不退去。人们易患四肢少力,饮食不下,大便泄泻,小便淋沥,腹满,足胫寒冷,阴痿,大便不通,尿失禁,小便频数。\\
少阳不退位,春天出现炎热的天气,暑热滞留不去,冬天温暖不上冻,流水不能结冰,蛰虫出现。人们易患少气,寒热往来,便血,上部发热,小腹坚硬胀满,小便赤,甚则出血。\\
阳明不退位,春天气候清凉,草木繁荣推后,气候寒热相间。人们易患呕吐,泄泻如注,或饮食不下,或大便干燥,四肢不能抬举,头目眩晕。\\
太阳不退位,春季寒冷气候又至,冰雹降下,阴沉之气昏暗覆盖,至二之气时寒气仍未退去。人们易患痹证厥冷,阴痿,小便失禁,腰膝疼痛,温病疫疠晚发。\\
黄帝说:司天之气的早晚,我已经知道了,希望听听在泉之数,可以让我听听吗?\\
岐伯说:地下迁正、升天及退位不能如时前进,可应于大地上物产变化,使万物失去正常时令的生化。\\
黄帝说:我听说天地二甲子,十干与十二支,上下相合,经纬天地之气,其数相互之间有更移的,有失守其位的,可以说明吗?\\
岐伯说:失去更移之正位,则虽得当岁之正位,而未能得其司正位之气,就使得四时节令变化失序,将发生大疫疠了。注《玄珠密语》上说过:阳年三十年,除去六年天刑,计有二十四个太过年,除此六年,皆是太过的。若不然的话,是因为刚柔迭失其位,虽是太过有余,亦当作为不及。\\
假如甲子阳年,土运太过而抑制,如癸亥年司天之数有余,年虽已交得甲子,可是去年司天之厥阴尚未退位,今年在泉的阳明已经迁正,去年在泉之少阳已退作在泉右间,就是去年的厥阴仍在司天的位置,在泉之阳明已迁正,所以上下不能相合了。癸巳相会,虽是土运太过,但其气已虚,反受木克,所以就不是太过,怎么能说土运太过呢?况且六律之太宫的黄钟不应太窒,木既胜土,则土之子金必来报复,金既来报复而少阴司天忽至,则木反助火克金,故金的报复力必微,如此则甲己失守,其后三年化成土疫,最迟在丁卯年,最早在丙寅年,土疫就要发生。其大小轻重和预后善恶,就要察看当年司天在泉之气的盛衰和北极星所指的月令了。又如甲子年,在上的甲与子合,相应司天之位,在下的阳明己卯未能迁正在泉,去年戊寅的少阳未曾退位,也就形成甲己与在下之戊寅相合,土运就不是太过,而木乃乘虚克土,它所生的金又行复胜,即反化成病邪。在上的司天与在下的在泉,阴阳属性不同,所以产生疫疠的大小与善恶,和司天在泉的变化是一样的。\\
假如丙寅阳年太过,如去年乙丑年司天之数有余,今年虽交得丙寅,而去年司天之太阴尚未退位。但今年在泉的厥阴已经迁正,就是去年在泉之太阳已经退位而作地之右间,形成司天太阴、司地厥阴的局面,所以地下不能承奉天令所化。如上乙下辛相会,水运太虚,反受土克,故不能算阳土太过,即如太簇与太羽音律不能相应,土胜而雨化,木来相应则化为风,这是丙辛失守其会,后三年化成水疫,最迟到己巳年,最早到戊辰年,甚者其至速,微者其至迟,水疫就要发生。其大小与善恶,要根据司天在泉的气数及北斗所指的月令来推算。又如丙寅年,丙与寅合,少阳应作司天,即辛巳厥阴未得迁正在泉,庚辰太阳未能退位,那上位司天之丙不能得下位在泉之辛,使水运小虚而有小胜小复,以后三年化为疫疠,名为水疠,病状如水疫。治法同前。\\
假如庚辰年阳年太过,如去年己卯司天之数有余,今年虽交得庚辰,阳明还在司天,下面的太阴已经迁正在泉,去年在泉之己卯少阴退位,己作地之右间,就成为司天阳明而司地太阴,所以司地不能承奉天令所化。上乙下巳相会,金运太虚,反受火克,故不能算阳土太过,即如姑洗与太商不能相应,火胜水复,气候当先热后寒,这是乙庚失守,其后三年当化成金疫,最快在壬午年,最慢在癸未年,金疫就要发生。其病的大小与善恶,要根据本年司天在泉的气数及北斗所指之月令而定。又如庚辰应时迁正司天,而下乙未未得迁正在泉,去年甲午少阴未得退位,那么上位司天便孤立,乙庚不能合德,即在下乙未柔干不能合刚,亦金运小虚,有小胜或无复,后三年化成疫疠,名为金疠,其症状与金疫相似。治法同前。\\
假如壬午年阳年太过,如去年辛巳司天之数有余,今天虽交得壬午,但厥阴尚在司天,下面的阳明已迁正在泉,去年在泉之丙申少阳已作地之右间,成为司天厥阴而司地阳明,所以地不能承奉天令所化。如上丁下辛相会合,木运太虚,反受金克,故不能算阳土太过,即如蕤宾与太角不能相应,金气行令而燥胜,火化热气之复,丁壬不能合德,其后三年化成木疫,严重的其至快,轻微的其至慢。疫至大小与善恶,当看疫至之年的天数与北斗所指的月令。又如壬午应时迁正司天,而下位丁酉未得迁正在泉,去年在泉之丙申少阳未得退位,那么上位司天便孤立,上下不能合德,这就是丁柔干不能合刚,木运亦小虚,有小胜,也有小复,其后三年化成疫疠,名为木疠,病状如风疫。治法同前。\\
假如戊申年阳年太过,如去年丁未司天之数有余,今年虽交得戊申,太阴犹尚司天,下面的厥阴已迁正在泉,去年在泉之壬戌太阳,已退位作地之右间,就成为司天丁未,司地癸亥,所以地不能承奉天令所化。上丁下癸相会,火运太虚,反受水克,故不能算作阳土太过,即如夷则与太徵不能相应,这时戊癸失守其会,后三年将化疫疠,最快在庚戌年。其疫大小与善恶,要看疫至之年的天数和北斗所指的月令。又如戊申应时迁正司天,而下面癸亥未得迁正在泉,壬戌太阳未得退位,那么上位司天便孤立,不能与在泉合德,这是下癸柔干不能合刚,火运稍衰,或有小胜,或无复,其后三年化成疫疠,名为火疠。治法同前,治疗的方法可用寒法泄法。\\
黄帝说:人体正气不足,天气又虚,神气失守,神光不聚,病邪伤人,致有夭亡,可以听听吗?\\
岐伯说:人的五脏,如果有一脏不足,再遇到天虚,感受的邪气就会深入。人过度忧愁思虑,就会损伤心脏,又如遇到少阴司天而天数不足,太阴作为接替的主司,这样就叫天虚,也就是人气和天气都虚。如果再遇到惊吓而损伤心精,汗出伤其心液,便成为三虚,以致神明失守。心是君主之官,神明出于此,神明失守其位,也就是神明游离于上丹田泥丸宫之下。神明既失其位,神光不能聚集,却遇到火运不及的年岁,则水疫之邪致病,使人猝然死亡。\\
人饮食不节、劳倦过度则伤脾脏,又遇太阴司天,天数不足,少阳作司天的左间来代表,这就叫虚,就是人气虚天气也虚。又遇到饮食过饱,伤胃出汗,或者酒醉行房,出汗伤脾,因而形成三虚,以致脾神失守。脾是谏议之官,智慧周密由此而出。脾神既失其位,神光失位而不能聚集,又遇岁土不及之年,或遇己年或甲年失守,或太阴司天天数不足,便有木疫之邪为病,使人猝死。\\
人久居潮湿之地,或强力入水作业,就会伤肾。肾是作强之官,伎巧由此而出。今有三虚,肾神失守,神志失位,神光不聚,却遇到水运不及之年,或者辛不会合,或者逢丙年失守,或者太阳司天不及,就有土疫之邪致病,使人猝死。\\
人如愤怒,气上逆而不能下降,就会损伤肝脏。又遇到厥阴司天,天数不足,即少阴作司天左间来代替,这叫天虚,成为天人两虚。又如奔跑恐惧,汗出伤肝。肝是将军之官,谋虑由此而出。精神失守,神明不聚,又遇到木运不及之年,或丁年不相会合,或壬年失守,或厥阴司天不及,就有金疫之邪致病,使人猝死。\\
(按:原文脱“伤肺”一节。)\\
以上五种失守其位的,因为天虚和人虚,使精神游散,失于守藏,就有五种病邪侵袭,使人猝死,这叫尸厥。人或患了五脏之神失其常位,就会神光不圆满了。不止疫邪为患,就是一切邪气侵袭,都是由于精神不守的缘故。这就是所说的,精神能够守藏则生存,不能守藏则死亡。得神的就会昌盛,失神的就要死亡。\\
(按:以上《刺法论》与《本病论》两篇,在王冰注释之前就已亡佚,王冰次注本中只存篇目,所以称为“遗篇”。)\\
黄帝内经(全二册)——传世经典 文白对照\\
黄帝内经 灵枢\\
叙\\
卷一 九针十二原第一法天\\
卷一 本输第二法地\\
卷一 小针解第三法人\\
卷一 邪气脏腑病形第四法时\\
卷一 根结第五法音\\
······\\
叙\\
昔黄帝作《内经》十八卷,《灵枢》九卷,《素问》九卷,迺其数焉。世所奉行唯《素问》耳。越人得其一二而述《难经》,皇甫谧次而为《甲乙》。诸家之说,悉自此始。其间或有得失,未可为后世法。则谓如《南阳活人书》称“咳逆者,哕也”。谨按《灵枢》经曰:“新谷气入于胃,与故寒气相争,故曰哕。”举而并之,则理可断矣。又如《难经》第六十五篇,是越人标指《灵枢·本输》之大略,世或以为流注。谨按《灵枢》经曰:“所言节者,神气之所游行出入也,非皮肉筋骨也。”又曰:“神气者,正气也。神气之所游行出入者,流注也。井、荥、输、经、合者,本输也。”举而并之,则知相去不啻天壤之异。但恨《灵枢》不传久矣,世莫能究。\\
夫为医者,在读医书耳。读而不能为医者,有矣;未有不读,而能为医者也。不读医书,又非世业,杀人尤毒于梃刃。是故古人有言曰:为人子而不读医书,犹为不孝也。\\
仆本庸昧,自髫迄壮,潜心斯道,颇涉其理。辄不自揣,参对诸书,再行校正家藏旧本《灵枢》九卷,共八十一篇,增修音释,附于卷末,勒为二十四卷。庶使好生之人,开卷易明,了无差别。除已具状经所属申明外,准使府指挥依条申转运司,选官详定,具书送秘书省国子监。今崧专访请名医,更乞参详,免误将来,利益无穷,功实有自。\\
时宋绍兴乙亥仲夏望日锦官史崧题\\
从前黄帝创作了《内经》共计十八卷,其中《灵枢》九卷,《素问》九卷,就是其总数。但是世上所遵行流传的只有《素问》。秦越人取其中的十分之一二,而发挥成《难经》,皇甫谧重新编纂排列为《针灸甲乙经》。各家学说,都是从此开始的。这其中或有得有失,不能成为后世的准则。比如《南阳活人书》中说“咳逆,就是哕证”。我谨慎地考察了《灵枢》经,其中说:“刚刚食用的五谷之气进入胃中,与胃中旧有的寒气相争,所以发生哕证。”列举并排于此,哪个更合乎道理就可以判断了。又比如《难经》的第五十六篇是秦越人标示《灵枢·本输》篇的内容大要,世上有人认为是讲“流注”的。我谨慎地考察了《灵枢》经,其中说:“所说的节是神气运行出入之所,不是指皮肉筋骨。”又说:“神气,就是正气。神气的运行出入,是流注。井、荥、输、经、合是经脉输注的关键所在。”列举并排于此,就可以知道所讲的道理相距不仅是天壤之别。只是遗憾《灵枢》不传于世,已经很久了,世人不能探究清楚。\\
行医治病,关键在于研读医书。读了医书,不能做个良医,有这种情况;没有不读医书,而能做个好医生的。不研读医书,又不是世代以医为业,那样害人就比刀枪棍棒厉害得多了。所以古人曾说过:作为人子,而不读医书,还不能算是孝子。\\
我本来平庸愚昧,从幼年至壮年,潜心研究医道,较深入地探究了其中的道理。于是不揣愚陋,参考核对各种书籍,再校正家藏旧本的《灵枢》九卷,共计八十一篇,增加修订了注音和释词附在卷末,刻成二十四卷。或许能使爱护生命的人,打开书卷,很容易明白,不会有差错。除了已经陈述情况向所属部门说明外,还恳请府里的指挥依据条例申请转运司,指定官员详细审定,全部誊写清楚送秘书省和国子监。现在我又专门访问和聘请名医,进一步请他们审慎详尽地参订,以免贻误今后的读者,给后世带来无穷的利益,医学的功效有所来源。\\
时间是有宋绍兴二十五年五月十五日锦官史崧题\\
卷一\\
九针十二原第一 法天\\
黄帝问于岐伯曰:余子万民,养百姓,而收其租税。余哀其不给,而属有疾病。余欲勿使被毒药,无用砭石,欲以微针通其经脉,调其血气,营其逆顺出入之会。令可传于后世,必明为之法。令终而不灭,久而不绝,易用难忘,为之经纪。异其章,别其表里,为之终始,令各有形,先立《针经》。愿闻其情。\\
岐伯答曰:臣请推而次之,令有纲纪,始于一,终于九焉。请言其道。小针之要,易陈而难入。粗守形,上守神。神乎神,客在门。未睹其疾,恶知其原?刺之微,在速迟。粗守关,上守机。机之动,不离其空。空中之机,清静而微。其来不可逢,其往不可追。知机之道者,不可挂以发;不知机道,叩之不发。知其往来,要与之期。粗之暗乎,妙哉!工独有之。往者为逆,来者为顺,明知逆顺,正行无问。逆而夺之,恶得无虚?追而济之,恶得无实?迎之随之,以意和之,针道毕矣。\\
凡用针者,虚则实之,满则泄之,宛陈则除之,邪胜则虚之。《大要》曰:徐而疾则实,疾而徐则虚。言实与虚,若有若无。察后与先,若存若亡。为虚与实,若得若失。虚实之要,九针最妙。补泻之时,以针为之。泻曰:必持内之,放而出之,排阳得针,邪气得泄。按而引针,是谓内温,血不得散,气不得出也。补曰:随之,意若妄之,若行若按,如蚊虻止,如留如还,去如弦绝。令左属右,其气故止,外门以闭,中气乃实。必无留血,急取诛之。持针之道,坚者为宝,正指直刺,无针左右,神在秋毫,属意病者,审视血脉,刺之无殆。方刺之时,必在悬阳,及与两卫,神属勿去,知病存亡。血脉者,在腧横居,视之独澄,切之独坚。\\
九针之名,各不同形:一曰优针,长一寸六分;二曰员针,长一寸六分;三曰氻针,长三寸半;四曰锋针,长一寸六分;五曰铍针,长四寸,广二分半;六曰员利针,长一寸六分:七曰毫针,长三寸六分;八曰长针,长七寸;九曰大针,长四寸。优针者,头大末锐,去泻阳气;员针者,针如卵形,揩摩分间,不得伤肌肉,以泻分气;氻针者,锋如黍粟之锐,主按脉勿陷,以致其气;锋针者,刃三隅,以发痼疾;铍针者,末如剑锋,以取大脓;员利针者,尖如氂,且员且锐,中身微大,以取暴气;毫针者,尖如蚊虻喙,静以徐往,微以久留之而养,以取痛痹;长针者,锋利身长,可以取远痹;大针者,尖如梃,其锋微员,以泻机关之水也。九针毕矣。\\
夫气之在脉也,邪气在上;浊气在中,清气在下,故针陷脉则邪气出,针中脉则浊气出,针太深则邪气反沉,病益。故曰:皮肉筋脉,各有所处,病各有所宜,各不同形,各以任其所宜。无实无虚,损不足而益有余,是谓甚病,病益甚。取五脉者死,取三脉者恇。夺阴者死,夺阳者狂。针害毕矣。刺之而气不至,无问其数;刺之而气至,乃去之,勿复针。针各有所宜,各不同形,各任其所为。刺之要,气至而有效,效之信,若风之吹云,明乎若见苍天。刺之道毕矣。\\
黄帝曰:愿闻五脏六腑所出之处。\\
岐伯曰:五脏五腧,五五二十五腧;六腑六腧,六六三十六腧。经脉十二,络脉十五。凡二十七气,以上下。所出为井,所溜为荥,所注为输,所行为经,所入为合。二十七气所行,皆在五腧也。节之交,三百六十五会。知其要者,一言而终;不知其要,流散无穷。所言节者,神气之所游行出入也,非皮肉筋骨也。\\
睹其色,察其目,知其散复;一其形,听其动静,知其邪正。右主推之,左持而御之,气至而去之。凡将用针,必先诊脉,视气之剧易,乃可以治也。五脏之气已绝于内,而用针者反实其外,是谓重竭。重竭必死,其死也静。治之者辄反其气,取腋与膺。五脏之气已绝于外,而用针者反实其内,是谓逆厥。逆厥则必死,其死也躁。治之者反取四末。刺之害,中而不去,则精泄;不中而去,则致气。精泄则病益甚而恇,致气则生为痈疡。\\
五脏有六腑,六腑有十二原,十二原出于四关,四关主治五脏。五脏有疾,当取之十二原。十二原者,五脏之所以禀三百六十五节之会也。五脏有疾也,应出十二原,而原各有所出,明知其原,睹其应,而知五脏之害矣。\\
阳中之少阴,肺也,其原出于太渊,太渊二。阳中之太阳,心也,其原出于大陵,大陵二。阴中之少阳,肝也,其原出于太冲,大冲二。阴中之至阴,脾也,其原出于太白,太白二。阴中之太阴,肾也,其原出于太溪,太溪二。膏之原,出于鸠尾,鸠尾一。肓之原,出于脖胦,脖胦一。凡此十二原者,主治五脏六腑之有疾者也。胀取三阳,飧泄取三阴。\\
今夫五脏之有疾也,譬犹刺也,犹污也,犹结也,犹闭也。刺虽久,犹可拔也;污虽久,犹可雪也;结虽久,犹可解也;闭虽久,犹可决也。或言久疾之不可取者,非其说也。夫善用针者,取其疾也,犹拔刺也,犹雪污也,犹解结也,犹决闭也。疾虽久,犹可毕也。言不可治者,未得其术也。\\
刺诸热者,如以手探汤;刺寒清者,如人不欲行。阴有阳疾者,取之下陵三里。正往无殆,气下乃止,不下复始也。疾高而内者,取之阴之陵泉;疾高而外者,取之阳之陵泉也。\\
黄帝问岐伯说:我养万民、养百官,而征收他们的租税。很怜悯他们不能终尽天年,还接连不断地生病。我想叫他们不服药,也不用砭石,只用细针,刺入肌肤,就能疏通经脉,调和气血,使气血运行,在经脉中逆来顺往出入会合。使这种疗法,可以传到后世,就必须明确地制定出针经大法。为使针法永远不会磨灭,历久相传而不断绝,在学习中,容易运用,难以忘记,这又必须制定出微针使用的准则。另外,更要辨章析句,辨别表里,讲明用针的终始之道,把九针的形状写清楚,首先编成一部《针经》。我希望听到实际内容。\\
岐伯答道:我愿意把所知道的按着次序来谈,这样才有条理,从一到九,终始不乱。先谈谈针刺治疗的一般道理。小针的关键所在,说起来容易,可是达到精微的境界却很难啊!粗率的医生拘守形体,仅知在病位上针刺,高明的医工却懂得根据病人的神气变化针治疾病。神啊!神啊!气血循行经脉,出入有一定的门户,病邪可从门户侵入体内,医生看不出是什么病,哪能了解病变的原因呢?针刺的巧妙,在于如何运用疾徐手法。粗率的医生拘守四肢关节的穴位治疗,高明的医工却能观察经气。经气的循行,离不开腧穴。邪气随着经气而流动,腧穴所体现的经气虚实变化是清静微妙的,必须仔细体验。当邪气盛时,不可迎而补之,在邪气衰时,不可追而泻之。懂得气机变化的道理,就不会有毫发的差失;不懂得气机变化的道理,就像箭扣弦上,不能射出一样。所以针刺必须掌握气的往来顺逆盛衰之机,才能确有疗效。粗率的医生对此昏昧无知,这种妙处,只有高明的医工才能有。什么是逆顺呢?正气去叫做“逆”,正气来复叫做“顺”,明白逆顺之理,就可以放胆直刺,无须四顾询问了。那正气已虚,反而用泻法,怎么不会更虚呢?邪气正盛,反而用补法,怎么不会更实呢?必须迎其邪而泻,随其去而补,对于补泻手法,能用心体察,那么针刺之道,也就尽在其中了。\\
凡是针刺时,正气虚用补法,邪气实用泻法,有淤血的用破除法,邪气胜的用攻邪法。《大要》说:慢进针而快出针,急按针孔的为补法,快进针而慢出针,不按针孔的为泻法。说到针下有气感为实,针下无气感为虚,因为气本无形,所以似有似无。根据疾病的缓急及气的虚实来决定补泻的先后次序,根据气之虚实,而决定是否留针及留针的久暂。总的说来,如掌握得法,就能达到补虚泻实的目的,使患者感到补之若有所得,泻之若有所失。补虚泻实的要点,在于巧妙地使用九针。或补或泻,用针刺手法来解决。泻法的要领是:持针纳入,得气后,摇大针孔,转而出针,这可使邪气随针外泄。假如出针随即按闭针孔,会使邪气蕴郁于内,淤血不散,邪气不得外泄。补法的要领是:顺随经脉循行的方向进针,好像漫不经心地轻轻刺入。在行针引气,按穴下针时,像蚊虫叮咬一样似留似去的感觉,得气以后,急速出针像箭离弓弦一样快。右手出针,左手急闭针孔,经气因而留止,针孔已闭,中气就会充实了。如有皮下出血,应该速予除去。持针的准则,以手下坚牢有力最可贵。对准腧穴,端正直刺,针不偏左偏右,行针者的精神要集中在针端,注意观察病人,仔细看其血脉,进针时避开它,这样,就不会发生危险了。要进针的时候,一定要注意病人的精神状态及卫气、脾气的状况,而针者也须聚精会神,毫不疏忽,从而测知病气的存亡。血脉之所在,横布在腧穴周围,看起来显得很清楚,用手去摸按也会感到坚实。\\
九针的名称不同,形状也各异:第一种叫奷针,长一寸六分;第二种叫员针,长一寸六分;第三种叫奼针,长三寸五分;第四种叫锋针,长一寸六分;第五种叫铍针,长四寸,宽二分半;第六种叫员利针,长一寸六分;第七种叫毫针,长三寸六分;第八种叫长针,长七寸;第九种叫大针,长四寸。奷针,针头大而针尖锐利,适于浅刺以泻皮肤之热;员针,针形如卵,用于按摩分肉之间,既不会损伤肌肉,又能够疏泄分肉的邪气;奼针,针尖像小米粒的微圆,用于按摩经脉,流通气血,但不能深陷肌肉之内,否则反伤正气;锋针,三面有刃,用以治疗积久难治之病;铍针,针尖像剑锋一样锐利,用以刺痈排脓;员利针、针尖像马尾,圆而锐利,针身稍粗,用于治疗急证;毫针,针尖像蚊虻的嘴,轻缓的刺入皮内,留针养神,可以治疗痛痹;长针,针锋锐利,针身薄而略长,可以治疗久痹证;大针,针尖如折竹,其锋稍圆,可用以泻去关节积水。所有九针的情况,大致如此而已。\\
邪气在人体经脉之内,风热之气常在上部;饮食积滞之气常停中部,寒湿之气常留下部,因而针刺部位也就不同了,刺上部各经腧穴可使风热之气外出;刺阳明之脉,可以排除胃肠积滞;病在浅表而针刺太深了,能够引邪入里,加重病势。因此说:皮肉筋脉各有它的部位,病证各有它的适应孔穴,情况不同,就应该随着病情慎重施针。不能实证用补法,虚证用泻法,这就是损不足而益有余,会加重病情。精气虚的病人,误泻五脏腧穴,会致人于死;阳气不足的病人,误泻三阳经的腧穴,可以致正气怯弱,神志错乱。总之,误泻阴经,耗伤了脏气,会致死;误泻阳经,损伤了阳气,会发狂证。用针不当的害处大致如此。针刺时,需要候气,如刺后尚未得气,不应拘泥手法次数的多少,必须等待经气到来;如果针已得气,就可去针不再刺了。九针各有不同适用证,针形也不一样,在使用时,要根据病情分别选用。总之,针刺的关键,是要得气,针下得气,必有疗效,疗效显著的,就像风吹云散,可以看到明朗的天空那样。这些都是针刺的道理。\\
黄帝说:我希望听到脏腑脉气所出之处的情况。\\
岐伯说:五脏经脉,各有井、荥、输、经、合五个腧穴,五五共二十五个腧穴;六腑经脉,各有井、荥、输、原、经、合六个腧穴,六六共三十六个腧穴,人体有十二经脉,每经各有一络,加上任督之脉各一络和脾之大络,共十五络,这二十七脉之气循行周身。脉气所出之处叫“井”,脉气流过之处叫“荥”,脉气灌注运输之处叫“输”,脉气通过之处叫“经”,脉气汇聚之处叫“合”。这二十七气出入于上下手足之间,它的脉气由始微而趋向正盛,最后入合于内。这二十七气流注运行都在这五腧之中,昼夜不息。人体关节等相交部位的间隙,共有三百六十五个会合处。知道这些要妙所在,就可以一言以蔽之,否则就漫无边际了。这里所说的“节”,都是血气游行出入和络脉渗灌诸节的地方,不是指皮肉筋骨说的。\\
在针刺时,注意察看患者的面色和眼神,可以了解血气的耗散与还复;从病人形态动静、声音变化,可以了解邪正虚实。然后右手推而进针,左手护持针身,等到针下得气,就可出针了。凡用针的时候,必先诊察脉象以了解脏气的和与不和,然后治疗。如五脏之气已绝于内,属阴虚,而用针反补在外的阳经,会使阳愈盛而阴愈虚,这叫“重竭”。重竭必死,死时安静。这是因为医生每违反经气补泻原则,误取腋和胸的腧穴,使脏气虚竭所致。五脏之气已虚于外,属阳虚,而用针反补在内阴经,阴愈盛而阳愈虚,引起四肢厥冷,这叫“逆厥”。逆厥必死,死时烦躁。这是误取四肢末端穴位,使阳气愈竭而致。针刺的要害,刺已中病而不出针就会耗伤精气;不中病而出针,会使邪气留滞不散。伤经气会加重病情而使人虚弱,气滞很容易发生痈疡。\\
五脏联系在外的六腑,六腑之外有十二原联属,十二原穴出于四肢关节,四关原穴主治五脏病变。所以五脏有病,就应该取十二原穴。因为十二原穴,是五脏聚三百六十五节经气而集中的地方。因此五脏有了病变,就反应到十二原,而十二原也各有所属之内脏,了解原穴的性质,观察它的反应,就可知五脏的病情。\\
肺属阳中之少阴,它的原穴是太渊,太渊左右共二穴。心属阳中之太阳,它的原穴是大陵,大陵左右共二穴。肝属阴中之少阳,它的原穴是太冲,太冲左右共二穴。脾属阴中之至阴,它的原穴是太白,太白左右共二穴。肾属阴中之太阴,它的原穴是太溪,太溪左右共二穴。膏的原穴,是任脉的鸠尾,鸠尾一穴。肓的原穴,是脐下的气海,气海一穴。以上五脏的原穴各有二穴,加膏、肓两原穴,共计十二原穴,以通脏腑表里之气,所以能治五脏六腑之病。大凡腹胀病,可取足三阳经的腧穴;飧泄病,可取足三阴经的腧穴治疗。\\
现在五脏有病,好比肌肉扎刺,物体被污染,绳索打了结,河水淤塞一样。但是,扎了刺虽然日子长,还可以拔掉;污染日子虽久,还可以洗净;结拴了好久,还可以解开;河道淤塞时间虽然长些,还可以疏通。有人认为久病就不能治愈,这样说不对。善于用针的医生治疗疾病,就像拔刺、涤污、解扣、疏淤一样。疾病的时间虽然很长,还可以达到治愈效果。说久病不能治,是因为未掌握针刺技术。\\
针刺热病,应当浅刺快刺,像用手探试沸腾的热水一样,一触即起;针刺寒冷病,应当深刺留针,像行人恋恋不愿离开一样。阴分有热病的,取阳明经的足三里穴。用针要正确,不能懈怠,邪气消退即出针,如邪气不退,还可再刺。如病位在上而病本属于脏,取足太阴经的合穴阴陵泉;病位在上而病本属于腑,取足少阳经的合穴阳陵泉。\\
本输第二 法地\\
黄帝问于岐伯曰:凡刺之道,必通十二经络之所终始,络脉之所别处,五输之所留,六腑之所与合,四时之所出入,五脏之所溜处,阔数之度,浅深之状,高下所至。愿闻其解。\\
岐伯曰:请言其次也。肺出于少商,少商者,手大指端内侧也,为井木;溜于鱼际,鱼际者,手鱼也,为荥;注于太渊,太渊,鱼后一寸陷者中也,为腧;行于经渠,经渠,寸口中也,动而不居,为经;入于尺泽,尺泽,肘中之动脉也,为合。手太阴经也。\\
心出于中冲,中冲,手中指之端也,为井木;溜于劳宫,劳宫,掌中中指本节之内间也,为荥;注于大陵,大陵,掌后两骨之间方下者也,为腧;行于间使,间使之道,两筋之间,三寸之中也,有过则至,无过则止,为经;入于曲泽,曲泽,肘内廉下陷者之中也,屈而得之,为合。手少阴也。\\
肝出于大敦,大敦者,足大指之端及三毛之中也,为井木;溜于行间,行间,足大指间也,为荥;注于太冲,太冲,行间上二寸陷者之中也,为腧;行于中封,中封,内踝之前一寸半,陷者之中,使逆则宛,使和则通,摇足而得之,为经;入于曲泉,曲泉,辅骨之下,大筋之上也,屈膝而得之,为合。足厥阴也。\\
脾出于隐白,隐白者,足大指之端内侧也,为井木;溜于大都,大都,本节之后,下陷者之中也,为荥;注于太白,太白,腕骨之下也,为输;行于商丘,商丘,内踝之下,陷者之中也,为经;入于阴之陵泉,阴之陵泉,辅骨之下,陷者之中也,伸而得之,为合。足太阴也。\\
肾出于涌泉,涌泉者,足心也,为井木;溜于然谷,然谷,然骨之下者也,为荥;注于太溪,太溪,内踝之后,跟骨之上,陷中者也,为输;行于复留,复留,上内踝二寸,动而不休,为经;入于阴谷,阴谷,辅骨之后,大筋之下,小筋之上也,按之应手,屈膝而得之,为合。足少阴经也。\\
膀胱出于至阴,至阴者,足小指之端也,为井金;溜于通谷,通谷,本节之前外侧也,为荥;注于束骨,束骨,本节之后,陷中者也,为腧;过于京骨,京骨,足外侧大骨之下,为原;行于昆仑,昆仑,在外踝之后,跟骨之上,为经;入于委中,委中,腘中央,为合;委而取之。足太阳也。\\
胆出于窍阴,窍阴者,足小指次指之端也,为井金;溜于侠溪,侠溪,足小指次指之间也,为荥;注于临泣,临泣,上行一寸半陷者中也,为腧;过于丘墟,丘墟,外踝之前,下陷者中也,为原;行于阳辅,阳辅,外踝之上,辅骨之前,及绝骨之端也,为经;入于阳之陵泉,阳之陵泉,在膝外陷者中也,为合,伸而得之。足少阳也。\\
胃出于厉兑,厉兑者,足大指内次指之端也,为井金;溜于内庭,内庭,次指外间也,为荥;注于陷谷,陷谷者,上中指内间上行二寸陷者中也,为腧;过于冲阳,冲阳,足跗上五寸陷者中也,为原;摇足而得之,行于解溪,解溪,上冲阳一寸半陷者中也,为经;入于下陵,下陵,膝下三寸,胻骨外三里也,为合;复下三里三寸为巨虚上廉,复下上廉三寸为巨虚下廉也;大肠属上,小肠属下,足阳明胃脉也。大肠小肠,皆属于胃。是足阳明也。\\
三焦者,上合手少阳,出于关冲,关冲者,手小指次指之端也,为井金;溜于液门,液门,小指次指之间也,为荥;注于中渚,中渚,本节之后陷者中也,为腧;过于阳池,阳池,在腕上陷者之中也,为原;行于支沟,支沟,上腕三寸,两骨之间陷者中也,为经;入于天井,天井,在肘外大骨之上陷者中也,为合,屈肘乃得之;三焦下腧,在于足大指之前,少阳之后,出于腘中外廉,名曰委阳,是太阳络也,手少阳经也。三焦者,足少阳、太阳之所将,太阳之别也,上踝五寸,别入贯腨肠,出于委阳,并太阳之正,入络膀胱,约下焦。实则闭癃,虚则遗溺;遗溺则补之,闭癃则泻之。\\
小肠者,上合手太阳,出于少泽,少泽,小指之端也,为井金;溜于前谷,前谷,在手外廉本节前陷者中也,为荥;注于后溪,后溪者,在手外侧本节之后也,为腧;过于腕骨,腕骨在手外侧腕骨之前,为原;行于阳谷,阳谷,在锐骨之下陷者中也,为经;入于小海,小海,在肘内大骨之外,去端半寸陷者中也,伸臂而得之,为合。手太阳经也。\\
大肠上合手阳明,出于商阳,商阳,大指次指之端也,为井金;溜于本节之前二间,为荥;注于本节之后三间,为腧;过于合谷,合谷,在大指歧骨之间,为原;行于阳溪,阳溪,在两筋间陷者中也,为经;入于曲池,在肘外辅骨陷者中也,屈臂而得之,为合。手阳明也。\\
是谓五脏六腑之腧,五五二十五腧,六六三十六腧也。六腑皆出足之三阳,上合于手者也。\\
缺盆之中,任脉也,名曰天突;一次任脉侧之动脉,足阳明也,名曰人迎;二次脉,手阳明也,名曰扶突;三次脉,手太阳也,名曰天窗;四次脉,足少阳也,名曰天容;五次脉,手少阳也,名曰天牖;六次脉,足太阳也,名曰天柱;七次脉,颈中央之脉,督脉也,名曰风府。腋内动脉,手太阴也,名曰天府;腋下三寸,手心主也,名曰天池。\\
刺上关者,呿不能欠;刺下关者,欠不能呿。刺犊鼻者,屈不能伸;刺两关者,伸不能屈。\\
足阳明,挟喉之动脉也,其腧在膺中。手阳明,次在其腧外,不至曲颊一寸。手太阳,当曲颊。足少阳,在耳下曲颊之后。手少阳,出耳后,上加完骨之上。足太阳,挟项大筋之中发际。阴尺动脉,在五里,五腧之禁也。\\
肺合大肠,大肠者,传道之腑。心合小肠,小肠者,受盛之腑。肝合胆,胆者,中精之腑。脾合胃,胃者,五谷之腑。肾合膀胱,膀胱者,津液之腑也。少阳属肾,肾上连肺,故将两脏。三焦者,中渎之腑也,水道出焉,属膀胱,是孤之腑也。是六腑之所与合者。\\
春取络脉诸荥大经分肉之间,甚者深取之,间者浅取之。夏取诸腧孙络肌肉皮肤之上。秋取诸合,余如春法。冬取诸井诸腧之分,欲深而留之。此四时之序,气之所处,病之所舍,藏之所宜。转筋者,立而取之,可令遂已。痿厥者,张而刺之,可令立快也。\\
黄帝问岐伯说:大凡针治的道理,必须通晓十二经脉循行的起止点,络脉从经脉别行的通路,井、荥、输、经、合五输穴的所在部位,六腑阳经与五脏阴经的表里相合处,随着四时的变化,血气出入流行的变化,五脏血气通过经脉流注于体表的部位,经络的宽窄度数和部位深浅,血气循行通及上下各处。希望听到对于这些问题的讲解。\\
岐伯说:请让我依次说明。肺经的血气,从少商穴发出,少商在手大指端内侧,为井穴,属木;流行到鱼际穴,鱼际在手鱼的边缘,为荥穴;灌注到太渊穴,太渊在手鱼后一寸的凹陷中,为腧穴;运行到经渠穴,经渠在腕后寸口之中,脉动而不停之处,为经穴;汇入到尺泽穴,尺泽在肘中动脉处,为合穴。这是手太阴经的五输穴。\\
心经的血气,从中冲穴发出,中冲在中指之端,为井穴,属木;流行到劳宫穴,劳宫在中指本节后手掌中间,为荥穴;灌注到大陵穴,大陵在掌后腕与臂两骨之间的凹陷中,为输穴;运行到间使穴,间使在掌后三寸两筋之间,当本经有病时,这一部位会出现反应,无病时就无反应,为经穴;汇入到曲泽穴,曲泽在肘内侧,屈肘时的凹陷处才能取得,为合穴。这是手少阴经的五输穴。\\
肝经的血气,从大敦穴发出,大敦在足大趾尖端及三毛之中,为井穴,属木;流行到行间穴,行间在足大趾次趾之间,为荥穴;灌注到太冲穴,太冲在行间穴上二寸凹陷之中,为输穴;运行到中封穴,中封在内踝前一寸半凹陷之中,令患者足尖逆而上举,可见有宛宛陷窝,再令患者将足恢复自如,则进针可通,或令患者将足微摇而取得,为经穴;汇入到曲泉穴,曲泉在内辅骨之下,大筋之上,屈膝取之即得,为合穴。这是足厥阴经的五腧穴。\\
脾经的血气,从隐白穴发出,隐白在足大趾端内侧,为井穴,属木;流行到大都穴,大都在本节之后的凹陷中,为荥穴;灌注到太白穴,太白在本节后核骨之下,为输穴;运行于商丘穴,商丘在内踝之下凹陷中,为经穴;汇入到阴陵泉穴,阴陵泉在内辅骨之下的凹陷中,伸足取之即得,为合穴。这是足太阴经的五输穴。\\
肾经的血气,从涌泉穴发出,涌泉在足底心,为井穴,属木;流行到然谷穴,然谷在足内踝前大骨下陷中,为荥穴;灌注到太溪穴,太溪在内踝骨后,跟骨之上凹陷中,跳动不止,为输穴;运行到复溜穴,复溜在内踝上二寸,为经穴;汇入到阴谷穴,阴谷在内辅骨之后,大筋之下,小筋之上,按之应手,屈膝取之即得,为合穴。这是足少阴经的五输穴。\\
膀胱经的血气,从至阴穴发出,至阴在足小趾端外侧,为井穴,属金;流行到通谷穴,通谷在小趾本节之前外侧,为荥穴;灌注到束骨穴,束骨在本节之后的凹陷中,为输穴;过于京骨穴,京骨在足外侧大骨之下,为原穴;运行到昆仑穴,昆仑在足外踝之后,跟骨之上,为经穴;汇入到委中穴,委中在膝弯中央,为合穴;可以屈而取之。这是足太阳经脉的五腧穴。\\
胆经的血气,从窍阴穴发出,窍阴在足小趾侧的次趾尖端,为井穴,属金;流行到夹溪穴,夹溪在足小趾与次趾之间,为荥穴;流注到临泣穴,临泣由夹溪再向上行一寸半处凹陷中,为输穴;过于丘墟穴,丘墟在外踝骨前之下凹陷中,为原穴;运行到阳辅穴,阳辅在外踝之上四寸余,辅骨的前方,绝骨的上端,为经穴;汇入到阳陵泉穴,阳陵泉在膝外侧凹陷中,为合穴,伸足取之而得。这是足少阳经的五腧穴。\\
胃经的血气,从厉兑穴发出,厉兑在足大趾侧的次趾尖端,为井穴,属金;流行到内庭穴,内庭在次趾外侧与中趾之间,为荥穴;灌注到陷谷穴,陷谷在中趾的内侧上行二寸的凹陷中,为输穴;过于冲阳穴,冲阳在足背上自趾缝向上约五寸的凹陷中,为原穴,摇足而取得之;运行到解溪穴,解溪在冲阳之上一寸半的凹陷中,为经穴;汇入到下陵穴,下陵就是在膝下三寸,胻骨外缘的三里穴,为合穴,再从三里下三寸,是上巨虚穴,大肠属之;自上巨虚再下三寸,为下巨虚穴,小肠属之。由于大肠小肠,在体内连属于胃腑之下,因而在经脉上也有连属足阳明胃脉之处。这是足阳明经的五腧穴。\\
三焦,上合手少阳经脉,其血气从关冲穴发出,关冲在无名指之端,为井穴,属金;流行到液门穴,液门在小指与无名指之间,为荥穴;灌注到中渚穴,中渚在无名指本节后之凹陷中,为输穴;过于阳池穴,阳池在腕上凹陷中,为原穴;运行到支沟穴,支沟在腕后三寸的两骨间凹陷中,为经穴;汇入到天井穴,天井在肘外大骨上的凹陷中,为合穴,屈肘取之即得;三焦之气输于下部者,在足太阳经之前,足少阳经之后,出于膝腘窝外缘,名叫委阳,是足太阳经的大络,又是手少阳的经脉。三焦虽属手少阳经,在下则有足少阳、太阳二经为之输给。所以又自足太阳经别出在外踝上五寸处,别入通过腿肚,出于委阳,与足太阳经的正脉相并,入腹内联络膀胱,约束着下焦。其气实则为小便不通,气虚则为遗尿;遗尿当用补法,小便不通当用泻法。\\
小肠,上合手太阳经脉,其血气从少泽穴发出,少泽在手小指外侧端,为井穴,属金;流行到前谷穴,前谷在手外侧本节前的凹陷中,为荥穴:灌注到后溪穴,后溪在手外侧小指本节的后方,为输穴;过于腕骨穴,腕骨在手外侧腕骨之前,为原穴;运行到阳谷穴,阳谷在腕后锐骨前下方的凹陷中,为经穴;汇入到小海穴,小海在肘内侧大骨之外,距离骨尖半寸处的凹陷中,伸臂取之即得,为合穴。这是手太阳经的五腧穴。\\
大肠,上合手阳明经脉,其血气从商阳穴发出,商阳在食指内侧端,为井穴,属金;流行到二间穴,二间在食指本节之前,为荥穴;灌注到三间穴,三间在本节之后,为输穴;过于合谷穴,合谷在大指次指歧骨之间,为原穴;运行到阳溪穴,阳溪在大指本节后,腕上两筋之间的凹陷中,为经穴;汇入到曲池穴,曲池在肘外侧辅骨的凹陷处,屈臂取之即得,为合穴。这是手阳明经的五腧穴。\\
以上所述,就是五脏六腑的腧穴,五脏阴经五五二十五个要穴,六腑阳经六六三十六个要穴。而六腑的血气,都从足三阳经脉出行,又上合于手三阳经脉。\\
左右两缺盆的中央,是任脉所行之处,有穴名天突;次于任脉后第一行的动脉,是足阳明经脉所行之处,有穴名人迎;第二行是手阳明经脉所行之处,有穴名扶突;第三行是手太阳经脉所行之处,有穴名天窗;第四行是足少阳经脉所行之处,有穴名天容;第五行是手少阳经脉所行之处,有穴名天牖;第六行是足太阳经脉所行之处,有穴名天柱;第七行在项中央,是督脉所行之处,有穴名风府。在腋下上臂内侧的动脉,是手太阴经脉所行之处,有穴名天府;在侧胸部当腋下三寸,是手厥阴心包经脉所行之处,有穴名天池。\\
刺上关穴,应张口而不能合口;刺下关穴,应合口而不能张口。刺犊鼻穴,应屈膝而不能伸足;刺内关与外关穴,要应伸手而不能弯屈。\\
足阳明经挟喉而行的动脉,其腧穴分布在胸之两旁膺部。手阳明经的腧穴,在它的外侧,距离曲颊一寸。手太阳经的腧穴,在曲颊处。足少阳经的腧穴,在耳下曲颊之后。手少阳经的腧穴,在耳后完骨之上。足太阳经的腧穴,在项后,挟大筋两旁发际下的凹陷中。阴尺动脉处,是手阳明的五里穴,不能屡刺,以防五腧穴所内通的血气竭尽。\\
肺合大肠,大肠是输送小肠已经消化之物的器官。心合小肠,小肠是受盛由胃而来之腐熟的水谷的器官。肝合胆,胆是居中接受精汁的器官。脾合胃,胃是消化水谷的器官。肾合膀胱,膀胱是贮存小便的器官。手少阳也属肾,肾又上连于肺,所以能统率三焦和膀胱两脏器。三焦,是像沟渎一样行水的器官,水道由此而出,属于膀胱,这是一个单独的器官。这就是六腑与五脏配合的情况。\\
春天治疗,取络穴、荥穴和经脉分肉之间,病重的应深取,病轻的应浅取。夏天治疗,取输穴、孙络,孙络在肌肉皮肤上。秋天治疗,除取合穴外,其他参照春天的刺法。冬天治疗,取井穴或输穴,应深刺,且留针。这是根据四时气候变化的规律,血气运行的深浅,病邪停留的部位,以及时令、经络皮肉等与脏腑的关系,来确定四时的刺法。治疗转筋,让病人站立而取穴施针,可使痉挛马上消除。治疗痿厥,让病人舒张四肢而取穴施针,可使病人马上感到轻快。\\
小针解第三 法人\\
所谓易陈者,易言也。难入者,难著于人也。粗守形者,守刺法也。上守神者,守人之血气,有余不足,可补泻也。神客者,正邪共会也。神者,正气也,客者,邪气也。在门者,邪循正气之所出入也。未睹其疾者,先知邪正,何经之疾也。恶知其原者,先知何经之病,所取之处也。\\
刺之微,在数迟者,徐疾之意也。粗守关者,守四肢而不知血气正邪之往来也。上守机者,知守气也。机之动,不离其空中者,知气之虚实,用针之徐疾也。空中之机,清净以微者,针以得气,密意守气勿失也。其来不可逢者,气盛不可补也。其往不可追者,气虚不可泻也。不可挂以发者,言气易失也。扣之不发者,言不知补泻之意也,血气已尽而气不下也。\\
知其往来者,知气之逆顺盛虚也。要与之期者,知气之可取之时也。粗之暗者,冥冥不知气之微密也。妙哉!工独有之者,尽知针意也。往者为逆者,言气之虚而小,小者,逆也。来者为顺者,言形气之平,平者,顺也。明知逆顺,正行无问者,言知所取之处也。迎而夺之者,泻也;追而济之者,补也。\\
所谓虚则实之者,气口虚而当补之也。满则泄之者,气口盛而当泻之也。宛陈则除之者,去血脉也。邪胜则虚之者,言诸经有盛者,皆泻其邪也。徐而疾则实者,言徐内而疾出也。疾而徐则虚者,言疾内而徐出也。言实与虚,若有若无者,言实者有气,虚者无气也。察后与先,若亡若存者,言气之虚实,补泻之先后也,察其气之已下,与常存也。为虚与实,若得若失者,言补者佖然若有得也,泻则怳然若有失也。\\
夫气之在脉也,邪气在上者,言邪气之中人也高,故邪气在上。浊气在中者,言水谷皆入于胃,其精气上注于肺,浊溜于肠胃,若寒温不适,饮食不节,而病生于肠胃,故命曰浊气在中也。清气在下者,言清湿地气之中人也,必从足始,故曰清气在下也。针陷脉,则邪气出者,取之上。针中脉,则浊气出者,取之阳明合也。针太深,则邪气反沉者,言浅浮之病,不欲深刺也,深则邪气从之入,故曰反沉也。皮肉筋脉,各有所处者,言经络各有所主也。取五脉者死,言病在中,气不足,但用针尽大泻其诸阴之脉也。取三脉者恇,言尽泻三阳之气,令病人恇然不复也。夺阴者死,言取尺之五里五往者也。夺阳者狂,正言也。\\
睹其色,察其目,知其散复,一其形,听其动静者,言上工知相五色于目,有知调尺寸小大缓急滑涩,以言所病也。知其邪正者,知论虚邪与正邪之风也。右主推之,左持而御之者,言持针而出入也。气至而去之者,言补泻气调而去之也。调气在于终始一者,持心也。节之交三百六十五会者,络脉之渗灌诸节者也。\\
所谓五脏之气,已绝于内者,脉口气,内绝不至,反取其外之病处,与阳经之合,有留针以致阳气,阳气至,则内重竭,重竭则死矣;其死也,无气以动,故静。所谓五脏之气,已绝于外者,脉口气,外绝不至,反取其四末之输,有留针以致其阴气,阴气至,则阳气反入,入则逆,逆则死矣;其死也,阴气有余,故躁。所以察其目者,五脏使五色循明,循明则声章。声章者,言声与平生异也。\\
所谓“易陈”,就是说起来容易。“难入”,是说它的精微之处,一般人难以掌握。“粗守形”是指粗浅的医生,只懂得机械地拘守刺法。“上守神”,是指高明的医生,能辨明病人血气虚实,作为补泻的根据。“神客”,是正气与邪气互相交争。“神”是人体的正气,“客”是致病的邪气。“在门”,是说邪气随着正气在腠理出入。“未睹其疾”,是说应先了解邪气、正气的盛衰,以及疾病属哪一经。“恶知其原”是说应先了解疾病在哪一经,及所应取的穴位。\\
“刺之微,在数迟”,是说针刺手法的微妙在于快慢的技巧。“粗守关”,是说粗浅的医生,只局限于四肢的腧穴,而不懂血气虚实与正邪的进退。“上守机”,是说高明的医生懂得气机的变化。“机之动,不离其空中”,是说要知道气机的虚实变化和用针的快慢。“空中之机,清净以微”,是说要了解用针的关键在于得气,应仔细注意和掌握气机变化,不可失去时机。“其来不可逢”,是说邪气正盛时不能用补法。“其往不可追”,是说正气已虚不能用泻法。“不可挂以发”,是说得气的时机容易失去。“扣之不发”,是说不知补泻的意义,使病人血气耗尽,而邪气未能祛除。\\
“知其往来”,是说知道气方来为顺为盛,气如已往为逆为虚。“要与之期”,是说要知道懂得候气,等待适当的针刺时机。“粗之暗”,是说粗浅的医生,昏昧无知,不懂得气的精微细密作用。“妙哉!工独有之”,是说高明的医生,完全掌握用针的意义。“往者为逆”,是说邪气已去时其人气虚而小,小就是逆。“来者为顺”,是说正气来时形气和平,平就是顺。“明知逆顺,正行无问”,是说要知道所取腧穴的确切部位。“迎而夺之”,是泻法;“追而济之”是补法。\\
所谓“虚则实之”,是说气口脉虚而当补。“满则泄之”,是说气口脉盛而当泻。“宛陈则除之”,是说去除脉中的淤血。“邪胜则虚之”,是说诸经有邪盛的应当泻其邪气。“徐而疾则实”,是进针慢而出针快,为补法。“疾而徐则虚”,是进针快而出针慢,为泻法。“言实与虚,若有若无”,是说用补法使正气恢复,用泻法使邪气消退。“察后与先,若亡若存”,是说了解气的虚实来决定补泻的先后,从此可以辨别邪气是已去还是存留。“为虚与实,若得若失”,是说用补法要使病人有饱满的感觉,好像有所得一般,用泻法要使病人有空虚的感觉,好像有所失一般。\\
所谓“气之在脉也,邪气在上”,是说邪气侵入经脉后,风邪多伤人的头部,所以说“邪气在上”。“浊气在中”,是说水谷入于胃后,它的精微之气上注于肺,它的浓浊部分留于肠胃,如果寒温不适宜,饮食不节制,那么肠胃中就会发病,所以说“浊气在中”。“清气在下”,是说清冷潮湿之气,使人发病,必从足部开始,所以说“清气在下”。“针陷脉,则邪气出”是说风邪伤人上部,要取头部的腧穴治疗。“针中脉,则浊气出”,是说肠胃的浊气发病,要取胃经的合穴足三里治疗。“针太深,则邪气反沉”,是说邪气浅浮之病,不能用深刺的针法,如误用了,反会使邪气随之深入,所以说“反沉”。“皮肉筋脉,各有所处”,是说皮肉筋脉各有一定的部位,也就是经络各有主治的所在。“取五脉者死”,是说病在内脏而元气不足的,仅用针竭力大泻五脏的腧穴,会造成死亡。“取三脉者恇”,是说竭力泻六腑腧穴之气,就会使病人精神怯弱,不易复元。“夺阴者死”,是说刺尺泽后的五里穴如果泻到五次,则必会泻尽脏阴之气而死。“夺阳者狂”,是说大泻三阳经脉之气,会使病人精神变化而成狂证。\\
“睹其色、察其目,知其散复,一其形,听其动静”,是说高明的医生,懂得观察患者颜面,和眼睛的色泽变化,又能够细察尺肤和寸口部位所表现出的小大、缓急、滑涩等脉象,从而说明患者所以发病的原因。“知其邪正”,是说知道患者所感受的是虚邪,还是正邪。“右主推之,左持而御之”,是说进针和出针的两种不同动作。“气至而去之者”,是说运用补泻手法,等到气机调和,就应该停针。“调气在于终始一者”,是说医生在运针调气时要专心致志,聚精会神,使神不外驰。“节之交三百六十五会”,是说周身三百六十五穴,乃是脉络中的气血渗灌各部的通会之处。\\
所谓“五脏之气,已绝于内”,是说脉口所主内部之气,已经断绝不至,反而取其表现在外表的病所与阳经的合穴,又用留针法以补益阳气,使阳气亢盛致使内脏阴气更加衰竭,阴气衰竭就要死亡;临死时,气口之脉没有气血鼓动,所以“安静”。所谓“五脏之气,已绝于外”,是说脉口所主外部之气,已经断绝不至,反而取其四肢之腧穴,又用留针法以补益阴气,阴气盛致使阳气内陷,阳气内陷是逆证,所以会死亡;临死时,由于阴气偏盛而暴露于外,所以“躁动”。所以要观察患者眼睛,是因为脏腑的精气,都输注于目,而使五色鲜明。脏腑的精气充足,五色鲜明,声音就高亢清晰;但患者的声音高亢清晰,与正常人是有区别的。\\
邪气脏腑病形第四 法时\\
黄帝问于岐伯曰:邪气之中人也,奈何?\\
岐伯答曰:邪气之中人高也。\\
黄帝曰:高下有度乎?\\
岐伯曰:身半已上者,邪中之也;身半已下者,湿中之也。故曰:邪之中人也,无有常。中于阴则溜于腑,中于阳则溜于经。\\
黄帝曰:阴之与阳也,异名同类,上下相会,经络之相贯,如环无端。邪之中人,或中于阴,或中于阳,上下左右,无有恒常,其故何也?\\
岐伯曰:诸阳之会,皆在于面。中人也,方乘虚时,及新用力,若饮食汗出,腠理开,而中于邪。中于面则下阳明,中于项则下太阳,中于颊则下少阳,中于膺背两胁亦中其经。\\
黄帝曰:其中于阴,奈何?\\
岐伯曰:中于阴者,常从臂胻始。夫臂与胻,其阴皮薄,其肉淖泽,故俱受于风,独伤其阴。\\
黄帝曰:此故伤脏乎?\\
岐伯答曰:身之中于风也,不必动脏。故邪入于阴经,则其脏气实,邪气入而不能客,故还之于腑。故中阳则溜于经,中阴则溜于腑。\\
黄帝曰:邪之中人脏,奈何?\\
岐伯曰:愁忧恐惧则伤心,形寒寒饮则伤肺。以其两寒相感,中外皆伤,故气逆而上行。有所堕坠,恶血留内,若有所大怒,气上而不下,积于胁下则伤肝。有所击仆,若醉入房,汗出当风则伤脾。有所用力举重,若入房过度,汗出浴水则伤肾。\\
黄帝曰:五脏之中风,奈何?\\
岐伯曰:阴阳俱感,邪气乃往。\\
黄帝曰:善哉。\\
黄帝问于岐伯曰:首面与身形也,属骨连筋,同血合于气耳。天寒则裂地凌冰,其卒寒,或手足懈惰,然而其面不衣,何也?\\
岐伯答曰:十二经脉,三百六十五络,其血气皆上于面而走空窍,其精阳气上走于目而为睛,其别气走于耳而为听,其宗气上出于鼻而为臭,其浊气出于胃走唇舌而为味。其气之津液皆上熏于面,而皮又厚,其肉坚,故热甚,寒不能胜之也。\\
黄帝曰:邪之中人,其病形何如?\\
岐伯曰:虚邪之中身也,洒淅动形;正邪之中人也微,先见于色,不知于身,若有若无,若亡若存,有形无形,莫知其情。\\
黄帝曰:善哉。\\
黄帝问于岐伯曰:余闻之,见其色,知其病,命曰明;按其脉,知其病,命曰神;问其病,知其处,命曰工。余愿闻见而知之,按而得之,问而极之,为之奈何?\\
岐伯答曰:夫色脉与尺之相应也,如桴鼓影响之相应也,不得相失也。此亦本末根叶之殊候也,故根死则叶枯矣。色脉形肉不得相失也,故知一则为工,知二则为神,知三则神且明矣。\\
黄帝曰:愿卒闻之。\\
岐伯答曰:色青者,其脉弦也;赤者,其脉钩也;黄者,其脉代也;白者,其脉毛也;黑者,其脉石也。见其色而不得其脉,反得其相胜之脉则死矣;得其相生之脉则病已矣。\\
黄帝问于岐伯曰:五脏之所生,变化之病形,何如?\\
岐伯答曰:先定其五色五脉之应,其病乃可别也。\\
黄帝曰:色脉已定,别之奈何?\\
岐伯曰:调其脉之缓急、小大、滑涩,而病变定矣。\\
黄帝曰:调之奈何?\\
岐伯答曰:脉急者,尺之皮肤亦急;脉缓者,尺之皮肤亦缓;脉小者,尺之皮肤亦减而少;脉大者,尺之皮肤亦贲而起;脉滑者,尺之皮肤亦滑;脉涩者,尺之皮肤亦涩。凡此变者,有微有甚,故善调尺者,不待于寸;善调脉者,不待于色。能参合而行之者,可以为上工,上工十全九;行二者为中工,中工十全七;行一者为下工,下工十全六。\\
黄帝曰:请问脉之缓急、小大、滑涩之病形,何如?\\
岐伯曰:臣请言五脏之病变也。心脉急甚者,为瘛疭;微急,为心痛引背,食不下。缓甚,为狂笑;微缓,为伏梁,在心下,上下行,时唾血。大甚,为喉吤;微大,为心痹引背,善泪出。小甚,为善哕;微小,为消瘅。滑甚,为善渴;微滑,为心疝引脐,小腹鸣。涩甚,为瘖;微涩,为血溢,维厥,耳鸣,巅疾。\\
肺脉急甚,为癫疾;微急,为肺寒热,怠惰、咳唾血、引腰背胸,若鼻息肉不通。缓甚,为多汗;微缓,为痿瘘、偏风,头以下汗出,不可止。大甚,为胫肿;微大,为肺痹,引胸背,起恶日光。小甚,为泄;微小,为消瘅。滑甚,为息贲上气;微滑,为上下出血。涩甚,为呕血;微涩,为鼠瘘,在颈支腋之间,下不胜其上,其应善痠矣。\\
肝脉急甚,为恶言;微急,为肥气,在胁下,若覆杯。缓甚,为善呕;微缓,为水瘕痹也。大甚,为内痈,善呕,衄;微大,为肝痹,阴缩,咳引小腹。小甚,为多饮;微小,为消瘅。滑甚,为毌疝;微滑,为遗溺。涩甚,为溢饮;微涩,为瘛挛筋痹。\\
脾脉急甚,为瘛疭;微急,为膈中,食饮入而还出,后沃沫。缓甚,为痿厥;微缓,为风痿,四肢不用,心慧然若无病。大甚,为击仆;微大,为疝气,腹里大脓血,在肠胃之外。小甚,为寒热;微小,为消瘅。滑甚,为毌癃。微滑,为虫毒蛕蝎,腹热。涩甚,为肠毌;微涩,为内毌,多下脓血。\\
肾脉急甚,为骨癫疾;微急,为沉厥,奔豚,足不收,不得前后。缓甚,为折脊;微缓,为洞,洞者,食不化,下嗌还出。大甚,为阴痿;微大,为石水,起脐已下至小腹,腄腄然,上至胃脘,死不治。小甚,为洞泄;微小,为消瘅。滑甚,为癃毌;微滑,为骨痿,坐不能起,起则目无所见;涩甚,为大痈;微涩,为不月,沉痔。\\
黄帝曰:病之六变,刺之奈何?\\
岐伯答曰:诸急者多寒,缓者多热,大者多气少血,小者血气皆少,滑者阳气盛、微有热,涩者多血少气、微有寒。是故刺急者,深内而久留之;刺缓者,浅内而疾发针,以去其热;刺大者,微泻其气,无出其血;刺滑者,疾发针而浅内之,以泻其阳气而去其热;刺涩者,必中其脉,随其逆顺而久留之,必先按而循之,已发针,疾按其痏,无令其血出,以和其脉;诸小者,阴阳形气俱不足,勿取以针,而调以甘药也。\\
黄帝曰:余闻五脏六腑之气,荥输所入为合,令何道从入,入安连过?愿闻其故。\\
岐伯答曰:此阳脉之别入于内,属于腑者也。\\
黄帝曰:荥输与合,各有名乎?\\
岐伯答曰:荥输治外经,合治内腑。\\
黄帝曰:治内腑奈何?\\
岐伯曰:取之于合。\\
黄帝曰:合各有名乎?\\
岐伯答曰:胃合于三里,大肠合入于巨虚上廉,小肠合入于巨虚下廉,三焦合入于委阳,膀胱合入于委中央,胆合入于阳陵泉。\\
黄帝曰:取之奈何?\\
岐伯答曰:取之三里者,低跗取之;巨虚者,举足取之;委阳者,屈伸而索之;委中者,屈而取之;阳陵泉者,正竖膝,予之齐,下至委阳之阳取之;取诸外经者,揄申而从之。\\
黄帝曰:愿闻六腑之病。\\
岐伯答曰:面热者,足阳明病;鱼络血者,手阳明病;两跗之上脉竖陷者,足阳明病。此胃脉也。\\
大肠病者,肠中切痛而鸣濯濯,冬日重感于寒即泄,当脐而痛,不能久立。与胃同候,取巨虚上廉。\\
胃病者,腹尒胀,胃脘当心而痛,上支两胁,膈咽不通,食饮不下,取之三里也。\\
小肠病者,小腹痛,腰脊控睾而痛,时窘之后,当耳前热,若寒甚,若独肩上热甚,及手小指次指之间热,若脉陷者,此其候也。手太阳病也,取之巨虚下廉。\\
三焦病者,腹气满,小腹尤坚,不得小便,窘急,溢则为水留,即为胀。候在足太阳之外大络,大络在太阳少阳之间,亦见于脉,取委阳。\\
膀胱病者,小腹偏肿而痛,以手按之,即欲小便而不得,肩上热若脉陷,及足小指外廉及胫踝后皆热。若脉陷,取委中央。\\
胆病者,善太息,口苦,呕宿汁,心下澹澹恐人将捕之,嗌中吤吤然,数唾。在足少阳之本末,亦视其脉之陷下者灸之,其寒热者取阳陵泉。\\
黄帝曰:刺之有道乎?\\
岐伯答曰:刺此者,必中气穴,无中肉节。中气穴则针游于巷,中肉节即皮肤痛。补泻反则病益笃,中筋则筋缓,邪气不出,与其真相搏,乱而不去,反还内著。用针不审,以顺为逆也。\\
黄帝问岐伯说:外邪伤人的情况怎样呢?\\
岐伯回答说:邪气伤人会在人体的上部。\\
黄帝又问说:部位的上下,有一定的常规吗?\\
岐伯说:上半身发病的,是受了风寒外邪所致;下半身发病的,是受了湿邪所致。因此说:外邪侵犯人体,没有固定部位。外邪侵犯阴经,会流传到六腑,外邪侵犯阳经,也可能流传在本经的通路而发病。\\
黄帝问:经脉的阴和阳,名称虽然不同,其实同属于经络系统,上下互相会合,经络之间彼此联贯,如圆环没有开端。外邪伤人,有的侵入阴经,有的侵入阳经,或上、或下、或左、或右,没有固定部位,是什么道理呢?\\
岐伯说:手足的三阳经,都聚合到头面部。邪气伤人,往往趁着体虚之时,以及刚劳累用力后,或热饮热食出了汗,腠理开泄,而被邪气侵袭。邪气侵入面部,就会下行至足阳明胃经;邪气侵入项部,就会下行至足太阳膀胱经;邪气侵入颊部,就会下行至足少阳胆经;如果邪气侵入胸膺、脊背、两胁,也会分别下行它所属的阳明经、太阳经、少阳经。\\
黄帝问道:邪气侵入阴经,怎么样呢?\\
岐伯回答说:邪气侵入阴经,往往是从手臂或足胫开始的。因为手臂和足胫内侧的皮肤较薄,肌肉柔润,所以身体各部同样受了风邪,而这些部位最易受伤。\\
黄帝又问道:这邪气也会伤及五脏吗?\\
岐伯回答说:人身感受风邪,不一定会伤及五脏。假若外邪侵入了阴经,而脏气充实,就是邪气入里也留不住,必定还归六腑。因此阳经受了邪,就流注于本经而发病;阴经受了邪,就会流注于六腑而发病。\\
黄帝问:邪气伤及五脏是怎样的?\\
岐伯说:愁忧恐惧会使心脏受伤,形体受寒,又喝了冷水,就会使肺脏受伤。因为两种寒邪交感,内外受伤,就会发生肺气上逆的病变。如从高处跌坠,淤血留滞体内,又因大怒刺激,气上冲而不下,郁结胁下,就会伤肝。被人打击跌倒,或醉后行房,出汗冒风,就会伤脾。如用力举重,或房事过度,或出汗以后,浴于水中,就会伤肾。\\
黄帝又问:五脏为风邪所伤,为什么呢?\\
岐伯说:一定是内脏先伤再感受外邪,内外之邪结合,风邪才能侵入内脏。\\
黄帝说:说得真好。\\
黄帝问岐伯说:人的头面和全身形体,都是由筋骨支撑的,由气血滋养。当天寒地裂,滴水成冰的时候,如突然感受寒气,手足就会瑟缩不伸,麻木不灵,可是面部却不用衣服御寒,这是什么缘故?\\
岐伯回答说:周身十二经脉和三百六十五络,所有血气都上行达到头面部,分别流入各个孔窍,那精阳之气上注于目,使眼睛能看;那旁行的经气上达于耳,使耳能听;那宗气上出于鼻,使鼻能嗅;那由胃生出来的谷气,上走唇舌,使唇舌有味觉。所有这些气和津液,都上行熏蒸于面部,面部的皮又厚,肌肉坚实,因此面上的阳热已极,就是天气极寒冷也能适应。\\
黄帝问:外邪侵犯人体,发病的症状是怎样的呢?\\
岐伯说:虚邪伤了人,病人会战栗恶寒;正邪伤人发病比较轻微,先看到气色方面有点变异,身上没有什么感觉,像有病又像没有病,似有症状又似没有症状,不容易知道它的病情。\\
黄帝说:讲得好。\\
黄帝问岐伯说:我听说医生看病人气色,就知道病情的叫“明”;按病人脉象,就知道病情的叫“神”;问病情,就知道病情的叫“工”。我希望听一下,闻声、望色就能知道病情,切脉就能得到病况,问病就可彻底了解病苦的所在,怎么做才能有如此水平呢?\\
岐伯回答说:病人的气色、脉象、尺肤都与疾病有相应关系,如响随鼓、如影随形,不会有差错。这也像树木的根本和枝末一样,根衰败,枝叶必然枯槁。人的面色,脉象与皮肉外形的表现是不会不一致的。知其一为工,知其二为神,知其三就是神医了。\\
黄帝说:希望听你详尽解释。\\
岐伯回答说:面色青的,脉象应弦;面色红的,脉象应钩;面色黄的,脉象应代;面色白的,脉象应毛;面色黑的,脉象应石。如果看到面色与脉象不合,反而诊得相克脉象,就会死亡;若能诊得相生脉象,疾病就会痊愈。\\
黄帝问岐伯说:五脏所生疾病的变化和表现是怎样的?\\
岐伯回答说:必先确定五色和五脉的相应关系,疾病就可以区别。\\
黄帝又问:气色和脉象已经确定了,怎么区别病情呢?\\
岐伯说:只要诊察出脉的缓急、小大、滑涩,病变就确定了。\\
黄帝问:诊察的方法如何呢?\\
岐伯回答说:脉急促的,尺肤的皮肤也紧急;脉徐缓的,尺肤的皮肤也弛缓;脉象小的,尺肤的皮肤也瘦小;脉象大的,尺肤的皮肤也大而突起;脉象滑的,尺肤的皮肤也滑润;脉象涩的,尺肤的皮肤也涩滞。以上六种变化,有轻有重,所以善于诊察尺肤的,不必等诊寸口脉;善于诊察脉象的,不必等望色。能够察色、辨脉、观察尺肤三者配合起来而进行诊断的,称为上工,上工治愈十分之九;能够运用两种方法诊察的,称为中工,中工治愈十分之七;仅能运用一种方法进行诊察的,称为下工,下工治愈十分之六。\\
黄帝问:请问缓急、小大、滑涩的脉象,所主的病状是怎样的呢?\\
岐伯说:请让我谈谈五脏的病变。心脉急甚为手足抽搐;微急为心痛牵引脊背,饮食不下。心脉缓甚为神志失常的狂笑;微缓为久积之伏梁病,其症状是:在心下即上腹部,上下移动,常有唾血。心脉大甚为喉中如有物梗阻;微大为心痹,疼痛牵引肩背,时时流泪。心脉小甚为呃逆;微小为消谷善饥的消渴病。心脉滑甚为消渴;微滑为心疝,疼痛牵引脐部,小腹鸣响。心脉涩甚为瘖哑不能言;微涩为出血,四肢厥逆,耳鸣,头部疾病。\\
肺脉急甚为癫疾;微急为肺有寒热:懈怠乏力,咳嗽咳血,疼痛牵引胸部和腰背部,或鼻中息肉阻塞,呼吸不畅。肺脉缓甚为多汗;微缓为痿瘘,半身不遂,头部以下汗出不止。肺脉大甚为足胫肿;微大为肺痹,牵引胸背胀痛,怕见日光。肺脉小甚为泄泻;微小为消瘅。肺脉滑甚为息贲,咳喘气逆;微滑在上为吐、衄血,在下为便血,肺脉涩甚为呕血;微涩为鼠瘘,发于颈项与腋下,下肢软弱,难以支撑躯体,四肢酸楚难耐。\\
肝脉急甚为恶言恶语;微急为肥气病,位于胁下,状如覆杯。肝脉缓甚为呕吐;微缓为水积胸胁,小便不通。肝脉大甚为内有痈肿,常呕吐、衄血;微大为肝痹,阴器收缩,咳嗽牵引小腹作痛;肝脉小甚为多饮,微小为消谷善饥的消瘅病。肝脉滑甚为阴囊肿大的厊疝;微滑为遗尿病。肝脉涩甚为溢饮水肿;微涩为筋脉拘挛不舒的筋痹。\\
脾脉急甚为四肢抽搐;微急为食入而吐的膈中病,大便多泡沫。脾脉缓甚为四肢痿软无力,四肢厥冷;微缓为风痿病,四肢痿废不用,但神志清楚如无病之人。脾脉大甚为突然仆倒的厥病;微大为痞气病,腹中多脓血,聚在肠胃之外。脾脉小甚为寒热病;微小为内热消瘅。脾脉滑甚为阴囊肿大的厊)疝和小便不通的癃闭。微滑为肠中有蛔虫等寄生虫病,腹中发热。脾脉涩甚为广肠脱出的肠伒)病;微涩是肠内溃脓,大便下脓血。\\
肾脉急甚为邪深至骨的骨癫疾;微急为下肢沉重逆冷,发为奔豚,两足伸不能屈,大小便不通。肾脉缓甚为腰脊痛如折;微缓为洞泄病,洞泄的症状是饮食不化,食入之后即从大便排出。肾脉大甚为阴痿不起;微大为石水病,从脐以下至小腹部胀满下坠,上至胃脘不适,预后不良。肾脉小甚为洞泄病;微小为消瘅。肾脉滑甚为小便不通,或为厊疝;微滑为骨痿病,可坐而不能站起,站起则视物不清;肾脉涩甚为大的痈肿;微涩为月经不行,或痔疾日久不愈。\\
黄帝问道:疾病出现六种脉象变化,怎样针刺呢?\\
岐伯回答说:凡是脉象紧的多属寒,脉象缓的多属热,脉象大的多属气有余而血不足,脉象小的多属气血都不足,脉象滑的属阳气盛而微有热,脉象涩的血少气少而微有寒。因此,在针刺急脉的病变,进针要深,留针时间要长;针刺缓脉的病变,进针应该浅,出针要快,以散其热;针刺大脉的病变,略微泻其气,不能出血;针刺滑脉的病变,应快出针,浅刺,以泻阳气,排除热邪;针刺涩脉的病变,必须刺中经脉,随着气行的逆顺方向行针,留针时间要长,还要先按摩经脉,使脉气舒缓,出针以后,赶快按住针孔,不使出血,以调和经脉;凡是脉象小的,阴阳形气都虚弱,不宜用针刺,而用缓和之药调治。\\
黄帝说:我听说五脏六腑的脉气,都出于井穴,从荥、输而进入合穴。这是从哪条经脉进入合穴的?进入后又和哪些脏腑经脉有联系呢?希望听听其中的缘故。\\
岐伯回答说:这就是手足阳经,由别络进入内部而又属于六腑的。\\
黄帝问:荥、输与合穴,在治疗上各有一定的作用吗?\\
岐伯说:荥、输的脉气浮浅,可以治外经的病,合的脉气深入,可以治疗内腑的病。\\
黄帝问:治疗体内的腑病,怎样取穴呢?\\
岐伯说:应取合穴。\\
黄帝问:合穴各有名称吗?\\
岐伯回答说:胃的合穴在三里,大肠的合穴在巨虚上廉,小肠的合穴在巨虚下廉,三焦的合穴在委阳,膀胱的合穴在委中,胆的合穴在阳陵泉。\\
黄帝说:怎样取合穴呢?\\
岐伯回答说:取三里穴,应足背低平;取巨虚穴,应举足;委阳穴,应先屈后伸下肢取穴;委中穴,应屈膝取穴;阳陵泉穴,应正立竖膝使两膝齐平,至委中的外侧取穴;凡取治在外经脉的病变,应该用或摇或伸的方式取穴。\\
黄帝说:希望听一下六腑的病变。\\
岐伯回答说:面部发热是足阳明的病变;手鱼部出现郁滞的血斑是手阳明的病变;足背的冲阳脉出现坚实而极隐伏的现象,也是足阳明的病变。这是胃的经脉。\\
大肠病,肠中痛如刀割,阵阵肠鸣,冬天再感受寒邪,就会泄泻,当脐部疼痛,痛时不能久立。肠与胃有密切联系,可取胃经的上巨虚穴治疗。\\
胃病,会出现腹胀满,胃脘当心部位疼痛,支撑两胁,胸膈和咽喉间不通,饮食不下,可取足三里穴治疗。\\
小肠病,少腹作痛,腰背牵引睾丸疼痛,大便窘急,觉得耳前发热,或发冷,或只是肩上很热,以及手小指与无名指间发热,若络脉虚陷不起,这就是手太阳小肠经病变的症候。手太阳小肠病变,可取下巨虚穴治疗。\\
三焦病,腹部胀满,小腹胀得尤甚,小便不通,感到窘迫难受,水溢于皮下成为水肿,留在腹部为胀病。三焦病候也会呈现在足太阳外侧的大络上,这络脉在太阳经和少阳经之间,三焦有病,此处脉现异常,取委阳治疗。\\
膀胱病,小腹部偏肿而痛,用手按揉痛处,就要小便,又尿不出来,肩部发热,或脉陷不起,以及足小趾外侧,胫骨和足踝后都显有热象。若络脉虚陷不起,可取委中穴治疗。\\
胆病,经常叹气,口苦,呕出苦水,心跳不安,好像怕人逮捕他一样,咽喉里如物梗塞,频频咳嗽、吐唾沫。在足少阳经起点至终点的循行通路上,也可以出现络脉陷下的情况,可以用灸法治疗;如胆病而有寒热现象的,可取足少阳经的合穴阳陵泉刺治。\\
黄帝问:针刺有一定的规律吗?\\
岐伯回答说:针刺这些穴位,一定要刺中气穴,不可刺中肉或刺中节。因为刺中气穴,则经气运行于脉道之内,经脉就通了;如果误中肉节,只能损伤好肉,使皮肤疼痛。如果补泻手法用反了,就会加重病情;如果误刺中筋,筋就会弛缓,邪气也出不去,与真气相争,由于邪气扰乱不去,反回到内里为病。这都是用针不审慎,反顺为逆的恶果。\\
卷二\\
根结第五 法音\\
岐伯曰:天地相感,寒暖相移,阴阳之道,孰少孰多?阴道偶,阳道奇。发于春夏,阴气少,阳气多,阴阳不调,何补何泻?发于秋冬,阳气少,阴气多,阴气盛而阳气衰,故茎叶枯槁,湿雨下归,阴阳相移,何泻何补?\\
奇邪离经,不可胜数,不知根结,五脏六腑,折关败枢,开阖而走,阴阳大失,不可复取。九针之玄,要在终始。故能知终始,一言而毕,不知终始,针道咸绝。\\
太阳根于至阴,结于命门。命门者,目也。阳明根于厉兑,结于颡大者。颡大者,钳耳也。少阳根于窍阴,结于窗笼。窗笼者,耳中也。太阳为开,阳明为阖,少阳为枢。故开折,则肉节渎,而暴病起矣。故暴病者,取之太阳,视有余不足。渎者,皮肉宛膲而弱也。阖折,则气无所止息,而痿疾起矣。故痿疾者,取之阳明,视有余不足。无所止息者,真气稽留,邪气居之。枢折,即骨繇而不安于地。故骨繇者,取之少阳,视有余不足,骨繇者,节缓而不收。所谓骨繇者,摇故也。当穷其本也。\\
太阴根于隐白,结于太仓。少阴根于涌泉,结于廉泉。厥阴根于大敦,结于玉英,络于膻中。太阴为开,厥阴为阖,少阴为枢。故开折,则仓廪无所输,膈洞,膈洞者,取之太阴,视有余不足。故开折者,气不足而生病也。阖折,即气绝而喜悲,悲者,取之厥阴,视有余不足。枢折,则脉有所结而不通,不通者,取之少阴,视有余不足。有结者,皆取之。\\
足太阳根于至阴,溜于京骨,注于昆仑,入于天柱、飞扬也。\\
足少阳根于窍阴,溜于丘墟,注于阳辅,入于天容、光明也。\\
足阳明根于厉兑,溜于冲阳,注于下陵,入于人迎、丰隆也。\\
手太阳根于少泽,溜于阳谷,注于少海,入于天窗、支正也。\\
手少阳根于关冲,溜于阳池,注于支沟,入于天牖、外关也。\\
手阳明根于商阳,溜于合谷,注于阳谿,入于扶突、偏历也。\\
此所谓十二经者,盛络皆当取之。\\
一日一夜五十营,以营五脏之精,不应数者,名曰狂生。所谓五十营者,五脏皆受气。持其脉口,数其至也。五十动而不一代者,五脏皆受气。四十动一代者,一脏无气;三十动一代者,二脏无气;二十动一代者,三脏无气;十动一代者,四脏无气;不满十动一代者,五脏无气。予之短期,要在《终始》。所谓五十动而不一代者,以为常,以知五脏之期。予之短期者,乍数乍疏也。\\
黄帝曰:逆顺五体者,言人骨节之小大,肉之坚脆,皮之厚薄,血之清浊,气之滑涩,脉之长短,血之多少,经络之数。余已知之矣,此皆布衣匹夫之士也。夫王公大人,血食之君,身体柔脆,肌肉软弱,血气慓悍滑利。其刺之徐疾,浅深多少,可得同之乎?\\
岐伯答曰:膏粱菽藿之味,何可同也。气滑即出疾,其气涩则出迟,气悍则针小而入浅,气涩则针大而入深,深则欲留,浅则欲疾。以此观之,刺布衣者深以留之,刺大人者微以徐之,此皆因气慓悍滑利也。\\
黄帝曰:形气之逆顺,奈何?\\
岐伯曰:形气不足,病气有余,是邪胜也,急泻之。形气有余,病气不足,急补之。形气不足,病气不足,此阴阳气俱不足也。不可刺之。刺之,则重不足,重不足则阴阳俱竭,血气皆尽,五脏空虚,筋骨髓枯,老者绝灭,壮者不复矣。形气有余,病气有余,此谓阴阳俱有余也,急泻其邪,调其虚实。故曰有余者泻之,不足者补之,此之谓也。\\
故曰:刺不知逆顺,真邪相搏。满而补之,则阴阳四溢,肠胃充郭,肝肺内尒,阴阳相错。虚而泻之,则经脉空虚,血气竭枯,肠胃聂辟,皮肤薄著,毛腠夭膲,予之死期。故曰:用针之要,在于知调阴与阳,调阴与阳。精气乃光,合形与气,使神内藏。故曰:上工平气,中工乱脉,下工绝气危生。故曰:下工不可不慎也。必审五脏变化之病,五脉之应,经络之实虚,皮之柔粗,而后取之也。\\
岐伯说:天地之气互相交感,气候的寒暖相互转换,阴阳变化的规律,究竟谁多谁少?阴是双数,阳是单数。如果疾病发生在春夏,则阴气少而阳气多,阴阳之气不协调,补哪一经?泻哪一经?病发生在秋冬,则阳气少而阴气多。自然界阴气盛而阳气衰,所以植物茎叶枯槁,雨水湿气下归于根部,在疾病上表现为阴阳之气相移,补哪一经?泻哪一经?\\
异常的邪气,侵入经脉,所致病变不能尽数。针刺时,如果不懂得经脉的起始与终止,及五脏六腑的相互关系,就会导致六经的关守折损、枢纽败坏,以至开合不当,真气走泄,阴阳之气大量损耗,用针刺治也不能起作用了。因此,运用九针的奥妙,关键在于懂得经脉的起止。能懂得经脉起止的,一句话就能说完,不懂得经脉的起止,针治的道理就完全无从谈起了。\\
足太阳经脉,起始于至阴穴,终结于命门。“命门”就是眼睛的睛明穴。足阳明经脉,起始于厉兑,终结于颡大。“颡大”就是指额之大角,钳束于耳上之头维穴。足少阳经脉,起始于足窍阴穴,终结于窗笼。“窗笼”就是耳中的听宫穴。三阳经中太阳为关,阳明为阖,少阳为枢。如果“关”的作用失常,则肌肉关节异常而生暴病。所以暴病可取足太阳经脉,根据虚实来治疗。“渎”是指皮肉消瘦干枯。如果“阖”的作用失常,则精气无处止息而发生痿疾,所以痿疾可取足阳明经脉,根据虚实来治疗。“无所止息”是指真气留滞,邪气盘踞于内,而发为痿疾。如果“枢”的作用失常,则骨动摇而不便行走站立,所以“骨繇”可取足少阳经脉,根据虚实来治疗。“骨繇”是骨节弛缓而失去约束。所以叫作骨繇,就是因为骨节摇动的缘故。以上各病,都必须彻底弄清它的病因,才能正确治疗。\\
足太阴经,起始于隐白穴,终结于中脘穴。足少阴经脉,起始于涌泉穴,终结于廉泉穴。足厥阴经,起始于大敦穴,终结于玉堂穴,且有支脉连络膻中。三阴经中太阴为关,厥阴为阖,少阴为枢。如果“关”的作用失常,则仓廪不能运化而病膈洞,膈洞,当取足太阴经脉,根据虚实来治疗。所谓“关折”,是由脾胃气虚不能运化水谷而致。如果“阖”的作用失常,则气机不畅而易生悲哀之感,悲哀的,当取足厥阴经脉,根据虚实来治疗。如果“枢”的作用失常,则脉道淤结不通,淤结不通的,当取足少阴经脉,根据虚实来治疗。凡脉道淤结的都可取治。\\
足太阳经脉,起始于至阴穴,流行到京骨穴,灌注到昆仑穴,上入项后之天柱穴而走头部,下入下肢之飞扬穴而交足少阴经。\\
足少阳经脉,起始于足窍阴穴,流行到丘墟穴,灌注到阳辅穴,上入颈部之天冲穴而走头部,下入下肢之光明穴而交足厥阴经。\\
足阳明经脉,起始于厉兑穴,流行到冲阳穴,灌注到解谿穴,上入颈部之人迎穴而走面部,下入丰隆穴而交足太阴经。\\
手太阳经脉,起始于少泽穴,流行到阳谷穴,灌注到小海穴,上入颈部之天窗穴而走头部,下入下肢之支正穴而交手少阴经。\\
手少阳经脉,起始于关冲穴,流行到阳池穴,灌注到支沟穴,上入项部之天牖穴而走头部,下入上肢之外关穴而交手厥阴经。\\
手阳明经脉,起始于商阳穴,流行到合谷穴,灌注到阳谿穴,上入颈部之扶突穴而走面部,下入上肢的偏历穴而交手太阴经。\\
这就是十二经脉的根流注入的穴位,凡因邪气侵入而经络盛满的,都可泻治这些穴位。\\
人体经脉血气,一昼夜共循环五十周次,以运营五脏精气,如果不符合此数,就会生病。所谓“五十营”,是五脏都受到血气的灌注营养。这可以从诊察寸口的脉搏,计数搏动次数而测知脏气盛衰。如脉跳五十次,没有一次停止的,是五脏精气都充足。在四十跳中有一次停止的,就有一脏气衰;在三十跳中有一次停止的,就是两脏气衰;在二十跳中有一次停止的,就是三脏气衰;在十跳中有一次停止的,就是四脏气衰;如不满十跳而有一次停止的,是五脏之气都已亏虚。据此,可以判断死期。其主要内容,已见《终始》篇中。所谓脉五十跳没有一次停止的,就是正常脉象,由此可测知五脏经气的状况。如果脉跳忽快忽慢,可以预测临近死期了。\\
黄帝说:“逆顺五体”,是讲骨节有大有小,肌肉有坚有脆,皮肤有厚有薄,血液有清有浊,气的运行有滑有涩,经脉有长有短,营血有多有少,以及经络的数目。我已经懂得了,这些都是一般百姓的情况。至于王公大人,肉食者,身体柔脆,肌肉软弱,血气运行疾速滑利。给他们针刺的快慢、浅深、多少,与百姓是否相同呢?\\
岐伯说:饮食肥甘厚味的王公大人和吃粗粮的百姓,怎么能相同呢!气行滑利的出针应快,气行涩滞的出针应慢,气轻浮的用小针而进针宜浅,气涩滞的用大针而进针宜深,深刺应留针,浅刺应出针快。由此看来,刺百姓时,应深而留针;刺大人时,应轻而慢,这是因为这些人气行滑利。\\
黄帝问:人体形与气的顺逆情况如何?\\
岐伯说:外形气息不足,而病气有余的,是邪偏胜,应该急泻邪气。外形气息有余,而病气不足的,应该急补正气。形体气息不足,病气也不足,这是正邪都不足。不能用针刺。如用针治,则加重衰弱,加重衰弱则阴阳表里的血气都将枯竭,五脏空虚,筋骨精髓枯槁,老年人会死亡,壮年人也难康复。形体正气充足,病气也有余,这是阴阳表里都有余,可先泻去邪气,而后根据虚实进行治疗。所谓“有余者泻之,不足者补之”,说的就是这个道理。\\
所以说:针治不懂得经脉循行的逆顺,真气与邪气斗争的情况。实证用补法,会使阴阳表里之邪气弥漫,充满肠胃,肝肺内胀,则阴阳表里的气血逆乱。虚证用泻法,会使经脉空虚,血气枯竭,肠胃皱叠,瘦得皮包骨,毫毛腠理枯折憔悴,就离死期不远了。因此说:用针治病的关键,在于懂得调节阴阳。调节阴阳,则精气充足,形体与内气的活动合一,使神气内藏而不外散。所以说:高明的医生,能平调阴阳之气;一般的医生,会扰乱经脉之气血运行;低劣的医生,会危及生命。所以说:对低劣的医生,病人不能不特别小心。运用针治,必须仔细审察五脏病情的变化,四时五脏脉象的相应变化,以及经络的虚实,皮肉的柔脆或坚实,然后才能治疗。\\
寿夭刚柔第六 法律\\
黄帝问于少师曰:余闻人之生也,有刚有柔,有弱有强,有短有长,有阴有阳,愿闻其方。\\
少师答曰:阴中有阴,阳中有阳,审知阴阳,刺之有方,得病所始,刺之有理,谨度病端,与时相应。内合于五脏六腑,外合于筋骨皮肤,是故内有阴阳,外亦有阴阳。在内者,五脏为阴,六腑为阳;在外者,筋骨为阴,皮肤为阳。故曰病在阴之阴者,刺阴之荥输;病在阳之阳者,刺阳之合;病在阳之阴者,刺阴之经;病在阴之阳者,刺络脉。故曰病在阳者命曰风,病在阴者命曰痹,阴阳俱病命曰风痹。病有形而不痛者,阳之类也;无形而痛者,阴之类也。无形而痛者,其阳完而阴伤之也,急治其阴,无攻其阳;有形而不痛者,其阴完而阳伤之也,急治其阳,无攻其阴。阴阳俱动,乍有形,乍无形,加以烦心,命曰阴胜其阳,此谓不表不里,其形不久。\\
黄帝问于伯高曰:余闻形气,病之先后、外内之应,奈何?\\
伯高答曰:风寒伤形,忧恐忿怒伤气。气伤脏,乃病脏。寒伤形,乃应形。风伤筋脉,筋脉乃应。此形气外内之相应也。\\
黄帝曰:刺之奈何?\\
伯高答曰:病九日者,三刺而已;病一月者,十刺而已。多少远近,以此衰之。久痹不去身者,视其血络,尽出其血。\\
黄帝曰:外内之病,难易之治,奈何?\\
伯高答曰:形先病而未入脏者,刺之半其日;脏先病而形乃应者,刺之倍其日。此外内难易之应也。\\
黄帝问于伯高曰:余闻形有缓急,气有盛衰,骨有大小,肉有坚脆,皮有厚薄,其以立寿夭,奈何?\\
伯高答曰:形与气相任则寿,不相任则夭;皮与肉相裹则寿,不相裹则夭;血气经络胜形则寿,不胜形则夭。\\
黄帝曰:何谓形之缓急?\\
伯高答曰:形充而皮肤缓者则寿,形充而皮肤急者则夭。形充而脉坚大者顺也,形充而脉小以弱者气衰,衰则危矣。若形充而颧不起者骨小,骨小则夭矣。形充而大肉夬坚而有分者肉坚,肉坚则寿矣;形充而大肉无分理不坚者肉脆,肉脆则夭矣。此天之生命,所以立形定气而视寿夭者。必明乎此立形定气,而后以临病人,决死生。\\
黄帝曰:余闻寿夭,无以度之。\\
伯高答曰:墙基卑,高不及其地者,不满三十而死;其有因加疾者,不及二十而死也。\\
黄帝曰:形气之相胜,以立寿夭奈何?\\
伯高答曰:平人而气胜形者寿;病而形肉脱,气胜形者死,形胜气者危矣。\\
黄帝曰:余闻刺有三变,何谓三变?\\
伯高答曰:有刺营者,有刺卫者,有刺寒痹之留经者。\\
黄帝曰:刺三变者,奈何?\\
伯高答曰:刺营者,出血;刺卫者,出气;刺寒痹者,内热。\\
黄帝曰:营卫寒痹之为病,奈何?\\
伯高答曰:营之生病也,寒热少气,血上下行。卫之生病也,气痛时来时去,怫忾贲响,风寒客于肠胃之中。寒痹之为病也,留而不去,时痛而皮不仁。\\
黄帝曰:刺寒痹内热,奈何?\\
伯高答曰:刺布衣者,以火焠之。刺大人者,以药熨之。\\
黄帝曰:药熨奈何?\\
伯高答曰:用淳酒二十斤,蜀椒一斤,干姜一斤,桂心一斤,凡四种,皆伀咀,渍酒中。用绵絮一斤,细白布四丈,并内酒中。置酒马矢煴中,盖封涂,勿使泄。五日五夜,出布绵絮,曝干之,干复渍,以尽其汁。每渍必晬其日,乃出干。干,并用滓与绵絮,复布为复巾,长六七尺,为六七巾。则用之生桑炭炙巾,以熨寒痹所刺之处,令热入至于病所,寒复炙巾以熨之,三十遍而止。汗出以巾拭身,亦三十遍而止。起步内中,无见风。每刺必熨,如此病已矣。此所谓内热也。\\
黄帝问少师说:我听说人的先天禀赋,有刚柔、强弱、长短、阴阳的区别,希望听一下其中的道理。\\
少师回答说:就人体阴阳来说,阴当中还有阴,阳当中还有阳,只有了解阴阳的规律,才能很好的运用针刺方法,了解疾病发生的情况,才能在针刺时做出适当的手法,同时要认真地揣度发病的经过与四时变化的相应关系。人体的阴阳,在内合于五脏六腑,在外合于筋骨皮肤,所以人体内有阴阳,体外也有阴阳。在体内的,五脏为阴,六腑为阳;在体外的,筋骨为阴,皮肤为阳。因此,病在阴中之阴的,当刺阴经的荥输;病在阳中之阳的,当刺阳经的合穴;病在阳中之阴的,当刺阴经的经穴;病在阴中之阳的,当刺阳经的络穴。这是根据阴阳内外与疾病的关系,而选取针刺穴位的基本法则。阴阳也可以作为疾病的分类准则,病在阳经的叫风,病在阴经的叫痹,阴阳两经都有病的叫风痹。病有形态变化而不疼痛的,属于阳经一类;病无形态变化而疼痛的,属于阴经一类。没有形态变化而感到疼痛的,是阳经未受侵害,只是阴经有病,赶快在阴经取穴治疗,不要攻治阳经;有形态变化而不感觉疼痛的,是阴经未受侵害,只是阳经有病,赶快在阳经取穴治疗,不要攻治阴经。阴阳表里都有病,忽然有形态变化,忽然又没了,更加上心烦,叫阴病重于阳,这是所谓的不表不里,预后不良。\\
黄帝问伯高说:我听说形气与发病有先后内外的相应关系,是什么道理?\\
伯高回答说:风寒外袭,先伤形体,忧恐忿怒的精神刺激,先伤内气。气逆伤了五脏之和,就会使五脏有病。寒邪侵袭形体,就会使肌表皮肤发病。风邪伤了筋脉,就会使筋脉发病。这就是形气与疾病外内相应的关系。\\
黄帝问:怎样针刺治疗呢?\\
伯高回答说:病九天的,刺三次可以好;病一个月的,刺十次可以好。病程时日的多少远近,都可以根据三日一刺的标准来计算。经久不愈的痹证,根据血络变化,尽力去掉淤血。\\
黄帝又问:人体在内在外的疾病,针刺难易的情况怎样呢?\\
伯高回答说:形体先有病还未传入内脏的,针刺的次数,可以根据已病的日数减半计算;内脏先有病而形体也有反应的,针刺的日数就要加倍。这就是疾病有内外、针治有难易的对应关系。\\
黄帝问伯高说:我听说人的外形有缓有急,正气有盛有衰,骨胳有大有小,肌肉有坚有脆,皮肤有厚有薄,从这些怎样来确定人的寿夭呢?\\
伯高回答说:外形与正气相称的多长寿,不相称的多夭亡;皮肤与肌肉结合紧密的多长寿,不紧密的多夭亡;血气经络充盛胜过外形的多长寿,血气经络衰弱不能胜过外形的多夭亡。\\
黄帝问:什么叫做形体的缓急?\\
伯高回答说:形体充实而皮肤柔软的人,多长寿;形体充实而皮肤坚紧的人,多短命。形体充实而脉气坚大的为顺;形体充实而脉气弱小的属于气衰,气衰是危险的。如果形体充实而面部颧骨不突起的人,骨胳必小,骨胳小的多短命。形体充实而臂腿臀部肌肉突起坚实而有肤纹的,称为肉坚,肉坚的人多长寿;形体充实而臂腿臀部肌肉没有肤纹的,称为肉脆,肉脆的人多短寿。这是自然界赋予人生命所形成的形体与生气的自然状态,可据此来判断人的寿命长短。医者,必须了解形体与生气的状态,然后可以临床治病,判断死生。\\
黄帝说:我听说人有寿夭,但无法推测。\\
伯高回答说:衡量人的寿夭,凡是面部肌肉陷下,而四周的骨胳显露,不满三十岁就会死的;再加上疾病影响,不到二十岁,就可能死亡。\\
黄帝问:从形与气的相胜,怎样用它去确定寿命长短呢?\\
伯高回答说:健康人,正气胜过形体的可以长寿;有病的人,形体肌肉很消瘦,即使其气胜过形体,也是要死的;即使形体尚可,但元气已衰,也很危险。\\
黄帝问:我听说刺法有三种变化,什么叫三种变化呢?\\
伯高回答说:有刺营分,刺卫分和刺寒痹稽留经络三种。\\
黄帝问:怎样刺三种变化?\\
伯高回答说:刺营分时要刺出恶血;刺卫分时要祛除邪气;刺寒痹时要使热气入内。\\
黄帝问:营分、卫分、寒痹的症状,是怎样的?\\
伯高回答说:营分病证见寒热往来,呼吸少气,血上下妄行。卫分病证见痛无定时,胸腹满闷或者窜动作响,这是风寒侵入肠胃所致。寒痹多由病邪久留而不解,因此时常感到筋骨作痛,甚至皮肤麻木不仁。\\
黄帝问:怎样刺寒痹才能使体内有热感?\\
伯高回答说:一般百姓,可用烧红的火针刺治。而王公大人,多用药熨。\\
黄帝问:怎样药熨呢?\\
伯高回答说:用醇酒二十升,蜀椒一升,干姜、桂心各一斤,共四种药,都剉碎,浸泡酒中。再用丝绵一斤、细白布四丈,一齐放入酒中。把酒器加上盖,放在燃着的干马粪内煨,并用泥封固,不使泄气。经过五天五夜,将细布与丝绵取出晒干,干后再浸入酒内,如此反复地将药酒浸干为度。每次要浸泡一整天,然后拿出来再晒干。等酒浸干后,将布做成每个长六七尺的夹袋,共做六七个,将药渣与丝绵装入袋内。用时取生桑炭火,将夹袋放在上面烘热,熨敷于寒痹所刺的地方,使热气能深透病处。夹袋凉了再将其烘热,如此熨敷三十次,每次都使患者出汗。出汗后用手巾揩身,也要三十遍。并让患者在室内行走,但不能见风。按照这样的方法,每次针治时,再加用熨法,病就会好了。这就是“内热”的方法。\\
官针第七 法星\\
凡刺之要,官针最妙。九针之宜,各有所为;长短大小,各有所施也。不得其用,病弗能移。疾浅针深,内伤良肉,皮肤为痈。病深针浅,病气不泻,支为大脓。病小针大,气泻太甚,疾必为害;病大针小,气不泄泻,亦复为败。失针之宜,大者泻,小者不移。已言其过,请言其所施。\\
病在皮肤无常处者,取以镵针于病所,肤白勿取。病在分肉间,取以员针于病所。病在经络痼痹者,取以锋针。病在脉气少,当补之者,取以氻针,于井荥分输。病为大脓者,取以铍针。病痹气暴发者,取以员利针。病痹气痛而不去者,取以毫针。病在中者,取以长针。病水肿不能通关节者,取以大针。病在五脏固居者,取以锋针。泻于井荥分输,取以四时。\\
凡刺有九,以应九变。一曰输刺。输刺者,刺诸经荥输脏腧也。二曰远道刺。远道刺者,病在上,取之下,刺腑腧也。三曰经刺。经刺者,刺大经之结络经分也。四曰络刺。络刺者,刺小络之血脉也。五曰分刺。分刺者,刺分肉之间也。六曰大泻刺。大泻刺者,刺大脓以铍针也。七曰毛刺。毛刺者,刺浮痹皮肤也。八曰巨刺。巨刺者,左取右,右取左。九曰焠刺。焠刺者,刺燔针则取痹也。\\
凡刺有十二节,以应十二经。一曰偶刺。偶刺者,以手直心若背,直痛所,一刺前,一刺后,以治心痹。刺此者,傍针之也。二曰报刺。报刺者,刺痛无常处也,上下行者;直内,无拔针,以左手随病所,按之,乃出针,复刺之也。三曰恢刺。恢刺者,直刺傍之,举之前后,恢筋急,以治筋痹也。四曰齐刺。齐刺者,直入一,傍入二;以治寒气小深者。或曰三刺。三刺者,治痹气小深者也。五曰扬刺。扬刺者,正内一,傍内四,而浮之;以治寒气之博大者也。六曰直针刺。直针刺者,引皮乃刺之;以治寒气之浅者也。七曰输刺。输刺者,直入直出,稀发针而深之;以治气盛而热者也。八曰短刺。短刺者,刺骨痹,稍摇而深之,致针骨所,以上下摩骨也。九曰浮刺。浮刺者,傍入而浮之;以治肌急而寒者也。十曰阴刺。阴刺者,左右率刺之,以治寒厥,中寒厥,足踝后少阴也。十一曰傍针刺。傍针刺者,直刺傍刺各一,以治留痹,久居者也。十二曰赞刺。赞刺者,直入直出,数发针而浅之,出血,是谓治痈肿也。\\
脉之所居,深不见者,刺之;微内针,而久留之,以致其空脉气也。脉浅者,勿刺;按绝其脉,乃刺之;无令精出,独出其邪气耳。所谓三刺则谷气出者,先浅刺绝皮,以出阳邪;再刺则阴邪出者,少益深,绝皮致肌肉,未入分肉间也;已入分肉之间,则谷气出。故《刺法》曰:始刺浅之,以逐邪气,而来血气;后刺深之,以致阴气之邪;最后刺极深之,以下谷气。此之谓也。故用针者,不知年之所加,气之盛衰,虚实之所起,不可以为工也。\\
凡刺有五,以应五脏。一曰半刺。半刺者,浅内而疾发针,无针伤肉,如拔毛状;以取皮气,此肺之应也。二曰豹文刺。豹文刺者,左右前后针之,中脉为故;以取经络之血者,此心之应也。三曰关刺。关刺者,直刺左右,尽筋上;以取筋痹,慎无出血,此肝之应也。或曰渊刺,一曰岂刺。四曰合谷刺。合谷刺者,左右鸡足,针于分肉之间;以取肌痹,此脾之应也。五曰输刺。输刺者,直入直出,深内之至骨;以取骨痹,此肾之应也。\\
针刺的要领,以选用适当的针具为最妙。九种针具各有不同的作用,长短大小不同的针,各有不同的使用方法。假如使用不当,病患就不能除去。如果病浅而针刺深,就会损伤体内的好肉,以致表皮发生痈肿。病深而针刺浅,病邪就不能排除,扩散而酿成大的脓肿。病重而用大针,泻气太过,就会加重病情;病重而用小针,邪气不能外泻,也能使治疗失败。总之,针刺失当在于过度用针则泻伤正气,用针不足则邪气不除。已经说明了用针的过失,请让我再谈正确的用针方法。\\
病在皮肤而游移无定的,可用镵针刺患处,但病患处皮肤发白的不能用。病在分肉之间的,可用员针。病在经络痹阻已久的,可用锋针。病在脉,脉气不足,当用补法,可用奼针,按摩各经的“井、荥、输、原、经、合”穴。病患为大的脓肿,可用铍针。病患为突发性的痹证,可用员利针。痹证疼痛日久不解的,可用毫针。病在体内的,可用长针。病属关节间水肿,可用大针。病在五脏,久而不愈,可用锋针。在各经的“井、荥、输、原、经、合”穴用泻法,并且要根据四时不同而取穴。\\
针刺有九种方法,适用于九种病变。第一种叫输刺。输刺是刺各经的井、荥、输、经、合穴,以及在足太阳经上的五脏六腑之背腧穴。第二种叫远道刺。远道刺是病在上部取刺下部,以刺属六腑的足三阳经的腧穴为多。第三种叫经刺。经刺是刺深部大经出现于浅表的硬结或压痛。第四种叫络刺。络刺是刺浅表的小络,泻出淤血。第五种叫分刺。分刺是刺各经分肉之间。第六种叫大泻刺。大泻刺是用铍针刺大的脓肿。第七种叫毛刺。毛刺是刺皮肤间浮浅的痹证。第八种叫巨刺。巨刺是左病取右,右病取左。第九种叫焠刺。焠刺是将针烧热来治疗寒痹。\\
刺法有十二种,以适应十二经的病证。第一种叫偶刺。偶刺用手当其前胸及后背的痛处寻摸一下,然后进针,一针前胸,一针后背,用来治疗心痹证。治这种病,应当从旁斜刺以免刺伤内脏。第二种叫报刺。报刺是刺痛无定处,上下游走的;直刺,留针不拔,再以左手寻痛处揉按,然后将针拔出,重新再刺。第三种叫恢刺。恢刺是直刺筋的四旁,或前或后地提插,来扩大针孔,解除筋脉拘急,治疗筋痹证。第四种叫齐刺。齐刺是直刺一针,左右两旁再各刺一针;治疗寒气稽留稍深的痹证。因为这是三针并用,所以又叫“三刺”。三刺,是治疗寒痹稍深的疾病。第五种叫阳刺。阳刺是在病位正中刺一针,周围刺四针,用浅刺法;可以治疗寒气稽留范围较大的病证。第六种叫直针刺。直针刺是先把皮肤提起,然后将针沿皮刺入;用来治疗寒气稽留部位较浅的病证。第七种叫输刺。输刺是针刺时直出直入,用针少而针入深;治疗气盛而有热的病证。第八种叫短刺。短刺适于治疗骨痹,进针时轻轻摇针,深入,使针尖达到骨处,上下提插摩骨。第九种叫浮刺。浮刺是从旁斜刺而入于浮浅的肌表;治疗肌肉拘急而属寒的病证。第十种叫阴刺。阴刺是左右都针刺,用以治疗受寒厥冷的病证,刺中寒厥取足踝后少阴经穴。第十一种叫傍针刺。傍针刺是直刺一针、傍刺一针,治疗痹痛久而不去的病证。第十二种叫赞刺。赞刺是直入直出,多下针而浅刺,使患部出血,用来治疗痈肿。\\
经脉深居体内不显露于外的,可刺;刺时应轻轻进针而长时间留针,引导孔穴中的脉气运行。脉在皮下浅表的,不能直刺其脉;先用指切按住,避开脉管,然后再进针;不要使精气外泄,仅将邪气祛除。所谓“三刺则谷气出”的刺法,是先浅刺皮部,疏泄在表之阳邪;所谓“再刺则疏泄阴分之邪”,是进针较皮部稍深,至肌肉而未到分肉间;最后刺到分肉间,则谷气出而产生痠胀感。所以《刺法》上说:开始浅刺,以驱除邪气,流通血气;稍后刺略深,以疏泄阴分之邪;最后刺极深,可以通导谷气。说的就是这种刺法。因此,运用针刺法的人,不懂得每年气候加临于人体的情况、正气的盛衰、虚实证的形成,就不能称为医工。\\
凡刺法有五种,可用于五脏病证。第一种叫半刺。半刺是进针浅而出针快,针刺不能伤肌肉,好像拔毫毛一样,使皮肤感受一下轻微的刺激;合谷刺与肺气相应。第二种叫豹文刺。豹文刺是左右前后都进针,像豹的斑纹,以刺中络脉出血为目的,这种刺法与心气相应。第三种叫关刺。关刺是在四肢的关节附近,筋的尽端处进针,用来治疗筋痹,但刺时要谨慎,不能出血,这种刺法与肝气相应。此法又叫“渊刺”或“岂刺”。第四种叫合谷刺。合谷刺是直刺进针到分肉间以后,复将针提到皮下向左右分肉间各斜刺一针,如鸡足分叉;用以治疗肌痹,这种刺法与脾气相应。第五种叫输刺。输刺是进针时直入直出,深刺至骨的附近;用以治疗骨痹,这种刺法与肾气相应。\\
本神第八 法风\\
黄帝问于岐伯曰:凡刺之法,先必本于神。血、脉、营、气、精、神,此五脏之所藏也。至其淫泆离脏则精失,魂魄飞扬,志意恍乱,智虑去身者,何因而然乎?天之罪与?人之过乎?何谓德、气、生、精、神、魂、魄、心、意、志、思、智、虑?请问其故。\\
岐伯答曰:天之在我者,德也;地之在我者,气也。德流气薄而生者也。故生之来谓之精,两精相搏谓之神,随神往来者谓之魂,并精而出入者谓之魄,所以任物者谓之心,心之所忆谓之意,意之所存谓之志,因志而存变谓之思,因思而远慕谓之虑,因虑而处物谓之智。\\
故智者之养生也,必顺四时而适寒暑,和喜怒而安居处,节阴阳而调刚柔,如是则僻邪不至,长生久视。\\
是故怵惕思虑者则伤神,神伤则恐惧,流淫而不止。因悲哀动中者,竭绝而失生。喜乐者,神惮散而不藏。愁忧者,气闭塞而不行。盛怒者,迷惑而不治。恐惧者,神荡惮而不收。\\
心,怵惕思虑则伤神,神伤则恐惧自失,破夬脱肉,毛悴色夭,死于冬。\\
脾,愁忧不解则伤意,意伤则悗乱,四肢不举,毛悴色夭,死于春。\\
肝悲哀动中则伤魂,魂伤则狂忘不精,不精则不正,当人阴缩而挛筋,两胁骨不举,毛悴色夭,死于秋。\\
肺喜乐无极则伤魄,魄伤则狂,狂者意不存人,皮革焦,毛悴色夭,死于夏。\\
肾盛怒而不止则伤志,志伤则喜忘其前言,腰脊不可以俯仰屈伸,毛悴色夭,死于季夏。\\
恐惧而不解则伤精,精伤则骨痠痿厥,精时自下。是故五脏主藏精者也,不可伤,伤则失守而阴虚,阴虚则无气,无气则死矣。是故用针者,察观病人之态,以知精神魂魄之存亡,得失之意,五者以伤,针不可以治之也。\\
肝藏血,血舍魂。肝气虚则恐,实则怒。脾藏营,营舍意。脾气虚则四肢不用,五脏不安,实则腹胀,经溲不利。心藏脉,脉舍神。心气虚则悲,实则笑不休。肺藏气,气舍魄。肺气虚,则鼻塞不利,少气;实则喘喝,胸盈仰息。肾藏精,精舍志,肾气虚则厥,实则胀,五脏不安。必审五脏之病形,以知其气之虚实,谨而调之也。\\
黄帝问岐伯说:针刺的法则,必须先研究病人的精神状态。因为血、脉、营、气、精、神,这都是五脏所藏的。至其失了正常,离开所藏之脏,五脏精气走失,魂魄也飞扬了,志意也烦乱了,智慧和思考能力离开了自身,为什么会这样呢?是上天的惩罚呢,还是人为的过失呢?什么叫德、气、生、精、神、魂、魄、心、意、志、思、智、虑?希望听到其中的道理。\\
岐伯回答说:天赋予我们人类的是德,地赋予我们人类的是气,由于天德下流与地气上交,阴阳相结合,使万物化生成形,人才能生存。所以,人体生命的原始物质,叫精;阴阳两精相结合产生的生命活动,叫神;随着神的往来活动而出现的知觉机能,叫魂;跟精气一起出入而产生的运动机能,叫魄;可以支配外来事物的,叫心;心里有所忆念而留下的印象,叫意;意念所在,形成了认识,叫志;根据认识而反复研究事物的变化,叫思;因思考而有远的推想,叫虑;因思虑而能定出相应的处理事物方法,叫智。\\
因此,智者养生必定顺着四时来适应寒暑的气候,调和喜怒而安定起居,节制房事,调和刚柔。这样,虚邪贼风就不能侵袭人体,自然可以延寿,不易衰老了。\\
所以过分的恐惧忧思,就会损伤心神,损伤心神就恐惧,使阴精流失不止。悲哀过度伤了内脏,会使气机竭绝,丧失生命。喜乐过度,会致喜极气散不能收藏。愁忧过度,就会使气机闭塞,不能流畅。大怒,就会使神志昏迷,失去常态。恐惧过度,就会由于精神动荡而精气不能收敛。\\
心过度恐惧忧思,就会伤神,神伤,就会时时恐惧不能自控,时间久了,肌肉消瘦,毛发憔悴,面色异常,死在冬季。\\
脾过度忧愁不能解除,就会伤意,意伤,就会苦闷烦乱,手足乏力,不能抬起来,进而毛发憔悴,面色异常,死在春季。\\
肝过度悲哀影响内脏,就会伤魂,魂伤,会出现精神紊乱症状,导致肝脏失去藏血作用,使人阴器萎缩,筋脉挛急,两胁不能舒张,进而毛发憔悴,面色异常,死在秋季。\\
肺过度喜乐,就会伤魄,魄伤,会形成狂病,狂者思维混乱,不识旧人,皮肤枯槁,进而毛发憔悴,面色异常,死在夏季。\\
肾大怒不能遏止,就会伤志,志伤,就容易忘记自己说过的话,腰脊不能随意俯仰,进而毛发憔悴,面色异常,死在季夏。\\
过度恐惧而解除不了,就会伤精,精伤,就会发生骨节酸痛和痿厥,并常有遗精。所以五脏是主藏精气的,不可被损伤;伤了,就会使精气失守,形成阴虚,阴虚则阳气的化源断绝,离死就不远了。所以运用针刺的人,必定要观察病人的形态,以了解他的精、神、魂、魄等精神活动的旺盛或衰亡,如果五脏精气已经损伤,就不能用针刺治疗了。\\
肝贮藏血,魂依附血液。肝气虚,会恐惧;肝气盛,容易发怒。脾贮藏营气,意念依附营气。脾气虚,会使四肢运用不灵,五脏不能调和;脾气壅实,会使腹部胀满,大小便不利。心藏神,神寄附在血脉中。心气虚,会悲伤;心气太盛,会笑而不止。肺藏气,魄依附在肺气中。肺气虚,会感到鼻塞,呼吸不便,气短;肺气壅实,会大喘,胸满,甚至仰面而喘。肾藏精,意志依附精气。肾气虚,会手足厥冷,肾有实邪,会腹胀,并连及五脏不能安和。因此说:治病必须审察五脏病的症状,以了解元气虚实,从而谨慎地加以调治。\\
终始第九 法野\\
凡刺之道,毕于《终始》。明知终始,五脏为纪,阴阳定矣。阴者主脏,阳者主腑。阳受气于四末,阴受气于五脏。故泻者迎之,补者随之。知迎知随,气可令和。和气之方,必通阴阳。五脏为阴,六腑为阳。传之后世,以血为盟。敬之者昌,慢之者亡。无道行私,必得夭殃。\\
谨奉天道,请言终始!终始者,经脉为纪。持其脉口人迎,以知阴阳,有余不足,平与不平。天道毕矣。所谓平人者不病。不病者,脉口人迎应四时也,上下相应而俱往来也,六经之脉不结动也,本末之寒温之相守司也,形肉血气必相称也。是谓平人。少气者,脉口人迎俱少而不称尺寸也。如是者,则阴阳俱不足。补阳则阴竭,泻阴则阳脱。如是者,可将以甘药,不可饮以至剂。如是者,弗灸。不已者,因而泻之,则五脏气坏矣。\\
人迎一盛,病在足少阳;一盛而躁,病在手少阳。人迎二盛,病在足太阳;二盛而躁,病在手太阳。人迎三盛,病在足阳明;三盛而躁,病在手阳明。人迎四盛,且大且数,名曰溢阳,溢阳为外格。脉口一盛,病在足厥阴;一盛而躁,在手心主。脉口二盛,病在足少阴;二盛而躁,在手少阴。脉口三盛,病在足太阴;三盛而躁,在手太阴。脉口四盛,且大且数者,名曰溢阴,溢阴为内关。内关不通,死不治。人迎与太阴脉口俱盛四倍以上,命名关格。关格者,与之短期。\\
人迎一盛,泻足少阳而补足厥阴,二泻一补,日一取之,必切而验之,疏取之上,气和乃止。人迎二盛,泻足太阳,补足少阴,二泻一补,二日一取之,必切而验之,疏取之上,气和乃止。人迎三盛,泻足阳明而补足太阴,二泻一补,日二取之,必切而验之,疏取之上,气和乃止。脉口一盛,泻足厥阴而补足少阳,二补一泻,日一取之,必切而验之,疏而取之上,气和乃止。脉口二盛,泻足少阴而补足太阳,二补一泻,二日一取之,必切而验之,疏取之上,气和乃止。脉口三盛,泻足太阴而补足阳明,二补一泻,日二取之,必切而验之,疏而取之上,气和乃止。所以日二取之者,太阴主胃,大富于谷气,故可日二取之也。人迎与脉口俱盛三倍以上,命曰阴阳俱溢,如是者不开,则血脉闭塞,气无所行,流淫于中,五脏内伤。如此者,因而灸之,则变易而为他病矣。\\
凡刺之道,气调而止。补阴泻阳,音气益彰,耳目聪明。反此者,血气不行。\\
所谓气至而有效者,泻则益虚。虚者,脉大如其故而不坚也。坚如其故者,适虽言快,病未去也。补则益实。实者,脉大如其故而益坚也。夫如其故而不坚者,适虽言快,病未去也。故补则实,泻则虚。痛虽不随针,病必衰去。必先通十二经脉之所生病,而后可得传于终始矣。故阴阳不相移,虚实不相倾,取之其经。\\
凡刺之属,三刺至谷气。邪僻妄合,阴阳易居。逆顺相反,沉浮异处。四时不得,稽留淫泆。须针而去。故一刺则阳邪出,再刺则阴邪出,三刺则谷气至,谷气至而止。所谓谷气至者,已补而实,已泻而虚,故以知谷气至也。邪气独去者,阴与阳未能调,而病知愈也。故曰补则实,泻则虚。痛虽不随针,病必衰去矣。\\
阴盛而阳虚,先补其阳,后泻其阴而和之。阴虚而阳盛,先补其阴,后泻其阳而和之。\\
三脉动于足大指之间,必审其实虚。虚而泻之,是谓重虚。重虚,病益甚。凡刺此者,以指按之。脉动而实且疾者则泻之,虚而徐者则补之。反此者,病益甚。其动也,阳明在上,厥阴在中,少阴在下。膺腧中膺,背腧中背。肩膊虚者,取之上。重舌,刺舌柱以铍针也。手屈而不伸者,其病在筋;伸而不屈者,其病在骨。在骨守骨,在筋守筋。\\
泻一方实,深取之,稀按其痏,以极出其邪气;补一方虚,浅刺之,以养其脉,疾按其痏,无使邪气得入。邪气来也紧而疾,谷气来也徐而和。脉实者,深刺之,以泄其气;脉虚者,浅刺之,使精气无得出,以养其脉,独出其邪气。刺诸痛者,其脉皆实。\\
故曰:从腰以上者,手太阴阳明皆主之;从腰以下者,足太阴阳明皆主之。病在上者下取之,病在下者高取之,病在头者取之足,病在腰者取之腘。病生于头者头重,生于手者臂重,生于足者足重。治病者,先刺其病所从生者也。\\
春,气在毛;夏,气在皮肤;秋,气在分肉;冬,气在筋骨。刺此病者各以其时为齐。故刺肥人者,以秋冬之齐;刺瘦人者,以春夏之齐。病痛者,阴也。痛而以手按之不得者,阴也,深刺之。痒者,阳也,浅刺之。病在上者,阳也;病在下者,阴也。\\
病先起阴者,先治其阴而后治其阳;病先起阳者,先治其阳而后治其阴。刺热厥者,留针,反为寒;刺寒厥者,留针,反为热。刺热厥者,二阴一阳;刺寒厥者,二阳一阴。所谓二阴者,二刺阴也;一阳者,一刺阳也。久病者,邪气入深。刺此病者,深内而久留之,间日而复刺之。必先调其左右,去其血脉。刺道毕矣。\\
凡刺之法,必察其形气。形肉未脱,少气而脉又躁,躁疾者,必为缪刺之。散气可收,聚气可布。深居静处,占神往来;闭户塞牖,魂魄不散。专意一神,精气之分,毋闻人声,以收其精,必一其神,令志在针。浅而留之,微而浮之,以移其神,气至乃休。男内女外,坚拒勿出。谨守勿内,是谓得气。\\
凡刺之禁:新内勿刺,新刺勿内。已醉勿刺,已刺勿醉。新怒勿刺,已刺勿怒。新劳勿刺,已刺勿劳。已饱勿刺,已刺勿饱。已饥匆刺,己刺勿饥。已渴勿刺,已刺勿渴。大惊大怒,必定其气,乃刺之。乘车来者,卧而休之,如食顷乃刺之。出行来者,坐而休之,如行十里顷乃刺之。\\
凡此十二禁者,其脉乱气散,逆其营卫,经气不次。因而刺之,则阳病入于阴,阴病出于阳,则邪气复生。粗工勿察,是谓伐身。形体淫泆,乃消脑髓,津液不化,脱其五味,是谓失气也。\\
太阳之脉,其终也,戴眼、反折、瘛疭,其色白,绝皮乃绝汗,绝汗,则终矣。少阳终者,耳聋,百节尽纵,目系绝,目系绝,一日半则死矣。其死也,色青白,乃死。阳明终者,口目动作,喜惊,妄言,色黄,其上下之经盛而不行,则终矣。少阴终者,面黑,齿长而垢,腹胀闭塞,上下不通,而终矣。厥阴终者,中热嗌干,喜溺心烦,甚则舌卷,卵上缩,而终矣。太阴终者,腹胀闭,不得息,气噫,善呕,呕则逆,逆则面赤,不逆则上下不通,上下不通,则面黑皮毛燋,而终矣。\\
大凡针刺的法则,全在《终始》篇里。明确了解终始的意义,就必须以五脏为纲纪,可以确定阴经阳经的关系。阴经是与五脏相通,阳经是与六腑相通。阳经承受四肢的脉气,阴经承受五脏的脉气。所以泻法是迎而夺之,补法是随而济之。知道迎随补泻的方法,可以使脉气调和。而调和脉气的关键,必定要明白阴阳的规律。五脏在内为阴,六腑在外为阳。要将刺法流传于后世,必须严肃认真地对待,如同“以血为盟”一样。重视此法会使它发扬光大,忽视此法能使其散失消亡。如果不懂装懂,一定会危害人的生命。\\
慎重地遵循天地阴阳变化规律,让我谈谈针刺的终始意义吧!所谓终始,是以十二经脉为纲纪,从脉口、人迎两部的脉象了解阴经阳经的脉象是实是虚,上下之脉是相应平衡还是不平衡。这样,阴阳变化就大致掌握了。所谓平人,就是没有病的人,无病人的脉口和人迎的脉象是和四时相应的;脉口,人迎互相呼应,往来不息;六经之脉搏动不止;人体上下内外,在寒温不同的环境里能够保持平衡;形肉和血气也能够协调一致。这就是没有病的人。气虚的人,脉口、人迎的脉象细小,而尺肤和脉象不相称。像这样,就是阴阳都不足的病证。补阳就会使阴气衰竭,泻阴就会使阳气亡脱。这样的病人,只可以用缓剂补养,不能用峻猛的药物攻泻。这种病证也不能用灸法。因为病未愈,而用泻法,那就会败坏五脏真气。\\
人迎脉象大于寸口一倍,病在足少阳胆经;大一倍而躁动,病在手少阳三焦经。人迎脉象大于寸口二倍,病在足太阳膀胱经;大二倍而躁动,病在手太阳小肠经。人迎脉象大于寸口三倍,病在足阳明胃经;大三倍而躁动,病在手阳明大肠经。人迎脉象大于寸口四倍,大而且速,名叫“溢阳”,溢阳是六阳偏盛,格拒六阴在外,所以叫“外格”。寸口脉象大于人迎一倍,病在足厥阴肝经;大一倍而躁动,病在手厥阴心包络经。寸口脉象大于人迎二倍,病在足少阴肾经;大二倍而躁动,病在手少阴心经。寸口脉象大于人迎三倍,病在足太阴脾经;大三倍而躁动,病在手太阴肺经。寸口脉象大于人迎四倍而且速的,名叫“溢阴”。溢阴是六阴经偏盛至极而泛滥于内,称为“内关”。内关闭塞不通,是不治的死证。如果人迎与寸口的脉象,都大四倍以上,名叫“关格”。关格预测必死。\\
人迎脉大于寸口一倍,泻足少阳胆经,而补足厥阴肝经,二分泻一分补,每天治一次,必须切脉以验其偏盛的情况,疏取肝胆两经上之穴,脉气平和为止。人迎脉大于寸口两倍,泻足太阳膀胱经,而补足少阴肾经,二分泻一分补,每两天治一次,必须切脉以验其偏盛的情况,疏取肾与膀胱两经上之穴,脉气平和为止。人迎脉大于寸口三倍,泻足阳明胃经,而补足太阴脾经,二分泻一分补,每天治两次,必须切脉以验其偏盛的情况,疏取脾胃两经上之穴,脉气平和为止。寸口脉大于人迎一倍,泻足厥阴肝经,而补足少阳胆经,二分补一分泻,每天治一次,必须切脉以验其偏盛的情况,疏取肝胆两经上之穴,脉气平和为止。寸口脉大于人迎两倍,泻足少阴肾经,而补足太阳膀胱经,二分补一分泻,每两天治一次,必须切脉以验其偏盛的情况,疏取肾与膀胱两经上之穴,脉气平和为止。寸口脉大于人迎三倍,当泻足太阴脾经,而补足阳明胃经,二分补一分泻,每天可治两次,必须切脉以验其偏盛的情况,疏取脾胃两经上之穴,脉气平和为止。之所以每天治两次,是因为足太阴脾与足阳明胃相表里,谷气最丰富,因此可以每天针治两次。人迎与寸口的脉象,都大三倍以上,叫做“阴阳俱溢”,这样就不能开通,血脉闭塞而脉气无法通行,淫溢于中而五脏内伤。像这样,若使用灸法,更伤其阴,就可能变成其他病证。\\
大凡针刺的原则,阴阳之气调和了,就要停针。要注意阴阳补泻,这样才会有语音清朗,耳聪目明的效果。相反,血气就不能正常运行。\\
所谓针下气至而获得疗效,是说实证用了泻法,就会由实转虚。这虚的脉象仍旧大,却不坚实。如果脉象坚实照旧,虽说一时觉得舒服,其实病情并没有减轻。虚证用了补法,就会由虚转实。这实的情况,是脉象仍旧大些,并且更坚实了。如果脉象大虽照旧而并不坚实,虽说一时觉得舒服,其实病情并没有减轻。所以准确地运用补法,会使正气充实;准确地运用泻法,会使病邪衰退。即使病不随着针立即除去,但病势必定减轻。必须先明白十二经脉与各种疾病的关系,然后才可以做到有始有终。阴经和阳经是不会互相改变的,虚证和实证也是不会相反的,所以针治疾病,就要取其所属的经脉。\\
大凡针刺所应该注意的是采用三刺法使正气徐徐而来。那邪僻不正之气与血气混合,使阴阳失其常位而逆乱。气血运行的逆顺颠倒,脉象沉浮异常。脉气与四时不相应合,患者或血气留滞,或血气妄行。所有这许多病变,都有待用针刺去排除。因此要注意三刺法:初刺能使阳分的病邪排出,再刺会使阴分的病邪排出,三刺就会使正气徐徐而来,这时就应该出针了。所谓谷气至,是说已经用了补法,就觉得气充实些;已经用了泻法,就觉得病邪衰退些。从这些表现就知道谷气已至。起初,仅是邪气排除了,阴与阳之间的血气还没有调和,但是已能知道病要痊愈了。所以说用补法而能使正气充实,用泻法而能使邪气衰退。病痛虽未能随针立即消除,但病势必会减轻。\\
阴经邪气盛,阳经正气虚,先补阳经正气,后泻阴经邪气,从而调和有余和不足。阴经正气虚,阳经邪气盛,先补阴经正气,后泻阳经邪气,从而调和有余和不足。\\
足阳明经、足厥阴经、足少阴经三条经脉,都有动脉散布于足大指之间,在针刺时,必须审察它是属于虚证,或是属于实证。假如虚证误用了泻法,这叫重虚。虚而更虚,病就更厉害了。大凡针刺这些病证时,先用手指去按动脉,脉的搏动实而快的就用泻法,脉的搏动虚而缓的就用补法。如所用的补泻之法,与此相反,那么病就会更加重。至于动脉的所在,足阳明经在足跗之上,足厥阴经在足跗之内,足少阴经在足跗之下。取胸部腧穴必中其胸。取背部腧穴必中其背。肩膊出现酸胀麻木的虚证,应取上肢经脉的腧穴。对于重舌的患者,应该用铍针,刺舌下根柱,使之出血。手指弯曲而不能够伸直,那病在筋上;伸直了而不能够弯曲,那病在骨上。病在骨,应该求之于主骨的各个穴位去治疗;病在筋,应该求之于主筋的各个穴位去治疗。\\
泻的大法,在于泻的时候要注意脉气之实,深刺,出针后,缓按针孔,以尽量泄去邪气;补的时候要注意脉气之虚,浅刺,以保养所取的经脉,出针后,急按针孔,不叫邪气侵入。邪气来了,针下会感到拘急;谷气来了,针下会感到徐和。脉气盛实的,深刺,使邪气外泄;脉气虚弱的,浅刺,使精气不外泄,以养其经脉,而仅让邪气排出。对于各种疼痛的病证,要一律深刺,因为疼证的脉象都是实的。\\
所以说:腰以上的病,都在手太阴肺经,手阳明大肠经的主治范围;腰以下的病,都在足太阴脾经、足阳明胃经的主治范围。病在上部的,可以取下部的穴位;病在下部的,可以取上部的穴位;病在头部的,可取足部的穴位;病在腰部的,可取腘部的穴位。病患于头部的,头必觉得重;病患于手部的,臂必觉得重;病生于足部的,足必觉得重。治疗这些病证,应当先针刺疾病开始发生的部位。\\
春天,邪气在毫毛;夏天,邪气在皮肤;秋天,邪气在分肉;冬天,邪气在筋骨。治疗这些与时令有关的病证,针刺的浅深,应该根据季节而变化。所以刺胖人,要用适于秋冬的深刺法;刺瘦人,就用适于春夏的浅刺法。感到疼痛的病人,多属阴证。疼痛时用手按压,不能缓解的,也是属于阴证,要深刺。患者身上发痒,是病邪在外属阳,要浅刺。病在上部的属阳,病在下部的属阴。\\
病先起于阴经的,应该先治疗阴经,然后再治疗阳经;病先起于阳经的,应该先治疗阳经,然后再治疗阴经。针刺热厥,留针可以由热转寒;针刺寒厥,留针可以由寒转热。针刺热厥,当刺阴经二次,刺阳经一次;针刺寒厥,当刺阳经二次,阴经一次。所谓二阴的意思,就是在阴经针刺二次,一阳的意思,就是在阳经针刺一次。患病的时间长了,病邪深入脏腑。针治这类宿疾,应该深刺并且长时间地留针,每隔一日,再继续针刺。还要首先察明病邪在左在右的偏盛现象,去掉血脉中的淤滞。针刺的原则无非就是这些。\\
大凡针刺的法则,必须诊察患者的形气。形肉虽然不显消瘦,但是气短,脉又躁动而快,出现了躁动而且快的脉象,就应当采用缪刺法。使耗散的真气可以收住,积聚的邪气可以散去。在针刺时,医生就好像深居静处,只与神往来;又像闭户塞窗,意识不乱。念头单纯,心神一贯,精气不分,听不到旁人的声音,从而使精神内守,专一地集中在针刺上。浅刺留针,或微捻提针,以转移病人的精神紧张,直到针下得气为止。针刺之时,男子浅刺候气于外,女子深刺候气于内,坚拒正气不使之出。严防邪气不使之入,这叫做得气。\\
大凡针刺禁忌:行房不久的不能刺,针刺不久的不能行房。酒醉的不能刺,针刺不久的不能醉酒。刚发怒的不能刺,刚针刺的不能动怒。疲劳的不能刺,刚针刺的不能过劳。刚吃饱的不能刺,刚针刺的不能饱餐。饥饿的不能刺,刚针刺的不能饥饿。口渴的不能刺,刚针刺的不能口渴。大惊大恐的,必先安定神气,再行针刺。乘车从远来的,要安卧休息约一顿饭的工夫,再行针刺。步行来的,要坐下来休息,大约走十里路的时间,再行刺针。\\
以上总计十二种针刺禁忌,都是因为血脉运行紊乱,正气耗散,营卫失调,经气不能依次循行。在此情形下针刺,则使阳分病深入到阴分,阴分病外出波及到阳分,以致邪气更盛而加重病情。低劣的医生不能体察这些,而妄用针刺,这叫做摧残人的身体。结果病者正气耗损,体力衰弱,甚至脑髓消耗,不能化生津液,五味营养脱失,这叫做真气丧失。\\
太阳经脉血气将绝的症候是:病者两目上视而不转动,角弓反张,手足抽搐,面色苍白,皮肤绝无血色,乃至绝汗,汗出如珠,绝汗一出,便要死亡了。少阳经脉血气将绝的症候是:病者耳聋,周身骨节松弛,眼球后连于脑的脉络气血断绝,见到目系绝的现象,约一天半就要死亡。死时,面色青白,那就要死亡了。阳明经脉血气将绝的症候是:病者口眼抽动,易于惊惕,胡言乱语,面色发黄,手足阳明经脉循行的部位上脉躁动而盛,血气不行,就要死亡了。少阴经脉血气将绝的症候是:病者面色黎黑,齿龈萎缩而牙齿变长,并且污垢不泽,脘腹胀满,气机滞塞,上下不通,就要死亡了。厥阴经脉血气将绝的症候是:病者内热,喉咙干渴,尿失禁,心中烦乱,甚至舌卷曲,阴囊与睾丸上缩,就要死亡了。太阴经脉血气将绝的症候是:病者腹胀,大便不通,呼吸不利,嗳气,常常呕吐,呕时气就上逆,气上逆就面部发红,如气不上逆,就上下不通而面色发黑,皮毛憔悴而死亡了。\\
卷三\\
经脉第十\\
雷公问于黄帝曰:《禁服》之言,凡刺之理,经脉为始。营其所行,制其度量。内次五脏,外别六腑。愿尽闻其道。\\
黄帝曰:人始生,先成精,精成而脑髓生;骨为干,脉为营,筋为刚,肉为墙;皮肤坚而毛发长。谷入于胃,脉道以通,血气乃行。\\
雷公曰:愿卒闻经脉之始生。\\
黄帝曰:经脉者,所以能决死生,处百病,调虚实,不可不通。\\
肺手太阴之脉,起于中焦,下络大肠,还循胃口,上膈属肺。从肺系横出腋下,下循臑内,行少阴心主之前,下肘中,循臂内,上骨下廉,入寸口,上鱼,循鱼际,出大指之端;其支者,从腕后直出次指内廉,出其端。\\
是动则病肺胀满,膨膨而喘咳,缺盆中痛,甚则交两手而瞀,此为臂厥。是主肺所生病者,咳,上气喘渴,烦心胸满,臑臂内前廉痛厥,掌中热。气盛有余,则肩背痛,风寒,汗出中风,小便数而欠。气虚,则肩背痛寒,少气不足以息,溺色变。为此诸病,盛则泻之,虚则补之,热则疾之,寒则留之,陷下则灸之,不盛不虚,以经取之。盛者寸口大三倍于人迎,虚者则寸口反小于人迎也。\\
大肠手阳明之脉,起于大指次指之端,循指上廉,出合谷两骨之间,上入两筋之中,循臂上廉,入肘外廉,上臑外前廉,上肩,出髃骨之前廉,上出于柱骨之会上,下入缺盆络肺,下膈属大肠;其支者,从缺盆上颈贯颊,入下齿中,还出挟口,交人中,左之右,右之左,上挟鼻孔。\\
是动则病齿痛颈肿。是主津液所生病者,目黄,口干,鼽衄,喉痹,肩前臑痛,大指次指痛不用。气有余,则当脉所过者热肿;虚,则寒栗不复。为此诸病,盛则泻之,虚则补之,热则疾之,寒则留之,陷下则灸之,不盛不虚,以经取之。盛者人迎大三倍于寸口,虚者人迎反小于寸口也。\\
胃足阳明之脉,起于鼻之交伔中,旁纳太阳之脉,下循鼻外,入上齿中,还出挟口,环唇,下交承浆,却循颐后下廉,出大迎,循颊车,上耳前,过客主人,循发际,至额颅;其支者,从大迎前下人迎,循喉咙,入缺盆,下膈,属胃,络脾;其直者,从缺盆下乳内廉,下挟脐,入气街中;其支者,起于胃口,下循腹里,下至气街中而合,以下髀关,抵伏兔,下膝膑中,下循胫外廉,下足跗,入中指内间;其支者,下廉三寸而别,下入中指外间;其支者,别跗上,入大指间,出其端。\\
是动则病洒洒振寒,善伸,数欠,颜黑,病至则恶人与火,闻木声则惕然而惊,心欲动,独闭户塞牖而处,甚则欲上高而歌,弃衣而走,贲响腹胀,是为骭厥。是主血所生病者,狂瘛,温淫汗出,鼽衄,口㖞,唇胗,颈肿,喉痹,大腹水肿,膝膑肿痛,循膺、乳、气街、股、伏兔、骭外廉、足跗上皆痛,中指不用。气盛,则身以前皆热,其有余于胃,则消谷善饥,溺色黄。气不足,则身以前皆寒栗,胃中寒则胀满。为此诸病,盛则泻之,虚则补之,热则疾之,寒则留之,陷下则灸之,不盛不虚,以经取之。盛者,人迎大三倍于寸口;虚者,人迎反小于寸口也。\\
脾足太阴之脉,起于大指之端,循指内侧白肉际,过核骨后,上内踝前廉,上踹内,循胫骨后,交出厥阴之前,上膝股内前廉,入腹属脾络胃,上膈,挟咽,连舌本,散舌下;其支者,复从胃,别上膈,注心中。\\
是动则病舌本强,食则呕,胃脘痛,腹胀善噫,得后与气,则快然如衰,身体皆重。是主脾所生病者,舌本痛,体不能动摇,食不下,烦心,心下急痛,溏、瘕泄、水闭,黄疸,不能卧,强立,股膝内肿、厥,足大指不用。为此诸病,盛则泻之,虚则补之,热则疾之,寒则留之,陷下则灸之,不盛不虚,以经取之。盛者,寸口大三倍于人迎;虚者,寸口反小于人迎也。\\
心手少阴之脉,起于心中,出属心系,下膈络小肠;其支者,从心系上挟咽,系目系;其直者,复从心系却上肺,下出腋下,下循臑内后廉,行手太阴心主之后,下肘内,循臂内后廉,抵掌后锐骨之端,入掌内后廉,循小指之内出其端。\\
是动则病嗌干心痛,渴而欲饮,是为臂厥。是主心所生病者,目黄胁痛,臑臂内后廉痛厥,掌中热痛。为此诸病,盛则泻之,虚则补之,热则疾之,寒则留之,陷下则灸之,不盛不虚,以经取之。盛者,寸口大再倍于人迎;虚者,寸口反小于人迎也。\\
小肠手太阳之脉,起于小指之端,循手外侧上腕,出踝中,直上循臂骨下廉,出肘内侧两筋之间,上循臑外后廉,出肩解,绕肩胛,交肩上,入缺盆络心,循咽下膈,抵胃属小肠;其支者,从缺盆循颈上颊,至目锐眦,却入耳中;其支者,别颊上帄抵鼻,至目内眦,斜络于颧。\\
是动则病嗌痛颔肿,不可以顾,肩似拔,臑似折。是主液所生病者,耳聋、目黄、颊肿,颈、颔、肩、臑、肘、臂外后廉痛。为此诸病,盛则泻之,虚则补之,热则疾之,寒则留之,陷下则灸之,不盛不虚,以经取之。盛者,人迎大再倍于寸口;虚者,人迎反小于寸口也。\\
膀胱足太阳之脉,起于目内眦,上额交巅;其支者,从巅至耳上角;其直者,从巅入络脑,还出别下项,循肩髃内,挟脊抵腰中,入循膂,络肾属膀胱;其支者,从腰中下挟脊贯臀,入腘中;其支者,从髆内左右,别下,贯胛,挟脊内,过髀枢,循髀外,从后廉下合腘中,以下贯踹内,出外踝之后,循京骨,至小指外侧。\\
是动则病冲头痛,目似脱,项似拔,脊痛,腰似折,髀不可以曲,腘如结,踹如裂,是为踝厥。是主筋所生病者,痔、疟、狂、癫疾,头匢项痛,目黄、泪出、鼽衄,项、背、腰、尻、腘、踹、脚皆痛,小指不用。为此诸病,盛则泻之,虚则补之,热则疾之,寒则留之,陷下则灸之,不盛不虚,以经取之。盛者,人迎大再倍于寸口;虚者,人迎反小于寸口也。\\
肾足少阴之脉,起于小指之下,邪走足心,出于然谷之下,循内踝之后,别入跟中,以上踹内,出腘内廉,上股内后廉,贯脊,属肾,络膀胱;其直者,从肾上贯肝膈,入肺中,循喉咙,挟舌本;其支者,从肺出络心,注胸中。\\
是动则病饥不欲食,面如漆柴,咳唾则有血,喝喝而喘,坐而欲起,目丼丼,如无所见,心如悬,若饥状;气不足则善恐,心惕惕,如人将捕之,是为骨厥。是主肾所生病者,口热舌干,咽肿上气,嗌干及痛,烦心,心痛,黄疸,肠澼,脊股内后廉痛,痿厥嗜卧,足下热而痛。为此诸病,盛则泻之,虚则补之,热则疾之,寒则留之,陷下则灸之,不盛不虚,以经取之。灸则强食生肉,缓带披发,大杖重履而步。盛者,寸口大再倍于人迎;虚者,寸口反小于人迎者。\\
心主手厥阴心包络之脉,起于胸中,出属心包络,下膈,历络三焦;其支者,循胸出胁,下腋三寸,上抵腋,下循臑内,行太阴少阴之间,入肘中,下臂行两筋之间,入掌中,循中指出其端;其支者,别掌中,循小指次指出其端。\\
是动则病手心热,臂肘挛急,腋肿,甚则胸胁支满,心中澹澹大动,面赤目黄,喜笑不休。是主脉所生病者,烦心心痛,掌中热。为此诸病,盛则泻之,虚则补之,热则疾之,寒则留之,陷下则灸之,不盛不虚,以经取之。盛者,寸口大一倍于人迎;虚者,寸口反小于人迎也。\\
三焦手少阳之脉,走于小指次指之端,上出两指之间,循手表腕,出臂外两骨之间,上贯肘,循臑外,上肩,而交出足少阳之后,入缺盆,布膻中,散络心包,下膈,循属三焦;其支者,从膻中上出缺盆,上项,系耳后直上,出耳上角,以屈下颊至帄;其支者,从耳后入耳中,出走耳前,过客主人前,交颊,至目锐眦。\\
是动则病耳聋浑浑焞焞,嗌肿喉痹。是主气所生病者,汗出,目锐眦痛,颊痛,耳后肩臑肘臂外皆痛,小指次指不用。为此诸病,盛则泻之,虚者补之,热则疾之,寒则留之,陷下则灸之,不盛不虚,以经取之。盛者,人迎大一倍于寸口;虚者,人迎反小于寸口也。\\
胆足少阳之脉,起于目锐眦,上抵头角,下耳后,循颈行手少阳之前,至肩上,却交出手少阳之后,入缺盆;其支者,从耳后入耳中,出走耳前,至目锐眦后;其支者,别锐眦,下大迎,合于手少阳,抵于帄,下加颊车,下颈合缺盆,以下胸中,贯膈络肝属胆,循胁里,出气街,绕毛际,横入髀厌中;其直者,从缺盆下腋,循胸过季胁,下合髀厌中,以下循髀阳,出膝外廉,下外辅骨之前,直下抵绝骨之端,下出外踝之前,循足跗上,入小指次指之间;其支者,别跗上,入大指之间,循大指歧骨内出其端,还贯爪甲,出三毛。\\
是动则病口苦,善太息,心胁痛,不能转侧,甚则面微有尘,体无膏泽,足外反热,是为阳厥。是主骨所生病者,头痛颔痛,目锐眦痛,缺盆中肿痛,腋下肿,马刀侠瘿,汗出振寒,疟,胸、胁、肋、髀、膝外至胫绝骨外踝前及诸节皆痛,小指次指不用。为此诸病,盛则泻之,虚则补之,热则疾之,寒则留之,陷下则灸之,不盛不虚,以经取之。盛者,人迎大一倍于寸口;虚者,人迎反小于寸口也。\\
肝足厥阴之脉,起于大趾丛毛之际,上循足跗上廉,去内踝一寸,上踝八寸,交出太阴之后,上腘内廉,循股阴入毛中,过阴器,抵小腹,挟胃属肝络胆,上贯膈,布胁肋,循喉咙之后,上入颃颡,连目系,上出额,与督脉会于巅;其支者,从目系下颊里,环唇内;其支者,复从肝别贯膈,上注肺。\\
是动则病腰痛不可俯仰,丈夫毌疝,妇人少腹肿,甚则嗌干,面尘脱色。是主肝所生病者,胸满呕逆,飧泄狐疝,遗溺闭癃。为此诸病,盛则泻之,虚则补之,热则疾之,寒则留之,陷下则灸之,不盛不虚,以经取之。盛者,寸口大一倍于人迎;虚者,寸口反小于人迎也。\\
手太阴气绝,则皮毛焦。太阴行气,温于皮毛者也。故气不荣,则皮毛焦;皮毛焦,则津液去皮节;津液去皮节者,则爪枯毛折;毛折者,则毛先死。丙笃丁死,火胜金也。\\
手少阴气绝,则脉不通。少阴者,心脉也;心者,脉之合也。脉不通,则血不流;血不流,则髦色不泽。故其面黑如漆柴者,血先死。壬笃癸死,水胜火也。\\
足太阴气绝者,则脉不荣肌肉。唇舌者,肌肉之本也。脉不荣,则肌肉软;肌肉软,则舌萎,人中满;人中满,则唇反;唇反者,肉先死。甲笃乙死,木胜土也。\\
足少阴气绝,则骨枯。少阴者,冬脉也,伏行而濡骨髓者也。故骨不濡,则肉不能著也;骨肉不相亲,则肉软却;肉软却,故齿长而垢,发无泽;发无泽者,骨先死。戊笃己死,土胜水也。\\
足厥阴气绝,则筋绝。厥阴者,肝脉也;肝者,筋之合也;筋者,聚于阴器,而脉络于舌本也。故脉弗荣,则筋急;筋急,则引舌与卵。故唇青、舌卷、卵缩,则筋先死。庚笃辛死,金胜木也。\\
五阴气俱绝,则目系转,转则目运。目运者,为志先死。志先死,则远一日半死矣。六阳气绝,则阴与阳相离,离则腠理发泄,绝汗乃出。故旦占夕死,夕占旦死。\\
经脉十二者,伏行分肉之间,深而不见;其常见者,足太阴过于外踝之上,无所隐故也。诸脉之浮而常见者,皆络脉也。六经络手阳明少阳之大络,起于五指间,上合肘中。饮酒者,卫气先行皮肤,先充络脉,络脉先盛,故卫气已平,营气乃满,而经脉大盛。脉之卒然动者,皆邪气居之,留于本末,不动则热。不坚则陷且空,不与众同,是以知其何脉之动也。\\
雷公曰:何以知经脉之与络脉异也?\\
黄帝曰:经脉者常不可见也,其虚实也,以气口知之。脉之见者,皆络脉也。\\
雷公曰:细子无以明其然也。\\
黄帝曰:诸络脉皆不能经大节之间,必行绝道而出,入复合于皮中,其会皆见于外。故诸刺络脉者,必刺其结上。甚血者虽无结,急取之以泻其邪而出其血,留之发为痹也。凡诊络脉,脉色青则寒且痛,赤则有热。胃中寒,手鱼之络多青矣;胃中有热,鱼际络赤。其暴黑者,留久痹也;其有赤有黑有青者,寒热气也;其青短者,少气也。凡刺寒热者皆多血络。必间日而一取之,血尽而止,乃调其虚实。其小而短者少气,甚泻之则闷,闷甚则仆,不得言。闷则急坐之也。\\
手太阴之别,名曰列缺。起于腕上分间,并太阴之经直入掌中,散入于鱼际。其病实,则手锐掌热;虚,则欠匟,小便遗数。取之,去腕半寸。别走阳明也。\\
手少阴之别,名曰通里。去腕一寸半,别而上行,循经入于咽中,系舌本,属目系。其实则支隔,虚则不能言。取之掌后一寸。别走太阳也。\\
手心主之别,名曰内关。去腕二寸,出于两筋之间,别走少阳。循经以上,系于心,包络心系。实则心痛,虚则为烦心。取之两筋间也。\\
手太阳之别,名曰支正。上腕五寸,内注少阴;其别者,上走肘,络肩髃。实则节弛肘废,虚则生肬,小者如指痂疥。取之所别也。\\
手阳明之别,名曰偏历。去腕三寸,别入太阴;其别者,上循臂,乘肩髃,上曲颊偏齿;其别者,入耳,合于宗脉。实则龋齿耳聋,虚则齿寒痹隔。取之所别也。\\
手少阳之别,名曰外关。去腕二寸,外绕臂,注胸中,合心主。病实则肘挛,虚则不收。取之所别也。\\
足太阳之别,名曰飞阳。去踝七寸,别走少阴。实则鼽窒,头背痛;虚则鼽衄。取之所别也。\\
足少阳之别,名曰光明。去踝五寸,别走厥阴,下络足跗。实则厥,虚则痿躄,坐不能起。取之所别也。\\
足阳明之别,名曰丰隆。去踝八寸,别走太阴;其别者,循胫骨外廉,上络头项,合诸经之气,下络喉嗌。其病气逆则喉痹瘁瘖。实则狂癫,虚则足不收,胫枯。取之所别也。\\
足太阴之别,名曰公孙。去本节之后一寸,别走阳明;其别者,入络肠胃。厥气上逆则霍乱。实则肠中切痛,虚则鼓胀。取之所别也。\\
足少阴之别,名曰大钟。当踝后绕跟,别走太阳;其别者,并经上走于心包,下贯腰脊。其病气逆则烦闷,实则闭癃,虚则腰痛。取之所别者也。\\
足厥阴之别,名曰蠡沟。去内踝五寸,别走少阳;其别者,经胫上睾,结于茎。其病气逆则睾肿卒疝。实则挺长,虚则暴痒。取之所别也。\\
任脉之别,名曰尾翳。下鸠尾,散于腹。实则腹皮痛,虚则痒搔。取之所别也。\\
督脉之别,名曰长强。挟膂上项,散头上,下当肩胛左右,别走太阳,入贯膂。实则脊强,虚则头重。高摇之,挟脊之有过者。取之所别也。\\
脾之大络,名曰大包。出渊腋下三寸,布胸胁。实则身尽痛,虚则百节尽皆纵。此脉若罗络之血者,皆取之脾之大络脉也。\\
凡此十五络者,实则必见,虚则必下。视之不见。求之上下。人经不同,络脉异所别也。\\
雷公问黄帝说:《禁服》篇说过,针刺的道理,从研究经脉开始。揣度它的运行,知道它的长短,向内联系五脏,在外联系六腑。希望详细地听听其中的道理。\\
黄帝说:人最初生成,首先形成精,由精发育而生脑髓;此后就逐渐形成人体。以骨为支柱,以经脉作为营运气血的通道,以坚劲的筋来约束骨骼,肌肉像墙一样卫护机体;到皮肤坚韧、毛发生长,人形即成。出生以后,水谷入胃,化生精微,脉道内外贯通,血气即可在脉中运行不止。\\
雷公说:我希望听到经脉最初发生的情况。\\
黄帝说:经脉的作用,可以决断死生,处理百病,察明虚实,作为医生,不可不明白。\\
肺手太阴的经脉,从中焦腹部起始,下绕大肠,返回循着胃的上口,上膈膜,属于肺。再从气管横走而出腋下,沿着上臂内侧,行在手少阴与手厥阴两经的前面,下至肘内,沿着臂的内侧和掌后高骨下缘,入寸口,沿着鱼际,出拇指尖端;它的支脉,从手腕后,直出食指尖端内侧,与手阳明大肠经相接。\\
外邪侵犯本经而发生的病证是肺部膨膨胀满,咳嗽气喘,缺盆疼痛,重则可见两手交叉按于胸前,视物模糊不清,这是臂厥病。本经所主的肺脏发生病变,可见咳嗽,呼吸急迫,心烦胸闷,臑臂部内侧前缘疼痛厥冷,或掌心发热。气盛有余,则肩背疼痛,畏风寒,汗出中风,小便频数而量少。气虚,则肩背疼痛发凉,气短,小便颜色改变。治疗这些病证,实证用泻法,虚证用补法,热证用速刺法,寒证用留针法,络脉虚陷的用灸法,不实不虚的从本经取治。本经气盛,寸口脉比人迎脉大三倍;气虚,寸口脉反小于人迎脉。\\
大肠手阳明的经脉,起始于食指尖端,沿食指上侧,出合谷穴拇指、食指歧骨之间,上入腕上两筋凹陷处,沿前臂上方,入肘外侧,再沿上臂外侧前缘,上肩,出肩端的前缘,上出于肩胛上,与诸阳经会合于大椎,向下入缺盆络肺,下贯膈膜,会属于大肠;它的支脉,从缺盆上走颈部,通过颊部,下入齿缝中,回转过来绕至上唇,左右两脉交会于人中,左脉向右,右脉向左,上行挟于鼻孔两侧,与足阳明胃经相接。\\
外邪侵犯本经而发生的病证是牙齿疼痛,颈部肿大。本腑所主的津液发生病变,可出现眼睛发黄,口中发干,鼻流清涕或出血,喉中肿痛,肩前及上臂作痛,食指疼痛,不能运动。气有余的实证是,在本经脉循行所过的部位上发热而肿;气不足的虚证是,恶寒战栗,且难以回复温暖。治疗这些病证,实证用泻法,虚证用补法,热证用速刺法,寒证用留针法,络脉虚陷的用灸法,不实不虚的从本经取治。本经气盛,寸口脉比人迎脉大三倍;气虚,寸口脉反小于人迎脉。\\
胃足阳明的经脉,起于鼻孔两旁的迎香穴,旁入足太阳的经脉,下沿鼻外侧,入上齿缝中,回来环绕口唇,下交于承浆穴处,再沿腮下后方,出大迎穴,沿颊车穴,上至耳前,通过客主人穴,沿发际,至额颅部;它的支脉,从大迎穴的前面,向下至人迎穴,沿喉咙入缺盆,下贯膈膜,会于胃腑,与脾脏联系;它另有一支直行经脉,从缺盆下至乳房的内侧,再向下挟脐,入毛际两旁气街部;另一支脉,起胃下口,下循腹里,至气街前与直行的经脉相合,循髀关穴,至伏兔部,下至膝盖,沿胫骨前外侧,下至足背,入中指内侧;另一支脉,从膝下三寸处别行,下至足中指外侧;它另一支脉,从足背面,进入足大指,直出大指尖端,与足太阴脾经相接。\\
外邪侵犯本经而发生的病证有发寒战抖,频频伸腰呵欠,额部黧黑,发病时厌恶见人和火光,听到木器声响就会害怕,心跳不安,喜欢关闭门窗独居内室等症状,甚至会登高歌唱,脱衣而跑,且有肠鸣腹胀,这叫“骭厥”。本经主血,所发病变,有因高热以致发狂抽搐,温病,汗自出,鼻流清涕或出血,口唇生疮疹,颈肿,喉肿闭塞;因水停而腹肿大,膝盖部肿痛,沿胸侧、乳部、气街、股、伏兔、足胫外缘、足背上均有痛感,足中趾不能屈伸。气盛见,胸腹部都发热;胃热盛则消谷而易饥,小便色黄。气不足见胸腹部寒凉,如胃中有寒气,可发胀满。治疗这些病证,实证用泻法,虚证用补法,热证用速刺法,寒证用留针法,络脉虚陷的用灸法,不实不虚的从本经取治。本经气盛,寸口脉比人迎脉大三倍;气虚,寸口脉反小于人迎脉。\\
脾足太阴的经脉,起于足大指尖端,沿着大指内侧白肉处,经过核骨,上行至内踝前面,再上小腿肚,沿胫骨后方,与厥阴肝经交叉出于其前,上行膝股内侧前缘,入腹,属脾、络胃,上过膈膜,挟行咽喉部,连于舌根,并散布于舌下;它的支脉,又从胃腑分出,别出上走膈,注入心中,与手少阴心经相接。\\
外邪侵犯本经发生的病证是舌根发硬不柔和,食后就呕吐,胃脘疼痛,腹胀,嗳气,大便或矢气后,感到轻松如病去一样,周身沉重。本经所主的脾脏发生病变,会出现舌根疼痛,身体不能活动,饮食不下,心烦,心下牵引疼痛,大便稀薄或下痢,或小便不通,黄疽,不能安卧,勉强站立时,则大腿、膝内侧肿痛厥冷,足大趾不能活动。治疗这些病证,实证用泻法,虚证用补法,热证用速刺法,寒证用留针法,络脉虚陷的用灸法,不实不虚的从本经取治。本经气盛,寸口脉比人迎脉大三倍;气虚,寸口脉反小于人迎脉。\\
心手少阴的经脉,起于心脏里,出属于心的脉络,下贯膈膜,联络小肠;它的支脉,从心系的脉络上行,挟于咽喉,关联到目珠连于脑的脉络;它另有直行的经脉,又从心脏的脉络上行于肺部,向下横出腋下,再向下沿上臂内侧的后缘,行于手太阴肺经和手厥阴心包络经的后面,下行肘内,沿着前臂内侧的后缘,到掌后小指侧高骨的尖端,入掌内后侧,沿着小指的内侧至指端。\\
外邪侵犯本经发生的病证是咽喉干燥,心痛,渴欲饮水,这是臂间经气厥逆的臂厥。本经所主的心脏发生病变会出现眼睛发黄,两胁疼痛,上臂膊和小臂内侧后缘疼痛、厥冷,掌心热痛。治疗这些病证,实证用泻法,虚证用补法,热证用速刺法,寒证用留针法,络脉虚陷的用灸法,不实不虚的从本经取治。本经气盛,寸口脉比人迎脉大两倍;气虚,寸口脉反小于人迎脉。\\
小肠手太阳的经脉,起于手小指尖端,循行手外侧,上入腕部,出小指侧的高骨,直上沿前臂骨的下缘,出肘内侧两筋之间,再向上沿上臂外侧后缘,出肩后骨缝,绕行肩胛部,交于肩上,入缺盆,联络心脏。再沿咽部下穿横膈膜,至胃,再向下属于小肠;它的支脉,从缺盆沿头颈上抵颊部,至眼外角,回入耳中;另有支脉,从颊部上眼眶下,至鼻,再至眼内角。斜行络于颧骨部,与足太阳经相接。\\
外邪侵犯本经而发生的病证是咽喉疼痛,颔部肿,不能回头,肩痛如拔,臂痛如折。本经所主的液发生的病变,会出现耳聋,眼睛发黄,颊肿,颈、颔、肩、臑、肘、臂后缘疼痛。治疗这些病证,实证用泻法,虚证用补法,热证用速刺法,寒证用留针法,络脉虚陷的用灸法,不实不虚的从本经取治。本经气盛,人迎脉比寸口脉大两倍;气虚,人迎脉反小于寸口脉。\\
膀胱足太阳的经脉,起于眼内角,向上过额部,会于头顶之上;它的支脉,从头顶至耳上角;它的直行经脉,从头顶入络于脑,还出,另下行过项,沿肩胛骨内侧,夹脊椎两旁,直至腰部,沿脊肉深入,联系肾脏,会于膀胱;它另有支脉,从腰中,会于后阴,通过臀部,直入膝腘窝中;它又有直脉,从左右肩胛骨内侧,另向下行,贯穿肩胛,挟行脊内,过髀枢部,沿大腿外侧后缘,向下行合于膝弯内,又向下通过小腿肚,出外踝骨的后边,沿着京骨,至小指外侧尖端,与足少阴肾经相接。\\
外邪侵犯本经发生的病证,为气上冲而头痛,眼球疼得像脱出一样,项部疼痛似拔,脊背疼痛,腰痛似折,大腿不能屈伸,腘窝如结扎,小腿肚疼痛如裂,这叫踝厥。本经所主的筋发生病变,会出现痔疮、疟疾、狂病、癫疾,囱门及颈项疼痛,眼睛发黄,流泪,鼻流清涕或出血,项、背、腰、尻、腘、踹及脚都疼痛,足小趾不能活动。治疗这些病证,实证用泻法,虚证用补法,热证用速刺法,寒证用留针法,络脉虚陷的用灸法,不实不虚的从本经取治。本经气盛,寸口脉比人迎脉大两倍;气虚,寸口脉反小于人迎脉。\\
肾足少阴的经脉,起于足小指之下,斜向足掌心,出于然谷穴之下,沿着内踝骨的后方,另入足跟,上小腿肚内侧,出腘内侧,上行股部内侧后缘,穿过肾脏,与膀胱联系;其直行的经脉,从肾脏向上经过肝和横膈膜,进入肺脏,沿着喉咙,归结于舌根;它的支脉,从肺联系心脏,注于胸中,与手厥阴心包经相接。\\
外邪侵犯本经而发生的病证是虽有饥饿感却不想进食,面色黑无光,咳吐带血,喘息有声,刚坐下就想起来,两目视物模糊,好像什么都看不见,心慌如悬,如饥饿状;气虚就容易恐惧,心中惊悸,好像有人要来捕捉他一样,这是骨厥。本经脉所主的肾脏发生病变,会出现口热,舌干,咽部肿,气上逆,喉咙发干疼痛,心烦心痛,黄疸,痢疾,脊背、大腿内侧后缘疼痛,足部痿软厥冷,嗜睡,足心发热而痛。治疗这些病证,实证用泻法,虚证用补法,热证用速刺法,寒证用留针法,络脉虚陷的用灸法,不实不虚的从本经取治。如果使用灸法,就该勉强吃生肉,松缓衣带,放开头发,扶着大杖,穿着重履,缓步而走。本经气盛,寸口脉比人迎脉大两倍;气虚,寸口脉反小于人迎脉。\\
心主手厥阴心包络的经脉,起于胸中,出属于心包络,下穿膈膜,依次地联系胸腹的上中下三焦;它的支脉,循行胸中横出胁下,当腋缝下三寸处,又向上行至腋部,沿着上臂内侧,行于手太阴肺经与手少阴心经的中间,入肘中,下循臂,行掌后两筋之间,进入掌中,循中指,至指端;它另有支脉,从掌内分出,沿无名指直达指端,与手少阳三焦经相接。\\
外邪侵犯本经而发生的病证是手心发热,臂肘部拘挛,腋部肿,甚至胸胁胀满,心中动摇不安,面赤,眼黄,喜笑不止。本经所主的脉发生病变,会出现心烦心痛,掌心发热。治疗这些病证,实证用泻法,虚证用补法,热证用速刺法,寒证用留针法,络脉虚陷的用灸法,不实不虚的从本经取治。本经气盛,寸口脉比人迎脉大一倍;气虚,寸口脉反小于人迎脉。\\
三焦手少阳的经脉,起于无名指尖端,上出小指与无名指之间,沿着手背,出前臂外侧两骨的中间,向上穿过肘,沿上臂外侧,上肩,而交出于足少阳胆经之后,入缺盆,分布于膻中,散布络于心包,下过膈膜,依次会属于上中下三焦;它的支脉,从膻中上出缺盆,上颈项,夹耳后,直上出耳上角,由此屈而下行额部,至眼眶下;它另有支脉,从耳后进入耳中,再出走耳前,通过客主人穴的前方,与前支脉会于颊部,而至眼外角,与足少阳胆经相接。\\
外邪侵犯本经而发生的病证是耳聋,听不清楚,咽肿,喉痹。本经所主的气发生病变,出现自汗出,外眼角痛,颊痛,耳后、肩、臑、肘、臂外侧都疼痛,无名指不能运动。治疗这些病证,实证用泻法,虚证用补法,热证用速刺法,寒证用留针法,络脉虚陷的用灸法,不实不虚的从本经取治。本经气盛,人迎脉比寸口脉大一倍;气虚,人迎脉反小于寸口脉。\\
胆足少阳的经脉,起于眼外角,上至额角,向下绕至耳后,沿颈部,行于手少阳三焦经的前面,至肩上,又交叉到手少阳三焦经的后面,而进入缺盆;它的支脉,另从眼外角,下行至大迎穴附近,与手少阳三焦经相合,至眼眶下,向颊车,下颈,与前入缺盆的支脉相合,然后下行胸中,贯膈,络肝,属胆,沿着胁内,出少腹两侧的气街,绕过阴毛际,横入环跳部;它的直行经脉,从缺盆下走腋,沿胸部过季胁,与前支脉合于环跳部,再下沿髀部外侧,出阳陵泉,下行于腓骨之前,直下抵阳辅穴,下出外踝之前,沿着足背,出足小趾与第四趾之间;它的另一支脉,由足背走向足大趾间,沿着大趾的骨缝,到它的尖端,又返回穿入爪甲,出三毛与足厥阴肝经相接。\\
外邪侵犯本经所发生的病证是口苦,时常叹气,胸胁部疼痛,不能翻身,病重的面色晦暗无光,全身皮肤枯槁不泽,足外侧发热,这叫阳厥。本经所主的骨发生病变,会出现头痛,下颔及外眼角疼痛,缺盆肿痛,腋下肿,腋下或颈旁生瘰疬,自汗出而发冷,疟疾,胸、胁、肋、大腿、膝外侧直至胫骨、绝骨、外踝前以及诸关节皆痛,足第四趾不能运动。治疗这些病证,实证用泻法,虚证用补法,热证用速刺法,寒证用留针法,络脉虚陷的用灸法,不实不虚的从本经取治。本经气盛,人迎脉比寸口脉大一倍;气虚,人迎脉反小于寸口脉。\\
肝足厥阴的经脉,起于足大趾丛毛上的大敦穴,沿着足背上侧,至内踝前一寸处,向上至踝骨上八寸处,交叉于足太阴脾经的后方,上腘内缘,沿阴股,入阴毛中,环绕阴器一周,至小腹,夹行于胃的两旁,属肝,络胆,上通膈膜,散布于胁腹部,沿喉咙的后侧,入喉咙的上孔,联系眼球深处的脉络,与督脉会合于巅顶的百会,它的支脉,从眼球深处脉络,向下行于颊部内侧,环绕口唇之内;它另有一支脉,又从肝脏通过膈膜,上注于肺脏与手太阴肺经相接。\\
外邪侵犯本经而发生的病证是腰痛不能俯仰,男子患厊疝,妇女患少腹肿胀,病重的咽喉发干,面色晦暗无光。本经所主的肝脏发生病变,会出现胸中满闷,呕吐气逆,腹泻完谷不化,狐疝,遗尿或小便不通。治疗这些病证,实证用泻法,虚证用补法,热证用速刺法,寒证用留针法,络脉虚陷的用灸法,不实不虚的从本经取治。本经气盛,寸口脉比人迎脉大一倍;气虚,寸口脉反小于人迎脉。\\
手太阴肺经脉气衰竭,皮毛就会憔悴枯槁。手太阴肺,能运行精气,以温润皮毛。所以肺虚而不能营养皮毛,皮毛就憔悴枯槁;皮毛憔悴枯槁,是因为皮肤关节失去了津液的滋养;皮肤关节失去了津液的滋养,就会爪甲枯槁,毫毛折断脱落;毫毛折断脱落,是肺经精气先衰竭的征象。这种病证,逢丙日危重,逢丁日死亡,这是由于火胜金的缘故。\\
手少阴心经的脉气衰竭,则脉道不通。手少阴经是心脏的经脉;心与血脉相配合。如脉道不通,血流就不畅;血流不畅,面色就无光润。所以面色暗黑无光泽是血脉先枯竭的征象。这种病证,逢壬日危重,逢癸日死亡,这是因为水胜火的缘故。\\
足太阴脾经脉气衰竭,经脉就不能营养肌肉。唇舌是肌肉之本。经脉不能输布营养,就会使肌肉松软;肌肉松软则舌体萎缩,人中肿满;人中肿满,口唇就外翻;口唇外翻,是肌肉先死的征象。这种病证,逢甲日危重,逢乙日死亡,这是由于木胜土的缘故。\\
足少阴肾经脉气衰竭,就会使骨枯槁。足少阴肾与冬天相应,其脉伏行在深部而濡养骨髓。如果骨髓得不到肾气濡养,肌肉就不能附着于骨;骨肉不能亲合而分离,肌肉就软弱萎缩;肌肉软缩,就显得齿长而多垢,头发失去光泽;头发不光泽,是骨气先死的征象。这种病证,逢戊日危重,逢己日死亡,这是由于土胜水的缘故。\\
足厥阴肝经脉气衰竭,筋的功能就断绝。足厥阴属于肝脏的经脉;肝脉的外合是筋;经筋聚合在阴器,向上联络舌根。如果肝脉不能养筋,则筋缩拘急;筋急牵引阴囊和舌根。所以出现口唇发青、舌卷、阴囊抽缩,是筋先死的征象。这种病证,逢庚日危重,逢辛日死亡,这是由于金胜木的缘故。\\
五脏阴经的精气都衰竭会出现目系转动;目系转动则眼晕;眼晕为神志先死;神志既丧,最长一天半就要死亡。六腑阳经的精气衰败,阴气与阳气互相分离;阴阳分离则腠理开张,精气外泄,出现绝汗,汗出如珠不止。所以早晨出现这种危象,预计晚上必定死亡,夜间出现危象,预计明天早晨必定死亡。\\
十二经脉,隐伏在体内而行于分肉之间,其深不能看到;经常可以见到的,只是手太阴肺经在经过手外踝之上气口部分,这是由于该处骨露皮浅无所隐蔽的缘故。其他各脉在浅表而经常可见到的,都是络脉。在手足六经络脉中,手阳明大肠经,手少阳三焦经的大络,分别起于手五指之间,上合于肘中。饮酒的人,它的酒气随着卫气行于皮肤,充溢络脉,首先使络脉满盛。就会使卫气均平,营气满盛,那经脉也就很充盛了。人的经脉突然充盛,这都是邪气侵袭于内,留在脏腑经脉里,聚而不动,可以化热。如浮络不现坚实,就是病邪深入,经气虚空,不与一般相同,所以知道哪条经脉发病了。\\
雷公问:怎样能够知道经脉和络脉的不同呢?\\
黄帝说:经脉在平常是看不到的,它的虚实从气口切脉可知。显露在外的脉,都是络脉。\\
雷公说:我不明白这种区别。\\
黄帝说:所有络脉,都不能经过大关节之间,而行于经脉所不到之处,出入流注,再结合皮部的浮络,共同会合而显现在外面。所以针刺所有络脉的病变,必须刺其聚结之处。若血聚过多,虽然没有显现淤结之络,也应该急刺,泻去病邪,放出淤血。如果淤血留内,会发为痹证。凡是察看络脉:脉现青色,是寒邪凝滞并有疼痛;脉现赤色,是有热。胃里有寒,手鱼部的络脉多现青色;胃里有热,鱼际的络脉会出现赤色。鱼际络脉出现黑色的,是日久不愈的痹病。如兼有赤、黑、青三色出现的,是寒热错杂的病变。如青色而短,属于气弱。凡是针刺胃里或寒或热的病证,都是多刺血络。必须间日一刺,把淤血泻完为止。然后察明病证的虚实,如脉现青色而短,是气衰的病人,过用泻法,就会使病人感到心里烦乱,烦乱极了,就会跌倒,不能说话。对于这种烦乱的病人,赶快扶他坐下,施行急救。\\
手太阴经的别出络脉叫列缺。起于腕上的分肉之间,与手太阴经经脉并行,直入手掌中,散布于鱼际处。本络脉发病,邪实的见腕后高骨及手掌发热;正虚的见张口呵欠,小便不禁或频数。治疗这些病证,取腕后一寸半的列缺穴。本络由此别走手阳明经脉。\\
手少阴经的别出络脉名叫通里。它起于腕上一寸半处,别出上行,循本经入于咽中,联系舌根,联属目系。本络脉发病,邪实的出现胸膈间有支撑不舒之感;正虚的出现不能言语。治疗这些病证,取掌后一寸的通里穴。本络由此别走手太阳经脉。\\
手厥阴心包经的别出络脉名叫内关。起于腕上二寸处的两筋之间,由此别走手少阳经。并循本经上行,系于心包,联络心系。本络脉发病,邪气实的见心痛;正气虚的见心中烦乱。治疗这些病证,取腕上二寸两筋间的内关穴。\\
手太阳经的别出络脉名叫支正。起于腕上五寸,向内注于手少阴心经;其别出的向上过肘,向上络于肩髃穴。本络脉发病,邪实的见骨节弛缓,肘关节萎废不用;正虚的会长赘肉,小的赘肉多如指间痂疥那样。治疗这些病证,取本经别出的络穴支正。\\
手阳明经的别出络脉名叫偏历。起于腕上三寸处,别行走入手太阴经;其别而上行的沿臂上肩髃,再上行过颈到曲颊,偏络于齿根;另一别出的络脉,上入耳中,合于该部的主脉。本络脉发病,邪实的见龋齿,耳聋;正虚的见齿寒,膈间闭塞不通。治疗这些病证,取本经别出的络穴偏历。\\
手少阳经的别出络脉名叫外关。起始于腕上二寸处,向外绕行于臂部,注入胸中与手厥阴心包经相会合。本络脉发病,邪实的见肘关节拘挛;正虚的见肘部弛缓不收。治疗这些病证,取本经别出的络穴外关。\\
太阳经的别出络脉名叫飞阳。起于外踝上七寸处,别行走入足少阴经。本络脉发病,邪实的出现鼻流清涕、阻塞不通,头背疼痛;正虚的出现鼻流清涕或出血。治疗这些病证,取本经别出的络穴飞阳。\\
足少阳经的别出络脉名叫光明。起于外踝上五寸,别行走入足厥阴经,向下络于足背。本络脉发病,邪实的见四肢厥冷;正虚的见下肢痿软无力,不能行走,坐而不能起立。治疗这些病证,取本经别出的络穴光明。\\
足阳明经的别出络脉名叫丰隆。它起于外踝上八寸,别行走入足太阴经;其别出而上行的,沿着胫骨的外缘,向上络于头项,与其他各经经气会合,向下绕络于喉咽。本络脉发病,其病气上逆,出现喉痹和突然失音。邪实则神志失常而癫狂;正虚则两足弛缓不收,小腿肌肉萎缩。治疗这些病证,取本经别出的络穴丰隆。\\
足太阴经的别出络脉名叫公孙。起于足大趾本节后一寸,别行走入足阳明经;其别出而上行的,入腹络于肠胃。本络脉发病,其厥气上逆则发为霍乱。邪气实则肠中痛如刀割;正气虚则腹胀如鼓。治疗这些病证,取本经别出的络穴公孙。\\
足少阴经的别出络脉名叫大钟。起于足内踝的后面,环绕足跟,别行走入足太阳经;别出而行的络脉与本经经脉相并上行,走入心包络,向下贯穿腰脊。本络脉发病,其病气上逆则心烦闷乱,邪气实则小便不通,正气虚则腰痛。治疗这些病证,取本经别出的络穴大钟。\\
足厥阴经的别出络脉名叫蠡沟。起于内踝上五寸,别行走入足少阳经;别出而上行的络脉,沿小腿向上到达睾丸,聚于阴茎。其病气上逆则睾丸肿大,突然疝气。邪气实则阳强不倒;正气虚则阴部暴痒。治疗这些病证,取本经别出的络穴蠡沟。\\
任脉经的别出络脉名叫尾翳。由此别出下行,散布于腹部。本络脉发病,邪气实则腹部皮肤痛;正气虚则腹部皮肤作痒。治疗这些病证,取本经别出的络穴尾翳。\\
督脉经的别出络脉名叫长强。由此别出挟脊膂上行到项部,散布于头上,再向下行于肩胛两旁,别行走入足太阳膀胱经,深入贯穿脊膂内。本络脉发病,邪气实则脊柱强直,正气虚则头部沉重。检查时,摇动患者的头项部,可以发现挟脊之脉有病变。取本经别出的络穴长强治疗。\\
足太阴脾经别出的最大络脉名叫大包。从渊腋下三寸,散布于胸胁部。本络脉发病,邪气实则全身疼痛;正气虚则周身骨节弛纵无力。因这一络脉包罗诸络之血,治疗这些病症取本络脉的大包穴。\\
以上十五络脉,邪气实则血满脉中而明显可见,正气虚则脉络陷下而不易看见。如果在外表看不见,可在络脉的上下寻求。由于每个人的经脉不同,络脉也有差异。\\
经别第十一\\
黄帝问于岐伯曰:余闻人之合于天道也,内有五脏,以应五音、五色、五时、五味、五位也;外有六腑,以应六律,六律建阴阳诸经,而合之十二月、十二辰、十二节、十二经水、十二时、十二经脉者,此五脏六腑之所以应天道。夫十二经脉者,人之所以生,病之所以成,人之所以治,病之所以起。学之所始,工之所止也。粗之所易,上之所难也。请问其离合出入,奈何?\\
岐伯稽首再拜曰:明乎哉问也!此粗之所过,上之所息也,请卒言之。\\
足太阳之正,别入于腘中;其一道,下尻五寸,别入于肛,属于膀胱,散之肾,循膂,当心入散;直者,从膂,上出于项,复属于太阳。此为一经也。足少阴之正,至腘中,别走太阳而合,上至肾,当十四椎,出属带脉;直者,系舌本,复出于项,合于太阳。此为一合。或以诸阴之别,皆为正也。\\
足少阳之正,绕髀入毛际,合于厥阴;别者,入季胁之间,循胸里属胆,散之肝,上贯心,以上挟咽,出颐颔中,散于面,系目系,合少阳于外眦也。足厥阴之正,别跗上,上至毛际,合于少阳,与别俱行。此为二合也。\\
足阳明之正,上至髀,入于腹里,属胃,散之脾,上通于心,上循咽,出于口,上伔帄,还系目系,合于阳明也。足太阴之正,上至髀,合于阳明,与别俱行,上络于咽,贯舌中。此为三合也。\\
手太阳之正,指地,别于肩解,入腋走心,系小肠也。手少阴之正,别入于渊腋两筋之间,属于心,上走喉咙,出于面,合目内眦。此为四合也。\\
手少阳之正,指天,别于巅,入缺盆,下走三焦,散于胸中也。手心主之正,别下渊腋三寸,入胸中,别属三焦,出循喉咙,出耳后,合少阳完骨之下,此为五合也。\\
手阳明之正,从手循膺乳,别于肩髃,入柱骨,下走大肠,属于肺,上循喉咙,出缺盆,合于阳明也。手太阴之正,别入渊腋少阴之前,入走肺,散之大肠,上出缺盆,循喉咙,复合阳明。此六合也。\\
黄帝问岐伯说:我听说人与自然天道是相应合的,在内属阴的五脏,与五音、五色、五时、五味、五位相应合;在外属阳的六腑,与六律相应合,六律分六阴律六阳律,而与十二月、十二辰、十二节、十二经水、十二时、十二经脉相应合。这是五脏六腑与自然天道相应合的情况。十二经脉是人体之所以能生存,疾病之所以能形成,人体之所以能维持健康,疾病之所以能治愈的根源。是学习医学的起始,是医生应当终生全力用功学习的,却为粗工所轻视,而高明的医生认为掌握它是很困难的。请你谈谈经脉在人体是怎样离合出入的情况?\\
岐伯恭敬地行礼回答说:问得很高明!这是粗工容易忽略的问题,只有高明的医生才会认真地钻研它。请让我详细地说明。\\
足太阳经脉别出而行的正经,分道而入于腘窝中;另一道至尻下五寸处,别行入于肛门,向内属于膀胱本腑,再散行至肾脏,沿脊内上行,当心部而分散;其直行的,从脊上出于项部,再入属于足太阳本经经脉。这是足太阳别行的一经。足少阴经脉别出而行的正经至腘窝中,别出一脉与太阳经相会合,上行至肾,当十四椎处,外出属于带脉;其直行的经脉,上行系于舌根,复出于项部,与足太阳经相合。这是足太阳与足少阴表里相配的第一合。或以诸阴经的经别与诸阳经的经别相互配合,都称为正经。\\
足少阳经脉别出而行的正经,上行绕于髀部而入阴毛处,与足厥阴经脉会合;其别出一脉入季胁间,沿胸里,入内属于本经胆腑,散行于肝,向上贯穿心部,上行挟咽喉两旁,出于腮部及颔中,散于面部,系于目系,与足少阳本经会合于外眼角。足厥阴经脉别出而行的正经,从足背别行,上行至阴毛处,与足少阳别行的正经相合,向上偕行。这是阴阳表里相配的第二合。\\
足阳明经脉别出而行的正经,上行髀部,进入腹内,入内属于本经胃腑,散行至脾脏,上通于心,上行沿咽部,出于口,再上行至鼻梁及眼眶下方,联系目系,与足阳明本经会合。足太阴经脉别出而行的正经,别而上行至髀部,与足阳明经别行的正经会合而向上偕行,上络于咽部,贯入舌中。这是阴阳表里相配的第三合。\\
手太阳经脉别出而行的正经,自下而上行,从肩后骨缝别行入于腋下,走入心脏,系于小肠本腑。手少阴经脉别出而行的正经,走入腋下三寸足少阳经渊腋穴处两筋之间,入内属于心脏,上走喉咙,出于面部,与手太阳经的一条支脉会合于眼内角。这是阴阳表里相配的第四合。\\
手少阳经脉别出而行的正经,是从上而下的,从巅顶,别行入于缺盆,下走三焦本腑,散于胸中。手厥阴心包经脉别出而行的正经,别出渊腋下三寸处,入于胸中,别行联属三焦本腑,外出上行,沿喉咙,出耳后,与手少阳三焦经会合于完骨的下方。这是阴阳表里相配的第五合。\\
手阳明经脉别出而行的正经,从手上行至侧胸、乳之间,别行出于肩髃穴,入于柱骨,而后向下走入大肠本腑,向上联属肺脏,再向上沿喉咙出缺盆,与手阳明本经会合。手太阴经脉别出而行的正经,别出入于渊腋部手少阴经之前,入肺本脏,散行于大肠,上行出缺盆,沿喉咙,再与手阳明经相合。这是阴阳表里相配的第六合。\\
经水第十二\\
黄帝问于岐伯曰:经脉十二者,外合于十二经水,而内属于五脏六腑。夫十二经水者,其有大小、深浅、广狭、远近各不同,五脏六腑之高下、大小,受谷之多少,亦不等,相应奈何?夫经水者,受水而行之,五脏者,合神气魂魄而藏之;六腑者,受谷而行之,受气而扬之;经脉者,受血而营之。合而以治,奈何?刺之深浅,灸之壮数,可得闻乎?\\
岐伯答曰:善哉问也!天至高,不可度;地至广,不可量。此之谓也。且夫人生于天地之间,六合之内,此天之高、地之广也,非人力之所能度量而至也。若夫八尺之士,皮肉在此,外可度量切循而得之,其死可解剖而视之。其脏之坚脆,腑之大小,谷之多少,脉之长短,血之清浊,气之多少,十二经之多血少气,与其少血多气,与其皆多血气,与其皆少血气,皆有大数。其治以针艾,各调其经气,固其常有合乎。\\
黄帝曰:余闻之,快于耳,不解于心,愿卒闻之。\\
岐伯答曰:此人之所以参天地而应阴阳也,不可不察。足太阳外合于清水,内属于膀胱,而通水道焉;足少阳外合于渭水,内属于胆;足阳明外合于海水,内属于胃;足太阴外合于湖水,内属于脾;足少阴外合于汝水,内属于肾;足厥阴外合于渑水,内属于肝;手太阳外合于淮水,内属于小肠,而水道出焉;手少阳外合于漯水,内属于三焦;手阳明外合于江水,内属于大肠;手太阴外合于河水,内属于肺。手少阴外合于济水,内属于心;手心主外合于漳水,内属于心包。凡此五脏六腑十二经水者,外有源泉,而内有所禀,此皆内外相贯,如环无端,人经亦然。故天为阳,地为阴,腰以上为天,腰以下为地。故海以北者,为阴;湖以北者,为阴中之阴;漳以南者,为阳;河以北至漳者,为阳中之阴;漯以南至江者,为阳中之太阳。此一隅之阴阳也,所以人与天地相参也。\\
黄帝曰:夫经水之应经脉也,其远近浅深,水血之多少,各不同,合而以刺之,奈何?\\
岐伯答曰:足阳明,五脏六腑之海也,其脉大血多,气盛热壮;刺此者,不深弗散,不留不泻也。足阳明,刺深六分,留十呼;足太阳,深五分,留七呼。足少阳,深四分,留五呼;足太阴,深三分,留四呼;足少阴,深二分,留三呼。足厥阴,深一分,留二呼。手之阴阳,其受气之道近,其气之来疾,其刺深者,皆无过二分;其留,皆无过一呼。其少长大小肥瘦,以心撩之,命曰法天之常。灸之亦然。灸而过此者得恶火,则骨枯脉涩;刺而过此者,则脱气。\\
黄帝曰:夫经脉之小大,血之多少,肤之厚薄,肉之坚脆,及夬之大小,可为量度乎?\\
岐伯答曰:其可为度量者,取其中度也,不甚脱肉而血气不衰也。若失度之人,痟瘦而形肉脱者,恶可以度量刺乎。审切循扪按,视其寒温盛衰而调之,是谓因适而为之真也。\\
黄帝问岐伯说:人体的十二经脉,在外与地上的十二条河流相应合,在内则连属五脏六腑。十二条河流,有或大或小、或深或浅、或宽或窄、或远或近的不同;五脏六腑也有或在上或在下、或大或小和容纳饮食多少的不同,两者是怎样相应的呢?经水是受纳地上的水而流行到各地;五脏是聚合神、气、魂、魄而藏储于内;六腑是受纳运输水谷,并将分别出的水谷精气,输送布散到全身;经脉是受纳血液,营运于周身。把这些内容配合地来,运用于治疗,该怎样做呢?还有针刺的深浅,施灸壮数的多少,能说给我听吗?\\
岐伯回答说:你问得很好!天极高,不能测;地极广,不能量。说的就是这个道理。人生活在天地之间,六合之内,这天高地广,不是用人力所能测量的。但是人的身体,皮肉俱在,从外部可计算测量,用手指切按而获得各部的情况;死了以后,可以通过解剖来观察内在的情况。人体五脏的坚脆,六腑的大小,受谷的多少,脉道的长短,血液的清浊,气的多少,以及十二经是多血少气,少血多气,气血都多,还是气血都少等情况,都有一定的规律。运用针刺艾灸治病,调节各经的经气,也都有一定规律。\\
黄帝说:我听了你说的这些道理,听起来很清楚,但心里还不能深刻理解,希望你能详尽地讲给我听。\\
岐伯说:这是人与天地阴阳相参相合的大道理,不可不详细体察。足太阳经在外与清水相合,在内联属膀胱腑,主要功能是通利水道;足少阳经在外与渭水相合,在内联属胆腑;足阳明经在外与海水相合,在内联属胃腑;足太阴经在外与湖水相合,在内联属脾脏;足少阴经在外与汝水相合,在内联属肾脏;足厥阴经在外与渑水相合,在内联属肝脏;手太阳经在外与淮水相合,在内联属小肠,水道由此而出;手少阳经在外与漯水相合,在内联属三焦;手阳明经在外与江水相合,在内联属大肠;手太阴经在外与河水相合,在内联属肺脏;手少阴经在外与济水相合,在内联属心脏;手厥阴经在外与漳水相合,在内联属心包络。以上所说的五脏六腑和十二经水,在外的十二经水各有源泉,在内的五脏六腑各有自然禀赋,这都是内外相互贯通,如圆环一样周而复始没有尽头,人的经脉循行也是如此。天气轻清属阳,地气重浊属阴;人体腰部以上象天属阳,腰部以下象地属阴。以十二经水分阴阳,海水以北属阴;湖水以北,属阴中之阴;漳水以南属阳;河水以北至漳水之间,属阳中之阴;漯水以南至江水之间,属阳中之太阳。这是举大地一部分区域河流的阴阳属性,来说明人与天地阴阳的相应情况。\\
黄帝问:十二经水与十二经脉相应合,它们的远近、深浅以及水血的多少各不相同,把两者结合起来,用于指导针刺是怎样的呢?\\
岐伯回答说:足阳明胃是五脏六腑的“海”,其经脉最大而气血最多,发病时出现大热;所以针刺这一经时,不深刺则邪不能散,不留针则邪不能泻。足阳明经,针刺六分深,留针呼吸十次的时间;足太阳经,针刺五分深,留针呼吸七次的时间;足少阳经,针刺四分深,留针呼吸五次的时间;足太阴经,针刺三分深,留针呼吸四次的时间;足少阴经,针刺二分深,留针呼吸三次的时间;足厥阴经,针刺一分深,留针呼吸两次的时间。手三阴三阳经脉,因其均循行于人体上半身,接受心肺气血的距离较近,气来得迅速,针刺深度一般不超过二分,留针时间一般不超过一次呼吸。至于年龄有老少,身材有大小,身形有胖瘦的不同,医者必须心中有数,因人施治,这叫做顺从天道的常规。灸法也是如此。如果施灸过度,变成“恶火”,就会骨髓枯槁,血脉凝涩;针刺过度,会发生正气虚脱。\\
黄帝问:经脉的大小,血的多少,皮肤的厚薄,肌肉的坚脆,以及刐肉的大小,都可以计量吗?\\
岐伯回答说:可以计量的,要选择中等身材以肌肉不甚消瘦,血气不甚衰弱的人为标准。如果失度的人形体消瘦,以致肌肉尽脱,怎么可以计量以作针刺的标准呢?所以必须通过切、循、扪、按等方法检查,根据病证的寒热虚实来进行调治,这叫做各适其宜而慎重地运用针刺治疗。\\
卷四\\
经筋第十三\\
足太阳之筋,起于足小指,上结于踝,邪上结于膝,其下循足外踝,结于踵,上循跟,结于腘,其别者,结于踹外,上腘中内廉,与腘中并上结于臀,上挟脊,上项;其支者,别入结于舌本;其直者,结于枕骨,上头下颜,结于鼻;其支者,为目上网,下结于圴;其支者,从腋后外廉,结于肩髃;其支者,入腋下,上出缺盆,上结于完骨;其支者,出缺盆,邪上出于圴。其病小指支跟肿痛,腘挛,脊反折,项筋急,肩不举,腋支,缺盆中纽痛,不可左右摇。治在燔针劫刺,以知为数,以痛为输。各曰仲春痹也。\\
足少阳之筋,起于小指次指,上结外踝,上循胫外廉,结于膝外廉;其支者,别起外辅骨,上走髀,前者结于伏兔之上,后者结于尻;其直者,上乘尐季胁,上走腋前廉,系于膺乳,结于缺盆;直者,上出腋,贯缺盆,出太阳之前,循耳后,上额角,交巅上,下走颔,上结于伄;支者,结于目眦,为外维。其病小指次指支转筋,引膝外转筋,膝不可屈伸,腘筋急,前引髀,后引尻,即上乘尐季胁痛,上引缺盆膺乳颈,维筋急,从左之右,右目不开,上过右角,并\\
脉而行,左络于右,故伤左角,右足不用,命曰维筋相交。治在燔针劫刺,以知为数,以痛为输。名曰孟春痹也。\\
足阳明之筋,起于中三指,结于跗上,邪外上加于辅骨。上结于膝外廉,直上结于髀枢,上循胁,属脊;其直者,上循骭,结于膝,其支者,结于外辅骨,合少阳,其直者,上循伏兔,上结于髀,聚于阴器,上腹而布,至缺盆而结,上颈,上挟口,合于伄,下结于鼻,上合于太阳,太阳为目上网,阳明为目下网;其支者,从颊结于耳前。其病足中指支,胫转筋,脚跳坚,伏兔转筋,髀前肿,毌疝,腹筋急,引缺盆及颊,卒口僻,急者目不合,热则筋纵,目不开。颊筋有寒,则急引颊移口,有热则筋弛纵缓,不胜收,故僻。治之以马膏,膏其急者,以白酒和桂,以涂其缓者,以桑钩钩之,即以生桑灰置之坎中,高下以坐等,以膏熨急颊,且饮美酒,噉美炙肉,不饮酒者,自强也,为之三拊而已。治在燔针劫刺,以知为数,以痛为输。名曰季春痹也。\\
足太阴之筋,起于大指之端内侧,上结于内踝;其直者,络于膝内辅骨,上循阴股,结于髀,聚于阴器,上腹,结于脐,循腹里,结于肋,散于胸中;其内者,著于脊。其病足大指支内踝痛,转筋痛,膝内辅骨痛,阴股引髀而痛,阴器纽痛,下引脐两胁痛,引膺中脊内痛。治在燔针劫刺,以知为数,以痛为输。命曰孟秋痹也。\\
足少阴之筋,起于小指之下,并足太阴之筋,邪走内踝之下,结于踵,与太阳之筋合,而上结于内辅之下,并太阴之筋而上循阴股,结于阴器,循脊内挟膂,上至项,结于枕骨,与足太阳之筋合。其病足下转筋,及所过而结者皆痛及转筋。病在此者,主痫瘛及痉,在外者不能俯,在内者不能仰。故阳病者腰反折不能俯,阴病者不能仰。治在燔针劫刺,以知为数,以痛为输,在内者熨引饮药。发数甚者,死不治。名曰仲秋痹也。\\
足厥阴之筋,起于大指之上,上结于内踝之前,上循胫,上结内辅之下,上循阴股,结于阴器,络诸筋。其病足大指支内踝之前痛,内辅痛,阴股痛转筋,阴器不用,伤于内则不起,伤于寒则阴缩入,伤于热则纵挺不收。治在行水,清阴气。其病转筋者,治在燔针劫刺,以知为数,以痛为输。命曰季秋痹也。\\
手太阳之筋,起于小指之上,结于腕,上循臂内廉,结于肘内锐骨之后,弹之应小指之上,入结于腋下;其支者,后走腋后廉,上绕肩胛,循颈,出走太阳之前,结于耳后完骨;其支者,入耳中;直者,出耳上,下结于颔,上属目外眦。其病小指支肘内锐骨后廉痛,循臂阴,入腋下,腋下痛,腋后廉痛,绕肩胛引颈而痛,应耳中鸣痛,引颔,目瞑良久,乃得视,颈筋急,则为筋瘘颈肿。寒热在颈者,治在燔针劫刺,以知为数,以痛为输。其为肿者,复而锐之。本支者,上曲牙,循耳前,属目外眦,上颔,结于角。其痛当所过者,支转筋。治在燔针劫刺,以知为数,以痛为输。名曰仲夏痹也。\\
手少阳之筋,起于小指次指之端,结于腕,中循臂,结于肘,上绕臑外廉,上肩走颈,合手太阳;其支者,当曲颊,入系舌本;其支者,上曲牙,循耳前,属目外眦,上乘颔,结于角。其病当所过者即支转筋,舌卷。治在燔针劫刺,以知为数,以痛为腧。名曰季夏痹也。\\
手阳明之筋,起于大指次指之端,结于腕,上循臂,上结于肘外,上臑,结于髃;其支者,绕肩胛,挟脊,直者,从肩髃上颈;其支者,上颊,结于伄,直者,上出手太阳之前,上左角,络头,下右颔。其病当所过者,支痛及转筋,肩不举,颈不可左右视。治在燔针劫刺,以知为数,以痛为输,名曰孟夏痹也。\\
手太阴之筋,起于大指之上,循指上行,结于鱼后,行寸口外侧,上循臂,结肘中,上臑内廉,入腋下,出缺盆,结肩前髃,上结缺盆,下结胸里,散贯贲,合贲下,抵季胁。其病当所过者,支转筋,痛甚成息贲,胁急吐血。治在燔针劫刺,以知为数,以痛为输。名曰仲冬痹也。\\
手心主之筋,起于中指,与太阴之筋并行,结于肘内廉,上臂阴,结腋下,下散前后挟胁;其支者,入腋,散胸中,结于贲。其病当所过者,支转筋,前及胸痛,息贲。治在燔针劫刺,以知为数,以痛为输。名曰孟冬痹也。\\
手少阴之筋,起于小指之内侧,结于锐骨,上结肘内廉,上入腋,交太阴,挟乳里,结于胸中,循贲,下系于脐。其病内急,心承伏梁,下为肘网。其病当所过者,支转筋,筋痛。治在燔针劫刺,以知为数,以痛为输。其成伏梁唾血脓者,死不治。经筋之病,寒则反折筋急,热则筋弛纵不收,阴痿不用。阳急则反折,阴急则俯不伸。焠刺者,刺寒急也,热则筋纵不收,无用燔针。名曰季冬痹也。\\
足之阳明,手之太阳,筋急则口目为噼,眦急不能卒视,治皆如右方也。\\
足太阳膀胱经的筋,起于足小趾,上结于足外踝,斜上再结于膝,在下面的沿足外侧,结于踵部,由踵部沿足跟上行结于膝腘窝;别行的另一支,结于腿肚外侧,上行至膝腘内缘,与前在腘中的一支并行,上结于臀部,再向上挟脊柱两侧至项部;由此又分出一支,别行入内结于舌根;自项部直行的那支,结于枕骨,上行头顶,下至颜面,结于鼻部;由此分出一支,是上眼皮的纲维,下行结于颧骨部;又分出一支,从腋窝后方外缘,结于肩髃穴;又有一分支,入腋下方,再上出于缺盆部,上行结于耳后的完骨部;又有一分支,自缺盆部,斜上出于颧骨部。本经筋发生的病证,为足小趾牵引着足跟部肿痛,膝腘拘挛,脊柱反张,项部拘急,肩臂不能上举,腋部引及缺盆部纠结作痛,不能左右摇动。治疗用火针,用快速的手法,以病见效为针刺次数的限度,以病处的痛点为腧穴。这种病叫仲春痹。\\
足少阳经的筋,起于足的第四趾,上结于外踝,上沿胫骨外缘,结于膝外缘的阳陵泉;由此分出一支,自外辅骨处别行,上走髀部,前支结于伏兔部,分支结于尻部;直行的上行至季胁下空软处,再向上走腋部的前缘,系于侧胸与乳部,上结于缺盆部;又一支直行的上出于腋部,通过缺盆,出太阳经筋之前,沿耳后,绕上额角,交会于巅顶,再向下走下巴颏,上结于颧骨部;另一分支结于眼外角,为眼之外维。本经筋发生的病证,为足第四趾抽筋,牵引膝外侧抽筋,膝关节屈伸不利,膝腘的筋拘急,前方牵引髀部,后方牵引尻部,并且上乘季胁下空软处与季胁部疼痛,向上牵引缺盆、侧胸、乳、颈等部所维系的筋都拘急,左右相交,向上至面部,从左向右的筋拘急则右目不能张开,上至右额角与刉脉并行,因阴阳刉脉在此互相交叉,左边的筋与右部相联络,如果左角处的筋受伤,会引起右足不能活动,这种情况,叫做维筋相交。治疗用火针,用快速的手法,以病见效为针刺次数的限度,以病处的痛点为腧穴。这种病叫孟春痹。\\
足阳明经的筋,起于足次趾与中趾,结于足背,斜行于外侧上方,加于辅骨。上结于膝外侧,直上结于髀枢部,上沿胁肋,入内联属于脊;其直行的,从足背向上沿胫骨,结于膝部;分出的一支,结于外辅骨,合足少阳经的筋;其直行的,上沿大腿前肌肉隆起部,向上结于髀部,聚于阴器,再向上行而散布于腹部,到缺盆处集结,上颈部,挟口两旁,合于颧骨,在下的结于鼻,在上的合于太阳经的筋,太阳经的筋网维于上眼皮,阳明经的筋网维于下眼皮;分出一支,从颊部结于耳前。本经筋发生的病证,为足中趾牵引到胫部抽筋,脚部筋肉跳动而坚硬,大腿前方伏兔部抽筋,髀前部肿,厊疝,腹筋拘急,引及缺盆与颊部,突然口角歪斜,拘急的一方,眼不能闭合,如有热则筋弛纵,而眼不能开;颊部的筋有寒则拘急,牵引颊部使口角移动,有热则筋弛纵而不能收束,所以口角就会歪斜。治疗方法是采用马膏,贴在拘急的一侧,用白酒调肉桂末,涂在松弛的一侧,并用桑钩钩于口角,另用桑柴的炭火,置于小壶中,高低位置以病人坐着可得到暖气为准。一面用马膏熨于拘急一侧的颊部,同时喝一些酒,多吃一些熏肉之类的美味,不能喝酒的人,也要勉强喝一些,并在患处再三抚摩,这样就能愈病。其他的疾患,可用火针,取快速的手法,以病愈为针刺次数的限度,以病部的痛点为腧穴。这种病叫季春痹。\\
足太阴经的筋,起于足大趾的内侧端,向上结于内踝;直行的络于膝内侧辅骨,上沿大腿内侧结于髀部,聚会于阴器,然后上行至腹,结于脐部,再沿腹里,结于肋部,散于胸中;在内部深层的,附着于脊内。本经筋发生的病证,为足大趾牵引内踝作痛,转筋疼痛,膝内辅骨疼痛,大腿内侧引髀部作痛,阴器纽痛,由下向上牵引脐腹与两胁肋作痛,并牵引到胸部与脊内疼痛。治疗用火针,用快速的手法,以病见效为针刺次数的限度,以病部的痛点为腧穴。这种病叫孟秋痹。\\
足少阴经的筋,起于足小趾的下方,与足太阳经筋并行,斜走内踝骨下方,结于足后跟,与足太阳经筋相合,而上结于内辅骨之下,再与足太阴经筋并行,而向上沿大腿内侧,结于阴器,沿脊内,挟脊肉上行至项,结于脑后的枕骨,与足太阳经筋相合。本经筋发生的病证,为足下抽筋,以及其经过的部位与结聚处,都疼痛及抽筋。在本经筋的病证,主要有癫痫、拘挛证、痉证,在背部外侧的不能前俯,在胸腹内侧的不能后仰。所以阳病的腰向后反折不能前俯,阴病的不能后仰。治疗用火针,用快速的手法,以病见效为针刺次数的限度,以病部的痛点为腧穴,病在内的并可用药物熨贴患处,按摩导引,饮服汤药。如多次发作而剧烈的,是不治的死证。这种病叫仲秋痹。\\
足厥阴经的筋,起于足大趾之上,上行结于内踝骨之前,再向上沿胫骨,结于膝内辅骨之下,向上沿大腿内侧,结于阴器,联络其他各经筋。本经筋发生的病证,为足大趾牵引内踝骨前疼痛,膝内辅骨痛,大腿内侧疼痛抽筋,阳萎不用,如伤于房事过度,则阳萎不举,如伤于寒则阴器缩入,如伤于热则阴器弛纵挺长不收。治疗应疏通肾脏而清理本经的经气。对于转筋一类的病证,治疗用火针,用快速的手法,以病见效为针刺次数的限度,以病部痛点为腧穴。这种病叫季秋痹。\\
手太阳经的筋,起于手小指上,上结于腕部,上沿前臂内缘,结于肘部内侧锐骨后方,医生用手指弹之,则痠麻感可传导到小指尖,向上行,入内侧结于腋下;其分支,走腋窝后缘,上行绕于肩胛,沿颈部出走足太阳经筋的前方,结于耳后的完骨;另一分支,走入耳中;直行的分支,出耳上,再下行结于颔部,又上行联属于眼外角。本经筋发生的病证,为小指牵引肘内锐骨后缘疼痛,沿上臂内侧入腋下而见腋下疼痛,腋后缘疼痛,绕肩胛牵引颈部疼痛,并有耳鸣作痛,更牵及颔部疼痛,必须闭目很久才能睁眼看清东西,如果颈部的筋拘急,可能形成鼠瘰颈肿,颈部有寒热。治疗用火针,用快速的手法,以病见效为针刺次数的限度,以病部痛点为腧穴。如有肿大,当再用锐针刺治。这种病叫仲夏痹。\\
手少阳经的筋,起于手小指侧的无名指之端,结于腕部,向上沿前臂两骨之间,结于肘部,再绕至臑外缘,上行肩部,至颈部与手太阳经筋相合;其分支,从曲颊部深入连系舌根;又一分支,上行曲牙部,沿耳前,联属于眼外角,上乘额部结于额角。本经筋发生的病证,在它循行的部位上,见牵引抽筋,舌体卷缩。治疗用火针,用快速的手法,以病见效为针刺次数的限度,以病部痛点为腧穴。这种病叫季夏痹。\\
手阳明经的筋,起于拇指侧的食指之端,结于腕部,上沿前臂,上结于肘外,上行臑部,结于肩髃;其分支绕于肩胛部,挟脊两侧;直行的分支,从肩髃上行至颈;又一分支,上行颊部,结于颧骨部;直行的上出于手太阳经筋之前,上左额角,络于头部,下行到右侧颔部。本经筋发生的病证,在其循行的部位上,牵引疼痛抽筋,肩不能上举,颈部旋转不利,不能左右环视。治疗用火针,用快速的手法,以病见效为针刺次数的限度,以病部痛点为腧穴。这种病叫孟夏痹。\\
手太阴经的筋,起于手拇指之上,沿指上行,结于鱼际之后,循行寸口外侧,上沿臂,结于肘中,上臑部内侧,入腋下,出于缺盆,结于肩前方,再上结于缺盆,下结于胸里,散贯于胃之上口贲门部,再集合于贲门而下抵软肋部。本经筋发生的病证,是在其循行部位上牵引抽筋,痛甚会成为息贲证,胁肋拘急而吐血。治疗用火针,用快速的手法,以病见效为针刺次数的限度,以病部的痛点为腧穴。这种病叫仲冬痹。\\
手厥阴经的筋,起于手中指,与手太阴经筋并行,结于肘部内侧,上行上臂内侧,结于腋下,下行分散为前后而挟于胁肋;其分支入于腋部,散于胸中,结于贲门。本经筋发生的病证,是在其循行的部位,牵引抽筋,向前方连及胸部疼痛,成为息贲证。治疗用火针,用快速的手法,以病见效为针刺次数的限度,以病部的痛点为腧穴。这种病叫孟冬痹。\\
手少阴经的筋,起始于手小指内侧,结于掌后小指侧的锐骨;上行结于肘部内侧,再上行入腋下,与手太阴之筋交叉,挟行于乳内,结于胸中,沿贲部下系于脐部。本经筋发生的病证,在内的拘急时会承于心下而成伏梁证;在上肢的如罗网牵急肘部,在循行的部位上,都牵引抽筋疼痛。治疗用火针,用快速的手法,以病见效为针刺次数的限度,以病部的痛点为腧穴。如果已成伏梁证,见吐脓血,是不治的死证。大凡经筋的病,因寒的就曲折而拘挛,因热的就松弛而不收,阴痿而不举,背部的筋拘急就会向后反张,腹部的筋拘急就会向前俯屈而不能伸直。焠刺的方法是用于因寒而拘急的病证,如因热而筋弛缓不收的,不能用燔针。这种病叫季冬痹。\\
足阳明、手太阳经筋拘急,则为口眼歪斜,眼角拘急不能猝然视物。治疗都可采用以上各种方法。\\
骨度第十四\\
黄帝问于伯高曰:《脉度》言经脉之长短,何以立之?\\
伯高曰:先度其骨节之大小、广狭、长短,而脉度定矣。\\
黄帝曰:愿闻众人之度,人长七尺五寸者,其骨节之大小长短,各几何?\\
伯高曰:头之大骨围二尺六寸,胸围四尺五寸,腰围四尺二寸。发所覆者,颅至项尺二寸,发以下至颐长一尺。君子终折。\\
结喉以下至缺盆中长四寸,缺盆以下至巿旡长九寸,过则肺大,不满肺小。巿旡以下至天枢长八寸,过则胃大,不及则胃小。天枢以下至横骨长六寸半,过则回肠广长,不满则狭短。横骨长六寸半,横骨上廉以下至内辅之上廉长一尺八寸,内辅之上廉以下至下廉长三寸半,内辅下廉下至内踝长一尺三寸,内踝以下至地长三寸,膝腘以下至跗属长一尺六寸,跗属以下至地长三寸。故骨围大则太过,小则不及。\\
角以下至柱骨长一尺,行腋中不见者长四寸。腋以下至季胁长一尺二寸,季胁以下至髀枢长六寸,髀枢以下至膝中长一尺九寸,膝以下至外踝长一尺六寸,外踝以下至京骨长三寸,京骨以下至地长一寸。\\
耳后当完骨者广九寸,耳前当耳门者广一尺三寸,两颧之间相去七寸,两乳之间广九寸半,两髀之间广六寸半。足长一尺二寸,广四寸半。肩至肘长一尺七寸,肘至腕长一尺二寸半,腕至中指本节长四寸,本节至其末长四寸半。\\
项发以下至背骨长二寸半,膂骨以下至尾骶二十一节长三尺,上节长一寸四分分之一,奇分在下,故上七节至于膂骨,九寸八分分之七。此众人骨之度也,所以立经脉之长短也。是故视其经脉之在于身也,其见浮而坚,其见明而大者,多血,细而沉者,多气也。\\
黄帝问伯高说:《脉度》篇中说经脉的长短,是怎样确定的呢?\\
伯高说:应该先测量骨节的大小、宽窄、长短,从而就可以测定经脉的长度。\\
黄帝又问道:想听听普通人的骨度,成人身长以七尺五寸长计算,其骨节的大小、长短各是多少呢?\\
伯高说:头颅大骨周围二尺六寸,胸围四尺五寸,腰围四尺二寸。头发所覆盖的部位,颅至项为一尺二寸,前发际以下至颐长一尺,后发际至颐共二尺二寸。明达的君子还要参校计算。\\
喉结以下至缺盆中央长四寸,缺盆以下至胸骨剑突长九寸,超过九寸的是肺大,不满九寸的是肺小。胸骨剑突以下至天枢长八寸,超过八寸的是胃大,不满八寸的是胃小。天枢向下至耻骨长六寸半,超过六寸半的是回肠宽而长,不满六寸半的是回肠狭而短。耻骨横长为六寸半,横骨的上缘向下至膝内辅骨的上缘长一尺八寸,内辅骨上缘向下至内辅骨下缘长三寸半,内辅骨下缘向下至内踝骨尖长一尺三寸,内踝骨尖至足底长三寸,膝腘窝向下至足跗两踝之周围所属长一尺六寸,跗属向下至足底长三寸。以上这些骨的尺寸数字,头骨围粗大的会超过,头骨围细小的会不及。\\
两侧头角向下至柱骨长一尺,肩骨至腋中尽处长四寸,腋部向下至软肋长一尺二寸,软肋向下至髀枢长六寸,髀枢向下至膝盖中央长一尺九寸,膝向下至外踝骨尖长一尺六寸,外踝骨尖向下至小趾侧后的京骨长三寸,京骨向下至足底长一寸。\\
耳后当完骨部之间宽九寸,耳前当两耳门之间宽一尺三寸,两颧骨之间宽七寸,两乳之间宽九寸半,两髀之间宽六寸半。足长一尺二寸,宽四寸半。肩峰至肘关节长一尺七寸,肘至腕关节长一尺二寸半,腕至中指本节长四寸,中指本节至中指端长四寸半。\\
项后发际向下至背骨第一节的大椎处长二寸半,大椎骨向下至尾骶骨共二十一节长三尺,上面的七节每节长一寸四分一厘,零数在下,所以上七节共长九寸八分七厘。以上所述是普通人骨的长度,根据这个标准,然后来确定经脉的长短。所以说经脉在人体中,其浮于表面,坚实明显而粗大的多血,细小而隐于内的多气。\\
五十营第十五\\
黄帝曰:余愿闻五十营,奈何?\\
岐伯答曰:天周二十八宿,宿三十六分,人气行一周,千八分。日行二十八宿,人经脉上下、左右、前后二十八脉,周身十六丈二尺,以应二十八宿。\\
漏水下百刻,以分昼夜。故人一呼,脉再动,气行三寸;一吸,脉亦再动,气行三寸。呼吸定息,气行六寸;十息,气行六尺,日行二分;二百七十息,气行十六丈二尺,气行交通于中,一周于身,下水二刻,日行二十五分。五百四十息,气行再周于身,下水四刻,日行四十分。二千七百息,气行十周于身,下水二十刻,日行五宿二十分;一万三千五百息,气行五十营于身,水下百刻,日行二十八宿,漏水皆尽,脉终矣。所谓交通者,并行一数也。故五十营备,得尽天地之寿矣,凡行八百一十丈也。\\
黄帝说:我想听听五十营是如何计算的?\\
岐伯回答说:天空一周有恒星二十八宿,每宿距离三十六分,一昼夜运行五十周,共计一千零八分。在一昼夜中日行周历了二十八宿,人体的经脉分布在上下、左右、前后,共二十八脉,脉气在全身运转一周共十六丈二尺,恰好相应于二十八宿。\\
铜壶滴漏以一百刻计算,来分白天和黑夜。人一呼,脉搏动两次,营气在脉中运行三寸;一吸,脉也搏动两次,营气也运行三寸。一呼一吸,称为“息”,营气运行六寸;十息,营气行六尺,日行二分;二百七十息,营气运行十六丈二尺,气行交通于中,脉气行遍周身,此时漏水降下二刻,日在星宿之间移行二十五分。人呼吸五百四十息时,营气就再运行全身一周,此时漏水降下四刻,日在星宿之间移行四十分有零。人呼吸二千七百息时,营气已周行于全身十次,此时漏水降下二十刻,日在星宿之间移行五宿又二十分有零;人呼吸一万三千五百息的时间,脉气就营运周身五十次,此时漏水降下百刻,日遍行二十八宿,漏水已尽,而人体的经脉之气也运行周遍了。所谓“交通”,是营气并二十八脉通行一周之数。因此,每日营气运行五十周次,不失其常,共计八百一十丈,则能保持健康,尽其天年。\\
营气第十六\\
黄帝曰:营气之道,内谷为宝。谷入于胃,乃传之肺,流溢于中,布散于外。精专者行于经隧,常营无已,终而复始,是谓天地之纪。故气从太阴出注手阳明,上行注足阳明,下行至跗上,注大指间,与太阴合,上行抵髀。从脾注心中,循手少阴,出腋下臂,注小指,合手太阳,上行乘腋出\\
内,注目内眦,上巅下项,合足太阳,循脊下尻,下行注小指之端,循足心注足少阴,上行注肾,从肾注心,外散于胸中;循心主脉,出腋下臂,出两筋之间,入掌中,出中指之端,还注小指次指之端,合手少阳;上行注膻中,散于三焦,从三焦注胆,出胁注足少阳,下行至跗上,复出跗,注大指间,合足厥阴,上行至肝,从肝上注肺,上循喉咙,入颃颡之窍,究于畜门。其支别者,上额循巅下项中,循脊入骶,是督脉也。络阴器,上过毛中,入脐中,上循腹里,入缺盆,下注肺中,复出太阴。此营气之所行也,逆顺之常也。\\
黄帝说:营气运行周身,以受纳饮食谷物为最可贵。水谷入胃,它化生的精微,传到肺脏,流溢于内营养脏腑,布散于外滋养形体。其精华部分流行于经脉之中,常常营运而不休止,终而复始,这是自然的规律。营气首先从手太阴肺经出发,流注于手阳明大肠经,上行流注于足阳明胃经,下行到足背,流注足大指间,与足太阴脾经相合;上行抵达脾经,从脾的支脉,上注于心中,由此沿着手少阴心经,出腋窝,下沿臂内侧后缘,流注到手小指之端,与手太阳小肠经相合,由此上行越过腋外,出于眼眶下的内侧,流注到眼内角,然后再上至巅顶,下行于颈项,与足太阳膀胱经相合,又沿脊柱向下经尻部,下行流注于足小趾之端,再沿着足心,流注到足少阴肾经,循经上行而注入肾脏,从肾注于心包络,外散于胸中;再沿心包络脉,出腋窝,下行前臂,入两筋的中间,入掌中,直出手中指之端,再转回来流注到无名指的尖端,与手少阳三焦经相合;由此上行注于膻中,散布于上中下三焦,再从三焦流注到胆腑,出胁部,注于足少阳胆经,下行到足背,又从足背流注到足大指间,与足厥阴肝经相合,循肝经上行至肝脏,再从肝脏上注于肺中,向上沿喉咙后面,入上额之窍,深入于鼻内通脑之处。其分支别行的,上行额部,沿头顶中央,下行项中,沿脊柱,入骶骨部,这是督脉。由此再通过任脉,络绕阴器,上过毛际,入于脐中,向上沿腹内,入缺盆,复向下流注到肺中,又从手太阴肺经开始循环周流。这就是营气运行的径路,无论上行下行,都循此常道而不变。\\
脉度第十七\\
黄帝曰:愿闻脉度。\\
岐伯答曰:手之六阳,从手至头,长五尺,五六三丈。手之六阴,从手至胸中,三尺五寸,三六一丈八尺,五六三尺,合二丈一尺。足之六阳,从足上至头,八尺,六八四丈八尺。足之六阴,从足至胸中,六尺五寸,六六三丈六尺,五六三尺,合三丈九尺。\\
脉从足至目,七尺五寸,二七一丈四尺,二五一尺,合一丈五尺。督脉任脉各四尺五寸,二四八尺,二五一尺,合九尺。凡都合一十六丈二尺,此气之大经隧也。经脉为里,支而横者为络,络之别者为孙。盛而血者,疾诛之。盛者泻之,虚者饮药以补之。\\
五脏常内阅于上七窍也。故肺气通于鼻,肺和,则鼻能知臭香矣。心气通于舌,心和,则舌能知五味矣。肝气通于目,肝和,则目能辨五色矣。脾气通于口,脾和,则口能知五谷矣。肾气通于耳,肾和,则耳能闻五音矣。五脏不和,则七窍不通;六腑不和,则留为痈。故邪在腑,则阳脉不和,阳脉不和,则气留之,气留之,则阳气盛矣。阳气太盛,则阴脉不利,阴脉不利,则血留之,血留之,则阴气盛矣。阴气太盛,则阳气不能荣也,故曰关;阳气太盛,则阴气弗能荣也,故曰格;阴阳俱盛,不得相荣,故曰关格。关格者,不得尽期而死也。\\
黄帝曰:\\
脉安起安止?何气荣也?\\
岐伯答曰:\\
脉者,少阴之别,起于然骨之后,上内踝之上,直上循阴股入阴,上循胸里入缺盆,上出人迎之前,入圴,属目内眦,合于太阳、阳\\
而上行,气并相还,则为濡目,气不荣则目不合。\\
黄帝曰:气独行五脏,不荣六腑,何也?\\
岐伯答曰:气之不得无行也,如水之流行不休。故阴脉荣其脏,阳脉荣其腑,如环之无端,莫知其纪,终而复始。其流溢之气,内溉脏腑,外濡腠理。\\
黄帝曰:\\
脉有阴阳,何脉当其数?\\
岐伯答曰:男子数其阳,女子数其阴,当数者为经,其不当数者为络也。\\
黄帝说:我希望听听经脉的长度。\\
岐伯回答说:手的六阳经脉,从手至头部,每脉长五尺,五六共三丈。手的六阴经脉,从手至胸中,每脉长三尺五寸,三六一丈八尺,五六三尺,合计二丈一尺。足的六阳经脉,从足上行至头部,每脉长八尺,六八共四丈八尺。足的六阴经脉,从足至胸中,每脉长六尺五寸,六六三丈六尺,五六三尺,合计三丈九尺。刉脉从足至眼部,每脉长七尺五寸,二七一丈四尺,二五一尺,左右合计一丈五尺。督脉和任脉,各长四尺五寸,二四八尺,二五一尺,合计九尺。以上二十八脉,共长十六丈二尺,这是营气运行的大经脉的情况。经脉循行体内,由经脉分支而横向循行的是络脉,由络脉再分出的是孙络。如果气盛而有淤血,应速行针刺出血。总之,邪气盛的用泻法,正气虚的饮汤药来补益。\\
五脏的精气,经常从体内而上通于七窍。肺气通于鼻,肺气调和,则鼻能辨别香臭。心气通于舌,心气调和,则舌能辨别五味。肝气通于目,肝气调和,则目能辨别五色。脾气通于口,脾气调和,饮食就有滋味。肾气通于耳,肾气调和,则耳能听五音。如果五脏不调和,则七窍就不畅通。六腑不调和,则气滞血淤而外生痈疡。因此,邪气滞留六腑,则阳脉不和,阳脉不和,则气稽留,气稽留,则阳气偏盛。邪气滞留五脏,则阴脉不和,阴脉不和则血稽留,血稽留,则阴气偏盛。阴气太盛,则阳气不能营运于内,所以称为“关”;阳气太盛,则阴气不能营运于外,所以称为“格”;阴阳之气都盛,不能互相营运,则叫做“关格”。出现“关格”,就不能寿终正寝了。\\
黄帝问:刉脉从哪里起始,到哪里终止?是借助哪条经脉之气而运行的呢?\\
岐伯回答说:刉脉是从足少阴肾经分别而出,起于内踝前的然骨之后,向上经过内踝上部,直上沿股内侧进入前阴,上沿胸腹内部,进入缺盆,再上行至人迎之前,入噂骨部,至眼内角,与足太阳经脉、阳刉脉会合而上行,三经之气合并,还而下行,濡养两眼,如果阴刉脉气不能上营,眼睛就不能闭合。\\
黄帝问:阴刉经脉之气独运行于五脏而不营养六腑,为什么呢?\\
岐伯回答说:气的运行不能停止,就像水的流动不息一样。所以阴刉脉营养五脏,阳刉脉营养六腑。它们的运行如圆环一样没有开端,无法知道它的开始,只是终而复始的循环着。它们输送流溢的精气,在内灌溉脏腑,在外濡养腠理。\\
黄帝问:刉脉有阴刉、阳刉的不同,究竟依据哪一条脉来计算呢?\\
岐伯回答说:男子以阳刉计算,女子以阴刉计算,凡作为计数的就是经脉,不作为计数的就是络脉。\\
营卫生会第十八\\
黄帝问于岐伯曰:人焉受气?阴阳焉会?何气为营?何气为卫?营安从生?卫于焉会?老壮不同气,阴阳异位,愿闻其会。\\
岐伯答曰:人受气于谷。谷入于胃,以传于肺,五脏六腑,皆以受气。其清者为营,浊者为卫。营在脉中,卫在脉外。营周不休,五十而复大会。阴阳相贯,如环无端。卫气行于阴二十五度,行于阳二十五度,分为昼夜。故气至阳而起,至阴而止。故曰:日中而阳陇为重阳,夜半而阴陇为重阴。故太阴主内,太阳主外。各行二十五度,分为昼夜。夜半为阴陇,夜半后而为阴衰,平旦阴尽,而阳受气矣。日中为阳陇,日西而阳衰。日入阳尽,而阴受气矣。夜半而大会,万民皆卧,命曰合阴。平旦阴尽而阳受气。如是无已,与天地同纪。\\
黄帝曰:老人之不夜瞑者,何气使然?少壮之人不昼瞑者,何气使然?\\
岐伯答曰:壮者之气血盛,其肌肉滑,气道通,营卫之行,不失其常,故昼精而夜瞑。老者之气血衰,其肌肉枯,气道涩,五脏之气相搏,其营气衰少而卫气内伐,故昼不精,夜不瞑。\\
黄帝曰:愿闻营卫之所行,皆何道从来?\\
岐伯答曰:营出于中焦,卫出于下焦。\\
黄帝曰:愿闻三焦之所出。\\
岐伯答曰:上焦出于胃上口,并咽以上,贯膈而布胸中,走腋,循太阴之分而行,还至阳明,上至舌,下足阳明。常与营俱行于阳二十五度,行于阴亦二十五度,一周也。故五十度而复大会于手太阴矣。\\
黄帝曰:人有热,饮食下胃,其气未定,汗则出,或出于面,或出于背,或出于身半,其不循卫气之道而出,何也?\\
岐伯曰:此外伤于风,内开腠理,毛蒸理泄,卫气走之,固不得循其道。此气慓悍滑疾,见开而出,故不得从其道,故命曰漏泄。\\
黄帝曰:愿闻中焦之所出。\\
岐伯答曰:中焦亦并胃中,出上焦之后。此所受气者,泌糟粕,蒸津液,化其精微,上注于肺脉,乃化而为血。以奉生身,莫贵于此。故独得行于经隧,命曰营气。\\
黄帝曰:夫血之与气,异名同类,何谓也?\\
岐伯答曰:营卫者,精气也;血者,神气也。故血之与气,异名同类焉。故夺血者无汗,夺汗者无血。故人生有两死,而无两生。\\
黄帝曰:愿闻下焦之所出。\\
岐伯答曰:下焦者,别回肠,注于膀胱,而渗入焉。故水谷者,常并居于胃中,成糟粕而俱下于大肠,而成下焦。渗而俱下,济泌别汁,循下焦而渗入膀胱焉。\\
黄帝曰:人饮酒,酒亦入胃,谷未熟而小便独先下,何也?\\
岐伯答曰:酒者,熟谷之液也,其气悍以清,故后谷而入,先谷而出焉。\\
黄帝曰:善。余闻上焦如雾,中焦如沤,下焦如渎,此之谓也。\\
黄帝问岐伯说:人的精气来自哪里?阴和阳在哪里会合?什么叫做营气?什么叫做卫气?营卫之气是从哪里产生的?卫营之气在哪里会合?老年人和壮年人气的盛衰不同,昼夜气行的位置各异,我希望听听会合的道理。\\
岐伯回答说:人的精气,来源于饮食物。当饮食入胃,它的精微就传给了肺脏,五脏六腑都因此接受了营养。其中清的称为营气,浊的称为卫气。营气运行于脉中,卫气运行于脉外。在周身运行不休,营卫各运行五十周次又会合。阴阳相互贯通,如环周一样没有开头。卫气行于阴分二十五周次,又行于阳分二十五周次,昼夜各半。所以卫气的循行,从属阳的头部起始,到手足阴经为止。所以说:卫气行于阳经,中午阳气最盛,称为重阳;夜半行于阴经,阴气最盛,称为重阴,太阴主管人体内部,太阳主管人体外部,营卫在其中各运行二十五周次,都是以昼夜来划分的。半夜是阴气最盛的时候,夜半以后阴气渐衰,黎明阴气衰退而阳气继起。中午阳气最盛,日落而阳气衰退。当日入黄昏,阳气已尽而阴气继起。到夜半,营卫之气始相会合,这时人们都入睡,这叫合阴。到黎明阴气衰尽,而阳气又继起了。如此循行不止,和自然界日月运行的道理一致。\\
黄帝问:老人往往夜里入睡困难,是什么气使他这样呢?青壮人白天往往不睡觉,是什么气使他这样呢?\\
岐伯回答说:壮年人的气血充盛,肌肉滑润,气道通畅,营气卫气的运行不失常规,所以白天神气清爽,夜里睡得香。老人的气血衰退,肌肉消瘦,气道涩滞,五脏之气损耗,营气衰少,卫气内乏,所以白天神不清爽,夜里也不易入睡。\\
黄帝问:我希望听到营、卫二气的运行,都是从哪里发出来的?\\
岐伯回答说:营气发于中焦,卫气发于上焦。\\
黄帝说:希望听一下发于上焦的情况。\\
岐伯回答说:上焦之气从胃上口发出,并食道上行,穿过膈膜,散布胸中,横走腋下,沿手太阴肺经范围下行,返回到手阳明大肠经,上行至舌,又下流注于足阳明胃经,卫气与营气一样都是运行于阳分二十五周,运行于阴分二十五周,这就是昼夜一周的大循环。所以卫气五十周次行遍全身,再与营气大会于手太阴肺经。\\
黄帝说:人在有热时,就会饮食刚入胃,其精微之气还未化成,汗就先出来了。或出于面,或出于背,或出于半身,并不沿着卫气运行的道路而出,是什么道理呢?\\
岐伯说:这是为风邪所伤,以致腠理舒张,皮毛为风热所蒸,腠理开泄,卫气行至肌表疏松的地方,就不沿着它的流行道路走了。卫气的性质慓悍滑利,见到开泄的地方就走,所以不能从它正常运行之道而出,这叫漏泄。\\
黄帝说:希望听到中焦的出处?\\
岐伯回答说:中焦的部位与胃并列,在上焦之后。这里主化生水谷之味,泌去糟粕,蒸腾津液,化生精微,向上传注于肺脉,再化生而为血液。用它奉养周身,没有比它更宝贵的了。所以独能行于经脉之内,叫做营。\\
黄帝说:血和气,名称虽不一样,而其实却是同类,这是为什么?\\
岐伯回答说:营和卫都是水谷精气化成;血是精气化生的最宝贵的物质,称为“神气”。因此血和气,名虽不同,却属于同类。凡失血过多的人,其汗也少;出汗过多的人,其血亦少。所以说人体夺血或夺汗均可死亡,而血与汗缺一则不能生存。\\
黄帝说:我希望听到下焦的出处。\\
岐伯回答说:下焦可另将糟粕输送到回肠,又将水液渗透注入膀胱。所以水谷一类,常并存在胃中,经过消化,形成了糟粕,向下输送到大肠,成为下焦的主要功能。至于水液,也都是向下渗灌,排去其水,保留清液,其中浊秽部分,就沿着下焦而渗入膀胱。\\
黄帝说:人喝酒,酒入胃中,谷物还未腐熟,而酒液先从小便排泄,这是什么缘故?\\
岐伯回答说:酒是谷类发酵而酿成的液体,其气慓悍清纯,所以比食物后入,反比食物先从小便排出。\\
黄帝说:很对。我听说,三焦的功能,上焦像雾一样,中焦像沤物池一样,下焦像水沟一样,就是这样。\\
四时气第十九\\
黄帝问于岐伯曰:夫四时之气,各不同形。百病之起,皆有所生。灸刺之道,何者为定?\\
岐伯答曰:四时之气,各有所在,灸刺之道,得气穴为定。故春取经、血脉、分肉之间,甚者深刺之,间者浅刺之。夏取盛经孙络,取分间,绝皮肤;秋取经腧,邪在腑,取之合。冬取井荥,必深以留之。\\
温疟,汗不出,为五十九痏。风仴肤胀,为五十七痏。取皮肤之血者,尽取之。飨泄,补三阴之上,补阴陵泉,皆久留之,热行乃止。转筋于阳,治其阳;转筋于阴,治其阴,皆卒刺之。\\
徒仴,先取环谷下三寸,以铍针针之,已刺而筩之,而内之,入而复之,以尽其仴,必坚。来缓则烦悗,来急则安静。间日一刺之,仴尽乃止。饮闭药,方刺之时,徒饮之。方饮无食,方食无饮,无食他食,百三十五日。\\
著痹不去,久寒不已,卒取其三里。肠中不便,取三里,盛泻之,虚补之。疠风者,素刺其肿上,已刺,以锐针针其处,按出其恶气,肿尽乃止。常食方食,无食他食。\\
腹中常鸣,气上冲胸,不能久立,邪在大肠,刺肓之原、巨虚上廉、三里。小腹控睾,引腰脊,上冲心,邪在小肠者,连睾系,属于脊,贯肝肺,络心系。气盛则厥逆,上冲肠胃,熏肝,散于肓,结于脐。故取之肓原以散之,刺太阴以予之,取厥阴以下之,取巨虚下廉以去之,按其所过之经以调之。\\
善呕,呕有苦,长太息,心中憺憺,恐人将捕之,邪在胆,逆在胃,胆液泄则口苦,胃气逆则呕苦,故曰呕胆。取三里以下胃气逆,刺少阳血络以闭胆逆,却调其虚实,以去其邪。饮食不下,膈塞不通,邪在胃脘。在上脘则刺抑而下之,在下脘则散而去之。\\
小腹痛肿,不得小便,邪在三焦约,取之太阳大络,视其络脉与厥阴小络结而血者,肿上及胃脘,取三里。\\
睹其色,察其目,知其散复者,视其目色,以知病之存亡也。一其形,听其动静者,持气口人迎,以视其脉。坚且盛且滑者,病日进;脉软者,病将下;诸经实者,病三日已。气口候阴,人迎候阳也。\\
黄帝问岐伯说:四时的气候变化,性质各不相同,百病的起始,都受气候影响而发生。针灸治疗的原则,怎样来决定呢?\\
岐伯回答说:四时的邪气侵入人体,各有一定的部位,针灸治疗的原则,以掌握四时气候与腧穴的关系而定。所以春季可取大经、血脉、分肉的穴位,病重的可用深刺法,病轻的可用刺浅法。夏季可取气盛的六阳经脉或孙络的穴位,或刺分肉之间,以及透过皮肤的浅刺法;秋季可取“经穴”及“输穴”,如邪在六腑,可取“合穴”。冬季可取“井穴”及“荥穴”,一定要深刺而且留针。\\
患温疟不出汗的,治疗有热病五十九个腧穴。患风水皮肤浮肿,治疗有水肿病五十七个腧穴。若皮下有血络,都应当刺出其血。患飧泄证,可取三阴交、阴陵泉,用补法,并且都要长期间留针,必等到患者觉得针下有热感为止。四肢外侧转筋,应治阳经;四肢内侧转筋,应治阴经;都可用火针焠刺。\\
单纯的水肿病,先取环跳之下三寸的风市穴,用铍针刺,针后在针孔处插入竹筒,吸收水液,反复进行,将水放尽,必使肌肉恢复坚实。在针刺时,必急刺,刺的慢病人会感到烦闷,刺得快则安静。隔一天刺一次,直到水肿退尽为止。可以内服开闭利水的药,在初刺时就可饮服。但要注意刚饮药不能进食,刚进食不能饮药,除了正常的饮食以外,禁食其他食物一百三十五天。\\
患著痹经久不愈,常觉寒冷不解,可用火针刺三里穴。大小肠功能失常,都可取足三里穴,实证用泻法,虚证用补法。麻风病可针刺肿起的部位,刺后再用锐利的针刺其患处,用手按压,出其恶气,肿消尽停止。常吃普通的食物,不可吃其他对病不利的食物。\\
腹内时常鸣响,气上逆而冲胸部,不能久站,这是病邪在大肠,可刺气海、上巨虚、足三里穴。小腹部控引睾丸,连及腰脊作痛,向上冲及心胸,是邪在小肠,小肠连及睾丸的系脉,附属于脊椎,上贯于肝肺,绕络于心系。因此,邪气盛的则厥气上逆,上逆冲及肠胃,熏灼肝脏,散于肓膜,聚结于脐部。所以取气海穴以散其结气,再刺手太阴肺经的腧穴来补虚,取足厥阴肝经的腧穴来泻实,取小肠经的合穴下巨虚以去其邪气,按出现症状的经脉进行调治。\\
病人时常呕吐,呕出苦水,长叹气,心中恐惧,害怕有人来抓他,这是病邪在胆,而邪气横逆于胃,胆汁外泄则口苦,胃气上逆则呕苦,所以叫“呕胆”。治疗应取足三里穴,以降下胃气之逆,刺足少阳胆经的血络,以止上逆之胆气,再根据虚实属性,以去除病邪。饮食不下,膈间闭塞不通,这是邪在胃脘。如果上脘不通,可用针刺降其上逆之气,如果下脘不通,可用针刺疏散其病邪。\\
小腹肿痛,不能小便,这是邪在三焦,约束而不行,可取太阳经的大络委阳,看它的络脉与厥阴经的小络交结而有淤血之处,若是肿胀上至胃脘部,并取足三里穴。\\
所谓的“睹其色,察其目,知其散复者”,就是察看病人眼睛的五色,来判断病邪的存留与消失的情况。所谓的“一其形,听其动静者”,是说持诊气口与人迎之脉,察看他的脉象。如脉象强盛而滑利的,病情会日渐加重;脉象虚软的,病邪将要减退;各经脉气尚充实的,三天后病就会好。这就是所谓气口是候阴分的,人迎是候阳分的。\\
卷五\\
五邪第二十\\
邪在肺,则病皮肤痛,寒热,上气喘,汗出,咳动肩背。取之膺中外腧,背三节五脏之傍。以手疾按之,快然,乃刺之;取之缺盆中,以越之。\\
邪在肝,则两胁中痛,寒中,恶血在内,行善掣节,时脚肿。取之行间,以引胁下;补三里,以温胃中;取血脉,以散恶血;取耳间青脉,以去其掣。\\
邪在脾胃,则病肌肉痛。阳气有余,阴气不足,则热中善饥;阳气不足,阴气有余,则寒中肠鸣腹痛;阴阳俱有余,若俱不足,则有寒有热。皆调于三里。\\
邪在肾,则病骨痛,阴痹。阴痹者,按之而不得,腹胀腰痛,大便难,肩背颈项痛,时眩。取之涌泉、昆仑,视有血者,尽取之。\\
邪在心,则病心痛,喜悲,时眩仆。视有余不足而调之其输也。\\
病邪在肺脏,会发生皮肤疼痛,恶寒发热,气上逆而喘,汗出,咳嗽牵引肩背疼痛。治疗取侧胸上部的中府、云门穴,及背部第三椎骨旁的肺腧穴。进针时,先用手速按其处,病者觉得舒适,就在该处进针;同时可取缺盆穴,使肺中邪气向上越出。\\
病邪在肝脏,会发生两胁疼痛,寒气留中,恶血淤留在内,走路时经常关节牵引作痛,并且时有脚肿症状。治疗取行间穴,以引胁肋间的郁结之气下行;并补足三里穴以温其胃中;对淤血的络脉,针刺散其恶血;再取耳轮后青络上的瘛脉穴,以除去掣痛。\\
病邪在脾胃,会发生肌肉疼痛。如果阳气有余,阴气不足,则热邪在中而易饥;阳气不足,阴气有余,则寒邪在中而肠鸣、腹痛;若阴阳均有余或均不足,则有寒有热。这些病证,都可取三里穴来治疗。\\
病邪在肾脏,会发生骨痛、阴痹。所谓阴痹,在形体表面触按不到,证见腹胀、腰痛,大便难,肩、背、颈、项疼痛,时常目眩。治疗取涌泉、昆仑穴,凡有淤血的,尽刺出其血。\\
病邪在心脏,会发生心痛,易悲伤,时时目眩跌仆。治疗时先要分析其虚实属性,而后调治本经的输穴。\\
寒热病第二十一\\
皮寒热者,不可附席,毛发焦,鼻槁腊,不得汗。取三阳之络,以补手太阴。\\
肌寒热者,肌痛,毛发焦而唇槁腊,不得汗。取三阳于下,以去其血者,补足太阴以出其汗。\\
骨寒热者,病无所安,汗注不休。齿未槁,取其少阴于阴股之络;齿已槁,死不治。骨厥亦然。\\
骨痹,举节不用而痛,汗注烦心。取三阴之经,补之。\\
身有所伤,血出多,及中风寒,若有所堕坠,四支懈惰不收,名曰体惰。取其小腹脐下三结交。三结交者,阳明,太阴也,脐下三寸,关元也。\\
厥痹者,厥气上及腹。取阴阳之络,视主病也。泻阳补阴经也。\\
颈侧之动脉人迎,人迎,足阳明也,在婴筋之前。婴筋之后,手阳明也,名曰扶突。次脉,足少阳脉也,名曰天牖。次脉,足太阳也,名曰天柱。腋下动脉,臂太阴也,名曰天府。\\
阳迎头痛,胸满不得息,取之人迎。暴瘖气鞕,取扶突与舌本出血。暴聋气蒙,耳目不明,取天牖。暴挛痫眩,足不任身,取天柱。暴瘅内逆,肝肺相搏,血溢鼻口,取天府。此为天牖五部。\\
臂阳明有入圴遍齿者,名曰大迎,下齿龋取之。臂恶寒补之,不恶寒泻之。足太阳,有入圴遍齿者,名曰角孙,上齿龋取之,在鼻与圴前。方病之时,其脉盛,盛则泻之,虚则补之。一曰取之出鼻外。\\
足阳明有挟鼻入于面者,名曰悬颅,属口,对入系目本,视有过者取之。损有余,益不足,反者益甚。足太阳有通项入于脑者,正属目本,名曰眼系。头目苦痛取之,在项中两筋间,入脑乃别。阴\\
阳\\
,阴阳相交,阳入阴,阴出阳,交于目锐眦。阳气盛则瞋目,阴气盛则瞑目。\\
热厥取足太阴、少阳,皆留之。寒厥取足阳明、少阴于足,皆留之。\\
舌纵涎下,烦悗,取足少阴。振寒洒洒,鼓颔,不得汗出,腹胀烦悗,取手太阴。刺虚者,刺其去也;刺实者,刺其来也。\\
春取络脉,夏取分腠,秋取气口,冬取经输。凡此四时,各以时为齐。络脉治皮肤,分腠治肌肉,气口治筋脉,经输治骨髓、五脏。\\
身有五部:伏兔一;腓二,腓者,腨也;背三;五脏之腧四;项五。此五部有痈疽者,死。\\
病始手臂者,先取手阳明,太阴而汗出。病始头首者,先取项太阳而汗出,病始足胫者,先取足阳明而汗出。臂太阴可汗出,足阳明可汗出。故取阴而汗出甚者,止之于阳;取阳而汗出甚者,止之于阴。\\
凡刺之害:中而不去则精泄,不中而去则致气。精泄则病甚而恇,致气则生为痈疽也。\\
邪在皮肤而发生寒热病,不能着席而卧,毛发憔悴,鼻子干枯,不得出汗。治疗可取足太阳膀胱经的络穴飞扬,并补手太阴肺经的穴位。\\
邪在肌肉而发生寒热病,肌肉疼痛,毛发憔悴,口唇干枯,不得出汗。治疗可取足太阳膀胱经在下肢的络穴飞扬,以祛除淤血,并补足太阴脾经的穴位,以出其汗。\\
邪在骨而发生寒热病,病人烦躁不安,汗出如流。如牙齿尚未枯槁,可取足少阴经的络穴大钟;如牙齿已经枯槁的,是不治的死证。对骨厥的诊治,也是如此。\\
骨痹证,周身关节活动不便而疼痛,汗出如流,心烦。治疗可取三阴经的穴位,用补法。\\
因外伤出血过多,又受了风寒,或从高处坠伤,以致四肢怠惰,不能运动,名叫“体惰”。治疗取小腹脐下的三结交。所谓“三结交”,是足阳明、足太阴与任脉三经交结之处,在脐下三寸,名叫关元。\\
厥痹证,有厥气上及于腹部,治疗可取阴经或阳经的络穴,但必须诊察以何经之病为主。总的原则是泻阳经,补阴经。\\
颈间结喉两侧的动脉处的腧穴,名叫人迎,属于足阳明经,在“婴筋”的前方。在“婴筋”后方的是手阳明经脉的腧穴,名叫扶突。向后次一行的经脉是足少阳的腧穴,名叫天牖。向后再次一行的经脉是足太阳的腧穴,名叫天柱。腋窝下方的动脉是手太阴经脉,其腧穴,名叫天府。\\
如阳邪上逆头痛,胸中满闷,呼吸不利,可取人迎穴。如突然音哑,舌强,可取扶突穴,并刺舌根出血。如突然耳聋,经气蒙蔽不通,耳不聪,目不明,可取天牖穴。如突然拘挛癫痫,头目眩晕,两足站立不稳,可取天柱穴。如突然患消瘅,内脏气机逆乱,肝肺两经邪火相争,血往上溢,口鼻出血,可取天府穴。这是天牖五部穴位的所在及其主治的病证。\\
手阳明经脉,有走入颧骨下,遍络于齿龈的,其腧穴,名叫大迎,下齿龋痛时,可以取大迎穴治疗。臂部恶寒的用补法,臂部不恶寒的用泻法。足太阳经脉,也有走入颧骨,遍络齿龈的,其腧穴,名叫角孙,上齿龋痛时,可以取足太阳经在鼻与颧骨前的穴位治疗。初病的时候,如脉气盛,盛的可用泻法,如脉虚的可用补法。另一说,上齿痛可取鼻外侧的禾髎、迎香等穴治疗。\\
足阳明经脉,有夹行于鼻两侧而走面部的,其腧穴名叫悬颅,这条脉下行的属口,上行的对着口角而走入眼睛深部,根据病情,可取悬颅穴。实泻虚补,如果治法相反,就会加重疾病。足太阳经脉有通于项后而走入脑部的,联系到眼睛深部,叫做目系。如见头目疼痛,可以取项中两筋之间的玉枕穴,此脉入脑后才分道而行。阳刉和阴刉是阴阳相交的,阳入于阴,阴出于阳,相交于目内眦。如阳气偏盛眼睛就瞪大,阴气偏盛眼睛就常闭。\\
热厥证,取足太阴经与足少阳经的腧穴,都留针。寒厥证,取足阳明经与足少阴经的腧穴,都留针。\\
舌纵缓不收,口涎自下,心中烦闷,取足少阴经的腧穴。洒洒恶寒,甚至两颔颤抖,汗不得出,腹胀,烦闷,取手太阴经的腧穴。总的原则是刺虚证用补法,应顺着脉气去的方向转针;刺实证用泻法,应迎着脉气来的方向转针。\\
在春季针刺时,可取络脉间的穴位;夏季针刺时,可取肌肉与皮肤间的穴位;秋季针刺时,可取气口部的穴位;冬季针刺时,可取经穴。大凡四季刺法,各有取穴范围。取络脉可治皮肤病,取肌肉可治肌肉病,取气口可治筋脉病,取经脉可治骨髓和五脏病。\\
体表的重要部位有五处:一是大腿前方的伏兔部;二是腓,腓是小腿肚部;三是背部中行的督脉部;四是背部的五脏腧部;五是项间的督脉经部。这五个部位如发生痈疽,多死亡。\\
疾病起于手部臂部的,取手阳明经与手太阴经的穴位使其出汗。疾病起于头部的,取项间足太阳经的穴位使其出汗;疾病起于足部胫部的,取足阳明经的穴位使其出汗。刺手太阴经可以发汗,刺足阳明经也可以发汗。如果取阴经的穴位而汗出过多时,可取阳经的穴位来止汗;取阳经的穴位而汗出过多时,可取阴经的穴位来止汗。\\
大凡误用针刺法的害处:当刺已中病而不出针,则易使精气外泄;尚未刺中病而即出针,则可使邪气内留。精气外泄则会使病加重而患者更衰弱,如邪气内留则易变生痈疽外证。\\
癫狂第二十二\\
目眦外决于面者,为锐眦。在内近鼻者,为内眦。上为外眦,下为内眦。\\
癫疾始生,先不乐,头重痛,视举目赤,甚作极,已而烦心,候之于颜。取手太阳、阳明、太阴,血变而止。\\
癫疾始作,而引口啼呼者,候之手阳明、太阳。左强者,攻其右;右强者,攻其左,血变而止。癫疾始作,先反僵,因而脊痛,候之足太阳、阳明、太阴、手太阳,血变而止。\\
治癫疾者,常与之居,察其所当取之处。病至,视之有过者泻之,置其血于瓠壶之中,至其发时,血独动矣;不动,灸穷骨二十壮。穷骨者,骶骨也。\\
骨癫疾者,圢齿诸腧、分肉皆满而骨居,汗出烦悗;呕多沃沫,气下泄,不治。\\
筋癫疾者,身倦挛急脉大,刺项大经之大杼脉;呕多沃沫,气下泄,不治。\\
脉癫疾者,暴仆,四肢之脉皆胀而纵。脉满,尽刺之出血,不满,灸之挟项太阳,灸带脉于腰,相去三寸,诸分肉本输。呕多沃沫,气下泄,不治。\\
癫疾者,疾发如狂者,死不治。\\
狂始生,先自悲也,喜忘、苦怒、善恐者,得之忧饥。治之取手太阴、阳明,血变而止,及取足太阴、阳明。狂始发,少卧不饥,自高贤也,自辩智也,自尊贵也,善骂詈,日夜不休。治之取手阳明、太阳、太阴、舌下、少阴。视之盛者,皆取之,不盛,释之也。\\
狂言、惊、善笑、好歌乐,妄行不休者,得之大恐。治之取手阳明、太阳、太阴。狂,目妄见、耳妄闻,善呼者,少气之所生也。治之取手太阳、太阴、阳明、足太阴、头、两圢。狂者多食,善见鬼神,善笑而不发于外者,得之有所大喜。治之取足太阴、太阳、阳明,后取手太阴、太阳、阳明。狂而新发,未应如此者,先取曲泉左右动脉,及盛者见血,有顷已;不已,以法取之,灸骨骶二十壮。\\
风逆暴四肢肿,身漯漯,唏然时寒,饥则烦,饱则善变。取手太阴表里,足少阴、阳明之经。肉清,取荥,骨清,取井、经也。\\
厥逆为病也,足暴清,胸若将裂,肠若将以刀切之,烦而不能食,脉大小皆涩。暖取足少阴,清取足阳明。清则补之,温则泻之。厥逆腹胀满,肠鸣,胸满不得息,取之下胸二胁,咳而动手者,与背腧,以手按之,立快者,是也。\\
内闭不得溲,刺足少阴、太阳与骶上,以长针。气逆则取其太阴、阳明、厥阴,甚取少阴、阳明动者之经也。\\
少气,身漯漯也,言吸吸也,骨痠体重,懈惰不能动,补足少阴。短气,息短不属,动作气索,补足少阴,去血络也。\\
眼角向外凹陷于面颊一侧的,叫目锐眦。在眼的内侧靠近鼻梁的,叫目内眦。上眼胞属目外眦,下眼胞属目内眦。\\
癫病开始发作,病人先闷闷不乐,头重痛,两眼上视而发红,发作严重时,出现心中烦乱,可通过颜面部的色泽、表情来候察。治疗可取手太阳、手阳明、手太阴三经的一些腧穴,等到面部的血色转为正常时就停针。\\
癫病开始发作,口角抽搐歪斜,发出啼叫声,应诊察手阳明、手太阳两经。根据病情而施治,凡左侧正常的,应刺右侧;右侧正常的,应刺左侧。患者面部的血色转为正常时停针。癫病开始发作时,先见腰脊反张僵硬,接着觉得脊柱作痛,候察其病变所在,可取足太阳、足阳明、足太阴、手太阳经的一些腧穴,等到患者面部的血色转为正常时停针。\\
治疗癫病时,医生应常和病者一起居住,观察所应取刺的腧穴。病发时,根据其有病的经脉,使用泻法出血,将泻出的血放在葫芦内,等到再复发时,其血就会动;如果不动,可灸穷骨二十壮。穷骨就是骶骨。\\
病深入骨的骨癫疾,腮齿诸腧分肉皆胀满,而骨骼僵直,常出汗,胸中烦闷;倘呕吐多白沫,肾气下泄,就是死证。\\
病深入筋的筋癫疾,身体踡缩,筋脉拘急,脉大。宜刺项后足太阳膀胱经的大杼穴;倘呕吐多白沫,肾气下泄,就是死证。\\
病深入脉的脉癫疾,发病时突然仆倒,四肢的脉都胀满弛纵。在脉满处,都可以针刺出血;如脉不满,宜灸挟行于项后两侧的足太阳经的腧穴,并灸带脉穴于腰相距三寸许的地方,及诸经的分肉之间与四肢的腧穴;倘呕吐多白沫,及气下泄的,就是死证。\\
癫疾病,如发作时像狂证一样,就是不治的死证。\\
狂证开始发作,先有悲伤之情,健忘易怒,时常恐惧,这是因为过度忧愁与饥饿所致。治疗取手太阴经、手阳明经的腧穴,等到面部的血色转为正常时停针,并取足太阴经、足阳明经的一些腧穴。狂证开始发作,少睡眠,不饥饿,自以为最伟大,自以为最聪明,自以为最尊贵,好骂人,日夜吵闹不休。治疗取手阳明、手太阳、手太阴、手少阴经的腧穴及舌下的廉泉穴。但要注意只有血脉盛的才可施针,血脉不盛的不能用。\\
患者语言狂妄,易惊好笑,喜欢歌唱,行为反常而不停止,这是大恐所致。治疗取手阳明、手太阳、手太阴经的腧穴。狂证发作时,幻视幻听,大喊大叫,这是神气衰少所致。治疗可取手太阳、手太阴、手阳明、足太阴经的腧穴,以及头部和两吇的腧穴。发狂的病人,多食而不饱,时常看到神鬼,窃笑而不表现于外,这是喜乐过度所致。治疗先取足太阴、足太阳、足阳明的腧穴,后再取手太阴、手太阳、手阳明的腧穴。如狂初起,还没有出现以上严重症状时,先取左右曲泉,以及血脉盛处,用针泻血,不久就可痊愈了;如果还没有治愈,再按上述治法治疗,并灸骨骶二十壮。\\
外感风邪,厥气内逆,突然四肢疼痛,身体出汗,时常因寒冷而发出唏嘘声,饥饿时心中烦乱,吃饱后又多动而不安。治疗可取手太阴与手阳明表里两经,以及足少阴、足阳明经的腧穴。如果肌肉寒冷的,取荥穴;骨骼寒冷的,取井穴与经穴。\\
厥逆病的症状,两足突然发冷,胸痛如裂,肠痛如刀割,心中烦乱而不能进食,脉无论大小都兼涩象。如身体温暖的,取足少阴经的腧穴;如身体发冷的,取足阳明经的腧穴。身体寒冷的用补法,身体温热的用泻法。厥逆病见腹胀肠鸣,胸中满闷,呼吸不利,取胸下两胁肋间,咳嗽则脉动应手的腧穴,再取背腧穴,用手按压感觉轻快,就是应刺的穴位。\\
内气闭阻而小便不通,取足少阴与足太阳两经的腧穴与骶骨的长强穴,用长针刺之。气机上逆,取足太阴、足阳明、足厥阴经的腧穴,严重的,取足少阴与足阳明经发生变动的腧穴。\\
少气的病人,身体出汗,言语不相连续,骨节发酸,身体沉重,身体懈惰无力而不能动作,取足少阴经的腧穴用补法。气息短促,呼吸不能连续,活动就感到气虚疲乏,补足少阴经的腧穴,其脉有淤血时,应刺之出血。\\
热病第二十三\\
偏枯,身偏不用而痛,言不变,志不乱,病在分腠之间,巨针取之。益其不足,损其有余,乃可复也。痱之为病也,身无痛者,四肢不收,智乱不甚,其言微知,可治;甚则不能言,不可治也。病先起于阳,后入于阴者,先取其阳,后取其阴,浮而取之。\\
热病三日,而气口静、人迎躁者,取之诸阳,五十九刺,以泻其热而出其汗,实其阴以补其不足者。身热甚,阴阳皆静者,勿刺也。其可刺者,急取之,不汗出则泄。所谓勿刺者,有死征也。\\
热病七日、八日,脉口动,喘而眩者,急刺之,汗且自出,浅刺手大指间。\\
热病七日、八日,脉微小,病者溲血,口中干,一日半而死。脉代者,一日死。热病已得汗出,而脉尚躁,喘且复热,勿刺肤,喘甚者,死。热病七日、八日,脉不躁,躁不散数,后三日中有汗。三日不汗,四日死,未曾汗者,勿腠刺之。\\
热病先肤痛,窒鼻充面,取之脉,以第一针,五十九。苛轸鼻,索皮于肺,不得索之火。火者,心也。\\
热病先身涩,倚而热,烦悗,干唇,口嗌,取之皮,以第一针,五十九;肤胀,口干,寒汗出,索脉于心,不得索之水。水者,肾也。\\
热病,嗌干多饮,善惊,卧不能安,取之肤肉,以第六针,五十九;目眦青,索肉于脾,不得索之木。木者,肝也。\\
热病面青脑痛,手足躁,取之筋间,以第四针,于四逆;筋躄,目浸,索筋于肝,不得索之金。金者,肺也。\\
热病数惊,瘛疭而狂,取之脉,以第四针,急泻有余者。癫疾毛发去,索血于心,不得索之水。水者,肾也。\\
热病身重骨痛,耳聋而好瞑,取之骨,以第四针,五十九,刺骨;病不食,啮齿,耳青,索骨于肾,不得索之土。土者,脾也。\\
热病不知所痛,耳聋,不能自收,口干,阳热甚,阴颇有寒者,热在髓,死不可治。\\
热病头痛,颞颥目刓脉痛,善衄,厥热病也。取之以第三针,视其有余不足。\\
热病体重,肠中热,取之以第四针,于其腧及下诸指间,索气于胃络,得气也。\\
热病挟脐急痛,胸胁满,取之涌泉与阴陵泉,取以第四针,针嗌里。\\
热病而汗且出,及脉顺可汗者,取之鱼际、太渊、大都、太白,泻之则热去,补之则汗出,汗出太甚,取内踝上横脉,以止之。\\
热病已得汗而脉尚躁盛,此阴脉之极也,死;其得汗而脉静者,生。热病者脉尚盛躁而不得汗者,此阳脉之极也,死;脉盛躁得汗静者,生。\\
热病不可刺者,有九:一曰:汗不出,大颧发赤,哕者,死;二曰:泄而腹满甚者,死;三曰:目不明,热不已者,死;四曰:老人婴儿,热而腹满者,死;五曰:汗不出,呕下血者死;六曰:舌本烂,热不已者,死;七曰:咳而衄,汗不出,出不至足者,死;八曰:髓热者,死;九曰:热而痉者,死。腰折,瘛疭,齿噤妀也。凡此九者,不可刺也。\\
所谓五十九刺者,两手外内侧各三,凡十二痏;五指间各一,凡八痏,足亦如是;头入发一寸傍三分各三,凡六痏;更入发三寸边五,凡十痏;耳前后口下者各一,项中一,凡六痏;巅上一,囟会一,发际一,廉泉一,风池二,天柱二。\\
气满胸中喘息,取足太阴大指之端,去爪甲如薤叶。寒则留之,热则疾之,气下乃止。\\
心疝暴痛,取足太阴、厥阴,尽刺去其血络。\\
喉痹,舌卷,口中干,烦心心痛,臂内廉痛,不可及头,取手小指次指爪甲下,去端如韭叶。\\
目中赤痛,从内眦始,取之阴\\
。风痉身反折,先取足太阳及腘中及血络出血;中有寒,取三里。\\
癃,取之阴\\
及三毛上及血络出血。\\
男子如蛊,女子如怚,身体腰脊如解,不欲饮食,先取涌泉见血,视跗上盛者,尽见血也。\\
偏枯病的症状,半身不遂而疼痛,但言语如常,神志清楚,这是病在分肉腠理之间,应当用大针治疗。补益正气祛除邪气,才能恢复正常。痱病的症状,身体并不疼痛,但四肢弛缓而不收,意志错乱而不甚,讲话略微还能听明白的,尚可治疗;病重不能讲话的,就无法治愈了。这种病如果先从阳分开始,后转入阴分的,应当先治阳分,然后再治阴分,用浅刺的方法。\\
热病已经三天,气口脉象平静而人迎脉象躁动不宁,应取治各阳经,治疗热病的五十九个腧穴,以泻其热邪,发汗,并充实三阴经,以补阴分的不足。如果身热很重,但人迎、气口的脉象却平静,不可用针刺。如果还可针刺的,应立即施治,即使不得汗出,邪气也会外泄。所以“勿刺”的缘故,是有死亡的征象。\\
热病七八天,寸口脉象躁动,气喘头眩,应赶快针刺,汗将自出,浅刺手大指间的少商穴。\\
热病七八天,脉象微小,病人尿血,口中干燥,一天半就可能死亡。再出现代脉,一天就会死亡。热病已出汗,而脉象仍然躁动,气喘,又见发热,不宜再浅刺皮肤以重伤正气,如气喘严重的就会死亡。热病七八天,脉象并不躁动,或虽有躁动而并无“散”、“数”之象,三天之内可能出汗。如果三天还不出汗,第四日就会死亡;未曾得汗的患者,也不可再浅刺其腠理以发汗解表。\\
热病先见皮肤疼痛,鼻孔阻塞,面部肿胀,当取治于皮,用九针中的第一针镵针,选用热病五十九腧。如鼻部有小疹,取肺经腧穴,以治皮肤之病,不可取治于“火”。所谓“火”,就是心经。\\
热病先见皮肤粗涩不爽,无力而热,烦闷,口唇咽喉干燥,治疗当取治于肺,用九针的第一针镵针,选用热病五十九腧;如果肌肤胀满,口干,出冷汗,应取心经腧穴治疗脉的病,不可取治于“水”。所谓“水”,就是肾经。\\
热病见咽喉干燥,饮水多,易惊,卧床不起,应取治肤肉,用九针中的第六针员利针,选用热病五十九输穴;如见眼角发青,因为脾主肌肉,应当取脾经的输穴以治疗肌肉的病,不可取治于“木”。所谓“木”,就是肝经。\\
热病见面色青,脑部疼痛,手足躁动不安,应当取治于筋间,用九针中的第四针锋针,针其四肢厥逆;如筋躄足不能行,泪出不收,应取肝经腧穴治疗筋病,不可取治于“金”。所谓“金”就是肺经。\\
热病屡发惊痫,手足抽搐而狂躁,应取治于脉,用九针中的第四针锋针,急泻其热邪。癫疾而毛发脱落的,应取心经腧穴治疗血病,不可取治于“水”。所谓“水”,就是肾经。\\
热病身体沉重,骨节疼痛,耳聋,嗜睡,应当取治于骨,用九针的第四针锋针,选用热病五十九腧穴;如患者不思饮食,咬牙,耳色青,因为肾主骨,应取肾经腧穴治疗骨病,不可取治于“土”。所谓“土”,就是脾经。\\
热病不能自知其痛处,耳聋失聪,四肢懈惰不能自主运动,口干,阳热已极而阴分仅有寒意,这是热在骨髓的征象,为不治的死证。\\
热病见头痛,颞颥部连及眼睛的脉络抽掣作痛,鼻易出血,这是厥热病。用九针中的第三针奼针,根据疾病的虚实,用不同的手法进行补泻。\\
热病身体沉重,肠中有热,用九针中的第四针锋针,取脾胃二经的“输穴”太白、陷谷,以及诸足趾间之腧穴,并可取胃经的络穴丰隆导引经气,而后才能得气。\\
热病见挟脐两侧拘急疼痛,胸胁胀满,取涌泉与阴陵泉穴,用九针中的第四针锋针,针咽喉的廉泉穴。\\
热病,汗将出,脉象为顺,可以用针取汗时,当取鱼际、太渊、大都、太白穴,用泻法则热邪可去,用补法则能出汗,如果汗出太多的,可取内踝上横脉处的三阴交穴来制止。\\
热病已出汗而脉尚躁动盛大的,这是阴脉衰极的征象,为死证;汗出后脉象平缓的,可生。热病脉象尚躁动盛大,不得汗出的,是阳脉亢盛至极的症象,为死证;如脉象虽盛大躁动,汗出后而脉象平静的,可生。\\
热病有九种死证,不可针刺:一是汗不得出,两颧骨发红而呃逆的,是死证;二是泄泻而腹胀满严重的,是死证;三是目视不明,热势不退的,是死证;四是老年人与婴儿,发热而腹胀满的,是死证;五是汗不得出,呕血便血的,是死证;六是舌根溃烂,热势不退的,是死证;七是咳嗽而鼻出血,汗不得出,或虽汗出而不到脚的,是死证;八是热邪已深入骨髓的,是死证;九是热极而发痉的,是死证。所谓“热而痉”的症状是腰脊反张、手足抽搐、口噤咬牙等。凡是以上九种征象,都不宜使用针刺。\\
所谓治疗热病的五十九穴,就是两手外侧与内侧各有三穴,共计十二穴;手五指间各一穴,共计八穴,足五趾间也是这样;头部入发际一寸,中行旁开有三处,每侧各三穴,左右共六穴;再从入发际的中行向后三寸,每侧各五穴,左右共十穴;耳前后各一穴,口下一穴,项中一穴,共计六穴;巅顶上一穴,囱会一穴,前发际一穴,后发际一穴,廉泉一穴,左右风池共两穴,左右天柱共两穴。\\
气逆壅满胸中,喘息急促,可取足太阴经脉在足大趾端距离爪甲约薤叶宽处的隐白穴。寒证留针,热证快出针,待逆气下降而不喘,就停止。\\
心疝病,突然疼痛,可取足太阴、足厥阴二经,针刺放尽络脉的淤血。\\
喉痹,舌卷,口干,心烦,心痛,臂部内侧疼痛,不能上举至头,可取无名指爪甲端距离如韭菜叶宽处的关冲穴。\\
眼睛红肿疼痛,从内眼角开始的,取阴刉脉的照海穴。因风而致痉挛,角弓反张,先取足太阳经及膝腘中的委中穴,以及浮浅的络脉,刺出其血;中焦有寒,取足三里穴。\\
小便不通,取阴刉脉的照海穴和足大趾三毛处的大敦穴,以及浮浅的络脉,刺出其血。\\
男子腹胀如蛊,女子郁阻之病,身体腰脊如散解,不思饮食,先取涌泉穴出血,再看足背上有血盛的络脉,同样刺出其血。\\
厥病第二十四\\
厥头痛,面若肿起而烦心,取之足阳明、太阴。\\
厥头痛,头脉痛,心悲善泣,视头动,脉反盛者,刺尽去血,后调足厥阴。\\
厥头痛,贞贞头重而痛,泻头上五行,行五,先取手少阴,后取足少阴。\\
厥头痛,意善忘,按之不得,取头面左右动脉,后取足太阴。\\
厥头痛,项先痛,腰脊为应,先取天柱,后取足太阳。\\
厥头痛,头痛甚,耳前后脉涌有热,泻出其血,后取足少阳。\\
真头痛,头痛甚,脑尽痛,手足寒至节,死不治。\\
头痛不可取于腧者,有所击堕,恶血在于内,若肉伤,痛未已,可则刺,不可远取也。\\
头痛不可刺者,大痹为恶,日作者,可令少愈,不可已。\\
头半寒痛,先取手少阳、阳明,后取足少阳、阳明。\\
厥心痛,与背相控,如从后触其心,伛偻者,肾心痛也,先取京骨、昆仑,发针不已,取然谷。\\
厥心痛,腹胀胸满,心尤痛甚,胃心痛也,取之大都、太白。\\
厥心痛,痛如以锥针刺其心,心痛甚者,脾心痛也,取之然谷,太溪。\\
厥心痛,色苍苍如死状,终日不得休息,肝心痛也,取之行间、太冲。\\
厥心痛,卧若从居,心痛间,动作痛益甚,色不变,肺心痛也,取之鱼际、太渊。\\
真心痛,手足清至节,心痛甚,旦发夕死,夕发旦死。\\
心痛不可刺者,中有盛聚,不可取于腧。\\
肠中有虫瘕及蛟蛕,皆不可取以小针。\\
腹中痛,发作肿聚,往来上下行,痛有休止,腹热,喜涎出者,是蛟蛕也。以手聚按而坚,持之,无令得移,以大针刺之,久持之,虫不动,乃出针也。\\
耳聋无闻,取耳中。\\
耳鸣,取耳前动脉。\\
耳痛不可刺者,耳中有脓,若有干耵聍,耳无闻也。\\
耳聋,取手足小指次指爪甲上与肉交者,先取手,后取足。耳鸣,取手中指爪甲上,左取右,右取左,先取手,后取足。\\
髀不可举,侧而取之,在枢合中,以员利针,大针不可刺。\\
病注下血,取曲泉。\\
风痹淫病不可已者,足如履冰,时如入汤中。股胫淫泺,烦心头痛,时呕时悗,眩已汗出,久则目眩,悲以喜恐,短气不乐,不出三年,死也。\\
厥头痛,面部浮肿,心中烦躁的,可取足阳明与足太阴经的腧穴。\\
厥头痛,头部脉络疼痛,心中悲伤,好哭,可以诊察到头部颤动,络脉充盛,用针刺出血,然后取足厥阴经的腧穴。\\
厥头痛,痛有定处,并有沉重感,治疗用泻法,取头部正中督脉与两旁的足太阳、足少阳经,共计五行,每行五穴,合计二十五穴;先取手少阴经,后取足少阴经的腧穴。\\
厥头痛,长叹气、健忘,触摸不到疼痛部位,先取在头面部左右的动脉,然后再取足太阴经的腧穴。\\
厥头痛,从项部先痛,腰脊部也相应疼痛,先取天柱穴,后取足太阳经的腧穴。\\
厥头痛,痛得很厉害,耳前耳后的脉络都怒胀有热,先泻出其血,后取足少阳经的腧穴。\\
真头痛,疼痛剧烈,如果整头疼痛,手足发凉至关节的,这是不治的死证。\\
头痛,有的不能取腧穴治疗,如被击伤或从高处跌落,淤血留阻于内或肌肉受伤而疼痛不止,可在受伤的局部针刺,不能用远处的腧穴。\\
头痛不能仅用针刺,是由大痹为患,每天发作,用针刺可使痛势减轻,但不能根治。\\
头一侧发冷疼痛,先取手少阳、手阳明经的腧穴,后取足少阳、足阳明经的腧穴。\\
厥心痛,牵引至背部,好像从背后触动心脏一样,以致病人不敢伸直腰板,这是肾气厥逆的心痛,先取京骨、昆仑穴,出针后疼痛立止,如不止,再取然谷穴。\\
厥心痛,胸腹胀满,心口疼痛剧烈,这是胃气厥逆的心痛,可取大都、太白穴。\\
厥心痛,痛如以锥刺心,心口疼痛剧烈,这是脾气厥逆的心痛,可取然谷、太溪穴。\\
厥心痛,面色青苍如死人,整天疼痛不止,这是肝气厥逆的心痛,可取行间、太冲穴。\\
厥心痛,当安卧或从容闲居时,疼痛缓解,活动时加重,但面色不变,这是肺气厥逆的心痛,可取鱼际、太渊穴。\\
邪气在心的真心痛,手足寒冷至肘膝关节,心胸痛势剧烈,早上发作到晚上就会死亡,晚上发作到次日早上就会死亡。\\
心痛不能用刺法治疗的,是因为内有积聚或淤血,所以不能取穴治疗。\\
肠内有虫积或蛔虫,都不适宜用小针治疗。\\
腹部疼痛,发作时有肿块聚起,上下游走不定,时痛时止,腹部热,易流口涎,这是有蛔虫的征象。手指并拢,按紧肿块,不让它移动,然后用大针刺之,长时按住,等虫不动,才出针。\\
耳聋听不到声音,取耳中的听宫穴。\\
耳鸣,取耳前动脉处的耳门穴。\\
耳内疼痛,不能针刺的是指耳中有脓,或有干耳垢,以致听觉失常。\\
治疗耳聋,先取无名指爪甲上的关冲穴,后取足第四趾的窍阴穴。治疗耳鸣,可取手中指爪甲上端的中冲穴,左耳鸣取右穴,右耳鸣取左穴,先取手上的腧穴,再取足部的大敦穴。\\
大腿抬不起来,治疗时,让病人侧卧,在髀枢中的环跳穴,用员利针刺之,大针不能用。\\
下血如水注,取曲泉穴。\\
风痹病久不愈,两足时如履冰之感,时如入汤之热。大小腿部因邪气蔓延而痠痛无力,并见心烦,头痛,时呕吐或饱闷,目眩稍停就出虚汗,停一会又目眩,悲伤过后又生恐惧,呼吸短促,闷闷不乐,有这些症状三年内可能死亡。\\
病本第二十五\\
先病而后逆者,治其本;先逆而后病者,治其本。先寒而后生病者,治其本;先病而后寒者,治其本;先病而后泄者,治其本。先热而后生病者,治其本;先泄而后生他病者,治其本。必且调之,乃治其他病。先病而后中满者,治其标;先病而后泄者,治其本。先中满而后烦心者,治其本。有客气,有同气。大小便不利,治其标;大小便利,治其本。\\
病发而有余,本而标之,先治其本,后治其标。病发而不足,标而本之,先治其标,后治其本。谨详察间甚,以意调之,间者并行,甚者独行。先小大便不利而后生病者,治其本也。\\
先患某种病而后气血不和的,应当先治其本;先有气血不和而后患病的,先治气血不和的本病。先感受寒邪而后发生其他病变的,应当先治其本;先患病而后生寒变的,也当先治其本病。先患热病而后发生其他病变的,应当治其本病;先患泄泻而后又生其他病的,当先治疗泄泻。一定得先把泄泻治好,才可治疗其他病证。先患病而后发生中满的,应当先治它的标病;先患病而后发生泄泻的,应当先治本病;先患胸腹胀满证,而后又增加了心烦不舒的,应当治其本病。有邪气,也有真气。大小便不利的,应当先治其标病;大小便通利的,应当先治其本病。\\
若发病表现为有余的实证,应当用本而标之的治法,即先治其本,后治其标;若发病表现为不足的虚证,应当用标而本之的治法,即先治其标,后治其本。要谨慎地观察病情的轻重,根据具体病情而进行治疗。病轻的可以标本兼治,病重的就要根据病情,或治本或治标。先大小便不通利,而后并发其他疾病的,应当先治其本病。\\
杂病第二十六\\
厥,挟脊而痛者,至顶,头沉沉然,目丼丼然,腰脊强,取足太阳腘中血络。\\
厥,胸满面肿,唇累累然,暴言难,甚则不能言,取足阳明。\\
厥,气走喉而不能言,手足清,大便不利,取足少阴。\\
厥,而腹响响然,多寒气,腹中穀穀,便溲难,取足太阴。\\
嗌干,口中热如胶,取足少阴。\\
膝中痛,取犊鼻,以员利针,发而间之。针大如氂,刺膝无疑。\\
喉痹不能言,取足阳明;能言,取手阳明。\\
疟不渴,间日而作,取足阳明;渴而日作,取手阳明。\\
齿痛,不恶清饮,取足阳明;恶清饮,取手阳明。\\
聋而不痛者,取足少阳;聋而痛者,取手阳明。\\
衄而不止,衃血流,取足太阳;衃血,取手太阳;不已,刺腕骨下;不已,刺腘中出血。\\
腰痛,痛上寒,取足太阳、阳明;痛上热,取足厥阴;不可以俯仰,取足少阳;中热而喘,取足少阴、腘中血络。\\
喜怒而不欲食,言益少,刺足太阴;怒而多言,刺足少阳。\\
圢痛,刺手阳明与圢之盛脉,出血。\\
项病,不可俯仰,刺足太阳;不可以顾,刺手太阳也。\\
小腹满大,上走胃至心,淅淅身时寒热,小便不利,取足厥阴。\\
腹满,大便不利,腹大,亦上走胸嗌,喘息喝喝然,取足少阴。\\
腹满食不化,腹响响然,不能大便,取足太阴。\\
心痛引腰脊,欲呕,取足少阴。\\
心痛,腹胀。啬啬然,大便不利,取足太阴。\\
心痛引背不得息,刺足少阴;不已,取手少阳。\\
心痛引小腹满,上下无常处,便溲难,刺足厥阴。\\
心痛,但短气,不足以息,刺手太阴。\\
心痛,当九节刺之,按,已刺按之,立已;不已,上下求之,得之立已。\\
圢痛,刺足阳明曲周动脉见血,立已;不已,按人迎于经,立已。\\
气逆上,刺膺中陷者与下胸动脉。\\
腹痛,刺脐左右动脉,已刺按之,立已;不已,刺气街,已刺按之,立已。\\
痿厥为四末束悗,乃疾解之,日二,不仁者,十日而知,无休,病已,止。\\
哕,以草刺鼻,嚏,嚏而已;无息,而疾迎引之,立已;大惊之,亦可已。\\
经气厥逆,挟脊两旁疼痛,放散至头顶,头昏沉重,视物不清,腰脊强直,取足太阳经在腘中的委中穴处的络脉刺血。\\
经气厥逆,胸中气满,面目浮肿,口唇肿厚,突然语言难出,甚至不能言语,取足阳明经的腧穴治疗。\\
经气厥逆,上及喉部以致不能言语,手足寒冷,大便不利,取足少阴经的腧穴治疗。\\
经气厥逆,腹部膨胀,弹之有声,寒气滞留,腹中有水声,二便不利,取足太阴经的腧穴治疗。\\
咽喉干燥,口中热而唾液如胶,取足少阴经的腧穴治疗。\\
膝关节疼痛,取犊鼻穴,用员利针刺之,出针后隔些时候还可再刺。这种针身大如牛尾的长毛,用刺膝部最为适宜。\\
喉痹肿痛,不能说话的,取足阳明经的腧穴治疗;还能说话的,取手阳明经的腧穴治疗。\\
疟疾口不渴,隔日发作一次的,可取足阳明经的腧穴治疗;口渴,每日发作的,取手阳明经的腧穴治疗。\\
牙齿疼痛,不怕冷饮的,取足阳明经的腧穴治疗;怕冷饮的,取手阳明经的腧穴治疗。\\
耳聋不疼的,取足少阳经的腧穴治疗;耳聋兼见疼痛的,取手阳明经的腧穴治疗。\\
鼻出血不止,且有黑血流出的,取足太阳经的腧穴治疗;如淤血结滞,取手太阳经的腧穴治疗;不愈,刺腕骨下的腕骨穴治疗;再不愈,可刺委中出血。\\
腰痛,痛处发凉的,取足太阳、足阳明两经的腧穴治疗;痛处发热的,取足厥阴经的腧穴治疗;腰痛不能俯仰的,取足少阳经的腧穴治疗;内有热而气喘的,可取足少阴经的腧穴与委中处络脉刺血。\\
易发怒,不思饮食,讲话少的,刺足太阴经的腧穴;易怒而言语特别多的,可刺足少阳经的腧穴。\\
下巴疼痛,取手阳明经的腧穴与足阳明经的颊车穴泻血。\\
颈项疼痛,不能俯仰的,刺足太阳经的腧穴;不能左右盼顾的,刺手太阳经的腧穴。\\
小腹胀满膨大,向上波及胃脘及心胸,恶寒战栗,时常寒热,小便不利,取足厥阴经的腧穴治疗。\\
腹部胀满,大便不利,腹膨大,向上影响到胸部与喉咙,气喘有声,取足少阴经的腧穴治疗。\\
腹中胀满,食物积滞不化,腹中鸣响,大便不通,取足太阴经的腧穴治疗。\\
心痛牵引腰脊疼痛,恶心欲吐,取足少阴的腧穴治疗。\\
心痛,腹中胀满。肠中涩滞不通,大便不利,取足太阴经的腧穴治疗。\\
心痛牵引背部疼痛,呼吸不利,刺足少阴经的腧穴;不愈,取手少阴经的腧穴治疗。\\
心痛牵引小腹胀满,上下走窜,痛无定处,二便不利,刺足厥阴经的腧穴。\\
心痛,只见气短,呼吸困难,刺手太阴经的腧穴。\\
心痛,可在第九胸椎棘突下的筋缩穴针刺,先在穴位上按揉,刺后再继续按揉,可以立即止痛;痛还不止,再在该处上下寻求痛点刺治,就可立即止痛。\\
下巴痛,刺足阳明经在曲周部的颊车穴处出血,可以立即止痛;如果痛还不止,再按摩人迎部,就可立即止痛。\\
气逆上冲,可刺胸膺中凹陷处的膺窗穴,以及胸前下方的动脉处。\\
腹中疼痛,可刺脐左右动脉处的天枢穴,刺后再按摩该处,可以立即止痛;如痛还未止,可刺气冲穴,刺后再按摩,就可立即止痛。\\
痿与厥病,可将四肢束缚起来,待病者感觉气闷,就立即解开,每天两次,不知痛痒的,治疗十天就可恢复感觉,但不可中止,需继续至病愈为止。\\
呃逆证,可用草茎刺入鼻孔,使之喷嚏,打了喷嚏就能好;或者屏住呼吸,很快的将上逆之气引而下行,呃逆即止;或者使之突然受到惊吓,也可以立愈。\\
周痹第二十七\\
黄帝问于岐伯曰:周痹之在身也,上下移徙,随其脉上下,左右相应,间不容空,愿闻此痛,在血脉之中邪?将在分肉之间乎?何以致是?其痛之移也,间不及下针,其慉痛之时,不及定治,而痛已止矣,何道使然?愿闻其故。\\
岐伯答曰:此众痹也,非周痹也。\\
黄帝曰:愿闻众痹。\\
岐伯对曰:此各在其处,更发更止,更居更起,以右应左,以左应右,非能周也,更发更休也。\\
黄帝曰:善。刺之奈何?\\
岐伯对曰:刺此者,痛虽已止,必刺其处,勿令复起。\\
帝曰:善。愿闻周痹何如?\\
岐伯对曰:周痹者,在于血脉之中,随脉以上,随脉以下,不能左右,各当其所。\\
黄帝曰:刺之奈何?\\
岐伯对曰:痛从上下者,先刺其下以遏之,后刺其上以脱之;痛从下上者,先刺其上以遏之,后刺其下以脱之。\\
黄帝曰:善。此痛安生?何因而有名?\\
岐伯对曰:风寒湿气,客于外分肉之间,迫切而为沫,沫得寒则聚,聚则排分肉而分裂也,分裂则痛,痛则神归之,神归之则热,热则痛解,痛解则厥,厥则他痹发,发则如是。此内不在脏,而外未发于皮,独居分肉之间,真气不能周,故命曰周痹。故刺痹者,必先切循其上下之六经,视其虚实,及大络之血结而不通,及虚而脉陷空者而调之,熨而通之,其瘛坚,转引而行之。\\
黄帝曰:善。余已得其意矣,亦得其事也。\\
黄帝问岐伯说:周痹在人体,邪气随着经脉上下移动,上下左右相应疼痛,疼痛交替,几乎没有一点点空隙,请问这种疼痛,是在血脉之中呢?还是在分肉之间?是怎么形成的?疼痛转移很快,而来不及下针;邪气聚积疼痛时,还未及下针,而疼痛已停止了,这是什么道理?希望听听其中的缘故。\\
岐伯答道:这是众痹,而不是周痹。\\
黄帝说:希望听听众痹的情况。\\
岐伯答道:众痹的疼痛各有一定部位,交互发作和停止,交互留居和复起,右对应左,左对应右,左右相应,但不能周遍全身,而是交互发作和停止的。\\
黄帝说:好。怎样针刺这种病呢?\\
岐伯说:针刺这种痹证,疼痛虽然已经停止,但还要在病处刺治,不让它复发。\\
黄帝说:讲得好。希望再听听周痹的情况怎样?\\
岐伯答道:周痹的病邪留居血脉之中,所以能随着经脉上下移动,但是不能左右交互发作,而固定在一定的部位。\\
黄帝问:怎么针刺这种病呢?\\
岐伯答道:疼痛由上向下移动的,先刺其下部以遏止病邪,然后再刺其上部尽除病根;疼痛由下向上移动的,先刺其上部以遏止病邪,然后再刺其下部以尽除病根。\\
黄帝说:很好。这种痛疼是怎样产生的?又是根据什么定名的呢?\\
岐伯答道:风、寒、湿三气,侵袭到体表分肉之间,逼迫该部津液形成汁沫,汁沫遇到寒气而凝聚,汁沫凝聚处的分肉被排挤而分裂,分肉裂开则产生疼痛,痛时卫气就贯注到局部而发热,痛遇热就能缓解,痛势缓解后,就会厥气上逆,厥气上逆到其他部位,其他部位的痹痛又发作,发作时的情况也是这样。这种病既不在内脏,又不在外表的皮肤,仅仅留居分肉之间,使人身的真气不能正常周行,所以命名为“周痹”。针刺这种痹证时,必先要依次切循检查痹疼在上下六经的哪一经,并分析其虚实属性,以及大络之间有无淤血凝结不通,或虚弱而脉下陷,然后再调治,并可采用熨法,疏通经脉,如果拘急坚硬的,可以牵引运动病人肢体,帮助血气的运行。\\
黄帝说:很好。我已懂得了其中的道理,也懂得怎样治疗的方法。\\
口问第二十八\\
黄帝闲居,辟左右而问于岐伯,曰:余已闻九针之经,论阴阳逆顺,六经已毕,愿得口问。\\
岐伯避席再拜曰:善乎哉问也!此先师之所口传也。\\
黄帝曰:愿闻口传。\\
岐伯答曰:夫百病之始生也,皆生于风雨寒暑,阴阳喜怒,饮食居处,大惊卒恐。则血气分离,阴阳破败,经络厥绝,脉道不通,阴阳相逆,卫气稽留,经脉虚空,血气不次,乃失其常。论不在经者,请道其方。\\
黄帝曰:人之欠者,何气使然?\\
岐伯答曰:卫气昼日行于阳,夜半则行于阴。阴者主夜,夜者卧。阳者主上,阴者主下。故阴气积于下,阳气未尽,阳引而上,阴引而下,阴阳相引,故数欠。阳气尽,阴气盛,则目瞑;阴气尽而阳气盛,则寤矣。泻足少阴,补足太阳。\\
黄帝曰:人之哕者,何气使然?\\
岐伯曰:谷入于胃,胃气上注于肺。今有故寒气与新谷气俱还入于胃,新故相乱,真邪相攻,气并相逆,复出于胃,故为哕。补手太阴,泻足少阴。\\
黄帝曰:人之唏者,何气使然?\\
岐伯曰:此阴气盛而阳气虚,阴气疾而阳气徐,阴气盛而阳气绝,故为唏。补足太阳,泻足少阴。\\
黄帝曰:人之振寒者,何气使然?\\
岐伯曰:寒气客于皮肤,阴气盛,阳气虚,故为振寒寒栗。补诸阳。\\
黄帝曰:人之噫者,何气使然?\\
岐伯曰:寒气客于胃,厥逆从下上散,复出于胃,故为噫。补足太阴、阳明。\\
黄帝曰:人之嚏者,何气使然?\\
岐伯曰:阳气和利,满于心,出于鼻,故为嚏。补足太阳荣、眉本。\\
黄帝曰:人之亸者,何气使然?\\
岐伯曰:胃不实则诸脉虚,诸脉虚则筋脉懈惰,筋脉懈惰则行阴用力,气不能复,故为亸。因其所在,补分肉间。\\
黄帝曰:人之哀而泣涕出者,何气使然?\\
岐伯曰:心者,五脏六腑之主也;目者,宗脉之所聚也,上液之道也;口鼻者,气之门户也。故悲哀愁忧则心动,心动则五脏六腑皆摇,摇则宗脉感,宗脉感则液道开,液道开故泣涕出焉。液者,所以灌精濡空窍者也,故上液之道开则泣,泣不止则液竭,液竭则精不灌,精不灌则目无所见矣,故命曰夺精。补天柱经侠颈。\\
黄帝曰:人之太息者,何气使然?\\
岐伯曰:忧思则心系急,心系急则气道约,约则不利,故太息以伸出之。补手少阴、心主、足少阳,留之也。\\
黄帝曰:人之涎下者,何气使然?\\
岐伯曰:饮食者皆入于胃,胃中有热则虫动,虫动则胃缓,胃缓则廉泉开,故涎下。补足少阴。\\
黄帝曰:人之耳中鸣者,何气使然?\\
岐伯曰:耳者,宗脉之所聚也。故胃中空则宗脉虚,虚则下,溜脉有所竭者,故耳鸣。补客主人,手大指爪甲上与肉交者也。\\
黄帝曰:人之自啮舌者,何气使然?\\
岐伯曰:此厥逆走上,脉气辈至也。少阴气至则啮舌,少阳气至则啮颊,阳明气至则啮唇矣。视主病者,则补之。\\
凡此十二邪者,皆奇邪之走空窍者也。故邪之所在,皆为不足。故上气不足,脑为之不满,耳为之苦鸣,头为之苦倾,目为之眩;中气不足,溲便为之变,肠为之苦鸣;下气不足,则乃为痿厥心悗。补足外踝下,留之。\\
黄帝曰:治之奈何?\\
岐伯曰:肾主为欠,取足少阴。肺主为哕,取手太阴、足少阴。唏者,阴盛阳绝,故补足太阳,泻足少阴。振寒者,补诸阳。噫者,补足太阴、阳明。嚏者,补足太阳、眉本。亸,因其所在,补分肉间。泣出,补天柱经侠颈,侠颈者,头中分也。太息,补手少阴、心主、足少阳,留之。涎下,补足少阴。耳鸣,补客主人,手大指爪甲上与肉交者。自啮舌,视主病者则补之。目眩头倾,补足外踝下留之。瘘厥心悗,刺足大指间上二寸留之,一曰足外踝下,留之。\\
黄帝闲居,避开左右的人,对岐伯说:我已经知道九针在医经上所论述的阴阳经的逆顺走向和手足六经的道理,我还希望听听从问答口授中得到医学知识。\\
岐伯离开座席,拜了两拜,说:您问得好啊!这些知识都是先师口传给我的。\\
黄帝说:我希望听听这些口传的医学知识。\\
岐伯回答说:大凡疾病的发生,都是因为风雨寒暑,房事过度,喜怒不节,饮食不调,居处不适,大惊猝恐等。导致了血气分离,阴阳衰败,经络闭绝,脉道不通,阴阳逆乱,卫气滞留,经脉空虚,气血紊乱,人体失去正常状态。这些内容古代医经上没有记载,请让我说明这些方法。\\
黄帝问:人打呵欠,是什么气所致?\\
岐伯回答说:卫气白天循行阳分,夜间循行阴分。阴气主夜,入夜则睡眠。阳气升发而主上,阴气沉降而主下。所以人在夜间将睡之时,阴气聚集于下部,阳气还未全入阴分,阳仍有引气上升的作用;而同时,阴气开始引阳气向下降,阴阳上下相引,于是连连呵欠。等到阳气都入阴分,阴气大盛时,就能闭目安眠;等到阴气尽而阳气盛,就醒了。这样的症状,泻足少阴经,补足太阳膀胱经。\\
黄帝问:人发生呃逆,是什么气所致?\\
岐伯说:正常情况下,饮食物进入胃中,经过脾胃的腐熟运化,把精微上注到肺。如果中焦先感受寒邪,又新进饮食,寒邪与食物都存留胃中,新进的饮食与原有的寒邪互相扰乱,邪正相争,邪气与胃气相攻相逆,再从胃中出,而上入胸膈,所以发生呃逆。这样的症状,补手太阴肺经,泻足少阴肾经。\\
黄帝问:人发生哽咽,是什么气所致?\\
岐伯说:这是由于阴气盛而阳气虚,阴气运行快,阳气运行慢,甚至阴气过盛,阳气衰绝,所以造成哽咽。这样的症状,应补足太阳经,泻足少阴经。\\
黄帝问:人发冷打战,是什么气所致?\\
岐伯说:寒邪侵入皮肤,寒邪偏盛,体表阳气偏虚,所以出现发冷、战抖的症状。这样的症状,当采用温补各阳经的方法。\\
黄帝问:人发生嗳气,是什么气所致?\\
岐伯回答说:寒邪侵入胃中,厥逆之气从下向上扩散,再从胃中出,所以出现嗳气。这样的症状,应该补足太阴脾经和足阳明胃经。\\
黄帝问:人打喷嚏,是什么气所致?\\
岐伯说:阳气和利,布满于心胸而上出于鼻,就会打喷嚏。这样的症状,应补足太阳荥穴通谷,以及眉根部的攒竹穴。\\
黄帝问:人发生全身无力、疲困懈惰,是什么气所致?\\
岐伯说:胃气虚,以致各经脉皆虚;各经脉的虚衰就导致筋脉懈惰无力;筋脉懈惰无力,再勉强行房,则元气不能恢复,于是出现懈惰无力的亸证。这样的症状,应根据病变发生部位,在分肉间施用补法。\\
黄帝问:人因哀伤而涕泪俱出,是什么气所致?\\
岐伯答道:心是五脏六腑的主宰;眼睛是诸多经脉聚会的地方,也是津液在上部外泄的道路;口鼻是气出入的门户。大凡悲哀忧愁等情志变化,就会扰动心神,心神扰动不安则五脏六腑受到影响不安,脏腑不安进而触动各经脉,经脉被触动,从而使眼及口鼻的液道开张,液道开张涕泪就由此出来了。人体之液是渗灌精气濡养孔窍的,所以上液之道开张就流泪,而流泪不止则精液耗竭,精液耗竭不能渗灌精气而濡养孔窍,所以眼目看不见东西,这叫作“夺精”。这样的症状,应补足太阳经在项部的天柱穴。\\
黄帝问:人的叹气,是什么气所致?\\
岐伯说:忧愁思虑则心系急迫,心系急迫则约束气道,气道约则呼吸不利,所以不时长出气,以伸展其气。这样的症状,应补手少阴经、手厥阴经、足少阳经,用留针的方法。\\
黄帝问:人流口涎,是什么气所致?\\
岐伯说:饮食都入胃中,若胃中有热,寄生虫因热而蠕动,虫动会使胃气弛缓,胃缓则舌下廉泉开张而流涎。这样的症状,应补足少阴肾经。\\
黄帝问:人发生耳鸣,是什么气所致?\\
岐伯答道:耳部是宗脉聚集的地方。若胃中空虚,水谷精气供给不足,则宗脉必虚,宗脉虚则阳气下陷不升,精微不能上入耳部的经脉,致气血不充而耗竭,所以耳鸣。这样的症状,应在足少阳胆经的客主人穴,以及位于手大指爪甲角的手太阴肺经的少商穴施以补法。\\
黄帝问:人有时自己咬舌,是什么气所致?\\
岐伯说:这是由于厥逆之气上行,影响到各经脉之气分别上逆而致。如少阴脉气上逆,就会咬舌;少阳脉气上逆,就会咬颊部;阳明脉气上逆,就会咬唇。这样的症状,应该根据发病部位,确定属于何经,而施以补法。\\
大凡这十二种病邪,都是奇邪侵入孔窍形成的。所以邪气侵害的部位,都是因为正气不足。所以上气不足,则脑髓不满,症见耳鸣、头倾、目眩;中气不足,症见二便失常、肠中鸣响;下气不足,两足痿弱无力、厥冷,心胸窒闷。治疗时,补足太阳经位于足外踝后部的昆仑穴,并用留针法。\\
黄帝问:以上各病,怎样治疗呢?\\
岐伯说:肾主呵欠,故呵欠应取足少阴肾经。肺主呃逆,故呃逆应取手太阴肺经以及足少阴肾经。哽咽是由于阴盛阳衰,所以要补足太阳膀胱经、泻足少阴肾经。发冷打战,要补各阳经。嗳气,应补足太阴脾经和足阳明胃经。喷嚏,当补足太阳膀胱经的攒竹穴。肢体懈惰无力,根据发病部位,补分肉间。哭泣流泪不止,当补位于项后中行两旁的足太阳经的天柱穴。叹气,当补手少阴心经、手厥阴心包经和足少阳胆经,用留针法。流口涎,补足少阴肾经。耳鸣,补足少阳胆经的客主人穴,以及位于手大指爪甲角部的手太阴肺经的少商穴。自咬舌颊等部位,应据发病部位的所属经脉分别施用补法。目眩、头倾,补足外踝后的昆仑穴,用留针法。肢痿无力而厥冷、心胸窒闷的,刺足大趾本节后二寸处,用留针法,一说可针刺足外踝后的昆仑穴,并用留针法。\\
卷六\\
师传第二十九\\
黄帝曰:余闻先师,有所心藏,弗著于方。余愿闻而藏之,则而行之。上以治民,下以治身,使百姓无病。上下和亲,德泽下流。子孙无忧,传于后世。无有终时,可得闻乎?\\
岐伯曰:远乎哉问也!夫治民与自治,治彼与治此,治小与治大,治国与治家,未有逆而能治之也,夫惟顺而已矣。顺者,非独阴阳脉论气之逆顺也,百姓人民皆欲顺其志也。\\
黄帝曰:顺之奈何?\\
岐伯曰:入国问俗,入家问讳,上堂问礼,临病人问所便。\\
黄帝曰:便病人奈何?\\
岐伯曰:夫中热消瘅则便寒,寒中之属则便热。胃中热则消谷,令人悬心善饥。脐以上皮热,肠中热,则出黄如糜。脐以下皮寒,肠中寒,则肠鸣飧泄。胃中寒,肠中热,则胀而且泄。胃中热,肠中寒,则疾饥,小腹痛胀。\\
黄帝曰:胃欲寒饮,肠欲热饮,两者相逆,便之奈何?且夫王公大人血食之君,骄恣从欲,轻人,而无能禁之,禁之则逆其志,顺之则加其病,便之奈何?治之何先?\\
岐伯曰:人之情,莫不恶死而乐生。告之以其败,语之以其善,导之以其所便,开之以其所苦。虽有无道之人,恶有不听者乎?\\
黄帝曰:治之奈何?\\
岐伯曰:春夏先治其标,后治其本;秋冬先治其本,后治其标。\\
黄帝曰:便其相逆者奈何?\\
岐伯曰:便此者,食饮衣服,亦欲适寒温。寒无凄怆,暑无出汗。食饮者,热无灼灼,寒无沧沧,寒温中适。故气将持。乃不致邪僻也。\\
黄帝曰:《本脏》以身形支节夬肉,候五脏六腑之小大焉。今夫王公大人,临朝即位之君而问焉,谁可扪循之而后答乎?\\
岐伯曰:身形支节者,脏腑之盖也,非面部之阅也。\\
黄帝曰:五脏之气,阅于面者,余已知之矣,以肢节而阅之奈何?\\
岐伯曰:五脏六腑者,肺为之盖,巨肩陷咽,候见其外。\\
黄帝曰:善。\\
岐伯曰:五脏六腑,心为之主,缺盆为之道,刐骨有余,以候巿旡。\\
黄带曰:善。\\
岐伯曰:肝主为将,使之候外,欲知坚固,视目小大。\\
黄帝曰:善。\\
岐伯曰:脾主为卫,使之迎粮,视唇舌好恶,以知吉凶。\\
黄帝曰:善。\\
岐伯曰:肾主为外,使之远听,视耳好恶,以知其性。\\
黄帝曰:善。愿闻六腑之候。\\
岐伯曰:六腑者,胃为之海,广骸、大颈、张胸,五谷乃容;鼻隧以长,以候大肠;唇厚、人中长,以候小肠;目下果大,其胆乃横;鼻孔在外,膀胱漏泄,鼻柱中央起,三焦乃约。此所以候六腑者也。上下三等,脏安且良矣。\\
黄帝说:我听说先师有许多心得,没记载在书籍中。我希望听听这些心得而珍藏起来,作为准则推行,上以治民,下以治身,使百姓无病。上下和美亲善,恩德教泽在民间流行。子孙无病可虑,传于后代,永无终止。所有这些,可以让我听到吗?\\
岐伯说:您问得深远啊!治民和治己,治彼和治此,治小和治大,治国和治家,从来没有用逆行的方法而能治理好的,只有采取顺行的方法。但所说的顺,不仅是指阴阳经脉营卫的逆顺,对待人民百姓,也要顺着他们的意愿。\\
黄帝问:顺之怎样去做呢?\\
岐伯说:进入一个国家,要问明当地的风俗,进入人家,要问明他家的忌讳,登堂更要问明人家的礼节,医生临证也要问病人怎样觉得舒适。\\
黄帝问:怎样使病人觉得舒适呢?\\
岐伯说:人内热患了消瘅病,适宜于寒治法;寒中病适于热治法。胃中有热,谷物消化得就快,人心如悬,总有饿感。脐以上的皮肤发热,是肠中有热,排出的粪便黄如糜粥。脐以下的皮肤觉寒,是肠中有寒,会肠鸣飧泄。胃中有寒,肠中有热,会出现腹胀腹泻。胃中有热,肠中有寒,出现易饿,小腹胀痛。\\
黄帝问:胃热宜于寒饮,肠寒宜于热饮,二者寒热相反,应该怎样治疗呢?尤其像王公大人,肉食之君,都骄傲纵欲,轻视别人,无法劝阻他们,劝阻就违背他们的意志,如顺着他们的意志,就会加重病情。像这样,如何治疗?先从哪里着手呢?\\
岐伯说:人之常情,没有不怕死而喜爱活着的。告诉他哪些对人有害处,哪些对人有好处,用适宜的指导他,解开他心中的苦痛。就是不太懂理的人,怎么会不听劝告呢?\\
黄帝问:怎样治疗呢?\\
岐伯说:春夏,先治在外的标病,后治在内的本病;秋冬,先治在内的本病,后治在外的标病。\\
黄帝问:怎样从病人的喜爱来适应其病情呢?\\
岐伯说:顺应这样的病人,在饮食衣服方面,应注意使他寒温适中。天寒时,多加衣服,不要着凉;天热时,要少穿,不要热得出汗。在饮食上不要过热过凉,应寒温合适。这样,真气就能内守,外邪就不能侵入体内。\\
黄帝说:《本脏》篇根据人体的外形与肢节刐肉情况,来测候五脏六腑的大小。现在见到王公大人和临朝即位的君主,如果他们问这个问题时,有谁敢在他们的身上抚摸探测,然后再作答复呢?\\
岐伯说:形体肢节,覆盖着脏腑,生理上与脏腑相通,因而五脏六腑之精气可以外显于形体肢节,故察其外,可知其内,而不只是依靠诊察面部。\\
黄帝说:五脏之精气显现于面部,从面部诊察五脏精气的方法,我已经懂得了,但根据肢节形体来了解内脏的变化,又是怎样的呢?\\
岐伯说:五脏六腑,肺的部位最高而称为“盖”,可从肩骨及咽喉的高突与陷下外形来测候。\\
黄帝说:对。\\
岐伯说:五脏六腑,以心为主宰,缺盆作为通路,肩骨两端距离较大,可以测候缺盆骨的部位,从而了解心脏的大小。\\
黄帝说:对。\\
岐伯说:肝的功能像将军,有勇有谋,有防御外侮的能力,要了解它坚固与否,可从眼睛的大小来测候。\\
黄帝说:对。\\
岐伯说:脾是主捍卫全身的,用它来接受水谷精微,运输周身,观察唇舌色泽及对食物的嗜好,可以测候脾病的吉凶。\\
黄帝说:对。\\
岐伯说:肾气通于耳而主外,用它远听声音,从听力的好坏,可以测候肾的功能。\\
黄帝说:对。请问怎么从外在形体以测候六腑情况呢?\\
岐伯说:六腑以胃为水谷之海,骸骨宽阔、颈围粗壮、胸廓舒张的人,容纳五谷就多;鼻窍的隧道长短,可以测候大肠的情况;唇厚,人中沟长,可以测候小肠的情况;下眼泡大,则胆姿横;鼻孔外翻,是膀胱不固而小便滴漏,鼻柱中央隆起,三焦固密。这就是测候六腑的方法。总之,外形的上、中、下三部相称,内脏一定安定而健康。\\
决气第三十\\
黄帝曰:余闻人有精、气、津、液、血、脉,余意以为一气耳,乃辨为六名,余不知其所以然。\\
岐伯曰:两神相搏,合而成形,常先身生,是谓精。\\
何谓气?\\
岐伯曰:上焦开发,宣五谷味,熏肤,充身、泽毛,若雾露之溉,是谓气。\\
何谓津?\\
岐伯曰:腠理发泄,汗出溱溱,是谓津。\\
何谓液?\\
岐伯曰:谷入气满,淖泽注于骨,骨属屈伸。泄泽,补益脑髓,皮肤润泽,是谓液。\\
何谓血?\\
岐伯曰:中焦受气取汁,变化而赤,是谓血。\\
何谓脉?\\
岐伯曰:雍遏营气,令无所避,是谓脉。\\
黄帝曰:六气者,有余不足,气之多少,脑髓之虚实,血脉之清浊,何以知之?\\
岐伯曰:精脱者,耳聋;气脱者,目不明;津脱者,腠理开,汗大泄;液脱者,骨属屈伸不利,色夭,脑髓消,胫酸,耳数鸣;血脱者,色白,夭然不泽;脉脱者,其脉空虚。此其候也。\\
黄帝曰:六气者,贵贱何如?\\
岐伯曰:六气者,各有部主也,其贵贱善恶,可为常主,然五谷与胃为大海也。\\
黄帝问:我听说人身有精、气、津、液、血、脉,我本来以为它是一气,现在却分为六种名称,我不知道为什么要这样分?\\
岐伯说:男女交媾,合和而结成新的形体,这种产生形体的物质在形体之先,叫精。\\
什么叫做气呢?\\
岐伯说:从上焦开发,发散五谷精微,温和皮肤,充实形体,润泽毛发,像雾露滋润草木一样,叫气。\\
什么叫津呢?\\
岐伯说:腠理发泄,出的汗很多,叫津。\\
什么叫做液呢?\\
岐伯说:谷物入胃,气充满全身,湿润的汁液渗到骨髓,使骨骼关节屈伸自如。渗出的部分,在内补益脑髓,在外润泽皮肤,叫液。\\
什么叫血呢?\\
岐伯说:中焦脾胃纳受食物,吸收汁液的精微,经过变化而成红色的液质,叫血。\\
什么叫脉呢?\\
岐伯说:像设堤防一样限制着气血,使它无所回避和妄行,叫脉。\\
黄帝问:六气在人体的有余不足,如精气的多少,津液的虚实,血脉的清浊,怎样才知道呢?\\
岐伯说:精虚的,会耳聋;气虚的,会目不明;津虚的,会腠理开,大量出汗;液虚的,会骨节屈伸不利,面色无华,脑髓不充,小腿发酸,常耳鸣;血虚的,肤色苍白,晦暗无光;脉虚的,脉象空虚无神。这就是六气有余不足的主要表现。\\
黄帝问道:六气的主次是怎样的呢?\\
岐伯说:六气各有它所主的脏器,其主次主要是从它们发挥的作用来划分的,但六气的来源都是以五谷和胃作为资生的源泉。\\
肠胃第三十一\\
黄帝问于伯高曰:余愿闻六腑传谷者,肠胃之小大长短,受谷之多少,奈何?\\
伯高曰:请尽言之。谷所从出入浅深远近长短之度:唇至齿长九分,口广二寸半。齿以后至会厌,深三寸半,大容五合。舌重十两,长七寸,广二寸半。咽门重十两,广一寸半,至胃长一尺六寸。胃纡曲屈,伸之,长二尺六寸,大一尺五寸,径五寸,大容三斗五升。小肠后附脊,左环回周迭积,其注于回肠者,外附于脐上,回运环十六曲,大二寸半,径八分分之少半,长三丈二尺。回肠当脐,左环,回周叶积而下,回运环反十六曲,大四寸,径一寸寸之少半,长二丈一尺。广肠傅脊,以受回肠,左环叶积,上下辟,大八寸,径二寸寸之大半,长二尺八寸。肠胃所入至所出,长六丈四寸四分,回曲环反,三十二曲也。\\
黄帝问伯高说:我希望听听消化谷物的六腑的状况,肠胃的大小长短,受纳水谷的多少,情况是怎样的?\\
伯高说:让我全部讲给您听。水谷的出入以及浅深、远近、长短的度数是:口唇到牙齿距离九分,两口角之间宽度是二寸半。牙齿向后到会厌部,深三寸半,大小能容水谷五合。舌重十两,长七寸,宽二寸半。咽门重十两,宽一寸半,由咽门到胃的长度是一尺六寸。胃的形态是纡屈曲折的,伸直长二尺六寸,周长一尺五寸,直径是五寸,大小能容水谷三斗五升。小肠在腹腔,后系附于脊柱之前,从左向右环行,而后周回重迭于腹内,下口注于回肠,在外侧附着于脐上,回行环转共有十六个弯曲,周长二寸半,直径八分又三分之一,长三丈二尺。回肠当脐处向左回环,迭积向下,回行环绕也有十六个弯曲,周长四寸,直径一寸又三分之一,共长二丈一尺。广肠附着于脊前,接受回肠所传下的糟粕,向左回环迭积在脊椎之前,由上向下而逐渐宽大,最宽处周长八寸,直径二寸又三分之二,长二尺八寸。肠胃运化水谷的过程,从口唇至肛门总长六丈零四寸四分,共有三十二个回环弯曲。\\
平人绝谷第三十二\\
黄帝曰:愿闻人之不食,七日而死,何也?\\
伯高曰:臣请言其故。胃大一尺五寸,径五寸,长二尺六寸,横屈受水谷三斗五升。其中之谷常留二斗,水一斗五升而满。上焦泄气,出其精微,慓悍滑疾,下焦下溉诸肠。小肠大二寸半,径八分分之少半,长三丈二尺,受谷二斗四升,水六升三合合之大半。回肠大四寸,径一寸寸之少半,长二丈一尺。受谷一斗,水七升半。广肠大八寸,径二寸寸之大半,长二尺八寸,受谷九升三合八分合之一。肠胃之长,凡五丈八尺四寸,受水谷九斗二升一合合之大半,此肠胃所受水谷之数也。\\
平人则不然,胃满则肠虚,肠满则胃虚。更虚更满,故气得上下,五脏安定,血脉和利,精神乃居。故神者,水谷之精气也。故肠胃之中,当留谷二斗,水一斗五升。故平人日再后,后二升半,一日中五升,七日五七三斗五升,而留水谷尽矣。故平人不食饮七日而死者,水谷精气津液皆尽故也。\\
黄帝说:希望听听人不饮食七天而死的道理?\\
伯高说:让我说明它的缘故。胃的周长一尺五寸,直径五寸,长二尺六寸,纡曲屈伸的容量,可以受纳水谷三斗五升,其中经常留着食物二斗,水液一斗五升,而充满胃中。通过上焦的宣发作用,输出食物的精微,随着慓悍滑疾之气营养全身,在下焦下面,起着清涤作用,泄于小肠。小肠大二寸半,直径八分又一分的三分之一,长三丈二尺,它的容量能受纳食物二斗四升,水液六升三合又一合的三分之二。回肠周长四寸,直径一寸又三分之一,长二丈一尺。它的容量能受纳食物一斗,水液七升半。广肠周长八寸,直径二寸又三分之二,长二尺八寸,受纳水谷的糟粕九升三合八分又一合的八分之一。肠胃的长度,总共五丈八尺四寸,可以受纳水谷九斗二升一合又一合的三分之二,这是肠胃装满水谷容量的总数。\\
平人就不这样,因为胃里充满食物,肠中是空的;肠中充满来自胃中的食物,胃里就已空虚。肠胃只有更虚更满,体内气机才能升降正常,五脏安定,血脉和利,精神安宁。所以说人的神气,主要由水谷精气所化生。因此肠胃里,经常存留谷物二斗,水液一斗五升。所以平人每天排便两次,每次排便二升半,一天里排便五升,七天五七三斗五升,所有留存于肠胃中的水谷就会竭尽的。所以平人不吃不喝七天而死,是因为水谷津液都已竭尽。\\
海论第三十三\\
黄帝问于岐伯曰:余闻刺法于夫子,夫子之所言,不离于营卫血气。夫十二经脉,内属于腑脏,外络于肢节,夫子乃合之于四海乎?\\
岐伯答曰:人亦有四海、十二经水。经水者,皆注于海,海有东西南北,命曰四海。\\
黄帝曰:以人应之奈何?\\
岐伯曰:人有髓海,有血海,有气海,有水谷之海,凡此四者,以应四海也。\\
黄帝曰:远乎哉!夫子之合人天地四海也。愿闻应之奈何?\\
岐伯答曰:必先明知阴阳表里荥输所在,四海定矣。\\
黄帝曰:定之奈何?\\
岐伯曰:胃者,水谷之海,其输上在气街,下至三里;冲脉者,为十二经之海,其输上在于大抒,下出于巨虚之上下廉;膻中者,为气之海,其输上在于柱骨之上下,前在于人迎;脑为髓之海,其输上在于其盖,下在风府。\\
黄帝曰:凡此四海者,何利何害?何生何败?\\
岐伯曰:得顺者生,得逆者败;知调者利,不知调者害。\\
黄帝曰:四海之逆顺奈何?\\
岐伯曰:气海有余者,气满胸中,悗息面赤;气海不足,则气少不足以言。血海有余,则常想其身大,怫然不知其所病;血海不足,亦常想其身小,狭然不知其所病。水谷之海有余,则腹满;水谷之海不足,则饥不受谷食。髓海有余,则轻劲多力,自过其度;髓海不足,则脑转耳鸣,胫痠眩冒,目无所见,懈怠安卧。\\
黄帝曰:余已闻逆顺,调之奈何?\\
岐伯曰:审守其输,而调其虚实,无犯其害。顺者得复,逆者必败。\\
黄帝曰:善。\\
黄帝问岐伯说:我听夫子您讲过刺法,您所讲的离不开营卫气血。十二经脉,在内连属于五脏六腑,在外网络于四肢关节,怎么把它和四海相配合呢?\\
岐伯回答说:人体也有四海、十二经水。十二经水的流行,都从四方会合注入大海,海有东西南北,所以叫四海。\\
黄帝问:人体怎样和四海相应呢?\\
岐伯说:人体有髓海,有血海,有气海,有水谷之海,以上四者,所以和四海相应。\\
黄帝说:讲得真深远啊!先生把人体和天地四海配合起来了。希望再听听它们是怎样相应的?\\
岐伯说:必先明确知道经脉的阴阳表里荥输的部位,就可以确定髓、血、气、水谷这四海了。\\
黄帝问:究竟是怎样确定呢?\\
岐伯说:胃是水谷之海,它的输注要穴,上在气冲,下在三里穴;冲脉是十二经之海,也就是血海,它的输注要穴,上在大杼,下在上巨虚和下巨虚穴;膻中是气海,它的输注要穴,在柱骨上的痖门、柱骨下的大椎,前在人迎穴;脑是髓海,它的输注要穴,上在百会,下在风府穴。\\
黄帝说:关于人身的四海,怎样会有益?怎样会有害?怎样会生机旺盛?怎样会衰退?\\
岐伯说:人身的四海顺乎生理规律的就生机旺盛,反之就会衰退;懂得调养四海的就有益于身体,否则就有害。\\
黄帝问:四海的逆顺情况怎样呢?\\
岐伯说:气海有余,是邪气盛,就会气满胸中,呼吸急促,面赤;气海不足,就会气短,说话无力。血海有余,因为血多脉盛,就会想象身体似大起来,虽然心情怫郁,而说不出病来;血海不足,就会经常感觉身体轻小,虽然心情不舒,也说不出病来。水谷之海有余,就会腹部胀满;水谷之海不足,就会觉得饥饿而不想吃东西。髓海有余,就会使身体轻劲多力,耐劳超过常度;髓海不足,就会脑似旋转,耳鸣,小腿发痠,眩晕,眼睛看不见东西,懈怠,嗜睡。\\
黄帝问:我已听到逆顺的情况,怎样调治呢?\\
岐伯说:精确掌握那些与四海相通的上下输穴,来调治,依据虚则补之,实则泻之的法则,不犯虚虚实实的错误。能这样做,病人就会安康;否则,病人就会衰败。\\
黄帝说:说得好。\\
五乱第三十四\\
黄帝曰:经脉十二者,别为五行,分为四时,何失而乱?何得而治?\\
岐伯曰:五行有序,四时有分,相顺则治,相逆则乱。\\
黄帝曰:何谓相顺?\\
岐伯曰:经脉十二者,以应十二月。十二月者,分为四时。四时者,春秋冬夏,其气各异。营卫相随,阴阳已和,清浊不相干,如是则顺之而治。\\
黄帝曰:何谓逆而乱?\\
岐伯曰:清气在阴,浊气在阳,营气顺脉,卫气逆行。清浊相干,乱于胸中,是谓大悗。故气乱于心,则烦心密嘿,俯首静伏,乱于肺,则俯仰喘喝,接手以呼;乱于肠胃,则为霍乱;乱于臂胫,则为四厥;乱于头,则为厥逆,头重眩仆。\\
黄帝曰:五乱者,刺之有道乎?\\
岐伯曰:有道以来,有道以去,审知其道,是谓身宝。\\
黄帝曰:善。愿闻其道。\\
岐伯曰:气在于心者,取之手少阴心主之输。气在于肺者,取之手太阴荥、足少阴输。气在于肠胃者,取之足太阴,阳明;不下者,取之三里。气在于头者,取之天柱、大杼,不知,取足太阳荥输。气在于臂足,取之先去血脉,后取其阳明,少阳之荥输。\\
黄帝曰:补泻奈何?\\
岐伯曰:徐入徐出,谓之导气。补泻无形,谓之同精。是非有余不足也,乱气之相逆也。\\
黄帝曰:允乎哉道!明乎哉论!请著之玉版,命曰治乱也。\\
黄帝问:人身十二经脉,分别与五行配合,又分别属于四时,因为什么失调而引起气行的逆乱?又是依靠什么保证脉气的正常运行?\\
岐伯说:五行的交替有一定秩序,四时气候的变化有季节的分别,经脉运行与四时五行的规律相适应,就能保持正常的活动,违反了这个规律,就会运行逆乱。\\
黄帝问:什么叫相顺呢?\\
岐伯说:十二经脉和十二个月相应。十二个月分为四时。四时就是春、夏、秋、冬,其气候各不相同。人体营气与卫气是内外相随,阴阳调和的,清气与浊气不互相影响,这样就能顺应四时而保持健康。\\
黄帝问:什么叫相逆而乱呢?\\
岐伯说:属营的清气本在阴分,属浊的卫气本在阳分,营气在脉内与脉顺行,卫气在脉外与脉逆行。如果清浊之气受邪气干扰而乱于胸中,就叫“大悗”。所以气乱于心,可见心中烦扰,沉默不言,低头静伏;气乱于肺,可见俯仰不安,喘息喝喝有声,两手按在胸前而呼吸;气乱于肠胃,则发生霍乱;气乱于手臂与足胫,可见四肢厥冷;气乱于头,可见厥气上逆,头重眩晕,甚则跌倒。\\
黄帝问:这五种乱证,针刺有原则吗?\\
岐伯说:营卫之气运行,都有一定的往来规律,能掌握这种往来规律,实在是养生的要领。\\
黄帝道:对。希望听听针刺原则。\\
岐伯说:气乱于心,取手少阴心经与手厥阴心包络经的输穴神门、大陵。气乱于肺,取手太阴经的荥穴鱼际和足少阴经的输穴太溪。气乱于肠胃,取足太阴、足阳明的经穴太白、陷谷;如果无效的,可取足三里穴。气乱于头,取天柱、大杼二穴;如果病仍不见轻,再取足太阳经的荥穴通谷与输穴束骨。气乱于手臂与足胫,应先刺淤结不通的血脉,然后再取阳明、少阳两经的荥穴与输穴。\\
黄帝问:补泻的手法如何呢?\\
岐伯说:慢慢地进针,慢慢地出针,这叫“导气”。补泻手法轻巧无形,目的在于调和精气。这些病证,并不是有余的实证和不足的虚证,而仅是气机一时逆乱所致。\\
黄帝问:道理讲得很恰当!论证也很明白!让我把它刻在珍贵的玉版上,命名为“治乱”。\\
胀论第三十五\\
黄帝曰:脉之应于寸口,如何而胀?\\
岐伯曰:其脉大坚以涩者,胀也。\\
黄帝曰:何以知脏腑之胀也?\\
岐伯曰:阴为脏,阳为腑。\\
黄帝曰:夫气之令人胀也,在于血脉之中耶,脏腑之内乎?\\
岐伯曰:三者皆存焉,然非胀之舍也。\\
黄帝曰:愿闻胀之舍。\\
岐伯曰:夫胀者,皆在于脏腑之外,排脏腑而郭胸胁,胀皮肤,故命曰胀。\\
黄帝曰:脏腑之在胸胁腹里之内也,若匣匮之藏禁器也,各有次舍,异名而同处,一域之中,其气各异,愿闻其故。\\
岐伯曰:夫胸腹,脏腑之郭也。膻中者,心主之宫城也。胃者,太仓也。咽喉小肠者,传送也。胃之五窍者,闾里门户也。廉泉、玉英者,津液之道也。故五脏六腑者,各有畔界,其病各有形状。营气循脉,卫气逆,为脉胀;卫气并脉循分,为肤胀。三里而泻,近者一下,远者三下,无问虚实,工在疾泻。\\
黄帝曰:愿闻胀形。\\
岐伯曰:夫心胀者,烦心短气,卧不安。肺胀者,虚满而喘咳。肝胀者,胁下满而痛引小腹。脾胀者,善哕,四肢烦悗,体重不能胜衣,卧不安。肾胀者,腹满引背央央然,腰髀痛。\\
六腑胀:胃胀者,腹满,胃脘痛,鼻闻焦臭,妨于食,大便难。大肠胀者,肠鸣而痛濯濯,冬日重感于寒,则飧泄不化。小肠胀者,少腹尒胀,引腰而痛。膀胱胀者,少腹满而气癃。三焦胀者,气满于皮肤中,轻轻然而不坚。胆胀者,胁下痛胀,口中苦,善太息。凡此诸胀者,其道在一,明知逆顺,针数不失。泻虚补实,神去其室,致邪失正,真不可定,粗之所败,谓之夭命。补虚泻实,神归其室,久塞其空,谓之良工。\\
黄帝曰:胀者焉生?何因而有?\\
岐伯曰:卫气之在身也,常并脉循分肉,行有逆顺,阴阳相随,乃得天和,五脏更始,四时循序,五谷乃化。然后厥气在下,营卫留止,寒气逆上,真邪相攻,两气相搏,乃合为胀也。\\
黄帝曰:善。何以解惑?\\
岐伯曰:合之于真,三合而得。\\
帝曰:善。\\
黄帝问于岐伯曰:《胀论》言无问虚实,工在疾泻,近者一下,远者三下。今有其三而不下者,其过焉在?\\
岐伯对曰:此言陷于肉肓,而中气穴者也。不中气穴,则气内闭,针不陷肓,则气不行;上越中肉,则卫气相乱,阴阳相逐。其于胀也,当泻不泻,气故不下。三而不下,必更其道,气下乃止。不下复始,可以万全,乌有殆者乎?其于胀也,必审其诊,当泻则泻,当补则补,如鼓应桴,恶有不下者乎?\\
黄帝问:脉象反应在寸口,什么样的脉是胀病?\\
岐伯说:脉象大而坚劲兼带涩滞的,就是胀病。\\
黄帝问:如何鉴别脏胀与腑胀呢?\\
岐伯说:病在阴分属于脏胀,病在阳分属于腑胀。\\
黄帝问:气机运行不畅使人发胀,其病所是在血脉之中呢,还是在脏腑之内?\\
岐伯说:胀病与血脉、脏、腑三者都有关系,但都不是胀的病所。\\
黄帝道:希望听听胀的病所。\\
岐伯说:胀病,都是在脏腑之外,排挤脏腑而充斥胸胁,使皮肤胀满,所以叫“胀”。\\
黄帝说:五脏六腑居于胸胁和腹腔之内,就像禁物藏在柜匣中一样,各有固定位置,不同名称的脏器,虽然同处在一个区域之中,但功能各不相同,希望听听其中的道理。\\
岐伯说:胸腹好比是五脏六腑的外郭。胸中好比是心脏君主的宫城。胃容纳食物好比是仓库。咽喉至小肠是传送饮食物的道路。咽门、贲门、幽门、阑门、魄门五窍,是胃肠道的门户。廉泉至玉英,是津液运行的道路。所以五脏六腑各有界限,发病后也各有不同的症状。营气本来循行脉中,如果卫气逆行于脉中则为脉胀;如果卫气并经脉同行于分肉之间,则成为肤胀。治疗时取足三里穴,用泻法,病邪近而轻的可刺泻一次,病邪远而重的可刺泻三次,不论虚证或实证,关键在于急用泻法以祛其邪。\\
黄帝道:希望听听胀病的各种症状。\\
岐伯说:心胀的症状,心中烦乱,气息短促,睡眠不安。肺胀的症状,胸中气胀而虚满,气喘咳嗽。肝胀的症状,胁下胀满,疼痛牵引小腹。脾胀的症状,呃逆,四肢烦扰闷胀,身体沉重而不能胜衣,睡眠不安。肾胀的症状,腹中胀满,连及背部不舒,腰髀部疼痛。\\
六腑胀:胃胀的症状,腹部胀满,胃脘疼痛,鼻中如闻焦臭,妨碍饮食,大便困难。大肠胀的症状,肠鸣疼痛,濯濯有声,如在冬天再感受寒邪,就会发生飧泻而完谷不化。小肠胀的症状,少腹妅胀,引及腰部作痛。膀胱胀的症状,少腹胀满,小便不利。三焦胀的症状,气充满于皮肤而肿起,用手按之浮而不实。胆胀的症状,胁下胀痛,口苦,经常叹息。以上各种胀病,其治疗的道理都是一样的,只要懂得气血运行的逆顺,针治次数就不会失误。如果虚证而用泻法,实证而用补法,神气离其所藏之处,引致邪气内入,正气散失,真气不能安定,这是因为粗工误治败坏而成,这叫折人寿命。如果虚证用补法,实证用泻法,神气能归藏其所,再逐步地补益其不足,这才叫做良工。\\
黄帝问:胀病是从何而生?因为什么原因有了胀的名称?\\
岐伯说:卫气在身体运行,经常与经脉相并循行于分肉之间,运行时虽有上下逆顺的不同,但总是阴阳内外相随相和,这样才是天然的和谐,五脏之气更相主时,四时气候遵循一定次序转移,这样五谷才能化生精微。如果病气从下上逆,导致营卫之气运行不畅而稽留,加之寒气上逆,真气与邪气相互攻击,两气互相搏结,纠合起来就成为胀病。\\
黄帝说:对。疑惑的问题怎样才能解决呢?\\
岐伯说:结合人体的真气,从血脉、脏、腑三方面反映的症状,互相参照就可知道了。\\
黄帝道:对。\\
黄帝问岐伯说:以上《胀论》所说胀病的治疗,不论虚实,关键在于急用泻法,病邪近而轻的一次,病邪远而重的三次。现有已刺三次而胀病仍未痊愈,失误在哪里呢?\\
岐伯答道:这是说针治时要将针刺到分肉的空隙之间,中于气穴之内。如果刺不中气穴,则病气仍然郁阻于内,针不到分肉空隙,则经气仍然不能运行;仅刺入皮肤而未陷肓,或刺不中气穴,误中分肉之间,就会扰乱卫气的正常活动而乱行,阴阳之气互相争逐而不能相随。对于胀病,当泻而不泻,以致病气不能下泄。如果已刺三次而病气还未下泄,必须更换刺治的穴位,直到病气下泄为止。如果病气还不下泄,还应从头开始刺治,这样一定能够治愈的,哪里会有危险呢?对于胀病,必须仔细诊察其脉象,当泻则泻,当补则补,就会像桴鼓相应一样,病邪哪里还有不去的道理呢?\\
五癃津液别第三十六\\
黄帝问于岐伯曰:水谷入于口,输于肠胃,其液别为五,天寒衣薄则为溺与气,天热衣厚则为汗,悲哀气并则为泣,中热胃缓则为唾,邪气内逆,则气为之闭塞而不行,不行则为水胀。余知其然也,不知其何由生?愿闻其道。\\
岐伯曰:水谷皆入于口,其味有五,各注其海,津液各走其道。故三焦出气,以温肌肉,充皮肤,为其津;其留而不行者,为液。\\
天暑衣厚则腠理开,故汗出;寒留于分肉之间,聚沫则为痛。天寒则腠理闭,气湿不行,水下流于膀胱,则为溺与气。\\
五脏六腑,心为之主,耳为之听,目为之候,肺为之相,肝为之将,脾为之卫,肾为之外。故五脏六腑之津液,尽上渗于目。心悲气并则心系急,心系急则肺举,肺举则液上溢。夫心系与肺,不能常举,乍上乍下,故咳而泣出矣。\\
中热则胃中消谷,消谷则虫上下作,肠胃充郭故胃缓,胃缓则气逆,故唾出。\\
五谷之津液和合而为膏者,内渗入于骨空,补益脑髓,而下流于阴股。阴阳不和,则使液溢而下流于阴,髓液皆减而下,下过度则虚,虚故腰背痛而胫痠。\\
阴阳气道不通,四海闭塞,三焦不泻,津液不化,水谷并行肠胃之中,别于回肠,留于下焦,不得渗膀胱,则下焦胀,水溢则为水胀。此津液五别之逆顺也。\\
黄帝问岐伯说:水谷从口进入,输送到肠胃,它化生的津液分别为五种,当天气寒冷时,或衣服穿得薄时,就变为小便与哈气;当天气炎热时,或衣服穿得厚时,就成为汗液;遇悲伤哀痛时,气合于心,则为眼泪;当中焦有热,胃功能弛缓时,就化为唾液;当邪气内阻,阳气闭塞不行,阳气不行则水液潴留而为水胀。我知道这些情况,但不知道是怎样发生的,希望听听其中的道理。\\
岐伯说:水谷都从口进入人体,它有五种味道,分别注入其所喜的五脏,津液亦随其所喜而各走其道。所以从三焦输出其气,来温养肌肉,充实皮肤,这就叫做“津”;其留而不行的叫做“液”。\\
天气炎热,衣服穿得过厚,则腠理开张,所以出汗;如果寒气停留于分肉之间,使津液凝聚为沫汁则发生疼痛。天气寒冷时腠理闭密,气湿不能从汗孔排泄,向下流于膀胱,就为小便与气。\\
五脏六腑,以心为主宰,耳主远听,眼主瞭望,肺像宰相,肝像将军,脾像护卫,肾脏主骨而支撑形体。所以五脏六腑的津液,都向上渗灌到眼睛。当心中悲哀,气并于心时,心系就会引急,心系引急则肺叶上举,肺叶上举使津液向上溢出。但心系与肺叶不能经常上举,气行忽上忽下,所以咳嗽与流泪。\\
中焦有热,胃中容易消化食物,容易消化食物则肠中寄生虫上下活动,水谷与寄生虫使肠胃胀满,则胃的活动弛缓,胃弛缓则气上逆,所以唾液随之而出。\\
五谷的津液,和合而成为脂膏,向内渗灌到骨孔,向上补益脑髓,向下流注到生殖器。如果阴阳不调和,则使液从阴窍流泄,髓液也同时减少,流泄过度使真阴虚损,真阴虚损则出现腰背疼痛、足胫酸软。\\
如果阴阳气道不通,则四海闭塞,三焦不能输泻,津液不能化生,水谷聚集在肠胃之中,最后从回肠别出,停留在下焦,不能将水分渗入膀胱,则下焦胀满,水液充溢而为水胀。这就是津液分别为五的正常与反常情况。\\
五阅五使第三十七\\
黄帝问于岐伯曰:余闻刺有五官五阅,以观五气。五气者,五脏之使也,五时之副也。愿闻其五使当安出?\\
岐伯曰:五官者,五脏之阅也。\\
黄帝曰:愿闻其所出,令可为常。\\
岐伯曰:脉出于气口,色见于明堂。五色更出,以应五时,各如其常。经气入脏,必当治理。\\
帝曰:善。五色独决于明堂乎?\\
岐伯曰:五官已辨,阙庭必张,乃立明堂。明堂广大,蕃蔽见外,方壁高基,引垂居外。五色乃治,平博广大,寿中百岁。见此者,刺之必已,如是之人者,血气有余,肌肉坚致,故可苦以针。\\
黄帝曰:愿闻五官。\\
岐伯曰:鼻者,肺之官也;目者,肝之官也;口唇者,脾之官也;舌者,心之官也;耳者,肾之官也。\\
黄帝曰:以官何候?\\
岐伯曰:以候五脏。故肺病者,喘息鼻张;肝病者,眦青;脾病者,唇黄;心病者,舌卷短,颧赤;肾病者,颧与颜黑。\\
黄帝曰:五脉安出,五色安见,其常色殆者如何?\\
岐伯曰:五官不辨,阙庭不张,小其明堂,蕃蔽不见,又埤其墙,墙下无基,垂角去外。如是者,虽平常殆,况加疾哉。\\
黄帝曰:五色之见于明堂,以观五脏之气,左右高下,各有形乎?\\
岐伯曰:腑脏之在中也,各以次舍,左右上下,各如其度也。\\
黄帝问岐伯说:我听说刺法里有通过五官反映的五脏之气的外在表现的“五阅”,来观察五种气色。五气是五脏功能的外在表现,是与五时相配合的。希望听听五使从哪里反映出来呢?\\
岐伯说:五官是五脏的外候。\\
黄帝说:希望听到五官表现五脏变化的情况,以作为察病的常规。\\
岐伯说:五脏的脉色可从气口反映出来,气色可从鼻部反映出来。五色交替出现,和五时相应,各如常规。如果邪气从经脉传入内脏,就要治内。\\
黄帝说:说得好。观察五色仅是取决于鼻吗?\\
岐伯说:五官之色,已经分明,天庭的部位必定明显,才可决定明堂的测候。明堂广大,颊侧和耳门部位显露于外,面部方正、丰厚,齿龈的本肉在外护着牙齿。五色正常,五官的位置平正开阔,这样的人,其寿命应活到百岁。见到这样的人,针刺一定能治好病。因为这样的人,血气有余,肌肉坚实,可以急用针刺治疗。\\
黄帝说:希望听听五官的职能。\\
岐伯说:鼻是肺之官;目是肝之官;口唇是脾之官;舌是心之官;耳是肾之官。\\
黄帝问:从五官诊察什么呢?\\
岐伯说:可以诊察五脏。所以肺脏有了病,可见喘息急促,鼻孔扇动;肝脏有了病,可见眼角发青;脾脏有了病,可见口唇发黄;心脏有了病,可见舌短,两颧发红;肾脏有了病,可见两颧和额部色黑。\\
黄帝问:有人五脉正常,五色也正常,其气色与常人一样,而一旦有病就危险极了,这是什么道理呢?\\
岐伯说:五官分野不清,天庭不开阔,鼻子很小,颊侧和耳门瘦削不饱满,耳周及耳下的肌肉不厚,耳垂和下颏像削去了一部分。这种人,虽在平时无病,但已有短寿的征象,何况再加上疾病呢?\\
黄帝问:五色表现在鼻部,可以观察五脏之气,其中左右高下,各有一定形象吗?\\
岐伯说:五脏在胸腹腔之内,各有位置,它反映在面部的五色,左右上下也各有常度。\\
逆顺肥瘦第三十八\\
黄帝问于岐伯曰:余闻针道于夫子,众多毕悉矣。夫子之道应若失,而据未有坚然者也。夫子之问学熟乎,将审察于物而心生之乎?\\
岐伯曰:圣人之为道者,上合于天,下合于地,中合于人事。必有明法,以起度数、法式检押,乃后可传焉。故匠人不能释尺寸而意短长,废绳墨而起平木也;工人不能置规而为圆,去矩而为方。知用此者,固自然之物,易用之教,逆顺之常也。\\
黄帝曰:愿闻自然奈何。\\
岐伯曰:临深决水,不用功力,而水可竭也;循掘决冲,而经可通也。此言气之滑涩,血之清浊,行之逆顺也。\\
黄帝曰:愿闻人之白黑肥瘦少长,各有数乎?\\
岐伯曰:年质壮大,血气充盈,肤革坚固,因加以邪。刺此者,深而留之,此肥人也。广肩腋项,肉薄厚皮而黑色,唇临临然,其血黑以浊,其气涩以迟。其为人也,贪于取与。刺此者,深而留之,多益其数也。\\
黄帝曰:刺瘦人奈何?\\
岐伯曰:瘦人者,皮薄色少,肉廉廉然,薄唇轻言。其血清气滑,易脱于气,易损于血。刺此者,浅而疾之。\\
黄帝曰:刺常人奈何?\\
岐伯曰:视其白黑,各为调之。其端正敦厚者,其血气和调,刺此者,无失常数也。\\
黄帝曰:刺壮士真骨者奈何?\\
岐伯曰:刺壮士真骨,坚肉缓节监监然。此人重则气涩血浊,刺此者,深而留之,多益其数。劲则气滑血清,刺此者,浅而疾之。\\
黄帝曰:刺婴儿奈何?\\
岐伯曰:婴儿者,其肉脆血少气弱,刺此者,以毫针,浅刺而疾发针,日再可也。\\
黄帝曰:临深决水,奈何?\\
岐伯曰:血清气滑,疾泻之,则气竭焉。\\
黄帝曰:循掘决冲,奈何?\\
岐伯曰:血浊气涩,疾泻之,则经可通也。\\
黄帝曰:脉行之逆顺,奈何?\\
岐伯曰:手之三阴,从脏走手;手之三阳,从手走头;足之三阳,从头走足;足之三阴,从足走腹。\\
黄帝曰:少阴之脉独下行,何也?\\
岐伯曰:不然。夫冲脉者,五脏六腑之海也,五脏六腑皆禀焉。其上者,出于颃颡,渗诸阳,灌诸精;其下者,注少阴之大络,出于气街,循阴股内廉,入腘中,伏行骭骨内,下至内踝之后属而别;其下者,并于少阴之经,渗三阴;其前者,伏行出跗属,下循跗入大指间,渗诸络而温肌肉。故别络结则跗上不动,不动则厥,厥则寒矣。\\
黄帝曰:何以明之?\\
岐伯曰:以言导之,切而验之,其非必动,然后乃可明逆顺之行也。\\
黄帝曰:窘乎哉!圣人之为道也,明于日月,微于毫厘,其非夫子,孰能道之也。\\
黄帝问岐伯说:我听夫子讲针道,知道很多了。根据夫子的理论针刺,常常手到病除,从没有坚不可除的病证。先生是向前辈的先生询问继承的呢?还是从审察事物中而发明的呢?\\
岐伯说:圣人所作针刺的道理,对上合于天文,对下合于地理,对中合于社会人事。一定有明确的法则,以立尺度长短,模式规矩,然后才可传于后世。所以匠人不能丢掉尺寸而妄揣短长,放弃绳墨而求平直;工人不能丢开规而去画圆,去了矩而去画方。知道运用这一法则的,是顺应了自然的物理,是便于应用的教法,也就是衡量逆顺的常规。\\
黄帝问:希望听听自然之道是怎样的。\\
岐伯说:到深河那里放水,不用多大功力,就可以把水放完;从洞穴里开地道,则直行的大道很容易通开。这是说人身的气有滑有涩,血有清有浊,气血的运行有逆有顺。治疗时应该顺应其自然。\\
黄帝说:我希望听听人的白黑肥瘦少长,在针刺时,是否有不同呢?\\
岐伯说:壮年而体质强壮的人,血气充足旺盛,皮肤坚密,在感受病邪时,针刺这种人,应该深刺、留针,这是刺肥壮人的标准。另有一种人,肩腋很开阔,颈项肉薄、皮厚、色黑,唇厚,血色黑浊,气行涩迟。这种人,贪图便宜,追求利益。针刺是应该深刺,留针,多增加针刺的次数。\\
黄帝问:针刺瘦人用什么针法呢?\\
岐伯说:瘦人皮薄颜色淡,肌肉消瘦,唇薄,语声低。他的血清稀而气滑利,像这样,气、血都容易虚脱、损耗。针刺时应该浅刺、急速出针。\\
黄帝问:针刺普通人用什么针法呢?\\
岐伯说:观察他的肤色白黑,分别配合针刺深浅的标准。属于端正纯厚的人,它的血气和调,针刺时依据正常的针法标准。\\
黄帝问:针刺壮士用什么针法呢?\\
岐伯说:壮士骨骼坚固,肌肉丰厚,关节坚大。这样的人,性情稳重的,气涩血浊,针刺就当深刺、留针,并且增加针刺次数。而性情好动的,气滑血清,针刺就当浅刺而急速出针。\\
黄帝问:针刺婴儿用什么针法呢?\\
岐伯说:婴儿肉软、血少、气弱,针刺时用毫针,浅刺进针要快,一天针刺两次就够了。\\
黄帝问:临深决水,运用于针刺上是怎样的?\\
岐伯说:血清气滑的人,用疾泻的针法,就会使真气衰竭。\\
黄帝问:循掘决冲,运用于针刺上是怎样的?\\
岐伯答说:血浊气涩的人,用疾泻的针法,会使真气通畅。\\
黄帝问:经脉循行的逆顺情况怎样?\\
岐伯说:手三阴经脉,从内脏走向手部;手三阳经脉,从手部走向头部;足三阳经脉,从头部走向足部;足三阴经脉,从足部走向腹部。\\
黄帝问:只有足少阴经脉下行,为什么?\\
岐伯说:不是这样的。冲脉是五脏六腑气血汇聚之处,五脏六腑都从它那禀受气血以濡养。它上行的部分,出于鼻道上窍,渗入阳经,灌注精气;下行的部分,输注到足少阴经的大络,从气街出行,沿大腿内侧,下入膝腘窝中,伏行于胫骨之内,再下至内踝后跟骨上缘而别行;下行的又一支脉,与足少阴经相并而行,渗入三阴经;行于其前面的,从内踝后的深部出于跟骨结节上缘,下沿足背走入足大趾内,渗入诸络脉而温养肌肉。所以该脉的别络淤结时,在足背上的脉就不跳动,不跳动则经气厥逆,经气厥逆而下肢寒冷。\\
黄帝问:用什么方法查明冲脉和少阴经气的逆顺呢?\\
岐伯说:开导病人,问明症状,用手切足背动脉,验其是否跳动,如果不是厥逆,必定有脉跳动,然后就可辨明经脉循行的逆顺情况。\\
黄帝说:这个问题真重要呀!圣人所作的针道,比日月还光明,比毫厘还细微,若不是夫子,有谁能讲明白呢!\\
血络论第三十九\\
黄帝曰:愿闻奇邪而不在经者。\\
岐伯曰:血络是也。\\
黄帝曰:刺血络而仆者,何也?血出而射者,何也?血出黑而浊者,何也?血出清而半为汁者,何也?发针而肿者,何也?血出若多若少而面色苍苍者,何也?发针而面色不变而烦悗者,何也?多出血而不动摇者,何也?愿闻其故。\\
岐伯曰:脉气盛而血虚者,刺之则脱气,脱气则仆。血气俱盛而阴气多者,其血滑,刺之则射。阳气畜积,久留而不泻者,其血黑以浊,故不能射。新饮而液渗于络,而未合和于血也,故血出而汁别焉。其不新饮者,身中有水,久则为肿。阴气积于阳,其气因于络,故刺之血未出而气先行,故肿。阴阳之气其新相得而未和合,因而泻之,则阴阳俱脱,表里相离,故脱色而苍苍然。刺之血出多,色不变而烦悗者,刺络而虚经;虚经之属于阴者,阴脱,故烦悗。阴阳相得而合为痹者,此为内溢于经,外注于络,如是者,阴阳俱有余,虽多出血而弗能虚也。\\
黄帝曰:相之奈何?\\
岐伯曰:血脉者,盛坚横以赤,上下无常处,小者如针,大者如筋,刺而泻之,万全也。故无失数矣,失数而反,各如其度。\\
黄帝曰:针入而肉著者,何也?\\
岐伯曰:热气因于针则针热,热则肉著于针,故坚焉。\\
黄帝说:希望听听由奇邪所致,但不在经脉中的病变的情况。\\
岐伯说:这是一种在络脉之中的病变。\\
黄帝说:刺血络放血,病人就昏倒了,是什么原因?刺后血向外射出,是什么原因?血出色黑稠浊的,是什么原因?血出清稀,一半是汁液,是什么原因?出针后而皮肤肿起,是什么原因?血出或多或少而面色苍白,是什么原因?出针后面色虽然不变而心中烦闷,是什么原因?血出虽多而不觉痛苦,是什么原因?希望听听上述情况的道理。\\
岐伯说:脉中气偏盛而血偏虚,针刺时容易脱气,气脱就会仆倒。血气虽然都盛而阴气更多的,血行滑疾,针刺时血向外喷射。阳气积蓄于络脉之内,停留已久,不能外泄,致使血色变黑而稠浊,所以不能远射。刚刚喝了汤水,水液渗入络脉,还没有和血液融和,所以刺出的血中有水液夹杂。如果不是刚饮汤水,而病者体内原有水液停留,久之会形成水肿。阴气积蓄在阳分,已经渗入到络脉,因此当刺治时,血还未出而气已先行,故使局部肿起。阴气与阳气刚刚接触,还没有融合调和,此时用针刺行泻法,阴阳两气同时脱失,表里分离,以致色脱而面色苍白。刺时血出过多,虽然面色没有变化,而胸中烦闷,是因为刺络血出过多,使经脉也空虚了;若属于阴经空虚,就会因阴脱而产生烦闷。阴分阳分邪气相结合而形成痹证,是因为在内泛滥于经脉,在外渗注于络脉,这样,阴分与阳分都是邪气有余,虽然多出一些血,也不会致虚。\\
黄帝问:怎样观察呢?\\
岐伯说:血脉盛的,坚硬充满而皮下发红,上下没有固定的部位,小的像针,大的像筋,针刺用泻法,是非常安全的。所以不能违背刺络的原则,违背原则,非但无效,反会加重病情。\\
黄帝问:针入后,常见肌肉胶着针身,这是什么原因?\\
岐伯说:人体热气使针身发热,热则致肌肉胶着针体,所以坚涩难以转动。\\
阴阳清浊第四十\\
黄帝曰:余闻十二经脉,以应十二经水者。其五色各异,清浊不同,人之血气若之,应之奈何?\\
岐伯曰:人之血气,苟能若一,则天下为一矣,恶有乱者乎。\\
黄帝曰:余问一人,非问天下之众。\\
岐伯曰:夫一人者,亦有乱气,天下之众,亦有乱人,其合为一耳。\\
黄帝曰:愿闻人气之清浊。\\
岐伯曰:受谷者浊,受气者清。清者注阴,浊者注阳。浊而清者,上出于咽;清而浊者,则下行。清浊相干,命曰乱气。\\
黄帝曰:夫阴清而阳浊,浊者有清,清者有浊,清浊别之奈何?\\
岐伯曰:气之大别,清者上注于肺,浊者下走于胃。胃之清气,上出于口,肺之浊气,下注于经,内积于海。\\
黄帝曰:诸阳皆浊,何阳浊甚乎?\\
岐伯曰:手太阳独受阳之浊,手太阴独受阴之清。其清者上走空窍,其浊者下行诸经。诸阳皆清,足太阴独受其浊。\\
黄帝曰:治之奈何?\\
岐伯曰:清者其气滑,浊者其气涩,此气之常也。故刺阴者,深而留之;刺阳者,浅而疾之;清浊相干者,以数调之也。\\
黄帝问:我听说人体的十二经脉,和地上的十二经水相应。那十二经水五色不同,清浊也不同,而人体的血气如一,说它和十二经水相应,是怎么回事呢?\\
岐伯说:人体的血气,如果能够如一,那么,天下的一切,就都可以为一,怎么会发生混乱呢?\\
黄帝说:我问的是一个人的经脉血气,不是问天下众人的事情。\\
岐伯说:在一个人身体内有乱气,天下的众人,也有乱气,道理是一个。\\
黄帝说:我希望听听人体内的清气和浊气。\\
岐伯说:人吃的谷物是浊气,吸的空气是清气。清气注入阴,浊气注入阳。由水谷浊气化生的清气,上出于咽喉;在清气内的浊气则下行。若清浊升降失常,互相干扰,就叫乱气。\\
黄帝问:阴清阳浊,浊中有清气,清中有浊气,清气、浊气怎样区别呢?\\
岐伯说:气的大致区别是,清气向上注入肺脏,浊气向下流入胃腑。胃中化生的清气,上出于口;肺中所含的浊气,向下注入经脉,在内积聚在气海中。\\
黄帝问:诸阳经都是浊气所在,哪个阳腑浊气最多呢?\\
岐伯说:手太阳小肠接受的浊气最多,手太阴肺接受的清气最多。清气上走于孔窍,浊气下行于各经脉。五脏受纳的都是清气,只有足太阴脾接受胃中之浊气。\\
黄帝问道:清浊之气,应怎样调治呢?\\
岐伯说:清气滑利,浊气涩滞,这是气的正常情况。因此,针刺阴脏的病,深刺而留针;针刺阳腑的病,浅刺而快出针;如果清浊之气互相干扰,根据情况,进行调治。\\
卷七\\
阴阳系日月第四十一\\
黄帝曰:余闻天为阳,地为阴,日为阳,月为阴,其合之于人,奈何?\\
岐伯曰:腰以上为天,腰以下为地,故天为阳,地为阴。故足之十二经脉,以应十二月,月生于水,故在下者为阴;手之十指,以应十日,日主火,故在上者为阳。\\
黄帝曰:合之于脉,奈何?\\
岐伯曰:寅者,正月之生阳也,主左足之少阳;未者,六月,主右足之少阳;卯者,二月,主左足之太阳;午者,五月,主右足之太阳。辰者,三月,主左足之阳明;巳者,四月,主右足之阳明。此两阳合明,故曰阳明。申者,七月之生阴也,主右足之少阴;丑者,十二月,主左足之少阴;酉者,八月,主右足之太阴;子者,十一月,主左足之太阴;戌者,九月,主右足之厥阴;亥者,十月,主左足之厥阴。此两阴交尽,故曰厥阴。\\
甲主左手之少阳,己主右手之少阳。乙主左手之太阳,戊主右手之太阳。丙主左手之阳明,丁主右手之阳明。此两火并合,故为阳明。庚主右手之少阴,癸主左手之少阴。辛主右手之太阴,壬主左手之太阴。故足之阳者,阴中之少阳也;足之阴者,阴中之太阴也。手之阳者,阳中之太阳也;手之阴者,阳中之少阴也。腰以上者为阳,腰以下者为阴。\\
其于五藏也,心为阳中之太阳,肺为阳中之少阴,肝为阴中之少阳,脾为阴中之至阴,肾为阴中之太阴。\\
黄帝曰:以治之,奈何?\\
岐伯曰:正月、二月、三月,人气在左,无刺左足之阳;四月、五月、六月,人气在右,无刺右足之阳;七月、八月、九月,人气在右,无刺右足之阴;十月、十一月、十二月,人气在左,无刺左足之阴。\\
黄帝曰:五行以东方为甲乙木王春,春者,苍色,主肝。肝者,足厥阴也。今乃以甲为左手之少阳,不合于数,何也?\\
岐伯曰:此天地之阴阳也,非四时五行之以次行也。且夫阴阳者,有名而无形,故数之可十,离之可百,散之可千,推之可万,此之谓也。\\
黄帝问:我听说天属阳,地属阴,日属阳,月属阴,它们与人体是怎样相应合的?\\
岐伯说:人体的腰以上属天,腰以下属地,所以腰以上属天为阳,腰以下属地为阴。所以在下的足部的十二条经脉,与十二个月份相应。因为月生于水,属阴,所以在下的属阴;在上的手十指,与十日相应,日主于火,属阳,所以在上的属阳。\\
黄帝问:十二月、十日与经脉相配合是怎样的?\\
岐伯说:正月配寅,称为正月建寅,阳气初生的时候,主左足的少阳经;六月建未,主右足的少阳经;二月建卯,主左足的太阳经;五月建午,主右足的太阳经;三月建辰,主左足的阳明经;四月建巳,主右足的阳明经。因三四两个月夹在两阳的中间,而为两阳合明,所以叫做阳明。七月建申,是阴气渐生之时,主右足的少阴经;十二月建丑,主左足的少阴经;八月建酉,主右足的太阴经;十一月建子,主左足的太阴经;九月建戌,主右足的厥阴经;十月建亥,主左足的厥阴经。因九十两个月夹在两阴的中间,为阴气交会的时间,所以称为厥阴。\\
甲日主左手的少阳经,己日主右手的少阳经。乙日主左手的太阳经,戊日主右手的太阳经。丙日主左手的阳明经,丁日主右手的阳明经。丙丁都属火,丙、丁日是两火合并,所以称为阳明。庚日主右手的少阴经。癸日主左手的少阴经。辛日主右手的太阴经,壬日主左手的太阴经。足在下属阴,所以足的阳经,为阴中的少阳;足的阴经,为阴中的太阴。手在上属阳,所以手的阳经,为阳中的太阳;手的阴经,为阳中的少阴。总之,腰以上属于阳,腰以下属于阴。\\
至于五脏方面,心为阳中的太阳,肺为阳中的少阴,肝为阴中的少阳,脾为阴中的至阴,肾为阴中的太阴。\\
黄帝问:在治疗上如何运用这些道理呢?\\
岐伯说:正月、二月、三月分主左足的少阳、太阳、阳明经,此时的人气偏重在左,所以不宜针刺左足的三阳经;四月、五月、六月分主右足的阳明、太阳、少阳经,此时的人气偏重在右,所以不宜针刺右足的三阳经;七月、八月、九月分主右足的少阴、太阴、厥阴经,此时的人气偏重在右,所以不宜针刺右足的三阴经;十月、十一月、十二月分主左足的厥阴、太阴、少阴经,此时的人气偏重在左,所以不宜针刺左足的三阴经。\\
黄帝问:从五行来说,东方为天干中的甲、乙,同属于木气,旺于春季,春季的颜色,为苍色,主肝脏。肝的经脉,是足厥阴。现在把甲配属左手的少阳,与五行配天干的规则不同,为什么呢?\\
岐伯说:这是根据天地阴阳的规律来说明手足经脉的阴阳属性的,不是按四时配合五行的次序来分阴阳的。而且阴阳是有名无形的抽象概念,所以用阴阳对立统一来说明事物,可以由一推到十,进一步分析,可以由百推到千,推演至万,就是这个意思。\\
病传第四十二\\
黄帝曰:余受九针于夫子,而私览于诸方。或有导引行气,乔摩、灸、熨、刺、焫、饮药。之一者可独守耶,将尽行之乎?\\
岐伯曰:诸方者,众人之方也,非一人之所尽行也。\\
黄帝曰:此乃所谓守一勿失,万物毕者也。今余已闻阴阳之要,虚实之理,倾移之过,可治之属。愿闻病之变化,淫传绝败而不可治者,可得闻乎?\\
岐伯曰:要乎哉问!道,昭乎其如日醒;窘乎其如夜瞑。能被而服之,神与俱成。毕将服之,神自得之。生神之理,可著于竹帛,不可传于子孙。\\
黄帝曰:何谓日醒?\\
岐伯曰:明于阴阳,如惑之解,如醉之醒。\\
黄帝曰:何谓夜瞑?\\
岐伯曰:瘖乎其无声,漠乎其无形。折毛发理,正气横倾。淫邪泮衍,血脉传溜。大气入藏,腹痛下淫。可以致死,不可以致生。\\
黄帝曰:大气入藏,奈何?\\
岐伯曰:病先发于心,一日而之肺,三日而之肝,五日而之脾。三日不已,死。冬夜半,夏日中。\\
病先发于肺,三日而之肝,一日而之脾,五日而之胃。十日不已,死。冬日入,夏日出。\\
病先发于肝,三日而之脾,五日而之胃,三日而之肾。三日不已,死。冬日入,夏早食。\\
病先发于脾,一日而之胃,二日而之肾,三日而之膀胱。十日不已,死。冬人定,夏晏食。\\
病先发于胃,五日而之肾,三日而之膀胱,五日而上之心。二日不已,死。冬夜半,夏日昳。\\
病先发于肾,三日而之膀胱,三日而上之心,三日而之小肠。三日不已,死。冬大晨,夏晏晡。\\
病先发于膀胱,五日而之肾,一日而之小肠,一日而之心。二日不已,死。冬鸡鸣,夏下晡。\\
诸病以次相传,如是者,皆有死期,不可刺也!间一脏及至三四脏者,乃可刺也。\\
黄帝问:我从夫子那里学到了九针知识,自己又看了记载其他疗法的方书,又有导引行气,按摩、灸、熨、刺、烧、饮药。在治疗时,是只用其中一种方法呢?还是导引等法都综合使用呢?\\
岐伯说:多样的治疗方法,是适应于众人疾病的,不是某一个人都需要使用的。\\
黄帝问:这就是所谓坚守一个总的原则,而不放弃,就能解决各种复杂病情。现在我已听到阴阳的要领,虚实的道理,腠理不固与正气不足的病变,以及病还有可治的机会等。此外,希望再听一下疾病的变化,淫邪传递,正气绝败,以致不可治疗,可以听听吗?\\
岐伯说:你问的是非常重要的。道,它的明显就像“日醒”一样,它的迫切就像“夜瞑”一样。能按照去做,时刻不离于身,心领神会,就会与道合一,始终运用它,自然就会得到神妙。这种“生神”的医理,可以刻在竹帛上,传于后世,不可自私地传给子孙。\\
黄帝问:什么叫日醒?\\
岐伯说:明白了阴阳的规律,好像解开疑惑,又像醉酒醒过来一样。\\
黄帝问:什么叫夜瞑?\\
岐伯说:外邪侵害身体,既没有声响,也没有形迹。只是在不知不觉中毛发折断,腠理开泄,正气随时耗散,淫邪散溢肌体,邪气传留血脉之中。因之流入内脏,腹部作痛,下焦脏气逆乱。可以致死,而不可以使人再活下去。\\
黄帝问:邪气入脏,怎样传变呢?\\
岐伯说:疾病开始发于心脏的,过了一日,就传到肺脏,过了三日,又传到肝,过了五日,又传到脾脏。如果再过三日,病还不好,就会死的。冬季死在半夜,夏季死在中午。\\
疾病开始发于肺脏的,过了三日,就传到肝脏,再过一日,就传到脾脏,过了五日,就传到胃腑。如果再过十日,病还不好,就会死的。冬季死在日入的时候,夏季死在日出的时候。\\
疾病开始发于肝脏的,过了三日,就传到脾脏,过了五日,就会传到胃腑,再过三日,就传到肾脏。如再过三日,还不好,就会死。冬季死在日入的时候,夏季死在早饭的时候。\\
疾病开始发生在脾脏的,一日就传到胃腑,过了二日,就传到肾脏,经过三日,就会传到膀胱。如再过十日,还不好,就会死。冬季死在人定的时候,夏季死在晚饭的时候。\\
疾病开始发生于胃的,过了五日,就传到肾脏,再过三日,就传到了膀胱,再经过五日,就向上传到心脏。如再过二日,还不好,就会死。冬季死在夜半,夏季死在午后。\\
疾病开始发生于肾的,过了三日,就传到膀胱,再过三日,向上传到心脏,传到小肠。如再过三日,还不好,就会死。冬季死在黎明,夏季死在夜间。\\
疾病开始发生在膀胱的,过了五日,就传到肾脏,再过一日,就传到小肠,再过一日,就传到心脏。如再过二日,还不好,就会死。冬季死在夜半后鸡叫时分,夏季死在午后未时。\\
各种疾病都是按着一定的次序相互传变的。像这样的传变,都可预期死亡,不能用针刺治疗!如果疾病的传变次序是间隔一脏或间隔三脏、四脏的,才可以用针刺治疗。\\
淫邪发梦第四十三\\
黄帝曰:愿闻淫邪泮衍,奈何?\\
岐伯曰:正邪从外袭内,而未有定舍,反淫于藏,不得定处,与营卫俱行,而与魂魄飞扬,使人卧不安而喜梦。气淫于府,则有余于外,不足于内;气淫于藏则有余于内,不足于外。\\
黄帝曰:有余不足,有形乎?\\
岐伯曰:阴气盛,则梦涉大水而恐惧;阳气盛,则梦大火而燔焫;阴阳俱盛则梦相杀。上盛,则梦飞;下盛,则梦堕;甚饥,则梦取;甚饱,则梦予。肝气盛,则梦怒;肺气盛,则梦恐惧、哭泣;心气盛,则梦善笑;脾气盛,则梦歌乐,身体重不举;肾气盛,则梦腰脊两解不属。凡此十二盛者,至而泻之,立已。\\
厥气客于心,则梦见丘山烟火;客于肺,则梦飞扬,见金铁之奇物;客于肝,则梦山林树木;客于脾,则梦见丘陵大泽,坏屋风雨;客于肾,则梦临渊,没居水中;客于膀胱,则梦游行;客于胃,则梦饮食;客于大肠,则梦田野;客于小肠,则梦聚邑冲衢;客于胆,则梦斗讼自刳;客于阴器,则梦接内;客于项,则梦斩首;客于胫,则梦行走而不能前,及居深地窌苑中;客于股肱,则梦礼节拜起;客于胞圠,则梦溲便。凡此十五不足者,至而补之,立已也。\\
黄帝说:我想听听邪气在体内弥散而引起梦境的情况是怎样的?\\
岐伯说:正邪从外侵袭体内,没有固定的部位,却流窜到内脏,也没有固定处所,与营卫之气一起流行,随着魂魄一起飞扬,使人睡眠不安而多梦。若邪气扰乱六腑,在外的阳气就有余,在里的阴气就不足;若邪气干扰五脏,在里的阴气就有余,在外的阳气就不足。\\
黄帝问:有余与不足,都有什么表现呢?\\
岐伯说:阴气盛,就会梦见自己趟渡大水而感到恐惧;阳气盛,就会梦见大火燃烧而感到灼热;阴阳都盛,就会梦见互相击杀。上部邪盛,就会梦见向上飞腾;下部邪盛,就会梦见向下坠堕。过度饥饿,就会梦见向人索要东西;过饱,就会梦见给别人东西。肝气盛,就会梦见愤怒;肺气盛,就会梦见恐惧、哭泣;心气盛,就会梦见喜笑;脾气盛,就会梦见歌唱娱乐、身体沉重难以抬举;肾气盛,就会梦见腰脊相互分离而不相连属。以上这十二种气盛所致的梦境发生时,针刺时可在相应腧穴使用泻法,就能消除梦境。\\
邪气侵犯心脏,会梦见山丘烟火弥漫;侵犯肺脏,会梦见飞扬腾越,或看到金铁制成的奇物;侵犯肝脏,会梦见山林树木;侵犯脾脏,会梦见丘陵和大湖,风雨损毁房屋;侵犯肾脏,会梦见面临深渊,或浸没水中;侵犯膀胱,会梦见到处游荡;侵犯到胃,会梦见饮食;侵犯到大肠,会梦见身处广阔的原野;侵犯到小肠,会梦见身居人群熙攘的交通要道;侵犯到胆,会梦见与人打架斗殴、打官司或剖腹自杀;侵犯到生殖器,会梦中性交;侵犯到项部,就会梦见被杀头;侵犯到足胫,会梦见想行走而不能,以及被困于地窖、苑囿之中;侵犯到大腿和上臂,就会在梦中行跪拜之礼;侵犯到膀胱和直肠,就会梦到小便和大便。以上这十五种正气不足所致的梦境,可根据梦境辨别邪气所居之处,针刺相应的腧穴,施以补法,就能使梦境很快痊愈。\\
顺气一日分为四时第四十四\\
黄帝曰:夫百病之所始生者,必起于燥湿、寒暑、风雨、阴阳、喜怒、饮食、居处。气合而有形,得脏而有名,余知其然也。夫百病者,多以旦慧昼安,夕加夜甚,何也?\\
岐伯曰:四时之气使然。\\
黄帝曰:愿闻四时之气。\\
岐伯曰:春生夏长,秋收冬藏,是气之常也,人亦应之。以一日分为四时,朝则为春,日中为夏,日入为秋,夜半为冬。朝则人气始生,病气衰,故旦慧;日中人气长,长则胜邪,故安;夕则人气始衰,邪气始生,故加;夜半人气入藏,邪气独居于身,故甚也。\\
黄帝曰:其时有反者,何也?\\
岐伯曰:是不应四时之气,脏独主其病者,是必以脏气之所不胜时者甚,以其所胜时者起也。\\
黄帝曰:治之奈何?\\
岐伯曰:顺天之时,而病可与期。顺者为工,逆者为粗。\\
黄帝曰:善。余闻刺有五变,以主五输,愿闻其数。\\
岐伯曰:人有五脏,五脏有五变,五变有五输,故五五二十五输,以应五时。\\
黄帝曰:愿闻五变。\\
岐伯曰:肝为牡脏,其色青,其时春,其日甲乙;其音角,其味酸;心为牡脏,其色赤,其时夏,其日丙丁,其音徵,其味苦;脾为牝脏,其色黄,其时长夏,其日戊己,其音宫,其味甘;肺为牝脏,其色白,其时秋,其日庚辛,其音商,其味辛;肾为牝脏,其色黑,其时冬,其日壬癸,其音羽,其味咸。是为五变。\\
黄帝曰:以主五输,奈何?\\
岐伯曰:脏主冬,冬刺井;色主春,春刺荥;时主夏,夏刺输;音主长夏,长夏刺经;味主秋,秋刺合。是谓五变,以主五输。\\
黄帝曰:诸原安合,以致六输?\\
岐伯曰:原独不应五时,以经合之,以应其数,故六六三十六输。\\
黄帝曰:何谓脏主冬,时主夏,音主长夏,味主收,色主春?愿闻其故。\\
岐伯曰:病在脏者,取之井;病变于色者,取之荥;病时间时甚者,取之输;病变于音者,取之经;经满而血者,病在胃,及以饮食不节得病者,取之于合,故命曰味主合。是谓五变也。\\
黄帝说:百病开始发生,一定起于燥湿寒暑风雨等外感,或是由于男女喜怒饮食居处等内伤。邪气侵入体内,就会有症状表现出来,邪入内脏,也有不同的病名,这些我已知道了。很多疾病,多是早晨清爽、白天安静,傍晚加重,夜里更重,这是什么缘故呢?\\
岐伯说:这是因为四时气候使它这样的。\\
黄帝说:希望听一下四时之气。\\
岐伯说:春生、夏长、秋收、冬藏,这是四时气候变化的规律,人体也与此相应。把一天分为四时:早晨是春天,中午是夏天,日落是秋天,半夜是冬天。早晨人体正气,生发如春气,病邪衰退,病者会感觉清爽;中午人体正气盛大如夏气,盛就胜邪,所以病者安静;傍晚人体正气如收敛的秋气,邪气开始生发,所以病势加重;夜半人体正气如闭藏的冬气,邪气独居体内,所以病势更加严重。\\
黄帝问:疾病在一天中的轻重变化,有时没有旦慧、昼安、夕加、夜甚的情况,这是什么原因呢?\\
岐伯说:这是和四时之气不相应,而只由五脏决定病情,这样的病,必定在受病五脏被时日所克的时候就加重,若受病五脏能克制时日的时候病就轻减。\\
黄帝问:怎么治疗呢?\\
岐伯说:治疗必须顺应自然天时的变化,就可以预测疾病的好坏。能这样做,就是高明的医生,相反,就是粗率的医生。\\
黄帝说:好。我听说刺法中有五变,以五输穴为主,希望听听其中的规律。\\
岐伯说:人有五脏,五脏各有相应的色、时、日、音、味的五种变化,每种变化都有井、荥、输、经、合五种腧穴,五五相乘,所以就有二十五个腧穴,与一年中的五季相应。\\
黄帝说:希望听听五变的情况?\\
岐伯说:肝为阳脏,在色为青,在时为春,在日为甲乙,在音为角,在味为酸;心为阳脏,在色为赤,在时为夏,在日为丙丁,在音为徵,在味为苦;脾为阴脏,在色为黄,在时为长夏,在日为戊己,在音为宫,在味为甘;肺为阴脏,在色为白,在时为秋,在日为庚辛,在音为商,在味为辛;肾为阴脏,在色为黑,在时为冬,在日为壬癸,在音为羽,在味为咸。这就是与五脏相应的五变。\\
黄帝问:以五变分主五输穴,是什么情况?\\
岐伯说:五脏主冬,冬季刺井穴;五色主春,春刺荥穴;五时主夏,夏季刺输穴;五音主长夏,长夏刺经穴;五味主秋,秋季刺合穴。这是五变分主五输的情况。\\
黄帝问:六腑的原穴怎样配合,而成为六输呢?\\
岐伯说:只有原穴不与五时相配,而把它归于经穴中,以应五时六输之数,所以六六三十六个腧穴。\\
黄帝问:什么叫做脏主冬,时主夏,音主长夏,味主秋,色主春?希望听听其中的道理。\\
岐伯说:病在五脏的,治疗时应刺井穴;病变表现于面色的,治疗时应刺荥穴;病情时轻时重的,治疗时应刺输穴;疾病表现在声音方面发生变化的,应刺经穴;经脉盛满而有淤血的,病在胃腑,以及因饮食不节引起的疾病,治疗时都应刺合穴,所以说味主合。这就是五变的针治法则。\\
外揣第四十五\\
黄帝曰:余闻九针九篇,余亲受其词,颇得其意。夫九针者,始于一而终于九,然未得其要道也。夫九针者,小之则无内,大之则无外,深不可为下,高不可为盖。恍惚无穷,流溢无极。余知其合于天道、人事、四时之变也。然余愿杂之毫毛,浑束为一,可乎?\\
岐伯曰:明乎哉问也!非独针道焉,夫治国亦然。\\
黄帝曰:余愿闻针道,非国事也。\\
岐伯曰:夫治国者,夫惟道焉。非道,何可小大深浅,杂合而为一乎?\\
黄帝曰:愿卒闻之。\\
岐伯曰:日与月焉,水与镜焉,鼓与响焉。夫日月之明,不失其影;水镜之察,不失其形;鼓响之应,不后其声。动摇则应和,尽得其情。\\
黄帝曰:窘乎哉!昭昭之明不可蔽。其不可蔽,不失阴阳也。合而察之,切而验之,见而得之,若清水明镜之不失其形也。五音不彰,五色不明,五脏波荡,若是则内外相袭,若鼓之应桴,响之应声,影之似形。故远者司外揣内,近者司内揣外。是谓阴阳之极,天地之盖。请藏之灵兰之室,弗敢使泄也。\\
黄帝说:我听过九针九篇,亲自领略着智慧的理论,深受其益。这九针,是从一到九,道理深刻,可是还没有完全懂得其中的主要道理。九针的道理,精到不能再细,大到不能再大,深到不能再深,高到无盖可盖。它的奥妙恍惚无穷,它的运用流溢不尽。以上种种,我知道它是合于天道、人事、四时变化的,我希望把这像毫毛一样细的东西,归纳成为一个总纲,这可以吗?\\
岐伯说:你问得高明极了,不仅是针道要有一个总纲,就是治国也是这样的。\\
黄帝说:我希望听的是针道,并不是国事。\\
岐伯说:治理国事,就是要有一个一以贯之的“道”。没有“道”,怎么能把小大、深浅的许多复杂的事务,综合为一个总纲呢?\\
黄帝说:希望详尽地听一下。\\
岐伯说:这可用日和月,水和镜,鼓和响来比喻。日月照明,物影立现;水镜之光,容颜不失;击鼓作响,其声紧随。所以形与影,声与响是相互应和的,明白了这些,就能够掌握针刺的法则了。\\
黄帝说:这个问题说起来真困难啊!尽管困难,但深刻的真理之光,是不可遮蔽的。它所以不可遮蔽,是由于不失去阴阳相对的道理。在临证时,综合病人的情况而观察它,切诊来验证脉象,望诊来得到外部情况,这就像清水明镜之不失真一样。人的声音色泽,是内脏功能的反应,如果五音不响亮,五色不鲜明,五脏动摇,像这样内外相因,就像鼓与槌相和,响与声相应,影与形相类一样。因此说,从远看,观察在外的声音色泽,可以测知内脏的症候;从近看,观察在内的脏腑,可以测知声音色泽的变化。这可说是阴阳变化的极点,天地所包的道理也尽在其中。希望把它藏在灵兰室,不敢使它散失。\\
五变第四十六\\
黄帝问于少俞曰:余闻百疾之始期也,必生于风雨寒暑,循毫毛而入腠理。或复还,或留止,或为风肿汗出,或为消瘅,或为寒热,或为留痹,或为积聚。奇邪淫溢,不可胜数,愿闻其故。夫同时得病,或病此,或病彼,意者天之为人生风乎,何其异也?\\
少俞曰:夫天之生风者,非以私百姓也。其行公平正直,犯者得之,避者得无殆,非求人而人自犯之。\\
黄帝曰:一时遇风,同时得病,其病各异,愿闻其故。\\
少俞曰:善乎哉问!请论以比匠人。匠人磨斧斤,砺刀削,斫材木。木之阴阳,尚有坚脆。坚者不入,脆者皮弛。至其交节,而缺斤斧焉。夫一木之中,坚脆不同。坚者则刚,脆者易伤。况其材木之不同,皮之厚薄,汁之多少,而各异耶。夫木之早花先生叶者,遇春霜烈风,则花落而叶萎。久曝大旱,则脆木薄皮者,枝条汁少而叶萎。久阴淫雨,则薄皮多汁者,皮溃而漉。卒风暴起,则刚脆之木,枝折杌伤。秋霜疾风,则刚脆之木,根摇而叶落。凡此五者,各有所伤,况于人乎。\\
黄帝曰:以人应木奈何?\\
少俞答曰:木之所伤也,皆伤其枝。枝之刚脆而坚,未成伤也。人之有常病也,亦因其骨节皮肤腠理之不坚固者,邪之所舍也,故常为病也。\\
黄帝曰:人之善病风厥漉汗者,何以候之?\\
少俞答曰:肉不坚,腠理疏,则善病风。\\
黄帝曰:何以候肉之不坚也?\\
少俞答曰:夬肉不坚,而无分理。理者粗理,粗理而皮不致者,腠理疏。此言其浑然者。\\
黄帝曰:人之善病消瘅者,何以候之?\\
少俞答曰:五藏皆柔弱者,善病消瘅。\\
黄帝曰:何以知五藏之柔弱也?\\
少俞答曰:夫柔弱者,必有刚强,刚强多怒,柔者易伤也。\\
黄帝曰:何以候柔弱之与刚强?\\
少俞答曰:此人薄皮肤而目坚固以深者,长冲直扬,其心刚,刚则多怒,怒则气上逆,胸中蓄积,血气逆留,夃皮充肌,血脉不行,转而为热,热则消肌肤,故为消瘅。此言其人暴刚而肌肉弱者也。\\
黄帝曰:人之善病寒热者,何以候之?\\
少愈答曰:小骨弱肉者,善病寒热。\\
黄帝曰:何以候骨之小大,肉之坚脆,色之不一也?\\
少愈答曰:颧骨者,骨之本也。颧大则骨大,颧小则骨小。皮肤薄而其内无夬,其臂懦懦然,其地色炲然,不与其天同色,污然独异,此其候也。然臂薄者,其髓不满,故善病寒热也。\\
黄帝曰:何以候人之善病痹者?\\
少俞答曰:粗理而肉不坚者,善病痹。\\
黄帝曰:痹之高下有处乎?\\
少俞答曰:欲知其高下者,各视其部。\\
黄帝曰:人之善病肠中积聚者,何以候之?\\
少俞答曰:皮肤薄而不泽,肉不坚而淖泽,如此则肠胃恶,恶则邪气留止,积聚乃伤。脾胃之间,寒温不次,邪气稍至,稸积留止,大聚乃起。\\
黄帝曰:余闻病形,已知之矣,愿闻其时。\\
少俞答曰:先立其年,以知其时。时高则起,时下则殆。虽不陷下,当年有冲通,其病必起,是谓因形而生病。五变之纪也。\\
黄帝问少俞说:我听说各种疾病开始时,必定由风雨寒暑外感引起,邪气沿着毫毛而进入腠理。或传变,或留止,或形成风肿而出汗,或发为消瘅,或寒热往来,或成为久痹,或形成积聚。不正的邪气散漫于体内,以致病证难以尽数,希望听听其中的缘故。至于同时得病,有的生这种病,有的生那种病,我认为是自然界气候对人的影响不同,否则,为什么发生的病变各不相同呢?\\
少俞说:自然界发生的风,不会偏私某个人,它普遍吹动,公平正直,触犯它,就会得病;避开它,就没有危险。不是风邪找人,是人自己去触犯它,才生病的。\\
黄帝说:同一时候遇到风,又同时得了病,可是病情不一样,希望听一下其中的原因。\\
少俞说:问得很好。让我拿匠人来比喻吧。匠人磨斧、磨刀,砍削木材。树木的阴面阳面,有坚硬与脆薄的区别。坚者不易砍入,脆者容易裂开,遇到结节,能够损坏刀斧。就木材说,坚脆不一样,坚硬的就强,脆薄的易折。何况木材种类不同,外皮的厚薄,内含汁液的多少,也各不相同呢!像那早开花先生叶的,遇到春霜烈风,就会花落而叶萎。或久经暴晒,大旱,脆弱皮薄的木材,枝条中含的水分少了,而致树叶枯萎。或久经阴天,阴雨连绵,木材薄皮而多含水分的,就会树皮溃烂渗水。或遭到狂风暴起,就会使刚脆的树木,树枝折断,树干损伤。或遇到秋霜疾风,就会使刚脆的树木,树根摇动,树叶零落。以上这五种木材的情况,分别有不同的伤损,何况人呢?\\
黄帝说:将人和树木相比,是怎样的?\\
少俞回答说:树木所受的损伤,都是树枝受伤。如果树枝刚实坚硬,就未必受到损伤。人经常有病,也是因为它的骨节皮肤腠理不坚固,往往是病邪所留止的地方,所以经常有病。\\
黄帝问:人有常患风厥病,汗出不止,应该怎样诊察呢?\\
少俞回答说:肌肉不坚实,腠理疏松,就会常感受风病。\\
黄帝问:怎样来诊察肌肉不坚实呢?\\
少俞回答说:那是肩、肘、髀、膝等处的肌肉不坚实,又没有肤纹的。由于肌肉不坚实,肤粗,皮亦不致密。因此,腠理疏松,就容易感受风邪。这仅说是大致如此吧。\\
黄帝问:人有常患消瘅病,应该怎样诊察呢?\\
少俞回答说:五脏都很柔弱的人,就容易发生消瘅病。\\
黄帝问:怎么知道五脏柔弱呢?\\
少俞回答说:五脏柔弱的人,必定性气刚强,性气刚强则多怒,所以五脏柔弱的人就容易受到损伤。\\
黄帝问:怎样诊察五脏柔弱与性气刚强呢?\\
少俞回答说:这种人皮肤薄,但是眼睛坚固深入,眉毛竖起,性气刚暴,性气刚暴就容易发怒,怒则使气向上逆,而积聚胸中,血气停留不畅,肌肉皮肤肿胀,使血脉不得畅流而生郁热,郁热则消烁肌肉皮肤,而成为消瘅。这是指性情刚暴而肌肉脆弱的人啊。\\
黄帝问:人有常患寒热病,应怎样诊察呢?\\
少俞回答说:骨骼细小,肌肉脆弱的人,容易患寒热病。\\
黄帝问:怎样诊察骨骼的大小、肌肉的坚脆、气色的不同呢?\\
少俞回答说:面部颧骨是全身骨骼之本。颧骨大的骨骼也大,颧骨小的骨骼也小。皮肤薄弱,肌肉也不能隆起,臂膊柔弱而无力,下巴的气色晦浊无神,与天庭的气色不一致,像蒙罩着一层污垢,这就是诊候骨、肉、色的方法。同时,如果两臂肌肉薄弱,其骨髓必不充满,所以容易患寒热病。\\
黄帝问:怎样诊察病人容易患痹证呢?\\
少俞回答说:腠理粗疏而肌肉不坚实,就容易患痹证。\\
黄帝问:痹证部位上下有一定的处所吗?\\
少俞回答说:要知道痹证部位的高下,必须观察五脏的分部情况。\\
黄帝问:人有常患肠中积聚的,怎样诊察呢?\\
少俞回答说:皮肤薄弱,缺乏光泽,肌肉不坚实而缺乏滑泽,这样,可知肠胃功能不好,邪气容易积聚而伤及脾胃的功能。如果在脾胃之间,因寒温不调,邪气渐渐侵入,就会聚集停留,而形成积聚病。\\
黄帝说:关于疾病的症状,我已经知道了,希望再听听疾病与时令的关系。\\
少俞回答说:首先要确定一年的气候变化规律,然后再掌握各个时令的气候。凡在气候对疾病有利之时,其病就会好转;气候对疾病不利之时,其病就会恶化。有时虽然某一时令的气候变化并不剧烈,但因该年气候对其人体不适应,也一定引发疾病。这叫因形体素质不同而引发的疾病。这就是五变为病的纲要。\\
本脏第四十七\\
黄帝问于岐伯曰:人之血气精神者,所以奉生而周于性命者也。经脉者,所以行血气而营阴阳,濡筋骨,利关节者也;卫气者,所以温分肉,充皮肤,肥腠理,司开阖者也;志意者,所以御精神,收魂魄,适寒温,和喜怒者也。是故血和则经脉流行,营复阴阳,筋骨劲强,关节清利矣。卫气和则分肉解利,皮肤调柔,腠理致密矣。志意和则精神专直,魂魄不散,悔怒不起,五脏不受邪矣。寒温和则六腑化谷,风痹不作,经脉通利,肢节得安矣。此人之常平也。五脏者,所以藏精神血气魂魄者也;六腑者,所以化水谷而行津液者也。此人之所以具受于天也,无愚智贤不肖,无以相倚也。然有其独尽天寿,而无邪僻之病,百年不衰,虽犯风雨卒寒大暑,犹有弗能害也;有其不离屏蔽室内,无怵惕之恐,然犹不免于病,何也?愿闻其故。\\
岐伯对曰:窘乎哉问也!五脏者,所以参天地,副阴阳,而连四时,化五节者也。五脏者,固有大小、高下、坚脆、端正、偏倾者;六腑亦有小大、长短、厚薄、结直、缓急。凡此二十五者,各不同,或善或恶,或吉或凶。请言其方。\\
心小则安,邪弗能伤,易伤以忧;心大则忧不能伤,易伤于邪。心高则满于肺中,悗而善忘,难开以言;心下则脏外,易伤于寒,易恐以言。心坚则脏安守固;心脆则善病消瘅热中。心端正则和利难伤;心偏倾则操持不一,无守司也。\\
肺小则少饮,不病喘喝;肺大则多饮,善病胸痹、喉痹、逆气。肺高则上气肩息咳;肺下则居贲迫肺,善胁下痛。肺坚则不病咳上气;肺脆则苦病消瘅易伤。肺端正则和利难伤;肺偏倾则胸偏痛也。\\
肝小则脏安,无胁下之病;肝大则逼胃迫咽,迫咽则苦膈中,且胁下痛。肝高则上支贲切,胁悗,为息贲;肝下则逼胃,胁下空,胁下空则易受邪。肝坚则脏安难伤;肝脆则善病消瘅易伤。肝端正则和利难伤;肝偏倾则胁下痛也。\\
脾小则脏安,难伤于邪也;脾大则苦凑尐而痛,不能疾行。脾高则尐引季胁而痛;脾下则下加于大肠,下加于大肠则脏苦受邪。脾坚则脏安难伤;脾脆则善病消瘅易伤。脾端正则和利难伤,脾偏倾则善满善胀也。\\
肾小则脏安难伤;肾大则善病腰痛,不可以俯仰,易伤以邪。肾高则苦背膂痛,不可以俯仰;肾下则腰尻痛,不可以俯仰,为狐疝。肾坚则不病腰背痛;肾脆则善病消瘅易伤。肾端正则和利难伤;肾偏倾则苦腰尻痛也。凡此二十五变者,人之所苦常病。\\
黄帝曰:何以知其然也?\\
岐伯曰:赤色小理者心小,粗理者心大。无巿旡者,心高;巿旡小、短、举者,心下。巿旡长者,心下坚;巿旡弱小以薄者,心脆。巿旡直下不举者,心端正;巿旡倚一方者,心偏倾也。\\
白色小理者,肺小;粗理者,肺大。巨肩反膺陷喉者,肺高;合腋张胁者,肺下。好肩背厚者,肺坚;肩背薄者,肺脆。背膺厚者,肺端正;胁偏疏者,肺偏倾也。\\
青色小理者,肝小;粗理者,肝大。广胸反骹者,肝高;合胁兔骹者,肝下。胸胁好者,肝坚;胁骨弱者,肝脆。膺腹好相得者,肝端正;胁骨偏举者,肝偏倾也。\\
黄色小理者,脾小;粗理者,脾大。揭唇者,脾高;唇下纵者,脾下。唇坚者,脾坚;唇大而不坚者,脾脆。唇上下好者,脾端正;唇偏举者,脾偏倾也。\\
黑色小理者,肾小;粗理者,肾大。高耳者,肾高;耳后陷者,肾下。耳坚者,肾坚;耳薄不坚者,肾脆。耳好前居牙车者,肾端正,耳偏高者,肾偏倾也。凡此诸变者,持则安,减则病也。\\
帝曰:善。然非余之所问也。愿闻人之有不可病者,至尽天寿,虽有深忧大恐,怵惕之志,犹不能感也,甚寒大热,不能伤也;其有不离屏蔽室内,又无怵惕之恐,然不免于病者,何也?愿闻其故。\\
岐伯曰:五脏六腑,邪之舍也,请言其故。五脏皆小者,少病,苦燋心,大愁忧;五脏皆大者,缓于事,难使以忧。五脏皆高者,好高举措;五脏皆下者,好出人下。五脏皆坚者,无病;五脏皆脆者,不离于病。五脏皆端正者,和利得人心;五脏皆偏倾者,邪心而善盗,不可以为人,卒反复言语也。\\
黄帝曰:愿闻六腑之应。\\
岐伯答曰:肺合大肠,大肠者,皮其应;心合小肠,小肠者,脉其应。肝合胆,胆者,筋其应;脾合胃,胃者,肉其应;肾合三焦膀胱,三焦膀胱者,腠理毫毛其应。\\
黄帝曰:应之奈何?\\
岐伯曰:肺应皮。皮厚者大肠厚,皮薄者大肠薄。皮缓,腹裹大者大肠大而长,皮急者大肠急而短。皮滑者大肠直,皮肉不相离者大肠结。\\
心应脉。皮厚者脉厚,脉厚者小肠厚;皮薄者脉薄,脉薄者小肠薄;皮缓者脉缓,脉缓者小肠大而长;皮薄而脉冲小者,小肠小而短。诸阳经脉皆多纡屈者小肠结。\\
脾应肉。肉夬坚大者胃厚,肉夬幺者胃薄。肉夬小而幺者胃不坚;肉夬不称身者胃下,胃下者下管约不利。肉夬不坚者胃缓,肉夬无小裹累者胃急。肉夬多少裹累者胃结,胃结者上管约不利也。\\
肝应爪。爪厚色黄者胆厚,爪薄色红者胆薄。爪坚色青者胆急,爪濡色赤者胆缓。爪直色白无约者胆直,爪恶色黑多纹者胆结也。\\
肾应骨,密理厚皮者,三焦膀胱厚,粗理薄皮者,三焦膀胱薄。疏腠理者,三焦膀胱缓,皮急而无毫毛者,三焦膀胱急。毫毛美而粗者,三焦膀胱直,稀毫毛者,三焦膀胱结也。\\
黄帝曰:厚薄美恶皆有形,愿闻其所病。\\
岐伯答曰:视其外应,以知其内脏,则知所病矣。\\
黄帝问岐伯说:人体的血气精神,是养生而使性命存续的物质。人的经脉是运行血气,转输清浊之气,濡润筋骨,滑利关节的;人的卫气是温养肌肉,充养皮肤,肥盛腠理,管理皮肤腠理开合的;人的志意是驾驭精神,收聚魂魄,适应寒温变化,调节情绪的。所以血脉调和则经脉流行,营养周身内外,筋骨强劲,关节滑利。卫气调和则分肉感到舒畅滑利,皮肤和柔,腠理致密。志意和顺则精神专一,魂魄不散漫,悔怒不妄起,五脏不受邪气侵袭。适应气候的寒温变化,则六腑能正常运化水谷,不致发生风痹,经脉畅通,四肢关节活动正常。这些都是人体协调的常态。五脏是储藏精神血气魂魄的;六腑是运化谷物而布散津液的。这些都是人天然的禀受,不论愚智贤不肖,没有不同的。但有的人独享大寿,未发生过什么疾病,直到百岁,身体不衰,虽然遇到了风雨、暴冷、大暑的气候,也不能损害其健康;还有的人从不离开屏风、室内,也没遭到惊恐害怕的事,但仍然免不了生病,这是为什么?希望听一下其中的缘故。\\
岐伯回答说:你问得很难回答啊!五脏,与天地相参,阴阳相配,与四时五季的变化相应。五脏本来有小大、高下、坚脆、端正偏倾等不同;六腑也有小大、长短、厚薄、曲直、缓急等差异。这二十五种变化,各不相同,或善或恶,或吉或凶,请让我说说它的道理吧。\\
心脏小的,则心气安定,外邪不能伤害,但易被内忧所伤;心脏大的,不致被内忧所伤,但易为外邪所伤。心脏位置高,则充满肺部,多烦闷,好忘事,很难用言语开导他;心脏位置低,则脏气不紧密,易为寒邪所伤,又容易用言语去恐吓他。心脏坚实的,则所藏的神气安定,内守固密;心脏脆弱的,则多患消瘅热中。心脏位置端正,则脏气和谐,外邪难以伤害,心脏位置偏倾不正,则操持各种事物不能如一,这是精神不能内守去约束。\\
肺脏小的,就饮水少,也不患喘喝的病;肺脏大的,就饮水多,容易患胸痹、喉痹、逆气等证。肺脏位置高的,就会气逆向上、肩息、咳嗽等证;肺脏位置低的,就会逼迫胸膈,多胁下痛。肺脏坚实的,就不会患咳嗽、气逆向上的病;肺脏脆弱的,就会患消瘅病,容易感受外邪。肺脏位置端正,则肺气和利,外邪难以伤害;肺脏位置偏倾不正,就会影响胸胁偏痛。\\
肝脏小的,则脏气安定,没有胁下作痛的病;肝脏大的,就会逼近胃部,上迫咽喉,胸中膈塞不通,并且胁下疼痛。肝脏位置高的,就会上支胸膈,并且胁下拘急,发为息贲;肝脏位置低的,则胃部安和,胁下空虚,因为空虚就容易感受外邪。肝脏坚实,则脏气安定,外邪难以伤害;肝脏脆弱,则多患消瘅,而易被外邪所伤。肝脏的位置端正,则肝气和利,不易为外邪伤害;肝脏的位置偏倾的,则胁下也会偏痛的。\\
脾脏小的,则脏气安定,外邪难以伤害;脾脏大的,就会经常影响腋下胁上空软部分作痛,走路不快。脾脏位置高的,胁下空软处会牵引季胁作痛;脾脏位置低,就向下加于大肠之上,常受邪气伤害。脾脏坚实的,则脏气安和,难被外邪所伤;脾脏脆弱的,就会患消瘅病,容易为外邪侵害。脾脏位置端正,则脾气和利,不易为外邪伤害;脾脏位置偏倾,就容易发生胀满。\\
肾脏小的,则脏气安定,外邪难以伤害;肾脏大的,则常患腰痛,不能俯仰,容易为邪所伤。肾脏位置高,经常有脊背疼痛,不能俯仰;肾脏位置低,就会腰尻部疼痛,不能前后俯仰,且有狐疝。肾脏坚实,就没有腰背痛;肾脏脆弱,就多病消瘅,容易为邪气所伤。肾脏位置端正,则肾气和利,不易为外邪伤害;肾脏位置偏倾,就会经常发生腰尻偏痛。以上这二十五种变化,是人经常发生的疾病。\\
黄帝说:怎样知道五脏的大小、高低、坚脆、端正与偏倾呢?\\
岐伯说:皮肤红色,纹理细密的,心脏就小,纹理粗疏的,心脏就大。看不见胸骨剑突的,心脏的位置就高;胸骨剑突小,短而鸡胸的,心脏的位置就低。胸骨剑突长的,心脏就坚实;胸骨剑突弱小而较薄的,心脏就脆弱。胸骨剑突直下而不突起的,心脏就端正;胸骨剑突偏在一面的,心脏就偏倾不正。\\
皮肤白色,纹理细密的,肺脏就小;纹理粗疏的,肺脏就大。两肩高大,胸部向外突出,而咽喉内陷的,肺脏的位置就高;两腋收敛,两胁开张的,肺脏的位置就低。肩背部宽厚的,肺脏就坚实;肩背部薄弱的,肺脏就脆弱。背部及胸膺宽厚的,肺脏就端正;胸部偏斜的,肺就偏倾不正。\\
皮肤青色,纹理细密的,肝脏就小;纹理粗疏的,肝脏就大。胸部宽阔,胁骨隆起的,肝脏的位置就高;胁部狭窄,胁骨低的,肝脏的位置就低。胸胁健壮的,肝脏就坚实;胁骨柔软的,肝脏就脆弱。胸腹好,比例匀称的,肝脏就端正;胁骨偏斜而高起的,肝脏就偏倾不正。\\
皮肤黄色,纹理细密的,脾脏就小;纹理粗疏的,脾脏就大。嘴唇上翻的,脾脏的位置就高;嘴唇下垂的,脾脏的位置就低。嘴唇坚实的,脾脏就坚实;嘴唇大而不坚实的,脾脏就脆弱。嘴唇上下均匀的,脾脏就端正;嘴唇偏耸的,脾脏就偏倾不正。\\
皮肤黑色,纹理细密的,肾脏就小;纹理粗疏的,肾脏就大。两耳高的,肾脏的位置就高;两耳向后陷下的,肾脏的位置就低。耳朵皮肉坚实的,肾脏就坚实;耳薄而皮肉不坚实的,肾脏就脆弱。两耳皮肉丰厚,位于两侧颊车之前的,肾脏就端正;两耳一边偏高的,肾脏就偏倾不正。以上各种变化情况,如能注意调养,就仍能保持正常,如不善调理,有所伤损,就会发生疾病。\\
黄帝说:说得好。但这些不是我要问的。我希望听听有的人从不患病,能享大寿。虽然遇到深忧大恐,情绪上极坏,还不能损伤他,酷寒炎暑,都不能损伤他;还有的人,不离开屏风室内,也没有深忧大恐,可仍不免患病,这是什么道理?希望知道其中的缘故。\\
岐伯说:五脏六腑,是可以被外邪侵入之处,请让我讲讲其中的缘故。五脏都小的,生病就少,但经常要劳心焦虑,免不了忧愁;五脏都大的,做事缓慢,很难使他忧愁。五脏的位置都高,举动措置,好高骛远而不切实际;五脏的位置都低,意志薄弱,情愿居于人下。五脏都坚实的,不会生病;五脏都脆弱的,病患缠身。五脏的位置都端正的,性情和顺而受人喜欢;五脏的位置都偏倾的,居心不正而常为盗窃,不够做人的条件,他的言语竟反复无常。\\
黄帝说:希望听一下六腑与人体组织的相应情况。\\
岐伯回答说:肺与大肠表里配合,大肠外应于皮肤;心与小肠表里配合,小肠外应于血脉。肝与胆表里配合,胆外应于筋;脾与胃表里配合,胃外应于肉;肾与三焦膀胱表里配合,三焦膀胱外应于毫毛腠理。\\
黄帝问:脏腑和各组织的相应关系怎样呢?\\
岐伯说:肺与皮肤相应,又与大肠相表里。那么皮肤厚的,大肠就厚;皮肤薄的,大肠就薄。皮肤松,肚囊大的,大肠就缓纵而长;皮肤紧,大肠就紧而短。皮肤滑润的,大肠就滑利;皮肉不相附丽的,大肠就不滑利。\\
心与血脉相应,又与小肠相表里。脉在皮中,那么皮肤厚的,血脉就厚,血脉厚的,小肠就厚;皮肤薄的,血脉就薄,血脉薄的,小肠就薄;皮肤弛缓的,血脉就弛缓,血脉弛缓的,小肠的形状就大而长;皮肤薄,血脉虚少的,小肠的形状就小而短。各条阳经脉络显现有纡屈现象的,就可知小肠之气也会有所郁结的。\\
脾与肉相应,而与胃相表里。脾主肉,那么肉刐坚大的,胃体就厚;肉刐小的,胃体就薄。肉刐小而且薄的,胃就不坚实;肉刐与身体不相称的,胃的位置偏下,而致胃下口被压迫拘束,食物不能顺利通过。肉刐不坚实的,则胃弛缓;肉刐上没有小颗粒累累相连的,则胃体紧敛。肉刐上出现很多小颗粒的,则胃气郁结,这样,则胃上口拘束,就会饮食困难。\\
肝与爪甲相应,而与胆相表里。肝主筋,爪甲是筋之余,爪甲厚而色黄的,胆囊就厚;爪甲薄而色红的,胆囊就薄。爪甲坚硬而色青的,胆紧敛;爪甲柔润而色赤的,胆弛缓。爪甲平直无纹而白色的,胆气舒畅和顺;爪甲畸形色黑而多纹的,胆气郁结不舒。\\
肾与骨相应,而肾主骨,内与三焦膀胱相应。纹理密,皮肤厚,则三焦膀胱厚;纹理粗,皮肤薄,则三焦膀胱薄。腠理疏松的,则三焦膀胱之气就和缓,皮肤紧绷,而无毫毛的,则三焦膀胱之气就紧促。毫毛美好而粗的,则三焦膀胱之气就条达;毫毛稀少的,则三焦膀胱之气就郁结不舒了。\\
黄帝说:脏腑的厚薄美恶,既然都有形状,希望再听一下它所发生的疾病。\\
岐伯回答说:观察它在外的相应情况,可以测知内脏变化,也就知道所发生的疾病。\\
卷八\\
禁服第四十八\\
雷公问于黄帝曰:细子得受业,通于《九针》六十篇,旦暮勤服之,近者编绝,久者简垢,然尚讽诵弗置,未尽解于意矣。《外揣》言浑束为一,未知所谓也。夫大则无外,小则无内,大小无极,高下无度,束之奈何?士之才力,或有厚薄,智虑褊浅,不能博大深奥,自强于学若细子,细子恐其散于后世,绝于子孙,敢问约之奈何?\\
黄帝曰:善乎哉问也!此先师之所禁,坐私传之也,割臂歃血之盟也,子若欲得之,何不斋乎?\\
雷公再拜而起曰:请闻命。于是也,乃斋宿二日,而请曰:敢问今日正阳,细子愿以受盟。黄帝乃与俱入斋室,割臂歃血。黄帝亲祝,曰:今日正阳,歃血传方,有敢背此言者,反受其殃。\\
雷公再拜曰:细子受之。黄帝乃左握其手,右授之书,曰:慎之慎之,吾为子言之。\\
凡刺之理,经脉为始,营其所行,知其度量;内刺五藏,外刺六府;审察卫气,为百病母;调其虚实,虚实乃止;泻其血络,血尽不殆矣。\\
雷公曰:此皆细子之所以通,未知其所约也。\\
黄帝曰:夫约方者,犹约囊也,囊满而弗约,则输泄;方成弗约,则神弗与俱。\\
雷公曰:愿为下材者,弗满而约之。\\
黄帝曰:未满而约之以为工,不可为天下师。\\
雷公曰:愿闻为工。\\
黄帝曰:寸口主中,人迎主外,两者相应,俱往俱来,若引绳大小齐等。春夏人迎微大,秋冬寸口微大,如是者名曰平人。\\
人迎大一倍于寸口,病在足少阳,一倍而躁,在手少阳;人迎二倍,病在足太阳,二倍而躁,病在手太阳;人迎三倍,病在足阳明,三倍而躁,病在手阳明。盛则为热,虚则为寒,紧则为痛痹,代则乍甚乍间。盛则泻之,虚则补之;紧痛则取之分肉,代则取血络,且饮药;陷下则灸之;不盛不虚,以经取之,名曰经刺。人迎四倍者,且大且数,名曰溢阳,溢阳为格,死不治。必审按其本末,察其寒热,以验其脏腑之病。\\
寸口大于人迎一倍,病在足厥阴;一倍而躁,在手心主;寸口二倍,病在足少阴;二倍而躁,在手少阴;寸口三倍,病在足太阴;三倍而躁,在手太阴。盛则胀满、寒中、食不化;虚则热中、出糜、少气、溺色变;紧则痛痹;代则乍痛乍止。盛则泻之,虚则补之。紧则先刺而后灸之,代则取血络而后调之。陷下则徒灸之。陷下者,脉血结于中,中有著血,血寒,故宜灸之。不盛不虚,以经取之。寸口四倍者,名曰内关,内关者,且大且数,死不治。必审察其本末之寒温,以验其藏府之病。\\
通其营输,乃可传于大数。大数曰:盛则徒泻之,虚则徒补之。紧则灸刺且饮药。陷下则徒灸之。不盛不虚,以经取之。所谓经治者,饮药,亦曰灸刺。脉急则引,脉大以弱,则欲安静,用力无劳也。\\
雷公问黄帝说:我自从跟随您接受学业,通读《九针》六十篇,从早到晚勤奋学习,尽管编绝简垢,还不停地阅读背诵,但还不能完全了解其中的精义。如《外揣》篇里说的“浑束为一”,不知是什么意思。既然说九针的道理,大到不可再大,细到不可再精,大到无限大,小到无限小,既然大小高下都是无限的,又怎样将其约束总结呢?况且,人们的智力有高低的不同,有的人浅见薄识,不能领会博大高深的道理,又不能像我一样的勤奋努力,我担心学术会在后世流散失传,子孙不能继承下去,请问怎样由博返约呢?\\
黄帝说:你问得很好!这正是先师再三告诫,不能传给那种不劳而获、专谋私利的人,所以要通过割臂歃血的盟誓,才能秘密地传授。你要想得到它,为什么不斋戒呢?\\
雷公拜了两拜,说:我愿遵照你说的去做。于是雷公很诚恳地斋戒独宿二天,又来请求说:今天正午时分,我愿受盟传方。黄帝和他一同进入斋室,举行割臂歃血的盟誓。黄帝亲自祝告说:今天正午时分,歃血为盟,传授医方,有敢违背今天誓言的,必遭受祸殃。\\
雷公再拜说:我愿接受盟戒。黄帝左手握住雷公的手,右手将书授予雷公,并说:慎重啊慎重!我现在给你讲解其中的道理。\\
大凡针刺的道理,以掌握经脉为开始,要知道经脉循行的走向,并知道它的长短大小;病在内的,针刺五脏所属的经脉,病在外的,针刺六腑所属的经脉;同时要审察卫气的变化,因为卫气是人体的护卫,卫气失常则外邪易入,百病由此而生;实则泻之,虚则补之,如能调治其虚实,补泻得宜,则虚实病变就会停止;病在血络的,刺络泻血,淤血消除,病情就会好转。\\
雷公说:这些道理我是知道的,但还不知如何把它们归纳起来,以掌握其要领。\\
黄帝说:约方,就像把袋口扎住一样,袋子满了,如果不扎袋口,则装的东西就会倒出来;学了许多医方,如果不能提纲挚领加以总结归纳,则杂而不精,就不能出神入化,运用自如。\\
雷公说:愿做下等人材的人,不求学识渊博,就想要归纳精简、提纲挚领。\\
黄帝说:这样的人只能做个一般的医生,而不能做天下医师的导师。\\
雷公说:我希望听听如何做个一般医生。\\
黄帝说:寸口脉是诊察内在的五脏病变,人迎脉是诊察外在的六腑病变,这两个部位的脉象搏动往来不息,大小均等。春夏时节,人迎脉略大一些,秋冬时节,寸口脉略大一些,像这样的就是正常人。\\
人迎比寸口的脉大一倍的,病在足少阳经,大一倍而躁疾的,病在手少阳经;人迎脉比寸口大两倍的,病在足太阳经,大二倍而躁疾的,病在手太阳经;人迎脉比寸口大三倍的,病在足阳明经,大三倍而躁疾的,病在手阳明经。人迎脉盛大的为热;虚小的为寒;脉紧的为痛痹;脉结代的病情时轻时重。治疗时,脉盛的实证用泻法,脉虚的虚证用补法;脉紧的痛证,针刺分肉间的腧穴,脉代的取血络放血,并配合服用汤药;经脉陷下的用灸法;不盛不虚的,根据发病经脉取穴,叫做“经刺”。人迎脉比寸口大四倍,大而且数,名叫“溢阳脉”,溢阳是阴气格阳于外的死证。必须详细研究发病的终始本末,辨清寒热属性,以察验脏腑的病变。\\
寸口脉大于人迎一倍的,病在足厥阴经;大一倍而躁急的,病在手厥阴经;寸口脉大于人迎二倍的,病在足少阴经;大二倍而躁急的,病在手少阴经;寸口脉大于人迎三倍的,病在足太阴经;大三倍而躁急的,病在手太阴经。寸口脉盛大的,可见胀满、内寒、食不消化等证;寸口脉虚弱的,可见内热、大便如糜、少气、小便色变;寸口脉紧的,出现痛痹;脉代的是血脉不调,或痛或止。脉盛的用泻法,脉虚的用补法。脉紧的先针刺而后用灸,脉代的刺血络泄淤血,然后用药物调治。脉虚陷不起的,用灸法。脉虚陷不起的是因脉中血行凝结,并有淤血附着脉中,血因寒凝,所以宜用灸法。不盛不虚的,取本经腧穴位治疗。寸口脉大于人迎四倍的,叫做“内关”,内关的脉象大而且数,是不治的死证。总之,必须详细研究发病的终始本末,辨清寒热属性,以察验脏腑的病变。\\
必须通晓经脉的循行和输注的生理,才能进一步传授针灸治病的大法。大法是:脉象盛大的用泻法,脉象虚小的用补法。脉象紧的可灸、刺、药并用,脉虚陷不起的用灸法。脉不盛不虚的本经自病,取本经腧穴治疗。所谓经治,就是或服药,或灸刺,随其经脉所宜而选用治疗方法。脉急的是邪盛,可兼用导引法,脉大而弱的,宜安心静养,不要过劳或勉强用力。\\
五色第四十九\\
雷公问于黄帝曰:五色独决于明堂乎?小子未知其所谓也。\\
黄帝曰:明堂者,鼻也;阙者,眉间也;庭者,颜也;蕃者,颊侧也;蔽者,耳门也。其间欲方大,去之十步,皆见于外。如是者寿,必中百岁。\\
雷公曰:五官之辨奈何?\\
黄帝曰:明堂骨高以起,平以直。五藏次于中央,六府挟其两侧。首面上于阙庭,王宫在于下极。五藏安于胸中,真色以致,病色不见。明堂润泽以清。五官恶得无辨乎?\\
雷公曰:其不辨者,可得闻乎?\\
黄帝曰:五色之见也,各出其色部。部骨陷者,必不免于病矣。其色部乘袭者,虽病甚,不死矣。\\
雷公曰:官五色奈何?\\
黄帝曰:青黑为痛,黄赤为热,白为寒。是谓五官。\\
雷公曰:病之益甚,与其方衰,如何?\\
黄帝曰:外内皆在焉。切其脉口滑小紧以沉者,病益甚,在中;人迎气大紧以浮者,其病益甚,在外。其脉口浮滑者,病日进;人迎沉而滑者,病日损。其脉口滑以沉者,病日进,在内;其人迎脉滑盛以浮者,其病日进,在外。脉之浮沉及人迎与寸口气小大等者,病易已。病之在脏,沉而大者,易已,小为逆;病在腑,浮而大者,其病易已。人迎盛坚者,伤于寒;气口盛坚者,伤于食。\\
雷公曰:以色言病之间甚,奈何?\\
黄帝曰:其色粗以明,沉夭者为甚。其色上行者,病益甚,其色下行,如云彻散者,病方已。五色各有藏部,有外部,有内部也。色从外部走内部者,其病从外走内;其色从内走外者,其病从内走外。病生于内者,先治其阴,后治其阳。反者益甚。其病生于阳者,先治其外,后治其内。反者益甚。其脉滑大以代而长者,病从外来。目有所见,志有所恶,此阳气之并也,可变而已。\\
雷公曰:小子闻风者,百病之始也;厥逆者,寒湿之起也。别之奈何?\\
黄帝曰:常候阙中,薄泽为风,冲浊为痹,在地为厥。此其常也。各以其色言其病。\\
雷公曰:人不病卒死,何以知之?\\
黄帝曰:大气入于脏腑者,不病而卒死矣。\\
雷公曰:病小愈而卒死者,何以知之?\\
黄帝曰:赤色出两颧,大如母指者,病虽小愈,必卒死。黑色出于庭,大如母指,必不病而卒死。\\
雷公再拜曰:善哉!其死有期乎?\\
黄帝曰:察色以言其时。\\
雷公曰:善乎!愿卒闻之。\\
黄帝曰:庭者,首面也;阙上者,咽喉也;阙中者,肺也;下极者,心也;直下者,肝也;肝左者,胆也;下者,脾也;方上者,胃也;中央者,大肠也;挟大肠者,肾也;当肾者,脐也;面王以上者,小肠也;面王以下者,膀胱、子处也;颧者,肩也;颧后者,臂也;臂下者,手也;目内眦上者,膺乳也;挟绳而上者,背也;循牙车以下者,股也;中央者,膝也;膝以下者,胫也;当胫以下者,足也;巨分者,股里也;巨屈者,膝膑也。此五藏六府肢节之部也,各有部分。有部分,用阴和阳,用阳和阴。当明部分,万举万当。能别左右,是谓大道。男女异位,故曰阴阳。审察泽夭,谓之良工。\\
沉浊为内,浮泽为外。黄赤为风,青黑为痛,白为寒。黄而膏润为脓,赤甚者为血。痛甚为挛,寒甚为皮不仁。五色各见其部,察其浮沉,以知浅深。察其泽夭,以观成败。察其散抟,以知远近。视色上下,以知病处。积神于心,以知往今。故相气不微,不知是非。属意勿去,乃知新故。色明不粗,沉夭为甚,不明不泽,其病不甚。其色散,驹驹然,未有聚;其病散而气痛,聚未成也。\\
肾乘心,心先病,肾为应。色皆如是。\\
男子色在于面王,为小腹痛,下为卵痛。其圜直为茎痛。高为本,下为首。狐疝毌阴之属也。\\
女子在于面王,为膀胱、子处之病。散为痛,抟为聚。方员左右,各如其色形。其随而下至胝为淫。有润如膏状,为暴食不洁。\\
左为左,右为右。其色有邪,聚散而不端。面色所指者也。色者,青、黑、赤、白、黄,皆端满有别乡。别乡赤者,其色赤,大如榆荚,在面王为不日。其色上锐,首空上向,下锐下向,在左右如法。以五色命藏,青为肝,赤为心,白为肺,黄为脾,黑为肾。肝合筋,心合脉,肺合皮,脾合肉,肾合骨也。\\
雷公问黄帝说:观察面部的五色,仅是取决于明堂吗?我还不太了解。\\
黄帝说:明堂,就是鼻;阙,就是两眉之间;天庭,就是额部;蕃,就是两颊之侧;蔽,就是耳门。这些部位之间,端正丰厚,在十步之外,一望而见。这样的人,一定会享百岁高寿。\\
雷公问:五官各部的病色应怎样辨别呢?\\
黄帝说:鼻骨高而隆起,正而且直。五脏部位,依次排列在鼻部的中央,六腑挟附在它的两旁。在上的阙中和天庭,主头面;在两目之间的下极,主心之王宫。当胸中五脏安和,相应部位就会出现正常色泽,看不到病色。鼻部的色泽,显得清润。这样,五官的病色,哪会辨别不出来呢?\\
雷公问:还有不这样辨别的,可以听听吗?\\
黄帝说:五脏病色都有一定的显现部位,如该部的不正气色,有深陷入骨的征象,必然要患病。如它的部色,有彼此相生的征象,就是病情严重,也不会死亡。\\
雷公问:五色所主的是什么?\\
黄帝说:青黑主痛,黄赤主热,白主虚寒。这就是五色所主。\\
雷公问:疾病加重和病邪将衰,怎样去认识呢?\\
黄帝说:应该色脉结合,全面诊察。按切病人的脉口,出现滑、小、紧、沉的,其病会日趋严重,这是病在五脏;人迎脉气,出现大、紧、浮的,其病情也会日趋严重,这是病在六腑。若脉口部脉现浮滑的,病就日渐加重;人迎脉现沉而滑的,病就日渐轻减。如脉口部脉现滑而沉的,病就日加严重,属于五脏病;如人迎部脉现滑盛而浮的,病也会日加严重,属于六腑病。至于脉象或沉或浮及人迎和脉口部的小大相等的,病就容易好。病在五脏,脉现沉而大的,病就容易好;脉现沉而小的,就是逆象;病在六腑,脉现浮而大的,病就容易好。人迎主表,脉现盛而坚的,是伤于寒;脉口主里,脉现盛而坚的,是伤于食。\\
雷公问:从面部病色来判断病情轻重,怎样呢?\\
黄帝说:如病人面部色泽微亮的是病轻,沉滞晦暗的是病重。如病色向上走的,病就加重;如病色向下走,像浮云散去的,病就要好了。五脏的病色,各有脏腑的部位,有属于外部的六腑,有属于内部的五脏。病色从外部走向内部的,是病邪从表入里;病色从内部走向外部的,是病邪从里出表。病生于里的,先治其脏,后治其腑。治反了,病就更加严重。病生于外的,先治其表,后治其里。治反了,病就更加严重。脉象滑大或代或长,是病邪从外而来。目有妄见,神志反常,这是阳盛之病,可以泻阳补阴,病就会好的。\\
雷公问:我听说风邪是百病的起因;厥痹是由于寒湿之气所致。从色泽怎样辨别呢?\\
黄帝说:这应该观察眉间的气色,色现浮薄光泽的是风病,色现沉滞晦浊的是痹病,病色出现在面的下部是厥病。这是一般规律。总的说来,要分别根据色泽说明病变。\\
雷公问:有的人没有病象而突然死亡,怎样预知呢?\\
黄帝说:大邪之气侵入脏腑,虽然没有病象,也会突然死亡的。\\
雷公问:病稍微见好,而突然死亡的,怎样预知呢?\\
黄帝说:赤色出现在两颧上,如拇指大,病虽稍微好转,还会突然死亡;黑色出现在天庭,如拇指大,虽没有显著病象,也会突然死亡。\\
雷公再拜问道:说得好!那猝死的人,能预知死期吗?\\
黄帝说:观察面部色泽的变化,可以断定死亡的时日。\\
雷公说:好呀!我希望完全知道。\\
黄帝说:天庭,主头面病;眉心之上,主咽喉病;眉心,主肺脏病;两目之间,主心脏病;由两目之间直下的鼻柱的部位,主肝脏病;在这部位的左面,主胆病;从鼻柱以下的鼻准之端,主脾脏病;挟鼻准之端而略上,主胃病;面之中央,主大肠病;挟两颊部,主肾脏病;当肾脏所属颊部的下方,主脐部病;在鼻准的上方两侧,主小肠病;在鼻准以下的人中部,主膀胱和子宫病;至于各部所主的四肢疾病,就是颧骨主肩;颧骨的后方主臂;在此之下主手;眼内角的上方,主胸部和乳部;颊的外部以上应背;沿牙车以下之处,主大腿部;两牙床的中央部位,主膝部;膝以下的部位,主胫部;由胫以下,主足部;口角大纹处,主大腿内侧;颊下曲骨的部位,主膝盖骨。以上是五脏六腑肢体分布在面部的情况,各有一定的部位。在治疗时,用阴和阳,用阳和阴。只要审明各部分所表现的色泽,就会诊治不失。能够辨别阳左阴右,就了解阴阳的变化规律。男女病色的顺逆,其位置是不同的,所以说必须了解阴阳的规律。再观察面色的润泽和晦滞,从而诊断出疾病的好坏,这就是高明的医生。\\
面色沉滞晦浊的是在里在脏的病,浅浮光亮的是在表在腑的病。色见黄赤属于热,色见青黑属于痛,色见白属于寒。黄而油亮的是疮疡将要化脓,深红的是有淤血。痛极就会拘挛,受寒重就出现皮肤麻木。五色表现在各部位上,观察它的或浮或沉,可以知道疾病的浅深。观察它的光润和枯滞,可以看出病情的或好或坏。观察它的散在和聚结,可以知道病程的或远或近。观察病色的在上在下,可以知道病变部位。全神贯注,心中了了,可以知道病的过去和现在。因此观察病色,如不仔细,就不知道病的虚实。专心致志,毫不走神,才能了解病情的过去和目前情况。面色光亮而不粗糙,病就不会太重。面色既不明亮,又不润泽,而显得沉滞晦暗的,病就比较严重;若其色散而不聚在固定的地方,则其病势也要消散,仅有气痛,还没成为积聚。\\
肾的邪气侵犯心脏,是因为心脏先有了病,肾的黑色,相应出现在心所属的部位上。一般说,病色的出现,都像这样。\\
男子病色出现在鼻准的上方,主小腹疼痛,下引睾丸作痛。若病色出现在圜直的人中沟上,就会发生阴茎作痛。在人中的上半部,主茎根病痛;在人中下半部,主茎头作痛。这是属于狐疝、阴伒一类的病。\\
女子病色出现在鼻准的上方,主膀胱与子宫病。病色散在的主痛,病色集结的主积聚。积聚或方或圆、或左或右,分别像病色在外面所显现的形状。如其色随着下行至唇部,就会有淫浊疾患。如面色光润如脂的,那是暴食,或是吃了不洁食物的象征。\\
病色见于左,是左侧有病;病色见于右,是右侧有病。如面部有病色,或聚或散而不正的,一如面色所指,就可知道发病的脏腑。所谓五色,就是青、黑、赤、白、黄,它的色泽都是端正充润,表现在所属部位。有时也会出现在其他部位上,如心的赤色不出现在心所属的部位,而出现在面王部位上,色深的,大如榆荚,不多天内,病情就会有变化。如果它的病色形状,在上的边缘尖锐,是因为头部气虚,病邪就会向上发展;在下的边缘尖锐,病邪就会向下发展;尖端的在左在右,都可以根据这个原则去测候病邪的发展趋向。以五色与五脏相应的关系来说:青色属肝,赤色属心,白色属肺,黄色属脾,黑色属肾。肝与筋相配合,心与脉相配合,肺与皮相配合,脾与肉相配合,肾与骨相配合。\\
论勇第五十\\
黄帝问于少俞曰:有人于此,并行并立,其年之长少等也,衣之厚薄均也,卒然遇烈风暴雨,或病或不病,或皆病,或皆不病,其故何也?\\
少俞曰:帝问何急?\\
黄帝曰:愿尽闻之。\\
少俞曰:春温风,夏阳风,秋凉风,冬寒风。凡此四时之风者,其所病各不同形。\\
黄帝曰:四时之风,病人如何?\\
少俞曰:黄色薄皮弱肉者,不胜春之虚风;白色薄皮弱肉者,不胜夏之虚风;青色薄皮弱肉者,不胜秋之虚风;赤色薄皮弱肉者,不胜冬之虚风也。\\
黄帝曰:黑色不病乎?\\
少俞曰:黑色而皮厚肉坚,固不伤于四时之风。其皮薄而肉不坚,色不一者,长夏至而有虚风者,病矣。其皮厚而肌肉坚者,长夏至而有虚风,不病矣。其皮厚而肌肉坚者,必重感于寒,外内皆然,乃病。\\
黄帝曰:善。\\
黄帝曰:夫人之忍痛与不忍痛者,非勇怯之分也。夫勇士之不忍痛者,见难则前,见病则止;夫怯士之忍痛者,闻难则恐,遇痛不动。夫勇士之忍痛者见难不恐,遇痛不动;夫怯士之不忍痛者,见难与痛,目转而盻,恐不能言,失气惊,颜色变化,乍死乍生。余见其然也,不知其何由,愿闻其故。\\
少俞曰:夫忍痛与不忍痛者,皮肤之薄厚,肌肉之坚脆缓急之分也,非勇怯之谓也。\\
黄帝曰:愿闻勇怯之所由然。\\
少俞曰:勇士者,目深以固,长衡直扬,三焦理横,其心端直,其肝大以坚,其胆满以傍,怒则气盛而胸张,肝举而胆横,眦裂而目扬,毛起而面苍,此勇士之由然者也。\\
黄帝曰:愿闻怯士之所由然。\\
少俞曰:怯士者,目大而不减,阴阳相失,三焦理纵,巿旡短而小,肝系缓,其胆不满而纵,肠胃挺,胁下空。虽方大怒,气不能满其胸,肝肺虽举,气衰复下,故不能久怒,此怯士之所由然者也。\\
黄帝曰:怯士之得酒,怒不避勇士者,何藏使然?\\
少俞曰:酒者,水谷之精,熟谷之液也,其气慓悍,其入于胃中,则胃胀,气上逆,满于胸中,肝浮胆横。当是之时,固比于勇士,气衰则悔。与勇士同类,不知避之,名曰酒悖也。\\
黄帝问于少俞说:假使有几个人在这里,同行同立,年龄大小相同,穿的衣服厚薄也相同,突然遭到狂风暴雨,有的生病,有的不生病,或者都生病,或者都不生病,这是什么缘故?\\
少俞说:你先问哪一个问题呢?\\
黄帝说:我都想听一听。\\
少俞说:春季当令的是温风,夏季是热风,秋季是凉风,冬季是寒风。大凡四季之风,性质不同,影响到人体发病的情况也不一样。\\
黄帝问:四季之风,使人发病情况如何呢?\\
少俞说:色黄皮薄而肌肉柔弱的人,经不住春天的虚邪贼风;色白皮薄肌肉柔弱的人,经不住夏天的虚邪贼风;色青皮薄肌肉柔弱的人,经不住秋天的虚邪贼风;色赤皮薄肌肉柔弱的人,经不住冬天的虚邪贼风。\\
黄帝问:色黑的人不生病吗?\\
少俞说:色黑皮厚、肉坚的人,就不会被四季虚邪贼风所伤。如果其人皮肤薄弱,肌肉不坚实,肤色不一定,到了长夏的季节,遇到了虚邪贼风就会生病。如果其人色黑皮厚,肌肉坚实,虽遇到长夏季节的虚风,也不会发病。皮厚,肌肉坚实的人必须既感于风,又感于寒,内外都受伤,就不免生病了。\\
黄帝说:讲得好。\\
黄帝说:人能够忍受疼痛与否,并不是单从性格的勇敢和怯弱来分得。有些勇敢的人而不能耐受疼痛,而遇到危难却能勇往直前,而当遭到疼痛时,则退缩不前;有些怯弱的人能耐受疼痛,听到有危难的事就恐惧不安,而遇到疼痛时,却能忍受坚持不动。有些勇敢且能忍受疼痛的人,见到危难不恐惧,遭到疼痛能忍受;有些怯弱而又不能忍受疼痛的人,见到危难与疼痛,眼珠转动,怒目而视,但吓得不敢说话,心惊气促,吓得变了面色,疑死疑生。我看到这种情况,不知是什么原因,想听听其中的道理。\\
少俞说:忍痛与不忍痛,与皮肤的厚薄、肌肉的坚实、脆弱、或松或紧的不同有关,并不是勇敢、怯弱能说明的。\\
黄帝说:我希望听听人为什么会有勇敢和怯懦的不同性格。\\
少俞说:勇敢的人,目光深邃而凝视不动,长眉竖起,肌肉纹理粗横,心脏端正,肝脏大而坚实,胆囊盛满,在发怒时,会气盛而胸张,肝叶上举而胆横,眼眶欲裂,目光四射,毛发竖起,面现青色,这些是勇士性格的内在因素和外在表现。\\
黄帝说:希望听听怯懦人的性格是怎样产生的。\\
少俞说:怯懦的人两目虽大,但不深固,阴阳不协调,肌肉纹理纵而不横,胸骨剑突短而小,肝脏缓纵,胆汁不满,胆囊松弛,肠胃不强健,少弯曲而挺直,胁下空虚而肝气不能充满。虽正大怒,怒气也不能充满胸中,肝肺之叶虽能上举,但不能持久,怒气消失,又垂下了,所以不能长时间发怒。这些是怯士性格的内在因素和外在表现。\\
黄帝问:怯懦的人酒后发怒,也和勇士差不多,这是哪一脏的功能使他这样的呢?\\
少俞说:酒是水谷的精华,熟谷的液汁。其气迅利猛急,酒入胃中,会胃部胀满,气机上逆,充满于胸中,使肝气浮动,胆气恣横。在酒醉时,他的言谈举止,固然和勇士差不多,但酒劲一过,则怯态如故,反而后悔自己的冲动行为。这种酒醉后的言谈举止,看上去好像勇士一样,但不知怎样去做,所以称为酒悖。\\
背腧第五十一\\
黄帝问于岐伯曰:愿闻五脏之腧,出于背者。\\
岐伯曰:胸中大腧在杼骨之端,肺腧在三焦之间,心腧在五焦之间,膈腧在七焦之间,肝腧在九焦之间,脾腧在十一焦之间,肾腧在十四焦之间,皆挟脊相去三寸所,则欲得而验之,按其处,应在中而痛解,乃其腧也。灸之则可,刺之则不可。气盛则泻之,虚则补之。以火补者,毋吹其火,须自灭也;以火泻者,疾吹其火,传其艾,须其火灭也。\\
黄帝问岐伯说:我想了解五脏出于背部的腧穴。\\
岐伯说:胸中的大腧是在项后第一椎下,肺腧在第三椎下,心腧在第五椎下,膈腧在第七椎下,肝腧在第九椎下,脾腧在第十一椎下,肾腧在第十四椎下,这些穴位,都在脊骨的两旁,左右穴位相距三寸,要确定这些穴位,检验的方法是:用手按其腧穴部位,病人感到痠麻、胀痛,或者原有疼痛不适,通过按压而缓解,就是穴位所在。这些腧穴,在治疗上以灸法为宜,不可妄用针刺。在用灸时,邪气盛的可用泻法,正气虚的可用补法。用艾火补时,艾火燃着后,不要吹火,让它慢慢燃烧,待其自灭;用艾火泻时,艾火燃着后,迅速吹旺其火,随即加上艾炷再灸,使之急燃而迅速熄灭。\\
卫气第五十二\\
黄帝曰:五脏者,所以藏精神魂魄者也;六腑者,所以受水谷而行化物者也。其气内于五脏,而外络肢节。其浮气之不循经者,为卫气。其精气之行于经者,为营气。阴阳相随,外内相贯,如环之无端,亭亭淳淳乎,孰能穷之。然其分别阴阳,皆有标本虚实所离之处。能别阴阳十二经者,知病之所生;候虚实之所在者,能得病之高下;知六腑之气街者;能知解结绍于门户。能知虚石之坚软者,知补泻之所在;能知六经标本者,可以无惑于天下。\\
岐伯曰:博哉圣帝之论!臣请尽意悉言之。足太阳之本。在跟以上五寸中,标在两络命门。命门者,目也。足少阳之本,在窍阴之间,标在窗笼之前。窗笼者,耳也。足少阴之本,在内踝下上三寸,标在背腧与舌下两脉也。足厥阴之本,在行间上五寸所,标在背腧也。足阳明之本,在厉兑,标在人迎颊挟颃颡也。足太阴之本,在中封前上四寸之中,标在背腧与舌本也。\\
手太阳之本,在外踝之后,标在命门之上一寸也。手少阳之本,在小指次指之间上二寸,标在耳后上角下外眦也。手阳明之本,在肘骨中,上至别阳,标在颜下合钳上也。手太阴之本,在寸口之中,标在腋内动也。手少阴之本,在锐骨之端,标在背腧也。手心主之本,在掌后两筋之间二寸中,标在腋下三寸也。凡候此者,下虚则厥,下盛则热,上虚则眩,上盛则热痛。故实者绝而止之,虚者引而起之。\\
请言气街:胸气有街,腹气有街,头气有街,胫气有街。故气在头者,止之于脑;气在胸者,止之膺与背腧;气在腹者,止之背腧,与冲脉于脐左右之动脉者。气在胫者,止之于气街,与承山踝上以下。取此者用毫针,必先按而在久应于手,乃刺而予之。所治者,头痛眩仆,腹痛中满暴胀,及有新积。痛可移者,易已也,积不痛,难已也。\\
黄帝说:五脏是储藏精神魂魄的;六腑是受纳水谷和运输消化之物的。六腑运输的水谷精微之气,在内进入五脏,在外络于肢节。其中浮于脉外之气,不沿经脉循行的,叫卫气;其中精气循行经脉之中的,叫营气。营卫阴阳相随而行,内外贯通,有如圆环没有开端,如水之源远流长,无有穷尽。但从阴阳属性来说,都有标本、虚实判断的标准。能分别三阴三阳十二经的就可以知道病是怎样产生的;能判断虚实所在,就能找出疾病的上下部位;能知道六腑之气往来的通道,就能知道解开结聚,疏通经穴;能知道虚实的软硬属性,就能知道补虚泻实的关键所在;能知手足六经的标和本,在治疗复杂的疾病时就能应对自如了。\\
岐伯说:圣帝的议论博大极了!我愿尽我所知的尽量地说出来。足太阳膀胱经之本,在足跟以上五寸中的跗阳穴;标在两目的睛明穴。命门指眼睛。足少阳胆经之本,在足第四趾外侧端的窍阴穴之间;标在窗笼之前,即耳前的听宫穴。足少阴肾经之本,在内踝下三寸的交信穴;标在背部的肾腧穴以及舌下两脉的廉泉穴。足厥阴肝经之本,在行间穴上五寸的中封穴;标在背部的肝腧穴。足阳明胃经之本,在足次趾端的厉兑穴;标在颊下结喉两旁的人迎穴。足太阴脾经之本,在中封穴前上四寸的三阴交穴;标在背部的脾与舌根。\\
手太阳小肠经之本,在手外踝之后的养老穴;标在睛明穴上一寸处。手少阳三焦经之本,在手无名指之间的液门穴,标在耳后上角的角孙穴与下外眦的丝竹空穴。手阳明大肠经之本,在肘骨中的曲池穴,上至臂臑穴处,标在颊下一寸,人迎之后,扶突之上。\\
手太阴肺经之本,在寸口中的太渊穴;标在腋内动脉,即腋下三寸的天府穴处。手少阴心经之本,在掌后锐骨之端的神门穴,标在背部的心腧穴。手厥阴心包经之本,在掌后两筋之间二寸的内关穴,标在腋下三寸的天池穴处。凡要观察十二经标本上下的病变,一般在下的为本,下虚则发为厥逆,下盛则为热痛;在上者为标,上虚则为眩晕,上盛则为热痛。实证当泻,杜绝邪气,止其发作;虚证当补,助其正气而振其不足。\\
我再谈谈气街吧:胸气有它的道路,腹气有它的道路,头气有它的道路,胫气有它的道路。气在头部的,聚于脑;气在胸部的,聚于胸之两旁的膺部和背腧;气在腹部的,聚于背腧,与腹部冲脉在脐左右的经脉搏动之处;气在胫部的,聚于足阳明经的气街穴及足太阳经的承山穴和足踝部上下等处。凡刺这些穴位都要用毫针,操作时,必须先用手长时间按压穴位,待其气至,然后针刺与之补泻。刺气街能治疗头痛、眩晕、跌仆、腹痛、中满、腹部突然胀满,及新得的积聚。疼痛按之移动的,治之易愈;积证不疼痛的,难愈。\\
论痛第五十三\\
黄帝问于少俞曰:筋骨之强弱,肌肉之坚脆,皮肤之厚薄,腠理之疏密,各不同,其于针石火焫之痛何如?肠胃之厚薄坚脆亦不等,其于毒药何如?愿尽闻之。\\
少俞曰:人之骨强、筋弱、肉缓、皮肤厚者耐痛,其于针石之痛,火焫亦然。\\
黄帝曰:其耐火焫者,何以知之?\\
少俞答曰:加以黑色而美骨者,耐火焫。\\
黄帝曰:其不耐针石之痛者,何以知之?\\
少俞曰:坚肉薄皮者,不耐针石之痛,于火焫亦然。\\
黄帝曰:人之病,或同时而伤,或易已,或难已,其故何如?\\
少俞曰:同时而伤,其身多热者易已,多寒者难已。\\
黄帝曰:人之胜毒,何以知之?\\
少俞曰:胃厚、色黑、大骨及肥者,皆胜毒;故其瘦而薄胃者,皆不胜毒也。\\
黄帝问少俞说:人体筋骨的强弱,肌肉的坚脆,皮肤的厚薄,腠理的疏松和致密,各不同,他们对于针石和艾火灸灼所致疼痛的耐受性怎样呢?人的肠胃的厚薄、坚脆也不同,他们对药物的耐受性怎样呢?希望都听听。\\
少俞说:骨强、筋软弱、肌肉舒缓、皮肤厚实的人,能耐受疼痛,这种人对针刺或艾火烧灼所致疼痛的耐受力也一样。\\
黄帝问:怎样知道有人能耐受艾火的灼痛呢?\\
少俞回答说:骨强筋弱肉缓皮肤厚,而加上皮肤色黑、骨骼强劲的人,能耐艾火的灼痛。\\
黄帝问:怎样知道有人不能耐受针刺疼痛呢?\\
少俞说:肉坚而皮薄的人,不能耐受针刺疼痛,对于艾火疼痛也不能耐受。\\
黄帝问:人们患病,有些是同时为外邪所伤,有的人容易痊愈,有的人不易痊愈,是什么道理呢?\\
少俞说:同时为外邪所伤,身体多热的,容易痊愈;身体多寒的,就不易痊愈。\\
黄帝问:怎样知道人对药物耐受力的大小呢?\\
少俞说:胃厚、皮色黑、骨骼粗及肥壮的人,耐药力较强;身体瘦弱而胃薄的人,耐药力就差。\\
天年第五十四\\
黄帝问于岐伯曰:愿闻人之始生,何气筑为基?何立而为楯?何失而死?何得而生?\\
岐伯曰:以母为基,以父为楯。失神者死,得神者生也。\\
黄帝曰:何者为神?\\
岐伯曰:血气已和,荣卫已通,五脏已成,神气舍心,魂魄毕具,乃成为人。\\
黄帝曰:人之寿夭各不同,或夭或寿,或卒死,或病久,愿闻其道。\\
岐伯曰:五脏坚固,血脉和调。肌肉解利,皮肤致密。营卫之行,不失其常。呼吸微徐,气以度行。六腑化谷,津液布扬。各如其常,故能长久。\\
黄帝曰:人之寿百岁而死,何以致之?\\
岐伯曰:使道隧以长,基墙高以方。通调营卫,三部三里起。骨高肉满,百岁乃得终。\\
黄帝曰:其气之盛衰,以至其死,可得闻乎?\\
岐伯曰:人生十岁,五脏始定,血气已通,其气在下,故好走。二十岁,血气始盛,肌肉方长,故好趋。三十岁,五脏大定,肌肉坚固,血脉盛满,故好步。四十岁,五脏六腑十二经脉,皆大盛以平定。腠理始疏,荣华颓落,发颇斑白,平盛不摇,故好坐。五十岁,肝气始衰,肝叶始薄,胆汁始减,目始不明。六十岁,心气始衰,苦忧悲,血气懈隋,故好卧。七十岁,脾气虚,皮肤枯。八十岁,肺气衰,魄离,故言善误。九十岁,肾气焦,四脏经脉空虚。百岁,五脏皆虚,神气皆去,形骸独居而终矣。\\
黄帝曰:其不能终寿而死者,何如?\\
岐伯曰:其五脏皆不坚,使道不长。空外以张,喘息暴疾。又卑基墙,薄脉少血,其肉不石。数中风寒,血气虚,脉不通。真邪相攻,乱而相引。故中寿而尽也。\\
黄帝问岐伯说:人在生命开始的时候,是以什么为基础?以什么作为外卫?失去什么就会死亡?得到什么才会生存呢?\\
岐伯说:以母为基础,以父为外卫。没了神气就会死亡,有了神气才能生存。\\
黄帝问:什么叫神呢?\\
岐伯说:血气已经和调,荣卫已经通畅,五脏已经形成,神气潜藏于心,魂魄具备了,就成为人。\\
黄帝说:人的年岁长短各不相同,有的命短,有的寿长,有的突然死亡,有的患病日久,希望听到其中的道理。\\
岐伯说:五脏形质坚固,血脉和顺协调。肌肉滑润,皮肤细密。营卫之气的运行,不背离常规。呼吸徐缓,经气循度而行。六腑消化谷物,津液布散周身。以上各方面,都能正常活动,寿命就能长久。\\
黄帝问:人怎样才能活到百岁而死呢?\\
岐伯说:长寿者的人中沟深而长,鼻的部位,高大方正。营卫循行畅通无阻,面部的三停高起而不平陷,骨骼高起,肌肉丰满,这种健壮的形体,是能活到百岁的象征。\\
黄帝问:人的体气盛衰,从幼年直到死亡,可以听听吗?\\
岐伯说:人生到十岁,五脏才开始健全,血气已经通畅,这时他的经气,还在下肢,所以喜跑。到了二十岁,血气开始旺盛,肌肉正在发达,所以喜快走。到了三十岁,五脏完全健全,肌肉坚固,血脉盛满,所以喜欢缓行。到了四十岁,五脏六腑和十二经脉已发育很好,并且稳定。腠理开始稀疏,面部华色开始衰落,发鬓斑白,经气平定盛满至极,精力已不十分充足,所以好坐。到了五十,肝气开始衰退,肝叶薄弱,胆汁逐渐减少,眼睛开始有不明的感觉。到了六十岁,心气开始衰退,经常有忧虑悲伤之苦,血气运行缓慢,所以喜欢躺卧。到了七十岁,脾气虚弱,皮肤干枯。到了八十岁,肺气衰退,魂魄离散,所以言语常常错误。到了九十岁,肾气焦竭,肝、心、脾、肺四脏和经脉都空虚了。到了百岁,五脏就都空了,神气也都没有了,这时,就仅留下形体而死亡了。\\
黄帝问:有人不能享尽天年就死了,是为什么?\\
岐伯说:那是五脏都不坚实,人中不长。鼻孔向外张开,呼吸急速。鼻梁骨低,脉小血少,肌肉不坚实。屡受风寒,血气虚弱,经脉不通。正邪相攻,体内血气失常,引邪深入。所以中年就会死。\\
逆顺第五十五\\
黄帝问于伯高曰:余闻气有逆顺,脉有盛衰,刺有大约,可得闻乎?\\
伯高曰:气之逆顺者,所以应天地阳阳、四时、五行也,脉之盛衰者,所以候血气之虚实有余不足也。刺之大约者,必明知病之可刺,与其未可刺,与其已不可刺也。\\
黄帝曰:候之奈何?\\
伯高曰:《兵法》曰:无迎逢逢之气,无击堂堂之阵。《刺法》曰:无刺熇熇之热;无刺漉漉之汗;无刺浑浑之脉;无刺病与脉相逆者。\\
黄帝曰:候其可刺,奈何?\\
伯高曰:上工,刺其未生者也;其次,刺其未盛者也;其次,刺其已衰者也。下工,刺其方袭者也,与其形之盛者也,与其病之与脉相逆者也。故曰:“方其盛也,勿敢毁伤,刺其已衰,事必大昌。”故曰:“上工治未病,不治已病。”此之谓也。\\
黄帝问伯高说:我听说气的运行有逆有顺,血脉有盛有衰,针刺有大法,能听听吗?\\
伯高说:气行的逆与顺,是和自然界的阴阳、四时、五行相适应的;脉象的盛衰是用以诊察气血的虚实有余不足的。针刺的大法是必须明确掌握疾病可以刺、还是不可以刺、或已经到了不可针刺的程度这三种情况。\\
黄帝问:怎样诊察可刺与不可刺呢?\\
伯高说:《兵法》上曾说:作战时当敌方来势凶猛,气焰正盛时,不可迎击其锐势,面对敌军盛大整齐的阵容,不可冒然出击。《刺法》上说:热势炽盛的,不可刺;大汗淋漓的,不可刺;脉象模糊浊乱时,不可刺;脉象和病情不符的,不可刺。\\
黄帝说:怎样诊察可刺的时机呢?\\
伯高说:高明的医生在疾病未发作之前针刺;其次,在病发,但邪气未盛时针刺;再次,在邪气已衰,正气欲复时针刺。技术低劣的医生,却在邪气正旺时针刺,或刺外形貌似强盛而实则内虚的人,或刺病情与脉象不符的人。所以医经上说:“在邪气盛的时候不能针刺,在邪气衰退时针刺,就会取得很好的疗效。”所以《四气调神大论》中说:“高明的医生是在未病之前预先防治,并不是已经发病才去治疗的。”就是这个道理。\\
五味第五十六\\
黄帝曰:愿闻谷气有五味,其入五脏,分别奈何?\\
伯高曰:胃者,五脏六腑之海也。水谷皆入于胃,五脏六腑皆禀气于胃。五味各走其所喜。谷味酸,先走肝;谷味苦,先走心;谷味甘,先走脾;谷味辛,先走肺;谷味咸,先走肾。谷气津液已行,营卫大通,乃化糟粕,以次传下。\\
黄帝曰:营卫之行奈何?\\
伯高曰:谷始入于胃,其精微者,先出于胃之两焦,以溉五脏。别出两行,营卫之道。其大气之抟而不行者,积于胸中,命曰气海。出于肺,循喉咽,故呼则出,吸则入。天地之精气,其大数常出三入一。故谷不入,半日则气衰,一日则气少矣。\\
黄帝曰:谷之五味,可得闻乎?\\
伯高曰:请尽言之。五谷:秔米甘,麻酸,大豆咸,麦苦,黄黍辛。五果:枣甘,李酸,栗咸,杏苦,桃辛。五畜:牛甘,犬酸,猪咸,羊苦,鸡辛。五菜:葵甘,韭酸,藿咸,薤苦,葱辛。\\
五色:黄色宜甘,青色宜酸,黑色宜咸,赤色宜苦,白色宜辛。凡此五者,各有所宜。\\
五宜:所言五宜者,脾病者,宜食粳米饭,牛肉枣葵;心病者,宜食麦,羊肉杏薤;肾病者,宜食大豆黄卷,猪肉栗藿;肝病者,宜食麻,犬肉李韭;肺病者,宜食黄黍,鸡肉桃葱。\\
五禁:肝病禁辛,心病禁咸,脾病禁酸,肾病禁甘,肺病禁苦。\\
肝色青,宜食甘,粳米饭、牛肉、枣、葵,皆甘。\\
心色赤,宜食酸,犬肉、麻、李、韭,皆酸。\\
脾色黄,宜食咸,大豆、豕肉、栗、藿,皆咸。\\
肺色白,宜食苦,麦、羊肉、杏、薤,皆苦。\\
肾色黑,宜食辛,黄黍、鸡肉、桃、葱,皆辛。\\
黄帝问:希望听一下,谷气五味进入五脏后,是怎样转输的呢?\\
伯高说:胃像是五脏六腑营养汇聚的大海。水谷都要进入胃中,因此,五脏六腑都从胃接受水谷的精微之气。饮食物的五味,分别进入它所喜爱之脏。味酸的,先进入肝;味苦的,先进入心;味甘的,先进入脾;味辛的,先进入肺;味咸的,先进入肾。谷气化生的津液,已在体内运行,因而营卫通畅,其中废物就化为糟粕,随着二便由上而下地排出体外。\\
黄帝问:营卫的运行怎样呢?\\
伯高说:水谷入胃后,所化生的精微部分,从胃出后至中上二焦,经肺灌溉五脏。它在输布于全身时,分别为两条途径,其清纯部分化为营气,浊厚部分化为卫气,分别从脉内外的两条道路运行于周身。同时所产生的大气,则聚于胸中,称为气海。这种气自肺沿咽喉而出,呼则出,吸则入,保证人体正常呼吸运动。天地的精气,它在体内代谢的大概情况,是宗气、营卫和糟粕三方面输出,但另一方面又要从天地间吸入空气与食入饮食物,以补给全身营养的需要。所以半日不吃饭,就会感到气衰,一天不进饮食,就会感到气少了。\\
黄帝说:谷物的五味,可以听听吗?\\
伯高说:我愿意详尽地说一下。在五谷里:秔米味甘,芝麻味酸,大豆味咸,小麦味苦,黄黍味辛。在五果里:枣味甘,李味酸,栗味咸,杏味苦,桃味辛。在五畜里:牛肉味甘,犬肉味酸,猪肉味咸,羊肉味苦,鸡肉味辛。在五菜里:葵菜味甘,韭菜味酸,豆叶味咸,薤白味苦,葱味辛。\\
五种病色所宜之味:黄色适宜甜味,青色适宜酸味,黑色适宜咸味,红色适宜苦味,白色适宜辣味。大凡这五种病色各有适宜之味。\\
五脏病所宜之食:所说的五宜是指脾病宜食粳米饭,大枣和冬葵;心病宜食麦食,羊肉,杏子和薤白;肾病宜食大豆黄卷,猪肉,栗子和藿叶;肝病宜食芝麻,狗肉,李子,韭菜;肺病宜食黄黍,鸡肉,桃子,葱。\\
五脏病禁忌:肝病禁忌辣味,心病禁忌咸味,脾病禁忌酸味,肾病禁忌甜味,肺病禁忌苦味。\\
肝主青色,宜食甜味,粳米饭,牛肉,大枣,冬葵,都是甜味。\\
心主红色,宜食酸味,狗肉,芝麻,李子,韭菜,都是酸味。\\
脾主黄色,宜食咸味,大豆,猪肉,栗子,藿叶,都是咸味。\\
肺主白色,宜食苦味,麦子,羊肉,杏子,薤白,都是苦味。\\
肾主黑色,宜食辣味,黄黍,鸡肉,桃子,大葱,都是辣味。\\
卷九\\
水胀第五十七\\
黄帝问于岐伯曰:水与肤胀、鼓胀、肠覃、石瘕、石水,何以别之?\\
岐伯答曰:水始起也,目窠上微肿,如新卧起之状,其颈脉动,时咳,阴股间寒,足胫瘇,腹乃大,其水已成矣,以手按其腹,随手而起,如裹水之状,此其候也。\\
黄帝曰:肤胀,何以候之?\\
岐伯曰:肤胀者,寒气客于皮肤之间,仜仜然不坚,腹大,身尽肿,皮厚,按其腹窅而不起,腹色不变。此其候也。\\
黄帝曰:鼓胀何如?\\
岐伯曰:腹胀,身皆大,大与肤胀等,色苍黄,腹筋起。此其候也。\\
肠覃何如?\\
岐伯曰:寒气客于肠外,与卫气相搏,气不得荣,因有所系,瘕而内著,恶气乃起,瘜肉乃生。其始生也,大如鸡卵,稍以益大,至其成,如怀子之状,久者离岁,按之则坚,推之则移,月事以时下,此其候也。\\
石瘕何如?\\
岐伯曰:石瘕生于胞中,寒气客于子门,子门闭塞,气不得通,恶血当泻不泻,衃以留止,日以益大,状如怀子,月事不以时下。皆生于女子。可导而下。\\
黄帝曰:肤胀、鼓胀,可刺邪?\\
岐伯曰:先泻其胀之血络,后调其经,刺去其血络也。\\
黄帝问岐伯说:水胀、肤胀、鼓胀、肠覃、石瘕、石水,怎样区别呢?\\
岐伯回答说:水胀病开始时,病人下眼睑微肿,好像刚睡醒的样子,他的人迎脉搏动明显,时时咳嗽,大腿内侧有寒凉之感,足胫浮肿,腹部胀大,具备这些症状,水胀病就已经形成了。以手按他的腹部,松手后,随手而起,好像里面裹着水一样。这就是水胀病的证候。\\
黄帝问:肤胀怎样诊断呢?\\
岐伯说:肤胀病是因寒邪侵入皮肤之间,表现为腹部胀大,用手叩击如鼓之中空不实,全身肿,皮肤厚,用手按其腹部深陷不能随手而起,腹部的皮色没变化。这就是肤胀病的证候。\\
黄帝问:鼓胀的证候怎样呢?\\
岐伯说:鼓胀病的腹部胀大,全身肿胀与肤胀病相同,但鼓胀的肤色青黄,腹部青筋暴露。这是鼓胀病的证候。\\
黄帝问:肠覃病的证候怎样呢?\\
岐伯说:寒邪侵袭,停留在肠外,和卫气相搏,正气不能正常运行营养周身,因而邪气留滞,血淤不通,附著在肠外,病邪逐渐滋长,瘜肉就产生了。这种病开始时像鸡蛋一样大,逐渐长大,等到最后形成时,像怀孕似的,病程长的可以经过几年,用手按压,患部很硬,推之能移动,月经仍能按期来潮。这是肠覃的证候。\\
黄帝问:石瘕的证候怎样呢?\\
岐伯说:石瘕是长在子宫里,因寒气入侵子门,使子门闭阻,气血不通,恶血不能排泄,以致凝结成块,滞留在胞中,逐渐长大,其形状像怀孕一样,月经不能按期来潮。这种病都发生在妇女。可用通导攻下的方法治疗淤血。\\
黄帝问:肤胀和鼓胀,可以针刺吗?\\
岐伯说:首先用针泻淤滞的络脉,然后调理经脉,但必须先刺去血络中的淤血。\\
贼风第五十八\\
黄帝曰:夫子言贼风邪气之伤人也,令人病焉。今有其不离屏蔽,不出空穴之中,卒然病者,非不离贼风邪气,其故何也?\\
岐伯曰:此皆尝有所伤于湿气,藏于血脉之中,分肉之间,久留而不去;若有所堕坠,恶血在内而不去。卒然喜怒不节,饮食不适,寒温不时,腠理闭而不通。其开而遇风寒,则血气凝结,与故邪相袭,则为寒痹。其有热则汗出,汗出则受风。虽不遇贼风邪气,必有因加而发焉。\\
黄帝曰:今夫子之所言者,皆病人之所自知也。其毋所遇邪气,又毋怵惕之所志,卒然而病者,其故何也?唯有因鬼神之事乎?\\
岐伯曰:此亦有故邪留而未发,因而志有所恶,及有所慕,血气内乱,两气相搏。其所从来者微,视之不见,听而不闻,故似鬼神。\\
黄帝曰:其祝而已者,其故何也?\\
岐伯曰:先巫者,因知百病之胜,先知其病之所从生者,可祝而已也。\\
黄帝问:您说过四时不正之气伤害人体,使人生病。可是有人不离开屏风,亦不出屋中,忽然生病,并不是没有避开贼风邪气,这是什么缘故呢?\\
岐伯说:这都是曾经为湿邪所伤,湿邪蕴藏在血脉和分肉之内,长久留止而不能排除;或者有因堕落,淤血在内未散。忽然喜怒过度,饮食不适宜,寒温不调,致使腠理闭塞,壅滞不通。或在腠理开张之时,遭遇风寒,就会使血气凝聚,以前湿邪和新感风寒相合,就成为寒痹。或有因热出汗,出汗时受了风。以上这些,虽然没有遇到贼风邪气,也会因为原有宿邪加上新感之邪而发病。\\
黄帝问:像夫子您所说的这些,都是病人自己所知道的。那些没有遭到四时不正之气,也没有恐惧等情志刺激,忽然就发病了,是什么缘故?是真有鬼神作祟吗?\\
岐伯说:这也是先有宿邪留在体内,还没发作,由于思想上有厌烦的事,或向往的事,不能遂心,以致血气不和,新病与宿邪相搏,所以突然发病。它的病因极为微妙,既看不见,也听不见,所以像有鬼神作祟一样。\\
黄帝问:那些用祝由术而治好的病,道理何在?\\
岐伯说:前代的巫师,因为懂得各种疾病之间相互制约的关系,首先掌握疾病发生的由来,所以用祝由术能把病治好。\\
卫气失常第五十九\\
黄帝曰:卫气之留于腹中,稸积不行,菀蕴不得常所,使人支胁胃中满,喘呼逆息者,何以去之?\\
伯高曰:其气积于胸中者,上取之;积于腹中者,下取之;上下皆满者,傍取之。\\
黄帝曰:取之奈何?\\
伯高曰:积于上,泻人迎,天突、喉中;积于下者,泻三里与气街;上下皆满者,上下取之,与季胁之下一寸;重者,鸡足取之。诊视其脉,大而弦急,及绝不至者,及腹皮急甚者,不可刺也。\\
黄帝曰:善。\\
黄帝问于伯高曰:何以知皮肉、气血、筋骨之病也。\\
伯高曰:色起两眉薄泽者,病在皮。唇色青黄赤白黑者,病在肌肉。营气濡然者,病在血气。目色青黄赤白黑者,病在筋。耳焦枯受尘垢者,病在骨。\\
黄帝曰:病形何如,取之奈何?\\
伯高曰:夫百病变化,不可胜数,然皮有部,肉有柱,血气有输,骨有属。\\
黄帝曰:愿闻其故。\\
伯高曰:皮之部,输于四末;肉之柱,在臂胫诸阳,分肉之间,与足少阴分间;血气之输,输于诸络,气血留居,则盛而起;筋部无阴无阳,无左无右,候病所在;骨之属者,骨空之所以受液,而益脑髓者也。\\
黄帝曰:取之奈何?\\
伯高曰:夫病变化,浮沉深浅,不可胜穷,各在其处。病间者浅之,甚者深之,间者小之,甚者众之。随变而调气,故曰上工。\\
黄帝问于伯高曰:人之肥瘦大小寒温,有老壮少小,别之奈何?\\
伯高对曰:人年五十已上为老,三十已上为壮,十八已上为少,六岁已上为小。\\
黄帝曰:何以度知其肥瘦?\\
伯高曰:人有肥有膏有肉。\\
黄帝曰:别此奈何?\\
伯高曰:夬肉坚,皮满者,肥;夬肉不坚,皮缓者,膏,皮肉不相离者,肉。\\
黄帝曰:身之寒温,何如?\\
伯高曰:膏者其肉淖;而粗理者身寒,细理者身热。脂者其肉坚;细理者热,粗理者寒。\\
黄帝曰:其肥瘦大小,奈何?\\
伯高曰:膏者,多气而皮纵缓,故能纵腹垂腴。肉者,身体容大。脂者,其身收小。\\
黄帝曰:三者之气血多少,何如?\\
伯高曰:膏者多气,多气者热,热者耐寒。肉者多血则充形,充形则平。脂者,其血清,气滑少,故不能大。此别于众人者也。\\
黄帝曰:众人奈何?\\
伯高曰:众人皮肉脂膏不能相加也,血与气不能相多,故其形不小不大,各自称其身,命曰众人。\\
黄帝曰:善。治之奈何?\\
伯高曰:必先别其三形,血之多少,气之清浊,而后调之,治无失常经。是故膏人,纵腹垂腴;肉人者,上下容大;脂人者,虽脂不能大者。\\
黄帝问:卫气滞留在腹中,蓄积不能正常运行,郁结没有固定部位,使人胁肋支拄,胃部胀满,喘息气逆,怎样除去呢?\\
伯高说:气蓄积在胸中的,取上部的穴位;蓄积在腹中的,取下部的穴位;如果胸腹上下都胀满的,取上下和附近经脉的穴位。\\
黄帝问:具体取哪些穴位呢?\\
伯高说:蓄积在胸中的,取足阳明胃经的人迎穴,及任脉的天突和廉泉穴以泻之;蓄积在腹中的,取足阳明胃经的三里穴和气冲穴以泻之;胸腹部都蓄积的,上下部的穴位都取,并取季胁之下一寸处的章门穴;病重的,采用鸡足针法。若在诊察时,见脉大弦急,或脉绝不至的,以及腹皮绷急的,都不可用针刺。\\
黄帝说:说得好。\\
黄帝问伯高说:怎么诊察皮肉、气血、筋骨的病变呢?\\
伯高说:病色出现在两眉之间,浮薄光泽的,主病在皮。口唇出现青、黄、赤、白、黑之色的,主病在肌肉。营气濡弱而致皮肤湿润而多汗的,主病在血气。目现青、黄、赤、白、黑之色的,主病在筋。耳轮焦枯如尘垢的,主病在骨。\\
黄帝问:疾病的表现怎样,怎么治疗?\\
伯高说:诸多疾病都是千变万化的,不能完全说清楚,但是皮有分部,肉有突起处,血气有输注的隧道,骨有跗属。\\
黄帝说:希望听听其中的道理。\\
伯高说:皮之分部,在于四末;肉之柱,在上臂、下胫的手足六阳经分肉之间,与足少阴经循行通路上的分肉之间;血气之输注,在诸经的络穴,所以气血郁滞,则络脉壅盛而高起;病在筋的,不论其在阴在阳,在左在右,候察发病部位就可以了;病在骨的,当取治于骨之所属,因为骨孔是接受并输注精气而能补益脑髓的。\\
黄帝问:怎样取穴治疗呢?\\
伯高说:疾病变化不一,病位的浮沉,刺治的浅深,难以穷尽,根据发病的具体部位决定治法。病轻的用浅刺,病重的用深刺,病轻的用针宜少,病重的用针宜多。根据病情的变化而调整气机,这就是高明的医生。\\
黄帝问伯高说:人体的肥瘦、身量的大小、体质的寒温,以及年龄的有老、壮、少、小,怎样区别呢?\\
伯高说:年龄五十岁以上为老,三十岁以上为壮,十八岁以上为少,六岁以上为小。\\
黄帝问:根据什么度量人的肥瘦呢?\\
伯高说:人有肥、有膏、有肉的不同。\\
黄帝问:怎样分别这三种类型呢?\\
伯高说:刐肉坚实,皮下丰满的为肥;刐肉不坚实、皮肤弛缓的为膏;皮肉相连不相分离的为肉。\\
黄帝问:体质属寒属温的情况怎样呢?\\
伯高说:属于膏型的人,肌肉柔润;其中,纹理粗疏的,身体多寒,纹理致密的,身体多热。属于脂型的人,肌肉坚厚;其中,纹理致密的,身体多热,纹理粗疏的,身体多寒。\\
黄帝问:人体的肥瘦、大小怎样区别呢?\\
伯高说:膏型的人,阳气充盛,皮肤宽纵弛缓,所以呈现腹肌宽纵,肥肉下垂的形态。肉型的人,身体宽大。脂型的人,肉坚而身形小。\\
黄帝问:这三种类型人气血的多少怎样呢?\\
伯高说:膏型的人多气,多气为阳性体质,阳盛的人能耐寒。肉型的人多血,多血则形体充盛,形体充盛则体质平和。脂型的人,其血清,气滑而少,所以身形不大。这是有别于一般人的三种人的气血多少的情况。\\
黄帝问:一般人的情况怎样呢?\\
伯高说:一般人的皮、肉、脂、膏不能相加,以及血、气都没有偏多,所以形体不大不小,很匀称,这就是一般人。\\
黄帝说:讲得好!怎样治疗呢?\\
伯高说:首先必须辨别三种不同类型的人,掌握各型人血的多少,气的清浊,然后调治,治疗不能违背常规治法。所以膏人的体型是腹肌宽纵,肥肉下垂;肉人的体型是上下肢体都很宽大;脂型的人,虽脂肪多,体型却不大。\\
玉版第六十\\
黄帝曰:余以小针为细物也,夫子乃言上合之于天,下合之于地,中合之于人,余以为过针之意矣,愿闻其故。\\
岐伯曰:何物大于针乎?夫大于针者,惟五兵者焉。五兵者,死之备也,非生之具也。且夫人者,天地之镇也,其可不参乎?夫治民者,亦惟针焉。夫针之与五兵,其孰小乎?\\
黄帝曰:病之生时,有喜怒不测,饮食不节,阴气不足,阳气有余,营气不行,乃发为痈疽。阴阳不通,两热相搏,乃化为脓,小针能取之乎?\\
岐伯曰:圣人不能使化者,为之,邪不可留也。故两军相当,旗帜相望,白刃陈于中野者,此非一日之谋也。能使其民,令行禁止,卒无白刃之难者,非一日之教也,须臾之得也。夫至使身被痈疽之病,脓血之聚者,不亦离道远乎?夫痈疽之生,脓血之成也,不从天下,不从地出,积微之所生也。故圣人自治于未有形也,愚者遭其已成也。\\
黄帝曰:其已形,不予遭,脓已成,不予见,为之奈何?\\
岐伯曰:脓已成,十死一生,故圣人弗使已成,而明为良方,著之竹帛,使能者踵而传之后世,无有终时者,为其不予遭也。\\
黄帝曰:其已有脓血,不以小针治乎?\\
岐伯曰:以小治小者,其功小;以大治大者,其功大;以小治大者,多害。故其已成脓血者,其唯砭石铍锋之所取也。\\
黄帝曰:多害者,其不可全乎?\\
岐伯曰:其在逆顺焉。\\
黄帝曰:愿闻逆顺。\\
岐伯曰:以为伤者,其白眼青黑眼小,是一逆也;内药而呕者,是二逆也;腹痛渴甚,是三逆也;肩项中不便,是四逆也;音嘶色脱,是五逆也。除此五者,为顺矣。\\
黄帝曰:诸病皆有逆顺,可得闻乎?\\
岐伯曰:腹胀,身热,脉大,是一逆也;腹鸣而满,四肢清,泄,其脉大,是二逆也;衄而不止,脉大,是三逆也;咳且溲血,脱形,其脉小劲,是四逆也;咳,脱形、身热,脉小以疾,是谓五逆也。如是者,不过十五日而死矣。\\
其腹大胀,四末清,脱形,泄甚,是一逆也;腹胀便血,其脉大,时绝,是二逆也;咳,溲血,形肉脱,脉搏,是三逆也;呕血,胸满引背,脉小而疾,是四逆也;咳呕腹胀,且飧泄,其脉绝,是五逆也。如是者,不及一时而死矣。工不察此者而刺之,是谓逆治。\\
黄帝曰:夫子之言针甚骏,以配天地,上数天文,下度地纪,内别五脏,外次六腑,经脉二十八会,尽有周纪。能杀生人,不能起死者。子能反之乎?\\
岐伯曰:能杀生人,不能起死者也。\\
黄帝曰:余闻之则为不仁,然愿闻其道,弗行于人。\\
岐伯曰:是明道也,其必然也,其如刀剑之可以杀人,如饮酒使人醉也,虽勿诊,犹可知矣。\\
黄帝曰:愿卒闻之。\\
岐伯曰:人之所受气者,谷也。谷之所注者,胃也。胃者,水谷气血之海也。海之所行云气者,天下也。胃之所出气血者,经隧也。经隧者,五脏六腑之大络也,迎而夺之而已矣。\\
黄帝曰:上下有数乎?\\
岐伯曰:迎之五里,中道而止,五至而已,五往而脏之气尽矣,故五五二十五而竭其输矣,此所谓夺其天气者也,非能绝其命而倾其寿者也。\\
黄帝曰:愿卒闻之。\\
岐伯曰:匜门而刺之者,死于家中;入门而刺之者,死于堂上。\\
黄帝曰:善乎方,明哉道。请著之玉版,以为重宝,传之后世,以为刺禁,令民勿敢犯也。\\
黄帝说:我认为小针是细小之物,先生却说它的作用上合于天,下合于地,中合于人,我认为这是过分夸大了针的意义,希望听听其中的道理。\\
岐伯说:还有什么东西能够比针更大呢?比针大的,惟有五种兵器。但五种兵器都是准备杀人的,不是活人的工具。而且人是天地间最高贵的,怎么可以不参赞天地的化育之德呢?治疗人民的疾病,只有用小针。针和五兵的作用到底哪个小,不是很清楚了吗?\\
黄帝问:疾病的发生,因喜怒无度,饮食无节,阴气不足,阳气有余,以致营气郁滞不行,而发生痈疽。营卫气血阻滞不通,阳热之气与邪热互相搏结,而化为脓,这种病,小针能治疗吗?\\
岐伯说:圣人不能使邪气消失自化,要及早治疗,因为病邪不可久留体内。譬如两军对敌,旗帜相望,刀光剑影遍于旷野,决不是一天策划的结果。能够使民众,有令必行,有禁必止,最终不致招致断头之祸,这也不是一天教育,顷刻就能实现的。等到身体患上痈疽之病,脓血聚集于身,不也是背离了养生之道的结果吗?痈疽的产生,脓血的形成,既不是从天而降,也不是从地而生,而是邪气侵入人体,逐渐积累而成的。所以圣人自己能够在痈疽没有形成之前,积极预防,不使其发生;而愚者,不知防治,就会遭受疾病形成的痛苦。\\
黄帝问:怎么做才能不遭遇痈疽发生,看不到脓血形成呢?\\
岐伯说:脓血已成,十死一生,所以圣人不让痈疽形成,明确地制定了防止痈疽的良方,记载在竹帛上,使有才能的人能够继承下来,并世代相传,永无终止,为的是不让人们遭受痈疽病的痛苦。\\
黄帝问:已经形成脓血的,不能用小针治疗吗?\\
岐伯说:用小针治疗小痈疽,功效小;用大针来治疗大痈疽,功效大;用小针治疗大痈疽,多有伤害。所以已形成脓血的痈疽,只有用砭石、铍针、锋针取治脓血,最为适宜。\\
黄帝问:因为用针不当,而造成痈疽病的恶化,病人的生命就不能保全了吗?\\
岐伯说:这要看病证属逆还是属顺了。\\
黄帝说:希望听听痈疽病的逆顺。\\
岐伯说:白睛青,黑眼小,是逆证之一;服药而呕吐的,是逆证之二;腹痛而口渴甚的,是逆病之三;肩项转移不便的,是逆证之四;声音嘶哑,面无血色的,是逆证之五。除了这五种逆证之外,就都是顺证了。\\
黄帝问:各种病都有顺证逆证,可以听听吗?\\
岐伯说:腹部胀满,身体发热,脉大,是逆证之一;腹满肠鸣,四肢逆冷,腹泻,脉大,是逆证之二;鼻出血不止,脉大,是逆证之三;咳嗽而且尿血,肌肉消瘦,脉小而劲疾,是逆证之四;咳嗽,形体羸弱异常,身体发热,脉小而搏动疾速,是逆证之五。出现以上五逆证,不过十五天就会死亡。\\
还有腹大而胀,四肢厥冷,形肉已脱,泄泻不止的,是逆证之一;腹部胀满,大便下血,脉大而时有断绝,是逆证之二;咳嗽而尿血,形肉已脱,脉来搏指散乱,是逆证之三;呕血,胸部胀满牵引背部,脉小而快,是逆证之四;咳嗽呕吐,腹胀,泄泻完谷不化,而脉绝不至,是逆证之五。若出现这些症状的,不过一天就会死。医生对这些逆象,不加以审察而妄行针刺,就是误治。\\
黄帝问:先生把针刺的作用说得很大,可以与天地相配,上合天文,下合地理,在内则与五脏相联,在外则依次和六腑相通。全身二十八脉经气流注都有一定的规律,所以针刺能疏通经脉,宣畅气血,但针刺搞不好也会把活人治死,而不能让死者复生。您能够改变这种情况吗?\\
黄帝说:我听到针刺不当,致人于死,认为太不仁道了,但还是希望听听其中的道理,不再把错误用于病人。\\
岐伯说:这是明显的道理,也是必然的结果,好如刀剑可以杀人,饮酒可以醉人,不用诊察,也可以知道。\\
黄帝说:希望全部听听。\\
岐伯说:人所禀受的精气,来源于水谷。水谷注入的器官,是胃。胃好比是受纳水谷、化生气血的大海。海洋里的水,要上升到天空,化为云气,才能布散运行天下。胃中精微化生的气血,要运行周身,需要有经脉隧道。经隧,就是联络五脏六腑的大络,如果在这些地方用迎而夺之的刺法,就会误泻真气,而置人死地。\\
黄帝问:上下手足经脉,有多少穴位不能针刺呢?\\
岐伯说:误用迎而夺之的泻法,针刺手阳明大肠经的五里穴,致使脏气运行到中途而止,一脏的真气大约是五至而已,所以若连续五次用迎而夺之的泻法,则一脏的真气泻尽,若连续泻二十五次,就会使五脏所输注的精气都竭绝,这就是劫夺了人的天真之气,并非由于他命之自绝而终其寿的。\\
黄帝说:希望全部听听。\\
岐伯说:在气血出入的要害处妄行针刺,若刺得浅则其害迟,病人回到家中才会死亡;若刺得深则其害速,病者就会死在医者的堂上。\\
黄帝说:你讲的方法很完善,道理也很明白。请把它刻录在玉版上,作为最宝贵的文献,流传后世,作为针刺的禁忌,使人不敢触犯。\\
岐伯说:针治搞不好,确能致人于死;但针治得当,也不能起死复生。\\
五禁第六十一\\
黄帝问于岐伯曰:余闻刺有五禁。\\
岐伯曰:禁其不可刺也。\\
黄帝曰:余闻刺有五夺。\\
岐伯曰:无泻其不可夺者也。\\
黄帝曰:余闻刺有五过。\\
岐伯曰:补泻无过其度。\\
黄帝曰:余闻刺有五逆。\\
岐伯曰:病与脉相逆,命曰五逆。\\
黄帝曰:余闻刺有九宜。\\
岐伯曰:明知九针之论,是谓九宜。\\
黄帝曰:何谓五禁?愿闻其不可刺之时。\\
岐伯曰:甲乙日自乘,无刺头,无发蒙于耳内。丙丁日自乘,无振埃于肩喉廉泉。戊己日自乘四季,无刺腹去爪泻水。庚辛日自乘,无刺关节于股膝。壬癸日自乘,无刺足胫。是谓五禁。\\
黄帝曰:何谓五夺?\\
岐伯曰:形肉已夺,是一夺也;大夺血之后,是二夺也;大汗出之后,是三夺也;大泄之后,是四夺也;新产及大血之后,是五夺也。此皆不可泻。\\
黄帝曰:何谓五逆?\\
岐伯曰:热病脉静,汗已出,脉盛躁,是一逆也;病泄,脉洪大,是二逆也;著痹不移,夬肉破,身热,脉偏绝,是三逆也;\\
淫而夺形,身热,色夭然白,及后下血衃,血衃笃重,是谓四逆也;寒热夺形,脉坚搏,是谓五逆也。\\
黄帝问岐伯说:我听说刺有五禁。\\
岐伯说:是指遇到五个禁日,某些部位不可针刺。\\
黄帝说:我听说针刺的禁忌有五夺。\\
岐伯说:是指气血衰弱的人,不可用泻法。\\
黄帝说:我听说针刺的禁忌还有五过。\\
岐伯说:五过就是针刺补泻不能过其常度。\\
黄帝说:我听说针刺有五逆之证。\\
岐伯说:五种病证与脉象相反,就叫五逆。\\
黄帝说:我听说针刺有九宜。\\
岐伯说:明确知道九针的理论,就是九宜。\\
黄帝说:什么叫五禁?希望听听不可针刺的忌日。\\
岐伯说:天干应于人身,甲乙应头,所以逢甲乙日,不要刺头部,也不要用发蒙的针法去刺耳内。丙丁应肩喉,逢丙丁日,不要用振埃法刺肩、喉及廉泉穴。戊己应手足四肢,逢到戊己日,不可刺腹部和用去爪法泻水。庚辛应于股膝,逢庚辛日,不可刺股膝的穴位。壬癸应足胫,逢壬癸日,不可刺足胫的穴位。这就叫五禁。\\
黄帝问:什么叫五夺?\\
岐伯说:久病形肉瘦削至极,是一夺;大失血之后,是二夺;大汗出之后,是三夺;大泄利之后,是四夺;新产大失血之后,是五夺。五夺证都不可用泻法。\\
黄帝问:什么叫五逆?\\
岐伯说:热性病,而脉反沉静,汗出后,而脉反见躁动的,是为逆证之一;泄泻,而脉反洪大,是为逆证之二;肢体麻木沉重,肘、膝等处高起的肌肉破溃,身体发热,而脉现偏绝,是为逆证之三;\\
久病遗、泄、淋、浊、汗等阴血受损之病,致使形体消瘦,若见发热,肤色苍白,或见大便中夹杂紫黑血块,病情极重,是为逆证之四;久患寒热,形体消瘦,脉反见坚硬搏指的,是逆证之五。\\
动输第六十二\\
黄帝曰:经脉十二,而手太阴、足少阴、阳明独动不休,何也?\\
岐伯曰:是明胃脉也。胃为五脏六腑之海,其清气上注于肺,肺气从太阴而行之。其行也,以息往来,故人一呼脉再动,一吸脉亦再动,呼吸不已,故动而不止。\\
黄帝曰:气之过于寸口也,上十焉息,下八焉伏?何道从还?不知其极。\\
岐伯曰:气之离藏也,卒然如弓弩之发,如水之下岸,上于鱼以反衰,其余气衰散以逆上,故其行微。\\
黄帝曰:足之阳明,何因而动?\\
岐伯曰:胃气上注于肺,其悍气上冲头者,循咽,上走空窍,循眼系,入络脑,出圢,下客主人,循牙车,合阳明,并下人迎,此胃气别走于阳明者也。故阴阳上下,其动也若一。故阳病而阳脉小者为逆,阴病而阴脉大者为逆。故阴阳俱静,俱动,若引绳相倾者病。\\
黄帝曰:足少阴,何因而动?\\
岐伯曰:冲脉者,十二经之海也,与少阴之大络,起于肾下,出于气街,循阴股内廉,邪入腘中,循胫骨内廉,并少阴之经,下入内踝之后,入足下,其别者,邪入踝,出属跗上,入大指之间,注诸络,以温足胫。此脉之常动者也。\\
黄帝曰:营卫之行也,上下相贯,如环之无端,今有其卒然遇邪气,及逢大寒,手足懈惰,其脉阴阳之道,相输之会,行相失也,气何由还?\\
岐伯曰:夫四末阴阳之会者,此气之大络也。四街者,气之径路也。故络绝则径通,四末解则气从合,相输如环。\\
黄帝曰:善。此所谓如环无端,莫知其纪,终而复始,此之谓也。\\
黄帝问:十二经脉之中,为什么只有手太阴肺经、足少阴肾经和足阳明胃经三经的经脉搏动不已呢?\\
岐伯说:这就是足阳明胃脉与脉搏跳动的关系。胃是五脏六腑的营养来源,胃中水谷化生的清气,向上流注肺中,肺气从手太阴肺经开始,运行周身十二经脉。肺气上下运行,呼吸往来,故人一呼脉跳动两次,一吸脉亦跳动两次,呼吸不停,所以寸口脉搏动不止。\\
黄帝说:脉气通过寸口,脉来时其气较盛,脉去时其气较衰,脉气从哪里回到本脉?不知道其所以然。\\
岐伯说:脉气从内脏向外输注到经脉时,快如离弦之箭,如离岸之洪水,当脉气上达鱼际后,却呈现由盛而转衰,但其衰散之余力还能逆而上行,所以脉气就微弱了。\\
黄帝问:足阳明胃脉是因为什么而搏动的?\\
岐伯说:胃气上注于肺,其上冲于头的慓悍之气,循咽喉向上走行孔窍,循着眼系,入络于脑,从脑出吇部,向下会于客主人穴,沿着颊车,合于足阳明本经,并向下行到人迎穴,这就是胃气别走而又合于阳明的过程。因此,手太阴寸口脉和足阳明人迎脉搏动一致。所以阳病而阳脉反小的为逆;阴病而阴脉反大的为逆。在正常情况下,寸口和人迎脉是平衡的,静则俱静,动则俱动,像牵引绳索一样均匀,如果上下之脉若引绳不匀而一方偏盛,就是病态。\\
黄帝问:足少阴肾脉是因为什么而搏动呢?\\
岐伯说:冲脉为十二经脉之海,它和足少阴的络脉,同起源于肾下,出于足阳明胃经的气街,沿大腿内侧,向下斜行入腘中,沿着胫骨内侧,与足少阴经相合而下行入于足内踝之后,入于脚下。它分出的支脉,斜入内踝,出而入于足背上,进入小趾之间,注入诸络脉,来温养足胫部。这就是足少阴经脉常动不休的原因。\\
黄帝问:营气和卫气的循行,上下互相贯通,如圆环一样没有开端,现在突然遭遇邪气侵袭,以及遇到了严寒天气,手足懈惰无力,经脉阴阳之道,气血相输之会,将运行失常,在这样的情况下,营卫之气是怎样往返循环的呢?\\
岐伯说:四肢是阴阳会合的地方,也是营卫之气通行的大络。四街是营卫之气运行的必经之路。所以络脉被邪气阻塞后,则四街这些径路开通,当四肢的邪气解除后,则络脉又复沟通,气又从这里输运会合,如环无端,周而复始。\\
黄帝说:好!经气运行,如环无端,莫知其纪,周而复始,就是这个道理。\\
五味论第六十三\\
黄帝问于少俞曰:五味入于口也,各有所走,各有所病。酸走筋,多食之,令人癃;咸走血,多食之,令人渴;辛走气,多食之,令人洞心;苦走骨,多食之,令人变呕;甘走肉,多食之,令人悗心。余知其然也,不知其何由,愿闻其故。\\
少俞答曰:酸入于胃,其气涩以收,上之两焦,弗能出入也。不出即留于胃中,胃中和温,则下注膀胱。膀胱之胞薄以懦,得酸则缩绻,约而不通,水道不行,故癃。阴者,积筋之所终也,故酸入而走筋矣。\\
黄帝曰:咸走血,多食之,令人渴,何也?\\
少俞曰:咸入于胃,其气上走中焦,注于脉,则血气走之。血与咸相得则凝,凝则胃中汁注之。注之则胃中竭,竭则咽路焦,故舌本干而善渴。血脉者,中焦之道也,故咸入而走血矣。\\
黄帝曰:辛走气,多食之,令人洞心,何也?\\
少俞曰:辛入于胃,其气走于上焦,上焦者,受气而营诸阳者也。姜韭之气熏之,营卫之气不时受之,久留心下,故洞心。辛与气俱行,故辛入而与汗俱出。\\
黄帝曰:苦走骨,多食之,令人变呕,何也?\\
少俞曰:苦入于胃,五谷之气,皆不能胜苦。苦入下脘,三焦之道皆闭而不通,故变呕。齿者,骨之所终也,故苦入而走骨,故入而复出,知其走骨也。\\
黄帝曰:甘走肉,多食之,令人悗心,何也?\\
少俞曰:甘入于胃,其气弱小,不能上至于上焦,而与谷留于胃中者,令人柔润者也。胃柔则缓,缓则虫动,虫动则令人悗心。其气外通于肉,故甘走肉。\\
黄帝问少俞说:五味进入口中,各进入所喜的脏器,各有所发生的病变。酸味走筋,多食酸味,会使人小便不通;咸味走血,多食咸味,会使人发渴;辛味走气,多食辛味,会使人心闷;苦味走骨,多食苦味,会使人呕吐;甘味走肉,多食甘味,会使人心闷。我已知道五味食之过度,能发生这些病证,但不理解其中的道理,希望听到其中的缘故。\\
少俞回答说:酸味入胃以后,因气味涩滞,而有收敛作用,只能行于上、中二焦,不能遽行出入。既然不出,就流于胃里,胃里温和,就向下渗注到膀胱。由于膀胱之皮薄而软,受到酸味,就会缩屈,使膀胱出口处约束不通,以致小便不畅,因此发生癃闭。人体的阴器,是周身诸筋终聚之处,所以酸味入胃而走肝经之筋。\\
黄帝问:咸味走血分,多食咸味,使人口渴,为什么?\\
少俞说:咸味入胃以后,它所化之气向上走于中焦,再由中焦流注到血脉,与血相和。咸与血相和,脉就要凝涩,脉凝涩则胃的水液也要凝涩,胃的水液凝涩则胃里干竭,由于胃液干竭,咽路感到焦躁,因而舌干多渴。血脉是输送中焦精微于周身的道路,血亦出于中焦,咸味上行于中焦,所以咸入胃后,就走入血分。\\
黄帝问:辛味走气分,多食辛味,使人感觉如烟熏心,为什么?\\
少俞说:辛味入胃以后,其气走向上焦,上焦有受纳饮食精气以运行腠理而卫外的功能。姜韭之气,熏至营卫,不时受到辛味的刺激,如久留在胃中,所以有如烟熏心的感觉。辛走卫气,与卫气同行,所以辛味入胃以后,就会和汗液发散出来。\\
黄帝问:苦味善走骨,多食令人呕吐,为什么?\\
少俞说:苦入胃后,五谷之气味都不能胜过苦味。当苦味进入下脘后,三焦的气机阻闭不通,三焦不通,则入胃之水谷,不得通调而散,胃阳受到苦味的影响而功能失常,胃气上逆而变为呕吐。牙齿是属骨的部分,称骨之所终,苦味入胃后,走骨也走齿。因此,如已入胃的苦味而重复吐出,就可以知其已经走骨了。\\
黄帝问:甘味善走肌肉,多食则令人心中烦闷,为什么?\\
少俞说:甘味入胃后,甘气柔弱而小,不能上达上焦,与饮食物一同留于胃中,所以胃气也柔润。胃柔则胃功能减弱,胃的功能减弱则肠中寄生虫乘机而动,虫动则使人心中闷乱。另外,由于甘味入脾,脾主肌肉,所以甘味外通于肌肉。\\
阴阳二十五人第六十四\\
黄帝曰:余闻阴阳之人,何如?\\
伯高曰:天地之间,六合之内,不离于五,人亦应之。故五五二十五人之形而阴阳之人不与焉。其态又不合于众者五,余已知之矣。愿闻二十五人之形,血气之所生,别而以候,从外知内,何如?\\
岐伯曰:悉乎哉问也!此先师之秘也,虽伯高犹不能明之也。\\
黄帝避席,遵循而却,曰:余闻之,得其人弗教,是谓重失,得而泄之,天将厌之。余愿得而明之,金柜藏之,不敢扬之。\\
岐伯曰:先立五形金木水火土,别其五色,异其五形之人,而二十五人具矣。\\
黄帝曰:愿卒闻之。\\
岐伯曰:慎之慎之,臣请言之。\\
木形之人,比于上角,似于苍帝。其为人,苍色,小头,长面,大肩背,直身,小手足,好有才,劳心,少力,多忧,劳于事。能春夏不能秋冬,感而病生,足厥阴佗佗然。大角之人,比于左足少阳,少阳之上遗遗然。左角之人,比于右足少阳,少阳之下随随然。奼角之人,比于右足少阳,少阳之上推推然。判角之人,比于左足少阳,少阳之下栝栝然。\\
火形之人,比于上徵,似于赤帝。其为人赤色,广圣,锐面小头,好肩背髀腹,小手足,行安地,疾行摇,肩背肉满,有气轻财,少信,多虑,见事明,好颜,急心,不寿暴死。能春夏不能秋冬,秋冬感而病生,手少阴核核然。质徵之人比于左手太阳,太阳之上肌肌然。少徵之人,比于右手太阳,太阳之下慆慆然。右徵之人,比于右手太阳,太阳之上鲛鲛然。质判之人,比于左手太阳,太阳之下支支颐颐然。\\
土形之人,比于上宫,似于上古黄帝。其为人黄色,圆面,大头,美肩背,大腹,美股胫,大手足,多肉,上下相称,行安地,举足浮,安心,好利人,不喜权势,善附人也。能秋冬不能春夏,春夏感而病生,足太阴敦敦然。太宫之人,比于左足阳明,阳明之上婉婉然。加宫之人,比于左足阳明,阳明之下坎坎然。少宫之人,比于右足阳明,阳明之上枢枢然。左宫之人,比于右足阳明,阳明之下兀兀然。\\
金形之人,比于上商,似于白帝。其为人方面,白色,小头,小肩背,小腹,小手足,如骨发踵外,骨轻,身清廉,急心,静悍,善为吏。能秋冬不能春夏,春夏感而病生,手太阴敦敦然。奼商之人,比于左手阳明,阳明之上廉廉然。右商之人,比于左手阳明,阳明之下脱脱然。大商之人,比于右手阳明,阳明之上监监然。少商之人,比于右手阳明,阳明之下严严然。\\
水形之人,比于上羽,似于黑帝。其为人黑色,面不平,大头,廉颐,小肩,大腹,大手足,发行摇身,下尻长,背延延然,不敬畏,善欺绐人,戮死。能秋冬不能春夏,春夏感而病生,足少阴汙汙然。大羽之人,比于右足太阳,太阳之上颊颊然。少羽之人,比于左足太阳,太阳之下纡纡然。众之为人。比于右足太阳,太阳之下洁洁然。桎之为人,比于左足太阳,太阳之上安安然。是故五形之人,二十五变者,众之所以相异者是也。\\
黄帝曰:得其形,不得其色,何如?\\
岐伯曰:形胜色,色胜形者,至其胜时年加,感则病行,失则忧矣。形色相得者,富贵大乐。\\
黄帝曰:其形色相胜之时,年加可知乎?\\
岐伯曰:凡人之大忌常加九岁。七岁,十六岁,二十五岁,三十四岁,四十三岁,五十二岁,六十一岁,皆人之大忌,不可不自安也,感则病行,失则忧矣。当此之时,无为奸事,是谓年忌。\\
黄帝曰:夫子之言,脉之上下,血气之候,以知形气,奈何?\\
岐伯曰:足阳明之上,血气盛,则髯美长;血少气多,则髯短;故气少血多,则髯少;血气皆少,则无髯,两吻多画。足阳明之下,血气盛,则下毛美长至胸;血多气少,则下毛美短至脐,行则善高举足,足指少肉,足善寒;血少气多,则肉而善瘃;血气皆少,则无毛,有则稀枯悴,善痿厥足痹。\\
足少阳之上,气血盛,则通髯美长;血多气少,则通髯美短;血少气多,则少髯;血气皆少,则无须。感于寒湿,则善痹,骨痛、爪枯也。足少阳之下,血气盛,则胫毛美长,外踝肥;血多气少,则胫毛美短,外踝皮坚而厚;血少气多,则胻毛少,外踝皮薄而软;血气皆少,则无毛,外踝瘦,无肉。\\
足太阳之上,血气盛,则美眉,眉有毫毛;血多气少,则恶眉,面多小理;血少气多,则面多肉;血气和,则美色。足太阳之下,血气盛,则跟肉满,踵坚;气少血多,则瘦,跟空;血气皆少,则喜转筋,踵下痛。\\
手阳明之上,血气盛,则髭美;血少气多,则髭恶;血气皆少,则无髭。手阳明之下,血气盛,则腋下毛美,手鱼肉以温;气血皆少,则手瘦以寒。\\
手少阳之上,血气盛,则眉美以长,耳色美;血气皆少,则耳焦恶色。手少阳之下,血气盛,则手拳多肉以温;血气皆少,则寒以瘦;气少血多,则瘦以多脉。\\
手太阳之上,血气盛,则有多须,面多肉以平,血气皆少则面瘦恶色。手太阳之下,血气盛则掌肉充满,血气皆少则掌瘦以寒。\\
黄帝曰:二十五人者,刺之有约乎?\\
岐伯曰:美眉者,足太阳之脉,气血多;恶眉者,血气少;其肥而泽者,血气有余;肥而不泽者,气有余,血不足;瘦而无泽者,气血俱不足。审察其形气有余不足而调之,可以知逆顺矣。\\
黄帝曰:刺其阴阳,奈何?\\
岐伯曰:按其寸口人迎,以调阴阳,切循其经络之凝涩,结而不通者,在于身皆为痛痹,甚则不行,故凝涩。凝涩者,致气以温之,血和乃止。其结者,脉结,血不行,决之乃行。故曰:气有余于上者,导而下之;气不足于上者,推而休之,其稽留不至者,因而迎之。必明于经隧,乃能持之。寒与热争者,导而行之,其宛陈血结者,则而予之。必先明知二十五人,则血气之所在,左右上下,刺约毕矣。\\
黄帝问:我听说人的阴阳属性不同,怎样区别呢?\\
伯高问:天地之间,四方上下之内,一切事物都离不开“五行”,人也是如此。所以五五二十五人之形,不包括阴阳之人在内。那阴阳之人的五种形态与一般人不同,我已经知道了。我希望听听二十五人的形态,及其血气的生成,分别进行候察,从外部表现来测知内部情况,如何?\\
岐伯说:你问得真全面啊!这是先师秘而不传的,即使是伯高,也不能明白其中的道理。\\
黄帝离开座席,后退了几步,很恭谨地说:我听说,遇到可以传授医道的人而不教给他,就是双重损失,得到了医道而随便泄漏,上天也要厌弃他。我希望得到医道并且彻底明白,把它保存在金匮里,不敢随便传扬出去。\\
岐伯说:先要明确金、木、水、火、土五种类型,辨别五色,区别五种基本形态之人,这样二十五种人的形态就清楚了。\\
黄帝说:希望详细听听。\\
岐伯说:慎重啊慎重!请让臣下说明。\\
木形的人,属于木音中的上角,好像东方地区的人。这种人皮肤苍色,头小,面长,大肩平背,身直,手足小,有才干,好用心机,体力差,多忧劳事物,能耐受春夏,不能耐受秋冬,易感受病邪而发病,属于足厥阴肝经,其性格特征是雍容自得,柔美安重。秉木气之偏的有左右上下四种类型:左之上方,属于大角一类,可类比左足少阳经之上,其性格特征是自得和蔼。右之下方,属于左角一类,可类比右足少阳经之下,其性格特征是随和顺从。右之上方,属于幵角一类,可类比右足少阳经之上,其性格特征是勇于进取。左之下方,属于判角一类人,可类比左足少阳经之下,其性格特征是刚正不阿。\\
火形的人,属于火音中的上徵,好像南方地区的人。这种人皮肤色红,脊背宽广,面尖瘦,头小,肩背髀腹发育很好,手足小,步履稳重,走路快而肩摇,背部的肌肉丰满,有气魄,轻视钱财,缺少信心,多忧虑,看事明白,爱漂亮,性情躁急,不能享高寿,多暴死。能耐受春夏的温暖,不能耐受秋冬的寒凉,秋冬时感受外邪,易发病,属于手少阴心经,性格特征是为人真诚。秉火气之偏的有上下左右四种类型:左之上方,属于质徵一类,可类比左手太阳之上,其性格特征是识见浅薄。右之下方,属于少徵一类,可类比右手太阳经之下,其性格特征是猜忌多疑。右之上方,属于右徵一类,可类比右手太阳之上,其性格特征是勇于进取。左之下方,属于质判一类,可类比左手太阳之下,其性格特征是乐观愉快,怡然自得而无忧无虑。\\
土形的人,属于土音中的上宫,好像中央地区的人。这种人皮肤色黄,圆脸,大头,肩背丰厚健美,腹大,大腿和足胫部都健壮,手足大,肌肉丰满,全身上下各部都很匀称,走路稳重,步伐轻盈,内心安静,喜欢帮助别人,不争逐权势,善于团结人。能耐受秋冬的寒冷,不能耐受春夏的温热,春夏感受外邪容易生病,属于足太阴脾经,性格特征是诚恳忠厚。集土气之偏的有左右上下四种类型:左之上方,属于太宫一类,可类比于左足阳明经之上,其性格特征是和平柔顺。左之下方,属于加宫一类,可类比于左足阳明经之下,其性格特征是神情怡悦。右之上方,属于少宫一类,可类比于右足阳明经之上,其性格特征是为人圆滑。右之下方,属于左宫一类,可类比于右足阳明之下,其性格特征是专心致志、不怕困难。\\
金形的人,属于金音中的上商,好像西方地区的人。这种人是方脸,皮肤白色,小头,小肩背,小腹,小手足,足跟坚壮,好像骨头长在足踵的外面一样,骨骼轻,身体轻捷,禀性廉洁,性情急躁,能动能静,善于做小吏。能耐受秋冬的寒冷,不能耐受春夏的温热,感受了春夏的邪气易于生病,属于手太阴肺经,性格特征是坚不可屈。禀金气之偏的有上下左右四种类型:左之上方,在金音中属于幵商一类的人,类属于左手阳明经之上,其性格特征是廉洁自守。左之下方,在金音中属于右商一类的人,类属于左手阳明经之下,其性格特征是潇洒而美好。右之上方,在金音中属于大商一类的人,类属于右手阳明经之上,其性格特征是善于明察是非。右之下方,在金音中属于少商一类的人,类属于右手阳明经之下,其性格特征是威严而庄重。\\
水形的人,属于水音中的上羽,好像北方地区的人。这种人皮肤黑色,面不平正,大头,腮有棱角,两肩小,腹大,手足大,行路时摇晃身体,下尻较长,脊背也长,这种人既不恭敬又无畏惧,经常欺诈,常被杀戮致死。能耐受秋冬的寒冷,不能耐受春夏的温热,春夏感受外邪容易发病,属于足少阴肾经,性格特征是为人卑下。禀水气之偏的有左右上下四种类型:右之上方,属于大羽一类,可类比于右足太阳经之上,其性格特征是洋洋自得。左之下方,属于少羽一类,可类比于左足太阳经之下,其性格特征是性情纡曲,不直爽。右之下方,属于众羽一类,可类比于右足太阳经之下,其性格特征是文静如水之清澈。左之上方,属于桎羽一类,可类比于左足太阳之上,其性格特征是舒缓徐和。所以金、木、水、火、土五行之人,及其二十五种类型的变化,是与一般人有所不同的。\\
黄帝问:形体是五行之一,但其肤色与五行的类型不符,怎样呢?\\
岐伯说:形体的五行属性克肤色的五行属性,或肤色的五行属性克形体的五行属性,再遇到年忌相加,感受病邪就会生病,如果失治难免有性命之忧。如果形色相称,就会富贵康乐。\\
黄帝问:在形色相克时,年忌能够知道吗?\\
岐伯说:总的说来,年忌的计算方法是,七岁是大忌之年,在此基础上递加九年。则十六岁、二十五岁、三十四岁、四十三岁、五十二岁、六十一岁,这些年龄,都是大忌之年,不能不安定好自己的身心,否则感受病邪而发生疾病,又有所疏忽,就有性命之忧了。所以,在这些年龄时,不能做奸邪之事。这叫年忌。\\
黄帝问:夫子所说在手足三阳经脉的上部和下部,候察气血的多少,来了解形气的强弱,是怎样的呢?\\
岐伯说:足阳明在上部的经脉,如果血气充足,则两颊的胡须美而长;血少气多的,则胡须短;气少血多的,胡须稀少;血气皆少的,则两颊全无胡须,而且口角两旁皱纹多。足阳明在下部的经脉,如果气血充足,阴毛美而长,可上及胸部;血多气少,则阴毛虽美而短,可及脐部,走路时喜欢高抬足,足趾肉少,两足部常寒冷;血少气多的,则易生冻疮;血气皆不足的,则无阴毛,即便有也很稀少,枯槁憔悴,易患痿厥足痹等病。\\
足少阳在上部的经脉,如果气血充盛,则生于两颊连鬓的胡须美而长;若血多气少,则连鬓的胡须虽美好而短;血少气多的,则胡须少;血气皆少,则不长胡须。感受寒湿,容易患痹证,骨痛、爪甲干枯。足少阳在下部的经脉,如果血气充盛的,则小腿毫毛美长,外踝丰满;血多气少的,则小腿毫毛美短,外踝处皮坚而厚;血少气多的,则小腿的毫毛少,外踝处皮薄而软;血气都少的,则不生毛,外踝处瘦而没有肌肉。\\
足太阳在上部的经脉,如果血气充足,则眉毛清秀而长,眉中出现长毛;血多气少,则眉毛枯悴,脸部多细小皱纹;血少气多,则面部肌肉丰满;气血平和,则面色秀美。足太阳在下部的经脉,如果气血充盛,则足跟部肌肉丰满,踵坚实;气少血多,则脚后跟瘦而无肉;气血都少的,易发生转筋、脚后跟痛等证。\\
手阳明在上部的经脉,如果气血充盛,则嘴上边的胡须秀美;血少气多的,则嘴上边的胡须粗恶;血与气都少,则嘴上边的不长胡须。手阳明在下部的经脉,如果气血充盛,则腋毛秀美,手鱼部肌肉温暖;若气血皆不足,则手部肌肉瘦削而寒凉。\\
手少阳在上部的经脉,如果气血充盛,则眉毛美而长,耳色美好;血气都少,则耳廓焦枯色晦。手少阳在下部的经脉,如果气血充盛,则手拳多肉而温暖;气血都不足的,则手拳消瘦且寒凉;气少血多的,则手拳消瘦,而络脉显见。\\
手太阳在上部的经脉,如果血气充盛,则须多,面部肌肉丰满而平坦;血气都少,则面部消瘦而颜色晦暗。手太阳在下部的经脉,如果气血充盛,则手掌肌肉丰满;气血都少,则掌部肌肉消瘦而寒凉。\\
黄帝说:这二十五种类型的人,针刺有一定的准则吗?\\
岐伯说:眼眉秀美的,足太阳经脉气血充足;眼眉粗恶的,气血均少;肥胖而光泽的,血气有余;肥胖而无光泽的,气有余,血不足;消瘦而无光泽的,气血均不足。审察形气有余、不足的表现来调治,就可以知道疾病的顺逆了。\\
黄帝问:怎样针刺手足阴阳经脉的病变呢?\\
岐伯说:诊察人迎、寸口脉,来审察阴阳盛衰的变化,再沿着经络切按,如见气血凝滞,郁结不通的,在身体上会出现痛痹,严重的气血不能运行,所以气血凝结涩滞。气血凝涩的,用留针补益,以温通气机,气血通和则停止。由于凝结,脉中郁结,血运不畅,可刺出淤血,开决脉络,则气血就可正常运行。所以说:上部邪气亢盛的,应该导之下行;上部正气不足的,用推而扬之的针法,催气上行;若留针已久而气仍未至的,应用多种针法,以迎导其气。必须明白经脉的走行,才能正确使用各种针刺手法。如有寒热交争的,应引导其气血运行;若血脉中郁滞而日久淤血凝结的,根据情况予以治疗。总之,必须先了解二十五种人的不同类型,那么其气血在左右上下的盛衰及病变就清楚了。因而,针刺的准则,也就都包括在其中了。\\
卷十\\
五音五味第六十五\\
右徵与少徵,调右手太阳上。\\
左商与左徵,调左手阳明上。\\
少徵与大宫,调左手阳明上。\\
右角与大角,调右足少阳下。\\
大徵与少徵,调左手太阳上。\\
众羽与少羽,调右足太阳下。\\
少商与右商,调右手太阳下。\\
桎羽与众羽,调右足太阳下。\\
少宫与大宫,调右足阳明下。\\
判角与少角,调右足少阳下。\\
奼商与上商,调右足阳明下。\\
奼商与上角,调左足太阳下。\\
上徵与右徵同,谷麦,畜羊,果杏,手少阴,脏心,色赤,味苦,时夏。\\
上羽与大羽同,谷大豆,畜彘,果栗,足少阴,脏肾,色黑,味咸,时冬。\\
上宫与大宫同。谷稷,畜牛,果枣,足太阴,脏脾,色黄,味甘,时季夏。\\
上商与右商同,谷黍,畜鸡,果桃,手太阴,脏肺,色白,味辛,时秋。\\
上角与大角同,谷麻,畜犬,果李,足厥阴,脏肝,色青,味酸,时春。\\
大宫与上角,同右足阳明上。\\
左角与大角,同左足阳明上。\\
少羽与大羽,同右足太阳下。\\
左商与右商,同左手阳明上。\\
加宫与大宫,同左足少阳上。\\
质判与大宫,同左手太阳下。\\
判角与大角,同左足少阳下。\\
大羽与大角,同右足太阳上。\\
大角与大宫,同右足少阳上。\\
右徵、少徵、质徵、上徵、判徵。\\
右角、奼角、上角、大角、判角。\\
右商、少商、奼商、上商、左商。\\
少宫、上宫、大宫、加宫、左宫。\\
众羽、桎羽、上羽、大羽、少羽。\\
黄帝曰:妇人无须者,无血气乎?\\
岐伯曰:冲脉、任脉,皆起于胞中,上循脊里,为经络之海。其浮而外者,循腹上行,会于咽喉,别而络唇口。气盛则充肤热肉,血独盛则澹渗皮肤,生毫毛。今妇人之生,有余于气,不足于血,以其数脱血也,冲任之脉,不荣口唇,故须不生焉。\\
黄帝曰:士人有伤于阴,阴气绝而不起,阴不用,然其须不去,其故何也?宦者独去,何也?愿闻其故。\\
岐伯曰:宦者,去其宗筋,伤其冲脉,血泻不复,皮肤内结,唇口不荣,故须不生。\\
黄帝曰:其有天宦者,未尝被伤,不脱于血,然其须不生,其故何也?\\
岐伯曰:此天之所不足也,其任冲不盛,宗筋不成,有气无血,唇口不荣,故须不生。\\
黄帝曰:善乎哉!圣人之通万物也,若日月之光影,音声鼓响,闻其声而知其形,其非夫子,孰能明万物之精。是故圣人视其颜色,黄赤者多热气,青白者少热气,黑色者多血少气。美眉者太阳多血,通髯极须者少阳多血,美须者阳明多血。此其时然也。夫人之常数,太阳常多血少气,少阳常多气少血,阳明常多血多气,厥阴常多气少血,少阴常多血少气,太阴常多血少气。此天之常数也。\\
属于火音中的右徵和少徵之类的人,应当调治右侧手太阳小肠经的上部。\\
属于金音中的左商和火音中左徵之类的人,应当调治左侧手阳明大肠经的上部。\\
属于火音中少徵和土音中大宫之类的人,应当调治左侧手阳明大肠经的上部。\\
属于木音中的右角和大角之类的人,应当调治右侧足少阳胆经的下部。\\
属于火音中的大徵和少徵之类的人,应当调治左侧手太阳小肠经的上部。\\
属于水音中的众羽和少羽之类的人,应当调治右侧足太阳膀胱经的下部。\\
属于金音中的少商和右商之类的人,应当调治右侧手太阳小肠经的下部。\\
属于水音中的桎羽和众羽一类的人,应当调治右侧足太阳膀胱经的下部。\\
属于土音中的少宫和大宫一类的人,应当调治右侧足阳明胃经的下部。\\
属于木音中的判角和少角一类的人,应当调治右侧足少阳胆经的下部。\\
属于金音中的幵商和上商一类的人,应当调治右侧足阳明胃经的下部。\\
属于金音中的幵商和木音中的上角之类的人,应当调治左侧足太阳膀胱经的下部。\\
属木音中的大角与属土音中的大宫之类的人,都可以调治右侧足少阳胆经的上部。\\
右徵、少徵、质徵、上徵、判徵五种人,都是属火音中的类型。\\
右角、幵角、上角、大角、判角五种人,都是属木音中的类型。\\
右商、少商、幵商、上商、左商五种人,都是属金音中的类型。\\
少宫、上宫、大宫、加宫、左宫五种人,都是属土音中的类型。\\
众羽、桎羽、上羽、大羽、少羽五种人,都是属水音中的类型。\\
黄帝问:妇女不长胡须,是没有血气吗?\\
岐伯说:冲脉和任脉都从胞中起始,向上循行在脊椎里面,是经脉之海。那在体表循行的,沿腹部上行,交会于咽喉部,从咽喉部别出一个分支,循行环绕于口唇的周围。气充盛的人则皮肤温热,若血独盛则渗灌到皮肤,生长毫毛。妇女的生理特点是气有余而血不足,原因是每月排出月经,冲任之脉,不能营养口唇,所以妇女不长胡须。\\
黄帝说:有人损伤了阴器,阴器萎废,不能勃起,丧失了性功能,但他的胡须仍然生长,这是什么原因呢?而太监却没有胡须,又是什么原因呢?希望听听其中的道理。\\
岐伯说:宦官是被割掉阴茎和睾丸后,冲脉受伤,血泻出后不能恢复正常,皮肤被伤后,伤口干结,唇口得不到冲、任脉气血的营养,所以胡须就不能生长了。\\
黄帝问:有一种人是天宦,其宗筋没有受伤,也不像妇女那样因排月经而伤血,但他不长胡须,是什么原因呢?\\
岐伯说:这是先天发育不足,他的任、冲二脉不充盛,阴茎和睾丸发育不完全,虽然有气,但血不足,不能营养唇口,所以不长胡须。\\
黄帝说:说得好极了!圣人能够通晓万事万物的道理,就像日月有光和影,鼓响有音和声一样,如果不是夫子您,谁能明白万物的精微道理呢?所以圣人看到面现黄赤色的,就知体内气血热,出现青白色,就知气血寒,面现黑色,就知多血少气。眉毛秀美的,太阳经多血;通髯和胡须相连的,少阳经多血;胡须华美的,阳明经多血。这是一般的规律。人体经脉中气血多少的规律:太阳经常多血少气,少阳经常多气少血,阳明经常气血均多,厥阴经常多气少血,少阴经常多血少气,太阴经常多血少气。这是先天获得的生理的正常规律。\\
百病始生第六十六\\
黄帝问于岐伯曰:夫百病之始生也,皆生于风雨寒暑,清湿喜怒。喜怒不节则伤脏,风雨则伤上,清湿则伤下。三部之气,所伤异类,愿闻其会。\\
岐伯曰:三部之气各不同,或起于阴,或起于阳,请言其方。喜怒不节则伤脏,脏伤则病起于阴也;清湿袭虚,则病起于下;风雨袭虚,则病起于上,是谓三部。至其淫泆,不可胜数。\\
黄帝曰:余固不能数,故问先师,愿卒闻其道。\\
岐伯曰:风雨寒热,不得虚,邪不能独伤人。卒然逢疾风暴雨而不病者,盖无虚,故邪不能独伤人。此必因虚邪之风,与其身形,两虚相得,乃客其形。两实相逢,众人肉坚。其中于虚邪也,因于天时,与其身形,参以虚实,大病乃成。气有定舍,因处为名,上下中外,分为三员。\\
是故虚邪之中人也,始于皮肤,皮肤缓则腠理开,开则邪从毛发入,入则抵深,深则毛发立。毛发立则淅然,故皮肤痛。留而不去,则传舍于络脉。在络之时,痛于肌肉,其痛之时息,大经乃代。留而不去,传舍于经。在经之时,洒淅喜惊。留而不去,传舍于输。在输之时,六经不通,四肢则肢节痛,腰脊乃强。留而不去,传舍于伏冲之脉。在伏冲之时,体重身痛。留而不去,传舍于肠胃。在肠胃之时,贲响腹胀。多寒则肠鸣飧泄,食不化;多热则溏出麋。留而不去,传舍于肠胃之外,募原之间,留著于脉。稽留而不去,息而成积。或著孙脉,或著络脉,或著经脉,或著输脉,或著于伏冲之脉,或著于膂筋,或著于肠胃之募原,上连于缓筋,邪气淫泆,不可胜论。\\
黄帝曰:愿尽闻其所由然。\\
岐伯曰:其著孙络之脉而成积者,其积往来上下。臂手孙络之居也,络浮而缓,不能拘积而止之,故往来移行,肠胃之间。水湊渗注灌,濯濯有音。有寒则腹尒满雷引,故时切痛。其著于阳明之经,则挟脐而居,饱食则益大,饥则益小。其著于缓筋也,似阳明之积,饱食则痛,饥则安。其著于肠胃之募原也,痛而外连于缓筋,饱食则安,饥则痛。其著于伏冲之脉者,揣之应手而动,发手则热气下于两股,如汤沃之状。其著于膂筋,在肠后者,饥则积见,饱则积不见,按之不得。其著于输之脉者,闭塞不通,津液不下,孔窍干壅。此邪气之从外入内,从上下也。\\
黄帝曰:积之始生,至其已成,奈何?\\
岐伯曰:积之始生,得寒乃生,厥乃成积也。\\
黄帝曰:其成积奈何?\\
岐伯曰:厥气生足悗,悗生胫寒,胫寒则血脉凝涩,血脉凝涩则寒气上入于肠胃。入于肠胃则胀满,胀满则肠外之汁沫迫聚不得散,日以成积。卒然多食饮,则肠满,起居不节,用力过度,则络脉伤。阳络伤则血外溢,血外溢则衄血;阴络伤则血内溢,血内溢则后血。肠胃之络伤,则血溢于肠外,肠外有寒,汁沫与血相搏,则并合凝聚不得散,而积成矣。卒然外中于寒,若内伤于忧怒,则气上逆,气上逆则六输不通,温气不行,凝血蕴裏而不散,津液涩渗,著而不去,而积皆成矣。\\
黄帝曰:其生于阴者,奈何?\\
岐伯曰:忧思伤心;重寒伤肺;忿怒伤肝;醉以入房,汗出当风伤脾;用力过度,若入房汗出浴,则伤肾。此内外三部之所生病者也。\\
黄帝曰:善。治之奈何?\\
岐伯答曰:察其所痛,以知其应。有余不足,当补则补,当泻则泻。毋逆天时,是谓至治。\\
黄帝问岐伯说:各种疾病开始发生,都是风雨寒暑,清湿喜怒内外各种因素所致。喜怒没有节制,会伤及内脏;外感风雨,会伤及人体上部;感受寒湿,会伤及人体下部。上中下三部之气伤人,各不相同,希望你讲一下大概的道理。\\
岐伯说:三部之气各不相同,病或先生于阴分,或先生于阳分,请让我讲一讲大概情况。喜怒没有节制,则病起于内部;清湿乘虚袭人筋骨,则病起于下部;风雨乘虚袭人肌表,则病起于上部。这是邪气侵袭的三个主要部位。待至病邪漫延深入,发生的症状,就不能计数了。\\
黄帝说:我本来对千变万化的病变不能尽数了解,所以请教你,希望告诉我全部的道理。\\
岐伯说:风雨寒暑,如不得虚邪之气,也不能单独伤人。有人突然遇到疾风暴雨,但没生病,这大多是没有虚邪,因此不能伤人。疾病的发生,必有虚邪贼风,与人身体素虚,两虚相遇,邪气才能侵入人体。若气候正常,体质强健,两实相逢,多数人又皮肉坚实,虚邪不能侵害。为虚邪所伤,那必定是因为四时不正之气以及身体衰弱,体虚邪实,相互结合,才酿成大病。邪气有固定侵袭的部位,根据邪气停留部分,来命名。有上中下或表里半表半里,纵横分为三部。\\
所以虚邪伤害人体,从皮肤开始,皮肤弛缓则腠理开泄,腠理开泄,则邪气从毛发侵入,到达深部后,会促使毛发竖起。毛发竖起,就会感觉寒栗,皮肤疼痛。邪气留而不除,就会传入络脉。传入络脉,就会肌肉作痛;如疼痛止时,经脉就要代受其邪。滞留不除,就会传入于经脉。邪在经脉,寒栗恶冷,多惊。滞留不除,就会传入于输脉。邪在输脉,手之六经不通,四肢感到疼痛,腰脊不能屈伸。滞留不除,就会传入伏冲之脉,邪在于伏冲之脉,就会体重身痛,滞留不除,就会传入于肠胃。邪在肠胃,会腹胀肠鸣。多寒就要肠鸣泄泻,食物不化;多热就要便溏,赤白相兼。滞留不除,就会传入于肠胃之外,募原之间,留著在募原血脉之中。滞留不除,就会停在这里成为积块。总之,邪气侵入人体,或留著于孙脉,或留著于络脉,或留著于经脉,或留著于输脉,或留著于伏冲之脉,或留著于脊膂之筋,或留著于肠胃之募原,或留著于缓筋,邪气在体内泛滥,变化多端,不能说尽。\\
黄帝说:希望听听疾病形成的原由始末。\\
岐伯说:邪气留着孙络而成为积证,积块能够上下移动。因它聚在孙络,孙络浮而缓,不能使积块固定,所以它往来移入而慢慢进入肠胃之间。若有水液出现聚渗注灌于内,会发出濯濯的水声。若有寒气,就会腹部胀满雷鸣,相互牵引,经常急痛。如邪气留著于阳明之经,那积块就会夹在脐部周围,饱食后,脉络粗大;饥饿时,脉络细小。如邪气留著于缓筋,就像阳明经的积证一样,饱食后,感觉胀痛;饥饿时,反感觉舒适。如邪气留著肠胃募原之间,疼痛会向外牵连到缓筋,饱食后,感觉舒适;饥饿时,感到疼痛。如邪气留著于伏冲之脉,用手触摸有动感,手离开后,似有热气向两股下行,就像浇了热水一样。如邪气留著于膂筋,饥饿时,则积块可以看清;饱食后,积块不易看清,用手摸不到。如邪气留著于输脉,会使脉道闭塞不通,津液不能布散,致使孔窍干燥壅塞。这都是邪气自外而内,从上而下的一般表现。\\
黄帝问:积证从开始到形成,是怎样的?\\
岐伯说:积证开始发生,是由于感受了寒气,寒气上逆就形成了积证。\\
黄帝问:积证的具体形成过程,是怎样的?\\
岐伯说:感受寒厥之气,使两脚发生疼痛,行动不便,因此引起小腿寒冷,由于小腿寒冷,以致血脉凝涩,血脉凝涩,则寒气自下而上进入肠胃之中。寒气进入肠胃后,引起腹部胀满,胀满则肠胃之外的汁沫,为寒邪所迫,聚留不散,日久就形成积证。或因突然暴食暴饮,肠里充满食物,消化困难,又加上起居无节,用力过度,就会损伤络脉。如阳络伤,血就向外渗溢,血向外渗溢,就会鼻出血;如阴络伤,血就向内渗溢,血向内渗溢,就会便血。如肠胃的络脉伤,血就渗溢出肠外,如果肠外适有寒气,汁沫和溢出的血相搏结,就凝聚在一起散不开了,就成为积证。还有因突然感受寒邪,情绪忧怒,就会使气上逆,气上逆,则六经的经气会壅滞不通,卫气不行,血液凝结,蕴郁于里,不能散开,津液因而凝涩。这样,久留不除,也会形成积证。\\
黄帝问:疾病发生于内脏,又是怎样的?\\
岐伯说:忧思会伤心;重寒会伤肺;忿怒会伤肝;醉酒行房,出汗之后,当风受凉,会伤脾;用力过多,及房事后,汗出沐浴,会伤肾。这都是身体内外上中下三部所发生的病证。\\
黄帝说:说得好。怎样治疗呢?\\
岐伯答说:观察它疼痛的部位,了解病变所在。根据邪盛有余和正虚不足,当补的就补,当泻的就泻。不违反四时气候规律,这就是最好的治疗原则。\\
行针第六十七\\
黄帝问于岐伯曰:余闻九针于夫子,而行之于百姓,百姓之血气各不同形,或神动而气先针行,或气与针相逢,或针已出,气独行,或数刺乃知,或发针而气逆,或数刺病益剧。凡此六者,各不同形,愿闻其方。\\
岐伯曰:重阳之人,其神易动,其气易往也。\\
黄帝曰:何谓重阳之人?\\
岐伯曰:重阳之人,熇熇高高,言语善疾,举足善高,心肺之脏气有余,阳气滑盛而扬,故神动而气先行。\\
黄帝曰:重阳之人而神不先行者,何也?\\
岐伯曰:此人颇有阴者也。\\
黄帝曰:何以知其颇有阴也?\\
岐伯曰:多阳者多喜,多阴者多怒,数怒者易解,故曰颇有阴,其阴阳之离合难,故其神不能先行也。\\
黄帝曰:其气与针相逢,奈何?\\
岐伯曰:阴阳和调而血气淖泽滑利,故针入而气出,疾而相逢也。\\
黄帝曰:针已出而气独行者,何气使然?\\
岐伯曰:其阴气多而阳气少,阴气沉而阳气浮,其气沉者内藏,故针已出,气乃随其后,故独行也。\\
黄帝曰:数刺乃知,何气使然?\\
岐伯曰:此人之多阴而少阳,其气沉而气往难,故数刺乃知也。\\
黄帝曰:针入而气逆者,何气使然?\\
岐伯曰:其气逆与其数刺病益甚者,非阴阳之气,浮沉之势也,此皆粗之所败,工之所失,其形气无过焉。\\
黄帝问岐伯说:我从夫子那里听到了关于九针的知识,运用针刺技术为百姓治病,百姓的血气盛衰的不同,针下的反应也不同,有的人精神紧张,还没有针刺,生理上就有反应;有的人进针马上就有得气的感觉;有的人出针后才有反应;有的人针刺几次后,才渐有反应;有的人刚下针,就出现不良反应;有的人针治几次,反而病情加重。这六种情况,表现各不相同,希望听听其中的道理。\\
岐伯说:重阳的人,他的心神易动,容易有针感。\\
黄帝问:什么样是重阳之人呢?\\
岐伯说:重阳之人,性格像火一样炽热,高傲不屈人下,说话语速快,走路高抬足,心肺二脏之气有余,阳气滑利充盛而激扬发越,所以神易动,容易有针感。\\
黄帝问:有些重阳之人,他的心神并不易激动,为什么呢?\\
岐伯说:这种重阳之人,是略有阴气在内。\\
黄帝问:怎么知道他略有阴气在内呢?\\
岐伯说:多阳者,常有喜悦之情,多阴者,常易发怒,但气消得也快,所以说是阳中略有阴,这种人的阴阳结合困难,所以其心神气不易激动,神气不能先行。\\
黄帝问:有的人进针后,马上有得气的感觉,为什么呢?\\
岐伯说:这是由于阴阳协调,气血润泽和畅,所以进针后很快就会得气。\\
黄帝问:有的人出针后才有反应,这是什么气使他这样的呢?\\
岐伯说:他的阴气多而阳气少,阴气主沉,阳气主浮,阴气主沉潜敛藏,所以出针后,针感随后才出现,这就是“独行”。\\
黄帝问:几次针刺后才有反应,是什么气使然呢?\\
岐伯说:这种人阴气多而阳气少,其气沉降而气至难,所以要几次针刺,才出现反应。\\
黄帝问:有的人刚进针,就出现晕针反应,这是什么气使然呢?\\
岐伯说:针刺后气逆和多次针刺后病情加重的,绝不是阴阳的盛衰,或经气的浮沉所致,而是医生的草率,治疗失误造成的,而病人的形气体质没有过错。\\
上膈第六十八\\
黄帝曰:气为上膈,上膈者,食入而还出,余已知之矣。虫为下膈,下膈者,食晬时乃出,余未得其意,愿卒闻之。\\
岐伯曰:喜怒不适,食饮不节,寒温不时,则寒汁流于肠中,流于肠中则虫寒,虫寒则积聚,守于下管,则肠胃充郭,卫气不营,邪气居之。人食则虫上食,虫上食则下管虚,下管虚则邪气胜之,积聚以留,留则痈成,痈成则下管约。其痈在管内者,则沉而痛深;其痈在外者,则外而痛浮,痈上皮热。\\
黄帝曰:刺之奈何?\\
岐伯曰:微按其痈,视气所行,先浅刺其傍,稍内益深,还而刺之,毋过三行。察其沉浮,以为深浅。已刺必熨,令热入中,日使热内,邪气益衰,大痈乃溃。参伍以禁,以除其内,恬惔无为,乃能行气,后以咸苦,化谷乃下矣。\\
黄帝说:由于气机郁结而形成上膈证,上膈证饮食入胃马上就吐出,我已知道了。因虫积而形成下膈证,进食后一昼夜才吐出,我还不了解其道理,希望详尽听听。\\
岐伯说:由于喜怒等情志不畅,饮食不节制,寒温不按时调适,损伤胃气,寒湿流注于肠中,肠中寒湿流注,肠寄生虫觉得寒冷,虫得寒则积聚一团,盘踞在下脘,这样肠胃形成壅塞充胀,脾胃阳气不得温通,邪气稽留在此。人饮食时,虫也向上求食,虫上行求食时下方空虚,邪气乘虚侵入,积聚停留在内,稽留日久,就形成内痈,内痈既成,会使拘束不利。痈在下脘之内的,沉而痛深;痈在下脘外面的,浅而痛浮,在痈的部位上,皮肤发热。\\
黄帝问:怎样针刺治疗呢?\\
岐伯说:刺治之前,用手轻按患部,以观察病气的走向,先浅刺痈肿周围,进针后稍有感觉,再逐渐深刺,然后照样反复进行刺治,但不能超过三次。观察病位的深浅,来确定针刺的深浅。针刺以后,必须加以热熨法,使热气进入内部,只要阳气日渐温通,邪气就会日趋衰退,内痈自然溃散。再配合适当的调理,不要触犯禁忌,以消除体内的致病因素,清虚恬惔,无欲无求,元气才能畅达,然后再服用咸苦之味,饮食就能消化而传下,病也就好了。\\
忧恚无言第六十九\\
黄帝问于少师曰:人之卒然忧恚而言无音者,何道之塞,何气不行,使音不彰?愿闻其方。\\
少师答曰:咽者,水谷之道也。喉咙者,气之所以上下者也。会厌者,音声之户也,口唇者,音声之扇也。舌者,音声之机也。悬雍垂者,音声之关也。颃颡者,分气之所泄也。横骨者,神气之所使,主发舌者也。故人之鼻洞涕出不收者,颃颡不开,分气失也。是故厌小而薄,则发气疾,其开阖利,其出气易;其厌大而厚,则开阖难,其气出迟,故重言也。人卒然无音者,寒气客于厌,则厌不能发,发不能下至,其开阖不致,故无音。\\
黄帝曰:刺之奈何?\\
岐伯曰:足之少阴,上系于舌,络于横骨,终于会厌。两泻其血脉,浊气乃辟。会厌之脉,上络任脉,取之天突,其厌乃发也。\\
黄帝问少师说:有的人因突然忧郁或愤怒而失音,是体内哪条气血的道路阻塞?是什么气机不通畅,以致声不响亮呢?希望听听其中的道理。\\
少师回答说:咽部是胃受纳水谷的必经通路。喉咙是呼吸之气上下出入的要道。会厌好比是音声的门户,口唇好比是音声的门扇。舌好比是音声的机关。悬雍垂好比是音声的关隘。颃颡是气从此分出到口鼻,鼻涕和唾液,从此而出。横骨受神气支配,为控制舌体运动的枢机。所以人患鼻渊,流涕不止,这是颃颡开阖不利,分气失职的原因。所以会厌薄小的人,呼气畅快,开阖流利,他出气容易,所以音声爽利;会厌厚大的,就开阖不利,出气迟缓,所以说话口吃,声音不清。人突然失音的,是因为风寒侵入会厌,以致会厌不能开,或开而不能阖,开阖失利,就形成了失音证。\\
黄帝问:怎样来针刺治疗呢?\\
岐伯说:足少阴肾的经脉,向上走行,联系到舌根,联络横骨,终止于会厌。针刺治时,两次泻足少阴经上联于会厌的血脉,浊气才能除去。会厌的脉络向上与任脉相联,再针刺任脉的天突穴,会厌就可以恢复开阖而能正常发音了。\\
寒热第七十\\
黄帝问于岐伯曰:寒热瘰疬在于颈腋者,皆何气使生?\\
岐伯曰:此皆鼠瘘寒热之毒气也,留于脉而不去者也。\\
黄帝曰:去之奈何?\\
岐伯曰:鼠瘘之本,皆在于脏,其末上出于颈腋之间。其浮于脉中,而未内著于肌肉,而外为脓血者,易去也。\\
黄帝曰:去之奈何?\\
岐伯曰:请从其本,引其末,可使衰去,而绝其寒热。审按其道,以予之;徐往徐来,以去之。其小如麦者,一刺知,三刺而已。\\
黄帝曰:决其生死,奈何?\\
岐伯曰:反其目视之,其中有赤脉,上下贯瞳子。见一脉,一岁死;见一脉半,一岁半死;见二脉,二岁死;见二脉半,二岁半死;见三脉,三岁而死。见赤脉不下贯瞳子,可治也。\\
黄帝问岐伯说:恶寒发热的瘰疬病,生在颈腋之下,是什么气使之发生的?\\
岐伯说:这都是鼠瘘证,是寒热的毒气稽留在经脉中不能排除而致。\\
黄帝问:如何除去呢?\\
岐伯说:鼠瘘的病根,都在内脏,它出现标末现象即症状,却上出于颈腋之间。如果毒气浅浮,仅在脉中,还没有内伤肌肉,只是外面腐化为脓血的,较容易治愈。\\
黄帝问:怎样除去呢?\\
岐伯说:从致病的根源着手来充实五脏正气,以使毒气排出,就可以使邪毒渐渐消退,停止寒热发作。要察明鼠瘘的部位,按照主病的脏腑经脉,给予适当的针刺,用针时应缓入缓出,达到祛除病邪的目的。瘰疬初起,形小如麦粒,一次针刺就有效,针三次就可以治愈。\\
黄帝问:怎样决断这种病人的生死呢?\\
岐伯说:翻开眼皮观察,如果眼中有赤脉,从上下贯穿瞳子。出现一条赤脉的,死期在一年;出现一条半赤脉的,死期在一年半;出现两条赤脉的,死期在两年;出现两条半赤脉的,死期在两年半;出现三条赤脉的,死期在三年。没有出现赤脉并下贯瞳子的,可以医治。\\
邪客第七十一\\
黄帝问于伯高曰:夫邪气之客人也,或令人目不瞑、不卧出者,何气使然?\\
伯高曰:五谷入于胃也,其糟粕、津液、宗气分为三隧。故宗气积于胸中,出于喉咙,以贯心脉,而行呼吸焉。营气者,泌其津液,注之于脉,化以为血,以荣四末,内注五脏六腑,以应刻数焉。卫气者,出其悍气之慓疾,而先行于四末分肉皮肤之间,而不休者也。昼行于阳,夜行于阴,常从足少阴之分间,行于五脏六腑。今厥气客于五脏六腑,则卫气独卫其外,行于阳,不得入于阴。行于阳则阳气盛,阳气盛则阳\\
满,不得入于阴,阴虚故目不瞑。\\
黄帝曰:善。治之奈何?\\
伯高曰:补其不足,泻其有余,调其虚实,以通其道,而去其邪。饮以半夏汤一剂,阴阳已通,其卧立至。\\
黄帝曰:善。此所谓决渎壅塞,经络大通,阴阳和得者也。愿闻其方。\\
伯高曰:其汤方以流水千里以外者八升,扬之万遍,取其清五升煮之,炊以苇薪火,沸,置秫米一升,治半夏五合,徐炊,令竭为一升半,去其滓,饮汁一小杯,日三,稍益,以知为度。故其病新发者,覆杯则卧,汗出则已矣;久者,三饮而已也。\\
黄帝问于伯高曰:愿闻人之肢节,以应天地奈何?\\
伯高答曰:天圆地方,人头圆足方以应之。天有日月,人有两目。地有九州,人有九窍。天有风雨,人有喜怒。天有雷电,人有音声。天有四时,人有四肢。天有五音,人有五脏。天有六律,人有六腑。天有冬夏,人有寒热。天有十日,人有手十指。辰有十二,人有足十指、茎、垂以应之,女子不足二节,以抱人形。天有阴阳,人有夫妻。岁有三百六十五日,人有三百六十五节。地有高山,人有肩膝。地有深谷,人有腋腘。地有十二经水,人有十二经脉。地有泉脉,人有卫气。地有草蓂,人有毫毛。天有昼夜,人有卧起。天有列星,人有牙齿。天有小山,人有小节。地有山石,人有高骨。地有林木,人有募筋。地有聚邑,人有夬肉。岁有十二月,人有十二节。地有四时不生草,人有无子。此人与天地相应者也。\\
黄帝问于岐伯曰:余愿闻持针之数,内针之理,纵舍之意,扞皮开腠理,奈何?脉之屈折,出入之处,焉至而出,焉至而止,焉至而徐,焉至而疾,焉至而入?六腑之腧于身者?余愿尽闻其序。别离之处,离而入阴,别而入阳,此何道而从行?愿尽闻其方。\\
岐伯曰:帝之所问,针道毕矣。\\
黄帝曰:愿卒闻之。\\
岐伯曰:手太阴之脉,出于大指之端,内屈,循白肉际,至本节之后太渊,留以澹,外屈,上于本节下,内屈,与阴诸络会于鱼际,数脉并注,其气滑利,伏行壅骨之下,外屈,出于寸口而行,上至于肘内廉,入于大筋之下,内屈,上行臑阴,入腋下,内屈走肺。此顺行逆数之屈折也。\\
心主之脉,出于中指之端,内屈,循中指内廉以上,留于掌中,伏行两骨之间,外屈,出两筋之间,骨肉之际,其气滑利,上二寸,外屈,出行两筋之间,上至肘内廉,入于小筋之下,留两骨之会,上入于胸中,内络于心脉。\\
黄帝曰:手少阴之脉独无腧,何也?\\
岐伯曰:少阴,心脉也。心者,五脏六腑之大主也,精神之所舍也,其脏坚固,邪弗能容也。容之则心伤,心伤则神去,神去则死矣。故诸邪之在心者,皆在于心之包络。包络者,心主之脉也,故独无腧焉。\\
黄帝曰:少阴独无腧者,不病乎?\\
岐伯曰:其外经病而脏不病,故独取其经于掌后锐骨之端。其余脉出入屈折,其行之徐疾,皆如手太阴、心主之脉行也。故本腧者,皆因其气之虚实疾徐以取之,是谓因冲而泻,因衰而补。如是者,邪气得去,真气坚固,是谓因天之序。\\
黄帝曰:持针纵舍,奈何?\\
岐伯曰:必先明知十二经脉之本末,皮肤之寒热,脉之盛衰滑涩。其脉滑而盛者,病日进;虚而细者,久以持;大以涩者,为痛痹;阴阳如一者,病难治。其本末尚热者,病尚在,其热已衰者,其病亦去矣。持其尺,察其肉之坚脆、大小、滑涩、寒温、燥湿。因视目之五色,以知五脏而决死生。视其血脉,察其色,以知其寒热痛痹。\\
黄帝曰:持针纵舍,余未得其意也。\\
岐伯曰:持针之道,欲端以正,安以静,先知虚实,而行疾徐,左手执骨,右手循之,无与肉果。泻欲端以正,补必闭肤,辅针导气,邪得淫泆,真气得居。\\
黄帝曰:扞皮开腠理,奈何?\\
岐伯曰:因其分肉,左别其肤,微内而徐端之,适神不散,邪气得去。\\
黄帝问于岐伯曰:人有八虚,各何以候?\\
岐伯答曰:以候五脏。\\
黄帝曰:候之奈何?\\
岐伯曰:肺心有邪,其气留于两肘;肝有邪,其气留于两腋;脾有邪,其气留于两髀;肾有邪,其气留于两腘;凡此八虚者,皆机关之室;真气之所过,血络之所游,邪气恶血,固不得住留,住留则伤筋络,骨节机关不得屈伸,故病挛也。\\
黄帝问伯高说:邪气侵犯人体,有时使人不能闭目入睡安卧,是什么气造成的?\\
伯高说:饮食五谷进入胃中,它的糟粕、津液、宗气分为三条隧道。宗气积聚在胸中,上出喉咙,贯通心脉,而行呼吸。营气分泌津液,灌注到脉中,化为血液,向外营养四肢,向内灌注五脏六腑,循行于周身与昼夜百刻计数相应。卫气是水谷化生的慓悍之气,首先循行于四肢的分肉、皮肤之间,而不停息。白天出表,夜间入里,常以足少阴肾经为起点,循行于五脏六腑。如有邪气侵入五脏六腑,使得卫气只能行于阳分,而不得入于阴分。卫气只能循行于阳分,则阳气偏盛,阳气偏盛则使阳刉脉气充满,不得入于阴分,而致阴虚,所以不能闭目而眠。\\
黄帝说:好!怎样治疗呢?\\
伯高说:用针刺补其阴分的不足,泻其阳分的有余,调理虚实,沟通阴阳交会的隧道,从而消除邪气。再饮用半夏汤一剂,使阴阳经气畅通,马上可以安卧入睡。\\
黄帝说:讲得好!这种治法,好比是开决水道,疏通淤塞,使经络畅通,阴阳得以调和。希望听听半夏汤方的配制。\\
伯高说:半夏汤方,是用千里以外的长流水八升,用杓扬之万遍,取其中轻浮在上的清水五升,用苇薪做燃料,用急火煮沸后,放入秫米一升,制半夏五合,接着用苇火慢慢煎熬,煎至药汤浓缩到一升半时,除去药渣,每次饮药汁一小杯,一日服三次,逐渐加量,以见效为度。如果是新发的病,服完药后立即就能安眠,出汗就好了;病程较长久的,要服三剂才能痊愈。\\
黄帝问伯高说:希望听听人的四肢百节,是怎样和天地相应的?\\
伯高回答说:天是圆的,地是方的,人的头是圆的,足是方的,与之相应。天上有日月,人有两眼。地下有九州,人身有九窍。天有风雨,人有喜怒。天有雷电,人有声音。天有四季,人有四肢。天有五音,人有五脏。天有六律,人有六腑。天有冬夏,人有寒热。天干有十,人的手指有十。地支有十二,人的十趾加上阴茎、睾丸也是十二,女子缺少两节,但能够怀孕。天有阴阳,人有夫妻。一年有三百六十五天,人有三百六十五个腧穴。地有高山,人有肩、膝。地有深谷,人有腋窝和腘窝;地上有十二大河流,人体有十二条主要经脉。地下有泉脉流通,人体有卫气运行。地上生长丛草,人身长有毫毛。天有昼夜,人有起卧。天上有群星,人口内有牙齿。地上有小山,人体有小关节。地有山石,人有颧肩膝踝等高骨。地上有树林,人体内有筋膜。地上有人群聚集的城镇,人体有肌肉隆起的刐肉。一年有十二个月,人的四肢有十二大关节。大地有四时不生草木的,人也有终生不生育子女的。这些,就是人体和天地相应的具体情况。\\
黄帝问岐伯说:我希望听听针刺中持针的技巧,进针的道理,纵舍迎随的手法,以及扞分皮肤,开达腠理的刺法,是怎样的?还有经脉的屈折,和出入之处,在脉气流注中,它到哪而出,到哪而止,到哪而慢,到哪而快,到哪而入?六腑的腧穴是怎么流注到全身的?我希望都听听其中的道理。还有经脉的支别离合之处,阳经是怎样从腧穴别出走入阴经,阴经又是怎样由腧穴别出走入阳经?这些都是从哪些条道路来运行沟通的?希望详细听听这些道理。\\
岐伯说:你所提到的问题,针刺的道理尽在其中了。\\
黄帝说:希望全部听听。\\
岐伯说:手太阴的经脉,从手大指的尖端发出,向内屈折,沿着内侧的白肉际,到达大指本节后的太渊穴,经气流注于此,略作停留,而形成寸口动脉,然后向外屈折,上行到达本节之下,又向内屈行,和诸阴络会合在鱼际,由于几条阴经之脉都灌注于此,其脉气流动滑利,伏行于壅骨之下,再由此向外屈折,浮出寸口而循经上行,向上到达肘内侧,进入大筋之下,又向内屈折上行,通过臑部内侧走入腋下,向内屈行走入肺中。这是按照手太阴肺经的顺行和逆行径路的屈折、出入情况。\\
手厥阴心主的经脉,从中指尖端发出,由此向内屈折,沿着中指内侧上行,流注到掌中,伏行在两骨之间,向外屈行,出于两筋的中间,骨肉的交界处,它的脉气圆滑流利,上行二寸后,向外屈折,出行于两筋的中间,向上到肘内侧,进入小筋之下,流注到两骨的交会处,再上行进入胸中,向内连络心脉。\\
黄帝问:唯独手少阴经脉,没有自己的腧穴,是什么道理呢?\\
岐伯说:手少阴经,是心的经脉。心是五脏六腑的主宰,又是精神的聚居地,它的器质坚固,不容许邪气侵入。一旦有邪气侵入,就会损伤心脏,心脏损伤,精神气就会耗散离去,人就死了。所以各种病邪侵犯心脏的,都在心的包络上。包络,是心主宰的经脉。所以唯独手少阴心经没有腧穴。\\
黄帝问:手少阴经没有腧穴,就不会生病了吗?\\
岐伯说:在外的手少阴经脉会受病,而内在的心脏不会受病,所以在外经有病时,只需取它的经脉在掌后锐骨之端的神门穴。其余经脉的出入屈折,运行的快慢,都与手太阴、心主二脉循行的情况相似。所以本经的输穴神门,都要根据经气的虚实缓急来取治,这就是所谓的因邪气盛而用泻法,因正气虚而用补法。这样,使邪气可以消除,而真气可以坚固,这种治法,叫因顺天然秩序的治法。\\
黄帝问道:持针纵舍的手法是怎样的呢?\\
岐伯说:首先必须清楚地明白十二经脉的起止循行,以及诊察皮肤的寒热,脉象的盛衰、滑涩。如脉滑而盛大的,是病情日渐严重;脉虚弱而细微的,是迁延不愈的久病;脉大而涩的,是痛痹;表里俱伤,气血皆败的,病难治。凡胸腹和四肢还在发热的,是病邪还在,热势衰减病邪也将离去了。诊尺肤以观察患者肌肉的坚或脆、大或小、滑或涩、寒或温、燥或湿。根据两目的五色变化,以分辨五脏的病变,决断生死。观察血络反映于外部的色泽,以了解寒热痛痹等证。\\
黄帝说:持针纵舍的具体手法,我还不太理解。\\
岐伯说:用针的方法在于端正身形,内心安静,先了解疾病的虚实,然后再运用缓急补泻的手法,用左手按定骨骼的位置,右手循穴进针,但不可用力过猛,以防针被肉裹。泻法必须垂直下针,补法出针时,必须闭其针孔,并用辅助行针的手法,以疏导经气,使邪气消散,真气内守。\\
黄帝问:扞皮肤,开腠理,怎样操作呢?\\
岐伯说:顺着病人的分肉,左手扞别开腧穴出的皮肤,然后微微用力进针,针刺方向要端正,不能偏斜,恰好不致神气散乱而又能开泄腠理,病邪可以散去。\\
黄帝问岐伯说:人身的八虚,各自用来诊察哪些疾病呢?\\
岐伯说:用来诊察五脏的病变。\\
黄帝问:怎样诊察呢?\\
岐伯说:肺与心两经有了邪气,邪气留住在左右两肘;肝经有了邪气,邪气流注到两腋窝;脾经有了邪气,邪气留住到两髀;肾精有了邪气,邪气留住到两腘。这八个虚弱部位,都是四肢关节屈伸的枢纽,是真气运行所过和血络游行之处,所以邪气恶血不可留住在这些部位,如有邪气恶血留住,就会损伤筋络筋骨,以致关节不能屈伸,所以发生拘挛的症状。\\
通天第七十二\\
黄帝问于少师曰:余尝闻人有阴阳,何谓阴人,何谓阳人?\\
少师曰:天地之间,六合之内,不离于五,人亦应之,非徒一阴一阳而已也。而略言耳,口弗能遍明也。\\
黄帝曰:愿略闻其意,有贤人圣人,心能备而行之乎?\\
少师曰:盖有太阴之人,少阴之人,太阳之人,少阳之人,阴阳和平之人。凡五人者,其态不同,其筋骨气血各不等。\\
黄帝曰:其不等者,可得闻乎?\\
少师曰:太阴之人,贪而不仁,下齐湛湛,好内而恶出,心和而不发,不务于时,动而后之,此太阴之人也。\\
少阴之人,小贪而贼心,见人有亡,常若有得,好伤好害,见人有荣,乃反愠怒,心疾而无恩。此少阴之人也。\\
太阳之人,居处于于,好言大事,无能而虚说,志发于四野,举措不顾是非,为事如常自用,事虽败而常无悔。此太阳之人也。\\
少阳之人,扐谛好自贵,有小小官,则高自宜,好为外交而不内附。此少阳之人也。\\
阴阳和平之人,居处安静,无为惧惧,无为欣欣,婉然从物,或与不争,与时变化,尊则谦谦,谭而不治,是谓至治。古人善用针艾者,视人五态乃治之。盛者泻之,虚者补之。\\
黄帝曰:治人之五态奈何?\\
少师曰:太阴之人,多阴而无阳。其阴血浊,其卫气涩。阴阳不和,缓筋而厚皮。不之疾泻,不能移之。少阴之人,多阴少阳,小胃而大肠,六腑不调。其阳明脉小而太阳脉大,必审调之。其血易脱,其气易败也。\\
太阳之人,多阳而少阴。必谨调之,无脱其阴,而泻其阳。阳重脱者易狂,阴阳皆脱者,暴死,不知人也。\\
少阳之人,多阳少阴,经小而络大。血在中而气在外,实阴而虚阳,独泻其络脉则强,气脱而疾,中气不足,病不起也。\\
阴阳和平之人,其阴阳之气和,血脉调。谨诊其阴阳,视其邪正,安容仪。审有余不足。盛则泻之,虚则补之,不盛不虚,以经取之。此所以调阳阳,别五态之人者也。\\
黄帝曰:夫五态之人者,相与毋故,卒然新会,未知其行也,何以别之?\\
少师答曰:众人之属,不如五态之人者,故五五二十五人,而五态之人不与焉。五态之人,尤不合于众者也。\\
黄帝曰:别五态之人奈何?\\
少师曰:太阴之人,其状黮黮然黑色,念然下意,临临然长大,腘然未偻。此太阴之人也。\\
少阴之人,其状清然窃然,固以阴贼,立而躁崄,行而似伏。此少阴之人也。\\
太阳之人,其状轩轩储储,反身折腘。此太阳之人也。\\
少阳之人,其状立则好仰,行则好摇,其两臂两肘则常出于背。此少阳之人也。\\
阴阳和平之人,其状委委然,随随然,颙颙然,愉愉然,氶氶然,豆豆然,众人皆曰君子。此阴阳和平之人也。\\
黄帝问少师说:我曾经听说人有阴与阳的不同,什么是属阴的人?什么是属阳的人?\\
少师说:天地之间,四方上下之内,都离不开五行,人也和五行相应,并不是仅有相对的一阴一阳而已。这只是大概一说,至于其复杂情形,用语言难以说清。\\
黄帝问:希望听到大概的情况,有贤人圣人,他们是否能够达到阴阳平衡呢?\\
少师说:人大致可以分为太阴、少阴、太阳、少阳、阴阳和平五种类型。这五种类型的人,他们的形态不同,筋骨强弱,气血盛衰,也各不相同。\\
黄帝说:那不同情况,可以让我听听吗?\\
少师说:属于太阴的人,性情贪婪不仁厚,表面谦虚,假装正经,内心却深藏阴险,好得恶失,喜怒不形于色,不识时务,只知利己,看风使舵,行动上惯用后发制人的手段。具有这些特性的,就是太阴之人。\\
属于少阴的人,贪图小利,而有害人之心,看到别人有了损失,就像拣到便宜一样高兴,好伤人,好害人,看到别人光荣,就恼怒,心怀嫉妒,没有同情心。有这些特征的,就是少阴之人。\\
属于太阳的人,平时自鸣得意,好讲大事,无能却空说大话,言过其实,好高骛远。行动不顾是非,做事经常自以为是,做事虽然失败,却没有后悔之心。有这些特征的,就是太阳之人。\\
属于少阳的人,做事审慎,好抬高自己,有了小小的官职,就自以为了不起,向外宣扬,好交际,而不能踏踏实实地工作。有这些特征的就是少阳之人。\\
属于阴阳和平的人,生活安静,心安无所畏惧,不追求过分喜乐,顺从事物发展的自然规律,遇事不与人争,善于适应形势的变化,地位虽高却很谦虚,以理服人,而不是用压服的手段来治人,具有极好的治理才能。具有这些特性的,就是阴阳和平之人。古代善用针灸疗法的医生,观察五类人的形态,分别给以治疗。气盛的用泻法,气虚的用补法。\\
黄帝问:针治五种形态的人,是怎样的?\\
少师说:属于太阴的人,阴偏多,却无阳。他们的阴血重浊,卫气涩滞。阴阳不调和,形体表现为筋缓皮厚的特征。像这样的人,不用急泻针法,就不能去除他的病。属于少阴的人,阴多阳少,他们的胃小而肠大,六腑的功能不协调。因为他的足阳明经脉气偏小,而手太阳经脉气偏大,一定要审慎调治。因为他的血容易耗损,他的气也容易败伤。\\
属于太阳的人,阳多阴少。一定谨慎地进行调治,不能再耗损其阴,只可泻其阳。阳大脱就易发狂躁,如果阴阳都耗损就会突然死亡,或不省人事。\\
属于少阳的人,阳多阴少,经脉小而络脉大。血在中而气在外,在治疗时,应当充实阴经而泻其阳络,但是单独过度地泻其阳络,就会迫使阳气很快的耗损,以致中气不足,病也就难以痊愈了。\\
属于阴阳和平的人,他们的阴阳之气和谐,血脉调顺。在治疗时,应当谨慎地观察他的阴阳变化,了解他的邪正盛衰,看明他的容颜表现。然后细审是哪一方面有余,哪一方面不足。邪盛用泻法,正虚用补法,如果不盛不虚,就治疗病证所在的本经。这就是调治阴阳,辨别五种不同形态人的标准。\\
黄帝问:与五种形态的人,素不相识,乍一见面,很难知道他们的作风和性格属于哪一类型的人,应怎样来辨别呢?\\
少师回答说:一般人不具备这五种人的特性,所以“阴阳二十五人”,不包括在五态人之内。因为五态之人是具有代表性的五种类型,他们和一般人是不相同的。\\
黄帝问:怎样分别五种形态的人呢?\\
少师说:属于太阴的人,面色阴沉黑暗,而假意谦虚,身体本来高大,却卑躬屈膝,故作姿态,而并非真有佝偻病,这就是太阴之人的形态。\\
属于少阴的人,外貌好像清高,但是行动鬼祟,偷偷摸摸,深怀阴险害人之贼心,站立时躁动不安,显示出邪恶之象,走路时状似伏身向前。这是少阴之人的形态。\\
属于太阳的人,外貌表现高傲自满,仰腰挺胸,好像身躯向后反张和两腘曲折那样。这是太阳之人的形态。\\
属于少阳的人,在站立时惯于把头仰得很高,行走时惯于摇摆身体,常常反挽其手于背后。这是少阳之人的形态。\\
属于阴阳和平的人,外貌从容稳重,举止大方,性格和顺,善于适应环境,态度严肃,品行端正,待人和蔼,目光慈祥,作风光明磊落,举止有度,处事条理分明,众人都说有德行。这是阴阳和平之人的形态。\\
卷十一\\
官能第七十三\\
黄帝问于岐伯曰:余闻九针于夫子,众多矣,不可胜数。余推而论之,以为一纪。余司诵之,子听其理。非则语余,请其正道,令可久传,后世无患。得其人乃传,非其人勿言。\\
岐伯稽首再拜曰:请听圣王之道。\\
黄帝曰:用针之理,必知形气之所在,左右上下,阴阳表里。血气多少,行之逆顺,出入之合,谋伐有过。知解结,知补虚泻实,上下气门,明通于四海,审其所在。寒热淋露,以输异处。审于调气,明于经隧,左右肢络,尽知其会。寒与热争,能合而调之;虚与实邻,知决而通之。左右不调,把而行之。明于逆顺,乃知可治;阴阳不奇,故知起时。审于本末,察其寒热,得邪所在,万刺不殆。知官九针,刺道毕矣。\\
明于五输,徐疾所在。屈伸出入,皆有条理。言阴与阳,合于五行。五藏六府,亦有所藏。四时八风,尽有阴阳。各得其位,合于明堂。各处色部,五藏六府。察其所痛,左右上下;知其寒温,何经所在。审皮肤之寒温滑涩,知其所苦;膈有上下,知其气所在。先得其道,稀而疏之。稍深以留,故能徐入之。大热在上,推而下之。从下上者,引而去之。视前痛者,常先取之。大寒在外,留而补之。入于中者,从合泻之。针所不为,灸之所宜。上气不足,推而扬之。下气不足,积而从之。阴阳皆虚,火自当之。厥而寒甚,骨廉陷下。寒过于膝,下陵三里。阴络所过,得之留止。寒入于中,推而行之。经陷下者,火则当之。结络坚紧,火所治之。不知所苦,两\\
之下。男阳女阴,良工所禁。针论毕矣。\\
用针之服,必有法则。上视天光,下司八正,以辟奇邪,而观百姓。审于虚实,无犯其邪。是得天之露,遇岁之虚。救而不胜,反受其殃。故曰:必知天忌,乃言针意。法于往古,验于来今。观于窈冥,通于无穷。粗之所不见,良工之所贵。莫知其形,若神仿佛。\\
邪气之中人也,洒淅动形。正邪之中人也,微先见于色,不知于其身。若有若无,若亡若存。有形无形,莫知其情。是故上工之取气,乃救其萌芽,下工守其已成,因败其形。\\
是故工之用针也,知气之所在,而守其门户。明于调气,补泻所在,徐疾之意,所取之处。泻必用员,切而转之,其气乃行。疾而徐出,邪气乃出。伸而迎之,摇大其穴,气出乃疾。补必用方,外引其皮,令当其门。左引其枢,右推其肤,微旋而徐推之。必端以正,安以静,坚心无解。欲微以留,气下而疾出之。推其皮,盖其外门,真气乃存。用针之要,无忘其神。\\
雷公问于黄帝曰:《针论》曰“得其人乃传,非其人勿言”,何以知其可传?\\
黄帝曰:各得其人,任之其能,故能明其事。\\
雷公曰:愿闻官能奈何?\\
黄帝曰:明目者,可使视色。聪耳者,可使听音。捷疾辞语者,可使传论。语徐而安静,手巧而心审谛者,可使行针艾,理血气而调诸逆顺,察阴阳而兼诸方。缓节柔筋而心和调者,可使导引行气。疾毒言语轻人者,可使唾痈咒病。爪苦手毒,为事善伤者,可使按积抑痹。各得其能,方乃可行,其名乃彰。不得其人,其功不成,其师无名。故曰“得其人乃传,非其人勿言”,此之谓也。手毒者,可使试按龟,置龟于器下而按其上,五十日而死矣;手甘者,复生如故也。\\
黄帝问岐伯说:我听你讲解九针之学,已经很多,都不能计数了。我推究其中的道理,经过归纳整理,成为系统的理论。我现在读出来给你听,你听其理论。有错误的地方,就告诉我加以修正,使它永远传给后世,以便学习和运用。当然要传教可靠的人,不能教不可靠的人。\\
岐伯叩头再拜说:我希望听一下圣王所讲的针道。\\
黄帝说:用针治病的道理,一定要知道脏腑形气所在,左右上下的部位,阴阳表里的关系,血气的或多或少,以及脉气在全身的逆行和顺行,和由里出表或由表入里的会合处所等等,这样,才能祛除邪气恶血。更要懂得解其结聚,了解补虚泻实,上下气穴;明确知道四海腧穴的部位及其生理病理表现。寒热雨露的不同病因,会侵入人体不同部位。要谨慎地调和脉气,必须搞清十二经脉及周身左右上下的支络的循行交会。若有寒热相争,能参合各种情况进行调治;对于虚实错杂,应能决断而调治。左右不协调,应用左病刺右,右病刺左的手法,只有明确经脉循行的顺逆,才知道怎样治疗;脏腑阴阳调和,就可预知病愈之时。审察清楚疾病的标本、寒热,确定邪气所在部位,针刺治疗就不会错误。再掌握了九针的不同性能,针刺方法就全面了。\\
明白五腧穴的主治范围和针刺的徐疾手法。经脉往来的屈伸出入,都有一定的规律。人体的阴阳与五行是相合的。五脏六腑,分别有藏神藏谷的功能。四时八风的变化,全有阴阳的关系。疾病各有其发生的部位,结合面部五色诊,寻求各部显现出的不同色泽,来诊察五脏六腑的疾病。观察疾病的部位,是在左在右还是在上在下;判断疾病的寒温属性,知道病在哪条经脉。审察皮肤的寒温滑涩,知道它的疾病属性;再诊察膈膜上下,可以知道病气所在。首先掌握经脉的通路,取穴要少而精。或如疾病深在则留针,使正气徐徐内入。病人上部大热,当用推而下之的针法。如病邪从下向上发展,就引病邪向下而排除。同时注意病人以前所患之病,应该先治前病,以除宿因。身体寒冷的,采用留针而补之使热的针法。如寒邪深入于里,从合穴泻去寒邪。凡不适应针刺的病,用灸法较适宜。上气不足的病,用推而扬之的方法。下气不足的病,当采用留针随气的针法。若阴阳皆虚的病,可以用火灸法治疗。厥逆而寒象重的,或骨侧的肌肉下陷,或寒冷超过两膝,宜灸三里穴。又如阴络所过之处,寒邪留滞在内,寒邪深入到了内脏,就当用推而行之的针法。经脉陷下的,就用艾灸治疗。络脉结而坚紧的,也用艾灸治疗。有不知确切部位的病痛,当灸阳刉所通的申脉穴和阴刉所通的照海穴,男子取阳刉,女子取阴刉,若男取阴刉而女取阳刉,就犯了治疗上的错误,这是技术精良的医生所禁忌的。懂得了这些,用针的理法就完备了。\\
用针治疗,一定要有法则。上要观察日月星辰之运行规律,下要了解四时八节气的不同,以避免四时不正之气,而提醒百姓知道,使他审察虚实,能够预防,不为邪气侵袭。如天风雨不时,或时令不正。医生救治,没有掌握气候变化的情况,反会使病情趋于危险。所以说:必须知道天时的宜忌,然后才谈得上针法的作用。取法古人的学术,用临床实践来检验。仔细观察那些微妙难见的变化,才能通晓变化无穷的疾病。这是粗率的医工认识不到,而良医认为宝贵的。之所以难知,是由于看不到形迹,好像神灵,若有若无。\\
邪气侵入人体后,出现寒栗怕冷的现象。正邪侵入人体,先略微表现在气色上,而身体没有异常感觉。像有病又像无病,像病邪消失,又像病邪还留存。像有病形,又像无病形,不易知道真实的病情。所以上工治病是在病的初期,根据脉气的变化进行治疗;下工不掌握这个方法,到病已形成以后,按常规治疗,这样会使病人的形体受到伤害。\\
所以医生用针,应该知道脉气的运行所在,按照相应的腧穴治疗。明白如何调气,什么应补,什么应泻,进针或快或慢,该取什么穴位。泻法须用流利圆活的手法,直迫病所而转针,正气就可正常运行。进针快而出针慢,邪气就会随针散出。进针时,屈伸而迎其气之来;出针时,摇大针孔,邪气就能很快排出。补法须用端正从容的手法,外引皮肤,使正当其穴。左手持针,右手推针进入皮肤,轻微捻转,缓缓进针。针者一定端正,精神安静,心坚不懈地进行刺治。待气至以后,要略微留针,等到邪气已出,就要赶快出针,随即按压穴位的皮肤,扪住针孔,真气就内存不泄。用针的关键在于千万勿忘“得神”。\\
雷公问黄帝说:《针论》所说“得其人乃传,非其人勿言”,怎样知道他是可以传授的人呢?\\
黄帝说:传授学术要分别选择适当的人材,教他可以胜任的工作,才能做好事业。\\
雷公说:希望听一下怎样才能量材取用呢?\\
黄帝说:目明的,可以使他看色泽。耳聪的,可以使他听声音。口齿流利的,善于言辞的,可以使他传达言论。语言徐缓安静的,手巧,心又仔细,可以使他操作针灸,以疏通血气,调治一切逆顺反常病证,观察阴阳变化而兼用各种治疗方法。手缓筋柔,心性和顺的,可以让他导引行气。嫉妒、刻薄,说话轻视人的,可以使他做唾痈祝病的事。爪甲粗,手下狠,做事爱伤人的,可以使他按揉积聚,治疗痹证。总之,使每个人,各尽其能,各种治疗方法,才可以推行,名声才可以显扬。如果传授的不得其人,不仅没有功效,其师也没有名誉。所以说“得其人乃言,非其人勿传”,就是这个意思。检验手毒的方法,可叫人试按乌龟,把乌龟放在器具下面,在上面按压,到五十天乌龟就死了;如果手善的,按压过五十天后,乌龟仍然活着。\\
论疾诊尺第七十四\\
黄帝问于岐伯曰:余欲无视色持脉,独调其尺,以言其病,从外知内,为之奈何?\\
岐伯曰:审其尺之缓急、小大、滑涩,肉之坚脆,而病形定矣。\\
视人之目窠上,微痈,如新卧起状,其颈脉动,时咳,按其手足上,窅而不起者,风水肤胀也。\\
尺肤滑以淖泽者,风也。尺肉弱者,解汃,安卧脱肉者,寒热,不治。尺肤滑而泽脂者,风也。尺肤涩者,风痹也。尺肤粗如枯鱼之鳞者,水泆饮也。尺肤热甚,脉盛躁者,病温也,其脉盛而滑者,病且出也。尺肤寒,其脉小者,泄、少气。尺肤炬然,先热后寒者,寒热也。尺肤先寒,久持之而热者,亦寒热也。\\
肘所独热者,腰以上热;手所独热者,腰以下热。肘前独热者,膺前热;肘后独热者,肩背热。臂中独热者,腰腹热。肘后廉以下三四寸热者,肠中有虫。掌中热者,腹中热;掌中寒者,腹中寒。鱼上白肉有青血脉者,胃中有寒。尺炬然热,人迎大者,当夺血。尺紧,人迎脉小甚,少气。悗有加,立死。\\
目赤色者病在心,白在肺,青在肝,黄在脾,黑在肾。黄色不可名者,病在胸中。\\
诊目痛,赤脉从上下者,太阳病;从下上者,阳明病;从外走内者,少阳病。\\
诊寒热瘰疬,赤脉上下至瞳子,见一脉,一岁死;见一脉半,一岁半死;见二脉,二岁死;见二脉半,二岁半死;见三脉,三岁死。\\
诊龋齿痛,按其阳明之来,有过者独热,在左左热,在右右热,在上上热,在下下热。\\
诊血脉者,多赤多热,多青多痛,多黑为久痹,多赤、多黑、多青皆见者,寒热。\\
身痛而色微黄,齿垢黄,爪甲上黄,黄疸也。安卧,小便黄赤,脉小而涩者,不嗜食。\\
人病,其寸口之脉与人迎之脉小大等,浮沉等者,病难已也。\\
女子手少阴脉动甚者,妊子。\\
婴儿病,其头毛皆逆上者,必死。耳间青脉起者,掣痛。\\
大便青瓣,飧泄,脉小者,手足寒,难已;飧泄,脉小,手足温,泄易已。\\
四时之变,寒暑之胜,重阴必阳,重阳必阴,故阴主寒,阳主热,故寒甚则热,热甚则寒。故曰:寒生热,热生寒,此阴阳之变也。故曰:冬伤于寒,春生瘅热;春伤于风,夏生飧泄肠澼;夏伤于暑,秋生痎疟;秋伤于湿,冬生咳嗽。是谓四时之序也。\\
黄帝问岐伯说:我想不使用望色、诊脉,仅靠诊察尺肤,来说明疾病,从外在变化,推测内在病情,怎样做呢?\\
岐伯说:诊察尺肤的缓或急、小或大、滑或涩,以及肌肉的坚实或脆弱,就可确定病情了。\\
看到病人的眼睑轻微浮肿,好像刚睡醒起来的样子,颈部人迎脉搏动明显,时时咳嗽,按压患者手足,深陷不起的,这是风水肤胀的证候。\\
尺肤滑而光泽的,是风病。尺部肌肉柔弱的,身体困倦,四肢懈怠,喜好睡眠,肌肉瘦削的,是寒热时发的虚痨证,不易治愈。尺肤滑润如膏脂的,是风病。尺肤涩滞不滑的,是风痹病。尺肤粗糙像干枯鱼鳞的,是脾土虚衰,水饮不化的溢饮病。尺肤灼热,脉盛大而躁动的,是温病,若脉象盛大而滑利的,是病邪将去之象。尺肤寒冷而脉小的,是泄泻与气虚的病。尺肤高热灼手,先热后冷的,是寒热病。尺肤先寒冷,久按之后感觉发热的,也是寒热病。\\
肘部皮肤单独发热的,主腰以上发热;手部皮肤单独发热的,主腰以下发热。肘前单独发热的,主胸发热;肘后单独发热的,主肩背发热。臂中单独发热的,主腰腹发热。肘后缘以下三四寸发热的,主肠中有虫。手掌发热的,主腹中发热;手掌发凉的,主腹中发凉。手鱼际白肉有青色血脉的,主胃中有寒。尺肤高热炙手,颈部人迎脉盛大的,主失血。尺肤紧,人迎脉小甚的,主气虚。若加有烦闷现象,会立即死亡。\\
目见赤色的,主病在心;见白色的,主病在肺;见青色的,主病在肝;见黄色的,主病在脾;见黑色的,主病在肾。黄色而夹杂其他颜色而不能辨明的,主病在胸中。\\
诊察目痛,有红色的络脉从上向下的,属于太阳经的病;从下向上的,属于阳明经的病;从外眼角向内行走的,属于少阳经的病。\\
诊察寒热瘰疬,眼中有赤脉从上向下贯穿瞳子,见一条赤脉,一年死;见一条半赤脉,一年半死;见两条赤脉,两年死;见两条半赤脉,两年半死;见三条赤脉,三年死。\\
诊察龋齿痛,按压阳明之脉,有病变的部位必单独发热,病在左侧的左侧热,病在右侧的右侧热,病在上的上热,病在下的下热。\\
诊察络脉时,皮肤多赤色络脉的,多属热证;多青色的,多属痛证;多黑色的,是久痹;多赤、多黑、多青皆见的,为寒热病。\\
身体疼痛,面色微黄,牙齿垢黄,指甲上也现黄色,是黄疸病。好安卧,小便黄赤,脉小而涩的,不嗜饮食。\\
有病的人,他的寸口脉和人迎脉搏动的小大以及浮沉相等的,属难治之病。\\
妇女手少阴心脉动甚的,主怀孕。\\
婴儿有病,头发都向上竖起的,必定死亡。耳部络脉色青而隆起的,主抽搐腹痛。\\
大便绿色有乳瓣,泄下完谷不化,脉小弱,手足寒冷的,其病难治;若泄泻脉小,手足温暖的,易治。\\
四季的气候变化,寒暑往复,阴盛至极则转变为阳,阳盛至极则转变为阴,阴性主寒,阳性主热。所以寒到极点就会变热,热到极点就会变寒。所以说寒能生热,热能生寒,这是阴阳变化的道理。所以说:冬天感受寒邪,到了春天就发生温热病;春天感受了风邪,到了夏天就发生泄泻、痢疾病;夏天感受暑邪,到了秋天就容易发生疟疾;秋天感受了湿邪,到了冬天就发生咳嗽病。这是因为四季气候不同,因而生病的规律。\\
刺节真邪第七十五\\
黄帝问于岐伯曰:余闻刺有五节,奈何?\\
岐伯曰:固有五节:一曰振埃,二曰发蒙,三曰去爪,四曰彻衣,五曰解惑。\\
黄帝曰:夫子言五节,余未知其意。\\
岐伯曰:振埃者,刺外经,去阳病也。发蒙者,刺腑输,去腑病也。去爪者,刺关节之支络也。彻衣者,尽刺诸阳之奇输也。解惑者,尽知调阴阳,补泻有余不足,相倾移也。\\
黄帝曰:刺节言振埃,夫子乃言刺外经,去阳病,余不知其所谓也,愿卒闻之。\\
岐伯曰:振埃者,阳气大逆,上满于胸中,愤尒肩息,大气逆上,喘喝坐伏,病恶埃烟,噎不得息,请言振埃,尚疾于振埃。\\
黄帝曰:善。取之何如?\\
岐伯曰:取之天容。\\
黄帝曰:其咳上气,穷诎胸痛者,取之奈何?\\
岐伯曰:取之廉泉。\\
黄帝曰:取之有数乎?\\
岐伯曰:取天容者,无过一里,取廉泉者,血变而止。\\
帝曰:善哉。\\
黄帝曰:刺节言发蒙,余不得其意。夫发蒙者,耳无所闻,目无所见。夫子乃言刺腑输,去腑病,何输使然?愿闻其故。\\
岐伯曰:妙呼哉问也!此刺之大约,针之极也,神明之类也,口说书卷,犹不能及也,请言发蒙耳,尚疾于发蒙也。\\
黄帝曰:善。愿卒闻之。\\
岐伯曰:刺此者,必于日中,刺其听宫,中其眸子,声闻于耳,此其输也。\\
黄帝曰:善。何谓声闻于耳?\\
岐伯曰:刺邪以手坚按其两鼻窍而疾偃,其声必应于针也。\\
黄帝曰:善。此所谓弗见为之,而无目视,见而取之,神明相得者也。\\
黄帝曰:刺节言去爪,夫子乃言刺关节之支络,愿卒闻之。\\
岐伯曰:腰脊者,身之大关节也。肢胫者,人之所以趋翔也。茎垂者,身中之机,阴精之候,津液之道也。故饮食不节,喜怒不时,津液内溢,乃下留于睾,水道不通,日大不休,俯仰不便,趋翔不能,此病荥然有水,不上不下,铍石所取,形不可匿,裳不得蔽,故命曰去爪。\\
帝曰:善。\\
黄帝曰:刺节言彻衣,夫子乃言尽刺诸阳之奇输,未有常处也,愿卒闻之。\\
岐伯曰:是阳气有余而阴气不足。阴气不足则内热,阳气有余则外热,两热相搏,热于怀炭,外畏绵帛,衣不可近身,又不可近席。腠理闭塞,则汗不出,舌焦唇槁,腊干嗌燥,饮食不让美恶。\\
黄帝曰:善。取之奈何?\\
岐伯曰:取之于其天府、大杼三痏,又刺中膂,以去其热,补足手太阴以去其汗,热去汗稀,疾于彻衣。\\
黄帝曰:善。\\
黄帝曰:刺节言解惑,夫子乃言尽知调阴阳,补泻有余不足,相倾移也,惑何以解之?\\
岐伯曰:大风在身,血脉偏虚,虚者不足,实者有余,轻重不得,倾侧宛伏,不知东西,不知南北,乍上乍下,乍反乍复,颠倒无常,甚于迷惑。\\
黄帝曰:善。取之奈何?\\
岐伯曰:泻其有余,补其不足,阴阳平复,用针若此,疾于解惑。\\
黄帝曰:善。请藏之灵兰之室,不敢妄出也。\\
黄帝曰:余闻刺有五邪,何谓五邪?\\
岐伯曰:病有持痈者,有容大者,有狭小者,有热者,有寒者,是谓五邪。\\
黄帝曰:刺五邪,奈何?\\
岐伯曰:凡刺五邪之方,不过五章。瘅热消灭;肿聚散亡;寒痹益温;小者益阳,大者必去。请道其方。\\
凡刺痈邪无迎陇,易俗移性不得脓。诡道更行去其乡,不安处所乃散亡。诸阴阳过痈者,取之其输泻之。\\
凡刺大邪日以小,泄其有余乃益虚。剽其通,针去其邪肌肉亲,视之毋有反其真。刺诸阳分肉间。\\
凡刺小邪日以大,补其不足乃无害。视其所在迎之界,远近尽至,其不得外,侵而行之乃自费。刺分肉间。\\
凡刺热邪越而沧,出游不归乃无病。为开道乎辟门户,使邪得出病乃已。\\
凡刺寒邪日以温,徐往疾去致其神。门户已闭气不分,虚实得调真气存。\\
黄帝曰:官针奈何?\\
岐伯曰:刺痈者用铍针,刺大者用锋针,刺小者用员利针,刺热者用镵针,刺寒者用毫针也。\\
请言解论。与天地相应,与四时相副,人参天地,故可为解。下有渐洳,上生苇蒲,此所以知形气之多少也。阴阳者,寒暑也。热则滋而在上,根荄少汁。人气在外,皮肤缓,腠理开,血气减,汗大泄,肉淖泽。寒则地冻水冰,人气在中,皮肤致,腠理闭,汗不出,血气强,肉坚涩。当是之时,善行水者,不能往冰;善穿地者,不能凿冻;善用针者,亦不能取四厥,血脉凝结,坚搏不往来者,亦未可即柔。故行水者,必待天温冰释冻解而水可行,地可穿也。人脉犹是也。治厥者,必先熨调和其经,掌与腋、肘与脚、项与脊以调之,火气已通,血脉乃行,然后视其病,脉淖泽者,刺而平之;坚紧者,破而散之,气下乃止,此所以解结者也。\\
用针之类,在于调气。气积于胃,以通营卫,各行其道。宗气留于海,其下者注于气街,其上者走于息道。故厥在于足,宗气不下,脉中之血,凝而留止,弗之火调,弗能取之。\\
用针者,必先察其经络之实虚,切而循之,按而弹之,视其应动者,乃后取之而下之。六经调者,谓之不病,虽病,谓之自已也。一经上实下虚而不通者,此必有横络盛加于大经,令之不通,视而泻之。此所谓解结也。\\
上寒下热,先刺其项太阳,久留之,已刺则熨项与肩胛,令热下合乃止。此所谓推而上之者也。\\
上热下寒,视其虚脉而陷之于经络者取之,气下乃止。此所谓引而下之者也。\\
大热遍身,狂而妄见、妄闻、妄言,视足阳明及大络取之,虚者补之,血而实者泻之。因其偃卧,居其头前,以两手四指挟按颈动脉,久持之,卷而切推,下至缺盆中,而复止如前,热去乃止。此所谓推而散之者也。\\
黄帝曰:有一脉生数十病者,或痛、或痈、或热、或寒,或痒、或痹、或不仁,变化无穷,其故何也?\\
岐伯曰:此皆邪气之所生也。\\
黄帝曰:余闻气者,有真气,有正气,有邪气,何谓真气?\\
岐伯曰:真气者,所受于天,与谷气并而充身也。正气者,正风也。从一方来,非实风,又非虚风也。邪气者,虚风之贼伤人也,其中人也深,不能自去。正风者,其中人也浅,合而自去,其气来柔弱,不能胜真气,故自去。\\
虚邪之中人也,洒淅动形,起毫毛而发腠理。其入深,内搏于骨,则为骨痹。搏于筋,则为筋挛。搏于脉中,则为血闭不通,则为痈。搏于肉,与卫气相搏。阳胜者则为热;阴胜者,则为寒。寒则真气去,去则虚,虚则寒。搏于皮肤之间,其气外发,腠理开,毫毛摇,气往来行,则为痒。留而不去,则痹。卫气不行,则为不仁。\\
虚邪偏客于身半,其入深,内居荣卫,荣卫稍衰,则真气去,邪气独留,发为偏枯。其邪气浅者,脉偏痛。\\
虚邪之入于身也深,寒与热相搏,久留而内著,寒胜其热,则骨疼肉枯;热胜其寒,则烂肉腐肌为脓,内伤骨,内伤骨为骨蚀。有所结,筋屈不得伸,邪气居其间而不反,发为筋溜。有所结,气归之,卫气留之,不得反,津液久留,合而为肠溜,久者数岁乃成,以手按之,柔。已有所结,气归之,津液留之,邪气中之,凝结日以益甚,连以聚居,为昔瘤,以手按之,坚。有所结,深中骨,气因于骨,骨与气并,日以益大,则为骨疽。有所结,中于肉,宗气归之,邪留而不去,有热则化而为脓,无热则为肉疽。凡此数气者,其发无常处,而有常名也。\\
黄帝问岐伯说:我听说刺法有五节之说,具体是怎样的呢?\\
岐伯说:刺法确实有五节:一是振埃,二是发蒙,三是去爪,四是彻衣,五是解惑。\\
黄帝说:夫子您说的五节,我还不知道它的意义。\\
岐伯说:振埃的刺法,是刺外经,治疗阳病。发蒙的刺法,是针六腑的输穴,治疗腑病。去爪的刺法,是刺关节的支络。彻衣的刺法,是遍刺六腑的别络。解惑的刺法,是完全知道阴阳的变化,据之以补不足,泻有余,相互之间反复发生变化。\\
黄帝说:刺节中的振埃,夫子说是刺外经,治阳病,我不理解其中的道理,希望详尽地听听。\\
岐伯说,振埃的针法,是治疗阳气逆上,充满胸中,胸部胀满,呼吸抬肩,或胸中大气上逆而致喝喝气喘,坐伏不安,害怕尘埃和烟熏,咽部噎塞,呼吸不畅,请让我解释振埃的命名,是比喻针刺治疗这类病,疗效比振落尘埃还要快。\\
黄帝说:很好。取什么腧穴呢?\\
岐伯说:取天容穴。\\
黄帝问:如果咳嗽气逆,气机不伸,而胸痛的,取什么穴呢?\\
岐伯说:取廉泉穴。\\
黄帝问:取穴时针刺深浅有规律吗?\\
岐伯说:取天容穴时,下针不要超过一寸,取廉泉穴时,血络通了就止针。\\
黄帝说:很好。\\
黄帝说:刺节中所说的发蒙针法,我还不理解其中的意义。本来发蒙的针法,是治疗两耳无闻、两眼不见之证的,夫子却说针刺腑输,除去腑病,哪个输穴能有这种作用呢?我希望听听其中的道理。\\
岐伯说:问得太好了!这是针刺的关键要领,也是针法奥妙的极致,属于神明之类,口中说的和书上记载的,还不能完全表达出来。所谓的发蒙,是说其奏效比开发蒙瞆还要快。\\
黄帝说:好。希望详细听听。\\
岐伯说:针刺这种病,必须在中午时分,刺听宫穴,针刺感应达到瞳子,并使耳中听到声响,这就是针刺的输穴。\\
黄帝说:好。什么叫“声闻于耳”呢?\\
岐伯说:就是在针刺听宫时,让病人用手紧捏住两鼻孔,赶快仰卧,必然有声音应针而响。\\
黄帝说:好。这真是所谓的用眼睛看不见内里怎样的作为,可见的是医生的取穴针刺,却有得心应手,出神入化的神奇疗效。\\
黄帝说:刺节所说的去爪针法,夫子说是刺关节的支络,希望详尽听听。\\
岐伯说:腰脊是身体最大的关节。肢和胫是行走的器官。阴茎、睾丸为身中机,阴精由此排泄,小便由此排出。如果饮食不节制,喜怒过度,使津液内溢,下行聚集于睾丸,水道不通,阴囊日渐肿大,俯仰困难,行走受限。这种病是由于水液蓄积,上下水道不通,取用铍针放水,阴囊水肿之形不能藏匿,下裳不能遮蔽,治疗这种病,就好像修剪掉多余的指甲一样,所以叫去爪。\\
黄帝说:好!\\
黄帝说:刺节中所说的彻衣针法,夫子却说遍刺诸阳经之奇穴,没有固定部位,希望详尽听听。\\
岐伯说:这种刺法是治疗阳气有余而阴气不足的病。阴气不足则生内热,阳气有余则生外热,两热相搏结,则热甚于怀抱炭火,外怕靠近绵帛之物,衣服也不能贴近身体,身热不敢靠近坐席。腠理闭塞,不得出汗,舌焦、唇槁、肌肉枯瘦、咽喉干燥,饮食无味,不分好坏。\\
黄帝说:好。怎样治疗呢?\\
岐伯说:治疗这种病取天府、大杼穴各刺三次,再刺中膂腧,以泻热,再补手、足太阴经,使发汗,待热退汗少时,病就痊愈了,取效比脱掉衣服还要快。\\
黄帝说:好。\\
黄帝问:刺节中所说的解惑针法,夫子却说要完全懂得调整阴阳和运用补泻,使虚实相互改变,怎样解除迷惑呢?\\
岐伯说:大风侵入人体,血气必有偏虚之处,虚是正气不足,实是邪气有余,这样身体左右轻重不相称,身体不能倾斜反侧,也不能宛转俯伏,甚至不能辨别东西南北,症状忽上忽下,反复多变,颠倒无常,比一般神志迷惑的病要严重。\\
黄帝说:好。怎样治疗呢?\\
岐伯说:泻除有余的邪气,补益不足的正气,使阴阳平衡。像这样用针,取效就比突然解除迷惑、豁然开朗还快。\\
黄帝说:很好。请让我把这些针刺理论知识,储藏在灵兰之室,不敢随便拿出示人。\\
黄帝问:我听说有刺五邪的方法,什么叫五邪呢?\\
岐伯说:有痈邪,有实邪,有虚邪,有热邪,有寒邪,这叫五邪。\\
黄帝说:怎样刺治五邪之病呢?\\
岐伯说:大凡刺治五邪的方法,不过五条。对瘅热的病应消灭热邪;肿聚不散的应当使其消散;寒邪痹病应益气温阳;虚弱者补益阳气,邪实有余的必须驱除邪气。请让我说明具体的针刺方法。\\
大凡刺痈邪,不可迎着痈邪的锐势妄用针刺或排脓,应和缓地像移风易俗,移情易性一样,耐心地进行调治,这样痈疽就不会化脓而治愈。若已化脓就采用其他方法治疗,离开脓之所在,使脓毒不能留聚,脓液排出,邪毒就消亡了。所以不论是阳经或阴经生痈者,都要循本经取穴以泻之。\\
大凡针刺大邪之病,是使邪气减小。用泄法,泄去有余的邪气,则邪气日渐虚衰。用砭刺使正气运行的道路开通。用针刺祛除其邪气,则肌肉自然亲附致密。观察邪气已经祛除,真气恢复乃停针。盛大的实邪,多在三阳,故宜针刺诸阳经分肉间的穴位。\\
大凡针刺虚邪致病的方法,是使正气日渐盛大,补其正气的不足,邪气就不能为害了。审察邪气所在,迎而夺之,使远近的正气尽至而不外泄,若外邪入侵在体内泛滥流行,正气就会损耗。刺虚邪之法,当取分肉间的穴位。\\
大凡针刺热邪,将邪气发散于外,使其外出不再回返,身体不发热就没病了。在针刺时应当为邪气疏通道路,开辟门户,使邪热得以外泄,病就痊愈了。\\
大凡针刺寒邪,应逐日温养正气,用徐进疾出的补法,使神气恢复正常。出针后揉按闭合针孔,正气不散,虚实调和,真气就密固内存了。\\
黄帝说:用什么针刺治五邪呢?\\
岐伯说:刺痈疡用铍针,刺实邪用锋针,刺虚邪用员利针,刺热邪用镵针,刺寒邪用毫针。\\
请让我谈谈解结的理论。人与天地相适应,与四季相符合,因为人与天地相参,所以才可以谈到解结。比如在下面有水湿的地方,在上面才能生长蒲苇,根据这个道理,从人体外形的变化,就可以测知内在气血的多少了。阴阳的变化,可以用寒暑的变化来比喻。炎热时,地面的水分被蒸发成云,草木根荄就缺少水分。人体受热气熏蒸,阳气浮越在外,所以皮肤弛缓,腠理开张,血气衰减,汗液大泄,肌肉润泽。寒冷时,土地上冻,流水结冰,人的阳气也潜藏在内,所以皮肤致密,腠理闭合,汗不出,血气强,肌肉坚涩。这个时候,善于行水的人不能在冰上行船;善于穿地的人,也不能凿开冻土;善于用针的人,也不能治疗四肢厥逆的病证。血脉因寒凝结,坚聚不能流畅往来,是不能立即使它柔软的。所以行水的人,必须等到气候转暖,冰冻化解后才能在水上行舟;穿地的人,也必须等到大地解冻才能穿地。人体的血脉,也是如此。治疗厥逆病,必先用温熨,以调和经脉,在两掌、两腋、两肘、两脚,以及项、脊等关节交会之处,实行温熨,待温热之气通达,血脉就恢复正常运行了。然后再观察病情,如脉气滑润流畅的,用针刺使其平复;如脉象坚紧的,用破坚散结法,使厥逆之气下行而止针。这些都是用来解结的具体方法。\\
大凡用针刺治病的法则,主要在于调整经气。水谷精气先积于胃中,化生的营气和卫气,各循行于自己的道路。宗气,积聚胸中而为气海,其下行的灌注于气街穴处;其上行的走向呼吸之道。所以足部发生厥逆时,宗气就不能下行,脉中之血也凝滞留止,若不先用火灸温熨来通调气血,就不能取穴针刺。\\
用针刺治病,必须首先察看经络的虚实,用手切循经脉,按揉并弹动经脉,看到应指而动的部位,然后取穴,下针。手足六经经脉调和的,是无病的征象,就是有轻微的病,也可以不治自愈。如果一经出现上实下虚而不通的,这必定是横络的亢盛之气加于正经,使其不通,根据疾病的所在而用泻法。这也就是所说的解结的方法。\\
腰以上寒冷,腰以下发热的,当先刺项间足太阳经的穴位,长时间留针,针刺以后,还要温熨项部及肩胛部,使热气上下相合,才可止针。这就是所谓推而上之的方法。\\
腰以上发热,腰以下发冷,察看哪条虚脉陷于经络,取适当的穴位,使阳气下行后止针。这就是所谓引而下之的方法。\\
周身高热,热极发狂,且有妄见、妄闻、妄言的,察看足阳明经及其大的络脉,取穴刺治,虚的用补法,有血淤而属实的用泻法。让病人仰卧,医者在病人头前,用两手拇指、食指,挟按患者颈部的动脉,要长时间挟持,并用卷而按切的手法,向下推按至缺盆,再重复上述动作,身热退去才休止。这就是所谓推而散之的方法。\\
黄帝问:有在一脉之中发生几十种病证的,或疼痛,或成痈,或发热,或恶寒,或作痒,或为痹痛,或麻木不仁,变化无穷,是什么原因呢?\\
岐伯说:这都是邪气所造成的。\\
黄帝说:我听说气有真气,有正气,有邪气的不同,什么叫真气呢?\\
岐伯说:所谓真气,由先天的元气与后天的谷气合并而成,并充养全身。所谓正气,即正风,它是从与四季相符合的方位而来,不是实风,也不是虚风。所谓邪气,就是能够戕贼伤害人体的虚邪贼风,它侵入人体,部位比较深,不能自行消散。正风,侵入人体,部位表浅,与体内真气接触后,能自行散去,因为正风来势柔弱,不能战胜体内真气,所以能自行离去。\\
虚邪贼风侵入人体,扰动形体,出现寒栗怕冷,毫毛竖起,腠理开泄。若邪气深入而搏结于骨的,就发为骨痹。搏结于筋的,就出现筋挛。搏结于脉中,就出现血脉闭塞不通或成为痈。搏结于肌肉的,与卫气相搏。如果阳邪偏胜,就为热证;阴邪偏胜,就为寒证。寒邪偏盛,则真气离去,真气离去则虚衰,虚衰则畏寒。邪气搏结于皮肤之间,会向外发泄,腠理开疏,毫毛动摇脱落,致邪气在皮腠间轻微地往来流行,所以皮肤发痒。若邪气留滞不去,则痹阻不通。如果卫气不能畅行,则为麻木不仁。\\
虚邪贼风侵犯半侧身体,入犯深部,在体内居留于营卫之中,使营卫的功能渐渐减弱,所以真气离去,邪气单独存留,就发生半身不遂。如果邪气留在表浅部位,会发生半身经脉偏痛。\\
虚邪侵入人体,部位较深,寒与热相互搏结,久留不去而停着于内,如果寒胜过热,会引起骨节疼痛,肌肉枯萎;如果热胜过寒,会发生肌肉腐烂化脓,进一步向内伤到骨,伤骨便成为“骨蚀”。邪气结聚,中于筋,筋屈而不伸,邪气久留其间而不消,可发为筋瘤。邪气结聚于内,气郁于内,因而卫气也停留而不能正常循行,以致津液久留肠胃与邪气相合成为肠瘤,发展缓慢的要数年才能形成,用手按摸感到柔软。邪气结聚而气郁于内,津液停留不行,又感受邪气,凝结不散,日益加重,接连积聚起来,便成为昔瘤,用手按摸感到坚硬。邪气结聚,伤及深层的骨部,骨与邪气并合,一天天地增大,则形成骨疽。邪气结聚,伤及肌肉而宗气归于内,邪气留着不去,如有内热可化为脓,如无热可成为肉疽。上述这几种邪气,发病没有固定的部位,但都有一定的名称。\\
卫气行第七十六\\
黄帝问于岐伯曰:愿闻卫气之行,出入之合,何如?\\
岐伯曰:岁有十二月,日有十二辰,子午为经,卯酉为纬。天周二十八宿,而一面七星,四七二十八星。房昴为纬,虚张为经。是故房至毕为阳,昴至心为阴。阳主昼,阴主夜,故卫气之行,一日一夜五十周于身,昼日行于阳二十五周,夜行于阴二十五周,周于五脏。\\
是故平旦阴气尽,阳气出于目,目张,则气上行于头,循项下足太阳,循背下至小指之端。其散者,别于目锐眦,下手太阳,下至手小指外侧。其散者,别于目锐眦,下足少阳,注小指次指之间。以上循手少阳之分,下至小指之间。别者以上至耳前,合于颔脉,注足阳明,以下行至跗上,入五指之间。其散者,从耳下下手阳明,入大指之间,入掌中。其至于足也,入足心,出内踝下,行阴分,复合于目,故为一周。\\
是故日行一舍,人气行于身一周与十分身之八;日行二舍,人气行于身三周与十分身之六;日行三舍,人气行于身五周与十分身之四;日行四舍,人气行于身七周与十分身之二;日行五舍,人气行于身九周;日行六舍,人气行于身十周与十分身之八;日行七舍,人气行于身十二周与十分身之六;日行十四舍,人气行二十五周于身有奇分与十分身之二,阳尽而阴受气矣。其始入于阴,常从足少阴注于肾,肾注于心,心注于肺,肺注于肝,肝注于脾,脾复注于肾为一周。是故夜行一舍,人气行于阴脏一周与十分脏之八,亦如阳行之二十五周,而复合于目。阴阳一日一夜,合有奇分十分身之二,与十分藏之二,是故人之所以卧起之时有早晏者,奇分不尽故也。\\
黄帝曰:卫气之在于身也,上下往来不以期,候气而刺之,奈何?\\
伯高曰:分有多少,至有长短,春秋冬夏,各有分理,然后常以平旦为纪,以夜尽为始。是故一日一夜,水下百刻,二十五刻者,半日之度也,常如是毋已,日入而止,随日之长短,各以为纪而刺之。谨候其时,病可与期;失时反候者,百病不治。故曰:刺实者,刺其来也;刺虚者,刺其去也。此言气存亡之时,以候虚实而刺之。是故谨候其气之所在而刺之,是谓逢时。病在于三阳,必候其气在于阳而刺之;病在于三阴,必候其气在于阴分而刺之。\\
水下一刻,人气在太阳;水下二刻,人气在少阳;水下三刻,人气在阳明;水下四刻,人气在阴分。水下五刻,人气在太阳;水下六刻,人气在少阳;水下七刻,人气在阳明;水下八刻,人气在阴分。水下九刻,人气在太阳;水下十刻,人气在少阳;水下十一刻,人气在阳明;水下十二刻,人气在阴分。水下十三刻,人气在太阳;水下十四刻,人气在少阳,水下十五刻,人气在阳明;水下十六刻,人气在阴分。水下十七刻,人气在太阳;水下十八刻,人气在少阳;水下十九刻,人气在阳明;水下二十刻,人气在阴分。水下二十一刻,人气在太阳;水下二十二刻,人气在少阳;水下二十三刻,人气在阳明;水下二十四刻,人气在阴分。水下二十五刻,人气在太阳,此半日之度也。从房至毕一十四舍,水下五十刻,日行半度,回行一舍,水下三刻与七分刻之四。《大要》曰:常以日之加于宿上也,人气在太阳。是故日行一舍,人气行三阳行与阴分,常如是无已,与天地同纪,纷纷氿氿,终而复始,一日一夜,水下百刻而尽矣。\\
黄帝问岐伯说:希望听听卫气是怎样运行,出入于阴阳表里,又是怎样会合的?\\
岐伯说:一年有十二个月,一天有十二个时辰,子午位居南北,为经;卯酉位居东西,为纬。天体循行一周有二十八个星宿,每一方各有七个星宿,东南西北四方共有二十八个星宿。房宿居东方,昴宿居西方,所以房昴为纬;虚宿居北方,张宿居南方,所以虚张为经。从东方的房宿到西方的毕宿属阳,从西方的昴宿到东方的心宿属阴。阳主白天,阴主夜间,卫气的运行,在一日一夜之中,要循行于全身五十周次,白天行于阳二十五周次,夜间行于阴二十五周次,环周行于五脏之间。\\
卫气昼行于阳,夜行于阴,所以到黎明时分,阴气已尽,阳气浮出于目,眼睛张开,卫气上行于头,沿项后下行足太阳经,再沿着背部向下到足小趾外侧端。其散行的,从目锐眦别出,向下沿手太阳经,下行至手小指外侧端。另一条散行的,也从目锐眦别出,沿着足少阳经下行,注入足小趾、第四趾之间。再向上循手少阳经之分,下行到小指、无名指之间。从手少阳别行的上行至耳前,合于颔部经脉,注入足阳明经,向下行至足背,入五趾间。又一条散行的,从耳下向下,沿手阳明经,入手大指次指端,再络入掌中。卫气抵达足部,进入足心,出内踝,行于阴分,向上行复合于目内眦,这就是卫气运行一周的顺序。\\
因此,在白昼当太阳运行一宿时,卫气行身一又十分之八周;运行二宿时,卫气行身三又十分之六周;运行三宿时,卫气行身五又十分之四周;运行四宿时,卫气行身七又十分之二周;运行五宿时,卫气行身九周;运行六宿时,卫气行身十又十分之八周;运行七宿时,卫气行身十二又十分之六周;运行十四宿时,卫气行身二十五又十分之二周,这时卫气在白昼行于阳的过程就结束了,而阴分开始承受卫气。卫气开始进入阴分时,通常是由足少阴肾经传注于肾脏,由肾脏注入心脏,由心脏注入肺脏,由肺脏注入肝脏,由肝脏注入脾脏,由脾脏再传到肾脏,为一周。因此,夜间运行一宿的时间,卫气行于阴分也是一又十分之八周,也和行于阳分的二十五周一样,在眼部会合。阴分阳分一日一夜本应运行五十周,可是按每宿卫气运行一又十分之八周来计算,行于阳分的多出十分之二周,行于阴分的也多出十分之二周,所以人睡和醒的时间,有或早或晚的不同,就是这些余数造成的。\\
黄帝问:卫气在人体内的循行,或上或下或往或来,时间不固定,怎样候气而针刺呢?\\
伯高说:昼夜阴阳的多少不同,就有天长天短差异,春夏秋冬四季,各有不同的节气,因而昼夜长短都有一定的规律,一般根据太阳初出的时候为准,以夜尽为卫气行于阳分的开始。因此,一昼夜之中,计时的水漏下百刻,二十五刻正好是半天的度数,总是这样环周不已,到日入时白昼结束,随着日出日入的长短,分别作为标准进行针刺。针刺时,要候其气至再下针,疾病才可痊愈;若失去时机,违反了候气的原则,则任何疾病都不能治愈。所以说:针刺实证,是迎其气之来而刺;针刺虚证,是随其气之去而刺。这是说根据邪气的盛衰留去,诊候虚实而针刺。所以,谨慎地候察邪气的所在而针刺,就叫做逢时。病在三阳经,必候气在阳分时针刺;病在三阴经,必候气在阴分时针刺。\\
漏水下注一刻,卫气行于手足太阳经;漏水下注二刻,卫气行于手足少阳经;漏水下注三刻,卫气行于手足阳明经;漏水下注四刻,卫气行于足少阴肾经。漏水下注五刻,卫气行于手足太阳经;漏水下注六刻,卫气行于手足少阳经;漏水下注七刻,卫气行于手足阳明经;漏水下注八刻,卫气行于足少阴肾经。漏水下注九刻,卫气行于手足太阳经;漏水下注十刻,卫气行于手足少阳经;漏水下注十一刻,卫气行于手足阳明经;漏水下注十二刻,卫气行于足少阴肾经。漏水下注十三刻,卫气行于手足太阳经;漏水下注十四刻,卫气行于手足少阳经;漏水下注十五刻,卫气行于手足阳明经;漏水下注十六刻,卫气行于足少阴肾经。漏水下注十七刻,卫气行于手足太阳经;漏水下注十八刻,卫气行于手足少阳经;漏水下注十九刻,卫气行于手足阳明经;漏水下注二十刻,卫气行于足少阴肾经。漏水下注二十一刻,卫气行于手足太阳经;漏水下注二十二刻,卫气行于手足少阳经;漏水下注二十三刻,卫气行于手足阳明经;漏水下注二十四刻,卫气行于足少阴肾经。漏水下注二十五刻,卫气行于手足太阳经,这是半日中卫气运行的度数。从房宿到毕宿运转一十四舍,经过整个白昼,漏水下注五十刻,日行半个周天,每当日行周列一宿,需时漏水下注三刻又七分之四刻。《大要》说:通常是以日行环周二十八宿的每一宿之时,卫气也恰恰运行在手足太阳经。所以日行一宿的过程,卫气也恰恰运行过三阳经与阴分,经常这样周行不已,与天地的变化规律相同,卫气在体内的运行,虽然纷繁,却有条不紊,终而复始,一日一夜,漏水下注百刻,卫气在体内完成了五十周的运行。\\
九宫八风第七十七\\
太一常以冬至之日,居叶蛰之宫四十六日,明日居天留四十六日,明日居仓门四十六日,明日居阴洛四十五日,明日居天宫四十六日,明日居玄委四十六日,明日居仓果四十六日,明日居新洛四十五日,明日复居叶蛰之宫,曰冬至矣。\\
太一日游,以冬至之日,居叶蛰之宫,数所在,日从一处,至九日,复反于一,常如是无已,终而复始。\\
太一移日,天必应之以风雨。以其日风雨则吉,岁美民安少病矣。先之则多雨,后之则多旱。\\
太一在冬至之日有变,占在君;太一在春分之日有变,占在相;太一在中宫之日有变,占在吏;太一在秋分之日有变,占在将;太一在夏至之日有变,占在百姓。所谓有变者,太一居五宫之日,病风折树木,扬沙石。各以其所主占贵贱。\\
因视风所从来而占之。风从其所居之乡来为实风,主生长,养万物;从其冲后来为虚风,伤人者也,主杀主害者。谨候虚风而避之,故圣人曰:避虚邪之道,如避矢石然,邪弗能害,此之谓也。\\
是故太一徙,立于中宫,乃朝八风,以占吉凶也。\\
风从南方来,名曰大弱风。其伤人也,内舍于心,外在于脉,其气主为热。\\
风从西南方来,名曰谋风。其伤人也,内舍于脾,外在于肌,其气主为弱。\\
风从西方来,名曰刚风。其伤人也,内舍于肺,外在于皮肤,其气主为燥。\\
风从西北方来,名曰折风。其伤人也,内舍于小肠,外在于手太阳脉,脉绝则溢,脉闭则结不通,善暴死。\\
风从北方来,名曰大刚风。其伤人也,内舍于肾,外在于骨与肩背之膂筋,其气主为寒也。\\
风从东北方来,名曰凶风。其伤人也,内舍于大肠,外在于两胁腋下及肢节。\\
风从东方来,名曰婴儿风。其伤人也,内舍于肝,外在于筋纽,其气主为湿。\\
风从东南方来,名曰弱风。其伤人也,内舍于胃,外在肌肉,其气主体重。\\
此八风皆从其虚之乡来,乃能病人。三虚相搏,则为暴病卒死。两实一虚,病则为淋露寒热。犯其雨湿之地,则为痿。故圣人避风,如避矢石焉。其有三虚而偏中于邪风,则为击仆偏枯矣。\\
北极星太一常从冬至日开始,居于正北方叶蛰宫四十六天;期满的下一天,就移居东北方天留宫四十六天;期满的下一天,就移居正东方仓门宫四十六天;期满的下一天,就移居东南方阴洛宫四十五天;期满的下一天,就移居正南方上天宫四十六天;期满的下一天,就移居西南方玄委宫四十六天;期满的下一天,就移居正西方仓果宫四十六天;期满的下一天,就移居西北方新洛宫四十五天;期满后的下一天,重居叶蛰宫,就又到了冬至日。\\
太一游宫的日子,开始于冬至日,居于正北叶蛰宫,以此为起点,来推算其所在之处,到第九天,重又回到坎位,经常这样循环不休,终而复始地运转着。\\
太一每逢交节的日子,必有风雨出现。如果当天风调雨顺,则年景好,民众安居,很少生病。假若交节之前有风雨,这一年就会多雨;交节之后出现风雨,这一年就会多旱。\\
太一在冬至那一天,气候剧变,预测应在君;太一在春分那一天,气候剧变,预测应在相;太一在中宫那一天,气候剧变,预测应在吏;太一在秋分那一天,气候剧变,预测应在将;太一在夏至那一天,气候剧变,预测应在百姓。所谓气候剧变,是指太一分别居于五宫的那一天,气候突变,折断树木,飞沙走石。分别从太一所主的方位来占验病者的贵贱。\\
观察风所来的方向,作为预测气象的依据。风来自其所当令的方位与季节相适应的,是实风,主生长,养育万物;若风来自于其所当令相对的方位与季节相抵触的,是虚风,能够伤人致病,主残害万物。谨慎候察虚风的到来而躲避,所以圣人说:防避虚邪贼风的方法,就像躲避箭矢飞石,外邪就不能侵害,说的就是这个道理。\\
所以北极星太一迁移,位居中宫,才能朝向八风,来推测气象的吉凶。\\
从南方来的风,名叫大弱风。它伤害到人体,内可侵入于心,外在于血脉,其气主热性病。\\
从西南方来的风,名叫谋风。它伤害到人体,内可侵入于脾,外在于肌肉,其气主虚弱病。\\
从西方来的风,名叫刚风。它伤害到人体,内可侵入于肺,外则留于皮肤之间,其气主燥病。\\
从西北方来的风,名叫折风。它伤害到人体,内可侵入小肠,外在于手太阳经脉,若其脉气竭绝,则邪气充满流溢;若其脉气闭塞,则结聚不通,会突然死亡。\\
从北方来的风,名叫大刚风。它伤害到人体,内可侵入于肾,外在于骨骼和肩背的膂筋,其气主寒性病。\\
从东北方来的风,名叫凶风。它伤害到人体,内可侵入大肠,外在于两胁腋骨下及上肢关节部。\\
从东方来的风,名叫婴儿风。它伤害到人体,内可侵入于肝,外在于筋的联结之处,其气主为湿病。\\
从东南方来的风,名叫弱风。它伤害到人体,内可侵入于胃,外在肌肉,其气主身体沉重的病证。\\
以上这八风,都是从虚乡来的,才能使人生病。人与自然界是息息相通的,如果是虚人,又遇到年、月、时之三虚,就会暴发疾病,突然死亡。如果三虚之中为两实一虚,则能发生疲困,寒热相杂之证。或在雨湿之地,感受湿气,会发生痿病。所以圣人,避免虚邪贼风如同躲避箭矢飞石。如果逢到三虚,就可能偏中邪风,突然昏仆倒地,以致引起半身不遂一类的病证。\\
卷十二\\
九针论第七十八\\
黄帝曰:余闻九针于夫子,众多博大矣,余犹不能寤,敢问九针焉生?何因而有名?\\
岐伯曰:九针者,天地之大数也,始于一而终于九。故曰:一以法天,二以法地,三以法人,四以法时,五以法音,六以法律,七以法星,八以法风,九以法野。\\
黄帝曰:以针应九之数,奈何?\\
岐伯曰:夫圣人之起天地之数也,一而九之,故以立九野,九而九之,九九八十一,以起黄钟数焉,以针应数也。\\
一者,天也。天者,阳也。五脏之应天者肺,肺者,五脏六腑之盖也。皮者,肺之合也,人之阳也。故为之治针,必以大其头而锐其末,令无得深入而阳气出。\\
二者,地也。人之所以应土者,肉也。故为之治针,必筩其身而员其末,令无得伤肉分,伤则气竭。\\
三者,人也。人之所以生成者,血脉也。故为之治针,必大其身而员其末,令可以按脉勿陷,以致其气,令邪气独出。\\
四者,时也。时者,四时八风之客于经络之中,为痼病者也。故为之治针,必筩其身而锋其末,令可以泻热出血,而痼病竭。\\
五者,音也。音者,冬夏之分,分于子午,阴与阳别,寒与热争,两气相搏,合为痈脓者也。故为之治针,必令其末如剑锋,可以取大脓。\\
六者,律也。律者,调阴阳四时而合十二经脉,虚邪客于经络而为暴痹者也。故为之治针,必令尖如氂,且员且锐,中身微大,以取暴气。\\
七者,星也。星者,人之七窍,邪之所客于经,而为痛痹,舍于经络者也。故为之治针,令尖如蚊虻喙,静以徐往,微以久留,正气因之,真邪俱往,出针而养者也。\\
八者,风也。风者,人之股肱八节也,八正之虚风,八风伤人,内舍于骨解腰脊节腠理之间,为深痹也。故为之治针,必长其身,锋其末,可以取深邪远痹。\\
九者,野也。野者,人之节解皮肤之间也。淫邪流溢于身,如风水之状,而溜不能过于机关大节者也。故为之治针,令尖如梃,其锋微员,以取大气之不能过于关节者也。\\
黄帝曰:针之长短,有数乎?\\
岐伯曰:一曰镵针者,取法于巾针,去末半寸,卒锐之,长一寸六分,主热在头身也。二曰圆针,取法于絮针,筩其身而卵其锋,长一寸六分,主治分间气。三曰氻针,取法于黍粟之锐,长三寸半,主按脉取气,令邪出。四曰锋针,取法于絮针,筩其身,锋其末,长一寸六分,主痈热出血。五曰铍针,取法于剑锋,广二分半,长四寸,主大痈脓,两热争者也。六曰员利针,取法于氂针,微大其末,反小其身,令可深内也,长一寸六分,主取痈痹者也。七曰毫针,取法于毫毛,长一寸六分,主寒热痛痹在络者也。八曰长针,取法于綦针,长七寸,主取深邪远痹者也。九曰大针,取法于锋针,其锋微员,长四寸,主取大气不出关节者也。\\
针形毕矣,此九针大小长短法也。\\
黄帝曰:愿闻身形应九野,奈何?\\
岐伯曰:请言身形之应九野也。左足应立春,其日戊寅己丑;左胁应春分,其日乙卯;左手应立夏,其日戊辰己巳;膺喉首头应夏至,其日丙午;右手应立秋,其日戊申己未;右胁应秋分,其日辛酉;右足应立冬,其日戊戌己亥;腰尻下窍应冬至,其日壬子;六腑膈下三脏应中州,其大禁,大禁太一所在之日及诸戊己。凡此九者,善候八正所在之处,所主左右上下。身体有痈肿者,欲治之,无以其所直之日,溃治之。是谓天忌日也。\\
形乐志苦,病生于脉,治之以灸刺。形苦志乐,病生于筋,治之以熨引。形乐志乐,病生于肉,治之以针石。形苦志苦,病生于咽喝,治之以甘药。形数惊恐,筋脉不通,病生于不仁,治之以按摩醪药。是谓形。\\
五藏气:心主噫,肺主咳,肝主语,脾主吞,肾主欠。\\
六腑气:胆为怒,胃为气逆、哕,大肠小肠为泄,膀胱不约为遗溺,下焦溢为水。\\
五味:酸入肝,辛入肺,苦入心,甘入脾,咸入肾,淡入胃。是谓五味。\\
五并:精气并于肝则忧,并于心则喜,并于肺则悲,并于肾则恐,并于脾则畏。是谓五精之气,并于脏也。\\
五恶:肝恶风,心恶热,肺恶寒,肾恶燥,脾恶湿。此五脏气所恶也。\\
五液:心主汗,肝主泣,肺主涕,肾主唾,脾主涎。此五液所出也。\\
五劳:久视伤血,久卧伤气,久坐伤肉,久立伤骨,久行伤筋。此五久劳所病也。\\
五走:酸走筋,辛走气,苦走血,咸走骨,甘走肉,是谓五走也。\\
五裁:病在筋,无食酸;病在气,无食辛;病在骨,无食咸;病在血,无食苦;病在肉,无食甘。口嗜而欲食之,不可多也,必自裁也。命曰五裁。\\
五发:阴病发于骨;阳病发于血;以味发于气;阳病发于冬,阴病发于夏。命曰五发。\\
五邪:邪入于阳,则为狂;邪入于阴,则为血痹;邪入于阳,搏则为癫疾;邪入于阴,搏则为瘖;阳入于阴,病静,阴出于阳,病喜怒。\\
五藏:心藏神,肺藏魄,肝藏魂,脾藏意,肾藏精志也。\\
五主:心主脉,肺主皮,肝主筋,脾主肌,肾主骨。\\
阳明多血多气,太阳多血少气,少阳多气少血,太阴多血少气,厥阴多血少气,少阴多气少血。故曰刺阳明出血气,刺太阳出血恶气,刺少阳出气恶血,刺太阴出血恶气,刺厥阴出血恶气,刺少阴出气恶血也。\\
足阳明太阴为表里,少阳厥阴为表里,太阳少阴为表里。是谓足之阴阳也。手阳明太阴为表里,少阳心主为表里,太阳少阴为表里。是谓手之阴阳也。\\
黄帝问:我听你讲解九针,内容真是丰富多彩,但有些我还是不明白,请问九针是怎样创造的?根据什么而定名的?\\
岐伯说:九针的创造是取法于天地大数而来的,天地的数理,从一开始,到九而终止。所以说第一种针具取法于天,第二种针具取法于地,第三种针具取法于人,第四种针具取法于四时,第五种针具取法于五音,第六种针具取法于六律,第七种针具取法于七星,第八种针具取法于八风,第九种针具取法于九野。\\
黄帝问:以针和九数相应,情况怎样呢?\\
岐伯说:圣人确立了天地的数理,是从一到九,因此把大地划分为九个分野,若九九相乘,等于八十一,从而建立黄钟之数,九针正与此数相应。\\
一之数,取象于天。天属阳。在五脏中,外与天气相应的是肺,肺的位置最高,为五脏六腑的“华盖”。皮毛是肺的外合,皮毛在体表,属人体之阳分。取象于此而制成镵针,针的式样,必须针头大,针尖锐利,使之不能深刺,而只能浅刺,以防深刺泄伤阳气。\\
二之数,取象于地。地即土,人体与土相应的是肉。取象于此而制成员针,针的式样是针身圆直如竹管状,针尖呈卵圆形,使之刺治时不得损伤分肉,若过伤肌肉,易使脾气衰竭。\\
三之数,取象于人。人的生命形成,依赖于血脉输送营养。取象于此而制成奼针,针的式样是针身大,针尖圆,微尖而钝,可以按摩穴位,疏通血脉,但不能刺入过深,而引邪内陷,以引导正气,则邪气自然外出。\\
四之数,取象于四时。所谓“时”,就是四时八方的风邪,侵入经络中,痹阻血脉,气滞血淤,而渐成顽固性的病证。取象于此而制成锋针,针的式样是针身长直、针尖锋利,用来刺络放血,泻其淤热,而根治顽疾。\\
五之数,取象于五音。音为五数,位于一、九两数的中间。一数,代表冬至一阳初生之时,月建在子;九数,代表夏至一阴初生之时,月建在午。而五数正当一到九数的中央,如暑往寒来,阴阳消长的复迁,由此可分。比象人体的寒热不调,两气搏结,形成痈肿化脓之证。取象于此而制成铍针,针的式样是针头锋利如剑,可以刺破痈疽,排出脓血。\\
六之数,取象于六律。六律调节声音,分为阴阳,应于四时十二辰,比象于人体十二经脉,虚邪贼风,侵入经络,阴阳失调,气血雍闭,会暴发痹证。取象于此而制成圆利针,针的式样是针尖状如氂牛尾,圆而锐利,针身略粗大,适用于刺治急性病。\\
七之数,取象于七星。七星,比象于人体的七窍,邪从孔穴侵入经络,久留不去,发生痛痹,是因为客舍于经络的缘故。取象于此而制成毫针,针的式样是针尖微细,好像蚊虻嘴那样,刺治时,要静候其气,慢慢进针,轻微地提插,长时间留针,这样正气得到充实,只要邪气消散,真气也就随之恢复,出针以后,还要继续疗养。\\
八之数,取象于八风。八风与人的股肱八节相应,如四时气候反常,八方虚风就会伤人,侵入骨缝腰及关节和腠理之间,形成邪深在里的痹证。取象于此而制成长针,针的式样是针身长而针尖锋利,可以治疗邪深病久的痹证。\\
九之数,取象于九野。九野与人的周身关节骨缝和皮肤之间相应。邪气流溢漫延周身,状似风水,水气不能通过大关节,而形成肌肤水肿。取象于此而制成大针,针的式样是尖如破竹,针尖略圆,治疗水气停留,大气不能通过关节的疾病。\\
黄帝问:针的长短,有一定度数吗?\\
岐伯说:第一种叫镵针,模仿巾针的式样制成,针头大,在距离针尖末端约半寸许,突然尖锐,状如箭头,长度共一寸六分,适用于浅刺,主治热在头身的病证。第二种叫员针,模仿絮针的式样制成,针身圆直如竹管状,针尖呈卵圆形,长一寸六分,主治邪在分肉间的疾病。第三种叫奼针,仿照黍粟的形状,圆而微尖,长三寸半,用来按摩经穴,行气活血,驱邪外出。第四种叫锋针,模仿絮针的式样制成,针身圆直,针尖锋利,长一寸六分,主治痈疡热毒之证,以刺络放血。第五种叫铍针,模仿剑锋的式样制成,针尖锋利如剑,宽二分半,长四寸,主治寒热搏结的痈肿化脓之证。第六种叫员利针,模仿氂牛尾制成,针的形状细长如氂牛尾,针尖稍大,针身反小,用于深刺,长一寸六分,主治痈肿和痹证。第七种叫毫针,其形细如毫毛,长一寸六分,主治邪在于络的寒热和痛痹。第八种叫长针,模仿綦针的式样制成,长七寸,主治邪深病久的痹证。第九种叫大针,针的形状,是模仿锋针制作,针长略圆,长四寸,主治大气不能通利关节,积水成肿的病证。\\
以上所述,九针的式样已尽在其中了。这也是九针大小长短及其制法的根据。\\
黄帝问:我希望听听人的身形和九野相应的情况,是怎样的?\\
岐伯说:请让我讲讲身形应九野的情况。春夏属阳,阳气从左而升,自下而上,所以人的左足应于艮宫(东北方),在节气应于立春,在日辰正当戊寅、己丑;左胁应于震宫(正东方),在节气应于春分,在日辰正当乙卯;左手应于巽宫(东南方),在节气应于立夏,在日辰正当戊辰、己巳;前胸、咽喉、头面应于离宫(正南方),在节气应于夏至,在日辰正当丙午。秋冬属阴,阴气从右而降,自上而下,所以右手应于坤宫(西南方),在节气应于立秋,在日辰正当戊申、己未;右胁应于兑宫(正西方),在节气应于秋分,在日辰正当辛酉;右足应于乾宫(西北方),在节气应于立冬,在日辰正当戊戌、己亥;腰、尻、下窍应于坎宫(正北方),在节气应于冬至,在日辰正当壬子。六腑和肝、脾、肾三脏,都在膈下腹中的部位,应于中宫,它属于大禁,所谓大禁,是指正交八节(四立、二分、二至)的太一所在之日,以及各个戊、己日。掌握了人体九个部位和九个方位的相应关系,就可以测候八方当令节气的所在,及其相应于形体左右上下的各部位,从而也就明确了刺法上的禁忌日期。身体某个部位发生了痈肿,如果正当“太一所在”及“戊己”所值之日,就不能用溃破法治疗,犯了大禁,因为土气正旺或节气当令之日,是有助于人体正气充实之时,破溃反伤正气。所以不宜针刺的日期,叫做天忌日。\\
形体安乐,精神苦闷,发病易在经脉,用针刺治疗。形体劳苦,精神愉快,发病易在筋骨,用熨引治疗。形体安乐,精神愉快,治疗肌肉,用针刺和砭石治疗。形体劳苦,精神苦闷,发病易在咽嗌,用甘药治疗。形体屡次遭受惊恐,筋脉运行不畅,发病易出现肢体不仁的症状,用按摩、药酒治疗。这就是所谓五种形志的病。\\
五脏之气失调,会出现不同症状:心主要是噫气,肺主要是咳嗽,肝主要是多语,脾主要是吞酸,肾主要是呵欠。\\
六腑之气失调:胆易发怒,胃易哕逆,大肠小肠易为泄泻,膀胱不能约束是遗尿,下焦泛滥,易发水肿。\\
五味入胃后,按其属性各归其所合的脏腑:酸味属木,入于肝;辛味属金,入于肺;苦味属火,入于心;甘味属土,入于脾胃;咸味属水,入于肾;淡味亦附属于土,先入于胃。这就是五味各有所入。\\
五脏精气相并之证:精气并于肝则肝气抑郁,而生忧虑;并于心则心气有余,而生喜笑;并于肺则气郁胸窄,而生悲哀;并于肾则水盛火衰,而心悸善恐;并于脾则痰盛中虚,胆怯生畏。这是五脏精气相并,邪气入脏之证。\\
五脏各有所厌恶:肝厌恶风,心厌恶热,肺厌恶寒,肾厌恶燥,脾厌恶湿。这是五脏所恶。\\
五脏化生五液:心主汗液,肝主泪液,肺主涕液,肾主唾液,脾主涎液。这是五脏主五液。\\
五种劳逸过度所致的损伤:久视伤心血,久卧伤肺气,久坐伤肌肉,久立伤骨,久行伤筋。这是五种久劳所伤。\\
五味各有所走:酸味入肝而走筋,辛味入肺而走气,苦味入心而走血,咸味入肾而走骨,甘味入脾而走肉。这叫五走。\\
饮食的五种节制:酸性收敛,病在筋,不能多食酸味;辛能发散,病在气,不能多食辛味;咸能软坚,病在骨,不宜多食咸味;苦能化燥,病在血,不能多食苦味;甘能壅满助湿,病在肉,不宜多食甘味。即使嗜好欲食,也不可多食,必须自我节制。这叫五裁。\\
五脏发病的部位和季节各不相同:肾为阴脏而主骨,发病多在骨骼;心为阳脏而主血脉,发病多在血脉;脾为阴脏,主肌肉,发病多在肌肉。肝阳虚而发病,多发于冬季;肺阳虚而发病,往往发于夏季。这叫五发。\\
五脏为邪所扰的病变:病邪入于阳分,则为狂;病邪入于阴分,血脉凝涩,发生痹证;病邪入于阳,邪气搏结于上,发生头部疾患;五脏阴经通于喉舌之间,病邪入于阴,搏结不去,伤阴而瘖哑;病邪由阳入阴,病多平静,病邪由阴出阳,病多怒。\\
五脏所藏精神活动:心脏藏神,肺脏藏魄,肝脏藏魂,脾脏藏意,肾脏藏精志。\\
五脏各有所主:心主宰血脉,肺主宰皮毛,肝主宰筋膜,脾主宰肌肉,肾主宰骨胳。\\
手足各经有气血多少的不同:阳明经多血多气,太阳经多血少气,少阳经多气少血,太阴经多血少气,厥阴经多血少气,少阴经多气少血。所以说,刺阳明经可以出血与气,刺太阳经能出血不能出气,刺少阳经能出气而不能出血,刺太阴经能出血不能出气,刺厥阴经能出血不能出气,刺少阴经能出气而不能出血。\\
足阳明胃经与足太阴脾经相为表里,足少阳胆经与足厥阴肝经为表里,足太阳膀胱经与足少阴肾经为表里。这是足三阳经与足三阴经的表里关系。手阳明大肠经与手太阴肺经为表里,手少阳三焦经与手厥阴心包经为表里,手太阳小肠经与手少阴心经为表里。这是手三阴经与手三阳经的表里关系。\\
岁露论第七十九\\
黄帝问于岐伯曰:经言夏日伤暑,秋病疟。疟之发以时,其故何也?\\
岐伯对曰:邪客于风府,循膂而下。卫气一日一夜,大会于风府,其明日下一节,故其日作尚晏。此其先客于脊背也。故每至于风府则腠理开,腠理开则邪气入,邪气入则病作,此所以日作尚晏也。卫气之行风府,日下一节,二十一日,下至尾底,二十二日,入脊内,注于伏冲之脉,其行九日,出于缺盆之中,其气上行,故其病稍益早。其内搏于五脏,横连募原,其道远,其气深,其行迟,不能日作,故次日乃稸积而作焉。\\
黄帝曰:卫气每至于风府,腠理乃发,发则邪入焉。其卫气日下一节,则不当风府,奈何?\\
岐伯曰:风无常府,卫气之所应,必开其腠理,气之所舍,则其府也。\\
黄帝曰:善。夫风之与疟也,相与同类,而风常在,而疟特以时休,何也?\\
岐伯曰:风气留其处,疟气随经络,沉以内搏,故卫气应乃作也。\\
帝曰:善。\\
黄帝问于少师曰:余闻四时八风之中人也,故有寒暑,寒则皮肤急而腠理闭,暑则皮肤缓而腠理开。贼风邪气,因得以入乎?将必须八正虚邪,乃能伤人乎?\\
少师答曰:不然。贼风邪气之中人也,不得以时,然必因其开也,其入深。其内极也疾,其病人也卒暴。因其闭也,其入浅以留,其病人也徐以迟。\\
黄帝曰:有寒温和适,腠理不开,然有卒病者,其故何也?\\
少师答曰:帝弗知邪入乎?虽平居,其腠理开闭缓急,其故常有时也。\\
黄帝曰:可得闻乎?\\
少师曰:人与天地相参也,与日月相应也。故月满则海水西盛,人血气积,肌肉充,皮肤致,毛发坚,腠理郄,烟垢著。当是之时,虽遇贼风,其入浅不深。至其月郭空,则海水东盛,人血气虚,其卫气去,形独居,肌肉减,皮肤纵,腠理开,毛发残,腠理薄,烟垢落。当是之时,遇贼风则其入深,其病人也卒暴。\\
黄帝曰:其有卒然暴死暴病者,何也?\\
少师答曰:得三虚者,其死暴疾也;得三实者,邪不能伤人也。\\
黄帝曰:愿闻三虚。\\
少师曰:乘年之衰,逢月之空,失时之和,因为贼风所伤,是谓三虚。故论不知三虚,工反为粗。\\
帝曰:愿闻三实。\\
少师曰:逢年之盛,遇月之满,得时之和,虽有贼风邪气,不能危之也,命曰三实。\\
黄帝曰:善乎哉论!明乎哉道!请藏之金匮,然此一夫之论也。\\
黄帝曰:愿闻岁之所以皆同病者,何因而然?\\
少师曰:此八正之候也。\\
黄帝曰:候之奈何?\\
少师曰:候此者,常以冬至之日,太一立于叶蛰之宫,其至也,天必应之以风者矣。风从南方来者,为虚风,贼伤人者也。其以夜半至也,万民皆卧而弗犯也,故其岁民少病。其以昼至者,万民懈惰,而皆中于虚风,故万民多病。虚邪入客于骨,而不发于外,至其立春,阳气大盛,腠理开,因立春之日,风从西方来,万民又皆中于虚风,此两邪相搏,经气结代者矣。故诸逢其风而遇其雨者,命曰遇岁露焉。因岁之和,而少贼风者,民少病而少死;岁多贼风邪气,寒温不和,则民多病而死矣。\\
黄帝曰:虚邪之风,其所伤贵贱何如?候之奈何?\\
少师答曰:正月朔日,太一居天留之宫,其日西北风不雨,人多死矣。正月朔日,平旦北风,春,民多死。正月朔日,平旦北风行,民病多者,十有三也。正月朔日,日中北风,夏,民多死。正月朔日,夕时北风,秋,民多死。终日北风,大病死者十有六。正月朔日,风从南方来,命曰旱乡;从西方来,命曰白骨将将,国有殃,人多死亡。正月朔日,风从东方来,发屋,扬沙石,国有大灾也。正月朔日,风从东南方行,春有死亡。正月朔日,天和温不风,籴贱,民不病;天寒而风,籴贵,民多病。此所谓候岁之风,贼伤人者也。二月丑不风,民多心腹病;三月戌不温,民多寒热;四月巳不暑,民多瘅病;十月申不寒,民多暴死。诸所谓风者,皆发屋,折树木,扬沙石,起毫毛,发腠理者也。\\
黄帝问岐伯说:医经中说夏天伤了暑邪,到秋天会发生疟疾。疟疾的发作有一定时间,是什么原因呢?\\
岐伯回答说:邪气侵入风府,沿着脊椎下行。人体卫气循行的规律,是一日一夜在风府穴会合,然后循着脊椎逐日下行一节,这样卫气与邪气相遇,就一天比一天晚了,所以,疟疾的发作时间也就一天天向后推迟。这是邪气先侵入脊背的原因。每当卫气运行到风府时,则腠理开张,腠理开张则邪气便乘隙侵入,邪气侵入与卫气相搏,病就发作,这是疟疾发作时间常常逐渐推迟的原因。卫气运行至风府,每日下行一节,经二十一日,下行到尾骶骨,至二十二日,又入于脊内,流注于伏冲之脉,再沿经脉上行,到第九日,上出于两缺盆的中间,由于气上行逐日升高,因此发病的时间就一天比一天早了。如果邪气内迫于五脏,横连于膜原,邪气的道路距离体表已远,深藏体内,运行也较迟缓,不能每天发病,所以要积累到第二天才会发作。\\
黄帝问:卫气每到风府时,腠理就张开,张开则邪气乘隙而入,致人发病。但卫气逐日下移一节,有时不在风府处,疟疾为什么也发作呢?\\
岐伯说:风邪侵入体内并没有固定位置,只是卫气与邪气相搏,就有所反应,必定使腠理开张而发病,所以邪气留滞之处,就是发病的所在。\\
黄帝说:讲得好。风邪致病和疟疾,相互之间属于同类,但是,风邪为病,常常持续存在,而疟疾的发作却按时休止,为什么呢?\\
岐伯说:因为风邪常停留在发病部位,而疟邪之气却能随着经络,深入而搏结于内,所以与卫气相遇,发生搏击,引起抗邪反应,就会发作。\\
黄帝说:讲得好。\\
黄帝问少师说:我听说四时八风伤害人体,与寒暑气候的不同有关。寒冷时人的皮肤紧急,腠理闭合;暑热时人的皮肤舒缓,腠理开张。贼风邪气是乘人体皮腠开泄而侵入的呢?还是必须遇到八节虚邪,才会伤人呢!\\
少师回答说:不是这样。有的贼风邪气伤人,没有固定时期,但必须乘人体皮腠开张时,才能侵入。邪气侵入部位深的,病就严重,所以发病也急暴。如果在皮腠闭合时,邪气即使侵入,也只能停留在浅表部位,发病也比较迟缓。\\
黄帝问:有的人能够调和适应寒温变化,腠理也不开张,但突然发病了,是什么缘故?\\
少师回答说:您不知道邪气侵入吗?即使在人们平时的生活中,腠理的开闭缓急,也都有一定的时间规律。\\
黄帝问:可以让我听听吗?\\
少师说:人与天地自然相参,与日月运行相应。所以在满月时,海水西盛。这时人的血气清和,肌肉充实,皮肤致密,毛发坚竖,腠理闭合,皮脂多而表固。在这个时候,即使遇到贼风侵入,也浅不能深。如果到了月缺时,海水东盛。这时人的气血较虚,卫气离开体表,深入于里,外形虽然如常,但肌肉消减,皮肤弛缓,腠理开张,毛发残损,腠理疏薄,皮脂剥落。在这个时候,若遇到贼风,它就能深入内里,使人发病急暴。\\
黄帝问:有的人突然死亡,或突然生病,这是什么原因?\\
少师回答说:遇到三虚的,会出现暴病暴死的情况;遇到三实的,就不会为邪气所伤害了。\\
黄帝说:希望听听三虚的道理。\\
少师说:在岁气不及的衰年,又遇到月缺无光的黑夜,四时节令又反常,因而被贼风所伤,这就叫三虚。所以讨论医道,如果不懂得三虚致病的理论,就是学识浅陋的粗工。\\
黄帝说:希望听听三实。\\
少师说:逢上岁气旺盛之年,又遇到月光圆满,再有调和的气候,即使有贼风邪气,也不能危害人体,这就叫三实。\\
黄帝说:说得好极了!道理也讲得很明白!请把它保存在金匮里,命名为“三实”,不过,这只是指个人发病而说的。\\
黄帝问:在一年之内,人们都患同样的病,是什么原因造成的呢?\\
少师说:这要候察八方气候的变化。\\
黄帝问:怎样候察呢?\\
少师说:候察确定这种情况,通常是在冬至日这一天,太一北斗星立于叶蛰之宫的时候去观察,因为太一移行到这一天,必有风出现。风从南方来的,叫作虚风,能够伤害人体。如果风在半夜来,这时人们都已入睡,邪气不能侵犯,所以当年人们很少生病。若风在白天来,由于人们防护松懈,容易被虚风所伤,因此多数人会生病。如果冬季虚邪侵入骨髓,没有发泄于外,到了立春,阳气逐渐旺盛,腠理开张,在立春这天,刮来了西风,人们又会被这种虚风所伤,此时潜伏在体内的伏邪与新感之邪合并,留结在经脉之中,两邪交替而发病。所以凡是遇到风或雨而使人发生疾病,就命名做“遇岁露”。由于一年之中气候调和,很少有贼风,人们患病和死亡的就少;如果一年中多有贼风邪气,气候寒温不调,人们患病和死亡的就多。\\
黄帝问:虚邪这种风,它伤人轻重的情况怎样呢?怎么候察呢?\\
少师回答说:正月初一日,太一移居天留宫,这一天刮西北风而不下雨,人多病死。正月初一日,早晨刮北风,到了春季,患病的人多死。正月初一日,早晨刮北风,患病的人多,约有十分之三。正月初一日。中午刮北风,到了夏季,人多病死。正月初一日,傍晚刮北风,秋天人多病死。整天刮北风,人患大病而死的约有十分之六。正月初一日,风从南方来,叫旱乡;风从西方来,叫白骨堆积,全国会有祸殃流行,人多死亡。正月初一日,风从东方刮来,掀翻房屋,飞沙走石,国家将有大灾发生。正月初一日,风从东南方来,春天人多病死。正月初一日,天气温和,不刮风,是丰年的先兆,粮价贱,人们也少病;如果天气寒冷刮风,这是荒年的先兆,粮价贵,人们也多病。这就是所说的,在正月初一日观察风向,可以预测虚邪贼风伤人的情况。如果二月的丑日,不起风,人们多患心腹病;三月的戌日,气候不温暖,人们多患寒热病;四月的巳日不热,人们多患黄疸病;十月的申日不冷,人们多暴死。以上所谓的风,都是指能损毁房屋,折断树木,飞沙走石,吹得使人毫毛竖起,腠理开张的大风。\\
大惑论第八十\\
黄帝问于岐伯曰:余尝上于清泠之台,中阶而顾,葡匐而前,则惑。余私异之,窃内怪之,独瞑独视,安心定气,久而不解,独博独眩,披发长跪,俯而视之,后久之不已也。卒然自止,何气使然?\\
岐伯对曰:五脏六腑之精气,皆上注于目而为之精。精之窠为眼;骨之精为瞳子;筋之精为黑眼;血之精为其络窠;气之精为白眼;肌肉之精为约束。裹撷筋骨血气之精而与脉并为系,上属于脑,后出于项中。故邪中于项,因逢其身之虚,其入深,则随眼系以入于脑,入于脑则脑转,脑转则引目系急,目系急则目眩以转矣。邪中其精,其精所中不相比也,则精散,精散则视歧,视歧见两物。\\
目者五脏六腑之精也,营卫魂魄之所常营也,神气之所生也。故神劳则魂魄散,志意乱。是故瞳子黑眼法于阴,白眼赤脉法于阳也,故阴阳合传,而精明也。目者,心使也。心者,神之舍也。故神精乱而不抟,卒然见非常处,精神魂魄,散不相得,故曰惑也。\\
黄帝曰:余疑其然。余每之东苑,未曾不惑,去之则复,余唯独为东苑劳神乎?何其异也?\\
岐伯曰:不然也。心有所喜,神有所恶,卒然相感,则精气乱,视误,故惑,神移,乃复。是故间者为迷,甚者为惑。\\
黄帝曰:人之善忘者,何气使然?\\
岐伯曰:上气不足,下气有余,肠胃实而心肺虚。虚则营卫留于下,久之不以时上,故善忘也。\\
黄帝曰:人之善饥而不嗜食者,何气使然?\\
岐伯曰:精气并于脾,热气留于胃,胃热则消谷,谷消故善饥。胃气逆上,则胃脘塞,故不嗜食也。\\
黄帝曰:病而不得卧者,何气使然?\\
岐伯曰:卫气不得入于阴,常留于阳。留于阳,则阳气满,阳气满,则阳\\
盛;不得入于阴,则阴气虚,故目不瞑矣。\\
黄帝曰:病目而不得视者,何气使然?\\
岐伯曰:卫气留于阴,不得行于阳。留于阴,则阴气盛,阴气盛,则阴\\
满;不得入于阳,则阳气虚,故目闭也。\\
黄帝曰:人之多卧者,何气使然?\\
岐伯曰:此人肠胃大而皮肤涩,而分肉不解焉。肠胃大则卫气留久,皮肤涩则分肉不解,其行迟。夫卫气者,昼日常行于阳,夜行于阴。故阳气尽则卧,阴气尽则寤。故肠胃大,则卫气行留久;皮肤涩,分肉不解,则行迟。留于阴也久,其气不精,则欲瞑,故多卧矣。其肠胃小,皮肤滑以缓,分肉解利,卫气之留于阳也久,故少瞑焉。\\
黄帝曰:其非常经也,卒然多卧者,何气使然?\\
岐伯曰:邪气留于上焦,上焦闭而不通,已食若饮汤,卫气留久于阴而不行,故卒然多卧焉。\\
黄帝曰:善。治此诸邪,奈何?\\
岐伯曰:先其藏府,诛其小过,后调其气,盛者泻之,虚者补之。必先明知其形志之苦乐,定乃取之。\\
黄帝问岐伯说:我曾经在登上清泠台时,走到中间的台阶,回头看了一下,然后匍匐前行,感到神魂迷惑。我心里觉得奇怪,内心感到很诧异。于是就一会儿闭眼,一会儿又睁眼,让自己平心静气,好久还是没有消除。反而看得越远眩得越厉害,于是披散开头发,久跪在台上,低头向下看,之后很久不能停止。后来突然就自动停止了,这是什么气造成的?\\
岐伯回答说:五脏六腑的精气都向上输送而汇聚到两眼,形成视觉功能。这些精气汇集合并,成为眼目,其中骨之精注入瞳子;筋之精注入黑眼,血之精注入血络;气之精注入白眼;肌肉之精注入眼胞。包裹了筋骨血气等的精气,与脉合并,成为目系,向上行内属于脑,向后出于项中。所以邪气侵入项部,因逢身体虚弱,邪气深入,随着眼系侵入脑部,侵入脑部,就发生脑转,脑转又会牵引目系紧急,目系紧急则两目眩晕而脑转。邪气伤害了精气,精气为邪气所伤则相互之间不能紧密联系,而致精气涣散,精气涣散则视物分歧,视物分歧就是视一为二。\\
眼目,是五脏六腑的精气和营、卫、魂、魄经常营运的地方,也是产生神气的部位。所以当精神劳累时,会使魂魄分散,志意混乱。所以瞳孔、黑眼取法于阴气,白眼、赤脉取法于阳气,所以阴阳之精气相互聚合,就能产生眼睛的视觉。眼睛为心所指使。心是神气所居之所,所以神气混乱而使精气不能抟聚于目时,突然看到非常的事物,精神魂魄散乱而不相安和,就发生眩惑。\\
黄帝说:我怀疑你讲的这些道理。我每次去东苑,总是眩惑,离开那里就恢复正常,难到我只因为东苑才消耗神气吗?为什么会出现这种特殊的现象呢?\\
岐伯说:不是这样。心里虽是喜爱的,但是精神上又厌恶,这样爱憎两种情绪,突然相感,则精气紊乱,视觉产生错误,使人感到眩惑,等精神转移,就恢复正常。所以这种情况,轻的称为“迷”,重的称为“惑”。\\
黄帝问:有的人常常健忘,是什么气形成的?\\
岐伯说:由于上部之气不足,下部之气有余,也就是肠胃之气充实而心肺之气虚弱,心肺气虚。心肺气虚则营卫之气稽留在下部,久而不能按时上行,所以容易健忘。\\
黄帝问:有的人容易饥饿但又不想吃东西,这是什么气形成的?\\
岐伯说:精气积并于脾,热气蕴留于胃,胃热太甚易于消化水谷,水谷易消化所以容易饥饿。由于胃气上逆,胃脘壅塞不通,所以又不想吃东西。\\
黄帝问:人患病不能入睡,是什么气形成的?\\
岐伯说:这是卫气不能入于阴分而经常滞留于阳分的缘故。稽留在阳分则阳气盛满,阳气盛满,使阳刉脉的脉气偏盛;不得入阴分则阴分偏虚,所以不能闭目入睡。\\
黄帝问:有的人患两目紧闭而不能视物的病,是什么气造成的?\\
岐伯说:这是卫气稽留在阴分,而不能运行到阳分的缘故。稽留在阴分则阴气偏盛,阴气偏盛,使阴刉脉盈满;不行于阳分,则阳分气虚,所以闭目而不张。\\
黄帝问:有人时常嗜睡,是什么气造成的?\\
岐伯说:这种人肠胃宽大,皮肤涩滞,肌肉不滑利。肠胃宽大则使卫气停留的时间长,皮肤涩滞则肌肉不滑利,卫气运行迟缓。卫气白天行于阳分,夜晚行于阴分。所以卫气在阳分行尽就入睡,在阴分行尽就起床。所以肠胃大,卫气运行过久;皮肤涩滞,分肉不滑利而卫气运行缓慢。停留在阴分的时间长,其气不精,则使人欲闭目,所以这种人嗜睡。如果肠胃狭小,皮肤滑利而弛缓,分肉也解利,卫气停留在阳分的时间较长,两眼少闭而不想睡觉。\\
黄帝问:有的人不是经常好睡,而是突然发生嗜睡,这是什么气造成的?\\
岐伯说:邪气留滞在上焦,上焦闭塞不通,如果已吃过饭或喝了汤水,使卫气久留在阴分而不能外行,所以会出现突然嗜睡。\\
黄帝说:好。怎么治疗这些疾病呢?\\
岐伯说:首先明确疾病所属的脏腑,祛除轻微的邪气,然后再调理营卫之气,实证用泻法,虚证用补法。必须首先清除形体和情志的苦乐,决定以后才能治疗。\\
痈疽第八十一\\
黄帝曰:余闻肠胃受谷,上焦出气,以温分肉,而养骨节,通腠理。中焦出气如露,上注谿谷,而渗孙脉,津液和调,变化而赤为血。血和则孙脉先满溢,乃注于络脉,皆盈注于经脉。阴阳已张,因息乃行,行有经纪,周有道理,与天合同,不得休止。切而调之,从虚去实,泻则不足。疾则气减,留则先后。从实去虚,补则有余。血气已调,形气乃持。余已知血气之平与不平,未知痈疽之所从生,成败之时,死生之期,有远近,何以度之,可得闻乎?\\
岐伯曰:经脉流行不止,与天同度,与地合纪。故天宿失度,日月薄蚀,地经失纪,水道流溢,草蓂不成,五谷不殖,径路不通,民不往来,巷聚邑居,则别离异处,血气犹然,请言其故。夫血脉营卫,周流不休,上应星宿,下应经数。寒邪客于经络之中则血泣,血泣则不通,不通则卫气归之,不得复反,故痈肿。寒气化为热,热胜则腐肉,肉腐则为脓,脓不泻则烂筋,筋烂则伤骨,骨伤则髓消,不当骨空,不得泄泻,血枯空虚,则筋骨肌肉不相荣,经脉败漏,熏于五藏,藏伤故死矣。\\
黄帝曰:愿尽闻痈疽之形,与忌、日、名。\\
岐伯曰:痈发于嗌中,名曰猛疽。猛疽不治,化为脓,脓不泻,塞咽,半日死,其化为脓者,泻则合豕膏,冷食,三日而已。\\
发于颈,名曰夭疽。其痈大以赤黑,不急治,则热气下入渊腋,前伤任脉,内熏肝肺,熏肝肺,十余日而死矣。\\
阳气大发,消脑留项,名曰脑烁。其色不乐,项痛而如刺以针。烦心者,死,不可治。\\
发于肩及臑,名曰疵痈。其状赤黑,急治之,此令人汗出至足,不害五藏,痈发四五日,逞焫之。\\
发于腋下赤坚者,名曰米疽。治之以砭石,欲细而长,疏砭之,涂以豕膏,六日已,勿裹之。其痈坚而不溃者,为马刀挟瘿,急治之。\\
发于胸,名曰井疽。色青,其状如大豆,三四日起,不早治,下入腹,不治,七日,死矣。\\
发于膺,名曰甘疽。色青,其状如谷实瓜蒌,常苦寒热,急治之,去其寒热,十日,死,死后出脓。\\
发于胁,名曰败疵。败疵者,女子之病也。久之,其病大痈脓。治之,其中乃有生肉,大如赤小豆,剉肊阞草根各一升,以水一斗六升,煮之,竭为取三升,则强饮厚衣,坐于釜上,令汗出至足,已。\\
发于股胫,名曰股胫疽。其状不甚变,而痈脓搏骨,不急治,三十日,死矣。\\
发于尻,名曰锐疽。其状赤坚大,急治之,不治,三十日,死矣。\\
发于股阴,名曰赤施。不急治,六十日,死。在两股之内,不治,十日而当死。\\
发于膝,名曰疵痈。其状大痈,色不变,寒热,如坚石。勿石,石之者,死;须其柔,乃石之者,生。\\
诸痈疽之发于节而相应者,不可治也。发于阳者,百日死;发于阴者,三十日死。\\
发于胫,名曰兔啮。其状赤至骨,急治之,不治害人也。\\
发于内踝,名曰走缓。其状痈也,色不变,数石其输,而止其寒热,不死。\\
发于足上下,名曰四淫。其状大痈,急治之,百日死。\\
发于足傍,名曰厉痈。其状不大,初如小指发,急治之,去其黑者,不消辄益,不治,百日死。\\
发于足指,名脱痈。其状赤黑,死不治;不赤黑,不死。不衰,急斩之;不,则死矣。\\
黄帝曰:夫子言痈疽,何以别之?\\
岐伯曰:营气稽留于经脉之中,则血泣而不行,不行则卫气从之而不通,壅遏而不得行,故热。大热不止,热胜则肉腐,肉腐则为脓。然不能陷,骨髓不为燋枯,五藏不为伤,故命曰痈。\\
黄帝曰:何谓疽?\\
岐伯曰:热气淳盛,下陷肌肤,筋髓枯,内连五脏,血气竭,当其痈下,筋骨良肉皆无余,故命曰疽。疽者,上之皮夭以坚,上如牛领之皮;痈者,其皮上薄以泽。此其候也。\\
黄帝说:我听说肠胃受纳水谷,至上焦化为卫气,来温煦肌肉,营养骨节,通利腠理。在中焦化为营气,像雾露一样,注入到谿谷,渗入到孙脉,与津液调和后,变成红色的血液。血和则先把孙脉充满,充满则溢入络脉,络脉充满后,再传注到经脉。这样阴经阳经都得到补给,随着呼吸运行于全身的经脉,经脉的运行有规律,环周有其道理。与天地自然的规律相同,运动不休。应专心诊察调治,用泻法驱除实邪,但过泻会伤正气而为不足。针刺时,快速出针,就可以削减邪气,如用留针法就不能泻而病情先后如一,没有变化。用补法可以驱除虚证,但过补会助长余邪。治疗的目的在于达到血气调和,形气之间的机能恢复正常。我已经懂得了血气的平与不平的道理,但不知道痈疽是怎样发生的,以及形成和消散的日期,痊愈或死亡的日期,时间的远近,这些怎样预测呢?可以听听吗?\\
岐伯说:经脉运行不止,与天地自然有一样的规则。所以天上的星宿运行失去常度,则有日蚀或月蚀,地上的十二经水流行失去常规,则会漫出水道,泛滥为灾,而致草木涝死不成,五谷不能成长,甚至道路被淹,不能沟通,人们不相往来,或聚集街巷或居住小邑,隔离在不同的地方。血气的运行也是如此,让我谈谈其中的道理。人体的血脉和营卫是循环周流,永不停息的,与在上的二十八宿和在下的十二经水相参相应。如果寒邪侵入经络,血行就会凝涩,凝涩以致不通,不通则卫气运行受到阻碍而停在不通之处,影响它的循环往反,因而形成痈肿。寒邪之气化热,热盛就会腐蚀肌肉,肌肉腐蚀了就会化脓。脓不得排泄就会烂筋,筋烂了就会伤骨,骨伤则骨髓消减,如脓汁不在骨节的缝隙处,就无从排泄,就会引起营血虚亏,这样筋骨肌肉都得不到营养,经脉因之衰败损伤,毒气蔓延五脏,五脏受到严重伤害就会死亡。\\
黄帝说:我想全面听听痈疽的形状以及禁忌、预后、名称。\\
岐伯说:痈发生在咽喉的,叫猛疽。不及时治疗则化脓,脓液不能排出,堵塞咽喉,半天即死。已经化浓的,排出脓液,可配合猪油膏冷服,三天可以痊愈。\\
发生在颈部的痈,叫夭疽。范围大而颜色紫黑,不及时治疗,热毒之气下移至渊腋部,前伤任脉,向内熏灼肝肺,熏灼肝肺,在十几天内,就会死亡。\\
阳邪亢甚,消烁脑髓,而毒邪留结在项部,形成的痈疽叫脑烁。神色惨淡,项痛有如针刺。如果出现心烦,便是死证,不能治。\\
发生在肩、臂部的,叫疵痈。颜色紫黑,要赶紧治疗,否则,会使病人汗出至足,但不致伤及五脏,在痈发四五天内,赶快用灸法。\\
发生在腋下,红肿坚硬的,叫米疽。用砭石治疗,选用细长的砭石,用稀疏的砭刺法,再涂上猪油膏,六天能痊愈,不需要包扎。如果痈疽坚硬不溃时,是马刀挟瘿,要赶紧治疗。\\
发生在胸部的痈疽叫井疽。颜色发青,形状如大豆,三四天内,不及时治疗,邪毒向下内陷入腹,成为不治之症,到七天就会死。\\
发生在膺部的痈疽,叫甘疽。皮色发青,形状如谷粒或瓜蒌,常发寒热,赶快治疗,除去寒热,如不及时治疗,十天后会死亡,死的时候才流出脓液。\\
发生于胁部的痈疽,名叫败疵。败疵多发于妇女。迁延日久,可变成大痈脓。治疗后,其中长出新肉,有赤小豆大小,剉碎菱草和连翘的根,各一升,加水一斗六升,煮取三升,乘热勉强喝下,饮后穿厚衣服,坐在热釜上熏蒸,使汗出到脚,便愈。\\
发生在股胫部的痈疽,名叫股胫疽。它的外形没有太大变化,但脓液贴附骨上,不赶紧治疗,三十天内会死。\\
发生在尾骶骨部的痈疽,名叫锐疽。颜色赤,坚硬而肿大,应急治,如果不治,三十天内可以致死。\\
发生在大腿内侧的痈疽,叫赤施。不赶紧治疗,六十天内会死。假使两股同时发生,不急治,当在十天内死亡。\\
发生在膝部的叫疵痈。痈的外形肿大,患处皮色不变,恶寒发热,坚硬如石。此时不可用砭刺法,如果误用砭刺,会致死亡;必须等它柔软时,再用砭刺,可以复生。\\
凡痈疽生在关节上下左右相对的,都是不治之证。生在阳分的一百天死亡;生在阴分的三十天死亡。\\
发生在足胫的,叫做兔啮。外形色红而深至骨,赶紧治疗,不抓紧治疗要危及生命。\\
发生在内踝部的痈疽,名叫走缓。形状像痈而皮色不变,多次用砭石刺其肿处,除去寒热,可以不死。\\
发生在足部上下的痈疽,名叫四淫。外形如大痈,应赶紧治疗,一百天内会死。\\
发生在足旁的痈疽,名叫厉痈。外形不大,初发如小指大,要赶紧治疗,把已发黑的部分消除,如果不能消散,很快会加重,如不治疗,一百日内可致死亡。\\
发生在足趾上的痈疽,名叫脱痈。外现紫黑色的,为不治之死证;不见赤黑色的,不死。如病情没有衰退之象,赶快截除,否则,难免死亡。\\
黄帝说:夫子说的痈和疽,怎样区别呢?\\
岐伯说:营气稽留在经脉之中,则血液凝涩而不行,不行则卫气因之而不能通畅,壅阻而不能运行,所以蕴郁生热。大热不能休止,热毒偏盛则使肌肉腐烂,肌肉腐烂则化为脓。但不会内陷入里,不会使骨髓焦枯,五脏不会被损伤,所以命名为痈。\\
黄帝说:什么叫疽呢?\\
岐伯说:热毒之气亢盛,陷入肌肤,使筋髓枯萎,向内连及五脏,气血耗竭,在痈肿部位,筋骨和肌肉全都败坏无余,所以命名为疽。疽证,患部皮色枯暗而且硬如牛颈项的皮;痈证,患部皮薄而且颜色光泽。这就是痈证和疽证的候察要点。\\
\end{document}