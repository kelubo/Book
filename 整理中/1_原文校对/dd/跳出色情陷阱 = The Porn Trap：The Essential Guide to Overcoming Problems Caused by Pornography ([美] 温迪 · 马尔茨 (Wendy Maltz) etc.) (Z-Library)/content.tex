\documentclass{article}
\usepackage{ctex}
\begin{document}
\title{跳出色情陷阱 = The Porn Trap:The Essential Guide to Overcoming Problems Caused by Pornography ([美] 温迪 · 马尔茨 (Wendy Maltz) etc.) (Z-Library)}
\maketitle
\section{内容}
中文版序\\
我们非常高兴看到简体中文版《跳出色情陷阱》的问世。我们要感谢法律出版社的领导和编辑们,他们支持该项目,并决定发布简体中文版。我们也要感谢宋尚鸿小姐为此书出版做出的努力,她积极勤奋,坚持不懈,最终将出版的梦想变为现实。法律出版社和宋尚鸿小姐都认识到,当下借助于电子科技传播的色情问题己经超越了文化和国度的范围。色情问题已经在全球范围内传播开来,影响着世界各地的人群。或许我们的生活方式有所不同,但我们相信本书提供的信息和治疗策略将惠及中国读者。打破沉默,在国际领域共享信息,这些强有力的措施将会加强我们的性健康,巩固我们的家庭,改善我们的生活。——温迪·马尔茨、拉瑞·马尔茨\\
两位美国性学家给中国人带来的启发 ——夏海新 中国反色情网负责人\\
我们不可否认,色情是人类最大的诱惑。在今天这个多元化的年代里,对色情的各种解读层出不穷,光怪陆离。\\
作为一个关注和抵制色情危害的公益机构的义工,我常常会遇到这样的质问:你们为什么反对色情?色情,给人类带来了快乐和享受;它有你们说的那么可怕吗?它为什么就不能代表人类的尊贵?\\
我们有这样一种回答:如果色情能给人带来尊贵,那么那些艺术家为什么不夜夜享受性带来的快乐,而去追求耀眼灯光下的掌声与鲜花?许多动物没有道德法律的约束,它们的性是自由的,为什么它们不随时随地地性交?\\
但可怕的是,在色情越来越泛滥的当下,我们身边的许多人正成为色情之魔的俘虏,不能自拔。\\
在欲望的追逐中,许多人误解了幸福的本意:真正的幸福是安详、宁静,是内心的甜蜜与满足,是健康的愉悦……而相对于此,色情给人类带来了什么呢?宝贵的生命能量被空耗,原本纯洁的心变得扭曲、猥琐、空虚,治愈身体疾病的阳气被掏空,原本通畅的命运之途变得坎坷而卑下。就像弗兰西斯·培根所总结的那样:放荡的青春必然迎来愧悔的晚年。\\
长期从事反色情的教育救助工作,我们目睹了太多的人间悲剧:迷恋黄网引发的心理变态、婚外恋导致的家庭悲剧、因色成仇的血案、早早死去的性乱者、病苦不堪的人、因父母性乱而性格扭曲的孩子……如果一一记录下来,简直是一幅人间地狱图,而他们,原本都可以有尊严而幸福地活着。\\
告诉您一组数字:中国每天有5000个家庭破裂,超过2/3的原因是婚外恋;每年中国有1300万胎儿被人工流产、1000万婴儿被药流,主要的原因是青少年未婚同居;监狱里,1/3的刑事犯罪因色情而发,80%的罪犯都是来自问题家庭(离婚、父母不和,主因还是色情)……\\
性是自由的吗?性是个人的吗?当您看了这组数字之后,您会怎样认为?一个社会的人,自然有追求幸福的权利,有追求快乐的权利,但没有给他人和社会带来危害的权利!\\
“万恶淫为首,百善孝为先”、“色字头上一把刀,有福消福,无福消寿”、“劝君莫借风流债,家中自有代还人,借得快来还得快,你要赖时她不赖”、“色是少年第一关”……这些一度被国人摒弃的老祖宗的教诲,被一个个生命和家庭痛苦地印证,它们真实不虚!\\
近年,互联网改变了世界,带给了人们便捷的同时也给人类的幸福安康带来了巨大危害。我们痛心地看到,许多人因迷恋于网上轻易获得的色情,而变得卑琐不堪,甚至有种种变态非法的举动。色情的危害,不亚于第二次鸦片战争。许多学校的体育运动会都取消了长跑,因为许多学子都被掏空了身子。\\
与泛滥的网络色情相对应,手淫无害论更是让一代年轻人误入迷途,有许多孩子向我们求助,因为手淫,他们早早衰老不堪!因为手淫,他们丧失了大好的人生前途,“本是玉堂人物,弄成邪僻儿郎”。\\
更令我们痛心的是,由于性乱,许多青少年怀孕之后听信无痛人流的谎言随意堕胎,每年数以千万的生命被扼杀在子宫之中,而且,导致5000万人婚后不孕不育,而本来最聪明、最健康的头胎被轻而易举地堕掉,人口素质在下降。色情,已经危及中华民族的人种人根!\\
为此,我们建立了中国反色情网,就是要建立一道网上的绿色长城,保卫孩子,保卫家庭,保卫正在崛起的中华民族!\\
我们高兴地发现,许多有志之士聚拢到我们周围,许多天各一方的同仁在与我们一起前行!\\
这一次,我们有幸与温迪·马尔茨和拉瑞·马尔茨结缘,这一对美国的社工夫妻在做着与我们同样的努力,他们帮助了大量因色情危害导致心理、家庭等问题的受害者,令我们尊敬和赞叹。\\
我们非常愿意将他们这本《跳出色情陷阱》推荐给中国的广大读者。《跳出色情陷阱》一书来到中国恰逢其时,他们让我们看到美国性学家严谨的科学态度,让我们看到了曾走入性解放误区、现已开始觉醒的美国人对性的态度。\\
书中大量的事实、广泛的调查、深入的探索让我们看到了色情对人类的危害。我们相信,这本书也会给许多性科学领域的真正研究者带来参考和启发。\\
由于文化背景的不同,关于性,以及治疗色情瘾心理疾患的方法,两位美国专家的观点会与我们不同。中国传统文化中,有对性深刻的认识,对性乱后果的改善也有根本治疗的做法,这些在我们的网站上都有摘录,不一一赘述。\\
导言\\
我不酗酒,我从来不磕药,我也不喜欢抽烟;这些都不是我的问题。我的问题就是色情。——艾利克斯\\
追忆到25年前,那时笔者从未想过要写一本以色情为主题的著作。20世纪80年代,社会刚刚开始探索治疗性问题和情感问题的方式,而并未曾重视色情这个问题。和心理学领域内的大多数同行一样,笔者虽然觉得色情粗俗低级,但还是无伤大雅。那时笔者参加性治疗培训时获得的相关资料,常常建议我们向客户推荐限制级录像和色情品,以促进客户和其伴侣之间的亲密度。\\
直到20世纪90年代中期,笔者才开始重视色情问题。因为自那时起,越来越多的客户上门咨询如何解决由色情引发的问题,因为色情已经严重影响到他们的社交生活。很快,笔者就发现色情极易扭曲个人的性趣和性欲,导致人不去享受正常性关系,却转而寻求性工具,或者幻想人物和场景来体验性爱。归根结底,原因在于色情的功能往往从促进双方的亲密度,转变成让色情观看者要依赖于色情来体验性。\\
这就是问题所在:以前生产色情录像、杂志和书籍的目的是为了给两人世界添料的,而如今,色情自身逐渐成为了性欲的对象。它会给沉溺其中的人洗脑,让人只想着身体部位和特定的性行为,却再没有能力去和现实中的伴侣体验浪漫和激情,也无法享受情感和身体上的亲昵。这演变成一场色情和伴侣之间的战争,他们在竞争着成为为色情观看者的性释放对象。\\
并非无害的幻想\\
自我们意识到色情问题后患无穷后的l10年间,社会上色情品数量激增,要接触色情也越来越方便。正因如此,越来越多的美国人,以及世界其他地区的人们,都前赴后继地陷入了或正在落入色情这个巨大的陷阱。因为色情,情侣分手,家庭破碎。单身汉们觉得,沉溺于色情之后,自己对伴侣的忠诚度下降,交往的持续时间也变短。自以为顺利治愈色情瘾数年的人,经不住诱惑,再次沦陷,而复发后的情况往往更为严重。\\
色情往往是个人私下偷偷摸摸观看的,而那些被抓个正着的人事后总会羞愧不已,产生抑郁的情绪,他们觉得自己虚伪,有悖伦理,孤立无助,有时甚至会产生轻生的念头。也有很多人变得焦躁不安,彻夜失眠。有人向笔者倾诉,色情引诱他们走上不法、高危的歧途,像观看儿童色情,搞婚外恋,在成人书店滥交,召妓,工作时看色情。大部分客户承认,即使他们明白了色情的负面影响,也无法抵制诱惑。色情与烟、酒和毒品一样。不能戒除它,就是因为上了瘾。\\
色情极易从偶尔的消遣变成积习难改的问题。从而给现实生活带来无穷的灾难。受访的色情观看者意识到这一点时往往不敢置信。对他们来说,色情是逃避现实的感官娱乐方式。刚开始就是为了找乐子,那令人脸红心跳的禁忌世界令人沉迷,而事实上,色情却最终会化成恐怖的陷阱,如同流沙一般,把陷入其中的人悄无声息地湮没,而当事人却浑然不觉。以一些最悲剧的人来说,色情毁掉了他们的全部生活,阻碍了他们的社交,砸掉了他们的饭碗,践踏了他们的自尊,甚至扼杀了他们所有的梦想和希望。\\
请注意,并不是只有那些陷人色情陷阱的人才会尝到苦果,他们的伴侣也会向笔者寻求帮助。伴侣们抱怨说,色情观看者强迫自己进行性行为,两人的夫妻生活冷淡。一些伴侣会感到自己的身体、外貌,或者性表现受到了对方的嘲笑,使得她们深感自己的吸引力降低。\\
根据美国婚姻律师学会报道,强迫性网络使用症是导致2002年美国离婚率居高不下的罪魁祸首之一,而其中至少50%的案例都与色情有关。这个数据让笔者最终确定,色情对于色情观看者的伴侣有着不容忽视的影响。而就在8年之前,色情和离婚基本上毫无关系。\\
色情观看者的伴侣不仅担心两人未来的共同生活,也经常担心孩子是否会受到色情的不良影响。这种担忧不无道理,孩子们很有可能会发现家长收藏的色情,而家长对性行为、色情的态度也会在潜移默化中影响孩子。如果一位家长时常观看色情,而另一位家长觉得不妥,这会让成长中的孩子感到困惑,无法判断怎样的性行为才算正常。总之,色情观看者的伴侣常常感到被冷落,想保护孩子却心有余而力不足。总之一句话,色情陷阱所困扰的,不仅仅是色情观看者本人而已。\\
什么不同了\\
几乎每位被色情所困扰的人都认为,自己会身陷色情陷阱,网络和其他电子移动设备责不可逃。无论何时,只要你一点鼠标,一按遥控器,淫秽的的图片和录像、色情聊天室、情色游戏就会出现在你面前,操作简单,无后顾之忧,马上就能让你享受性快感,如此诱惑实在难以抗拒。现代高端科技的发展,不需要消费者跑到人前去租借或购买限制级内容,匿名就可以获取色情信息。\\
30年前,想要接触色情,你要付出时间,金钱和精力。现在呢?你要付出时间、金钱和精力才能远离色情。不管人们主观上是否情愿,色情都会见缝插针,无孔不人,电脑中群发的邮件,莫名的链接,弹出来的窗口。都是色情接近我们的途径。这话说得有理:“你不必去找色情,它自己会送上门来!”\\
30年前,只有很少一部分人经常观看色情;而如今,色情吸引的人数之多,前所未有。男女老少。各行各业的人,都会成为色情的俘虏。1年365天,1周7天,1天24个小时,色情无时无刻不在发挥着腐蚀作用。单在美国,约4000万人每月最少浏览一次色情网站:其中一些人每次就看个几分钟,而有些人会经常浏览色情内容,一看就是几个小时。在网络上,25%的日常搜索,35%的下载,都涉及色情信息。\\
沉溺于色情的人群以男性为主,占总数的75%~85%,这会让人觉得合情合理。但是近年来,女性观看色情的现象越来越普遍,而下面这个事实更会让人觉得触目惊心:未成年群体已经成为色情的最大消费人群。毋庸置疑,接触色情的年纪越小,上瘾就越快,情况就会越加严重,这导致如今色情成瘾者群体之庞大,史无前例。\\
我们为什么写书\\
倾听了色情给许多人的生活带来的严重影响,了解了许多痛苦而励志的个人经历后,笔者决定,除了提供传统的治疗方式,我们要付出更多来帮助这个群体,经过大量的调查研究,我们很快就发现,受色情影响的群体基数正在迅猛膨胀,而色情给个人带来的问题也越来越严重:原来不过是一个小群体的小问题,如今已经演变成了严峻的社会问题。这个问题犹如脱缰野马,越发不可控,给无数人的生活带来了无尽的问题。我们也曾经与其他心理治疗师,尤其是性和感情治疗师沟通交流,他们也察觉到了同样的趋势:10年前,很少有客户会上门咨询色情问题,而如今,色情问题已经成为他们的工作重心所在。\\
笔者开始查阅相关文献和书籍,却发现有效信息寥寥无几,色情引发的一系列问题也未成为社会关注的焦点。美国文化倾向于避免公开或者严肃地谈论性问题,因此现有的大多数调查研究仅着眼于个人短期接触色情的后果,以此来判断色情是否会引发性暴力,但没有任何一组研究关注日常生活中观看色情以及伴随的手淫行为。鉴于现状,笔者最终肯定,调查色情对观看者的人际交往以及伴侣关系影响的研究寥寥无几。\\
作为性以及情爱关系问题专家,我们认为色情成瘾问题即人际关系问题。色情会影响个人的精神健康,也会影响个人与伴侣及其他家庭成员的互动。习惯性地观看色情,很可能会损害个人自尊,影响性生活质量。要克服色情的诱惑,仅靠停止观看色情的行为,或是强迫自己克服瘾头是不够的,这还需要个人重新树立正直的品行,确立健康的性观念。而笔者在收集文献的过程中,未曾看到任何一组研究从这个角度讨论过色情问题。\\
鉴于信息不足,我们便决定动手写一本著作来填补这个空缺。在与同行无数次长谈,对专业文献进行地毯式的梳理之后,我们开始着手准备《跳出色情陷阱》这本书。动笔之初。我们便决定从个人自尊和人际关系角度,有条理地、客观地描述色情问题,描写一系列与色情抗争的真人真事,分析他们的观念想法和观察角度,并进一步介绍咨询专家提供的专业建议。\\
为保护客户的个人隐私,我们征聘受访者,也和其他治疗师推荐的志愿者交流沟通。在采访过程中,受访者的勇气和坦诚深深打动了我们;他们愿意将自己最私密的伤痛和快乐和盘托出。希望为他人敲响警钟,以免落入色情陷阱。受访者之一罗布告诉我们,他和大家分享经历,是希望别人不要重蹈覆辙,他自己曾因为在工作场所的电脑上观看儿童色情,当场被撞,结果饭碗砸了,老婆跑了,他也失去了家人和朋友的尊重。罗布感慨道:“色情不仅仅是个人问题,它是一个社会和文化问题。希望我的经历可以帮助那些同病相连的人,让他们不会觉得太孤独,太羞愧,也希望他们能够得到帮助,顺利康复。”\\
我们也和色情成瘾者的伴侣交流沟通。卡伦是一位28岁的美容师,当她发现丈夫的隐藏电脑文件夹中放着淫秽图片时,如同五雷轰顶。“要和他一起生孩子的念头一下子就变得很恐怖”,她说,“看到这种垃圾他竟然会性兴奋?这意味着什么?我再也不信任他了。我们花了几年的时间咨询心理医生,之后我才慢慢信任他,重新考虑和他组建一个家庭。”\\
此外,我们还专访了治疗专家、性瘾专家和宗教咨询师,他们有丰富的治疗色情瘾经验。正是因为他们无私地奉献出独家的治疗方法与策略,此书才更具实践意义。\\
我们衷心希望《跳出色情陷阱》这本书可以帮助色情受害者打破长久以来的沉默,鼓励他们主动寻求帮助,积极克服问题。这本书是你逃离陷阱的最佳指导,文本不带任何歧视色彩,旨在帮助你直视问题。书中提供的一系列信息可以帮助你选择适合自已的治疗方式,具体到何时开始治疗色情瘾。怎样治疗色情瘾。\\
读者将会看到\\
我们选择《跳出色情陷阱》这个书名,正是因为它能生动地表现出色情的本质。同时,它也说明,许多人己经清楚了色情的恶果,却仍然不愿意或者不能够抵抗色情的诱惑;此外,这个比喻也形象地描绘了许多人治疗路途的艰辛:从第一次接触色情,到沉溺于色情,到无力自拔的绝望。最后又成功从色情中走出来。\\
在本书中,你将会阅读色情成瘾者的真实经历,他们或许被深深伤害过,最终下决心要改变生活;在认清形势、下决心改变之前,必是要经历一场风风雨雨的。也许,他们的经历波润起伏,难以置信,不过这也证明,尽管个人心中往往明白此举不妥,却仍轻易地陷入色情这个迷幻陷阱,但到了结局,这些故事都是不乏希冀的。这证明,不论色情问题有多严重,个人还是可以通过科学知识和外界帮助来克服。\\
书中大部分的事例和引用来源于我们对色情观看者及其伴侣的采访。为了保护个人隐私,文中皆为化名,并隐去部分细节;笔者也对受访者的言语进行整理,使阐述更为清晰,以文字形式呈现。\\
无论读者是刚开始接触色情,或者已经沉溺于色情,还是意识到色情的危害并参与到色情成瘾康复项目中,《跳出色情陷阱》一书都具有指导作用,书中还提供了一系列简单有效的治疗策略。本书的目的包括:\\
·判断、评估色情的影响\\
·帮助个人判断何时可以开始戒除色情\\
·学习如何克服色情瘾,控制自身欲望\\
·重树自尊,重建诚信\\
·挽救受色情所影响的人际关系\\
·远离色情,建立幸福美满的性生活\\
在经典电影《人猿泰山》中、当有人陷人流沙时,泰山会荡着一根绳子去拯救他们。这本书是我们像泰山一样竭尽全力,想要拯救你生命的努力。在本书中,你不仅会学到跳出色情陷阱的各种治疗方式,也会找到再不复入陷阱的方法。要做到真正远离色情,是一件非常困难的事,这一点,笔者并没有危言耸听。戒除色情瘾的过程同戒烟戒酒一样,你可能会遇到重重困难,克制不住欲望而康复受挫。即使读者阅读完本书,笔者也鼓励你去咨询专业的治疗师和咨询师,他们能帮助你完善治疗方案,提供必要的指导。\\
读者可以根据自己的需求阅读本书,也可以将本书作为平台,与其他读者交流沟通。本书将会帮助你了解色情所可能引发的所有问题,也能作行动指南,为你现在或是未来的改变指引方向。如果你在咨询心理医生,无论选择的治疗方式是信仰型项目还是十二步治疗项目。《跳出色情陷阱》一书都会为你的康复疗程提供额外帮助。\\
如果你本人不观看色情,但关心的人深陷其中,《跳出色情陷阱》将会帮助你采取主动措施。阅读完本书后,你将会学会如何与色情观进行建设性的对话,在相互信任的基础之上共同努力,解决问题。\\
此书的意义在于我们相信每一位个体都有享受键康的、以爱为基础的性权利,而如今现代化传媒的迅猛发展导致色情过剩,已严重威胁到个体的性权利。或许色情会给你带来暂时的性快感,但最终它会压抑你的性本能,因为它会扭曲个人自然的性态度。破坏个人的性自控能力,危及个人与伴侣之间以爱情为基础的、健康的性关系。早日摆脱色情对你的影响,你才能得到真正的解放,重享健康美满的人生。\\
第一部分认识色情陷阱\\
要学会辨别假币\\
它或许可以给你带来片刻的喜悦\\
但会长久地拖垮你\\
就好像拉在一头放屁的骆驼后面的\\
疲惫不堪的人\\
—14世纪波斯诗人:哈菲兹(在14世纪的波斯,即现伊朗,用假钱交易会受到刑罚,被绳子捆绑起来让骆驼拖着走,骆驼如果放屁,会直接冲到受罚人的脸上,这对受罚人是非常耻辱的。)\\
第一章 色情的隐患\\
托尼是一名25岁的本科生。坐在笔者的在面前接受采访时,他愣愣地望着窗外。轻轻地摇着头,感慨自己是多么快就深深陷入了色情的陷阱。“那时候,我和同居女友发生了一点矛盾。之后她离开了小镇几天,我就决定上网看色情来慰藉自己,在那之前我对色情根本没什么兴趣。那个周五下午,我开始浏览免费的色情网站。周六的时候,我就已经开始色情聊天,到了周日,我就已经加人了一个滥交网站。事情接二连三地发生,似乎一切都是顺理成章的。两个礼拜以后,我女朋友看到了我的网络浏览记录,知道了我看色情网站的事情。我对她撒谎,说我上这些网站是在为政府做调查。一个月以后,女友离开了我。我从来都没想过色情可以颠覆我原本的生活。”\\
43岁的玛丽是一名会计,两个孩子的单身母亲。当她最终意识到色情的影响之大时,大为震惊。“之前我一直觉得色情很无聊,我最多偶尔看看色情杂志和录像。后来,我开始边看色情边自慰,这样子比光看刺激得多。渐渐地,我就好像染上了毒瘾一样,总是想要看色情。因为我不想让孩子们发现,就转移阵地到网上去看色情。事情就一发不可收拾了。在网络上,我随时都可以找到色情信息:鼠标点得越快,看到的色情信息也就越多。一天晚上,我满脑子都是色情,就像磕了药一样。接着,我达到性高潮了,我甚至没有用到手!电脑完全控制了我的身心,色情实在太可怕了。”\\
戴夫是一名年仅50的牧师,尽管他深爱着妻子,却仍然自认为是一名色情成瘾者。他到现在还在与色情做抗争,想尽力摆脱色情的负面影响。因为色情,他丢了之前学校辅导员的工作。“看色情让我体会了最美好的性爱,快感强烈到不行。我不需要付出任何情感,只要选择就好了。在色情网站上,我追求过金发美女伊娃。那一切都非常诱惑,非常神秘,非常酷。不过,我被逮了个正着的时候,就没有那么酷了。色情的另一面是毁灭。我丢了饭碗,还差点跟老婆离婚。如果你一直看色情,那么总有一天,色情会毁了你全部的生活。我认为,现在很多人都还没有意识到色情的毁灭性。”\\
长久以来,观看色情一直是许多人羞以启齿的秘密。大部分女性都害怕承认自己接触色情;而多数男性都认为同性都观看色情,就算不是高频也至少是偶尔为之。毕竟,这是件“爷们儿都会做的事情”,事实确实如此,3/4的色情观看者都是男性。但是不论如何,色情总会让人觉得龌龊,你不会在半生不熟的人面前谈论这档子事,更没有人会在求职面试的时候提。就算你是位男性,而且你老爸早就心知肚明了,你也不会跟老妈提;你可能只会在哥们面前把色情当作笑话讲。\\
要承认自己是色情成瘾者,哪怕承认自己接触色情,都会令人难堪。毕竟,大家总是会认为,沉溺于色情不能自拔这种事不是一般人会做的。没有人愿意把大学生、会计、牧师等各行各业的人看成是色情成瘾的人。而事实上,正是这些普通人观看色情:普通人,像你,我,飞机上邻座的人,你的医生,你的修车师傅。\\
且不论各人不同的情况,托尼、玛丽和戴夫,以及笔者为本书所接触的受访人群之间的共同点在于:了解色情的毁灭后果时,大家都分外惊讶,此前自己观看色情,不外是为了找点乐子。“怎么会呢?”大家会想,“这又不是嗑药,搞婚外恋,而且又不是真的和人发生肉体关系,怎么会导致那么可怕的后果?比如离婚,丢饭碗,甚至失去性欲。”\\
事实证明,观看色情会让人变得盲目。不知不觉中,淫秽作品的强大腐蚀作用会最终扭曲个人的生活。就算我们提高警惕,也难以察觉,色情为何、又是如何侵蚀我们的生命。你知道吗?长期观看色情会改变大脑内部结构。这就是为什么观看色情的人会沉溺于性,产生不健康的性欲望甚至有性功能障碍的原因之一。如果你的大脑已经改变,就很难理智地察觉生活中异于寻常的变化,以及这些变化对生活的影响。\\
色情作为一种“产品”,非常具有挑逗意味,让人难以抗拒。不过它在给人带来性快感的同时,也为你将来的痛苦挖好了天衣无缝的陷阱。色情和其他管制物品——酒和烟一样,看起来很诱人,有时候会给你点甜头,但它最终带来的痛苦会远远超过当初的快感。而与酒精和烟草不同的是,几乎没有人会警告我们色情潜在的副作用。\\
我们总是对色情毫无防备。上学的时候,老师总是教导我们,酒精、毒品、香烟,甚至暴饮暴食都是危险的。但是很有可能,你根本没有听到老师说过色情会带来的影响。色情杂志、书籍、影视作品和网站没有像一般产品那样写着配料成分,也无从得知他们的生产标准和“药力”强弱。你什么时候见过卫生局局长温馨提醒大家,不要接触色情品?\\
本书的目标在于帮助读者更好地了解色情,包括了解其对个人精神、情感以及社交生活的影响。作者希望,无论是读者本人或是读者的亲朋挚友沉溺于色情陷阱,通过阅读本书,你将会了解色情是怎样一步步侵蚀个人的生活。笔者坚信,只要你已经下定决心戒掉色情、恢复美好的生活,书中提供的信息将会协助你一起克服色情的诱惑。本书的另一宗旨在于促进读者用宽容的心态来看待那些被色情所伤害的人,包括读者自己。\\
抵制色情,是一项艰巨而复杂的工程,需要个人花费大量的时间和精力来进行自我反省,与外界交流沟通,并充分了解抵制色情的意义所在。你对色情的科学了解越多,你便越能坦然地与他人谈论色情问题,同时避免自己落入色情陷阱。没有了色情的阴影,生活会变得更加美好:你能够享受健康的性和愉悦的爱恋,同时,也能更好地实现自身的价值。\\
色情难以界定\\
20世纪50年代,一部经典电视连续剧《超人》风靡全美。在剧中,小镇上的居民常常聚集在街道上,指着天空惊呼:“快看!一只小鸟!一架飞机!一个超人!”色情有这么容易定义就好了。笔者认为,色情的概念难以界定。\\
通常人们对色情的认识取决于个人主观上是否喜欢并使用它,或是害怕并认为它应当被禁止。提倡色情的人会将它称之为“无害的视觉刺激”、“自由的言论”或是“视觉享受”,反对者则会称之为“对性的侮辱”、“文化污染”以及“性犯罪指南”。\\
随着观看者环境和视角的变化,色情的功能多变:它可以是一个供人消费的产品,一种天马行空的幻想,一针让人欲仙欲死的性兴奋剂,一种让人抬不起头的罪过,它也可以是一种开放的自由言论;它可以是一款供人消遣的游戏,一种让人迷失的毒品,一个强有力地性发泄竞争者,一种性爱生活的引导,一场夺人眼球的表演,一种功能强大的性捕食工具,一种赤裸裸的性变态,一种情色的艺术,一项违法违纪的罪名,也可以看作是一种黑色幽默。在不同的人生阶段,色情的意义不尽相同;而在同一时期,色情也可能同时具备以上多项功能。人们对色情定义的含糊之处,也是导致色情隐患越演越烈的原因之一。正是因为整个社会缺少对色情定义的共识,公众总是在争论其界限,而忽视了色情的影响,忽视了怎样去发掘治疗性瘾的途径。\\
当然,每个人都有自我意识去判断色情,这种判断能力与个人的年龄、性别、成长和文化背景相关,同时,这也取决于色情对个人精神。情感和道德感的震撼力。这就是为什么伴侣常常为了一些事发生矛盾,比如说,小两口会争论《体育画报》杂志的年度泳装特辑到底算不算是色情。\\
40多年前,美国最高法院大法官波特·斯图尔特(Potter Stewart )对色情的评论被传为佳话:“我不能给色情下定义,但是我看到的时候我知道它是。”他的话让人忍俊不禁,因为这位大法官的视力之差在当时是出了名的。不过他的话说明了一个关键问题:色情存在于观看者眼中,也就是说,你觉得它看起来是色情,那么,在那一刻,对你来说它就是色情。\\
笔者将色情定义为:试图成为或者已经成为个人性欲发泄途径的、关于性爱的露骨作品。这个定于并不拘泥于色情是否暴露,也不局限于色情主题的本质,而是着眼于个人与色情之间的关系本质。性教育材料会提供关于性的科学信息,情色艺术和文学是为了崇尚人体和性,而色情的最终目的就是为了刺激性欲并让观看者对色情本身产生依赖,将之人化并发展出与之的某种性关系。\\
色情的效力之强在于它所能刺激的性兴奋可以即时带来满足感。色情会诱惑潜在的观看者,让人认为,其实只要点一下鼠标或者遥控器,就可以看到色情,又何必要调整自己去满足别人的需要呢?色情让人不脱衣服就可以产生性兴奋!\\
色情信息高速公路\\
“色情”(Pornography)一词源自古希腊词汇“porno”和“graphie”,字面意思是“和妓女有关的写作”。千年之前,色情就已经成为公众消费的商品,随着时间的推进、技术的发展,色情不断演变,形式多样,威力愈强。印刷行业的发展促进了色情业第一次历史性的飞跃;随着新科技的发展,例如相机、电视、录像机、电脑和其他电子产品,包括iPod和手机的广泛使用,色情的内容日益丰富,形式多样,网络中的色情有无数种途径可以进入你的电脑。\\
随着信息高速公路的迅速发展,色情也在不断演变,如今网络能飞速带你进入前所未闻的新领域,而色情始终没有背离它产生时的初衷:与娼妓相关。无论观看者想要多么强烈的性行为,信息化高速公路都会让人满足,不需要顾及其他人的心情和需求。色情就像嫖娼一样,故意回避性行为某些重要的部分。例如真挚情感的表现,伴侣之间的交流,性爱前戏,还有对性安全、性行为后果的顾虑。美国阿拉巴马大学传媒研究者,道夫·奇尔曼(Dolf Zillman)在1989年出版的著作《色情》中如是写道:“色情描绘的是刚相识的、毫无瓜葛又互不负责的人之间发生的事,而且大家很快就会散场,再也不见面。”\\
色情作为一种性兴奋剂,威力与日俱增。比如说,历史上的情色画像就像一条泥路,可以承载一架架载着色情淫秽画像的马车;照相机问世后,色情照片成了沥青公路,更容易让观看者性兴奋。色情电影和录像出现后,色情的道路就开始变得刺激了,而随着色情聊天、现场直播、网络摄像头、专业电子控制性玩具的兴起,色情观看者算是真正驶向了信息化高速公路的大道。这些新技术让色情越来越直观,越来越逼真;如此一来,色情已经超越了性幻想的范围,晋升为观看者在电子世界中的性经历。无论你在信息高速公路上走的是哪一条路,想要去任何地方都轻而易举。借助强大的科技支持,当今的色情才得以与观看者的现实伴侣竞争,霸占观看者的性注意力。\\
当下,如此庞大的群体会有这种“色情问题”,原因之一就是色情无处不在,廉价易得,影响力不容忽视。现如今,网络上的色情信息超过4亿页!奈徳是一名65岁的单身汉,这本该是当外公的年纪,而他却仍然在和色情做着苦苦的抗争。奈徳向笔者倾诉:“过去50年间,我亲眼目睹色情从稀少到过剩。我过去终是要避开熟人,跑到专门的商店和剧院去看色情。现在,你要是能躲开色情,铁树都能开花了。”\\
奈徳的话从侧面反映了一个事实:色情在过去短短的几十年内,已经形成了一个巨大的商业市场。这块大蛋糕不仅被许多小型企业抢占,而且美国主要的传媒巨头也加入其中,从中分羹。全球的商业性情色网站、杂志、书籍、录像、DVD、有线电视等产业的年利润总额超过970亿美元(从2003年到2007年的增长率为70%)。你是不是一直以为运动员是赚大钱的?其实,仅在美国,色情产业的经济效益就超过了所有专业足球、棒球和篮球经销商利润的总和。\\
现今的色情还附加了新的优势:种类繁多,接触便利。从优雅的内衣模特到人兽杂交,应有尽有。你想要什么,就能找到什么。色情内容丰富多样,里面的情景都是平日生活中看不到,在现实生活中无法尝试的。罗波,一位43岁的宣传主管,也曾被色情所困惑。他说:“过去,我喜欢浏览无穷无尽的免费色情网站。点击,点击,点击!哇,还有呢!我还在想,有没有更多?点击,点击,点击!又有了!”\\
观看者被淹没在色情幻想之中时,这个禁忌世界所带来的欢愉是不可抵抗的。随着时代的发展,色情在不断扩展它的极限,变得更加露骨淫秽。柯克是一名48岁的邮局工作人员,去年他成功戒掉了色情瘾,他分享道:“过去30年间,我见证了色情越来越露骨,越来越淫秽的过程。我还是个孩子的时候,第一次看到的艳照杂志甚至都没有露点。我看到的第一部色情录像,性描写还是很隐晦的,既没有特写,也没有无聊的情节。后来,录像中渐渐出现了真枪实弹的性特写,过分关注身体部位而没有情节。在我戒除性瘾之前,一种叫做‘标准色情’的东西开始困扰我;那就是特写一系列你难以想象的性暴力、性变态和性侮辱行为。我根本就不敢相信,那种东西也会让我性兴奋?!”\\
色情能带来的\\
色情的隐患不仅仅在于它难以定义的本质、泛滥的数量以及它那种让人无法抗拒的诱惑力,还在于它能够让观看者的体验变得十分愉悦。下文会详细说明色情是如何给人带来愉悦体验的。\\
1.即刻性兴奋\\
色情的卖点之一在于它会让人性兴奋。色情的初衷在于让你性兴奋,体验欲仙欲死的感觉。从这一点来看,色情类似性爱玩具、春药和伟哥。色情会让观看者的性器官兴奋,血流加快,感官愉悦。不过,与一般性兴奋用品不同的是,色情会长期保存在大脑里面,之后随时可以读取这段记忆来获取新的性快感;而除了你自己,谁也不知道你正在大脑中享受色情。\\
大部分的色情一开场就把注意力放在性生殖器上,不断特写性器官和性行为。色情对于性的单一特写,是为了避免现实生活中可能阻碍性欲发泄的因素:和伴侣争吵后分床而睡,性生活中途手机响了,工作上的烦恼影响到了性爱的兴致,甚至是为了对方着想而产生温柔呵护的情感。布莱恩是一位18岁的小伙子,他放弃了小时候喜欢的冲浪运动,却染上了观看色情的恶习。他告诉笔者:“要说性兴奋,色情一点都不跟你拐弯抹角,它就是要给你当头一震!”\\
色情既不是口服的,也不是外用的。它通过人的感官,比如视觉或者听觉,直接进入身体内部,对观看者的中枢神经,即大脑的兴奋中心,产生直接影响。当你在心中呐喊“快点给我!”的时候,色情就会带给观看者最直接的性兴奋感和真实的满足感。在观看者精神上起变化的同时,身体也在同步变化着:心跳加快,呼吸急促,生殖器兴奋。\\
调查证明,观看色情图片时,男女两性都会有本能的性反应。色情作为一种性兴奋剂,其威力实在不可小觑,就算观看者不喜欢色情,觉得内容让人不适,但在生理上还是会其反应。看到屏幕上的性爱和看到现实中的性爱一样,都会让观看者性兴奋。所以,当你在看色情内容被挑起性欲时,就等于你在性爱现场。\\
色情作为性兴奋剂还会激发睾丸素(性欲荷尔蒙)的分泌。实验证明,当公猴子看到其他猴子在交配时,自身的睾丸素会增加。睾丸素不仅仅会引发性欲和性冲动,也会刺激性动机和侵略性。尽管男女两性都会分泌睾丸素,但男性分泌的平均数量是女性的10~15倍之多。也就是说,就算身体不会直接吸收或消化色情,单单是看到色情图片、听到色情声音就会刺激身体分泌荷尔蒙,欲望火烧火燎。\\
詹姆士是一名23岁的大学生,他花在色情上的时间远远多于参加社交结识新朋友的时间。他说:“色情纯粹就是兴奋剂,它就是要让你兴奋!你都不需要去谈恋爱,光是看就会让你很爽。”另一位大学生凯尔补充说:“色情无处不在,非常有效,比单单手淫更有意思得多。”\\
2.像毒品一样给你绝顶的快感\\
读者可能在想:“得了吧,色情怎么可能会像嗑药?我不吸它,不喝它,也不会注射它。”但事实是,色情对人体和大脑产生所能产生的影响绝不亚于可卡因、兴奋剂或者酒精。色情会刺激前脑内侧部分,此处常被称为“享乐通道”,即神经递质多巴胺的感受器。\\
人性兴奋时,大脑会释放多巴胺。其他让人兴奋的活动,如亲吻、亲热、抽烟和嗑药,都会让大脑产生这种物质。色情会刺激大脑分泌大量多巴胺,让人产生嗑药以后的快感。一些研究者将这种快感比作吸食可卡因。山姆在现实生活中十分内向,他就是这样形容色情的:“沉溺于色情的感觉,就好像一道激流闪电击中身体,而好处就是,我随时都可以重温这种快感。”\\
观看色情也会增加大脑分泌其他兴奋性化学物质,如肾上腺素、脑内咖和羟色胺。而问题在于,色情在使大脑产生过剩性化学物质的同时,也大大削弱了人体自身在正常情况下的分泌能力。这就是为什么色情观看者需要不断追求新刺激来满足自己不断膨胀的性欲。泰德,一位30岁的证券经纪人,跟笔者说:“不论看过多少色情,我总想看的更多。”\\
如此一来,色情可以产生类似镇静剂或麻醉剂的效果。如果观看色情的同时还在自慰,就可以得到性高潮。我们都知道,性高潮带来的愉悦的感觉,舒缓痛苦,让人得到无限释放。\\
色情的功能多样,它能够带来愉悦感,让人放松心情,甚至能让人逃避现实中的痛苦,让人上瘾。久而久之,人就需要依靠色情来逃避现实中的痛苦。有色情瘾的人,往往是欲火焚身, 一门心思就渴望色情,色情性爱成了人最大的欲望和需求。如果人经常依靠色情来获取快感,那么一旦停止观看后,人就会变得焦躁、压抑,甚至失眠。这种症状,和在戒烟戒酒戒毒时的反应一模一样。如果个人需要治疗性瘾,平均来说,仅要使身体的多巴胺感受器恢复到正常状态就需要18个月的时间。\\
观看后.人就会变得焦躁、压抑,甚至失眠。这种症状,和在戒烟戒酒戒毒时的反应一模一样。如果个人要治疗性瘾,仅要使身休的多巴胺感受器恢复到正常状态就需要18个月的时间。\\
3.力量之旅\\
大权在握的感觉让人兴奋,尤其是控制的还是人类最基本、最原始的性,而色情就能让人拥有这种快感。把色情当作乐子,人就会产生幻觉:觉得自己很好很强大,一切尽在掌握之中。提姆是一位年近50的图书管理员,对他来说,色情真的燃烧起了他的小宇宙:“在色情里面,我就是个妻妾成群、左拥右抱的爷们儿。我只要出现就可以了,所有的女人都会为我表演。她们大跳艳舞,卖弄风情,暴露身体。我控制了她们所有人,她们只为我一个人表演。”\\
史蒂文是一位29岁的精神健康义工,他很赞同提姆的话:“我喜欢色情最重要的原因就是:我才是掌控性行为的人。我都不用去求别人跟我做爱,就可以随心所欲地在想要的时候,用我想要的方式,挑选我想要的人,享受我想要的性。我不用顾忌任何人的感受,这就是因为我一个人表演的秀。”\\
在电脑、电视、成人店里大肆寻找色情时的感受,堪比人在运动,比如打猎时产生的权力感。跟打猎一样,先是锁定,包抄,瞄准,然后拿下;你购买或者下载色情也是一样。一些色情观看者认为,随着性欲而来的猎奇感和控制欲望比高潮还会让人满足。色情观看者能够找到免费的色情资料偷偷观看,再想方设法地掩盖,整个过程都会让色情观看者感到自己的能耐。\\
很多人喜欢色情是因为色情让他们成为了窥淫癖者(喜欢窥视他人性行为的人)。这值得人思考,因为这种行为在现实生活中是非法行为。窥淫癖者通过观察那些毫不知情的人赤身裸体进行性爱得到性兴奋,看到的都是本不该看到的事。你能看到别人在做的事,而别人却无法回看你、阻止你。在这样的关系中,窥淫癖者完全占了上风,因为他们有特权,可以随意指指点点,只要用鼠标点击一下,就可以拒绝不喜欢的内容,而自己却不用被指责。\\
当然,人们都清楚色情带来的权力感无法转化为现实。在屏幕、杂志上性感妩媚、极具诱惑的对象,在现实中都无法拥有;事实上,个人也根本无法控制色情中的性情景和幻想,制作人才可以控制色情节目的录制过程。悲哀的是,部分色情观看者就是愿意在那么几分钟,最多几小时的时间里臆想自己就是权力掌控者。乐意享受掌控别人的快感,甚于跟现实中的伴侣亲近温存。色情给人带来的掌控感,会导致以自我为中心的性观念,扭曲个人在现实生活中的行为举止。\\
4.赌博般的兴奋\\
观看色情的本质上与赌博非常相似。如今色情都是通过高科技设备传播,人们只要点击一下鼠标就可以得到源源不断的色情资源。在寻找心仪的色情网站、最渴望的性活动、最中意的幻想对象时,你的心情与你在拉斯维加斯赌城玩老虎机想要发大财的状态时一样的。商人奈吉尔说:“观看色情的时候,我心里很清楚自己想要的是什么类型的色情内容、什么类型的性爱,我知道这些东西肯定找得到。我喜欢去搜索最劲爆的内容!”\\
不过,就算你再怎么拼命搜索幻想中的完美内容,你也只能看到一张所谓“完美”的照片或是性爱动作。这种情况称为“间接性奖励机制”,它的目的就是要吊住观看者的胃口,就像赌博场里面的游戏机一样。正是这个机制的存在让你能轻易地在网络上浏览到色情内容,而且观看色情时你会觉得时间飞逝。\\
这个“间接性奖励机制”带着不固定比率时制,你永远不知道要浏览多少内容才能找到自己想要的色情信息。这个过程是随机的,而这就是色情改变个人行为最大的机制体现。在科学实验中,鸽子采用这种随机性机制,执着地试图啄一颗可能存在的大米,而看点被饿死。\\
在下载色情图片或者打开色情网站的一瞬间,你心中预期的色情内容就会引发大脑分泌让人兴奋的化学物质:多巴胺。正是因为这种类似于赌博的快感,再加上它可以使人性兴奋的本质,使得色情的威力愈加强大。网络色情数不胜数,只要你想要,色情就会通过各种形式的高科技设备出现在你的面前,而你不仅仅会被色情内容所影响,也会被色情的传递机制所左右。在过去,你邮箱每月只会收到一起色情电子杂志,那个时候色情的间接性奖励机制远远没有现今的威力。\\
5.风流韵事\\
你会不会爱上色情?至少色情能让人产生性欲,这是千真万确的。笔者采访的诸多调查对象认为色情能够成为一种可选的性欲发泄途径,虽然他们不会为色情写情诗、唱情歌,但他们总是会用恋爱式的词汇来形容色情,像“真是让我性欲大开啊”,“我跟它在一块儿的时间总是过得特别快”,“我真是迫不及待想要见它啊”,还有“它让我体验了最美好的高潮”。\\
色情看起来不过是和虚拟对象做爱,事实上它给人的感觉要真实得多。虽然色情杂志和成人录像中的男男女女不过存在于一个纸质平面或者屏幕上的二维空间里,但是一边性幻想一边自慰所产生的快感,丝毫不亚于在现实生活中发生关系时所产生的快感。\\
观看色情会让大脑产生性兴奋的化学物质,比如多巴胺和睾丸激素等引发性兴奋和快感的化学物质。在现实生活中如果人恋爱了或者动情之时,大脑也会分泌这些物质。再者,高潮时,大脑会释放荷尔蒙,包括催产素和加压素,同时,这也会使人习惯于依赖色情来达到高潮。借助色情得到的高潮次数越多,你就越容易在情感和性欲方面对色情产生依赖性。\\
色情和搞婚外情一样,会占用你大量的时间和精力,让你无暇去照顾已有的伴侣。观看色情的人常常要偷偷摸摸、鬼鬼祟祟的,这和出轨的表现没有什么差别。如果伴侣质问,色情观看者就会忙不迭加地否认、撒谎,试图掩盖自己的罪状。在不知不觉之中,人们更加沉溺于和色情的恋爱,而忽视现实生活中的人际关系。当然,如果人总是观看色情,在做爱的时候脑海中会不由自主浮现出色情的影像,这样一来,人就很难和现实生活中的伴侣享受亲密无间的情愫了。\\
弊大于利\\
毋庸置疑,色情充满诱惑、威力无穷,它能挑逗人的性欲,让人性欲得到满足,让人逃离现实烦恼,让人感到大权在握、欲壑难填。但是观看色情所导致的问题是无穷无尽的,而且一切都是在不知不觉之中发生,难以察觉。直到问题发展到极其严重的地步,个人才会惊觉色情的影响竟是如此之大。在本书以下章节中,笔者将会详细描述色情:\\
·与个人的价值观、信念、人生信条相悖\\
·使人变得虚伪,在与他人交往的过程中多少会遮遮掩掩\\
·会使伴侣难过,并与伴侣竞争成为个人的性欲发泄渠道\\
·损害个人的身体、精神健康\\
·让你逐渐失去对伴侣的吸引力\\
·会导致性欲、性功能问题\\
·毁掉个人健康的性欲\\
·会导致一系列影响家庭和工作的问题,引发违法行为,损害个人的精神健康\\
笔者坚信观看色情的风险要远远大于它短期之内给人带来的蝇头小利,因为笔者在调查和采访过程中见证了太多负面实例,证明色情足以阻碍个人与伴侣之间的情感关系。麦克斯才刚刚20出头,就已经意识到问题的严重性:“色情会扭曲正常的性观念。在色情的世界里面根本没有真实的满足、平等和相互尊重的感情。它会让你变得只知道索取,不知道付出。色情描绘的根本不是我们应该追求的健康性爱,那些都是浮云。你根本不能找到任何真实、长久的性快乐。”\\
无论色情再怎么强大,也无法比拟真实生活中正常性爱时产生的美好感情,而这些真实的感情才会帮助个人树立自信心,促进个人和伴侣之间的亲密度。没有什么比跟一个大活人谈恋爱,卿卿我我更美好温馨的事了。即使个人已经对色情产生了依赖性、习得了色情倡导的态度,但只要你痛下决心,就一定可以戒除这不良的习惯。请记住,一切为了戒除色情瘾而付出的努力都是值得的。\\
很多人告诉笔者,戒除色情瘾以后,自己的性生活更加美满,道德感愈加健全,社交生活更加丰富,精神上更加充实。34岁的寇里曾经被色情害得吃牢饭,他告诉笔者:“现在我终于明白了,色情根本就不能让人感到满足。在现实生活中和伴侣亲热得到的幸福感,远远超过看色情得到的快感。虽然从某些方面来看,和现实中的伴侣亲热需要付出更多,不过显然回报也是更好的。健康的性生活会提高生活质量,不会像色情那样只能给你带来短暂的快感。色情根本没有什么所谓的长远利益,只有长远的痛苦。没有了色情,我现在的生活更好了,如果当初我没有沉溺于色情,我就不会损人又害己了。”\\
第二章 色情初体验\\
讲到色情,大家通常会直觉地认为它是供成人消费的。毕竟,那些卖色情杂志和录像的商店都叫“成人商店”,电视播出色情秀之前总会提示说“少儿不宜”。不过事实却令人吃惊:大部分人初次接触色情的平均年龄是11岁!而几乎每个人在18岁之前就已经过了把眼瘾。\\
如果现在的你有色情瘾问题,很可能这个问题可以追溯到你初次接触色情时的情形。童年时个人性格的成形时期,孩子非常容易受到不良因素的影响,而这段时期的经历会塑造个人的人生态度和行为举止,从孩童到青少年的这段过渡时期则是个人性欲望和性取向的形成时期。\\
布莱恩的经历具有一定代表性,证明当今社会孩童往往过早接触到色情。布莱恩未成年时就已经深陷色情陷阱不可自拔,他如今是一名27岁的商人,成家立业,育有一子,但在过去4年间他都在苦苦地与色情瘾作斗争。\\
布莱恩的自述\\
我初次接触色情时,只有6岁。那天我和哥哥一起在当地小学的停车场骑车,调皮捣蛋。那是暑假的一个周末,除了两个混混在单杠那儿玩之外,学校空无一人。我和哥哥起初也没有注意到他们,直到后来他们把我们叫了过去,说是要给我们看好东西。\\
其中一个人打开了一本《阁楼》杂志,让我们看里面的裸体女性。当时我并没有觉得惊讶,但这段回忆却十分清晰,直到现在还深深烙印在我的记忆中。我当时觉得有点奇怪,但又夹杂着快感。那时我还不知道什么叫淫秽,因为年纪太小,根本不懂那种内容,也不了解自己微妙的情绪变化。\\
我和哥哥回家后,把这件事告诉了妈妈。她火冒三丈,一路狂奔到学校操场,想要找到那两个浑小子,但他们已经走了,走前还把那本色情杂志撕成一页一页,撒在地上,似乎就是想要大肆宣传。妈妈把所有的纸一张张捡起来,全都扔到了垃圾桶。我当时不知道为什么妈妈反应如此之大,不过就是裸体而已嘛。对当时的我来说这件事没有什么大不了,但是妈妈反应过激,而且事后也没有给我们任何解释,这让我印象深刻。\\
之后,我对于色情的好奇心暂时消失了。关于性的想法,被其他事情挤到边角旮旯去了,我忙着和邻居孩子打成一片,一起骑车,玩气枪大战,时常参加各种运动,还要上学。直到7年之后的那个夏天,我才开始主动看色情。\\
那个暑假特别闷热。父母都去上班了,留我和哥哥看家,还要干一大堆农场的活儿。我们俩烦得要命。为了表示抗议,我们就在家看哥哥从某解码频道录制下来的色情录像来打发时间,有时候,我们也去堂哥家里看花花公子频道。\\
妈妈对操场事件的态度就已经说明,色情在家里是个禁忌。我们知道看色情的事一定要隐瞒得天衣无缝,万一被老妈发现就惨了。不过我还是照常看,一方面因为哥哥说这个很棒,另一方面我也已经到了性冲动的年纪了。刚开始,我还没有学会自慰,但是很快我就无师自通了。我的第一次高潮就是在看那个色情录像带时来的。\\
很快地,我们就不再抱怨父母布置的家务活了,只一心盼着父母出门。我满脑子都是想要看更多色情的念头,渴望更多的性幻想。就算父母在家,我也会在脑子里面偷偷回味录像中的画面。我开始每天频繁地自慰,已经有强迫症性质了,但是那时的我,很傻很天真地不知道自己的行为会带来怎样的后果。我只知道,这件事做起来很爽,只想着做的次数越多越好。\\
这就是我染上色情瘾的起点,在那之后,我对色情的欲望有增无减。只要有机会,不管是限制级杂志还是录像,我都会千方百计弄到手。如果我在朋友家玩,而且他的父母正好不在的话,我就会怂恿朋友去找他老爸的色情珍藏品。等到我父母发现我沉溺在色情中的时候,我已经病入膏肓了。他们再怎么训斥,我都当作是耳边风。表面上,我装的可怜兮兮,满口认错,保证以后再也不会看色情。但是第二天,我就会迫不及待地想要看更多更刺激的色情了。\\
笔者点评:那天早上,当布莱德和哥哥骑车去学校操场玩的那一刻,他绝对料想不到那天与色情的第一次亲密接触会成为他青少年时期性欲的焦点。色情导致他丧失诚信,最后,还几乎毁掉他的婚姻。\\
孩子是很容易接触到色情的\\
笔者调查发现,一个人往往在童年或者青少年时期就开始观看色情了。每个人初次接触色情的情景各不相同,但是对很多人来说,初次接触色情时年纪过小。\\
今年32岁的泰勒还记得他第一次看色情时,他只有5岁。那时,他在一个朋友家玩。“朋友的父亲把一本色情杂志随随便便地放在起居室的桌子上面,似乎完全不在意这种事。我到现在都记得,当时我在杂志里看到了很多大胸脯的女人照片,裸体的男人和女人在做爱,”泰勒说,“我觉得那本杂志非常神奇,当然啦,那个时候的我完全不知道人体的性机制问题,但我就是被那种新奇又神秘的感觉给吸引住了,没有什么人在5岁的时候能够天天看到这种东西。\\
吉尔是一名34岁的矿工。他9岁的时候,得了流感在家养病。某天他在父亲的抽屉里发现了一本《阁楼》色情杂志。“我当时并没有刻意在找什么,知识偶尔发现了那本杂志。我随便翻了翻,感到很好奇,又有点疑惑。就是从那个时候开始,我改变了对爸爸的看法。我把杂志挪了一个地方,想让爸爸知道我已经看到了这本杂志,但他都没有提起这件事。”\\
翻出父母的色情收藏品是孩子们初次接触色情的途径之一。对孩子们来说,找到父母藏着的色情就好像找到宝藏一样。色情肯定是稀奇又珍贵的东西,不然父母怎么会这样大费周章藏起来?找到了父母藏起来的色情,一看之后,果然是很有料的东西,劲爆无比,抢人眼球。但同时,大部分的孩子都明白,自己不应该跟父母说这件事,一旦被发现的话,肯定会被狠狠地罚一顿,然后再也看不到这些色情品了。为了保住自己的战利品,孩子们自然会守口如瓶。一旦人染上了色情瘾,就只能一辈子用这种偷偷摸摸的办法看色情。\\
孩子们接触色情的另外一种途径就是通过亲朋好友的介绍。当长辈或者孩子崇拜的人兴奋地给孩子秀色情时,孩子们会觉得很刺激,很兴奋。这就是所谓的“传染性兴奋”,我们小时候总是会刻意模仿偶像或者竞争对手的行为以及他们对事物的态度。\\
贾斯汀就是通过亲人的介绍接触到色情的。他9岁的时第一次看到的色情是叔叔让他看的。“我叔叔照看我们时,把我和两个弟弟放在沙发上,坐在他身边。我们还以为他要讲故事,但是他拿出了一本《花花公子》杂志,让我们看里面的图片,我的眼珠子都要蹦出来了。叔叔弗莱德非常兴奋地翻着书页,我们在一旁看的也是热血沸腾。我当时就好像酒鬼第一次喝到了酒一样,激动得不得了。后来叔叔只要来照看我们,就常常给我们看色情杂志。\\
另一个实例,罗布,43岁,他小时候第一次看到的色情图片是哥哥给的。“那时候我7岁。哥哥让我坐在床上,大声宣布说:‘做好准备,它会燃烧你的小宇宙!’然后他拿出一张图片,放在我面前。那是一张半裸女人的图片,她真的是个大美人,我到现在还记得她的姿容,记忆犹新。”\\
大部分受访者第一次接触色情时的情形,不外乎是以下几种情况:和朋友一起看,亲人介绍看,看到父母藏着的色情品。而如今,大部分孩子第一次接触色情都是通过互联网。调查表明,每年都有1/4的孩子在上网时会不经意地看到色情信息。8岁到16岁之间的孩子中,90%都成人在网上看过色情。另一项调查表明,半数以上的孩子在初次接触色情之后都会念念不忘。也就是说,本该是为成人服务的内容,却导致孩子们过早地接触到色情,这已经成了整个社会的一场大灾难。\\
初体验色情时的个人反应不尽相同\\
有些人说:“色情有什么好看的吗?!”而另一些人觉得:“色情实在是太有意思了!”吉尔,前书里提到过的矿工,就是这样说的:“看着那些美貌的、魅惑的女人照片,就让我有一种‘世界又重生了’的感觉“。\\
大部分受访者称,色情的初体验是又刺激又劲爆的,只有很小一部分人说自己感到不适。小部分受访者用“雷人”、“困惑”这类字眼来形容色情初体验。贝奇,19岁的大二女生,在刚上高中时第一次接触到色情。“太让我震惊了!”我还感到困惑,抑郁消沉。第一次看到这种对女性实行变态性暴力的特写,让我失望透顶,因为色情彻底侮辱了女性的尊严。父母跟我说,性是很美好的事,但色情里面的性不是。我想要和男人亲热,但是我不想要对方把我当做色情角色一样来侮辱,这让我完全无法接受。”\\
初次接触色情时的反应,取决于各人的特定因素:年纪的大小;组团还是独自一人;是自己翻箱倒柜找出来的,还是别人为了炫耀而秀出来的;看到的是什么内容,等等。你懂的,一个6岁的孩子看到重口味色情录像时的反应,跟12岁的孩子找到了老爸收藏的色情图片时的反应,肯定是不一样的。性别也会影响个人的态度。与男孩相比,年幼的少女初次接触色情时,更容易情绪低落,甚至害怕到失声痛哭。\\
其他影响个人对色情态度的因素还包括宗教信仰、家族期望和社会环境。大家同样都是13岁,但A的家庭会毫不避讳地公开讨论性问题,而B的家庭非常保守,不愿提及任何与性有关的问题,那么两个人对色情的态度也将会是截然不同的。\\
33岁的艾利克斯的经历说明,宗教信仰和家族期盼如何影响个人对色情的看法。艾利克斯觉得色情这个禁忌世界的快感可以帮助自己脱离童年时苛刻的禁欲教育。“我出生在一个虔诚的门诺派教徒家庭,你听说过门诺派的一个笑话吗?为什么门诺派严禁婚前性行为?因为这会让人想要跳舞!所以,色情这种词是根本不可能出现在现在家庭里。9岁的时候,我从好朋友的哥哥那里买了20本《花花公子》之类的色情杂志,用做作业的时间偷偷地看,那面红耳赤、偷偷摸摸的感觉实在是很刺激。”\\
父母的态度会导致情况恶的化\\
还记得前文中布莱德和哥哥去学校操场玩,两个混混给他们看色情杂志的故事吗?事后布莱德和哥哥立刻冲回家告诉妈妈这件事,其实他们有很多困惑希望得到解答,但妈妈并没有冷静地给孩子分析这件事,而是直接发火,去操场找那两个混混算账。因此,布莱德孩童时代,从来没有机会向长辈询问相关的问题,也没有从长辈那里听到对此事的任何评价。\\
布莱德现在已经成家立业,做了父亲,自然也就能理解当时母亲的反应了:竟然有人拿那种垃圾给这么小的孩子看?她自然是很生气,也担心那些混混会引诱或是猥亵儿童。但是,她不理性的态度让布莱德更加困惑。\\
那时的布莱德和其他同龄儿童一样,对人体和男女之差非常好奇。然而,布莱德母亲的过激反应让他对自己本能的好奇心产生了羞耻感和负罪感,再也不敢向母亲提起任何有关色情的事。当布莱德到了青春期并沉溺于色情时,父母才开始关心家庭性教育,可惜,木已成舟。长辈和孩子们之间交流的渠道已经堵塞了。布莱德已经开始沉溺于色情,在被逮个正着的时候又满口谎言。\\
大部分受访者最初接触色情时,父母的态度也几乎如出一辙。如果孩童想要尝试跟父母谈论这个问题,父母的反应让他们很快就意识到,性话题是越界的。大部分的父母都像布莱德的双亲一样,不知道如何培养孩子们的科学性观念。所以,这些父母就采取闷声发大财的态度,绝口不提任何有关色情的事;而有些父母不愿意谈及色情问题,就是害怕孩子发现自己也在偷偷地看色情。总的来说,大部分家庭中,父母对色情的态度就是“不问,不说”。\\
这样一来,问题就开始恶化:父母不能帮助孩子们有效地解决他们对色情的困惑。这将会引起一系列的连锁反应:原本孩子对父母毫无保留的真诚态度、父母与孩子之间原本亲密无间的关系开始出现裂痕,孩子们会不必要地为自己的性意识感到获耻,而这更容易诱发强迫性色情瘾。\\
看到这儿,读者可以停下来回想一下自己的色情初体验。下面是笔者设计的问卷,旨在帮助读者正确理解自己初体验时的情况以及影响。\\
色情初体验调查表\\
1.第一次接触色情时,你多大?\\
2.当时的环境怎样?你是独自一人还是和他人在一起?\\
3.你接触的是什么类型的色情?描绘的是什么场景、什么类型的性行为?\\
4.你当时的第一反应是什么?你的感觉怎样?比如:兴奋、迷惑、焦虑、羞耻、好奇、害怕、产生性欲、恶心、愤怒、高兴、低落,等等。\\
5.你有和其他人谈论过自己的色情初体验吗?如果有,倾诉对象的反应如何?如果没有,独自背负着这个“不能说的秘密”,你感觉如何?\\
6.你的色情初体验会让你想要观看、搜索更多的色情吗?如果是,通过什么样的途径?\\
7.你的色情初体验,给现在的你留下了哪些遗憾?如果可以重来一次,你希望事情可以怎样改变?比如:当时的年纪过小、色情内容不适、当时的环境等方面。为什么?\\
孩子因为各种原因继续观看色情\\
无论初次体验色情时的年龄有多大,除了小部分人称自从第一次接触色情之后就开始上瘾以外,大部分人称,自己会根据时间、环境的变化。时不时地选择去观看或不观看色情。不管你是七零后,八零后,九零后,还是零零后,如果你定期观看色情,多半是因为你发现色情或多或少可以满足你童年时期的这些正常需求:\\
1.学习性知识\\
2.群体归属感\\
3.性开放和性快感\\
4.消除精神压力\\
下文会逐一分析每类需求,并说明每项需求是如何让人自童年就患上色情瘾的。\\
1.“这真是太不可思议了!”(学习性知识)\\
孩子们天生就好奇性是什么,裸体是什么样子,自己是从哪里来的等生理问题。到了青少年阶段,人体分泌的荷尔蒙激增,这时候人不仅仅是想要了解性,而是必须要了解!而且是现在就必须要了解!\\
事实上,几乎所有学校和家庭的性教育都像是蜻蜓点水,根本无法满足我们的求知欲,不能让人了解性关系怎么发生,人又是怎么享受性的。光学习如何禁欲,如何避孕,如何预防传染病都远远不够。小孩子就是好奇,想知道人在进行性行为时是什么样的,以什么样的姿势,会带着什么样的表情,发出什么样的声音。\\
凯文是一名34岁的银行投资家,他14岁的时候在神学院修道,学习成为牧师。他越是得不到性信息,就越是渴望了解。“电影之夜,每当镜头里出现接吻或者亲热的镜头,主教都会用手帕把屏幕遮住。他这种大张旗鼓的遮遮掩掩,激发了我的好奇心,更是想要看色情镜头。到后来,我不可遏制地想要看‘不该看’的部分。”\\
如今,色情的确是唾手可得。与传统性教育相比,色情利用真实的生活场景,利用活生生的影像,逼真的音效,来展现更真实的性行为。大部分受访人群都把色情当作性启蒙教材。赫克托是一名医学专业学生,他说:“我10岁左右开始看色情,还和哥哥一起去他朋友家看色情杂志和录像。其实对我来说,色情就是性启蒙教材,我看色情就是想要搞清楚性到底是什么,怎么做,用什么样的姿势。我知道学了这些,迟早有一天会用得上。”\\
赞恩也是一名大学生,过去他做的“家庭作业”很特别。“早在高中的时候,我就在网络上发现了裸体女人,我就像发了疯一样痴迷色情。我当时就是一个性欲旺盛的青少年,对性很好奇;另一方面,我在网上总是能找到新鲜刺激的内容。有了初次体验,每天我就会花几小时本该用来做作业的时间去看色情。”\\
帕蒂是一名55岁的女性,她小时候缺乏性教育资料的时候,色情就是她的性启蒙老师。“我那时候经常帮人照看小孩。有天晚上,我把孩子哄上床之后,在雇主家客厅沙发背后的壁橱里找到了一大摞《花花公子》的杂志,那时候我不过12岁,没人敢在我家里看这种杂志的。我看了那些杂志才知道,原来还有这么一个我一直不了解的性世界!从此,我就专门挑这种家里有色情的家庭当保姆,乘机去看他们的色情杂志。”\\
人们在年少无知时,很难察觉色情中描绘的性根本就是误导性的,在某些情况下,根本就是荒谬到可笑!尽管色情会让我们了解身体的构造,但这样说吧,色情里百的身体比例都不正常。哪个男人看到男艳星,不会哀叹上天如此偏心?或者看着自己的伴儿在心中哀叹!为啥我媳妇儿就是个“飞机场”?\\
你相信吗?\\
正常男人勃起后阴茎平均长度是5.8英寸\\
男色情演员勃起后的阴茎平均长度足是8英寸\\
85%的女色情演员都隆过胸\\
100%的色情杂志折叠插页都是PS过的\\
色情明星经常要进行清除毛发手术、生殖器整容手术和抽脂手术。——数据来源:《男性健康》2004年第3期,《给聪明女孩看的色情指南》\\
可能,色情最误导人的“教育”在于,它会让孩子们以为色情中的性是最完美的,随意的一夜情是美好的性爱,也没有人在事后会感到不满,伴侣可以接受并且享受任何一种形式的性行为。而只有在人成熟之后才会惊觉,这根本不现实。\\
2.“爷儿们都这祥”(群体归属感)\\
对很多人来说,色情会拉近自己与朋友圈或者家庭成员的关系。在调查过程中,笔者听闻许多和亲朋好友共享资源、共同观看色情的例子,但是这种现象往往仅限于单一性别团体。就像前文中提到的布莱德,他花了整个署假的时间和哥哥一起观看色,后来还去他同学家下载色情内容;也有很多人告诉笔者,他们和其他人一起看过色情杂志和录像。如今科技发达,和朋友家人一起分享色情也变得起来越方便:一个短信就能把色情网站的网址转发给朋友,相互之间也可以把装满色情图片的移动硬盘借来借去。\\
年少时,分享这种禁忌内容是为兄弟两肋插刀的表现,可以帮助个人加强社交纽带。当个人和其他团体成员有着共同之处时,就会产生团队归属感;如果这个共同之处还不能被大人知道的话,会加强团体亲密度。色情在团体内的传播,就好像和朋友一起分享香烟、啤酒和大麻,这样子同伴会觉得你很酷,这种情况在男性团体中尤为常见。\\
伊万是一位22岁的汽车修理师,他认为色情拉近了他和一帮好哥们的关系。“我们7年级的时候,有个同学在学校附近的超市偷了一本《阁楼》色情杂志。我们好几个男生一起看完了那本杂志。到了高中的时候,我们就在朋友圈里面分享色情录像,还大开玩笑,评头论足.。但是我们从来没有认真讨论过自己的真实感受,这本来就是要给生活加点搞笑、劲爆的料,尽管这些内容变态、荒谬到了恶心的地步。\\
我们看色情不是为了性兴奋,只是想找个聚会的理由。我要是知道了这些哥们有共享的色情,就会跟他们预定,不过我从来不跟外人提起我跟这一群人混在一起。现在再回过头去看.我们那时候对女人裸体的迷恋,不过是一种所有人都上贼船,所以大家都要万众一心的感觉。”\\
现年38岁的多恩也是通过色情来巩固与哥们的关系。他12岁的时候就和同学一起看他爸爸的《花花公子》杂志。“我们翻过每一本,看完之后热烈地讨论,评头论足:‘天哪,她真漂亮!’,‘老兄,很刺激吧?’或者足‘如果是我的话我就会这样子做。’我整个童年都在和朋女一起看色情。我对色情的依赖越来越重,不过这也让我与父亲更加亲近。”\\
大部分男性受访者都承认色情加强了自己与他人的纽带。他们沉溺于色情,是因为他们认为”爷们儿都会看色情”,而且还会让他们显得成熟。寇里现在34岁,而他在有生之年中都在和色情做苦苦的斗争、他的经历也是大同小异:“过去我和朋友总是一起熬夜喝洒,别人看整夜的体育节目,我们就看整夜的色情录像。我心里其实有点内疚,但是一想到色情无处不在,没有男生不看的时候,也就坦然了。”\\
拉尔夫是一名36岁的修理师,色情拉近了他和父亲、兄弟的关系。“我出生在一个大家庭,我有五个哥哥,爸爸和哥哥们都看色情。《阁楼》和《花花公子》这类色情杂志在厕所地上随便乱放。家里有一个罐子,里而的色情资源都是共享的。我15岁的时候,哥哥们就开始让我看他们收藏的色情录像了。”\\
女性一般不会为了改善闺蜜间的友谊而一起坐下来看色情。就算有这样的事,她们的态度与男性也是大不相同的。蕾西是一名27岁的理发师,她回忆说:“在寄宿学校上学的时候,我们一群女孩子窝在一起看色情录像,其实也就是抱着玩玩的心态。刚开始看我们还觉得很搞笑,但后来就觉得非常恶心,没过多久,我们就开始抱怨里面的男主角太猥琐,性交太粗野,对白也很没有营养。”\\
如此看来,女性通过集体排斥色情来促进集体亲密度。有意思的是、女孩子之间可以公开地谈论、谴责色情;而男孩子之间则达成了一种默契;永远不能说色情的坏话,就算你真的觉得恶心,你也不能说出来。\\
3.“我有这种感觉是正常的”(性开放和性快感)\\
年龄在8-18岁之间的孩子,最怕被人当作“怪物”或者“异类”。所有人都想要合群,这样才显得自已正常。青春期来得懵懵懂懂,许多人都以为自己有点异样,特别是缺乏家长和其他成年人正确指导的情况下。孩子不知道自己的性冲动和性幻想都是正常的、自然的。\\
色情对性的描绘十分露骨,表达性欲和性活动的方式也是“一切皆有可能”。色情就像是孩子们的伊甸园,在这里,他们可以自由自在地探索、尝试性体验。情色艺术摄影师兼作家大卫·斯坦伯格评论“色情仍然是性开放者公开、自由地表达性观点的方式,而对美国这个仍以清教徒思想为主流的社会而言性仍然是比较忌讳的话题,色情也算是比较激进的言论了。”\\
色情的潜在信息是,有了性欲望就应该身体力行去实践。去享受性高潮,这些都是名正言顺的事。色情不仅仅允许青少年发展那种萌芽期的性欲,更鼓动个人去享受随之而来的性快感。总之一句话,色情诉我们:有性欲是很正常的事。\\
对成长过程中性欲被压抑的人来说,色情就是一种解放。罗里已经60岁了,他是所有受访者中最年长的一位。他说:“在我成长的过程中,一提起性欲,我就觉得自己卑鄙无耻,自责得很。但我对裸体女人喜欢到了极点,看着她们的图片,自己的身体就会性兴奋起来。《花花公子》这本杂志让我知道,色情其实没什么大不了,成熟稳重、教育良好的男人都干这事。心理有了这样的转变后,我就不会因为有生殖器而感到猥琐。我不用猥亵任何人,只要看着杂志插图就可以性幻想了。我老是开玩笑说,我到了20岁的时候,才知道不是每个女人的肚脐上都带着环。”\\
许多男同性恋者也称,同性恋色情让他们不会因为自己的性欲和性冲动而感到羞耻。跟异性恋相比,同性恋者了解相关信息的机会更少,所以对他们来说,色情可以在很大程度上弥补这个空缺。艾伦是一位38岁的厨师,他告诉笔者,在他第一次看到GV之前,体内不断觉醒的同性恋性意识让自己一直觉得不合群。“色情让我第一次了解男人之间的性爱。我从小成长在一个保守社区里、除了看色情,我根本没有其他途径去了解我的性特征,也不知道该怎么和我的另一半亲热。”\\
先撇去性取向不说,色情能让年轻人放纵地去享受刚在萌芽阶段的性欲望。在成长时期,自慰是一种很普遍的性发泄方式,而以高潮为目的的自慰次数与年龄的增长成正比。色情允许甚至是鼓励自慰。你看色情角色总是爱抚自己,而且周围的人也没有指责。艾德现年45岁,之前他曾经一度沉溺于色情:“色情对我来说,最重要意义就是它打破了我的性压抑,它让我明白,自我性快感和性释放是无可厚非的。”\\
色情鼓吹自慰的特点尤其吸引青少年,不过你很容易发现色情鼓吹的不仅仅只是自慰而已;它还要成为自慰的焦点、促进自慰。这就是为什么色情要展现一系列现成的淫秽图片来刺激观看者的性欲。\\
玛尔塔是一名年近50的艺术家,她少女时期就十分喜欢《花花女子》(一本美国女性休闲杂志,以男色为主题——译者注)这本杂志。她说:“我就是个视觉控。我第一次看到这本杂志的时候是70年代中期。那时我还是个高中生。直到今天我还记得杂志里面一张让人眼前一亮的男性图片。他真的很俊美,同时给人非常淫靡的感觉。他随意地躺着,看起来纯洁无瑕。乌黑的头发,微笑的眼睛,古铜色的光滑肌肤。不过亮点在于,他勃起了。过去我就常常把他当作性幻想的对象。”\\
色情最大的特性之一在于它会让人明白,在追求性欲的道路上,我们并不孤单。也许,这就是为什么色情在男性中会这么受欢迎,因为打小成长的环境常常逼迫男性成为大爷们,来证明他们一点都不娘娘腔。对异性恋的男性来说。色情就让自慰这件事也“异性化”了。男性会有意识地忽略是自己在刺激自己的生殖器,因为色情强调的是女性的存在。也就是说,色情让自慰这种体验从“自我爱抚”变成了和别人一起”。至于“别人”只是一张纸做的图片或者屏幕上显示的影像,也就无关紧要了。\\
56岁的乔治已经是爷爷辈的人了。他告诉笔者,他少年时期沉溺于色情就是为了自慰。“60年代的时候,跟很多同龄人一样,我看着色情图片来自慰。那个时候,我能看的就只有《国家地理》杂志里面露胸的女性,还有美国西尔斯公司目录本的内衣秀里面那些衣不遮体的女人。等成年了,我就去买色情杂志。对着色情自慰真的很刺激,让我觉得快乐,让我得到解脱。”\\
现年22岁的麦克斯成长在如今这个充斥着色情的时代,他和乔治一样,从少年时代就要依赖色情来自慰。他跟笔者分享:“我并不喜欢色情对待女性的方式,所以我不会为了看色情而去看,除非是到了我需要发泄性欲的时候,只要我想看,就能在网上找到,让我又快又轻松地得到满足。不过,自慰之后,我就不会再继续看色情了。”\\
麦克斯还说,他少年时期借助色情来发泄性欲是合理的,因为那时候他还没有在现实中发生性关系的想法。“看色情、自慰,这些都让我的性欲得到及时发泄。不会有什么恶果。如果不是色情,我可能就会更加迫切地想要和现实中发生关系,那个时候的我还没有那么大的勇气,而且我当时年纪也太小,还没有足够的时间和精力来开始一段诚心诚意的性关系。”\\
4.“让压力消失、消失”(消除精神压力)\\
导致青少年在色情陷阱中不断沦陷的另一个原因是,青少年会发现色情能带着他们暂时逃离现实生活。色情就好像看书、看电视、玩游戏一样,可以让人暂时忘却现实。进入一个美好的幻想世界。在这个世界里,没有无聊的作业,没有一个念念叨叨的老妈,没有在你背后戳脊梁的明友,也没有让你一见倾心、却不幸地发现他(她)根本不知道你存在的暗恋对象。\\
和电视节目不同的是,色情不仅仅可以轻易地转移你的注意力,它也是一种现成的娱乐方式。因为色情类似于毒品的特性,能激起大脑和肉体的性欲,所以,那些经常觉得自己懦弱,或是感觉被别人忽视、抛弃,或是在精神上、肉体上受到侮辱、攻击,或是感到被人背叛的孩子更容易迷恋.上色情。一般来说,如果成长家庭中存在着酗酒、毒品、性虐待问题,或者家人脾气暴躁的话,孩子更容易迷失在色情陷阱之中。\\
还记得前文中布莱德和他哥哥的故事吗?他们在那个暑假里疯狂地看色情。其实整个暑假布莱德都被困在家里做农活,所以他心情郁闷,这是他迷上色情的诱因之一。布莱德抱怨他被父母晾在一边,不够自由,不能出去和死党一起玩,不能打篮球,更不用提和父母一起出去度假了。所以看色情就成了他向父母表达不满的一种方式,而色情为他敞开了这个伊甸园的大门。在这个新世界里,无聊的暑假生活就变得激动人心了。布莱德和哥哥一起看了一个暑假的色情,这可真的把现实枯燥的生活转变成了激情四射的日子。\\
且不论童年的经历。笔者经常会听到受访者说色情是孩童时期减压的好办法。再说个实例,柯克·富兰克林(Kirk Franklin)是一名格莱美奖得主,主唱褔音音乐。他曾在电视采访中说他童年时期会沉溺于色情,与他自幼被双亲抛弃有关。孩童时期的柯克觉得自己被人看不起,没有什么安全感,所以他需要到色情世界里去寻找“伙伴”。\\
伊桑是一名40岁的建筑师,他从小和酗酒成性、脾气暴躁的父亲生活,现实中的焦虑和恐惧在他边看色情边自慰时才会暂时地消失。“小时候,我是个容易焦虑,没有安全感的孩子。我爸爸太暴力了,他总是让我感到非常害怕。我早早就发现色情可以让自己逃离现实,寻求安慰。色情很刺激,会让我慢慢镇定下来。7岁的时候,我就从父母的房间里面偷色情杂志,然后横躺在自己床上慢慢欣赏。9岁的时候,我就在看色情的时候达到了高潮。家里的气氛比较紧张的时候,我就经常会看色情杂志来放松心情。”\\
贾斯汀成长在一个感情冷漠的家庭,他要用色情来抑制自己的火暴脾气。他告诉笔者,他把色情当作药物来控制自己的火暴脾气。“我童年时期很容易发火。我总是和兄弟们打架,有一次我真的想要把哥哥掐死,还好有个叔叔跑过来把我拉开了。色情对我来说就是一种药剂,因为我很难控制自己的冲动,而色情杂志可以转移我的注意力。看色情图片的时候,我会畅快地性幻想,这样我就可以控制自己的怒火了。到了青少年时期,我边看色情边自慰,每天多达两到四次。”\\
劳拉是一名35岁的女商人,她11岁时受到了性侵犯,需要时常观看色情来缓解痛苦。“我的两个哥哥迷上了爸爸的色情杂志。他们让我看那些图片,给我念那些故事,然后他们就对我做那些事,我沦为他们学习性爱的工具。我这么说你可能会觉得奇怪,但后来我会偷偷跑去哥哥们的房间看那些色情杂志。我尤其喜欢那些强迫性的性爱,每晚都要看那些色情来自慰。这样,我就可以暂时忘却哥哥们给我带来的伤害;这样子,我才能入睡。即使后来哥哥们不再这样对我,我还是继续看色情。”\\
这一章的事例充分解释了导致孩童陷入色情陷阱的种种原因。色情刚开始就是孩子好奇的新鲜事,而最后则成为一种舒缓压力的方式,让人忘却痛苦的回忆。尽管一些人成年后,正常的恋爱会逐渐帮助他们逃离色情,但更多的情况下,童年时期产生的性瘾会在成年后的几十年里继续影响个人的生活。关于这点,笔者将会在下章详细叙述。\\
可能童年时期的你也是如此,初次接触色情并不是为了发泄自己的性欲,尽管事实上色情确实可以让人达到发泄性欲的效果。很多孩子过早地接触色情,导致他们过早地对性发生兴趣,开始性经历。大部分受访者都称色情是他们人生的第一段性关系,而且这段关系到现在都仍然是进行式,给人的心理和性生活带来不容忽视的影响。\\
第三章 与色情恋爱\\
你可能不相信自己会跟色情恋爱,但如果你习惯性地看色情,那就是和色情在恋爱。没有人会发布征人帖子说:征人:征虚拟对象一名,可以在任何时问、任何地点、无条件满足本人的性欲望。”不过色情就能满足这样的条件,它像一个锲而不舍的求爱者一样,无时无刻不在诱惑着你来和它谈一场恋爱。和任何爱恋一样,和色情的恋爱也是难以割舍的。\\
当被问及是什么导致他们和色情“真诚恋爱”时,几乎所有受访者都说自己当初根本没有察觉到这回事。对他们来说,看色情不过就是为了找点乐子,舒缓压力,减轻烦恼,做点在现实生活中无法实现的事;他们不会停下来想一想,其实在享受色情的过程中,他们对色情就产生了一种情感和性的依赖关系,而这种感情可能会——其实笔者想说的是肯定会——导致一系列连锁反应。\\
如果个人没有正确分析自己与色情的关系,不了解色情的意义,不清楚色情的误导性作用,那么潜移默化中,色情就会悄然变成那个“重要的另一半”。如果个人经常性地向某人某事寻求情感慰藉或者性满足感,那么就会在情感和肉体上依赖于它。\\
上一章中提到,当人们年幼无知初次接触色情时,往往对色情的实质以及影响一无所知,也就很容易沉溺在色情中不能自拔。但到了成年时期,我们已经是生命的总舵手,要为自己的行为负贵,我们需要认清一个事实:只有断绝和色情的联系,才能在现实中发展积极健康的情感关系。\\
成年以后,一些人只会偶尔看色情来调剂,而另外一些人则是深陷其中不能自拔。研究表明,童年时期就接触过色情的人群中,1/2的男性和1/10的女性在成年后会沉溺于色情。这个数据表明,有些人在成年之后继续深陷色情,而另一些人则可以摆脱色情,继续自己的人生,这其中必然有些因素在发挥作用。\\
随着生活环境与个人经历的变化,个人的色情习性也会随之改变,一名男性在年轻气盛的单身学生时期,与他身为人父、子女成群时相比,更容易沉溺于色情:而一名女性,在还是足球队员时,与她生为人母、带着孩子踢足球时相比,更容易色情成瘾。一般情况下,看的色情越多,需求就会越大,因为人脑会在潜意识中把色情与快感联系起来。而用来观看色情的时间,是我们本该用来提高社交能力,加强人际交往的机会。\\
通过阅读本章内容,读者将能了解自己与色情的关系在长期内的变化:成年之后,你已经摆脱了色情,还是越陷越深?你是正在远离色情,还是和色情的关系稳定?\\
本章将会讨论影响色情与个人关系的重要因素,包括你是否沉溺于色情,你为何会沉溺于色情,以及你是如何沉溺于色情的。这些信息将帮助你更好地了解自己与色情的关系,更好地为自己的行为负责。在了解这些因素之后,你可以调整自己的观念和举止,以把握与色情的关系。\\
影响个人与色情关系的因素大致可以分为两类。一类是抑制因素,抑制个人看色情的欲望,为大脑降温,减少对色情的渴望;另一类是促进因素,诱使个人看更多的色情,在色情之中越陷越深,换言之,促进因素会促进个人对色情的欲望。\\
其实,在任何一个时候,抑制因索和促进因素都在影响着我们。一方让我们远离色情,另一方将我们拉近色情。其中的关键就在于,两类因素中哪一方对个人来说起着主导作用。如果抑制因素足够强大,就会减少个人对色情的欲望,让人逐渐远离色情。当个人的促进因素强于抑制因素时,个人就会在色情中无法自拔。\\
逐渐远离色情\\
本章首先要介绍的是今年27岁的杰克。他是一家体育运动品商店的经理。杰克生长在偏远的农村,在青少年时期接触到了色情。有那么一两年时间里,他常常看色情,但自从他上了大学,和女朋友保持性生活起。他就不现那么热衷于色情了。杰克的经历证明,随着年纪的增长。抑制因素逐渐增强,最终战胜了促进因素,有效地断绝了他与色情之间的联系。通过阅读杰克的故事,读者可以判断杰克的个人背景、生活方式、人生态度、人生经历等各方面的因素如何让他在色情道路上及时刹车。\\
杰克的自述\\
童年时期我基本上没有机会接触色情。因为居住在山区,我们家连电视机和DVD都没有。10岁那年,一天我在爸爸的橱柜里面找鱼竿,第一次看到了色情杂志《阁楼》。不用说,内容就是非常露骨刺激,但我也感到奇怪:为什么爸爸会买这种杂志,还把它藏得这么好?\\
杂志里面的女人看着有点奇怪,倒不是因为她们赤身裸体,我见过很多裸体的人。在我们当地的小区里,墓本上每个人都会在大热天去晒日光浴,在河里裸泳。问题在于,色情图片里面的女人看起来一点都不真实。她们狂野不羁,眼神狂热。我当时觉得她们很魅惑、很性感,但也让人不安。她们跟我在现实中认识的女性,我的三姑六婆,学校老师,一点都不像;这些女人看起来一点都不开心。\\
十几岁的时候,我还是个处男,我把看那些色情杂志当作一种实践,为第一次和女孩子上床做准备工作,这就好像为了学骑车在自行车后轮两侧装上的稳定轮。刚开始。我就看一些女人裸体,很快我发现自己还足比较喜欢穿衣服的女人我天生就喜欢健康的运动型女生,所以和电视节目“维多利亚的秘密”相比,我还足喜欢“大地尽头”(Land’s End)里那类模特。\\
我只看过几次色情录像,觉得非常倒胃口。那些录像对性交和高潮的描写真是一点都没有爱。这种重口味的内容禽兽不如,让我时做爱的欲望都没了。这与我的个性和价值观完全不符。我的朋友会公开说他们边看重口味的色情边自慰,可我就是一点儿兴致都没有。如果我要看色情,我绝对不会看那种暴露的、重口味的杂志,而只是看一些软性色情杂志。\\
曾经有段时间,我尝试着在看重口味色情的时侯自慰,但我一点都不喜欢这种感觉。高潮来得太快,不能让我满意,反倒会让我觉得空虚。如果我通过自己的性幻想来自慰,快感持续的时间会更长久,因为这种快感是肉体和精神双方面的享受。\\
我发现,接触的色情细节越少,我的性幻想反而更加丰满。如果看到赤裸裸的性交,还有什么好幻想的?那时候,我看了重口味的内容之后,把影像记在脑中,然后把里面的人物替换成我的暗恋对象来幻想着自慰。这样,我把色情影像现实化,也就更真实了。在我开始了稳定的性生活之后,色情自然退出了我的生活。即使后来我和女朋友吹了,在那之后的好几年里,我也几乎没看过色情。\\
阅读完杰克的故事之后,你觉得杰克将来会沉溺于色情还是会远离色情?根据笔者的临床经验,我们打包票保证杰克不会沉溺于色情。从他对色情的反应、利用色情的方式、喜欢的色情内容、性经历、恋爱的目的等一系列因素都能看出,他将来不会对色情有所需求。他更可能会和一名女性保持长期稳定的性关系,色情自然而然就会退出他的生活。\\
杰克的故事证明,抑制因素会让成年人失去对色情的兴趣,这些因素包括:\\
1.个人不好色情这口\\
2.接触色情的途径有限\\
3.对性关系有安全感、满足感\\
4.追求亲密的感情\\
笔者会对以上抑制因素一一进行详尽描述,读者可以独立判断这些因素在自己生活中的意义以及作用。许多人发现,通过强调这些抑制因素,他们能有效地控制自己对色情的迷恋。\\
1.个人喜好\\
不管是做人还是做事,我们都有自己的偏好。你喜欢北方人还是南方人?你喜欢中国足球队还是巴西足球队?等离子电视还是液晶电视?斯巴鲁还是悍马?饭后甜点,你喜欢苹果还是冰淇淋?同样地,个人对性画面的偏好也不同。有些人第一眼看到色情就喜欢上了,而另一些人只会觉得它很黄很暴力。这很正常,因为每个人对色情的直觉反应也是一直在变化;不过事后,个人会一次又一次重温初次接触色情时的感觉。\\
有些人告诉笔者,他们对色情的厌恶是本能的、发自肺腑的,那种感觉就好像你被逼着吃你最最讨厌的食物。而另外一些人表示,他们抗拒色情是因为色情违背了他们的哲学观、世界观、价值观或者政治观。\\
杰克与很多青少年一样,第一次看到《阁楼》杂志时觉得里面的内容很性感、很刺激,但是他个人同时对色情抱着消极的态度,这样就难以被色情完全控制。杰克除了觉得色情很吸引人外,也觉件图片中的女性不真实,让人生畏。爸爸买了这种杂志,还藏在渔具箱里而,这意味着什么呢?这个疑问也让杰克产生了一些焦虑感。尽管当时只有10岁,杰克似乎己经形成了明确的男性性价值观。\\
部分受访者觉得色情就是单方面的一头热,一点都没有爱,所以对色情毫无兴趣。无论如何变花样,色情的情节都很肤浅,那些演员会被选中就是因为他们能在摄影机前表演那档子事,而不是因为演技出众。20岁的山姆是一位救生员、他对笔者说:“刚开始色情会让我激动一下,但是几个月之后我就没什么兴趣了。色情里面的女人在现实中根本不存在。她们跟我在现实生活中接触到的女性完全不一样,看起来她门就很假,我也根本不会拿她们性幻想。性幻想一个我永远都不可能拥有的女人,有什么意义?”\\
同样对色情持否定态度的还有邦妮,一位21岁的女学生,她觉得色情中的角色让人反感。“我原以为色情很刺激,但其实很无聊。色情里的男人比我还要自恋;色情里的一夜情、3P我一辈子都不会想要,最让我不能接受的就是里面主角的滥情、不专一。”\\
25岁上下的杰瑞是一名康复中心咨询师,他不喜欢色情并不是因为演员气质猥琐,或是滥交的剧情,而是因为他不喜欢色情对待女性的态度、描绘女性的方式。“高中的时候,我总和一群哥们一块看色情。但我总觉得别扭。其实我不喜欢色情。有时候,我会和哥们讨论,色情里面的女主角也是有兄弟姐妹、父母双亲的,这让人感觉怪怪的。”\\
24岁的查德是一名单身汉,他不喜欢色情是因为观看色情会带来负能量,还耗费精力。“色情确实很有煽动性,也会激发起男性的捕食者欲望,这让我很不自在。上个月,有次篮球比赛结束以后,我和篮球队的哥们儿在一个朋友家里喝酒、吃烤牛肉。那个朋友放起了色情录像,大家就开始大喊大叫,冲着屏幕上的女人嚷嚷:‘来吧,宝贝!’之类的下流话。他们对着屏幕上的女孩子评头论足,讨论那些女人应该怎么怎么做。我一点都不喜欢这种发泄、感觉就像亲眼目睹了一场轮奸。我走到房间后面,以最快的速度离开了。”按理讲,强大的同辈压力会迫使查德留下来,但他的负面情感甚至超过了这种压力。\\
此外,一些人不接触色情是因为他们不喜欢冒险,害怕被当场捉住,也不希望承受曝光后的恶果。还有一些人称,他们不接触色情是因为他们认为色情和雾霾一样,会污染人的身心。\\
个人对于色情的态度通常根据性别而大相径庭。女性更容易抵触色情,因为色情本来描绘的就是男性的性幻想,来促进男性自慰,让男人泄欲,它忽略的是女性需求和性权益。比如说,女性常会性幻想以关爱和亲昵为基础的感官体验,或是以承诺为基础的精神爱恋,而色情仅着眼于某些身体部位,以及某些性爱动作,或是一夜情。即使色情会激发女性性欲,但是女性普遍还是反感色情,因为色情践踏、侮辱女性的尊严。1975年,《花花公子》杂志采访了美国著名作家埃利卡·琼(Erica Jong) ,她谈论了自己看色情时的感觉:“看了头10分钟,我就想要回家去大干一场;20分钟以后。我觉得在有生之年中,我都不想再干了。”\\
2.限制接触\\
不管色情多么易得,它还是需要你去接触存储和传送的信息设备,观看需要时间,有时还需要资费,此外,个人还需要小心掩盖以防被人发现。如果没有这些条件,个人观看色情就会有条件限制。如果个人没把看色情当成头等大事,或者看色情要付出的代价过于巨大,那么个人就会控制观看色情的行为。\\
举个例子,杰克非常热爱运动品店经理这份工作。与其宅在房间,他和其他男性室友更喜欢做菜,练瑜伽,静心阅读,跟恋人或者朋友出去玩。他们房里的旧电视机和录像机废置已久,杰克在办公室有一台电脑,用来追踪存货,与其他分店和供货商交流,但他不喜欢在电脑前待得太.久,而是更喜欢和店里的顾客交流。空余时间,他喜欢做些户外运动,亲近大自然,而不是在家沉醉于那个光与影的虚拟世界。\\
对于杰克来说,他需要改变日常生活中的习惯才能找到看色情的机会。他告诉笔者,他根本没有时间来做他喜欢的每一件事,而色情根本排不上号。\\
即使现在网络上的免费色情泛滥,经济原因还是会限制个人观看色情。购买录像机、DVD和其他播放工具需要花钱,很多色情网站的下载也是收费的。举个例子,弗兰克告诉笔者,他忍着不买色情品,就是因为他不愿意把自己辛苦赚来的血汗钱花在这些性幻想的产品上。“我从来不会想花钱去买色情,要是买了我才二呢,我会觉得被坑了。”\\
不管是什么原因限制个人接触色情,不论是时间、金钱,还是喜好程度,只要你与色情接触的渠道减少了,这就会抑制你继续观看色情。这类似于摆脱前任,你和她的交集越少,你就越容易走上正轨。\\
3.性安全感和性满足感\\
许多人在观看色情时会体验到性冲动,这种快感如洪水般充斥着全身,给人带来极度的感官享受,让人欲火焚身、急着泄欲。但是这种以性幻想为导向、产品为驱动的性体验并不能吸引每一个人。那些与伴侣性关系和谐,不需要借助色情来自慰的个人,对色情的欲望和需求都比较小。在访问过程中,笔者发现大部分不依赖色情的人群有着一个共通之处:他们对自己的性生活有安全感,对自己的性能力有自信心。\\
举个例子,45岁的杰夫做点小本生意,结婚已经20年了。他对笔者说:“我二十几岁的时候,对色情的欲望特别强烈,当时我特别想要知道各种类型的女孩子裸体是什么样子,不同的女人在性欲旺盛的时候表现有什么不同。我还想要学习一些新奇的床上招数,回去了能和老婆一起分享。不过,年纪大些之后,我对色情的好奇心就慢慢没有了。我想这是因为我和妻子的性生活很美满。我也能感到,妻子在性方面对我很包容。而且她也很享受我给她的快感。我们的交流没有任何障碍,和她在一起就能满足我的性需求,我也没有必要感到羞耻。我和她在一起时的快乐根本不能从色情那里得到:她的气息,她的触感,身体之间的交流。感觉到她的兴奋知道她得到了享受,也会把我带动起来。每次自慰的时候,我也喜欢幻想与妻子的性生活。而看了色情之后,那些画面会一直浮现在我脑海里,剥夺我的快乐和满足感,我一点都不喜欢那样!”\\
笔者接触过很多持相同观点的人,一旦他们在性生活上感到满足和自信,他们就不会再依赖于色情。对运动品店经理杰克来说也是如此,他在现实生活中也体验过性爱,所以色情作为“辅导教材”的功能也就不复存在了。现在的他更喜欢没有色情影响的性快感。他放弃了色情,把现实生活中的的性生活作为幻想的来源。有性需求时,他会结合自已的经历和性幻想来解决。\\
34岁的菲尔告诉笔者,他想要全心全意地去体验现实中的性爱,却与那个虚拟的色情世界产生了冲突。“我不喜欢色情让我放浪形骸。对我来说,只有本色的时候才能体会到最美好的性——感官清晰,性欲正常。我不喜欢色情就是因为这个自私的原因:我想要全权控制自己的性爱过程。”\\
跟真实的性爱相比,色情给人带来的感官享受不值一提,这就是为什么一些成年人渐渐远离色情的原因。对他们来说,色情永远都无法替代和活生生的人一起享受性爱时的快感。\\
4.渴望亲密情感\\
希望发展一段亲密性关系的愿望,可以帮助你远离色情。一个有理性判断能力的人不用看多少色情就会发现,色情根本不注重两性关系中的情感亲密度,也不鼓励负责任的性关系。“抵制色情男性团体”的发言人这样写道:“很多男性发现,色情会腐蚀人与人之间的亲密关系。色情似乎会‘让你兴奋’,但实际上,它怂恿你放弃了许多亲近他人的机会。”\\
菲尔喜欢与现实生活中的伴侣发生性关系,相比之下,色情逊色不少。“我跟爱人在一起时的性才是最好的。对我来说,性是神圣的。跟伴侣在一起的时候,我才会全身心地投入,和她很好地互动。”\\
理查德是一名28岁的服务生,他一也非常珍惜和伴侣的亲密关系。一次他和女友一起看色情录像的经历让他醒悟到,色情会阻碍双方之间的亲密感。“那是一部非常热辣的电影。我们俩看得热血沸腾,才看到一半就开始做爱。但是,在我爱抚女友的时候,电影中的镜头不停在我脑中回放。那次性爱很激烈。但是我们俩都感觉精神上很不合拍,都觉得自己只是利用对方来发泄生理上的欲望。我根本找不到两个人做爱时的亲密感,那种只有两个人在床上,能够清醒享受到的性爱。”理查德说,那件事给他留下了阴影.从此他再也没有租过色情录像。\\
麦克斯刚20出头,亲身体会到父母那幸福美满的婚姻,让他也下决心要培养一份同样美好的感情。对天长地久婚姻的向往,让他渐渐夫去了对色情的兴趣。“和异性的性生活,还有父母的好榜样,都让我了解到色情只是一个幻想世界,”他说,“那根本不是现实生活的榜样,色情根本没有半点温馨的亲密感,也无法体会双方之间的爱意。为什么还要苦苦追求呢?”\\
此外,麦克斯赞成,真挚的恋爱关系以开诚布公的态度为基础,这对那些沉溺于色情的人来说,要绝对坦诚是不可能的。麦克斯解释说:“看色情的时候我从来没有心安理得过。色情太隐秘了。这倒不是因为我害怕会被逮到。因为我父母很尊重我的个人隐私,我就是不喜欢关起门来做这种见不得光的事,也不喜欢疏远我在乎的人。”\\
如果个人开始一段认真的爱恋,他们可能就会选择远离色情,以此表示对伴侣的尊重。邓肯是一名22岁的应届毕业生,他说:“我女朋友不待见色情。而且,她不止一次向我表明,她不喜欢我看色情。色情会让她感到她不能让我满足,所以我才需要去寻找其他慰藉来得到性满足,事实根本不是这样的,我觉得她比色情火辣多了。不过,我理解她的感受,又不想惹她生气,所以我就不再看色情了。”\\
从这些实例中,读者能发现阻碍个人依赖色情的因素多种多样,相辅相成。如果色情妨碍你发展现实中的感情,那么你就会控制观看色情的行为,不会沉迷其中。同样,如果色情违背了你的宗教信仰和价值观,并且你对自己的性能力有信心,这些因素综合起来就会削弱你对色情的依赖性。生活中的抑制因素越强,个人就越可能会远离色情。\\
介于读者生活中的抑制因素或许会和上文中的事例有所不同,笔者希望读者完成以下测试“什么会让我远离色情”,了解自己生活中可以预防或限制观看色情的因素。请记住,随着读者生活经历、人生视野的改变,个人对色情的态度也在变化,所以读者的回答会随着时间而改变。\\
“什么让我远离色情”\\
判断我的抑制因素\\
列表可以帮助你判断哪些因素会减少自己以色情的依赖度,请在符合自己情况的项目前打钩:\\
—我没有便捷的途径来观看色情,我也没有意愿改变现状。\\
—我觉得色情很无聊,没有意思。\\
一我不喜欢色情里面描绘人和性爱的方式。\\
—我宁愿从非色情渠道中获取性信息。\\
—我觉得自己是一个很棒的性伴侣。\\
—我不喜欢被陌生人挑起性欲。\\
—我不希望对色情产生依赖性。\\
—观看色情会让我感到羞耻。\\
—我根本没有时间看色情。\\
—我不希望因为以观看色情而触怒或者伤害我的伴侣。\\
—做人要诚实。\\
—色情和我的性爱现相去甚远。\\
—我喜欢自己创造性幻想。\\
—我更喜欢不依赖于色情的自慰。\\
—色情无法满足我的性需求。\\
—最美好的性经历是我和在乎的人一起创造的。\\
—我喜欢性爱时身心的存在感。\\
—我不希望在观看色情时被人逮到。\\
—我不喜欢色情行业,不想为它做经济贡献。\\
—我不喜欢观看色情时的个人感受。\\
—我不需要借助于色情就能有效地舒缓生活中的压力。\\
—我不喜欢在观看色情时达到高潮。\\
—我更喜欢和伴侣一起看色情,而不是一个人看。\\
—我希望对我爱的人开诚布公。\\
—我认为色情不应该出现在家庭或工作场合。\\
—我觉得孩子们不应该接触色倩。\\
—我觉得色情是“黄毛小子”的性行为。\\
—我觉得色情太过激烈。\\
—有比色情更美好的事情等着我去做。\\
—年纪越大,对色情的兴趣就自然少了。\\
—总分\\
每隔6个月重新评估一次,确定让自己沉溺于色情的因素变化。\\
在色情中越陷越深\\
介绍了让人减少对色情依赖性的因素后,下文要着重介绍加剧个人对色情依赖性的因素。\\
前文中提到过科里,一位34岁的电脑分析师,他童年初次接触色情以后就越陷越深。和杰克一样,科里也是在小乡村中成长,成年离家之前,他都没有色情瘾。但在此之后两人的经历就大不相同了。科里的成长环境与杰克非常相似;但在现实生活各种因素的驱使下,科里开始沉溺色情 ,这给他的生活带来了许多严重的后果。\\
科里的自述\\
在我的家庭里,谈论色情或是任何与性有关的话题都是出格的。我从来不敢问父母任何关于性的话题,我们生活的社区也十分保守。社区学校禁止开展任何实践性的性教育。我从小就内向,对性一无所知。觉得自慰都是罪过。跟其他男孩子一样,我偶尔会看看色情,一般看的都是朋友之间传来传去的色情杂志,那些图片让我感觉很刺激。不过总的来说,我年纪小的时候看色情不算频繁。色情资料我拿得到,也真的想看,但是我觉得自己想要看色情的念头很龌龊。不过,我自慰的时候还是会看色情图片。大学的时候,我偶尔会去租色情录像,买色情杂志,但因为有自责心理,不敢看得太过频繁。\\
大学毕业之后,我开始和一名叫爱丽丝的女孩子约会,她很可爱,但我并不是很了解她。我当时只觉得,既然一个女孩子对我有了好感,我又觉得她顺眼,我就应该跟她结婚。她同意了,但是因为宗教信仰的关系,我们约定绝不进行婚前性行为。爱丽丝和我订婚4年,期间我们连接吻都没有过。想象一下,那时我20出头,正是血气方刚的时候,这极大地压抑了我的性欲。\\
毕业后,我进入一家电脑公司工作。我的本职工作是在新闻组网(Usenet Groups)上搜索信息,编辑目录。在这种网络论坛上,任何人都可以匿名发各种内容的帖子。新闻组网上的大部分用户都看色情。无论你喜欢哪种重口味的性爱,都能找到。只要你用新闻组网,你就会发现其他用户也在看色情。渐渐地,我开始沉溺在色情世界里。我有点恋足癖,最喜欢一个专门特写裸足的群组,我是那个群组的常客;我也喜欢在网页上面看未成年少女的色情图片。\\
在未婚妻面前,我从来没有避讳看色情这件事。我感觉得到爱丽丝内心的不满,但她还是忍耐着。其实,我对色情和自慰有很深重的罪恶感,但是这些负面情绪只会让我的高潮来得更加猛烈。我会自我安慰说老这样性压抑也不行,看色情总比跟别的女人发生关系好啊。慢慢地,我的性关注点就转移到了色情上,对未婚妻的态度越来越冷漠。只要我觉得孤单沮丧,或是百无聊赖的时候,就会边看色情边自慰。没有性生活,加上我的工作让我很轻松就能接触到网络色情,我就对色情上了瘾。\\
笔者分析,当科里的生活环境改变后,他与色情的关系得到了实质性的发展。他和女友长期保持恋爱关系但又禁欲,而他在工作时又能常常接触色情,这挑起了他的性欲。这些促进因素让他在色情之中不能自拔的,为将来生活中的诸多问题埋下伏笔。\\
笔者将科里的经历当作具有普遍意义的事例,总结出诱使个人在色情中越陷越探的促进因素,包括:\\
1.将色情和快感相联系\\
2.接触色情的途径便捷\\
3.利用色情来对抗抑郁和压力\\
4.无法维持恋爱关系\\
笔者将会详细分析上述促进因素,帮助读者更好地了解这些因索是如何影响个人与色情的关系。\\
1.将色情和快感相联系\\
虽然科里对着色情自慰后会有负罪感,但他还是对色情有着浓厚的兴趣,不会觉得倒胃口。跟很多人一样,科里喜欢色情是因为色情可以满足他的性好奇心,让他体会性快感。从初体验开始,科里就体会到了强烈的快感,从而对色情产生了浓厚的情感。\\
色情可以有效地辅助自慰,光这点就足以让人对色情迷恋不已,玛丽是一名43岁的会计,她说:“自从我把自慰和色情结合起来的时候,就已经没有回头路了。单纯的自慰没什么意思,但是和色情结合在一起就非常劲爆了。我在看色情节日的时候,感到自己融入性爱角色中去了,那种感觉和嗑药一样嗨。”\\
色情会引发性幻想,这是人们会把色情和快感挂钩的另一个原因。丹是一名20出头的年轻人,他就是喜欢色情能够立即带他进入“一切皆有可能”的性幻想世界中。他说:“我感觉,色情就是一个无处不在、随叫随到、心甘情愿的性伴侣。”\\
在性幻想的世界中,色情尤其能够满足男性的性趣和需求。在色情里,只要男性勾勾手指,女性就会二话不说主动送上门来。就算对最MAN的男人来说,女人对自己感兴趣的性幻想也是非常愉悦的。因为男性在现实生活中总是先要“客套一下”,彬彬有礼问问对方是否愿意和自己发生关系,何时才能发性关系。男人能在色情中找到无限的快感,是因为在色情世界里没有人会抗拒他们的欲望。\\
喜欢享受视觉刺激的个人更易患上色情瘾。现实中,大部分男性都会被视觉刺激引起性欲,这就是为什么对男性来说。相比较其他类型的网络色情活动,如网络聊天、网络约会和性教育网站,他们更喜欢看露骨的色情画面。研究证明,男性沉溺于视觉色情的可能性是女性的两倍。但一些专家认为,这种性别差异会渐渐缩小,因为如今越来越多的年轻女性过早地接触到了网络视觉色情信息。\\
此外,色情体验的其他方面同样可以促进快感。如第一章所述,色情的诸多属性会让人产生强烈的愉悦感,借助于高科技,色情可以诱发如嗑药的亢奋感。人们会把色情和快感联系起来,就是因为人在搜寻新鲜资料的过程中可以得到猎艳般的刺激感,如手拿冲锋枪扫射般的畅快感,更不用说它还能为高潮增色了。\\
前文提及,高潮会刺激大脑释放一些令人产生快感的化学物质,而任何能促进高潮的事物都会让人印象深刻。所以,人们通过,也可以说是需要,色情来达到性爱的顶点,如此享受自慰和高潮之后,人自然就会对色情产生强大的情感和肉需体求。一些人在单身时观看色情,即使婚后也无法断绝与色情的联系,这就是因为他们需要依赖于色情来寻求快感。\\
如果人需要依赖色情来达到愉悦的高潮,那么无论你是否有主观意愿,人在潜意识中都会想要更多的色情。有事物让你感觉很好,你就会想要不断重温这种感觉,这是人之常情。做那些让人愉悦的事情次数越多,越会让人习惯于此事,人会逐渐丧失判断力。色情带来的愉悦感会增加个人对色情的容忍度和接受度,从而忽略心底的顾虑。\\
2.接触色情的途径便捷\\
尽管如今色情泛滥,但要长期在私下观看花费很少甚至完全免费的色情资料,还是需要一个比较完善的接触渠道,因此,接触渠道仍是影响个人与色情关系的重要因索之一。过去,想看色情,我们还要出花钱买,回家了偷偷看,心里还在暗暗期盼买来的内容会是惊喜。但自从有了网络之后,人们就可以坐在办公室或者家中,随心所欲地搜索自己喜欢的色情。如今,无论你有没有主动去搜索,都可能会在网上看到色情的预告广告,无意间弹出的窗口,网页植入的色情广告。你会发现,几乎每一次上网,自己都要纠结到底要不要点击那些弹出来的色情窗口。\\
随着社会文化的发展,高科技已经改变了无数人观看色情的习惯。根据《男性健康》杂志最新的调查,71%的男性自使用互联网以来,他们会观看更多的色情,而1/2的男性会自忖是否过于频繁或者过久地观看色情。\\
接触色情的途径便捷是科里沉溺其中的重要原因之一。他说:“网络性爱把我牢牢困住了,搜索我想要的图像既方便又快捷,很难让人抗拒。只要我想,就马上可以沉浸在我喜欢的裸足或是少女图片里面。网络上其他人也在看相同的内容,这让我觉得自己的小爱好是‘正常’的。过去,要在商店里找到自己喜欢的色情需要大量的时间和钱财,现在的条件便利太多了。”\\
维克多是一名50岁的社区义工,如今已是他戒除色情瘾的第四个年头了。自从他能接触网络色情以后,观看的色情量曾经一度猛增。“婚后我偶尔会去买一些色情品。虽然我很喜欢,但还是觉得把色情品藏起来不太安全。所以看完后我都会销毁掉。在过去20年间,这样的事我只做过几次。1998年起,我开始看网络色情,之后我对色情的迷恋达到了顶峰,直到4年前,我才能控制自己对色情的依赖。我从来没有嗑过药,但是我现在觉得网络色情就是‘性瘾的可卡因’,我就是一个活生生的例子。”\\
托德是一名35岁的邮递员,对他来说,正是廉价的网络色情使他的色情瘾一发不可收拾。在使用互联网以前,色情的费用问题限制了他观看色情的行为。“我的经济一直比较紧张。我最多就是偶尔花16美元买本色情杂志,20美元去脱衣舞夜总会放纵一下。但我还是没有条件每天都这样去消费。我手头稍微宽裕一点的时候,很快就会被妻子榨干。那时接触色情的途径太少了。但有了网络以后,这事就容易多了。过去我一年之内最多偷偷摸摸进行十次和色情有关的活动。现在呢?我每天都可以在网络上看2到3小时的色情信息。所以,那种‘我要是真的干了这事回家没法交代’的顾虑完全不存在了。”\\
对布莱德来说,便捷的电影付费点播系统激起了他对色情的欲望。“结婚以前。我的性幻想对象是我的准妻子。结婚后6个月,我找了一份销售的工作,经常要出差。我总是在外面奔波,跑遍了全国。每一个住过的旅馆都有电影付费点播系统,而且,色情在点播系统中总是排名第一的。所以,只要一出差我就开始看色情。我的色情瘾又复发了,而且比之前还要严重。”\\
牧师吉姆·托马斯在美国俄勒冈州尤金的信念中心主持一个专门针对男性色情瘾者的项目。他认为,如今便捷的网络和电视色情频道是引发公众色情瘾问题的重要原因。“开车去一家成人商店、色情商店,被人发现的风险还挺大,人们很难拉下脸来做这事儿。但如果只需要鼠标轻轻点击就可以进人色情网页,人们就比较容易迷失自己了。其实,有些人在纠结到底要不要看色情的时候,如果不是面对着色情的直接诱惑,或许不会失守。偷偷摸摸看色情,不会有旁人知情,这样的风险就小了。所以,有些人就无力抗拒了。”\\
本是一名22岁的大学生。一次,他的室友们都出远门了,他一个人不分日夜泡在网上,就染上了色情瘾。“刚开始,我都没有想要看色情。但浏览网页的时候,色情无处不在。我的好奇心就这样被激发了,然后一眨眼,我就已经点击开了网页。在我有意识之前,这一切就已经发生了,我好像被色情这个巨大黑洞给吞噬掉了。”\\
接触色情的机会越多,人们就会越想要看。这个道理和超市结账台上面放着的糖果是一样的:就算我们不饿,也还是会想要买。便捷的渠道极大地满足了我们的欲望,削弱了我们理性判断的能力。如果色情时不时在生活中出现,我们或许就很难对色情说“不”。这就好像你实际上不想要吃结账台上的糖果,但你还是会掏钱买一样,色情会在你觉醒之前就在潜移默化中浸透你的生活。作者大卫·穆拉(David Mura)在他的著作《男性的悲哀:对色情和成瘾的解读》里这样说道:“色情图片出现的频率越高,就越能消磨人们的抵抗力。”\\
3.利用情色来对抗抑郁和压力\\
习惯用色情来消极对抗性以及感情问题的人,更容易在色情陷阱中越陷越深。色情带来的性体验,似乎是抚慰心灵伤口的灵丹妙药。如果你打小就把色情当作一种抵制机制,那你在孤单和脆弱,心情沮丧绝望之时自然会到色情中寻找慰藉。如果色情曾让你彻底释放,即使只是哲时而已,那么之后遇到了经济问题,工作压力,家庭矛盾,都会增加你点击色情网站、打开电视色情频道的概率。\\
38岁的东是一名音乐家,他虽然很享受和妻子的性生活。但他还是需要色情来对抗日常生活中的压力。他告诉笔者:“妻子只是给了我一个性欲发泄的渠道,这对我来说还远远不够。我还需要性幻想,这是我逃避现实的方式。老婆只存在于现实中,而现实才是我痛苦的来源;色情的性幻想世界,没有痛苦,只有享受。最重要的是,这一切都在我的掌控之中。”\\
艾伯特,中年、离异,3个孩子的父亲。他到现在还要看色情来逃避现实:“色情可以让我暂时忘记现实的痛苦。最重要的是,这一刻,我解脱了。”要说解压的效果,色情和电视是一样的。我们不过是选择了另一种现实的方式来躲避面临的问题。就像艾伯特总结的那样:“色情是现成的逃避机制。”色情能带来舒缓的功效,这不仅仅是因为色情能让人暂时逃离现实世界,进入性幻想的愉悦世界,还因为色情会使人体内的血流循环加快,刺激高潮的来临,舒缓肌肉紧绷。\\
如果色情被当作对抗焦虑、悲伤、愤怒、怨恨等负面情绪的工具,那么人们在色情中越陷越深也就不足为奇了。和前妻的婚姻亮起红灯的那段日子里,色情是凯文最忠实的伙伴。“那时我和前妻的关系已经开始恶化,我开始频繁地去一家成人商店。”他告诉笔者,“看色情会让我暂时忘却婚姻的痛苦。”\\
患有抑郁症、强迫症、多动症、毒瘾性格等心理疾病的人更容易色情成瘾。托德就是这样一个例子,他告诉笔者:“我患有多动症和抑郁症。色情对我来说就是个紧急时刻的防空洞。压力快把我击垮的时候,色情就在那儿等着我,我就可以暂时逃避一下现实了。”\\
对性生活的不满也同样会引发或者加剧色情瘾。一些人不去努力和现实中的伴侣解决问题,却转而求助于色情。以科里为例,他订婚4年,没有和未婚妻发生过性关系,连接吻也没有。还记得他是如何把性不满归咎为自己陷入色情陷阱的原因吗?性欲旺盛者会将色情视为后备胎,伴侣不能满足自己时,色情就是后备方案。一名23岁的年轻人总结的好:“色情为我的性欲添料,我就是把它当作备胎。”色情跟现实中的伴侣不同,它从来不会说“不”,也从来不会因为对方的特殊性需求或者自己性需求不满而发脾气,是实至名归的安全备胎。\\
40岁的吉姆是一位电工,在和妻子进行受孕治疗期问,他大大增加了色情观看量。在那段日子里,他被迫“随时为性备命”。他说:“我和妻子在日历上圈好了要做爱的日子,在一段时期内每天定时做爱。那段时间我压力很大,不管愿不愿意我都要做爱,不然医生和妻子就会不满。我喜欢色情,就是因为它给了我开放的性自由,我自己能决定是不是要高潮。”\\
很多受访者称性障碍会让自己加深色情瘾,也可以说,色情为性冷淡者、阳痿患者、不孕不育者带来了福音。卡森,一名55岁的退役运动员,在反思了自己观看色情的经历之后,总结道:“几年前我会沉溺于色情,就是因为我不希望自己性欲衰退。我想要重新体验年轻时那种强烈的性兴奋和性欲望。失去了性欲,让我觉得自己老了。不中用了。”\\
在现实中缺乏性伴侣也是导致个人产生、加深色情瘾的原因之一。兰就是一个典型的例子,把色情视为恋爱的替代品。他说:“我一个人住,没有女友。色情让我性兴奋,几乎每天我都要看色情来发泄性欲。除非我开始恋爱,不然我还会继续这样子下去。”\\
如果把色情当作改善人际关系,解决性欲等现实问题的工具,那么人们就很难探索出积极方式来解决现实问题。色情在短期内快速有效,看起来似乎是解决现实问题的简易途径,但它和毒品一样,只会让人产生依赖性。如果我们一味依赖于色情,不在现实中解决问题,那么我们只会在色情陷阱中越陷越深。\\
4.无法维持亲密关系\\
另一个导致色情瘾的因素是个人无法在现实中维持亲密关系。笔者采访过的很多色情成瘾者都称:“谈感情太浪费精力了”,“我不喜欢和别人分享自己的情感”,“性就是为了快感,不是为了爱”,“我根本不可能在恋爱时完全忠诚”。\\
对那些不喜欢对别人敞开心扉,不习惯展示内心的脆弱和温柔的人来说,色情就是绝佳的性欲发泄途径。真正的恋爱关系是要花费一番心思的。但色情到手很容易。就像牧师吉姆·托马斯所说:“在色情里面,性和爱恋是没有关联的,所以对那些不知道如何接近女性的男人来说,色情确实很有吸引力。从性方面来说,男人就是用下半身思考的动物,而色情带来的信号就是:他们这样做无可厚非,色情就是不需要恋爱就可以体验性的办法。”\\
26岁的杰克逊是一名法官助理,他身陷于色情瘾是因为他觉得色情可以庇护他,让他免受指责。他说:“看色情不需要我行动,也不需要我挖空心思去讨好别人。屏幕上、色情电话那头的妞儿让我很安心,风险也很小;而且,我也不需要担心我是不是早泄了。在现实中做爱根本没有这种自由感。”\\
皮特是一个25岁左右的年轻人,色情让他体验了以个人为中心的性快感。他回忆了自己的体验后说:“看色情不需要任何前戏。除了自己,我不需要满足其他任何人的性欲。色情给的就是我想要的,我就是个想要即刻性快感的任性孩子。”\\
对自己的性吸引力或者性技巧不够自信的人〔在生活中,每个人某些时候都多少会有这样的不确定感)也可能对色情毫无抵抗力。观看色情就是一场不需要别人评判的性活动。没有人在看你,没有真实的肉体接触。有些人相信要自始至终掌控主动权才能达到最完美的性爱,那么色情对他们来说就是完美的体验,因为色情观看者自己可以决定何时、何地,以及进行何种性行为。\\
罗恩·费因奇是一名波兰籍性治疗师,他说:“色情是一张通往模拟高潮的免费门票,还免除了现实性爱的风险。”男性的性欲望在色情这个世界中绝不会受挫,因此他们自然容易于沉溺其中。\\
不过话说回来,这并不是说色情不会带来任何风险。色情观看者经常要扯谎来解释自己的行为,还要把看的内容藏起来。研究显示,70%的人是偷偷摸摸地在看色情,而且就算人们公开自已观看色情的行为,前文中提到科里对他的未婚妻就不避讳这件事,但他们还是会尽量避免谈论细节内容,以免让伴侣感到不适。他们可能会闭口不谈最重要。最敏感的信息,比如他们在看的是什么类型的色情,他们自慰时也需要色情的刺激。\\
科里从来没有告诉过未婚妻爱丽丝,他经常边看未成年少女图片边自慰。如今科里才醒悟过来:恋爱关系没有满足他的性需求,他对爱丽丝的不忠,最终导致了这段关系的破裂。他说:“恋爱的时候,我的需求从来没满足过。从订婚到结婚这段时问里,我观看色情的频率暴涨,而这种双面人生最终毁了我,也影响到了他人。”\\
在生命的不同阶段,各种促进因素或许会驱使个人在色情道路上迷失自我。我们或许对性生活不满,想要发泄心理压力,或许希望重现年少时期的旺盛性欲,如果抑制因素不能完全抵消这些现实问题,那么色情问题失控的可能性就很大。\\
下文提供的清单详细介绍了促进个人与色情联系的各种因素。阅读完别人的经历,你可以花几分钟的时间来分析自身的促进因素。\\
“什么导致我在色情陷阱中越陷越深?”\\
分析促进因素\\
此清单旨在帮助个人确定目前生活中促进你和色情联系的因素。如果你有所述表现,请在前面横线上打钩。\\
—我把自己看的色情资料藏起来。不让别人看到。\\
—我为了能够继续观看色情不惜撒谎。\\
—我平时会幻想自己看色情的情形。\\
—我觉得自己想看多少色情都行。\\
—我很容易上瘾。\\
—我认识沉溺于色情的人。\\
—我有自己的色情资料储备。\\
—无聊的时候我看色情解闷。\\
—我想要体验色情中看到的性行为。\\
—我的朋友和熟人也沉溺于色情。\\
—我看到能传递色情信息的高科技设备就很激动。\\
—我最棒的性体验是在看色情时经历的。\\
—沮丧的时候,为了改善心情我会看色情。\\
—我倾向于在色情中,而不是在现实中寻找慰藉。\\
—尽管色情违背了我的价值观和信仰,我还是会观看色情。\\
—我需要观看色情或者幻想色情来刺激性欲。\\
—我把色情当作性行为楷模。\\
—我最棒的高潮都是在看色情时体验到的。\\
—我在性生活时也会幻想色情画面。\\
—我喜欢特写非法或者虐待倾向的色情。\\
—我抽出时间定期观看色情。\\
—只要我想要,就能找到色情信息。\\
—我很喜欢那些看起来像是AV演员的人。\\
—没有对象的时候,我需要色情来发泄性欲。\\
—只有看着色情或者想着色情的时候,我自慰才会感到自然。\\
—我都是独自观看色情,不和别人一起看。\\
—自从观看色情以来,我的性欲大增。\\
—观看色情被逮到的可能性让我觉得很刺激。\\
—我想到要放弃色情就觉得不舍。\\
—随着时间的推移,我越来越依赖色情了。\\
—总分\\
读者可以每6个月重新评估这些促进因素。\\
这些促进色情瘾的因素会引导人们在生活中作出选择,让个人与色情的关系或如开了闸门的水库,或如及时停止的刹车。比较一下你在促进因素和抑制因素清单的得分,哪一个总分更高?哪些促进因素和抑制因素对你的影响最大?这些信息可以帮助你评估现阶段与色情的关系,并预测走向。\\
每一天,我们或者更加靠近,抑或更加远离色情。如果我们想要和体育用品店的经理杰克一样远离色情,那么我们每一次成功抵制色情都会减少自身与色情的联系。但是,如果你和下章中将要提到的色情观看者一样,在色情陷阱中越陷越深,没有停下脚步去思考自己究竟有了什么变化、为什么会发生这些变化,那么结局可能会是灾难性的。就好像开车到中途,忽然发现自己迷路了,甚至完全不知道如何到了如今的地步。科里的遭遇就是如此,色情驱使他踏上一条不归路,乃至坠入深渊:他猥亵了继女,妻子离开了他,而他蹲了大牢。\\
在色情的道路上,你必须要清醒地意识到现阶段的状态以及未来发展的趋势。正如一位男性受访者所说:“这不是一个随便玩玩的游戏。色情这个幻想世界最终会入侵现实生活。”\\
第四章色情导致的恶果——勇于直视还是装聋作哑?\\
在法学院念书的时候,我约了班上一位聪明漂亮的女生来公寓听音乐,”布兰特,一名27岁的律师回忆道,“我让她先去卧室坐坐,自己跑去厨房给她弄点喝的。然而不到1分钟,她出现在厨房门口,神色有异,满嘴借口,说要急着回家。我为她叫了出租车,目送她离开之后回到公寓,才发现在卧室的咖啡桌上,在一盒抽纸旁,散乱放着几本色情杂志,几张色情DVD,这些都是我前一晚看了忘记收好的。从那晚起,尽管我和那个女孩每天上课的时候都是同桌,但那个优秀的女生再一也没有正眼瞅过我。”\\
这个事件让布兰特充分认识到了色情的恶果:原本可能会和他拍拖的女性,在发现他的不雅癖好之后不仅仅在性方而鄙视他,甚到彻底否定了他的人格。即使如此.布兰特却仍未想要就此戒除色情瘾。他只是决定,下次要是带女生回家一定事先要把色情品藏得好好的。\\
布兰特这种的心态是可以理解的,毕竟,某个事物给人带来快感,我们就不愿意放手。有趣又刺激的事情,谁不爱做?但问题在于,很多生活中会给人带来短暂快感的事物,迟早也会带来伤痛。很多情况下、快感背后往往存在着危险,而人往往在潜意识中不愿意放弃那片刻的享受,所以即使心知肚明也视而不见问题的阴暗面,避重就轻.。\\
给人带来快感的事物可能会带来严重的后果。酒精就是一个例子。很多人偶尔会小酌两杯红酒、啤酒或是混合酒。酒精风味独特,让人放松;在一个寒冷的夜晚、一场冷场的交谈中,还可以用来驱驱寒、热热场。酒精和色情一样,很容易获得;酒精也分不同的口味和度数,可以在社交场所合饮,也可以独自一人时喝两口。想要完全回避酒精也是十分困难的事。酒类广告无处不在,鼓励我们畅饮,许多场合也提供洒精饮品。\\
但是,如果个人经常喝洒,就会面临一系列不良后果。人喝高了之后会口齿不清,满嘴胡话,判断力受损,自制力全失,健忘,走路踉跄,不能开车。长期的酗酒会导致个人无法维持正常的家庭生活,无法必履行工作和社会职责,甚至伤害亲属友人。严重情况下,酗酒成性还会引发严重的健康问题,引发肝脏疾病和性功能等问题,让人对化学物质产生依赖性、也有人酒后驾车而身陷囹圄。\\
但是,一旦人对酒精上了瘾,一扎冰啤酒、一杯红酒都是很难让人抗拒的。对很多人来说,喝酒己经成为生活的一部分了,他们每餐必喝:下班后泡酒吧时,一片比萨饼、一条热狗就着冰啤酒就下了肚。即使认识到酒精给个人生活带来的严重后果,个人也会本能地为自己的行为辩护,否认喝酒的负而影响。如果我们选择无视现实,问题只会进一步恶化,要想彻底根治也会更加艰难。\\
色情观看者也会如此为色情的恶果辩护。起初,你能看到的都是色情给生活带来的积极影响。在潜意识中,你可能已经意识到这样子做是会带来间题的,但是快乐逍遥之际,哪里还会考虑将来的问题呢?\\
但是,和酗酒一样,沉溺于色情终会带来无穷无尽的恶果。到那时,即使你欲盖弥彰,都无法再假装一切正常了。从大部分受访者的经历中可以看出,他们的生活最终还是失控了。而不幸的是,许多人开始看色情之初都不清楚后果的严重性,直到木已成舟,一切无法挽回。\\
罗波,43岁,他自从14岁起每天看着色情自慰。他的经历充分说明,人们会如何熟视无睹色情带来的恶果。他告诉笔者:“和其他的坏习惯相比,色情好像一点杀伤力都没有。要是赌博,你的钱就会输光;要是吸毒,你的身子就坏了,身体机能衰退,体弱多病。色情不会影响我开车,也不会有什么明显的坏处。之前我就是看不出色情的坏处,它对我身体的影响也少。所以,我以前从来不担心会有什么后果。要不是后来它毁了我,我根本不会意识到问题的严重性。”\\
在本章中,我们将会详述一系列列由色情引发的问题,指明色情为何会成为盲点,讨论个人即使意识到色情的后患,却还是会为色情辩护的原因。通过叙述部分色情观看者的经历,色情瘾康复者的励志故事,笔者希望读者能更好地了解,色情会引起生理和情感问题,还有性生活问题,甚至影响到个人的事业前程。\\
观看色情所引发的后果\\
观看色情时,人会沐浴在性快感之中,但潜在问题如同潜伏的暴风,虎视眈眈。据色情瘾康复者总结,观看色情导致最普遍的九大恶果为:\\
1.“我很容易感到暴躁不安、抑郁。”\\
2.“我变得越来越孤僻。”\\
3.“我将他人视为性物体。”\\
4.“我忽视了生活中的重要因素。”\\
5.“我的性功能有障碍。”\\
6.“我让伴侣不快了。”\\
7.“自我感觉非常不好。”\\
8.“我在做一些有风险的危险举动。”\\
9.“我对色情上瘾了。”\\
但凡出现上述任意一种症状就表明色情已经严重影响到了你,而符合的症状越多,就表示个人在色情陷阱中沉溺越深,要治愈色情瘾的难度也会随之增加。九大后果中每一项症状都说明了色情所带来的问题,而个人体验的症状越多,也就表示个人在色情陷阱中沉溺越深,要治愈色情瘾问题的难度也会随之增加。正如上表所示,色情所导致的后果体现在个人心理和人际关系两个层面。色情会影响个人的思维方式、内心情感,以及个人的社交方式。在某些情况下,即使个人对色情的“上瘾”程度没有那么严重,但还是会遭遇色情的恶果。\\
下面,笔者将会洋细介绍每项后果的情况。\\
1.“我很容易感到不安、抑郁”\\
如果人私底下做了什么见不得人的事,即使有时主观意识还未察觉,这都会在潜意识里影向个人的情绪。长期观看色情的人通常脾气火暴,鸡毛蒜皮的事也会让他们暴怒,而最后则发展成为抑郁。无论观看者能从色情中获取多少快感,大部分人还是会意识到——至少在潜意识中意识到——他们的行为不受社会认可。绝大部分色情观看者都是偷偷摸摸的,因为他们知道这个社会总体上来说,会把观看色情的人当作性“变态”或者是“色狼”。\\
就算个人把看色情的行为隐瞒得天衣无缝,绝无他人知情,但是把色情当作一种性欲发泄的方式常常会让人自惭。一方面,个人对色情怀着强烈的欲望,但另一方而又觉得看色情很无耻,这种矛盾心理会在潜移默化中影响观看者的心理健康,让人左右为难,在精神和情感上受压。个人的心理承受能力是有限的,而色情的恶果终会渗透到个人生活的各个方面。个人要有足够的勇气,才能够和爱人、朋友、同事或是心理医生谈论这些问题,所以大部分人只是默默承受,脾气变得更加暴躁,内心更加焦虑,情绪更加压抑。\\
许多成功克服色情瘾者称,在持续观看色情的过程中,他们非常容易因为一些鸡毛蒜皮的事变得焦躁,而他们的本质并非如此。在我们后面开车的那个司机就是个“傻子”,爱人忘记在商店买该买的东西,真是“二到家了”,孩子真是“烦人”。我们在无意识中会将观看色情而引发的负面情绪迁怒到生活中其他的人和事上去了。\\
35岁的比尔称:“老看色情的时候,我很容易发火。我在工作和生活方而像是变了一个人。我心中觉得自己配不上妻子。觉得自惭形秽。她在生活中效率真的很高。我想到自已的缺点就觉得非常郁闷,戒备心特别强,同时也担心看色情会被逮到。这种负面情绪就转化成了怒火。我一看到不顺眼的事情就大发脾气。这是色情的副作用,而我当时根木没有意识到这个问题。”\\
愤怒和焦虑这些负面情绪会加深个人对色情的依赖性。负面情绪的爆发,会自动在色情观看者和他人之间竖起交流的屏障。很多色情观看者都相信这句老话:“先发制人”。他们主动出击,打击对方,以此来转移别人的注意力。此外,他们还会打架斗殴,怨天怨地,为自已的行为找借口。鲁迪已经成功戒除了色情瘾,他说:“观看色情的那段日子里,我的脾气非常坏——满口脏话,对人苛刻,控制欲强。我和老婆老是闹矛盾,之后我就去看色情来补偿自已,还安慰自己说,是她逼我这么做的。”\\
当色情观看者意识到自己己经深陷在色情之中无法自拔时,也会感到愤怒。怒火会在潜意识中掩盖他们无法克制观看色情的欲望而产生的自责心理。凯斯,一名30岁出头的父亲,说:“我以为结婚以后我就可以戒除色情的。好吧,婚后我确实有几个月没有碰色情。然后我以为有个孩子就好了。这次也是一样,我也就忍住了几个月的时间。第二个孩子出生的时候,情况也是差不多。每次戒瘾失败后,我都会变得更加暴躁。只要是个活物,我都会发脾气。”\\
有些人和凯斯一样,会把怒火发泄出来,而另外一些人则会在内心堆理负面情绪,这样怒火就会转化成为抑郁。色情成瘾者不把愤怒撒出来,就会在内心备受精神折磨。陷入困境中,人的压力倍增,却无法探索解脱之道,却无法摆脱负面情绪,就容易产生抑郁。对一部分人来说,抑郁和无助会长期折磨他们,甚至产生轻生的念头。\\
只可惜,色情不会贴什么温馨提示的标签,等到色情观看者发觉色情引发的负面情绪之时,往往是为时己晚。科里告诉笔者:“以前,我从来没有意识到自己长期以来的负面情绪是因为看了色情。举个例子,我过去很容易累,其实无精打采就是抑郁的主要表现之一。但是我当时并没有想到把这点和看色情的习惯联系起来,误以为我饮食不合理才会有这种问题。直到我进了监狱,停止看色情,这些症状才开始改善。”\\
在理想情况中,一个人有了抑郁、愤怒,焦虑等负面情绪时。就该主动去寻求帮助,改变以往的生活方式。但有时为了对抗负面情绪,一些人选择了色情,把它当作舒缓感情压力的“麻醉剂”。色情确实能有效控制情绪,但一切都是暂时的;就其长远影响来说,色情只会让人更易感到愤怒,心理抑郁,坠入情感折磨的无底深渊之中。\\
2.“我变得越来越孤僻”\\
经常观看色情并不需要一个人躲到世外桃源去,但某种意义上来说,看乐对情会给人一种生活在深山老林里的感觉。许多色情观看者称,色情的副作用之一就是会让他们越来越孤僻,渐渐疏远生命中最重要的那些人。想当初,个人观看色情就是出于对人体的好奇心,而这样看来,这个副作用就显得有些讽刺了。色情观看者发现,长期以来在这个幻想世界中的趣味旅行,最终导致他们很难在现实中维持真挚的爱恋关系。\\
经常观看色情的人是孤独的,因为他们要忽视别人的存在,躲避他人的注意力。色情观看者常常说,他们习惯“宅”着看色情——脱离现实生活,一个人的时候偷偷看。一位男性受访者称:“观看色情不是我现实生活的一部分,这只是我在闲暇之余为了找点乐子,宣泄过剩的精力。色情跟我生命中的其他部分没关系,也不会影响到我的人际关系。”其实,问题就在这里:要偷偷“宅”着看色情就意味着避人耳目,对别人撒谎。为了保证自己有机会观看色情,乐对情观看者自然而然就会疏远他人。考虑到这层关系,色情观看者是牺牲了社交的机会,转而去看色情。\\
要找到和色情独处的机会,搜索劲爆的色情内容需要消耗大量的时间和精力,而这本该用来参加社交活动。西蒙是一名交换项目的大学生,他告诉笔者,他周六晚上常常一个人在寝室对着电脑浏览“学生妹色情”网站,放弃去参加校园活动或聚会的机会。\\
如果在网络上面花费的时间过久,个人就会倍感孤独,与现实世界愈加疏远。现在平均每个美国人只有两个挚友,与1985年的统计数据3个相比还少了1个,网络的日益普及、高科技的发展是造成这一现象的主要原因.对网络色情上了瘾就会影响个人发展真挚的爱恋关系,在精神和肉体上越来越疏远他人。这个问题后果不可估量,因为对每个人来说,真挚的面对面交流是最基本需求。\\
要色情成瘾者去建立新的人际关系,他们或许会有所犹豫。如果这个新朋友或者爱人指责我观看色情怎么办?如果我向他们伸出友谊之一手却被拒绝了,又该怎么办?如果他们要我放弃色情,才能和我继续交往的话怎么办?社交焦虑会让人精神紧张。一名色情瘾康复者告诉笔者:“不停止观看色情,我就无法维持一段真挚的情谊,无论是恋爱关系还是朋友友谊。长久以来,观看色情也好,自慰也好,都让我非常内疚,但那时我的色情瘾实在太重,既不想要结交朋友,也不想要接触任何要求我绝对坦诚的人。我有太多罪恶的秘密了。”\\
对于33岁的连来说,色情已经成为阻碍他结识新朋友、与伴侣交往的屏障了。“我很难开始一段新的感情,因为色情已经给了我现成的满足渠道,释放我的性欲,让我得到满足。这样子,我就不需要费劲去结交新朋友,为人际交往犯愁了。”\\
观看色情导致个人越来越孤僻的另一个原因是,色情所教授的爱恋方法会让人失去对他人的兴趣,长期观看色情会导致个人变得越来越自私。毕竟,个人在观看色情时,所有的重心都是在自己身上。色情也会在个人心中树立这个理念:在性爱和恋爱中,力量和控制比怜爱或关爱更重要。如果个人长期观看色情,尤其是在形成同情、关爱、体贴和爱心这些特质的关键年龄段,那么个人可能很难在现实中发展真挚的人际关系。个人可能更希望和伴侣培养色情中的单纯性爱关系,而不是亲昵温存的互动。\\
色情作为个人的性爱“伴侣”,其占有欲十分强烈。它要让你在观看时独自一人,而且它还限制了个人在现实中与性伴侣维持亲密关系的能力。\\
3.“我将他人视为性物体”\\
“此刻,不管你喜不喜欢,我带着性欲上下打量你。我根本不知道你的名字,也根本不关心;我不知道你是谁,也不在乎。你就是为我提供性快感的一个物体。我只关心你会让我引起怎样的性欲。”一名色情瘾康复者如此生动地描述色情瘾者的看人方式。\\
笔者将这种现象称为“色情化”个体,即以充满性欲的方式打量他人,在本质上将人转化为色情影像中的一个角色。“色情化”就是一种把人当作性物体,将现实生活中的个人转化为色情中的幻想对象。个人观看的色情越多,被挑起的性欲越强,个人就更有可能将现实生活中的个人视为性物体。\\
玛莎,一名中年艺术家,她在现实生活中开始将他人“性物体化”时,才开始担心色情的负面影响。“这跟喜欢看养眼的人不一样,”她说,“我去公园跑步,就会不由自主地打量看到的每一个人,幻想他们能挑起我多少性欲。我就跟酒鬼一样,去看一场比赛,去一家餐馆,逛一家食品店,就一定要喝酒。看到一名男性在教堂做礼拜的时候,我就会想,他如果裸奔的话,该有多刺激啊。不管我想要如何在主观上控制,这种臆想都不会消失。”\\
色情化不仅仅会在现实生活中分散人的注意力,它还会让人对现实生活中的伴侣性趣全无。如果你只想要和长得像艳星的女孩子约会的话,可供你选择的对象就少了。赞恩,一名高校的高年级学生,说:“网络上那些女孩子太火辣了,个个都是魔鬼身材。我花了好多时间性幻想她们。但是,最近我意识到自已无法忍受现实中女生的不完美。在酒吧我会和一个挺好的女孩子搭讪,她长得可爱,又很风趣,但是我的兴趣就此打住。因为她不会是一个‘满分’,她终究还是有缺陷。她胸部太小,腰太粗,大腿太肥。我知道,因为这些女孩子没有色情女星的超级模特身材就否定她们,这样是不对的,但是色情让我无法对喜欢的女生产生性欲。”\\
兰迪,一名26岁的男性,坦白道,他多年来性物体化女性的习惯阻碍了他和现实女性发展恋爱关系的能力。色情剥夺了他和女性交友的本基能力和技巧,也让他失去了对女性的尊重。他说:“我过去只把女性当作性物休,如果她能挑起我的性欲,我就跟她聊;如果她不能,我就不搭理她。大部分女孩子都会反感这种以性爱为目的的攻势。除非我觉得我能把她勾引上床,否则再漂亮的女孩子也引不起我的兴趣。我过去从来不知道利用某人做爱和与他人一起分享性爱之间有着本质差别。色情让我没法公平地看待女性,也让我没法真心待她们,我真的是太坏了。”\\
在色情世界中沉溺的时间越久,个人就越倾向于把他人当作利用的工具,忽略双方之间的分享和互动。观看过多的色情。会让人分不清性幻想和现实之间的界限,这会损害个人共鸣怜悯之心,以及坦然面对自己、面对他人的能力。\\
4.“我忽视了生活中的重要因素”\\
一开始,观看色情是一种刺激的娱乐。一种促进自慰的方式,但是日积月累,它就从一种忙里偷闲时找的乐子,变成了一项严重阻碍生活追求的活动。许多色情观看者称,他们沉溺于色情无法自拔,导致事业前程受挫,家庭责任感松懈,身体健康受到威胁,精神生活堕落。\\
任何一件能提供强烈快感的事物都会让人上瘾——不管是修理好了一辆古董老爷车,玩在线的扑克游戏,还是看现场直播的比赛。如今大量的色情是通过大众化的高科技设备传播的,人们很容易就在工作学习时被色情吸引了注意力。即使个人并没有在看色情,光是回忆看过的色情图片,期待将会看到的内容,计划好独自一人观看的时间地点,掩藏好一切蛛丝马迹以免被抓等,都会占用个人大量的时间和精力。大量调查表明,个人观看色情的时间越长,色情带来的问题就越多。\\
沉溺于色情不仅浪费时间,还耗费精力。严重影响个人身体健康。沉溺于色情者往往不坚持合理运动、健康饮食、按时睡眠,甚至顾不上洗澡、梳头这些个人卫生问题。睡眠问题,如失眠、睡眠不足普遍困扰色情观看者,因为很多人都是在晚上睡觉前或者半夜看色情。他们常常会置自己的身体健康、家庭关系、工作事业于不顾,长期熬夜看色情。如果个人原本的生活就十分忙碌,色情更会雪上加霜。罗彼说:“我有一份全职工作来养活全家,晚上还要去夜校上课。我还包揽了家务活,努力维护一个好丈夫、好父亲的形象。我晚上从9点到12点做些家庭作业,然后在网上搜索色情,直到凌晨1、2点,这样我每天的睡眠时间不过4、5个小时。我只要一看色情就没法刹车,到了第二天,我几乎无法正常工作学习。”\\
除了占用睡眠时问,色情还会导致睡眠问题。看了那么多刺激内容后。大脑的视觉中枢会持续保持兴奋的状态。长时间在电脑前看色情还会影响视力和颈椎健康,这也会影响个人的睡眠质量,而白天的生活也变得更加难耐。\\
在色情上花费过多的时间,也会削弱个人的工作或学习追求。如果大脑一直渴望着色情,那么就算人不在看色情,也很难全心投人工作和学习中去。有人告诉笔者,当他沉溺于色情的时候,工作效率降低大约75%,最终让他丢了饭碗。\\
科里坦白色情如何扼杀了他的电脑事业。他说:“性瘾越来越严重的时候,我就开始忽视生活中的其他事情,尤其是工作。我开始临时抱佛脚,工作表现也是垫底。我每天花3到4个小时看的那些东西,对自我完善、技能学习、了解世界、促进人际都完全无益。这事我现在想起来就伤心,我浪费了好多好多宝贵的时间,一寸光阴一寸金啊。”\\
对比尔来说,对色情的执着不仅仅浪费了他的时间,还让他白白丢失了许多机会。他说:“我在生活中是一个要强的人,但色情很容易就转移了我的注意力。那时我没有督促自己勤奋努力,每天都要看几小时的色情。我是一个股票经纪人,工作松懈了之后,我错失了很多宝贵的机会。即使这样,我还继续看色情来转移注意力。遗失掉的志向,挥霍掉的机会,让我失落不已。”\\
长期观看色情也会让人忽略家庭和感情责任。正如前文所述,色情观行者常常不惜远离伴侣和孩子,挤出更多的私密时间观看色情。观看色情显得比家庭活动,如陪伴孩子、陪伴侣散心、做家务还要重要。一名男性告诉笔者:“我牺牲了对家庭成员的关爱,只为了满足自己对色情的欲望。”另外一名男性称,色情彻底改变了他与家人的交流方式:“过去我养成了一个习惯,就是趁着妻子和女儿去购物的时候,躲在家里边看色情录像边自慰。有时候,她们会叫我一起去,但是我都拒绝了。有些时候,她们的车还没有开出家门前的车道,我就已经迫不及待地开始看了。”\\
当色情占据了一个人所有注意力时,流失掉的不仅仅是时间。一些色情观看者为色情掏空了腰包,就是为了上一些特殊的网站以及获取杂志、书籍、录像、DVD、电子设备和订阅服务。这些资金本可以用来提高家庭生活质量。当然了,色情也会引发更严重的经济问题,如个人在上班时间登录色情网站而被炒了鱿鱼,或者个人因为进行不法色情行为,如观看儿童色情而被逮捕。\\
就算只在家里藏了点色情内容,像是电脑文件夹里保存的信息,也会对孩子造成威胁,增加他们过早接触色情、受到精神伤害的风险。正如笔者所述,儿童时期过早接触色情往往会成为色情瘾的源头;而且,如果孩子发现父母观看色情的话,对父母也会改观。\\
如果色情已经成为个人生命中头等要事,这对人的精神生活也会产生极大的影响。《男性键康》杂志曾做过调查,采访了部分色情瘾痊愈的男性,询问他们戒瘾之后生活哪个方面得到的改善最为明显,答案是:精神。色情不仅仅掠夺了个人原本可以在教堂做礼拜、在社区做义工的时间,也会动摇个人的核心价值观,让人产生羞愧感和虚伪感。其实,只要你仔细想一想,你就会发现,崇尚信仰时的表现,如对信仰的忠诚,对精神领域的思考,对宗教的着迷等,同样符合“祟尚”色情的表现。部分色情观看者告诉笔者,他们羞愧到不愿去教堂,不愿意用积极的方式表达信仰。\\
5.“性功能有障碍”\\
色情观看者往往只把情当做一种高效无害的刺激性欲方式,但实际上色情最终会导致严重的性功能问题。近期,一名57岁的男性在笔者的官方网站HealthySex.com上留言说:“我有性功能障碍。我边看色情边自慰的历史太久了,现在和我心爱的女友在一起的时候,我根本不能性兴奋,这对我们两人的打击都很大。我想要和她享受正常的性关系,不受色情影响的性关系。其实我要勃起没有问题。说得这么露骨很抱歉,但是我真的不想要失去这个女人,我只是想要回到正常的我。”\\
类似的情况并不少见。习惯性观看色情会导致一系列性功能问题。下列清单是笔者在临床观察中总结出色情瘾导致的性功能病症。读者可以由此判断,自己或者伴侣是否存在以下问题。\\
色情观看者的十大性功能问题\\
_1.避免或者无法对伴侣产生兴趣\\
_2.面对伴侣无法产生性欲\\
_3.与伴侣一起时,无法或者无法维持勃起\\
_4.性生活时无法达到高潮\\
_5.性生活时,脑中会闪现色情影像和念头\\
_6.对性伴侣苛刻、粗暴\\
_7.过性生活时情感硫远,精神游离\\
_8.性生活后,对伴侣不满\\
_9.无法或者很难建立或者维护爱恋关系\\
_10.进行出格、有风险的危险性行为\\
肉体天生就会享受性爱,但是人类大部分的性行为习惯都是后天习得的。色情能提供强有力的性训练,塑造个人的性兴趣用体验性快感的方式。比方说,色情教会我们独自一人获得性高潮,对陌生人产生性欲,而不是找活生生的、有血有肉的、含情脉脉的伴侣。\\
如果把色情看作行为模范,人就很容易对现实伴侣抱着不切实际的幻想。个人会感到,现实中和伴侣的性生活,远没有色情描绘的那样刺激有趣,反而是无趣得很。现实中的性生活好像就是代替色情性生活的次品。一名男性告诉笔者:“长久以来,我总是一个人对着色情杂志上的夸张插页到达高潮,相比之下,现实中和相貌平平的人做普普通通的爱就让我觉得既尴尬又无聊。”\\
对伦来说,常年来对着色情自慰让他在高潮时不愿意有旁人存在。他说:“我在现实中和女性发生关系时感觉很别扭。我总是一个人偷偷看色情,身边没有其他人在场。我在现实中和女性就发生过那么几次性关系,总觉得性爱很奇怪,无法接受。”\\
色情还会让人对性生活的次数和频率产生不切实际的幻想。20岁出头的艾利克斯刚刚结婚,他就抱着错误的期望,希望和妻子的性生活能和色情影像一样,激情火爆,随心所欲。他说:“我一直以为,在一段婚姻里面,性生活的频率是由我来做主的。色情先入为主,让我以为性关系可以随时随地发生,我想要多少次都可以。现实让我非常失望,还一度影响到了我的婚姻。我后来才明白,只要是地球人。都不会像色情里描写的那样持久。”\\
史蒂芬一位年近30的男性,对现实生活的性伴侣都非常失望,因为他根本找不到一位女性会乐意尝试色情里面刺激的性行为。说:”我在大学时期开始性生活之后。才发现现实中没有女生愿意做色情里面的姿势。看色情时,我觉得有些动作非常刺激,但那在现实中根本没有市场。我原本以为这些动作是很正常的,后来才发现不是这样,这太让人扫兴了。”\\
现年40岁的伊桑试图实践色情中的性行为,也受到了打击。“着色情这么久,我一直充满幻想,在心中早就臆想好了性伴侣的容貌、性行为时的行为。她要一头金发、大胸脯、小蛮腰。我只想和那些男人们看了会流鼻血的美女约会。最重要的是,她一定要把我看成是充满l量和欲望的控制者。我过去从来没有质疑过自已的幻想,所以总是对现实中的性伴侣不满,某些时候还强迫她们和我发生性关系。我无法欣赏每位女性独特的美和性感,我也不知道正常的性行为是什么样子。”\\
色情会引发性功能问题,是因为色情会误导人过度依赖于视觉影像来引起性欲,如此一来,个人就容易忽视性生活中的其他因素,如心理感受,肉体感官等。杰克·强斯顿,一名网络成人性教育家告诉笔者:“色情会麻木性快感时的心理感受。个人在多大程度上依赖于外在的感官来达到性快感,就会异致多大程度上内在性欲感的缺失。长期观看色情录像会麻醉个人情感,让人习惯性地依赖于色情,无法在现实中与伴侣享受性生活。”\\
《拒绝成为男性》(Refusing to Be a Man)这本著作的作者——斯托尔·腾贝尔格,在这段话中明确表明了自已对色情作为性教育角色的看法:“一旦影像技术让男性设想了心中最完美的性经历,那么他有可能再也无法体验性爱,而彻底沦为一个机械的窥淫癖者,不时地在自慰中抽搐。”\\
一位受访者称:“为了在性生活中保持勃起的状态,我必须要不停地回想色情中的镜头。有时候,色情对我来说就是‘视觉伟哥’。约会前,我也要先看色情,让那些镜头在脑中不停回放。我要把女友的形象在脑中转化成色情中的角色,才能达到高潮。”\\
除了不断追求视觉上的刺激,色情观看者还会想追求更加强烈的生理刺激。高频的、强迫性的自慰会让人对一般的爱抚和触碰变得迟钝,只有极其强烈、粗暴的触碰生殖器官才能让人获得性欲,而普通的性交无法让人满足。如果性生活时,男方希望通过做爱来达到自慰时的刺激感,这就会引发问题。如一位男性所说:“我的妻子总是抱怨, 我根本不是在爱抚她,而是在蹂躏她。”\\
色情观看者一旦接受了色情的性教育,就无法在现实生活中维持对伴侣的性兴趣和性欲。一名男性说:“我看色情时,可以不停的翻页,点击新的网站,看不同的女人来保持自己的性兴奋。但和女友做爱的时候,我老觉得无聊,容易分神。毕竟,就算她再怎么好,也只是一个人。我要是把全部注意力放在她一个人身上,我就会性欲全失。”\\
色情成瘾者常常表示,即使他们可以在性生活中保持正常的状态,但是结束后,他们还是会觉得失望,对伴侣不满。真正的性关系是需要大量的“工作”的:要激发、保持性欲,要和脑海中不断闪现的色情画面作斗争,要有精神和肉体的双重刺激才能得到令人满足的高潮。一名30岁的女性,小时候就开始对着爸爸的色情品自慰。她说为了要达到高潮,我要把自己幻想成色情里面某个女主角:我根本不知道在现实生活中自己的这个身体应该如何放松地去享受性生活。虽然我可以让自己达到高潮,但是我没法和伴侣一起产生共鸣,我们的 性生活不过是貌合神离而己。他是个好人,但我觉得我只是在有需求的时候利用了他的身体而已,这对他来说很不公平。我担心在生活中我变成了一个旁观者,无法真正投入到感情中,无法和真心爱我的人一起享受性爱。”\\
6.“我让我的伴侣不快了”\\
当35岁的查克还是个单身汉时,他每天花一两个小时看色情杂志和网页。他最喜欢浏览业余的色情网站,里面特写的都是相貌平平的普通女人裸体:查克说:“我喜欢上了年纪的女人,有点赘肉的那种,这样看起来才真实,就好像那种你会在街上、超市里碰到的女人。”查克交了女友之后,曾经一度停止观看色情,但很快又旧习复发,因为边看色情边自慰已经是他“长久以来的老习惯”了。\\
查克和许多刚开始恋爱的色情观看者一样,觉得色情是无伤大雅的小事。对他来说。看色情不过是爷们儿都会做的事。他没有预料到女友发现之后受到的伤害。某天,女友恰巧发现了他电脑中的色情文件夹。她“火冒三丈”,威胁说,如果他不立刻停止观看色情,她就要分手。女友的这种要求让他进退两难。他确实不想让女朋友生气,但同时他也不想要放弃色情。查克的这种困境是好多色情成瘾者都会经历的,而他们不约而同地选择了一个两全其美的解决办法:他答应女友再也不会看色情,好让她放心,同时采取更隐秘的办法偷偷地看色情,更加小心翼。\\
尽管查克似乎暂时解决了色情带来的问题,但是他发现,两人的关系发生了微妙的变化:女友开始不信任他,老是疑神疑鬼。他总是担心,女友会偷偷监视他,趁他不家的时候翻他的电脑,或者乱翻他的东西寻找罪证。\\
色情瘾者的伴侣未必能帮忙解决实际问题,但他们迟早会怀疑对方的忠诚度,质疑这段感情。一名男性告诉笔者:“我老婆注意到我们之间的关系发了变化。她质问我是不是有了外遇。当然啦,我肯定说没有,因为我确实没有找小三。一天晚上,我在电脑前看色情的时候,恰巧被她撞见了,她问我是不是对色情上了瘾,我马上瞎编了一些理由,说自己在看新闻的时候,几个色情网站自己弹出来的,刚好被她进房的时候看见了。她当时看起来好像是相信了,但是自从那晚起,她跟我在感情上越来越疏远,肉体接触也越来越少了。我非常纠结,不知道该继续撒谎,还是该诚实地告诉她我需要色情。我想对大部分已婚男人来说,最大的恐惧就是让妻子发觉他们是多么软弱。我当时就是处在这两难之中。”\\
只要有一方是色情成瘾者,两人之间的性亲密度就肯定会受到影响,毕竟,色情旨在吸引个人的性精力和性注意力。色情观看者的伴侣常会抱怨对方忽视了自己的性需求。在一段爱恋关系之中,性亲密感的减少就是矛盾爆发的前兆。詹妮弗·施耐德博士是性瘾专家,她通过研究发现,70%有网络色情问题的伴侣中,至少有一名伴侣对性关系兴趣不大。而有悖于主流观点的发现是,通常情况下,是色情观看者,而非其伴侣,才是对性关系提不起兴趣的一方。\\
色情观看者与其伴侣的性生活也会受到影响。不能满足伴侣的需求、不能在性生活中表达爱意,这些都会让色情成瘾者在性生活中显得苛刻而冷漠。贾斯汀告诉笔者,他无法对妻子表达爱意,这让妻子很是委屈。“我不过是把性爱当做一种程序化的东西,绝对不会想到要把它想象成什么神圣的感情交流。我把妻子晾在一边,她就会委屈,但我要和她做爱,她又觉得我是在强迫她。就这样,我们的性生活非常不和谐,这成了我们两个人摩擦的主要原因。”\\
7.“自我感觉非常不好”\\
色情最严重的后果莫过于它对自尊的腐蚀。个人的自我评价建立于自尊和正直品格;同时,也和个人的社交生活和人际交往有关。如果你发现自己认为“我真的不了解现在的这个自己了”、“我讨厌我自己”、“我成了一个伪君子”等,你的自尊就岌岌可危了。一位男性回忆:“我不喜欢自己变成那样。我是个骗子,满嘴谎言。对色情的迷恋让我厌恶自己到了极点。”\\
色情带来的恶果,不论是不满的伴侣,或是性功能问题,还是差劲的工作表现,都会极大打击个人自尊。自尊心受到伤害的情形。类似于踢足球时受的伤。开头几次,我们还可以装作没什么事,但是最终,所有的大伤小伤加起来,会迫使我们不得不提前出场。\\
如果你饱受羞耻心的困扰,总是生活在恐惧之中,还必须时刻防备,不让周围人发现自己的这个小秘密,那么,想要维持良好的自我感觉是非常困难的。尼克这样解释:”一方面,我有很深重的罪恶感,但是另一方面,我又觉得看色情是心安理得的事,我还会安慰自己说,‘反正我就一人渣,我要用行动来证明这一点’。我陷人这个恶性循环中,根本无力改变现状。”\\
布拉德认识到,他的自卑心理导致他和妻子宝拉发生无数次争执。“我心底有无数怒火,又有深深的愧疚感,”他说,“我会把这些不满都发泄到妻子身,然后她就爆发了。其实,长久以来我都有很强的罪恶感,因为我一直做着我不该做,也不想做的事。当时的生活状态,根本不是我想要的。我厌恶自己,为什么就是不能下决心去改变,过自己想要的生活呢?”\\
色情毁灭自我有另一种方式就是:色情与正常的道德价值观和宗教价值观背道而驰。举个例子,美国有个叫“承诺守护者”的基督教福音派组织,旨在提倡男性在家族和社会生活中发挥如“上帝般慈爱的影响”,而其中超过50%的的男性都有色情问题。当一个人成为道德领袖时,还在同时做着违背自己价值观和行为规范的事,这种分裂的滋味肯定不好受。一位男性称:“色情让我在教堂几乎抬不起头来。我不为人知的秘密生活,我内心的自责,让我根本尤法成为一位真正的行为楷模,这让我备受折磨。”\\
罗波说,当他发现自己做的事与自我定位格格不入时,觉得自己很堕落。他满是悔恨地回忆道:“妻子带着孩子们去购物的时候,我说我有工作要做,其实我整个下午都在网上看淫秽的内容。后来,我听到妻子在家门口倒车的声音,就站起来去帮她提东西,悲剧的是,我忘了关电脑。”\\
“我女儿走进房间。看到了屏幕上的内容,就大叫‘妈妈!’妻子走进房间,质问:‘这都是什么东西?’我当时羞愧无比,但我还是对妻子撒谎说,那只是网络搜索时弹出来的网页而已。我真不希望让她知道我会在网络上搜索这种东西,也不希望她知道我会做这种事。这件事就是一个彻底的灾难——对女儿来说,对妻子来说,对我来说。我自认为在生活的很多方而我都是一个正直的人,但我在色情世界里就好像换了一个人似的。”\\
“从那时起,我的秘密生活被迫曝光,妻子肯定会在心里怀疑:‘这个人是准?这个人到底是谁?’讽刺的是,就在这件事的前几天,我还记得要妻告诉岳母说我是她认识的最诚实的男人。当时我在心底呐喊:老天,我其实是你认识的最龌龊的人渣。”\\
在某种程度上,色情观看者会意识到,色情总是包含对女性、儿童以及有色人种的践踏和压榨,这点也会增加观看者的自卑感。一位男性说:“我知道在制作色情电影时,演员都是被压榨的。有些时候,我都能感觉到电影对人的剥削和压迫,只要看里面女人的眼神和表情我就知道了。男孩子就是这么矛盾。明明知道录像里面的事情.都是不对的,还会去看。”\\
有些色情观看者喜欢看一些口味特别重的内容,比如性虐待、强奸、身体伤害、儿童性侵、人兽特写等,这些人往往会在心理上承受巨大的罪恶感,导致个人无法真诚地与他人交往,无法获得内心的平静,更无法形成正常的自我价值观。伦喜欢看父母性虐待孩子的影像,他告诉笔者:“有时候,我心里也会觉得自己道德败坏,没有道德操守。我会从这种一般人无法接受的内容里面得到快感,你说我是个什么样的人?”\\
色情还会诱使观看者迷恋上性虐待和性犯罪行为,这对于个人自尊的打击是灾难性的。一名男性曾经双眼含泪、浑身发抖地告诉笔者,色情使他成为了一个“变态”,一个’‘视觉强奸犯”。\\
8.“我在做一些有风险的危险举动”\\
“色情会带来你想要的东西。但是它也会带来你起初根本不想要的东西。”这是笔者经常从色情观看者口中听到的话,有时候他们指的是自已后来喜欢的极端色情内容,有时指的是自己在现实生活中性趣的变化。\\
通过观看某些重口味的色情,比如一夜情、性暴力、无安全措施性爱等内容,个人在潜意识中会不自觉地相信,在现实中这种行为也是合理的。看到的行为我们就会想要去模仿,这是人之常情,尤其是镜头中的那些主角看起来好像还很享受。但是,如果,一个人经常边看重口味画面边自慰的话,个人会习惯于把注意力都放在色情带来的刺激感上,忽视色情的粗暴本质。一名男性说:“色情瘾不断加深之后,我越来越喜欢那些变态的暴力色情画面,这些曾经会让我反胃的东西已经成为了我最喜欢的性幻想对象。”观看色情时体验到的性高潮会让人失去判断力,让人做出不敢做的事。另一名男性说:“色情看到情浓时,理智什么的都是浮云。”任何一种危险型性行为都会刺激大脑分泌化学物质,如多巴胺和肾上腺素,这些化学物质会进一步促进性欲,让人产生一种强大到无所不能的良好感觉。如果在观看色情的同时摄入一些含有刺激神经的物质,如酒精、兴奋剂、可卡因,也会加大个人做出性虐待等高危性行为的概率。\\
沉溺于高危性行为的色情会让人越陷越深,正如伦所说,这就像是被自己的本能好奇心牵着鼻子走一样。他告诉笔者:“为了满足自己的好奇心,我会去看任何内容的色情,捆绑、乱伦、轮奸、性虐待这类奇奇怪怪的内容,这些内容在一般尺度的色情中是看不到的。这是一个渐渐习惯的问题:我刚开始就是想找一些新鲜的内容看,对这些内容腻了之后就觉得无聊了。如果你每天都可以拿到免费的巧克力曲奇,起初你还会觉得惊喜,时间一久,无论曲奇多么美味,你还是会腻,接着你就会想要尝尝其他的东西了。相同的道理,我也不需要不停去找新鲜的色情,事情就这样不断发展下去了。”\\
没过多久,看同一种类型的色情就无法让人满足了,观看者就会想要探索更多极端的色情来满足自己不断膨胀的欲望。色情观看者都会意识到,为了不断得到新鲜的性快感,自己就需要不断适应新类型的色情所带来的震惊感和羞耻感。詹姆士是一名大学生,他说:“我需要更加变态的内容,要带着一点危险的因索,这样我才能得到想要的快感。只要一想到内容真的是很下作,我就会像打了鸡血一样兴奋。而且,现在想要找到那种真枪实弹、粗俗下作的内容也并不难。我知道这样不好,但是我就是被那种性快感吸引了,越陷越深。”\\
观看过多的极端色情会使人对现实产生错误的估计,误以为那些侮辱性的暴力性行为在现实中能被人接受。某些色情观看者会觉得,如果自己不尝试一下色情中的行为,就会错过一些刺激的快感。而且,色情不会描绘性行为带来的潜在问题,色情观看者可能不了解性行为的风险,以为在现实中做极端的性行为不会引起任何不良后果。\\
习惯性地将色情作为性欲发泄途径也会让人对暴力产生麻木感。因为色情教会我们,只要把人看成是性物体就好,而不是有情感、有需求、有基本权利的个人。研究证明,有暴力倾向或者无法自控的人如果常常对着色情自慰,他或她施行性暴力的可能性会增加。\\
在房间或是电脑里收藏色情内容也是有风险的,这点我们也可以从各大新闻媒体中得知。笔者曾与几名因在电脑中收藏儿童色情而锒铛入狱的男性交流过。他们之中有人是“恋童癖”,希望或者已经与经与未成年人发生关系,他们利用儿童图片作为激发性欲的方式;也有人不过是好奇心作祟,或者觉得其他形式的色情不够刺激;还有一些人是意外接触到了儿童色情。尽管这种情况并不多见,但是儿童色情会在个人不知情的情况下就出现在其电脑之中。邮件、网站、网络聊天工具、网络社区、资源共享网站等网络资源都可能会自动在个人电脑中添加儿童色情信息。\\
专家证明,网络色情中20%一25%的内容都是关于儿童色情的。裸体、性感的孩子照片,进行性行为的孩童(18岁以下)是孩子被性虐的影像证据。如果有人通过任何途径拥有儿童色情资源,那么不管你有没有把自己当作恋童癖,这种行为都算是犯罪行为。执法者没有那么多精力来揣摩一个人的性幻想,他们只会通过证据来定案。\\
个人获取色情的地点也很重要。不管你看的是什么内容的色情,只要是在工作地点的电脑上看的话,都属于高风险行为。网络匿名和个人隐私不过是美好的谎言而已。网络上进行的大部分活动都是有迹可循的,即通过集体网络系统追踪。南希·弗林,《网络政策手册》的作者这样写道:“如果你在办公室上班,你就要做好被监视的准备。”笔者采访过的部分色情观看者就是因为在上班时间看色情而丢了饭碗,他们以为在办公地点看要比在家里看安全。他们这是被色情迷了心窍,而没有意识到,如果被发现,他们就会被炒鱿鱼。\\
9.“我对色情上瘾了”\\
很多人以为,上瘾这种词只能用来形容磕药、酗酒的后果,或许没有多少人会想到,某些行为也会让人上瘾,比如通过看色情来寻求性快感。如果一个事物既能让人产生快感又能缓解痛苦心理的话,任何人都可能会上瘾。同理,赌博、购物、观看色情,这些事物都会让人上瘾。\\
本书第一章中提到,观看色情会改变大脑分泌的化学物质。色情会刺激大脑中的快感中枢,引发快感荷尔蒙和化学物质的大量分泌,比如多巴胺、内啡肽、肾上腺素和催产素等,这些物质会改变个人的感受。一些科学家将观看色情时大脑分泌的化学物质变化等同于吸食可卡因时大脑产生的变化。大脑扫描显示,色情成瘾者的大脑和平常人有着一定的差别。大量证据表明,色情通过感官中枢进入身体后,会像毒品一样改变人体的生物系统。\\
一个人若是对色情上了瘾,神经生理和体内的化学物质都会发生变化,这会改变个人看色情的初衷。个人不再是利用色情来获得快感,而是身体已经开始习惯色情,需要色情。相对于物质上瘾,行为上瘾的过程比较难识别,因为这一过程在生理层而上发生,周期十分缓慢,就好像人们无法察觉体内的细胞更新一样,大家也很难察觉每次看了色情以后体内的生理变化,因此,人们总是难以发现自己的上瘾倾向,等到醒悟时,人已被牢牢困在了陷阱之中。\\
要确定“我是否对色情上瘾了”,需要自我评估。色i情成瘾者都具有三个关键性的特征。色情成瘾者:\\
对色情有强烈的、持续的渴望。\\
无法控制自己,想要戒除色情,但每次都以失败告终。\\
尽管明白色情会带来恶果,但还会继续观看.。\\
渴望(Craving),无法控制自己(Can’t control it),不顾后果、继续观看(Continuing despite consequences),这三个以C开头的单词可以帮助读者记忆色情成瘾的三大特征。\\
色情成瘾后,个人会失去自控力。要不要看色情?什么时候看?看什么样的色情?怎样看?看多少?这些问题已经不在个人能力范围之内了,起决定性因素的是人体内的化学物质。不知不觉,个人就发展到只能通过色情来获取性快感的地步。罗波自青少年时期就开始天天边看色情边自慰,但他从未觉得这有任何不妥。也没有意识到自己的行为带着强迫症性质,对他来说,这只是一件他喜欢做的事,而且习惯了天天做。\\
真正让罗波吃惊的是,即使婚后他和妻子保持着活跃的美满性生活,他还是无法抑制想要观看色情的欲望。用他的话来说,就是:“尽管我们几乎每天都会做爱,但是我还是需要色情这服药。我不买色情杂志了,而是经常去成人书店,那儿有放色情录像的小隔间,我在那里消磨了很多时间。无沦我和妻子发生过多少次性关系,我还是需要看情来获得那种间接的快感来满足自己。”\\
玛丽在丈夫过一世之后才开始定期观看色情。刚开始看色情,不过是为了让自己从悲痛中解脱出来,但这很快就成了她每天哄孩子睡觉后必做的功课。“刚开始看色情不过是件自发的事,但很快这就成了一种强迫性的习惯。我对色情产生的那种强烈欲望,就跟毒瘾发作一样。哪天晚上要是没看色情,就会感觉不自在。”\\
随着个人对色情欲望的加强,色情观看者也会不断想出新的获取色情的渠道,存储色情的办法,以及忙里偷闲看色情的机会。每天上最喜欢的网站看更新的图片,下班后顺道去逛逛成人书店,三更半夜时看色情,这些都是色情成瘾者常干的事。对色情的欲望膨胀到一定程度后,其重要性胜过努力工作、真诚恋爱、家庭责任、维护健康等人生目标。一位男性称:“我不仅仅为自己创造观看色情的机会,我只要一有机会就会去看。”\\
色情成瘾者会发现,即使他们想要控制自己观看色情的行为,却总会以失败告终。比方说,一位男性告诉笔者,他想要停止观看色情,但最多就只能撑三四天。\\
伦是最近才发现自己对色情的控制力少之又少。他告诉笔者:“我对着电脑看色情的时候,时不时地会意识到:我真的做过头了,我在这上面浪费太多的时间了。我清空了电脑收藏夹,删除了电脑里所有的色情链接和色情图片。但是坚持不了多久,最多就几天,我就会反悔:我真不该那样做,然后我又会把所有的东西都找回来,从头开始,从网上下载所有的资料保存起来。”\\
海克特是一名医学博士,他认识到自己没有培养健康的抗压方式,而是习惯在色情里寻找慰藉。他说:“要写论文还有一大堆事情要忙,但是我又去看色情打手枪了。然后,我才会发现自己又浪费了多少时间,接着我就会狠狠地骂自己:‘我到底在干什么?’”\\
有时,色情瘾者还会不断追求更大量的新鲜内容。前文提及,色情观看者对某类色情迟钝了以后,就要寻找其他更刺激的色情来达到以往的快感。也就是说,大脑对色情提供的感官刺激变得越来越迟钝。个人需要越来越重口味的色情才会“有效”得到性快感。罗波解释说:“我对刺激性的事物越来越迟钝了。我那时还觉得《阁楼》和《花花公子》杂志里面的女人很诱人,但是时间一久,我需要更加淫秽色情的内容。我就好像有了毒瘾一样,需要量更大、劲更猛的毒品。我的色情瘾从刚开始看的裸体女人,到模拟做爱,到真实做爱,到群交,到同性做爱,到与未成年人做爱,这个链条好像会不停地延续下去,完全不受我的掌控。”\\
对色情上了瘾的人很难控制自己的行为,原因还在于:对他们来说,戒除过程非常难熬。很多人告诉笔者,他们在戒除色情的过程中,出现了坐立难安、压抑抑郁、难以人眠、暴躁易怒等症状。一位男性说:“我努力想要戒除色情瘾、但是我真的做不到。我没法入眠,甚至身体打颤。我知道色情里面的女孩子受到非人待遇,但我就是控制不了自己。我觉得自己就是个悲剧,深陷其中,但无能为力。”\\
色情已经给贾斯汀带来一系列恶果:色情影响了他的性生活,导致他和妻子离婚,让他与别人越来越疏远,但他不管不顾,仍在继续观看色情。尽管色情给他带来如此之多的伤痛,他还是无法停止观看色情。他说:“这对我来说就好像是一场噩梦,看色情就好像在吸毒,而且是我现在就要吸这个毒。我有试过去戒,但做不到。我想要边看色情边自慰的欲望太强烈了,让我非常恐惧。”\\
应该认真对待,还是轻描淡写?\\
色情的负面影响会不断警醒观看者的良知。其实这足以激励观看者下决心做出彻底的改变。当他或她意识到问题所在,就会想要逐步将色情永远驭逐出自己的生活舞台。但是,很少有色情成瘾者能真正下决心做出改变。大部分人告诉笔者,当问题发生时,他们的本能反应就是要忽视它,假装一切正常。\\
这种反应和烟鬼的反应一样:即使抽烟者已经意识到抽烟会引起呼吸系统疾病和睡眠问题,也会在他和朋友、爱人之间产生隔阂,他们还是不愿意相信,这种能够产生快感、补充能量的事物,会产生巨大的负面影响。\\
比方说,那天罗波的妻子和年幼的女儿购物回家后。发现了他在看淫秽色情,这让罗波很难受。但是,他并没有将问题归咎于色情;相反,他只是质问自己为何粗心大意忘记关掉色情网站。罗彼后来承认:“其实那件事已经很清楚地说明我有严重的色情瘾,但我当时甚至都没有意识到这事做得多出格。我那个时候工作勤奋,收人丰厚,就觉得自己有权力享受色情。我安慰自己说,我是因为性欲比较强才会需要色情。然后我就会想,只要把色情藏好了,下次多留意,就不会有事了。”\\
色情观看者会费尽全身解数来避免直视自己的色情瘾问题。也许,他们会在一段时间内停止观看色情,希望自己对色情的欲望减弱后,问题自然就会消失;也许,他们还会改变自己看色情的方式,变换观看色情的类型。受到他人质问时,他们会满口否认,不愿意谈论此事或承诺说一定会戒;抑或他们会如刺猬般防备,言语激烈地反驳、斥责那些指出自己的问题的人。\\
色情观看者所采用的这些对策,只会导致战时越拉越长,加重问题的严重性,或许,在此期间。他们可以视而不见色情的恶果,不用时刻忍受良心上的谴责。但这并不是长久之计,随着时间的推进,这些后果只会越来越严重,越来越明显,越来越恶化。色情的各种恶果之间会相互促进。比如说,长期的色情瘾问题会加剧孤独感和抑郁症,而对刺激色情的追求也会促使观看者去搜索非法色情,或进行其他性犯罪活动。\\
这就是发生在罗波.身上的悲剧。他因为在电脑上观看儿童色情。最后落得锒铛入狱。罗波告诉笔者:“我的色情瘾发展到后来,就想要下载未成年人的性爱图片看。我那时一直说服自己,说肯定不会有事的;我还安慰自己说,那些色情里的女孩子都至少满16岁了。我从来没有直视过色情带来的问题,所以色情毁了我最珍惜的一切。我失去了贤惠的妻子,两个漂亮的孩子,一份体面的工作,一幢高档别墅。我就是自作聪明,不见黄河不落泪。”\\
如果个人长期忽视色情带来的严重问题,拒绝而对现实,这些问题只会随着时间发酵。如果不立刻着手解决,那么色情给自己以及身边每一个关心自己的人都会带来更加严重的后果。\\
第五章 备受煎熬的伴侣\\
半夜时分,22岁的梅根醒来,嘴角还带着甜蜜的笑意。在一个热带度假胜地和新婚丈夫杰西渡过的蜜月。五天五夜无止境的、充满激情的性爱回忆还历历在目。她下意识地伸出手,想要在另一半床上寻找杰西温暖的体温。但是,他不在。她想,也许杰西去了厨房,她便急急地下了床去找他。当她经过一间空着的卧室时,发现房门紧锁,门缝透出微弱的光。梅根想他可能在房里睡着了,便轻轻打开门走了进去。那时,杰西就在那儿,背对着她,蜷缩在电脑前,快速点击着鼠标,全神一贯注地盯着屏幕。\\
梅根走近了些,越过他的肩膀,想看看是什么让他如此着迷。“我简直不敢相信,”她后来回忆说,“杰西居然在看裸体女人张开双腿的图片。这简直就是晴天霹雳,我之前根本不知道他喜欢色情。在那一瞬问,我们蜜月里的性爱就一文不值了。意识到我在他身后时,杰西发火了,说我打搅了他。我质问他为什么要看色情,他说这没有什么。他还说,我要习惯这事,不要小题大做。听了这话,我顿时觉得五雷轰顶,只想要吐。”\\
当梅根发现杰西喜欢色情时,她绝望了。尽管杰西觉得自己看色情对梅根没有彩响,但事实并非如此。和杰西一样,很多色情观看者总认为自己可以把色情和爱恋区分开来;即使被伴侣发现,他们也希望伴侣可以接受。至少是可以容忍这件事,不要小题大做。色情观看者这种单纯片面的想法低估了色情对伴侣的影响,导致他们无法真心体会伴侣的感受。尽管杰西知道大部分女性都不喜欢色情,但是梅根所受的伤害之深远出乎他的预料。杰西说:“第二天,梅根收拾了行李说要离开我,我惊呆了。她说,除非我尊重她的感受,放弃色情,不然她要和我离婚。”\\
了解彼此的经历、体谅彼此的感受,这对维护一段健康、真挚的爱恋关系来说非常重要:.观看色情和真心满足的伴侣,这两者难以并存。对色情的迷恋会损害一段情感的基础价值观,包括诚实、忠诚、亲密、尊重、信任和爱意等因素。谈恋爱的时候,女孩子们期盼男生能够遵守这些价值观,而男性观看色情对女性而言就是一个危险信号:她是否已经没法引起伴侣的性兴趣和注意力了?而色情观看者常用谎言和欺骗来掩盖自己的行为,从而破坏了伴侣间诚实、信任和尊重等情感基础。\\
杰西并不是有意伤害新婚妻子,但梅根还是伤透了心,也失去了对丈夫的信任。大部分色情观看者往往抱着侥幸心理,但在现实中,一旦有一方沉溺于色情,就一定会给爱恋关系带来影响。\\
本章旨在回答以下问题:色情成瘾者的习惯是如何影响伴侣的?女性从刚开始并不知情(至少不知道伴侣成瘾已深)到后来发现问题无法回避时,会经历怎样的心路历程?为什么色情成瘾者的伴侣会有如此反应?由于大部分色情成瘾者的伴侣是女性,所以异性恋中的女性将是本章的主要讨论对象。笔者通过咨询和调查发现,异性恋中女性的感情机制,与大部分色情成瘾者的伴侣表现高度吻合。\\
笔者认为,与色情成瘾者保持爱恋关系的女性,通常会在心理上经历四个基础段:这些阶段之间可能存在着重合现象,基于爱恋关系的变化以及色情成瘾者对女性的诚实度变化,女性也可能会循环体验这四个阶段。这四个阶段即:\\
第一阶段:蒙在鼓里\\
第二阶段:大吃一惊\\
第三阶段:情感伤痕\\
第四阶段:设法解决\\
下文将具体分析每一阶段中伴侣的心路历程。\\
第一阶段:蒙在鼓里\\
和梅根一样,许多女性起初并没有察觉到伴侣喜欢色情;女性通常会一厢情愿地相信,伴侣是忠诚的,在性关系问题上也是清白的:大部分女性会理所当然地认为,伴侣知道遮遮掩掩地看色情跟搞外遇的性质一样严重。女性对伴侣无条件的信任,对色情后果的无知,伴侣隐瞒看色情的事实,这些因素都会让女性长期蒙在鼓里,时间长达几个月甚至几年。\\
但是,无知并不是一种幸福:随着时间的推移,尤其是在伴侣的色情瘾越来越严重的情况下,女性只会感到更加困惑,更加压抑。\\
“有什么事情不对劲——但到底是什么呢?”\\
想象一下,一个人觉得不舒服却不知道病源;再试想,自己想要升迁却屡遭失败,没有人告诉自己原因。这就是女性受到色情问题的波及,却丝奄不知情时所经历的心理感受。她们的第六感认为,这段关系似乎出现了问题,但她们无法确定问题所在。\\
在女性不知情的情况下,首先出现问题的通常是性生活。一些女性会觉得伴侣是在强迫自己进行性生活,这对她们的心理和肉体来说都是一种折磨;另一些女性会觉得伴侣抗拒性生活。现年53岁的家庭主妇黛比回忆当初新婚时,自己就是不明白为什么身体健康、魅力十足的老公罗吉不喜欢做爱。“婚前他还是很喜欢做爱的,”黛比说,“但是结婚之后,我们的性生活就变得不稳定了,做爱的时候还要我主动.”虽然两人的性生活和谐,但是黛比就是不满足,觉得性爱时两人不够亲昵。她以为这是因为两人既要上课又要打工,都太累了。有时,她又会担心,是不是罗吉不再觉得她性感,对她失去性趣了。“我真的想不出来,”她说,“我以为男人永远喜欢做爱的。他不想做爱,我就会胡思乱想,是他有问题还是我有问题?”\\
对伴侣色情成瘾毫不知情的女性,最初总是会在伴侣的情绪变化和兴趣转移中发现一些端倪。28岁的科伦是一名美容师,她和强尼婚后几年中就有过相似的经历。两人的性生活一直很和谐,但男方突然不愿意做两人原本喜欢一起做的事情了,比如一起去教堂。“我们对上帝的信仰、对宗教的虔诚,是我们相识相知的原因,”科伦说,“但他忽然不愿意去教堂了,说他不再确定自己对上帝的忠诚了;而且,他变得越来越暴躁了。在那之前,他一直很温柔,对我很好。我当时不知道他到底有什么问题。”科伦当时不知道强尼有色情瘾,还以为他的精神方而有了问题。她觉得伤心,也对丈夫充满了怨念。\\
黛比和科伦根本不清楚,她们丈夫行为举止的改变正是因为他们偷偷地沉溺于色情。妻子们从来没有在家里看到色情品,也没有听丈夫说喜欢色情,她们都是事后才发现了真相。而在此期间,她们的婚姻早已千疮百孔,岌岌可危了。婚姻并没有如女性期盼的那样,越来越亲密,越来越幸福。妻子们感到困惑无措,不知道色情正是症结所在。\\
“我还以为这没有什么大不了的”\\
即使有些女性知道伴侣观看色情,却仍然不清楚伴侣沉溺的程度以及色情对于伴侣的重要性。她们或许对伴侣喜欢色情的事略知一二,但完全不清楚色情会给两人的生活带来多大的影响。当汉娜让男友理查德——她后来的丈夫——搬进来跟她同居的时候,汉娜在他的行李里面发现了满满一盒子色情录像、书籍和杂志。不过,汉娜很开放,她当时并不在意,之后也没有和理查德提过这件事。“我知道有些人要靠色情来得到性快感,就好像有些人一欢用性爱玩具一样,所以我当时也没有多想,”汉娜说,“我们的性生活一直很完美,很开放,不断有新招式。我一直觉得他看色情没有什么坏处。直到很久以后,我才意识到,其实那一盒子色情就是一盏警示灯啊!”\\
婚后一年,汉娜渐渐发觉,作为大厨的理查德晚上去工作,但日间总是窝在电脑屏幕前。他越来越不负责,越来越疏远汉娜,也越来越没有爱意了。他的改变让汉娜很是困扰,她开始担心两人的关系是不是变质了。“尽管之前我就知道理查德看色情,但我当时真没想到他会那样长地看色情,也没有想到色情就是一切问题的起因。我们结婚的那个年代,还没有一个词叫‘色情瘾’。我从来不知道看色情是不健康的行为,而且最终会毁掉一段感情。”\\
26岁的宝拉是一名秘书,在婚姻早期她也知道丈夫布莱德喜欢色情。丈夫向她坦白的时候,她并没有觉得有什么不妥。“我年少的时候也看过色情杂志上的性感图片,我没觉得那有什么不好,”她说,我当时只是想,色情是很多爷们都会看的;而且,我以为这事只在我们婚前发生过几次。我不想让他有不必要的负罪感,所以我轻描淡写地回答。‘谢谢你的坦白。别担心,没事的。’”即使后来布莱德在性生话时要求宝拉做一些出格的举动,而且越来越不关心她的需求,宝拉也没有意识到色情就是一切问题的起源。\\
汉娜和宝拉都天真地低估了色情的潜在威胁。她们和万千女性一样,以为色情是每个爷们儿都会看的,没有什么大不了。出于维护关系的考虑,她们觉得最好还是不要”无事生非”、“小题大做”的好。正是因为她们从刚开始就排除了色情这个因索,所以汉娜和宝拉很难把婚姻问题和丈夫观看色情这件事联系起来。因此和那些不知道伴侣沉溺于色情的女性一样, 这些知情的女性也同样摸不着头脑,只是空担忧两人关系中的变化,却束手无策。\\
“是不是因为他有色情瘾?”\\
陷入情感困境的女性迟早都会开始行动,寻找解决问题的办法,这种行为的动力则来自于她的忧虑、她对这段情感的珍惜。女性或许会开始阅读探讨情感问题的书籍,咨询心理医生,向闺蜜倾诉心事。\\
在一些情况下,女性的努力探索会让她意识到伴侣可能沉溺于色情。黛比告诉笔者,她足足忍受了感情沉重、性生活不和谐的十年婚姻之后,才意识到色情可能是导致自己婚姻问题的罪魁祸首。一位朋友在倾听了她的问题之后,质疑她的丈夫是不是有色情成瘾问题。“那天晚上我回到家就开始质问罗杰。他脸色惨白,彻底崩溃。他告诉我:‘是的’,他说他之前就看过几次色情,但已经没有再看了;而且,他还向我保证再也不会看了。”\\
黛比终于找到了所有问题的答案,而且她也相信罗杰不会再看色情了,这时她心中的一块大石终于落了地。但是,她没有意识到的是罗杰和千千万万沉溺于色情的人一样,因为太羞愧而不敢承认自己陷入色情陷阱不可自拔,根本没有勇气告诉伴侣真相。起初受到质问时,大部分色情观看者要不矢口否认,要不粉饰真相。“起初受到质问时,大部分色情观看者要不矢口否认,要不粉饰真相。”根据罗杰的话来推断。”黛比说,“我猜他也就是买过几次《花花公子》,我当时根本不知道他的色情问题有多么严重。”\\
而对黛比的质问,罗杰承认了自己观看色情的事实,但是他并没有告诉妻子,他每天都在看色情,而且边看边自慰。黛比对色情潜在的负面影响所知甚少。再加上罗杰并没有开诚布公,在此后的几年中,黛比又被蒙在鼓里。她告诉笔者:“那晚之后,我继续竭尽全力想要让他恢复正常,想让我们的婚姻回到正轨上。我假装我们之间还是有默契的,但情况根本没有改善,之后我就彻底地迷茫了。”\\
怀疑和困惑同样折磨着苏,一名45岁的银行柜员,尽管她早就知道丈夫鲍伯曾经一度痴迷色情。两人结婚20余年,在头几个月的时间里,他们曾经试过在做爱前一起看色情录像。刚开始苏还觉得有点刺激,到了后来就觉得没意思。她觉得色情会损害两人之间的亲密度,而她也以为,鲍伯会和她一样彻底放弃色情。\\
但是去年发生的一些事情,让苏怀疑鲍伯又开始看色情了。“我接到一个电话,说鲍伯欠了一盘色情录像的钱。”她说,“我真的很震惊,鲍伯对此倒是一笑而过,辩解说这只是一个误会,他会处理的。但几个月之后,我又接到了类似的电话,说鲍伯又欠了好几盘色情录像的钱。小样儿,如果只是意外,怎么会接二连三地发生?”苏越来越觉得压抑,就去质问鲍伯,但是他含糊其辞,说些不靠谱的话。“我就像是一个酒鬼的妻子,闻到丈夫的呼吸里而有酒味,却听到丈夫说他很清醒,”她说,“我觉得很困惑,不知道该相信谁。鲍伯跟我保证说他没有看色情,但是我心里很清楚,有什么事情就是就是不对劲。”\\
尽管苏心里很清楚丈夫的回答不靠谱,但她还是没有把丈夫的老底掀出来。“我不愿相信他还会在看色情。这事情想想都觉得可怕。我任鲍伯给出无数解释,听他承诺说再也不会看了。主要是我不敢相信鲍伯竟然会对我撒谎;而且我也不愿意面对现实,承认我们的婚姻已经有问题了。”\\
很多的女性跟黛比和苏一样,打从心底里信任丈夫或者男友。她们会千方百计地说服自己,不愿意把伴侣看成是沉溺于色情的人,一个竟然会刻意隐瞒,肆意欺骗自己的人。这种自欺欺人的心理在女性中很是普遍。32岁的酒吧女招待詹尼特说,“虽然男朋友一再否认,但各种证据表明他仍然沉溺于色情,诡异的邮箱账户,卡车里面的《花花公子》杂志,裸体女性钥匙吊坠。尽管事实摆在眼前,我还是选择了自欺欺人:我就是不相信他会喜欢这种垃圾。”\\
一些情况下,女性会撞见铁i正如山,表明伴侣有严重的色情问题。某天,汉娜提前下班回家,发现丈夫理查德坐在卧室的电脑前。“房间里灯光很暗.而且一股腥味,”汉娜说,“垃圾捅里面扔着成堆的卫生纸,这太明显了。我简直就不敢相信,质问他,‘你是不是对着色情打手枪’‘你是不是在网上看色情了?’,理查德只是不断否认,之后他就再不愿提起这件事了。我很郁闷,也担心他的色情瘾比我想象的更严重。我后来当这件事没发生过,免得他觉得尴尬。”\\
很多色情观看者的伴侣们和黛比、苏、詹尼特、汉娜有过类似的经历,她们告诉笔者,由于无法找到症结所在,她们一度相当困惑,都快被“逼疯”了。她们第六感认为发生过的事情,伴侣矢口否认,这只会让女性越来越抑郁,越来越焦虑。由此引发的各种身体疾病、如头痛、失眠也会影响她们。内心的情感压力会让陷入僵局的女性变得越来越脆弱越敏感。\\
再回顾当时的情况,很多的女性都说她们当时根本不知道如何判断伴侣是不是沉溺于色情。在美国社会中,很少有人会公开谈论成瘾时的症状,因此,女性很难获取足够的信息来证实自己的猜测。许多女性告诉笔者,她们女性告诉笔者,她们希望自己曾经关注到伴侣行为中的某些改变,因为现在看起来,这些改变直指色情问题。如果时间能够倒转,她们会尤其注意伴侣的以下行为:\\
沉溺于色情的迹象\\
1.某段时间内行踪不明,却没法给出合理理由\\
2.拥有色情资料或者上网浏览色情网页\\
3.过度玩电脑或者通宵玩电脑\\
4.看电视或者电脑的时候,要把旁人都支开\\
5.床上表现有异\\
6.社交生活不如以前活跃,情感上与他人越来越疏远\\
7.拥有额外的邮箱账号、信用卡和其他电话号码\\
8.为行为找含糊的托词来糊弄人\\
9.一旦被追问起色情问题,防备心就会格外强烈\\
10.证据确凿表明他遮遮掩掩、满口谎言、鬼鬼祟祟\\
11.不合常理的疲劳、愤怒和/或暴躁\\
12.越来越关心自己的性魅力和性表现\\
13.越来越缺乏怜爱之心,抵触非性生活时的接触\\
14.说出没心没肺的性评论,使用怪异的性语言\\
15.在恋爱中无法维持情感上的亲密度\\
16.失去性欲,有性功能问题\\
17.越来越需要性刺激来得到性释放\\
18.对新奇的性举动、性爱工具越来越感兴趣\\
此清单可帮助女性判断她所处的情感困境是否源于伴侣的色情问题。读者必须要意识到,此清单中的多项表现也可能是由其他个人问题,包括上瘾问题引发的。\\
恍然大悟\\
女性怀疑丈夫或者男友沉溺于色情,但是对方矢口否认,试图消除她的疑虑,这会让女性心生怀疑。在这种情况下。许多女性会进一步发掘证据,找出事实的真相。一些女性反复盘问伴侣,为什么他的举止有异,进而质疑他不靠谱的回答。还有一些女性会向伴侣倾诉,说出自己对这段感情的不满,女性希望通过此举激发男性的保护欲,让男性更直率地说出真相。还有一部分女性会掏心掏肺地恳求伴侣敞开心扉,说出事实。也有很多陷入困境的女性开始参加情感和婚姻咨询,希望咨询师可以帮助自己找出事实的真相。\\
这种沮丧心理如大山般压在女性心头,只会让女性更坚韧地寻求证据。达莲娜是一名28岁的软件技术师,她和男友同居几个月之后就开始担心他是不是又犯色情瘾了。她说:“不知何时开始,他的行为开始变得诡异。每次我们做爱以后,他都看起来很困,说他要打盹儿。我要和他一起睡,他都只是笑笑,吻我一下,然后说他一个人就好了。这个打盹儿的事在短短几个礼拜之内就发生了好几次,我起了疑心。第二天,我去翻看他的电脑网页浏览历史,结果发现他在所谓打盹儿的时候。其实就是在看色情网站。我气得要死,就去质问他。”\\
达莲娜和其他很多女性一样,会像猎犬一样拼命在暗中寻找伴侣看色情的蛛丝马迹,以此来缓解心中的矛盾感和压力感;或许女性也不喜欢自已对伴侣遮遮掩掩,但她们还是会自我安慰说,这样子做总比被蒙在鼓里好。矛盾的是,就算女性在第六感得到证实后会松一口气,她知道了伴侣确实沉溺于色情时还是会非常沮丧。\\
第二阶段:大吃一惊\\
很多时候,伴侣沉溺于色情的事实迟早会被女性发现。很多情况下,女性总是在无意间发现确实的证据;有时候,其他家族成员、工作同事、朋友熟人等旁人先发现女性的伴侣在看色情,然后转告于她。在某些情况下,色情观看者会突然坦白。总之,不管女性是以怎样的方式发现事实,问题的严重性、问题对两人关系的影响都会让女性感到震惊。下方描绘的是证据确凿表明伴侣沉溺于色情后,女性比较普遍的反应。\\
“我以为他已经戒了”\\
10年前,罗杰告诉黛比他已经戒了色情,而10年后,黛比发现罗杰还在看色情,她震惊了。一天早上,黛比在忙不迭露营帐篷准备度假时,发现了一张租借色情录像的收据,上面的日期就是前一天。“我盘问他,刚开始罗杰还不肯承认,到后来才说他确实是租了色情录像。我当时就迷茫了。接着他承认说,他其实根本没有戒除色情瘾,10以来他一直在对我撒谎。我愤怒了、爆发了,这就好像一块烙铁烫在了我的心头。我开始疑神疑鬼:他说用来买午餐的那些钱,真的是花在了午餐上吗?还是他拿了那些钱去色情场所?他是不是利用工作、赚钱的时间去看色情?每次他说加班,每次他不愿意做爱——是不是都是因为色情?我已经不知道什么是真,什么是假了,不知道我嫁了个什么货色。”\\
“我根本不知道原来他沉溺于色情”\\
露西是一名女性研究专业的大学生,她发现同居男友托尼一直沉溺于色情时,简直不敢相信。恋爱3年以来,她一直以为托尼和她一样,觉得色情侮辱女性,不想再看了,她也觉得他不像是会沉溺于色情的那种人:“有天下午,托尼一直在电脑前待着,”她回忆说,“后来我忍不住问他在干吗,他说他在做‘政府调查’?!这太离谱了。我马上不爽了,直接要看他遮遮掩掩在看的内容,他拒绝了,然后就发火了,还威胁我说他要离家出走。到最后他才承认,他在看色情。我自己到电脑前去看,看到了滥交网站、社交妓女、波霸女人的图片。他过去总跟我说他很讨厌色情,但是现在呢?他竟然在看色情。现在我老想,是不是所有的男人都是混蛋?所有的男人都不忠诚?我是想要再给他一次机会的,但是我们再也回不到过去了。”\\
“我简宜不相信他会喜欢这种垃级”\\
就算女性已经知道伴侣沉溺于色情,但是她如果进一步了解他看的都是什么性质的色情、他对着色情自慰、他看的色情尺度的话,也会感到吃惊。宝拉意外地看到布莱德自慰时看的图片,大受打击。“我当时在电脑文件夹里面找一封信,一些色情内容就忽然在屏幕上跳了出来,”宝拉回忆说,“我当时吓了一大跳,那种图片根本不是我在孩童时期看到的色情图片。那种根本不叫做爱,那是侮辱、残暴、冷漠。屏幕上的女孩子就被当作破烂的洋娃娃一样被虐待,这给我当头一棒。我还记得当时自己反复想:他会喜欢这种垃圾?他怎么会喜欢这种垃圾?”\\
“然后我终于醒悟过来,明白了布菜德正是因为偷偷摸摸看色情才导致了他那过度的戒备心理,那暴躁的性格,大部分时间里我们性生话不和谐,做爱时没有默契。那一瞬间,我顿时明白了色情对他的生活、对我们生活的影响。之前,从他嘴里说出来的‘色情’不过就是一个词,但是看着那些裸体女人的图片时,色情对他来说就是活生生的了。忽然间,我感觉丈夫不过把我当作性物体,一种拿来看、拿来用的物体。我坐在那儿,坐在电脑前,就开始痛哭。我在这段感情中受到的所有伤害,忍受的所有痛苦,如潮水般将我淹没。我整整哭了两天。”\\
“那一瞬间改变了我的一生”\\
不是所有色情观看者的伴侣都会和黛比、露西、宝拉一样反应激烈,但是发现真相时产生的震惊和愤怒,那种被背叛的感觉,对性生活的不满和抱怨,是不约而同的。那种负面情绪会占据人的心头;这种感觉和发现伴侣有外遇有毒瘾、有赌瘾时的心情相差无几。幻想很丰满,现实很骨感,所有的一切,他们对伴侣的看法、对自已的看法、对这段感情的看法,在一瞬间彻底改变了。而且,女性发现真相当下所受到的惊吓。会在瞬间打击她们的精神和肉体。\\
那种深人骨髓的伤痛,如黛比所说的“一块烙铁烫在了我的心头”并不罕见。一些女性觉得自己被恐惧包围,仿佛浑身血液凝固; 另外一些人呼吸困难,心跳加速;还有一些人觉得恶心反胃;一些人在泪水中崩溃,一些人尖叫发狂。她们想要逃离,想要毁灭色情,也有女性因为过度伤心而在随后的一段时间内厌食、失眠。这种打击对女性的影响不可小觑。\\
女性发现伴侣沉溺于色情时的感受,就好像坐过山车时被抛了出来,这种感觉会持续几天、几周甚至几个月。上一刻她觉得怒火中烧,下一刻又觉得绝望无比。好多伴侣觉得孤立无援,束手无策。有些女性告诉笔者,她们知遴真相后无比震惊,在很长一段时问内根木无法理清内心的头绪。\\
影响女性反应强弱的因索很多,包括:她对色情的看法,伴侣观看色情的性质和程度,她对伴侣的在意程度。这段感情持续的时间越久,女性投人越多,她们的反应就会越强烈。\\
如果女性和黛比、露西、宝拉一样,对一段长期稳定的情感倾尽所有,对感情寄托了自己所有的梦想和期望的话,她们所受的影响远远大于随性恋爱的女性。由于女性为情感掏心掏肺,付出了如此多的时间、精力和情感,这就不难理解她们发现真相时所受的打击之大了。其实,让她们绝望的不仅仅是因为伴侣色情成瘾,还因为他们为了掩饰而撒谎逃避,这都让这段感情千疮百孔;彻底了解了伴侣的谎言和虚伪本质。将会毁灭女性对伴侣的所有信任。\\
“他对色情的兴趣让我害怕”\\
对凯伦来说,发现强尼在偷偷看色情让她伤心无比,他看的是什么类型的色情啊!其实,当时他们已经在考虑结婚了。“我在我们的电脑上打开了他的一个文件夹,”凯伦说,“里面全是年轻女孩子的性感照片。我绝望了,不知道该怎么办。我在想:我真的能跟这种人过一辈子吗?他几乎就是在儿童色情的边缘玩火了,这太让我害怕了。如果照片里面是30岁的成熟女人,也会让我担心,但是担心的性质就大不相同了。看了他收藏的色情内容后,我醒悟到,我绝对不能让我们的孩子降临到这个世界上来,因为孩子可能会受到来自他的威胁。我怀疑我以后还有没有勇气跟他生孩子。我爱强尼,看到我爱的人这么下作,我很心痛。”\\
一些女性的丈夫或男友和凯伦的伴侣一样,喜欢一些比较刺激的色情,如强迫、侮辱、折磨、未成年人性关系,这会引起女性的格外注意。发现自己的丈夫或者男友喜欢看强暴的性爱,或是性虐待的色情内容后,女性会担心自己以及他人的安全。\\
“这就好像我又被性侵犯了一次”\\
那些经历过性虐待、性折磨的女性在发现伴侣沉溺于色情时,反应往往更加激烈。因为色情中描写的性交、折腾人的方式类似于性剥削和性虐待,这会激起女性受性虐时的痛苦回忆。\\
38岁的弗兰曾遭到性虐待。她在恋爱初期就告诉男友大卫,自己对色情很反感,大卫向她保证说他在两人恋爱以前就已经戒掉了色情瘾。因此,某天晚当弗兰在大卫的手提电脑里面发现他收藏的色情网站时,简直就是晴天霹雳。“我很傻很天真地想了一下,心里还想可能是我弄错了,这个不是什么色情网站。”弗竺说,“但它确实就是。我马上关了网页,头脑一片空白,整个人都麻木了。我滚动了一下鼠标,就在收藏夹里看到了另外一个链接,这个是未成年人色情网。这一切都让我觉得恐惧,因为我年幼时被一个看色情的偎琐大叔猥亵过,大卫知道这件事。”\\
“我知道大卫还沉溺于色情的时候非常震惊,自嘲竞然相信了他的鬼话。我觉得自己就是一个脑残,特别绝望。我在感情上需要他,但是我明白我已经不能再和他在一起了。我的脑子无法运转,也不知道如何是好,感觉好像又被性侵了一次,觉得自己好脏。我恨他,恨他为什么这样子对我,尤其是他还知道我的痛苦经历。我觉得迷惘,觉得困惑,脆弱无助,同时我也感到非常害怕,内心充满怨恨和怒火。那晚,他陪着我,包容了我的激烈反应,让我稍微觉得好了一些。但这种复杂的心情一直持续到现在,我还是不能确定自己是否还能继续留在他身边。”\\
第三阶段:情感伤痕\\
在女性怒火爆发、尖叫发狂告一段落之后,内心的被背叛感和失望感,对不和谐性生活的失望,对伴侣的抗拒感会持续几个月甚至几年的时间。女性内心遭受的折磨,往往是色情观看者所不能察觉,也不能了解的。女性的自尊心受挫,安全感丧失,开放度和性反应感都会大大受损,而这些情感伤痕很难在短期内愈合。女性负面情绪持续的时间之久,也会让色情观看者吃惊不已。要使女性的心理伤痕快速愈合,需要色情瘾者产生情感上的共鸣,良好地解决色情瘾问题,只有这样他才能重获女性的信任,重建亲密关系。\\
下文将会更详细地介绍色情成瘾者难以察觉、理解的女性内心伤痛。\\
“叫我如何再信他?”\\
当黛比终于知道两人在一起的20年间,丈夫罗杰一直偷偷对着色情自慰时,她再也不相信这个男人和这段感情了。“这些年我一直怀疑自己是不是神经质,因为每次我质问他的时候,罗杰都说根本没有什么事,”她说,“我一直都相信他说的是真话,真实他根本就是在撒谎。我再也没法相信他了,我会想:等等,这次我是不是又被忽悠了?那次我是不是又被耍了?我根本不知道什么时候应该相信他,就好像一个人站在暴风雪中,要去分辨两片雪花之间的差别,眼前一片漆黑,脑里一片空白。我不知道如何是好。”\\
黛比和很多女性一样,一厢情愿地相信自己和伴侣都应该在情感上开诚布公,彼此坦诚。两人共同创建生活的感情基础就是诚恳,而当女性发现伴侣长期以来一直在撒谎时,幻想就被彻底扼杀了,信任感不复存在。\\
很多女性认为伴侣沉溺于色情等于是违反了情感关系的另一原则:在性方面彼此忠诚,即双方都应该把自己的性趣和性能量释放在对一方身上。所以,女性无法继续相信伴侣会在性问题上对自己忠诚。“我到现在还在怀疑罗杰是不是在搞婚外恋,不过两件事情性质其实是一样的,”黛比说,“他的性注意力、性精力、性幻想对象,原本都是属于我的,而不是其他任何一个人的。我觉得自己的权利被剥夺了,我又怎么相信他不会再欺骗我一次呢?”\\
发现男友大卫浏览色情网站以后,弗兰也曾怀疑自己是否还能信任他。过去她一直以为,之前他说的有关性生活和性活动的话都是真的。“现在我发现,自己很难再相信他了,和他在一起时也没有安全感了,”她说,“他在我们交往期间,说的全是假话,谁知道他还会撒怎样的谎呢?”\\
“我彻底鄙视他”\\
对伴侣失去信任感之后,女性很快就会转而鄙视他。如果你已经揭开了他的假面具,发现他无法自控,性嗜好不健康,又无法维持亲密关系的话,你也就很难一再尊重他的人格了。前一刻,女性还将他视为值得尊敬的人;下一刻,她就将他视为一个麻木不仁、面目可憎的人了。\\
女性之所以鄙视色情成瘾的伴侣,是因为她很清楚,他喜欢的那种东西是人人皆知下三滥的内容,而且还侮辱女性和儿童的尊严。弗兰知道大卫沉溺于色情之前,一直认为他是一个聪明伶俐有爱心的人。但是自从知道他喜欢色情之后,她的看法就有了一百八十度的转变。“他根本没有我想象中的那样正直,”她说,“我需要一个知道如何珍惜女性的男人,而不是一个把女性物体化,对着PS过的女性图片来得到性快感的男人。现在我只觉得他是一个会虐待女性,龌龊下作的病态男人。我再也没法去尊重他了。”\\
“我觉得自已缺乏性魅力,不能满足他”\\
女性常常会将伴侣对色情的喜好,诠释为对自己外貌和性魅力的否定。看到男友托尼在电脑前看的那些色情照片时,露西就开始怀疑自己的性魅力了。她郁闷地说:“忽然间,我就觉得自己胸部不够大,我之前从来不会这么想,现在我看着自己比较平坦的胸部,总觉得自己少了点什么。在我脑中,那些色情图片就像是毒药。”\\
和许多发现伴侣沉溺于色情的女性一样,露西立刻沉入了自责的汪洋之中,觉得自己没有性魅力,无法在性生活方面满足对象,这彻底扭曲了露西对自己身体和性能力的看法。“我一直以来都对自己的身体非常自信,”露西说,“但是当我知道他要看色情里面那些女性来引起性欲的时候,我生命中第一次开始怀疑自己的魅力了。我现在很想找回从前那种自信心理。”\\
女性们把自己和色情中的女性相比,就很容易因为自己的身材和性能力而感到自卑。“和色情里面那些年轻、苗条、性感魅惑的女孩子相比,我觉得自己既丑又肥,没有魅力,”黛西说,“我担心丈夫是不是总拿我和色情里面的女孩子比,因为我肯定没法和那些天使脸蛋、魔鬼身材的艳星比。那些女人不需要男人付出感情,也不需要交流,她们不需要任何事情。她们完美无缺,任由男人摆布。我觉得自己不够好,身材太挫,永远都没有办法满足他。”\\
弗兰知道男友大卫沉溺于色情后,也产生了类似的焦虑。“我不能跟色情里年轻性感的女性比,”她说,“我对自己的性能力没有信心,我也不能和色情演员一样做那些动作。”\\
“和他做爱时我觉得别扭”\\
很多女性在发现伴侣沉溺于色情之后,在特定时间段内都不想和伴侣发生性关系。那种被背叛感会转化成不信任,对伴侣的鄙视,包括她们否定自已性能力后产生的焦虑,所有这些复杂情感会导致女性抗拒肉体上的接触。“尽管在发现大卫沉溺于色情之前,我们的性生活非常美满,”弗兰说,“现在我再也不想和他做爱了。”\\
汉娜再也不想和丈夫理查德保持夫妻生活了,这不仅是因为她觉得自己没有性魅力,也因为她不再信任他了。尽管证据确凿,理查德还是矢口否认他沉溺于色情,还拒绝和她谈论这事。“他一直以来说谎欺骗我,我再也不信任他了。我的精神彻底崩溃,性生活也没有了。只要一想到他在网络色情里面找刺激,我就没有一点儿性欲。他的性欲世界根本不是基于现实的,这简直就是匪夷所思,就好像他玩宠物小精灵上了瘾一样。”\\
在发现伴侣沉溺于色情后,两人的性生活就会立刻降温不少,而女性从中得到的快感会减少。许多女性会猜测伴侣在性生活时脑中的想法。“要做爱就要脱衣服,”达莲娜说,“但是每次我脱衣服的时候,男友都用那种色迷迷的眼神盯着我,好像我在跳脱衣舞一样。这让我觉得自己很廉价,不受珍惜。”\\
黛比也有同感:“现在,每次罗杰看着我或是和我做爱的时候。我就是觉得他脑子里在回放色情。我担心他看了色情之后,脑中会长期重复色情画面。有时候,性生活进行到一半,我忽然感觉他的人根本不在我身边,因为我感觉不到我们之间的默契。他的肉体确实在我身边,他在肉体上和我做爱,但是他是在神游。这让我觉得自己不能在性方而满足他,而自己对他来说不过是一具会呼吸的肉体而已。”\\
在性爱时,如果女性觉得伴侣心里想的根本不是自己,这会损害女性开l放的本性。“我无数次感到尴尬,”凯伦说,“做爱的时候。我不希望丈夫把我当作色情里面的女人,一旦他建议我们尝试一点新花样,我就起疑心。他的那些建议,是一般丈夫对妻子的合理请求而己,但我立刻就会怀疑,那些是不是他在某些变态网站上而看到的花样,想要在我身上做试验,有了这种心理,我根本不能放松地享受性爱。”\\
宝拉知道丈夫看过大量色情之后,就越来越抵触性生活。她说:“有些时候,布莱德会在大白天逮着我就要做一些很别扭的性行为,我就会想,是什么事情刺激了他?我就是认定了他肯定是从色情里面看到这种事,然后想要我来表演给他看。我真是恨得要命,做爱的时候我觉得自己就是性爱玩具,所以有时候我还故意疏远他。”\\
有些女性对性爱的兴趣大减,是因为她们觉得,伴侣沉溺于色情就是没有向自己看齐,没有尊重、崇尚身体结合。黛比说:“罗杰把我们两从间最珍贵的交流方式:他的性爱,随便地和无数幻想中的女性去分享。我一直认为我们的性爱表达了彼此的爱意,这只是两人之间秘密的事。”同样地,弗兰觉得大卫观看色情,玷污了两人之间的亲密关系。她说:“我觉得自己被侮辱了。我一直认为我们的结合特殊而神圣,但对大卫来说,那太廉价了。”\\
“他在感情上抛弃我了,我感觉自已没有被爱借”\\
最终得知丈夫鲍伯在两人20年的婚姻期间一直沉溺于色情时,苏最大的恐惧就是她已经意识到,无沦她怎样抗议,丈夫也没有要戒除色情的打算,这让她觉得自己一文不值,却无能为力,这导致她对丈夫、对自己、对这段婚姻出现了信任危机。“我反复告诉鲍伯,他看色情会让我觉得很困扰,让我觉得自己被侮辱了。我不想要在家里看到色情,也不想要他在地球上任何一个角落看色情。他说我神经过敏,小题大做。我就知道他到现在还在对我撒谎,他现在还在看色情。我不知道怎么做才能让事情有所好转。他对我的忠诚度,远远不如他对色情的忠诚度。我一直以为我才是他生命中的挚爱。尽管他知道我很难过,但他还是在看色情,这让我受伤很深。”\\
尽管梅根是新婚,但她同样也没有体会到被珍惜、被爱护的感觉。在她逮到丈夫看色情之后,丈夫只是一味为自己辩护,根本没有考虑到她所受的伤害,这让她非常伤心。她要利用离婚作为威胁才能引起他的注意。“他就是不管不顾我的感受。一心想要继续看色情。”她说,“我和色情,杰西选择的是色情。那些曾经说爱我、珍惜我的海誓山盟全都是空头支票吗?如果他真的爱我,又怎么会执意去做那些会让我伤心的事?”\\
苏、梅根和其他很多女性‘一样。都认为感悄需要彼此的体谅和谅解。她们希望丈夫可以听取她们的意见,体贴她们的感受,意识到色情问题的严重性。因为她们原本以为,只要自己和伴侣清楚地表达了自己对色情的抵触,表明自己受的伤有多深,对方就会毫不犹豫地答应戒除色情,再也不会看。当伴侣们并没有急着要戒除色情,而且还无视此事给她们带来的伤痛时,这无疑是在她们的情感伤痕上撒了一把盐。\\
如果女性在情感压抑时,没有得到伴侣足够的安慰和关怀,她就会觉得孤立无援。弗兰告诉笔者,让她受伤最深的是,男友大卫不是立刻承诺改过自新。“我不敢相信,在我发现他在网上收藏色情的那晚之后,他还在继续看!”她说,“他觉得自己可能色情成瘾了。他说我们的关系变质了,他不再关心我了,他还以为我不会意识到。我希望他坦诚,但现在的我只觉得被抛弃了。他根本没在感情上关怀过我,尽管他知道,他看色情的话我会非常难过,但是他还是不管不顾,继续看色情,这让我非常绝望。现在大卫告诉我,他要珍爱生命,远离色情,但我觉得一切都已经太晚了。如果他真的看重我们这段感情,真的爱我的话,难道他不该早早地就去寻求解决问题的办法吗?”\\
第四阶段:设法解决\\
女性在发现伴侣沉溺于色情之时,往往会经历巨大的情感波动,刺激女性采取应对措施。通常情况下,女性不会就此放弃一段感情,至少不会立刻就分手,而是采取措施,试图凭自己的能力去控制、修复对方的色情瘾问题。这种控制欲望是可以理解的,这只是当自己的信任、诚恳和忠诚被践踏后。想要扭转局势的应对方式。\\
变身成为色情警察\\
部分女性往往会化身为“色情警察”来控制局面,让自己安心。她会像警察一样监控伴侣的行为,不停地审问他。甚至不惜设置圈套来揭露伴侣偷偷看色情的活动。大部分女性并不喜欢做这种事,这是她们不惜一切,单方面想要挽救这段关系的努力。\\
南希,38岁,3个孩子的母亲,她抓到丈夫洛根在网上偷偷看色情时,才知道丈夫在10年婚姻间都在偷看色情。从此,南希化身成了色情警察。“我变成了一名非常厉害的侦探,”南希说,“我学会了如何在电脑里面查看隐藏文件夹,给电脑加锁,不给他任何机会接触色情,我还追踪他所有的银行和信用卡账单。我在发现他沉溺于色情后的几个礼拜里都在疯狂地想办法让他戒除色情。我觉得我可以解决这个问题,我再也不想被欺骗,也不想再失望一次。”\\
毋庸置疑,色情警察这个角色会让色情观看者愤怒不已,变得充满攻击性,因为这会让他觉得自己被当成了小孩或是罪犯,自己的隐私被侵犯,因此变得如刺猬般防备。女性伴侣也最终会厌倦色情警察的角色,因为她要时刻保持警惕,不能有丝毫松懈。色情警察这个办法并不能很好地带来安全感,反而会使感情关系变得紧张,这充其量也就只能给女性一种控制一切的假象而已。因为,如今这个电子化的世界里充斥着色情信息,色情成瘾者只要有意,很容易就能找到看色情的私密途径。\\
和色情竞争\\
另一种极端的情况是女性会做出努力来和色情竞争。这些女性认为,只要自已在外形上变得和AV女优一样的话,她们就能挑起对方的性欲。重新吸引伴侣的性注意力。这种方法也体现出她潜意识中对自己性生活不满、被性拒绝时产生的忧虑。一些女性花费大量时间和精力来修饰自己的外貌,希望自己能和AV女优一拼高低。她们会染发、隆胸、给生殖器除毛发、做指甲、抽脂、减肥,等等。她们也会逼迫自己过更多的性生活,强迫自己在床上表现得和色情中的女主角一样。为了满足伴侣,女性还会做一些自己并不喜欢的性举动,比如暴力性爱、角色扮演,或者一些侮辱情感、伤害一身体健康的举动。\\
可能女性的这种表现会让伴侣高兴一阵子,但是时间一久,女性就会觉得自己很下作,因为她们在做着违背自己价值观的事,而且,几乎所有的女性都会意识到,自己根本无法和色情中的那些年轻又高挑,PS过、精心装扮过的艳星媲美。25岁的沙拉说:“我不能和那些AV女优比,我也不能像她们一样让男人疯狂。就算那些男人折磨她们,令她们窒息,无论男性做出什么样的出格举动,她们都不会抱怨,让男人达到高潮就是她们唯一使命。”梅根对笔者说,“我想要模仿AV女来取悦丈夫,但这让我感觉自己就是个出来卖的。我希望性爱是两人间独特的联系,而不是什么表演。”罗尔是一名女作家,她补充说:“当我穿上男友要求的衣着时,我永远不知道他爱的究竞是我,还是他色情幻想中的活人版本。”\\
寻求外援\\
第三种策略,也是唯一一种可以让色情瘾者的伴侣真正舒缓自己痛苦的方式,就是寻求外援。如果确定色情观看者不愿改变,那么女性必须要向亲朋好友、闺蜜蓝颜、教堂神职人员、十二步治疗法或其他专业人员寻求帮助,获取建议,听取意见。悲剧的是,大部分女性都心有顾虑,不愿和他人谈论色情问题,因为她们担心这会让自己看起来很失败,也担心这会让别人对自己的男性伴侣留下负面印象。同时,女性还会担心色情问题会带来风言风语。\\
卡伦说:“在很长的一段时间里,我都觉得自己不能向任何人倾诉心事,我甚至都不愿相信我自己。我怕别人知道以后会在强尼背后指指点点,我因为这事对强尼的印象就大打折扣了。我害怕这件事会一辈子影响两人的关系,我就是想让这件事大事化小、小事化了,不影响我们的生活 。但是鼓终,我还是对教堂里的一位咨询师开了口。”\\
起初,苏也没有得到有效的帮助。后来,她向一位性和情感治疗师寻求帮助。“我觉得鲍伯的色情问题让我很羞愧。我不知道该跟谁说这件事。我不想和闺蜜们说,因为这好像是侵犯了丈夫的个人隐私。破坏了我们之间的感情。我无时无刻不在惦记着他的色情问题,感觉自己好孤单。如果他迷上的是酒精或者毒品就好了,因为这些问题更容易被人所理解。而色情呢?它有关性行为,太私密了。谁又能真的了解呢?当我的咨询师告诉我,如今色情问题非常普遍,而我的丈夫也不是唯一一个沉溺于色情的男性时,我才松了一口气。”\\
丈夫理查德拒绝面对自己的色情问题时,汉娜得到了很多来自亲朋好友的支持和帮助。“我并没有向父母和闺蜜们隐瞒这件事,这样做真是明智。他们都很有同情心。理查德因为色情变得更加孤僻,完全失去了自理的能力,看起来惨兮兮的。我的亲友团们注意到了他的变化,也为他担心。我们试过夫妻治疗,但治疗师都说除非理查德能控制自己的问题、自己主动寻求帮助,他才能帮助我们,而结果只有我一个人去接受治疗。我对治疗师说我没有安全感,对未来迷惘不安,而治疗师以我进行开导后,我才意识到理查德的色情瘾并不是因我而起的,那跟我没有一毛钱关系。这么多人给了我强有力的关爱和支持,这对我来说就像是全世界。我给广大的建议就是:不要一个人承受。”\\
尽管整个过程非常痛苦,但汉娜还是得到了可喜的康复。她意识除非理查德承认自己有问题并且愿意采取相应措施,不然她无能为力。如果女性希望自己的生活能改善,就要自己寻求帮助,做出改变,让生活符合自己的价值观。\\
如果一方有严重的色情瘾问题,那么不管问题是无人知晓还是众所周知,它就像房屋的基开始腐蚀,动摇了整段关系的基础。如果不及时采取行动解决问题,后果不堪设想。“目睹色情毁灭我们婚姻的过程,我伤心绝望过,”汉娜说,“我试过和理查德谈论这件事,但他就是心不在焉,然后又开始自闭、暴躁不安。色情瘾腐蚀了他的整个心灵,而那个心灵里面已经没有了对我的爱。他的心已经不在我身上了。也不在家庭里。他行尸走肉地活着就是为了等待下一次色情给他带来的刺激,他的身心都被色情所控制,而我的爱再也无法撼动他了。”\\
第六章 穷途末路\\
色情给人虚幻的快感:前一刻它让人飘飘欲仙,后一刻它就牢牢地将人拖下无底深渊。——罗波\\
如果色情成瘾者不管不顾色情带来的问题和风险,一意要保持和色情的性爱关系,结果会如何呢?对一些人来说,结局就是:生活受到巨大冲击,被逼到穷途末路之上。这种巨大的生活变故会带来不可估量的后果。而色情观看者之前满口的否认和狡辩全都会变得苍白无力,因为血淋淋的现实已经证明,他或她已经对自己和别人犯下了不可弥补的错误。\\
穷途末路包括外部危机和内部危机。外部危机指现实生活中的变故,如在工作时观看色情被逮住,或是和女朋友吹了,内部危机则是指以情感和精神崩溃为表现的内心变化。色情带来的某些影响是可以预测的,比如说,那些强迫性观看色情并且深受其苦的人,比只是偶瞄一两眼色情的人更容易走上绝境。但是,色情能引发的某些后果超乎想象。比如说,第一次观看色情的人无意间下载了儿童色情而被抓去蹲了大牢。再比如说,有人虽然如今不怎么看色情,但早就遗忘的限制级录像被妻子发现,引发了一系列家庭问题,最终导致离婚。\\
现实很残酷:无论你认为自己如何控制或者限制了观看色情的行为,你都不能百分之百确定所有的事情都在掌控之中。就像酗酒、嗑药、找小三一样,色情观看者的行为有着潜在的风险,这是在人为控制和能力范围以外的。\\
本章将会讲述四位被逼上穷途末路的色情成瘾者经历。读者将会认识麦切,他55岁,已婚,有三个孩子,曾是一名老师;汉克,一名47岁的焊工,离异;玛丽,一名43岁的单身妈妈,有2个孩子;还有汤姆,一名26岁的药店职员。这些人都曾经坚信色情给了他们最美好、最刺激的性爱,觉得自有权继续在色情中寻找刺激,而这种幻想被现实击得粉碎。 笔者选择描绘不同人的真实经历,展示个人被逼上绝路的不同方式。读者将会看到,色情是如何给这些人的生活带来致命的一击,而色情观看者在变故后又经历了怎样的心路历程。尽管经历各不相同,麦切、汉克、玛丽和汤姆四人的生活有一共通之处:色情最终让他们的生活失控了。他们对色情的欲望以及色情给他们带来的即刻快感,诱使他们沉溺其中,还自欺欺人地以为这不会影响到他们的生活,不会伤害那些关心爱护自己的人。这些人的经历说明一个共同主题:色情成瘾者每一次观看色情之时,和生活中最可怕的变故不过一纸之隔。\\
麦切的经历\\
东窗事发之前的麦切认为,自己的生活已经幸福美满到了极点。他刚刚步入50岁大关,有着幸福的家庭,3个青春期的女儿很是崇拜他,而他作为高中教师和教练的事业也受人尊敬,他还是教育理事会和董事会的重要成员。他平易近人,聪明能于,学生和家一长都对他赞赏有加,同事也很尊重他。表面上看来,他就是一个成功美国男人的真实写照。\\
和许多男孩子一样,麦切自打青春期发现了老爸的色情杂志以后就开始看色情了。年轻的时候,他偶尔会在出差时买些色情杂志,租借色情录像带。进人20世纪90年代后期,随着网络的普及,麦切发现,只要他想,就可以轻松在电脑上找到各种色情,之后他观看色情就更加频繁了。\\
“我不想在家里放任何色情品,也不想在电脑里面保存色情信息,因为3个女儿还在家里住着,”他说,“我不想让色情危及她们,所以我就趁着工作时间上网看色情,等到学生们放学之后,一个人窝在单人办公室,拉上窗帘,锁上门。在那时看来,在办公室看色情很保险,很安全。”\\
麦切认为自己看的色情完全无害。“我看的大部分色情都很隐晦,”他说,“只是裸体。成年男性和女性发生正常的性关系,没有什么出格的行为,没有变态的东西,没有非法的内容。我喜欢看的那种图片和录像讲的是爱情故事,两个人最后会做爱的那种。剧情一般都比较狗血,里面会有很多性爱镜头。”\\
尽管麦切婚后的30年间和妻子的性生活一直很稳定,但他还是需要色情来填补自己的欲望。“妻子不会像我一样喜欢做爱,”他说,“她从来都不会特别渴望性生活,觉得就那么回事儿,婚后不久我也就随她去了。但是色情随时都能得到,现在的网络太发达了。色情里面的性爱太劲爆,会让我热血沸腾,达到非常强烈的高潮。喜欢看色情就好像迷上了毒品一样刺激。”\\
那时麦切已经意识到自己看色情的行为带着强迫性,但他认为这和生活中其他强迫性的活动一样,都是无伤大雅的。“我每天早上都会看报纸,我在放学后要看一点色情,我在半夜的时候要吃夜宵。如果我不做这些事,就会觉得不踏实。色情是我当时唯一个不良嗜好,那时的我也没觉得这个习惯有多不好,只是觉得这件事不应该让别人知道。”\\
一个周五放学铃声响起后,学生们都离开了学校。麦切坐在办公桌前收拾周末需要修改的试卷。忽然,校长、区教育局长和其助手走了进来,三人脸色凝重。“我热情地招呼了他们,我当时根本不知道他们的目的,”麦切说,“教育局长低沉地说:‘我们追踪了你的网络使用记录,知道你一直在工作电脑上看色情。今天我们来就是要把你的电脑带走,调查你,把你赶走。把所有教室和教学楼的钥匙全都拿出来。就是现在!’我当时呆若木鸡,好像身处梦境。他们这样子根本就是把我当成了色狼、变态。”\\
“他们还问可不可以调查我的家用电脑。我说:‘可以,可以。你们可以全部都拿走,那里面根本什么都没有,我就只在这里看那些东西。’教介局长接过话:‘我们会调查每一台你用过的电脑,一定会找出足够的证据送你进监狱。’”\\
麦切这才意识到纸包不住火了,便决定马上向生命中每一个重要的人坦白。“我先是告诉了牧师,”他说,“那天晚上我向妻子和女儿坦白了一切,她们一点心理准备都没有,家里一下子就翻了锅。”随后,麦切联系了协会教育代表,聘用了一位律师。律师告诉麦切,如果麦切和学区打官司,或许他可以重回工作岗位。\\
但是麦切并没有决定要打官司,他选择了辞职。“学区威胁说要公开我的丑事,”他说,“他们之前已经用这种手段对付了两个人了。我接到一个学区代表的电话,他威胁说:‘如果你不来辞职,我们会把这件事爆料给媒体。’他们想要杀鸡儆猴。到了这种地步,就算我从来没有对任何一名我教过、训练过的孩子说过或做过任何不得体的事,也是无济于事;就算我有30年地区优秀教育经验也没有用。他们就是想要把我逼上绝路,我知道他们说到做到。如果真的闹起来。我在这个地区就名声扫地了。不管事实如何,只要人们在媒体中看到或者听到。一切不分青红皂白的指责就全成了铁板钉钉的事实了。”\\
麦切如今再去回顾他那段经历,认为这是生命中一段“艰难的黑暗时期”。30年的教书育人生涯就这样被一笔勾销,而家人、朋友和同事一夜之问都失去了对他的敬重。“但是,”他说,“虽然那时候我心里特别不平衡,但我终究了解,遮遮掩掩看色情是一件极其危险的事。后来我被判了缓刑,还主动接受了性侵犯测试,参加了治疗项目。经历过这些波折之后,我才拿回了教师资格证书。现在我已经改行了,但还是一直把资格证放在钱包里,时刻提醒自己曾经受过的痛苦。我曾经有色情瘾,但是没有勇气直视它,反而把这个问题越藏越深。如果当时的我有足够的勇气,就应该早早向妻子和牧师求助。如果我真的这样做了,也许现在我还在教书育人。但事实就是,色情瘾占据了我的身心,控制了我的行为:色情毁了我的生活。”\\
麦切发现,观看色情极易侵蚀人的判断力,诱使个人在不知不觉中承担巨大的风险。尽管麦切又聪明又能干,但是在欲望的驱使下,做了一些他现在称之为“很脑残的行为”。尽管有时候他心里不安,知道工作时间看色情不太明智,但理智被欲望所蒙蔽,他也就艇而走险了。他曾经自我安慰说:“我看的不过是软性色情,没有人会知道的。我在学校看是为了保护家人啊。”他从来没有想过要花点时间,好好反思自己看的内容,也没有去预估被逮着之后的后果:观看色情让他奋不顾身。\\
当麦切说色情完全“毁”了他的生活时,他指的并不仅仅是30年卓越的教育者生涯毁于一旦,还包括他自己的生活、他生命中其他人的生活都受到巨大伤害。麦切的妻子既震惊又愤怒,失望透顶,她认为麦切看色情就是性不忠,他的女JL们也对他很是失望。他还要在学生、家长和其他人面前找借口来掩饰自己辞职的真正原因。麦切被迫改行来养家糊口,而且在事后很长一段时间内,他都过得战战兢兢,生怕事情被捅出去,影响他的声誉。\\
汉克的经历\\
汉克,47岁,离异。他和同辈人一样,在青年时期就常常看《花花公子》杂志。他当并没有觉得那些内容特别淫秽,不过不久之后,他就开始看那些惹火的色情图片来自慰了。\\
17岁时,他和一个女孩子发生性关系之后就做了爸爸。两人经不住父母双方的压力,被迫结婚了。尽管这位新娘很喜欢性爱,只要汉克想要,她随时愿意做爱,但是汉克发现自己还是需要不时地对着色情自慰才能满足。汉克说:“和妻子在一起的时候我从来没有性满足过。那里我18岁。有一个性欲正常,积极配合的性伴侣。但是不知道为什么,我就是想要看色情。”\\
后来汉克离婚了。虽然之后他有过无数床伴,但还是老去色情那里寻找慰藉。“我一直觉得自己的性欲没有得到过满足,后来我明白过来了,并不是我需要什么新花样,而是我一直在寻找完美女人,和色情里而一样的女人。色情不能帮助我和现实中的女人相处。我就是想要完美的女人,宁缺毋滥。”\\
在离婚后的几年中,汉克的色情瘾愈加严重了。他说:“那段婚姻画上句号的时候,我就像是一个饿到前胸贴后背的人,进了一家自助餐饭店。忽然之间,我想看多少色情就能看多少,不用考虑其他任何人。我不顾一切地看色情,对我来说,在现实中做爱不外是另一种形式的色情。”\\
汉克说自己变“花心”了,一心想要找女人上床。“过去有一段时间里,我跟种马一样四处留情。我同时有好几个床伴,但是每一个都不能让我满足。我和女人的关系一般就持续几周,最多几个月。在最初的迷恋和神秘感消失之后,我的性欲马上就得到了满足,然后我就觉得那个女人很下作,觉得她低人一等,配不上我,也不值得我去爱。我和她们在一起的时候也会感到很内疚,我不过是利用她们来得到性高潮,根本没有认真对待这份感情。我找的不是真爱,是可以上床的女人,但更多时候,只有色情才能满足我的性欲。30岁左右的时候,我就不再和女人们纠缠了,更乐意看看色情,在觉得自已需要一个活人的时候才搞搞一夜情。”\\
汉克完全不介意脱离现实,边看色情边自慰已经满足了他的肉体和精神需求,他也很满意在色情中自己可以掌控性爱的整个过程。他说:“我不需要考虑别人的需要,而且还能控制自己何一个性反应。”\\
到而立之年时,汉克已经进化到光靠自己就可以解决性需求,不需要现实的女性。“我要的就只是视觉和文字色情给我带来的刺激,”他说,“有性需求的时候就去看色情自慰。我可以在10分钟之内完事,也可以花几小时的时间幻想,延长快感。”\\
汉克在观看色情时虽然获得了肉体欢愉,却牺牲了自己的社交生活。汉克反思道:“慢慢地,我已经没能力和女性保持恋爱关系了,也没有了那份热情,因为我变得太自私了。如果我去了聚会,找到了可以发生一夜情的女人,这整件事还是以我自己为中心。有时候我和女人睡觉仅仅是为了证明我没有什么问题,我还有人性。”\\
在此后的10年间,汉克和色情的关系发生了本质变化。对着色情自慰已经不是他为了满足性欲而选择做的事;相反地,他需要经常边看色情边自慰才能开心。“这是一种强迫症,我要是不做就会觉得自己欲求不满。我开始频繁地对着色情自慰,每天一两次,这已经不单单是为了发泄性欲。自慰的时间越来越长,而且我要看更重口味的色情才觉得刺激。我后来就不看《花花公子》这类色情杂志。喜欢上了儿童色情,我甚至还觉得这个转变非常正常。我需要更加重口味的东西,所以《花花公子》没法再让我满足了,它太无趣、太保守了。我想要看那些非法杂志,想边喝酒、边磕药、边看色情来提高刺激度,我对色情的依赖已经发生了质的改变。”\\
汉克渐渐察觉到自己的胃口越来越大了。“看那些暴露的影像已经让我无感了。虽然我看的时候也会有快感,但是真正能让我满足的是我自己的想法,自己的想象力。我感觉自己都可以写色情小说了,于是,我就开始构思,为自己写小说。写作是我为自己辩护的最好理由,我自我安慰说,这种写作是益智的,会锻炼我的写作能力。”\\
汉克在房里一躲就是几天,磕药到飘飘欲仙,喝几打伏特加,一边写着色情。“我可以勃起几个小时,真的是几个小时不间断。而且我这样子一来就是几个月,每个周末我都把自己关在房间里干这些事。刚开始我还觉得很棒,但到后来就觉得于心不忍了。我开始觉得,我看着色情照片中赤身裸体的女人,却一点儿也不在意她作为一个人的尊严,我真是道德败坏。我用她的照片,等到我看腻了就把照片扔在一边。我为她感到羞愧,更为我自己感到羞愧。”\\
“其实我知道自己在做些什么。我太自私自利了,得到回报实在太少。但是,这种罪恶心理好像更加刺激了我的性欲。我再也没法主导自己的性生活了,色情才是起着主导作用的。色情反客为主,它好像有了生命力,它才是独裁者。它在导演着整场戏,而我就是个跑龙套的。”\\
汉克全身心地投人色情,周末时间也把自己弄到筋疲力尽,这直接影响了他的工作。“我是焊工,工作要求我聚精会神,不能出丝毫差错,因为我是和火焰、有毒气体在打交道。一不小心,我就可能受伤甚至有生命危险。一次,我连续写了3天的色情小说,磕了3天的药,喝了3天的酒,睡了3个小时的觉,就去上班。我就站在那儿,好像完全忘记了工作,忘记了我是谁。我就站在那儿,精神崩溃了。这种感觉就好像是我的良心在呐喊,我感到深入骨髓的羞愧感,就好像精神快要崩溃,道德快要投降的时候,良知在用它唯一可以触动我的方式,敲击着我的心灵。”\\
“我从来没有跟人提过这件事,而且那种感受也很难用言语来形容。那种感受非常奇妙。在那一瞬间,我只觉得自己再也不想喝酒,不想磕药,再不想看色情了。那一瞬间仿佛一切都已经结束,我彻底崩溃了。我觉得自己堕落到了极点,接着我就开始抽泣,就好像我身体中的一部分已经死掉了,我在为它哀悼。我再也不能、也不愿意这样子对待自己了,我不想这样子堕落下去,我想要证明自己还是一个完整的人。”\\
和麦切不同的是,汉克在内心走到了穷途末路。并没有人逮到他看色情,没有伴侣威胁着要离开他,他也没有丢掉饭碗。色情把他困在精神和心理的空白区,他已经全然抛弃了自己的价值观和尊严;那时,他才意识到,不管曾经多么自信,认为他可以隐瞒所有人,但他终不能自欺欺人,因为一切已经突破了个人的道德底线。\\
情感崩溃体现了内心关于为或不为的矛盾斗争。精神压力之大,使得原本在潜意识中徘徊的情感如火山喷发,进人主观意识中。到此时,身心疲惫,转而崩溃。汉克的精神崩溃是他忽略了主观的价值观,精神感受和各种需求所导致的必然结果。他在崩溃之后才意识到,他不愿意再看色情,也不想伤害自己的身体、情感和精神了。汉克到了山穷水尽之时,才发现生命失控了。现在,他再也不能否认自己的色情、毒品、情感和自慰等问题如脱缰野马不可控制。正如汉克反思:“极度沉溺症,和自我毁灭近在咫尺。”\\
玛丽的经历\\
玛丽是一位中年的工薪阶层单身妈妈,有两个正处在青春期的孩子。自从她发现对着色情自慰可以缓解精神压力之后,便沉溺其中不能自拔了。和汉克、麦切一样,玛丽在青少年时期就开始看色情了,她看的是满是性爱描写的侦探小说和言情小说。年近20的时候,她在汽车旅馆当房间保洁。“我在那里上班的时候,经’常发现各种形式的色情品,去那儿住宿的人常会买色情杂志看,之后就随手扔在房间里。我把能找到的每一样东西都收集起来,战利品很是丰厚,各种色情杂志应有尽有。手头存货太多的时候,我就开始当‘托儿’,把这些色情杂志送给家人朋友。我会先看,不过那些内容一般都很狗血,翻来覆去就是那些套路。最大的刺激在于:这些色情杂志是被禁的。我辞了汽车旅馆的工作以后,就没有再收藏色情了。”\\
玛丽年幼时双亲就离异了,她轮流随着双亲中的一方生活。她的父母都经常约会,也经常在电视上看色情。玛丽趁着他们不在的时候会去看他们的色情。“我被色情吸引住了,感觉很刺激。只要能找到,我就会去看。”玛丽现在才明白,她的成长环境充满了性爱。这导致她从年幼就喜欢上了色情。她的双亲都看色情,而且,她年幼时被父亲性骚扰过好几次,而她14岁时还被母亲的男友性侵犯过。\\
玛丽2S岁左右时结了婚,成了一名基督教徒。婚后,她在电视上看过几次色情。“作为一名基督教徒,我不该看那些内容,但看电视的时候,我就会想要看色情。如果换台的时候,某个频道正在播放性爱镜头.那我就会看下去。”\\
尽管和丈夫的性生活活跃,玛丽还是要看色情。“我新婚时用性爱来解压。性爱就是我的‘合法途径’,名正言顺地和丈夫发生性关系,无所顾忌。我爱我的丈夫,但是我不知道什么是健康的性爱。当时,对我来说。性不过是一种释放压力的方式而已。”\\
结婚七年后,当他们的孩子一个5岁,另一个18个月大的时候,她年仅32岁的丈夫因为哮喘突发猝死。“我打电话叫救护车,还拼命给他做人工呼吸,但是一点用都没有。护理人员刚给他戴上氧气面罩的时候,他就死在了我的怀里。我儿子上幼儿园的第一天就是我丈夫的葬礼,他的离世是我人生最大的悲哀。”\\
自从丈夫逝世后,玛丽就变得越来越孤僻了。“我不想和任何人保持联系。我受伤太深,发誓说再也不会让自己受到这样的伤害了。我不和朋友一起出去,就待在家里看色情。就是在那段时间里。我的色情瘾开始加深了。沉溺在色情里的时候,我才会遗忘失去丈夫的痛苦,忘记孤苦伶仃的寂寞。我不会在家里放色情书籍和杂志,因为我不想让孩子们发现。而且,我住的是个小镇,我不想让别人看到我去买色情杂志。我也没有开通有线电视,不过“秀台”(show time)频道还是渗透进来了。我看得到很清楚的影像,听得到声音。”\\
玛丽很快就发现,对着色情自慰,比单看色情要刺激得多。“那时候,我甚至都不需要看到画面,只要能听到性爱的声音,知道他们在做的事,这样就够了。我开始边看色情频道边自慰,让自己放松入睡。如果哪天我过得不顺,心情不好了,就会躲到那个幻想的世界里,在大脑里面重演色情情节;我也会看着色情自慰,增加兴奋度。有时候,我心里意识到:这样子是不对的!但是,我就是停不下来。”\\
当玛丽开始观看网络色情之后,问题更加恶化了。“孩子们睡着之后,我就去上网,花几个小时的时间点击鼠标,看那些免费色情,”她说,“我几乎不挑,不管是什么内容跳出来,只要不是特别变态,我都会点击进去看。过了一段时间,我才意识到自己在看的内容是我之前根本想象不到的。比如说,我会去同性恋聊天室看看那些家伙在说什么。我从来不和任何人交谈,因为我觉得他们不靠谱,不过他们会分享一些怪异的图片,都是一大群男人用各种姿势展示勃起的图片。我也没有把这些人当同性恋,就把他们当作一群雄性动物而已。”\\
玛丽对着电视和网络色情自慰已经有7年的历史了。“我基本上每天都会做,有时候一天做好几次,”她说。“这就是一种狂热,好像嗑了药一样。我会一直惦记着这事,做了还想做,这事完全占据了我的身心。有时候,我会觉得自己和色情中的角色达到了情感上的共鸣,我会意淫自己和他们一起做爱,然后我会想在现实中用这种方式做爱。这种想法时时浮现,我内心知道这样不道德。我不想和母亲一样,孩子在家的时候还随便带男人回家过夜。但是一段时间后,色情、幻想和自慰对我来说已经不够刺激了,不能让我满足,我想要更多。我有这种念头的时候,我非常害怕。我知道我有麻烦了。”\\
此外,我也开始感觉到,自慰时我失去了对自己思想和身休的控制。我对着电脑点击鼠标,一晃几个小时。等回过神来,已经是凌晨三四点钟了,我还要早起。好几次我在网上看色情,随心所欲地点击,忽然间我就高潮了,我甚至都没有触碰自己的身体,那是一种‘无为高潮’。我只是点击,浏览,意淫,然后就高潮了。我失去了对自身的控制,其他东西占据了我的肉体,控制了我的反应。色情的威力太大了,让我堕落成了现在的样子,我非常害怕。”\\
“色情让我自愧不已,它影响了我和上帝的关系。我不能既追随上帝。又继续观看色情、意淫、自慰。在圣经中,有经文要求基督教徒不能同时侍奉两个上帝。但是曾经一度,我的上帝是色情。好几次我坐在教堂里的时候,脑子里不停地回放着限制级的画面,我会看着来参加礼拜的某个人,对着他意淫。我花很多时间回想自己在网络和电视中看到的色情。我觉得罪过。我觉得上帝不会再爱我了。”\\
因为色情,玛丽的自尊和性欲受损,社交生活减少,宗教信仰动摇,此外,色情还影响了她和孩子们的关系。“我变得越来越孤僻。我完成了单亲妈妈的基本义务,有份工作,给孩子们饭吃,送他们上学,但是我没法真心的和他们交流。我满脑子想的就是什么时候可以再看色情,应该怎么看。我和孩子们天天住在一起,但我没有真正关心过他们的感受,他们作为人的需求,我们甚至从来没有一起玩过游戏。两个孩子都很抑郁,后来我才发现,可能就是因为我很少陪伴在他们身边,爱护他们吧。”\\
一天,玛丽8岁大的女儿生气了。”她开始哭,说我根本不知道她是谁。她在怀疑我是不是真的爱她,”玛丽说,“我的心都碎了。这让我反思,我浪费了多少时间和精力在那个臆想的世界里,却没有认真地和自己的孩子交流感情。哮喘已经让他们失去了父亲,而现在他们又会因为色情失去母亲。那时候我意识到自己有问题,需要帮助。但是,我根本不知道该如何面对,如何处理,觉得丢不起这个脸。”\\
和女儿对话之后没多久,玛丽的电脑坏了,教堂的青年牧师正好来她家探望她的儿子。玛丽说一切都是“上天的旨意”。她知道这位牧师很懂电脑 ,就问他能不能帮忙修电脑。“我对电脑不太在行,不知道电脑会保存网页浏览历史,还可以删掉,所以我看过的网页记录都在电脑里面。”\\
“那位牧师开始修到理我的电脑,忽然,我发现他在看我的网页浏览历史。他眼睛瞪得老大,下巴都快要掉下来了:他看到了那一长串我浏览过的限制级色情网页。我当时恨不得在地上找个缝钻进去!我们两人都没有说什么,整个场面尴尬得要死。我想:完了,现在我该怎么办?我只是跟他说:‘我会和牧师说这件事的。’我当时觉得老脸都丢尽了,尴尬无比,但这件事让我清醒地意识到,我有严重的色情问题,是时候必须要停止这一切了。现在我的所作所为曝光了,别人已经知道我在看这些网站,这已经不是什么秘密了。”\\
麦切和汉克都是被一场主要事件逼得走投无路的,而玛丽的生活早就踏上了毁灭之路,色情已经影响到了她生活的各个方而。她早就开始疏远孩子和其他人,也动摇了自己的宗教信仰,性兴趣越来越重口味,让她自己都觉得害怕。尽管玛丽当时一就清楚色情带来的问题了,但她束手无策,认为没有人会理解她。“我是一名看色情女性,我感到很羞耻,”玛丽说,“人们都认为只有爷们才会看色情,但是他们永远不会想到女性也会看。我只想把事情掩盖掉,能瞒多久就多久。”\\
牧师发现她观看色情这件事给她带来的压力,终于逼着她走上了绝境。这是压死骆驼的最后一根稻草,玛丽的孤单寂寞全部赤裸裸的暴露在阳光下。这件事让她无比尴尬、无比窘迫,逼着她不得不寻求外界的帮助。一周后,玛丽找到了牧师夫妇,向他们倾诉了自己的色情问题。他们耐心地聆听,丝毫没有指责的意思,还为她提供了很多治疗色情瘾的意见和建议。玛丽说:“跟他们说这件事让我觉得很不舒服,但奇怪的是,这对我来说又是一种解脱。”\\
汤姆的经历\\
汤姆,26岁,单身,一名药店职员。今年是他参加色情瘾治疗项目的第三个年头了。打有记忆以来,色情就一直在生命中纠缠着他。我爸爸在家里车库的小木箱里面收藏了上百本《花花公子》、《阁楼》之类的色情杂志,”汤姆说,“我把他的东西偷过来作为自己的珍藏。我爸爸从来没有提过这件事,所以我也觉得没有什么大不了的。有天,我妈妈大发脾气,让爸爸把收藏的所有色情杂志都扔掉,爸爸照做了,但是我很快就发现,他在水床下的抽屉里面又开始囤积色情杂志了。有时候,我会偷偷躲在他的躺椅后而,和他一起看《花花公子》电视频道,不让他发现。”\\
汤姆的童年一直i缺失父爱。“爸爸总是在外面,就算他在家,也从来不和我一起玩。我们从来没有一起运动过,也没有一起修过车,他一点都不关心我。我对他非常不满,所以我要看他的色情,这会让我感到我们之间存在某种联系的,我们有着共同的秘密。”\\
汤姆成年之后找了一份保安的工作,手头充裕了些。“我会去成人书店看录像,这些东西对我来说很新鲜。从那时起,我对《花花公子》之类的内容就提不起兴趣了,那些都太保守了。我当时还是处男,看到真实的性交会让我激情澎湃。我会买一大堆录像,后来又去买充气娃娃之类的性爱玩具。我的生活满是色情,虽然那时我还和父母一块儿住,但是我己经迷失在自己的色情小世界里了。”\\
“我太喜欢看色情了,事后又老是会愧疚,”汤姆对笔者说,”好几次.我把所有买来的录像都砸烂,把带子弄坏,把所有的色情杂志都扔到垃圾捅里。我并不想和父亲一样依赖色情,这样会让我觉得自己很恶心。在生理上我渴望色情,但在内心深处,我知道这并不是我想要的。但是,我最多就只能熬一个月时间不看色情,接着我会安慰自己,这没有什么大不了的,然后我又去租带子在工作的时候看。因为我干的是安保,身边都没有什么人,我就乘着工作时间对着色情自慰。”\\
汤姆也会在父母的房子里面看色情,收藏色情,每天对着色自慰五六次,但他也一直在忍受着精神上的折磨·。“记得有一次,我买了一大堆录像出了成人店去上班,接着我就哭了,我觉得精神上受到了巨大的折磨。我想要改变,但无能为力,也不知通到哪里去寻求帮助。那时我不过19岁,但已经离不开色情了。有时我暂停一段时间,但是脑子里无时无刻不在想着色情,看过的图片会在脑海里不停地重演,我尽力想要和这种念头作斗争。晚上的时候,我总是满脑子色情,没法入睡。”\\
频繁观看色情让汤姆在现实生活中无法确定健康性爱的界限。“我开始意淫身边的女孩子,对三个姐妹都起了淫欲,还意淫保安公司的那些女同事们。其实我内心也很羞愧,我当时想,要是我的姐妹和女同事们知道我这种龌龊念头的话,一定会和我绝交。这种羞愧感太压抑了,我深受折磨。”\\
汤姆有一位女同事情绪压抑,而她丈夫总是出差,他就和同事开始了婚外恋。汤姆还安慰自己说,自己会让她开心点。他那时还是处男。“她比我大一轮,而且有了家室,但她很看得开。我和她的这段不伦之恋持续了好久。第一次和她上床之后,我哭了。这一切都不是我想的那样。我以为性爱会给人很美好的感觉,会让人欲仙欲死。让我永生难忘,但事实根本就不是这样,那感觉和对着色情慰没有什么差别。”\\
“有了第一次之后,我也有了很多次性爱,但我 一直很难达到高潮。女友说我可能自慰多了,只有我自己心里清楚, 这是因为我脑里的垃圾太多了,所以很难对一个女人投人全部的身心,在精神上和她产生共鸣。那么多次自慰,那么多色情照片,那么多女孩子龌龊的念头。这一切都让我无地自容。我脑海中总是把女孩子们想象成色情受害者的样子,所以我觉得自已不配和女孩子交往。不配得到高潮和快感。”\\
“自慰能舒缓羞愧感。我会告诉自己:这就是生活的一部分而已,没什么大不了,我之前也把这事处理得挺好。我在那千分之一秒钟会感到羞愧,然后就自动屏蔽这个念头。但是和女孩子在一起的时候,事情就变得比较难控制了。我习惯了只关注自己的生殖器,忽略旁人的存在。因此,就算我有女友,有一个活生生的人在那儿,我还是不能满足。我把气都撒在女孩子身上,尽管我才是那个无法和她亲密的人。”\\
汤姆对色情越来越着迷了,他开始在女友办公室的电脑上看f色情,在家里父亲的电脑上看。那时他父亲已经退休了,父亲看完色情之后才轮得到他看。“有时候我很火大,‘’汤姆说,“我在心里为自己辩解:我爸爸没有权力看色情,但我还没有结婚,我才有权力看。”\\
汤姆和女友分手以后,又陷入了一段忘年恋中,这段感情持续了几个月。“那个时候,我父母己经离异,父亲再婚了,”汤姆说,“我一个人住,一天到l晚都在上网。我下班回家以后就开始看电脑,每天花8个小时光看色情。我那时候四处搜寻心中最完美的那幅画面。不过我总是可以找到一张在当时就可以满足我、让我高潮的图片。我不会重复看任何一张图片,每天晚上找的都是新图片。我喜新厌旧,看过一眼的图片很快就没了兴趣。我变得越来越麻木了。”\\
“我脑子里全是这种垃圾,想的全是色情。有大晚上,我在父亲的房间里给12岁的继妹按摩背部,结果我竟然去按摩她的胸部!当时我并不认为有什么不妥,好像这一切都很自然,而且她看起来好像有18岁了。但是,一瞬间,我的良心重重地惊醒了,好像那一刻我抛弃了所有的色情念头。我心里想:住手!这样是不对的!我离开了房间,在回家的路上把车子停在路边,泪流满面,号啕大哭。那是生命中唯一一次,我真的有了轻生的念头。我觉得生命己经被色情控制了,毁灭了。我并没有预谋要侵犯继妹,但事情就这样发生了,好像我的世界中其他一切都已经不存在了。这种状态类似于看色情时的感觉,我只要聚精会神做一件事就可以:只要想着自己,让自己得到性满足就可以了。”\\
“后来,我父亲和继母质问我关于继妹的事情,那是我一直等待的时机。他们质问:‘你搞什么鬼名堂?你到底过的是什么日子?’我彻底崩溃,泪流满面,说:‘爸爸,我自己也搞不清楚状况。’父亲老泪纵横,他说:‘我自己有色情瘾,还以为保持自己和你的距离,就不会让你受影响。我还是错了。’”\\
汤姆自首了,他以性侵犯继妹的罪名被逮捕。那刻,他觉得自己一无所有了。“但是现在我回头去看,发觉那才是我生命中最大的福分,那是我无数次在内心祈祷才得到的回应。我那时候厌恶自己到了极点,我成了父亲的翻版,我们都被色情所控制。就是因为色情把我逼到了穷途末路,我才走上了康复的道路。”\\
汤姆发现,如果个人有一长期的严重色情瘾,那色情就会在他的潜意识中培养性虐待倾向。性幻想和现实混杂在一起,在那个唯我独尊的自我世界里,性欲会暂时控制理性,让人不计后果。\\
深刻反思后,汤姆才醒悟到自己已经在色情陷阱中陷得太深,到最后势必会做出不妥的行为来。他知道自己有色情瘾,但无能为力他和已婚女人搞外遇,他对亲戚产生乱伦的性幻想,他的所作所为已经违背了健康性爱的标准了。色情扭曲了他的性思想,这必会引发他做出性侵行为。“我真希望时光可以流转,”汤姆感叹遒。“我愿意付出一切,来换取自已没有越界、没有性俊我继妹。”\\
尽管山穷水尽之时,人会极度心神不宁,但是本章提及的4位色情瘾者都最终意识到,他们的生活境带来的并不全是黑暗。那件导致他们无路可走的事件,也同时标志着生命的转折点,激励他们开始一段新的旅程,最终重获自由。麦切丢了饭碗以后,接受了一系列的咨询,并成功地开创了新事业,之后他还在教堂里领导创立并管理一个男性色情瘾康复项目。汉克在工作时间彻底崩溃之后,参加了一个住院的上瘾康复项目,成功戒除了他多种病瘾。在此后的人年间,他再也没有碰过色情,清清白白、神志清醒。玛丽参加了咨询项目、治愈研讨会和一个专门为女性设计的十二步性瘾治疗项目,她和孩子们又恢复了亲密的关系,而且她也开始结交新朋友。汤姆呢?在服刑之后,汤姆参加了个人咨询、十二步治疗法和集体咨询项目来戒除色情瘾,习得了健康的性观念,现在的他能够正确规划自己的性生活,也成功修复了和家人的关系,他感到前所未有的自信。\\
麦切、汉克、玛丽和汤姆,他们并没有一直在悲剧里呻吟,而是用自己山穷水尽时的痛苦来激励自己戒除色情,开始人生旅程的新篇章。\\
第二部分 治愈\\
人生最重要的事莫过于:\\
你曾经爱过有多深?\\
你曾经活得多完整?\\
你曾经学会放手得多彻底?——杰克·科菲尔德,佛学教师\\
第七章 获取戒除色情的动力\\
世上没有什么可以阻挡一个有良好心态的人达成目标;;也没有什么可以帮助一个心态不良的人有所成就。——托马斯·杰弗逊(美国《独立宣言》起草者,美国第二任总统)\\
不需要提醒,你也知道能戒除色情瘾是件好事,可能你也早就想这样做了。当色情带来的影响超过其所能提供的快感时,多数深陷色情陷阱的人都会渴望从中逃离。因为他们早已厌倦了自欺欺人,厌倦了在人前的伪装,更厌倦了怕被逮到而惶惶不可终日的焦虑。真心想要治愈色情瘾的人,定是想要改善生活状态,增进和伴侣之间的感情,保持自己正直的人格,得到家人与社会的尊重。一句话,他们想要重获生命的主控权。\\
治愈色情的决定意义重大,足以算是生命中的一个里程碑。它代表着更高层次的成熟和自律,也是个人开始摆脱色情的起点;同时,这个决定了为个人敞开无色情生活的大门。但是,要真正治愈色情瘾确实不是一件易事。前文讨论过,色情的吸引力之大,足以让陷人迷幻状态,失去思考和判断的能力。色情习惯会演变成为一种对抗压力的惯用方式,一种释放能量的快捷途径。\\
决心要医戒除色情后,人们总是会悲哀地发现,要彻底放放手的很难。这不仅仅是简单地做承诺、扔掉所有色情品或是切断网络就可以解决的问题。就算戒除的意愿非常真切,但想要看色情的欲望还是会非常强烈。一般人戒除色瘾的初期尝试通常都是以失败告终的。\\
40岁的达勒医生,在尝试戒除色情瘾初期时遇到了重重阻碍。“为了维护婚姻我打算戒除色情瘾,当时我以为很容易,”达勒说,“几天后,一天晚上上床睡觉时,我突然特别想看色情,那时候,保证承诺什么的都抛到了脑后,我起床去开电脑。走了几步我就停在客厅里,想起自己做出的承诺。纠结几分钟以后。我想:承诺都去死吧!就径自走到电脑旁。后来我又改变了主意,走回卧室去睡觉。我犹像不定,几个小时之内都在卧室和客厅之间不停地穿梭。那晚我还是没有看色情,但那个晚上就是一场噩梦。一周后,我就开始看色情了,那时我才意识到色情对我来说有多么重要,要彻底摆脱色情没有想象的容易。”\\
如同被困在旋转门里,人在逃离和沉溺之间摇摆,极易感到沮丧和疲惫。62岁的尼克成功克服了色情,他对笔者说:“在康复阶段初期,我觉得自己陷入了一个恶性循环,我会去看色情,然后感到内疚,告诉自己再也不能看了,但是我又会去看,事后又觉得罪过,羞愧不已。我又暗暗发誓说,我绝对不会再犯了,但几天之内我又会违背誓言,那么轻易地就自己的承诺忘得一干二净,又去看色情,真是不可思议啊。”\\
达勒和尼克想要戒除色情瘾的初次尝试都以失败告终,让他们失望不已。也引发他们内心的激烈斗争,很多人都经历过这样的内心煎熬。我们不禁要问,为什么那些已经下了决心的人,内心仍然会动摇,继续观看色情呢?\\
要回答这个问题,首先要理解两个关键词:矛盾(Ambivalence)和动力(Motivation)。\\
矛盾是指内心存在着两种截然相反的想法:尽管个人已经下定了决心,但潜在精神层面上仍存在着反对的声音,这导致意念容易动摇。矛盾是一种普遍存在的心理,要放弃某些曾让我们感到欢愉和满足的事物时,内心往往会有这种情感。色情成瘾者常常会有这种矛盾心理,如“我想要偶尔看看色情,会感到很快乐和刺激”,“我想要戒除色情,它给我带来了很多麻烦,伤害了我和关心我的人”。起初下决心戒除色情而后产生犹豫不定的心态,即矛盾心理。\\
要成功戒除色情瘾,首先要克服这种矛盾心理。也就是说,“我想停止看色情”的心理强过“我想看色情”的心理,这就是戒除色情瘾的动力来源,也是心理天平倾斜的关键。动力并不仅仅指想要完成某事的意愿,它还包括明确的目标,调整行为的能力以及持久的恒心。强烈而持久的内在动力是戒除成瘾性行为的前提条件,在戒除色情的过程中,内在动力能打下稳定的基石,让所有的康复步骤有序进行。戒除色情的决心必须足够清晰、强烈,才能让人抗拒色情的诱惑。动力越强大,越持续,就更容易克服矛盾心理。“我想要戒除色情”的愿望就会大大超越“我想要看色情”的欲望。\\
本章将着重谈论如何有效解决矛盾的心理,加强并保持个人动力。笔者提供的方法和练习会帮助读者认识到,我为什么要治愈色情瘾?生命中什么人和事对我来说是最重要的?我该保持怎样的正确心态,成功地实现目标?\\
矛盾心理越少,动力越强,就证明你已经做好了准备,能够勇敢地直视前途中的艰难于康复迸·程二险阻。你不需要浪费时间和精力做心理斗争,怀疑自己是不是已经做好准备,而是要充分利用资源,利用本书第八章中的策略来实现康复。如果个人有了充分的准备,那么戒除色情的路途就会更加坦荡,更容易实现目标。或许某时你会担心功亏一篑,只要你能够调整情绪,转变想法,避免重蹈覆辙,保持头脑清醒,就能专注于康复进程。\\
要解决矛盾心理,加强并维持动力,最终实现目标,以下4种策略必不可少,本章会详细介绍每一种策略。读者在阅读时可以结合自己的处境,思考如何实施策略来刺激个人动力。4种策略包括:\\
1.承认色情带来的问题\\
2.确定人生中什么才是最重要的\\
3.直视自己的恐惧\\
4.为自己的康复负责\\
策略一:承认色情带来的问题\\
这种策略指:确认、追踪色情给你带来的问题,预测继续沉迷色情将会带来的问题。如果你能清醒地认识到色情在过去如何伤害了你,现在如何继续破坏你的生活,未来又将如何继续威胁你的人生,你就能越坚定地在康复道路上坚持下去。这一步至关重要,因为许色情成瘾者都已习惯视而不见这个嗜好的影响。\\
部分色情观看者不愿意直视色情的负面影响,拒绝承认事实。对那些在主观意识中只关注色情好处的人来说,这种拒绝心理非常有效。为了保持戒除色情瘾的动力,首先要消除这种否认心理。只有理性地正视色情给自己和他人带来的麻烦,才能认识到问题所在,如果能长期保持清醒的头脑,就可以杜绝拒绝心理。\\
要分辨色情带来的影响,办法有很多。你可以和挚友或者咨询师谈论已有的或是潜在的问题,在日记中写下自己的经历和忧虑,时常自我反省,这些方式都非常有效。读者也可以翻阅第四章第67页中讨论的色情消极后果作为参考。此外,读者还可以回答以下练习,“色情带来了什么麻烦?”\\
色情带来了什么麻烦\\
给自己一小时的时间,确保自己不会受到打扰,然后认真回答以下问题:\\
过去色情给我带来了什么恶果?\\
现今的我又面临着色情带来的什么问题?\\
色情如何以我不喜欢的方式,改变了我的生活?\\
色情如何伤害了我的伴侣和那些关爱我的人?\\
如果我继续沉迷色情,还可能会有怎样的恶果?\\
笔者建议读者在纸上写下回答,然后把指放在伸手可及的地方,比如钱包里、书桌上、床柜里或是电脑旁,确保自己随时可以拿到。反复阅读自己的回答,尤其是自己快无法遵守承诺,不可自制地想要观看色情时。当你对色情的负面影响有更清晰的认识时,请修改或补充答案。当然,读者也可以和咨询师或者挚友交流。\\
很多色情瘾康复者告诉笔者,他们经常回想色情给自己、给自己在意的人带来的痛苦。比如,伊森经常反思一件让他十分痛苦和丢脸的事。几年以来边观看色情边吸食大麻的习惯,让他喜欢在现实中实践自己喜欢的性爱镜头,如捆绑,在伴侣身上小解等。伊森以为性伴侣都不会抗拒,但是有次他这么做了之后,女友非常委屈,还警告小镇上的其他女孩不要接近他。一天,在一家人山人海的店铺里,伊森遇到了那位女友的闺蜜,对方大声谴责他的性行为。“我的私生活就这样在大庭广众之下被广播出来了,我毫无防备。”伊森说,“当时我呆呆的,觉得自己的脸都丢尽了,但这件事也让我意识到自己的性欲已经变得危险,它会让我忽视伴侣的需求。尽管这件事让我非常尴尬,但我还是会经常回想,把它当做警示,提醒自己不要再看色情。”\\
35岁的劳拉已经成功戒除了色情瘾,回首往事,色情曾让她陷入非常危险的境地。“我那时候养成了习惯,一出差就在酒店房间看即时付费的色情节目,”她说,“我会性兴奋到不行,然后就开始自慰。一段时间以后,这样已经不能满足我了,于是我看完色情以后就会在酒店休息室随便找男人搞一夜情。色情让我性幻想,而独自一个人的性不能满足我。事情开始失控了。我染上了性病,还受到了性虐待。我一辈子都不会忘记,当时我有多堕落,感觉有多恶心。”\\
色情观看者也可以预测继续观看色情的后果,以此来加强动力。预测甚至确定未来的问题,会有效加强动力,让自己在戒瘾道路上继续前行。色情的恶果包括:被逮到观看或者收藏色情;产生不健康的性兴趣;失去一段重要的情感;丢饭碗;失去朋友、家人和同事的尊重;在公众场合被羞辱;参与危险的、非法的或含有性虐倾向的性行为;让孩子过早地接触色情;对色情上了瘾,没有它就活不下去;触犯法律;等等。读者也可以从实际情况出发,确定色情将会给自己带来的后果。比尔是一名35岁的证券经纪人,因为害怕被别人逮到自己看色情惶惶而不可终日,最终痛下决心戒除色情瘾。“我会在公共大楼的办公室里打手枪,尽管门锁着、窗户关着,但我知道,在楼道里走来走去的人会经过我的窗户,如果他们透过窗帘往里而看,就一定会看到我。在家的时候,我无时无刻不在担心妻子会逮到我,或者会有警察来敲门,因为我电脑里面那么多的儿童色情图片来逮捕我,我知道这些罪名足以让我进监狱。所有这些忧虑让我变得神经兮兮的。”\\
艾利克斯19岁时就已经沉溺于色情了,而他展望未来的时候,终于醒悟色情最终会影响自己的恋爱和婚姻。他说:“色情的唯一目的就是让你对着不是你老婆、不是你女友的女人产生性趣。我知道再过几年自己就会结婚,所以我预感现在是时候该停了。我不想有了老婆和孩子以后,还躲在办公室意淫。”尽管艾利克斯还年轻,但他也担心色情瘾会影响自己的性健康,危及他在教会团体中的精神领袖地位。“在我生命的这个阶段,色情所带来的问题已经远远超过了它的好处。色情确实给了我乐子,但是我已经知道,色情将来只会影响我的恋爱关系,扭曲我对性的看法。我能康复,就是因为我时常提醒自己:色情真的会毁了我的人生。”\\
不管着眼于过去,现在,还是未来,只要下决心消除矛盾心理,不断激励自我,就迈出了可喜的一大步。每个人都会经历一定程度的情感波动,请不要回避。时刻牢记,其实这种情感在加强、促进你的决心。适度的情感压抑会促使个人作出重大决定:不然,个人很难维持动力;如果压抑过度,个人又容易被击垮,重新回到老习惯那里寻求安慰。\\
策略二:确定人生中什么才是最重要的\\
另外一种能有效加强和维持个人动力的办法就是直面色情的影响。承认色情是如何违背自己的价值观,破坏自己的人生信念。笔者多年的咨询经验证明,当个人开始说出这样的话时,就证明他己经在戒除色情瘾的道路上稳步前进了:“色情完全违背了我的人生观”、“ 我必须要作出抉择,而我选择戒除色情瘾”i、“我不能一边希望达到理想状态,又一边沉迷于色情”。\\
要确定色情是否违背人生信念、个人首先应该清楚,生命的追求是什么,价位观是什么,自己想为家人和社会做出怎样的贡献,又希望给别人留下怎样的印象。也许你认为每个人都明确生命的意义所在,但许多身陷色情世界的人,并未曾认真思考生命的目标和人生的价什。一些孩子们甚至都还未曾考虑过,自己想要成为什么样的人,想在生命中获得怎样的成就、就稀里糊涂陷入了色情陷阱。\\
建立并坚持自已的价值观不变,这对每一个人来说都至关重要。而对色情瘾者来说,能确定自己的价值观就意味着生命中一个历史性的飞跃。观看色情时,价值观不仅会激起对自我身份的疑惑,引起激烈的心理斗争,也能提供良好的心理基础,为康复过程提供更多积极的心理暗示、以加强决心。\\
若想更清楚地认清并维护生命之重,办法有很多。个人可以每天抽空评估自己的价值观和人生目标,阅读一些关于设定价值观、道德观和人生目标的书籍,和家人、咨询师或者知己谈论自己生命之重。读者也可以通过回答以下问题,来更明确地了解“人生中什么才是最重要的”。\\
对我来说什么才是最重要的\\
下列问题可以帮助读者确定自己的价值观和人生目标。通过思考,你能够更清楚地了解生命中最重要的事。请确保自己有足够的时间来认真考虑和发掘自己的想法。尽可能准确、详尽地回答。\\
1.我生命中最重要的六件事是什么?\\
2.我想达到怎样的人生目标?\\
3.我信仰的道德观和价值观是什么?\\
4.我的宗教信仰或者精神饭依是什么?\\
5.我生命中最重要的人是谁?我准备如何时待他或她?\\
6.我想给他人留下怎样的印象?\\
7.我想怎样为我在意的人付出?\\
8.我想怎样为自己所在的群体和社会做贡敲?\\
9.色情瘾如何违背了我的价值观、信仰和人生目标?\\
读者回答完这些问题以后,请将答案放在随手可及的地方,也可以和挚友、伴侣和咨询师等人交流。尽管别人不能替你决定问题的答案,你也可以从交流中寻求建议。获取灵感,加强个人意识从中得到领悟。随着治疗的推进,读者需要随时更.新、补充这些问题的答案。\\
许多色情瘾康复者证实,时常反思生命之重有助于个人坚定戒除色情瘾的决心。作为教堂青年团队的领袖,西蒙发现强迫性色情瘾有悖于自己的现实身份。他说:“我明自了,自己嘴上说信仰的东西,和色情放到我脑子里面的东西不能兼容。尽管现实生活中我没有通奸,也没有犯什么罪过,但我在脑子里已经把坏事都做尽了。在我的意识中,一山不容二虎,精神信仰和观看色情的念头绝不可能共存。我不能既领导着青年集体,拥有高尚的精神生活,又跑去店买《阁楼》杂志,然后回家对着书打手枪。这样太不和谐了。”\\
科里要追求的是长期的稳定情感,他说:“我最终醒悟,色情只会让我越来越孤僻,没法跟人亲近。色情派发我和性欲,给我性快感,但带不来认同感和爱情;相反,它让我很难拥有健康的性关系。色情就像是毒品,会让我性紧张,让我渴望性爱。我想体会自然舒适的性爱,学会真正地去爱惜、关系一个真实的伴侣。\\
当马科斯开始明确自己的价值观时,才察觉虚幻中的色情和自己尊重女性的想法无法融合。“大学期间,我认识了很多女性同学和同事,我们经常一起工作,参加项目,还经常聊天,这种关系跟性完全不搭界。这些女性都是我的朋友,我们都是平等的人,而色情总是侮辱女性,把她们踩在脚下。我不能和女性朋友们聚会之后,又回家对对着色情自慰。在那些色情镜头里面,女性只被当成了性玩物。”\\
一旦明确了自己的价值观和目标,个人就能更清楚地了解色情是如何阻碍自己实现个人价值的。尼克意识到自己因为色情而变得虚伪之后,便决心戒除色情。“我厌倦了自欺欺人,厌倦了对人撒谎,”他说,“我整个生命就像是一个弥天大谎。只要我还在看色情,就不可能有端正的品行。我必须要言行一致。”\\
从西蒙、科里和马科斯的经历可以看出,了解色情和个人价值观之间的冲突,可为改变现状、坚持治疗提供强有力的支持。\\
许多色情瘾康复者认为观看色情与自身的一些品性特质格格不人。而正是因为他们发现了这种矛盾的存在,才能促进自己成功脱离色情陷阱。下面的练习“我想要成为……的人”列出了与观看色情相矛盾的个人品性特征,读者可以选择自己重视的内容,在相应的横线上打钩。另外,请读者思考色情如何与自己选择的这些品性相冲突,色情又是如何毁了自已的抱负,破坏了自身的价值观。\\
我想要成为……的人\\
_诚实可信\\
_言行一致,遵守诺言\\
_言行中体现出对他人的尊重\\
_付出真心来获得真爱,只把性欲留给对方\\
_非常诚恳\\
_参加高效的、有意义的活动\\
_在群体中塑造楷模形象\\
_大家公认的好人\\
_自尊,自信\\
_能控制自己的冲动\\
_有道德感的守法好公民\\
_有自豪感\\
_尊重女性和儿童\\
_能得到别人的敬佩和尊重\\
_不会在精神上、肉体上折磨自己的伴侣\\
_过着有信仰、有精神依托的生活\\
_保护家人不受伤害\\
_无不良嗜好,无强迫症\\
_不支持色情产业\\
_精神和肉体上的双重健康\\
_在性爱时可以表达出浓浓爱意\\
_(其他)\\
阅读完上表,请读者以书面形式说明,自己选择的品性特征和色情之间存在怎样不可调节的冲突。比如说,读者可以写:“我不能在看色情的同时还想着做一个诚实可信的人。”大声说出自己不能做的原因,这个办法非常有效。“我不能既看着色情,又想着做一个诚实可信的人。观看色情让我变得偷偷摸摸的,还要对别人撒谎,说我在干别的事。要是这样,我就不能奢望别人信任我。”同理,如果读者想要成为自豪的人,就可以说:“我不能既看看色情,又想着要成为有自豪感的人。每次我看了色情之后,都会觉得羞愧、罪过。”因为自己选择的每一个品性写下它和色情格格不入的原因,然后每天大声回答。这是一个明确自己价值观和人生目标的过程,这会让色情观看者了解到,色情是如何和这些价值观以及人生目标相冲突的,这是一种加强个人动力的绝好方式。\\
策略三:直视自己的恐惧\\
如此多的色情观看者常常摇摆不定,失去了在康复道路上继续前进的动力,原因之一就是:恐惧。在戒除色情的过程中,恐惧感不可避免。要作出改变人生的重大决定,独自面对未知的前途,起初都会让人觉得害怕,即使内心清楚这种改变有百利而无一害。在戒除色情的早期阶段,许多人害怕,断绝与色情的联系就意味着失去了即刻的快感和感情的慰藉。戒除色情会彻底改变个人的生活方式,比如说和色情瘾说拜拜,找到新方式来对抗情感压力,学习如何更坦诚地和他人交往。不用说,在情感和性方面对色情依赖性最强的人,最害怕放弃色情。\\
很多情况下,恐惧感深埋在潜意识之中,被其他情感所掩盖。比如说,戒除色情瘾的初期,很多人都会感到焦虑和压抑。他们不了解,正是恐诱发了这些负面情感。如果人们无法发现这种恐惧心理的存在,不做出具体行动来克服,那么恐惧就会渐渐浸蚀个人的动力,破坏个人康复的努力。就好像悄然扎进汽车车胎中的一个钉子。这种无法确定的恐惧会阻碍康复进度,而你却毫不知情。个人潜意识中的恐惧感越强,就越难彻底戒除色情。\\
认清并且承认内心的恐惧感,这是克服负面心理的第一步。大部分受访者都认为在戒除色情的初期,这样做非常有利。下表列出受访者的回答,希望帮助读者判断内心的恐惧。\\
失去色情后的普遍恐惧心理\\
请在将合自己条件的横线上打钩。\\
_我害怕变得抑郁。\\
_我害怕变得易怒、沮丧。\\
_我害怕感到孤独。\\
_我害怕感到焦虑不安。\\
_我害怕没有色情就无法自慰。\\
_我害怕失去性能力。\\
_我害怕对性爱失去兴趣。\\
_我害怕不能从性爱中得到足够的快感。\\
_我害怕性爱受挫。\\
_我害怕做出危险的性行为。\\
_我害怕在性爱上过于依赖伴侣。\\
_我害怕自己变成‘’伪娘”,或者在性方面放不开。\\
_我害怕自己向别人倾诉之后,别人会鄙视我。\\
_我害怕没人能了解我、帮助我。\\
_我害怕戒除色情的努力会以失败告终。\\
_其他恐惧\\
正如上表所示,戒除色情瘾的过程中会产生的恐惧感可分为三个方而:情感状态、性爱欢愉和人际关系,观看色情会至少暂时满足一个方面。回到上表,查看自己选择的选项,请区分哪些和情感、性爱或者恋爱相关。你的恐惧分数在每一种类别项,还是集中于某一类别?了解自己内心的恐惧,有助于发掘自己在康复过程中需要特别留意的方面,如此你就可以放弃色情,通过其他方式来满足这一方面的需求。比如说,读者害怕戒除色情以后自己会变得孤独,那么你就通过其他事物来排遣自已的孤单感,以此克服这种恐惧心理。即使没有了色情,自己也不会感到孤单。\\
失去重要的情感依赖后,除了会产生孤单感之外,还会产生其他负面情绪,如害怕、压抑、易怒、焦虑不安等。这些负面情感在戒除色情的头6个月内最为明显。如果个人坚守阵地,那么这些负面心理在强度、频率和周期上都会减弱直至消失。如果出于某些原因,这些负面心理困扰个人的时间超过预期,甚至影响到日常生活,那么你就可以向外界寻求帮助,通过药物和精神治疗来对抗负面情绪。\\
很多色情观看者担心戒除色情会影响他们的性能力,因为他们觉得戒除色情瘾就等于失去性爱的机会,导致自己以性失去积极性。这种心理情有可原,尤其对那些把色情当作性发泄首选的人来说,这种担心更为明显。遇到这种情况,你可以提醒自己,戒除色情只是关上了一扇窗,让你失去一种体验性的途径;但它也同时为你打开了无数种欢愉的性体验大门。色情中描绘的性行为会误导你的性理念,而戒除色情之后,健康的性爱方式取而代之,帮助个人树立自尊,培养稳定的恋爱关系。把色情当作无数种性爱方式中的沧海一粟,而且还是有缺陷的一种,戒除色情并不意味着生命中的性选择的减少,而是增多。如果这样想,内心的恐惧感就会减少。\\
艾德在戒除色情瘾的过程中,把自己的性恐惧感全都记录下来了。在日记中,他不断鼓励自己,坚信凭借自己的力量,一定可以应对未来的挑战。“我的性欲可能会经历一个低谷,发生一些改变,”他说,“我要做好万全准备。改变会带来各种问题,我会找其他法子来得到性快感,把这些新方式培养成习惯,这样就没有后顾之忧了。我可以创造自己的性幻想。”读者也可以时常反思,把自己的恐惧,以及对抗它们的方式用书面形式记录下来。\\
挖掘恐惧下的错误观念,用真相来粉碎它们,是保持动力的另一种有效方式。比如说,一些男性色情观看者担心,放弃色情意味着损害对自我身份的认同感,失去爷们儿本质。他们会想,如果我不看色情我就不算是一个爷们儿。这种恐惧基于陈旧的理念,即传统教育对男性本质的误解,而这种误解被色情一再强化。仔细分析,质疑这种观念的合理性,就可以轻易地推翻这种理论。兰迪正在与色情瘾作斗争,他说:“我打不就认为,爷们都看色情。下决心戒除色情瘾的时候,我曾经担心别的男孩子要是知道了就会对我有看法。后来我才意识到,只有一个真正强大的男人才能戒除色情瘾这种顽固的恶习,而且,只有一个真正的男人、一个不看色情的男人,才能真心去一个女人。”\\
大声说出自已的恐惧也是消除恐惧的好办法。当你大声说出内心的恐惧,这种负面情绪的影响力就不再显得那么绝对、那么强大了。光天化日之下,有些恐惧忽然间就显得无足轻重了。适应能力与生俱来,尽管看了色情多年,个人还是可以通过学习新方式来克服色情瘾。你可以向朋友或专业人员求助,利用各种资源来助自己一臂之力。你就越能确定自己的恐惧,并且合理表达出来,那你就越能获取所需的帮助。\\
不管个人恐惧的本质是什么,大胆地说出来,减少它的负面影响,保持戒除色情的动力。当你开始挑战自己的恐俱。培养积极的观念,你就会感到自已的强大和勇气,继续在戒除色情的.道路上越走越远。请时常提醒自已,任何一个勇敢戒除了色情瘾的人,都曾经体验过你如今面临的恐惧,而他们劫后重生,更加强大,只因他们有足够的勇气。不让恐惧阻止自己前行的脚步。\\
策略四:为自己的康复负责\\
如果你想要维持动力,成功越出色情这个监狱,那么最重要的心理准备莫过于自己要为康复承担责任的决定。如果你认为自己是迫于他人的压力才戒除色情,主动权并不在自己手里的话,就很难维持这种动力。这种情况下,你还需要克服重重怨恨和痛苦心理障碍来实现成功。\\
他人或许可以为你提供温情的鼓励和支持,但若真要打破自己对色情的情感和生理依赖,个人还是要自己努力。汉克说:“色情不会哪天有空了,走上前来跟你说:‘小子,你不看我也没有关系。’改变要从内心开始,只有你才能决定什么时候开始行动。你必须要告诉自己:我再也不会看色情了。总有一天,我要和色情彻底划清界限。”\\
对自己负责需要个人培养新技巧,在这场一个人的战役中兼职教练、中卫、裁判和拉拉队等数职。你主管的内容包括设计策划,实施方案,立下规定,监督表现,在碰壁时还要策划新方案,在成功给自己奖励,等等。\\
想要在这场战役中占上风,最好的战略就是:列出一张清单,详细描述你曾克服过的每一个艰难险阻。人们都遇到过困难,不管是要戒除坏习惯,还是要与病魔和伤痛抗争,或者面对经济问题。所以我们每一个人都应该挖掘自己的潜力,培养适合自己的策略,以应对新的问题和困难。部分人会选择和挚友或者专业人员探讨,而其他人更愿意独立解决问题。一些人希望尝试所有可行的办法,直到发现最好的对策为止,I而另一些人则希望设计一个万全计划,万变不离其宗。简单回顾自己的成功史,可以让自己确信,尽管改变落畏艰难、具有挑战性,但你已经克服了那么多困难,这次也一样会成功。\\
伊森选择“不成功便成仁”的心理暗示方式来戒除色情,让自己掌控康复过程,以保持动力,他以前也用这种方法戒除了大麻。“刚开始,我只是远离色情一段时间,”他说,“我表现得就像是确定要戒除色情一样,而不是等着自己想通的那一天。这样我就有足够的机会来发现,没有色情的生活多么健康。你懂的,你要是不给狗狗其他诱饵,它是不会把嘴里咬着的骨头吐出来的。我也是个俗人。如果我自己很享受一件事,但是有人告诉我,这事对我不好,让我戒掉,我是戒不掉的。我需要亲身体验一下,没有了它之后生活是不是改善了。放弃色情刚开始感觉就像是禁欲,但我就是坚持不碰色情,几周后我就发现,这样做是对的。”\\
有些人发现,通过改变自己的内心想法,就可以让自己承担更多责任。当脑海中出现了消极想法,如“我需要色情”,“我控制不住自己”,“我永远都不可能戒掉”,他们就立刻培养积极的心态,如“只要我能想,没有什么做不到的事”,“我知道到哪里寻求帮助”,“我过去就做过很大的改变,这次我也会成功”。\\
一句老话,你不能在上山的时候就想着下山的路。这个道理很简单,一旦改变了每天内心的想法,就能在康复道路上拥有强大的精神动力。色情抢走了罗波的婚姻之后,他不断给自己积极的心理蝉不,强调自己内心向善,从而成功戒除了色情,改善了生活。“我有坚定的精神信仰,”他说,“我信仰的教条之一就是:世界上有一种崇高的力量,世人称之为上帝,我是他的爱子。我本人就是这种种精神力的表现,在这种力量中,有善、有爱、有光。正因为如此,我能够成为一个善人,过没有色情影响的生活。”\\
无论是通过言语、思想还是行动,只要你在康复过程中越能取得掌控权、就越能感到自己的强大,能够控制命运的魄力。“这个过程不只是寻求戒除色情瘾的办法,”尼克作为教堂团队的领袖,曾一度沉溺于色情,他如是说,“它也是一种负责任的表现。你一定要足够坚定、决断,为自己的行为负责。你必须要清楚自己所想,然后坚持自己所想。如果你不够坚定,那你就一次也没法对抗色情的诱惑了。你要足够坚定,每一次想起色情时,你都要完全掌控自己的身心,和色情决战到底。”\\
承认自己的色情问题,清楚自己的生命之重、直视自己的恐惧,为自己的康复全权负责,以此加强动力。能将这些行为付诸实践的话,成功女神就在不远处对你招手了。请注意,你不需要等到动力十足时才开始戒除色情瘾,只要你愿意主动出击,动力自然而然就会加强。笔者在下章中介绍的康复步骤将会帮助你排遣孤独感,逃脱色情的窒息、掌控之后,生活会得到极大的改善,同时个人也会在更加健康的性活动中得到满足。\\
读者在接下来的几章中将会了解,要戒除色情不仅仅要求个人转变思想,也要改变行动。健康的心态和积极的行为相辅相成:积极的心态激励人改变行为,而积极的行为改变又会促进个人培养良好的心态。不管使用哪一种方法,要肯定的一点就是:只有你自己能选择改变。戒除色情瘾的有效办法很多,但掌控权都在你手中,只有你才能确定什么时候实施,什么时候开始改变。\\
正如汉克所说:“如果你真心想要戒除色情,那就付诸实践吧。只要你勇敢地踏出第一步,一切就会自然而然发生,你会感受到没有色情影响的生活是多么美好。但是,除非你自己决定改变,不然一切依旧。”\\
第八章 六个基础步骤\\
戒除色情瘾需要个人付出时间精力,也需要他人给予的关爱和理解;它需要个人的坦诚,也需要他人的信任;而它最需要的是个人的当机立断。——麦切\\
成功戒除色情瘾者的康复故事各不相同。一些人分阶段逐渐戒除色情,每周减少观看次数,直到最后杜绝。而另一些人从观看硬性色情转变到观看软性色情,最后什么也不看。还有一些人立刻停止观看色情,利用强大的意志力来抵抗随之而来的不适感,直到欲望逐渐消失。尽管各人策略各不相同,每一位色情瘾康复者都不约而同地采取了一些特定步骤,确保成功戒瘾。\\
基于笔者多年的咨询经历、临床研究以及采访成果,本章将会介绍六个戒除色情瘾的基础步骤,介绍如何向他人求助,如何减少接触色情的机会,如何培养健康的生活方式和行为举止来创建美好无色情的生活。如果读者目标明确,付出实际行动来执行步骤,那色情的诱惑自然逊色不少,抵制色情也就变得轻松了。下一章将会介绍下一阶段的策略技巧,帮助个人进一步强化目标。了解自己的选择范围,知道如何执行,这样才能帮助你审时度势,向无色情的美满生活迈进。\\
六个基础步骤\\
笔者将这六个步骤称为“基础 ”,并不是指这些步骤容易执行。而是因为这是戒瘾过程的根基。未完成这六个步骤,就很难作出彻底的改变。这些基本步骤可以在戒除色情瘾的过程中促进和培养个人自我支持、自我关爱的心理。只有善待自己,给自己足够的时间和自由。才能找到足够的情感支持和现实资源来实现康复。综合实施这六个步骤,可以削弱前文提及的诱导,即恶化色情瘾的因素,包括孤独感、色情轻易可得的特性、个人的精神压力、上瘾倾向、性安全感缺乏以及性沮丧感。\\
戒除色情所必须采取的六个步骤:\\
1.对别人坦白自己的色情瘾问题\\
2.参加治疗项目\\
3.创造一个零色情的环境\\
4.建立24小时支持和责任机制\\
5.爱护你的身体和精神健康\\
6.开始治疗自己的性欲\\
实施这些步骤的难易程度取决于各人的不同情况,实际上,最具挑战性的步骤,才是最有效的步骤。如果读者觉得某个步骤实施起未有难度的话,就要扪心自问:我为什么会有这种反应?对我来说。这个步骤的挑战在哪里?如果要尝试这个步骤,我需要担心什么?要完成这个步骤,我需要做什么?如果某个步骤对读者来说挑战性太高,那么笔者建议读者不要有所顾虑,而是要勇敢地迈出第一步,一步步向着目标迈进,不管进度有多缓慢。为了更直观,笔者在介绍每一个步骤时都会提出可供选择的实施方案。\\
请注意,这六个步骤不是毫无关联的,它们相辅相成,共同发力。不要把这些步骤当成水沟里零散的垫脚石,而要把它们当做一座大桥的各个部分。每个部分越是坚固,整座桥也就更加稳固,这样才能更好地面对未来的挑战。比如说,有人关心你、鼓励你,承受的压力刺激出了自己的正能量,这些都会成为你参加治疗项目的动力,其中任何一个步骤都缺一不可。笔者发现,许多人无法成功戒除色情瘾,就是因为他们没有按部就班,全面实行这六个基础步骤。因此在戒除色情的过程中定期评估每一个步骤的实施进程,那么,你就应该花费更多的时间和精力专注于此。\\
下文将详细介绍六个步骤。\\
第一步:对别人坦白自己的色情瘾问题\\
“我有色情瘾问题,现在想要戒了。”当色情观看者第一次对着别人大声说出这句话的时候,他已经从色情陷阱中逃离了可喜的一大步。前文讨论过,个人对色情的依恋源自于个人的孤单寂寞和否认心理。开诚布公地和别人谈论自己的问题,自然会削弱和色情之间的联系,谎言和欺骗就显得不再那么重要了。一瞬间,长久以来自己千方百计偷看色情,维持这种私密生活的努力也就显得微不足道了。仅仅是有人知道你看色情,就会大大削弱色情的吸引力。许多色情瘾康复者称,主动“打开天窗说亮话”引发了他们戒瘾的动力。\\
倾诉色情问题并非易事,因为这意味着主动承认自身弱点,承认自己对生命的失控,甚至承认自己恶劣的变态性行为。因此,要说出这些话,会让人产生负面心理,觉得羞愧、罪过、焦虑不安。\\
色情观看者和大部分地球人一样,很少开诚布公地谈论性,免得尴尬。你可能会担心,倾诉对象会因为色情问题而指责你,批评你,鄙视你,最终否定你的人格。这种顾虑普遍存在,也是可以理解的。47岁的艾德,已经成功地戒除了色情瘾,他将自己袒露心声初期时的担忧总结为:“我觉得很丢人。我当时认为,如果其他人知道我沉溺于色情,就再也不会喜欢我了,他们会瞧不起我,觉得我变态,不想和我在一起。”艾德最终还是选择向别人倾诉,尽管他不情不愿,但还是选择开口,因为他已经到了不得不戒除色情瘾的关键时刻,而且他也意识到,这是要戒除色情瘾的必经之路。\\
担心被当作性变态的心理在女性色情观看者中特别常见,因为色情被当作男人们的消遣。女性会担心,比起男性,女性观看色情的行为更让人难以接受。并且,不论男女,只要自己的色情问题涉及暴力内容或者儿童色情,个人都会守口如瓶,这是人之常情。考虑到以上种种因素就可以发现,要走出这第一步确实需要足够的勇气和决心。请记住,向别人披露自己的色情问题并不是为了让自己感觉良好(好与不好视情况而定),而是因为这是戒瘾的必经之路。\\
选一名支持者倾诉:倾诉的时机、对象、地点、方式和程度,都是由个人自己掌控的。稳妥起见,个人可能会选择一位稳重可靠、成熟理智的人作为第一个倾诉对象。\\
对训练有素的专业人士倾诉自己的色情问题是个不错的问题。笔者的许多客户称,我们是他们第一次倾诉色情问题的对象。就环境来说,治疗师的办公室相对比较安全,而且倾诉对象是有治疗色情瘾经验的专业治疗师。也有一部分人选择神职人员或健康专家作为第一倾诉对象。近几年来,越来越多的牧师、拉比(犹太教经师或神职人员—译者注)、神父、医师等群体对沉溺于色情瘾者的包容性越来越强。当然,读者也可以选择打电话咨询保密的全国性瘾和精神健康热线咨询师,接听热线的工作人员会推荐当地的咨询和援助资源。\\
一些色情观看者倾向于对熟人倾诉,比如伴侣、近亲、师长和挚友。如果读者有意选择这些对象,那么请回答下表中的问题,这有助于你选择合适的倾诉对象。\\
·谁知道了我的色情问题以后,还会包容我?\\
·谁值得我信赖,不会羞辱我、谴责我?\\
·我曾经向谁倾吐心事后得到了积极回应?\\
·谁不爱在人后说闲话?\\
·谁一直都比较尊重他人隐私?\\
·谁对他人的私生活抱着同情心和怜悯心?\\
·谁了解甚至精通上瘾问题,知道如何康复?\\
请选择一位阅历丰富、成熟可靠的对象来帮助你一起了解色情问题,他/她会提出睿智的建议,对你的倾诉作出积极回应。但是无论你多么谨慎小心,都不能百分之百地保证倾诉之后你会有舒畅感,这其中的风险也是所有人必须要学会承担的。\\
受访的成功戒除色情者一致赞成:自我曝光是跳出色情陷阱的重要一步。56岁的乔治已成功戒除了色情瘾,他建议说:“如果你有色情问题,那么找一位可信的对象,可以是神父、治疗师、挚友,也可以是配偶或者伴侣。坦诚地告诉他们。跟他们说,是孤独和孤僻逼你走上了色情这条不归路。”汤姆也赞成说:“你一定要勇敢地走出第一步,找一位你信任的、让你有安全感的人。跨过那道坎儿,克制恐惧心理,一五一十地说出来。只要你说出来了,再要对别人、对自己坦诚就容易多了。你再也不需要隐瞒了。”\\
对伴侣倾诉:如果你有伴侣,你迟早要对他/她说出自已的色情瘾问题。即使出于维持对方信任感的考虑,没有首先告诉伴侣,也务必要让他/她成为第一批知情人。\\
向伴侣坦白的那一刻,请不要奢望他/她的支持和理解.至少不是当下。正如第五章所述、伴侣首次知道对方有色情问题时,情绪失控是不可避免的。伴侣起初会感到震惊和愤怒,感到对方在性方面背叛了自己。不管伴侣的反应多么激烈。个人都必须坚定地完成这个步骤,绝不动摇。尽管伴侣从初会感到伤心难过,但你还是要旗帜鲜明地坚持自我曝光这个步骤。只有这样,你才能打破过去那种孤独的不健康的方式,为两人以后长远的关系发展打下坚实的基础。第十章伴侣治疗法中将进一步讨论伴侣们就如何去治疗色情带来的创伤。\\
柯克对妻子坦白后,如释重负。“我内心斗争足足6年之久,才对妻子说出了真话。因为我实在没法继续说谎了。那是我第一次告诉妻子,我沉溺于网络色情。对她坦白的感觉很不好,但这是我重新找回自信的第一步。”\\
倾诉的方式:有些人事先未加考虑,就冲动地把色情问题和盘托出,导火线往往是一些意外事件。比如说,一次出差,布莱德在宾馆房间里不眠不休地观看付费色情,根本没有要对妻子宝拉坦白的打算,但结果他却说出了口。“那次出差,我没日没夜地看色情。回家后,我根本没法跟宝拉像往常那样沟通,”他说,“我们开始打架,又停战。在那个时候,我忽然意识到色情瘾在毁灭我的婚姻。那天晚上,我的神经崩溃。像胎儿一样蜷缩在床角,身子开始颤抖,情形非常诡异。那时候,我没有来由地感到恐惧,宝拉开始担心我,想要帮忙。她问:‘出什么事了?你怎么了?’我开口了,告诉宝拉我有色情瘾,我们还讨论怎么去寻求帮助。慢慢地,颤抖消失了,之后我感到好过些了。”\\
一些色情观看者细心准备,完整规划,希望有计划性地倾诉这个问题。他们认真选择倾诉对象、计划好要说的内容、策划收场的方式,如果读者想要采取全面策划的方式,以下建议将会对你有所帮助:\\
关于倾诉色情瘾问题的建议\\
·选择一位能够帮助你顺利实现倾诉过程的精神健康专家或宗教咨询师作为倾诉对象。\\
选择合适的时机,确保倾诉对象不会受打扰。比如说,关掉手机,在私密场合交谈,这样会给你带来安全感。\\
·让他人知道你有隐私问题要讨论,不要来打扰。\\
·事先询问倾诉对象是否愿意聆听,适合你的时机未必适合他人。\\
·告诉对方,张口坦白多么不易。让对方知道,你内心挣扎了很久才决定要对他/她说出自己的问题。告诉对方,你选择他/她作为倾诉对象是出于信任,你尊重他/她的意见,认为他/她给人安全感。\\
·让对方知道,为什么你选择此时透露这个问题。你可以以说:“我现在要把这件事告诉你,因为我再也不想继续保守这个秘密了;我不想继续伤害自己,伤害别人;我已经做好了坦白的准备,我承认我有问题,而且我必须要采取措施。”\\
·让对方知道,你需要从他/她那里得到何种帮助,你只需要对方耐心聆听,还是需要对方提出建议,分享他/她的想法,还是希望得到其他形式的帮助?\\
反应各不相同:遵循了以上建议,倾诉对象可能会给予你同情和尊重。但想要完全控制或者预测他人的反应并不现实。戴夫告诉笔者:“有些倾听者确实退缩了。我觉得,这个话题不是让他们觉得恶心,就是觉得害怕。他们拒绝继续讨论这个问题,。不过是因为反感这个话题。只要想通这一点,他们的反应也就不会让我困扰了,这让我更加珍惜那些可以理解和支持我的人。”\\
很多成功戒除色情者称,他们选择的倾诉对象能给于他们包容和理解。凯文和两位老友去钓鱼旅行时,决定向朋友说出自己的问题。“旅行的时候,我有足够的时间来思考色情对妻子和家庭的影响,”他说。“我下了决心要戒除色情,也知道有别人的帮助才会成功。一天吃晚饭的时候,我把这件事告诉了我的俩哥们儿。真的太好了,他们完全支持我,其中一个说,他也有这个问题;我这才意识到,凡是人都会有自己的强迫、恐惧心理以及秘密,不管是什么错误,都可能会有人犯。我发现,如果你大方地说出自己的问题,别人就会同情你,也会反过来诚恳对待你。”\\
艾利克斯初次倾诉色情问题时,得到了教堂一群伙伴的支持,让他大为感动。“我向他们倾诉色情问题的时候,在场的每一个人都哭了。”他说,“没有人厌恶我,没有人说我有罪。能对他们坦白,我大大地松了一口气。他们设身处地为我着想,我的生命对他们来说也很重要,这证明,我不是一个人在战斗。”\\
笔者期望读者倾诉后会得到一个积极的结果。但不论你初次倾诉后得到的是支持和同情,还是愤怒和拒绝,这都是你必须承受的孤独和痛苦。\\
第二步:参加治疗项目\\
如果你下决心戒除色情瘾,那么参加治疗项目是必不可少的。没有治疗项目的指点与引导,个人很容易失去对目标的洞察力,继而被压力击垮,复而回到老路上,恢复以前的思维和行为方式,这对治愈色情瘾毫无益处。\\
参加咨询会或者团体会议的想法起初会让人觉得恐怖,动力全无。要迈出参加治疗这一步所需的勇气和毅力,远甚于对小部分人坦露的情形。要搜索、定位提供治愈色情瘾服务的咨询师和组织,需要个人投入足够的时间精力;为了参加咨询或者团体会议而调整日程,也确实会给生活带来诸多不便;参加治疗项目可能也会产生一定的费用;此外,这还需要个人做好尝试不同选择的准备,因为不去探索就不知道什么适合自己、什么可以满足自己;当然,个人还要学会信任素不相识的个人或集体。\\
参加治疗项口是一个非常有效的步骤。大部分受访者坦言,治疗项目给他们的生活带来了许多积极因素。通过参加个人或者团体咨询,他们习得了有效的戒瘾方法,确立了行为楷模,对自我的行为也有了更深刻的了解,从而激发了自己的新想法,在戒除色情瘾的过程中能够合理评估自己的进退。参加治疗项目也能帮助人调整心态,因为它给了个人结识同道中人的机会。与他人的沟通会带来感情支持,减少内心的罪恶心理,激励色情观看者持之以恒。\\
科里表示,他在男性性瘾康复项目中得到的支持和鼓励,最终帮助他成功戒除了色情瘾。“色情给我带来这么多性满足感,要彻底戒除还是很不容易的。我需要时时有人在耳边提醒我再看下去是不对的,这样我才能保持动力,这也是我成功的关键。只有这样,我才不会一意孤行,硬要争辩色情对我有好处。”\\
针对色情瘾人群的各种治疗项目在方式、形式和收费等方而各不相同。有些人愿意和咨询师、健康治疗师或者其他专家进行一对一的交流,而另一些人更喜欢参加团体会议,和其他想要戒除色情瘾或者克服其他性瘾问题的人群一起努力。要找到最适合的项目,读者要做足调查来发掘自己能够支配的资源,确定何种治疗项目符合自己的价值观和人生信念。你可能要花费一番心思才能找到满足自己需求,符合自己人格,同时不影响正常生活的治疗方式。\\
常见的色情瘾治疗项目包括:\\
·个人咨询\\
·伴侣或夫妻咨询\\
·咨询师指导的团体咨询\\
·十二步成瘾治疗法\\
·信仰治疗法\\
·住院治疗法\\
·特殊项目及研讨会\\
要找到适合的治疗项目,最有效的办法就是和精神健康咨询师、宗教领袖或者戒瘾专家进行面对面的交流。他/她精通领域内的各种服务、你可以自在地和他们谈沦自己的处境。有了他们的帮助和指导,你可以制定符合个人情况、满足个人需求、符合个人目标的治疗方案。\\
艾德和一名精神健康咨询师交流之后,制订了以下治疗方案:\\
1.我每周都会去性瘾治疗专家处接受治疗,为期至少3个月;\\
2.我每周至少参加一次匿名性瘾者聚会,为期至少3个月;\\
3.3个月后,我会自我评估这3个月内的表现,完善计划,设计下一步的治疗步骤。\\
艾德选择个人咨询和团体康复项目相结合的治疗方式非常有效。各种方式相辅相成,可以帮助个人维持长期稳定的进步。不过,读者要了解,最初采用的计划需要随着康复阶段的改变而不断完善。\\
接下来介绍各种戒除色情瘾途径的特点和优势。\\
私人咨询:可咨询对象包括专业受训的咨询师、临床义工、精神医师以及其他持有资格证书的精神健康师,那些专门在性瘾领域深造的行家是最合适的,他们推荐的戒瘾疗程一般是每周一次或每月两次,持续时间从几个月到几年不等。治疗师可以为你提供指导和帮助,以修复诱导色情问题的潜在情感伤痕,克服其他心理障碍。长期和某人单独交流可以促进人际交往的能力,让个人更加坦然地敞开心扉。\\
比尔起初也犹豫要不要进行个人咨询,后来他才万分庆幸自己当初的选择。在所有项目中,个人咨询对他来说是最保险的,这还有效地控制了他对儿童色情的迷恋。“如果你和我一样会被儿童色情引起性欲的话,那你最大的问题就是孤独,”他说,“我担心自己要是坦白的话,别人会骂我是‘变态’、‘禽兽’。和咨询师的交流是我有生以来做过最正确的决定,他让我突破了沉默。咨询师公正客观地帮我分析问题,还教给我一些策略技巧,让我转变行为,用更加积极的办法来调整自己的性兴趣。”\\
伴侣咨询法:伴侣关系岌岌可危时,伴侣咨询非常有效,因为双方有了能够在理性公正的环境中讨论色情问题的机会。这种康复工作可以帮助两人了解,色情是如何破坏了他们的关系,损害了双方的信任感和亲密感。在专业婚姻和伴侣咨询师的引导下,色情瘾者可以和伴侣齐心协力,共同修复色情给这段关系所带来的伤害。\\
笔者鼓励处在爱恋关系中的色情观看者尝试伴侣咨询法,但建议将其作为个人治疗的辅助或者后续,才能发挥出最大的效果。在个人治疗阶段早期,伴侣咨询可以稳定两我关系,因为双方都会试图缓解过激的感情反应。伴侣咨询还会让个人更好地了解到,色情损害了两人之间的关系,深刻了解这一形势会更好地鼓励个人。但是在没有解决色情瘾问题之前,不要奢望解决两人之间的矛盾。通常来说,在色情观看者作出戒瘾承诺,而且已经成功远离色情一段时日的情况下,伴侣咨询法是最有效的。\\
团体项目:团体咨询和其他形式的集体项目,如十二步团体治疗法是色情康复过程中不可或缺的部分。毕竟,康复需要长期的付出,如果能定期和其他同道中人交流沟通,个人也能更好地维护自己的康复目标。 团体康复项目包括咨询师引导的团体咨询项目、以宗教信仰为基础的团体咨询项目和十一步康复项目,如嗜性匿名互诫协会(Sexaholics Anonymous)和性沉溺者匿名协会Sex Addicts Anonymous)。美国这种正规团体康复项目收费较为合理,有些甚至是免费的,项目规模从五六人到二十多人不等。许多团体采用固定的模式来分享成员经历,每个礼拜都遵循固定的日程安排,集中讨论康复中的各个问题。这种方法通常能让每位成员都感到轻松自在,畅所欲言地讨论自己的色情问题,并且相互学习,共同进步。\\
参加团体项目的色情观看者普遍认为这种方式效果显著,团体不仅让他们有了集体归属感,也会给他们提供必要的支持,并且,团体活动强调个人责任感和诚实的重要性。如果你知道,别人时不时都会关心一下你的康复进度,你就更能把持住自己,远离色情了。“知道别人跟我一样苦苦挣扎,让我轻松不少,”伊万说,“我已经下决心要戒除色情瘾了,但很多次我还是想看色情。团体的其他成员时不时会检查我的康复进度。他们会问:‘情况怎么样了?’我知道他们任何时候都有可能会问我这个问题,这个挂念帮助了我抵制色情。”\\
参加团体治疗还有一个好处,经常听闻其他色情瘾者的经历,会督促个人坦陈自己的感受和行为。如果个人听闻色情给同组成员带来的打击,,就会对自己的问题有不一样的看法。布莱德今年27岁,四年前,他曾参加教堂男性色情瘾康复团体治疗项目。“我能戒除色情全靠这个项目。”他i兑,“开了几次团体会议后,色情对我就失去了那种诱惑力。我没有想要看色情的欲望了,不会去幻想色情。也不会对着色情自慰了,别人也有类似的感觉。远离色情瘾的感觉很好,我能够更加了解自己的行为,明白色情如何影响到我的生活,要拉我下水。我开始清楚,我在抗争什么,如何掌控局势。”\\
此外,团体治疗项目还可以帮助个人发展人际交往能力,而这往往是色情观看者所欠缺的。贾斯汀对笔者说:“色情瘾让我没法成熟起来,我不信任女人。在性沉溺者匿名协会中,我们花了很多精力来分析自己的情绪。团体成员教我怎样维持感情,怎样真心地去体会别人的感受。我能够更加亲近他人、与他人交流的时候,强迫性自慰的需求就消失了。这样看来,对着色情自慰不过是最次的替代品而已。”\\
知道团队里的每一个成员都面对着同样的问题,可以帮助个人克服羞耻感。迪克,34岁,他说:“我们的男性团队否定了那种是爷们就不能承认弱点、不承认自己也会受伤的大男子主义。新参加团队的男生总是想要美化色情,队里的其他人就会说:‘得了吧,我们之前都跟你一样,但是所有人都搞砸了,别犯傻了。’’我们能够理性地发掘自己行为的本质,但不会在道德上俯视别人,对他们指指点点。大家成了铁哥们儿,相互信任,相互尊重隐私。这种战友式的友谊跟军队一样,就好像我们一起经历过生死,一起承受过、伤痛过,就是这种友情让我们团结在一起。我们相互帮助,一起提高社交能力,一起学习如何更加尊重另一半。多亏了这群兄弟,我的自我感觉好了很多。”\\
如果你不住在城市,要找到团体治疗的项目或许有难度。玛丽要开三小时的车,才能找到嗜性匿名者互诫会。她告诉笔者,长途开车去参加聚会是值得的,因为团体治疗的好处实在太多了。“我提前一天晚上到,第二天一早去见治疗师,”她说,“有时候我是团队里唯一个女性。之前我一直觉得有点别扭,不过现在我习惯了,就觉得自己不过是其中一员。跟别的地方相比,我在那里更有安全感。虽然性别不同,但我和队员们都专注于自己的问题。大家聚在一块儿,就是为了同一个目的:远离色情,改善生命,相互帮助。”\\
住院疗法:住院治疗项目对无法控制性行为、承受巨大情感压力的色情瘾者最为有效。这类项目大多设立于校园式的医院机构中,提供全方位的服务,包括治疗需求评估、个人咨询、团体咨询和教育信息,还有出院后的后续治疗项口。尽管和其他治疗项目相比,这种项目费用较高,但它提供的贵宾式、个人化的照顾方式能够有效地帮助患者,让他们远离色情,建立新的生活习惯,会有专业人员观察其临床表现,让患者得到充分休息,治愈情感和肉体上的伤痛。\\
特殊治疗项目:能供色情瘾者选择的还有其他特殊项目和服务, 包括短期强化门诊项目和研讨会,成员可以每天参加,周期从几天到几周不等。这些项目比住院项目要价低,通常在类似教室的场所举行,参加项目的成员也包括接受其他治疗的色情瘾者和性瘾者。劳拉参加了一个Bethesda研讨会,一个专门针对美国田纳西州纳什维尔地区的女性基督教信徒的性瘾治疗项目,踏上了戒瘾之路。之前我一直认为,在性瘾问题中,自己是女性版的‘独孤求败’,”劳拉说,“ 直到参加了那个研讨会我才意识到,很多女性也有同样的问题。”\\
只有你自己才能确定何种治疗项目最为有效,找到一个合适的项目,来帮助你保持内心的诚恳,坚守承诺,这样的精神支撑能引导你在康复的道路上不懈地走下去。\\
第三步:创造一个零色情的环境\\
当然了,把色情清扫出自己的环境也是康复中关键的一环。想要戒除色情瘾,你就需要把色情踢出自己的房间、办公室和其他自己习惯性观看色情的地点。就好像戒烟者一定要把所有的烟全都处理掉,戒i酒者要把酒全都倒掉一样,色情瘾康复者也要远离色情。如果身处零色情的环境里,想要再毫无顾忌地观看色情也就没有那么容易了,需要突破物质和精神的双重限制才行。总之,零色情的环境会增强个人追求健康生活的积极性。\\
与香烟和酒精不同的是,色情不是一种在商店柜台上坐等顾客的商品;任何时候,你想要接触色情都非常容易,色情价格低廉,形式多样,传播途径丰富。仅仅把色情清扫出门,远离出售色情品的地点还不足以成功远离色情。要完全脱离色情需要人坚定不移的努力。一旦发现色情卷土重来,个人一定要坚守阵地。\\
创建色情环境的方式可简单归类为:扫出门,关上门,不接触。下文将详细介绍每种方式。\\
扫出门:清除掉放在家里、工作地点和车上的色情品,在生活中可能接触到的色情品全都要彻底清除。销毁所有色情杂志、书籍、录像、DVD和电脑中的文件夹,取消电视色情频道。把这看成是除害的努力,色情毒害你的人生,唯一一个能够改善个人健康、保持环境清静的方法就是把所有的色情品彻底清除掉。\\
把所有的东西都扔掉其实很简单,但清除色清的过程也许会让人感到压抑,压力倍增。在这个过程中,个人普遍会产生沮丧、愤怒等负面情绪,类似于结束一段曾带来愉悦性体验的感情。一些色情观看者花费数年时间收集色情品,收藏的都是专门满足个人的需求、符合个人品位的内容,他们和这些收藏品难割难舍;此外,还有一些人意志不够坚定,认为“康复”没成的话,色情又可以成为自己的安全港湾了。只有清涂色情,才会让这种努力变得真实,因为这意味着实实在在与色情决裂。\\
色情瘾痊愈者将这个过程比作“和老友绝交”,觉得“很不好”,“是我做过最艰难的决定”。这种情感反应的强烈度体现了个人对色情的入迷程度和依赖性。和长期观看色情者相比,清扫色情品对偶尔观看之的人来说会轻松些。如果清除色情品带来的情感压力太大,可以向专家和康复小组的成员寻求帮助,找到最好的处理方式,克服负面情绪。许多人说,适应了无色情的环境之后,他们如释重负,感到自豪。如果想再看色情,就需要内心的渴望再加上具体的行动才可以实现,这会让人安心不少。“色情不在你面前晃悠的时候,要远离色情也就变得比较容易了。”玛丽如是说。\\
关上门:将生活环境中的色情都清扫出去之后,接下来的任务就是“关上门”了。色情和类色情信息随时都可能在网上弹出,在电视节目中出现,在电影镜头中闪现,在印刷品中现,色情时刻潜伏在日常生活中。广告和娱乐业都要把暴露图片来当噱头,很多企业不惜下血本来制作“十八禁”产品,最大限度地扩张产品的潜在市场,如付费电视频道、色情杂志、电视卫星、有线电视和其他一箩筐高科技电子产品。许多网站的内容其实与色情无关,却也会用那些性感的弹出图片来吸引点击率。\\
想要和色情彻底决裂需要个人的坚定毅力。笔者发现,想要永久地将色情挡在门外,最佳对策就是制造障碍。请记住,色情就是因为匿名、免费和易得等特征才会泛滥,如果你能够抵抗这些因素,就能将色情永远地关在门外。请仔细阅读以下清单,判断哪些选项有助于创造和维护无色情的环境:\\
减少个人接触色情的途径\\
_取消订制的色情品(包括网页、杂志、无线电视和手机服务)。\\
_选择以家庭服务为导向的宽带供应商。\\
_更改自己的邮箱地址。\\
_在电脑上安装网络监管软件。\\
_将电脑移到家中客厅等公共区域。\\
_取消电脑宽带服务。\\
_购买新的硬盘驱动器或电脑。\\
_有旁人在场,且其能看到屏幕时才使用电脑。\\
_订购百分百无色情的电视节目套餐。\\
_处理掉自己所有的电视机、录像机和DVD播放器。\\
_屏蔽带有色情的电视频道。\\
_避免出入成人书店或脱衣舞夜总会等场所。\\
_避免出入销售色情杂志和其他色情品的商店。\\
_避免出入出租成人录像的商店。\\
_回避那些常和你一起看色情的人或者是能让你看到色情的人。\\
_入住旅店前先电话咨询,取消你入住期间的电视色情节目,确保客房电视不带色情内容。\\
_告诉朋友、亲戚和同事,不要再给你色情品或是色情桩接。\\
以上这些办法可以帮助个人更顺利地戒除色情瘾。“色情就像是毒品”维斯说。“你越是允许它入侵你的生活,就会越依赖于它。看了色情之后的几天,那些画面常常会出现在我的脑中,逼着我去看更多的色情。10年前我戒了烟,这段经历告诉我如果想要戒除一种瘾,远离这种瘾的时间越久,欲望就越淡薄,接下来你就等着时间发挥作用吧。把生命中所有的色情都扫出门,就好像掐灭了我的欲望。\\
汤姆发现,只要改变一个老习惯,就可以远离色情。“我在一家大型药店上班,店里有一个很大的杂志取阅处,放的杂志在我看来都是色情,像Maxim、《男人装》和《时尚》这种封面上都是半裸女人的杂志。为了戒除色情瘾,我确保自己远离那个书架。同事有时候去那里,随手拿起一本杂志就开始看。我要是看到心里也会痒痒,所以我就尽量远离那个区域,同事看的时候我也不去凑热闹;我也不看电视了,因为电视中总是有和色情相关的画面和荤段子。尽我所能,把日常生活中的色情全都扫出门去,这样就不会刺激我想要看色情了。”\\
不接触:清注意,不管你多么努力地想要把色情清扫出自己的生活环境,你始终都会接触到色情。开车的时候,你可能会看到路边一幅性感的广告牌;电视音乐频道中的歌手和舞者扭动身体,让人想起AV女的举动;打开周末版的报纸,其中夹杂的一则女性内衣广告会让人联想到脱衣舞。那你又该如何应对呢?\\
既然不可能完全躲避那些刺激性的内容,最好的对策就是要以最快的速度远离它。事物的存在并不意味着我们一定要将其记在脑中。可能你没有意识到,我们每天每分每秒都在做着选择:选择注意或者忽视环境中成千上万件不同的事物。个人可以把性感的图片归到“忽略”名单上,不去想它;也可以简单地闭上眼睛,拿遥控器换台,关掉电子设备,或是站起来挪个地儿。请记住,你总是有充分的掌控权,在视觉上忽视色情的存在,在情感上对色情无动于衷。\\
汉克意外看到能让人想起色情的画面和场景时,就会采取一种“眼球快闪”的方式。“我会不由自主地想到色情,我之前就经常那样,大脑也已经习惯了,”他说,“所以现在,我要求自己一看到性感内容,就马上闪开。哦,那里有个性感女孩的照片,快闪!转移视线。看电视的时候,我的眼睛总要不停地快闪,因为很多广告镜头都会特写女性身材,性感火辣。我就只听广告,不用眼睛看。刚开始会不习惯,但现在‘快闪’已经成了我和新习惯。我越是这么做,感觉就越是自然。过去我总是带着一股子欲火生活着,因为我现在长期的控制,欲火开始减少,有时甚至完全感觉不到了,这就好像我的身体自由了,我可以去做自己想要的事情,这种感觉很好。”\\
将环境中的色情清除出去,意外遇到色情时采取积极的措施,这有助于个人远离色情。色情离你的生活越远,你就越能过上更健康、更符合个人价值观和人生目标的生活。\\
第四步:建立24 小时支持和责任机制\\
个人不仅要未雨绸缪,应对随时可能出现的色情内容,还要学会遏制内心的欲望。深呼吸,心中默数到10,这种方法适用于某些人,但是对大部分色情观看者来说,还是行不通。仅凭自己的意志力来控制冲动有一定的风险,当你内心挣扎,不知如何控制欲望时,如果有一个随时候命的亲友团就能帮上大忙。\\
麦切因为在办公场所看色情而丢了老师和教练的工作.为了重新开始,他改过自新,决定要戒除色情瘾。“我是不是心里还在惦记?我是不是要跟一个老习惯说再见?这是肯定的。是教堂男性性瘾康复小组帮助我走上了正轨。”麦切在这个过程中发现,在戒瘾道路上走得越久,他就越需要更多的支持来克服内心的欲望。“我和牧师说好了,我任何时候都可以给他打电话。我会告诉他,‘现在我特别纠结。你能不能为我祈祷?你能不能建议我去做点别的事情?’牧师会耐心地听着、安慰我,给我一些建议。他的支持陪伴我走过了这段最艰难的日子,帮助我戒除了色情瘾。”\\
读者可以考虑将以下人员当作亲友团成员:专业咨询师,神职人员,十二步戒除性瘾提倡者,康复小组成员(也称为“责任伙伴”),伴侣,家庭成员,朋友或者是地方、全国性的热线咨询师。\\
为了避免某位亲友团成员压力过大或者没有空闲之类的情形,能培养多位可以电话沟通的倾诉对象最好不过。汤姆告诉笔者,他的母亲和姐妹是他康复过程中不可或缺的支柱。“她们一直坚守在我身边,听我说我的问题。她们知道因为爸爸的影响,我从小就生活在色情的阴影里。她们明白这对我的生活有多大的影响,也知道我在戒除瘾的这段时间里,内心经受了怎样的煎熬。”\\
汤姆的亲友团并不只限于母亲和姐妹。他说:“我也和姐夫说这件事,他是一个强有力的支持。十二步治疗项目中认识的朋友,还有教堂的教友们,都知道我过去的问题,了解我现在的煎熬。有他们在,真的很好。我把这件事告诉了很多人,之前我一直害怕把这件事说出来,觉得别人要是知道了就会把我当作变态。但事实证明,我错了。朋友们真心在乎我,也支持我的康复进程。”\\
伴侣也可以成为有力的支持者,但是他们能发挥的作用因人而异。许多伴侣知道了对方的色情欲望之强、观看的内容之龌龊、戒除色情瘾过程中色情瘾之反复时,往往觉得难以接受。他们很容易恢复之前的惯性思维,想要控制对方的行为,妄加评沦,并且监督对方的一举一动,这种反应在色情瘾康复初期尤其明显。不把伴侣当成是唯一的支持者,并且伴侣也充分信任你的情况下,伴侣的支持会非常有效。\\
埃里克斯觉得在戒瘾的过程中,妻子艾丽莎给予了他莫大的支持。他22岁结婚,婚前几年他就意识到自己有色情瘾问题并且停止观看色情了。艾丽莎从未觉得被背叛了,也没有因此吃醋。“结婚前,我就告诉艾丽莎我有色情瘾。色情瘾很难戒掉,当时我就预感色情瘾随时会发作。后来,每当我特别想要看色情的时候,就会告诉妻子。她会拥抱我,安慰我,让我好受些。艾丽莎不会羞辱我,也不会指责我。知道她随时陪在我身边,让我很安心。”\\
亚当是性沉溺者匿名协会的提倡者,他有多年和色情瘾者打交道的经验,一些色情瘾者无法遏制欲望时,常常会给他电话求助。亚当说:“我建议大家量力而行,要弄清楚自己的感觉和需求。我一直鼓励人们去了解自己内心的需求。内心深处的情感,很难用言语来形容,这也常常是人痛苦的来源。要康复,就需要个人增强自我意识,清楚自己选择的范围,而且有需要的时候还有人陪伴在你身边。”\\
第五步:爱护你的身体和精神健康\\
戒除色情瘾会带来身体和精神的双重压力。陷入色情陷阱的人停止观看色情之后,会出现脱瘾症状,类似于毒瘾者停止使用可卡因或其他烈性毒品后产生的症状。像戒毒一样,色情瘾康复者也需要足够的时间,让因观看色情而改变的大脑化学物质恢复正常。在这个调整精神状态的过程中,个人可能会出现焦虑、失眠等症状。\\
要应对色情瘾给人带来的精神、身体健康问题,最好的办法就是循序渐进,保持个人健康。拥有健康的体魄、坚定的信念和强大精神力量的人群不仅能较快地恢复,也会有更多的精力来培养替代色情的瘾的生活习惯,以此改善人际关系,重获新生。\\
自我爱护的基本理念对大部分人来说都是老生常谈,但如果你已经陷入色情有些时日,那么可能你长久以来一直忽视了自己的健康。笔者接下来要介绍的一些信息或许可以让读者受益。笔者的建议包括:\\
·去看医生或者健康师评估个人当前的健康状况。咨询对抗压力、保持健康的方法。\\
·如果还没有每天锻炼身体的习惯,那么现在是时候了。培养锻炼的习惯,增强自己的力童、耐力和灵活度(如举重、跑步、伸展等运动)。保持良好的身材会帮助个人改善精神状态,增加抗压性。无论是慢跑、游泳还是拳击,任何形式的运动都可以帮助抵抗戒瘾过程中的负面情绪,比如焦虑、愤怒、忧虑等。\\
·找一位运动同伴,监督自已的运动状况:找支持你戒瘾的运动迷一起运动,这样一箭双雕。想要懈怠时,同伴会激励你·\\
·每天保证七小时的高质量睡眠。良好的睡眠会加强人体健康,有助个人作出理性的决定。戒除色情瘾的过程中,个人需要充足的睡眠来抵抗精神压力。\\
·评估自己的饮食习惯,作出相应的改进。可以咨询营养师,了解自已应该怎样改善饮食习惯。\\
·选择几种抗压方式,定期实践以舒缓紧张感。可供选择的抗压方式,小到听歌、遛狗,大到有体系的方式,如练习瑜伽、冥想。\\
·评估并且解决任何可能影响康复的潜在问题,比如说毒品或酒精问题、临床抑郁、强迫性神经失调、荷尔蒙失调等问题。\\
戒除色情瘾的同时,也可以改善个人健康,何乐而不为。托尼说:“我几乎每天晚上都在健身房锻炼,不吃垃圾食品,营养均衡,每天保证7小时的睡眠。我现在的身体状况和精神状况前所未有的好,感觉很不错,根本想不到看色情。”\\
保持自己的健康状态,也包括应对自我情绪变化,产生的问题时及时采取应对措施。当心情落到低谷,比如感到无聊、孤独、焦虑,觉得受伤、愤怒或者压力过大时,想要看色情的欲望会无限扩大。能否戒除色情,很大程度上取决于自己能否透彻地了解内心的感受,作出合适的调整。\\
比尔注意到,工作午休时间他总想要看色情,考虑过后,他认为罪魁祸首就是“无聊”二字,这为他的行动指明了方向。“我就开始在午休时间找点有意思的事做,”他说,“我去玩游戏。尽管这样子也没有完全打消我想要看色情的念头,但还是让我舒服了很多,我能做点更有意思的事。有时候,我也会给妻子打电话,和同事闲聊,出去散步或者骑车出去玩。”\\
如果读者能够积极地参加一些有利于身心健康的活动,要远离色情就容易多了。你喜欢的健康活动,不管是运动、园艺、乐器、登山还是猜字游戏,都可以增强自信心,给人带来正能量,以对抗戒除色情瘾带来的压力。\\
玛丽告诉笔者:“过去,我从来不会善待自己。现在,我上瑜伽课,每天都会练习。每个月我都会做治疗按摩。如果我困了,就去休息一会儿。过去我从来没有这样爱过自己。我一直以为自己没时间来做这些事,而我现在学到的一个道理就是,自己的健康是康复过程中的重中之重。”\\
看起来,戒除色情瘾的同时又要培养健康的生活习惯,这好像让人难以负荷。但是你很快就会发现,更好地照顾自己是值得的。这不仅会让你的戒瘾之路更为顺l畅,而且保证色情瘾不会再次复发。\\
第六步:开始治疗自己的性欲\\
六步基本步骤的最后一步,就是要了解并克服观看色情给人的性观念和性行为造成的影响。正如第二章所述,个人首次接触色情的情形往往可以追溯到幼年时期,因此,很多色情观看者并不是从正规的性教育资源和真实生活中学习“性”,而是以色情为启蒙老师。因此,个人可能已经习惯了色情的性模式,但这并不适用于现实生活。\\
如果个人沉溺于色情,这很可能意味着个人幻想中的性爱和采取性行为的方式都已受到色情的影响了。悲剧的是,色情是自私的,它鼓励人长期沉溺其中,而色情传达的信息并不利于培养健康的性理念,无助于个人享受亲密的性爱。即使个人已经把色情全部清扫了出去,也很难让生活走上正轨,因为那些色情信息,哪怕只是残留在脑中的影像,仍可能影响个人的性生活。\\
性态度和性举动并不会因为个人停止观看色情就消失。个人需要重新定义性,改变对性的偏颇看法,学习新方法,和伴侣一起享受愉悦的性生活。时常反省,及时调整自己的性态度和性举动,这非常有必要。如果有幸福美满的性生活来替代色情,那么要远离色情就是水到渠成的事了。\\
要培养正确的性态度和性理念有很多方式,如阅读科学性教育的报纸杂志,参加社区的性教育课程,咨询性瘾康复咨询师、持证性治疗师和性教育者等。读者也可以和挚友,尤其是那些具有积极的性价值观和行为举止的对象,讨论什么是健康的性;当然你也可以和信仰小组或精神小组的带头人讨论,他们通常都会有比较正面的性观念。\\
同时,了解培养健康的性体验和情感关系的前提条件,也会有所帮助。读者可以浏览笔者的官方网站www.HealthySex.com,阅读题为《马尔茨性互动层次》(Meltz Hierarchy of Sexual Interaction)的文章,了解此方面的信息。能够区分色情驱使的性欲和健康性欲之间的差异,有助于个人更快更彻底地摆脱色情瘾。\\
下表通过对比,区分色情滋养的性态度与性举动和健康性观念之间的主要差别。思考哪一项会引起你的深思。\\
你知道它们的差别吗?\\
色情中的性:\\
性是利用别人\\
性是为了性爱\\
性是给别人欣赏的秀\\
性是强迫性的\\
性是公共商品\\
性是观看别人的性行为\\
性和爱是分离的\\
性可能是带伤害性的\\
性与情是疏离的\\
性可以在任何时间发生\\
性是不安全的\\
性侮辱人\\
性可以是不负责任的\\
性是违背道德的\\
性缺乏健康的交流\\
性包含了欺骗\\
性基于虚幻\\
性没有伦理约束\\
性分裂人格\\
性违背个人的价值观\\
性是羞耻的\\
性是即兴满足;\\
健康的性:\\
性是关爱别人\\
性是和伴侣分享\\
性是私人体验\\
性是自然欲望驱使的\\
性是个人财富\\
性是真挚的结合\\
性是爱的表现\\
性是滋养的\\
性与情相结合\\
性需要一些特定条件\\
性是安全的\\
性尊重人\\
性要负责任地对待\\
性符合道德观和价值观\\
性需要健康的交流\\
性需要诚恳\\
性包容所有感官\\
性有伦理约束\\
性修养人格\\
性反映个人的价值观\\
性提升个人自信心\\
性是持久满足\\
治疗色情瘾的这个最后步骤不仅需要个人培养更健康的性观念,还要个人停止任何带色情含义的性举动。处在康复期的色情瘾常常会犹豫:我要不要停止手淫?我要不要停止性生活?他们担心特定的性行为会诱发以往受色情影响的性方式,将他们重新拉回色情瘾的深渊之中;色情瘾康复者往往还想要知道如何在合适的时间用积极的方式表达正常性欲和性冲动。很多人没有坐等命运的安排,而是很明智地认识到,能否康复取决于他们是否能了解并应对自己的性感觉和性冲动。\\
性的表达方式多种多样。因人而异,适用于他人的方式未必普遍适用。鉴于色情具有强大的诱惑力,还能诱发强迫性性行为,所以笔者建议求助于专业咨询师,寻找最适合的方式。专业咨询师会结合每个人不同的个性、色情问题的严重程度、个人的生活方式、宗教信仰以及精神信仰和治疗史进行综合考量,提供安全高效的策略来帮助个人培养健康的性观念。\\
如果性欲已经到了与色情密不可分的地步,个人可以暂时全面停止任何形式的性活动,包括特定的性表达方式,这种方式尤其适用于有性强迫行为和/或对高风险、违法和危险性活动感兴趣的群体。许多十二步治疗法、信仰治疗法和性瘾治疗项目都力荐个人在康复初期禁欲,术语称之为“给性放个假”。有必要的话,停止性活动,为期几周到几个月或者更久,这种方式具有立竿见影的效果。知道在特定时间内,任何形式的性都是出格行为,这就要求个人主动严格控制自己的行为。“没有性的3个月是我的救命稻草,”艾德告诉笔者,“就是因为找知道自己不能做那样的事、在那期间我没有受到色情念头的骚扰。”\\
“给性放个假”可以紧急停止以色情为中心、被色情所影响的性表达方式,这样,个人就有机会也解,没有了色情驱使,地球照样转;个人的性理念也会突破以往色情所带来的局限。在禁欲的这段时间内,你可以学习新的爱抚和性生活方式,远离色情的影响。\\
一些色情瘾痊愈者在戒瘾的过程中还能享受性爱,而不影响康复进程。究其原因,这些人已经培养了超愉悦的性行为模式,这一切与色情无关。他们的性趣并不是强迫性的,也无关色情的性幻想。举个例子,德里克是一名35岁的公车司机,几年前婚姻亮起红灯期间,他患上了色情瘾,此后色情瘾会间歇地发作。“找自慰、过性生活的时候,会性幻想我妻子,”他说,“如果在修复夫妻感情的时间里要禁止性生活,那就有点二了,没有这个必要。”\\
要彻底消除色情的性理念,培养健康的性方式,需要个人付出大量的精力。这种以性欲为中心的治疗方式会帮助个人走出色情瘾康复的第一步。确定方向,避免未解决的性问题来阻碍康复进程。第十一章将会介绍如何“以全新的方式对待性”,从色情瘾中彻底解放出来,需要个人培养全新的性习惯,减少对色情的兴趣,尝试挖掘体验性快感的潜能。如果读者有足够的耐心,希望将想法转换为实践,就需要培养新的性欲表达方式,提升个人自尊,增强和伴侣之间的亲密度。\\
本章所述的六步基本治疗方式建起了一座结实的桥梁,让读者有机会逃离色情陷阱。个人实施这些步骤时,需要避免常见的康复障碍,培养全新的态度和举止,为成功而努力改变。但是,请注意,想要彻底康复,仅仅遵循这六步还不够。接下来的章节会继续介绍,如何应对康复中的性瘾复发,如何解决色情引发的潜在问题,如何修复一段破裂的关系,.如何培养新的性技巧和亲昵技巧等。\\
第九章 应对以及预防色情瘾复发\\
旧习惯和我不过一天一时之隔,——艾德\\
德鲁,35岁,身为人父,在他自以为色情瘾痊愈后许久,色情瘾的恶化突如其来,他又开始重新看色情,这让他很是懊悔,同时也惊异不已。“我康复之后的3年内,都没有想要看色情,”他说,“有天晚上,妻子有事出了镇、我就开始在网络上看色情;之后我几乎每天晚上都要对着色情自慰。现在看色情比以前方便多了,现在的网速快得很,而且我的电脑也没有屏蔽色情的软件,我完全陷进去了。”\\
科里康复后第一年内,色情瘾渐渐复发,这让他心烦意乱。“我上网的时候不知不觉就去看那些内衣广告,”他说,“看了几个之后,我想:没事的,不过是内衣而已。我自己骗自己,不肯承认被挑起了性欲。接着我就去浏览泳装网站,心里想,没事的,不过是泳装而已。性欲克制不住的时候,我就想,对着画面打手枪应该没什么问题,这根本算不上色情嘛。这样五六天一过,我就慢慢回到了看色情的老路。”\\
德鲁和科里在色情瘾复发时都非常震惊,对自己失望不己。即使他们和其他色情瘾者交流时会了解到复发是正常现象,但他们还是不希望发生在自己身上。他们会觉得羞愧,也担心这个复发的含义。这只是一时疏忽吗?自己是不是又回到看色情的老路上了?幸运的是,他们都有一个强大的亲友团随时听命,帮助他们顺利地在康复道路上前行。他们听取了咨询师和色情康康复小组其他成员的意见和建议,来评估复发如何发生、为何发生,并采取对策,加强康复措施。\\
本章将会详细讨论复发的症状:它的定义、发生原因以及应对的办法。笔者希望提供相关信息和工具,让读者武装自己,以免重新陷入色情陷阱。笔者将会帮助读者判断刺激色情欲望的因素。寻求减少复发风险的途径,让人及时远离色情,在第一时间避免重陷深渊的策略。\\
笔者将会为读者提供具有建设性意义的对策,以防色情瘾复发。事实上,戒除色情瘾并不是直线进步的过程;这个过程常常会包含一系列的进步和挫折。牢记这一点就可以帮助你带着自怜和宽容的心态来审视复发的事实,这有利于实现康复目标。如果你善待自己,充分发挥创造力,那你就能找到合理的方式来戒除色情瘾。你只要掌控了主动权,在复发时就能及时应对,重回正轨。\\
要成功戒除色情瘾,你就必须了解如何应对复发,并最终学会如何克服复发。只要端正了态度,选对了方法,你就有能力将潜在的复发,转化为加强自身正直品行的正能量,帮助自己远离色情,追求美好生活。\\
什么是色情瘾复发?\\
笔者将色情瘾复发定义为:恢复之前观看色情的行为。个人在认识到由色情发的问题后,作出戒瘾承诺,但之后又重新观看色情,就称为色情瘾复发。\\
复发的表现形式多样,小到偶尔的疏忽,如拿起色情杂志翻看了几分钟,大到彻底的倒退,如为了释放性欲,重新定期寻找和观看色情。对大部分人来说,复发的时间越短,频率越低,涉及性的程度越低,复发的威胁就越小。如果个人能迅速应对,就能将小小的疏忽转化成警示灯,提高对自我的关注度,进一步弥补生活方式中的薄弱环节。\\
如果复发时间持续了几周或者几个月,色情又能够激发个人性欲、促进高潮,个人基本上.恢复了以往定期观看色情的行为,则是比较棘手的。个人在远离色情一段时间后,重新恢复了以色情的情感依赖和性依赖,这实际上会巩固色情瘾,因为复发会激发个人对色情的欲望,使其行为更为恶劣,戒瘾过程更耗时、更具杀伤力。正如布莱德所说:“我坚持不看色情才几个月,就又重新开始看了,好像潜}意识里面要把错过的那些都补回来。”\\
为什么会复发?\\
如果个人习惯于用色情来对抗情感压力,追求性快感、那么不论你想要戒瘾的决心有多坚定、都无法抗拒想要继续看色情的欲望。色情瘾是一个周期性的症状,它并不会因为个人不看色情就自动消失。第四章中介绍过,长期观看色情和酗酒、吸毒一样,会改变大脑分泌的化学物质,想要一切恢复正常需要时间。\\
色情瘾康复者常常会极度渴望色情,即使在色情瘾痊愈之后,这种欲望还会继续困扰他们数个月甚至几年。能引发色情联想的情感、想法和情景,都会引起他们的兴奋回忆,激起他们潜意识中对色情的渴望。布莱德说:“酒鬼都是‘一辈子的’,只要世上一天有酒。他们就不会停。烟鬼也会很爽快地承认他们是‘一辈子的烟鬼’。好吧,对我们这些有色情瘾的人来说也是这样。一旦你和色情有了性关系,那你就永远跳不出它的魔掌了。”\\
色情瘾容易反复,源于现在的社会环境充斥着大量刺激的性信息和影像。惹火的像到处都是,要长期避开所有的色情信息并不现实,杂志里面的啤酒广告,电视屏幕上穿着比基尼的表演者,在非色情网站弹出来的色情链接。不过话说回来,就算那一大堆刺激性的影响不存在于现实世界,看过的色情影像还是会残留在脑海中。埃里克斯色情瘾痊愈已有十余年了,他告诉笔者,“现在想要远离色情还是很困难。我的脑子里还满是那些念头。任何时候,我都能回想起过去喜欢的色情故事和图像,那些东西已经印在脑子里了。”\\
毒品和酒精康复者告诉笔者,色情让人难以抗拒,数量繁多,又能引发强烈的性快感,这让他们觉得和烟瘾毒瘾相比,色情瘾更容易复发。拉斐尔,一名36岁的色情和酒精成瘾者告诉笔者,他对色情的欲望胜过对酒精的渴望。他说,观看色情不会像喝酒那样带来直接伤害。\\
“喝酒的时候,”他说,“‘喝多了会狂吐,不能开车,宿醉,但是色情不一样。我没有哪次对着色情自慰不嗨的。要强迫自己停止看色情,需要了很强大的理智。”\\
柯克曾经是大麻和色情成瘾者,他说色情瘾的反复比戒除大麻还要痛苦。“要戒除大麻,我只要不去买,不和毒友鬼混就可以了,”他说,“我不去聚会了,也不碰朋友家里的大麻烟卷,我现在都不跟那些毒友来往了。可是要避开色情就没那么容易了,性感图片到处都是,任何时候只要我想,就能看到色情。要远离色情,还要付出更多。”\\
很多观看者低估了色情的影响力,觉得自己能够避免、抗拒色情带来的冲动和欲望,这种自负心理并不利于克服复发。牧师吉姆·托\\
马斯会提醒每位加入男性色情康复团体的新成员:“这个房间里的每位成员或多或少都有过复发。不要掉以轻心,认为这种事不会发生在自己身上,在你以为自己行的时候,它可能就潜伏在你的身边。”\\
即使个人下定决心要戒除色情瘾,也得到了他人的关怀和支持,可是复发还可能会出现。一些色情瘾痊愈者认为,康复早期是复发的高风险时期,因为他们对这个决定还有所迟疑,或者疏忽大意,也还未戒掉通过色情来获取性快感的习惯,所有这些因素让他们难以彻底远离色情。\\
另外一些色情瘾痊愈者告诉笔者,在康复后期,当他们信心十足,认为自己已经掌控了进程时,复发的风险很大。那时他们己经忘记自己曾经怎样依赖于色情,也忘记了色情会带来的问题,甚至忘记了即使稍稍接触色情也会非常危险。成功远离色情的那段经历也会让他们降低戒备心。“我以为自己可以的,”科里说,“我只是想要测试,自己还可以接受什么程度的色情。”科里也发现这种“测试”会增加自己暴露于色情的风险,削弱抵抗力,甚至在自己察觉之前就导致了复发。\\
复发如何发生?\\
色情瘾的反复同不小心踩到路上的水坑不一样。康复中的色情瘾者不会毫无缘由地就重新陷入色情陷阱之中。跟生活中很多事情一样,复发不是一蹴而就的,而是阶段性发展的。它在特定时间段内发生,阶段性地发生。起初个人和色情无关时,笔者称这个状态为“无色情区域”;进而演变到个人对色情没有抵抗力的状态,即“一触即发区域”。在这个区域内,就意味着个人会阶段性地重拾色情瘾。图1“复发过程”显示,色情瘾复发在个人重新接触色情之前,就已经显出苗头了。【图中内容为,复发过程按照 无色情区域(与色情无关系);一触即发区域(容易复发);复发区域[第一阶段(想着观看色情),第二阶段(接触色情):感官上接触色情,第三阶段(将色情作为性发泄途径):全面复发]依次发生】\\
治疗色情瘾的最终目标是停留在无色情区域,即在精神和身体等各方面完全不接触色情,不受色情影响。如果读者培养出本书第十二章中所描绘的基本行为准则,就可以保证自己停留在无色情区域内。\\
当个人身处无色情区域时,环境中的外在影响或者是自身的情感和肉体感受,都可能会刺激个人进入一触即发区域。一旦进入这个区域,即使个人没有采取任何具体措施,心理上就会产生想要看色情的念头。外在的导火线,包括意外地接触到性意味的影象和信息,遇到挑逗性欲的人,接触色情传递系统,如电视和网络。内在导火线包括精神压力,觉得沮丧、孤独,或是感到愤怒、压抑,精神焦虑、精疲力端,性方面受到压力或者是受到酒精、毒品的影响。查理说:“我在看MTV,看报纸的时候,就知道自己已经在一触即发区域,就快要进人复必区域了,我老安慰自己说,总没有坏到看色情。”\\
导火线因人而异。比如说,看带着性爱镜头的限制级电影会触发人看色情的念头。但对另外一些人来说完全无效。\\
很多色情瘾康复者己经进人触发阶段,却还蒙在鼓里,这种情况很普遍。很多时候,我们甚至都没有注意到存在的问题,可能当下并没有想看色情的念头,但是几天或是几个月之后,个人就失去了对色情的戒备,那么色情瘾随时都有可能复发。“我不知道那是复发给我挖的陷阱,”布莱德说,“我先是感受特别,紧接着有些事会激发我,我还蒙在鼓里的时候,其实心里早就想看色情了。到了这个时候,我要不就是咬紧牙关,好好过小日子,尽最大的努力不去看,要不我就马上缴械投降了。导火线可以仅仅是开车时看到啤酒广告牌上面的大胸脯女人。”\\
即使身处一触即发区域,个人还是有机会避免色情复发。清楚的自己的软肋,当刺激性事物出现时,采取措施去平息,以免自己陷入复发阶段。而且,身处一触即发阶段时,要远离色情回到无色情区域还来得及。\\
处在复发区域的第一阶段时,个人会想看色情。享受色情的回忆浮上心头,个人就会怀念看色情时的舒坦,开始计划何时再看色情。“我总是会想到性,开始自娱自乐,”尼克分析了过去的复发情景,总结说,“要是有了这种念头,就算只有零点零一秒,我就死定了。色情的幻想很刺激,而且那时我也没办法让自己不去想那些。在复发阶段的时候,我就已经把色情的害处全都抛到脑后了。我原本以为自己可以轻松就付,但其实那是很危险的。这就好像两根电线碰到了一起,我都没来得及把它们拉扯开来,它们就已经碰撞出火花来了。一旦火花点燃了,我就被欲望牵着鼻子走,想去搜索色情了。”\\
如果没有及时遏制欲望,且欲望一再被激起的话,个人就会进入复发区域的第二阶段,开始看色情是迟早的事。不管是随便翻阅色情杂志,租借色情DVD,还是登录色情网站,就是个人重新看色情、进一步陷人复发区域的第三阶段,将色情当作性发泄的方式。如果个人身处第三阶段,将色情当作激发性欲,促进自慰和高潮的方法,继续频繁看色情的行为非常危险,因为色情带来的快感会刺激个人的欲望。如果复发的第三阶段反复循环,那么个人戒瘾的长期努力就可能全部化为泡影。\\
个人在复发区域陷得越深,就越难回头。这就好像流沙的吞噬力量,让人落到无底深渊,想要看色情的欲望会无限扩大。和色情保持性关系的念头,先是让人觉得好玩,然后是期盼和向往,最后就是实施了。\\
如何遏制复发\\
不幸的是,当个人陷入复发区域时,没有什么外界的警告和指示——没有霓虹灯也没有警示灯——来提醒你。这样一来,个人可能只有到了想要看色情或者开始看色情的时候,才会意识到自己处在复发阶段了。不论个人在复发区域的哪个阶段,最好的办法就是,采取正确的措施,回头是岸。尽快回到无色情区域。\\
通过实行以下五步预防法,读者可以有效地遏制复发:\\
1.停止自己正在做的事,承认自己确实已经进入了危险区\\
2.远离色情念头和色情\\
3.安抚自己的精神和情感\\
4.求助于外界\\
5.重申自己的康复承诺\\
图2显示,读者可以在任何时刻,通过实施以下5个步骤来防止自己在复发区域越陷越深,并重回无色情区域。【图中表示,一旦发现进入复发区域,就要立即停止,远离,安抚,求助,重申(就是上面说的那5点),然后返回无色情区域】\\
接下来,笔者将会详细介绍五步防复发法。\\
1.停止自己正在做的事,承认自己确实已经进入了危险区。如果意识到自己正在做的事会让你想要看色情,那就必须马上停止。如果你掉以轻心,忽视这些危险信号的话,就是在给自己挖陷阱,让自己在复发区域越陷越深。另一方面来说,正视自己身处危险境地,摆脱色情的想法和行为,可以打破自我麻醉的状态,遏制复发。“停止”就是给自己一个机会,承认自己做的事很危险的。“如果我意识到自己在看的内容让我想起了色情,”科里称,“我就会马上停止,告诉自己,这就是色情,我不能再看了!以着自己大声地说出这个事实,我就会恢复清醒,这让我没法跟自己撒谎。自欺欺人这种事,我以前看色情的时候经常做。\\
停止行为并承认事实的办法非常有效,这是抵抗色情瘾反复的必备良药。一旦发现自已有想看色情的想法,打住!就算你是在对着色情自慰的半路上,也要马上打住!打破过去观看色情的模式,重新控制自身的思维和行动,作出正确的决定。\\
2.远离色情念头和色情。一旦你意识到自己身处复发区域,马上采取措施。尽可能远离色情。如果你想看色情,马上转移注意力。\\
如果你在用电脑,马上关机,转身离开。如果你在看电视,马上换台,或者关掉电视,或者站起身来,离开房间。出去散步,给朋友打电话,听听音乐。享受大自然空气…做一件转移注意力的事(请参见以下“转移注意力”方法)。\\
转移注意力\\
一个简单的感官意识测试可以帮助你停止现有的念头,转移注意力,转而去想身边的亨物。首先说出这个短语:“现在我发现……”,然后把这个句子补充完整,把所见的事物补充进来。比如说,“现在我发现太阳从窗户外面照射进来”。通过观察身边的不同事物,连续重复、补充这个短语“现在我发现……”五次。接下来,根据自己所听的声音造五个句子;然后用五个句子,说出你可以接触的事物或是内心的不同情感。这个测试可以帮助你将意力转移到现实环境中,远离色情的虚幻世界。\\
不管多么欲火焚身,你还是有机会改变改变当下的想法,拒绝正在观看的事,停止正在做的事情。玛丽就是在复发的半路上改变了主意。”“我花了整整一大的时间开车在镇上转悠,想买色情杂志带回家,从头看到尾。我路过很多家熟悉的杂志店,最后停在了一家门口,走进去买杂志。我走到柜台前准备结账,就在那一瞬间,忽然醒悟到我真的太二了,我就是被困在老习惯里面了。我把杂志放在柜台上,转身离开了那家店。”\\
3.安抚自己的精神和情感。转移注意力非常必要,但同时你还要意识到自己已经被想法或行为刺激到了。心中一直想着看色情,这会引发特定的精神反应,如心跳加速、血压升高、生殖器充血、瞳孔放大,这些都会刺激复发。此外,幻想观看色情也会促使大脑分泌多巴胺和其他快感化学物质,这也会激起看色情的欲望。只要你停止想象观看色情,停止接触色情,这些精神和大脑化学物质的刺激作用就会渐渐消失。在此期间,你可能会因为没有得到色情带来的快感,感觉到身体上的不适。这就是为什么在性欲被激起时,学会安抚自已的重要性,\\
即使你想要看色情或者已经在观看色情,要安抚自已的方法还有很多。比如说,放松地坐下或者躺下,平稳呼吸,手放在胸口处,稳定心跳频率,让自己镇静下来;堵住自己的右鼻孔,用左鼻孔进行深呼吸,持续大约5分钟,这会让人放松;另一种有效的方法就是按摩双耳的外轮廓,这样会带来轻松感,让人镇定;按摩自己的双足也可以转移注意力,释放性兴奋。以非性的发泄方式让自己放松,这可以安抚被.刺激的大脑和身体。有些人发现,安睁的祷告或是冥想有镇定作用,让人放松,另一些人则发现激烈运动,比如慢跑、举重、骑车,可以燃烧多余的精力。\\
4.向外界求助。有效抑制复发的方式之一就是向他人求助。给了解你情况的咨询师,靠谱的朋友,教堂带头人或是伴侣打电话,这会把你从色情的世界中解放出来。牧师吉姆·托马师解释道:“当你意识到自己处在复发期,这是你打电话求助的黄金时间,向他人求助,跟人接触,打断复发的过程。马上就和别人沟通,不要一个人在里面越陷越深。”\\
麦切就是通过向牧帅求助,从而成功将自己从复发的第二阶段解救出来。“有天我在上网,不自觉地就开始浏览以前经常看的色情网站,”他说,“我知道如果当时不立刻解决这个问题,我就会把这件事深埋在心里,神不知鬼不觉,表现得好像没事人一样。当下我就马上打电话给牧师,说:‘我现在尽快去找你,我想要你为我祈祷。我要杆悔、我要自责。’牧师帮助我认识到,除了这个差错,我还是有进步的。在过去,我根本不会这样做,但现在我很在意这件事。”\\
在复发的过程中或者是事后向外界求助,有一定的挑战性。布莱德必须要努力克制自己的老习惯。“色情瘾复发的时候,我条件反射地就是想要‘冬眠’——远离所有人,享受私密空间,”他说,“我真不敢相信戒瘾竞然会失败。康复小组里的哥们儿都那么信任我,我也跟他们保证过,但我还是失败了。我当时在想,我要怎么跟他们交代?”布莱德在小组里坦白了复发的事。他发现所有的成员都很有同情心,善解人意,他们也很熟悉复发的情况,纷纷给布莱德很多防止复发恶化的建议。\\
埃里克斯一发现色情瘾复发的迹象,就会向妻子求助,“我要是看到什么刺激性欲的内容,就会跟她说,她喜欢我向她求助。她一直在身边支持我,变成了我康复过程的一部分,这让我很安心。”当然,前一章已经讨论过,依靠伴侣来获取支持是否可靠,还要视情况而定。\\
抵抗复发时求助的对象,最好是了解你的情况、能提供有效建议,不断鼓肋你的人士。\\
5.重申康复承诺。如果你自控成功,转移了自己的注意力、让自己冷静下来,并积极寻求帮助,那最后一步就是要坚定地重申戒除色情瘾的承诺。这个步骤需要你回想当初戒除色情瘾的原因,并明确自己想要达到的人生高度。如果你再次确立康复目标,重振旗鼓,重新明确承诺,你就会发现自己又回到了通往无色情生活的大道之上。\\
要重申承诺,最有效的办法之一就是重温自己在第七章中的回答来加强自信心。提醒自己色情会带来的问题,它是如何让你与他人产生隔阂,如何让你看不起自己,如何置你于危险之中,又是如何限制了你的性知识。明确自己的价值观和人生目标,重新认识自己,了解自己的生命价值以及人生目标。\\
既然决定戒除色情瘾,就要有始有终,就算遇到复发,也要为自己的行为负责,采取相应的措施,确保自己不会再犯。如果你需要建议或指导才能回到康复的正轨上,请借鉴前一章介绍的六步基本步骤法。评估哪个是你之前忽略的步骤,哪个细节执行得不够彻底。头脑风暴如何执行这些步骤,自己的康复才会更加顺畅。\\
一些康复中的色情瘾者认为,培养新习惯的办法十分有效,当自己在复发边缘时,新习惯可以及时遏制欲望。肯是一名40岁的已婚男子,他不希望自己一看到性感女人就想去看色情。在咨询师的指导下,他养成了一个新习惯来时刻提醒自已,就是马上触摸、观赏自己无名指上的结婚戒指,这样做会让他想起自己对妻子的爱和付出,还有他想要把色情清扫出生命的决心。培养这个习惯几个月后,肯就很少会有想看色情的念头了。\\
这五个策略——停止、远离、安抚、求助和重申,为戒除色情瘾提供指导,能有效遏制复发。接下来,笔者将会详细介绍,如何自信地运用自我意识,培养更多的技巧,让自己安全地远离复发。\\
预防未来复发\\
不论色情瘾是否复发过,探索预防复发的策略都非常必要。如果个人能及时察觉自己接近或是进入了一触即发区域,并预备了有效办法,就可以在复发进一步恶化前,火速将自己安全抢救回到无色情区域。(请参见图3,预防复发。)\\
此处提供两种预防复发的方式:了解自己的软肋,遵守自己的极限。下文将会仔细介绍这两种方式。了解自己的软肋。要了解点燃复发的导火线,最佳办法便是分析过去复发的情形,从中吸取教训。这个过程并不轻松,因为回想过去的复发经历会引发愧疚和懊悔的负面情绪。但是,如果你将此视为色情瘾康复教育的一部分,就能将自责情绪转化成具有教育性的正面启发。\\
科里回忆了过去的复发情形后,和治疗师进行了沟通,他希望了解复发是怎样发生,又为何会发生。遭受了色情带来的毁灭性重击:破裂的婚姻、他犯下的性侵罪行、在监狱中服刑,他简直不敢相信自己还会重新陷进去:“要是不弄清楚到底是怎么回事,我肯定还会再犯。”在治疗师的帮助下,科里制定了一个时间表,时间段包含复发的整个过程,他需要分析在这个时间段内自己每一刻的感受、想法和行为。他希望通过这种方式来确定自己为什么对复发毫无抵抗力,以帮助他在将来更好地判断复发的迹象。\\
科里发现,自己感到孤独寂寞,或是性得不到满足时,脑海中浮现的性爱画面就是复发前兆。“起初我会说服自已说,我只是在看泳装广告,”他说,“然后我开始看内衣目录。我骗自己说,这没什么大不了。打开色情网站的时候,我就马上兴奋起来了,既然已经在兴头上了,就没有干不出来的事情了。”科里现在明白了,就算从理论上来说,泳衣和内衣广告不算是色情,但是它们跟色情一样会让他性冲动。他甚至都不需要自慰或者达到高潮,光是搜索图片的行为,期待搜索的念头,就会让他兴奋。答案很明显,不是他看的内容,而是他为什么看的原因、怎么样看的方式,才是复发的诱因。\\
玛丽为了防止复发,也分析了最近几次复发的情况,她认为主因是精神脆弱之时,她急切需要安慰情绪的事物。“压力大的时候,我就会觉得自己在生活中只是个loser,然后大脑就好像条件反射一样,开始回放之前看过的,用来放松情绪的色情。接着我就被激发了,一心想着去买色情来看。我现在明白了,平时压力太大的时候要格外注意,马上减压,这样心里会舒坦点。”\\
德鲁也回头审视了两次复发经历,总结了教训。“第一次复发的时候,咨询师分析是因为我太孤单了,戒瘾的决心也不够坚定。她建议我去参加男性康复团体,但是我当时根本没有听进去。短短几个月之后,我就又去看色情了,到那时我才听取了她的建议,参加了一个团体。在过去的4年里,我每周都会去参加团体活动,这给了我很大的帮助。我再也没有复发过了。”\\
读者可以借鉴科里、玛丽和德鲁的实例,从复发经历中吸取教训。分析自己某次复发的情形,制订相应的时间表。判断在复发过程中不同时间点,自己的情绪、心里的想法以及自己的行为。是什么诱惑你进入复发区域?怎样做才不会再次受到诱惑?\\
还有一种了解自已软肋的方式就是全面想象在怎样的情况下自己会把持不住,进而复发。这种视觉化的想象可以帮助你确定一系列的因素,确定什么会促使你想到色情、想要获取色情、为了性目的而观看色情。不论你是否复发过,都可以从这个治疗策略中受益。\\
以下测试“我的色情瘾是不是要复发了”可以帮助读者认识到,引发色情瘾的因素复杂多样,了解这一点可以帮助你掌控复发过程。这个测试会引发你的思索,激发联想,因此,笔者建议读者在咨询师等人在场的情况下,进行这个测试。\\
我的色情瘾是不是要复发了\\
这个测试将会指导你幻想一段色情瘾复发的经历,促使你周全考虑可能引起复发的因素。\\
想象自己即将色情瘾复发,以设想为基拙,回答以下问题。结合亲身经历来完善答案。每一题请给出尽可能多的答案。\\
你可能身处的地点是(家里、工作场所、车里、旅馆、学校等)?\\
怎样的色情传递体系随时待命(电脑、电视、色情杂志、手机等)?\\
可能发生的时间(早上、中午、下午、晚上或是深夜)?\\
你可能和谁在一起(单独、朋友、伴侣、陌生人、一群人等)?\\
你可能在进行怎样的活动(工作、学习、进食、旅行、休息、放松时、娱乐时、运动时、阅读时、人际交往时等)?\\
你可能会感到的身体状况(疲惫、饥饿、愤怒、性兴奋、疼痛、精疲力尽、病态、寒冷、发烧、衣冠不整、身体状态不佳、健康等)?\\
如果还伴有其他上瘾症状或者问题行为,你可能还在干什么(吸烟、喝酒、吸毒、赌博、购物、熬夜、过度进食、过度工作等)?\\
可能刚刚发生的事件(一件让人失望的事情、道到拒绝、有一个成就、得到了奖项、付款、争吵、失去了一个机会、离开了某人等)?\\
可能所处的情感状态(孤独、压抑、愤怒、焦虑、沮丧、伤心、开心、无聊、失望、有力量等)?\\
你可能未受满足的需求(需要陪伴、激动、新颖、竞争、友谊、同感、爱、确定、放松、安慰等)?\\
可能会激发的情感伤痕(感觉被抛弃、被背叛、被拒绝、觉得自己无能、魅力不足、丢人、无能为力、不妥等)?\\
你不经意间接触到的性意味内容可能出现在(电视节目中的暴露画面、电影中的性爱镜头、网络上弹跳出的窗口、广告、露天广告牌、杂志封面、露骨的性描写文字等)?\\
你可能是以何种方式“越界”,做出了会让人联想到色情的行为(四下无人时上网、看电视时不停换台找性画面、在杂志上寻找性感的图片、看限制级电影、在旅店看付费色情电视节目、参加色情聊天室活动、私底下进行其他秘密行为、空闲时独自一人等)?\\
如果色情瘾已经复发,可能是因为你之前没有采取的措施是(停止行为并且承认自己在复发的危险区域、转移自己对色情念头和色情画面的注意力、让自己在精神和情感上冷静下来、迅速求助于外援、重新明确自己的康复承诺等)?\\
读者在这个刚试中确认的事物会促使复发。分析自己的答案,思考你如何才能加强自己对色情的抵抗力。你需要更好地照顾自己吗?需要学习如何安抚自己的感情吗?你需要改善和他人的关系吗?\\
多种导火线同时被激发,各种因素相互促进、相互加强的情况很是常见。比如说,独自一人、筋疲力尽、无聊、深夜在电脑前,这些因素相互结合就会产生高危状态。请在以下横线上写出相互联系的因素,未来你需要特别注意这些因素。\\
最后,根据你在这个测试中学到的内容,在以下横线上列出一系列健康的选择,对色情的抵抗力减弱时,你可以利用这些选择来抵杭诱惑。\\
杰西,25岁,已婚,他在治疗色情瘾的过程中,并未经历过复发。他设想了复发的情景,想象下班后,自己回到家里的车库,妻子还没回来,他独自一个在车上。他想象自己因为害怕饭碗不保,焦躁不安。他想象自己在开车回家的路上进了一家便利店,本来是想要买冷饮,但从自动苏打饮料机走到收银台的途中,他就不自觉地拿了一本色情杂志。由此,杰西判断,自己需要被满足的需求包括放松的身体、确定的安全感、工作表现良好以及自己能够得到重视。\\
杰西结合这些启示,制订出了一个防止复发的方案。他每天下班后就给妻子梅根打电话。这个方案还有个意想不到的附加好处,就是梅根每天晚上都会对他更加温柔体贴。如果杰西下班后想要喝冷,吃糖果,他只会去那些不出售色情杂志的商店。他决定减少每天的咖啡因摄取量,减少紧张焦虑感。他还让梅根每天晚上帮他按摩脖子和肩膀,让自己得到舒缓。杰西的这些改变,帮助他应对并控制了情绪波动,扼杀了他想要看色情的机会。\\
静下心来想象自己色情瘾复发的情景有一定的难度。没有人想承认自己重新陷入色情的可能性,更不会想详细地描绘复发的过程。不过,做过测试的色情瘾康复者一致认为,这个测试具有预见性,能帮助他们意识到,其实自己很容易被诱惑,同时,他们也了解到怎样的措施可以避免复发。\\
遵守自己的极限。另一个防止复发的重要技巧就是要清楚自己的极限所在,避免置身于高风险的境况,不然很容易失守。了解自己的软肋,个人就会知道自己行为和选择的极限。“我正在学习怎样远离悬崖,而不是去测试我到底能靠它多近,”何克解释道,“有一些事情,不管是不是跟性有关,\\
我都不再做了,我知道那是陷阱,会让我色情瘾复发。我要时时注意自己在哪里、在做什么,这样才能保证自己的安全。我现在就像是一个微博,需要经常更新自己的状态。”\\
以下举出一些个人改变的实例,这些色情瘾者采用的康复措施,遵守个人极限,能有效减少复发的风险。\\
·汤姆喜欢举重,也会阅读相关的报纸杂志。健身杂志里面经常会出现半裸女性的图片,他过去常对着那些图片自慰,现在他已经不碰那种杂志了。\\
·贾斯汀不再喝啤酒了,他过去常在看色情的时候喝。\\
·劳拉现在只看PG-13(适于13岁以下儿童)级的电影,因为R(适合18岁以上的成年人)级电影常有各种性瘾导火线,包括性爱镜头、黄色笑话、暴力,让她想去看色情。\\
了解自己的复发导火线,清楚并且遵守自己的极限,这些是预防复发的重要条件。花点时间考虑一下自己可以如何调整行为以减少甚至消除对色情的渴望。尝试这些改变,坚持一段时间,你就可以见证,这些改变是如何帮助你巩固了无色情的目标。\\
深入治疗阶段,预防色情瘾复发\\
对色情瘾深重的人来说,了解并规避导火线还远远不够。他们需要更强有力的策略来预防复发。如第二章所述,大部分色情瘾者自幼年或青春期以来,每每经受压力时便会去看色情发泄。即使现在已是成人,他们对色情的依赖性还和年少时一样,无法减弱。\\
如果确实如此那么色情瘾者需要解开童年时期的心结,找出自己和色情的纠葛源头。尼克在戒除色情瘾的头几年间,一直在和复发作斗争。“刚开始我看到的只是问题的表面现象,就是性吸引。我当时以为只要解决这个问题就可以了。后来我才意识到,我对色情的兴趣高温不退,肯定有其他更重要的因素。”\\
在牧师和男性治疗团体的帮助下,尼克发掘出了行为背后的驱使因素,那是一种想要得到肯定,希望被人接受的心理。“三年级的时候,同伴们很没人性地嘲笑我。我觉得自己没用,低人一等,特别自卑。只有在看色情小说和色情杂志的时候才能解脱。其实我就是用色情来弥补失去的那个自我。不管什么时候,只要我觉得自己无能,比不上别人的时候,我就会回来找色情,就这样。色情就成了一个根深蒂固的习惯。现在,我能把色情和不幸的童年联系起来之后,对色情的兴趣也少了很多。”\\
和尼克一样,劳拉也是通过了解深层的精神问题来加强对色情的抵抗力。“我在接受治疗的时候分析自己迷恋色情的原因,”她说,“我最爱看那种描写高危性爱的文字,女主角非常柔弱,在身体上受到威胁的那种。这其实就很明显了,我喜欢这种情景,就是因为我小时候被兄弟性侵,经常觉得孤独无助。”\\
分析自己最痴迷的色情幻想类型,是一种分析内心深处思想的有效方法。请在专业精神健康师的指导下回答以下问题,读者会有所启发:\\
·你最初喜欢的色情都是什么样的故事套路和情节?\\
·里面的主角是什么样的?在故事中主角如何和他人交流?为什么会用这种交流方式?\\
·你最理想的色情幻想是否和过去的需求不满或者痛苦经历相关?\\
伊森告诉笔者:“我分析了自己从8岁以来的色情幻想,之后对色情l就没有那么热衷了。我之前喜欢的色情情节都是一名女性被男人践踏、侮辱,甚至受到虐待。不管男人怎么对她,女性都会觉得这个男人了不起,死守着不肯离开。我分析了那些幻想,觉得那不仅仅跟性爱有关系,它们也表现了我年幼时因为妈妈常常不在而对妈妈的不满,自己无能为力,没有安全感。”\\
伊森分析了色情幻想中的人际机制,就意识到这种想法有悖常理,会阻碍他在现实中追求真挚的情感。“我从小幻想的那种女性在现实中根本不存在,”他说,“就算真有女人会受虐到这种程度的话,我也不会喜欢的。再抱着这种幻想就太傻了,我已经没兴趣了。到了我这个年纪,我更喜欢现实一点,追求一段健康的关系。”\\
色情瘾复发会让人心情低落,觉得羞愧又自惭。不过你始终可以扭转局势,充分利用过去的复发经历,更好地了解自己,发掘出色情成瘾的潜在因素,激励自己走上无色情的人生道路。科里很明智地告诉笔者:“你可以把复发看成是失败。但是我体会到,复发只是一个暂时的后退。不管你对自己有多失望,这都是一个很好的教训。”\\
第十章 伴侣治疗法\\
治疗色清瘾的同时,我们夫妻俩也解决了婚姻中的问题,重建了默契。 —罗根\\
丈夫罗杰因为色情问题丢了饭碗,黛比当时考虑过离婚。“我当时快疯了,”她说,“我跟他说:‘我告诉你,如果你还想跟我在一起的话,你就要去找专业帮助来戒掉色情。我已经不相信你了。’”罗杰不想失去黛比,也不希望一段20年的婚姻就此结束。他答应了黛比,开始参加每周一次的色情瘾康复小组,也开始进行一对一的咨询治疗。\\
之后数个月内,两人虽同住一个屋据下,却貌合神离,双方都对这段婚姻没什么安全感,只能走一步看一步。“我很受伤,愤怒到不行,”黛比说,“唯一能安慰我的事就是知道他在接受治疗。”\\
对罗杰来说,这也是一段非常艰难的日子。他和色情做着苦苦的斗争,怀念着当初与黛比的亲密无间,卿卿我我。”终于有天晚上,我坐在黛比身边,”他说,“直直地看着她的眼睛,告诉她,‘我不知道该怎么做,我已经全部按照你的要求去做了,而我也有进步。但是你好像一直在生我的气,我做什么你都不开心。刚开始我还能理解,毕竟是我背叛了你,但是现在我觉得你是在惩罚我,再这样下去的话,我们俩再也好不起来了。”\\
黛比也清楚地记得当时的情形。“我听得出来罗杰说的是真心话。如果还要维持这段婚姻的话,我必须原谅他。”这次交流鼓励了黛比,她开始自我治疗,阅读有关色情瘾、病态互依症和伴侣治疗法的书籍,“我们开始培养两人之间的默契,努力维护这段婚姻,之后事情就不一样了,”黛比说,“我们更加了解对方了,能更好地表达自己,这个关键性的转机,帮助我们维护了婚姻。”\\
罗杰和黛比的经历表明,色情问题对伴侣关系的影响之大。即使色情瘾者在努力戒瘾,但要真正修复两人之间的亲密度,还要解决很多其他问题。双方都想要重建信任感,愿意沟通,希望向对方表达爱意时,修复才能实现。\\
本章将会介绍几对成功经受住色情考验,并促进两人感情的真实事例,并进一步讨论伴侣们应该采取何种措施来修复色情带来的伤害。笔者也会为色情瘾者的伴侣们提供建议,鼓励她们借用这个机会来治愈内心深处的创伤,最终创建一段比以往更有爱、更幸福美满的感情。\\
以下是伴侣间可以采用的四步法,帮助修复色情对两人感情的破坏。\\
重树信任\\
体谅伴侣\\
3.放下愤怒,真心原谅\\
4.改善沟通方式,重建亲密关系\\
尽管笔者在此将步骤分为四步,不过各步骤之间存在着重合或互补之处。接下来笔者将会详细介绍。\\
1.重树信任\\
重树信任是伴侣治疗法的基础。当色情已经严重影响到了两人的关系,那么色情瘾者一定有不忠的表现,而不是他们的不忠让伴侣对他们所说的每一句话、所做的每一件事,包括要戒瘾的承诺,都抱着怀疑的态度。她担心如果这次又相信了他,到时候自己会更伤心、更失望。另一方面来说,不管色情瘾者戒瘾的决心有多坚定,付出了多少努力,也会因为没有得到伴侣的信任和尊重而受打击。伴侣火气不消,不信任的态度会让他觉得两人再也回不到过去了。\\
既然信任感是色情瘾者自己破坏的,那就理当由他来重新争取伴侣对自己的信任感。杰告诉笔者:“我花了这么多年的时间给自已挖了陷阱,让她没法相信我,我这都是自找的。我不能奢望黛比马上就原谅我,像以前一样信任我。我要再拉长战线,事无巨细,都表现出自己的诚意,让她知道我还是值得她信任的。”\\
信任很脆弱,一瞬间就会被击得粉碎,比如说,伴侣在你电脑里发现了一个装满了色情内容的文件夹,再要获得她的信任,可能要等上几个月甚至几年了,也就是说,色情瘾者要重获伴侣的信任,需要足够的耐心和毅力。只因为你现在很“乖”,就要伴侣重新信任自己,这不现实。此外,你也要在一段尽可能长的时间内,在所有事件中表现出自己的可靠性和可信度。\\
说到做到。说到重树信任这件事,行动远比言语重要。牧师吉姆·托马斯说:“一名不信任伴侣的女性会想:你爱怎么天花乱坠都行,不过你最好拿出具体行动来。我要着的是你行为上的变化,你的承诺对我来说已经一文不值了。”\\
能体现可信度的行为包括:参加治疗项目,采取措施避免色情刺激,包括意外看到色情或类色情信息,善待自己,建立亲友团、坦诚地向伴侣汇报康复情况。尽管南希逮到丈夫罗根看色情时很伤心,但3个礼拜之后,罗根就开始见个人咨询师,参加男性色情瘾团体,而且还同意参加伴侣咨询,他的这番努力让南希很是安慰。“罗根马上就去寻求帮助,参加了很多座谈会和小组,这让我看到了他的诚意。这证明,他很努力地想要了解色情问题的源头,彻底解决这个问题,他这是言行一致。在他开始治疗的6个月后,我又开始信任他了。”\\
如果色情瘾者能谨守承诺,紧跟康复计划,就算当下两人关系还存在问题,他还是能够通过长期的努力来重得伴侣的信任。\\
说出实话。要让对方重新信任自己,你必须足够诚实、真挚,因为当初就是色情瘾者的不忠和欺骗打破了信任。要让伴侣知道,你会一五一十地回答她所有的疑问,你还会主动汇报自己康复的进程。想重得信任,你必须坦诚所有的色情行为和其他个人行为。\\
但是,色情瘾治疗者或许会有所畏缩,不敢彻底坦诚,担心自己的不良行为会惹伴侣生气,拒他于千里之外。这种担心不可避免,但如果色情瘾者没有对伴侣全盘托出,没有把心里的想法和感受告诉对方,那么对方知道实情之后反应只会更加激烈。如果色情瘾者足够坦诚,坚守戒除色情瘾的目标,邵么伴侣就可能会愿意接受他治疗中的退步,继续修复两人关系。\\
为了避免意见不一。或是产生互掐之类的情况,双方可以在治疗师或神职人员的协调下,就如何达成如下康复目标达成共识:\\
·色情瘾者将会采取的戒瘾步骤;\\
·色情瘾治疗者参加治疗项目的疗程和频率;\\
·需要避免的特定性行为;\\
·色情治瘾疗者会多详细地向伴侣透露自己的色情想法和实际接触色情的情况;\\
·会导致复发的行为;复发之后多久会让伴侣知情,\\
尽管很多伴侣坚持要了解真相,但真要让他们亲耳听到对方在治疗中的退步,伴侣们总会失望不已。如果伴侣对戒除色情瘾的任务之艰巨有了彻底的了解,那他们就能理解对方了。\\
“我告诉罗根,如果他想看色情了,就跟我说,”南希说,“虽然我承认,他这样跟我坦白复发的时候,我会很难过,但是我不能把他诚恳的话当作武器,反过来去回击他、伤害他,那不是我想要的沟通方式。希望伴侣对你坦诚,唯一的办法就是要确定自己信任他,虽然这不是件简单的事。”宝拉也赞成这个意见:“事实就是事实,我知道某种程度上的复发属于正常现象,但是对我来说最重要的是确定丈夫是不是坦诚,是不是在采取行动认真康复,是不是总体上来说有进步。”\\
在色情瘾治疗的过程中,谨慎的交流模式有助于修复伴侣关系。许多治疗者的伴侣会采用24小时法,即色情瘾者要在复发后的24小时内告诉伴侣。这个暂时的延缓可以给人一个反思自己行为,接受他人建议的机会。如此,色情瘾者可以更冷静、更理性地和伴侣讨论复发问题。\\
布莱德认识到过去和妻子谈论复发问题的方式不合理,就决定改变。“我以前跟妻子说自己复发的时候.很容易说着说着就来气了,”他说,“那就好像我往她身上‘泼脏水’来减轻自己的负担,这样我是会好受些,因为我确实说了实话,但这样把宝拉毁了,这种闹剧让她非常为难。现在,和她沟通之前,我都会事先和咨询师、牧师还有康复团体的哥们儿谈谈,这给了我一个反思问题、修补问题的机会,让我学会自己承担责任。他们帮我想尽办法,防止类似的事再次发生,最后我才会和宝拉坐下来,好好进行一段理智的对话。我会坦白自己做的事,然后说我已经学到教训了。虽然她还是会失望,但至少我这种坦白的态度不会让她难过。她不喜欢看到我犯错,但是她还是说,我能主动寻求帮助,主动告诉她,不断进步,这样子就让她对我另眼相看了。”\\
分享负担。色情瘾者努力重树伴侣信任感的同时,伴侣也击要打开自己的心结。参加个人咨询,参与十二步治疗项目,比如说COSA(Co-Sex Addict A nonymous,两性成瘾者匿名会),和S-Anon治疗项目,治疗效果立竿见影。如第五章所述,许多伴侣发现对方有色情问题题之后,总是会产生强烈的控制欲望,想要凭一己之力就把问题解决掉。伴侣们监视对方的一举一动,成了名副其实的“色情警察”。这种行为虽然无可厚非,但也会在重建信任的道路上设下不可逾越的障碍。\\
伴侣双方都理应得到尊重。如果伴侣花过多精力来窥探对方的行为,控制欲望过强,草木皆兵的话,那么互相尊重不过是纸上谈兵。伴侣要清楚一点,对方的色情瘾问题并不是因她而起,她也无法完全控制局面,决定不了对方是否能戒瘾成功。做伴侣的一定要自爱,关注自己,合理表达自己的感情和需求,给对方足够的自由空间。伴侣要照顾好自己,平息愤怒,而不是一味要去改变色情瘾者。“我发现丈夫罗根在看色情之后,第一个倾诉对象就是教堂牧师,”南希说,“她劝我,不要再当什么色情侦探了,这样做一点好处都没有,我会再也没法信任罗根的,我想要控制的事情,我永远都控制不了。她建议我先解决自己的问题。当时她的话我没有听进去,在无数姐妹淘苦口婆心的劝告之后,我才决定给他自由的空间。知道罗根也在努力地改变,我很欣慰。让自己放手,不去打扰他的康复进程,我的生活也轻松了很多。”\\
如果伴侣的心已经变了,那么伴侣也要做好一切准备,即使暂时分开一段时间,也要尊重自己,维护安全感。黛比告诉笔者,她彻底摆脱了那种“怨妇”、“咨询师”,“警察”、“逼供者”等角色,她只警告丈夫,如果他不好好康复,两人就吹了。从那以后,她才算是真正解脱了出来。“我告诉罗杰:‘我不想再管你的康复问题了。我不会再问你在哪儿,也不会问你有没有去参加治疗会,不过我希望你能坦诚一点。’我这是以退为进,因为我知道还有很多人在监督他。而且,我已经说得很清楚了,如果他再不老实,我就会跟他说拜拜。”这种治疗策略乍听起来有些犀利,但在某些睛况下.这是重树彼此信任感的途径之一。\\
正确的行动、稳定的信任感、双方共同支持的康复目标,只要实现了这些条件,伴侣之间就能重树信任,修复被色情所拖累的情感。\\
2.体谅伴侣\\
打下了信任的基石,治疗的下一阶段需要伴侣更好地了解色情瘾者经受的苦楚。要实现这个步骤,需要色情瘾者和伴侣分享每一个细节,包括自己色情瘾问题以及自己在治疗中付出的努力;而伴侣则要说出对方的色情瘾如何波及自己。在沟通的过程中,如果两人能够相互谅解,那么伴侣会能更好地了解对方,更宽容地看待关系中的问题,这对推进色情瘾者的康复来说至关重要。越能诚恳地分享自己的经历,耐心地倾听伴侣的心声,色情瘾者就能更好地也解色情是如何离间了这段感情。\\
透露更多关于色情瘾的信息。色情瘾者承认自己有色情瘾问题或者意外曝光的那一刻,双方都会非常难受、非常痛苦。在那一刻,色情瘾者往往会非常羞愧,不愿和盘托出,不愿淡论细节,尤其是那些比较敏感的信息。同时,伴侣也会无比震惊,甚至不能理智地询问或倾听。所以,曝光或坦白的那一刻,并不是伴侣能够理性解决这个问题的最好时机,其中原因有很多,包括她们很难理解色情瘾从何而来,色情瘾问题的含义以及它给色情瘾者带来的困扰。\\
当治疗已初见成效,且伴侣也已经克服了最初的负面情绪,这便是双方可以进一步交流的时机。治疗几个月之后,罗根和妻子南希坐下来,更详细地坦白了他的色情瘾问题。“我对南希说,自从11岁以来,色情就成为了我生命中一个不能说的秘密,我过去是怎么对着色情自慰来缓解压力,又是多喜欢看拉拉色情。她问问题,我尽我所能地回答。我可以跟以前一样骗她,但是我也清楚,为了弥补这一切,我不能有所保留,必须说实话。”\\
尽管对南希来说,听到罗根说实话确实让她很难受,但是她承认,罗根这样做拯救了他们的婚姻。“我喜欢他老实地跟我坦白,不遮遮掩掩。如果当时他的态度没这么诚恳,而是随便找些借口打发我,把责任推我头上的话,我当时就会离开他。我之前一直担心,他需要找色情来发泄是因为我不能满足他。罗根再三以我保证,他觉得我很性感,还解释说他对色情的兴趣跟我没有关系,色情瘾就好像是毒瘾。”\\
听了罗根的解释,南希终于意识到他的色情瘾并不是自己的过错。她了解到罗根对色情的兴趣早在他们交往以前就有了,所以这不是说他对自己失去了兴趣,知道了这点,她如释重负。“我是没法满足罗根的幻想的,”南希说,“早在我们认识前,他就已经看了几年有色情了!”\\
更好地认识色情瘾问题也可以让伴侣释怀,停止对色情瘾者的鄙视和厌恶。伴侣可以不认同对方看色情的行为,但是深入了解他陷人色情陷阱的原因,会让她变得更宽容、更富同情心。\\
但是,要和伴侣分享自己色情问题的历史以及细节,不是毫无风险的。有些伴侣可以理性地消化很多信息,但另外一些就会被部分细节困扰,无法接受对方看的色情口味之重。即便如此,调查发现,如果准备坦白,不管开口有多难,色情瘾者最好还是全盘托出,不要遮遮掩掩,有所隐瞒。坦白不彻底,刻意的隐满、撒谎,会继续破坏这段情感,再次引起对方的疑心,加重对方的被背叛感和不信任感。\\
诸多专家一致认为,伴侣需要了解足够的信息,清楚现阶段自己对这段关系的期望,对色情瘾者的治疗进展期望合理。才能作出理性的决定。因为各人情况不同,笔者建议读者在坦白某些敏感信息前,一定要和伴侣商量好双方都可以接受的细节程度。伴侣可以事先写好问题清单,列出自已想要了解的事。许多伴侣们也觉得专业治疗师的建议非常有效,尤其是精通改善受色情影响的伴侣关系。\\
在沟通的过程中,最关键的可能并不是交流的内容,而是表达的方式。如果色情瘾者够诚恳,愿意坦白,对自己的所作所为负全部的责任,这样双方关系才能改善。此外,伴侣也需要认识到,色情瘾者需要鼓足勇气才能坦白,所以伴侣应该尊重也,不要利用他的坦诚来伤害他。\\
更好地了解色情对伴侣的影响。色情瘾者自然需要足够坦白,而同等重要的是,伴侣也要分享自己对色情问题和对方康复过程的看法,现时要意识到,对方确实在很诚恳地关心她,也很在意她的感受。伴侣可以通过言语或是书信的方式表达自己的态度,包括她对性爱物化的看法,色情问题对她的伤离之深以及她对这段感情未来的恐惧。\\
前文提及,许多伴侣发现对方的色情问题后,会感到极度受伤,既失望又忧虑,很难在短期内恢复信任感,还会抗拒对方。如果伴侣能够适宜地表达自己的不安心理,并且体会到对方真心在意她的感受,这样可以逐渐排遣忧虑感,让伴侣重新确定对方还是重视、珍爱自已的。\\
话虽如此,但是要伴侣说出自己的真实感受,也有一定的难度。和色情瘾者一样。她担心彻底的坦诚会进一步损害这段关系。实际上,如果伴侣善于表达,不去攻击对方,分享的成功概率更大。色情瘾者则专注于真心去体谅对方,设身处地为对方着想,使双方关系得到缓和,而不是为自己辩护、找借口来推卸责任,也不要得寸进尺。\\
伴侣或许对这种分享的办法心存顾虑,所以事先安排好沟通模式最好不过。针对了解伴侣心路历程的特殊伴侣咨询会议是一个不错的选择。此外,一些治疗师也建议,伴侣可以用书信来说明,对方的色情问题如何影响了她的生活。\\
用书信的方式沟通可以给伴侣足够的时间来思考想要说的话,斟酌合理的表达方式。如果她参加了咨询,可以向治疗师寻求指导,一起书写、修改这封信,一直到她满意为止。一些伴侣选择对着色情瘾者大声朗读书信的内容,而其他人希望色情瘾者独自一人安静地阅读。以书信方式呈现色情问题的影响,好处在于色情瘾者可以把这封信当作警钟,带在身边,以时刻提醒自己不要忘记曾带给伴侣的痛苦,从而坚持远离色情的目标。\\
如果色情瘾者设身处地为伴侣着想,体谅对方的心情,那么伴侣治疗就已经取得了关键性的进步了。“我知道老婆艾莉斯一直很反感色情、但我从来不知道其中的原因,直到那天晚上我们坐在后院走廊上彻夜长谈的时候,我才知道了原因,”乔治说,“她小时候,她爸爸就喜欢看色情,像是半裸女人的挂历,他床边还放着一沓沓色情杂志和小说;而且,不管妈妈和她怎样反对,她爸爸还一定要在客厅的咖啡桌上放色情杂志。艾莉斯最恨她爸爸下流地对女人的身体评头论足。看到色情里面的女性受到虐待,她会觉得无能为力,甚至觉得反胃。听了她的话,我才知道自己看色情犯了她的大忌。”\\
艾德了解妻子的经历之后,也有同样的感触。“我以前总以为看看色情没什么大不了,不过就是我一个人的幻想而已。不过看了妻子写给我的信,我才知道,要说对我们婚姻的影响,看色情跟在现实中搞外遇,找小妞没什么两样。”\\
艾玛向丈夫德鲁坦白说,她在发现他在看色情的时候,非常失望。德鲁这才发现,倾听是一门技术活儿。“要回想之前犯的错让我挺难受的,但这真的很重要,”他说,“艾玛一直以为我们的婚姻很美满——两个孩子,体面的工作,美满的性生活。但自从她发现这个秘密以来,我的形象就全毁了。我的形象,从一个绝世好男人就直接变成了家里的贻害。之后很久,她都担心要是我的色情问题传了出去,会影响到孩子们的生活,我们在这个小镇也混不下去了。我知道,她的担心不是没有道理的。”\\
这种深层次交流不仅仅可以增强伴侣之间的信任感,让彼此相互理解,也能重建双方之间的默契。“我发现罗根有色情瘾的时候,当下直觉告诉我。这段婚姻走到头了,”南希说,“但是现在我们能更好地理解对方,并肩作战。这确实花了我们不少精力,但是我最终了解到,他对色情的兴趣和我没关系,他也知道了为什么我会感觉这么受伤。”\\
深层次地分析自己最初的反应、了解伴侣的经历之后,你就会认识到,色情问题并不是一方施暴,另一方受害那么简单;实际上,色情是一把“双刃剑”,它同时伤害了双方,只要两人齐心协力,就可以相互支撑,相互帮助,彼此治愈心灵上的痛苦。\\
3.放下愤怒,真心原谅\\
伴侣冶疗的另一个重要步骤需要伴侣平息愤怒,试着去原谅。许多伴侣在发现对方的色情问题时,都怒不可遏。她们的愤怒不仅仅反映了心中的愤恨,受伤的情感,同时这也是她们用来反击色情瘾者背叛行为的武器。\\
伴侣火气的大小和发泄途径因人而异。黛比告诉笔者,罗杰开始戒瘾治疗的一年之内,她的火气还消不下去。“我气得暴跳如雷,”她说,“刚开始,我没有直接表现出生气的样子。知道他对我撒谎、背着我偷偷看了几年的色情,我比死还要难受。我对婚姻的梦想和期望全都破灭了。我恨他,怪他毁了这一切。我太生气了,怪他害我遭受这一切,我们没有夫妻生活了,他还丢了饭碗。刚开始我还可以掩饰自己的愤怒,但是同住一个屋檐下,我们俩却好像陌生人一样。一段日子之后,我决定不再伪装了。我并没有刻意想要报复他,伤害他,但实际上我还是借用这个机会报复了他。我再也不掩饰了,想说什么就说什么,两人的关系也一度降到了冰点。有几次,他进门的时候,我会直接说:‘不要走过来,今天我不想看到你。’接着我就会走出房间。”\\
黛比意识到,自已在把怒火当作武器去惩罚罗杰,而许多伴侣们在和对方发生摩擦的时候,这股怒火才会迸发出来。尽管卡伦平时没闹过什么脾气,但每次和丈夫强尼吵架的时候,她都会觉得心里有一股子闷火要发泄。“我们为了鸡毛蒜皮的小事吵架,他觉得我太任性,那时候我会暴露出自己的阴暗面,争辩说:‘我为了你承受了这么多,忍了你这么久,你竞然有胆子说我任性?’我这种恶劣的态度让两人没法沟通,我也不想一直把气氛弄得这么僵。”\\
一些伴侣们无意识地长期压制内心的怒火,等到最终爆发时,情况就容易失控。“整整一个月的时间里,我都是麻木的,”艾玛说,“最终发火的时候,就好像火山爆发一样,就算老公已经在治疗也没用。他这么多年不在我身边,我一个人抚养孩子有多艰辛?他对我撒了那么多谎,我婚后一直觉得特别孤独,所有的这些痛苦开始清晰起来。在很长一段时间里,我都在用怒火反击他、谴责他,这也是我知道的唯一一种发泄方式。”\\
尽管伴侣的怒火情有可原,但是怒火,尤其是用过激方式表现出来的愤怒,或是持续时间太久的怒火,会极大地阻碍两人关系的改善。怒火会蒙蔽伴侣的双眼,让她们看不到色情瘾者做出的积极改变,这会继续破坏两人之间的信任和默契,让双方感到爱已耗尽,情分已绝。\\
一起平息怒火。色情瘾者和伴侣要齐心协力,相互谅解,一起平息怒火。伴侣可以采取以下方法来平息怒火:承认自己正在生气,善于理性地表达自己的情绪,不要一直把对方看成是个负心汉,给对方一个改进的机会,支持对方的治疗。\\
黛比意识到,自己恶劣的态度已经影响到了她们的婚姻了,就向丈夫商量应对的方式。“我跟他说,在我生气闹别扭的时候,他应该直接告诉我,两人坐下来好好谈一谈我的感受,当下就把我的心结打开,这样我们的关系才能好转。”卡伦和丈夫强尼在日常生活中吵嚷的时候,她尽量控制自己,不去提他过去的色情问题,她也不会天天把这件事挂在嘴上,让他难堪。\\
如果怒火久久不能平息,就会掩盖内心的失落和悲伤,让人无法分辨潜在的情感。要平息怒火,关键在于发掘心中柔弱的一面。艾玛用这种方式克服了自己的怒火。“我必须要弄清楚,我的怒火下面隐藏的是什么,然后分析这些感情。我现在发现了,这是一种掩饰自己悲伤和失望的心理。接受伴侣治疗的时候,我找到了沟通的诀窍,知道怎样更加直接、更加诚恳地和德鲁沟通。一开始,我不喜欢说出自己的内心感受,因为我觉得这样就是把自己最柔弱的一面暴露出来了。现在我领悟到,要平息怒火,就要放松自己的戒备,向他展示我最脆弱的一面。我之前就是太生气了,都没有意识到愤怒背后的感情。”\\
艾玛学会的这种表达艺术,改善了他们的婚姻。“以前只要艾玛一发火、我就会让步,因为我一直觉得对不起她,但其实这样两人的关系只会一直恶化,”德鲁说,“自从她开始和我倾诉说她有多么担心、多么孤独之后。我渐渐能够理解她的感受,也愿意继续守在她身边。这对我们来说是破天荒的进步,和以前相比,我们的关系改善太多了。”\\
色情瘾者既要毫无保留,坦诚相待,又要严格遵守承诺,此外,他们还要体谅伴侣的感情,对伴侣表示最真挚的歉意,来帮助伴侣消除怒火。当然,怒火不是表达一次就完事的,而是要尽可能多的重复,直到伴侣的情绪平复为止。\\
色情瘾者也同样可以用书信来表达自己的情绪。在信里,他可以安抚伴侣,比如说,伴侣担心他会威胁家庭氛围,担心他的性注意力已经不在自己身上了,也担心他长久以来一直不够坦诚。乔恩,一名45岁的色情瘾者,给妻子凯写了以下这封信,来表达自己的歉意。\\
我最最亲爱的老婆:\\
我看色情背叛了你,对你不忠,我其的很抱歉。我对你撒了这么多年的谎,瞒着你看色情,瞒着你花钱,喝酒,有时候我的行为还触及了法律。\\
我怀疑这辈子我都不能彻底了解你受的伤害之深。我知道你爱我,希望我成为一个温情的男人,够忠诚,够靠谱,因为只有这样的男人才配得上你。我也知道,我的行为彻底打破了你的期望,让你觉得我根本不在乎你,甚至连你自己都开始怀疑自己,不敢相信自己对现实的判断,让你的自尊和自信都受到了伤害,我愿意付出一切来弥补你。\\
我知道,我让你受伤了,但是你并没有选择结束这段感情,还要对你的闺蜜们解释我们之间的问题,到头来却发现我还对你有所隐瞒,我还在偷偷地看着色情自慰,老跑成人书店。只要我一天不改变这种行为,你就一直要疏远闺蜜们,孤孤单单。\\
我知道,我所了解的远远不够全面。我只是想要写下我已经认识到的,对你造成的伤害。写完这封信,我会一直把这封信带在身上,放在触手可及的地方,把这当成是一个永恒的警钟,时刻提醒自己你为我受的伤。我每天都在祈祷,希望自己可以保持清醒。不犯任何性方面的错误,我会对你毫无保留的坦诚。我相信,我们最终还是拆不散的一对,上天也注定要我们在一起。我真心诚意地想要成为那个配得上你的男人,温情、忠诚。用什么语言也不能表达我对你的感激之情,谢谢你一直守护在我的身边。\\
爱你的老公\\
乔恩的这封信是两人重归于好的大功臣,它平息了妻子凯的怒火,让她原谅了丈夫。乔恩的这封信不仅表现出了真挚的歉意,也让凯听到了最重要的一句话,就是对他来说,她比色情重要,而且他会为自己的行为负责,保证不会再伤害她。看了这封信之后,凯又对丈夫敞开了心扉。\\
信任回来了,并不意味着就可以要求甚至强迫对方原谅自己。原谅,只会在相互理解和稳健的长期治疗基础之上自然而然地发生。伴侣克服了自己的负面情绪,原谅了对方,但这并不意味着伴侣不能要求对方为自己的行为负责,而是需要伴侣调整心态,接受对方的为人、他的过去和他的不完美。确信对方有能力改变,有能力抚平他给你带来的伤痛。。而色情瘾者需要保持诚恳的态度,为自己犯下的错承担后果,认真专注于康复,对伴侣的伤痛表现出怜悯之心,表达自己真挚的歉意,那么你就够资格得到对方的原谅。当色情瘾者为弥补过错而努力踏出每一步时,他就会发现,不仅和伴侣的关系开始改善,而且自已的人格也得到了提升。\\
4.改善沟通方式,重建亲密关系\\
伴侣治疗的第四步要求双方学习沟通的技巧,重树亲密默契。许多伴侣们在接受色情瘾治疗的同时,也顺利修复了情感关系,他们认为成功的秘诀在于建立分享信息和交流情感的沟通模式。\\
罗根和南希采取的治疗方式还包括阅读有关沟通交流的书籍,参加伴侣治疗会议,治疗师也经常指导他们如何有效地表达情绪。“我们学会了怎样辨别内心的感受,要和对方诚恳地交流,不要把气氛弄得太僵,”南希说,\\
“除了治疗,我们还养成了一个习惯,就是晚饭后聚在一起,交流一下感情。我们会去散步,或是坐在沙发上聊天,谈谈自己的心情,各自的工作、讨论想要解决的问题。这样,我知道罗根每天都做了些什么,我们就不会是生活在两个不同世界的人了。”\\
罗根也这样认为:“过去我们总是设法避免冲突,但现在我们已经可以表达自己的不同看法,进行精彩的辩论,然后解决问题。改善了沟通方式之后,我们俩变得特别亲密。也更加信任对方了。\\
乔恩和凯每周日早晨会坐下来好好聊聊,说说自已在做的事,交流各自对康复过程还有怎样的疑惑。这对乔恩来说是一个挑战,因为他从小就学会把事情藏在心底。“我慢慢适应了以后,就能够更好地表达内心的想法,”他说,“我最擅长把自已藏得很深,而我们的交流就是要帮我克服心里的怨恨,不要把情绪都理在心里。这对我的康复帮助很大,只有这样我才能消除愤怒,不会觉得自己是受害者,这样就能避免会引起复发的负面情绪。”乔恩能好好地和凯交流,这也让凯觉得安心。“我觉得他不会那样排斥我了,”凯说,“乔恩会告诉我他都做了些什么,我知道对他来说我还是很重要的。”\\
即使两人意见不一,也需要两人求同存异,将每次交流的重心放在如何更好地理解和支持对方上,保持通畅的沟通方式,这种定期的伴侣沟通途径效果最好。适合伴侣讨论的话题包括:\\
·当下生活中的难题\\
·自己的感受和忧虑\\
·对自己的新看法和新感受\\
·自己的进步\\
·自己经受过的失败\\
·自己希望从伴侣处得到何种帮助\\
·赞赏、欣赏对方的行为和想法\\
·两人关系的改善\\
定期的面对面交流可以帮助伴侣双方学习、实践更多积极的沟通方式,巩固两人关系。“现在,黛比说出她想法的时候,我会认真地听,不会显得不耐烦,”罗杰说,“就算她闹脾气或者我们意见不一,我也可以很理性地处理。我要是动了想看色情的念头,也会老老实实地告诉她。这种真诚的沟通方式,非常有助于我的康复。我们俩都学会了怎么去了解对方的心思,怎么表达自己的需求。”\\
黛比还说:·‘我现在会坦率地说自已很担心,会直接问他自己想知道的事。我看得出来,罗杰很认真地在听我倾述,没有逃避。\\
他还教我怎么有技巧地问他治疗的事,才不会让他为难。罗杰会主导我们的谈话,一直和我保持眼神上的交流。现在我们能够这样畅快也和对方分享心事,我觉得他像是变了一个人似的。”\\
认直积极地去关心对方生活中的问题,用一种有爱的方式和对方沟通,重新建立起两人之间打从色情问题发生以来就不复存在的纽带。这会让双方感受到,彼此是独一无二的,彼此之间也相互理解,这会给双方都带来安全感,增加亲昵度,自然而然促使双方在身体和情感上更加亲近。\\
这四步修复伴侣关系的方法:重树信任,增加理解,平息怒火,增强沟通,能稳固伴侣关系。色情瘾者需要用持续可靠的行为向伴侣证明,自己是铁定了心要戒瘾,自己也很坦诚,并且自己也在努力弥补错误,这点至关重要。对伴侣来说,了解对方色情瘾背后的深层次原因,平息怒火,一起努力垂重建沟通的桥梁、都有助于修复色情带来的伤害。\\
采取这四步治疗法的伴侣们常常会看到可喜的改善。“对我来说,最大的好处就是我对这段感情又恢复了憧憬,”凯伦说,“我之前一直以为我面前只有两条路,要么离婚,要么下半辈子就只能继续过这样悲惨的日子。但是,我们现在的婚姻就是我理想中的状态,现在我很满足。”同样地,接受了5年的治疗后,艾玛称:“我和德鲁的婚姻从来没有像现在这样稳定过。我们更加了解对方了,有冲突的时候也能很好地解决。我坚信,我们会一直这样幸福地走下去。现在的我,比任何时候都更爱德鲁。我们相互尊重,而且有着共同的人生目标。”\\
对黛比和罗杰来说,治疗让他们的心走得更近。“罗杰和我又是一对儿了,”黛比说,“对付色情问题的时候,他不是孤军作战,而是我们在一起并肩作战,对抗共同的敌人。罗杰过去总觉得我碍着他看色情,现在,他觉得色情碍着他来全心全意爱我。”罗杰也赞成黛比的说法:“4年治疗之后,我们的关系改善了很多。现在我们之间的感情,更加稳固,更加健康。我们得到了那么多帮助,付出了那么多努力,好像我们拆掉了之前婚姻的不良基础,重建了一个更加稳固的。我很感谢黛比,就是因为她一直陪伴在我身边,我们现在还能在一起。”\\
仅仅四步是不够的\\
尽管这四步对大部分伴侣们来说非常有效,但是对另一些人来说,过去的欺骗太多太深,现在想要在短期内重获信任,得到原谅,恢复亲密关系并不现实,韦斯和玛吉就是这么一对。他们接受了治疗师的建议,决定使用谎言探测器,也称为测谎机测试,来帮助他们重建信任,跳出僵局。这种客观可靠的测试可以帮助伴侣了解色情瘾者是否说出了实话。当下,这种科技作为伴侣治疗的附加工具,在色情瘾治疗中非常流行。只要肯定对方说的是实话,伴侣就能根据四步治疗法继续治疗,并从中得益。\\
谎言探测器如何拯救了一段婚姻:韦斯和玛吉的故事\\
50岁左右的韦斯和玛吉十指相扣,跟笔者分享,在当初色情问题发生后,他们是如何修复了这段感情。他们的婚姻已经经营了25年,养育了3个儿女。在这段婚姻的大部分时间里,韦斯没离开过色情,而且一直瞒着妻子。他骗玛吉说他已经戒瘾了,瞒着复发的事情不说。玛吉发现的时候彻底崩溃,对他失去了全部的信任,还把他踢出了家门。他们当时认为,最小的孩子离家去上大学之时,就是这段婚姻走到头的日子。\\
韦斯:“我们分居了两年。我从结婚开始就一直瞒若玛吉看色情,玛吉知道以后快气疯了。我当时以为,我们俩就这么吹了。一天,我午睡后醒来的那一刻,忽然觉得我真正想要的是和玛吉一起共度余生,色情和其他的偷吃根本不重要,那都是浮云。为了玛吉,我什么都愿意做。现在很难去解释那一瞬间的奇妙感受,那好像是一种顿悟,忽然而来的灵光一动,现在想起来确实很神奇。我当时就决定了,不惜一切代价,也要重新赢得她的信任和爱。”\\
玛吉:“韦斯告诉我,他决定好好做人,放弃色情,他问我会不会重新考虑跟他在一起。我知道他本质不坏,也很有前途。我当时不知道该怎么做。这么多年来,我一直蒙在鼓里。虽然分居让我心痛,但是有什么理由可以让我再去相信他呢?”\\
韦斯:“当时我是百分之百确定要戒除色情,但根本没人相信我。治疗师建议说,可以用谎言探测器来证明我对玛吉的真心,我想,哇,这真是一个好主意!我已经’下定决心不会再对她撒谎了,谎言探测器是个证明自己的好办法,有了它,我就不用花十几二十年的时间来让玛吉相信我的真心了。”\\
玛吉:“当韦斯告诉我他要测谎的时候,我被震住了,这再次证明他是认真的。如果你的伴侣为了色情问题撒谎,那么你不仅不相信他,会连自己的想法都否定掉。因为之前你感觉蹊跷,但开口问的时候,对方连连否定,要不就是闪烁其词。这次不一样,压力全在他身上了,我只要轻松地去就可以了,不需要再怀疑,因为他自己会证明自己的真心。”\\
韦斯:“我决心准备做心理测试的时候,心里非常激动。我当时有点担心测试会出错,但误测的情况很少见,可以忽略不计。治疗师推荐了一位测谎专家,他擅长用测试来改善色情瘾者和性瘾者的婚姻。整个过程需要3个小时,那位专家先和我们打了招呼,然后他单独和玛吉讨论了要提的问题,确定了提问的方式。接着,他用半个小时的时间给我测试,这段时间里,玛吉出去走了走。”\\
玛吉:“测谎专家很明确地告诉我他是站在我这边的。他告诉我:‘我们在这里的唯一原因是因为你被骗了,受到了伤害。我想让你把心里的疑问都说出来,一次解决你的所有疑惑。’这话很让我安心,也让我心潮涌动。天啊,我们之所以会在这里就是因为我被骗了15年。只有做这个测验我才能知道韦斯说的是不是实话,够不够靠谱。这个测试确实意义重大。”\\
韦斯:“我躺在椅子上,浑身贴满了测试绷带,刚开始我心里还有\\
点小紧张。所有的机械原理和步骤我都事先了解过了,我也清楚误测的概率非常低,但是第一次接受测试总是会胡思乱想。不过我当时就知道了所有的题目,也就没有那么紧张了。当测试师问我:‘过去6个月里,你有没有为了性快感而观看或者是保存性影像’的时候,我很自然地说出了实话:‘没有。’”\\
玛吉说:“我散步回来的时候,忐忑不安。测试师用一个肯定的微笑迎接我:‘韦斯这次是真心的。他通过测试了。’他可是测谎的专家,他的判断给了我无限信心,我又可以相信韦斯了。我如释重负,实在太开心、太激动了。”\\
韦斯:“我也松了一大口气,并不仅仅是轱辘为我通过了测试——因为我说的都是实话——还因为玛吉很开心。她又给了我一次机会,那我们还有希望。我们预定每6个月我就会来进行一次测试,而且玛吉可以随时要求我来做测试。那之后,我就搬回了家和她一起住。初次测试是几年前的事了,之后每6个月一次的测试,我都通过了。不用再隐瞒的感觉很好,我也不需要再撒谎了。撒谎的时候。我总是提心吊胆,缩手缩脚的,不用撒谎的时候,心里坦荡荡的,一片清明。”\\
玛吉:“测试让我们两个都卸下了重担,我现在再也不会追问他在做什么,也不会怀疑他说的话。色情完全不可能影响我们的婚姻了。韦斯能这样诚恳,让我很爱他,很尊敬他。我觉得,如果当初不是有测谎机测试,可能我们现在就已经散了。”\\
韦斯:“知道我每6个月就要接受一次测试。我就切断了自己所有对色情的幻想。这就好像系上了安全带,穿上了救生圈。我不会再花心思去想色情了,因为我绝对不能再看了。虽然有时候这种紧张感也算是一种负担,但是我不会放松警惕,去想那些有的没的。想看色情的想法最多持续几秒钟,我很快就会打消那l种念头,只去想着测谎机测试太神奇了。有谁想对伴侣表决心的话,我强烈推荐测谎机测试。”\\
韦斯和玛吉的经历是一个有力的例子。它证明色情瘾者为了要摆脱色情带来的影响,可以变得多么坚定,多么勇敢;这也见证了即使在最危难的时候,爱和承诺力量无穷。本章讲述的所有事例都证明,如果伴侣两人能够齐心协力,坚持不懈,就能摆脱色情瘾,让两人都爱意更浓,情意绵绵。\\
第十一章 培养新的健康性观念\\
我的色情问题曾严重到让我害怕性爱,怕自己做爱的时候又恢复旧习惯。我知道,如果不学习新的性表达方式,以前那种具有破坏性的坏习惯又将卷土重来。 —玛丽\\
玛丽一样,很多色情瘾康复者都对性生活有所顾虑。他们经常怀疑,自己会不会在性生活时想起色情,被激起看色情的欲望。单身的色情瘾痊愈者会担心自已无力发展健康美好的性关系,而对有伴侣的色情瘾康复者来说,他们曾让伴侣备受困扰,这会让他们怀疑自己是否还能和伴侣重享性爱。尽管疑虑多多,但是大部分色情瘾康复者都希望找到合适的办法。恢复不受色情影响的美满性生活。\\
或许读者也有类似的疑惑,不知道如何用新方式来调整和表达性欲。但可以肯定的是,色情瘾者通过调整性举动,培养新技巧,就可以拥有幸福美满、不受色情影响的性生活。举个例子,曾观看了25年色情的贾斯汀,在成功地将色情扫出生活之后,对自己现今的性生活非常满意。“我戒除色情一年之后,遇到了一个很特别的女孩子,”他告诉笔者,“我们在一起4年了,我们的性关系一直和谐得不得了。女朋友说我是一个绝种好爱人。这话我听了很受用。而且,因为在我前一段婚中我总爱看色情,妻子老是抱怨我做爱太机械,把她当发泄工具。现在和女友做爱的时候,我不会觉得内疚,只会有精神爱恋上的超快感。”\\
色情瘾痊愈后,性治疗的重任首当其冲。如果个人已经恢复了正直感,确立了自我价谊,发展出了一个强大的亲友团,在人际交往中重树了自己的形象,而且别人对自己也有了信任感之后,培养新的健康性观念就是自然而然的需求。下一步,就是要将自己的性欲转化为生活中的积极因素,如此一来,生活会有翻天覆地的改变,个人将会培养出良好的自我感觉,以及体验性快感的能力。\\
许多色情瘾康复者都成功培养了健康的性观念,保持着无色情的生话状态。人人都有性欲,性需求是一种基本需求,也是人之本性。如果个人表达性欲和性冲动的方式符合自己的价值观,那么色情陷阱就很难再吸引他了。几次色情瘾复发后,伊森总结说:“我未来的幸福全看我能不能找到替代色情的东西了。我的性欲并不会消失,所以我需要找到替代色情的事物,才不会走禁欲、性行为失控这些极端。”\\
以亲密为导向的性方式\\
何种形式的性足以和色情抗衡?笔者在采访色情瘾康复者的过程中发现,无论个人有伴侣与否,克服色情瘾的有效办法之一就是需要个人明确目标,确定要和活生生的人维持亲密关系。色情性爱都是幻想而已,对色情瘾康复者来说,和现实中的人进行性接触才是新鲜的、有诱惑力的。以亲密为导向的性爱需要个人综合对现今或者未来伴侣的积极情感,包括愿意只为对方表现自己的性欲,只会性挑逗对方一个人,只会在对方而前展示自己的高潮。如此,个人便能专注于现实生活中的感情,合理表达自己的真实感受。这种方式以尊重健康性爱为前提,强调包括责任、平等、尊重和爱护等正面的积极感情。没有人被压榨,也没有人会受伤,这样的性爱不会带来羞耻感,因为它符合普世的价值观和人生目标。\\
以亲密为导向的性爱鼓励个人去发掘体验无色情性爱的种种可能,比如调动全身的感官、树立个人的自尊、确定彼此之间的信任感、感受伴侣的体温,享受两人之间的嬉戏,发笑,温柔的爱抚,进行有爱的精神交流。这样的性爱建立在彼此了解和彼此相爱的基础之上,表达的都是最真挚的亲昵。伴侣双方都会感到满足,也能热烈地说出自己的欲望和需求,包括自己对性的感觉。性生活是相互满足的过程,在此过程中,个人可以感受到爱与被爱的喜悦。正是因为以亲密为导向的性爱可以让人在那么多层面上感到满足,所以它能帮助个人将色情幻想重塑成健康的性观念。\\
本章将会介绍一系列概念和练习,以帮助个人学习技巧,发展以亲密为导向的性爱方式。你可以独立完成部分练习,而另一些练习则需要伴侣的参与。如果读者目前没有伴侣,可以先阅读并练习本章内容,为将来的性生活打下墓础。如果读者有伴侣,那么你可以和伴侣共同努力,抵抗色情给性生活带来的不良影响,一起体验更幸福的性爱。\\
要实现亲密导向的性生活,关键在于远离色情之后,你是否还能找到一条健康发泄性精力的健康途径。为实现这一目标,本章将会提供不同的练习。帮你打破色情和性欲之间的联系。长期实践这些练习,个人就可以在身体和精神双重层面上体验性爱。本章大部分练习旨在修复个人意识,培养亲密技巧,防止个人进行过激的性爱活动,同时这也能帮助你享受无色情的美满性生活。\\
笔者总结出以下多项技巧,旨在帮助个人建立亲密导向的性爱方式,包括:\\
1.采取求爱行动\\
2.和伴侣讨论性问题\\
3.增强感官意识\\
4.用新方式看待伴侣\\
5.增加触感词汇\\
6.探索感官愉悦\\
7.用心做爱\\
这七个技巧循序渐进,从帮助个人建立性爱关系的健康基础,到加强特定感官意识,培养健康性行为,每一个技巧都会促进双方在性爱中的亲密度。由于部分练习包括触摸和刺激性器官,笔者建议读者结合自己的具体情况,选择合理的方式实施。如果读者正处于康复计划中的禁欲阶段,那么笔者建议读者去请教咨询师或治疗师,确定何时可以实施以下技巧和练习。\\
下文将会详细介绍这七个技巧。\\
技巧一:采取求爱行动\\
蓝鲸会用鳍状肢相互磨蹭对方;雄性狒狒昂首阔步,前后摆动身体,拂佛也会接吻、牵手;雌性负鼠面对求婚者时,会高傲地抬起下颚,直视对方。所有的哺乳动物,以及许多爬行动物和鸟类都有自己独特的求爱方式。求爱就好像一支优雅的舞蹈,采用一系列吸引异性的爱抚和行为,最终达到性爱目的。尽管求爱这个概念好像有点过时,而且在和色情中描绘的性爱对比时更显老土,但所有关于人类交际和求偶的研究都证明,伴侣间若要长久维持亲密的性关系,关键在于求爱。\\
人类求爱的方式包括:微笑,调情,沟通了解对方,约会,跳舞,吃饭,谈心,牵手,直视对方,搂着对方的肩膀,轻抚对方的膝盖,轻楼着对方的腰,拥抱,接吻。这些行为会让伴侣适应彼此,并为进一步的身体接触作铺垫。爱抚身体极度敏感的地带——比如胸部和生殖器,是人类伴侣求爱的最后阶段。\\
不论是在一段感情的朦胧阶段或是修补过程中,求爱行为都马虎不得。跳过求爱的某些步骤或是草草了事都会影响亲密度。如果个人在求爱过程中发现了伴侣的喜好,那么个人就可以投其所好,取悦双方。\\
要培养亲密感,求爱至关重要。因为它能确保在性爱之前,双方在身体和情感层面上逐渐培养出熟悉感和信任感。这就好像双方确定关系之前先打下感情基础而从朋友开始。\\
求爱行为给个人足够的时间来分析对方的表情、语气、举止、笑声、随意的触碰和其他的非言语交流方式。判断伴侣这些小动作所传达的讯息,能够帮助个人提高性爱交流的质量,不然,他也算不上是一个合格的爱人。\\
求爱行为可以帮助双方相互适应,因为个人往往不知不觉就开始模仿伴侣的行为举止。两人在一起时间越久,双方在动作、思维方式、言谈举止等方面就越相像。一旦双方培养出了这些具有杀伤力的强大情感,那么性趣和性欲就自然会产生,当两人最终结合在一起的时候,双方都能在各方面去适应对方,以享受更加舒适、更加美满的性爱。\\
个人需要深入了解伴侣,投其所好,不要只把注意力局限在性爱里。两人并肩散步时牵着对方的手,晚餐前为伴侣按摩背部,谈话的时候带笑意,直视对方,放一些舒缓的音乐,一起跳一支浪漫的舞。求爱给彼此充分的时间,在发生性爱关系以前,培养出双方之间的默契。求爱目标的重中之重就是要培养信赖感,满足双方的身体和情感需求。\\
“亲密镜像”的练习给读者一个机会。体验怎样协调自己和伴侣的行为来刺激双方的整体感和结合感,以构建亲密性爱的基石。双方轮流主导练习,相互配合,相互鼓励,帮助双方改善日常交流的方式。\\
亲密镜像\\
目标:加强伴侣之间的情感交流,通过同步个人和伴侣的身体动作来增强求爱进度。\\
建议时间5-10分钟\\
两人面对面,距离近至两人可掌心相对。想象伴侣是你的镜中影像。双手相隔5-7公分,用任意方式,如向上,向下,向左,向右或是圆弧运动,慢慢移动双手手掌。伴侣的任务就是把自己当成是你的镜中影像,跟着你的动作走。你要控制动作的节奏,确保件侣跟得上你的节奏。时间到了之后互换角色。这次,由你的伴侣主导,你的目标就是要模仿伴侣的动作。\\
变化:\\
1.可以双手合十进行练习\\
2.轮流模仿对方的表情和动作\\
3.站起身,轮流模仿对方的整个身体动作\\
对女性来说,对性爱的需求,永远不可能甚于对求爱的需求。不论伴侣的年龄阶段、两人相处时间的长短,求爱都必不可少,其方式也需要不断变换,以不断给人带来新鲜感。个人求爱的频率越高,两人和谐性生活的基础也就越稳固。\\
技巧二:和伴侣讨论性问题\\
良好的言语交流是亲密性爱的一个重要组成部分,谈论性可以让双方了解彼此的性爱好,磨合双方的偏好,减小需求的差异,探索新的性活动,解决潜在的性问题。交流可以确定双方是否都得到了满足,是否有一方产生了负面情绪,或是感情受到伤害。调查表明,畅所欲言谈论性爱的夫妻,性生活的满足度最高,这也是情理之中的事。\\
然而,畅谈性并不是一件易事,尤其当色情成了色情瘾者的性爱榜样之时。在色情中,没有人会直率地谈论性爱,也没有人会说:“等等,再给我点时间培养一下情绪”,或是“那样很不舒服,你还是用别的方式爱抚我吧。”没有人会问:“你有没有保护措施?”或者是“你愿意做这个动作吗?”而且在色情中也没有人会批判恶劣行径,如性虐待幼童。色情有强大的误导作用,会讨让人对性问题抱着玩世不恭的态度。\\
要坦率地和伴侣沟通性问题,需要个人有意识的努.力和练习。现在也有很多资源可以帮助个人学习有效的性交流技巧(请参考本书英文版的书尾资料部分)。总体来说,如果两人之间有一个能坦然讨论性爱的氛围,相互尊重,给对方足够的安全感,鼓励对方说出各自的性需求和忧虑,那么两人都能从中受益。\\
为维持一段稳定的关系,伴侣之间可以讨论的性话题有很多。下文列出的问题清单可以作为读者的话引子。\\
能够深人讨论某个问题最好不过,因此笔者建议伴侣们每次选择其中的一两个话题来进行详细讨沦。\\
创建亲密性爱的话题\\
·你最喜欢性爱中的哪个部分?你希望在性爱中体验到怎样的感受?你认为性爱的目的和意义在哪里?\\
·你对自己作为性伴侣的角色有什么看法?色情是如何影响了你的性欲的?\\
·.过去的经历如何影响了你对性的看法?比如说,你是否得过\\
·性传染疾病?你是否被性功能障碍困扰?你是否曾经被性虐待过?\\
·你喜欢何时、何地以及何种方式进行性爱?\\
·你希望伴侣如何进行性爱的前戏?怎样的行为会激起你的欲望?你如何表达想要进行性生活的想法?\\
·讨论身体部位和性活动时,你喜欢用怎样的语言?你偏好用\\
俗语,医学专业术语,还是更喜欢一般的语言?\\
·你想采用何种方式来保护自己,避免怀孕或性传染疾病?\\
·你觉得怎样才能保持自己的安全感和舒适感?比如说:清洁\\
问题,指甲护理,保持私密空间,枕头或者是润滑油。\\
·在性爱之后,怎样的事可以让你继续感受到爱意?\\
·你对隐私、忠贞和未来的性爱关系有什么期望?\\
· 你希望伴侣如何进行性爱的前戏?怎样的行为会激起你的欲望?你如何表达想要进行性生活的想法?\\
·讨论身体部位和性活动时,你喜欢用怎样的语言?你偏好用\\
俗语,医学专业术语,还是更喜欢一般的语言?\\
·你想采用何种方式来保护自己,避免怀孕或性传染疾病?\\
·你觉得怎样才能保持自己的安全感和舒适感?比如说:清洁问题,指甲护理,保持私密空间,枕头或者是润滑油。\\
·在性爱之后,怎样的事可以让你继续感受到爱意?\\
·你对隐私、忠贞和未来的性爱关系有什么期望?\\
请记住,这些问题并没有标准。因为每一个人都是有不同的需求和欲望,而交流的目的就是要更好地了解对方,在不损害价值观,不影响身心安全,不降低个人舒适度,不伤害个人自尊心的前提下,适当磨合,不勉强对方。色情瘾者必须要清楚,色情中的性爱并不现实,而且它对两人人的感情具有相当大的杀伤力。两性之间要综合双方的想法,有创意地解决问题。做一个善于聆听,尊重对方的人,积极找办法,综合双方的需求和欲望。\\
除了讨论重要的性话题之外,通过“大脑风暴”想出来的性生活基本准则也很实用。事先商量好两人都可以接受的行为,减少不必要的猜忌,为积极的性爱生活做好准备。读者可以采用下文提出的准则,这些准则来自于许多接受治疗的伴侣们的推荐:\\
·轻松地询问对方的期盼\\
·即使是在开玩笑的时候,也不能贬低对方\\
·任何一方可以随时喊停某种特定的爱抚或者性爱\\
·可以随时暂停或停止性生活\\
·舒适和安全排第一位、是重中之重\\
·双方共同珍惜亲密感,一起追求欢愉感\\
坦诚地和伴侣沟通性爱的想法会带来意想不到的惊喜。布莱德告诉笔者:“我和妻子终于知道我们想要什么样的性生活了。我们现在的性生活,形式多到爆。宝拉很喜欢尝试新方式,我们俩也更亲近了,两人一起探索性爱方式的想法也比以前多了十几倍。我之前从来不知道她可以这么开放。她告诉我,做爱让她和我更亲近了,也让她敞开了心扉。\\
罗根对妻子坦白说,以前他对自己性表现不够自信,而现在他在性生活时能放得开了。“以前,我总是逃避性生活,怕自己半途不举,早早收场。”他说,“现在,我非常喜欢性爱。我告诉南希,我喜欢性爱,但同时我又觉得焦虑,和她沟通会让我平静下来。我们现在的性生活已经改善了很多,这也给了我很大的自信。”\\
技巧三:增强感官意识\\
如果个人能够体会并且享受感官乐趣,包括视觉、听觉、噢觉、味觉和触觉在内的所有感官愉悦,这会增进性爱中的亲密感。全方位的感官体验是一种动态资源,可以增加性快感。此外,如果个人能够随时协调各种感官,就更能投入性生活,让性生活更具温情。\\
色情让人依赖于视觉刺激。看着色情影像生殖器就会兴奋起来。这就是许多色情瘾者很难在现实性爱中和伴侣培养默契的原因:他们的思绪会自动转移到色情影像上去,没有色情就没有了刺激感。色情充斥了视觉感官,让人难以接受更多的感官体验。好在这种思维模式是可以改变的。\\
通过有意识的改变,个人可以培养技巧,用新的方式来刺激、唤醒视觉和其他感官。以下“感官探索”练习提供的技巧可以帮助你更好地体验感官感受,以及各方面感官带来的特殊快感。这个练习需要在非性爱环境中进行,所以在练习过程中避免任何性念头,放慢步伐,集中注意力,学习如何更好地锻炼、发掘自己的感官。本练习旨在培养个人通过感官产生、表达性欲的能力、接受性刺激等技巧。练习操作简单,无论读者是否单身,都可以进行操作。\\
感官探索\\
目标:增强个人感官意识,测试感官感受体验,发掘自己喜欢的感官,在感官受到刺激时也能放松,心神宁静。\\
建议时间:10-20分钟\\
收集一系列小体积的自然物体,比如石块、贝壳、香料、不同的布料、松果、鲜花和水果。选择自己喜欢的物体,凝视、触摸、嗅、听或是尝,最后将物体放在碗里或桌上。\\
找一个舒适的地方坐下,放松,深呼吸。将注意力集中在自己的呼吸上,持续几分钟时间。如果走神了,就把思绪拉回来,继续专注于自己平稳的呼吸。感觉准备好的时候,拿一个物体,花几分钟时间探索这个物体,近距离观察它,查看它的样式、颜色和材质。然后闭上眼睛,把它放到耳边磨蹭,或是摇晃。它会发出怎样的声音?声音动听吗?接下来,将物体放在鼻子下轻嗅。是麝香味还是香甜味?有没有勾起你的回忆(味道和记忆是相连的)?现在,用物体轻轻磨蹄自己的手臂内侧或是脸倾,感觉是柔软还是坚硬?光滑还是粗糙?让你起了鸡皮疙瘩还是让你觉得痒痒的?如果你喜欢,并且安全的话,可以用舌头品尝物体的味道。享受就好,不需要拘束。\\
探索完第一个物体,可以继续探索其他物体,仔细用感官去探索每一个物体,在整个过程中保持专心,放松心情。检查完毕后,回想自己的经历。哪个物体最让你有感觉?你最依赖的感官是哪一个?怎样用物体刺激感官,会让你有新奇、愉悦的感受?\\
变化:\\
选择另一组不同的物体,可以是日常生活中常接触的事物,比如柔软的T恤、树枝上的树叶或是要吃的食品,重复这个练习。\\
放松身心,用同样好奇的方式探索身体的某个部分,如手、足。把注意力集中到指尖,体会自己触碰的手或足的内部变化,体验被触碰的感觉。和触碰物体不同的是,触碰人体会同时引起两种不同的快感,制造触碰与被触碰的体脸。\\
“感官探索”练习通过感官上的互动,培养个人对事物的新奇感,学会用欣赏的眼光看事物。用这种方式来触越爱人,在和伴侣交流时,能有意识地去探索和欣赏身体不同层面的愉悦感。如果双方能在感官交流中达成默契,他们往往会在身体和精神等层面上更为亲近。\\
技巧四:用新方式看待伴侣\\
眼睛通常被称为体验浪漫的首要器官。凝视爱人会引发盎然的微笑,激发浪漫的爱火。现觉感官也是刺激和维持性欲的重要因索之一。人们常常发现,色情瘾者很难用自然的感官去积极欣赏伴侣。观看色情的习惯会让人习惯于用冷漠的眼光看待他人,这只会让伴侣退却,也无法敞开心扉。依赖于色情获取性刺激的习惯,也会让人产生不现实的性欲望,设定不切实际的性魅力标准,让个人对现实伴侣失去兴趣。要清除这些色情瘾的后遗症,恢复健康的性欲,个人就要用眼睛真正去“看”,增加性亲密度。\\
用爱去看。你如何看待伴侣的身体?你打量伴侣身体的方式,是类似于浏览色情图片,还是带着爱护和尊重?和色情图片不同的是,伴侣会作出反应,她有个人喜恶。如果伴侣喜欢你在性生活时看她的方式,那么做爱时她会更加自在。\\
诚然,要时刻留意自已看待伴侣的方式确实不是一件易事。因为个人已经习惯用看色情的方式来看别人,从中获取性刺激。色情对个人影响够深的话,足以让人不自觉地用看色情的方式来看待伴侣。\\
要用视觉感官来提高性亲密度,就需要你明确伴侣是如何感受你的行为举止,又是如何解读你的凝视的。询问伴侣,当你在不同情况下看着她的时候,她有怎样不同的感受。鼓励伴侣说出内心感受,如果她不喜欢你看她的方式,那就作出改变。请记住,就像你的言语和抚摸一样,和伴侣的眼神交流同样内涵丰富。\\
“爱意凝视”是一个简单的技巧,目的在于训练用眼神来表达爱意。这个技巧非常实用,能够帮助你改变之前看人的方式,让伴侣在你的眼神中找到更多的爱意。当你充满爱意地凝视对方,向对方发出性爱讯息时,伴侣就不会觉得自己沦为了性爱工具。\\
爱意凝视\\
目的:训练利用眼神传递爱意的交流方式\\
建议时间:几分钟\\
把眼睛想象成是心灵的窗户。看着你的伴侣,心中充满爱意和珍惜。让这种感觉从心底弥漫开来,一直渗入眼底,经常微笑,眼神交流,让伴侣看到你眼底的幸福感。不要吝惜,将自己对伴侣的爱意和珍惜全都用言语表达出来。只要是和伴侣一起时,在不同的场合都可以经常重复这个练习。\\
发掘对方的性感地带。如果伴侣认为,在对方的眼中自己是性感的,她在性爱中就会更加开放。如果色情瘾者还是紧抓着以往狭隘的性观念不放,一味追求不符合实际的性魅力和性刺激,那么他就很难觉得伴侣性感。很多伴侣能感觉得到,对方在用色情标准衡量自己,而这只会引起伴侣的负面情绪,破坏亲密度。如果读者已经成功戒除了色情瘾,那么接下来,你就需要学习更加珍爱伴侣的独特之处,为两人创建更坚固的情感纽带,增强性快感。\\
没有哪个人能保持百分百的性感,包括你自己。所以,你应该接受并且欣赏伴侣的性感点,主动发掘伴侣最性感的部位。说出自己最爱伴侣的哪一个特点,比如说,对方亮闪闪的双眸,灿烂的笑容,飘飘的长发或是脖子的优美曲线。在用眼睛看的同时,也要注意其他非视觉方面的特征,包括声音、气息、举止或是肌肤。专注于自己最喜爱的部分,这样一来,其他不那么完美的部分就不会那么显眼了。个人可以告诉伴侣,自己有多么喜欢她的某些特征。\\
改变自己看伴侣的方式可以促进愉悦的性生活,黛比告诉笔者:“我和罗杰现在放得很开,我们会玩很多新花样。有时候,我光着身子走来走去,说:‘你有没有看到我的睡衣?我找不到了。’他抛个媚眼,调戏说:‘没见过,不过我觉得你现在穿的这套就不错。’现在他看我的时候,我很自在,因为我觉得他重视我,尊敬我。”\\
如果你和伴侣珍惜彼此,能够接受对方的一切,那么两人在性交流的过程中会更有自信,更富有表现力。\\
技巧五:增加触感词汇\\
触摸是性爱的真正语言。从最初的温柔指尖触碰,到最后的结合,触摸是伴侣双方建立亲密纽带、体验性快感的主要手段。如果能用触摸来表达不同的情感,传达各种信息,将会为性体验增色不少。\\
学会表达自己的性观点,增强自己“看”伴侣身体的能力非常重要,而用多种方式触摸也是极其重要的一课。触摸的方式,触摸的部位,触摸的时机,都会向伴侣传达你的心情和欲望。\\
触摸交流并不是那么容易就能理解的。人们原本希望用手或手指表达的含义,传达到伴侣那里就可能变了味。如果沟通不足,两人确实很容易误解对方的信息,错误估计对方想要的性行为。毋庸置疑,很多伴侣们在性方面的分歧,都来自于双方对触摸行为的误解。下面的触摸练习可以帮助伴侣扩充触摸词库,更清楚地用触摸来表达情感,传达信息。\\
辨别触摸类型\\
目的:增强自己的触摸交流能力,分析对方触摸自己时传达的信息\\
工具:两支铅笔,两张纸\\
建议时间:20一30分钟\\
让伴侣背对着你,在纸上写下以下四种触摸方式(可打乱顺序):“调皮式”、“治疗式”、“爱意式”和“激情式”。完成后,用一只手或者双手,根据单子上的第一种方式触摸伴侣的后背。如果选择的是“调皮式”,你可以用一种轻快的、挠痒的方式移动手指,在伴侣的后背上用手指虚拟弹钢琴或是画画。结束以后,用手掌扫过伴侣的背部,让伴侣猜测自己的触摸属于哪一种类型,并将答案写在纸上。如果伴侣不能确定的话,你可以重复抚摸,直到伴侣写下答案。\\
接下采用纸上的第二种触摸方式,进行下一轮触摸。继续练习,直到伴侣将四个答案全都写下来为止。\\
结束以后,两人面对面,核对答案。如果伴侣猜错了某种类型的触摸,请教伴侣自己如何改变触摸方式来更准确地传递信息。用新方式触摸伴侣的背部并询问伴侣新的触摸方式是否有所改善。\\
接下来,交换角色,让伴侣触摸你的背部来传达四种不同的触摸方式。请记住,每种类型相对应的触摸方式并没有固定模式。这个练习的目的在于增加伴侣间的触摸词汇,更好地了解双方触摸的含义,统一交流方式。\\
变化:\\
1.用不同的触摸方式进行练习,比如:’‘调情式”、“温情式”、“性感式”和“冒险式”。\\
2.用不同的方式,触摸伴侣身体的其他部分,比如头、脸部、手、足、胸膛、小腹和生殖器部位。\\
3.脱下衣服进行这个练习。\\
不管是不是在床上,一个颇为丰富的触摸词库总是能够促进两人之间的亲密交流。做爱时,你可以采用两人都喜欢的触摸方式来爱抚时方。\\
技巧六:探萦感宫愉悦\\
对每一个成年人,尤其是曾经沉溺于色情的人来说,要学习在性生活时不以刺激生殖器和达到高潮为目的,而是要相互给予、交流性快感是很重要的。和伴侣的性生活,可以促进双方在身体上的满足感,也能让伴侣感受到你的爱意,鼓励两人更开放地尝试新方式。\\
此处介绍的“感官愉悦”练习提供一种循序渐进的方法,旨在促进双方快感。放慢步骤,不以性结合为首要目的,相互爱抚,用足够的时间尝试各种刺激的办法,更多的了解对方。\\
受访的伴侣们认为这个练习教会他们如何调整亲密技巧,包括和对方互动,给对方提示,根据对方的要求改变触摸的方式,放下所有防备,充满爱意的看着对方。真实性结合不在这个练习的范围以内,所以伴侣们在练习时可以随心所欲,跟着感觉走就好。感官上的愉悦会让人自然地放松,不会感到无聊,也不会让人想到色情,这种自然的练习可以让双方都感到愉悦。\\
“感官愉悦”练习可以让人习得许多技巧,以提升自已的性活动能力。请注意,练习这些技巧时要避免性活动,至少在练习初期尽量避免。读者和伴侣就事先达成共识,这个练习的目的并不是性结合。\\
感官愉悦\\
目的:温柔地探索对方的身体,了解对方喜欢的触摸方式,扩大体验快感的范围。\\
建议时间:30分钟或更久\\
根据自身的舒适程度着装,可以穿宽松的衣服或者只穿内衣,或者裸体。裸体更方便直接触碰,但如果你担心这样会引起性冲动,可以穿着衣服。你也可以用其他方式来放松,比如说练习前洗澡,确保室内温度适宜,修剪指甲,身体涂抹乳液,确保两人不会在练习中途受到干扰。\\
让伴侣在柔软的床上躺下,可以是背部着床,也可以是腹部着床。你的任务就是从头到脚爱抚伴侣。开始时,探索伴侣那些一直以来被你忽略的身体部分,暂时不要触摸脚部和生殖器。温柔地触摸对方的身体。感官上充分熟悉每一处后,再继续触摸其他部分。注意伴侣不同身体部分的触感,毛发、光滑、坚硬、柔软。你最喜欢触摸哪个部位?\\
用不同的方式触摸伴侣的身体来表达不同的情感,例如调皮,激情,充满爱意。感受自己闭眼时的体验和睁眼时有何不同。用心体验触摸时的手感,以及其他感官受到的刺激——听到的,看到的,尝到的,闻到的。鼓励伴侣说出她最喜欢的方式和最不喜欢的方式。询问伴侣怎样改变会让你的触摸变得更加舒服,然后根据她的回答调整方式。\\
完成以后,让伴侣翻个身,你继续触摸伴侣的另一半身体。当你和伴侣完成触摸之后,你们可以相拥,或者坐起身来交谈,讨论各自从这个练习中享受到的、学到的内容。\\
接下来,两人转换角色,让伴侣来探索你的身体。当伴侣触摸你时,平稳地深呼吸,保持放松。放松胸膛和腹部,增强性体验。身体被触摸时,请保持清醒,用心体验不同的感官感受。把注意力集中到被伴侣触摸的部位,捕捉你最喜欢的方式,然后把内心感受说出来,让伴侣知道,她应该怎样改变自己的触摸方式会让你更加舒适。同时,也要听伴侣说出她的内心感受。\\
让伴侣触摸自己的身体会令人热血沸腾,伴侣的触摸是随意的,无法预测。享受那种未知感。随时准备接受惊喜。如果练习中途萌发了色情念头,马上告诉伴侣。然后做出相应的调整,减少触发的可能性。当你又放松下来,能够适应触摸方式时,练习可以继续进行。\\
变化:\\
1.你和伴侣可以同时相互触摸时方的身体:,两人可以拥抱,或并肩躺着。你可以尝试模仿游戏,一方以特定的方式触摸时方特定的身体部位,另一方进行模仿,以相同的方式触摸叶方的身体。平稳呼吸,并注意观察伴侣的呼吸。深情地凝视时方,给对方一个微笑,让自己的行为和伴侣相协调。这种由相互爱抚而产生的快感,可以加深双方之间的结合感。\\
2.在练习中加入触摸胸部和生殖器的步骤。触摸时,请确保你和伴侣将这种行为当作是平常的触摸方式,触摸目的保持不变。再次申明、要放松,平稳地深呼吸,用心体会触摸和被触摸的愉悦感。胸部乳头、生殖器和腹股沟等身体部分的肌肤非常娇嫩,触摸时必须轻柔\\
在练习过程中,产生性欲是正常的。亲密接触会引起勃起和其他生殖器充血等正常反应,你要学会把这视为正常性体验的一部分。\\
要真心体验并且熟练进行全身的感官交流,需要反复练习。读者可以每隔几周定期练习一次“感官愉悦”。通过这个练习习得的高超触摸和交流技巧,将会丰富你的性经历,让你能够更熟练、更有技巧地触摸对方的身体。你能更开放地和伴侣交流,享受伴侣触摸时带来的独特感和爱意。\\
如果个人能将这些技巧,融入自己的色情瘾治疗过程,这足以唤醒自身的性意识。增强性体验。个人可以将这些促进感官愉悦的技巧作为前戏和性爱的一部分,让自己全身心地投入性爱体验中去。贾汀改变了以往触摸伴侣和性生活的方式,受益匪浅。“我有色情瘾的时候,做爱非常机械。我从来没有注意我在做什么,感觉有什么不同。现在,我在探索自己喜欢的感官体验,看什么方式能给我带来快感。”\\
技巧七:用心傲爱\\
爱意会为性结合增添浓墨重彩的一笔。如果对彼此有着浓浓的爱意,再结合性欲望和性刺激,双方可以得到前所未有的性快感。可悲的是,被色情毒害了的人很难在产生性欲时保持爱意。色情发挥作用的方式是一种直接的刺激——反应方式,即色情影像(刺激)会直接引起生殖器兴奋(反应),这个过程否定了爱意的存在。一名男性和笔者分享说;“色情就是生殖器交流,和爱没有半毛钱关系。”\\
要学习如何将心中的爱意和生殖器兴奋联系起来的过程包括两个步骤。首先,个人必须要建立爱意和生殖器之间的关联;其次,和伴侣一起时,个人要想办法提醒自己这个关联的存在,用心去感受。\\
接下来的“传递心声”练习可以帮助你同时实现这两个目标。这是一个操作相对简单的练习,旨在加强爱意和生殖器感官之间的积极联系。练习前,放松身心,可以独自一人,也可以让伴侣共同参与。此练习需要你触摸自己的生殖器,而不需要激烈的性刺激活动或是达到高潮。之后,当你和伴侣进行性生活时,可以选择练习“传递心声”中的“变化”部分的技巧,来唤醒并增强性爱过程中的爱意。\\
传递心声\\
目的:培养并体验爱意和生殖器之间的联系\\
建议时间:5-10分钟\\
穿着舒适的衣服,在私人空间,如卧室,放松地坐下或躺下。将一只手放在心脏上方。深呼吸、放松,感受每次呼吸时手掌随着胸膛的起伏而起伏。呼吸时,尽量放松胸膛和小腹。在心中偷悦地接纳心脏的稳定跳动,感激心脏给予维持生命的能量。个人可以自由选择闭眼或是睁眼,看何种方式可以让自己更放松,心神宁静。\\
将手心贴在心脏上方,凝神体会自己的感受,培养自爱的情绪。回忆过去自己的成就,自己最优秀的特质。如果你有伴侣,也可以以伴侣为思考对象,分析你深爱她,敬仰她的原因。\\
将一只手放在心脏上方的同时,另一只手轻柔地放在或者靠近生殖器。保持深呼吸,放松身心,用心体会手心下心脏和生殖器之间的微妙联系。通过体验同时触摸胸口和生殖器的感觉,体会这两个部分的感官联系。用心体会生殖器给生命带来的积极因素。呼吸的同时,将意识轮流锁定在心脏和生殖器上。如果你的思绪飘远了,拉回思绪,放松心情,深呼吸。维持这个姿势几分钟即可,如果你觉得舒适,也可以长时间保持这个动作。当你准备停止的时候,挪开双手,深呼吸几分钟,回味刚才的体验。\\
变化:\\
1.裸体进行此练习。\\
2.如果伴侣同意,可以尝试双人练习。练习过程中凝视着时方,保持徽笑。做好准备后,将放在自己生殖器上的手移到伴侣的生殖器部位。继续深呼吸,放松身心。有必要的话,放在心脏上方的手掌可以时常轻拍、轻抚胸腔,心中要保持对自己以及伴侣的爱意。\\
3.在任何性活动以前,都可以进行这个传达心意的练习。\\
4.当你们正在进行性生活时,花一点时间,带着浓浓的爱意,抚摸自己或伴侣的心脏胸膛。\\
在性爱过程中表达爱意的方式多种多样。你可以尝试以下方式,包括本章介绍的技巧:\\
·花点时间对着伴侣微笑,凝视对方的双眸。\\
·将注意力从自己兴奋的生殖器暂时转移到自己最喜欢、最敬佩伴侣的某点特制上。\\
·花点时间用言语表达自己对伴侣的爱慕之情。\\
·用已经学会的触摸方式来表达自己的深情,这会让伴侣开心不已。\\
·在性爱中表达自己的爱意十分重要,请记住,性爱是互动的爱意交流,伴侣也会对你表达敬仰之情和浓浓爱意。体会伴侣温柔的话语、触摸、动作和表情,想象所有的这些回馈已经深深扎根在你心底。让自己感受和真正爱自己的人在一起的满足感。你可以要求伴侣在性生活时,将她的手时不时放在你心脏上方,作为一种非言语的表达爱意方式,提醒自己要珍惜伴侣的爱意。\\
性爱后两人腻在一起的时.光也是亲密的最佳时机。性爱的高温渐渐褪去后,用充满爱意的方式继续交流,相互依偎,说说甜蜜的情话、轮流听对方的心跳、一起来个鸳鸯浴、在对方的臂膀里入眠……性爱后的亲密交流,如触摸,交谈,享受美好时光,可以开启新一轮的求爱过程,成为下一次J性生活的前戏。\\
个人也可以在性生活结束后反思经历过的亲密接触,心存感激。回想自已和伴侣结合过程中得到的快感,重组自己的性幻想,用和伴侣一起创造的美好性爱记忆来取代色情影像,作为个人性欲和性刺激的来源。\\
以亲密为导向的性爱需要读者调动身体,感官和思想,最重要的是你的真心,再加上和伴侣一起进行的练习。本章介绍的七个技巧能帮助你和伴侣共同建立一段更加稳固、更加持久、更加令人满意的性关系。这种健康的性爱方式,不仅可以治愈色情瘾者的伤痕,也可以抚慰伴侣的痛苦。因为之前强尼沉溺于色情,导致他和妻子凯伦的性生活曾经一度充满了失望,而现在,强尼不仅停止观看色情,更把重心从“看色情”转移到了“做爱”,这让凯伦终于放了心。她告诉笔者:“现在色情对我们的感情没有任影响了,我们性生活的方式也已经完全改变了。我现在很信任强尼,跟他做爱的时候也更加开放。我不需要当刺猬来保护自己。现在,我在他面前很开放,自然而然地就想要去满足他。”\\
重新学习如何成为一个合格的性伴侣,和对方建立起深厚的纽带。会给人带来身心上的双重回报,为个人带来深层次的满足感。乔治20岁出头时就有了看色情的习惯,他说过一句很经典的话:“我以前的目标就是要找一段爆、劲爆、再劲爆的性爱,那就是色情的精髓。但是,我已经发现了,性的内涵其实非常丰富,还包括亲昵的调情,淘气的撒娇,情意绵绵,温柔体贴等。我已经56岁了,但我现在的性生活比以前年轻气盛的时候还要丰富。我过去以为,性生活的频率越高。我和伴侣就会更加亲近,但结果正好相反。身体上的接触并不代表情感上的亲密,有时候效果恰恰相反。美满的性生活需要双方的真诚和爱意。”\\
第十二章 真正的自由和满足感\\
确实,治对的时候我受了很多煎熬,但我最终还是成功了。我现在是一个正常人,我的第二春终于来了——春天来了,我的世界春意盎然,阳光明媚。 ——比尔\\
要到达成功的彼岸困难重重,但只有跳出色情陷阱,生命才算得上是真正的美好。受访的每一位色情瘾康复者都迫不及待地想要和笔者分享,他们现在的自我感觉更好了,人际关系得到了改善,性欲更加健康,人生的未来也更加明朗。他们拥有了前所未有的幸福感和满足感,他们成功的故事,都闪耀着热情和自信。\\
不论读者现在是处在治疗色情瘾的中途,或是考虑要不要戒除瘾,或是刚刚迈出了治疗的第一步,或是在和复发作斗争,抑或是在戒除瘾道路上一路通畅,笔者都希望通过本章介绍色情瘾康复者的经历,给读者带来灵感,为前程照亮时曙光。下文将要介绍,前文中提及的色情瘾者在成功摆脱色情瘾以后,生活变得更加自由,人生变得更加满足的经历。\\
读者会发现,每个人的戒瘾途径各不相同,但大部分色情瘾康复者都证明,在戒瘾之后,他们的人生态度更为积极,而生活也更轻松了,自己的情绪变得稳定。也更喜欢在现实生活中和人交往了,因为他们已经学会如何尊重他人,在为人处世时体贴关怀,保持和他人的顺畅沟通。这点对伴侣来说极为有利,而重新建立起来的信任感。畅通无阻的沟通渠道加强了两人之间的纽带。总之,只要个人可以从色情陷阱中跳出来,那么生活的每一个方面,包括健康、性爱、情感和人际关系,都会得到改善。\\
当然,戒瘾有实实在在的好处,但有时候也会留给人一点点遗憾。戒瘾需要人们放弃一些可以带来快感的事物。尽管有部分受访者会谈到自己的失落感,但是他们很快就转而强调,现在无色情生活的美好。比如说,劳拉承认她挺怀念色悄这个老朋友,她告诉笔者:“我在戒瘾之前,已经看了30多年的色情。那些幻想就好像我的闺蜜,我难过的时候逗我开心。但是,同样是那些幻想,诱导我去尝试那些危险的性经历,害得我半死不活。我现在坚决抵制这些诱惑,活得清清白白!”伊森说:“我怀念色情,跟怀念磕药一样。但是,我现在和一个很好的女人在一起,她又诚实又温柔。我真正在乎的是她。如果不是当初戒掉了色情,我不可能会有今天。”\\
读者可以把本常分享的故事看成是未来的生活蓝图,前提是读者能够坚持不懈地按照笔者在本书中介绍的方法,坚持戒瘾。读者将会发现,跳出色情陷阱可以怎样调整人生方向,为成功打下基础,再创人生新高。这是关于自由和勇气的故事。那些色情康复瘾者做到了,你也可以的。\\
坚持时间越久,康复就越容易\\
在色情瘾治疗的过程中,最有盼头的一件事情就是,成瘾最终会变得比较轻松。比尔告诉笔者,他参加会议、见咨询师已经几年了,现在的他很自信,相信自己一定可以远离色情,亨受美好生活。“治疗的前六个月最艰辛。事实证明,我之前的想法太天真了。我太轻敌,觉得它没什么大不了,不会伤害到谁,色情不过就是个随便玩玩的游戏,这些想法部错得离谱。能够认识到这些,我的康复也就更加顺利了,也不会老想着要看色情。”\\
玛丽的经历也是如此,她过了整整5年的时间来戒瘾,她说:“我戒瘾很成功。跟牧师在我电脑上发现色情的那天比,现在的我改变了很多。集体活动和咨询教我勇于直视自己的问题,心情不好的时候也要学会放下。我现在可以很自信地说,色情已经影响不了我了。”\\
通过长久的坚持,许多色情瘾康复者能成功地将色情清扫出门。当色情意外出现时,他们也能坚守承诺。坚决抗拒色情。色情再也不能诱使他们偏离正轨了,只因为他们可以坚们自己的价值观和人生目标不动摇。科里自豪地告诉笔者,他最近顺利控制了一个潜在的复发局势。“我上网去找一个运作系统的工具,一个链接把我带到了一个色情新闻小组的档案馆网站.,当时一个念头一闪而过:嗨 ,点一下就好了,就在那里。但是我没有点击。我只是笑了笑,冷静地把网页关了。经验告诉我,不管色情会暂时带来多少快感,它根本就不能在现实中满足我。”\\
如果个人在康复过程中培养了敏锐的洞察力,认识到色情的害处,绝不可以再犯,那么康复之路就会十分坦荡。“我最近发现了,色情还真不是我的菜,”罗根说,“要说到喝酒,有些人随便喝几口就能打住,但是我从来不知道哪个人随便看几眼色情就打住的。反正我做不到。就好像大家说的:‘一天酒鬼,一辈子洒鬼’,我觉得色情瘾也是样。我就是一个血淋淋的例子。如果我想要健康地活着,那我就不能靠近色情。”当色情瘾者最终下定决心,必须在余生中都要杜绝色情时,他们的渴望就已经减弱了。就像罗根所说:“小姑娘我最终下定决心,绝对不会再看色情之后,我就再也不会有特别强烈的欲望了。”\\
自由就是每天的选择\\
诚然,这个世界充斥着各类色情信息和类色情彩像,要完全避免接触并不现实。因此,每一位色情瘾康复者都会发现,自己不断重温自己的承诺,才能远离色情。这是一个需要你坚守终身的承诺,如果你能把色情清扫出生活,这也会给你极大的成就感。兰迪下决心要一辈子戒除色情之后,他的内心也变得更加强大。“我以前被色情坑了,好像被活活绑架了一样。我现在知道现实的性欲是什么,真正的激情是什么。现在我好后悔以前花了那么多时间在这件事情上,这非但没给我带来什么好处,还害得我差点跳不出来。但是,现在我已经得新掌握了自己的生活,成为了主导。做决定的是我,而不是什么色情,这让我感觉自己很强大。虽然色情是强劲对手,但是我每天都在打胜仗。”\\
埃里克斯也说他经常会受到色情的诱惑,他一直遵守诺言远离色情,他为此无比自豪。他对笔者说:“以前我看色情的时候,一直把这个秘密憋在心里。我以前总是觉得自己太罪过,弄得自己不敢和别人打交道,放不开手脚。现在,我每次成功抗拒色情诱惑的时候,都会感到一种真正的自由,好像心灵有了归属。我坚守诺言,大大方方,清清白白。当然了,其实每次要抗拒色情都不容易,但是我每次成功之后,心里都会感觉特别舒坦。”\\
个人回报令人满意\\
如果你决定要一辈子远离色情,那么生活定会得到改善。汉克告诉笔者:“我打小就开始看色情,直到40多岁人生跌到谷底,这30年里,我一点都不觉得满足,总是闷闷不乐,自己也没有真正成熟起来。3年前,我戒了瘾之后,生命就开始改变了。现在,我终于觉得自己是一个成熟的男人,我的生活很美满,也够刺激。现在我能理性地表达自己的感受,做回真正的我。”\\
在戒瘾的过程中,个人的内心会不断强大。治疗会帮助个人理清情绪,增加抗压能力,不盲目追求快感。个人也能更好地了解自己,了解生命之重,清晰人生目标。如此,个人便不会臣服于非理性的冲动,而是自尊自爱。肯定生命价值,抵抗负面情绪。“现在,我会把精力放到重要的事情上面去,”比尔说。”现在我更加警惕了,专心去享受当下的生活,不受幻想影响,不幼稚,不老是生闷气,觉得畏惧,或是跟刺猬一样防备。色情就是前进道路上的毒瘤,我喜欢的是有效率的生活。\\
放弃色情的最大回报之一就是,个人的人际关系会得到改善。前文提及,色情观看者往往喜欢独处,社交活动不活跃,待人不诚恳。遵循康复步骤可以培养个人对外求助的能力,对人更为宽容,和和命中重要的人更为亲密,改善和伴侣、家庭成员、朋友和同事的关系。“我现在状态不错,和别人的交流也很顺畅,”乔治兴奋地告诉笔者,“’我可以用爱去欣赏女性,珍惜她的心意,把她当作是一个独特的个体来尊重。戒了色情以后,我变成了理想中的男人。”\\
尽管罗波花了多年的时间才彻底改变,他很庆幸自己戒瘾成功,他不需要自欺欺人,也不需要遮遮掩掩。“我看网络色情的时候被逮了个正着,我丢了饭碗,妻子和孩子也离开了我,”他说,“戒除色情并不轻松,但只有这样我的生活才能走上正轨。现在,我人生中第一次可以和别人真心的交流,再也不用戴着面具生活了,也不需要对别人撒谎。我又是一个完整的,健全的人了。没有了色情,生活好了太多,之前我千方百计想要在色情里寻找的东西才一文不值呢。”\\
培养人际交往的能力,会有无穷的额外好处,其中之一就是能够加强个人的自尊和自信。麦切说:“我重生了,我不会再纠结了,生命跟我的道德价值观和精神信仰相符。我现在是一个很坦诚的人。以前,我说话之前要先在心里把真话掂量几遍,现在我不需要遮遮掩掩了。我的婚姻也改善了,性生活很和谐,和伴侣的沟通非常顺畅。我现在的生活有原则,很完整,生活在这个自由世里时,比在肤浅的色情世界里面仿徨好得多。”\\
尼克与许多色情瘾康复者一样,觉得自己的性生活得到了改善。“现在我和妻子在床上很投入,我再也不会有什么色情幻想了。现在性生活的时候,我不会觉得羞耻,不会有负罪感,得到的快感很持久,很满足。我在生命中第一次感到自己是完整的,在性方面也很健康。”\\
汉克总结了远离色情的诸多好处。“戒除了色情,现在的我才算是一个完整的人。我不会胡思乱想,不会刻意疏远别人。现在我是一个成熟的男人,不会再把别人当作性工具。我和伴侣相处融洽了很多,我以前真是自活了。远离色情,是天赐的良机,这唤醒了我的个人意识,促进了我的独立人格。”\\
用自己成功的事例来激励他人\\
经受了色情的各种折磨,许多色情瘾康复者都希望他人可以引以为戒,避免重蹈覆辙。只要自己的治疗上了正轨,他们通常就会利用自己的前车之鉴,宣传色情危害,为那些尚在康复初期的人群提供帮助。\\
维克结束了为期5年的嗜性匿名者互诫协会(Sex Addicts Anonymous)治疗之后,开始为小组中的新成员举办励志性演讲。“想当初,多亏了咨询师指点迷津,他告诉我,我已经没法控制自己的色情瘾了,我的生活已经失控了,是他鼓励我勇敢地走出了第一步。现在,我要发挥自己的积极主观能动性,把精力放到治疗项目中去,这不仅加强了我的精神信仰,同时也能为别人带去一点好处,为世界做点小小的贡献,我觉得这样子做非常值得。”\\
尼克戒瘾已经7年了,最近的一次复发还是在6年前,但是他一直积极地参加教堂信仰康复小组会议。他说:“我们小组里面有些人被色情吃得死死的,我给予他们无限的同情。既然我自己己经康复得差不多了,我希望自己能够带领队友一起远离色情。我和他们分享自己的经历,希望能引发他们的思考:怎么样采取具体的措施抵抗色情,怎么样克服性生活中的羞耻感。在治疗小组中做个领头人,证明了我的能力。我能为小组做出贡献,也能被大众所认可,这样我就非常满足了。”\\
汤姆无法忘怀,当他沉溺于色情瘾时,自己的独孤无助。现在。他会有意识地关注身边人的行为,捕捉他人行为中的蛛丝马迹来判断他们是否在和色情作斗争,需要时提供及时的帮助。“我还挺大方的,不会不好意思和朋友讨论性,”他说,“如果我觉得朋友情绪压抑,性格孤僻,我就会婉转地问他是不是有色情瘾和强迫性自慰问题。我改变了观念,不会为讨论性感到羞耻,我只想打破色情瘾的恶性循环,尽力帮助别人,因为这是我困在色情里的时候最希望得到的帮助,现在我会去这样帮助他人。”\\
部分色情瘾康复者会用各种方式来伸出援手。他们支持相关组织,为有性瘾或被其他色情相关问题所困扰的人群提供服务:,这些热心的资助者出钱出力,做志愿者,帮助管理、运营治疗项目和资源,如免费热线、网站、全国性治疗服务协会、信仰服务和当地的十二步治疗项目。他们的默默付出,.为那些希望戒除色情瘾的群体建立起了完善的服务网络。\\
色情瘾康复者想要出力,途径还有很多,方式之一就是加入预防色情问题的工作中来。他们普遍赞成一句活:“点滴预防胜过灵丹妙药。”比如说,科里告诉笔者,他有个想法,希望各大媒体和网站打出公益广告,让人们了解色情的危险。他解释说:“想象一下,如果所有的色情网站都和香烟盒子或药品盒子一样,温馨提示色情的负面影响,那会有多大的影响力。色情问题不是单纯的个人问题,而是社会和文化问题。整个社会需要更加关注这个问题,做到能公开地理性讨论色情的影响。消除内心的羞耻感,大胆地说出网络色情的影响:它会怎样让你在阴沟里翻船。我一辈子最多花了10美元买色情杂志。然后却蹲了监狱。我真不希望别人走上我的老路。”\\
艾德在自家社区的服务小组和教室中授课,宣讲色情瘾瘾可能带来的负面后果。“看色情,害人害已。现在,我积极教育大众,.希望公众打破沉默,也希望由此来实现自我价值。和别人沟通,不仅可以时刻提醒自己要继续远离色情,也能让我帮助到那些受到色情伤害的人。”\\
如今,劳拉积极地想要打破公众对女性色情问题的沉默。“色情伤我很深,但是我还是找到了康复的办法,现在我要尽其所能,帮助其他女性,”她说,“我建议那些沉溺于色情的年轻女孩子:千万不要看!看色情很容易上瘾。可能你刚开始以为自己只是玩玩而已,忽然某天你就会发现,看色情己经全面影响到了自己的身体、精神和情感健康。”\\
保护孩子\\
许多色情瘾康复者都非常关注对孩子的保护,希望让孩子能远离色情的影响。他们清楚,现在的孩子年幼时就能接触到色情,如果缺乏正确的指导,孩子们很容易产生色情瘾问题。成功跳出色情陷阱之后,将精力转移到对孩童的保护上,以免孩子受到色情的彩响,这可以说非常具有现实意义。\\
劳拉现在致力于组织活动,提醒孩子们远离色情。她告诉笔者:“一想到现在很多无辜的少男少女在受到色情的荼毒,我就觉得难受。我希望公众可以意识到,色情的毒害会扭曲孩子们的性观念,阻碍孩子们形成健全的性意识我们必须马上行动起来,不然现在的孩子们有很大的风险。我们要保护孩子,不让他们接触色情;教育他们,让他们了解人体是美丽的、神圣的,尊重别人,美好的性爱只是真挚爱恋的产物。孩子们有权利知道,色情和毒品一样,会让人感觉很刺激,但实际上却是致命的,它根本不能让人觉得满足。”\\
许多色情瘾康复者都认识到,如今的文化氛围中色情无处不在,无辜的孩子们很容易成为受害者。杰克认为,家长不教育孩子有关色情的危害,就是不称职。“现在的孩子日子不好过,”杰克说,“他们生活的这个社会既‘鼓励’色情,又‘谴责’色情。将来我有了孩子,我就一定会告诉他们色情的危害。不让他们接触色情,那是不现实的。如果我的父母早早提醒我,色情有误导作用,会让人误解女人的身体构造,让人产生厌女症,学习到错误的性知识,那我早该知道,色情不现实,描绘的也不是真实状况,里面对人的方式都不对。”\\
如果读者身为人父/母,那么在自我康复的过程中,另一个十二分关键的任务就是要保护孩子,不让他们接触到色情,避免色情瘾。尽你所能,不让孩子们接触色情,同时,和孩子们培养亲密的关系,和孩子们讨论色情可能带来的后果,并教育孩子如何培养健康性观念,维持爱恋关系。如此双管齐下,可以确保色情问题不会变成遗传问题。请参见以下清单:\\
如何减少孩子染上色情瘾的可能性\\
以下清单列出的措施,可以帮助保护孩子免受色情的负面影响。笔者建议读者根据孩子的年龄阶段,适当地调整教育方式。\\
_维持无色情的家庭环境。\\
_和孩子培养亲密的关系,给予孩子足够的关爱。\\
_定期和孩子沟通,了解孩子的忧虑,帮助孩子解决生活中的困难。\\
_在家庭中树立健康的性观念,确定合理的性界限。\\
_宽容对待孩子对性的好奇和欲望。\\
_鼓励孩子看到了性暴露内容后和你交流。\\
_理性地回应孩子对色情的问题和忧虑,发现孩子接触色情时也要理性地处理。\\
_教育孩子,色情中描绘的性爱是不正确的,有误导性。\\
_让孩子了解色情瘾的真面目和其他负面影响。\\
_公开谈论当今社会中普遍存在的性问题,例如性虐待、性瘾、性传染疾病和意外怀孕等问题。\\
_如有需要,帮助孩子利用共同资源,如参加社区性教育和性咨询活动。\\
如果孩子无意间接触到了色情或者想要主动接触色情,那么家长可以教导孩子。帮助他们避免对色情产生依赖性。一位父亲告诉笔者,他发现青春期的儿子半夜在网络上看色情图片。“其实我当时可以对他大吼,让他滚去睡觉,但我把这当作一个教育的机会,”他说“我坐下来,好好告诉儿子,色情从哪里来,谁把色情放在网上,解释他们这样做的原因。我跟儿子说,色情就是要让人兴奋,让人上瘾,然后乘机掏空你的钱袋。我要儿子了解真相,他就可以理性地作出判断,了解幻想和现实的区别。我还问他,他的朋友里面,有哪个人的行为和色情中一样。我们还讨论了什么叫做色情瘾,我告诉儿子,色情瘾就好像吸毒一样,会消磨人的自尊。我还告诉他,看色情不能帮他追女孩子,然后,我们达成共识,一致认为现实和点击鼠标看屏幕是不同的。”\\
发现孩子在看色情的场景确实尴尬,家长往往容易情绪失控,但请用宽容的心态来解决这个问题,让孩子了解足够的信息。如果家长能给孩子灌输健康的性观念,这有利一孩子的长远发展,让孩子在当今这个充斥着色情的世界中健康成长。\\
庆祝真正的自由和满足感\\
本书主要讨论充斥着色情的生命之痛。对许多人来说、刚开始刺激火辣的习惯,最终将会扼杀生命中的一切美好,打击个人自尊心,影响个人性欲,破坏恋爱关系,影响个人与家庭成员和朋友的关系,最终影响个人生活,剥夺生命的自由。要远离色情很艰难,也很漫长,这个过程中,个人会感到困惑,思想矛盾,或许会引发复发。\\
但是,本书中记叙的那些色情瘾康复者事例证明书,戒除色情瘾给生命带来最重要的奖励就是:自由,这远胜于色情给予的虚拟快感。这里所说的自由并不是指字面上的自由,即人可以随时随地做自己想做的事;真正意义上的自由是指能够作出明智的决定,让自己拥有最健康、最幸福的生活。如果生命曾被色情这样强大的事物控制过,你就能够理解这种自由的宝贵。\\
真正意义上的自由促使人培养坦诚的人生态度,给生命带来生机,让生活符合个人的人生目标、使命、价值观,帮助个人追求梦想。这种意义上的自由才能让人构建健康的人际关系,让人体验爱情,尊重他人,培养自尊自爱。而且,这种自由可以调整性欲,让人在性生活中和伴侣保持良好的互动,感到爱和被爱,肯定自身价值。\\
布茱德戒瘾之后,理解了真正的自由他告诉笔者:‘“之前我没有意识到这一点,就沦为了色情的奴隶,被洗脑到不知道什么叫自由。现在,我能对色情说不,我能自由地选择生活方式。”麦切也意识到“奴隶”和拥有“自由”两者之间的差别:“我的自我感觉好了很多。我觉得自己很幸运,也进步了很多。以前看色情的时候,自己是被控制的。如果色情再来诱惑我,我能自由地做出选择。现在我是自由的个体,我能从灵魂的最深处表达我自己。”\\
要找到一条跳出色情陷阱的路并不容易。但是我们希望打破长期以来围绕这个话题的沉寂,让公众知情,因为这个问题已经严重威胁着许多社会大众,而这波及的不仅仅是色情观看者自己,还影响到了他们关爱的人,关爱他们的人。公众可以即刻行动解决这些问题。改变就在当下。我们有充分的理由相信,本书介绍的方法将帮助你成功戒除色情瘾,抚慰自己的情感伤痕,修复受损的人际关系,改善自己的人生。如果你可以在康复道路上坚持到底,你就会发现回报是可喜的,值得你的一切付出。\\
在跳出色情陷阱的旅程中,色情瘾者从无知到精通,从回避到行动,从欺骗到诚恳,从羞愧到正直,从自私到爱与被爱。木书介绍的实例表明,即使你被色情所伤至深,你还是可以重获新生,治愈伤痛,从色情的枷锁中永远地解脱出来。\\
译者的话\\
《跳出色情胳阱》一书是由美国著名性学专家温迪·马尔和茨拉瑞·马尔茨共著。我与此书的缘分始于二O一一年夏,正值我开始浏览“戒邪淫网”以及“中国反色情网”这两家国内最具影响力的反色情网站,初步了解到色情对中国社会尤其是青年群体的影响之大。受此启发,我自愿加入到网络志愿者组织“华文爱心翻译义工之家”。成为了一名英语翻译志愿者,参与翻译治疗性瘾及色情瘾的英文资料。在此期间,我有幸拜读并翻译了此书。\\
介于性瘾以及色情瘾涉及到的是心理学领域,要在现实生活中给这个群体做出统一的书面定义并不容易,而性学领域的学者对“性瘾”概念的真伪还尚未定论。基于我和为数不少的性瘾者的交流沟通,我认为“性瘾者”这个概念确实存在:这是一群接受过良好教育并且拥有强烈道德意识的青年人,意识到自身对性或者色情的追求已经影响到了正常的生活和工作、损害了身体乃至精神健康,他们不畏承认自己的性瘾问题,并且积极寻求一切资源来帮助自己摆脱这种病瘾的负面影响。\\
随着社会包容度的逐步提高,性瘾者团体自我意识的觉醒以及寻求援助意念的增强,这个群体正在为人们所理解,并引起主流媒体的关注。与此同时,国内越来越多的反色情机构也纷纷为性瘾者提出切实可行的建议,如“戒邪淫网”,通过强调精神信仰的重要性鼓励性瘾者内心向善,劝诫公众远离性瘾;“反色情网”则是通过提倡中国悠久的道德伦理观念,希望唤醒性瘾者的道德感和良知,以支持个人克服 性瘾的决心;非本土的公益组织,如性与爱上瘾无名会(Sex and Love Addicts Anonymous)也进入了中国社会,为性瘾者提供一些治疗的课程,等等。\\
总之,中外公益组织所提供的各类信息和资源正在帮助性瘾者更好的了解自身情况、采取相应的治疗措施改善自身的现状;同时,这些机构的存在和宣传,也让更多的社会大众意识到性瘾的社会化趋势,以及探索治疗性瘾方式的迫切性。\\
译者认为随着社会的不断发展,大众思想开放程度不断增加,将性瘾放到道德对立面劝诫或者以宗教教义化导的教诲方式,在目前中国社会传统文化断代、宗教信仰普遍缺失的大时代背景下未必具有普适性。而国内相关心理知识的匮乏,也导致许多性瘾者难以了解这种瘾患的起源,更不知如何采取系统的治疗方式。\\
《跳出色情陷阱》此书最重要的意义便是:它脱离了传统的道德规劝方式,而是将性瘾的根源追溯到个人的成长教育时期和个体心理的发展机制,用科学实验的方式证明过度观看色情能改变人脑中化学物质的构成,从而让人深陷其中难以自拔;同时,作者根据丰富的临床实验,提供了一套系统的、切实可行的治疗方案来帮助读者克服色情瘾。\\
译者认为,此书不仅可以作为克服色情瘾的指导用书,也可以用作青少年性启蒙教材,更可以作为性心理学研究的学术资料来研究。换言之,这本书的目标读者,不仅仅是有自我救赎意识的性瘾者,也包括所有希望完善自身性观念或性教育的个体,因此,此书也可以成为家庭性教育的启蒙材料; 至于本书的科研价值,还有待性学领域的科研工作者进一步发掘。\\
《跳出色情陷阱》的顺利出版,标志着国内首部科学性教育书籍问世,希望此书能够为正在迷失中的性瘾者照亮希望之光,更希望此书可以起到抛砖引玉的作用,促使更多有志者共同努力探索中国社会性学教育方面的问题,并将研究推向健康、科学的轨道。由于本人专业知识有限,翻译存在许多不足之处,真诚欢迎广大读者批评指正。\\
我衷心感谢法律出版社的领导以及各位编辑,对我的热情指导与帮助;非常感谢作者温迪马尔茨和拉瑞马尔茨长久以来对我的支持和理解;商丘市寿康文化研究学会徐冉会长在此书出版期间给予了我诸多的宝贵建议;中国反色情网站负责人夏海新为此书写序言,在此我深表感谢。\\
我衷心感谢每一位参与本书校对的网络志愿者们:曹艳珺、黄宇、程蒙、王利文、候晓可、高悦、刘雪瑶、周洪侠、邹斯婷、张斐奕,蒋荟蓉、崔翘、周堃、钟锦英、范睿哲、张韫、吴懿文、邓之欣、黄明瑜、张璐、蒋荟蓉、胡敏、邹艳涛、吴璇。\\
宋尚鸿\\
二O一四年春于北京\\
\end{document}