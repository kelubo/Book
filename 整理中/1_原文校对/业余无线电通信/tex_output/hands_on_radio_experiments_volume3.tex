\documentclass[12pt,UTF8]{ctexbook}

% 设置纸张信息
\usepackage{geometry}
\geometry{a4paper, left=2.5cm, right=2.5cm, top=3cm, bottom=3cm}

% 设置字体
\xeCJKsetup{AutoFallBack=true}
\setCJKfamilyfont{hei}{SimHei}
\setCJKfamilyfont{kai}{KaiTi}

% 目录 chapter 级别加点(.)
% \usepackage{titletoc}
% \titlecontents{chapter}[0pt]{\vspace{3mm}\bf\addvspace{2pt}\filright}{\contentspush{\thecontentslabel\hspace{0.8em}}}{}{\titlerule*[8pt]{.}\contentspage}

% 设置 part 和 chapter 标题格式
\ctexset{
	part/name= {第,卷},
	part/number={\chinese{part}},
	chapter/name={第,章},
	chapter/number={\arabic{chapter}}
}

% 图片相关设置
\usepackage{graphicx}
\graphicspath{{../epub_extracted/images/}}

% 列表项向右偏移
\usepackage{enumitem}

% 强制表格位置
\usepackage{float}

% 超链接设置
\usepackage{hyperref}
\hypersetup{
    colorlinks=true,
    linkcolor=blue,
    filecolor=magenta,
    urlcolor=cyan,
}

% 标题设置
\title{ARRL's Hands-On Radio Experiments, Volume 3}
\author{H. Ward Silver}
\date{2018}

\begin{document}

\maketitle
\tableofcontents

\frontmatter
\chapter{前言}

本书是 ARRL's Hands-On Radio Experiments 系列的第三卷,包含了多个关于业余无线电的实验项目。这些实验涵盖了电子电路、天线与传播、传输线与阻抗匹配、电子基础、电子元件、测试与测试设备、RF 技术以及实用电台实践等多个领域。

通过这些实验,读者可以深入了解业余无线电的各种技术原理和实践应用,提高自己的无线电技能。

\mainmatter

% 增加空行
~\\

\part{电子电路}
\chapter{实验 #122:电池特性(上)}
\section{英文原文}

This month, we begin a two part article about batteries and battery characteristics with a discussion of what makes batteries "go" and how the different types behave electrically. Next month, we'll observe the differences for ourselves.

\subsection{Basic Battery Construction  and Chemistry}

Viewed simply, the battery is a self-contained chemical reaction vessel in which chemical substances give up electrons that flow through an external circuit to other types of substances that accept electrons. The battery's \textbf{separator} keeps the chemicals, well, separated so that the electrons have to make the trip through the circuit to be exchanged. 

The electrons do some useful work along the way, converting the chemical energy to electrical energy and then to whatever type of energy the user extracts from the circuit — mechanical, heat, electromagnetic, etc. The reaction continues until all of the available electrons have been exchanged through the external circuit, depleting the chemicals and discharging the battery.

The types of atoms or molecules involved in giving up and accepting the electrons — called the \textbf{battery chemistry} — determine the \textbf{electromotive force} (EMF) that pushes the electrons through the circuit. Each type of atom or molecule has a certain affinity for electrons: Via a specific chemical reaction, some want to get rid of them and some want to acquire them. The strength of that reaction's electron exchange is the reaction's \textbf{electropotential}, which is measured in volts. The difference in electropotential between the chemicals is what determines the \textbf{terminal voltage }of the battery. (The list of electropotentials for common materials and reactions is also known as the \textbf{galvanic series}.)

Batteries are classified in two groups; \textbf{primary} (non-rechargeable) and \textbf{secondary} (rechargeable). In both groups, giving up and accepting electrons changes the chemicals into different compounds. (The atoms are still the same types of atoms but the rearrangement of their electrons changes the structure of molecules made from those atoms.) In a \textbf{primary} battery, the reaction is not reversible even if a voltage is applied externally to make the electrons flow "the other way." In a \textbf{secondary} or rechargeable battery, the reaction will run in reverse if powered by an external voltage, restoring the original chemicals and recharging the battery.

\subsection{Basic Battery Terminology}

Let's start with \textbf{capacity,} which is given in \textbf{ampere-hours} (\textbf{Ah}). Amperes (coulombs of charge per second) multiplied by time results in an amount of charge. (1 Ah = 1 coulomb/s × 3600 s/h = 3600 coulombs) Thus, capacity measures the number of electrons that a battery can cause to flow through an external circuit. Because capacity is an amount of charge, it is abbreviated C for coulomb. Causing large numbers to flow (high current) discharges a battery's capacity quickly and low current discharges it slowly.

Capacity is independent of terminal voltage: A specific pair of chemicals has the same relative electropotential no matter what quantity of those chemicals is contained in the battery. Thus, terminal voltage is the same for large and small batteries of the same battery chemistry. The larger the quantity of chemicals in the battery, however, the more electrons can be exchanged. 

Capacity also indicates the amount of energy stored in a battery. Since the terminal voltage is relatively constant and voltage is joules (J) per coulomb, multiplying capacity times ter- minal voltage yields energy, usually in units of \textbf{watt-hours} (Wh). For example, a capacity of 2 Ah and a 1.5 V terminal voltage represents 2 Ah × 1.5 V = 3 Wh. (Note that not all of the stored energy can be delivered to the external circuit and that terminal voltage drops as the battery is discharged.)

Another important battery characteristic related to its capacity is the battery's \textbf{specific energy,} which is given in watt-hours per kilogram (Wh/kg). Batteries with high specific energy store a lot of energy for a given weight. 




In battery literature, you will also encounter \textbf{C-rate,} which is the rate at which a battery is charged or discharged measured in terms of its capacity. If a 1000 mAh battery is discharged at a current of 1000 mA, it is being discharged at a rate of 1 C. At 500 mA, the rate is 0.5 C, and at 100 mA the rate is 0.1 C. Because batteries are made in so many different sizes using the same chemistry, C-rate is a useful way to talk about battery performance and maintenance independent of size.

\subsection{Types of Battery Chemistry}

Several battery chemistries account for most needs of the Amateur Radio operator: alka- line, lead-acid, nickel-cadmium (NiCd), nickel-metal hydride (NiMH) and lithium-ion (Li-ion). Table 1 gives the basic characteristics of these battery types.\textsuperscript{1} Alkaline batteries are primary (non-rechargeable) and the rest are secondary (rechargeable). While re-chargeable batteries offer higher specific energies and lower costs over the life of the battery, alkaline batteries do not require a charger, which can be important for emergency situations when ac power is not available. They also have a long shelf life.

\begin{figure}[H]
    \centering
    \includegraphics[width=0.7\linewidth]{00114.jpeg}
    \caption{电池类型特性}
    \label{fig:battery_types}
\end{figure}

\subsection{Effect of Discharge Rate}

Table 1 shows peak discharge rates for the various types of batteries in terms of capacity, however, that is not the recommended rate of discharge during normal use. The amount of energy a battery can deliver is maximized at a much lower rate, shown in the row "Best discharge rate" in Table 1. Note that all types of batteries perform best well below the peak discharge rate. Figure 1 shows the effect on a NiCd battery pack's lifetime at different discharge rates. Other types of batteries are even more strongly affected by discharge rate.

\begin{figure}[H]
    \centering
    \includegraphics[width=0.7\linewidth]{00115.jpeg}
    \caption{不同放电率对电池寿命的影响}
    \label{fig:discharge_rate_effect}
\end{figure}

\subsection{Effect of Depth of Discharge}

Depth of Discharge (DoD) has a large effect on the number of charge-discharge cycles a battery can provide. The more, or "deeper," =a battery is discharged, the more it is stressed. Table 2 provides an idea of the effect on a Li-ion battery lifetime when repeatedly discharged to a specific level. Partial discharges preserve battery life.

\begin{figure}[H]
    \centering
    \includegraphics[width=0.7\linewidth]{00116.jpeg}
    \caption{放电深度对电池寿命的影响}
    \label{fig:dod_effect}
\end{figure}

\subsection{Effect of Repeated  Charge-Discharge Cycles}

As most users of rechargeable batteries quickly discover, a battery performs like  new over a number of charge-discharge cycles and then begins to lose capacity. This is due to changes at the microscopic level in the materials of the battery. For example, when a battery is new, the chemicals are typically in the form of very small crystals that provide lots of surface area to exchange electrons. With each cycle, however, the crystals grow in size and that reduces the  total surface area and battery capacity.

Figure 2 shows the effect of repeated cycling on the capacity of a set of identical, new 1500 mAh Li-ion batteries. Even though the battery may be able to support hundreds of cycles, after 200 cycles most of the batteries had lost 10% of their initial capacity.

\begin{figure}[H]
    \centering
    \includegraphics[width=0.7\linewidth]{00117.jpeg}
    \caption{重复充放电循环对电池容量的影响}
    \label{fig:cycle_effect}
\end{figure}

\subsection{Effect of Temperature}

Because batteries are based on chemical reactions that vary in rate with temperature, capacity is also affected. For example, a lead-acid battery loses about 10 to 20% of its capacity between room temperature (23°C) and freezing (0°C). By –20°C, about half of the capacity is lost. Thus it is very important for capacity to be measured at a specific temperature when comparing batteries. Automobile and deep cycle batteries have a \textbf{Cold Cranking Amps} (\textbf{CCA}) rating that must be specified at –18°C (0°F) for just this reason.

That is not to say that everything gets better with increasing temperature. From \textbf{Batteries in a Portable World}, "The optimum operating temperature for a [lead-acid] battery is 25°C (77°F). As a guideline, every 8°C (15°F) rise above this temperature cuts battery life in half. At 33°C (95°F), that battery's lifetime is cut in half and is reduced to 10% at 45°C (107°F)."

\subsection{Measuring Battery Performance}

Next month we are going to both measure some of these effects on actual batteries and introduce you to a new type of voltmeter that has the ability to act as a \textbf{data logger}. Once you realize how useful data logging is, you'll find all sorts of uses for it in the shack and around your home.

\section{注释}

\begin{enumerate}
    \item I. Buchmann, \textbf{Batteries In a Portable World}, 2011, pp 34-37. Available from your ARRL dealer or the ARRL Bookstore, ARRL order no. 1156. Telephone 860-594-0355, or toll-free in the US 888-277-5289; \href{http://www.arrl.org/shop}{www.arrl.org/shop}; \href{mailto:pubsales@arrl.org}{pubsales@arrl.org}.
\end{enumerate}

\chapter{实验 #123:电池特性(下)}
\section{英文原文}

In last month's column, I explored some of the basic terminology used to describe and compare batteries.\textsuperscript{1} Material from \textbf{Batteries For a Portable World} clearly showed the differences between the common types of batteries that hams use to supply power for radios and accessories.\textsuperscript{2} This month, we'll measure some common batteries by using the \textbf{data logging }function of an inexpensive digital multimeter (DMM).

\subsection{Data Logging Voltmeter}

Let's face it, taking regular measurements of a slowly changing parameter is bor-r-r-r-ring. I've done my share of watching a ticking clock and meter or gauge but today there are automated tools to do that job. They never get distracted, forget, or misread the data. The tool we'll use this month is the data logging DMM.

A full fledged data logger, such as a Fluke 2625A Hydra model (\href{http://www.fluke.com}{www.fluke.com}) with multiple channels and high speed high accuracy precision measurements, is way beyond the needs of a typical ham. What we need is a single channel voltmeter with an interface to a PC.

The smaller sibling of the 2625A is Fluke's 289 DMM, which has impressive specifications for a voltmeter. Features include a USB interface and companion software so you can store or \textbf{log} data on a PC. This is a top-of-the-line DMM with a $600 price tag (the older Fluke 189 sells for a couple hundred less). I love my ultra reliable Fluke DMM but for ham shack data logging, I needed a less expensive solution.

A trip to the Jameco catalog (\href{http://www.jameco.com}{www.jameco.com}) turned up the under $100 \textbf{house brand} MS8226, Jameco p/n 137462, with decent specs and an RS-232C interface. Searching the usual Internet bargain sites also turned up similar meters, some for as little as $30. You might also get lucky by watching for used meters from Fluke and other high end manufacturers.

The MS8226 has all the usual functions, plus temperature (°C) with an included thermocouple, capacitance (50 nF to 100 µF), frequency (to 5 MHz), duty cycle and true RMS measurements with an unspecified upper frequency limit. The RS-232C interface requires a USB-to-serial converter or running the host software on an older PC with a serial port.

Without making this a product review, I'll just say that the meter works as advertised and includes an easy to understand manual. The software is very basic, but useful as a means of creating time stamped data files of measurements from the voltmeter. Once the data is on the PC, you can export it into spreadsheet format for graphing or analysis as described later.

\begin{center}
\textbf{Winding Your Own Wirewound Resistors}
\end{center}

If you don't have a power resistor handy, you can wind your own resistor from common copper wire. The \textbf{ARRL Handbook} gives resistance in \textbf{Ω / 1000 feet}. For a 1 Ω resistor, #20 AWG at 10.1 Ω / 1000 feet requires about 100 feet, and #30 AWG at 104 Ω /1000 feet requires about 10 feet. Wind enameled wire on a ceramic, glass or non melting plastic tube. A drop of epoxy will hold the wire in place.

\subsection{Loading and Testing Batteries}

We're going to record battery voltages with a resistive load applied every few seconds over an extended period of time. Comparisons will be made by manually transferring data into a multi-column spreadsheet for graphing. (The spreadsheet used for this column is available on the Hands-On Radio website.)

Remember that batteries store a lot of energy. When choosing a load, plan for the heat that must be dissipated by using resistors with an adequate power rating and keeping them off of surfaces that can be damaged by elevated temperatures. The maximum power dissipation during a load test will be E\textsuperscript{2}/R, where E is the battery terminal voltage. 

Use load resistances that will draw some- what more than the Best Discharge Rate current in Table 1 of the previous experiment. Values around 1 Ω will work well for these tests, drawing a maximum current of 1.5 V / 1 Ω = 1.5 A, and dissipating a little over 2 W (1.5 V\textsuperscript{2} / 1 Ω = 2.25 W). With this power dissipation a 5 W resistor can get hot enough to burn you or a workbench surface. (Don't use incandescent lamps as loads — their resistance varies with current.) 

If you make your own loads, use several resistors in series or parallel to spread out the heat as shown in the photo of my test set in Figure 1. My loads are made out of paralleled 2 W metal oxide resistors. They are soldered between SO-239 UHF coax sockets made into a frame with #6-32 screws and threaded spacers. A banana plug fits snugly into the SO-239 making a fine high current connection or, at lower currents, clip leads can be used. Typical plastic battery holder contacts and wiring may not be heavy enough to handle the higher than normal discharge currents in these tests. I used copper pennies held in an insulating vise as my fixture contact with an extra heavy clip lead to the load.

\begin{figure}[H]
    \centering
    \includegraphics[width=0.7\linewidth]{00118.jpeg}
    \caption{电池测试装置}
    \label{fig:battery_test_setup}
\end{figure}

\subsection{AA to AAA Comparison}

Before beginning, it's worth noting that what is usually referred to as a \textbf{battery} is a single package of chemicals more correctly referred to as a \textbf{cell.} A set of cells connected together form a \textbf{battery}, which derives from the original meaning of a group of identical pieces, such as an artillery battery of several guns. An assembly of six individual lead acid cells, each producing 2 V and connected in series create a vehicle's 12 V starting battery. In the case of single cell batteries, the word \textbf{cell} and \textbf{battery} are interchangeable.

Let's start by comparing two fresh batteries that use the same chemistry but have different capacities. I used AA and AAA alkaline batteries sold by Costco under their \textbf{Kirkland} brand name. Battery capacity is not specified by Costco but third party testing has found the AA cells to supply approximately 2300 mAh and the AAA capacity is probably about half that.

Figure 2 shows the initial portion of a 30 minute battery comparison when connected to a 1 Ω resistor load at room temperature (about 21°C). You can see that the initial terminal voltages are approximately the same and that the capacity of the smaller AAA battery is depleted more quickly. At the end of the load test, (not shown in Figure 2)the AA cell terminal voltage recovered to 1.38 V and the AAA cell to 1.22 V.

\begin{figure}[H]
    \centering
    \includegraphics[width=0.7\linewidth]{00119.jpeg}
    \caption{AA和AAA电池比较}
    \label{fig:aa_aaa_comparison}
\end{figure}

\subsection{Alkaline to NiMH Comparison}

The open circuit voltage of the fresh rechargeable NiMH AA cell shown in Figure 3 is lower than that of a fresh alkaline cell by about 0.3 V. Both drop about the same amount when connected to a 1 Ω load. That difference narrows to 0.2 V after about  6 minutes. The 0.3 V margin can translate to a lot of extra operating time if alkaline cells are used.

\begin{figure}[H]
    \centering
    \includegraphics[width=0.7\linewidth]{00120.jpeg}
    \caption{碱性电池和NiMH电池比较}
    \label{fig:alkaline_nimh_comparison}
\end{figure}

\begin{figure}[H]
    \centering
    \includegraphics[width=0.7\linewidth]{00121.jpeg}
    \caption{温度对电池内阻的影响}
    \label{fig:temperature_effect}
\end{figure}

\subsection{Temperature and  Internal Resistance}

A battery's temperature affects its internal resistance quite a bit as you can see in Fig-ure 4. The same fresh battery was tested at 0, 20 and 80°C by connecting it to a 1 Ω load for 10 s with a 10 s rest between load periods. The open circuit terminal voltage was approximately the same at all three temperatures, varying only 8 mV from 1.604 to 1.612 V. The cold battery voltage dropped substantially under load — initially about 0.26 V at 1.6 A load, implying an internal resistance, R\textsubscript{INT} of 0.26 / 1.6 = 0.16 Ω. At room temperature, voltage dropped 0.125 V for R\textsubscript{INT} = 0.078 Ω. At 80°C (hot enough to burn the experimenter's fingers!) the voltage drop of 0.064 V indicates R\textsubscript{INT} = 0.04 Ω, a 2:1 variation with temperature. This can be important when trying to get the most performance from a battery over a wide temperature range!

\subsection{Other Data Logging Tasks}

Obviously, data logging can be put to many other uses, such as recording temperature or current consumption. As with the radio astronomy project, logging can record audio (noise levels in that case) as well. A spreadsheet can convert voltages from sensors directly into physical units and combine different data elements to measure differential temperatures, ratios, minimum and maximum values and so forth. Best of all, a data logger can patiently record data to catch a power dropout or intermittent noise that never seems to happen when you're around.

\section{注释}

\begin{enumerate}
    \item All previous Hands-On Radio experiments are available to ARRL members at \href{http://www.arrl.org/hands-on-radio}{www.arrl.org/hands-on-radio}.
    \item I. Buchmann, \textbf{Batteries In a Portable World}, 2011, pp 34-37. Available from your ARRL dealer or the ARRL Bookstore, ARRL order no. 1156. Telephone 860-594-0355, or toll-free in the US 888-277-5289; \href{http://www.arrl.org/shop}{www.arrl.org/shop}; \href{mailto:pubsales@arrl.org}{pubsales@arrl.org}.
    \item \href{http://www.batteryshowdown.com/results-lo.html}{www.batteryshowdown.com/results-lo.html}
\end{enumerate}

\chapter{实验 #142:高频电感}
\section{英文原文}

In this experiment, we'll make and measure the performance of inductors at RF. We'll start with the basic relationship between inductance and inductive reactance, then measure some commercial inductors, and finally make our own air-core inductors from the formula found in the \textbf{ARRL Handbook.}

\subsection{Inductance and Inductive Reactance}

Inductance (L) is the ability to store energy in a magnetic field and is measured in henries (H). A changing current through an inductor causes a change in the magnetic field, which in turn creates a voltage in the inductor that opposes the change in current. This is known as \textbf{electromagnetic induction} or \textbf{self induction.}

When an alternating current (AC) flows through an inductor, the opposition to that current is called inductive reactance (X\textsubscript{L}) and is measured in ohms (Ω). The relationship between inductance, frequency, and inductive reactance is given by the formula:

X\textsubscript{L} = 2πfL

where f is the frequency in hertz (Hz). From this formula, we can see that inductive reactance increases with both frequency and inductance. This means that inductors are more effective at blocking high-frequency signals than low-frequency ones.

\subsection{Commercial Inductors}

Let's start by measuring the inductance of some commercial inductors using an inductance meter or a multimeter with inductance measurement capability. You can also use an oscillator and a resistor to form a resonant circuit, then calculate the inductance from the resonant frequency.

For this experiment, we'll use a set of common inductors with values ranging from a few microhenries (µH) to several millihenries (mH). Record the nominal value of each inductor and then measure its actual inductance. Note any discrepancies between the nominal and measured values.

\subsection{Making Air-Core Inductors}

Now let's make our own air-core inductors using the formula from the \textbf{ARRL Handbook}. The formula for the inductance of a single-layer air-core inductor is:

L = (r\textsuperscript{2} × N\textsuperscript{2}) / (9r + 10l)

where:
- L is the inductance in microhenries (µH)
- r is the radius of the coil in inches
- N is the number of turns
- l is the length of the coil in inches

Let's make a 10 µH inductor. Using the formula, we can calculate the number of turns needed for a given radius and length. For example, if we use a 1/2 inch diameter form (r = 0.25 inches) and a coil length of 0.5 inches, we can solve for N:

10 = (0.25\textsuperscript{2} × N\textsuperscript{2}) / (9 × 0.25 + 10 × 0.5)
10 = (0.0625 × N\textsuperscript{2}) / (2.25 + 5)
10 = (0.0625 × N\textsuperscript{2}) / 7.25
10 × 7.25 = 0.0625 × N\textsuperscript{2}
72.5 = 0.0625 × N\textsuperscript{2}
N\textsuperscript{2} = 72.5 / 0.0625 = 1160
N ≈ 34 turns

Wind 34 turns of insulated wire around a 1/2 inch diameter form. Remove the form and measure the inductance of the coil. Compare the measured value with the calculated value.

\subsection{Measuring Inductive Reactance}

Now let's measure the inductive reactance of our inductors at different frequencies. We can do this using a signal generator, an oscilloscope, and a resistor.

1. Connect the inductor in series with a known resistor (e.g., 100 Ω).
2. Apply a sine wave signal from the signal generator to the series circuit.
3. Measure the voltage across the resistor (V\textsubscript{R}) and the voltage across the inductor (V\textsubscript{L}) using the oscilloscope.
4. Calculate the current through the circuit using Ohm's law: I = V\textsubscript{R} / R.
5. Calculate the inductive reactance using Ohm's law: X\textsubscript{L} = V\textsubscript{L} / I.
6. Repeat this measurement at several different frequencies (e.g., 1 kHz, 10 kHz, 100 kHz, 1 MHz).

\subsection{Q Factor}

The quality factor (Q) of an inductor is a measure of its efficiency. It is defined as the ratio of inductive reactance to the resistance of the inductor:

Q = X\textsubscript{L} / R

A higher Q factor indicates a more efficient inductor with less energy loss. Let's measure the Q factor of our inductors at different frequencies.

1. Measure the DC resistance of the inductor using a multimeter.
2. Calculate the inductive reactance at a given frequency as described above.
3. Calculate the Q factor using the formula.

\subsection{Experiments with Inductors}

1. **Inductor in Series with a Resistor**: Connect an inductor in series with a resistor and apply an AC signal. Observe how the voltage across the inductor and resistor changes with frequency.

2. **Inductor in Parallel with a Resistor**: Connect an inductor in parallel with a resistor and apply an AC signal. Observe how the current through the inductor and resistor changes with frequency.

3. **LC Resonant Circuit**: Connect an inductor in parallel with a capacitor to form a resonant circuit. Apply an AC signal and observe the voltage across the circuit as you vary the frequency. The resonant frequency is where the voltage is maximum.

\begin{figure}[H]
    \centering
    \includegraphics[width=0.7\linewidth]{00122.jpeg}
    \caption{电感测试电路}
    \label{fig:inductor_test_circuit}
\end{figure}

\section{注释}

\begin{enumerate}
    \item 公式来源:\textbf{ARRL Handbook for Radio Amateurs}
\end{enumerate}

\chapter{实验 #125:施密特触发器}

\section{英文原文}

Once you start building radio gear, you learn that a great deal of "radio" has little to do with RF. This month we're going to work with one of the non-radio building blocks you'll encounter when an analog signal and a digital function come together — the \textbf{Schmitt trigger}. Long-time readers may recall the Schmitt trigger making an appearance in Experiment #11 about op-amp comparators.\textsuperscript{1, 2} Adding positive feedback to the basic comparator circuit creates \textbf{hysteresis} — a switching threshold that changes depending on whether the circuit's output is on or off. This turns out to be desirable in certain applications.

If you're not familiar with op-amp comparators, download and read Experiment #11 from the "Hands-On Radio" web page. Page two covers hysteresis and how to design the comparator-based Schmitt trigger with discrete components. If you have an LM311 (or one of the equivalents listed) build a Schmitt trigger circuit and perform the experiment.

\subsection{The Logic IC Schmitt Trigger}

The ability to switch reliably in the presence of noise is a valuable function for digital circuits that have analog signals as inputs. Op-amps and discrete components take up valuable printed circuit board space, so the Schmitt trigger function was packaged into an IC. The basic set of six hex inverters (7414-type or CD4069 ICs) and quad NAND gates (74132 or CD4093) are the most common Schmitt trigger ICs. They are inexpensive and widely available.\textsuperscript{3}

The difference in switching characteristics between standard logic gates and Schmitt triggers can be seen in the device data sheets. Download the Texas Instruments data sheets for the CD4011 (standard quad NAND) and CD4093 ICs from \href{http://www.datasheetcatalog.org}{www.datasheetcatalog.org}. Look in the dc or Static Electrical Characteristics tables and find the Input Low and Input High voltage specifications for a V\textsubscript{DD} value of 5 V at 25°C. For the standard gate the typical values of V\textsubscript{IH} and V\textsubscript{IL} are 3.5 and 1.5 V, respectively, and the response to an input signal in that 2 V range is undefined. For the Schmitt trigger IC, V\textsubscript{N} and V\textsubscript{P} are 1.9 and 2.9 V, only 1.0 V apart — a much smaller switching window — and drawing (b) at the bottom of the CD4093 datasheet's first page shows the \textbf{transfer characteristic} of the gate. Note that the output voltage is defined for all values of input voltage. Let's put that to work.

\subsection{Sensing Slowly Changing Signals}

Any device controlled by a microprocessor needs to have a \textbf{POWER_OK} signal to prevent its digital circuits from attempting to operate before the power supply is fully up and running. Such premature operation can yield strange results.

Similarly, when power is lost, the same signal notifies the digital circuits to shut down in a hurry. Specialized power monitoring ICs are available for this task, but the simple circuit in Figure 1 can also do the job.

\begin{figure}[H]
    \centering
    \includegraphics[width=0.7\linewidth]{00004.jpeg}
    \caption{Power Monitoring Circuit}
    \label{fig:power_ok}
\end{figure}

When power turns \textbf{ON}, capacitor C1 charges slowly through R1, keeping V\textsubscript{IN} below the buffer's V\textsubscript{P} threshold for a time delay of about one time constant, R1 × C1. (The back-to-back inverters form a non-inverting \textbf{buffer}.) After that time, the buffer's \textbf{POWER_OK} signal goes high to indicate the power supply has had enough time to stabilize. If power is lost at any time during the charging process, or if power is turned \textbf{OFF} after stabilizing, C1 is rapidly discharged through the 1N4148 and the \textbf{POWER_OK} signal goes low.

You can build this circuit using two of the inverters in a 74HC14 IC powered by 5 V. To make it easy to observe the time delay, use a 1 MΩ resistor for R1 and a 1 μF capacitor for C1. The 1 kΩ resistor provides current limiting during the experiment and is small compared to R1, so it has an insignificant effect on the charging time of C1.

Build the power-down detector using a CMOS hex inverter such as the 74HC14 or CD40106. TTL versions (7414 or 74LS14) draw too much current for the 1 MΩ resistor to act as a pull-up. Also, in a real-world design, the inverter would be powered from a large capacitor to allow it to hold \textbf{POWER_OK} low for several msec, insuring a controlled shut-down period.

Watching the \textbf{POWER_OK} signal with a voltmeter or oscilloscope, apply power to the circuit and verify that the signal stays low and does not go high until about 1 second has passed. If you connect one end of a clip lead or piece of wire to ground and simulate a power dropout by brushing the other end against the cathode of the diode (point A in Figure 1), the output signal should immediately go low, signifying power is \textbf{not} OK and there should be a 1 second delay before it returns to the OK state. \textbf{POWER_OK} might be used as a reset signal for a digital circuit.

\subsection{Switch Debouncing}

A switch may feel quite solid to you, leaving little doubt that when you close a switch, it instantly closes and stays closed. In truth, the contacts of almost all mechanical switches and relays literally bounce for a few milliseconds before settling down to stay closed. Because digital devices are so fast, software can react to those bounces as multiple switch closings and openings. While it's possible to "debounce" a switch signal in software, it can be done with hardware, too.

Reconfigure your circuit as shown in Figure 2; only one inverter section is needed here. That wire you just used to simulate a power dropout can also simulate the noisy signal from a switch, or you can use a real momentary switch. The two time delays, t\textsubscript{CL} and t\textsubscript{OP} depend on the values of R1 and R2, respectively:

\begin{figure}[H]
    \centering
    \includegraphics[width=0.7\linewidth]{00005.jpeg}
    \caption{Switch Debouncing Circuit}
    \label{fig:switch_debounce}
\end{figure}

\begin{center}
t\textsubscript{CL} ≈ R2 × C1 and t\textsubscript{OP} ≈ R1 × C1
\end{center}

Start with R1 = 1 MΩ, R2 = 10 kΩ, and C1 = 1 μF. You'll need an oscilloscope to see the bounces of the switch contacts and the short delay, t\textsubscript{CL}.

If you swap R1 and R2, you'll find that R1 has to be much larger than R2 for \textbf{SW_CLOSE} to go high. If R1 is too small, closing the switch does not discharge C1 below the V\textsubscript{N} threshold for the inverter because R1 "overpowers" R2 and keeps the capacitor charged to higher than V\textsubscript{N}. Experiment with different values; you'll find that R1 has to be about three times larger than R2 to get reliable results.

The squelch function in your radio also requires a continuously changing signal to cause switching at a threshold. If the input signal to R2 is the rectified and filtered output of a receiver's audio amplifier, the output of the inverter indicates whether a signal is present.

\subsection{Edge Detectors}

In many cases you want to be able to tell when the input signal changes state without having to monitor it continuously. This requires an elementary form of memory that will allow a circuit that can compare "then" to "now." The circuit in Figure 3 uses a two-input XOR gate to make the comparison and an R-C circuit to create the time delay that acts as memory. The Schmitt trigger input is required because of the slowly changing R-C circuit output.

\begin{figure}[H]
    \centering
    \includegraphics[width=0.7\linewidth]{00006.jpeg}
    \caption{Edge Detector Circuit}
    \label{fig:edge_detector}
\end{figure}

V\textsubscript{A} is the "now" state of V\textsubscript{IN} while V\textsubscript{B} is the "then" state. An XOR gate's output is high only when just one of its inputs is high. If both inputs are in the same state — high or low — the output, \textbf{CHANGE}, is also low. During the delay period while C1 is charging or discharging (approximately equal to the time constant R1 × C1), one input lags behind the other so that the XOR function is true and \textbf{CHANGE} is high — but \textbf{only} during the charge/discharge period. The Schmitt trigger action insures that the slowly-changing voltage V\textsubscript{B} causes only one pulse with every transition of V\textsubscript{IN}.

You can build the circuit in Figure 3 using one section of a 74HC86 quad exclusive-OR gate with Schmitt trigger inputs. R1 and C1 values of 100 kΩ and 0.01 μF will provide output pulses about 1 msec long. Use a 50% duty-cycle, 5 V pulse output from a function generator or a 555 timer circuit as V\textsubscript{IN} (don't use a square wave with a negative voltage) with a repetition rate of about 100 Hz. Vary the time constant of R1-C1 to see the effect on the output pulse width, viewed on an oscilloscope.

Another name for this circuit is a \textbf{frequency doubler}. Two output pulses occur for every input pulse — one at each edge of the input pulse. If you insert a Schmitt trigger inverter at point B, the \textbf{CHANGE} pulses change from positive pulses at each edge to negative pulses.

\subsection{Shopping List}

\begin{itemize}[leftmargin=2cm]
    \item 74HC14 hex inverter
    \item 74HC86 quad XOR gate
    \item 1N4148 diode
    \item 1/4-watt resistors: 1 kΩ, 10 kΩ, 100 kΩ, 1 MΩ
    \item Capacitors: 0.01 μF, 1 μF (ceramic or electrolytic)
    \item Momentary switch (optional)
\end{itemize}

\subsection{Notes}

\begin{enumerate}
    \item All previous "Hands-On Radio" experiments are available to ARRL members at \href{http://www.arrl.org/hands-on-radio}{www.arrl.org/hands-on-radio}.
    \item The Schmitt trigger is named for Otto Schmitt who identified the function when studying properties of squid nerves in the 1930s!
    \item 7400-family model ICs are available in many different logic families such as HC, LS, and AC.
\end{enumerate}

\section{中文翻译}

一旦你开始构建无线电设备,你会发现很多"无线电"工作与射频几乎无关。本月我们将使用一种非无线电构建模块,当模拟信号和数字功能结合时你会遇到它——\textbf{施密特触发器}。长期读者可能会记得施密特触发器在关于运算放大器比较器的实验 #11 中出现过。\textsuperscript{1, 2} 向基本比较器电路添加正反馈会创建\textbf{迟滞}——一种开关阈值,根据电路输出是开还是关而变化。这在某些应用中被证明是可取的。

如果你不熟悉运算放大器比较器,请从"Hands-On Radio"网页下载并阅读实验 #11。第二页涵盖了迟滞以及如何使用分立元件设计基于比较器的施密特触发器。如果你有 LM311(或列出的等效器件),请构建一个施密特触发器电路并进行实验。

\subsection{逻辑 IC 施密特触发器}

在存在噪声的情况下可靠切换的能力对于将模拟信号作为输入的数字电路来说是一项有价值的功能。运算放大器和分立元件占用了宝贵的印刷电路板空间,因此施密特触发器功能被封装到 IC 中。基本的六个六反相器(7414 型或 CD4069 IC)和四与非门(74132 或 CD4093)是最常见的施密特触发器 IC。它们价格便宜且广泛可用。\textsuperscript{3}

标准逻辑门和施密特触发器之间的开关特性差异可以在器件数据手册中看到。从\href{http://www.datasheetcatalog.org}{www.datasheetcatalog.org}下载德州仪器的 CD4011(标准四与非门)和 CD4093 IC 的数据手册。查看直流或静态电气特性表,找到 V\textsubscript{DD} 值为 5 V、25°C 时的输入低电平和输入高电平规格。对于标准门,V\textsubscript{IH} 和 V\textsubscript{IL} 的典型值分别为 3.5 和 1.5 V,在该 2 V 范围内的输入信号的响应是未定义的。对于施密特触发器 IC,V\textsubscript{N} 和 V\textsubscript{P} 分别为 1.9 和 2.9 V,仅相差 1.0 V——一个小得多的开关窗口——CD4093 数据手册第一页底部的图 (b) 显示了门的\textbf{传输特性}。请注意,输出电压对于所有输入电压值都是定义的。让我们利用这一点。

\subsection{感知缓慢变化的信号}

任何由微处理器控制的设备都需要一个\textbf{POWER_OK}信号,以防止其数字电路在电源完全启动之前尝试运行。这种过早操作会产生奇怪的结果。

同样,当电源丢失时,相同的信号会通知数字电路尽快关闭。有专门的电源监控 IC 可用于此任务,但图 1 中的简单电路也可以完成这项工作。

\begin{figure}[H]
    \centering
    \includegraphics[width=0.7\linewidth]{00004.jpeg}
    \caption{电源监控电路}
    \label{fig:power_ok_cn}
\end{figure}

当电源\textbf{ON}时,电容器 C1 通过 R1 缓慢充电,使 V\textsubscript{IN} 低于缓冲器的 V\textsubscript{P} 阈值约一个时间常数 R1 × C1。(背对背的反相器形成一个非反相\textbf{缓冲器}。)在该时间之后,缓冲器的\textbf{POWER_OK}信号变高,表明电源已经有足够的时间稳定下来。如果在充电过程中的任何时间电源丢失,或者在稳定后电源被\textbf{OFF},C1 通过 1N4148 快速放电,\textbf{POWER_OK}信号变低。

你可以使用 74HC14 IC 中的两个反相器构建此电路,由 5 V 供电。为了易于观察时间延迟,使用 1 MΩ 电阻器作为 R1,1 μF 电容器作为 C1。1 kΩ 电阻器在实验期间提供电流限制,与 R1 相比很小,因此对 C1 的充电时间影响不大。

使用 CMOS 六反相器(如 74HC14 或 CD40106)构建电源下降检测器。TTL 版本(7414 或 74LS14)的电流消耗太大,无法让 1 MΩ 电阻器作为上拉电阻。此外,在实际设计中,反相器将由大型电容器供电,使其能够将\textbf{POWER_OK}保持低电平几毫秒,确保受控的关闭周期。

用电压表或示波器观察\textbf{POWER_OK}信号,向电路施加电源,并验证信号保持低电平,直到大约 1 秒后才变高。如果你将夹线或电线的一端连接到地,并通过将另一端刷到二极管的阴极(图 1 中的点 A)来模拟电源 dropout,输出信号应立即变低,表示电源\textbf{不}正常,并且在返回正常状态之前应该有 1 秒的延迟。\textbf{POWER_OK}可以用作数字电路的复位信号。

\subsection{开关去抖动}

开关对你来说可能感觉非常坚固,让你几乎毫无疑问,当你关闭开关时,它会立即关闭并保持关闭。事实上,几乎所有机械开关和继电器的触点在稳定关闭之前都会反弹几毫秒。由于数字设备如此之快,软件可能会将这些反弹反应为多个开关闭合和断开。虽然可以在软件中"去抖动"开关信号,但也可以通过硬件完成。

将你的电路重新配置为图 2 所示;这里只需要一个反相器部分。你刚才用来模拟电源 dropout 的电线也可以模拟来自开关的噪声信号,或者你可以使用真正的瞬时开关。两个时间延迟 t\textsubscript{CL} 和 t\textsubscript{OP} 分别取决于 R1 和 R2 的值:

\begin{figure}[H]
    \centering
    \includegraphics[width=0.7\linewidth]{00005.jpeg}
    \caption{开关去抖动电路}
    \label{fig:switch_debounce_cn}
\end{figure}

\begin{center}
t\textsubscript{CL} ≈ R2 × C1 和 t\textsubscript{OP} ≈ R1 × C1
\end{center}

从 R1 = 1 MΩ、R2 = 10 kΩ 和 C1 = 1 μF 开始。你需要示波器来观察开关触点的反弹和短延迟 t\textsubscript{CL}。

如果你交换 R1 和 R2,你会发现 R1 必须比 R2 大得多,\textbf{SW_CLOSE} 才能变高。如果 R1 太小,关闭开关不会使 C1 放电到反相器的 V\textsubscript{N} 阈值以下,因为 R1"压倒"R2 并使电容器保持充电到高于 V\textsubscript{N} 的值。尝试不同的值;你会发现 R1 必须比 R2 大约三倍才能获得可靠的结果。

你的收音机中的静噪功能也需要一个持续变化的信号才能在阈值处引起切换。如果 R2 的输入信号是接收器音频放大器的整流和滤波输出,反相器的输出表示是否存在信号。

\subsection{边缘检测器}

在许多情况下,你希望能够判断输入信号何时改变状态,而无需持续监控它。这需要一种基本形式的内存,允许电路比较"当时"和"现在"。图 3 中的电路使用双输入 XOR 门进行比较,并使用 R-C 电路创建充当内存的时间延迟。由于 R-C 电路输出缓慢变化,需要施密特触发器输入。

\begin{figure}[H]
    \centering
    \includegraphics[width=0.7\linewidth]{00006.jpeg}
    \caption{边缘检测器电路}
    \label{fig:edge_detector_cn}
\end{figure}

V\textsubscript{A} 是 V\textsubscript{IN} 的"现在"状态,而 V\textsubscript{B} 是"当时"状态。XOR 门的输出仅在其一个输入为高时才为高。如果两个输入处于相同状态——高或低——输出\textbf{CHANGE}也为低。在 C1 充电或放电的延迟期间(大约等于时间常数 R1 × C1),一个输入落后于另一个,因此 XOR 函数为真,\textbf{CHANGE} 为高——但\textbf{仅}在充电/放电期间。施密特触发器动作确保缓慢变化的电压 V\textsubscript{B} 随着 V\textsubscript{IN} 的每次转换仅产生一个脉冲。

你可以使用 74HC86 四异或门的一个部分构建图 3 中的电路,该门具有施密特触发器输入。100 kΩ 和 0.01 μF 的 R1 和 C1 值将提供约 1 毫秒长的输出脉冲。使用函数发生器或 555 定时器电路的 50\% 占空比、5 V 脉冲输出作为 V\textsubscript{IN}(不要使用带有负电压的方波),重复率约为 100 Hz。改变 R1-C1 的时间常数,观察对示波器上查看的输出脉冲宽度的影响。

该电路的另一个名称是\textbf{倍频器}。每个输入脉冲会产生两个输出脉冲——在输入脉冲的每个边缘一个。如果你在点 B 插入施密特触发器反相器,\textbf{CHANGE} 脉冲从每个边缘的正脉冲变为负脉冲。

\subsection{购物清单}

\begin{itemize}[leftmargin=2cm]
    \item 74HC14 六反相器
    \item 74HC86 四异或门
    \item 1N4148 二极管
    \item 1/4 瓦电阻器:1 kΩ, 10 kΩ, 100 kΩ, 1 MΩ
    \item 电容器:0.01 μF, 1 μF(陶瓷或电解)
    \item 瞬时开关(可选)
\end{itemize}

\subsection{注释}

\begin{enumerate}
    \item 所有以前的"Hands-On Radio"实验都可在 ARRL 成员的\href{http://www.arrl.org/hands-on-radio}{www.arrl.org/hands-on-radio}获取。
    \item 施密特触发器以 Otto Schmitt 命名,他在 1930 年代研究鱿鱼神经特性时发现了该功能!
    \item 7400 系列型号 IC 有许多不同的逻辑系列,如 HC、LS 和 AC。
\end{enumerate}

\chapter{实验 #129:星形-三角形和 Pi-T 电路}

\section{英文原文}

I didn't realize Hands-On Radio had so many electricians and power engineers as readers! Experiment #127's hypothetical example of three-phase power to explain how phasors operate (combined with a sloppy math error) was not intended to be a tutorial on ac wiring practices, but the references cited in the article and on the Hands-On Radio web page should clear up any confusion I unintentionally generated on this important topic.\textsuperscript{1} During the post-column post-mortem, I realized that I'd been given a golden opportunity to dig deeper into some important supporting circuit concepts and get us back to radio at the same time.

\subsection{Why a Delta?}

When studying three-phase power systems, you'll soon encounter the terms \textbf{wye} (pronounced "why") and \textbf{delta}. These refer to how the three individual phases are connected with respect to a neutral reference. Figure 1 illustrates the basic idea. The coils shown here — typical of motor or transformer windings — could also be voltage or current sources, resistors, capacitors, or generic impedances. The origin of the names for each type of system is clear — the schematics for each system take the shape of a \textbf{Y} or a \textbf{Δ}.

\begin{figure}[H]
    \centering
    \includegraphics[width=0.7\linewidth]{00007.jpeg}
    \caption{Wye and Delta Connections}
    \label{fig:wye_delta}
\end{figure}

In a wye system, the three phases (a, b and c) all share a common neutral point, so there are four connections: Phase A, B and C (also commonly labeled Line 1, 2 and 3) and neutral. In a delta system, there is no common neutral connection because the sources or loads are connected together in a loop. That means there are only three connections and the voltages between them. (A neutral reference point can be created in a delta network through various techniques discussed in the reference articles.)

The angle between the phase voltages is always 120°, but whether the angle is positive or negative depends on the \textbf{phase sequence} which can be a-b-c or a-c-b in order of increasing angle. In Figure 1A, the phase sequence is a-b-c, which is positive rotation.

Say — how can three voltages be connected together in a loop that doesn't contain any resistance and not have the current go to infinity? If the voltages were dc, we would indeed have a problem! Instead, these are ac sine waves with the same voltage magnitude (V) but different phase angles. Adding up the voltages around the circuit gives us V∠0° + V∠120° + V∠240°. Changing the phasors to rectangular coordinates allows us to calculate the sum: (V + j0) + (–0.5V + j0.866V) + (–0.5V – j0.866V) = 0. So the net voltage around the loop is zero and no circulating current flows at all!

In both wye and delta systems the loads can be connected between phase or line voltages. In a wye system, a load can also be connected between a phase voltage and neutral. Imbalanced loads in either type of system can cause substantial \textbf{error currents} to flow. Non-linear loads such as switchmode supplies and loads controlled by SCRs and TRIACs create harmonic currents. Both of these cause problems, too. Obviously, generating, transferring, and using multi-phase power is a complex subject. You can learn more about it in the references mentioned earlier and at \href{http://www.allaboutcircuits.com/vol_2/chpt_10/1.html}{www.allaboutcircuits.com/vol\_2/chpt\_10/1.html}.

But what do wye and delta power systems have to do with radio? From the standpoint of ac power, not much — unless you happen to need a \textbf{really} big power supply. How- ever, we use wye and delta all the time in our circuits — we just refer to them as \textbf{Pi} and \textbf{T}!

\subsection{Having T with Pi}

Figure 2 shows two circuits made of generic impedances — one is a Pi network like you'll find in nearly every tube-type amplifier and the other takes the shape of a T network that you'll find in most antenna tuners. Figure 2 also shows how a Pi network is the same as a delta network and a T network is a wye network. Who knew?

\begin{figure}[H]
    \centering
    \includegraphics[width=0.7\linewidth]{00008.jpeg}
    \caption{Pi and T Networks as Delta and Wye}
    \label{fig:pi_t_networks}
\end{figure}

That's handy to know, but there is another neat trick to apply. You can turn a circuit of one type (Pi or T) into its exact equivalent circuit of the other type (T or Pi) by using some math called the \textbf{wye-delta transformation}. From the perspective of the input and output connections, the equivalent circuit will behave \textbf{exactly} the same (with a caveat explained later). The following equations show the math, although you can use an online calculator such as this one at \href{http://www.elektro-energetika.cz/calculations/transfigurace.php?language=english}{www.elektro-energetika.cz/calculations/transfigurace.php?language=english}\textsuperscript{2}

Pi (Delta) to T (Wye)

Z\textsubscript{A} = Z\textsubscript{1}Z\textsubscript{2} / ΣZ; Z\textsubscript{B} = Z\textsubscript{1}Z\textsubscript{3} / ΣZ; Z\textsubscript{C} = Z\textsubscript{2}Z\textsubscript{3} / ΣZ

Where ΣZ = Z\textsubscript{1} + Z\textsubscript{2} + Z\textsubscript{3}

T (Wye) to Pi (Delta)

Z\textsubscript{1} = Z\textsubscript{P} / Z\textsubscript{C}, Z\textsubscript{2} = Z\textsubscript{P} / Z\textsubscript{B}, Z\textsubscript{3} = Z\textsubscript{P} / Z\textsubscript{A}

Where Z\textsubscript{P} = Z\textsubscript{A}Z\textsubscript{B} + Z\textsubscript{A}Z\textsubscript{C} + Z\textsubscript{B}Z\textsubscript{C}

\subsection{Using the Transformation}

More than just math sleight-of-hand, transforming the circuits from one form to the other can be quite useful. Let's say the circuit you start with has component values that are hard to make work well — maybe they are very large or very small values. By changing the circuit from one form to another, the component values also change and may become more reasonable. Let's try an example:

The circuit in Figure 3A is a 40 dB T network attenuator with symmetrical input and output impedances of 50 Ω. The 49 Ω series resistors aren't an issue, but the 1 Ω parallel resistance could be significantly affected by extra wiring resistance in the common connection. Transforming the circuit into its Pi equivalent in Figure 3B changes the resistors to 51 and 2500 Ω. The 2500 Ω resistance is much less affected by wiring resistance and other small variations. You can use an online calculator such as \href{http://www.microwaves101.com/encyclopedia/calcattenuator.cfm}{www.microwaves101.com/encyclopedia/calcattenuator.cfm} or tables of resistor values for Pi and T attenuators allow you to pick the form that makes the most sense.\textsuperscript{3}

\begin{figure}[H]
    \centering
    \includegraphics[width=0.7\linewidth]{00009.jpeg}
    \caption{T Network Attenuator and Its Pi Equivalent}
    \label{fig:attenuator_example}
\end{figure}

This can also work with circuits made out of reactances (Ls and Cs). Let's try an example in which we are transforming an output impedance of 800 Ω to 50 Ω at a frequency of 7 MHz and with a circuit Q of 6. If we use a typical antenna tuner's T network with series capacitors and a parallel inductor, the component values are C\textsubscript{IN} = 76 pF, L = 5.9 μH, and C\textsubscript{OUT} = 25 pF (calculated with the T match calculator at \href{http://www.eeweb.com/toolbox/t-match}{www.eeweb.com/toolbox/t-match}). The value of L is fine but the values for C\textsubscript{IN} and C\textsubscript{OUT} are small enough that it would not take much stray capacitance to upset the tuning of the network which makes the settings "touchy." In addition, the higher the reactances of the series capacitors (–j300 and –j910 Ω, respectively) the higher the voltages across them for a given current.

What happens if the T (wye) network is transformed to a Pi (delta) network? The resulting reactances are arranged with a shunt (parallel) inductor at the input (j 44 Ω or 1 μH) and output (j 132 Ω or 3 μH) and a larger series capacitor (–j155 Ω or 147 pF) which would reduce the effect of stray capacitance and lowers series reactance, too. The online calculator can also change the Pi network to a series-L form if that's more convenient.\textsuperscript{4}

Here's the caveat mentioned earlier: When using LC networks it's important to remember that when using LC networks, the transformation works \textbf{only} at the frequency for which the components have the reactances you specify. As frequency changes, so do the reactances, and you'll have to re-calculate the component values to get an exactly equivalent circuit. Using variable components allows you to use the circuit at different frequencies.

Putting the transformation calculators in your software toolbox is very helpful when you are trying to select and design an impedance matching circuit. This is particularly true at QRO power levels where heavy-duty (i.e. expensive) components are required to stand up to the high voltages and currents. A few iterations of your design over the range of frequencies and impedances you want helps avoid extreme component values and the high voltages and currents that often go with them.

\subsection{References}

\begin{enumerate}
    \item All previous Hands-On Radio experiments are available to ARRL members at \href{http://www.arrl.org/hands-on-radio}{www.arrl.org/hands-on-radio}.
    \item Step 1 of the calculator asks for the "Shape of the complex numbers," meaning that you should select either rectangular (R + jX) or phasor (Z∠θ) form for the impedances. Use a comma for the decimal in the European convention. i.e. 1.0 becomes 1,0.
    \item See the Component Data and References chapter of the \textbf{ARRL Handbook}, available from your ARRL dealer, or from the ARRL Store, ARRL order no. 6948. Telephone toll-free in the US 888-277-5289, or 860-594-0355; fax 860-594-0303; \href{http://www.arrl.org/shop/}{www.arrl.org/shop/}; \href{mailto:pubsales@arrl.org}{pubsales@arrl.org}.
    \item The calculator can switch between series-C (block dc current) and series-L (pass dc current) forms of the network.
\end{enumerate}

\section{中文翻译}

我没想到《实践无线电》有这么多电工和电力工程师读者!实验 #127 中用三相电力的假设例子来解释相量如何工作(加上一个草率的数学错误)并不是为了成为交流 wiring 实践的教程,但文章和《实践无线电》网页上引用的参考资料应该能澄清我在这个重要话题上无意中造成的任何混淆。\textsuperscript{1} 在专栏后的反思中,我意识到我得到了一个绝佳的机会,可以更深入地挖掘一些重要的支持电路概念,同时让我们回到无线电话题。

\subsection{为什么是三角形?}

当研究三相电力系统时,你很快就会遇到 \textbf{星形}(发音为"why")和 \textbf{三角形} 这两个术语。这些指的是三个独立相位相对于中性参考的连接方式。图 1 说明了基本概念。这里显示的线圈 — 典型的电机或变压器绕组 — 也可以是电压或电流源、电阻器、电容器或通用阻抗。每种系统名称的起源很清楚 — 每种系统的示意图都呈 \textbf{Y} 或 \textbf{Δ} 形状。

\begin{figure}[H]
    \centering
    \includegraphics[width=0.7\linewidth]{00007.jpeg}
    \caption{星形和三角形连接}
    \label{fig:wye_delta_cn}
\end{figure}

在星形系统中,三个相位(a、b 和 c)都共享一个公共中性点,因此有四个连接:A、B 和 C 相(也通常标记为 1、2 和 3 线)和中性线。在三角形系统中,没有公共中性连接,因为源或负载连接成一个环路。这意味着只有三个连接和它们之间的电压。(可以通过参考文章中讨论的各种技术在三角形网络中创建中性参考点。)

相位电压之间的角度始终为 120°,但角度是正还是负取决于 \textbf{相序},相序可以是 a-b-c 或 a-c-b(按角度增加的顺序)。在图 1A 中,相序是 a-b-c,这是正旋转。

问 — 三个电压如何在不包含任何电阻的环路中连接在一起,而电流不会无限大?如果是直流电压,我们确实会有问题!相反,这些是交流正弦波,具有相同的电压幅度 (V) 但不同的相位角。将电路周围的电压相加,得到 V∠0° + V∠120° + V∠240°。将相量转换为直角坐标可以计算总和:(V + j0) + (–0.5V + j0.866V) + (–0.5V – j0.866V) = 0。因此,环路周围的净电压为零,根本没有循环电流流动!

在星形和三角形系统中,负载都可以连接在相电压或线电压之间。在星形系统中,负载也可以连接在相电压和中性线之间。任何一种系统中的不平衡负载都可能导致大量 \textbf{误差电流} 流动。非线性负载(如开关模式电源和由 SCR 和 TRIAC 控制的负载)会产生谐波电流。这两种情况都会导致问题。显然,生成、传输和使用多相电力是一个复杂的主题。你可以在前面提到的参考资料和 \href{http://www.allaboutcircuits.com/vol_2/chpt_10/1.html}{www.allaboutcircuits.com/vol\_2/chpt\_10/1.html} 上了解更多信息。

但是星形和三角形电力系统与无线电有什么关系呢?从交流电源的角度来看,关系不大 — 除非你恰好需要一个 \textbf{非常} 大的电源。然而,我们在电路中一直使用星形和三角形 — 我们只是将它们称为 \textbf{Pi} 和 \textbf{T}!

\subsection{与 Pi 一起玩 T}

图 2 显示了两个由通用阻抗组成的电路 — 一个是 Pi 网络,你会在几乎所有电子管放大器中找到,另一个是 T 网络,你会在大多数天线调谐器中找到。图 2 还显示了 Pi 网络如何与三角形网络相同,T 网络如何与星形网络相同。谁知道呢?

\begin{figure}[H]
    \centering
    \includegraphics[width=0.7\linewidth]{00008.jpeg}
    \caption{作为三角形和星形的 Pi 和 T 网络}
    \label{fig:pi_t_networks_cn}
\end{figure}

这很有用,但还有另一个巧妙的技巧可以应用。你可以通过使用一种称为 \textbf{星形-三角形变换} 的数学方法,将一种类型的电路(Pi 或 T)转换为另一种类型的精确等效电路(T 或 Pi)。从输入和输出连接的角度来看,等效电路的行为将 \textbf{完全} 相同(带有稍后解释的警告)。以下等式显示了数学,尽管你可以使用在线计算器,例如 \href{http://www.elektro-energetika.cz/calculations/transfigurace.php?language=english}{www.elektro-energetika.cz/calculations/transfigurace.php?language=english} 上的这个计算器。\textsuperscript{2}

Pi(三角形)到 T(星形)

Z\textsubscript{A} = Z\textsubscript{1}Z\textsubscript{2} / ΣZ; Z\textsubscript{B} = Z\textsubscript{1}Z\textsubscript{3} / ΣZ; Z\textsubscript{C} = Z\textsubscript{2}Z\textsubscript{3} / ΣZ

其中 ΣZ = Z\textsubscript{1} + Z\textsubscript{2} + Z\textsubscript{3}

T(星形)到 Pi(三角形)

Z\textsubscript{1} = Z\textsubscript{P} / Z\textsubscript{C}, Z\textsubscript{2} = Z\textsubscript{P} / Z\textsubscript{B}, Z\textsubscript{3} = Z\textsubscript{P} / Z\textsubscript{A}

其中 Z\textsubscript{P} = Z\textsubscript{A}Z\textsubscript{B} + Z\textsubscript{A}Z\textsubscript{C} + Z\textsubscript{B}Z\textsubscript{C}

\subsection{使用变换}

不仅仅是数学技巧,将电路从一种形式转换为另一种形式可能非常有用。假设你开始的电路具有难以正常工作的组件值 — 也许它们是非常大或非常小的值。通过将电路从一种形式更改为另一种形式,组件值也会改变,可能变得更加合理。让我们尝试一个例子:

图 3A 中的电路是一个 40 dB T 网络衰减器,具有对称的 50 Ω 输入和输出阻抗。49 Ω 串联电阻不是问题,但 1 Ω 并联电阻可能会受到公共连接中额外布线电阻的显著影响。将电路转换为图 3B 中的 Pi 等效电路,电阻变为 51 和 2500 Ω。2500 Ω 电阻受布线电阻和其他小变化的影响要小得多。你可以使用在线计算器,例如 \href{http://www.microwaves101.com/encyclopedia/calcattenuator.cfm}{www.microwaves101.com/encyclopedia/calcattenuator.cfm} 或 Pi 和 T 衰减器的电阻值表,让你选择最合理的形式。\textsuperscript{3}

\begin{figure}[H]
    \centering
    \includegraphics[width=0.7\linewidth]{00009.jpeg}
    \caption{T 网络衰减器及其 Pi 等效电路}
    \label{fig:attenuator_example_cn}
\end{figure}

这也可以用于由电抗(L 和 C)组成的电路。让我们尝试一个例子,我们在 7 MHz 频率和电路 Q 值为 6 的情况下,将 800 Ω 的输出阻抗转换为 50 Ω。如果我们使用典型的天线调谐器 T 网络,带有串联电容器和并联电感器,组件值为 C\textsubscript{IN} = 76 pF,L = 5.9 μH,C\textsubscript{OUT} = 25 pF(使用 T 匹配计算器在 \href{http://www.eeweb.com/toolbox/t-match}{www.eeweb.com/toolbox/t-match} 计算)。L 的值很好,但 C\textsubscript{IN} 和 C\textsubscript{OUT} 的值很小,以至于不需要太多杂散电容就会扰乱网络的调谐,使设置变得"敏感"。此外,串联电容器的电抗越高(分别为 –j300 和 –j910 Ω),给定电流下它们两端的电压就越高。

如果将 T(星形)网络转换为 Pi(三角形)网络会发生什么?结果是电抗排列为输入(j 44 Ω 或 1 μH)和输出(j 132 Ω 或 3 μH)的并联电感器,以及更大的串联电容器(–j155 Ω 或 147 pF),这将减少杂散电容的影响并降低串联电抗。在线计算器还可以将 Pi 网络更改为串联 L 形式,如果更方便的话。\textsuperscript{4}

这是前面提到的警告:使用 LC 网络时,重要的是要记住,当使用 LC 网络时,变换 \textbf{仅} 在组件具有你指定的电抗的频率下工作。随着频率的变化,电抗也会变化,你必须重新计算组件值以获得完全等效的电路。使用可变组件允许你在不同频率下使用电路。

将变换计算器放入你的软件工具箱中,当你尝试选择和设计阻抗匹配电路时非常有帮助。这在 QRO 功率水平上尤为重要,因为需要重型(即昂贵)组件来承受高电压和电流。在你想要的频率和阻抗范围内进行几次设计迭代,有助于避免极端组件值以及通常随之而来的高电压和电流。

\subsection{参考资料}

\begin{enumerate}
    \item 所有以前的"实践无线电"实验都可在 ARRL 成员的 \href{http://www.arrl.org/hands-on-radio}{www.arrl.org/hands-on-radio} 获取。
    \item 计算器的第 1 步要求输入"复数的形状",这意味着你应该为阻抗选择矩形(R + jX)或相量(Z∠θ)形式。在欧洲惯例中使用逗号作为小数点。即 1.0 变为 1,0。
    \item 参见《ARRL 手册》的组件数据和参考章节,可从你的 ARRL 经销商或 ARRL 商店获取,ARRL 订单号 6948。美国免费电话 888-277-5289,或 860-594-0355;传真 860-594-0303;\href{http://www.arrl.org/shop/}{www.arrl.org/shop/};\href{mailto:pubsales@arrl.org}{pubsales@arrl.org}。
    \item 计算器可以在串联 C(阻断直流电流)和串联 L(通过直流电流)网络形式之间切换。
\end{enumerate}

\chapter{实验 #130:通信扬声器}

\section{英文原文}

Okay, okay — enough with math and the phasors and the coordinates! Several columns dedicated to phase rotation and spinning around at the carrier frequency is enough to make anyone a bit dizzy, the author included.\textsuperscript{1} I'm sure we all need something more on the order of drilling and soldering, so let's return to the workbench and cobble together an accessory that has a home in every mobile station — the communications speaker. But we'll jazz it up a bit.

We all get speakers built into our mobile rigs. However, they are often chosen simply because they will fit in the box and not because they are the best solution for competing with wind and road noise in a vehicle, usually while trying to understand the limited fidelity speech of another operator, who may also be driving.

An external communication speaker is substantially larger and able to reproduce speech with better fidelity at volumes that can overcome ambient noise. Many vendors offer fine products in this regard and if what you need is only one speaker for one radio, that's probably the right solution.

But as our friends and family well know, one radio is rarely enough! Getting separate speakers for each rig leads to clutter, as well as a volume arms race as each radio is turned up louder and louder to be heard.

The most common multi-radio mobile installation has a pair of rigs, perhaps a ham radio and a scanner (or, as in the author's car, a VHF/UHF mobile FM transceiver and an HF rig). If you only have a single speaker, it's simple to add a switch and select one or the other. Yet it's pretty common to have both radios on at the same time — perhaps you are operating HF and keeping an ear out for a call on the repeater or vice versa.

\subsection{Passive Mixers}

The simplest way of being able to satisfy the requirement of listening to either or both radios at the same time is to substitute a \textbf{balance} control for the A/B switch. By adjusting the balance control, you can listen to either radio A, radio B or a combination of both. Balance should not be confused with \textbf{pan} (from "panoramic") which refers to positioning a particular audio source in multiple audio channels (\href{http://thedawstudio.com/Tips/PanPots.html}{thedawstudio.com/Tips/PanPots.html}).

Figure 1 shows a very simple way of being able to listen to either or both radios in a single speaker using a passive balance control. The speaker outputs of most radios can supply a few watts of audio power into the typical communications speaker impedance of 4 to 32 Ω. A fixed resistor is in series with each speaker output to isolate the individual radio audio outputs from each other. The variable resistor is connected so that when the wiper is at either end, the speaker gets full output from one radio and very little from the other.

\begin{figure}[H]
    \centering
    \includegraphics[width=0.7\linewidth]{00185.jpeg}
    \caption{Passive Balance Control}
    \label{fig:passive_balance}
\end{figure}

Because of differences in radio audio outputs, available speaker impedances, and your personal volume preferences, a range of values is shown on the schematic. If you want a lot of volume, choose lower values for the fixed resistors, with the tradeoff being higher minimum volume for the undesired channel. Wirewound resistors are fine in this low-frequency application. This simple circuit can be installed inside the speaker housing, too.

It may take a couple of tries before you get the right combination of volume settings on the radios and resistor values in the mixer. Because the radios are different, the fixed resistor values may need to be different, too. In fact, the fixed resistors can be replaced by potentiometers if you like.

\subsection{Active Mixing}

A more flexible method of controlling the volume from more than one source through a single speaker is to use an \textbf{active mixer} — an amplifier that combines audio from multiple inputs with each level adjustable in the output. There are many types of active mixers, ranging from a simple summing circuit based on an op-amp\textsuperscript{2} to sophisticated designs with two or more output channels that have pan, balance, and frequency equalization on each input. These may have a place in the well-rounded shack at home but we're talking about your mobile station, so let's not go overboard.

You probably already have a type of mixer in your vehicle — it's part of the audio entertainment system. A standard feature on most vehicle audio systems these days is an \textbf{AUX} (auxiliary) input with a three-conductor, stereo and a 1/8 inch phone jack mounted somewhere on the dashboard or console. Plug in your stereo audio player or smartphone and away you go. The audio from your ham rigs can be plugged into the stereo, too, if you make sure to keep the signal levels down. Figure 2 shows an audio attenuator circuit suitable for use with the audio system's \textbf{AUX} input.

\begin{figure}[H]
    \centering
    \includegraphics[width=0.7\linewidth]{00186.jpeg}
    \caption{Audio Attenuator for AUX Input}
    \label{fig:aux_attenuator}
\end{figure}

In this circuit, both radio outputs are still connected to a fixed resistor but now they are not connected together in the output balance control. Instead, a resistive divider limits the signal level into the \textbf{AUX} input. If the load connected to the resistive dividers is 32 Ω, typical of small headphones or earbuds, the voltage from each channel is attenuated by about 14 dB. How did I determine that? Since 32 Ω // 100 Ω ≈ 25 Ω (// is used to indicate "in parallel with"), the output is reduced by 20 log (25 / (100+25)) = 13.9 dB. If you need more volume, decrease the input fixed resistor value. A high audio system input impedance reduces attenuation to 6 dB.

With the output of the divider connected to audio system's \textbf{AUX} input you can listen to one radio in the left channel and the other in the right channel. The fidelity of my car's stereo system is a lot better than that of the speakers in the radio! Non-hams find the audio system's output a lot easier to listen to, as well. Hams have gotten used to really poor mobile audio with lots of distortion and no bass. You might be surprised at how good a radio can sound if its output isn't trying to overdrive a minimal speaker over the road noise.

\subsection{Customize It!}

Don't stop here — add more features. You can use a splitter at the \textbf{AUX} input for your audio player, but why not add a parallel jack and switch on the speaker housing? Add a headphone jack or adjustable resistors for independent level setting. Don't be afraid to experiment with different resistor values and configurations.

In my vehicle, I wanted to be able to switch the speaker on or off independently of the audio system so I could listen to both radios and some entertainment at the same time. Figure 3 shows how I have my circuit configured and Figure 4 shows a photo of the final product. The circuits of Figures 1 and 2 are connected "in parallel" to the radio audio outputs so that they can act independently.

\begin{figure}[H]
    \centering
    \includegraphics[width=0.7\linewidth]{00187.jpeg}
    \caption{Complete Circuit Diagram}
    \label{fig:complete_circuit}
\end{figure}

\begin{figure}[H]
    \centering
    \includegraphics[width=0.7\linewidth]{00188.jpeg}
    \caption{Final Product}
    \label{fig:final_product}
\end{figure}

I used what I had in my junk box. These values were "close enough" for reasonable radio volume settings. It got the job done and let me proceed with hamming it up. As a bonus, I have to say that listening to a big CW pileup with the widest IF filters when it's played through a powerful audio system is some kind of amazing. It's not quite opera, doc, but it's close!

\section{注释}

\begin{enumerate}
    \item All previous Hands-On Radio experiments are available to ARRL members at \href{http://www.arrl.org/hands-on-radio}{www.arrl.org/hands-on-radio}.
    \item Silver, Ward NØAX, "Experiment 3: Basic Operational Amplifiers," QST, April 2003, pp 63-64.
\end{enumerate}

\chapter{实验 #132:电阻网络}

\section{英文原文}

Resistors may seem to be the lowliest of components, but they are by far the most common. This month's column discusses several different types of useful resistor circuits that are handy designs to have in your toolbox. Dig in and try a few on your own.

\subsection{Finding a Needed Parallel Value}

We should all know the basic formulas for series and parallel resistors. In series, just add the values together: \textbf{R\textsubscript{SER} = R1 + R2 + R3…etc.} For parallel resistors, the equivalent value is the "reciprocal of the sum of reciprocals": \textbf{R\textsubscript{PAR} = 1 / (1/R1 + 1/R2 + 1/R3 + …etc).} Happily, for two resistors this simplifies to \textbf{R\textsubscript{PAR} = R1 R2 / (R1 + R2).} But what if you need to create a certain resulting value of \textbf{R\textsubscript{PAR}} — what two resistances in parallel can you use? Start by picking a value for \textbf{R1} that is higher than the desired value.

\begin{center}
R2 = R\textsubscript{PAR} R1 / (R1 – R\textsubscript{PAR}) [1]
\end{center}

This is a handy formula to have in a calculator or spreadsheet so I've created an \textbf{Excel} spreadsheet containing all the formulas in this column. It's available on the "Hands-On Radio" web page.\textsuperscript{1}

\subsection{Power Dissipation}

Power dissipation for a single resistor is equal to \textbf{V\textsuperscript{2}/R} or \textbf{I\textsuperscript{2}R}, but what happens when you have more than one resistor in series or parallel? It sounds complicated, until you realize that in a series circuit the same current flows through all of the resistors — use \textbf{I\textsuperscript{2}R}. Similarly, the same voltage appears across all of the resistors in a parallel circuit — use \textbf{V\textsuperscript{2}/R}.

\subsection{Voltage Dividers}

If you are given both resistor values in the voltage divider shown in Figure 1 and for a moment ignore the load resistor, RL, it is straightforward to figure the output voltage: \textbf{V\textsubscript{OUT} = V\textsubscript{IN} R2 / (R1 + R2)}. But what if you need a specific division ratio (\textbf{V\textsubscript{OUT} / V\textsubscript{IN}}) and want to know what resistor values to use? Start by choosing the total resistance of the divider, \textbf{R\textsubscript{TOT} = R1 + R2}, then

\begin{figure}[H]
    \centering
    \includegraphics[width=0.7\linewidth]{00010.jpeg}
    \caption{Voltage Divider}
    \label{fig:voltage_divider}
\end{figure}

\begin{center}
R2 = R\textsubscript{TOT} (V\textsubscript{OUT} / V\textsubscript{IN}) and R1 = R\textsubscript{TOT} (1 – V\textsubscript{OUT} / V\textsubscript{IN}) [2]
\end{center}

Or maybe you know the division ratio and already have a value for \textbf{R2} — note that the division ratio is inverted in this equation:

\begin{center}
R1 = R2 (V\textsubscript{IN} / V\textsubscript{OUT} – 1) [3]
\end{center}

The voltage division ratio is also affected by the load, \textbf{RL}, attached across \textbf{R2} at the output of the divider. If you need a precise ratio, remember to include the effect of \textbf{RL} by substituting \textbf{RL // R2} for \textbf{R2} in equations 2 and 3. (The symbol // means "in parallel with.")

A good rule is that to avoid large effects on the voltage division ratio, \textbf{RL} should be at least 10 times greater than \textbf{R2}. If this is not practical, an alternative is to connect the voltage divider output to a high-impedance buffer circuit such as an emitter-follower ("Hands-On Radio" Experiment #2) or op-amp voltage follower (Experiment #3).

\subsection{Current Dividers}

Sometimes, instead of dividing a voltage, you need to divide a current, such as when a load current is too large to measure directly. For resistors in parallel as in Figure 2A, the current through any one of them, \textbf{R\textsubscript{N}}, is

\begin{figure}[H]
    \centering
    \includegraphics[width=0.7\linewidth]{00011.jpeg}
    \caption{Current Divider}
    \label{fig:current_divider}
\end{figure}

\begin{center}
I\textsubscript{N} = I\textsubscript{TOT} (R\textsubscript{PAR} / R\textsubscript{N}) [4]
\end{center}

where \textbf{I\textsubscript{TOT}} = total current through all resistors and \textbf{R\textsubscript{PAR}} is the combined parallel resistance of all the resistors.

Let's do a sample calculation for two resistors. If you have 1.5 mA flowing through the parallel combination of a 1 kΩ and a 470 Ω resistor, how much current is flowing through the 470 Ω resistor? And through the 1 kΩ resistor?

I\textsubscript{470} = 1.5 mA (1 kΩ // 470 Ω / 470 Ω) = 1.5 mA (320 Ω / 470 Ω) = 1.0 mA

I\textsubscript{1k} = 1.5 mA (1 kΩ // 470 Ω / 1 kΩ) = 1.5 mA (320 Ω / 1 kΩ) = 0.5 mA

This makes sense — the lower-value resistor carries more current. It's always good to check on your calculations!

Here's a more common situation: diverting a specific fraction of the circuit's total current, \textbf{I\textsubscript{TOT}}, through a measurement circuit as in Figure 2B. With two resistors in parallel and \textbf{R2} the resistor in the measurement path, the current division ratio is \textbf{I\textsubscript{2} / I\textsubscript{TOT} = I\textsubscript{2} / (I\textsubscript{1} + I\textsubscript{2})}. To limit the effects of adding resistance to the circuit, you must also pick a suitably small value for the total amount of resistance you are adding, \textbf{R\textsubscript{PAR} = R1 // R2}. By rearranging equation 4, we can find \textbf{R2 = R\textsubscript{PAR} / (I2 / I\textsubscript{TOT})} and rearranging equation 1 gives us the value of \textbf{R1}.

An example would be nice! Let's say I want to divert 1\% of a 750 mA current through the measurement branch of my circuit. (The division ratio for 1% is 0.01.) If I can add 1 Ω of total resistance to the circuit, I have \textbf{R\textsubscript{PAR} = 1}. Start by finding the measurement branch resistance, \textbf{R2} = 1 / (0.01) = 100 Ω. The main current-carrying resistance must be \textbf{R1} = 1 × 100 / (100 – 1) = 100 / 99 = 1.01 Ω. One more thing — make sure \textbf{R1} can dissipate the total power of 0.75\textsuperscript{2} × 1.01 = 0.57 W. And also remember that resistors change value when they heat up, so it might be smart to use a 5 W or larger resistor for \textbf{R1} to keep its temperature rise to a minimum.

\subsection{Meter Shunts}

Another very common application of current dividers is the meter shunt. When you see a meter calibrated in amps of current, the delicate meter movement itself is not carrying all that current. Usually the meter is a milli- or microammeter that measures a small sample of current diverted through it as we just described. Analog meters with a full-scale current (\textbf{I\textsubscript{FS}}) of 100 µA to 10 mA are fairly common. Here's how to use them for measuring larger currents.

If you don't know the meter's \textbf{I\textsubscript{FS}}, connect a 1.5 V battery or other low voltage source, a 10 kΩ potentiometer, and DMM (in current-measuring mode) in series with the meter. Adjust the resistance for full-scale current (start at maximum resistance) and read \textbf{I\textsubscript{FS}} from the DMM.

Next determine the meter's internal resistance, \textbf{R\textsubscript{M}}, but don't hook it up to an ohmmeter! The current from the ohmmeter could damage a sensitive meter movement, so use the circuit in Figure 3. Meters with \textbf{I\textsubscript{FS}} of less than 1 mA typically have an \textbf{R\textsubscript{M}} of 1000-5000 Ω and for I\textsubscript{FS} of 1 to 10 mA, \textbf{R\textsubscript{M}} ranges from a few to several hundred ohms. Open S1 and adjust \textbf{R1} for full-scale deflection. Now close S1 and adjust \textbf{R2} for half-scale deflection. Remove the voltage source, disconnect \textbf{R2} and measure it — it has the same value as \textbf{R\textsubscript{M}}.

\begin{figure}[H]
    \centering
    \includegraphics[width=0.7\linewidth]{00012.jpeg}
    \caption{Meter Shunt Circuit}
    \label{fig:meter_shunt}
\end{figure}

Now that you know the meter's full-scale current and internal resistance, you can calculate the value of shunt resistance necessary for a current, \textbf{I}, through the shunt to cause a full-scale reading on the meter:

\begin{center}
R\textsubscript{SHUNT} = R\textsubscript{M} I\textsubscript{FS} / (I – I\textsubscript{FS}) [5]
\end{center}

Let's give this a try: find the shunt resistance that causes a 1 mA full-scale meter with a resistance of 150 Ω to indicate at full-scale with a current of 1 A through the shunt.

\begin{center}
R\textsubscript{SHUNT} = 150 Ω × 0.001 A / (1 – 0.001) = 0.150 Ω
\end{center}

Remember that the shunt may have to dissipate some power and that it shouldn't get very warm to avoid temperature effects. A common choice for shunt resistors is a coil of small enameled wire. Tables of resistance for copper wire of different sizes are available in \textbf{The ARRL Handbook} and other sources.\textsuperscript{2} For example, AWG 20 wire can carry 1 A of current and has a resistance of 10.12 Ω per 1000 feet. We'll need 1000 × 0.150 / 10.12 = 14.82 feet of wire to make our shunt.

These circuits are simple but they are everywhere, once you start looking for them. Troubleshooting and circuit design become easier if you understand how they work.

\subsection{References}

\begin{enumerate}
    \item All previous Hands-On Radio experiments are available to ARRL members at \href{http://www.arrl.org/hands-on-radio}{www.arrl.org/hands-on-radio}.
    \item The ARRL Handbook is available from your ARRL dealer, or from the ARRL Store, ARRL order no. 0007. Telephone toll-free in the US 888-277-5289, or 860-594-0355; fax 860-594-0303; \href{http://www.arrl.org/shop/}{www.arrl.org/shop/}; \href{mailto:pubsales@arrl.org}{pubsales@arrl.org}.
\end{enumerate}

\section{中文翻译}

电阻器可能看起来是最低级的组件,但它们是迄今为止最常见的。本月的专栏讨论了几种不同类型的有用电阻器电路,这些电路设计在你的工具箱中很方便。深入研究并自己尝试一些。

\subsection{找到所需的并联值}

我们都应该知道串联和并联电阻器的基本公式。串联时,只需将值相加:\textbf{R\textsubscript{SER} = R1 + R2 + R3…等}。对于并联电阻器,等效值是"倒数之和的倒数":\textbf{R\textsubscript{PAR} = 1 / (1/R1 + 1/R2 + 1/R3 + …等)}。幸运的是,对于两个电阻器,这简化为\textbf{R\textsubscript{PAR} = R1 R2 / (R1 + R2)}。但是,如果你需要创建某个特定的\textbf{R\textsubscript{PAR}}值 — 你可以使用哪两个并联电阻?首先选择一个高于所需值的\textbf{R1}值。

\begin{center}
R2 = R\textsubscript{PAR} R1 / (R1 – R\textsubscript{PAR}) [1]
\end{center}

这是一个在计算器或电子表格中很方便的公式,所以我创建了一个包含本专栏中所有公式的\textbf{Excel}电子表格。它可以在"实践无线电"网页上找到。\textsuperscript{1}

\subsection{功率耗散}

单个电阻器的功率耗散等于\textbf{V\textsuperscript{2}/R}或\textbf{I\textsuperscript{2}R},但当你有多个串联或并联电阻器时会发生什么?这听起来很复杂,直到你意识到在串联电路中,相同的电流流过所有电阻器 — 使用\textbf{I\textsuperscript{2}R}。同样,相同的电压出现在并联电路中的所有电阻器上 — 使用\textbf{V\textsuperscript{2}/R}。

\subsection{电压分压器}

如果你在图 1 所示的电压分压器中给定了两个电阻器值,并且暂时忽略负载电阻 RL,那么计算输出电压很简单:\textbf{V\textsubscript{OUT} = V\textsubscript{IN} R2 / (R1 + R2)}。但是,如果你需要特定的分压比(\textbf{V\textsubscript{OUT} / V\textsubscript{IN}})并想知道使用什么电阻器值呢?首先选择分压器的总电阻,\textbf{R\textsubscript{TOT} = R1 + R2},然后

\begin{figure}[H]
    \centering
    \includegraphics[width=0.7\linewidth]{00010.jpeg}
    \caption{电压分压器}
    \label{fig:voltage_divider_cn}
\end{figure}

\begin{center}
R2 = R\textsubscript{TOT} (V\textsubscript{OUT} / V\textsubscript{IN}) 和 R1 = R\textsubscript{TOT} (1 – V\textsubscript{OUT} / V\textsubscript{IN}) [2]
\end{center}

或者,你可能知道分压比并且已经有了\textbf{R2}的值 — 注意这个等式中的分压比是反转的:

\begin{center}
R1 = R2 (V\textsubscript{IN} / V\textsubscript{OUT} – 1) [3]
\end{center}

分压比还受到连接在分压器输出端\textbf{R2}两端的负载\textbf{RL}的影响。如果你需要精确的比率,请记住通过在等式 2 和 3 中用\textbf{RL // R2}代替\textbf{R2}来包含\textbf{RL}的影响。(符号//表示"与...并联"。)

一个好的规则是,为了避免对分压比的大影响,\textbf{RL}应该至少是\textbf{R2}的10倍。如果这不实际,另一种方法是将电压分压器输出连接到高阻抗缓冲电路,例如射极跟随器("实践无线电"实验 #2)或运算放大器电压跟随器(实验 #3)。

\subsection{电流分压器}

有时,你需要分流电流而不是分压,例如当负载电流太大而无法直接测量时。对于如图 2A 所示的并联电阻器,通过其中任何一个\textbf{R\textsubscript{N}}的电流是

\begin{figure}[H]
    \centering
    \includegraphics[width=0.7\linewidth]{00011.jpeg}
    \caption{电流分压器}
    \label{fig:current_divider_cn}
\end{figure}

\begin{center}
I\textsubscript{N} = I\textsubscript{TOT} (R\textsubscript{PAR} / R\textsubscript{N}) [4]
\end{center}

其中\textbf{I\textsubscript{TOT}} = 通过所有电阻器的总电流,\textbf{R\textsubscript{PAR}}是所有电阻器的组合并联电阻。

让我们为两个电阻器做一个示例计算。如果你有1.5 mA的电流流过1 kΩ和470 Ω电阻器的并联组合,那么有多少电流流过470 Ω电阻器?流过1 kΩ电阻器的电流是多少?

I\textsubscript{470} = 1.5 mA (1 kΩ // 470 Ω / 470 Ω) = 1.5 mA (320 Ω / 470 Ω) = 1.0 mA

I\textsubscript{1k} = 1.5 mA (1 kΩ // 470 Ω / 1 kΩ) = 1.5 mA (320 Ω / 1 kΩ) = 0.5 mA

这是有道理的 — 较低值的电阻器承载更多电流。检查你的计算总是好的!

这里有一个更常见的情况:如图 2B 所示,将电路总电流\textbf{I\textsubscript{TOT}}的特定部分分流通过测量电路。对于两个并联电阻器,其中\textbf{R2}是测量路径中的电阻器,电流分流比是\textbf{I\textsubscript{2} / I\textsubscript{TOT} = I\textsubscript{2} / (I\textsubscript{1} + I\textsubscript{2})}。为了限制向电路添加电阻的影响,你还必须为要添加的总电阻量选择一个适当小的值,\textbf{R\textsubscript{PAR} = R1 // R2}。通过重新排列等式4,我们可以找到\textbf{R2 = R\textsubscript{PAR} / (I2 / I\textsubscript{TOT})},重新排列等式1给出了\textbf{R1}的值。

举个例子会很好!假设我想将750 mA电流的1%分流通过我的电路的测量分支。(1%的分流比是0.01。)如果我可以向电路添加1 Ω的总电阻,我有\textbf{R\textsubscript{PAR} = 1}。首先找到测量分支电阻,\textbf{R2} = 1 / (0.01) = 100 Ω。主电流承载电阻必须是\textbf{R1} = 1 × 100 / (100 – 1) = 100 / 99 = 1.01 Ω。还有一件事 — 确保\textbf{R1}可以耗散0.75\textsuperscript{2} × 1.01 = 0.57 W的总功率。还要记住,电阻器在加热时会改变值,所以使用5 W或更大的电阻器作为\textbf{R1}来保持其温度升高最小可能是明智的。

\subsection{电表分流器}

电流分压器的另一个非常常见的应用是电表分流器。当你看到一个以安培为单位校准的电表时,精致的电表机芯本身并没有承载所有电流。通常,电表是一个毫安或微安表,测量通过它分流的小样本电流,就像我们刚才描述的那样。满量程电流(\textbf{I\textsubscript{FS}})为100 µA至10 mA的模拟电表相当常见。以下是如何使用它们来测量更大的电流。

如果你不知道电表的\textbf{I\textsubscript{FS}},请连接一个1.5 V电池或其他低压源、一个10 kΩ电位器和DMM(在电流测量模式下)与电表串联。调整电阻以获得满量程电流(从最大电阻开始),并从DMM读取\textbf{I\textsubscript{FS}}。

接下来确定电表的内部电阻\textbf{R\textsubscript{M}},但不要将其连接到欧姆表!欧姆表的电流可能会损坏敏感的电表机芯,所以使用图3中的电路。\textbf{I\textsubscript{FS}}小于1 mA的电表通常具有1000-5000 Ω的\textbf{R\textsubscript{M}},对于I\textsubscript{FS}为1至10 mA的电表,\textbf{R\textsubscript{M}}范围从几欧姆到几百欧姆。打开S1并调整\textbf{R1}以获得满量程偏转。现在关闭S1并调整\textbf{R2}以获得半量程偏转。移除电压源,断开\textbf{R2}并测量它 — 它与\textbf{R\textsubscript{M}}具有相同的值。

\begin{figure}[H]
    \centering
    \includegraphics[width=0.7\linewidth]{00012.jpeg}
    \caption{电表分流器电路}
    \label{fig:meter_shunt_cn}
\end{figure}

现在你知道了电表的满量程电流和内部电阻,你可以计算分流器电阻的值,该值对于通过分流器的电流\textbf{I}在电表上引起满量程读数是必要的:

\begin{center}
R\textsubscript{SHUNT} = R\textsubscript{M} I\textsubscript{FS} / (I – I\textsubscript{FS}) [5]
\end{center}

让我们尝试一下:找到使1 mA满量程、电阻为150 Ω的电表在分流器通过1 A电流时指示满量程的分流电阻。

\begin{center}
R\textsubscript{SHUNT} = 150 Ω × 0.001 A / (1 – 0.001) = 0.150 Ω
\end{center}

记住,分流器可能必须耗散一些功率,并且它不应该变得很热以避免温度影响。分流电阻器的常见选择是小漆包线线圈。不同尺寸铜线的电阻表可在\textbf{The ARRL Handbook}和其他来源中找到。\textsuperscript{2}例如,AWG 20线可以承载1 A的电流,每1000英尺的电阻为10.12 Ω。我们需要1000 × 0.150 / 10.12 = 14.82英尺的电线来制作我们的分流器。

这些电路很简单,但一旦你开始寻找它们,它们无处不在。如果你了解它们的工作原理,故障排除和电路设计会变得更容易。

\subsection{参考资料}

\begin{enumerate}
    \item 所有以前的"实践无线电"实验都可在 ARRL 成员的\href{http://www.arrl.org/hands-on-radio}{www.arrl.org/hands-on-radio}获取。
    \item The ARRL Handbook 可从你的 ARRL 经销商或 ARRL 商店获取,ARRL 订单号 0007。美国免费电话 888-277-5289,或 860-594-0355;传真 860-594-0303;\href{http://www.arrl.org/shop/}{www.arrl.org/shop/};\href{mailto:pubsales@arrl.org}{pubsales@arrl.org}。
\end{enumerate}

\chapter{实验 #141:窗口比较器和零点检测器}

\section{英文原文}

In the early days of Hands-On Radio, one of the first experiments (#11) described comparator circuits.\textsuperscript{1} These are pretty handy for detecting when the voltage at one of their inputs is greater or less than at the other input. I'm finally getting around to describing two variations of the comparator — the \textbf{window comparator} and the \textbf{null detector}. There are plenty of applications for these circuits in radio and for test circuits.

\subsection{The Window Comparator}

Figure 1 shows a comparator circuit that incorporates hysteresis to shift the setpoint "a little bit" when the output changes state. That helps prevent "chatter" — rapid on-off switching of the output due to noise on the input signal or voltage reference. This comparator-with-hysteresis is known as a \textbf{Schmitt trigger}. Schmitt triggers are so useful that logic ICs intended for interfacing to switch and analog signals use them at gate inputs. Typical parts are the CD4093 and CD40106, along with variations of the 7414.

\begin{figure}[H]
    \centering
    \includegraphics[width=0.7\linewidth]{00013.jpeg}
    \caption{Comparator with Hysteresis}
    \label{fig:comparator_hysteresis}
\end{figure}

The comparator provides information about a single level — is the input voltage higher or lower than the setpoint? It is often more useful to know if an input voltage is between an upper and lower limit. This type of comparator is a \textbf{window comparator}, sometimes called a \textbf{range detector}.

The window comparator in Figure 2 is simply a pair of comparator circuits — one detects whether the input is below the lower limit and the other detects when it is higher than the upper limit. Hysteresis is not included in order to keep the circuit simple. (Power and ground for the LM393 are on Pins 8 and 4 in the DIP and SOIC packages.)

\begin{figure}[H]
    \centering
    \includegraphics[width=0.7\linewidth]{00014.jpeg}
    \caption{Window Comparator Circuit}
    \label{fig:window_comparator}
\end{figure}

The comparator setpoint voltages (labeled Upper Limit and Lower Limit) are easy to calculate: Upper Limit = V+ × (R2+R3)/(R1+R2+R3) = 2/3 V+ and Lower Limit = 1⁄3 V+. If V+ is 12 V, then the two limits are 8 V and 4 V, creating a 4 V window from 4 to 8 V.

Remember that the output of the comparator is an open-collector transistor acting as a switch that can be open or closed. By connecting the output to a pull-up resistor (R4 in Figure 2) the output voltage is \textbf{HIGH} when the transistor is off (approximately V+) and \textbf{LOW} (at the transistor's collector-emitter saturation voltage) when the transistor is on. When the comparator's non-inverting (+) input is at a higher voltage than the non-inverting (–) input, the output is \textbf{HIGH}. When the opposite is true (non-inverting input lower than the inverting input) then the open-collector transistor output is \textbf{LOW}.

Connecting both comparator outputs together creates a \textbf{wired-AND} gate in which \textbf{both} output transistors must be \textbf{HIGH} to allow the voltage at R4 to rise and turn on Q1, lighting the LED. If either comparator output is \textbf{LOW}, Q1 remains off. For the window comparator, if the input voltage is higher than Upper Limit, the output of U1A is \textbf{LOW} and Q1 is off. If the input voltage is lower than Lower Limit, the output of U1B is \textbf{LOW} and Q1 is off. The only condition in which both comparator outputs are \textbf{HIGH} occurs when the input voltage is between Upper Limit and Lower Limit.

\subsection{Look In the Window}

Build the circuit, using either a variable power supply or adjustable resistor to create the input voltage. Vary the input voltage from below 4 V to above 8 V, while watching the LED. It will turn on when the input voltage is in the target window.

Now experiment with the value of R2 by substituting fixed-value resistors or using an adjustable resistor. What happens to the window range as R2 is increased? (The window range increases.) Why? (A higher value for R2 lowers Lower Limit and increases Upper Limit.) Decrease R2 and verify that the opposite effect occurs. Return R2 to 3.3 kΩ and increase the value of R3 — what happens to the window? (The window shifts higher.) Return R3 to 3.3 kΩ and increase the value of R1 to observe the effect on the window. (It shifts lower.)

If you used a different voltage for V+ the values of Upper Limit and Lower Limit would change right along with it. What if there was noise on the power supply output? Or if you were using a battery supply — what would happen as the battery discharged? Obviously, it's not a good idea to use an unregulated voltage source for your setpoints. Some kind of regulated voltage is necessary. (Note that if you use a bipolar supply with Pin 4 of U1 and the negative end of R3 connected to the negative supply voltage, the window limits can be made positive or negative.)

This is such a common circuit that many manufacturers offer single and dual comparator ICs with a voltage reference built in. One such IC is the Linear Technology LTC1442. You can also use a Zener diode or voltage reference IC. Regardless, for consistent and reliable operation, the voltage source to which the voltage divider string of R1, R2, and R3 must be clean and stable. It is good practice, particularly around RF sources like transmitters, to include small-value capacitors (such as 0.01 µF) from each setpoint to ground.

Window comparators are useful for lots of radio-related chores: making sure your power supply or battery voltage is in the right range, for example. If you sample some RF power and use a peak detector as described in the previous column, a window comparator can be used to make sure the power is within a desired range. Another use is for decoding the Icom transceiver BAND data output that changes from 0 to 7 V with the band selected. K6XX has designed a circuit that uses an LED meter driver IC with many window comparators built in to sort the voltage levels into digital-compatible outputs.

\subsection{Null Detector}

There are many instances in which it's useful to adjust a circuit or system to produce a voltage exactly equal to some reference level, such as when balancing a bridge circuit or trying to adjust dc offset in an amplifier or demodulator circuit. A special type of window detector called a \textbf{null detector} responds to a narrow range right around a 0 V difference between its two inputs. "Null" in this case means "no difference between" the voltage being measured and a reference voltage.

Figure 3A shows a null detector circuit that uses an analog center-zero meter to allow fine adjustments to obtain the null. The input section consisting of U1A, R1, and R2 sets the gain of the detector equal to R2 / R1. The higher the gain, the more sensitive the detector. Note that U1 is a dual op-amp and not a dual comparator! A bipolar power supply of at least ±6V should be used.

\begin{figure}[H]
    \centering
    \includegraphics[width=0.7\linewidth]{00015.jpeg}
    \caption{Null Detector Circuit}
    \label{fig:null_detector}
\end{figure}

U1B converts the voltage at the non-inverting input into current through the meter. R3 calibrates the voltage-to-current ratio: I = V / R3. If R3 is 1kΩ, then 1 V at the non-inverting input to U1B is converted to 1 mA of current through the meter. (1-0-1 mA center-zero meters are often used with a shunt for battery charging systems.)

To calibrate the meter; set R2 to 10 kΩ so that gain equals 1, ground V\textsubscript{ref}, then connect a voltage of 0 to 1 V to the input. Measure the voltage at the output of U1A and adjust R2 so that the meter shows the same number of mA. i.e. – for a voltage of +0.5 V, the meter should indicate 0.5 mA in the positive direction.

To use the null detector, connect V\textsubscript{ref} to the reference voltage desired, whether 0 V or some other value, and adjust input gain (R2) to the desired sensitivity. For initial adjustments of the external circuit being measured, keep gain low. As the null is reached, increase input gain for more and more sensitive adjustments.

An alternative to the analog meter is shown in Figure 3B, a pair of LEDs connected back-to-back. The calibration pot, R3, is adjusted with V\textsubscript{ref} and Input shorted so that both LEDs are dark. (Noise, hum, or ripple on either signal may cause the LEDs to light dimly. A 0.1 µF capacitor across R2 creates a low-pass filter to avoid responding to ac components.) If the input voltage is higher than V\textsubscript{ref}, then the bottom LED will be dark and the top LED will be lit. This is often sufficient to adjust a circuit or balance a bridge without requiring a high-precision display.

\subsection{Parts List}

\begin{itemize}[leftmargin=2cm]
    \item LM393 dual comparator (or equivalent)
    \item TL082 dual op-amp (or equivalent)
    \item 2N3904 NPN transistor
    \item Standard LED (2)
    \item 470 Ω, 3.3 kΩ (3 ea.), 10 kΩ (2 ea.) ¼-watt, 5% resistors
    \item 5 kΩ, 10 kΩ trimpot
    \item 1-0-1 mA center-zero dc milliammeter (optional)
\end{itemize}

\subsection{References}

\begin{enumerate}
    \item All previous Hands-On Radio experiments are available to ARRL members at \href{http://www.arrl.org/hands-on-radio}{www.arrl.org/hands-on-radio}.
\end{enumerate}

\section{中文翻译}

在《实践无线电》的早期,第一个实验之一(#11)描述了比较器电路。\textsuperscript{1} 这些电路对于检测其输入之一的电压何时大于或小于另一个输入的电压非常方便。我终于开始描述比较器的两种变体 —— \textbf{窗口比较器} 和 \textbf{零点检测器}。这些电路在无线电和测试电路中有许多应用。

\subsection{窗口比较器}

图 1 显示了一个比较器电路,该电路包含迟滞,以在输出状态改变时将设定点"稍微"偏移。这有助于防止"抖动" —— 由于输入信号或电压参考上的噪声导致输出快速开关切换。这种带迟滞的比较器被称为 \textbf{施密特触发器}。施密特触发器非常有用,以至于旨在与开关和模拟信号接口的逻辑 IC 在门输入处使用它们。典型的部件包括 CD4093 和 CD40106,以及 7414 的变体。

\begin{figure}[H]
    \centering
    \includegraphics[width=0.7\linewidth]{00013.jpeg}
    \caption{带迟滞的比较器}
    \label{fig:comparator_hysteresis_cn}
\end{figure}

比较器提供有关单个电平的信息 —— 输入电压是高于还是低于设定点?通常更有用的是知道输入电压是否在上限和下限之间。这种类型的比较器是 \textbf{窗口比较器},有时称为 \textbf{范围检测器}。

图 2 中的窗口比较器只是一对比较器电路 —— 一个检测输入是否低于下限,另一个检测输入何时高于上限。为了保持电路简单,不包括迟滞。(LM393 的电源和接地在 DIP 和 SOIC 封装的引脚 8 和 4 上。)

\begin{figure}[H]
    \centering
    \includegraphics[width=0.7\linewidth]{00014.jpeg}
    \caption{窗口比较器电路}
    \label{fig:window_comparator_cn}
\end{figure}

比较器设定点电压(标记为上限和下限)很容易计算:上限 = V+ × (R2+R3)/(R1+R2+R3) = 2/3 V+,下限 = 1⁄3 V+。如果 V+ 是 12 V,那么两个限制是 8 V 和 4 V,创建一个从 4 到 8 V 的 4 V 窗口。

请记住,比较器的输出是一个开集电极晶体管,充当可以打开或关闭的开关。通过将输出连接到上拉电阻(图 2 中的 R4),当晶体管关闭时,输出电压为 \textbf{高}(约为 V+),当晶体管打开时,输出电压为 \textbf{低}(在晶体管的集电极-发射极饱和电压)。当比较器的同相 (+) 输入电压高于反相 (–) 输入电压时,输出为 \textbf{高}。当相反情况发生时(同相输入低于反相输入),则开集电极晶体管输出为 \textbf{低}。

将两个比较器输出连接在一起创建一个 \textbf{线与} 门,其中 \textbf{两个} 输出晶体管必须都为 \textbf{高} 才能允许 R4 处的电压上升并打开 Q1,点亮 LED。如果任一比较器输出为 \textbf{低},Q1 保持关闭。对于窗口比较器,如果输入电压高于上限,U1A 的输出为 \textbf{低},Q1 关闭。如果输入电压低于下限,U1B 的输出为 \textbf{低},Q1 关闭。只有当输入电压在上限和下限之间时,两个比较器输出都为 \textbf{高} 的情况才会发生。

\subsection{查看窗口}

构建电路,使用可变电源或可调电阻创建输入电压。将输入电压从低于 4 V 变化到高于 8 V,同时观察 LED。当输入电压在目标窗口内时,它会打开。

现在通过替换固定值电阻器或使用可调电阻器来实验 R2 的值。随着 R2 的增加,窗口范围会发生什么变化?(窗口范围增加。)为什么?(R2 的值越高,下限越低,上限越高。)减小 R2 并验证相反的效果发生。将 R2 恢复到 3.3 kΩ 并增加 R3 的值 —— 窗口会发生什么变化?(窗口向上移动。)将 R3 恢复到 3.3 kΩ 并增加 R1 的值以观察对窗口的影响。(它向下移动。)

如果你为 V+ 使用不同的电压,上限和下限的值会随之改变。如果电源输出上有噪声会怎样?或者如果你使用电池供电 —— 当电池放电时会发生什么?显然,使用未调节的电压源作为设定点不是一个好主意。需要某种调节电压。(请注意,如果你使用双极性电源,U1 的引脚 4 和 R3 的负端连接到负电源电压,则窗口限制可以是正的或负的。)

这是一个如此常见的电路,以至于许多制造商提供内置电压参考的单路和双路比较器 IC。一个这样的 IC 是 Linear Technology LTC1442。你也可以使用齐纳二极管或电压参考 IC。无论如何,为了获得一致和可靠的操作,R1、R2 和 R3 的分压器串所连接的电压源必须清洁且稳定。特别是在像发射机这样的 RF 源周围,良好的做法是在每个设定点到地之间包含小值电容器(例如 0.01 µF)。

窗口比较器对于许多与无线电相关的任务很有用:例如,确保你的电源或电池电压在正确的范围内。如果你采样一些 RF 功率并使用上一列中描述的峰值检测器,窗口比较器可用于确保功率在所需范围内。另一个用途是解码 Icom 收发器的 BAND 数据输出,该输出随所选频段从 0 变为 7 V。K6XX 设计了一个电路,使用内置许多窗口比较器的 LED 仪表驱动 IC 将电压电平分类为数字兼容输出。

\subsection{零点检测器}

在许多情况下,调整电路或系统以产生与某个参考电平完全相等的电压是有用的,例如在平衡桥接电路或尝试调整放大器或解调器电路中的直流偏移时。一种称为 \textbf{零点检测器} 的特殊类型的窗口检测器响应其两个输入之间恰好为 0 V 差异的窄范围。在这种情况下,"零点"意味着"被测电压与参考电压之间没有差异"。

图 3A 显示了一个零点检测器电路,该电路使用模拟中心零仪表来允许精细调整以获得零点。由 U1A、R1 和 R2 组成的输入部分将检测器的增益设置为等于 R2 / R1。增益越高,检测器越敏感。请注意,U1 是双运算放大器,而不是双比较器!应使用至少 ±6V 的双极性电源。

\begin{figure}[H]
    \centering
    \includegraphics[width=0.7\linewidth]{00015.jpeg}
    \caption{零点检测器电路}
    \label{fig:null_detector_cn}
\end{figure}

U1B 将同相输入处的电压转换为通过仪表的电流。R3 校准电压-电流比:I = V / R3。如果 R3 是 1kΩ,那么 U1B 同相输入处的 1 V 被转换为通过仪表的 1 mA 电流。(1-0-1 mA 中心零仪表通常与分流器一起用于电池充电系统。)

要校准仪表:将 R2 设置为 10 kΩ,使增益等于 1,将 V\textsubscript{ref} 接地,然后将 0 到 1 V 的电压连接到输入。测量 U1A 输出处的电压并调整 R2,使仪表显示相同数量的 mA。即 — 对于 +0.5 V 的电压,仪表应在正方向指示 0.5 mA。

要使用零点检测器,将 V\textsubscript{ref} 连接到所需的参考电压,无论是 0 V 还是其他值,并将输入增益 (R2) 调整到所需的灵敏度。对于被测外部电路的初始调整,保持增益较低。当达到零点时,增加输入增益以获得越来越敏感的调整。

模拟仪表的替代方案如图 3B 所示,是一对背对背连接的 LED。校准电位器 R3 调整为 V\textsubscript{ref} 和输入短路,使两个 LED 都变暗。(任一信号上的噪声、嗡嗡声或纹波可能导致 LED 暗淡点亮。R2 两端的 0.1 µF 电容器创建低通滤波器,以避免响应交流分量。)如果输入电压高于 V\textsubscript{ref},则底部 LED 将变暗,顶部 LED 将点亮。这通常足以调整电路或平衡桥接,而不需要高精度显示。

\subsection{零件清单}

\begin{itemize}[leftmargin=2cm]
    \item LM393 双比较器(或等效物)
    \item TL082 双运算放大器(或等效物)
    \item 2N3904 NPN 晶体管
    \item 标准 LED(2 个)
    \item 470 Ω、3.3 kΩ(各 3 个)、10 kΩ(各 2 个) ¼ 瓦、5% 电阻器
    \item 5 kΩ、10 kΩ 微调电位器
    \item 1-0-1 mA 中心零直流毫安表(可选)
\end{itemize}

\subsection{参考资料}

\begin{enumerate}
    \item 所有以前的"实践无线电"实验都可在 ARRL 成员的 \href{http://www.arrl.org/hands-on-radio}{www.arrl.org/hands-on-radio} 获取。
\end{enumerate}

\chapter{实验 #143:延迟电路}

\section{英文原文}

The Warner Brothers cartoon character Marvin the Martian seethes, "delays, delays!" But there is no need for a ham radio electronics designer to become "verry annngry." Not at all! Delay circuits are found in many types of ham radio gear, and might even prevent an Earth-shattering ka-boom! This month's column serves up two sample circuits to satisfy your search for spare seconds.

\subsection{Pulse Stretcher}

There are many applications to "stretch" a short pulse into a longer one. A circuit that detects RF might only generate a very short pulse if the incoming signal is brief or weak. That might generate a pulse too short to reliably trigger a logic circuit, be detected by a microprocessor, or light an indicator long enough to be easily seen by the naked eye. Switching or power transients are another notoriously unreliable input signal. One simple and inexpensive solution is to use a spare logic gate or two and an RC timing network, as shown in Figure 1.

\begin{figure}[H]
    \centering
    \includegraphics[width=0.7\linewidth]{00016.jpeg}
    \caption{Pulse Stretcher Circuits}
    \label{fig:pulse_stretcher}
\end{figure}

Let's take the circuit in Figure 1A as an example. In its resting or \textbf{quiescent} state, the input signal is \textbf{LOW}, the input connected to R and C is \textbf{LOW}, and the output of the \textbf{OR} gate formed by the combined \textbf{NOR} gate and inverter is also \textbf{LOW.} As soon as the input pulse changes to \textbf{HIGH}, the output of the \textbf{OR} gate also goes high. This causes current to flow through C, creating a voltage across R. Since C is assumed to be discharged, the initial voltage across R is approximately the same as the output signal, close to V+. Then it begins to drop according to the RC circuit's time constant \textbf{τ = RC}. In a bit more than one time constant \textbf{τ} the voltage will have dropped enough at the \textbf{OR} gate's input to be a logic-level \textbf{LOW}. If the input pulse has ended by then, both inputs to the \textbf{OR} gate will be \textbf{LOW} and so the output of the \textbf{OR} gate will return to \textbf{LOW}. If \textbf{τ} is longer than the input pulse, then the output pulse will be stretched to \textbf{τ} seconds long.

You can follow similar steps to figure out how the circuits work in Figures 1B – 1D. In all, there are four circuits for stretching and inverting either positive- or negative-going pulses. You'll find that using an oscilloscope is the best way of watching both the input and output pulses. Use a 555 timer circuit as described in Experiment #5 as your pulse generator.\textsuperscript{1}

The exact amount of stretching depends on the logic switching thresholds of the logic family you are using. The closer the gate switching thresholds are to V+ and ground, the more the pulse will be stretched. For example, switching thresholds for the 4000-series of CMOS logic are about 10 and 90 percent of V+. Pulses will be stretched longer for this family of logic than for a logic family with thresholds closer to ½ V+. Why? (Because the voltage across R will have to decay longer to reach the lower switching threshold and that means the output pulse will stay \textbf{HIGH} longer.)

\subsection{Soft-start Circuit}

Linear amplifiers and other equipment with high-voltage (HV) power supplies need a bit of delay between the time the power switch is turned \textbf{ON} and the time full voltage is applied to the HV rectifiers and filter. The reason is \textbf{inrush current}. If a linear power supply's filter capacitors are discharged when power is applied, they act like a short-circuit during those first few cycles of rectified ac. This causes very high current pulses in the transformer windings and through the rectifiers as the capacitors charge up.

After a few cycles of ac and depending on the resistance of the rectifiers and transformer secondary winding, the filter capacitors are charged to near their peak value and the amount of current drops dramatically. During the charging period however, peak currents can be 10 to 20 times normal current or even higher, placing significant stress on all of the power supply components.

Circuits have been employed that slowly increase transformer input voltage with a \textbf{TRIAC} or other variable ac source. The \textbf{soft-start} circuit presented in Figure 2 is simpler and satisfies quite nicely the requirement to limit that surge current. It limits inrush current with a 10 Ω power resistor. Until the relay activates, the 10 Ω resistor is in series with the primary winding of the main power transformer. After a suitable delay, the relay contacts short out the 10 Ω resistor and full power is applied to the main transformer.

\begin{figure}[H]
    \centering
    \includegraphics[width=0.7\linewidth]{00017.jpeg}
    \caption{Soft-start Circuit}
    \label{fig:soft_start}
\end{figure}

To power the relay, an auxiliary power supply circuit is required. A small 12 V transformer supplies a 1N4001 diode and 2200 μF in a half-wave rectifier circuit. At light loads, the filtered output voltage, \textbf{V\textsubscript{PS}}, will be about \textbf{12 × 1.4 – 0.7 ≈ 16 V.} (\textbf{V\textsubscript{PS}} will drop closer to 12 V when the relay coil draws current from the supply.)

The timing of when the relay switches is determined by the 33 k Ω resistor and 470 µF capacitor. When power is applied with the 470 μF capacitor completely discharged, it begins charging towards 16 V with a time constant of \textbf{τ = RC = 33 × 10\textsuperscript{3} × 470 × 10\textsuperscript{-6} = 15.5 s.}

As the voltage across the 470 µF capacitor (V\textsubscript{CAP}) increases, it approaches the \textbf{gate threshold voltage} (\textbf{V\textsubscript{GS}(Th)}) of the BS170 FET. This is the point at which the FET will rapidly turn on and conduct drain current, acting more or less like a switch. How quickly the gate voltage will reach 2 V is determined by the equation:

\begin{figure}[H]
    \centering
    \includegraphics[width=0.5\linewidth]{00018.jpeg}
    \caption{Capacitor Charging Equation}
    \label{fig:charging_equation}
\end{figure}

Two seconds is plenty of time for a power supply's filter capacitors to charge up. (This and similar formulas for RC timing circuits are posted on the Hands-On Radio web page for this experiment.)

Most 12 V relay coils have a resistance of around 100 Ω and so draw about 120 mA from a 12 V supply. If the \textbf{on-resistance} (\textbf{R\textsubscript{DS}(On)}) of the FET is a few ohms, it will dissipate \textbf{P\textsubscript{D}=I\textsubscript{D}\textsuperscript{2} R\textsubscript{DS}(On)= 0.014 R\textsubscript{DS}(On)} watts, which is a minimal amount of heat. The 470 kΩ resistor discharges the timing capacitor when power is removed so that the circuit will operate properly when power is again switched on. The second 1N4001 diode clamps the coil voltage so that a nasty \textbf{inductive kickback} transient doesn't destroy the FET when the relay is turned off.

All of this is quite loosely estimated, which means there is plenty of room for experimentation by you! There is a lot of variation between relays — not only in the coil's resistance but the relay's \textbf{pull-in voltage} at which the coil will actually switch the contacts. Try changing the timing components, use different types of FETs or redesign the circuit to use an NPN transistor like the 2N4401, or maybe scrounge up some different relays and try them out. This will help you get a feel for how much variation you can expect out there in the real world.

You needn't actually apply the soft-start circuit to a high-voltage supply — it will operate just fine on its own. You can get a sense for the timing just by listening to the relay or wiring an LED circuit through the normally-open contacts. If you use this circuit on an actual ac power supply, though, please remember that the 10 Ω resistor and the relay contacts carry the full ac mains voltage. That hazard is easy to forget when working with low-voltage circuits! Make sure you keep the ac wiring insulated and well away from the low-voltage dc circuits. No surprises, please!

\subsection{References}

\begin{enumerate}
    \item All previous Hands-On Radio experiments are available to ARRL members at \href{http://www.arrl.org/hands-on-radio}{www.arrl.org/hands-on-radio}.
\end{enumerate}

\section{中文翻译}

华纳兄弟卡通角色火星人马文愤怒地说:"延迟,延迟!" 但业余无线电电子设计师不需要变得"非常生气"。完全不需要!延迟电路在许多类型的业余无线电设备中都有发现,甚至可能防止地球爆炸的灾难!本月的专栏提供了两个示例电路,以满足你对额外秒数的搜索。

\subsection{脉冲展宽器}

有许多应用需要将短脉冲"展宽"为长脉冲。如果输入信号短暂或微弱,检测 RF 的电路可能只会生成非常短的脉冲。这可能会生成太短的脉冲,无法可靠地触发逻辑电路、被微处理器检测到,或点亮指示器足够长的时间以被肉眼轻易看到。开关或电源瞬变是另一种众所周知的不可靠输入信号。一个简单且便宜的解决方案是使用一两个备用逻辑门和 RC 定时网络,如图 1 所示。

\begin{figure}[H]
    \centering
    \includegraphics[width=0.7\linewidth]{00016.jpeg}
    \caption{脉冲展宽器电路}
    \label{fig:pulse_stretcher_cn}
\end{figure}

让我们以图 1A 中的电路为例。在其静止或 \textbf{静态} 状态下,输入信号为 \textbf{低},连接到 R 和 C 的输入为 \textbf{低},由组合 \textbf{NOR} 门和反相器形成的 \textbf{OR} 门的输出也为 \textbf{低}。一旦输入脉冲变为 \textbf{高},\textbf{OR} 门的输出也变高。这导致电流流过 C,在 R 两端产生电压。由于假设 C 已放电,R 两端的初始电压近似等于输出信号,接近 V+。然后它开始根据 RC 电路的时间常数 \textbf{τ = RC} 下降。在略多于一个时间常数 \textbf{τ} 后,\textbf{OR} 门输入处的电压将下降到足以成为逻辑电平 \textbf{低}。如果输入脉冲在那时结束,\textbf{OR} 门的两个输入都将为 \textbf{低},因此 \textbf{OR} 门的输出将返回 \textbf{低}。如果 \textbf{τ} 长于输入脉冲,则输出脉冲将被展宽为 \textbf{τ} 秒长。

你可以按照类似的步骤来理解图 1B – 1D 中的电路如何工作。总共有四个电路用于展宽和反相正脉冲或负脉冲。你会发现使用示波器是观察输入和输出脉冲的最佳方式。使用实验 #5 中描述的 555 定时器电路作为你的脉冲发生器。\textsuperscript{1}

展宽的确切量取决于你使用的逻辑系列的逻辑开关阈值。门开关阈值越接近 V+ 和接地,脉冲将被展宽得越多。例如,4000 系列 CMOS 逻辑的开关阈值约为 V+ 的 10% 和 90%。对于这个逻辑系列,脉冲将比阈值更接近 ½ V+ 的逻辑系列展宽得更长。为什么?(因为 R 两端的电压必须衰减更长时间才能达到较低的开关阈值,这意味着输出脉冲将保持 \textbf{高} 更长时间。)

\subsection{软启动电路}

线性放大器和其他具有高压 (HV) 电源的设备需要在电源开关打开的时间和全电压施加到 HV 整流器和滤波器的时间之间有一点延迟。原因是 \textbf{涌入电流}。如果线性电源的滤波电容器在施加电源时已放电,它们在整流交流的前几个周期中表现得像短路。这会在电容器充电时在变压器绕组和整流器中产生非常高的电流脉冲。

经过几个交流周期后,取决于整流器和变压器次级绕组的电阻,滤波电容器被充电到接近其峰值,电流大大降低。然而,在充电期间,峰值电流可能是正常电流的 10 到 20 倍甚至更高,对所有电源组件造成显著压力。

已经采用了使用 \textbf{TRIAC} 或其他可变交流电源缓慢增加变压器输入电压的电路。图 2 中呈现的 \textbf{软启动} 电路更简单,并且很好地满足了限制该浪涌电流的要求。它使用 10 Ω 功率电阻器限制涌入电流。在继电器激活之前,10 Ω 电阻器与主电源变压器的初级绕组串联。经过适当的延迟后,继电器触点短路 10 Ω 电阻器,全功率施加到主变压器。

\begin{figure}[H]
    \centering
    \includegraphics[width=0.7\linewidth]{00017.jpeg}
    \caption{软启动电路}
    \label{fig:soft_start_cn}
\end{figure}

为了给继电器供电,需要一个辅助电源电路。一个小型 12 V 变压器为半波整流电路中的 1N4001 二极管和 2200 μF 提供电源。在轻负载下,滤波后的输出电压 \textbf{V\textsubscript{PS}} 将约为 \textbf{12 × 1.4 – 0.7 ≈ 16 V}。(当继电器线圈从电源汲取电流时,\textbf{V\textsubscript{PS}} 将下降到接近 12 V。)

继电器切换的时间由 33 k Ω 电阻器和 470 µF 电容器决定。当 470 μF 电容器完全放电时施加电源,它开始以 \textbf{τ = RC = 33 × 10\textsuperscript{3} × 470 × 10\textsuperscript{-6} = 15.5 s} 的时间常数向 16 V 充电。

随着 470 µF 电容器两端的电压 (V\textsubscript{CAP}) 增加,它接近 BS170 FET 的 \textbf{栅极阈值电压} (\textbf{V\textsubscript{GS}(Th)})。这是 FET 将快速开启并传导漏极电流的点,或多或少像一个开关。栅极电压达到 2 V 的速度由以下等式确定:

\begin{figure}[H]
    \centering
    \includegraphics[width=0.5\linewidth]{00018.jpeg}
    \caption{电容器充电等式}
    \label{fig:charging_equation_cn}
\end{figure}

两秒钟足够电源的滤波电容器充电了。(此公式和 RC 定时电路的类似公式发布在本实验的 Hands-On Radio 网页上。)

大多数 12 V 继电器线圈的电阻约为 100 Ω,因此从 12 V 电源汲取约 120 mA。如果 FET 的 \textbf{导通电阻} (\textbf{R\textsubscript{DS}(On)}) 为几欧姆,它将耗散 \textbf{P\textsubscript{D}=I\textsubscript{D}\textsuperscript{2} R\textsubscript{DS}(On)= 0.014 R\textsubscript{DS}(On)} 瓦,这是最小量的热量。470 kΩ 电阻器在移除电源时放电定时电容器,以便在再次打开电源时电路正常工作。第二个 1N4001 二极管钳位线圈电压,以便在继电器关闭时不会有讨厌的 \textbf{电感反冲} 瞬变损坏 FET。

所有这些都是非常宽松的估计,这意味着你有足够的实验空间!继电器之间有很多变化 —— 不仅在线圈的电阻上,而且在继电器实际切换触点的 \textbf{吸合电压} 上。尝试更改定时组件,使用不同类型的 FET 或重新设计电路以使用 NPN 晶体管如 2N4401,或者可能找到一些不同的继电器并尝试它们。这将帮助你了解在现实世界中可以期望多少变化。

你不必实际将软启动电路应用于高压电源 —— 它自己运行得很好。你可以通过听继电器或通过常开触点接线 LED 电路来了解定时。但是,如果你在实际的交流电源上使用此电路,请记住 10 Ω 电阻器和继电器触点承载全交流市电电压。在处理低压电路时,这种危险很容易被忘记!确保保持交流布线绝缘并远离低压直流电路。请不要有意外!

\subsection{参考资料}

\begin{enumerate}
    \item 所有以前的"实践无线电"实验都可在 ARRL 成员的 \href{http://www.arrl.org/hands-on-radio}{www.arrl.org/hands-on-radio} 获取。
\end{enumerate}

\chapter{实验 #155:负电压电路}

\section{英文原文}

It may come as a surprise in this 12-volt world, but not every circuit is powered by 13.8 ± 0.5 V dc. Negative voltages were quite common in the days of tubes, but today's digital logic and analog circuitry mostly requires only one voltage polarity — positive. Is there nothing left "below ground?"

\subsection{ALC Control}

You don't have very far to look for a circuit that depends on a negative voltage, and that is your radio's Automatic Level Control (ALC) input. Just like the radio's internal ALC system, an external amplifier generates an ALC signal to reduce transmitter power and prevent overdrive. The ALC signal is a dc voltage varying from 0 V (full power, no ALC action) to a few volts negative. For example, the ALC range of my TS-590S is 0 to –7 V. Armed with this knowledge, you can turn your 100 W radio into a milli-watter with the simple battery-powered circuit shown in Figure 1.

\begin{figure}[H]
    \centering
    \includegraphics[width=0.7\linewidth]{00019.jpeg}
    \caption{ALC Control Circuit}
    \label{fig:alc_control}
\end{figure}

Set the 1 kΩ pot to about 500 Ω between the wiper and either end of the element. Connect a 9 V battery (paying careful attention to which terminal is positive or negative) across the element of the pot. (You can't use a ground-referenced power supply for the 9 V battery!) The output ALC signal is created between the wiper of the pot and positive terminal of the battery, which is connected to the radio's chassis ground. Use a voltmeter to verify that the \textbf{ALC_Out} signal is negative and about half of the battery voltage, then adjust the pot for an output of about –0.1 V. The 0.01 μF capacitor filters out any RF that might be picked up by the connecting cable.

Disconnect the battery and fashion a cable to connect the \textbf{ALC_Out} signal to your radio's ALC input. (This might be a dedicated phono jack, but it's probably two pins on an accessory connector.) Be sure the positive battery terminal will be connected to the radio's common or ground pin. Reconnect the battery and turn on the radio. Set the radio for full power output, then generate a steady carrier level into a dummy load. Adjust the pot so that \textbf{ALC_Out} becomes more negative and output power from the radio should eventually drop. Generate some CW and adjust the pot until you have 5 W output. \textbf{Voilà} — you have a QRP rig. Or, you can keep going and see what you can work with 100 mW!

\subsection{Un-common Emitter}

The common emitter (CE) amplifier circuit from "Hands-On Radio" Experiment #1 makes another appearance in Figure 2 — almost!\textsuperscript{1} Look closely and you'll see that Q1 is a 2N3906, the PNP twin of the common 2N3904. They have very similar specifications for gain, voltage and current rating, switching speed, etc. You'll also notice that the battery polarity is inverted so the "top rail" is the circuit's common.

\begin{figure}[H]
    \centering
    \includegraphics[width=0.7\linewidth]{00020.jpeg}
    \caption{PNP Common Emitter Amplifier}
    \label{fig:pnp_common_emitter}
\end{figure}

Everything about this PNP version of the CE is inverted. Instead of pushing current \textbf{into} the base to turn on the transistor, in this circuit you pull current \textbf{out} of the base. The more current you allow to flow out of the base (by lowering the value of R1), the more I\textsubscript{C} will flow. Even the equations for gain and dc operating point are the same.

Wire up this circuit (use a battery or isolated power supply) and verify that I\textsubscript{C} is about –4 mA and the voltage from collector to emitter, V\textsubscript{CE}, is about –5 V. (The minus sign for I\textsubscript{C} indicates that current is coming \textbf{out} \textbf{of} instead of \textbf{into} the collector.) You can experiment with resistor values to see the effect changing R1 and R2 has on I\textsubscript{C}. In fact, building both the NPN CE circuit and the PNP CE circuit side by side is a good way to get more comfortable with PNP transistors.

\subsection{Grid-block Keying}

Many of us have electronic keyers that ground our rig's positive-voltage keying input to turn on the transmitter. What about keying a vacuum tube rig? Many later model tube rigs used \textbf{grid-block keying}, in which a high negative bias voltage (–50 to –150 V) on a tube's grid cut off plate current during non-transmitting periods. Grounding the negative voltage allowed plate current to flow, turning on the transmitter. This was no problem with a hand key, "bug," or relay, but today's solid-state keyers can't handle negative voltage. What to do?

Chuck Olson, WB9KZY, of Jackson Harbor Press (\href{http://wb9kzy.com/ham.htm}{wb9kzy.com/ham.htm}), designed the circuit shown in Figure 3 to adapt the common positive-voltage keyer to grid-block vacuum tube circuits.\textsuperscript{2} In this circuit, the keyer's output (usually an NPN transistor, as shown) is connected between J1 and J3. To key the tube transmitter, the keyer's output transistor grounds the base of Q1 through R1, turning on the PNP MJE350, which is rated for the higher grid-block voltage. This connects J2, the radio's grid-block input, to +5 V, turning on the transmitter. (Voltage at J2 is assumed to be negative.) R2 is a pull-up resistor that keeps the MJE350 turned off when the keyer transistor is off. C1 and D1 protect the transistor from transients and keep RF out of the circuit.

\begin{figure}[H]
    \centering
    \includegraphics[width=0.7\linewidth]{00021.jpeg}
    \caption{Grid-block Keying Adapter}
    \label{fig:grid_block_keying}
\end{figure}

\subsection{Bipolar Op-Amp Circuits}

Another place you'll find a requirement for a negative voltage supply is in op-amp circuits, such as the one shown in Figure 4. This circuit is from "Hands-On Radio" Experiment #3 — a \textbf{difference amplifier} with an output equal to the difference between V\textsubscript{in1} and V\textsubscript{in2}. (All the resistor values must be equal for this to be true.) While the venerable 741 is shown on the schematic, any common op-amp will do the job at audio frequencies or below. The inputs can be dc, ac, or a combination of the two. It doesn't matter if both input voltages are positive or negative or which is greater. (Just keep them between the power supply voltages to avoid damaging the op-amp.)

\begin{figure}[H]
    \centering
    \includegraphics[width=0.7\linewidth]{00022.jpeg}
    \caption{Bipolar Op-Amp Circuit}
    \label{fig:bipolar_opamp}
\end{figure}

The output can be of either polarity so the power supply must supply both polarities. Figure 4 also shows a handy (and portable) means of generating a negative supply — splitting a set of batteries. A string of D-cell batteries will supply clean dc power for a long time at the low loads of most circuit experimenting. Use at least four cells for each power supply voltage to generate ±6 V. If you have a large multi-cell holder, you can create the circuit's common voltage by soldering a wire to a bit of brass or copper shim and slipping it between adjacent cells. A pair of surplus rechargeable gel-cell batteries is a good choice, too.

\subsection{Generating Negative Voltages}

Batteries are terrific, but over the long-term, even a small current drain will deplete a good-sized battery. If you have a reliable source of positive voltage you can use a voltage converter IC and a couple of capacitors to create your negative supply instead.

The ICL7660 has been a staple of the analog designer for a long time, along with its many variations. Start by downloading and reading the ICL7660 data sheet from Maxim Integrated.\textsuperscript{3} Not only will you get a detailed description of how the IC works, but also plenty of interesting and innovative ways to use it.

The ICL7660 uses \textbf{switched-capacitor technology} to convert positive voltages to negative voltages, double a positive voltage, divide voltages, can be connected in series to create higher negative voltages, or connected in parallel to increase the available current.

If you build a simple negative voltage converter, it can be used as the power supply for the ALC power control circuit in Figure 1. Input power can be obtained from the radio's +12 V supply. Place a Zener diode across the circuit's output, in parallel with the 0.01 μF capacitor to limit the output voltage to the maximum for your radio. By packaging the circuit in a simple plastic container or enclosure, you create a handy power-control accessory for your station. Just right for the QRP ARCI's "1000 Miles Per Watt" award — how low can you go?\textsuperscript{4}

\subsection{References}

\begin{enumerate}
    \item All previous "Hands-On Radio" columns are available to ARRL members at \href{http://www.arrl.org/hands-on-radio}{www.arrl.org/hands-on-radio}.
    \item Jackson Harbor Press sells the Keyall and KeyallHV kits if you prefer to purchase the components and pre-made PCB to make your own adapter.
    \item \href{http://datasheets.maximintegrated.com/en/ds/ICL7660-MAX1044.pdf}{datasheets.maximintegrated.com/en/ds/ICL7660-MAX1044.pdf}
    \item \href{http://qrparci.org/awards}{qrparci.org/awards}
\end{enumerate}

\section{中文翻译}

在这个12伏的世界里,这可能会让人惊讶,但并不是每个电路都由13.8 ± 0.5 V直流供电。在电子管时代,负电压相当常见,但今天的数字逻辑和模拟电路大多只需要一种电压极性 —— 正电压。"地电位以下"就没有什么了吗?

\subsection{ALC控制}

你不需要很远就能找到依赖负电压的电路,那就是你的收音机的自动电平控制(ALC)输入。就像收音机的内部ALC系统一样,外部放大器会生成一个ALC信号来减少发射机功率并防止过载。ALC信号是一个直流电压,从0 V(全功率,无ALC动作)变化到几伏负电压。例如,我的TS-590S的ALC范围是0到–7 V。有了这些知识,你可以用图1所示的简单电池供电电路将你的100 W收音机变成毫瓦级收音机。

\begin{figure}[H]
    \centering
    \includegraphics[width=0.7\linewidth]{00019.jpeg}
    \caption{ALC控制电路}
    \label{fig:alc_control_cn}
\end{figure}

将1 kΩ电位器设置为在抽头和元件的任一端之间约500 Ω。将9 V电池(仔细注意哪个端子是正还是负)连接到电位器的元件两端。(你不能使用接地参考电源来代替9 V电池!)输出ALC信号在电位器的抽头和电池的正端子之间产生,电池的正端子连接到收音机的底盘接地。使用电压表验证\textbf{ALC_Out}信号是负的,约为电池电压的一半,然后调整电位器使输出约为–0.1 V。0.01 μF电容器过滤掉可能被连接电缆拾取的任何RF。

断开电池并制作一条电缆,将\textbf{ALC_Out}信号连接到收音机的ALC输入。(这可能是一个专用的唱机插孔,但可能是附件连接器上的两个引脚。)确保电池的正端子将连接到收音机的公共或接地引脚。重新连接电池并打开收音机。将收音机设置为全功率输出,然后向假负载生成稳定的载波电平。调整电位器,使\textbf{ALC_Out}变得更负,收音机的输出功率最终应该下降。生成一些CW并调整电位器,直到你有5 W输出。\textbf{瞧} —— 你有了一个QRP设备。或者,你可以继续前进,看看你能用100 mW做什么!

\subsection{不常见的发射极}

"实践无线电"实验#1中的共发射极(CE)放大器电路在图2中再次出现 —— 几乎!\textsuperscript{1} 仔细看,你会发现Q1是2N3906,是常见的2N3904的PNP孪生兄弟。它们在增益、电压和电流额定值、开关速度等方面具有非常相似的规格。你还会注意到电池极性是反转的,所以"顶轨"是电路的公共端。

\begin{figure}[H]
    \centering
    \includegraphics[width=0.7\linewidth]{00020.jpeg}
    \caption{PNP共发射极放大器}
    \label{fig:pnp_common_emitter_cn}
\end{figure}

这个CE的PNP版本的所有东西都是反转的。在这个电路中,你不是将电流\textbf{推入}基极来打开晶体管,而是将电流\textbf{拉出}基极。你允许从基极流出的电流越多(通过降低R1的值),I\textsubscript{C}就会越多。即使是增益和直流工作点的方程也是相同的。

搭建这个电路(使用电池或隔离电源)并验证I\textsubscript{C}约为–4 mA,集电极到发射极的电压V\textsubscript{CE}约为–5 V。(I\textsubscript{C}的负号表示电流是从集电极\textbf{流出}而不是\textbf{流入}集电极。)你可以用电阻值进行实验,看看改变R1和R2对I\textsubscript{C}的影响。事实上,并排构建NPN CE电路和PNP CE电路是更熟悉PNP晶体管的好方法。

\subsection{栅极阻塞键控}

我们中的许多人都有电子键控器,它们将我们设备的正电压键控输入接地以打开发射机。那么键控真空管设备呢?许多后期型号的电子管设备使用\textbf{栅极阻塞键控},其中电子管栅极上的高负偏置电压(–50到–150 V)在非发射期间切断板电流。将负电压接地允许板电流流动,打开发射机。这对于手键、"bug"或继电器来说没有问题,但今天的固态键控器无法处理负电压。怎么办?

Jackson Harbor Press的Chuck Olson,WB9KZY(\href{http://wb9kzy.com/ham.htm}{wb9kzy.com/ham.htm})设计了图3所示的电路,将常见的正电压键控器适配到栅极阻塞真空管电路。\textsuperscript{2} 在这个电路中,键控器的输出(通常是NPN晶体管,如图所示)连接在J1和J3之间。要键控电子管发射机,键控器的输出晶体管通过R1将Q1的基极接地,打开PNP MJE350,它额定用于更高的栅极阻塞电压。这将J2(收音机的栅极阻塞输入)连接到+5 V,打开发射机。(假设J2处的电压为负。)R2是一个上拉电阻,当键控器晶体管关闭时,它保持MJE350关闭。C1和D1保护晶体管免受瞬变影响并防止RF进入电路。

\begin{figure}[H]
    \centering
    \includegraphics[width=0.7\linewidth]{00021.jpeg}
    \caption{栅极阻塞键控适配器}
    \label{fig:grid_block_keying_cn}
\end{figure}

\subsection{双极性运算放大器电路}

另一个你会发现需要负电压电源的地方是运算放大器电路,如图4所示。这个电路来自"实践无线电"实验#3 —— 一个\textbf{差分放大器},其输出等于V\textsubscript{in1}和V\textsubscript{in2}之间的差值。(所有电阻值必须相等才能做到这一点。)虽然原理图上显示的是古老的741,但任何常见的运算放大器都可以在音频频率或更低频率下完成这项工作。输入可以是直流、交流或两者的组合。两个输入电压是正还是负,或者哪个更大都无关紧要。(只要保持它们在电源电压之间以避免损坏运算放大器。)

\begin{figure}[H]
    \centering
    \includegraphics[width=0.7\linewidth]{00022.jpeg}
    \caption{双极性运算放大器电路}
    \label{fig:bipolar_opamp_cn}
\end{figure}

输出可以是任一极性,因此电源必须提供两种极性。图4还显示了一种方便(且便携式)的生成负电源的方法 —— 分割一组电池。一串D电池将以大多数电路实验的低负载长时间提供清洁的直流电源。每个电源电压使用至少四个电池以生成±6 V。如果你有一个大型多电池支架,你可以通过将电线焊接到一点黄铜或铜垫片上并将其滑入相邻电池之间来创建电路的公共电压。一对多余的可充电胶体电池也是一个不错的选择。

\subsection{生成负电压}

电池很棒,但从长远来看,即使是很小的电流消耗也会耗尽一个相当大的电池。如果你有可靠的正电压源,你可以使用电压转换器IC和几个电容器来创建你的负电源。

ICL7660及其许多变体长期以来一直是模拟设计师的主食。首先从Maxim Integrated下载并阅读ICL7660数据手册。\textsuperscript{3} 你不仅会获得关于IC如何工作的详细描述,还会获得许多有趣和创新的使用方法。

ICL7660使用\textbf{开关电容技术}将正电压转换为负电压,将正电压加倍,分压,可以串联连接以创建更高的负电压,或并联连接以增加可用电流。

如果你构建一个简单的负电压转换器,它可以用作图1中ALC功率控制电路的电源。输入功率可以从收音机的+12 V电源获得。在电路输出两端放置一个齐纳二极管,与0.01 μF电容器并联,以将输出电压限制在收音机的最大值。通过将电路包装在简单的塑料容器或外壳中,你可以为你的电台创建一个方便的功率控制附件。正好适合QRP ARCI的"每瓦1000英里"奖项 —— 你能走多低?\textsuperscript{4}

\subsection{参考资料}

\begin{enumerate}
    \item 所有以前的"实践无线电"专栏都可在ARRL成员的\href{http://www.arrl.org/hands-on-radio}{www.arrl.org/hands-on-radio}获取。
    \item 如果你喜欢购买组件和预制PCB来制作自己的适配器,Jackson Harbor Press出售Keyall和KeyallHV套件。
    \item \href{http://datasheets.maximintegrated.com/en/ds/ICL7660-MAX1044.pdf}{datasheets.maximintegrated.com/en/ds/ICL7660-MAX1044.pdf}
    \item \href{http://qrparci.org/awards}{qrparci.org/awards}
\end{enumerate}

\chapter{实验 #156:设计广播拒绝滤波器}

\section{英文原文}

If you live within a few miles of an AM broadcast station, you will be interested in broadcast-reject filters to prevent overload, this month's subject. This will actually be a "two-fer" column because we're going to use the latest version of Jim Tonne's, W4ENE, \textbf{ELSIE} filter design software to do the design work. This handy software makes it easy to design your own passive LC filters in a wide variety of configurations. You can juggle design inputs to your heart's content, watching the filter respond until it's just right.

\subsection{Downloading \textbf{ELSIE}}

Begin by downloading the set of utility programs from \textbf{The ARRL Handbook}'s web page.\textsuperscript{1} Open the ZIP file, double-click on \textbf{LCinstall275.exe} and after installing, run \textbf{ELSIE}. Be sure you are running version 2.75 (look at the lower left of the opening display) so that the directions in this article will agree with what you see on your screen. Begin by clicking \textbf{NEW DESIGN}.

\subsection{Specifying the Filter}

Now it's time to tell \textbf{ELSIE} what kind of circuit you want. This is the filter's \textbf{topology} describing the general arrangement of the filter components. (Filter basics were covered in "Hands-On Radio" experiments #50 and #51.\textsuperscript{2}) Because we are designing a broadcast-reject filter for 160 meter reception, we will want a high-pass response. \textbf{ELSIE} gives us two choices for high-pass filters: capacitive input and inductive input. A capacitor in series with the filter at the input (capacitive input) blocks any dc and low-frequency signals, so select that topology.

Next, we must select from the many types of LC filter circuits, called \textbf{families}, and each has a slightly different set of characteristics. For example, the Butterworth family has a smooth rolloff between the passband and the stop band. The Chebyshev family allows some ripple in either passband or stop band in trade for a steeper rolloff. Bessel filters have a constant time delay through the filter in the passband. If you click the \textbf{?} button next to \textbf{BUTTERWORTH} in the \textbf{FAMILY} section, a pop-up window will show the general behavior for each family.

In our case, we need a very sharp rolloff, passing signals with little attenuation at 1.8 MHz, the bottom end of 160 meters, but lots of attenuation at 1.6 MHz, the highest frequency of the AM broadcast band at which full-power stations are permitted.\textsuperscript{3} Chebyshev would be a good choice, but the Cauer family is even better at creating the necessary steep rolloff. The tradeoff apparent from the filter characteristics is that attenuation of the Cauer filters varies quite a bit in the stop band. That's okay, as long as a certain minimum attenuation is maintained, so select \textbf{Cauer} as the filter family.

Now the program needs some performance specifications entered at the right-hand side of the screen. How much attenuation (\textbf{STOP BAND DEPTH, A\textsubscript{S}}, in dB) is enough for our filter? In my experience, 40 dB is enough to keep even nearby AM stations from clobbering the front end of a late-model receiver. For \textbf{RIPPLE BANDWIDTH (F\textsubscript{C})}, enter "1.8M" (1.8 MHz) as the lowest frequency of the filter passband. The highest frequency at which we want our 40 dB of attenuation, "1.6M" (1.6 MHz), is the \textbf{STOPBAND WIDTH (F\textsubscript{S})}.

Filter \textbf{ORDER (N)} can be thought of as the number of resonances created by the filter components. The higher the order, the more components (Ls and Cs) are required to create the circuit. Start by entering a filter order of 3 to see if we can meet our design goals.

\subsection{Viewing the Response}

To set up the program's calculations and display configuration, click the \textbf{ANALYSIS} tab. This is where the inductor and capacitor Q are specified (250 and 1000, respectively). Lower values of Q results in less stop band attenuation and less sharp rolloff, among other effects. Leave them at their default settings for this exercise. Specify an \textbf{ANALYSIS START FREQUENCY} of 0.5 MHz and an \textbf{ANALYSIS STOP FREQUENCY} of 10 MHz. Leave all other selections and values at their original default settings.

Before looking at the filter response plot, click the \textbf{SCHEMATIC} tab at the top of the screen. The information is summarized in Figure 1. Passband ripple (21.3 dB) and the component maximum/minimum ratios are calculated by the program.

\begin{figure}[H]
    \centering
    \includegraphics[width=0.7\linewidth]{00023.jpeg}
    \caption{Filter Schematic}
    \label{fig:filter_schematic}
\end{figure}

Now click the \textbf{PLOT} tab for the frequency response graph of Figure 2. Place the cursor on the blue response line and hold down the left mouse button. At the bottom of the screen, you will see the filter performance at that frequency. The figure shows performance at the response peak of 1.93 MHz. Move the mouse to the stop band notch near 1.5 MHz to find attenuation there (–69 dB at 1.49 MHz). Attenuation varies by more than 20 dB over the 160 meter band — that's too much! The program is trying but can't meet our specifications without more components to create a higher-order filter. (While you're at it, use the \textbf{DESIGN} window to select different filter families and compare their responses.)

\begin{figure}[H]
    \centering
    \includegraphics[width=0.7\linewidth]{00024.jpeg}
    \caption{Filter Frequency Response (Order 3)}
    \label{fig:filter_response_3}
\end{figure}

\subsection{Interactive Design}

This is where the value of easy-to-use design software becomes apparent. Instead of re-starting a laborious design process, simply re-enter the new specifications and try again. Return to the \textbf{DESIGN} tab and increase the filter order from 3 to 4, then click \textbf{PLOT}. Performance is improved, but attenuation still varies by more than 10 dB across 160 meters and the rolloff isn't sharp enough; only 28 dB of attenuation at 1.6 MHz.

Increase the filter order to 5, resetting F\textsubscript{S} to 1.6 MHz. (The program changes some values when order is changed between odd and even. Check the settings when order is changed.) This response (see Figure 3) is much more useful. Attenuation varies by about 3 dB (1⁄2 S-unit) across the 160 meter band and we just meet our design goal with –40 dB of attenuation at 1.59 MHz. Click the \textbf{Save} tab to hold on to this design version before proceeding.

\begin{figure}[H]
    \centering
    \includegraphics[width=0.7\linewidth]{00025.jpeg}
    \caption{Filter Frequency Response (Order 5)}
    \label{fig:filter_response_5}
\end{figure}

\subsection{Using Standard Value Components}

The schematic shows all of the component values are in a reasonable range. Nevertheless, I don't think your local component vendor will have, say, 538.436 pF capacitors in stock, nor do you want to have to adjust variable capacitors. Now is the time to redesign the filter using standard fixed-value parts. This will degrade filter performance a bit, but remember that we can continue to work with the design.

Return to the \textbf{DESIGN} window and click the \textbf{NEAREST 5\%} tab. You'll be presented with several options, starting with changing all of the components to the nearest standard value in the 5\% series. Other options include just changing the capacitors or inductors, assuming you'll wind the Ls or tune the Cs. You can also change just the capacitors or inductors and the program will re-calculate the remaining values exactly so that you can tune up the filter yourself.

Let's take the easy way out and select the first option to use all standard values. Return to the \textbf{DESIGN} tab, then check the schematic shown in Figure 4. How about performance? Viewing the frequency response, not much has changed. We have a little more variation across the band (now 4 dB) but attenuation at 1.6 MHz is still the same, only failing to reach 40 dB below 840 kHz by less than a dB.

\begin{figure}[H]
    \centering
    \includegraphics[width=0.7\linewidth]{00026.jpeg}
    \caption{Filter Schematic with Standard Values}
    \label{fig:filter_schematic_standard}
\end{figure}

I think we're done! It's easy to see how you could continue to experiment with the filter, possibly trading some stop band attenuation for less passband attenuation, or smoother response in the passband, and we haven't even begun to work on the input and output impedances or delay time. Nevertheless, this design can be built with off-the-shelf components requiring no tuning to provide useful performance. Thanks to W4ENE and his terrific software.

\subsection{References}

\begin{enumerate}
    \item A set of free student version circuit design utilities from Tonnesoft are available for download from \href{http://www.arrl.org/arrl-handbook-reference}{www.arrl.org/arrl-handbook-reference}.
    \item All previous "Hands-On Radio" columns are available to ARRL members at \href{http://www.arrl.org/hands-on-radio}{www.arrl.org/hands-on-radio}.
    \item Above 1600 kHz in the US, AM stations are limited to 10 kW during the day and 1 kW at night. These smaller stations are less likely to cause overload problems than the full-power 50 kW transmitters.
\end{enumerate}

\section{中文翻译}

如果你住在AM广播电台几英里范围内,你会对广播拒绝滤波器感兴趣,以防止过载,这是本月的主题。这实际上是一个"二合一"专栏,因为我们将使用Jim Tonne的W4ENE最新版本的\textbf{ELSIE}滤波器设计软件来完成设计工作。这个方便的软件使设计各种配置的无源LC滤波器变得容易。你可以随心所欲地调整设计输入,观察滤波器的响应,直到它恰到好处。

\subsection{下载 \textbf{ELSIE}}

首先从\textbf{The ARRL Handbook}的网页下载实用程序集。\textsuperscript{1} 打开ZIP文件,双击\textbf{LCinstall275.exe},安装后运行\textbf{ELSIE}。确保你运行的是版本2.75(查看打开显示的左下角),以便本文中的说明与你屏幕上看到的一致。首先点击\textbf{NEW DESIGN}。

\subsection{指定滤波器}

现在是时候告诉\textbf{ELSIE}你想要什么样的电路了。这是滤波器的\textbf{拓扑结构},描述了滤波器组件的一般排列。(滤波器基础知识在"实践无线电"实验#50和#51中介绍。\textsuperscript{2})因为我们正在设计用于160米接收的广播拒绝滤波器,我们需要高通响应。\textbf{ELSIE}为我们提供了两种高通滤波器选择:电容输入和电感输入。输入处与滤波器串联的电容器(电容输入)会阻断任何直流和低频信号,因此选择该拓扑结构。

接下来,我们必须从许多类型的LC滤波器电路中选择,称为\textbf{系列},每种都有略有不同的特性集。例如,巴特沃斯系列在通带和阻带之间有平滑的滚降。切比雪夫系列允许在通带或阻带中有一些纹波,以换取更陡峭的滚降。贝塞尔滤波器在通带中通过滤波器的时间延迟恒定。如果你点击\textbf{FAMILY}部分中\textbf{BUTTERWORTH}旁边的\textbf{?}按钮,会弹出一个窗口显示每个系列的一般行为。

在我们的情况下,我们需要非常陡峭的滚降,在160米的低端1.8 MHz处通过信号几乎无衰减,但在AM广播频段的最高频率1.6 MHz处有大量衰减,这是允许全功率电台的频率。\textsuperscript{3} 切比雪夫将是一个不错的选择,但考尔系列在创建必要的陡峭滚降方面甚至更好。从滤波器特性中明显看出的权衡是,考尔滤波器的衰减在阻带中变化很大。这没关系,只要保持一定的最小衰减,所以选择\textbf{Cauer}作为滤波器系列。

现在程序需要在屏幕右侧输入一些性能规格。我们的滤波器需要多少衰减(\textbf{阻带深度,A\textsubscript{S}},以dB为单位)?根据我的经验,40 dB足以防止即使是附近的AM电台干扰新型接收机的前端。对于\textbf{纹波带宽(F\textsubscript{C})},输入"1.8M"(1.8 MHz)作为滤波器通带的最低频率。我们希望40 dB衰减的最高频率"1.6M"(1.6 MHz)是\textbf{阻带宽度(F\textsubscript{S})}。

滤波器\textbf{阶数(N)}可以被认为是滤波器组件创建的谐振次数。阶数越高,创建电路所需的组件(Ls和Cs)越多。首先输入滤波器阶数3,看看我们是否能满足设计目标。

\subsection{查看响应}

要设置程序的计算和显示配置,点击\textbf{ANALYSIS}选项卡。这里指定了电感器和电容器的Q值(分别为250和1000)。Q值越低,阻带衰减越少,滚降越不陡峭,以及其他影响。在本练习中保持默认设置。指定\textbf{分析开始频率}为0.5 MHz,\textbf{分析停止频率}为10 MHz。保持所有其他选择和值为原始默认设置。

在查看滤波器响应图之前,点击屏幕顶部的\textbf{SCHEMATIC}选项卡。信息总结在图1中。通带纹波(21.3 dB)和组件最大/最小比率由程序计算。

\begin{figure}[H]
    \centering
    \includegraphics[width=0.7\linewidth]{00023.jpeg}
    \caption{滤波器原理图}
    \label{fig:filter_schematic_cn}
\end{figure}

现在点击\textbf{PLOT}选项卡查看图2的频率响应图。将光标放在蓝色响应线上并按住鼠标左键。在屏幕底部,你将看到该频率下的滤波器性能。该图显示了1.93 MHz响应峰值处的性能。将鼠标移动到1.5 MHz附近的阻带凹口,找到那里的衰减(1.49 MHz时为–69 dB)。在160米频段上,衰减变化超过20 dB —— 这太多了!程序正在尝试,但没有更多组件来创建更高阶的滤波器,就无法满足我们的规格。(同时,使用\textbf{DESIGN}窗口选择不同的滤波器系列并比较它们的响应。)

\begin{figure}[H]
    \centering
    \includegraphics[width=0.7\linewidth]{00024.jpeg}
    \caption{滤波器频率响应(3阶)}
    \label{fig:filter_response_3_cn}
\end{figure}

\subsection{交互式设计}

这就是易于使用的设计软件的价值所在。不需要重新开始繁琐的设计过程,只需重新输入新规格并再次尝试。返回\textbf{DESIGN}选项卡,将滤波器阶数从3增加到4,然后点击\textbf{PLOT}。性能有所改善,但在160米频段上衰减仍然变化超过10 dB,滚降不够陡峭;在1.6 MHz处只有28 dB的衰减。

将滤波器阶数增加到5,将F\textsubscript{S}重置为1.6 MHz。(当阶数在奇数和偶数之间变化时,程序会更改一些值。当阶数更改时,请检查设置。)这种响应(见图3)更有用。在160米频段上,衰减变化约为3 dB(1⁄2 S单位),我们正好满足设计目标,在1.59 MHz处有–40 dB的衰减。点击\textbf{Save}选项卡以保存此设计版本,然后继续。

\begin{figure}[H]
    \centering
    \includegraphics[width=0.7\linewidth]{00025.jpeg}
    \caption{滤波器频率响应(5阶)}
    \label{fig:filter_response_5_cn}
\end{figure}

\subsection{使用标准值组件}

原理图显示所有组件值都在合理范围内。尽管如此,我认为你的当地组件供应商不会有,比如说,538.436 pF的电容器库存,你也不想必须调整可变电容器。现在是时候使用标准固定值部件重新设计滤波器了。这会稍微降低滤波器性能,但请记住,我们可以继续使用该设计。

返回\textbf{DESIGN}窗口,点击\textbf{NEAREST 5\%}选项卡。你将看到几个选项,首先是将所有组件更改为5\%系列中的最接近标准值。其他选项包括仅更改电容器或电感器,假设你将缠绕Ls或调整Cs。你也可以仅更改电容器或电感器,程序将精确重新计算剩余值,以便你可以自己调整滤波器。

让我们采取简单的方法,选择第一个选项使用所有标准值。返回\textbf{DESIGN}选项卡,然后检查图4所示的原理图。性能如何?查看频率响应,变化不大。我们在整个频段上有更多的变化(现在为4 dB),但在1.6 MHz处的衰减仍然相同,只是在840 kHz以下未能达到40 dB,差值不到1 dB。

\begin{figure}[H]
    \centering
    \includegraphics[width=0.7\linewidth]{00026.jpeg}
    \caption{使用标准值的滤波器原理图}
    \label{fig:filter_schematic_standard_cn}
\end{figure}

我认为我们完成了!很容易看出你如何继续实验滤波器,可能会以一些阻带衰减换取更少的通带衰减,或通带中更平滑的响应,我们甚至还没有开始研究输入和输出阻抗或延迟时间。尽管如此,这种设计可以用现成的组件构建,无需调整即可提供有用的性能。感谢W4ENE和他的出色软件。

\subsection{参考资料}

\begin{enumerate}
    \item 一套来自Tonnesoft的免费学生版电路设计实用程序可从\href{http://www.arrl.org/arrl-handbook-reference}{www.arrl.org/arrl-handbook-reference}下载。
    \item 所有以前的"实践无线电"专栏都可在ARRL成员的\href{http://www.arrl.org/hands-on-radio}{www.arrl.org/hands-on-radio}获取。
    \item 在美国,1600 kHz以上,AM电台白天限制为10 kW,夜间限制为1 kW。这些较小的电台不太可能像全功率50 kW发射机那样引起过载问题。
\end{enumerate}

\chapter{实验 #164:分频器}
\section{英文原文}
Dividers --- they seem like such trivial things. That is, until you need one and then the mad scramble through the reference books begins. There are electronic dividers for every aspect of a signal; voltage, current, power, phase, frequency. We'll take a look at voltage and current dividers this month. You should have these circuits tucked away in your radio toolbox, ready for use wherever the need arises.

\subsection{Passive Voltage and Current Dividers}
The passive version of voltage and current dividers in Figures 1 and 2 are about the simplest resistive circuits around. The equation for output voltage across R\textsubscript{3} (the load resistor) as a fraction of the input voltage, V\textsubscript{IN} is:

\begin{figure}[htbp]
    \centering
    \includegraphics{../epub_extracted/images/00027.jpeg}
    \caption{Voltage Divider}
    \label{fig:voltage-divider}
\end{figure}

\begin{figure}[htbp]
    \centering
    \includegraphics{../epub_extracted/images/00028.jpeg}
    \caption{Current Divider}
    \label{fig:current-divider}
\end{figure}

where the symbol || represents "in parallel with." Note that you have to take into account the loading effect of R\textsubscript{3} in parallel with R\textsubscript{2} to calculate the exact division ratio.

How much does the loading effect matter? Well, that also depends on the ratio of R\textsubscript{1} and R\textsubscript{2}, but let's take a typical situation with R\textsubscript{1} (the divider input resistor) being 10 × R\textsubscript{2}. Varying the ratio of R\textsubscript{3} to R\textsubscript{2} from 0.1 (R\textsubscript{3} = 10% of R\textsubscript{2}) to infinite (R\textsubscript{3} = an open circuit), the voltage divide ratio varies from 0.009 to 0.09, which is a 10:1 range! So yes, it matters. (An Excel spreadsheet showing how the ratios R\textsubscript{1}:R\textsubscript{2} and R\textsubscript{3}:R\textsubscript{2} affect V\textsubscript{OUT}/V\textsubscript{IN} is available on the "Hands-On Radio" web page for this experiment.) With this ratio of R\textsubscript{1}:R\textsubscript{2},

Where the voltage divider is essentially a series circuit, the current divider in Figure 2 is a parallel circuit. The figure shows a four-resistor divider, but any number of resistors, n, can be used. The current through resistor R\textsubscript{n} is:

I\textsubscript{n} / I\textsubscript{IN} = R\textsubscript{eq} / (R\textsubscript{n} + R\textsubscript{eq} )

where R\textsubscript{eq} is the parallel value of all resistors other than R\textsubscript{n}. In Figure 2, R\textsubscript{eq} = R\textsubscript{1} || R\textsubscript{2} || R\textsubscript{3}.

\subsection{Active Voltage Dividers}
Op-amps are great for building circuits that perform analog math operations. In fact, that's what the "op" in "op-amp" actually means --- "operational." The inexpensive IC amplifiers you buy for pennies today are the descendants of the original amplifiers used in analog computers to solve all sorts of hard problems.

The circuit in Figure 3 performs the operation of multiplication or division with the ratio determined by the ratio of the two resistors, R\textsubscript{i} and R\textsubscript{f}. (See Experiment #3 for an explanation of how the circuit works.)

\begin{figure}[htbp]
    \centering
    \includegraphics{../epub_extracted/images/00030.jpeg}
    \caption{Active Voltage Divider}
    \label{fig:active-voltage-divider}
\end{figure}

Add an inverting buffer circuit to get rid of the minus sign and you have a multiplier or divider, depending on whether the value of R\textsubscript{f} is larger or smaller than that of R\textsubscript{i}.

\subsection{Active Current Dividers}
You may have seen the circuit in Figure 4A before --- the current mirror. When the transistors Q1 and Q2 are closely matched, the collector current through Q2 will equal or "mirror" the current in Q1. Current mirrors work because matched bipolar junction transistors with the same base-to-emitter voltage will have the same collector currents. Because the bases and emitters are connected together, V\textsubscript{BE} must be the same for both transistors and so are the collector currents.

\begin{figure}[htbp]
    \centering
    \includegraphics{../epub_extracted/images/00031.jpeg}
    \caption{Current Mirror}
    \label{fig:current-mirror}
\end{figure}

The two transistors are matched if they are made out of the same material (Si, Ge, GaAs, etc), have the same current gains (β), and are at the same temperature. Without careful measurements, it's hard to find two individual transistors that are precisely matched. If the transistors are part of the same IC, however, they will be very closely matched. For example, the MPQ2222 contains four 2N2222 transistors in either DIP or SMT packages.

The current mirror can be turned into a current divider by adding resistance in the emitter circuit of each transistor. With the bases connected together, the voltage from the bases to common is equal for both transistors:

I\textsubscript{n} / I\textsubscript{IN} = R\textsubscript{eq} / (R\textsubscript{n} + R\textsubscript{eq})

If V\textsubscript{BE1}=V\textsubscript{BE2}, then the transistor currents are controlled by the ratio of the two resistors: I\textsubscript{IN}/I\textsubscript{OUT} = R\textsubscript{2}/R\textsubscript{1} and I\textsubscript{OUT} = I\textsubscript{IN}(R\textsubscript{1}/R\textsubscript{2}). In the truest sense, this circuit doesn't divide a single current into two individual currents, but it does create an output current that is a known and controllable fraction of the first.

\subsection{Probing the Effect of Reactance}
There is no requirement that the voltage divider in Figure 1 be constructed only from resistors. Reactances will do nicely, and the Colpitts oscillator relies on a capacitive voltage divider in its feedback circuit. (See Experiment #34, "RF Oscillators, Part 1," for a description of the Colpitts circuit.) If reactances and resistances are mixed, there will be a phase shift between the input and output voltage that depends on frequency. If inductive and capacitive reactances are used, the result is a series tuned circuit and a resonance is created at the frequency where the reactances are equal, creating equal and opposite phase shifts that cancel, as well.

In keeping with our recent theme of oscilloscope-related ideas, a voltage divider with reactance describes the combination of a scope input connected to a 10× probe. Figure 5 shows the scope's input has approximately 20 pF of parallel capacitance and the probe has a 10× smaller adjustable capacitance of 2 pF in parallel with its series resistance.

\begin{figure}[htbp]
    \centering
    \includegraphics{../epub_extracted/images/00032.jpeg}
    \caption{Scope Probe as Voltage Divider}
    \label{fig:scope-probe-divider}
\end{figure}

While it is expected that the scope's high-impedance input might include some capacitance, what is the function of the small adjustable capacitance in the probe? At dc, both capacitors have infinite reactance and can be ignored, leaving a 10:1 voltage divider circuit. As the signal frequency increases however, the reactance of the scope's input capacitance goes down. This would act as a low-pass filter and higher frequency signals and signal components would be attenuated, distorting the input signal.

The scope probe capacitance acts as a compensating high-pass filter. As frequency increases, its reactance goes down, creating a larger signal at the scope input. The effect --- when the probe capacitor is properly adjusted --- exactly balances the low-pass effect of the scope's input capacitance.

In your scope's manual, you will find instructions for adjusting a probe's compensating capacitor. The probe is connected to a square wave, usually provided at a test point right on the scope's front panel. This square wave is rich in harmonics that give it the sharp corners. If the probe's capacitance is too high, the high-pass effect allows too much harmonic energy into the scope and the square wave overshoots on each transition. If the probe's capacitance is too low, the low-pass effect dominates and the square wave's corners are rounded. By adjusting the probe compensating capacitor until the square wave's corners are as sharp and close to right angles as possible, the low- and high-pass effects balance and the frequency response of the combined probe and scope input is flat across the bandwidth of the scope.

Don't take my word for it --- fire up your scope and give it a try! If you don't have a manual for your scope, the online probe compensation tutorial from Pico Technology will fill in the blanks for you. Carefully adjust the compensating capacitor while watching the effect on the test square wave's corners. You will clearly see the high- and low-pass effects.

\begin{enumerate}
    \item All previous "Hands-On Radio" experiments are available to ARRL members at \url{http://www.arrl.org/hands-on-radio}.
    \item \url{http://www.picotech.com/library/application-note/how-to-tune-x10-oscilloscope-probes}
\end{enumerate}

\chapter{实验 #167:舒适收听的清晰音频}
\section{英文原文}
Throughout my years in ham radio, I've frequently heard something like, "It's a good receiver, but it just sounds harsh (or hissy)." Also popular is, "That radio just wears out my ears." And the ever-present, "The signals all sound like mush." Some of those might be symptoms of problems elsewhere in the receive chain, like in the Automatic Gain Control (AGC) sub-system. Quite often, though, the problems stem from a weak audio output stage.

The latest radios cost hundreds (if not thousands) of dollars and have sophisticated RF electronics and processing power. There should be an equal amount of attention paid to the actual interface with you, the user, and that is the audio output stage.

\subsection{Frequency Response}
In bygone days when "hi-fi sets" used vacuum tubes, frequency response was a serious part of "specs-man-ship." Bigger (heavier) audio output transformers helped extend the bottom end below 100 Hz. Higher-gain tubes and circuits extended the high end above 10 kHz. There were actually meaningful differences between the frequency response of different models. Today's integrated circuits (ICs) offer responses from well below 10 Hz to 20 kHz and beyond human hearing. Problem solved?

Not really. A communications receiver can actually have a frequency response that's too wide. Maybe for broadcast-quality AM, a wide response is necessary for that "warm" sound AM operators tout, but for HF SSB/CW, it just allows a lot of noise and interference into your "channel." On VHF+ FM, too much bass response means you'll begin to hear those no-longer-sub-audible tones. Too much high-frequency audio and noise will drive you crazy. You need a response appropriate to what you're trying to copy and no more; 200 to 5000 Hz is more than adequate, because IF filtering will limit signal audio response to much less than that.

\subsection{Distortion}
Another gremlin of audio stages is distortion, specified as total harmonic distortion (THD). Measured for a single-tone sine wave at a specific power level, THD is the sum of the power of all distortion products divided by the power of the fundamental frequency. Home entertainment audio gear sports THD of far less than 1%, but a typical ham transceiver specifies THD as high as several percent for a specific power level and load impedance. Is this a bad thing?

A lot of THD is definitely a bad thing, as it makes the receiver sound harsh or shrill even if the received signal is clean. You can see distortion by viewing the audio output of your rig with PC-based spectrum analyzer software such as DL4YHF's Spectrum Lab. (Use the speaker output for lowest output impedance.) Tune in a steady carrier, such as WWV received in USB or LSB mode, and experiment with different audio levels, watching the relative amount of harmonics as you change the volume. Place different loads on the output by using a splitter to connect one or more paralleled speakers to the output. You'll see the harmonic content increasing with heavier loads and higher volume.

How can you tell if your radio has an audio problem? The Hands-On Radio web supplement for this column (www.arrl.org/hands-on-radio) has a short set of diagnostics.

\subsection{Hiss}
Receiver hiss is a high-frequency "white" or wide-spectrum noise generated by the internal electronic devices. It is present whenever the receiver is turned on, regardless of what external noise is coming in through the antenna connector. While the hiss may not be loud, over long periods, it is fatiguing as it continually stimulates your inner ear and the auditory system. Rather than apply external filters, the best way to remove the hiss is by rolling off gain above a few kHz in the audio circuitry.

\subsection{Audio Power Output Stages}
For simple construction, it's been hard to beat the venerable LM386 audio power amplifier IC. It has been used for the audio output stage of simple radios for many years, and the price is certainly reasonable. But it can be hissy with a frequency response of 300 kHz and has a ho-hum THD of 0.2% at only 125 mW of output power. This is okay for driving light headphones or a small speaker, but we can do better.

The proliferation of portable music players drives many improvements in audio output electronics. For example, the TPA1517 is made specifically to drive stereo headphones or speakers. Its THD is still 0.1% at an output power of 1 W, which is plenty to drive even full-size headphones and small speakers. (This comes at the price of higher supply current over a pair of LM386 ICs.) Figure 1 shows the schematic for a bare-bones amplifier using this chip. The datasheet has additional circuit ideas.

\begin{figure}[htbp]
    \centering
    \includegraphics{../epub_extracted/images/00033.jpeg}
    \caption{Bare-bones Amplifier using TPA1517}
    \label{fig:tpa1517-amplifier}
\end{figure}

There are many other audio output ICs made for driving low (4 Ω) and medium (32 Ω) load impedances. You can find them by searching for audio amplifiers from distributors like Digi-Key Electronics (digikey.com) and Mouser Electronics (mouser.com). From the many options, optimize your selection for power consumption, power output, or some desirable set of features. For example, the TPA1517 has a mute/standby control input, perfect for use in a transceiver.

Experimenters might also want to build an audio amplifier out of discrete components to become more familiar with using transistors. In that case, Figure 2 shows an audio power stage from Experimental Methods in RF Design. This is a high-gain (46 dB, x400) amplifier that can provide clean audio to headphones or small speakers. Note the parallel combination of a 100 kΩ and 220 pF capacitor for both the input buffer stage and in the feedback loop from the output to the middle op-amp. This combination rolls off signals above 7 kHz (approximately 1⁄2πRC) to limit hiss and high-frequency noise. A low-noise op-amp, such as the NE5532, should be used, as well.

\begin{figure}[htbp]
    \centering
    \includegraphics{../epub_extracted/images/00034.jpeg}
    \caption{Audio Power Stage from Experimental Methods in RF Design}
    \label{fig:discrete-audio-amplifier}
\end{figure}

Regardless of which design or IC you use, good wiring practices are required. Speaker or headphone connections should be made directly to the circuit board --- do not use the chassis as the only signal return. If you are building the amplifier in a metal enclosure and using a stereo headphone jack, the body (sleeve contact) of the jack should be securely connected to the enclosure to keep common-mode RF on the cable outside. Run a lead from the jack's ground terminal to the common ground point of the amplifier. A ferrite bead or two on the audio output lead helps keep RF out of the amplifier, as well.

\subsection{Headphones}
Let's not forget about the actual final audio stage --- the headphones or speaker. All this fine audio is wasted if you're not using good-quality headphones and speakers. Noise-canceling headphones are also beneficial, and they should be comfortable --- with audio this clean, you'll be wearing them for many hours, enjoying what the band really sounds like!

\begin{enumerate}
    \item \url{http://www.qsl.net/dl4yhf/spectra1.html}
    \item \url{http://www.ti.com/lit/ds/symlink/lm386.pdf}
    \item \url{http://www.ti.com/lit/ds/symlink/tpa1517.pdf}
    \item W. Hayward, W7ZOI, R. Campbell, KK7B, R. Larkin, W7PUA, Experimental Methods in RF Design, ARRL, 2003.
    \item J. Brown, K9YC, "A Ham's Guide to RFI, Ferrites, Baluns, and Audio Interfacing," Rev 5a, \url{http://audiosystemsgroup.com/RFI-Ham.pdf}
\end{enumerate}

\chapter{实验 #174:开关放大器}
\section{英文原文}
As part of our General and Amateur Extra class license studies, we have to learn a few things about amplifier classes. Class refers to how the amplifying devices (tube or transistor) operate, and how the output impedance matching and filter circuits work.

\subsection{Class Background}
The early Hands-On Radio experiments (#1 and #2, for example) stress the importance of quiescent- or Q-point. The Q-point is the combination of voltage and current in a circuit when no input signal is present.

The output will follow a load line (see Experiment #77) that describes voltages and currents possible for the load at the amplifier's output. The device's output current can be increased until it reaches saturation, the point at which further changes in input have no effect. Similarly, the output current can be reduced to zero at cutoff.

The location of the Q-point on the load line determines how "close" the amplifier is to saturation or cutoff. The amplifier circuit's configuration, Q-point location, load, gain, and output circuit characteristics all combine to determine the amplifier's output waveform, linearity, and efficiency. There are several common combinations and resulting behaviors of the amplifier --- these are called amplifier classes.

\subsection{Classes A, B, and AB}
The easiest way to compare these classes is in terms of conduction angle --- the number of degrees in one cycle during which the device is conducting current in its linear region, from 0° to 360°.

In Class A operation, the device (Figure 1A) conducts during the entire cycle; a conduction angle of 360°. If the load line is linear and the input doesn't "overdrive" the circuit out of its linear region (between saturation and cutoff), the output waveform is an exact reproduction of the input. This is a true "linear amplifier." Class A amplifiers have poor efficiency (about 50% maximum) because of their continuous current.

\begin{figure}[htbp]
    \centering
    \includegraphics{../epub_extracted/images/00035.jpeg}
    \caption{Amplifier Classes}
    \label{fig:amplifier-classes}
\end{figure}

In Class B operation, the conduction angle is 180° (half of the cycle). Class B amplifiers are more efficient (a maximum of 78.5%) than Class A, because they are only in their linear region half the time. But they are quite non-linear. Class B amplifiers typically use a pair of devices that conduct on alternate half cycles (push-pull operation). Their outputs are combined to produce a complete output signal.

Class AB operation (Figure 1A) "splits the difference" between Class A and B with a conduction angle between 180° and 360°. This is a compromise --- Class AB amplifiers are more linear than Class B, and more efficient than Class A. Class AB is very popular for SSB operation, whether tube or transistor. Some readers are probably thinking, "But Class AB is still really non-linear. How can that work and still put out a clean signal?" Hold that thought.

\subsection{Class C and Harmonics}
The final "traditional" class is Class C (Figure 1B), in which the device is in its linear region for a short period, with a conduction angle of only 25 – 50% (90 – 180°). The output current consists of sine-like pulses. Class C amplifiers are very non-linear and are used to amplify signals for which only the frequency is preserved --- FM and CW, for example. Because the device is only in the linear region for short periods, the efficiency of Class C amplifiers can be as high as 85%.

Clearly, Class AB, B, and C amplifiers generate a lot of harmonic distortion, because they cut off some or most of the waveform. (Remember that we are talking about an RF waveform, not audio.) There would be significant in-band distortion products as well, particularly for Class C.

Let's do an experiment. Build the common-emitter amplifier of Experiment #1 with a 50 kΩ pot instead of R1 and R2. Adjust the bias to change the Q-point, and with a 1 kHz input signal, observe the output signal spectrum by using a sound card audio analyzer (see Experiments #64 and #65). Compare the spectra for Class A, B, AB, and C operation.

To get rid of the harmonic energy requires the output circuit to act as a filter as well as provide an impedance match. When you adjust the TUNE function of a tube amplifier, you are bringing the output circuit to resonance where it acts as a band-pass filter. (If you have a linear amplifier, tune it up at reduced power into a dummy load, then listen for your second or third harmonic on a second receiver. Change the tuning of the amplifier while watching the S-meter to get an idea of signal level. You should be able to see (and hear) the effects of mistuning.

\subsection{Class D and Switching Amplifiers}
Class D goes farther than Class C --- the amplifier acts as a pulse width modulated (PWM) switch by setting the bias and drive so that the device is either fully on (saturated) or fully off (cutoff). As Experiment #9 shows, these two states have low-power dissipation compared to the linear region. Class D is a type of switching amplifier.

The switch operates at f\textsubscript{PWM}, which is many times the frequency of the signal to be amplified. The duration of each pulse is controlled by the amplitude of the input waveform to change the pulse's average energy. The switch is turned on and off so rapidly that it spends very little time in the linear region, and its conduction angle is very small. This creates a series of current pulses.

A low-pass filter then removes the high-frequency switching components and leaves only the desired average-energy signal. Essentially, a Class D amplifier is a switching mode power supply, with the input signal acting as the output voltage control signal.

Because the switching frequency must be many times the frequency of the signals to be amplified, Class D circuits are only used for audio. The switching frequency is typically greater than 100 kHz and easily separated from the desired audio range below 20 – 30 kHz. There are many Class D audio amplifier integrated circuits (ICs), and because they are so efficient (more than 90%), they are the circuit of choice in battery-powered gear like smartphones and music players. (The Sparkfun BOB-11044 Class D amplifier kit based on the TPA2005D1 chip --- www.sparkfun.com --- only costs a few dollars and makes a great experiment.)

\subsection{Class E and F}
At RF, the frequency of the switching pulses, f\textsubscript{0}, is so high that it takes more work to keep the device out of the linear region. Remember that current should be high or voltage should be high, but not both at the same time. This is accomplished by the tuned output networks, as shown in Figure 2, described more completely by Iulian Rosu, YO3DAC/VA3IUL.

\begin{figure}[htbp]
    \centering
    \includegraphics{../epub_extracted/images/00036.jpeg}
    \caption{Class E and F Amplifier Output Networks}
    \label{fig:class-e-f-networks}
\end{figure}

For Class E, the output network's series-LC network is tuned to the desired RF frequency of operation. The capacitance across the switch is charged and discharged at the same rate. The tuned circuit causes voltage at the device to go to zero just as the current is beginning to increase. This keeps switching losses low, and the tuned circuit takes care of the harmonics. A separate tuned network is required for each band of operation.

The related Class F amplifier uses a parallel-LC trap in the output circuit to block f\textsubscript{3}, the third and strongest harmonic of a square wave, while passing the fundamental (f\textsubscript{0}). Again, with careful tuning, the voltage and current are never maximum at the same time, the switch stays out of its linear region, and efficiency is high.

\subsection{Out-Classed}
There are other classes, as well. Class G is similar to a Class AB amplifier, but switches between two voltage levels to reduce power dissipation at low signal levels. Class I uses two devices driven with complementary pulse duty cycles to cancel harmonics and follow the input waveform. Class S is a variation on Class D, and Class T uses digital signal processing (DSP) to optimize pulse widths in a Class D amplifier for better performance.

\begin{enumerate}
    \item All Hands-On Radio experiments are available online to ARRL members at \url{http://www.arrl.org/hands-on-radio}.
    \item I. Rosu, YO3DAC/VA3IUL, "RF Power Amplifiers," \url{http://www.qsl.net/va3iul}.
    \item "Amplifier Classes," Electronic Tutorials, \url{http://www.electronics-tutorials.ws/amplifier/amplifier-classes.html}.
\end{enumerate}

\part{天线与传播}
\chapter{实验 #124:Beta 匹配}

\section{英文原文}

This month's column will show you four different types of antenna feed-point impedance matching that all work in the same way even though they look quite different. In addition, I'll introduce a new friend for your computer toolbox — \textbf{SimSmith} by Ward Harrington, AE6TY.

The basic problem — impedance matching using inductance across an antenna's feed-point — is the same, but the solution goes by several names. This makes it more difficult to understand because giving the same things a different name (or giving different things the same name) is confusing. Nevertheless, as you read the column, keep in mind that all of the techniques presented here accomplish the same task.

\subsection{The Essential Problem and its Solution}

While discussing impedance matching of antennas, it's natural for most hams to imagine impedances \textbf{greater} than 50 \textbf{Ω}. The examples we use in learning about SWR primarily use higher impedance values for the calculation Z\textsubscript{LOAD} / Z\textsubscript{0}: 100 / 50 = 2:1, for example. In actuality, it's common for Z\textsubscript{LOAD} to be \textbf{less} than 50 \textbf{Ω}. A Yagi's driven element feed-point impedance is often in the range of 20 to 30 \textbf{Ω} and the natural impedance for a quarter-wave monopole (the common ground plane) is around 35 \textbf{Ω}.

Transforming this lower impedance to 50 \textbf{Ω} doesn't lend itself well to the most common techniques. The impedance ratio of 1.4 - 2:1 doesn't fall into the usual 1:2:4:9-type ratio of "easy" transformer impedance ratios, nor are there coaxial cables with a characteristic impedance of 35 to 40 \textbf{Ω} that would enable simple quarter-wave "Q sections" to do the job.\textsuperscript{1} [A pair of 75 \textbf{Ω} cables in parallel will get in range, though they are a bit clunky. — \textbf{Ed}.]

Nevertheless, the clever approach illustrated in Figure 1 gives the electrical schematic view of how this problem is solved. First, we have to give up the usual assumption that the feed-point impedance is resonant — that is, R + \textbf{j}0 \textbf{Ω}. It's part of our "ham DNA" that makes us think antennas need to be resonant to work, but in this case resonance actually makes the problem harder. 

\begin{figure}[H]
    \centering
    \includegraphics[width=0.7\linewidth]{00037.jpeg}
    \caption{Impedance Matching Approach}
    \label{fig:impedance_matching}
\end{figure}

By making the antenna a little shorter than its resonant length, the feed-point impedance becomes slightly capacitive (A). That capacitive reactance can then be used as part of an \textbf{L} network by adding an inductor across the feed point as shown in (B). Redrawing the circuit (C) results in the \textbf{L} network's more familiar form. (See Experiment #21 for more about \textbf{L} networks.)

There is a bit of a trick involved. You can't have just \textbf{any} amount of resistance and capacitive reactance. The combination has to be in the right range so that adding the inductance transforms the impedance to 50 + \textbf{j}0 \textbf{Ω}. How do you figure that out?

\subsection{SimSmith}

The standard way of visualizing transmission line and impedance matching mechanics is by using a Smith chart. (If you are unfamiliar with the Smith chart, read the introduction in Hands-On Radio experiments #59-61. Recent editions of the \textbf{ARRL Antenna Book} include a detailed tutorial on the Smith chart, either in print or on the CD-ROM.\textsuperscript{2}) Yesterday's compass and straightedge have been replaced by interactive computer software such as the easy-to-use \textbf{SimSmith} (\href{http://www.ae6ty.com/Smith_Charts.html}{www.ae6ty.com/Smith_Charts.html}). Written in Java, AE6TY's free tool is available for a wide variety of computers. Furthermore, he has created videos and guides to explain how to use the software and the Smith chart, so there is no reason not to have a copy and begin learning the power of "seeing inside" transmission lines and matching networks. Before taking a look at our current problem using \textbf{SimSmith}, allow me to point out a few of its features that I will use here.

First, as you can see in Figure 2, the program shows the usual \textbf{constant-resistance circles} and \textbf{constant-reactance arcs }in light red. Less usual are the \textbf{constant-conductance circles} and \textbf{constant-susceptance arcs} shown in light blue. (Susceptance, B, is the reciprocal of reactance, X.) The normalized 1.0 + \textbf{j}0 point is at the center. (In our 50 \textbf{Ω} world, that represents an impedance of 50 + \textbf{j}0 \textbf{Ω} or an admittance of 0.02 + \textbf{j}0 S, where S is the symbol for siemens, the unit of conductance.) From any impedance or admittance point on the chart, adding resistance or reactance in series "moves" along the red circles and arcs. Adding resistance or reactance in parallel or \textbf{shunt} "moves" along the blue circles and arcs. (Instead of move, I'll use the correct term \textbf{transform} from here on.)

\begin{figure}[H]
    \centering
    \includegraphics[width=0.7\linewidth]{00038.jpeg}
    \caption{SimSmith Smith Chart}
    \label{fig:simsith_chart}
\end{figure}

At the upper left, \textbf{SimSmith} shows the transmission line circuit you've constructed, including a load at the left and the source (\textbf{generator} in Smith chart speak) at the right. A collection of subcircuits is available at the lower left — there is everything from series and shunt components, to stubs, to tuned circuits, to general-purpose blocks that perform specific math functions. Add a subcircuit by drag-and-dropping it onto the transmission line circuit at the desired point. Then fill in the values (too small to reproduce in the figure) underneath the subcircuit. \textbf{SimSmith} does the rest. Let's try it.

\subsection{Using a Coil}

Figure 2 is a screen shot from \textbf{SimSmith} showing the equivalent of Figure 1C. An easier to read version is on the Hands-On Radio web page. By entering \textbf{1.5} in the generator's SWR value window, \textbf{SimSmith} drew a \textbf{constant-SWR} \textbf{circle} around the center — all points within this circle represent SWR values of 1.5:1 or less — for reference. I selected a frequency of 7 MHz because I use a quarter-wave vertical on 40 meters at my station. Game on!

While designing an antenna, achieving the goal of being able to use a single shunt inductor as your matching network requires the right feed-point impedance, shown as the load on \textbf{SimSmith}. Adding the shunt inductor will transform the impedance counter-clockwise parallel to one of the light blue constant-conductance circles as shown by a heavy blue line. (The inductor adds susceptance but does not affect the conductance. Parallel capacitance transforms clockwise.) The feed-point impedance should be designed such that adding inductive susceptance can transform the impedance to within the desired constant-SWR circle.

Starting with the antenna's feed-point impedance at the left, I've entered a value of 35 – \textbf{j}25 \textbf{Ω}. This is a reasonable value for an aluminum tubing vertical over a good ground system, adjusted to a bit less than its natural resonant length. By "fiddling with" (technical term) the value of inductance, I found that a value of 1.7 \textbf{µ}H presented a resulting impedance of 52.3 – \textbf{j} 0.4 \textbf{Ω} to the feed line for an SWR value of 1.05:1. I'd say that works. In fact, this is quite close to the size of inductor I use to match my vertical antenna on 40 meters.

\subsection{Try a Shorted Stub}

Assuming you've downloaded \textbf{SimSmith} and are running it, enter the same values for load impedance but replace the parallel inductor with a shorted stub — the sub-circuit directly below the parallel inductor. Drag-and-drop the parallel inductor sub-circuit into the trash can symbol. Then drag the shorted-stub subcircuit to the transmission line circuit. 

From the values \textbf{SimSmith }assumes about the stub (such as it being made of 50 \textbf{Ω} coax), adjust the length until you get about the same match as with the inductor (≈55°). For fun, increase stub length to 90° — the stub now presents an open circuit so that it does nothing. (See Experiment #22 for more about stubs.)

Can you use a longer length to create an inductive feed-point impedance and match it with a parallel capacitance? Change the feed-point impedance to 35 + \textbf{j}25 Ω and find out. (300 pF should get you close.) 

\subsection{Using a Hairpin}

Along with shunt inductance and shorted stubs, the third name and fourth idea covered here is the \textbf{hairpin} or \textbf{beta match} shown in Figure 3. You should recognize the matching device as a shorted stub of open-wire transmission line. Typical hairpins are made of heavy wire with wide spacing that results in a high characteristic impedance. What length of hairpin is required to match our original load if its characteristic impedance is 300 \textbf{Ω}? (Roughly15° or 2.8 feet at 7 MHz for a 95% velocity factor.) 

\begin{figure}[H]
    \centering
    \includegraphics[width=0.7\linewidth]{00039.jpeg}
    \caption{Hairpin or Beta Match}
    \label{fig:hairpin_beta}
\end{figure}

If the driven element is insulated and electrically balanced, the very center of the feed-point and the matching hairpin are electrically neutral. Hy-Gain antennas add mechanical stability to the design by attaching that point of the hairpin to the antenna boom — also electrically neutral with respect to the feed line — creating the \textbf{beta match}.

\subsection{A Common Theme}

You should now see the common theme of all four matching designs. By creating capacitive reactance in the feed-point impedance and applying a shunt inductance across the feed point, the ratio and phase of voltage and current can be altered to create a purely resistive impedance of the desired value. 

\subsection{References}

\begin{enumerate}
    \item See Hands-On Radio experiment #81, "Synchronous Transformers." All previous Hands-On Radio experiments are available to ARRL members at \href{http://www.arrl.org/hands-on-radio}{www.arrl.org/hands-on-radio}.
    \item The ARRL Antenna Book, 22nd Edition. Available from your ARRL dealer or the ARRL Bookstore, ARRL order no. 6948. Telephone 860-594-0355, or toll-free in the US 888-277-5289; \href{http://www.arrl.org/shop}{www.arrl.org/shop}; \href{mailto:pubsales@arrl.org}{pubsales@arrl.org}.
\end{enumerate}

\section{中文翻译}

本月的专栏将向你展示四种不同类型的天线馈电点阻抗匹配,尽管它们看起来很不同,但工作原理相同。此外,我将为你的计算机工具箱介绍一个新朋友 —— Ward Harrington(AE6TY)的 \textbf{SimSmith}。

基本问题 —— 使用天线馈电点上的电感进行阻抗匹配 —— 是相同的,但解决方案有几个名称。这使得理解更加困难,因为给相同的东西不同的名称(或给不同的东西相同的名称)会造成混淆。尽管如此,当你阅读本专栏时,请记住,这里介绍的所有技术都完成相同的任务。

\subsection{基本问题及其解决方案}

在讨论天线的阻抗匹配时,大多数业余无线电爱好者自然会想象阻抗\textbf{大于}50\textbf{Ω}。我们在学习SWR时使用的例子主要使用更高的阻抗值进行计算Z\textsubscript{LOAD} / Z\textsubscript{0}:例如100 / 50 = 2:1。实际上,Z\textsubscript{LOAD}\textbf{小于}50\textbf{Ω}是很常见的。八木天线的驱动元件馈电点阻抗通常在20到30\textbf{Ω}之间,四分之一波长单极天线(常见的接地平面)的自然阻抗约为35\textbf{Ω}。

将这种较低的阻抗转换为50\textbf{Ω}并不适合最常见的技术。1.4-2:1的阻抗比不属于通常的1:2:4:9型"容易"变压器阻抗比,也没有特性阻抗为35到40\textbf{Ω}的同轴电缆可以实现简单的四分之一波长"Q部分"来完成这项工作。\textsuperscript{1} [一对并联的75\textbf{Ω}电缆可以达到范围,尽管它们有点笨重。——\textbf{编辑}。]

尽管如此,图1中所示的巧妙方法给出了如何解决这个问题的电气原理图。首先,我们必须放弃通常的假设,即馈电点阻抗是谐振的 —— 即R + \textbf{j}0\textbf{Ω}。我们的"业余无线电DNA"使我们认为天线需要谐振才能工作,但在这种情况下,谐振实际上使问题更加困难。

\begin{figure}[H]
    \centering
    \includegraphics[width=0.7\linewidth]{00037.jpeg}
    \caption{阻抗匹配方法}
    \label{fig:impedance_matching_cn}
\end{figure}

通过使天线比其谐振长度短一点,馈电点阻抗变得略微容性(A)。然后,通过在馈电点添加电感,如(B)所示,该容性电抗可以用作\textbf{L}网络的一部分。重新绘制电路(C)会得到\textbf{L}网络更熟悉的形式。(有关\textbf{L}网络的更多信息,请参见实验#21。)

这里涉及一点技巧。你不能有\textbf{任何}数量的电阻和容性电抗。组合必须在正确的范围内,以便添加电感将阻抗转换为50 + \textbf{j}0\textbf{Ω}。你如何计算出来?

\subsection{SimSmith}

可视化传输线和阻抗匹配机制的标准方法是使用史密斯圆图。(如果你不熟悉史密斯圆图,请阅读"实践无线电"实验#59-61中的介绍。\textbf{ARRL天线手册}的最新版本包括关于史密斯圆图的详细教程,无论是印刷版还是CD-ROM版。\textsuperscript{2})昨天的指南针和直尺已经被交互式计算机软件所取代,例如易于使用的\textbf{SimSmith}(\href{http://www.ae6ty.com/Smith_Charts.html}{www.ae6ty.com/Smith_Charts.html})。AE6TY的免费工具是用Java编写的,可用于各种计算机。此外,他还创建了视频和指南来解释如何使用该软件和史密斯圆图,因此没有理由不拥有一份副本并开始学习"看到传输线和匹配网络内部"的力量。在使用\textbf{SimSmith}查看我们当前的问题之前,请允许我指出我将在这里使用的一些功能。

首先,正如你在图2中看到的,程序显示通常的\textbf{恒定电阻圆}和\textbf{恒定电抗弧}为浅红色。不太常见的是显示为浅蓝色的\textbf{恒定导纳圆}和\textbf{恒定电纳弧}。(电纳B是电抗X的倒数。)归一化的1.0 + \textbf{j}0点位于中心。(在我们的50\textbf{Ω}世界中,这表示阻抗为50 + \textbf{j}0\textbf{Ω}或导纳为0.02 + \textbf{j}0 S,其中S是西门子的符号,电导的单位。)从圆图上的任何阻抗或导纳点,串联添加电阻或电抗会沿着红色圆和弧"移动"。并联或\textbf{分流}添加电阻或电抗会沿着蓝色圆和弧"移动"。(从这里开始,我将使用正确的术语\textbf{变换}而不是移动。)

\begin{figure}[H]
    \centering
    \includegraphics[width=0.7\linewidth]{00038.jpeg}
    \caption{SimSmith史密斯圆图}
    \label{fig:simsith_chart_cn}
\end{figure}

在左上角,\textbf{SimSmith}显示你构建的传输线电路,包括左侧的负载和右侧的源(史密斯圆图中的\textbf{发生器})。左下角有一系列子电路 —— 从串联和分流组件,到短截线,到调谐电路,到执行特定数学函数的通用块,应有尽有。通过将子电路拖放到传输线电路的所需点来添加子电路。然后在子电路下方填写值(太小而无法在图中重现)。\textbf{SimSmith}会完成其余工作。让我们尝试一下。

\subsection{使用线圈}

图2是\textbf{SimSmith}的屏幕截图,显示了图1C的等效物。更易于阅读的版本在"实践无线电"网页上。通过在发生器的SWR值窗口中输入\textbf{1.5},\textbf{SimSmith}在中心周围绘制了一个\textbf{恒定SWR} \textbf{圆} —— 该圆内的所有点表示SWR值为1.5:1或更小 —— 作为参考。我选择了7 MHz的频率,因为我在我的电台使用40米的四分之一波长垂直天线。开始!

在设计天线时,实现能够使用单个分流电感作为匹配网络的目标需要正确的馈电点阻抗,如\textbf{SimSmith}上的负载所示。添加分流电感将使阻抗逆时针平行于一条浅蓝色恒定导纳圆,如粗蓝线所示。(电感增加电纳但不影响电导。并联电容顺时针变换。)馈电点阻抗应设计为,添加电感电纳可以将阻抗变换到所需的恒定SWR圆内。

从左侧的天线馈电点阻抗开始,我输入了35 – \textbf{j}25\textbf{Ω}的值。这对于在良好接地系统上的铝管垂直天线来说是合理的值,调整到略小于其固有谐振长度。通过"摆弄"(技术术语)电感值,我发现1.7\textbf{µ}H的值向馈线呈现52.3 – \textbf{j}0.4\textbf{Ω}的结果阻抗,SWR值为1.05:1。我认为这可行。事实上,这非常接近我用于匹配40米垂直天线的电感大小。

\subsection{尝试短截线}

假设你已经下载了\textbf{SimSmith}并正在运行它,输入相同的负载阻抗值,但用短截线替换并联电感 —— 并联电感正下方的子电路。将并联电感子电路拖放到垃圾桶符号中。然后将短截线子电路拖到传输线电路上。

根据\textbf{SimSmith}对短截线的假设值(例如它由50\textbf{Ω}同轴电缆制成),调整长度直到获得与电感相同的匹配(≈55°)。为了好玩,将短截线长度增加到90° —— 短截线现在呈现开路,因此它什么也不做。(有关短截线的更多信息,请参见实验#22。)

你能使用更长的长度来创建电感馈电点阻抗并用并联电容匹配它吗?将馈电点阻抗更改为35 + \textbf{j}25Ω并找出答案。(300 pF应该接近。)

\subsection{使用发夹}

除了分流电感和短截线外,这里介绍的第三个名称和第四个想法是图3所示的\textbf{发夹}或\textbf{beta匹配}。你应该将匹配装置识别为开路传输线的短截线。典型的发夹由粗线制成,间距宽,导致高特性阻抗。如果其特性阻抗为300\textbf{Ω},匹配我们原始负载需要多长的发夹?(对于95%的速度因子,在7 MHz时约为15°或2.8英尺。)

\begin{figure}[H]
    \centering
    \includegraphics[width=0.7\linewidth]{00039.jpeg}
    \caption{发夹或Beta匹配}
    \label{fig:hairpin_beta_cn}
\end{figure}

如果驱动元件是绝缘的且电平衡的,则馈电点的正中心和匹配发夹是电中性的。Hy-Gain天线通过将发夹的该点连接到天线 boom 上来增加设计的机械稳定性 —— 相对于馈线也是电中性的 —— 从而创建\textbf{beta匹配}。

\subsection{共同主题}

现在你应该看到所有四种匹配设计的共同主题。通过在馈电点阻抗中创建容性电抗并在馈电点上应用分流电感,可以改变电压和电流的比率和相位,以创建所需值的纯电阻阻抗。

\subsection{参考资料}

\begin{enumerate}
    \item 参见"实践无线电"实验#81,"同步变压器"。所有以前的"实践无线电"实验都可在ARRL成员的\href{http://www.arrl.org/hands-on-radio}{www.arrl.org/hands-on-radio}获取。
    \item The ARRL Antenna Book,第22版。可从你的ARRL经销商或ARRL书店获取,ARRL订单号6948。电话:美国免费电话888-277-5289,或860-594-0355;传真:860-594-0303;\href{http://www.arrl.org/shop}{www.arrl.org/shop};\href{mailto:pubsales@arrl.org}{pubsales@arrl.org}。
\end{enumerate}

\chapter{实验 #133:扩展双 Zepp 天线}

\section{英文原文}

The reference to the Extended Double Zepp (EDZ) antennas in Experiment #131 (on the coax to open-wire balun) certainly generated some interest!\textsuperscript{1} The Zepp is one of the oldest antennas. Patented in 1909 by Hans Beggerow (German patent 225204, \href{http://www.aktuellum.com/circuits/antenna-patent}{www.aktuellum.com/circuits/antenna-patent}) the antenna is shown suspended vertically from a balloon (naming it after the Zeppelin airship came later) looking for all the world like an upside-down J-pole, which, in fact, it is! 

The Zepp is usually imagined as horizontal, as in Figure 1A, and the J-pole as vertical, but electrically they are essentially the same antenna. Both use a quarter-wave section of transmission line to convert the high impedance at the end of a half-wave radiating element to a lower impedance suitable for attaching to feed line.

\begin{figure}[H]
    \centering
    \includegraphics[width=0.7\linewidth]{00040.jpeg}
    \caption{Zepp and J-pole Antennas}
    \label{fig:zepp_jpole}
\end{figure}

\subsection{Pepping Up the Zepp}

The half-wave Zepp offers no advantage over a center-fed dipole in terms of gain or directivity. The only difference is the feed point being located at the end of the Zepp and in the middle of the dipole. As a result, the basic Zepp has a gain of 0 dBd.\textsuperscript{2}

Like arranging a pair of dipoles in an array to focus the radiated energy in a desired direction, creating gain, a pair of Zepps can be connected end to end as shown in Figure 1B. This creates the "two half-waves in phase" antenna that narrows the broadside pattern a bit and has gain of 1.9 dB over a single Zepp. (The net gain is less than 3 dB due to coupling between the separate elements.) The "double Zepp" is a basic collinear array with both elements lying along the same line.

The "missing" 1.1 dB of gain would be available if the radiation patterns of the two half-wave antennas could be added together independently. Coupling between the two antenna halves can be reduced by moving the elements farther apart, but feeding them would then become complicated. This problem was solved in a 1936 IRE paper by GH Brown who lengthened or extended each element from 180 degrees (half-wavelength) to 230 degrees, as seen in Figure 1C.

This antenna was introduced to amateurs in the June 1938 issue of \textbf{QST} by W2NB as the Extended Double Zepp or EDZ. Not only does the antenna "recover" the missing gain to a full 3 dBd but can also be easily matched to either open-wire or coaxial feed line through the use of transmission line techniques. A more recent \textbf{QST} article by W5JH gives the EDZ design information shown in Table 1 for HF bands from 40 through 10 meters.\textsuperscript{3}

\begin{figure}[H]
    \centering
    \includegraphics[width=0.7\linewidth]{00041.jpeg}
    \caption{EDZ Antenna Design}
    \label{fig:edz_design}
\end{figure}

\subsection{Feeding the Zepp}

Matching the EDZ to a feed line is an interesting story. With each element of the array being longer than one half-wavelength, the feed point impedance is quite reactive. For example, W5JH gives the feed point impedance of a 20 meter EDZ as 155.3 – \textbf{j}889.6 Ω for an SWR of about 15.5:1. If connected directly to 100 feet of 50 Ω RG-8X coax that has 0.9 dB of matched loss at 20 meters, \textbf{TLW} calculates that a 15.5:1 SWR would result in 9.8 dB of additional loss for a total of 10.7 dB in the feed line.\textsuperscript{4} Obviously, it is a good idea to lower the SWR in some way!

Luckily, the impedance transforming properties of transmission lines can be used to change the impedance. (See Experiments #59 – 61 on the Smith Chart.) In this particular case, using a short length of high-impedance open-wire line (also called a ladder or window line) transforms the high feed point impedance (the "Antenna Z" column in Table 1) to something quite close to 50 Ω (the "Feed Point Z" column in Table 1). Figure 2 shows how a 10.3 foot section of 600 Ω line transforms the antenna feed point impedance point to very nearly 50 + \textbf{j}0 Ω at the center of the chart.

\begin{figure}[H]
    \centering
    \includegraphics[width=0.7\linewidth]{00042.jpeg}
    \caption{Impedance Transformation with Open-wire Line}
    \label{fig:impedance_transformation}
\end{figure}

At the end of this short feed line section, you have two options — connect a 50 Ω feed line or extend the open-wire line by some multiple of half-wavelengths so that the 50 Ω point is reached again and attach the 50 Ω feed line (or transmitter) there. By attaching 50 Ω coax at a point where the high-impedance open-wire line presents a 50 Ω impedance the SWR will then remain low in the coax all the way to the transmitter.

If your antenna is up in the air, the end of the matching section of 600 Ω line will be dangling well off the ground and you might not want to lift that much coax. Furthermore, the solid conductors of most open-wire feed lines will break from flexing in the wind with this load attached. Since the impedances in a transmission line repeat every half-wavelength along the line, you can add feed line in half-wavelength sections and reach another 50 Ω point, hopefully near the ground or a mechanical support. 

In our case, for 600 Ω open-wire line with a velocity factor of 0.92, one half-wavelength at 14.175 MHz is 32 feet. By adding multiples of 32 feet to the overall length, you can bring the 50 Ω point to a location where it is more convenient to attach a coaxial feed line, perhaps using a balun, as was shown in Experiment #131.

450 Ω window line also works but not quite as well. I found that \textbf{TLW} indicated that a length of 11.6 feet transformed the impedance to about 27 Ω for a minimum SWR of nearly 2:1. This is a lot better than 15.5:1 and will be lower at the end of the coax due to its loss but it might be worth buying or making your own 600 Ω line if you want to use this antenna system design.

I decided to go further by using \textbf{EZNEC} antenna modeling software (\href{http://www.eznec.com}{www.eznec.com}), creating a design with a low SWR point on two bands, 20 and 15 meters while using 450 Ω window line. I started with the dimensions of the 20 meter EDZ in Table 1, and dug in. You can do this too — start with the W5JH dimensions then optimize for your needs.

By lengthening the antenna to 89 feet and using a transmission line length of 11.1 feet, an SWR of 1.6 was obtained at 14.05 MHz and 1.4 at 21.1 MHz. Bringing the low SWR point to ground level required more window line. Since I needed an integer number of half-wavelengths at both 14.05 and 21.1 MHz, I added 2 half-wavelengths at 14.05 MHz (63.8 feet according to \textbf{TLW}) for a total of 74.9 feet which is also 3 half-wavelengths at 21.1 MHz. A bit more optimizing gave a line length of 74.7 feet and an SWR of 1.5 and 1.3:1 on 20 and 15 meters, respectively. Including ground reflections, gain is around 8.7 dBi on both bands with no tuner required.

I've installed a trio of these antennas in a triangle so the three 20 meter patterns cover all the main DX azimuths. Being electrically long on 15 meters, the antenna generates "four-leaf clover" patterns so my next project is redesigning the antenna for a single main lobe on both bands. Luckily, W7SX tackled that question in a July 1999 \textbf{QEX} article so I will be putting \textbf{EZNEC} to work once again!\textsuperscript{5}

\subsection{References}

\begin{enumerate}
    \item All previous Hands-On Radio experiments are available to ARRL members at \href{http://www.arrl.org/hands-on-radio}{www.arrl.org/hands-on-radio}.
    \item dBd specifies gain with respect to a dipole, usually in free space. Add 2.15 dB to obtain dBi, gain with respect to an isotropic radiator. 
    \item J. Haigwood, W5JH, "The Extended Double Zepp Revisited," Sep 2006, \textbf{QST}, pp 35 – 36.
    \item TLW or Transmission Line Program for Windows, by Dean Straw, N6BV, is a transmission line calculator program included with the ARRL Antenna Book available from your ARRL dealer, or from the ARRL Store, ARRL order no. 6948. Telephone toll-free in the US 888-277-5289, or 860-594-0355; fax 860-594-0303; \href{http://www.arrl.org/shop/}{www.arrl.org/shop/}; \href{mailto:pubsales@arrl.org}{pubsales@arrl.org}.
    \item R. Zavrel, Jr., W7SX, "The Multiband Extended Double Zepp and Derivative Designs," July 1999 \textbf{QEX}, pp 34 – 39.
\end{enumerate}

\section{中文翻译}

实验#131(关于同轴到开路导线巴伦)中对扩展双Zepp(EDZ)天线的引用肯定引起了一些兴趣!\textsuperscript{1} Zepp是最古老的天线之一。由Hans Beggerow于1909年申请专利(德国专利225204,\href{http://www.aktuellum.com/circuits/antenna-patent}{www.aktuellum.com/circuits/antenna-patent}),该天线显示为从气球垂直悬挂(以齐柏林飞艇命名是后来的事),看起来就像一个倒置的J形天线,事实上,它就是!

Zepp通常被想象为水平的,如图1A所示,而J形天线为垂直的,但从电气上讲,它们本质上是相同的天线。两者都使用四分之一波长的传输线段将半波辐射元件末端的高阻抗转换为适合连接到馈线的低阻抗。

\begin{figure}[H]
    \centering
    \includegraphics[width=0.7\linewidth]{00040.jpeg}
    \caption{Zepp和J形天线}
    \label{fig:zepp_jpole_cn}
\end{figure}

\subsection{增强Zepp}

半波Zepp在增益或方向性方面并不比中心馈电偶极子有优势。唯一的区别是馈电点位于Zepp的末端和偶极子的中间。因此,基本Zepp的增益为0 dBd。\textsuperscript{2}

就像排列一对偶极子成阵列以将辐射能量聚焦在所需方向上,创建增益一样,一对Zepp可以像图1B所示那样端到端连接。这创建了"两个同相半波"天线,稍微收窄了宽边方向图,并且比单个Zepp有1.9 dB的增益。(由于分离元件之间的耦合,净增益小于3 dB。)"双Zepp"是一个基本的共线阵列,两个元件都沿同一条线排列。

如果两个半波天线的辐射方向图可以独立相加,那么"缺失"的1.1 dB增益将可用。通过将元件移得更远,可以减少两个天线半部之间的耦合,但这样馈电就会变得复杂。这个问题在1936年的IRE论文中由GH Brown解决,他将每个元件从180度(半波长)延长到230度,如图1C所示。

这种天线在1938年6月的\textbf{QST}杂志中由W2NB介绍给业余无线电爱好者,称为扩展双Zepp或EDZ。该天线不仅"恢复"了缺失的增益至完整的3 dBd,而且还可以通过使用传输线技术轻松匹配到开路导线或同轴馈线。W5JH最近的\textbf{QST}文章给出了表1中所示的EDZ设计信息,适用于40至10米的HF频段。\textsuperscript{3}

\begin{figure}[H]
    \centering
    \includegraphics[width=0.7\linewidth]{00041.jpeg}
    \caption{EDZ天线设计}
    \label{fig:edz_design_cn}
\end{figure}

\subsection{馈电Zepp}

将EDZ匹配到馈线是一个有趣的故事。由于阵列的每个元件都长于一个半波长,馈电点阻抗相当电抗。例如,W5JH给出20米EDZ的馈电点阻抗为155.3 – \textbf{j}889.6 Ω,SWR约为15.5:1。如果直接连接到100英尺的50 Ω RG-8X同轴电缆,该电缆在20米处有0.9 dB的匹配损耗,\textbf{TLW}计算得出15.5:1的SWR将导致9.8 dB的额外损耗,馈线总损耗为10.7 dB。\textsuperscript{4} 显然,以某种方式降低SWR是个好主意!

幸运的是,可以使用传输线的阻抗变换特性来改变阻抗。(参见关于史密斯圆图的实验#59-61。)在这种特殊情况下,使用短长度的高阻抗开路导线(也称为梯形或窗口线)将高馈电点阻抗(表1中的"Antenna Z"列)变换为非常接近50 Ω的值(表1中的"Feed Point Z"列)。图2显示了10.3英尺的600 Ω线如何将天线馈电点阻抗变换到圆图中心非常接近50 + \textbf{j}0 Ω的点。

\begin{figure}[H]
    \centering
    \includegraphics[width=0.7\linewidth]{00042.jpeg}
    \caption{使用开路导线的阻抗变换}
    \label{fig:impedance_transformation_cn}
\end{figure}

在这个短馈线段的末端,你有两个选择 —— 连接50 Ω馈线或通过添加半波长的倍数来延长开路导线,以便再次达到50 Ω点并在那里连接50 Ω馈线(或发射机)。通过在高阻抗开路导线呈现50 Ω阻抗的点连接50 Ω同轴电缆,SWR将在同轴电缆中保持低位,一直到发射机。

如果你的天线在空中,600 Ω线的匹配段末端将悬空在离地面很高的地方,你可能不想举起那么多同轴电缆。此外,大多数开路导线馈线的实心导体在这种负载下会因风中的弯曲而断裂。由于传输线中的阻抗沿线路每半波长重复一次,你可以添加半波长段的馈线并到达另一个50 Ω点,希望靠近地面或机械支撑。

在我们的例子中,对于速度因子为0.92的600 Ω开路导线,14.175 MHz时的一个半波长为32英尺。通过将32英尺的倍数添加到总长度,你可以将50 Ω点带到更方便连接同轴馈线的位置,也许使用巴伦,如实验#131所示。

450 Ω窗口线也有效,但效果不如600 Ω。我发现\textbf{TLW}指示11.6英尺的长度将阻抗变换到约27 Ω,最小SWR接近2:1。这比15.5:1好得多,并且由于其损耗,在同轴电缆末端会更低,但如果你想使用这种天线系统设计,购买或制作自己的600 Ω线可能是值得的。

我决定进一步使用\textbf{EZNEC}天线建模软件(\href{http://www.eznec.com}{www.eznec.com}),创建一个在20和15米两个频段上具有低SWR点的设计,同时使用450 Ω窗口线。我从表1中20米EDZ的尺寸开始,深入研究。你也可以这样做 —— 从W5JH尺寸开始,然后根据你的需求进行优化。

通过将天线延长到89英尺并使用11.1英尺的传输线长度,在14.05 MHz处获得了1.6的SWR,在21.1 MHz处获得了1.4的SWR。将低SWR点带到地面需要更多的窗口线。由于我需要在14.05和21.1 MHz处都有整数个半波长,我在14.05 MHz处添加了2个半波长(根据\textbf{TLW}为63.8英尺),总长度为74.9英尺,这也是21.1 MHz处的3个半波长。再进行一些优化,得到74.7英尺的线长,在20和15米处的SWR分别为1.5和1.3:1。包括地面反射,两个频段的增益约为8.7 dBi,无需调谐器。

我已经安装了三个这样的天线成三角形,以便三个20米方向图覆盖所有主要的DX方位角。由于在15米处电气长度较长,天线会产生"四叶草"方向图,因此我的下一个项目是重新设计天线,使其在两个频段上都有单个主瓣。幸运的是,W7SX在1999年7月的\textbf{QEX}文章中解决了这个问题,所以我将再次使用\textbf{EZNEC}!\textsuperscript{5}

\subsection{参考资料}

\begin{enumerate}
    \item 所有以前的"实践无线电"实验都可在ARRL成员的\href{http://www.arrl.org/hands-on-radio}{www.arrl.org/hands-on-radio}获取。
    \item dBd指定相对于偶极子的增益,通常在自由空间中。添加2.15 dB以获得dBi,即相对于各向同性辐射器的增益。
    \item J. Haigwood, W5JH,"The Extended Double Zepp Revisited",2006年9月,\textbf{QST},第35-36页。
    \item TLW或Transmission Line Program for Windows,由Dean Straw, N6BV开发,是一个传输线计算器程序,包含在ARRL Antenna Book中,可从你的ARRL经销商或ARRL商店获取,ARRL订单号6948。美国免费电话888-277-5289,或860-594-0355;传真860-594-0303;\href{http://www.arrl.org/shop/}{www.arrl.org/shop/};\href{mailto:pubsales@arrl.org}{pubsales@arrl.org}。
    \item R. Zavrel, Jr., W7SX,"The Multiband Extended Double Zepp and Derivative Designs",1999年7月\textbf{QEX},第34-39页。
\end{enumerate}

\chapter{实验 #136:端馈天线}

\section{英文原文}

An HF antenna currently enjoying some popularity, especially with backpackers such as those active in the Summits On the Air (\href{http://www.sota.org.uk}{www.sota.org.uk}) program, is the End-Fed Half-Wave (EFHW). One of the oldest antennas, it was originally known as the "Zepp" and is widely used today in its VHF/UHF disguise as the J-pole (see Experiment #133\textsuperscript{1}). Mechanically, it can be convenient to attach the feed line at one end, which is also a support point. It is also easy to toss one support rope over a high point and let the EFHW slope to a lower point where the feed line connection is made. Electrically, however, there is more to this antenna than meets the eye, and that can lead to some unexpected results.

\subsection{At the End}

The impedance of a half-wavelength piece of wire varies from a minimum of about 73 Ω when the feed point is at the middle to a much higher value at the end. Off-center-fed (OCF) antennas take advantage of this by locating the feed point somewhere at which a medium-sized impedance occurs on several bands. A fixed-ratio impedance transformer then creates a coax-friendly impedance on several bands.

At first glance, the impedance at the end of the wire should be infinite because the current has to be zero. In the real world, however, there is a fair amount of capacitance between the antenna and anything close to it that conducts electricity. This lowers impedance, especially at the end of the antenna. For example, at 14 MHz, 10 pF of capacitance is approximately 1.3 kΩ of reactance. It doesn't take many pF of capacitance in parallel with the impedance at the end of a wire to lower the resulting impedance dramatically.

Yes, but capacitance to what? Conductive material within a quarter to a half-wavelength of the antenna. That includes capacitance to ground which in a center-fed half-wave is responsible for some current flow at the end of the antenna and partly responsible for making the wire seem longer electrically than it is physically, helping to create the familiar formula for dipole length: \( l = 468 / f \) (see Experiment #92).

\subsection{The Whole Enchilada}

What else is nearby that conducts electricity? The feed line, of course! In the absence of stern measures to prevent it, there will be a hefty amount of common-mode current flowing on the feed line, whether coaxial or parallel-conductor. But wait, isn't parallel-conductor feed line balanced? Yes, but only to the differential-mode currents (see Experiment #91). Along with the equal-and-opposite currents carrying power in the line, each of the conductors can pick up common-mode current just like any other wire and re-radiate signals just like any other antenna. For coax, the center conductor and the inside surface of the shield may be isolated from the antenna's radiated field but the outer surface is completely exposed and it, too, picks up common-mode current and re-radiates a signal. The combination of the antenna wire and the feed line's common-mode current path form the entire \textbf{antenna system}.

Based on the end-fed antenna model in W7EL's \textbf{EZNEC} User Manual, Figure 1 illustrates why considering the entire system is important.\textsuperscript{2,3} You can see that antenna current varies from zero (this is a simulation) at the un-fed end to a small but non-zero value at the feed point end. The approximately half-wavelength feed line, in this case parallel-conductor, is terminated at the bottom by the power source and has one conductor attached to the antenna at the top. The remaining conductor is left open.

\begin{figure}[H]
    \centering
    \includegraphics[width=0.7\linewidth]{00043.jpeg}
    \caption{End-Fed Antenna System}
    \label{fig:end_fed_system}
\end{figure}

There are two currents in the feed line (phase is not shown in this drawing, only magnitude) that are \textbf{almost} the same magnitude. The currents differ by the amount of common-mode current, \( I_{CM} \), that increases \( I_{Line1} \) and reduces \( I_{Line2} \) so that \( I_{CM} = I_{Line1} - I_{Line2} \). \( I_{CM} \) is also the value of current at the end of the antenna wire. From the standpoint of a radiated signal, the EFHW antenna is not really "end-fed" at all!

Radiating currents in the EFHW antenna system consist of the current on the antenna wire \textbf{and} the common-mode current on the feed line. The EFHW is in reality somewhat off-center-fed. As you might imagine, the resulting radiation pattern is nearly omnidirectional and not very much like a classic dipole. That may be better for a portable station than an antenna with nulls along its axis. Nevertheless, the EFHW user should be aware of where the antenna system current is flowing!

Another thing the EFHW user should be aware of is that when an antenna system is unbalanced, \textbf{everything} that is connected to it and is not isolated by chokes or other methods (such as detuning) should be assumed to be part of the antenna system. A typical EFHW system includes not only the antenna and feed line but the ground connection, all of the radio equipment, and the operator when touching anything.

When running QRP, the resulting common-mode RF current and voltage on the equipment may not be very noticeable. But, at and above 100 W, RF voltages high enough to cause RF burns can be present at different points in the system — such as on the microphone or key! When this happens, you can sometimes "move" the hot spots around in the system by attaching quarter-wavelength wires to the high-voltage point. Because the high-voltage point moves to the open end of the wire, secure it where it won't be accidentally touched while in use.

\subsection{Experiencing the System}

You can demonstrate these effects with an SWR analyzer covering 2 meters and a simple experimental test antenna. Cut 38 inches of stiff wire or rod into two equal pieces (coat hanger wire will do nicely), creating a dipole antenna for 2 meters. Include about ¾-inch on each wire to form loops or use terminals as in Figure 2. To support the antenna and hold it steady for testing, I used a piece of ½-inch PVC pipe and a T fitting as shown in Figure 2. 8-32 screws through the fitting hold the antenna, as shown in Figure 2.

\begin{figure}[H]
    \centering
    \includegraphics[width=0.7\linewidth]{00044.jpeg}
    \caption{Test Antenna Setup}
    \label{fig:test_antenna_setup}
\end{figure}

Next, create an RG-8/X or RG-58 coaxial cable with an RF connector on one end and alligator clips on the other. (This is a handy antenna test accessory.) Any length from 6 to 10 feet is fine. Attach the SWR analyzer to the antenna with the coax and tape the coax along the vertical support for a distance of 4 or 5 feet. This stabilizes the antenna system and keeps it in the same configuration as you experiment — very important for antenna testing.

Locate an open spot where you can secure the antenna support to hold the antenna horizontally a couple of feet over your head. There should be a wavelength or so (6 feet) of clearance between the antenna and any conducting wires or metal surfaces.

Use the SWR analyzer to find the frequency at which the antenna is closest to resonance — mine was about 145.15 MHz. (You may not be able to find the X = 0 resonance due to the setup or analyzer performance.) Record both the resistive (R) and reactive (X) values, whichever are available from the analyzer. In my case, at 145.15 MHz the feed point impedance was measured to be 43 ± \( j \)20 Ω through 8 feet of RG-8/X cable. (Most analyzers don't show the sign of the reactance and that's not important for this experiment.)

Touch the fingers of one hand to the outside of the feed line and watch the impedance measurement as you move your hand up and down the feed line within a couple of feet of the feed point. You won't see a lot of variation (a few ohms for R and X) because the low impedance of the antenna feed point masks the effect of the coax shield's outer surface.

To create a high feed point impedance, tune the analyzer 20 or 30 MHz lower so that the feed point R is 300 to 400 Ω. Perform the same experiment of touching the coax jacket while observing the effect on feed point impedance. At 115 MHz, R varied from 290 – 390 Ω and X from 0 to 115 Ω. This wide variation showed the antenna system included the common-mode current path and any impedance added to it, such as fingers!

If you have some clamp-on ferrite cores, snap them on to the feed line just below the feed point, as shown in Figure 3. The common #43 mix used for suppressing VHF/UHF EMI is just right although other mixes will have some effect, as well. As you add each core, repeat the touch-and-observe experiment. By adding five cores, as shown in the photo, variation in impedance for my antenna was reduced to ±10 Ω of resistance and ±20 Ω of reactance.

\begin{figure}[H]
    \centering
    \includegraphics[width=0.7\linewidth]{00045.jpeg}
    \caption{Ferrite Cores on Feed Line}
    \label{fig:ferrite_cores}
\end{figure}

The wide variations illustrate the sensitivity of a high impedance feed point to the presence of other conductors connected to and in the near field (within a few wavelengths) of the antenna. You can also see the effect of adding impedance to the feed line's common-mode current path with ferrite cores. By isolating the shield's outer surface with ferrite cores, the effect of common-mode current paths on the system can be reduced.

Be careful, however! Your EFHW system may depend on that common-mode current to reduce the feed point impedance of the antenna system. Without it, the EFHW feed point impedance will generally be much higher and difficult to match. It may be better to move any feed line choke closer to the transmitter and let the feed line between the choke and antenna radiate.

For another look at the EFHW through the eyes of an antenna designer, read AE6TY's \textbf{QRP Quarterly} article, "Refining an End Fed Antenna."\textsuperscript{4} By understanding the EFHW as an antenna system, you'll start to look at your other antennas as systems, too. This will help you plan, install, understand, and use your growing antenna farm more effectively, at home or in the field.

\subsection{References}

\begin{enumerate}
    \item All previous Hands-On Radio experiments are available to ARRL members at \href{http://www.arrl.org/hands-on-radio}{www.arrl.org/hands-on-radio}.
    \item \textbf{EZNEC} User Manual, version 5.0, \href{http://eznec.com/misc/EZNEC_Printable_Manual/5.0/EZW50_User_Manual.pdf}{eznec.com/misc/EZNEC_Printable_Manual/5.0/EZW50_User_Manual.pdf}.
    \item Lewallen, Roy W7EL, "Baluns, What They Do and How They Do It," \href{http://www.eznec.com/.Amateur/Articles/Baluns.pdf‎}{www.eznec.com/.Amateur/Articles/Baluns.pdf‎}
    \item \href{http://www.ae6ty.com/Papers_files/Refining%20an%20End%20Fed%20Vertical%20Dipole.pdf}{www.ae6ty.com/Papers_files/Refining%20an%20End%20Fed%20Vertical%20Dipole.pdf}
\end{enumerate}

\section{中文翻译}

一种目前颇受欢迎的HF天线,尤其是在背包客中,如活跃在Summits On the Air(\href{http://www.sota.org.uk}{www.sota.org.uk})项目中的人,是端馈半波(EFHW)天线。它是最古老的天线之一,最初被称为"Zepp",如今在VHF/UHF频段以J形天线的形式广泛使用(见实验#133\textsuperscript{1})。从机械角度来看,将馈线连接在一端(也是支撑点)可能很方便。也很容易将一根支撑绳扔过高处,让EFHW倾斜到较低的点,在那里进行馈线连接。然而,从电气角度来看,这种天线比表面看起来要复杂得多,这可能会导致一些意想不到的结果。

\subsection{在末端}

半波长导线的阻抗从馈电点在中间时的最小值约73 Ω变化到末端的更高值。偏置馈电(OCF)天线利用这一点,将馈电点定位在多个频段上出现中等阻抗的位置。然后,固定比率的阻抗变换器在多个频段上创建适合同轴电缆的阻抗。

乍一看,导线末端的阻抗应该是无限大的,因为电流必须为零。然而,在现实世界中,天线与其附近任何导电物体之间存在相当大的电容。这会降低阻抗,尤其是在天线末端。例如,在14 MHz时,10 pF的电容约为1.3 kΩ的电抗。与导线末端阻抗并联的电容不需要很多pF就能显著降低最终阻抗。

是的,但电容是相对于什么的?天线四分之一到半波长范围内的导电材料。这包括对地电容,在中心馈电半波天线中,对地电容是天线末端有一些电流流动的原因之一,也是导致导线在电气上看起来比物理上更长的部分原因,这有助于创建熟悉的偶极子长度公式:\( l = 468 / f \)(见实验#92)。

\subsection{完整系统}

附近还有什么导电的东西?当然是馈线!在没有采取严厉措施防止的情况下,馈线上会有大量的共模电流流动,无论是同轴电缆还是平行导体。但是,平行导体馈线不是平衡的吗?是的,但仅对差模电流(见实验#91)。除了在线路中携带功率的大小相等、方向相反的电流外,每个导体都可以像任何其他导线一样拾取共模电流,并像任何其他天线一样重新辐射信号。对于同轴电缆,中心导体和屏蔽层的内表面可能与天线的辐射场隔离,但外表面完全暴露,它也会拾取共模电流并重新辐射信号。天线导线和馈线的共模电流路径的组合形成了整个\textbf{天线系统}。

基于W7EL的\textbf{EZNEC}用户手册中的端馈天线模型,图1说明了为什么考虑整个系统很重要。\textsuperscript{2,3}你可以看到,天线电流从未馈电端的零(这是模拟)变化到馈电点端的小但非零值。大约半波长的馈线(在这种情况下是平行导体)在底部由电源端接,其中一根导体在顶部连接到天线。另一根导体保持开路。

\begin{figure}[H]
    \centering
    \includegraphics[width=0.7\linewidth]{00043.jpeg}
    \caption{端馈天线系统}
    \label{fig:end_fed_system_cn}
\end{figure}

馈线中有两种电流(此图中未显示相位,仅显示幅度),它们的幅度\textbf{几乎}相同。电流的差异在于共模电流的量,\( I_{CM} \),它增加\( I_{Line1} \)并减少\( I_{Line2} \),因此\( I_{CM} = I_{Line1} - I_{Line2} \)。\( I_{CM} \)也是天线导线末端的电流值。从辐射信号的角度来看,EFHW天线根本不是真正的"端馈"!

EFHW天线系统中的辐射电流由天线导线上的电流\textbf{和}馈线上的共模电流组成。EFHW实际上是某种偏置馈电。正如你可能想象的那样,产生的辐射方向图几乎是全向的,与经典偶极子不太相似。对于便携式电台来说,这可能比沿其轴线有零点的天线更好。尽管如此,EFHW用户应该知道天线系统电流在哪里流动!

EFHW用户还应该意识到的另一件事是,当天线系统不平衡时,\textbf{所有}连接到它并且未通过扼流圈或其他方法(如失谐)隔离的东西都应被视为天线系统的一部分。典型的EFHW系统不仅包括天线和馈线,还包括接地连接、所有无线电设备,以及当触摸任何东西时的操作员。

当运行QRP时,设备上产生的共模RF电流和电压可能不太明显。但是,在100 W及以上时,系统不同点可能存在足以导致RF灼伤的RF电压——例如在麦克风或电键上!当这种情况发生时,你有时可以通过将四分之一波长导线连接到高压点来"移动"系统中的热点。由于高压点会移动到导线的开路端,请将其固定在使用时不会被意外触摸的地方。

\subsection{体验系统}

你可以使用覆盖2米波段的SWR分析仪和简单的实验测试天线来演示这些效果。将38英寸的硬 wire或杆切成两段相等的部分(衣架线效果很好),创建一个2米波段的偶极子天线。在每根导线上留出约¾英寸以形成环路,或使用图2中的端子。为了支撑天线并在测试期间保持其稳定,我使用了一段½英寸的PVC管和一个T形接头,如图2所示。通过接头的8-32螺钉固定天线,如图2所示。

\begin{figure}[H]
    \centering
    \includegraphics[width=0.7\linewidth]{00044.jpeg}
    \caption{测试天线设置}
    \label{fig:test_antenna_setup_cn}
\end{figure}

接下来,创建一根RG-8/X或RG-58同轴电缆,一端带有RF连接器,另一端带有鳄鱼夹。(这是一个方便的天线测试附件。)6到10英尺的任何长度都可以。将SWR分析仪通过同轴电缆连接到天线,并将同轴电缆沿垂直支架胶带固定4或5英尺的距离。这可以稳定天线系统,并在你进行实验时保持相同的配置——这对天线测试非常重要。

找到一个开阔的地方,你可以固定天线支架,将天线水平保持在你头顶上方几英尺处。天线与任何导电线或金属表面之间应有一个波长左右(6英尺)的间隙。

使用SWR分析仪找到天线最接近谐振的频率——我的大约是145.15 MHz。(由于设置或分析仪性能,你可能无法找到X = 0的谐振。)记录电阻(R)和电抗(X)值,无论分析仪提供哪些值。在我的例子中,在145.15 MHz时,通过8英尺的RG-8/X电缆测量的馈电点阻抗为43 ± \( j \)20 Ω。(大多数分析仪不显示电抗的符号,这对本实验不重要。)

用一只手的手指触摸馈线的外部,观察当你在馈电点附近几英尺内上下移动手时的阻抗测量值。你不会看到太多变化(R和X的变化只有几欧姆),因为天线馈电点的低阻抗掩盖了同轴电缆屏蔽层外表面的影响。

要创建高馈电点阻抗,将分析仪调谐到低20或30 MHz,使馈电点R为300至400 Ω。执行相同的实验,触摸同轴电缆护套,同时观察对馈电点阻抗的影响。在115 MHz时,R从290-390 Ω变化,X从0到115 Ω。这种广泛的变化表明,天线系统包括共模电流路径和添加到其上的任何阻抗,例如手指!

如果你有一些夹式铁氧体磁芯,将它们夹在馈电点正下方的馈线上,如图3所示。用于抑制VHF/UHF EMI的常见#43混合材料效果很好,尽管其他混合材料也会有一些效果。每次添加一个磁芯时,重复触摸和观察实验。通过添加五个磁芯,如图所示,我的天线的阻抗变化减少到电阻±10 Ω和电抗±20 Ω。

\begin{figure}[H]
    \centering
    \includegraphics[width=0.7\linewidth]{00045.jpeg}
    \caption{馈线上的铁氧体磁芯}
    \label{fig:ferrite_cores_cn}
\end{figure}

这种广泛的变化说明了高阻抗馈电点对连接到天线并在其近场(几个波长内)的其他导体的存在的敏感性。你还可以看到通过铁氧体磁芯向馈线的共模电流路径添加阻抗的效果。通过使用铁氧体磁芯隔离屏蔽层的外表面,可以减少共模电流路径对系统的影响。

然而,要小心!你的EFHW系统可能依赖于该共模电流来降低天线系统的馈电点阻抗。没有它,EFHW馈电点阻抗通常会高得多,难以匹配。将任何馈线扼流圈移近发射机并让扼流圈和天线之间的馈线辐射可能会更好。

要从天线设计师的角度另眼看EFHW,请阅读AE6TY的\textbf{QRP Quarterly}文章,"Refining an End Fed Antenna。"\textsuperscript{4}通过将EFHW理解为一个天线系统,你也会开始将你的其他天线视为系统。这将帮助你更有效地规划、安装、理解和使用你不断增长的天线群,无论是在家中还是在野外。

\subsection{参考资料}

\begin{enumerate}
    \item 所有以前的"实践无线电"实验都可在ARRL成员的\href{http://www.arrl.org/hands-on-radio}{www.arrl.org/hands-on-radio}获取。
    \item \textbf{EZNEC}用户手册,版本5.0,\href{http://eznec.com/misc/EZNEC_Printable_Manual/5.0/EZW50_User_Manual.pdf}{eznec.com/misc/EZNEC_Printable_Manual/5.0/EZW50_User_Manual.pdf}。
    \item Lewallen, Roy W7EL,"Baluns, What They Do and How They Do It," \href{http://www.eznec.com/.Amateur/Articles/Baluns.pdf‎}{www.eznec.com/.Amateur/Articles/Baluns.pdf‎}
    \item \href{http://www.ae6ty.com/Papers_files/Refining%20an%20End%20Fed%20Vertical%20Dipole.pdf}{www.ae6ty.com/Papers_files/Refining%20an%20End%20Fed%20Vertical%20Dipole.pdf}
\end{enumerate}

\chapter{实验 #150:对数周期天线基础}
\section{英文原文}
The standard design for rotatable directional ham antennas has been the Yagi-Uda array, known today just as "the Yagi," nearly since its introduction in the late 1920s. Chester Buchanan, W3DZZ, added parallel LC circuits, aka "traps," to dipoles and Yagis in 1955, putting rotatable directivity on 14 Mc and up within reach of the average station builder.

Then came 30, 17, and 12 meters. Hams wanted "pointable gain" on these bands, and that changed their antenna requirements dramatically. Suddenly, the \emph{log-periodic} became an all-bands-on-one-boom solution.

\subsection{Frequency Independence}
One of the common claims for log-periodic antennas, most commonly a Log-Periodic Dipole Array (LPDA), is that they are frequency-independent. Why so? The fundamental idea (somewhat oversimplified) is that by defining an antenna entirely in terms of angles and ratios, it will behave consistently when scaled to any frequency. This is related to the notions of \emph{self-similarity} and \emph{scale-invariance}. If the antenna's structure remains consistent when scaled by some factor, the antenna's behavior with frequency turns out to be periodic (repeating) according to the logarithm of that factor. Thus the name, log-periodic.

In its most common amateur form, the LPDA consists of a set of linear \(\lambda\)/2 dipoles covering the lowest to highest frequency of the antenna's range, which is usually one octave at HF from 14 to 30 MHz. (Three octave ranges of 3 – 30 MHz are real monsters!) Tennadyne (\url{http://www.tennadyne.com}) makes HF logs and a five-octave 50 – 1300 MHz model.

So, where are the ratios? The three primary parameters of LPDA design are:
\begin{itemize}
    \item Apex angle, \(\alpha\) (alpha), which controls the shape of the triangular LPDA outline
    \item Scale factor, \(\tau\) (tau), which controls the ratio between spacing and length of adjacent elements
    \item Relative spacing, \(\sigma\) (sigma), which controls how many elements fill the triangular outline
\end{itemize}

These three parameters completely define the shape and internal structure of an LPDA, whether it is intended for use at HF, VHF, microwave, or light. An LPDA designed from the same three parameter values will look the same at any scale.

The scale factor, \(\tau\), captures the relationship of L, R, and D for the antenna elements as illustrated in Figure 1:

\begin{figure}[htbp]
    \centering
    \includegraphics{../epub_extracted/images/00046.jpeg}
    \caption{LPDA Element Parameters}
    \label{fig:lpda-parameters}
\end{figure}

\begin{figure}[htbp]
    \centering
    \includegraphics{../epub_extracted/images/00047.jpeg}
    \caption{LPDA Parameter Relationships}
    \label{fig:lpda-relationships}
\end{figure}

As \(\tau\) increases, the elements get farther apart and the lengths of adjacent elements differ more.

The three parameters are related by the following equation:

\begin{figure}[htbp]
    \centering
    \includegraphics{../epub_extracted/images/00048.jpeg}
    \caption{LPDA Parameter Equation}
    \label{fig:lpda-equation}
\end{figure}

By picking values for two of the parameters, the third can be determined — a lot like Ohm's Law.

\subsection{Building a Log}
First, we define the frequency range to be covered to determine longest and shortest dipole lengths (typically resonant a few percent outside the desired range). Then, we pick a boom length. This sets the overall size of the triangle and determines \(\tau\). Next, we have to specify how much "ripple" we can tolerate in the antenna's behavior over that range. That determines how many elements will fill the triangle. We do this by choosing a value for \(\sigma\) or by specifying the number of elements. Figure 2 shows some examples of antennas which all have the same gain but cover different frequency ranges, or that have the same frequency ranges and different element spacings.

\begin{figure}[htbp]
    \centering
    \includegraphics{../epub_extracted/images/00049.jpeg}
    \caption{LPDA Examples}
    \label{fig:lpda-examples}
\end{figure}

As a practical matter, we use charts or software to design the antenna. Figure 3 shows the most common chart used for log-periodic design. On the horizontal axis for \(\tau\), the triangle gets "pointier" toward the right. On the vertical axis for \(\sigma\), toward the top there are more and more elements packed into the triangle. The slanted straight lines show different values for the apex angle, \(\alpha\).

\begin{figure}[htbp]
    \centering
    \includegraphics{../epub_extracted/images/00050.jpeg}
    \caption{LPDA Design Chart}
    \label{fig:lpda-design-chart}
\end{figure}

Overlaid on top of the parameter scales are the curved grey lines. As on a topographic map, these represent combinations of the three parameters which result in the same values of gain, labeled for each line. The dotted line running across all of the curves shows the value of \(\sigma\) required to obtain the maximum gain for a particular value of \(\tau\). (This chart is based on a particular length-to-diameter ratio for the dipoles and characteristic impedance of the antenna.) Using the optimum value usually results in an antenna too large to be practical but typical designs (represented by black dots at the bottom of Figure 3) have acceptable performance.

The program \emph{LPCAD} by Roger Cox, WBØDGF, is a more practical method of designing your own antenna (\url{http://wb0dgf.com/LPCAD.htm}). Using this software, after establishing the antenna's frequency range, you can enter values for \(\tau\) and \(\sigma\) directly to see the results. (This is an easy way to find out why the optimum value of \(\sigma\) is impractical.) Or you can enter boom length and number of elements, which is a much more practical way of designing a log-periodic!

\subsection{Feeding the Log}
The LPDA is fed from the forward end of the array at the triangle's apex. For frequencies toward the middle or low end of the antenna's range, the "front" dipoles are very short compared to \(\lambda\)/2, and so will have a high impedance. A traveling wave develops as the signal moves along the transmission line toward the longer dipoles until it encounters the dipoles close to resonance, which are excited by the wave and radiate its energy. This \emph{active region} moves back and forth with the operating frequency.

Phase reversal is key to the antenna's performance. If each successive dipole is fed out of phase with the adjacent dipoles, the array develops a \emph{back-fire} pattern to the front of the array to the left in Figure 1. If phase is not reversed — a common error made by first-time log-periodic assemblers, such as myself — radiation is in the \emph{end-fire} direction to the back of the array, resulting in poor SWR and gain.

Feeding successive elements out of phase can be accomplished by constructing the boom from a pair of conductive tubes insulated from each other forming a parallel conductor line. The feed line runs through one tube and is connected to the parallel conductors at the front of the array. Another method, more common in smaller TV antennas and in really large military or commercial LPDAs, is to insulate all of the elements from a single supporting boom and use crisscrossing straps to connect all of the elements.

\subsection{Filling Your Log Book}
This column just scratches the surface of log-periodic and frequency-independent antenna design. There are dozens of designs in common use from MF through mm-wave as described in \emph{The ARRL Antenna Book} and numerous other references. Try downloading a log-periodic manual online and entering the element lengths and spacings to see what value you obtain for \(\sigma\), \(\tau\), and \(\alpha\), then find where that design falls on the chart. You'll never look at a TV antenna the same way again!

\begin{enumerate}
    \item The first amateur to use a Yagi was 1CCZ in 1928. His neighbors thought it was either a Ferris Wheel, a dirigible, or a ship ("Strays," \emph{QST}, Oct 1928). The Yagi was described a few months earlier: H. Yagi, "Beam Transmission of Ultra Short Waves," Proc. IRE, Jun 1928, Vol 26, pp 715 – 741.
    \item Buchanan, C., W3DZZ, "The Multimatch Antenna System," \emph{QST,} March 1955, p 22.
    \item All Hands-On Radio experiments are available to ARRL members at \url{http://www.arrl.org/hands-on-radio}.
    \item \emph{The ARRL Antenna Book}, 22nd edition, Chapter 7, ARRL.
    \item Johnson & Jasik, \emph{Antenna Engineering Handbook}, 2nd edition, Chapter 14, McGraw-Hill.
\end{enumerate}

\chapter{实验 #165:传播预测}
\section{英文原文}
While we all knew Solar Cycle 24 was headed to a solar minimum sooner or later, over the past few months there has been a lot of hope that maybe there would be a few more months of sunspots. Well, the reality is that there have been quite a few spotless days lately, and we are assured of more to come. Luckily, there are great tools and more data than ever before to help you make your on-the-air minutes count. This month, we're going to learn about an excellent and free resource, \emph{VOACAP Online}.

\subsection{Propagation Prediction — Then and Now}
Before the Internet, the only information available to most hams were the hourly announcements from WWV and WWVH, of the solar flux and K indices along with the condition of the geomagnetic field. A simple chart gave you some idea of whether conditions were better or worse than "normal," whatever "normal" meant. Numeric tables of monthly estimates generated by programs like \emph{IONCAP} (Ionospheric Communications Analysis and Prediction Program) and the early coverage maps by pioneering PC programs like \emph{W6ELProp} have been replaced with sophisticated graphic presentations a far more nuanced view.

The network of NCDXF beacon stations (\url{http://www.ncdxf.org/pages/beacons.html}) is as useful as ever, but is supplemented by the worldwide Reverse Beacon Network (\url{http://www.reversebeacon.net}) of automated receivers, including signal reports of beacon signals. Reception reports ("spots") once distributed by packet radio bulletin-board systems using \emph{PacketCluster} software are now available to all via Telnet connections and on websites like \url{http://dxmaps.com}, \url{http://dxsummit.fi}, and \url{http://dxheat.com}. Get on the air, call CQ on CW or RTTY, and within a few seconds, your presence will be made known worldwide!

With all this information around, who needs predictions? Unless you have 24 hours a day to spend watching a computer screen (or watching for DX alerts coming in by text message) you need to decide when you'll be at the rig. With sunspots on the decline, band openings above 10 MHz will decline too. While the lower-frequency bands improve for long-haul contacts with declining solar flux, you still need to plan for the short openings between daytime absorption and the MUF (maximum usable frequency) falling at night. Good planning makes for happy hams!

\subsection{VOACAP Online}
\emph{VOACAP} (Voice of America Coverage Analysis Program) was developed to predict broadcast coverage using detailed ionospheric models and the continually improving understanding of interactions between the Sun and Earth's geomagnetic environment.

While you can download \emph{VOACAP} and run it on your PC, the software has been made available with an online interface (\url{http://www.voacap.com}) by Jari Perkiömäki, OH6BG/OG6G. It manages a lot of the setup and configuration so that the program is easily usable by beginners. (Once you are familiar with the online version, the PC-based version will allow you additional flexibility and customizing to suit your station more exactly. A user's manual is available on the \emph{VOACAP Online} home page.)

\subsection{Coverage Area Maps}
Let's start with a map of locations for which a particular band is expected to support contacts. Browse to \url{http://www.voacap.com/coverage.html} or click COVERAGE AREA MAP on the \emph{VOACAP} home page. You'll see a screen like that shown in Figure 1. For first-time visitors, it will be centered on the "East Pole" (the intersection of 0° E and 0° N). Start by selecting a QTH in the Transmitter Site panel: I selected Jefferson City, Missouri, the closest menu choice to my home. I then selected the transmitter parameters — antenna (dipole at 33 feet), power (100 W), and mode (CW) — and a band (14.1 MHz). The website automatically loads the SSN (smoothed sunspot number) from a solar observatory (35 on the day the map was generated) and uses the date and time of the PC's clock (2100 UTC on 31 July 2016). The receiving antenna is assumed to be a dipole at 33 feet, as well. Accept those defaults for now.

\begin{figure}[htbp]
    \centering
    \includegraphics{../epub_extracted/images/00051.jpeg}
    \caption{VOACAP Coverage Area Map Interface}
    \label{fig:voacap-interface}
\end{figure}

Click the RUN THE PREDICTION! button to see where 20 meter CW might be open between these two types of stations, and after about 5 seconds of computation you will see a map like that in Figure 2. (Note the skip zone [dark area] around the transmitting location indicated by the bright red dot.) The brighter the color, the more likely it is that you'll be able to make contact with the signal qualities built in to the online software. (You can configure "quality of service" values in the PC-based version.) Even on a summer afternoon with solar activity the minimum and using low dipoles, there should be opportunities to make contacts galore! Why not get on and call CQ or tune around?

\begin{figure}[htbp]
    \centering
    \includegraphics{../epub_extracted/images/00052.jpeg}
    \caption{VOACAP Coverage Area Map Results}
    \label{fig:voacap-results}
\end{figure}

Here are your first assignments: Vary the transmitting parameters to see what effect they have on your coverage map. (All of the antennas are assumed to be oriented in the preferred direction when calculating signal strength.)
\begin{itemize}
    \item What happens if you switch from CW to SSB? (Spreading your signal over a 3 kHz bandwidth instead of 300 Hz certainly reduces your available coverage!)
    \item What happens if you switch from the low dipole to a ¼-wave vertical with a good ground system? (This could be a significant improvement.)
    \item Experiment with raising the dipole in 10-meter increments. At greater heights, why do secondary "holes" in coverage appear? (The elevation pattern of the dipole breaks up into lobes and nulls between them.)
    \item Try small Yagis at different heights, too. Extra credit for experimenting with long path (selectable in the GREAT-CIRCLE PATH menu) to see how much power and antenna it might take to work, say, Japan the long way around!
    \item Now change the time through the day from around sunrise at your location to night and watch the effects of the Earth's rotation. (The Sun's position is shown as a yellow dot on the map.)
\end{itemize}

\subsection{Point-to-Point Predictions}
Let's say you do want to see when the bands might be open to a particular location. Maybe there is a DXpedition on, or you might have a friend with whom you make regular contacts. \emph{VOACAP Online} will "run the numbers" between two points as well. Browse to \url{http://www.voacap.com/prediction.html} and enter your transmitter information as before. This time, select a receiving location such as PY1 — Rio de Janeiro, and you'll see the hour-by-hour band availability chart update to that in Figure 3.

\begin{figure}[htbp]
    \centering
    \includegraphics{../epub_extracted/images/00053.jpeg}
    \caption{VOACAP Point-to-Point Prediction}
    \label{fig:voacap-point-to-point}
\end{figure}

The bands from 80 through 10 meters each have their own ring showing the expected communication probability for each hour throughout the entire 24-hour day. At least one of the HF bands is open between WØ and PY1 at all times!

Now click the button at the lower right, labeled RUN PREDICTION! to see a bands-by-the-hour chart of propgation reliability.

Planning for propagation is a lot of fun — not as much fun as you'll have getting on the air and making QSOs you've investigated online, but nothing whets one's DX appetite like a chart saying, "Come and get it!" Once you're up to speed on \emph{VOACAP Online}, give the PC-based version a try and you'll find it to be a very valuable tool in your hands-on repertoire.

\begin{enumerate}
    \item See \url{http://tf.nist.gov/stations/iform.html} for the schedule of information transmitted by WWV and WWVH.
\end{enumerate}

\chapter{实验 #176:偶极子馈电点}
\section{英文原文}
The ordinary half-wavelength dipole is one of the best (and oldest) antennas — hard to beat for good performance, ease of construction, and bang for the builder's buck.

\subsection{Dipole Fundamentals}
When the dipole is 1⁄2-wavelength long it acts just like a vibrating string at its \emph{fundamental} frequency. Find a string, stretch it tight between a pair of sturdy supports a foot or two apart, and pluck the string with your finger. Look closely at it from the side. You will see the string's maximum displacement occurs in the middle, gradually reducing to zero at each end. Imagine that displacement represents electrical current, and you have a mental picture of a dipole at its half-wave resonant frequency. Current flows in one direction (just as the string is displaced in one direction) for half of a cycle, goes to zero, then reverses and builds to a maximum in the other direction for the second half cycle.

Figure 1 shows the dipole's current and voltage along the antenna. When current is flowing from left to right, voltage on the left side is higher than on the right, and vice versa. This is the root of the dipole's name; di– (meaning "two") and –pole (meaning "electrical polarity"). The two "poles" of the dipole (the left and right halves) have opposite electrical polarities so that current is always flowing from one to the other. The polarity reverses with every half cycle, so the direction of current flow also reverses.

\begin{figure}[htbp]
    \centering
    \includegraphics{../epub_extracted/images/00054.jpeg}
    \caption{Dipole Current and Voltage Distribution}
    \label{fig:dipole-current-voltage}
\end{figure}

\subsection{The Feed Point}
Note that I haven't mentioned the feed point yet — that's because it doesn't affect the basics of how the dipole "works." Just as you can pluck an instrument string anywhere along its length, you can feed a dipole anywhere as well. Plucking the string in various places may affect how loud it sounds as a result, but it doesn't change the fundamental frequency. Different points along the string are more effective for making the string sound louder or softer — but still at the same frequency.

For a dipole, the feed point is simply where you attach the feed line. It can be anywhere along the dipole — centered, off-center, or at the end. The feed point is just the place where you "pluck the string" and apply energy to the antenna. The dipole itself doesn't care. Changing the feed point of a resonant, half-wave dipole doesn't change the distribution of current and voltage along the dipole or its radiation pattern or its resonant frequency. (This assumes no current can flow back down the feed line as common-mode current.)

Impedance (Z) is the ratio of voltage (V) to current (I), whether at an antenna feed point or in a circuit. If the feed point is at the center, voltage is minimum and current is maximum, so we expect the impedance to be low, as well. And it is — the impedance of a resonant half-wave dipole at the center is approximately 72 Ω in free space.

If you move the feed point away from the center of the dipole, however, the situation changes. Voltage begins to increase and current to decrease, so the farther away from the center of the dipole, the higher impedance gets. If the feed line is attached somewhere off-center, the higher feed point impedance means that the feed line should have a higher characteristic impedance, if energy is to be efficiently transferred to the dipole. Toward the end of the dipole, impedance gets very high, indeed, perhaps a few thousand ohms. This can make it challenging to transfer energy to the dipole near its end. (See Hands-On Radio Experiment #136: "End-Fed Antennas.")

That said, the half-wave dipole simply doesn't care where you attach the feed line. It will have the same voltage and current distribution if excited with the same amount of power no matter where that power source is attached.

\subsection{Multiband Operation}
For single-band, coax-fed operation, there's no electrical reason to feed the dipole anywhere else but the center; the impedance is a good match to 50 or 75 Ω feed line, and it's mechanically simple. Moving the feed point away from the center makes the antenna system asymmetrical. This results in the feed line picking up a lot of common-mode current, requiring choke baluns and other techniques to isolate the line. So for a single-band dipole, leave the feed point centered.

For multi-band operation, the situation is quite different. Figure 2 shows the current and voltage distribution along the same dipole on its second (blue) and third (red) harmonic. At the second harmonic, the central feed point impedance is now very high, because voltage is high and current is low. You can see a number of high- and low-impedance points along the dipole for the different frequencies. Feeding the dipole at the center means low feed point impedance on the odd harmonics, beginning with the fundamental and high impedances on the even harmonics. The result is a severe impedance mismatch on at least half the bands. If the feed line is low-loss (such as heavy open-wire line), that may be acceptable, and an antenna tuner can be used at the transmitter.

\begin{figure}[htbp]
    \centering
    \includegraphics{../epub_extracted/images/00055.jpeg}
    \caption{Dipole Current and Voltage on Harmonics}
    \label{fig:dipole-harmonics}
\end{figure}

\subsection{Hitting the Spot}
More often, though, a high standing-wave ratio (SWR) is not acceptable due to the high losses it causes in coaxial cable. The compromise is to find one spot along the dipole where feed line impedance is similar on different bands. An impedance transformer converts the impedances to a value that does not create a high SWR in coaxial cable. The SWR will not be 1:1, but the feed line loss will be modest and the impedance will be within the range of most tuners.

The most common such design places the feed point at 1⁄3 of the total dipole length from one end and uses a 4:1 impedance transformer. This results in SWR of less than 2:1 on the fundamental, second, and fourth harmonic, e.g. 40, 20, and 10 meters. (The feed point impedance depends on height above ground, as well.) A current choke should be used at the feed point to block common-mode current and decouple the feed line from interacting with the antenna. Several commercial antennas are available in this configuration and the antenna is popularly known as the off-center-fed dipole (OCFD).

Another popular configuration is to feed the antenna with open-wire line and use a tuner to match the feed line impedance to 50 Ω. There will be more feed line loss than matching at the antenna, but the convenience of using a single feed line outweighs the tolerable performance loss.

In 1996, K1POO analyzed the OCFD and came up with an alternative feed point position at approximately 1⁄6 of the length from one end. Figure 3 shows a conventional OCFD and the K1POO design.

\begin{figure}[htbp]
    \centering
    \includegraphics{../epub_extracted/images/00056.jpeg}
    \caption{OCFD Designs}
    \label{fig:ocfd-designs}
\end{figure}

\subsection{Children of a Common Mother}
What's particularly important to realize, and the reason for this discussion, is that all three of the antennas we've discussed — the half-wave dipole, the end-fed half-wave (EFHW), and the off-center-fed dipole (OCFD) — are the \emph{same antenna!} The only difference is where the feed line is attached. Assuming they have the same amount of power applied to them, they all have the same radiation pattern and gain. Certainly, you may prefer one design over the other for mechanical or aesthetic convenience. Our oldest antenna friend, the dipole, however, will work just as it always has.

\begin{enumerate}
    \item All previous Hands-On Radio experiments are available to ARRL members at \url{http://www.arrl.org/hands-on-radio}.
    \item B. Shackleford, W6YE, "Custom Open-wire Line — It's a Snap," \emph{QST}, July 2011, pp. 33 – 36.
    \item \url{http://rsars.files.wordpress.com/2013/01/k1poo-4-band-ocfd-40-20-15-10m-richard-formato-iss-1-3.pdf}
\end{enumerate}

\part{传输线与阻抗匹配}
\chapter{实验 #131:同轴到开路导线巴伦}

\section{英文原文}

I recently designed some Extended Double Zepp (EDZ) antennas that present a reasonable SWR on 14 and 21 MHz. The design uses a specific length of 450 Ω ladder line, resulting in an SWR of less than 2:1 at the end of the ladder line on both bands. Since that length was too short to reach the shack, I chose to transition from the 450 Ω line to 50 Ω coaxial cable. (The EDZ design will be presented in a future column or article.)

One can just connect the coax to the ladder line and hope for the best — it might work, as some designs for multiband antennas will function that way. Unfortunately, the \textbf{outside} of the coax shield is also connected at the junction of the two feed lines, creating a \textbf{common-mode current path }with impedance depending on the length of the coax and the operating frequency. 

The basic idea is explained in Roy Lewallen's, W7EL, classic article "Baluns & What They Do," at \href{http://www.eznec.com/Amateur/Articles/Baluns.pdf}{www.eznec.com/Amateur/Articles/Baluns.pdf}. If you haven't read it, this would be a good time to do so.

\subsection{The Case for Using a Balun}

Roy's article shows why a current or choke balun is needed at the transition from the coax to a dipole with the wires at right angles to the coax. What if instead of a dipole, the coax is connected to ladder line? Is a balun still necessary? In transmission lines, the conductors are tightly coupled so that the currents are equal and in opposite directions.\textsuperscript{1} That means the same current should flow on the inside of the coax shield and the conductor of the ladder line to which the shield is connected. If any of the current escaped on the outside of the coax feed line as common-mode current, then the balanced current rule would be violated, upsetting the impedance presented at the junction of the two feed lines. 

While the coupling of the two conductors in the feed line \textbf{should} be sufficient to guarantee balanced currents in each, it's a good idea to raise the impedance of the common-mode current path, especially because you don't know the impedance of that path. Common-mode current on feed lines can cause the antenna system to behave unpredictably.

There is another reason to add some common-mode impedance to the feed line — preserving the symmetry of the antenna system. With a balanced antenna such as a dipole or EDZ, \textbf{decoupling} of the feed line's common-mode current path from the antenna's radiated field is also important, as explained in W7EL's article. Since common-mode chokes are difficult to create for ladder line, I oriented that portion of the feed line at close to right angles from the antenna to preserve antenna balance. Adding a choke balun at the junction of coax and ladder line was the next step. (If the coax is parallel to the antenna, add a coiled-coax choke or two along the coax to minimize common-mode current all along the feed line.)

The choke balun can take many forms, as explained in the \textbf{ARRL Handbook} and \textbf{ARRL Antenna Book}.\textsuperscript{2} I decided against the W2DU-style balun of many ferrite beads on the coax because of the expense, and against the coiled-coax balun because it is somewhat heavy and unwieldy when suspended by the feed line (particularly if form-wound). It is also hard to create a scramble-wound choke that works well over the range of 40 to 10 meters (the EDZ is tunable on the WARC bands and 40) so I selected a compromise between all three designs.

\subsection{Balun in a Jiffy}

My choke balun was easily wound on a ferrite toroid core, using a bifilar winding that is really just a very closely spaced parallel-conductor feed line. By using the right mix of ferrite, the choke will create enough impedance across the HF range.

Following the guidance of Jim Brown's, K9YC, tutorials on ferrites and chokes, I chose a 2.4 inch diameter #31 mix with a winding of 10 turns.\textsuperscript{3,4} The ferrite tutorial estimates that the balun's choking impe-dance at 7, 14, and 28 MHz is 3000, 3500, and 2000 Ω, respectively, as shown in Figure 1. For the bifilar winding, I used two-conductor PVC-insulated #16 zip cord which is fine for 100 W power levels. (If you plan on running high power, use #12 or larger wire.) For this core you need about 3 inches of wire per turn plus the connections at either end for a total of about 36 inches of wire, including the input and output connections.\textsuperscript{5}

\begin{figure}[H]
    \centering
    \includegraphics[width=0.7\linewidth]{00057.jpeg}
    \caption{Balun Choking Impedance}
    \label{fig:balun_impedance}
\end{figure}

The balun would be installed outside, suspended in mid-air, with up to 30 feet of coax hanging from the balun. Therefore, I needed a lightweight, non-conductive enclosure that could accommodate an SO-239 connector and the ladder line. New enclosures all seemed to be rectangular, heavy, and expensive. PVC pipe and caps would be \textbf{really} heavy. While sorting through a bag of empty food containers that I use to hold parts, I found my balun enclosure in the form of a peanut butter jar. 

The 16 oz size turned out to be perfect for a 100 W balun and 28 oz jars are large enough for high-power models (Figures 2-4 show balun assembly). The clear jar is tough and a 2.4 inch toroid fits inside after winding, although you have to squeeze the jar a bit to get it through the threaded part of the jar. To get the ladder line through the lid, punch some holes with an awl or small drill bit. Drill or cut a hole in the bottom of the jar that is a little bit bigger than the shell of a PL-259 connector. Drill three or four small holes around the bottom of the jar for drainage. For UV protection, spray paint the jar and lid with outdoor enamel.

\begin{figure}[H]
    \centering
    \includegraphics[width=0.7\linewidth]{00058.jpeg}
    \caption{Balun Assembly Step 1}
    \label{fig:balun_assembly_1}
\end{figure}

\begin{figure}[H]
    \centering
    \includegraphics[width=0.7\linewidth]{00059.jpeg}
    \caption{Balun Assembly Step 2}
    \label{fig:balun_assembly_2}
\end{figure}

\begin{figure}[H]
    \centering
    \includegraphics[width=0.7\linewidth]{00060.jpeg}
    \caption{Balun Assembly Step 3}
    \label{fig:balun_assembly_3}
\end{figure}

Begin winding by securing the first turn with high-quality electrical tape such as Scotch 33+ or a wire tie. Then wind 10 turns on the core, making sure each turn is snug on the core, securing the final turn. I have tried both a single end-to-end winding and the crossover style of winding introduced by W1JR in which after half the turns are wound, the winding crosses through and over to the opposite side of the core, then continues to the point opposite the first turn. The crossover winding has little effect at HF but it conveniently places the input and output connections on opposite sides of the core. This makes the balun easier to assemble and holds it straight between the top and bottom of the jar. Both styles work fine in this use. The input and output leads should be short enough (about 1 inch for the low-power version) that they are not bent against the jar with the lid on.

\begin{figure}[H]
    \centering
    \includegraphics[width=0.7\linewidth]{00058.jpeg}
    \caption{Balun Winding}
    \label{fig:balun_winding}
\end{figure}

To attach the winding to the SO-239, tin the hollow tip of a #4 self-tapping sheet metal screw. Then place a small amount of anti-oxidation compound such as Penetrox on the screw threads and turn it into one of the SO-239 flange holes. (A #6 screw also works but you'll probably have to drill out the SO-239 hole a little bit, depending on the manufacturer.) Then solder one winding wire to the screw and the other to the SO-239 center conductor. Attaching the SO-239 to the jar with more sheet metal screws during installation is optional.

To test the balun before attaching the ladder line, solder a 47 or 51 Ω resistor across the output winding and use an antenna analyzer to measure the balun's input impedance. It should be close to 50 Ω with an SWR of 1:1. Move your hand along the coax and make sure the SWR doesn't change, a symptom of common-mode current on the coax. Polarity of the input and output windings is not important unless you are making a set of baluns in which case you should be consistent in how the windings are attached to the SO-239 and ladder line.

Poke the ladder line conductors through the lid, and then use needlenose pliers to curl the wire into a circle or U for soldering. Solder the output leads to the ladder line wires. If you want, coat those connections with liquid electrical tape or aquarium RTV sealant. Leave the lid off for now.

\begin{figure}[H]
    \centering
    \includegraphics[width=0.7\linewidth]{00059.jpeg}
    \caption{Ladder Line Connection}
    \label{fig:ladder_line_connection}
\end{figure}

To install the balun, insert the coaxial feed line's PL-259 through the hole in the jar and screw it on to the SO-239. Use the coax to pull the core into the jar until the lid is against the threads. Screw the jar back into its lid. The PL-259/SO-239 should turn freely and not bind in the hole. Waterproof the coax connectors and you are done!

\subsection{Parts List}

\begin{itemize}[leftmargin=2cm]
    \item 2.4" #31 mix ferrite toroid core (Fair Rite 2631803802, Mouser 623-2631803802)
    \item 36" of two-conductor, #16 PVC-insulated zip cord (RadioShack 55057440)
    \item 16 oz peanut butter jar
    \item SO-239
    \item #4 self-tapping screw
\end{itemize}

\subsection{References}

\begin{enumerate}
    \item See the discussion of mutual induction and Lenz's law in Experiments #117 and #118. All previous Hands-On Radio experiments are available to ARRL members at \href{http://www.arrl.org/hands-on-radio}{www.arrl.org/hands-on-radio}.
    \item \textbf{The ARRL Handbook} and \textbf{ARRL Antenna Book} are available from your ARRL dealer, or from the ARRL Store. Telephone toll-free in the US 888-277-5289, or 860-594-0355; fax 860-594-0303; \href{http://www.arrl.org/shop/}{www.arrl.org/shop/}; \href{mailto:pubsales@arrl.org}{pubsales@arrl.org}.
    \item \href{http://audiosystemsgroup.com/RFI-Ham.pdf}{audiosystemsgroup.com/RFI-Ham.pdf}.
    \item \href{http://audiosystemsgroup.com/CoaxChokesPPT.pdf}{audiosystemsgroup.com/CoaxChokesPPT.pdf}.
    \item The 36" bifilar winding starts to have an appreciable electrical length at 10 meters and has an additional transforming effect on the impedance that increases with frequency.
\end{enumerate}

\section{中文翻译}

我最近设计了一些扩展双Zepp(EDZ)天线,在14和21 MHz上呈现合理的SWR。该设计使用特定长度的450 Ω梯形线,在两个频段上梯形线末端的SWR均小于2:1。由于该长度太短无法到达 shack,我选择从450 Ω线过渡到50 Ω同轴电缆。(EDZ设计将在未来的专栏或文章中介绍。)

人们可以简单地将同轴电缆连接到梯形线并希望一切顺利 —— 它可能会工作,因为一些多频段天线的设计会以这种方式运行。不幸的是,同轴电缆屏蔽层的\textbf{外部}也在两个馈线的连接处连接,创建了一个\textbf{共模电流路径},其阻抗取决于同轴电缆的长度和工作频率。

Roy Lewallen的W7EL经典文章"Baluns & What They Do"在\href{http://www.eznec.com/Amateur/Articles/Baluns.pdf}{www.eznec.com/Amateur/Articles/Baluns.pdf}中解释了基本思想。如果你还没有读过,现在是个好时机。

\subsection{使用巴伦的理由}

Roy的文章展示了为什么在同轴电缆到偶极子的过渡处需要电流或扼流巴伦,偶极子的导线与同轴电缆成直角。如果同轴电缆不是连接到偶极子,而是连接到梯形线呢?还需要巴伦吗?在传输线中,导体紧密耦合,使得电流大小相等、方向相反。\textsuperscript{1} 这意味着相同的电流应该在同轴电缆屏蔽层的内部和梯形线的导体上流动,屏蔽层连接到该导体。如果任何电流作为共模电流逃逸到同轴馈线的外部,那么平衡电流规则将被违反,扰乱两个馈线连接处呈现的阻抗。

虽然馈线中两个导体的耦合\textbf{应该}足以保证每个导体中的平衡电流,但提高共模电流路径的阻抗是个好主意,尤其是因为你不知道该路径的阻抗。馈线上的共模电流会导致天线系统行为不可预测。

在馈线上添加一些共模阻抗还有另一个原因 —— 保持天线系统的对称性。对于平衡天线(如偶极子或EDZ),将馈线的共模电流路径与天线的辐射场\textbf{解耦}也很重要,正如W7EL的文章中所解释的。由于为梯形线创建共模扼流圈很困难,我将馈线的那部分定向为与天线成近直角,以保持天线平衡。在同轴电缆和梯形线的连接处添加扼流巴伦是下一步。(如果同轴电缆与天线平行,沿同轴电缆添加一个或两个线圈同轴扼流圈,以最小化整个馈线上的共模电流。)

正如\textbf{ARRL手册}和\textbf{ARRL天线手册}中所解释的,扼流巴伦可以采用多种形式。\textsuperscript{2} 我决定不使用同轴电缆上有许多铁氧体珠的W2DU风格巴伦,因为成本高,也不使用线圈同轴巴伦,因为当由馈线悬挂时,它有点重且笨重(特别是如果是成型绕制的)。创建一个在40到10米范围内(EDZ在WARC频段和40米上可调)工作良好的混乱绕制扼流圈也很困难,所以我选择了三种设计之间的折衷方案。

\subsection{快速制作巴伦}

我的扼流巴伦很容易在铁氧体环形磁芯上绕制,使用双线绕制,这实际上只是非常紧密间隔的平行导体馈线。通过使用正确的铁氧体混合材料,扼流圈将在HF范围内创建足够的阻抗。

遵循Jim Brown的K9YC关于铁氧体和扼流圈的教程,我选择了直径2.4英寸的#31混合材料,绕制10匝。\textsuperscript{3,4} 铁氧体教程估计,巴伦在7、14和28 MHz的扼流阻抗分别为3000、3500和2000 Ω,如图1所示。对于双线绕制,我使用了两导体PVC绝缘#16拉链线,适用于100 W功率水平。(如果你计划运行高功率,请使用#12或更大的线。)对于这个磁芯,你需要每匝约3英寸的线,加上两端的连接,总共约36英寸的线,包括输入和输出连接。\textsuperscript{5}

\begin{figure}[H]
    \centering
    \includegraphics[width=0.7\linewidth]{00057.jpeg}
    \caption{巴伦扼流阻抗}
    \label{fig:balun_impedance_cn}
\end{figure}

巴伦将安装在室外,悬挂在空中,最多30英尺的同轴电缆从巴伦垂下。因此,我需要一个轻量级、非导电的外壳,可以容纳SO-239连接器和梯形线。新的外壳似乎都是矩形的,重且昂贵。PVC管和盖子会\textbf{非常}重。在整理我用来存放零件的空食品容器袋时,我发现了我的巴伦外壳 —— 一个花生酱罐。

16盎司的尺寸对于100 W巴伦来说非常完美,28盎司的罐子足够大,可以容纳高功率型号(图2-4显示巴伦组装)。透明的罐子很坚固,绕制后2.4英寸的环形磁芯可以放入其中,尽管你必须稍微挤压罐子才能通过罐子的螺纹部分。要让梯形线穿过盖子,用锥子或小钻头打一些孔。在罐子底部钻或切一个比PL-259连接器外壳稍大的孔。在罐子底部周围钻三四个小孔用于排水。为了防紫外线,用户外搪瓷喷漆喷涂罐子和盖子。

\begin{figure}[H]
    \centering
    \includegraphics[width=0.7\linewidth]{00058.jpeg}
    \caption{巴伦组装步骤1}
    \label{fig:balun_assembly_1_cn}
\end{figure}

\begin{figure}[H]
    \centering
    \includegraphics[width=0.7\linewidth]{00059.jpeg}
    \caption{巴伦组装步骤2}
    \label{fig:balun_assembly_2_cn}
\end{figure}

\begin{figure}[H]
    \centering
    \includegraphics[width=0.7\linewidth]{00060.jpeg}
    \caption{巴伦组装步骤3}
    \label{fig:balun_assembly_3_cn}
\end{figure}

开始绕制时,用高质量的电工胶带(如Scotch 33+)或扎线带固定第一匝。然后在磁芯上绕10匝,确保每一匝都紧贴磁芯,固定最后一匝。我尝试过两种绕制方式:单端到端绕制和W1JR引入的交叉绕制,其中在绕制一半匝数后,绕制穿过并越过到磁芯的另一侧,然后继续到与第一匝相对的点。交叉绕制在HF下影响不大,但它方便地将输入和输出连接放在磁芯的相对侧。这使得巴伦更容易组装,并在罐子的顶部和底部之间保持笔直。两种风格在这种使用中都工作良好。输入和输出引线应该足够短(低功率版本约1英寸),以至于在盖子盖上时不会弯曲抵着罐子。

\begin{figure}[H]
    \centering
    \includegraphics[width=0.7\linewidth]{00058.jpeg}
    \caption{巴伦绕制}
    \label{fig:balun_winding_cn}
\end{figure}

要将绕制连接到SO-239,用焊锡镀#4自攻钣金螺丝的空心尖端。然后在螺丝螺纹上放置少量抗氧化化合物(如Penetrox),并将其拧入SO-239法兰孔之一。(#6螺丝也可以工作,但你可能需要根据制造商的不同稍微钻大SO-239孔。)然后将一根绕制线焊接到螺丝上,另一根焊接到SO-239中心导体上。在安装过程中用更多的钣金螺丝将SO-239固定到罐子上是可选的。

在连接梯形线之前测试巴伦,在输出绕制上焊接47或51 Ω电阻器,并使用天线分析仪测量巴伦的输入阻抗。它应该接近50 Ω,SWR为1:1。沿着同轴电缆移动你的手,确保SWR不变,这是同轴电缆上共模电流的症状。输入和输出绕制的极性不重要,除非你正在制作一组巴伦,在这种情况下,你应该在绕制如何连接到SO-239和梯形线上保持一致。

将梯形线导体穿过盖子,然后使用尖嘴钳将线卷曲成圆形或U形以便焊接。将输出引线焊接到梯形线导线上。如果需要,用液体电工胶带或水族馆RTV密封剂覆盖这些连接。暂时不要盖上盖子。

\begin{figure}[H]
    \centering
    \includegraphics[width=0.7\linewidth]{00059.jpeg}
    \caption{梯形线连接}
    \label{fig:ladder_line_connection_cn}
\end{figure}

要安装巴伦,将同轴馈线的PL-259穿过罐子上的孔,然后将其拧到SO-239上。使用同轴电缆将磁芯拉入罐子,直到盖子抵着螺纹。将罐子拧回盖子。PL-259/SO-239应该可以自由转动,不会在孔中卡住。防水处理同轴连接器,你就完成了!

\subsection{零件清单}

\begin{itemize}[leftmargin=2cm]
    \item 2.4" #31混合铁氧体环形磁芯(Fair Rite 2631803802,Mouser 623-2631803802)
    \item 36"两导体,#16 PVC绝缘拉链线(RadioShack 55057440)
    \item 16盎司花生酱罐
    \item SO-239
    \item #4自攻螺丝
\end{itemize}

\subsection{参考资料}

\begin{enumerate}
    \item 参见实验#117和#118中关于互感和楞次定律的讨论。所有以前的"实践无线电"实验都可在ARRL成员的\href{http://www.arrl.org/hands-on-radio}{www.arrl.org/hands-on-radio}获取。
    \item \textbf{ARRL手册}和\textbf{ARRL天线手册}可从你的ARRL经销商或ARRL商店获取。美国免费电话888-277-5289,或860-594-0355;传真860-594-0303;\href{http://www.arrl.org/shop/}{www.arrl.org/shop/};\href{mailto:pubsales@arrl.org}{pubsales@arrl.org}。
    \item \href{http://audiosystemsgroup.com/RFI-Ham.pdf}{audiosystemsgroup.com/RFI-Ham.pdf}。
    \item \href{http://audiosystemsgroup.com/CoaxChokesPPT.pdf}{audiosystemsgroup.com/CoaxChokesPPT.pdf}。
    \item 36"双线绕制在10米处开始具有明显的电气长度,并对阻抗产生随频率增加的额外变换效果。
\end{enumerate}

\chapter{实验 #137:选择馈线}

\section{英文原文}

An antenna system begins at a radio's output connector and ends in the space around the antenna, including the ground and everything conductive that is nearby (in terms of wavelengths). One never uses just an antenna — rather, an antenna system! What often seems a simple task — measuring, cutting, and hanging up a skyhook — really involves many more questions than you might expect. 

One of the questions you must answer is "What feed line do I use?" and, as with many technical endeavors, the answer begins, "It depends…" While the choice also involves cost and the mechanical aspects of installation, the focus of this column is on the electrical aspects. Assuming the feed line is adequately rated for the power levels you'll be using, the choice boils down to loss in the line.

\subsection{Evaluating Loss with \textbf{TLW}}

Recently, Dean Straw, N6BV, former editor of \textbf{The ARRL Antenna Book}, updated his incredibly useful \textbf{TLW} program (\textbf{Transmission Line for Windows}).\textsuperscript{1} The result is a more accurate assessment of systems using open-wire feed line, aka ladder line or window line. Often used as the feed line for multi-band "doublets" (wire antennas fed in the middle but not necessarily resonant on any band used), open-wire line (OWL) is often assumed to be less lossy than an equivalent length of coaxial cable. With the new version of \textbf{TLW} available, it seemed like a good idea to check this out. (This discussion doesn't include the loss of effects of baluns anywhere in the system.)

Starting with the center-fed, half-wave dipole used on a single band (and maybe on the third harmonic, too), the use of coax is a good choice for moderate feed line length. The feed line SWR and associated losses will be reasonably low, it's easy to connect at both ends, and it lends itself well to switching arrangements between several antennas.

Let's have a look at an example of this type of antenna system: a center-fed, 66-foot long inverted V, 50 feet above average ground, with 100 feet of feed line. According to \textbf{EZNEC} (\href{http://www.eznec.com}{www.eznec.com}), at 7.1 MHz where the antenna is very nearly ½-wavelength long, the antenna's feed point impedance is 65 – \textbf{j} 41 Ω or 77 ∠-32°.\textsuperscript{2} This load creates an SWR of 2.1:1 for 50 Ω cable. One hundred feet of RG-213 coax has a total line loss of 0.7 dB and the SWR at the transmitter end of the line is 1.9:1. (The online \textbf{VSWR-RL-Reflection Coefficient} calculator by Giangrandi is another excellent free tool.\textsuperscript{3})

If instead we fed the inverted V with the same length of 450 Ω OWL, the new version of \textbf{TLW} calculates SWR at the antenna to be 6.3:1 and the line loss as 0.43 dB. While it looks like OWL is slightly less lossy, that figure doesn't include any losses in the impedance matching unit required at the line's input where SWR is still 5.7:1. ARRL Lab measurements of full-legal limit antenna tuners shows an average loss of about 11% for an SWR of 4:1 on 40 meters, which is 0.7 dB.\textsuperscript{3} The total system loss when using OWL in this antenna system is about 1.1 dB. The advantage swings slightly more to coax for shorter feed line lengths and to OWL as feed line length increases.

\subsection{Tuned Feeders}

You can also use an old trick — \textbf{tuned feeders} — to reduce the need for impedance matching. If the OWL is extended to the closest length at which it is an integer number of ½-wavelengths long, the input and output impedances of the line will be very nearly the same. One wavelength of OWL at 7.1 MHz is 126 feet. At this length, a 50 Ω transmitter output sees an impedance of 73 – \textbf{j} 41 Ω (SWR of 2.1:1) and system loss is reduced to 0.5 dB, slightly better than for the coaxial cable. Adjusting the line length to 128.1 feet brings the system almost exactly to resonance with an input impedance of 73 Ω, an SWR of 1.5:1, and total system losses of 0.56 dB. Remember that this will \textbf{only} work on the one band and at the design frequency — you may still wind up needing impedance matchng at the band edges but the SWR will be lower than with the 100-foot feed line.

\subsection{Feed Lines for Multi-band Doublets}

The fun really begins when the antenna is to be used on several bands, including those for which the antenna is not an odd number of ½-wavelengths long, occasionally creating extreme feed point impedances. George Cutsogeorge, W2VJN, created an \textbf{EZNEC} model for \textbf{The ARRL Antenna Book} that shows center-fed feed point impedances on the HF ham bands for a 100-foot long doublet, installed at 50 feet over average ground.\textsuperscript{4} I used the new version of \textbf{TLW} to calculate losses and SWR at the antenna feed point for 100 feet of both RG-213 coax and 450 Ω OWL. All of the data is shown in Table 1.

\begin{figure}[H]
    \centering
    \includegraphics[width=0.7\linewidth]{00061.jpeg}
    \caption{Feed Line Loss Comparison}
    \label{fig:feed_line_loss}
\end{figure}

In this common situation it's not even close — feed line losses for OWL are uniformly lower than for coax, sometimes by a lot. That's not the entire story because you'll still have to provide some kind of impedance matching device at the transmitter most of the time. As we have seen, that adds additional losses, regardless of whether the feed line is OWL or coax. In general and except in extraordinary cases, the old advice is still good: OWL is best for feeding multi-band, non-resonant antennas from the perspective of system losses.

\subsection{Feed Lines for Matched Antennas}

Most of our ham-band antennas, whether Yagis, dipoles, or ground-plane verticals, are designed with a feed point impedance that matches 50 Ω coaxial cables. For these antennas, OWL would not be a good choice due to mechanical and impedance matching considerations. So, how does one choose the right coax for these antennas?

You could start with the manufacturer's specified loss per standard length (usually 100 feet) and pick the lowest-loss line you can afford. However, antenna system designers generally approach the problem from the standpoint of \textbf{allowable loss}. In other words, what is the maximum acceptable amount of feed line loss? They select feed line that for the required length will have less loss than the maximum amount. This is a somewhat harder problem to solve because of the calculations involved.

You're in luck, however, because Frank Donovan, W3LPL, has performed those calculations already. The results are in Table 2 (related tables are also available in \textbf{The ARRL Antenna Book, 22nd Edition}), which shows the length of cable that results in 1 dB of \textbf{matched loss} — the loss for a feed line terminated in its characteristic impedance.

To use these figures, begin by dividing the length of feed line you'll need to use by the maximum acceptable loss in dB. This determines the lower limit for the number of feet per dB of loss in the cable. Find the column showing the frequency at which you are working. Locate the entry in that column with the lowest value of feet/dB \textbf{greater} than the calculated lower limit. That cable is the lossiest you can use and still satisfy your total loss requirement. For example, if you can tolerate 5 dB of feed line loss in a 300-foot run at 440 MHz, your lower loss limit is 300 / 5 = 60 feet / dB. In the 440 MHz column, LDF4-50A hardline is the lossiest cable you can use — start shopping! Note that the values of feet/dB don't increase uniformly from bottom to top due to variations in cable performance over the wide frequency range shown in the table.

What if the cable is \textbf{not} matched? Additional loss will result as the power is reflected back and forth in the feed line. Figure 1 shows just how much more loss. This chart, developed by Joe Reisert, W1JR, combines matched loss and additional SWR-caused loss into a single set of curves.\textsuperscript{5} To find the total loss of your feed line, start with the load SWR (at the antenna) on the horizontal axis. Travel vertically to the curve (or an interpolation between curves) corresponding to the matched loss of the feed line you've selected. Travel horizontally to the vertical axis to find the total attenuation of the feed line. If this total is still below your maximum allowed loss, you can use that cable. If not, find the point at which the maximum allowed loss and load SWR intersect. That will show the maximum matched loss you can accept. You can then work backwards using Table 2 to find the cable you need.

\begin{figure}[H]
    \centering
    \includegraphics[width=0.7\linewidth]{00062.jpeg}
    \caption{Feed Line Loss Chart}
    \label{fig:loss_chart}
\end{figure}

\begin{figure}[H]
    \centering
    \includegraphics[width=0.7\linewidth]{00063.jpeg}
    \caption{Feed Line Loss Data}
    \label{fig:loss_data}
\end{figure}

Feed line selection is not always so complicated. Most of the time you can just hook up the antenna and you'll work stations. To be sure you're getting the most out of your precious watts, though, these sorts of tools can help you make your decision.

\subsection{References}

\begin{enumerate}
    \item \textbf{Transmission Line for Windows}, version 3.21, by Dean Straw, N6BV. See \href{http://www.arrl.org/qst-in-depth}{www.arrl.org/qst-in-depth} for software availability.
    \item Joel Hallas, W1ZR, \textbf{The ARRL Guide to Antenna Tuners}, Chapter 14, ARRL, 2010.
    \item "Conversions between VSWR — Return Loss — Reflection coefficient," Iacopo Giangrandi, \href{http://www.giangrandi.ch/electronics/anttool/swr.html}{www.giangrandi.ch/electronics/anttool/swr.html}.
    \item \textbf{The ARRL Antenna Book, 22nd Edition}, Chapter 24, ARRL, 2012.
    \item Hands-On Radio Experiment #119 "The Q3Q Balun Redux." All previous Hands-On Radio experiments are available to ARRL members at \href{http://www.arrl.org/hands-on-radio}{www.arrl.org/hands-on-radio}.
\end{enumerate}

\section{中文翻译}

天线系统从收音机的输出连接器开始,结束于天线周围的空间,包括地面和附近的所有导电物体(以波长为单位)。人们从不只使用天线 —— 而是使用天线系统!通常看似简单的任务 —— 测量、切割和悬挂空中天线 —— 实际上涉及的问题比你想象的要多得多。

你必须回答的问题之一是"我使用什么馈线?",就像许多技术尝试一样,答案开始于"这取决于..."。虽然选择还涉及成本和安装的机械方面,但本专栏的重点是电气方面。假设馈线的额定功率足以满足你将使用的功率水平,选择归结为线路中的损耗。

\subsection{使用 \textbf{TLW} 评估损耗}

最近,前\textbf{ARRL天线手册}编辑Dean Straw(N6BV)更新了他非常有用的\textbf{TLW}程序(\textbf{Windows传输线})。\textsuperscript{1} 结果是对使用开路导线馈线(也称为梯形线或窗口线)的系统进行更准确的评估。开路导线(OWL)通常用作多频段"偶极子"(在中间馈电但不一定在使用的任何频段上谐振的线天线)的馈线,通常被认为比等效长度的同轴电缆损耗更小。随着新版本\textbf{TLW}的推出,检查这一点似乎是个好主意。(本讨论不包括系统中任何地方的巴伦损耗效应。)

从用于单个频段(也可能用于三次谐波)的中心馈电半波偶极子开始,同轴电缆是中等长度馈线的良好选择。馈线SWR和相关损耗将相当低,两端连接容易,并且非常适合多个天线之间的切换安排。

让我们看一下这种类型的天线系统的例子:中心馈电、66英尺长的倒V形天线,高出平均地面50英尺,馈线长度为100英尺。根据\textbf{EZNEC}(\href{http://www.eznec.com}{www.eznec.com}),在7.1 MHz处,天线非常接近½波长,天线的馈电点阻抗为65 – \textbf{j} 41 Ω或77 ∠-32°。\textsuperscript{2} 这种负载为50 Ω电缆创建2.1:1的SWR。100英尺的RG-213同轴电缆的总线损耗为0.7 dB,线路发射机端的SWR为1.9:1。(Giangrandi的在线\textbf{VSWR-RL-反射系数}计算器是另一个优秀的免费工具。\textsuperscript{3})

如果我们改用相同长度的450 Ω OWL馈电倒V形天线,新版本的\textbf{TLW}计算出天线处的SWR为6.3:1,线路损耗为0.43 dB。虽然OWL看起来损耗稍小,但该数字不包括线路输入处所需的阻抗匹配单元的任何损耗,那里的SWR仍然为5.7:1。ARRL实验室对全法定限制天线调谐器的测量显示,在40米处SWR为4:1时,平均损耗约为11%,即0.7 dB。\textsuperscript{3} 在该天线系统中使用OWL时的总系统损耗约为1.1 dB。对于较短的馈线长度,同轴电缆的优势略大,而随着馈线长度的增加,OWL的优势更大。

\subsection{调谐馈线}

你还可以使用一个老技巧 —— \textbf{调谐馈线} —— 来减少对阻抗匹配的需求。如果将OWL延长到最接近的整数个½波长长度,线路的输入和输出阻抗将非常接近相同。在7.1 MHz处,OWL的一个波长为126英尺。在此长度下,50 Ω发射机输出看到的阻抗为73 – \textbf{j} 41 Ω(SWR为2.1:1),系统损耗减少到0.5 dB,略好于同轴电缆。将线路长度调整到128.1英尺,使系统几乎完全谐振,输入阻抗为73 Ω,SWR为1.5:1,总系统损耗为0.56 dB。请记住,这\textbf{仅}在一个频段和设计频率上工作 —— 你可能仍然需要在频段边缘进行阻抗匹配,但SWR将低于100英尺馈线的情况。

\subsection{多频段偶极子的馈线}

当天线要用于多个频段时,真正的乐趣开始了,包括那些天线不是奇数个½波长长度的频段,偶尔会产生极端的馈电点阻抗。George Cutsogeorge(W2VJN)为\textbf{ARRL天线手册}创建了一个\textbf{EZNEC}模型,显示了100英尺长偶极子在50英尺高的平均地面上安装时在HF业余频段上的中心馈电点阻抗。\textsuperscript{4} 我使用新版本的\textbf{TLW}计算了100英尺RG-213同轴电缆和450 Ω OWL在天线馈电点的损耗和SWR。所有数据都显示在表1中。

\begin{figure}[H]
    \centering
    \includegraphics[width=0.7\linewidth]{00061.jpeg}
    \caption{馈线损耗比较}
    \label{fig:feed_line_loss_cn}
\end{figure}

在这种常见情况下,结果一目了然 —— OWL的馈线损耗始终低于同轴电缆,有时差距很大。这不是全部情况,因为你仍然需要在大多数时候在发射机处提供某种阻抗匹配设备。正如我们所看到的,无论馈线是OWL还是同轴电缆,这都会增加额外的损耗。一般来说,除了特殊情况外,旧的建议仍然有效:从系统损耗的角度来看,OWL最适合为多频段、非谐振天线馈电。

\subsection{匹配天线的馈线}

我们大多数业余频段天线,无论是八木天线、偶极子还是接地平面垂直天线,都设计有与50 Ω同轴电缆匹配的馈电点阻抗。对于这些天线,由于机械和阻抗匹配考虑,OWL不是一个好选择。那么,如何为这些天线选择正确的同轴电缆呢?

你可以从制造商指定的每标准长度(通常为100英尺)的损耗开始,选择你能负担得起的损耗最低的线路。然而,天线系统设计师通常从\textbf{允许损耗}的角度来解决这个问题。换句话说,馈线损耗的最大可接受量是多少?他们选择的馈线在所需长度下的损耗小于最大量。由于涉及计算,这是一个稍微难以解决的问题。

不过,你很幸运,因为Frank Donovan(W3LPL)已经执行了这些计算。结果在表2中(相关表格也可在\textbf{ARRL天线手册,第22版}中找到),显示了导致1 dB\textbf{匹配损耗}的电缆长度 —— 馈线在其特性阻抗下终端的损耗。

要使用这些数字,首先将你需要使用的馈线长度除以最大可接受的dB损耗。这确定了电缆中每dB损耗的英尺数的下限。找到显示你工作频率的列。在该列中找到每dB英尺值最低且\textbf{大于}计算出的下限的条目。该电缆是你可以使用的损耗最大的电缆,同时仍能满足你的总损耗要求。例如,如果你能在440 MHz下的300英尺运行中容忍5 dB的馈线损耗,你的低损耗限制是300 / 5 = 60英尺/dB。在440 MHz列中,LDF4-50A硬线是你可以使用的损耗最大的电缆 —— 开始购物!请注意,由于电缆在表中显示的宽频率范围内的性能变化,英尺/dB值不会从底部到顶部均匀增加。

如果电缆\textbf{不}匹配会怎样?当功率在馈线中来回反射时,会导致额外的损耗。图1显示了额外的损耗有多大。这个由Joe Reisert(W1JR)开发的图表将匹配损耗和额外的SWR引起的损耗组合成一组曲线。\textsuperscript{5} 要找到馈线的总损耗,从水平轴上的负载SWR(在天线处)开始。垂直移动到与你选择的馈线的匹配损耗相对应的曲线(或曲线之间的插值)。水平移动到垂直轴以找到馈线的总衰减。如果这个总数仍然低于你最大允许的损耗,你可以使用该电缆。如果不是,找到最大允许损耗和负载SWR相交的点。这将显示你可以接受的最大匹配损耗。然后你可以使用表2向后工作,找到你需要的电缆。

\begin{figure}[H]
    \centering
    \includegraphics[width=0.7\linewidth]{00062.jpeg}
    \caption{馈线损耗图表}
    \label{fig:loss_chart_cn}
\end{figure}

\begin{figure}[H]
    \centering
    \includegraphics[width=0.7\linewidth]{00063.jpeg}
    \caption{馈线损耗数据}
    \label{fig:loss_data_cn}
\end{figure}

馈线选择并不总是如此复杂。大多数时候,你只需连接天线,就能工作。不过,为了确保你充分利用宝贵的瓦特,这些工具可以帮助你做出决定。

\subsection{参考资料}

\begin{enumerate}
    \item \textbf{Transmission Line for Windows},版本3.21,由Dean Straw(N6BV)编写。有关软件可用性,请参见\href{http://www.arrl.org/qst-in-depth}{www.arrl.org/qst-in-depth}。
    \item Joel Hallas,W1ZR,\textbf{ARRL天线调谐器指南},第14章,ARRL,2010年。
    \item "Conversions between VSWR — Return Loss — Reflection coefficient",Iacopo Giangrandi,\href{http://www.giangrandi.ch/electronics/anttool/swr.html}{www.giangrandi.ch/electronics/anttool/swr.html}。
    \item \textbf{ARRL天线手册,第22版},第24章,ARRL,2012年。
    \item 实践无线电实验#119 "The Q3Q Balun Redux"。所有以前的实践无线电实验都可在ARRL成员的\href{http://www.arrl.org/hands-on-radio}{www.arrl.org/hands-on-radio}获取。
\end{enumerate}

\chapter{实验 #157:匹配网络设计与构建}

\section{英文原文}

There comes a time in every ham's experiences when a single-band antenna needs to have its feed point impedance matched to 50 \textOmega. If practical, matching the antenna to 50 \textOmega at the feed point is the way to go and the subject of this month's column. You can apply the general process to your antenna farm, too.

I recently built two 60-foot towers to be used as base-fed monopole verticals on 160 and 80 meters. Both had significant top-loading from beams and mast extensions that lowered the frequency of quarter-wave resonance below the intended bands, so I decided to whip up an L network for each.\textsuperscript{1}

It's possible to just find a variable inductor and capacitor, hook them up, and start cranking in hopes of finding a match. But wouldn't it show a little more radio \textbf{savoir faire} to do a proper design?

\subsection{Building by Design}

Any design of a matching network begins with measuring the impedance you want to match, which sounds obvious but is often overlooked. This was a perfect opportunity to make use of my new SARK 110 Antenna Analyzer (\href{http://www.sark110.com}{www.sark110.com}) to characterize what was happening at the base of my towers. Table 1 shows the impedances at selected frequencies across 160 and 80 meters.

The 160 meter tower consists of 60 feet of Rohn 25 topped by a C-31XR triband Yagi and 20 additional feet of 2-inch aluminum mast. Forty copper radials surround the bottom. The resonant frequency is 1.65 MHz (Z = 18.5 + \textbf{j} 0.03 \textOmega) and the resistance is stable between 15 and 17 \textOmega across the band with the positive inductive reactance climbing to 42.3 \textOmega at the top of the band. The average impedance across the band is 15.8 + \textbf{j} 32.1 \textOmega.

The 80 meter tower is the same height, topped with an EF-230 two-element 30 meter beam and an XR5-T multiband HF Yagi on 6 feet of mast above the tower. There were no resonances observed between 2 and 6 MHz although reactance does dip to --65 \textOmega just above 3 MHz. Across the 80/75 meter band, impedance varies quite a bit as shown in the table, averaging 116.9 -- \textbf{j} 151.8 \textOmega.

\subsection{Striking a Match}

Once you have a good set of measurements (hold that thought) you can start coming up with a buildable network. I entered the average values of impedance into the online matching network calculator by John Wetherell, which gives component values for 16 types of networks.\textsuperscript{2} For both bands, there were two unworkable versions of the L network: either the math "blew up," resulting in NaN (Not a Number), or negative values were generated for one or more of the component values. 

Note that the Q value for an L network is fixed at the ratio of the source and load impedances. Q determines the matching bandwidth, and higher values of Q also means higher values of circulating current and peak voltage in the components. If the input and output impedances are very different, investigate networks that transform impedance in smaller steps.

Of the two remaining networks, I selected the circuit in Figure 1 for both networks, because I had a pair of heavy-duty 2000 pF vacuum-variable capacitors thanks to K9SD. The inductors I would wind myself, so I logged on to the K7MEM Single-Layer Air-Core Inductor Design website.\textsuperscript{3} This calculator produces a fairly accurate single-layer, air-core design with lots of options for adjusting mechanical dimensions.

\begin{figure}[H]
    \centering
    \includegraphics[width=0.7\linewidth]{00064.jpeg}
    \caption{L Network Configuration}
    \label{fig:l_network}
\end{figure}

Figure 2 shows the completed 80 meter matching network. Wanting a low-loss, mechanically sturdy coil, I wound it on a 2\textsuperscript{3}/\textsuperscript{4}-inch O.D. PVC coupling from 20 feet of \textsuperscript{1}/\textsuperscript{4}-inch O.D. soft-copper refrigeration tubing. (This tubing is available inexpensively from hardware stores in a number of diameters.) The 20-foot length was enough tubing for a couple of extra turns. I soldered quick-disconnect male terminals on the final three turns to allow for adjustment.

\begin{figure}[H]
    \centering
    \includegraphics[width=0.7\linewidth]{00065.jpeg}
    \caption{80 Meter Matching Network}
    \label{fig:80m_network}
\end{figure}

Winding a coil with tubing this large and soft can be difficult to do well. Kinks and bends are easy to make and impossible to remove. I enlisted the help of a friend to carefully feed the tubing out of the coil in which it is supplied. I formed the tubing around the PVC form by hand as he fed me the tubing, rotating the form and pulling the tubing into a close-wound coil. I then stretched the coil to the desired length, separating the turns by about \textsuperscript{1}/\textsuperscript{8} inch, as you can see in the photo.

The network is built on \textsuperscript{1}/\textsuperscript{2}-inch plywood and uses copper strap and heavy wire to connect the components. Input and output connectors are mounted on a piece of heavy plastic sheet. The coil is strong enough to support itself, but I left it on the PVC form held in place by two sheet-metal screws. Before I installed the network in an enclosure on the tower, I used a DVM with a capacitance function to set the capacitor to 1560 pF. Installed, at 3.55 MHz the SWR at the input was less than 1.1:1! I made some small adjustments of C, both up and down, to verify the network was behaving as it should, then set it back to the original value and left it alone.

Now I have a confession to make. When I first measured the 160 meter tower's base impedance, I confused impedance values between the two frequency markers on the SARK 110 screen. I saw "55 + \textbf{j} 88 \textOmega" and did not realize this was measured outside the band! My desire for a simple match (a series capacitor of 980 pF would do the job) overpowered caution. Assuming a series capacitor would suffice, I did not have any heavy wire or tubing on hand to wind a second inductor.

As shown in Figure 3, I mounted the second 2000 pF vacuum-variable on a small plastic cutting board made of high-density polyethylene (HDPE). These inexpensive (a set of three boards for $10) and tough kitchen items make great resources for the thrifty RF builder. While the 80 meter capacitor came with its custom mounting bracket, I had to make my own for this second unit. Luckily, two of the body rings fit 2-inch \textbf{U}-bolts very nicely, only requiring that the saddles be supported with an extra set of nuts. The input SO-239 was attached directly to a screw on the capacitor's body ring. A large hose clamp was used to make a connection to the second body ring.

\begin{figure}[H]
    \centering
    \includegraphics[width=0.7\linewidth]{00066.jpeg}
    \caption{160 Meter Matching Network}
    \label{fig:160m_network}
\end{figure}

\begin{figure}[H]
    \centering
    \includegraphics[width=0.7\linewidth]{00067.jpeg}
    \caption{Capacitor Mounting Detail}
    \label{fig:capacitor_mounting}
\end{figure}

Needless to say, designed from the wrong value of input impedance, my tuner...didn't! Chastened, I took a second set of data shown in Table 1. Using average impedance values with the online calculators, I determined that I needed a shunt coil of about 3 \textmu H. 

Away from my usual supplies, I was faced with another improvisation (remember Experiment #152?) but some #14 AWG stranded THHN wire and an empty soft-drink bottle saved the day. I wound a couple of extra turns on the bottle, attached it between the capacitor and SO-239, and headed back out to the tower. After removing a turn from the coil (I'd wound two extra), adjusting the capacitor gave me an SWR of 1.5:1 at 1.840 MHz! The SWR increases to 1.8:1 at 1800 and stays below 2:1 all the way to 1.915 MHz.

\subsection{Your Assignment}

You might be thinking, "Well, I can do that!" and of course, you can. Do you have a single-band vertical or wire antenna that has an elevated SWR? You can measure its feed point impedance with a snazzy analyzer like an SARK 110 or by making measurements with an MFJ or similar analog instrument. #12 or #14 AWG solid copper wire salvaged from unused ac wiring makes a fine coil, too. If you don't have an antenna to match, model a design or two and begin with that data. With a little design and workbench time, you'll soon be singing a happy tune.

\subsection{Notes}

\begin{enumerate}
    \item The L network was covered in "Hands-On Radio" experiment #21. All previous "Hands-On Radio" columns are available to ARRL members at \href{http://www.arrl.org/hands-on-radio}{www.arrl.org/hands-on-radio}.
    \item The calculator is available at \href{http://home.sandiego.edu/~ekim/e194rfs01/jwmatcher/matcher2.html}{home.sandiego.edu/~ekim/e194rfs01/jwmatcher/matcher2.html} and on other websites.
    \item There are a number of design calculators at \href{http://www.k7mem.com/Electronic_Notebook/inductors/coildsgn.html}{www.k7mem.com/Electronic_Notebook/inductors/coildsgn.html}.
\end{enumerate}

\section{中文翻译}

在每个火腿的经历中,总有一个时候需要将单频段天线的馈电点阻抗匹配到50 \textOmega。如果可行,在馈电点将天线匹配到50 \textOmega是最佳选择,这也是本月专栏的主题。你也可以将一般过程应用到你的天线群中。

我最近建造了两座60英尺高的塔,用作160和80米频段的底部馈电单极垂直天线。两座塔都有来自波束和桅杆延伸的显著顶部加载,这降低了四分之一波长谐振的频率,低于预期频段,因此我决定为每座塔制作一个L网络。\textsuperscript{1}

你可以只是找到一个可变电感器和电容器,将它们连接起来,然后开始调整,希望找到匹配。但是,进行适当的设计难道不会显示出更多的无线电\textbf{ savoir faire}吗?

\subsection{按设计构建}

任何匹配网络的设计都始于测量你想要匹配的阻抗,这听起来很明显,但经常被忽视。这是一个很好的机会,可以使用我的新SARK 110天线分析仪(\href{http://www.sark110.com}{www.sark110.com})来表征塔底发生的情况。表1显示了160和80米频段选定频率的阻抗。

160米塔由60英尺的Rohn 25组成,顶部有一个C-31XR三频段八木天线和20英尺的2英寸铝制桅杆。底部周围有40根铜辐射器。谐振频率为1.65 MHz(Z = 18.5 + \textbf{j} 0.03 \textOmega),整个频段的电阻稳定在15到17 \textOmega之间,正电感电抗在频段顶部攀升至42.3 \textOmega。整个频段的平均阻抗为15.8 + \textbf{j} 32.1 \textOmega。

80米塔高度相同,顶部有一个EF-230两单元30米波束和一个XR5-T多频段HF八木天线,位于塔上方6英尺的桅杆上。在2到6 MHz之间没有观察到谐振,尽管电抗在3 MHz以上略降至-65 \textOmega。在80/75米频段,阻抗变化很大,如表所示,平均为116.9 - \textbf{j} 151.8 \textOmega。

\subsection{实现匹配}

一旦你有了一组好的测量值(记住这一点),你就可以开始设计可构建的网络。我将平均阻抗值输入到John Wetherell的在线匹配网络计算器中,该计算器给出16种网络的组件值。\textsuperscript{2} 对于两个频段,L网络有两个不可行的版本:要么数学计算"爆炸",导致NaN(非数字),要么为一个或多个组件值生成负值。

请注意,L网络的Q值固定为源阻抗和负载阻抗的比率。Q决定了匹配带宽,Q值越高,组件中的循环电流和峰值电压也越高。如果输入和输出阻抗非常不同,请研究以较小步骤变换阻抗的网络。

在剩下的两个网络中,我为两个网络都选择了图1中的电路,因为我有一对由K9SD提供的重型2000 pF真空可变电容器。电感器我会自己绕制,所以我登录了K7MEM单层空芯电感器设计网站。\textsuperscript{3} 这个计算器产生相当准确的单层空芯设计,有很多选项可以调整机械尺寸。

\begin{figure}[H]
    \centering
    \includegraphics[width=0.7\linewidth]{00064.jpeg}
    \caption{L网络配置}
    \label{fig:l_network_cn}
\end{figure}

图2显示了完成的80米匹配网络。为了获得低损耗、机械坚固的线圈,我将其绕在2\textsuperscript{3}/\textsuperscript{4}英寸外径的PVC接头上,使用20英尺的\textsuperscript{1}/\textsuperscript{4}英寸外径软铜制冷管。(这种管道可以从五金店以多种直径廉价购买。)20英尺的长度足够用于额外的几圈。我在最后三圈上焊接了快速断开公端子,以允许调整。

\begin{figure}[H]
    \centering
    \includegraphics[width=0.7\linewidth]{00065.jpeg}
    \caption{80米匹配网络}
    \label{fig:80m_network_cn}
\end{figure}

用这么大且柔软的管道绕制线圈可能很难做好。扭结和弯曲很容易形成且无法去除。我请了一位朋友帮忙,小心地将管道从供应的线圈中拉出。当他给我送管道时,我用手将管道绕在PVC模板上,旋转模板并将管道拉入紧密绕制的线圈中。然后我将线圈拉伸到所需长度,将匝间距分开约\textsuperscript{1}/\textsuperscript{8}英寸,如照片所示。

网络建在\textsuperscript{1}/\textsuperscript{2}英寸胶合板上,使用铜带和粗线连接组件。输入和输出连接器安装在一块重型塑料板上。线圈足够坚固,可以自行支撑,但我将其留在PVC模板上,由两个钣金螺丝固定。在将网络安装到塔上的外壳之前,我使用带有电容功能的DVM将电容器设置为1560 pF。安装后,在3.55 MHz时,输入处的SWR小于1.1:1!我对C进行了一些小调整,上下都有,以验证网络是否按预期工作,然后将其设置回原始值并保持不变。

现在我要坦白一件事。当我第一次测量160米塔的底部阻抗时,我混淆了SARK 110屏幕上两个频率标记之间的阻抗值。我看到"55 + \textbf{j} 88 \textOmega",但没有意识到这是在频段外测量的!我对简单匹配的渴望(980 pF的串联电容器就可以完成工作)战胜了谨慎。假设串联电容器就足够了,我手头没有任何粗线或管道来绕制第二个电感器。

如图3所示,我将第二个2000 pF真空可变电容器安装在由高密度聚乙烯(HDPE)制成的小塑料切菜板上。这些廉价(一套三块板10美元)且坚固的厨房用品为节俭的RF建设者提供了很好的资源。虽然80米电容器带有其定制安装支架,但我必须为这个第二个单元制作自己的支架。幸运的是,两个本体环非常适合2英寸的\textbf{U}形螺栓,只需要用额外的一组螺母支撑鞍座。输入SO-239直接连接到电容器本体环上的螺丝。使用一个大软管夹连接到第二个本体环。

\begin{figure}[H]
    \centering
    \includegraphics[width=0.7\linewidth]{00066.jpeg}
    \caption{160米匹配网络}
    \label{fig:160m_network_cn}
\end{figure}

\begin{figure}[H]
    \centering
    \includegraphics[width=0.7\linewidth]{00067.jpeg}
    \caption{电容器安装细节}
    \label{fig:capacitor_mounting_cn}
\end{figure}

不用说,根据错误的输入阻抗值设计的调谐器...不起作用!我受到了教训,获取了表1中显示的第二组数据。使用在线计算器的平均阻抗值,我确定需要一个约3 \textmu H的并联线圈。

远离我通常的用品,我面临着另一个即兴创作(记得实验#152吗?),但一些#14 AWG绞合THHN电线和一个空软饮料瓶拯救了这一天。我在瓶子上多绕了几圈,将其连接在电容器和SO-239之间,然后回到塔上。从线圈上移除一匝(我多绕了两匝)后,调整电容器使我在1.840 MHz时获得1.5:1的SWR!SWR在1800时增加到1.8:1,并在整个1.915 MHz范围内保持低于2:1。

\subsection{你的任务}

你可能在想,"嗯,我能做到!",当然,你能。你有单频段垂直或线天线的SWR升高吗?你可以使用像SARK 110这样的时髦分析仪或通过使用MFJ或类似的模拟仪器进行测量来测量其馈电点阻抗。从未使用的交流布线中回收的#12或#14 AWG实心铜线也可以制作出很好的线圈。如果你没有要匹配的天线,模拟一两个设计并从该数据开始。通过一点设计和工作台时间,你很快就会唱着快乐的调子。

\subsection{注释}

\begin{enumerate}
    \item L网络在"实践无线电"实验#21中有所介绍。所有以前的"实践无线电"专栏都可在ARRL成员的\href{http://www.arrl.org/hands-on-radio}{www.arrl.org/hands-on-radio}获取。
    \item 计算器可在\href{http://home.sandiego.edu/~ekim/e194rfs01/jwmatcher/matcher2.html}{home.sandiego.edu/~ekim/e194rfs01/jwmatcher/matcher2.html}和其他网站上获取。
    \item 有许多设计计算器在\href{http://www.k7mem.com/Electronic_Notebook/inductors/coildsgn.html}{www.k7mem.com/Electronic_Notebook/inductors/coildsgn.html}。
\end{enumerate}

\chapter{实验 #159:更多 L 网络设计}

\section{英文原文}

My recent column on designing matching networks generated a fair amount of interest from readers interested in solving a particular matching problem or learning more about impedance matching.\textsuperscript{1} I thought it would be a good idea to have a practice session and explore a little bit of the Smith chart at the same time.

\subsection{A Moving Tale}

Let's start by having a look at Figure 1, which shows the various possibilities of using \textbf{L} networks to match a load (Z\textsubscript{LOAD}) to some nominal reference impedance (Z\textsubscript{0}).\textsuperscript{2, 3} These are simplified Smith charts, with the unshaded half representing the set of impedance values that are matched using the circuit immediately below. Each of the four network configurations shown can be designed to \textbf{transform} (match) any impedance in the unshaded region to the impedance at the center of the chart. 

\begin{figure}[H]
    \centering
    \includegraphics[width=0.7\linewidth]{00068.jpeg}
    \caption{L Network Matching Regions}
    \label{fig:l_network_regions}
\end{figure}

Each particular starting point — the load, Z\textsubscript{LOAD} — can be transformed into Z\textsubscript{0} in two steps. The first step takes you to a point labeled Z\textsubscript{1} on the border of the matchable region. The second step takes you from Z\textsubscript{1} to the chart's reference impedance, Z\textsubscript{0}, which is usually 50 \textOmega for Amateur Radio systems. Each move is following some kind of curved path. What are those curves?

Figure 2 has two parts. The first part is a standard Smith chart, showing impedance coordinates on the black outer circle and center line, plus a set of red arcs and circles. The second part shows the coordinates of \textbf{admittances}, which are the reciprocals of impedances. Admittance coordinates share the outer circle and center line with impedance coordinates but use a mirror-imaged set of blue arcs and circles. 

\begin{figure}[H]
    \centering
    \includegraphics[width=0.7\linewidth]{00069.jpeg}
    \caption{Smith Chart with Impedance and Admittance Coordinates}
    \label{fig:smith_chart}
\end{figure}

Any point on the chart's center line represents a pure resistance or conductance (R + \textbf{j}0 or G + \textbf{j}0) and any point on the outer circle represents a pure reactance or susceptance (0 + \textbf{j}X or 0 + \textbf{j}B). Points on one of the red circles all have the same resistance and points on one of the blue circles all have the same conductance. Points on a red arc all have the same reactance and points on a blue arc all have the same susceptance.\textsuperscript{4} Think of this set of intersecting circles, arcs, and lines as our impedance-matching chessboard.

How does this chessboard "work?" Pieces (impedances or admittances) are moved (transformed) by adding reactances connected in series or parallel (called a \textbf{shunt} connection). As shown in the circuits of Figure 1, the moves can only be along the constant resistance (or conductance) circles, since they don't add or subtract resistance (or conductance). For example, if I add some positive inductive reactance in series with a load, the added reactance will "move" the load clockwise along a constant-resistance circle. Similarly, if I add some positive capacitive susceptance in parallel with a load, the point will move clockwise along a constant-conductance circle. Adding reactance or susceptance in series or parallel with load can only "move the load" around the chart on these circles.

If I want to transform Z\textsubscript{LOAD} to the reference impedance Z\textsubscript{0}, then I have to move part of the way along one type of circle until it encounters the complementary circle that intersects the chart's center. Then I move along that new circle to Z\textsubscript{0}. Just like a knight in chess or a Manhattan cab. I have to move first in one direction, then switch to another direction. 

Figure 3 shows how there are two solutions to every such problem. Start with the point labeled Z\textsubscript{LOAD}. It has an impedance of approximately 0.5 – \textbf{j} 0.1 \textOmega. (Remember, everything is normalized so that the center point, presumably 50 + \textbf{j} 0 \textOmega, is represented as 1.0 + \textbf{j} 0 \textOmega.) By adding series X\textsubscript{L} or X\textsubscript{C}, I can move clockwise (X\textsubscript{L}) or counterclockwise (X\textsubscript{C}) along the red R = 0.5 \textOmega circle. I need to add enough reactance to reach the blue circle for G = 1.0 S (S stands for siemens, the unit of conductance). Once there, I can move to Z\textsubscript{0} by adding either shunt L or shunt C as noted. One set of "moves" represents what happens in the circuit of Figure 1C and the other in Figure 1D. Depending on the value of Z\textsubscript{LOAD}, there are always two possible circuits that will get you to Z\textsubscript{0}.

\begin{figure}[H]
    \centering
    \includegraphics[width=0.7\linewidth]{00070.jpeg}
    \caption{Two Solutions on Smith Chart}
    \label{fig:two_solutions}
\end{figure}

Why would I choose one circuit over the other since both choices wind up at the same point on the chart? Unlike the chess knight that must always move in an \textbf{L} of 2 squares by 1 square, the\textbf{ L} network's pair of reactances can take a wide range of values. One of those pairs will have more practical or convenient values (or be cheaper to purchase) than the other. 

\subsection{Working On Your Knight Moves}

Okay, your turn! You should now understand why Wetherell's L network calculator can solve an impedance matching problem for only two out of the four possible L-C L networks.\textsuperscript{5} Those are the two network configurations with Z\textsubscript{LOAD} in matchable regions of the Smith chart.

Another thing you've just learned: If an L network can't match an impedance — turn it around! If you look at the matchable regions of the two networks that are mirror images of each other, such as in Figure 1A and 1C, you'll see they are complementary. So by turning the network around, the previously unmatchable impedance is guaranteed to be in the matchable region for the new circuit's configuration.

Let's try some example calculations to get you into the game and moving your pieces around. Find the two circuits in Figure 1 that can be used for each example and the component values required. (Answers are at the end of the Notes list.)

Example 1: Z\textsubscript{LOAD }is the feed point impedance of a 10.1 MHz full-wave loop: 110 + \textbf{j} 50 \textOmega.

Example 2: Z\textsubscript{LOAD} is the feed point impedance of a Yagi's driven element at 144.2 MHz: 22 – \textbf{j} 10 \textOmega.

Example 3: Z\textsubscript{LOAD} is the feed point impedance of an off-center-fed dipole at 24.9 MHz: 250 – \textbf{j} 150 \textOmega.

Once you get the hang of it, you can have a lot of fun trying different versions of the matching networks, pushing the impedances to extreme values (high, low, highly reactive), and evaluating which circuit you'd choose to build.

\subsection{Notes}

\begin{enumerate}
    \item See Experiments #21, "The L Network" and #157 "Matching Network — Design and Build." All previous "Hands-On Radio" columns are available to ARRL members at \href{http://www.arrl.org/hands-on-radio}{www.arrl.org/hands-on-radio}.
    \item A complete set of all reactive networks and their matchable regions of the Smith chart are available in the RF Techniques chapter of \textbf{The ARRL Handbook}.
    \item \textbf{The ARRL Handbook} is available from the ARRL at \href{http://www.arrl.org/shop}{www.arrl.org/shop} and from dealers who carry ARRL publications.
    \item "Hands-On Radio" Experiments #59 – 61 explain the Smith chart, and there is additional material on the CD-ROM accompanying recent editions of \textbf{The ARRL Handbook}.
    \item Wetherell's calculator is available at \href{http://home.sandiego.edu/~ekim/e194rfs01/jwmatcher/matcher2.html}{home.sandiego.edu/~ekim/e194rfs01/jwmatcher/matcher2.html} and on other websites.
\end{enumerate}

\subsection{Answers to example problems:}

Example 1: A and B; A – 1013 nH & 207 pF; B – 2515 nH & 245 pF

Example 2: D and C; D – 48.9 nH and 74.5 pF; C – 38.4 nH and 24.9 pF

Example 3: A and C; A – 770 nH & 34 pF; C – 722 nH & 53 pF

\section{中文翻译}

我最近关于设计匹配网络的专栏引起了读者的相当大兴趣,他们有兴趣解决特定的匹配问题或了解更多关于阻抗匹配的知识。\textsuperscript{1} 我认为进行一次练习课程并同时探索一下史密斯圆图是个好主意。

\subsection{移动的故事}

让我们从查看图1开始,图1显示了使用\textbf{L}网络将负载(Z\textsubscript{LOAD})匹配到某个标称参考阻抗(Z\textsubscript{0})的各种可能性。\textsuperscript{2, 3} 这些是简化的史密斯圆图,其中未着色的一半表示使用正下方电路匹配的阻抗值集合。所示的四种网络配置中的每一种都可以设计为\textbf{变换}(匹配)未着色区域中的任何阻抗到圆图中心的阻抗。

\begin{figure}[H]
    \centering
    \includegraphics[width=0.7\linewidth]{00068.jpeg}
    \caption{L网络匹配区域}
    \label{fig:l_network_regions_cn}
\end{figure}

每个特定的起点 —— 负载,Z\textsubscript{LOAD} —— 可以通过两步变换为Z\textsubscript{0}。第一步将你带到标记为Z\textsubscript{1}的可匹配区域边界上的点。第二步将你从Z\textsubscript{1}带到圆图的参考阻抗Z\textsubscript{0},对于业余无线电系统,通常为50 \textOmega。每次移动都沿着某种曲线路径。这些曲线是什么?

图2有两部分。第一部分是标准史密斯圆图,显示黑色外圆和中心线上的阻抗坐标,以及一组红色弧线和圆。第二部分显示\textbf{导纳}的坐标,导纳是阻抗的倒数。导纳坐标与阻抗坐标共享外圆和中心线,但使用镜像的蓝色弧线和圆。

\begin{figure}[H]
    \centering
    \includegraphics[width=0.7\linewidth]{00069.jpeg}
    \caption{带有阻抗和导纳坐标的史密斯圆图}
    \label{fig:smith_chart_cn}
\end{figure}

圆图中心线上的任何点都代表纯电阻或电导(R + \textbf{j}0或G + \textbf{j}0),外圆上的任何点都代表纯电抗或电纳(0 + \textbf{j}X或0 + \textbf{j}B)。红色圆上的点都具有相同的电阻,蓝色圆上的点都具有相同的电导。红色弧上的点都具有相同的电抗,蓝色弧上的点都具有相同的电纳。\textsuperscript{4} 将这组相交的圆、弧和线视为我们的阻抗匹配棋盘。

这个棋盘如何"工作"?棋子(阻抗或导纳)通过添加串联或并联连接的电抗(称为\textbf{分流}连接)来移动(变换)。如���1的电路所示,移动只能沿着恒定电阻(或电导)圆,因为它们不添加或减去电阻(或电导)。例如,如果我添加一些正电感电抗与负载串联,添加的电抗将使负载沿恒定电阻圆顺时针移动。同样,如果我添加一些正电容电纳与负载并联,该点将沿恒定电导圆顺时针移动。与负载串联或并联添加电抗或电纳只能使"负载"在这些圆上围绕圆图移动。

如果我想将Z\textsubscript{LOAD}变换为参考阻抗Z\textsubscript{0},那么我必须沿一种圆移动一部分,直到它遇到与圆图中心相交的互补圆。然后我沿着那个新圆移动到Z\textsubscript{0}。就像国际象棋中的骑士或曼哈顿出租车一样。我必须先向一个方向移动,然后切换到另一个方向。

图3显示了每个这样的问题如何有两个解决方案。从标记为Z\textsubscript{LOAD}的点开始。它的阻抗约为0.5 – \textbf{j} 0.1 \textOmega。(记住,一切都被归一化,因此中心点,假设为50 + \textbf{j} 0 \textOmega,被表示为1.0 + \textbf{j} 0 \textOmega。)通过添加串联X\textsubscript{L}或X\textsubscript{C},我可以沿红色R = 0.5 \textOmega圆顺时针(X\textsubscript{L})或逆时针(X\textsubscript{C})移动。我需要添加足够的电抗以达到G = 1.0 S(S代表西门子,电导的单位)的蓝色圆。一旦到达那里,我可以通过添加分流L或分流C移动到Z\textsubscript{0},如注释所示。一组"移动"代表图1C电路中发生的情况,另一组代表图1D中的情况。根据Z\textsubscript{LOAD}的值,总有两种可能的电路可以让你到达Z\textsubscript{0}。

\begin{figure}[H]
    \centering
    \includegraphics[width=0.7\linewidth]{00070.jpeg}
    \caption{史密斯圆图上的两种解决方案}
    \label{fig:two_solutions_cn}
\end{figure}

既然两种选择最终都到达圆图上的同一点,为什么我会选择一种电路而不是另一种?与必须总是以2格乘1格的\textbf{L}形移动的国际象棋骑士不同,\textbf{L}网络的一对电抗可以采用广泛的值范围。其中一对将具有比另一对更实用或更方便的值(或购买更便宜)。

\subsection{练习你的骑士移动}

好的,轮到你了!你现在应该理解为什么Wetherell的L网络计算器只能为四种可能的L-C L网络中的两种解决阻抗匹配问题。\textsuperscript{5} 那些是Z\textsubscript{LOAD}在史密斯圆图可匹配区域中的两种网络配置。

你刚刚学到的另一件事:如果L网络无法匹配阻抗 —— 把它翻转过来!如果你查看彼此镜像的两个网络的可匹配区域,例如图1A和1C,你会看到它们是互补的。因此,通过翻转网络,以前无法匹配的阻抗保证在新电路配置的可匹配区域中。

让我们尝试一些示例计算,让你进入游戏并移动你的棋子。找到图1中可用于每个示例的两个电路以及所需的组件值。(答案在注释列表的末尾。)

示例1:Z\textsubscript{LOAD }是10.1 MHz全波环的馈电点阻抗:110 + \textbf{j} 50 \textOmega。

示例2:Z\textsubscript{LOAD}是八木天线在144.2 MHz的驱动元件的馈电点阻抗:22 – \textbf{j} 10 \textOmega。

示例3:Z\textsubscript{LOAD}是24.9 MHz偏置馈电偶极子的馈电点阻抗:250 – \textbf{j} 150 \textOmega。

一旦你掌握了它,你可以尝试不同版本的匹配网络,将阻抗推到极端值(高、低、高电抗),并评估你会选择构建哪个电路,这会很有趣。

\subsection{注释}

\begin{enumerate}
    \item 参见实验#21,"The L Network"和#157 "Matching Network — Design and Build"。所有以前的"实践无线电"专栏都可在ARRL成员的\href{http://www.arrl.org/hands-on-radio}{www.arrl.org/hands-on-radio}获取。
    \item 所有电抗网络及其史密斯圆图可匹配区域的完整集合可在\textbf{ARRL手册}的RF技术章节中找到。
    \item \textbf{ARRL手册}可从ARRL的\href{http://www.arrl.org/shop}{www.arrl.org/shop}和销售ARRL出版物的经销商处获取。
    \item "实践无线电"实验#59 – 61解释了史密斯圆图,最近版本的\textbf{ARRL手册}随附的CD-ROM上有额外的材料。
    \item Wetherell的计算器可在\href{http://home.sandiego.edu/~ekim/e194rfs01/jwmatcher/matcher2.html}{home.sandiego.edu/~ekim/e194rfs01/jwmatcher/matcher2.html}和其他网站上获取。
\end{enumerate}

\subsection{示例问题的答案:}

示例1:A和B;A – 1013 nH & 207 pF;B – 2515 nH & 245 pF

示例2:D和C;D – 48.9 nH和74.5 pF;C – 38.4 nH和24.9 pF

示例3:A和C;A – 770 nH & 34 pF;C – 722 nH & 53 pF

\chapter{实验 #166:优化短截线放置}

\section{英文原文}

The idea of making harmonic filters using stubs — resonant sections of transmission line — is attractive because stubs are frugal, use widely available material, and give pretty good performance. Or do they? I’ve heard more than one confused stub-builder asking why they only saw a few dB drop in the unwanted signal. They want to know what they’re doing wrong. Their stub could be just fine — the problem might stem from where the stub is attached!

\subsection{Stub Review}

Lets back up for a minute and review.\textsuperscript{1} Stubs used as harmonic filters are usually 1⁄4- or 1⁄2-wavelength long and terminated in a short- or open-circuit. They are connected in parallel with the main feed line by using a \textbf{T} adaptor, assuming coaxial cable is being used. Figure 1 shows the basic connection technique.

\begin{figure}[H]
    \centering
    \includegraphics[width=0.7\linewidth]{00071.jpeg}
    \caption{Stub Connection Technique}
    \label{fig:stub_connection}
\end{figure}

A wave encountering the junction splits, with some of the wave traveling along the stub to the short or open circuit where it is completely reflected back toward the junction. The stub length and the termination are selected so that at the frequency of the undesired harmonic, the wave returning to the junction will be out of phase with and cancel harmonic waves in the main feed line, but leave waves of the fundamental frequency unaffected.\textsuperscript{2}

That simplified view neglects one important thing — \textbf{wave impedance} in the feed line. We’re not talking about the feed line’s characteristic impedance, which is generally 50 or 75 \textOmega. Wave impedance is the ratio of voltage to current in the feed line. If the feed line is not terminated in its characteristic impedance (i.e. SWR > 1:1), the reflected wave interferes with the forward wave to create standing waves along the feed line. At some points, separated by  1⁄2 wavelength, the voltage to current ratio (wave impedance) will be a maximum. Just 1⁄4 wavelength away, the wave impedance will be a minimum.

You can see this change by using a Smith chart that shows the wave impedance in a feed line. From the point representing the termination impedance (Z\textsubscript{LOAD} in Figure 2), as you move along the line away from the termination (“Toward the Generator” on the chart) each new point on the chart shows the different wave impedance at that location in the feed line. It is important to remember that the electrical distance in wavelengths will be different for the fundamental and for each harmonic!

\begin{figure}[H]
    \centering
    \includegraphics[width=0.7\linewidth]{00072.jpeg}
    \caption{Smith Chart Showing Wave Impedance}
    \label{fig:wave_impedance}
\end{figure}

If a feed line is terminated by a short circuit (0 + \textbf{j}0 \textOmega at the left side of chart), moving toward the junction increases the wave impedance along the way. After 1⁄4 wavelength (halfway around the chart) the wave impedance reaches infinity. If the stub is long enough, wave impedance will return to zero 1⁄2 wavelength from the termination. 

Note that if the shorted stub is 1⁄4-wavelength long at the fundamental — presenting a high impedance at the junction — it will be 1⁄2-wavelength long at the harmonic and present a low impedance at the junction. The result is to short out the harmonic while leaving the fundamental unaffected. Neat trick, huh? Table 1 shows a variety of 1⁄4- and 1⁄2-wavelength stubs that can be used for a number of useful harmonic-canceling jobs. (See W2VJN’s book for a lot more!)

\begin{figure}[H]
    \centering
    \includegraphics[width=0.7\linewidth]{00073.jpeg}
    \caption{Stub Types for Harmonic Filtering}
    \label{fig:stub_types}
\end{figure}

\subsection{Why Impedance Matters}

There is one more thing to consider. The wave impedance in the main feed line determines how much effect the stub’s low impedance will have at their junction. If the main feed line’s wave impedance at the junction is high, the stub’s low impedance will effectively “short out” the harmonic, reducing its amplitude quite a bit. If the main feed line’s wave impedance at the junction is low, however, the stub’s low impedance placed across it will have comparatively little effect. 

For a monoband antenna like a single-band Yagi or ground plane matched to 50 \textOmega, the wave impedance along a 50 \textOmega feed line will be relatively constant \textbf{at the fundamental}. However, as anyone knows who has tried to transmit on 10 meters into a 20 meter monoband antenna, there is a sizable mismatch at the harmonic. This means the wave impedance for the harmonic will vary dramatically at different locations along the line and so will the effectiveness of a harmonic-canceling stub.

How big is the effect of stub placement? Both W2VJN and K9YC have modeled the harmonic attenuation of a stub at different locations in a feed line.\textsuperscript{3, 4} For the typical example of trying to cancel the 10 meter harmonic of a 20 meter signal, W2VJN’s simulation found that harmonic suppression could range from a worst-case low of 6.2 dB to a best-case maximum of 51.1 dB! (George’s model also simulated the effects of a transmitter’s output circuit, which is usually mistuned at the harmonic frequency.) So it’s quite realistic to find the performance of a filter stub very hit-or-miss just because of the varying wave impedance at different points along the feed line.

\subsection{Finding the Right Place}

How \textbf{do} you find the right place to locate a stub — or avoid the wrong place? The best way to start is by measuring the impedance of the antenna connected to the feed line — at the frequency of the harmonic you are trying suppress. Then find impedance along the feed line (again, at the harmonic) by using a modeling program like \textbf{TLW} (which comes with the \textbf{ARRL} \textbf{Antenna Book}), or \textbf{TLDetails} and \textbf{ZPlots }from AC6LA (\href{http://ac6la.com}{ac6la.com}), or \textbf{SimSmith} from AE6TY (\href{http://www.ae6ty.com/smith_charts.html}{www.ae6ty.com/smith_charts.html}).

What if you can’t measure or model the antenna system? Or you can’t install the stub where a model says you should? The most common location for a harmonic filter stub is at the output of an amplifier or transmitter. The impedance at this point is strongly influenced by the output impedance of the transmitter so you can make a good guess about whether this is a high- or low-impedance point based on the output circuit.

If the final component (closest to the output connector, not including any RF chokes) of the transmitter’s filter or tuning circuit is a series inductor (as for a Pi-L network), the impedance at the harmonic will be high. (See \href{http://k9yc.com/Coax-Stubs.pdf}{k9yc.com/Coax-Stubs.pdf} for a table listing the output circuits of many common amplifiers as well as more applications of stubs.) Attach the stub at the amplifier output or some multiple of 1⁄2-wavelengths away (at the harmonic). If the output component is a shunt capacitor (as for a Pi network), the impedance will be low so there should be 1⁄4-wavelength (or some odd multiple) of feed line between the transmitter output and the stub to place the junction at a high impedance point.

Another option is to force a high-impedance point by using a \textbf{double stub} as in Figure 3. Place the initial stub wherever you can. Then add 1⁄4 wavelength of cable (again, at the harmonic) to the main feed line and attach a second harmonic-canceling stub. The first stub creates a low-impedance point in the system and a high-impedance point  1⁄4-wavelength away. The second stub can take full advantage of the high-impedance point and provide an additional 30 dB or more of suppression.

\begin{figure}[H]
    \centering
    \includegraphics[width=0.7\linewidth]{00074.jpeg}
    \caption{Double Stub Configuration}
    \label{fig:double_stub}
\end{figure}

The key to getting the most out of your stubs is to measure their effectiveness. Don’t expect one stub placed randomly in your antenna system to solve your harmonic problems completely. Study how the stubs work, learn to make the necessary measurements, and use modeling tools like those listed here to design and install them. You will be glad you did!

\subsection{Notes}

\begin{enumerate}
    \item How transmission line stubs work and how to design the basic types were covered in “Hands-On Radio” experiments 22, 57, and 58. Experiment 81, on synchronous transformers, covers related ideas. All previous “Hands-On Radio” experiments are available to ARRL members at \href{http://arrl.org/hands-on-radio}{arrl.org/hands-on-radio}.
    \item There are many variations on this basic scheme. See the \textbf{ARRL Antenna Book} chapter on Transmission Lines (\href{http://www.arrl.org/antenna-book}{www.arrl.org/antenna-book}) for more information. Even more detail is available in W2VJN’s \textbf{Managing Interstation Interference}, available from International Radio (\href{http://www.inrad.net}{www.inrad.net}).
    \item J. Brown, K9YC, “Optimizing the Placement of Stubs for Harmonic Suppression,” \textbf{National Contest Journal}, July/August 2015,  pp 8 – 11.
    \item G. Cutsogeorge, W2VJN, “Optimizing the  Performance of Harmonic Attenuation Stubs,” \textbf{National Contest Journal}, Jan/Feb 2015, pp 3 – 4.
\end{enumerate}

\section{中文翻译}

使用短截线(传输线的谐振段)制作谐波滤波器的想法很有吸引力,因为短截线节省材料,使用广泛可用的材料,并且性能相当好。或者说它们真的好吗?我不止一次听到困惑的短截线构建者问为什么他们只看到不需要的信号下降了几dB。他们想知道他们哪里做错了。他们的短截线可能没问题 —— 问题可能出在短截线的连接位置!

\subsection{短截线回顾}

让我们后退一步并回顾一下。\textsuperscript{1} 用作谐波滤波器的短截线通常是1⁄4或1⁄2波长长,端接短路或开路。假设使用同轴电缆,它们通过使用\textbf{T}适配器与主馈线并联连接。图1显示了基本连接技术。

\begin{figure}[H]
    \centering
    \includegraphics[width=0.7\linewidth]{00071.jpeg}
    \caption{短截线连接技术}
    \label{fig:stub_connection_cn}
\end{figure}

遇到结的波会分裂,部分波沿着短截线传播到短路或开路,在那里它被完全反射回结。选择短截线长度和端接,使得在不需要的谐波频率下,返回结的波将与主馈线中的谐波波异相并抵消它们,但不影响基频波。\textsuperscript{2}

这个简化的观点忽略了一个重要的因素 —— 馈线中的\textbf{波阻抗}。我们不是在谈论馈线的特性阻抗,通常为50或75 \textOmega。波阻抗是馈线中电压与电流的比率。如果馈线没有以其特性阻抗端接(即SWR > 1:1),反射波会干扰正向波,在馈线沿线产生驻波。在间隔1⁄2波长的某些点,电压与电流的比率(波阻抗)将达到最大值。仅1⁄4波长远的地方,波阻抗将达到最小值。

你可以通过使用显示馈线中波阻抗的史密斯圆图来看到这种变化。从代表端接阻抗的点(图2中的Z\textsubscript{LOAD}),当你沿着线路远离端接移动时(圆图上的“朝向发电机”),圆图上的每个新点都显示了馈线中该位置的不同波阻抗。重要的是要记住,基频和每个谐波的波长电距离将不同!

\begin{figure}[H]
    \centering
    \includegraphics[width=0.7\linewidth]{00072.jpeg}
    \caption{显示波阻抗的史密斯圆图}
    \label{fig:wave_impedance_cn}
\end{figure}

如果馈线由短路端接(圆图左侧的0 + \textbf{j}0 \textOmega),向结移动会沿途增加波阻抗。经过1⁄4波长(圆图的一半)后,波阻抗达到无穷大。如果短截线足够长,波阻抗将在离端接1⁄2波长处返回零。

请注意,如果短路短截线在基频时为1⁄4波长长 —— 在结处呈现高阻抗 —— 它在谐波时将为1⁄2波长长,在结处呈现低阻抗。结果是短路谐波,同时不影响基频。巧妙的技巧,不是吗?表1显示了各种1⁄4和1⁄2波长短截线,可用于许多有用的谐波抵消工作。(有关更多信息,请参阅W2VJN的书!)

\begin{figure}[H]
    \centering
    \includegraphics[width=0.7\linewidth]{00073.jpeg}
    \caption{用于谐波滤波的短截线类型}
    \label{fig:stub_types_cn}
\end{figure}

\subsection{为什么阻抗很重要}

还有一件事需要考虑。主馈线中的波阻抗决定了短截线的低阻抗在其结处的影响程度。如果结处主馈线的波阻抗高,短截线的低阻抗将有效地“短路”谐波,相当大地降低其幅度。然而,如果结处主馈线的波阻抗低,短截线的低阻抗跨接在其上的效果将相对较小。

对于匹配到50 \textOmega的单频段天线(如单频段八木天线或接地平面天线),50 \textOmega馈线上的波阻抗在\textbf{基频}时将相对恒定。然而,正如任何尝试在10米波段向20米单频段天线发射的人所知道的那样,谐波处存在相当大的失配。这意味着谐波的波阻抗在沿线路的不同位置会有很大变化,谐波抵消短截线的有效性也会如此。

短截线放置的影响有多大?W2VJN和K9YC都模拟了短截线在馈线不同位置的谐波衰减。\textsuperscript{3, 4} 对于尝试抵消20米信号的10米谐波的典型示例,W2VJN的模拟发现谐波抑制范围从最坏情况的6.2 dB到最佳情况的51.1 dB!(George的模型还模拟了发射机输出电路的影响,该电路通常在谐波频率下失谐。)因此,仅仅因为沿馈线不同点的波阻抗变化,发现滤波短截线的性能非常不稳定是相当现实的。

\subsection{找到正确的位置}

你如何找到放置短截线的正确位置 —— 或避免错误的位置?最好的开始方法是测量连接到馈线的天线的阻抗 —— 在你试图抑制的谐波频率下。然后通过使用建模程序如\textbf{TLW}(随\textbf{ARRL} \textbf{天线手册}提供),或AC6LA的\textbf{TLDetails}和\textbf{ZPlots}(\href{http://ac6la.com}{ac6la.com}),或AE6TY的\textbf{SimSmith}(\href{http://www.ae6ty.com/smith_charts.html}{www.ae6ty.com/smith_charts.html})来找到沿馈线的阻抗(同样在谐波下)。

如果你无法测量或建模天线系统怎么办?或者你无法在模型说你应该的地方安装短截线怎么办?谐波滤波短截线最常见的位置是在放大器或发射机的输出处。这一点的阻抗受到发射机输出阻抗的强烈影响,因此你可以根据输出电路猜测这是高阻抗点还是低阻抗点。

如果发射机滤波器或调谐电路的最终组件(最靠近输出连接器,不包括任何RF扼流圈)是串联电感器(如Pi-L网络),谐波处的阻抗将很高。(有关列出许多常见放大器输出电路以及短截线更多应用的表格,请参阅\href{http://k9yc.com/Coax-Stubs.pdf}{k9yc.com/Coax-Stubs.pdf}。)将短截线连接在放大器输出或(在谐波下)远离1⁄2波长的倍数处。如果输出组件是并联电容器(如Pi网络),阻抗将很低,因此在发射机输出和短截线之间应该有1⁄4波长(或某些奇数倍数)的馈线,以将结放置在高阻抗点。

另一种选择是通过使用如图3所示的\textbf{双短截线}来强制高阻抗点。将初始短截线放在你可以的任何地方。然后在主馈线上添加1⁄4波长的电缆(同样在谐波下)并连接第二个谐波抵消短截线。第一个短截线在系统中创建一个低阻抗点和一个1⁄4波长远的高阻抗点。第二个短截线可以充分利用高阻抗点并提供额外30 dB或更多的抑制。

\begin{figure}[H]
    \centering
    \includegraphics[width=0.7\linewidth]{00074.jpeg}
    \caption{双短截线配置}
    \label{fig:double_stub_cn}
\end{figure}

充分利用短截线的关键是测量它们的有效性。不要期望在天线系统中随机放置一个短截线就能完全解决你的谐波问题。研究短截线如何工作,学习进行必要的测量,并使用这里列出的建模工具来设计和安装它们。你会很高兴你这样做了!

\subsection{注释}

\begin{enumerate}
    \item 传输线短截线的工作原理和基本类型的设计在“Hands-On Radio”实验22、57和58中介绍。关于同步变压器的实验81涵盖了相关思想。所有以前的“Hands-On Radio”实验都可在ARRL成员的\href{http://arrl.org/hands-on-radio}{arrl.org/hands-on-radio}获取。
    \item 这个基本方案有许多变体。有关更多信息,请参阅\textbf{ARRL天线手册}中关于传输线的章节(\href{http://www.arrl.org/antenna-book}{www.arrl.org/antenna-book})。W2VJN的\textbf{管理台站间干扰}中提供了更详细的信息,可从International Radio(\href{http://www.inrad.net}{www.inrad.net})获取。
    \item J. Brown, K9YC, “Optimizing the Placement of Stubs for Harmonic Suppression,” \textbf{National Contest Journal}, July/August 2015,  pp 8 – 11.
    \item G. Cutsogeorge, W2VJN, “Optimizing the  Performance of Harmonic Attenuation Stubs,” \textbf{National Contest Journal}, Jan/Feb 2015, pp 3 – 4.
\end{enumerate}

\chapter{实验 #177:馈线比较}

\section{英文原文}

“Feed Lines, Decibels, and Dollars” was the title of Steve Ford’s, WB8IMY, article in the August 2017 issue of \textbf{QST} on choosing a feed line — but you have to have the data! His column reminded me of some important graphs and tables in \textbf{The} \textbf{ARRL Handbook} and \textbf{Antenna Book} that can help you make those decisions.\textsuperscript{1}

I don’t know when you last purchased some new coax, but it’s sold by the foot and can cost an arm and a leg! We have limited resources to build our stations, so how do we balance feed line dollars against radio, antenna, computer, and gadget dollars? It’s not enough to say, “Just buy the best you can afford,” and hope you didn’t overspend.

\subsection{Matched Loss and  Non-Matched Loss}

Let’s start by revisiting the term \textbf{matched loss}. This is the loss when the feed line is terminated in its characteristic impedance and standing-wave ratio (SWR) is 1:1, such as 50 \textOmega cable attached to a 50 \textOmega load. Matched loss is specified in dB/100 feet by most North American manufacturers, and you can get it from their websites or from tables in the ARRL reference books.

In the real world of antennas, SWR is almost always greater than 1:1. That mismatch causes some of the power to be reflected back and forth in the line until it is either radiated as a signal or turned into heat. Figuring out this extra \textbf{mismatch loss} can be a chore, but it has been simplified by Joseph Reisert, W1JR, in Figure 1.\textsuperscript{2}

\begin{figure}[H]
    \centering
    \includegraphics[width=0.7\linewidth]{00075.jpeg}
    \caption{Mismatch Loss vs SWR}
    \label{fig:mismatch_loss}
\end{figure}

Start with the matched loss for your length of feed line. Let’s say the feed line has 2.0 dB of matched loss and the SWR at the antenna is 3:1. From 3:1 on the \textbf{x}-axis, go vertically to the 2.0 dB curve. Then turn left and intersect the \textbf{y}-axis at 2.8 dB. The 3:1 SWR “costs” another 2.8 – 2.0 = 0.8 dB of loss. This scenario isn’t very dramatic, but you can see from the graph that higher SWR and matched loss can eat up a lot of your transmitter’s output pretty quickly.

\subsection{How Much You Can Tolerate}

Let’s use the graph in a different way to answer the question, “What is the feed line with the highest matched loss I can accept?” This is a very common question when you are designing an antenna system. Perhaps you set a goal of not having more than 6 dB (one S-unit) of feed line loss. If you know the maximum SWR of your antenna, you have enough information to find the answer. Let’s say you have a non-resonant doublet that has one difficult band for which the SWR is 5:1.

Start on the \textbf{y-}axis of Figure 1 at 6 dB and follow that horizontal line across the chart. Now go to the \textbf{x-}axis and the vertical line for an SWR of 5:1. Where those two lines cross, interpolate between the curves to get a loss of about 4 dB (the curves are spaced logarithmically, so you have to take that into account). That’s how much matched loss you can tolerate.

Note that for a given maximum amount of allowable loss, as SWR increases the maximum matched loss goes \textbf{down}. That’s a little counter-intuitive until you realize that for higher SWR, you need cable that is less lossy to keep total loss below your maximum loss.

There is one more missing piece of the puzzle, and that is the length of the line between your transmitter and the antenna. Let’s say your feed line is 400 feet long — not too unusual for a Field Day or expedition-type setup. You need cable with a matched loss of 4 dB (or less) at the frequency of your “problem” band — let’s say that’s 15 meters, or 21.2 MHz. 

You can start browsing through websites, but you’ll find that the loss data is typically only specified at 1, 10, 100, and 1,000 MHz. This makes the process harder than it needs to be, so Table 1 was developed by W3LPL using a feed line loss calculator created by VK1OD.\textsuperscript{3} It shows the “feet per dB” for a wide variety of coaxial cables. The lossier the cable (toward the bottom of the table), the fewer feet required to incur 1 dB of loss.

\begin{figure}[H]
    \centering
    \includegraphics[width=0.7\linewidth]{00076.jpeg}
    \caption{Coaxial Cable Feet per dB Loss}
    \label{fig:cable_feet_per_db}
\end{figure}

In the case of our 15-meter problem, find the column labeled “21.2.” Our matched loss budget is 4 dB and the feed line length is 400 feet, so we can accept any cable with more than  400 / 4 = 100 feet/dB of loss. If we begin at the bottom, we see that RG-174, RG-58, and RG-8X are too lossy — it doesn’t take enough cable before 1 dB of loss at 21.2 MHz has been created. RG-213 shows a loss of 111 feet per dB, so a 400-foot length will result in 400 / 111 = 3.6 dB. You could spend the extra money on a cable with less loss, but you don’t have to!

How much RG-213 could you use and still make that 4.0 dB loss budget? You could use 111 feet/dB × 4 dB = 444 feet. If you wanted to make things a little simpler and just use 500-foot spools without having to divide the cable, select a type for which it takes \textbf{more} than 500 / 4 = 125 feet to create 1 dB of loss. Continuing our journey up the list, LMR-400 is the first cable to make the cut, so to speak.

This is a very handy table. If you have a copy of the \textbf{Antenna Book} (23rd edition), take a look at Chapter 23’s section on “Choosing and Installing Feed Lines.” There you’ll find two other tables that show how long a cable would have to be before you would gain 1 dB by replacing it with 7⁄8-inch or 1⁄2-inch Heliax hardline.

\subsection{Does It Really Matter?}

Maybe not so much at 160 meters, but as Steve Ford’s article points out, it certainly can make a difference, particularly above HF. VHF/UHF and microwave operations are incredibly sensitive to feed line loss, because the cable losses increase quickly with frequency. At 1,296 MHz, a popular EME frequency, it doesn’t take much of even the best flexible cable before replacing it with hardline begins to pay off. And signals converted to heat can \textbf{never} be recovered, not even with the best antennas or preamps. Sometimes, that expensive cable is the best bargain around.

\subsection{Coaxial Cable Types}

Hams use the familiar RG- numbers for coaxial cable like RG-8, RG-58, and RG-213. However, these designators no longer mean that the cable meets any particular specification. It is simply a matter of convenience that vendors still use them and there are wide variations in loss, velocity factor, and quality between cables with the same  RG- designator. For example, the Mouser Electronics website (\href{http://www.mouser.com}{www.mouser.com}) lists 72 different cables as “RG-58.” Pay less attention to the RG-type designator and more to the manufacturer’s part number, such as Belden 8267 or Alpha 9213, which are cables meeting the usual specifications for what we call “RG-213” coax. Look for the manufacturer and part number printed on the cable. If it’s not there, you can’t be sure what you’re getting.

\subsection{Notes}

\begin{enumerate}
    \item The \textbf{ARRL Handbook} (2017 edition, Item 0628) and \textbf{Antenna Book} (23rd edition, Item 0444) are available from the ARRL Bookstore at \href{http://www.arrl.org/shop}{www.arrl.org/shop}.
    \item J. Reisert, W1JR, “VHF/UHF World,” \textbf{Ham Radio}, Oct. 1987, pp. 27 – 38.
    \item \textbf{Antenna Book}, 23rd edition, Table 23.4.
\end{enumerate}

\section{中文翻译}

“馈线、分贝和美元”是Steve Ford,WB8IMY,在2017年8月出版的\textbf{QST}杂志上关于选择馈线的文章标题 —— 但你必须有数据!他的专栏让我想起了\textbf{ARRL手册}和\textbf{天线手册}中的一些重要图表和表格,可以帮助你做出这些决定。\textsuperscript{1}

我不知道你最后一次购买新同轴电缆是什么时候,但它是按英尺出售的,可能要花一大笔钱!我们构建电台的资源有限,所以我们如何平衡馈线费用与收音机、天线、计算机和小工具的费用?仅仅说“买你能负担得起的最好的”并希望你没有超支是不够的。

\subsection{匹配损耗和非匹配损耗}

让我们从重新审视\textbf{匹配损耗}这个术语开始。这是馈线以其特性阻抗端接且驻波比(SWR)为1:1时的损耗,例如连接到50 \textOmega负载的50 \textOmega电缆。大多数北美制造商以dB/100英尺指定匹配损耗,你可以从他们的网站或ARRL参考书中的表格中获取。

在天线的现实世界中,SWR几乎总是大于1:1。这种失配导致一些功率在线路中来回反射,直到它作为信号辐射或转化为热量。计算这种额外的\textbf{失配损耗}可能是一项繁琐的工作,但Joseph Reisert,W1JR,在图1中简化了这一点。\textsuperscript{2}

\begin{figure}[H]
    \centering
    \includegraphics[width=0.7\linewidth]{00075.jpeg}
    \caption{失配损耗与SWR关系}
    \label{fig:mismatch_loss_cn}
\end{figure}

从你的馈线长度的匹配损耗开始。假设馈线有2.0 dB的匹配损耗,天线处的SWR为3:1。从x轴上的3:1开始,垂直向上到2.0 dB曲线。然后向左转并在y轴上相交于2.8 dB。3:1的SWR“成本”为另一个2.8 – 2.0 = 0.8 dB的损耗。这种情况不是很戏剧性,但你可以从图表中看到,更高的SWR和匹配损耗会很快消耗掉发射机的大量输出。

\subsection{你能容忍多少}

让我们以不同的方式使用图表来回答这个问题:“我能接受的最高匹配损耗的馈线是什么?”这是设计天线系统时一个非常常见的问题。也许你设定了一个目标,即馈线损耗不超过6 dB(一个S单位)。如果你知道天线的最大SWR,你有足够的信息来找到答案。假设你有一个非谐振偶极子,它有一个困难的波段,其SWR为5:1。

从图1的y轴上的6 dB开始,沿着那条水平线穿过图表。现在转到x轴和SWR为5:1的垂直线。在这两条线交叉的地方,在曲线之间插值以获得约4 dB的损耗(曲线按对数间隔,所以你必须考虑到这一点)。这就是你能容忍的匹配损耗量。

请注意,对于给定的最大允许损耗量,随着SWR的增加,最大匹配损耗会\textbf{下降}。这有点违反直觉,直到你意识到对于更高的SWR,你需要损耗更小的电缆来保持总损耗低于你的最大损耗。

还有一个缺失的拼图,那就是发射机和天线之间线路的长度。假设你的馈线长400英尺 —— 对于Field Day或远征型设置来说并不太罕见。你需要在“问题”波段频率下具有4 dB(或更少)匹配损耗的电缆 —— 假设是15米,或21.2 MHz。

你可以开始浏览网站,但你会发现损耗数据通常只在1、10、100和1,000 MHz下指定。这使得过程比需要的更困难,所以表1是由W3LPL使用VK1OD创建的馈线损耗计算器开发的。\textsuperscript{3} 它显示了各种同轴电缆的“每dB英尺数”。电缆越损耗(朝向表格底部),产生1 dB损耗所需的英尺数越少。

\begin{figure}[H]
    \centering
    \includegraphics[width=0.7\linewidth]{00076.jpeg}
    \caption{同轴电缆每dB损耗英尺数}
    \label{fig:cable_feet_per_db_cn}
\end{figure}

在我们15米问题的情况下,找到标记为“21.2”的列。我们的匹配损耗预算为4 dB,馈线长度为400英尺,所以我们可以接受任何每dB损耗超过400 / 4 = 100英尺的电缆。如果我们从底部开始,我们看到RG-174、RG-58和RG-8X太损耗了 —— 在21.2 MHz下产生1 dB损耗不需要足够的电缆。RG-213显示每dB损耗为111英尺,所以400英尺长度将导致400 / 111 = 3.6 dB。你可以花额外的钱购买损耗更小的电缆,但你不必这样做!

你可以使用多少RG-213并仍然保持4.0 dB的损耗预算?你可以使用111英尺/dB × 4 dB = 444英尺。如果你想让事情更简单,只是使用500英尺的线轴而不必分割电缆,选择一种需要\textbf{超过}500 / 4 = 125英尺来创建1 dB损耗的类型。继续向上查看列表,LMR-400是第一个符合要求的电缆,可以这么说。

这是一个非常方便的表格。如果你有\textbf{天线手册}(第23版)的副本,请看第23章的“选择和安装馈线”部分。在那里你会找到另外两个表格,显示电缆需要多长时间才能通过用7⁄8英寸或1⁄2英寸Heliax硬线替换它来获得1 dB的增益。

\subsection{真的重要吗?}

在160米可能不那么重要,但正如Steve Ford的文章指出的那样,它肯定会产生差异,特别是在HF以上。VHF/UHF和微波操作对馈线损耗非常敏感,因为电缆损耗随频率快速增加。在1,296 MHz,一个流行的EME频率,即使是最好的柔性电缆也不需要太多,用硬线替换它就开始有回报了。而转化为热量的信号\textbf{永远}无法恢复,即使是最好的天线或前置放大器也不行。有时,那条昂贵的电缆是周围最好的交易。

\subsection{同轴电缆类型}

业余无线电爱好者使用熟悉的RG-编号来表示同轴电缆,如RG-8、RG-58和RG-213。然而,这些标识符不再意味着电缆符合任何特定规范。供应商仍然使用它们只是为了方便,并且具有相同RG-标识符的电缆在损耗、速度因子和质量方面存在很大差异。例如,Mouser Electronics网站(\href{http://www.mouser.com}{www.mouser.com})列出了72种不同的电缆作为“RG-58”。较少关注RG类型标识符,更多关注制造商的部件号,如Belden 8267或Alpha 9213,这些是符合我们称为“RG-213”同轴电缆通常规范的电缆。寻找印在电缆上的制造商和部件号。如果不在那里,你无法确定你得到的是什么。

\subsection{注释}

\begin{enumerate}
    \item \textbf{ARRL手册}(2017版,项目0628)和\textbf{天线手册}(第23版,项目0444)可从ARRL书店的\href{http://www.arrl.org/shop}{www.arrl.org/shop}获取。
    \item J. Reisert, W1JR, “VHF/UHF World,” \textbf{Ham Radio}, Oct. 1987, pp. 27 – 38.
    \item \textbf{天线手册},第23版,表23.4。
\end{enumerate}

\part{电子基础}
\chapter{实验 #126, 127, 128:相量,第 1、2、3 部分}

\section{英文原文}

\subsection{实验 #126:相量,第 1 部分}

Last month, we introduced the phasor — a way of representing a sinusoid in terms of its amplitude and some value of phase compared to a reference. Phasor notation looks like V∠ φ where V is the amplitude and φ is the phase. Let’s learn a few more things about phasors.

\subsubsection{Basic Phasor Math}

One of the nice things about phasors is that multiplying them is pretty easy. Multiplying phasor A by phasor B requires you to multiply the magnitudes and add the angles:

V\textsubscript{A}∠φ\textsubscript{A} × V\textsubscript{B}∠φ\textsubscript{B} = V\textsubscript{A}V\textsubscript{B}∠( φ\textsubscript{A} + φ\textsubscript{B})

Similarly, to divide phasors, divide the magnitudes and subtract one angle from the other.

V\textsubscript{A}∠φ\textsubscript{A} ÷ V\textsubscript{B}∠φ\textsubscript{B} = (V\textsubscript{A} / V\textsubscript{B})∠(φ\textsubscript{A} - φ\textsubscript{B})

Remember that to use phasor notation this way requires both signals to have exactly the same frequency so that φ\textsubscript{A} and φ\textsubscript{B} are constant. If that isn’t true, the math gets a lot fancier.

How about adding phasors? Not quite as easy. Because we are operating in polar notation, you must break down each phasor into its X axis and Y axis components, add those components together and then change them back to phasors:

V\textsubscript{C}∠φ\textsubscript{C} = V\textsubscript{A}∠φ\textsubscript{A} + V\textsubscript{B}∠φ\textsubscript{B}

X axis component =   [V\textsubscript{A} cos φ\textsubscript{A} + V\textsubscript{B} cos φ\textsubscript{B}]

Y axis component = [V\textsubscript{A} sin φ\textsubscript{A} + V\textsubscript{B} sin φ\textsubscript{B}]

V\textsubscript{C}∠φ\textsubscript{C} = X + \textbf{j} Y = √(X² + Y²) ∠ (tan⁻¹ Y/X)

\textbf{Bleh!}

Fortunately, scientific calculators and software usually have routines to do this math automatically — look in the manual or Help file under \textbf{polar notation}. (Remember that online tutorials for this kind of math are listed on the ARRL website.\textsuperscript{1})

\subsubsection{Graphically Adding and Subtracting}

We have been drawing all of the phasors with their head at the origin and their tail (where the arrowhead is) at the point representing the magnitude and angle. Phasors can be drawn anywhere on the X-Y plane, though, as long as they have the same magnitude and angle! This makes adding them together graphically very simple, as shown in Figure 1A, by arranging the phasors “head to tail.” The resulting phasor is drawn from the head of the first phasor to the tail of the last phasor. 

\begin{figure}[H]
    \centering
    \includegraphics[width=0.7\linewidth]{00080.jpeg}
    \caption{Graphical Addition and Subtraction of Phasors}
    \label{fig:phasor_addition}
\end{figure}

Just like ordinary numbers, you can add phasors together in any order. What about subtraction? Turn the phasor to be subtracted 180° and add as in Figure 1B — just like subtracting an ordinary number by multiplying it by -1 and adding instead. Now you know how to add, subtract, multiply, and divide phasors all having a common frequency. 

Let’s learn another neat trick — if the phasors represent voltages, how do you find the difference in voltage between two phasors? When you measure voltage at a point in a circuit, you measure voltage “from” ground “to” the point. In effect you are measuring the voltage at the point and then subtracting the voltage at your ground reference, which is zero. If our phasor ground reference is at the origin as in Figure 2, the tail of the phasor (with the arrowhead) shows the voltage measurement with respect to ground.

\begin{figure}[H]
    \centering
    \includegraphics[width=0.7\linewidth]{00081.jpeg}
    \caption{Voltage Between Phasors}
    \label{fig:voltage_between_phasors}
\end{figure}

When you measure voltage between two ungrounded points in a circuit, your meter’s negative probe is the reference and you measure voltage “from” the reference point “to” the point where the positive probe is. Figure 2 shows how this works if the two voltages are phasors and our reference “ground” point is at the origin. The voltage “from” phasor A “to” phasor B is itself a phasor, written V\textsubscript{AB} and calculated as V\textsubscript{B} – V\textsubscript{A}. We could also measure the voltage from phasor B to phasor A as V\textsubscript{BA} = V\textsubscript{A} – V\textsubscript{B}. You can see that V\textsubscript{BA} has exactly the same magnitude but the opposite angle to V\textsubscript{AB}. Take a minute and sketch out the subtraction of the phasors to make sure you see how I came up with V\textsubscript{AB} and V\textsubscript{BA}.

\subsubsection{Phasor-to-Phasor Voltages}

This is all fine and dandy, but does it have any practical value? Would you ever encounter phasor-to-phasor voltages? Yes and closer to home than you imagined. Residential ac power electrical service supplies two phases to the main breaker box, each 120 V. The power comes from a transformer at the utility pole with a single primary winding and two secondary windings. Figure 3 shows the secondary windings each supplying one phase of your electrical service and connected together at one end as the neutral. The polarity of the windings is opposite so that the phasors representing their voltages point in opposite directions as shown in the phasor diagram. This is called \textbf{split-phase} power.

\begin{figure}[H]
    \centering
    \includegraphics[width=0.7\linewidth]{00082.jpeg}
    \caption{Split-Phase Power}
    \label{fig:split_phase_power}
\end{figure}

If you have two equal-and-opposite phasors, what is the magnitude of the voltage between them? (Answer: The sum of the phasor magnitudes.) If each phasor has a magnitude of 120 V, the magnitude of the voltage between them is 120 + 120 = 240 V. If you connect one hot wire to each phase and one to neutral, that’s where the ac for your amplifier (or your clothes dryer) comes from!

\subsubsection{Three-phase Power}

Now let’s take this one step further — three-phase power.\textsuperscript{2} The ac coming from generating facilities like dams and power plants has three phases. That’s why there are three wires (or pairs of wires) making up the high-voltage lines (not counting any protective ground wires). Large power consumers would unbalance the power grid if they used power from just one of the phases, so they are wired to use some power from each of the phases and the electricians are in charge of configuring things so each phase is loaded by about the same amount. That is why buildings and businesses of any size have ac service with three phases, not just two.

The phasors representing each of the three phases — A, B, and C — are shown in  Figure 4. They are all spaced equally around the circle, 1⁄3 of the circumference or 120° apart. Let’s say your apartment in a big building is supplied with two phases of power, just like residential split-phase ac power, and each phasor has an amplitude of 120 V. What happens when you try to run the drier by connecting it to the two phases (let’s say phase A and B)? Why don’t you get 240 V?

\begin{figure}[H]
    \centering
    \includegraphics[width=0.7\linewidth]{00083.jpeg}
    \caption{Three-Phase Power}
    \label{fig:three_phase_power}
\end{figure}

Look at Figure 4 and the phasor V\textsubscript{BA}. The two phasors representing the phases of electrical service, V\textsubscript{A} and V\textsubscript{B}, are not pointing in opposite directions — they are only 120° apart — so their magnitudes don’t add as in the split-phase situation. In fact, if you look up the trigonometry, the magnitude of phasor V\textsubscript{BA} = √3 V\textsubscript{A} = 1.732 V\textsubscript{A}, not 2 V\textsubscript{A}. If each phase is supplying 120 V, what voltage will your dryer see if it is connected across two phases? (120 × 1.732 = 208 V)

This dependence on how your electrical service is derived from the utility grid makes a big difference when running a heavy load — such as an amplifier. If your amplifier is designed to run from 240 V power and you connect it to 224 V instead, that is about 7% low. It’s common for amplifiers not to supply their full rated output power when run at slightly lower input voltage. The opposite case — higher than expected input voltage — can stress high-voltage components, too. 

If your equipment does not have a “universal” power rating of something like 90 to 260 V ac, determine how to configure it for the voltage you have available. Many appliances and amplifiers have selectable input voltage “taps” or connections on the primary winding of a power transformer that can accommodate 240, 220 or 208 V power. (Where does 208 come from? Two hundred and eight is approximately 1.732 × 120 V, the usual voltage for home ac service.)

I said we’d have a two-part article, but it will take one more to get to some real radio meat-and-potatoes: AM and PM modulation from the perspective of phasors. That, in turn will usher you to the gates of modern data communication: I-Q modulation.

\subsection{实验 #127:相量,第 2 部分}

No, not the kind of phaser you set on stun, silly! If you passed your General exam, you learned about phase and a little bit about angular frequency. Amateur Extra class licensees (and those studying for the Extra) have even used phasor notation, although it was called by another name. 

In this two-part column, we’ll start by developing basic concepts to show what a phasor is and how it relates to things you already understand. Then we’ll progress to examples of using phasors to describe electrical and radio phenomena, such as modulation. As ham radio begins to use more advanced types of modulation, understanding phasors will provide an important bridge between the familiar AM/SSB modulation and the future.

\subsubsection{The Sinusoid}

Like many meals in radio, this dish begins with a sine wave and is seasoned with complex numbers. (You can find these subjects in the math tutorials for the General and Extra exams listed on the ARRL website.\textsuperscript{3}) The sine wave (or \textbf{sinusoid}) looks like a regularly increasing and decreasing wave but as Figure 1 shows, it is really related to rotation.

\begin{figure}[H]
    \centering
    \includegraphics[width=0.7\linewidth]{00077.jpeg}
    \caption{Sinusoid and Rotation}
    \label{fig:sinusoid_rotation}
\end{figure}

Imagine a point revolving counter-clockwise in a circle around the origin of the complex plane as at the left of Figure 1. If the point is one unit from the origin the coordinates for each location visited by the point are [cosθ, \textbf{i} sinθ], where θ (the Greek letter theta) is the angle from the positive x axis to the line from origin to the point. This circle, not surprisingly, is called the \textbf{unit circle} because the value of its radius is 1 or unity.

As the point revolves around the origin,  θ steadily increases from 0 to 360° and begins again at 0° with each cycle. (Counterclock-wise is considered the positive direction.) If the point always moves at the same speed, the frequency of the cycles around the origin, f, does not change and the point moves 360 × f degrees every second. That means the number of degrees a point has moved in t seconds, θ = 360 × f × t. 

There are 2π radians (another unit of angular measurement) in a circle so θ = 2π × f × t. The quantity 2πf is referred to as angular frequency, ω, and you see it used in the formulas for reactance and many other electrical calculations that depend on frequency. 

Tying it all back together, the coordinates for the position of the point for every point in time as it moves around the circle are [cos(2πft), \textbf{i} sin(2πft)] and the graph at the right of Figure 1 plots the point’s y (or imaginary) coordinate versus time, creating a sine wave. If we plotted the point’s x (or real) coordinate versus time, it would create a cosine wave.

Making the leap from a point moving around the circle to something more electrical, the magnitude of the point’s imaginary coordinate, sin(2πft) = sin(ωt), can also be a voltage or current or field strength. In fact, the familiar sine wave of ac power comes from the rotational motion of a generator’s field coil. As the coil passes through the magnetic field in the generator, the angle between the coil and the field changes in the same way as our point moves around the origin. This changing relationship between the coil and the field creates a sinusoidal voltage in the coil.

While both sine and cosine waves are generally referred to as sinusoids, they differ from each other in an important way. Starting from t = 0, the cosine wave starts at a value of 1 + 0\textbf{i} and the sine wave at 0 + \textbf{i}. Other than starting at different values, the waves are identical. The cosine wave describes the real coordinate and the sine wave describes the imaginary coordinate. 

You can also think of that difference in starting value as a difference in angle, in which the sine wave is π/2 radians (90°) ahead of the cosine wave. This difference never changes because both waves are describing the same thing — constant rotation. The position of a particular point on the wave is its phase and the amount of the difference is called the \textbf{phase angle}, 90° in this case.

This is where the following trigonometric identities come from: sinθ = cos(θ – 90°) = cos(θ – π/2) and cosθ = sin(θ + 90°) = sin  (θ + π/2). Many, many more such relationships between sine and cosine waves become obvious (or at least more understandable) when viewed from the standpoint of rotation and the unit circle.

\subsubsection{Polar Notation}

So far we have used the \textbf{rectangular} form for the coordinates of our moving point: x + \textbf{i} y. In most engineering technical literature, the letter \textbf{j} is used instead of \textbf{i} to avoid confusion with current and from here on in the column, we’ll do so, as well.

Next is the form that you may have already learned (or will learn!) for your Extra class exam and that is the polar form in which the coordinates take the form of a radius and an angle: r ∠θ. Polar form is read “r at (an angle of) theta.” Using polar form coordinates for a point on the unit circle is easy because they’re always 1∠θ. If you are describing the point’s position as it whirls around the circle, you can use the equation for angular frequency we figured out earlier and the coordinates become 1∠(2πft) = 1∠(ωt). So this particular method is a good shorthand way of describing what the moving point is doing.

\subsubsection{Introducing the Phasor}

When dealing with RF signals and circuitry, it’s often true that the frequency of the signals doesn’t change. Think of an RC low-pass filter, for example: the input signal V\textsubscript{IN} sin (ωt + 0) and output signal V\textsubscript{OUT} sin (ωt + ϕ) have the same frequency, even though their amplitudes are different by the ratio of  V\textsubscript{OUT}/V\textsubscript{IN} and they are offset in phase by ϕ. 

Assuming the same frequency for both signals, our polar form can now be simplified to V∠ ϕ where ϕ is just the phase angle between a signal and some reference signal or phase. The input signal to a circuit is usually the reference for measuring phase differences.

Hey, guess what? V∠ ϕ is a phasor! A phasor is just a complex number that represents the amplitude and phase of a sinusoid and the V∠θ polar notation is just convenient mathematical shorthand. Phasors are a type of vector – quantities that have both a magnitude and a direction. In the case of V∠ ϕ, the magnitude is |V| and the direction is the phase angle, ϕ, so the more cumbersome name “phase vector” was shortened to “phasor.” (As a vector, a phasor is often shortened even further and written as a single bold letter, such as \textbf{V} or \textbf{I}.)

\begin{figure}[H]
    \centering
    \includegraphics[width=0.7\linewidth]{00078.jpeg}
    \caption{Bicycle Wheel Demonstration}
    \label{fig:bicycle_wheel}
\end{figure}

\begin{figure}[H]
    \centering
    \includegraphics[width=0.7\linewidth]{00079.jpeg}
    \caption{Phasor Diagram}
    \label{fig:phasor_diagram}
\end{figure}

If our original sine wave is the reference signal, the phasor describing the sine wave is \textbf{V}∠0 and the cosine wave is \textbf{V}∠–90° or  \textbf{V}∠–π/2. Remember, the frequency is assumed to be the same for both signals, whether 60 Hz from the power grid or 14.200 MHz on 20 meters. Figure 3 shows a \textbf{phasor diagram} for the signals at the input and output of a filter.

There is a final way to describe the signal — the \textbf{exponential form} in which it is represented as V e\textsuperscript{jθ}. This form comes from the mathematics behind Euler’s equation\textsuperscript{4} in which the coordinates of our point are miraculously shown to be equivalent to e\textsuperscript{jθ} = cos θ + \textbf{j} sin θ. This form comes from the mathematics behind Euler’s equation in which the coordinates of our point are miraculously shown to be equivalent to e\textsuperscript{jθ} = cos θ + \textbf{j} sin θ. The serious and beautiful math\textsuperscript{5} behind this equation lies at the heart of much of electrical engineering and leads to the jaw-dropping Euler’s identity: e\textsuperscript{jπ} = –1 which unites the two most widely used transcendental numbers (e and π), imaginary numbers (\textbf{j}), negation and unity. Not bad for a point moving in a simple circle, huh?

\subsection{实验 #128:相量,第 3 部分}

I confess to using the wrong value for √3 in last month’s column. It’s 1.732. The implications of that mistake are addressed on the “Hands-On Radio” web page.\textsuperscript{6} Now, on to modulation!

What if, as seems to happen on a regular basis, one tuner-upper attracts a competing tuner-upper with another unmodulated carrier identical in frequency and amplitude except for having a slightly different phase — say 45° — ahead of the original carrier? The new signal’s phasor is given as A∠45°, just ahead of the first signal by 45°. Even though both phasors are rotating around the origin, that relationship never changes.

Since both of the signal phasors have the same frequency, why not do away with the rotating and look only at the differences? What would happen if you take a seat on the first carrier’s phasor, looking out toward the arrow’s head from the origin, and spin around with it? From your new perspective, the phasor doesn’t move or change at all because you’re rotating with it at the same rate (frequency) and its length (amplitude) is constant. The second carrier with the 45° phase difference is pointed off to the left, halfway between straight ahead and to your left. It, too, doesn’t move or change, but the phase difference means it points in a different direction.

Let’s say that the competing tuner-upper starts to drift down a little bit in frequency. As the frequency of the second signal drops, the rate at which its phasor rotates gets a little slower, too. That means it will start to fall behind the original phasor and from your perspective, the second phasor appears to rotate clockwise or backwards according to our counterclockwise-equals-positive convention. The lower the second signal’s frequency, the faster it rotates backwards. Let’s say the second signal stabilizes at a frequency 1 Hz lower. To you, it appears to rotate backwards, passing backwards across your phasor once per second. Similarly, if the frequency of the second signal increases, it will appear to rotate counterclockwise. 

Another possibility is that the phase of the second signal (with respect to the original signal) jumps around. In this case, what you would see is the phasor for the second signal shifting its position relative to the first signal — sometimes ahead, sometimes behind.

\subsubsection{AM from the Phasor Point of View}

AM produces three signals when a carrier is multiplied by a modulating signal. The first signal is the carrier with frequency, f\textsubscript{c}. If the modulating signal is a single tone with frequency, f\textsubscript{m}, two sidebands are created with frequencies, f\textsubscript{c}+f\textsubscript{m} (the upper sideband) and f\textsubscript{c}-f\textsubscript{m} (the lower sideband). See the Modulation chapter of the \textbf{ARRL Handbook}.\textsuperscript{7} Each of these signals can be treated as a phasor and the trio can be added together as we discussed in the previous column.

The amplitude of the three phasors doesn’t change but their relative directions do. Figure 1A shows what the three phasors look like from your perspective, sitting comfortably on the carrier phasor rotating at the carrier frequency, f\textsubscript{c}. Since the upper sideband (USB) phasor has a higher frequency than the carrier, you see it rotating counterclockwise at the modulating frequency, f\textsubscript{m}. Similarly, you see the LSB phasor rotating clockwise at f\textsubscript{m}. (Viewed all by themselves, the USB and LSB phasors are actually rotating at f\textsubscript{c}±f\textsubscript{m}.)

\begin{figure}[H]
    \centering
    \includegraphics[width=0.7\linewidth]{00084.jpeg}
    \caption{AM Modulation Phasors}
    \label{fig:am_phasors}
\end{figure}

Note that these counter-rotating sideband phasors have the same amplitude and are always ahead of or behind the carrier phasor. Think about what this means for the sum of the three phasors. Using the tip-to-tail method of adding phasors, the resulting AM signal’s phasor will always be aligned with the carrier phasor because of the symmetry of the sideband phasors. However, the amplitude of the AM phasor will grow and shrink as the two sidebands add to, then oppose, the carrier phasor. 

What happens if each sideband has exactly half the amplitude of the carrier? When the sideband phasors are both “pointing out” the resulting AM phasor’s amplitude equals the sum of the carrier plus the two sidebands: twice the original carrier’s amplitude. When the sideband phasors are “pointing in” their sum cancels with the carrier and there is no signal. Thus, the AM phasor’s amplitude varies from zero to twice that of the original carrier — just as you see in Figure 1B, which represents 100% modulation.

\subsubsection{FM and PM from the  Phasor Point of View}

From the standpoint of the unmodulated carrier, the phasor of an FM or PM signal moves ahead and behind that of the carrier as the amplitude of the modulating signal changes. (For the rest of this column, FM will be used to mean both FM and PM.)

Just as for AM, a pair of counter-rotating sideband phasors with frequencies of f\textsubscript{c}±f\textsubscript{m} add and cancel just as for AM. Unlike AM, however, they are oriented so that they are creating a separate modulating phasor at right angles to the carrier phasor as in Figure 2A. The resulting FM phasor created by the sum of the carrier and the modulating phasor shifts ahead of and behind its unmodulated position as in Figure 2B.

\begin{figure}[H]
    \centering
    \includegraphics[width=0.7\linewidth]{00085.jpeg}
    \caption{FM Modulation Phasors}
    \label{fig:fm_phasors}
\end{figure}

It’s not that simple, however, because FM and PM signals have constant amplitudes — only the frequency (or phase) may shift with modulation. That means the final sum of the phasors must have a constant amplitude, that of the original unmodulated carrier, shown as the arc in Figure 2B. The figure shows the small amplitude error created by including just the one set of modulation sidebands. When the modulation level is low, the error is small enough that one pair of modulation sidebands is acceptable and this is called “narrowband FM.”

As the modulation level increases (“wideband” FM) and the resulting FM phasor moves farther and farther from the unmodulated carrier, the resulting amplitude error would become larger. To keep the FM phasor close enough to the required amplitude, additional sets of sideband phasors are required. Each successive set operates at right angles to the previous set. This is the complex set of sidebands.

\subsubsection{IQ Modulation with Phasors}

As Figure 3 shows, the phasor of a modulated carrier moves around in an area defined by whether the modulation is AM or FM. If AM, the movement is horizontal, changing the phasor’s amplitude. If FM, the movement is along an arc, changing the relative phase. There’s no reason a signal can’t have both AM and FM components with the resulting phasor located anywhere within the indicated area.

\begin{figure}[H]
    \centering
    \includegraphics[width=0.7\linewidth]{00086.jpeg}
    \caption{Modulation Phasor Movements}
    \label{fig:modulation_movements}
\end{figure}

Oversimplifying to a degree, this is what IQ modulation is in which two different modulated signals are combined: the I signal (for \textbf{in phase}) and the Q signal (for \textbf{quadrature}). Both the I and Q signals are regular carrier signals, but the Q signal is 90° ahead of the I signal as shown in Figure 4. Modulating the I and Q signals independently and combining them can cause the resulting phasor to move around in the pattern of any of the AM or FM phasors discussed previously.

\begin{figure}[H]
    \centering
    \includegraphics[width=0.7\linewidth]{00087.jpeg}
    \caption{IQ Modulation Phasors}
    \label{fig:iq_phasors}
\end{figure}

Digital data can be transmitted by turning the I and Q signals on (1) and off (0) independently (also called \textbf{amplitude shift keying}), creating four possible combinations (00, 01, 10, and 11). By adding the on-or-off I and Q phasors together, the result is four different phasors shown in Figure 4. This is called \textbf{quadrature amplitude modulation} or \textbf{QAM} and each position of the phasor is called a \textbf{symbol}. If there are four possible symbols, it is called 4-QAM. A receiver demodulates the I and Q signals separately and decodes the phasors into the same on/off combinations, reproducing the same stream of digital data. 

From your perspective, sitting on the I signal’s phasor, the end points of the four phasors form a square called the modulation’s \textbf{constellation diagram}. Complex schemes with hundreds of points in the constellation have been devised — for example, digital cable TV signals use 64 or 256 points, called 64-QAM or 256-QAM, respectively.

All this from simple rotation! The interested reader may want to tackle additional information found online. You can learn more about IQ Modulation at \href{http://www.home.agilent.com/upload/cmc_upload/All/IQ_Modulation.htm?cmpid=zzfindnw_iqmod}{www.home.agilent.com/upload/cmc_upload/All/IQ_Modulation.htm?cmpid=zzfindnw_iqmod}, Amplitude and Frequency/Phase Modulation at \href{http://www.zhinst.com/blogs/michele/files/downloads/2012/12/AMFM.pdf}{www.zhinst.com/blogs/michele/files/downloads/2012/12/AMFM.pdf} and Digital Modulation at \href{http://ee.eng.usm.my/eeacad/mandeep/EEE436/CHAPTER2.pdf}{http://ee.eng.usm.my/eeacad/mandeep/EEE436/CHAPTER2.pdf}.

Nevertheless, even if you stop here, you’ll have traveled from a basic definition of phasors to how they can be used to visualize the modulation processes we use every day.

\section{中文翻译}

\subsection{实验 #126:相量,第 1 部分}

上个月,我们介绍了相量 —— 一种用振幅和相对于参考的某个相位值来表示正弦波的方法。相量表示法看起来像V∠ φ,其中V是振幅,φ是相位。让我们学习更多关于相量的知识。

\subsubsection{基本相量数学}

相量的一个优点是它们的乘法非常简单。将相量A乘以相量B需要你乘以幅度并相加角度:

V\textsubscript{A}∠φ\textsubscript{A} × V\textsubscript{B}∠φ\textsubscript{B} = V\textsubscript{A}V\textsubscript{B}∠( φ\textsubscript{A} + φ\textsubscript{B})

同样,要除法相量,需要除法幅度并相减角度:

V\textsubscript{A}∠φ\textsubscript{A} ÷ V\textsubscript{B}∠φ\textsubscript{B} = (V\textsubscript{A} / V\textsubscript{B})∠(φ\textsubscript{A} - φ\textsubscript{B})

请记住,以这种方式使用相量表示法要求两个信号具有完全相同的频率,以便φ\textsubscript{A}和φ\textsubscript{B}是恒定的。如果不是这样,数学会变得更加复杂。

那么如何相加相量呢?这就不那么简单了。因为我们在极坐标表示法中操作,你必须将每个相量分解为其X轴和Y轴分量,将这些分量相加,然后将它们转换回相量:

V\textsubscript{C}∠φ\textsubscript{C} = V\textsubscript{A}∠φ\textsubscript{A} + V\textsubscript{B}∠φ\textsubscript{B}

X轴分量 = [V\textsubscript{A} cos φ\textsubscript{A} + V\textsubscript{B} cos φ\textsubscript{B}]

Y轴分量 = [V\textsubscript{A} sin φ\textsubscript{A} + V\textsubscript{B} sin φ\textsubscript{B}]

V\textsubscript{C}∠φ\textsubscript{C} = X + \textbf{j} Y = √(X² + Y²) ∠ (tan⁻¹ Y/X)

\textbf{哎呀!}

幸运的是,科学计算器和软件通常有自动执行此数学运算的例程 —— 查看手册或帮助文件中的\textbf{极坐标表示法}。(请记住,ARRL网站上列出了这种数学的在线教程。\textsuperscript{1})

\subsubsection{图形化加减}

我们一直将所有相量绘制为头部在原点,尾部(箭头所在处)在表示幅度和角度的点。然而,相量可以绘制在X-Y平面上的任何位置,只要它们具有相同的幅度和角度!这使得图形化相加它们变得非常简单,如图1A所示,通过将相量排列为“头对尾”。结果相量从第一个相量的头部绘制到最后一个相量的尾部。

\begin{figure}[H]
    \centering
    \includegraphics[width=0.7\linewidth]{00080.jpeg}
    \caption{相量的图形化加减}
    \label{fig:phasor_addition_cn}
\end{figure}

就像普通数字一样,你可以以任何顺序相加相量。那么减法呢?将待减的相量旋转180°并如图1B所示相加 —— 就像通过乘以-1并相加来减去普通数字一样。现在你知道如何加减乘除具有共同频率的相量了。

让我们学习另一个巧妙的技巧 —— 如果相量表示电压,你如何找到两个相量之间的电压差?当你测量电路中某点的电压时,你测量的是“从”地“到”该点的电压。实际上,你是在测量该点的电压,然后减去你的接地参考点的电压,即零。如果我们的相量接地参考点在原点,如图2所示,相量的尾部(带有箭头)显示相对于地的电压测量值。

\begin{figure}[H]
    \centering
    \includegraphics[width=0.7\linewidth]{00081.jpeg}
    \caption{相量之间的电压}
    \label{fig:voltage_between_phasors_cn}
\end{figure}

当你测量电路中两个未接地的点之间的电压时,你的电表的负探头是参考点,你测量“从”参考点“到”正探头所在点的电压。如果两个电压是相量,并且我们的参考“接地”点在原点,图2显示了这是如何工作的。“从”相量A“到”相量B的电压本身就是一个相量,写为V\textsubscript{AB},计算为V\textsubscript{B} – V\textsubscript{A}。我们也可以测量从相量B到相量A的电压为V\textsubscript{BA} = V\textsubscript{A} – V\textsubscript{B}。你可以看到V\textsubscript{BA}具有完全相同的幅度,但与V\textsubscript{AB}相反的角度。花一点时间画出相量的减法,以确保你理解我是如何得出V\textsubscript{AB}和V\textsubscript{BA}的。

\subsubsection{相量到相量的电压}

这一切都很好,但它有任何实际价值吗?你会遇到相量到相量的电压吗?是的,而且比你想象的更接近家。住宅交流电源向主断路器箱供应两个相位,每个120 V。电力来自电线杆上的变压器,具有单个初级绕组和两个次级绕组。图3显示了次级绕组各自供应一个相位的电力,并在一端连接在一起作为中性线。绕组的极性相反,因此表示其电压的相量指向相反的方向,如相量图所示。这称为\textbf{分相}电力。

\begin{figure}[H]
    \centering
    \includegraphics[width=0.7\linewidth]{00082.jpeg}
    \caption{分相电力}
    \label{fig:split_phase_power_cn}
\end{figure}

如果你有两个相等且相反的相量,它们之间的电压幅度是多少?(答案:相量幅度的总和。)如果每个相量的幅度为120 V,它们之间的电压幅度为120 + 120 = 240 V。如果你将一根热线连接到每个相位,一根连接到中性线,那就是你的放大器(或你的烘干机)的交流电来源!

\subsubsection{三相电力}

现在让我们更进一步 —— 三相电力。\textsuperscript{2} 来自水坝和发电厂等发电设施的交流电具有三个相位。这就是为什么高压线路由三根电线(或电线对)组成(不包括任何保护地线)。大型电力用户如果只使用一个相位的电力会使电网不平衡,因此它们被布线为使用每个相位的一些电力,电工负责配置事物,以便每个相位的负载大致相同。这就是为什么任何规模的建筑物和企业都有三相交流服务,而不仅仅是两相。

表示三个相位 —— A、B和C —— 的相量如图4所示。它们都均匀分布在圆圈周围,周长的1⁄3或120°间隔。假设你在一栋大楼里的公寓供应两相电力,就像住宅分相交流电一样,每个相量的幅度为120 V。当你尝试通过将烘干机连接到两相(假设是相位A和B)来运行它时会发生什么?为什么你没有得到240 V?

\begin{figure}[H]
    \centering
    \includegraphics[width=0.7\linewidth]{00083.jpeg}
    \caption{三相电力}
    \label{fig:three_phase_power_cn}
\end{figure}

看看图4和相量V\textsubscript{BA}。表示电力服务相位的两个相量V\textsubscript{A}和V\textsubscript{B}不是指向相反方向 —— 它们只有120°的间隔 —— 所以它们的幅度不像分相情况那样相加。事实上,如果你查阅三角学,相量V\textsubscript{BA}的幅度 = √3 V\textsubscript{A} = 1.732 V\textsubscript{A},而不是2 V\textsubscript{A}。如果每个相位供应120 V,你的烘干机在连接到两相时会看到什么电压?(120 × 1.732 = 208 V)

这种对电力服务如何从公用电网派生的依赖在运行重负载时产生很大差异 —— 例如放大器。如果你的放大器设计为使用240 V电源,而你连接到224 V,这大约低7%。放大器在略微较低的输入电压下运行时通常不会提供其全额额定输出功率。相反的情况 —— 高于预期的输入电压 —— 也会给高压组件带来压力。

如果你的设备没有像90到260 V交流这样的“通用”功率额定值,请确定如何为你可用的电压配置它。许多电器和放大器在电源变压器的初级绕组上有可选择的输入电压“抽头”或连接,可以适应240、220或208 V电源。(208从哪里来?208大约是1.732 × 120 V,家庭交流服务的通常电压。)

我说过我们会有一个两部分的文章,但还需要一个部分来讨论一些真正的无线电核心内容:从相量的角度看AM和PM调制。这反过来将引导你进入现代数据通信的大门:I-Q调制。

\subsection{实验 #127:相量,第 2 部分}

不,不是那种你设置为击晕的相位器,傻瓜!如果你通过了General考试,你学习了相位和一点角频率。业余Extra级执照持有者(以及那些正在学习Extra的人)甚至使用了相量表示法,尽管它被称为另一个名称。

在这个两部分的专栏中,我们将首先发展基本概念,以展示相量是什么以及它如何与你已经理解的事物相关。然后我们将进展到使用相量描述电气和无线电现象的例子,如调制。随着业余无线电开始使用更先进的调制类型,理解相量将为熟悉的AM/SSB调制和未来之间提供重要的桥梁。

\subsubsection{正弦波}

就像无线电中的许多餐点一样,这道菜以正弦波开始,并用复数调味。(你可以在ARRL网站上列出的General和Extra考试的数学教程中找到这些主题。\textsuperscript{3})正弦波(或\textbf{正弦曲线})看起来像一个定期增加和减少的波,但如图1所示,它实际上与旋转有关。

\begin{figure}[H]
    \centering
    \includegraphics[width=0.7\linewidth]{00077.jpeg}
    \caption{正弦波和旋转}
    \label{fig:sinusoid_rotation_cn}
\end{figure}

想象一个点在复平面的原点周围逆时针旋转,如图1左侧所示。如果该点距离原点一个单位,则该点访问的每个位置的坐标为[cosθ, \textbf{i} sinθ],其中θ(希腊字母theta)是从正x轴到从原点到该点的线的角度。这个圆,毫不奇怪,被称为\textbf{单位圆},因为其半径的值为1或单位。

当点绕原点旋转时,θ从0稳定增加到360°,并在每个周期重新开始于0°。(逆时针被认为是正方向。)如果点总是以相同的速度移动,则绕原点的周期频率f不变,点每秒移动360 × f度。这意味着点在t秒内移动的度数,θ = 360 × f × t。

一个圆中有2π弧度(另一种角度测量单位),因此θ = 2π × f × t。量2πf被称为角频率,ω,你会在电抗和许多其他依赖频率的电气计算的公式中看到它。

将所有内容联系起来,点在绕圆移动时每个时间点的位置坐标为[cos(2πft), \textbf{i} sin(2πft)],图1右侧的图形绘制了点的y(或虚数)坐标与时间的关系,创建了一个正弦波。如果我们绘制点的x(或实数)坐标与时间的关系,它将创建一个余弦波。

从绕圆移动的点跃升到更电气的东西,点的虚数坐标的幅度,sin(2πft) = sin(ωt),也可以是电压、电流或场强。事实上,交流电源的熟悉正弦波来自发电机场线圈的旋转运动。当线圈通过发电机中的磁场时,线圈和磁场之间的角度以与我们的点绕原点移动相同的方式变化。线圈和磁场之间的这种变化关系在线圈中产生正弦电压。

虽然正弦波和余弦波通常都被称为正弦曲线,但它们在一个重要方面有所不同。从t = 0开始,余弦波从1 + 0\textbf{i}的值开始,正弦波从0 + \textbf{i}开始。除了从不同的值开始外,波是相同的。余弦波描述实坐标,正弦波描述虚坐标。

你也可以将起始值的差异视为角度差异,其中正弦波比余弦波领先π/2弧度(90°)。这种差异永远不会改变,因为两个波都描述相同的事情 —— 恒定旋转。波上特定点的位置是其相位,差异的量称为\textbf{相位角},在这种情况下为90°。

这就是以下三角恒等式的来源:sinθ = cos(θ – 90°) = cos(θ – π/2)和cosθ = sin(θ + 90°) = sin(θ + π/2)。当从旋转和单位圆的角度查看时,正弦波和余弦波之间的许多、许多这种关系变得明显(或至少更易于理解)。

\subsubsection{极坐标表示法}

到目前为止,我们使用了移动点坐标的\textbf{矩形}形式:x + \textbf{i} y。在大多数工程技术文献中,字母\textbf{j}被用来代替\textbf{i}以避免与电流混淆,从这里开始,我们也将这样做。

接下来是你可能已经为Extra级考试学习(或将学习!)的形式,即极坐标形式,其中坐标采用半径和角度的形式:r ∠θ。极坐标形式读作“r在(角度)theta处”。使用极坐标形式坐标表示单位圆上的点很容易,因为它们总是1∠θ。如果你描述点在绕圆旋转时的位置,你可以使用我们之前计算的角频率方程,坐标变为1∠(2πft) = 1∠(ωt)。因此,这种特殊方法是描述移动点正在做什么的好简写方式。

\subsubsection{介绍相量}

在处理RF信号和电路时,信号的频率通常不变。例如,考虑一个RC低通滤波器:输入信号V\textsubscript{IN} sin (ωt + 0)和输出信号V\textsubscript{OUT} sin (ωt + ϕ)具有相同的频率,即使它们的幅度不同,比值为V\textsubscript{OUT}/V\textsubscript{IN},并且它们的相位偏移为ϕ。

假设两个信号的频率相同,我们的极坐标形式现在可以简化为V∠ ϕ,其中ϕ只是信号与某个参考信号或相位之间的相位角。电路的输入信号通常是测量相位差的参考。

嘿,你猜怎么着?V∠ ϕ是一个相量!相量只是一个表示正弦波幅度和相位的复数,V∠θ极坐标表示法只是方便的数学简写。相量是一种向量 —— 具有大小和方向的量。在V∠ ϕ的情况下,大小是|V|,方向是相位角ϕ,因此更繁琐的名称“相位向量”被缩短为“相量”。(作为向量,相量通常进一步缩短并写为单个粗体字母,如\textbf{V}或\textbf{I}。)

\begin{figure}[H]
    \centering
    \includegraphics[width=0.7\linewidth]{00078.jpeg}
    \caption{自行车轮演示}
    \label{fig:bicycle_wheel_cn}
\end{figure}

\begin{figure}[H]
    \centering
    \includegraphics[width=0.7\linewidth]{00079.jpeg}
    \caption{相量图}
    \label{fig:phasor_diagram_cn}
\end{figure}

如果我们的原始正弦波是参考信号,描述正弦波的相量是\textbf{V}∠0,余弦波是\textbf{V}∠–90°或\textbf{V}∠–π/2。请记住,两个信号的频率被假设为相同,无论是来自电网的60 Hz还是20米波段的14.200 MHz。图3显示了滤波器输入和输出信号的\textbf{相量图}。

还有最后一种描述信号的方式 —— \textbf{指数形式},其中它表示为V e\textsuperscript{jθ}。这种形式来自欧拉方程背后的数学\textsuperscript{4},其中我们点的坐标神奇地显示为等于e\textsuperscript{jθ} = cos θ + \textbf{j} sin θ。这种形式来自欧拉方程背后的数学,其中我们点的坐标神奇地显示为等于e\textsuperscript{jθ} = cos θ + \textbf{j} sin θ。这个方程背后的严肃而美丽的数学\textsuperscript{5}是许多电气工程的核心,并导致了令人惊叹的欧拉恒等式:e\textsuperscript{jπ} = –1,它将两个最广泛使用的超越数(e和π)、虚数(\textbf{j})、否定和单位统一起来。对于一个在简单圆中移动的点来说,这不错,不是吗?

\subsection{实验 #128:相量,第 3 部分}

我承认在上个月的专栏中使用了错误的√3值。它是1.732。这个错误的影响在“Hands-On Radio”网页上得到了解决。\textsuperscript{6} 现在,让我们讨论调制!

如果,正如经常发生的那样,一个调谐器吸引了另一个调谐器,另一个未调制的载波在频率和幅度上相同,除了相位略有不同 —— 比如比原始载波领先45°?新信号的相量表示为A∠45°,比第一个信号领先45°。即使两个相量都绕原点旋转,这种关系永远不会改变。

由于两个信号相量具有相同的频率,为什么不消除旋转并只看差异呢?如果你坐在第一个载波的相量上,从原点看向箭头的头部,并随着它旋转,会发生什么?从你的新角度来看,相量不会移动或改变,因为你以相同的速率(频率)旋转,并且其长度(幅度)是恒定的。具有45°相位差的第二个载波指向左侧,在正前方和左侧之间的中间。它也不会移动或改变,但相位差意味着它指向不同的方向。

假设竞争的调谐器开始频率略微下降。随着第二个信号的频率下降,其相量旋转的速率也会略微减慢。这意味着它将开始落后于原始相量,从你的角度来看,第二个相量似乎会根据我们的逆时针等于正的约定顺时针或向后旋转。第二个信号的频率越低,它向后旋转的速度就越快。假设第二个信号稳定在频率低1 Hz。对你来说,它似乎向后旋转,每秒向后穿过你的相量一次。同样,如果第二个信号的频率增加,它将似乎逆时针旋转。

另一种可能性是第二个信号的相位(相对于原始信号)跳跃。在这种情况下,你会看到第二个信号的相量相对于第一个信号的位置发生变化 —— 有时领先,有时落后。

\subsubsection{从相量角度看AM}

当载波乘以调制信号时,AM产生三个信号。第一个信号是频率为f\textsubscript{c}的载波。如果调制信号是频率为f\textsubscript{m}的单音,则创建两个边带,频率为f\textsubscript{c}+f\textsubscript{m}(上边带)和f\textsubscript{c}-f\textsubscript{m}(下边带)。参见\textbf{ARRL手册}的调制章节。\textsuperscript{7} 这些信号中的每一个都可以被视为相量,并且三人组可以像我们在上一专栏中讨论的那样相加。

三个相量的幅度不会改变,但它们的相对方向会改变。图1A显示了从你的角度看三个相量的样子,舒适地坐在以载波频率f\textsubscript{c}旋转的载波相量上。由于上边带(USB)相量的频率高于载波,你看到它以调制频率f\textsubscript{m}逆时针旋转。同样,你看到LSB相量以f\textsubscript{m}顺时针旋转。(单独查看时,USB和LSB相量实际上以f\textsubscript{c}±f\textsubscript{m}旋转。)

\begin{figure}[H]
    \centering
    \includegraphics[width=0.7\linewidth]{00084.jpeg}
    \caption{AM调制相量}
    \label{fig:am_phasors_cn}
\end{figure}

请注意,这些反向旋转的边带相量具有相同的幅度,并且始终领先或落后于载波相量。思考一下这对三个相量的总和意味着什么。使用相量的头对尾相加方法,由于边带相量的对称性,产生的AM信号的相量将始终与载波相量对齐。然而,AM相量的幅度会随着两个边带相加然后反对载波相量而增长和收缩。

如果每个边带的幅度恰好是载波的一半,会发生什么?当边带相量都“向外指向”时,产生的AM相量的幅度等于载波加上两个边带的总和:原始载波幅度的两倍。当边带相量“向内指向”时,它们的总和与载波抵消,没有信号。因此,AM相量的幅度从零变化到原始载波的两倍 —— 正如你在图1B中看到的,这表示100%调制。

\subsubsection{从相量角度看FM和PM}

从未调制的载波的角度来看,FM或PM信号的相量随着调制信号的幅度变化而领先和落后于载波的相量。(在本专栏的其余部分,FM将用于表示FM和PM。)

与AM一样,频率为f\textsubscript{c}±f\textsubscript{m}的一对反向旋转边带相量相加和抵消,就像AM一样。然而,与AM不同,它们的方向使得它们在与载波相量成直角的方向上创建一个单独的调制相量,如图2A所示。由载波和调制相量的总和创建的FM相量如图2B所示,领先和落后于其未调制位置。

\begin{figure}[H]
    \centering
    \includegraphics[width=0.7\linewidth]{00085.jpeg}
    \caption{FM调制相量}
    \label{fig:fm_phasors_cn}
\end{figure}

然而,事情并不那么简单,因为FM和PM信号具有恒定的幅度 —— 只有频率(或相位)可能随调制而变化。这意味着相量的最终总和必须具有恒定的幅度,即原始未调制载波的幅度,如图2B中的弧线所示。该图显示了仅包含一组调制边带所产生的小幅幅度误差。当调制水平低时,误差足够小,一组调制边带是可接受的,这称为“窄带FM”。

随着调制水平的增加(“宽带”FM)和产生的FM相量从未调制载波移动得越来越远,产生的幅度误差会变得更大。为了使FM相量足够接近所需的幅度,需要额外的边带相量组。每个连续的组与前一组成直角操作。这是复杂的边带组。

\subsubsection{用相量进行IQ调制}

如图3所示,调制载波的相量在由调制是AM还是FM定义的区域内移动。如果是AM,移动是水平的,改变相量的幅度。如果是FM,移动是沿弧线的,改变相对相位。没有理由信号不能同时具有AM和FM分量,结果相量位于所示区域内的任何位置。

\begin{figure}[H]
    \centering
    \includegraphics[width=0.7\linewidth]{00086.jpeg}
    \caption{调制相量移动}
    \label{fig:modulation_movements_cn}
\end{figure}

在某种程度上简化,这就是IQ调制,其中两个不同的调制信号被组合:I信号(用于\textbf{同相})和Q信号(用于\textbf{正交})。I和Q信号都是常规载波信号,但Q信号比I信号领先90°,如图4所示。独立调制I和Q信号并将它们组合可以导致产生的相量以我们之前讨论的任何AM或FM相量的模式移动。

\begin{figure}[H]
    \centering
    \includegraphics[width=0.7\linewidth]{00087.jpeg}
    \caption{IQ调制相量}
    \label{fig:iq_phasors_cn}
\end{figure}

数字数据可以通过独立地打开(1)和关闭(0)I和Q信号(也称为\textbf{幅度键控})来传输,创建四种可能的组合(00, 01, 10, 和11)。通过将开或关的I和Q相量相加,结果是图4中显示的四个不同的相量。这称为\textbf{正交幅度调制}或\textbf{QAM},相量的每个位置称为\textbf{符号}。如果有四种可能的符号,则称为4-QAM。接收器分别解调I和Q信号,并将相量解码为相同的开/关组合,重现相同的数字数据流。

从你的角度来看,坐在I信号的相量上,四个相量的端点形成一个正方形,称为调制的\textbf{星座图}。已经设计了星座中有数百个点的复杂方案 —— 例如,数字有线电视信号使用64或256个点,分别称为64-QAM或256-QAM。

所有这些都来自简单的旋转!有兴趣的读者可能想在线处理额外的信息。你可以在\href{http://www.home.agilent.com/upload/cmc_upload/All/IQ_Modulation.htm?cmpid=zzfindnw_iqmod}{www.home.agilent.com/upload/cmc_upload/All/IQ_Modulation.htm?cmpid=zzfindnw_iqmod}了解更多关于IQ调制,在\href{http://www.zhinst.com/blogs/michele/files/downloads/2012/12/AMFM.pdf}{www.zhinst.com/blogs/michele/files/downloads/2012/12/AMFM.pdf}了解幅度和频率/相位调制,在\href{http://ee.eng.usm.my/eeacad/mandeep/EEE436/CHAPTER2.pdf}{http://ee.eng.usm.my/eeacad/mandeep/EEE436/CHAPTER2.pdf}了解数字调制。

尽管如此,即使你在这里停下来,你也会从相量的基本定义走到它们如何用于可视化我们每天使用的调制过程。

\subsection{注释}

\begin{enumerate}
    \item \href{http://www.arrl.org/studying-for-the-general-license}{www.arrl.org/studying-for-the-general-license},点击“Math Tutorials”,然后点击“Tutorials on Math for License Exams”
    \item 三相电力的Y和delta连接的全面处理可在\href{http://www.ece.msstate.edu/~donohoe/ece3414three_phase_power.pdf}{www.ece.msstate.edu/~donohoe/ece3414three_phase_power.pdf}在线获取。
    \item \href{http://www.arrl.org/studying-for-the-general-license}{www.arrl.org/studying-for-the-general-license},点击“Math Tutorials”,然后点击“Tutorials on Math for License Exams”
    \item Nahin, Paul J., \textbf{Dr. Euler’s Fabulous Formula: Cures Many Mathematical Ills}(Princeton University Press, 2006)。
    \item 在他的\textbf{Lectures on Physics}中,物理学家Richard Feynman将该方程描述为“我们的宝石”和“所有数学中最非凡、几乎令人震惊的公式之一”。
    \item 所有以前的Hands-On Radio实验都可在ARRL成员的\href{http://www.arrl.org/hands-on-radio}{www.arrl.org/hands-on-radio}获取。
    \item 可从你的ARRL经销商或ARRL商店获取,ARRL订单号6948。美国免费电话888-277-5289,或860-594-0355,传真860-594-0303;\href{http://www.arrl.org/shop/;}{www.arrl.org/shop/;} <a href="mailto:pubsales@arrl.org">pubsales@arrl.org</a>。
\end{enumerate}

\chapter{实验 #134:结温的影响}
\section{英文原文}

The equations describing how a bipolar junction transistor circuit works are usually simplified to ratios of resistances or currents. Many assumptions are made so that the calculations are straightforward — and most of the time this works just fine. The designer and technician should understand that underlying the simplifications and assumptions is a fundamental relationship that can have a major effect on how a circuit behaves as temperature changes. This month, you'll observe the effect directly.

\subsection{PN Junction Volt-Amp  Characteristic}

You've probably seen the basic graphs of Figure 1, perhaps combined in a single graph with different current scales for forward and reverse current. These graphs show the relationship between current and voltage — the I-V characteristic — for any semiconductor PN junction. The equation that generates the graphs is called the Fundamental Diode Equation. With positive for both voltage and current defined as from the P-type to the N-type material:

\begin{figure}[H]
    \centering
    \includegraphics[width=0.7\linewidth]{00088.jpeg}
    \caption{PN结伏安特性}
    \label{fig:pn_junction_iv}
\end{figure}

\begin{figure}[H]
    \centering
    \includegraphics[width=0.7\linewidth]{00089.jpeg}
    \caption{基本二极管方程}
    \label{fig:fundamental_diode_equation}
\end{figure}

where I\textsubscript{D} is the forward current through the diode, V\textsubscript{D} is the forward voltage across the diode, and I\textsubscript{S} is the reverse-bias saturation current that flows when V\textsubscript{D} is negative. (I\textsubscript{s} is measured at the lowest voltage that will produce a stable current level — typically a few tens of mV of reverse bias and is much smaller than the typical values for reverse leakage current specified in datasheets.) η is the ideality factor (also called the quality factor or emission coefficient) that depends on how the carriers of current recombine at the diode junction and varies from 1 to 2. For normal currents, η = 1 is usually sufficient.

The most interesting bit of this equation is the thermal voltage, V\textsubscript{T} = kT/q where k is the Boltzmann constant that relates the energy of a particle (the electron) and its temperature, T is the absolute temperature in degrees Kelvin, and q is the charge of an electron. At room temperature (300 K is commonly used in simulation software), V\textsubscript{T} = 28.85 mV. A common simplification is that V\textsubscript{T} ≈ T / 11,600.

What happens as temperature increases? Since V\textsubscript{T} goes up along with T, if everything else on the right side of the equation stays constant, I\textsubscript{D} "should" go down. How about an experiment? Figure 2 shows a simple test circuit for evaluating a diode's I-V characteristic. V and A represent a voltmeter and ammeter, respectively. Assuming you are using multimeters, set them to read 0 – 1 V and an initial current of 0 – 200 or 0 – 300 µA.

\begin{figure}[H]
    \centering
    \includegraphics[width=0.7\linewidth]{00090.jpeg}
    \caption{测试电路}
    \label{fig:test_circuit}
\end{figure}

The op-amp is connected as an adjustable voltage source with its output set by the 10 kΩ pot. Any garden variety op-amp will suffice, such as the venerable LM741. Your power supply needs to have an output of at least ±3 V and up to ± 15 V will do if it is within the op-amp's maximum voltage ratings.

Select an ordinary silicon diode such as a 1N914, 1N4148, or 1N4000-series part. Before connecting the diode, set the op-amp output voltage to 0.3 V on the voltmeter. Then connect the diode and slowly increase the voltage to 0.7 or 0.75 V in  0.05 V steps, recording both voltage and current. (A spreadsheet to generate a graph is provided on the Hands-On Radio web page.\textsuperscript{1}) The measurements start at 0.3 V because current is too low to be measured with ordinary test equipment below that level. Most multimeters are not very accurate at the low end of their ranges so expect your measured values to diverge quite a bit from the calculated values.

Now cool the diode by at least 30 °C and measure the currents again. One way to get a relatively consistent temperature during the measurements is to put a metal object that is many times larger than the diode, such as a large nut, in your home freezer for an hour or so. Orient the diode so you can sit the object directly on it. Measure current through the diode and when the reading stabilizes begin taking data — you'll have to work fast so the temperature remains about the same.

Why does I\textsubscript{D} have different values when the diode is cooled? Because the value of I\textsubscript{S} is strongly dependent on temperature, as is V\textsubscript{T}. In fact, for silicon diodes, the value of I\textsubscript{S} changes by about 7 % / °C, a positive shift with temperature. That means I\textsubscript{S} will double (or halve) with every 10 °C increase (or decrease) in temperature! 

The two temperature dependencies for V\textsubscript{T} and I\textsubscript{S} work against each other. As temperature increases, V\textsubscript{T} goes up which works to lower I\textsubscript{D} because it is in the exponent's denominator. On the other hand, increasing I\textsubscript{S} causes I\textsubscript{D} to increase. Thus, it is a balancing act with the change in I\textsubscript{S} having the larger effect. While the exact change requires some detailed calculations, around room temperature the net result is that V\textsubscript{D} changes about –2.2 mV / °C if current through the diode is held constant.

Next, go the other way: heat the diode by about the same amount (putting the metal object in hot water will do the job) and take the same set of measurements. Figure 3 shows a set of data I took for a 1N4148 diode. As you can see, below about 100 µA, my measurements started to show some signs of being inaccurate. (This is typical of low-level home-lab test setups, so don't be too concerned if the data doesn't make a nice straight line.) The diode you choose and the temperatures you obtain will probably give significantly different values of current — all you are trying to do is observe the effect of temperature.

\begin{figure}[H]
    \centering
    \includegraphics[width=0.7\linewidth]{00091.jpeg}
    \caption{测量数据}
    \label{fig:measurement_data}
\end{figure}

If you'd like some extra credit, try taking the measurements by controlling I\textsubscript{D} and measuring V\textsubscript{D}. Figure 2B shows how to turn the op-amp into a current source. (See Experiment #3 — Op-Amps for an explanation of this circuit.)\textsuperscript{1} Start with a short circuit instead of the diode and confirm that with the pot set to maximum resistance, approximately +V / 10 kΩ of current is flowing. Reinstall the diode and adjust the current to get about the same values of V\textsubscript{D} as in the previous set of measurements. You should see fairly similar results.

Double your extra credit by trying a germanium diode, such as the common 1N34A. Germanium (Ge) diodes have a much higher value of I\textsubscript{S} than do silicon (Si) diodes by three to four orders of magnitude! Thus, the current values you measure will be much larger for a given value of V\textsubscript{D}. That can work to our advantage, however, because for a given amount of I\textsubscript{D}, a germanium diode will have a lower V\textsubscript{D} than a silicon diode. That may be an advantage in a sensitive circuit, such as a diode detector.

While you are at it, remember that a bipolar junction transistor is constructed from a pair of back-to-back PN junctions. Substitute the collector-base or emitter-base junction of an inexpensive PNP (2N3906 or 2N4404) or NPN (2N3904 or 2N4401) transistor for the diode. The resulting measurements should be fairly similar. Remember that temperature dependency when designing your next transistor amplifier!

\subsection{Temperature Sensing}

Since the effect of temperature on diode current and voltage are so predictable, it's quite possible to make a temperature sensor out of a diode. Rearranging the diode equation gives

\begin{figure}[H]
    \centering
    \includegraphics[width=0.7\linewidth]{00092.jpeg}
    \caption{温度传感方程}
    \label{fig:temperature_sensing_equation}
\end{figure}

So you can see that measuring V\textsubscript{D} while holding I\textsubscript{D} constant gives a pretty good idea of temperature. Using a microprocessor to "do the math" or applying a comparator circuit to detect when V\textsubscript{D} crosses a threshold is an excellent method of temperature control at low cost.

\subsection{Parts list}

All parts can be substituted by any equivalent.

\begin{itemize}
    \item Op-amp — LM741
    \item Silicon diode — 1N4148 or 1N4000-series
    \item Germanium diode — LM34A
    \item Transistor — 2N4401/4403 or 2N3904/3906
    \item Potentiometer — 10 kΩ
\end{itemize}

\section{注释}

\begin{enumerate}
    \item All previous Hands-On Radio experiments are available to ARRL members at \href{http://www.arrl.org/hands-on-radio}{www.arrl.org/hands-on-radio}.
\end{enumerate}

\chapter{实验 #138:E 与 V}
\section{英文原文}

One of the first bits of technical lore every ham, indeed, every electronic hobbyist learns is the venerable Ohm's Law. This simple formula explains the relationship between current through a material with a known resistance, and the voltage across the material. Students dutifully memorize it in all its forms: I = E / R, E = I × R, and R = E / I but often ask the question, "Where do E and I come from?" They understand the use of R to represent resistance. The use of E for voltage and I for current is a little confusing. Why not V and C or V and A? That's a very good question and its answer goes back to the beginning of the electric era.

Instructors correctly respond that "E stands for electromotive force or emf." Isn't that just a fancy way of saying "voltage?" Not really. Let's go back to the early 1800s, when experimenters like Faraday and Henry and Ørsted were discovering electric and magnetic fields, the relationships between them, and their ability to influence the movement of a mysterious substance known as electric charge. (Some of these early discoveries are related in Hands-On Radio Experiments #117 and #118.\textsuperscript{1}) 

\subsection{Before the Electron}

The early experimenters were completely in the dark about what was moving around in their wires and reacting to the presence or motion of electric or magnetic fields. Even Faraday's concept of a field that explained how energy could be distributed in space was radical and new. So all they knew was electric charge and that there was a force that could make the charge move — the electromotive force.

Remember that no one at the time had the slightest idea what electric charge was. Dalton's description of the atomic role in chemistry had only been introduced between 1802 and 1805, and Lord Kelvin's discovery and description of the electron as an individual particle didn't come until 1897. Bohr's model of the hydrogen atom with electrons arranged in shells around a nucleus and which could leave an atom and move about, was not developed until 1913. Even today, we don't really understand what electric charge actually is, why it exists, or even what the electron consists of. That those early experimenters were able to ascertain fundamental laws which still hold true today is nothing short of remarkable!

\subsection{From E to V}

Yes, yes, but what about E and V? If you browse through an introductory electrical engineering textbook that introduces students to the world of electric fields and circuits, you will likely see that the book begins with these same basic relationships between fields and current. Students learn a general description of what makes electrons move in response to electric fields and magnetic fields and the initial equations usually use E or e to represent the electromotive force. (The symbol ε is also used.)

In this general environment, it is appropriate for the equations to use electromotive force. The resulting electron motion is analyzed through a volume or across a surface that has some conductivity (σ) which describes the ease with which electrons move through it. Figure 1 shows a diagram of charge moving through a cross-section area, A. Resistivity (ρ) is the reciprocal of conductivity, so that ρ = 1 / σ. Resistivity is measured as ohm-meters or ohm-m.

\begin{figure}[H]
    \centering
    \includegraphics[width=0.7\linewidth]{00093.jpeg}
    \caption{电荷通过横截面}
    \label{fig:charge_cross_section}
\end{figure}

So far, we haven't seen a hint of a circuit and all the equations are using E and the general parameter current density (J) which is the amount of current per unit of area. Boring! When do we get to the good stuff? Patience is eventually rewarded as the conducting material is formed into a thin loop — a circuit — and subjected to electric and magnetic fields. Now the electrons are moving around in this circuit and J can be replaced by current, I, the flow of all electric charge through the entire cross-sectional area of this thin conducting volume.\textsuperscript{2} Similarly, it now makes sense to talk about the loop's resistance, represented by R, since the current is now flowing in this highly constrained path. R accounts for all of the material's resistance and for a cylindrical wire as in Figure 2, R = ρ l / A, where l is the wire's length, A is its cross-sectional area, and ρ is the material's resistivity.

\begin{figure}[H]
    \centering
    \includegraphics[width=0.7\linewidth]{00094.jpeg}
    \caption{圆柱形导线}
    \label{fig:cylindrical_wire}
\end{figure}

We still have no batteries or power supplies, just an externally generated field that is a source of emf, represented by E and so the relationship between the three parameters becomes the familiar I = E / R. Here E represents the emf developed all the way around the loop by the external field. If you did the experiments of Experiment #118, you observed this current as the magnet moved near the coil of wire which made a long loop. 

\begin{figure}[H]
    \centering
    \includegraphics[width=0.7\linewidth]{00095.jpeg}
    \caption{自制环路}
    \label{fig:homemade_loop}
\end{figure}

Figure 3 shows how to see this emf by constructing your own loop and placing it near a power transformer or motor which is the source of a changing magnetic field. (For a field to generate emf in a loop that makes a current flow, the field must be changing or the circuit must be moving through the field.)

To measure the emf the loop is cut as in Figure 3, creating a pair of terminals. Current flow stops, of course, but an electric potential now exists between those terminals.\textsuperscript{3} By measuring it, we can tell how much energy (in joules) an amount of electric charge (in coulombs) will gain from the effects of the field on the loop. Electric potential is thus measured in joules/coulomb and the value of 1 joule/coulomb defined as 1 volt in honor of Alexander Volta, inventor of the battery. Electric potential was given the simpler name of voltage as well. The symbol for volts is the familiar V and voltage is represented in an equation by V or v.

Can V be substituted for E without risk of confusion? If the discussion is of ordinary electronic circuits, the answer is almost always yes. It is safe to say that substituting I = V / R for I = E / R will not get you into trouble around ham radio equipment! You should use E, however, when the discussion is about fields — for motors, generators, antennas, transformers, etc. Another place you'll see emf is the term kickback emf that refers to the reverse voltages generated by inductances when current through them is suddenly interrupted or turned off, causing the magnetic field created by current through the inductance. Similarly, a motor develops a back emf that opposes the applied voltage, limited current through the motor to only that required to drive the load and account for internal losses.

Don't confuse units of measurement, such as volts (V), amperes (A), and ohms (Ω), with their corresponding physical phenomena, such as electric potential (also V), current, (I), or resistance (R). For example, Ohm's Law is never written as A = V / Ω.

We have answered almost all of the original question but what about the mysterious I? Returning back to the early days, experimenters had learned to discriminate between a quantity of charge and the  flow of charge (current). The French experimenter Ampere (for whom the unit of current was later named) gave flow of charge the name "Intensité de Courant" (Intensity of Current) and assigned it the symbol I or i in equations. (This is one reason why electrical engineers use j instead of the mathematician's i to represent the imaginary square root of negative one in their equations.)

The ampere is a flow of one coulomb (C) per second. How many electrons make up 1 C? 6.24×10\textsuperscript{18} electrons! Basic Radio from 1942 gives us an idea of how many that is: "If 3 million people were to count for 8 hours per day at the rate of 200 per minute, they would have to count from the time of the Trojan Wars in 500 BC down to the present" in order to count the number of electrons passing by at a rate of 1 ampere.

Another source of frequent confusion is the overuse of the symbol C, which is used at various points to represent coulombs, capacitance (measured in farads with the symbol F), battery capacity and battery charge or discharge rates, not to mention the speed of light, c — see?

In closing, I would like to acknowledge the comments of Dana Brown, AD5VC, and Sam Neal, N5AF, on the "ham_instructor" e-mail reflector.\textsuperscript{4} Sharing expertise is one of the hallmarks of Amateur Radio and it is particularly important for instructors so that our newest licensees have a common (and correct) understanding of radio's physical environment.

\section{注释}

\begin{enumerate}
    \item All previous Hands-On Radio experiments are available to ARRL members at \href{http://www.arrl.org/hands-on-radio}{www.arrl.org/hands-on-radio}. 
    \item As an example — if a current, I, of 1 ampere is flowing in a 16 AWG wire 0.051 inches in diameter, the wire has a cross-sectional area of πd\textsuperscript{2}/4 = 0.00204 in\textsuperscript{2} and the corresponding current density, J = 1 / 0.00204 = 490 A / in\textsuperscript{2}. If the wire is made smaller, I will not change but J will increase.
    \item The electric potential across the open-circuited terminals of a source of emf is also known as the open-circuit voltage. 
    \item \href{https://groups.yahoo.com/neo/groups/ham_instructor/info}{https://groups.yahoo.com/neo/groups/ham_instructor/info}
\end{enumerate}

\chapter{实验 #139:数字代码基础}
\section{英文原文}

While tuning the bands, I've become more and more fascinated by the sheer variety of "digital" signals I encounter. Ever since the FCC allowed amateurs to develop their own protocols and codes in the early 1990s, we have seen an explosion of amateur digital innovation. The Fldigi software includes support for more than 30 digital modes, along with numerous variations on the protocols and modulations. WSJT, originally developed for VHF meteor scatter, is now a full suite of protocols. PACTOR and WINMOR are pushing the boundaries of digital data transfer over the very difficult HF channel. For voice communication, CODEC2 is now a fully-capable digital voice protocol. More are on the way as amateurs put the processing power of the modern PC to work, even copying that most venerable of all digital modes, CW!

Regardless of whether you are a phone, CW, or digital fan (or all three!) it's important to understand the basics of modern digital technology. There are certain bits of vocabulary that describe the data stream that underlies wireless data. This column will grapple with a few.

\subsection{Bits of Data}

Let's start with that most elementary of digital concepts — the bit. This is the 1-or-0 information element from which all other data is constructed. When using CW, RTTY, or PSK31, bits are what you hear being transmitted, either as a tone or as the presence of a carrier. 

As you can see from Figure 1, the Morse code character for E is seven bits long — three 0 bits, a single 1 bit, and a concluding three 0 bits! Morse code is composed of two fundamental elements: the dot (labeled 1 in the figure) and the inter-element space (labeled 0 in the figure). Dashes are composed of three dot elements with no intervening space and the inter-character space is composed of three inter-element spaces.

\begin{figure}[H]
    \centering
    \includegraphics[width=0.7\linewidth]{00096.jpeg}
    \caption{摩尔斯电码字符E}
    \label{fig:morse_code_e}
\end{figure}

The bits are assembled into characters on the receiving end by the human operator in the case of CW or by software in the case of RTTY or PSK31. The decoding process also requires some information about when a character's-worth of bits begins and ends, so framing bits are added. In the case of Morse code (CW), the framing bits are actually the inter-element spaces: three inter-element spaces in a row means a character has just finished or is just about to start. Don't discount the value of framing bits — if you've ever tried to copy CW with spaces that are too short, you know how important framing bits are!

\subsection{Know the Codes}

Amateurs refer to Morse as "the code" but there are really lots of codes in the digital world. In this sense, a code is simply a method of representing characters as a pattern of bits. To encode a character means to turn it into its proper bit pattern and to decode it means to turn the pattern of bits back into the character. There may be additional codes operating on the characters such as compression (like a ZIP file) or abbreviations (like Q signals) but at the basic level all we're talking about is the rules for turning whatever represents 1s and 0s into characters and vice versa.

Morse and PSK31 are somewhat unusual in that the required number of bits to send a character varies from character to character. Not including the framing bits or prosigns, Morse code requires 1 bit for its shortest character (E) and several characters require 19 bits. These are examples of Huffman codes, in which the length of the individual characters are controlled to improve the rate at which information can be transmitted. 

Most other codes are fixed-length, such as RTTY's Baudot code (named for its inventor, Emile Baudot) and the common computer character code, ASCII. Figure 2 shows how a character is constructed from mark and space elements in the Baudot code. Each Baudot character consists of five consecutive bits with no intervening elements.

\begin{figure}[H]
    \centering
    \includegraphics[width=0.7\linewidth]{00097.jpeg}
    \caption{Baudot代码字符结构}
    \label{fig:baudot_character_structure}
\end{figure}

Baudot has two framing bits: a start bit, which consists of one bit period of the space element and a stop bit, which consists of at least one bit period of the mark element. Stop bits can be configured to have a minimum length of 1, 1.5, or 2 bit periods so that the decoder functions properly.

ASCII characters can have 7- or 8-character data bits plus the start and stop bits. A parity bit can also be added to indicate whether there are an even or odd number of bits set to 1 in the character. This is a very simple method of error detection — the number of 1 bits in the character are counted and if the count and the parity bit do not agree, there was an error in that character. ASCII (and the extended 16-bit Unicode) character sets are used in digital communication that is not conducted via RTTY or PSK31.

\subsection{Symbols and Bits}

One of the more important measurements or specifications of a digital communication channel is its bit rate. In Morse code, RTTY, and PSK31, during the period when one bit is sent you would hear the pattern or tone for a 1 or a 0. The reciprocal of that period is the bit rate, given in bit/s or bps. For example, if the period for one bit is 1 msec, the channel's bit rate is the reciprocal of  1 / 1 msec = 1000 bps. Take care not to confuse bit/s with byte/s!

Some methods of encoding data into transmittable (and receivable) signals are not restricted to sending only one bit at a time. For example, if the transmitted signal consisted of two tones, each corresponding to a 1 or a 0, then during each bit period four different combinations of two bits could be sent: 00, 01, 10, or 11. Each of these combinations is called a symbol and the rate at which different combinations are sent is called the symbol rate. In this example, if the symbol rate was 300 symbols per second, the bit rate would be 2 × 300 = 600 bps. 

In honor of Baudot, symbol rate is measured in units of baud with 1 baud = 1 symbol / second. Thus, one does not say "baud rate," because baud is already a rate — there is no need to say "symbol rate rate." Just "baud" or "bauds" will do.

Packaging more than one bit into a symbol is common, and amateurs have been experimenting with protocols that use up to 63 different tones (MT63) to send an entire character during one symbol period. This speeds up data transmission at the expense of a wider signal bandwidth and requiring more sophisticated data coding and decoding equipment and software.

One of the more popular modulation methods to transmit symbols representing multiple bits is called quadrature amplitude modulation or QAM. In this method, two carriers are transmitted with a 90° phase shift between them. One is called the I signal and the other is the Q signal. They are turned on and off in various combinations to represent 00, 01, 10, and 11.\textsuperscript{1} The four combinations are plotted on a graph shown in Figure 3, which is called a constellation display. (QAM and the similar QPSK are well-suited to digital signal processing techniques. Advanced modulation schemes having up to 256 different symbols have been used!)

\begin{figure}[H]
    \centering
    \includegraphics[width=0.7\linewidth]{00098.jpeg}
    \caption{QAM星座图}
    \label{fig:qam_constellation}
\end{figure}

Figure 3 shows how the received signal moves from point to point in the constellation. When the signal reaches any of the four points, the receiving decoder generates the combination of 1s and 0s represented by that point. Variations in the trajectories between each of the four corner points are caused by noise and non-linearities in modulation circuits. The "fuzzier" the constellation display, the harder it is for the receiving decoder to make correct decisions about which symbol was actually transmitted.

In order to help the receiver operate better under adverse conditions, further coding schemes such as Viterbi encoding place restrictions on which codes can be transmitted in sequence. This reduces the number of possibilities the receiving decoder must consider and so makes it easier to reject noise and distortion. 

Finally we get to what really matters to a communication system user — data transfer rate. Measured in bytes / second, data transfer rate describes the ability of the entire system to move data from end to end and includes the slowing-down effect of the extra bits sent with each character, characters added to create packets or other structures, protocol timing delays, and so forth. It also includes the speeding-up effect of multiple bits per symbol. Generally speaking, data transfer rates in most systems are anywhere from 1⁄2 to 1⁄10 of the system's bit rate.

\subsection{Fielding a Recommendation}

Readers of Hands-On Radio may remember a few columns that addressed the early experiments establishing the link between electricity and magnetism.\textsuperscript{2} Recently, I discovered a terrific book that takes the subject further — to the electromagnetic field. Faraday, Maxwell, and the Electro- magnetic Field: How Two Men Revolu-tionized Physics, by Nancy Forbes and Basil Mahon explains very well how the idea of a "field" grew out of Faraday's discoveries and suggestions to be given its mathematical description by Maxwell. Not only is it an interesting tale of technical history but using minimal math, it provides some good background on what a field is — a hard thing to understand clearly as one begins learning about radio.

\section{注释}

\begin{enumerate}
    \item I-Q modulation was discussed in Experiment 128. All previous Hands-On Radio experiments are available to ARRL members at \href{http://www.arrl.org/hands-on-radio}{www.arrl.org/hands-on-radio}.
    \item Experiments 117 and 118, "Laying Down the Laws."
\end{enumerate}

\chapter{实验 #154:功率因数和相位角}
\section{英文原文}

Someone just learning electronics first learns Ohm's Law relating voltage, current, and resistance: R = E / I. The next step is usually to learn how to calculate power: P = E × I, but this equation must be qualified by the term, "in a resistive circuit." Inductive and capacitive circuits are somehow different, we learn. Why would inductance and capacitance have anything to do with power?

A similar conversation takes place when the terms "reactance" and "impedance" are introduced. Understanding resistance is fairly intuitive — it's like electrical friction — but this reactance stuff can seem a little odd at first: "Okay, capacitive reactance is like a spring and inductive reactance is like a flywheel. I think I get that. But energy is energy, isn't it? Why does this affect power, which is E × I? What is power factor? Phase angle?"

Understanding why inductance and capacitance and their abilities to store energy affect the relationship between power and energy can feel like deep water. In this column, we'll explore these electrical fundamentals.

\subsection{Power at Work}

Before jumping in, let's start from solid ground — power and energy in a resistor. When voltage is applied to a resistor, electrons begin to move (i.e. current) through whatever material the resistor is made of; carbon or metal film, nickel-chromium (nichrome) wire, a chunk of metal oxide, etc. The actual speed of the electrons is surprisingly low\textsuperscript{1} but there are an incredible number of them, even in the thinnest conductor. Each electron collides with the atoms making up the conductor and transfers some of its energy to the atom, causing it to vibrate, which is heat. Because the electron moves through the material in response to the applied voltage, the source of the electromotive force is said to have done work in the physics sense.\textsuperscript{2} In this case, the work is heating up the material.

Work (W) is measured in the same units as energy (joules) and can be thought of as "expended energy." The rate at which work is performed per unit of time (t) is power: P = W / t and is measured in watts if t is measured in seconds. One watt (also abbreviated W) is equal to one joule of work done (or energy expended) per second. Higher power means more work has been done in a specific period of time or that a specific amount of work done in a shorter period of time. What is important is that the electron moves through the material at the same time the voltage is applied.

\subsection{Power and Phase}

If the voltage applied to a resistor is a steady value, the resulting direct current (dc) will be a steady value, too. At each moment of time, you can calculate the instantaneous power by multiplying voltage and current and that, too, will be a steady value. If the applied voltage periodically reverses, creating alternating current (ac), the instantaneous power dissipated by the resistor as heat will also vary. 

Figure 1 shows voltage, current, and instantaneous power in an 8 Ω resistor when a 4 V sine wave is applied.\textsuperscript{3} It is important to note that when an ac voltage is applied to a resistor, voltage and current have exactly the same phase; reaching zero, maximum, and minimum at the same time. Instantaneous power (E × I) in this case is always positive even when both voltage and current are negative to indicate reversal. 

\begin{figure}[H]
    \centering
    \includegraphics[width=0.7\linewidth]{00099.jpeg}
    \caption{电阻中的电压、电流和瞬时功率}
    \label{fig:voltage_current_power}
\end{figure}

What happens if an ac voltage and current are not precisely in phase? First, why wouldn't they be precisely in phase? Well, if the component in the circuit stores and returns some of the energy instead of dissipating it as heat, that alters the timing between the current and voltage waveforms. For example, if an ac voltage is applied to the capacitive circuit of Figure 2A, the resulting current leads the applied voltage by 90°. (For an explanation, see the "Electrical Fundamentals" chapter of The ARRL Handbook or the section "Reactance and Impedance" in the General Class License Manual.\textsuperscript{4,5,6}) 

\begin{figure}[H]
    \centering
    \includegraphics[width=0.7\linewidth]{00100.jpeg}
    \caption{电容和电感电路}
    \label{fig:capacitive_inductive_circuits}
\end{figure}

Instead of instantaneous power always being positive as in the resistive circuit, it is positive for half the cycle (0° to 90° and 180° to 270°) and negative for the remaining half-cycle. When we add up instantaneous power throughout the cycle, the result is zero and no net work has been done at all! Figure 2B shows the complementary situation for an inductive circuit — the net result is the same. If total work is zero, then total power is also zero and no energy has been consumed. 

\subsection{Minding the Ps and Qs}

"But, but, but…" I hear you exclaim, "sure, total power is zero over the whole cycle but during half the cycle, power is being consumed! What happened to that power?" An excellent question! During the positive power half-cycle, energy is not being consumed or dissipated, it is being stored in an electric field (for a capacitor) or in a magnetic field (for an inductor). During the negative-power half-cycles, energy is returned to the source.

Power for which the voltage and current are in phase is called real power because the power does "real work" and is not stored or returned. Real power is labeled P and is measured in watts. Power for which the voltage and current are 90° out of phase (see Figure 2) is called reactive power because of the reactance creating the phase shift. Reactive power is labeled Q and is measured in volt-amperes reactive or VAR. 

If a circuit contains both resistance and reactance, also called a reactive load, the resulting instantaneous power waveform is made up of both real and reactive power. Figure 3 uses complex numbers to show the relationship between real and reactive power in such a circuit. Q is drawn parallel to +90° imaginary axis if the reactance is inductive (as in Figure 3) or at –90° if the reactance is capacitive.

\begin{figure}[H]
    \centering
    \includegraphics[width=0.7\linewidth]{00101.jpeg}
    \caption{复功率关系}
    \label{fig:complex_power_relationship}
\end{figure}

The addition of the values for real (P) and reactive power (Q) results in the complex number S = P + jQ, representing complex power. The magnitude of complex power, |S| = √(P² + Q²) is called apparent power, which is measured in volt-amperes or VA. Apparent power is also equal to the magnitude of the voltage and current measured separately and multiplied together without regard for phase differences. 

\subsection{Power Factor and Phase Angle}

We've finally arrived in the deep water! The phase angle, θ, from P to S is the amount of phase difference between the applied voltage and the resulting current in the circuit. Power factor, PF, is the cosine of the phase angle, θ, and so is always in the range of 0 to 1.

A PF of 1 means the voltage and current are exactly in phase and all power is real power. PF becomes gradually smaller as Q increases, eventually reaching 0 when all of the power is Q with no P. It doesn't matter whether Q is inductive (θ > 0) or capacitive (θ < 0) because the cosine of θ is the same for positive and negative angles.

Power system engineers use PF to specify how much reactive power is present in a particular circuit and they prefer PF to be close to 1. Why? If PF gets smaller for a given amount of real power, P, that means reactive power, Q, must be increasing. Oversimplifying somewhat, that means more volts and amps are dedicated to reactive power and not doing any real work. Those reactive volts and amps stress insulation and cause heating from I²R losses so wires have to be bigger and insulation thicker. Keeping PF close to 1 minimizes the bad effects of Q while the power system delivers the required amount of P to customers. 

This might be starting to sound familiar to hams. RF designers use θ when calculating impedances and designing matching circuits to make your transmitter happy with a 50 + j 0 Ω impedance — just another way of saying PF = 1 and θ = 0°! So you see, power factor and phase angle are just another way of discussing the relationship between voltage and current in an ac circuit, whether at 60 Hz or in our RF bands.

\section{注释}

\begin{enumerate}
    \item For typical currents in copper, the progress of a single electron through a wire (drift velocity) is much less than 1 mm/sec (\href{http://hyperphysics.phy-astr.gsu.edu/hbase/electric/ohmmic.html#c2}{hyperphysics.phy-astr.gsu.edu/hbase/electric/ohmmic.html#c2}).
    \item There are various definitions of work depending on the system involved. The precise definition of electrical work is discussed at \href{http://www.physicsclassroom.com/calcpad/energy}{www.physicsclassroom.com/calcpad/energy}.
    \item Unchanging voltage is referred to as a "dc voltage" and a regularly-reversing voltage as an "ac voltage" even if current is zero.
    \item The ARRL Handbook, 93rd edition, ARRL, 2015.
    \item General Class License Manual, 8th edition, ARRL, 2014.
    \item All previous "Hands-On Radio" columns are available to ARRL members at \href{http://www.arrl.org/hands-on-radio}{www.arrl.org/hands-on-radio}.
\end{enumerate}

\chapter{实验 #170:噪声系数}
\section{英文原文}

On our MF and lower HF bands, the received signal-to-noise ratio (SNR) is dominated by galactic noise from 15 to 30 MHz at night by atmospheric noise from storms and man-made sources below 15 MHz. 

The type of noise we are mostly concerned with in sensitive RF receiving systems is thermal noise, also called Johnson-Nyquist noise. There are many other types of noise — shot noise, flicker noise, and even popcorn noise — that are generated inside our electronic devices. From the external world, atmospheric noise is accompanied at VHF and above by galactic noise and sun noise.

Thermal noise is generated in all conducting materials by the random vibration of free electrons due to their thermal energy. The higher the material's temperature, the larger the vibrations of the free electrons become. This motion of charge creates a noise voltage in any conductor not at absolute zero.

Based on material\textsuperscript{1} by Paul Wade, W1GHZ, in The ARRL Handbook, the basics of thermal noise are as follows: Every resistance (and all conductors have resistance) generates a root-mean-square (RMS) noise voltage:

e = √(4kTRB)

where k is Boltzmann's constant (1.38 × 10\textsuperscript{–23} joules / K), T is the absolute temperature in kelvin (K, which is equal to the temperature in Celsius plus 273), R is the resistance, and B is the bandwidth in hertz. Converting to power, e² / 4R, the noise power generated by the resistor is: 

P\textsubscript{n} = kTB (watts)

This independence of frequency is why thermal noise is called white noise and has a power that depends only on temperature.

Note that all resistances at the same temperature generate the same noise power. Similarly, if two noise sources generate the same power in the same bandwidth, they are said to have the same noise temperature, T\textsubscript{n}. This is the temperature at which a resistor at the same temperature would generate the same noise power as the source, whether it is an electronic circuit, a cable, or an antenna. 

The amount of power per unit of bandwidth is called power density or spectral density and is equal to kT watts per Hz. Because power is proportional to the square of voltage, the corresponding voltage density or spectral density is measured in volts / √Hz, spoken as "volts per root hertz." 

Calculating noise power density at 290 K (290 K = 16.9°C = 62.6°F) gives:

P\textsubscript{n} = (1.38 × 10\textsuperscript{–23} × 290) B = 400 × 10\textsuperscript{–23} B

Multiply by the bandwidth in hertz to get the available noise power at 290 K. The choice of 290 K is simply for convenient calculations because P\textsubscript{n} is 400 B at that temperature. Converting to dBm by calculating 10 log(P\textsubscript{n}), we get –174 dBm / Hz. 

\subsection{Measuring Noise with Noise Figure}

Continuing to build on W1GHZ's material in The ARRL Handbook, all amplifiers add additional noise to the noise present at their input. The input noise per unit of bandwidth is N\textsubscript{i} = kT\textsubscript{g}, where T\textsubscript{g} is the noise temperature at the amplifier's input. Amplified by power gain, G, the output noise power is kT\textsubscript{g}G. The noise power added by the amplifier, kT\textsubscript{n}, is then added to the amplified input noise to produce a total output noise, N\textsubscript{o} = kT\textsubscript{g}G + kT\textsubscript{n} = k(T\textsubscript{g}G + T\textsubscript{n}).

We can model the amplifier as ideal and noise-free and add a noise-generating resistor of temperature T\textsubscript{e} = T\textsubscript{n} / G at the input. In this way, all sources of noise can be treated as inputs to the amplifier, as illustrated by Figure 1. Substituting for T\textsubscript{n} in the previous equation, the output noise is then N\textsubscript{o} = kG (T\textsubscript{g} + T\textsubscript{e}).

\begin{figure}[H]
    \centering
    \includegraphics[width=0.7\linewidth]{00102.jpeg}
    \caption{放大器噪声模型}
    \label{fig:amplifier_noise_model}
\end{figure}

The noise added by an amplifier can then be represented as kGT\textsubscript{e}, which is the amplifier's noise temperature amplified by the amplifier's gain. T\textsubscript{e} is sometimes referred to as excess temperature. 

Because amplifiers have many different values of gain, there needs to be a way to compare their noise performance without a bunch of calculations. The answer is noise factor and noise figure.

The noise factor F of an amplifier is the ratio of the total noise output of an amplifier with an input T\textsubscript{g} of 290 K to the noise output of an equivalent noise-free amplifier. The easiest way to do this is to use noise temperatures: F = 1 + T\textsubscript{e} / T\textsubscript{g}. It is often more convenient to work with noise figure, NF, the logarithm of noise factor expressed in dB: 

NF = 10 log F and F = log\textsuperscript{–1}(NF / 10)

If the signal-to-noise ratio (SNR) in decibels is known at the input and output: 

NF = SNR\textsubscript{in} – SNR\textsubscript{out} or SNR\textsubscript{out} = SNR\textsubscript{in} – NF

Noise figure is sometimes stated as input noise figure to emphasize that all noise sources and noise contributions are converted to an equivalent set of noise sources at the input of a noiseless device. In this way, noise performance can be compared on equal terms across a wide variety of devices. It also makes comparing their relative contributions to output noise much easier. 

\subsection{Location, Location, Location}

When more than one noise-generating component is connected in series or cascade, they all contribute differently to the overall output noise level and SNR. Even a simple receiving system consists of three such components — antenna, feed line, and receiver — with each adding to, or even amplifying, the noise at its input. The Friis equation takes into account the gain or loss of each component and gives an overall system noise factor for N devices connected in series.

\begin{figure}[H]
    \centering
    \includegraphics[width=0.7\linewidth]{00103.jpeg}
    \caption{Friis方程}
    \label{fig:friis_equation}
\end{figure}

\begin{figure}[H]
    \centering
    \includegraphics[width=0.7\linewidth]{00104.jpeg}
    \caption{系统噪声系数}
    \label{fig:system_noise_factor}
\end{figure}

where device 1 is the one at the input to the system. Clearly, if the gain of the first stage, G\textsubscript{1}, is large, then the noise contributions of the succeeding stages become too small to be significant. In addition, the noise temperature of the first stage is the largest contributor to the overall system noise because it is amplified by all remaining stages.

The important thing to learn from the Friis equation is that noise-reduction efforts should be made at the input to the system. A good example is the question of where to put a preamp in an antenna system. Figure 2 shows the difference between placing the preamp at the input to a lossy feed line and placing it at the receiver input.

\begin{figure}[H]
    \centering
    \includegraphics[width=0.7\linewidth]{00105.jpeg}
    \caption{前置放大器位置比较}
    \label{fig:preamp_location_comparison}
\end{figure}

You can hear the Friis equation in action with a simple experiment using a VHF or UHF FM transceiver, a long length of coax or an attenuator, and a preamplifier.\textsuperscript{2,3,4} 

Choose some coax with about 3 – 6 dB of loss — 100 feet of RG-58 or RG-8X will do nicely at VHF or UHF. Connect the coax between a small antenna and your transceiver, then tune in a weak (distant) repeater, which is received with a noisy signal at your location. (If you use an attenuator, just add attenuation until a local repeater signal becomes noisy.) Add the preamp at the antenna and observe how much the SNR improves. Then move it to the other end of the coax, connecting it directly to the transceiver input, and observe SNR again. You will find that placing the high-gain preamp closer to the input of your receiving system results in a better SNR.

\section{注释}

\begin{enumerate}
    \item Available from your ARRL dealer, or from the ARRL Store, ARRL Item no. 0628. Telephone toll-free in the US 888-277-5289 or 860-594-0355, fax 860-594-0303, \href{http://www.arrl.org/shop}{www.arrl.org/shop}, \href{mailto:pubsales@arrl.org}{pubsales@arrl.org}.
    \item \href{http://m2inc.com}{m2inc.com}
    \item \href{http://amsat-uk.org/info/g0mrf-144-mhz-preamp-kit}{amsat-uk.org/info/g0mrf-144-mhz-preamp-kit}
    \item \href{http://www.w1ghz.org/small_proj/Simple_Cheap_MMIC_Preamps.pdf}{www.w1ghz.org/small_proj/Simple_Cheap_MMIC_Preamps.pdf}
\end{enumerate}

\chapter{实验 #175:不同金属}
\section{英文原文}

When putting up my station with tower maven Don Daso, K4ZA, I got an earful of good advice about what happens when different types of metal are clamped together out in the weather. Hint — nothing good! Don gave me some good tips on how to manage the situation at reasonable cost.\textsuperscript{1} Because metal-to-metal connections are so common in amateur antenna and ground systems, I thought it was a great candidate for a "Hands-On Radio" column.

\subsection{Electrochemistry at Work}

Atoms that have an affinity for electrons are called oxidizing agents. Similarly, atoms that donate electrons are called reducing agents. There is a whole range of strengths for oxidizing and reducing agents. The greater the difference between the material's relative affinity for electrons, the stronger the reaction between them can be, with atoms of one material acquiring electrons from the donating material in an oxidation-reduction reaction. The relative difference in strength is called electropotential. 

If two materials with different electropotentials are in contact with each other, or there is a conducting path between them, electrons will move from the reducing to the oxidizing agent. This results in a change in the chemical makeup of the two materials. When the reaction occurs between parts of our antenna system, we apply a more descriptive word — corrosion. Corrosion is generally bad because the result of the reaction is generally a compound with less strength and conductivity than the original metal. Rust is one such corrosion product and so is that crusty stuff that builds up on your car battery terminal. Corrosion can proceed until the connection no longer conducts current or fails mechanically, i.e. the wire, antenna, or tower falls down.

Because not everything is made from the same metal, contacts between dissimilar metals are very common. If the metals are kept dry, corrosion proceeds slowly. But outside, the water from rain and condensation collects in the junction. This creates the conducting path, enabling electrons to flow and cause corrosion.

\subsection{The Galvanic Series and Anodic Index}

It is useful to know which materials can "get along" and which cannot. Chemists developed a list called the galvanic series, ranking materials from the strongest electron donors to the most inert material. The farther apart the two materials are in the series, the stronger the reaction between them will be. The more active material will act as the cathode, which donates the electrons. (Remember that chemistry is based on electronic current — the flow of electrons — and not conventional current, which is the flow of positive charge that is used in radio electronics. Confusing? Yes.)

Electropotential can be measured with a voltmeter. Measured in volts, it is what Alessandro Volta discovered and made use of to create the voltaic pile, known today as a battery. Because electropotential is a relative difference, it is measured as a voltage with respect to some reference material. That voltage is a metal's anodic index with respect to gold as the reference material. Table 1 shows the anodic series for materials common in our stations.

Because no two metals have the same anodic index value, won't there always be some corrosion when there is contact between dissimilar metals? Yes, but very small differences in anodic index result in very slow corrosion. If the environment is harsh, such as exposed to the weather or salt spray, limiting the difference in anodic index to 0.15 V or less means the corrosion will be manageable. In normal conditions, up to 0.25 V difference can be tolerated. (This is discussed in more detail at corrosion-doctors.org/ Definitions/galvanic-series.htm.)

\subsection{Connecting to Galvanized Surfaces}

Because zinc is an extremely active material (which is why it is so effective at protecting steel and iron), connecting copper wire or bronze clamps directly to a galvanized surface creates a very strong corrosion potential. What should you do?

Although there are anti-corrosion compounds, continuous exposure to weather will eventually wash them away. Instead, an inexpensive method of protecting the connection is to place a thin piece of 300-series stainless steel between the galvanized surface and the clamp or wire. 24-gauge (0.025-inch) shim stock is sufficient and acts to slow down the corrosion process while maintaining good electrical connectivity for lightning protection and RF connections.\textsuperscript{2} Figure 1 shows a photo of a ground clamp on a galvanized tower leg with a shim between the clamp and the leg. Thin shim can be cut with heavy scissors and bent by hand. Another option is to use a tower leg clamp designed specifically for this application, such as the Rohn R-CPC1/1.25 (available from several vendors).

\begin{figure}[H]
    \centering
    \includegraphics[width=0.7\linewidth]{00106.jpeg}
    \caption{接地夹与不锈钢垫片}
    \label{fig:ground_clamp_with_shim}
\end{figure}

Some areas have corrosive soils and other environmental characteristics that make corrosion an ongoing issue with towers. Tony Fisher's, K1KP, article in the October 2010 issue of QST — "Is Your Tower Still Safe?" — discusses setting up a sacrificial anode system that will protect your tower from corrosion. You can also check with local contractors to see if corrosion is a special problem in your area.

\subsection{Making Your Own Galvanic Cell}

This is all very academic until you see it for yourself. The easy and non-destructive way is to make your own galvanic cell and take some measurements.

Here's what you need: a plastic or glass jar (12 – 16 oz. capacity), a tablespoon of kitchen salt, a multimeter, two clip leads, and some test metals. I used common materials: copper, aluminum, stainless steel, galvanized steel, and 63/37 lead-tin solder. Any metal you can carry in the palm of your hand is safe to test, including coins and household metals. Remove any oxidation or grease from the surface, so that it's shiny and clean.

Dissolve the salt in enough water to fill about 3⁄4 of the jar. Set the multimeter to measure voltage (the 2 V scale will work best) and attach clip leads to the probes. (This keeps your probes from getting salt water on them.) Set the jar on an absorbent towel (not a good one!) and put the copper wire in the water with one end held out of the water by the clip lead or jar lip. (Keep the clip lead out of the water to keep its surface clean, too.) Now attach the other clip lead to one of the remaining pieces of metal placed in the water, but not touching the copper wire. You can see the experimental setup in Figure 2. The voltage shown on the meter will be approximately the difference in anodic potential between the two metals in Table 1. (The exact voltage will probably vary from Table 1 values because the metal may be an alloy or there may be surface oxidation that alters the chemistry a bit.)

\begin{figure}[H]
    \centering
    \includegraphics[width=0.7\linewidth]{00107.jpeg}
    \caption{实验装置}
    \label{fig:experimental_setup}
\end{figure}

\begin{figure}[H]
    \centering
    \includegraphics[width=0.7\linewidth]{00108.jpeg}
    \caption{测量电压}
    \label{fig:voltage_measurement}
\end{figure}

\section{注释}

\begin{enumerate}
    \item D. Daso, K4ZA, Antenna Towers for Radio Amateurs, Appendix B, ARRL, 2010.
    \item \href{https://www.galvanizeit.org/images/uploads/drGalv/Stainless_Steel_in_Contact_with_Galvanized_Steel.pdf}{https://www.galvanizeit.org/images/uploads/drGalv/Stainless_Steel_in_Contact_with_Galvanized_Steel.pdf}
\end{enumerate}

\chapter{实验 #178-179:麦克斯韦方程组}
\section{英文原文}

\subsection{实验 #178:麦克斯韦方程组 — Grad, Div, Curl}

Given their fundamental nature, it is natural to think of Maxwell's equations as describing laws of nature. They do, but it was not Maxwell who discovered them. As we learned in Experiments #117 and 118, those insights came from Faraday, Gauss, and Ampere.\textsuperscript{1} What Maxwell contributed was to see the relationships of electric and magnetic fields, voltage and current, as different components of a single natural phenomenon — electromagnetism. He simplified the relationships down to the level of "first principles" that aren't derived from any other more fundamental ideas — these are bedrock descriptions of the universe.

In the process, he came to understand that electric charge and magnetism are deeply related. After Maxwell, it became apparent that they are just different ways in which electromagnetic energy interacts with matter as it moves through space. An electron (or any electric charge) can respond to electric fields or magnetic fields. The same electron generates an electric field or, if in motion, a magnetic field.

The key is "motion." If anything could be considered "Maxwell's Law" it would be his modification to Ampere's Law. He made that equation symmetrical with Faraday's Law so that a time-changing magnetic field was linked to an induced electromotive force or voltage. This was the true spark of genius (so to speak), going beyond Faraday's Law to suggest that a time-changing electric field can produce a current. 

Maxwell then made a second leap from "time-changing" to "moving," and got a wave. (Faraday had suggested there might be waves associated with induction, but hadn't incorporated electric fields into the idea.) As they say, this changed everything. As Einstein himself acknowledged, this equivalence of motion and changes with time lies at the heart of relativity and its equivalence of space and time. (Pretty deep stuff for a ham radio magazine, I must say.)

Maxwell's original 20 equations were difficult to understand. We owe their current form to Oliver Heaviside, who simplified them to four equations using modern notation in 1885.\textsuperscript{2,3}

Table 1 explains as text what each of Maxwell's equations says and then states them mathematically. (Rautio's explanations are more detailed and the equations shown here are somewhat over-simplified.) This is the mountain. The following sections cover three basic math concepts that are used to express the equations. These are your climbing gear.

\begin{figure}[H]
    \centering
    \includegraphics[width=0.7\linewidth]{00109.jpeg}
    \caption{麦克斯韦方程组}
    \label{fig:maxwell_equations}
\end{figure}

\subsubsection{Gradient (∇)}

Let's start with gradient, which is pretty easy to visualize. Gradient, represented by the ∇ or nabla symbol, is a measure of how fast some parameter changes with respect to some other parameter. For example, a steep hill has a large gradient of height with respect to distance. (The word "grade" springs from the same root as "gradient.")

In ham radio, one gradient that concerns us is a change in voltage over distance. An insulator's breakdown voltage — the maximum gradient the insulator can withstand — is expressed in volts per inch. This is why high-voltage circuits and components have rounded or smooth edges and points — to reduce the voltage gradient near these surfaces, avoiding arcs and corona. 

Figure 1 is an illustration of gradient on a topographic map, which measures gravitational potential, also known as "elevation." The two blue lines, A and B, each represent a horizontal distance of 2,000 feet. Along which line is the gradient of elevation (vertical distance per horizontal distance) the greatest from end to end? The heavy brown contour lines are spaced 100 feet apart, so the net gradient along A from end to end is 200 feet in 2,000 feet or 0.1 feet per foot. B touches six heavy brown lines for a gradient of 600 / 2,000 = 0.3 feet per foot. The gradient symbol in Maxwell's equations includes the gradient in all three dimensions, not just two as in this map.

\begin{figure}[H]
    \centering
    \includegraphics[width=0.7\linewidth]{00110.jpeg}
    \caption{地形图上的梯度}
    \label{fig:gradient_topographic_map}
\end{figure}

\subsubsection{Divergence (•)}

Divergence, represented by the • symbol, can also be illustrated on a topographic map. Divergence describes whether a value, such as gravitational potential (elevation) or electrical potential (voltage), is increasing or decreasing through a curve or across a surface. Start with the contour circuit C, which surrounds the central peak, and imagine a rolling ball as your "gravity-o-meter." What would a ball do if placed on the contour circuit line? Everywhere around circuit C, gravitational potential increases to the "inside" and decreases "outside," so the ball would roll away from the central peak. We would say there is a high positive divergence in gravitational potential (elevation) across the circuit. (If the circuit was drawn around a sinkhole, there would be a high negative divergence.) 

The case of the circuit labeled D is not as simple. Part of the circle is on the slope of a nearby peak, two parts are in separate parts of a valley, and some is on the slopes of the central peak. Depending on location, a ball dropped on the circuit would roll toward the interior (1 and 6 o'clock positions, negative divergence), away from the interior (4 o'clock, positive divergence), or along the circuit (10 o'clock, zero divergence). Figuring out whether net divergence was positive or negative would require you to sum it up at each point around the circuit. Mathematically, this is an integration around the whole circuit, and it is shown as an integration symbol with a small circle in the middle (see Rautio's website in Note 3). Like the gradient, divergence in Maxwell's equations includes all three dimensions.

\subsubsection{Curl (×)}

The final tool in the set is curl, and that is something we can't show on a topographic map. Curl is derived from circulation, which could be understood as the push from gravity along a closed path such as one of our contour circles. Curl is denoted by the × symbol, which is also used to represent the mathematical cross product of two vectors.

Along circuit C in Figure 1, you would get no push anywhere because the gravitational potential at each point (elevation) is the same. Along circuit D, the push might be in one direction then in the other, but around the whole circuit, the net circulation is zero. Otherwise you could go up or down forever like an Escher staircase.

Curl is the amount of circulation per unit of area, and would be experienced as a twisting or turning force. You can experience curl for yourself. Anyone with boating experience has experienced curl when the current vectors at one end of the vessel are stronger (or have a different direction) than at the other. The twisting force shows the curl of the current's vector field across the surface of the water. Whirlpools and hurricanes also illustrate curl.\textsuperscript{4}

Gradient, divergence, and curl play a role in our day-to-day radio operating and are illustrated by the online video in Note 5. We'll take a closer look at the equations and find out how they lead to electromagnetic waves in the next column.

\subsection{实验 #179:麦克斯韦方程组 — 波的产生}

Using the mathematical equations of divergence, gradient, and curl from last month, let's find out where radio waves come from.\textsuperscript{6}

Maxwell's first equation, also known as Gauss' law for electric fields (∇•E = q\textsubscript{V}/ε\textsubscript{0}), tells us that if we measure where an electric field is pointing and how strong it is (the E-field's divergence or ∇•E) all around some arbitrary point, then we can tell how much electric charge, q\textsubscript{V}, is at that point. In a very oversimplified way, we can think of the electric field's divergence as mapping out a distortion of otherwise-neutral space. Multiplying the distortion by permittivity, ε\textsubscript{0}, the "electric stretchiness" of space, tells us the amount of charge, q\textsubscript{V}, it takes to produce that distortion. 

In more visual terms, you can "put a bag" around a point as in Figure 1A, add up the electric field everywhere it crosses the surface of that bag, and deduce how much electric charge must be inside the bag. There must be an equivalent magnetic charge that produces its own type of distortion we experience as, H, the magnetic field, right? No. 

\begin{figure}[H]
    \centering
    \includegraphics[width=0.7\linewidth]{00111.jpeg}
    \caption{高斯定律示意图}
    \label{fig:gauss_law_illustration}
\end{figure}

Maxwell's second equation, also known as Gauss' law for magnetic fields (∇•H/μ\textsubscript{0} = 0), tells us there can never be any "magnetic charge" inside the bag. Where the electric field can be visualized as field lines from an electric charge streaming off into space, the equivalent lines of the magnetic field are all closed loops with no beginning or end. Every loop crossing the bag's surface both enters and leaves, so the net total is always zero, as in Figure 1B. (If the loop is completely inside or outside the bag, it isn't counted.)

\subsubsection{Magnetism — Charge in Motion}

If there are no magnetic charges, what generates a magnetic field? This is where Maxwell's addition to Ampere's law, his fourth equation, comes in. It has two terms on the right-hand side: I + m\textsubscript{0}(dE/dt). The first is current, I, which is moving charge, at the point where the magnetic field is created. The second is the rate at which the electric field at that point is changing multiplied by permittivity. To add these two quantities together, they have to be of the same type with the same units. The first term is unambiguous — it's current — so the second quantity must also be current, or at least something equivalent to current. Maxwell called this term displacement current.

Displacement current comes from a time-changing electric field — how can that be created? Maxwell's third equation (Faraday's law) says that a time-changing magnetic field will do the job, but we're trying to create the magnetic field in the first place. According to the four equations, the only other way is to change the amount of electric charge at the point where we're trying to create the field.

Regarding the change in the amount of charge in our "bag of charge," the Law of Conservation of Charge is pretty clear — charge cannot be created or destroyed. If I want to change the amount of charge, I have to move some charge into or out of the bag, and moving charge is current. To shorten a really long story, the only way to create a time-changing electric field is by moving or displacing charge, which is current. That's how the electric field in a capacitor is created, by moving electrons off one electrode and onto the other. What we have just learned is that magnetic fields are the effect of electric charge in motion.

\subsubsection{The Wave Equation}

We're not quite there yet. We need one more equation. (Serious students of electromagnetism will want to dive into Fleisch's book on Maxwell's equations.\textsuperscript{7}) The equation below is the ideal wave equation\textsuperscript{8} (without loss or non-linearities), and it describes all waves:

\begin{figure}[H]
    \centering
    \includegraphics[width=0.7\linewidth]{00112.jpeg}
    \caption{理想波动方程}
    \label{fig:ideal_wave_equation}
\end{figure}

Let's break this into digestible pieces. A is whatever field in which the wave is created. We use E or H fields for radio waves. The right-hand side's first term (1/v²) is just the reciprocal of the velocity of the wave squared. In the second term, the superscript "2" above the d and t means "do this twice," not "squared." If the rate of change of our field is dA/dt, then d²A/dt² means "the rate of change of the rate of change." (For example, acceleration is the rate of change of velocity which is the rate of change of position. Hold that thought.) The intimidating term on the left (∇²A) describes how the gradient of the field in all three dimensions is changing — the gradient of the gradient, basically.

Reading from right to left, what the equation tells us is, Changes in the rate at which the field A varies, create changes in the field's strength throughout space. The changes are inversely proportional to the speed at which the changes propagate. Solutions to that equation, such as sine waves, describe wave motion, whether as a plucked string, a ripple across water, or a radio wave in space.

\subsubsection{Coupling E and H}

The left-hand side of Maxwell's third and fourth equations don't describe the E and H fields directly. They give us the curl of the fields (∇×E and ∇×H). I don't have room to show the complete process (it's in Fleisch's book), but with two mathematical tools (Stoke's theorem and the divergence theorem), we can get from the third and fourth equations to a wave equation that replaces A in the wave equation with E or μ\textsubscript{0}H.\textsuperscript{9,10} 

We are really close now — take three big steps then one final leap. Step 1: Equations three and four show that time-changing E and H fields generate each other (and if you follow the vector math for waves, E and H are at right angles in free space, as shown in Figure 2). 

\begin{figure}[H]
    \centering
    \includegraphics[width=0.7\linewidth]{00113.jpeg}
    \caption{电磁波形}
    \label{fig:electromagnetic_wave}
\end{figure}

Step 2: The only way to get those time-changing fields is to move charge — change its position with time — and the only way for that right-hand term in the wave equation to not be zero is for the charge to be accelerating (or decelerating). 

Step 3: Motion is a change in position relative to an observer (i.e., me), so the resulting changes in the fields also appear to be moving from the perspective of the observer.

And here we are, at last: The wave equation describes those changing fields, and we have our electromagnetic radio waves, generated by accelerating and decelerating charge, better known as ac current.

The coupling of the electric and magnetic fields creating each other explains why an electromagnetic wave is more than just an electric field and a magnetic field that just happen to be at the same place and just happen to be at right angles. They aren't separate things at all. The E and H fields are two aspects of the same thing — an electromagnetic wave moving through space.\textsuperscript{11}

Even with all the math, we're still not all that clear on what's really going on. From Jim Rautio, AJ3K:

"We talk about electric fields and magnetic fields as though they are real. Sure, you have seen iron filings move around, but no one has ever seen or touched a field. Like lines of force, fields are just a mathematical convenience that allows us to predict what happens when we do an experiment."\textsuperscript{12} 

Nor does anyone know what a photon or electrical charge actually is. We just know how to observe their effect on the space we live in.

Having completed our journey through the Land of Maxwell, this is a great place to end "Hands-On Radio" — at the headwaters of all radio waves, whose magic is at the heart of ham radio and the people who enjoy it. 73!

\section{注释}

\begin{enumerate}
    \item All previous "Hands-On Radio" experiments are available to ARRL members at \href{http://www.arrl.org/hands-on-radio}{www.arrl.org/hands-on-radio}.
    \item \href{http://en.wikipedia.org/wiki/Oliver_Heaviside}{en.wikipedia.org/wiki/Oliver_Heaviside} and Paul Nahin, Oliver Heaviside: The Life, Work, and Times of an Electrical Genius of the Victorian Age, IEEE, 2002.
    \item J. Rautio, AJ3K, "The Long Road to Maxwell's Equations," Dec. 2014, IEEE Spectrum, pp. 36 – 40, 54 – 56, and \href{http://www.microwaves101.com/encyclopedias/maxwell-s-equations}{www.microwaves101.com/encyclopedias/maxwell-s-equations}.
    \item \href{http://earth.nullschool.net}{earth.nullschool.net}
    \item \href{http://www.youtube.com/watch?v=qOcFJKQPZfo}{www.youtube.com/watch?v=qOcFJKQPZfo}
    \item All previous "Hands-On Radio" experiments are available to ARRL members at \href{http://www.arrl.org/hands-on-radio}{www.arrl.org/hands-on-radio}.
    \item D. Fleisch, The Student's Guide to Maxwell's Equations, Cambridge University Press, 2008.
    \item The correct form uses partial differentials, δ, but for simplicity, the simple derivative symbol is used here. A full discussion of the wave equation is well beyond the scope of this overview.
    \item The relationship between B and H is explained at \href{http://www.physicsforums.com/threads/in-magnetism-what-is-the-difference-between-the-b-and-h-fields.370525}{www.physicsforums.com/threads/in-magnetism-what-is-the-difference-between-the-b-and-h-fields.370525}.
    \item It was not lost on Maxwell that the same process also shows v, the wave's velocity, is equal to 1/√(ε\textsubscript{0}μ\textsubscript{0}), which just happens to also be c, the speed of light.
    \item See the electromagnetic wave animation at \href{http://commons.wikimedia.org/wiki/File:EM-Wave.gif}{commons.wikimedia.org/wiki/File:EM-Wave.gif}.
    \item J. Rautio, AJ3K, "The Long Road to Maxwell's Equations," Dec. 2014, IEEE Spectrum, pp. 36 – 40, 54 – 56, and \href{http://www.microwaves101.com/encyclopedias/maxwell-s-equations}{www.microwaves101.com/encyclopedias/maxwell-s-equations}.
\end{enumerate}

\part{电子元件}
\chapter{实验 #122, 123:电池特性,第 1 和 2 部分}

% 内容将从 EPUB 中提取

\chapter{实验 #142:RF 下的电感器}
\section{英文原文}
In one of the many strange-but-true things that happen at RF, that innocent-looking coil of wire or cable has more than one personality as the frequency changes! This month we'll explore the wacky world of inductors and learn how to use a neglected function of a common antenna analyzer along the way.

\subsection{Inductor Basics}
This formula for the inductance \(L\) of a basic single-layer, air-wound inductor has been in articles and handbooks for generations:

\begin{figure}[htbp]
    \centering
    \includegraphics{../epub_extracted/images/00122.jpeg}
    \caption{Inductance Formula}
    \label{fig:inductance-formula}
\end{figure}

Where \(d\) is the diameter of the coil in inches from wire center to wire center, \(\ell\) is the coil's length in inches, and \(n\) is the number of turns. This approximation works reasonably well but there are innumerable corrections. The formula makes several assumptions that the coil: is made from wire that is not too thick; is not too long or too short; has leads not too long; and has a reasonable \emph{pitch} (the number of turns per unit of length).

Why does frequency matter? The inductor model in Figure 1 gives part of the answer. This parallel-series circuit represents what an RF signal encounters in an inductor. Instead of just inductance (\(L\) in the schematic), there are three other \emph{parasitic characteristics} shown; \(C_P\), \(R_P\), and \(R_S\) resulting from the physical construction of the inductor. \(L\) is the inductance independent of parasitic effects.

\begin{figure}[htbp]
    \centering
    \includegraphics{../epub_extracted/images/00123.jpeg}
    \caption{Inductor Model with Parasitics}
    \label{fig:inductor-model}
\end{figure}

\(R_P\) is the simplest of the three parasitics, representing \emph{leakage resistance}, resistive current paths "around" the inductor for current. Dirt or grease on a circuit board and dust buildup on the turns of the inductor or the body of an encapsulated inductor are the most common sources of leakage resistance. It becomes significant when there is a high voltage across the inductor, such as might exist in a transmitter or tuning unit. This is a good reason to vacuum out high-power circuits from time to time.

\(R_S\) has a larger effect on the inductor's performance than \(R_P\), especially at high frequencies. At dc, \(R_S\) is specified as \emph{DCR}, or dc resistance. The resulting voltage drop or resistive heating can be important when the inductor has to carry dc current, such as when an RF choke is used in a bias T or a plate blocking choke.

If the inductor is used at RF, \emph{skin effect} comes into play, restricting current flow to a layer near the surface of the conductor. This causes \(R_S\) to increase with frequency. Inductors used in transmitters and tuning units often carry significant current so it is important to consider skin effect when selecting the size of a coil's wire or tubing. Resistive losses lower the inductor's Q, its ratio of reactance to resistance: \(X_L / R_S\).

\(C_P\) has the largest effect on inductor performance at RF. By creating a parallel-LC circuit, the combination of \(C_P\) and \(L\) means the inductor will resonate without any other external components. This creates the inductor's \emph{self-resonant frequency} or \emph{SRF}. We observed the effects of self-resonance in Experiment #111 on coiled-coax chokes. The coiled-coax choke makes use of the parallel resonance's high impedance to block current flow on the outside of the coax shield over a range of frequencies.

Where does \(C_P\) come from? Figure 2 shows that \(C_P\) results from \emph{inter-turn capacitance}. Each spot on the inductor wire forms a small capacitance to every other spot on adjacent turns, even though they are connected together by the wire. Over the entire inductor, \(C_P\) is called \emph{distributed capacitance}. Ways to reduce \(C_P\) include stretching the coil so that the turns are farther apart or in the case of multi-layer coils, carefully arranging the winding layers and winding the coil in sections.

\begin{figure}[htbp]
    \centering
    \includegraphics{../epub_extracted/images/00124.jpeg}
    \caption{Inter-turn Capacitance}
    \label{fig:inter-turn-capacitance}
\end{figure}

\subsection{Measuring Inductance}
If you have access to an antenna analyzer that displays reactance, you can measure an inductor's SRF and see the effects for yourself. As in Experiment #111, we'll use the popular MFJ-259/269-series of antenna analyzers.

Start by obtaining 8 to 10 feet of coaxial cable. Any of the RG-8/213/58/59 family will do — the characteristic impedance of the cable is unimportant as we are only interested in what happens on the \emph{outside} of the shield. Wind the cable around a non-conducting form such as the peanut butter jar in Figure 3. Remove a short section of jacket from each end, twist the shield into a lead, and connect it to the analyzer.

\begin{figure}[htbp]
    \centering
    \includegraphics{../epub_extracted/images/00125.jpeg}
    \caption{Coaxial Cable Inductor Setup}
    \label{fig:coaxial-inductor-setup}
\end{figure}

My coil of RG-58 cable has 9½ turns, it is 3 inches (76 mm) in diameter, and is 2 inches (51 mm) long. According to the equation at the start of this article, its inductance should be 6 μH. However, the equation isn't intended to apply to a close-wound coax coil, so I turned to the online inductance calculator by ON4AA at \url{http://hamwaves.com/antennas/inductance.html}. This calculator takes the diameter of my "wire" (4.5 mm) into account, as well, producing an inductance of 5.6 μH at a frequency of 1 MHz.

Connect the coil to your antenna analyzer, using a binding post adapter as shown in Figure 3. (Keep the analyzer and coil away from metal surfaces.) The following instructions apply to the MFJ-259/269 analyzer. Turn on the analyzer and press the MODE button until the display shows Inductance in uH. Clockwise from the upper left, the display shows frequency, reactance value, the label XL, and inductance in μH. The manual explains that the calculation is based on reactance and the analyzer itself can't tell whether the reactance is inductive or capacitive. You have to figure that out — if increasing the frequency causes reactance to increase, the reactance is inductive.

Start with your analyzer at its lowest frequency. (1.7 MHz on my analyzer.) The inductance value of my coil was 6.6 μH. This is not too far from the calculated value which didn't account for the jacket plas-tic's effect on \(C_P\) or the extra lead length from the coil to the analyzer. Slowly increase frequency. My inductance value stayed fairly steady near the calculated value until I passed 3 MHz and then began increasing. Why? As Figure 3B shows, the impedance of the resonant circuit of the inductor increases faster than that of an ideal inductor, causing the analyzer to see a "bigger" inductance.

Press MODE until you are back in the analyzer's usual "Impedance R&X" mode. Keep increasing frequency while watching the X value. (Ignore the SWR and Resistance meters.) You'll see it increase faster and faster until it exceeds the meter's ability to measure reactance and it displays Xs = 0. Keep increasing frequency and watch the Rs display, the equivalent series resistance value of the impedance. It will continue to increase, exceeding the analyzer's range of 1500 Ω as it approaches the coil's SRF.

Continue to increase frequency and after you pass the SRF, impedance will eventually come back into range and keep dropping as frequency increases, just like in Figure 3B. Switch back to inductance measurement and repeat the sweep through the coil's SRF. Note that above the SRF, reactance drops as frequency increases, showing that the reactance is capacitive. The inductor has changed into a capacitor!

Connect a 1.5 kΩ resistor (carbon composition or film will do) across the analyzer terminals to keep the impedance within the analyzer's range and switch back to impedance mode. You can find the coil's SRF by adjusting frequency to find the maximum value of Rs. My coil's SRF was 13.6 MHz where the meter displayed 1488 Ω. This would be a good choke for a 20 meter antenna! Now try spreading the turns apart, reducing \(C_P\), to see how that affects the SRF. You can also try an equivalently sized inductor out of insulated wire. A simple inductor? Not really!

What would happen if I tried to use this coil at VHF? \(C_P\) would make the inductor unusable. In fact, even small coils can become unusable at and above VHF due to parasitic capacitance. Capacitors with significant amounts of parasitic inductance can become unusable at those frequencies for similar reasons. Knowing the actual characteristics of your components is important for successful RF design and construction.

\begin{enumerate}
    \item H. A. Wheeler, "Simple Inductance Formulas for Radio Coils," \emph{Proc. I.R.E.,} Vol. 16, p 1398, Oct 1928.
    \item F. E. Terman, \emph{Radio Engineers' Handbook}, McGraw-Hill, p. 55, 1943.
    \item F. W. Grover, \emph{Inductance Calculations}, Dover Publications, 2009.
    \item \emph{The ARRL Handbook}, 91st edition, ARRL, Sections 5.3.4 through 5.3.7.
    \item All previous Hands-On Radio experiments are available to ARRL members at \url{http://www.arrl.org/hands-on-radio}.
\end{enumerate}

\chapter{实验 #147:RF 下的电容器}
\section{英文原文}
Paraphrasing the opening sentence of Experiment #142, "Inductors at RF," when is a capacitor not a capacitor? When it is an inductor. This odd fact of life at RF causes a lot of head-scratching and troubleshooting, particularly as the frequency of interest rises beyond the upper HF range. A capacitor's parasitic inductances and resistances are either hidden inside the capacitor or hidden in plain sight. Intriguing, isn't it? It's all part of learning to think like a radio wave.

\subsection{The Non-Ideal Capacitor}
Figure 1 shows a simple test circuit that includes a general-purpose model of a capacitor at RF and some simulated response curves. A nanofarad is not a nanofarad, it seems! The two resistances and the inductance are \emph{parasitic components}. They are consequences of the way the capacitor is constructed and are present to varying degrees in all capacitors.

\begin{figure}[htbp]
    \centering
    \includegraphics{../epub_extracted/images/00126.jpeg}
    \caption{Non-Ideal Capacitor Model}
    \label{fig:non-ideal-capacitor-model}
\end{figure}

\begin{itemize}
    \item L\textsubscript{S} is \emph{equivalent series inductance}, or \emph{ESL}, created by the capacitor's internal construction and connecting leads.
    \item R\textsubscript{S} is \emph{equivalent series resistance}, or \emph{ESR}, and represents the resistance from skin effect of the electrodes and leads along with dielectric losses.
    \item R\textsubscript{P} is \emph{equivalent parallel resistance}, or \emph{EPR}, and represents leakage losses.
\end{itemize}

Because the two primary sources of R\textsubscript{S} (skin effect and dielectric loss) both change with frequency in different ways, they are often modeled as separate resistors in simulation calculations. In this column, we'll just combine them for simplicity.

Figure 2 shows how two common types of capacitors are constructed. At A, you can see a common "roll-type" construction used for inexpensive electrolytics and many plastic film-type capacitors used in applications below 100 kHz. From the view of the end of the roll, you can probably guess that L\textsubscript{S} is pretty high compared to other types of capacitors. In fact, from the Cornell-Dubilier Electronics \emph{Aluminum Electrolytic Capacitor Application Guide}, we learn that a typical value for L\textsubscript{S} of an axial lead electrolytic ranges from the low tens of nH to around 200 nH.

\begin{figure}[htbp]
    \centering
    \includegraphics{../epub_extracted/images/00127.jpeg}
    \caption{Capacitor Construction Types}
    \label{fig:capacitor-construction-types}
\end{figure}

Figure 2B shows how a typical ceramic SMT (surface mount technology) ceramic capacitor is constructed. The capacitor is manufactured by placing an interleaved stack of metal foil and ceramic in a press and heating it to create a single \emph{monolithic} (literally "one stone") metal-ceramic block. The ends of the metal layers stick out of each side, where a metal cap is applied to form the terminal. This type of capacitor has very, very low L\textsubscript{S} — less than 1 nH. Disc ceramic capacitors are similar, with leads attached to the metal layers on each end of the stack. Other types of capacitor such as tantalum, silvered-mica, air variable, trimmers, and so on have intermediate values of L\textsubscript{S}.

\subsection{Parasitic Effects}
Measuring the effects of these parasitic components at RF is not all that easy, so we'll return to our \emph{LTSpice} roots to have a look. (See Experiments #83 – 85.)

Figure 1B shows the difference in frequency response from 100 kHz to 1 GHz between a nearly ideal 1 nF capacitor (R\textsubscript{S} and L\textsubscript{S} are zero and R\textsubscript{P} is 1 MΩ) and one with 1 nH of inductance added. Just like an inductor's series capacitance creates a self-resonance, so does a capacitor's series inductance. As frequency increases through self-resonance, the dotted trace shows the capacitor's reactance changing to inductive, which then continues to increase with frequency.

The ideal component behaves as expected at all frequencies. The non-ideal component, with its series resonance near 2 meters, might not be the best choice for a bypass capacitor at VHF and above!

\subsection{Circuit Behavior with Parasitics}
Figure 3 shows a common-emitter (CE) amplifier, the subject of Hands-On Radio Experiment #1, way back when. The ac gain of a CE amplifier primarily depends on two things — the transistor's gain-bandwidth product (h\textsubscript{FE}) and the ratio of the collector and emitter impedances, both of which change dramatically with frequency. In the emitter circuit, however, the bypass capacitor is represented by the RF model.

\begin{figure}[htbp]
    \centering
    \includegraphics{../epub_extracted/images/00128.jpeg}
    \caption{Common-Emitter Amplifier with RF Capacitor Model}
    \label{fig:ce-amplifier-with-rf-capacitor}
\end{figure}

If the bypass capacitor is assumed to have a very low impedance, the amplifier's gain is limited by the resistance through the transistors internal emitter resistance, r\textsubscript{e}, which depends on dc current flow through the transistor, but is typically around 25 Ω. That means the voltage gain of the circuit will be approximately A\textsubscript{V} ≈ –3.3 kΩ / r\textsubscript{e} = –132 = 42 dB. If the bypass capacitor wasn't there at all, A\textsubscript{V} would drop to –3.3 kΩ / 330 Ω = –10 = 20 dB. So if the capacitor parasitic components are present, their effects should be clearly visible. Let's try it!

Start by carefully creating the entire circuit as shown in Figure 3. (The \emph{LTSpice IV} schematic file and two sample response traces are available on the Hands-On Radio web page for the experiment.) The voltage source V1 is a 12 V power supply. The voltage source V2 is the ac sine wave input to the circuit. Set the parasitic component values of R\textsubscript{S} (R7) and L\textsubscript{S} (L1) to zero and of R\textsubscript{P} (R8) to a very high value. Next, perform a dc operating point (DC OP PNT) simulation. Make sure the value of V\textsubscript{C} is somewhere near 1⁄2 V1 and the transistor's collector current, Ic(Q1) ≈ 2.5 mA. If the operating point is not correct, the simulation will not produce accurate results.

Create an AC ANALYSIS simulation command with 10 points per decade of frequency between 100 Hz and 100 MHz (1e8 Hz). Run the simulation and click the simulation probe on the OUT and IN ports. In the response window (available on the website), you can see V(in) at the –20 dB level and V(out) near 30 dB, so voltage gain is approximately 48 dB, which is within reason of our previous estimate. The amplifier's bandwidth at the –3 dB points extends from about 1.2 kHz to 9 MHz, more or less.

Now start adding in the parasitic component values and observe the effect on the amplifier's frequency response. For example, changing the value of R\textsubscript{S} to 10 Ω drops the gain of the circuit to 43 dB without changing the frequency response very much. Changing the value of L\textsubscript{S} to 1 μH shifts the entire frequency response lower and reduces the circuit bandwidth so that the –3 dB points are now at 700 Hz and 3 MHz.

Next, change the bypass capacitor to the non-ideal 1 nF capacitor of Figure 1 and run the simulation from 100 Hz to 1 GHz. At low frequencies where the small capacitance is ineffective, amplifier gain is 20 dB as predicted. Above 100 kHz, gain increases to 40 dB, then falls off as transistor gain decreases and series inductance begins to have an effect.

Experiment by changing the coupling capacitors (C2 and C4) to their RF models, changing the transistor type, and adding small inductances in the signal path to simulate lead length. Keep an eye on the circuit's phase response (dashed line on the response graphs) to see how the extra inductances "color" the response.

\begin{enumerate}
    \item All previous Hands-On Radio experiments are available to ARRL members at \url{http://www.arrl.org/hands-on-radio}.
    \item \url{http://www.cde.com/resources/catalogs/AEappGUIDE.pdf}
    \item "Parasitic Inductance of Multilayer Ceramic Capacitors," AVX Corporation, \url{http://www.avx.com/docs/techinfo/parasitc.pdf}.
    \item \emph{LTSpice IV} is available from Linear Technologies at \url{http://www.linear.com/designtools/software}.
    \item Remember that the input signal is 0.1 V creating a –20 dB input reference for gain calculations, so an output level of 0 dB represents 20 dB of gain.
\end{enumerate}

\chapter{实验 #160:高频下的晶体管}

% 内容将从 EPUB 中提取

\chapter{实验 #161:晶体管微妙之处}

% 内容将从 EPUB 中提取

\chapter{实验 #169:气体放电管}

% 内容将从 EPUB 中提取

\chapter{实验 #172:RF 下的导线特性}

% 内容将从 EPUB 中提取

\chapter{实验 #173:RF 下的 PC 走线}

% 内容将从 EPUB 中提取

\part{测试与测试设备}
\chapter{实验 #140:RF 测量工具}

% 内容将从 EPUB 中提取

\chapter{实验 #158:测试装置}

% 内容将从 EPUB 中提取

\chapter{实验 #162:示波器触发和 RF}

% 内容将从 EPUB 中提取

\chapter{实验 #163:E 场和 H 场探头}

% 内容将从 EPUB 中提取

\chapter{实验 #168:评估滤波器}
\section{英文原文}
A person just learning about radio could be forgiven for thinking radio is mostly just about filters, with the occasional oscillator or modulator tossed in for good measure. Because the filtering function is everywhere, whether analog or digital, we need to know how to describe filters. There are three important parameters that we'll cover this month — amplitude response, ultimate rejection, and return loss.

\subsection{Filter Fundamentals}
The basics of filters were covered by two earlier Hands-On Radio experiments: #50 and #51, "Filter Design #1 and #2." More information about filters can be found in experiments #87 and #88, which are about using \emph{Elsie} (filter design software from Tonne Software, \url{http://www.tonnesoftware.com}), and #156, which uses \emph{Elsie} to design a broadcast interference (BCI) reject filter.

\subsection{Testing a Real Filter}
I got an e-mail from Scott Roleson, KC7CJ, titled "Exp #156 BCI Filter — It works!" Scott built a nice version of the filter (see Figure 1) in a die-cast aluminum box, and made the PCB from scratch. The capacitors are silvered-mica with a 5% tolerance. For the inductors, he used miniature encapsulated components.

\begin{figure}[htbp]
    \centering
    \includegraphics{../epub_extracted/images/00167.jpeg}
    \caption{Scott's BCI Filter}
    \label{fig:scott-bci-filter}
\end{figure}

I set up my vector network analyzer (VNA) to measure his filter's input-to-output attenuation from 1 kHz (referred to as "zero" frequency) through 10 MHz. Along with Scott's version of the Hands-On BCI filter, I also tested a commercial BCI filter from ICE Communications. Both are intended to be used at and above 1.8 MHz, the 160-meter band, where strong local AM BC stations can cause severe receiver overload. Scott's version is receive-only and uses components rated for low power. The ICE filter is rated for 300 W and can be installed in the output of a regular transceiver.

\subsection{Ultimate Rejection}
\emph{Ultimate rejection} is the attenuation the filter applies to signals far from the cutoff or rolloff frequency. In Figure 2, you can see there are deep notches in the filter's stop band. (Compare the as-built filter to the predicted performance in Figure 3 of experiment #156 — an important verification step.) That's okay — we selected a filter family (Cauer) that obtains a steep rolloff by placing notches at strategic frequencies. But it's not realistic to give the attenuation of those notches (51, 63, and 58 dB right-to-left) as the filter's ability to reject AM BC signals. The ultimate rejection of this filter is measured at the maximum of the two peaks in the stop band, 39 dB of attenuation at about 750 kHz. Did this amount of attenuation satisfy the original design specification for 40 dB at 1.6 MHz? There is plenty of attenuation at 1.6 MHz due to the notch placed there, but across the BC band, we just barely missed by about 1 dB. Pretty good, nevertheless.

\begin{figure}[htbp]
    \centering
    \includegraphics{../epub_extracted/images/00168.jpeg}
    \caption{Filter Stop Band Response}
    \label{fig:filter-stop-band}
\end{figure}

\subsection{Insertion Loss}
A close look at the filter's response shows that the shape is very close to what \emph{Elsie} predicted, right down to the passband ripple between 1.8 and 4 MHz above the filter's cutoff. Scott's filter rolls off a little higher than expected, hitting 10 dB of attenuation at 1.8 MHz. Because this is a receive-only filter, that extra 7 dB of attenuation at 1.8 MHz (we only wanted 3 dB) isn't a serious deficiency, and the filter will work fine. As the response flattens out, we can see there is about 3 dB of attenuation between 4 and 5 MHz. Similarly to how we measure ultimate rejection, this "worst-case" attenuation in the filter's passband is the filter's \emph{insertion loss} value.

In Figure 3, you see a very different filter response (the blue trace). Because this filter is expected to be used at 100 W power levels or higher, insertion loss must be minimized. From the inset photo of the filter components, you can see that full-size toroidal inductors are used. These have far lower resistance, causing less loss than the subminiature inductors used in the receive-only filter that are wound with very fine wire.

\begin{figure}[htbp]
    \centering
    \includegraphics{../epub_extracted/images/00169.jpeg}
    \caption{Filter Response and Return Loss}
    \label{fig:filter-response-rl}
\end{figure}

The ICE filter's insertion loss is less than 0.5 dB from about 2.2 through 10 MHz (0.5 dB of loss is the same as 10.9%). The tradeoff, as we discussed in experiment #156, is the steepness at which the response rolls off. While the ICE filter has very good ultimate rejection (70 dB), it doesn't reach our attenuation spec of 40 dB until 1.25 MHz, which is well inside the BC band. This is a consequence of selecting the different filter family, requiring extra components to construct.

\subsection{SWR and Return Loss}
VNAs and single-port analyzers like the Array Solutions AIM 4170 (array solutions.com), SteppIR SARK-110 (stepper.com), and others measure the S-parameter S11, also called \emph{return loss}, or RL. This parameter is the ratio in dB of how much power is reflected back to the source by the load. (The value of return loss is positive, just attenuation is specified in positive values of dB.)

RL can easily be converted to SWR and vice versa. Higher values of return loss mean less power is reflected and so indicate a lower value of SWR. For example, RL = 6 dB is an SWR of 3:1, 9.5 dB is an SWR of 2:1, 14 dB is an SWR of 1.5:1, and so forth.

Figure 3 also shows the filter's RL (S11) as the red trace. Below 1.8 MHz, RL is very low and SWR into the filter is quite high. But just below 1.8 MHz, RL is 10 dB (SWR approx. 2:1) and is greater than 30 dB (SWR = 1.07:1) for 80 meters and higher frequency bands. While RL is not so important for receiving filters, it is obviously quite important for a transmitting filter.

\subsection{Making Filter Measurements}
It would be nice if we all had VNAs, but even though they are more affordable than before, most amateurs use impedance analyzers and antenna analyzers for measuring return loss or SWR. An oscilloscope and signal generator can be used to make measurements of filter response. If you don't have a signal generator, use a manually-tuned SWR analyzer such as an MFJ-259/269-series unit. Be sure to measure both input and output voltages at each point. Be aware that most filters are designed to be terminated in a specific impedance, such as 50 Ω. Connect the filter to a dummy load when making your measurements, to avoid errors.

\section{Notes}
\begin{enumerate}
    \item All previous Hands-On Radio experiments are available to ARRL members at \url{http://www.arrl.org/hands-on-radio}.
    \item S-parameters are discussed in Hands-On experiment #72.
    \item A Return Loss — SWR converter is available online at \url{http://www.microwaves101.com/calculators/872-vswr-calculator}.
\end{enumerate}

\part{RF 技术}
\chapter{实验 #135:压接连接器}

% 内容将从 EPUB 中提取

\chapter{实验 #144:RF 接地的神话}

% 内容将从 EPUB 中提取

\chapter{实验 #145:接地和 bonding 系统}

% 内容将从 EPUB 中提取

\chapter{实验 #146:bonding 和屏蔽笔记}

% 内容将从 EPUB 中提取

\chapter{实验 #149:意外混频器}

% 内容将从 EPUB 中提取

\part{实用电台实践}
\chapter{实验 #130:通信扬声器}

% 内容将从 EPUB 中提取

\chapter{实验 #148:性能证明}

% 内容将从 EPUB 中提取

\chapter{实验 #151:Quist 测验}

% 内容将从 EPUB 中提取

\chapter{实验 #152:即兴创作}

% 内容将从 EPUB 中提取

\chapter{实验 #153:通过修复学习}

% 内容将从 EPUB 中提取

\chapter{实验 #171:交流电源分配}

% 内容将从 EPUB 中提取

\backmatter

\chapter{附录}

% 附录内容

\chapter{索引}

% 索引内容

\end{document}
