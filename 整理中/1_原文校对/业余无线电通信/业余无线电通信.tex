% 业余无线电通信
% 业余无线电通信.tex

\documentclass[12pt,UTF8]{ctexbook}

% 设置纸张信息。
% 纸张设置配置文件
% 用于定义书籍的页面尺寸和边距

\usepackage[a4paper,twoside]{geometry}
\geometry{
	left=25mm,
	right=20mm,
	top=25mm,
	bottom=25.4mm,
	headsep=1cm, 
    footskip=1cm,
	bindingoffset=10mm
}

% 设置字体,并解决显示难检字问题。
\xeCJKsetup{AutoFallBack=true}
\setCJKmainfont{SimSun}[BoldFont=SimHei, ItalicFont=KaiTi, FallBack=SimSun-ExtB]

% 目录 chapter 级别加点(.)。
\usepackage{titletoc}
\titlecontents{chapter}[0pt]{\vspace{3mm}\bf\addvspace{2pt}\filright}{\contentspush{\thecontentslabel\hspace{0.8em}}}{}{\titlerule*[8pt]{.}\contentspage}

% 设置 part 和 chapter 标题格式。
\ctexset{
	chapter/name={第,章},
	chapter/number={\arabic{chapter}}
}

% 图片相关设置。
\usepackage{graphicx}
\graphicspath{{Images/}}

% 设置署名格式。
\newenvironment{shuming}{\hfill\zihao{4}}

% 注脚每页重新编号,避免编号过大。
\usepackage[perpage]{footmisc}

\title{\heiti\zihao{0} 业余无线电通信}
\author{童效勇(BA1AA), 陈方(BA4RC)编著}
\date{}

\begin{document}

\maketitle
\tableofcontents

\frontmatter

\begin{figure}[htbp]
	\centering
	\includegraphics[width=0.7\linewidth]{cover}
	\caption{}
	\label{fig:1}
\end{figure}



\chapter{内容提要}

本书是由业余无线电家童效勇(BA1AA)和陈方(BA4RC)为广大业余无线电爱好者编写的业余无线电通信入门教材。

本书系统地介绍了开设、操作业余无线电台的相关知识和法律法规,主要内容包括:业余无线电通信简史、业余无线电通信操作实践、收发报技术的自我训练、业余无线电奖励证书和竞赛活动、不同业余无线电波段的运用、业余短波天线、业余无线电收发信机、依法设置和使用业余无线电台等。

本书既可作为开展业余无线电活动的教材,也可作为业余无线电爱好者的自修读本和手册。




\chapter{编著者的话}

业余爱好是人类社会进步的产物,是社会文明进步的标志。古今中外大凡发明创造者都有其业余爱好,而伟大的发明出自业余爱好者之手的例子更是不胜枚举。电气研究先驱者富兰克林12岁当印刷学徒并从未离开过印刷业;揭示电磁感应的法拉第也曾是报童、装订工,后来还成为一名化学专业研究者;电报机发明者莫尔斯发明电报时正从事大学的工艺美术教学……科学巨匠爱因斯坦说:“智慧并不产生于学历,而是来自对于知识的终生不懈的追求。”孔子也说:“知之者不如好之者,好之者不如乐之者。”不要被拜金主义、享乐主义和其他世俗的观点淹没了你的兴趣、爱好和激情!我们的祖国正需要千千万万个爱迪生式的发明家,而当今世界对人才的激烈竞争也正呼唤着每一个有志者从自己的业余爱好中去钻研、去实践、去塑造,以发现崭新的自我。

业余无线电通信活动以其极为丰富的内涵吸引了并将继续吸引着无数爱好者。科技性、先进性、实用性、群众性、国际性使这项活动与其他任何业余兴趣活动有着很大的不同;培养高素质的技术人才,丰富人们的文化生活,为抢险救灾提供有效的通信服务,促进各国人民间的交流,增进友谊,这一切正是改革开放不断深入的中国所迫切需要的。正因为这样,业余无线电通信活动及其标志—业余电台正越来越受到国家和各有关方面的重视,推动发展和加强管理的一系列法规、政策也已日趋完善。

我国有着大量的无线电技术爱好者,但进行业余无线电通信实践的人还不是很多。编写本书的目的是帮助更多的朋友学习和掌握业余无线电通信的基本知识和技能,尽可能地为乐于此道的爱好者们提供一本较为翔实的自我训练的教材。

改革开放的春风已吹绿了中国业余无线电通信芳草地,业余电台正如雨后春笋般出现在神州大地。愿爱好者在这里汲取更多的雨露和阳光,培育出更加绚丽夺目的奇葩—HAM之花!

1995年1月

\chapter{修订说明}

《业余无线电通信》一书自1995年出版以来,历经了数次改版、重印,2016年6月推出的第四版,也已陆续重印了三十余次。这说明在无线电技术和电子科学迅速发展的今天,这本“入门砖”性质的小册子,在广大业余无线电爱好者群体中还有一定的需求量。能够为我国业余无线电的发展尽一份微薄之力,这让我们感到十分欣慰。

为能适应科学技术和社会的快速发展以及业余无线电实践的丰富、进步,《业余无线电通信》先后于2004年、2011年和2015年进行过3次修订,出版了《业余无线电通信》第二版、第三版和第四版。在这几次修订中,除了对正文和附录里一些时效性较强的内容作必要的修改、调整外,改写了第8章《依法设置和使用业余电台》,增加了业余无线电在我国的发展简史,介绍了我国业余无线电爱好者群体在2008年汶川地震抢险救援应急通信工作中的突出表现,增加和改写了部分业余无线电通信操作实践方面的内容。

2020年10月,我们完成了《业余无线电通信》第4次修订。为能使书稿继续与时俱进,更好地服务于广大读者,我们在不改动原书总体结构的原则下,对涉及时效性的叙述及附录再次进行了修改调整,增加了软件无线电通信技术介绍、如何在线学习CW技术等方面的内容,在附录中增加了设计制作业余无线电测向机的相关内容。

《业余无线电通信》的每一次修订,都得到了许多业余无线电组织、业余无线电家和爱好者的帮助,在此一并向他们致谢。衷心感谢中国无线电运动协会,江苏、上海、天津等省、市无线电运动协会,中国无线电协会业余无线电分会以及龚万聪(BA1DU),陈平(BA1HAM),范斌(BA1RB),焦亮梅(BD1AYL),尹虎(BD1AZ),穆新宇(BD1ES),李彬(BA4REB),陈新宇(BA4RF),李家伟(BA4WI),卜宪之(BD4RG),姜锦中(BD4RQ),王龙(BD4RX),薛立人(BA5RX),郑英俊(BA5TX),陈衡(BD5RV/4),刘旭(BA8DX),刘虎(BG8AAS)等HAM在书稿校对、资料提供、翻译、新增内容的撰写等方面所给予的无私帮助,同时也感谢指正原书中的错漏之处并提出修改意见和建议的读者朋友们!

编著者

2020年10月22日

\mainmatter

无线电通信诞生于19世纪末。1888年,德国物理学家赫兹进行了一项著名的实验,他用火花隙激励一个环状天线,用另一个带缝隙的环状天线接收,证实了麦克斯韦关于电磁波存在的预言。赫兹实验激发了人们探索电磁波奥秘的热情,许多科学家都在努力研究如何利用电磁波传递信息,于是无线电技术蓬勃发展,人们的通信距离不断延伸。1901年,马可尼使用高功率的发射器首次完成了横跨大西洋的通信,揭开了无线电通信的新纪元。

20世纪初,无线电爱好者纷纷从事起无线电通信实验,商业电台数量迅速增加,电波干扰日益严重,各国政府都认识到需要制定一个法规,以保证频谱资源的合理分配和使用。1927年在华盛顿召开了世界无线电报大会,成立了国际无线电咨询委员会,并对广播、移动等各类无线电通信业务所用的频率进行了初次划分,将业余业务列入频率划分表,使业余无线电频率得到国际社会的确认。

业余无线电为无线电爱好者提供了一个广阔的舞台,无线电通信技术的发展同样也凝聚着全世界数百万业余无线电爱好者的智慧。短波通信、流星余迹通信、月面反射通信、无线电数字通信、无线电图像通信、低轨道卫星通信……业余无线电家们不断探索新的课题,在不到一百年的时间里,业余无线电从最初的无线电报逐渐发展成为利用计算机硬件、软件处理各种数字信号的全新技术。

\chapter{初识业余无线电}

从这个窗口看到的是整个世界。
——Carry V.Hammond VE3XN

1901年,马可尼成功实现了跨越大西洋的3200km距离的通信实验。1923年,业余无线电爱好者用小功率短波电台同样成功实现了横跨大西洋的通信实验。他们通过实验发现,波长越短通信距离越远,只需要较小的功率就能实现远距离通信。业余无线电爱好者的这一重大发现,是无线电发展史上最重要的成就之一,它为全球短波通信奠定了基础。

就无线电技术而言,业余无线电和其他无线电没有本质的区别。世界各国的业余无线电爱好者对无线电通信技术的发展起着重要的推动作用。在短波通信、流星余迹通信、单边带技术的研究与应用、月面反射通信、无线电数字通信、无线电图像通信、低轨道卫星通信等许多领域都留下了业余无线电家们不断探索的身影。在登山探险、抢险救灾、增进世界各国人民的友谊、培养青少年科技素质等方面,业余无线电通信也都展现了其独特的魅力和巨大潜力。

\section{独特的业余电台}

你可能很想参加业余无线电通信活动,急于了解一系列的问题,比如:

\begin{itemize}
	\item 什么是业余电台?
	\item 业余电台可以进行哪些通信实验?
	\item 业余电台能通联多远?
	\item 我能够玩转业余电台吗?
\end{itemize}  

\subsection{“火腿”与业余电台}

这里所说的“火腿”不是“金华火腿”,它特指对业余无线电通信有着浓厚兴趣的人。业余无线电爱好者的英文是“Radio Amateur”,又称为“HAM”。由于“HAM”在英语中还可以被解释为“火腿”(见图1-1),所以“火腿”又成了业余无线电通信爱好者们的另一个名字。

\begin{figure}[htbp]
	\centering
	\includegraphics[width=0.7\linewidth]{1}
	\caption{业余无线电通信爱好者}
	\label{fig:1}
\end{figure}

业余无线电台(Amateur Radio Station)是经过国家无线电主管部门正式批准,业余无线电爱好者为了试验收发信设备,进行技术探讨、通信训练和比赛而设立的电台。业余电台中的“业余”一词,并不表示从事业余无线电通信活动的人缺少专业知识与技能,而是强调业余无线电不能用于商业目的。国际电信联盟(ITU)根据不同的用途将全世界所有无线电通信分为43种业务,业余电台属于其中的“业余业务”。ITU对业余业务的定义为“供业余无线电爱好者进行自我训练、相互通信和技术研究的无线电通信业务。业余无线电爱好者系指经正式批准的、对无线电技术有兴趣的人,其兴趣纯系个人爱好而不涉及谋取利润”。

世界各国政府对业余无线电活动都给予了多方面的支持,允许无线电爱好者通过无线电波跨越国界进行国际间的交流,因此业余无线电爱好者也被称为各国的民间友谊大使。目前全世界拥有电台呼号的爱好者约300万人,如果你拥有了自己的电台和呼号,成为“火腿族”的一员,你就可以和国内外的业余无线电爱好者进行空中对话了。

下面摘录的是美国业余无线电协会(ARRL)出版的《业余无线电爱好者手册》中一段有关业余无线电通信的说明。

“一年365天,全世界的业余无线电爱好者都随时相互进行着通信。通信是人们切磋有趣的技术,进行富于变化的、激动人心的试验,发现新朋友的手段。业余无线电家作为具有共同的广泛兴趣的人们,通过全球规模的‘友谊之桥’,进行空中通信,交换思考的话题,互相学习。因此,业余无线电通信具有超越国界、增强理解和友谊的作用。这一点是其他爱好所无法实现的”。

\subsection{集体电台和个人电台}

根据设台者的身份,业余电台分为集体电台和个人电台两种。由团体申请设置,并由设台团体使用的称为集体业余电台,国际上常称其为俱乐部台(Club Station)。我国现有的集体业余电台,多为体育、教育、科协等系统所设立,是组织、培训爱好者的活动中心。个人业余电台是指爱好者本人申请设置并由其本人操作使用的电台。当今世界300万个业余电台中,绝大多数是个人电台。

中国的业余无线电活动开始于20世纪20年代,在当时极其简陋的条件下,老一辈业余无线电家怀着“以科学报效祖国”的理想,自己动手制作无线电收、发报机,互相联络,成为掌握无线电通信技术的先锋。新中国成立后,党和国家领导人十分重视在青少年中开展无线电活动,参照前苏联建立“陆海空志愿协会”军事后备组织的模式,我国建立了中国人民无线电俱乐部等机构,陆续从东欧引进快速收/发报、无线电测向、无线电多项通信等活动,组建专业运动队参加国际比赛。1958年在北京建立了我国第一座集体业余电台BY1PK。1992年,经国务院批准,我国恢复开放个人业余电台。从此,我国的业余无线电活动进入了一个新的阶段。近年来,随着中国经济的飞速发展和世界范围内文化交流的加强,业余无线电已揭开它神秘的面纱,逐步成为许多爱好者业余生活的重要内容,如图1-2所示。各地的业余无线电爱好者积极加入到普及通信知识和操作技能的活动中,时刻准备在突发灾害到来时为社会服务。还有的爱好者正在进行数据通信、空间通信等各种新技术的研究。

\begin{figure}[htbp]
	\centering
	\includegraphics[width=0.7\linewidth]{2}
	\caption{业余电台BY4RWT}
	\label{fig:1}
\end{figure}

\section{业余无线电活动}

业余无线电爱好者进行通信实验的内容是极其丰富的,既有“古老”的电报通信,也有各种各样的数据通信和图像通信。世界各国的爱好者还把电台设备带到野外相互通联,进行移动通信、小功率通信实验。捕捉“突发E层”、“大气波导”进行VHF/UHF波段超远程通信试验是最具魅力的研究课题之一,将互联网与电台结合在一起的Ecohlink同样也吸引着众多的业余无线电爱好者。

\subsection{数字通信}

数字通信是指传递数字信号的通信模式。通信的数字化,是当今通信技术的发展趋势之一。业余通信中的数字通信模式可以说是层出不穷,直到今天还在业余界广泛使用的电报(CW)应该算是最古老的一种数字通信模式。无线电电传(RTTY)是一种对设备要求较低的数字通信模式,在发送端用键盘代替话筒或电键输入信息,接收端则把来自对方的文字显示在显示器上,如图1-3所示。AMTOR是一种具有纠错码功能的电传通信方式。这种通信方式把每一个信息分散在不同时间里重复发送,在不增加设备的情况下,可以大大改善通信系统的可靠性。

\begin{figure}[htbp]
	\centering
	\includegraphics[width=0.7\linewidth]{3}
	\caption{数字通信软件}
	\label{fig:1}
\end{figure}

PSK31是指带宽为31Hz的移相键控调制模式。它只需要31Hz的带宽,占用频带非常窄,抗噪声能力强。当非常微弱甚至是几乎不可闻的信号出现时,只要在计算机屏幕上能出现信号轨迹,一般都能正确地解调。

另一种被广泛使用的数字通信模式是PACKET,利用广泛存在的PACKET网络,业余爱好者们每时每刻都能传递各种各样的信息,美国的爱好者甚至还利用PACKET网络提供全国范围的业余定位系统服务。

除此之外还有PACTOR、CLOVER、G-TOR等其他的数字通信模式,每种不同的模式适合不同的应用场合,各有各的特点。各种模式的通信实验给热衷于数字通信的业余无线电爱好者带来了许多乐趣。

\subsection{图像通信}

图像通信是现代通信技术综合发展的结果,如图1-4所示。在业余无线电通信领域里,也一直在探讨电视图像的传送技术,并不断对其进行改进。图像通信有两种主要方式;一种称为业余无线电视(ATV),用于传送活动图像;另一种称为慢扫描电视(SSTV),用于传送静止图像。

\begin{figure}[htbp]
	\centering
	\includegraphics[width=0.7\linewidth]{4}
	\caption{图像通信}
	\label{fig:1}
\end{figure}

普通电视为保证图像清晰,每秒需传送50幅图像,要占用6MHz宽的频带。业余频段范围较窄,难以传送一般的电视信号。爱好者采用几秒传送一幅图像的办法,用图像信号的慢变化来减小频带宽度,使得在短波波段传送图像的愿望得以实现。现在,许多爱好者借助计算机技术,配合简单的接口电路及软件,就可以完成SSTV方式的通联,这极大地丰富了业余无线电通信的内容。

\subsection{业余卫星通信}

1957年10月4日前苏联发射了世界上第一颗人造地球卫星,与此同时,美国的一些爱好者萌发了发射业余通信卫星的设想。他们组织起来并将这个计划命名为OSCAR。4年之后,他们的努力获得成功,世界上第一颗业余通信卫星OSCAR-1号在美国加利福尼亚的范登伯格空军基地发射升空。这不但证明了业余无线电爱好者有能力研制、开发、控制自己的卫星,也标志着业余无线电通信从此进入了太空时代。

业余卫星通信系统分为空间部分和地面部分。空间部分是指业余通信卫星,地面部分就是我们的业余电台。一个业余卫星地面电台由收发信设备、天线和跟踪控制系统组成。

在卫星通信中,由于通信距离远,信号传输损耗较大,卫星通信天线需要有较高的增益。VHF/UHF频段多采用8~10单元的八木天线,更高的频段使用20单元的八木天线或是抛物面天线。由于低轨道卫星运行周期较短,天线系统还应具备自动跟踪的功能。一般把发送、接收天线集中安装在可沿水平轴和垂直轴旋转的云台上,根据室内提供的驱动信号完成跟踪。

业余卫星通信爱好者可以选用专用的卫星收、发信机,能够异频双工工作。如YAESU公司的FT-736就是一种VHF/UHF频段全模式的卫星收、发信机,能在50MHz、144MHz、220MHz、430MHz和1200MHz业余波段工作,有SSB、CW和FM模式,能够同时异频收、发。这对于在低轨道卫星通信中克服多普勒频移很有作用。

低轨道卫星的特点是运行速度快,一般每两小时绕地球一周。要与低轨道卫星通信,必须能精确地确定卫星的位置。现在已有专门的软件可以计算出业余卫星的轨道,可以知道卫星什么时候经过我们的上空,算出卫星的仰角和方位角,由计算机控制天线方向控制器,使天线始终指向卫星,自动完成对卫星的跟踪,如图1-5所示。由于天线的波束宽度有一定的范围,也可通过手动方式控制天线方向。

\begin{figure}[htbp]
	\centering
	\includegraphics[width=0.7\linewidth]{5}
	\caption{卫星跟踪软件}
	\label{fig:1}
\end{figure}

世界各国的业余无线电爱好者已经成功地发射了二十几颗业余卫星。2009年12月15日,“希望一号”卫星(HO-68,见图1-6)在太原卫星发射中心搭载“长征四号丙”运载火箭成功地进入太空,见图1-7。这是中国第一颗青少年科普卫星。广大的业余无线电爱好者可以通过“希望一号”卫星进行各种方式的通信实验。

\begin{figure}[htbp]
	\centering
	\includegraphics[width=0.7\linewidth]{6}
	\caption{技术人员在测试“希望一号”卫星}
	\label{fig:1}
\end{figure}

\begin{figure}[htbp]
	\centering
	\includegraphics[width=0.7\linewidth]{7}
	\caption{“希望一号”卫星搭载“长征四号丙”运载火箭成功进入太空}
	\label{fig:1}
\end{figure}

\subsection{国际空间站业余无线电通信计划}

和国际空间站宇航员对话是一种独特的体验。国际空间站业余无线电通信计划(ARISS)是由美国业余无线电协会、国际业余卫星公司、美国国家宇航局等共同组织和发起的一项活动,是美国国家宇航局面向青少年的科技教育项目之一,图1-8所示为美国等16个国家共同建造的国际空间站。这个计划给学生们提供了一个利用业余无线电和国际空间站宇航员直接交流的机会。该计划要求学生用英语提出有关太空、太空飞行、宇航员的太空生活等的问题,宇航员对学生提出的问题进行解答。因此要求学生有较好的英语口语和听力水平,同时对太空知识有一定的了解。从2000年到2010年,世界上大约有400所学校参与了这项活动。

\begin{figure}[htbp]
	\centering
	\includegraphics[width=0.7\linewidth]{8}
	\caption{国际空间站}
	\label{fig:1}
\end{figure}

“Can you see the Great Wall from the ISS?”2007年8月26日北京时间18:44,南京市第三中学初二学生唐洁雯用流利的英语向宇航员提出了第一个问题,克莱顿·安德森在距地球300km的国际空间站上听到来自中国学生的呼叫。经过两年的不懈努力和等待之后,BY4RRR与国际空间站的通联活动终于得以实现。图1-9所示是中国学生第一次通过ARISS的国际计划,直接与国际空间站上的宇航员对话的情景。

\begin{figure}[htbp]
	\centering
	\includegraphics[width=0.7\linewidth]{9}
	\caption{BY4RRR的同学与国际空间站宇航员对话}
	\label{fig:1}
\end{figure}

\subsection{月面反射通信}

依靠月面反射完成通联的这种通信方式称为EME(Earth Moon Earth),见图1-10。地球与月球之间的距离大约为380000km,电波往返一次衰减极大,月面反射通信返回到地面的信号非常微弱。要想完成一次EME联络,必须掌握无线电波的传播特点,知道月亮的阴晴圆缺产生的不同的天体噪声对接收的影响。进行EME通信必须配备大功率发射机、高增益天线(见图1-11)以及高灵敏度的接收机。目前,完成双向EME联络大多数为CW方式。

\begin{figure}[htbp]
	\centering
	\includegraphics[width=0.7\linewidth]{10}
	\caption{月面反射通信}
	\label{fig:1}
\end{figure}

\begin{figure}[htbp]
	\centering
	\includegraphics[width=0.7\linewidth]{11}
	\caption{用于月面反射通信试验的八木天线阵}
	\label{fig:1}
\end{figure}

20世纪40年代,业余无线电家们已向EME通信发起了挑战。经过许多次失败之后,终于在1960年,美国的W6HB和W1BU在1296MHz首次实现了EME通信。

我国业余无线电爱好者也在积极探索EME通信。1997年10月19日,清华大学业余电台BY1QH在2m波段和瑞典SM5FRH等业余电台成功地进行了双向的EME联络,实现了我国在这一领域零的突破。

现在,EME通信已采用由计算机完成的新的调制技术,这大大降低了EME通信对设备的要求。2007年5月20日,江苏省无线电运动协会业余电台BY4RSA在1296.090MHz频率上,以JT65C的模式成功地与荷兰的业余电台PA3CSG进行了双向EME通联。这次实验使用的是一台IC706电台,一个自制的变频器和直径为1.8m、增益为25dB的抛物面天线,一台自制的功率放大器,发射功率仅90W。

\subsection{应急通信}

当地震、洪水、恐怖袭击等重大灾害事故突然发生时,往往伴随着电力供应以及公众电信、道路交通等设施遭到破坏,常规通信设施会随之陷入瘫痪,与外界的联系暂时中断。这时,灾害现场的业余无线电爱好者可以利用身边的通信设备,尽快找到可供电台使用的电源(如小型发电机、蓄电池、汽车电源等),选择有利地形,迅速架设天线,并立即进行紧急“求救呼叫”。在美国“9·11”恐怖袭击事件、印尼大海啸、四川汶川地震等灾难中,业余无线电爱好者在应急通信中都发挥了重要的作用。

在VHF/UHF频段,可用语言直接进行求救呼叫。

例如:“May Day、May Day、May Day! BA4AAA求救,BA4AAA求救,听到请回答。”

在求救呼叫得到灾害现场以外地区的回答后,呼救电台应向外发送以下信息:

①受灾的精确地点及性质(即遭受何种灾害);

②受灾的程度及受灾现场的情况;

③灾害现场现有的救援力量及迫切需要何种支援。

④其他一切有利于援助的资料。

求救呼叫是最高级别的信号,任何业余电台收听到求救呼叫时,必须立即无条件中断发射,改为守听状态,并给予必要的协助。

呼救电台在紧急情况得到解决后,应及时在结束联络时清楚地发出解除呼救的信号,以免其他电台继续长时间守听。
例如:“我是BA4AAA,救援人员已经到达,BA4AAA解除呼救,BA4AAA解除呼救,再见。”

承担救灾应急通信是业余电台的社会责任和优良传统,也是业余频段得到保护的主要理由之一。世界上许多国家都成立了业余无线电应急服务组织,美国ARRL有专业的业余无线电通信应急服务(ARES)委员会,我国的无线电爱好者也在积极开展应急通信演练活动(见图1-12),筹划、建立各地区的业余无线电应急通信网络。

\begin{figure}[htbp]
	\centering
	\includegraphics[width=0.7\linewidth]{12}
	\caption{应急通信演练}
	\label{fig:1}
\end{figure}

\section{依法设置业余电台}

为了有效利用无线电频率资源,保证各种无线电业务的正常进行,国际电信联盟制定了无线电管理的国际法规——《无线电规则》,用来作为各国无线电管理的依据。1993年国务院、中央军委颁布了《中华人民共和国无线电管理条例》,国家无线电管理机构先后颁布了一系列规章和管理文件。这些规定是我们开展业余无线电活动必须遵循的准则。
业余无线电爱好者不同于一般的电子技术爱好者,设置和使用个人业余电台,进行无线电波的发射,必须向国家或者地方无线电管理机构提出申请,得到批准并取得《中华人民共和国无线电台执照》(以下简称《电台执照》),严格按照指配的频率、功率工作。未经批准擅自使用无线电通信设备,干扰其他无线电通信业务的正常工作,造成空中电波秩序混乱,违规者将受到警告、查封及没收设备、罚款、吊销《电台执照》等处罚。造成严重后果、构成犯罪的,将依法追究刑事责任。

\subsection{取得业余电台《操作证书》}

根据国家无线电管理机构的有关规定,设置个人业余电台必须持有《中华人民共和国业余无线电台操作证书》(以下简称《操作证书》),如图1-13所示。《操作证书》是设置操作业余电台所要求的技术凭证。它共分5个等级,一级为最高级,五级为收听级。

\begin{figure}[htbp]
	\centering
	\includegraphics[width=0.7\linewidth]{13}
	\caption{《中华人民共和国业余无线电台操作证书》}
	\label{fig:1}
\end{figure}

申请《操作证书》者,应是经过无线电管理机构委托的中国无线电运动协会或其他单位的培训,并经《操作证书》考核机构考试合格的中华人民共和国公民。申领《操作证书》没有年龄限制。

取得《操作证书》后,可凭证在国内的个人和集体业余电台上进行操作,但必须得到电台的设台人或负责人同意,使用所操作电台的呼号,并严格遵守该台及《电台执照》中的各项规定。操作时使用的功率、频率及操作方式不得超越本人《操作证书》允许的范围。

取得《操作证书》后,如果打算设置个人业余电台,可按当地无线电管理机构规定的渠道递交《设置个人业余电台申请表》,并按当地无线电管理机构的要求对电台设备进行检验、检测,批准并核发《电台执照》,如图1-14所示,同时指配业余电台呼号。只有获得《电台执照》后,爱好者才能使用分配给自己的呼号,按照规定与国内外业余电台进行联络和交换QSL卡片。

\begin{figure}[htbp]
	\centering
	\includegraphics[width=0.7\linewidth]{14}
	\caption{《中华人民共和国无线电台执照》}
	\label{fig:1}
\end{figure}

\subsection{设置业余电台}

需要注意的是,部分初学者不了解有关的法规,购买到对讲机等通信设备后,自行设置频率,无证(照)发射。或虽有《操作证书》和《电台执照》,但使用时随意改变《操作证书》所允许的频率、功率范围,干扰其他电台,这都是与有关法规相违背的。

\subsection{工业和信息化部无线电管理局}

工业和信息化部无线电管理局是国家无线电管理机构,负责无线电频率的划分、分配与指配;依法监督管理无线电台(站);负责卫星轨道位置协调和管理;协调处理军地间无线电管理相关事宜;依法组织实施无线电管制;负责无线电监测、检测、干扰查处,协调处理电磁干扰事宜,维护空中电波的秩序。国家无线电管理机构和地方无线电管理机构负责业余电台的管理工作。

4.中国无线电运动协会
中国无线电运动协会(CRSA,见图1-15)是全国无线电爱好者的群众性组织,是主管业余无线电活动的政府部门联系广大无线电爱好者的纽带和管理业余无线电活动的助手。中国无线电运动协会在国际业余无线电联盟(IARU)中代表中国业余无线电爱好者。

\begin{figure}[htbp]
	\centering
	\includegraphics[width=0.7\linewidth]{15}
	\caption{CRSA会徽}
	\label{fig:1}
\end{figure}

目前,不同省、自治区、直辖市的业余无线电爱好者加入中国无线电运动协会,参加业余电台操作培训和考核,申请设置使用业余电台所需履行的具体手续有所不同,可向当地无线电管理机构、中国无线电运动协会或地方业余无线电社团咨询。中国无线电运动协会的总部设在北京,通信地址为“北京6106信箱”,邮政编码:100061,网址www.crsa.org.cn。

\subsection{国际电信联盟}

国际电信联盟是联合国的一个专门机构,也是联合国机构中历史最长的一个国际组织,简称“国际电联”、“电联”或“ITU”。国际电联是主管信息通信技术事务的联合国机构,主要负责各种业务间的频率分配。国际电联总部设在瑞士日内瓦,其成员包括191个成员国和700多个部门成员。

\chapter{选购一部业余电台}

引子
最好的选择未必是选择最好的。

可供爱好者选择的业余电台品种很多。按照不同的分类方法,可将业余电台分为单波段电台、双波段电台、多波段电台,或是手持式电台、车载电台、短波电台和全波段电台。尽管各种电台的外形、大小不尽相同,每部电台都由一些基本组件构成,它们是:电源、发射/接收机和天线,如图2-1所示。

\begin{figure}[htbp]
	\centering
	\includegraphics[width=0.7\linewidth]{16}
	\caption{电源、发射/接收机和天线是业余电台的3个基本组件}
	\label{fig:1}
\end{figure}

业余电台是指可以工作在业余频段的电台设备。购买电台之前,先要知道与电台设备密切相关的一些知识,如频率、频段、调频、调幅、功率等,以便在选购时对电台的特点有一个初步的了解。购买电台需要考虑许多因素,首先要确定电台的类型,然后再根据通信距离和通信环境来选择工作频段及输出功率。电台又是一种特殊的商品,购买、使用时还必须了解有关的法规和政策。

\section{业余频段}

人们可利用的电磁波的频率从千分之几赫直至1030Hz。为了便于研究、使用和管理,国际上把整个无线电频谱划分为若干频带,通常称为频段或波段。无线电波频段的划分见表2-1。电磁波是人类共享的资源,为了合理使用这一资源,国际电信联盟(ITU)对广播、移动等各种无线电业务可以使用的频率作了规定和划分。业余无线电通信作为业余业务在各频段上也都占有一席之地。这就是供全世界业余无线电爱好者使用的业余频段,我国业余电台使用的频率以《中华人民共和国无线电频率划分规定》为依据,表2-2列出的是常用的业余频段名称和频率范围。

\begin{figure}[htbp]
	\centering
	\includegraphics[width=0.7\linewidth]{17}
	\caption{无线电波频段的划分}
	\label{fig:1}
\end{figure}

\begin{figure}[htbp]
	\centering
	\includegraphics[width=0.7\linewidth]{18}
	\caption{业余业务频段}
	\label{fig:1}
\end{figure}

\section{VHF/UHF手持业余电台}

手持业余电台又称为对讲机,它是一种小型的移动通信工具,如图2-2所示。手持业余电台的最大优点是携带方便,野外活动时可以将手持业余电台抓在手上或放入衣袋中,手持业余电台主要用于FM通信。

VHF(甚高频)是指频率从30~300MHz的无线电波,UHF(特高频)是指频率从300~3000MHz的无线电波。实际上某个VHF/UHF手持业余电台的工作频率范围只是甚高频或特高频整个频段的一部分。

手持业余电台多为VHF/UHF单频段电台,也有许多手持业余电台可以多频段通信。例如八重洲VX-7R手持业余电台可以在50MHz、144MHz、430MHz 3个业余频段上通信。

由于受到体积的限制,手持业余电台不能提供较大的输出功率,手持业余电台所配的橡胶天线效率很低(见图2-3),这些因素使得手持业余电台的通信距离很有限,开阔地的通话距离为几千米。

\begin{figure}[htbp]
	\centering
	\includegraphics[width=0.7\linewidth]{19}
	\caption{FM手持业余电台}
	\label{fig:1}
\end{figure}

\begin{figure}[htbp]
	\centering
	\includegraphics[width=0.7\linewidth]{20}
	\caption{螺旋橡胶天线}
	\label{fig:1}
\end{figure}

VHF/UHF波段手持业余电台设备的价格低廉,可选范围大。VHF/UHF波段又是爱好者的“入门波段”,世界上大多数爱好者都活跃在这个波段上。

\section{VHF/UHF车载业余电台}

车载业余电台是专门为汽车和其他交通工具设计制造的移动通信设备,车载业余电台仅能在业余频段发射信号。车载业余电台设计紧凑,外壳坚固结实,结构耐冲击和震动。电台大小一般按汽车收放机的大小标准设计,主操作通过前面板进行,较先进的机型还采用分离式面板的安装形式,方便在车内狭小的空间内安装。图2-4所示为一部八重洲双段车载业余电台。车载业余电台为我们带来驾驶车辆以外的另一种感受,利用车里的业余无线电设备进行通联,可以体验野外通信的乐趣。图2-5所示为一部车载业余电台在车内的安装位置。

\begin{figure}[htbp]
	\centering
	\includegraphics[width=0.7\linewidth]{21}
	\caption{八重洲FT-7800R VHF/UHF双段车载业余电台}
	\label{fig:1}
\end{figure}

\begin{figure}[htbp]
	\centering
	\includegraphics[width=0.7\linewidth]{22}
	\caption{车载业余电台安装的位置}
	\label{fig:1}
\end{figure}

车载业余电台具有较大的输出功率。在进行FM通信时,大功率能够通联更远的距离。但是,业余电台选用设备的最大输出功率不得超过无线电管理部门核发的《电台执照》所规定的功率。当然,车载业余电台也可以放在室内使用,接上电源和室外天线,它就是一部基地电台了。

在移动通信中,由于电台位置不断变化,为保证通信质量,一般要求移动电台及基地台天线在水平面内无方向性,并且采用垂直极化方式。我们通常看到工作在VHF/UHF频段的手持业余电台、车载业余电台、中继台都使用了各种形式的垂直极化无指向性天线。

车载天线(见图2-6)有吸盘天线和夹边天线两种。车载天线在结构上有1/4波长天线、1/2波长天线、5/8波长天线等形式。一般情况下天线越长,其增益也越高。若通信范围主要是在市区通过中继台进行通联,则可以选择增益较低的短天线。当用于郊外车辆间远距离联络时,宜选用高增益的长天线。由于高度的限制,加上车辆高速行驶时会遇到很大的风阻,所以车载天线的尺寸要依照工作频段采用不同的设计。

\begin{figure}[htbp]
	\centering
	\includegraphics[width=0.7\linewidth]{23}
	\caption{车载天线}
	\label{fig:1}
\end{figure}

车载天线大多数是VHF/UHF双频段天线,这样只要一根天线就可以同时进行VHF频段和UHF频段的发射和接收。一般车载天线都是采用垂直极化方式,这种天线在水平方向上是均匀辐射的,这可以让我们在驾驶中不需要时刻想着天线的方向。图2-7所示是平面接地(GP)天线的立体方向图,这是一种典型的垂直极化全向天线。

\begin{figure}[htbp]
	\centering
	\includegraphics[width=0.7\linewidth]{24}
	\caption{GP天线方向图类似苹果形状}
	\label{fig:1}
\end{figure}

\section{短波业余电台}

短波(HF)业余电台是指发射频率在1.5~30MHz之间各业余频段的电台设备。图2-8所示为一部IC-718短波业余电台。短波业余电台的优点之一就是不需要很大的发射功率就能实现远距离通信。短波业余电台的输出功率通常为100W,在传播条件良好的情况下,这个功率可以进行全球范围的通信。

\begin{figure}[htbp]
	\centering
	\includegraphics[width=0.7\linewidth]{25}
	\caption{ICOM IC-718短波业余电台}
	\label{fig:1}
\end{figure}

短波波段是比较常用的业余无线电波段,在这个波段上,业余无线电爱好者可以使用多种通信模式,其中包括:

\begin{itemize}
	\item SSB—单边带语音模式;
	\item CW—电报模式;
	\item AM—调幅语音模式;
	\item FM—调频语音模式;
	\item RTTY—无线电电传模式;
	\item SSTV—慢扫描电视模式。
\end{itemize}

目前,有些制造厂商设计出了HF/VHF/UHF多波段、多模式业余电台,发射频率范围可达1.9~440MHz间的各业余频率。我们将这类电台称作全波段电台,它是目前非常流行、畅销的电台,如图2-9所示。全波段电台既可以作为移动台,也可以作为基地台。一些全波段电台还具有业余卫星通信等扩展功能。

\begin{figure}[htbp]
	\centering
	\includegraphics[width=0.7\linewidth]{26}
	\caption{建伍TS-2000 1.8~440MHz全波段电台}
	\label{fig:1}
\end{figure}

天线是无线电通信设备的耳目,天线的质量直接影响着电台通联的效果。电台通过天线向空中发射无线电波,同时通过天线从空中接收无线电信号,因此对电台来说,天线具有特别重要的作用。每部电台必须配备性能良好的天线才能实现好的通联效果。

短波业余电台常用的天线有半波偶极天线和八木天线。半波偶极天线是最容易制作的天线。它的结构非常简单,只需要将两根长度相同、总长度约为半波长的导线水平架设起来即可。偶极天线的长度由工作频率决定。以7MHz频段为例,它的波长λ≈42.85m。业余无线电爱好者通常将这个数字简称为40m,因此7MHz频段也称作40m波段。半波偶极天线两段导线的长度是1/4λ=10.7m,其外形如图2-10所示。

\begin{figure}[htbp]
	\centering
	\includegraphics[width=0.7\linewidth]{27}
	\caption{水平半波偶极天线}
	\label{fig:1}
\end{figure}

八木天线是一种典型的定向天线。这种天线是由日本人八木秀次于1926年发明的,因此叫做八木天线。八木天线具有良好的方向性和高增益,配合转向器可以很好地工作在HF、VHF和UHF频段,图2-11所示为一个典型的短波八木天线。将多个八木天线组合起来,构成天线阵列,可以进一步提高八木天线的方向性,特别适合VHF/UHF频段远距离通联。

\begin{figure}[htbp]
	\centering
	\includegraphics[width=0.7\linewidth]{28}
	\caption{HF频段14单元八木天线}
	\label{fig:1}
\end{figure}

\section{几个常用术语}

\subsection{频率范围}

为了合理地利用频率资源,保证用户之间不受干扰,无线电管理部门对手持业余电台的使用频率进行了划分,规定了不同行业使用的频率范围。常用的VHF、UHF业余频段一般是50~54MHz、144~148MHz和430~440MHz频段。业余无线电爱好者购买手持业余电台时首先要了解它的收、发频率范围是否包含144~148MHz、430~440MHz两个常用的业余频段,并且只能在业余频段发射。

\subsection{单段、双段}

单段电台只能在一个频段上收、发,双段电台可以工作在两个频段上。除单段、双段电台,有的对讲机制造商还设计出多频段电台。如八重洲FT-8900车载业余电台可以在29MHz、50MHz、144MHz、430MHz 4个业余频段上工作。

\subsection{单显、双显}

单显是指电台显示屏上显示一个信道、频率等其他指示图标。单显电台一般只能在当前频段上的一个信道(频率)上收、发。双显是指电台显示屏上显示两个信道、频率及指示图标。双显电台可以同时接收两个信道的信号,并能在其中任意一个信道上发射。

频段、显示方式  典型机型

单段单显     灵通LT-6100plus

单段双显     欧讯KG-689

双段单显     八重洲FT-7800

双段双显     八重洲FT-8800

\subsection{单工、双工}

单工通信是指在同一时刻信号只能单方向进行传输。单工电台的工作通常是以按键控制收、发的转换。当按下发射键时,发射机处于工作状态,接收机处于不工作状态;反之,松开发射按键时,接收机处于工作状态,发射机处于不工作状态。单工电台收、发交替工作,只能你说我听或者我说你听。

单工机根据频率使用情况,又分为同频单工机和异频单工机(半双工机)。同频单工机是指发射和接收都工作在同一频率上,它能有效地使用频率资源。异频单工机一般在有中转台的无线电通信系统中使用。现在很多单工机既能同频工作又能异频工作。

双工通信是指在同一时刻信号可以进行双向传输,和打电话一样,说话的同时也可以收听。双工电台的发射和接收分别工作在两个不同的频率上,所以也称为异频双工机。双工手持机大多在VHF频段和UHF频段上跨段工作。

\subsection{输出功率}

射频输出功率是决定通信距离的重要因素。功率越大,通信距离越远。手持式对讲机的输出功率一般在0.5~5W,车载业余电台的输出功率通常为5~50W。购买时可根据实际通信距离、《电台执照》的核定功率和《操作证书》等级来选择相应发射功率的电台设备。

\subsection{灵敏度}

灵敏度是指通信设备接收微弱信号的能力。灵敏度高的接收机能够收到较远电台微弱的信号。通常以输入信号电压的大小来表示灵敏度,单位是微伏(μV),这个数值越小,表示接收机的灵敏度越高。

\subsection{FM和SSB}

FM和SSB是业余无线电通信中最常用的两种语音通信模式。我们可以通过特定的电路,来改变无线电信号的幅度或频率,以便承载声音、图像信息。如果我们让信号的频率随语音发生改变,这就是调频,即FM。

FM通信的最大优点是它的接收性能良好,信号清晰,几乎没有噪声,机器调整也比较简单。现在,手持业余电台和车载业余电台多采用FM调频模式。

SSB是单边带语音通信的缩写。一般通信系统中,载波经音频信号调制后,包含载波和上、下两个边带信号,这两个边带含有相同的信息。为了提高通信效率和节约通信频带,我们可以通过单边带电路,将载波和另一边带去除掉,只发送一个边带,这种通信方式就称为单边带通信。单边带电台与调幅、调频电台相比,具有节省频谱、节约功率等优点,特别适合远距离通信。在短波(HF)段上一般都采用SSB通联。

\chapter{第一次通联}

引子
“CQ CQ CQ,这里是BY4RWT, BRAVO YANKEE FOUR ROMEO WHISKEY TANGO, BY4RWT呼叫……”

业余电台通话不能像打电话那样,想怎么说就怎么说。业余无线电通信有它特定的通话语言、通话规则和通信程序。要顺利地完成一次通联,业余无线电爱好者除了要具备基本的电台操作技能外,还要熟练掌握通信语言,了解业余通信基本程序,以保证通信中取得最佳的通联效果。

\section{通联前的准备}

\subsection{业余通信基本程序}

\subsubsection{呼叫}

呼叫前在你准备使用的频率上收听一会儿。业余频段上的电台较多,相互之间要避免干扰。如果听到频率上已经有电台在工作,则应更换频率呼叫。

为防止干扰,可以先问一声“这个频率上有人吗”,英语则用“Any body here”,如果听到有人回答,就应主动更换一个频率。

· 普遍呼叫

没有特定联络对象的主动呼叫称为普遍呼叫。听到普遍呼叫后的业余电台都可以回答。如BY4RWT普遍呼叫的程序是:

“CQ,CQ,CQ,这里是BY4RWT,BRAVO YANKEE FOUR ROMEO WHISKEY TANGO,BY4RWT呼叫,听到请回答……”

· 区域性呼叫

将联络对象限制在一定范围内的呼叫称为区域性呼叫。区域性呼叫程序与普遍呼叫程序基本一致,当听到其他电台在进行区域性呼叫而自己又不在被呼叫的范围内时不要应答。

如只呼叫中国电台:

“CQ BRAVO,CQ BRAVO,CQ BRAVO……”

呼叫不包括本国在内的远距离电台:

“CQ DX,CQ DX,CQ DX……”

· 插入呼叫

当两个电台正在联络,你想呼叫其中的一个,可以等他们讲完后呼叫。在有特殊情况时,可在双方谈话告一段落时插入呼叫。随便插入或打断别人的谈话,都是不礼貌的行为。

例如:“BY4RWT请求插入……”

\subsubsection{回答}

在听到某台呼叫,而自己属于被呼叫范围,你可以准备与其联络。回答的程序是:“BY4RWT,BRAVO YANKEE FOUR ROMEO WHISKEY TANGO,我是BG4RYN,BRAVO GOLF FOUR ROMEO YANKEE NOVEMBER,请回答。”

\subsubsection{沟通后的联络}

当双方已经相互收到对方的呼号后,就可以进行其他各项内容的交流。每次发信时可加上“听到”、“明白”、“抄收”、“Roger”等用语,表示刚才对方发过来的信息已全部抄收。讲话结束时要加上“完毕”、“请讲”、“请过来”、“Over”等用语,让对方知道自己已经转为收听状态。例如:

“听到。BG4RYN,这里是BY4RWT,很高兴在空中见面,您的信号是59。我叫孙倩,我的位置在南京五塘中学。BG4RYN,这里是BY4RWT,请讲。”

“Roger,BY4RWT,我是BG4RYN,你过来的信号也是59。我姓任,我在中央门附近。BY4RWT,我是BG4RYN,Over。”

\subsubsection{结束联络}

双方完成联络后,要以明确的语言结束联络,以便其他电台知道该频率已经空出。例如:

“……谢谢今天良好的联络,希望再次见到你。BG4RYN,这里是BY4RWT。73,再见!

“BY4RWT,我是BG4RYN。73,再见!

一般情况下,业余通信时还可以介绍自己的姓名、电台所在的地理位置、交换QSL卡片的地址、交换所用设备和天气情况等。每次通信内容应力求简洁,不要长谈。在业余电台上漫无边际的长时间聊天甚至调笑是违反业余电台有关规定的行为。

一次成功的联络,至少应包括下面3个内容:

\begin{itemize}
	\item 双方都正确地抄收了对方的呼号;
	\item 交换信号报告;
	\item 将这次的联络情况正确地记录在电台日记上。
\end{itemize}

通信程序举例如表3-1所示。

\begin{figure}[htbp]
	\centering
	\includegraphics[width=0.7\linewidth]{29}
	\caption{}
	\label{fig:1}
\end{figure}

\begin{figure}[htbp]
	\centering
	\includegraphics[width=0.7\linewidth]{30}
	\caption{}
	\label{fig:1}
\end{figure}

\subsection{呼号}

就像我们每个人有不同的姓名一样,业余电台的呼号是用来识别电台身份的一个代号。每个业余电台呼号由3个部分组成:前缀、业余分区、后缀。

前缀位于呼号的最前面,由1~2个英文字母或字母和数字组合而成,例如呼号BY4RWT、W3EP中的BY、W。它是由各国根据国际电信联盟排定的“国际呼号系列划分表”划分给业余电台使用的,所以前缀是每个业余电台所属国家或地区的标志。当你听清了一个呼号的前缀后,可以从“国际呼号系列划分表”查出这个电台的国籍。

有些呼号还包含有对业余电台性质的说明。我国的业余电台呼号中的第二个字母即属此例,如BA表示持有一级《操作证书》的个人业余电台,BY代表集体台,BT表示为某个重大活动临时设立的特设电台。

呼号的第二部分是“国内业余分区”,用一位数字表示。你可以根据其分区情况了解到这个电台位于该国家的哪个地区,我国的业余电台分区见表3-2。

表3-2 我国业余电台分区表

\begin{figure}[htbp]
	\centering
	\includegraphics[width=0.7\linewidth]{31}
	\caption{}
	\label{fig:1}
\end{figure}

\begin{figure}[htbp]
	\centering
	\includegraphics[width=0.7\linewidth]{32}
	\caption{}
	\label{fig:1}
\end{figure}

呼号的最后一部分是后缀,这是电台自己特有的名字。后缀一般由1~3个字母组成。如清华大学业余电台的呼号是BY1QH,前缀中的字母B代表中国,Y代表集体电台,1表示业余分区在第1区(即北京市),QH为呼号的后缀。世界各国遵照国际电信联盟(ITU)有关呼号划分的规定,制定出本国字母的排列原则,从而保证了电台呼号的唯一性。我国业余电台的呼号由各省(市、自治区)无线电管理机构分配。

\subsection{信号报告}

信号报告是业余无线电通信中最基本的技术数据,准确的信号报告可使双方了解自己收、发信设备的效率及电波的传播情况。业余电台的信号报告由信号可辨度(R)、信号强度(S)以及信号音调(T)3部分组成,所以信号报告也常称为“RST”。在语言通信中,只报告前2项;在电报、电传及其他一些数字通信中应报告全部3项。

\subsubsection{信号可辨度}

信号可辨度“R”位于信号报告的第一位,共分5级,用1~5中的一位数字表示,如表3-3所示。信号可辨程度的判断依靠主观经验。在通常情况下,对于能顺利进行联络的信号总是给予最高等级——5的报告。

表3-3 信号可辨度分级表

\begin{figure}[htbp]
	\centering
	\includegraphics[width=0.7\linewidth]{33}
	\caption{}
	\label{fig:1}
\end{figure}

\subsubsection{信号强度}

信号强度“S”位于信号报告的第二位,共分9个级别,用1~9中的一位数字表示,如表3-4所示。信号强度报告可以借助收、发信机面板上的仪表指示判读,见图3-2。有些收信机没有“S”表,则应参照定级标准依靠主观判断。

表3-4 信号强度分级表

\begin{figure}[htbp]
	\centering
	\includegraphics[width=0.7\linewidth]{34}
	\caption{}
	\label{fig:1}
\end{figure}

\begin{figure}[htbp]
	\centering
	\includegraphics[width=0.7\linewidth]{35}
	\caption{IC-756PRO电台左上角“S”表可指示信号强度}
	\label{fig:1}
\end{figure}

信号音调“T”位于信号报告的第三位,也用1~9中的一位数字表示。在报告信号时,代表“RST”的3位数字应同时报出,因此,对于好信号的报告,在语言通信中就是“59”,而在电报、电传等通信中的报告就是“599”了。

在VHF/UHF波段使用FM调频方式通话,当信号强度大于一定水平时,接收机听到的声音都很清楚,如果电台上没有信号强度指示,凭听觉很难对信号强度进行评估,这种情况下的信号报告只能粗略估计了。

\subsection{英文字母解释法}

在日常生活中,人们有时需要对某些汉字进行解释,如“木子李”、“弓长张”。语言通信中,由于部分英文字母发音相近,如字母S和X、M和N,另外信号在传播过程中受到干扰等原因,我们对有些字母分辨变得非常困难。为解决这个问题,人们约定用一些大家熟悉的单词来代表、解释相应的字母。根据国际电信联盟(ITU)的规定,业余无线电通信中采用国际民航组织(ICAO)使用的解释法作为“标准解释法”。此外还常用一些人们熟悉的地名、人名等来解释。

例如:“CQ CQ CQ,这里是BA4RM,BRAVO ALPHA FOUR ROMEO MIKE呼叫,听到请回答。”

英文字母解释法是业余无线电通信中必须掌握的基本知识。电台呼号、姓名、地址等许多信息的说明都需用到英文字母解释法,如表3-5所示。

表3-5 英文字母解释法

\begin{figure}[htbp]
	\centering
	\includegraphics[width=0.7\linewidth]{36}
	\caption{}
	\label{fig:1}
\end{figure}

\section{开始通联}

知道业余电台通信程序、电台呼号、英文字母解释法等基本内容后,便可以开始通联了。第一次通联总是令人紧张、兴奋的。打开电台后,不要急于通联,先仔细收听其他爱好者的通联,熟悉通联步骤、通联用语,掌握电台频率、功率的调整方法,没有问题后,再拿起话筒开始通联。

与其他爱好者建立通联有两种方法:一是自己呼叫CQ,等待其他电台的回答;二是回答正在呼叫CQ的其他电台。

\subsection{呼叫}

在开始呼叫CQ之前,首先要寻找一个未被其他火腿使用的空闲频率,养成呼叫之前先监听的好习惯。开机收听一会儿后,可以询问一次:“这个频率有人使用吗?这里是BA4RM。”确认该频率上没有人使用,再进行呼叫。呼叫CQ之后,如果没有人回答,可以稍等片刻,再次呼叫。如果呼叫三四遍后,一直没有人回答,可能这个频率上确实没有人收听,这时,你可以更换其他频点试试。

用车载业余电台进行呼叫时,应在自己的呼号后面加上“/所在的分区”或“/M”。比如BA4RM驾驶汽车来到北京,汽车停在某处呼叫:

“CQ CQ CQ,这里是BA4RM/1,BRAVO ALPHA FOUR ROMEO MIKE PORTABLE ONE呼叫,请回答。”

操作者报告自己的呼号(BA4RM/1)之后,又用字母解释法再次报告一遍,是让对方听清楚呼号,以防抄错,其中的斜线读作“Portable”。

业余电台操作首先要学会使用话筒。在通话过程中,手持话筒姿势要端正,一般用拇指或食指按压收、发控制键(PTT),嘴离话筒不要太远,大约10cm的距离为宜。

由于业余电台通联多为单工方式,说话和收听不能同时进行,而是通过PTT键控制说话和收听的转换。说话时按下PTT键,说完就立即松开,听对方讲话。别人呼叫时,自己不能随便按下PTT键说话,否则两人都无法听到对方的声音。

业余电台通联中,通话内容要简洁,语速平稳,语音清晰,语言礼貌、规范。无论何时何地进行通联,都要记住你代表的是一个群体,是整个业余无线电界的形象。

\subsection{移动通信的乐趣}

在汽车或其他交通工具上装备无线电通信设备,人们可以在移动中进行通信,这种通信方式叫做移动通信。移动通信方式灵活机动,能保持随时随地通联。现在,移动通信已广泛应用于各个行业及人们的日常生活当中。
业余电台联络方式可分为直发通信和中继通信。直发通信是指两电台直接进行通联,不经过中继台或其他电台转发。通常直发通信电台的发射频率和接收频率设置在同一个频点上,操作者通过PTT键来切换收、发状态。与邻近电台通联一般就选用这种通信方式。当然,这种模式也可以用于车队,让所有车载业余电台工作在相同的频率上,构成一个专门的移动通信网络,这也是最简单的一种组网方式了,如图3-3所示。需要注意的是,车载业余电台频率设置要严格遵守当地无线电管理部门的有关规定。

\begin{figure}[htbp]
	\centering
	\includegraphics[width=0.7\linewidth]{37}
	\caption{F1是移动通信网络的固定频点}
	\label{fig:1}
\end{figure}

目前汽车上装备的通信设备大多数是VHF(144MHz)和UHF(430MHz)频段的车载业余电台。VHF和UHF频段的电波以视距传播为主,沿地面传播距离一般在50km以内。考虑到车载业余电台的设置条件和移动通信的不稳定性,实际通信距离还会缩短。在平坦的开阔地段车载业余电台之间的通信距离一般在5~15km,所以,若想实现与较远电台的联络,还要充分利用地形,灵活掌握通联时机。

一般来说,当汽车行驶到开阔地或地势较高的地段时,往往可以呼叫到较远的联络对象。当车辆进入山谷或高楼林立的市区,通联常常中断。在高速公路上,汽车行驶到信号较弱的地区,会出现信号周期性消失的现象,工作频率越高,汽车速度越快,信号断续得就越频繁。若这种影响使得车载业余电台无法通联时,可降低汽车速度或将车辆停在信号较强的地方完成联络。

移动通信中的电波传播非常复杂,特别是车辆进入丘陵地带和城市中,山峰或高楼对电波传播的阻挡、反射,使得到达车载业余电台的信号时大时小,联络时通时断。这种情况下,电台操作者可以采取短呼叫、勤呼叫的方法进行联络,一旦信号好时就快速完成通联。若两台无法直接联络,可考虑通过其他电台转发或中继台完成通联。

\subsection{中继通信}

业余电台依靠中继台(Repeater)对电波的转发而建立通信称为中继通信,如图3-4所示。中继台是一种自动接力设备,当它收到信号以后,便启动发射机,把信号转发出去。由于中继台具有很好的接收性能和比较大的发射功率,所以通过它的转发,手持业余电台或车载业余电台的通信距离可以增加到几十千米甚至几百千米。

\begin{figure}[htbp]
	\centering
	\includegraphics[width=0.7\linewidth]{38}
	\caption{中继通信}
	\label{fig:1}
\end{figure}

\subsubsection{上行频率和下行频率}

中继台由一个发射机、一个接收机和一副天线组成,因此它与一般电台没有本质的区别。中继台的最大特点是接收机与发射机使用不同的频率。习惯上,把中继台的接收频率称为上行频率,发射频率称为下行频率。上行频率和下行频率的差值,叫做频差。频差的大小各地不尽相同,推荐标准是:UHF频段5MHz,VHF频段0.6MHz。

在使用中继台之前,必须将自己电台的发射频率和接收频率设置为两个不同的值。例如中继台上行频率是434.225MHz,下行频率是439.225MHz,则我们应该把439.225MHz设置成为自己的接收频率,把434.225MHz设置为自己的发射频率;或者在439.225MHz的基础上设置5MHz的频差。

\subsubsection{亚音}

“亚音”是“连续音调静噪控制系统(CTCSS)”或“亚音频静噪控制系统”的俗称。使用中继台通常都要求操作者发送亚音,中继台只有接收到你的亚音信号,才允许使用它。启用亚音功能并不是要限制爱好者使用中继台,而是为了防止外界干扰。如果没有亚音,任何无关的信号都有可能触发中继台。

实际上,亚音是一个低频信号,亚音的频率比我们耳朵可以听到的最低音频还要低,但中继台可以识别它。那么如何向中继台发送亚音呢?这就需要启动电台的CTCSS功能。启动CTCSS功能之后,电台会将亚音信号附加在正常的音频信号上一同发射出去。目前大部分VHF/UHF电台都有CTCSS功能。常用的亚音频率见表3-6。

表3-6 常用的亚音频率(Hz)

\begin{figure}[htbp]
	\centering
	\includegraphics[width=0.7\linewidth]{39}
	\caption{中继通信}
	\label{fig:1}
\end{figure}

若中继台CTCSS亚音频率设置为88.5Hz,则你使用的电台亚音也必须设置在88.5Hz。当然,只需让发射信号带有亚音就行了,接收可以不用设置。

\subsubsection{中继操作}

· “上”中继

当一切设置就绪,在频率空闲的情况下,应先测试能否“上”中继。拿起话筒说:“BA4AAA”以便引起别人注意。松开发射键(PTT)后,你会收到一个很短的没有调制的载频信号,手持台发出“嚓”的一声,这说明中继台处于正常的工作状态。如果有人想与你联络的话,他会呼叫你。只要通信各方都能“上”中继,就能建立中继通信。

· 报呼号

业余电台在每次通信建立以及结束时,应当主动报出本台的完整呼号。如果你要呼叫某个人,可直接呼叫:“BA4BBB,我是BA4AAA”。不要在呼叫时省略自己的呼号,而让被叫电台回答你时再询问你是谁。如果只是两个人在中继台上通话,可以不报对方的呼号,只报自己的呼号即可。

· 留有间隙

不要在对方松开PTT键时立即按键发射,每次发信前留有一定的间隙,可使其他人利用这一短暂的时间插入。如果你在通话中听到他人要求插入,应给予答复。不答复或答复却不给他发信的权力都是不礼貌的。

· 移动台优先

中继台通常是供直接联络有困难的爱好者使用的。在中继台繁忙的时候,应优先让移动台使用。移动台往往在某个地区才能打开中继台,移动到其他地区可能就无法使用中继。每次通联时间应尽可能短,以便让更多的爱好者使用中继台。如果能直接通联的话,就不要使用中继台。

我国许多城市都建立了2m和70cm频段的中继台。在你出差、旅游时,可以上网查找沿途城市中继台的频率,了解当地中继台的使用方法。

\section{QSL卡片}

QSL卡片是业余电台之间确认互相联络或收听报告的凭证,图3-6和图3-7所示分别为QSL卡片的正、反面。它多以14cm×9cm或15cm×10cm的硬质纸卡印制而成。每个业余电台都应有自己的QSL卡片,以便与世界各地的电台联络后互相交换。QSL卡片不仅是在一定程度上反映每个电台通信能力的标志,而且还是业余电台向有关机构申请各类奖状、证书的凭证。

\begin{figure}[htbp]
	\centering
	\includegraphics[width=0.7\linewidth]{40}
	\caption{QSL卡片正面}
	\label{fig:1}
\end{figure}

\begin{figure}[htbp]
	\centering
	\includegraphics[width=0.7\linewidth]{41}
	\caption{QSL卡片反面}
	\label{fig:1}
\end{figure}

来自世界各地的QSL卡片都填有对本台信号的收听情况及有关的各类数据,所以它又是研究、改进电台设备极好的技术资料。业余无线电爱好者们都以能收集到世界各地,尤其是那些业余电台稀少地区的QSL卡片为最大的乐趣。

\subsection{QSL卡片上的内容}

\begin{itemize}
	\item 醒目的本台呼号。
	\item 中国无线电运动协会会徽图案。
	\item CQ分区、ITU分区。
	\item 发至何台。
	\item 确认双方的联络或收听台的报告。
	\item 联络的时间。
	\item 联络时所用的频率。
	\item 联络时所用的操作方式。
	\item 对方的信号情况,即RST报告。
	\item 本台所在地的中英文详细地址。
	\item 集体台应有本台的中英文台名。
	\item 操作者签名。
\end{itemize}

除以上必须包括的内容外,许多爱好者的QSL卡片上还印有介绍自己设备、天气情况及附言等的栏目。

\subsection{正确填写QSL卡片}

QSL卡片既然是一种联络凭证,填写时就必须如实、认真,字体应清楚、正规,对方呼号中的英文字母最好使用大写印刷体,时间一律使用世界协调时(UTC),数字“0”应写成“Ø”。签名时应把全名写上,寄给国内的用汉字,寄给国外的用汉语拼音。在图3-7所示的示例中,如果发出的是确认联络的卡片,则在“OUR QSO”一栏中打上勾;如果是发给某收信台,确认收听报告的卡片,只要在“YOUR REPORT”一栏前画上勾。收信台卡片中“CLG/WKD”一栏,如果听到某台正在呼叫,也可以填写收听报告卡片,只需将这一栏的“WKD”划去,留下“CLG”并将你听到的呼叫内容填上即可,如呼叫“CQ”即填写“CQ”,如呼叫“BY1PK”就填写“BY1PK”。任何项目都不能填错,一旦填错,必须另换一张重新填写。涂改过的QSL卡片一律无效。

\subsection{QSL卡片的交换}

交换QSL卡片通常采用直接交换或经卡片管理局交换的方式。

\subsubsection{直接交换}

直接交换就是将卡片按对方的通信地址直接邮寄,因此在联络时应将对方的详细地址询问清楚。同样,如果希望对方也将卡片直接寄给自己,也应在联络时将自己的详细地址报给对方。

寄往国外的信封书写格式与我们平时使用的国内信函不一样。国际信函的收信人姓名、地址写在右下方,寄信人的姓名、地址写在左上角或信封的背面。姓名、地址的书写顺序也正好与国内信函相反,姓名在地址的前面,地址由小到大排列。图3-8所示是南京市五塘中学寄往美国的信封,右下角4行的内容从上到下分别为收信人姓名、门牌号及街道名、城市名及州名的缩写和邮政编码、国名。

\begin{figure}[htbp]
	\centering
	\includegraphics[width=0.7\linewidth]{42}
	\caption{寄往国外的信封书写格式}
	\label{fig:1}
\end{figure}

\subsubsection{经过卡片管理局交换}

很多国家和地区都设有全国性或地方性的卡片管理局,专门负责转寄国外爱好者的QSL卡片。通过管理局交换卡片比较方便,还可以节省邮资,但由于中转环节较多,收卡时间比较长。

寄发收听报告卡片的一般原则是只要听到对方信号,无论是呼叫还是联络,都可以向这个电台寄发收听报告卡片,当然,这要以听到对方的呼号为前提。没有听到呼号或从联络对方的呼叫中知道其呼号是不能寄发收听报告卡片的。

\subsection{QSL卡片的制作}

QSL卡片分为双面印刷和单面印刷两种。双面印刷的QSL卡片,一般在正面印有图案、照片和自己的呼号等,反面则为需要报告的各项内容。单面印刷的QSL卡片,就是把自己的呼号和要报告的内容都印在卡片的一面,这种卡片简单、明了,印刷成本较低,为不少爱好者所青睐。

除购买空白的QSL卡片或委托印刷厂印制外,还可以用黑白或彩色打印机打印卡片。每次打印之前,爱好者都可以临时设计一种新式样或者一张新图片,因此这种卡片是完全个性化的卡片。制作时选用250~300g的A4纸正好可以打印4张。所需的中国无线电运动协会会徽图案及QSL卡片模板可从网上下载(http://www.qslprinter.com)。

图3-9所示为2008年北京奥运会特设业余电台QSL卡片。

\begin{figure}[htbp]
	\centering
	\includegraphics[width=0.7\linewidth]{43}
	\caption{2008年北京奥运会特设业余电台QSL卡片}
	\label{fig:1}
\end{figure}

\section{电台日志}

电台日志(Station Log)是无线电爱好者在电台联络时,用来登记各种数据、资料以及联络情况的原始记录,它是业余电台唯一的工作记录,也是寄发QSL卡片的依据。

各国电台日志的内容大致相同。我国电台日志包括的内容有:日期、开始时间、结束时间、使用频率、对方呼号、操作方式、信号情况、内容摘要、QSL卡片的收发日期、签名。

填写电台日志和填写QSL卡片一样,时间一律使用协调世界时(UTC),呼号中的英文字母应使用大写印刷体,数字“0”应写成“Ø”,以防止和英文字母“O”相混,所有内容填写应真实可靠。电台日志必须永久保存。

\section{协调世界时}

协调世界时(UTC)是由国际无线电咨询委员会规定和推荐的国际标准时间。1979年12月在日内瓦举行的世界无线电行政大会已决定采用协调世界时来取代格林尼治时间,作为无线电通信的标准时间。

由于地球的自转,世界各地的时间是不相同的。为了计时方便,人们将地球按经度分为24个时区,以经度0°(本初子午线)为基准,东经7°30'与西经7°30'之间的区域为零时区,东经7°30'与22°30'之间的区域为东1区,西经7°30'与22°30'之间的区域为西1区,以此类推。每个时区横跨经度15°,相邻两个时区时间相差1小时。

协调世界时与零时区时间相同,它与世界各国的时间有一个固定的差值。我们可以根据各地所处的时区将当地时间换算成协调世界时。例如,北京时间是把我国东8区的时间作为全国统一的时间,它比协调世界时早8小时。北京时间2008年1月1日22点换算成协调世界时就是2008年1月1日14点,协调世界时用4位数字表示,14点写成1400。

\chapter{跟着电波旅行}

引子
“能收听到中国吗?”
“这需要非常好的天气。”
“能和月亮通话吗?”
“如果有强大的无线电,我看可以。”
——电影《接触未来》中艾莉与父亲的对话

电波的传播是一种奇妙的现象。对业余无线电爱好者来说,了解无线电波是如何传播的,对于顺利完成远距离通信实验很有帮助。当你操作业余电台与远方的朋友通联,亲自体验到电波传播的奇异特性时,才会感受到业余无线电通信的真正魅力。

\section{电磁波的产生}

1888年德国物理学家赫兹用实验证明了电磁波的存在,为电磁波的应用打开了大门。1901年,马可尼用10kW的音响火花式电报发射机,首次完成了横跨大西洋的无线电远距离通信。由于马可尼的卓越贡献,他获得了1909年度的诺贝尔物理学奖,如图4-1所示。

\begin{figure}[htbp]
	\centering
	\includegraphics[width=0.7\linewidth]{44}
	\caption{无线电通信的奠基人马可尼}
	\label{fig:1}
\end{figure}

马可尼早期的无线电发射机属于“火花隙发射机”(见图4-2),即用高电压在两个金属电极之间形成“火花隙”激发火花放电,电火花产生的电磁波从天线辐射出去。接收机采用粉末检波器,可以带动记录仪或者电铃等其他电路。无线电通信就是利用电磁波来传递信号的。

\begin{figure}[htbp]
	\centering
	\includegraphics[width=0.7\linewidth]{45}
	\caption{马可尼早期发射机电路}
	\label{fig:1}
\end{figure}

一块石头丢在水里,激起的水波会向四周传播。人讲话时声带振动引起周围空气的振动,使声音由近及远地向外传播,形成声波。波是自然界普遍存在的现象。当导体中有迅速变化的电流时,在周围空间就会有电磁波向外辐射。

在水波情形中,水面上出现的凸起部分和凹下部分分别称为波峰和波谷,在1s内出现的波峰(或波谷)数叫做水波的频率,频率单位叫赫兹(Hz)。相邻两个波峰(或波谷)之间的距离叫做波长,1s内波传播的距离叫做波速。跟水波类似,电磁波也有自己的频率和波长,同样也可以用波形图来描述。电磁波的传播速度非常快,与光速相同,在真空中传播速度约为3×108m/s,参见图4-3和图4-4。

\begin{figure}[htbp]
	\centering
	\includegraphics[width=0.7\linewidth]{46}
	\caption{波是自然界普遍存在的一种现象}
	\label{fig:1}
\end{figure}

\begin{figure}[htbp]
	\centering
	\includegraphics[width=0.7\linewidth]{47}
	\caption{波长}
	\label{fig:1}
\end{figure}

\section{电离层与对流层}

地球被大气包围着,包围地球的大气圈也称为“大气层”。以地面为界,越向上空气密度越小,最后进入星际空间的空气极其稀薄。人类生活在大气海洋的最底层。大气圈在垂直方向上因物理、化学特性的差异可分为若干层次。如以温度随高度分布的特征,在垂直方向上可分为:对流层、平流层、中间层、热层和外大气层。如按大气层电离状况,可把大气分为电离层和非电离层。大气圈质量的99.9%集中在50km以下的大气层里。
在大气层的不同高度,无线电波传播方式是不一样的。对流层和电离层是电波传播的主要区域,绝大多数的电波反射和折射都发生在这两个区域内。

\subsection{电离层}

电离层的高度从距地面50km一直延伸到约1000km的高度,太阳辐射使这个高度的部分气体分子电离成自由电子和正离子,所以这个区域叫做电离层。无线电波射入电离层后会产生折射、反射和散射现象,如图4-5所示。

\begin{figure}[htbp]
	\centering
	\includegraphics[width=0.7\linewidth]{48}
	\caption{大气中的电离层}
	\label{fig:1}
\end{figure}

注意:白天、黑夜电离层是不一样的(夜晚D层消失,F1、F2合为一层)。

电离层从低到高依次是D层、E层和F层,白天太阳辐射使F层分离为F1、F2两层。电离层对短波传播具有重要的影响。

高度最低的D层距离地面60~90km,D层白天出现,夜晚消失。D层的电子密度不足以反射短波,不过电波在通过D层时,将遭受严重的衰减。E层出现在距地面100~120km的高度,和D层一样,E层出现在太阳升起的时候,太阳落山后,E层对短波传播已不起作用。F层距地面约300km的高度,对于短波传播,F层是最重要的。在通常情况下,短波远距离通信都是通过F层的反射完成的。

电离层一般不能反射频率为30MHz以上的无线电波,VHF/UHF频段的无线电波通常穿过电离层而无法反射回地面。只有电离层中出现较强的Es层时,VHF电波才有可能被反射。

\subsection{对流层}

对流层的高度从地面向上延伸至约10km,对无线电波传播产生影响的通常是2km以下的大气层。VHF/UHF频段的无线电波通常在对流层中传播的距离近得多,不能像经电离层反射那样传播得很远。由于大气的折射,电波在对流层中传播时路径会发生弯曲。在一些特定的气候条件下,某个地区能形成一条传播VHF/UHF电波的通道,电波沿着这个通道可以传播到很远的地方,这种现象通常称为“大气波导”,如图4-6所示。

\begin{figure}[htbp]
	\centering
	\includegraphics[width=0.7\linewidth]{49}
	\caption{“大气波导”现象}
	\label{fig:1}
\end{figure}

\subsection{电波传播方式}

\subsubsection{地面波传播}

无线电波沿着地球表面传播的方式称为地面波传播。地面波信号比较稳定,基本上不受气象条件及季节变化的影响。但随着电波频率的增高,传播损耗也迅速增大。因此,这种传播方式主要用于长波和中波的广播、导航以及短波与超短波的近距离通信。

\subsubsection{天波传播}

天波传播是指电波从地面射向空中,经高空电离层反射后回到地面的传播方式,如图4-7所示。由于天波传播损耗小,电波经电离层一次反射(单跳)可传播4000km,还可能被地面再次反射进入电离层,形成电离层的多次反射。对长波、中波和短波,可利用电离层反射实现远距离甚至环球传播。

\begin{figure}[htbp]
	\centering
	\includegraphics[width=0.7\linewidth]{50}
	\caption{天波传播}
	\label{fig:1}
\end{figure}

在短波波段上利用天波传播时,用较小的功率就可以进行远距离通信。短波设备简单、成本低,在业余无线电通信中,运用最多的就是天波传播方式。

\subsubsection{天波传播}

无线电波在视线范围内的传播称为视距传播,如图4-8所示。视距传播可分为地—地视距传播和地—空视距传播。通常高于30MHz的电波能直接穿过电离层,视距传播的工作频段主要为超短波及微波波段。

\begin{figure}[htbp]
	\centering
	\includegraphics[width=0.7\linewidth]{51}
	\caption{视距传播}
	\label{fig:1}
\end{figure}

视距传播也是业余无线电爱好者常用的一种电波传播方式,卫星通信、月面反射通信都是利用了这种传播方式。

\section{天气对传播的影响}

天气对VHF/UHF电波的传播有着极大的影响。在特定的天气条件下,2m、70cm波段的通信距离可达4000km以上。而通常情况下VHF/UHF电波沿对流层只能传播几十千米。

\subsection{视线距离}

电波与光波相似。从天线发射出来的无线电波沿着直线传播,遇到障碍物发生反射或衍射。电波传播越远,强度越小,最终变得极其微弱。

由于地球是球形,凸起的地表面会挡住视线。视线所能达到的最远距离称为视线距离,如图4-9所示。在标准大气折射时,视线距离D可用公式$ D = 4.12\left ( \sqrt{H_{1} } +  \sqrt{H_{2} } \right ) km $来计算。从公式中可以看出,随着收、发天线高度H1、H2的增大,视线距离也逐渐增加。

\begin{figure}[htbp]
	\centering
	\includegraphics[width=0.7\linewidth]{52}
	\caption{视线距离与地球曲率的关系}
	\label{fig:1}
\end{figure}

通常VHF/UHF频段无线电波和光线一样沿直线传播。从视线距离的计算公式可以看出,超短波电台之间的通信距离由天线的高度来确定。但实际情况要比公式计算复杂得多。

\subsection{大气波导}

一般我们认为超短波无线电波和光线一样是直线传播,但由于大气折射率的微小变化,实际上它们在对流层中的传播路径通常是弯曲的。对于对流层电波传播而言,影响无线电波传播的主要因素是大气折射。

这个现象很容易用光线来说明。当光线通过不同折射率的媒质时,传播方向将发生改变,如图4-10所示。这也很容易说明当一根棒子斜插入水中时看上去变得弯曲了。电波传播也具有相同的性质。

\begin{figure}[htbp]
	\centering
	\includegraphics[width=0.7\linewidth]{53}
	\caption{光的折射}
	\label{fig:1}
\end{figure}

不同地区、不同季节和时间,大气折射率的分布情况是不一样的。天气的变化直接影响到对流层不同高度折射率的大小,在不同的天气条件下,无线电波传播的路径是不一样的。如图4-11所示,大气折射使电波传播距离超过视距。当出现超折射现象时,射向对流层中的无线电波经大气折射返回地面,再经地面反射后进入大气,然后在大气中折射,重新回到地面。也就是说此时的电波借助于连续的大气折射和地面反射传播,这一现象和金属波导中的无线电传播很相似,因而称为“大气波导”。电波沿着大气波导层可以传播到意想不到的远方。

\begin{figure}[htbp]
	\centering
	\includegraphics[width=0.7\linewidth]{54}
	\caption{大气折射使电波传播距离超过视距}
	\label{fig:1}
\end{figure}

超折射通常出现在稳定、晴朗的天气条件下。当大气中某一高度范围内出现气温随高度的增加而升高的逆温层,并且湿度随高度急剧下降,这时容易出现超折射现象。

逆温层与VHF/UHF远距离通信是密切相关的。在对流层中,气温通常是随高度的增加而降低。在晴朗微风的夜间,地面辐射冷却很快,贴近地面的空气层也随之降温,于是便形成了自地面开始的逆温层。这种由于地面强烈辐射冷却而形成的逆温称为辐射逆温。一般情况下,辐射逆温可以增加VHF/UHF电波的通信距离。随着逆温的加强以及逆温层厚度的增加,还可能形成大气波导。实际通联记录表明,VHF/UHF电波沿大气波导可以传播4000km之遥。

我国是一个地域辽阔、地形和气候复杂的国家,出现适合远距离传播的时间在不同的地区是不一样的。全国各地的爱好者都可以研究这一有趣的现象。每年春夏,在我国沿海地区常有相对稳定的逆温层,这给VHF/UHF波段远距离通信创造了非常有利的条件。如青岛的爱好者可以打开浙江平湖、江苏昆山、上海等地的400MHz中继台和当地的HAM进行QSO。

\subsection{手持业余电台能通联多远}

许多初学者购买手持业余电台时常常询问一个问题:这部手持业余电台能通联多远的距离?这个问题可不好回答。电台的通信距离是一个不确定的数值,它不仅取决于发信机功率的大小、天线的增益、收信机灵敏度,还与电波传播环境有关。这个看似简单的一个问题,其实是一个非常复杂的电波传播问题。

下面我们以输出功率5W的手持业余电台为例,介绍不同传播方式下的通联距离。

发射功率5W的手持业余电台在开阔地的通信距离一般可达6km。在市区内,由于电波通过建筑物的反射及衍射产生的损耗,通信距离大为缩短,一般只有1~3km。如果通信中的一方站在高山上,通信距离又会成倍增加,甚至有的爱好者直接用手持业余电台打开300km高的卫星中继台,进行业余卫星通信。
为什么当通信中的一方站在高楼或高山上时,通信距离会显著增加呢?这要从电波传播的特性来理解。图4-12中站在地面上B位置的人移至山顶C处,电波传播路径发生了改变,由靠近地面的AB变成了远离地面的AC,此时由地面引起的传播损耗减少了许多,通信距离自然也就增加了。

\begin{figure}[htbp]
	\centering
	\includegraphics[width=0.7\linewidth]{55}
	\caption{不同的传播路径电波的损耗是不同的}
	\label{fig:1}
\end{figure}

假设甲、乙二人用手持业余电台在没有任何障碍物、没有干扰的宇宙空间中通联,发射功率为5W(37dBm)、接收灵敏度为0.15μV(-129dBm)、通信频率为435MHz,他们最远的通信距离有多大呢?根据自由空间传播损耗的规律,我们可以计算出通信距离达10000km!有点难以置信,这可远远超出我们的经验。

从上面的叙述中可以看出,电波在自由空间中可以传播很远的距离,这就是为什么在某些特殊的条件下,VHF/UHF电波可以传播几千千米的原因了。

\section{神秘的电波}

无线电波在传播过程中受到地面、大气及电离层的影响,地球的自转和季节的交替都会导致无线电波的传播情况发生改变,这使得远程通信变得非常复杂。

经过长期的研究,人们已经掌握电波传播的一些规律。例如你可以根据定期发布的地球磁场报告,预测HF波段电波的传播;根据天气情况推算VHF/UHF波段的传播什么时候开放。尽管如此,仍然有许多未知因素影响着电波的传播。或许有一天你在VHF频段上会突然听到1000km以外某个火腿的呼叫。也许正是这种不可预测的特性,吸引了无数爱好者在不断探索电波传播的奥秘。

\subsection{太阳黑子周期}

太阳黑子是人们最早发现也是人们最熟悉的一种太阳表面活动。明亮的太阳光球表面经常出现一些暗黑斑点,这就是太阳黑子,见图4-13。太阳黑子是太阳活动的基本标志。

\begin{figure}[htbp]
	\centering
	\includegraphics[width=0.7\linewidth]{56}
	\caption{太阳黑子}
	\label{fig:1}
\end{figure}

太阳黑子的数量并不是固定的,它会随着时间的变化而改变,每11年会达到一个最大值,这11年的时间就被称为一个太阳黑子周期。从1755年开始,人们已经记录了23个太阳黑子周期。第23个太阳黑子周期开始于1996年的10月份,目前太阳黑子正处在第24个周期中,见图4-14。

\begin{figure}[htbp]
	\centering
	\includegraphics[width=0.7\linewidth]{57}
	\caption{太阳黑子周期}
	\label{fig:1}
\end{figure}

电离层与太阳黑子活动有着密切的联系。在太阳黑子最多的时候,电离层的电子密度增大,可以反射回地面的无线电波频率也随之增高。因此短波高频段在太阳黑子极大期的时候通联得很好。当我们接近太阳黑子活动的极小期时,会产生另一个有趣的现象,就是低频段的传播好于高频段。

在太阳活动高峰时期出现的太阳耀斑,会辐射大量的紫外线、X射线,对电离层产生扰动,电波传播状态急剧变化,常常出现短波通信中断。

\subsection{Es层传播}

电波传播中存在着许多奇怪的现象。例如6m波段会突然开放,能进行几千千米的远距离通信。为什么会发生这种现象?科学家们至今还没有给出一个完整的理论解释。

电离层偶发E层简称Es层。它是距地面95~130km的电离层E层内不规则的电离密集薄层,它的电子密度超出邻近区域许多,可以反射30~150MHz的无线电波。经过Es层的一次或多次反射,电波传播距离可达1000~10000km。

Es层的出现具有突然性,形成和持续的时间无法预测。Es层的成因目前人们还没有完全弄清楚,这也是有待于今后研究的一个课题。Es层活动有昼夜变化及季节变化,并随地理经、纬度而异。能反射2m波段的Es层很少出现。各国电波研究机构经过长期的研究发现,在中纬度地区,每年5月至8月是Es层出现最频繁的时期。从每天的统计结果来看,中纬度地区Es层经常出现在当地时间8∶00~12∶00及19∶00~23∶00两个时间段上。东亚及我国大部分地区是Es层的频发区,如果我们注意利用这个有利条件,可能在今后的通联中会有意想不到的收获。

由于Es层的产生和消失具有突然性,持续时间也无法估计,有时甚至传播开通只有几秒钟,这就要求操作者能够熟练地使用通信设备和通信用语,在较短的时间内快速完成一次远距离通联。

国内外有许多热衷于调频广播远距离接收活动的爱好者,接收远地调频广播电台信号也是了解Es层电波传播规律的一个可行的方法。我国调频广播频率(87~108MHz)介于6m波段(50~54MHz)和2m波段(144~148MHz)之间。调频广播远距离接收爱好者统计的Es层出现的规律与许多电波研究机构的结论是一致的。图4-15所示是美国雷克星敦的Girard M. Westerberg通过接收远距离调频广播统计的2003年Es层出现天数的情况。

\begin{figure}[htbp]
	\centering
	\includegraphics[width=0.7\linewidth]{58}
	\caption{Es层传播的季节变化}
	\label{fig:1}
\end{figure}

\subsection{信标台}

为了帮助业余无线电爱好者及时了解电波传播的规律,许多国家和地区的业余无线电组织在世界各地建立了信标台(Beacon),如图4-16所示。这些信标台24小时连续工作,以固定的频率、内容和方式向外发出信号,爱好者通过接收各地的信标台信号可以掌握电波传播情况。

\begin{figure}[htbp]
	\centering
	\includegraphics[width=0.7\linewidth]{59}
	\caption{位于澳大利亚西部的HF波段信标台VK6RBP}
	\label{fig:1}
\end{figure}

美国北加利福尼亚州远程通信基金会(NCDXF)与国际业余无线电联盟(IARU)合作,在全世界建立了18个HF波段信标台。这个信标网在14.100MHz、18.110MHz、21.150MHz、24.930MHz、28.2008MHz频点上用莫尔斯电码发送信标呼号,最大发射功率是100W。详细情况可访问www.ncdxf.org/beacons.html获取。

\chapter{莫尔斯电码与电报通信}

引子

1844年5月24日,莫尔斯在华盛顿向64km外的巴尔的摩发出了人类历史上的第一份电报。报文只有一句话:“上帝创造了何等的奇迹”。

无线电报是最早的无线电通信模式,无线电报又称为CW。电报通信曾在无线电通信中起过重要的作用。时至今日,CW仍然被一些火腿所喜爱。

\section{莫尔斯电码}

1837年,美国科学家莫尔斯经过反复试验,终于成功研制出最早的电磁式电报机。他用较短的电流脉冲信号“嘀”和较长的电流脉冲信号“嗒”给26个英文字母和数字建立了编码。短信号“嘀”用符号表示为“·”,长信号“嗒”用符号表示为“—”,这就是今天的莫尔斯电码,如表5-1~表5-3所示。为了准确传递电码,不论收、发报速度有多快,构成电码符号的点划之间、字符之间、单词之间的间隔必须保持一定的比例关系。如果以“嘀”的持续时间作为基本单位,那么长信号“嗒”相当于3个“嘀”的长度,同一字符中点划之间的间隔都是一个“嘀”的长度,字母之间要有3个“嘀”的间隔,单词之间要有5~7个“嘀”的间隔。

\begin{figure}[htbp]
	\centering
	\includegraphics[width=0.7\linewidth]{60}
	\caption{字母的莫尔斯电码}
	\label{fig:1}
\end{figure}

\begin{figure}[htbp]
	\centering
	\includegraphics[width=0.7\linewidth]{61}
	\caption{数字的莫尔斯电码}
	\label{fig:1}
\end{figure}

\begin{figure}[htbp]
	\centering
	\includegraphics[width=0.7\linewidth]{62}
	\caption{标点符号的莫尔斯电码}
	\label{fig:1}
\end{figure}

莫尔斯电码很简单,特别适合无线电通信。早期的无线电先驱们使用非常简单的设备便可以将信息发送出去。

除了业余无线电爱好者学习使用莫尔斯电码外,普通人在日常生活中也会遇到莫尔斯电码。有的手机短消息的提示音“嘀嘀嘀、嗒嗒、嘀嘀嘀”就是莫尔斯电码的“SMS”,即短消息服务“Short Message Service”的缩写。电信号可以传递莫尔斯电码,声音、光线也可以作为莫尔斯电码的载体。在电影《尼罗河上的惨案》中,侦探波洛发现身边有一条眼镜蛇时,不敢大声求救,而是轻轻敲击墙壁,发出类似“嘀嘀嘀、嗒嗒嗒、嘀嘀嘀”(SOS)的声音,来通知隔壁的上校军官来救他。

学习莫尔斯电码需要经常练习。你可以从网上下载一个莫尔斯电码练习软件,如CW\_PLAYER,如图5-2所示。通过参数设置,计算机可以产生不同速度的报文,供听抄练习。在我国,申请二级以上个人业余无线电台操作证书的爱好者都要参加莫尔斯电码收、发内容的考核。

\begin{figure}[htbp]
	\centering
	\includegraphics[width=0.7\linewidth]{63}
	\caption{莫尔斯电码练习软件}
	\label{fig:1}
\end{figure}

\section{Q简语}

汉语中有许多由4个字组成的成语,简单明了,用来表达特定的含义。在业余无线电通信中,有一种特殊的通信语言——Q简语。Q简语以字母“Q”开头,由3个字母组成的词组来表达一个完整的意思,如QRZ、QTH、QSO等。Q简语可以用于CW通信,也可以在语音通信中使用。

例如:“你的信号QRM,请QSY到431.500MHz。”意思是“你的信号受到其他台的干扰,请将频率改到431.500MHz”。

常用Q简语见表5-4。

\begin{figure}[htbp]
	\centering
	\includegraphics[width=0.7\linewidth]{64}
	\caption{常用Q简语}
	\label{fig:1}
\end{figure}

\section{业余通信常用缩语}

业余电台通信中除使用Q简语外,还常常使用一些通信缩语。这些缩语有的用于CW通信,有的是业余电台通信中惯用的。灵活使用Q简语和通信缩语,可使通信中的语言更加简洁、专业,可以提高业余电台的通信效率。

业余通信常用缩语见表5-5。

\begin{figure}[htbp]
	\centering
	\includegraphics[width=0.7\linewidth]{65}
	\caption{业余通信常用缩语}
	\label{fig:1}
\end{figure}

\begin{figure}[htbp]
	\centering
	\includegraphics[width=0.7\linewidth]{66}
	\caption{业余通信常用缩语}
	\label{fig:1}
\end{figure}

\section{CW通信}

CW通信已有100多年的历史。随着语音通信模式、数字通信模式的出现,使用CW的爱好者有所减少。不过CW模式占用带宽小,设备简单,可以在恶劣的环境下比语言模式更可靠地通信。这些优点仍然吸引着许多HAM坚持使用这一模式。实际上很多实验性的低功率发射机都采用CW模式。

绝大多数HF电台都包含CW模式,有的VHF/UHF电台上也有CW模式。如果你想进行CW通信,需要准备一个拍发莫尔斯电码的电键。

拍发莫尔斯电码的最基本设备是手工电键和自动电键。如果电键是球形键钮,一般采用“跪式”,即右手拇指自然扶住键钮内侧腰部,中指弯曲,第一关节跪于键钮外侧底盘上,食指成弧形,指端放在键盘钮顶部前沿,用三指合力握住键钮。如果使用的是平键钮,则食指、中指并拢后弯成弧形放在键钮的平顶上,拇指自然靠在键钮的左侧,如图5-3所示。

\begin{figure}[htbp]
	\centering
	\includegraphics[width=0.7\linewidth]{67}
	\caption{拍发莫尔斯电码的手工电建}
	\label{fig:1}
\end{figure}

自动电键(见图5-4)也是发送莫尔斯电码的常用设备。自动电键中的电子电路能够产生连续的“嘀”或“嗒”信号,使用时只要控制“嘀”或“嗒”信号的多少。自动电键的操作者将两个手指放在键体的两侧,拨动左边的键体,拍发“嗒”信号,拨动右边的键体,拍发“嘀”信号。自动电键有一个调节装置,可以选择不同的拍发速度。自动电键比较容易掌握。经过一段时间的练习后,就可以拍发出流畅的莫尔斯电码了。

\begin{figure}[htbp]
	\centering
	\includegraphics[width=0.7\linewidth]{68}
	\caption{自动电键}
	\label{fig:1}
\end{figure}

爱好者可以在业余频段上的某个频点拍发CQ,然后等待其他火腿应答。拍发CQ的格式是:

CQ CQ CQ DE BA4RM BA4RM BA4RM K

如果听到有人呼叫CQ,当听到最后的字母K时,你可以这样拍发应答对方:

JM1XAA JM1XAA DE BA4RM BA4RM K

通信结束时,你可以拍发下列报文结束CW通信:

TNX QSO 73 SK JM1XAA DE BA4RM TU

\chapter{业余电台竞赛活动}

引子

我喜欢参加业余无线电竞赛,我会为获得好成绩而欢呼,也会为失去一些分数而懊恼。但最重要的是我参加了,我感受到了快乐。
——BA4RF

业余无线电通信竞赛是一项充满刺激和挑战的比赛活动。世界各国业余无线电团体每年都定期举行多种多样的通信竞赛。竞赛是对每位参赛者的技术和设备的综合考核。为了不断创造新的成绩,爱好者们常常废寝忘食、乐此不疲。

\section{DX通信}

DX通信意为与远距离电台进行通联,一般与本国以外电台的通联均可称为DX通信。DX通信除了与通信距离有关外,还与通信难度和电台稀有程度有关。有许多爱好者花费大量的时间与精力搜寻世界各地的电台,不断刷新他们的DX通联纪录。

\subsection{DXCC和“DXCC国家”}

DXCC是美国的业余无线电协会(ARRL)的一个活动。ARRL按照一定的规则,将全世界分成很多个“DXCC国家”。这种用于爱好者计算通联成绩的“国家”与传统意义上的国家并不完全相同。目前,全世界一共有341个(2011年)这样的“DXCC国家”,见表6-1。大多数火腿的目标是尽可能多的联络不同“DXCC国家”的电台。每当联络到一个稀有的业余电台,尤其是那种人迹罕至地区的电台,火腿们都会兴奋不已,其快乐的心情是难以用语言形容的。

表6-1 DXCC呼号前缀对照表(部分)

\begin{figure}[htbp]
	\centering
	\includegraphics[width=0.7\linewidth]{69}
	\caption{}
	\label{fig:1}
\end{figure}

\begin{figure}[htbp]
	\centering
	\includegraphics[width=0.7\linewidth]{70}
	\caption{}
	\label{fig:1}
\end{figure}

DX通信比本地通联更有乐趣、更具有挑战性。DX通联的数量与难度反映了操作者专业水平的高低,也是爱好者申请各种奖状的主要依据。每到比赛时,在竞赛频点上挤满了猎取远距离稀有电台的火腿们。

\subsection{DX远征}

一些稀有的“DXCC国家”没有火腿居住,那里也没有业余电台。如果某天一群火腿访问这些“国家”,并在有限的时间内进行了DX通信,那么这种活动就叫做DX远征。例如从1994年至2007年,中外业余无线电爱好者对黄岩岛共进行了4次远征活动,如图6-1所示。

\begin{figure}[htbp]
	\centering
	\includegraphics[width=0.7\linewidth]{71}
	\caption{BS7H黄岩岛远征}
	\label{fig:1}
\end{figure}

DX远征的主要目的是满足全球无线电爱好者与远征所在地通联的需求。远征计划一般会在网上发布。规模稍大一些的远征活动通常还有宣传网站,让全球的爱好者能够迅速了解远征的情况。

\subsection{寻找DX电台}

寻找DX电台除了不停地转动调谐旋钮外,还可以利用互联网访问DX节点。大部分DX节点采用网站的形式发布信息,如图6-2所示。比如你可以浏览www.dxsummit.fi来了解DX电台的信息,知道哪些频率上有哪些电台在工作。

\begin{figure}[htbp]
	\centering
	\includegraphics[width=0.7\linewidth]{72}
	\caption{DX节点网站}
	\label{fig:1}
\end{figure}

DX通信时还要了解一天中最佳的通信时间,知道灰线什么时候掠过本地。灰线是地球上白天和黑夜交界的地方(见图6-3),即晨曦和黄昏的区域。这块区域具有一些独特的传播特性。由于没有阳光照射,所以低空的D层还没有形成或者刚好消失,电波被吸收得很少,传播损耗小,而F层的反射能力较强,这个时段对DX通信十分有利。爱好者可以通过专门的传播预测软件,了解在什么地方、什么时候通联的可能性最大。

\begin{figure}[htbp]
	\centering
	\includegraphics[width=0.7\linewidth]{73}
	\caption{灰线是有利于DX通信的区域}
	\label{fig:1}
\end{figure}

\section{CQWW远距离世界比赛}

业余无线电通信竞赛的目标很明确,就是在规定的时间内尽可能多地通联到远地电台。

CQWW远距离世界比赛是一个在业余无线电界中影响最大、参加人数最多的竞赛活动,它是由美国《CQ》杂志举办的。CQWW DX比赛时间为:每年9月最后一个周末为无线电电传(RTTY)比赛,10月最后一个周末为单边带语音通信(SSB)比赛,11月最后一个周末为电报(CW)比赛。每次比赛时间都是从星期六UTC0000开始,持续48小时,到星期日UTC2400结束。

比赛分为单操作员全波段(SOAB)、单操作员单波段(SOSB)、多操作员单发射机(MS)、多操作员多发射机(MM)等组别。按发射机功率分为高功率(HP,功率大于100W)、低功率(LP)、小功率(QRP,功率小于5W)。

CQVVW DX比赛的交换报告为信号报告加所在的CQ分区。比如我国上海的业余电台在CQ24区,其给出的报告就是“5924”(SSB)或“59924”(CW和RTTY)。

各种竞赛都有时间限制,比赛时往往要争分夺秒,用最简洁的语言完成通联。下面是PY2YU与BY4RWT在CQWW DX比赛(SSB)中的一段通话记录:

PY2YU:CQ contest,CQ contest,PAPA YANGKEE two YANGKEE UNIFORM contest.

BY4RWT:BRAVO YANGKEE four ROMEO WHISKEY TANGO.

PY2YU:BRAVO YANGKEE four ROMEO WHISKEY TANGO,Five nine one one.

BY4RWT:Five nine two four.

PY2YU:Thank you, 73!

CQWW DX比赛的记分方法是:基本分之和乘系数和。基本分的计算:每个波段中联络同一个DXCC实体内的电台为0分,联络一个本实体以外的本洲电台得1分,联络一个非本洲电台得3分。系数分的计算:每个波段中每增加一个DXCC表所列的新字头(呼号前缀)系数增加1分,每增加一个CQ分区系数增加1分。总分为各波段基本分之和与总系数分的乘积。

比赛的联络记录(LOG),SSB的应于当年12月1日前通过电子邮件发至ssb@cqww.com,CW的应于次年1月15日通过电子邮件发至cw@cqww.com,或于规定时间前将磁盘光碟寄往CQ Magazine,25 Newbribge Rd, Hicksville, NY11801, USA。

\section{IOTA竞赛}

1964年,英国短波收听爱好者Geoff Watts提出了“空中之岛”(Islands On The Air, IOTA)活动计划。1985年,英国无线电协会(Radio Society of Great Britain,RSGB)完善了这一计划,并成立了IOTA委员会。根据地理位置,IOTA委员会将全世界的海岛划分为1200个左右的岛组,赋予有业余电台活动的岛组以特定的编号,并制定了竞赛规则,设立了奖项。

IOTA是一项激动人心的DX通联活动。每年都有许多爱好者克服种种困难,来到分布于世界各地的岛屿上进行通信活动,IOTA远征(奖章见图6-4)为全世界业余无线电爱好者创造了十分难得的联络机会。

\begin{figure}[htbp]
	\centering
	\includegraphics[width=0.7\linewidth]{74}
	\caption{IOTA奖章}
	\label{fig:1}
\end{figure}

每年7月最后一个星期六UTC1200时至星期日UTC1200时,是世界IOTA竞赛时间。竞赛有3种参加对象:IOTA岛屿业余电台、全世界非岛业余电台和SWL收听台。对于IOTA岛屿台,每一次联络都有要求给出信号报告、从“001”开始递增的QSO序数和IOTA岛组编号;对于非岛台,则要求给出信号报告和QSO序数。
竞赛基本得分为非岛台与一个IOTA台联络得15分;IOTA岛台与非岛台及本岛组其他台联络得3分,与其他岛台联络得15分。

联络日志(LOG)使用电子文档,通过电子邮件或磁盘于当年9月1日之前递交。电子LOG以电子邮件的附件方式发送至iota.logs@rsgb.org。

IOTA竞赛的具体规则每年都可能会发生细微的变化,在参赛前应访问http://www.rsgbhfcc.org,以获得当年的规则文本。

\section{奖励证书}

全世界有关业余电台的奖状有数百种,这些由各国或国际业余无线电组织颁发的证书旨在鼓励爱好者在远距离通信试验中取得更好的成绩。当一个业余电台按照规则要求联络到一定数量的电台,并有对方寄来的QSL卡片作为凭证时,就可以向颁发该奖状的协会提出申请。

\subsection{CRSA0-9区奖状}

图6-5所示是由中国无线电运动协会向全世界颁发的奖状。任何国家和地区的业余电台,在任何波段、任何时间,使用任何操作方式,只要在中国大陆0~9区的每一个分区都联络到一个电台,即可向CRSA竞赛及奖状工作组提交申请。

\begin{figure}[htbp]
	\centering
	\includegraphics[width=0.7\linewidth]{75}
	\caption{CRSA0~9区奖状}
	\label{fig:1}
\end{figure}

\subsection{WAC奖状}

图6-6是由国际业余无线电联盟(IARU)颁发的奖状。爱好者只需和全世界六大洲(亚洲、欧洲、非洲、北美洲、南美洲、大洋洲)的业余电台进行联络并收到对方寄来的QSL卡片,即可向IARU总部提出申请。

\begin{figure}[htbp]
	\centering
	\includegraphics[width=0.7\linewidth]{76}
	\caption{WAC奖状}
	\label{fig:1}
\end{figure}

\subsection{DXCC奖状}

图6-7所示是由美国业余无线电协会(ARRL)颁发的在业余无线电界影响最大的证书之一。ARRL根据行政区域和地理地域的不同,把全世界各个国家和地区列为不同的“DX国家”,将这些“DX国家”的呼号前缀编制成了“DXCC表”。DXCC证书是以联络到100个“DXCC表”上列出的不同“国家”的业余电台为基本条件。

\begin{figure}[htbp]
	\centering
	\includegraphics[width=0.7\linewidth]{77}
	\caption{DXCC奖状}
	\label{fig:1}
\end{figure}

DXCC奖状有12种:基本奖、话、报、无线电传、1.8MHz、3.5MHz、7MHz、28MHz、50MHz、144MHz、卫星及五波段(5BDXCC)。其中5BDXCC必须在80m、40m、20m、15m和10m波段都得到100个不同实体的QSL卡片才可申请,它是一个非常难得的奖状,如图6-8所示。

\begin{figure}[htbp]
	\centering
	\includegraphics[width=0.7\linewidth]{78}
	\caption{五波段DXCC奖状}
	\label{fig:1}
\end{figure}

1990年8月,BY4RSA成为我国第一个获得DXCC奖状的集体电台。

\chapter{什么是业余无线电通信}

在科学技术迅速发展的今天,无线电通信已经深入到包括人们日常生活在内的各个领域。无论是天上的飞机、卫星,海上的轮船、舰艇,陆地上的各种车辆,还是人们熟悉的收音机、电视机、移动电话、Wi-Fi无线网络……全都离不开无线电通信技术。

业余无线电通信(以下有时简称“业余通信”)是整个无线电通信世界当中一个重要的组成部分。它是一项鼓励人们去从事无线电收信和发信实践的业余兴趣爱好活动。业余无线电通信的英语名字是“Amateur Radio”,符合国际电信联盟ITU定义的业余无线电爱好者是“Radio Amateur”,在世界上又普遍被称为“HAM”。由于“HAM”在英语中被解释为“火腿”,所以“火腿”又成了从事业余无线电通信的爱好者们的另一个名字。

业余无线电通信技术是一项内涵极其丰富的专门技术,所以人们还把获得发信执照、精通业余无线电技术和通信的爱好者称为“业余无线电家”,以区别于一般的电子技术爱好者。业余无线电通信的天地是博大的,当打开自己的收、发信机时,你可以听到来自世界各个角落的HAM的声音。当你获得业余无线电执照后,你可以轻松地和任何一个国家和地区的HAM交谈而无须办理出国护照,也可以从无数不见面的朋友那里得到技术上的支持。你会为自己第一次成功地和远方的朋友通信而兴高采烈,更可能会为自己在电子技术、通信技巧以及语言、人文地理等许多方面知识才能的迅速提高而大吃一惊!到那时,你才会更深切地体会到:业余无线电通信确实是一项遍及全世界的十分有意义的兴趣爱好活动。

\section{什么是业余电台}

联合国下设的专业机构“国际电信联盟”(ITU,International Telecommunication Union)根据不同的用途将全世界所有无线电通信分为若干种“业务”(Service),其中有两种业务用于业余无线电(Amateur Radio):“业余业务”(Amateur Service)和“卫星业余业务”(Amateur-Satellite Service)。ITU对业余业务的定义为“供业余无线电爱好者进行自我训练、相互通信和技术研究的无线电通信业务。业余无线电爱好者系指经正式批准的、对无线电技术有兴趣的人,其兴趣纯系个人爱好而不涉及谋取利润”。对卫星业余业务的定义是:“利用地球卫星上的空间电台开展的与业余业务相同目的的无线电通信业务。”用于业余业务的电台称业余电台(Amateur Radio Station)。业余电台是经过国家主管部门正式批准,业余无线电爱好者为了试验收发信设备、进行技术探讨、通信训练和比赛而设立的电台。

根据设台者的身份,业余电台可分为个人设置和团体(单位)设置两种。根据电台核准使用的频率和发射功率,我国又将业余电台分为A、B、C三类,以及特殊业余电台。只收听而不发射的电台被称为收听台,简称“SWL”(Short Wave Listener)。SWL虽然不发出信号,但它同样可以体会到HAM世界的美妙风光,帮助你和其他爱好者取得联系,而不用担心在稠密的住宅群中因为你的发信干扰了邻居的电视而招来不快。世界上有许多收听爱好者。

由团体(单位)申请设置的业余电台常被称为俱乐部电台(Club Station),我国曾于2013年前将这种电台定义为“集体业余电台”,并曾规定这类电台的呼号前缀(见本章1.3.3)为“BY”。这些“BY电台”多为学校、各类校外青少年教育机构、协会所设立,曾经为普及业余无线电知识、增进青少年爱好者对无线电科技爱好的兴趣发挥了积极作用。目前,仍有不少BY电台活跃着。现在,俱乐部电台和个人电台的呼号前缀已不做区分。本书附录3记录了部分BY电台的呼号,以便于了解这段历史和作为备查的资料。

个人业余电台是指爱好者本人申请设置并由其本人操作使用的电台。当今世界200多万个业余电台中,绝大多数是个人台。

在任何国家、任何地方,未经国家主管部门批准的无线电发信(包括试验发信)都是被严格禁止的。关于如何在我国申请设立和使用业余电台,请参阅本书第8章。

\section{}1.2 业余无线电的起源及在我国的发展历程
\subsection{}1.2.1 业余无线电通信的起源
\subsection{}1.2.2 中国业余无线电简史
\subsection{}1.2.3 我国业余无线电爱好者在突发事件中的几个真实故事
\section{}1.3 怎样寻找业余电台
\subsection{}1.3.1 电磁波以及波段的划分
\subsection{}1.3.2 业余电台的分区
\subsection{}1.3.3 业余电台的呼号
\subsection{}1.3.4 业余电台通信用的时间
\section{}1.4 业余电台的活动内容
\subsection{}1.4.1 多种多样的通信操作实践
\subsection{}1.4.2 各种数据通信研究
\subsection{}1.4.3 各种图像通信研究
\subsection{}1.4.4 业余无线电卫星通信
\subsection{}1.4.5 月面反射通信研究
\subsection{}1.4.6 移动通信研究
\subsection{}1.4.7 小功率通信研究
\subsection{}1.4.8 V/U波段通信
\subsection{}1.4.9 网络业余无线电
\subsection{}1.4.10 业余无线电测向

\chapter{业余无线电通信操作实践}

\section{}2.1 业余电台的通信内容
\section{}2.2 业余电台的信号报告
\section{}2.3 业余电台地理位置的报告
\section{}2.4 业余电台的QSL卡片
\subsection{}2.4.1 什么是QSL卡片
\subsection{}2.4.2 如何制作QSL卡片
\subsection{}2.4.3 如何填写QSL卡片
\subsection{}2.4.4 如何交换QSL卡片
\section{}2.5 业余电台的登记
\subsection{}2.5.1 电台日记
\subsection{}2.5.2 收听日记
\section{}2.6 业余无线电通信的语言
\subsection{}2.6.1 通信中的“字母解释法”
\subsection{}2.6.2 通信用Q简语
\subsection{}2.6.3 电码符号
\subsection{}2.6.4 无线电通信用的缩语
\subsection{}2.6.5 通信用英语
\section{}2.7 业余无线电通信基本程序
\subsection{}2.7.1 呼叫前的准备工作
\subsection{}2.7.2 普遍呼叫
\subsection{}2.7.3 区域性呼叫
\subsection{}2.7.4 回答程序
\subsection{}2.7.5 预约联络呼叫
\subsection{}2.7.6 未听清对方呼号时的询问呼叫
\subsection{}2.7.7 双方沟通后的联络程序
\subsection{}2.7.8 异频工作的呼叫方法
\subsection{}2.7.9 插入呼叫的方法
\section{}2.8 完整通信程序举例
\section{}2.9 网络通信
\section{}2.10 遇险通信和应急救援通信
\subsection{}2.10.1 遇险通信
\subsection{}2.10.2 应急救援通信

\chapter{收发报技术的自我训练}

\section{}3.1 正确地记忆电码符号
\subsection{}3.1.1 准确把握“点”“划”比例和“间隔”
\subsection{}3.1.2 怎样记忆电码符号
\section{}3.2 收报训练
\subsection{}3.2.1 收报的基本知识
\subsection{}3.2.2 收报的自我训练
\subsection{}3.2.3 巧用CW学习软件
\subsection{}3.2.4 适时进行机上抄收
\section{}3.3 发报练习
\subsection{}3.3.1 手键发报
\subsection{}3.3.2 自动键发报
\section{}3.4 严格自我要求,保证练习质量

\chapter{业余电台的奖励证书和竞赛活动}

\section{}4.1 业余电台的奖励证书
\subsection{}4.1.1 联络到中国Ø~9区奖状(Worked Chinese Ø~9 district)
\subsection{}4.1.2 WAC联络到世界各大洲奖状(Worked All Continents)
\subsection{}4.1.3 DXCC联络远距离电台俱乐部证书(DX Century Club)
\subsection{}4.1.4 WAS联络全美奖状(Worked All States)
\subsection{}4.1.5 WAZ联络全部CQ分区奖状(Worked All Zone)
\section{}4.2 业余电台的竞赛
\subsection{}4.2.1 业余电台竞赛的一般要求
\subsection{}4.2.2 主要的国际性竞赛介绍
\subsection{}4.2.3 国内的业余无线电比赛
\section{}4.3 IOTA“空中之岛”活动
\subsection{}4.3.1 IOTA岛屿编号
\section{}4.3.2 IOTA奖状
\section{}4.3.3 IOTA活动常用频率
\section{}4.3.4 IOTA远征
\section{}4.3.5 IOTA竞赛
\section{}4.4 FCC业余无线电执照资格考试

\chapter{怎样运用不同的业余波段}

\section{}5.1 无线电波的传播方式
\section{}5.2 电离层与天波传播
\section{}5.2.1 电离层概况
\section{}5.2.2 电离层对电波传播的影响
\section{}5.3 太阳黑子的影响
\section{}5.4 怎样利用几个不同的主要业余波段
\section{}5.4.1 160m波段(1.8~2.0MHz)
\section{}5.4.2 80m波段(3.5~3.9MHz)
\section{}5.4.3 40m波段(7.0~7.1MHz)
\section{}5.4.4 20m波段(14.0~14.35MHz)
\section{}5.4.5 15m波段(21.0~21.45MHz)
\section{}5.4.6 10m波段(28.0~29.7MHz)
\section{}5.4.7 6m波段(50~54MHz)
\section{}5.4.8 2m波段(144~148MHz)
\section{}5.4.9 70cm波段(430~440MHz)
\section{}5.5 业余波段上的信标(Beacons)

\chapter{业余短波天线}

\section{}6.1 天线
\section{}6.1.1 天线的主要特征
\section{}6.1.2 常用天线
\section{}6.1.3 天线的安全架设
\section{}6.2 传输线
\section{}6.2.1 传输线基础知识
\section{}6.2.2 传输线和天线间的匹配
\section{}6.2.3 平衡/不平衡转换
\section{}6.2.4 天线假负载
\section{}6.2.5 自制短波小环天线

\chapter{业余无线电收发信机}

\section{}7.1 短波收信机
\section{}7.1.1 业余无线电通信对收信机的要求
\section{}7.1.2 收信机介绍
\section{}7.1.3 收音机改装简易收信机实验
\section{}7.1.4 RTL-SDR软件无线电接收机入门应用
\section{}7.2 短波发信机
\section{}7.2.1 对发信机的要求
\section{}7.2.2 DIY CW QRP收发信机介绍
\section{}7.2.3 AX94 DIY单边带发信机介绍
\section{}7.3 超短波收发信机
\section{}7.3.1 FM调频通信
\section{}7.3.2 超短波数字化通信
\section{}7.4 成品业余无线电收发信机介绍
\section{}7.4.1 手持式对讲机
\section{}7.4.2 车载电台
\section{}7.4.3 中继台
\section{}7.4.4 短波电台
\section{}7.5 收发信设备中常见英文名字的意义
\section{}7.5.1 收信部分
\section{}7.5.2 发信部分
\section{}7.5.3 共用部分
\section{}7.6 自己动手制作辅助器材
\section{}7.6.1 功率计和驻波表
\section{}7.6.2 DIY电子电键

\chapter{依法设置和使用业余电台}

\section{}8.1 业余电台的分类管理及相应操作能力要求
\section{}8.2 个人设置业余电台的基本条件和申办程序
\section{}8.3 单位或团体设置业余电台的申办程序
\section{}8.4 特殊业余无线电台站
\section{}8.5 竞赛中的临时专用呼号
\section{}8.6 如何申办和使用业余无线电中继台
\section{}8.7 业余电台涉外交流活动方面的有关规定
\section{}8.7.1 有关外籍人员在华操作的规定
\section{}8.7.2 境外爱好者如何申请、办理《来访者业余无线电台临时操作证书》

\backmatter

\chapter{附录}

\section{}附录1 《中华人民共和国无线电管理条例》
\section{}附录2 卡片局各区分局负责人及各省联络站联系人
\section{}附录3之(1) 南极洲各科学考察站业余电台呼号前缀分布图
\section{}附录3之(2) 我国部分BY业余电台呼号
\section{}附录4 国内普通邮件及港澳台地区函件资费表(节选)
\section{}附录5 各类无线电通信业务通用的Q简语(节录)
\section{}附录6 无线电通信用缩语表(节录)
\section{}附录7之(1) CRSAØ~9区奖状式样
\section{}附录7之(2) CRSAØ~9区奖状申请表
\section{}附录8之(1) WAC基本奖状式样
\section{}附录8之(2) 1.8MHz WAC奖状式样
\section{}附录9之(1) DXCC基本证书式样
\section{}附录9之(2) 五波段DXCC证书式样
\section{}附录10之(1) WAS基本奖状式样
\section{}附录10之(2) 五波段WAS奖状式样
\section{}附录11 WAZ联络全部CQ分区奖状式样
\section{}附录12 计算通信方位角和大圆距离的BASIC程序
\section{}附录13 国际电信联盟《无线电规则》有关业余无线电部分的摘录
\section{}附录14之(1) 《业余无线电台管理办法》
\section{}附录14之(2) 《中华人民共和国无线电管制规定》
\section{}附录15之(1) 工业和信息化部文件
\section{}附录15之(2) 《关于进一步明确和规范业余无线电台管理有关工作的通知》
\section{}附录16之(1) 业余无线电台操作技术能力验证暂行办法
\section{}附录16之(2) 关于修订《各类别业余无线电台操作技术能力暂行验证考核标准》的通知
\section{}附录16之(3) 业余无线电中继台信息填报注意事项
\section{}附录17 CRAC业余频率使用及应急频点推荐规划
\section{}附录18之(1) 《内地业余无线电操作者逗留或到访香港特别行政区时申请业余电台牌照及操作授权证明的指引》
\section{}附录18之(2) 香港业余电台牌照的操作权限——操作频率及功率限制
\section{}附录18之(3) 《内地居民来港申请业余电台牌照/操作授权证明表格》
\section{}附录19之(1) 《来访者业余无线电台临时操作证书》申请办法
\section{}附录19之(2) 工业和信息化部关于香港特别行政区永久性居民在内地设置和使用业余无线电台有关事项的通告
\section{}附录20 A类业余电台操作证书考试内容提要
\section{}附录21 在轨业余卫星状态表
\section{}附录22 我国岛屿的IOTA编号表
\section{}附录23 业余无线电测向机的设计与制作


你是火腿吗:业余无线电通信/徐辉编著.—北京:人民邮电出版社,2012.4
(无线电科普丛书)
ISBN 978-7-115-24539-7

Ⅰ.①你… Ⅱ.①徐… Ⅲ.①无线电通信-普及读物 Ⅳ.①TN92-49

中国版本图书馆CIP数据核字(2010)第244730号

\end{document}