\documentclass{book}
\usepackage{ctex}
\usepackage{graphicx}
\usepackage{hyperref}
\usepackage{longtable}
\usepackage{booktabs}
\graphicspath{{./temp_extract/OEBPS/}}
\begin{document}
\title{跟老男孩学Linux运维:MySQL入门与提高实践}
\author{老男孩}
\maketitle
\tableofcontents


% From text00000.html


\par Linux/Unix技术丛书

\par 跟老男孩学Linux运维:MySQL入门与提高实践

\par 老男孩 著

\par ISBN:978-7-111-61367-1

\par 本书纸版由机械工业出版社于2019年出版,电子版由华章分社(北京华章图文信息有限公司,北京奥维博世图书发行有限公司)全球范围内制作与发行。

\par 版权所有,侵权必究

\par 客服热线:+ 86-10-68995265

\par 客服信箱:service@bbbvip.com

\par 官方网址:www.hzmedia.com.cn

\par 新浪微博 @华章数媒

\par 微信公众号 华章电子书(微信号:hzebook)


% From text00001.html


\chapter*{目录}
目录

\section*{
\href{text00002.html}{前言}
}


\section*{
\href{text00003.html#a1}{第1章 数据库介绍与分类}
}


\subsection*{
\href{text00003.html#a1}{1.1 数据库介绍}
}


\subsection*{
\href{text00004.html}{1.2 数据库的种类}
}


\subsubsection*{
\href{text00005.html}{1.2.1 关系型数据库介绍}
}


\subsubsection*{
\href{text00006.html}{1.2.2 非关系型数据库介绍}
}


\subsection*{
\href{text00007.html#b1}{1.3 常用关系型数据库产品介绍}
}


\subsubsection*{
\href{text00007.html#b1}{1.3.1 Oracle数据库}
}


\subsubsection*{
\href{text00008.html}{1.3.2 MySQL数据库}
}


\subsubsection*{
\href{text00009.html}{1.3.3 MariaDB数据库}
}


\subsubsection*{
\href{text00010.html}{1.3.4 SQL Server数据库}
}


\subsubsection*{
\href{text00011.html}{1.3.5 Access数据库}
}


\subsubsection*{
\href{text00012.html}{1.3.6 PostgreSQL数据库}
}


\subsubsection*{
\href{text00014.html#b1}{1.3.7 其他不常用的关系型数据库}
}


\subsection*{
\href{text00014.html#b1}{1.4 常用非关系型数据库产品介绍}
}


\subsubsection*{
\href{text00014.html#b1}{1.4.1 Memcached(key-value)}
}


\subsubsection*{
\href{text00015.html}{1.4.2 redis(key-value)}
}


\subsubsection*{
\href{text00016.html}{1.4.3 MongoDB(document-oriented)}
}


\subsubsection*{
\href{text00017.html}{1.4.4 Cassandra(column-oriented)}
}


\subsubsection*{
\href{text00019.html#b1}{1.4.5 其他非关系型数据库}
}


\subsection*{
\href{text00019.html#b1}{1.5 数据库相关知识}
}


\subsubsection*{
\href{text00019.html#b1}{1.5.1 数据库发展历史大事记}
}


\subsubsection*{
\href{text00020.html}{1.5.2 数据库软件企业应用排名及发展趋势参考}
}


\subsection*{
\href{text00021.html}{1.6 本章重点}
}


\subsection*{
\href{text00022.html}{1.7 章节试题}
}


\section*{
\href{text00023.html#a1}{第2章 MySQL数据库入门知识介绍}
}


\subsection*{
\href{text00023.html#a1}{2.1 MySQL介绍}
}


\subsubsection*{
\href{text00023.html#b1}{2.1.1 MySQL简介}
}


\subsubsection*{
\href{text00024.html}{2.1.2 MariaDB数据库的诞生背景介绍}
}


\subsubsection*{
\href{text00025.html}{2.1.3 为什么选择MySQL数据库}
}


\subsection*{
\href{text00026.html}{2.2 MySQL数据库分类与版本升级}
}


\subsubsection*{
\href{text00027.html}{2.2.1 MySQL数据库企业版与社区版的区别}
}


\subsubsection*{
\href{text00028.html}{2.2.2 MySQL数据库的四种发布版本介绍}
}


\subsection*{
\href{text00029.html}{2.3 MySQL数据库软件的命名介绍}
}


\subsection*{
\href{text00030.html#b1}{2.4 MySQL产品路线}
}


\subsubsection*{
\href{text00030.html#b1}{2.4.1 MySQL产品路线变更历史背景}
}


\subsubsection*{
\href{text00031.html}{2.4.2 MySQL-5.0.xx到MySQL-5.1.xx的产品线}
}


\subsubsection*{
\href{text00032.html}{2.4.3 MySQL-5.4.xx到MySQL-5.7.xx产品线}
}


\subsubsection*{
\href{text00033.html}{2.4.4 MySQL-Cluster-6.0.xx到MySQL-Cluster-7.5.xx产品线}
}


\subsection*{
\href{text00034.html#b1}{2.5 生产场景中如何选择MySQL版本}
}


\subsubsection*{
\href{text00034.html#b1}{2.5.1 MySQL数据库发布特性}
}


\subsubsection*{
\href{text00035.html}{2.5.2 企业生产场景选择MySQL数据库的建议}
}


\subsection*{
\href{text00037.html#a1}{2.6 章节试题}
}


\section*{
\href{text00037.html#a1}{第3章 MySQL数据库安装方法及安装实践}
}


\subsection*{
\href{text00037.html#a1}{3.1 MySQL数据库的安装方法及选择}
}


\subsubsection*{
\href{text00038.html}{3.1.1 yum/rpm方式安装MySQL}
}


\subsubsection*{
\href{text00039.html}{3.1.2 采用常规方式编译安装MySQL}
}


\subsubsection*{
\href{text00040.html}{3.1.3 采用cmake方式编译安装MySQL}
}


\subsubsection*{
\href{text00041.html}{3.1.4 采用二进制方式免编译安装MySQL}
}


\subsubsection*{
\href{text00042.html}{3.1.5 如何正确选择MySQL的安装方式}
}


\subsection*{
\href{text00043.html#b1}{3.2 安装并配置MySQL数据库}
}


\subsubsection*{
\href{text00043.html#b1}{3.2.1 安装MySQL数据库}
}


\subsubsection*{
\href{text00044.html}{3.2.2 创建MySQL数据库配置文件并对数据库目录授权}
}


\subsubsection*{
\href{text00045.html}{3.2.3 初始化MySQL数据库文件}
}


\subsubsection*{
\href{text00046.html}{3.2.4 配置并启动MySQL数据库}
}


\subsubsection*{
\href{text00047.html}{3.2.5 将MySQL相关命令加入全局路径}
}


\subsubsection*{
\href{text00048.html}{3.2.6 登录MySQL测试}
}


\subsubsection*{
\href{text00049.html}{3.2.7 基本的MySQL安全配置}
}


\subsection*{
\href{text00050.html}{3.3 MySQL安装FAQ}
}


\subsection*{
\href{text00051.html}{3.4 MySQL 5.6编译常见参数选项说明}
}


\subsection*{
\href{text00052.html}{3.5 章节试题}
}


\section*{
\href{text00053.html#a1}{第4章 MySQL多实例数据库企业级应用实践}
}


\subsection*{
\href{text00053.html#a1}{4.1 MySQL多实例介绍}
}


\subsubsection*{
\href{text00054.html}{4.1.1 什么是MySQL多实例}
}


\subsubsection*{
\href{text00055.html}{4.1.2 MySQL多实例的作用与问题}
}


\subsection*{
\href{text00056.html#b1}{4.2 MySQL多实例的生产应用场景}
}


\subsubsection*{
\href{text00056.html#b1}{4.2.1 资金紧张型公司的选择}
}


\subsubsection*{
\href{text00057.html}{4.2.2 并发访问不是特别大的业务}
}


\subsubsection*{
\href{text00058.html}{4.2.3 门户网站应用MySQL多实例场景}
}


\subsection*{
\href{text00059.html#b1}{4.3 MySQL多实例常见的配置方案}
}


\subsubsection*{
\href{text00059.html#b1}{4.3.1 单一配置文件、单一启动程序多实例部署方案}
}


\subsubsection*{
\href{text00060.html}{4.3.2 多配置文件、多启动程序部署方案}
}


\subsection*{
\href{text00061.html#b1}{4.4 安装并配置多实例MySQL数据库}
}


\subsubsection*{
\href{text00061.html#b1}{4.4.1 安装MySQL多实例}
}


\subsubsection*{
\href{text00062.html}{4.4.2 创建MySQL多实例的数据文件目录}
}


\subsubsection*{
\href{text00063.html}{4.4.3 创建MySQL多实例的配置文件}
}


\subsubsection*{
\href{text00064.html}{4.4.4 创建MySQL多实例的启动文件}
}


\subsubsection*{
\href{text00065.html}{4.4.5 配置MySQL多实例的文件权限}
}


\subsubsection*{
\href{text00066.html}{4.4.6 MySQL相关命令加入全局路径的配置}
}


\subsubsection*{
\href{text00067.html}{4.4.7 初始化MySQL多实例的数据库文件}
}


\subsubsection*{
\href{text00068.html}{4.4.8 启动MySQL多实例数据库}
}


\subsubsection*{
\href{text00069.html}{4.4.9 MySQL多实例数据库启动故障排错说明}
}


\subsection*{
\href{text00070.html}{4.5 配置及管理MySQL多实例数据库}
}


\subsection*{
\href{text00071.html}{4.6 参考资料}
}


\subsection*{
\href{text00072.html}{4.7 章节试题}
}


\section*{
\href{text00073.html#a1}{第5章 MySQL常用管理基础知识实践}
}


\subsection*{
\href{text00073.html#a1}{5.1 启动与关闭MySQL}
}


\subsubsection*{
\href{text00074.html}{5.1.1 单实例MySQL启动与关闭知识}
}


\subsubsection*{
\href{text00075.html}{5.1.2 多实例MySQL启动与关闭方法示例}
}


\subsection*{
\href{text00076.html#b1}{5.2 MySQL连接原理方法及提示符设置}
}


\subsubsection*{
\href{text00076.html#b1}{5.2.1 客户端连接MySQL服务器原理结构}
}


\subsubsection*{
\href{text00077.html}{5.2.2 默认单实例MySQL登录方法}
}


\subsubsection*{
\href{text00078.html}{5.2.3 默认多实例MySQL登录方法}
}


\subsubsection*{
\href{text00079.html}{5.2.4 异地远程登录MySQL方法}
}


\subsubsection*{
\href{text00080.html}{5.2.5 MySQL连接提示符说明}
}


\subsubsection*{
\href{text00081.html}{5.2.6 退出MySQL数据库}
}


\subsection*{
\href{text00082.html}{5.3 查看MySQL命令帮助}
}


\subsection*{
\href{text00083.html#b1}{5.4 设置及修改mysql root用户密码}
}


\subsubsection*{
\href{text00083.html#b1}{5.4.1 MySQL数据库用户安全策略介绍}
}


\subsubsection*{
\href{text00084.html}{5.4.2 为管理员root用户设置及修改密码}
}


\subsection*{
\href{text00086.html}{5.5 找回MySQL root用户密码}
}


\subsubsection*{
\href{text00086.html}{5.5.1 找回MySQL单实例root用户密码的方法}
}


\subsubsection*{
\href{text00087.html}{5.5.2 找回MySQL多实例root用户的密码方法}
}


\subsection*{
\href{text00088.html}{5.6 章节试题}
}


\section*{
\href{text00089.html#a1}{第6章 MySQL常用管理SQL语句应用实践}
}


\subsection*{
\href{text00089.html#a1}{6.1 SQL介绍}
}


\subsubsection*{
\href{text00089.html#b1}{6.1.1 什么是SQL}
}


\subsubsection*{
\href{text00090.html}{6.1.2 SQL的分类}
}


\subsection*{
\href{text00091.html#b1}{6.2 SQL解析原理流程}
}


\subsubsection*{
\href{text00091.html#b1}{6.2.1 MySQL体系结构简介}
}


\subsubsection*{
\href{text00092.html}{6.2.2 SQL解析流程介绍}
}


\subsection*{
\href{text00093.html}{6.3 SQL语句实践}
}


\subsubsection*{
\href{text00094.html}{6.3.1 DDL语句之管理数据库}
}


\subsubsection*{
\href{text00095.html}{6.3.2 DDL&&DCL语句之管理用户}
}


\subsubsection*{
\href{text00096.html}{6.3.3 DDL语句之管理表}
}


\subsubsection*{
\href{text00097.html}{6.3.4 DML语句之管理表中的数据}
}


\subsection*{
\href{text00098.html}{6.4 参考资料}
}


\subsection*{
\href{text00099.html}{6.5 章节试题}
}


\section*{
\href{text00100.html#a1}{第7章 MySQL数据库备份与恢复基础实践}
}


\subsection*{
\href{text00100.html#a1}{7.1 MySQL数据库的备份与恢复}
}


\subsubsection*{
\href{text00100.html#b1}{7.1.1 备份数据的意义}
}


\subsubsection*{
\href{text00101.html}{7.1.2 使用mysqldump进行数据库备份实践}
}


\subsubsection*{
\href{text00102.html}{7.1.3 mysqldump重要关键参数说明}
}


\subsubsection*{
\href{text00103.html}{7.1.4 生产场景下,不同引擎的mysqldump备份命令}
}


\subsubsection*{
\href{text00104.html}{7.1.5 利用SQL语句方式对表进行导入导出}
}


\subsection*{
\href{text00105.html#b1}{7.2 恢复数据库实践}
}


\subsubsection*{
\href{text00105.html#b1}{7.2.1 数据库恢复基本事项}
}


\subsubsection*{
\href{text00106.html}{7.2.2 利用source命令恢复数据库}
}


\subsubsection*{
\href{text00107.html}{7.2.3 利用mysql命令恢复(标准)}
}


\subsubsection*{
\href{text00108.html}{7.2.4 利用mysql-e参数查看mysql数据}
}


\subsection*{
\href{text00110.html}{7.3 mysqlbinlog增量恢复工具}
}


\subsubsection*{
\href{text00110.html}{7.3.1 mysql的binlog日志是什么}
}


\subsubsection*{
\href{text00111.html}{7.3.2 mysql的binlog日志的作用}
}


\subsubsection*{
\href{text00112.html}{7.3.3 mysql的binlog日志功能如何开启}
}


\subsubsection*{
\href{text00113.html}{7.3.4 mysqlbinlog工具解析binlog日志实践}
}


\subsubsection*{
\href{text00115.html}{7.3.5 mysqlbinlog命令常用参数}
}


\subsection*{
\href{text00115.html}{7.4 本章重点}
}


\section*{
\href{text00116.html#a1}{第8章 MySQL企业级备份应用知识与实践}
}


\subsection*{
\href{text00116.html#a1}{8.1 数据库备份的最高层次思想}
}


\subsection*{
\href{text00117.html}{8.2 数据库管理员的两大工作核心}
}


\subsection*{
\href{text00118.html#b1}{8.3 全量备份与增量备份}
}


\subsubsection*{
\href{text00118.html#b1}{8.3.1 全量备份的概念}
}


\subsubsection*{
\href{text00119.html}{8.3.2 增量备份的概念}
}


\subsubsection*{
\href{text00120.html}{8.3.3 全量与增量如何结合备份}
}


\subsection*{
\href{text00122.html}{8.4 MySQL常用的备份方式}
}


\subsubsection*{
\href{text00122.html}{8.4.1 逻辑备份方式}
}


\subsubsection*{
\href{text00123.html}{8.4.2 物理备份方式}
}


\subsubsection*{
\href{text00124.html}{8.4.3 物理备份与逻辑备份的区别}
}


\subsection*{
\href{text00125.html#b1}{8.5 逻辑备份的企业级应用实战}
}


\subsubsection*{
\href{text00125.html#b1}{8.5.1 中小企业的MySQL备份实战}
}


\subsubsection*{
\href{text00126.html}{8.5.2 中小企业MySQL增量恢复案例实战}
}


\subsection*{
\href{text00127.html#b1}{8.6 分库分表的生产备份策略}
}


\subsubsection*{
\href{text00127.html#b1}{8.6.1 为什么要分库分表备份}
}


\subsubsection*{
\href{text00128.html}{8.6.2 如何进行分库备份}
}


\subsubsection*{
\href{text00129.html}{8.6.3 如何进行分表备份}
}


\subsection*{
\href{text00130.html}{8.7 MySQL生产常用备份架构方案}
}


\subsection*{
\href{text00131.html}{8.8 本章重点}
}


\section*{
\href{text00132.html#a1}{第9章 MySQL物理备份工具Xtrabackup应用实践}
}


\subsection*{
\href{text00132.html#a1}{9.1 Xtrabackup介绍}
}


\subsection*{
\href{text00133.html}{9.2 Xtrabackup备份涉及的数据库名词}
}


\subsection*{
\href{text00134.html}{9.3 Xtrabackup备份的工作原理流程}
}


\subsection*{
\href{text00135.html#b1}{9.4 Xtrabackup工具安装}
}


\subsubsection*{
\href{text00135.html#b1}{9.4.1 系统环境说明}
}


\subsubsection*{
\href{text00136.html}{9.4.2 安装Xtrabackup}
}


\subsection*{
\href{text00137.html#b1}{9.5 Xtrabackup应用实践}
}


\subsubsection*{
\href{text00137.html#b1}{9.5.1 用于Xtrabackup数据备份的用户}
}


\subsubsection*{
\href{text00138.html}{9.5.2 用于恢复的MySQL配置文件}
}


\subsubsection*{
\href{text00139.html}{9.5.3 Xtrabackup软件附带的备份工具说明}
}


\subsubsection*{
\href{text00140.html}{9.5.4 Innobackupex工具语法介绍}
}


\subsubsection*{
\href{text00141.html}{9.5.5 全备与恢复全备实践}
}


\subsubsection*{
\href{text00142.html}{9.5.6 增量备份与恢复增量数据实践}
}


\subsubsection*{
\href{text00143.html}{9.5.7 中小企业MySQL Xtrabackup物理增量恢复案例实战}
}


\subsubsection*{
\href{text00144.html}{9.5.8 使用Xtrabackup物理分库分表备份}
}


\subsubsection*{
\href{text00145.html}{9.5.9 使用Xtrabackup物理分库分表备份的恢复}
}


\section*{
\href{text00146.html#a1}{第10章 MySQL数据库日志知识与企业应用实践}
}


\subsection*{
\href{text00146.html#a1}{10.1 MySQL常用日志文件知识}
}


\subsection*{
\href{text00147.html}{10.2 错误日志的介绍与配置}
}


\subsection*{
\href{text00148.html}{10.3 普通查询日志的介绍与配置}
}


\subsection*{
\href{text00149.html}{10.4 二进制日志的介绍与配置}
}


\subsection*{
\href{text00150.html}{10.5 慢查询日志}
}


\subsection*{
\href{text00151.html}{10.6 本章重点}
}


\subsection*{
\href{text00153.html#a1}{10.7 参考资料}
}


\section*{
\href{text00153.html#a1}{第11章 MySQL数据库字符集}
}


\subsection*{
\href{text00153.html#a1}{11.1 MySQL数据库字符集知识}
}


\subsubsection*{
\href{text00153.html#b1}{11.1.1 什么是字符集}
}


\subsubsection*{
\href{text00154.html}{11.1.2 MySQL数据库字符集}
}


\subsubsection*{
\href{text00155.html}{11.1.3 常用字符集介绍与选择建议}
}


\subsection*{
\href{text00156.html}{11.2 MySQL数据库字符集配置}
}


\subsection*{
\href{text00157.html}{11.3 如何防止数据库的中文显示乱码}
}


\subsection*{
\href{text00158.html#b1}{11.4 如何更改MySQL数据库库表的字符集}
}


\subsubsection*{
\href{text00158.html#b1}{11.4.1 更改库的字符集}
}


\subsubsection*{
\href{text00159.html}{11.4.2 更改表的字符集}
}


\subsubsection*{
\href{text00160.html}{11.4.3 生产环境更改数据库(含数据)字符集的方法}
}


\subsection*{
\href{text00161.html}{11.5 本章重点}
}


\section*{
\href{text00162.html#a1}{第12章 MySQL数据库存储引擎知识}
}


\subsection*{
\href{text00162.html#a1}{12.1 MySQL引擎概述}
}


\subsubsection*{
\href{text00162.html#b1}{12.1.1 什么是存储引擎?}
}


\subsubsection*{
\href{text00163.html}{12.1.2 MySQL存储引擎的架构}
}


\subsection*{
\href{text00164.html}{12.2 查看MySQL支持的存储引擎}
}


\subsection*{
\href{text00165.html}{12.3 MySQL 5.6支持的存储引擎}
}


\subsection*{
\href{text00166.html}{12.4 MySQL常用存储引擎特性对比}
}


\subsection*{
\href{text00167.html}{12.5 设置与更改MySQL的引擎}
}


\subsection*{
\href{text00168.html#b1}{12.6 MyISAM引擎}
}


\subsubsection*{
\href{text00168.html#b1}{12.6.1 什么是MyISAM引擎?}
}


\subsubsection*{
\href{text00169.html}{12.6.2 MyISAM引擎的存储方式}
}


\subsubsection*{
\href{text00170.html}{12.6.3 MyISAM引擎的主要特点}
}


\subsubsection*{
\href{text00171.html}{12.6.4 MyISAM引擎适用的生产业务场景}
}


\subsection*{
\href{text00172.html#b1}{12.7 InnoDB引擎}
}


\subsubsection*{
\href{text00172.html#b1}{12.7.1 什么是InnoDB引擎?}
}


\subsubsection*{
\href{text00173.html}{12.7.2 InnoDB引擎的存储方式}
}


\subsubsection*{
\href{text00174.html}{12.7.3 InnoDB引擎特点}
}


\subsubsection*{
\href{text00175.html}{12.7.4 InnoDB引擎适用的生产业务场景}
}


\subsubsection*{
\href{text00176.html}{12.7.5 InnoDB引擎相关参数介绍}
}


\subsubsection*{
\href{text00177.html}{12.7.6 InnoDB引擎调优的基本方法}
}


\subsection*{
\href{text00178.html}{12.8 Memory存储引擎}
}


\subsection*{
\href{text00179.html}{12.9 ARCHIVE存储引擎}
}


\subsection*{
\href{text00180.html}{12.10 NDB存储引擎}
}


\subsection*{
\href{text00181.html}{12.11 有关MySQL引擎常见的企业面试题}
}


\section*{
\href{text00182.html#a1}{第13章 MySQL引擎之InnoDB}
}


\subsection*{
\href{text00182.html#a1}{13.1 InnoDB存储引擎介绍}
}


\subsection*{
\href{text00183.html}{13.2 InnoDB和ACID模型}
}


\subsection*{
\href{text00184.html}{13.3 InnoDB多版本控制MVCC}
}


\subsection*{
\href{text00185.html}{13.4 InnoDB体系结构}
}


\subsubsection*{
\href{text00186.html}{13.4.1 缓存池(buffer pool)}
}


\subsubsection*{
\href{text00187.html}{13.4.2 change pool缓存池}
}


\subsubsection*{
\href{text00188.html}{13.4.3 自适应哈希索引(AHI)}
}


\subsubsection*{
\href{text00189.html}{13.4.4 doublewrite缓存}
}


\subsubsection*{
\href{text00190.html}{13.4.5 重做日志缓存(redo log buffer)}
}


\subsubsection*{
\href{text00191.html}{13.4.6 重做日志(redo log)}
}


\subsubsection*{
\href{text00192.html}{13.4.7 系统(共享)表空间}
}


\subsubsection*{
\href{text00193.html}{13.4.8 File-per-table独立表空间设置}
}


\subsubsection*{
\href{text00194.html}{13.4.9 undo日志}
}


\subsubsection*{
\href{text00195.html}{13.4.10 临时表空间}
}


\subsubsection*{
\href{text00196.html}{13.4.11 InnoDB后台线程}
}


\subsection*{
\href{text00197.html#b1}{13.5 InnoDB其他相关配置}
}


\subsubsection*{
\href{text00197.html#b1}{13.5.1 启动配置}
}


\subsubsection*{
\href{text00198.html}{13.5.2 指定配置文件位置}
}


\subsubsection*{
\href{text00200.html}{13.5.3 数据页配置}
}


\subsubsection*{
\href{text00200.html}{13.5.4 InnoDB只读设置}
}


\subsubsection*{
\href{text00201.html}{13.5.5 InnoDB优化器统计信息配置}
}


\subsubsection*{
\href{text00202.html}{13.5.6 索引页之间合并阈值}
}


\subsection*{
\href{text00203.html}{13.6 InnoDB普通表空间}
}


\subsection*{
\href{text00204.html#b1}{13.7 InnoDB表}
}


\subsubsection*{
\href{text00204.html#b1}{13.7.1 InnoDB表存储结构}
}


\subsubsection*{
\href{text00205.html}{13.7.2 创建InnoDB表}
}


\subsubsection*{
\href{text00207.html}{13.7.3 修改表的存储引擎}
}


\subsubsection*{
\href{text00207.html}{13.7.4 自增长字段设置}
}


\subsubsection*{
\href{text00208.html}{13.7.5 InnoDB表主要的限制}
}


\section*{
\href{text00209.html#a1}{第14章 MySQL主从复制知识与应用实践}
}


\subsection*{
\href{text00209.html#a1}{14.1 MySQL主从复制}
}


\subsubsection*{
\href{text00210.html}{14.1.1 MySQL主从复制介绍}
}


\subsubsection*{
\href{text00211.html}{14.1.2 MySQL主从复制企业级应用场景}
}


\subsubsection*{
\href{text00212.html}{14.1.3 MySQL主从读写分离实现方案}
}


\subsubsection*{
\href{text00213.html}{14.1.4 MySQL主从复制原理}
}


\subsubsection*{
\href{text00214.html}{14.1.5 MySQL主从复制原理及过程详细描述}
}


\subsection*{
\href{text00215.html#b1}{14.2 MySQL主从复制实践}
}


\subsubsection*{
\href{text00215.html#b1}{14.2.1 主从复制实践准备}
}


\subsubsection*{
\href{text00216.html}{14.2.2 在主库Master(51)上执行操作配置}
}


\subsubsection*{
\href{text00217.html}{14.2.3 在MySQL从库上执行的操作过程}
}


\subsubsection*{
\href{text00218.html}{14.2.4 启动从库同步开关并测试主从复制}
}


\subsubsection*{
\href{text00219.html}{14.2.5 MySQL主从复制问题汇总}
}


\subsubsection*{
\href{text00220.html}{14.2.6 MySQL主从复制配置步骤小结}
}


\subsubsection*{
\href{text00221.html}{14.2.7 MySQL主从复制线程状态说明及用途}
}


\subsubsection*{
\href{text00222.html}{14.2.8 生产场景中部署MySQL主从复制方案}
}


\subsection*{
\href{text00223.html}{14.3 MySQL主从复制在企业中的故障案例}
}


\subsection*{
\href{text00224.html}{14.4 本章重点}
}


\subsection*{
\href{text00225.html}{14.5 参考资料}
}


\section*{
\href{text00226.html#a1}{第15章 MySQL主从复制高级方案与应用实践}
}


\subsection*{
\href{text00226.html#a1}{15.1 MySQL集群企业级架构方案}
}


\subsection*{
\href{text00227.html}{15.2 MySQL企业级备份策略方案}
}


\subsection*{
\href{text00228.html}{15.3 MySQL主从复制生产场景的常见延迟原因及防范方案}
}


\subsection*{
\href{text00229.html}{15.4 MySQL主从复制数据一致性企业级方案}
}


\subsection*{
\href{text00230.html}{15.5 MySQL多线程复制解决复制延迟实践}
}


\subsection*{
\href{text00231.html}{15.6 让MySQL主从复制的从库只读访问}
}


\subsection*{
\href{text00232.html}{15.7 MySQL主从复制读写分离Web用户生产设置方案}
}


\subsection*{
\href{text00233.html}{15.8 MySQL主从延迟复制方案及恢复实践}
}


\subsection*{
\href{text00234.html}{15.9 本章重点}
}


\subsection*{
\href{text00235.html}{15.10 参考资料}
}


\section*{
\href{text00236.html#a1}{第16章 MySQL复制高级方案应用实践}
}


\subsection*{
\href{text00236.html#a1}{16.1 MySQL级联复制}
}


\subsubsection*{
\href{text00237.html}{16.1.1 MySQL级联复制介绍}
}


\subsubsection*{
\href{text00238.html}{16.1.2 MySQL级联复制实现要点}
}


\subsubsection*{
\href{text00239.html}{16.1.3 MySQL级联复制的应用场景}
}


\subsection*{
\href{text00240.html}{16.2 MySQL主主复制}
}


\subsubsection*{
\href{text00242.html}{16.2.1 MySQL主主复制介绍}
}


\subsubsection*{
\href{text00242.html}{16.2.2 MySQL主主复制能够解决的企业问题}
}


\subsubsection*{
\href{text00243.html}{16.2.3 MySQL主主复制的企业级实现方案}
}


\subsubsection*{
\href{text00244.html}{16.2.4 主主复制实践(自增ID)准备}
}


\subsubsection*{
\href{text00245.html}{16.2.5 在主库Master(51)上执行操作配置}
}


\subsubsection*{
\href{text00246.html}{16.2.6 在主库2Master(52)上执行操作配置}
}


\subsubsection*{
\href{text00247.html}{16.2.7 在主库1(51)上执行复制配置}
}


\subsubsection*{
\href{text00248.html}{16.2.8 在主库1和主库2进行测试}
}


\subsection*{
\href{text00249.html}{16.3 本章重点}
}


\subsection*{
\href{text00251.html#a1}{16.4 MySQL双主复制my.cnf完整配置对比}
}


\section*{
\href{text00251.html#a1}{第17章 MySQL半同步复制与GTID复制实践}
}


\subsection*{
\href{text00251.html#a1}{17.1 MySQL复制的多种工作方式}
}


\subsubsection*{
\href{text00252.html}{17.1.1 异步复制介绍}
}


\subsubsection*{
\href{text00253.html}{17.1.2 全同步复制介绍}
}


\subsubsection*{
\href{text00254.html}{17.1.3 半同步复制}
}


\subsection*{
\href{text00255.html#b1}{17.2 MySQL半同步复制原理及实践准备}
}


\subsubsection*{
\href{text00255.html#b1}{17.2.1 MySQL半同步复制介绍}
}


\subsubsection*{
\href{text00256.html}{17.2.2 MySQL半同步复制原理}
}


\subsubsection*{
\href{text00257.html}{17.2.3 MySQL半同步复制准备}
}


\subsection*{
\href{text00258.html#b1}{17.3 MySQL半同步复制应用实践}
}


\subsubsection*{
\href{text00258.html#b1}{17.3.1 MySQL半同步复制插件介绍}
}


\subsubsection*{
\href{text00259.html}{17.3.2 MySQL主库(db01)半同步插件安装和配置}
}


\subsubsection*{
\href{text00260.html}{17.3.3 MySQL半同步复制参数介绍}
}


\subsubsection*{
\href{text00261.html}{17.3.4 MySQL从库(db02)半同步插件安装和配置}
}


\subsubsection*{
\href{text00262.html}{17.3.5 实践1:半同步复制是否配置成功测试}
}


\subsubsection*{
\href{text00263.html}{17.3.6 实践2:半同步复制超时等待测试}
}


\subsubsection*{
\href{text00264.html}{17.3.7 实践3:主从复制故障时的半同步复制测试}
}


\subsection*{
\href{text00265.html}{17.4 生产半同步复制建议及其他方案说明}
}


\subsection*{
\href{text00266.html#b1}{17.5 MySQL GTID复制}
}


\subsubsection*{
\href{text00266.html#b1}{17.5.1 GTID复制简介}
}


\subsubsection*{
\href{text00267.html}{17.5.2 基于GTID复制技术的优缺点及工作原理}
}


\subsubsection*{
\href{text00268.html}{17.5.3 GTID的优缺点}
}


\subsubsection*{
\href{text00269.html}{17.5.4 MySQL GTID复制的应用及实践}
}


\subsubsection*{
\href{text00270.html}{17.5.5 GTID如何跳过事务冲突}
}


\subsection*{
\href{text00271.html}{17.6 本章重点}
}


\section*{
\href{text00272.html#a1}{第18章 MySQL集群高可用方案MHA应用实践}
}


\subsection*{
\href{text00272.html#a1}{18.1 什么是MHA}
}


\subsection*{
\href{text00273.html}{18.2 MHA的基本架构组成}
}


\subsection*{
\href{text00274.html}{18.3 MHA的工作原理}
}


\subsection*{
\href{text00275.html}{18.4 MHA工具包介绍}
}


\subsection*{
\href{text00276.html}{18.5 MHA解决方案的优点}
}


\subsection*{
\href{text00277.html#b1}{18.6 MHA方案实战}
}


\subsubsection*{
\href{text00277.html#b1}{18.6.1 搭建MHA的先决必要条件}
}


\subsubsection*{
\href{text00278.html}{18.6.2 MySQL节点规划}
}


\subsubsection*{
\href{text00279.html}{18.6.3 配置SSH密钥实现免密码登录}
}


\subsubsection*{
\href{text00280.html}{18.6.4 对所有的MySQL节点安装MHA Node软件}
}


\subsubsection*{
\href{text00281.html}{18.6.5 MHA管理节点安装}
}


\subsubsection*{
\href{text00282.html}{18.6.6 配置MHA管理节点}
}


\subsection*{
\href{text00283.html#b1}{18.7 启动及测试MHA}
}


\subsubsection*{
\href{text00283.html#b1}{18.7.1 启动MHA前需要检测的要素说明}
}


\subsubsection*{
\href{text00284.html}{18.7.2 检测SSH免密码登录配置}
}


\subsubsection*{
\href{text00285.html}{18.7.3 检测MySQL集群主从复制状况}
}


\subsection*{
\href{text00286.html#b1}{18.8 配置VIP漂移}
}


\subsubsection*{
\href{text00286.html#b1}{18.8.1 虚拟IP管理的两种方式}
}


\subsubsection*{
\href{text00287.html}{18.8.2 配置脚本}
}


\section*{
\href{text00288.html#a1}{第19章 MySQL读写分离Atlas工具实践}
}


\subsection*{
\href{text00288.html#a1}{19.1 什么是Atlas}
}


\subsection*{
\href{text00289.html}{19.2 Atlas的主要功能}
}


\subsection*{
\href{text00290.html}{19.3 Atlas与官方mysql-proxy的对比}
}


\subsection*{
\href{text00291.html}{19.4 安装Atlas}
}


\subsection*{
\href{text00292.html}{19.5 Atlas配置文件}
}


\subsection*{
\href{text00293.html}{19.6 启动Atlas}
}


\subsection*{
\href{text00294.html}{19.7 Atlas管理操作}
}


\section*{
\href{text00295.html}{第20章 云关系型数据库}
}


\subsection*{
\href{text00296.html}{20.1 阿里云RDS}
}


\subsection*{
\href{text00297.html}{20.2 阿里云RDS for MySQL}
}


\subsection*{
\href{text00298.html#b1}{20.3 阿里云RDS云数据库的相关概念}
}


\subsubsection*{
\href{text00298.html#b1}{20.3.1 地域}
}


\subsubsection*{
\href{text00299.html}{20.3.2 可用区}
}


\subsubsection*{
\href{text00300.html}{20.3.3 RDS实例}
}


\subsubsection*{
\href{text00301.html}{20.3.4 RDS for MySQL只读实例}
}


\subsubsection*{
\href{text00302.html}{20.3.5 RDS for MySQL克隆实例}
}


\subsubsection*{
\href{text00303.html}{20.3.6 RDS for MySQL灾备实例}
}


\subsubsection*{
\href{text00304.html}{20.3.7 RDS数据库}
}


\subsubsection*{
\href{text00305.html}{20.3.8 RDS数据库账号}
}


\subsubsection*{
\href{text00308.html}{20.3.9 RDS连接数}
}


\subsubsection*{
\href{text00308.html}{20.3.10 RDS磁盘容量}
}


\subsubsection*{
\href{text00308.html}{20.3.11 RDS for MySQL读写分离}
}


\subsubsection*{
\href{text00309.html}{20.3.12 RDS for MySQL三节点企业版}
}


\subsubsection*{
\href{text00310.html}{20.3.13 RDS for MySQL单机版}
}


\subsubsection*{
\href{text00311.html}{20.3.14 RDS for MySQL跨可用去迁移}
}


\subsection*{
\href{text00312.html#b1}{20.4 阿里云RDS for MySQL数据库实战}
}


\subsubsection*{
\href{text00312.html#b1}{20.4.1 RDS for MySQL创建实例}
}


\subsubsection*{
\href{text00313.html}{20.4.2 RDS for MySQL升级实例}
}


\subsubsection*{
\href{text00314.html}{20.4.3 RDS for MySQL查看基本信息}
}


\subsubsection*{
\href{text00315.html}{20.4.4 RDS for MySQL数据库管理}
}


\subsubsection*{
\href{text00316.html}{20.4.5 RDS for MySQL远程访问}
}


\subsubsection*{
\href{text00317.html}{20.4.6 RDS for MySQL备份与恢复}
}


\subsection*{
\href{text00318.html}{20.5 RDS for MySQL性能优化、报警管理及安全控制}
}


\subsubsection*{
\href{text00319.html}{20.5.1 RDS for MySQL资源监控}
}


\subsubsection*{
\href{text00320.html}{20.5.2 RDS for MySQL数据安全性}
}


\subsubsection*{
\href{text00321.html}{20.5.3 RDS for MySQL性能优化}
}


\subsection*{
\href{text00322.html}{20.6 RDS for MySQL日志管理}
}


\subsection*{
\href{text00323.html#b1}{20.7 RDS for MySQL的只读实例和克隆}
}


\subsubsection*{
\href{text00323.html#b1}{20.7.1 RDS for MySQL只读实例}
}


\subsubsection*{
\href{text00324.html}{20.7.2 RDS for MySQL只读实例功能特点}
}


\subsubsection*{
\href{text00325.html}{20.7.3 RDS for MySQL只读实例创建过程}
}


\subsubsection*{
\href{text00326.html}{20.7.4 RDS for MySQL创建只读实例}
}


\subsection*{
\href{text00327.html}{20.8 RDS for MySQL只读实例实现读写分离}
}


\subsection*{
\href{text00328.html}{20.9 RDS for MySQL克隆实例}
}


\subsection*{
\href{text00329.html#b1}{20.10 RDS for MySQL克隆实例使用场景}
}


\subsubsection*{
\href{text00329.html#b1}{20.10.1 克隆实例用于数据回溯}
}


\subsubsection*{
\href{text00330.html}{20.10.2 克隆实例用于准生产测试}
}


\subsection*{
\href{text00331.html}{20.11 RDS for MySQL重点回顾}
}



% From text00002.html


\chapter{前言}


\subsubsection{为什么要写这本书}


\par “跟老男孩学Linux运维”系列书籍出版以来,得到了广大网友的一致好评和赞扬,但是也有很多读者和网友从各种渠道对老男孩提出了新的期待,其中之一就是系列书籍中缺少企业中最为关键的MySQL实战方向的书籍。

\par 毋庸置疑,所有互联网网站最大的瓶颈就是企业的后端数据库,而MySQL更是重中之重,谁掌握了数据库技术,谁就能轻松拿到高薪,并且数据库管理岗位比其他岗位更受企业重视,因为数据安全是企业最重要的生命线,没有之一。由于老男孩平时教学十分繁忙,还要承担公司的管理工作,使得本书的写作一直断断续续。但是,在每次教学讲到MySQL技术时,老男孩就会想起读者和网友们的殷殷期待,于是又投入到写作中,本书就是在这种情况下完成的,在此特别感谢所有的读者和网友,没有你们的持续期待和支持,这本书就不会面世。

\par 在长期的运维工作以及深度教学中,老男孩发现很多Linux运维人员以及大部分开发人员,都对数据库的技术一知半解,只停留在基本的安装和SQL简单使用上,更要命的是大家都觉得数据库很重要,但是在工作中又都很惧怕数据库的管理和维护。数据库的重要性是毋庸置疑的,但是数据库技术真的没那么难,更没那么可怕,只要稍加努力,普通人也可以掌握胜任数据库管理员岗位的绝大部分技能本领。

\par 鉴于以上,作为一个曾经维护过数十台规模的混合数据库集群的过来人,老男孩决定写一本能让零基础新手以及Linux运维、开发入门人员都有信心掌握好数据库管理与维护的实战书籍,相信本书一定会让众多读者受益,提升他们的数据库管理和维护能力,实现加薪升职。本书旨在面向非专业专职数据库管理员,让所有的非专职技术人员能够具备独立(兼)管理中小企业数据库的实战能力。

\par 本书是“跟老男孩学Linux运维”系列的第五本书(前4本已由机械工业出版社出版),更多“跟老男孩学Linux运维”实战系列图书在持续写作中,敬请期待。

\subsubsection{读者对象}


\par ·Linux入门与开发人员

\par ·Linux运维工程师

\par ·初中级数据库管理人员

\par ·网络管理员和项目实施工程师

\par ·Linux相关售前售后技术工程师

\par ·开设Linux相关课程的大中专院校

\par ·对Linux、MySQL数据库感兴趣的人群

\subsubsection{如何阅读本书}


\par 本书依然延续老男孩写书的特点,是一本偏重实战的MySQL图书,并非大而全,但处处可以体现实战二字,丰富的知识讲解取自企业中的实战案例解决方案,并结合老男孩十几年的数据库运维工作和教学工作进行了梳理。全书从脉络上共分为20章:

\par 第1~4章为数据库知识简介,以及MySQL数据库单/多实例安装介绍和实践,内容简单易懂,让读者能够快速上手掌握MySQL。

\par 第5~9章讲解的是企业中MySQL数据库的常用维护和管理知识及实践、常用管理SQL语句知识及实践、数据库备份和管理知识与实践,以及企业级数据库逻辑备份与物理备份实战案例,让读者切实掌握中小企业的数据库维护本领。

\par 第10~13章讲解的是企业中MySQL数据库常用的日志、字符集、引擎等知识,并深入讲解核心引擎InnoDB,为学好数据库知识打下坚实的基础。

\par 第14~17章讲解的是企业中MySQL数据库的最核心技术——主从复制知识,同时讲解主从复制的各种架构在企业中的实战应用、半同步与GTID下同步的应用。

\par 第18~20章讲解的是企业中MySQL数据库集群的高可用方案以及MHA的实战案例、数据库读写分离中间件的实践、阿里云数据库RDS的基本应用实践。

\subsubsection{勘误和支持}


\par 由于老男孩的教学任务很重,课程较多,这本书基本上都是利用早晨和夜晚的时间完成写作的,限于本人的水平和能力,加之编写的时间仓促,书中难免有疏漏和不当之处,恳请读者批评指正。你可以将书中的错误发布在专门为本书准备的博客“\href{http://www.itblogs.cn}{http://www.itblogs.cn}
 ”评论处,同时不管你遇到何种问题,都可以加入为本书准备的QQ交流群465216827(加群说明:MySQL),我将尽力为读者提供最满意的解答。书中所需的工具等都将发布在上述博客中,我也会将工具相应功能的更新及时发布出来。如果你有更多的宝贵意见,欢迎发送邮件至我的邮箱oldboy@oldboyedu.com或者加老男孩助理的微信17600131504,加入本书的交流群,期待能够听到读者的真挚反馈。

\subsubsection{致谢}


\par 感谢前阿里云资深专家肖海波为本书贡献了第20章RDS数据库内容。

\par 感谢老男孩教育高级讲师曾老师为本书贡献了MHA/Atlas等章节内容。

\par 感谢老男孩教育高级讲师郭老师对本书部分内容的校对和提出的修改建议。

\par 感谢老男孩教育的每一位在校学员——你们自觉努力地学习,使得我有较多的时间持续写作。感谢你们对老男孩教育的支持。

\par 感谢老男孩教育的每一位老师,正是你们辛勤努力的工作,让我得以有时间完成此书。

\par 感谢森华易腾的陆锦云女士及其同事,感谢你们提供的IDC机房带宽资源并长期支持,使得本书得以顺利完成!

\par 感谢机械工业出版社华章公司的编辑杨绣国和温莉芳女士,感谢你们的不懈支持、包容和鼓励,正是你们的鼓励和帮助引导我顺利完成全部书稿。

\par 感谢没有提及名字的所有学生、网友以及关心关注老男孩的每一个人。

\par 最后要感谢我的父母、家人,正是你们的支持和体谅,让我有无限信心和力量去写作,并最终完成此书!

\par 谨以此书,献给支持老男孩教育的每一位朋友、学员以及众多热爱Linux运维技术的人。

\par 老男孩

\par 中国,北京,2018年11月


% From text00003.html


\chapter{第1章 数据库介绍与分类}


\section{
1.1 数据库介绍}
}


\subsubsection{什么是数据库}


\par 简单地说,数据库(Database)就是一个存放计算机数据的仓库,该仓库按照一定的数据结构(数据结构是指数据的组织形式或数据之间的联系)对数据进行组织和存储,我们可以通过数据库提供的多种方法来管理其中的数据。

\par 若以生活中的案例来进行更形象的描述,那么计算机里的数据库就是类似于人们存放杂物的储物间和仓库,它们的区别只是存放的东西不同,杂物间存放的是实体物件,而数据库里存储的是计算机数据。


% From text00004.html


\section{1.2 数据库的种类}


\par 按照早期的数据库理论,比较流行的数据库模型分为三种,分别为层次数据库、网状数据库和关系型数据库。而在当今的互联网企业中,最常用的数据库模型主要分为两种,即关系型数据库
 和非关系型数据库
 。我们不是写教科书,更不是研究数据库理论,因此,本书主要讲解关系型数据库和非关系型数据库NoSQL这两类最重要也是目前互联网企业里实际使用最多的数据库类型。


% From text00005.html


\subsection{1.2.1 关系型数据库介绍}


\par (1)关系型数据库的由来

\par 虽然网状数据库和层次数据库已经很好地解决了数据的集中和共享问题
 ,但是在数据的独立性和抽象级别上仍然存在很大的欠缺。用户在对这两种数据库进行存取时,仍然需要明确数据的存储结构,指出存取路径。而关系型数据库则可以较好地解决这些问题。

\par (2)关系型数据库介绍

\par 关系型数据库模型可将复杂的数据结构归结为简单的二元关系(即二维表格形式
 )。例如,老男孩IT教育某一期的学生表格(见表1-1)关系就是一个二元关系。在关系型数据库中,对数据的操作几乎全部建立在一个或多个关系表格上,通过对这些关联的表格进行分类、合并、连接或选取等运算来实现对数据的管理。

\par 关系型数据库的诞生距今已有几十年,已经从理论逐步发展成为大家耳熟能详的企业产品,例如,最常见的MySQL和Oracle数据库,其中,Oracle在数据库领域里上升到了霸主地位,形成了每年高达数百亿美元的庞大产业市场,而MySQL也是不容忽视的数据库,近些年逐渐崛起,以至于被Oracle重金收购,目前它们占据了互联网领域90\%以上的数据库市场份额。

\par (3)关系型数据库表格之间的关系举例

\par 前面已经讲过,数据库简单来说就是一个存放计算机数据的仓库,但是数据存放在数据库里还是有一定的组织形式的。在一个关系型数据库里,二维表格是组织数据的基本形式,每个不同的数据库里可以包含多张表格,例如,oldboy库里可以包含如下的表1-1~1-3。

\par 表1-1 学生表

\begin{figure}[htbp]
\centering
\includegraphics[width=0.8\textwidth]{Image00002.jpg}
\end{figure}



\par 表1-2 课程表

\begin{figure}[htbp]
\centering
\includegraphics[width=0.8\textwidth]{Image00003.jpg}
\end{figure}



\par 表1-3 学生选课表

\begin{figure}[htbp]
\centering
\includegraphics[width=0.8\textwidth]{Image00004.jpg}
\end{figure}



\par 而数据库里的每个表格又是由若干个列组成的,例如,表1-3中的学号、课程号、成绩都是组成表格的列,此外,一个表格可以有多个列,每个列又存储着不同的信息。

\par 由此,我们可以理解列的概念了,列是表中的字段,所有的表都是由一个或多个列组成的。

\par 数据库中的每个列(或称字段)都有相应的数据类型,数据类型定义每个列存储数据的类别,例如,如果某个列中希望存储的内容为数字,那么可以选择的数据类型为数字类型(int等);如果列中存储的是少量文本或日期,甚至是大量文本等类别,那么数据库都会对应不同的数据类型来进行存储。

\par 由此我们就可以理解数据类型的概念了,数据类型用于定义每个列需要存储什么样的数据,每个列都可以根据存储数据的实际需要来设置数据类型,在指定数据类型的同时还可以指定存储数据的长度。

\par 图1-1是关系型数据库表格的基本信息说明。

\begin{figure}[htbp]
\centering
\includegraphics[width=0.8\textwidth]{Image00005.jpg}
\end{figure}



\par 图1-1 关系型数据库表格基本信息

\par 
关系型数据库知识和特点小结


\par 1)关系型数据库在存储数据时实际上采用的就是二维表格形式(和Word、Excel里的表格几乎一样)。

\par 2)市场上占有量较大的是MySQL和Oracle数据库,而互联网场景最常用的是MySQL数据库。

\par 3)通过SQL(结构化查询语言)
 来存取、管理关系型数据库的数据。

\par 4)关系型数据库在保持数据安全和数据一致性方面功能很强,遵循ACID理论(后文会讲此知识)。


% From text00006.html


\subsection{1.2.2 非关系型数据库介绍}


\par 1.非关系型数据库诞生的背景

\par 
非关系型数据库也称为NoSQL数据库,
 请注意,NoSQL的本意是“Not Only SQL”,指的是非关系型数据库,而不是“No SQL”(没有SQL)的意思。因此,NoSQL数据库的产生之初并不是要彻底地否定和终结关系型数据库,而是作为传统关系型数据库的一个有效补充。

\par 随着互联网Web2.0、Web3.0网站的兴起,传统的关系型数据库在应付这些网站,特别是对于规模日益扩大的海量数据,超大规模和高并发的微博、微信等类型的动态网站时已经显得力不从心,暴露了很多难以克服的问题,例如,传统关系型数据库的I/O瓶颈、性能瓶颈等都难以有效突破。于是出现了大批针对特定场景,以高性能、高并发以及使用便利为目的的功能特异化的数据库产品
 ,非关系型数据库就是在这样的情景中诞生并得到非常迅速发展的。在这些特定的场景下,NoSQL数据库可以发挥出难以想象的高效率和高性能。近年来,NoSQL这个术语得到了广泛认同。

\par NoSQL是非关系型数据库的广义定义。它打破了长久以来关系型数据库与ACID理论大一统的局面。NoSQL数据库的数据存储不需要固定的表结构,通常也不存在连接操作。其在大数据存取上具备关系型数据库无法比拟的性能优势,满足了企业应用需要将数据存储在横向且伸缩性上更强的功能需求。例如,Google的BigTable、Amazon的Dynamo都是非常成功的商业NoSQL实现。在开源的NoSQL体系中,从早期的Memcached缓存软件到当今Facebook的Cassandra、Apache的HBase,都得到了广泛应用,redis、MongoDB等新兴的NoSQL数据库,也逐渐受到各类公司的欢迎和追捧。NoSQL数据库没有标准的查询语言(SQL),因此进行数据库查询需要制定数据模型。许多NoSQL数据库都有REST式的数据接口或者查询API。

\par 
非关系型数据库(NoSQL)知识小结


\par 1)NoSQL数据库不是否定关系型数据库,而是作为关系型数据库的一个重要补充。

\par 2)NoSQL数据库为了适应灵活及高性能、高并发需求而生,忽略影响高性能、高并发的功能。

\par 3)在NoSQL数据库领域,当今的最典型产品为redis(持久化缓存)、MongoDB、Memcached(纯内存)等。

\par 4)NoSQL数据库没有标准的查询语言(SQL),通常使用REST式的数据接口或者查询API。

\par 2.非关系型数据库种类介绍

\par (1)键值存储数据库

\par 键值(key-value)数据库类似于传统语言中使用的哈希表,可以通过key来添加、查询或者删除数据,因为是使用key主键访问,所以会获得很高的性能及扩展性。

\par 键值数据库主要使用一个哈希表,表中有一个特定的键和一个指针(指向特定的数据)。对于IT系统来说,key/value模型的优势在于简单、易部署、高并发。

\par 下面就来举例说明。

\par k1→oldboy,k1是键,相当于学号,oldboy对应的就是真实的数据,相当于具体的人。

\par k2→oldgirl,k2是键,相当于学号,oldgirl对应的就是真实的数据,相当于具体的人。

\par 我们要找人,首先定位学号(k1,k2),然后再找到具体的数据(oldboy、oldgirl)。

\par 键值存储数据库的典型产品为Memcached、redis。

\par 
\begin{figure}[htbp]
\centering
\includegraphics[width=0.8\textwidth]{Image00006.jpg}
\end{figure}

 提示:
 Memcached、redis是互联网领域里中小型企业网站使用最多的NoSQL数据库种类。

\par (2)列存储数据库

\par 列存储(column-oriented)数据库会将数据储存在列族(column family)中,一个列族通常存储会被一起查询的相关数据。举个例子,对于Person类,我们通常会查询他们的姓名和年龄,而不是薪资。在这种情况下,姓名和年龄就会放入一个列族中,而薪资则放在另一个列族中。

\par 这部分数据库通常用于应对分布式存储的海量数据。键仍然存在,但是它们的特点是指向了多个列。这些列是由列族来安排的。

\par 列存储数据库的典型产品为Cassandra、HBase。

\par 
\begin{figure}[htbp]
\centering
\includegraphics[width=0.8\textwidth]{Image00006.jpg}
\end{figure}

 提示:
 Cassandra、HBase是互联网领域大型门户网站使用较多的NoSQL数据库种类。例如,新浪、京东网站使用的是HBase(曾经笔者所在培训机构的DBA老师就曾经在新浪公司维护数台HBase集群),360安全公司使用的是Cassandra(笔者的学生曾在该公司维护数百台Cassandra集群)。

\par (3)面向文档的数据库

\par 面向文档(document-oriented)的数据库的灵感来自于Lotus Notes办公软件,而且它同第一种键值存储相类似。该类型的数据模型是版本化的文档,半结构化的文档以特定的格式存储,比如JSON。面向文档(document-oriented)的数据库可以看作键值数据库的升级版,允许之间嵌套键值。而且面向文档(document-oriented)的数据库比键值数据库的查询更高效。

\par 面向文档的数据库会将数据以文档的形式储存。每个文档都是自包含的数据单元,是一系列数据项的集合。每个数据项都有一个与名称对应的值,值既可以是简单的数据类型,如字符串、数字和日期等;也可以是复杂的类型,如有序列表和关联对象等。数据存储的最小单位是文档,同一个表中存储的文档属性可以是不同的,数据可以使用XML、JSON或者JSONB等多种形式存储。

\par 面向文档的数据库的典型产品为MongoDB。

\par 
\begin{figure}[htbp]
\centering
\includegraphics[width=0.8\textwidth]{Image00006.jpg}
\end{figure}

 提示:
 MongoDB是互联网领域里中小型企业比较常用的面向文档的数据库种类,在部分企业里,可以替代MySQL作为数据库使用。

\par (4)图形数据库(了解即可)

\par 图形(graph)数据库允许我们将数据以图形的方式进行存储。实体会被作为顶点,而实体之间的关系则会被作为边。比如我们有三个实体Steve Jobs、Apple和Next,则会有两个“Founded by”的边将Apple和Next连接到Steve Jobs。

\par 图形结构的数据库与其他行列以及刚性结构的SQL数据库不同,它使用的是灵活的图形模型,并且能够扩展到多个服务器上。

\par 图形数据库的典型产品为Neo4J、InfoGrid

\par 
\begin{figure}[htbp]
\centering
\includegraphics[width=0.8\textwidth]{Image00006.jpg}
\end{figure}

 提示:
 在笔者工作的十多年里,没有使用过图形数据库,不过根据国外网站数据库排名信息来看,Neo4J上升到了21位(截至笔者写作时),看来该软件正在被越来越多的人所熟知,我们对它保持关注吧,建议简单了解就可以了。


% From text00007.html


\section{1.3 常用关系型数据库产品介绍}


\subsection{
1.3.1 Oracle数据库}
}


\begin{figure}[htbp]
\centering
\includegraphics[width=0.8\textwidth]{Image00007.jpg}
\end{figure}



\par Oracle前身叫作SDL,由Larry Ellison和另外两个开发人员于1977年创办,他们开发了自己的拳头产品,在市场上大量销售,1979年,Oracle公司引入了第一个商用SQL关系型数据库管理系统。Oracle公司是最早开发关系型数据库的厂商之一,其产品支持最广泛的操作系统平台。目前Oracle关系型数据库产品的市场占有率很高。

\par Oracle公司是目前全球最大的数据库软件公司,也是近年来业务增长极为迅速的软件提供商与服务商。

\par 2007年7月,Oracle公司在美国纽约宣布推出数据库Oracle 11g,这是Oracle数据库的较新版本。据官方介绍,Oracle 11g有400多项功能,经过了1500万个小时的测试,开发工作量达到了3.6万人/月。Oracle 11g在安全、XML DB、备份等方面得到了很大的提升。

\par 主要的应用企业包括传统大企业、大公司、政府、金融、证券等。

\par 版本包括Oracle 8i、Oracle 9i、Oracle 10g、Oracle 11g、Oracle 12c等。


% From text00008.html


\subsection{1.3.2 MySQL数据库}


\begin{figure}[htbp]
\centering
\includegraphics[width=0.8\textwidth]{Image00008.jpg}
\end{figure}



\par MySQL数据库是一个开源的中小型关系型数据库管理系统,由瑞典MySQL AB公司开发
 。该公司于2008年被Sun公司收购,后Sun公司又被Oracle公司收购。目前MySQL被广泛地应用于各大中小型网站中
 。由于其具有体积小、速度快、总体拥有成本低,且开放源码等特点,因此许多大中小型网站选择它作为网站数据库,从而降低网站总体拥有成本,甚至国内知名的淘宝网也选择弃用Oracle而更换成更为开放的MySQL。

\par MySQL数据库的应用范围主要包括互联网领域、大中小型网站、游戏公司、电商平台等,因用户广泛,其产生了很多高并发的成熟解决方案,因此传统企业的用户也在逐渐增多。


% From text00009.html


\subsection{1.3.3 MariaDB数据库}


\begin{figure}[htbp]
\centering
\includegraphics[width=0.8\textwidth]{Image00009.jpg}
\end{figure}



\par MariaDB数据库管理系统是MySQL数据库的一个分支
 ,主要由开源社区维护,采用GPL授权许可。开发这个MariaDB数据库分支的可能原因之一是:Oracle公司收购了MySQL之后,有将MySQL闭源的潜在风险,因此MySQL开源社区采用分支的方式来避开这个风险。

\par 开发MariaDB数据库的目的是完全兼容MySQL数据库,包括API和命令行,使之能够轻松地成为MySQL的替代品。在存储引擎方面,它使用XtraDB来代替MySQL的InnoDB。MariaDB由MySQL的创始人Michael Widenius主导开发,他早前曾以10亿美元的价格,将自己创建的公司MySQL AB卖给Sun,此后,随着Sun被甲骨文收购,MySQL的所有权也落入Oracle的手中。MariaDB数据库的名称来自MySQL的创始人Michael Widenius的女儿Maria的名字。

\par MariaDB基于事务的Maria存储引擎,替换了MySQL的MyISAM存储引擎,使用Percona的XtraDB替换了MySQL的InnoDB存储引擎。

\par MariaDB数据库的早期版本,均依照MySQL的版本发行。因此,使用MariaDB的人都会从MySQL中了解到MariaDB的相关功能,学习MySQL数据库的人,也可以轻松上手掌握MariaDB数据库。


% From text00010.html


\subsection{1.3.4 SQL Server数据库}


\par Microsoft SQL Server是微软公司开发的大型关系型数据库系统
 。SQL Server的功能比较全面、效率高,可以作为中型企业或单位的数据库平台。由于SQL Server与Windows操作系统紧密集成,所以不论是应用程序的开发速度,还是系统事务处理的运行速度,都得到了较大的提升。对于在Windows平台上开发的各种企业级信息管理系统来说,不论是C/S(客户机/服务器)架构还是B/S(浏览器/服务器)架构,SQL Server都是一个很好的选择。但SQL Server也有它的缺点,即只能在Windows系统下运行
 。

\par 1987年,微软和IBM合作开发完成OS/2,IBM在其销售的OS/2 ExtendedEdition系统中绑定了OS/2 Database Manager,而微软产品线中尚缺少数据库产品。为此,微软将目光投向Sybase,同Sybase签订了合作协议,使用Sybase的技术开发了基于OS/2平台的关系型数据库。1989年,微软发布了SQL Server 1.0版。Microsoft在与Sybase分道扬镳之后,在其6.05和7.0版本中重写了核心数据库系统,最终形成了如今的SQL Server。

\par SQL Server数据库主要应用于使用Windows服务器平台的企业。


% From text00011.html


\subsection{1.3.5 Access数据库}


\par 美国Microsoft公司于1994年推出了微机数据库管理系统。它结合了Microsoft Jet Database Engine和图形用户界面两项特点,是Microsoft Office的成员之一。具有界面友好、易学易用、开发简单、接口灵活等特点,是典型的新一代桌面关系型数据库管理系统。Access能够存取Access/Jet、Microsoft SQL Server、Oracle,或者任何ODBC兼容数据库的资料。Access界面友好而且易学易用,作为Office套件的一部分,可以与Office集成,实现无缝连接,Access提供了表(Table)、查询(Query)、窗体(Form)、报表(Report)、宏(Macro)、模块(Module)等用来建立数据库系统的对象,而且还提供了多种向导、生成器、模板,把数据存储、数据查询、界面设计、报表生成等操作规范化。

\par Access是入门级小型桌面数据库,性能、安全性都很一般。可供个人管理或小型网站之用。Access不是数据库语言,只是一个数据库程序。集成于Office中。其主要具有如下特点。

\par ·完善地管理各种数据库对象,具有强大的数据组织、用户管理、安全检查等功能。

\par ·强大的数据处理功能,Access具备了许多先进的大型数据库管理系统所具备的特征,如事务处理/出错回滚能力等。

\par ·可以方便地生成各种数据对象,利用存储的数据建立窗体和报表,可视性好。

\par ·作为Office套件的一部分,可以与Office集成,实现无缝连接。

\par ·能够利用Web检索和发布数据,实现与Internet的连接。Access主要适用于中小型应用系统,或者作为客户机/服务器系统中的客户端数据库。

\par 早期应用领域为小型程序系统ASP+Access系统,例如,留言板、校友录、BBS等,目前在互联网场景下使用得较少。


% From text00012.html


\subsection{1.3.6 PostgreSQL数据库}


\par PostgreSQL是加州大学伯克利分校计算机系开发的,以POSTGRES为基础,后来更名为PostgreSQL,它是除了MySQL之外发展较快的另一个关系型数据库管理系统(RDBMS)。

\par PostgreSQL在灵活的BSD-风格许可证下发行。它提供了相对其他开放源代码数据库系统(比如MySQL)和专有系统(比如Oracle、Sybase、Microsoft SQL Server)之外的另一种选择。

\par PostgreSQL支持大部分SQL标准,并且提供了非常多的特性:复杂查询、外键、触发器、视图、事务完整性、MVCC。同样,PostgreSQL可以用许多方法扩展,比如,通过增加新的数据类型、函数、操作符、聚集函数、索引等来扩展。任何人都可以免费使用、修改和分发PostgreSQL,不管是私用、商用,还是学术研究使用,目前数据库产品排名在第四位,是一个很有潜力的数据库,但不建议新手或者经验欠缺的企业人员使用。


% From text00013.html


\subsection{1.3.7 其他不常用的关系型数据库}


\par 本书主要关注互联网领域常用的数据库产品,像DB2、Informix、Sybase等在互联网公司几乎见不到,因此这里就不多介绍了。


% From text00014.html


\section{1.4 常用非关系型数据库产品介绍}


\subsection{
1.4.1 Memcached(key-value)}
}


\begin{figure}[htbp]
\centering
\includegraphics[width=0.8\textwidth]{Image00010.jpg}
\end{figure}



\par Memcached是一个开源的、支持高性能、高并发的分布式内存缓存系统,由C语言编写,总共2000多行代码。从软件名称上看,前3个字符“Mem”就是内存的意思,而接下来的5个字符“Cache”就是缓存的意思,最后一个字符d,是daemon的意思,代表是服务端守护进程模式服务。

\par Memcached服务分为服务端和客户端两部分,其中,服务端软件的名字形如Memcached-1.4.24.tar.gz,客户端软件的名字形如Memcached-2.25.tar.gz。

\par Memcached软件诞生于2003年,最初由LiveJournal的Brad Fitzpatrick开发完成。Memcache是整个项目的名称,而Memcached是服务器端的主程序名,因其协议简单,应用部署方便且支持高并发,因此广泛应用于互联网企业,直到现在仍然如此。其官方网站地址为:\href{http://memcached.org/}{http://memcached.org/}
 。

\par 缓存一般用于保存一些经常被存取的对象或数据(例如,浏览器会把经常访问的网页缓存起来),通过缓存来存取对象或数据要比在磁盘上存取快很多,因为前者使用的是内存,后者使用的是磁盘。Memcached是一种纯内存缓存系统,把经常存取的对象或数据缓存在Memcached的内存中,程序通过API的方式存取这些被缓存的数据,Memcached里面的数据就像一张巨大的HASH表,数据以key-value对的方式存在。Memcached通过缓存经常被存取的对象或数据,从而减轻频繁读取数据库的压力,提高网站的响应速度,构建出速度更快的可扩展的Web应用。

\par 由于Memcached为纯内存缓存软件,一旦重启,所有的数据都会丢失,因此,新浪网基于Memcached开发了一个开源项目Memcachedb。通过为Memcached增加Berkeley DB的持久化存储机制和异步主辅复制机制,使Memcached具备了事务恢复能力、持久化数据存储能力和分布式复制能力,Memcachedb非常适合需要超高性能读写速度、持久化保存的应用场景,但是最近几年其正逐渐被其他的持久化产品所替代,例如redis。

\par 
有关Memcached的知识小结


\par 1)Memcached为key-value型数据库。

\par 2)Memcached为纯内存数据库。

\par 3)Memcached相关持久化产品Memcachedb、Tokyo Cabinet\textbackslash\{\}Tokyo Tyrant(ttserver)。

\par 4)Memcached企业级讲解见《跟老男孩学Linux运维:Web集群实战》
\href{text00014.html#ch1_back}{[1]}

 第13章。

\par 
\href{text00014.html#ch1}{[1]}
 该书已由机械工业出版社出版,书号为ISBN978-7-111-52983-5。——编辑注


% From text00015.html


\subsection{1.4.2 redis(key-value)}


\begin{figure}[htbp]
\centering
\includegraphics[width=0.8\textwidth]{Image00011.jpg}
\end{figure}



\par 和Memcached类似,redis也是一个key-value型存储系统。但redis支持的value类型相对更多,包括string(字符串)、list(链表)、set(集合)和zset(有序集合)等。这些数据类型都支持push/pop、add/remove和取交集、并集和差集及更丰富的操作,而且这些操作都是原子性的。在此基础之上,redis支持各种不同方式的排序。与Memcached一样,为了保证效率,redis的数据都是缓存在内存中的。区别是redis会周期性地把更新的数据写入磁盘或者把修改操作写入追加的记录文件中,并且在此基础之上实现master-slave(主从)复制。

\par redis是一个高性能的key-value数据库。redis的出现,在很大程度上补偿了Memcached这类key/value软件的不足,在部分场合还可以对关系型数据库起到很好的补充作用。它提供了Python、Ruby、Erlang和PHP客户端,使用很方便。

\par redis的官方网址为\href{http://www.redis.io/documentation}{http://www.redis.io/documentation}
 。

\par 
redis知识小结


\par 1)支持内存缓存,这个功能和Memcached相同。

\par 2)支持持久化存储,这个功能和Memcachedb、ttserver相同。

\par 3)数据类型更丰富。比其他key-value库功能更强。

\par 4)支持主从、分布式等集群模式。

\par 5)支持队列等特殊功能。


% From text00016.html


\subsection{1.4.3 MongoDB(document-oriented)}


\begin{figure}[htbp]
\centering
\includegraphics[width=0.8\textwidth]{Image00012.jpg}
\end{figure}



\par MongoDB是一个介于关系型数据库和非关系型数据库之间的产品,是非关系型数据库当中功能最丰富,最像关系型数据库的一种产品。它支持的数据结构非常松散,类似JSON的bjson格式,因此可以存储比较复杂的数据类型。MongoDB最大的特点是其所支持的查询语言非常强大,其语法有点类似于面向对象的查询语言,几乎可以实现类似关系型数据库单表查询的绝大部分功能,而且还支持对数据建立索引。它的特点是高性能、易部署、易使用,存储数据非常方便。

\par 其主要功能特性具体如下。

\par ·面向集合存储,易存储对象类型的数据。“面向集合”(collenction-orented),意思是数据被分组存储在数据集中,称为一个集合(collenction)。每个集合在数据库中都有一个唯一的标识名,并且可以包含无限数目的文档。集合的概念类似于关系型数据库(RDBMS)里的表(table),不同的是它不需要定义任何模式(schema)。

\par ·模式自由(schema-free),这意味着对于存储在MongoDB数据库中的文件,我们不需要知道它的任何结构定义。如果需要,你完全可以把不同结构的文件存储在同一个数据库里。

\par ·支持动态查询。

\par ·支持完全索引,包含内部对象。

\par ·支持查询。

\par ·支持复制和故障恢复。

\par ·使用高效的二进制数据存储,包括大型对象(如视频等)。

\par ·自动处理碎片,以支持云计算层次的扩展性。

\par ·支持Ruby、Python、Java、C++、PHP等多种语言。

\par ·文件存储格式为BSON(JSON的一种扩展)。BSON(Binary Serialized document Format)存储形式是指:存储在集合中的文档,被存储为键-值对的形式。键用于唯一标识一个文档,为字符串类型,而值则可以是各种复杂的文件类型。

\par 此外,还可以通过网络访问。MongoDB服务端可运行在Linux、Windows或OS X平台,支持32位和64位应用,而且它是把数据存储在文件中的,为提高效率其使用内存映射文件进行管理。

\par 关于MongoDB更详细的文档请见\href{http://www.mongodb.org/display/DOCS/Manual}{http://www.mongodb.org/display/DOCS/Manual}


\par Documentation:\href{http://www.mongodb.org/display/DOCS/Home}{http://www.mongodb.org/display/DOCS/Home}
 。

\par MongoDB是很有发展潜力的课程之一,后续会重点讲解该软件。


% From text00017.html


\subsection{1.4.4 Cassandra(column-oriented)}


\par Apache Cassandra是一套开源分布式key-value存储系统。它最初由Facebook开发,用于储存特别大的数据。Facebook目前使用的正是此系统。

\par Cassandra的主要特性如下所示。

\par ·分布式。

\par ·基于列的结构化。

\par ·高伸展性。

\par Cassandra的主要特点是它不是一个数据库,而是由一堆数据库节点共同构成的一个分布式网络服务
 ,对Cassandra的一个写操作,会被复制到其他节点上去,对Cassandra的读操作,也会被路由到某个节点上去读取。对于一个Cassandra群集来说,扩展性能是比较简单的事情,只管在群集里面添加节点就可以了。

\par Cassandra是一个混合型的非关系型数据库,类似于Google的BigTable。其主要功能比Dynomite(分布式的key-value存储系统)更丰富,Cassandra最初由Facebook开发,后转变成了开源项目。它是一个网络社交云计算方面很理想的数据库。以Amazon专有的完全分布式的Dynamo为基础,结合了Google BigTable基于列族(column family)的数据模型。P2P去中心化的存储,在很多方面都可以称为Dynamo 2.0。

\par Cassandra的官方网址为:\href{http://cassandra.org}{http://cassandra.org}
 。


% From text00018.html


\subsection{1.4.5 其他非关系型数据库}


\par 除此之外,其他非关系型数据库还有HBase、BerkeleyDB、Tokyo Cabinet\textbackslash\{\}Tokyo Tyrant(ttserver)等。


% From text00019.html


\section{1.5 数据库相关知识}


\subsection{
1.5.1 数据库发展历史大事记}
}


\par 1951:Univac系统使用磁带和穿孔卡片作为数据存储。

\par 1956:IBM公司在其Model 305 RAMAC中第一次引入了磁盘驱动器。

\par 1961:通用电气(GE)公司的Charles Bachman开发了第一个数据库管理系统IDS。

\par 1969:E.F.Codd发明了关系型数据库。

\par 1973:由John J.Cullinane领导Cullinane公司开发了针对IBM主机的基于网络模型的数据库IDMS。

\par 1976:Honeywell公司推出了第一个商用关系型数据库产品Multics Relational Data Store。

\par 1979:Oracle公司引入了第一个商用SQL关系型数据库管理系统。

\par 1983:IBM推出了DB2数据库产品。

\par 1985:为Procter & Gamble系统设计的第一个商务智能系统产生。

\par 1991:W.H.“Bill”Inmon发表了“构建数据仓库”。


% From text00020.html


\subsection{1.5.2 数据库软件企业应用排名及发展趋势参考}


\par 2013年数据库软件企业应用排名如图1-2所示。

\begin{figure}[htbp]
\centering
\includegraphics[width=0.8\textwidth]{Image00013.jpg}
\end{figure}



\par 图1-2 2013年数据库软件企业应用排名

\par 2015年数据库软件企业应用排名如图1-3所示。

\begin{figure}[htbp]
\centering
\includegraphics[width=0.8\textwidth]{Image00014.jpg}
\end{figure}



\par 图1-3 2015年数据库软件企业应用排名

\par 2016年数据库软件企业应用排名如图1-4所示。

\begin{figure}[htbp]
\centering
\includegraphics[width=0.8\textwidth]{Image00015.jpg}
\end{figure}



\par 图1-4 2016年数据库软件企业应用排名

\par 2017年数据库软件企业应用排名趋势如图1-5所示。

\begin{figure}[htbp]
\centering
\includegraphics[width=0.8\textwidth]{Image00016.jpg}
\end{figure}



\par 图1-5 2017年数据库软件企业应用排名

\par 2018年数据库软件企业应用排名如图1-6所示。

\begin{figure}[htbp]
\centering
\includegraphics[width=0.8\textwidth]{Image00017.jpg}
\end{figure}



\par 图1-6 2018年数据库软件企业应用排名

\par 上述内容源自\href{http://db-engines.com/en/ranking}{http://db-engines.com/en/ranking}
 。

\par 从图1-6所示的2018年的数据库软件企业应用排名趋势可以看出,Oracle依然是稳坐第一,但是已有落后的趋势了,即将讲解的“主人公”MySQL稳坐第二,如果再加上同源的MariaDB,则已经超过了Oracle的排名,而后面的PostgreSQL、MongoDB的上升趋势也较明显,因此,建议大家重点掌握MySQL,其次是MongoDB。


% From text00021.html


\section{1.6 本章重点}


\par 1)关系型数据库知识及特点介绍。

\par 2)关系型数据库的典型产品介绍及应用场景说明。

\par 3)非关系型数据库知识及特点介绍。

\par 4)非关系型数据库典型产品介绍及应用场景说明。

\par 5)常见数据库产品的发展趋势排名。


% From text00022.html


\section{1.7 章节试题}


\par 1)什么是关系型数据库,其有什么特点?

\par 2)关系型数据库的典型产品有哪些,各自对应什么应用场景?

\par 3)什么是非关系型数据库,其有什么特点?

\par 4)非关系型数据库的常见种类及相应的典型产品有哪些?

\par 5)请分别说出5个最流行的关系型和非关系型数据库产品名称。


% From text00023.html


\chapter{第2章 MySQL数据库入门知识介绍}


\section{
2.1 MySQL介绍}
}


\subsection{
2.1.1 MySQL简介}
}


\par 前面已经介绍过,MySQL属于传统的关系型数据库产品,其开放式的架构使得用户的选择性很强,而且随着技术的逐渐成熟,MySQL支持的功能也越来越多,性能也在不断地提高,对平台的支持也在增多,此外,社区的开发与维护人数也很多。当下,MySQL因为其功能稳定、性能卓越,且在遵守GPL协议的前提下,可以免费使用与修改,因此深受用户喜爱。

\par 我们知道,关系型数据库的特点是将数据保存在不同的表中,再将这些表放入不同的数据库中,而不是将所有的数据统一放在一个大仓库里,这样的设计加快了MySQL的读取速度,而且它的灵活性和可管理性也得到了很大的提高。访问及管理MySQL数据库的最常用标准化语言为SQL——结构化查询语言。SQL使得对数据库进行存储、更新和存取信息的操作变得更加容易。例如,你能用SQL为一个网站检索产品信息及存储用户信息、博文、帖子等,有关SQL的知识后文会详细讲解。


% From text00024.html


\subsection{2.1.2 MariaDB数据库的诞生背景介绍}


\par 自甲骨文公司收购MySQL之后,MySQL在商业数据库与开源数据库领域的市场占有份额都跃居第一,这样的格局引起了部分业内人士的担忧,因为商业数据库的老大有可能将MySQL闭源,为了避免Oracle将MySQL闭源,而无开源的类MySQL数据库可用,MySQL社区采用了分支的方式——MariaDB数据库就这样诞生了,MariaDB是一个向后兼容的数据库产品,可能会在以后替代MySQL,其官方地址为\href{https://mariadb.org/}{https://mariadb.org/}
 。不过,这里还是建议大家选择更稳定且使用更广泛的MySQL数据库,可以先测试MariaDB数据库,等使用的人员多一些,社区更活跃后再正式考虑使用也不迟。


% From text00025.html


\subsection{2.1.3 为什么选择MySQL数据库}


\par 毫无疑问,选择MySQL数据库已是既成事实,绝大多数使用Linux操作系统的互联网网站都在使用MySQL作为其后端的数据库存储方式,从大型的BAT门户到电商平台、分类门户等无一例外。那么,MySQL数据库到底有哪些优势和特点,让大家不约而同地选择它呢?

\par 原因可能有以下几点。

\par ·MySQL性能卓越,服务稳定,很少出现异常宕机的情况。

\par ·MySQL开放源代码且无版权制约,自主性强,使用成本低。

\par ·MySQL历史悠久,社区及用户非常活跃,遇到问题,可以寻求帮助。

\par ·MySQL软件体积小,安装使用简单,并且易于维护,安装及维护成本低。

\par ·MySQL品牌口碑效应好,使得企业无须考虑即可直接用之。

\par ·LAMP、LNMP、LNMT(tomcat)等流行Web架构都含有MySQL。

\par ·MySQL支持多种操作系统,提供了多种API,支持多种开发语言,特别是对流行的Java、Python、PHP等语言都有很好的支持。

\par 
\begin{figure}[htbp]
\centering
\includegraphics[width=0.8\textwidth]{Image00006.jpg}
\end{figure}

 提示:
 官方给出的使用MySQL理由见\href{http://www.mysql.com/why-mysql/topreasons.html}{http://www.mysql.com/why-mysql/topreasons.html}
 。


% From text00026.html


\section{2.2 MySQL数据库分类与版本升级}


\par MySQL数据库的官方网站为\href{http://www.mysql.com}{http://www.mysql.com}
 ,其发布的MySQL版本采用双授权政策,和大多数开源产品的路线一样,MySQL数据库也有社区版和企业版
 之分,且这两个版本又各自分了四个版本依次发布,这四个版本分别为:Alpha版、Beta版、RC版和GA版本
 。


% From text00027.html


\subsection{2.2.1 MySQL数据库企业版与社区版的区别}


\par 1.MySQL数据库企业版介绍

\par MySQL企业版由MySQL AB公司内部专门的人员负责开发及维护,但同时也会吸纳社区人员编写的优秀代码及算法,并且由他们严格按照软件测试流程对这些采纳的代码进行测试,确定没有问题之后才会进行发布。简单地说,MySQL企业版是由MySQL公司内部发布的,它参考了社区版的先进代码功能和算法,是MySQL公司的赢利产品,需要付费才能使用及提供服务支持,稳定性和可靠性无疑都是最好的,当然了,企业腰包得够鼓才能买得起。据老男孩了解的信息,某知名分类门户网站2008年就购买过MySQL企业版,价格不比那些闭源的商业数据库便宜,也是大几十万。

\par 2.MySQL数据库社区版介绍

\par MySQL社区版则是由分散在世界各地的MySQL开发者、爱好者以及用户参与开发与测试的,包括软件代码的管理、测试工作,也是他们在负责。社区也会设立BUG汇报机制,收集用户在使用过程中遇到的BUG情况,相比于企业版,社区版的开发及测试环境没有那么严格。

\par 3.MySQL数据库企业版与社区版的区别

\par MySQL是成熟产品,企业版与社区版之间在性能方面相差不大。它们的区别主要集中在以下几个方面。

\par ·企业版本组织管理与测试环节控制更严格,稳定性更好。

\par ·企业版不遵守GPL协议,社区版遵守GPL协议,可以免费使用!

\par ·使用企业版后可以购买相关的服务,比如,享受7×24小时的技术支持以及定时打补丁等服务,但是用户必须为此支付服务费用。社区版的服务质量与时效性等就无法与企业版相比了。

\par ·社区版本的维护服务只能靠社区提供,其无法像企业版本那样获得故障及补丁解决服务,但是,使用社区版是完全免费的方式。


% From text00028.html


\subsection{2.2.2 MySQL数据库的四种发布版本介绍}


\par 前面已经阐述过,MySQL的版本发布采用的是双授权政策,即分为社区版和企业版,而这两个版本又各自分为四个版本,依次发布:Alpha版、Beta版、RC版和GA版本
 。

\par 这四种发布版本之间的说明及区别具体如下。

\par (1)Alpha版

\par 
Alpha版一般只在开发的公司内部运行,不对外公开
 。主要是开发者自己对产品进行测试,检查产品是否存在缺陷、错误,验证产品功能与说明书、用户手册是否一致。MySQL属于开放源代码的开源产品,因此需要世界各地的开发者、爱好者和用户参与软件的开发、测试和手册编写等工作。自然也就必须对外公布此版本的源码和产品了,以方便他人参与开发或测试工作,甚至编写与修改用户手册。

\par (2)Beta版

\par Beta版一般是完成功能的开发和所有的测试工作之后的产品,不会存在较大的功能或性能上的BUG,通常会邀请或提供给用户体验与测试,以便更全面地测试软件的不足之处或存在的问题。

\par (3)RC版

\par RC版属于生产环境发布之前的一个小版本或称候选版,是根据Beta版本的测试结果,收集到的BUG或缺陷等信息,进行修复和完善之后的一版产品。

\par (4)GA版

\par 
GA版是软件产品正式发布的版本
 ,也称生产版本的产品。一般情况下,企业的生产环境都会选择GA版本的MySQL软件,用于真实的生产环境中。偶尔有个别大型企业会为追求新功能驱动而牺牲稳定性使用其他版本,但只是个例。

\par 既然有四种发布版本,那么如何进行选择呢?

\par MySQL AB官方网站会把各种数据库版本都上传到网站,以供不同的用户下载,主要是MySQL数据库是属于开放源代码的数据库产品,其鼓励全球的技术爱好者参与研发、测试、文档编写和经验分享,甚至还包括产品的发展规划。对于Development版本、Alpha版本和Beta版本,应禁止在生产环境中使用,因为很可能会存在重大的问题或是有些功能未被完全实现。绝大多数情况下,RC版本也是不允许使用在生产环境中的,毕竟这是一个在GA版本之前(也即生产版本发布之前)的小版本。另外,对于MySQL数据库GA版本,也需要慎重选择,开源社区的产品毕竟不是经过严格的测试工序完成的,是全球开源技术人员自愿完成的,会存在比商业产品稳定性弱的缺陷。更严格的选择后面会有进一步的说明。

\par 国内门户提供的MySQL下载地址:\href{http://mirrors.sohu.com/mysql/}{http://mirrors.sohu.com/mysql/}



% From text00029.html


\section{2.3 MySQL数据库软件的命名介绍}


\par MySQL数据库软件的版本号是由3个数字和一个后缀组成。例如,mysql-5.6.40.tar.gz的版本号,其数字的含义分别如下。

\par ·第1个数字(5)是主版本号,描述了文件格式。所有版本5的发行都有相同的文件格式。

\par ·第2个数字(6)是发行级别。主版本号和发行级别组合到一起便构成了发行序列号。

\par ·第3个数字(40)是指在此发行系列的版本号,随每个新分发版而递增。通常你需要已经选择的发行(release)的最新版本(版本)。

\par 每次更新之后,版本字符串的最后一个数字会递增。如果是相对于前一个版本增加了新功能或有微小的不兼容性,那么字符串的第二个数字会递增。如果文件格式改变,那么第一个数字会递增。

\par 至于版本号中的后缀,那是用来显示发行的稳定性级别的,通过这一系列后缀显示如何改进稳定性。可能的后缀有:alpha版、beta版、rc版、没有后缀(稳定版或GA版)。这几个后缀对应的说明可参见2.2.2节。


% From text00030.html


\section{2.4 MySQL产品路线}


\subsection{
2.4.1 MySQL产品路线变更历史背景}
}


\par 早期,MySQL也是遵循版本号逐渐增加的方式发展,格式形如mysql-X.XX.XX.tar.gz,例如,老DBA都非常熟悉的生产场景下的版本:MySQL 4.1.7、MySQL 5.0.56等。

\par 近几年,为了提高MySQL产品的竞争优势
 ,提高性能、降低开发维护成本等,也为了方便企业用户更精准地选择合适的版本,MySQL在发展到5.1系列之后,又重新规划为三条产品线。下面一起来看看。


% From text00031.html


\subsection{2.4.2 MySQL-5.0.xx到MySQL-5.1.xx的产品线}


\par 第一条产品线为:MySQL 5.0.xx升级到MySQL 5.1.xx,这条产品线继续完善与改进其用户体验和性能,同时又增加了新功能,可以说这条路线是MySQL早期产品的延续,这一系列产品的发布情况及历史版本如下。

\par MySQL 5.1是当前的稳定发布系列版本。只针对漏洞修复重新发布;没有增加会影响稳定性的新功能。

\par MySQL 5.0是前一个稳定发布系列版本。只针对严重漏洞修复和安全修复重新发布;没有增加会影响该系列的重要功能,并且官方也很快停止了对其的支持。

\par MySQL 4.0和MySQL 3.23是旧的稳定发布系列版本。该版本建议不再使用,并且官方已经停止了对其的支持。

\par 目前Red hat linux 7以及CentOS 7以前的官方yum源里附带的还是MySQL 5.1的版本,但是在绝大多数的大型互联网企业的重要场景里,已经罕见5.1版本了,当下,本系列的产品线已经发布到了5.2系列(非稳定版)。


% From text00032.html


\subsection{2.4.3 MySQL-5.4.xx到MySQL-5.7.xx产品线}


\par 为了更好地整合MySQL AB公司社区和第三方公司开发的新存储引擎,以及吸收新的实现算法等,从而更好地支持SMP架构并提高性能,第二条产品线做了大量的代码重构。版本编号从5.4.xx开始,最新稳定版为5.7.xx,目前发展到了6.0.xx和8.0.xx(非稳定版)。

\par 据官方讲,MySQL 5.6是有史以来最好的版本,是世界上使用最广的开源数据库,它提供了一套新的、先进的功能,使我们能够建设新一代基于网络和嵌入式的应用和服务。

\par 在企业中,目前是MySQL 5.5和MySQL 5.6并存的主流时代,MySQL 5.7是未来,本书的讲解主要以MySQL 5.6大版本为主,同时兼顾MySQL 5.5,并尽可能兼顾MySQL 5.7。


% From text00033.html


\subsection{2.4.4 MySQL-Cluster-6.0.xx到MySQL-Cluster-7.5.xx产品线}


\begin{figure}[htbp]
\centering
\includegraphics[width=0.8\textwidth]{Image00018.jpg}
\end{figure}



\par 图2-1 MySQL的各版本发展的参考图

\par 为了更好地推广MySQL Cluster版本,提高MySQL Cluster的性能和稳定性,改进和增加功能,使其对Cluster存储引擎提供更有效的支持与优化,因此研发了第三条产品线。该产品线的版本号为6.0.xx,目前已发展到7.5.xx。

\par 图2-1为MySQL各版本的发展参考图,对应时间一般为最后更新时间,而非发布时间。


% From text00034.html


\section{2.5 生产场景中如何选择MySQL版本}


\subsection{
2.5.1 MySQL数据库发布特性}
}


\par 商业软件研发和发行公司,都会提供经过完整测试,甚至多种用户环境模拟测试及试用之后,才推出的稳定的生产环境版本,这也使得技术人员对于商业数据库软件的选择,一般不需要有太多的顾虑与考虑。

\par 在大公司(像BAT等),对于数据库软件版本的选择,也会有相关人员详细阅读其新功能或改进点知识,并且会做很多相关的研究和测试工作,最后才逐步上线,而且是从边缘业务慢慢过渡到核心业务。

\par 目前,对于MySQL的企业版本,需要注册账号才可以下载到编译过的安装包或源代码,而且账号只有一个月的有效期;社区版本的MySQL产品,则不需要注册账号,只要填写一些信息即可转到下载页面。


% From text00035.html


\subsection{2.5.2 企业生产场景选择MySQL数据库的建议}


\par 企业生产场景下,数据库是重中之重,因此选择MySQL数据库一定要慎重。下面是给出的一些选择建议。

\par 1)一定要选稳定版版本,即选择开源的社区版的稳定版。

\par 2)产品线选择,建议选择第二条产品线中的5.5或5.6版本。目前互联网公司主流版本是5.5和5.6。

\par 3)选择MySQL数据库至少发布半年以上的稳定版本。

\par 4)要尽可能选择前后几个月没有大的BUG修复的版本,而不是大量修复BUG的集中版本。

\par 5)最好选择向后较长时间没有更新发布的版本。

\par 6)要考虑开发人员开发程序使用的版本是否兼容你所选的版本。

\par 7)首先作为内部开发测试数据库环境,测试运行几个月的时间。

\par 8)优先对企业非核心业务采用新的数据库稳定版本软件。

\par 9)向高手请教或者在及技术氛围好的群里和大家一起交流,使用真正的高手用过的、好用的且BUG少的GA版本产品。

\par 经过上述工序之后,若是没有重要的功能BUG或性能瓶颈,则可以开始考虑作为任何业务数据服务的后端数据库软件了。

\par 上述为选择的流程建议,实际上很多中小公司都会直接选择GA版的最新版本,如果业务中使用的功能比较单一,且数据和访问量不是很大,一般也不会有什么大问题,上述选择的建议,是以大规模大数据量高并发,使用数据库功能较多的场景为前提进行讲解的,对于这类场景,一定要慎重选择。

\par 参考资料:

\par 
\href{http://baike.baidu.com/view/24816.htm}{http://baike.baidu.com/view/24816.htm}


\par 
\href{http://baike.baidu.com/view/2521908.htm}{http://baike.baidu.com/view/2521908.htm}



% From text00036.html


\section{2.6 章节试题}


\par 1)在企业中如何正确选择MySQL的版本?

\par 2)请描述MySQL的不同产品路线及发展情况?

\par 3)MySQL与MariaDB的基本区别是什么?


% From text00037.html


\chapter{第3章 MySQL数据库安装方法及安装实践}


\section{
3.1 MySQL数据库的安装方法及选择}
}


\par 在当今的互联网企业里,MySQL数据库大多运行在Linux系列操作系统上,当然,你也可以运行在Windows/Unix等商业操作系统上,本书主要以国内互联网公司应用最多的数据库服务操作系统——CentOS 6最新版(6.8)x86\_64 Linux系统为例进行讲解,使用其他系统的读者同样也可以从本书受益!

\par 即使是在CentOS 6 x86\_64 Linux系统环境下,若应用场景不同或版本不同,MySQL数据库的安装方法也会有所区别,下面我们就把最常见的几种方法一一介绍给大家!


% From text00038.html


\subsection{3.1.1 yum/rpm方式安装MySQL}


\par MySQL官方网站及相关镜像网站提供了不同版本的RPM安装包,并且针对不同的硬件或操作系统平台,安装包的类型也会有区别。在使用时,可以到官方网站的下载页面进行选择,国内有一些互联网公司提供了镜像文件下载,比如搜狐、阿里云公司提供的镜像资源就非常不错。

\par 下面是搜狐网提供的数据库软件镜像地址:\href{http://mirrors.sohu.com/mysql}{http://mirrors.sohu.com/mysql}
 ,如果本书使用的版本不再发布,那么可以选择5.6的其他相关版本,老男孩也在本书开头提供的地址中上传了本书使用的版本的软件。

\par 
\begin{figure}[htbp]
\centering
\includegraphics[width=0.8\textwidth]{Image00019.jpg}
\end{figure}

 注意:
 yum/rpm安装方式适合所有MySQL软件产品。

\par 1.rpm包方式安装MySQL

\par rpm包的安装方式非常简单,这里以el6平台下的MySQL 5.6.40版本为例,首先,要通过上述搜狐镜像地址下载到如下四个MySQL相关软件安装包。

\noindent\rule{\textwidth}{0.4pt}


\begin{verbatim}MySQL-client-5.6.40-1.el6.x86\_64.rpm
MySQL-devel-5.6.40-1.el6.x86\_64.rpm
MySQL-server-5.6.40-1.el6.x86\_64.rpm
MySQL-shared-5.6.40-1.el6.x86\_64.rpm
\end{verbatim}


\noindent\rule{\textwidth}{0.4pt}


\par 
\begin{figure}[htbp]
\centering
\includegraphics[width=0.8\textwidth]{Image00006.jpg}
\end{figure}

 提示:
 我们可以从Linux的对应系统盘或系统镜像里找到类似的rpm包,但是版本一般会较低一些。

\par 一般来说,其中的MySQL-server-5.6.40-1.el6.x86\_64.rpm和MySQL-client-5.6.40-1.el6.x86\_64.rpm这两个软件包是必须要安装的,至于另外两个软件包,则可视实际需要进行安装,不过一般建议一起安装。

\par 可以把这四个rpm包上传到服务器的目录中,然后执行如下rpm命令进行安装:

\noindent\rule{\textwidth}{0.4pt}


\begin{verbatim}[root@oldboy tools]\# rpm -qa|grep mysql         \#<==查找已经安装的mysql的包。
mysql-libs-5.1.73-7.el6.x86\_64
rpm -e mysql-libs-5.1.73-7.el6.x86\_64 --nodeps  \#<==卸载系统已经安装的mysql依赖包。
rpm -ivh MySQL-client-5.6.40-1.el6.x86\_64.rpm
rpm -ivh MySQL-devel-5.6.40-1.el6.x86\_64.rpm
rpm -ivh MySQL-shared-5.6.40-1.el6.x86\_64.rpm
rpm -ivh MySQL-server-5.6.40-1.el6.x86\_64.rpm
\end{verbatim}


\noindent\rule{\textwidth}{0.4pt}


\par 这里的el6表示适合操作系统的版本,还有el5、el7等。i686表示适合32位的系统,x86\_64表示适合64位的系统。

\par 执行上述命令即可完成MySQL软件的安装。

\par 在采用rpm包安装方式时,必须要官方或第三方提供了现成的rpm软件包,否则是无法使用该方式安装的。另外,和直接采用yum的安装方式相比,rpm包的安装方式往往可以选择更新的版本,但是rpm包安装也有自身的问题,例如,无法满足定制化安装,比如,不能进行编译参数、路径等的更改。

\par 2.yum方式安装MySQL

\par yum方式安装MySQL数据库时,只需要执行一个命令“yum install mysql-server-y”即可,yum方式的安装原理是在执行yum安装命令之后,其会自动从yum源地址下载相应名称的MySQL数据库rpm包,然后到系统上安装,并自动解决各种软件包之间的依赖问题。这是一个非常不错的安装软件的方式,不仅仅是针对MySQL,安装其他软件也是如此。

\par yum安装方式的最大优点就是超级简单,但是它也有自身的问题:例如它继承了rpm包的无法定制化安装的问题;另外一个缺点是采用默认的yum安装时,一般随yum源附带的软件版本都比较低,例如,截至作者写作本文时(2018.5),使用CentOS 6.9 Linux默认yum安装的MySQL版本仅为5.1.73。


% From text00039.html


\subsection{3.1.2 采用常规方式编译安装MySQL}


\par 采用常规方式编译安装MySQL时,适合使用第一条最正宗的MySQL产品线5.2及以前版本:

\par 所谓常规方式编译安装MySQL就是延续早期MySQL的3部曲安装方式,即“./configure;make;make install”,下面是老男孩在早期的企业生产场景下操作过的具体命令及参数:

\noindent\rule{\textwidth}{0.4pt}


\begin{verbatim}tar zxf mysql-5.1.73.tar.gz
cd mysql-5.1.73
./configure \textbackslash\{\}
--prefix=/application/mysql5.1.73 \textbackslash\{\}
--with-unix-socket-path=/application/mysql5.1.73/tmp/mysql.sock \textbackslash\{\}
--localstatedir=/application/mysql5.1.73/data \textbackslash\{\}
--enable-assembler \textbackslash\{\}
--enable-thread-safe-client \textbackslash\{\}
--with-mysqld-user=mysql \textbackslash\{\}
--with-big-tables \textbackslash\{\}
--without-debug \textbackslash\{\}
--with-pthread \textbackslash\{\}
--enable-assembler \textbackslash\{\}
--with-extra-charsets=complex \textbackslash\{\}
--with-readline \textbackslash\{\}
--with-ssl \textbackslash\{\}
--with-embedded-server \textbackslash\{\}
--enable-local-infile \textbackslash\{\}
--with-plugins=partition,innobase \textbackslash\{\}
--with-mysqld-ldflags=-all-static \textbackslash\{\}
--with-client-ldflags=-all-static
make
make install
ln -s /application/mysql-5.1.73/ /application/mysql
\end{verbatim}


\noindent\rule{\textwidth}{0.4pt}


\par 安装到这里,MySQL数据库还不能正常启动使用,还需要进行初始化数据库等工作,具体可以参考后文的安装部分。

\par 此种方式适合所有MySQL 5.2.xx及以前的产品系列,是最常规的编译方式,在当下的互联网企业中,此种编译安装的方法已经很少使用了,原因是第一条产品线的产品(MySQL 5.2.xx及以前的产品系列)用得越来越少了,被第二条产品线(MySQL 5.4.xx及以后的产品系列)的产品逐渐替代了,因此,老男孩也不建议读者再使用第一条产品线的产品作为对外的业务库。


% From text00040.html


\subsection{3.1.3 采用cmake方式编译安装MySQL}


\par 考虑到MySQL 5.4.xx及以后系列产品的特殊性,其编译方式和早期的第一条产品线的有所不同,这里采用cmake或gmake的方式编译安装。即“./cmake;make;make install”,生产场景的具体命令及参数具体如下:

\noindent\rule{\textwidth}{0.4pt}


\begin{verbatim}tar zxf mysql-5.6.40.tar.gz
cd mysql-5.6.40
cmake . -DCMAKE\_INSTALL\_PREFIX=/application/mysql-5.6.40 \textbackslash\{\}
-DMYSQL\_DATADIR=/application/mysql-5.6.40/data \textbackslash\{\}
-DMYSQL\_UNIX\_ADDR=/application/mysql-5.6.40/tmp/mysql.sock \textbackslash\{\}
-DDEFAULT\_CHARSET=utf8 \textbackslash\{\}
-DDEFAULT\_COLLATION=utf8\_general\_ci \textbackslash\{\}
-DWITH\_EXTRA\_CHARSETS=all \textbackslash\{\}
-DWITH\_INNOBASE\_STORAGE\_ENGINE=1 \textbackslash\{\}
-DWITH\_FEDERATED\_STORAGE\_ENGINE=1 \textbackslash\{\}
-DWITH\_BLACKHOLE\_STORAGE\_ENGINE=1 \textbackslash\{\}
-DWITHOUT\_EXAMPLE\_STORAGE\_ENGINE=1 \textbackslash\{\}
-DWITH\_ZLIB=bundled \textbackslash\{\}
-DWITH\_SSL=bundled \textbackslash\{\}
-DENABLED\_LOCAL\_INFILE=1 \textbackslash\{\}
-DWITH\_EMBEDDED\_SERVER=1 \textbackslash\{\}
-DENABLE\_DOWNLOADS=1 \textbackslash\{\}
-DWITH\_DEBUG=0
\#提示,编译时可配置的选项很多,具体可参考本章最后一部分的内容或官方文档。
make
make install
ln -s /application/mysql-5.6.40/ /application/mysql
\end{verbatim}


\noindent\rule{\textwidth}{0.4pt}


\par 安装到这里,MySQL数据库仍然无法正常启动使用,还需要进行初始化数据库等工作,具体可以参考后文的安装部分,另外,cmake等用于编译的工具也需要提前进行安装。相关参数的说明,请见本章结尾的内容。

\par 如果上述操作未出现错误,则MySQL 5.6.40软件cmake方式的安装就算成功了。


% From text00041.html


\subsection{3.1.4 采用二进制方式免编译安装MySQL}


\par 采用二进制方式免编译安装MySQL的方法和yum/rpm包安装的方式类似,适合各类MySQL产品系列,不需要复杂的编译设置及编译时间等待,直接解压下载软件包,就相当于编译方式的make install步骤完成了,然后只要进行初始化数据库的操作,即可完成并启动。此方式的MySQL软件包一般都比较大,最大可达180MB,采用二进制方式免编译安装
 MySQL的方法在后文会详细讲解。


% From text00042.html


\subsection{3.1.5 如何正确选择MySQL的安装方式}


\par 若是对数据库要求不太高的场景,则可以采用yum/rpm方式安装MySQL,例如,并发不大,只是在公司内部(wiki系统)、企业内部(Zabbix监控系统,OpenStack后台管理)等需要数据库的一些应用场景,当然,生产场景下也是可以选择yum或rpm方式进行安装的。

\par 但是,有很多大型网站或门户网站,在安装MySQL时,往往会有各种定制化、初始化的需求,这时,要根据企业的需求先把源码包制作成rpm包,然后搭建自己的yum仓库,最终采用“yum install mysql-server-y”的方式进行安装,这样做的优点是既兼顾了yum/rpm安装方式简单的优点,又用到了源码包安装方式的可定制性,但是,使用这个方法需要一定的技术能力,此部分的内容建议参考老男孩教育为读者提供的如下博文。

\par 自动化运维必备技能——定制属于自己的RPM包\href{http://blog.oldboyedu.com/autodeploy-rpm/}{http://blog.oldboyedu.com/autodeploy-rpm/}
 。

\par 自动化运维必备技能——搭建属于自己的YUM仓库\href{http://blog.oldboyedu.com/autodeploy-yum/}{http://blog.oldboyedu.com/autodeploy-yum/}
 。

\par 二进制免编译安装方式简单方便,且适合5.0-5.1和5.5-5.7系列,是不少专业DBA的选择,普通Linux运维人员一般多采用编译的方式进行安装,对应到MySQL 5.0-5.1系列就是常规编译方式,对应到MySQL 5.5-5.7系列就是cmake编译方式。

\par 所以综合来讲,这些安装方式都是可以使用的,只是不同层次的人习惯不同,实际应用的性能差距不是很大。

\par 老男孩的建议:首先是选择MySQL5.5或以上的数据库版本
 ,当数据库服务器机器数量少的话,可以采用cmake编译方式安装,这是很多运维人员的习惯选择。在数据库服务器机器数量多的情况下,可采用二进制免编译方式安装,这是某些DBA的偏爱,若是数据库服务器机器数量特别大,且对定制化要求很高,则可以选择通过源码定制rpm包,搭建yum仓库的安装方式。当然了,采用此种方法的读者也要具备这方面的能力才行,前文已经给出了做rpm定制以及yum仓库搭建的地址,这里不再赘述。


% From text00043.html


\section{3.2 安装并配置MySQL数据库}


\subsection{
3.2.1 安装MySQL数据库}
}


\par 1.MySQL数据库的安装环境准备

\par 如果读者没有物理服务器环境,则可以搭建vmware等虚拟机环境学习,相应地需要准备如下内容。

\par 1)请提前加大VM虚拟机硬件的内存,最好将内存设置成2GB以上。

\par 2)最好提前下载好要安装的MySQL相关软件包(\href{http://mirrors.sohu.com/mysql/}{http://mirrors.sohu.com/mysql/}
 ),如果本书使用的软件版本已更新,请更新至5.6的最新版即可。

\par 3)重视每个操作过程的输出,有错误要解决掉然后再继续,不能忽略掉操作中的错误(error)。

\par 4)建议进入虚拟机界面去执行make以及make install,通过SSH操作有时会导致网络中断。

\par 有关vmware虚拟化学习软件和CentOS 6操作系统的安装详细步骤见《跟老男孩学Linux运维:Web集群实战》,或者参看免费部署文章:\href{http://book.51cto.com/art/201605/510756.htm}{http://book.51cto.com/art/201605/510756.htm}
 。

\par 
\begin{figure}[htbp]
\centering
\includegraphics[width=0.8\textwidth]{Image00006.jpg}
\end{figure}

 特别提示:
 Linux操作系统的安装选包很重要,如果系统安装的不合适(不要按照上述文章来安装,除非你真正知道自己在做什么),会导致MySQL无法正常安装及使用。

\par 2.安装MySQL需要的依赖包和编译软件

\par 当前的Linux系统环境情况如下:

\noindent\rule{\textwidth}{0.4pt}


\begin{verbatim}[root@oldboy \textasciitilde{}]\# cat /etc/redhat-release   \#<==操作系统版本
CentOS release 6.9 (Final)
[root@oldboy \textasciitilde{}]\# uname -r                  \#<==内核版本
2.6.32-696.el6.x86\_64
[root@oldboy \textasciitilde{}]\# uname -m
x86\_64                                     \#<==64位系统
\end{verbatim}


\noindent\rule{\textwidth}{0.4pt}


\par (1)安装MySQL需要的依赖包

\par 安装MySQL之前,最好先安装MySQL需要的依赖包,不然后面会出现很多报错信息,到那时还得再回来安装MySQL的依赖包。安装命令如下:

\noindent\rule{\textwidth}{0.4pt}


\begin{verbatim}[root@oldboy \textasciitilde{}]\# yum install ncurses-devel libaio-devel -y  \#<==确保系统可以
                                                         上网再执行。
[root@oldboy \textasciitilde{}]\# rpm -qa ncurses-devel libaio-devel
ncurses-devel-5.7-4.20090207.el6.x86\_64
libaio-devel-0.3.107-10.el6.x86\_64
\end{verbatim}


\noindent\rule{\textwidth}{0.4pt}


\par 
\begin{figure}[htbp]
\centering
\includegraphics[width=0.8\textwidth]{Image00006.jpg}
\end{figure}

 提示:
 安装后使用rpm-qa ncurses-devel libaio-devel命令进行检查,如果出现两行如上信息则表示安装成功。

\par (2)安装编译MySQL需要的软件

\par 由于MySQL 5.5及以上的系列产品要采用特殊的编译方式安装,因此,需要先安装常用的编译MySQL的工具cmake软件包,命令如下:

\noindent\rule{\textwidth}{0.4pt}


\begin{verbatim}[root@oldboy \textasciitilde{}]\# yum install cmake -y
[root@oldboy \textasciitilde{}]\# rpm -qa cmake
cmake-2.8.12.2-4.el6.x86\_64
\end{verbatim}


\noindent\rule{\textwidth}{0.4pt}


\par 
\begin{figure}[htbp]
\centering
\includegraphics[width=0.8\textwidth]{Image00006.jpg}
\end{figure}

 提示:
 安装后使用“rpm-qa cmake”检查,如果出现一行如上信息则表示安装成功。

\par 此外,也有网友采用源码包的方式安装cmake的,但是操作起来比较复杂,因此一般建议读者选择这个简单的yum安装方法。

\par 3.开始安装MySQL

\par 为了让大家学习更多的MySQL技术,本文选择以相对复杂的源代码安装方式为例来讲解MySQL多实例安装,大型公司一般都会将MySQL软件定制成rpm包,然后放到yum仓库里,使用yum安装,中小型企业里的二进制和编译安装的区别不大。

\par 使用二进制方式安装MySQL的方法见\href{http://oldboy.blog.51cto.com/2561410/1893734}{http://oldboy.blog.51cto.com/2561410/1893734}
 。

\par (1)建立MySQL用户账号

\par 首先以root身份登录到Linux系统中,然后执行如下命令创建mysql用户账号:

\noindent\rule{\textwidth}{0.4pt}


\begin{verbatim}[root@oldboy \textasciitilde{}]\# useradd -s /sbin/nologin -M mysql \#<==默认会创建和mysql用户
                                                 同名的组。
[root@oldboy \textasciitilde{}]\# id mysql
uid=500(mysql) gid=500(mysql) groups=500(mysql)
\end{verbatim}


\noindent\rule{\textwidth}{0.4pt}


\par 根据上述结果输出,可以看到mysql用户和组已经创建成功。

\par (2)获取MySQL软件包

\par MySQL软件包的下载地址为:\href{http://dev.mysql.com/downloads/mysql/}{http://dev.mysql.com/downloads/mysql/}
 (如果地址变更无法下载,则可以去\href{http://mirrors.sohu.com/mysql}{http://mirrors.sohu.com/mysql}
 下载或者去本章开头提供的云盘地址去下载相关的软件包)。可以把软件下载到客户端电脑本地后,使用rz等工具传到Linux里,或者找到网络下载地址后,直接在Linux里使用wget命令下载下来。

\par 
\begin{figure}[htbp]
\centering
\includegraphics[width=0.8\textwidth]{Image00006.jpg}
\end{figure}

 提示:
 本书以MySQL编译的方式进行讲解,使用二进制方式安装的完整过程前文已经给过地址了。在生产场景中,二进制和源码包两种安装方法都是可用的,其应用场景一般没什么太大差别。不同之处在于:二进制的安装包较大,名字和源码包也有些区别,二进制的安装过程比源码更快,更多区别在前文选择安装方式章节中已经讲解过了,这里就不再赘述了。

\par MySQL源码包和二进制安装包的名称见表3-1。

\par 表3-1 MySQL二进制和源码包名称

\begin{figure}[htbp]
\centering
\includegraphics[width=0.8\textwidth]{Image00020.jpg}
\end{figure}



\par (3)采用编译方式安装MySQL

\par 配置及编译安装的步骤具体如下。

\par 第一步,下载mysql软件包,命令及操作如下:

\noindent\rule{\textwidth}{0.4pt}


\begin{verbatim}[root@oldboy \textasciitilde{}]\# mkdir -p /home/oldboy/tools  \#<==这是老男孩习惯使用的软件目录。
[root@oldboy \textasciitilde{}]\# cd /home/oldboy/tools/
[root@oldboy tools]\# wget -q http://mirrors.163.com/mysql/Dowloads/MySQL-5.6/mysql-5.6.40.tar.gz
\#<==如果数据库软件更新了,请参考前文执行更换。
[root@oldboy tools]\# ls -lh
total 31M
-rw-r--r--. 1 root root 31M Nov 28 07:46 mysql-5.6.40.tar.gz  \#<==源码包大小31M。
\end{verbatim}


\noindent\rule{\textwidth}{0.4pt}


\par 第二步,解压并配置mysql,命令及操作如下:

\noindent\rule{\textwidth}{0.4pt}


\begin{verbatim}[root@oldboy tools]\# tar xf mysql-5.6.40.tar.gz
[root@oldboy tools]\# cd mysql-5.6.40
[root@oldboy tools]\# cmake . -DCMAKE\_INSTALL\_PREFIX=/application/mysql-5.6.40 \textbackslash\{\}
-DMYSQL\_DATADIR=/application/mysql-5.6.40/data \textbackslash\{\}
-DMYSQL\_UNIX\_ADDR=/application/mysql-5.6.40/tmp/mysql.sock \textbackslash\{\}
-DDEFAULT\_CHARSET=utf8 \textbackslash\{\}
-DDEFAULT\_COLLATION=utf8\_general\_ci \textbackslash\{\}
-DWITH\_EXTRA\_CHARSETS=all \textbackslash\{\}
-DWITH\_INNOBASE\_STORAGE\_ENGINE=1 \textbackslash\{\}
-DWITH\_FEDERATED\_STORAGE\_ENGINE=1 \textbackslash\{\}
-DWITH\_BLACKHOLE\_STORAGE\_ENGINE=1 \textbackslash\{\}
-DWITHOUT\_EXAMPLE\_STORAGE\_ENGINE=1 \textbackslash\{\}
-DWITH\_ZLIB=bundled \textbackslash\{\}
-DWITH\_SSL=bundled \textbackslash\{\}
-DENABLED\_LOCAL\_INFILE=1 \textbackslash\{\}
-DWITH\_EMBEDDED\_SERVER=1 \textbackslash\{\}
-DENABLE\_DOWNLOADS=1 \textbackslash\{\}
-DWITH\_DEBUG=0
\end{verbatim}


\noindent\rule{\textwidth}{0.4pt}


\par 
\begin{figure}[htbp]
\centering
\includegraphics[width=0.8\textwidth]{Image00006.jpg}
\end{figure}

 提示:


\par 1)编译时可配置的选项很多,常见的参数选项请参见3.4节的选项说明讲解,此处就不特别说明了。

\par 2)编译MySQL需要安装gcc等工具,此部分在安装操作系统时就已经安装上了,具体请参考CentOS 6操作系统的详细安装步骤,可参见《跟老男孩学Linux运维:Web集群实战》——书或免费部署文章\href{http://book.51cto.com/art/201605/510756.htm}{http://book.51cto.com/art/201605/510756.htm}
 。

\par 3)如果读者感觉输入这些配置比较费时间,可以加本篇开头的QQ交流群获取文件。

\par 第三步,编译并安装mysql,命令及操作如下:

\noindent\rule{\textwidth}{0.4pt}


\begin{verbatim}[root@oldboy mysql-5.6.34]\# make      \#<==如果是多核cpu,则可指定make -j
                                        cpu核数,加快编译速度。
[root@oldboy mysql-5.6.34]\# make install
\end{verbatim}


\noindent\rule{\textwidth}{0.4pt}


\par 如果是虚拟环境测试,以上两步过程可能有些长,请耐心等待。

\par 第四步,为mysql安装路径设置不带版本号的软链接/application/mysql,操作命令如下:

\noindent\rule{\textwidth}{0.4pt}


\begin{verbatim}[root@oldboy mysql-5.6.34]\# ln -s /application/mysql-5.6.34/ /application/mysql
\#补充:如果系统里有曾经安装的数据库文件和启动程序,则最好停掉或删除,以免发生冲突。
[root@oldboy mysql-5.6.34]\# ls -l /application/
total 4
lrwxrwxrwx.  1 root root   26 Feb 26 17:49 mysql -> /application/mysql-5.6.34/
drwxr-xr-x. 13 root root 4096 Feb 26 17:49 mysql-5.6.34
[root@oldboy mysql-5.6.34]\# ls /application/mysql/    \#<==这是老男孩常用的mysql的
                                                安装路径。
bin  COPYING  data  docs  include  lib  man  mysql-test  README  scripts  share  sql-bench  support-files
\end{verbatim}


\noindent\rule{\textwidth}{0.4pt}


\par 如果上述操作未出现错误(每个步骤结束后,都可以使用“echo\$?”查看返回值是否为0,为0则表示正确),则查看/application/mysql/目录;若其下有内容,则表示MySQL 5.6.40源代码包采用cmake方式安装成功了。


% From text00044.html


\subsection{3.2.2 创建MySQL数据库配置文件并对数据库目录授权}


\par MySQL 5.5及老版数据库默认为用户提供了多个配置文件模板,但是MySQL 5.6的support-files目录下已经没有配置文件模板了。

\noindent\rule{\textwidth}{0.4pt}


\begin{verbatim}[root@oldboy mysql-5.6. 40]\# ll support-files/*.cnf
-rw-r--r--. 1 root root 1126 Feb 26 17:54 support-files/my-default.cnf
[root@oldboy mysql-5.6. 40]\# mv /etc/my.cnf /etc/my.cnf.bak
\#提示:在CentOS 6.9版操作系统最小化安装完成之后,在/etc目录下会存在一个my.cnf,需要将此文件更名为其他的名字,如:/etc/my.cnf.bak,否则,该文件会干扰源码安装的MySQL的正确配置,造成无法启动。
\#在启动MySQL服务时,会按照一定的顺序搜索my.cnf,先在/etc目录下找,若找不到则搜索"\$basedir/my.cnf",本例依旧选择大家熟悉的/etc/路径!
[root@oldboy mysql-5.6.40]\# cp support-files/my-default.cnf /etc/my.cnf
提示:此行操作可以省略,在下文初始化mysql时会自动生成my.cnf模板文件,如果已经执行了上述命令,则初始化后会生成my-new.cnf文件,my.cnf和my-new.cnf除了注释以外是一致的。
[root@oldboy mysql-5.6. 40]\# chown -R mysql.mysql /application/mysql/
\#<==授权mysql用户管理mysql的安装目录,此步不做会导致启动服务错误。
\end{verbatim}


\noindent\rule{\textwidth}{0.4pt}


\par 关于mysql的配置文件my.cnf参数的更多说明及调优,请参看本书后面的章节。


% From text00045.html


\subsection{3.2.3 初始化MySQL数据库文件}


\par 上述配置完毕后,就可以初始化数据库数据文件了,这个步骤其实也可以在编译安装MySQL之后就操作,只不过放到这里更合理一些。

\par (1)初始化MySQL数据库

\par 初始化数据库的核心命令为:

\noindent\rule{\textwidth}{0.4pt}


\begin{verbatim}/application/mysql/scripts/mysql\_install\_db --basedir=/application/mysql/ --datadir=/application/mysql/data --user=mysql
\end{verbatim}


\noindent\rule{\textwidth}{0.4pt}


\par 
\begin{figure}[htbp]
\centering
\includegraphics[width=0.8\textwidth]{Image00006.jpg}
\end{figure}

 提示:
 --basedir=/application/mysql/为MySQL的安装路径,--datadir为数据文件目录。另外,需要注意mysql\_install\_db和MySQL 5.1的路径不同,MySQL 5.1在MySQL bin路径下。

\par 整个初始化的操作过程为:

\noindent\rule{\textwidth}{0.4pt}


\begin{verbatim}[root@oldboy mysql-5.6.34]\# /application/mysql/scripts/mysql\_install\_db --basedir=/application/mysql/ --datadir=/application/mysql/data --user=mysql
\#<==初始化mysql数据库文件,会有很多信息提示,如果没有ERROR级别的错误,有两个OK的字样,
      则表示初始化成功,否则就要解决初始化的问题。
WARNING:The host‘oldboy’could not be looked up with/application/mysq1//bin/resolveip.
\#<==如果没做主机名解析,则提示警告。
Installing MySQL system tables...2018-05-10 18:00:25 0 [Warning] TIMESTAMP with implicit DEFAULT value is deprecated. Please use --explicit\_defaults\_for\_timestamp server option (see documentation for more details).
2018-05-10 00:08:25 0 [Note] Ignoring --secure-file-priv value as server is running with --bootstrap.
2018-05-10 00:08:25 0 [Note] /application/mysql//bin/mysqld (mysqld 5.6.40) starting as process 47993 ...
2018-05-10 00:08:25 16787 [Note] InnoDB: Using atomics to ref count buffer pool pages
2018-05-10 00:08:25 16787 [Note] InnoDB: The InnoDB memory heap is disabled
2018-05-10 00:08:25 16787 [Note] InnoDB: Mutexes and rw\_locks use GCC atomic builtins
2018-05-10 00:08:25 16787 [Note] InnoDB: Memory barrier is not used
2018-05-10 00:08:25 16787 [Note] InnoDB: Compressed tables use zlib 1.2.3
2018-05-10 00:08:25 16787 [Note] InnoDB: Using Linux native AIO
2018-05-10 00:08:25 16787 [Note] InnoDB: Using CPU crc32 instructions
2018-05-10 00:08:25 16787 [Note] InnoDB: Initializing buffer pool, size = 128.0M
2018-05-10 00:08:25 16787 [Note] InnoDB: Completed initialization of buffer pool
2018-05-10 00:08:25 16787 [Note] InnoDB: The first specified data file ./ibdata1 did not exist: a new database to be created!
2018-05-10 00:08:25 16787 [Note] InnoDB: Setting file ./ibdata1 size to 12 MB
2018-05-10 00:08:25 16787 [Note] InnoDB: Database physically writes the file full: wait...
...省略若干...
2018-05-10 00:08:28 16787 [Note] InnoDB: Waiting for purge to start
2018-05-10 00:08:28 16787 [Note] InnoDB: 5.6.34 started; log sequence number 0
2018-05-10 00:08:28 16787 [Note] Binlog end
2018-05-10 00:08:28 16787 [Note] InnoDB: FTS optimize thread exiting.
2018-05-10 00:08:28 16787 [Note] InnoDB: Starting shutdown...
2018-05-10 00:08:28 16787 [Note] InnoDB: Shutdown completed; log sequence number 1625977
OK \#<==两个OK是初始化成功的标志。
Filling help tables... 2018-05-10 00:08:28 0 [Warning] TIMESTAMP with implicit DEFAULT value is deprecated. Please use --explicit\_defaults\_for\_timestamp server option (see documentation for more details).
2018-05-10 00:08:28 0 [Note] Ignoring --secure-file-priv value as server is running with --bootstrap.
2018-05-10 00:08:28 0 [Note] /application/mysql//bin/mysqld (mysqld 5.6.34) starting as process 48015 ...
2018-05-10 00:08:28 16809 [Note] InnoDB: Using atomics to ref count buffer pool pages
2018-05-10 00:08:28 16809 [Note] InnoDB: The InnoDB memory heap is disabled
...省略若干...
2018-05-10 00:08:28 16809 [Note] InnoDB: FTS optimize thread exiting.
2018-05-10 00:08:28 16809 [Note] InnoDB: Starting shutdown...
2018-05-10 00:08:28 16809 [Note] InnoDB: Shutdown completed; log sequence number 1625987
OK \#<==两个OK是初始化成功的标志。
To start mysqld at boot time you have to copy
support-files/mysql.server to the right place for your system
PLEASE REMEMBER TO SET A PASSWORD FOR THE MySQL root USER !
To do so, start the server, then issue the following commands:
/application/mysql//bin/mysqladmin -u root password 'new-password'
/application/mysql//bin/mysqladmin -u root -h oldboy password 'new-password'
Alternatively you can run:
/application/mysql//bin/mysql\_secure\_installation
which will also give you the option of removing the test
databases and anonymous user created by default.  This is
strongly recommended for production servers.
See the manual for more instructions.
You can start the MySQL daemon with:
cd . ; /application/mysql//bin/mysqld\_safe &
You can test the MySQL daemon with mysql-test-run.pl
cd mysql-test ; perl mysql-test-run.pl
Please report any problems at http://bugs.mysql.com/
The latest information about MySQL is available on the web at
http://www.mysql.com
Support MySQL by buying support/licenses at http://shop.mysql.com
\#\#\#请注意如下几行英文的说明
New default config file was created as /application/mysql//my.cnf and
will be used by default by the server when you start it.
You may edit this file to change server settings
\#从上述说明中可以知道MySQL的默认配置文件已经变到了/application/mysql//my.cnf下。
WARNING: Default config file /etc/my.cnf exists on the system
This file will be read by default by the MySQL server
If you do not want to use this, either remove it, or use the
--defaults-file argument to mysqld\_safe when starting the server
\#从上述说明中可以看到数据库启动时会读取/etc/my.cnf,因此有可能会导致无法启动,避免的方法就是使用mysqld\_safe 启动服务时采用“--defaults-file”参数指定配置文件,前文已将/etc/my.cnf改名了,并进行了新的配置文件替换,所以,就不需要指定参数了,这里是一个坑,请读者注意。
\end{verbatim}


\noindent\rule{\textwidth}{0.4pt}


\par 此步骤必须要初始化成功,否则,后面会出现登录不了数据库等各种问题。

\par (2)初始化数据库的原理及结果说明

\par 初始化数据库的实质就是创建基础的数据库系统的库文件,例如,生成mysql库表等。

\par 初始化数据库之后,查看数据目录,可以看到多了如下文件:

\noindent\rule{\textwidth}{0.4pt}


\begin{verbatim}[root@oldboy mysql-5.6.40]\# ls -l /application/mysql/data
total 110604
-rw-rw----. 1 mysql mysql 12582912 May 10 00:08 ibdata1
-rw-rw----. 1 mysql mysql 50331648 May 10 00:08 ib\_logfile0
-rw-rw----. 1 mysql mysql 50331648 May 10 00:08 ib\_logfile1
drwx------. 2 mysql mysql     4096 May 10 00:08 mysql  \#<==用于存放管理mysql
                                                   的数据。
drwx------. 2 mysql mysql     4096 May 10 00:08 performance\_schema
\#<==5.5及以上增加的内部性能库。
drwxr-xr-x. 2 mysql mysql     4096 May 10 00:48 test \#<==用于测试的test数据库。
[root@oldboy mysql-5.6.40]\# tree /application/mysql/data
\#<==如果没有tree,则可以采用yum install tree –y方式安装
/application/mysql/data
├── ibdata1
├── ib\_logfile0
├── ib\_logfile1
├── mysql
│   ├── columns\_priv.frm
│   ├── columns\_priv.MYD
│   ├── columns\_priv.MYI
…省略若干…
│   ├── general\_log.CSV
│   ├── general\_log.frm
│   ├── help\_category.frm
│   ├── procs\_priv.frm
│   ├── procs\_priv.MYD
│   ├── procs\_priv.MYI
│   ├── proxies\_priv.frm
│   ├── proxies\_priv.MYD
│   ├── proxies\_priv.MYI
│   ├── servers.frm
…省略若干…
│   ├── time\_zone\_name.MYI
│   ├── time\_zone\_transition.frm
│   ├── time\_zone\_transition.MYD
│   ├── time\_zone\_transition.MYI
│   ├── time\_zone\_transition\_type.frm
│   ├── time\_zone\_transition\_type.MYD
│   ├── time\_zone\_transition\_type.MYI
│   ├── user.frm
│   ├── user.MYD
│   └── user.MYI
├── performance\_schema
│   ├── accounts.frm
│   ├── cond\_instances.frm
│   ├── db.opt
│   ├── events\_stages\_current.frm
│   ├── events\_stages\_history.frm
│   ├── events\_stages\_history\_long.frm
│   ├── events\_stages\_summary\_by\_account\_by\_event\_name.frm
…省略若干…
│   ├── threads.frm
│   └── users.frm
└── test
    └── db.opt
3 directories, 136 files
\end{verbatim}


\noindent\rule{\textwidth}{0.4pt}


\par (3)MySQL初始化故障排错集锦

\par 本节的所有故障都必须要解除,即必须要出现两个Yes字样,否则,后面就算数据库服务可以启动,也会出现登录不了MySQL数据库等各种问题,因此读者若遇到了故障无法解决,则可以提前看一下本章结尾的故障集锦内容。


% From text00046.html


\subsection{3.2.4 配置并启动MySQL数据库}


\par 1)设置MySQL启动脚本:

\noindent\rule{\textwidth}{0.4pt}


\begin{verbatim}[root@oldboy mysql-5.6.40]\# pwd
/home/oldboy/tools/mysql-5.6.40
[root@oldboy mysql-5.6.40]\# cp support-files/mysql.server /etc/init.d/mysqld
\#<==将mysql启动脚本复制到mysql的命令路径。
[root@oldboy mysql-5.6.40]\# chmod 700 /etc/init.d/mysqld  \#<==使脚本可执行,注意权限要小一些。
[root@oldboy mysql-5.6.40]\# ls -l /etc/init.d/mysqld
-rwx------. 1 root root 10619 May 10 00:20/etc/init.d/mysqld
\#提示:老男孩开发了一个专业、规范,且相对简单的Shell版MySQL服务启动脚本,地址如下:
http://oldboy.blog.51cto.com/2561410/1945183
\end{verbatim}


\noindent\rule{\textwidth}{0.4pt}


\par 2)启动MySQL数据库:

\noindent\rule{\textwidth}{0.4pt}


\begin{verbatim}\#<==这是启动数据库规范方法之一,还可以使用” /application/mysql/bin/mysqld\_safe
    --user=mysql &”启动。这个命令结尾 “&”符号的作用是,在后台执行mysql服务,这条
    命令执行完 还需要按下回车才能进入到命令行状态。
\#<==注意,如果已经执行了上面的/etc/init.d/mysqld start启动命令,还想尝试下面mysqld\_
     safe的命令,请先执行/etc/init.d/mysqld stop结束mysql进程。
[root@oldboy mysql-5.6.40]\# /etc/init.d/mysqld start
Starting MySQL. SUCCESS!
\end{verbatim}


\noindent\rule{\textwidth}{0.4pt}


\par 特别提示:这里有一个大坑,就是数据库可能会无法启动,报错信息如下:

\noindent\rule{\textwidth}{0.4pt}


\begin{verbatim}[root@oldboy mysql-5.6.40]\# /etc/init.d/mysqld start
Starting MySQL.Logging to '/application/mysql-5.6.40/data/oldboy.err'.
180510 00:21:04 mysqld\_safe Directory '/application/mysql-5.6.40/tmp' for UNIX socket file don't exists.
 ERROR! The server quit without updating PID file (/application/mysql-5.6.40/data/oldboy.pid).
提示:/application/mysql-5.6.40/tmp目录不存在,5.6.34前的早期版本没事。
\end{verbatim}


\noindent\rule{\textwidth}{0.4pt}


\par 解决办法:

\noindent\rule{\textwidth}{0.4pt}


\begin{verbatim}[root@oldboy mysql-5.6.40]\# mkdir -p /application/mysql-5.6.40/tmp
\#<==编译时指定的socket路径。
[root@oldboy mysql-5.6.40]\# chown -R mysql.mysql /application/mysql/
[root@oldboy mysql-5.6.40]\# /etc/init.d/mysqld start
Starting MySQL. SUCCESS!
\end{verbatim}


\noindent\rule{\textwidth}{0.4pt}


\par 
\begin{figure}[htbp]
\centering
\includegraphics[width=0.8\textwidth]{Image00006.jpg}
\end{figure}

 提示:
 禁止使用kill-9、killall-9等命令强制杀死数据库,这会引起数据库无法启动等故障发生。企业中曾发生过的血的教训案例请看\href{http://oldboy.blog.51cto.com/2561410/1431161}{http://oldboy.blog.51cto.com/2561410/1431161}
 。

\par 3)检查MySQL数据库是否启动:

\noindent\rule{\textwidth}{0.4pt}


\begin{verbatim}[root@oldboy \textasciitilde{}]\# netstat -lntup|grep mysql
tcp        0      0 :::3306           :::*         LISTEN    48065/mysqld
\end{verbatim}


\noindent\rule{\textwidth}{0.4pt}


\par 如果发现3306端口没起来,则请使用tail-100/application/mysql/data/机器名.err检查日志报错进行调试。经常查看服务运行日志是个很好的习惯,也是高手的习惯。

\par 4)查看MySQL数据库启动结果日志:

\noindent\rule{\textwidth}{0.4pt}


\begin{verbatim}[root@oldboy \textasciitilde{}]\# tail /application/mysql/data/oldboy.err
2018-05-10 17:38:36 48065 [Note] InnoDB: Waiting for purge to start
2018-05-10 17:38:36 48065 [Note] InnoDB: 5.6.40 started; log sequence number 1625987
2018-05-10 17:38:36 48065 [Warning] No existing UUID has been found, so we assume that this is the first time that this server has been started. Generating a new UUID: 4f94404a-fc74-11e6-8112-000c292ece3f.
2018-05-10 17:38:36 48065 [Note] Server hostname (bind-address): '*'; port: 3306
2018-05-10 17:38:36 48065 [Note] IPv6 is available.
2018-05-10 17:38:36 48065 [Note]   - '::' resolves to '::';
2018-05-10 17:38:36 48065 [Note] Server socket created on IP: '::'.
2018-05-10 17:38:36 48065 [Note] Event Scheduler: Loaded 0 events
2018-05-10 17:38:36 48065 [Note] /application/mysql-5.6.40/bin/mysqld: ready for connections.
Version: '5.6.34'  socket: '/application/mysql-5.6.40/tmp/mysql.sock'  port: 3306  Source distribution
\end{verbatim}


\noindent\rule{\textwidth}{0.4pt}


\par 本例查看了错误日志的命令及错误日志中的内容,这里省略了大部分日志内容,只给出了默认的10行,如果有错误,一般会显示error字样。

\par 5)设置MySQL开机自启动:

\noindent\rule{\textwidth}{0.4pt}


\begin{verbatim}[root@oldboy \textasciitilde{}]\# chkconfig --add mysqld
[root@oldboy \textasciitilde{}]\# chkconfig --list mysqld
mysqld               0:off       1:off      2:on      3:on      4:on      5:on      6:off
\end{verbatim}


\noindent\rule{\textwidth}{0.4pt}


\par 此外,将启动命令/etc/init.d/mysqld start放到/etc/rc.local里面也可以实现开机自启动。

\par 若MySQL安装及使用出现故障,则可根据下面的分析思路进行检查。

\par ·细看所有步骤执行命令返回的屏幕输出,不要忽略关键的输出内容。

\par ·查看mysql错误日志/application/mysql/data/机器名.err。

\par ·辅助查看系统日志/var/log/messages。

\par ·如果是MySQL关联了其他服务,则还要同时查看相关服务的日志。

\par ·仔细阅读,重新查看所有操作的步骤是否正确,书写的命令及字符是否都正确。

\par 经常查看各种服务运行日志是个很好的习惯,也是成长为高手的必经之路!


% From text00047.html


\subsection{3.2.5 将MySQL相关命令加入全局路径}


\par 如果不为MySQL的命令配置全局路径,就会无法在命令行直接输入mysql这样的客户端命令,只能使用全路径命令(/application/mysql/bin/mysql),这种附带路径输入命令的方式很麻烦。下面来看看配置的具体方法。

\par 1)确认mysql命令所在的路径:

\noindent\rule{\textwidth}{0.4pt}


\begin{verbatim}[root@oldboy /]\# ls /application/mysql/bin/mysql
/application/mysql/bin/mysql
\end{verbatim}


\noindent\rule{\textwidth}{0.4pt}


\par 2)在PATH变量前面增加/application/mysql/bin路径,并追加到/etc/profile文件中:

\noindent\rule{\textwidth}{0.4pt}


\begin{verbatim}[root@oldboy /]\# echo 'export PATH=/application/mysql/bin:\$PATH' >>/etc/profile
\#<==注意,echo后是单引号,双引号是不行的。
[root@oldboy /]\# tail -1 /etc/profile
export PATH=/application/mysql/bin:\$PATH \#<==放在前面也是有目的的,是为了防止执行
                                          时使用老版本的mysql命令。
[root@oldboy /]\# source /etc/profile
\#<==执行source使上一行添加到/etc/profile中,内容直接生效。
\#<==以上命令的用途为,定义mysql全局路径,实现在任意路径执行mysql命令。
[root@oldboy \textasciitilde{}]\# echo \$PATH
/application/mysql/bin:/usr/local/sbin:/usr/local/bin:/sbin:/bin:/usr/sbin:
/usr/bin:/root/bin
\#<==执行echo \$PATH,若有/application/mysql/bin输出则表示配置成功。
\end{verbatim}


\noindent\rule{\textwidth}{0.4pt}


\par 
\begin{figure}[htbp]
\centering
\includegraphics[width=0.8\textwidth]{Image00006.jpg}
\end{figure}

 提示:
 更简单的设置方法为使用下面的命令做软链接:ln-s/application/mysql/bin/*/usr/local/sbin/,把mysql命令所在的路径链接到全局路径/usr/local/sbin/的下面。

\par 要特别强调的是,务必把MySQL命令路径放在PATH路径中其他路径的前面,否则,可能会导致使用的mysql客户端等命令和本章编译安装的mysql服务对应的mysql命令不是同一个,进而产生错误。下面的网址中给出了因为MySQL路径配置问题而导致的错误案例:\href{http://oldboy.blog.51cto.com/2561410/1122867}{http://oldboy.blog.51cto.com/2561410/1122867}
 ,该错误实际上就是因为使用yum安装的MySQL客户端命令访问了编译安装的服务端而导致的。


% From text00048.html


\subsection{3.2.6 登录MySQL测试}


\par 登录并测试的命令如下:

\noindent\rule{\textwidth}{0.4pt}


\begin{verbatim}[root@oldboy \textasciitilde{}]\# mysql \#<==直接输入mysql就可以进入数据库了,而且身份还是root。
Welcome to the MySQL monitor.  Commands end with ; or \textbackslash\{\}g.
Your MySQL connection id is 1
Server version: 5.6.40 Source distribution
Copyright (c) 2000, 2018, Oracle and/or its affiliates. All rights reserved.
Oracle is a registered trademark of Oracle Corporation and/or its
affiliates. Other names may be trademarks of their respective
owners.
Type 'help;' or '\textbackslash\{\}h' for help. Type '\textbackslash\{\}c' to clear the current input statement.
mysql>
提示:你还可以使用如下三种命令写法登录mysql。
mysql -uroot -p
mysql -uroot
mysql -uroot -p '密码'
\end{verbatim}


\noindent\rule{\textwidth}{0.4pt}


\par 若出现登录故障,则可以通过以下方法排查。

\par 如果这里提示无法登录,排除了数据库启动问题之后,则很可能是数据库初始化文件有问题。例如,执行mysql后提示如下错误:

\noindent\rule{\textwidth}{0.4pt}


\begin{verbatim}[root@oldboy \textasciitilde{}]\# mysql
ERROR 1045 (28000): Access denied for user 'root'@'localhost' (using password: NO)
\end{verbatim}


\noindent\rule{\textwidth}{0.4pt}


\par 解决办法:重新初始化数据库即可,此问题一般都是数据库初始化问题或者数据库文件损坏以及目录权限问题。

\noindent\rule{\textwidth}{0.4pt}


\begin{verbatim}mysql> show databases;  \#<==查看当前的数据库
+--------------------+
| Database           |
+--------------------+
| information\_schema |
| mysql              |
| performance\_schema |
| test               |
+--------------------+
4 rows in set (0.00 sec)

mysql> select user();  \#<==查看当前的登录用户
+----------------+
| user()         |
+----------------+
| root@localhost |
+----------------+
1 row in set (0.00 sec)

mysql>   \#<==快捷键ctrl+d,也可以使用quit或exit等。
\end{verbatim}


\noindent\rule{\textwidth}{0.4pt}


\par MySQL安装完成后,默认情况下,管理员账号root是无密码的,极不安全,必须要处理一下。


% From text00049.html


\subsection{3.2.7 基本的MySQL安全配置}


\par 1.为root用户设置密码

\par MySQL管理员的账号root密码默认为空,极不安全,可以通过mysqladmin命令为mysql不同实例的数据库设置独立的密码:

\noindent\rule{\textwidth}{0.4pt}


\begin{verbatim}[root@oldboy \textasciitilde{}]\# mysqladmin -u root password 'oldboy123'
\#<==为root用户设置密码oldboy123。
Warning: Using a password on the command line interface can be insecure.
\#<==这里是一个警告,提醒用户命令行放置密码是不安全的,请读者尽量不要在命令行输入密码,
    而是采取交互式的方式输入密码。
[root@oldboy \textasciitilde{}]\# mysql \#<==无法直接输入命令登录了。
ERROR 1045 (28000): Access denied for user 'root'@'localhost' (using password: NO)
[root@oldboy \textasciitilde{}]\# mysql -uroot -p \#<==新的登录方式,也可以直接带密码mysql -uroot
                                 -poldboy123。
Enter password:                  \#<==输入新密码oldboy123,交互式输入密码不会记录
                                   在命令记录里,因此更安全。
Welcome to the MySQL monitor.  Commands end with ; or \textbackslash\{\}g.
Your MySQL connection id is 5
Server version: 5.6.40 Source distribution
Copyright (c) 2000, 2018, Oracle and/or its affiliates. All rights reserved.
Oracle is a registered trademark of Oracle Corporation and/or its
affiliates. Other names may be trademarks of their respective
owners.
Type 'help;' or '\textbackslash\{\}h' for help. Type '\textbackslash\{\}c' to clear the current input statement.
mysql>
\end{verbatim}


\noindent\rule{\textwidth}{0.4pt}


\par 读者也可以执行mysql\_secure\_installation命令交互式地设置系统的用户密码。

\par 2.清理mysql服务器内无用的用户

\noindent\rule{\textwidth}{0.4pt}


\begin{verbatim}mysql> select user,host from mysql.user;
+------+-----------+
| user | host      |
+------+-----------+
| root | 127.0.0.1 |
| root | ::1       |
|      | localhost |
| root | localhost |
|      | oldboy    |
| root | oldboy    |
+------+-----------+
6 rows in set (0.00 sec)

mysql> drop user root@'::1';
Query OK, 0 rows affected (0.00 sec)

mysql> drop user root@'oldboy';
Query OK, 0 rows affected (0.00 sec)

mysql> drop user ''@'oldboy';
Query OK, 0 rows affected (0.00 sec)

mysql> drop user ''@'localhost';
Query OK, 0 rows affected (0.00 sec)

mysql> select user,host from mysql.user;
+------+-----------+
| user | host      |
+------+-----------+
| root | 127.0.0.1 |
| root | localhost |
+------+-----------+
2 rows in set (0.00 sec)
\end{verbatim}


\noindent\rule{\textwidth}{0.4pt}


\par 
\begin{figure}[htbp]
\centering
\includegraphics[width=0.8\textwidth]{Image00006.jpg}
\end{figure}

 提示:
 对于drop命令及数据库安全,后文会有详细讲解,此处不执行也可以。

\par 有时使用drop命令可能会无法删除对应用户。比如,当数据库内的host等字段为大写及特殊Linux主机名时,删除用户就会遇到问题,例如:

\noindent\rule{\textwidth}{0.4pt}


\begin{verbatim}mysql> drop user ' '@'MySQL';
ERROR 1396 (HY000): Operation DROP USER failed for ' '@'mysql'
\end{verbatim}


\noindent\rule{\textwidth}{0.4pt}


\par 可使用DML语句,并采用delete命令删除来解决此问题,具体命令如下:

\noindent\rule{\textwidth}{0.4pt}


\begin{verbatim}mysql> delete from mysql.user where user='' and host='MySQL';
Query OK, 1 row affected (0.00 sec)
mysql> flush privileges;
Query OK, 0 rows affected (0.00 sec)
\end{verbatim}


\noindent\rule{\textwidth}{0.4pt}


\par 3.删除mysql数据库内无用的test库

\noindent\rule{\textwidth}{0.4pt}


\begin{verbatim}mysql> drop database test;
Query OK, 0 rows affected (0.00 sec)
mysql> show databases;
+--------------------+
| Database           |
+--------------------+
| information\_schema |
| mysql              |
| performance\_schema |
+--------------------+
3 rows in set (0.00 sec)
\end{verbatim}


\noindent\rule{\textwidth}{0.4pt}


\par 有关MySQL的更多安全措施,在后文会有更详细的讲解。


% From text00050.html


\section{3.3 MySQL安装FAQ}


\par 
问题1:在配置mysql时遇到错误。


\par 出现的错误如下:

\noindent\rule{\textwidth}{0.4pt}


\begin{verbatim}checking for tgetent in -ltinfo... no
checking for termcap functions library... configure: error: No curses/termcap library found
\end{verbatim}


\noindent\rule{\textwidth}{0.4pt}


\par 原因:缺少ncurses安装包。

\par 解决方法:执行“yum-y install ncurses-devel”命令。

\par 
问题2:初始化MySQL数据库时出现故障。


\par ·故障1:给出了警告信息“WARNING:The host'oldboy'could not be looked up with resolveip”。

\par 警告是可以忽略的,如果非要解决则需要修改对主机名的解析,使其和“uname-n”命令的结果一样,如下:

\noindent\rule{\textwidth}{0.4pt}


\begin{verbatim}[root@oldboy \textasciitilde{}]\# grep "`uname -n`" /etc/hosts
127.0.0.1  oldboy \#<==也可以使用网卡IP解析。
\end{verbatim}


\noindent\rule{\textwidth}{0.4pt}


\par ·故障2:错误提示“ERROR:1004 Can't create file'/tmp/\#sql300e\_1\_0.frm'(errno:13)”。

\par 在执行初始化数据库命令时可能就会遇到这样的错误,错误提示如下:

\noindent\rule{\textwidth}{0.4pt}


\begin{verbatim}ERROR: 1004  Can't create file '/tmp/\#sql300e\_1\_0.frm' (errno: 13)
120406 15:47:02 [ERROR] Aborting
120406 15:47:02 [Note] /application/mysql/libexec/mysqld: Shutdown complete
Installation of system tables failed!  Examine the logs in
/application/mysql/data for more information.
\end{verbatim}


\noindent\rule{\textwidth}{0.4pt}


\par 根据提示可知,/tmp目录不能创建文件,所以解决办法是对/tmp目录增加权限,具体如下。

\par 解决办法:还原/tmp目录权限。操作命令如下:

\noindent\rule{\textwidth}{0.4pt}


\begin{verbatim}[root@oldboy \textasciitilde{}]\# chmod 1777 /tmp      \#<==1777是/tmp的默认权限,除非曾经改动
                                        过此目录权限才会报错。
[root@oldboy \textasciitilde{}]\# ls -ld /tmp/
drwsrwxrwt. 8 root root 4096 3月  10 18:07 /tmp/
\end{verbatim}


\noindent\rule{\textwidth}{0.4pt}


\par 本示例的故障必须要解除,否则,后面会出现登录不了MySQL数据库等各种问题。

\par 
问题3:登录数据库时提示“Access denied for user'root'@'localhost'”。


\par 解答:正文里已经给出答案,此处不再赘述。

\par 
问题4:有时使用drop命令删除MySQL用户删不掉。


\par 解答:正文里已经给出答案,此处不再赘述。


% From text00051.html


\section{3.4 MySQL 5.6编译常见参数选项说明}


\par MySQL 5.6编译常见参数选项说明如表3-2所示。

\par 表3-2 MySQL 5.6编译参数及说明

\begin{figure}[htbp]
\centering
\includegraphics[width=0.8\textwidth]{Image00021.jpg}
\end{figure}



\begin{figure}[htbp]
\centering
\includegraphics[width=0.8\textwidth]{Image00022.jpg}
\end{figure}



\par 更多内容请参考官方MySQL 5.6的cmake编译参数:\href{http://dev.mysql.com/doc/refman/5.6/en/source-configuration-options.html}{http://dev.mysql.com/doc/refman/5.6/en/source-configuration-options.html}



% From text00052.html


\section{3.5 章节试题}


\par 1)MySQL的常见安装方法有哪些?

\par 2)在企业中如何选择不同的MySQL安装方法?

\par 3)请描述如何安装及配置MySQL服务?

\par 4)安装及配置MySQL服务遇到故障如何排错?

\par 5)MySQL安装后可以做哪些基础优化?


% From text00053.html


\chapter{第4章 MySQL多实例数据库企业级应用实践}


\section{
4.1 MySQL多实例介绍}
}


\par 前面3章已经针对MySQL数据库进行了介绍,并说明了为什么选择MySQL数据库,以及MySQL数据库在Linux系统下的多种安装方式,同时以单实例讲解了如何以编译方式安装MySQL和基础安全优化等内容,本章将为大家讲解更为实用的MySQL多实例安装,百度、淘宝、阿里、新浪等大公司无一例外地都会使用多实例的方式部署数据库,那么是什么原因促使他们选择多实例数据库的部署方式呢?请看下文。


% From text00054.html


\subsection{4.1.1 什么是MySQL多实例}


\par 根据老男孩的经验,简单地说,MySQL多实例就是在一台服务器上同时开启多个不同的服务端口(如3306、3307),同时运行多个MySQL服务进程,这些服务进程通过不同的socket监听不同的服务端口来提供服务。

\par 这些MySQL多实例共用一套MySQL安装程序,使用不同的my.cnf配置文件、启动程序(也可以相同)和数据文件。在提供服务时,多实例MySQL在逻辑上看起来是各自独立的,它们根据配置文件的对应设定值来获得服务器相应数量的硬件资源。

\par 打个比方吧,MySQL多实例就相当于房子的多个卧室,每个实例可以看作是一间卧室,整个服务器就是一套房子,服务器的硬件资源(cpu、mem、disk)、软件资源(CentOS操作系统)可以看作是房子的卫生间、厨房、客厅,是房子的公用资源。若你是北漂的小伙伴,与朋友一起租房,相信对此能有更好地理解,大家蜗居在一起,休息时在自己的卧室,但出来活动肯定是要共用上述公共资源的。图4-1是MySQL多实例形象示意图。

\begin{figure}[htbp]
\centering
\includegraphics[width=0.8\textwidth]{Image00023.jpg}
\end{figure}



\par 图4-1 MySQL多实例形象示意图

\par 其实很多网络服务都是可以配置多实例的,例如Nginx、Apache、Haproxy、redis、Memcahed等都可以。而且这在门户网站中使用得很广泛,当然,随着虚拟化、云计算的产生和发展,也会通过不同的虚拟机切割资源配置独立的服务,但虚拟化存在一定的缺陷,例如,相比于物理机,其性能会有所下降等。


% From text00055.html


\subsection{4.1.2 MySQL多实例的作用与问题}


\par 本节首先来看看MySQL多实例的作用,具体如下。

\par ·可有效利用服务器资源。当单个服务器资源有剩余时,可以充分利用剩余的资源提供更多的服务,且可以实现资源的逻辑隔离。

\par ·节约服务器资源。若公司资金紧张,但是数据库又需要各自尽量独立地提供服务,而且还需要用到主从复制等技术,那么选择多实例就再好不过了。

\par MySQL多实例有它的好处,也有其弊端,比如,会存在资源互相抢占的问题。当某个数据库实例并发很高或者有SQL慢查询时,整个实例会消耗大量的系统CPU、磁盘I/O等资源,导致服务器上的其他数据库实例提供服务的质量一起下降。这就相当于大家住在一个房子的不同卧室中,早晨起来上班,都要刷牙、洗脸等,这样卫生间就会长期处于占用状态,其他人则必须要等待。不同实例获取的资源是相对独立的,无法像虚拟化一样完全隔离。


% From text00056.html


\section{4.2 MySQL多实例的生产应用场景}


\subsection{
4.2.1 资金紧张型公司的选择}
}


\par 若公司资金紧张,公司业务访问量不太大,但又希望不同业务的数据库服务各自能够尽量独立地提供服务而互相不受影响,或者,还有需要主从复制等技术提供备份或读写分离服务的需求,那么,多实例就再好不过了。比如:可以通过3台服务器部署9~15个实例,交叉做主从复制、数据备份及读写分离,这样就可以达到9~15台服务器每个只装一个数据库才有的效果。这里需要强调的是,所谓的尽量独立是相对的。


% From text00057.html


\subsection{4.2.2 并发访问不是特别大的业务}


\par 当公司业务访问量不太大的时候,服务器的资源基本上都是浪费的,这时就很适合多实例的应用,如果对SQL语句的优化做得比较好,MySQL多实例会是一个很值得使用的技术,即使并发很大,合理分配好系统资源以及搭配好服务,也不会有太大的问题。


% From text00058.html


\subsection{4.2.3 门户网站应用MySQL多实例场景}


\par 门户网站通常都会使用多实例,因为配置硬件好的服务器,可以节省IDC机柜空间,同时,运行多实例也会减少硬件资源占用率不满的浪费。比如,百度公司的很多数据库都是多实例的,不过,一般是从库采用多实例,例如某部门中使用的IBM服务器为48核CPU,内存96GB,一台服务器运行3~4个实例;此外,新浪网也有采用多实例的情况,内存48GB左右。

\par 
\begin{figure}[htbp]
\centering
\includegraphics[width=0.8\textwidth]{Image00024.jpg}
\end{figure}

 说明:
 据老男孩调查,新浪网的数据库以单机1~4个数据库实例的居多。其中又数1~2个的最多,这是因为大业务占用的机器比较多。在该公司,服务器以DELL R510居多,CPU是E5210、内存48GB、磁盘为12*300G SAS、采用的是RAID10。可将此作为门户网站的服务器配置参考,希望能给读者购买服务器及部署实例提供一点数据帮助。

\par 另外,新浪网站安装数据库时,一般采用编译安装的方式,并且会在进行优化之后做成rpm包,以便统一使用
\href{text00058.html#ch1_back}{[1]}

 。

\par 
\href{text00058.html#ch1}{[1]}
 老男孩面授课程会讲rpm包定制以及yum仓库搭建技术。读者也可以参考老男孩内部的360云盘分享:http://yunpan.cn/cjJ4rWaT8bVRS,提取码为1e89,里面有介绍rpm包制作的内部免费视频与yum仓库搭建速成或相关打包知识。


% From text00059.html


\section{4.3 MySQL多实例常见的配置方案}


\subsection{
4.3.1 单一配置文件、单一启动程序多实例部署方案}
}


\par 下面是MySQL官方文档提到的单一配置文件、单一启动程序多实例部署方案,老男孩老师不推荐此方案,这里仅作为知识点提及,后文不会再涉及此方案。my.cnf配置文件示例(MySQL手册里提到的方法)如下:

\noindent\rule{\textwidth}{0.4pt}


\begin{verbatim}[mysqld\_multi]
mysqld     = /usr/bin/mysqld\_safe
mysqladmin = /usr/bin/mysqladmin
user       = mysql
[mysqld1]
socket     = /var/lib/mysql/mysql.sock
port       = 3306
pid-file   = /var/lib/mysql/mysql.pid
datadir    = /var/lib/mysql/
user       = mysql
[mysqld2]
socket     = /mnt/data/db1/mysql.sock
port       = 3302
pid-file   = /mnt/data/db1/mysql.pid
datadir    = /mnt/data/db1/
user       = mysql
skip-name-resolve
server-id=10
default-storage-engine=innodb
innodb\_buffer\_pool\_size=512M
innodb\_additional\_mem\_pool=10M
default\_character\_set=utf8
character\_set\_server=utf8
\#read-only
relay-log-space-limit=3G
expire\_logs\_day=20
\end{verbatim}


\noindent\rule{\textwidth}{0.4pt}


\par 启动程序的命令如下:

\noindent\rule{\textwidth}{0.4pt}


\begin{verbatim}mysqld\_multi --config-file=/data/mysql/my\_multi.cnf start 1,2
\end{verbatim}


\noindent\rule{\textwidth}{0.4pt}


\par 该方案的缺点是耦合度太高,只有一个配置文件,不好管理。工作开发和运维的统一原则是:降低耦合度。


% From text00060.html


\subsection{4.3.2 多配置文件、多启动程序部署方案}


\par 多配置文件、多启动程序部署方案,是本文主要讲解的方案,也是老男孩老师常用并极力推荐的方案。下面来看一下配置示例。

\par 以下是老男孩已经部署好的MySQL双实例的目录信息及文件注释说明:

\noindent\rule{\textwidth}{0.4pt}


\begin{verbatim}[root@oldboy \textasciitilde{}]\# tree /data
/data
|-- 3306
|   |-- data   \#<==3306实例的数据文件。
|   |-- my.cnf \#<==3306实例的配置文件。
|   `-- mysql  \#<==3306实例的启动文件。
`-- 3307
    |-- data   \#<==3307实例的数据文件。
    |-- my.cnf \#<==3307实例的配置文件。
    `-- mysql  \#<==3307实例的启动文件。
4 directories, 4 files
\end{verbatim}


\noindent\rule{\textwidth}{0.4pt}


\par 
\begin{figure}[htbp]
\centering
\includegraphics[width=0.8\textwidth]{Image00006.jpg}
\end{figure}

 提示:
 这里的配置文件my.cnf、启动程序mysql都是独立的文件,数据文件data目录也是独立的。

\par 多实例MySQL数据库的安装和前文讲解的单实例没有任何区别,因此,读者如果有前文单实例的安装环境,那么可以直接略过4.4.1节的内容。


% From text00061.html


\section{4.4 安装并配置多实例MySQL数据库}


\subsection{
4.4.1 安装MySQL多实例}
}


\par 1.安装MySQL需要的依赖包和编译软件

\par (1)安装MySQL需要的依赖包

\par 安装MySQL之前,最好先安装MySQL需要的依赖包,不然后面会出现很多报错信息,到那时再回来安装MySQL的依赖包会很麻烦。安装命令如下:

\noindent\rule{\textwidth}{0.4pt}


\begin{verbatim}[root@oldboy \textasciitilde{}]\# yum install ncurses-devel libaio-devel -y
[root@oldboy \textasciitilde{}]\# rpm -qa ncurses-devel libaio-devel
libaio-devel-0.3.107-10.el6.x86\_64
ncurses-devel-5.7-4.20090207.el6.x86\_64
\end{verbatim}


\noindent\rule{\textwidth}{0.4pt}


\par 
\begin{figure}[htbp]
\centering
\includegraphics[width=0.8\textwidth]{Image00006.jpg}
\end{figure}

 提示:
 安装后检查,如果出现两行信息则表示安装成功。

\par (2)安装编译MySQL需要的软件

\par 由于MySQL 5.5及以上系列产品采用了特殊的编译方式安装数据库,因此,需要安装常用的编译MySQL的工具cmake软件包,命令如下:

\noindent\rule{\textwidth}{0.4pt}


\begin{verbatim}[root@oldboy \textasciitilde{}]\# yum install cmake -y
[root@oldboy \textasciitilde{}]\# rpm -qa cmake
cmake-2.8.12.2-4.el6.x86\_64
\end{verbatim}


\noindent\rule{\textwidth}{0.4pt}


\par 
\begin{figure}[htbp]
\centering
\includegraphics[width=0.8\textwidth]{Image00006.jpg}
\end{figure}

 提示:
 安装后使用“rpm-qa cmake”检查,如果出现一行如上信息则表示安装成功。此外,网上有资料选用源码包的方式安装cmake,比较复杂,因此一般建议读者选择这种简单的yum安装cmake的方法。

\par 2.开始安装MySQL

\par 为了让大家学习更多的MySQL技术,这里选择以相对复杂的源代码安装为例来讲解MySQL多实例的安装。大型公司一般都会将MySQL软件定制成rpm包,然后放到yum仓库里,使用yum安装,在中小型企业里,二进制和编译安装方式的区别不大。

\par (1)建立MySQL用户账号

\par 首先以root身份登录到Linux系统中,然后执行如下命令创建mysql用户账号:

\noindent\rule{\textwidth}{0.4pt}


\begin{verbatim}[root@oldboy \textasciitilde{}]\# useradd -s /sbin/nologin -M mysql \#<==默认会创建和mysql用户
                                                 同名的组。
[root@oldboy \textasciitilde{}]\# id mysql
uid=500(mysql) gid=500(mysql) 组=500(mysql)
\end{verbatim}


\noindent\rule{\textwidth}{0.4pt}


\par 根据上述结果输出,可以看到mysql用户和组已经成功创建。

\par (2)获取MySQL软件包

\par MySQL软件包的下载地址为:\href{http://dev.mysql.com/downloads/mysql/}{http://dev.mysql.com/downloads/mysql/}
 (如果地址变更无法下载,则去\href{http://mirrors.sohu.com/mysql}{http://mirrors.sohu.com/mysql}
 上下载,或者通过老男孩在本文开头给出的云盘地址下载)。可以把软件下载到客户端电脑本地后使用rz等工具传到Linux里,或者找到网络下载地址后直接在Linux里使用wget下载。

\par 
\begin{figure}[htbp]
\centering
\includegraphics[width=0.8\textwidth]{Image00006.jpg}
\end{figure}

 提示:
 本例是以MySQL编译的方式来讲解。在生产场景中,二进制和源码包两种安装方法都是可用的,其应用场景一般没有什么太大的差别。不同之处在于,二进制的安装包较大,二进制的安装过程比源码更快,此外在名字和源码包方面也有些区别。

\par (3)采用编译方式安装MySQL

\par 本节所讲的安装过程与3.2.1节“安装MySQL数据库”一节的内容一模一样,读者可以去第3章查看。


% From text00062.html


\subsection{4.4.2 创建MySQL多实例的数据文件目录}


\par 在老男孩的企业中,通常以“/data”目录作为MySQL多实例总的根目录,然后规划不同的数字(即MySQL实例端口号)作为“/data”下面的二级目录;不同的二级目录对应的数字就作为MySQL实例的端口号,以区别不同的实例;数字对应的二级目录下包含MySQL的数据文件、配置文件以及启动文件等。

\par 下面以配置3306、3307两个实例为例进行讲解。创建MySQL多实例的目录如下:

\noindent\rule{\textwidth}{0.4pt}


\begin{verbatim}[root@oldboy \textasciitilde{}]\# mkdir -p /data/\{3306,3307\}/data
[root@oldboy \textasciitilde{}]\# tree /data/
/data/
|-- 3306       \#<==3306实例的目录。
|   `-- data   \#<==3306实例的数据文件目录。
`-- 3307       \#<==3307实例的目录。
    `-- data   \#<==3307实例的数据文件目录。
4 directories, 0 files
\end{verbatim}


\noindent\rule{\textwidth}{0.4pt}


\par 
\begin{figure}[htbp]
\centering
\includegraphics[width=0.8\textwidth]{Image00006.jpg}
\end{figure}

 提示:


\par 1)“mkdir-p/data/\{3306,3307\}/data”相当于“mkdir-p/data/3306/data;mkdir-p/data/3307/data”的两条命令。

\par 2)如果是创建多个目录,则可以增加如3308、3309这样的目录名,在生产环境中,一般以2\textasciitilde{}4个实例为佳。


% From text00063.html


\subsection{4.4.3 创建MySQL多实例的配置文件}


\par MySQL 5.5及以下数据库默认为用户提供了多个配置文件模板,但是MySQL 5.6版本只有一个模板文件。示例如下:

\noindent\rule{\textwidth}{0.4pt}


\begin{verbatim}[root@oldboy mysql-5.6.40]\# ll support-files/*.cnf
-rw-r--r--. 1 root root 1126 Feb 26 17:54 support-files/my-default.cnf
[root@oldboy mysql-5.6.34]\# mv /etc/my.cnf /etc/my.cnf.bak
\end{verbatim}


\noindent\rule{\textwidth}{0.4pt}


\par 
\begin{figure}[htbp]
\centering
\includegraphics[width=0.8\textwidth]{Image00019.jpg}
\end{figure}

 注意:
 在完成CentOS 6.9版操作系统的最小安装后,在/etc目录下会存在一个my.cnf,需要将此文件更名为其他的名字,如/etc/my.cnf.bak,否则,该文件会干扰源码安装的MySQL的正确配置,造成无法启动。

\par 在启动MySQL服务时,会按照一定的顺序搜索my.cnf,先在/etc目录下查找,若找不到则会搜索"\$basedir/my.cnf",在本例中就是/application/mysql-5.6.40/my.cnf,这是新版MySQL的配置文件的默认位置!

\par 上面是单实例的默认配置文件模板,如果配置多实例,则其与单实例会有所不同。为了让MySQL多实例之间彼此独立,需要为每一个实例建立一个my.cnf配置文件和一个启动文件mysql,让它们分别对应自己的数据文件目录data。

\par 首先,通过vim命令来添加配置文件内容,命令如下:

\noindent\rule{\textwidth}{0.4pt}


\begin{verbatim}vim /data/3306/my.cnf
vim /data/3307/my.cnf
\end{verbatim}


\noindent\rule{\textwidth}{0.4pt}


\par 不同的实例需要添加的my.cnf内容也会有区别,具体见表4-1,其中的配置是由官方的配置模板修改而来的,当然,在实际工作中则是拿早已配置好的模板来进行修改的,可以通过rz等方式上传配置文件模板my.cnf文件到相关目录下。

\par 表4-1 MySQL 3306、3307实例配置文件对比

\begin{figure}[htbp]
\centering
\includegraphics[width=0.8\textwidth]{Image00025.jpg}
\end{figure}



\par 
\begin{figure}[htbp]
\centering
\includegraphics[width=0.8\textwidth]{Image00006.jpg}
\end{figure}

 提示:
 本多实例的很多参数仅仅是方便学习环境之用,生产环境的配置可以参见本书后文,文中两个实例参数的区别已经标出。

\par 最终完成后的多实例根/data目录结果如下:

\noindent\rule{\textwidth}{0.4pt}


\begin{verbatim}[root@oldboy \textasciitilde{}]\# tree /data
/data
|-- 3306
|   |-- data
|   `-- my.cnf   \#<==这个就是3306实例的配置文件。
`-- 3307
    |-- data
    `-- my.cnf   \#<==这个就是3307实例的配置文件。
4 directories, 2 files
\end{verbatim}


\noindent\rule{\textwidth}{0.4pt}


\par 有关配置文件的参数说明,可以查看官方手册或者查找相关资料。


% From text00064.html


\subsection{4.4.4 创建MySQL多实例的启动文件}


\par MySQL多实例启动文件的创建和配置文件几乎一样,也可以通过vim命令来添加,示例如下:

\noindent\rule{\textwidth}{0.4pt}


\begin{verbatim}vim /data/3306/mysql
vim /data/3307/mysql
\end{verbatim}


\noindent\rule{\textwidth}{0.4pt}


\par 需要添加的MySQL启动文件内容见表4-2。当然,在实际工作中是拿早已配置好的模板来进行修改的,可以通过rz等方式上传配置文件模板mysql文件到相关目录下。

\par 表4-2 3306与3307 mysql实例启动文件对比

\begin{figure}[htbp]
\centering
\includegraphics[width=0.8\textwidth]{Image00026.jpg}
\end{figure}



\begin{figure}[htbp]
\centering
\includegraphics[width=0.8\textwidth]{Image00027.jpg}
\end{figure}



\par 
\begin{figure}[htbp]
\centering
\includegraphics[width=0.8\textwidth]{Image00006.jpg}
\end{figure}

 提示:
 这里的MySQL启动Shell脚本知识,请参考《跟老男孩学Linux运维:Shell编程实战》一书,京东售卖地址:\href{https://item.jd.com/12117874.html}{https://item.jd.com/12117874.html}
 。

\par 最终完成后的多实例根/data目录结果如下:

\noindent\rule{\textwidth}{0.4pt}


\begin{verbatim}[root@oldboy \textasciitilde{}]\# tree /data
/data
|-- 3306
|   |-- my.cnf  \#<==3306实例的配置文件。
|   `-- mysql   \#<==3306实例的启动文件。
`-- 3307
    |-- my.cnf  \#<==3307实例的配置文件。
    `-- mysql   \#<==3307实例的启动文件。
2 directories, 4 files
\end{verbatim}


\noindent\rule{\textwidth}{0.4pt}


\par 需要特别说明一下的是,在多实例启动文件中,启动MySQL不同实例服务所执行的命令实质上是有区别的,例如,启动3306实例的命令如下:

\noindent\rule{\textwidth}{0.4pt}


\begin{verbatim}mysqld\_safe --defaults-file=/data/3306/my.cnf > /dev/null  2>&1&
\end{verbatim}


\noindent\rule{\textwidth}{0.4pt}


\par 启动3307实例的命令如下:

\noindent\rule{\textwidth}{0.4pt}


\begin{verbatim}mysqld\_safe --defaults-file=/data/3307/my.cnf >/dev/null 2>&1 &
\end{verbatim}


\noindent\rule{\textwidth}{0.4pt}


\par 其中,“--defaults-file=/data/3307/my.cnf”表示指定配置文件启动,“>/dev/null 2>&1”表示将正确输出和错误输出定向到空。

\par 下面看看在多实例启动文件中,如何停止MySQL不同实例服务的命令。

\par 本次多实例脚本,老男孩统一采用了mysql官方推荐的kill的形式:

\noindent\rule{\textwidth}{0.4pt}


\begin{verbatim}printf "Stoping MySQL...\textbackslash\{\}n"
mysqld\_pid=`cat "\$mysqld\_pid\_file\_path"` \#<==获取进程pid。
if (kill -0 \$mysqld\_pid 2>/dev/null)     \#<==测试pid对应的mysql进程是否存在。
    then
        kill \$mysqld\_pid                 \#<==根据进程号杀死进程。
        sleep 2
fi
\end{verbatim}


\noindent\rule{\textwidth}{0.4pt}


\par 而在以往的教学中,老男孩讲解的是使用如下的mysqladmin命令关闭方法,这个命令的缺点是必须要有数据库的root用户密码才能运行

\par 停止3306实例的命令如下:

\noindent\rule{\textwidth}{0.4pt}


\begin{verbatim}mysqladmin -u root -poldboy123 -S /data/3306/mysql.sock shutdown
\end{verbatim}


\noindent\rule{\textwidth}{0.4pt}


\par 停止3307实例的命令如下:

\noindent\rule{\textwidth}{0.4pt}


\begin{verbatim}mysqladmin -u root -poldboy123 -S /data/3307/mysql.sock shutdown
\end{verbatim}


\noindent\rule{\textwidth}{0.4pt}



% From text00065.html


\subsection{4.4.5 配置MySQL多实例的文件权限}


\par 1)通过下面的命令授权mysql用户和组管理整个多实例的根目录/data:

\noindent\rule{\textwidth}{0.4pt}


\begin{verbatim}[root@oldboy \textasciitilde{}]\# chown -R mysql.mysql /data
[root@oldboy \textasciitilde{}]\# find /data -name mysql|xargs ls -l
-rw-r--r--. 1 mysql mysql 1192 Feb 26 21:39 /data/3306/mysql
-rw-r--r--. 1 mysql mysql 1307 Jul 21  2013 /data/3307/mysql
\end{verbatim}


\noindent\rule{\textwidth}{0.4pt}


\par 2)通过下面的命令授权mysql多实例所有启动文件的mysql可执行,设置700权限最佳,注意不要使用755权限,因为启动文件里有数据库管理员密码,会被读取到:

\noindent\rule{\textwidth}{0.4pt}


\begin{verbatim}[root@oldboy \textasciitilde{}]\# find /data -name mysql|xargs chmod 700
\end{verbatim}


\noindent\rule{\textwidth}{0.4pt}


\par 现在检查下上述命令的处理结果,看看权限是否处理完成:

\noindent\rule{\textwidth}{0.4pt}


\begin{verbatim}[root@oldboy \textasciitilde{}]\# find /data -name mysql -exec ls -l \{\} \textbackslash\{\};
-rwx------. 1 mysql mysql 1307 Jul 21  2013 /data/3307/mysql
-rwx------. 1 mysql mysql 1192 Feb 26 21:39 /data/3306/mysql
\end{verbatim}


\noindent\rule{\textwidth}{0.4pt}


\par 从输出来看,权限已经调整完毕,这里是引导读者在学习中使用记忆命令,其结果相当于“chmod 700/data/3306/mysql”和“chmod 700/data/3307/mysql”这两个命令的组合。


% From text00066.html


\subsection{4.4.6 MySQL相关命令加入全局路径的配置}


\par 1.配置全局路径的意义

\par 如果不为MySQL的命令配置全局路径,就会无法直接在命令行输入mysql这样的命令,这时只能使用全路径命令(“/application/mysql/bin/mysql”),而这种带着路径输入命令的方式很麻烦。

\par 2.配置MySQL全局路径的方法

\par 1)确认mysql命令所在的路径,命令如下:

\noindent\rule{\textwidth}{0.4pt}


\begin{verbatim}[root@oldboy \textasciitilde{}]\# ls /application/mysql/bin/mysql
/application/mysql/bin/mysql
\end{verbatim}


\noindent\rule{\textwidth}{0.4pt}


\par 2)在PATH变量前增加/application/mysql/bin路径,并追加到/etc/profile文件中,示例如下:

\noindent\rule{\textwidth}{0.4pt}


\begin{verbatim}[root@oldboy \textasciitilde{}]\# echo 'export PATH=/application/mysql/bin:\$PATH' >>/etc/profile
\#<==注意,echo后是单引号,双引号是不行的。
[root@oldboy \textasciitilde{}]\# tail -1 /etc/profile
export PATH=/application/mysql/bin:\$PATH
[root@oldboy \textasciitilde{}]\# source  /etc/profile
\#<==执行source使上一行添加到/etc/profile中,内容直接生效。
\#<==以上命令的用途为,定义mysql全局路径,实现在任意路径执行mysql命令。
[root@oldboy \textasciitilde{}]\# echo \$PATH
/application/mysql/bin:/usr/local/sbin:/usr/local/bin:/sbin:/bin:/usr/sbin:/usr/bin:/root/bin
\#<==执行echo \$PATH,若有/application/mysql/bin输出则表示配置成功。
提示:如果第3章已经配置过,这里就不用配置了。
\end{verbatim}


\noindent\rule{\textwidth}{0.4pt}


\par 
\begin{figure}[htbp]
\centering
\includegraphics[width=0.8\textwidth]{Image00006.jpg}
\end{figure}

 提示:
 更简单的设置方法为使用如下命令来做软链接:

\par ln-s/application/mysql/bin/*/usr/local/sbin/

\par 把mysql命令所在路径链接到全局路径/usr/local/sbin/的下面。

\par 3.因MySQL环境变量配置顺序导致的错误案例

\par 特别强调:务必把MySQL命令路径放在PATH路径中其他路径的前面,否则,可能会导致使用的mysql等命令和编译安装的mysql命令不是同一个,进而产生错误。下面的网址中给出了因为MySQL路径配置问题而导致的错误案例:\href{http://oldboy.blog.51cto.com/2561410/1122867}{http://oldboy.blog.51cto.com/2561410/1122867}
 ,实际上就是因为使用yum安装的MySQL客户端命令访问了编译安装的服务端而出了问题。


% From text00067.html


\subsection{4.4.7 初始化MySQL多实例的数据库文件}


\par 上述步骤全部配置完毕后,就可以初始化数据库文件了,这个步骤其实也可以在编译安装MySQL之后就操作,只不过放到这里更合理一些,注意其中与第3章内容的不同点。

\par (1)初始化MySQL数据库

\par 初始化命令为:

\noindent\rule{\textwidth}{0.4pt}


\begin{verbatim}cd /application/mysql/scripts
./mysql\_install\_db  --defaults-file=/data/3306/my.cnf --basedir=/application/mysql --datadir=/data/3306/data --user=mysql
./mysql\_install\_db  --defaults-file=/data/3307/my.cnf --basedir=/application/mysql --datadir=/data/3307/data --user=mysql
\end{verbatim}


\noindent\rule{\textwidth}{0.4pt}


\par 
\begin{figure}[htbp]
\centering
\includegraphics[width=0.8\textwidth]{Image00006.jpg}
\end{figure}

 提示:
 “--basedir=/application/mysql”为MySQL的安装路径,--datadir为不同的实例数据目录,这里多了一个特殊的参数,就是指定配置文件“--defaults-file=/data/3306/my.cnf”,而MySQL 5.6以前的版本在安装多实例时没有,单实例更不需要此参数。

\par 操作过程为:

\noindent\rule{\textwidth}{0.4pt}


\begin{verbatim}[root@oldboy \textasciitilde{}]\# cd /application/mysql/scripts
\end{verbatim}


\noindent\rule{\textwidth}{0.4pt}


\par 初始化3306实例数据库:

\noindent\rule{\textwidth}{0.4pt}


\begin{verbatim}[root@oldboy scripts]\# ./mysql\_install\_db  --defaults-file=/data/3306/my.cnf --basedir=/application/mysql --datadir=/data/3306/data --user=mysql
\#<==初始化mysql数据库文件,会有很多信息提示,如果没有ERROR级别的错误,只有两个OK的
    字样,则表示初始化成功,否则就要解决初始化的问题。
Installing MySQL system tables...2018-05-10 21:50:10 0 [Warning] 'THREAD\_CONCURRENCY' is deprecated and will be removed in a future release. \#<==如果一不小心,使用了MySQL 5.6已经遗弃的参数就会出现警告信息!
…省略若干行…
2018-05-10 21:50:10 7efded714720 InnoDB: Warning: Using innodb\_additional\_mem\_pool\_size is DEPRECATED. This option may be removed in future releases, together with the option innodb\_use\_sys\_malloc and with the InnoDB's internal memory allocator.
\#<==如果一不小心,使用了MySQL 5.6已经遗弃的参数就会出现警告信息!
…省略若干行…
2018-05-10 21:50:14 48474 [Note] InnoDB: Shutdown completed; log sequence number 1620041
OK  \#<==如果此处有两个OK,就表示初始化成功。
…省略若干行…
2018-05-10 21:50:14 48496 [Note] InnoDB: Using atomics to ref count buffer pool pages
2018-05-10 21:50:14 48496 [Note] InnoDB: The InnoDB memory heap is disabled
…省略若干行…
2018-05-10 21:50:16 48496 [Note] InnoDB: Shutdown completed; log sequence number 1620051
OK  \#<==如果此处有两个OK,就表示初始化成功。
To start mysqld at boot time you have to copy
…省略若干行…
WARNING: Found existing config file /application/mysql/my.cnf on the system.
Because this file might be in use, it was not replaced,
but was used in bootstrap (unless you used --defaults-file)
and when you later start the server.
The new default config file was created as /application/mysql/my-new.cnf,
please compare it with your file and take the changes you need.
\end{verbatim}


\noindent\rule{\textwidth}{0.4pt}


\par 初始化3307实例数据库:

\noindent\rule{\textwidth}{0.4pt}


\begin{verbatim}[root@oldboy scripts]\# ./mysql\_install\_db  --defaults-file=/data/3307/my.cnf --basedir=/application/mysql --datadir=/data/3307/data --user=mysql
Installing MySQL system tables... 2018-05-10 21:51:01 0 [Warning] 'THREAD\_CONCURRENCY' is deprecated and will be removed in a future release.
…省略若干行…
2018-05-10 21:51:05 48522 [Note] InnoDB: Shutdown completed; log sequence number 1620041
OK  \#<==如果此处有两个OK,就表示初始化成功。
…省略若干行…
2018-05-10 21:51:07 48544 [Note] InnoDB: Shutdown completed; log sequence number 1620051
OK  \#<==如果此处有两个OK,就表示初始化成功。
To start mysqld at boot time you have to copy
…省略若干行…
The new default config file was created as /application/mysql/my-new.cnf,
please compare it with your file and take the changes you need.
\end{verbatim}


\noindent\rule{\textwidth}{0.4pt}


\par (2)初始化数据库的原理及结果说明

\par 初始化数据库的实质就是创建基础的数据库系统的库文件,例如,生成MySQL库表等。

\par 初始化数据库之后,查看对应实例的数据目录,可以看到多了如下文件:

\noindent\rule{\textwidth}{0.4pt}


\begin{verbatim}[root@oldboy scripts]\# tree /data
/data
├── 3306
│   ├── data
│   │   ├── ibdata1
│   │   ├── ib\_logfile0
│   │   ├── ib\_logfile1
│   │   ├── ib\_logfile2
│   │   ├── mysql
│   │   │   ├── columns\_priv.frm
│   │   │   ├── columns\_priv.MYD
│   │   │   ├── columns\_priv.MYI
│   │   │   ├── db.frm
│   │   │   ├── db.MYD
│   │   │   ├── db.MYI
…省略部分…
\end{verbatim}


\noindent\rule{\textwidth}{0.4pt}


\par 多实例在安装过程中可能会遇到的问题与单实例是一样的,可参考第3章来解决,这些故障必须要解除,否则,后面会出现登录不了MySQL数据库等各种问题。


% From text00068.html


\subsection{4.4.8 启动MySQL多实例数据库}


\par 下面来看看启动MySQL多实例数据库的命令,启动前要做一些准备工作,否则数据库可能无法正常启动。

\noindent\rule{\textwidth}{0.4pt}


\begin{verbatim}[root@oldboy scripts]\# mkdir -p /application/mysql-5.6.40/tmp
\#<==编译时指定的socket路径。
[root@oldboy scripts]\# chown -R mysql.mysql /application/mysql/
\end{verbatim}


\noindent\rule{\textwidth}{0.4pt}


\par 如果按照第3章的内容安装了单实例数据库,则需要先停止之前启动的数据库。

\noindent\rule{\textwidth}{0.4pt}


\begin{verbatim}[root@oldboy \textasciitilde{}]\# /etc/init.d/mysqld stop
Shutting down MySQL.... SUCCESS!
[root@oldboy \textasciitilde{}]\# chkconfig mysqld off
[root@oldboy \textasciitilde{}]\# chkconfig --list mysqld
mysqld               0:off      1:off      2:off      3:off      4:off      5:off      6:off
\end{verbatim}


\noindent\rule{\textwidth}{0.4pt}


\par 第一个实例3306的启动命令为:

\noindent\rule{\textwidth}{0.4pt}


\begin{verbatim}[root@oldboy scripts]\# /data/3306/mysql start
Starting MySQL...
\end{verbatim}


\noindent\rule{\textwidth}{0.4pt}


\par 第二个实例3307的启动命令为:

\noindent\rule{\textwidth}{0.4pt}


\begin{verbatim}[root@oldboy scripts]\# /data/3307/mysql start
Starting MySQL...
\end{verbatim}


\noindent\rule{\textwidth}{0.4pt}


\par 现在,检查MySQL多实例数据库是否启动成功:

\noindent\rule{\textwidth}{0.4pt}


\begin{verbatim}[root@oldboy scripts]\# netstat -lntup|grep 330
tcp        0      0 :::3306        :::*           LISTEN     48065/mysqld
tcp        0      0 :::3307        :::*           LISTEN     50011/mysqld
\end{verbatim}


\noindent\rule{\textwidth}{0.4pt}


\par 
\begin{figure}[htbp]
\centering
\includegraphics[width=0.8\textwidth]{Image00006.jpg}
\end{figure}

 特别强调:
 因为第3章讲解过MySQL单实例,因此读者在实践第4章的内容时,最好将第3章的环境通过虚拟机还原到安装前的状态,否则可能会造成原来的进程处于启动状态,从而引发与多实例启动的冲突,老男孩的处理方法为,停止原来的单实例3306库,暂不使用。

\noindent\rule{\textwidth}{0.4pt}


\begin{verbatim}[root@oldboy scripts]\# /etc/init.d/mysqld stop
Shutting down MySQL.. SUCCESS!
[root@oldboy scripts]\# netstat -lntup|grep 330
[root@oldboy scripts]\# chkconfig mysqld off
[root@oldboy scripts]\# chkconfig --list mysqld
mysqld               0:off       1:off       2:off      3:off       4:off      5:off       6:off
\end{verbatim}


\noindent\rule{\textwidth}{0.4pt}


\par 根据输出可以看到,3306和3307实例均已正常启动。


% From text00069.html


\subsection{4.4.9 MySQL多实例数据库启动故障排错说明}


\par 如果MySQL多实例数据库有服务没有被启动,那么排查办法具体如下。

\par 如果发现没有显示MySQL对应实例的端口,则请稍微等待几秒再检查,MySQL服务的启动比Web服务等会慢一些。

\par 如果还不行,则请查看MySQL服务对应实例的错误日志,错误日志路径在my.cnf配置的最下面定义。例如,3306实例的错误日志为:

\noindent\rule{\textwidth}{0.4pt}


\begin{verbatim}[root@oldboy scripts]\# grep log-error /data/3306/my.cnf|tail -1
log-error=/data/3306/oldboy\_3306.err
\end{verbatim}


\noindent\rule{\textwidth}{0.4pt}


\par 那么,可以执行“tail-100/data/3306/mysql\_oldboy3306.err”检查mysql错误日志:

\noindent\rule{\textwidth}{0.4pt}


\begin{verbatim}[root@oldboy scripts]\# tail /data/3306/oldboy\_3306.err
2018-05-10 22:05:14 49285 [Note] Shutting down plugin 'MyISAM'
2018-05-10 22:05:14 49285 [Note] Shutting down plugin 'CSV'
2018-05-10 22:05:14 49285 [Note] Shutting down plugin 'MRG\_MYISAM'
2018-05-10 22:05:14 49285 [Note] Shutting down plugin 'sha256\_password'
2018-05-10 22:05:14 49285 [Note] Shutting down plugin 'mysql\_old\_password'
2018-05-10 22:05:14 49285 [Note] Shutting down plugin 'mysql\_native\_password'
2018-05-10 22:05:14 49285 [Note] Shutting down plugin 'binlog'
2018-05-10 22:05:14 49285 [Note] /application/mysql-5.6.40/bin/mysqld: Shutdown complete
\end{verbatim}


\noindent\rule{\textwidth}{0.4pt}


\par 上面的提示级别都为Note,表示没有错误。下面的提示中就有报错信息了:

\noindent\rule{\textwidth}{0.4pt}


\begin{verbatim}170226 22:05:14 mysqld\_safe mysqld from pid file /data/3306/mysqld.pid ended
/usr/local/mysql/libexec/mysqld: Can't find file: './mysql/plugin.frm' (errno: 13)
120412 15:23:33 [ERROR] Can't open the mysql.plugin table. Please run mysql\_upgrade to create it.
120412 15:23:34  InnoDB: Initializing buffer pool, size = 32.0M
120412 15:23:34  InnoDB: Completed initialization of buffer pool
120412 15:23:34  InnoDB: Operating system error number 13 in a file operation.
InnoDB: The error means mysqld does not have the access rights to
InnoDB: the directory.
InnoDB: File name ./ibdata1
InnoDB: File operation call: 'create'.
InnoDB: Cannot continue operation.
120412 15:23:34 mysqld\_safe mysqld from pid file /data/3306/mysqld.pid ended
\end{verbatim}


\noindent\rule{\textwidth}{0.4pt}


\par 此外,还可以通过以下方式来检查。

\par ·细看所有步骤执行命令返回的屏幕输出,不要忽略关键的输出内容。

\par ·查看mysql错误日志/application/mysql/data/机器名.err。

\par ·辅助查看系统日志/var/log/messages。

\par ·如果是MySQL关联了其他服务,则要同时查看相关服务的日志。

\par ·仔细阅读,重新查看所有操作的步骤是否正确,书写的命令及字符是不是都对。

\par 经常查看各种服务运行日志是一个很好的习惯,也是成为高手的必经之路!

\par 其他可能导致故障的问题如下。

\par ·数据库服务器磁盘空间满了(可用“df-h”查看磁盘空间是否满)。

\par ·防火墙iptables和selinux是否关闭。

\par ·更多故障问题可以搜索或者加入老男孩交流群(号码见前言部分)。


% From text00070.html


\section{4.5 配置及管理MySQL多实例数据库}


\par 1.配置MySQL多实例数据库开机自启动

\par 服务的开机自启动很关键,MySQL多实例的启动也不例外。把MySQL多实例的启动命令加入/etc/rc.local,可实现开机自启动:

\noindent\rule{\textwidth}{0.4pt}


\begin{verbatim}[root@oldboy scripts]\# echo "\#mysql multi instances" >>/etc/rc.local
[root@oldboy scripts]\# echo "/data/3306/mysql start" >>/etc/rc.local
[root@oldboy scripts]\# echo "/data/3307/mysql start" >>/etc/rc.local
[root@oldboy scripts]\# tail -3 /etc/rc.local
\#mysql multi instances
/data/3306/mysql start
/data/3307/mysql start
\end{verbatim}


\noindent\rule{\textwidth}{0.4pt}


\par 
\begin{figure}[htbp]
\centering
\includegraphics[width=0.8\textwidth]{Image00006.jpg}
\end{figure}

 提示:
 要确保MySQL脚本可执行,当然,也可以调整脚本,然后通过chkconfig进行开机启动管理,这部分内容在《跟老男孩学习Linux运维:Shell编程实战》一书中有深入讲解!

\par 2.登录MySQL测试

\par 测试命令如下:

\noindent\rule{\textwidth}{0.4pt}


\begin{verbatim}[root@oldboy scripts]\# mysql -S /data/3306/mysql.sock \#<==直接输入就可以进来了,而且身份还是root。但是多了-S /data/3306/mysql.sock,用于区别登录不同的实例。
Welcome to the MySQL monitor.  Commands end with ; or \textbackslash\{\}g.
Your MySQL connection id is 1
Server version: 5.6.40 Source distribution
Copyright (c) 2000, 2018, Oracle and/or its affiliates. All rights reserved.
Oracle is a registered trademark of Oracle Corporation and/or its
affiliates. Other names may be trademarks of their respective
owners.
Type 'help;' or '\textbackslash\{\}h' for help. Type '\textbackslash\{\}c' to clear the current input statement.
mysql> quit
Bye
[root@oldboy scripts]\# mysql -S /data/3307/mysql.sock \#<==登录3307实例。
Welcome to the MySQL monitor.  Commands end with ; or \textbackslash\{\}g.
...省略若干...
mysql> quit
Bye
\end{verbatim}


\noindent\rule{\textwidth}{0.4pt}


\par 3.MySQL多实例数据库的管理方法

\par MySQL安装完成之后,默认情况下,MySQL管理员的账号root是无密码的。登录不同的实例需要指定不同实例的sock路径及mysql.sock文件,这个mysql.sock是在my.cnf配置文件里指定的。

\par 下面是无密码情况下登录数据库的方法,关键点是“-S”参数及后面指定的/data/3306/mysql.sock,注意,对于不同实例的sock,虽然它们的名字相同,但是路径是不同的,因此指的是不同的文件:

\noindent\rule{\textwidth}{0.4pt}


\begin{verbatim}mysql -S /data/3306/mysql.sock
mysql -S /data/3307/mysql.sock
\end{verbatim}


\noindent\rule{\textwidth}{0.4pt}


\par 下面是重启对应实例数据库的命令:

\noindent\rule{\textwidth}{0.4pt}


\begin{verbatim}/data/3306/mysql stop \#<==停止数据库。
/data/3306/mysql start
\end{verbatim}


\noindent\rule{\textwidth}{0.4pt}


\par 4.MySQL安全配置

\par MySQL管理员的账号root密码默认为空,极不安全,可以通过mysqladmin命令为mysql不同实例的数据库设置独立的密码。示例如下:

\noindent\rule{\textwidth}{0.4pt}


\begin{verbatim}[root@oldboy scripts]\# mysqladmin -u root -S /data/3306/mysql.sock password 'oldboy123'
\#<==为mysql数据库3306实例root用户设置密码oldboy123。
Warning: Using a password on the command line interface can be insecure.
\#<==这里的Warning是5.6版本开始才有的, 是提醒用户不要在命令行输入密码,后文会讲解消除
       的方法。
[root@oldboy scripts]\# mysqladmin -u root -S /data/3307/mysql.sock password 'oldboy456'
\#<==为mysql数据库3307实例root用户设置密码oldboy456。
Warning: Using a password on the command line interface can be insecure.
[root@oldboy scripts]\# mysql -uroot -p -S /data/3306/mysql.sock
\#<==无法直接登录了
Enter password:
ERROR 1045 (28000): Access denied for user 'root'@'localhost' (using password: NO)
[root@oldboy scripts]\# mysql -uroot -p -S /data/3306/mysql.sock
Enter password:
ERROR 1045 (28000): Access denied for user 'root'@'localhost' (using password: NO)
[root@oldboy scripts]\# mysql -uroot -p -S /data/3306/mysql.sock
\#<==新的登录方式
Enter password:  \#<==输入新密码oldboy123
Welcome to the MySQL monitor.  Commands end with ; or \textbackslash\{\}g.
Your MySQL connection id is 5
Server version: 5.6.40 Source distribution
…省略若干…
Type 'help;' or '\textbackslash\{\}h' for help. Type '\textbackslash\{\}c' to clear the current input statement.
mysql>
\end{verbatim}


\noindent\rule{\textwidth}{0.4pt}


\par 
\begin{figure}[htbp]
\centering
\includegraphics[width=0.8\textwidth]{Image00006.jpg}
\end{figure}

 提示:
 3307实例的设置方法和3306实例的设置相同,只是连接时的sock路径不同而已,关于这点已提醒多次,大家要注意下。

\par 下面介绍带密码登录不同实例数据库的方法。

\par 登录3306实例:

\noindent\rule{\textwidth}{0.4pt}


\begin{verbatim}mysql -uroot -poldboy123 -S /data/3306/mysql.sock
\end{verbatim}


\noindent\rule{\textwidth}{0.4pt}


\par 登录3307实例:

\noindent\rule{\textwidth}{0.4pt}


\begin{verbatim}mysql -uroot -poldboy456 -S /data/3307/mysql.sock
\end{verbatim}


\noindent\rule{\textwidth}{0.4pt}


\par 
\begin{figure}[htbp]
\centering
\includegraphics[width=0.8\textwidth]{Image00006.jpg}
\end{figure}

 提示:
 基础弱的读者,在测试时尽量保证多实例的密码相同,可以减少麻烦,后面还原数据库会覆盖密码,因此这里也会把3306、3307的密码统一设为“oldboy123!”修改命令为:

\noindent\rule{\textwidth}{0.4pt}


\begin{verbatim}mysqladmin -u root -S /data/3307/mysql.sock -poldboy456 password oldboy123
\end{verbatim}


\noindent\rule{\textwidth}{0.4pt}


\par 更多内容,如删除默认的test数据库,以及删除默认的多实例的用户,请参见第3章,更多的数据库安全,请参考数据库优化章节。

\par 5.如何再增加一个MySQL的实例

\par 若在已经有了3306和3307实例的基础之上,还想要再增加一个MySQL实例,该怎么办?下面给出增加一个MySQL 3308端口实例的命令集合:

\noindent\rule{\textwidth}{0.4pt}


\begin{verbatim}mkdir -p /data/3308/data
\textbackslash\{\}cp /data/3306/my.cnf  /data/3308/
\textbackslash\{\}cp /data/3306/mysql  /data/3308/
sed -i 's/3306/3308/g' /data/3308/my.cnf
sed -i 's/server-id = 6/server-id = 8/g' /data/3308/my.cnf
sed -i 's/3306/3308/g' /data/3308/mysql
chown -R mysql:mysql /data/3308
chmod 700 /data/3308/mysql
cd /application/mysql/scripts
./mysql\_install\_db --defaults-file=/data/3308/my.cnf --datadir=/data/3308/data --basedir=/application/mysql --user=mysql
chown -R mysql:mysql /data/3308
egrep "server-id|log-bin" /data/3308/my.cnf
/data/3308/mysql start
sleep 5
netstat -lnt|grep 3308
\#提示:最好把server-id按照IP地址最后一个小数点的数字进行设置或者按照实例端口尾数进行设置。
\#成功标志:多了一个启动的端口3308
[root@oldboy scripts]\# netstat -lnt|grep 330
tcp        0      0 0.0.0.0:3306              0.0.0.0:*             LISTEN
tcp        0      0 0.0.0.0:3307              0.0.0.0:*             LISTEN
tcp        0      0 0.0.0.0:3308              0.0.0.0:*             LISTEN
\end{verbatim}


\noindent\rule{\textwidth}{0.4pt}


\par 如果配置完毕,服务启动后却没有运行起来,别忘了一定要看MySQL错误日志,在“/data/3308/my.cnf”最下面有错误日志路径地址。本例作为作业留给读者自己去实践,看看能否独自搞定增加这个实例。

\par 6.多实例MySQL登录问题分析

\par (1)多实例本地登录MySQL

\par 多实例本地登录一般是通过socket文件来指定具体登录到哪个实例的,此文件的具体位置是在mysql编译过程或者my.cnf文件里指定的。在本地登录数据库时,登录程序会通过socket文件来判断登录的是哪个数据库实例。

\par 例如,通过“mysql-uroot-p'oldboy123'-S/data/3307/mysql.sock”可知,登录的是3307这个实例。mysql.sock文件是MySQL服务端与本地MySQL客户端进行通信的Unix套接字文件。

\par (2)远程连接登录MySQL多实例

\par 远程登录MySQL多实例中的一个实例时,通过TCP端口(port)来指定所要登录的MySQL实例,此端口的配置是在MySQL配置文件my.cnf中指定的。

\par 例如,在“mysql-uoldboy-p'oldboy'-h 10.0.0.7-P 3307”中,“-P”为端口参数,后面接具体的实例端口,端口是一种“逻辑连接位置”,是客户端程序被分派到计算机上特殊服务程序的一种方式,强调的是提前在10.0.0.7上对oldboy用户做授权。


% From text00071.html


\section{4.6 参考资料}


\par 1)MySQL官方手册中关于MySQL 5.5和MySQL 5.6的内容。

\par 2)MySQL数据库企业级核心知识精品地址为:

\par 
\href{http://edu.51cto.com/course/course_id-4058.html}{http://edu.51cto.com/course/course\_id-4058.html}


\par 3)Heartbeat+DRBD+MySQL高可用架构方案与实施过程细节:

\par 
\href{http://oldboy.blog.51cto.com/2561410/1240412}{http://oldboy.blog.51cto.com/2561410/1240412}


\par 4)MySQL数据库企业级应用实战:

\par 
\href{http://edu.51cto.com/pack/view/id-214.html}{http://edu.51cto.com/pack/view/id-214.html}



% From text00072.html


\section{4.7 章节试题}


\par 1)什么是MySQL多实例?

\par 2)请描述MySQL多实例的优缺点?

\par 3)请描述如何配置MySQL多实例?

\par 4)如何在本地以及远程登录MySQL多实例?


% From text00073.html


\chapter{第5章 MySQL常用管理基础知识实践}


\section{
5.1 启动与关闭MySQL}
}


\par 因为MySQL数据库中存放着企业的重要数据,所以数据库数据必须保持高度的一致性,这也导致了MySQL的内部管理机制很复杂,而这就使得MySQL数据库服务不同于常规的Web服务,是不能随意关闭的,因此,能否正确地关闭和开启MySQL,是初学者管理MySQL数据库时最关键的事情之一。


% From text00074.html


\subsection{5.1.1 单实例MySQL启动与关闭知识}


\par 1.正确启动单实例MySQL数据库

\par 第3章曾经讲解了单实例MySQL数据库的安装和部署,不知读者是否还记得在初始化数据库时有这么一条命令:

\noindent\rule{\textwidth}{0.4pt}


\begin{verbatim}[root@oldboy mysql-5.6.40]\# cp support-files/mysql.server /etc/init.d/mysqld
\end{verbatim}


\noindent\rule{\textwidth}{0.4pt}


\par 这条命令是把数据库自带的启动脚本,复制到/etc/init.d/目录实现管理MySQL服务的启动和停止。

\par 下面就为大家介绍启动单实例数据库的两种正确方法。

\par (1)利用数据库自带的脚本启动数据库

\par 正确启动MySQL服务的命令如下:

\noindent\rule{\textwidth}{0.4pt}


\begin{verbatim}[root@oldboy \textasciitilde{}]\# /etc/init.d/mysqld start
Starting MySQL.... SUCCESS!
\end{verbatim}


\noindent\rule{\textwidth}{0.4pt}


\par 执行启动命令之后,检测数据库是否已经启动:

\noindent\rule{\textwidth}{0.4pt}


\begin{verbatim}[root@oldboy \textasciitilde{}]\# ss -lnt|grep 330
LISTEN     0      80                       :::3306                    :::*
\end{verbatim}


\noindent\rule{\textwidth}{0.4pt}


\par 
\begin{figure}[htbp]
\centering
\includegraphics[width=0.8\textwidth]{Image00006.jpg}
\end{figure}

 提示:
 根据输出可以得知数据库成功启动了。这里的“/etc/init.d/mysqld start”相当于“mysqld\_safe 2>&1>/dev/null &”。

\par (2)用初始化数据库时MySQL系统给出的方法启动

\par 首先,停止前面已启动的数据库,命令如下:

\noindent\rule{\textwidth}{0.4pt}


\begin{verbatim}[root@oldboy \textasciitilde{}]\# /etc/init.d/mysqld stop
Shutting down MySQL.. SUCCESS!
[root@oldboy \textasciitilde{}]\# ss -lnt|grep 330
\end{verbatim}


\noindent\rule{\textwidth}{0.4pt}


\par 然后,使用mysqld\_safe启动数据库,命令如下:

\noindent\rule{\textwidth}{0.4pt}


\begin{verbatim}[root@oldboy \textasciitilde{}]\# mysqld\_safe --user=mysql &
[1] 53854
[root@oldboy \textasciitilde{}]\# 170228 02:43:25 mysqld\_safe Logging to '/application/mysql
-5.6.40/data/oldboy.err'.
170228 02:43:25 mysqld\_safe Starting mysqld daemon with databases from /application/mysql-5.6.40/data
\end{verbatim}


\noindent\rule{\textwidth}{0.4pt}


\par 
\begin{figure}[htbp]
\centering
\includegraphics[width=0.8\textwidth]{Image00006.jpg}
\end{figure}

 提示:
 这里多出了很多提示信息,而且还需要再执行一次回车操作。

\par 还可以使用如下不输出提示的命令替代上述启动命令:

\noindent\rule{\textwidth}{0.4pt}


\begin{verbatim}[root@oldboy \textasciitilde{}]\# mysqld\_safe --user=mysql >/dev/null 2>&1 &
[1] 53981
\end{verbatim}


\noindent\rule{\textwidth}{0.4pt}


\par 现在,再次检测数据库是否已经启动,命令如下:

\noindent\rule{\textwidth}{0.4pt}


\begin{verbatim}[root@oldboy \textasciitilde{}]\# ss -lnt|grep 330
LISTEN     0      80                       :::3306                    :::*
\end{verbatim}


\noindent\rule{\textwidth}{0.4pt}


\par 可以通过如下命令查看MySQL服务进程信息:

\noindent\rule{\textwidth}{0.4pt}


\begin{verbatim}[root@oldboy \textasciitilde{}]\# ps -ef|grep mysql|grep -v grep
root  53981  53667  0 02:48 pts/1  00:00:00 /bin/sh /application/mysql/bin/mysqld\_safe --user=mysql
mysql     54070  53981  0 02:48 pts/1    00:00:00 /application/mysql-5.6.40/
bin/mysqld --basedir=/application/mysql-5.6.40 --datadir=/application/mysql-5.6.40/data --plugin-dir=/application/mysql-5.6.40/lib/plugin --user=mysql --log-error=/
application/mysql-5.6.40/data/oldboy.err --pid-file=/application/mysql-5.6.40/data/oldboy.pid
\end{verbatim}


\noindent\rule{\textwidth}{0.4pt}


\par 上述信息中一共有2个进程,一个是mysqld\_safe进程,类似于Web服务的nginx master(管理进程),另一个是mysqld进程,类似于Web服务的nginx worker(工作进程)。

\par 2.MySQL单实例服务启动的基本原理

\par “/etc/init.d/mysqld”是MySQL自带的使用Shell编写的启动脚本,执行脚本后最终会调用mysqld\_safe命令脚本,mysqld\_safe脚本执行后又会调用mysqld主程序启动MySQL服务,因此在前文查看MySQL进程时,会发现不仅有mysqld\_safe进程,还有mysqld进程。

\par 查看执行“/etc/init.d/mysqld start”命令的脚本中调用mysqld\_safe的程序代码如下:

\noindent\rule{\textwidth}{0.4pt}


\begin{verbatim}[root@oldboy \textasciitilde{}]\# sed -n '271,283p' /etc/init.d/mysqld
case "\$mode" in
  'start')
    \# Start daemon
    \# Safeguard (relative paths, core dumps..)
    cd \$basedir
    echo \$echo\_n "Starting MySQL"
    if test -x \$bindir/mysqld\_safe
    then
      \# Give extra arguments to mysqld with the my.cnf file. This script
      \# may be overwritten at next upgrade.
      \$bindir/mysqld\_safe --datadir="\$datadir" --pid-file="\$mysqld\_pid\_file\_path" \$other\_args >/dev/null & \#<==这正是在初始化数据库时系统给出的快速启动MySQL的方式。
\end{verbatim}


\noindent\rule{\textwidth}{0.4pt}


\par MySQL服务启动原理逻辑图如图5-1所示。

\begin{figure}[htbp]
\centering
\includegraphics[width=0.8\textwidth]{Image00028.jpg}
\end{figure}



\par 图5-1 MySQL服务启动原理逻辑图

\par 3.MySQL单实例服务启动小结

\par 1)使用“/etc/init.d/mysqld start”命令启动数据库的本质就相当于执行“mysqld\_safe--user=mysql &”命令。

\par 2)老男孩开发Shell脚本启动多实例数据库时就是调用的mysqld\_safe程序,最后由mysqld\_safe程序调用mysqld完成数据库的启动。

\par 3)在找回MySQL root密码时,也会使用mysqld\_safe程序,并且会认带忽略授权表的参数启动来找回root密码。

\par 4.正确关闭单实例MySQL数据库

\par (1)使用数据库自带脚本关闭数据库

\par 无论是使用自带脚本启动数据库,还是使用mysqld\_safe启动数据库,都可以采用自带脚本关闭MySQL服务,操作命令如下:

\noindent\rule{\textwidth}{0.4pt}


\begin{verbatim}[root@oldboy \textasciitilde{}]\# /etc/init.d/mysqld stop
Shutting down MySQL.. SUCCESS!
[root@oldboy \textasciitilde{}]\# ps -ef|grep mysql|grep -v grep \#<==采用过滤进程方法查看mysql服务。
\end{verbatim}


\noindent\rule{\textwidth}{0.4pt}


\par (2)使用mysqladmin管理命令关闭数据库

\par MySQL提供了一个mysqladmin管理命令,可以用来优雅地关闭数据库,命令如下:

\noindent\rule{\textwidth}{0.4pt}


\begin{verbatim}[root@oldboy \textasciitilde{}]\# mysqladmin -uroot -poldboy123 shutdown  \#<==注意密码要正确。
Warning: Using a password on the command line interface can be insecure.
                                                         \#<==忽略它。
[root@oldboy \textasciitilde{}]\# ps -ef|grep mysql|grep -v grep          \#<==已经关闭数据库。
\end{verbatim}


\noindent\rule{\textwidth}{0.4pt}


\par 5.MySQL单实例服务关闭的基本原理

\par 使用“/etc/init.d/mysqld stop”命令执行脚本关闭数据库的程序代码如下:

\noindent\rule{\textwidth}{0.4pt}


\begin{verbatim} [root@oldboy \textasciitilde{}]\# sed -n '298,315p' /etc/init.d/mysqld
  'stop')
    \# Stop daemon. We use a signal here to avoid having to know the
    \# root password.
    if test -s "\$mysqld\_pid\_file\_path"         \#<==如果\$mysqld\_pid\_file\_
                                             path变量大小不为0。
    then
      mysqld\_pid=`cat "\$mysqld\_pid\_file\_path"` \#<==则读取路径下的内容并赋值给
                                             mysqld\_pid。
      if (kill -0 \$mysqld\_pid 2>/dev/null)     \#<==如果\$mysqld\_pid对应的进程
                                             ID存在则为真。
      then
        echo \$echo\_n "Shutting down MySQL"     \#<==打印提示。
        kill \$mysqld\_pid               \#<==使用kill pid的方式关闭MySQL服务。
        \# mysqld should remove the pid file when it exits, so wait for it.
        wait\_for\_pid removed "\$mysqld\_pid" "\$mysqld\_pid\_file\_path"; return\_value=\$?                                           \#<==传参给函数。
      else
        log\_failure\_msg "MySQL server process \#\$mysqld\_pid is not running!"
                                               \#<==传消息给msg函数。
        rm "\$mysqld\_pid\_file\_path"             \#<==删除PID文件。
      fi
\end{verbatim}


\noindent\rule{\textwidth}{0.4pt}


\par 第4章的多实例数据库启动采用的就是这个方法,有关Shell编程的知识,请参考《跟老男孩学Linux运维:Shell编程实战》一书,京东售卖地址:\href{https://item.jd.com/12117874.html}{ttps://item.jd.com/12117874.html}
 。

\par MySQL自带脚本关闭数据库的原理就是通过kill pid的方式关闭数据库。

\par 6.使用关闭进程命令关闭数据库

\par 如果使用上述的常规方法无法关闭数据库,那么还有几个命令可以使用,它们分别是kill、killall、pkill,在不接任何参数的情况下,这几个命令发送的信号默认都是TERM(15),表示停止,但进程有权忽略该信号。示例如下:

\noindent\rule{\textwidth}{0.4pt}


\begin{verbatim}kill  pid            \#<==这里的pid是数据库服务对应的进程号。
killall mysqld       \#<==这里的mysqld是数据库服务对应的进程名字。
pkill mysqld         \#<==这里的mysqld是数据库服务对应的进程名字。
\end{verbatim}


\noindent\rule{\textwidth}{0.4pt}


\par 以下关闭数据库的操作应禁止使用,因为其很可能会导致MySQL数据库无法启动。

\noindent\rule{\textwidth}{0.4pt}


\begin{verbatim}killall -9 mysqld    \#<==这里的-9信号表示强制杀死进程。
kill -9 pid          \#<==这里的-9信号表示强制杀死进程。
\end{verbatim}


\noindent\rule{\textwidth}{0.4pt}


\par 可通过如下地址查看企业生产高并发环境下野蛮粗鲁杀死数据库所导致的故障案例:

\par 
\href{http://oldboy.blog.51cto.com/2561410/1431161}{http://oldboy.blog.51cto.com/2561410/1431161}


\par 
\href{http://oldboy.blog.51cto.com/2561410/1431172}{http://oldboy.blog.51cto.com/2561410/1431172}


\par 7.关闭MySQL数据库方法小结

\par 关闭MySQL数据库的方法选择顺序如下。

\par 第一种使用MySQL自带的管理脚本,示例如下:

\noindent\rule{\textwidth}{0.4pt}


\begin{verbatim}/etc/init.d/mysqld stop
\end{verbatim}


\noindent\rule{\textwidth}{0.4pt}


\par 第二种为mysqladmin管理方法:

\noindent\rule{\textwidth}{0.4pt}


\begin{verbatim}mysqladmin -uroot -poldboy123 shutdown \#<==使用这个命令的最大障碍就是必须事先知道密码。
\end{verbatim}


\noindent\rule{\textwidth}{0.4pt}


\par 第三种为利用系统进程管理命令关闭MySQL:

\noindent\rule{\textwidth}{0.4pt}


\begin{verbatim}kill  pid       \#<==这里的pid为数据库服务对应的进程号。
killall mysqld  \#<==这里的mysqld是数据库服务对应的进程名字。
pkill mysqld    \#<==这里的mysqld是数据库服务对应的进程名字。
\end{verbatim}


\noindent\rule{\textwidth}{0.4pt}



% From text00075.html


\subsection{5.1.2 多实例MySQL启动与关闭方法示例}


\par 下面我们看一下多实例MySQL启动与关闭方法的原理。

\par 启动3306实例命令服务的命令为“/data/3306/mysql start”,实际上就是mysqld\_safe加上不同的实例配置文件参数启动,示例如下:

\noindent\rule{\textwidth}{0.4pt}


\begin{verbatim}mysqld\_safe --defaults-file=/data/3306/my.cnf 2>&1 > /dev/null &
\end{verbatim}


\noindent\rule{\textwidth}{0.4pt}


\par 停止3306实例的命令为“/data/3306/mysql stop”,实际上使用的就是mysqladmin命令方法。示例如下:

\noindent\rule{\textwidth}{0.4pt}


\begin{verbatim}mysqladmin -uroot -poldboy123 -S /data/3306/mysql.sock shutdown
\end{verbatim}


\noindent\rule{\textwidth}{0.4pt}


\par 还有就是把多实例的停止改为使用和“/etc/init.d/mysqld stop”里面类似的kill方式,见MySQL默认启动脚本或第4章MySQL多实例启动脚本。


% From text00076.html


\section{5.2 MySQL连接原理方法及提示符设置}


\subsection{
5.2.1 客户端连接MySQL服务器原理结构}
}


\par 1.MySQL客户端简介

\par MySQL是一个典型的C/S服务结构软件,作为运维或DBA人员,我们经常使用MySQL自带的客户端程序(在“/application/mysql/bin”目录下)对其进行管理,常用的管理命令有mysql、mysqladmin、mysqldump、mysqlbinlog,mysql是登录MySQL最常用的客户端程序(有一些程序员也会使用类似phpmyadmin或者Navicat等的客户端管理软件)。

\par 2.客户端连接MySQL服务器原理结构

\par 图5-2为本地客户端和远程应用服务程序,连接MySQL服务的工作原理逻辑图。

\begin{figure}[htbp]
\centering
\includegraphics[width=0.8\textwidth]{Image00029.jpg}
\end{figure}



\par 图5-2 客户端连接MySQL服务器原理结构图

\par 3.MySQL连接方式介绍

\par 如图5-2所示,MySQL的连接方式有TCP/IP和Socket连接方式。TCP/IP连接方式一般是应用(PHP/Python/Java程序)和数据库服务不在一台机器上的连接方案,对于mysql命令来说就是指定-h参数登录,命令如下:

\noindent\rule{\textwidth}{0.4pt}


\begin{verbatim}mysql -uroot -poldboy123 -h 10.0.0.52
\end{verbatim}


\noindent\rule{\textwidth}{0.4pt}


\par 本地连接数据库常用的方式一般是Socket连接方式,特别是多实例本地MySQL登录,本书就首推socket的连接方式,命令如下:

\noindent\rule{\textwidth}{0.4pt}


\begin{verbatim}mysql -uroot -poldboy123 -S /tmp/mysql.sock
\end{verbatim}


\noindent\rule{\textwidth}{0.4pt}



% From text00077.html


\subsection{5.2.2 默认单实例MySQL登录方法}


\par 默认单实例MySQL的登录命令如下:

\noindent\rule{\textwidth}{0.4pt}


\begin{verbatim}1      mysql            \#<==刚装完系统无密码情况下的登录方式。不要密码。
2      mysql -uroot     \#<==刚装完系统无密码情况下的登录方式。不要密码。
3      mysql -uroot -p  \#<==这里是标准的dba命令行登录命令,交互式输入密码可有效防止密码泄露。
4      mysql -uroot -p’oldboy123’  \#<==命令行一般不这样使用,因为密码是明文的,5.6版本
                                  只要一发现这样使用就会输出警告。
\end{verbatim}


\noindent\rule{\textwidth}{0.4pt}


\par 下面先介绍几个防止MySQL密码泄露的小妙招(MySQL安全策略)。

\par 第一个小妙招:可以通过环境变量来强制Linux不记录敏感历史命令。

\par 在命令行执行“HISTCONTROL=ignorespace”后,再在输入带密码的命令前面加一个空格登录,登录命令不会被记录到历史记录里。示例如下:

\noindent\rule{\textwidth}{0.4pt}


\begin{verbatim}[root@oldboy \textasciitilde{}]\# HISTCONTROL=ignorespace     \#<==这里是临时生效,要想永久生效,
                                            请放入/etc/bashrc。
[root@oldboy \textasciitilde{}]\#  mysql -uroot -p'oldboy123' \#<==命令的开头要多一个空格。
\end{verbatim}


\noindent\rule{\textwidth}{0.4pt}


\par 第二个小妙招:操作完敏感的命令后及时删除命令行记录。

\par 可执行“history-d历史命令序号”清除指定历史记录命令:

\noindent\rule{\textwidth}{0.4pt}


\begin{verbatim}[root@oldboy \textasciitilde{}]\# history|tail -4     \#<==显示历史记录。
  252  mysql -uroot -p'oldboy123'    \#<==此条带密码,敏感,待删除。
  253  pwd
  254  history
  255  history|tail -4
[root@oldboy \textasciitilde{}]\# history -d 252      \#<==删除序号为252的历史记录。
[root@oldboy \textasciitilde{}]\# history|tail -5
  252  pwd                           \#<==序号252对应的带密码登录的命令已经消失。
  253  history
  254  history|tail -4
  255  history -d 252
  256  history|tail -5
\end{verbatim}


\noindent\rule{\textwidth}{0.4pt}


\par 可执行“history-c”清除所有记录:

\noindent\rule{\textwidth}{0.4pt}


\begin{verbatim}[root@oldboy \textasciitilde{}]\# history -c
[root@oldboy \textasciitilde{}]\# history
    1  history
\end{verbatim}


\noindent\rule{\textwidth}{0.4pt}


\par 也可执行“>\textasciitilde{}/.bash\_history”清除历史记录文件:

\par 第三个小妙招:为带密码的启动脚本以及备份脚本等加700权限,用户和组改为root。示例如下:

\noindent\rule{\textwidth}{0.4pt}


\begin{verbatim}chmod 700 /data/3306/mysql      \#<==如果采用kill信号的关闭方式数据库,可不执行。
chmod 700 /server/scripts/bak.sh    \#<==将密码写入my.cnf配置文件,使得执行备份
                                    命令不需要加密码。
\end{verbatim}


\noindent\rule{\textwidth}{0.4pt}


\par 第四个小妙招:把密码写入my.cnf配置文件并加600权限,用户和组改为mysql。示例如下:

\noindent\rule{\textwidth}{0.4pt}


\begin{verbatim}[root@oldboy \textasciitilde{}]\# cp /application/mysql/my.cnf /etc/
[root@oldboy \textasciitilde{}]\# grep -A 2 client /etc/my.cnf    \#<==配置文件开头添加如下三行,
                                                 无须重启服务。
[client]                   \#<==客户端模块标签。
User = root                \#<==用户参数及密码。
Password = oldboy123       \#<==密码参数及密码。
[root@oldboy \textasciitilde{}]\# mysql     \#<==此时登录数据库就不用输入密码了。
Welcome to the MySQL monitor.  Commands end with ; or \textbackslash\{\}g.
Your MySQL connection id is 8
Server version: 5.6.40 Source distribution
...省略若干行...
Type 'help;' or '\textbackslash\{\}h' for help. Type '\textbackslash\{\}c' to clear the current input statement.
mysql>
\end{verbatim}


\noindent\rule{\textwidth}{0.4pt}



% From text00078.html


\subsection{5.2.3 默认多实例MySQL登录方法}


\par 多实例MySQL本地登录的方法如下:

\noindent\rule{\textwidth}{0.4pt}


\begin{verbatim}[root@oldboy \textasciitilde{}]\# mysql -uroot -p -S /data/3306/mysql.sock
[root@oldboy \textasciitilde{}]\# mysql -uroot -p -S /data/3307/mysql.sock
\end{verbatim}


\noindent\rule{\textwidth}{0.4pt}


\par 多实例实际上就是通过mysql的“-S”参数指定不同的sock文件登录到不同的服务实例中。


% From text00079.html


\subsection{5.2.4 异地远程登录MySQL方法}


\par 无论是单实例还是多实例的远程登录(无须指定sock路径),均可通过指定主机和端口登录,但登录的用户必须提前被授权许可才行。

\par 单实例异地远程登录:

\noindent\rule{\textwidth}{0.4pt}


\begin{verbatim}mysql -uroot -p -h 127.0.0.1
\end{verbatim}


\noindent\rule{\textwidth}{0.4pt}


\par 多实例异地远程登录:

\noindent\rule{\textwidth}{0.4pt}


\begin{verbatim}mysql -uroot -p -h 127.0.0.1 -P3306
mysql -uroot -p -h 127.0.0.1 -P3307
\end{verbatim}


\noindent\rule{\textwidth}{0.4pt}


\par 这里的127.0.0.1可以被换成其他登录服务器的IP或IP段,注意必须提前被授权许可才行。


% From text00080.html


\subsection{5.2.5 MySQL连接提示符说明}


\par 1.MySQL提示符设置说明

\par 连接MySQL后的默认提示符是“mysql>”,这个提示符也是可以更改的,就像Linux命令行提示符一样。

\par 为了区分日常的正式环境和测试环境,从而避免操作失误,可以对提示符做一定的标记性修改,并且可将其写在配置里使之永久生效。

\par 更改MySQL数据登录提示符的方法具体如下。

\par (1)修改登录提示符

\par 命令行修改登录提示符的示例如下:

\noindent\rule{\textwidth}{0.4pt}


\begin{verbatim}mysql> prompt \textbackslash\{\}u@oldboy \textbackslash\{\}r:\textbackslash\{\}m:\textbackslash\{\}s->
PROMPT set to '\textbackslash\{\}u@oldboy \textbackslash\{\}r:\textbackslash\{\}m:\textbackslash\{\}s->'
root@oldboy 07:38:04->
root@oldboy 07:38:10->
\end{verbatim}


\noindent\rule{\textwidth}{0.4pt}


\par 其中,“\textbackslash\{\}u@oldboy\textbackslash\{\}r:\textbackslash\{\}m:\textbackslash\{\}s->”中的“\textbackslash\{\}u”为登录的数据库用户,“@”为分隔符,后面的oldboy为固定标签,“\textbackslash\{\}r:\textbackslash\{\}m:\textbackslash\{\}s”为时间信息,->为提示符标识。

\par (2)配置文件修改登录提示符

\par 在my.cnf配置文件的[mysql]模块下添加如下内容(注意,不是[mysqld])。保存后,无须重启MySQL,退出当前session,重新登录即可。如果是在my.cnf配置文件中添加的,可以使用“\textbackslash\{\}\textbackslash\{\}”符号,以避免转义带来的问题。

\noindent\rule{\textwidth}{0.4pt}


\begin{verbatim}[client]
prompt=\textbackslash\{\}\textbackslash\{\}u@oldboy \textbackslash\{\}\textbackslash\{\}r:\textbackslash\{\}\textbackslash\{\}m:\textbackslash\{\}\textbackslash\{\}s->
3)
\end{verbatim}


\noindent\rule{\textwidth}{0.4pt}


\par (3)多实例场景登录提示符说明

\par 在多实例场景下,要想使得提示符配置生效,不仅需要把参数放到my.cnf配置文件里,还需要在连接MySQL时增加“--defaults-extra-file”参数指定修改的配置文件。

\par 可通过如下命令修改MySQL多实例配置文件,增加一行配置:

\noindent\rule{\textwidth}{0.4pt}


\begin{verbatim}[root@oldboy \textasciitilde{}]\# head -4 /data/3307/my.cnf
[client]                            \#<==客户端模块标签。
port            = 3307
socket          = /data/3307/mysql.sock
user = root                         \#<==为了登录方面,也可以增加用户参数设置。
Password = oldboy123                \#<==为了登录方面,也可以增加用户密码参数设置。
prompt = \textbackslash\{\}\textbackslash\{\}u@oldboy \textbackslash\{\}\textbackslash\{\}r:\textbackslash\{\}\textbackslash\{\}m:\textbackslash\{\}\textbackslash\{\}s->   \#<==登录后提示。
\end{verbatim}


\noindent\rule{\textwidth}{0.4pt}


\par 增加密码参数后,请注意配置文件权限。

\par 现在重新登录,仔细看,可以发现增加了一个特殊的指定配置文件的参数:

\noindent\rule{\textwidth}{0.4pt}


\begin{verbatim}[root@oldboy \textasciitilde{}]\# mysql --defaults-extra-file=/data/3307/my.cnf -S /data/3307/mysql.sock
\#<==如果要带用户密码参数,则要放在--defaults-extra-file选项后面。
Warning: Using a password on the command line interface can be insecure.
Welcome to the MySQL monitor.  Commands end with ; or \textbackslash\{\}g.
Your MySQL connection id is 3
Server version: 5.6.40 Source distribution
...省略若干行...
root@oldboy 07:53:25->quit
\end{verbatim}


\noindent\rule{\textwidth}{0.4pt}


\par 2.登录后的显示信息

\par MySQL 5.6.40登录后的显示信息如下:

\noindent\rule{\textwidth}{0.4pt}


\begin{verbatim}[root@oldboy \textasciitilde{}]\# mysql -uroot -poldboy123
Warning: Using a password on the command line interface can be insecure.
Welcome to the MySQL monitor.  Commands end with ; or \textbackslash\{\}g. \#<==mysql命令以分
                                                      号;或\textbackslash\{\}g结束。
Your MySQL connection id is 13
Server version: 5.6.40 Source distribution                \#<==版本信息。
……省略若干行……
Type 'help;' or '\textbackslash\{\}h' for help. Type '\textbackslash\{\}c' to clear the current input statement.
                                                          \#<==帮助提示。
mysql>
\end{verbatim}


\noindent\rule{\textwidth}{0.4pt}


\par 
\begin{figure}[htbp]
\centering
\includegraphics[width=0.8\textwidth]{Image00006.jpg}
\end{figure}

 提示:
 很多网友都不注意这些登录信息,其实很多有用的提示内容都包含在登录信息里面。


% From text00081.html


\subsection{5.2.6 退出MySQL数据库}


\par 数据库维护完毕后,如果需要退出,则可以执行下面的命令:

\noindent\rule{\textwidth}{0.4pt}


\begin{verbatim}1、quit      (\textbackslash\{\}q) Quit mysql.
2、exit      (\textbackslash\{\}q) Exit mysql. Same as quit.
3、也可以使用快捷键Ctrl+c或Ctrl+d。
\end{verbatim}


\noindent\rule{\textwidth}{0.4pt}


\par 操作命令如下:

\noindent\rule{\textwidth}{0.4pt}


\begin{verbatim}mysql>quit
\end{verbatim}


\noindent\rule{\textwidth}{0.4pt}



% From text00082.html


\section{5.3 查看MySQL命令帮助}


\par MySQL中的查看帮助命令help与Linux命令行的普通命令帮助工具man类似,与Linux下bash内置命令的帮助工具help相同。要想查看MySQL中命令的使用语法等帮助信息,就需要用到help工具了,可以直接输入help,或者先输入help,然后在后面接要查看的命令及命令组合,例如help create。默认情况下MySQL中的命令是不区分大小写的,这与Linux命令行严格区分大小写是有区别的。

\par 下面直接在MySQL中键入help命令,输出的这部分帮助信息用途不是很大,简单了解即可:

\noindent\rule{\textwidth}{0.4pt}


\begin{verbatim}mysql> help
...省略若干行...
List of all MySQL commands:
Note that all text commands must be first on line and end with ';'
?         (\textbackslash\{\}?) Synonym for `help'.
clear     (\textbackslash\{\}c) Clear the current input statement.
connect   (\textbackslash\{\}r) Reconnect to the server. Optional arguments are db and host.
delimiter (\textbackslash\{\}d) Set statement delimiter.
edit      (\textbackslash\{\}e) Edit command with \$EDITOR.
ego       (\textbackslash\{\}G) Send command to mysql server, display result vertically.
exit      (\textbackslash\{\}q) Exit mysql. Same as quit.
go        (\textbackslash\{\}g) Send command to mysql server.
help      (\textbackslash\{\}h) Display this help.
nopager   (\textbackslash\{\}n) Disable pager, print to stdout.
notee     (\textbackslash\{\}t) Don't write into outfile.
pager     (\textbackslash\{\}P) Set PAGER [to\_pager]. Print the query results via PAGER.
print     (\textbackslash\{\}p) Print current command.
prompt    (\textbackslash\{\}R) Change your mysql prompt.
quit      (\textbackslash\{\}q) Quit mysql.
rehash    (\textbackslash\{\}\#) Rebuild completion hash.
source    (\textbackslash\{\}.) Execute an SQL script file. Takes a file name as an argument.
status    (\textbackslash\{\}s) Get status information from the server.
system    (\textbackslash\{\}!) Execute a system shell command.
tee       (\textbackslash\{\}T) Set outfile [to\_outfile]. Append everything into given outfile.
use       (\textbackslash\{\}u) Use another database. Takes database name as argument.
charset   (\textbackslash\{\}C) Switch to another charset. Might be needed for processing binlog with multi-byte charsets.
warnings  (\textbackslash\{\}W) Show warnings after every statement.
nowarning (\textbackslash\{\}w) Don't show warnings after every statement.
For server side help, type 'help contents'
\end{verbatim}


\noindent\rule{\textwidth}{0.4pt}


\par 比较重要的帮助命令是“help contents”,也可以使用“?contents”命令,它可以显示很多重要的帮助信息。示例如下:

\noindent\rule{\textwidth}{0.4pt}


\begin{verbatim}mysql> help contents
You asked for help about help category: "Contents"
For more information, type 'help <item>', where <item> is one of the following
categories:
    Account Management
    Administration
    Compound Statements
    Data Definition
    Data Manipulation
    Data Types
    Functions
    Functions and Modifiers for Use with GROUP BY
    Geographic Features
    Help Metadata
    Language Structure
    Plugins
    Procedures
    Storage Engines
    Table Maintenance
    Transactions
    User-Defined Functions
    Utility
\end{verbatim}


\noindent\rule{\textwidth}{0.4pt}


\par 为了方便读者辨识这些类别的知识,下面针对上面的输出结果进行了翻译,如表5-1所示。

\par 表5-1 “help contents”结果信息表

\begin{figure}[htbp]
\centering
\includegraphics[width=0.8\textwidth]{Image00030.jpg}
\end{figure}



\par 在上述“help contents”命令的输出结果中,还可以用help加上感兴趣的信息,继续深入查看帮助,示例如下:

\noindent\rule{\textwidth}{0.4pt}


\begin{verbatim}mysql> help Data Definition \#<==Data Definition的意思是数据定义分类,即数据库
                              数据定义命令。
You asked for help about help category: "Data Definition"
For more information, type 'help <item>', where <item> is one of the following
topics:
    ALTER DATABASE
    ALTER EVENT
    ALTER FUNCTION
    ALTER LOGFILE GROUP
    ALTER PROCEDURE
    ALTER SERVER
    ALTER TABLE
    ALTER TABLESPACE
    ALTER VIEW
    CONSTRAINT
    CREATE DATABASE
    CREATE EVENT
    CREATE FUNCTION
    CREATE INDEX
    CREATE LOGFILE GROUP
    CREATE PROCEDURE
    CREATE SERVER
    CREATE TABLE
    CREATE TABLESPACE
    CREATE TRIGGER
    CREATE VIEW
    DROP DATABASE
    DROP EVENT
    DROP FUNCTION
    DROP INDEX
    DROP PROCEDURE
    DROP SERVER
    DROP TABLE
    DROP TABLESPACE
    DROP TRIGGER
    DROP VIEW
    RENAME TABLE
    TRUNCATE TABLE
mysql>
\end{verbatim}


\noindent\rule{\textwidth}{0.4pt}


\par 同样为了方便读者辨识这些命令的前缀知识,下面针对上面的输出信息进行了翻译,如表5-2所示。

\par 表5-2 “help Data Definition”结果信息表

\begin{figure}[htbp]
\centering
\includegraphics[width=0.8\textwidth]{Image00031.jpg}
\end{figure}



\begin{figure}[htbp]
\centering
\includegraphics[width=0.8\textwidth]{Image00032.jpg}
\end{figure}



\par 在上述“help Data Definition”命令的输出中,还可以使用help加上感兴趣的具体管理数据库的命令信息,从而进一步深入查看帮助信息。比如,可以通过如下示例查看创建数据库的帮助信息:

\noindent\rule{\textwidth}{0.4pt}


\begin{verbatim}mysql> help CREATE DATABASE
Name: 'CREATE DATABASE'
Description:
Syntax:
CREATE \{DATABASE | SCHEMA\} [IF NOT EXISTS] db\_name
    [create\_specification] ...
create\_specification:
    [DEFAULT] CHARACTER SET [=] charset\_name
  | [DEFAULT] COLLATE [=] collation\_name
CREATE DATABASE creates a database with the given name. To use this
statement, you need the CREATE privilege for the database. CREATE
SCHEMA is a synonym for CREATE DATABASE.
URL: http://dev.mysql.com/doc/refman/5.6/en/create-database.html
mysql>
\end{verbatim}


\noindent\rule{\textwidth}{0.4pt}


\par 这样就查到了具体的管理数据库的命令帮助信息,查看其他的帮助信息也是如此,这里需要注意一点的是,读者要有一定的阅读英文或者使用工具翻译英文的能力。

\par 
\begin{figure}[htbp]
\centering
\includegraphics[width=0.8\textwidth]{Image00006.jpg}
\end{figure}

 提示:
 MySQL的help命令帮助就相当于Linux下的man命令一样,很好用,读者要牢记。


% From text00083.html


\section{5.4 设置及修改mysql root用户密码}


\subsection{
5.4.1 MySQL数据库用户安全策略介绍}
}


\par 安装完MySQL数据库之后,默认的管理员root密码为空,这很不安全。因此需要为root用户设置一个密码。其实前面在安装MySQL数据库之后,我们已经做了一些安全措施,具体如下。

\par ·数据库不设置外网IP。

\par ·为root用户设置比较复杂的密码。

\par ·删除无用的mysql库内的用户账号,只保留root@localhost以及root@127.0.0.1。

\par ·删除默认的test数据库。

\par ·增加用户的时候,授权的权限应尽量最小,允许访问的主机范围最小化。

\par ·登录命令行操作不携带密码,而是回车后输入密码,从MySQL 5.6版本起命令行携带密码会报警告提示。

\par 除了上面的方法之外,针对MySQL数据库的用户处理,还有更严格的做法,下面就来看看。

\par 以下是采用更安全的措施来删除root,添加新的管理员用户。

\par 1)删除所有mysql中的用户,包括root超级用户:

\noindent\rule{\textwidth}{0.4pt}


\begin{verbatim}mysql> delete from mysql.user;
Query OK, 2 rows affected (0.00 sec)
\end{verbatim}


\noindent\rule{\textwidth}{0.4pt}


\par 这里的root可以保留,修改为其他用户也可以。

\par 2)增加system并将该管理员用户提升为超级管理员,即与root等价的管理员用户,只是名字不同而已:

\noindent\rule{\textwidth}{0.4pt}


\begin{verbatim}mysql> grant all privileges on *.* to system@'localhost' identified by 'oldboy123' with grant option;
Query OK, 0 rows affected (0.00 sec)
mysql> flush privileges;
Query OK, 0 rows affected (0.00 sec)
提示:flush-privileges      Reload grant tables (same as reload)
\end{verbatim}


\noindent\rule{\textwidth}{0.4pt}


\par 此外,对于带密码的文件或脚本权限,最好是文件用600,脚本用700,用户和组则用root或mysql。


% From text00084.html


\subsection{5.4.2 为管理员root用户设置及修改密码}


\par 1.为root用户设置密码

\par 刚安装完MySQL时是没有密码的,此时可以使用下面的命令为MySQL设置密码。

\noindent\rule{\textwidth}{0.4pt}


\begin{verbatim}mysqladmin -u root password 'oldboy123' \#<==适合单实例。
mysqladmin -u root password 'oldboy123' -S /data/3306/mysql.sock
                                        \#<==适合多实例。
提示:命令是在Linux命令行执行的,而不是在mysql命令行。
\end{verbatim}


\noindent\rule{\textwidth}{0.4pt}


\par 2.为root用户修改密码的方法一:Linux命令行修改法

\par 在Linux命令行下修改密码适合于已知密码的场合,此时可利用mysqladmin进行密码修改,示例如下:

\noindent\rule{\textwidth}{0.4pt}


\begin{verbatim}mysqladmin -u root -p'oldboy123' password 'oldboy' \#<== oldboy456为原密码,
                                                 新密码为oldboy。
mysqladmin -u root -p'oldboy' password 'oldboy123' -S /data/3306/mysql.sock
                                              \#<==适合多实例方式。
提示:读者练习完后请把密码依旧修改为oldboy123,本书就是这样做的。
\end{verbatim}


\noindent\rule{\textwidth}{0.4pt}


\par 3.为root用户修改密码的方法二:SQL语句修改法

\par 在MySQL命令行下修改密码常用于遗忘了密码的情况,或者给不熟悉Linux命令行管理的人员使用,这里的操作命令略微复杂一些,不过后文会有细致讲解。

\noindent\rule{\textwidth}{0.4pt}


\begin{verbatim}mysql> UPDATE mysql.user SET password=PASSWORD("oldboy123") WHERE user='root' and host='localhost';
Query OK, 1 row affected (0.00 sec)
Rows matched: 1  Changed: 1  Warnings: 0
mysql> flush privileges; \#<==刷新权限使得修改密码生效。
Query OK, 0 rows affected (0.00 sec)
\end{verbatim}


\noindent\rule{\textwidth}{0.4pt}


\par 
\begin{figure}[htbp]
\centering
\includegraphics[width=0.8\textwidth]{Image00006.jpg}
\end{figure}

 提示:


\par 1)此法是SQL语句的DML语句,必须指定where条件(具体的用户和主机)。

\par 2)必须使用PASSWORD()函数来加密更改密码,才能保证数据库里的密码是加密的。

\par 3)此命令是在mysql命令行执行的,而不是在Linux命令行执行。

\par 4)此法更适合密码丢失后,通过“--skip-grant-tables”参数启动数据库,再对密码进行修改的情形。

\par 4.为root用户修改密码的方法三:SQL语句修改法

\par 下述方法看起来简单,但也有很大的局限性:

\noindent\rule{\textwidth}{0.4pt}


\begin{verbatim}mysql> set password=password('oldboy');
Query OK, 0 rows affected (0.00 sec)
mysql> flush privileges;
Query OK, 0 rows affected (0.00 sec)
mysql> quit
Bye
\end{verbatim}


\noindent\rule{\textwidth}{0.4pt}


\par 该方法的局限性表现在如下两个方面。

\par 1)本方法仅为修改当前用户密码。

\par 2)不适合通过“--skip-grant-tables”方式启动后修改密码,若此时采用该方法,将会报错,报错提示如下:

\noindent\rule{\textwidth}{0.4pt}


\begin{verbatim}mysql>set password=password('oldboy');
ERROR 1290 (HY000): The MySQL server is running with the --skip-grant-tables option so it cannot execute this statement
\end{verbatim}


\noindent\rule{\textwidth}{0.4pt}



% From text00085.html


\section{5.5 找回MySQL root用户密码}


\par 对于一些粗心的管理员,可能会遭遇忘记root密码的情况,那么,假如我们一不小心忘记了数据库密码,该如何找回呢?本节将对此进行讲解。


% From text00086.html


\subsection{5.5.1 找回MySQL单实例root用户密码的方法}


\par 首先停止MySQL服务,示例如下:

\noindent\rule{\textwidth}{0.4pt}


\begin{verbatim}[root@oldboy \textasciitilde{}]\# /etc/init.d/mysqld stop
Shutting down MySQL.... SUCCESS!
\end{verbatim}


\noindent\rule{\textwidth}{0.4pt}


\par 然后,使用mysqld\_safe附带的“--skip-grant-tables”(忽略授权登录验证)启动MySQL服务。示例如下:

\noindent\rule{\textwidth}{0.4pt}


\begin{verbatim}[root@oldboy \textasciitilde{}]\# mysqld\_safe --skip-grant-tables --user=mysql >/dev/null 2>&1 &
[1] 57303  \#<==在启动时加--skip-grant-tables参数,表示忽略授权表验证。
[root@oldboy \textasciitilde{}]\# ss -lnt|grep 330
LISTEN     0      80                       :::3306                    :::*
\end{verbatim}


\noindent\rule{\textwidth}{0.4pt}


\par 现在,无须密码即可登录MySQL,示例如下:

\noindent\rule{\textwidth}{0.4pt}


\begin{verbatim}[root@oldboy \textasciitilde{}]\# mysql \#<==登录时空密码即可。
Welcome to the MySQL monitor.  Commands end with ; or \textbackslash\{\}g.
Your MySQL connection id is 1
Server version: 5.6.34 Source distribution
……省略若干行……
Type 'help;' or '\textbackslash\{\}h' for help. Type '\textbackslash\{\}c' to clear the current input statement.
mysql> mysqld\_safe --skip-grant-tables --user=mysql >/dev/null 2>&1 &
mysql \#<==登录时空密码即可。
\end{verbatim}


\noindent\rule{\textwidth}{0.4pt}


\par 这时,可以将root密码修改为新密码了,示例如下:

\noindent\rule{\textwidth}{0.4pt}


\begin{verbatim}mysql> set password=password('oldboy123'); \#<==该方法在此处无效。
ERROR 1290 (HY000): The MySQL server is running with the --skip-grant-tables option so it cannot execute this statement
mysql> update mysql.user set password=PASSWORD("oldboy123") where user="root" and host="localhost";
Query OK, 0 rows affected (0.01 sec)       \#<==该方法正合适,mysqladmin的方法
                                          在这里也是无效的。
Rows matched: 1  Changed: 0  Warnings: 0
mysql> flush privileges;                   \#<==刷新权限使得修改密码生效。
Query OK, 0 rows affected (0.01 sec)
mysql> quit
Bye
\end{verbatim}


\noindent\rule{\textwidth}{0.4pt}


\par 重启MySQL服务,命令如下:

\noindent\rule{\textwidth}{0.4pt}


\begin{verbatim}[root@oldboy \textasciitilde{}]\# /etc/init.d/mysqld stop \#<==也可以执行mysqladmin -uroot
                                          -poldboy123 shutdown。
Shutting down MySQL.. SUCCESS!
[1]+  Done   mysqld\_safe --skip-grant-tables --user=mysql > /dev/null 2>&1
[root@oldboy \textasciitilde{}]\# /etc/init.d/mysqld start
Starting MySQL. SUCCESS!
\end{verbatim}


\noindent\rule{\textwidth}{0.4pt}


\par 可再登录进行测试,示例如下:

\noindent\rule{\textwidth}{0.4pt}


\begin{verbatim}[root@oldboy \textasciitilde{}]\# mysql -uroot -poldboy  \#<==使用错误的密码导致无法登录。
Warning: Using a password on the command line interface can be insecure.
ERROR 1045 (28000): Access denied for user 'root'@'localhost' (using password: YES)
[root@oldboy \textasciitilde{}]\# mysql -uroot -poldboy123 \#<==使用修改过后的密码即可登录了。
Warning: Using a password on the command line interface can be insecure.
Welcome to the MySQL monitor.  Commands end with ; or \textbackslash\{\}g.
Your MySQL connection id is 3
...省略若干行...
mysql>
\end{verbatim}


\noindent\rule{\textwidth}{0.4pt}


\par 需要说明的是,加“--skip-grant-tables”参数登录后,修改完密码一定要重启,然后再对外提供服务。如果你发现重启后使用MySQL不加密码依然可以登录,请查看是不是配置文件设置了密码。此外,在MySQL命令行使用的密码会覆盖my.cnf配置文件中配置的密码。


% From text00087.html


\subsection{5.5.2 找回MySQL多实例root用户的密码方法}


\par 这里以多实例的3307为例进行讲解。

\par 第一步,关闭多实例3307 MySQL服务。

\par 首先使用killall mysqld或者kill pid(需要使用ps查到pid的值)的方式关闭指定实例3307的数据库,对于多实例的脚本(/data/3307/mysql),老男孩之前是使用mysqladmin的方式进行关闭的,这种方式需要使用密码才行,因此这个脚本此时就无法关闭(假设忘记了密码)了。

\par 第二步,启动数据库时加“--skip-grant-tables”参数,注意,该参数要放到结尾:

\noindent\rule{\textwidth}{0.4pt}


\begin{verbatim}[root@oldboy \textasciitilde{}]\# mysqld\_safe --defaults-file=/data/3307/my.cnf --skip-grant-tables >/dev/null 2>&1 &
[1] 58437
[root@oldboy \textasciitilde{}]\# ss -lnt|grep 3307
LISTEN     0      128                      :::3307                    :::*
\end{verbatim}


\noindent\rule{\textwidth}{0.4pt}


\par 第三步,使用登录命令,示例如下:

\noindent\rule{\textwidth}{0.4pt}


\begin{verbatim}[root@oldboy \textasciitilde{}]\# mysql -S /data/3307/mysql.sock   \#<==登录时空密码即可。
\end{verbatim}


\noindent\rule{\textwidth}{0.4pt}


\par 第四步,修改密码,示例如下:

\noindent\rule{\textwidth}{0.4pt}


\begin{verbatim}mysql> update mysql.user set password=PASSWORD("oldboy123") where user="root" and host="localhost";
mysql> FLUSH PRIVILEGES;
mysql> quit
\end{verbatim}


\noindent\rule{\textwidth}{0.4pt}


\par 现在,可以重启服务,使用新密码登录了。

\par 改完密码后,停止数据库,以正常启动的方式重新启动并登录:

\noindent\rule{\textwidth}{0.4pt}


\begin{verbatim}[root@oldboy \textasciitilde{}]\# /data/3307/mysql stop
\#<==要提前编辑脚本,将修改后的密码写到脚本密码参数位置,否则会无法关闭数据库。
Stoping MySQL...
Warning: Using a password on the command line interface can be insecure.
[1]+  Done                   mysqld\_safe --defaults-file=/data/3307/my.cnf --skip-grant-tables > /dev/null 2>&1
[root@oldboy \textasciitilde{}]\# /data/3307/mysql start          \#<==正式重启多实例MySQL。
Starting MySQL...
[root@oldboy \textasciitilde{}]\# ss -lnt|grep 3307               \#<==检查是否重启成功。
LISTEN     0      128                      :::3307                    :::*
[root@oldboy \textasciitilde{}]\# mysql -S /data/3307/mysql.sock  \#<==不用密码登录不行了。
ERROR 1045 (28000): Access denied for user 'root'@'localhost' (using password: YES)
[root@oldboy \textasciitilde{}]\# mysql -uroot -poldboy -S /data/3307/mysql.sock
                                                 \#<==使用新密码登录。
Warning: Using a password on the command line interface can be insecure.
Welcome to the MySQL monitor.  Commands end with ; or \textbackslash\{\}g.
...省略若干行...
mysql>
\end{verbatim}


\noindent\rule{\textwidth}{0.4pt}


\par 特殊提示:本章在单实例和多实例数据库之间进行了各种切换,请在使用单实例数据库之前,停止掉多实例中的3306实例,停止命令为/data/3306/mysql stop,如果想使用多实例3306实例及3307实例,则需要提前停止单实例数据库,停止命令为/etc/init.d/mysqd stop。


% From text00088.html


\section{5.6 章节试题}


\par 1)请描述MySQL启动与关闭的命令原理。

\par 2)请描述MySQL启动与关闭的常见优雅方法。

\par 3)如何查看MySQL内部SQL命令的帮助?

\par 4)如何设置与修改MySQL管理用户的密码?

\par 5)MySQL数据库密码丢失如何找回?

\par 6)如何防止MySQL数据库密码泄露?


% From text00089.html


\chapter{第6章 MySQL常用管理SQL语句应用实践}


\section{
6.1 SQL介绍}
}


\subsection{
6.1.1 什么是SQL}
}


\par SQL,英文全称为Structured Query Language,中文意思是结构化查询语言,它是一种对关系型数据库中的数据进行定义和操作的语言,是大多数关系型数据库管理系统所支持的工业标准语言。

\par 结构化查询语言(SQL)是一种数据库查询和程序设计语言,用于存取数据以及查询、更新和管理关系型数据库系统。SQL常用作MySQL逻辑备份文件的扩展名。结构化查询语言是高级的非过程化编程语言,允许用户在高层数据结构上工作。它不要求用户指定数据的存放方法,也不需要用户了解具体的数据存放方式,所以,其完全不同于底层结构,不同的数据库系统可以使用相同的结构化查询语言作为数据输入与管理的接口。结构化查询语言语句可以嵌套,这使得它具有极大的灵活性和强大的功能。不同的数据库系统的SQL会有一些差别。


% From text00090.html


\subsection{6.1.2 SQL的分类}


\par 根据百科资料,管理关系型数据库的结构化查询语言包含6个部分,下面就来逐一介绍。

\par 1.数据查询语言(DQL)

\par DQL,全称为Data Query Language,其语句也称为“数据检索语句”,作用是从表中获取数据,确定数据应怎样在应用程序中给出。关键字SELECT是DQL(也是所有SQL)用得最多的,其他DQL常用的保留字有WHERE、ORDER BY、GROUP BY和HAVING。这些DQL保留字常与其他类型的SQL语句一起使用。具体语句示例如下:

\noindent\rule{\textwidth}{0.4pt}


\begin{verbatim}mysql> select user,host from mysql.user order by user;
+------+-----------+
| user | host      |
+------+-----------+
| root | localhost |
| root | 127.0.0.1 |
+------+-----------+
2 rows in set (0.00 sec)
\end{verbatim}


\noindent\rule{\textwidth}{0.4pt}


\par 2.数据操作语言(DML)

\par DML,全称为Data Manipulation Language,中文为数据操作语言。其语句的关键字为INSERT、UPDATE和DELETE。它们分别用于添加、修改和删除表中的行(数据)。具体语句示例如下:

\noindent\rule{\textwidth}{0.4pt}


\begin{verbatim}mysql> delete from mysql.user where user='oldboy';
Query OK, 0 rows affected (0.00 sec)
\end{verbatim}


\noindent\rule{\textwidth}{0.4pt}


\par 3.事务处理语言(TPL)

\par TPL,全称为Transaction Processing Language,TPL语句用于确保被DML语句影响的表的所有行能够及时得到更新。TPL语句包括BEGIN TRANSACTION、COMMIT和ROLLBACK。

\par 4.数据控制语言(DCL)

\par DCL,全称为Data Control Language,这类语句通过GRANT或REVOKE授权用户许可,确定单个用户和用户组对数据库对象的访问。某些RDBMS可用GRANT或REVOKE控制对表中单个列的访问。

\par 5.数据定义语言(DDL)

\par DDL,全称为Data Definition Language,其语句包括动词CREATE、DROP和ALTER。可使用该语言在数据库中创建新库表或删除库表,或者为表添加字段、索引等。

\par 6.指针控制语言(CCL)

\par CCL,全称为CURSOR Control Language,它的语句(像DECLARE CURSOR、FETCH INTO和UPDATE WHERE CURRENT)用于对一个或多个表的单独行进行操作。

\par 而开发人员、DBA以及运维人员常把SQL分为表6-1中所列的三大类。

\par 表6-1 SQL常见分类

\begin{figure}[htbp]
\centering
\includegraphics[width=0.8\textwidth]{Image00033.jpg}
\end{figure}



\begin{figure}[htbp]
\centering
\includegraphics[width=0.8\textwidth]{Image00034.jpg}
\end{figure}




% From text00091.html


\section{6.2 SQL解析原理流程}


\subsection{
6.2.1 MySQL体系结构简介}
}


\par 为了让读者能够清晰地理解SQL语句的执行原理,本节简单介绍下MySQL的基本内部结构,更详细的介绍请参见后文。MySQL的体系结构由外围应用程序(客户端)、连接层、SQL层、存储引擎层等组成,图6-1所示的即为MySQL的基本体系结构。

\begin{figure}[htbp]
\centering
\includegraphics[width=0.8\textwidth]{Image00035.jpg}
\end{figure}



\par 图6-1 MySQL基本体系结构

\par 在图6-1中,应用程序在连接MySQL数据库时,首先要经过连接池技术,然后通过连接池进入SQL解析层,经过SQL解析层的一系列处理及解析后,再进入到存储引擎层去查找数据,最后通过内存或磁盘读取到数据,然后再逐级返回,最终将数据返回给应用程序或用户。这里大家需要了解的是SQL层在体系结构中的位置,正是SQL层解析并读懂SQL语句从而得以读取到数据库的数据,那么SQL层的解析流程到底是怎么样的呢?请看6.2.2节。有关MySQL体系结构的细节请参见后文。


% From text00092.html


\subsection{6.2.2 SQL解析流程介绍}


\par 为了让读者清晰地了解SQL解析的流程原理,老男孩特意用一张图来阐述,图6-2所示的即为SQL解析流程图。

\begin{figure}[htbp]
\centering
\includegraphics[width=0.8\textwidth]{Image00036.jpg}
\end{figure}



\par 图6-2 SQL解析流程图

\par 1)在图6-2中,当接收到SQL语句后,数据库首先会判断SQL语句的正确性、类型(例如DDL、DML、DCL),然后根据不同类型,由命令分发模块投递到相应的模块处理。如果是SELECT语句,则首先会查找SQL高速缓存(query\_cache);如果命中目标数据,则SQL不需要再做相应的解析执行操作,直接将请求的数据返回客户端应用程序即可。

\par 2)如果用户请求的数据未在高速缓存中命中,那么会进入SQL的解析过程。

\par a)对于SQL层本身来说,其是无法直接读懂SQL语句的,不能直接执行。这时,就要靠SQL层的解析器(Parser)来进行SQL的词法、语法分析,最终得出1个或多个SQL语句的执行计划。

\par b)得出执行计划之后,还不能直接使用,因为解析器可能给出了一条SQL语句的多种执行方式,需要进行进一步的判断,那么哪条执行计划是最好的呢?这件事情将交由查询优化器(Optimizer)去判断。优化器会根据自身的算法,找到代价最低(一般是有合理索引的那条)的那个执行计划进行下发。(当然有些情况例外,不过其都是由优化器自身的算法来决定的,我们在将来的章节中会再进行分析,另外在不同的MySQL版本里,优化器中的处理又各不相同。)

\par c)这时我们已经找到合适的执行计划了,是否就可以执行查询了呢?其实SQL层中还有其他相关的模块控制,例如,是否有权限去执行、执行是否需要等待等。

\par d)经过SQL层的逐层处理,SQL终于被执行,然后将执行后的结果,投递到下层的存储引擎接口,最终经过存储引擎,获取到磁盘数据文件上的数据。

\par e)SQL执行完成后,会将此次获取的数据更新到查询缓存(若有特定情况则不会更新)。


% From text00093.html


\section{6.3 SQL语句实践}


\par 为了简化读者的学习环境,本章全部使用第3章讲解的单实例环境进行实践,请读者停掉所有多实例数据库,然后通过“/etc/init.d/mysqld start”启动数据库服务。

\par 学习环境准备的操作过程具体如下。

\par 1)停止多实例数据库并确认已停止:

\noindent\rule{\textwidth}{0.4pt}


\begin{verbatim}[root@oldboy \textasciitilde{}]\# /data/3306/mysql stop
Stoping MySQL...
Warning: Using a password on the command line interface can be insecure.
[root@oldboy \textasciitilde{}]\# /data/3307/mysql stop
Stoping MySQL...
Warning: Using a password on the command line interface can be insecure.
[root@oldboy \textasciitilde{}]\# ss -lnt|grep 330 \#<==执行后没有任何结果,表示多实例环境都已正确关闭。
\end{verbatim}


\noindent\rule{\textwidth}{0.4pt}


\par 2)启动单实例数据库,并确认已启动:

\noindent\rule{\textwidth}{0.4pt}


\begin{verbatim}[root@oldboy \textasciitilde{}]\# /etc/init.d/mysqld start
Starting MySQL. SUCCESS!
[root@oldboy \textasciitilde{}]\# ss -lnt|grep 330  \#<==启动单实例后,执行检查端口命令,若出现下面
                                    的内容则表示单实例环境准备完毕。
LISTEN     0      80                       :::3306                    :::*
\end{verbatim}


\noindent\rule{\textwidth}{0.4pt}



% From text00094.html


\subsection{6.3.1 DDL语句之管理数据库}


\par DDL的特点是对数据库内部的对象进行创建、修改、删除等操作,不涉及对表中内容的操作和更改。这部分是运维人员或DBA需要熟练掌握的内容,开发人员了解即可。

\par 下面就跟随老男孩一起来了解下DDL的常用操作。

\par 1.创建数据库

\par 创建数据库的命令语法为:

\noindent\rule{\textwidth}{0.4pt}


\begin{verbatim}create database <数据库名>;
\end{verbatim}


\noindent\rule{\textwidth}{0.4pt}


\par 
\begin{figure}[htbp]
\centering
\includegraphics[width=0.8\textwidth]{Image00006.jpg}
\end{figure}

 提示:


\par ·注意库名不能以数字开头。

\par ·所有字母对大小写不敏感。

\par 如果把创建数据库比喻为追求女生,那么就是表6-2中的对应关系。

\par 表6-2 建库的信息说明

\begin{figure}[htbp]
\centering
\includegraphics[width=0.8\textwidth]{Image00037.jpg}
\end{figure}



\par 在编译环境下,针对MySQL默认字符集建立数据库测试的步骤如下。

\par 1)建立一个名为oldboy的数据库:

\noindent\rule{\textwidth}{0.4pt}


\begin{verbatim}mysql> create database oldboy;      \#<==创建名为oldboy的数据库。
Query OK, 1 row affected (0.00 sec) \#<==执行后的输出结果,“Query OK”表示成功执行,“1 row affected”表示影响了1行,“(0.00 sec)”为执行的时间。
mysql> show databases;              \#<==查看所有的数据库。
+--------------------+
| Database           |
+--------------------+
| information\_schema | \#<==此表为系统表,存储数据库内置对象的信息,如用户,权限等。
| mysql              | \#<==此表为系统表,存储用户授权和权限相关的信息,授权时会用到。
| oldboy             | \#<==这里就是刚才创建的数据库。
| performance\_schema | \#<==这是MySQL第二条产品线增加的系统表,存储与性能相关的表数据。
+--------------------+
4 rows in set (0.00 sec)
mysql>
\end{verbatim}


\noindent\rule{\textwidth}{0.4pt}


\par 查看刚刚创建的oldboy库对应的SQL语句:

\noindent\rule{\textwidth}{0.4pt}


\begin{verbatim}mysql> help show                      \#<==通过show命令查看数据库语句的帮助。
SHOW CREATE DATABASE db\_name
mysql> show create database oldboy\textbackslash\{\}G  \#<==\textbackslash\{\}G是为了调整显示的格式。
*************************** 1. row ***************************
       Database: oldboy
Create Database: CREATE DATABASE `oldboy` /*!40100 DEFAULT CHARACTER SET utf8 */
\#<==当不指定字符集建库时,库的字符集默认和编译时指定的字符集一致。
1 row in set (0.00 sec)
\end{verbatim}


\noindent\rule{\textwidth}{0.4pt}


\par 2)建立一个GBK字符集数据库,名为oldboy\_gbk,并查看建库语句:

\noindent\rule{\textwidth}{0.4pt}


\begin{verbatim}mysql> create database oldboy\_gbk CHARACTER SET gbk COLLATE gbk\_chinese\_ci;
Query OK, 1 row affected (0.01 sec)
mysql> show create database oldboy\_gbk\textbackslash\{\}G
*************************** 1. row ***************************
       Database: oldboy\_gbk
Create Database: CREATE DATABASE `oldboy\_gbk` /*!40100 DEFAULT CHARACTER SET gbk */
\#<==字符集确实改为了gbk。
1 row in set (0.00 sec)
\end{verbatim}


\noindent\rule{\textwidth}{0.4pt}


\par 3)建立一个名为oldboy\_utf8的UTF8数据库:

\noindent\rule{\textwidth}{0.4pt}


\begin{verbatim}mysql> show create database oldboy\_utf8\textbackslash\{\}G
*************************** 1. row ***************************
       Database: oldboy\_utf8
Create Database: CREATE DATABASE `oldboy\_utf8` /*!40100 DEFAULT CHARACTER SET utf8 */
\#<==字符集确实改为了utf8了。
1 row in set (0.00 sec)
\end{verbatim}


\noindent\rule{\textwidth}{0.4pt}


\par 4)创建不同字符集格式的数据库命令集合:

\noindent\rule{\textwidth}{0.4pt}


\begin{verbatim}create database oldboy; \#<==默认数据库的字符集设置配置。
create database oldboy\_gbk DEFAULT CHARACTER SET gbk COLLATE gbk\_chinese\_ci;
                        \#<==创建gbk字符集数据库。
create database oldboy\_utf8 DEFAULT CHARACTER SET utf8  COLLATE utf8\_general\_ci;
                      \#<==创建utf8字符集数据库。
\end{verbatim}


\noindent\rule{\textwidth}{0.4pt}


\par 
\begin{figure}[htbp]
\centering
\includegraphics[width=0.8\textwidth]{Image00006.jpg}
\end{figure}

 提示:


\par ·字符集的不一致是数据库里中文内容出现乱码的罪魁祸首,有关字符集的知识详见后文。

\par ·查看字符集及校对规则名字(指定字符集建库结尾的语句,例如SET gbk COLLATE gbk\_chinese\_ci)的方法为“mysql>SHOW CHARACTER SET;”。

\par 到了这里,有必要了解一下有关数据库字符集的知识了。

\par 如果编译的时候指定了特定的字符集,则以后创建对应字符集的数据库时就不需要指定字符集了。示例如下:

\noindent\rule{\textwidth}{0.4pt}


\begin{verbatim}cmake . -DCMAKE\_INSTALL\_PREFIX=/application/mysql-5.6.40 \textbackslash\{\}
-DMYSQL\_DATADIR=/application/mysql-5.6.40/data \textbackslash\{\}
-DMYSQL\_UNIX\_ADDR=/application/mysql-5.6.40/tmp/mysql.sock \textbackslash\{\}
-DDEFAULT\_CHARSET=utf8 \textbackslash\{\}              \#<==这里就是建库时指定的默认数据库字符集。
-DDEFAULT\_COLLATION=utf8\_general\_ci \textbackslash\{\} \#<==这里就是建库时指定的默认数据库校对规则。
\end{verbatim}


\noindent\rule{\textwidth}{0.4pt}


\par 
\begin{figure}[htbp]
\centering
\includegraphics[width=0.8\textwidth]{Image00006.jpg}
\end{figure}

 提示:
 使用MySQL二进制包安装的数据库字符集默认为latin1!

\par 
那么,在企业里的不同场景下,要如何创建符合字符集需求的数据库呢?


\par 可根据公司开发的程序确定字符集(建议选择UTF8,移动互联网环境下可选utf8mb4)!也可以在编译的时候指定字符集,例如:

\noindent\rule{\textwidth}{0.4pt}


\begin{verbatim}-DDEFAULT\_CHARSET=utf8 \textbackslash\{\}
-DDEFAULT\_COLLATION=utf8\_general\_ci \textbackslash\{\}
\end{verbatim}


\noindent\rule{\textwidth}{0.4pt}


\par 此时,建库的时候默认创建即可,例如“create database oldboy;”。

\par 如果编译的时候没有指定字符集或者指定了与网站程序不同的字符集,又该如何解决?

\par 这时指定字符集创建数据库即可,也可以在数据库配置文件里,修改默认的字符集显示,具体见后文说明。

\par 2.显示数据库

\par 显示数据库,前文已经出现过,最基本的命令就是:

\noindent\rule{\textwidth}{0.4pt}


\begin{verbatim}show databases;
\end{verbatim}


\noindent\rule{\textwidth}{0.4pt}


\par 可执行help show查看与该命令对应的帮助信息。

\noindent\rule{\textwidth}{0.4pt}


\begin{verbatim}mysql> help show
...省略若干...
SHOW CREATE VIEW view\_name
SHOW DATABASES [like\_or\_where]  \#<==这就是显示数据库的命令帮助信息。
SHOW ENGINE engine\_name \{STATUS | MUTEX\}
SHOW [STORAGE] ENGINES
...省略若干...
\end{verbatim}


\noindent\rule{\textwidth}{0.4pt}


\par 如果不知道本节所讲的关键命令show,那么可以使用“help contents”命令,然后执行“help Administration”来查看。

\par 下面来看个使用show命令的示例:

\noindent\rule{\textwidth}{0.4pt}


\begin{verbatim}mysql> show databases like 'oldboy'; \#<==匹配oldboy字符串的内容。
+-------------------+
| Database (oldboy) |
+-------------------+
| oldboy            |
+-------------------+
1 row in set (0.00 sec)

mysql> show databases like 'oldboy\%'; \#<==\%为通配符,表示匹配以oldboy开头的所有内容。
+--------------------+
| Database (oldboy\%) |
+--------------------+
| oldboy             |
| oldboy\_gbk         |
| oldboy\_utf8        |
+--------------------+
3 rows in set (0.00 sec)
\end{verbatim}


\noindent\rule{\textwidth}{0.4pt}


\par 3.切换数据库

\par 所谓的切换库,就相当于系统中的切换路径一样,只不过,系统中使用的是类似于“cd/etc”的命令,而数据库里切换库使用的是use命令。

\par 命令语法为:

\noindent\rule{\textwidth}{0.4pt}


\begin{verbatim} use  <数据库名>;
\end{verbatim}


\noindent\rule{\textwidth}{0.4pt}


\par 如果oldboy数据库存在,则可通过如下命令尝试进入oldboy数据库里:

\noindent\rule{\textwidth}{0.4pt}


\begin{verbatim}mysql> select database(); \#<==查看当前管理员所在的库名。
+------------+
| database() |
+------------+
| NULL       |            \#<==空值,表示没有在任何库里。
+------------+
1 row in set (0.00 sec)

mysql> use oldboy         \#<==切换到oldboy库里。
Database changed
mysql> select database(); \#<==重新查看。
+------------+
| database() |
+------------+
| oldboy     |            \#<==表示存在于oldboy库里了。
+------------+
1 row in set (0.00 sec)
mysql>
\end{verbatim}


\noindent\rule{\textwidth}{0.4pt}


\par 4.查看数据库包含的表信息

\par 要查看数据库包含的表信息有两种常见的方法,第一种是切到数据库里面去查看,第二种是在外面查看库里的表信息。下面分别来看看。

\par 第一种方法:切到数据库里面去查看表信息。

\noindent\rule{\textwidth}{0.4pt}


\begin{verbatim}mysql> use oldboy
mysql> show tables;
Empty set (0.00 sec)  \#<==空表,因为是新库,还没有建立表。
\end{verbatim}


\noindent\rule{\textwidth}{0.4pt}


\par 第二种方法:在库外面查看库里的表信息。

\noindent\rule{\textwidth}{0.4pt}


\begin{verbatim}mysql> show tables from oldboy;            \#<==查看指定库oldboy中包含的表。
Empty set (0.00 sec)
mysql> show tables in oldboy\_gbk;
Empty set (0.00 sec)
mysql> show tables from mysql like 'db\%';  \#<==还可以匹配包含指定字符开头的表。
+-----------------------+
| Tables\_in\_mysql (db\%) |
+-----------------------+
| db                    |
+-----------------------+
1 row in set (0.01 sec)
\end{verbatim}


\noindent\rule{\textwidth}{0.4pt}


\par 如果不知道这里使用的命令show,则可以通过“help contents”命令,进而执行“help Administration”命令来查看。

\par 5.删除数据库

\par 命令语法为:

\noindent\rule{\textwidth}{0.4pt}


\begin{verbatim}drop database <数据库名>;
\end{verbatim}


\noindent\rule{\textwidth}{0.4pt}


\par 以下示例用于删除名为oldboy\_gbk的数据库:

\noindent\rule{\textwidth}{0.4pt}


\begin{verbatim}mysql> drop database oldboy\_gbk;
Query OK, 0 rows affected (0.02 sec)
mysql> show databases like 'oldboy\_gbk';
Empty set (0.00 sec)
\end{verbatim}


\noindent\rule{\textwidth}{0.4pt}


\par 学习时遇到忘记命令或命令用法时就可以随时查看帮助和提示,例如,若不知道此题的命令是什么,则可以直接执行下面的命令查看帮助:

\noindent\rule{\textwidth}{0.4pt}


\begin{verbatim}mysql> help drop database
\end{verbatim}


\noindent\rule{\textwidth}{0.4pt}


\par 当然如果连drop都不知道的话,就只能执行“help contents”,然后找到“Data Definition”来查看了,“Data Definition”的意思是数据定义,可想而知,“help Data Definition”就是用于查询具体的命令了。


% From text00095.html


\subsection{6.3.2 DDL&&DCL语句之管理用户}


\par 1.查看当前数据库的用户列表

\par 查看数据库用户列表属于DML负责的部分内容,此处为了方便读者理解,提前进行讲解。

\par 查看当前数据库用户列表的命令如下:

\noindent\rule{\textwidth}{0.4pt}


\begin{verbatim}mysql> select user,host from mysql.user;
+------+-----------+
| user | host      |
+------+-----------+
| root | 127.0.0.1 | \#<==数据库核心保留管理员用户。
| root | localhost | \#<==数据库核心保留管理员用户。
+------+-----------+
2 rows in set (0.00 sec)
\end{verbatim}


\noindent\rule{\textwidth}{0.4pt}


\par 这里的select关键字表示查询,是DML语句的关键字之一,user和hosts为要查找的mysql表的字段,from表示去哪查,mysql.user是mysql库里的user表,select语句的更多知识请见后文。此外,数据库的标准用户由“用户”@“主机名”共同组成的,两者加起来是数据库用户的唯一标识。

\par 此前我们安装完数据库之后已经做了基础优化,因此,现在看到的数据库用户就只剩下两个了。

\par 2.创建数据库用户

\par 创建数据库用户的语法和创建数据库的语法差不多,都使用的是关键字create。表6-3针对该语法进行了说明。

\noindent\rule{\textwidth}{0.4pt}


\begin{verbatim}CREATE USER '用户'@'主机' IDENTIFIED BY '密码';
\end{verbatim}


\noindent\rule{\textwidth}{0.4pt}


\par 表6-3 创建用户表格

\begin{figure}[htbp]
\centering
\includegraphics[width=0.8\textwidth]{Image00038.jpg}
\end{figure}



\par 现在创建一个blog用户,只允许本地主机登录,密码是blog123。

\noindent\rule{\textwidth}{0.4pt}


\begin{verbatim}mysql> create user blog@localhost identified by 'blog123'; \#<==有时为了方便会省略单引号,建议尽量不要省略。
Query OK, 0 rows affected (0.01 sec)
mysql> select user,host from mysql.user;
+------+-----------+
| user | host      |
+------+-----------+
| root | 127.0.0.1 |
| blog | localhost | \#<==这样blog新用户就建立了。
| root | localhost |
+------+-----------+
3 rows in set (0.00 sec)
\end{verbatim}


\noindent\rule{\textwidth}{0.4pt}


\par 企业里创建用户一般是授权一个内网网段登录,最常见的网段写法有两种。

\par 方法1:172.16.1.\%(\%为通配符,匹配所有的内容)。

\par 方法2:172.16.1.0/255.255.255.0,但是不能使用172.16.1.0/24,这是个小遗憾。

\par 下面创建一个用户bbs,授权172.16.1.0/24网段内机器访问:

\noindent\rule{\textwidth}{0.4pt}


\begin{verbatim}mysql> create user bbs@'172.16.1.\%' identified by 'bbs123';
Query OK, 0 rows affected (0.00 sec)
mysql> select user,host from mysql.user where user='bbs';
+------+----------+
| user | host     |
+------+----------+
| bbs  | 172.16.1.\% | \#<==这样bbs新用户就建立了。
+------+----------+
1 row in set (0.00 sec)
mysql>quit
\end{verbatim}


\noindent\rule{\textwidth}{0.4pt}


\par 登录测试:

\noindent\rule{\textwidth}{0.4pt}


\begin{verbatim}[root@oldboy \textasciitilde{}]\# mysql -ublog -pblog123              \#<==数据库服务器本机上登录。
[root@oldboy \textasciitilde{}]\# mysql -ubbs -pbbs123 -h 172.16.1.51 \#<==异地或本机登录bbs用户
                                                 必须要指定IP。
\end{verbatim}


\noindent\rule{\textwidth}{0.4pt}


\par 若授权主机不对,就会无法登录,这是MySQL的重要安全手段之一。

\par 需要特别强调的一点是,使用create创建的用户仅仅是空用户,即除了可以连接数据库之外,其没有任何数据库权限,有关用户授权还必须要使用grant命令。当然,grant命令可以同时完成创建用户和授权两种操作,因此,很多管理员使用grant替代create来创建用户。下面来看个示例:

\noindent\rule{\textwidth}{0.4pt}


\begin{verbatim}mysql> show grants for bbs@'172.16.1.\%'; \#<==查看用户对应的授权权限。
+------------------------------------------------------------------------+
| Grants for bbs@172.16.1.\%                                                |
+------------------------------------------------------------------------+
| GRANT USAGE ON *.* TO 'bbs'@'172.16.1.\%' IDENTIFIED BY PASSWORD <secret> |
\#<==USAGE表示连接权限。
+------------------------------------------------------------------------+
1 row in set (0.00 sec)
\end{verbatim}


\noindent\rule{\textwidth}{0.4pt}


\par 3.删除数据库用户

\par 删除数据库用户的语法如下:

\noindent\rule{\textwidth}{0.4pt}


\begin{verbatim}drop user 'user'@'主机域';
\end{verbatim}


\noindent\rule{\textwidth}{0.4pt}


\par 注意这里面的引号,可以是单引号也可以是双引号,不过最好是用单引号,这也是规范中建议的。

\par 这个命令的意思与创建用户极其类似,具体见表6-4。

\par 表6-4 删除数据库语法信息解释

\begin{figure}[htbp]
\centering
\includegraphics[width=0.8\textwidth]{Image00039.jpg}
\end{figure}



\par 下面使用该命令删除上述例子中的blog用户:

\noindent\rule{\textwidth}{0.4pt}


\begin{verbatim}mysql> select user,host from mysql.user where user='blog'; \#<==删除前检查。
+------+----------+
| user | host     |
+------+----------+
| blog |localhost |
+------+----------+
1 row in set (0.00 sec)
mysql> drop user 'blog'@'localhost';                        \#<==删除命令。
Query OK, 0 rows affected (0.00 sec)
mysql> select user,host from mysql.user where user='blog';  \#<==删除后检查。
Empty set (0.00 sec)     \#<==删除了。
mysql> flush privileges; \#<==使得处理用户后,对数据库生效,一般有数据库改动的情况,
                           最好执行这个固定命令。
Query OK, 0 rows affected (0.00 sec)
\end{verbatim}


\noindent\rule{\textwidth}{0.4pt}


\par 如果读者使用drop删除不了用户,则很可能是因为用户或主机部分是特殊字符或大写内容等,此时可以使用下面的方式删除,以bbs用户、172.16.1.\%主机为例,具体处理命令及操作过程如下:

\noindent\rule{\textwidth}{0.4pt}


\begin{verbatim}mysql> select user,host from mysql.user where user='bbs';
+------+----------+
| user | host     |
+------+----------+
| bbs  | 172.16.1.\% |
+------+----------+
1 row in set (0.00 sec)
mysql> delete from mysql.user where  user='bbs' and host='172.16.1.\%';
Query OK, 1 row affected (0.00 sec)
mysql> flush privileges;
Query OK, 0 rows affected (0.00 sec)
mysql> select user,host from mysql.user where user='bbs';
Empty set (0.00 sec)
mysql>
\end{verbatim}


\noindent\rule{\textwidth}{0.4pt}


\par 4.授权数据库用户

\par 查看grant的命令帮助,可以很容易地找到创建用户并授权的例子!在MySQL中输入“help grant”即可得到如下帮助信息:

\noindent\rule{\textwidth}{0.4pt}


\begin{verbatim}mysql> help grant
…省略部分…
CREATE USER 'jeffrey'@'localhost' IDENTIFIED BY 'mypass';
GRANT ALL ON db1.* TO 'jeffrey'@'localhost';
GRANT SELECT ON db2.invoice TO 'jeffrey'@'localhost';
GRANT USAGE ON *.* TO 'jeffrey'@'localhost' WITH MAX\_QUERIES\_PER\_HOUR 90;
…省略部分…
\end{verbatim}


\noindent\rule{\textwidth}{0.4pt}


\par 运维人员比较常用的创建用户的方法是,使用grant命令在创建用户的同时进行权限授权。下面列举一个授权的例子:

\noindent\rule{\textwidth}{0.4pt}


\begin{verbatim}GRANT ALL ON  db1.* TO 'jeffrey'@'localhost' IDENTIFIED BY 'mypass ';
\end{verbatim}


\noindent\rule{\textwidth}{0.4pt}


\par 授权用户的grant命令简单语法为:

\noindent\rule{\textwidth}{0.4pt}


\begin{verbatim}grant all privileges on dbname.* to username@localhost identified by 'passwd';
\end{verbatim}


\noindent\rule{\textwidth}{0.4pt}


\par 上述命令是授权localhost主机上通过用户username管理dbname数据库的所有权限,密码为passwd。其中username、dbname、passwd可根据业务的情况分别修改。表6-5针对语法内容进行了说明。

\par 表6-5 grant语法解释

\begin{figure}[htbp]
\centering
\includegraphics[width=0.8\textwidth]{Image00040.jpg}
\end{figure}



\par 操作案例1:创建test用户,对oldboy库具备所有权限,允许从localhost主机登录管理数据库,密码是test123。

\par 实现上述操作的具体命令为:

\noindent\rule{\textwidth}{0.4pt}


\begin{verbatim}grant all privileges on oldboy.* to 'test'@'localhost' identified by 'test123';
\end{verbatim}


\noindent\rule{\textwidth}{0.4pt}


\par 操作过程如下。

\par 查看当前数据库的用户情况,然后执行对应命令授权,示例如下:

\noindent\rule{\textwidth}{0.4pt}


\begin{verbatim}mysql> grant all privileges on oldboy.* to 'test'@'localhost' identified by 'test123';
Query OK, 0 rows affected (0.01 sec)
mysql> select user,host from mysql.user where user='test';
+------+-----------+
| user | host      |
+------+-----------+
| test | localhost |  \#<==test用户已经建立了。
+------+-----------+
1 row in set (0.00 sec)
\end{verbatim}


\noindent\rule{\textwidth}{0.4pt}


\par 查看授权用户test的具体权限,命令如下:

\noindent\rule{\textwidth}{0.4pt}


\begin{verbatim}mysql> show grants for 'test'@'localhost';
+------------------------------------------------------------------------+
| Grants for test@localhost                                              |
+------------------------------------------------------------------------+
| GRANT USAGE ON *.* TO 'test'@'localhost' IDENTIFIED BY PASSWORD '*676243218923905CF94CB52A3C9D3EB30CE8E20D'                                  |
| GRANT ALL PRIVILEGES ON `oldboy`.* TO 'test'@'localhost'
  \#<==这里就是授权的权限,注意授权与上文中create创建用户的区别。                 |
+------------------------------------------------------------------------+
2 rows in set (0.00 sec)
\end{verbatim}


\noindent\rule{\textwidth}{0.4pt}


\par 操作案例2:授权与root同等地位的system用户权限,如果读者有洁癖,创建完成就可以删除root用户了。

\noindent\rule{\textwidth}{0.4pt}


\begin{verbatim}mysql> show grants for root@localhost;  \#<==先查看root用户的权限。
+------------------------------------------------------------------------+
| Grants for root@localhost                                              |
+------------------------------------------------------------------------+
| GRANT ALL PRIVILEGES ON *.* TO 'root'@'localhost' IDENTIFIED BY PASSWORD '*FE28814B4A8B3309DAC6ED7D3237ADED6DA1E515' WITH GRANT OPTION                |
| GRANT PROXY ON ''@'' TO 'root'@'localhost' WITH GRANT OPTION           |
+------------------------------------------------------------------------+
2 rows in set (0.00 sec)
\end{verbatim}


\noindent\rule{\textwidth}{0.4pt}


\par 创建一个权限和root一样大的system用户,with grant option
 是下面命令的重点:

\noindent\rule{\textwidth}{0.4pt}


\begin{verbatim}mysql> grant all on *.* to 'system'@'localhost' identified by 'system123' with grant option;
Query OK, 0 rows affected (0.00 sec)
mysql> show grants for system@localhost;
+------------------------------------------------------------------------+
| Grants for system@localhost                                            |
+------------------------------------------------------------------------+
| GRANT ALL PRIVILEGES ON *.* TO 'system'@'localhost' IDENTIFIED BY PASSWORD '*B2A09C1B789FF64A8C3C70A4EC1CA78F34C635B7' WITH GRANT OPTION                |
+------------------------------------------------------------------------+
1 row in set (0.00 sec)

mysql> GRANT PROXY ON ''@'' TO 'system'@'localhost' WITH GRANT OPTION;
\#<==允许创建代理用户。
Query OK, 0 rows affected (0.00 sec)

mysql> show grants for system@localhost;
+------------------------------------------------------------------------+
| Grants for system@localhost                                            |
+------------------------------------------------------------------------+
| GRANT ALL PRIVILEGES ON *.* TO 'system'@'localhost' IDENTIFIED BY PASSWORD '*B2A09C1B789FF64A8C3C70A4EC1CA78F34C635B7' WITH GRANT OPTION                |
| GRANT PROXY ON ''@'' TO 'system'@'localhost' WITH GRANT OPTION         |
+------------------------------------------------------------------------+
2 rows in set (0.00 sec)
\end{verbatim}


\noindent\rule{\textwidth}{0.4pt}


\par 此外,还可以使用create和grant命令来配合完成授权操作。具体步骤如下。

\par 首先,创建用户username及密码passwd,授权主机localhost:

\noindent\rule{\textwidth}{0.4pt}


\begin{verbatim}CREATE USER 'username'@'localhost' IDENTIFIED BY 'passwd';
\end{verbatim}


\noindent\rule{\textwidth}{0.4pt}


\par 然后,授权localhost主机上通过用户username管理dbname数据库的所有权限,无须密码:

\noindent\rule{\textwidth}{0.4pt}


\begin{verbatim}GRANT ALL ON dbname.* TO 'username'@'localhost';
\end{verbatim}


\noindent\rule{\textwidth}{0.4pt}


\par 若是要授权局域网内的主机远程连接数据库,根据grant命令的语法,我们知道test@'localhost'位置为授权访问数据库的用户和主机,其中localhost可以用域名、IP地址或IP段来替代,因此,要授权局域网内的主机可以通过如下方法来实现。

\par 第一种方法:一条命令+百分号的匹配法,示例如下:

\noindent\rule{\textwidth}{0.4pt}


\begin{verbatim}mysql> grant all on *.* to test@'172.16.1.\%' identified by 'test123';
Query OK, 0 rows affected (0.00 sec)
\end{verbatim}


\noindent\rule{\textwidth}{0.4pt}


\par 第二种方法:一条命令+子网掩码的配置法,示例如下:

\noindent\rule{\textwidth}{0.4pt}


\begin{verbatim}mysql> grant all on *.* to test@'172.16.1.0/255.255.255.0' identified by 'test123';
Query OK, 0 rows affected (0.00 sec)
\end{verbatim}


\noindent\rule{\textwidth}{0.4pt}


\par 
\begin{figure}[htbp]
\centering
\includegraphics[width=0.8\textwidth]{Image00006.jpg}
\end{figure}

 提示:
 不能用test@'172.16.1.0/24'替代test@'172.16.1.0/255.255.255.0'

\par 第三种方法:通过两条命令来实现,具体如下。

\par 首先,创建用户并设置密码:

\noindent\rule{\textwidth}{0.4pt}


\begin{verbatim}mysql> create user test@'172.16.1.\%' identified by 'test123';
Query OK, 0 rows affected (0.00 sec)
\end{verbatim}


\noindent\rule{\textwidth}{0.4pt}


\par 然后,对用户授权指定权限和管理的库表,这样即可实现:

\noindent\rule{\textwidth}{0.4pt}


\begin{verbatim}mysql> grant all on *.* to test@'172.16.1.0/255.255.255.0';
Query OK, 0 rows affected (0.00 sec)
\end{verbatim}


\noindent\rule{\textwidth}{0.4pt}


\par 不过,要记得对上述每条grant命令最好都刷新权限(即使添加用户不需要,养成良好的一致性操作习惯也很重要):

\noindent\rule{\textwidth}{0.4pt}


\begin{verbatim}mysql> flush privileges;
Query OK, 0 rows affected (0.00 sec)
\end{verbatim}


\noindent\rule{\textwidth}{0.4pt}


\par 5.授权的权限列表

\par 可通过实验获得ALL PRIVILEGES都包括哪些权限。

\par 
\begin{figure}[htbp]
\centering
\includegraphics[width=0.8\textwidth]{Image00024.jpg}
\end{figure}

 说明:
 自己动手,丰衣足食,比到处问强多了,这是初学者(同学们)需要重点加强提高的地方。

\par 首先,看看前面授权过的test用户的权限,命令如下:

\noindent\rule{\textwidth}{0.4pt}


\begin{verbatim}mysql> show grants for 'test'@'localhost';
+------------------------------------------------------------------------+
| Grants for test@localhost                                              |
+------------------------------------------------------------------------+
| GRANT USAGE ON *.* TO 'test'@'localhost' IDENTIFIED BY PASSWORD '*676243218923905CF94CB52A3C9D3EB30CE8E20D'                                  |
| GRANT ALL PRIVILEGES ON `oldboy`.* TO 'test'@'localhost'
  \#<==这里就是授权的权限,注意授权与上文中create创建用户的区别。                 |
+------------------------------------------------------------------------+
2 rows in set (0.00 sec)
\end{verbatim}


\noindent\rule{\textwidth}{0.4pt}


\par 此时查看的还是ALL PRIVILEGES权限,但并未细分。

\par 然后,取消test用户的只读权限(SELECT),看看会是什么情况,示例如下:

\noindent\rule{\textwidth}{0.4pt}


\begin{verbatim}mysql> REVOKE SELECT ON oldboy.* FROM 'test'@'localhost';
Query OK, 0 rows affected (0.00 sec)

mysql> REVOKE SELECT ON oldboy.* FROM 'test'@'localhost';
Query OK, 0 rows affected (0.00 sec)
mysql> show grants for 'test'@'localhost';
+------------------------------------------------------------------------+
|Grants for test@localhost                                               |
+------------------------------------------------------------------------+
| GRANT USAGE ON *.* TO 'test'@'localhost' IDENTIFIED BY PASSWORD '*676243218923905CF94CB52A3C9D3EB8'                                          |
| GRANT INSERT, UPDATE, DELETE, CREATE, DROP, REFERENCES, INDEX, ALTER, CREATE TEMPORARY TABLES, LOCK TABLES, EXECUTE, CREATE VIEW, SHOW VIEW, CREATE ROUTINE, ALTER ROUTINE, EVENT, TRIGGER ON `oldboy`.* TO 'test'@'localhost'   |
\#<==看到了吧,权限all被拆分成了更细的权限。
+------------------------------------------------------------------------+
2 rows in set (0.00 sec)
mysql> select * from mysql.db where user='test' and host='localhost'\textbackslash\{\}G
\#<==去db表里查看用户权限。
*************************** 1. row ***************************
                 Host: localhost
                   Db: oldboy
                 User: test
          Select\_priv: N \#<==这就是刚刚用revoke命令收回的权限。
          Insert\_priv: Y
          Update\_priv: Y
          Delete\_priv: Y
          Create\_priv: Y
            Drop\_priv: Y
           Grant\_priv: N \#<==这个权限是N,为什么?这是因为创建用户all里没有授权
                             grant权限。
      References\_priv: Y
           Index\_priv: Y
           Alter\_priv: Y
Create\_tmp\_table\_priv: Y
     Lock\_tables\_priv: Y
     Create\_view\_priv: Y
       Show\_view\_priv: Y
  Create\_routine\_priv: Y
   Alter\_routine\_priv: Y
         Execute\_priv: Y
           Event\_priv: Y
         Trigger\_priv: Y
1 row in set (0.01 sec)
mysql>
\end{verbatim}


\noindent\rule{\textwidth}{0.4pt}


\par 此时我们再查看test用户权限,ALL PRIVILEGES权限已经被细分,但是没有select权限了。

\par 因此可以说,ALL PRIVILEGES的权限至少包括:SELECT、INSERT、UPDATE、DELETE、CREATE、DROP、REFERENCES、INDEX、ALTER、CREATE TEMPORARY TABLES、LOCK TABLES、EXECUTE、CREATE VIEW、SHOW VIEW、CREATE ROUTINE、ALTER ROUTINE、EVENT、TRIGGER等。

\par 表6-6给出了MySQL的ALL PRIVILEGES的权限情况。

\par 表6-6 MySQL的ALL PRIVILEGES的权限列表

\begin{figure}[htbp]
\centering
\includegraphics[width=0.8\textwidth]{Image00041.jpg}
\end{figure}



\par 工作中授权时,授权用户应尽量授权为最小的满足业务需求的权限,而不是直接授权为“ALL PRIVILEGES”。

\par 6.企业中grant授权权限问题说明

\par (1)企业里主数据库用户的授权问题说明

\par 在企业生产环境中,如果是以Web形式连接数据库的用户,那么尽量不要授予all权限,最好是分拆授权,比如,授予select、insert、update、delete等适合Web使用的DML语句关键字权限。下面示例中授予的权限就比较规范合理,当然,对于不同的企业,要具体问题具体分析。

\noindent\rule{\textwidth}{0.4pt}


\begin{verbatim}grant select,insert,update,delete on oldboy.* to test@'172.16.1.\%' identified by 'test123';
\end{verbatim}


\noindent\rule{\textwidth}{0.4pt}


\par 
\begin{figure}[htbp]
\centering
\includegraphics[width=0.8\textwidth]{Image00019.jpg}
\end{figure}

 注意:
 授予用户权限时有如下3条安全红线不要轻易跨过。

\par ·权限不能用all,而应用select、insert、update、delete等具体权限。

\par ·库不能用“*.*”,而应用“oldboy.*”格式具体到库。

\par ·主机不能用\%,而应用内网IP段,即'172.16.1.\%'格式。

\par PHP程序语言连接MySQL的简易程序代码如下:

\noindent\rule{\textwidth}{0.4pt}


\begin{verbatim}<?php
  //\$link\_id=mysql\_connect('数据库主机名','用户','密码');
    \$link\_id=mysql\_connect('172.16.1.7','test','test123') or mysql\_error();
    if(\$link\_id)\{
        echo "mysql successful by oldboy !";
    \}else\{
        echo mysql\_error();
    \}
?>
\end{verbatim}


\noindent\rule{\textwidth}{0.4pt}


\par (2)博客、CMS、BBS等产品的数据库授权

\par 前面说过,采用Web形式连接数据库的用户应尽量采用最小化原则进行授权,但是很多开源软件都是通过Web界面安装的,因此,在安装期间除了select、insert、update、delete这4个权限之外,有可能还需要
 create、drop等比较危险的权限。针对这种情况,需要建库、建表,授权例子如下:

\noindent\rule{\textwidth}{0.4pt}


\begin{verbatim}mysql> grant select,insert,update,delete,create,drop on blog.* to 'blog '@'172.16.1.\%' identified by 'blog123';
Query OK, 0 rows affected (0.00 sec)
\end{verbatim}


\noindent\rule{\textwidth}{0.4pt}


\par 生成数据库、表后,可以使用revoke命令收回create、drop授权:

\noindent\rule{\textwidth}{0.4pt}


\begin{verbatim}mysql> revoke create,drop on blog.* from 'blog'@'172.16.1.\%';
\#<==粗体部分一定要对上,否则会收不回来。
Query OK, 0 rows affected (0.00 sec)

mysql> show grants for 'blog'@'172.16.1.\%';
+------------------------------------------------------------------------+
| Grants for blog@172.16.1.\%                                             |
+------------------------------------------------------------------------+
| GRANT USAGE ON *.* TO 'blog'@'172.16.1.\%' IDENTIFIED BY PASSWORD '*D6.B..A'|
| GRANT SELECT, INSERT, UPDATE, DELETE ON `blog`.* TO 'blog'@'172.16.1.\%'|
+------------------------------------------------------------------------+
2 rows in set (0.00 sec)
\end{verbatim}


\noindent\rule{\textwidth}{0.4pt}


\par (3)生产环境针对主库(写为主读为辅)用户的授权策略

\par 如果是单机,即Web或应用程序和数据库在一台电脑上的数据库授权,则可以采用如下命令:

\noindent\rule{\textwidth}{0.4pt}


\begin{verbatim}GRANT all privileges ON `blog`.* TO 'blog'@'localhost' identified by 'blog123';
\end{verbatim}


\noindent\rule{\textwidth}{0.4pt}


\par 如果应用程序服务器和数据库服务器不在一个主机上,则授权命令如下:

\noindent\rule{\textwidth}{0.4pt}


\begin{verbatim}GRANT all privileges ON `blog`.* TO 'blog'@'172.16.1.\%' identified by 'oldboy456';
\end{verbatim}


\noindent\rule{\textwidth}{0.4pt}


\par 由于工作中异机环境比较多,因此下面都是针对异机情况进行说明的。

\par 下面的命令为严格授权,使用该命令虽然重视了安全,但却忽略了方便。

\noindent\rule{\textwidth}{0.4pt}


\begin{verbatim}GRANT SELECT, INSERT, UPDATE, DELETE ON `blog`.* TO 'blog'@'172.16.1.\%' identified by 'blog123';
\end{verbatim}


\noindent\rule{\textwidth}{0.4pt}


\par (4)生产环境从库(只读)用户的授权

\par 授权命令如下:

\noindent\rule{\textwidth}{0.4pt}


\begin{verbatim}GRANT SELECT ON `blog`.* TO 'blog'@'172.16.1.\%' identified by 'blog123';
\end{verbatim}


\noindent\rule{\textwidth}{0.4pt}


\par 这里表示为172.16.1.0/24的用户blog授予管理blog数据库中所有表(*表示所有表)的只读权限(SELECT),密码为blog123。

\par (5)生产环境主从库高级授权策略

\par 针对这种情况,有两种授权形式,具体如下。

\par 第一种,使用简单方法,见表6-7。

\par 表6-7 用户和密码相同,只有IP不同的表格

\begin{figure}[htbp]
\centering
\includegraphics[width=0.8\textwidth]{Image00042.jpg}
\end{figure}



\par 第二种,配置简单方法,见表6-8。

\par 表6-8 用户不同,IP不同的表格

\begin{figure}[htbp]
\centering
\includegraphics[width=0.8\textwidth]{Image00043.jpg}
\end{figure}



\par 显然,第一种授权方法是最专业的,第二种方法给开发者的感觉是有多个用户,不专业。

\par (6)生产场景下的具体授权

\par 主库授权的命令为:

\noindent\rule{\textwidth}{0.4pt}


\begin{verbatim}GRANT SELECT, INSERT, UPDATE, DELETE ON `blog`.* TO 'blog'@'172.16.1.\%' identified by 'blog123';
\end{verbatim}


\noindent\rule{\textwidth}{0.4pt}


\par 从库授权用户的命令为:

\noindent\rule{\textwidth}{0.4pt}


\begin{verbatim}GRANT SELECT ON `blog`.* TO 'blog'@'172.16.1.\%' identified by 'blog123';
\end{verbatim}


\noindent\rule{\textwidth}{0.4pt}


\par 当然,从库除了做SELECT的授权之外,还可以加read-only等只读参数,严格控制Web用户写从库。

\par (7)生产场景下,主从库读写分离授权难点与解决方案

\par 若主从库的mysql库和表是同步的,则会无法针对同一个用户授权不同的权限。主库授权后会自动同步到从库上,导致从库的授权只读失败。那么这种情况该怎么办呢?

\par 解决方法有如下几点。

\par ·取消数据库中mysql库的同步功能。

\par ·授权主库权限后,从库执行收回增删改权限,只保留查的权限。

\par ·不在授权上控制增删改,而是用read-only参数控制普通用户更新从库。注意,read-only参数对超级用户无效。

\par 这几个解决方法的具体策略在主从复制章节中会有详细讲解。

\par (8)授权不规范导致的生产血案

\par 以下案例是2007年老男孩群里网友讲述的真实案例。

\par 运维人员授权用户all权限,导致开发人员通过该用户自行修改了表结构(字段),造成服务出问题,最后黑锅甩在了运维人员身上。

\par 运维人员排查了半天也没有结果,终于在对比表结构(对比生产数据和备份的数据)的时候发现了问题,最后告诉开发人员,把字段改回去,服务就好了。

\par 
启发:
 生产场景下尽量不要给开发人员select以外的权限,对于网站的连接账号,不要授予select、insert、delete、update以外的权限。对别人的“仁慈”,就是对自己的岗位和公司最大的“背叛”。

\par 7.查看用户及授权

\par 可通过如下命令查看MySQL数据库中的用户和主机信息:

\noindent\rule{\textwidth}{0.4pt}


\begin{verbatim}mysql> select user,host from mysql.user;
+--------+------------+
| user   | host       |
+--------+------------+
| root   | 127.0.0.1  |
| blog   | 172.16.1.\% |
| root   | localhost  |
| system | localhost  |
| test   | localhost  |
+--------+------------+
5 rows in set (0.00 sec)
\end{verbatim}


\noindent\rule{\textwidth}{0.4pt}


\par 
\begin{figure}[htbp]
\centering
\includegraphics[width=0.8\textwidth]{Image00019.jpg}
\end{figure}

 注意:


\par 1)由于安全原因,上面的用户是经过处理的。

\par 2)MySQL的用户由“用户名@主机名”构成,所以在用户列具有相同的用户,不要奇怪。

\par 要查看授权用户blog的具体授权权限,可使用如下命令:

\noindent\rule{\textwidth}{0.4pt}


\begin{verbatim}mysql> show grants for 'blog'@'172.16.1.\%';
+------------------------------------------------------------------------+
| Grants for blog@172.16.1.\%                                             |
+------------------------------------------------------------------------+
| GRANT USAGE ON *.* TO 'blog'@'172.16.1.\%' IDENTIFIED BY PASSWORD '*D6.B..A'|
| GRANT SELECT, INSERT, UPDATE, DELETE ON `blog`.* TO 'blog'@'172.16.1.\%'|
+------------------------------------------------------------------------+
2 rows in set (0.00 sec)
\end{verbatim}


\noindent\rule{\textwidth}{0.4pt}



% From text00096.html


\subsection{6.3.3 DDL语句之管理表}


\par 下面以前文给出的MySQL单实例编译场景为例,针对默认字符集格式的oldboy库进行讲解,环境准备如下:

\noindent\rule{\textwidth}{0.4pt}


\begin{verbatim}mysql> create database oldboy;
ERROR 1007 (HY000): Can't create database 'oldboy'; database exists
\#<==提示已存在就不用创建了。
mysql> show create database oldboy\textbackslash\{\}G
*************************** 1. row ***************************
       Database: oldboy
Create Database: CREATE DATABASE `oldboy` /*!40100 DEFAULT CHARACTER SET utf8 */
\#<==utf8字符集。
1 row in set (0.00 sec)
\end{verbatim}


\noindent\rule{\textwidth}{0.4pt}


\par 1.建立表

\par MySQL数据库中的表和Word、Excel里的表是一模一样的,相关表的基本知识如图6-3所示。

\begin{figure}[htbp]
\centering
\includegraphics[width=0.8\textwidth]{Image00044.jpg}
\end{figure}



\par 图6-3 数据库中的表

\par 建表的基本命令语法为:

\noindent\rule{\textwidth}{0.4pt}


\begin{verbatim}create table <表名> (
<字段名1> <类型1> ,
…
<字段名n> <类型n>);
\end{verbatim}


\noindent\rule{\textwidth}{0.4pt}


\par 其中,create table是关键字,不能更改,但是大小写可以变化。

\par 下面是人工设计的建表语句示例,表名为student:

\noindent\rule{\textwidth}{0.4pt}


\begin{verbatim}create table student(
id int(4) not null,
name char(20) not null,
age tinyint(2)  NOT NULL default '0',
dept varchar(16)  default NULL
);
\end{verbatim}


\noindent\rule{\textwidth}{0.4pt}


\par 第二种MySQL生成的建表语句,表名为student:

\noindent\rule{\textwidth}{0.4pt}


\begin{verbatim}CREATE TABLE `student` (
  `id` int(4) NOT NULL,
  `name` char(20) NOT NULL,
  `age` tinyint(2) NOT NULL DEFAULT '0',
  `dept` varchar(16) DEFAULT NULL
) ENGINE=InnoDB DEFAULT CHARSET=utf8;
\end{verbatim}


\noindent\rule{\textwidth}{0.4pt}


\par 上述语句表示创建一个student表,该表有4个字段:

\noindent\rule{\textwidth}{0.4pt}


\begin{verbatim}CREATE TABLE `student` (   \#<== CREATE TABLE是创建表的固定关键字,student为表名。
  `id` int(4) NOT NULL,            \#<==学号列,数字类型,长度为4,不为空值。
  `name` char(20) NOT NULL,        \#<==名字列,定长字符类型,长度为20,不为空值。
  `age` tinyint(2) NOT NULL DEFAULT '0', \#<==年龄列,很小的数字类型,长度为2,
                                         不为空,默认为0值。
  `dept` varchar(16) DEFAULT NULL  \#<==系别列,变长字符类型,长度为16,默认为空。
) ENGINE=InnoDB DEFAULT CHARSET=utf8;    \#<==引擎和字符集,引擎默认为InnoDB,
                                         字符集,继承库的utf8。
\end{verbatim}


\noindent\rule{\textwidth}{0.4pt}


\par 对于上面的student表内容可以用表6-9直观显示,字段下面的5表示表的5条记录。

\par 表6-9 student表内容

\begin{figure}[htbp]
\centering
\includegraphics[width=0.8\textwidth]{Image00045.jpg}
\end{figure}



\par 可以看到,MySQL的表和Word、Excel表几乎没有什么区别。很简单吧!

\par 下面进行实战演示。

\par 首先,执行student建表语句:

\noindent\rule{\textwidth}{0.4pt}


\begin{verbatim}mysql> use oldboy;
Database changed
mysql> create table student( id int(4) not null, name char(20) not null, age tinyint(2)  NOT NULL default '0', dept varchar(16)  default NULL  );
Query OK, 0 rows affected (0.02 sec)
mysql> show tables;
+------------------+
| Tables\_in\_oldboy |
+------------------+
| student          |
+------------------+
1 row in set (0.00 sec)
\end{verbatim}


\noindent\rule{\textwidth}{0.4pt}


\par 可通过如下命令查看建表的语句:

\noindent\rule{\textwidth}{0.4pt}


\begin{verbatim}mysql> show create table student\textbackslash\{\}G
*************************** 1. row ***************************
       Table: student
Create Table: CREATE TABLE `student` (
  `id` int(4) NOT NULL,
  `name` char(20) NOT NULL,
  `age` tinyint(2) NOT NULL DEFAULT '0',
  `dept` varchar(16) DEFAULT NULL
) ENGINE=InnoDB DEFAULT CHARSET=utf8
1 row in set (0.00 sec)
\end{verbatim}


\noindent\rule{\textwidth}{0.4pt}


\par 需要注意的是,在MySQL 5.1和MySQL 5.5及以上环境中,默认建表语句中的引擎是不同的,如果希望控制表的引擎,就要在建表语句里显式地指定表引擎建表。

\par MySQL 5.1及以前的版本中,默认引擎为MyISAM
 ,MySQL 5.5.5及以后的版本中,默认引擎为InnoDB
 。

\par 通过以下命令可查看表结构:

\noindent\rule{\textwidth}{0.4pt}


\begin{verbatim}mysql> desc student;
+-------+-------------+------+-----+---------+-------+
| Field | Type        | Null | Key | Default | Extra |
+-------+-------------+------+-----+---------+-------+
| id    | int(4)      | NO   |     | NULL    |       |
| name  | char(20)    | NO   |     | NULL    |       |
| age   | tinyint(2)  | NO   |     | 0       |       |
| dept  | varchar(16) | YES  |     | NULL    |       |
+-------+-------------+------+-----+---------+-------+
4 rows in set (0.00 sec)
\end{verbatim}


\noindent\rule{\textwidth}{0.4pt}


\par 上面提及的表的字段类型知识见表6-10。

\par 表6-10 字段类型表的对应列的类型说明

\begin{figure}[htbp]
\centering
\includegraphics[width=0.8\textwidth]{Image00046.jpg}
\end{figure}



\par 表6-11说明了CHAR和VARCHAR之间的差别。

\par 表6-11 CHAR和VARCHAR之间的差别

\begin{figure}[htbp]
\centering
\includegraphics[width=0.8\textwidth]{Image00047.jpg}
\end{figure}



\par 针对表6-11中存储需求的说明如下。

\par VARCHAR(10)列可以容纳最大长度为10的字符串。实际存储需求是字符串(L)的长度,加上一个记录字符串长度的字节。对于字符串'abcd',L是4,存储需要5字节。

\par 
CHAR和VARCHAR的差别小结


\par char类型是定长,不够的在右边用空格补全,这会浪费存储空间,以此列为查询条件时,速度更快,多数系统表的字段都是定长。

\par varchar类型是变长,节省存储空间,以此列为查询条件时速度较慢。

\par 下面来看看生产环境下的标准的UTF8格式表结构语句。下面是某sns产品生产中的正式建表语句:

\noindent\rule{\textwidth}{0.4pt}


\begin{verbatim}use sns;
set names gbk;
CREATE TABLE `subject\_comment\_manager` (
  `subject\_comment\_manager\_id` bigint(12) NOT NULL auto\_increment COMMENT '主键',
  `subject\_type` tinyint(2) NOT NULL COMMENT '素材类型',
  `subject\_primary\_key` varchar(255) NOT NULL COMMENT '素材的主键',
  `subject\_title` varchar(255) NOT NULL COMMENT '素材的名称',
  `edit\_user\_nick` varchar(64) default NULL COMMENT '修改人',
  `edit\_user\_time` timestamp NULL default NULL COMMENT '修改时间',
  `edit\_comment` varchar(255) default NULL COMMENT '修改的理由',
  `state` tinyint(1) NOT NULL default '1' COMMENT '0代表关闭,1代表正常',
  PRIMARY KEY  (`subject\_comment\_manager\_id`),
  KEY `IDX\_PRIMARYKEY` (`subject\_primary\_key`(32)), \#<==括号内的32表示对前
                                                  32个字符做前缀索引。
  KEY `IDX\_SUBJECT\_TITLE` (`subject\_title`(32))
  KEY `index\_nick\_type` (`edit\_user\_nick`(32),`subject\_type`)
  \#<==联合索引,此行为是新加的,用来为大家讲解。实际表语句内没有此行。
) ENGINE=InnoDB AUTO\_INCREMENT=1 DEFAULT CHARSET=utf8;
\end{verbatim}


\noindent\rule{\textwidth}{0.4pt}


\par 2.查看表结构

\par 
查看表结构的
 命令语法为:

\noindent\rule{\textwidth}{0.4pt}


\begin{verbatim}desc表名或者show [full] columns from 表名 FROM db\_name;
\end{verbatim}


\noindent\rule{\textwidth}{0.4pt}


\par 查看表结构有如下几种方法。

\par 方法1:先通过use进入到指定库,然后再查看。

\noindent\rule{\textwidth}{0.4pt}


\begin{verbatim}mysql> use oldboy
Database changed
mysql> desc student;
+-------+-------------+------+-----+---------+-------+
| Field | Type        | Null | Key | Default | Extra |
+-------+-------------+------+-----+---------+-------+
| id    | int(4)      | NO   |     | NULL    |       |
| name  | char(20)    | NO   |     | NULL    |       |
| age   | tinyint(2)  | NO   |     | 0       |       |
| dept  | varchar(16) | YES  |     | NULL    |       |
+-------+-------------+------+-----+---------+-------+
4 rows in set (0.00 sec)
\end{verbatim}


\noindent\rule{\textwidth}{0.4pt}


\par 方法2:无须进入指定库,通过如下命令直接查看。

\noindent\rule{\textwidth}{0.4pt}


\begin{verbatim}mysql> show columns from oldboy.student;
+-------+-------------+------+-----+---------+-------+
| Field | Type        | Null | Key | Default | Extra |
+-------+-------------+------+-----+---------+-------+
| id    | int(4)      | NO   |     | NULL    |       |
| name  | char(20)    | NO   |     | NULL    |       |
| age   | tinyint(2)  | NO   |     | 0       |       |
| dept  | varchar(16) | YES  |     | NULL    |       |
+-------+-------------+------+-----+---------+-------+
4 rows in set (0.00 sec)
mysql> show full columns from student from oldboy;
+-------+-------------+-----------------+------+-----+---------+-------+-
--------------------------------+---------+
| Field | Type        | Collation       | Null | Key | Default | Extra | Privileges                      | Comment |
+-------+-------------+-----------------+------+-----+---------+-------+-
--------------------------------+---------+
| id    | int(4)      | NULL            | NO   |     | NULL    |       | select,insert,update,references |         |
| name  | char(20)    | utf8\_general\_ci | NO   |     | NULL    |       | select,insert,update,references |         |
| age   | tinyint(2)  | NULL            | NO   |     | 0       |       | select,insert,update,references |         |
| dept  | varchar(16) | utf8\_general\_ci | YES  |     | NULL    |       | select,insert,update,references |         |
+-------+-------------+-----------------+------+-----+---------+-------+-
--------------------------------+---------+
4 rows in set (0.01 sec)
\end{verbatim}


\noindent\rule{\textwidth}{0.4pt}


\par 3.更改表名

\par 更改表名有两种方法。

\par 第一种是采用rename命令更改表名。

\par 命令的语法为:

\noindent\rule{\textwidth}{0.4pt}


\begin{verbatim}rename table 原表名 to 新表名;
\end{verbatim}


\noindent\rule{\textwidth}{0.4pt}


\par 例如,要将表student名字更改为test,执行命令与结果如下:

\noindent\rule{\textwidth}{0.4pt}


\begin{verbatim}mysql> rename table student to test;
Query OK, 0 rows affected (0.03 sec)
mysql> show tables;
+------------------+
| Tables\_in\_oldboy |
+------------------+
| test             |
+------------------+
1 row in set (0.00 sec)
\end{verbatim}


\noindent\rule{\textwidth}{0.4pt}


\par 第二种方法为采用alter法修改表名。

\par 命令示例如下:

\noindent\rule{\textwidth}{0.4pt}


\begin{verbatim}mysql> alter table test rename to student;
Query OK, 0 rows affected (0.01 sec)

mysql> show tables;
+------------------+
| Tables\_in\_oldboy |
+------------------+
| student          |
+------------------+
1 row in set (0.00 sec)
\end{verbatim}


\noindent\rule{\textwidth}{0.4pt}


\par 4.增、删、改表的字段

\par 添加字段

\par 命令语法为:

\noindent\rule{\textwidth}{0.4pt}


\begin{verbatim}alter table 表名 add 字段 类型 其他;
\end{verbatim}


\noindent\rule{\textwidth}{0.4pt}


\par 在添加字段时,先通过如下命令测试表数据:

\noindent\rule{\textwidth}{0.4pt}


\begin{verbatim}CREATE TABLE `test` (
  `id` int(4) NOT NULL AUTO\_INCREMENT,
  `name` char(20) NOT NULL,
  PRIMARY KEY (`id`)
) ENGINE=InnoDB DEFAULT CHARSET=UTF8;
\end{verbatim}


\noindent\rule{\textwidth}{0.4pt}


\par 然后,查看表结构:

\noindent\rule{\textwidth}{0.4pt}


\begin{verbatim}mysql> desc test;
+-------+----------+------+-----+---------+----------------+
| Field | Type     | Null | Key | Default | Extra          |
+-------+----------+------+-----+---------+----------------+
| id    | int(4)   | NO   | PRI | NULL    | auto\_increment |
| name  | char(20) | NO   |     | NULL    |                |
+-------+----------+------+-----+---------+----------------+
2 rows in set (0.01 sec)
\end{verbatim}


\noindent\rule{\textwidth}{0.4pt}


\par 若要在表test中添加字段sex、age和qq,类型分别为char(4)、int(4)、varchar(15),那么可以通过如下命令来完成。

\par 先添加性别列,长度为4,内容非空。示例如下:

\noindent\rule{\textwidth}{0.4pt}


\begin{verbatim}mysql> alter table test add sex char(4);
Query OK, 0 rows affected (0.02 sec)
Records: 0  Duplicates: 0  Warnings: 0
mysql> desc test;
+-------+----------+------+-----+---------+----------------+
| Field | Type     | Null | Key | Default | Extra          |
+-------+----------+------+-----+---------+----------------+
| id    | int(4)   | NO   | PRI | NULL    | auto\_increment |
| name  | char(20) | NO   |     | NULL    |                |
| sex   | char(4)  | YES  |     | NULL    |                |  \#<==此行为新增
                                                           的sex列。
+-------+----------+------+-----+---------+----------------+
3 rows in set (0.00 sec)
\end{verbatim}


\noindent\rule{\textwidth}{0.4pt}


\par 
\begin{figure}[htbp]
\centering
\includegraphics[width=0.8\textwidth]{Image00006.jpg}
\end{figure}

 提示:
 默认情况下,列会增加到所有字段的结尾。

\par 下面指定添加年龄列到name列后面的位置,示例如下:

\noindent\rule{\textwidth}{0.4pt}


\begin{verbatim}mysql> alter table test add age int(4) after name;
Query OK, 0 rows affected (0.02 sec)
Records: 0  Duplicates: 0  Warnings: 0
mysql> desc test;
+-------+----------+------+-----+---------+----------------+
| Field | Type     | Null | Key | Default | Extra          |
+-------+----------+------+-----+---------+----------------+
| id    | int(4)   | NO   | PRI | NULL    | auto\_increment |
| name  | char(20) | NO   |     | NULL    |                |
| age   | int(4)   | YES  |     | NULL    |                |
| sex   | char(4)  | YES  |     | NULL    |                |
+-------+----------+------+-----+---------+----------------+
4 rows in set (0.00 sec)
\end{verbatim}


\noindent\rule{\textwidth}{0.4pt}


\par 现在通过下面的命令在第一列添加qq字段:

\noindent\rule{\textwidth}{0.4pt}


\begin{verbatim}mysql> alter table test add qq varchar(15) first;
Query OK, 0 rows affected (0.04 sec)
Records: 0  Duplicates: 0  Warnings: 0
mysql> desc test;
+-------+-------------+------+-----+---------+----------------+
| Field | Type        | Null | Key | Default | Extra          |
+-------+-------------+------+-----+---------+----------------+
| qq    | varchar(15) | YES  |     | NULL    |                |
| id    | int(4)      | NO   | PRI | NULL    | auto\_increment |
| name  | char(20)    | NO   |     | NULL    |                |
| age   | int(4)      | YES  |     | NULL    |                |
| sex   | char(4)     | YES  |     | NULL    |                |
+-------+-------------+------+-----+---------+----------------+
5 rows in set (0.00 sec)
\end{verbatim}


\noindent\rule{\textwidth}{0.4pt}


\par 若要删除字段,可采用如下命令:

\noindent\rule{\textwidth}{0.4pt}


\begin{verbatim}mysql> alter table test drop qq;
Query OK, 0 rows affected (0.02 sec)
Records: 0  Duplicates: 0  Warnings: 0

mysql> alter table test drop age;
Query OK, 0 rows affected (0.02 sec)
Records: 0  Duplicates: 0  Warnings: 0
\end{verbatim}


\noindent\rule{\textwidth}{0.4pt}


\par 若要同时添加两个字段,可采用如下命令:

\noindent\rule{\textwidth}{0.4pt}


\begin{verbatim}mysql> alter table test add age tinyint(2) first,add qq varchar(15);
Query OK, 5 rows affected (0.06 sec)
Records: 5  Duplicates: 0  Warnings: 0
\end{verbatim}


\noindent\rule{\textwidth}{0.4pt}


\par 下面来看一些生产环境下的命令使用案例。

\par 增加1个字段的命令如下:

\noindent\rule{\textwidth}{0.4pt}


\begin{verbatim}ALTER TABLE `etiantian` ADD `FIRSTPHOTO\_URL` varchar(255) default NULL COMMENT '第一张图片URL'
\end{verbatim}


\noindent\rule{\textwidth}{0.4pt}


\par 增2个字段的命令如下:

\noindent\rule{\textwidth}{0.4pt}


\begin{verbatim}ALTER TABLE `basic` ADD `adhtml\_top`  varchar(1024) default NULL COMMENT
'顶部广告html' ,
                    ADD `adhtml\_right` varchar(1024) default NULL COMMENT '右侧广告html' ;
\end{verbatim}


\noindent\rule{\textwidth}{0.4pt}


\par 改变字段的命令如下:

\noindent\rule{\textwidth}{0.4pt}


\begin{verbatim}alter table ett\_ambiguity change ambiguity\_state  ambiguity\_state tinyint  comment '状态,默认1=正常,0=失效';
ALTER TABLE `ett\_photo`
MODIFY COLUMN `PHOTO\_DESCRIPTION` varchar(512) CHARACTER SET utf8 COLLATE utf8\_general\_ci NOT NULL COMMENT '描述' AFTER PHOTO\_TITLE`;
\end{verbatim}


\noindent\rule{\textwidth}{0.4pt}


\par 修改字段类型的命令如下:

\noindent\rule{\textwidth}{0.4pt}


\begin{verbatim}mysql> alter table test modify age char(4) after name;
Query OK, 6 rows affected (0.00 sec)
Records: 6  Duplicates: 0  Warnings: 0
\end{verbatim}


\noindent\rule{\textwidth}{0.4pt}


\par 修改字段名称的命令如下:

\noindent\rule{\textwidth}{0.4pt}


\begin{verbatim}mysql> alter table test change age oldboyage char(4) after name;
Query OK, 6 rows affected (0.01 sec)
Records: 6  Duplicates: 0  Warnings: 0
\end{verbatim}


\noindent\rule{\textwidth}{0.4pt}


\par 这里要说明一下企业里更改数据的流程:开发人员写出SQL语句,发给运维人员或DBA检验并执行(如图6-4所示)!

\par 
\begin{figure}[htbp]
\centering
\includegraphics[width=0.8\textwidth]{Image00006.jpg}
\end{figure}

 提示:
 工作中添加字段的需求来自开发人员,运维人员或DBA执行开发人员提供的语句。

\par 对数据库表的修改,应尽量选在代码上线的时候或者业务低谷的时候执行,不要在流量高峰期处理大表的更改。

\begin{figure}[htbp]
\centering
\includegraphics[width=0.8\textwidth]{Image00048.jpg}
\end{figure}



\par 图6-4 开发人员提交的修改数据库的SQL语句

\par 下班的时候尽量不要独自在生产线上更改东西,老男孩在刚入行的时候就犯过一次这样的错误,还好提前备份了,迅速改了回来,不过还是吓出了一身冷汗。

\par 5.创建和删除索引

\par 数据库的索引就像书的目录一样,如果在字段上建立了索引,那么以索引列为查询条件时可以加快查询数据的速度,这是MySQL优化的重要内容之一,后文会详细讲解,这里老男孩先带大家尝个鲜。

\par 常见的为表内字段建立索引的方法有如下两种。

\par 方法1:建表后利用alter命令增加普通索引。

\par 在此之前,要删除建表时创建的index\_name索引,命令如下:

\noindent\rule{\textwidth}{0.4pt}


\begin{verbatim}mysql> alter table student drop index index\_name;
Query OK, 0 rows affected (0.01 sec)
Records: 0  Duplicates: 0  Warnings: 0
mysql> desc student;
+-------+-------------+------+-----+---------+----------------+
| Field | Type        | Null | Key | Default | Extra          |
+-------+-------------+------+-----+---------+----------------+
| id    | int(11)     | NO   | PRI | NULL    | auto\_increment |
| name  | char(20)    | NO   |     | NULL    |                |
| age   | tinyint(2)  | NO   |     | 0       |                |
| dept  | varchar(16) | YES  |     | NULL    |                |
+-------+-------------+------+-----+---------+----------------+
4 rows in set (0.01 sec)
\end{verbatim}


\noindent\rule{\textwidth}{0.4pt}


\par 然后,就可以在test表的name列上添加索引了,索引名为index\_name,命令如下:

\noindent\rule{\textwidth}{0.4pt}


\begin{verbatim}mysql> alter table test add index index\_name(name);
Query OK, 0 rows affected (0.01 sec)
Records: 0  Duplicates: 0  Warnings: 0
\end{verbatim}


\noindent\rule{\textwidth}{0.4pt}


\par 方法2:使用create为test表的qq列创建普通索引。

\par 示例命令如下:

\noindent\rule{\textwidth}{0.4pt}


\begin{verbatim}mysql> create index index\_qq on test(qq);
Query OK, 0 rows affected (0.01 sec)
Records: 0  Duplicates: 0  Warnings: 0
\end{verbatim}


\noindent\rule{\textwidth}{0.4pt}


\par 查看结果:

\noindent\rule{\textwidth}{0.4pt}


\begin{verbatim}mysql> desc test;
+-------+-------------+------+-----+---------+----------------+
| Field | Type        | Null | Key | Default | Extra          |
+-------+-------------+------+-----+---------+----------------+
| age   | tinyint(2)  | YES  |     | NULL    |                |
| id    | int(4)      | NO   | PRI | NULL    | auto\_increment |
\#<==PRI为主键的标识。
| name  | char(20)    | NO   | MUL | NULL    |                |
\#<==MUL这里原来是空。
| sex   | char(4)     | YES  |     | NULL    |                |
| qq    | varchar(15) | YES  | MUL | NULL    |                |
\#<==MUL这里原来是空。
+-------+-------------+------+-----+---------+----------------+
5 rows in set (0.00 sec)
\end{verbatim}


\noindent\rule{\textwidth}{0.4pt}


\par 其中,PRI为主键索引的标识,MUL为普通索引的标识。

\par 删除建表时创建的index\_name索引的命令为:

\noindent\rule{\textwidth}{0.4pt}


\begin{verbatim}mysql> alter table student drop index index\_name;
\end{verbatim}


\noindent\rule{\textwidth}{0.4pt}


\par 
生产场景下的经验


\par 当数据量以及访问量很大的时候,不适合临时建立索引,因为会影响用户访问。老男孩曾经在生产上为有着四五百万条记录的表建立索引,花了90~180秒,说多了都是血和泪啊,读者应尽量选择在业务流量低谷时建立索引,以避免重蹈覆辙。

\par 6.查看建表语句

\par 可通过如下命令查看已建表对应的SQL语句:

\noindent\rule{\textwidth}{0.4pt}


\begin{verbatim}mysql> show create table test\textbackslash\{\}G  \#<==\textbackslash\{\}G垂直显示结果。
*************************** 1. row ***************************
       Table: test
Create Table: CREATE TABLE `test` (
  `age` tinyint(2) DEFAULT NULL,
  `id` int(4) NOT NULL AUTO\_INCREMENT,
  `name` char(20) NOT NULL,
  `sex` char(4) DEFAULT NULL,
  `qq` varchar(15) DEFAULT NULL,
  PRIMARY KEY (`id`)
) ENGINE=InnoDB DEFAULT CHARSET=utf8
1 row in set (0.00 sec)
\end{verbatim}


\noindent\rule{\textwidth}{0.4pt}


\par 7.删除表

\par 命令的语法为:

\noindent\rule{\textwidth}{0.4pt}


\begin{verbatim}drop table <表名>;
\end{verbatim}


\noindent\rule{\textwidth}{0.4pt}


\par 例如,删除表名为test的表,执行的命令和结果如下:

\noindent\rule{\textwidth}{0.4pt}


\begin{verbatim}mysql> show tables from oldboy;
+------------------+
| Tables\_in\_oldboy |
+------------------+
| student          |
| test             |
+------------------+
2 rows in set (0.00 sec)
mysql> drop table student;   \#<==删除表名为test的表。
Query OK, 0 rows affected (0.01 sec)
mysql> show tables from oldboy;
+------------------+
| Tables\_in\_oldboy |
+------------------+
| test             |
+------------------+
1 row in set (0.00 sec)
\end{verbatim}


\noindent\rule{\textwidth}{0.4pt}



% From text00097.html


\subsection{6.3.4 DML语句之管理表中的数据}


\par 1.往表中插入数据

\par 命令语法为:

\noindent\rule{\textwidth}{0.4pt}


\begin{verbatim}insert into <表名> [( <字段名1>[,..<字段名n > ])] values ( 值1 )[, ( 值n )];
\end{verbatim}


\noindent\rule{\textwidth}{0.4pt}


\par 在插入数据前,先新建一个简单的测试表test,语句如下:

\noindent\rule{\textwidth}{0.4pt}


\begin{verbatim}use oldboy
drop table test;
CREATE TABLE test (
   id int(4) NOT NULL AUTO\_INCREMENT,
  name char(20) NOT NULL,
  PRIMARY KEY (id)
) ;
desc test;
\end{verbatim}


\noindent\rule{\textwidth}{0.4pt}


\par 往表中插入数据有几种不同的语法,具体如下。

\par 第一种:按规则指定所有的列名,并且每列都插入值。命令如下:

\noindent\rule{\textwidth}{0.4pt}


\begin{verbatim}insert into test(id,name) values(1,'oldboy');
\end{verbatim}


\noindent\rule{\textwidth}{0.4pt}


\par 第二种:只在name列插入值,由于id列是自增的,所以可以这样操作。命令如下:

\noindent\rule{\textwidth}{0.4pt}


\begin{verbatim}insert into test(name) values('oldgirl');
\end{verbatim}


\noindent\rule{\textwidth}{0.4pt}


\par 也可以执行下面的语句来实现,注意,这两种请只选择一条执行:

\noindent\rule{\textwidth}{0.4pt}


\begin{verbatim}insert into test(id,name) values(null,'zhangsan');
\end{verbatim}


\noindent\rule{\textwidth}{0.4pt}


\par 第三种:如果不指定列,就要按规则为每列都插入恰当的值。命令如下:

\noindent\rule{\textwidth}{0.4pt}


\begin{verbatim}insert into test values(3,'inca');
\end{verbatim}


\noindent\rule{\textwidth}{0.4pt}


\par 第四种:批量插入数据的方法,可提升效率。命令如下:

\noindent\rule{\textwidth}{0.4pt}


\begin{verbatim}insert into test values(4,'zuma'),(5,'kaka'); \#<==批量插入2条记录,提升了效率。
delete from test;
INSERT INTO `test` VALUES (1,'oldboy'),(2,'oldgirl'),(3,'inca'),(4,'zuma'), (5,'kaka');
\end{verbatim}


\noindent\rule{\textwidth}{0.4pt}


\par 下面是插入数据的实践:

\noindent\rule{\textwidth}{0.4pt}


\begin{verbatim}mysql> use oldboy
Database changed
mysql> drop table test;
Query OK, 0 rows affected (0.00 sec)
mysql> CREATE TABLE test (
    ->    id int(4) NOT NULL AUTO\_INCREMENT,
    ->   name char(20) NOT NULL,
    ->   PRIMARY KEY (id)
    -> ) ;
Query OK, 0 rows affected (0.02 sec)
mysql> desc test;
+-------+----------+------+-----+---------+----------------+
| Field | Type     | Null | Key | Default | Extra          |
+-------+----------+------+-----+---------+----------------+
| id    | int(4)   | NO   | PRI | NULL    | auto\_increment |
| name  | char(20) | NO   |     | NULL    |                |
+-------+----------+------+-----+---------+----------------+
2 rows in set (0.00 sec)
\end{verbatim}


\noindent\rule{\textwidth}{0.4pt}


\par 通过以下命令插入第一行,请注意语法的细微不同:

\noindent\rule{\textwidth}{0.4pt}


\begin{verbatim}mysql> insert into test(id,name) values(1,'oldboy');
Query OK, 1 row affected (0.00 sec)
mysql> insert into test(name) values('oldgirl');
Query OK, 1 row affected (0.01 sec)
mysql> insert into test values(3,'inca');
Query OK, 1 row affected (0.01 sec)
mysql> insert into test values(4,'zuma'),(5,'kaka');
Query OK, 2 rows affected (0.00 sec)
Records: 2  Duplicates: 0  Warnings: 0
mysql> select * from test;  \#<==查看插入后的结果。
+----+---------+
| id | name    |
+----+---------+
|  1 | oldboy  |
|  2 | oldgirl |
|  3 | inca    |
|  4 | zuma    |
|  5 | kaka    |
+----+---------+
5 rows in set (0.00 sec)
mysql> delete from test;  \#<==全部删除,批量重新插入。
Query OK, 5 rows affected (0.01 sec)
mysql> select * from test;
Empty set (0.00 sec)
mysql> INSERT INTO `test` VALUES (1,'oldboy'),(2,'oldgirl'),(3,'inca'),(4,'zuma'),(5,'kaka');
Query OK, 5 rows affected (0.00 sec)
Records: 5  Duplicates: 0  Warnings: 0
mysql> select * from test; \#<==查看批量插入后的结果。
+----+---------+
| id | name    |
+----+---------+
|  1 | oldboy  |
|  2 | oldgirl |
|  3 | inca    |
|  4 | zuma    |
|  5 | kaka    |
+----+---------+
5 rows in set (0.00 sec)
\end{verbatim}


\noindent\rule{\textwidth}{0.4pt}


\par 
\begin{figure}[htbp]
\centering
\includegraphics[width=0.8\textwidth]{Image00006.jpg}
\end{figure}

 提示:
 请大家注意观察输出结果。

\par 测试完毕,退出数据库,然后备份上述数据,留着备用,备份的命令如下:

\noindent\rule{\textwidth}{0.4pt}


\begin{verbatim}mysql> exit
Bye
[root@oldboy \textasciitilde{}]\# mysqldump -uroot -poldboy123 -B oldboy >/opt/bak.sql
Warning: Using a password on the command line interface can be insecure.
\end{verbatim}


\noindent\rule{\textwidth}{0.4pt}


\par 备份后要检查一下备份的SQL数据内容,同时过滤无用信息:

\noindent\rule{\textwidth}{0.4pt}


\begin{verbatim}[root@oldboy \textasciitilde{}]\# ls -l /opt/bak.sql
-rw-r--r--. 1 root root 2051 Mar  2 03:32 /opt/bak.sql
\#<==这里的2051位置对应的数字不能太小,否则备份可能会有问题。
[root@oldboy \textasciitilde{}]\# egrep -v "\#|\textbackslash\{\}/|\textasciicircum{}\$|--" /opt/bak.sql
USE `oldboy`;
DROP TABLE IF EXISTS `test`;
CREATE TABLE `test` (
  `id` int(4) NOT NULL AUTO\_INCREMENT,
  `name` char(20) NOT NULL,
  PRIMARY KEY (`id`)
) ENGINE=InnoDB AUTO\_INCREMENT=6 DEFAULT CHARSET=utf8;
LOCK TABLES `test` WRITE;
INSERT INTO `test` VALUES (1,'oldboy'),(2,'oldgirl'),(3,'inca'),(4,'zuma'), (5,'kaka');
UNLOCK TABLES;
\end{verbatim}


\noindent\rule{\textwidth}{0.4pt}


\par 可以看到上面备份的语句,就是我们执行的语句内容,后面会用到此处备份的数据还原填充数据。

\par 
\begin{figure}[htbp]
\centering
\includegraphics[width=0.8\textwidth]{Image00006.jpg}
\end{figure}

 补充强调:
 我们平时登录网站发帖子、博文、微信,实质上都是通过Web网站的程序连接MySQL数据库,然后使用上述的insert语句把数据存入数据库。

\par 2.查询表中的数据

\par 命令语法为:

\noindent\rule{\textwidth}{0.4pt}


\begin{verbatim}select <字段1,字段2,...> from < 表名 > where < 表达式 >;
\end{verbatim}


\noindent\rule{\textwidth}{0.4pt}


\par 其中,select、from、where是不能随便更改的,它们是关键字,支持大小写。

\par 若要查看表test中的所有数据,可采用如下两种方法。

\par 方法1:进入指定库后再查询(常用)。

\noindent\rule{\textwidth}{0.4pt}


\begin{verbatim}mysql> use oldboy
Database changed
mysql> select * from test;
+----+---------+
| id | name    |
+----+---------+
|  1 | oldboy  |
|  2 | oldgirl |
|  3 | inca    |
|  4 | zuma    |
|  5 | kaka    |
+----+---------+
5 rows in set (0.00 sec)
\end{verbatim}


\noindent\rule{\textwidth}{0.4pt}


\par 方法2:直接在库外,使用“库.表”的方式查询。

\noindent\rule{\textwidth}{0.4pt}


\begin{verbatim}mysql> select * from oldboy.test; \#<==用点号分隔库和表
+----+---------+
| id | name    |
+----+---------+
|  1 | oldboy  |
|  2 | oldgirl |
|  3 | inca    |
|  4 | zuma    |
|  5 | kaka    |
+----+---------+
5 rows in set (0.00 sec)
\end{verbatim}


\noindent\rule{\textwidth}{0.4pt}


\par 这里的“*”表示查询表的所有字段,详情可查看help select。

\par 可通过如下命令查询mysql库use表对应user和host列的用户,这是前面用过的方法:

\noindent\rule{\textwidth}{0.4pt}


\begin{verbatim}mysql> select user,host from mysql.user;
+--------+------------+
| user   | host       |
+--------+------------+
| root   | 127.0.0.1  |
| blog   | 172.16.1.\% |
| root   | localhost  |
| system | localhost  |
| test   | localhost  |
+--------+------------+
5 rows in set (0.00 sec)
\end{verbatim}


\noindent\rule{\textwidth}{0.4pt}


\par 若要根据指定条件查询表的部分数据,这里面又分为以下几种情况,具体来看看。

\par 第一种,查看表test中的前2行数据,命令如下:

\noindent\rule{\textwidth}{0.4pt}


\begin{verbatim}mysql> select * from test limit 2;
+----+---------+
| id | name    |
+----+---------+
|  1 | oldboy  |
|  2 | oldgirl |
+----+---------+
2 rows in set (0.00 sec)
mysql> select * from test limit 0,3; \#<==从第0行开查,查3行记录。
+----+---------+
| id | name    |
+----+---------+
|  1 | oldboy  |
|  2 | oldgirl |
|  3 | inca    |
+----+---------+
3 rows in set (0.00 sec)
\end{verbatim}


\noindent\rule{\textwidth}{0.4pt}


\par 第二种,指定固定条件查询数据,一般结尾接where关键字并指定字段的条件查询。

\par 假设要查询id为1的行,则命令如下:

\noindent\rule{\textwidth}{0.4pt}


\begin{verbatim}mysql> select * from test where id=1; \#<==查询数字列无须加引号,但是看起来是数字,
                                      实际属于字符列就需要加引号了。
+----+--------+
| id | name   |
+----+--------+
|  1 | oldboy |
+----+--------+
1 row in set (0.00 sec)
mysql> select * from test where name='oldgirl'; \#<==查询字符串务必要加单引号。
+----+---------+
| id | name    |
+----+---------+
|  2 | oldgirl |
+----+---------+
1 row in set (0.00 sec)
\end{verbatim}


\noindent\rule{\textwidth}{0.4pt}


\par 
\begin{figure}[htbp]
\centering
\includegraphics[width=0.8\textwidth]{Image00006.jpg}
\end{figure}

 提示:
 查询字符类型的条件列的值要带单引号,整形列的数字值不带引号,一切以字段类型为准。

\par 若要同时查询多个条件,则取交集(and),例如查询id为2,并且name为oldgirl的命令如下:

\noindent\rule{\textwidth}{0.4pt}


\begin{verbatim}mysql> select * from test where id=2 and name='oldgirl'; \#<==多个条件。
+----+---------+
| id | name    |
+----+---------+
|  2 | oldgirl |
+----+---------+
1 row in set (0.00 sec)
\end{verbatim}


\noindent\rule{\textwidth}{0.4pt}


\par 第三种,指定固定条件范围查询数据。

\par 执行命令如下:

\noindent\rule{\textwidth}{0.4pt}


\begin{verbatim}mysql> select id,name from test where id>2 and id<5; \#<==多个条件,and取交集。
+----+------+
| id | name |
+----+------+
|  3 | inca |
|  4 | zuma |
+----+------+
2 rows in set (0.00 sec)
mysql> select id,name from test where id>3 or id<2; \#<==多个条件,or取并集。
+----+--------+
| id | name   |
+----+--------+
|  1 | oldboy |
|  4 | zuma   |
|  5 | kaka   |
+----+--------+
3 rows in set (0.00 sec)
\end{verbatim}


\noindent\rule{\textwidth}{0.4pt}


\par 第四种,其他查询功能。

\par 若要查询排序功能,则可采用如下命令:

\noindent\rule{\textwidth}{0.4pt}


\begin{verbatim}mysql> select id,name from test where id>3 order by id asc;
+----+------+
| id | name |
+----+------+
|  4 | zuma |
|  5 | kaka |
+----+------+
2 rows in set (0.00 sec)
mysql> select id,name from test where id<3 order by id desc;
+----+---------+
| id | name    |
+----+---------+
|  2 | oldgirl |
|  1 | oldboy  |
+----+---------+
2 rows in set (0.00 sec)
\end{verbatim}


\noindent\rule{\textwidth}{0.4pt}


\par 
小结:
 前文已经讲解过,对于MySQL的DML语句,开发人员需要熟练掌握,运维人员只需简单了解即可,因此本书将其他的查询如子查询、join、union、多表关联查询、分组、having等放到了开发章节来讲,运维人员了解到这就可以了,有兴趣的运维读者可以去开发章节深入学习,较长的select语句展示见\href{http://www.ithao123.cn/content-5282534.html}{http://www.ithao123.cn/content-5282534.html}
 。

\par 3.修改表中的数据

\par 命令语法为:

\noindent\rule{\textwidth}{0.4pt}


\begin{verbatim}update 表名 set 字段=新值,… where 条件;
\end{verbatim}


\noindent\rule{\textwidth}{0.4pt}


\par 这里一定要注意条件。

\par 下面是修改指定的行字段内容案例实践。

\par 可通过如下命令查看要修改的表:

\noindent\rule{\textwidth}{0.4pt}


\begin{verbatim}mysql> select * from test;
+----+---------+
| id | name    |
+----+---------+
|  1 | oldboy  |
|  2 | oldgirl |
|  3 | inca    |
|  4 | zuma    |
|  5 | kaka    |
+----+---------+
5 rows in set (0.00 sec)
\end{verbatim}


\noindent\rule{\textwidth}{0.4pt}


\par 若要将id为3的行的名字修改为gongli,可采用如下命令:

\noindent\rule{\textwidth}{0.4pt}


\begin{verbatim}mysql> update test set name='xiaoting' where id=3; \#<==等号可以改为其他任一
                                                符号或增加多条件。
Query OK, 1 row affected (0.00 sec)
Rows matched: 1  Changed: 1  Warnings: 0
mysql> select * from test where id=3;
+----+----------+
| id | name     |
+----+----------+
|  3 | xiaoting |
+----+----------+
1 row in set (0.00 sec)
\end{verbatim}


\noindent\rule{\textwidth}{0.4pt}


\par 4.修改数据导致的事故案例和解决方案

\par 注意,以下介绍的是严重的事故案例,若误操作,则可能会导致数据丢失。

\par 如果不带条件更改所有表的记录,示例如下:

\noindent\rule{\textwidth}{0.4pt}


\begin{verbatim}mysql> update test set name='xiaoting'; \#<==如果不加条件则要十分小心。专业做法是要多向开发人员确认,发送给开发人员的语句一定要括号注明,以防止DBA误会。
Query OK, 4 rows affected (0.01 sec)
Rows matched: 5  Changed: 4  Warnings: 0
\end{verbatim}


\noindent\rule{\textwidth}{0.4pt}


\par 上述命令将会更改表的所有记录,此时如果是错误的修改,那么对企业来说就是重大的事故。查看更改后的结果:

\noindent\rule{\textwidth}{0.4pt}


\begin{verbatim}mysql> select * from test;
+----+----------+
| id | name     |
+----+----------+
|  1 | xiaoting | \#<==可以发现,所有的name列的内容都是相同的了。
|  2 | xiaoting |
|  3 | xiaoting |
|  4 | xiaoting |
|  5 | xiaoting |
+----+----------+
5 rows in set (0.00 sec)
\end{verbatim}


\noindent\rule{\textwidth}{0.4pt}


\par 针对上面的故障,开始用备份的数据进行恢复(备份的重要性):

\noindent\rule{\textwidth}{0.4pt}


\begin{verbatim}mysql> drop table test;
Query OK, 0 rows affected (0.01 sec)
mysql> source /opt/bak.sql
Query OK, 0 rows affected (0.01 sec)
mysql> select * from oldboy.test;
+----+---------+
| id | name    |
+----+---------+
|  1 | oldboy  |
|  2 | oldgirl |
|  3 | inca    |
|  4 | zuma    |
|  5 | kaka    |
+----+---------+
5 rows in set (0.00 sec)
\end{verbatim}


\noindent\rule{\textwidth}{0.4pt}


\par 至于防止因发生误操作而导致上述数据库故障案例的方法之一,请读者到老男孩的博客\href{http://oldboy.blog.51cto.com/2561410/1321061}{http://oldboy.blog.51cto.com/2561410/1321061}
 查看。

\par 5.删除表中的数据

\par 命令语法为:

\noindent\rule{\textwidth}{0.4pt}


\begin{verbatim}delete from 表名 where 表达式;
\end{verbatim}


\noindent\rule{\textwidth}{0.4pt}


\par 下面是操作演示。

\par 若要删除表test中编号为1的记录,可采用如下命令:

\noindent\rule{\textwidth}{0.4pt}


\begin{verbatim}mysql> use oldboy
Database changed
mysql> delete from test where id=1;          \#<==删除id为1的行。
Query OK, 1 row affected (0.00 sec)
mysql> delete from test where name='oldboy'; \#<==删除name 等于oldboy的行。
Query OK, 0 rows affected (0.00 sec)
mysql> select * from test;
+----+---------+
| id | name    |
+----+---------+
|  2 | oldgirl |
|  3 | inca    |
|  4 | zuma    |
|  5 | kaka    |
+----+---------+
4 rows in set (0.00 sec)
\end{verbatim}


\noindent\rule{\textwidth}{0.4pt}


\par 
\begin{figure}[htbp]
\centering
\includegraphics[width=0.8\textwidth]{Image00006.jpg}
\end{figure}

 提示:


\par ·不加任何条件(where)就是全部删除,这也是非常危险的操作,这里就不演示了。对于不加条件的delete、update,一定要慎用回车键,按照流程操作(运维人员和开发人员一起操作)。

\par ·开发人员在程序里可能不用delete语句,而是用update语句来更新显示的状态(实现逻辑删除)。

\par 以下是一个通过update伪删除数据的企业案例。

\par 有些企业开发人员在开发程序时,会通过状态来判断页面内容显示。比如,在test表中有如下数据环境,环境接上文:

\noindent\rule{\textwidth}{0.4pt}


\begin{verbatim}mysql> alter table test add  state tinyint(2) not null default 1;
Query OK, 0 rows affected (0.04 sec)
Records: 0  Duplicates: 0  Warnings: 0
mysql> desc test;
+-------+------------+------+-----+---------+----------------+
| Field | Type       | Null | Key | Default | Extra          |
+-------+------------+------+-----+---------+----------------+
| id    | int(4)     | NO   | PRI | NULL    | auto\_increment |
| name  | char(20)   | NO   |     | NULL    |                |
| state | tinyint(2) | NO   |     | 1       |                |
+-------+------------+------+-----+---------+----------------+
3 rows in set (0.00 sec)
\end{verbatim}


\noindent\rule{\textwidth}{0.4pt}


\par 然后,程序显示内容时,使用如下的语句:

\noindent\rule{\textwidth}{0.4pt}


\begin{verbatim}mysql> select * from test where state=1;
+----+---------+-------+
| id | name    | state |
+----+---------+-------+
|  2 | oldgirl |     1 |
|  3 | inca    |     1 |
|  4 | zuma    |     1 |
|  5 | kaka    |     1 |
+----+---------+-------+
4 rows in set (0.00 sec)
\end{verbatim}


\noindent\rule{\textwidth}{0.4pt}


\par 此时,如果要删除数据,就可以用update替代delete实现逻辑删除,现在删除oldgirl所在的行:

\noindent\rule{\textwidth}{0.4pt}


\begin{verbatim}mysql> update test set state=0 where name='oldgirl';
Query OK, 1 row affected (0.00 sec)
Rows matched: 1  Changed: 1  Warnings: 0
\end{verbatim}


\noindent\rule{\textwidth}{0.4pt}


\par 查看删除结果,可以发现oldgirl所在的行确实消失了,但实际上数据表里依然有,只不过状态为0,从而不显示了:

\noindent\rule{\textwidth}{0.4pt}


\begin{verbatim}mysql> select * from test where state=1;
+----+------+-------+
| id | name | state |
+----+------+-------+
|  3 | inca |     1 |
|  4 | zuma |     1 |
|  5 | kaka |     1 |
+----+------+-------+
3 rows in set (0.00 sec)
\end{verbatim}


\noindent\rule{\textwidth}{0.4pt}


\par 
\begin{figure}[htbp]
\centering
\includegraphics[width=0.8\textwidth]{Image00006.jpg}
\end{figure}

 提示:


\par 网页正常显示的数据SQL语句为“select*from test where state=1;”

\par 删除上述oldgirl的记录的语句为“update test set state=0 where name='oldgirl';”

\par 因此实际上数据并未真的删除,而是显示状态变为0了,如果想要真正删除,还可以写个在夜里定时任务清理state为0的状态行。

\par 6.清空表中的数据

\par 命令语法为:

\noindent\rule{\textwidth}{0.4pt}


\begin{verbatim}truncate table表名;
\end{verbatim}


\noindent\rule{\textwidth}{0.4pt}


\par 以下是清空实践:

\noindent\rule{\textwidth}{0.4pt}


\begin{verbatim}mysql> select * from test; \#<==清空前查看下。
+----+---------+-------+
| id | name    | state |
+----+---------+-------+
|  2 | oldgirl |     0 |
|  3 | inca    |     1 |
|  4 | zuma    |     1 |
|  5 | kaka    |     1 |
+----+---------+-------+
4 rows in set (0.00 sec)
mysql> truncate table test; \#<==执行清空命令。
Query OK, 0 rows affected (0.00 sec)
mysql> select * from test; \#<==查询发现数据为空了。
Empty set (0.00 sec)
\end{verbatim}


\noindent\rule{\textwidth}{0.4pt}


\par truncate和delete是有区别的,简单说明如下。

\par ·TRUNCATE与不带WHERE子句的DELETE语句功能相同:两者均删除表中的全部行,但TRUNCATE比DELETE速度更快。

\par ·TRUNCATE通过释放存储表数据所用的数据页来删除数据,并且只在事务日志中记录页的释放,因此使用的系统和事务日志资源更少。

\par ·DELETE语句每次删除一行,并且会在事务日志中为所删除的每一行记录一项。


% From text00098.html


\section{6.4 参考资料}


\par 1)SQL介绍

\par 
\href{http://baike.baidu.com/view/595350.htm?fromId=34}{http://baike.baidu.com/view/595350.htm?fromId=34}


\par 2)SQL分类

\par 
\href{http://baike.baidu.com/view/595350.htm?fromId=34}{http://baike.baidu.com/view/595350.htm?fromId=34}



% From text00099.html


\section{6.5 章节试题}


\par 1)登录MySQL数据库。

\par 2)查看当前登录的用户。

\par 3)创建数据库oldboy,并查看已建库完整语句。

\par 4)创建用户oldboy,使之可以管理数据库oldboy。

\par 5)查看创建的用户oldboy拥有哪些权限。

\par 6)查看当前数据库里有哪些用户。

\par 7)进入oldboy数据库。

\par 8)查看当前所在的数据库。

\par 9)创建一张表test,字段id和name varchar(16)。

\par 10)查看建表结构及表结构的SQL语句。

\par 11)插入一条数据“1,oldboy”。

\par 12)再批量插入2行数据“2,老男孩”“3,oldboyedu”。

\par 13)查询名字为oldboy的记录。

\par 14)把数据id等于1的名字oldboy更改为oldgirl。

\par 15)在字段name前插入age字段,类型tinyint(2)。

\par 16)不退出数据库备份oldboy数据库。

\par 17)删除test表中的所有数据,并查看。

\par 18)删除表test和oldboy数据库并查看。

\par 19)不退出数据库恢复以上删除的数据。

\par 20)将id列设置为主键,在Name字段上创建普通索引。

\par 21)在字段name后插入手机号字段(shouji),类型为char(11)。

\par 22)所有字段上插入2条记录(自行设定数据)。

\par 23)删除Name列的索引。

\par 24)查询手机号以135开头的,名字为oldboy的记录(提前插入)。

\par 25)收回oldboy用户的select权限。

\par 26)删除oldboy用户。

\par 27)删除oldboy数据库。

\par 28)使用mysqladmin关闭数据库。

\par 29)MySQL密码丢了,请找回。

\par 30)简述SQL语句执行原理流程。


% From text00100.html


\chapter{第7章 MySQL数据库备份与恢复基础实践}


\section{
7.1 MySQL数据库的备份与恢复}
}


\subsection{
7.1.1 备份数据的意义}
}


\par 经常有网友问,运维到底是什么工作,到底要做些什么?老男孩认为,运维工作的核心简单概括起来就是两件事:第一个是保护公司的数据,第二个是让网站能够7*24小时提供服务。

\par 虽然这两件事情都很重要,但是相比较而言,丢失一部分数据和让网站7*24小时提供服务,哪个更重要呢?

\par 这就要看具体的公司和具体的业务了。

\par 例如,对于类似于百度搜索、腾讯这样的公司,丢失几万条记录(比如QQ聊天记录)可能并不算啥,这时7*24小时提供服务会更重要;而在金融行业(比如银行),数据是最重要的,一条都不能丢,而且不能泄漏。相较之下,可能宕机、停机的影响并没有那么大。

\par 对于绝大多数企业来讲,失去数据就相当于失去商机、失去产品、失去客户,甚至会造成公司倒闭,
 那么,在所有的数据中,最核心的数据又是哪些呢?这恐怕要属数据库中的数据了,当然,并不是说其他数据不重要,只是这一部分更具代表性。既然数据库中的数据地位这么高,那么数据库备份与恢复的重要性就不言而喻了。


% From text00101.html


\subsection{7.1.2 使用mysqldump进行数据库备份实践}


\par MySQL数据库自带了一个很优秀的备份命令,即mysqldump,下面就来了解它的用法。

\par mysqldump语法如下:

\noindent\rule{\textwidth}{0.4pt}


\begin{verbatim}mysqldump -u用户名 -p密码 参数 数据库名> 备份的文件名.sql \#<==-u和-p后可无空格。
\end{verbatim}


\noindent\rule{\textwidth}{0.4pt}


\par 
\begin{figure}[htbp]
\centering
\includegraphics[width=0.8\textwidth]{Image00006.jpg}
\end{figure}

 特别说明:
 为了防止密码外泄,在第5章中,我们已经将密码写入配置文件中,具体配置见5.2节,因此,本章会使用不带用户和密码的命令替换带用户和密码的命令,例如,使用如下命令:

\noindent\rule{\textwidth}{0.4pt}


\begin{verbatim}[root@oldboy \textasciitilde{}]\# mysql </opt/bak.sql
\end{verbatim}


\noindent\rule{\textwidth}{0.4pt}


\par 替换如下命令:

\noindent\rule{\textwidth}{0.4pt}


\begin{verbatim}[root@oldboy \textasciitilde{}]\# mysql -uroot -poldboy123 </opt/bak.sql
\#<==这是使用mysql将第5章的实验数据还原的命令。
\end{verbatim}


\noindent\rule{\textwidth}{0.4pt}


\par 1.不带参数备份单个数据库

\par 
范例
 :不加任何参数备份名字为oldboy的库。

\par 首先,查看备份前的数据:

\noindent\rule{\textwidth}{0.4pt}


\begin{verbatim}[root@oldboy \textasciitilde{}]\# mysql
Welcome to the MySQL monitor.  Commands end with ; or \textbackslash\{\}g.
...省略若干行...
mysql> use oldboy
Database changed
mysql> select * from test; \#<==一共5行记录。
+----+---------+
| id | name    |
+----+---------+
|  1 | oldboy  |
|  2 | oldgirl |
|  3 | inca    |
|  4 | zuma    |
|  5 | kaka    |
+----+---------+
5 rows in set (0.00 sec)
mysql>
\end{verbatim}


\noindent\rule{\textwidth}{0.4pt}


\par 然后,执行备份oldboy库的命令,命令如下:

\noindent\rule{\textwidth}{0.4pt}


\begin{verbatim}[root@oldboy \textasciitilde{}]\# mysqldump oldboy >/opt/mysql\_bak.sql  \#<==密码在配置文件里了。
\end{verbatim}


\noindent\rule{\textwidth}{0.4pt}


\par 最后,使用egrep命令检查备份的结果:

\noindent\rule{\textwidth}{0.4pt}


\begin{verbatim}[root@oldboy \textasciitilde{}]\# egrep -v "\#|\textbackslash\{\}*|--|\textasciicircum{}\$" /opt/mysql\_bak.sql
                                             \#<==为了清晰排除了注释等特殊字符。
DROP TABLE IF EXISTS `test`;                 \#<==删除表语句。
CREATE TABLE `test` (                        \#<==建表语句。
  `id` int(4) NOT NULL AUTO\_INCREMENT,
  `name` char(20) NOT NULL,
  PRIMARY KEY (`id`)
) ENGINE=InnoDB AUTO\_INCREMENT=6 DEFAULT CHARSET=utf8;
LOCK TABLES `test` WRITE;
INSERT INTO `test` VALUES (1,'oldboy'),(2,'oldgirl'),(3,'inca'),(4,'zuma'), (5,'kaka');                                      \#<==插入语句。
UNLOCK TABLES;
\end{verbatim}


\noindent\rule{\textwidth}{0.4pt}


\par 可能读者已经发现了一个现象,mysqldump备份的内容就是曾经执行过的SQL语句,所不同的是,为了恢复效率,mysqldump把数据写成了一个insert语句,另外多了两行锁表和解锁的操作。

\par 2.加-B参数备份的实践

\par 下面以备份oldboy库时加-B参数为例进行讲解。

\par 备份命令如下:

\noindent\rule{\textwidth}{0.4pt}


\begin{verbatim}[root@oldboy \textasciitilde{}]\#  mysqldump -B oldboy >/opt/mysql\_bak\_B.sql
\end{verbatim}


\noindent\rule{\textwidth}{0.4pt}


\par 将该例和上一个范例(没加参数的备份文件)进行对比,即可了解“-B”参数的作用:

\noindent\rule{\textwidth}{0.4pt}


\begin{verbatim}[root@oldboy \textasciitilde{}]\# diff /opt/mysql\_bak.sql /opt/mysql\_bak\_B.sql
\#<==使用diff或vimdiff比较。
18a19,26
> -- Current Database: `oldboy`
> --
>
> CREATE DATABASE /*!32312 IF NOT EXISTS*/ `oldboy` /*!40100 DEFAULT CHARACTER SET utf8 */;
>
> USE `oldboy`;
>
> --
51c59
< -- Dump completed on 2017-03-02 10:00:56
---
> -- Dump completed on 2017-03-02 10:05:39
\end{verbatim}


\noindent\rule{\textwidth}{0.4pt}


\par 可以看到,加“B”参数的作用是增加创建数据库和连接数据库的语句
 。即如下这两条语句:

\noindent\rule{\textwidth}{0.4pt}


\begin{verbatim}CREATE DATABASE /*!32312 IF NOT EXISTS*/ `oldboy` /*!40100 DEFAULT CHARACTER SET utf8 */;
USE `oldboy`;
\end{verbatim}


\noindent\rule{\textwidth}{0.4pt}


\par 别小看这两条语句,在恢复数据库时可是有大作用的,后面讲解恢复知识时会进一步说明。

\par 除此之外,若使用“B”参数备份数据库,那么后面可以直接接多个库名,即同时备份多个库。

\par 3.使用gzip压缩备份数据库的实践

\par 指定压缩命令gzip压缩备份oldboy数据库,命令如下:

\noindent\rule{\textwidth}{0.4pt}


\begin{verbatim}[root@oldboy \textasciitilde{}]\# mysqldump -B oldboy|gzip>/opt/mysql\_bak\_B.sql.gz
\#<==注意压缩命令前要加管道。
[root@oldboy \textasciitilde{}]\# ls -l /opt/
total 16
-rw-r--r--. 1 root root 2051 Mar  2 03:32 bak.sql
-rw-r--r--. 1 root root 2051 Mar  2 10:05 mysql\_bak\_B.sql
\#<==没有压缩的备份数据较大。
-rw-r--r--. 1 root root  779 Mar  2 10:16 mysql\_bak\_B.sql.gz
\#<==压缩过的数据库备份小了很多。
-rw-r--r--. 1 root root 1908 Mar  2 10:00 mysql\_bak.sql
\end{verbatim}


\noindent\rule{\textwidth}{0.4pt}


\par 同样的数据,没有压缩备份时是2051字节,压缩备份后为779字节,整整缩小了三分之二。

\par 
阶段性知识点小结


\par 通过以上几个范例,可以得出几个重要的mysqldump参数知识点。

\par ·使用-B参数备份数据库,会在备份的数据中增加建库及use库的语句。

\par ·使用-B参数备份数据库,后面还可以直接接多个库名,实现同时备份多个库。

\par ·使用gzip命令压缩备份的数据,备份的压缩包是原来的1/3大小。

\par 4.mysqldump命令工作原理

\par 利用mysqldump命令备份数据的过程,实际上就是把数据(包括库表)从MySQL库里以SQL语句的形式直接输出或者生成备份文件的过程,这种备份成SQL语句的方式称为逻辑备份。

\par 把备份的数据过滤掉注释信息,剩下的都是SQL语句,过滤结果如下:

\noindent\rule{\textwidth}{0.4pt}


\begin{verbatim}[root@oldboy \textasciitilde{}]\# egrep -v "\#|\textbackslash\{\}*|--|\textasciicircum{}\$" /opt/mysql\_bak.sql
DROP TABLE IF EXISTS `test`;
CREATE TABLE `test` (
  `id` int(4) NOT NULL AUTO\_INCREMENT,
  `name` char(20) NOT NULL,
  PRIMARY KEY (`id`)
) ENGINE=InnoDB AUTO\_INCREMENT=6 DEFAULT CHARSET=utf8;
LOCK TABLES `test` WRITE;
INSERT INTO `test` VALUES (1,'oldboy'),(2,'oldgirl'),(3,'inca'),(4,'zuma'), (5,'kaka');
UNLOCK TABLES;
\end{verbatim}


\noindent\rule{\textwidth}{0.4pt}


\par 使用mysqldump命令可以把数据库中的数据导出来,并通过SQL语句的形式存储。这种备份方式称为逻辑备份,效率不是很高。在当下的生产场景中,多用于数据量不是很大的备份情况,例如30GB以内的数据。若在数据库数据很大的时候采用此备份方法,则所用的时间就会很长,恢复的时间也会很长,因此,当数据大于30GB(参考值)后,建议选择其他的诸如Xtrabackup的物理方式进行备份和恢复,见第9章内容。

\par 5.备份多个库

\par (1)备份多个库的实践

\par 前文已经提到过,备份多个库时,可使用mysqldump命令的参数-B实现,操作结果如下:

\noindent\rule{\textwidth}{0.4pt}


\begin{verbatim}mysql> show databases;\#<==查看当前数据库里的库信息。
+--------------------+
| Database           |
+--------------------+
| information\_schema |
| mysql              | \#<==实验备用库mysql。
| oldboy             | \#<==实验备用库oldboy。
| oldboy\_utf8        | \#<==实验备用库oldboy\_utf8。
| performance\_schema |
+--------------------+
5 rows in set (0.00 sec)
mysql> quit
Bye
[root@oldboy \textasciitilde{}]\#  mysqldump -B oldboy oldboy\_utf8|gzip>/opt/test.sql.gz
\#<==同时备份两个库。
\#提示:使用-B参数即表示后面可接多个库,并且在备份文件里会增加use db,和create database db的信息。
[root@oldboy \textasciitilde{}]\#  mysqldump -B oldboy oldboy\_utf8 mysql|gzip>/opt/all.sql.gz
\#<==MySQL 5.5数据库备份mysql库时需要加一个--events参数否则会有警告信息,
    MySQL 5.6版就没有提示了。
[root@oldboy \textasciitilde{}]\# ls -l /opt/all.sql.gz
-rw-r--r--. 1 root root 178750 Mar  2 11:09 /opt/all.sql.gz
\#<==备份的数据库不能太小(例如几字节),否则可能是由于命令或参数等导致备份结果有问题。
\end{verbatim}


\noindent\rule{\textwidth}{0.4pt}


\par (2)如何做分库备份

\par 上文使用mysqldump命令备份时,是把多个库备份成了一个文件。有时一个企业的数据库实例里会有多个库,例如(www、bbs、blog),但是出问题的时候可能只是其中某一个库,如果在备份时把所有的库都备份成了一个数据文件的话,恢复某一个库的数据时就比较麻烦了,因此有了分库备份的需求。

\par 分库备份实际上就是每次只执行一个mysqldump备份命令语句备份一个库,如果数据库里有多个库,就执行多条相同的语句来备份各个库。注意每个库都可以用对应备份的库作为库名,结尾加“.sql”即可。备份多个库的命令如下:

\noindent\rule{\textwidth}{0.4pt}


\begin{verbatim}mysqldump -uroot -p'oldboy123' -B oldboy ...
mysqldump -uroot -p'oldboy123' -B oldboy\_utf8 ...
mysqldump -uroot -p'oldboy123' -B mysql ...
......
\end{verbatim}


\noindent\rule{\textwidth}{0.4pt}


\par 这里分享一个特殊技巧,即用如下命令实现分库备份,操作过程部分实现如下:

\noindent\rule{\textwidth}{0.4pt}


\begin{verbatim}[root@oldboy \textasciitilde{}]\# mysql -e "show databases;"|egrep -v "\_schema|atabase"
\#<==取出需要备份的库名列表。
mysql
oldboy
oldboy\_utf8
[root@oldboy \textasciitilde{}]\# mysql -e "show databases;"|egrep -v "\_schema|atabase"|sed -r 's\#\textasciicircum{}(.*)\#mysqldump -B \textbackslash\{\}1|gzip >/tmp/\textbackslash\{\}1.sql.gz\#g' \#<==利用命令拼接成多个mysqldump备份的具体命令,如下。
mysqldump -B mysql|gzip >/tmp/mysql.sql.gz
mysqldump -B oldboy|gzip >/tmp/oldboy.sql.gz
mysqldump -B oldboy\_utf8|gzip >/tmp/oldboy\_utf8.sql.gz
[root@oldboy \textasciitilde{}]\# mysql -e "show databases;"|egrep -v "\_schema|atabase"|sed -r 's\#\textasciicircum{}(.*)\#mysqldump -B \textbackslash\{\}1|gzip >/tmp/\textbackslash\{\}1.sql.gz\#g'|bash \#<==最后管道对bash执行这些备份命令,就是一次分开备份多个库了。
[root@oldboy \textasciitilde{}]\# ls -l /tmp/*.sql.gz \#<==大功告成,怎么样?
-rw-r--r--. 1 root root 178591 Mar  2 11:35 /tmp/mysql.sql.gz
-rw-r--r--. 1 root root    778 Mar  2 11:35 /tmp/oldboy.sql.gz
-rw-r--r--. 1 root root    510 Mar  2 11:35 /tmp/oldboy\_utf8.sql.gz
\end{verbatim}


\noindent\rule{\textwidth}{0.4pt}


\par 当然了,读者还可以用Shell脚本实现分库备份的脚本,可以看老男孩曾经录制的视频:\href{http://edu.51cto.com/course/course_id-808.html}{http://edu.51cto.com/course/course\_id-808.html}
 。

\par 6.备份单个表

\par 语法如下:

\noindent\rule{\textwidth}{0.4pt}


\begin{verbatim}mysqldump -u用户名 -p数据库名 表名>备份的文件名
\end{verbatim}


\noindent\rule{\textwidth}{0.4pt}


\par 当不加-B参数备份数据库时,例如“mysqldump oldboy test”,mysqldump命令默认就会把oldboy当作库,把test当作表,如果后面还有多个字符串,例如“mysqldump oldboy test test1”,那么除了oldboy为库之外,其他的test、test1都是oldboy库的表。

\par 执行结果如下:

\noindent\rule{\textwidth}{0.4pt}


\begin{verbatim}[root@oldboy \textasciitilde{}]\# mysqldump oldboy test>/tmp/oldboy\_test.sql
\#<==仅备份oldboy库里的test表。
\end{verbatim}


\noindent\rule{\textwidth}{0.4pt}


\par 7.备份多个表

\par 语法如下:

\noindent\rule{\textwidth}{0.4pt}


\begin{verbatim}mysqldump -u用户名 -p数据库名 表名1 表名2 > 备份的文件名
\end{verbatim}


\noindent\rule{\textwidth}{0.4pt}


\par 下面的命令可同时备份mysql库下的user表和db表:

\noindent\rule{\textwidth}{0.4pt}


\begin{verbatim}[root@oldboy \textasciitilde{}]\#  mysqldump mysql user db>/tmp/mysql.sql
提示:此时不能加-B参数了,因为库后面都是表了。
[root@oldboy \textasciitilde{}]\# egrep -v "\#|\textbackslash\{\}*|--|\textasciicircum{}\$" /tmp/mysql.sql
DROP TABLE IF EXISTS `user`;
CREATE TABLE `user` (  \#<==user表。
  `Host` char(60) COLLATE utf8\_bin NOT NULL DEFAULT '',
  `User` char(16) COLLATE utf8\_bin NOT NULL DEFAULT '',
  `Password` char(41) CHARACTER SET latin1 COLLATE latin1\_bin NOT NULL DEFAULT '',
...省略若干行...
  `password\_expired` enum('N','Y') CHARACTER SET utf8 NOT NULL DEFAULT 'N',
  PRIMARY KEY (`Host`,`User`)
) ENGINE=MyISAM DEFAULT CHARSET=utf8 COLLATE=utf8\_bin COMMENT='Users and global privileges';
LOCK TABLES `user` WRITE;
UNLOCK TABLES;
DROP TABLE IF EXISTS `db`;
CREATE TABLE `db` (  \#<==db表。
  `Host` char(60) COLLATE utf8\_bin NOT NULL DEFAULT '',
  `Db` char(64) COLLATE utf8\_bin NOT NULL DEFAULT '',
  `User` char(16) COLLATE utf8\_bin NOT NULL DEFAULT '',
...省略若干行...
  PRIMARY KEY (`Host`,`Db`,`User`),
  KEY `User` (`User`)
) ENGINE=MyISAM DEFAULT CHARSET=utf8 COLLATE=utf8\_bin COMMENT='Database privileges';
LOCK TABLES `db` WRITE;
INSERT INTO `db` VALUES ('\%','test','','Y','Y','Y','Y','Y','Y','N','Y','Y','Y','Y','Y','Y','Y','Y','N','N','Y','Y'),('\%','test\textbackslash\{\}\textbackslash\{\}\_\%','','Y','Y','Y','Y','Y','Y','N','Y','Y','Y','Y','Y','Y','Y','Y','N','N','Y','Y'),('localhost','oldboy','test','N','Y','Y','Y','Y','Y','N','Y','Y','Y','Y','Y','Y','Y','Y','Y','Y','Y','Y'),('172.16.1.\%','blog','blog','Y','Y','Y','Y','N','N','N','N','N','N','N','N','N','N','N','N','N','N','N');
UNLOCK TABLES;
\end{verbatim}


\noindent\rule{\textwidth}{0.4pt}


\par 可以看到mysql.sql文件里确实有了user表和db表的备份信息。

\par 企业中可能会存在这样的情况,一个库里有大表也有小表,有时可能需要只恢复某一个小表,上面实现的多表备份文件很难拆开,就像没有分库那样会导致恢复某一个小表也很麻烦。那么又如何进行分表备份呢?

\par 这里的思路和分库备份是一样的,每执行一条语句就备份一个表,生成不同的数据文件即可。命令如下:

\noindent\rule{\textwidth}{0.4pt}


\begin{verbatim}mysqldump oldboy test > oldboy\_test.sql
mysqldump oldboy test1> oldboy\_test1.sql
......
\end{verbatim}


\noindent\rule{\textwidth}{0.4pt}


\par 将上述命令放入一个脚本里就是脚本分表备份了,当然这样是很土的备份脚本了,更好的备份脚本见分库分表备份视频,我们在学习Shell脚本编程时,会实现这个分表分库备份的脚本,视频链接为\href{http://edu.51cto.com/course/course_id-808.html}{http://edu.51cto.com/course/course\_id-808.html}
 。

\par 分表备份的缺点:数据文件多,很碎,一旦需要全部恢复又很麻烦。

\par 解决办法具体如下。

\par ·做一个完整全备,再做一个分库分表备份。

\par ·虽然文件多、碎,但可以利用脚本批量操作多个SQL文件。

\par 8.企业备份案例解析

\par 企业面试案例:如果多个库或多个表备份到了一个文件里,那么这种情况下,如何恢复单个库或者单个表?

\par 解答具体如下。

\par ·找个第三方测试库,将所有备份都导入到这个测试库里,然后把需要的单库或表再备份出来,最后恢复到需要恢复的正式库里。

\par ·如果是单表恢复,还可以执行“grep-w表名bak.sql>表名.sql”命令。

\par 当然最好是备份时提前采用分库分表备份。

\par 9.备份数据库表结构(不包含数据)

\par 利用mysqldump的-d参数可以只备份表的结构,即建表的语句。下面的示例将展示如何备份oldboy库中所有表的结构:

\noindent\rule{\textwidth}{0.4pt}


\begin{verbatim}[root@oldboy \textasciitilde{}]\# mysqldump -d oldboy >/opt/oldboy.sql
\#<==备份oldboy库的所有表结构,也可以指定某一个表。
[root@oldboy \textasciitilde{}]\# egrep -v "\#|\textbackslash\{\}*|--|\textasciicircum{}\$" /opt/oldboy.sql
DROP TABLE IF EXISTS `test`;
CREATE TABLE `test` (
  `id` int(4) NOT NULL AUTO\_INCREMENT,
  `name` char(20) NOT NULL,
  PRIMARY KEY (`id`)
) ENGINE=InnoDB AUTO\_INCREMENT=6 DEFAULT CHARSET=utf8;
\end{verbatim}


\noindent\rule{\textwidth}{0.4pt}


\par 说明一下,因为oldboy库里只有test表,因此这里只显示了test的表结构。

\par 10.只备份数据库表的数据(不包含表结构)

\par 利用-t参数备份数据库表的数据(SQL语句形式),命令如下:

\noindent\rule{\textwidth}{0.4pt}


\begin{verbatim}[root@oldboy \textasciitilde{}]\# mysqldump -t oldboy >/opt/oldboy1.sql
\#<==备份oldboy库的所有表的内容,也可以指定某一个表。
[root@oldboy \textasciitilde{}]\# egrep -v "\#|\textbackslash\{\}*|--|\textasciicircum{}\$" /opt/oldboy1.sql
LOCK TABLES `test` WRITE;
INSERT INTO `test` VALUES (1,'oldboy'),(2,'oldgirl'),(3,'inca'),(4,'zuma'), (5,'kaka');
UNLOCK TABLES;
\end{verbatim}


\noindent\rule{\textwidth}{0.4pt}


\par 11.同时将数据和表结构分离导出

\par 利用-T参数可以实现将数据和表结构同时分离备份。不过有可能会碰到如下情况:

\noindent\rule{\textwidth}{0.4pt}


\begin{verbatim}[root@oldboy \textasciitilde{}]\# mysqldump oldboy test --compact -T /tmp/
\#<==5.6版本因为安全权限问题不能直接导出了。
mysqldump: Got error: 1290: The MySQL server is running with the --secure-file-priv option so it cannot execute this statement when executing 'SELECT INTO OUTFILE' \#<==提示是因为--secure-file-priv选项所导致。
\end{verbatim}


\noindent\rule{\textwidth}{0.4pt}


\par 如果依然想以此方式备份数据,那么可以调整配置文件参数并重启MySQL:

\noindent\rule{\textwidth}{0.4pt}


\begin{verbatim}[root@oldboy \textasciitilde{}]\# grep sec /etc/my.cnf
secure\_file\_priv='' \#<==在[mysqld]模块下增加
[root@oldboy \textasciitilde{}]\# /etc/init.d/mysqld restart
Shutting down MySQL.. SUCCESS!
Starting MySQL.... SUCCESS!
\end{verbatim}


\noindent\rule{\textwidth}{0.4pt}


\par 备份的命令如下:

\noindent\rule{\textwidth}{0.4pt}


\begin{verbatim}[root@oldboy \textasciitilde{}]\# mysqldump oldboy test --compact -T /tmp/
\#<==--compact是减少无用输出的意思。
[root@oldboy \textasciitilde{}]\# ls -l /tmp/test*
-rw-r--r--. 1 root  root  331 Mar  3 03:09 /tmp/test.sql
-rw-rw-rw-. 1 mysql mysql  40 Mar  3 03:09 /tmp/test.txt
[root@oldboy \textasciitilde{}]\# cat /tmp/test.sql              \#<==建表的语句。
/*!40101 SET @saved\_cs\_client     = @@character\_set\_client */;
/*!40101 SET character\_set\_client = utf8 */;
CREATE TABLE `test` (
  `id` int(4) NOT NULL AUTO\_INCREMENT,
  `name` char(20) NOT NULL,
  PRIMARY KEY (`id`)
) ENGINE=InnoDB AUTO\_INCREMENT=6 DEFAULT CHARSET=utf8;
/*!40101 SET character\_set\_client = @saved\_cs\_client */;
[root@oldboy \textasciitilde{}]\# cat /tmp/test.txt              \#<==纯文本数据,非SQL语句了。
1      oldboy
2      oldgirl
3      inca
4      zuma
5      kaka
\end{verbatim}


\noindent\rule{\textwidth}{0.4pt}


\par 
mysqldump参数新小结


\par 通过以上讲解,可以得出如下关于mysqldump参数的新知识点。

\par ·-d参数的作用是只备份库表结构(SQL语句形式)。

\par ·-t参数的作用是只备份表内的数据(SQL语句形式)。

\par ·-T将库表和数据分离成不同的文件,数据是纯文本,表结构是SQL语句。

\par 12.刷新binlog文件参数(-F)

\par (1)binlog是什么

\par Binlog是一个二进制格式的文件,用于记录用户对数据库更新的SQL语句信息,例如更改数据库库表和更改表内容的SQL语句都会记录到binlog里,但是对库表等内容的查询则不会记录到日志中。

\par (2)binlog对于备份的作用

\par 当有数据写入到数据库时,还会同时把更新的SQL语句写入到对应的binlog文件里,这个文件就是上文所说的binlog文件。

\par 使用mysqldump备份时,一般是对某一时刻的数据进行全备,例如,0点进行数据库备份。

\par 假设是每天0点对数据库进行备份,那么在两次备份之间就有24小时的数据没有备份,在这期间如果数据库发生故障,使用mysqldump全量恢复也只能恢复到当日0点,但是有了binlog文件,就可以将两次完整备份间隔之间的数据还原,因为binlog文件里的数据就是写入数据库的数据,使用binlog文件恢复数据,我们称之为二进制增量数据恢复。

\par (3)为什么要刷新binlog

\par 刷新(切割)binlog日志的目的就是确定全备和增量备(binlog文件)的临界点,当全备完成后,全备时刻以前的binlog文件就无用了(全备里已有这部分数据了),但是全备以后到下一次全备之前的数据就是十分重要的,这部分数据就存在于binlog文件里,因此在进行全备时需要找到全备之后和binlog增量之间的临界点,使得恢复时,需要的binlog文件数据一条不多(不能和全备的内容重合),一条不少(全备后的所有数据都要有)。图7-1是在企业中使用-F刷新日志的原理图。

\begin{figure}[htbp]
\centering
\includegraphics[width=0.8\textwidth]{Image00049.jpg}
\end{figure}



\par 图7-1 使用-F刷新日志的原理图

\par (4)如何开启binlog功能?

\par binlog文件生效需要一个参数:log\_bin,编辑配置文件增加log\_bin参数即可,注意,MySQL 5.6版本变成了下划线连接的格式(MySQL 5.6以前是中杠)。

\noindent\rule{\textwidth}{0.4pt}


\begin{verbatim}[root@oldboy \textasciitilde{}]\# grep log\_bin /etc/my.cnf
log\_bin \#<==默认binlog文件名前缀为“主机名-bin”,也可以自定义名字,例如log\_bin =
             mysql-bin。
[root@oldboy \textasciitilde{}]\# /etc/init.d/mysqld restart
Shutting down MySQL.. SUCCESS!
Starting MySQL. SUCCESS!
[root@oldboy \textasciitilde{}]\# ls -lrt /application/mysql/data/oldboy-bin*
-rw-rw----. 1 mysql mysql 143 Mar  3 05:50 /application/mysql/data/oldboy-bin.000001 \#<==日志文件。
-rw-rw----. 1 mysql mysql  40 Mar  3 05:50 /application/mysql/data/oldboy-bin.index  \#<==索引文件。
\end{verbatim}


\noindent\rule{\textwidth}{0.4pt}


\par binlog日志切割就是确定全备和binlog增量备份的临界点,当然,后文还有类似的参数,如“--master-data”。

\par (5)使用-F刷新binlog日志

\par 使用-F将会从备份后的时刻起重新记录binlog日志文件,将来增量恢复从新的binlog日志文件开始即可。

\par 例如,早晨10点丢失数据需要恢复数据,则数据恢复步骤具体如下。

\par 1)将0:00点时刻备份的全备数据还原到数据库,这个时候数据就恢复到了当日00点。

\par 2)0:00点-10:00丢失的数据,就要从全备后当天的所有binlog里恢复,而使用-F切割日志,就是找到0:00点这个时刻全备和binlog接缝的起始binlog文件。

\noindent\rule{\textwidth}{0.4pt}


\begin{verbatim}[root@oldboy \textasciitilde{}]\# mysqldump -F -B oldboy|gzip >/opt/bak\_\$(date +\%F).sql.gz
\#<==带-F备份。
[root@oldboy \textasciitilde{}]\# ls -lrt /application/mysql/data/oldboy-bin*
-rw-rw----. 1 mysql mysql 143 Mar  3 05:50 /application/mysql/data/oldboy-bin.000001
-rw-rw----. 1 mysql mysql 168 Mar  3 05:57 /application/mysql/data/oldboy-bin.000002 \#<==新binlog。
-rw-rw----. 1 mysql mysql  60 Mar  3 05:57 /application/mysql/data/oldboy-bin.index
[root@oldboy \textasciitilde{}]\# mysqldump -F -B oldboy|gzip >/opt/bak\_\$(date +\%F).sql.gz
[root@oldboy \textasciitilde{}]\# ls -lrt /application/mysql/data/oldboy-bin*
-rw-rw----. 1 mysql mysql 143 Mar  3 05:50 /application/mysql/data/oldboy-bin.000001
-rw-rw----. 1 mysql mysql 168 Mar  3 05:57 /application/mysql/data/oldboy-bin.000002
-rw-rw----. 1 mysql mysql 168 Mar  3 05:57 /application/mysql/data/oldboy-bin.000003 \#<==新binlog。
-rw-rw----. 1 mysql mysql  80 Mar  3 05:57 /application/mysql/data/oldboy-bin.index
提示:可以看到,使用-F参数后,每备份一次,都会重新生成新的binlog记录数据。
\end{verbatim}


\noindent\rule{\textwidth}{0.4pt}


\par 13.记录binlog位置的特殊参数(--master-data)

\par mysqldump里提供了一个参数,使得管理员不用刷新binlog,也可以找到全量和增量的临界点,这就是“--master-data”参数。使用这个参数备份后,在备份的文件对应的SQL语句里会添加CHANGE MASTER语句及binlog文件及位置点信息。

\par 当“--master-data=1”时,备份结果为可执行的“CHANGE MASTER...”语句;当“--master-data=2”时,备份结果为注释的“--CHANGE MASTER...”语句,“--”在SQL语句里为注释的意思。“--master-data”参数除了确定增量恢复和全备之间的临界点之外,进行主从复制时的作用更大,详情见后文。

\par 操作示例如下:

\noindent\rule{\textwidth}{0.4pt}


\begin{verbatim}[root@oldboy \textasciitilde{}]\# mysqldump --master-data=1 oldboy --compact|head -1
\#<==--master-data=1的结果。
CHANGE MASTER TO MASTER\_LOG\_FILE='oldboy-bin.000004', MASTER\_LOG\_POS=120;
[root@oldboy \textasciitilde{}]\# mysqldump --master-data=2 oldboy --compact|head -1
\#<==--master-data=2的结果。
-- CHANGE MASTER TO MASTER\_LOG\_FILE='oldboy-bin.000004', MASTER\_LOG\_POS=120;
\end{verbatim}


\noindent\rule{\textwidth}{0.4pt}


\par 14.锁定所有表备份(-x参数)

\par 默认情况下,在使用mysqldump命令备份期间,数据库依然是可以写入数据的,此时,备份的数据就不是某一个时刻的一致性备份了,因此也会在恢复时造成困扰。例如备份时使用-x锁表备份就会取得0点时刻的完整备份,即在0点时刻停止所有写入操作,然后导出备份数据,备份完毕,放开写入,备份时刻停止写入的情况如图7-2所示。

\par 如果不锁表进行备份,就会无法得到某一个时刻的完整备份,由于是一边备份数据一边写入数据,因此最后备份的数据很可能就是一个时间段的数据,而非某一个时刻的一致性全备(如图7-3所示)。

\begin{figure}[htbp]
\centering
\includegraphics[width=0.8\textwidth]{Image00050.jpg}
\end{figure}



\par 图7-2 锁表备份示意图

\begin{figure}[htbp]
\centering
\includegraphics[width=0.8\textwidth]{Image00051.jpg}
\end{figure}



\par 图7-3 不锁表备份示意图

\par 15.innodb表特有的备份参数(--single-transaction)

\par 当使用mysqldump的“--single-transaction”对innodb表进行备份时,会开启一个事务,并将整个备份过程放到一个事务里,以确保执行本次dump会话时,不会看到其他连接会话已经提交了的数据,即备份开始时刻的数据是什么样,备份出来就是什么样子(如图7-4所示)。这也相当于是锁表之后备份的数据,但是这个参数是允许在备份期间写入数据的,而不是在使用“-x”参数锁表之后,备份期间无法写入任何数据。操作例子如下:

\noindent\rule{\textwidth}{0.4pt}


\begin{verbatim}mysqldump -B --master-data=2 --single-transaction oldboy|gzip >/opt/all.sql.gz
\end{verbatim}


\noindent\rule{\textwidth}{0.4pt}


\begin{figure}[htbp]
\centering
\includegraphics[width=0.8\textwidth]{Image00052.jpg}
\end{figure}



\par 图7-4 --single-transaction备份原理示意图

\par 这就是为什么要强调“--single-transaction”是innodb表的特有备份参数了,只有事务型引擎,才能支持这样的备份功能。


% From text00102.html


\subsection{7.1.3 mysqldump重要关键参数说明}


\par mysqldump的参数很多,下面就把企业工作中常见的参数列举出来,如表7-1所示。

\par 表7-1 mysqldump重要参数说明

\begin{figure}[htbp]
\centering
\includegraphics[width=0.8\textwidth]{Image00053.jpg}
\end{figure}



\begin{figure}[htbp]
\centering
\includegraphics[width=0.8\textwidth]{Image00054.jpg}
\end{figure}



\par 标*的参数为工作中比较重要的参数。如果想了解更多的参数请执行“mysqldump-help”。


% From text00103.html


\subsection{7.1.4 生产场景下,不同引擎的mysqldump备份命令}


\par innodb引擎的备份命令如下:

\noindent\rule{\textwidth}{0.4pt}


\begin{verbatim}mysqldump -A -B --master-data=2 --single-transaction |gzip >/opt/all.sql.gz
\end{verbatim}


\noindent\rule{\textwidth}{0.4pt}


\par “--single-transaction”是innodb引擎专有的备份参数,优势就是备份期间数据依然可以执行写操作。“--master-data”的作用具体如下。

\par 适合多引擎混合(例如,myisam与innodb混合)的备份命令如下:

\noindent\rule{\textwidth}{0.4pt}


\begin{verbatim}mysqldump -A -B --master-data=2 |gzip >/opt/alL\_\$(date +\%F).sql.gz
\end{verbatim}


\noindent\rule{\textwidth}{0.4pt}


\par 这个备份的特点是会锁表,在备份期间会影响数据写入,使用“--master-data”会自动开启-x锁表参数功能。

\par 
\begin{figure}[htbp]
\centering
\includegraphics[width=0.8\textwidth]{Image00024.jpg}
\end{figure}

 说明:
 如果数据库用到了存储过程和函数特殊功能,就加上“-R”备份,如果用到了触发器功能则加上“--triggers”备份;如果考虑切割binlog日志也可以加“-F”;如果不想备份所有库,也可以取消“-A”,独立指定多个库名备份。


% From text00104.html


\subsection{7.1.5 利用SQL语句方式对表进行导入导出}


\par 1.导出表

\par (1)导出表SQL语句语法

\par 工作中有时候可能会遇到一些特殊的需求,例如想把表的数据导出到excel里,而不是SQL语句的形式,这时就可能会用到下面的导出表的方法,当然,前文的mysqldump的-T备份结果也可以导入到excel里。

\par 语法如下:

\noindent\rule{\textwidth}{0.4pt}


\begin{verbatim}SELECT * FROM 表名 INTO OUTFILE 'file\_name' export\_options
\end{verbatim}


\noindent\rule{\textwidth}{0.4pt}


\par 其中,export\_options包括如下内容说明。

\par 表7-2 export\_options内容说明

\begin{figure}[htbp]
\centering
\includegraphics[width=0.8\textwidth]{Image00055.jpg}
\end{figure}



\par (2)导出表实践

\par 准备数据:

\noindent\rule{\textwidth}{0.4pt}


\begin{verbatim}mysql> use oldboy
Database changed
mysql> select * from test;
+----+--------+
| id | name   |
+----+--------+
|  1 | full01 |
|  2 | full02 |
|  3 | full03 |
|  4 | full04 |
|  5 | full05 |
+----+--------+
5 rows in set (0.00 sec)
\end{verbatim}


\noindent\rule{\textwidth}{0.4pt}


\par 将oldboy库里的test表数据导出为纯文本:

\noindent\rule{\textwidth}{0.4pt}


\begin{verbatim}mysql> select * from test into outfile "/tmp/oldboy\_test1.txt";
Query OK, 9 rows affected (0.00 sec)
mysql> system cat /tmp/oldboy\_test.txt
1      full01
2      full02
3      full03
4      full04
5      full05
\#提示:默认情况下导出的数据不同域之间用tab键分隔。
\end{verbatim}


\noindent\rule{\textwidth}{0.4pt}


\par 导出时设置字符集:

\noindent\rule{\textwidth}{0.4pt}


\begin{verbatim}select * from test into outfile "/tmp/oldboy\_test2.txt" character set utf8;
\end{verbatim}


\noindent\rule{\textwidth}{0.4pt}


\par 以指定的分隔符导出,这里指定“-”为不同域之间的分隔符:

\noindent\rule{\textwidth}{0.4pt}


\begin{verbatim}mysql> select * from test into outfile "/tmp/oldboy\_test3.txt" fields terminated by '-';
Query OK, 5 rows affected, 1 warning (0.01 sec)
mysql> system cat /tmp/oldboy\_test3.txt
1-full01
2-full02
3-full03
4-full04
5-full05
\end{verbatim}


\noindent\rule{\textwidth}{0.4pt}


\par 导出时设置对字段内容进行引用,使用双引号:

\noindent\rule{\textwidth}{0.4pt}


\begin{verbatim}mysql> select * from test into outfile "/tmp/oldboy\_test4.txt" fields enclosed by '"';
Query OK, 5 rows affected (0.01 sec)
mysql> system cat /tmp/oldboy\_test4.txt
"1"      "full01"
"2"      "full02"
"3"      "full03"
"4"      "full04"
"5"      "full05"
\#提示不同域的内容已经加上了双引号。
\end{verbatim}


\noindent\rule{\textwidth}{0.4pt}


\par 更多参数功能,读者可以自行测试,其实本节内容在工作中用的都不多,读者了解即可。

\par 2.导入表

\par (1)导入表命令语法

\par 使用上述select语句以及用mysqldump导出的纯文本数据,都可以使用如下“load data”方法进行导入,除此之外,还有mysqlimport也可以实现相同的效果,本文以load data语句为例。

\par 语法如下:

\noindent\rule{\textwidth}{0.4pt}


\begin{verbatim}LOAD DATA INFILE 'file\_name' INTO TABLE tbl\_name import option
\end{verbatim}


\noindent\rule{\textwidth}{0.4pt}


\par 其中,import options也包括和前文导入类似的选项,内容说明如表7-3所示。

\par 表7-3 import\_options内容说明

\begin{figure}[htbp]
\centering
\includegraphics[width=0.8\textwidth]{Image00056.jpg}
\end{figure}



\par (2)导入表数据实践

\par 准备数据,清空test表:

\noindent\rule{\textwidth}{0.4pt}


\begin{verbatim}mysql> delete from test;
Query OK, 9 rows affected (0.00 sec)
\end{verbatim}


\noindent\rule{\textwidth}{0.4pt}


\par 将前文oldboy\_test1.txt格式的数据导入数据库(即默认导出格式):

\noindent\rule{\textwidth}{0.4pt}


\begin{verbatim}mysql> system cat /tmp/oldboy\_test1.txt
1      full01
2      full02
3      full03
4      full04
5      full05
mysql> load data infile '/tmp/oldboy\_test1.txt' into table test;
Query OK, 5 rows affected (0.00 sec)
Records: 5  Deleted: 0  Skipped: 0  Warnings: 0
mysql> select * from test;
+----+--------+
| id | name   |
+----+--------+
|  1 | full01 |
|  2 | full02 |
|  3 | full03 |
|  4 | full04 |
|  5 | full05 |
+----+--------+
5 rows in set (0.00 sec)
\end{verbatim}


\noindent\rule{\textwidth}{0.4pt}


\par 将前文oldboy\_test3.txt格式的数据导入数据库(指定“-”分隔符导出的格式):

\noindent\rule{\textwidth}{0.4pt}


\begin{verbatim}mysql> system cat /tmp/oldboy\_test3.txt
1-full01
2-full02
3-full03
4-full04
5-full05
mysql> load data infile '/tmp/oldboy\_test3.txt' into table test fields terminated by '-';
Query OK, 5 rows affected (0.00 sec)
Records: 5  Deleted: 0  Skipped: 0  Warnings: 0
mysql> select * from test;
+----+--------+
| id | name   |
+----+--------+
|  1 | full01 |
|  2 | full02 |
|  3 | full03 |
|  4 | full04 |
|  5 | full05 |
+----+--------+
5 rows in set (0.00 sec)
mysql> load data infile '/tmp/oldboy\_test1.txt' into table test fields terminated by '-';
ERROR 1265 (01000): Data truncated for column 'id' at row 1
\#提示:数据格式不对就无法导入。
\end{verbatim}


\noindent\rule{\textwidth}{0.4pt}


\par 将前文oldboy\_test4.txt格式的数据导入数据库(使用双引号引用的数据):

\noindent\rule{\textwidth}{0.4pt}


\begin{verbatim}mysql> delete from test;
Query OK, 5 rows affected (0.00 sec)
mysql> system cat /tmp/oldboy\_test4.txt
"1"      "full01"
"2"      "full02"
"3"      "full03"
"4"      "full04"
"5"      "full05"
mysql> load data infile '/tmp/oldboy\_test4.txt' into table test fields enclosed by '"';
Query OK, 5 rows affected (0.00 sec)
Records: 5  Deleted: 0  Skipped: 0  Warnings: 0
mysql> select * from test;
+----+--------+
| id | name   |
+----+--------+
|  1 | full01 |
|  2 | full02 |
|  3 | full03 |
|  4 | full04 |
|  5 | full05 |
+----+--------+
5 rows in set (0.00 sec)
\end{verbatim}


\noindent\rule{\textwidth}{0.4pt}


\par 更多功能,读者可以自行测试。


% From text00105.html


\section{7.2 恢复数据库实践}


\subsection{
7.2.1 数据库恢复基本事项}
}


\par mysql命令以及source命令恢复数据库的原理就是在数据库里重新执行文件的SQL语句的过程。数据恢复和字符集的关联很大,如果字符集不正确则会导致恢复的数据乱码,有关字符集的知识详见后文。


% From text00106.html


\subsection{7.2.2 利用source命令恢复数据库}


\par 1.使用source命令恢复数据库的说明

\par 进入mysql数据库控制台后,切换到想恢复数据的数据库。

\noindent\rule{\textwidth}{0.4pt}


\begin{verbatim}mysql>use 数据库
\end{verbatim}


\noindent\rule{\textwidth}{0.4pt}


\par 接着,使用source命令进行恢复,后面接.sql文件,即上文使用mysqldump备份的文件或者人工编辑的SQL语句文件:

\noindent\rule{\textwidth}{0.4pt}


\begin{verbatim}mysql>source oldboy\_db.sql
\end{verbatim}


\noindent\rule{\textwidth}{0.4pt}


\par 这个oldboy\_db.sql文件是系统的相对路径,默认是登录MySQL前的系统路径,也可以使用完整的路径。

\par 2.恢复演示

\par 先执行一次正式的备份:

\noindent\rule{\textwidth}{0.4pt}


\begin{verbatim}[root@oldboy \textasciitilde{}]\# mysqldump -B --master-data=2 --single-transaction oldboy| gzip >/opt/oldboy.sql.gz
[root@oldboy \textasciitilde{}]\# ls -l /opt/oldboy.sql.gz
-rw-r--r--. 1 root root 867 Mar  3 08:34 /opt/oldboy.sql.gz
\end{verbatim}


\noindent\rule{\textwidth}{0.4pt}


\par 恢复前解压为SQL文件:

\noindent\rule{\textwidth}{0.4pt}


\begin{verbatim}[root@oldboy \textasciitilde{}]\# gzip -d /opt/oldboy.sql.gz
[root@oldboy \textasciitilde{}]\# ll /opt/
total 4
-rw-r--r--. 1 root root 2204 Mar  3 08:34 oldboy.sql
\end{verbatim}


\noindent\rule{\textwidth}{0.4pt}


\par 登录数据库删除oldboy数据库,然后准备用source恢复:

\noindent\rule{\textwidth}{0.4pt}


\begin{verbatim}mysql> drop database oldboy;           \#<==删除数据库oldboy。
Query OK, 1 row affected (0.03 sec)
mysql> select * from oldboy.test;      \#<==查询数据看是否还有?
ERROR 1146 (42S02): Table 'oldboy.test' doesn't exist \#<==报错了。
mysql> source /opt/oldboy.sql          \#<==执行恢复命令。
Query OK, 0 rows affected (0.00 sec)
..省略若干...
Query OK, 0 rows affected (0.00 sec)
mysql> select * from oldboy.test;      \#<==查询数据看是否还有?显然有了。
+----+---------+
| id | name    |
+----+---------+
|  1 | oldboy  |
|  2 | oldgirl |
|  3 | inca    |
|  4 | zuma    |
|  5 | kaka    |
+----+---------+
5 rows in set (0.00 sec)
\end{verbatim}


\noindent\rule{\textwidth}{0.4pt}


\par 如果是utf8数据库,人工编辑的SQL文件,则建议使用“UTF8没有签名”格式。


% From text00107.html


\subsection{7.2.3 利用mysql命令恢复(标准)}


\par 1.使用mysql命令恢复基本实践

\par mysql命令是MySQL数据库自带的重要命令之一,除了日常登录数据库之外,还可以通过mysqldump备份的文件或者人工编辑的SQL语句文件对数据库进行数据恢复:

\noindent\rule{\textwidth}{0.4pt}


\begin{verbatim}mysql> drop database oldboy;
Query OK, 1 row affected (0.01 sec)
mysql> quit
Bye
[root@oldboy \textasciitilde{}]\# mysql </opt/oldboy.sql          \#<==这就是恢复命令,简单吧。
[root@oldboy \textasciitilde{}]\# mysql -e "select * from oldboy.test;"  \#<==使用-e参数,可在命令行查MySQL数据。
+----+---------+
| id | name    |
+----+---------+
|  1 | oldboy  |
|  2 | oldgirl |
|  3 | inca    |
|  4 | zuma    |
|  5 | kaka    |
+----+---------+
\end{verbatim}


\noindent\rule{\textwidth}{0.4pt}


\par 2.使用开发人员提交的SQL语句恢复文件

\par 假定开发人员让运维人员或DBA插入数据到数据库(可能是通过邮件发送的,内容可能是字符串或者SQL文件)。

\par 此时的SQL文件里很可能没有use db这样的字样,此时如果使用mysql命令导入就要指定数据库名了,即:

\noindent\rule{\textwidth}{0.4pt}


\begin{verbatim}[root@oldboy \textasciitilde{}]\# mysql oldboy </opt/oldboy.sql
\end{verbatim}


\noindent\rule{\textwidth}{0.4pt}


\par 指定库名oldboy的作用就相当于是在数据库里执行use oldboy,因此如果使用mysqldump备份时不使用-B参数,那么在恢复时不但可能会提示没有数据库,还可能会提示没有选择数据库。

\par 并且在SQL语句文件里尽可能地加入字符集设置,以防止乱码。假设开发人员提交了N行插入语句,则需要插入到数据库(这里以一行来代替):

\noindent\rule{\textwidth}{0.4pt}


\begin{verbatim}insert into test(name) values('小陶');
\end{verbatim}


\noindent\rule{\textwidth}{0.4pt}


\par 放入文件里执行(也可以直接在数据库命令行执行):

\noindent\rule{\textwidth}{0.4pt}


\begin{verbatim}[root@oldboy \textasciitilde{}]\# cat /opt/insert.sql
set names utf8;
insert into test(name) values('小陶');
\end{verbatim}


\noindent\rule{\textwidth}{0.4pt}


\par 如果是utf8数据库,人工编辑的SQL文件,建议使用“UTF8没有签名”的格式。

\par 此时恢复数据到数据库,命令如下:

\noindent\rule{\textwidth}{0.4pt}


\begin{verbatim}[root@oldboy \textasciitilde{}]\# mysql </opt/insert.sql
ERROR 1046 (3D000) at line 2: No database selected  \#<==提示没有数据库可选。
[root@oldboy \textasciitilde{}]\# mysql oldboy</opt/insert.sql       \#<==这里oldboy的作用就相当于是在库里执行use oldboy一样。
[root@oldboy \textasciitilde{}]\# mysql -e "select * from oldboy.test;"
+----+---------+
| id | name    |
+----+---------+
|  1 | oldboy  |
|  2 | oldgirl |
|  3 | inca    |
|  4 | zuma    |
|  5 | kaka    |
|  6 | 小陶     | \#<==数据已经正常插入。
+----+---------+
\end{verbatim}


\noindent\rule{\textwidth}{0.4pt}


\par 看到了吧,mysql不仅可以恢复mysqldump的备份文件,只要是sql语句,都可以通过mysql命令执行到数据库中。

\par 如果在使用mysqldump备份时指定了-B参数,那么恢复时就无须指定库名恢复了,为什么?

\par 因为使用-B参数以后,备份结果中会带use oldboy和“create database oldboy”;语句,而恢复时在mysql命令后指定库名就相当于是执行use oldboy。

\par 所以,如果mysqldump备份时指定了-B参数,则恢复时可以用如下方法:

\noindent\rule{\textwidth}{0.4pt}


\begin{verbatim}mysql </opt/oldboy.sql
\end{verbatim}


\noindent\rule{\textwidth}{0.4pt}


\par 否则只能用如下方法,而且条件是oldboy库必须事先存在,否则需要提前创建:

\noindent\rule{\textwidth}{0.4pt}


\begin{verbatim}mysql oldboy</opt/insert.sql  \#<==条件是oldboy库必须事先存在,否则需要提前创建。
\end{verbatim}


\noindent\rule{\textwidth}{0.4pt}


\par 此处的oldboy库相当于use oldboy。

\par 因此建议使用mysqldump备份库时指定-B参数,效果会更好。

\par 3.针对压缩的备份数据进行恢复

\par 方法1:使用gzip解压(会删除压缩文件)。

\noindent\rule{\textwidth}{0.4pt}


\begin{verbatim}gzip -d /opt/oldboy.sql.gz
mysql </opt/oldboy.sql
\end{verbatim}


\noindent\rule{\textwidth}{0.4pt}


\par 方法2:使用gzip解压(不会删除压缩文件)。

\noindent\rule{\textwidth}{0.4pt}


\begin{verbatim}[root@oldboy \textasciitilde{}]\# !mysqldump          \#<==调用最近的mysqldump命令,重复执行备份。
mysqldump -B --master-data=2 --single-transaction oldboy|gzip >/opt/oldboy.sql.gz
[root@oldboy \textasciitilde{}]\# gzip -cd /opt/oldboy.sql.gz >/opt/oldboy1.sql \#<==特殊解压方法。
[root@oldboy \textasciitilde{}]\# mysql </opt/oldboy1.sql
\end{verbatim}


\noindent\rule{\textwidth}{0.4pt}


\par 方法3:使用gunzip解压(不会删除压缩文件)。

\noindent\rule{\textwidth}{0.4pt}


\begin{verbatim}[root@oldboy \textasciitilde{}]\# gunzip -cd /opt/oldboy.sql.gz >/opt/oldboy.sql
[root@oldboy \textasciitilde{}]\# mysql </opt/oldboy2.sql
\end{verbatim}


\noindent\rule{\textwidth}{0.4pt}


\par 或者:

\noindent\rule{\textwidth}{0.4pt}


\begin{verbatim}[root@oldboy \textasciitilde{}]\# gunzip</opt/oldboy.sql.gz|mysql
\end{verbatim}


\noindent\rule{\textwidth}{0.4pt}


\par 方法4:使用zcat读取压缩包数据。

\noindent\rule{\textwidth}{0.4pt}


\begin{verbatim}[root@oldboy \textasciitilde{}]\# zcat /opt/oldboy.sql.gz >/opt/oldboy3.sql
[root@oldboy \textasciitilde{}]\# mysql </opt/oldboy3.sql
\end{verbatim}


\noindent\rule{\textwidth}{0.4pt}



% From text00108.html


\subsection{7.2.4 利用mysql-e参数查看mysql数据}


\par mysql命令提供了一个功能,可以让使用者无须登录数据库,在Linux命令行就可以执行SQL语句。

\par 1.查看数据库oldboy库test表数据

\par 在Linux命令行可通过如下命令实现上述功能:

\noindent\rule{\textwidth}{0.4pt}


\begin{verbatim}[root@oldboy \textasciitilde{}]\# mysql -e "use oldboy;select * from test;"
+----+---------+
| id | name    |
+----+---------+
|  1 | oldboy  |
|  2 | oldgirl |
|  3 | inca    |
|  4 | zuma    |
|  5 | kaka    |
|  6 | 小陶    |
+----+---------+
\end{verbatim}


\noindent\rule{\textwidth}{0.4pt}


\par 2.利用mysql-e参数查看SQL线程执行状态

\par 在Linux命令行可通过如下命令查看SQL线程执行状态:

\noindent\rule{\textwidth}{0.4pt}


\begin{verbatim}[root@oldboy \textasciitilde{}]\# mysql -e "show processlist;"
+----+------+-----------+------+---------+------+-------+------------------+
| Id | User | Host      | db   | Command | Time | State | Info             |
+----+------+-----------+------+---------+------+-------+------------------+
| 21 | root | localhost | NULL | Query   |    0 | init  | show processlist |
+----+------+-----------+------+---------+------+-------+------------------+
\end{verbatim}


\noindent\rule{\textwidth}{0.4pt}


\par 以下命令用于查看完整的线程状态,此命令在查看大量慢查询语句时非常有用:

\noindent\rule{\textwidth}{0.4pt}


\begin{verbatim}[root@oldboy \textasciitilde{}]\# mysql -e "show full processlist;"
+----+------+-----------+------+---------+------+-------+-----------------------+
| Id | User | Host      | db   | Command | Time | State | Info                  |
+----+------+-----------+------+---------+------+-------+-----------------------+
| 22 | root | localhost | NULL | Query   |    0 | init  | show full processlist |
+----+------+-----------+------+---------+------+-------+-----------------------+
\end{verbatim}


\noindent\rule{\textwidth}{0.4pt}


\par 来看一个企业故障案例,故障原因是mysql系统的sleep线程过多,有大量的慢查询语句,导致数据库无法接受正常的请求。

\noindent\rule{\textwidth}{0.4pt}


\begin{verbatim}mysql>show full processlist;
\end{verbatim}


\noindent\rule{\textwidth}{0.4pt}


\par 结果分别如图7-5和图7-6所示。

\begin{figure}[htbp]
\centering
\includegraphics[width=0.8\textwidth]{Image00057.jpg}
\end{figure}



\par 图7-5 数据库里正在执行的SQL语句的列属性

\begin{figure}[htbp]
\centering
\includegraphics[width=0.8\textwidth]{Image00058.jpg}
\end{figure}



\par 图7-6 数据库里正在执行的SQL语句的信息

\par 图7-5和图7-6实际上是一张图,为了显示的清晰分为两张图片展示给读者。

\noindent\rule{\textwidth}{0.4pt}


\begin{verbatim}mysql> kill 212555221;
\end{verbatim}


\noindent\rule{\textwidth}{0.4pt}


\par 在“mysql>kill 212555221;”语句中,84是ID,若使用kill(insert,update)命令则可能会丢失数据。

\par 
解决办法


\par 先调整MySQL的如下两个超时参数配置:

\noindent\rule{\textwidth}{0.4pt}


\begin{verbatim}mysql> show variables like '\%\_timeout\%';
+----------------------------+----------+
| Variable\_name              | Value    |
+----------------------------+----------+
| connect\_timeout            | 10       |
| interactive\_timeout        | 28800    |
| wait\_timeout               | 28800    |
+----------------------------+----------+
10 rows in set (0.00 sec)
set global wait\_timeout = 60;
set global interactive\_timeout = 60;
\end{verbatim}


\noindent\rule{\textwidth}{0.4pt}


\par 然后在配置文件里修改:

\noindent\rule{\textwidth}{0.4pt}


\begin{verbatim}[mysqld]
interactive\_timeout = 120  \#<==此参数设置后wait\_timeout自动生效。
wait\_timeout = 120
\end{verbatim}


\noindent\rule{\textwidth}{0.4pt}


\par 其他补充方法具体如下。

\par 1)在PHP程序中,不使用持久链接,即使用mysql\_connect而不是pconnect。

\par 2)PHP程序执行完毕,应该显式调用mysql\_close。

\par 3)Java程序调整连接池(C3P0)或者调整JAVA服务(Tomcat有关连接池参数)。

\par 4)逐步分析MySQL的SQL查询及慢查询日志,找到查询过慢的SQL,优化之。

\par 3.利用mysql-e参数查看mysql变量及性能状态

\par 通过以下命令查看mysql的所有参数配置:

\noindent\rule{\textwidth}{0.4pt}


\begin{verbatim}[root@oldboy \textasciitilde{}]\# mysql -e "show variables;"
Variable\_name      Value
auto\_increment\_increment      1
auto\_increment\_offset      1
autocommit      ON
automatic\_sp\_privileges      ON
...省略上百行...
\end{verbatim}


\noindent\rule{\textwidth}{0.4pt}


\par 可通过以下命令查看my.cnf配置文件的配置有没有在数据库中生效:

\noindent\rule{\textwidth}{0.4pt}


\begin{verbatim}[root@oldboy \textasciitilde{}]\# mysql -e "show variables like 'log\_bin';"
+---------------+-------+
| Variable\_name | Value |
+---------------+-------+
| log\_bin       | ON    |
+---------------+-------+
\end{verbatim}


\noindent\rule{\textwidth}{0.4pt}


\par 查看mysql数据库运行状态的命令如下:

\noindent\rule{\textwidth}{0.4pt}


\begin{verbatim}[root@oldboy \textasciitilde{}]\# mysql -e "show global status;"|head -5
Variable\_name            Value
Aborted\_clients            0
Aborted\_connects            0
Binlog\_cache\_disk\_use      0
Binlog\_cache\_use            7
\end{verbatim}


\noindent\rule{\textwidth}{0.4pt}


\par 高级管理员可以对这个状态信息和配置文件参数进行比较,从而找出优化数据库配置的方法,案例可以参考老男孩学员的文章,地址为:\href{http://www.xuliangwei.com/xubusi/213.html}{http://www.xuliangwei.com/xubusi/213.html}
 。

\par 4.利用mysql-e参数不重启数据库修改数据库参数

\par 不重启数据库修改数据库参数还能生效的示例如下:

\noindent\rule{\textwidth}{0.4pt}


\begin{verbatim}[root@oldboy \textasciitilde{}]\# mysql -e "show variables;"|grep key\_buffer
key\_buffer\_size      8388608
[root@oldboy \textasciitilde{}]\# mysql -e "set global key\_buffer\_size = 1024*1024*16;"
[root@oldboy \textasciitilde{}]\# mysql -e "show variables;"|grep key\_buffer
key\_buffer\_size      16777216
\end{verbatim}


\noindent\rule{\textwidth}{0.4pt}


\par 从上面的示例可以得知,实现不重启数据库更改数据库参数的命令为:

\noindent\rule{\textwidth}{0.4pt}


\begin{verbatim}set global key\_buffer\_size = 1024*1024*16;  \#<==及时生效,重启mysql失效。
\end{verbatim}


\noindent\rule{\textwidth}{0.4pt}


\par 在该过程中,配置文件也要增加或修改,编辑/etc/my.cnf,修改key\_buffer\_size=16M。

\par 5.利用mysql-e参数引出的重要命令

\par 下面将给出利用mysql-e参数引出的重要命令,这些命令虽然没有详细讲解,但是,在工作中还是很重要的,请读者牢记:

\noindent\rule{\textwidth}{0.4pt}


\begin{verbatim}show processlist;    \#<==查看数据库里正在执行的SQL语句,可能无法看全完整的SQL语句。
show full processlist  \#<==查看正在执行的完整SQL语句,完整显示。
set global key\_buffer\_size = 1024*1024*16 \#<==不重启数据库调整数据库参数,直接生效,
                                        重启后失效。
show variables;        \#<==查看数据库的配置参数信息,例如,my.cnf里参数的生效情况。
show variables like '\%log\_bin\%';"
kill ID                \#<==杀掉SQL线程的命令,ID为线程号。
show session status;   \#<==查看当前会话的数据库状态信息。
show global status;    \#<==查看整个数据库运行的状态信息,很重要,要分析并要做好监控。
show engine innodb status; \#<==显示innodb引擎的性能状态(早期版本show innodb status)。
\end{verbatim}


\noindent\rule{\textwidth}{0.4pt}


\par 6.mysqladmin命令常用参数

\par mysqladmin命令也有很多查看数据库信息的命令,具体如下:

\noindent\rule{\textwidth}{0.4pt}


\begin{verbatim}mysqladmin的相关命令:
mysqladmin password oldboy123                  \#<==设置密码,前文用过的。
mysqladmin -uroot -poldboy123 password oldboy  \#<==修改密码,前文用过的。
mysqladmin -uroot -poldboy123 status           \#<==查看状态,相当于show status。
mysqladmin -uroot -poldboy123 -i 1 status      \#<==每秒查看一次状态。
mysqladmin -uroot -poldboy123 extended-status  \#<==等同于“show global status;”。
mysqladmin -uroot -poldboy123 flush-logs       \#<==切割日志。
mysqladmin -uroot -poldboy123 processlist      \#<==查看执行的SQL语句信息。
mysqladmin -uroot -poldboy123 processlist -i 1 \#<==每秒查看一次执行的SQL语句。
mysqladmin -uroot -p'oldboy123' shutdown       \#<==关闭mysql服务,前文用过的。
mysqladmin -uroot -p'oldboy123' variables      \#<==相当于show variables。
\end{verbatim}


\noindent\rule{\textwidth}{0.4pt}


\par 7.mysql命令常用参数

\par 表7-4针对mysql命令的常用基础参数进行了说明。

\par 表7-4 mysql命令常用参数总结

\begin{figure}[htbp]
\centering
\includegraphics[width=0.8\textwidth]{Image00059.jpg}
\end{figure}



\par 更多命令信息,可执行“mysql-help”查看。


% From text00109.html


\section{7.3 mysqlbinlog增量恢复工具}


\par mysqlbinlog工具的作用是解析mysql的二进制binlog的日志内容,把二进制日志解析成可以在MySQL数据库里执行的SQL语句。


% From text00110.html


\subsection{7.3.1 mysql的binlog日志是什么}


\par mysql数据目录下的如下文件就是mysql的binlog日志:

\noindent\rule{\textwidth}{0.4pt}


\begin{verbatim}oldboy-bin.000001
oldboy-bin.000002
oldboy-bin.000003
……
\end{verbatim}


\noindent\rule{\textwidth}{0.4pt}


\par 
\begin{figure}[htbp]
\centering
\includegraphics[width=0.8\textwidth]{Image00006.jpg}
\end{figure}

 提示:
 必须要打开log\_bin功能才能生成上述文件。


% From text00111.html


\subsection{7.3.2 mysql的binlog日志的作用}


\par mysql的binlog日志用于记录MySQL内部的增删改等操作,也就是对MySQL数据库更新内容的记录(对数据库的改动),对数据库进行查询的语句(如以show、select开头的语句),不会被binlog日志记录。binlog日志的主要作用是数据库的主从复制以及数据灾难后的增量恢复。


% From text00112.html


\subsection{7.3.3 mysql的binlog日志功能如何开启}


\par 在mysql的配置文件my.cnf中,增加log\_bin参数即可开启binlog日志,也可以通过赋值来指定binlog日志的文件名,示例如下:

\noindent\rule{\textwidth}{0.4pt}


\begin{verbatim}[root@oldboy \textasciitilde{}]\# grep log\_bin /etc/my.cnf
log\_bin
\end{verbatim}


\noindent\rule{\textwidth}{0.4pt}


\par 
\begin{figure}[htbp]
\centering
\includegraphics[width=0.8\textwidth]{Image00006.jpg}
\end{figure}

 提示:
 也可以按照“log\_bin=/application/mysql/logs/oldboy-bin”命名,目录要存在。


% From text00113.html


\subsection{7.3.4 mysqlbinlog工具解析binlog日志实践}


\par 默认情况下,binlog日志是二进制格式的,不能使用查看文本工具的命令(比如,cat、vi等)查看:

\noindent\rule{\textwidth}{0.4pt}


\begin{verbatim}[root@oldboy \textasciitilde{}]\# cd /application/mysql/data/
[root@oldboy data]\# ls -l oldboy-bin*
-rw-rw----. 1 mysql mysql  143 Mar  3 05:50 oldboy-bin.000001
-rw-rw----. 1 mysql mysql  168 Mar  3 05:57 oldboy-bin.000002
-rw-rw----. 1 mysql mysql  168 Mar  3 05:57 oldboy-bin.000003
-rw-rw----. 1 mysql mysql 6967 Mar  3 09:44 oldboy-bin.000004
\#<==找一个内容大的文件做测试。
-rw-rw----. 1 mysql mysql   80 Mar  3 05:57 oldboy-bin.index
[root@oldboy data]\# file oldboy-bin.000004
oldboy-bin.000004: MySQL replication log        \#<==复制log文件。
[root@oldboy data]\# tail -2 oldboy-bin.000004   \#<==使用tail查看内容,很混乱,
                                                确定其是非纯文本文件。
) ENGINE=InnoDB AUTO\_INCREMENT=7 DEFAULT CHARSET=utf8TNF                                                       oldboyoldboy/*!40000 ALTER TABLE `test` DISABLE KEYS */\#ЪF    std!!!
          oldboyoldboyBEGIN8A
           oldboyoldboyINSERT INTO `test` VALUES (1,'oldboy'),(2,'oldgirl'), (3,'inca'),(4,'zuma'),(5,'kaka'),(6,'小陶')\textasciitilde{}F¤筙x7
          td!!!
               oldboyoldboy/*!40000 ALTER TABLE `test` ENABLE KEYS */z4
\end{verbatim}


\noindent\rule{\textwidth}{0.4pt}


\par 1.解析指定库的binlog日志

\par 利用“mysqlbinlog-d”参数解析指定库的binlog日志:

\noindent\rule{\textwidth}{0.4pt}


\begin{verbatim}[root@oldboy data]\# mysqlbinlog -d oldboy oldboy-bin.000004 -r bin.sql
\#<==-d指定库,-r指定生成的文件。
[root@oldboy data]\# grep -i insert bin.sql \#<==过滤内容,看到了曾经恢复的SQL语句。
/*!40019 SET @@session.max\_insert\_delayed\_threads=0*/;
INSERT INTO `test` VALUES (1,'oldboy'),(2,'oldgirl'),(3,'inca'),(4,'zuma'), (5,'kaka')
INSERT INTO `test` VALUES (1,'oldboy'),(2,'oldgirl'),(3,'inca'),(4,'zuma'), (5,'kaka')
SET INSERT\_ID=6/*!*/;
insert into test(name) values('小陶')
INSERT INTO `test` VALUES (1,'oldboy'),(2,'oldgirl'),(3,'inca'),(4,'zuma'), (5,'kaka'),(6,'小陶')
INSERT INTO `test` VALUES (1,'oldboy'),(2,'oldgirl'),(3,'inca'),(4,'zuma'), (5,'kaka'),(6,'小陶')
INSERT INTO `test` VALUES (1,'oldboy'),(2,'oldgirl'),(3,'inca'),(4,'zuma'), (5,'kaka'),(6,'小陶')
INSERT INTO `test` VALUES (1,'oldboy'),(2,'oldgirl'),(3,'inca'),(4,'zuma'), (5,'kaka'),(6,'小陶')
\end{verbatim}


\noindent\rule{\textwidth}{0.4pt}


\par 结论:mysqlbinlog可以指定-d实现分库导出binlog,如果使用-d参数,那么在更新数据时,必须要有use库名,才能分出指定库的binlog,例如,写入数据库的语句必须采用如下写法:

\noindent\rule{\textwidth}{0.4pt}


\begin{verbatim}use oldboy;
insert into test values(1,'oldboy')
\end{verbatim}


\noindent\rule{\textwidth}{0.4pt}


\par 下面的这种写法就不行:

\noindent\rule{\textwidth}{0.4pt}


\begin{verbatim}insert into oldboy.test values(2,'oldgirl')
\end{verbatim}


\noindent\rule{\textwidth}{0.4pt}


\par 2.按照位置截取binlog内容

\par 按照位置截取binlog内容的优点是精确,但是要花费时间寻找位置,例如,要截取oldboy-bin.000009文件从位置365到位置456的日志,命令如下。

\noindent\rule{\textwidth}{0.4pt}


\begin{verbatim}mysqlbinlog oldboy-bin.000009 --start-position=365 --stop-position=456 -r pos.sql
\end{verbatim}


\noindent\rule{\textwidth}{0.4pt}


\par 
\begin{figure}[htbp]
\centering
\includegraphics[width=0.8\textwidth]{Image00006.jpg}
\end{figure}

 提示:
 开始位置点必须存在于binlog里,结尾位置点可以不存在。

\par 若指定了开始位置,而不指定结束位置,那么请问结束位置是?可通过如下命令查看:

\noindent\rule{\textwidth}{0.4pt}


\begin{verbatim}mysqlbinlog oldboy-bin.000009 --start-position=365 -r pos.sql
\end{verbatim}


\noindent\rule{\textwidth}{0.4pt}


\par 若指定了结束位置,而不指定开始位置,那么请问开始位置是?可通过如下命令查看:

\noindent\rule{\textwidth}{0.4pt}


\begin{verbatim}mysqlbinlog oldboy-bin.000009 --stop-position=456 -r pos.sql
\end{verbatim}


\noindent\rule{\textwidth}{0.4pt}


\par 所谓的位置点,就是mysqlbinlog解析文件里的不同行行首的“\#at数字”标识的数据。

\par 3.按照时间截取binlog内容

\par 按照时间截取binlog内容的缺点是模糊、不准确,截取的内容会丢失部分数据,精确到秒,1秒也可能会有多条语句。

\par 下面语句标识截取的是oldboy-bin.000009文件中从‘2014-10-1617:14:15’时间到‘2014-10-1617:15:15’时间的数据:

\noindent\rule{\textwidth}{0.4pt}


\begin{verbatim}mysqlbinlog oldboy-bin.000009 --start-datetime='2014-10-16 17:14:15' --stop -datetime='2014-10-16 17:15:15' -r time.sql
\end{verbatim}


\noindent\rule{\textwidth}{0.4pt}


\par 若指定了开始时间,而不指定结束时间,那么请问结束时间是?可通过如下命令查看:

\noindent\rule{\textwidth}{0.4pt}


\begin{verbatim}mysqlbinlog oldboy-bin.000009 --start-datetime='2014-10-16 17:14:15'  -r time.sql
\end{verbatim}


\noindent\rule{\textwidth}{0.4pt}


\par 若指定了结束时间,而不指定开始时间,那么请问开始时间是?可通过如下命令查看:

\noindent\rule{\textwidth}{0.4pt}


\begin{verbatim}mysqlbinlog oldboy-bin.000009  --stop-datetime='2014-10-16 17:15:15' -r time.sql
提示:所谓的时间点就是mysqlbinlog解析文件里的不同行行首的“\#170303  9:44:22”标识的数据。
\end{verbatim}


\noindent\rule{\textwidth}{0.4pt}


\par 后文的增量恢复案例中会有mysqlbinlog命令的实战应用。


% From text00114.html


\subsection{7.3.5 mysqlbinlog命令常用参数}


\par 表7-5为mysqlbinlog命令常用参数说明。

\par 表7-5 mysqlbinlog命令常用参数说明

\begin{figure}[htbp]
\centering
\includegraphics[width=0.8\textwidth]{Image00060.jpg}
\end{figure}




% From text00115.html


\section{7.4 本章重点}


\par 1)备份的意义?

\par 2)mysqldump备份的原理?

\par 3)mysqldump常用备份参数说明及实践。

\par 4)mysql恢复命令source实践。

\par 5)mysql恢复命令mysql常用参数说明及实践。

\par 6)利用select导出和导入数据参数说明及实践。

\par 7)mysql的连接及状态常见管理命令说明。

\par 8)mysqladmin常用命令说明。

\par 9)mysqlbinlog常用参数说明及实践。


% From text00116.html


\chapter{第8章 MySQL企业级备份应用知识与实践}


\section{
8.1 数据库备份的最高层次思想}
}


\par 数据库备份的最高层次,就是永远都用不上备份。

\par ——老男孩

\par 这就像我们日常购买大病保险一样,任何人购买大病保险都肯定不是希望得大病,我们做数据库备份也是一样,备份策略无论做得多么完备,我们还是不希望故障发生。因此,除了具备高超的备份策略和精湛的恢复能力之外,还要在未雨绸缪上多下功夫以达到防患于未然的目的。


% From text00117.html


\section{8.2 数据库管理员的两大工作核心}


\par 1.能够让数据安全得到保护

\par 所谓的数据安全,最容易被人误以为是只有数据丢失,其实还包含数据被脱库、泄密等方面,本章主要讲解对数据的备份和恢复,以防止企业数据丢失,后面的文章还会讲解更高级的安全防范措施。

\par 2.能7×24小时提供服务

\par 数据库具备7×24小时提供服务的能力,是数据库管理员的重要职责,后文会为大家讲解MySQL各种优化策略、主从复制集群、MySQL MHA+Keepalived集群、读写分离等综合保障数据库具备7×24的服务能力的措施。


% From text00118.html


\section{8.3 全量备份与增量备份}


\subsection{
8.3.1 全量备份的概念}
}


\par 全量数据就是数据库中所有的数据(或某一个库的全部数据);全量备份就是把数据库中所有的数据进行备份。

\par 下面就以InnoDB引擎数据库为例进行讲解。

\par 备份数据库中所有库的所有数据的命令为:

\noindent\rule{\textwidth}{0.4pt}


\begin{verbatim}mysqldump -B --master-data=2 --single-transaction -A |gzip >/opt/all.sql.gz
\end{verbatim}


\noindent\rule{\textwidth}{0.4pt}


\par 备份oldboy一个库中所有数据的命令为:

\noindent\rule{\textwidth}{0.4pt}


\begin{verbatim}mysqldump -B --master-data=2 --single-transaction oldboy|gzip >/opt/oldboy.sql.gz
\end{verbatim}


\noindent\rule{\textwidth}{0.4pt}



% From text00119.html


\subsection{8.3.2 增量备份的概念}


\par 增量数据就是指上一次全量备份数据之后到下一次全量备份之前数据库所更新的数据。在使用mysqldump命令做全备时,增量数据就是MySQL的binlog日志,因此,对binlog日志的备份在此处就可以称为增量备份,当然,有些工具本身就可以实现全量以及增量数据备份,例如Xtrabackup工具,此知识点会在后文的第9章讲解。


% From text00120.html


\subsection{8.3.3 全量与增量如何结合备份}


\par 下面以mysqldump命令和binlog日志增量数据为例讲解企业中按天全备和按周全备的方法,具体见表8-1和表8-2。

\par 表8-1 按天全备与增量备份数据示例

\begin{figure}[htbp]
\centering
\includegraphics[width=0.8\textwidth]{Image00061.jpg}
\end{figure}



\par 按天全备的特点如下。

\par ·优点:恢复数据时需要的数据文件数量少,恢复时间短,维护成本低。

\par ·缺点:每天一个全备,占用空间多,占用系统资源多,经常备份会影响用户体验。

\par 中小企业用得最多的策略就是按天全备,然后根据空间情况保留全备份数,例如仅保留7天内的备份数据,如果企业数据很重要,则可以使用磁带机等设备留存1年以上的备份数据。

\par binlog增量的清理可以通过在my.cnf中配置“过期清理天数”的相关参数(expire\_logs\_days=7)来实现,例如保留7天内的binlog日志,理论上如果每天进行全备,那么binlog只要保留1天的就够了。

\par 表8-2 按周全备与增量备份数据示例

\begin{figure}[htbp]
\centering
\includegraphics[width=0.8\textwidth]{Image00062.jpg}
\end{figure}



\par 按周全备的特点如下。

\par ·优点:每周仅有一个完整备份,因此占用磁盘总空间小,占用系统资源少,备份次数少,用户体验好一些。

\par ·缺点:恢复时数据文件多,导致恢复麻烦,维护成本高,恢复时间长。

\par 大型企业由于数据量特别大,每天全备时间太长,因此有可能会采用周备的策略,这样不仅有利于节省数据存储空间而且不会影响用户访问数据库的体验。


% From text00121.html


\section{8.4 MySQL常用的备份方式}


\par MySQL备份的常用方式有逻辑备份和物理备份。


% From text00122.html


\subsection{8.4.1 逻辑备份方式}


\par 1.逻辑备份

\par MySQL的逻辑备份其实就是使用MySQL自带的mysqldump命令或其他相关工具,把MySQL数据以SQL语句的形式导出或备份成文件。在恢复的时候则通过执行mysql恢复命令(或source等)将存储的SQL语句文件数据还原到MySQL数据库中。

\par 实现逻辑备份的常用工具为MySQL自带的mysqldump命令,备份所有库的命令为:

\noindent\rule{\textwidth}{0.4pt}


\begin{verbatim}mysqldump -A -B --master-data=2 --single-transaction |gzip >/opt/all.sql.gz
\end{verbatim}


\noindent\rule{\textwidth}{0.4pt}


\par 恢复数据库的方法之一为:

\noindent\rule{\textwidth}{0.4pt}


\begin{verbatim}zcat opt/all.sql.gz|mysql
\end{verbatim}


\noindent\rule{\textwidth}{0.4pt}


\par 使用此种逻辑备份方式进行全量备份后的增量数据就是数据库记录的binlog日志文件,那么,如何增量恢复binlog日志呢?mysqlbinlog工具可以把binlog日志转换成SQL语句,然后通过mysql恢复命令(或source等)将SQL语句还原到MySQL数据库中。

\par 恢复增量数据的命令方法为:

\noindent\rule{\textwidth}{0.4pt}


\begin{verbatim}mysqlbinlog oldboy-bin.000008 oldboy-bin.000009 >bin.sql
\#<==将binlog文件解析为SQL语句。
mysql <bin.sql                 \#<==恢复到数据库。
\end{verbatim}


\noindent\rule{\textwidth}{0.4pt}


\par 2.逻辑备份的特点

\par 逻辑备份的优点为操作简单、方便、可靠,并且备份的数据可以跨平台、跨版本、甚至跨软件、跨操作系统,还可以实现分库分表备份;逻辑备份也有一定的缺点,例如,备份速度比物理备份慢、恢复的效率也不是特别高等。

\par 3.逻辑备份的常用工具

\par mysqldump是MySQL官方自带的最常用的逻辑备份工具,还能实现分表分库备份,也是本章重点讲解的备份工具。除此之外,还有一个mydumper工具,它是一个在GPL许可下发布的高性能MySQL备份和恢复工具集,有兴趣的读者也可以研究下。

\par 4.逻辑备份的企业应用场景

\par 适用于数据量不是特别大的场景,例如,老男孩所给的参考值为打包前不大于30GB的数据库数据,30GB的值主要是考虑备份效率的问题,以及管理员使用复杂度的平衡。

\par 不过,在跨版本、跨软件升级或迁移数据的时候,此时物理备份一般就不能使用了。


% From text00123.html


\subsection{8.4.2 物理备份方式}


\par 1.物理备份

\par (1)冷备方法

\par MySQL的物理备份方法之一是使用cp、rsync、tar、scp等复制工具把MySQL数据文件复制成多份,由于在备份期间数据仍然有写入操作,所以,直接复制的备份方式会引起数据丢失。另外在恢复数据库时,对新数据库的路径、配置也有要求,一般要和原库的配置保持一致(版本、路径、配置尽可能一样)。

\par 为了确保备份期间数据的一致性,可以选择人工停库或者锁库后再进行物理复制,而这在生产环境中一般是不允许的,除非是可以申请停机或锁表时间,所以使用传统Linux命令复制工具还是比较粗的冷备份方式,应避免使用,后文还会提及根据MySQL主从复制利用从库进行冷备份的策略。

\par 一般在进行大规模数据库迁移时,先停库,然后物理迁移,这样做是很有效率的方案。

\par (2)热备方法

\par 除了在Linux命令行通过命令直接复制MySQL数据文件之外,还有一些其他的第三方的开源或商业物理热备份工具,如Xtrabackup。使用这个工具可以实现物理全备及增量备份。

\par 2.物理备份的特点

\par 物理备份的优缺点正好与逻辑备份相反,因此在企业里应根据需求,互补使用。

\par ·优点:速度快,效率高。

\par ·缺点:不容易跨平台、跨版本、跨软件、跨操作系统,可以实现分库分表备份,但恢复时会麻烦很多,软件的使用也比较复杂一些。

\par 3.物理备份的常用工具或方法

\par Linux下冷备份工具为cp、tar,备份时需要锁表或者停库以确保数据的一致性;开源的热备份(基于InnoDB)工具则是Xtrabackup。

\par 4.物理备份的企业应用场景

\par ·数据库总数据量超过30GB的,可使用Xtrabackup热备工具进行备份,以提升效率。

\par ·可以选择在数据库的从库上进行备份,备份时停止SQL线程应用数据到数据库,然后通过cp或tar打包备份,这也是一种不错的冷备方案,不会影响数据库的服务。


% From text00124.html


\subsection{8.4.3 物理备份与逻辑备份的区别}


\par 表8-3给出了逻辑备份与物理备份在各方面的对比。

\par 表8-3 物理备份与逻辑备份的对比

\begin{figure}[htbp]
\centering
\includegraphics[width=0.8\textwidth]{Image00063.jpg}
\end{figure}



\par 那么实际工作中使用的到底是逻辑备份还是物理备份呢?生产场景下又是如何具体应用的呢?且看下文详细讲解。


% From text00125.html


\section{8.5 逻辑备份的企业级应用实战}


\subsection{
8.5.1 中小企业的MySQL备份实战}
}


\par 1.中小企业全备备份策略与实践

\par 中小企业一般会采用逻辑备份,常用的工具就是前文提到的mysqldump命令,备份的策略一般是每日进行全量备份,备份会选择在数据库业务流量低谷时执行,备份时可以锁表或者采用事务方式备份,其中一个简单的实战备份脚本案例为:

\noindent\rule{\textwidth}{0.4pt}


\begin{verbatim}[root@oldboy scripts]\# cat bak.sh
\#!/bin/bash
\#Author: oldboy
export PATH=/application/mysql/bin:/usr/local/bin:/sbin:/bin:/usr/sbin:/usr/bin
bak\_path=/server/bakup
[ ! -d \$bak\_path ] && mkdir -p \$bak\_path   \#<==若备份路径不存在则创建。
mysqldump -B -A --master-data=2 |gzip >\$bak\_path/\$\{file\_name\}.sql.gz
\#<==如果仅为innodb引擎,则可以再加上--single-transaction参数。

\#rsync all data to backup server           \#<==备份完成后立刻推送至备份服务器,
                                           需要提前部署rsync服务。
rsync -az \$bak\_path/ rsync\_backup@172.16.1.31::mysql/ --password-file=/etc/
rsync.password
\#del expiries file         \#<==删除本地的7天备份,如果空间紧张那么保留三天也可以。
find \$bak\_path/ -type -f -name "*.sql.gz" -mtime +7|xargs rm -f
\end{verbatim}


\noindent\rule{\textwidth}{0.4pt}


\par 稍微复杂点的备份脚本如下:

\noindent\rule{\textwidth}{0.4pt}


\begin{verbatim}[root@oldboy scripts]\# cat bak.sh
\#!/bin/bash
\#Author: oldboy
export PATH=/application/mysql/bin:/usr/local/bin:/sbin:/bin:/usr/sbin:/usr/bin
bak\_path=/server/bakup
[ ! -d \$bak\_path ] && mkdir -p \$bak\_path
\#if week is 6 then bak other file name.
if [ \$(date +\%w) -eq 6 ]             \#<==如果时间为周六,则,
  then
      file\_name=bak\_\$(date +\%w\_\%F)   \#<==将备份文件名改为周和日期,目的是在备份
                                      服务器上保留每周六的数据。
else
      file\_name=bak\_\$(date +\%F)      \#<==否则,备份文件名为日期。
fi
mysqldump -B -A --master-data=2 |gzip >\$bak\_path/\$\{file\_name\}.sql.gz
md5sum \$bak\_path/\$\{file\_name\}.sql.gz >\$bak\_path/\$\{file\_name\}.flag
\#<==做md5指纹的目的是用于未来检查备份及传输结果是否正常。
\#rsync all data to backup server     \#<==备份完成后立刻推送至备份服务器,需要提前
                                      部署rsync服务。
rsync -az \$bak\_path/ rsync\_backup@172.16.1.31::mysql/ --password-file=/etc/
rsync.password
\#del expiries file         \#<==删除本地的7天备份,如果空间紧张那么保留三天也可以。
find \$bak\_path/ -type -f -name "*.sql.gz" -mtime +7|xargs rm -f
\#send mail to self or administrator  \#<==还可以将备份结果发送邮件给管理员,不过最佳策略是在备份服务器上检查备份结果统一发管理员。
\end{verbatim}


\noindent\rule{\textwidth}{0.4pt}


\par 上述两个脚本是相对较简单的企业级实战数据库备份脚本,在实际工作中,可能在判断及逻辑上会更复杂一些。

\par 最后配置定时任务,使其每日0点执行上述脚本:

\noindent\rule{\textwidth}{0.4pt}


\begin{verbatim}[root@oldboy scripts]\# crontab -l|tail -2
\#bak mysql for oldboy at 20170318
00 00 * * * /bin/sh /server/scripts/bak.sh &>/dev/null
\end{verbatim}


\noindent\rule{\textwidth}{0.4pt}


\par 最终的数据库本地(备份服务器相同)备份结果示例如下:

\noindent\rule{\textwidth}{0.4pt}


\begin{verbatim}[root@oldboy scripts]\# ll /server/bakup/
total 360
-rw-r--r--. 1 root root     70 Mar 19  2017 bak\_2017-03-19.flag
\#<==普通备份对应的指纹文件。
-rw-r--r--. 1 root root 178891 Mar 19  2017 bak\_2017-03-19.sql.gz
\#<==普通的备份文件。
-rw-r--r--. 1 root root     72 Mar 18 00:00 bak\_6\_2017-03-18.flag
\#<==周六备份对应的指纹文件。
-rw-r--r--. 1 root root 178892 Mar 18 00:00 bak\_6\_2017-03-18.sql.gz
\#<==周六的备份文件。
[root@oldboy scripts]\# cat /server/bakup/bak\_6\_2017-03-18.flag
50ae9d28d0719682b89f6c929d02dfa4  /server/bakup/bak\_6\_2017-03-18.sql.gz
\end{verbatim}


\noindent\rule{\textwidth}{0.4pt}


\par 将来在备份服务器上可以检查备份是否成功,以及数据在备份过程中是否被改变等:

\noindent\rule{\textwidth}{0.4pt}


\begin{verbatim}[root@oldboy bakup]\# md5sum -c bak\_6\_2017-03-18.flag
\#<==检测备份文件是否正常以及在传输过程中是否有损坏或被篡改。
/server/bakup/bak\_6\_2017-03-18.sql.gz: OK  \#<==OK表示一切正常。
\end{verbatim}


\noindent\rule{\textwidth}{0.4pt}


\par 一般会在备份服务器上保留最近7天的所有备份,同时保留每周六的全部备份命令:

\noindent\rule{\textwidth}{0.4pt}


\begin{verbatim}[root@oldboy bakup]\# find /server/bakup/ -type f -name "bak\_*" -mtime +7 ! -name "bak\_6*"
/server/bakup/bak\_2017-03-19.flag
/server/bakup/bak\_2017-03-19.sql.gz
[root@oldboy bakup]\# find /server/bakup/ -type f -name "bak\_*" -mtime +7 ! -name "bak\_6*"|xargs rm -f
\end{verbatim}


\noindent\rule{\textwidth}{0.4pt}


\par 
\begin{figure}[htbp]
\centering
\includegraphics[width=0.8\textwidth]{Image00006.jpg}
\end{figure}

 提示:
 上述备份保留策略是让备份的内容尽量多,但又要尽可能地减少重复。

\par 有关服务器数据备份,老男孩曾经讲过一个很完备的备份和检查并发邮件的方案,请参考:\href{http://edu.51cto.com/course/course_id-8198.html}{http://edu.51cto.com/course/course\_id-8198.html}
 。

\par 2.全备的数据什么时候可以派上用场

\par 使用mysqldump全备的数据什么时候可以派上用场呢?

\par ·迁移或者升级数据库时。

\par ·增加从库的时候。

\par ·人为执行DDL、DML语句破坏数据库数据时(此时若使用主从库就会无法防止数据丢失,因为所有库都会执行破坏的语句)。

\par ·跨机房灾备时,此时需要将全备份复制到异地。

\par 若是因为硬件或删除物理文件导致数据库故障,就不需要用备份数据恢复了,可以直接把主库关闭,在从库上配置好VIP等配置后,启动从库提供服务即可。

\par 3.中小企业增量备份策略与实践

\par 中小企业增量备份就是备份binlog文件,在MySQL没有主从复制功能或主从复制功能不完善的时候,我们就曾采取定时或实时推binlog文件的方法。例如每分钟推一次binlog到备份服务器上,或者通过mysqlbinlog参数read-from-remote-server,在其他服务器上远程读取binlog。

\par 但是这类方法都不是最佳的,因为有可能会丢失数据。比较好的binlog增量备份或MySQL备份方法就是为MySQL数据库配置异机主从复制功能(实时复制功能),即binlog会被实时地发送到从服务器上,这样效果才是最好的。当然,也要相应地在主从复制的从库上实现全备。

\par 4.备份binlog增量文件何时可以派上用场

\par 当需要完整恢复数据库数据的时候,就会需要binlog增量恢复。

\par 5.企业里MySQL备份策略选择

\par 大多数中小企业的数据库环境都为一主多从,因此,可采取在一个从库服务器上专门做全量以及增量备份(需要开启从库记录binlog日志功能),至于备份方法,采用mysqldump、Xtrabackup均可。


% From text00126.html


\subsection{8.5.2 中小企业MySQL增量恢复案例实战}


\par 前文讲解了需要利用备份进行恢复的一些场景,本节将为大家讲解的案例是较为复杂的恢复场景,比如,内部管理人员通过SQL语句误删除数据。

\par 首先,思考一下,具备什么条件才能完整恢复数据库数据?

\par ·具备全量备份(mysqldump)。

\par ·除全量备份以外,还有全量备份之后产生的的所有binlog增量日志。

\par 假设当前数据库内的数据如下:

\noindent\rule{\textwidth}{0.4pt}


\begin{verbatim}[root@oldboy \textasciitilde{}]\# mysql  -e "select * from oldboy.test;"
+----+---------+
| id | name    |
+----+---------+
|  1 | oldboy  |
|  2 | oldgirl |
|  3 | inca    |
|  4 | zuma    |
|  5 | kaka    |
+----+---------+
\end{verbatim}


\noindent\rule{\textwidth}{0.4pt}


\par 模拟0点开始对数据库oldboy数据进行全备,命令如下:

\noindent\rule{\textwidth}{0.4pt}


\begin{verbatim}[root@oldboy \textasciitilde{}]\# mkdir /data/backup -p
[root@oldboy \textasciitilde{}]\# date -s "2017/03/19"
Sun Mar 19 00:00:00 EDT 2017
[root@oldboy \textasciitilde{}]\# mysqldump -B --master-data=2 --single-transaction oldboy| gzip>/data/backup/oldboy\_\$(date +\%F).sql.gz
[root@oldboy \textasciitilde{}]\# ls -l /data/backup/
total 4
-rw-r--r--. 1 root root 867 Mar 19 00:00 oldboy\_2017-03-19.sql.gz
\end{verbatim}


\noindent\rule{\textwidth}{0.4pt}


\par 模拟0点全备后用户继续写入数据,示例如下:

\noindent\rule{\textwidth}{0.4pt}


\begin{verbatim}[root@oldboy \textasciitilde{}]\# mysql -e "use oldboy;insert into test values(6,'bingbing');"
[root@oldboy \textasciitilde{}]\# mysql -e "use oldboy;insert into test values(7,'xiaoting');"
[root@oldboy \textasciitilde{}]\# mysql -e "select * from oldboy.test;"
+----+----------+
| id | name     |
+----+----------+
|  1 | oldboy   |
|  2 | oldgirl  |
|  3 | inca     |
|  4 | zuma     |
|  5 | kaka     |
|  6 | bingbing |     \#<==全备后写入的数据,实际工作中会写入很多条。
|  7 | xiaoting |
+----+----------+
\end{verbatim}


\noindent\rule{\textwidth}{0.4pt}


\par 模拟上午10:00点管理人员删除oldboy数据库,示例如下:

\noindent\rule{\textwidth}{0.4pt}


\begin{verbatim}[root@oldboy \textasciitilde{}]\# date -s "2017/03/19 10:00"
Sun Mar 19 10:00:00 EDT 2017
[root@oldboy \textasciitilde{}]\# mysql  -e "drop database oldboy;show databases;"
+--------------------+
| Database           |
+--------------------+
| information\_schema |
| mysql              |
| oldboy\_utf8        |
| performance\_schema |
+--------------------+
\end{verbatim}


\noindent\rule{\textwidth}{0.4pt}


\par 再假设数据库出问题10分钟后,公司的网站运营人员报网站故障,联系运维人员DBA解决。此时,DBA或开发人员会查看网站报错(或者查看后台日志),可以看到连不上oldboy数据库的提示,登录数据库排查,发现数据库oldboy库已经不存在了,突然想起来早晨管理员说要处理下数据库,于是询问管理员是如何处理的,管理员答曰,10点左右刚刚清除了一个“没有用的”数据库……问题的原因终于找到了。事实上,登录系统分析系统审计日志,以及分析数据库binlog也可以发现库丢失的原因;若是开发人员,通过程序日志判断应该也可以查出来。

\par 
\begin{figure}[htbp]
\centering
\includegraphics[width=0.8\textwidth]{Image00006.jpg}
\end{figure}

 提示:
 数据库的权限管理流程制度很重要,如果能未雨绸缪,就不会这么麻烦地恢复数据了。俗话说得病容易去病难,数据库的恢复也是同样的道理。

\par 找到原因后,就可以开始准备恢复了。恢复前要做如下准备。移走所有binlog增量文件,防止被二次破坏,并确认是否有全备:

\noindent\rule{\textwidth}{0.4pt}


\begin{verbatim}[root@oldboy \textasciitilde{}]\# cp -a /application/mysql/data/oldboy-bin.* /data/backup/
[root@oldboy \textasciitilde{}]\# ls -l /data/backup/
total 28
-rw-r--r--. 1 root  root   867 Mar 19 00:00 oldboy\_2017-03-19.sql.gz
\#<==全备在这里。
-rw-rw----. 1 mysql mysql  143 Mar  3 05:50 oldboy-bin.000001
-rw-rw----. 1 mysql mysql  168 Mar  3 05:57 oldboy-bin.000002
-rw-rw----. 1 mysql mysql  168 Mar  3 05:57 oldboy-bin.000003
-rw-rw----. 1 mysql mysql 7737 Mar 19 10:00 oldboy-bin.000004
-rw-rw----. 1 mysql mysql   80 Mar  3 05:57 oldboy-bin.index
\end{verbatim}


\noindent\rule{\textwidth}{0.4pt}


\par 下面开始恢复实战。

\par 1)停止数据库对外访问。因为是通过drop命令删除数据库的,后面不会有写入操作,因此,可以不用额外停止写入。但如果是因为update导致的数据破坏,最好是停库处理或对外停止写入。这里采用iptables防火墙屏蔽所有应用程序的写入:

\noindent\rule{\textwidth}{0.4pt}


\begin{verbatim}[root@oldboy \textasciitilde{}]\# iptables -I INPUT -p tcp --dport 3306 ! -s 172.16.1.51 -j DROP
\#<==非172.16.1.51禁止访问数据库3306端口。
\end{verbatim}


\noindent\rule{\textwidth}{0.4pt}


\par 2)解压全备的数据,命令如下:

\noindent\rule{\textwidth}{0.4pt}


\begin{verbatim}[root@oldboy \textasciitilde{}]\# cd /data/backup/
[root@oldboy backup]\# gzip -cd oldboy\_2017-03-19.sql.gz >oldboy.sql
[root@oldboy backup]\# ls -lrt oldboy.sql
-rw-r--r--. 1 root root 2205 Mar 19 10:29 oldboy.sql
\end{verbatim}


\noindent\rule{\textwidth}{0.4pt}


\par 3)解析binlog文件增量数据。

\par 那么应该从哪个binlog文件的哪个位置开始呢?

\par 我们在全备时增加了--master-data=2参数,还记得这个参数的作用吗?

\par 这个参数就是帮我们记录全备后的binlog文件名以及binlog文件对应的恢复位置点的。这里就可以用到这个参数的作用了:

\noindent\rule{\textwidth}{0.4pt}


\begin{verbatim}[root@oldboy backup]\# sed -n '22p' oldboy.sql
-- CHANGE MASTER TO MASTER\_LOG\_FILE='oldboy-bin.000004', MASTER\_LOG\_POS=7181;
\end{verbatim}


\noindent\rule{\textwidth}{0.4pt}


\par 从上面的代码可以看到,要从oldboy-bin.000004文件的7181位置点开始恢复增量数据:

\noindent\rule{\textwidth}{0.4pt}


\begin{verbatim}[root@oldboy backup]\# mysqlbinlog -d oldboy oldboy-bin.000004 --start-position= 7181 -r bin.sql
\end{verbatim}


\noindent\rule{\textwidth}{0.4pt}


\par 工作中除了oldboy-bin.000004文件之外,还可能生成oldboy-bin.000005、oldboy-bin.000006……多个binlog文件,现在可以继续恢复后面的所有binlog文件:

\noindent\rule{\textwidth}{0.4pt}


\begin{verbatim}[root@oldboy backup]\# mysqlbinlog -d oldboy oldboy-bin.000005 oldboy-bin. 000006 -r bin1.sql
\end{verbatim}


\noindent\rule{\textwidth}{0.4pt}


\par 
\begin{figure}[htbp]
\centering
\includegraphics[width=0.8\textwidth]{Image00006.jpg}
\end{figure}

 提示:
 本节恢复就只有一个binlog文件,即oldboy-bin.000004,因此此条命令就不要再执行了。

\par 4)说到这里,需要剔除误删数据库的drop语句,否则,直接恢复就又进入了删库故障的坑:

\noindent\rule{\textwidth}{0.4pt}


\begin{verbatim}[root@oldboy backup]\# grep -w drop  bin.sql \#<==过滤drop单词的行。
drop database oldboy
[root@oldboy backup]\# sed -i '/drop database oldboy/d' bin.sql
\#<==删除drop数据库oldboy的语句。
[root@oldboy backup]\# grep -w drop  bin.sql
\end{verbatim}


\noindent\rule{\textwidth}{0.4pt}


\par 现在,准备彻底完成,开始正式恢复。

\par 先恢复0点以前的全备数据:

\noindent\rule{\textwidth}{0.4pt}


\begin{verbatim}[root@oldboy backup]\# mysql </data/backup/oldboy.sql  \#<==先恢复全备,即0点
                                                  以前的备份。
[root@oldboy backup]\# mysql -e "select * from oldboy.test;"
+----+---------+
| id | name    |
+----+---------+
|  1 | oldboy  |
|  2 | oldgirl |
|  3 | inca    |
|  4 | zuma    |
|  5 | kaka    |
+----+---------+
\end{verbatim}


\noindent\rule{\textwidth}{0.4pt}


\par 看到了吧,0点以前的数据已恢复了,但是0点到出故障时间的增量还没有。

\par 再恢复增量数据,即从binlog中解析出清理了drop语句的bin.sql文件:

\noindent\rule{\textwidth}{0.4pt}


\begin{verbatim}[root@oldboy backup]\# mysql oldboy</data/backup/bin.sql   \#<==恢复增量文件。
[root@oldboy backup]\# mysql -e "select * from oldboy.test;"
+----+----------+
| id | name     |
+----+----------+
|  1 | oldboy   |
|  2 | oldgirl  |
|  3 | inca     |
|  4 | zuma     |
|  5 | kaka     |
|  6 | bingbing | \#<==增量的数据回来了。
|  7 | xiaoting | \#<==增量的数据回来了。
+----+----------+
\end{verbatim}


\noindent\rule{\textwidth}{0.4pt}


\par 到此,利用MySQL的全量备份以及MySQL的binlog增量进行恢复的案例就结束了。


% From text00127.html


\section{8.6 分库分表的生产备份策略}


\subsection{
8.6.1 为什么要分库分表备份}
}


\par 下面的全备命令把2个库备份成了一个备份文件:

\noindent\rule{\textwidth}{0.4pt}


\begin{verbatim}mysqldump -B --master-data=2 --single-transaction oldboy mysql|gzip>/data/backup/all.sql.gz
\end{verbatim}


\noindent\rule{\textwidth}{0.4pt}


\par 但在还原时,很多时候只需要还原一个库或者多个库的一个表,这个时候,整个备份文件就会很难拆分,给恢复也会带来麻烦,关于这点,前面也曾讲解过。对于这种情况,最好是分库分表备份。


% From text00128.html


\subsection{8.6.2 如何进行分库备份}


\par 最佳的方法就是从数据库中取出所有库名,然后对每个数据库执行一次备份。

\par 分库备份的脚本如下:

\noindent\rule{\textwidth}{0.4pt}


\begin{verbatim}[root@oldboy scripts]\# cat fenku.sh
\#!/bin/bash
\#Author: oldboy
export PATH=/application/mysql/bin:/usr/local/bin:/sbin:/bin:/usr/sbin:/usr/bin
bak\_path=/server/bakup/\$(date +\%F)
[ ! -d \$bak\_path ] && mkdir -p \$bak\_path
for dbname in `mysql -e "show databases"|sed '1,2d'|grep -v \_schema`
\#<==取库名轮询备份。
do
    mysqldump -B --master-data=2 |gzip >\$bak\_path/\$\{dbname\}\_\$(date +\%F).sql.gz
    \#<==注意备份的名字。
done
\end{verbatim}


\noindent\rule{\textwidth}{0.4pt}


\par 
\begin{figure}[htbp]
\centering
\includegraphics[width=0.8\textwidth]{Image00006.jpg}
\end{figure}

 提示:
 有关Shell脚本的实战知识,请参考老男孩已出版的新书《跟老男孩学Linux运维:Shell编程实战》一书。

\par 下面就来执行分库脚本并查看备份结果:

\noindent\rule{\textwidth}{0.4pt}


\begin{verbatim}[root@oldboy scripts]\# sh fenku.sh
[root@oldboy scripts]\# ll /server/bakup/2017-03-19/
total 12
-rw-r--r--. 1 root root 130 Mar 19 11:10 mysql\_2017-03-19.sql.gz
-rw-r--r--. 1 root root 130 Mar 19 11:10 oldboy\_2017-03-19.sql.gz
-rw-r--r--. 1 root root 130 Mar 19 11:10 oldboy\_utf8\_2017-03-19.sql.gz
\end{verbatim}


\noindent\rule{\textwidth}{0.4pt}



% From text00129.html


\subsection{8.6.3 如何进行分表备份}


\par 分表备份比分库更细,实际上就是先取一个库名,然后循环读取该库里的表进行备份,备份完之后,再取下一个库名,继续循环库里的所有表进行备份,直到所有库里的所有表都备份完毕。

\par 实际脚本如下:

\noindent\rule{\textwidth}{0.4pt}


\begin{verbatim}[root@oldboy scripts]\# cat fenbiao.sh
\#!/bin/bash
\#Author: oldboy
export PATH=/application/mysql/bin:/usr/local/bin:/sbin:/bin:/usr/sbin:/usr/bin
bak\_path=/server/bakup/\$(date +\%F)
[ ! -d \$bak\_path ] && mkdir -p \$bak\_path
for dbname in `mysql -e "show databases"|sed '1,2d'|grep -v \_schema`
do
    for tablename in `mysql -e "show tables from \$dbname;"|sed '1d'`
    do
        mysqldump -B --master-data=2 |gzip >\$bak\_path/\$\{dbname\}\_\$\{tablename\} \_\$(date +\%F).sql.gz
    done
done
\end{verbatim}


\noindent\rule{\textwidth}{0.4pt}


\par 下面执行分表脚本并查看备份结果:

\noindent\rule{\textwidth}{0.4pt}


\begin{verbatim}[root@oldboy scripts]\# sh fenbiao.sh
[root@oldboy scripts]\# ll -lt /server/bakup/2017-03-19/
total 116
-rw-r--r--. 1 root root 130 Mar 19 11:28 oldboy\_test\_2017-03-19.sql.gz
\#<==oldboy库只有test表。
-rw-r--r--. 1 root root 130 Mar 19 11:28 mysql\_user\_2017-03-19.sql.gz
-rw-r--r--. 1 root root 130 Mar 19 11:28 mysql\_time\_zone\_transition\_type\_2017-03-19.sql.gz
-rw-r--r--. 1 root root 130 Mar 19 11:28 mysql\_time\_zone\_transition\_2017-03-19.sql.gz
-rw-r--r--. 1 root root 130 Mar 19 11:28 mysql\_time\_zone\_name\_2017-03-19.sql.gz
-rw-r--r--. 1 root root 130 Mar 19 11:28 mysql\_time\_zone\_leap\_second\_2017-03-19.sql.gz
-rw-r--r--. 1 root root 130 Mar 19 11:28 mysql\_time\_zone\_2017-03-19.sql.gz
-rw-r--r--. 1 root root 130 Mar 19 11:28 mysql\_tables\_priv\_2017-03-19.sql.gz
-rw-r--r--. 1 root root 130 Mar 19 11:28 mysql\_slow\_log\_2017-03-19.sql.gz
\end{verbatim}


\noindent\rule{\textwidth}{0.4pt}


\par 分表分库虽然对恢复单个库表来说比较方便,但是如果要完整恢复又该怎么呢?

\par 事实上,分库分表的数据尽量不要用于完整恢复,因为binlog有可能会有写入操作,用于单库单表的恢复,如果非要用来做完整恢复,那就通过写脚本来实现吧!


% From text00130.html


\section{8.7 MySQL生产常用备份架构方案}


\par 在中小公司一般比较常用的做法是,每日0点执行全备任务,先把数据按照日期备份到数据库本地,然后推送到数据库备份服务器,由于本地空间有限,因此本地仅保留3~7日的全备。

\par 如果有备用的服务器资源可用,那么最好通过主从同步的方式进行备份,这样即使物理机损坏了也可以很快地切换到新服务器(还可以HA自动切换),但是主从复制的缺点是不能解决错误执行SQL语句的问题,例如,执行“drop database oldboy”命令之后,所有的从库都会傻傻地执行同样的删除动作(这个问题可以通过在从库上实现延迟复制来缓解)。

\par 因此,我们一般会在某一台不对外提供业务的从库上使用上述的mysqldump或Xtrabackup来进行定时备份。这里有个需要特别注意的地方,用于备份从库的二进制日志记录功能必须打开,关于这点具体会在主从复制章节讲解。


% From text00131.html


\section{8.8 本章重点}


\par 1)数据库管理员两大工作核心。

\par 2)全量备份与增量备份的概念。

\par 3)了解逻辑备份的特点、常用工具和应用场景。

\par 4)了解物理备份的特点、常用工具和应用场景。

\par 5)物理备份和逻辑备份区别对比。

\par 6)中小企业的MySQL增量备份及恢复实战过程。

\par 7)如何进行分库分表备份?


% From text00132.html


\chapter{第9章 MySQL物理备份工具Xtrabackup应用实践}


\section{
9.1 Xtrabackup介绍}
}


\par Xtrabackup是Percona公司专门针对MySQL数据库开发的一款开源免费的物理备份(热备)工具,可以对InnoDB和XtraDB等事务引擎的数据库实现非阻塞(即不锁表)方式的备份,也可以针对MyISAM等非事务引擎实现锁表方式备份。

\par Xtrabackup的主要特点具有如下。

\par ·直接复制物理文件,备份和恢复数据的速度非常快,安全可靠。

\par ·在备份期间执行的事务不会间断,备份InnoDB数据不会影响业务。

\par ·备份期间不会增加太多数据库的性能压力。

\par ·支持对备份的数据进行自动校验。

\par ·支持全量、增量、压缩备份及流备份。

\par ·支持在线迁移表以及快速创建新的从库。

\par ·支持几乎所有版本的MySQL和MariaDB。


% From text00133.html


\section{9.2 Xtrabackup备份涉及的数据库名词}


\par 1.MySQL数据文件扩展名知识说明

\par 想要很好地理解本章的内容,首先要做一些数据库相关知识的功课才行,老男孩以表格的形式列出了Xtrabackup备份中涉及的一些数据库专业信息知识,如表9-1所示。

\par 表9-1 Xtrabackup备份中涉及的一些数据库专业信息知识

\begin{figure}[htbp]
\centering
\includegraphics[width=0.8\textwidth]{Image00064.jpg}
\end{figure}



\par 2.事务型引擎的ACID特性

\par 在MySQL中,InnoDB和MariaDB中的XtraDB都是事务型引擎,事务型引擎的共同特征是具备事务的4个特性,这4个特性分别是:原子性(atomicity)、一致性(consistency)、隔离性(isolation)、持久性(durability),又称为ACID特性。表9-2针对这些特性给出了进一步说明。

\par 表9-2 ACID特性说明

\begin{figure}[htbp]
\centering
\includegraphics[width=0.8\textwidth]{Image00065.jpg}
\end{figure}



\par 3.InnoDB引擎内部知识概念

\par 有部分读者对InnoDB的知识知之甚少,但这些知识又非常重要,因此这里列举了一些本章会涉及的InnoDB基础概念知识,以供读者参考,具体见表9-3。

\par 表9-3 InnoDB基础概念知识

\begin{figure}[htbp]
\centering
\includegraphics[width=0.8\textwidth]{Image00066.jpg}
\end{figure}



\par 4.InnoDB引擎内部知识结构图及说明

\par 简单地说,InnoDB的表空间分为共享表空间和独立表空间(推荐)两种,表空间里存放数据的最小单位是页,每个页的默认值为16KB;多个连续的页(默认是64个)组成一个区段;而多个区段和页构成一个段。初始时,InnoDB首先会为每个段分配32个页,之后根据实际需要再将区段分配给段,InnoDB可以一次性添加4个区给一个大的段,从而确保数据存储时能有一个良好的顺序性。

\par 有关上述知识的简单逻辑图如图9-1所示。

\begin{figure}[htbp]
\centering
\includegraphics[width=0.8\textwidth]{Image00067.jpg}
\end{figure}



\par 图9-1 InnoDB引擎内部知识结构

\par 5.InnoDB备份相关名词

\par 下面将介绍InnoDB日志及Xtrabackup备份原理所涉及的词汇知识,见表9-4。

\par 表9-4 InnoDB日志及Xtrabackup备份原理词汇

\begin{figure}[htbp]
\centering
\includegraphics[width=0.8\textwidth]{Image00068.jpg}
\end{figure}




% From text00134.html


\section{9.3 Xtrabackup备份的工作原理流程}


\par 1.Xtrabackup恢复的工作原理

\par Percona Xtrabackup软件是基于InnoDB等事务引擎自带的redo日志和undo日志功能来保持备份和恢复前后数据一致性的,从而确保数据库的数据安全可靠。在InnoDB引擎中存在一个redo日志(事务日志)功能。redo日志文件会存储每一个InnoDB表中的数据修改记录。当InnoDB数据库启动时,会检查数据文件和redo日志文件,将已经提交到事务日志(redo日志文件)中的信息应用(提交)到数据文件并保存,然后根据undo日志信息将修改过但没有提交的数据记录进行回滚(不提交到数据文件)。

\par 2.Xtrabackup执行全备份的原理过程说明

\par 当执行Xtrabackup程序开始备份时,Xtrabackup首先会记录当前redo日志的位置(即对应的LSN号),同时还会在后台启动一个进程持续监视redo日志文件的变化,并将变化的信息都记录到xtrabackup\_logfile中,之后就会针对所有的InnoDB数据文件进行备份(复制),待InnoDB数据文件备份完成之后,再执行“flush tables with read lock”命令对整个数据库锁表,然后备份(复制)MyISAM等非事务引擎的数据文件。待数据文件全部(包括InnoDB、MyISAM数据文件和redo日志数据记录)都备份完毕之后,获取binlog二进制日志位置点信息,最后执行unlock tables解锁命令,恢复整个数据库的可读写状态。

\par 整个备份过程的原理如图9-2所示。

\begin{figure}[htbp]
\centering
\includegraphics[width=0.8\textwidth]{Image00069.jpg}
\end{figure}



\par 图9-2 Innobackupex全备份的原理流程

\begin{figure}[htbp]
\centering
\includegraphics[width=0.8\textwidth]{Image00070.jpg}
\end{figure}



\par 图9-2 (续)

\par 3.Xtrabackup执行全备份恢复的过程

\par 当执行Xtrabackup工具恢复数据时,要经过准备恢复(prepare)和实际恢复(restore)两个步骤。在准备恢复过程结束后,InnoDB表的数据(即备份的物理文件)就恢复到了复制InnoDB文件结束时的时间点,这个时间点也是全库锁表复制MyISAM引擎数据时的起点,所以最终恢复的数据和数据库的数据是一致的。全备的数据有两部分,一部分是全备的物理文件,一部分是Xtrabackup log日志文件,整个恢复过程如图9-3所示。

\begin{figure}[htbp]
\centering
\includegraphics[width=0.8\textwidth]{Image00071.jpg}
\end{figure}



\par 图9-3 准备恢复的基本过程图

\par 4.Xtrabackup执行增量备份的过程

\par Innobackupex增量备份的(仅对InnoDB引擎有效)核心就是复制全备之后的InnoDB中变更的“页”数据,复制时会以全备中xtrabackup\_checkpoints文件对应的LSN号为依据,将大于给定的LSN号的页数据(就是增量数据)进行备份,因为要比对全备的LSN号,所以第一次增量备份是基于全备的,以后实施的每一次增量备份都要基于上一次的增量备份,最终实现备份的数据是连续的、无缺失的,针对MyISAM引擎的备份依然是锁表备份。

\par 增量备份的过程具体如下。

\par 首先在全备的xtrabackup\_checkpoints logfile中找到并记录最后一个checkpoint(last checkpint LSN),然后从该LSN的位置开始复制InnoDB的redo日志到xtrabackup\_logfile,然后开始复制全部的数据文件.ibd,待全部数据复制结束后,就停止复制logfile,增量备份的过程与全备份基本类似,区别就是第二步,仅复制InnoDB中变化的页数据,而非所有物理文件。

\par 5.Xtrabackup执行增量恢复的过程

\par 增量数据的恢复过程与全量备份的恢复过程类似,所不同的是增量恢复是以全备份数据为基础的,增量恢复的数据主要涉及全备的数据、增量的数据、Xtrabackup\_log日志文件。恢复过程是先将增量备份中变化的页数据应用到全备数据中,然后,读取Xtrabackup\_log应用redo数据到全备数据中,同时回滚未提交的事务,整个过程如图9-5所示。

\begin{figure}[htbp]
\centering
\includegraphics[width=0.8\textwidth]{Image00072.jpg}
\end{figure}



\par 图9-4 增量备份具体过程图

\begin{figure}[htbp]
\centering
\includegraphics[width=0.8\textwidth]{Image00073.jpg}
\end{figure}



\par 图9-5 Xtrabackup执行增量恢复过程图解

\par 如果读者至此对备份和恢复过程还不是很理解的话,先别惊慌,请继续阅读实践部分,实践完了再回来看这些流程就很简单了。


% From text00135.html


\section{9.4 Xtrabackup工具安装}


\subsection{
9.4.1 系统环境说明}
}


\par 本文采用的是CentOS Linux系统,其他系统与此大同小异,具体环境为:

\noindent\rule{\textwidth}{0.4pt}


\begin{verbatim}[root@oldboy \textasciitilde{}]\# cat /etc/redhat-release
CentOS release 6.9 (Final)
[root@oldboy \textasciitilde{}]\# uname -r
2.6.32-696.el6.x86\_64
[root@oldboy \textasciitilde{}]\# uname -m
x86\_64
\end{verbatim}


\noindent\rule{\textwidth}{0.4pt}



% From text00136.html


\subsection{9.4.2 安装Xtrabackup}


\par 默认情况下系统中并没有安装Xtrabackup程序,因此,可以执行如下命令进行安装(前提是要确保服务器可以联网)。

\par 1)配置epel源,命令如下:

\noindent\rule{\textwidth}{0.4pt}


\begin{verbatim}[root@oldboy \textasciitilde{}]\# wget -O /etc/yum.repos.d/epel.repo  http://mirrors.aliyun.com/repo/epel-6.repo
\end{verbatim}


\noindent\rule{\textwidth}{0.4pt}


\par 2)安装Xtrabackup软件需要的基础环境包:

\noindent\rule{\textwidth}{0.4pt}


\begin{verbatim}[root@oldboy \textasciitilde{}]\# yum -y install perl perl-devel libaio libaio-devel perl-Time-HiRes perl-DBD-MySQL
\end{verbatim}


\noindent\rule{\textwidth}{0.4pt}


\par 3)下载Xtrabackup软件并安装:

\noindent\rule{\textwidth}{0.4pt}


\begin{verbatim}[root@oldboy \textasciitilde{}]\# wget https://www.percona.com/downloads/XtraBackup/Percona- XtraBackup-2.4.4/binary/redhat/6/x86\_64/percona-xtrabackup-24-2.4.4-1.el6.x86\_64.rpm \#<==如果地址无效,请到官方网站寻找合适的版本。
[root@oldboy \textasciitilde{}]\# ls -l percona-xtrabackup-24-2.4.4-1.el6.x86\_64.rpm
-rw-r--r--. 1 root root 8534624 Jul 21  2016 percona-xtrabackup-24-2.4.4-1.el6.x86\_64.rpm
[root@oldboy \textasciitilde{}]\# yum -y install percona-xtrabackup-24-2.4.4-1.el6.x86\_64.rpm
\#<==自动解决依赖包。
\end{verbatim}


\noindent\rule{\textwidth}{0.4pt}


\par 4)最后,查看安装后的结果:

\noindent\rule{\textwidth}{0.4pt}


\begin{verbatim}[root@oldboy \textasciitilde{}]\# ls -l `which xtrabackup innobackupex`
lrwxrwxrwx. 1 root root   10 Mar 20 06:04 /usr/bin/innobackupex -> xtrabackup
\#<==软链接了。
-rwxr-xr-x. 1 root root 22136096 Jul 21  2016 /usr/bin/xtrabackup
\end{verbatim}


\noindent\rule{\textwidth}{0.4pt}


\par 到此为止,Xtrabackup软件就安装完成了。


% From text00137.html


\section{9.5 Xtrabackup应用实践}


\subsection{
9.5.1 用于Xtrabackup数据备份的用户}
}


\par 用于Xtrabackup数据备份的用户也是需要一定权限的,具体配置如下(backup用户):

\noindent\rule{\textwidth}{0.4pt}


\begin{verbatim}mysql> CREATE USER 'backup'@'localhost' IDENTIFIED BY 'backup123';
mysql> GRANT RELOAD, LOCK TABLES, PROCESS, REPLICATION CLIENT ON *.* TO 'backup' @'localhost';
mysql> FLUSH PRIVILEGES;
\end{verbatim}


\noindent\rule{\textwidth}{0.4pt}


\par 此外,用于操作备份的Linux系统用户有读取和写入相关目录的权限,一般情况下,还是建议读者使用root管理员权限进行备份,不管是MySQL还是Linux系统,特别是在学习期间,简单方便是最重要的,本文也都采用root权限(生产中也是采用root权限居多)来演示。


% From text00138.html


\subsection{9.5.2 用于恢复的MySQL配置文件}


\par 先创建存放MySQL日志文件的目录,将MySQL日志和数据文件分离到不同的目录中,以便于Xtrabackup恢复正式数据:

\noindent\rule{\textwidth}{0.4pt}


\begin{verbatim}[root@oldboy mysql]\# mkdir /application/mysql/logs -p
[root@oldboy mysql]\# chown -R mysql.mysql /application/mysql/logs
\end{verbatim}


\noindent\rule{\textwidth}{0.4pt}


\par 然后,修改MySQL的配置文件,把二进制日志、错误日志、慢查询日志从MySQL数据文件目录中迁移出来,并增加数据文件对应的目录。

\noindent\rule{\textwidth}{0.4pt}


\begin{verbatim}[root@oldboy mysql]\# egrep -v "\#|\textasciicircum{}\$" /etc/my.cnf
[client]
user=root
password=oldboy123
[mysqld]
basedir = /application/mysql/                 \#<==增加MySQL根目录。
datadir = /application/mysql/data/            \#<==MySQL数据目录,Xtrabackup
                                               恢复数据时需要这个目录。
\#\#\#\#\#\#\#\#\#\#\#binlog\#\#\#\#\#\#\#\#\#\#\#\#
log\_bin = /application/mysql/logs/oldboy-bin  \#<==二进制日志路径调整,尽可能在
                                              装数据库前就调整好。
expire\_logs\_days = 7
\#\#\#\#\#\#\#\#\#\#\#slow log\#\#\#\#\#\#\#\#\#\#\#
slow-query-log = ON
long\_query\_time = 2
log\_queries\_not\_using\_indexes = ON
slow-query-log-file = /application/mysql/logs/slow.log  \#<==慢查询日志路径调整。
min\_examined\_row\_limit = 800
[mysqld\_safe]
log-error = /application/mysql/logs/oldboy.err         \#<==慢查询日志路径调整。
sql\_mode=NO\_ENGINE\_SUBSTITUTION,STRICT\_TRANS\_TABLES
\end{verbatim}


\noindent\rule{\textwidth}{0.4pt}


\par 
\begin{figure}[htbp]
\centering
\includegraphics[width=0.8\textwidth]{Image00006.jpg}
\end{figure}

 提示:
 上面注释过的行就是增加或更改的参数,其他参数和前面章节使用的一致。

\par 调整完配置文件,别忘了重启MySQL:

\noindent\rule{\textwidth}{0.4pt}


\begin{verbatim}[root@oldboy \textasciitilde{}]\# /etc/init.d/mysqld restart
Shutting down MySQL.. SUCCESS!
Starting MySQL. SUCCESS!
\end{verbatim}


\noindent\rule{\textwidth}{0.4pt}


\par 需要特别强调的是,其实本节实现MySQL配置文件的调整,就是为了后文直接使用Xtrabackup(--copy-back)将数据文件还原到数据文件的目录中,如果使用mv等命令移动还原数据文件,则可以不调整本节的参数。


% From text00139.html


\subsection{9.5.3 Xtrabackup软件附带的备份工具说明}


\par Xtrabackup软件主要包含两个常用的工具。

\par (1)Xtrabackup命令

\par Xtrabackup命令是专门用于对InnoDB和XtraDB等事务引擎的数据库热备份的工具,不能用于备份MyISAM等其他类型的引擎数据,其主要特点是备份数据时完全不用锁表。

\par (2)Innobackupex命令

\par Innobackupex命令是将上述Xtrabackup命令使用perl脚本进行二次封装的工具,除了可以用于InnoDB和XtraDB等引擎之外,还可以备份MyISAM及多种引擎混合使用的场景,该命令的主要特点是备份事务引擎数据而不用锁表,可以备份非事务引擎数据,但要锁表。

\par 鉴于国内运维人员和DBA在企业里通常使用功能更多的Innobackupex命令来进行备份和恢复,因此本章也以此命令为例来讲解Xtrabackup软件的原理及备份恢复应用实践。


% From text00140.html


\subsection{9.5.4 Innobackupex工具语法介绍}


\par Innobackupex工具语法如下:

\noindent\rule{\textwidth}{0.4pt}


\begin{verbatim}innobackupex [--user=name] [--host=name] [--password=WORD] [--port=PORT] [--socket=SOCKET] [--defaults-file=MY.CNF] PATH
\end{verbatim}


\noindent\rule{\textwidth}{0.4pt}


\par 相关参数说明见表9-5。

\par 表9-5 Innobackupex工具参数说明

\begin{figure}[htbp]
\centering
\includegraphics[width=0.8\textwidth]{Image00074.jpg}
\end{figure}



\begin{figure}[htbp]
\centering
\includegraphics[width=0.8\textwidth]{Image00075.jpg}
\end{figure}




% From text00141.html


\subsection{9.5.5 全备与恢复全备实践}


\par 1.数据准备

\par 本次实践使用第8章的数据,如下:

\noindent\rule{\textwidth}{0.4pt}


\begin{verbatim}mysql> use oldboy
Database changed
mysql> select * from test;
+----+----------+
| id | name     |
+----+----------+
|  1 | oldboy   |
|  2 | oldgirl  |
|  3 | inca     |
|  4 | zuma     |
|  5 | kaka     |
|  6 | bingbing |
|  7 | xiaoting |
+----+----------+
7 rows in set (0.00 sec)
\end{verbatim}


\noindent\rule{\textwidth}{0.4pt}


\par 2.开始全备

\par 1)先创建一个规范的备份目录:

\noindent\rule{\textwidth}{0.4pt}


\begin{verbatim}[root@oldboy \textasciitilde{}]\# mkdir /server/backup -p
[root@oldboy \textasciitilde{}]\# innobackupex --defaults-file=/etc/my.cnf --user=root --password= oldboy123 --socket=/application/mysql-5.6. 40/tmp/mysql.sock --no-timestamp /server/backup/full
\end{verbatim}


\noindent\rule{\textwidth}{0.4pt}


\par 
\begin{figure}[htbp]
\centering
\includegraphics[width=0.8\textwidth]{Image00006.jpg}
\end{figure}

 提示:
 有关参数的说明,请参照表9-5,注意socket的路径为编译时指定的路径。

\par 执行备份时,屏幕会有非常多的输出信息,其实就是备份的工作流程细节,请读者仔细看,全备份与增量备份每当命令操作成功时,都会在日志的结尾处打印“completed ok!”作为标识:

\noindent\rule{\textwidth}{0.4pt}


\begin{verbatim}[root@oldboy \textasciitilde{}]\# innobackupex --defaults-file=/etc/my.cnf --user=root --password= oldboy123 --socket=/application/mysql-5.6.40/tmp/mysql.sock --no-timestamp /server/backup/full
...省略若干行提醒信息及读取数据库的innodb参数配置等...
180320 11:31:38 >> log scanned up to (1955436)    \#<==开始记录并监控redo日志。
180320 11:31:38 [01] Copying ./ibdata1 to /server/backup/full/ibdata1
\#<==开始复制innodb文件。
180320 11:31:39 [01]        ...done
...省略若干行...
180320 11:31:39 [01] Copying ./oldboy/test.ibd to /server/backup/full/oldboy/test.ibd
180320 11:31:39 [01]        ...done
180320 11:31:39 >> log scanned up to (1955436)
180320 11:31:39 Executing FLUSH NO\_WRITE\_TO\_BINLOG TABLES...
180320 11:31:39 Executing FLUSH TABLES WITH READ LOCK...  \#<==开始锁表备份非
                                                       事务引擎表。
180320 11:31:39 Starting to backup non-InnoDB tables and files
180320 11:31:39 [01] Copying ./performance\_schema/events\_statements\_summary\_by\_thread\_by\_event\_name.frm to /server/backup/full/performance\_schema/events\_statements\_summary\_by\_thread\_by\_event\_name.frm
180320 11:31:39 [01]        ...done
...省略若干复制MyISAM表的数据文件行...
180320 11:31:40 Finished backing up non-InnoDB tables and files
\#<==备份非事务引擎表完成。
180320 11:31:40 [00] Writing xtrabackup\_binlog\_info   \#<==记录binlog位置。
180320 11:31:40 [00]        ...done
180320 11:31:40 Executing FLUSH NO\_WRITE\_TO\_BINLOG ENGINE LOGS...
xtrabackup: The latest check point (for incremental): '1955436'
\#<==记录最新的checkpoint。
xtrabackup: Stopping log copying thread.    \#<==停止日志监控进程。
.180320 11:31:40 >> log scanned up to (1955436)
180320 11:31:41 Executing UNLOCK TABLES     \#<==停止锁表。
180320 11:31:41 All tables unlocked
...省略备份结果信息提示...
180320 11:31:41 completed OK!               \#<==备份成功标识。
\end{verbatim}


\noindent\rule{\textwidth}{0.4pt}


\par 最终的备份结果如下:

\noindent\rule{\textwidth}{0.4pt}


\begin{verbatim}[root@oldboy \textasciitilde{}]\# ll /server/backup/full/
total 77860
-rw-r-----. 1 root root      418 Mar 20 07:35 backup-my.cnf
\#<==配置文件备份。
-rw-r-----. 1 root root 79691876 Mar 20 07:35 ibdata1
\#<==共享表空间备份。
drwxr-x---. 2 root root     4096 Mar 20 07:35 mysql
\#<== mysql库的备份。
drwxr-x---. 2 root root     4096 Mar 20 07:35 oldboy
\#<== 3行oldboy等库的备份。
drwxr-x---. 2 root root     4096 Mar 20 07:35 oldboy\_utf8
drwxr-x---. 2 root root     4096 Mar 20 07:35 performance\_schema
-rw-r-----. 1 root root       22 Mar 20 07:35 xtrabackup\_binlog\_info
\#<== binlog位置信息。
-rw-r-----. 1 root root      113 Mar 20 07:35 xtrabackup\_checkpoints
\#<== checkpoints信息。
-rw-r-----. 1 root root      569 Mar 20 07:35 xtrabackup\_info
\#<== xtrabackup信息。
-rw-r-----. 1 root root     2560 Mar 20 07:35 xtrabackup\_logfile
\#<== xtrabackup日志文件。
[root@oldboy \textasciitilde{}]\# cd /server/backup/full/
[root@oldboy full]\# cat xtrabackup\_binlog\_info
oldboy-bin.000004      120   \#<==binlog位置信息。
[root@oldboy full]\# cat xtrabackup\_checkpoints \#<==存放备份的起始位置和结束位置。
backup\_type = full-backuped    \#<==备份类型
from\_lsn = 0                   \#<==checkpoints起始点。
to\_lsn = 1955065               \#<==checkpoints结束点。
last\_lsn = 1955065
compact = 0
recover\_binlog\_info = 0
\end{verbatim}


\noindent\rule{\textwidth}{0.4pt}


\par 3.利用全备恢复数据

\par 使用Innobackupex备份的数据有可能会处于不一致的状态,因此在恢复数据之前,首先要将数据调成一致性的状态,即需要应用备份过程中记录变化的日志(xtrabackup\_logfile),同时回滚未提交的事务日志数据,具体命令如下:

\noindent\rule{\textwidth}{0.4pt}


\begin{verbatim}[root@oldboy full]\# innobackupex --apply-log --use-memory=32M /server/backup/full/
\end{verbatim}


\noindent\rule{\textwidth}{0.4pt}


\par 
\begin{figure}[htbp]
\centering
\includegraphics[width=0.8\textwidth]{Image00006.jpg}
\end{figure}

 提示:


\par 1)“--apply-log”用于记录变化的日志,并回滚未提交的事务日志数据。

\par 2)“--use-memory=32M”是分配的使用内存大小。

\par 该命令同样会有大量的输出,若提示“18032007:57:33 completed OK!”,则表示操作成功。

\par 执行完此命令之后,备份的数据就可以用来恢复了。

\par 使用Xtrabackup备份的本质就是复制物理数据文件,因此,在进行最终恢复时,需要临时关闭数据库,把物理文件复制回去进行恢复,整个恢复过程如下:

\noindent\rule{\textwidth}{0.4pt}


\begin{verbatim}[root@oldboy \textasciitilde{}]\# /etc/init.d/mysqld stop  \#<==停库。
Shutting down MySQL.. SUCCESS!
[root@oldboy \textasciitilde{}]\# ss -lnt|grep 3306        \#<==检查端口。
[root@oldboy \textasciitilde{}]\# mv /application/mysql/data /application/mysql/data\_ori
\#<==模拟删除数据文件。
[root@oldboy \textasciitilde{}]\# mkdir -p /application/mysql/data  \#<==创建原始数据目录,
                                                  目录要为空。
[root@oldboy \textasciitilde{}]\# mv /server/backup/full/* /application/mysql/data/
\#<==使用mv直接复制会更舒服一些。
\#<==也可以使用下面的命令复制还原数据,和mv命令的功能是一样的,读者2选1即可,如下命令
    的缺点是要根据MySQL配置文件的路径进行回拷贝,因此配置文件没有指定数据文件路径、数据
    文件目录不为空等,都会导致无法回拷贝的问题。具体命令为innobackupex --defaults-
    file=/etc/my.cnf --copy-back --rsync /server/backup/full/
[root@oldboy \textasciitilde{}]\# chown -R mysql.mysql /application/mysql/data
\#<==如果权限不对,那么启动数据库可能报“Starting MySQL. ERROR! The server quit
    without updating PID file”错误。
[root@oldboy \textasciitilde{}]\# /etc/init.d/mysqld start
Starting MySQL. SUCCESS!
[root@oldboy \textasciitilde{}]\# mysql -e "select * from oldboy.test;"
+----+----------+
| id | name     |
+----+----------+
|  1 | oldboy   |
|  2 | oldgirl  |
|  3 | inca     |
|  4 | zuma     |
|  5 | kaka     |
|  6 | bingbing |
|  7 | xiaoting |
+----+----------+
\end{verbatim}


\noindent\rule{\textwidth}{0.4pt}


\par 恢复成功。


% From text00142.html


\subsection{9.5.6 增量备份与恢复增量数据实践}


\par 使用Xtrabackup程序做增量备份之前,首先要进行一次全备份,第一次增量备份是基于全备进行的,之后的每一次增量备份都会基于上一次增量备份的数据来实现,操作过程如下。

\par (1)创建基础全备份

\par 该命令与9.5.5节介绍全备时使用的命令一模一样:

\noindent\rule{\textwidth}{0.4pt}


\begin{verbatim}[root@oldboy \textasciitilde{}]\# innobackupex --defaults-file=/etc/my.cnf --user=root --password=oldboy123 --socket=/application/mysql-5.6. 40/tmp/mysql.sock --no-timestamp /server/backup/base\_full
\end{verbatim}


\noindent\rule{\textwidth}{0.4pt}


\par 
\begin{figure}[htbp]
\centering
\includegraphics[width=0.8\textwidth]{Image00006.jpg}
\end{figure}

 提示:
 注意全备的路径,后面增量备份时要用到。

\par (2)模拟增加数据,然后做增量备份

\par 增加如下数据:

\noindent\rule{\textwidth}{0.4pt}


\begin{verbatim}mysql> use oldboy
Database changed
2 rows in set (0.00 sec)
mysql> insert into test values(8,'outman');
Query OK, 1 row affected (0.00 sec)
mysql> insert into test values(9,'outgirl');
Query OK, 1 row affected (0.00 sec)
mysql> select * from test;
+----+----------+
| id | name     |
+----+----------+
|  1 | oldboy   |
|  2 | oldgirl  |
|  3 | inca     |
|  4 | zuma     |
|  5 | kaka     |
|  6 | bingbing |
|  7 | xiaoting |
|  8 | outman   |  \#<==全备以后的模拟增加了如下两行数据。
|  9 | outgirl  |
+----+----------+
9 rows in set (0.00 sec)
\end{verbatim}


\noindent\rule{\textwidth}{0.4pt}


\par 开始做第一次增量备份:

\noindent\rule{\textwidth}{0.4pt}


\begin{verbatim}[root@oldboy \textasciitilde{}]\# innobackupex --defaults-file=/etc/my.cnf --user=root --password=oldboy123 --socket=/application/mysql-5.6.40/tmp/mysql.sock --no-timestamp --incremental-basedir=/server/backup/base\_full --incremental /server/backup/one\_inc
\end{verbatim}


\noindent\rule{\textwidth}{0.4pt}


\par 注意,这里指定的全备路径是--incremental-basedir=/server/backup/base\_full。

\par 增量备份的结果在/server/backup/one\_inc下,感兴趣的读者可以认真看看,其实就是从全备信息里最后的LSN开始读取redo日志,对改变的数据进行增量备份,所以,我们刚刚增加的增量数据很少,InnoDB数据文件备份得也很少,增量备份的InnoDB数据备份的关键输出如下:

\noindent\rule{\textwidth}{0.4pt}


\begin{verbatim}xtrabackup: using the full scan for incremental backup
180320 12:37:41 [01] Copying ./ibdata1 to /server/backup/one\_inc/ibdata1.delta
180320 12:37:41 [01]        ...done
180320 12:37:41 [01] Copying ./mysql/innodb\_table\_stats.ibd to /server/backup/one\_inc/mysql/innodb\_table\_stats.ibd.delta
180320 12:37:41 [01]        ...done
180320 12:37:41 [01] Copying ./mysql/innodb\_index\_stats.ibd to /server/backup/one\_inc/mysql/innodb\_index\_stats.ibd.delta
180320 12:37:41 [01]        ...done
180320 12:37:41 >> log scanned up to (1959061)
...省略无用行...
180320 12:37:41 [01] Copying ./oldboy/test.ibd to /server/backup/one\_inc/oldboy/test.ibd.delta
180320 12:37:41 [01]        ...done
\end{verbatim}


\noindent\rule{\textwidth}{0.4pt}


\par 根据输出,我们也大概了解了插入数据到数据库时,对数据库会有什么影响,而对于非事务引擎的备份,则仍是锁表进行全备。

\par (3)再模拟增加的数据,然后做第二次增量备份

\par 增加如下数据:

\noindent\rule{\textwidth}{0.4pt}


\begin{verbatim}mysql> use oldboy
mysql> insert into test values(10,'two\_inc1');
Query OK, 1 row affected (0.00 sec)
mysql> insert into test values(11,'two\_inc2');
Query OK, 1 row affected (0.00 sec)
\end{verbatim}


\noindent\rule{\textwidth}{0.4pt}


\par 开始做第二次增量备份,注意,这里要用到第一次增量备份的目录,而不是全备目录:

\noindent\rule{\textwidth}{0.4pt}


\begin{verbatim}[root@oldboy \textasciitilde{}]\# innobackupex --defaults-file=/etc/my.cnf --user=root --password=oldboy123 --socket=/application/mysql-5.6. 40/tmp/mysql.sock --no-timestamp --parallel=3 --incremental-basedir=/server/backup/one\_inc/ --incremental /server/backup/two\_inc
\end{verbatim}


\noindent\rule{\textwidth}{0.4pt}


\par 到此为止,我们就完成了1次全备,以及全备之后的两次增量数据备份。

\par (4)开始做增量数据恢复

\par 增量恢复的步骤为先恢复全备(base\_full目录)数据,然后再恢复第一次增量(one\_inc)的数据,以及第二次增量(two\_inc)的数据,如果有更多次增量备份就以此类推。

\par 1)应用redo日志恢复全备数据:

\noindent\rule{\textwidth}{0.4pt}


\begin{verbatim}[root@oldboy \textasciitilde{}]\# innobackupex --apply-log --use-memory=32M --redo-only /server/backup/base\_full/
\end{verbatim}


\noindent\rule{\textwidth}{0.4pt}


\par 
\begin{figure}[htbp]
\centering
\includegraphics[width=0.8\textwidth]{Image00006.jpg}
\end{figure}

 特别强调:
 非最后一次合并增量数据一定要加--redo-only参数,即只应用redo日志恢复数据,而不执行undo回滚未提交的数据,等到最后一次增量备份合并完成后再进行undo日志回滚数据。

\par 2)合并第一次的增量数据到全备数据目录:

\noindent\rule{\textwidth}{0.4pt}


\begin{verbatim}[root@oldboy \textasciitilde{}]\# innobackupex --apply-log --use-memory=32M --redo-only --incremental-dir=/server/backup/one\_inc /server/backup/base\_full/
\end{verbatim}


\noindent\rule{\textwidth}{0.4pt}


\par 
\begin{figure}[htbp]
\centering
\includegraphics[width=0.8\textwidth]{Image00006.jpg}
\end{figure}

 提示:
 这里最关键的就是增量备份的参数和路径了。

\par 3)合并第二次的增量数据到全备数据目录:

\noindent\rule{\textwidth}{0.4pt}


\begin{verbatim}[root@oldboy \textasciitilde{}]\# innobackupex --apply-log --use-memory=32M --incremental-dir=/server/backup/two\_inc /server/backup/base\_full/
\end{verbatim}


\noindent\rule{\textwidth}{0.4pt}


\par 需要强调的是,最后一次合并增量数据到全备时,取消了“--redo-only”参数。

\par 4)对最终的全量数据做redo日志应用,并执行undo回滚数据:

\noindent\rule{\textwidth}{0.4pt}


\begin{verbatim}[root@oldboy \textasciitilde{}]\# innobackupex --apply-log --use-memory=32M /server/backup/base\_full/
\end{verbatim}


\noindent\rule{\textwidth}{0.4pt}


\par 
\begin{figure}[htbp]
\centering
\includegraphics[width=0.8\textwidth]{Image00006.jpg}
\end{figure}

 提示:
 本命令取消了“--redo-only”参数,目的是应用undo日志回滚数据,为最终的恢复数据做准备。

\par 5)开始正式恢复数据:

\noindent\rule{\textwidth}{0.4pt}


\begin{verbatim}[root@oldboy \textasciitilde{}]\# /etc/init.d/mysqld stop
Shutting down MySQL.. SUCCESS!
[root@oldboy \textasciitilde{}]\# ss -lnt|grep 3306
[root@oldboy \textasciitilde{}]\# mv /application/mysql/data /tmp/data\_ori1
[root@oldboy \textasciitilde{}]\# mkdir -p /application/mysql/data  \#<==原始数据目录/application/mysql/data要为空。
[root@oldboy \textasciitilde{}]\# innobackupex --defaults-file=/etc/my.cnf --copy-back --rsync /server/backup/base\_full/ \#<==用innobackupex看看效果和mv是不是一样的,注意MySQL的配置文件要正确。
[root@oldboy \textasciitilde{}]\# chown -R mysql.mysql /application/mysql/data
\#<==如果权限不对,那么启动数据库可能会报“Starting MySQL. ERROR! The server quit
    without updating PID file”错误。
[root@oldboy \textasciitilde{}]\# /etc/init.d/mysqld start
Starting MySQL. SUCCESS!
[root@oldboy \textasciitilde{}]\# mysql -e "select * from oldboy.test;"
+----+----------+
| id | name     |
+----+----------+
|  1 | oldboy   |
|  2 | oldgirl  |
|  3 | inca     |
|  4 | zuma     |
|  5 | kaka     |
|  6 | bingbing |
|  7 | xiaoting |
|  8 | outman   |
|  9 | outgirl  |
| 10 | two\_inc1 |
| 11 | two\_inc2 |
+----+----------+
\end{verbatim}


\noindent\rule{\textwidth}{0.4pt}



% From text00143.html


\subsection{9.5.7 中小企业MySQL Xtrabackup物理增量恢复案例实战}


\par 前文讲解的Xtrabackup增量恢复是不完善的,本节将讲解较为复杂的恢复方法,即内部管理人员通过SQL语句将数据误删除后,所采用的完善恢复方案会涉及为Xtrabackup进行全量与增量备份,以及binlog日志的恢复知识。

\par 在讲解案例之前,思考一下具备什么样的条件才能完整地物理恢复数据库数据。

\par 答案具体如下。

\par ·具备全量备份(Xtrabckup备份的全备)。

\par ·具备全量之后的所有增量备份(Xtrabckup备份的增量)。

\par ·具备最后一次增量备份以后的所有MySQL的binlog增量日志。

\par 现在,准备模拟数据。假设当前数据库内的数据如下:

\noindent\rule{\textwidth}{0.4pt}


\begin{verbatim}mysql> use oldboy
Database changed
mysql> truncate table test;
Query OK, 0 rows affected (0.02 sec)
mysql> insert into test values(1,'full01');
Query OK, 1 row affected (0.00 sec)
mysql> insert into test values(2,'full02');
Query OK, 1 row affected (0.01 sec)
mysql> insert into test values(3,'full03');
Query OK, 1 row affected (0.00 sec)
mysql> insert into test values(4,'full04');
Query OK, 1 row affected (0.00 sec)
mysql> insert into test values(5,'full05');
Query OK, 1 row affected (0.00 sec)
mysql> select * from test;
+----+--------+
| id | name   |
+----+--------+
|  1 | full01 |
|  2 | full02 |
|  3 | full03 |
|  4 | full04 |
|  5 | full05 |
+----+--------+
5 rows in set (0.00 sec)
\end{verbatim}


\noindent\rule{\textwidth}{0.4pt}


\par 先模拟3月21日0点开始对数据库oldboy数据进行全备:

\noindent\rule{\textwidth}{0.4pt}


\begin{verbatim}[root@oldboy \textasciitilde{}]\# date -s "2018/03/21"
Sun Mar 21 00:00:00 EDT 2018
[root@oldboy \textasciitilde{}]\# innobackupex --defaults-file=/etc/my.cnf --user=root --password= oldboy123 --socket=/application/mysql-5.6.34/tmp/mysql.sock --no-timestamp /server/backup/new\_base\_full
\end{verbatim}


\noindent\rule{\textwidth}{0.4pt}


\par 然后模拟3月21日0点全备之后(0:00-24:00点)用户继续写入数据:

\noindent\rule{\textwidth}{0.4pt}


\begin{verbatim}[root@oldboy \textasciitilde{}]\# mysql -e "use oldboy;insert into test values(6,'new\_inc\_one\_1');"
[root@oldboy \textasciitilde{}]\# mysql -e "use oldboy;insert into test values(7,'new\_inc\_one\_2');"
[root@oldboy \textasciitilde{}]\# mysql -e "select * from oldboy.test;"
+----+---------------+
| id | name          |
+----+---------------+
|  1 | full01        |
|  2 | full02        |
|  3 | full03        |
|  4 | full04        |
|  5 | full05        |
|  6 | new\_inc\_one\_1 |
|  7 | new\_inc\_one\_2 |
+----+---------------+
\end{verbatim}


\noindent\rule{\textwidth}{0.4pt}


\par 之后,在3月22日0点做增量备份(没做全备):

\noindent\rule{\textwidth}{0.4pt}


\begin{verbatim}[root@oldboy \textasciitilde{}]\# date -s "2018/03/22"
Wed Mar 22 00:00:00 EDT 2018
[root@oldboy \textasciitilde{}]\# innobackupex --defaults-file=/etc/my.cnf --user=root --password= oldboy123 --socket=/application/mysql-5.6.34/tmp/mysql.sock --no-timestamp --incremental-basedir=/server/backup/new\_base\_full --incremental /server/backup/new\_one\_inc
\end{verbatim}


\noindent\rule{\textwidth}{0.4pt}


\par 
\begin{figure}[htbp]
\centering
\includegraphics[width=0.8\textwidth]{Image00006.jpg}
\end{figure}

 提示:
 可以按周每天一直增量备份到下一次全备之前,这里就以备份一次增量为例,减少篇幅。

\par 在3月22日0点增备之后(0:00-24:00)用户继续写入数据:

\noindent\rule{\textwidth}{0.4pt}


\begin{verbatim}[root@oldboy \textasciitilde{}]\# mysql -e "use oldboy;insert into test values(8,'binlog\_data\_1');"
[root@oldboy \textasciitilde{}]\# mysql -e "use oldboy;insert into test values(9,'binlog\_data\_2');"
[root@oldboy \textasciitilde{}]\# mysql -e "select * from oldboy.test;"
+----+---------------+
| id | name          |
+----+---------------+
|  1 | full01        |
|  2 | full02        |
|  3 | full03        |
|  4 | full04        |
|  5 | full05        |
|  6 | new\_inc\_one\_1 |
|  7 | new\_inc\_one\_2 |
|  8 | binlog\_data\_1 |
|  9 | binlog\_data\_2 |
+----+---------------+
\end{verbatim}


\noindent\rule{\textwidth}{0.4pt}


\par 假设3月22日上午10:00点管理人员误删除了oldboy数据库:

\noindent\rule{\textwidth}{0.4pt}


\begin{verbatim}[root@oldboy \textasciitilde{}]\# date -s "2018/03/22 10:00"
Tue Mar 22 10:00:00 EDT 2018
[root@oldboy \textasciitilde{}]\# mysql  -e "drop database oldboy;show databases;"
+--------------------+
| Database           |
+--------------------+
| information\_schema |
| mysql              |
| oldboy\_utf8        |
| performance\_schema |
+--------------------+
\end{verbatim}


\noindent\rule{\textwidth}{0.4pt}


\par 数据库出问题10分钟后,公司的网站运营人员报网站故障,联系运维人员、DBA解决。此时,DBA或开发人员会查看网站报错(或者查看后台日志),可以看到连不上oldboy数据库的提示,然后登录数据库排查,发现数据库oldboy库已经不存在了,经过询问才知道是管理员10点左右刚刚清除了一个“没有用的”数据库……问题的原因终于找到了。事实上,登录系统分析系统审计日志,以及分析数据库binlog也可以发现库丢失的原因;开发人员通过程序日志判断应该也可以查出来原因。

\par 找到了原因,就可以开始进行恢复前的准备。

\par 1)移走所有binlog增量文件,防止被二次破坏。

\noindent\rule{\textwidth}{0.4pt}


\begin{verbatim}[root@oldboy \textasciitilde{}]\# cp -a /application/mysql/logs /server/backup/binlog
\end{verbatim}


\noindent\rule{\textwidth}{0.4pt}


\par 2)开始恢复数据前的增量备份和全备合并。

\par 先合并全备数据到新全备目录,即恢复第一天0点前的所有数据:

\noindent\rule{\textwidth}{0.4pt}


\begin{verbatim}[root@oldboy \textasciitilde{}]\# innobackupex --apply-log --use-memory=32M --redo-only /server/backup/new\_base\_full/
\end{verbatim}


\noindent\rule{\textwidth}{0.4pt}


\par 再合并增量数据到新全备目录,即恢复第二天0点前的所有数据:

\noindent\rule{\textwidth}{0.4pt}


\begin{verbatim}[root@oldboy \textasciitilde{}]\# innobackupex --apply-log --use-memory=32M --incremental-dir=/server/backup/new\_one\_inc /server/backup/new\_base\_full/
\end{verbatim}


\noindent\rule{\textwidth}{0.4pt}


\par 最后应用所有redo日志,并回滚未提交的数据:

\noindent\rule{\textwidth}{0.4pt}


\begin{verbatim}[root@oldboy \textasciitilde{}]\# innobackupex --apply-log --use-memory=32M /server/backup/new\_base\_full/
\end{verbatim}


\noindent\rule{\textwidth}{0.4pt}


\par 3)开始恢复binlog日志数据。

\par 由于3月22日0~10点之前的所有数据并不在全备里,也不在增量里,而是在MySQL的binlog日志里,因此还要进行如下操作来完成恢复。

\par 首先查看3月22日0点最后一次增量的binlog位置信息:

\noindent\rule{\textwidth}{0.4pt}


\begin{verbatim}[root@oldboy \textasciitilde{}]\# cat /server/backup/new\_one\_inc/xtrabackup\_binlog\_info
oldboy-bin.000006 588
\end{verbatim}


\noindent\rule{\textwidth}{0.4pt}


\par 这个位置信息以后的binlog才是我们需要的,以前的数据在Xtrabackup的全量和增量里已经有了,然后查看问题出现后第一时间备份的binlog文件信息:

\noindent\rule{\textwidth}{0.4pt}


\begin{verbatim}[root@oldboy \textasciitilde{}]\# ll /server/backup/binlog/oldboy-bin*
-rw-rw----. 1 mysql mysql  143 Mar 20 11:29 /server/backup/binlog/oldboy-bin.000001
-rw-rw----. 1 mysql mysql  143 Mar 20 12:14 /server/backup/binlog/oldboy-bin.000002
-rw-rw----. 1 mysql mysql 1058 Mar 20 13:52 /server/backup/binlog/oldboy-bin.000003
-rw-rw----. 1 mysql mysql  120 Mar 20 13:56 /server/backup/binlog/oldboy-bin.000004
-rw-rw----. 1 mysql mysql 1941 Mar 21 10:06 /server/backup/binlog/oldboy-bin.000005
-rw-rw----. 1 mysql mysql  709 Mar 22 10:12 /server/backup/binlog/oldboy-bin.000006
-rw-rw----. 1 mysql mysql  588 Mar 22 10:18 /server/backup/binlog/oldboy-bin.000007
-rw-rw----. 1 mysql mysql  294 Mar 22 10:14 /server/backup/binlog/oldboy-bin.index
\end{verbatim}


\noindent\rule{\textwidth}{0.4pt}


\par 从上面的信息可知,需要使用的两个binlog文件分别为oldboy-bin.000006和oldboy-bin.000007。

\par 现在从oldboy-bin.000006文件的588位置点开始恢复增量数据:

\noindent\rule{\textwidth}{0.4pt}


\begin{verbatim}[root@oldboy \textasciitilde{}]\# cd /server/backup/binlog
[root@oldboy binlog]\# mysqlbinlog -d oldboy oldboy-bin.000006 oldboy-bin. 000007 --start-position=588 -r bin.sql
\end{verbatim}


\noindent\rule{\textwidth}{0.4pt}


\par 这里要剔除误删数据库的drop语句,否则,直接恢复就又进入了删库故障的坑:

\noindent\rule{\textwidth}{0.4pt}


\begin{verbatim}[root@oldboy binlog]\# grep -w drop  bin.sql \#<==过滤drop单词的行。
drop database oldboy
[root@oldboy binlog]\# sed -i '/drop database oldboy/d' bin.sql
\#<==删除drop数据库oldboy的语句。
[root@oldboy binlog]\# grep -w drop  bin.sql
\end{verbatim}


\noindent\rule{\textwidth}{0.4pt}


\par 需要特别强调的是,本案例是删除数据库了,因此binlog日志就不再增长了,所以可以利用上文第一时间备份的binlog恢复,如果是update操作误改了数据又没有停库,那么,此时最好是在数据库服务器上用Iptables控制应用程序连接数据库之后,再从上述binlog位置点开始读取正式日志目录下的所有binlog日志,一直将Iptables限制访问时的所有binlog日志恢复完。

\par 下面开始正式恢复数据库数据的实战。

\par 1)停止数据库对外访问。因为是使用drop命令删除库,而且后面不会有写入了,所以可以不用额外停止写入,但如果是update导致的数据破坏,那么最好是在发现问题的第一时间就进行停库处理或对外停止写入。这里采用Iptables防火墙屏蔽所有应用程序的写入。

\noindent\rule{\textwidth}{0.4pt}


\begin{verbatim}[root@oldboy \textasciitilde{}]\# iptables -I INPUT -p tcp --dport 3306 ! -s 172.16.1.51 -j DROP
\#<==非172.16.1.51禁止访问数据库3306端口。
\end{verbatim}


\noindent\rule{\textwidth}{0.4pt}


\par 2)恢复处理好的Xtrabackup备份的全量及增量数据:

\noindent\rule{\textwidth}{0.4pt}


\begin{verbatim}[root@oldboy \textasciitilde{}]\# /etc/init.d/mysqld stop                \#<==恢复时要停库。
Shutting down MySQL.. SUCCESS!
[root@oldboy \textasciitilde{}]\# ss -lnt|grep 3306
[root@oldboy \textasciitilde{}]\# mv /application/mysql/data /tmp/data3  \#<==移走旧的数据。
[root@oldboy \textasciitilde{}]\# mkdir -p /application/mysql/data
\#<==原始数据目录/application/mysql/data要为空。
[root@oldboy \textasciitilde{}]\# mv /server/backup/new\_base\_full/* /application/mysql/data/
\#<==采用mv,而不是xtrabackup的--copy-back参数还原,使用mv的时候,前文的MySQL
       配置就可以采用了。
[root@oldboy \textasciitilde{}]\# chown -R mysql.mysql /application/mysql/data/
\#<==别忘了重新授权。
[root@oldboy \textasciitilde{}]\# /etc/init.d/mysqld start           \#<==恢复完毕,启动数据库。
Starting MySQL. SUCCESS!
[root@oldboy \textasciitilde{}]\# mysql -e "select * from oldboy.test;"
+----+---------------+
| id | name          |
+----+---------------+
|  1 | full01        |
|  2 | full02        |
|  3 | full03        |
|  4 | full04        |
|  5 | full05        |
|  6 | new\_inc\_one\_1 |
|  7 | new\_inc\_one\_2 |
+----+---------------+
\end{verbatim}


\noindent\rule{\textwidth}{0.4pt}


\par 到此,使用Xtrabackup备份的到3月21日的全备数据和3月22日0点之前增量备份的数据都回来了。

\par 接下来开始恢复3月22日0点到10点的数据库数据。

\par 这时需要恢复binlog数据,即从binlog解析出数据并且清理了drop语句的bin.sql文件:

\noindent\rule{\textwidth}{0.4pt}


\begin{verbatim}[root@oldboy \textasciitilde{}]\# cd /server/backup/binlog
[root@oldboy binlog]\# mysql oldboy<bin.sql \#<==恢复binlog数据。
[root@oldboy binlog]\# mysql -e "select * from oldboy.test;"
+----+---------------+
| id | name          |
+----+---------------+
|  1 | full01        |
|  2 | full02        |
|  3 | full03        |
|  4 | full04        |
|  5 | full05        |
|  6 | new\_inc\_one\_1 |
|  7 | new\_inc\_one\_2 |
|  8 | binlog\_data\_1 |  \#<==binlog里记录的数据回来了。
|  9 | binlog\_data\_2 |
+----+---------------+
\end{verbatim}


\noindent\rule{\textwidth}{0.4pt}


\par 到此,就实现了利用Xtrabackup程序进行全备和增量备份的恢复,并且使用binlog进行了完整的数据恢复。


% From text00144.html


\subsection{9.5.8 使用Xtrabackup物理分库分表备份}


\par 使用Xtrabackup物理分库分表备份的前提是开启独立表空间模式,即设置innodb\_file\_per\_table=On,MySQL 5.6版本默认已经开启了此模式。

\par 1)备份单个库oldboy:

\noindent\rule{\textwidth}{0.4pt}


\begin{verbatim}[root@oldboy \textasciitilde{}]\# innobackupex --defaults-file=/etc/my.cnf --user=root --password
=oldboy123 --socket=/application/mysql-5.6.40/tmp/mysql.sock --no-timestamp --databases=
"oldboy" /server/backup/oldboy\_full
[root@oldboy \textasciitilde{}]\# ls -l /server/backup/oldboy\_full/
total 77848
-rw-r-----. 1 root root      418 Mar 22 12:27 backup-my.cnf
-rw-r-----. 1 root root 79691876 Mar 22 12:27 ibdata1
drwxr-x---. 2 root root     4096 Mar 22 12:27 oldboy
-rw-r-----. 1 root root       22 Mar 22 12:27 xtrabackup\_binlog\_info
-rw-r-----. 1 root root      113 Mar 22 12:27 xtrabackup\_checkpoints
-rw-r-----. 1 root root      595 Mar 22 12:27 xtrabackup\_info
-rw-r-----. 1 root root     2560 Mar 22 12:27 xtrabackup\_logfile
\end{verbatim}


\noindent\rule{\textwidth}{0.4pt}


\par 2)备份多个库:

\noindent\rule{\textwidth}{0.4pt}


\begin{verbatim}[root@oldboy \textasciitilde{}]\# innobackupex --defaults-file=/etc/my.cnf --user=root --password=
oldboy123 --socket=/application/mysql-5.6.40/tmp/mysql.sock --no-timestamp --databases= "oldboy oldgirl" /server/backup/oldboy\_oldgirl
[root@oldboy \textasciitilde{}]\# ls -l /server/backup/oldboy\_oldgirl/
total 77852
-rw-r-----. 1 root root      418 Mar 22 12:30 backup-my.cnf
-rw-r-----. 1 root root 79691876 Mar 22 12:30 ibdata1
drwxr-x---. 2 root root     4096 Mar 22 12:30 oldboy
drwxr-x---. 2 root root     4096 Mar 22 12:30 oldgirl
-rw-r-----. 1 root root       22 Mar 22 12:30 xtrabackup\_binlog\_info
-rw-r-----. 1 root root      113 Mar 22 12:30 xtrabackup\_checkpoints
-rw-r-----. 1 root root      606 Mar 22 12:30 xtrabackup\_info
-rw-r-----. 1 root root     2560 Mar 22 12:30 xtrabackup\_logfile
\end{verbatim}


\noindent\rule{\textwidth}{0.4pt}


\par 3)备份单个表:

\noindent\rule{\textwidth}{0.4pt}


\begin{verbatim}[root@oldboy \textasciitilde{}]\# innobackupex --defaults-file=/etc/my.cnf --user=root --password=
oldboy123 --socket=/application/mysql-5.6.40/tmp/mysql.sock --no-timestamp --databases= "oldboy.test" /server/backup/oldboy\_test
\end{verbatim}


\noindent\rule{\textwidth}{0.4pt}


\par 4)备份多个库的多个表:

\noindent\rule{\textwidth}{0.4pt}


\begin{verbatim}[root@oldboy \textasciitilde{}]\# innobackupex --defaults-file=/etc/my.cnf --user=root --password=
oldboy123 --socket=/application/mysql-5.6.40/tmp/mysql.sock --no-timestamp --databases= "oldboy.test oldboy.student" /server/backup/oldboy\_test\_student
[root@oldboy \textasciitilde{}]\# innobackupex --defaults-file=/etc/my.cnf --user=root --password=
oldboy123 --socket=/application/mysql-5.6.40/tmp/mysql.sock --no-timestamp --databases= "oldboy.test oldboy.student mysql.user" /server/backup/all
\end{verbatim}


\noindent\rule{\textwidth}{0.4pt}


\par 其他用法还有“--tables-file”、“——include”等参数,读者可以参考官方说明进行尝试。


% From text00145.html


\subsection{9.5.9 使用Xtrabackup物理分库分表备份的恢复}


\par 在实施使用Xtrabackup物理分库分表备份的恢复时,需要使用“--export”来做恢复准备,最终恢复时,不能使用常规的“--copy-back”选项,而是要初始化一个带有mysql库(“mysql\_install\_db--user=mysql”)的干净的数据文件目录。

\par 1)恢复准备(请选择一个单独的测试环境来执行):

\noindent\rule{\textwidth}{0.4pt}


\begin{verbatim}[root@oldboy \textasciitilde{}]\# innobackupex --apply-log --export /server/backup/oldboy\_test/ \#<==恢复oldboy的test表。
\end{verbatim}


\noindent\rule{\textwidth}{0.4pt}


\par 2)初始化数据库(选择一个单独的测试环境来执行)。

\par 这里为了方便,在当前库演示:

\noindent\rule{\textwidth}{0.4pt}


\begin{verbatim}[root@oldboy \textasciitilde{}]\# /etc/init.d/mysqld stop                  \#<==停库。
Shutting down MySQL.. SUCCESS!
[root@oldboy \textasciitilde{}]\# mv /application/mysql/data /tmp/data.ori \#<==备份前面的数据。
[root@oldboy \textasciitilde{}]\# mkdir /application/mysql/data
[root@oldboy \textasciitilde{}]\# /application/mysql/scripts/mysql\_install\_db --basedir=/application/mysql/ --datadir=/application/mysql/data --user=mysql
\#<==初始化新的mysql数据文件。
[root@oldboy \textasciitilde{}]\# cp /server/backup/oldboy\_test/* /application/mysql/data/
\#<==将oldboy.test数据回拷贝。
[root@oldboy \textasciitilde{}]\# chown -R mysql.mysql /application/mysql/data \#<==别忘了授权。
[root@oldboy \textasciitilde{}]\# /etc/init.d/mysqld start
Starting MySQL. SUCCESS!
\end{verbatim}


\noindent\rule{\textwidth}{0.4pt}


\par 通过以下命令可登录数据库查看还原的数据:

\noindent\rule{\textwidth}{0.4pt}


\begin{verbatim}mysql> select * from oldboy.test;
+----+---------------+
| id | name          |
+----+---------------+
|  1 | full01        |
|  2 | full02        |
|  3 | full03        |
|  4 | full04        |
|  5 | full05        |
|  6 | new\_inc\_one\_1 |
|  7 | new\_inc\_one\_2 |
|  8 | binlog\_data\_1 |
|  9 | binlog\_data\_2 |
+----+---------------+
9 rows in set (0.01 sec)
\end{verbatim}


\noindent\rule{\textwidth}{0.4pt}


\par 如果有需要,可以通过mysqldump导出后恢复到正式库。

\par 最后别忘了将当前的环境恢复回来:

\noindent\rule{\textwidth}{0.4pt}


\begin{verbatim}[root@oldboy \textasciitilde{}]\# /etc/init.d/mysqld stop                  \#<==停止数据库。
Shutting down MySQL.. SUCCESS!
[root@oldboy \textasciitilde{}]\# mv /application/mysql/data /opt/data.new  \#<==移动旧的数据备份。
[root@oldboy \textasciitilde{}]\# mv /tmp/data.ori /application/mysql/data  \#<==将备份的数据还原。
[root@oldboy \textasciitilde{}]\# chown -R mysql.mysql /application/mysql/data  \#<==授权。
[root@oldboy \textasciitilde{}]\# /etc/init.d/mysqld start                 \#<==启动数据库。
Starting MySQL. SUCCESS!
\end{verbatim}


\noindent\rule{\textwidth}{0.4pt}



% From text00146.html


\chapter{第10章 MySQL数据库日志知识与企业应用实践}


\section{
10.1 MySQL常用日志文件知识}
}


\subsubsection{MySQL日志种类}


\par 为了帮助管理员快速发现数据库的相关运行信息,MySQL为用户提供了几种日志种类,具体见表10-1。

\par 表10-1 MySQL常用日志种类

\begin{figure}[htbp]
\centering
\includegraphics[width=0.8\textwidth]{Image00076.jpg}
\end{figure}



\par 默认情况下,以上所有的日志都处于非激活状态(Linux环境)。当激活日志时,所有的日志都默认配置在数据文件的目录下。管理员也可以对上述日志进行轮询切割,实现该功能常见的命令是mysqladmin flush-logs、mysqldump的“-F”或“--master-data”参数等,下面就分别介绍这几种日志知识。


% From text00147.html


\section{10.2 错误日志的介绍与配置}


\par 1.错误日志的介绍

\par MySQL的错误日志用于记录MySQL服务进程mysqld在启动/关闭或运行过程中遇到的错误信息。

\par 2.错误日志的记录配置

\par MySQL的错误日志通常由mysqld或mysqld\_safe程序产生,前文已经讲解过MySQL的启动原理,因此,可利用如下方法配置记录MySQL错误日志。

\par 方法1:在my.cnf配置文件中调整,注意,是在[mysqld\_safe]模块的下面进行配置。命令如下:

\noindent\rule{\textwidth}{0.4pt}


\begin{verbatim}[mysqld\_safe]
log-error = /application/mysql-5.6.40/data/oldboy.err
\end{verbatim}


\noindent\rule{\textwidth}{0.4pt}


\par 方法2:在启动MySQL服务的命令里加入记录错误日志的参数。

\par 示例如下:

\noindent\rule{\textwidth}{0.4pt}


\begin{verbatim}mysqld\_safe --log-error=/application/mysql-5.6.40/data/oldboy.err &
\end{verbatim}


\noindent\rule{\textwidth}{0.4pt}


\par 查看到的最终结果为:

\noindent\rule{\textwidth}{0.4pt}


\begin{verbatim}mysql> show variables like 'log\_error\%';
+---------------+-------------------------------------------+
| Variable\_name | Value                                     |
+---------------+-------------------------------------------+
| log\_error     | /application/mysql-5.6.40/data/oldboy.err |
+---------------+-------------------------------------------+
1 row in set (0.00 sec)
[root@oldboy \textasciitilde{}]\# tail -4 /application/mysql/data/oldboy.err
2018-03-03 05:50:09 68261 [Note] InnoDB: 128 rollback segment(s) are active.
2018-03-03 05:50:09 68261 [Note] InnoDB: Waiting for purge to start
2018-03-03 05:50:09 68261 [Note] InnoDB: 5.6.40 started; log sequence number 1854308
2018-03-03 05:50:09 68261 [Note] Server hostname (bind-address): '*'; port: 3306
\end{verbatim}


\noindent\rule{\textwidth}{0.4pt}


\par 
\begin{figure}[htbp]
\centering
\includegraphics[width=0.8\textwidth]{Image00006.jpg}
\end{figure}

 提示:
 默认的日志文件名部分和系统主机名一致,扩展名为“.err”。

\par 3.错误日志轮询

\par 管理员可以使用命令轮询错误日志,例如可以按天轮询,具体方法如下:

\noindent\rule{\textwidth}{0.4pt}


\begin{verbatim}[root@oldboy \textasciitilde{}]\# cd  /application/mysql/data/         \#<==切换到日志目录下。
[root@oldboy data]\# mv oldboy.err oldboy\_\$(date +\%F).err  \#<==将错误日志按天
                                                       移动改名。
[root@oldboy data]\# mysqladmin flush-logs             \#<==执行刷新日志命令。
[root@oldboy data]\# ls -l oldboy.err
-rw-rw----. 1 mysql mysql 0 Mar 19 19:34 oldboy.err   \#<==新的错误日志诞生了。
\end{verbatim}


\noindent\rule{\textwidth}{0.4pt}


\par 4.数据库故障排查案例分析

\par 新手安装数据库时,遇到数据库无法启动时的排查方法具体如下。

\par 1)先清空错误日志文件,然后重新启动MySQL服务,再查看日志文件报什么错误,并根据错误日志进行处理。

\par 2)如果无法解决,则删除数据文件,重新初始化数据库。

\par 假设在排查故障时,得到的错误日志提示为:

\noindent\rule{\textwidth}{0.4pt}


\begin{verbatim}180321 17:36:26 InnoDB: Completed initialization of buffer pool
180321 17:36:26  InnoDB: Operating system error number 13 in a file operation.
InnoDB: The error means mysqld does not have the access rights to
InnoDB: the directory.
InnoDB: File name ./ibdata1
\end{verbatim}


\noindent\rule{\textwidth}{0.4pt}


\par 根据提示可知,该错误是权限问题导致的问题,可对数据目录递归执行权限,然后再重启数据库。命令如下:

\noindent\rule{\textwidth}{0.4pt}


\begin{verbatim}[root@oldboy data]\# chown -R mysql.mysql /application/mysql/data/
\end{verbatim}


\noindent\rule{\textwidth}{0.4pt}



% From text00148.html


\section{10.3 普通查询日志的介绍与配置}


\par 1.普通查询日志的介绍

\par 普通查询日志的作用是记录客户端连接信息,以及执行的SQL语句信息。

\par 2.普通查询日志的功能配置

\par 可能官方考虑到普通查询日志的重要性比较低,因此默认情况下普通查询日志是关闭状态,如下所示:

\noindent\rule{\textwidth}{0.4pt}


\begin{verbatim}mysql> show variables like 'general\_log\%';
+------------------+-------------------------------------------+
| Variable\_name    | Value                                     |
+------------------+-------------------------------------------+
| general\_log      | OFF                                       |
| general\_log\_file | /application/mysql-5.6.40/data/oldboy.log |
+------------------+-------------------------------------------+
2 rows in set (0.00 sec)
\end{verbatim}


\noindent\rule{\textwidth}{0.4pt}


\par 可以执行在线修改的命令使其临时生效:

\noindent\rule{\textwidth}{0.4pt}


\begin{verbatim}mysql> set global general\_log = on;
Query OK, 0 rows affected (0.00 sec)
mysql> show variables like 'general\_log\%';
+------------------+-------------------------------------------+
| Variable\_name    | Value                                     |
+------------------+-------------------------------------------+
| general\_log      | ON                                        |
| general\_log\_file | /application/mysql-5.6.40/data/oldboy.log |
+------------------+-------------------------------------------+
2 rows in set (0.00 sec)
\end{verbatim}


\noindent\rule{\textwidth}{0.4pt}


\par 如果希望永久生效,则可以把参数写入my.cnf的配置文件里的[mysqld]模块下:

\noindent\rule{\textwidth}{0.4pt}


\begin{verbatim}general\_log = on
general\_log\_file = /application/mysql-5.6.40/data/oldboy.log
\end{verbatim}


\noindent\rule{\textwidth}{0.4pt}


\par 3.普通查询日志示例

\par 可在执行几个操作后,观察查询日志文件的变化:

\noindent\rule{\textwidth}{0.4pt}


\begin{verbatim}[root@oldboy data]\# tail oldboy.log
/application/mysql-5.6.40/bin/mysqld, Version: 5.6.40-log (Source distribution). started with:
Tcp port: 3306  Unix socket: /application/mysql-5.6.40/tmp/mysql.sock
Time                 Id Command    Argument
180319 19:53:47         67 Query      show variables like 'general\_log\%'
180319 19:55:45         67 Query      show variables like 'log\_error\%'
180319 19:55:52         67 Query      select * from oldboy.test
180319 19:55:57         67 Quit
\end{verbatim}


\noindent\rule{\textwidth}{0.4pt}


\par 4.普通查询日志的生产使用建议

\par 在高并发数据库的场景下,普通查询日志应该是关闭状态的(默认也是关闭的),主要是因为查询日志的信息量很大,容易导致磁盘I/O性能问题。当访问量不是很大,而企业又有审计执行的SQL语句的需求时,可以考虑开启该功能。


% From text00149.html


\section{10.4 二进制日志的介绍与配置}


\par 1.二进制日志的介绍

\par 二进制日志的作用是记录数据库里的数据被修改的SQL语句,一般为DDL和DML语句,例如含有insert、update、delete、create、drop、alter等关键字的语句。

\par 2.二进制日志的作用

\par 二进制日志最重要的作用有2个,具体如下。

\par 第一个是记录MySQL数据的增量数据,用来做增量数据库恢复,没有二进制日志功能,MySQL的备份将无法完整还原数据。

\par 第二个是实现主从复制功能,具体见MySQL主从复制的相关内容。

\par 3.二进制日志的配置

\par 二进制日志的调整,前文已经讲解并实际操作过了,查验如下:

\noindent\rule{\textwidth}{0.4pt}


\begin{verbatim}[root@oldboy data]\# grep log\_bin /etc/my.cnf
log\_bin   \#<==默认情况下记录日志前缀为“主机名-bin”。
mysql> show variables like 'log\_bin';
+---------------+-------+
| Variable\_name | Value |
+---------------+-------+
| log\_bin       | ON    | \#<==记录binlog开关。
+---------------+-------+
1 row in set (0.00 sec)

mysql> show variables like '\%log\_bin';
+---------------+-------+
| Variable\_name | Value |
+---------------+-------+
| log\_bin       | ON    | \#<==记录binlog开关。
| sql\_log\_bin   | ON    | \#<==临时不记录binlog开关。
+---------------+-------+
2 rows in set (0.00 sec)
\end{verbatim}


\noindent\rule{\textwidth}{0.4pt}


\par 有个参数可以实现在开启binlog功能的前提下,临时不记录binlog,示例如下:

\noindent\rule{\textwidth}{0.4pt}


\begin{verbatim}mysql> set session sql\_log\_bin = OFF; \#<==临时停止记录binlog,注意是session
                                       级别,不影响其他会话。
Query OK, 0 rows affected (0.00 sec)
mysql> show variables like '\%log\_bin';
+---------------+-------+
| Variable\_name | Value |
+---------------+-------+
| log\_bin       | ON    |
| sql\_log\_bin   | OFF   | \#<==已关闭。
+---------------+-------+
2 rows in set (0.00 sec)
mysql> create database oldgirl; \#<==建库测试。
Query OK, 1 row affected (0.00 sec)

mysql> show binary logs; \#<==查看binlog文件列表及位置点。
+-------------------+-----------+
| Log\_name          | File\_size |
+-------------------+-----------+
| oldboy-bin.000001 |       143 |
| oldboy-bin.000002 |       168 |
| oldboy-bin.000003 |       168 |
| oldboy-bin.000004 |      9299 |
| oldboy-bin.000005 |       211 | \#<==最新的binlog文件及位置点,也可以通过
                                    “show master status;”来确定。
+-------------------+-----------+
5 rows in set (0.00 sec)
mysql> system mysqlbinlog oldboy-bin.000005|grep "oldgirl"
\#<==过滤binlog文件,没有记录binlog。
mysql> set session sql\_log\_bin = On; \#<==开启开关。
Query OK, 0 rows affected (0.00 sec)
mysql> drop database oldgirl;        \#<==删除数据库。
Query OK, 0 rows affected (0.00 sec)
mysql> system mysqlbinlog oldboy-bin.000005|grep "oldgirl"
\#<==继续过滤,发现记录了binlog。
drop database oldgirl
\end{verbatim}


\noindent\rule{\textwidth}{0.4pt}


\par 到这里,读者应该知道sql\_log\_bin的功能了吧,这个功能通常用于在用户使用mysql恢复数据时不希望恢复的数据SQL记录到binlog里的情况。当然,还有其他的应用场景。

\par 4.二进制日志文件的刷新条件

\par 1)数据库重启会自动刷新binlog为新文件。

\par 2)执行“mysqldump-F”或“mysqladmin flush-logs”会将binlog刷新为新文件。

\par 3)binlog文件达到1GB左右时,会自动刷新binlog为新文件。

\par 4)人为配置切割及调整。

\par binlog最大值控制参数及默认大小查看方法如下:

\noindent\rule{\textwidth}{0.4pt}


\begin{verbatim}mysql> show variables like 'max\_binlog\_size';
+-----------------+------------+
| Variable\_name   | Value      |
+-----------------+------------+
| max\_binlog\_size | 1073741824 |
+-----------------+------------+
1 row in set (0.00 sec)
\end{verbatim}


\noindent\rule{\textwidth}{0.4pt}


\par 5.二进制日志索引文件

\par 除了很多按序列生成的binlog文件列表之外,还有一个索引文件,例如下文里的oldboy-bin.index:

\noindent\rule{\textwidth}{0.4pt}


\begin{verbatim}[root@oldboy data]\# pwd
/application/mysql/data
[root@oldboy data]\# ls -l oldboy-bin.*
-rw-rw----. 1 mysql mysql  143 Mar  3 05:50 oldboy-bin.000001
-rw-rw----. 1 mysql mysql  168 Mar  3 05:57 oldboy-bin.000002
-rw-rw----. 1 mysql mysql  168 Mar  3 05:57 oldboy-bin.000003
-rw-rw----. 1 mysql mysql 9299 Mar 19 19:34 oldboy-bin.000004
-rw-rw----. 1 mysql mysql  211 Mar 19 20:15 oldboy-bin.000005
-rw-rw----. 1 mysql mysql  100 Mar 19 19:34 oldboy-bin.index
\end{verbatim}


\noindent\rule{\textwidth}{0.4pt}


\par 索引文件的文件名和binlog文件一样,只是扩展名为index,查看索引文件内容的命令如下:

\noindent\rule{\textwidth}{0.4pt}


\begin{verbatim}[root@oldboy data]\# cat oldboy-bin.index
./oldboy-bin.000001
./oldboy-bin.000002
./oldboy-bin.000003
./oldboy-bin.000004
./oldboy-bin.000005
\end{verbatim}


\noindent\rule{\textwidth}{0.4pt}


\par binlog索引文件的控制参数为:

\noindent\rule{\textwidth}{0.4pt}


\begin{verbatim}mysql> show variables like 'log\_bin\_index';
+---------------+-------------------------------------------------+
| Variable\_name | Value                                           |
+---------------+-------------------------------------------------+
| log\_bin\_index | /application/mysql-5.6.40/data/oldboy-bin.index |
+---------------+-------------------------------------------------+
1 row in set (0.00 sec)
\end{verbatim}


\noindent\rule{\textwidth}{0.4pt}


\par 6.删除二进制日志的方法

\par binlog日志很重要,不能随意清除,有些读者看到所维护的服务器空间满了,竟然会直接删除binlog物理文件,这样的操作是错误的,应避免。那么如何正确删除binlog文件呢?

\par 首先,要确定什么时候可以删除binlog。

\par 理论上每天的数据库全备时刻以前的binlog都是无用的,但是工作中我们会根据需要保留3~7天的本地binlog文件。

\par 下面来看看具体的删除方式。

\par (1)设置参数自动删除binlog

\par 设置参数自动删除binlog是每个管理员都应该做的,参数设置示例如下。

\par 假设参数为:

\noindent\rule{\textwidth}{0.4pt}


\begin{verbatim}expire\_logs\_days = 7  \#<==删除7天前的日志
\end{verbatim}


\noindent\rule{\textwidth}{0.4pt}


\par 该参数默认是没有配置的,生产中可以同时实现在线更改以及永久更改配置文件:

\noindent\rule{\textwidth}{0.4pt}


\begin{verbatim}mysql> show variables like 'expire\_logs\_days';
+------------------+-------+
| Variable\_name    | Value |
+------------------+-------+
| expire\_logs\_days | 0     |
+------------------+-------+
1 row in set (0.00 sec)

mysql> set global expire\_logs\_days = 7;
Query OK, 0 rows affected (0.00 sec)

mysql> show variables like 'expire\_logs\_days';
+------------------+-------+
| Variable\_name    | Value |
+------------------+-------+
| expire\_logs\_days | 7     |
+------------------+-------+
1 row in set (0.00 sec)

[root@oldboy data]\# grep expir /etc/my.cnf
expire\_logs\_days = 7
\end{verbatim}


\noindent\rule{\textwidth}{0.4pt}


\par (2)从最开始一直删除到指定的文件位置(不含指定文件)

\par 这种方法一般用于处理临时的需求,操作如下:

\noindent\rule{\textwidth}{0.4pt}


\begin{verbatim}[root@oldboy data]\# cp oldboy-bin.* /tmp
\end{verbatim}


\noindent\rule{\textwidth}{0.4pt}


\par 登录数据库时执行如下命令:

\noindent\rule{\textwidth}{0.4pt}


\begin{verbatim}mysql> show binary logs;
+-------------------+-----------+
| Log\_name          | File\_size |
+-------------------+-----------+
| oldboy-bin.000001 |       143 |
| oldboy-bin.000002 |       168 |
| oldboy-bin.000003 |       168 |
| oldboy-bin.000004 |      9299 |
| oldboy-bin.000005 |       211 |
+-------------------+-----------+
4 rows in set (0.00 sec)
mysql> purge binary logs to 'oldboy-bin.000002';
Query OK, 0 rows affected (0.00 sec)
mysql> show binary logs;
+-------------------+-----------+
| Log\_name          | File\_size |
+-------------------+-----------+
| oldboy-bin.000002 |       168 | \#<==序列000002以前的就没了。
| oldboy-bin.000003 |       168 |
| oldboy-bin.000004 |      9299 |
| oldboy-bin.000005 |       211 |
+-------------------+-----------+
4 rows in set (0.00 sec)
\end{verbatim}


\noindent\rule{\textwidth}{0.4pt}


\par (3)按照时间删除binlog日志

\par 这种方法也是用于处理临时的需求,操作如下:

\noindent\rule{\textwidth}{0.4pt}


\begin{verbatim}[root@oldboy data]\# ls -l --time-style=long-iso oldboy-bin*
-rw-rw----. 1 mysql mysql  168 2018-03-03 05:57 oldboy-bin.000002
-rw-rw----. 1 mysql mysql  168 2018-03-03 05:57 oldboy-bin.000003
-rw-rw----. 1 mysql mysql 9299 2018-03-19 19:34 oldboy-bin.000004
-rw-rw----. 1 mysql mysql  211 2018-03-19 20:15 oldboy-bin.000005
-rw-rw----. 1 mysql mysql   80 2018-03-19 21:15 oldboy-bin.index
\end{verbatim}


\noindent\rule{\textwidth}{0.4pt}


\par 下面删除“2018-03-1919:34”以前的binlog文件:

\noindent\rule{\textwidth}{0.4pt}


\begin{verbatim}mysql> PURGE MASTER LOGS BEFORE '2018-03-19 19:34';
Query OK, 0 rows affected (0.00 sec)
mysql> system ls -l --time-style=long-iso oldboy-bin*
-rw-rw----. 1 mysql mysql 9299 2018-03-19 19:34 oldboy-bin.000004
-rw-rw----. 1 mysql mysql  211 2018-03-19 20:15 oldboy-bin.000005
-rw-rw----. 1 mysql mysql   40 2018-03-19 21:20 oldboy-bin.index
\end{verbatim}


\noindent\rule{\textwidth}{0.4pt}


\par (4)清除所有的binlog,并从000001开始重新记录

\par reset master指令可以清除数据库所有的binlog文件,并从000001开始重新记录:

\noindent\rule{\textwidth}{0.4pt}


\begin{verbatim}mysql> reset master;
Query OK, 0 rows affected (0.01 sec)
mysql> system ls -l --time-style=long-iso oldboy-bin*
-rw-rw----. 1 mysql mysql 120 2018-03-19 21:22 oldboy-bin.000001
-rw-rw----. 1 mysql mysql  20 2018-03-19 21:22 oldboy-bin.index
\end{verbatim}


\noindent\rule{\textwidth}{0.4pt}


\par 下面是binlog相关参数的设置和优化思路。

\par 使用如下命令可以查看binlog相关的参数:

\noindent\rule{\textwidth}{0.4pt}


\begin{verbatim}mysql> show variables like 'binlog\_\%';
mysql> show variables like '\%log\_bin\%';
\end{verbatim}


\noindent\rule{\textwidth}{0.4pt}


\par 这里只给大家讲解工作中比较常用的参数。

\par (1)binlog\_cache\_size

\par 二进制日志缓存是数据库为每一个客户连接分配的内存空间。对于事务引擎来说,适当调整该参数会获得更好的性能,该参数的默认值为:

\noindent\rule{\textwidth}{0.4pt}


\begin{verbatim}mysql> show variables like '\%binlog\_cache\%';
+-----------------------+----------------------+
| Variable\_name         | Value                |
+-----------------------+----------------------+
| binlog\_cache\_size     | 32768                |
| max\_binlog\_cache\_size | 18446744073709547520 |
+-----------------------+----------------------+
2 rows in set (0.00 sec)
\end{verbatim}


\noindent\rule{\textwidth}{0.4pt}


\par (2)max\_binlog\_size

\par 该参数用于设置binlog日志的最大大小,默认为1GB,但是该值并不能严格控制binlog的大小。若binlog大小接近1GB,而此时又在执行一个较大的事务,那么为了保证事务的完整性,数据库不会做日志刷新动作,而是直到该事务的日志全部记录进入当前binlog日志后才会进行刷新。该参数的默认值查询结果为:

\noindent\rule{\textwidth}{0.4pt}


\begin{verbatim}mysql> show variables like '\%max\_binlog\_size\%';
+-----------------+------------+
| Variable\_name   | Value      |
+-----------------+------------+
| max\_binlog\_size | 1073741824 |
+-----------------+------------+
1 row in set (0.00 sec)
\end{verbatim}


\noindent\rule{\textwidth}{0.4pt}


\par (3)sync\_binlog

\par 这个参数的作用是控制binlog什么时候同步到磁盘。对数据库来说,这是很重要的参数,它不仅会影响数据库的性能,还会影响数据库数据的完整性。

\par 对于“sync\_binlog”参数的说明具体如下。

\par ·“sync\_binlog=0”表示在事务提交之后,数据库不会将binlog\_cache中的数据刷新到磁盘,而是让文件系统自行决定什么时候来做刷新或者在缓存满了之后才刷新到磁盘。

\par ·“sync\_binlog=n”表示每进行n次事务提交之后,数据库都会进行一次将缓存数据强制刷新到磁盘的操作。

\par 该参数默认的设置是0,示例如下:

\noindent\rule{\textwidth}{0.4pt}


\begin{verbatim}mysql> show variables like '\%sync\_binlog\%';
+---------------+-------+
| Variable\_name | Value |
+---------------+-------+
| sync\_binlog   | 0     |
+---------------+-------+
1 row in set (0.00 sec)
\end{verbatim}


\noindent\rule{\textwidth}{0.4pt}


\par 设置为0时数据库的性能是最好的,但数据风险也是最大的,对于数据安全性要求较高的数据库,应该调整该参数将其改为1,值得注意的是,即使参数设置为1,仍然有binlog记录的内容与数据库的实际内容不一致的风险。

\par 7.记录二进制日志的三种模式

\par MySQL使用不同的模式记录二进制日志信息,常见的有三种模式。

\par (1)语句模式

\par 语句(statement-based)模式是MySQL5.6版本默认的模式,简单地说,就是每一条被修改的数据的SQL语句都会记录到master的binlog中。在复制slave库的时候,SQL进程会解析成与原来master端执行过的相同的SQL来再次执行。

\par 该模式的优点是不需要记录细到每一行数据的更改变化,因此,可减少binlog日志量,实际上是减少了很多,节约了磁盘I/O,提高了系统性能。

\par 但该模式同样有一些缺点,由于语句模式记录的是执行的SQL语句,所以,对于某些具有特殊功能的SQL语句来说,就可能会导致无法在从库上正确执行,从而导致主从库数据不一致的问题。

\par 例如,当特殊的函数被执行时,当触发器、存储过程等特殊功能被执行时,而row level模式是基于每一行来记录变化的,所以不会出现类似的问题(更多详情请参考混合模式)。

\par (2)行级模式

\par 简单地说,行级(row-based)模式就是将数据被修改的每一行的情况记录为一条语句。

\par 优点:在行级模式下,binlog中可以不记录执行的SQL语句的上下文相关信息,仅仅记录哪一条记录被修改了,修改成什么样了即可,所以row level的日志内容会非常清楚地记录下每一行数据修改的细节,非常容易理解。而且不会出现某些特定情况下的存储过程或function以及trigger的调用和触发无法被正确复制的问题。

\par 缺点:行级模式下,所有的执行语句都将根据修改的行来记录,而这就可能会产生大量的日志内容,例如一条语句修改了100万行,语句模式就用一条语句即可搞定,而行级模式执行之后,日志中记录的就是100万行的修改记录,binlog日志的量可能会大得惊人。

\par (3)混合模式

\par 混合(mixed-based)模式默认采用语句模式记录日志,在一些特定的情况下会将记录模式切换为行级模式记录,这些特殊情况包含但不限于以下情况。

\par ·当函数中包含UUID()时。

\par ·当表中有自增列(AUTO\_INCREMENT)被更新时。

\par ·当执行触发器(trigger)或者存储过程(stored function)等特殊功能时。

\par ·当FOUND\_ROWS()、ROW\_COUNT()、USER()、CURRENT\_USER()、CURRENT\_USER等执行时。

\par 8.企业中如何选择二进制日志模式

\par 在互联网公司中,使用MySQL的特殊功能比较少(存储过程、触发器、函数),此时可以选择默认的语句模式。

\par 如果公司较多用到MySQL的特殊功能,如存储过程、触发器、函数等,并且需要做主从复制请首选行级模式,次选mixed模式。

\par 9.二进制日志的模式配置调整

\par 临时调整命令如下:

\noindent\rule{\textwidth}{0.4pt}


\begin{verbatim}mysql> SET GLOBAL binlog\_format = 'STATEMENT';
mysql> SET GLOBAL binlog\_format = 'ROW';
mysql> SET GLOBAL binlog\_format = 'MIXED';
\end{verbatim}


\noindent\rule{\textwidth}{0.4pt}


\par 永久调整可以将“binlog\_format='模式名'”写入到my.cnf配置文件中,并重启服务。

\par 10.二进制日志的读取和增量恢复

\par 此部分请参考本书的第7章7.3节,此处不再赘述。

\par 11.行级模式二进制日志实际读取示例

\par 当在数据库中执行如下语句:

\noindent\rule{\textwidth}{0.4pt}


\begin{verbatim}mysql> SET GLOBAL binlog\_format = 'ROW';
\end{verbatim}


\noindent\rule{\textwidth}{0.4pt}


\par 或者在my.cnf中加入binlog\_format='ROW'配置生效后,此时如果更新或者删除多行数据就会发现binlog日志记录的内容有所不同,具体命令如下:

\noindent\rule{\textwidth}{0.4pt}


\begin{verbatim}[root@oldboy 3306]\# mysqlbinlog --base64-output=decode-rows -v oldboy-bin.000001
\#<==--base64-output=decode-rows -v以行级模式解析binlog日志。
/*!50530 SET @@SESSION.PSEUDO\_SLAVE\_MODE=1*/;
\# at 233
\#180306  6:26:27 server id 1  end\_log\_pos 233   Table\_map: `oldboy`.`student` mapped to number 64
\#180306  6:26:27 server id 1  end\_log\_pos 500   Update\_rows: table id 64 flags: STMT\_END\_F
\#\#\# UPDATE `oldboy`.`student`
\#\#\# WHERE
\#\#\#   @1=1
\#\#\#   @2='oldboy1'
\#\#\# SET
\#\#\#   @1=1
\#\#\#   @2='test'
\#\#\# UPDATE `oldboy`.`student`
\#\#\# WHERE
\#\#\#   @1=3
\#\#\#   @2='oldboy2'
\#\#\# SET
\#\#\#   @1=3
\#\#\#   @2='test'
\#\#\# UPDATE `oldboy`.`student`
\#\#\# WHERE
\#\#\#   @1=5
\#\#\#   @2='oldboy3'
\#\#\# SET
\#\#\#   @1=5
\#\#\#   @2='test'
\#\#\# UPDATE `oldboy`.`student`
\#\#\# WHERE
\#\#\#   @1=6
\#\#\#   @2='oldgirl1'
\#\#\# SET
\#\#\#   @1=6
\#\#\#   @2='test'
\# at 500
\end{verbatim}


\noindent\rule{\textwidth}{0.4pt}



% From text00150.html


\section{10.5 慢查询日志}


\par 1.慢查询日志介绍

\par 简单地理解,慢查询日志(slow query log)就是记录执行时间超出指定值(long\_query\_time)或其他指定条件(例如,没有使用到索引,结果集大于1000行)的SQL语句。

\par 2.慢查询日志相关参数说明

\par 慢查询的参数,对于数据库SQL的优化非常重要,是SQL优化的前提,因此,这里以表的形式进行说明,具体见表10-2。

\par 表10-2 慢查询的参数及说明

\begin{figure}[htbp]
\centering
\includegraphics[width=0.8\textwidth]{Image00077.jpg}
\end{figure}



\par 3.慢查询日志重要参数配置

\par 企业中常见的配置慢查询的参数为:

\noindent\rule{\textwidth}{0.4pt}


\begin{verbatim}slow-query-log = ON                        \#<==慢查询开启开关
long\_query\_time = 2                        \#<==记录大于2秒的SQL语句。
log\_queries\_not\_using\_indexes = ON         \#<==没有使用到索引的SQL语句。
slow-query-log-file = /application/mysql/slow.log  \#<==记录SQL语句的文件。
min\_examined\_row\_limit = 800               \#<==记录结果集大于800行的SQL语句。
\end{verbatim}


\noindent\rule{\textwidth}{0.4pt}


\par 
\begin{figure}[htbp]
\centering
\includegraphics[width=0.8\textwidth]{Image00019.jpg}
\end{figure}

 注意:
 MySQL 5.6与以前的版本略微有变化,主要变化就是中杠“-”和下划线“\_”。

\par 可将上述参数配置到my.cnf里,配置完毕重启MySQL服务,并进行检查:

\noindent\rule{\textwidth}{0.4pt}


\begin{verbatim}mysql> show variables like 'slow\_query\%';
+---------------------+-----------------------------+
| Variable\_name       | Value                       |
+---------------------+-----------------------------+
| slow\_query\_log      | ON                          | \#<==开关已打开。
| slow\_query\_log\_file | /application/mysql/slow.log | \#<==文件路径已生效。
+---------------------+-----------------------------+
2 rows in set (0.00 sec)
mysql> show variables like '\%long\_query\%';
+-----------------+----------+
| Variable\_name   | Value    |
+-----------------+----------+
| long\_query\_time | 2.000000 | \#<==记录大于2秒的查询已生效。
+-----------------+----------+
1 row in set (0.01 sec)
mysql> show variables like '\%log\_queries\_not\%';
+-------------------------------+-------+
| Variable\_name                 | Value |
+-------------------------------+-------+
| log\_queries\_not\_using\_indexes | ON    | \#<==记录没有使用索引的查询已生效。
+-------------------------------+-------+
1 row in set (0.00 sec)

mysql> show variables like '\%min\_examined\_row\_limit\%';
+------------------------+-------+
| Variable\_name          | Value |
+------------------------+-------+
| min\_examined\_row\_limit | 800   | \#<==记录查询结果集大于800行的SQL已生效。
+------------------------+-------+
1 row in set (0.00 sec)
\end{verbatim}


\noindent\rule{\textwidth}{0.4pt}


\par 到此,就已经设定好记录慢查询SQL语句的条件了,那么,对于每天所产生的大量慢查询,又该如何处理和分析呢?

\par 4.慢查询日志的刷新方法

\par 在工作中,可以利用定时任务按天对慢查询日志进行切割,然后再分析。

\par 示例切割脚本如下:

\noindent\rule{\textwidth}{0.4pt}


\begin{verbatim}[root@oldboy data]\# mkdir /server/scripts/ -p
[root@oldboy data]\# cat /server/scripts/cut\_slow\_log.sh
\#!/bin/bash
\# Author: oldboy
\# Organization: www.oldboyedu.com
\# Created Time : 2018-03-19 23:47:21
export PATH=/application/mysql/bin:/sbin:/bin:/usr/sbin:/usr/bin
cd /application/mysql &&\textbackslash\{\}
mv slow.log slow.log.\$(date +\%F)&&\textbackslash\{\}
mysqladmin flush-log
\end{verbatim}


\noindent\rule{\textwidth}{0.4pt}


\par 将上述脚本放入定时任务,每天0点执行切割任务,配置结果如下:

\noindent\rule{\textwidth}{0.4pt}


\begin{verbatim}[root@oldboy data]\# tail -2 /var/spool/cron/root
\#cut mysql slow log by oldboy at 20180324
00 00 * * * /bin/sh /server/scripts/cut\_slow\_log.sh >/dev/null 2>&1
\end{verbatim}


\noindent\rule{\textwidth}{0.4pt}


\par 5.使用工具分析慢查询日志案例

\par 实际工作中,慢查询的日志可能非常多,给运维人员的优化工作带来了一定的困难,MySQL官方提供了慢查询的分析工具mysqldumpslow,有兴趣的读者可以参考官方手册的4.6.9节“mysqldumpslow”了解该工具的用法。

\par 下面为大家介绍一款很不错的第三方分析工具mysqlsla(需要单独安装该工具)。

\par (1)安装mysqlsla

\par 请提前下载好mysqlsla-2.03.tar.gz到指定目录下,然后执行如下命令安装:

\noindent\rule{\textwidth}{0.4pt}


\begin{verbatim}yum install  perl-devel perl-DBI perl-DBD-MySQL -y
rpm -qa perl-devel perl-DBI perl-DBD-MySQL
tar xf mysqlsla-2.03.tar.gz
cd mysqlsla-2.03
perl Makefile.PL
make
make install
\end{verbatim}


\noindent\rule{\textwidth}{0.4pt}


\par (2)利用mysqlsla工具分析慢查询

\par mysqlsla命令的默认路径为:/usr/local/bin/mysqlsla。

\par 简单语法如下:

\noindent\rule{\textwidth}{0.4pt}


\begin{verbatim}mysqlsla -lt slow [SlowLogFilePath] > [ResultFilePath]
\end{verbatim}


\noindent\rule{\textwidth}{0.4pt}


\par 在实际工作中,老男孩通常使用脚本调用mysqlsla工具进行分析,然后每天早晨8点,把分析结果发给企业的核心人员(DBA、运维总监、CTO、研发总监、核心开发),最后由DBA配合核心开发共同优化这些棘手的SQL慢查询。

\par 简单的案例脚本如下,注意切割日志和分析合并为一个脚本了:

\noindent\rule{\textwidth}{0.4pt}


\begin{verbatim}[root@oldboy mysqlsla-2.03]\# cat  /server/scripts/slow\_log\_analyze.sh
\#!/bin/bash
\#Author: oldboy
\#Organization: www.oldboyedu.com
\#Created Time : 2018-03-19 23:47:21
export PATH=/application/mysql/bin:/sbin:/bin:/usr/sbin:/usr/bin
Date=`date +\%F -d -1day`

\#cut slow log
cd /application/mysql &&\textbackslash\{\}
mv slow.log slow.log\_\$Date &&\textbackslash\{\}
mysqladmin flush-log

\#analyze slow log
Time=`date +\%F`
Path=/usr/local/bin/mysqlsla
cd /application/mysql &&\textbackslash\{\}
\$Path/mysqlsla -lt slow slow.log\_\$Date >analyzed\_slow\_\$Date.log 2>&1

\#rsync analyzed\_slow to backup server 省略了此步。
\#send analyzed slow log to adminstrator on backup server by mail. 省略了此步。
\end{verbatim}


\noindent\rule{\textwidth}{0.4pt}


\par 将上述脚本放入定时任务,每天0点执行切割任务,配置结果如下:

\noindent\rule{\textwidth}{0.4pt}


\begin{verbatim}[root@oldboy data]\# tail -2 /var/spool/cron/root
\#analyzed mysql slow log by oldboy at 20180324
00 00 * * * /bin/sh /server/scripts/slow\_log\_analyze.sh >/dev/null 2>&1
\end{verbatim}


\noindent\rule{\textwidth}{0.4pt}


\par (3)分析前后的日志结果对比

\par 比如原始慢日志中有一堆语句如下:

\noindent\rule{\textwidth}{0.4pt}


\begin{verbatim}\# Time: 110417  0:00:09
\# User@Host: oldboy[oldboy] @  [172.16.1.51]
\# Query\_time: 3  Lock\_time: 0  Rows\_sent: 1  Rows\_examined: 17600
select min(BBS\_HI\_ID) AS BBS\_HI\_ID from t\_******** where BBS\_HI\_ISTEAMMATE=1 and BBS\_HI\_EDITOR\_USER\_ID\_ENCRYPT='asdfEWERADFS';
\# User@Host: oldboy[oldboy] @  [172.16.1.51]
\# Query\_time: 4  Lock\_time: 0  Rows\_sent: 1  Rows\_examined: 17600
select min(BBS\_HI\_ID) AS BBS\_HI\_ID from t\_******** where BBS\_HI\_ISTEAMMATE=1 and BBS\_HI\_EDITOR\_USER\_ID\_ENCRYPT='asdfEWERADFS';
\# User@Host: bbs[oldboy] @  [172.16.1.52]
\# Query\_time: 4  Lock\_time: 0  Rows\_sent: 1  Rows\_examined: 17600
select min(BBS\_HI\_ID) AS BBS\_HI\_ID from t\_******** where BBS\_HI\_ISTEAMMATE=1 and BBS\_HI\_EDITOR\_USER\_ID\_ENCRYPT='NFCDebkasdf';
\# User@Host: oldboy[oldboy] @  [172.16.1.51]
\# Query\_time: 3  Lock\_time: 0  Rows\_sent: 1  Rows\_examined: 17600
select min(BBS\_HI\_ID) AS BBS\_HI\_ID from t\_******** where BBS\_HI\_ISTEAMMATE=1 and BBS\_HI\_EDITOR\_USER\_ID\_ENCRYPT='NFCDebkasdf';
\# User@Host: bbs[oldboy] @  [172.16.1.52]
\# Query\_time: 5  Lock\_time: 0  Rows\_sent: 1  Rows\_examined: 17600
select min(BBS\_HI\_ID) AS BBS\_HI\_ID from t\_******** where BBS\_HI\_ISTEAMMATE=1 and BBS\_HI\_EDITOR\_USER\_ID\_ENCRYPT='nfEACwQEW5MICAN2';
....................
....................
\end{verbatim}


\noindent\rule{\textwidth}{0.4pt}


\par 使用mysqlsla处理后,结果呈现为:

\noindent\rule{\textwidth}{0.4pt}


\begin{verbatim}Count         : 23  (8.52\%)
Time          : 102 s total, 4.434783 s avg, 3 s to 7 s max  (6.79\%)
  95\% of Time : 88 s total, 4.190476 s avg, 3 s to 6 s max
Lock Time (s) : 0 total, 0 avg, 0 to 0 max  (0.00\%)
  95\% of Lock : 0 total, 0 avg, 0 to 0 max
Rows sent     : 1 avg, 1 to 1 max  (0.02\%)
Rows examined : 11.53k avg, 5.70k to 17.60k max  (1.07\%)
Database      : bbsdb
Users         :
        oldboy@ 172.16.1.51 : 86.96\% (20) of query, 11.11\% (30) of all users
        bbs@ 172.16.1.52 : 13.04\% (3) of query, 2.96\% (8) of all users
Query abstract:
SELECT MIN(BBS\_HI\_id) AS BBS\_HI\_id FROM t\_******** WHERE BBS\_HI\_isteammate=N AND BBS\_HI\_editor\_user\_id\_encrypt='S';
Query sample:
select min(BBS\_HI\_ID) AS BBS\_HI\_ID from t\_******** where BBS\_HI\_ISTEAMMATE=1 and BBS\_HI\_EDITOR\_USER\_ID\_ENCRYPT='asdfEWERADFS';
\end{verbatim}


\noindent\rule{\textwidth}{0.4pt}


\par 在上述结果中,语句的执行情况(执行次数、对象信息、查询记录量、时间开销、来源统计)等信息一目了然,更方便运维人员以及DBA做进一步分析。

\par 除了mysqlsla和mysqldumpslow慢查询分析工具之外,还有一些第三方分析工具,如pt-query-diges、myprofi、mysql-explain-slow-log、mysqllogfilter等,读者若感兴趣可以深入研究。

\par 当然还可以把日志收集到数据库中或使用ELK等流行工具收集慢查询日志,最后分析后可视化展现,具体情况如图10-1所示。

\begin{figure}[htbp]
\centering
\includegraphics[width=0.8\textwidth]{Image00078.jpg}
\end{figure}



\par 图10-1 企业中的慢查询可视化展示图


% From text00151.html


\section{10.6 本章重点}


\par 1)MySQL的日志种类有哪些?

\par 2)MySQL普通查询日志的特点和配置方法?

\par 3)MySQL二进制日志的作用特点与配置?

\par 4)如何正确删除MySQL二进制日志?

\par 5)MySQL二进制日志的三种模式及特点?

\par 6)企业中如何选择MySQL二进制日志模式?

\par 7)MySQL二进制日志的模式配置调整?

\par 8)如何记录及分析MySQL慢查询日志?


% From text00152.html


\section{10.7 参考资料}


\par MySQL 5.6官方手册第5章“MySQL Server Logs”。


% From text00153.html


\chapter{第11章 MySQL数据库字符集}


\section{
11.1 MySQL数据库字符集知识}
}


\subsection{
11.1.1 什么是字符集}
}


\par 大家都知道,计算机只能识别0和1这样的二进制数字,无论是处理计算机程序,还是进行科学运算,最终都要转换为二进制数据来完成操作;例如,我们输入一个数字“8”,计算机会将其识别成二进制数字“1000”。

\par 但是,计算机要处理的数据不仅仅是数字,还会有字母,为了处理字母,就产生了ASCII码系统。英文字母共有26种变化,算上大小写也才52种变化,即使加上特殊的英文标点符号、特殊字符,变化也不多,而用8位二进制数字可以表达256种字符,也就是说,8位二进制数字就足以胜任英文字符的处理工作了。

\par 但是,各个国家的语言文字大多不同,不仅仅是数字、字母以及特殊字符。例如中国的汉字数量就有数万之多,常用的有几千个。这时,使用ASCII编码就会无法满足需求,于是就有了GBK、BIG5、GB2312这类的字符编码,采用16位二进制数可以表达65535个汉字,这对于常用的汉字使用来说就足够用了。

\par 现在,在简体中文环境下,常用的编码除了GB2312和GB18030之外,还会用到UTF-8。GBK是专门用作中文的字符编码规范,UTF是通用转换格式的缩写,又可称为万国码,理论上来说,UTF可以表达各种文字的编码格式。

\par 那么我们现在就应该明白了,字符编码其实就是将人类使用的汉字(或其他语言)、英文字母、特殊符号等信息,通过预先设定的转换规则,将其转换为计算机可以识别的二进制数字的一种编码方式。


% From text00154.html


\subsection{11.1.2 MySQL数据库字符集}


\par 通过上文对字符集的描述,我们可以得出字符集的简单知识,字符集其实就是一套文字符号及编码,对应的文字及编码,可以将人类可以识别的内容与计算机可以识别的信息进行互相转换。

\par 假设我们有一个字母表使用了四个字母:A、B、a、b。我们为每个字母赋予一个数值:A=0,B=1,a=2,b=3。字母A是一个符号,数字0是A的编码,这四个字母和它们的编码组合在一起就可以称为一个字符集。

\par MySQL数据库的字符集不仅包括字符集(CHARACTER),还包括校对规则(COLLATION)。其中,校对规则的作用是定义比较字符串的方式。

\par 假设我们希望比较两个字符串的值:A和B。最简单的方法是查找编码:A为0,B为1。因为0小于1,所以可以说A小于B。我们所做的仅仅是在我们的字符集上应用了一个校对规则。可见,校对规则是一套规则,作用是对编码进行比较。


% From text00155.html


\subsection{11.1.3 常用字符集介绍与选择建议}


\par 1.常用字符集介绍

\par 在操作系统以及各类软件中都有字符集,MySQL也不例外,下面通过表11-1介绍一下MySQL常用的字符集。

\par 表11-1 常用字符集知识

\begin{figure}[htbp]
\centering
\includegraphics[width=0.8\textwidth]{Image00079.jpg}
\end{figure}



\par 2.MySQL如何选择合适的字符集

\par ·如果存储的是各种各样的语言文字,则可以选择UTF8,这是目前国内应用最为广泛的字符集,没有之一。

\par ·如果只需要支持中文,并且数据量很大,此外,还包含了大量的运算,则可以选择GBK,理论上其可以获得更高的性能,但不推荐使用。

\par ·对于新型的互联网以及移动互联网的混合业务,推荐使用utf8mb4字符集替代UTF8字符集。总之,如果没有极特别的需求,请选择UTF8或utf8mb4作为数据库的字符集。

\par ·如果使用开源程序,则可以根据上述说明进行选择,如果是公司开发人员自己开发产品,那么选择权就在开发人员手里,DBA只能提供建议。

\par 3.查看MySQL数据库字符集和校对规则

\par 查看当前MySQL系统支持的字符集

\par MySQL数据库支持的字符集有很多种,通过以下命令可以查看当前MySQL支持的字符集:

\noindent\rule{\textwidth}{0.4pt}


\begin{verbatim}mysql> show character set;
+----------+-----------------------------+---------------------+--------+
| Charset  | Description                 | Default collation   | Maxlen |
+----------+-----------------------------+---------------------+--------+
| big5     | Big5 Traditional Chinese    | big5\_chinese\_ci     |      2 |
...省略若干...
| latin1   | cp1252 West European        | latin1\_swedish\_ci   |      1 |
...省略若干...
| gb2312   | GB2312 Simplified Chinese   | gb2312\_chinese\_ci   |      2 |
| greek    | ISO 8859-7 Greek            | greek\_general\_ci    |      1 |
| cp1250   | Windows Central European    | cp1250\_general\_ci   |      1 |
| gbk      | GBK Simplified Chinese      | gbk\_chinese\_ci      |      2 |
| latin5   | ISO 8859-9 Turkish          | latin5\_turkish\_ci   |      1 |
| armscii8 | ARMSCII-8 Armenian          | armscii8\_general\_ci |      1 |
| utf8     | UTF-8 Unicode               | utf8\_general\_ci     |      3 |
...省略若干...
| utf8mb4  | UTF-8 Unicode               | utf8mb4\_general\_ci  |      4 |
| cp1251   | Windows Cyrillic            | cp1251\_general\_ci   |      1 |
| utf16    | UTF-16 Unicode              | utf16\_general\_ci    |      4 |
...省略若干...
+----------+-----------------------------+---------------------+--------+
40 rows in set (0.00 sec)
\end{verbatim}


\noindent\rule{\textwidth}{0.4pt}


\par 可以看到,前面提到的最常用的字符集也在其中:

\noindent\rule{\textwidth}{0.4pt}


\begin{verbatim}+----------+-----------------------------+---------------------+--------+
| Charset  | Description                 | Default collation   | Maxlen |
+----------+-----------------------------+---------------------+--------+
| latin1   | cp1252 West European        | latin1\_swedish\_ci   |      1 |
| gb2312   | GB2312 Simplified Chinese   | gb2312\_chinese\_ci   |      2 |
| gbk      | GBK Simplified Chinese      | gbk\_chinese\_ci      |      2 |
| utf8     | UTF-8 Unicode               | utf8\_general\_ci     |      3 |
| utf8mb4  | UTF-8 Unicode               | utf8mb4\_general\_ci  |      4 |
+----------+-----------------------------+---------------------+--------+
\end{verbatim}


\noindent\rule{\textwidth}{0.4pt}


\par 
\begin{figure}[htbp]
\centering
\includegraphics[width=0.8\textwidth]{Image00019.jpg}
\end{figure}

 注意:
 每个字符集至少会有一个校对规则,也可能会有好几个校对规则,查看字符集对应校对规则的命令为:“mysql>SHOW COLLATION LIKE'utf8\%';”。


% From text00156.html


\section{11.2 MySQL数据库字符集配置}


\par 设置MySQL的字符集需要考虑到很多个层次,老男孩将这些需要考虑的层次大概分为以下7个级别。

\par ·操作系统级别。

\par ·操作系统客户端级别(SSH)。

\par ·MySQL实例级别。

\par ·数据库中的库级别。

\par ·表级别(含字段级别)。

\par ·MySQL客户端级别(连接及返回结果)。

\par ·程序代码级别。

\par 下面以Linux系统及UTF8字符集为例进行讲解。

\par 1.Linux系统服务端字符集设置

\par 很多读者在使用MySQL时经常会被中文乱码所困扰,其中Linux系统和连接Linux系统客户端的字符集设置可能就是问题之一,对此,要尽量将系统的字符集和系统中软件的字符集进行统一,设置和生效的方法具体如下:

\noindent\rule{\textwidth}{0.4pt}


\begin{verbatim}[root@oldboy \textasciitilde{}]\# cat /etc/sysconfig/i18n    \#<==配置到配置文件里可以永久生效。
LANG="zh\_CN.UTF-8"               \#<==LANG为系统字符集环境变量,设置为中文UTF8。
SYSFONT="latarcyrheb-sun16"
[root@oldboy \textasciitilde{}]\# source /etc/sysconfig/i18n \#<==使得修改生效。
[root@oldboy \textasciitilde{}]\# echo \$LANG                 \#<==检查生效情况。
zh\_CN.UTF-8
\end{verbatim}


\noindent\rule{\textwidth}{0.4pt}


\par 如果在Linux服务器里使用MySQL登录到数据库,则请注意系统字符集的使用。

\par 2.Linux系统客户端字符集设置

\par 常见的连接Linux系统的客户端为SecureCRT、XShell,下面我们就来了解一下如何设置字符集的配置。

\par SecureCRT的配置如图11-1所示。

\par XShell的配置如图11-2所示。

\par 3.MySQL服务端数据库实例字符集设置

\par 设置服务器的字符集有很多种方法,下面介绍其中常用的三种。

\begin{figure}[htbp]
\centering
\includegraphics[width=0.8\textwidth]{Image00080.jpg}
\end{figure}



\par 图11-1 SecureCRT的字符集配置

\begin{figure}[htbp]
\centering
\includegraphics[width=0.8\textwidth]{Image00081.jpg}
\end{figure}



\par 图11-2 XShell的字符集配置

\par 方法1:前文在编译安装MySQL的时候就曾指定过如下的服务器端字符集。

\noindent\rule{\textwidth}{0.4pt}


\begin{verbatim}cmake .
-DDEFAULT\_CHARSET=utf8 \textbackslash\{\}
-DDEFAULT\_COLLATION=utf8\_general\_ci \textbackslash\{\}
-DWITH\_EXTRA\_CHARSETS=all \textbackslash\{\}
\end{verbatim}


\noindent\rule{\textwidth}{0.4pt}


\par 方法2:如果编译时没指定字符集,或者指定了不合适的字符集,那么还可以在安装后修改配置文件。

\par 可按如下要求更改my.cnf参数:

\noindent\rule{\textwidth}{0.4pt}


\begin{verbatim}[mysqld]
character-set-server=utf8 \#<==适合5.5及以后的版本,5.1及以前版本的参数为default-
                            character-set。
\end{verbatim}


\noindent\rule{\textwidth}{0.4pt}


\par 方法3:还可以在启动数据库时,增加选项指定的字符集。

\noindent\rule{\textwidth}{0.4pt}


\begin{verbatim}mysqld --character-set-server=utf8 \#<==不推荐使用此法。
\end{verbatim}


\noindent\rule{\textwidth}{0.4pt}


\par 以上三种方法对数据库的影响具体体现在输出结果中有注释的参数行上,具体如下:

\noindent\rule{\textwidth}{0.4pt}


\begin{verbatim}mysql> show variables like 'character\_set\%';
+--------------------------+-------------------------------------------+
| Variable\_name            | Value                                     |
+--------------------------+-------------------------------------------+
| character\_set\_client     | utf8                                      |
| character\_set\_connection | utf8                                      |
| character\_set\_database   | utf8  \#<==MySQL中库的字符集。               |
| character\_set\_filesystem | binary                                    |
| character\_set\_results    | utf8                                      |
| character\_set\_server     | utf8   \#<==MySQL服务器实例字符集。           |
| character\_set\_system     | utf8                                      |
| character\_sets\_dir       | /application/mysql-5.6.40/share/charsets/ |
+--------------------------+-------------------------------------------+
8 rows in set (0.00 sec)
\end{verbatim}


\noindent\rule{\textwidth}{0.4pt}


\par 4.MySQL数据库中的库的字符集设置

\par 在MySQL中,库的字符集设置一般是在建库的时候指定的,如果在建库的时候未指定,则库的字符集与MySQL数据库实例的字符集一致。可通过如下命令查看当前实例的字符集:

\noindent\rule{\textwidth}{0.4pt}


\begin{verbatim}mysql> show variables like 'character\_set\_database\%';
+------------------------+-------+
| Variable\_name          | Value |
+------------------------+-------+
| character\_set\_database | utf8  |
+------------------------+-------+
1 row in set (0.00 sec)

mysql> show variables like 'collation\_database\%';
+--------------------+-----------------+
| Variable\_name      | Value           |
+--------------------+-----------------+
| collation\_database | utf8\_general\_ci |
+--------------------+-----------------+
1 row in set (0.00 sec)
\end{verbatim}


\noindent\rule{\textwidth}{0.4pt}


\par 下面创建数据库test,并查看建库的字符集:

\noindent\rule{\textwidth}{0.4pt}


\begin{verbatim}mysql> create database tingting;
Query OK, 1 row affected (0.00 sec)
mysql> show create database tingting\textbackslash\{\}G
*************************** 1. row ***************************
       Database: tingting
Create Database: CREATE DATABASE `tingting` /*!40100 DEFAULT CHARACTER SET utf8 */
1 row in set (0.00 sec)
\end{verbatim}


\noindent\rule{\textwidth}{0.4pt}


\par 在编译MySQL时,若指定了正确的字符集或者修改配置文件调整过的服务器的字符集,那么,在以后建库的时候就可以直接执行简化的命令“create database tinting;”。

\par 也可以在建库的时候指定字符集和校对规则来建库,命令如下:

\noindent\rule{\textwidth}{0.4pt}


\begin{verbatim}create database oldboy DEFAULT CHARACTER SET UTF8 DEFAULT COLLATE = utf8\_general\_ci;
\end{verbatim}


\noindent\rule{\textwidth}{0.4pt}


\par 上面的语句为查看已建立的oldboy库的语句,其中,“CHARACTER SET UTF8”即为数据库字符集,而“utf8\_general\_ci”则为校对规则。

\par 
\begin{figure}[htbp]
\centering
\includegraphics[width=0.8\textwidth]{Image00019.jpg}
\end{figure}

 注意:
 采用二进制方式安装MySQL时,若没有指定字符集,则此时字符集默认为latin1,此时需要建立设置UTF8字符集的库,即需要指定UTF8字符集建库,否则就必须要提前修改配置文件,将服务器字符集调整为所需要的字符集。

\par 5.MySQL数据库表的字符集设置

\par 默认情况下,建表的字符集与库的字符集应一致,设置表字符集的命令具体如下:

\noindent\rule{\textwidth}{0.4pt}


\begin{verbatim}mysql> use tingting
Database changed
mysql> create table test(id int(4));
Query OK, 0 rows affected (0.03 sec)

mysql> show create table test\textbackslash\{\}G
*************************** 1. row ***************************
       Table: test
Create Table: CREATE TABLE `test` (
  `id` int(4) DEFAULT NULL
) ENGINE=InnoDB DEFAULT CHARSET=utf8  \#<==就是这里,如果在建表时需要调整字符集,
                                       则必须指定字符集。
1 row in set (0.00 sec)
\end{verbatim}


\noindent\rule{\textwidth}{0.4pt}


\par 6.MySQL数据库客户端字符集设置

\par 对MySQL数据库客户端字符集进行设置,对于防止MySQL更新时,出现中文乱码有极大的影响,设置方法也有几种,具体如下。

\par 方法1:临时生效单条命令法。

\noindent\rule{\textwidth}{0.4pt}


\begin{verbatim}mysql> set names utf8;
Query OK, 0 rows affected (0.00 sec)
\end{verbatim}


\noindent\rule{\textwidth}{0.4pt}


\par 其中,“set names utf8”也可以用下面三个命令来替代。

\noindent\rule{\textwidth}{0.4pt}


\begin{verbatim}SET character\_set\_client = utf8;
SET character\_set\_results = utf8;
SET character\_set\_connection = utf8;
\end{verbatim}


\noindent\rule{\textwidth}{0.4pt}


\par 方法2:登录数据库时指定字符集。

\noindent\rule{\textwidth}{0.4pt}


\begin{verbatim}mysql --default-character-set=utf8;
\end{verbatim}


\noindent\rule{\textwidth}{0.4pt}


\par 方法3:通过修改my.cnf实现修改MySQL客户端的字符集,配置方法如下。

\noindent\rule{\textwidth}{0.4pt}


\begin{verbatim}[client]
default-character-set=utf8
\end{verbatim}


\noindent\rule{\textwidth}{0.4pt}


\par 
\begin{figure}[htbp]
\centering
\includegraphics[width=0.8\textwidth]{Image00019.jpg}
\end{figure}

 注意:
 多实例MySQL场景下,这里的字符集修改要统一修改/etc/my.cnf。

\par 以上几种方法对数据库的影响具体体现在输出结果中有注释的参数行上,具体如下:

\noindent\rule{\textwidth}{0.4pt}


\begin{verbatim}mysql> show variables like 'character\_set\%';
+--------------------------+-------------------------------------------+
| Variable\_name            | Value                                     |
+--------------------------+-------------------------------------------+
| character\_set\_client     | utf8  \#<==MySQL客户端字符集。               |
| character\_set\_connection | utf8  \#<==MySQL连接的字符集。               |
| character\_set\_database   | utf8              |
| character\_set\_filesystem | binary                                    |
| character\_set\_results    | utf8  \#<==MySQL返回结果的字符集。            |
| character\_set\_server     | utf8          |
| character\_set\_system     | utf8                                      |
| character\_sets\_dir       | /application/mysql-5.6.40/share/charsets/ |
+--------------------------+-------------------------------------------+
8 rows in set (0.00 sec)
\end{verbatim}


\noindent\rule{\textwidth}{0.4pt}


\par 人工登录数据库执行“set names UTF8”,以及使用MySQL命令指定字符集登录操作,或者更改my.cnf配置文件客户端模块的参数,来实现更改客户端字符集,都是改变了MySQL客户端的client、connection、results 3个参数的字符集。

\par 7.程序代码字符集

\par 很多开源软件都会给出多种字符集的软件代码,例如,bbs软件就是如此。Discuz!X3.2在继承和完善Discuz!X3.1的基础上,针对社区移动端进行了新的尝试,比如,推出微信登录、微社区等功能。它作为安全稳定的程序可为站长提供更加可靠的保障。

\par 以下是Discuz!X3.2的下载地址。简体中文GBK的下载地址为:

\par 
\href{http://download.comsenz.com/DiscuzX/3.2/Discuz_X3.2_SC_GBK.zip}{http://download.comsenz.com/DiscuzX/3.2/Discuz\_X3.2\_SC\_GBK.zip}


\par 简体UTF8的下载地址为:

\par 
\href{http://download.comsenz.com/DiscuzX/3.2/Discuz_X3.2_SC_UTF8.zip}{http://download.comsenz.com/DiscuzX/3.2/Discuz\_X3.2\_SC\_UTF8.zip}


\par 如果是公司自己开发的产品,则需要咨询开发人员如何搭建符合程序的字符集环境。


% From text00157.html


\section{11.3 如何防止数据库的中文显示乱码}


\par 管理员在配置MySQL数据库字符集时,需要尽可能地确保11.2节开头提到的7大项字符集统一,对于管理员来说,查看数据库字符集的基本方法具体如下:

\noindent\rule{\textwidth}{0.4pt}


\begin{verbatim}mysql> show variables like 'character\_set\%';
Variable\_name            | Value
-------------------------+-----------------------------------------------
character\_set\_client     | utf8    \#客户端字符集;
character\_set\_connection | utf8    \#客户端连接字符集;
character\_set\_database   | utf8    \#数据库字符集,配置文件时指定或建库建表时指定;
character\_set\_filesystem | binary  \#文件系统字符集;
character\_set\_results    | utf8    \#客户端返回结果字符集;
character\_set\_server     | utf8    \#服务器字符集,配置文件时指定或建库建表时指定;
character\_set\_system     | utf8    \#系统字符集。
character\_sets\_dir       | /application/mysql-5.5.32/share/charsets/ |
\end{verbatim}


\noindent\rule{\textwidth}{0.4pt}


\par 更改Linux系统字符集变量之后,可以查看MySQL中字符集的变化。

\subsubsection{彻底防止MySQL数据库内的数据中文乱码方法}


\par 切记,字符集的不一致是数据库乱码的罪魁祸首,要想避免MySQL数据库内的数据中文乱码方法,就要遵循前文所说的7大项字符集设置规则,即Linux系统服务端与Linux系统客户端字符集、MySQL服务端数据库实例与MySQL数据库客户端字符集、MySQL数据库中的库和表的字符集、程序代码的字符集要一致。如果是利用文件还原数据,还要注意文件的编码问题。

\par 下面是一个不按要求来操作,导致出现乱码的范例:

\noindent\rule{\textwidth}{0.4pt}


\begin{verbatim}mysql> set names gbk;                      \#<==修改MySQL客户端字符集为GBK。
Query OK, 0 rows affected (0.00 sec)

mysql> use tingting
Database changed

mysql> create table t1(id int(4),name varchar(16));  \#<==默认建表。
Query OK, 0 rows affected (0.01 sec)

mysql> insert into t1 values(1,'老男');              \#<==插入中文。
Query OK, 1 row affected (0.00 sec)
mysql> select * from t1;
+------+--------+
| id   | name   |
+------+--------+
|    1 | 老男   |  \#<==正常。
+------+--------+
1 row in set (0.00 sec)

mysql> set names utf8; \#<==修改MySQL客户端字符集为UTF8。
Query OK, 0 rows affected (0.00 sec)

mysql> select * from t1;
+------+-----------+
| id   | name      |
+------+-----------+
|    1 | 鑰佺敺    | \#<==发现数据乱码了,这就是客户端字符集和服务端不一致时插入数据而
                       导致的乱码问题。
+------+-----------+
1 row in set (0.00 sec)
\end{verbatim}


\noindent\rule{\textwidth}{0.4pt}



% From text00158.html


\section{11.4 如何更改MySQL数据库库表的字符集}


\subsection{
11.4.1 更改库的字符集}
}


\par 管理员可以使用alter命令对数据库的字符集进行更改,命令如下:

\noindent\rule{\textwidth}{0.4pt}


\begin{verbatim}mysql> show create database oldboy\textbackslash\{\}G
*************************** 1. row ***************************
       Database: oldboy
Create Database: CREATE DATABASE `oldboy` /*!40100 DEFAULT CHARACTER SET utf8 */
1 row in set (0.00 sec)

mysql> alter database oldboy CHARACTER SET latin1 COLLATE = latin1\_swedish\_ci;
Query OK, 1 row affected (0.00 sec)

mysql> show create database oldboy\textbackslash\{\}G
*************************** 1. row ***************************
       Database: oldboy
Create Database: CREATE DATABASE `oldboy` /*!40100 DEFAULT CHARACTER SET latin1 */
1 row in set (0.00 sec)

mysql> alter database oldboy CHARACTER SET utf8 collate utf8\_general\_ci;
Query OK, 1 row affected (0.00 sec)

mysql> show create database oldboy\textbackslash\{\}G
*************************** 1. row ***************************
       Database: oldboy
Create Database: CREATE DATABASE `oldboy` /*!40100 DEFAULT CHARACTER SET utf8 */
1 row in set (0.00 sec)
\end{verbatim}


\noindent\rule{\textwidth}{0.4pt}



% From text00159.html


\subsection{11.4.2 更改表的字符集}


\par 管理员也可以使用alter命令对数据库的表的字符集进行更改,命令如下:

\noindent\rule{\textwidth}{0.4pt}


\begin{verbatim}mysql> use tingting
Database changed
mysql> show create table t1\textbackslash\{\}G
*************************** 1. row ***************************
       Table: t1
Create Table: CREATE TABLE `t1` (
  `id` int(4) DEFAULT NULL,
  `name` varchar(32) DEFAULT NULL
) ENGINE=InnoDB DEFAULT CHARSET=utf8
1 row in set (0.00 sec)

mysql> alter table t1 CHARACTER SET latin1;
Query OK, 0 rows affected (0.01 sec)
Records: 0  Duplicates: 0  Warnings: 0

mysql> show create table t1\textbackslash\{\}G
*************************** 1. row ***************************
       Table: t1
Create Table: CREATE TABLE `t1` (
  `id` int(4) DEFAULT NULL,
  `name` varchar(32) CHARACTER SET utf8 DEFAULT NULL
) ENGINE=InnoDB DEFAULT CHARSET=latin1
1 row in set (0.00 sec)
\end{verbatim}


\noindent\rule{\textwidth}{0.4pt}


\par 上面仅仅是修改字符集的基本操作,若是对于已经上线的数据库,并且已经包含了很多数据,就无法使用这个方法了,使用“alter database dbname character set*”修改库的字符集,或者使用“alter table tablename character set*”更改表的字符集,只对新创建的表或者更新的数据生效。

\par 那么如何对已经包含了数据的库表的字符集进行调整呢?见下文。


% From text00160.html


\subsection{11.4.3 生产环境更改数据库(含数据)字符集的方法}


\par 对于已经包含了数据的库表,若要对字符集进行调整,就需要将数据先导出,然后更改数据库环境,更改建库和表的字符集之后,重新导入数据,这样才能实现相应的调整。整个过程具体如下。

\par 1)确保数据库不要更新,然后导出所有数据为SQL的文件。

\par 2)针对导出的数据进行字符集替换(替换表和库),例如把GBK改为UTF8。

\par 3)修改my.cnf配置文件,更改MySQL客户端及服务端的字符集,重启生效。

\par 4)导入更改过新字符集的库表的数据,包括表结构语句,然后提供服务。

\par 5)将操作系统、SSH客户端,以及程序更改为对应的新字符集。

\par 另外,更改字符集时,要将小的字符集集合更改为大的字符集集合,不然可能会丢失数据。整个操作步骤比较简单,这里就不演示给读者了。


% From text00161.html


\section{11.5 本章重点}


\par 1)什么是字符集?

\par 2)MySQL如何选择合适的字符集?

\par 3)如何更改MySQL服务端数据库实例字符集设置?

\par 4)如何指定MySQL数据库中库的字符集进行建库?

\par 5)如何指定MySQL数据库表的字符集建表?

\par 6)如何设置MySQL数据库客户端字符集?

\par 7)如何防止数据库的中文显示乱码?

\par 8)如何更改MySQL数据库库表的字符集?

\par 9)生产环境下更改数据库(含数据)字符集的方法?


% From text00162.html


\chapter{第12章 MySQL数据库存储引擎知识}


\section{
12.1 MySQL引擎概述}
}


\subsection{
12.1.1 什么是存储引擎?}
}


\par 在讲解什么是存储引擎之前,我们先来个比喻,我们都知道录制的视频文件,可以转换成不同的格式,如mp4、avi、wmv等,而存储在电脑的磁盘上也会存在于不同类型的文件系统中,如Windows里常见的ntfs、fat32,存在于Linux里常见的ext3、ext4、xfs等,但是,用户看到的实际视频内容都是一样的。它们之间的直观区别是占用系统的空间大小与清晰程度可能不一样。

\par 数据库表里的数据存储在数据库里及磁盘上,与上述的视频格式及存储磁盘文件系统格式的特征类似,也有很多种存储方式。

\par 但是,对于用户和应用程序来说,同样一张表的数据,无论采用什么引擎来存储,用户看到的数据都是一样的。对于不同的引擎存取,引擎功能、占用的空间大小、读取性能等可能都有区别。

\par 通过上面的说明,我们给出存储引擎的官方解释,存储引擎是MySQL数据库用来处理不同表类型的SQL操作的组件。

\par MySQL早期最常用的存储引擎为:MyISAM和InnoDB。目前,InnoDB是最常用的存储引擎,也是MySQL5.6默认的存储引擎。


% From text00163.html


\subsection{12.1.2 MySQL存储引擎的架构}


\par MySQL的存储引擎是MySQL数据库的重要组成部分。MySQL的每种存储引擎在MySQL里都是通过插件的方式使用的,可以轻易地从MySQL中进行加载和卸载,MySQL中可以同时支持多种存储引擎。图12-1是MySQL存储引擎体系结构的简图。

\begin{figure}[htbp]
\centering
\includegraphics[width=0.8\textwidth]{Image00082.jpg}
\end{figure}



\par 图12-1 MySQL存储引擎体系结构图

\par 从图12-1b中,我们可以清晰地看见MySQL体系结构的组成部分,具体如下。

\par ·连接池部分。

\par ·数据库管理部分。

\par ·SQL接口、查询分析器、优化器、缓存缓冲。

\par ·存储引擎部分(像插座的部分)。

\par ·数据库数据文件和各种日志文件。

\par ·文件系统磁盘。

\par 本书主要关注的是第四部分,像插座一样的存储引擎。


% From text00164.html


\section{12.2 查看MySQL支持的存储引擎}


\par 可以在MySQL中使用显示引擎的命令来得到一个可用引擎的列表:

\noindent\rule{\textwidth}{0.4pt}


\begin{verbatim}mysql> select version();
+------------+
| version()  |
+------------+
| 5.6.34-log |
+------------+
1 row in set (0.00 sec)
mysql> show engines;
+--------------------+---------+-----------------------------------------+
| Engine             | Support | Comment
                     | Transactions | XA   | Savepoints |
+--------------------+---------+-----------------------------------------+
| MRG\_MYISAM         | YES     | Collection ofidentical MyISAM tables
                     | NO           | NO   | NO         |
| CSV                | YES     | CSV storage engine
                     | NO           | NO   | NO         |
| MEMORY             | YES     | Hash based, stored in memory, useful for temporary tables     | NO           | NO   | NO         |
| BLACKHOLE          | YES     | /dev/null storage engine (anything you write to it disappears)    | NO           | NO   | NO         |
| MyISAM             | YES     | MyISAM storage engine
                     | NO           | NO   | NO         |
| PERFORMANCE\_SCHEMA | YES     | Performance Schema
                     | NO           | NO   | NO         |
| ARCHIVE            | YES     | Archive storage engine
                     | NO           | NO   | NO         |
| InnoDB             | DEFAULT | Supports transactions, row-level locking, and foreign keys     | YES          | YES  | YES        |
| FEDERATED          | NO      | Federated MySQL storage engine
                     | NULL         | NULL | NULL       |
+--------------------+---------+-----------------------------------------+
9 rows in set (0.00 sec)
\end{verbatim}


\noindent\rule{\textwidth}{0.4pt}


\par 以上命令的结果显示了数据库可用引擎的全部名单,以及在当前的数据库中是否支持这些引擎,其中前四列比较重要,第一列是引擎名字,第二列是当前数据库是否支持,第三列是描述,第四列表示是否支持事务。


% From text00165.html


\section{12.3 MySQL 5.6支持的存储引擎}


\par 下面将MySQL 5.6支持的存储引擎用表格的方式进行说明,具体见表12-1。

\par 表12-1 MySQL5.6支持的存储引擎

\begin{figure}[htbp]
\centering
\includegraphics[width=0.8\textwidth]{Image00083.jpg}
\end{figure}



\par 读者只需要知道前三个,掌握第一个InnoDB,了解第二个MyISAM和第三个Memory即可,本章也将着重介绍这三个引擎。


% From text00166.html


\section{12.4 MySQL常用存储引擎特性对比}


\par MySQL的存储引擎有很多,不同引擎之间的区别具体有哪些呢?请参考表12-2 MySQL常用存储引擎特性对比。

\par 表12-2 MySQL常用存储引擎特性对比

\begin{figure}[htbp]
\centering
\includegraphics[width=0.8\textwidth]{Image00084.jpg}
\end{figure}



\par 
\begin{figure}[htbp]
\centering
\includegraphics[width=0.8\textwidth]{Image00006.jpg}
\end{figure}

 提示:
 NDB引擎是MySQL集群软件支持的,5.6社区版不支持NDB引擎。


% From text00167.html


\section{12.5 设置与更改MySQL的引擎}


\par 1.设置表的引擎

\par 如果建表的时候不指定引擎,那么表的引擎就会和数据库的默认配置一致。下面就来讲解下如何指定表的引擎建立表,例如建立一个学生表,代码如下:

\noindent\rule{\textwidth}{0.4pt}


\begin{verbatim}CREATE TABLE `student` (
  `Sno` int(10) NOT NULL COMMENT '学号',
  `Sname` varchar(16) NOT NULL COMMENT '姓名',
  `Ssex` char(2) NOT NULL COMMENT '性别',
  `Sage` varchar(16) default NULL,
  `Sdept` varchar(16) default NULL COMMENT '学生所在系别',
  KEY `ind\_sage` (`Sage`),
  KEY `ind\_sno` (`Sno`)
) ENGINE=MyISAM DEFAULT CHARSET=utf8 \#<==最后一行括号外,指定引擎。
\end{verbatim}


\noindent\rule{\textwidth}{0.4pt}


\par 2.更改表的引擎

\par 一般来说,更改MySQL引擎的需求并不多见,但偶尔也会有,下面就向大家介绍几种修改方法。

\par 方法1:利用SQL命令语句修改引擎,具体命令如下。

\noindent\rule{\textwidth}{0.4pt}


\begin{verbatim}ALTER TABLE oldboy ENGINE = INNODB;
ALTER TABLE oldboy ENGINE = MyISAM;
\end{verbatim}


\noindent\rule{\textwidth}{0.4pt}


\par 更改引擎实例:

\noindent\rule{\textwidth}{0.4pt}


\begin{verbatim}mysql> show create table test\textbackslash\{\}G
*************************** 1. row ***************************
       Table: test
Create Table: CREATE TABLE `test` (
  `id` int(4) NOT NULL AUTO\_INCREMENT,
  `name` char(20) NOT NULL,
  PRIMARY KEY (`id`),
  KEY `ind\_name` (`name`(8))
) ENGINE=InnoDB AUTO\_INCREMENT=10 DEFAULT CHARSET=utf8 \#<==默认是InnoDB。
1 row in set (0.00 sec)

mysql> ALTER TABLE test ENGINE=MyISAM;      \#<==将test表的引擎更改为MyISAM。
Query OK, 5 rows affected (0.02 sec)
Records: 5  Duplicates: 0  Warnings: 0

mysql> show create table test\textbackslash\{\}G
*************************** 1. row ***************************
       Table: test
Create Table: CREATE TABLE `test` (
  `id` int(4) NOT NULL AUTO\_INCREMENT,
  `name` char(20) NOT NULL,
  PRIMARY KEY (`id`),
  KEY `ind\_name` (`name`(8))
) ENGINE=MyISAM AUTO\_INCREMENT=10 DEFAULT CHARSET=utf8 \#<==已改为MyISAM。
1 row in set (0.00 sec)
特别提示:更改MySQL引擎之后也需要相关配置参数的支持,否则,效果可能不佳。
\end{verbatim}


\noindent\rule{\textwidth}{0.4pt}


\par 使用此方法若要批量修改,则需要通过开发脚本实现,与分表分库脚本差不多。

\par 方法2:使用sed对备份的SQL文件进行批量转换。

\par 使用sed对备份内容进行引擎转换:

\noindent\rule{\textwidth}{0.4pt}


\begin{verbatim}nohup sed -e 's/MyISAM/InnoDB/g' oldboy.sql > oldboy\_1.sql &
\end{verbatim}


\noindent\rule{\textwidth}{0.4pt}


\par 方法3:mysql\_convert\_table\_format命令修改。

\noindent\rule{\textwidth}{0.4pt}


\begin{verbatim}mysql\_convert\_table\_format  --user=root --password=oldboy123 --socket=/data/3306/mysql.sock --engine=MyISAM oldboy test
\end{verbatim}


\noindent\rule{\textwidth}{0.4pt}


\par 
\begin{figure}[htbp]
\centering
\includegraphics[width=0.8\textwidth]{Image00006.jpg}
\end{figure}

 提示:
 该命令需要一些依赖包,安装方法为:

\noindent\rule{\textwidth}{0.4pt}


\begin{verbatim}yum -y install perl-DBI perl-DBD-MySQL perl-Time-HiRes。
\end{verbatim}


\noindent\rule{\textwidth}{0.4pt}



% From text00168.html


\section{12.6 MyISAM引擎}


\subsection{
12.6.1 什么是MyISAM引擎?}
}


\par MyISAM引擎是MySQL关系型数据库管理系统的默认存储引擎(MySQL 5.5.5以前)。这种MySQL表存储结构可从旧的ISAM代码中扩展出许多有用的功能。在新版本的MySQL中,InnoDB引擎由于支持事务、外键等,有利于数据的一致性,以及其能支持更高的多用户并发性等优点,InnoDB已经取代了曾经常用的MyISAM引擎,不过由于数据库中的MySQL库的大部分表主要用于读取,因此,MyISAM引擎依然在使用。


% From text00169.html


\subsection{12.6.2 MyISAM引擎的存储方式}


\par 每一个MyISAM引擎的表都对应于硬盘上的三个文件。这三个文件虽然具有一样的文件名,但是其不同的扩展名指示了其不同的类型用途:“.frm”文件用于保存表的定义,该文件并不是MyISAM引擎的一部分,而是服务器的一部分;“.MYD”用于保存表的数据;“.MYI”则是表的索引文件。“.MYD”和.MYI是MyISAM的关键点。示例代码如下:

\noindent\rule{\textwidth}{0.4pt}


\begin{verbatim}[root@oldboy \textasciitilde{}]\# cd /application/mysql/data
[root@oldboy data]\# ls -l mysql/|head -10
总用量 1688
-rw-r-----. 1 mysql mysql  8820 3月 22 10:01 columns\_priv.frm \#<==存放表的定义。
-rw-r-----. 1 mysql mysql     0 3月 22 10:01 columns\_priv.MYD \#<==存放表的数据。
-rw-r-----. 1 mysql mysql  4096 3月 22 10:01 columns\_priv.MYI \#<==存放表的索引。
-rw-r-----. 1 mysql mysql  9582 3月 22 10:01 db.frm
-rw-r-----. 1 mysql mysql  1760 3月 22 10:01 db.MYD
-rw-r-----. 1 mysql mysql  5120 3月 22 10:01 db.MYI
-rw-r-----. 1 mysql mysql 10223 3月 22 10:01 event.frm
-rw-r-----. 1 mysql mysql     0 3月 22 10:01 event.MYD
-rw-r-----. 1 mysql mysql  2048 3月 22 10:01 event.MYI
\end{verbatim}


\noindent\rule{\textwidth}{0.4pt}


\par MySQL数据库系统的表大多数都使用MyISAM引擎。

\par 这里先向读者抛出两个小疑问,请读者带着疑问看看下面的知识。

\par 1)为什么MySQL5.5.5以前默认是MyISAM引擎,而MySQL5.5.5以后默认是InnoDB?

\par 2)为什么MySQL库里的表仍然默认为MyISAM引擎?


% From text00170.html


\subsection{12.6.3 MyISAM引擎的主要特点}


\par MyISAM引擎的主要特点一直以来都是面试官设置的考点,读者需要重点掌握,MyISAM引擎的特点具体请参考表12-3。

\par 表12-3 MyISAM引擎的主要特点

\begin{figure}[htbp]
\centering
\includegraphics[width=0.8\textwidth]{Image00085.jpg}
\end{figure}




% From text00171.html


\subsection{12.6.4 MyISAM引擎适用的生产业务场景}


\par 下面为大家介绍下,MyISAM引擎可以适用的生产业务场景。

\par 1)不需要事务支持并且对数据一致性要求不高的业务(例如转账就不行)。

\par 2)一般适用于读请求较多的应用,读写都频繁的场景不适合
 。

\par 3)读写并发访问相对较低的业务。

\par 4)数据修改相对较少的业务(阻塞问题)。

\par 5)硬件资源比较差的服务器。

\par 6)使用读写分离的MySQL从库可以使用MyISAM(早期有人这么玩过)。

\par 
\begin{figure}[htbp]
\centering
\includegraphics[width=0.8\textwidth]{Image00019.jpg}
\end{figure}

 注意:
 当下99\%的企业业务场景,都不需要使用MyISAM了,而是选择更有优势的InnoDB。


% From text00172.html


\section{12.7 InnoDB引擎}


\subsection{
12.7.1 什么是InnoDB引擎?}
}


\par InnoDB引擎是当下MySQL数据库最重要的存储引擎,其正在成为目前MySQL AB所发行新版的标准,被包含在所有的安装包里。与其他的存储引擎相比,InnoDB引擎的优点是更新数据行级锁定、支持ACID的事务、支持外键,它的设计目标是面向在线事务处理的应用,目前绝大多数互联网公司都在使用InnoDB引擎,该引擎替代了其他的引擎。MySQL 5.6版本的默认引擎已变为InnoDB引擎。


% From text00173.html


\subsection{12.7.2 InnoDB引擎的存储方式}


\par InnoDB存储引擎将数据存放在一个像黑盒一样的逻辑表空间中,这个表空间分为共享表空间和独立表空间,从MySQL 5.6开始,即默认支持将InnoDB引擎的表数据单独存放到各自独立的ibd文件中(独立表空间)。

\par 
范例12-1
 :创建表并查看表空间情况

\noindent\rule{\textwidth}{0.4pt}


\begin{verbatim}mysql> select version();
+------------+
| version()  |
+------------+
| 5.6.34-log |
+------------+
1 row in set (0.00 sec)
mysql> use oldboy
Database changed
mysql> create table test1(id int);
Query OK, 0 rows affected (0.04 sec)
mysql> show create table test1\textbackslash\{\}G
*************************** 1. row ***************************
       Table: test1
Create Table: CREATE TABLE `test1` (
  `id` int(11) DEFAULT NULL
) ENGINE=InnoDB DEFAULT CHARSET=utf8
1 row in set (0.00 sec)
\end{verbatim}


\noindent\rule{\textwidth}{0.4pt}


\par 
\begin{figure}[htbp]
\centering
\includegraphics[width=0.8\textwidth]{Image00006.jpg}
\end{figure}

 提示:
 由输出结果可以看到,只有test1.frm,没有MyISAM对应的数据文件和索引文件了,这是为什么呢?

\noindent\rule{\textwidth}{0.4pt}


\begin{verbatim}[root@oldboy \textasciitilde{}]\# cd /application/mysql/data
[root@oldboy data]\# ls -l
总用量 196660
-rw-r-----. 1 mysql mysql 79691776 3月  26 05:01 ibdata1
\#<==这里是共享表空间存放InnoDB数据的文件,MySQL 5.6以前默认情况下所有的InnoDB数据
       都存放在这个文件里。
-rw-r-----. 1 mysql mysql 50331648 3月  26 05:01 ib\_logfile0
-rw-r-----. 1 mysql mysql 50331648 3月  22 10:01 ib\_logfile1
-rw-r-----. 1 mysql mysql 12582912 3月  22 10:26 ibtmp1
drwxr-x---. 2 mysql mysql     4096 3月  22 10:01 mysql
drwxr-x---. 2 mysql mysql     4096 3月  26 05:01 oldboy
-rw-rw----. 1 mysql mysql        6 3月  22 13:10 oldboy.pid
drwxr-x---. 2 mysql mysql     4096 3月  22 10:01 oldboy\_utf8
drwx------. 2 mysql mysql     4096 3月  22 12:30 oldgirl
drwxr-x---. 2 mysql mysql     4096 3月  22 10:01 performance\_schema
drwx------. 2 mysql mysql     4096 3月  24 04:02 tingting
[root@oldboy data]\# ls -l oldboy \#<==查看oldboy数据库里表的情况。
总用量 256
-rw-r-----. 1 mysql mysql    61 3月  24 04:00 db.opt
-rw-rw----. 1 mysql mysql  8644 3月  24 21:21 student.frm
-rw-rw----. 1 mysql mysql     0 3月  24 21:21 student.MYD
-rw-rw----. 1 mysql mysql  1024 3月  24 21:21 student.MYI
-rw-rw----. 1 mysql mysql  8556 3月  26 05:01 test1.frm
\#<==存放InnoDB引擎test1.表的定义。
-rw-rw----. 1 mysql mysql 98304 3月  26 05:01 test1.ibd
\#<==这里是独立表空间,test1表数据文件。
-rw-rw----. 1 mysql mysql  8586 3月  24 21:22 test.frm
\#<==这里存放的是MyISAM引擎test表的定义。
-rw-rw----. 1 mysql mysql   325 3月  24 21:22 test.MYD
\#<==这里存放的是MyISAM引擎test表的数据。
-rw-rw----. 1 mysql mysql  3072 3月  24 21:22 test.MYI
\#<==这里存放的是MyISAM引擎test表的索引。
-rw-rw----. 1 mysql mysql  8556 3月  22 11:30 t.frm
-rw-rw----. 1 mysql mysql 98304 3月  22 11:31 t.ibd
\end{verbatim}


\noindent\rule{\textwidth}{0.4pt}


\par 控制InnoDB存储引擎数据文件的参数具体如下:

\noindent\rule{\textwidth}{0.4pt}


\begin{verbatim}mysql> show variables like 'innodb\%';
+-----------------------------+------------------------+
| Variable\_name               | Value                  |
+-----------------------------+------------------------+
| innodb\_autoextend\_increment | 64                     | \#<==数据文件自增参数。
| innodb\_data\_file\_path       | ibdata1:12M:autoextend | \#<==数据文件路径。
| innodb\_data\_home\_dir        |                        | \#<==数据家目录。
| innodb\_file\_per\_table       | ON                     | \#<==独立表空间开关。
...省略若干...
+-----------------------------+------------------------+
120 rows in set (0.00 sec) \#<==InnoDB相关参数共120个。
\end{verbatim}


\noindent\rule{\textwidth}{0.4pt}


\par 
\begin{figure}[htbp]
\centering
\includegraphics[width=0.8\textwidth]{Image00024.jpg}
\end{figure}

 说明:
 默认情况下数据库会初始化一个12MB的ibdata1共享表空间文件,并且以64MB为单位自动增长,innodb\_file\_per\_table控制是否开启独立表空间格式,开启之后,所有的表的数据都会以单独的数据文件形式存在。


% From text00174.html


\subsection{12.7.3 InnoDB引擎特点}


\par 一直以来,InnoDB引擎都是数据库的重中之重,它的主要特点一直以来更是面试官设置的考点,读者需要重点掌握,InnoDB引擎的主要特点具体如表12-4。

\par 表12-4 InnoDB引擎的主要特点

\begin{figure}[htbp]
\centering
\includegraphics[width=0.8\textwidth]{Image00086.jpg}
\end{figure}




% From text00175.html


\subsection{12.7.4 InnoDB引擎适用的生产业务场景}


\par 1)需要事务支持的业务(具有很好的事务特性)。

\par 2)行级锁定对高并发有很好的适应能力,但需要确保查询是通过索引来完成的。

\par 3)数据读写及更新都较为频繁的场景,如BBS、SNS、微博、微信等。

\par 4)数据一致性要求较高的业务,例如:充值转账、银行卡转账等。

\par 5)硬件设备资源较好,特别是内存要大,可以利用InnoDB较好的缓存能力来提高内存利用率,尽可能减少磁盘IO。


% From text00176.html


\subsection{12.7.5 InnoDB引擎相关参数介绍}


\par 从前面的说明可以看到,InnoDB引擎相关参数多达120个,但是在实际工作中,我们需要修改的参数却是非常少的,下面就为大家介绍下InnoDB引擎的重要参数,更多内容读者可以参考官方手册学习或者通过“mysql>show variables like'innodb\%';”查看。

\par 表12-5 InnoDB引擎的重要参数

\begin{figure}[htbp]
\centering
\includegraphics[width=0.8\textwidth]{Image00087.jpg}
\end{figure}




% From text00177.html


\subsection{12.7.6 InnoDB引擎调优的基本方法}


\par 1)主键应尽可能小,以避免对Secondary index带来过大的空间负担。

\par 2)建立有效索引避免全表扫描,因为会使用表锁。

\par 3)尽可能缓存所有的索引和数据,提高响应速度,减少磁盘IO消耗。

\par 4)在进行大批量小插入的时候,应尽量自己控制事务而不要使用autocommit自动提交。若有开关则可以控制提交方式。

\par 5)合理设置innodb\_flush\_log\_at\_trx\_commit参数值,不要过度追求安全性。

\par 6)应避免主键更新,因为这会带来大量的数据移动。


% From text00178.html


\section{12.8 Memory存储引擎}


\par Memory就是内存的意思,因此Memory存储引擎(又称为heap引擎)的数据存储是放在内存(注意:由max\_heap\_table\_size参数控制内存占用大小,默认为16MB。)中的,因此存取速度特别快,但是如果数据库宕机或重启,那么所有的数据就都会丢失,因此它比较适合用于存放临时表的数据,例如,discuz论坛数据库中的统计在线人数的session表采用的就是Memory引擎。Memory存储引擎默认采用的是Hash索引,而不像其他引擎(MyISAM和InnoDB)默认的是B-tree索引。

\par Memory存储引擎在使用上也有一些限制,例如,仅支持表锁,不支持TEXT和BlOB数据类型,还有当存储变长字段(varchar)时是按照定长字段(char)来进行的,这也会浪费一些内存空间。Memory存储引擎在企业工作中应用的不是很多,读者作为知识点了解一下即可。


% From text00179.html


\section{12.9 ARCHIVE存储引擎}


\par ARCHIVE的中文意思是归档,因此ARCHIVE适用于存放大量归档历史数据(可查询但不能删除)的保存。

\par ARCHIVE引擎仅支持select、insert操作;MySQL 5.1以后开始支持索引等操作。

\par ARCHIVE引擎使用zlib无损数据压缩算法,压缩比可达10:1,可大量节省磁盘空间,设计ARCHIVE引擎的目标是提供高速的插入和压缩等功能。

\par 如下范例所示,建立两个不同存储引擎的表,测试ARCHIVE存储引擎的表其占用空间的情况.

\par 首先建立一个MyISAM存储引擎的表,插入数据:

\noindent\rule{\textwidth}{0.4pt}


\begin{verbatim}mysql> create table t1 engine=myisam as select * from information\_schema.columns;
Query OK, 1683 rows affected (0.05 sec)
Records: 1683  Duplicates: 0  Warnings: 0
mysql> show table status like 't1'\textbackslash\{\}G
*************************** 1. row ***************************
           Name: t1
         Engine: MyISAM
        Version: 10
     Row\_format: Dynamic
           Rows: 1683
 Avg\_row\_length: 148
    Data\_length: 250716  \#<==总长度250716。
Max\_data\_length: 281474976710655
...省略若干...
1 row in set (0.00 sec)
\end{verbatim}


\noindent\rule{\textwidth}{0.4pt}


\par 再建立一个ARCHIVE引擎表,插入数据:

\noindent\rule{\textwidth}{0.4pt}


\begin{verbatim}mysql> create table t2 engine=archive as select * from information\_schema.columns;
Query OK, 1703 rows affected (0.03 sec)
Records: 1703  Duplicates: 0  Warnings: 0

mysql> show table status like 't2'\textbackslash\{\}G
*************************** 1. row ***************************
           Name: t2
         Engine: ARCHIVE
        Version: 10
     Row\_format: Compressed \#<==压缩了。
           Rows: 1703
 Avg\_row\_length: 21
    Data\_length: 36679      \#<==总长度36679。
...省略若干...
1 row in set (0.01 sec)
\end{verbatim}


\noindent\rule{\textwidth}{0.4pt}


\par 数据文件形式如下:

\noindent\rule{\textwidth}{0.4pt}


\begin{verbatim}[root@oldboy data]\# ll /application/mysql/data/oldboy/t2*
-rw-rw----. 1 mysql mysql 36679 3月  26 06:58 /application/mysql/data/oldboy/t2.ARZ
-rw-rw----. 1 mysql mysql 13658 3月  26 06:58 /application/mysql/data/oldboy/t2.frm
提示:.ARZ是数据压缩文件,.frm是表结构定义文件
\end{verbatim}


\noindent\rule{\textwidth}{0.4pt}


\par 其他种类引擎的说明请读者回看12.3节“MySQL 5.6支持的存储引擎”中表12-1的说明。

\par 实际工作中使用最多的就是InnoDB引擎,其次是MyISAM和NDB存储引擎,其他的例如Memory、Merge、CSV、Federate等存储引擎的使用场景都相对较少,本章也不进行过多讲解。


% From text00180.html


\section{12.10 NDB存储引擎}


\par NDB存储引擎是一个集群存储引擎,类似于Oracle的RAC集群,但它是Share Nothing的架构,因此NDB能够提供更高级别的高可用性和可扩展性。NDB的特点是数据全部存放在内存中,因此,通过主键进行查找的速度非常快。

\par 关于NDB,有一个问题需要注意,它的连接(join)操作是在MySQL数据库层完成的,而不是在存储引擎层完成的,这就意味着,复杂的join操作需要巨大的网络开销,查询速度会很慢,在中小型企业中,NDB引擎的使用频率极少,读者可以直接忽略,重点关注InnoDB引擎知识即可。


% From text00181.html


\section{12.11 有关MySQL引擎常见的企业面试题}


\par 1)MySQL有哪些存储引擎,各自有什么特点和区别?

\par 2)生产环境中应如何选用MySQL的存储引擎?

\par 3)不同的引擎应如何备份?混合引擎应如何备份?


% From text00182.html


\chapter{第13章 MySQL引擎之InnoDB}


\section{
13.1 InnoDB存储引擎介绍}
}


\par MySQL从5.5版本开始即将InnoDB作为默认存储引擎,该存储引擎是第一个完整支持事务ACID特性的存储引擎,且支持数据行锁、多版本并发控制(MVCC)、外键,以及一致性非锁定读。

\par InnoDB作为MySQL的默认存储引擎,这就意味着默认创建的表都会使用此存储引擎,除非使用“ENGINE=参数”指定创建其他存储引擎的表。

\par InnoDB的关键属性具体如下。

\par 1)ACID事务特性支持,包括commit、rollback以及crash恢复的能力。

\par 2)行级别锁以及多版本并发控制。

\par 3)利用主键的聚簇索引(clustered index)在底层存储数据,以提升对主键查询的IO性能。

\par 4)支持外键功能,管理数据的完整性。

\par 在MySQL实例中执行“show engines”命令查看存储引擎的情况。

\par “Support=YES”代表当前支持的存储引擎,“DEFAULT”代表默认的存储引擎,具体如下:

\noindent\rule{\textwidth}{0.4pt}


\begin{verbatim}mysql> show engines;
+--------------------+---------+-----------------------------------------+
| Engine             | Support | Comment                                                        | Transactions | XA   | Savepoints |
+--------------------+---------+-----------------------------------------+
| MyISAM             | YES     | MyISAM storage engine                                          | NO           | NO   | NO         |
| CSV                | YES     | CSV storage engine                                             | NO           | NO   | NO         |
| PERFORMANCE\_SCHEMA | YES     | Performance Schema                                             | NO           | NO   | NO         |
| BLACKHOLE          | YES     | /dev/null storage engine (anything you write to it disappears) | NO           | NO   | NO         |
| MRG\_MYISAM         | YES     | Collection ofidentical MyISAM tables                          | NO           | NO   | NO         |
| InnoDB             | DEFAULT | Supports transactions, row-level locking, and foreign keys     | YES          | YES  | YES        |
| ARCHIVE            | YES     | Archive storage engine                                         | NO           | NO   | NO         |
| MEMORY             | YES     | Hash based, stored in memory, useful for temporary tables      | NO           | NO   | NO         |
| FEDERATED          | NO      | Federated MySQL storage engine                                 | NULL         | NULL | NULL       |
+--------------------+---------+-----------------------------------------+
\end{verbatim}


\noindent\rule{\textwidth}{0.4pt}



% From text00183.html


\section{13.2 InnoDB和ACID模型}


\par 事务(Transaction)是数据库区别于文件系统的重要特性之一,事务可由一条非常简单的SQL语句组成,也可以由一组复杂的SQL语句组成。事务中的操作,要么都做修改,要么都不做,这就是事务的基本目的。理论上说,事务有着极其严格的定义,它必须同时满足四个特性,即通常所说的事务的ACID特性。

\par ACID模型是关系型数据库普遍支持的事务模型,用于保证数据的一致性,其中的ACID所代表的具体含义分别如下。

\par 1)A:atomicity原子性。事务是一个不可再分割的工作单位,事务中的操作要么都发生,要么都不发生。

\par 2)C:consistency一致性。事务开始之前和事务结束以后,数据库的完整性约束没有被破坏。也就是说,数据库事务不能破坏关系数据的完整性以及业务逻辑上的一致性。

\par 3)I:isolation独立性。多个事务并发访问时,事务之间是隔离的,一个事务不应该影响其他事务的运行效果。

\par 4)D:durability持续性。在事务完成以后,该事务对数据库所做的更改便持久地保存在数据库之中,并不会被回滚。

\par 举例来说,比如银行的汇款1000元的操作,简单来说,可以拆分成A账户的余额-1000,B账户的余额+1000,还要分别在A和B的账户流水上记录余额变更日志,这四个操作必须放在一个事务中完成,否则丢失其中的任何一条记录对整个系统来说都是不完整的。

\par 对于上述例子来说,原子性体现在要么四条操作每条都成功,这就意味着汇款成功;要么其中某一个操作失败,则整个事务中的四条操作都回滚,即汇款失败。一致性表示当汇款结束时,A账户和B账户里的余额变化和操作日志记录是可以对应起来的。独立性表示在汇款操作过程中,如果有C账户也在往B账户里汇款的话,那么两个事务之间相互不会影响,即“A->B”有四个独立操作,“C->B”也有四个独立操作。持久性表示当汇款成功时,A和B的余额就变更了,不管是数据库重启还是别的什么原因,该数据已经写入到磁盘中作为永久存储,不会再发生变化,除非有新的事务发生。

\par 其中事务的隔离性是通过MySQL锁机制来实现的,原子性、一致性、持久性则是通过MySQL的redo和undo日志记录来完成的

\par 
显式事务启动|结束


\par 1)以start transaction/begin开始事务。

\par 2)以commit/rollback transaction结束事务。

\par 
隐形事务提交


\par 主要是DDL、DCL会引发事务隐形提交,比如create、alter语句以及grant、revoke等语句。


% From text00184.html


\section{13.3 InnoDB多版本控制MVCC}


\par 我们知道对较为繁忙的MySQL系统来说,任何时刻都很少会出现只有一个活跃的数据库链接在操作数据库,通常都是多个并行活跃的数据库链接在同时执行操作,而且也极有可能会出现在对某个表或某些行进行并行读写操作,那么为了保证操作能够并行执行且支持对事务的回滚操作,InnoDB会将修改前的数据存放在回滚段中。

\par 多版本控制在InnoDB的默认隔离级别下,是指当有事务正在修改一些数据而未提交时,并行的读操作依然能够读到这部分数据而不需要等待事务提交,而读到的数据则是写事务开始之前的数据。InnoDB数据存储结构如图13-1所示。

\par 如图13-2所示,InnoDB会在数据库的每一行上额外增加三个字段以实现多版本控制,第一个字段是DB\_TRX\_ID,用来存放针对该行最后一次执行insert、update操作的事务ID,而delete操作也会被认为是update,只是会有额外的一位来代表事务为删除操作;第二个字段是DB\_ROLL\_PTR指针,其指向回滚段里对应的undo日志记录;第三个字段是DB\_ROW\_ID,代表每一行的行ID。

\begin{figure}[htbp]
\centering
\includegraphics[width=0.8\textwidth]{Image00088.jpg}
\end{figure}



\par 图13-1 InnoDB数据存储结构

\begin{figure}[htbp]
\centering
\includegraphics[width=0.8\textwidth]{Image00089.jpg}
\end{figure}



\par 图13-2 InnoDB行数据结构

\par 1)初始数据行的情况,六个字段的值分别是1、2、3、4、5、6。

\par 2)事务1修改该数据行,将6个字段的值分别*10,并生成回滚日志记录,如图13-3所示。

\begin{figure}[htbp]
\centering
\includegraphics[width=0.8\textwidth]{Image00090.jpg}
\end{figure}



\par 图13-3 InnoDB行数据与回滚段关系

\par 3)事务2读取该数据行。

\par 事务2按照自己的事务ID与行数据中的事务ID做对比,并按照事务隔离级别选取事务1修改前的回滚段中的数据进行返回。

\noindent\rule{\textwidth}{0.4pt}


\begin{verbatim}在两个数据库链接下实验默认隔离级别下的多版本控制
链接1:mysql> start transaction;
链接2:mysql> start transaction;
链接1:mysql> update score set score=88 where sid=1;
链接2:mysql> select * from score where sid=1;     \#\#\#链接1锁数据未释放,链接2也能访问修改前的数据
+------+-----------+-------+
| sid  | course\_id | score |
+------+-----------+-------+
|    1 |         1 |    90 |
|    1 |         2 |    90 |
|    1 |         3 |    90 |
|    1 |         4 |    90 |
链接1: mysql>commit;
链接2:mysql> select * from score where sid=1;     \#\#\#链接1锁释放,但链接2访问到的数据依然是之前的数据(这是由InnoDB的默认事务隔离级别决定的,可重复读的特性)
+------+-----------+-------+
| sid  | course\_id | score |
+------+-----------+-------+
|    1 |         1 |    90 |
|    1 |         2 |    90 |
|    1 |         3 |    90 |
|    1 |         4 |    90 |
链接2:mysql> commit;
链接2:mysql> select * from score where sid=1;     \#\#\#链接2提交之后,再访问到的数据就是修改后的数据了
+------+-----------+-------+
| sid  | course\_id | score |
+------+-----------+-------+
|    1 |         1 |    88 |
|    1 |         2 |    88 |
|    1 |         3 |    88 |
|    1 |         4 |    88 |
\end{verbatim}


\noindent\rule{\textwidth}{0.4pt}


\par 回滚段中的undo日志记录分为插入undo日志和update日志,且只有在事务commit提交之后才会被丢弃,为避免回滚段越来越大,需要注意应及时执行commit命令。


% From text00185.html


\section{13.4 InnoDB体系结构}


\par InnoDB存储引擎的体系架构如图13-4所示。

\par InnoDB存储引擎作为MySQL体系中的核心部件,其是由三部分组成的,包括各种缓存池、后台线程和底层的数据文件,其中的各种缓存池的作用具体如下。

\par 1)维护所有后台线程需要访问的内部数据结构。

\par 2)缓存磁盘上的数据,方便快速读取,同时在对磁盘文件进行数据修改之前也在这里缓存。

\par 3)重做日志缓存,等等。

\begin{figure}[htbp]
\centering
\includegraphics[width=0.8\textwidth]{Image00091.jpg}
\end{figure}



\par 图13-4 InnoDB存储引擎体系结构

\par 下面我们就来一一讲解。


% From text00186.html


\subsection{13.4.1 缓存池(buffer pool)}


\par buffer pool(缓存池)是InnoDB在内存中开辟的用来缓存表数据和索引数据的区域,一般可以设置为50\%~80\%的内存大小,通过将经常访问的数据放置到内存当中来加快访问速度。InnoDB buffer pool的组成如图13-5所示。

\par InnoDB数据的读写都需要经过缓存(缓存在buffer pool即内存中),数据以整页(16KB)为单位读取到缓存中。缓存中的数据以LRU策略换出(最少使用策略),IO效率高、性能好。如图13-6所示的是InnoDB buffer pool的作用示意图。

\begin{figure}[htbp]
\centering
\includegraphics[width=0.8\textwidth]{Image00092.jpg}
\end{figure}



\par 图13-5 InnoDB buffer pool组成

\begin{figure}[htbp]
\centering
\includegraphics[width=0.8\textwidth]{Image00093.jpg}
\end{figure}



\par 图13-6 InnoDB buffer pool作用示意图

\par buffer pool是内存中用来缓存数据和索引的存储区域,也是MySQL性能调优的重要一环。

\par 理想情况下,设置的size越大,则缓存到内存中的数据越多,InnoDB就越像是内存数据库。

\par buffer pool的底层是一个列表,通过LRU算法进行数据页的换进换出操作。当空间原因导致新页的加入需要换出一页时,InnoDB取出最近最少使用的页并将这个新的数据页加入到列表的中央。从方向上看,列表的头部是最常使用的数据页,而在尾部则是最少使用的数据页。

\par buffer pool中3/8的部分是用于保存最少使用的数据页,而中央部分其实是经常使用和最少使用的结合点。当在最少使用中保存的数据页被访问时,数据页就会被移动到列表的头部变成最常使用的。

\par 1.配置大小

\par InnoDB buffer pool的大小既可以在启动时配置,也可以在启动之后配置。

\par 增加和减少buffer pool的大小都是以大块的方式,块的大小由参数innodb\_buffer\_pool\_chunk\_size决定,默认为128MB。

\par Innodb\_buffer\_pool\_size的大小可以自行设定,但必须是innodb\_buffer\_pool\_chunk\_size*innodb\_buffer\_pool\_instances的整数倍,如果不是,则buffer pool会被调整成大于设定值且最接近的一个值,如下示例代码中,虽然buffer pool被设置为9GB,但实际生效的依然是10GB:

\noindent\rule{\textwidth}{0.4pt}


\begin{verbatim}shell> mysqld --innodb\_buffer\_pool\_size=9G --innodb\_buffer\_pool\_instances=16

mysql> SELECT @@innodb\_buffer\_pool\_size/1024/1024/1024;
+------------------------------------------+
| @@innodb\_buffer\_pool\_size/1024/1024/1024 |
+------------------------------------------+
|                          10.000000000000 |
+------------------------------------------+
\end{verbatim}


\noindent\rule{\textwidth}{0.4pt}


\par Innodb\_buffer\_pool\_chunk\_size可以自行设定,且增加和减少都要以MB为单位,并且只能在启动前修改,修改后的值*innodb\_buffer\_pool\_instances不能大于buffer pool的大小,否则修改无效:

\noindent\rule{\textwidth}{0.4pt}


\begin{verbatim}[mysqld]
innodb\_buffer\_pool\_chunk\_size=134217728
\end{verbatim}


\noindent\rule{\textwidth}{0.4pt}


\par 可以动态修改buffer pool的大小,用set语句直接修改,当语句发起时,会一直等到当前所有的事务都结束后才执行,而且一旦执行,则执行过程中的其他事务如果要访问buffer pool,就必须等待语句执行完毕:

\noindent\rule{\textwidth}{0.4pt}


\begin{verbatim}mysql> SET GLOBAL innodb\_buffer\_pool\_size=402653184;
\end{verbatim}


\noindent\rule{\textwidth}{0.4pt}


\par 当执行online的调整大小时,可以通过error log或者innodb\_buffer\_pool\_resize\_status查看进度:

\noindent\rule{\textwidth}{0.4pt}


\begin{verbatim}mysql> SHOW STATUS WHERE Variable\_name='InnoDB\_buffer\_pool\_resize\_status';
+----------------------------------+----------------------------------+
| Variable\_name                    | Value                            |
+----------------------------------+----------------------------------+
| Innodb\_buffer\_pool\_resize\_status | Resizing also other hash tables. |
+----------------------------------+----------------------------------+
\end{verbatim}


\noindent\rule{\textwidth}{0.4pt}


\par 2.配置多个buffer pool实例

\par 当buffer pool的大小是GB级别时,将一个buffer pool分割成几个独立的实例能够降低多个线程同时读写缓存页的竞争性而提高并发性。设置innodb\_buffer\_pool\_instances的参数可以调整实例个数。如果有多个实例,则缓存的数据页会随机放置到任意的实例中,且每个实例都有独立的buffer pool所有的特性。

\par Innodb\_buffer\_pool\_instances的默认值是1,最大可以调整成64。

\par 3.buffer pool内数据页控制

\par 新读取的数据页插入到buffer pool的LRU列表的中间位置,默认位置是从尾部开始算起的3/8的位置。当放入到buffer pool的页被第一次访问时其就开始往列表的前方移动,这样,列表的后部就是不经常访问的页甚至是从不访问的页。

\par 通过参数innodb\_old\_blocks\_pct可以控制列表中“old”数据页所占的百分比,默认是37\%,等同于3/8,取值范围是5~95。

\par Innodb\_old\_blocks\_time参数默认是1000毫秒,指定了页面读取到buffer pool后,但没有移动到经常受访问列表位置的时间窗口。

\par 4.InnoDB buffer pool预存取

\par read ahead是异步预先获取多个数据页到buffer pool的IO操作,这些数据页都是假定随后会被用到的。InnoDB可通过如下两种read-ahead算法提高IO性能。

\par 1)线性read ahead:预测哪些页会被顺序访问。通过innodb\_read\_ahead\_threshold参数调整顺序数据页的数量。当从一个区中顺序读取的页数量大于等于innodb\_read\_ahead\_threshold时,InnoDB会触发异步read ahead操作将整个区都读到buffer pool中。该参数的默认值是56,取值范围是0~64。

\par 2)随机read ahead:通过已经在buffer pool中的数据页来预测哪些页会被随后访问到。如果13个连续的处于相同区的页存在于buffer pool中,则InnoDB会把同一个区的其他页都读取进来。可通过设置innodb\_random\_read\_ahead=ON来开启此方式。

\par 可通过执行show engine innodb status命令显示的三个参数判断read-ahead算法的有效性,三个参数具体如下。

\par ·Innodb\_buffer\_pool\_read\_ahead

\par ·Innodb\_buffer\_pool\_read\_ahead\_evicted

\par ·Innodb\_buffer\_pool\_read\_ahead\_rnd

\par 5.InnoDB buffer pool磁盘持久化

\par InnoDB会在后台将buffer pool中的脏页(已经修改但没有写到数据文件)flush掉。当buffer pool中的脏页所占百分比达到innodb\_max\_dirty\_pages\_pct\_lvm时会触发flush操作,当所占比例达到innodb\_max\_dirty\_pages\_pct时,InnoDB会“强烈”地flush。

\par 针对数据修改操作频繁的系统,flush可能会严重滞后,从而导致大量的buffer pool内存占用,有一些参数专门针对修改繁忙的系统可以进行调整,具体如下。

\par ·Innodb\_adaptive\_flushing\_lwm:为防止redo log被填满,此参数设置了一个阈值,如果redo log的容量超过此阈值,则执行adaptive flush操作。

\par ·Innodb\_max\_drity\_pages\_pct\_lwm

\par ·Innodb\_io\_capacity\_max

\par ·Innodb\_flushing\_avg\_loops

\par 6.重置buffer pool状态

\par InnoDB可以通过配置innodb\_buffer\_pool\_dump\_at\_shutdown参数来确保在服务器重启时部分经常使用的数据页能够直接加载到buffer pool中,通过批量加载的方式,来节省重启服务器所导致的warmup时间(原先在buffer pool中的数据页要从磁盘中再次加载到内存中)。

\par buffer pool的状态可以在任意时刻被保存,而重置状态也可以恢复任意保存的副本。

\par 在数据库运行期间,动态配置buffer pool数据页保留占比的方式如下:

\noindent\rule{\textwidth}{0.4pt}


\begin{verbatim}SET GLOBAL innodb\_buffer\_pool\_dump\_pct=40;
\end{verbatim}


\noindent\rule{\textwidth}{0.4pt}


\par 而在配置文件中的配置方法为:

\noindent\rule{\textwidth}{0.4pt}


\begin{verbatim}[mysqld]
innodb\_buffer\_pool\_dump\_pct=40
\end{verbatim}


\noindent\rule{\textwidth}{0.4pt}


\par 当服务器关闭时,配置保存buffer pool的当前状态的方法如下:

\noindent\rule{\textwidth}{0.4pt}


\begin{verbatim}SET GLOBAL innodb\_buffer\_pool\_dump\_at\_shutdown=ON;
\end{verbatim}


\noindent\rule{\textwidth}{0.4pt}


\par 当服务器开启时,重新加载buffer pool的方法如下:

\noindent\rule{\textwidth}{0.4pt}


\begin{verbatim}mysqld --innodb\_buffer\_pool\_load\_at\_startup=ON;
\end{verbatim}


\noindent\rule{\textwidth}{0.4pt}


\par 默认情况下innodb\_buffer\_pool\_dump\_at\_shutdown和innodb\_buffer\_pool\_load\_at\_startup两个配置均是开启状态。

\par 数据库运行期间保存和重新加载buffer pool的方法如下:

\noindent\rule{\textwidth}{0.4pt}


\begin{verbatim}SET GLOBAL innodb\_buffer\_pool\_dump\_now=ON;
SET GLOBAL innodb\_buffer\_pool\_load\_now=ON;
\end{verbatim}


\noindent\rule{\textwidth}{0.4pt}


\par 查看buffer pool保存和重新加载的进度的方法如下:

\noindent\rule{\textwidth}{0.4pt}


\begin{verbatim}SHOW STATUS LIKE 'Innodb\_buffer\_pool\_dump\_status';
SHOW STATUS LIKE 'Innodb\_buffer\_pool\_load\_status';
\end{verbatim}


\noindent\rule{\textwidth}{0.4pt}


\par 7.监控buffer pool的状态情况

\par 通过show engine innodb status命令可以查看buffer pool的运行情况,代码如下:

\noindent\rule{\textwidth}{0.4pt}


\begin{verbatim}----------------------
BUFFER POOL AND MEMORY
----------------------
Total large memory allocated 2198863872
Dictionary memory allocated 776332
Buffer pool size   131072
Free buffers       124908
Database pages     5720
Old database pages 2071
Modified db pages  910
Pending reads 0
Pending writes: LRU 0, flush list 0, single page 0
Pages made young 4, not young 0
0.10 youngs/s, 0.00 non-youngs/s
Pages read 197, created 5523, written 5060
0.00 reads/s, 190.89 creates/s, 244.94 writes/s
Buffer pool hit rate 1000 / 1000, young-making rate 0 / 1000 not
0 / 1000
Pages read ahead 0.00/s, evicted without access 0.00/s, Random read
ahead 0.00/s
LRU len: 5720, unzip\_LRU len: 0
I/O sum[0]:cur[0], unzip sum[0]:cur[0]
\end{verbatim}


\noindent\rule{\textwidth}{0.4pt}


\par 表13-1 InnoDB buffer pool运行状态重点参数解释

\begin{figure}[htbp]
\centering
\includegraphics[width=0.8\textwidth]{Image00094.jpg}
\end{figure}




% From text00187.html


\subsection{13.4.2 change pool缓存池}


\par change buffering是MySQL5.5中加入的新特性,change buffering是insert buffer的加强。insert buffer只针对insert有效,由于对非聚簇索引叶子节点的插入往往不像主键插入一样是按顺序的,而是随机读写的,从而导致了插入操作性能的下降,所以为了提高性能,InnoDB存储引擎对非聚簇索引的插入或更新操作并不是每次都直接插入到索引页,而是先判断插入的索引页是否在缓冲池中,如果在则直接插入,如果不在则先将插入信息放到change buffer中,然后再以一定的频率和情况对buffer和索引叶子节点进行合并操作,这时通常能将多个插入合并到一个操作中,这也就大大提高了插入的性能

\par change buffering对insert、delete、update(delete+insert)、purge都有效。同样的,当修改一个索引块(secondary index)的数据时,索引块在buffter pool中不存在,修改信息就会被cache在change buffer中,当通过索引扫描把需要的索引块读取到buffer pool中时,其会与change buffer中的修改信息合并,再择机写回disk。

\par Innodb\_change\_buffering参数缓存所对应的操作(update会被认为是delete+insert)具体如下。

\par ·all:默认值,缓存insert、delete、purges操作。

\par ·none:不缓存。

\par ·inserts:缓存insert操作。

\par ·deletes:缓存delete操作。

\par ·changes:缓存insert和delete操作。

\par ·purges:缓存后台执行的物理删除操作。

\par 一般情况下,当辅助索引页被读取到缓冲池中时,例如在执行正常的select查询操作时,这时就会确认索引页是否在change buffer的B+树中,若有,则将结构中的记录和缓冲池中的索引页进行合并操作。

\par 可以通过show engine innodb status命令查看change buffer的操作信息:

\noindent\rule{\textwidth}{0.4pt}


\begin{verbatim}-------------------------------------
INSERT BUFFER AND ADAPTIVE HASH INDEX
-------------------------------------
Ibuf: size 1, free list len 0, seg size 2, 0 merges
merged operations:
 insert 0, delete mark 0, delete 0
\end{verbatim}


\noindent\rule{\textwidth}{0.4pt}


\par 其中,merged operations代表了辅助索引页与change buffer的合并操作次数,insert代表insert buffer,delete mark代表delete buffer,delete代表purge buffer。

\par innodb\_change\_buffer\_max\_size参数用于配置change buffer在buffer pool中所占的最大百分比,默认是25\%,最大可以设置为50\%。当MySQL实例中有大量的修改操作时,需要考虑增大innodb\_change\_buffer\_max\_size的值。


% From text00188.html


\subsection{13.4.3 自适应哈希索引(AHI)}


\par AHI(Adaptive Hash index)属性使得InnoDB更像是内存数据库。该属性通过innodb\_adapitve\_hash\_index开启,也可以通过“——skip-innodb\_adaptive\_hash\_index”参数关闭。

\par 哈希(hash)是一种非常快的等值查找方法,一般情况下,这种查找的时间复杂度为O(1),即一般仅需要一次查找就能定位数据。而B+树的查找次数,则取决于B+树的高度,在生产环境中,B+树的高度一般为3~4层,故需要3~4次的查询。

\par InnoDB会监控对表上个索引页的查询。如果观察到建立哈希索引可以带来速度提升,则自动建立哈希索引,称为自适应哈希索引(Adaptive Hash Index,AHI)。

\par AHI是MySQL数据库InnoDB存储引擎中用于加速索引查找的一个结构。InnoDB本身不支持hash索引,所有的索引检索都通过B树查询。AHI可以认为是“索引的索引”。当对一个页面的访问次数满足一定的条件之后,存储引擎会自动建立哈希索引,维护InnoDB叶子页记录的索引键值(或键值前缀)到叶子节点记录的Hash映射关系。这样就能够根据索引键值(或前缀)定位到记录的地址。这样就可以不用再搜索B+树从root到叶子页的路径定位过程。

\par AHI有一个要求,那就是对这个页的连续访问模式必须是一样的。

\par 例如,对于(a,b)访问模式情况:

\par where a=xxx

\par where a=xxx and b=xxx

\par 根据MySQL官网的文档显示,AHI启动后,读写速度提高2倍时,辅助索引的连接操作性能可以提高5倍。

\par 另外,AHI是数据库自动优化的,DBA只需要指导开发人员去尽量使用符合AHI条件的查询即可,以提高效率。


% From text00189.html


\subsection{13.4.4 doublewrite缓存}


\par 如果说change buffer带给InnoDB存储引擎的是性能上的提升,那么doublewrite带给InnoDB存储引擎的则是数据页的可靠性。

\par doublewrite(两次写)缓存是位于系统表空间的存储区域,用来缓存InnoDB的数据页从InnoDB缓存池中flush之后并写入到数据文件之前。所以当操作系统或者数据库进程在数据页写磁盘的过程中崩溃时,InnoDB可以在doublewrite缓存中找到数据页的备份而用来执行crash恢复。

\par 数据页写入到doublewrite缓存的动作所需要的IO消耗要小于写入到数据文件的消耗,因为此写入操作会以一次大的连续块的方式写入。

\par 在应用(apply)重做日志之前,用户需要一个页的副本,当写入失效发生时,先通过页的副本来还原该页,再进行重做,这就是doublewrite。

\par doublewrite缓存示意图如图13-7所示,doublewrite的组成具体如下。

\par 内存中的doublewrite buffer,大小为2MB。

\par 物理磁盘上共享表空间中连续的128个页,即2个区(extend),大小同样为2MB。

\par 对缓冲池的脏页进行刷新时,并不是直接写磁盘,而是通过memcpy()函数将脏页先复制到内存中的doublewrite buffer中,之后通过doublewrite再分两次,每次1MB,顺序地写入共享表空间的物理磁盘上,在这个过程中,因为doublewrite页是连续的,因此这个过程是顺序写的,开销并不是很大。在完成doublewrite页的写入后,再将doublewrite buffer中的页写入各个表空间文件中,此时的写入就是离散的。如果操作系统在将页写入磁盘的过程中发生了崩溃,那么在恢复过程中,InnoDB可以从共享表空间中的doublewrite中找到该页的一个副本,然后将其复制到表空间文件,再应用重做日志。

\begin{figure}[htbp]
\centering
\includegraphics[width=0.8\textwidth]{Image00095.jpg}
\end{figure}



\par 图13-7 doublewrite缓存示意图


% From text00190.html


\subsection{13.4.5 重做日志缓存(redo log buffer)}


\par redo log buffer是一块用来存放写入redo log文件内容的内存区域,内存的大小由innodb\_log\_buffer\_size参数确定。该buffer的内容会定期刷新到磁盘的redo log文件中。

\par 参数innodb\_flush\_log\_at\_trx\_commit决定了buffer刷新到文件的方式,参数innodb\_flush\_log\_at\_timeout决定了buffer刷新的频率。

\par innodb\_flush\_log\_at\_trx\_commit的参数取值及说明如下。

\par ·0:每秒写入并持久化一次(不安全,但性能高,无论是MySQL还是服务器宕机,都会丢数据,丢失最多1秒的数据)。

\par ·1:每次commit都持久化(安全,但性能低,IO负担重)。

\par ·2:每次commit都写入内存的缓存中,每秒再刷新到磁盘(安全,性能折中,若中MySQL宕机则数据不会丢失,若是服务器宕机则会丢失最多1秒的数据)。

\par innodb\_flush\_log\_at\_timeout参数决定最多丢失多少秒的数据,默认是1秒。

\begin{figure}[htbp]
\centering
\includegraphics[width=0.8\textwidth]{Image00096.jpg}
\end{figure}



\par 图13-8 redo log buffer示意图

\par redo log buffer的示意图如图13-8所示。


% From text00191.html


\subsection{13.4.6 重做日志(redo log)}


\par redo日志是存在于磁盘上的文件,默认情况下包括ib\_logfile0和ib\_logfile1两个文件,常用于在crash(恢复)发生时将还没来得及写入到数据文件中但已经完成提交的事务在数据库初始化时重新执行一遍。

\par 关于redo log buffer写入到redo log文件的方式,InnoDB提供了组提交(group commit)的方式,这就意味着针对一次写磁盘操作可以包含多个事务数据,用此方法可以提高性能。

\par 为了提高IO效率,数据库修改的文件都是在内存缓存中完成的;那么,我们知道一旦断电,内存中的数据都将消失,而数据库又是如何保证数据的完整性的?答案是数据持久化与事务日志。

\par 如果宕机了则会应用已经持久化好了的日志文件,读取日志文件中没有被持久化到数据文件里面的记录,然后将这些记录重新持久化到我们的数据文件中。redo log的示意图如图13-9所示。

\par 1.redo log文件配置

\par 默认情况下InnoDB会在数据文件夹下创建两个48MB的日志文件,分别是ib\_logfile0和ib\_logfile1。

\par Innodb\_log\_group\_home\_dir参数用来定义redo日志的文件位置:

\noindent\rule{\textwidth}{0.4pt}


\begin{verbatim}[mysqld]
innodb\_log\_group\_home\_dir = /dr3/iblogs
\end{verbatim}


\noindent\rule{\textwidth}{0.4pt}


\par innodb\_log\_files\_in\_group参数用来定义日志文件的个数,默认和推荐值都是2,在日志组中每个重做日志文件的大小均一致,并以循环写入的方式运行。例如,如果是默认配置,则InnoDB存储引擎先写重做日志文件1,当达到文件的最后时,会切换到重做日志文件2,再当重做日志文件2也被写满时,再切换到重做日志文件1。

\par 如果是拥有三个重做日志文件,则运行方式如图13-10所示。

\begin{figure}[htbp]
\centering
\includegraphics[width=0.8\textwidth]{Image00097.jpg}
\end{figure}



\par 图13-9 redo log示意图

\begin{figure}[htbp]
\centering
\includegraphics[width=0.8\textwidth]{Image00098.jpg}
\end{figure}



\par 图13-10 redo log文件组示意图

\par innodb\_log\_file\_size参数定义了每个日志文件的大小,日志文件越大则意味着buffer pool进行文件间切换的操作就越少,从而减少IO,一般至少要保证在高峰期的1个小时之内,所有的日志都能存放在一个日志文件里而不发生切换,当然文件大小也有最大限制,也就是所有日志文件的总大小不能超过512G。

\par 2.重置InnoDB redo log文件大小

\par 更改redo log文件大小的方法具体如下。

\par 1)关闭MySQL。

\par 2)通过innodb\_log\_file\_size更改文件大小,通过innodb\_log\_files\_in\_group更改文件数量。

\par 3)启动MySQL。


% From text00192.html


\subsection{13.4.7 系统(共享)表空间}


\par InnoDB的系统表空间可用来存放表和索引数据,同时也是doublewriter缓存、change缓存和回滚日志的存储空间,系统表空间是被多个表共享的表空间。

\par 默认情况下,系统表空间只有一个系统数据文件,名为ibdata1。系统数据文件的位置和个数均由参数innodb\_data\_file\_path来决定。InnoDB表与表空间文件依赖如图13-11所示。

\begin{figure}[htbp]
\centering
\includegraphics[width=0.8\textwidth]{Image00099.jpg}
\end{figure}



\par 图13-11 InnoDB表与表空间文件依赖

\par 1.系统表空间数据文件配置

\par 可以通过innodb\_data\_file\_path和innodb\_data\_home\_dir来配置系统表空间数据文件。

\par Innodb\_data\_file\_path可以包含一个或多个数据文件,中间用“;”分开:

\noindent\rule{\textwidth}{0.4pt}


\begin{verbatim}innodb\_data\_file\_path=datafile\_spec1[;datafile\_spec2]...
datafile\_spec1 = file\_name:file\_size[:autoextend[:max:max\_file\_size]]
\end{verbatim}


\noindent\rule{\textwidth}{0.4pt}


\par 其中autoextend和max选项只能用作最后的这个数据文件。Autoextend在默认情况下是一次增加64MB,如果要修改此值,可通过innodb\_autoextend\_increment参数。Max用来指定可扩展数据文件的最大容量以避免数据文件大小超过可用磁盘空间的大小。

\par 举例如下:

\noindent\rule{\textwidth}{0.4pt}


\begin{verbatim}[mysqld]
innodb\_data\_file\_path=ibdata1:50M;ibdata2:50M:autoextend
\end{verbatim}


\noindent\rule{\textwidth}{0.4pt}


\par 上述语句表示指定ibdata1和ibdata2两个数据文件,其中ibdata1文件为固定的50MB大小,而ibdata2文件则是初始化为50MB并可自动扩展容量。

\noindent\rule{\textwidth}{0.4pt}


\begin{verbatim}[mysqld]
innodb\_data\_file\_path=ibdata1:12M:autoextend:max:500M
\end{verbatim}


\noindent\rule{\textwidth}{0.4pt}


\par 上述语句表示指定ibdata1一个数据文件,ibdata文件初始化为12MB,并可自动扩展容量到500MB。

\par innodb\_data\_home\_dir参数用于显示指定数据文件的存储目录,默认是MySQL安装后的数据文件目录,举例如下:

\noindent\rule{\textwidth}{0.4pt}


\begin{verbatim}[mysqld]
innodb\_data\_home\_dir = /path/to/myibdata/
innodb\_data\_file\_path=ibdata1:50M;ibdata2:50M:autoextend
\end{verbatim}


\noindent\rule{\textwidth}{0.4pt}


\par 当然也可以在innodb\_data\_file\_path中指定绝对路径的数据文件:

\noindent\rule{\textwidth}{0.4pt}


\begin{verbatim}[mysqld]
innodb\_data\_home\_dir =
innodb\_data\_file\_path=/path/to/myibdata/ibdata1:50M;/path/to/myibdata/ibdata2:
50M:autoextend
\end{verbatim}


\noindent\rule{\textwidth}{0.4pt}


\par 2.重置InnoDB系统表空间

\par 最简单的增加系统表空间的办法就是在初始化阶段配置数据文件的自增长,配置最后一个文件的autoextend属性,当数据文件空间不足时默认自动增长64MB大小。也可以通过修改innodb\_autoextend\_increment参数来修改自动增长的大小。

\par 还可以通过增加另一个数据文件方法来扩展表空间,具体步骤如下。

\par 1)关闭MySQL。

\par 2)检查配置的最后一个数据文件是否为autoextend,如果是则根据当前数据文件的大小去掉自动扩展属性,改成当前大小。

\par 3)在配置文件的innodb\_data\_file\_path参数里增加一个新的数据文件,选择是否自动扩展。

\par 4)启动MySQL。

\par 代码如下:

\noindent\rule{\textwidth}{0.4pt}


\begin{verbatim}innodb\_data\_home\_dir =
innodb\_data\_file\_path = /ibdata/ibdata1:10M:autoextend
\#\#\#\#改成
innodb\_data\_home\_dir =
innodb\_data\_file\_path = /ibdata/ibdata1:988M;/disk2/ibdata2:50M:autoextend
\end{verbatim}


\noindent\rule{\textwidth}{0.4pt}


\par 减小系统表空间大小的方法具体如下。

\par 1)MySQL dump出所有的InnoDB表,包括MySQL系统数据库下的五个表,代码如下:

\noindent\rule{\textwidth}{0.4pt}


\begin{verbatim}mysql> select table\_name from information\_schema.tables where table\_schema= 'mysql' and engine='InnoDB';
+----------------------+
| table\_name           |
+----------------------+
| innodb\_index\_stats   |
| innodb\_table\_stats   |
| slave\_master\_info    |
| slave\_relay\_log\_info |
| slave\_worker\_info    |
+----------------------+
\end{verbatim}


\noindent\rule{\textwidth}{0.4pt}


\par 2)关闭MySQL。

\par 3)删除InnoDB的所有数据文件和日志文件,包括“*.ibd”和ib\_log文件,还有在MySQL库文件夹下的“*.ibd”文件。

\par 4)删除所有“.frm”的InnoDB表文件。

\par 5)在配置文件里配置新的表空间文件。

\par 6)启动MySQL。

\par 7)导入备份出的dump文件。


% From text00193.html


\subsection{13.4.8 File-per-table独立表空间设置}


\par 通过设置File-per-table表空间,使得InnoDB的数据表不是共享一个系统表空间,而是每个表一个独立的表空间。可以通过设置innodb\_file\_per\_table来开启此属性。开启之后每个表数据和索引数据都会默认单独存放在数据文件夹下的“.ibd”数据文件中。

\noindent\rule{\textwidth}{0.4pt}


\begin{verbatim}mysql> show variables like '\%per\_table\%';
+-----------------------+-------+
| Variable\_name         | Value |
+-----------------------+-------+
| innodb\_file\_per\_table | ON    |
+-----------------------+-------+
\end{verbatim}


\noindent\rule{\textwidth}{0.4pt}


\par 
配置单表数据文件表空间


\par InnoDB的单表数据文件表空间代表每个InnoDB表的数据和索引数据都存放在单独的“.ibd”数据文件中,每个“.ibd”数据文件均代表独立的表空间。此属性可通过innodb\_file\_per\_table配置。

\par 此配置的主要优势具体如下。

\par 1)当删除表或者truncate表的时候,意味着对磁盘空间可以回收。而共享表空间时,若删除一个表则空间不会释放而只是文件里有空闲空间。

\par 2)truncate table命令比共享表空间要快。

\par 3)通过定义“create table…data directory=”绝对路径,可以将特定的表放在特定的磁盘或者存储空间中。

\par 4)可以将单独的表物理复制到另外的MySQL实例中。

\par 此配置的劣势具体如下。

\par 每个表都有未使用的空间,这就意味着磁盘空间有些浪费。

\par 启动单独表空间的方式如下:

\noindent\rule{\textwidth}{0.4pt}


\begin{verbatim}[mysqld]
innodb\_file\_per\_table=1
\end{verbatim}


\noindent\rule{\textwidth}{0.4pt}


\par 当设置innodb\_file\_per\_table=0时,所有创建的新表都会放置到共享表空间里,除非在create table命令里显式地使用tablespace选项。

\par 将已经存在于共享表空间的表修改为独立表空间的方法如下:

\noindent\rule{\textwidth}{0.4pt}


\begin{verbatim}SET GLOBAL innodb\_file\_per\_table=1;
ALTER TABLE table\_name ENGINE=InnoDB;
\end{verbatim}


\noindent\rule{\textwidth}{0.4pt}


\par 通过命令“create table…data directory=绝对路径”可以将单表数据文件创建在另外的目录里。在指定的绝对路径下,会创建与数据库名相同的文件夹,里面含有此表的“.ibd”文件,同时在MySQL的数据库名文件夹下会创建table\_name.isl文件,文件内容包含了此表的真实路径,相当于link文件。

\noindent\rule{\textwidth}{0.4pt}


\begin{verbatim}mysql> USE test;
Database changed

mysql> SHOW VARIABLES LIKE 'innodb\_file\_per\_table';
+-----------------------+-------+
| Variable\_name         | Value |
+-----------------------+-------+
| innodb\_file\_per\_table | ON    |
+-----------------------+-------+
1 row in set (0.00 sec)

mysql> CREATE TABLE t1 (c1 INT PRIMARY KEY) DATA DIRECTORY = '/alternative/directory';
Query OK, 0 rows affected (0.03 sec)

\# MySQL creates a .ibd file for the new table in a subdirectory that corresponding
\# to the database name

db\_user@ubuntu:\textasciitilde{}/alternative/directory/test\$ ls
t1.ibd

\# MySQL creates a .isl file containing the path name for the table in a directory
\# beneath the MySQL data directory

db\_user@ubuntu:\textasciitilde{}/mysql/data/test\$ ls
db.opt  t1.frm  t1.isl
\end{verbatim}


\noindent\rule{\textwidth}{0.4pt}


\par 当没有开启innodb\_file\_per\_table时,可以将tablespace和data directory两个参数配合使用:

\noindent\rule{\textwidth}{0.4pt}


\begin{verbatim}CREATE TABLE t2 (c1 INT PRIMARY KEY) TABLESPACE = innodb\_file\_per\_table
  DATA DIRECTORY = '/alternative/directory';
\end{verbatim}


\noindent\rule{\textwidth}{0.4pt}


\par 不管是出于备份复制还是别的什么原因,若要将单表复制到另外的数据库实例下,可以使用传输表空间的方法。

\par 1)在原实例下创建表,代码如下:

\noindent\rule{\textwidth}{0.4pt}


\begin{verbatim}mysql> use test;
mysql> CREATE TABLE t(c1 INT) engine=InnoDB;
\end{verbatim}


\noindent\rule{\textwidth}{0.4pt}


\par 2)在目标实例下创建表,代码如下:

\noindent\rule{\textwidth}{0.4pt}


\begin{verbatim}mysql> use test;
mysql> CREATE TABLE t(c1 INT) engine=InnoDB;
\end{verbatim}


\noindent\rule{\textwidth}{0.4pt}


\par 3)在目标实例下去除表的表空间属性,代码如下:

\noindent\rule{\textwidth}{0.4pt}


\begin{verbatim}mysql> ALTER TABLE t DISCARD TABLESPACE;
\end{verbatim}


\noindent\rule{\textwidth}{0.4pt}


\par 此命令不支持有外键的表,必须首先执行foreign\_key\_checks=0

\par 4)在原实例下表加锁仅允许读操作,并生成“.cfg”元文件:

\noindent\rule{\textwidth}{0.4pt}


\begin{verbatim}mysql> use test;
mysql> FLUSH TABLES t FOR EXPORT;
\end{verbatim}


\noindent\rule{\textwidth}{0.4pt}


\par 5)将“.ibd”和“.cfg”文件复制到目标实例的指定目录下:

\noindent\rule{\textwidth}{0.4pt}


\begin{verbatim}shell> scp /path/to/datadir/test/t.\{ibd,cfg\} destination-server:/path/to/datadir/test
\end{verbatim}


\noindent\rule{\textwidth}{0.4pt}


\par 6)在原实例下释放锁:

\noindent\rule{\textwidth}{0.4pt}


\begin{verbatim}mysql> use test;
mysql> UNLOCK TABLES;
\end{verbatim}


\noindent\rule{\textwidth}{0.4pt}


\par 7)在目标实例下执行导入表空间操作:

\noindent\rule{\textwidth}{0.4pt}


\begin{verbatim}mysql> use test;
mysql> ALTER TABLE t IMPORT TABLESPACE;
\end{verbatim}


\noindent\rule{\textwidth}{0.4pt}



% From text00194.html


\subsection{13.4.9 undo日志}


\par undo日志是由一系列事务的undo日志记录组成的,每一条undo日志记录均包含了事务数据回滚的相关原始信息,所以当其他的事务需要查看修改前的原始数据时,会从此undo日志记录中获取。undo日志存放在回滚段中的undo日志段中。默认情况下,回滚段是作为系统表空间的一部分的,但也可以通过设置innodb\_undo\_tablespaces和innodb\_undo\_directory两个参数而拥有自己独立的undo表空间。

\par InnoDB支持最大128个回滚段,其中的32个用于服务临时表的相关事务操作,剩下的96个用于服务非临时表,每个回滚段都可以同时支持1023个数据修改事务,也就是总共96K个数据修改事务。

\par Innodb\_undo\_logs参数可用来设置回滚段的个数。

\par undo log的原理很简单,为了满足事务的原子性,在操作任何数据之前,首先将数据备份到一个地方(这个存储数据备份的地方称为undo log)。然后进行数据的修改。如果出现了错误,或者用户执行了ROLLBACK语句,那么系统可以利用undo log中的备份将数据恢复到事务开始之前的状态。

\par 1.undo表空间配置

\par 默认情况下,undo日志是存放在系统表空间里的,但也可以选择在独立的一个或多个undo表空间中存放undo日志。

\par ·Innodb\_undo\_directory参数决定了独立的undo表空间存放目录。

\par ·Innodb\_undo\_logs参数决定了回滚段的个数,该变量可以动态调整。

\par ·Innodb\_undo\_tablespaces参数决定了独立的undo表空间的个数,比如将该参数设置为16时,则会在undo表空间的存放目录下创建16个undo文件,默认为10MB。

\par 示例代码如下:

\noindent\rule{\textwidth}{0.4pt}


\begin{verbatim}mysql> show variables like '\%innodb\_undo\%';
+--------------------------+-------+
| Variable\_name            | Value |
+--------------------------+-------+
| innodb\_undo\_directory    | ./    |
| innodb\_undo\_log\_truncate | OFF   |
| innodb\_undo\_logs         | 128   |
| innodb\_undo\_tablespaces  | 0     |
\end{verbatim}


\noindent\rule{\textwidth}{0.4pt}


\par 2.设置undo log独立表空间

\par 默认情况下,undo log是存储在系统表空间里的,我们也可以将其存放在一个或多个独立表空间下。

\par ·Innodb\_undo\_tablespaces参数定义了有多少个undo表空间,此参数只能在建立MySQL实例时被配置。

\par ·innodb\_undo\_directory参数定义了undo表空间的存放路径。

\par ·innodb\_undo\_logs参数定义了回滚段的数量。

\par 3.清空undo表空间的内容

\par MySQL实例最少会有两个undo表空间,在清空一个undo表空间的同时需要保证至少有一个undo表空间是可用的:

\noindent\rule{\textwidth}{0.4pt}


\begin{verbatim}mysql> SELECT @@innodb\_undo\_tablespaces;
+---------------------------+
| @@innodb\_undo\_tablespaces |
+---------------------------+
|                         2 |
+---------------------------+
\end{verbatim}


\noindent\rule{\textwidth}{0.4pt}


\par Innodb\_undo\_logs参数决定了回滚段的数量,取值范围是35~128,默认是128,其中的前33个分布在系统表空间和临时表空间里,剩余的则在表空间之间通过round-robin方式依次分布:

\noindent\rule{\textwidth}{0.4pt}


\begin{verbatim}mysql> SELECT @@innodb\_undo\_logs;
+--------------------+
| @@innodb\_undo\_logs |
+--------------------+
|                128 |
+--------------------+
1 row in set (0.00 sec)
\end{verbatim}


\noindent\rule{\textwidth}{0.4pt}


\par Innodb\_undo\_log\_truncate参数决定是否开启undo表空间清空:

\noindent\rule{\textwidth}{0.4pt}


\begin{verbatim}mysql> SET GLOBAL innodb\_undo\_log\_truncate=ON;
\end{verbatim}


\noindent\rule{\textwidth}{0.4pt}


\par 当将此参数设置为ON后,则代表将undo文件大小超过innodb\_max\_undo\_log\_size(默认值是128MB)的都标记为清空:

\noindent\rule{\textwidth}{0.4pt}


\begin{verbatim}mysql> SELECT @@innodb\_max\_undo\_log\_size;
+----------------------------+
| @@innodb\_max\_undo\_log\_size |
+----------------------------+
|                 1073741824 |

mysql> SET GLOBAL innodb\_max\_undo\_log\_size=2147483648;
Query OK, 0 rows affected (0.00 sec)
\end{verbatim}


\noindent\rule{\textwidth}{0.4pt}


\par 当标记为清空后,若回滚段标记为非激活状态则表示不接收新的事务,而已存在的事务会等到完成;然后通过purge操作释放回滚段空间;当undo表空间的所有回滚段都释放后,表空间就会清空成初始大小10MB;然后回滚段重新变成激活状态以接收新的事务。


% From text00195.html


\subsection{13.4.10 临时表空间}


\par 临时表空间(temporary)用于存放临时表,默认情况下,是在数据文件夹下的ibtmp1数据文件,此数据文件被设置为每次自动增长12MB大小,当然也可以设置innodb\_temp\_data\_file\_path来指定临时表空间文件的存放位置。

\par 临时表空间文件在正常的shutdown之后会自动清除,如果发现临时表空间数据文件比较大,那么可以考虑重启MySQL来释放空间大小,如图13-12所示。

\begin{figure}[htbp]
\centering
\includegraphics[width=0.8\textwidth]{Image00100.jpg}
\end{figure}



\par 图13-12 重启MySQL

\par 但在crash发生时不会清除,这就需要DBA手动删除表空间文件或重启服务器。

\par 查看临时表空间文件的方法如下:

\noindent\rule{\textwidth}{0.4pt}


\begin{verbatim}mysql> show variables like '\%innodb\_temp\%';
+----------------------------+-----------------------+
| Variable\_name              | Value                 |
+----------------------------+-----------------------+
| innodb\_temp\_data\_file\_path | ibtmp1:12M:autoextend |
\end{verbatim}


\noindent\rule{\textwidth}{0.4pt}



% From text00196.html


\subsection{13.4.11 InnoDB后台线程}


\par 1.InnoDB主线程

\par InnoDB的主线程在后台承担了诸多任务,绝大多数是与IO操作相关的,比如将buffer pool中修改后的数据异步刷新到磁盘文件中,保证数据的一致性,包括脏页的刷新、合并change buffer、undo回滚页的回收等。

\par 从MySQL 5.6开始,master线程的工作已经被大大减轻,类似purge、page clean都分配给独立的后台线程来进行。

\par Innodb\_io\_capacity参数设置了InnoDB的整体IO能力。该参数应该被设置为等同于操作系统每秒的IO操作数量。该参数可以设置为100及以上的任意数值,默认值是200。其中设置为100时相当于7200RPM的磁盘性能。

\par 2.InnoDB后台IO线程

\par 在InnoDB存储引擎中可大量使用AIO(Async IO)来处理IO请求,这样可以提高数据库的性能,而IO线程的主要工作是负责这些IO请求的回调处理。

\par 通过配置innodb\_read\_io\_threads和innodb\_write\_io\_threads参数来指定后台读和写数据页的线程的个数,默认值是4,容许的取值范围是1-64。

\par 可以通过show engine innodb status命令来查看InnoDB中IO的线程情况:

\noindent\rule{\textwidth}{0.4pt}


\begin{verbatim}mysql> show engine innodb status\textbackslash\{\}G
FILE I/O
--------
I/O thread 0 state: waiting for completed aio requests (insert buffer thread)
I/O thread 1 state: waiting for completed aio requests (log thread)
I/O thread 2 state: waiting for completed aio requests (read thread)
I/O thread 3 state: waiting for completed aio requests (read thread)
I/O thread 4 state: waiting for completed aio requests (read thread)
I/O thread 5 state: waiting for completed aio requests (read thread)
I/O thread 6 state: waiting for completed aio requests (write thread)
I/O thread 7 state: waiting for completed aio requests (write thread)
I/O thread 8 state: waiting for completed aio requests (write thread)
I/O thread 9 state: waiting for completed aio requests (write thread)
\end{verbatim}


\noindent\rule{\textwidth}{0.4pt}


\par 3.InnoDB purge线程

\par 事务被提交之后,可能会不再需要其所使用的undo log,因此需要purge线程来回收已经使用并分配的undo页。

\par InnoDB的purge操作是一种垃圾回收操作,是由一个或多个独立的线程自动执行的。可通过innodb\_purge\_threads参数设置purge线程的数量,如果DML操作比较复杂且会涉及多个表时,则可以考虑增加此值,最大可以设置为32。

\noindent\rule{\textwidth}{0.4pt}


\begin{verbatim}[mysqld]
innodb-purge\_threads=1
\end{verbatim}


\noindent\rule{\textwidth}{0.4pt}


\par 4.InnoDB脏页刷新线程

\par 对磁盘数据页的更新都是先缓存在内存缓冲池中,那些已经被后台线程修改但还没有刷新到磁盘的数据页称为脏页,由于脏页的存在,为保证数据的持久化,需要有一个线程来承担定期刷新的任务。在早期的版本中,主要是由主线程来承担刷新任务;在后续的InnoDB版本中,脏页的刷新被单独分成一个线程处理,即脏页刷新线程。

\par 5.InnoDB线程并发度配置

\par InnoDB利用操作系统的线程技术可达到多线程实现。

\par ·Innodb\_thread\_concurrency参数用于限制同时执行的线程数目。默认值是0,代表没有限制。

\par ·Innodb\_thread\_sleep\_delay参数确定。


% From text00197.html


\section{13.5 InnoDB其他相关配置}


\subsection{
13.5.1 启动配置}
}


\par InnoDB合理的规划方法是在创建数据库实例之前就定义好数据文件、日志文件和数据页大小等相关属性。


% From text00198.html


\subsection{13.5.2 指定配置文件位置}


\par MySQL实例启动需要依赖my.cnf配置文件,而配置文件可以存在于多个操作系统目录下。my.cnf文件的默认查找路径,从上到下找到的文件先读,但优先级是逐级提升的。MySQL配置文件的位置如图13-13所示。

\begin{figure}[htbp]
\centering
\includegraphics[width=0.8\textwidth]{Image00101.jpg}
\end{figure}



\par 图13-13 MySQL配置文件的位置

\par 比如在/etc/my.cnf中指定了端口是3307,而在用户根目录下的“.my.cnf”中同时指定了端口是3308,则MySQL启动之后的侦听端口是3308。因为对相同的参数来说,根目录下的配置文件优先级更高。而如果在两个文件中仅在/etc/my.cnf中指定了端口,则会以该指定的端口开启MySQL。

\par 为了确保使用的是指定的配置文件,可以使用“--defaults-file”参数指定启动文件的位置:

\noindent\rule{\textwidth}{0.4pt}


\begin{verbatim}mysqld --defaults-file=path\_to\_configuration\_file
\end{verbatim}


\noindent\rule{\textwidth}{0.4pt}



% From text00199.html


\subsection{13.5.3 数据页配置}


\par Innodb\_page\_size参数用来指定所有InnoDB表空间的数据页大小。默认是16KB大小,也可以设置为64KB、32KB、8KB和4KB。一般设置为与存储磁盘的block size接近的大小。


% From text00200.html


\subsection{13.5.4 InnoDB只读设置}


\par InnoDB可以通过“——innodb-read-only”参数设置数据表只能读取:

\noindent\rule{\textwidth}{0.4pt}


\begin{verbatim}[mysqld]
innodb-read-only=1

mysql> update temp set id2=100;
 ERROR 1015 (HY000): Can't lock file (errno: 165 - Table is read only)
\end{verbatim}


\noindent\rule{\textwidth}{0.4pt}



% From text00201.html


\subsection{13.5.5 InnoDB优化器统计信息配置}


\par InnoDB表的优化器统计信息分为永久和非永久两种。

\par 对于永久的优化器统计信息,即使是服务器重启的情况下也会存在,其用于选出更优的执行计划以便提供更好的查询性能。

\par 配置innodb\_stats\_auto\_recalc参数可用来控制统计信息在表发生了巨大变化(超过10\%的行)之后是否自动更新,但是由于自动更新统计信息本身是异步的,所以有时未必能马上更新,这时可以执行analysis table语句来同步更新统计信息。

\par create table和alter table语句中的stats\_persistent、stats\_auto\_recalc、stats\_sample\_pages子句可用来配置单个表的优化器统计信息规则,具体如下。

\par 1)stats\_persistent用来指定是否对此表开启永久统计资料,1代表开启,0代表不开启。当开启之后,可以执行analyze table命令收集统计资料。

\par 2)stats\_auto\_recalc表示是否对表的永久统计资料自动进行重新计算,默认值与全局参数innodb\_stats\_auto\_recalc一致。1代表当表中10\%以上的数据更新时需要重新计算,0代表不自动更新,而是通过analyze table命令重新计算。

\par 3)stats\_sample\_pages表示计算索引列的统计资料是需要的索引页的样本数量。

\par 代码如下:

\noindent\rule{\textwidth}{0.4pt}


\begin{verbatim}CREATE TABLE `t1` (
`id` int(8) NOT NULL auto\_increment,
`data` varchar(255),
`date` datetime,
PRIMARY KEY  (`id`),
INDEX `DATE\_IX` (`date`)
) ENGINE=InnoDB,
  STATS\_PERSISTENT=1,
  STATS\_AUTO\_RECALC=1,
STATS\_SAMPLE\_PAGES=25;
\end{verbatim}


\noindent\rule{\textwidth}{0.4pt}


\par 优化器统计资料数据存储在系统表mysql.innodb\_table\_stats和mysql.innodb\_index\_stats表中,这两个表中有一个字段last\_update可以用来判断统计信息最后更改的时间。这两个表的数据也可以通过手工更改。当手工更改完数据之后,要执行“flush table表名”命令来重新load此表的统计资料。innodb\_table\_stats表中的每个目标表示一行记录,而innodb\_index\_stats表中的每个索引都会有多条记录。

\par Innodb\_table\_stats表结构如表13-2所示。

\par 表13-2 Innodb\_table\_stats表结构

\begin{figure}[htbp]
\centering
\includegraphics[width=0.8\textwidth]{Image00102.jpg}
\end{figure}



\begin{figure}[htbp]
\centering
\includegraphics[width=0.8\textwidth]{Image00103.jpg}
\end{figure}



\par Innodb\_table\_stats表结构代码如下:

\noindent\rule{\textwidth}{0.4pt}


\begin{verbatim}mysql> select * from mysql.innodb\_table\_stats where table\_name='students'\textbackslash\{\}G
*************************** 1. row ***************************
           database\_name: test
              table\_name: students
             last\_update: 2017-04-21 17:12:07
                  n\_rows: 5
    clustered\_index\_size: 1
sum\_of\_other\_index\_sizes: 2
\end{verbatim}


\noindent\rule{\textwidth}{0.4pt}


\par Innodb\_index\_stats表结构如表13-3所示。

\par 表13-3 Innodb\_index\_stats表结构

\begin{figure}[htbp]
\centering
\includegraphics[width=0.8\textwidth]{Image00104.jpg}
\end{figure}



\par Innodb\_index\_stats表结构代码如下:

\noindent\rule{\textwidth}{0.4pt}


\begin{verbatim}mysql> select * from mysql.innodb\_index\_stats where table\_name='students';
+---------------+------------+--------------+---------------------+--------------+
| database\_name | table\_name | index\_name   | last\_update         | stat\_name    | stat\_value | sample\_size | stat\_description                  |
+---------------+------------+--------------+---------------------+--------------+
| test          | students   | PRIMARY      | 2017-04-21 17:12:07 | n\_diff\_pfx01 |          5 |           1 | sid                               |
| test          | students   | PRIMARY      | 2017-04-21 17:12:07 | n\_leaf\_pages |          1 |        NULL | Number of leaf pages in the index |
| test          | students   | PRIMARY      | 2017-04-21 17:12:07 | size         |          1 |        NULL | Number of pages in the index      |
| test          | students   | idx\_st\_sid   | 2017-04-21 17:12:07 | n\_diff\_pfx01 |          5 |           1 | sid                               |
| test          | students   | idx\_st\_sid   | 2017-04-21 17:12:07 | n\_leaf\_pages |          1 |        NULL | Number of leaf pages in the index |
| test          | students   | idx\_st\_sid   | 2017-04-21 17:12:07 | size         |          1 |        NULL | Number of pages in the index      |
\end{verbatim}


\noindent\rule{\textwidth}{0.4pt}


\par “Stat\_name=n\_diff\_pfxNN”参数:当参数是n\_diff\_pfx01时,stat\_value列表示索引第一列上的区别值有几个;当参数是n\_diff\_pfx02时,stat\_value列表示索引第一、二列上的区别值有几个,以此类推。而stat\_description列显示了对应的逗号隔开的索引列值。

\par 默认情况下,永久优化器统计信息的属性是开启的,即innodb\_stats\_persistent=ON。

\par 非永久优化器统计信息会在每次进行服务器重启或者其他一些操作时被清理。

\par 设置“innodb\_stats\_persistent=ON”参数(默认),可将优化器统计信息存储在磁盘上。

\par MySQL的查询优化器会基于评估好的统计资料选择合适的索引参与到执行计划中,而类似于analyze table的语句则会从索引中随机选取数据页参与到每个索引的基数评估中。而参数innodb\_stats\_persistent\_sample\_pages则决定了参与评估的数据页的数量,默认值是20。当语句执行的执行计划不是最优选择时,可考虑增加此参数,以便获得正确的统计资料。

\par 当设置参数innodb\_stats\_persistent=OFF或者对单个表设置stats\_persistent=0时,对应的统计资料就仅存在于内存中而非磁盘上,当服务器重启之后统计资料丢失,代码如下:

\noindent\rule{\textwidth}{0.4pt}


\begin{verbatim}mysql> alter table students stats\_persistent=0;
Query OK, 0 rows affected (0.00 sec)
Records: 0  Duplicates: 0  Warnings: 0
mysql> truncate table students;
mysql> analyze table students;
+-----------------+---------+----------+----------+
| Table           | Op      | Msg\_type | Msg\_text |
+-----------------+---------+----------+----------+
| course.students | analyze | status   | OK       |
mysql> select * from mysql.innodb\_index\_stats where table\_name=‘students';
\#\#最新统计信息在表中不更新
+---------------+------------+------------+---------------------+--------------+
| database\_name | table\_name | index\_name | last\_update         | stat\_name    | stat\_value | sample\_size | stat\_description
+---------------+------------+------------+---------------------+--------------+
| course        | students   | PRIMARY    | 2017-05-18 09:14:21 | n\_diff\_pfx01 |      97344 |          20 | sid                               |
| course        | students   | PRIMARY    | 2017-05-18 09:14:21 | n\_leaf\_pages |        234 |        NULL | Number of leaf pages in the index |
| course        | students   | PRIMARY    | 2017-05-18 09:14:21 | size         |        289 |        NULL | Number of pages in the index      |
| course        | students   | idx\_1      | 2017-05-18 09:15:30 | n\_diff\_pfx01 |          0 |           2 | gender                            |
| course        | students   | idx\_1      | 2017-05-18 09:15:30 | n\_diff\_pfx02 |     100710 |          20 | gender,sid                        |
| course        | students   | idx\_1      | 2017-05-18 09:15:30 | n\_leaf\_pages |         90 |        NULL | Number of leaf pages in the index |
| course        | students   | idx\_1      | 2017-05-18 09:15:30 | size         |        161 |        NULL | Number of pages in the index

mysql> alter table students stats\_persistent=1;
mysql> analyze table students;
+-----------------+---------+----------+----------+
| Table           | Op      | Msg\_type | Msg\_text |
+-----------------+---------+----------+----------+
| course.students | analyze | status   | OK       |
+-----------------+---------+----------+----------+
1 row in set (0.01 sec)
mysql> select * from mysql.innodb\_index\_stats where table\_name='students';
+---------------+------------+------------+---------------------+--------------+
| database\_name | table\_name | index\_name | last\_update         | stat\_name    | stat\_value | sample\_size | stat\_description                  |
+---------------+------------+------------+---------------------+--------------+
| course        | students   | PRIMARY    | 2017-05-18 09:33:39 | n\_diff\_pfx01 |          0 |           1 | sid                               |
| course        | students   | PRIMARY    | 2017-05-18 09:33:39 | n\_leaf\_pages |          1 |        NULL | Number of leaf pages in the index |
| course        | students   | PRIMARY    | 2017-05-18 09:33:39 | size         |          1 |        NULL | Number of pages in the index      |
| course        | students   | idx\_1      | 2017-05-18 09:33:39 | n\_diff\_pfx01 |          0 |           1 | gender                            |
| course        | students   | idx\_1      | 2017-05-18 09:33:39 | n\_diff\_pfx02 |          0 |           1 | gender,sid                        |
| course        | students   | idx\_1      | 2017-05-18 09:33:39 | n\_leaf\_pages |          1 |        NULL | Number of leaf pages in the index |
| course        | students   | idx\_1      | 2017-05-18 09:33:39 | size         |          1 |        NULL | Number of pages in the index      |
\end{verbatim}


\noindent\rule{\textwidth}{0.4pt}


\par 比如执行analyze table语句手动刷新统计资料,或者在innodb\_stats\_on\_metadata选项打开之后执行show table status/show index,或者查询information\_schema.tables/statistics表时,非永久统计资料都会自动更新,当InnoDB检测到1/16的表数据被修改时也会更新。


% From text00202.html


\subsection{13.5.6 索引页之间合并阈值}


\par 当索引页的数据由于删除操作或者修改操作低于阈值时,InnoDB可以通过配置merge\_threshold来确保其会将此索引页和邻近的索引页合并。merge-threshold的默认值是50,取值范围是1到50。

\par merge\_threshold参数可以定义在表上,也可以定义在一个独立的索引上。

\noindent\rule{\textwidth}{0.4pt}


\begin{verbatim}CREATE TABLE t1 (
   id INT,
  KEY id\_index (id)
) COMMENT='MERGE\_THRESHOLD=45';
CREATE TABLE t1 (
   id INT,
  KEY id\_index (id)
);

ALTER TABLE t1 COMMENT='MERGE\_THRESHOLD=40';
CREATE TABLE t1 (
   id INT,
  KEY id\_index (id) COMMENT 'MERGE\_THRESHOLD=40'
);
CREATE TABLE t1 (id INT);
CREATE INDEX id\_index ON t1 (id) COMMENT 'MERGE\_THRESHOLD=40';
\end{verbatim}


\noindent\rule{\textwidth}{0.4pt}


\par 评估merge\_threshold参数的合理方法是查看innodb\_metrics表里的相关参数,确保发生了较少的索引页合并,且合并请求和成功合并的数量相当:

\noindent\rule{\textwidth}{0.4pt}


\begin{verbatim}mysql> SELECT NAME, COMMENT FROM INFORMATION\_SCHEMA.INNODB\_METRICS
WHERE NAME like '\%index\_page\_merge\%';
+-----------------------------+----------------------------------------+
| NAME                        | COMMENT                                |
+-----------------------------+----------------------------------------+
| index\_page\_merge\_attempts   | Number ofindex page merge attempts    |
| index\_page\_merge\_successful | Number of successful index page merges |
+-----------------------------+----------------------------------------+
\end{verbatim}


\noindent\rule{\textwidth}{0.4pt}



% From text00203.html


\section{13.6 InnoDB普通表空间}


\par InnoDB除了上文中讲述的系统表额空间和独立表空间之外,也可以创建普通表空间,可以将多个表指定放在此表空间上,以便与系统表空间文件区别对待。

\par 利用create tablespace命令可以创建一个共享的InnoDB表空间,与系统表空间一样,多个表可以在此表空间上存储数据,此表空间的数据文件可以放置在任意的文件夹下:

\noindent\rule{\textwidth}{0.4pt}


\begin{verbatim}CREATE TABLESPACE tablespace\_name
    ADD DATAFILE 'file\_name'
    [FILE\_BLOCK\_SIZE = value]
        [ENGINE [=] engine\_name]

mysql> CREATE TABLESPACE `ts1` ADD DATAFILE 'ts1.ibd' Engine=InnoDB;
\#\#创建在MySQL数据目录下
mysql> CREATE TABLESPACE `ts1` ADD DATAFILE '/my/tablespace/directory/ts1.ibd' Engine=InnoDB;
\end{verbatim}


\noindent\rule{\textwidth}{0.4pt}


\par 创建完表空间之后,就可以通过“create table…tablespace”或者“alter table…tablespace”命令将表增加到此表空间上:

\noindent\rule{\textwidth}{0.4pt}


\begin{verbatim}mysql> CREATE TABLE t1 (c1 INT PRIMARY KEY) TABLESPACE ts1 ROW\_FORMAT=COMPACT;
mysql> ALTER TABLE t2 TABLESPACE ts1;
\end{verbatim}


\noindent\rule{\textwidth}{0.4pt}


\par alter table命令可以使得InnoDB表在系统表空间、独立表空间和普通表空间之间互相转化:

\noindent\rule{\textwidth}{0.4pt}


\begin{verbatim}ALTER TABLE tbl\_name TABLESPACE [=] tablespace\_name
\#\#从系统表空间或者独立表空间上转移到普通表空间
ALTER TABLE tbl\_name ... TABLESPACE [=] innodb\_system
\#\#从普通表空间或者独立表空间上转移到系统表空间
ALTER TABLE tbl\_name ... TABLESPACE [=] innodb\_file\_per\_table
\#\#从系统表空间或者普通表空间上转移到独立表空间
\end{verbatim}


\noindent\rule{\textwidth}{0.4pt}


\par “Alter table…tablespace”语句的执行都会导致此表重建,即使表空间的属性和之前的是一样的。

\par 当删除一个普通表空间时,首先需要保证此表空间上的所有表都已被删除,否则会报错。删除表空间是通过drop tablespace语句来执行的。Drop database的动作会删除所有的表,但创建的tablespace不会被自动删除,必须通过drop tablespace显式执行。

\noindent\rule{\textwidth}{0.4pt}


\begin{verbatim}mysql> drop tablespace ts1;
ERROR 1529 (HY000): Failed to drop TABLESPACE ts1
mysql> drop table temp123;
Query OK, 0 rows affected (0.00 sec)
mysql> drop tablespace ts1;
Query OK, 0 rows affected (0.01 sec)
\end{verbatim}


\noindent\rule{\textwidth}{0.4pt}


\par 普通表空间不支持临时表,而且也不支持“alter table…discard tablespace”和“alter table…import tablespace”命令。


% From text00204.html


\section{13.7 InnoDB表}


\subsection{
13.7.1 InnoDB表存储结构}
}


\par 在InnoDB存储引擎中,表都是按照主键的顺序组织存放的,表的这种存储方式称为索引组织表。在InnoDB存储引擎中,每张表都有一个主键,如果没有自定义的主键和唯一索引,则存储引擎会自动创建一个6个字节大小的指针作为表的主键。

\par 图13-14展示了InnoDB表中的数据存储格式,可以看到在主键或者说聚簇索引的叶子节点上就存储了主键列的值,同时也存储了非主键列的值,所以说InnoDB表的表数据都存放在聚簇索引上。

\begin{figure}[htbp]
\centering
\includegraphics[width=0.8\textwidth]{Image00105.jpg}
\end{figure}



\par 图13-14 InnoDB表结构

\par 图13-15展示了InnoDB的主键索引、二级索引和MyISAM存储引擎的区别:对InnoDB的二级索引来说,虽然也是B+树结构,但叶子节点仅需要存放索引的键值与主键值的对应关系即可,因为完全可以通过此对应关系上的主键值找到主键索引上的其他列值;而MyISAM存储引擎则更像是其他关系型数据库,在所有索引的叶子节点上都存放了对应表数据行的物理存放路径指针。


% From text00205.html


\subsection{13.7.2 创建InnoDB表}


\par 如下所示的代码是通过create table语句创建InnoDB表,因为默认存储引擎就是InnoDB,所以不需要在创建表的语句的最后指定engine=innodb:

\noindent\rule{\textwidth}{0.4pt}


\begin{verbatim}CREATE TABLE `students` (
  `sid` int(11) DEFAULT NULL,
  `sname` varchar(10) DEFAULT NULL,
  `gender` int(11) DEFAULT NULL,
  `dept\_id` int(11) DEFAULT NULL
)
\end{verbatim}


\noindent\rule{\textwidth}{0.4pt}


\par InnoDB的表数据和索引数据默认都是存储在系统表空间中的,但可以通过开启innodb\_file\_per\_table选项将表数据和索引数据存放在独立的表空间中。表创建完之后,会在表所在的数据库文件夹里创建“.frm”文件用来存储表的结构,系统表空间对应的“.ibdata”文件存储数据文件,而当开启独立表空间时,则会在表所在的数据库文件夹里创建“.ibd”用来存储表数据和索引数据。

\begin{figure}[htbp]
\centering
\includegraphics[width=0.8\textwidth]{Image00106.jpg}
\end{figure}



\par 图13-15 InnoDB和MyISAM表结构对比

\par 通过show table status语句可以查看InnoDB的表属性:

\noindent\rule{\textwidth}{0.4pt}


\begin{verbatim}mysql> show table status like 'students'\textbackslash\{\}G
*************************** 1. row ***************************
           Name: students
         Engine: InnoDB
        Version: 10
     Row\_format: Dynamic
           Rows: 8
 Avg\_row\_length: 2048
    Data\_length: 16384
Max\_data\_length: 0
   Index\_length: 0
      Data\_free: 0
 Auto\_increment: NULL
    Create\_time: 2017-05-05 23:10:11
    Update\_time: 2017-05-05 23:25:29
     Check\_time: NULL
      Collation: latin1\_swedish\_ci
       Checksum: NULL
 Create\_options:
        Comment:
1 row in set (0.00 sec)
\end{verbatim}


\noindent\rule{\textwidth}{0.4pt}



% From text00206.html


\subsection{13.7.3 修改表的存储引擎}


\par alter table语句可以修改已有表的存储引擎:

\noindent\rule{\textwidth}{0.4pt}


\begin{verbatim}ALTER TABLE table\_name ENGINE=InnoDB;
\end{verbatim}


\noindent\rule{\textwidth}{0.4pt}



% From text00207.html


\subsection{13.7.4 自增长字段设置}


\par 当对InnoDB表设置了自增长字段之后,表会在内存中保存一个自增长计数器。

\par 默认情况下,自增长字段的初始值是1,但也可以通过配置auto\_increment\_offset参数将所有的自增长字段的初始值设置为另外的值,若要向表中插入数值,则InnoDB会求出当前表中该列的最大值,然后在此基础上加1作为插入的数据。默认是以+1为增长的进度,但也可以通过auto\_increment\_increment配置所有自增长字段的自定义增长进度。


% From text00208.html


\subsection{13.7.5 InnoDB表主要的限制}


\par ·InnoDB表目前只支持最多1017个列。

\par ·InnoDB表目前支持最大64个二级索引。

\par ·多列索引目前支持最大16个列。

\par ·如果表中不存在text或者blob类型字段,则行数据整体的最大长度是65535个字节:

\noindent\rule{\textwidth}{0.4pt}


\begin{verbatim}mysql> CREATE TABLE t (a VARCHAR(8000), b VARCHAR(10000),
    -> c VARCHAR(10000), d VARCHAR(10000), e VARCHAR(10000),
    -> f VARCHAR(10000), g VARCHAR(10000)) ENGINE=InnoDB;
ERROR 1118 (42000): Row size too large. The maximum row size for the
used table type, not counting BLOBs, is 65535. You have to change some
columns to TEXT or BLOBs
\end{verbatim}


\noindent\rule{\textwidth}{0.4pt}



% From text00209.html


\chapter{第14章 MySQL主从复制知识与应用实践}


\section{
14.1 MySQL主从复制}
}


\par MySQL数据库的主从复制技术与使用scp/rsync等命令进行的异机文件级别复制类似,都是数据的远程传输,只不过MySQL的主从复制技术是其软件自身携带的功能,无须借助第三方工具,而且,MySQL的主从复制并不是直接复制数据库磁盘上的文件,而是将逻辑的记录数据库更新的binlog日志发送到需要同步的数据库服务器本地,然后再由本地的数据库线程读取日志中的SQL语句并重新应用到MySQL数据库中,从而即可实现数据库的主从复制。


% From text00210.html


\subsection{14.1.1 MySQL主从复制介绍}


\par MySQL数据库支持单向、双向、链式级联、环状等不同业务场景的主从复制。在复制过程中,一台服务器(严格来讲是实例)作为主数据库(Master),接收来自用户的、对其内容的更新,而一个或多个其他的服务器则作为从服务器(Slave),接收来自主服务器binlog文件的日志内容,然后将该日志内容解析出的SQL语句重新应用到其他从服务器中,使得主从服务器数据达到一致。

\par 如果设置了链式级联复制,那么,从服务器本身除了作为从服务器之外,也会同时作为其下面从服务器的主数据库服务器。链式级联复制形式类似于A==>B==>C。

\par 图14-1和图14-2均为单向主从复制架构逻辑图,此架构只能在Master服务器端进行数据写入。

\begin{figure}[htbp]
\centering
\includegraphics[width=0.8\textwidth]{Image00107.jpg}
\end{figure}



\par 图14-1 一主一从逻辑图示例

\begin{figure}[htbp]
\centering
\includegraphics[width=0.8\textwidth]{Image00108.jpg}
\end{figure}



\par 图14-2 一主多从逻辑图

\par 图14-3为双向主主复制逻辑架构图,此架构可以在Master服务器1端或Master服务器2端进行数据写入,或者在两端同时写入数据(需要经过特殊设置)。

\begin{figure}[htbp]
\centering
\includegraphics[width=0.8\textwidth]{Image00109.jpg}
\end{figure}



\par 图14-3 双向主主复制逻辑图

\par 图14-4为线性级联单向双主复制逻辑架构图,此架构只能在Master服务器1端进行数据写入,工作场景Master服务器1和Master服务器2作为主主互备,Slave服务器1作为从库,中间的Master服务器2需要进行特殊的设置。

\begin{figure}[htbp]
\centering
\includegraphics[width=0.8\textwidth]{Image00110.jpg}
\end{figure}



\par 图14-4 线性级联单向双主复制逻辑图

\par 图14-5为环状级联单向多主同步逻辑架构图,任意一个点都可以写入数据,此架构比较复杂,属于极端环境下的“作品”,一般场景应慎用。

\begin{figure}[htbp]
\centering
\includegraphics[width=0.8\textwidth]{Image00111.jpg}
\end{figure}



\par 图14-5 环状级联单向多主同步逻辑架构图

\par 此外,MySQL官方手册也给出了常见的复制架构图,如图14-6所示。

\begin{figure}[htbp]
\centering
\includegraphics[width=0.8\textwidth]{Image00112.jpg}
\end{figure}



\par 图14-6 MySQL官方的常见复制架构图

\par 在当前的生产工作环境中,MySQL主从复制默认都是异步的复制方式,即不是严格实时的数据同步,但是在正常情况下带给用户的体验几乎都是实时的(延迟的情况见后文讲解)。


% From text00211.html


\subsection{14.1.2 MySQL主从复制企业级应用场景}


\par MySQL主从复制集群技术使得MySQL数据库支持大规模高并发的读写操作成为可能,同时又能有效地解决物理服务器宕机场景的数据备份和进行快速业务切换的问题。对于企业生产环境来说,MySQL主从复制主要有以下几个重要的应用场景。

\par 1.从服务器作为主服务器的实时数据备份

\par 主从服务器架构的设计,可以大大加强MySQL数据库架构的健壮性。例如,当主服务器出现问题时,我们可以人工切换或设置成自动切换到从服务器继续提供服务,此时从服务器的数据和宕机时的主数据库几乎是一致的。

\par 这有些类似于NFS存储数据通过inotify+rsync将数据同步到备份的NFS服务器,只不过MySQL的复制方案是其自带的工具,实现的方式是逻辑的复制,而非文件层级的复制。

\par 利用MySQL的主从复制技术进行数据备份,在硬件故障、软件故障、人为在数据库外误操作的场景下,该数据备份是有效的;但对于人为地在数据库中执行drop、delete等语句删除数据的情况,从库的备份功能就没有用了,因为从服务器也会执行删除的语句。

\par 2.主从服务器实现读写分离,从服务器实现负载均衡

\par MySQL主从服务器架构可通过程序(PHP、Java等)或代理软件(maxscale、atlas)实现对用户(客户端)的请求按读和写进行分离访问,即让从服务器仅仅处理用户的select(查询)请求,以降低用户查询的响应时间及同时在主服务器上读写所带来的访问压力。对于更新的数据(例如update、insert、delete语句)仍然会交给主服务器处理,以确保主服务器和从服务器保持实时同步。

\par 百度、淘宝、新浪等绝大多数的网站都是用户浏览的页面多于用户发布内容的页面,因此通过在从服务器上接收只读请求,就可以很好地减轻主库的读压力,且从服务器可以很容易地扩展为多台,使用LVS进行负载均衡(读写分离软件自身大多也有负载均衡的功能),效果就非常棒了,这就是传说中的数据库读写分离架构。上述架构的逻辑图如图14-7所示。

\par 3.根据业务重要性对多个从服务器进行拆分访问

\par 根据公司的业务,可以对几个不同的从服务器进行拆分。例如,有为外部用户提供浏览查询服务的从服务器,有内部DBA用来进行数据备份的从服务器,还有为公司内部人员提供访问的后台、脚本、财务统计、日志分析及供开发人员查询使用的从服务器。这样的拆分除了能够减轻主服务器的压力之外,还可以使得数据库对外部用户浏览、内部用户业务处理及DBA的备份操作等互不影响。具体如何对从服务器进行拆分可以用图14-8所示的简单架构来说明。


% From text00212.html


\subsection{14.1.3 MySQL主从读写分离实现方案}


\par (1)通过程序实现读写分离(需要程序支持)

\par PHP和Java程序都可以通过设置多个连接文件轻松地实现对数据库的读写分离,即当语句关键字为select时,就去连接读库的连接文件,若为update、insert、delete时,则连接写库的连接文件。

\begin{figure}[htbp]
\centering
\includegraphics[width=0.8\textwidth]{Image00113.jpg}
\end{figure}



\par 图14-7 一主多从读写分离及从库负载均衡架构逻辑图

\begin{figure}[htbp]
\centering
\includegraphics[width=0.8\textwidth]{Image00114.jpg}
\end{figure}



\par 图14-8 MySQL主从复制根据业务的重要性拆分从库的方案

\par 通过程序实现读写分离的缺点就是需要开发人员对程序进行改造,程序本身无法直接支持读写分离。

\par (2)通过开源的软件实现读写分离

\par Maxscale、Atlas、Mycat等代理软件也可以实现读写分离的功能,并且无须对应用程序做任何修改,而且它们还支持负载均衡等功能;缺点是又引入了单点服务,并且稳定性不如程序实现好。

\par (3)大型门户独立开发DAL层综合软件

\par 百度、阿里等大型门户都有开发牛人,会花大力气开发适合自己业务的读写分离、负载均衡、监控报警、自动扩容、自动收缩等一系列功能的DAL层软件,此部分可以参考老男孩架构师分布式数据库集群的课程内容。

\par MySQL读写分离的基本逻辑图如图14-9所示。

\begin{figure}[htbp]
\centering
\includegraphics[width=0.8\textwidth]{Image00115.jpg}
\end{figure}



\par 图14-9 MySQL读写分离的基本逻辑图


% From text00213.html


\subsection{14.1.4 MySQL主从复制原理}


\par MySQL的主从复制是一个异步的复制过程(虽然一般情况下感觉是实时的),数据将从一个MySQL数据库(我们称之为Master)复制到另一个MySQL数据库(我们称之为Slave),在Master与Slave之间实现整个主从复制的过程是由三个线程参与完成的。其中有两个线程(SQL线程和IO线程)在Slave端,另外一个线程(binlog dump线程)在Master端(MySQL 5及以前是3个线程完成复制,从MySQL 6起SQL线程可以是多个)。

\par 要实现MySQL的主从复制功能,首先必须打开Master端的binlog日志功能。因为整个复制过程实际上就是Slave从Master端获取binlog日志,然后再在Slave上以相同的顺序执行获取的binlog日志中所记录的各种SQL操作,从而实现主从数据一致的功能。

\par 可以通过MySQL的配置文件my.cnf中的mysqld模块([mysqld]标识后的参数部分)增加“log\_bin”参数选项来实现打开MySQL的binlog记录功能,具体信息如下:

\noindent\rule{\textwidth}{0.4pt}


\begin{verbatim}[mysqld]
log\_bin = /application/mysql/logs/oldboy-bin
\#<==路径/application/mysql/logs/要提前创建,路径非必须,oldboy-bin是binlog日志
       的前缀。
\end{verbatim}


\noindent\rule{\textwidth}{0.4pt}


\par 
\begin{figure}[htbp]
\centering
\includegraphics[width=0.8\textwidth]{Image00006.jpg}
\end{figure}

 提示:
 有同学因为将log\_bin放在了配置文件的结尾(其他模块标识的下面),而不是[mysqld]标识之后,从而导致配置复制不成功。


% From text00214.html


\subsection{14.1.5 MySQL主从复制原理及过程详细描述}


\par 下面简单描述下MySQL Replication复制的原理及过程。

\par 1)在Slave服务器上执行start slave命令开启主从复制开关,主从复制开始进行。

\par 2)此时,Slave服务器的I/O线程会通过在Master上已经授权的复制用户权限请求连接Master服务器,并请求从指定binlog日志文件的指定位置(日志文件名和位置就是在配置主从复制服务时执行change master命令指定的)之后开始发送binlog日志内容。

\par 3)Master服务器接收到来自Slave服务器的I/O线程的请求之后,其上负责复制的binlog dump线程会根据Slave服务器的I/O线程请求的信息,分批读取指定binlog日志文件所指定位置之后的binlog日志信息,然后返回给Slave端的I/O线程。返回的信息中除了binlog日志内容之外,还包括在Master服务器端记录的新的binlog文件名称,以及在新的binlog中的下一个指定的更新位置。

\par 4)当Slave服务器的I/O线程获取到Master服务器上I/O线程发送的日志内容及日志文件和位置点之后,会将binlog日志内容依次写入到Slave端自身的Relay Log(即中继日志)文件(MySQL-relay-bin.xxxxxx)的最末端,并将新的binlog文件名和位置记录到master-info文件中,以便下一次读取Master端新binlog日志时,能够告诉Master服务器需要从新binlog日志的指定文件及位置开始请求新的binlog日志内容。

\par 5)Slave服务器端的SQL线程会实时地检测本地Relay Log中I/O线程新增加的日志内容,然后及时地把Relay Log文件中的内容解析成SQL语句,并在自身Slave服务器上按解析SQL语句的位置顺序执行和应用这些SQL语句,并将当前应用中继日志的文件名及位置点记录在relay-log.info中。

\par 经过了上面的过程,就可以确保在Master端和Slave端执行了同样的SQL语句。在复制状态正常的情况下,Master端和Slave端的数据是完全一样的。当然,MySQL的复制机制也包含一些特殊的情况,具体请参考官方的说明,大多数情况下,大家是不用担心的。

\par 图14-10为MySQL Replication的复制原理逻辑图。

\begin{figure}[htbp]
\centering
\includegraphics[width=0.8\textwidth]{Image00116.jpg}
\end{figure}



\par 图14-10 MySQL主从复制基本原理逻辑图

\par 
\begin{figure}[htbp]
\centering
\includegraphics[width=0.8\textwidth]{Image00024.jpg}
\end{figure}

 特别说明:
 当企业面试MySQL主从复制原理时,不管是面试还是笔试,都要尽量画图表达,而不是口头讲或者通过文字描述,面试时可以找黑板或者拿出纸来向面试官详细讲解,自MySQL 5.6起,Slave服务器端的SQL线程可以是很多个,针对多个库可以并行执行SQL解析及应用操作。

\par 下面针对MySQL主从复制原理的重点进行小结。

\par ·主从复制是异步的逻辑的SQL语句级的复制。

\par ·复制时,主库有一个binlog dump线程,从库有两个线程,I/O线程和SQL线程。

\par ·从MySQL 5.6起,Slave服务器端的SQL线程可以是很多个。

\par ·实现主从复制的必要条件是主库要开启记录binlog的功能。

\par ·用于复制的所有MySQL节点配置中的server-id都不能相同。

\par ·binlog文件只记录对数据库有更改的SQL语句(来自主数据库内容的变更),不记录任何查询(select、show)以及未对数据库做出更改的语句。


% From text00215.html


\section{14.2 MySQL主从复制实践}


\subsection{
14.2.1 主从复制实践准备}
}


\par 1.主从复制数据库实战环境准备

\par MySQL主从复制实践对环境的要求比较简单,可以是单机单数据库多实例的环境,也可以是两台服务器,每台机器一个独立数据库的环境。本文以两台服务器,每台服务器上均安装本书第3章所讲的单机单实例数据库的环境为例进行讲解,单机单数据库多实例的环境讲解请参考《跟老男孩学Linux运维:Web集群实战》一书第9章的内容。

\par 
\begin{figure}[htbp]
\centering
\includegraphics[width=0.8\textwidth]{Image00006.jpg}
\end{figure}

 特别提示


\par 1.如果是VMware虚拟环境克隆虚拟机,则会导致网卡启动无法连网的问题,解决方法请参考\href{http://oldboy.blog.51cto.com/2561410/1363853}{http://oldboy.blog.51cto.com/2561410/1363853}
 。

\par 2.如果是克隆安装好的MySQL虚拟机,则需要注意数据目录下auto.cnf文件(本文是/application/mysql/data/auto.cnf)里的UUID要人为修改一下,使其与原始虚拟机不一致。

\par 2.定义主从复制需要的服务器角色

\par 主库及从库IP、端口信息如表14-1所示。

\par 表14-1 主库及从库IP和端口信息

\begin{figure}[htbp]
\centering
\includegraphics[width=0.8\textwidth]{Image00117.jpg}
\end{figure}



\par 3.数据库中英文名称约定

\par MySQL主库,也可称为Master,本文所对应服务的端口号为3306!

\par MySQL从库,也可称为Slave,本文所对应服务的端口号为3306!

\par 下面的内容中,可能把主库称为Master,把从库称为Slave,或者反过来称呼,代表的意思都是一样的。

\par 4.检查数据库的当前状态

\par 分别执行命令检查主从数据库的启动状态:

\noindent\rule{\textwidth}{0.4pt}


\begin{verbatim}[root@db01 \textasciitilde{}]\# lsof -i :3306
COMMAND  PID  USER   FD   TYPE DEVICE SIZE/OFF NODE NAME
mysqld  3044 mysql   11u  IPv6  15565      0t0  TCP *:mysql (LISTEN)
[root@db02 \textasciitilde{}]\# lsof -i :3306
COMMAND  PID  USER   FD   TYPE DEVICE SIZE/OFF NODE NAME
mysqld  3028 mysql   11u  IPv6  15422      0t0  TCP *:mysql (LISTEN)
\end{verbatim}


\noindent\rule{\textwidth}{0.4pt}


\par 根据结果可以确定数据库已处于正常启动状态。


% From text00216.html


\subsection{14.2.2 在主库Master(51)上执行操作配置}


\par 1.设置server-id的值并开启binlog功能参数

\par 根据前文介绍的MySQL主从复制原理我们可以知道,要实现主从复制,关键是要开启binlog日志功能,所以,首先我们需要打开主库的binlog日志参数。

\par 1)修改主库的配置文件。执行vi/etc/my.cnf,编辑MySQL的配置文件,两个参数按如下内容进行修改:

\noindent\rule{\textwidth}{0.4pt}


\begin{verbatim}[mysqld]
server\_id = 1  \#<==用于同步的每台机器或实例server\_id都不能相同。
log\_bin = /application/mysql/logs/oldboy-bin   \#<==红色加粗可省略,本文是与前面
                                               的章节保持一致。
特别提示:从MySQL5.6起,参数中改用"\_"代替"-"了,但是常规情况下后面这种写法依然兼容。
\end{verbatim}


\noindent\rule{\textwidth}{0.4pt}


\par 
\begin{figure}[htbp]
\centering
\includegraphics[width=0.8\textwidth]{Image00006.jpg}
\end{figure}

 提示:


\par ·上面的两个参数要放在my.cnf中的[mysqld]模块下,否则会出错。

\par ·server-id的值使用服务器IP地址最后一个小数点后面的数字如19,目的是避免不同的机器或实例ID重复(不适合多实例)。server-id的取值范围为0<server-id<232
 -1的自然数。

\par ·首先要在my.cnf配置文件中查找相关参数,并按要求进行修改。若发现不存在,再添加参数,切记,参数不能重复。

\par ·修改my.cnf配置后需要重启数据库,注意要确认已经真正重启了。

\par 2)检查配置参数之后的结果:

\noindent\rule{\textwidth}{0.4pt}


\begin{verbatim}[root@db01 \textasciitilde{}]\# egrep "server\_id|log\_bin" /etc/my.cnf
log\_bin = /application/mysql/logs/oldboy-bin \#<==后面可以不带等号部分,MySQL会
                                             使用默认路径及日志名。
server\_id = 1
\end{verbatim}


\noindent\rule{\textwidth}{0.4pt}


\par 3)重启主库MySQL服务:

\noindent\rule{\textwidth}{0.4pt}


\begin{verbatim}[root@db01 \textasciitilde{}]\# /etc/init.d/mysqld restart
Shutting down MySQL.. SUCCESS!
Starting MySQL. SUCCESS!
\end{verbatim}


\noindent\rule{\textwidth}{0.4pt}


\par 4)登录数据库检查参数的更改情况:

\noindent\rule{\textwidth}{0.4pt}


\begin{verbatim}[root@db01 \textasciitilde{}]\# mysql -e "show variables like 'log\_bin';" \#<==不登入数据库检查,
                                                       密码已写入配置文件。
+---------------+-------+
| Variable\_name | Value |
+---------------+-------+
| log\_bin       | ON    |  \#<==Binlog功能已开启
+---------------+-------+
[root@db01 \textasciitilde{}]\# mysql -e "show variables like 'server\_id';"
+---------------+-------+
| Variable\_name | Value |
+---------------+-------+
| server\_id     | 1     | \#<==通过检查可以得知配置的server\_id为1
+---------------+-------+
提示:通过使用show variables在数据库里查询变量的方式是真正确认数据库参数是否生效的关键。
\end{verbatim}


\noindent\rule{\textwidth}{0.4pt}


\par 到此,主库的binlog功能就开启了。

\par 2.在主库上建立用于主从复制的账号

\par 根据主从复制的原理,从库要想与主库同步,必须要有一个可以连接主库的账号,并且这个账号是在主库上创建的,权限是允许从库连接主库并同步数据。

\par 1)登录主数据库(172.16.1.51),命令如下:

\noindent\rule{\textwidth}{0.4pt}


\begin{verbatim}[root@db01 \textasciitilde{}]\# mysql \#<==为了操作安全方便,密码已写入my.cnf配置文件。
\end{verbatim}


\noindent\rule{\textwidth}{0.4pt}


\par 2)建立用于从库复制的账号及对应的权限:

\noindent\rule{\textwidth}{0.4pt}


\begin{verbatim}mysql> grant replication slave on *.* to 'rep'@'172.16.1.\%' identified by 'oldboy123';
Query OK, 0 rows affected (0.01 sec)
\#replication slave为允许MySQL Slave同步的必需权限,此处不要授权all权限。
\# “*.*”表示所有库所有表,也可以指定具体的库和表进行复制。例如oldboy.test中,oldboy为库,test为表。
\# 'rep'@'172.16.1.\%'中的rep为同步账号。172.16.1.\%为授权主机网段,使用了“\%”表示允许整个172.16.1.0网段以rep用户访问。
\# identified by 'oldboy123'中的oldboy123为密码,在实际环境中还是复杂一点为好。
mysql> flush privileges; \#<==刷新权限使得授权的权限生效,对账号进行更改一般需要此
                           命令,此处是为养成习惯而执行,而不是必须执行。
Query OK, 0 rows affected (0.02 sec)
\end{verbatim}


\noindent\rule{\textwidth}{0.4pt}


\par 3)检查主库创建的rep复制账号:

\noindent\rule{\textwidth}{0.4pt}


\begin{verbatim}mysql> select user,host from mysql.user; \#<==检查账号命令。
+------+------------+
| user | host       |
+------+------------+
| root | 127.0.0.1  |
| rep  | 172.16.1.\% | \#<==出现这一行表示复制账号已经配置好了。
| root | localhost  |
+------+------------+
3 rows in set (0.00 sec)
mysql> show grants for rep@'172.16.1.\%'; \#<==查看授权权限命令。
+------------------------------------------------------------------------+
| GRANT REPLICATION SLAVE ON *.* TO 'rep'@'172.16.1.\%' IDENTIFIED BY PASSWORD '*FE28814B4A8B3309DAC6ED7D3237ADED6DA1E515' | \#<==结果显示,rep账号权限正是配置的主
                                          从复制所需要的权限。
REPLICATION SLAVE
+------------------------------------------------------------------------+
\end{verbatim}


\noindent\rule{\textwidth}{0.4pt}


\par 3.对主数据库锁表只读后进行备份

\par 1)对主数据库锁表只读(当前窗口不要关掉)的命令如下:

\noindent\rule{\textwidth}{0.4pt}


\begin{verbatim}mysql> flush table with read lock;
Query OK, 0 rows affected (0.00 sec)
\end{verbatim}


\noindent\rule{\textwidth}{0.4pt}


\par 
\begin{figure}[htbp]
\centering
\includegraphics[width=0.8\textwidth]{Image00006.jpg}
\end{figure}

 提示:
 对于不同引擎的情况,这个锁表命令的时间会受到下面参数的控制。锁表时,如果超过了设置的时间则会自动解锁。

\par 默认情况下,自动解锁的时长参数值设置如下:

\noindent\rule{\textwidth}{0.4pt}


\begin{verbatim}mysql> show variables like '\%timeout\%';
\#<==截取部分关键内容如下
+----------------------------+-------+
| Variable\_name              | Value |
+----------------------------+-------+
| interactive\_timeout        | 28800 |
| wait\_timeout               | 28800 |
+----------------------------+-------+
10 rows in set (0.00 sec)
\end{verbatim}


\noindent\rule{\textwidth}{0.4pt}


\par 
\begin{figure}[htbp]
\centering
\includegraphics[width=0.8\textwidth]{Image00006.jpg}
\end{figure}

 提示:
 关于这两个参数的具体详情,请读者自行测试。

\par 2)锁表后查看主库的状态:

\noindent\rule{\textwidth}{0.4pt}


\begin{verbatim}mysql> show master status;
+-------------------+----------+--------------+------------------+-------------------+
| File              | Position | Binlog\_Do\_DB | Binlog\_Ignore\_DB | Executed\_Gtid\_Set |
+-------------------+----------+--------------+------------------+-------------------+
| oldboy-bin.000023 |      626 |              |                  |                   |
+-------------------+----------+--------------+------------------+-------------------+
1 row in set (0.00 sec)
\end{verbatim}


\noindent\rule{\textwidth}{0.4pt}


\par 最关键的是前两项,由上述代码可以看到当前binlog日志文件名和二进制binlog日志偏移量的位置。

\par 注意,“show master status;”命令显示的信息需要记录在案,将后面的从库导入全备后,继续对主库进行复制时就是要从这个文件以及对应的位置开始(MySQL 5.6版本支持GTID功能,关于此功能,后文会详细讲解)。

\par 3)导出主库数据

\par 锁表后,一定要单开一个新的SSH窗口(否则锁表会失效),导出数据库中的所有数据,如果数据量很大(30GB以上),并且允许停机,则也可以停库直接打包数据文件进行迁移,那样做速度更快,使用第9章提供的Xtrabackup进行热备份也可以。

\noindent\rule{\textwidth}{0.4pt}


\begin{verbatim}[root@db01 \textasciitilde{}]\# mkdir /server/backup -p  \#<==创建备份目录。
[root@db01 \textasciitilde{}]\# mysqldump -A -B |gzip >/server/backup/bak\_\$(date +\%F).sql.gz
\#<==实施备份导出数据。
[root@db01 \textasciitilde{}]\# ls -l /server/backup/bak\_\$(date +\%F).sql.gz \#<==检查备份。
-rw-r--r--. 1 root root 215482 9月  12 22:12 /server/backup/bak\_2017-09-12. sql.gz
\end{verbatim}


\noindent\rule{\textwidth}{0.4pt}


\par 为了确保导出数据期间,数据库没有数据插入,导库完毕后可以再检查下主库的状态信息:

\noindent\rule{\textwidth}{0.4pt}


\begin{verbatim}mysql> show master status;
+-------------------+----------+--------------+------------------+-------------------+
| File              | Position | Binlog\_Do\_DB | Binlog\_Ignore\_DB | Executed\_Gtid\_Set |
+-------------------+----------+--------------+------------------+-------------------+
| oldboy-bin.000023 |      626 |              |                  |                   |
+-------------------+----------+--------------+------------------+-------------------+
1 row in set (0.00 sec)
\end{verbatim}


\noindent\rule{\textwidth}{0.4pt}


\par 
\begin{figure}[htbp]
\centering
\includegraphics[width=0.8\textwidth]{Image00006.jpg}
\end{figure}

 提示:
 若无特殊情况,binlog文件及位置点与锁表后导出数据前是一致的,即没有发生变化,如果使用了-F参数导出数据,因为会切割日志文件,因此上述状态可能会发生变化,但是两次状态变化之间并没有数据写入。

\par 导出数据完毕后,解锁主库,恢复可写,因为主库还要对外提供服务,因此不能一直锁定不让用户访问:

\noindent\rule{\textwidth}{0.4pt}


\begin{verbatim}mysql> unlock tables;
\end{verbatim}


\noindent\rule{\textwidth}{0.4pt}


\par 可能会有读者因为锁表后的binlog位置问题而犯迷糊,实际上在做从库之前,无论主库更新了多少数据,最后从库都可以从上面show master status的位置很快赶上主库的进度。

\par 4.把从主库导出的MySQL数据迁移到从库

\par 这里常用的命令有scp、rsync等,可将备份的数据复制到异地。

\par 本文采用scp进行复制,命令如下:

\noindent\rule{\textwidth}{0.4pt}


\begin{verbatim}[root@db01 \textasciitilde{}]\# scp -rp /server/backup/bak\_\$(date +\%F).sql.gz root@172.16.1.52:/opt
The authenticity of host '172.16.1.52 (172.16.1.52)' can't be established.
RSA key fingerprint is a3:42:ca:c2:e6:a9:cb:0b:9f:aa:8a:23:15:2c:c6:87.
Are you sure you want to continue connecting (yes/no)? yes
Warning: Permanently added '172.16.1.52' (RSA) to the list of known hosts.
reverse mapping checking getaddrinfo for bogon [172.16.1.52] failed - POSSIBLE BREAK-IN ATTEMPT!
root@172.16.1.52's password:
bak\_2017-09-12.sql.gz                          100\%  210KB 210.4KB/s   00:00
\end{verbatim}


\noindent\rule{\textwidth}{0.4pt}


\par 若出现如上提示,则表示数据库备份已经复制到172.16.1.52从数据库服务器的/opt目录下了。


% From text00217.html


\subsection{14.2.3 在MySQL从库上执行的操作过程}


\par 1.设置server\_id的值并关闭binlog功能参数

\par 一般来说,数据库的server\_id在一套主从复制体系内是唯一的,这里从库的server\_id必须与主库及其他的从库不同,并且需要注释掉从库的binlog参数配置。如果不对从库进行级联复制,并且不作为备份使用,就不要开启binlog了,开启了反而会增加从库磁盘I/O的压力。

\par 但是,如下两种情况需要打开从库的binlog记录功能,记录数据库更新的SQL语句:

\par ·作为形如A→B→C的级联同步,中间的B数据库服务,需要开启binlog记录功能。

\par ·在从库中做数据库备份时需要开启binlog记录功能。因为数据库备份必须要有全备和binlog日志,才是完整的备份。

\par 1)修改配置文件,配置从库(172.16.1.52)的相关参数。

\par 执行vi/etc/my.cnf,编辑my.cnf配置文件,按如下两个参数对内容进行修改:

\noindent\rule{\textwidth}{0.4pt}


\begin{verbatim}[mysqld]
server\_id = 2  \#<==调整等号后的数值,与任何一个数据库实例都不相同。
\#log\_bin = /application/mysql/logs/oldboy-bin  \#<==如果有log\_bin参数一行就注释掉,如果配置没有此行则可以忽略。
\end{verbatim}


\noindent\rule{\textwidth}{0.4pt}


\par 2)检查配置参数后的结果,命令如下:

\noindent\rule{\textwidth}{0.4pt}


\begin{verbatim}[root@db02 \textasciitilde{}]\# egrep "server\_id|log\_bin" /etc/my.cnf
server\_id = 2
log\_bin = /application/mysql/logs/oldboy-bin
\end{verbatim}


\noindent\rule{\textwidth}{0.4pt}


\par 3)重启从数据库:

\noindent\rule{\textwidth}{0.4pt}


\begin{verbatim}[root@db02 \textasciitilde{}]\# /etc/init.d/mysqld restart
Shutting down MySQL.. SUCCESS!
Starting MySQL. SUCCESS!
\end{verbatim}


\noindent\rule{\textwidth}{0.4pt}


\par 4)登录数据库检查参数的改变情况:

\noindent\rule{\textwidth}{0.4pt}


\begin{verbatim}[root@db02 \textasciitilde{}]\# mysql -e "show variables like 'log\_bin';" \#<==不登入数据库检查,密码已写入配置文件。
+---------------+-------+
| Variable\_name | Value |
+---------------+-------+
| log\_bin       | OFF   |  \#<==Binlog功能已关闭。
+---------------+-------+
[root@db02 \textasciitilde{}]\# mysql -e "show variables like 'server\_id';"
+---------------+-------+
| Variable\_name | Value |
+---------------+-------+
| server\_id     | 2     |  \#<==通过检查得知配置的server\_id为2。
+---------------+-------+
\end{verbatim}


\noindent\rule{\textwidth}{0.4pt}


\par 2.将从主库mysqldump中导出的数据恢复到从库

\par 在做主从复制之前,我们需要让主库与从库的数据保持一致,因此需要将主库导出的数据全恢复到从库之后再设置主从复制。

\par 操作命令如下:

\noindent\rule{\textwidth}{0.4pt}


\begin{verbatim}[root@db02 opt]\# zcat /opt/bak\_2017-09-12.sql.gz|mysql \#<==直接读取压缩包恢复数据,不能有报错。
\end{verbatim}


\noindent\rule{\textwidth}{0.4pt}


\par 
\begin{figure}[htbp]
\centering
\includegraphics[width=0.8\textwidth]{Image00006.jpg}
\end{figure}

 提示:
 如果备份时使用了“-A”参数,则在还原数据到其他从库时,登录的root密码也会与主库一致,因为授权表也被主库数据覆盖了。

\par 3.登录从库(52)配置复制参数

\par 1)MySQL从库连接主库的配置信息如下:

\noindent\rule{\textwidth}{0.4pt}


\begin{verbatim}CHANGE MASTER TO
MASTER\_HOST='172.16.1.51',  \#<==这里是主库的IP。
MASTER\_PORT=3306,           \#<==这里是主库的端口,从库的端口可以与主库的不同。
MASTER\_USER='rep',          \#<==这里是主库上建立的用于复制的用户rep。
MASTER\_PASSWORD='oldboy123', \#<==这里是rep用户的密码。
MASTER\_LOG\_FILE='oldboy-bin.000023', \#<==这里是show master status时查看到的
                                      二进制日志文件名称,注意不能多出空格。
MASTER\_LOG\_POS=626; \#<==这里是show master status时查看到的二进制日志偏移量,注意
                         不能多出空格。
\end{verbatim}


\noindent\rule{\textwidth}{0.4pt}


\par 
\begin{figure}[htbp]
\centering
\includegraphics[width=0.8\textwidth]{Image00006.jpg}
\end{figure}

 提示:
 字符串用单引号括起来,数值不用引号,注意内容前后不能有空格。

\par 2)登录从数据库后,应去掉上述语句中注释的部分,执行代码如下:

\noindent\rule{\textwidth}{0.4pt}


\begin{verbatim}CHANGE MASTER TO
MASTER\_HOST='172.16.1.51',
MASTER\_PORT=3306,
MASTER\_USER='rep',
MASTER\_PASSWORD='oldboy123',
MASTER\_LOG\_FILE='oldboy-bin.000023',
MASTER\_LOG\_POS=626;
\end{verbatim}


\noindent\rule{\textwidth}{0.4pt}


\par 
\begin{figure}[htbp]
\centering
\includegraphics[width=0.8\textwidth]{Image00006.jpg}
\end{figure}

 提示:
 这个步骤的参数信息一定不能出错,否则,数据库复制会前功尽弃。

\par 也可以不登录数据库内部的命令行,在Linux命令行快速执行CHANGE MASTER的语句可以实现相应的功能,下面的语句特别适合在脚本中一键配置从库使用,语句如下:

\noindent\rule{\textwidth}{0.4pt}


\begin{verbatim}    mysql<< EOF
CHANGE MASTER TO
    MASTER\_HOST='172.16.1.51',
    MASTER\_PORT=3306,
    MASTER\_USER='rep',
    MASTER\_PASSWORD='oldboy123',
    MASTER\_LOG\_FILE='oldboy-bin.000023',
    MASTER\_LOG\_POS=626;
EOF
\end{verbatim}


\noindent\rule{\textwidth}{0.4pt}


\par 配置MySQL从库连接主库信息操作的整个过程如下:

\noindent\rule{\textwidth}{0.4pt}


\begin{verbatim}[root@db02 opt]\# mysql<< EOF
> CHANGE MASTER TO
> MASTER\_HOST='172.16.1.51',
> MASTER\_PORT=3306,
> MASTER\_USER='rep',
> MASTER\_PASSWORD='oldboy123',
> MASTER\_LOG\_FILE='oldboy-bin.000023',
> MASTER\_LOG\_POS=626;
> EOF
\end{verbatim}


\noindent\rule{\textwidth}{0.4pt}


\par 上述操作过程的原理实际上是把用户密码等信息写入从库新的master.info文件中:

\noindent\rule{\textwidth}{0.4pt}


\begin{verbatim}[root@db02 opt]\# ls -l /application/mysql/data/master.info
-rw-rw----. 1 mysql mysql 91 9月  12 23:16 /application/mysql/data/master.info
[root@db02 opt]\# cat /application/mysql/data/master.info
23
oldboy-bin.000023 \#<==这里是show master status时查看到的二进制日志文件名称。
626               \#<==这里是show master status时查看到的二进制日志偏移量。
172.16.1.51   \#<==这里是主库的IP。
rep           \#<==这里是主库上建立的用于复制的用户rep。
oldboy123     \#<==这里是rep用户的密码。
3306          \#<==这里是主库的端口。
…省略若干…
\end{verbatim}


\noindent\rule{\textwidth}{0.4pt}



% From text00218.html


\subsection{14.2.4 启动从库同步开关并测试主从复制}


\par 1)启动从库主从复制开关,并查看复制状态。相关语句如下:

\noindent\rule{\textwidth}{0.4pt}


\begin{verbatim} [root@db02 opt]\# mysql -e "start slave;"
[root@db02 opt]\# mysql -e "show slave status\textbackslash\{\}G;"
*************************** 1. row ***************************
               Slave\_IO\_State: Waiting for master to send event
                  Master\_Host: 172.16.1.51
                  Master\_User: rep
                  Master\_Port: 3306
                Connect\_Retry: 60
              Master\_Log\_File: oldboy-bin.000023
          Read\_Master\_Log\_Pos: 626
               Relay\_Log\_File: db02-relay-bin.000002
                Relay\_Log\_Pos: 284
        Relay\_Master\_Log\_File: oldboy-bin.000023
             Slave\_IO\_Running: Yes
            Slave\_SQL\_Running: Yes
              Replicate\_Do\_DB:
          Replicate\_Ignore\_DB:
           Replicate\_Do\_Table:
       Replicate\_Ignore\_Table:
      Replicate\_Wild\_Do\_Table:
  Replicate\_Wild\_Ignore\_Table:
                   Last\_Errno: 0
                   Last\_Error:
                 Skip\_Counter: 0
          Exec\_Master\_Log\_Pos: 626
              Relay\_Log\_Space: 456
              Until\_Condition: None
               Until\_Log\_File:
                Until\_Log\_Pos: 0
           Master\_SSL\_Allowed: No
           Master\_SSL\_CA\_File:
           Master\_SSL\_CA\_Path:
              Master\_SSL\_Cert:
            Master\_SSL\_Cipher:
               Master\_SSL\_Key:
        Seconds\_Behind\_Master: 0
Master\_SSL\_Verify\_Server\_Cert: No
                Last\_IO\_Errno: 0
                Last\_IO\_Error:
               Last\_SQL\_Errno: 0
               Last\_SQL\_Error:
  Replicate\_Ignore\_Server\_Ids:
             Master\_Server\_Id: 1
                  Master\_UUID: 4c92bf66-0f0b-11e7-ba4a-000c292ece3d
             Master\_Info\_File: /application/mysql/data/master.info
                    SQL\_Delay: 0
          SQL\_Remaining\_Delay: NULL
Slave\_SQL\_Running\_State: Slave has read all relay log; waiting for the slave I/O thread to update it
           Master\_Retry\_Count: 86400
                  Master\_Bind:
      Last\_IO\_Error\_Timestamp:
     Last\_SQL\_Error\_Timestamp:
               Master\_SSL\_Crl:
           Master\_SSL\_Crlpath:
           Retrieved\_Gtid\_Set:
            Executed\_Gtid\_Set:
                Auto\_Position: 0

Master\_SSL\_Verify\_Server\_Cert: No
                Last\_IO\_Errno: 0
                Last\_IO\_Error:
               Last\_SQL\_Errno: 0
               Last\_SQL\_Error:
  Replicate\_Ignore\_Server\_Ids:
             Master\_Server\_Id: 1
提示:有关show slave status结果的详细说明,请大家查看MySQL手册。
\end{verbatim}


\noindent\rule{\textwidth}{0.4pt}


\par 主从复制是否配置成功了,最关键的是查看下面3项状态参数:

\noindent\rule{\textwidth}{0.4pt}


\begin{verbatim}[root@db02 opt]\# mysql -e "show slave status\textbackslash\{\}G;"|egrep "IO\_Running|SQL\_Running: |\_Behind\_Master"
             Slave\_IO\_Running: Yes
            Slave\_SQL\_Running: Yes
        Seconds\_Behind\_Master: 0
\end{verbatim}


\noindent\rule{\textwidth}{0.4pt}


\par ·Slave\_IO\_Running:Yes,这个表示I/O的线程状态,I/O线程负责从主库中读取Binlog日志,并将Binlog日志写入从库的中继日志中,状态为Yes表示I/O线程工作正常,否则异常。

\par ·Slave\_SQL\_Running:Yes,这个表示SQL的线程状态,SQL线程负责读取中继日志(relay-log)中的数据并转换为SQL语句应用到从数据库中,状态为Yes表示I/O线程工作正常,否则异常。

\par ·Seconds\_Behind\_Master:0,这个表示在复制过程中,从库比主库延迟的秒数,这个参数很重要,但企业里有更准确地判断主从延迟的方法:在主库中写时间戳,然后通过从库读取时间戳,与当前数据库时间进行比较,从而认定是否真的延迟。

\par 2)测试主从复制结果。在主库中写入数据,然后观察从库的数据状况:

\noindent\rule{\textwidth}{0.4pt}


\begin{verbatim}[root@db01 \textasciitilde{}]\# mysql -e "create database alex\_python;"  \#<==注意提示符:这是
                                                       主库建库操作。
[root@db02 \textasciitilde{}]\# mysql -e "show databases like 'alex\%';"   \#<==这是从库查询检查。
+------------------+
| Database (alex\%) |
+------------------+
| alex\_python      |
+------------------+
\end{verbatim}


\noindent\rule{\textwidth}{0.4pt}


\par 根据测试可以判断,从库(172.16.1.52)和主库(172.16.1.51)是同步的,到此,主从复制的基本实践就已经全部完成了。


% From text00219.html


\subsection{14.2.5 MySQL主从复制问题汇总}


\par 问题1:主库“show master status;”没有返回状态结果。

\noindent\rule{\textwidth}{0.4pt}


\begin{verbatim}mysql> show master status;
Empty set (0.00 sec)
\end{verbatim}


\noindent\rule{\textwidth}{0.4pt}


\par 解答:上述问题的原因是主库binlog功能开关没开或是没生效。

\par 问题2:出现错误信息“Last\_IO\_Error:Got fatal error 1236 from master when reading data from binary log:'Could not find first log file name in binary log index file'”。

\par 解答:上面故障的原因是执行CHANGE MASTER命令时某一个参数的值多出了空格,因而产生了错误,如下:

\noindent\rule{\textwidth}{0.4pt}


\begin{verbatim}CHANGE MASTER TO
MASTER\_HOST='172.16.1.51',
MASTER\_PORT=3306,
MASTER\_USER='rep',
MASTER\_PASSWORD='oldboy123',
MASTER\_LOG\_FILE=' oldboy-bin.000023 ', \#<== oldboy-bin.000023文件两端不能有空格。
MASTER\_LOG\_POS=626;
\end{verbatim}


\noindent\rule{\textwidth}{0.4pt}


\par 问题3:MySQL服务无法启动。

\par 故障语句如下:

\noindent\rule{\textwidth}{0.4pt}


\begin{verbatim}[root@db01 \textasciitilde{}]\# /etc/init.d/mysqld start
 ERROR! MySQL server PID file could not be found!
\end{verbatim}


\noindent\rule{\textwidth}{0.4pt}


\par 解决:上述问题的原因有很多种可能,具体如下。

\par 1)数据库目录权限和用户属主问题,数据库目录使用mysql用户。

\par 2)my.cnf配置出错,需要调整my.cnf中有问题的配置。

\par 3)数据库初始化时有问题,需要重新初始化数据库。


% From text00220.html


\subsection{14.2.6 MySQL主从复制配置步骤小结}


\par MySQL主从复制配置的完整步骤具体如下。

\par 1)准备两台数据库环境,关闭Selinux和Iptables防火墙,确定其能正常启动和登录。

\par 2)配置my.cnf文件:主库配置log\_bin和server\_id参数;从库配置server\_id参数,该值不能像主库及其他从库一样,从库一般不开启log\_bin功能。配置参数后需要重启才能生效。

\par 3)登录主库增加用于同步的账户,例如rep,并授权replication slave同步的权限。

\par 4)登录主库,整库锁表flush table with read lock(窗口关闭后即失效,超时参数设置的时间到了,锁表也会失效),然后show master status查看binlog的位置状态。

\par 5)新开窗口,在Linux命令行备份导出原有的数据库数据,并复制到从库所在的服务器目录。如果数据库数据量很大,则使用热备工具Xtrabackup或停机打包,而不用mysqldump。

\par 6)导出主库中数据之后,即可刻执行unlock tables解锁主库。

\par 7)把从主库中导出的数据恢复到从库。

\par 8)根据主库的show master status查看到的binlog的位置状态,在从库中执行“change master to...”语句。

\par 9)从库开启复制开关,即执行“start slave;”。

\par 10)从库“show slave status\textbackslash\{\}G”,检查同步状态(两个Yes和一个延迟秒数是否为0),并在主库中进行更新,在从库中检查测试复制是否成功。


% From text00221.html


\subsection{14.2.7 MySQL主从复制线程状态说明及用途}


\par 1.MySQL主从复制主库I/O线程状态说明

\par 1)登录主数据库查看MySQL线程的同步状态:

\noindent\rule{\textwidth}{0.4pt}


\begin{verbatim}mysql>  show processlist\textbackslash\{\}G  \#<==查看主库所有线程的状态。
*************************** 1. row ***************************
     Id: 1
   User: rep
   Host: 172.16.1.52:39041
     db: NULL
Command: Binlog Dump  \#<==用于复制的主库线程。
   Time: 755
  State: Master has sent all binlog to slave; waiting for binlog to be updated
  \#<==当前状态
   Info: NULL
1 rows in set (0.00 sec)
\end{verbatim}


\noindent\rule{\textwidth}{0.4pt}


\par 
\begin{figure}[htbp]
\centering
\includegraphics[width=0.8\textwidth]{Image00006.jpg}
\end{figure}

 提示:
 上述状态的意思是线程已经从binlog日志中读取到了所有的更新,并已经将更新发送到了从数据库服务器。线程现在为空闲状态,等待主服务器上二进制日志中的新事件更新。

\par 表14-2中列出了主服务器的binlog Dump线程的State列的最常见状态。如果你没有在主服务器上看见任何binlog Dump线程,则说明复制没有运行,二进制binlog日志由各种事件组成,一个事件通常会为一个更新添加一些其他的信息。

\par 表14-2 主库I/O线程工作状态

\begin{figure}[htbp]
\centering
\includegraphics[width=0.8\textwidth]{Image00118.jpg}
\end{figure}



\par 2)登录从数据库查看MySQL线程的工作状态,从库有两个线程,即I/O和SQL线程。

\par 下面是从库I/O线程的状态。

\noindent\rule{\textwidth}{0.4pt}


\begin{verbatim}mysql> show processlist\textbackslash\{\}G
*************************** 1. row ***************************
     Id: 4
   User: system user
   Host:
     db: NULL
Command: Connect
   Time: 2044
  State: Waiting for master to send event
   Info: NULL
\end{verbatim}


\noindent\rule{\textwidth}{0.4pt}


\par 表14-3列出了从服务器的I/O线程的State列的最常见的状态。该状态也出现在Slave\_IO\_State列,由SHOW SLAVE STATUS显示。

\par 表14-3 从库I/O线程的工作状态

\begin{figure}[htbp]
\centering
\includegraphics[width=0.8\textwidth]{Image00119.jpg}
\end{figure}



\par 下面是从库SQL线程的状态:

\noindent\rule{\textwidth}{0.4pt}


\begin{verbatim}*************************** 2. row ***************************
     Id: 5
   User: system user
   Host:
     db: NULL
Command: Connect
   Time: 1600
  State: Slave has read all relay log; waiting for the slave I/O thread to update it
   Info: NULL
\end{verbatim}


\noindent\rule{\textwidth}{0.4pt}


\par 表14-4列出了从服务器的SQL线程其State列的最常见的状态。

\par 表14-4 从库SQL线程状态及说明

\begin{figure}[htbp]
\centering
\includegraphics[width=0.8\textwidth]{Image00120.jpg}
\end{figure}



\par 有关MySQL主从复制参与线程状态的更多信息,请参考MySQL官方手册。

\par 2.查看MySQL线程同步状态的用途

\par 通过MySQL线程同步状态可以看到同步是否正常进行,故障的位置是什么,另外还可以查看数据库同步是否完成,可用于主库宕机切换数据库或者人工数据库主从切换迁移等。

\par 例如,主库宕机,要选择最快的从库将其提升为主库,就需要查看主从库的线程状态,如果是主从复制在正常情况下进行角色切换,也需要查看主从库的线程状态,根据复制状态确定数据库更新是否完成。


% From text00222.html


\subsection{14.2.8 生产场景中部署MySQL主从复制方案}


\par 1.快速配置MySQL主从复制

\par 快速配置MySQL主从复制的步骤具体如下。

\par 1)安装好要配置的从库的数据库,配置好log\_bin和server\_id参数。

\par 2)无须配置主库my.cnf文件,主库的log\_bin和server\_id参数默认就是配置好的。

\par 3)登录主库,增加从库连接主库同步的账户,例如rep,并授权replication slave同步的权限。

\par 4)使用曾经在半夜通过mysqldump带“-x”和“--master-data=1”的命令及参数
 定时备份的全备数据备份文件,把它恢复到从库。

\par 5)在从库中执行“change master to...”语句,无须binlog文件及对应位置点。

\par 6)从库开启同步开关,start slave。

\par 7)在从库中执行“show slave status\textbackslash\{\}G”,检查同步状态,并在主库中进行更新测试。

\par 2.无须熬夜,轻松部署MySQL主从复制

\par 实战过程如下。

\par 1)在主库上通过定时任务使其半夜执行如下所示的命令,备份并导出主库数据:

\noindent\rule{\textwidth}{0.4pt}


\begin{verbatim}mysqldump -A -B -x --master-data=1|gzip >/opt/bak1\_\$(date +\%F).sql.gz
\end{verbatim}


\noindent\rule{\textwidth}{0.4pt}


\par “--master-data=1”参数会在备份数据里增加如下语句:

\noindent\rule{\textwidth}{0.4pt}


\begin{verbatim}-- Position to start replication or point-in-time recovery from
CHANGE MASTER TO MASTER\_LOG\_FILE='oldboy-bin.000024', MASTER\_LOG\_POS=107;
\end{verbatim}


\noindent\rule{\textwidth}{0.4pt}


\par 2)白天找机会在需要做复制的从库上导入全备

\noindent\rule{\textwidth}{0.4pt}


\begin{verbatim}zcat /opt/bak1\_\$(date +\%F).sql.gz|mysql
mysql << EOF
CHANGE MASTER TO
MASTER\_HOST='172.16.1.51',
MASTER\_PORT=3306,
MASTER\_USER='rep',
MASTER\_PASSWORD='oldboy123';
EOF
提示:这里忽略了两个参数MASTER\_LOG\_FILE和MASTER\_LOG\_POS,这是因为在备份的时候使用了“--master-data=1”,在导入数据库时,这两个参数已经自动设置好了。
\end{verbatim}


\noindent\rule{\textwidth}{0.4pt}


\par 这里的CHANGE MASTER后面无须指定binlog文件名及具体位置,因为这部分已经在还原数据时提前应用到数据库里了(备份时使用“--master-data=1”的功劳)。

\par 最后在从库上开启复制开关,检验成果:

\noindent\rule{\textwidth}{0.4pt}


\begin{verbatim}start slave;         \#<==开启主从复制开关。
show slave status\textbackslash\{\}G  \#<==查看主从复制状态。
\end{verbatim}


\noindent\rule{\textwidth}{0.4pt}



% From text00223.html


\section{14.3 MySQL主从复制在企业中的故障案例}


\par 
工作中MySQL从库停止复制的故障案例


\par 模拟并重现故障的能力是运维人员最重要的能力。下面就来进行模拟操作。先在从库中创建一个库,然后在主库中创建同名的库来模拟数据冲突,代码如下:

\noindent\rule{\textwidth}{0.4pt}


\begin{verbatim}show slave status\textbackslash\{\}G:结果为:
Slave\_IO\_Running: Yes
Slave\_SQL\_Running: NO
Seconds\_Behind\_Master: NULL
                   Last\_Error: Error 'Can't create database 'xiaoliu'; database exists' on query. Default database: 'xiaoliu'. Query: 'create database xiaoliu'
\end{verbatim}


\noindent\rule{\textwidth}{0.4pt}


\par 对于该冲突,解决方法1为:

\noindent\rule{\textwidth}{0.4pt}


\begin{verbatim}stop slave; \#<==临时停止同步开关。
set global sql\_slave\_skip\_counter =1 ; \#<==将同步指针向下移动一个,如果多次不同步,
                                       则可以重复操作。
start slave; \#<==开启同步开关。
\end{verbatim}


\noindent\rule{\textwidth}{0.4pt}


\par 对于普通的互联网业务,上述移动指针的命令操作所带来的问题不是很大。当然,要在确认不影响公司业务的前提下进行上述操作。

\par 若是在企业场景下,对当前业务来说,解决主从同步问题比主从数据不一致问题更重要,如果主从数据一致也是很重要的,那么就再找个时间来恢复下这个从库。

\par 主从数据不一致更重要还是保持主从同步持续状态更重要,则要根据具体业务进行选择。

\par 这样Slave就会和Master同步了,主要关键点为:

\noindent\rule{\textwidth}{0.4pt}


\begin{verbatim}Slave\_IO\_Running:  Yes
Slave\_SQL\_Running: Yes
Seconds\_Behind\_Master是否为0,\#0表示状态已经同步
\end{verbatim}


\noindent\rule{\textwidth}{0.4pt}


\par 
\begin{figure}[htbp]
\centering
\includegraphics[width=0.8\textwidth]{Image00006.jpg}
\end{figure}

 提示:
 set global sql\_slave\_skip\_counter=n;\#n取值>0,表示忽略执行N个更新。

\par 解决方法2:根据可以忽略的错误号事先在配置文件中进行配置,跳过指定的不影响业务数据的错误,例如:

\noindent\rule{\textwidth}{0.4pt}


\begin{verbatim}[root@MySQL \textasciitilde{}]\# grep slave-skip /data/3306/my.cnf
slave-skip-errors = 1032,1062,1007
\end{verbatim}


\noindent\rule{\textwidth}{0.4pt}


\par 
\begin{figure}[htbp]
\centering
\includegraphics[width=0.8\textwidth]{Image00006.jpg}
\end{figure}

 提示:
 类似由于入库重复所导致的失败可以忽略,其他情况下是不是可以忽略则需要根据不同公司的具体业务来进行评估。

\par 其他可能引起复制故障的问题具体如下。

\par ·MySQL自身的原因及人为重复插入数据。

\par ·不同的数据库版本会引起不同步,低版本到高版本可以,但是高版本不能往低版本同步。

\par ·MySQL的运行错误或者程序BUG。

\par ·binlog记录模式,例如row level模式就比默认的语句模式要好。


% From text00224.html


\section{14.4 本章重点}


\par 1)MySQL主从复制的常见架构。

\par 2)MySQL主从复制的原理(面试时经常会问到)。

\par 3)MySQL主从复制的实践。

\par 4)生产场景下如何快速部署MySQL主从复制。

\par 5)MySQL主从复制生产故障解决思路。


% From text00225.html


\section{14.5 参考资料}


\par 1)MySQL官方手册5.6

\par 2)MySQL数据库企业级核心知识精品

\par 
\href{http://edu.51cto.com/course/course_id-4058.html}{http://edu.51cto.com/course/course\_id-4058.html}


\par 3)Heartbeat+DRBD+MySQL高可用架构方案与实施过程细节

\par 
\href{http://oldboy.blog.51cto.com/2561410/1240412}{http://oldboy.blog.51cto.com/2561410/1240412}


\par 4)MySQL数据库企业级应用实战

\par 
\href{http://edu.51cto.com/pack/view/id-214.html}{http://edu.51cto.com/pack/view/id-214.html}



% From text00226.html


\chapter{第15章 MySQL主从复制高级方案与应用实践}


\section{
15.1 MySQL集群企业级架构方案}
}


\par 在实际生产环境中,一主多从的数据库环境是最常采用的数据库架构,那么我们应该怎么分配不同的主从以向用户提供更高效的访问呢?请阅读下文。

\par 1.根据对数据库的访问请求实现读写分离

\par 在实际生产环境中,99\%的业务场景都是读多写少,而一主多从的环境恰恰是主不容易扩展,而从更容易增加多台的环境,因此,把更少请求的写访问业务调度到主库上,而把读请求较多的业务调度到多个从库上,就是一个很好的策略。此方案对应的方法和工具已经在第14章中介绍过,此处就不再累述了,后文对本方案的实践还会进行讲解。

\par 2.根据不同的业务拆分多个从库以提供访问

\par 从大的方向来分,用户的请求主要有两种,一个是外部用户的访问请求,一个是内部用户的访问请求。而大多数情况下,都是外部用户访问请求更重要,需要更好的用户体验效果,而对内部用户访问请求的体验要求相对外部来说就要低得多,甚至宕机一段时间也不会有太大的影响。

\par 因此,首先要对外部用户和内部用户读数据库的访问请求进行分离,拿一主五从的数据库环境来说,可以让前三个从库(利用读写分离软件或LVS等工具实现负载均衡)提供外部用户读请求访问,让第四个从库用于内部用户读访问(业务后台、数据分析、搜索业务、财务统计、定时任务、开发查询等),让第五个数据库用于数据库定时全备份以及增量备份(开启binlog功能)。整个逻辑图如图15-1所示。

\begin{figure}[htbp]
\centering
\includegraphics[width=0.8\textwidth]{Image00121.jpg}
\end{figure}



\par 图15-1 MySQL主从复制根据业务重要性拆分从库方案

\par 3.实现对主库的高可用

\par 一主多从的MySQL数据库环境,最脆弱的就是主库了,不但扩展困难,所承受的压力也很大,而且,还会面临宕机没有机器替补的问题。

\par 本节为大家介绍的是MySQL的高可用方案。MySQL的高可用方案包含很多种,具体如下。

\par 1)heartbeat+dbrd+mysql方案

\par 此方案是通过dbrd工具对主数据库服务器实现基于block的异机物理复制(类似于网络raid1),速度很快,但缺点是与备节点对应的复制的数据分区是不可见状态(即不能被访问,除非主节点宕机,备节点才可提供访问),浪费一台机器,早期受MySQL官方推荐,现如今在企业场景中使用的不是很多了,但仍然不失为一套成熟的MySQL高可用方案。

\par 2)mysql-MMM(Master-Master replication Manager)方案

\par 此方案是通过MySQL的replication实现的主主之间的数据同步,同时还可以实现slaves负载均衡等功能,但业界普遍认为MMM无法完全保持数据的一致性,因此对数据一致性要求较高的数据库场景很少采用,导致该方案也逐步没落了。

\par 3)mysql MHA(Master High Availability)+keepalived方案

\par MHA是业界公认的MySQL高可用的相对来说比较成熟的方案,此方案也是通过MySQL的replication实现的数据库服务器之间的数据同步,同时还可以实现从库负载均衡、主库宕机后自动选择最优的从库,将其切换为主库,并尽最大的努力对所有的库做数据补全操作一直到最新,并对其他从库和新主库实现复制,再加上keepalived是为了实现VIP漂移,因此该方案是业界采用较多的方案,也是本书后面将会精讲的方案。

\par 4)更多方案将在后续出版的书籍中再与大家探讨,例如,pxc、共享存储方案、分布式方案等。


% From text00227.html


\section{15.2 MySQL企业级备份策略方案}


\par 前文讲解过MySQL主从复制的企业应用场景之一,就是作为数据库的一个异地相当于实时性的备份,但是,这个实时性备份有一个无法解决的问题,那就是如果主库存在语句级误操作(例如“drop database oldboy;”),那么从库也会执行“drop database oldboy;”,这样MySQL主从库就都删除了该数据,因此,除了MySQL主从复制备份之外,我们还需要定时的数据库备份方案。

\par 1.利用MySQL主从复制的从库进行数据备份策略

\par 根据前面的章节可知,定时备份比较常用的策略是使用mysqldump(逻辑备份)以及Xtrabackup(物理备份)进行备份。

\par 虽然这二者都支持热备功能,但是由于在备份期间消耗了大量的磁盘IO以及内存、CPU等系统资源,导致为用户提供服务的能力大大减弱,哪怕是在访问低谷时进行的备份,还是会影响一部分用户访问,那么有没有更好的备份策略呢?这就是接下来我们要讨论的话题。

\par 高性能高并发业务场景下进行数据库备份时,可以选择MySQL主从复制中的一个不对外服务的从库作为定时数据库备份服务器,备份策略具体如下。

\par 例如15.1节的图15-1所示的是一主多从的数据库集群环境,可以选择在一台从库上进行备份(Slave5),而不参与提供外部用户访问,此时需要在该从库(Slave5)中开启binlog功能,其逻辑如图15-1所示。

\par 2.利用MySQL主从复制的从库进行数据备份实战的步骤

\par 1)选择一个不对外提供服务的从库,这样可以确保其与主库更新最接近,专门做数据备份用,从而告别在主库上进行备份但影响主库访问的方案。

\par 2)开启从库的binlog功能,目的是将来数据丢失时可使用binlog进行增量恢复,注意,此处最好开启从库的binlog功能,而不是在数据恢复时拿主库的binlog进行数据恢复。

\par 3)以下提供的是3种常见的备份方案。

\par ·当数据量低于30GB时,可采用mysqldump逻辑备份,优缺点前文已经讲解过了。

\par ·当数据量大于30GB时,可采用Xtrabackup物理热备工具,优缺点前文中也已经讲解过了。

\par ·采用从库备份也可以使用冷备的方案,即备份时可以选择停止从库的SQL线程,停止应用SQL语句到数据库,I/O线程将继续保持工作状态,执行命令为“stop slave sql\_thread;”,备份方式可以采取cp、tar(针对“/data/”目录)工具进行备份,最后把全备文件和binlog数据发送到备份服务器上留存即可。


% From text00228.html


\section{15.3 MySQL主从复制生产场景的常见延迟原因及防范方案}


\par 笔者总结了MySQL主从复制的生产场景,延迟的常见问题、原因及解决方案,具体如下。

\par 
问题一:一个主库的从库太多,易导致复制延迟。


\par 建议从库数量在3~5个为宜,要复制的从节点数量过多,会导致主库系统及网络带宽压力过大,从而最终导致从库复制延迟。

\par 
问题二:从库硬件比主库差,易导致复制延迟。


\par master和slave的硬件配置应该保持一致,主从库硬件差异包括磁盘IO、CPU、内存等各方面因素,在高并发大数据量写入的场景下易造成复制的延迟。

\par 
问题三:慢SQL语句过多,易导致复制延迟。


\par 假如有一条SQL语句,在主库上执行了20秒完毕,此时到从库上可能也得20秒左右才能查到数据,这样就延迟了20秒。

\par SQL语句的优化应作为常规工作不断地进行监控和优化,如果是单个SQL的写入时间过长,那么可以在修改后分多次写入。通过查看慢查询日志或“show full processlist”命令找出执行时间长的查询语句或者大的事务,建立索引或者分拆多条小的语句执行。

\par 
问题四:主从复制的设计问题,易导致复制延迟。


\par 例如,主从复制单线程中,如果主库写并发太大,来不及传送到从库则会导致延迟。

\par MySQL 5.6可以支持SQL多线程复制,但其也是针对不同的库级别的,大型门户网站则可能会开发多线程同步功能。

\par 
问题五:主从库之间的网络延迟,易导致复制延迟。


\par 主从库的网卡、网线、连接的交换机等网络设备都可能会成为复制的瓶颈,导致复制延迟,另外,跨公网主从复制很容易导致主从复制延迟。

\par 
问题六:主库读写压力太大,导致复制延迟。


\par 主库硬件要搞好一点,架构的前端要加buffer以及cache层,以减轻主库的压力。


% From text00229.html


\section{15.4 MySQL主从复制数据一致性企业级方案}


\par MySQL数据库主从复制延迟是高并发场景中时而会发生的问题,大家除了要注意上述15.3节中提到的6个问题之外,还有如下方法可以应对非预期的主从复制延迟问题。

\par 1.采用半同步复制方案

\par 主从不一致时因为主从复制默认是异步的复制方式,即主库完成后,binlog将数据传输到从库并写到中继日志中,最后在由从库解析执行到从库里。而半同步复制方案实际上就是实时复制方案,即让主库和从库同时写入完毕,再把请求结果返回给用户,这样就可以实现主从是实时一致性的,该方案的详细介绍和实践见本书后文。该方案的优点可以最大限度地保证主从一致,但是,增加了主库写入的时间,超过实时同步设置的超时时间都会转化成异步复制。

\par 2.当复制发生延迟时让程序改读主库

\par 1)当数据写入后,程序在从库读不到该数据时,自动判断并请求主库,这是多年前老男孩曾为公司提供的方案,最终被采纳了(思路很重要,结果导向,倒推方法,解决方法的缺陷和对应的困难),此方案有一个小问题那就是对于新数据没有任何问题,但是对于更新的数据,可能会在从库读到旧的数据,从而不会再读取主库,另外还会增加主库的压力。

\par 2)可以部署一个缓存服务器,当数据写入数据库的同时,将更新主库的数据的key、时间戳以及数据超时时间(例如1秒)写入到缓存服务器中,然后在进行读请求访问的时候,先查缓存服务器,如果有对应的key,则表示主库刚写完不到1秒,从库也有可能还没有,所以主动读主库,如果缓存里没有对应的key,则表示更新的主库对应key的数据已经写完超过1秒了,此时从库也应该有了,如果此时从库中还没有,则可以再请求写库。该方案的缺点是逻辑复杂,开发成本很高,并且还会增加主库的压力,最后不管key的超时时间设置的是大还是小,如果复制发生了故障,那么理论上还可能会导致访问到旧的数据。

\par 3)在产品程序逻辑上,应尽量延缓(例如1秒)用户访问刚更新过的数据,为数据库主从复制争取同步的时间,例如,自动返回设置3、2、1等,或者增加一个点击按钮,此方案是笔者曾经向公司推荐的方案,可能不是最佳的方案,这里将此方案提出来的目的是提醒读者,很多时候解决问题未必是技术层面的事情,也可以从产品层面解决问题,希望读者们能打开格局,不拘一格做运维。

\par 4)采用读写分离工具实现路由数据到主库,一般是大型企业自研,常规读写分离工具会将写请求自动路由到主库,将从请求路由到从库。此时,可以让工具路由在写请求的同时记录写(增加或更新)请求对应的key和更新的时间,若有相同的请求访问这个key对应的读数据,则判断一下这个写请求延续的时间是否超过了指定的时间(例如1-2秒,根据业务需要来指定),如果未超过,则让这个读请求读主库。

\par 以上方案的优点各有千秋,缺点也都很明显,最终的选择还是得依靠读者去折中平衡选择。

\par 当然,从哲学上来说,最核心的还是主从复制不发生故障或故障自动治愈(14.3节),不发生延迟(15.3节),才是解决数据一致性的最高境界。因此,读者需要在15.3节和14.3节的知识讲解上多下功夫,再加上本节的内容,则可无敌于天下了。


% From text00230.html


\section{15.5 MySQL多线程复制解决复制延迟实践}


\par 早在MySQL 5.6版本以前,就有一线门户公司开发过MySQL多线程复制功能,MySQL软件从5.6.3开始支持多线程复制功能,但这个多线程复制并不是完全的多线程复制,而是有条件的多线程复制,即只能针对不同的数据库多线程,而不能针对一个库里的多个表多线程,总的来说,多线程复制还是一个不错的功能。

\par 具体的配置如下。

\par 1)查看当前slave服务器的SQL线程状态,提示为“Waiting for master to send event
 ”,表示等待主库发送事件日志:

\noindent\rule{\textwidth}{0.4pt}


\begin{verbatim}mysql> show processlist;
+----+-------------+-----------+------+---------+-------+-------------------+
| Id | User        | Host      | db   | Command | Time  | State                            | Info             |
+----+-------------+-----------+------+---------+-------+-------------------+
| 21 | system user |           | NULL | Connect | 23240 | Waiting for master to send event | NULL             |
| 32 | root        | localhost | NULL | Query   |     0 | init                             | show processlist |
+----+-------------+-----------+------+---------+-------+-------------------+
2 rows in set (0.00 sec)
\end{verbatim}


\noindent\rule{\textwidth}{0.4pt}


\par 2)检查多线程的参数配置:

\noindent\rule{\textwidth}{0.4pt}


\begin{verbatim}mysql> show variables like '\%parallel\%';
+------------------------+-------+
| Variable\_name          | Value |
+------------------------+-------+
| slave\_parallel\_workers | 0     | \#<==默认为0,即单线程复制。
+------------------------+-------+
1 row in set (0.00 sec)
\end{verbatim}


\noindent\rule{\textwidth}{0.4pt}


\par 3)停止主从复制:

\noindent\rule{\textwidth}{0.4pt}


\begin{verbatim}mysql> stop slave;
Query OK, 0 rows affected (0.01 sec)
mysql> set global slave\_parallel\_workers = 4; \#<==在线修改为4个线程。
Query OK, 0 rows affected (0.00 sec)
mysql> show variables like '\%parallel\%';
+------------------------+-------+
| Variable\_name          | Value |
+------------------------+-------+
| slave\_parallel\_workers | 4     |            \#<==参数修改成功。
+------------------------+-------+
1 row in set (0.00 sec)
\end{verbatim}


\noindent\rule{\textwidth}{0.4pt}


\par 4)启动数据库查看修改效果:

\begin{figure}[htbp]
\centering
\includegraphics[width=0.8\textwidth]{Image00122.jpg}
\end{figure}



\par 从输出中可以看到,已经启动了4个SQL线程,如果希望永久生效,则可同时在my.cnf中配置该参数,无须重启:

\noindent\rule{\textwidth}{0.4pt}


\begin{verbatim}[mysqld]
slave\_parallel\_workers = 4
\end{verbatim}


\noindent\rule{\textwidth}{0.4pt}



% From text00231.html


\section{15.6 让MySQL主从复制的从库只读访问}


\par read-only参数选项可以让从数据库服务器只允许具备主从复制权限的线程或具有SUPER权限的数据库用户进行更新,从而确保从服务器不接受来自用户端的其他普通用户的更新。

\par 1.read-only参数允许数据库更新的条件

\par read-only参数允许数据库更新的条件主要包含如下两种情况。

\par ·具有SUPER权限的用户可以更新,不受read-only参数影响,例如,管理员root。

\par ·来自从服务器具备主从复制权限的线程可以更新,不受read-only参数的影响,例如,前文的rep用户。

\par 在生产环境中,可以在从库Slave中使用read-only参数,以确保从库数据不被非法更新。

\par 2.如何配置read-only参数

\par 方法一:启动数据库时直接带“--read-only”参数启动,代码如下。

\noindent\rule{\textwidth}{0.4pt}


\begin{verbatim}mysqld\_safe --read-only --user=mysql &
\end{verbatim}


\noindent\rule{\textwidth}{0.4pt}


\par 方法二:在my.cnf中[mysqld]模块下加read-only参数,然后重启数据库,配置如下。

\noindent\rule{\textwidth}{0.4pt}


\begin{verbatim}[mysqld]
read-only
\end{verbatim}


\noindent\rule{\textwidth}{0.4pt}



% From text00232.html


\section{15.7 MySQL主从复制读写分离Web用户生产设置方案}


\par 在配置好MySQL主从复制,并实现了读写分离以后,数据库授权程序访问的用户设置方法包含如下几种方案。

\par 1.主库和从库使用不同的用户,授权不同的权限。

\par 主库上对web\_w用户的授权如下:

\par 用户:web\_w密码:oldboy123端口:3306主库VIP:172.16.1.51

\noindent\rule{\textwidth}{0.4pt}


\begin{verbatim}权限:SELECT, INSERT, UPDATE, DELETE
命令:GRANT SELECT, INSERT, UPDATE, DELETE ON `web`.* TO 'web\_w'@'172.16.1.\%' identified by 'oldboy123';
\end{verbatim}


\noindent\rule{\textwidth}{0.4pt}


\par 从库上对web\_r用户单独授权如下:

\par 用户:web\_r密码:oldboy123端口:3306从库VIP:172.16.1.52

\noindent\rule{\textwidth}{0.4pt}


\begin{verbatim}权限:SELECT
命令:GRANT SELECT ON `web`.* TO 'web\_r'@'172.16.1.\%' identified by 'oldboy123';
\end{verbatim}


\noindent\rule{\textwidth}{0.4pt}


\par 
\begin{figure}[htbp]
\centering
\includegraphics[width=0.8\textwidth]{Image00006.jpg}
\end{figure}

 提示:
 此方法显得不够专业,而且从库上也会存在web\_w用户,一旦被开发等人利用,就可以对从库进行写库,这种授权也会对数据库带来潜在的风险。

\par 2.网站程序访问主库和从库时使用一套用户密码

\par 示例如下。

\par 主库:用户为web,密码为oldboy123,端口为3306,写IP为172.16.1.51。

\par 主库:用户为web,密码为oldboy123,端口为3306,读IP为172.16.1.52。

\par 从外表上看,只有读写分配的IP不同,其他都是相同的。运维人员应尽量为开发人员提供方便,如果是数据库前端有DAL层(实现了读写分离),则可以只为开发人员提供一套用户、密码、端口、VIP,这样就更专业了,剩下的都由运维人员未完成即可。

\par 下面是使用同一个Web连接用户授权访问的方案。

\par 方法1:主库和从库使用相同的用户,但授予不同的权限。

\par 主库上对web用户的授权如下:

\noindent\rule{\textwidth}{0.4pt}


\begin{verbatim}用户:web 密码:oldboy123  端口:3306  主库VIP:172.16.1.51
权限:SELECT, INSERT, UPDATE, DELETE
命令:GRANT SELECT, INSERT, UPDATE, DELETE ON `web`.* TO 'web'@'172.16.1.\%' identified by 'oldboy123';
\end{verbatim}


\noindent\rule{\textwidth}{0.4pt}


\par 从库上对web用户的授权如下:

\noindent\rule{\textwidth}{0.4pt}


\begin{verbatim}用户:web 密码:oldboy123  端口:3306  从库VIP:172.16.1.52
权限:SELECT
提示:由于主库和从库是同步复制的,所以从库上的web用户会自动与主库的一致,即无法实现只读select的授权。
\end{verbatim}


\noindent\rule{\textwidth}{0.4pt}


\par 要实现方法1中的授权方案,并且要求操作严谨,应采用如下两个方法。

\par 一是在主库上创建完web用户和权限之后,在从库上revoke收回对应的更新权限(insert、update、delete)。

\par 命令分别如下。

\noindent\rule{\textwidth}{0.4pt}


\begin{verbatim}主库执行:
GRANT SELECT, INSERT, UPDATE, DELETE ON `web`.* TO 'web'@'172.16.1.\%' identified by 'oldboy123';
从库执行:
REVOKE INSERT,UPDATE,DELETE ON web.* FROM 'web'@'172.16.1.\%';
\end{verbatim}


\noindent\rule{\textwidth}{0.4pt}


\par 二是忽略授权库mysql同步,主库的my.cnf配置参数如下:

\noindent\rule{\textwidth}{0.4pt}


\begin{verbatim}binlog-ignore-db = mysql    \#<==mysql库不记录Binlog日志。
replicate-ignore-db = mysql \#<==忽略复制MySQL库。
\end{verbatim}


\noindent\rule{\textwidth}{0.4pt}


\par 
\begin{figure}[htbp]
\centering
\includegraphics[width=0.8\textwidth]{Image00006.jpg}
\end{figure}

 提示:
 注意上面参数的等号两边必须要有空格。

\par 三是在从库上设置read-only参数,让从库只读。

\par 主库从库:主库和从库使用相同的用户,授予相同的权限(非ALL权限)。

\noindent\rule{\textwidth}{0.4pt}


\begin{verbatim}用户:web 密码:oldboy123  端口:3306  主库VIP:172.16.1.51,从库VIP:172.16.1.52
权限:SELECT, INSERT, UPDATE, DELETE
命令:GRANT SELECT, INSERT, UPDATE, DELETE ON web.* TO 'web\_w'@'172.16.1.\%' identified by 'oldboy123';
\end{verbatim}


\noindent\rule{\textwidth}{0.4pt}


\par 由于从库设置了read-only,非super权限是无法写入的,因此,通过read-only参数就可以很好地控制用户非法将数据写入从库。

\par 老男孩老师的生产工作场景的设置方案具体如下。

\par 1)忽略授权库mysql同步,主库的配置参数如下:

\noindent\rule{\textwidth}{0.4pt}


\begin{verbatim}binlog-ignore-db = mysql
replicate-ignore-db = mysql
\end{verbatim}


\noindent\rule{\textwidth}{0.4pt}


\par 
\begin{figure}[htbp]
\centering
\includegraphics[width=0.8\textwidth]{Image00006.jpg}
\end{figure}

 提示:
 注意上面参数等号两边必须要有空格。

\par 2)主库和从库使用相同的用户,但授予不同的权限。

\par 主库上对web用户的授权如下:

\noindent\rule{\textwidth}{0.4pt}


\begin{verbatim}用户:web 密码:oldboy123  端口:3306  写库VIP:172.16.1.51
权限:SELECT, INSERT, UPDATE, DELETE
命令:GRANT SELECT, INSERT, UPDATE, DELETE ON web.* TO 'web'@'172.16.1.\%' identified by 'oldboy123';
\end{verbatim}


\noindent\rule{\textwidth}{0.4pt}


\par 从库上对web用户的授权如下:

\noindent\rule{\textwidth}{0.4pt}


\begin{verbatim}用户:web 密码:oldboy123  端口:3306  读库VIP:172.16.1.52
权限:SELECT
\end{verbatim}


\noindent\rule{\textwidth}{0.4pt}


\par 3)在从库上设置read-only,增加第三个保险。

\par 以上三点的共同作用即可达到专业规范的授权Web用户的目的。


% From text00233.html


\section{15.8 MySQL主从延迟复制方案及恢复实践}


\par 1.MySQL主从延迟复制方案介绍

\par MySQL从5.6版本开始就支持主从延迟复制,这个功能主要解决的问题是,当主库有逻辑的数据删除或者错误更新时,所有的从库都会进行错误的更新,从而导致数据库的所有数据都异常,即使有定时的备份数据可以用于数据恢复,特别是数据库的数据量很大时,恢复时间也会很长,在恢复期间,数据库数据被删除或者出现错误数据都会影响正常的访问体验。

\par 而延迟复制就可以很好地解决这个问题。例如,可以设定某一个从库和主库的更新延迟1个小时,这样当主库数据出现问题以后,1个小时以内即可发现,可以对这个从库进行无害恢复处理,使之依然是正确的完整的数据,这样就省去了数据恢复占用的时间,用户体验也会有所提高。

\par 2.MySQL主从延迟复制配置实践

\par MySQL 5.6版本的延迟复制配置,是通过在Slave上执行以下命令实现的:

\noindent\rule{\textwidth}{0.4pt}


\begin{verbatim}CHANGE MASTER TO MASTER\_DELAY = N;   \#<==读者可在配置延迟从库Change Master时
                                      直接加上MASTER\_DELAY选项。
\end{verbatim}


\noindent\rule{\textwidth}{0.4pt}


\par 该语句设置Slave数据库延时N秒后,再与主数据库进行数据复制,具体操作为登录到Slave数据库服务器(本文是52),然后执行如下命令:

\noindent\rule{\textwidth}{0.4pt}


\begin{verbatim}mysql> stop slave;
Query OK, 0 rows affected (0.45 sec)
mysql> CHANGE MASTER TO MASTER\_DELAY = 20;   \#<==这是延迟的核心命令。
Query OK, 0 rows affected (0.22 sec)
mysql> start slave;
Query OK, 0 rows affected (0.15 sec)
mysql> show slave status\textbackslash\{\}G
*************************** 1. row ***************************
            ...省略若干...
            Slave\_IO\_Running: Yes
            Slave\_SQL\_Running: Yes
            ...省略若干...
            SQL\_Delay: 20             \#<==这里设置的数字就是延迟20秒后进行复制。
            SQL\_Remaining\_Delay: NULL \#<==还剩多少秒执行复制。
Slave\_SQL\_Running\_State: Slave has read all relay log; waiting for the slave I/O thread to update it                   \#<==SQL线程的状态。
...省略若干...
1 row in set (0.09 sec)
\#<==提示:复制状态里最常用的三个状态参数分别为SQL\_Delay、SQL\_Remaining\_Delay、Slave\_
       SQL\_Running\_State,说明上面已经分别进行注释了。
\end{verbatim}


\noindent\rule{\textwidth}{0.4pt}


\par 主库插入数据:

\noindent\rule{\textwidth}{0.4pt}


\begin{verbatim}mysql> create database lanlan;
Query OK, 1 row affected (0.00 sec)
\end{verbatim}


\noindent\rule{\textwidth}{0.4pt}


\par 主库插入完数据1秒以后,从库执行“show databases;”查看数据是否及时同步了,结果如下:

\noindent\rule{\textwidth}{0.4pt}


\begin{verbatim}mysql> show databases;
+--------------------+
| Database           |
+--------------------+
| information\_schema |
| alex\_python        |
| mysql              |
| performance\_schema |
+--------------------+
4 rows in set (0.36 sec)
\end{verbatim}


\noindent\rule{\textwidth}{0.4pt}


\par 在从库上并没有看到在主库上创建的数据库lanlan,此时可间歇性地执行“show slave status\textbackslash\{\}G”查看延迟的参数状态,输出如下:

\noindent\rule{\textwidth}{0.4pt}


\begin{verbatim}mysql> show slave status\textbackslash\{\}G
            ...省略若干...
                    SQL\_Delay: 20
          SQL\_Remaining\_Delay: 13 \#<==剩余13秒执行复制。
      Slave\_SQL\_Running\_State: Waiting until MASTER\_DELAY seconds after master executed event
            ...省略若干...
1 row in set (0.00 sec)
mysql> show slave status\textbackslash\{\}G
            ...省略若干...
                    SQL\_Delay: 20
          SQL\_Remaining\_Delay: 9 \#<==剩余9秒执行复制。
      Slave\_SQL\_Running\_State: Waiting until MASTER\_DELAY seconds after master executed event
            ...省略若干...
1 row in set (0.00 sec)
mysql> show slave status\textbackslash\{\}G
            ...省略若干...
                    SQL\_Delay: 20
          SQL\_Remaining\_Delay: NULL \#<==复制完成后,没有新数据更新的状态。
      Slave\_SQL\_Running\_State: Slave has read all relay log; waiting for the slave I/O thread to update it
            ...省略若干...
1 row in set (0.00 sec)
\end{verbatim}


\noindent\rule{\textwidth}{0.4pt}


\par 在从库没有更新数据,并且处于延迟复制还没到时间的期间,查看从库的中继日志:

\noindent\rule{\textwidth}{0.4pt}


\begin{verbatim} [root@db02 data]\# pwd
/application/mysql/data
[root@db02 data]\# mysqlbinlog db02-relay-bin.000002
SET @@session.lc\_time\_names=0/*!*/;
SET @@session.collation\_database=DEFAULT/*!*/;
create database lanlan \#<==中继日志中的确已经有了创建的语句,说明IO线程还是实时在工作的。
\end{verbatim}


\noindent\rule{\textwidth}{0.4pt}


\par 3.MySQL主从延迟复制原理解析及结论

\par MySQL的延迟复制实际上影响的只是SQL线程将数据应用到从数据库,而I/O线程早已把主库更新的数据写到了从库的中继日志中,因此,在延迟复制期间,即使主库宕机了,从库到了延迟复制的时间,也依然会把数据更新到与主库宕机时一致。

\par 
\begin{figure}[htbp]
\centering
\includegraphics[width=0.8\textwidth]{Image00006.jpg}
\end{figure}

 特别提示:
 其实MySQL的延迟复制功能早在几年前,老男孩老师就已经用思想实现了这个功能,并应用于企业生产备份和恢复中了,方法如下。

\par 1)正如15.2节已经介绍过的,执行“mysql>stop slave sql\_thread;”停掉SQL线程,然后进行备份,备份期间主库可能会宕机,但是主库的binlog依然会及时发到从库,最终从库依然可以恢复到主库宕机之前的状态。

\par 2)编写一个脚本,利用定时任务控制sql\_thread的停止和运行,进而库就可以控制实现简单的从库延迟复制功能了,这就是思想的重要性。当然了,对于5.6版本还是就用软件本身提供的功能吧,对于5.6以前的数据库若要实现延迟复制,则可以考虑下老男孩曾经使用过的延迟备份以及延迟复制的思路。

\par 4.使用MySQL主从延迟复制进行数据恢复实践

\par 在企业中,我们应根据业务需求为延迟复制指定一个时间段,例如1个小时后进行该从库复制,那么在这一个小时之内,如果主库误更新了数据,其他的从库也都会依样误更新数据,那么如何将这个延迟的从库恢复到正常没有误更新数据前的完整状态呢?且看下文的实践。

\par 1)模拟环境,将从库延迟调整为3600秒:

\noindent\rule{\textwidth}{0.4pt}


\begin{verbatim}mysql> stop slave;
Query OK, 0 rows affected (0.00 sec)
mysql> CHANGE MASTER TO MASTER\_DELAY = 3600;
Query OK, 0 rows affected (0.03 sec)
mysql> start slave;
Query OK, 0 rows affected (0.08 sec)
mysql> show slave status\textbackslash\{\}G
            ...省略若干...
                    SQL\_Delay: 3600
          SQL\_Remaining\_Delay: 2414
      Slave\_SQL\_Running\_State: Waiting until MASTER\_DELAY seconds after master executed event
            ...省略若干...
1 row in set (0.00 sec)
\end{verbatim}


\noindent\rule{\textwidth}{0.4pt}


\par 2)模拟在主库写入数据,每隔5秒写入1个库,就当是模拟用户写入数据了:

\noindent\rule{\textwidth}{0.4pt}


\begin{verbatim}[root@db01 \textasciitilde{}]\# for n in \{1..5\}
> do
>     mysql -e "create database oldboy\$n"
>     sleep 5
> done
提示:关于Shell脚本知识可以参考《跟老男孩学Linux运维:Shell编程实战》一书。
\end{verbatim}


\noindent\rule{\textwidth}{0.4pt}


\par 3)模拟人为破坏数据,也可以是不带where的update语句:

\noindent\rule{\textwidth}{0.4pt}


\begin{verbatim}mysql> drop database oldboy5;   \#<==删除oldboy5数据库,后面要做的就是把这个数据
                                  库恢复回来,别的数据还得保留。
Query OK, 0 rows affected (0.00 sec)
mysql> show databases like 'oldboy\%';
+--------------------+
| Database (oldboy\%) |
+--------------------+
| oldboy1            |
| oldboy2            |
| oldboy3            |
| oldboy4            |
+--------------------+
4 rows in set (0.00 sec)
\#此时,所有的从库都已经是坏数据了,只有延迟从库是好的,但是是一个小时之前的数据。
\end{verbatim}


\noindent\rule{\textwidth}{0.4pt}


\par 4)当数据库出现误删数据的情况时(特别是update不加条件破坏数据),要想完整恢复数据,最好选择对外停止访问措施,此时需要牺牲用户体验,除非业务可以忍受数据不一致,并且不会被二次破坏。从库可以适当地继续开放给用户读访问,但是也可能会导致用户读到的数据是坏的数据,这里就需要读者去衡量数据一致性和用户体验的问题。本例是使用iptables封堵用户对主库的所有访问:

\noindent\rule{\textwidth}{0.4pt}


\begin{verbatim}[root@db01 \textasciitilde{}]\# iptables -I INPUT -p tcp --dport 3306 ! -s 172.16.1.53 -j DROP
\#<==非172.16.1.53禁止访问数据库3306端口,51是主库IP,172.16.1.53为远程连接SSH
       客户端的IP。
\end{verbatim}


\noindent\rule{\textwidth}{0.4pt}


\par 5)登录主库执行“show processlist;”对binlog是否全部发送到该延迟从库进行确认,当然了,也可以登录延迟从库执行“show processlist;”对从库IO线程是否完全接收binlog进行状态查询确认:

\noindent\rule{\textwidth}{0.4pt}


\begin{verbatim}mysql> show processlist;
+----+----------------------------------------------------------------------+------------------+
| 12 | rep  | 172.16.1.52:39043 | NULL | Binlog Dump |  709 | Master has sent all binlog to slave; waiting for binlog to be updated | NULL             |
+----+----------------------------------------------------------------------+------------------+
2 rows in set (0.00 sec)
\#上述提示表示主库已经发送完所有的Binlog日志到从库了。
\end{verbatim}


\noindent\rule{\textwidth}{0.4pt}


\par 6)在从库上执行“stop slave;”暂停主从复制,并查看数据库是否已同步过来:

\noindent\rule{\textwidth}{0.4pt}


\begin{verbatim}mysql> stop slave;
mysql> show databases;
+--------------------+
| Database           |
+--------------------+
| information\_schema |
| alex\_python        |
| mysql              |
| performance\_schema |
+--------------------+
4 rows in set (0.00 sec)
\#提示:因为还未到延迟时间,因此数据不会同步到该延迟从库。
\end{verbatim}


\noindent\rule{\textwidth}{0.4pt}


\par 7)根据relay-log.info记录的SQL线程读取relay-log的位置,解析未应用到从库的relay-bin日志:

\noindent\rule{\textwidth}{0.4pt}


\begin{verbatim}[root@db02 data]\# pwd           \#<==进入到中继日志所在的目录。
/application/mysql/data
[root@db02 data]\# ls -l *relay* \#<==查看中继日志的相关信息。
-rw-rw----. 1 mysql mysql 172 9月  13 17:32 db02-relay-bin.000001 \#<==中继日志。
-rw-rw----. 1 mysql mysql 993 9月  13 17:37 db02-relay-bin.000002 \#<==中继日志。
-rw-rw----. 1 mysql mysql  48 9月  13 17:32 db02-relay-bin.index  \#<==中继日志索引。
-rw-rw----. 1 mysql mysql  61 9月  13 17:32 relay-log.info \#<==SQL线程读取中继日志
                                                    的位置信息。
[root@db02 data]\# cat relay-log.info  \#<==查看中继日志应用的位置信息。
7
./db02-relay-bin.000002  \#<==SQL线程读取中继日志的文件名信息。
284                      \#<==SQL线程读取中继日志的位置点信息。
oldboy-bin.000024
309
3600
0
1
\end{verbatim}


\noindent\rule{\textwidth}{0.4pt}


\par 8)解析SQL线程未解析的全部剩余relay-bin中继日志数据,由于本例模拟数据量不够大,因此本例中只有db02-relay-bin.000002一个中继日志,实际工作中可能会有多个,将它们一并解析到一个指定文件或者分成不同的文件存放也可以。

\noindent\rule{\textwidth}{0.4pt}


\begin{verbatim}[root@db02 data]\# mysqlbinlog --start-position=284 db02-relay-bin.000002 > relay.sql \#<==根据上述的relay-log.info的中级日志文件和位置信息进行中继日志解析,关于此命
            令的用法前面章节已经讲解过了,此处不再累述。
\end{verbatim}


\noindent\rule{\textwidth}{0.4pt}


\par 9)找到破坏数据库数据的SQL语句,并从已解析的SQL语句中将其删除,这里使用的是“drop database oldboy5”:

\noindent\rule{\textwidth}{0.4pt}


\begin{verbatim}[root@db02 data]\# egrep "drop database oldboy5" relay.sql \#<==检查是否存在误删
                                                      的SQL语句。
drop database oldboy5
[root@db02 data]\# sed -i '/drop database oldboy5/d' relay.sql \#<==删除,注意
                                                          别删多了。
[root@db02 data]\# egrep "\textasciicircum{}drop database oldboy5" relay.sql  \#<==检查删除结果。
\end{verbatim}


\noindent\rule{\textwidth}{0.4pt}


\par 10)将解析后并处理好的relay.sql数据文件恢复到延迟从库:

\noindent\rule{\textwidth}{0.4pt}


\begin{verbatim}[root@db02 data]\# mysql<relay.sql  \#<==这一步就是从停止slave复制开始,根据relay- log.info位置将剩下的所有日志数据手动恢复到数据库中,需要注意的是应提前清理破坏数据库的语句,然后再恢复。
[root@db02 data]\# mysql -e "show databases like 'oldboy\%';"
+--------------------+
| Database (oldboy\%) |
+--------------------+
| oldboy1            |
| oldboy2            |
| oldboy3            |
| oldboy4            |
| oldboy5            |   \#<==之前删除的oldboy5数据库已经找回。
+--------------------+
\end{verbatim}


\noindent\rule{\textwidth}{0.4pt}


\par 行文至此,利用延迟数据库恢复数据完毕,此时还需要要将此从库提升为主库(见手工实现主从角色切换章节的内容),将VIP指向该“延迟从库”,即作为新主库提供用户访问,然后,再对其他遭到破坏的主从数据库进行修复。


% From text00234.html


\section{15.9 本章重点}


\par 1)MySQL集群场景企业级架构方案。

\par 2)MySQL集群场景企业级备份策略方案。

\par 3)MySQL主从复制生产常见延迟原因及防范方案。

\par 4)MySQL主从复制数据一致性企业级方案。

\par 5)MySQL多线程复制解决复制延迟实践。

\par 6)让MySQL主从复制的从库只读访问。

\par 7)MySQL主从复制读写分离Web用户生产设置方案。

\par 8)MySQL主从延迟复制方案及恢复实践。


% From text00235.html


\section{15.10 参考资料}


\par 1)MySQL官方手册5.1、5.5及5.6。

\par 2)MySQL数据库企业级核心知识精品:

\par 
\href{http://edu.51cto.com/course/course_id-4058.html}{http://edu.51cto.com/course/course\_id-4058.html}


\par 3)Heartbeat+DRBD+MySQL高可用架构方案与实施过程细节:

\par 
\href{http://oldboy.blog.51cto.com/2561410/1240412}{http://oldboy.blog.51cto.com/2561410/1240412}


\par 4)MySQL数据库企业级应用实战:

\par 
\href{http://edu.51cto.com/pack/view/id-214.html}{http://edu.51cto.com/pack/view/id-214.html}



% From text00236.html


\chapter{第16章 MySQL复制高级方案应用实践}


\section{
16.1 MySQL级联复制}
}


\par 前文曾经介绍过,对于MySQL数据库的复制技术除了一主多从的常用复制架构之外,还可以是级联复制、双主(双向主从)复制以及多主(环状)复制等方案,本节就为大家介绍MySQL级联复制的方案。


% From text00237.html


\subsection{16.1.1 MySQL级联复制介绍}


\par MySQL级联复制的特点是从(slave)服务器本身除了作为从服务器之外,同时也会作为其下面从服务器的主数据库服务器。级联复制就是型如A==>B==>C的复制形式。

\par 如图16-1所示的就是级联单向复制逻辑架构图,本套架构一般只能在Master服务器A端进行数据写入,工作场景A作为主库,B作为A的从库,同时B又作为C的主库,C作为B的从库,此时中间的B需要进行特殊的设置(开启binlog功能)。

\begin{figure}[htbp]
\centering
\includegraphics[width=0.8\textwidth]{Image00123.jpg}
\end{figure}



\par 图16-1 级联复制逻辑图


% From text00238.html


\subsection{16.1.2 MySQL级联复制实现要点}


\par 我们已经知道了,MySQL主从复制技术的核心特点是通过传输binlog日志实现复制的功能。级联复制是主从复制的扩展变种,因此,级联复制架构中作为中间复制的从库B角色,如果同时还想做主库的话,就需要在B从库上开启binlog日志功能。

\par 1.从库开启binlog日志功能常见应用场景说明

\par 从库开启binlog日志功能的常见应用场景具体如下。

\par 1)当前的从库还要同时作为其他从库的主库,例如,级联复制或双主互为主从场景的情况下,就必须在从库上开启binlog日志功能。

\par 2)前文提到的将从库作为数据库全备的服务器,此时也要开启binlog日志用于全备恢复之后的增量数据恢复。

\par 2.从库记录binlog日志方法

\par 下面介绍一下从库记录binlog日志的方法。

\par 在从库的my.cnf中加入如下参数,然后重启服务即可生效:

\noindent\rule{\textwidth}{0.4pt}


\begin{verbatim}log\_bin = /application/mysql/logs/oldboy-bin
log\_slave\_updates    \#<==必须要有这个参数,否则不会记录Binlog日志。
expire\_logs\_days = 7 \#<==相当于find /path -type f -name " oldboy-bin.000*"
                         -mtime +7 |xargs rm -f
\end{verbatim}


\noindent\rule{\textwidth}{0.4pt}


\par 3.MySQL级联复制的实现说明

\par 除了作为新主库的从库需要开启binlog日志功能之外,级联复制的其他实现步骤与普通的主从复制是相同的,没有任何差别,因此这里就不再赘述了,请读者参考前文尝试下级联复制的实现。


% From text00239.html


\subsection{16.1.3 MySQL级联复制的应用场景}


\par 级联复制作为MySQL主从复制的一个扩展,其常用的应用场景具体如下。

\par 1)作为主库的级联从库,在大并发场景下,可以减轻主库下面的直接从库过多带来的数据复制压力,同时把级联从库作为一级从库的一个物理备份(从库宕机热备可快速补充从库节点)。

\par 2)级联从库用于数据备份、数据分析、企业内部等对数据实时性要求不是很高的业务应用。


% From text00240.html


\section{16.2 MySQL主主复制}


\par MySQL主主复制是级联复制的特殊形式,前文的级联复制是A==>B==>C的单向复制形式,而主主复制实际上是将A和C的角色合并为一个A,即A和B是对等的双向复制结构。

\par 图16-2为双向主主复制逻辑架构图,此架构可以在Master1端或Master2端进行数据写入,或者在两端同时写入数据(需要进行特殊设置)。

\begin{figure}[htbp]
\centering
\includegraphics[width=0.8\textwidth]{Image00124.jpg}
\end{figure}



\par 图16-2 双向主主复制逻辑图


% From text00241.html


\subsection{16.2.1 MySQL主主复制介绍}


\par MySQL主主复制也是使用mysql replication的复制方式,只不过复制方向是双向的,因此,复制过程与主从复制大同小异,只是在配置参数上多了几个参数。


% From text00242.html


\subsection{16.2.2 MySQL主主复制能够解决的企业问题}


\par 很多企业设计MySQL主主复制架构的可能原因是希望增加写并发的能力,但其实双主是否真的能够增加写并发,还有待实践检验,因为一个库写入时,另一个库也会同时写入(由复制进程写入),写入量较大时,从库的写压力也是不小的。

\par 如果要切实增加大量用户的写并发请求,那么MySQL主主复制架构可能并不是最好的选择,这里有如下的更优选择可供读者参考。

\par 1)分库:例如原本一个数据库服务器里有www/bbs/blog等多个库,那么可以将每个产品库都做成一套集群,这样就可以提高写的压力了,这样的拆分简单有效,DBA或运维人员自己就可以实现,几乎不需要业务程序做任何改动。

\par 2)分表:a)横拆——将最常使用的小字段放在一个表里,将不常使用的大字段放在单独的表里,等需要查询大字段表时,再通过连表进行查询;b)纵拆——根据表的记录进行拆分,例如,1000万条记录的一张表,可以根据key或uid将1~1000000条放在第一张表,1000001~2000000条放在第二张表,依此类推拆成10张表,然后前端根据key或uid进行调度决定应去访问哪张表对应的数据。这里的横拆和纵拆需要开发人员的或者数据库研发人员的配合才能完成。

\par 3)百度千台数据库集群拆分案例:当单表达到指定的记录数时,自动化扩容拆表,自动化收缩合表,具体详情见后文。

\par 老男孩个人建议参考:即使配置了MySQL双主复制,最好也只单写,配置双主的目的并不是为了增加写并发,而是为了实现在主库宕机后角色能够快速切换提供服务。

\par 正常主库宕机切换到从库,需要开启binlog功能,取消read-only参数,同时还需要更改Web用户授权等操作,而配置双主就不需要这些操作了,只要进行VIP的切换即可,当然了,后文讲解的MHA访问可以帮我们自动实现这个角色的切换,以及修改参数的过程。


% From text00243.html


\subsection{16.2.3 MySQL主主复制的企业级实现方案}


\par MySQL主主复制的常见实现方案具体如下。

\par 1.通过MySQL的参数配置使表主键自增的方案

\par 即通过为MySQL的表主键ID字段增加auto\_increment功能,实现主键的自动增长(默认步长为1),ID的增长具体如下。

\par ·M1库:每张表的主键都为奇数,例如:1、3、5....。

\par ·M2库:每张表的主键都为偶数,例如:2、4、6....。

\par 该实现的具体参数配置。

\par 解决主键自增长ID冲突,M1库与/M2库的参数配置如下。

\par Master1的参数配置如下。

\par ·auto\_increment\_increment=2\#<==自增ID的间隔,如135,间隔为2。

\par ·auto\_increment\_offset=1\#<==ID的初始位置。

\par (所有库里表的ID都将形成1、3、5、7...序列)

\par Master2的参数配置如下。

\par ·auto\_increment\_increment=2\#<==自增ID的间隔,如246,间隔为2。

\par ·auto\_increment\_offset=2\#<==ID的初始位置

\par (所有库里表的ID都将形成2、4、6、8...序列)

\par 
该种方法的优缺点


\par 缺点:该方法会导致表的ID号不连续。例如,写数据时ID的特征为:当M1中的数据写入ID号为1、3、5,此时如果数据写入M2,则数据写入ID就会从6开始,而不是从2开始。假设M2写入的连续ID为6、8、10,此时M1开始写,则会从11开始写。结论:总是以当前表里最大的ID值为基础递增开始写ID。

\par 优点:前端网站程序不需要做任何修改,就可以实现双主的架构。

\par 应用:尽量不要双写,而是单写,其作用可以作为主库宕机的备用切换选择(不用更改配置即可进行角色切换),当然要双写也是没有问题的,不会导致数据冲突,但会产生表里数据的ID值不连续的问题。

\par 2.使用序列服务实现MySQL双主方案

\par 使用序列(sequence)服务,由序列服务提供ID(就像银行的取号机器发号一样),当用户通过程序写数据库时,由序列服务分配ID,然后按照分发的ID号顺序调度到不同的数据库写入,以确保写入不同双主数据库的ID是不同的,并确保ID是值连续的。

\par 优点:表记录的ID号是连续的。

\par 缺点:增加了序列服务或服务器,引入了单点,同时,程序若要改动,则实现会比较复杂。

\par 总的建议,尽可能不要用双写模式。


% From text00244.html


\subsection{16.2.4 主主复制实践(自增ID)准备}


\par 1.主主复制数据库实战环境准备

\par MySQL主主复制实践对环境的要求比较简单,与主从复制环境一致即可,环境可以参考前文讲解,此处不现赘述。

\par 2.定义主主复制需要的服务器角色

\par 主库及从库IP、端口信息如表16-1所示。

\par 表16-1 主库及从库IP、端口信息

\begin{figure}[htbp]
\centering
\includegraphics[width=0.8\textwidth]{Image00125.jpg}
\end{figure}



\par 3.检查数据库的当前状态

\par 分别执行命令检查主主数据库的启动状态:

\noindent\rule{\textwidth}{0.4pt}


\begin{verbatim}[root@db01 \textasciitilde{}]\# lsof -i :3306
COMMAND  PID  USER   FD   TYPE DEVICE SIZE/OFF NODE NAME
mysqld  3044 mysql   11u  IPv6  15565      0t0  TCP *:mysql (LISTEN)
[root@db02 \textasciitilde{}]\# lsof -i :3306
COMMAND  PID  USER   FD   TYPE DEVICE SIZE/OFF NODE NAME
mysqld  3028 mysql   11u  IPv6  15422      0t0  TCP *:mysql (LISTEN)
\end{verbatim}


\noindent\rule{\textwidth}{0.4pt}


\par 根据结果可以确定数据库处于正常启动状态。


% From text00245.html


\subsection{16.2.5 在主库Master(51)上执行操作配置}


\par 1.设置主主复制的相关参数

\par 1)修改主库1的配置文件。执行vi/etc/my.cnf,编辑MySQL的配置文件,两个参数按如下内容进行修改:

\noindent\rule{\textwidth}{0.4pt}


\begin{verbatim}[mysqld]
server\_id = 1                \#<==用于同步的每台机器或实例的server\_id都不能相同。
log\_bin = /application/mysql/logs/oldboy-bin  \#<==可省略红色加粗部分,本文是与
                                              前面的章节保持一致。
log\_slave\_updates                       \#<==从库记录binlog必备参数。
expire\_logs\_days = 7                    \#<==自动清除7天前的binlog日志文件。
auto\_increment\_increment          = 2   \#<==自增ID的间隔,如 1 3 5,间隔为2。
auto\_increment\_offset             = 1   \#<==ID的初始位置。
slave-skip-errors = 1032,1062,1007,1008,1146,1049  \#<==忽略一些不影响业务的
                                                 复制错误提示。
\end{verbatim}


\noindent\rule{\textwidth}{0.4pt}


\par 2)检查参数配置之后的结果:

\noindent\rule{\textwidth}{0.4pt}


\begin{verbatim}[root@db01 \textasciitilde{}]\# egrep "server\_id|log|auto" /etc/my.cnf
log\_bin = /application/mysql/logs/oldboy-bin
expire\_logs\_days = 7
server\_id = 1
log\_slave\_updates
auto\_increment\_increment    = 2
auto\_increment\_offset   = 1
log-error = /application/mysql/logs/oldboy.err
\#特别提示:部分参数在前面章节中已经配置过,请注意检查时不要重复。
\end{verbatim}


\noindent\rule{\textwidth}{0.4pt}


\par 3)重启主库1的MySQL服务,并登录数据库检查参数的生效情况(这里略):

\noindent\rule{\textwidth}{0.4pt}


\begin{verbatim}[root@db01 \textasciitilde{}]\# /etc/init.d/mysqld restart
Shutting down MySQL.. SUCCESS!
Starting MySQL. SUCCESS!
\end{verbatim}


\noindent\rule{\textwidth}{0.4pt}


\par 到此,主库1的binlog功能和ID的起始及自增配置就配置好了。

\par 2.在主库1上建立用于主主复制的账号

\par 该步骤与主从复制的配置一样,如果前文已经配置好了账号,那么本节内容也可以忽略。

\par 1)登录主数据库1(172.16.1.51),命令如下:

\noindent\rule{\textwidth}{0.4pt}


\begin{verbatim}[root@db01 \textasciitilde{}]\# mysql \#<==为了操作安全方便,密码已经写入my.cnf配置文件。
\end{verbatim}


\noindent\rule{\textwidth}{0.4pt}


\par 2)建立用于主库2复制的账号及对应的权限:

\noindent\rule{\textwidth}{0.4pt}


\begin{verbatim}mysql> grant replication slave on *.* to 'rep'@'172.16.1.\%' identified by 'oldboy123';
Query OK, 0 rows affected (0.01 sec)
mysql> flush privileges; \#<==刷新权限使得授权的权限生效,一般对账号的更改需要应用此
                           命令,此处是为养成习惯而执行,而不是必须执行的。
Query OK, 0 rows affected (0.02 sec)
\end{verbatim}


\noindent\rule{\textwidth}{0.4pt}


\par 3.备份导出主库1的数据

\par 1)备份导出主库1的数据,也可以半夜在主库1上通过定时任务执行如下面命令:

\noindent\rule{\textwidth}{0.4pt}


\begin{verbatim}mysqldump -A -B -x --master-data=1|gzip >/opt/bak1\_\$(date +\%F).sql.gz
\#<==备份。
scp -rp /opt/bak1\_\$(date +\%F).sql.gz root@172.16.1.52:/opt/ \#<==复制到从库。
\end{verbatim}


\noindent\rule{\textwidth}{0.4pt}


\par 参数“--master-data=1”会在备份数据库里增加如下语句:

\noindent\rule{\textwidth}{0.4pt}


\begin{verbatim}-- Position to start replication or point-in-time recovery from
CHANGE MASTER TO MASTER\_LOG\_FILE='oldboy-bin.000024', MASTER\_LOG\_POS=107;
\end{verbatim}


\noindent\rule{\textwidth}{0.4pt}



% From text00246.html


\subsection{16.2.6 在主库2Master(52)上执行操作配置}


\par 1.设置主库2上复制的相关参数

\par 1)修改主库2的配置文件。执行vi/etc/my.cnf,编辑MySQL的配置文件,两个参数按如下内容进行修改:

\noindent\rule{\textwidth}{0.4pt}


\begin{verbatim}[mysqld]
server\_id = 2                 \#<==用于同步的每台机器或实例的server\_id都不能相同。
log\_bin = /application/mysql/logs/oldboy-bin   \#<==可省略红色加粗部分,本文是与
                                               前面的章节保持一致。
log\_slave\_updates                       \#<==从库记录Binlog必备参数。
expire\_logs\_days = 7                    \#<==自动清除7天前的binlog日志文件。
auto\_increment\_increment          = 2   \#<==自增ID的间隔,如 1 3 5,间隔为2。
auto\_increment\_offset             = 2   \#<==ID的初始位置。
slave-skip-errors = 1032,1062,1007,1008,1146,1049    \#<==忽略一些不影响业务的
                                                   复制错误提示。
\end{verbatim}


\noindent\rule{\textwidth}{0.4pt}


\par 2)检查参数配置之后的结果:

\noindent\rule{\textwidth}{0.4pt}


\begin{verbatim}[root@db02 \textasciitilde{}]\# egrep "server\_id|log|auto" /etc/my.cnf
log\_bin = /application/mysql/logs/oldboy-bin
expire\_logs\_days = 7
server\_id = 2 \#<==这个ID与主库1的ID是不同的。
log\_slave\_updates
auto\_increment\_increment    = 2
auto\_increment\_offset   = 2   \#<==该起始数字与主库1也是不同的。
log-error = /application/mysql/logs/oldboy.err
\#特别提示:部分参数在前面章节中已经配置过,请注意检查不要重复。
\end{verbatim}


\noindent\rule{\textwidth}{0.4pt}


\par 3)重启主库2的MySQL服务,并登录数据库检查参数的生效情况(这里略)。

\noindent\rule{\textwidth}{0.4pt}


\begin{verbatim}[root@db02 \textasciitilde{}]\# /etc/init.d/mysqld restart
Shutting down MySQL.. SUCCESS!
Starting MySQL. SUCCESS!
\end{verbatim}


\noindent\rule{\textwidth}{0.4pt}


\par 到此,主库2的binlog功能和ID起始,以及自增配置就配置好了。

\par 2.在需要做复制的主库2上导入全备做复制

\par 特别提示:主库2上不需要再设置主库1用于复制的账号了,因为备份恢复到数据库,主库2就有了账号了,导入的命令如下:

\noindent\rule{\textwidth}{0.4pt}


\begin{verbatim}zcat /opt/bak1\_\$(date +\%F).sql.gz|mysql
mysql << EOF
CHANGE MASTER TO
MASTER\_HOST='172.16.1.51',
MASTER\_PORT=3306,
MASTER\_USER='rep',
MASTER\_PASSWORD='oldboy123';
EOF
提示:这里忽略了两个参数MASTER\_LOG\_FILE和MASTER\_LOG\_POS,这是因为在备份的时候使用了“--master-data=1”,在导入数据库时,这两个参数已经自动设置好了。
\end{verbatim}


\noindent\rule{\textwidth}{0.4pt}


\par 这里的CHANGE MASTER后无须指定binlog文件名及其具体位置,因为这部分已经在还原数据时提前应用到数据库里了(备份时“--master-data=1”的功劳)。

\par 3.启动从库同步开关并测试主主复制

\par 1)启动从库主主复制开关,并查看复制状态。相关语句如下:

\noindent\rule{\textwidth}{0.4pt}


\begin{verbatim} [root@db02 opt]\# mysql -e "start slave;"
[root@db02 opt]\# mysql -e "show slave status\textbackslash\{\}G;"
               Slave\_IO\_State: Waiting for master to send event
                  Master\_Host: 172.16.1.51
                  Master\_User: rep
                  Master\_Port: 3306
                Connect\_Retry: 60
              Master\_Log\_File: oldboy-bin.000025
          Read\_Master\_Log\_Pos: 120
               Relay\_Log\_File: db02-relay-bin.000011
                Relay\_Log\_Pos: 284
        Relay\_Master\_Log\_File: oldboy-bin.000025
             Slave\_IO\_Running: Yes
            Slave\_SQL\_Running: Yes
              ...省略若干...
1 row in set (0.00 sec)
提示:有关show slave status结果的详细说明,请大家查看MySQL手册。
\end{verbatim}


\noindent\rule{\textwidth}{0.4pt}


\par 主主复制是否配置成功了,其中最关键的是如下所示的3项状态参数:

\noindent\rule{\textwidth}{0.4pt}


\begin{verbatim}[root@db02 opt]\# mysql -e "show slave status\textbackslash\{\}G;"|egrep "IO\_Running|SQL\_Running: |\_Behind\_Master"
             Slave\_IO\_Running: Yes
            Slave\_SQL\_Running: Yes
        Seconds\_Behind\_Master: 0
\end{verbatim}


\noindent\rule{\textwidth}{0.4pt}


\par 特别强调:在首次开启复制的过程中,如果出现复制错误,可以执行如下语句。

\noindent\rule{\textwidth}{0.4pt}


\begin{verbatim}stop slave; \#<==临时停止同步开关。
set global sql\_slave\_skip\_counter = 1; \#<==将同步指针向下移动一个,如果多次移动仍
                                        不同步,则可以重复操作。
start slave; \#<==开启同步开关。
\end{verbatim}


\noindent\rule{\textwidth}{0.4pt}


\par 最后记录下主库2的binlog状态,用于在主库1上进行复制的binlog复制的位置点:

\noindent\rule{\textwidth}{0.4pt}


\begin{verbatim}mysql> show master status;
+-------------------+----------+--------------+------------------+-------------------+
| File              | Position | Binlog\_Do\_DB | Binlog\_Ignore\_DB | Executed\_Gtid\_Set |
+-------------------+----------+--------------+------------------+-------------------+
| oldboy-bin.000027 |  1303324 |              |                  |                   |
+-------------------+----------+--------------+------------------+-------------------+
1 row in set (0.01 sec)
\end{verbatim}


\noindent\rule{\textwidth}{0.4pt}



% From text00247.html


\subsection{16.2.7 在主库1(51)上执行复制配置}


\noindent\rule{\textwidth}{0.4pt}


\begin{verbatim}[root@db01 opt]\# mysql << EOF
CHANGE MASTER TO
MASTER\_HOST='172.16.1.52',
MASTER\_PORT=3306,
MASTER\_USER='rep',
MASTER\_PASSWORD='oldboy123',
MASTER\_LOG\_FILE='oldboy-bin.000027', \#<==从上文获得。
MASTER\_LOG\_POS=1303324;              \#<==从上文获得。
EOF
[root@db01 opt]\# mysql -e "start slave;"
[root@db01 opt]\# mysql -e "show slave status\textbackslash\{\}G;"
*************************** 1. row ***************************
               Slave\_IO\_State: Waiting for master to send event
                  Master\_Host: 172.16.1.52
                  Master\_User: rep
                  Master\_Port: 3306
                Connect\_Retry: 60
              Master\_Log\_File: oldboy-bin.000027
          Read\_Master\_Log\_Pos: 1303324
               Relay\_Log\_File: db01-relay-bin.000002
                Relay\_Log\_Pos: 284
        Relay\_Master\_Log\_File: oldboy-bin.000027
             Slave\_IO\_Running: Yes
            Slave\_SQL\_Running: Yes
              ...省略若干...
1 row in set (0.00 sec)
\end{verbatim}


\noindent\rule{\textwidth}{0.4pt}


\par 至此主库1(51)和主库2(52)就实现互相同步了。


% From text00248.html


\subsection{16.2.8 在主库1和主库2进行测试}


\par 1)在主库1上建库建表,命令如下:

\noindent\rule{\textwidth}{0.4pt}


\begin{verbatim}create database two\_master;
use two\_master;
CREATE TABLE test(
  id bigint(12) NOT NULL auto\_increment COMMENT '主键',
  name varchar(12) NOT NULL COMMENT '姓名',
  PRIMARY KEY  (id)
);
\end{verbatim}


\noindent\rule{\textwidth}{0.4pt}


\par 2)然后插入数据,执行过程如下:

\noindent\rule{\textwidth}{0.4pt}


\begin{verbatim}mysql> insert into test(name) values('jeacen'),('alex'),('老男孩');
Query OK, 3 rows affected (0.01 sec)
Records: 3  Duplicates: 0  Warnings: 0
mysql> select * from test;
+----+-----------+
| id | name      |
+----+-----------+
|  1 | jeacen    |
|  3 | alex      |
|  5 | 老男孩    |
+----+-----------+
3 rows in set (0.00 sec)
\#提示:可以看到ID都是奇数。
\end{verbatim}


\noindent\rule{\textwidth}{0.4pt}


\par 3)此时登录主库2(52)查看数据:

\noindent\rule{\textwidth}{0.4pt}


\begin{verbatim}mysql> select * from test;
+----+-----------+
| id | name      |
+----+-----------+
|  1 | jeacen    |
|  3 | alex      |
|  5 | 老男孩    |
+----+-----------+
3 rows in set (0.00 sec)
\#数据和主库1相同,因为实现了复制,ID也为奇数。
\end{verbatim}


\noindent\rule{\textwidth}{0.4pt}


\par 4)登录主库2(52)继续插入数据:

\noindent\rule{\textwidth}{0.4pt}


\begin{verbatim}mysql> insert into test(name) values('卧底'),('wu\_sir'),('小曹');
Query OK, 3 rows affected (0.00 sec)
Records: 3  Duplicates: 0  Warnings: 0

mysql> select * from test;
+----+-----------+
| id | name      |
+----+-----------+
|  1 | jeacen    |
|  3 | alex      |
|  5 | 老男孩    |
|  6 | 卧底      |
|  8 | wu\_sir    |
| 10 | 小曹      |
+----+-----------+
6 rows in set (0.00 sec)
\end{verbatim}


\noindent\rule{\textwidth}{0.4pt}


\par 新插入的数据是从当前表中最大数字偶数6的基础上开始进行插入的,至此,双主复制测试成功。


% From text00249.html


\section{16.3 本章重点}


\par 1)MySQL主主复制的常见架构。

\par 2)MySQL主主复制的原理(面试时经常会问到)。

\par 3)MySQL主主复制的实践。

\par 4)生产场景下如何快速部署MySQL主主复制。

\par 5)MySQL主主复制生产故障解决思路。


% From text00250.html


\section{16.4 MySQL双主复制my.cnf完整配置对比}


\par 表16-2 MySQL双主复制my.cnf完整配置对比

\begin{figure}[htbp]
\centering
\includegraphics[width=0.8\textwidth]{Image00126.jpg}
\end{figure}




% From text00251.html


\chapter{第17章 MySQL半同步复制与GTID复制实践}


\section{
17.1 MySQL复制的多种工作方式}
}


\par 稍微深入学习MySQL的读者都会接触到复制的概念,就像前文所讲的MySQL主从复制原理是异步的复制工作方式,那么对于数据库复制,除了异步以外还有同步和半同步等工作方式,本节就对数据库的复制方式进行一个基本的介绍,并带领读者实践半同步复制的技术。


% From text00252.html


\subsection{17.1.1 异步复制介绍}


\par 这是可以用一个比喻来描述一主两从异步的概念,假设你买了两份礼物,需要同时送给你的两个好朋友,骑车找到第一个朋友家,放在她家的门口,然后打个电话告知她自己去取,接着继续到第二个朋友家,也是放到家门口,同样打个电话告知她自己去取,那么这两个传输的过程就都是异步的方式。异步的特点是不保证实时收到或者一定会收到,但效率比较高,因为不需要等着朋友出来取,直接放到她的家门口即可。

\par 默认情况下,MySQL的复制就是异步的,即在Master上将所有的更新操作都写入Binlog之后并不能确保(也不关心)所有的更新是否都复制到了Slave服务器的中继日志中,以及是否应用到了Slave数据库里。

\par 异步复制(async replication)的明显优势就是复制效率很高,但是其缺点也十分明显,那就是不同的服务器进行复制时可能会存在数据不一致的问题,甚至还可能会丢失数据;异步方式适合于常规的、普通的互联网应用场景。


% From text00253.html


\subsection{17.1.2 全同步复制介绍}


\par 同样,这里再用一个比喻来描述一主两从同步复制的概念,你依然买了两份礼物,需要同时送给你的两个好朋友,骑车找到第一个朋友家,然后打个电话把她叫出来,把礼物交到她的手里,接着继续到第二个朋友家,也是打电话告知她出来,同样把礼物交到她的手里,那么这两个传输的过程就都是同步复制的方式,同步的特点是保证礼物实时收到,效率会低一些,因为要等待朋友出来取,如果朋友再磨蹭一下,就会额外浪费很多时间。

\par 全同步复制(full sync repication)是指当主库执行完一个事务后,需要确保所有的从库都执行了该事务才返回给客户端。因为需要等待所有的从库都执行完该事务才能返回,所以全同步复制的性能必然会受到所有从库更新的拖累。

\par MySQL自身不支持同步复制,需要用第三方工具如DRBD(sync模式)等实现同步复制,严格来说,把下文讲解的半同步复制技术默认(或人为)全部应用到所有从库上也算是全同步复制。

\par 同步复制的优点是能够确保将数据实时复制到所有的从库,但是主库需要等待所有从库的写入完成,这会影响主库的更新效率,可能还会导致主库的更新延迟,适合于对数据一致性要求较高的应用场景。


% From text00254.html


\subsection{17.1.3 半同步复制}


\par 再用一个同样的比喻来描述一主两从半同步复制
 的概念,你还是买了两份礼物,需要送给你的两个好朋友,骑车找到第一个朋友家,然后打个电话把她叫出来,把礼物交到她的手里,接着继续到第二个朋友家,这次是把礼物放到她的家门口,打个电话告知她自己去取,那么第一个传输的过程就是半同步复制的方式,第二个传输的过程就是异步的方式,半同步的目的是保证礼物会被实时收到,异步的目的是提高送礼物的效率。

\par 半同步复制(semi-sync repication)介于异步复制和全同步复制之间,主库在执行完客户端提交的事务之后不是立刻返回给客户端,而是等待至少一个从库接收到并写到relay log中才返回给客户端。相对于异步复制,半同步复制提高了数据的安全性,同时它也造成了一定程度的延迟,半同步复制最好在低延时的复制节点之间使用。


% From text00255.html


\section{17.2 MySQL半同步复制原理及实践准备}


\subsection{
17.2.1 MySQL半同步复制介绍}
}


\par MySQL从5.5版本开始,即支持半同步复制。引入半同步复制功能的目的是为了保证在Master出现问题的时候,至少有一台Slave的数据是完整的。在超时的情况下半同步复制也会转换为异步复制,以保障主库业务的正常更新。

\par 半同步复制模式在一定程度上可以保证所提交的事务至少会发送给一个从库。但仅仅是保证事务已经传递到了从库上,并不能确保已经在从库上应用完成。


% From text00256.html


\subsection{17.2.2 MySQL半同步复制原理}


\par 在半同步复制的模式下,每一个事务都需要等待从库返回正常接收日志数据的信息之后,才会返回给用户更新完毕。如果在更新期间,主从库发生了网络故障或者从库宕机,那么此时主库在事务提交后会等待10秒(默认值),如果10秒内还是不能联系到该从库,就会放弃更新该从库,并向用户返回数据。这时主从库就会恢复到默认的异步复制状态。

\par 如果主机间的通信延迟较小,而且更新的事务较小,则半同步复制可以在主库更新性能损失很小的情况下,最大限度地实现某一台从库的零数据丢失。

\par 图17-1为MySQL半同步复制的逻辑图,希望读者能够更好地理解半同步复制技术。

\begin{figure}[htbp]
\centering
\includegraphics[width=0.8\textwidth]{Image00127.jpg}
\end{figure}



\par 图17-1 MySQL半同步复制的逻辑图

\par 在上述图17-1所示的逻辑图中,我们可以得知,要实现半同步复制功能,必须要在主从库中同时开启半同步复制功能,并进行适当的配置才行,在实际工作中,至少需要配置一台半同步从服务器,在主服务器更新事务处理返回给用户之前,需要确认更新已经收到并写入了从服务器的中继日志中。当出现连接从服务器超时(默认为10秒)时,主服务器会暂时切换到异步复制模式进行复制,恢复半同步复制模式的从服务器并及时接收来自主库的更新信息。


% From text00257.html


\subsection{17.2.3 MySQL半同步复制准备}


\par 1.MySQL半同步复制环境准备

\par MySQL半同步复制实践对环境的要求比较简单,既可以是单机单数据库多实例的环境,也可以是两台服务器,每个机器一个独立数据库的环境。本文以两台服务器,每个服务器均安装本书第3章单机单实例数据库的环境为例进行讲解,单机单数据库多实例的环境讲解请参考《跟老男孩学习Linux运维:Web集群实战》一书第9章的内容。

\par 
\begin{figure}[htbp]
\centering
\includegraphics[width=0.8\textwidth]{Image00006.jpg}
\end{figure}

 特别提示:


\par 1)如果是VMware虚拟环境克隆虚拟机,则会导致网卡启动无法连网的问题,解决方法请参考\href{http://oldboy.blog.51cto.com/2561410/1363853}{http://oldboy.blog.51cto.com/2561410/1363853}
 。

\par 2)如果是对安装好的MySQL的虚拟机进行克隆,则需要注意数据目录下文件auto.cnf(本文是/application/mysql/data/auto.cnf)里的UUID需要人为修改一下,使其与原始虚拟机不一致。

\par 2.MySQL半同步复制服务器角色说明

\par 主库及从库IP、端口信息如表17-1所示。

\par 表17-1 主库及从库IP、端口信息

\begin{figure}[htbp]
\centering
\includegraphics[width=0.8\textwidth]{Image00128.jpg}
\end{figure}



\par 3.MySQL半同步复制主从库配置文件对比说明

\par 请注意主从库中配置不同的部分,见表17-2半同步主从库配置文件对比说明。

\par 配置完毕后,重启所有的MySQL服务。

\par 4.配置MySQL主从复制

\par 半同步复制实现的基础是主从复制环境,因此,请读者参考前面的章节准备好主从复制的环境,当然主主复制环境也可以,此处不再赘述,配置完最好在虚拟机上做好快照,以后进行其他复制时还会用到。

\par 表17-2 半同步主从库配置文件对比说明表

\begin{figure}[htbp]
\centering
\includegraphics[width=0.8\textwidth]{Image00129.jpg}
\end{figure}




% From text00258.html


\section{17.3 MySQL半同步复制应用实践}


\subsection{
17.3.1 MySQL半同步复制插件介绍}
}


\par MySQL从5.5版本开始即支持半同步复制,本文的环境是5.6.40(同样也适合其他5.6系列的版本)版,因此,软件本身就支持了,无须再额外安装软件了,默认情况下,半同步的插件在MySQL安装目录的如下路径中:

\noindent\rule{\textwidth}{0.4pt}


\begin{verbatim}[root@db01 plugin]\# pwd
/application/mysql/lib/plugin
[root@db01 plugin]\# ls -l semi*
-rwxr-xr-x. 1 mysql mysql 425141 2月  26 2017 semisync\_master.so
-rwxr-xr-x. 1 mysql mysql 248175 2月  26 2017 semisync\_slave.so
\end{verbatim}


\noindent\rule{\textwidth}{0.4pt}



% From text00259.html


\subsection{17.3.2 MySQL主库(db01)半同步插件安装和配置}


\par 登录MySQL主库(db01)安装半同步插件,操作命令如下:

\noindent\rule{\textwidth}{0.4pt}


\begin{verbatim}mysql> install plugin rpl\_semi\_sync\_master soname 'semisync\_master.so';
Query OK, 0 rows affected (0.00 sec)
\end{verbatim}


\noindent\rule{\textwidth}{0.4pt}


\par 开始配置主库半同步插件。

\par 临时生效方法具体如下。

\par 控制Master是否开启半同步,需要将如下参数设置为ON或OFF(1或者0):

\noindent\rule{\textwidth}{0.4pt}


\begin{verbatim}mysql> set global rpl\_semi\_sync\_master\_enabled = ON; \#<==此处用1不生效。
Query OK, 0 rows affected (0.00 sec)
\end{verbatim}


\noindent\rule{\textwidth}{0.4pt}


\par 同时控制Master等待多长时间被告知Slave已收到日志,也就是所谓的超时时间:

\noindent\rule{\textwidth}{0.4pt}


\begin{verbatim}mysql> set global rpl\_semi\_sync\_master\_timeout = 10000;
Query OK, 0 rows affected (0.00 sec)
\end{verbatim}


\noindent\rule{\textwidth}{0.4pt}


\par 永久生效方法具体如下。

\par 编辑配置文件调整如下参数到[mysqld]模块下。注意,别放错位置了。参数调整如下:

\noindent\rule{\textwidth}{0.4pt}


\begin{verbatim}[mysqld]
rpl\_semi\_sync\_master\_enabled = ON
rpl\_semi\_sync\_master\_timeout = 1000
\end{verbatim}


\noindent\rule{\textwidth}{0.4pt}


\par 以上两种方法如果同时进行配置,则可以不重启数据库,生产库可以同时配置。

\par 检查主库开启及配置的插件生效方法具体如下:

\noindent\rule{\textwidth}{0.4pt}


\begin{verbatim}mysql> select * from mysql.plugin;
+----------------------+--------------------+
| name                 | dl                 |
+----------------------+--------------------+
| rpl\_semi\_sync\_master | semisync\_master.so |
+----------------------+--------------------+
1 row in set (0.00 sec)
mysql> show plugins;\#<==也可以查询INFORMATION\_SCHEMA.PLUGINS库表。
+----------------------+--------+----------------+--------------------+---------+
| Name                 | Status | Type           | Library            | License |
+----------------------+--------+----------------+--------------------+---------+
| binlog               | ACTIVE | STORAGE ENGINE | NULL               | GPL     |
...省略若干...
| partition            | ACTIVE | STORAGE ENGINE | NULL               | GPL     |
| rpl\_semi\_sync\_master | ACTIVE | REPLICATION    | semisync\_master.so | GPL     |
+----------------------+--------+----------------+--------------------+---------+
43 rows in set (0.00 sec)
mysql> show global status like '\%semi\%'; \#<==查看主库半同步复制的状态参数,细节
                                          见表17-4的详细说明。
+--------------------------------------------+-------+
| Variable\_name                              | Value |
+--------------------------------------------+-------+
| Rpl\_semi\_sync\_master\_clients               | 0     | \#<==默认值几乎都是0。
| Rpl\_semi\_sync\_master\_net\_avg\_wait\_time     | 0     |
| Rpl\_semi\_sync\_master\_net\_wait\_time         | 0     |
| Rpl\_semi\_sync\_master\_net\_waits             | 0     |
| Rpl\_semi\_sync\_master\_no\_times              | 0     |
| Rpl\_semi\_sync\_master\_no\_tx                 | 0     |
| Rpl\_semi\_sync\_master\_status                | ON    | \#<==这个为ON,表示开启
                                                     半同步复制状态。
| Rpl\_semi\_sync\_master\_timefunc\_failures     | 0     |
| Rpl\_semi\_sync\_master\_tx\_avg\_wait\_time      | 0     |
| Rpl\_semi\_sync\_master\_tx\_wait\_time          | 0     |
| Rpl\_semi\_sync\_master\_tx\_waits              | 0     |
| Rpl\_semi\_sync\_master\_wait\_pos\_backtraverse | 0     |
| Rpl\_semi\_sync\_master\_wait\_sessions         | 0     |
| Rpl\_semi\_sync\_master\_yes\_tx                | 0     |
+--------------------------------------------+-------+
14 rows in set (0.00 sec)
mysql> show global variables like '\%semi\%'; \#<==查看主库半同步复制的相关参数设置,
                                            细节详见表17-3的说明。
+------------------------------------+-------+
| Variable\_name                      | Value |
+------------------------------------+-------+
| rpl\_semi\_sync\_master\_enabled       | ON    | \#<==为ON表示已开启。
| rpl\_semi\_sync\_master\_timeout       | 1000  |
| rpl\_semi\_sync\_master\_trace\_level   | 32    |
| rpl\_semi\_sync\_master\_wait\_no\_slave | ON    |
+------------------------------------+-------+
4 rows in set (0.00 sec)
\end{verbatim}


\noindent\rule{\textwidth}{0.4pt}



% From text00260.html


\subsection{17.3.3 MySQL半同步复制参数介绍}


\par (1)主库半同步复制配置参数说明

\par 主库半同步复制可配置的参数较少,可通过“show global variables like'\%semi\%';”查看,见上文代码,具体参数说明见表17-3主库半同步复制参数说明。

\par 表17-3 主库半同步复制参数说明

\begin{figure}[htbp]
\centering
\includegraphics[width=0.8\textwidth]{Image00130.jpg}
\end{figure}



\par (2)主库半同步复制状态说明

\par 主库半同步复制状态的参数较多,但都不是很重要,了解一下即可。可通过“show global status like'\%semi\%';”查看,见上文代码,具体参数说明见表17-4主库半同步复制参数说明。

\par 表17-4 主库半同步复制状态说明

\begin{figure}[htbp]
\centering
\includegraphics[width=0.8\textwidth]{Image00131.jpg}
\end{figure}




% From text00261.html


\subsection{17.3.4 MySQL从库(db02)半同步插件安装和配置}


\par 登录MySQL从库(db02)安装半同步插件,操作命令如下:

\noindent\rule{\textwidth}{0.4pt}


\begin{verbatim}mysql> INSTALL PLUGIN rpl\_semi\_sync\_slave SONAME 'semisync\_slave.so';
Query OK, 0 rows affected (0.34 sec)
\end{verbatim}


\noindent\rule{\textwidth}{0.4pt}


\par 开始配置从库半同步插件。

\par 临时生效方法具体如下。

\par 控制Slave是否开启半同步,需要将如下参数设置为ON或OFF:

\noindent\rule{\textwidth}{0.4pt}


\begin{verbatim}mysql> SET GLOBAL rpl\_semi\_sync\_slave\_enabled = ON;
Query OK, 0 rows affected (0.00 sec)
\end{verbatim}


\noindent\rule{\textwidth}{0.4pt}


\par 永久生效方法具体如下。

\par 编辑配置文件“/etc/my.cnf”调整如下参数到[mysqld]模块下:

\noindent\rule{\textwidth}{0.4pt}


\begin{verbatim}[mysqld]
rpl\_semi\_sync\_slave\_enabled = ON
\end{verbatim}


\noindent\rule{\textwidth}{0.4pt}


\par 以上两种方法如果同时配置,则可以不重启数据库,生产库可以同时进行配置。

\par 如果在一个正在运行的Slave上开启半同步复制的功能,那么在配置半同步以后,需要重启停止Slave的I/O线程:

\noindent\rule{\textwidth}{0.4pt}


\begin{verbatim}mysql> show slave status\textbackslash\{\}G
*************************** 1. row ***************************
               Slave\_IO\_State: Waiting for master to send event
...省略若干...
             Slave\_IO\_Running: Yes  \#<==确保主从复制是两个Yes。
            Slave\_SQL\_Running: Yes
...省略若干...
 1 row in set (0.00 sec)
mysql> stop slave io\_thread; \#<==停止IO线程。
Query OK, 0 rows affected (0.00 sec)
mysql> start slave io\_thread;  \#<==启动IO线程。
Query OK, 0 rows affected (0.00 sec)
\end{verbatim}


\noindent\rule{\textwidth}{0.4pt}


\par 检查主库
 (db01)开启及配置的插件的生效方法:

\noindent\rule{\textwidth}{0.4pt}


\begin{verbatim}mysql> show global status like '\%semi\%'; \#<==查看从库半同步复制的状态。
+----------------------------+-------+
| Variable\_name              | Value |
+----------------------------+-------+
| Rpl\_semi\_sync\_slave\_status | ON    |   \#<==为ON表示已开启。
+----------------------------+-------+
1 row in set (0.00 sec)
mysql> show variables like '\%semi\%';     \#<==查看从库半同步复制插件的参数,共2个。
+---------------------------------+-------+
| Variable\_name                   | Value |
+---------------------------------+-------+
| rpl\_semi\_sync\_slave\_enabled     | ON    | \#<==表示是否激活半同步插件。
| rpl\_semi\_sync\_slave\_trace\_level | 32    | \#<==从库半同步复制的调试级别。
+---------------------------------+-------+
2 rows in set (0.00 sec)
\end{verbatim}


\noindent\rule{\textwidth}{0.4pt}


\par 同时查看主库db01上面的半同步插件是否开启,输出如图17-2,注意第一个参数clients的值变为1了(之前是0),表示主从半同步复制连接成功。

\begin{figure}[htbp]
\centering
\includegraphics[width=0.8\textwidth]{Image00132.jpg}
\end{figure}



\par 图17-2 主从半同步复制连接成功


% From text00262.html


\subsection{17.3.5 实践1:半同步复制是否配置成功测试}


\par 1)登录主库db01再次确定半同步复制插件是否已经开启,状态是否正常。输出如图17-3所示

\begin{figure}[htbp]
\centering
\includegraphics[width=0.8\textwidth]{Image00133.jpg}
\end{figure}



\par 图17-3 主库db01半同步复制插件已开启

\par 图17-3中,方框结尾的值为ON,表示主库半同步插件都已处于工作状态,请注意其他状态值都为0。

\par 2)登录从库db02确定半同步复制插件是否已经开启,并且从库的复制状态是正常的:

\noindent\rule{\textwidth}{0.4pt}


\begin{verbatim}mysql> show variables like '\%semi\%';
+---------------------------------+-------+
| Variable\_name                   | Value |
+---------------------------------+-------+
| rpl\_semi\_sync\_slave\_enabled     | ON    |
| rpl\_semi\_sync\_slave\_trace\_level | 32    |
+---------------------------------+-------+
2 rows in set (0.00 sec)
mysql> show slave status\textbackslash\{\}G
*************************** 1. row ***************************
               Slave\_IO\_State: Waiting for master to send event
...省略若干...
             Slave\_IO\_Running: Yes  \#<==确保主从复制是两个Yes。
            Slave\_SQL\_Running: Yes
...省略若干...
 1 row in set (0.00 sec)
\end{verbatim}


\noindent\rule{\textwidth}{0.4pt}


\par 3)开始写入数据,操作命令如下:

\noindent\rule{\textwidth}{0.4pt}


\begin{verbatim}mysql> create database oldboy1; \#<==建立oldboy1库。
Query OK, 1 row affected (0.00 sec)
mysql> create database oldboy2; \#<==建立oldboy2库。
Query OK, 1 row affected (0.00 sec)
mysql> show databases like 'oldboy\%';
+--------------------+
| Database (oldboy\%) |
+--------------------+
| oldboy1            |
| oldboy2            |
+--------------------+
2 rows in set (0.00 sec)
\end{verbatim}


\noindent\rule{\textwidth}{0.4pt}


\par 4)查看主库半同步复制状态,操作命令如下:

\noindent\rule{\textwidth}{0.4pt}


\begin{verbatim}mysql> show global status like '\%semi\%';
+--------------------------------------------+-------+
| Variable\_name                              | Value |
+--------------------------------------------+-------+
| Rpl\_semi\_sync\_master\_clients               | 1     |
| Rpl\_semi\_sync\_master\_net\_avg\_wait\_time     | 387   |\#<==平均网络等待时间为387毫秒。
| Rpl\_semi\_sync\_master\_net\_wait\_time         | 775   |\#<==总网络等待时间为775毫秒。
| Rpl\_semi\_sync\_master\_net\_waits             | 2     |\#<==总网络等待次数为2。
| Rpl\_semi\_sync\_master\_no\_times              | 0     |
| Rpl\_semi\_sync\_master\_no\_tx                 | 0     |
| Rpl\_semi\_sync\_master\_status                | ON    |
| Rpl\_semi\_sync\_master\_timefunc\_failures     | 0     |
| Rpl\_semi\_sync\_master\_tx\_avg\_wait\_time      | 466   |\#<==平均事务等待时间为466毫秒。
| Rpl\_semi\_sync\_master\_tx\_wait\_time          | 933   |\#<==总事务等待时间为775毫秒。
| Rpl\_semi\_sync\_master\_tx\_waits              | 2     |\#<==总事务等待次数为2。
| Rpl\_semi\_sync\_master\_wait\_pos\_backtraverse | 0     |
| Rpl\_semi\_sync\_master\_wait\_sessions         | 0     |
| Rpl\_semi\_sync\_master\_yes\_tx                | 2     | \#<==收到了2次接受的确认。
+--------------------------------------------+-------+
14 rows in set (0.00 sec)
\end{verbatim}


\noindent\rule{\textwidth}{0.4pt}


\par 从主库半同步复制状态的信息上看,复制是成功的。


% From text00263.html


\subsection{17.3.6 实践2:半同步复制超时等待测试}


\par 1)停止从库的IO线程复制状态,操作命令如下:

\noindent\rule{\textwidth}{0.4pt}


\begin{verbatim}Query OK, 0 rows affected (0.02 sec)
\end{verbatim}


\noindent\rule{\textwidth}{0.4pt}


\par 
\begin{figure}[htbp]
\centering
\includegraphics[width=0.8\textwidth]{Image00019.jpg}
\end{figure}

 注意:
 前文提到过,半同步复制原理是等待数据传输到从库的中继日志,以及向客户返回结果,因此,半同步复制和IO线程直接相关,与SQL线程无关。

\par 2)在从主库中更新数据,操作命令如图17-4所示。

\begin{figure}[htbp]
\centering
\includegraphics[width=0.8\textwidth]{Image00134.jpg}
\end{figure}



\par 图17-4 更新主从库数据

\par 从如图17-4所示的测试结果可以得出,从库停止IO线程复制之后,第一次执行删除数据库oldboy1的操作时等待了10秒之后才提交完事务,但是,第二次执行删除数据库oldboy2的操作就很快了,这是因为第一次等待从库超时之后,半同步复制状态自动转换为异步了,所以第二次及以后都会很快了。

\par 3)开启从库的IO线程复制状态,操作命令如下:

\noindent\rule{\textwidth}{0.4pt}


\begin{verbatim}mysql> start slave io\_thread;
Query OK, 0 rows affected (0.02 sec)
\end{verbatim}


\noindent\rule{\textwidth}{0.4pt}


\par 4)再次查看主库的半同步复制状态,如图17-5所示。

\begin{figure}[htbp]
\centering
\includegraphics[width=0.8\textwidth]{Image00135.jpg}
\end{figure}



\par 图17-5 查看主库的半同步复制状态

\par 因为监测到从库恢复了,因此半同步复制状态自动又恢复为ON,继续保持半同步复制状态了,至此,半同步的测试完成。


% From text00264.html


\subsection{17.3.7 实践3:主从复制故障时的半同步复制测试}


\par 1)模拟复制故障:先在从库(db02)上创建oldboy10的数据库(create database oldboy10),然后在主库(db01)上再创建oldboy10的同名数据库(create database oldboy10),最后在从库上查看复制状态,如图17-6所示。

\par 
\begin{figure}[htbp]
\centering
\includegraphics[width=0.8\textwidth]{Image00019.jpg}
\end{figure}

 注意:
 从库配置文件里不能有“slave-skip-errors=1032,1062,1007,1008,1146,1049”参数行。

\begin{figure}[htbp]
\centering
\includegraphics[width=0.8\textwidth]{Image00136.jpg}
\end{figure}



\par 图17-6 模拟复制故障

\par 2)此时查看主库的半同步复制状态,发现其依然为ON,如图17-7所示,可见半同步复制和SQL线程没有关系。

\begin{figure}[htbp]
\centering
\includegraphics[width=0.8\textwidth]{Image00137.jpg}
\end{figure}



\par 图17-7 查看主库的半同步复制状态

\par 
\begin{figure}[htbp]
\centering
\includegraphics[width=0.8\textwidth]{Image00006.jpg}
\end{figure}

 提示:
 作者在做这个实验时,出现了一次OFF状态,后来再重试就没有了。

\par 3)处理从库复制故障,操作命令如下:

\noindent\rule{\textwidth}{0.4pt}


\begin{verbatim}mysql> stop slave;
Query OK, 0 rows affected (0.01 sec)
mysql> set global sql\_slave\_skip\_counter =1 ;
Query OK, 0 rows affected (0.00 sec)
mysql> start slave;
Query OK, 0 rows affected (0.00 sec)
mysql> show slave status\textbackslash\{\}G
*************************** 1. row ***************************
               Slave\_IO\_State: Waiting for master to send event
...省略若干...
             Slave\_IO\_Running: Yes  \#<==确保主从复制是两个Yes。
            Slave\_SQL\_Running: Yes
...省略若干...
 1 row in set (0.00 sec)
\end{verbatim}


\noindent\rule{\textwidth}{0.4pt}


\par 4)分别查看主库和从库的半同步复制状态。

\par 主库半同步复制状态依然是ON,如图17-8所示。

\begin{figure}[htbp]
\centering
\includegraphics[width=0.8\textwidth]{Image00138.jpg}
\end{figure}



\par 图17-8 主库半同步复制状态

\par 从库半同步复制状态业依然是ON,如图17-9所示。

\begin{figure}[htbp]
\centering
\includegraphics[width=0.8\textwidth]{Image00139.jpg}
\end{figure}



\par 图17-9 从库半同步复制状态


% From text00265.html


\section{17.4 生产半同步复制建议及其他方案说明}


\par 对于常规的互联网应用,主从复制就够用了,对于主从一致性要求较高的应用可以使用半同步复制方案,这是一个很不错的选择,但是半同步复制方案也会有缺点,那就是半同步复制会导致主库更新性能的下降,尤其是在从库网络不稳定时会对主库更新带来更大的更新性能下降。

\par 解决办法具体如下。

\par 1)将主库半同步超时调短(例如,1~2秒)。

\par 2)半同步复制的从库硬件与主库之间的网络配置要更好一些。

\par 3)半同步从库不用提供任何业务服务(包括读服务)。

\par 除了半同步复制技术之外,还有一些方案可以用于解决数据一致性问题,具体如下。

\par 1)客户端程序实现双写数据库。

\par 2)客户端程序在写数据库的同时,写一段时间数据到磁盘上或内存中(redis)。


% From text00266.html


\section{17.5 MySQL GTID复制}


\subsection{
17.5.1 GTID复制简介}
}


\par GTID(global transaction identifier)是一个在主数据库上对每个已经提交到数据库的事务的唯一编号,这个标识不仅在主数据库上是唯一的,而且在整个复制架构中的所有数据库中都是唯一的。

\par 一个GTID由一对坐标表示,用冒号(:)分隔,代码如下所示:

\noindent\rule{\textwidth}{0.4pt}


\begin{verbatim}GTID = source\_id :transaction\_id
\end{verbatim}


\noindent\rule{\textwidth}{0.4pt}


\par 在上面的定义中,每一个GTID均代表一个数据库的事务,等号右边的source\_id表示执行事务的源服务器主库的uuid(也就是server\_uuid),而transaction\_id是一个从1开始的自增的序列号,表示在这个主库上执行的第n个事务。只要保证每台数据库的server\_uuid全局唯一,以及每台数据库生成的transaction\_id自身唯一,就能保证GTID的全局唯一性。

\par GTID是事务在数据库中提交时创建分配的唯一标识符,所有事务均与GTID一一映射。下面是一个GTID的具体形式:

\noindent\rule{\textwidth}{0.4pt}


\begin{verbatim}2E11FA47-61CA-11E1-9E33-C70AA9429562:28
\end{verbatim}


\noindent\rule{\textwidth}{0.4pt}


\par 什么是sever\_uuid?

\par MySQL 5.6用128位的server\_uuid代替了原本32位的server\_id的大部分功能。原因很简单,server\_id依赖于my.cnf的手工配置,有可能会产生冲突,而自动产生128位uuid的算法可以保证所有的MySQL uuid都不会发生冲突。

\par 在进行首次启动时,MySQL会自动生成一个server\_uuid,并且保存到数据库目录下的auto.cnf文件里,这个文件目前存在的唯一目的就是保存server\_uuid。在MySQL再次启动时其会读取auto.cnf文件,继续使用上次生成的server\_uuid。


% From text00267.html


\subsection{17.5.2 基于GTID复制技术的优缺点及工作原理}


\par 1.GTID复制的工作原理简介

\par GTID复制的工作原理具体如下。

\par 1)当主数据库进行数据更新时,会在事务前产生GTID号,一同记录到binlog日志中。

\par 2)从数据库端的I/O线程将变更的binlog数据,写入到本地的中继日志(relay log)中。

\par 3)从数据库端的SQL线程从中继日志中获取GTID号,然后对比本地的Binlog查看其是否有记录。如果有记录,则说明该GTID的事务已经执行,此时从数据库会忽略。

\par 4)如果没有记录,则从数据库就会从中继日志中获取数据并执行该GTID的事务,并记录到binlog中。

\par 根据GTID号可以知道事务最初是在哪个数据库上提交的,GTID的存在方便了主从复制的宕机切换(failover)。

\par 2.传统主从复制的宕机切换

\par 由图7-10可以看到,在MySQL的GTID功能出现以前,主从复制的宕机切换操作过程。

\begin{figure}[htbp]
\centering
\includegraphics[width=0.8\textwidth]{Image00140.jpg}
\end{figure}



\par 图17-10 主从复制的宕机切换操作过程

\par 若A服务器宕机,则需要将业务切换到B服务器上。同时,我们又需要使得C服务器与B服务器重新保持复制。这种情况的操作过程命令很简单,即在C服务器上执行“CHANGE MASTER TO MASTER\_HOST='xxx',MASTER\_LOG\_FILE='xxx',MASTER\_LOG\_POS=nnn。”

\par 问题在于同一个事务在每台数据库服务器上所在的binlog名字和位置点可能都不一样,那么怎样才能找到C当前同步的停止位置点呢,对应B的master\_log\_file和master\_log\_pos是什么的时候就变成了难题,这可能是官方推出GTID复制的原因。

\par 老男孩:其实,作者在前文中已经介绍过了,master.info文件记录了master\_log\_file和master\_log\_pos的位置,可以通过读master.info文件来找到真正的位置点,因此,在没有GITD功能以前,我们也是很快就能够搞定切换的。更进一步的,我们还可以通过第三方工具,例如MHA等MySQL集群管理工具来实现主从复制主库宕机的切换以及从库和新主库复制的问题。

\par 不过MySQL在5.6的GTID功能出现以后,对于上述问题的解决似乎变得更简单一些了。由于同一事务的GTID在所有节点上的值都是一致的,那么根据C服务器,当前停止点的GTID就能唯一定位到B上的GTID。而由于MASTER\_AUTO\_POSITION功能的出现,管理员不需要知道GTID的具体值,直接使用“CHANGE MASTER TO MASTER\_HOST='xxx',”MASTER\_AUTO\_POSITION命令就可以直接完成宕机切换的工作了。


% From text00268.html


\subsection{17.5.3 GTID的优缺点}


\par 1.GTID复制的优点

\par 
GTID复制的优点具体如下。


\par ·根据GTID可以知道事务最初是在哪个数据库上提交的。

\par ·GTID对于宕机切换有一定的方便性。

\par 2.GTID复制的缺点

\par 
GTID复制的缺点具体如下。


\par ·GTID模式实例和非GTID模式实例之间是不能进行复制的,不能混用,要么都是GTID,要么都不是。

\par ·在同一事务中更新事务表与非事务表将导致多个GTIDs分配给同一事务。

\par 官方指出,GTID包含如下三个限制,当enforce-gtid-consistency=ON时,以下三类SQL语句是不支持的:“CREATE TABLE...SELECT statements”“CREATE TEMPORARY TABLE”和“DROP TEMPORARY TABLE statements inside transactions”

\par ·同时更新传统事务表和非传统事务表时,有一个例外,如果所有非事务表都是临时表时,非事务的DML被允许在同一事务中或在同一语句中作为传统的DML。

\par Transactions or statements that update both transactional and nontransactional tables.There is an exception that nontransactional DML is allowed in the same transaction or in the same statement as transactional DML,if all nontransactional tables are temporary.


% From text00269.html


\subsection{17.5.4 MySQL GTID复制的应用及实践}


\par 1.环境准备

\par 参考17.2.3节“MySQL半同步复制准备”中1和2的内容,本文GTID复制安装和初始化的步骤与该节环境一致,只是配置文件有区别,具体见下文。

\par 2.GTID复制主从库配置文件对比说明

\par 请注意,GTID主从库配置的不同部分,请参考表17-5半同步主从库配置文件的对比说明。

\par 表17-5 GTID复制主从库配置文件对比说明

\begin{figure}[htbp]
\centering
\includegraphics[width=0.8\textwidth]{Image00141.jpg}
\end{figure}



\par 
\begin{figure}[htbp]
\centering
\includegraphics[width=0.8\textwidth]{Image00024.jpg}
\end{figure}

 说明:
 除了server-id之外,其他完全一致。配置完毕后,重启所有的MySQL服务。

\par 3.GTID复制特殊配置参数说明

\par GTID复制特殊配置参数说明具体见表17-6。

\par 表17-6 GTID复制特殊配置参数说明

\begin{figure}[htbp]
\centering
\includegraphics[width=0.8\textwidth]{Image00142.jpg}
\end{figure}



\par 4.配置MySQL GTID主从复制

\par 1)建立用于从库复制的账号及对应权限:

\noindent\rule{\textwidth}{0.4pt}


\begin{verbatim}mysql> grant replication slave on *.* to 'rep'@'172.16.1.\%' identified by 'oldboy123';
Query OK, 0 rows affected (0.01 sec)
mysql> flush privileges;
Query OK, 0 rows affected (0.02 sec)
\end{verbatim}


\noindent\rule{\textwidth}{0.4pt}


\par 2)导出主库的数据:

\noindent\rule{\textwidth}{0.4pt}


\begin{verbatim}mysqldump -A -B -x --set-gtid-purged=OFF|gzip >/opt/bak1\_\$(date +\%F).sql.gz
\end{verbatim}


\noindent\rule{\textwidth}{0.4pt}


\par 如果不加“--set-gtid-purged=OFF”参数备份,则会在dump文件里看到以下信息:

\noindent\rule{\textwidth}{0.4pt}


\begin{verbatim}SET @@GLOBAL.GTID\_PURGED='4c92bf66-0f0b-11e7-ba4a-000c292ece3d:1-2';
\end{verbatim}


\noindent\rule{\textwidth}{0.4pt}


\par 3)将主库导出的MySQL数据迁移到从库。

\par 本文采用scp进行复制,命令如下:

\noindent\rule{\textwidth}{0.4pt}


\begin{verbatim}[root@db01 \textasciitilde{}]\# scp -rp /opt/bak1\_\$(date +\%F).sql.gz root@172.16.1.52:/opt
\end{verbatim}


\noindent\rule{\textwidth}{0.4pt}


\par 4)把从主库mysqldump导出的数据恢复到从库。

\noindent\rule{\textwidth}{0.4pt}


\begin{verbatim}[root@db02 etc]\# cd /opt/
[root@db02 opt]\# gzip -d bak1\_\$(date +\%F).sql.gz
[root@db02 opt]\# mysql < bak1\_2017-09-18.sql
\end{verbatim}


\noindent\rule{\textwidth}{0.4pt}


\par 5)登录从库(52)配置复制参数。

\par 登录从数据库之后,执行去掉上述语句中注释部分的代码如下:

\noindent\rule{\textwidth}{0.4pt}


\begin{verbatim}CHANGE MASTER TO
MASTER\_HOST='172.16.1.51',
 MASTER\_PORT=3306,
MASTER\_USER='rep',
MASTER\_PASSWORD='oldboy123',
MASTER\_AUTO\_POSITION=1;              \#<== GTID复制的参数。
MASTER\_LOG\_FILE='oldboy-bin.000023', \#<==删除传统复制的binlog文件参数。
MASTER\_LOG\_POS=626;                  \#<==删除传统复制的binlog文件位置点参数。
\end{verbatim}


\noindent\rule{\textwidth}{0.4pt}


\par 6)启动从库同步开关并测试主从复制:

\noindent\rule{\textwidth}{0.4pt}


\begin{verbatim}mysql>start slave;
mysql> show slave status\textbackslash\{\}G
*************************** 1. row ***************************
               Slave\_IO\_State: Waiting for master to send event
                  Master\_Host: 172.16.1.51
                  Master\_User: rep
                  Master\_Port: 3306
                Connect\_Retry: 60
              Master\_Log\_File: oldboy-bin.000002
          Read\_Master\_Log\_Pos: 532
               Relay\_Log\_File: db02-relay-bin.000002
                Relay\_Log\_Pos: 411
        Relay\_Master\_Log\_File: oldboy-bin.000002
             Slave\_IO\_Running: Yes   \#<==为Yes就对了。
            Slave\_SQL\_Running: Yes   \#<==为Yes就对了。
            ...省略若干...
          Exec\_Master\_Log\_Pos: 532
              Relay\_Log\_Space: 614
              Until\_Condition: None
              ...省略若干...
             Master\_UUID: 4c92bf66-0f0b-11e7-ba4a-000c292ece3d
             Master\_Info\_File: /application/mysql/data/master.info
                    SQL\_Delay: 0
          SQL\_Remaining\_Delay: NULL
Slave\_SQL\_Running\_State: Slave has read all relay log; waiting for the slave I/O thread to update it
           Master\_Retry\_Count: 86400
           ...省略若干...
            Executed\_Gtid\_Set: 4c92bf66-0f0b-11e7-ba4a-000c292ece3d:1-2
                Auto\_Position: 1
1 row in set (0.00 sec)
提示:有关show slave status结果的详细说明,请大家查看MySQL手册。
\end{verbatim}


\noindent\rule{\textwidth}{0.4pt}


\par 主从复制是不是配置成功了,其中最关键的是下面3项的状态参数:

\noindent\rule{\textwidth}{0.4pt}


\begin{verbatim}[root@db02 opt]\# mysql -e "show slave status\textbackslash\{\}G;"|egrep "IO\_Running|SQL\_Running: |\_Behind\_Master"
             Slave\_IO\_Running: Yes
            Slave\_SQL\_Running: Yes
        Seconds\_Behind\_Master: 0
\end{verbatim}


\noindent\rule{\textwidth}{0.4pt}


\par 7)测试主从复制结果。在主库上写入数据,然后观察从库的数据状况:

\noindent\rule{\textwidth}{0.4pt}


\begin{verbatim}[root@db01 \textasciitilde{}]\# mysql -e "create database alex\_python;"  \#<==注意提示符:这是
                                                     主库建库的操作。
[root@db02 \textasciitilde{}]\# mysql -e "show databases like 'alex\%';"   \#<==这是从库查询检查。
+------------------+
| Database (alex\%) |
+------------------+
| alex\_python      |
+------------------+
\end{verbatim}


\noindent\rule{\textwidth}{0.4pt}


\par 根据测试可以判断,从库(172.16.1.52)和主库(172.16.1.51)是同步的,到此,GTID主从复制的基本实践就完成了。


% From text00270.html


\subsection{17.5.5 GTID如何跳过事务冲突}


\par 这个功能主要是跳过事务,代替第14章讲解的传统复制错误时使用的“set global sql\_slave\_skip\_counter=1”。

\par 由于在这里GTID必须是连续的,因此正常情况下,同一个服务器产生的GTID是不会存在空缺的,所以不能简单地skip掉一个事务,只能通过注入空事务的方法替换掉一个实际操作的事务。注入空事务的方法如下:

\noindent\rule{\textwidth}{0.4pt}


\begin{verbatim}stop slave;
set gtid\_next='xxx:N';
begin;commit;
set gtid\_next='AUTOMAIC';
start slave;
\end{verbatim}


\noindent\rule{\textwidth}{0.4pt}


\par 这里的“xxx:N”就是你的slave sql thread报错的GTID号,也即你想要跳过的GTID号。

\par 案例如下:模拟在从库中先建立数据库inca,在主库中继续创建inca,制造冲突。

\par 从库建立数据库inca的代码如下:

\noindent\rule{\textwidth}{0.4pt}


\begin{verbatim}[root@db02 opt]\# mysql -e "create database inca;"
\end{verbatim}


\noindent\rule{\textwidth}{0.4pt}


\par 主库建立数据库inca的代码如下:

\noindent\rule{\textwidth}{0.4pt}


\begin{verbatim}[root@db01 opt]\# mysql -e "create database inca;"
\end{verbatim}


\noindent\rule{\textwidth}{0.4pt}


\par 从库:

\noindent\rule{\textwidth}{0.4pt}


\begin{verbatim}[root@db02 opt]\# mysql -e "show slave status\textbackslash\{\}G;"
*************************** 1. row ***************************
               Slave\_IO\_State: Waiting for master to send event
                  Master\_Host: 172.16.1.51
           ...省略若干...
             Slave\_IO\_Running: Yes
            Slave\_SQL\_Running: No
                        ...省略若干...
                   Last\_Errno: 1007
                   Last\_Error: Error 'Can't create database 'inca'; database exists' on query. Default database: 'inca'. Query: 'create database inca'
                 Skip\_Counter: 0
          Exec\_Master\_Log\_Pos: 831
              Relay\_Log\_Space: 1055
           ...省略若干...
        Seconds\_Behind\_Master: NULL
Master\_SSL\_Verify\_Server\_Cert: No
                Last\_IO\_Errno: 0
                Last\_IO\_Error:
               Last\_SQL\_Errno: 1007
               Last\_SQL\_Error: Error 'Can't create database 'inca'; database exists' on query. Default database: 'inca'. Query: 'create database inca'
  Replicate\_Ignore\_Server\_Ids:
           ...省略若干...
     Last\_SQL\_Error\_Timestamp: 170918 15:29:39
               Master\_SSL\_Crl:
           Master\_SSL\_Crlpath:
           Retrieved\_Gtid\_Set: 4c92bf66-0f0b-11e7-ba4a-000c292ece3d:3-5
            Executed\_Gtid\_Set: 4c92bf66-0f0b-11e7-ba4a-000c292ece3d:1-4,
4c92bf66-0f0b-11e7-ba4a-000c292ece3f:1
                Auto\_Position: 1
\end{verbatim}


\noindent\rule{\textwidth}{0.4pt}


\par 故障解决实战:

\noindent\rule{\textwidth}{0.4pt}


\begin{verbatim}mysql> stop slave;     \#<==停止复制。
Query OK, 0 rows affected (0.01 sec)
mysql> set gtid\_next='4c92bf66-0f0b-11e7-ba4a-000c292ece3d:5';
\#<==这里是上文“Retrieved\_Gtid\_Set:”里的最大事务号。
Query OK, 0 rows affected (0.00 sec)
mysql> begin;commit;    \#<==执行一个空事务。
Query OK, 0 rows affected (0.00 sec)
Query OK, 0 rows affected (0.00 sec)
mysql> set gtid\_next='AUTOMATIC'; \#<==自动比较GTID,使复制继续进行。
Query OK, 0 rows affected (0.00 sec)
mysql> start slave;   \#<==开启复制。
Query OK, 0 rows affected (0.32 sec)
mysql> show slave status\textbackslash\{\}G   \#<==检查修复的结果。
*************************** 1. row ***************************
               Slave\_IO\_State: Waiting for master to send event
                  Master\_Host: 172.16.1.51
                  Master\_User: rep
                  Master\_Port: 3306
                 ...省略若干...
             Relay\_Master\_Log\_File: oldboy-bin.000002
             Slave\_IO\_Running: Yes
            Slave\_SQL\_Running: Yes
           ...省略若干...
             Master\_UUID: 4c92bf66-0f0b-11e7-ba4a-000c292ece3d
             Master\_Info\_File: /application/mysql/data/master.info
             Retrieved\_Gtid\_Set: 4c92bf66-0f0b-11e7-ba4a-000c292ece3d:3-5
            Executed\_Gtid\_Set: 4c92bf66-0f0b-11e7-ba4a-000c292ece3d:1-5,
4c92bf66-0f0b-11e7-ba4a-000c292ece3f:1
                Auto\_Position: 1
1 row in set (0.00 sec)
\#<==提示:有两个Yes,证明复制修复成功。
\end{verbatim}


\noindent\rule{\textwidth}{0.4pt}



% From text00271.html


\section{17.6 本章重点}


\par 1)MySQL复制的多种方式。

\par 2)MySQL半同步复制的原理。

\par 3)MySQL半同步复制的实践。

\par 4)生产场景中如何选择半同步复制?

\par 5)实时复制的其他解决方案思路。

\par 6)MySQL GTID概念与工作原理。

\par 7)MySQL GTID实践与故障解决方案。


% From text00272.html


\chapter{第18章 MySQL集群高可用方案MHA应用实践}


\section{
18.1 什么是MHA}
}


\par MHA(Master High Availability)是用Perl编写的一套非常流行和实用的MySQL高可用解决方案软件
 ,它的作用是保证MySQL主从复制集群中主库的高可用性,同时保证整个集群业务服务不受影响
 。当主库异常宕机后,MHA能够在1~30秒的时间内实现故障自动检测和故障自动转移,选择一个最优的从库变成新的主库,同时使得其他的从库和新的主库继续保持数据一致的状态,同时还需要保证在数据库发生故障时,将集群所有数据的损失都降到最低。

\par 在主从复制集群框架中,MHA方案能够很好地解决主从复制宕机切换过程中业务持续服务和数据一致性的问题,因此MHA方案备受企业欢迎,其在互联网企业中的应用率很高,是一套值得读者掌握好的解决方案。

\par 高可用性(high availability):通常用于描述一个系统经过专门的设计,使其减少停工时间,从而保持业务服务的高度可用性的功能说明。


% From text00273.html


\section{18.2 MHA的基本架构组成}


\par 整个MHA软件由两部分角色组成,即MHA Manager(管理节点)和MHA Node(数据节点)。MHA Manager服务可以独立部署在一台服务器(含虚拟机)上管理多个主从复制集群,也可以部署在某一台主从复制从节点或者其他应用服务器节点上,而MHA Node服务需要运行在每一个MySQL服务器上。MHA Manager会定时通过主库上的MHA Node服务监测主库,当主库出现故障时,它可以自动将最优从库(可以提前指定或由MHA判定)提升为新的主库,然后让所有其他的从库与新的主库重新保持正常的复制状态。故障的整个切换和转移的过程对客户以及应用程序几乎是完全透明的。


% From text00274.html


\section{18.3 MHA的工作原理}


\par 前面已经说过,MHA的两大重要功能,其一是主库宕机切换到新的主库,同时保证其他从库与新主库保持一致的复制状态,其二是保证整个故障切换转移过程中整个集群的数据丢失应尽可能最小,那么MHA是怎么做到的呢?

\par MHA的工作原理大体可以分为如下三个过程。

\par (1)选择新主

\par 主动宕机需要在集群中选择适合的从库作为新主库,根据MHA的配置,可选择的方法具体如下。

\par 1)根据其他从库的binlog位置点选择最新的从库作为新的主库。

\par 2)设置半同步从库,直接选择半同步的从库作为新的主库。

\par (2)数据补全

\par 在进行故障切换和转移之前,必须要进行数据补全操作,否则就算是进行故障切换了,也会面临数据丢失的问题,这是不可接受的,因此数据补全的步骤具体如下。

\par 1)如果主库服务器仍然可以连接,那么MHA会试图通过SSH连接主库,把宕机主库内的所有binlog日志保存下来;如果不可进行SSH连接,则放弃主库的binlog日志数据。

\par 2)以选择好的最新主库的binlog位置点为基准点,通过中继日志(relay log)进行数据补全,使得其他所有从库与新主库的数据保持一致。

\par 3)将宕机时从主库上保存下来的binlog日志(如果有)恢复到所有数据库节点上。

\par (3)角色切换

\par 1)将选择好的新主库提升为正式主库的角色。

\par 2)让其他的从库重新与新主库保持主从复制状态。

\par 
\begin{figure}[htbp]
\centering
\includegraphics[width=0.8\textwidth]{Image00024.jpg}
\end{figure}

 说明:
 这里面还有一个问题那就是主库IP的问题,可以通过Keepalived的VIP漂移功能解决应用程序访问数据库集群的问题,Keepalived高可用软件内容见老男孩的Web集群实战一书的第13章。

\par 图18-1表示的是MHA工作原理基本架构图解。

\begin{figure}[htbp]
\centering
\includegraphics[width=0.8\textwidth]{Image00143.jpg}
\end{figure}



\par 图18-1 MHA工作原理基本架构图解


% From text00275.html


\section{18.4 MHA工具包介绍}


\par 前文中已经说明过,MHA的解决方案由两部分组成,Manager节点和Node节点,不同的节点软件,对应的命令工具的说明具体如下。

\par (1)Manager节点软件包含的一些主要工具如表18-1所示。

\par 表18-1 Manager节点工具及说明

\begin{figure}[htbp]
\centering
\includegraphics[width=0.8\textwidth]{Image00144.jpg}
\end{figure}



\par (2)Node节点软件包含的一些主要工具如表18-2所示。

\par 表18-2 Node节点工具及说明

\begin{figure}[htbp]
\centering
\includegraphics[width=0.8\textwidth]{Image00145.jpg}
\end{figure}




% From text00276.html


\section{18.5 MHA解决方案的优点}


\par ·主库宕机后,从库可以立刻接替主库提供服务,并且故障转移的速度会很快。

\par ·主库崩溃不存在数据一致性问题。

\par ·部署MHA不需要对当前MySQL集群环境做出重大更改。

\par ·不需要添加额外的服务器(仅一台manager就可以管理上百个节点,以及多套不同的集群)。

\par ·功能强大,适应场景多,性能优秀,可工作在半同步复制和异步复制场景中,当监控MySQL状态时,仅需要每隔N秒向master发送一次ping包(默认3秒),所以对性能无影响。

\par ·只要是MySQL主从复制支持的存储引擎,MHA都支持,不会仅局限于InnoDB。


% From text00277.html


\section{18.6 MHA方案实战}


\subsection{
18.6.1 搭建MHA的先决必要条件}
}


\par ·两台以上MySQL服务器或虚拟机环境(最好不用多实例环境)。

\par ·在多台MySQL之间实现主从复制。

\par 检验方法,登录到所有从库,执行如下命令:

\noindent\rule{\textwidth}{0.4pt}


\begin{verbatim}mysql> show slave status\textbackslash\{\}G                   \#<==查看复制状态命令。
 Slave\_IO\_Running: Yes                       \#<==有两个Yes,证明主从复制成功。
 Slave\_SQL\_Running: Yes                      \#<==有两个Yes,证明主从复制成功。
\end{verbatim}


\noindent\rule{\textwidth}{0.4pt}


\par ·MySQL配置文件参数的更改

\par 用于提升主库的从库my.cnf的额外设置(也可以使全部数据库均按此参数进行设置):

\noindent\rule{\textwidth}{0.4pt}


\begin{verbatim}relay\_log\_purge = 0              \#<==不自动删除relay log,以便于宕机后修复数据。
log\_bin = /application/mysql/logs/oldboy-bin  \#<==从库开启binlog,以便于宕机
                                            后修复数据。
expire\_logs\_days = 7             \#<==自动删除7天前的binlog。
log-slave-updates = 1            \#<==从库开启Binlog,以便于宕机后修复数据。
\end{verbatim}


\noindent\rule{\textwidth}{0.4pt}



% From text00278.html


\subsection{18.6.2 MySQL节点规划}


\par 本文实战案例规划了一主两从的MySQL主从集群环境,MHA管理节点放置在一台从库上,具体的规划见表18-3。

\par 表18-3 MySQL节点规划

\begin{figure}[htbp]
\centering
\includegraphics[width=0.8\textwidth]{Image00146.jpg}
\end{figure}




% From text00279.html


\subsection{18.6.3 配置SSH密钥实现免密码登录}


\par MHA管理节点在工作时,会通过SSH服务连接到其他节点服务器进行探测或复制数据,因此,在部署MHA方案之前,需要让每套MySQL主从复制集群内的每台机器之间都能做到免密码登录(包括其本身),这个过程相对来说比较简单,步骤如下。

\par 分别在db01、db02、db03上执行如下命令,创建密钥对,并执行复制密钥的命令。

\noindent\rule{\textwidth}{0.4pt}


\begin{verbatim}ssh-keygen -t dsa -P '' -f \textasciitilde{}/.ssh/id\_dsa >/dev/null 2>&1  \#<==创建密钥对。
ssh-copy-id -i \textasciitilde{}/.ssh/id\_dsa.pub 10.0.0.51     \#<==复制到不同的机器上,含本身。
ssh-copy-id -i \textasciitilde{}/.ssh/id\_dsa.pub 10.0.0.52
ssh-copy-id -i \textasciitilde{}/.ssh/id\_dsa.pub 10.0.0.53
\#说明:
\#最终达到的结果就是,每台机器之间都要互相信任,都可以互相免密码登录到任意的其他机器。
\#例:有三台机器db01、db02、db03
\#在db01上执行ssh可以不用输入密码连接到db02、db03(也包括自己)。
\#同样,db02和db03也是一样,可以免密ssh连接到其他两台机器(包括自己)。
\#因为MHA在检测本机的时候,也是通过SSH来进行检测的,必须要做到免密。
\end{verbatim}


\noindent\rule{\textwidth}{0.4pt}


\par 
重点强调:
 在manager节点安装在MySQL主从复制集群中的一台服务器的前提下,一定要让本机也可以免密ssh登录本机。


% From text00280.html


\subsection{18.6.4 对所有的MySQL节点安装MHA Node软件}


\par 对所有MySQL节点(db01、db02、db03)安装MHA Node软件的详细步骤如下。

\par 1)安装MHA软件需要依赖包(本章开头介绍了MHA是用perl语言编写的,因此需要安装perl语言的环境):

\noindent\rule{\textwidth}{0.4pt}


\begin{verbatim}yum install perl-DBD-MySQL -y
\end{verbatim}


\noindent\rule{\textwidth}{0.4pt}


\par 2)安装MHA-node包(需要去Google上下载或者加入开篇老男孩读者群来获取):

\noindent\rule{\textwidth}{0.4pt}


\begin{verbatim}rpm -ivh mha4mysql-node-0.56-0.el6.noarch.rpm
\end{verbatim}


\noindent\rule{\textwidth}{0.4pt}


\par 在任意安装完MHA node包的机器上执行如下命令,可以看到前文提到的MHA node包里包含的工具,这些工具都是perl语言编写的脚本,功能说明前文已述,此处不再累述。

\noindent\rule{\textwidth}{0.4pt}


\begin{verbatim}[root@db02 bin]\# ls -l /usr/bin/*\_*log*
-rwxr-xr-x. 1 root root 16367 3月  31 2014 /usr/bin/apply\_diff\_relay\_logs
-rwxr-xr-x. 1 root root  4807 3月  31 2014 /usr/bin/filter\_mysqlbinlog
-rwxr-xr-x. 1 root root  8261 3月  31 2014 /usr/bin/purge\_relay\_logs
-rwxr-xr-x. 1 root root  7525 3月  31 2014 /usr/bin/save\_binary\_logs
\end{verbatim}


\noindent\rule{\textwidth}{0.4pt}


\par 3)对命令实现软链接,以便于直接使用:

\noindent\rule{\textwidth}{0.4pt}


\begin{verbatim}ln -s /application/mysql/bin/mysqlbinlog /usr/bin/mysqlbinlog
ln -s /application/mysql/bin/mysql /usr/bin/mysql
\#说明:之所以要做这两条命令的软链接,是因为MHA-Manager在做检测的时候会去这两个路径下寻找MySQL的命令。也可以直接把这两条命令复制到/usr/bin目录下,如果在前文安装MySQL时设置好了/application/mysql/bin路径对应的环境变量也可以。
\end{verbatim}


\noindent\rule{\textwidth}{0.4pt}


\par 4)登录所有MySQL数据库添加MHA管理账号并检查:

\noindent\rule{\textwidth}{0.4pt}


\begin{verbatim}grant all privileges on *.* to mha@'10.0.0.\%' identified by 'mha';
select user,host from mysql.user;
\end{verbatim}


\noindent\rule{\textwidth}{0.4pt}


\par 特别说明:所有MySQL节点全部都要执行本节的操作。


% From text00281.html


\subsection{18.6.5 MHA管理节点安装}


\par 管理节点既可以部署在任何一台机器上,也可以部署在MySQL集群的其中一台机器上,并且一个管理节点可以管理多套MySQL集群。

\par 1)使用epel源安装perl相关语言环境依赖:

\noindent\rule{\textwidth}{0.4pt}


\begin{verbatim}wget -O /etc/yum.repos.d/epel.repo http://mirrors.aliyun.com/repo/epel-6.repo
\end{verbatim}


\noindent\rule{\textwidth}{0.4pt}


\par 2)安装管理节点所需的依赖:

\noindent\rule{\textwidth}{0.4pt}


\begin{verbatim}yum install -y perl-Config-Tiny epel-release perl-Log-Dispatch perl-Parallel-ForkManager perl-Time-HiRes
\end{verbatim}


\noindent\rule{\textwidth}{0.4pt}


\par 3)安装MHA-node包(需要去Google上下载或者加入开篇老男孩读者群来获取),本文为了节省服务器,选择在db03上安装MHA-manger,上文已经安装好了node包:

\noindent\rule{\textwidth}{0.4pt}


\begin{verbatim}rpm -ivh mha4mysql-node-0.56-0.el6.noarch.rpm
\end{verbatim}


\noindent\rule{\textwidth}{0.4pt}


\par 4)安装MHA-manger包(需要去Google上下载或者加入开篇老男孩读者群来获取):

\noindent\rule{\textwidth}{0.4pt}


\begin{verbatim}rpm -ivh mha4mysql-manager-0.56-0.el6.noarch.rpm
\end{verbatim}


\noindent\rule{\textwidth}{0.4pt}


\par 在安装完MHA-manger包的机器上执行如下命令,可以看到前文提到的MHA-manger包里所包含的工具,这些工具依然都是perl语言编写的脚本,功能说明前文已述,此处不再累述:

\noindent\rule{\textwidth}{0.4pt}


\begin{verbatim}[root@db03 bin]\# ls -l /usr/bin/masterha\_*
-rwxr-xr-x. 1 root root 1995 3月  31 2014 /usr/bin/masterha\_check\_repl
-rwxr-xr-x. 1 root root 1779 3月  31 2014 /usr/bin/masterha\_check\_ssh
-rwxr-xr-x. 1 root root 1865 3月  31 2014 /usr/bin/masterha\_check\_status
-rwxr-xr-x. 1 root root 3201 3月  31 2014 /usr/bin/masterha\_conf\_host
-rwxr-xr-x. 1 root root 2517 3月  31 2014 /usr/bin/masterha\_manager
-rwxr-xr-x. 1 root root 2165 3月  31 2014 /usr/bin/masterha\_master\_monitor
-rwxr-xr-x. 1 root root 2373 3月  31 2014 /usr/bin/masterha\_master\_switch
-rwxr-xr-x. 1 root root 5171 3月  31 2014 /usr/bin/masterha\_secondary\_check
-rwxr-xr-x. 1 root root 1739 3月  31 2014 /usr/bin/masterha\_stop
\end{verbatim}


\noindent\rule{\textwidth}{0.4pt}


\par 
\begin{figure}[htbp]
\centering
\includegraphics[width=0.8\textwidth]{Image00006.jpg}
\end{figure}

 提示:
 MHA-manager不要安装在主库所在的服务器上,一旦主库因为物理宕机或者网络故障,则会无法完成故障切换,MHA是通过SSH来进行检测的,MHA-manager相当于这套高可用方案中的“大脑”。


% From text00282.html


\subsection{18.6.6 配置MHA管理节点}


\par 安装完MHA管理节点之后,需要进行相关文件的配置,实操步骤如下:

\noindent\rule{\textwidth}{0.4pt}


\begin{verbatim}[root@db03 \textasciitilde{}]\# mkdir -p /etc/mha           \#<==在/etc下创建mha目录。
[root@db03 \textasciitilde{}]\# mkdir -p /var/log/mha/app1  \#<==在/etc下创建mha目录。
[root@db03 \textasciitilde{}]\# vim /etc/mha/app1.cnf       \#<==编辑mha配置文件,增加配置内容。
[server default]                           \#<==默认模块标签。
manager\_log=/var/log/mha/app1/manager.log  \#<==配置日志路径。
manager\_workdir=/var/log/mha/app1.log      \#<==配置工作日志路径。
master\_binlog\_dir=/application/mysql/data  \#<==配置MHA保存主库binlog日志的路径。
user=mha                                   \#<==MySQL数据中授权的用户。
password=mha                               \#<==MySQL数据中授权的用户。
ping\_interval=2                       \#<==设置监控主库发送ping数据包的时间间隔,
                                       若尝试三次没有回应则自动进行failover。
repl\_user=rep                              \#<==主从复制对应的用户。
repl\_password=oldboy123                    \#<==主从复制用户对应的密码。
ssh\_user=root                              \#<==ssh远程连接服务器的用户。
report\_script=/usr/local/send\_report       \#<==设置故障发生切换后触发执行的脚本。
secondary\_check\_script=/usr/local/bin/masterha\_secondary\_check -s db03 -s db02 --user=root --master\_host=db01 --master\_ip=10.0.0.51 --master\_port=3306
\#<==当MHA Manager节点到MASTER节点(db01)的监控之间出现问题时,MHA Manager将会
       尝试从其他路径登录到MASTER(db01)节点。
\#<==注:此配置在MHA Manager节点只有单独一台机器时起作用。意思就是,在Manager节点
       联系不上Master时,通过两个从节点(db02、db03)去探视Master(db01)节点的状态。
shutdown\_script="" \#<==设置故障发生后执行主机脚本关闭故障机(防止故障机活过来发生脑裂)
[server1]                                  \#<==第一个mysql主机模块标签。
hostname=10.0.0.51                         \#<==第一个mysql主机IP。
port=3306                                  \#<==第一个mysql主机端口。
[server2]
hostname=10.0.0.53
port=3306
candidate\_master=1           \#<==设定此参数后,server2标签的主机,将优先作为主库
                               宕机的候选服务器(切换主库优先选择)。
check\_repl\_delay=0  \#<==设定此参数后,MHA会忽略主从复制延迟,将此服务器作为后选主机。
[server3]
hostname=172.16.1.52
port=3306
\end{verbatim}


\noindent\rule{\textwidth}{0.4pt}


\par 最终去掉解释说明及无关配置的案例代码如下:

\noindent\rule{\textwidth}{0.4pt}


\begin{verbatim}[server default]
manager\_log=/var/log/mha/app1/manager.log
manager\_workdir=/var/log/mha/app1.log
master\_binlog\_dir=/application/mysql/data
master\_ip\_failover\_script=/usr/local/bin/master\_ip\_failover
user=mha
password=mha
ping\_interval=2
repl\_user=rep
repl\_password=oldboy123
ssh\_user=root
shutdown\_script=""
[server1]
candidate\_master=1
check\_repl\_delay=0
hostname=10.0.0.51
port=3306
[server2]
hostname=10.0.0.52
port=3306
candidate\_master=1
check\_repl\_delay=0
[server3]
hostname=10.0.0.53
port=3306
\end{verbatim}


\noindent\rule{\textwidth}{0.4pt}


\par 至此,我们的MHA就已经安装并配置完毕了,但是我们还需要做一些动作。


% From text00283.html


\section{18.7 启动及测试MHA}


\subsection{
18.7.1 启动MHA前需要检测的要素说明}
}


\par 在启动MHA之前,我们需要先进行启动测试,如下是两个MHA启动的必要条件。

\par ·SSH免密登录是否成功。

\par ·MySQL主从复制是否成功。

\par 当然在启动MHA的时候,MHA也会进行这两项检测,但是手动检测可以更直观地看到结果,如果没有检测成功,那么问题出在哪里,这里会用到上文提及的MHA工具。


% From text00284.html


\subsection{18.7.2 检测SSH免密码登录配置}


\par 通过前文提供的脚本工具加上MHA的配置文件进行检测执行命令及结果如下:

\noindent\rule{\textwidth}{0.4pt}


\begin{verbatim}[root@db03 \textasciitilde{}]\# masterha\_check\_ssh --conf=/etc/mha/app1.cnf
Wed Feb 28 22:02:41 2018 - [warning] Global configuration file /etc/masterha\_default.cnf not found. Skipping.
Wed Feb 28 22:02:41 2018 - [info] Reading application default configuration from /etc/mha/app1.cnf..
Wed Feb 28 22:02:41 2018 - [info] Reading server configuration from /etc/mha/app1.cnf..
Wed Feb 28 22:02:41 2018 - [info] Starting SSH connection tests..
...省略若干行...
reverse mapping checking getaddrinfo for bogon [10.0.0.53] failed - POSSIBLE BREAK-IN ATTEMPT!
reverse mapping checking getaddrinfo for bogon [10.0.0.52] failed - POSSIBLE BREAK-IN ATTEMPT!
Wed Feb 28 22:02:44 2018 - [debug]   ok.
Wed Feb 28 22:02:44 2018 - [info] All SSH connection tests passed successfully.
\#<==说明:当看到结尾出现‘All SSH connection tests passed successfully’字样,
       则证明所有服务器之间的SSH免密钥登录没有问题,否则还需要参考前文重新配置SSH免密钥登录。
\end{verbatim}


\noindent\rule{\textwidth}{0.4pt}



% From text00285.html


\subsection{18.7.3 检测MySQL集群主从复制状况}


\noindent\rule{\textwidth}{0.4pt}


\begin{verbatim}masterha\_check\_repl --conf=/etc/mha/app1.cnf
\#当看到出现‘MySQL Replication Health is OK’字样,则证明主从复制没有问题。
\#启动MHA
nohup masterha\_manager --conf=/etc/mha/app1.cnf --remove\_dead\_master\_conf --ignore\_last\_failover < /dev/null > /var/log/mha/app1/manager.log 2>&1 &
\#--conf指定配置文件。
\#--remove\_dead\_master\_conf在配置文件中删除宕机的master信息。
\#--ignore\_last\_failover忽略上一次故障切换。
\end{verbatim}


\noindent\rule{\textwidth}{0.4pt}



% From text00286.html


\section{18.8 配置VIP漂移}


\subsection{
18.8.1 虚拟IP管理的两种方式}
}


\par ·通过keepalived的方式,管理VIP。

\par ·通过MHA自带的脚本方式,管理VIP。

\par 本章主要讲解如何使用MHA自带的脚本方式来管理VIP,因为keepalived的弊端在于只允许两台服务器之间进行互相切换,若使用了MHA自带的脚本,则可以实现所有机器都可以切换。


% From text00287.html


\subsection{18.8.2 配置脚本}


\noindent\rule{\textwidth}{0.4pt}


\begin{verbatim}\#在配置文件中添加以下内容,指定master\_ip\_failover脚本的路径
master\_ip\_failover\_script=/usr/local/bin/master\_ip\_failover
\#在脚本中添加VIP相关的内容(这个脚本是MHA-Manager自带的)
my \$vip = '10.0.0.55/24';
my \$key = '0';
my \$ssh\_start\_vip = "/sbin/ifconfig eth0:\$key \$vip";
my \$ssh\_stop\_vip = "/sbin/ifconfig eth0:\$key down";
\#为脚本添加执行权限(否则MHA会无法启动)
chmod +x /etc/mha/master\_ip\_failover
\#然后在主库上手动添加一个VIP
ifconfig eth0:0 10.0.0.55/24
\end{verbatim}


\noindent\rule{\textwidth}{0.4pt}


\par 
\begin{figure}[htbp]
\centering
\includegraphics[width=0.8\textwidth]{Image00006.jpg}
\end{figure}

 提示:
 大家可以将MHA与一些读写分离的软件一起结合使用,例如第19章会讲到的Atlas。


% From text00288.html


\chapter{第19章 MySQL读写分离Atlas工具实践}


\section{
19.1 什么是Atlas}
}


\par Atlas是由QIHU 360软件有限公司(纽约证券交易代码:QIHU)的Web平台部门的基础架构团队开发和维护的、基于MySQL协议的数据库中间件项目。它修复了大量的Bug,并在MySQL-Proxy 0.8.2的基础上添加了许多新功能。目前该项目已广泛应用于QIHU公司,许多MySQL业务均已连接到Atlas平台。Atlas转发的读取和写入请求数量已达到数十亿。

\par 
读写分离:
 为了确保数据库产品的稳定性,很多数据库都拥有双机热备功能。也就是说,第一台数据库服务器,是对外提供增、删、改业务的生产服务器;第二台数据库服务器,则主要是进行读的操作。


% From text00289.html


\section{19.2 Atlas的主要功能}


\par Atlas的主要功能具体如下。

\par ·读/写分离。

\par ·从库负载均衡。

\par ·IP过滤。

\par ·数据分片,自动分表。

\par ·DBA可平滑上下线数据库。

\par ·自动删除宕机的数据库服务器。

\par ·不停机重新加载配置文件。

\par 
\begin{figure}[htbp]
\centering
\includegraphics[width=0.8\textwidth]{Image00006.jpg}
\end{figure}

 提示:
 通过配置文件中的管理用户和管理端口可以针对配置文件进行修改,并保存,无须重启。


% From text00290.html


\section{19.3 Atlas与官方mysql-proxy的对比}


\par Atlas与官方mysql-proxy的对比如下。

\par ·Atlas将主流程中所有的Lua代码用C语言进行重写,Lua仅用于管理接口。

\par ·重写网络模型、线程模型。

\par ·实现了真正意义上的连接池。

\par ·优化了锁机制,性能提高数十倍。


% From text00291.html


\section{19.4 安装Atlas}


\par 根据官方提供的安装包,Atlas分为如下两种。

\par ·Atlas(普通):Atlas-2.2.1.el6.x86\_64.rpm。

\par ·Atlas(分片):Atlas-sharding\_1.0.1-el6.x86\_64.rpm。

\par Atlas的安装代码如下:

\noindent\rule{\textwidth}{0.4pt}


\begin{verbatim}\#下载Atlas安装包(普通)
wget https://github.com/Qihoo360/Atlas/releases/download/2.2.1/Atlas-2.2.1. el6.x86\_64.rpm
\#安装Atlas
rpm -ivh Atlas-2.2.1.el6.x86\_64.rpm
\end{verbatim}


\noindent\rule{\textwidth}{0.4pt}



% From text00292.html


\section{19.5 Atlas配置文件}


\par Atlas配置文件的代码如下:

\noindent\rule{\textwidth}{0.4pt}


\begin{verbatim}\#配置文件路径
/usr/local/mysql-proxy/conf/test.conf
\#配置文件标签
[mysql-proxy]
\#带\#号的为非必需的配置项目
\#管理接口的用户名
admin-username = user
\#管理接口的密码
admin-password = pwd
\#Atlas后端连接的MySQL主库的IP和端口,可设置多项,用逗号分隔
proxy-backend-addresses = 192.168.1.10:3306,192.168.1.11:3306
\#Atlas后端连接的MySQL从库的IP和端口, “@”后面的数字代表权重,用来作负载均衡,若省略则默认为1,可设置多项,用逗号分隔。
proxy-read-only-backend-addresses = 192.168.1.231:3306
\#用户名与其对应的加密过的MySQL密码,密码使用PREFIX/bin目录下的加密程序encrypt加密,以下行的user1和user2为示例,将其替换为你的MySQL的用户名和加密密码!
pwds = root: 1N/CNLSgqXuTZ6zxvGQr9A==
\#设置Atlas的运行方式,设为true时为守护进程方式,设为false时为前台方式,一般开发调试时将其设为false,线上运行时设为true,true后面不能有空格。
daemon = true
\#设置Atlas的运行方式,设为true时Atlas会启动两个进程,一个为monitor,一个为worker,monitor在worker意外退出后会自动将其重启,设为false时只有worker,没有monitor,一般开发调试时设为false,线上运行时设为true,true后面不能有空格。
keepalive = true
\#工作线程数,对Atlas的性能有很大的影响,可根据情况进行适当设置。
event-threads = 8
\#日志级别,分为message、warning、critical、error、debug五个级别。
log-level = message
\#日志存放的路径。
log-path = /usr/local/mysql-proxy/log
\#SQL日志的开关,可设置为OFF、ON、REALTIME,OFF代表不记录SQL日志,ON代表记录SQL日志,REALTIME代表记录SQL日志且实时写入磁盘,默认为OFF。
sql-log = ON
\#慢日志输出设置。若设置了该参数,则日志只输出执行时间超过sql-log-slow(单位为ms)的日志记录。若不设置该参数则输出全部日志。
\#sql-log-slow = 10
\#实例名称,用于同一台机器上多个Atlas实例间的区分。
\#instance = test
\#Atlas监听的工作接口IP和端口。
proxy-address = 0.0.0.0:3307
\#Atlas监听的管理接口IP和端口。
admin-address = 0.0.0.0:2345
\#分表设置,此例中,person为库名,mt为表名,id为分表字段,3为子表数量,可设置多项,以逗号分隔,若不分表则不需要设置该项。
\#tables = person.mt.id.3
\#默认字符集,设置该项后客户端不再需要执行SET NAMES语句。
charset = utf8
\#允许连接Atlas的客户端的IP,可以是精确IP,也可以是IP段,以逗号分隔,若不设置该项则允许所有IP连接,否则只允许列表中的IP连接。
\#client-ips = 127.0.0.1, 192.168.1
\#Atlas前面挂接的LVS的物理网卡的IP(注意不是虚IP),若有LVS且设置了client-ips,则必须设置此项,否则可以不设置。
\#lvs-ips = 192.168.1.1
\end{verbatim}


\noindent\rule{\textwidth}{0.4pt}


\par 
\begin{figure}[htbp]
\centering
\includegraphics[width=0.8\textwidth]{Image00006.jpg}
\end{figure}

 提示:
 对用户密码加密,在/usr/local/mysql-proxy/bin下有个encrypt工具,执行该工具将需要加密的密码输入在后面即可。

\noindent\rule{\textwidth}{0.4pt}


\begin{verbatim}\#例:
\#root的密码是oldboy123
/usr/local/mysql-proxy/bin/encrypt oldboy123
\#执行后会出现如下加密后的密码,写入配置文件即可。
1N/CNLSgqXuTZ6zxvGQr9A==
Atlas配置文件示例:
[mysql-proxy]
admin-username = user
admin-password = pwd
proxy-backend-addresses = 10.0.0.55:3306
proxy-read-only-backend-addresses = 10.0.0.52:3306,10.0.0.53:3306
pwds = root: 1N/CNLSgqXuTZ6zxvGQr9A==
daemon = true
keepalive = true
event-threads = 8
log-level = message
log-path = /usr/local/mysql-proxy/log
sql-log=ON
proxy-address = 0.0.0.0:33060
admin-address = 0.0.0.0:2345
charset=utf8
\end{verbatim}


\noindent\rule{\textwidth}{0.4pt}



% From text00293.html


\section{19.6 启动Atlas}


\par 启动Atlas代码如下:

\noindent\rule{\textwidth}{0.4pt}


\begin{verbatim}\#启动Atlas
/usr/local/mysql-proxy/bin/mysql-proxyd test start
\#test是上文中配置文件中的实例名字,可以自行更改
\end{verbatim}


\noindent\rule{\textwidth}{0.4pt}


\par 
\begin{figure}[htbp]
\centering
\includegraphics[width=0.8\textwidth]{Image00019.jpg}
\end{figure}

 注意:
 启动Atlas的工具是“mysql-proxyd”而不是“mysql-proxy”,它们在同一目录下,请勿混淆。


% From text00294.html


\section{19.7 Atlas管理操作}


\par 通过配置文件中的配置,可以使用管理用户、管理密码及管理端口,连接到Atlas,代码如下:

\noindent\rule{\textwidth}{0.4pt}


\begin{verbatim}\#根据上文配置连接Atlas
mysql -uuser -ppwd -h127.0.0.1 -P2345
\#查看帮助
select * from help;
测试读写分离:
mysql -uuser -ppwd -h127.0.0.1 –P33060
show variables like "server\_id";
\end{verbatim}


\noindent\rule{\textwidth}{0.4pt}


\par 图19-1是Atlas的帮助界面。

\begin{figure}[htbp]
\centering
\includegraphics[width=0.8\textwidth]{Image00147.jpg}
\end{figure}



\par 图19-1 Atlas帮助界面

\noindent\rule{\textwidth}{0.4pt}


\begin{verbatim}\#帮助注解
SELECT * FROM backends;          \#《== 查看后端代理的数据库信息。
SET OFFLINE \$backend\_id ;        \#《== 平滑下线数据库。
SET ONLINE \$backend\_id ;         \#《== 平滑上线数据库。
ADD MASTER \$backend ;            \#《== 添加一个主库。
ADD SLAVE \$backend ;             \#《== 添加一个从库。
REMOVE BACKEND \$backend\_id ;     \#《== 移除的数据库。
SELECT * FROM clients ;          \#《== 查看客户端信息。
ADD CLIENT \$client ;             \#《== 添加一个客户端。
REMOVE CLIENT \$client ;          \#《== 删除一个客户端。
SELECT * FROM pwds ;             \#《== 查看数据库的连接用户。
ADD PWD \$pwd ;                   \#《== 添加一个用户(密码不用加密)。
ADD ENPWD \$pwd ;                 \#《== 添加一个用户(密码加密)。
REMOVE PWD \$pwd ;                \#《== 删除一个用户。
SAVE CONFIG ;                    \#《== 保存配置文件。
SELECT VERSION ;                 \#《== 查看版本信息。
\end{verbatim}


\noindent\rule{\textwidth}{0.4pt}


\par 
\begin{figure}[htbp]
\centering
\includegraphics[width=0.8\textwidth]{Image00006.jpg}
\end{figure}

 提示:
 以上命令执行完毕后,无须重启服务,执行“SAVE CONFIG;”即可将配置保存到配置文件中。


% From text00295.html


\chapter{第20章 云关系型数据库}


\par 云数据库是一种被优化或部署在一个虚拟计算环境中的数据库,是一种稳定可靠、具有弹性伸缩能力的、在线的数据库服务。云数据库能够根据用户的需求,进行资源的按需扩展(CPU、内存、存储),同时进行按需付费。

\par 云数据库同比传统自建数据库具有如下优势。

\par (1)高性价比

\par 云数据库的使用者不需要购买硬件设备,也不需要构建自己的硬件基础环境,即无须投入人力维护物理基础环境。同时云数据库是基于X86的服务器,通过集群的方式来实现,性价比极高。付费方式有包年包月和按量付费,能从很大程度上降低成本。

\par (2)简单部署

\par 我们只需要注册相关云厂商的账号,然后在控制台内点击购买云数据库,即可完成云数据库的部署,即开即用。

\par 云厂商还会提供一套功能完善的Web工具,无须我们自己使用第三方工具对数据库进行管理和监控,这从很大程度上降低了数据库管理的复杂性。

\par (3)高可用性

\par 云数据库一般使用多节点部署模式。例如,在Web界面购买一个数据库实例,云厂商会提供一个不可见的备份实例,当主数据库发生异常时,云数据库会自动进行故障切换,保证数据库的高可用性。

\par 云数据库还具备自动备份等功能,保证数据库实例的高可用进行和数据安全。


% From text00296.html


\section{20.1 阿里云RDS}


\par 阿里云RDS是指阿里云的云关系型数据库(Relational Database Service,RDS),本章我们会对阿里云的云关系型数据库——RDS for MySQL做详细的介绍。

\par 目前RDS兼容MySQL、SQL Server、PostgreSQL和PPAS(高度兼容Oracle)四种关系型数据库。相关数据库的语法可以直接通用,RDS数据库还提供了容灾、备份、恢复、监控、迁移等方面的全套解决方案,彻底解决了数据库运维的烦恼。


% From text00297.html


\section{20.2 阿里云RDS for MySQL}


\par RDS for MySQL是阿里云提供的一种稳定可靠、可弹性伸缩的在线数据库服务,强调的是稳定可靠、可弹性伸缩和在线使用数据库服务,与传统数据库是有区别的,RDS for MySQL即开即用,我们从线上即时购买,即时开通就可以使用了。

\par RDS for MySQL提供了完善的Web管理界面,可通过Web界面,对数据库进行数据库实例的管理,例如,数据库的创建和删除、数据库实例数据备份和恢复、数据库实例的性能监控、数据库实例升降级等功能。

\par 
RDS for MySQL特点


\par 1)RDS本身是主从备份架构,数据有三份以上的存储备份,具有非常高的可用性和数据可靠性,服务可用性的承诺是99.95\%,数据可靠性的承诺是六个九,99.9999\%。使用克隆实例可以轻松地将数据恢复至7天内的任意时刻。

\par 2)RDS专门为用户准备了在线的数据库管理平台DMS,这是阿里云自主研发的,是专门为RDS量身定制的一个数据库管理平台,用户完全可以通过浏览器在线,安全方便地对数据库进行管理和维护,还可以在线实现轻松的数据回溯,根据备份文件和log日志可以恢复7天内的任意时刻并精确到分钟。

\par 3)阿里云通过MySQL所有的SQL操作日志记录和慢SQL的分析,能够为用户提供完整的分析报告,告知客户一些优化的建议,比如说,主键的检查、是不是有主键、索引建立得是否合适,以及在线可以看到的SQL执行计划,帮助客户优化自己的SQL语句等。

\par 4)RDS有完整的监控体系,RDS具有将近20种的性能监控和资源监控视图,可对部分资源设置报警阀值,比如说硬盘使用率达到多少时可以进行报警。


% From text00298.html


\section{20.3 阿里云RDS云数据库的相关概念}


\subsection{
20.3.1 地域}
}


\par 地域(region)是所购买的RDS实例的服务器所处的地理位置,阿里云数据中心目前提供华东1、华东2、华北1、华北2、华北3、华南1、美国西部1、美国东部1、亚太东南1、亚太东北1、欧洲中部1、中东东部1等地域的服务。

\par 
\begin{figure}[htbp]
\centering
\includegraphics[width=0.8\textwidth]{Image00019.jpg}
\end{figure}

 备注:
 不同地域之间的阿里云产品,内网不互通。


% From text00299.html


\subsection{20.3.2 可用区}


\par 可用区(zone)是指在同一地域内,电力和网络互相独立的物理区域。同一可用区内的ECS和RDS实例网络延时更小。在同一地域内可用区与可用区之间内网互通,可用区之间能够做到故障隔离。是否将云服务器ECS和RDS实例放在同一可用区内,主要取决于对容灾能力和网络延时的要求。

\par 如果您的应用需要较高的容灾能力,那么建议您将ECS和RDS实例部署在同一地域的不同可用区内。如果您的应用在实例之间需要较低的网络时延,则建议您将ECS和RDS实例创建在相同的可用区内。单可用区可有效控制云产品间的网络延迟,多可用区可轻松实现同城容灾。


% From text00300.html


\subsection{20.3.3 RDS实例}


\par RDS实例是阿里云关系型数据库的运行环境,各实例(instance)之间相互独立、资源隔离,同一实例中的不同数据库之间是资源共享的,RDS for MySQL实例目前支持的通用实例,最大CPU16核、内存为96GB、最大磁盘容量为2000GB(独显主机可以达到3TB的存储)。独享套餐实例最大CPU 16核、内存为128G、最大磁盘容量为2000GB(独显主机可以达到3TB的存储)。独占物理主机实例最大CPU60核、内存为470G、最大磁盘容量为3000GB。


% From text00301.html


\subsection{20.3.4 RDS for MySQL只读实例}


\par RDS for MySQL只读实例可分担数据库压力,提高应用的吞吐量,在对数据库有少量写请求,但有大量读请求的应用场景下,单个RDS实例可能无法应对读取带来的压力,读取压力过大会对主数据库产生影响,从而影响业务。为了实现读取能力的弹性扩展,分担数据库压力,RDS支持在同一个地域中创建一个或多个只读实例。使用只读实例能够满足大量的数据库读取需求,以此提高应用的吞吐量。

\par 只读实例是单个物理节点的架构(没有备节点),采用MySQL的原生复制功能,会将主实例的更改同步到所有只读实例,如图20-1所示。只读实例与主实例在同一地域,但可以在不同的可用区。只读实例是按时收费的。


% From text00302.html


\subsection{20.3.5 RDS for MySQL克隆实例}


\par 克隆实例可以按照指定的RDS实例批量复制出与原实例一模一样的新RDS实例,复制的内容包括实例数据和实例中可设置的参数(如备份设置、参数设置的参数)。对于需要批量创建相同实例的用户,可以使用克隆实例的功能,在一个现有实例上快速复制出多个实例。

\begin{figure}[htbp]
\centering
\includegraphics[width=0.8\textwidth]{Image00148.jpg}
\end{figure}



\par 图20-1 高冗余版云数据库架构图

\par 克隆实例目前仅支持MySQL类型的数据库。


% From text00303.html


\subsection{20.3.6 RDS for MySQL灾备实例}


\par 对于数据可靠性有较强需求的业务场景,或者是有监管需求的金融业务场景,RDS提供异地灾备实例,帮助用户提升数据的可靠性。

\par RDS通过数据传输服务(DTS)实现了主实例和异地灾备实例之间的实时同步,如图20-2所示。主实例和灾备实例均搭建了主备高可用架构,当主实例所在的区域发生了突发性自然灾害等状况,主节点(Master)和备节点(Slave)均无法连接时,可将异地灾备实例切换为主实例,在应用端修改数据库链接地址后,即可快速恢复应用的业务访问。

\begin{figure}[htbp]
\centering
\includegraphics[width=0.8\textwidth]{Image00149.jpg}
\end{figure}



\par 图20-2 云数据库灾备说明图


% From text00304.html


\subsection{20.3.7 RDS数据库}


\par 数据库是用户在一个实例下创建的逻辑单元,一个实例可以创建多个数据库,在实例内数据库命名唯一,MySQL类型的实例,最多可以创建500个数据库,而SQL Server类型的实例最多则可以创建50个数据库,所有的数据库都会共享该实例下的资源,如CPU、内存、磁盘容量等。


% From text00305.html


\subsection{20.3.8 RDS数据库账号}


\par 每个数据库账号都可以应用于多个数据库,同时每个数据库的读写权限也可以被分配给多个数据库账号,一个账号可以创建多个实例,对于MySQL和SQL Server类型的实例来说,每个实例最多可以创建500个数据库账号。


% From text00306.html


\subsection{20.3.9 RDS连接数}


\par 应用程序可以同时连接到RDS实例的连接数量。


% From text00307.html


\subsection{20.3.10 RDS磁盘容量}


\par 用户购买RDS实例时,所选择购买的磁盘大小。磁盘总容量包含数据空间的使用量、日志空间的使用量、临时文件的使用量,等。


% From text00308.html


\subsection{20.3.11 RDS for MySQL读写分离}


\par 读写分离功能当前只有RDS for MySQL5.6双机高可用版支持。RDS for MySQL5.6双机高可用版,添加只读实例之后,均可免费开通读写分离功能。

\par 此功能目前仅支持在华东2地域使用,其他地域将会逐步开通读写分离功能。


% From text00309.html


\subsection{20.3.12 RDS for MySQL三节点企业版}


\begin{figure}[htbp]
\centering
\includegraphics[width=0.8\textwidth]{Image00150.jpg}
\end{figure}



\par 图20-3 云数据库金融版架构图

\par MySQL三节点企业版(以下简称三节点)是阿里云关系型数据库RDS于2017年3月份面向高端企业级用户推出的一款完全自研的云数据库系列,除了维持原有的MySQL兼容性和可用性,我们在AliSQL内核中引入了Raft协议,借助MySQL Semi-sync Replication可实现日志多副本同步复制,以确保数据的强一致性,提供金融级的可靠性,如图20-3所示。

\par 三节点企业版已从2017年3月17日开始进行免费公测。


% From text00310.html


\subsection{20.3.13 RDS for MySQL单机版}


\par 单机版是云数据库RDS推出的全新系列,采用单个数据库节点部署架构,与主流的主备双机高可用版相比,它只包含一个节点,没有备用节点用于故障恢复,如图20-4所示。现在,MySQL和SQL Server均已支持这种新的系列。

\begin{figure}[htbp]
\centering
\includegraphics[width=0.8\textwidth]{Image00151.jpg}
\end{figure}



\par 图20-4 双机高可用版和单机版架构图

\par 双机高可用版的备库仅用于故障转移恢复,没有对外提供服务,并且数据库复制也为主库带来了额外的性能开销,因此从这个角度分析,单机基础版的性能不仅不会低于双机高可用版,甚至还会有所提升。

\par RDS单机基础版可以通过底层数据分布式存储层来保证数据多副本的可靠性,一个物理节点故障损坏不会造成数据丢失。同时,减少一个数据库节点,可以大幅节省用户的成本,售价可低至双机高可用版的一半。


% From text00311.html


\subsection{20.3.14 RDS for MySQL跨可用去迁移}


\par 一键式迁移可满足业务部署,如果当前可用区不满足业务部署的需要,则RDS支持跨可用区迁移。跨可用区迁移必须在相同的物理地域下进行,例如可将支持华东1可用区A的实例迁移至华东2可用区B,可用区在迁移过程中不会影响正常使用,数据库实例的状态,会显示为升降级中数据迁移完毕切换时候的状态,会有30s的闪断,应用程序需要预先做好数据库重连的机制。


% From text00312.html


\section{20.4 阿里云RDS for MySQL数据库实战}


\subsection{
20.4.1 RDS for MySQL创建实例}
}


\par 1)在浏览器内,打开阿里云官方网站\href{http://www.aliyun.com}{www.aliyun.com}
 ,如图20-5所示。

\par 2)输入你阿里云的账号和密码,点击登录(阿里云账号注册的过程,相信读者可以自行完成),如图20-6所示。

\par 3)登录以后界面显示如图20-7所示,点击右上角的“控制台”。

\begin{figure}[htbp]
\centering
\includegraphics[width=0.8\textwidth]{Image00152.jpg}
\end{figure}



\par 图20-5 阿里云官方网端

\begin{figure}[htbp]
\centering
\includegraphics[width=0.8\textwidth]{Image00153.jpg}
\end{figure}



\par 图20-6 阿里云登录界面

\begin{figure}[htbp]
\centering
\includegraphics[width=0.8\textwidth]{Image00154.jpg}
\end{figure}



\par 图20-7 登录后的界面

\par 控制台界面如图20-8所示。

\begin{figure}[htbp]
\centering
\includegraphics[width=0.8\textwidth]{Image00155.jpg}
\end{figure}



\par 图20-8 控制台界面

\par 4)按照图20-9所示的操作,进入RDS购买页面。

\par 依次点击“产品与服务”→“云计算基础服务”→“数据库”→“云数据库RDS版”。

\begin{figure}[htbp]
\centering
\includegraphics[width=0.8\textwidth]{Image00156.jpg}
\end{figure}



\par 图20-9 RDS购买界面

\par 5)选择地域并在此地域创建RDS实例。

\par 上文中我们已经介绍过如何选择地域,这里不再赘述。点击“新建实例”,如图20-10所示。

\begin{figure}[htbp]
\centering
\includegraphics[width=0.8\textwidth]{Image00157.jpg}
\end{figure}



\par 图20-10 选择地域

\par 6)创建RDS实例,如图20-11所示。

\begin{figure}[htbp]
\centering
\includegraphics[width=0.8\textwidth]{Image00158.jpg}
\end{figure}



\par 图20-11 创建RDS实例

\par ·计费类型


\par 计费类型分为包年包月和按量付费。当前测试使用RDS,这里我们使用按量付费即可,如果是生产环境,且需要长期使用,建议选择包年包月。

\par ·基础配置


\par 地域:选择适当的地域(上文已经介绍过)。

\par 可用区:这里可以选择单可用区和多可用区,如图20-12所示。单可用区实例,主和备都会在一个可用区内;多可用区实例,主和备实例会在不同的可用区,以提高RDS的高可用性。

\par 数据库类型:当前RDS支持MySQL、MSSQLServer、PgSQL和PPAS,这里我们选择MySQL,如图20-13所示。

\begin{figure}[htbp]
\centering
\includegraphics[width=0.8\textwidth]{Image00159.jpg}
\end{figure}



\par 图20-12 可用区选择界面

\begin{figure}[htbp]
\centering
\includegraphics[width=0.8\textwidth]{Image00160.jpg}
\end{figure}



\par 图20-13 数据库选择界面

\par 数据库版本:MySQL数据库支持5.6和5.5(单机实例支持5.7版本)。

\par 系列:双机高可用版。

\par ·网络


\par 网络类型:RDS当前支持经典网络或专有网络,访问模式。

\par ·实例规格


\par 不同规格的实例,提供不同的性能,根据业务对性能的需要,选择不同规格的实例。

\par ·存储


\par 存储空间:根据需求选择存储空间,最大2000GB。

\par ·购买量


\par 数量:一次可以购买多个实例,最多10个。

\par 上述参数选择完毕之后,点击“立即购买”。

\par 7)支付费用,界面如图20-14所示。

\begin{figure}[htbp]
\centering
\includegraphics[width=0.8\textwidth]{Image00161.jpg}
\end{figure}



\par 图20-14 支付费用界面

\par 点击“管理控制台”,如图20-15所示。

\begin{figure}[htbp]
\centering
\includegraphics[width=0.8\textwidth]{Image00162.jpg}
\end{figure}



\par 图20-15 开通成功界面

\par 可以看到,数据库为创建中状态,如图20-16所示。

\begin{figure}[htbp]
\centering
\includegraphics[width=0.8\textwidth]{Image00163.jpg}
\end{figure}



\par 图20-16 数据库状态界面

\par 至此,RDS数据库实例创建完毕。


% From text00313.html


\subsection{20.4.2 RDS for MySQL升级实例}


\par 在业务量持续增大的场景下,RDS实例的性能可能会无法满足需求。此时我们就需要考虑对RDS的性能进行升级,以满足业务对性能的需求。

\par RDS规格升级的过程,不会影响到业务使用数据库。关于升级规则,如果本地物理机硬件配置足够,那么会直接进行本地升级,此种情况下升级速度较快。如果本地物理机硬件资源不足,那么RDS会在资源充足的物理机上创建新的实例,然后进行升级,此种升级会复制当前数据库的数据到远端的数据库,数据同步过程中,对业务没有影响,升级期间无法管理数据库,例如无法添加数据库账号等。当数据库数据同步完毕后,会进行主备切换,此时会有30秒左右的闪断,需要业务支持数据库断开重连。

\par 1)登录RDS控制台,如图20-17所示。

\begin{figure}[htbp]
\centering
\includegraphics[width=0.8\textwidth]{Image00164.jpg}
\end{figure}



\par 图20-17 RDS控制台界面

\par 2)选择需要升级的RDS实例,如图20-18所示。

\begin{figure}[htbp]
\centering
\includegraphics[width=0.8\textwidth]{Image00165.jpg}
\end{figure}



\par 图20-18 选择需要升级的RDS实例

\par 3)确定规格和存储,点击“确认变更”,如图20-19所示。

\begin{figure}[htbp]
\centering
\includegraphics[width=0.8\textwidth]{Image00166.jpg}
\end{figure}



\par 图20-19 确定规格和存储

\par 以上是按量付费实例升级的操作过程。

\par 包年包月实例升级的操作步骤与此类似,这里我们不再进行演示。

\par 升级中的实例,不允许做迁移管理等操作。需要等待一段时间,才能完成实例的配置升级。

\par 4)点击“管理控制台”,如图20-20所示。

\par 5)数据库运行状态为“升降级中”,等待状态变成“运行中”,升级完毕,如图20-21所示。


% From text00314.html


\subsection{20.4.3 RDS for MySQL查看基本信息}


\par 实例关系:主实例、只读实例、灾备实例的关系。

\begin{figure}[htbp]
\centering
\includegraphics[width=0.8\textwidth]{Image00167.jpg}
\end{figure}



\par 图20-20 升级成功界面

\begin{figure}[htbp]
\centering
\includegraphics[width=0.8\textwidth]{Image00168.jpg}
\end{figure}



\par 图20-21 数据库运行状态界面

\par 基本信息配置信息:地域、可用区、实例ID、实例名称、内存、磁盘数据库版本。

\par 运行状态运行周期:可用性、是否锁定、空间使用、创建时间、付费模式、到期时间。

\par 
登录控制台


\par 1)点击“管理”或“实例名称”,如图20-22所示。

\begin{figure}[htbp]
\centering
\includegraphics[width=0.8\textwidth]{Image00169.jpg}
\end{figure}



\par 图20-22 实例列表界面

\par 2)查看基本信息,如图20-23所示。

\begin{figure}[htbp]
\centering
\includegraphics[width=0.8\textwidth]{Image00170.jpg}
\end{figure}



\par 图20-23 查看基本信息界面

\par 如表20-1所示的是基本信息表。

\par 表20-1 基本信息表

\begin{figure}[htbp]
\centering
\includegraphics[width=0.8\textwidth]{Image00171.jpg}
\end{figure}



\begin{figure}[htbp]
\centering
\includegraphics[width=0.8\textwidth]{Image00172.jpg}
\end{figure}




% From text00315.html


\subsection{20.4.4 RDS for MySQL数据库管理}


\par 1.数据库创建管理

\par 1)创建数据库的账号,如图20-24和图20-25所示。

\begin{figure}[htbp]
\centering
\includegraphics[width=0.8\textwidth]{Image00173.jpg}
\end{figure}



\par 图20-24 创建数据库的账号界面

\begin{figure}[htbp]
\centering
\includegraphics[width=0.8\textwidth]{Image00174.jpg}
\end{figure}



\par 图20-25 填入数据库账号信息界面

\par 2)创建数据库并授权数据库的管理账号,如图20-26和图20-27所示。

\begin{figure}[htbp]
\centering
\includegraphics[width=0.8\textwidth]{Image00175.jpg}
\end{figure}



\par 图20-26 创建数据库管理的界面

\begin{figure}[htbp]
\centering
\includegraphics[width=0.8\textwidth]{Image00176.jpg}
\end{figure}



\par 图20-27 授权数据库的管理账号界面

\par 2.数据库权限管理

\par 1)数据库账号管理界面,如图20-28所示。

\begin{figure}[htbp]
\centering
\includegraphics[width=0.8\textwidth]{Image00177.jpg}
\end{figure}



\par 图20-28 数据库账号管理界面

\par 2)重置账号密码,点击“重置密码”标签,如图20-29所示。

\begin{figure}[htbp]
\centering
\includegraphics[width=0.8\textwidth]{Image00178.jpg}
\end{figure}



\par 图20-29 重置账号密码界面

\par 3)修改账号权限,点击“修改权限”标签,可设置账号,是哪个数据库的读权限或读写权限,以及授权账号到更多数据库,如图20-30所示。

\begin{figure}[htbp]
\centering
\includegraphics[width=0.8\textwidth]{Image00179.jpg}
\end{figure}



\par 图20-30 修改账号权限界面

\par 4)删除账号,点击“删除”标签,如图20-31所示。

\begin{figure}[htbp]
\centering
\includegraphics[width=0.8\textwidth]{Image00180.jpg}
\end{figure}



\par 图20-31 删除账号界面

\par 5)查看账号基本信息,如图20-32所示。

\begin{figure}[htbp]
\centering
\includegraphics[width=0.8\textwidth]{Image00181.jpg}
\end{figure}



\par 图20-32 查看账号基本信息界面

\par 由图20-32可以看到,账号的状态是激活状态,并且可以查询账号管理几个数据库,以及对应数据库的管理权限。

\par 3.数据库维护管理

\par 1)重启数据库,如图20-33和图20-34所示。

\begin{figure}[htbp]
\centering
\includegraphics[width=0.8\textwidth]{Image00182.jpg}
\end{figure}



\par 图20-33 数据库管理界面

\begin{figure}[htbp]
\centering
\includegraphics[width=0.8\textwidth]{Image00183.jpg}
\end{figure}



\par 图20-34 重启数据库界面

\par 输入验证码后,进行实例重启,重启过程中实例不可用。

\par 从如图20-35所示的RDS管理控制台中可以看到,数据库状态为“重启中”。

\begin{figure}[htbp]
\centering
\includegraphics[width=0.8\textwidth]{Image00184.jpg}
\end{figure}



\par 图20-35 RDS管理控制台界面

\par 点击“管理”或“实例名称”,进入实例基本信息页面,点击图20-36中的标识1,可以看到重启任务的进度。

\begin{figure}[htbp]
\centering
\includegraphics[width=0.8\textwidth]{Image00185.jpg}
\end{figure}



\par 图20-36 实例基本信息界面查看重启进度

\par 2)登录数据库,如图20-37所示,进入RDS实例的数据库管理界面,点击“登录数据库”。

\begin{figure}[htbp]
\centering
\includegraphics[width=0.8\textwidth]{Image00186.jpg}
\end{figure}



\par 图20-37 登录数据库界面

\par 点击“登录数据库”后,进入数据库管理工具DMS管理界面,1)选择正确的数据库,2)输入数据库用户名和密码,3)点击“登录”,如图20-38所示。

\begin{figure}[htbp]
\centering
\includegraphics[width=0.8\textwidth]{Image00187.jpg}
\end{figure}



\par 图20-38 DMS管理界面

\par 点击“登录”后,因为是第一次管理数据库,因此会提示添加白名单,允许图20-39中的地址段可以管理数据库,这些地址为DMS机器的IP地址。点击“设置所有实例”,如果不设置,DMS将无法管理数据库。

\begin{figure}[htbp]
\centering
\includegraphics[width=0.8\textwidth]{Image00188.jpg}
\end{figure}



\par 图20-39 添加白名单界面

\par 
\begin{figure}[htbp]
\centering
\includegraphics[width=0.8\textwidth]{Image00019.jpg}
\end{figure}

 备注:
 如果点击“不设置”,则没有添加上述地址到白名单,请手动添加白名单。

\par 通过数据管理DMS登录数据库后,可以通过DMS管理界面,查看当前数据库的IOPS、CPU、连接数、存储空间等信息等,性能信息监控每4秒采集一次,如图20-40所示。

\begin{figure}[htbp]
\centering
\includegraphics[width=0.8\textwidth]{Image00189.jpg}
\end{figure}



\par 图20-40 DMS管理界面,查看数据库信息


% From text00316.html


\subsection{20.4.5 RDS for MySQL远程访问}


\par 1.RDS for MySQL访问模式

\par RDS for MySQL支持两种访问模式:标准模式和高安全模式。

\par (1)标准模式

\par 使用负载均衡屏蔽了数据库引擎HA切换对应用层的影响,标准模式响应时间最短,如图20-41所示。

\begin{figure}[htbp]
\centering
\includegraphics[width=0.8\textwidth]{Image00190.jpg}
\end{figure}



\par 图20-41 RDS for MySQL标准模式

\par (2)高安全模式

\par 高安全模式可以预防90\%的链路闪断,因所有的SQL语句都会经过代理层,高安全模式可以支持SQL注入的防御,可以分析SQL语句,来防护SQL注入,目前支持MySQL和PostgreSQL,高安全模式会增加响应时间,如图20-42所示。

\begin{figure}[htbp]
\centering
\includegraphics[width=0.8\textwidth]{Image00191.jpg}
\end{figure}



\par 图20-42 RDS for MySQL高安全模式

\par 2.RDS for MySQL切换访问模式

\par RDS for MySQL切换访问模式依次如图20-43到图20-46所示。

\begin{figure}[htbp]
\centering
\includegraphics[width=0.8\textwidth]{Image00192.jpg}
\end{figure}



\par 图20-43 数据库连接界面选择“切换访问模式”

\begin{figure}[htbp]
\centering
\includegraphics[width=0.8\textwidth]{Image00193.jpg}
\end{figure}



\par 图20-44 切换访问模式界面

\begin{figure}[htbp]
\centering
\includegraphics[width=0.8\textwidth]{Image00194.jpg}
\end{figure}



\par 图20-45 输入手机验证码

\begin{figure}[htbp]
\centering
\includegraphics[width=0.8\textwidth]{Image00195.jpg}
\end{figure}



\par 图20-46 实例列表界面查看基本信息

\par 3.RDS for MySQL申请外网地址

\par (1)标准模式

\par RDS for MySQL申请外网地址界面,如图20-47和图20-48所示。

\begin{figure}[htbp]
\centering
\includegraphics[width=0.8\textwidth]{Image00196.jpg}
\end{figure}



\par 图20-47 数据库连接界面选择“申请外网地址”

\par 
\begin{figure}[htbp]
\centering
\includegraphics[width=0.8\textwidth]{Image00019.jpg}
\end{figure}

 注意:
 如果需要同时使用内网和外网地址,请将“标准模式”切换为“高安全模式”。外网服务器IP地址需要设置白名单之后,方可通过外网地址连接实例。

\begin{figure}[htbp]
\centering
\includegraphics[width=0.8\textwidth]{Image00197.jpg}
\end{figure}



\par 图20-48 申请外网地址界面

\par 申请外网地址,状态显示为“内外网切换中”,如图20-49所示。

\begin{figure}[htbp]
\centering
\includegraphics[width=0.8\textwidth]{Image00198.jpg}
\end{figure}



\par 图20-49 实例列表界面查看基本信息

\par (2)高安全模式

\par 申请内网地址,如图20-50所示。

\begin{figure}[htbp]
\centering
\includegraphics[width=0.8\textwidth]{Image00199.jpg}
\end{figure}



\par 图20-50 申请内网地址前切换为高安全模式

\par 切换回“高安全模式”,申请内网地址,高安全模式可以同时支持内网地址和外网地址。

\par 已经是高安全访问模式,申请内网地址,如图20-51和图20-52所示。

\begin{figure}[htbp]
\centering
\includegraphics[width=0.8\textwidth]{Image00200.jpg}
\end{figure}



\par 图20-51 数据库连接界面选择“申请内网地址”

\begin{figure}[htbp]
\centering
\includegraphics[width=0.8\textwidth]{Image00201.jpg}
\end{figure}



\par 图20-52 是否确定申请内网地址

\par 点击“确定”之后,回到RDS管理菜单可以查看工作状态,如图20-53所示。

\begin{figure}[htbp]
\centering
\includegraphics[width=0.8\textwidth]{Image00202.jpg}
\end{figure}



\par 图20-53 实例列表界面查看基本信息

\par 等待任务完成后,就可以看到RDS已经有了内网地址和外网地址,如图20-54所示。

\begin{figure}[htbp]
\centering
\includegraphics[width=0.8\textwidth]{Image00203.jpg}
\end{figure}



\par 图20-54 RDS已经有了内网地址和外网地址

\par 4.RDS for MySQL连接地址修改

\par RDS的连接地址,可以进行自定义,这样便于维护管理,如图20-55所示。

\begin{figure}[htbp]
\centering
\includegraphics[width=0.8\textwidth]{Image00204.jpg}
\end{figure}



\par 图20-55 RDS连接地址可进行修改

\par 点击图20-55中的“修改连接地址”,选择内网或外网地址,如图20-56所示。

\begin{figure}[htbp]
\centering
\includegraphics[width=0.8\textwidth]{Image00205.jpg}
\end{figure}



\par 图20-56 选择网络类型界面

\par 连接地址,可以更改为比较有意义的名称,例如WordPress-db,同时还可以自定义数据库的端口号,内网地址和外网地址都可以进行自定义,如图20-57和20-58所示。

\begin{figure}[htbp]
\centering
\includegraphics[width=0.8\textwidth]{Image00206.jpg}
\end{figure}



\par 图20-57 自定义数据库端口和内网地址

\begin{figure}[htbp]
\centering
\includegraphics[width=0.8\textwidth]{Image00207.jpg}
\end{figure}



\par 图20-58 自定义数据库端口和外网地址

\par 点击图20-59中的“点击获取”,输入验证码后开始更改。

\begin{figure}[htbp]
\centering
\includegraphics[width=0.8\textwidth]{Image00208.jpg}
\end{figure}



\par 图20-59 输入手机验证码


% From text00317.html


\subsection{20.4.6 RDS for MySQL备份与恢复}


\par (1)自动备份

\par RDS提供了多种类型的备份,MySQL支持物理备份和逻辑备份,MySQL支持全量备份和增量备份。备份开始时间可由用户根据自己的业务低峰进行灵活配置,所有备份文件免费保留7天。

\par (2)手动备份

\par 用户在需要时可以临时性发起备份操作,备份文件免费保留7天。

\par (3)数据回溯

\par 利用备份文件和日志,RDS可以生成一个7天内任意时刻的克隆实例,用户可在校验数据无误之后,再进行数据恢复操作。创建克隆实例操作不影响用户当前实例的正常运行。

\par (4)备份文件下载

\par RDS会将用户备份文件免费保留7天,在此期间用户可登录RDS管理控制台,将备份文件下载至本地。

\par 1.RDS for MySQL默认备份策略

\par RDS备份查询如图20-60所示。

\begin{figure}[htbp]
\centering
\includegraphics[width=0.8\textwidth]{Image00209.jpg}
\end{figure}



\par 图20-60 RDS备份查询界面

\par 当RDS创建完毕之后,默认会有备份策略设置。

\begin{figure}[htbp]
\centering
\includegraphics[width=0.8\textwidth]{Image00210.jpg}
\end{figure}



\par 图20-61 RDS默认备份策略设置界面

\par 2.RDS for MySQL自动备份策略设置

\par 如图20-62所示,选择好备份周期和备份时间之后,RDS将会根据设置开启自定义备份策略。

\par 
\begin{figure}[htbp]
\centering
\includegraphics[width=0.8\textwidth]{Image00019.jpg}
\end{figure}

 注意:
 应尽量避免在业务高峰期备份数据库。

\begin{figure}[htbp]
\centering
\includegraphics[width=0.8\textwidth]{Image00211.jpg}
\end{figure}



\par 图20-62 RDS根据备份设置开启自动备份策略

\par 3.RDS for MySQL手动备份

\par 如图20-63所示,手动备份可以选择物理备份和逻辑备份。

\begin{figure}[htbp]
\centering
\includegraphics[width=0.8\textwidth]{Image00212.jpg}
\end{figure}



\par 图20-63 手动备份选择备份方式界面

\par 如图20-64所示,逻辑备份可以选择实例备份和单库备份。

\begin{figure}[htbp]
\centering
\includegraphics[width=0.8\textwidth]{Image00213.jpg}
\end{figure}



\par 图20-64 逻辑备份选择界面

\par 逻辑备份既可以选择备份单个库,也可以选择实例级别的备份,即备份所有的库。

\par 任务执行完毕后可以查看任务的执行状态,如图20-65所示。

\begin{figure}[htbp]
\centering
\includegraphics[width=0.8\textwidth]{Image00214.jpg}
\end{figure}



\par 图20-65 查看任务的执行状态

\par 4.RDS for MySQL物理备份和逻辑备份比较

\par RDS使用mysqldump对MySQL数据库进逻辑全量备份,使用开源软件Xtrabackup对MySQL数据库进行实例级别的物理全量备份,具体见表20-2。

\par 表20-2 逻辑备份和物理备份对比表

\begin{figure}[htbp]
\centering
\includegraphics[width=0.8\textwidth]{Image00215.jpg}
\end{figure}



\begin{figure}[htbp]
\centering
\includegraphics[width=0.8\textwidth]{Image00216.jpg}
\end{figure}



\par 5.RDS for MySQL下载备份文件

\par RDS for MySQL下载备份文件如图20-66和图20-67所示。

\begin{figure}[htbp]
\centering
\includegraphics[width=0.8\textwidth]{Image00217.jpg}
\end{figure}



\par 图20-66 备份恢复界面选择数据备份下载

\begin{figure}[htbp]
\centering
\includegraphics[width=0.8\textwidth]{Image00218.jpg}
\end{figure}



\par 图20-67 实例备份文件下载确认界面

\par 下载到本地后是一个tar包,可以解压查看,如图20-68所示。

\begin{figure}[htbp]
\centering
\includegraphics[width=0.8\textwidth]{Image00219.jpg}
\end{figure}



\par 图20-68 下载到本地的压缩包

\par 6.RDS for MySQL恢复

\par 什么情况下需要恢复实例?

\par 1)系统上线前,做功能的反复验证时。

\par 2)系统运行中,出现脏数据,无法在线修复时。

\par 云数据库RDS恢复,可以在线完成。

\par 7.RDS for MySQL覆盖性恢复

\par RDS for MySQL覆盖性恢复界面如图20-69和图20-70所示。

\begin{figure}[htbp]
\centering
\includegraphics[width=0.8\textwidth]{Image00220.jpg}
\end{figure}



\par 图20-69 备份恢复界面选择数据备份恢复

\begin{figure}[htbp]
\centering
\includegraphics[width=0.8\textwidth]{Image00221.jpg}
\end{figure}



\par 图20-70 覆盖恢复实例备份集确认界面

\par 输入验证码后,开始恢复,如图20-71所示。

\begin{figure}[htbp]
\centering
\includegraphics[width=0.8\textwidth]{Image00222.jpg}
\end{figure}



\par 图20-71 输入验证码界面

\par 回到RDS主管理界面,可以看到数据库的状态是“备份恢复中”,根据备份量的大小,时间会有所不同,如图20-72所示。

\begin{figure}[htbp]
\centering
\includegraphics[width=0.8\textwidth]{Image00223.jpg}
\end{figure}



\par 图20-72 实例列表界面查看基本信息

\par 8.RDS for MySQL通过克隆实例进行数据恢复

\par RDS for MySQL通过克隆实例进行数据恢复如图20-73所示。

\begin{figure}[htbp]
\centering
\includegraphics[width=0.8\textwidth]{Image00224.jpg}
\end{figure}



\par 图20-73 RDS通过克隆实例进行数据恢复的界面

\par 克隆实例可以选择主实例的备份集,或者在备份有效存储时间内的时间点上复制出一个新的实例。克隆实例只支持主实例的克隆,若主实例下挂载有只读实例和灾备实例,那么克隆时只克隆该主实例,不克隆其下的只读实例和灾备实例。具体操作如图20-74和图20-75所示。

\begin{figure}[htbp]
\centering
\includegraphics[width=0.8\textwidth]{Image00225.jpg}
\end{figure}



\par 图20-74 克隆实例的方式

\begin{figure}[htbp]
\centering
\includegraphics[width=0.8\textwidth]{Image00226.jpg}
\end{figure}



\par 图20-75 订单确认界面

\par 回到RDS管理控制台,可以看到克隆实例正在创建中,如图20-76所示。

\par 进入主实例界面可以查看到,状态是“克隆实例中”,如图20-77所示。

\begin{figure}[htbp]
\centering
\includegraphics[width=0.8\textwidth]{Image00227.jpg}
\end{figure}



\par 图20-76 实例列表界面查看基本信息

\begin{figure}[htbp]
\centering
\includegraphics[width=0.8\textwidth]{Image00228.jpg}
\end{figure}



\par 图20-77 主实例界面查看克隆状态

\par 克隆完毕,如图20-78所示。

\begin{figure}[htbp]
\centering
\includegraphics[width=0.8\textwidth]{Image00229.jpg}
\end{figure}



\par 图20-78 克隆完毕界面

\par 下面我们手动删除原数据库里面的内容。点击数据库实例的名称,如图20-79所示。

\begin{figure}[htbp]
\centering
\includegraphics[width=0.8\textwidth]{Image00230.jpg}
\end{figure}



\par 图20-79 选择数据库实例的界面

\par 删除数据库实例里面的数据库,如图20-80所示。

\begin{figure}[htbp]
\centering
\includegraphics[width=0.8\textwidth]{Image00231.jpg}
\end{figure}



\par 图20-80 确认删除数据库的界面

\par 如图20-81所示,可以看到数据库是“删除中”的状态,此时数据库开始删除。

\begin{figure}[htbp]
\centering
\includegraphics[width=0.8\textwidth]{Image00232.jpg}
\end{figure}



\par 图20-81 开始删除数据库的界面

\par 回到RDS管理界面,点击新建立的克隆实例名称,如图20-82所示。

\begin{figure}[htbp]
\centering
\includegraphics[width=0.8\textwidth]{Image00233.jpg}
\end{figure}



\par 图20-82 选择新建立的克隆实例

\par 在克隆实例内,可以看到数据库还存在,如图20-83所示。

\begin{figure}[htbp]
\centering
\includegraphics[width=0.8\textwidth]{Image00234.jpg}
\end{figure}



\par 图20-83 克隆实例中的数据库还存在

\par 现在我们使用数据传输工具DTS,开始迁移恢复数据库。点击“迁移数据库”,后续步骤如图20-84到图20-87所示。

\begin{figure}[htbp]
\centering
\includegraphics[width=0.8\textwidth]{Image00235.jpg}
\end{figure}



\par 图20-84 选择迁移数据库界面

\begin{figure}[htbp]
\centering
\includegraphics[width=0.8\textwidth]{Image00236.jpg}
\end{figure}



\par 图20-85 创建迁移任务界面

\begin{figure}[htbp]
\centering
\includegraphics[width=0.8\textwidth]{Image00237.jpg}
\end{figure}



\par 图20-86 填写源库及目标库的信息

\par 数据迁移任务开始,如图20-88所示。

\par 数据迁移中,如图20-89所示。

\begin{figure}[htbp]
\centering
\includegraphics[width=0.8\textwidth]{Image00238.jpg}
\end{figure}



\par 图20-87 选择链路规格启动迁移界面

\begin{figure}[htbp]
\centering
\includegraphics[width=0.8\textwidth]{Image00239.jpg}
\end{figure}



\par 图20-88 数据迁移任务开始界面

\begin{figure}[htbp]
\centering
\includegraphics[width=0.8\textwidth]{Image00240.jpg}
\end{figure}



\par 图20-89 数据迁移进行中界面

\par 数据迁移完毕,如图20-90所示。

\begin{figure}[htbp]
\centering
\includegraphics[width=0.8\textwidth]{Image00241.jpg}
\end{figure}



\par 图20-90 数据迁移完毕界面

\par 回到数据,查看数据,是否从灾备实例内恢复到了源数据库中,如图20-91所示。

\begin{figure}[htbp]
\centering
\includegraphics[width=0.8\textwidth]{Image00242.jpg}
\end{figure}



\par 图20-91 查看源数据库中是否已恢复

\par 可以看到数据库已经恢复,如图20-92所示。

\begin{figure}[htbp]
\centering
\includegraphics[width=0.8\textwidth]{Image00243.jpg}
\end{figure}



\par 图20-92 源数据库中已恢复成功

\par 9.可维护的时间段

\par 阿里云的后台运维人员将在指定的时间段内进行例行维护。维护期间可能会造成RDS闪断,请确定应用程序有自动重连的功能。可维护时间段一定要选择在业务低峰期。需要根据业务特性进行维护时间段的选择。

\par 可维护时间段的选择如图20-93和图20-94所示。

\begin{figure}[htbp]
\centering
\includegraphics[width=0.8\textwidth]{Image00244.jpg}
\end{figure}



\par 图20-93 基本信息中选择可维护时间段的配置信息

\begin{figure}[htbp]
\centering
\includegraphics[width=0.8\textwidth]{Image00245.jpg}
\end{figure}



\par 图20-94 设置可维护时间段


% From text00318.html


\section{20.5 RDS for MySQL性能优化、报警管理及安全控制}


\par ·性能优化:慢日志查询、SQL运行报告、缺失索引。

\par ·阈值报警:磁盘空间、CPU使用率、连接数使用率、IOPS使用率。

\par ·安全控制:白名单设置、SQL注入告警。


% From text00319.html


\subsection{20.5.1 RDS for MySQL资源监控}


\par RDS for MySQL资源监控包括CPU和内存利用率(\%)、磁盘空间(单位:MB)、IOPS(单位:次/秒)、当前总连接数、网络流量(单位:KB),如图20-95所示。

\begin{figure}[htbp]
\centering
\includegraphics[width=0.8\textwidth]{Image00246.jpg}
\end{figure}



\par 图20-95 资源监控界面

\par 1.引擎监控

\par 引擎监控界面如图20-96所示。

\par 2.云监控报警

\par 监控与报警的报警界面如图20-97所示。

\par 图20-98所示的是默认的报警规则,可以根据需要自行修改。


% From text00320.html


\subsection{20.5.2 RDS for MySQL数据安全性}


\par 1.白名单设置

\par 顾名思义,白名单是允许哪些机器可以访问数据库,只有添加了白名单以后,客户端才可以连接上数据库,如图20-99和图20-100所示。

\begin{figure}[htbp]
\centering
\includegraphics[width=0.8\textwidth]{Image00247.jpg}
\end{figure}



\par 图20-96 引擎监控界面

\begin{figure}[htbp]
\centering
\includegraphics[width=0.8\textwidth]{Image00248.jpg}
\end{figure}



\par 图20-97 云监控报警界面

\begin{figure}[htbp]
\centering
\includegraphics[width=0.8\textwidth]{Image00249.jpg}
\end{figure}



\par 图20-98 默认的报警规则界面

\begin{figure}[htbp]
\centering
\includegraphics[width=0.8\textwidth]{Image00250.jpg}
\end{figure}



\par 图20-99 白名单设置界面

\begin{figure}[htbp]
\centering
\includegraphics[width=0.8\textwidth]{Image00251.jpg}
\end{figure}



\par 图20-100 添加名单分组界面

\par 2.SQL审计

\par 如图20-101所示的是开始SQL审计界面。

\begin{figure}[htbp]
\centering
\includegraphics[width=0.8\textwidth]{Image00252.jpg}
\end{figure}



\par 图20-101 开始进行SQL审计的界面


% From text00321.html


\subsection{20.5.3 RDS for MySQL性能优化}


\par 1.诊断报告

\par 诊断报告内会详细说明数据库内的语句是否有问题,以及数据库的相关优化建议。建议多看诊断报告以判断数据库的状态,如图20-102所示。

\begin{figure}[htbp]
\centering
\includegraphics[width=0.8\textwidth]{Image00253.jpg}
\end{figure}



\par 图20-102 创建诊断报告的界面

\par 2.SQL分析

\par SQL分析界面如图20-103所示。

\begin{figure}[htbp]
\centering
\includegraphics[width=0.8\textwidth]{Image00254.jpg}
\end{figure}



\par 图20-103 SQL分析的界面

\par 3.资源分析

\par 查看数据库的资源利用率,通过资源利用率可以分析数据资源是否被有效使用,如图20-104所示。

\begin{figure}[htbp]
\centering
\includegraphics[width=0.8\textwidth]{Image00255.jpg}
\end{figure}



\par 图20-104 资源分析界面


% From text00322.html


\section{20.6 RDS for MySQL日志管理}


\subsubsection{日志管理}


\par 日志管理可以进行数据库相关的错误日志和慢日志查询,如图20-105和图20-106所示。

\par 1.错误日志

\begin{figure}[htbp]
\centering
\includegraphics[width=0.8\textwidth]{Image00256.jpg}
\end{figure}



\par 图20-105 错误日志查询界面

\par 2.慢日志明细

\begin{figure}[htbp]
\centering
\includegraphics[width=0.8\textwidth]{Image00257.jpg}
\end{figure}



\par 图20-106 慢日志明细查询界面

\par 3.慢日志统计

\par 数据库慢日志,可以在控制台上进行统计,统计完毕之后还可以导出,以便于我们对数据库进行优化,如图20-107所示。

\begin{figure}[htbp]
\centering
\includegraphics[width=0.8\textwidth]{Image00258.jpg}
\end{figure}



\par 图20-107 慢日志统计查询界面


% From text00323.html


\section{20.7 RDS for MySQL的只读实例和克隆}


\subsection{
20.7.1 RDS for MySQL只读实例}
}


\par RDS只读实例和备份实例是一样的,都是通过Binlog生成的。

\par 只读实例采用MySQL的原生复制功能将源数据库实例(以下简称主实例)的更改同步到所有相关的只读节点。

\par 
\begin{figure}[htbp]
\centering
\includegraphics[width=0.8\textwidth]{Image00019.jpg}
\end{figure}

 注意:
 DTS无法订阅迁移只读实例。


% From text00324.html


\subsection{20.7.2 RDS for MySQL只读实例功能特点}


\par 一个主实例(限MySQL 5.6版本)最多可以创建5个只读实例,只读实例的规格大小可以与主实例不一致,可以更方便地进行弹性升降级。根据业务压力,在需要时可以随时对实例规格进行升级。只读实例计费模式目前仅支持按小时计费;

\par 只读实例不需要维护账号与数据库,全部通过主实例进行同步,白名单独立配置。


% From text00325.html


\subsection{20.7.3 RDS for MySQL只读实例创建过程}


\par 第1步:

\par 创建一个空的只读实例,该步骤大概在5~10分钟内完成。

\par 第2步:

\par 检查用户选择的RDS主实例最近24小时内是否进行了物理备份,如果未进行物理备份,则会对主实例进行临时备份。该步骤所花费的时间比较长。所以建议用户在购买只读实例之前,先对主实例进行一次物理备份,从而缩短只读实例的创建时间。

\par 第3步:

\par 将主实例的物理备份覆盖到只读实例中。该步骤的耗时将取决于主实例的数据大小。

\par 图20-108所示的是主实例备份文件不同大小在此步骤所花费的时间,仅供参考。

\begin{figure}[htbp]
\centering
\includegraphics[width=0.8\textwidth]{Image00259.jpg}
\end{figure}



\par 图20-108 备份文件大小与对应的时间共费

\par 第4步:

\par 同步24小时日志。


% From text00326.html


\subsection{20.7.4 RDS for MySQL创建只读实例}


\par 点击“添加只读实例”,如图20-109所示。

\begin{figure}[htbp]
\centering
\includegraphics[width=0.8\textwidth]{Image00260.jpg}
\end{figure}



\par 图20-109 添加只读实例界面

\par 如图20-110所示,选择只读实例的规格。因为只读实例提供的是长期的服务,所以只读实例是收费的,如图20-111所示。

\begin{figure}[htbp]
\centering
\includegraphics[width=0.8\textwidth]{Image00261.jpg}
\end{figure}



\par 图20-110 选择只读实例的规格

\begin{figure}[htbp]
\centering
\includegraphics[width=0.8\textwidth]{Image00262.jpg}
\end{figure}



\par 图20-111 只读实例收费确认界面

\par 创建只读实例成功界面如图20-112所示。

\begin{figure}[htbp]
\centering
\includegraphics[width=0.8\textwidth]{Image00263.jpg}
\end{figure}



\par 图20-112 创建只读实例成功界面

\par 回到RDS管理控制台,可以看到只读实例正在创建中,如图20-113所示。

\begin{figure}[htbp]
\centering
\includegraphics[width=0.8\textwidth]{Image00264.jpg}
\end{figure}



\par 图20-113 实例列表界面查看基本信息

\par 进入主实例可以看到,当前主实例中有一个只读实例,如图20-114所示。

\begin{figure}[htbp]
\centering
\includegraphics[width=0.8\textwidth]{Image00265.jpg}
\end{figure}



\par 图20-114 主实例中已有一个只读实例

\par 只读实例,同主实例是一样的,可以有内外网访问地址,只不过只读实例只能是只读,如图20-115所示。

\begin{figure}[htbp]
\centering
\includegraphics[width=0.8\textwidth]{Image00266.jpg}
\end{figure}



\par 图20-115 只读实例的内外网地址


% From text00327.html


\section{20.8 RDS for MySQL只读实例实现读写分离}


\subsubsection{读写分离案例}


\par 读写分离方案的优势具体如下。

\par 1)大量读请求指向只读实例。

\par 2)降低主数据库压力。

\par 3)对程序架构影响不大。

\par 读写分离方案结构如图20-116所示。

\begin{figure}[htbp]
\centering
\includegraphics[width=0.8\textwidth]{Image00267.jpg}
\end{figure}



\par 图20-116 读写分离方案结构图


% From text00328.html


\section{20.9 RDS for MySQL克隆实例}


\subsubsection{克隆实例的概念和创建过程}


\par 克隆实例可以按照指定的RDS实例批量复制出与原实例一模一样的新实例,复制的内容包括实例数据和实例中可设置的参数(如备份设置、参数设置的参数)。对于需要批量创建相同实例的用户,可以使用克隆实例功能,在一个现有实例上快速复制出多个实例。

\par 克隆实例当前仅支持MySQL类型的数据库,克隆实例的计费标准与主实例相同。

\par 克隆实例可以选择主实例的备份集,或者在备份有效存储时间内的时间点上复制出一个新的实例。克隆实例只支持主实例的克隆。若主实例下挂载有只读实例和灾备实例,则克隆时只克隆该主实例,不克隆其下的只读实例和灾备实例。

\par 上文20.4.6“RDS for MySQL备份与恢复”一节中我们已经实战练习过克隆实例如图恢复数据,实验步骤这里不再重复。


% From text00329.html


\section{20.10 RDS for MySQL克隆实例使用场景}


\subsection{
20.10.1 克隆实例用于数据回溯}
}


\par 什么情况下需要进行数据回溯?

\par 1)误删除数据。

\par 2)系统BUG“污染”了主数据库。

\par 数据回溯功能可以帮助您恢复到7天以内的任意一个时间点,精确到秒。如图20-117所示是从克隆实例自主实例迁移数据的过程。

\begin{figure}[htbp]
\centering
\includegraphics[width=0.8\textwidth]{Image00268.jpg}
\end{figure}



\par 图20-117 从克隆实例向主实例迁移数据


% From text00330.html


\subsection{20.10.2 克隆实例用于准生产测试}


\par 实际场景中可能会遇到如下的困惑。

\par ·系统升级前想要进行全面测试,但缺乏真实的数据。

\par ·拥有真实数据的备份,但恢复一次时间太长。

\par 克隆实例可以灵活、方便、随时、低成本满足业务需要。


% From text00331.html


\section{20.11 RDS for MySQL重点回顾}


\par ·RDS的特点、场景、相关概念。

\par ·RDS的购买、计费、升级。

\par ·RDS的数据库实例管理备份、恢复、迁移、监控、日志。

\par ·只读实例,读写分离。

\par ·克隆实例,数据回溯。

\end{document}