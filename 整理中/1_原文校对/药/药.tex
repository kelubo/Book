% 药
% 药.tex

\documentclass[12pt,UTF8]{ctexbook}

% 设置纸张信息。
\usepackage[a4paper,twoside]{geometry}
\geometry{
	left=25mm,
	right=25mm,
	bottom=25.4mm,
	bindingoffset=10mm
}

% 设置字体,并解决显示难检字问题。
\xeCJKsetup{AutoFallBack=true}
\setCJKmainfont{SimSun}[BoldFont=SimHei, ItalicFont=KaiTi, FallBack=SimSun-ExtB]

% 目录 chapter 级别加点(.)。
\usepackage{titletoc}
\titlecontents{chapter}[0pt]{\vspace{3mm}\bf\addvspace{2pt}\filright}{\contentspush{\thecontentslabel\hspace{0.8em}}}{}{\titlerule*[8pt]{.}\contentspage}

% 设置 part 和 chapter 标题格式。
\ctexset{
	chapter/name={},
	chapter/number={}
}

% 注脚每页重新编号,避免编号过大。
\usepackage[perpage]{footmisc}

\title{\heiti\zihao{0} 药}
\author{鲁迅}
\date{}

\begin{document}

\maketitle
\tableofcontents

\chapter{一}

秋天的后半夜,月亮下去了,太阳还没有出,只剩下一片乌蓝的天;除了夜游的东西,什么都睡着。华老栓忽然坐起身,擦着火柴,点上遍身油腻的灯盏,茶馆的两间屋子里,便弥满了青白的光。

“小栓的爹,你就去么?”是一个老女人的声音。里边的小屋子里,也发出一阵咳嗽。

“唔。”老栓一面听,一面应,一面扣上衣服;伸手过去说,“你给我罢”。

华大妈在枕头底下掏了半天,掏出一包洋钱\footnote{text},交给老栓,老栓接了,抖抖的装入衣袋,又在外面按了两下;便点上灯笼,吹熄灯盏,走向里屋子去了。那屋子里面,正在悉悉窣窣\footnote{象声词,形容轻微的摩擦声。这里形容穿衣服的声音。窸,xī。窣,sū。}的响,接着便是一通咳嗽。老栓候他平静下去,才低低的叫道,“小栓……你不要起来。……店么?你娘会安排的。”

老栓听得儿子不再说话,料他安心睡了;便出了门,走到街上。街上黑沉沉的一无所有,只有一条灰白的路,看得分明。灯光照着他的两脚,一前一后的走。有时也遇到几只狗,可是一只也没有叫。天气比屋子里冷得多了;老栓倒觉爽快,仿佛一旦变了少年,得了神通,有给人生命的本领似的,跨步格外高远。而且路也愈走愈分明,天也愈走愈亮了。

老栓正在专心走路,忽然吃了一惊,远远里看见一条丁字街,明明白白横着。他便退了几步,寻到一家关着门的铺子,蹩进\footnote{躲躲闪闪地走进。蹩,bié。}檐下,靠门立住了。好一会,身上觉得有些发冷。

“哼,老头子。”

“倒高兴……。”

老栓又吃一惊,睁眼看时,几个人从他面前过去了。一个还回头看他,样子不甚分明,但很像久饿的人见了食物一般,眼里闪出一种攫取的光。老栓看看灯笼,已经熄了。按一按衣袋,硬硬的还在。仰起头两面一望,只见许多古怪的人,三三两两,鬼似的在那里徘徊;定睛再看,却也看不出什么别的奇怪。

没有多久,又见几个兵,在那边走动;衣服前后的一个大白圆圈\footnote{清朝士兵穿的号衣(制服),前后都缀着一块圆形的白布,上面有个“兵”字或“勇”字。},远地里也看得清楚,走过面前的,并且看出号衣\footnote{text}上暗红色的镶边。——一阵脚步声响,一眨眼,已经拥过了一大簇人。那三三两两的人,也忽然合作一堆,潮一般向前进;将到丁字街口,便突然立住,簇成一个半圆。

老栓也向那边看,却只见一堆人的后背;颈项都伸得很长,仿佛许多鸭,被无形的手捏住了的,向上提着。静了一会,似乎有点声音,便又动摇起来,轰的一声,都向后退;一直散到老栓立着的地方,几乎将他挤倒了。

“喂!一手交钱,一手交货!”一个浑身黑色的人,站在老栓面前,眼光正像两把刀,刺得老栓缩小了一半。那人一只大手,向他摊着;一只手却撮着一个鲜红的馒头\footnote{指蘸有人血的馒头。旧时民间迷信,认为人血可以医治肺结核病,处决犯人时,有人向刽子手买蘸过人血的馒头治病。},那红的还是一点一点的往下滴。

老栓慌忙摸出洋钱,抖抖的想交给他,却又不敢去接他的东西。那人便焦急起来,嚷道,“怕什么?怎的不拿!”老栓还踌躇着;黑的人便抢过灯笼,一把扯下纸罩,裹了馒头,塞与老栓;一手抓过洋钱,捏一捏,转身去了。嘴里哼着说,“这老东西……”

“这给谁治病的呀?”老栓也似乎听得有人问他,但他并不答应;他的精神,现在只在一个包上,仿佛抱着一个十世单传的婴儿,别的事情,都已置之度外了。他现在要将这包里的新的生命,移植到他家里,收获许多幸福。太阳也出来了;在他面前,显出一条大道,直到他家中,后面也照见丁字街头破匾上“古囗亭口”\footnote{可念作“古某亭口”。囗,是文章里表示缺文的记号,作者是有意这样写的。浙江省绍兴县城内的轩亭口有一牌楼,匾上题有“古轩亭口”四个字。清末资产阶级民主主义革命家秋瑾于1907年在这里就义。这篇小说里夏瑜这个人物,一般认为是作者以秋瑾和其他一些资产阶级民主主义革命家的若干经历为素材而创造出来的。}这四个黯淡的金字。

\chapter{二}

老栓走到家,店面早经收拾干净,一排一排的茶桌,滑溜溜的发光。但是没有客人;只有小栓坐在里排的桌前吃饭,大粒的汗,从额上滚下,夹袄也帖住了脊心,两块肩胛骨高高凸出,印成一个阳文⑹的“八”字。老栓见这样子,不免皱一皱展开的眉心。他的女人,从灶下急急走出,睁着眼睛,嘴唇有些发抖。
“得了么?”
“得了。”
两个人一齐走进灶下,商量了一会;华大妈便出去了,不多时,拿着一片老荷叶回来,摊在桌上。老栓也打开灯笼罩,用荷叶重新包了那红的馒头。小栓也吃完饭,他的母亲慌忙说:“小栓——你坐着,不要到这里来。”一面整顿了灶火,老栓便把一个碧绿的包,一个红红白白的破灯笼,一同塞在灶里;一阵红黑的火焰过去时,店屋里散满了一种奇怪的香味。
“好香!你们吃什么点心呀?”这是驼背五少爷到了。这人每天总在茶馆里过日,来得最早,去得最迟,此时恰恰蹩到临街的壁角的桌边,便坐下问话,然而没有人答应他。“炒米粥⑺么?”仍然没有人应。老栓匆匆走出,给他泡上茶。
“小栓进来罢!”华大妈叫小栓进了里面的屋子,中间放好一条凳,小栓坐了。他的母亲端过一碟乌黑的圆东西,轻轻说:
“吃下去罢,——病便好了。”
小栓撮起这黑东西,看了一会,似乎拿着自己的性命一般,心里说不出的奇怪。十分小心的拗开⑻了,焦皮里面窜出一道白气,白气散了,是两半个白面的馒头。——不多工夫,已经全在肚里了,却全忘了什么味;面前只剩下一张空盘。他的旁边,一面立着他的父亲,一面立着他的母亲,两人的眼光,都仿佛要在他身上注进什么又要取出什么似的;便禁不住心跳起来,按着胸膛,又是一阵咳嗽。
“睡一会罢,——便好了。”
小栓依他母亲的话,咳着睡了。华大妈候他喘气平静,才轻轻的给他盖上了满幅补钉的夹被。
\chapter{三}
店里坐着许多人,老栓也忙了,提着大铜壶,一趟一趟的给客人冲茶;两个眼眶,都围着一圈黑线。
“老栓,你有些不舒服么?——你生病么?”一个花白胡子的人说。
“没有。”
“没有?——我想笑嘻嘻的,原也不像……”花白胡子便取消了自己的话。
“老栓只是忙。要是他的儿子……”驼背五少爷话还未完,突然闯进了一个满脸横肉的人,披一件玄色⑼布衫,散着纽扣,用很宽的玄色腰带,胡乱捆在腰间。刚进门,便对老栓嚷道:
“吃了么?好了么?老栓,就是运气了你!你运气,要不是我信息灵……”
老栓一手提了茶壶,一手恭恭敬敬的垂着;笑嘻嘻的听。满座的人,也都恭恭敬敬的听。华大妈也黑着眼眶,笑嘻嘻的送出茶碗茶叶来,加上一个橄榄,老栓便去冲了水。
“这是包好!这是与众不同的。你想,趁热的拿来,趁热的吃下。”横肉的人只是嚷。
“真的呢,要没有康大叔照顾,怎么会这样……”华大妈也很感激的谢他。
“包好,包好!这样的趁热吃下。这样的人血馒头,什么痨病都包好!”
华大妈听到“痨病”这两个字,变了一点脸色,似乎有些不高兴;但又立刻堆上笑,搭赸⑽着走开了。这康大叔却没有觉察,仍然提高了喉咙只是嚷,嚷得里面睡着的小栓也合伙咳嗽起来。
“原来你家小栓碰到了这样的好运气了。这病自然一定全好;怪不得老栓整天的笑着呢。”花白胡子一面说,一面走到康大叔面前,低声下气的问道,“康大叔——听说今天结果的一个犯人,便是夏家的孩子,那是谁的孩子?究竟是什么事?”
“谁的?不就是夏四奶奶的儿子么?那个小家伙!”康大叔见众人都耸起耳朵听他,便格外高兴,横肉块块饱绽,越发大声说,“这小东西不要命,不要就是了。我可是这一回一点没有得到好处;连剥下来的衣服,都给管牢的红眼睛阿义拿去了。——第一要算我们栓叔运气;第二是夏三爷赏了二十五两雪白的银子,独自落腰包,一文不花。”
小栓慢慢的从小屋子里走出,两手按了胸口,不住的咳嗽;走到灶下,盛出一碗冷饭,泡上热水,坐下便吃。华大妈跟着他走,轻轻的问道,“小栓,你好些么?——你仍旧只是肚饿?……”
“包好,包好!”康大叔瞥了小栓一眼,仍然回过脸,对众人说,“夏三爷真是乖角儿⑾,要是他不先告官,连他满门抄斩⑿。现在怎样?银子!——这小东西也真不成东西!关在牢里,还要劝牢头造反。”
“阿呀,那还了得。”坐在后排的一个二十多岁的人,很现出气愤模样。
“你要晓得红眼睛阿义是去盘盘底细的,他却和他攀谈了。他说:这大清的天下是我们大家的。你想:这是人话么?红眼睛原知道他家里只有一个老娘,可是没有料到他竟会这么穷,榨不出一点油水,已经气破肚皮了。他还要老虎头上搔痒,便给他两个嘴巴!”
“义哥是一手好拳棒,这两下,一定够他受用了。”壁角的驼背忽然高兴起来。
“他这贱骨头打不怕,还要说可怜可怜哩。”
花白胡子的人说,“打了这种东西,有什么可怜呢?”
康大叔显出看他不上的样子,冷笑着说,“你没有听清我的话;看他神气,是说阿义可怜哩!”
听着的人的眼光,忽然有些板滞⒀;话也停顿了。小栓已经吃完饭,吃得满头流汗,头上都冒出蒸气来。
“阿义可怜——疯话,简直是发了疯了。”花白胡子恍然大悟似的说。
“发了疯了。”二十多岁的人也恍然大悟的说。
店里的坐客,便又现出活气,谈笑起来。小栓也趁着热闹,拚命咳嗽;康大叔走上前,拍他肩膀说:
“包好!小栓——你不要这么咳。包好!”
“疯了!”驼背五少爷点着头说。

\chapter{四}

西关外靠着城根的地面,本是一块官地;中间歪歪斜斜一条细路,是贪走便道的人,用鞋底造成的,但却成了自然的界限。路的左边,都埋着死刑和瘐毙⒁的人,右边是穷人的丛冢⒂。两面都已埋到层层叠叠,宛然阔人家里祝寿时的馒头。
这一年的清明,分外寒冷;杨柳才吐出半粒米大的新芽。天明未久,华大妈已在右边的一坐新坟前面,排出四碟菜,一碗饭,哭了一场。化过纸⒃,呆呆的坐在地上;仿佛等候什么似的,但自己也说不出等候什么。微风起来,吹动他⒄短发,确乎比去年白得多了。
小路上又来了一个女人,也是半白头发,褴褛的衣裙;提一个破旧的朱漆圆篮,外挂一串纸锭⒅,三步一歇的走。忽然见华大妈坐在地上看她,便有些踌躇,惨白的脸上,现出些羞愧的颜色;但终于硬着头皮,走到左边的一坐坟前,放下了篮子。
那坟与小栓的坟,一字儿排着,中间只隔一条小路。华大妈看他排好四碟菜,一碗饭,立着哭了一通,化过纸锭;心里暗暗地想,“这坟里的也是儿子了。”那老女人徘徊观望了一回,忽然手脚有些发抖,跄跄踉踉退下几步,瞪着眼只是发怔。
华大妈见这样子,生怕她伤心到快要发狂了;便忍不住立起身,跨过小路,低声对他说,“你这位老奶奶不要伤心了,——我们还是回去罢。”
那人点一点头,眼睛仍然向上瞪着;也低声痴痴的说道,“你看,——看这是什么呢?”
华大妈跟了他指头看去,眼光便到了前面的坟,这坟上草根还没有全合,露出一块一块的黄土,煞是难看。再往上仔细看时,却不觉也吃一惊;——分明有一圈红白的花,围着那尖圆的坟顶。
他们的眼睛都已老花多年了,但望这红白的花,却还能明白看见。花也不很多,圆圆的排成一个圈,不很精神,倒也整齐。华大妈忙看他儿子和别人的坟,却只有不怕冷的几点青白小花,零星开着;便觉得心里忽然感到一种不足和空虚,不愿意根究。那老女人又走近几步,细看了一遍,自言自语的说,“这没有根,不像自己开的。——这地方有谁来呢?孩子不会来玩;——亲戚本家早不来了。——这是怎么一回事呢?”他想了又想,忽又流下泪来,大声说道:
“瑜儿,他们都冤枉了你,你还是忘不了,伤心不过,今天特意显点灵,要我知道么?”他四面一看,只见一只乌鸦,站在一株没有叶的树上,便接着说,“我知道了。——瑜儿,可怜他们坑了你,他们将来总有报应,天都知道;你闭了眼睛就是了。——你如果真在这里,听到我的话,——便教这乌鸦飞上你的坟顶,给我看罢。”
微风早经停息了;枯草支支直立,有如铜丝。一丝发抖的声音,在空气中愈颤愈细,细到没有,周围便都是死一般静。两人站在枯草丛里,仰面看那乌鸦;那乌鸦也在笔直的树枝间,缩着头,铁铸一般站着。
许多的工夫过去了;上坟的人渐渐增多,几个老的小的,在土坟间出没。
华大妈不知怎的,似乎卸下了一挑重担,便想到要走;一面劝着说,“我们还是回去罢”。
那老女人叹一口气,无精打采的收起饭菜;又迟疑了一刻,终于慢慢地走了。嘴里自言自语的说,“这是怎么一回事呢?……”
他们走不上二三十步远,忽听得背后“哑——”的一声大叫;两个人都竦然⒆的回过头,只见那乌鸦张开两翅,一挫身⒇,直向着远处的天空,箭也似的飞去了。

一九一九年四月二十五日 [1]


⑹阳文:刻在器物上的文字,笔画凸起的叫阳文,笔画凹下的叫阴文。
⑺炒米粥:用炒过的大米煮成的粥。
⑻拗(ǎo)开:用手掰开。拗,用手折断。
⑼玄色:黑色。
⑽搭赸(shàn):一般写作“搭讪”。为了跟人接近或把尴尬的局面敷衍过去而找话说。这里是后一种意思。
⑾乖角儿:机灵人。这里指善于看风使舵的人。
⑿满门抄斩:抄没财产,杀戮全家。
⒀板滞:呆板,停止不动。
⒁瘐(yǔ)毙:旧时关在牢狱里的人因受刑或饥寒、疾病而死亡。
⒂丛冢(zhǒng):乱坟堆。冢,坟墓。
⒃化过纸:烧过纸钱。旧时有迷信观念的人认为烧过的纸钱,死者可以在阴间使用。
⒄他:指华大妈。这篇小说里的第三人称代词,不分男女,一律写作“他”。
⒅纸锭(dìng):用纸或锡箔折成的“元宝”,纸钱的一种。
⒆竦(sǒng)然:惊惧的样子。竦,通“悚”。
⒇一挫身:身子一收缩。 [2]

\end{document}