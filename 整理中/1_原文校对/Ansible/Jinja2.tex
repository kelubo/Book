% Jinja2 模板引擎详解
% 使用xelatex编译

\documentclass[12pt,a4paper,twoside]{ctexbook}

% 页面设置
\usepackage{geometry}
\geometry{left=3cm, right=2.5cm, top=3cm, bottom=2.5cm}

% 字体设置
\usepackage{xeCJK}
\usepackage{fontspec}
\usepackage{microtype}

% 设置中文字体
\setCJKmainfont{SimSun}[  % 正文宋体
    BoldFont=SimHei,        % 粗体黑体
    ItalicFont=KaiTi        % 斜体楷体
]
\setCJKsansfont{SimHei}    % 无衬线字体黑体
\setCJKmonofont{SimSun}    % 等宽字体宋体
\setCJKfamilyfont{kai}[    % 楷体
    BoldFont=KaiTi
]{KaiTi}
\setCJKfamilyfont{fs}[     % 仿宋
    BoldFont=FangSong
]{FangSong}

% 常用字体命令
\newcommand{\song}{\CJKfamily{zhsong}}
\newcommand{\hei}{\CJKfamily{zhhei}}
\newcommand{\kai}{\CJKfamily{kai}}
\newcommand{\fs}{\CJKfamily{fs}}

% 标题格式设置
\usepackage{titlesec}
\ctexset{
    part/name={第,卷},
    part/number={\chinese{part}},
    chapter/name={第,章},
    chapter/number={\chinese{chapter}},
    section/name={第,节},
    section/number={\arabic{section}},
    subsection/number={\arabic{section}.\arabic{subsection}},
    chapter/format={\centering\hei\zihao{2}},
    section/format={\hei\zihao{4}},
    subsection/format={\hei\zihao{5}}
}

% 页眉页脚设置
\usepackage{fancyhdr}
\pagestyle{fancy}
\fancyhf{}
\fancyhead[LE,RO]{\zihao{5}\thepage}
\fancyhead[LO]{\zihao{5}\leftmark}
\fancyhead[RE]{\zihao{5}\rightmark}
\renewcommand{\chaptermark}[1]{\markboth{\chaptername\ \thechapter\ #1}{}}
\renewcommand{\sectionmark}[1]{\markright{\thesection\ #1}}
\fancyfoot[C]{\zihao{5} \thepage}
\renewcommand{\headrulewidth}{0.4pt}
\renewcommand{\footrulewidth}{0pt}

% 插图设置
\usepackage{graphicx}
\usepackage{float}
\usepackage{subfigure}
\graphicspath{{images/}}
\floatstyle{plaintop}
\restylefloat{figure}

% 表格设置
\usepackage{tabularx}
\usepackage{booktabs}
\usepackage{longtable}

% 数学公式设置
\usepackage{amsmath, amssymb, amsthm}
\usepackage{mathrsfs}

% 定理环境
\newtheorem{theorem}{定理}[chapter]
\newtheorem{definition}{定义}[chapter]
\newtheorem{lemma}{引理}[chapter]
\newtheorem{corollary}{推论}[chapter]
\newtheorem{example}{例}[chapter]

% 目录、摘要等设置
\usepackage{makeidx}
\makeindex

% 代码设置
\usepackage{listings}
\usepackage{color}
\usepackage{enumitem}

\definecolor{codegreen}{rgb}{0,0.6,0}
\definecolor{codegray}{rgb}{0.5,0.5,0.5}
\definecolor{codepurple}{rgb}{0.58,0,0.82}
\definecolor{backcolour}{rgb}{0.95,0.95,0.92}

\lstdefinestyle{mystyle}{
    backgroundcolor=\color{backcolour},
    commentstyle=\color{codegreen},
    keywordstyle=\color{magenta},
    numberstyle=\tiny\color{codegray},
    stringstyle=\color{codepurple},
    basicstyle=\footnotesize,
    breakatwhitespace=false,
    breaklines=true,
    captionpos=b,
    keepspaces=true,
    numbers=left,
    numbersep=5pt,
    showspaces=false,
    showstringspaces=false,
    showtabs=false,
    tabsize=2
}

\lstset{style=mystyle}

% 引用设置
\usepackage{hyperref}
\hypersetup{
    colorlinks=true,
    linkcolor=blue,
    citecolor=blue,
    urlcolor=blue,
    pdftitle={Jinja2 模板引擎详解},
    pdfauthor={Ansible 学习指南},
    pdfsubject={Jinja2 模板引擎},
    pdfkeywords={Jinja2 \quad Ansible \quad 模板引擎}
}

% 目录深度
\setcounter{tocdepth}{3}
\setcounter{secnumdepth}{3}

% 封面信息
\title{\hei\zihao{0} Jinja2 模板引擎详解}
\author{\song\zihao{2} Ansible 学习指南}
\date{\song\zihao{4} \today}

\begin{document}

% 封面
\begin{titlepage}
    \begin{center}
        \vspace*{6cm}
        \hei\zihao{0} Jinja2 模板引擎详解
        \vspace*{3cm}
        \song\zihao{2} Ansible 学习指南
        \vspace*{3cm}
        \song\zihao{4} \today
    \end{center}
\end{titlepage}

% 版权页
\newpage
\thispagestyle{empty}
\begin{center}
    \vspace*{8cm}
    \song\zihao{5} 版权所有 \textcopyright\ 2026 Ansible 学习指南
    \vspace*{1cm}
    \song\zihao{5} 出版社名称
\end{center}

% 摘要
\chapter*{摘要}
    Jinja2 是一个功能强大的 Python 模板引擎,由 Armin Ronacher 创建。它是 Django 模板系统的一个超集,但提供了更灵活、更强大的功能。在 Ansible 中,Jinja2 被广泛用于模板文件的渲染、Playbook 中的变量替换、条件语句和循环的实现,以及复杂数据结构的处理。
    
    本文详细介绍了 Jinja2 的基本语法、控制结构、过滤器、模板继承等特性,以及它在 Ansible 中的具体应用。通过学习本文,读者将能够掌握 Jinja2 的核心概念和使用方法,从而在 Ansible 自动化中更加灵活地使用模板技术。
    
    \begin{flushright}
        \textbf{关键词:}Jinja2 \quad Ansible \quad 模板引擎 \quad 自动化
    \end{flushright}

% 目录
\newpage
\tableofcontents

% 正文开始
\mainmatter

\chapter{什么是 Jinja2}

Jinja2 是一个功能强大的 Python 模板引擎,由 Armin Ronacher 创建。它是 Django 模板系统的一个超集,但提供了更灵活、更强大的功能。在 Ansible 中,Jinja2 被广泛用于:

\begin{itemize}
    \item 模板文件(.j2 文件)的渲染
    \item Playbook 中的变量替换
    \item 条件语句和循环的实现
    \item 复杂数据结构的处理
\end{itemize}

Jinja2 的设计目标是提供一个快速、安全、灵活的模板系统,同时保持代码的清晰和可维护性。

\chapter{Jinja2 的基本语法}

\subsection{变量}

变量是 Jinja2 中最基本的元素,用于在模板中插入动态内容。

\begin{lstlisting}
# 基本变量使用
Hello, {{ name }}!

# 访问对象属性
{{ user.name }}
{{ user['email'] }}

# 访问列表元素
{{ items[0] }}
{{ items[1] }}
\end{lstlisting}

\subsection{表达式}

Jinja2 支持各种表达式,包括算术运算、比较运算、逻辑运算等。

\begin{lstlisting}
# 算术运算
{{ 1 + 2 }}
{{ 10 - 5 }}
{{ 2 * 3 }}
{{ 10 / 2 }}
{{ 10 // 3 }}  # 整数除法
{{ 10 % 3 }}   # 取模
{{ 2 ** 3 }}   # 幂运算

# 比较运算
{{ 1 == 2 }}
{{ 1 != 2 }}
{{ 1 < 2 }}
{{ 1 <= 2 }}
{{ 1 > 2 }}
{{ 1 >= 2 }}

# 逻辑运算
{{ true and false }}
{{ true or false }}
{{ not true }}

# 成员运算
{{ 'a' in 'abc' }}
{{ 1 in [1, 2, 3] }}

# 测试运算
{{ variable is defined }}
{{ variable is none }}
{{ variable is string }}
{{ variable is number }}
{{ variable is iterable }}
\end{lstlisting}

\subsection{控制结构}

\subsubsection{条件语句}

条件语句用于根据条件执行不同的代码块。

\begin{lstlisting}
# 基本 if 语句

    <p>Welcome, admin!</p>


# if-else 语句

    <p>Welcome back, {{ user.name }}!</p>

    <p>Please log in.</p>


# if-elif-else 语句

    <p>Excellent!</p>

    <p>Good!</p>

    <p>Pass!</p>

    <p>Fail!</p>

\end{lstlisting}

\subsubsection{循环语句}

循环语句用于遍历可迭代对象,如列表、字典等。

\begin{lstlisting}
# 遍历列表

    <p>{{ item }}</p>


# 遍历字典

    <p>{{ key }}: {{ value }}</p>


# 带索引的循环

    <p>{{ loop.index }}. {{ item }}</p>  {# 从 1 开始 #}
    <p>{{ loop.index0 }}. {{ item }}</p> {# 从 0 开始 #}


# 循环控制

    
        
    
    
        
    
    <p>{{ item }}</p>


# 空循环处理

    <p>{{ item }}</p>

    <p>No items found.</p>

\end{lstlisting}

\subsubsection{宏}

宏类似于函数,用于定义可重用的代码块。

\begin{lstlisting}
# 定义宏

    <div class="user">
        <h3>{{ user.name }}</h3>
        <p>{{ user.email }}</p>
    </div>


# 使用宏
{{ render_user(user1) }}
{{ render_user(user2) }}

# 带默认参数的宏

    {{ greeting }}, {{ name }}!


{{ greet('John') }}
{{ greet('John', 'Hi') }}
\end{lstlisting}

\chapter{Jinja2 过滤器}

过滤器用于修改变量的值,格式为 `{{ variable|filter }}`。Jinja2 提供了大量内置过滤器,也支持自定义过滤器。

\subsection{常用内置过滤器}

\begin{lstlisting}
# 字符串过滤器
{{ 'hello'|upper }}
{{ 'HELLO'|lower }}
{{ 'hello world'|capitalize }}
{{ 'hello world'|title }}
{{ '  hello  '|trim }}
{{ 'hello'|length }}
{{ 'hello'|replace('h', 'H') }}
{{ 'hello'|reverse }}

# 数字过滤器
{{ 42|abs }}
{{ 42|round }}
{{ 42.5|int }}
{{ 42|float }}

# 列表过滤器
{{ [1, 2, 3]|length }}
{{ [3, 1, 2]|sort }}
{{ [1, 2, 3]|reverse }}
{{ [1, 2, 2, 3]|unique }}
{{ [1, 2, 3]|join(', ') }}

# 字典过滤器
{{ {'a': 1, 'b': 2}|length }}
{{ {'a': 1, 'b': 2}|keys }}
{{ {'a': 1, 'b': 2}|values }}

# 其他过滤器
{{ variable|default('default value') }}
{{ variable|bool }}
{{ variable|string }}
{{ variable|safe }}  # 标记为安全内容,不转义
{{ variable|tojson }}  # 转换为 JSON
{{ variable|datetimeformat('%Y-%m-%d') }}  # 日期格式化
\end{lstlisting}

\subsection{链式过滤器}

多个过滤器可以链式使用,前一个过滤器的输出作为后一个过滤器的输入。

\begin{lstlisting}
{{ '  hello world  '|trim|upper|replace('WORLD', 'ANSIBLE') }}
{{ [3, 1, 2]|sort|join(', ') }}
\end{lstlisting}

\subsection{自定义过滤器}

在 Ansible 中,可以通过插件系统创建自定义过滤器。

\chapter{Jinja2 模板继承}

模板继承是 Jinja2 的一个强大特性,允许创建基础模板并在子模板中扩展它。

\subsection{基础模板}

创建一个包含通用结构的基础模板。

\begin{lstlisting}
{# base.html.j2 #}
<!DOCTYPE html>
<html>
<head>
    <title>Default Title</title>
    
    
</head>
<body>
    <header>
        <h1>My Website</h1>
    </header>
    <main>
        
        
    </main>
    <footer>
        <p>&copy; 2023 My Website</p>
    </footer>
</body>
</html>
\end{lstlisting}

\subsection{子模板}

子模板继承基础模板并覆盖特定的块。

\begin{lstlisting}
{# home.html.j2 #}



    Home Page



    {{ super() }}  {# 调用父模板的内容 #}
    <link rel="stylesheet" href="home.css">



    <h2>Welcome to Home Page</h2>
    <p>This is the home page content.</p>

\end{lstlisting}

\chapter{Jinja2 在 Ansible 中的应用}

\subsection{模板文件}

使用 `template` 模块渲染 Jinja2 模板文件。

\begin{lstlisting}
- name: Copy nginx configuration
  template:
    src: nginx.conf.j2
    dest: /etc/nginx/nginx.conf
    owner: root
    group: root
    mode: '0644'
  notify: restart nginx
\end{lstlisting}

模板文件示例:

\begin{lstlisting}
{# nginx.conf.j2 #}
server {
    listen {{ nginx_port }};
    server_name {{ server_name }};
    
    root {{ web_root }};
    index index.html index.htm;
    
    
    listen 443 ssl;
    ssl_certificate {{ ssl_cert }};
    ssl_certificate_key {{ ssl_key }};
    
    
    location / {
        try_files $uri $uri/ =404;
    }
    
    
    gzip on;
    gzip_types text/plain text/css application/json application/javascript text/xml application/xml application/xml+rss text/javascript;
    
}
\end{lstlisting}

\subsection{Playbook 中的变量替换}

在 Playbook 中直接使用 Jinja2 语法进行变量替换。

\begin{lstlisting}
- name: Create user
  user:
    name: {{ username }}
    state: present
    groups: {{ user_groups|join(',') }}
    home: /home/{{ username }}

- name: Copy configuration file
  copy:
    content: |
      # Configuration for {{ service_name }}
      port: {{ service_port }}
      debug: {{ debug_mode|lower }}
      timeout: {{ timeout_seconds }}
    dest: /etc/{{ service_name }}/config.conf
\end{lstlisting}

\subsection{条件和循环}

在 Playbook 中使用 Jinja2 的条件和循环结构。

\begin{lstlisting}
- name: Install packages
  package:
    name: {{ item }}
    state: present
  loop: {{ packages }}
  when: ansible_os_family == 'RedHat'

- name: Configure services
  template:
    src: service.conf.j2
    dest: /etc/{{ item }}/config.conf
  loop: {{ services }}
  when: item in enabled_services
\end{lstlisting}

\chapter{Jinja2 的高级特性}

\subsection{上下文处理器}

上下文处理器用于在模板渲染前向上下文中添加变量。在 Ansible 中,这通常通过 `vars` 或 `set_fact` 模块实现。

\subsection{环境配置}

Jinja2 提供了多种环境配置选项,如自动转义、变量分隔符等。在 Ansible 中,这些配置通常在 `ansible.cfg` 中设置。

\subsection{沙箱模式}

Jinja2 支持沙箱模式,可以限制模板中可执行的操作,提高安全性。

\chapter{Jinja2 最佳实践}

\subsection{代码组织}

\begin{enumerate}
    \item **模板目录结构**:按照功能或服务组织模板文件
    \item **命名规范**:使用描述性的文件名,如 `nginx.conf.j2`
    \item **模块化**:将复杂模板拆分为多个小模板
    \item **注释**:为复杂的模板逻辑添加注释
\end{enumerate}

\subsection{性能优化}

\begin{enumerate}
    \item **缓存**:启用模板缓存以提高性能
    \item **避免复杂逻辑**:将复杂逻辑移至 Ansible tasks 中
    \item **变量预处理**:在渲染前预处理变量
    \item **避免重复计算**:使用 `set` 语句存储计算结果
\end{enumerate}

\subsection{安全性}

\begin{enumerate}
    \item **自动转义**:启用 HTML 自动转义以防止 XSS 攻击
    \item **安全过滤器**:使用 `safe` 过滤器时要谨慎
    \item **输入验证**:验证模板输入的安全性
    \item **限制权限**:限制模板文件的权限
\end{enumerate}

\subsection{调试技巧}

\begin{enumerate}
    \item **使用 debug 模块**:在 Playbook 中打印变量值
    \item **模板调试**:使用 `ansible-playbook --check` 检查模板渲染
    \item **错误处理**:使用 `default` 过滤器处理可能不存在的变量
    \item **日志记录**:在复杂模板中添加日志记录
\end{enumerate}

\chapter{Jinja2 示例}

\subsection{基本示例}

\begin{lstlisting}
{# 基本变量和表达式 #}
Hello, {{ name }}!
Your age is {{ age }}.
Next year you will be {{ age + 1 }}.

{# 条件语句 #}

    You are an adult.

    You are a minor.


{# 循环语句 #}

    {{ loop.index }}. {{ item }}


{# 过滤器 #}
{{ 'hello'|upper }}
{{ [3, 1, 2]|sort|join(', ') }}
{{ user|default('Guest') }}
\end{lstlisting}

\subsection{Ansible 模板示例}

\begin{lstlisting}
{# Apache 配置模板 #}
<VirtualHost *:{{ http_port }}>
    ServerName {{ server_name }}
    ServerAlias {{ server_aliases|join(' ') }}
    DocumentRoot {{ document_root }}
    
    <Directory {{ document_root }}>
        Options Indexes FollowSymLinks
        AllowOverride All
        Require all granted
    </Directory>
    
    ErrorLog {{ error_log }}
    CustomLog {{ access_log }} combined
    
    
    SSLEngine on
    SSLCertificateFile {{ ssl_cert }}
    SSLCertificateKeyFile {{ ssl_key }}
    
    
    
    <Location {{ location.path }}>
        AuthType Basic
        AuthName "{{ location.name }}"
        AuthUserFile {{ location.auth_file }}
        Require valid-user
    </Location>
    
</VirtualHost>
\end{lstlisting}

\subsection{复杂数据结构处理}

\begin{lstlisting}
{# 处理嵌套字典 #}

Service: {{ service }}
  Port: {{ config.port }}
  Enabled: {{ config.enabled }}
  
  Options:
    
    {{ key }}: {{ value }}
    
  


{# 处理列表的列表 #}

Group: {{ group.name }}
Members:
  
  - {{ member }}
  

\end{lstlisting}

\chapter{常见问题与解决方案}

\subsection{变量未定义}

\textbf{问题}:模板渲染时变量未定义导致错误

\textbf{解决方案}:
\begin{itemize}
    \item 使用 `default` 过滤器提供默认值
    \item 使用 `defined` 测试检查变量是否存在
    \item 在 Playbook 中确保变量被正确定义
\end{itemize}

\begin{lstlisting}
{{ variable|default('default value') }}

    {{ variable }}

    Default value

\end{lstlisting}

\subsection{模板语法错误}

\textbf{问题}:模板中存在语法错误导致渲染失败

\textbf{解决方案}:
\begin{itemize}
    \item 检查 Jinja2 语法是否正确
    \item 确保所有的块标签都有对应的结束标签
    \item 使用 `ansible-playbook --check` 进行测试
    \item 检查变量名和过滤器名的拼写
\end{itemize}

\subsection{性能问题}

\textbf{问题}:大型模板渲染速度慢

\textbf{解决方案}:
\begin{itemize}
    \item 拆分大型模板为多个小模板
    \item 减少模板中的复杂逻辑
    \item 预处理变量以减少模板中的计算
    \item 启用模板缓存
\end{itemize}

\subsection{安全问题}

\textbf{问题}:模板中可能存在安全漏洞

\textbf{解决方案}:
\begin{itemize}
    \item 启用自动转义
    \item 谨慎使用 `safe` 过滤器
    \item 验证所有用户输入
    \item 限制模板中可执行的操作
\end{itemize}

\chapter{总结}

Jinja2 是一个功能强大、灵活易用的模板引擎,在 Ansible 中发挥着重要作用。它不仅可以用于生成配置文件,还可以用于控制 Playbook 的执行逻辑。

掌握 Jinja2 的基本语法、过滤器和高级特性,可以大大提高 Ansible 自动化的效率和灵活性。通过合理组织模板文件、优化性能和确保安全性,可以创建出高质量的自动化配置。

Jinja2 的学习曲线相对平缓,但要真正掌握其高级特性和最佳实践,需要不断实践和积累经验。希望本章节的内容能够帮助你快速上手 Jinja2,并在 Ansible 自动化中发挥其强大功能。

\chapter{参考资料}

\begin{thebibliography}{99}
    \bibitem{jinja2-docs} Jinja2 官方文档:https://jinja.palletsprojects.com/
    \bibitem{ansible-templating} Ansible 模板文档:https://docs.ansible.com/ansible/latest/user_guide/playbooks_templating.html
    \bibitem{python-docs} Python 官方文档:https://docs.python.org/
    \bibitem{jinja2-templates} Jinja2 模板设计指南:https://jinja.palletsprojects.com/en/latest/templates/
\end{thebibliography}

% 索引
\printindex

\end{document}