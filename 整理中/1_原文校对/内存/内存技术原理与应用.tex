\documentclass{article}
\usepackage{ctex}
\usepackage{graphicx}
\usepackage{amsmath}
\usepackage{amssymb}
\usepackage{geometry}
\usepackage{multirow}
\usepackage{booktabs}
\usepackage{listings}
\usepackage{color}

\geometry{a4paper, margin=1in}
\title{内存技术原理与应用}
\author{}
\date{}

\begin{document}

\maketitle

\tableofcontents

\section{内存技术概述}

\subsection{内存的基本概念}

\subsection{内存的分类}

\subsection{内存的发展历程}

\section{内存基础理论}

\subsection{半导体存储原理}

\subsection{内存的工作原理}

\subsection{内存的性能指标}

\section{内存类型与技术}

\subsection{RAM(随机存取存储器)}

\subsubsection{SRAM(静态随机存取存储器)}

\subsubsection{DRAM(动态随机存取存储器)}

\subsection{ROM(只读存储器)}

\subsubsection{Mask ROM(掩膜只读存储器)}

\subsubsection{PROM(可编程只读存储器)}

\subsubsection{EPROM(可擦除可编程只读存储器)}

\subsubsection{EEPROM(电可擦除可编程只读存储器)}

\subsubsection{Flash ROM(闪存)}

\subsection{缓存(Cache)}

\subsubsection{L1缓存}

\subsubsection{L2缓存}

\subsubsection{L3缓存}

\subsection{虚拟内存}

\subsection{内存接口技术}

\subsubsection{DIMM}

\subsubsection{SO-DIMM}

\subsubsection{SIMM}

\section{内存技术详解}

\subsection{DRAM技术发展}

\subsubsection{SDRAM}

\subsubsection{DDR SDRAM}

\subsubsection{DDR2 SDRAM}

\subsubsection{DDR3 SDRAM}

\subsubsection{DDR4 SDRAM}

\subsubsection{DDR5 SDRAM}

\subsection{LPDDR技术}

\subsection{GDDR技术}

\subsection{HBM技术}

\subsection{3D内存技术}

\section{内存架构与设计}

\subsection{内存控制器}

\subsection{内存通道}

\subsection{内存时序}

\subsection{内存电压}

\subsection{内存散热}

\section{内存优化与管理}

\subsection{内存分配与回收}

\subsection{内存碎片管理}

\subsection{内存压缩}

\subsection{内存虚拟化}

\subsection{内存超频}

\section{内存测试与诊断}

\subsection{内存测试方法}

\subsection{内存诊断工具}

\subsection{内存故障分析}

\section{内存技术的应用}

\subsection{个人计算机内存}

\subsection{服务器内存}

\subsection{移动设备内存}

\subsection{嵌入式系统内存}

\subsection{图形卡内存}

\section{内存技术的发展趋势}

\subsection{高容量内存}

\subsection{高速内存}

\subsection{低功耗内存}

\subsection{智能内存}

\subsection{非易失性内存}

\section{结论}

\end{document}