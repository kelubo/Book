% 天线
% 天线.tex

\documentclass[12pt,UTF8]{ctexbook}

% 设置纸张信息。
% 纸张设置配置文件
% 用于定义书籍的页面尺寸和边距

\usepackage[a4paper,twoside]{geometry}
\geometry{
	left=25mm,
	right=20mm,
	top=25mm,
	bottom=25.4mm,
	headsep=1cm, 
    footskip=1cm,
	bindingoffset=10mm
}

% 设置字体,并解决显示难检字问题。
\xeCJKsetup{AutoFallBack=true}
\setCJKmainfont{SimSun}[BoldFont=SimHei, ItalicFont=KaiTi, FallBack=SimSun-ExtB]

% 目录 chapter 级别加点(.)。
\usepackage{titletoc}
\titlecontents{chapter}[0pt]{\vspace{3mm}\bf\addvspace{2pt}\filright}{\contentspush{\thecontentslabel\hspace{0.8em}}}{}{\titlerule*[8pt]{.}\contentspage}

% 设置 part 和 chapter 标题格式。
\ctexset{
	chapter/name={第,章},
	chapter/number={\chinese{chapter}}
}

% 图片相关设置。
\usepackage{graphicx}
\graphicspath{{Images/}}

% 设置署名格式。
\newenvironment{shuming}{\hfill\zihao{4}}

% 注脚每页重新编号,避免编号过大。
\usepackage[perpage]{footmisc}

\title{\heiti\zihao{0} 天线}
\author{佚名}
\date{}

\begin{document}

\maketitle
\tableofcontents

\frontmatter

\mainmatter

\chapter{天线概述}

\section{天线的定义}

天线是从空间辐射或接收电磁波(信息)的装置。天线是一种变换器,它把传输线上传播的导行波变换成在无界媒介(通常是自由空间)中传播的电磁波,或者进行相反的变换。天线广泛应用于无线电通信、广播、电视、雷达、导航、电子对抗、遥感、射电天文等工程系统。凡是利用电磁波来传递信息的,都依靠天线来工作。

\section{天线的作用}

天线的作用是实现电磁波的发射和接收。发射天线将高频电流转换为电磁波,向空间辐射电磁波。接收天线接收空间中的电磁波,将其转换为高频电流,供接收设备处理。天线的作用决定了无线通信系统的性能,包括通信距离、信号质量、抗干扰能力等。

\section{天线的分类}

天线按照不同的标准可以分为多种类型。按用途可以分为发射天线和接收天线,按工作频段可以分为长波天线、中波天线、短波天线、超短波天线、微波天线等,按结构形式可以分为线天线、面天线、阵列天线等,按极化方式可以分为垂直极化天线、水平极化天线、圆极化天线等。

\chapter{电磁波的发现和应用}

\section{电磁波的基本特性}

从科学的角度来说,电磁波是能量的一种,凡是高于绝对零度的物体,都会释放电磁波。且温度越高,释放出的电磁波波长就越短。

\begin{figure}[htbp]
\centering
\includegraphics[width=0.9\textwidth]{电磁波谱和频段.png}
\caption{电磁波谱和频段}
\label{fig:电磁波谱和频段}
\end{figure}

\section{电磁波的发现}

电磁波的发现是物理学史上的重要里程碑。1831年,法拉第首次发现电磁感应现象,即:当一块磁铁穿过一个闭合线路时,线路内就会有电流产生。1864年,英国的J.C.麦克斯韦总结了前人的科学成果,发表了《电磁场的动力学理论》,提出了电磁波学说。1887年,德国科学家H.R.赫兹用一个振荡偶子产生了电磁波,在历史上第一次直接验证了电磁波的存在。

\begin{description}
\item[赫兹实验的详细过程] 为了用实验来验证麦克斯韦高深莫测的电磁场理论,验证电磁波的存在,赫兹精心设计了一个电磁波发生器,对"电火花实验"进行了一系列深入的研究。赫兹用两块边长40.64cm的正方形锌板,每块锌板接上一个30.5cm长的铜棒,铜棒的一端焊上一个金属球,将铜棒与感应圈的电极相连。通电时,如果使两根铜棒上的金属球靠近,便会看到有火花从一个球跳到另一个球,这些火花表明电流在循环不息。在金属球之间产生的这种高频电火花,即电磁波。麦克斯韦的理论认为由此电磁波便会被送到空间去。赫兹为了捕捉这些电磁波,证明电磁波确实被送到了空间,他用一根两端带有铜球的铜丝弯成环状,当作检波器。他把这个检波器放到离电磁波发生器10m远的地方,当电磁波发生器通电后,检波器铜丝圈两端的铜球上产生了电火花。这些火花是怎样产生的呢?赫兹认为:这便是电磁波从发射器发出后,被检波器捉住了;电磁波不仅产生了,而且传播了10m远。
\end{description}

\section{电磁波的应用}

电磁波的应用范围广泛,涉及通信、广播、雷达、导航、医疗、工业等多个领域。在通信领域,电磁波用于无线电通信、移动通信、卫星通信等。在广播领域,电磁波用于中波广播、短波广播、调频广播、电视广播等。在雷达领域,电磁波用于搜索雷达、跟踪雷达、气象雷达等。在导航领域,电磁波用于GPS导航、北斗导航等。

在医疗领域,电磁波用于核磁共振、X射线成像、微波治疗等。核磁共振利用强磁场和射频脉冲,对人体进行成像。X射线成像利用X射线的穿透能力,对人体内部结构进行成像。微波治疗利用微波的热效应,治疗肿瘤等疾病。

在工业领域,电磁波用于感应加热、微波干燥、无损检测等。感应加热利用电磁感应原理,对金属进行加热。微波干燥利用微波的热效应,对物料进行干燥。无损检测利用电磁波的反射特性,检测材料内部的缺陷。

电磁波的应用推动了现代科技的发展,改变了人们的生活方式和工作方式。电磁波技术的进步使得信息传输更加便捷高效,使得医疗诊断更加准确安全,使得工业生产更加智能高效。电磁波的应用将继续扩展,为人类社会带来更多的便利和创新。无线电技术中使用的电磁波成为无线电波。

\begin{figure}[htbp]
\centering
\includegraphics[width=0.9\textwidth]{无线电波波段.png}
\caption{无线电波波段}
\label{fig:无线电波波段}
\end{figure}

\section{无线电通信的发展}

1888年,赫兹的发现激发了俄国科学家波波夫(亚历山大·斯塔帕诺维奇·波波夫,АлександрСтепановичПопов, 1859-1906)的研究兴趣。1889年,他多次重复了赫兹的实验,并提出"电磁波可以用来向远处发送信号"。在一次实验中,波波夫发现金属屑检波器的灵敏度异常地高。接收电磁波的距离比起平时有明显的增加。他没放过这个异常现象,仔细地观察了周围环境,也没发现什么变化。找了很多原因,但都一一排除了。他感到很奇怪,再试一次,灵敏度还是异常得高。忽然,他瞥见有一根导线搭在检波器上。很明显,这根导线增加了检波器的接收能力,增加了灵敏度。波波夫真是喜出望外,提高机器的灵敏度,增加传收距离的愿望竟在这无意中达到了。他使用的这根导线是世界上的第一根天线。波波夫对无线电通信的杰出贡献,就是他发现了天线的作用。

1894年,波波夫改进了赫兹的实验装置,利用撒了金属粉末的检波器,通过架在高空的导线,记录了大气中的放电现象。这是世界上第一台无线电接收机。

1895年5月7日,波波夫在俄国的物理学部年会上表演了他创造的这个"雷暴指示器"。1896年3月24日,波波夫又在彼得堡大学两幢相距250m的大楼之间表演了无线电通信,他和助手进行了一次正式的无线电传递莫尔斯电码的表演。

1897年,波波夫奉命在俄国波罗的海舰队的一些舰艇上建立无线电通信设备。1899年,波波夫将无线电投入军事应用,建立了40多km的无线电通信网。

1894年,意大利科学家G.马可尼在赫兹实验的基础上,在家中的楼上安装了发射电波的装置,楼下放置了检波器,并让检波器与电铃相接。他在楼上一接通电源,电磁波便穿过了检波器,让楼下的电铃迅速响了起来。第二年(1895年)夏天,马可尼又完成了一次非常成功的实验。到了秋天,实验又获得很大的进步。他把一只煤油桶展开,变成一块大铁板,作为发射的天线。把接收机的天线高挂在一棵大树上,用以增加接收信号的灵敏程度。他把发射机放在一座山岗的一侧,接收机安放在山岗另一侧的家中。当他的助手发送信号时,他有些紧张地守候着信号接收机。突然,电铃发出了清脆的响声。这响声对他来说比动人的交响乐更悦耳动听,让他几乎跳了起来。马可尼成功了!这次实验的距离达到2.7km,出现了历史性的突破。

1896年,马可尼抱着自己简陋的无线电发射机来到了工业革命的中心-英国,在伦敦开始了自己的创业生涯。1896年6月,他用电磁波进行了约14.4km距离的无线电通信试验;1898年,在英吉利海峡两岸进行无线电报跨海试验成功,通信距离为45公里;1899年又建立了106千米距离的通信联系。1901年12月,马可尼在加拿大用风筝牵引天线,成功地接收到了大西洋彼岸的无线电报,完成了横跨大西洋3600km的无线电远距离通信。

马可尼和波波夫关于无线电通信的发明,都是在1895年一1901年这短短的五六年时间内,各自独立完成的。因此可以说,无线电应用的大门是马可尼和波波夫同时打开的。天线也随着无线电的应用和发展而逐渐发展起来。

最早的发射天线是赫兹在1887年为了验证麦克斯韦根据理论推导所作关于存在电磁波的预言而设计的。它是两个约为30cm长、位于一直线上的金属杆,其远离的两端分别与两个约40$\text{cm}^2$的正方形金属板相连接,靠近的两端分别连接两个金属球并接到一个感应线圈的两端,利用金属球之间的火花放电来产生振荡。当时,赫兹用的接收天线是单圈金属方形环状天线,根据方环端点之间空隙出现火花来指示收到了信号。马可尼是第一个采用大型天线实现远洋通信的,所用的发射天线由30根下垂铜线组成,顶部用水平横线连在金属方形环状天线,根据方环端点之间空隙出现火花来指示收到了信号。这是人类真正付之实用的第一副天线。自从这副天线产生以后,天线的发展大致分为四个历史时期。

1.20世纪30年代之前:线天线时期

线天线是由线径远比波长小,长度可与波长相比的一根或多根金属导线构成的天线。主要用于长、中、短波及超短波波段,作为发射或接收天线。

在无线电获得应用的最初时期,真空管振荡器尚未发明,人们认为波长越长,传播中衰减越小。因此,为了实现远距离通信,所利用的波长都在1000m以上。在这一波段中,显然水平天线是不合适的,因为大地中的镜像电流和天线电流方向相反,天线辐射很小。此外,它所产生的水平极化波沿地面传播时衰减很大。因此,在这一时期应用的是各种不对称天线,如倒L形、T形、伞形天线等。由于高度受到结构上的限制,这些天线的尺寸比波长小很多,因而是属于电小天线的范畴。

倒L形天线

在单根水平导线的一端连接一根垂直引下线而构成的天线。因其形状像英文字母L倒过来,故称倒L形天线。俄文字母的Г字正好是英文字母L的倒写。故称Г型天线更方便。它是垂直接地天线的一种形式。为了提高天线的效率,它的水平部分可用几根导线排在同一水平面上组成,这部分产生的辐射可忽略,产生辐射的是垂直部分。倒L形天线一般用于长波通信。它的优点是结构简单、架设方便;缺点是占地面积大、耐久性差。

T形天线

T形天线是最常见的一种垂直接地的天线。一般用于长波和中波通信。在水平导线的中央,接上一根垂直引下线,形状像英文字母T,故称T形天线。它是最常见的一种垂直接地的天线。它的水平部分辐射可忽略,产生辐射的是垂直部分。为了提高效率,水平部分也可用多根导线组成。T形天线的特点与倒L形天线相同。它一般用于长波和中波通信。

伞形天线

在单根垂直导线的顶部,向各个方向引下几根倾斜的导体,这样构成的天线形状像张开的雨伞,故称伞形天线。它也是垂直接地天线的一种形式。其特点和用途与倒L形、T形天线相同。

20世纪20年代,业余无线电爱好者发现短波能传播到很远的距离。A.E.肯内利和 Q.亥维赛发现了电离层的存在和它对短波的反射作用,从而开辟了短波波段和中波波段领域。这时,天线尺寸可以与波长相比拟,促进了天线的顺利发展。这一时期,除了抗衰落的塔式广播天线外,还设计出各种水平天线和各种天线阵,采用的典型天线有偶极天线(见对称天线)、环形天线、长导线天线、同相水平天线、八木天线(见八木--宇田天线)、菱形天线和鱼骨形天线等。这些天线比初期的长波天线有较高的增益、较强的方向性和较宽的频带,后来一直得到使用并经过不断改进。

【天线阵】

由许多相同的单个天线(如对称天线)按一定规律排列组成的天线系统,也称天线阵。天线在通信、广播、电视、雷达和导航等无线电系统中被广泛应用,起到了传播无线电波的作用,是有效地辐射和接收无线电波必不可少的装置。

在这一时期,天线的理论工作也得到了发展。H.C.波克林顿在1897年建立了线天线的积分方程,证明了细线天线上的电流近似正弦分布。由于数学上的困难,他并未解出这一方程。后来E.海伦利用δ函数源来激励对称天线得到积分方程的解。同时,A.A.皮斯托尔哥尔斯提出了计算线天线阻抗的感应电动势法和二重性原理。R.W.P.金继海伦之后又对线天线做了大量理论研究和计算工作。将对称天线作为边值问题并用分离变量法来求解的有S.A.谢昆穆诺失、H.朱尔特、J.A.斯特拉顿和朱兰成等。

2.20世纪30年代到1945年:面天线时期

面天线是指具有初级馈源并由反射面形成次级辐射场的天线。主要应用于微波和毫米波波段。前馈式抛物面天线、卡塞格伦式和格雷戈里式双镜天线等均属面天线。

虽然早在1888年赫兹就首先使用了抛物柱面天线,但由于没有相应的振荡源,一直到20世纪30年代才随着微波电子管的出现陆续研制出各种面天线。这时已有类比于声学方法的喇叭天线、类比于光学方法的抛物反射面天线和透镜天线等。这些天线利用波的扩散、干涉、反射、折射和聚焦等原理获得窄波束和高增益。

第二次世界大战期间出现了雷达,大大促进了微波技术的发展。为了迅速捕捉目标,研制出了波束扫描天线,利用金属波导和介质波导研制出波导缝隙天线和介质棒天线以及由它们组成的天线阵。在面天线基本理论方面,建立了几何光学法、物理光学法和口径场法等理论。当时,由于战时的迫切需要,天线的理论还不够完善。天线的实验研究成了研制新型天线的重要手段,建立了测试条件和误差分析等概念,提出了现场测量和模型测量等方法(见天线参量测量)。在面天线有较大发展的同时,线天线理论和技术也有发展,如阵列天线的综合方法等。

3.从第二次世界大战结束到20世纪50年代末期:阵列天线、宽带和多频带天线

4.20世纪50年代以后

\chapter{天线原理}

\section{电磁波原理}

天线的工作原理基于电磁波理论。电磁波是由变化的电场和磁场相互激发而形成的波动现象。天线通过导体中的高频电流产生变化的电场和磁场,形成电磁波并向空间辐射。接收天线通过电磁波在导体中感应出高频电流,实现电磁波的接收。

\section{辐射原理}

天线的辐射原理是指天线将电能转换为电磁波并向空间辐射的过程。辐射原理包括电流辐射、电荷辐射、磁流辐射等。电流辐射是指导体中的高频电流产生电磁辐射。电荷辐射是指导体上的高频电荷产生电磁辐射。磁流辐射是指磁流的变化产生电磁辐射。

\section{接收原理}

天线的接收原理是指天线将电磁波转换为电能的过程。接收原理包括电磁感应、电磁耦合、电磁共振等。电磁感应是指电磁波在导体中感应出电流。电磁耦合是指电磁波通过耦合元件耦合到接收电路。电磁共振是指天线在特定频率下产生共振,提高接收效率。

\chapter{天线参数}

\section{增益}

增益是天线的一个重要参数,表示天线在特定方向上的辐射能力相对于参考天线的增强程度。增益通常以分贝表示,增益越高,天线在该方向上的辐射能力越强。增益与天线的尺寸、形状、效率等因素有关。

\section{方向性}

方向性是指天线在不同方向上的辐射或接收能力的不均匀性。方向性用方向图表示,方向图显示了天线在不同方向上的辐射或接收强度。方向性可以分为全向天线和定向天线,全向天线在所有方向上辐射或接收能力相同,定向天线在特定方向上辐射或接收能力较强。

\section{阻抗}

阻抗是天线的一个重要参数,表示天线对电流的阻碍作用。阻抗通常以欧姆表示,阻抗匹配是天线设计的关键问题。阻抗匹配是指天线阻抗与馈线阻抗相等,以实现最大功率传输。阻抗不匹配会导致反射,降低天线效率。

\section{带宽}

带宽是天线的一个重要参数,表示天线能够有效工作的频率范围。带宽通常以赫兹表示,带宽越宽,天线能够工作的频率范围越大。带宽与天线的结构、尺寸、材料等因素有关。宽带天线能够覆盖多个频段,窄带天线只能覆盖单个频段。

\section{驻波比}

驻波比是天线的一个重要参数,表示天线与馈线之间的阻抗匹配程度。驻波比通常以比值表示,驻波比越接近1,阻抗匹配越好。驻波比过高会导致反射,降低天线效率。驻波比通常要求小于2,理想情况下等于1。

\chapter{天线类型}

\section{偶极天线}

\begin{figure}[htbp]
\centering
\includegraphics[width=0.9\textwidth]{偶极子天线.jpg}
\caption{偶极子天线}
\label{fig:偶极子天线}
\end{figure}

偶极子天线用来发射和接收固定频率的信号。虽然在平时的测量中都使用宽带天线,但在场地衰减和天线系数的测量中都需要使用偶极子天线。

一天,戴柏(Diogenes Dipole)走过一个游乐场,发现狮子正在玩跷跷板,他发现这些狮子都能很快地保持平衡,于是突发奇想:如果天线也能保持平衡,效果会怎样呢?回到家后,戴柏马上拿了一条导线接上机器外壳,另一条导线则接到发射机输出,把两根导线对称摆开,这就成为一组新的天线。这种平衡天线非常好用!这就是大名鼎鼎的"双偶极"(Dipole)天线,为了纪念戴柏,以他的名字来命名。

由于家里空间不够大,无法架设双偶极天线,所以无线电爱好者崔伯(VonTrap)沿着天线,每隔几英尺就绕几个圈,把过长的部分缠绕起来,并且在缠绕的电感上并联电容,这就是"崔伯双偶极天线"(TrapDipole),也称陷波式偶极天线。

承袭者温顿(Raoul Windom)发现跷跷板放上不同重量的物体,通过调整距离也可以达到平衡,天线应该也可以像这样,以人工方式调整,达到平衡(匹配)。于是温顿天线(偏馈天线)被发明了。第二次世界大战期间,温顿天线广泛应用在战机上,直到喷气机时代才光荣退休。

偶极天线是最基本的天线类型,由两根长度为四分之一波长的导体组成。偶极天线具有简单的结构、良好的性能、易于制作的特点。偶极天线广泛应用于广播、电视、移动通信等领域,是天线设计的基础。

\section{单极天线}

单极天线是由一根长度为四分之一波长的导体组成的天线。单极天线通常需要地平面作为反射面,地平面可以是地面、金属板等。单极天线具有结构简单、成本低廉的特点,广泛应用于移动通信、广播等领域。

\section{环形天线}

\begin{figure}[htbp]
\centering
\includegraphics[width=0.9\textwidth]{环形天线.jpg}
\caption{环形天线}
\label{fig:环形天线}
\end{figure}

环形天线是将一根金属导线绕成一定形状,如圆形、方形、三角形等,以导体两端作为输出端的结构。绕制多圈(如螺旋状或重叠绕制)的称为多圈环天线。根据环形天线的周长L相对于波长λ的大小,环形天线可分为电大环(L≥λ)、中等环(λ/4≤L≤λ)和电小环(L<λ/4)三类。电小环天线是实际中应用最多的,如收音机中的天线、便携式电台接收天线、无线电导航定位天线、场强计的探头天线等。电大环天线主要用作定向阵列天线的单元。

环形天线是由环形导体组成的天线,环的周长通常为一个波长。环形天线具有体积小、抗干扰能力强的特点,广泛应用于接收设备、测向设备等领域。环形天线的方向图为环形,适合全向接收。

\section{八木天线}

\begin{figure}[htbp]
\centering
\includegraphics[width=0.9\textwidth]{八木天线.jpg}
\caption{八木天线}
\label{fig:八木天线}
\end{figure}

八木天线是由一个有源振子(一般用折合振子)、一个无源反射器和若干个无源引向器平行排列而成的端射式天线。在20世纪20年代,由日本东北大学的八木秀次和宇田太郎两人发明了这种天线,被称为"八木一宇田天线",简称"八木天线"。八木天线的确好用,它有很好的方向性,较偶极天线有高的增益。用它来测向、远距离通信效果特别好。如果再配上仰角和方位旋转控制装置,更可以随心所欲与包括空间飞行器在内的各个方向上的电台联络,这种感受从直立天线上是得不到的。

\section{同相水平天线}

同相水平天线(图1-3)是由同相馈电的水平对称振子组成的边射式平面天线阵,为了保证单向的辐射和接收,在阵面的一侧设置反射面。这种天线可用于短波干线通信或广播和米波警戒雷达等。

\section{对数周期天线}

对数周期天线是由多个偶极天线按对数周期排列组成的天线。对数周期天线具有宽频带、高增益的特点,广泛应用于电视接收、广播等领域。对数周期天线的偶极天线长度按对数比例变化,实现宽频带特性。

\section{螺旋天线}

螺旋天线是由螺旋形导体组成的天线。螺旋天线具有圆极化、宽频带的特点,广泛应用于卫星通信、移动通信等领域。螺旋天线的螺旋形状产生圆极化电磁波,适合卫星通信等需要圆极化的应用。

\section{菱形天线}

\begin{figure}[htbp]
\centering
\includegraphics[width=0.9\textwidth]{菱形天线.jpg}
\caption{菱形天线}
\label{fig:菱形天线}
\end{figure}

菱形天线是一种宽频带天线。它是由一个水平的菱形悬挂在四根支柱上构成的,菱形的一只锐角接在馈线上,另一只锐角接一个与菱形天线特性阻抗相等的终端电阻。

\section{鱼骨天线}

鱼骨形天线(Fishbone Antenna)利用简化方法来计算鱼骨形天线的结构尺寸。通过分析天线在垂直面和水平面内的受力情况,利用简单的力学方法计算出天线边吊索和振子尾线的长度,为天线结构设计和工程应用提供数据支撑。计算方法经过实际工程验证,适用于大多数柔索结构的分析计算。

\section{微带天线}

微带天线是制作在介质基板上的平面天线。微带天线具有体积小、重量轻、易于集成的特点,广泛应用于移动通信、卫星通信等领域。微带天线通常采用贴片天线、缝隙天线等形式,适合与集成电路集成。

\section{阵列天线}

阵列天线是由多个天线单元按一定规律排列组成的天线。阵列天线具有高增益、方向性可调的特点,广泛应用于雷达、卫星通信、移动通信等领域。阵列天线的天线单元可以同相馈电,通过控制各单元的相位实现波束扫描和方向图控制。

\section{抛物面天线}

抛物面天线是由抛物面反射器和馈源组成的天线。抛物面天线具有高增益、方向性强的特点,广泛应用于卫星通信、雷达、射电天文等领域。抛物面天线的抛物面反射器将电磁波聚焦到馈源,实现高增益接收或发射。

\section{喇叭天线}

\begin{figure}[htbp]
\centering
\includegraphics[width=0.9\textwidth]{喇叭天线.jpg}
\caption{喇叭天线}
\label{fig:喇叭天线}
\end{figure}

喇叭天线是面天线的一种,波导管终端渐变张开的圆形或矩形截面的微波天线,是使用最广泛的一类微波天线。它的辐射场是由喇叭的口面尺寸与传播型所决定的。其中,喇叭壁对辐射的影响可以利用几何绕射的原理来进行计算。如果喇叭的长度保持不变,口面尺寸与二次相位差会随着喇叭张角的增大而增大,但增益则不会随着口面尺寸变化。如果需要扩展喇叭的频带,则需要减小喇叭颈部与口面处的反射;反射会随着口面尺寸加大反而减小。喇叭天线的结构比较简单,方向图也比较简单而容易控制,一般作为中等方向性天线。频带宽、副瓣低和效率高的抛物反射面喇叭天线常用于微波中继通信。喇叭天线是一种应用广泛的微波天线,其优点是结构简单、频带宽、功率容量大、调整与使用方便。合理地选择喇叭尺寸,可以取得良好的辐射特性;相当尖锐的主瓣,较小的副瓣和较高的增益。喇叭天线在军事和民用上都非常广泛,是一种常见的测试用天线。

喇叭天线是由喇叭形辐射器组成的天线。喇叭天线具有宽频带、高增益的特点,广泛应用于微波通信、雷达等领域。喇叭天线的喇叭形状实现阻抗匹配和方向性控制,适合微波频段的应用。

\section{介质天线}

介质天线是由介质材料制成的天线。介质天线具有体积小、重量轻的特点,广泛应用于移动通信、卫星通信等领域。介质天线利用介质材料的介电常数实现电磁波的辐射或接收,适合与集成电路集成。

\section{智能天线}

智能天线是能够自适应调整方向图的天线系统。智能天线由多个天线单元和信号处理单元组成,能够根据信号环境自动调整方向图。智能天线具有提高容量、降低干扰、改善覆盖的特点,广泛应用于移动通信、雷达等领域。

\chapter{天线设计}

\section{设计原则}

天线设计的基本原则包括阻抗匹配、带宽优化、增益优化、方向性控制等。阻抗匹配是天线与馈线之间的阻抗相等,实现最大功率传输。带宽优化是天线覆盖所需的频率范围。增益优化是天线在特定方向上的辐射能力最大化。方向性控制是天线在所需方向上的辐射或接收能力最大化。

\section{设计方法}

天线设计的方法包括理论计算、仿真分析、实验测试等。理论计算是基于天线理论,通过数学公式计算天线的参数。仿真分析是使用电磁仿真软件,模拟天线的性能。实验测试是制作天线原型,通过实际测试验证天线性能。

\section{设计软件}

天线设计常用的软件包括HFSS、CST、FEKO等电磁仿真软件。HFSS是Ansys公司的高频结构仿真器,适用于三维天线仿真。CST是Dassault公司的电磁仿真软件,适用于时域和频域仿真。FEKO是Altair公司的电磁仿真软件,适用于大型天线阵列仿真。

\section{设计实例}

天线设计实例包括偶极天线设计、八木天线设计、阵列天线设计等。偶极天线设计包括计算偶极天线的长度、阻抗、增益等参数。八木天线设计包括计算有源振子和无源振子的长度、间距、增益等参数。阵列天线设计包括计算天线单元的排列、相位、波束扫描等参数。

\chapter{天线测量}

\section{测量参数}

天线测量的主要参数包括增益、方向图、阻抗、带宽、驻波比等。增益测量使用标准天线作为参考,比较被测天线和参考天线的接收功率。方向图测量使用转台旋转天线,测量不同方向上的辐射强度。阻抗测量使用网络分析仪,测量天线的输入阻抗。带宽测量测量天线的频率响应,确定有效工作频率范围。驻波比测量使用驻波比表,测量天线与馈线之间的阻抗匹配程度。

\section{测量设备}

天线测量的设备包括网络分析仪、频谱分析仪、信号发生器、功率计等。网络分析仪用于测量天线的阻抗、驻波比等参数。频谱分析仪用于测量天线的辐射频谱。信号发生器用于产生测试信号。功率计用于测量天线的发射或接收功率。

\section{测量环境}

天线测量需要在合适的测试环境中进行,包括开阔场、暗室、紧缩场等。开阔场是远离反射物的开阔场地,适合低频天线测量。暗室是内壁吸波的封闭空间,适合高频天线测量。紧缩场是缩小的测试环境,适合小型天线测量。

\chapter{天线应用}

\section{广播应用}

天线在广播领域有广泛应用,包括中波广播、短波广播、调频广播等。中波广播使用垂直天线,传播距离远但音质较差。短波广播使用水平天线,传播距离远但信号不稳定。调频广播使用垂直天线,传播距离近但音质较好。

\section{电视应用}

天线在电视领域有广泛应用,包括地面电视、卫星电视、有线电视等。地面电视使用八木天线或对数周期天线,接收地面电视信号。卫星电视使用抛物面天线,接收卫星电视信号。有线电视使用分配天线,将电视信号分配到用户。

\section{移动通信应用}

天线在移动通信领域有广泛应用,包括基站天线、手机天线等。基站天线使用定向天线或阵列天线,覆盖特定区域。手机天线使用内置天线或外置天线,实现移动通信。移动通信天线需要宽频带、多频段、小型化的特点。

\section{卫星通信应用}

天线在卫星通信领域有广泛应用,包括地球站天线、卫星天线等。地球站天线使用大口径抛物面天线,与卫星通信。卫星天线使用喇叭天线或阵列天线,向地球辐射信号。卫星通信天线需要高增益、高方向性、高可靠性的特点。

\section{雷达应用}

天线在雷达领域有广泛应用,包括搜索雷达、跟踪雷达、气象雷达等。搜索雷达使用阵列天线,实现大范围搜索。跟踪雷达使用跟踪天线,实现目标跟踪。气象雷达使用抛物面天线,探测气象目标。雷达天线需要高增益、高方向性、快速扫描的特点。

\section{导航应用}

天线在导航领域有广泛应用,包括GPS导航、北斗导航等。GPS天线使用微带天线或螺旋天线,接收GPS卫星信号。北斗天线使用微带天线或螺旋天线,接收北斗卫星信号。导航天线需要高灵敏度、低噪声、多频段的特点。

\chapter{天线维护}

\section{日常维护}

天线的日常维护包括清洁检查、紧固检查、防腐蚀处理等。清洁检查是清除天线表面的灰尘和污垢,保持天线性能。紧固检查是检查天线的连接部位,确保连接牢固。防腐蚀处理是给金属部件涂防锈漆,防止腐蚀。

\section{故障排除}

天线常见故障包括接触不良、阻抗不匹配、天线损坏等。接触不良会导致信号衰减,需要清洁或重新连接。阻抗不匹配会导致驻波比过高,需要调整天线或馈线。天线损坏会导致性能下降,需要修复或更换天线。

\section{安全注意事项}

天线维护的安全注意事项包括防触电、防坠落、防雷击等。防触电是在断电状态下进行维护,避免触电危险。防坠落是高空作业使用安全带,防止坠落事故。防雷击是安装避雷器,保护天线和设备免受雷击损坏。

\chapter{天线发展趋势}

\section{小型化}

天线的发展趋势之一是小型化。随着移动设备的发展,天线需要越来越小。小型化技术包括微带天线、介质天线、芯片天线等。小型化天线具有体积小、重量轻、易于集成的特点,适合移动设备和便携设备。

\section{宽频带}

天线的发展趋势之二是宽频带。随着通信技术的发展,天线需要覆盖越来越多的频段。宽频带技术包括超宽带天线、多频段天线、可重构天线等。宽频带天线具有频带宽、应用灵活的特点,适合多频段通信。

\section{智能化}

天线的发展趋势之三是智能化。随着信号处理技术的发展,天线需要越来越智能。智能化技术包括自适应天线、智能天线、MIMO天线等。智能化天线具有自适应调整、提高容量、降低干扰的特点,适合复杂信号环境。

\section{集成化}

天线的发展趋势之四是集成化。随着集成电路技术的发展,天线需要与电路集成。集成化技术包括芯片天线、封装天线、系统级封装等。集成化天线具有体积小、成本低、可靠性高的特点,适合大规模生产。

\section{新材料}

天线的发展趋势之五是新材料。随着材料技术的发展,天线需要使用新材料。新材料包括超材料、纳米材料、柔性材料等。新材料天线具有高性能、轻量化、可弯曲的特点,适合新型应用场景。

\backmatter

\end{document}
