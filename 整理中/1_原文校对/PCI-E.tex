\chapter{PCI Express (PCI-E)}

\section{PCI-E 简介}

PCI Express(Peripheral Component Interconnect Express,简称 PCI-E 或 PCIe)是一种高速串行计算机扩展总线标准,用于连接计算机系统中的各种硬件设备。PCI-E 取代了旧的 PCI、PCI-X 和 AGP 总线标准,成为现代计算机系统中最主要的扩展接口标准。

PCI-E 采用点对点串行连接架构,每个设备都拥有专用的连接通道,不需要像传统 PCI 总线那样共享带宽。这种设计大大提高了数据传输效率和系统性能。PCI-E 支持热插拔功能,允许在系统运行时添加或移除设备。

PCI-E 总线广泛应用于显卡、网卡、声卡、存储控制器、USB 控制器等各种扩展卡,是现代计算机系统不可或缺的组成部分。

\section{发展历史}

PCI-E 技术的发展始于 2000 年代初期,由 PCI-SIG(PCI Special Interest Group)组织制定标准。PCI-E 1.0 版本于 2003 年发布,标志着串行总线时代的开始。

随着计算机技术的快速发展,PCI-E 标准不断演进。2007 年发布了 PCI-E 2.0,带宽翻倍;2010 年发布 PCI-E 3.0,再次提升性能;2017 年发布 PCI-E 4.0;2019 年发布 PCI-E 5.0;2022 年发布 PCI-E 6.0。每一代新标准都在传输速率、带宽和功能特性上实现了显著提升。

PCI-E 的演进反映了计算机系统对更高带宽和更快数据传输速度的持续需求,特别是在高性能计算、人工智能和数据中心应用领域。

\section{PCI-E 架构}

\subsection{分层结构}

PCI-E 采用分层架构设计,将通信协议分为三个主要层次:事务层(Transaction Layer)、数据链路层(Data Link Layer)和物理层(Physical Layer)。这种分层设计使得协议的实现和优化更加灵活,各层可以独立演进。

事务层负责处理设备间的数据传输请求和响应,包括读写操作、内存映射 I/O、配置访问等。数据链路层确保数据在物理链路上的可靠传输,提供错误检测和纠正机制。物理层负责实际的信号传输,包括编码、解码、时钟恢复等底层功能。

这种分层架构使得 PCI-E 能够在保持向后兼容性的同时,不断引入新的特性和改进。

\subsection{事务层}

事务层是 PCI-E 协议的最高层,负责生成和处理事务层数据包(TLP)。TLP 包含请求者、完成者、地址、数据等信息,用于在设备之间传输数据。

事务层支持多种事务类型,包括内存读、内存写、I/O 读、I/O 写、配置读、配置写、消息等。每个事务都有一个唯一的标签,用于跟踪事务的完成状态。

事务层还实现了流量控制和信用机制,防止发送方过载接收方。接收方通过信用值告知发送方可用的缓冲区空间,发送方根据信用值控制数据发送速率。

\subsection{数据链路层}

数据链路层位于事务层和物理层之间,负责在物理链路上可靠地传输数据。数据链路层将事务层的数据包封装成数据链路层数据包(DLLP),并添加序列号和 CRC 校验码。

数据链路层实现了确认和重传机制,确保数据传输的可靠性。接收方收到数据包后,发送确认包(ACK);如果发送方未收到确认,会重传数据包。

数据链路层还负责链路初始化、训练和电源管理等功能。在链路初始化过程中,数据链路层与物理层协作,建立稳定的通信链路。

\subsection{物理层}

物理层是 PCI-E 协议的最底层,负责实际的信号传输。物理层采用差分信号传输技术,使用两根信号线传输一个比特的数据,提高了抗干扰能力。

物理层实现了 8b/10b 编码(PCI-E 1.0-3.0)或 128b/130b 编码(PCI-E 4.0+),将数据编码成适合传输的信号格式。编码还提供了时钟恢复和直流平衡功能。

物理层支持多种链路宽度和速度配置,从 x1 到 x32,速度从 2.5 GT/s 到 64 GT/s。物理层还实现了链路训练和均衡,确保信号质量。

\section{PCI-E 版本}

\subsection{PCI-E 1.0}

PCI-E 1.0 于 2003 年发布,是 PCI-E 标准的第一个版本。PCI-E 1.0 的传输速率为 2.5 GT/s(每秒 25 亿次传输),采用 8b/10b 编码,有效带宽为每通道 250 MB/s。

PCI-E 1.0 支持多种链路宽度,包括 x1、x4、x8 和 x16。x16 链路的理论带宽为 4 GB/s,足以满足当时显卡和其他高速设备的需求。

PCI-E 1.0 引入了点对点串行架构、热插拔支持、高级电源管理等创新特性,为后续版本的发展奠定了基础。

\subsection{PCI-E 2.0}

PCI-E 2.0 于 2007 年发布,传输速率提升至 5.0 GT/s,是 PCI-E 1.0 的两倍。PCI-E 2.0 继续采用 8b/10b 编码,有效带宽提升至每通道 500 MB/s。

PCI-E 2.0 增强了电源管理功能,引入了动态功耗管理机制。新标准还改进了事务层协议,提高了数据传输效率。

PCI-E 2.0 向后兼容 PCI-E 1.0 设备,允许不同版本的设备在同一系统中协同工作。x16 链路的理论带宽达到 8 GB/s,满足了更高性能显卡的需求。

\subsection{PCI-E 3.0}

PCI-E 3.0 于 2010 年发布,传输速率提升至 8.0 GT/s。虽然传输速率只比 PCI-E 2.0 提升了 60\%,但由于改用 128b/130b 编码,编码开销从 20\% 降低到 1.5\%,有效带宽提升至每通道 985 MB/s。

PCI-E 3.0 引入了多项新特性,包括 TPH(TLP Processing Hints)、原子操作、低功耗状态等。新标准还改进了错误报告和恢复机制。

PCI-E 3.0 x16 链路的理论带宽达到 15.75 GB/s,成为高性能计算和数据中心应用的主流选择。

\subsection{PCI-E 4.0}

PCI-E 4.0 于 2017 年发布,传输速率翻倍至 16.0 GT/s。继续采用 128b/130b 编码,有效带宽提升至每通道 1.97 GB/s。

PCI-E 4.0 引入了更精细的电源管理机制,支持更多低功耗状态。新标准还改进了信号完整性,提高了长距离传输的可靠性。

PCI-E 4.0 x16 链路的理论带宽达到 31.5 GB/s,广泛应用于高端显卡、高速存储和网络设备。PCI-E 4.0 也是消费级固态硬盘(NVMe SSD)的主流接口标准。

\subsection{PCI-E 5.0}

PCI-E 5.0 于 2019 年发布,传输速率再次翻倍至 32.0 GT/s。继续采用 128b/130b 编码,有效带宽提升至每通道 3.94 GB/s。

PCI-E 5.0 引入了更高级的信号完整性技术,包括前向纠错(FEC)和更复杂的均衡算法。新标准还改进了延迟和功耗特性。

PCI-E 5.0 x16 链路的理论带宽达到 63 GB/s,满足人工智能、机器学习和高性能计算等应用对带宽的巨大需求。

\subsection{PCI-E 6.0}

PCI-E 6.0 于 2022 年发布,传输速率达到 64.0 GT/s。为了在如此高的速率下保持信号完整性,PCI-E 6.0 引入了 PAM4 调制技术,每个符号可以表示 2 个比特。

PCI-E 6.0 采用 1b/1b 编码结合 PAM4 调制,有效带宽提升至每通道 8.0 GB/s。新标准还引入了前向纠错(FEC)机制,提高了数据传输的可靠性。

PCI-E 6.0 x16 链路的理论带宽达到 128 GB/s,为未来的超级计算、人工智能和数据中心应用提供了充足的带宽。

\section{通道配置}

\subsection{x1 通道}

x1 通道是最基本的 PCI-E 配置,使用 1 个差分对(2 根信号线)进行数据传输。x1 通道适用于低带宽设备,如声卡、USB 控制器、网络接口卡等。

x1 通道的带宽取决于 PCI-E 版本:PCI-E 1.0 为 250 MB/s,PCI-E 2.0 为 500 MB/s,PCI-E 3.0 为 985 MB/s,PCI-E 4.0 为 1.97 GB/s,PCI-E 5.0 为 3.94 GB/s,PCI-E 6.0 为 8.0 GB/s。

x1 通道的物理接口较小,占用空间少,适合集成在主板或小型扩展卡上。

\subsection{x4 通道}

x4 通道使用 4 个差分对(8 根信号线)进行数据传输,带宽是 x1 通道的 4 倍。x4 通道适用于中等带宽设备,如高性能网卡、RAID 控制器、某些固态硬盘等。

x4 通道的带宽:PCI-E 1.0 为 1 GB/s,PCI-E 2.0 为 2 GB/s,PCI-E 3.0 为 3.94 GB/s,PCI-E 4.0 为 7.88 GB/s,PCI-E 5.0 为 15.75 GB/s,PCI-E 6.0 为 32 GB/s。

x4 通道的物理接口比 x1 大,但仍然相对紧凑,广泛应用于服务器和高端工作站。

\subsection{x8 通道}

x8 通道使用 8 个差分对(16 根信号线)进行数据传输,带宽是 x1 通道的 8 倍。x8 通道适用于高带宽设备,如高端显卡、高速存储控制器、某些网络设备等。

x8 通道的带宽:PCI-E 1.0 为 2 GB/s,PCI-E 2.0 为 4 GB/s,PCI-E 3.0 为 7.88 GB/s,PCI-E 4.0 为 15.75 GB/s,PCI-E 5.0 为 31.5 GB/s,PCI-E 6.0 为 64 GB/s。

x8 通道的物理接口较大,通常用于需要高带宽但不需要 x16 全带宽的设备。

\subsection{x16 通道}

x16 通道使用 16 个差分对(32 根信号线)进行数据传输,是 PCI-E 标准支持的最大通道配置。x16 通道主要用于显卡和其他极高带宽设备。

x16 通道的带宽:PCI-E 1.0 为 4 GB/s,PCI-E 2.0 为 8 GB/s,PCI-E 3.0 为 15.75 GB/s,PCI-E 4.0 为 31.5 GB/s,PCI-E 5.0 为 63 GB/s,PCI-E 6.0 为 128 GB/s。

x16 通道的物理接口最大,通常位于主板的顶部,专用于显卡安装。

\section{物理接口}

\subsection{插槽类型}

PCI-E 插槽有多种类型,根据支持的通道数和版本而有所不同。常见的插槽类型包括 x1、x4、x8 和 x16,每种插槽都有不同的物理尺寸。

PCI-E 插槽采用机械键设计,确保设备只能插入兼容的插槽。插槽的长度和引脚数量与支持的通道数相关,x16 插槽最长,x1 插槽最短。

PCI-E 插槽还支持向下兼容,例如 x4 设备可以插入 x8 或 x16 插槽,但只能使用 x4 带宽。

\subsection{连接器引脚}

PCI-E 连接器包含多个引脚,用于传输数据、时钟、电源和信号。引脚排列根据通道数而有所不同,但基本结构相似。

PCI-E 连接器的引脚包括:差分信号对(用于数据传输)、时钟信号、电源引脚(3.3V、12V)、地线、辅助信号(如复位、中断、热插拔信号)等。

PCI-E 连接器还预留了备用引脚,用于未来扩展和特殊功能。

\subsection{机械规格}

PCI-E 插槽和连接器有严格的机械规格,确保不同厂商的设备能够互操作。规格包括插槽尺寸、引脚间距、插拔力、耐久性等。

PCI-E 插槽的长度与通道数相关:x1 插槽长约 25mm,x4 插槽长约 39mm,x8 插槽长约 56mm,x16 插槽长约 89mm。插槽的宽度和高度是固定的。

PCI-E 连接器支持热插拔,插拔寿命通常在 100 次以上。连接器还具有良好的电磁兼容性,减少信号干扰。

\section{传输特性}

\subsection{串行传输}

PCI-E 采用串行传输技术,每个差分对在同一时间只传输一个比特的数据。与并行传输相比,串行传输具有更高的抗干扰能力和更低的信号完整性要求。

串行传输使用差分信号,通过两根信号线传输互补信号。接收端通过比较两根线的电压差来恢复数据,这种技术大大提高了抗干扰能力。

串行传输还简化了物理层设计,降低了成本和功耗。随着信号处理技术的进步,串行传输的速率不断提高。

\subsection{全双工通信}

PCI-E 支持全双工通信,即数据可以同时在两个方向上传输。每个差分对都包含发送和接收两个通道,实现双向数据传输。

全双工通信大大提高了链路的利用率和总带宽。例如,x16 通道的带宽为 63 GB/s(PCI-E 5.0),是指每个方向都可以达到 63 GB/s,总带宽为 126 GB/s。

全双工通信还减少了延迟,提高了系统响应速度,特别适合需要同时进行读写操作的应用。

\subsection{点对点连接}

PCI-E 采用点对点连接架构,每个设备都有专用的连接通道,不需要共享带宽。这与传统的 PCI 总线形成鲜明对比,PCI 总线是共享总线架构。

点对点连接消除了总线竞争和仲裁开销,提高了数据传输效率。每个设备都可以充分利用链路带宽,不会受到其他设备的影响。

点对点连接还简化了系统设计,提高了可扩展性。系统可以轻松添加多个高速设备,而不会影响现有设备的性能。

\section{带宽性能}

\subsection{带宽计算}

PCI-E 带宽的计算公式为:带宽 = 通道数 × 每通道速率 × 编码效率。每通道速率以 GT/s(每秒传输次数)为单位,编码效率取决于编码方式。

对于 PCI-E 1.0-3.0,采用 8b/10b 编码,编码效率为 80%。对于 PCI-E 4.0-5.0,采用 128b/130b 编码,编码效率约为 98.5%。对于 PCI-E 6.0,采用 1b/1b 编码结合 PAM4 调制,编码效率为 100%。

例如,PCI-E 4.0 x16 的带宽计算:16 × 16 GT/s × 1.97 GB/s/GT/s = 31.5 GB/s。

\subsection{各版本带宽对比}

不同 PCI-E 版本的带宽对比(每通道):
- PCI-E 1.0: 250 MB/s
- PCI-E 2.0: 500 MB/s
- PCI-E 3.0: 985 MB/s
- PCI-E 4.0: 1.97 GB/s
- PCI-E 5.0: 3.94 GB/s
- PCI-E 6.0: 8.0 GB/s

可以看出,每代新标准的带宽都有显著提升,特别是从 PCI-E 3.0 到 PCI-E 4.0,带宽几乎翻倍。

\subsection{实际传输速率}

实际传输速率通常低于理论带宽,受多种因素影响,包括协议开销、设备性能、系统配置等。实际速率通常是理论带宽的 70-90%。

例如,PCI-E 4.0 x16 的理论带宽为 31.5 GB/s,但实际传输速率可能在 22-28 GB/s 之间,具体取决于设备和应用场景。

实际传输速率还受限于设备的处理能力、驱动程序优化、系统负载等因素。在实际应用中,需要综合考虑这些因素来评估系统性能。

\section{电源管理}

\subsection{电源供应}

PCI-E 插槽为连接的设备提供电源供应,包括 3.3V 和 12V 两种电压。不同规格的插槽提供不同的电源能力。

PCI-E x1 插槽通常提供 10W 的电源,PCI-E x4 插槽提供 25W,PCI-E x8 插槽提供 75W,PCI-E x16 插槽提供 75W。高功耗设备可能需要额外的电源连接器。

PCI-E 还支持动态电源管理,根据设备需求调整电源供应,提高能效。

\subsection{功耗限制}

PCI-E 规范定义了不同配置的功耗限制,确保系统稳定运行。功耗限制包括插槽供电能力和设备功耗要求。

PCI-E 设备必须遵守功耗限制,超过限制的设备需要额外的电源连接器。例如,高端显卡通常需要 6-pin 或 8-pin 电源连接器,提供额外的 75W 或 150W 电源。

功耗限制还考虑了热设计,确保设备在正常工作温度下运行。

\subsection{节能模式}

PCI-E 支持多种节能模式,包括 L0(正常工作)、L0s(低功耗待机)、L1(低功耗)、L1.1/L1.2(深度低功耗)等。

在节能模式下,链路可以降低传输速率或关闭部分通道,减少功耗。设备可以根据工作负载动态切换节能模式,平衡性能和功耗。

节能模式对于移动设备和数据中心尤为重要,可以显著降低总功耗和运营成本。

\section{热插拔支持}

\subsection{热插拔机制}

PCI-E 支持热插拔功能,允许在系统运行时添加或移除设备。热插拔机制包括物理检测、电气隔离、软件通知等步骤。

热插拔过程:1) 检测设备插入/移除;2) 隔离设备电源;3) 通知操作系统;4) 加载/卸载驱动程序;5) 恢复设备功能。

热插拔需要硬件和软件的协同支持,包括主板、设备、驱动程序和操作系统。

\subsection{设备识别}

热插拔设备插入后,系统需要识别设备并分配资源。设备识别过程包括枚举设备、读取配置空间、分配资源等步骤。

设备识别通过 PCI-E 配置空间实现,每个设备都有唯一的标识符(Vendor ID、Device ID)。系统根据这些标识符加载相应的驱动程序。

设备识别还需要考虑资源分配,包括内存地址、I/O 地址、中断号等。

\subsection{驱动加载}

设备识别后,系统需要加载相应的驱动程序。驱动加载过程包括查找驱动、加载驱动、初始化设备等步骤。

驱动加载可以是自动的,也可以是手动的。现代操作系统通常支持即插即用,自动识别和加载驱动。

驱动加载后,设备就可以正常工作,用户可以开始使用设备的功能。

\section{设备枚举}

\subsection{配置空间}

每个 PCI-E 设备都有一个配置空间,用于存储设备的配置信息和状态。配置空间包括标准配置头和设备特定配置区域。

标准配置头包含设备标识符(Vendor ID、Device ID)、类代码、基地址寄存器(BAR)、中断引脚等信息。设备特定配置区域由厂商自定义。

配置空间可以通过配置读/写事务访问,用于设备初始化和配置。

\subsection{设备ID分配}

系统启动时,会枚举所有 PCI-E 设备并分配设备 ID。设备 ID 包括总线号、设备号和功能号,用于唯一标识设备。

设备 ID 分配采用深度优先遍历算法,从根复合体开始,逐层扫描总线。每个设备都分配一个唯一的 ID。

设备 ID 用于访问设备的配置空间和内存映射 I/O 区域。

\subsection{资源分配}

设备枚举后,系统需要为每个设备分配资源,包括内存地址、I/O 地址、中断号等。资源分配必须避免冲突,确保每个设备都能正常工作。

内存地址分配通过基地址寄存器(BAR)实现,系统为每个 BAR 分配一个连续的内存区域。I/O 地址分配类似,但使用 I/O 空间。

中断分配需要考虑中断路由,确保中断能够正确传递到处理器。

\section{中断机制}

\subsection{传统中断}

传统中断(INTx)使用 4 个中断线(INTA#、INTB#、INTC#、INTD#),通过共享中断线传递中断信号。传统中断是电平触发或边沿触发。

传统中断需要中断控制器(如 APIC)的支持,中断线可以多个设备共享。传统中断的延迟较高,效率较低。

传统中断主要用于向后兼容,现代设备通常使用 MSI 或 MSI-X。

\subsection{MSI}

MSI(Message Signaled Interrupts)是 PCI-E 引入的新中断机制,通过内存写事务发送中断信号。MSI 不需要专用的中断线,减少了硬件复杂度。

MSI 支持最多 32 个中断向量,每个中断向量对应一个中断消息。MSI 中断直接发送到处理器的本地 APIC,延迟更低。

MSI 还支持中断合并,减少中断频率,提高系统性能。

\subsection{MSI-X}

MSI-X 是 MSI 的扩展版本,支持最多 2048 个中断向量。MSI-X 使用单独的中断表,每个中断向量都有独立的配置。

MSI-X 适用于需要大量中断的设备,如高端网卡、存储控制器等。MSI-X 提供了更灵活的中断分配和更好的性能。

MSI-X 还支持中断亲和性,可以将中断分配到特定的处理器核心,优化系统性能。

\section{错误处理}

\subsection{错误类型}

PCI-E 定义了多种错误类型,包括物理层错误、数据链路层错误、事务层错误等。错误类型包括信号错误、CRC 错误、超时错误、协议错误等。

物理层错误包括信号完整性问题、时钟恢复失败等。数据链路层错误包括 CRC 校验失败、重传超时等。事务层错误包括非法请求、超时等。

错误检测和报告机制确保系统能够及时发现和处理错误。

\subsection{错误报告}

PCI-E 设备检测到错误后,需要向系统报告错误。错误报告可以通过多种机制实现,包括中断、错误消息寄存器等。

错误报告包括错误类型、错误位置、严重程度等信息。系统根据错误信息决定处理策略,如重试、复位、禁用设备等。

错误报告机制对于系统可靠性和可维护性至关重要。

\subsection{错误恢复}

错误恢复是指系统在检测到错误后采取的恢复措施。恢复措施包括重试、链路复位、设备复位、热复位等。

对于可恢复的错误,系统可以尝试重试操作。对于严重错误,可能需要复位链路或设备。最严重的情况下,系统可能需要禁用设备。

错误恢复机制确保系统在错误发生时能够继续运行或优雅降级。

\section{应用场景}

\subsection{显卡连接}

显卡是 PCI-E 的主要应用之一,几乎所有现代显卡都使用 PCI-E 接口。显卡通常使用 x16 通道,提供最大的带宽。

高端显卡需要大量带宽来传输图形数据,PCI-E 4.0/5.0 x16 提供的 31.5-63 GB/s 带宽能够满足需求。显卡还通过 PCI-E 接口获取电源(最高 75W)。

PCI-E 还支持多显卡配置,如 NVIDIA SLI 和 AMD CrossFire,通过多个 PCI-E 插槽连接多张显卡。

\subsection{网卡连接}

网络接口卡(NIC)是 PCI-E 的另一个重要应用。网卡可以使用 x1、x4 或 x8 通道,具体取决于网络速度。

千兆网卡通常使用 x1 通道,10GbE 网卡使用 x4 通道,25GbE/40GbE 网卡使用 x8 通道。高速网卡需要大量带宽来处理网络流量。

PCI-E 网卡还支持 SR-IOV(Single Root I/O Virtualization),允许虚拟机直接访问网卡,提高虚拟化性能。

\subsection{存储设备}

PCI-E 存储设备包括 NVMe SSD、RAID 控制器等。NVMe SSD 直接连接到 PCI-E 总线,提供极高的性能。

NVMe SSD 通常使用 x4 通道,PCI-E 4.0 x4 提供 7.88 GB/s 带宽,远超 SATA SSD 的 600 MB/s。高端 NVMe SSD 使用 PCI-E 5.0,提供更高带宽。

PCI-E 存储控制器还支持多种 RAID 配置,提供数据保护和性能优化。

\subsection{扩展卡}

PCI-E 扩展卡包括声卡、USB 控制器、SATA 控制器、IEEE 1394 控制器等。这些扩展卡使用不同规格的 PCI-E 插槽。

声卡通常使用 x1 通道,USB 控制器使用 x1 或 x4 通道,SATA 控制器使用 x4 通道。扩展卡通过 PCI-E 接口提供额外的功能和性能。

PCI-E 扩展卡还支持热插拔,方便系统维护和升级。

\section{兼容性}

\subsection{向下兼容}

PCI-E 具有良好的向下兼容性,新版本的设备可以在旧版本的插槽中使用,但性能会降低到旧版本的规格。

例如,PCI-E 4.0 设备插入 PCI-E 3.0 插槽,设备会以 PCI-E 3.0 的速度运行。反之,旧版本设备插入新版本插槽,也会以旧版本的速度运行。

向下兼容确保了系统的平滑升级,用户可以逐步升级硬件,而不需要一次性更换所有设备。

\subsection{版本混用}

PCI-E 支持不同版本的设备在同一系统中混用。系统会自动协商每个链路的最佳速度,确保所有设备都能正常工作。

版本混用可能导致性能瓶颈,因为系统会以最低版本的设备速度运行。例如,PCI-E 4.0 和 PCI-E 3.0 设备混用,链路会以 PCI-E 3.0 速度运行。

版本混用需要考虑系统设计和性能需求,合理规划设备配置。

\subsection{速度协商}

PCI-E 链路在初始化时会进行速度协商,确定最佳的传输速度。速度协商通过物理层的训练和均衡过程实现。

速度协商考虑链路长度、信号质量、设备能力等因素。协商结果可能是最佳速度,也可能是较低速度,以确保链路稳定。

速度协商是自动的,用户无需干预。系统会自动选择最佳配置,平衡性能和稳定性。

\section{未来发展}

\subsection{新版本特性}

PCI-E 标准持续演进,未来版本将引入更多新特性。可能的特性包括更高的传输速率、更低的功耗、更强的信号完整性等。

PCI-E 7.0 可能在 2025 年左右发布,传输速率可能达到 128 GT/s,带宽达到 256 GB/s(x16)。

新版本还将改进电源管理、错误处理、虚拟化等功能,提高系统性能和可靠性。

\subsection{性能提升}

未来的 PCI-E 版本将继续提升性能,满足日益增长的带宽需求。性能提升包括更高的传输速率、更高效的编码、更低的延迟等。

性能提升将推动新一代应用的发展,包括人工智能、机器学习、高性能计算、数据中心等。

性能提升还需要考虑功耗和成本,确保技术能够广泛应用。

\subsection{应用扩展}

PCI-E 的应用领域将不断扩展,包括新兴的计算架构、存储技术、网络技术等。新的应用将推动 PCI-E 标准的演进。

可能的扩展应用包括:CXL(Compute Express Link)、NVMe over Fabrics、GPU 直接互连、AI 加速器等。

应用扩展将带来新的挑战和机遇,需要持续创新和协作。