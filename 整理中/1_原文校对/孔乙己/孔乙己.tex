% 孔乙己
% 孔乙己.tex

\documentclass[12pt,UTF8]{ctexbook}

% 设置纸张信息。
\usepackage[a4paper,twoside]{geometry}
\geometry{
	left=25mm,
	right=25mm,
	bottom=25.4mm,
	bindingoffset=10mm
}

% 设置字体,并解决显示难检字问题。
\xeCJKsetup{AutoFallBack=true}
\setCJKmainfont{SimSun}[BoldFont=SimHei, ItalicFont=KaiTi, FallBack=SimSun-ExtB]

% 目录 chapter 级别加点(.)。
\usepackage{titletoc}
\titlecontents{chapter}[0pt]{\vspace{3mm}\bf\addvspace{2pt}\filright}{\contentspush{\thecontentslabel\hspace{0.8em}}}{}{\titlerule*[8pt]{.}\contentspage}

% 设置 part 和 chapter 标题格式。
\ctexset{
	part/name= {第,卷},
	part/number={\chinese{part}},
	chapter/name={第,篇},
	chapter/number={\chinese{chapter}}
}

% 设置署名格式。
\newenvironment{shuming}{\hfill}

% 注脚每页重新编号,避免编号过大。
\usepackage[perpage]{footmisc}

\title{\heiti\zihao{0} 孔乙己}
\author{鲁迅}
\date{}

\begin{document}

\maketitle
\tableofcontents

\frontmatter
\chapter{前言、序言}

\mainmatter

鲁镇的酒店的格局,是和别处不同的:都是当街一个曲尺形的大柜台,柜里面预备着热水,可以随时温酒。做工的人,傍午傍晚散了工,每每花四文铜钱,买一碗酒,——这是二十多年前的事,现在每碗要涨到十文,——靠柜外站着,热热的喝了休息;倘肯多花一文,便可以买一碟盐煮笋,或者茴香豆,做下酒物了,如果出到十几文,那就能买一样荤菜,但这些顾客,多是短衣帮\footnote{旧指短打衣着的劳动人民。},大抵没有这样阔绰\footnote{阔气。chuò}。只有穿长衫的,才踱进店面隔壁的房子里,要酒要菜,慢慢地坐喝。

我从十二岁起,便在镇口的咸亨酒店里当伙计,掌柜说,样子太傻,怕侍候不了长衫主顾,就在外面做点事罢。外面的短衣主顾,虽然容易说话,但唠唠叨叨缠夹不清的也很不少。他们往往要亲眼看着黄酒从坛子里舀出,看过壶子底里有水没有,又亲看将壶子放在热水里,然后放心:在这严重监督之下,羼\footnote{chàn,混合,掺杂。}水也很为难。所以过了几天,掌柜又说我干不了这事。幸亏荐头\footnote{旧社会以介绍佣工为业的人,也泛指介绍职业的人。}的情面大,辞退不得,便改为专管温酒的一种无聊职务了。

我从此便整天的站在柜台里,专管我的职务。虽然没有什么失职,但总觉得有些单调,有些无聊。掌柜是一副凶脸孔,主顾也没有好声气\footnote{这里指态度。},教人活泼不得;只有孔乙己到店,才可以笑几声,所以至今还记得。

孔乙己是站着喝酒而穿长衫的唯一的人。他身材很高大;青白脸色,皱纹间时常夹些伤痕;一部乱蓬蓬的花白的胡子。穿的虽然是长衫,可是又脏又破,似乎十多年没有补,也没有洗。他对人说话,总是满口之乎者也\footnote{意思是满口文言词语。这里用来表现孔乙己的书呆子气。},教人半懂不懂的。因为他姓孔,别人便从描红纸\footnote{}上的“上大人孔乙己\footnote{旧时通行的描红纸(描红纸:一种印有红色楷字,供儿童摹写毛笔字用的字帖。旧时最通行的一种,印有“上大人孔(明代以前作丘)乙己化三千七十士尔小生八九子佳作仁可知礼也”这样一些笔划简单、三字一句和似通非通的文字。),印有“上大人孔乙己”这样一些包含各种笔画而又比较简单的字,三字一句。}”这半懂不懂的话里,替他取下一个绰号,叫作孔乙己。孔乙己一到店,所有喝酒的人便都看着他笑,有的叫道,“孔乙己,你脸上又添上新伤疤了!”他不回答,对柜里说,“温两碗酒,要一碟茴香豆。”便排出九文大钱。他们又故意的高声嚷道,“你一定又偷了人家的东西了!”孔乙己睁大眼睛说,“你怎么这样凭空污人清白……”“什么清白?我前天亲眼见你偷了何家的书,吊着打。”孔乙己便涨红了脸,额上的青筋条条绽出,争辩道,“窃书不能算偷……窃书!……读书人的事,能算偷么?”接连便是难懂的话,什么“君子固穷\footnote{语见《论语·卫灵公》。“固穷”即“固守其穷”,不以穷困而改变操守的意思。固,安守。}”,什么“者乎”之类,引得众人都哄笑起来:店内外充满了快活的空气。

听人家背地里谈论,孔乙己原来也读过书,但终于没有进学\footnote{明清科举制度,童生经过县考初试,府考复试,再参加由学政主持的院考(道考),考取的列名府、县学籍,叫进学,也就成了秀才。又规定每三年举行一次乡试(省一级考试),由秀才或监生应考,取中的就是举人。},又不会营生\footnote{谋生,筹划如何生活。};于是愈过愈穷,弄到将要讨饭了。幸而写得一笔好字,便替人家钞\footnote{现写作“抄”。}钞书,换一碗饭吃。可惜他又有一样坏脾气,便是好喝懒做。坐不到几天,便连人和书籍纸张笔砚,一齐失踪。如是几次,叫他钞书的人也没有了。孔乙己没有法,便免不了偶然做些偷窃的事。但他在我们店里,品行却比别人都好,就是从不拖欠;虽然间或没有现钱,暂时记在粉板上,但不出一月,定然还清,从粉板上拭去了孔乙己的名字。

孔乙己喝过半碗酒,涨红的脸色渐渐复了原,旁人便又问道,“孔乙己,你当真认识字么?”孔乙己看着问他的人,显出不屑置辩的神气。他们便接着说道,“你怎的连半个秀才也捞不到呢?”孔乙己立刻显出颓唐不安模样,脸上笼上了一层灰色,嘴里说些话;这回可是全是之乎者也之类,一些不懂了。在这时候,众人也都哄笑起来:店内外充满了快活的空气。

在这些时候,我可以附和着笑,掌柜是决不责备的。而且掌柜见了孔乙己,也每每这样问他,引人发笑。孔乙己自己知道不能和他们谈天,便只好向孩子说话。有一回对我说道,“你读过书么?”我略略点一点头。他说,“读过书,……我便考你一考。茴香豆的茴字,怎样写的?”我想,讨饭一样的人,也配考我么?便回过脸去,不再理会。孔乙己等了许久,很恳切的说道,“不能写罢?……我教给你,记着!这些字应该记着。将来做掌柜的时候,写账要用。”我暗想我和掌柜的等级还很远呢,而且我们掌柜也从不将茴香豆上账;又好笑,又不耐烦,懒懒的答他道,“谁要你教,不是草头底下一个来回的回字么?”孔乙己显出极高兴的样子,将两个指头的长指甲敲着柜台,点头说,“对呀对呀!……回字有四样写法\footnote{“回”字过去一般只有三种写法:“回”“囘”“囬”,极少有人用第四种写法“𡇌”。孔乙己这种深受科举教育毒害的读书人,常会注意一些没有用的字,而且把这看成学问和本领。},你知道么?”我愈不耐烦了,努着嘴走远。孔乙己刚用指甲蘸了酒,想在柜上写字,见我毫不热心,便又叹一口气,显出极惋惜的样子。

有几回,邻居孩子听得笑声,也赶热闹,围住了孔乙己。他便给他们茴香豆吃,一人一颗。孩子吃完豆,仍然不散,眼睛都望着碟子。孔乙己着了慌,伸开五指将碟子罩住,弯腰下去说道,“不多了,我已经不多了。”直起身又看一看豆,自己摇头说,“不多不多!多乎哉?不多也\footnote{语见《论语·子罕》:“大宰问于子贡曰:‘夫子圣者与?何其多能也!’子贡曰:‘固天纵之将圣,又多能也。’子闻之,曰:‘大宰知我乎?吾少也贱,故多能鄙事。’君子多乎哉?不多也。”这里与原意无关。}。”于是这一群孩子都在笑声里走散了。

孔乙己是这样的使人快活,可是没有他,别人也便这么过。

有一天,大约是中秋前的两三天,掌柜正在慢慢的结账,取下粉板,忽然说,“孔乙己长久没有来了。还欠十九个钱呢!”我才也觉得他的确长久没有来了。一个喝酒的人说道,“他怎么会来?……他打折了腿了。”掌柜说,“哦!”“他总仍旧是偷。这一回,是自己发昏,竟偷到丁举人家里去了。他家的东西,偷得的么?”“后来怎么样?”“怎么样?先写服辩\footnote{又作“伏辩”,即认罪书。这里指不经官府而自行了案认罪的书状。},后来是打,打了大半夜,再打折了腿。”“后来呢?”“后来打折了腿了。”“打折了怎样呢?”“怎样?……谁晓得?许是死了。”掌柜也不再问,仍然慢慢的算他的账。

中秋过后,秋风是一天凉比一天,看看将近初冬;我整天的靠着火,也须穿上棉袄了。一天的下半天,没有一个顾客,我正合了眼坐着。忽然间听得一个声音,“温一碗酒。”这声音虽然极低,却很耳熟。看时又全没有人。站起来向外一望,那孔乙己便在柜台下对了门槛坐着。他脸上黑而且瘦,已经不成样子;穿一件破夹袄,盘着两腿,下面垫一个蒲包,用草绳在肩上挂住;见了我,又说道,“温一碗酒。”掌柜也伸出头去,一面说,“孔乙己么?你还欠十九个钱呢!”孔乙己很颓唐的仰面答道,“这……下回还清罢。这一回是现钱,酒要好。”掌柜仍然同平常一样,笑着对他说,“孔乙己,你又偷了东西了!”但他这回却不十分分辩,单说了一句“不要取笑!”“取笑?要是不偷,怎么会打断腿?”孔乙己低声说道,“跌断,跌,跌……”他的眼色,很像恳求掌柜,不要再提。此时已经聚集了几个人,便和掌柜都笑了。我温了酒,端出去,放在门槛上。他从破衣袋里摸出四文大钱,放在我手里,见他满手是泥,原来他便用这手走来的。不一会,他喝完酒,便又在旁人的说笑声中,坐着用这手慢慢走去了。

自此以后,又长久没有看见孔乙己。到了年关\footnote{年底。旧社会年底结账时,债主要向欠债的人索债,欠债的人过年如同过关,所以叫“年关”。下文的端午和中秋,在旧社会里也是结账的期限。},掌柜取下粉板说,“孔乙己还欠十九个钱呢!”到第二年的端午,又说“孔乙己还欠十九个钱呢!”到中秋可是没有说,再到年关也没有看见他。

我到现在终于没有见——大约孔乙己的确死了。

\begin{shuming}
一九一九年三月。\footnote{本篇最初发表于1919年4月《新青年》第六卷第四号。本文发表时篇末有作者的《附记》如下:“这一篇很拙的小说,还是去年冬天做成的。那时的意思,单在描写社会上的一种生活,请读者看看,并没有别的深意。但用活字排印了发表,却已在这时候,——便是忽然有人用了小说盛行人身攻击的时候。大抵著者走入暗路,每每能引读者的思想跟他堕落:以为小说是一种泼秽水的器具,里面糟蹋的是谁。这实在是一件极可叹可怜的事。所以我在此声明,免得发生猜度,害了读者的人格。一九一九年三月二十六日记。”}
\end{shuming}

\backmatter

\end{document}