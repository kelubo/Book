% 只是为了好玩
% 只是为了好玩.tex

\documentclass[12pt,UTF8]{ctexbook}

% 设置纸张信息。
\usepackage[a4paper,twoside]{geometry}
\geometry{
	left=25mm,
	right=25mm,
	bottom=25.4mm,
	bindingoffset=10mm
}

% 设置字体,并解决显示难检字问题。
\xeCJKsetup{AutoFallBack=true}
\setCJKmainfont{SimSun}[BoldFont=SimHei, ItalicFont=KaiTi, FallBack=SimSun-ExtB]

% 目录 chapter 级别加点(.)。
\usepackage{titletoc}
\titlecontents{chapter}[0pt]{\vspace{3mm}\bf\addvspace{2pt}\filright}{\contentspush{\thecontentslabel\hspace{0.8em}}}{}{\titlerule*[8pt]{.}\contentspage}

% 设置 part 和 chapter 标题格式。
\ctexset{
	part/name= {第,卷},
	part/number={\chinese{part}},
	chapter/name={第,篇},
	chapter/number={\chinese{chapter}}
}

% 图片相关设置。
\usepackage{graphicx}
\graphicspath{{Images/}}

% 设置署名格式。
\newenvironment{shuming}{\hfill\zihao{4}}

% 注脚每页重新编号,避免编号过大。
\usepackage[perpage]{footmisc}

\title{\heiti\zihao{0} 只是为了好玩}
\author{Linus Torvalds,David Diamond}
\date{}

\begin{document}

\maketitle
\tableofcontents

\frontmatter

\chapter{前言:生活的意义之一(性、战争、Linux)}

背景:这本书开始写于一辆黑色的福特牌汽车上。

这辆车沿着州际五号公路朝南开去,行驶在加利福尼亚中部山谷的某个地方。林纳斯·托瓦兹、他的妻子朵芙·托瓦兹、他们的两个小女儿帕特里夏和丹妮亚拉,在一个外人的陪同下,旅行 351 英里去洛杉矶参观那里的动物园和一家宜家家居商店的分店。

大卫:我现在正在思索一个最基本的问题,而且非常重要。你在这部书里想表达什么?

林纳斯:我想解释生命的意义。

朵芙:林纳斯,你没有忘记给车子加油吧?

林纳斯:我对生命的意义有种理论。我们可以在第一章里对人们解释生命的意义何在。这样可以吸引住他们。一旦他们被吸引住,并且付钱买了书,剩下的章节里我们就可以胡扯了。

大卫:哦,是的。这倒像个计划。有人对我说,人类自从诞生起就一直有两个悬而未决的问题。第一个是:“生命的意义何在?”第二个是:“在一天结束时,我攒下的所有零花钱能干些什么?”

林纳斯:我有对第一个问题的回答。

大卫:答案是什么?

林纳斯:这个答案基本上简单而漂亮。它不会给你的生活以任何意义,但可以告诉你将发生什么。有三件事具有生命的意义。它们是你生活当中所有事情的动机,包括你所做的任何事情和一个生命体该做的所有事情。第一是生存,第二是社会秩序,第三是娱乐。生活中所有的事情都是按这个顺序发展的。娱乐之后便一无所的。因此从某种意义上说,这意味着生活的意义就是要达到第三个阶段。你一旦达到了第三个阶段,就算成功了。但首先要越过前两个阶段。

大卫:你需要详细解释一下。

帕特里夏:爸爸,我们能不能停车买个巧克力冰淇淋?我现在想吃冰淇淋。

朵芙:小宝贝,不行。你得等一等,等我们停下来去上厕所时你才可以吃冰淇淋。

林纳斯:我给你举几个例子来说明这一观点。最明显的是性,它开始只是一种延续生命的手段,后来变成了一种社会性的行为,比如你要结婚才能得到性。再后来,它成了一种娱乐。

帕特里夏:这么说我得上厕所了?

大卫:性为什么是娱乐?

林纳斯:好吧,我是在对牛弹琴。我举一个别的例子。

大卫:不必,还是说说性。

林纳斯:它是在另一个层次上的……

大卫(自言自语):哦,参与就是娱乐,而不是在一旁观看。好,我明白了。

林纳斯:……假如你从生物学的角度观察性行为,就是在另一个层次上。性一开始是怎么来的?是生存。最初它不是娱乐,后来两者融合在一起了。好,我们先把性放下。

大卫:别,我想这可以写整整一个章节。

林纳斯:我们还是来谈谈战争吧。很明显,它最初起源于生存,因为在你和水源之间有一个高个子家伙挡道。再后来,你必须和那个家伙为争夺一个妻子而搏斗。之后,就成了一种社会秩序。中世纪之前很长一段时间里战争就是这个样子。

大卫:战争是建立社会秩序的手段。

林纳斯:没错。但它也是把人自己塑造成社会秩序中一部分的手段。

大卫:如今战争已成为娱乐?

林纳斯:对。

大卫:也许那些在电视上观看战争节目的人,会觉得战争非常有意思。

林纳斯:电脑游戏。战争游戏。有线电视新闻网。战争的原因也常常很有意思。对战争的看法也是娱乐。对性的需求其原因也常常是娱乐。当然,生存的功能没有泯灭,尤其是当你是一个天主教徒的时候,对不对?但即使你是一个天主教徒,有时你想到性的时候也包含享乐的需要。所以这并非只是纯粹的娱乐。在所有的事物中,一部分动机可能是生存,另一部分可能是社会秩序,剩下的就是娱乐。好,我说说技术。技术最初也是生存。生存的意思并非只是生存而已,而是为了生存得更好。为了从井里打水人类才发明了风车……

大卫:火的发明也是这样。

林纳斯:对。这仍是为了生存,还没有达到社会秩序和娱乐的阶段。

大卫:技术是怎样进入社会秩序阶段的?

林纳斯:其实工业化的目的主要是生存需要,或者为了生存得更好。比如生产汽车,它便意味着制造出更快更漂亮的汽车。然后技术便达到了社会层面的阶段。这时我们有了电话。在某种程度上,也有了电视。早期许多电视节目主要是为了给大众洗脑。电台也是如此,许多国家常常投资电台,为的就是社会秩序的原因。

大卫:建立和维持社会秩序……

林纳斯:没错,然后又超越了那个阶段。今天,电视显然主要是为了娱乐。如今你才可以到处看到手机,手机现在大体上还处于社会的层面,但也正在朝娱乐的阶段发展。

大卫:那么技术的未来将会怎样?我们已经超越了生存阶段,现在处于社会阶段,是这样吗?

林纳斯:对。过去所有技术都是为了使生活更容易一些。是为了快点达到一个地点,货品更便宜一些,住进更好的房子等等。那么信息技术与过去的技术有何区别呢?人人都建立起联系后又会怎样呢,还有什么可做的?当然人们之间的联系可以建立得很好,但从根本上

说与过去没什么区别。因此技术将把我们引向何处?依我看,下一个巨大的步伐就是迈向娱乐。
大卫:你的意思是,一切的一切最终都将演变成娱乐……
林纳斯:这就是为什么Linux在某种程度上如此成功的原因。想一下那三个动机。第一个是生存,拥有计算机的人把这个视为理所当然。坦白地说,假如你有一台电脑,就意味着你已经不会再为基本的生计问题发愁了。第二个是社会秩序,建立社会秩序的动机显然是为了使各种各样的人能够各安其位。
大卫:你在Comdex电脑业会议上说的话非常不错,当时你说Linux的开发是一个全球性团队的体育项目。这一点基本上是由你创立的,伙计。
林纳斯:Linux表明了人们为什么喜欢团队体育项目,尤其是想成为团队中的一员。
大卫:是的,每天坐在电脑前,你大概希望觉得你是一个大团体中的一分子,任何事情中的一分子。
林纳斯:这就是社会层面,和其他团队体育项目一样。想象一下一支足球队里面的人,特别是高中的足球队。Linux的社会层面是非常非常重要的。但 Linux也是娱乐,这种娱乐是金钱很难买到的。当你处在生存阶段时,金钱是一个非常强大的动机,因为用金钱换取生存是件容易的事情。换取生存之类的东西是很容易的,但突然之间你进入了娱乐阶段,金钱就……

大卫:金钱就没用了?
林纳斯:不是,并不是没用,因为显然你可以用钱买电影影碟、速度更快的汽车、更豪华的假期。还有许多东西你也可以买,从而改善你的处境。
朵芙:林纳斯,我们该给丹妮亚拉换尿布了。帕特里夏也得去上厕所。我想喝一杯卡普契诺咖啡。我们在这儿能找到一家星巴克(Starbucks)咖啡馆吗?我们现在在哪儿?
大卫(抬起头):根据空中的味道,我想我们已经快到国王城了。
林纳斯:我们所说的都非常宏观,但我们指的不光是人,而且是生活。和熵的定律相似。根据“生活的熵定律”,一切事物都将从生存走向娱乐,但这并不意味着在某个局部地区没有倒退的现象,而且毫无疑问许多地方都有这种情况。有时事物往往会分裂开来。
大卫:但作为一个体系,一切事物都朝着一个方向发展……
林纳斯:一切事物都朝着一个方向发展,但并不是同步的。所以从根本上说,性已经达到了娱乐阶段,战争已经快接近娱乐阶段了,技术也已经达到了这一阶段。新生的事物开始是为了生存,比如太空旅行,它在某个阶段是为了生存,然后成为一种社会秩序,最后达到娱乐的目的。可以用膜拜的角度看一下文明。我是说,文明也是以同样的格局出现的。文明最早是为了生存,比如大家聚在一起就会生存得更好,于是建立起了社会结构。最后,文明的存在是为了纯粹的娱乐,当然也并非完全纯粹,而且这种娱乐也并非有什么不好。古希腊

人是以强大的社会秩序而著称的,他们也有不少娱乐。人人都知道那个时代产生了不少一流的哲学家。
大卫:那么这些和生活的意义有什么联系呢?
林纳斯:并非有何联系……只是说……存在着这类问题。
大卫:这其间的小小联系你还得想一想。
帕特里夏:妈妈,瞧那些牛。
林纳斯:所以,要是你知道生活的走向是这样的,那么毫无疑问,你的生活目标就中促成这一走向。而且这种走向并非是一个单一的行程。你做的一切都是许多走向中的一部分。你也可以问自己:“我做些什么才能使社会变得更好?”你知道你是这个社会的一部分。你知道社会正在朝这个方向发展,你也能帮助它朝这个方向走。
朵芙(扬起鼻子):什么味?真难闻。
林纳斯:所以归根结底,我们都是为了开心。我们也可以坐在这里,完全放松,享受着汽车旅行。
大卫:仅仅为了开心(Just for fun)?

\mainmatter


第一章 一个书呆子的诞生
1、大鼻子的孩子
2、外公的计算机
3、芬兰的严冬
4、我的家族
5、中学时代
6、长大成人
7、爱洗桑拿的国家
第二章 一种操作系统的诞生
1、昨天的电脑
2、上大学
3、从UNIX开始
4、第一台386和终端仿真
第三章 编程的美妙
1、开始编程
2、长腿的终端仿真器
3、Linux 的诞生
4、开放源代码
5、Linux能换来金钱吗?
6、MINIX对Linux
7、最后的冲刺
8、朵芙
第四章 舞会上的国王
1、1.0版本闪亮登场
2、版权之争
3、接受全美达的邀请
4、欢迎来到硅谷
5、一夜功成名就
6、Linux 呈蔓延之势
7、财富的到来
8、糟糕的展示会
9、媒体的攻击
10、道德不应制度化
11、舞会上的国王
12、还会再干
第五章 知识产权
1、各种观点
2、结束控制
3、未来的娱乐之旅
4、为何开放源代码
5、名声与财富
6、生活的意义之二

\backmatter

\end{document}