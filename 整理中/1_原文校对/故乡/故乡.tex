% 故乡
% 故乡.tex

\documentclass[12pt,UTF8]{ctexbook}

% 设置纸张信息。
\usepackage[a4paper,twoside]{geometry}
\geometry{
	left=25mm,
	right=25mm,
	bottom=25.4mm,
	bindingoffset=10mm
}

% 设置字体,并解决显示难检字问题。
\xeCJKsetup{AutoFallBack=true}
\setCJKmainfont{SimSun}[BoldFont=SimHei, ItalicFont=KaiTi, FallBack=SimSun-ExtB]

% 目录 chapter 级别加点(.)。
\usepackage{titletoc}
\titlecontents{chapter}[0pt]{\vspace{3mm}\bf\addvspace{2pt}\filright}{\contentspush{\thecontentslabel\hspace{0.8em}}}{}{\titlerule*[8pt]{.}\contentspage}

% 设置 part 和 chapter 标题格式。
\ctexset{
	part/name= {第,卷},
	part/number={\chinese{part}},
	chapter/name={第,篇},
	chapter/number={\chinese{chapter}}
}

% 设置古文原文格式。
\newenvironment{yuanwen}{\bfseries\zihao{4}}

% 设置署名格式。
\newenvironment{shuming}{\hfill\bfseries\zihao{4}}

% 注脚每页重新编号,避免编号过大。
\usepackage[perpage]{footmisc}

\title{\heiti\zihao{0} 故乡}
\author{鲁迅}
\date{}

\begin{document}

\maketitle
\tableofcontents

\frontmatter

我冒了严寒,回到相隔二千余里,别了二十余年的故乡去。

时候既然是深冬;渐近故乡时,天气又阴晦了,冷风吹进船舱中,呜呜的响,从篷隙向外一望,苍黄的天底下,远近横着几个萧索的荒村,没有一些活气。我的心禁不住悲凉起来了。

阿!这不是我二十年来时记得的故乡?

我所记得的故乡全不如此。我的故乡好得多了。但要我记起他的美丽,说出他的佳处来,却又没有影像,没有言辞了。仿佛也就如此。于是我自己解释说:故乡本也如此,——虽然没有进步,也未必有如我所感的悲凉,这只是我自己心情的改变罢了,因为我这次回乡,本没有什么好心绪。

我这次是专为了别他而来的。我们多年聚族而居的老屋,已经公同卖给别姓了,交屋的期限,只在本年,所以必须赶在正月初一以前,永别了熟识的老屋,而且远离了熟识的故乡,搬家到我在谋食的异地去。
=================================
第二日清早晨我到了我家的门口了。瓦楞上许多枯草的断茎当风抖着,正在说明这老屋难免易主的原因。几房的本家大约已经搬走了,所以很寂静。我到了自家的房外,我的母亲早已迎着出来了,接着便飞出了八岁的侄儿宏儿。

我的母亲很高兴,但也藏着许多凄凉的神情,教我坐下,歇息,喝茶,且不谈搬家的事。宏儿没有见过我,远远的对面站着只是看。

这时候,我的脑里忽然闪出一幅神异的图画来:深蓝的天空中挂着一轮金黄的圆月,下面是海边的沙地,都种着一望无际的碧绿的西瓜,其间有一个十一二岁的少年,手捏一柄钢叉,向一匹猹尽力的刺去,那猹却将身一扭,反从他的胯下逃走了。

这少年便是闰土。我认识他时,也不过十多岁,离现在将有三十年了;那时我的父亲还在世,家景也好,我正是一个少爷。那一年,这祭祀忙不过来,他便对父亲说,可以叫他的儿子闰土来管祭器的。

我的父亲允许了;我也很高兴,因为我早听到闰土这名字,而且知道他和我仿佛年纪,闰月生的,五行缺土,所以他的父亲叫他闰土。他是能装弶捉小鸟雀的。

我于是日日盼望新年,新年到,闰土也就到了。好容易到了年末,有一日,母亲告诉我,闰土来了,我便飞跑的去看。他正在厨房里,紫色的圆脸,头戴一顶小毡帽,颈上套一个明晃晃的银项圈。

第二日,我便要他捕鸟。他说:须大雪下了才好。我于是又很盼望下雪。闰土又对我说:“现在太冷,你夏天到我们这里来。我们日里到海边检贝壳去,红的绿的都有,鬼见怕也有,观音手也有。晚上我和爹管西瓜去,你也去。”

我素不知道天下有这许多新鲜事:海边有如许五色的贝壳;西瓜有这样危险的经历,我先前单知道他在水果店里出卖罢了。

阿!闰土的心里有无穷无尽的稀奇的事,都是我往常的朋友所不知道的。他们不知道一些事,闰土在海边时,他们都和我一样只看见院子里高墙上的四角的天空。

可惜正月过去了,闰土须回家里去,我急得大哭,他也躲到厨房里,哭着不肯出门,但终于被他父亲带走了。他后来还托他的父亲带给我一包贝壳和几支很好看的鸟毛,我也曾送他一两次东西,但从此没有再见面。

现在我的母亲提起了他,我这儿时的记忆,忽而全都闪电似的苏生过来,似乎看到了我的美丽的故乡了。我应声说:“这好极!他,——怎样?……”

一日是天气很冷的午后,我吃过午饭,坐着喝茶,觉得外面有人进来了,便回头去看。我看时,不由得非常出惊,慌忙站起身,迎着走去。

这来的便是闰土。虽然我一见便知道是闰土,但又不是我这记忆上的闰土了。他身材增加了一倍;先前的紫色的圆脸,已经变作灰黄,而且加上了很深的皱纹;眼睛也像他父亲一样。

我这时很兴奋,但不知道怎么说才好,只是说:“阿!闰土哥,——你来了?……”  我接着便有许多话,想要连珠一般涌出……但又总觉得被什么挡着似的,单在脑里面回旋,吐不出口外去。

我似乎打了一个寒噤;我就知道,我们之间已经隔了一层可悲的厚障壁了。我也说不出话。我问问他的景况。他只是摇头。母亲问他,知道他的家里事务忙,明天便得回去;又没有吃过午饭,便叫他自己到厨下炒饭吃去。

下午,他拣好了几件东西:两条长桌,四个椅子,一副香炉和烛台,一杆抬秤。他又要所有的草灰,待我们启程的时候,他用船来载去。

又过了九日,是我们启程的日期。闰土早晨便到了,我们终日很忙碌,再没有谈天的工夫。来客也不少,有送行的,有拿东西的,有送行兼拿东西的。

我想到希望,忽然害怕起来了。闰土要香炉和烛台的时候,我还暗地里笑他,以为他总是崇拜偶像,什么时候都不忘却。现在我所谓希望,不也是我自己手制的偶像么?只是他的愿望切近,我的愿望茫远罢了。

我在朦胧中,眼前展开一片海边碧绿的沙地来,上面深蓝的天空中挂着一轮金黄的圆月。我想:希望是本无所谓有,无所谓无的。这正如地上的路;其实地上本没有路,走的人多了,也便成了路。

一九二一年一月。

\backmatter

\end{document}