\documentclass{book}
\usepackage{ctex}
\usepackage{graphicx}
\usepackage{hyperref}

\begin{document}


\subsection{第1章 阳痿}

阳痿,即勃起功能障碍,是指阴茎勃起硬度不足以进入阴道或不能维持其硬度至射精。中医学称为“阳痿”或“阳事不举”“阴器不用”“举而不坚”。主要分为阴茎前型、阴茎型。

阴茎前型是指具有正常解剖的阴茎,在性兴奋期间,虽然在适当的环境与场所有足够的性刺激,但是阴茎不能勃起或勃起硬度与时间不足以达到完成正常性交的能力。由于精神心理因素导致勃起障碍称为心理性阳痿;全身代谢或局部病变引起的阳痿称为器质性阳痿。器质性勃起功能障碍往往继发于内分泌、血管与神经的病变。性生活从未获得满意勃起和正常性交者称为原发性阳痿;有过正常性生活,以后发生的阳痿为继发性阳痿。在任何场合下,均不能勃起者为完全性阳痿。在某些场合可勃起,而在另外环境下不能勃起者称境遇性阳痿。

阴茎型勃起功能障碍是指阴茎本身解剖异常而不能获得或维护勃起能力,属器质性勃起功能障碍。如尿道严重下裂、阴茎海绵体硬结病、阴茎海绵体纤维化、阴茎外伤、手术与感染以及包茎、淋巴水肿、两性畸形、小阴茎等。

勃起功能障碍的发病率随年龄增长而增高,过去普遍认为85\%~90\%的阳痿是心因性的,但随着医学科学技术的迅速发展,如今已认识到许多器质性因素在阳痿病因诊断中的重要性,全面查找病因是满意治疗的前提。

影响勃起的心理因素由于每个人所处环境、经历、心理状态及性格特点不同,对同样精神与社会因素的心理反应也有所不同。常见的影响勃起功能的精神心理因素是:缺乏性教育或错误的性教育、各种心理创伤、夫妻关系或社会上人际关系不协调、性生活场所不适当等,以及医源性如医生误导或错误的解释。

1.加味龙胆泻肝汤

【药物组成】龙胆草10g,黄芩10g,栀子10g,泽泻8g,木通8g,车前子8g,当归10g,柴胡10g,生地黄10g,蜈蚣2条,甘草3g。

【治疗方法】每日1剂,水煎服,连续服用5剂后进行疗效评定,服药期间忌房事。

【功 效】清热利湿,活血通精。

【临床运用】临床治疗40例。治愈6例,显效15例,好转15例,无效4例,总有效率为90\%。

治验:李某,男,31岁。2005年3月8日初诊,自述阳痿不举,临房力不从心1年余,伴下肢酸软,口干口苦,大便干结,阴囊潮湿,嗜食烟酒厚味,自购多种补肾壮阳药无效。查体:肥胖,舌红苔黄腻,脉滑数,证属肝胆湿热下注,治法:清肝胆湿热;方药:龙胆草10g,栀子10g,泽泻8g,蜈蚣2条,木通8g,车前子8g,当归10g,柴胡10g,生地黄10g,甘草3g。5剂后症状全部消失,随访1年未见复发。

【经验心得】湿热型阳痿在男科门诊中比较常见,尤其以中青年偏多,随着生活水平的提高,青年人追求快乐,追求刺激,经常酗酒,餐餐膏粱厚味,再加手淫助长,一有身体不适,盲目用滋补壮阳药,导致湿热内生下注。《类证治载·阳痿》篇说:“亦有湿热下注,宗筋弛纵而致阳痿者。”龙胆泻肝汤针对本病的主要病机,方中龙胆草、黄芩、栀子苦寒清下焦湿热,泽泻、木通、车前子清热利湿,使湿热从水道排除,生地黄、当归滋阴养血,使苦寒药物不伤耗阴血,柴胡引药入肝胆,甘草调和诸药,蜈蚣通络,使湿热阻滞之经络畅通,精血下注病变部位,综观全方是泻中有补,利中有滋,以使火降热清湿浊分清,循经所发诸症而愈。

【方剂出处】谭万顺.加味龙胆泻肝汤治疗湿热型阳痿病40例.云南中医中药杂志,2008,29(2):62

2.逍遥散加味

【药物组成】柴胡10g,当归10g,白芍10g,茯苓10g,薄荷6g,郁金10g,淫羊藿30g,菟丝子30g,石菖蒲10g。

【随症加减】阳虚甚者加肉桂、仙茅;兼有纳食减少、腹胀便溏者,加砂仁、陈皮;心悸健忘、失眠多梦者加炒酸枣仁、合欢花;心脾两虚者加归脾丸,日服3次;血瘀者酌加桃仁、红花。

【治疗方法】每日1剂,水煎分2次内服,连服10剂为1个疗程。一般1~2个疗程即有好转,多数3~4个疗程痊愈。

【功 效】疏肝解郁,补肾益精。

【临床运用】临床治疗36例。治愈28例,有效5例,无效3例,总有效率92\%。

治验:孙某,男,28岁。平素少言语,阳痿不举2个月,近半个月来症状加重,伴有心烦,夜寐不宁,多梦遗精,精神委靡,胸闷胁胀,四肢疲乏无力。查体:体温37℃,脉搏80/分,呼吸15/分,血压120/80mmHg。头颅、五官(-),颈胸部对称无畸形,心肺(-)。腹软无包块,肝脾不大,四肢、脊柱(-)。外生殖器检查无异常。舌尖红,薄黄苔,脉弦。证属肝郁脾虚,郁而化火兼肾虚精关不固。治以疏肝解郁泻火,兼以益肾。处方:柴胡10g,当归10g,白芍10g,白术10g,茯苓10g,薄荷6g,郁金10g,菖蒲10g,淫羊藿30g,菟丝子30g,炒酸枣仁20g,合欢皮10g,黄柏10g。水煎,分2次内服,每日1剂。自述10剂后阳事已举,继进原方3剂,诸症悉除,至今未发。

【经验心得】阳痿从肾治者认为命门火衰,精气虚寒乃本病主因,故温补肾阳真火,往往选用仙茅、淫羊藿、韭菜子、附桂或血肉有情之品峻补。但本证因情志内伤,夫妻感情不和或劳累思虑过度发病者亦不少见,其病机乃肝气郁结,气机不畅,抑或郁而化火而兼有肾虚兼症,因肝藏血,主疏泄调情志、畅气机、和血脉。又阴阳交和与冲、任关系密切,冲脉总督全身气血,为十二经之要冲,有“经脉之海”“血海”“五脏六腑之海”之说。冲、任不和,男子疝气阳痿,女子带下癥瘕,月经不调。冲、任二脉系在肝肾,肝失疏泄,肾失封藏,病由自生。

本方以逍遥散为主加益肾药物组成。其中逍遥诸味意在疏肝解郁,养血健脾。另有淫羊藿、菟丝子补肾填精益髓;石菖蒲有养心健脑聪耳之效;郁金入肝经,力专行气祛瘀,且能清心除烦。诸药合用重在疏肝解郁,兼能补肾益精,治肝顾肾,每获良效。

【方剂出处】杨俊.疏肝解郁法治疗阳痿36例初步分析.光明中医,2008,23(1):61

3.解郁起痿汤

【药物组成】醋柴胡、郁金、枳壳、合欢皮各10g,制香附、炒白芍、白蒺藜、枸杞子、肉苁蓉、淫羊藿、当归各12g,川芎8g,蜈蚣2条,丹参15g,炙甘草2g。

【随症加减】伴烦躁易怒者,加牡丹皮、栀子清热除烦;伴心悸失眠者,加酸枣仁、远志、茯神养心安神;伴食欲不振者,加茯苓、白术、党参健脾除湿;伴肾阳亏虚者,加肉桂、鹿角胶、巴戟天温阳补肾;伴精亏阴虚者去川芎,加黄精、熟地黄、女贞子养阴生精;伴遗精、早泄者,加芡实、五味子、山茱萸、五倍子固精止泄;伴有瘀血阻滞者,加丹参、广三七、赤芍、红花活血通络。

【治疗方法】上方每日1剂,水煎服,15日为1个疗程,连续治疗1~6个疗程。治疗期间禁烟酒、辛辣刺激、生冷之品。畅情志,忌气恼,并适当配合性感集中训练,心理咨询疏导等心理治疗。

【功 效】疏肝解郁,通络兴阳。

【临床运用】临床治疗120例。治愈66例,有效48例,无效6例,总有效率95\%。

治验:郭某,男,30岁。2005年9月16日初诊。阴茎勃起不坚,同房时不能随意勃起已3年余,近半年来阴茎勃起困难,逐渐无法性生活。伴有情志抑郁,胸胁胀闷,烦躁易怒,善太息,性欲淡漠,舌质黯紫、苔薄白,脉弦。查体:第二性征及阴茎发育正常,阴毛及腋毛分布均匀。睾丸体积大小正常,睾丸及附睾无肿块,无触痛,无精索静脉曲张。提睾反射及球海绵体反射存在。经化验,前列腺液常规及尿常规均正常。前列腺液细菌培养24小时无菌生长。经测定性激素六项均在正常范围。血管活性药物(罂粟碱30mg加酚妥拉明1mg)阴茎海绵体注射试验(+),阴茎彩色多普勒超声检查所见:阴茎动脉血流速度:29cm/s。中医诊断为阳痿,证属肝气郁结。治拟疏肝解郁:醋柴胡、郁金、牡丹皮、枳壳、合欢皮、焦栀子各10g,制香附、炒白术、白芍、白蒺藜、肉苁蓉、当归各12g,淫羊藿、茯苓、丹参各15g,蜈蚣2条,炙甘草3g,并配合心理疏导治疗。服药1个疗程后复诊,患者已能正常勃起,近期已能成功性交。情志抑郁、胸胁胀闷、烦躁易怒等症状均已明显改善,心情畅快,已未见太息。舌质淡红,脉缓有力。继用上方去牡丹皮、栀子30剂而痊愈,随访1年未复发。

【经验心得】肝气条达舒畅,上升适度,气机舒达,血行流畅,才可保持人体血液贮藏、调节的正常功能。宗筋的作用强弱与肝气的正常疏泄及肝血的旺盛充盈有密切关系。肝藏血,主疏泄,又主宗筋。肝血在肝气的疏导下对宗筋的快速充盈是阴茎勃起的物质基础。《内经》曰:“肝者,罢极之本,魂之居也,其华在爪,其充在筋,以生血气……”又曰:“故人卧,血归于肝,肝受血而能视,足受血而能步,掌受血而能握,指受血而能摄……”说明宗筋的正常功能活动是由肝的正常疏泄和肝的藏血功能共同协调完成的。人体七情致病突出表现在直接伤及内脏,影响脏腑气机,以及情绪的波动使病情加重。前阴为宗筋之汇,如果情志不遂,郁怒不伸,均可影响肝的疏泄功能,导致肝气郁结,经络不通,肝血运行失畅,不能灌注宗筋,即而发为阳痿一病。故在临床上由精神心理因素障碍形成的肝气郁结是阳痿发病的一个重要病机,应以疏肝解郁、疏理气机为主要治法。疏肝解郁法可使患者肝气条达,气机畅通,肝血充盈宗筋,故可使阳痿病人恢复正常性功能。自拟解郁起痿汤中柴胡、郁金、枳壳、白芍、合欢皮、香附疏肝解郁、条达情志;白芍、当归、川芎、丹参畅通血脉;白芍、枸杞子、当归、川芎、肉苁蓉滋补肝血,使肝血充盈;白蒺藜、淫羊藿解郁兴阳;蜈蚣通络散结。诸药配伍合用,共奏疏肝解郁、通络兴阳之效。

【方剂出处】杨德放.解郁起痿汤治疗功能性阳痿120例.陕西中医,2008,29(10):1331

4.疏肝兴阳汤

【药物组成】柴胡、香附各10g,当归、白芍、白蒺藜各12g,九香虫6g,蜈蚣2条。

【随症加减】兼血瘀者加川芎、桃仁、红花;兼湿热者加龙胆草、栀子、车前子;兼肾阴虚者加熟地黄、山茱萸、枸杞子;兼肾阳虚者加淫羊藿、菟丝子、巴戟天。

【治疗方法】每日1剂,水煎服,每日2次。

【功 效】疏肝解郁。

【临床运用】临床治疗54例患者,治愈32例,好转14例,无效8例,总有效率85\%。

【经验心得】历代医家认为阳痿与肾、肝、脾三脏功能失调密切相关。当今社会经济繁荣,因脾肾不足而致阳痿的虚证日趋减少;而因社会变革、竞争激烈、社会适应不良、心理障碍致肝气郁结、气血不畅、宗筋失养患阳痿的日渐增多,因此用疏肝解郁法治疗肝气郁结型阳痿多收良效。自拟疏肝兴阳汤,方中柴胡、香附疏肝解郁,调畅气机;当归、白芍养血活血,柔润宗筋;白蒺藜疏肝兴阳;九香虫理气解郁,补肾壮阳;蜈蚣走窜之力最速。诸药共奏理气活血、兴阳振痿之功。辅以心理疏导,消除心理障碍,可收事半功倍之效。

【方剂出处】马祥生.疏肝兴阳汤治疗阳痿54例.四川中医,2001,19(4):33

5.调肝益肾汤

【药物组成】怀牛膝、丹参各30g,淫羊藿15g,当归、赤芍、白芍、青皮、山茱萸、红花、远志、九香虫各10g,龟甲胶7g,柴胡6g,鹿角胶3g,蜈蚣3条。

【随症加减】以调补肝肾、养血活血为基本治疗原则,再根据兼症不同加减应用。如伴疲乏无力、汗出多、睡眠差者加黄芪30g,首乌藤、合欢皮各15g,伴腰膝酸软症状明显者加巴戟天、狗脊各15g,伴湿热内蕴者去鹿角胶、龟甲胶,加黄芩、龙胆草、车前子、栀子、泽泻各10g。

【治疗方法】每日1剂,水煎,早、晚空腹分2次口服。

【功 效】调肝益肾,助阳起痿。

【临床运用】治疗40例,显效32例,好转6例,无效2例。

【经验心得】明代王汝言曰:“少年阳痿,有因失志者,但宜舒郁。”可见古人认识到阳痿之病多以肾为重点,涉及肝、脾、心、胆。在治法上有内治、外治之分。内治占主导地位。在用药上以植物药为主,配合动、矿物药。立法以温补为主。但在临床中发现现代条件下的阳痿,由于生活节奏的加快,男性心理压力增大,其证候学规律也发生了与古代不同的变化,其性质上有阴阳、寒热、虚实之不同。虚实上以实证多而虚证少;阴阳上以阴虚多而阳虚少为普遍规律;寒证、热证二者差异不大。脏腑定位尤与肝最为密切,与脾次之,与肺、心、胆有一定关系。临床上此类患者病久必致情绪郁闷,肝失条达。因此以调肝益肾为主导思想,自拟调肝益肾汤。其中柴胡、当归、赤芍、白芍、青皮以疏肝解郁,条达气机;丹参、红花,养血活血,滋其化源;以鹿之三,龟之七配合牛膝、山茱萸、九香虫、淫羊藿、蜈蚣益肾通络;远志清心安神。诸药合用共奏调肝益肾之功效。对肝郁肾虚之阳痿,切中病机,随症加减,故能起到立竿见影之功效。

【方剂出处】刘茂君.调肝益肾汤治疗阳痿40例.陕西中医,2004,25(8):698

6.淡利通阳方

【药物组成】生薏苡仁、萆薢各15g,泽泻、滑石、车前子、石菖蒲、路路通各10g,通草6g。

【随症加减】湿重于热者加茯苓、猪苓、苍术;热重于湿者加竹叶、芦根;湿热并重者合甘露消毒饮;脾胃湿热者合三仁汤;肝经湿热者合龙胆泻肝汤;膀胱湿热者合八正散;阴虚湿热者合知柏地黄汤;合并慢性前列腺炎者加丹参、赤芍、王不留行或红藤、败酱草;合并尿道炎者加金钱草、马鞭草、紫草、灯心草、萹蓄、瞿麦。

【治疗方法】每日1剂,水煎,早、晚空腹分2次口服。

【功 效】利湿清热,通利阳气。

【临床运用】临床治疗56例,治愈29例,好转21例,未愈6例,总有效率89.29\%。

【经验心得】在临床实践中发现一些肥胖体形及湿热体质之人,或有慢性前列腺炎病史者,使用补肾壮阳药物,疗效不满意,且有加重病情的现象,究其原因,此类患者多形体肥胖,或饮食不节,嗜食辛辣炙煿之品及饮酒,湿蕴化热;或手淫、忍精不泄、残精败精瘀滞精道;或长久憋尿,溢逆精道,酿成湿热;或性事不洁,交感湿热之邪;或湿热之体,反服辛热滋腻之剂,助湿生热。湿热胶结,下注宗筋,壅阻经络,气机不畅,阳气不展,阳痿失用。此属湿热阻滞、阳气不展,非肾虚阳弱之故也,治疗不能温补,而应通阳。拟清利湿热,调畅气机,伸展阳气。叶天士在《外感湿热病篇》中提出“通阳不在温,而在利小便。”病属湿热阻滞,阳气不展,用辛温会助热,用苦寒可伤阳,故宜用甘淡、甘寒或微寒之品,渗湿利小便,湿祛则气道通,气道通则阳气复。处方中生薏苡仁、车前子甘淡微寒,清热利湿,利尿通淋;萆薢苦平,分清化浊,利湿降浊,举清升阳,泽泻甘淡性平,利水渗湿泄热,淡渗利膀胱之水湿,性寒泄肾及膀胱之热,二药配对,泌别清浊,利湿泄热,祛除留着于肌肉、筋脉、经络之湿热;滑石、通草甘淡微寒,淡可渗湿,寒可清热,泄热下行,滑石质滑利窍、通草性善通利,二药配对,清利湿热,滑利水道,通利小便而不伤阴;石菖蒲味辛微温,芳香开窍,化湿行气;路路通辛苦性平,通行十二经脉,辛开苦降,外泄风湿,内泄水湿,调理气机,利水消肿。诸药共用,渗利小便,利湿清热,湿去热孤,阳气自通,宗筋得充,阳痿自愈。

【方剂出处】翟海定,等.淡利通阳方治疗阳痿56例小结.甘肃中医,2005,18(8):25

7.土柏六五汤

【药物组成】土茯苓30g,黄柏15g,生地黄15g,熟地黄15g,山药15g,山茱萸15g,牡丹皮10g,茯苓15g,泽泻10g,韭菜子15g,菟丝子15g,枸杞子15g,覆盆子15g,车前子10g,淫羊藿15g,肉苁蓉15g。

【随症加减】肾阴虚甚,加知母15g,女贞子15g,墨旱莲15g;肾阳虚甚加仙茅15g,鹿角胶10g(烊化);湿热下注甚加萆薢10g,滑石30g,竹叶10g;肝郁不舒甚加柴胡10g,白芍15g,香附10g。

【治疗方法】每日1剂,水煎,分2次服。20日为1个疗程,一般治疗1~3个疗程。

【功 效】滋肾助阳,阴阳双补,清利湿热。

【临床运用】临床治疗140例,治愈84例,好转51例,有效率96.4\%,治愈率60\%。

【经验心得】土柏六五汤中,六味地黄汤滋阴补肾中又清肝渗湿,已是标本兼治,五子衍宗丸益肾助阳,兼以利湿,亦是标本兼治,两方合用,符合“善补阳者,必于阴中求阳;善补阴者,必于阳中求阴”之旨,如此则阴阳俱补。更加用土茯苓、黄柏清热燥湿,合牡丹皮、车前子同清相火,退阴虚发热,肉苁蓉、淫羊藿合韭菜子、菟丝子益肾助阳起痿,温而不燥烈。诸药同用,共奏滋肾助阳,阴阳双补,兼以清利湿热之功。如此则阴阳充,湿热清,标本俱治,阳事自兴。同时,患者尚需清心寡欲,耐心服药,不可滥用“春药”,取快一时,必致一蹶不振。

【方剂出处】邹世光,等.土柏六五汤治阳痿140例疗效观察.辽宁中医杂志,2007,34(6):770

8.龟鹿补肾汤

【药物组成】鹿角胶12g(熔化),龟甲胶12g(熔化),炙黄芪18g,熟地黄20g,淫羊藿9g,益智仁9g(打碎),枸杞子12g,巴戟天15g,肉苁蓉12g,阳起石15g(打碎先煎)。

【随症加减】若肾阳虚损明显者鹿角胶加倍量;兼血虚者加何首乌12~15g,当归12g;气虚者加党参12g,山药15g;腰痛甚者加杜仲12g,菟丝子10g。

【治疗方法】加清水800ml,煎至250~300ml,分2~3次于饭前服。

【功 效】补肾壮阳,填精补髓。

【临床运用】临床治疗95例,治愈70例,好转19例,无效6例。

【经验心得】阳痿是指性交时阴茎不能勃起或勃起不坚,以致不能完成正常性交活动的病证,为男性常见性功能障碍之一。本病在《内经》中称“阴痿、阴器不用、筋痿”。本病病机主要为宗筋弛纵,痿软不举,病理性质虚多实少。如《景岳全书·阳痿》说:“火衰者,十居七八,火盛者仅有之耳。”本方枸杞子、肉苁蓉、巴戟天均入肾经,能温肾助阳,补命门之火兼益精气;炙黄芪补益中气,以消除阳明之气衰;阴阳互损,肾阳虚损日久,必然导致肾阴之不足,故又以熟地黄滋补肾阴兼补肝血;龟鹿二味均为骨肉有情之品,既可滋肾壮阳,填精补髓,又可防止补阳药物燥热伤阴之弊,使阴阳保持相对平衡。诸药合用,具有补肾壮阳、填精补髓、益气补中之功,故临床应用每可取得满意效果。本病在治疗过程中要注意虚实的兼夹,用药要灵活变通。本病除服药治疗外,尚须注意精神、摄生调理。因体弱所致尚须加强营养,适当休息,并可配合针灸治疗,以增强疗效。

【方剂出处】蒋建.龟鹿补肾汤加减治疗阳痿95例.时珍国医国药,2004,15(9):645

9.龟鹿海马汤

【药物组成】龟甲胶、鹿角胶、人参、菟丝子、五味子、覆盆子、车前子、山药、山茱萸、茯苓、牡丹皮、淫羊藿、海马、仙茅、杜仲、乌药各10g,熟地黄、丹参各24g,泽泻8g,蜈蚣2g,枸杞子、白芍各20g,炙甘草3g。

【随症加减】伴遗精滑精者加金樱子、芡实,伴多梦者加龙骨、牡蛎,伴心悸者加酸枣仁、龙眼肉,伴健忘者加益智仁、石菖蒲。

【治疗方法】每日1剂,水煎服,1个月为1个疗程,未愈者继服下1个疗程,共治疗2个疗程。

【功 效】调和五脏,填补肾精。

【临床运用】临床治疗368例患者,治愈331例,好转33例,无效4例,总有效率98.9\%。

【经验心得】阳痿有虚有实,但所治368例中无实证,正如张景岳“火衰者十居七八,火盛者仅有之耳”之说。本病的发生以肾虚为主因,肾虚宗筋弛纵,痿弱无力,故阳事不举。龟鹿海马汤方中熟地黄、龟甲胶、枸杞子、山茱萸填肾精,鹿角胶、海马、淫羊藿、仙茅、菟丝子、杜仲、覆盆子助肾阳,两组物药相伍,阴中求阳,阳中求阴,肾气自旺。人参、山药、茯苓益气健脾,使水谷得化,脏腑得养。泽泻、牡丹皮、茯苓、车前子泻其补药之浊气,白芍入肝以调宗筋,丹参、蜈蚣养血活血,乌药理气以防补药之滞,五味子闭精关之门以防精液外泄,炙甘草调和诸药。诸药合用则五脏并调,肾精充足。治疗期间须注意生活有规律,并禁房室及手淫。每天做肛提肌运动2~3次,每次1~5分钟,而湿热或阴虚火旺者不宜用本方。

【方剂出处】熊竹林.龟鹿海马汤治疗阳痿368例.实用中医药杂志,2006,22(1):10

10.强 精 汤

【药物组成】黄芪20g,当归15g,熟地黄10g,何首乌10g,五味子10g,菟丝子15g,覆盆子10g,肉苁蓉20g,淫羊藿10g,川牛膝15g,蜈蚣2条。

【治疗方法】水煎,日2次口服,疗程为1个月。

【功 效】益气养血,补肾填精。

【临床运用】临床治疗36例。显效14例,有效19例,无效3例,总有效率91.7\%。

【经验心得】以上病例均属肾精亏虚型,多由先天禀赋不足,命门火衰;或房室过度,伤及肾气所致。治宜益气养血、补肾填精为法。方中,黄芪、当归配伍为方中君药,其为当归补血汤,中医有“精血同源”之说,故补精离不开补血。中医学认为,精是由血化生而成,也就是说血为精之源,血旺则精足,补血为治疗此病的根本。熟地黄养血滋阴、补精益髓,何首乌补益精血,二药共助补血之功,为精的生成奠定基础;菟丝子、覆盆子、五味子补肾固精,直接补充肾精,肾精不足多为肾气不固,所以加用补肾固精的覆盆子和五味子;淫羊藿、肉苁蓉补肾助阳,温阳化气,使阴津蒸腾气化,使气血津液布散周身,则气旺神足,阴阳协调;牛膝引血下行,改善供血,既防治淫羊藿、肉苁蓉之燥热炎上,又可引热下行直达病所;蜈蚣兴阳通络效佳,治痿必用。诸药合用,标本兼顾,必能精旺神足,阳器以用事。

【方剂出处】段雪光,等.自拟强精汤治疗阳痿36例临床观察.实用中医内科杂志,2008,22(6):92

11.清化振痿汤

【药物组成】生薏苡仁30g,白豆蔻10g,石菖蒲10g,蚕沙10g,白芷15g,柴胡10g,蛇床子15g,萆薢15g,虎杖15g,牛膝10g,甘草梢6g。

【随症加减】伴失眠多梦、神情烦躁者,加远志、合欢皮各10g,龙骨、牡蛎各15g;心悸、气短、乏力、面色少华者,加黄芪24g,山茱萸、白术各10g;肝郁气滞者,加香附、郁金各10g;喜食肥甘者,加山楂15g,莱菔子10g;嗜酒者,重用葛根24g;兼湿热伤阴者,加知母、玄参、麦冬各12g。

【治疗方法】每日1剂,水煎,分2次服,14日为1个疗程。

【功 效】清利湿热,通脉起痿。

【临床运用】临床治疗129例,治愈89例,改善27例,无效13例,总有效率为89.92\%。

【经验心得】阳痿一证,历代医家均责之肾虚,治疗亦以温补壮阳为主,但疗效往往不十分满意,正如《医津一筏》所言:“邪火内炽,阳事反萎,苦寒泻之,阳事勃然,火与真阳势不两立,世人以助火之剂,翼肆真阳,非徒无益,而又害之。”究其原因,主要是现代人营养丰富,体质甚好,加上工作和生活压力,神情不舒,肝气不展,恣食肥甘厚味与烟酒等酿湿助热之品,更有交媾不洁,湿毒余邪留注肝肾,而致宗筋不用。叶天士在《临证指南医案》中指出:“阳痿,更有湿热为患者,宗筋弛纵而不坚举,治用苦味坚阴,淡渗去湿,湿去热清,则病退矣。”根据上述理论及临床经验,自拟清化振痿汤治疗本病,疗效满意。方中薏苡仁、白豆蔻、石菖蒲、蚕沙清热化湿;柴胡疏肝行气;萆薢、虎杖清化、淡渗利湿,分清别浊;白芷为风药,又是阳明引经药,既能燥湿,又能起痿;蛇床子引药入肾而补不留邪,使邪去正安;虎杖、牛膝化瘀通血脉,引药下行。诸药合用,共奏湿热清、机关利、宗筋张、阳事兴之效。

【方剂出处】施云,等.清化振痿汤治疗阳痿129例.湖南中医杂志,2001,17(6):39

12.加味柴胡疏肝散

【药物组成】柴胡、枳壳、川芎、郁金、香附、川楝子、陈皮各10g,白芍15g,蜈蚣2条,甘草5g。

【随症加减】兼肾阴不足加枸杞子15g,生地黄10g;肾阳不足加仙茅、淫羊藿各10g,心神不宁加酸枣仁、远志各10g,湿热下注加黄柏、车前子各10g。

【治疗方法】每日1剂,水煎服,分3次服,30日为1个疗程。

【功 效】活血通络,解郁起痿。

【临床运用】临床治疗50例,治愈31例,好转14例,无效5例,总有效率90%。

【经验心得】阳痿的基本病机主要是气血不能充养阴器,于是肝经气血旺盛畅通与否是性功能能否正常发挥的关键所在。肝失疏泄,气机逆乱,气滞血瘀,宗筋失养而致阳痿。而且临床观察到,本型患者多有情志抑郁、少言寡语、精神紧张、阴茎睾丸或少腹部坠胀疼痛、舌质紫暗或边有瘀点瘀斑、脉弦涩等肝郁血瘀表现,且以青壮年患者为多。故治疗当以疏肝理气,活血通气立法。柴胡疏肝散出自张景岳《景岳全书》,由柴胡、香附、陈皮、枳壳、川芎、白芍、甘草组成,原为治疗肝气郁滞证,变通其用,在原方上加郁金、川楝子、蜈蚣,既能疏肝行气,又能活血通络,气血同治,气行血畅,宗筋得养,切中肝郁血瘀阳痿之病机,故能取效。

【方剂出处】常建国.加味柴胡疏肝散治疗肝郁血瘀型阳萎50例.四川中医,2005;23(3):60

13.双补四物汤

【药物组成】黄芪30g,山药20g,苍术12g,陈皮10g,熟地黄15g,枸杞子12g,巴戟天12g,当归12g,丹参15g,川芎12g,赤芍12g。

【随症加减】伴阳虚者加淫羊藿15g,菟丝子12g;阴虚火旺者加黄柏12g,生牡蛎15g;肝气郁结者加柴胡12g,白芍12g;湿热下注者加车前子15g,黄芩12g,泽泻10g。

【治疗方法】每日1剂,水煎服。同时严格执行糖尿病饮食,选用格列本脲(优降糖)、格列吡嗪(美吡达)、二甲双胍中的一种或两种控制血糖。

【功 效】填精益髓,化瘀通络。

【临床运用】临床治疗25例,其中治愈5例,显效8例,有效9例,无效3例,总有效率为88\%。

【经验心得】糖尿病性阳痿是由于控制阴茎勃起的骶副交感神经病变及阴茎动脉异常所致。糖尿病患者常有山梨醇、果糖、葡萄糖在神经细胞内积聚现象,这样会因渗透压作用使神经细胞发生肿胀变性;微血管病变包括神经内血管进行性硬化以及供应神经营养的血管闭塞所致循环障碍等。阴茎动脉血管异常包括内膜的纤维增生、中层的纤维化、钙化以及管腔的狭窄闭锁,这样的结果必然阻碍血流到达海绵体,因而影响阴茎的勃起。糖尿病性神经病变如能及早诊断与严格控制,则不论周围神经和自主神经病变均是可逆性的。糖尿病性阳痿患者经降糖药治疗后,多饮、多食、多尿均不明显,但多有倦怠乏力,腰膝酸软,口唇、舌质黯淡或紫黯、瘀斑,阴茎有瘀斑,龟头青紫等脾肾两虚,血脉瘀阻表现。中医学认为,气充则血行,气虚则血滞。瘀血阻滞肝肾脉络,气血灌注宗筋不利而致阳痿。双补四物汤中黄芪、山药益气健脾,补后天以助先天;苍术、陈皮行气燥湿,以助脾运;熟地黄、枸杞子滋补肾阴,填精益髓;巴戟天兴阳起痿;当归、丹参、川芎、赤芍活血化瘀通络。综观全方,脾肾双补,药性平和,补而不滞,温而不燥,可使精盛阳强,瘀去脉通。如此标本兼治,故获良效。

【方剂出处】李金水,等.自拟双补四物汤治疗糖尿病性阳痿25例.国医论坛,2005,20(3):32

14.滋阴起痿汤

【药物组成】熟地黄30g,何首乌40g,枸杞子20g,山药15g,阳起石30g(包煎),淫羊藿10g,麻黄1g,黄狗肾粉1g(每晚吞服)。

【随症加减】腰酸腰困加桑寄生20g,狗脊10g,焦杜仲10g;双下肢无力、膝软明显,加怀牛膝15g,白芍10g,木瓜10g;失眠多梦加枣仁15g,首乌藤30g,生龙骨30g;形寒肢冷加炮附子10g,干姜10g,肉桂5g;气短、乏力、多汗加党参20g,黄芪20g,山茱萸12g,五味子10g,头晕头闷加菊花10g,枸杞子25g,黄精10g;阴囊潮湿加巴戟天15g,菟丝子15g。

【治疗方法】水煎服,每日1剂。若服药1周效果不明显者,于每晚临睡前,改黄狗肾粉剂量为2g,吞服。服药期间禁忌房事、烟酒,忌食辛辣刺激性食物。15天为1个疗程,一般情况可连续服用2个疗程。

【功 效】滋阴添精,补肾充髓。

【临床运用】临床治疗50例。治愈30例,有效16例,无效4例,总有效率92\%。

治验:李某,男,28岁。2001年4月27日初诊。主诉患阳痿2年,追问病史,婚前频频遗精3年余,婚后2年一直不能进行正常性生活,伴头晕乏力,耳鸣耳聋,腰膝酸软,身体瘦弱,常觉足心发热,舌质红,苔少,脉沉细,诊断为"阳痿",辨证为阴精亏损、宗筋失养。以滋阴起痿法治疗,药用滋阴起痿汤原方剂,加用焦杜仲15g,桑寄生20g,狗脊10g,黄精15g。水煎服,6剂后,性欲增强,晨起阴茎能勃起,但不坚硬,且历时短暂。知药已中病机,上方略出入调治30余天,阴茎勃起坚硬,后以六味地黄丸巩固。随访2年,性生活正常,生育1男婴。

【经验心得】张景岳认为,阳痿“火衰者十居七八”,作者临床观察阳痿患者阴虚者十居七八,与古代医家的观点相背离。尤其是中青年患者,每因相火自旺,或欲火萌生、遗精、房室不节而致阴精亏损,阴不济阳,阳无所依,宗筋失养,从而使阴茎不能挺举,或举而不坚,或早泄,不能进行正常性生活。通过滋阴添精、补肾充髓法治疗,使阴精充足,能与阳相济,阳得阴助,宗筋受润则功能无穷。方剂中选用大剂量熟地黄、枸杞子、何首乌、山药、黄狗肾以补亏损之真阴,配伍少量淫羊藿、阳起石,意在阳中求阴,使生化源泉不竭。全方大补真阴,对肾阴虚、精亏所致阳痿尤为适宜。

【方剂出处】杨宝贵.滋阴起痿汤治疗阳痿50例临床报告.中国中医基础医学杂志,2007,(12):22

15.五子二仙汤

【药物组成】熟地黄20g,山药15g,山茱萸10g,牡丹皮10g,茯苓10g,泽泻10g,肉桂10g,菟丝子15g,金樱子15g,韭菜子15g,覆盆子15g,枸杞子15g,巴戟天15g,淫羊藿15g,仙茅15g,阳起石20g,煅龙骨20g(先煎)。

【随症加减】下阴、下肢发冷、阳虚偏重者加制附子6~9g;失眠健忘,神经衰弱者加远志10g。

【治疗方法】上药水煎服,每日1剂。15日为1个疗程,休息3~5日,继服第2个疗程,服用2~3个疗程后,若疗效较好可按原方加倍改散剂继服1~2个疗程,以巩固疗效。

【功 效】温阳益气,养阴填精。

【临床运用】临床治疗139例,全痿患者50例,服药好转40例,无效10例,总有效率达80\%;半痿患者48例,治愈45例,有效3例,总有效率达100\%,举而不坚者41例,治愈41例,总有效率达100\%。

【经验心得】历代医家认为本证与肝、肾、阳明三经有密切关系。因为阴茎为厥阴肝经所达,为宗筋所聚,阳明主润宗筋,阳明气衰则宗筋不振,肾主藏精又主肾气,肾气虚弱,则阳事不举。所以治疗阳痿应从整体观念出发,从心、肝、肾、脾四脏入手,特别应以心肾为重点。同时,阳痿受精神心理和环境因素影响,因此在药物治疗的同时,辅以精神心理调节,以调畅气机,增强自信心。采用金匮肾气汤温补肾气,惟恐温阳之药不足,难以胜病,故加淫羊藿、仙茅、巴戟天、阳起石相伍,有温阳不伤阴,且具兴奋性功能之长;五子汤佐远志、煅龙骨有滋阴填精、强心固涩之效,合奏温阳益气、养阴填精之效。

【方剂出处】黄五臣,等.金匮五子二仙汤加减治疗阳痿139例临床观察.内蒙古中医药,2005,(6):7

16.回 春 丹

【药物组成】生地黄30g,熟地黄30g,山茱萸12g,肉桂10g,茯苓30g,淫羊藿40g,仙茅15g,续断20g,黄柏10g,枸杞子20g,桑寄生40g,牛膝20g,阳起石20g,巴戟天12g,杜仲12g,乌梅20g。

【治疗方法】每日1剂,加少许盐、酒,水煎2次各30~40分钟,分早、晚空腹各温服1次。21日为1个疗程。

【功 效】补肾填精,壮阳起痿。

【临床运用】临床治疗22例。治愈21例,有效1例,无效0例,总有效率100\%。

治验:赵某,男,36岁。2003年3月26日初诊。主诉婚后12年未育,性事一向不谐,近半年临房时阴茎由举坚时短、未泄先痿渐致举而不坚、难以性交,成功率极低,时有夜寐欠佳、多愁寡欢,夫妻关系紧张。曾四处寻医求诊,多用温阳补肾药如右归丸或赞育丹等汤剂或汇仁肾宝、雄狮丸等成药而罔效。亦经血尿常规、血糖、血脂及前列腺液、精液、肝功等化验和肝脏、前列腺B超检查,未见器质性病变。刻下:性欲低落、交合不能、晨勃尚存,神疲乏力、腰膝酸软,头晕纳差、大便时干时稀,夜寐不安、失眠愁虑、耳鸣如蝉,舌淡红、苔薄腻,脉弦细。证属肝郁肾虚、心神不宁、宗筋失养而病发阳痿。治宜疏肝益肾,佐以滋阴健脾养心。方予回春丹水煎,取汁于早餐前、午睡起、晚间空腹温服。首试7剂后复诊:诉有所好转。续用7剂后诉:夫人开始煎药殷勤。再用7剂,患者信心益增,药后诸症消、精神畅,夜间玉茎多有勃起,舌淡红、苔薄白、脉缓。药已中的,守法治疗,嘱休息7天后续第2个疗程,慎房事、少烟酒。3周后诉:食、寐、便、神如常,顺利入房,性感增强。第3个疗程后性事顺畅,夫妻尽享鱼水之欢,感情甚佳。第4个疗程后不久发现夫人有喜。随访至今未复发。

【经验心得】阳痿系由于长期精神紧张,思虑过度,情志抑郁;或恣食辛辣肥甘厚味,饮酒无度,酿生湿热,湿热下注,阻碍阳气疏布,宗筋弛纵;或劳倦伤脾、气血亏虚,宗筋失养而致。《景岳全书》:“凡男子阳痿不起,多由命门火衰,精气虚冷,或以七情劳倦损伤生阳之气,多致此证……但火衰者十居七八,火盛者仅有之耳。”故特以补肾益精,壮阳起痿为主组方。但在临证中还要采取“阴中求阳”“阳中求阴”之法,即在补阳时加入少许滋阴药,在补阴时加入少许温阳药。正如张景岳所说:“善补阳者,必于阴中求阳,则阳得阴助而生化无穷;善补阴者,必于阳中求阴,则阴得阳升而源泉不竭。”回春丹取熟地黄、枸杞子养阴补血,益肝肾;生地黄、黄柏、乌梅养阴生津;淫羊藿、仙茅、阳起石温肾壮阳;肉桂补火助阳,温经通脉,益阳治阴;山茱萸补肾益精,山茱萸既能补阳,又能滋阴;杜仲、续断、牛膝、桑寄生、巴戟天补肾壮阳,强筋骨;茯苓健脾安神。全方合用则能补肾填精,壮阳起痿,使肾精充,阳道兴而病告愈。

【方剂出处】李桂民,等.回春丹治疗心理性阳痿22例.中国现代药物应用,2009,3(5):103

17.阳 痿 汤

【药物组成】仙茅、露蜂房、淫羊藿各15g,覆盆子、山茱萸、枸杞子、熟地黄各20g,蜈蚣1条(焙干研末冲服)。

【治疗方法】水煎服,每日1剂,渣再煎服。

【功 效】温肾阳,益精气。

【临床运用】临床治疗35例。1个疗程后痊愈12例,显效9例,有效11例,无效3例,总有效率91.43\%。

治验:某某,男,65岁,阳事不举4年。患者10年前丧偶,丧偶前性功能正常,4年前再婚后发现临房时阴茎不能勃起,性生活无法进行,多方求治,病情如故。求诊时,伴见头晕、耳鸣、眼花、腰膝酸软、失眠、纳差、畏寒、大便溏、小便清长、舌淡嫩、苔薄白,脉沉无力。患者年事已高,天癸将竭,精髓不足,肾阳亏虚,脾失温煦,脾肾两虚,宗筋失养,无力举阳,治宜温补脾肾,填精补髓。口服阳痿汤2个疗程,患者全身症状明显改善,临房时阴茎能勃起,基本上能完成性交过程,性生活满意。

【经验心得】本方以淫羊藿、覆盆子为君,取淫羊藿“补命门,益精……”(《本草备要》);覆盆子“强阳健阴,男子精虚阳痿……能令坚长。”(《本草纲目》)。以仙茅、山茱萸、枸杞子、露蜂房为臣,仙茅“益阳道,房事不倦。”(《本草纲目》);山茱萸“补肾气,兴阳道,坚阴茎,添精髓”(《药性本草》);枸杞子“生精益气”(《本草纲目》);露蜂房“灰之,酒服,主阳痿。”(《唐本草》)。以熟地黄为佐,取其“主补血气,滋水益阳。”(《珍珠囊》)。以蜈蚣为使,取其“走窜之力最速,内而脏腑,外而经络,凡气血凝聚之处皆能开之。”(《医学衷中参西录》)。

是方重在肾,旁及肝,既有温肾阳益精气之功,又有通筋脉行气血之用,使肾宅阳生阴长,绝无助阳劫阴之弊。现代医学认为:阳痿的发生,无论何种原因均导致阴茎海绵窦的供血不足,因此用露蜂房、蜈蚣活血通络,以改善阴茎内的血液循环。在治疗上抓住了主要矛盾,有的放矢,故获良效。

【方剂出处】马朝辉.阳痿汤治疗阳痿35例临床观察.基层医学论坛,2009,13(1):51

18.三紫振痿汤

【药物组成】紫霄花10g,紫河车10g,丹参15g,蜈蚣2条,白芍10g,淫羊藿10g,露蜂房6g,巴戟天10g,枸杞子15g,香附12g,柴胡10g,葛根10g,九香虫6g,牛膝6g。

【治疗方法】水煎,每日1剂,早、晚饭后分服,30日为1个疗程,连服3个疗程,收效后以2倍量作蜜丸,每丸9g,每日3次,每服1丸,以巩固疗效。

【功 效】益肾疏肝,活血通络,兴阳振痿。

【临床运用】临床治疗75例,近期治愈39例,显效11例,有效19例,无效6例,总有效率为92\%。

【经验心得】方中淫羊藿、巴戟天、露蜂房益肾壮阳,以兴阳事。紫霄花为淡水海绵科动物脆质针海绵的干燥群体,甘温益阳涩精,是治阳痿的专用药。紫河车味甘咸性温,是血肉有情之品,补气养血益精。枸杞子滋补肝肾以益阴助阳。蜈蚣入窜力最速,内而脏腑,外而经络,凡气血凝聚之处,皆能开之,故可通达瘀脉,善治阳痿。白芍养血活血,补肝柔肝,荣养宗筋,既能养血益精和调阴阳,又能监制蜈蚣辛温走窜伤阴之弊。瘀血阻于经络,宗筋失养,难以充盈而致阳痿,瘀血又是肾虚肝郁的病理产物,故用丹参活血通经散瘀。肝经络阴器,宗筋乃肝所主,肝失疏泄,气血失调,经络运行障碍,宗筋难得其养,故阳事不兴,故用香附、柴胡疏肝解郁通经。九香虫善入肝肾之位,功善理气化滞,温中助阳。古有“治痿独取阳明”(《内经》语)、“阴中求阳”(景岳语)之论述,故佐以葛根鼓舞阳明津气,兼起阴气,牛膝益肾而引血下行阴部。诸药协力,共奏益肾疏肝、活血通络以兴阳事之功。

【方剂出处】于增瑞,等.三紫振痿汤治疗阳痿75例.北京中医,2006,25(9):552

19.参归三七汤

【药物组成】红参15g(口干、口苦、尿黄者用白参或西洋

参),当归、川芎各10g,枸杞子24g,丹参12g,三七15g,鸡血藤20g。

【随症加减】偏肾阳虚者加制附子、肉桂各10g,蛇床子12g,羊睾丸1对;兼有湿热者加薏苡仁30g,滑石20g,伸筋草10g;肝郁者加柴胡6g,郁金10g,另配服逍遥丸;兼前列腺炎者加红藤30g,露蜂房6g;阳痿因外伤引起者加鸡血藤30g,泽兰10g,另加服大活络丸。

【治疗方法】每日1剂,水煎2次,共取汁1 000ml左右,分4次温服。另辅以精神疏导,服药期间禁食萝卜、白菜。服药3周内禁房事,50日为1个疗程。

【功 效】益气,补血,活血。

【临床运用】临床治疗57例,1个疗程痊愈者15例,2个疗程痊愈者22例,好转15例,无效5例,总有效率91.24\%。

【经验心得】气血旺盛通行,宗筋得以充养,则阳物易兴,且勃而坚硬;若气血亏虚或滞行,宗筋失养,阳物举而不坚,甚至痿而不用。故兴阳起痿应抓住调治气血为本,若概以壮阳治肾,一味温补,不仅阳痿难除,反会阴伤。壮阳不兼益气,则所补之阳犹如浮云,不久即失,故先贤以附、桂壮阳每多伍以参、芪益气。又气虚不仅血亏,而且血运随之无力,往往致瘀,故益气补血活血应同步进行,否则难奏其效。

【方剂出处】张良圣,等.益气补血活血法治疗阳痿57例.实用中医药杂志,2006,22(8):482


\subsection{第2章 早泄}

早泄是男女在性交时,勃起的阴茎刚接触阴唇或未插入阴道即射精,阴茎随之软缩,使性交不能继续下去而被迫中止的—种常见的性功能障碍。健康人在性交2~6分钟后射精是很普通的。早泄的实质是过快射精发生在男性的愿望之前,他们在性活动中经常缺乏对射精和性高潮的合理而随意的控制力,使男性在性反应周期中迅速由兴奋期进入了高潮期,而几乎没有体会到性生活带来的快感。没有性活动周期中不断增进性紧张度的平台期,或平台期太短,致使双方未能获得性满足。导致早泄的原因有以下几种:在精神心理因素中,其主要的表现形式是焦虑,它是几乎所有性功能障碍的共同特征。至于造成焦虑的原因则是多种多样的。焦虑可以掩盖或妨碍患者对射精即将来临感知的警觉。医生的治疗目标之一应该是帮助早泄患者清楚地把射精的先兆感鉴别出来,并把它从本质上与射精本身区别开来。由于潜在焦虑常常导致早泄患者对时间概念具有一种主观上的扭曲,这自然会影响到他们的性表现能力。患者似乎被卷入一个时间的漩涡,它否定了射精之前的先兆感受和这两种感受的先后顺序。在这一关键时刻的感知错位和焦虑使他们不可能把欲望和满足感正确地区分开来。如夫妻感情不融洽,相互间存在潜在的敌意、怨恨和恼怒,或丈夫对妻子过分的畏惧、崇拜,存在自卑心理,使男方产生焦虑和恐惧心理,有的由某种偶然的原因,出现1~2次早泄,就背上了思想包袱,产生了恐惧与不安;焦急情况下的婚前性交;女方对性交厌烦,希望赶快结束,促使射精提前;长期禁欲后的首次性交等均可引起早泄。

泌尿生殖系感染如尿道炎、前列腺炎、精囊炎、精阜炎等,因炎症的刺激,尿道敏感性增强,在发生充血时,前列腺和精囊的代谢和分泌发生紊乱的,局部的刺激可能会对部分人引起暂时的早泄,因为对刺激的反应处于敏感的临界状态,就会很快发生射精。

早泄如能采取正确的心理治疗、药物治疗及性行为疗法等综合施治,往往可获得较好的临床效果,其预后较好。

1.八正散加减

【药物组成】萹蓄15g,瞿麦12g,木通15g,车前子20g

(包),滑石30g,栀子12g,莲子心12g,金樱子20g,煅牡蛎30g,甘草6g。

【治疗方法】内用药物水煎服。配合外用五辛散洗剂:五倍子50g,细辛5g,打碎水煎至200ml,温度为50℃。将龟头置入药液外洗浸泡按摩,药凉即止,每晚1次,2周为1个疗程。

【功 效】清热,祛湿,泻火。

【临床运用】临床治疗68例,近期治愈22例,显效29例,有效13例,无效4例。

【经验心得】早泄是男性比较常见的性功能障碍性疾病。中医学认为,早泄多由淫欲过度,斫伐肾阴肾阳,以致肾气亏损,封藏失职,固摄无权;或相火炽盛,手淫不节,肾精亏损,以致阴虚火旺,封藏不固,精关易开所致。现代医学认为多与精神心理因素有关,前列腺炎、精囊炎等亦可加重病情。曾有报道病例均伴有慢性前列腺炎或慢性尿道炎,出现下焦湿热蕴结症状,这可能与南方地域炎热、潮湿有关,亦与部分人的性生活混乱有关。患者反复出现泌尿生殖系感染,表现为下焦湿热症状。因此,治疗上用萹蓄、瞿麦、木通、滑石清热祛湿;栀子、莲子心清心火;车前子泻肾中虚火;金樱子、牡蛎固摄下元,共奏清热祛湿泻火之效,使精关免受邪热及相火扰动。五倍子、细辛为敛涩之剂,外洗龟头能降低敏感性,增强耐受力。标本兼治,内外结合是治疗早泄的可靠方法。

【方剂出处】黄清春.清热祛湿法为主治疗早泄68例.山东中医药大学学报,1999,23(5):359

2.分心清肝饮

【药物组成】生地黄10g,黄连10g,栀子10g,龙胆草6g,柴胡6g,龙骨15g,牡蛎15g,莲子15g,刺蒺藜15g,芡实10g,朱茯神30g,车前子10g。

【治疗方法】每日1剂,水煎服。连续用药1个月以上。

【功 效】清心安神,泻热疏肝。

【临床运用】临床治疗78例,临床治愈49例,有效23例,无效6例,总有效率占92.3\%。

【经验心得】男性射精是由神经、内分泌等系统参与调节的一种反射现象,早泄是患者自己控制射精能力太差所致,其病因多数与精神因素有关。其主症属中医学“相火亢盛、精气不固”的范畴,常伴有头晕、腰膝酸软、失眠多梦、口苦、烦闷诸症。中医学认为,心主神明,故治以分心之法,清心火、养心阴、补心气、定心神,神明通则早泄自然好转。另外,中医学认为,肝主疏泄、调情志,分心清肝饮能泻肝火、疏肝气,使肝气条达,而肝脉循阴部、络阴器,肝主宗筋,肝气条达,不仅进一步调理情志使心神定、精关固,更能直接改善阴茎功能,从而大大地改善患者早泄的状况。采用分心清肝饮就是清其心火、定心神、泻肝热、疏肝气从而达到治疗早泄目的。

【方剂出处】袁国辉.自拟分心清肝饮治疗早泄78例.四川中医,2003,21(8):39

3.酸枣仁汤

【药物组成】酸枣仁30g,知母12g,川芎、黄柏各10g,茯苓、枸杞子、熟地黄各15g。

【随症加减】以肾气不固为主,表现为射精前阴茎欠坚者,加淫羊藿、沙苑子等;以相火偏旺为主、阴茎易起,射精前勃起尚坚者,加牡丹皮、泽泻。病久者,补则酌加党参、黄芪,涩则加龙骨、牡蛎、覆盆子,随症加减药量。

【治疗方法】每日1剂,水煎,分2次服,连续服药20日为1个疗程。

【功 效】益肝血,宁心神。

【临床运用】临床治疗63例患,经治疗1个疗程,其中显效32例,有效21例,无效10例,总有效率84\%。

【经验心得】“酸枣仁汤”源出《金匮要略》,为治“虚劳虚烦不得眠”而设,方中清补涩并举,心、肝、肾并调,与早泄的病机暗契。从现代医学理论角度看,遗精、早泄与性神经系统的兴奋性有关,早泄是因射精的阈值降低,平台期相对缩短。故降低兴奋,升高阈值,能使性交延长。而知母、黄柏、酸枣仁相伍,能降低性神经系统的兴奋性,从而能够治疗早泄,以药适证,也是应用酸枣仁汤治疗早泄的主要思路。早泄的病理与遗精相类似,清代尤怡《金匮翼·梦遗精》曰:“动于心者,神摇于上,则精遗于下也。”说明射精、早泄与神志因素密切相关。肝藏血、主疏泄,心主血脉,藏神。酸枣仁汤益肝血,宁心神,堪当其任。且足厥阴之经脉“循股阴”“过阴器”,肝经气血的盈亏与否亦必影响性功能的正常发挥,益肝血治早泄也体现了性功能障碍“从肝论治”的精神。肾主生殖,藏精,乃性事活动的生理基础,故益肝切勿忘补肾,用酸枣仁汤治疗早泄必加枸杞子、地黄之类,使肝、肾、心并调,方能达较好疗效。

【方剂出处】卢伟.酸枣仁汤为主治疗早泄63例报道.浙江中西医结合杂志,2001,11(3):186

4.龙胆泻肝汤

【药物组成】龙胆草、栀子、黄芩、黄柏、牡丹皮、赤芍、川牛膝、车前子(包煎)各10g,柴胡8g,生地黄15g,生甘草6g。

【随症加减】伴生殖道感染者减牡丹皮、赤芍,加败酱草、白花蛇舌草、甘草;伴焦虑、畏惧、心悸者去牡丹皮、赤芍,加酸枣仁、龙齿;伴性欲减退者减生地黄、牡丹皮、赤芍,加淫羊藿、补骨脂、菟丝子;伴性欲亢进者,黄柏、牛膝各增为15g。

【治疗方法】每日1剂,水煎分2次温服。每5剂为1个疗程,一般治疗1~3个疗程。

【功 效】疏肝清火,解郁祛湿。

【临床运用】临床治疗60例,显效23例,有效31例,无效6例,总有效率为90\%。其中1个疗程有效者16例,2个疗程有效者23例,3个疗程有效者15例。

【经验心得】早泄的病因病机以肝气郁结、肾气不固、阴虚火旺为主。根据临床观察,原发性早泄多以肝气郁结、肝经湿热、相火炽盛为主,参照西医使用抗焦虑、镇静药的思路,选用龙胆泻肝汤加减以泻肝胆实火,清三焦湿热,收到较理想的疗效。据《中药大辞典》记载,该方中大多数药物有降低血压,减慢心率,镇静,延长睡眠及缓解肌肉紧张的作用,这与性交中出现全身性肌紧张、心动加速、呼吸急促、血压升高的反应有明确的针对性。方中减去木通、泽泻、当归,加用牡丹皮、赤芍、黄柏、川牛膝,目的是减轻原方的通利作用,而加强清热凉血泻火之功,同时牛膝还有引药下行的作用,更符合原发性早泄的病机。在药物治疗的同时,医师给予患者适当的性知识指导,更有利于提高疗效。

【方剂出处】邱慧敏.龙胆泻肝汤加减治疗早泄60例.山西中医,2002,18(1):44

5.调 泄 方

【药物组成】龟甲10g,炒刺猥皮6g(冲),炒鹿角片18g,海马6g(冲),莲须10g,山茱萸18g,芡实30g,金樱根30g,肉苁蓉20g,炙黄芪30g,生山药30g,生龙骨、生牡蛎各18g,钩藤18g。

【随症加减】若偏重于肾阴不足者加知母6g,黄柏6g;肝气郁结者加柴胡7g,川楝子10g;湿热蕴结者加车前子18g,木通6g;瘀血阻滞者加泽兰10g,益母草18g。临床睾丸或前列腺或精阜肥大者加夏枯草30g,王不留行100g;性交后小腹抽痛者加川楝子10g,乌药18g;精索静脉曲张者加鸡血藤30g。

【治疗方法】①将上药用冷水浸泡30分钟,沙锅文火煎30分钟,取汁约200ml,1日2次,1次100ml。15剂为1个疗程。②另取五倍子30g,文火煎30分钟,晾温后将龟头浸泡此液中15~20分钟,1日2次,15天为1个疗程。

【功 效】温肾助阳,安神固精。

【临床运用】临床治疗30例。治愈16例,显效7例,有效4,无效3例,总有效率90\%。

治验:程某,男,30岁。1989年6月18日初诊。患者新婚半年,为求早孕,房事频频,几乎每夜2~3次交合欲泻。妻子身疲不快,催其快速射精,渐之出现早泄,且射精的性快感日渐减少,内心甚感郁闷,常以食用烟酒解愁,致使早泄加重。近3个月来性欲低下,阴茎插入阴道稍动即泄精。症见:神疲乏力,腰部酸困,头晕口干,五心烦热,纳食一般,睡眠不稳,精神沮丧,检查:体格健壮,血压、外生殖器、前列腺均正常;精液系列检查基本正常。女方经妇科系统检查证实有生育能力。诊舌质鲜红、苔薄黄稍腻、脉细数。诊断:①乍交即泄。②精泄无子。③原发性早泄性不育症。证系气阴两虚,精关不固,肝经郁热。给予益气滋阴,补肾固本,疏肝清热之法治之。遂拟调泄方加牡丹皮10g,竹茹10g,栀子10g,炒酸枣仁30g,焦槟榔7g,柴胡7g。1日1剂。同时外用五倍子煎液浸泡龟头。1日1次。守方据证加减治疗月余,诸症消失,让其尝试房事,已能在阴道内交合5~6分钟后射精,且有欣快之感,即停服中药,嘱其服六味地黄丸合逍遥丸善后。同年9月其妻怀孕,次年6月顺产1女婴,体重3 100g。

【经验心得】中医学认为,精液的封藏与排泄是由心、肝、肾三脏功能活动协调的结果。肾主生殖,主封藏。肾虚则封藏失职,固摄无权,精液自泄;肝主疏泄,统调气机。若情志抑郁,疏泄失常,约束无能,则使交合提前射精;心主神明,君主之官。心有欲念,引发相火妄动,扰动精关,不能摄敛而早泄。可见,协调肾、肝、心三脏的生理功能,是治疗、治愈早泄的根本病理基要。因此选用炒鹿角片、海马、肉苁蓉等药,温肾以助元阳,纳气固肾以坚其形,从而提高射精中枢控制力;黄芪、山药,补肾健脾,振中宫,生气血,以润养宗筋,改善阴茎血循环;龟甲、山茱萸、金樱根等药,补肾阴,填精血,清心火,交心肾,以敛精液外泄;芡实、莲须、生龙骨、生牡蛎等药,镇静安神,固肾涩精,以延长射精时间;钩藤甘凉,平肝清热,定惊息风,解郁疏肝,增加人体血流量,兴奋子宫肌,增大回肠肌的收缩力,加强膀胱肌挛缩,麻醉坐骨神经敏感度,以提高射精时的自控力。诸药合用,温肾填精,则阴平阳秘;益气健中,则精液乃固;清心除烦,则神明内守;肝平气畅,则开合有度。

【方剂出处】贾巧萍.调泄方治疗早泄性不育症30例临床观察.光明中医,2008,23(9):1332

6.固精鳅鱼汤

【药物组成】鲜泥鳅50g,淫羊藿15g,鹿角胶10g,熟地黄20g,天冬10g,黄柏3g,五味子5g,煅牡蛎20g,金樱子15g。

【随症加减】兼肝经湿热加柴胡、车前草各10g;兼心脾气虚加白术12g,龙眼肉10g。

【治疗方法】将鲜泥鳅放入清水中养3日,令其排净污物;入油锅中煎黄后加水,再加入用纱布包好的上述中药(除鹿角胶外)和生姜4片,煎至100ml,去渣取汁,烊化鹿角胶后内服。每日1剂,每剂煎3次,早、中、晚各服1次。10日为1个疗程,连服2个疗程。

【功 效】补肾固精,滋阴降火。

【临床运用】临床治疗39例患者,经治疗痊愈33例,好转4例,无效2例。总有效率达94.8\%。

【经验心得】方中泥鳅鱼味甘,性平,归脾、肾经,能补肾壮阳,暖中益气,除湿;淫羊藿功能补肾壮阳,鹿角胶功能补肾壮阳,固精,共为君药。熟地黄滋阴补肾精,为臣药。天冬滋阴降火,黄柏清虚火,五味子、煅牡蛎、金樱子收敛固精,均为佐药。全方合用有补肾固精,滋阴降火之功。结合临床,随症加减,如兼肝经湿热加柴胡疏肝清热,车前草清利湿热;兼心气虚加益气健脾之白术,益气养心之龙眼肉。

【方剂出处】杨扬.固精鳅鱼汤治疗早泄39例.广西中医药,2004,27(2):50

7.固 精 煎

【药物组成】党参15g,天冬15g,莲子15g,生地黄15g,北黄芪10g,五味子6g,五倍子6g,煅龙骨30g,煅牡蛎30g,芡实15g,黄柏10g,砂仁6g,甘草6g。

【治疗方法】每日1剂,水煎,分2次服。2周为1个疗程。

【功 效】滋阴补肾,固涩止泄,健脾益气。

【临床运用】临床治疗40例,经1~3个疗程治疗,40例中治愈16例,有效20例,无效4例,有效率为90\%。

治验:陈某,男,32岁。2004年5月26日初诊。结婚6年,既往夫妻性生活正常,近半年来出现性生活时射精过早,一般只能坚持1分钟以内,有时阴茎勃起,刚接触女方身体即泄,未能完成性交,双方不能尽兴。性欲无明显改变,勃起功能正常。诊见舌淡苔薄白,脉沉细。诊断为早泄,证属脾肾两虚,治宜补肾健脾、固精止泄,投固精煎7剂,每日1剂,水煎服。1周后复诊,自述药后夫妻性生活1次,性交时间能坚持5~6分钟。再投7剂,性交时间延长至10分钟左右。继服1个疗程,性交时间能坚持10分钟左右,观察半年无复发。

【经验心得】早泄多因素体阴虚或久病伤阴,阴虚火旺,精官失职或肾阳亏虚,肾气不固所致。固精煎由金锁固精丸合三才封髓丹加五味子、五倍子而成。方中芡实、莲子固肾涩精、健脾益气,龙骨、牡蛎涩精止泄,党参益气健脾,生地黄、天冬、黄柏滋阴降火,砂仁温脾止泻,甘草调和诸药,五味子、五倍子均为收涩之品。诸药合用,共奏滋阴补肾、固涩止泄、健脾益气之功。

【方剂出处】王桂如.固精煎治疗早泄40例.河南中医,2007,27(1):59

8.调神固精汤

【药物组成】炒酸枣仁10g,柏子仁15g,合欢花10g,莲子10g,沙苑子15g,韭菜子10g,覆盆子20g,白芍20g,山茱萸10g,乌药15g。

【随症加减】肾阳虚者,加菟丝子30g,仙茅15g;肾阴虚者,加知母10g,黄柏10g。

【治疗方法】水煎分2次服,每日1剂,1个月为1个疗程。

【功 效】疏肝调心安神,补益温肾涩精。

【临床运用】临床治疗50例。治愈18例,显效18例,无效14例,总有效率72.0\%。

【经验心得】早泄与心、肝、肾关系密切。精液的藏摄疏泄依赖于心、肝、肾等脏腑的共同作用,肾主藏精,肝主疏泄,一藏一泄全在于心所主的神志所系,心神安宁则藏泄有度,心神不宁则精液的藏泄失度,所以早泄其本在肾,其制在肝,其源在心。方中枣仁、柏子仁滋心阴、益肝血且可宁心安神,合欢花解郁,莲子清心,四药合力清早泄之源;白芍养血敛阴柔肝并可疏泄肝气,山茱萸补肝肾兼有酸涩收敛之功,二者一疏一涩,使肝气疏泄有度而平早泄之制;沙苑子、韭菜子、覆盆子大补肾水、收敛肾精而固早泄之本;乌药温肾顺气降逆引诸药下行。如此清源固本,早泄乃愈。

【方剂出处】李波,等.调神固精汤治疗早泄50例.江苏中医药,2008,40(9):94

9.摄精延射汤

【药物组成】何首乌30g,枸杞子30g,菟丝子30g,芡实30g,金樱子30g,五味子10g,五倍子3g,桑螵蛸15g,海螵蛸15g,鸡内金10g,川楝子10g,生龙骨30g,生牡蛎30g。

【随症加减】阴虚火旺者加黄柏、知母各10g。

【治疗方法】每日1剂,水煎2次分服,4周为1个疗程,同时,每晚临睡前加服盐酸氯丙咪嗪25mg;西药组仅服盐酸氯丙咪嗪25mg。

【功 效】滋补肝肾,收敛固涩。

【临床运用】所有患者均以阴茎进入阴道不到2分钟射精为标准,临床治疗43例,有效37例(86.0\%),其中轻度者21例,重度者16例;无效6例(14.0\%)。

【经验心得】摄精延射汤方中枸杞子、何首乌、菟丝子具有增强性腺功能和抗衰老作用,是传统滋补肝肾,治疗遗精、早泄之品,为方中主药。其他药物则均具收敛固涩作用,为涩精止遗要药,其中牡蛎、龙骨、五味子还具镇静、安定作用,能调节性神经,共为辅药。川楝子苦寒,疏肝泄热,既能抑制补益药的偏性,又可防止固涩药之过,为佐使。

【方剂出处】严仲庆,等.摄精延射汤合盐酸氯丙咪嗪治疗早泄43例.中国中西医结合杂志,2001,21(7):551

10.逍遥固精汤

【药物组成】柴胡、白芍、白术、沙苑子、益智仁、桑螵蛸各10g,薄荷6g,五味子10~15g,磁石、芡实各20~30g,莲子10枚。

【随症加减】烦闷口苦者加黄连6g;夜寐难安者加百合10g,合欢皮15g;腰痛甚者加杜仲10g,续断15g,气虚者加黄芪、党参各20g;肾阳虚者淫羊藿、巴戟天各10g。

【治疗方法】每日1剂,水煎,分2次服。4周为1个疗程。

【功 效】疏肝补肾,固精止遗。

【临床运用】临床治疗68例,显效28例,有效27例,无效13例,总有效率80.9\%。

【经验心得】早泄的发生与心、脾、肝、肾等脏腑的功能失调有密切的关系,多因肝失疏泄,制约无能,心脾两虚,心肾不交,阴虚火旺,肾失封藏,固摄无权所致。而心因性早泄主要是肝气郁结,所欲不适,疏泄无权,或情绪紧张,心存恐惧,肾气亏虚,开合失司,则临房早泄,或禀赋不足,久病体虚,肾气亏虚,封藏失职,则临房即泄。自拟逍遥固精汤,主要药物有柴胡、白芍、白术、薄荷,取逍遥散之意,疏肝解郁;取芡实、沙苑子、益智仁、桑螵蛸、莲子益肾固精;用五味子、磁石安神定志。综观全方,具有疏肝补肾、固精止遗的作用。

【方剂出处】陈成博,等.逍遥固精汤治疗心因性早泄68例临床观察.浙江中医杂志,2007,42(9):514

11.镇肝熄风汤加减

【药物组成】牛膝30g,代赭石30g,龙骨30g,牡蛎30g,天冬15g,五味子9g,龟甲15g,玄参15g,蜈蚣3条,甘草9g。

【随症加减】兼见肝经湿热者加龙胆草、泽泻,阴虚火旺者加知母、黄柏,肾气不固者加山药、山茱萸、熟地黄。

【治疗方法】每日1剂,水煎服。4周为1个疗程。

【功 效】镇肝息风,滋阴潜阳。

【临床运用】临床治疗45例患者,治愈26例,有效15例,无效4例,总有效率为91.1\%。

【经验心得】中医理论认为,肝主疏泄,肾为封藏之本,肝与肾之间相互制约,相反相成,共司精关的开合。肝藏血,以气为用,体阴而用阳;其性主升、主动,为风木之脏。肝疏泄功能的异常一般分为疏泄不及和疏泄太过两种,虽同属肝脏气机失调,但前者是功能减退,后者是功能亢进。肝脉“过阴器”,在性交过程中,阴茎所接受的性刺激作用于肝脉,促进了肝脉气血的运行,但须受肝之疏泄功能的调节:疏泄不及则气行迟缓,开启精关无力,易出现不射精;太过则气行迅疾如风,冲逆精关而易出现早泄。治法当镇肝息风为主,佐以滋阴潜阳,用镇肝熄风汤加减治疗。方中以牛膝引血下行;代赭石、龙骨、牡蛎和蜈蚣降逆潜阳,镇肝息风;龟甲、玄参、天冬、五味子滋养阴液,以制阳亢;甘草缓急,调和诸药。上药合用,共奏镇肝息风、滋阴潜阳之功。实验研究证明,镇肝熄风汤具有镇静与催眠作用,这与现代医学应用氯丙咪嗪降低中枢神经兴奋性异曲同工。临床观察发现,该方能显著地缓解患者的紧张与焦虑情绪,减轻性交时过度的兴奋与激动感,增强患者对射精预感的感受与控制能力,延迟射精发动的时间;与氯丙咪嗪比较,具有不良反应少、复发率低、治愈率高的优点。

【方剂出处】张培永,等.镇肝熄风汤加减治疗早泄45例临床观察,山东中医杂志,2003,22(5):274

12.滋肾泻火固精汤

【药物组成】莲子心15g,山茱萸24g,山药30g,生地黄15g,五味子10g,煅牡蛎30g,芡实15g,酸枣仁30g,知母10g(盐炒),黄柏10g(盐炒),牡丹皮10g,沙苑子18g。

【随症加减】肝经湿热加龙胆草、黄芩、栀子;肝气不疏,加柴胡、当归、白芍;肾气不固加菟丝子、淫羊藿;心脾两虚加党参、白术、黄芪。

【治疗方法】每日1剂,水煎服,分2次服,4周为1个疗程。

配合针刺取穴:①百会、神门、气海、三阴交、太溪、太冲透涌泉;②心俞、肝俞、脾俞、肾俞、三阴交、太溪。每日针刺1次,两组穴位交替使用,连续12天,中间休息5天。1个月为1个疗程。治疗前令病人排空膀胱,针刺腹部,以尿道根部有电击感为度;针背部组以局部酸胀而放射至臀部(或大腿、脚跟部)为佳,手、足、头部穴位均应酸而麻胀,行捻转平补平泻法,留针30分钟。

【功 效】滋阴补肾,固涩止泄,健脾益气。

【临床运用】临床治疗48例。治愈21例,显效14例,有效5例,无效8例,总有效率85.42\%。

【经验心得】早泄的发生与心、肝、肾的功能失调有密切的关系。朱丹溪说:"主封藏者肾也,司疏泄者肝也,两脏皆有相火,而其系上属于心。"故早泄以肾虚为本,病机多因素体阴虚或久病伤阴,阴虚火旺,精官失职或肾阳亏虚,肾气不固所致。滋肾泻火固精汤中以山茱萸、山药、熟地黄益肾滋肾固精;沙苑子、芡实、莲子固肾涩精,健脾益气;五味子、牡蛎涩精止泄;黄柏、知母、牡丹皮滋阴清虚火;酸枣仁除虚烦,安心神;知母、黄柏、酸枣仁相伍,能降低性神经系统的兴奋性。诸药合用,共奏滋阴补肾、固涩止泄、健脾益气之功,可使精关开合有度,药证相符,故能收到较好疗效。诸穴中气海、肾俞、百会通任督和阴阳;神门、心俞安定心神;肝俞、三阴交、太冲、太溪、涌泉平调肝肾,脾俞健脾益气,诸穴合用阴阳和调,心肾相交,心神安定而明显提高射精控制能力。

【方剂出处】曹永贺.中药结合针刺治疗早泄48例.现代中医药,2008,28(5):73

13.针 刺 法

【穴位选择】一组为关元及双侧大赫、三阴交;一组为双侧肾俞、大肠俞、次髎。

【治疗方法】每日1组,两组交替使用,每次留针30分钟,其间行针2~3次,10次为1个疗程,两疗程之间休息3天。其中前一组针刺关元穴得气后加艾灸30分钟。后一组针刺次髎穴后加上G-6805型电针仪,用连续波低频率电刺激30分钟。同时配合口服六味地黄丸合金锁固精丸,每日2次。

【功 效】益肾壮阳,涩精止泻。

【临床运用】临床治疗36例。治愈22例,有效13例,无效1例,总有效率97.2\%。

治验:李某,24岁,有手淫习惯,且从18岁起性生活过于放纵,婚后性能力下降,以致3个月前出现早泄,每次不足1分钟或数秒钟即泄出,但勃起功能无异常。就诊时症见精神委靡,腰膝酸软,潮热盗汗,舌红苔少,脉细弱。诊为早泄,证属阴虚阳亢型。用上述方法治疗1个疗程后,患者诉其房事能进行5分钟之久,续用本法治疗2个疗程后,诉其房事达10分钟以上。又巩固治疗1个疗程,半年后随访未见复发。

【经验心得】早泄,最早见于《辨证录·种嗣门》“男子有精滑之极,一到妇女之门即便泄精,欲勉强图欢不得,且泄精甚薄。”中医学认为,本病多因肾气虚衰,疏泄失常,封藏失职,约束无力,固摄失权所致。六味地黄丸滋阴补肾,金锁固精丸涩精止泻。关元穴为三阴脉与任脉之会,乃男子藏精,女子蓄血之处,针后加灸可补下焦元气,调理冲任,温肾益阳,使气血得行,冲任得和,真气得充。大赫穴为冲脉少阳之会,其穴阴气盛大,精气会聚。三阴交为足三阴经之交会穴,有益肾固阳、气血双调、阴阳两补之功。肾俞、大肠俞、次髎均为背俞穴,是治疗肾病之有效验穴。诸穴合用,前后配穴,共奏益肾壮阳、涩精止泻之效。

【方剂出处】张彦珂.针药结合治疗早泄36例.浙江中医杂志,2008,43(9):531


\subsection{第3章 遗精}

遗精是指在无性交活动、无手淫的情况下,精液自尿道口自行泄出。青少年男性或婚后长期分居的男性,每月发生1或2次甚或3或4次遗精属正常现象,如频繁发生遗精或稍有刺激、色情意念即发生遗精者则为病态。

遗精分生理性遗精和病理性遗精两种。青春期后男性若无正常性生活,偶然出现遗精,遗精后无不适,属于正常生理现象,不属于病态。据有关资料统计,大约80\%的未婚青年男性都有过遗精。男性初次遗精的年龄大约在16岁。随着青春期的发育,生殖系统也逐渐发育成熟,睾丸体积增大,雄性激素水平明显升高,睾丸、附睾、精囊及前列腺等附属性腺器官时刻都在产生精液。正常情况下,精液在体内储存一段时间后,一般在体内被吸收掉,也可在受到性刺激时,或外生殖器官受到各种外界刺激时,不自觉地排出体外,这就是遗精。如成年男子遗精次数频繁,每周2次以上或在清醒状态下有性意识活动即出现遗精,并伴有头晕、乏力、腰酸等症状,则为病理性遗精。

1.桂枝汤加味

【药物组成】黄芪20g,桂枝10g,白芍10g,白术10g,煅龙骨、煅牡蛎各10g,炙甘草6g,生姜10g,大枣7枚。

【治疗方法】每日1剂,水煎分2次服。

【功 效】健脾助运,温养肌肤。

【临床运用】临床治疗50例。治愈30例,显效4例,有效9例,无效7例,总有效率为86\%。

治验:王某,男,22岁。先天禀赋不足,后天未得调养,幼年即有潮热盗汗,19岁时骤患遗精,数日后形肉大脱。连服滋阴涩精之剂,更进“益肾”“锁阳固精”诸丸剂,亦罔效,以为不治,衍期三载。2005年8月来诊治。作者观其面色苍白,肉消形脱,自汗畏风,舌质稍淡,苔薄白,其脉细数而虚。伴纳差,膝胫自冷,溲后时有精液滑出。辨证:阴精走泄于下、阳气散失于上,肾病传脾,证属“虚劳”明矣。脾气衰败,则谷减形消;阴损及阳,卫气不得固护,则肢冷、自汗、畏风。脾气不升,肾关不固,则见遗精。经云:“精不足者,补之以味;形不足者,温之以气。”服上方3剂后诸症递减,去龙骨、牡蛎续服二旬,脾运健旺,饮食大增,乃予“六味地黄丸”滋阴以善后。

【经验心得】遗精后形肉大脱,是为脾败,非阴亏也。服滋阴涩精之药,欲止其遗而不达,反徒伤脾气。上方内以健脾助运,外而温养肌肤,各脏皆得其养,大气升腾而愈。方中桂枝加黄芪辛甘化阳,行周身皮肉之间,以温肤充形;白术健脾以护肾关;龙骨、牡蛎涩精止汗。因此在辨证治疗中,要针对脏腑病变的孰轻孰重进行选择用药,脏腑辨证准确才能提高疗效,一味补肾治疗遗精是不全面的。

【方剂出处】宋秀霞.桂枝汤加味治疗遗精50例.河南中医,2008,28(4):21

2.萆薢分清饮加味

【药物组成】萆薢、车前子、茯苓各20g,丹参、虎杖、败酱草各30g,黄柏、白术各15g,石菖蒲、莲子心各6g。

【随症加减】遗精严重加煅龙骨、煅牡蛎、鸡内金;前列腺液白细胞较多加蒲公英、白花蛇舌草、连翘各10g;失眠多梦加远志、合欢皮、首乌藤。

【治疗方法】每日1剂,水煎,分2次温服。

【功 效】分利清解,清心宁神。

【临床运用】临床治疗44例患者,治愈28例,好转12例,无效4例,总有效率90.91\%。

【经验心得】萆薢分清饮出自清代程国彭《医学心悟》一书。原方为治疗湿热渗入膀胱所致之赤白浊。运用该方加味以治疗前列腺炎所致之遗精,收效甚捷。方中萆薢、黄柏、车前子擅走下焦,清热利湿泄浊;茯苓、白术善运中焦,健脾利湿,能杜绝生湿之源;石菖蒲和中化浊,莲子心清心宁神;丹参、虎杖、败酱草活血化瘀,清热解毒。诸药合用,分利清解,兼而施之。有如《本草正义》所言:“湿浊去而肾无邪热之扰,肾气自能收摄。”故不尚涩精而精关自固。

【方剂出处】徐丹.加味萆薢分清饮治疗前列腺炎所致遗精44例.浙江中医,2003,(9):379

3.龙胆百合汤

【药物组成】龙胆草10g,栀子10g,黄芩10g,柴胡6g,木通6g,泽泻10g,车前子10g(包煎),百合15g,生龙骨30g,生牡蛎30g,酸枣仁30g,首乌藤10g,芡实10g。

【治疗方法】每日1剂,水煎,分2次服。14日为1个疗程。

【功 效】清肝泻火,镇心安神,补肾涩精。

【临床运用】临床治疗102例,治愈73例,好转24例,无效5例,总有效率95.1\%。

【经验心得】遗精的原因主要有二:一是湿热之邪下注,扰动精室,常见前列腺炎、精囊炎、包皮炎等;二为心肾不交,相火妄动,常为缺乏正确的性知识,思想过分集中在性方面,或为此担惊受怕,使大脑皮质出现持续的性兴奋,进而引发遗精。因此,补肾涩精法治之疗效欠佳。龙胆百合汤正是从病因病机出发,选用龙胆草、栀子、黄芩、柴胡、木通、泽泻、车前子清肝经湿热,使疏泄有度;百合、生龙骨、生牡蛎、酸枣仁、首乌藤镇心安神;芡实补肾涩精,使精关坚固,合泄有常。

【方剂出处】李芳琴,等.龙胆百合汤治疗遗精102例小结.甘肃中医,2007,20(1):32

4.清心止遗汤

【药物组成】黄连6g,麦冬12g,甘草10g,五味子15g,煅龙骨25g,珍珠母20g,玄参15g,煅牡蛎20g,地龙15g,车前子15g。

【治疗方法】水煎服,每日1剂,7日为1个疗程,共服2个疗程。

【功 效】清心镇静,安神涩精。

【临床运用】临床治疗40例,治愈20例,显效10例,有效6例,无效4例,总有效率为90.0\%。

【经验心得】清心止遗汤主要组成药物有黄连、麦冬、甘草、五味子、龙骨、珍珠母、玄参、牡蛎、地龙、车前子。方中黄连味苦,性寒,具有泻心火之功效。《本草新编》记载:“上吐利吞酸,解口渴,治火眼,安心,止梦遗,定狂燥,除痞满。”心火得降,神自安宁,精自固。五味子,味酸性温,具有滋阴涩精安神之功效。《本草通玄》谓“固精,敛汗”。药理研究证实:五味子对不同水平的中枢神经系统有作用,增强神经过程的灵活性,促进二种神经过程的相互平衡,提高大脑皮质的调节作用。玄参味苦咸,性凉,具有滋阴降火除烦之功效。车前子味甘性寒,《日华子本草》记载:“通小便淋涩,壮阳,治脱精,心烦。”使热从小便而出,助解心火。龙骨味甘涩,性平,具有镇惊安神,敛汗固精之功效。《本草经百种录》曰:“龙骨最黏涩,能收敛正气,凡心神耗散,肠胃滑脱之疾,皆能已之。”《医学衷中参西录》记载:“龙骨,质最黏涩,具有翕收之功,故能收敛元气,镇静安神,固涩滑脱,凡心中怔忡,多汗淋漓,吐血衄血,二便下血,遗精白浊,大便滑泄,小便不禁,女子崩带、皆能治之。”牡蛎味咸涩,性凉,具有敛阴、安神镇静、涩精之功效。《海药本草》谓:“主男子遗精,虚劳乏损,补肾正气,止盗汗,去烦热。”地龙味咸,性寒,具有清热镇静的作用,药理研究该药有镇静、解热、抗惊厥的作用。诸药合用,全方具有清心镇静兼安神涩精之功效。

【方剂出处】陈和亮,等.清心镇静法治疗遗精40例临床观察.上海中医药大学学报,2003,17(1):28

5.加味四妙汤

【药物组成】苍术15g,黄柏15g,薏苡仁30g,牛膝15g,车前子12g,草豆蔻10g,滑石20g,石菖蒲12g,萆薢15g,白术15g,茯苓20g,丹参30g。

【随症加减】遗精频繁者加金樱子15g,芡实30g,心烦少寐者加莲子心10g,炒栀子15g。

【治疗方法】每日1剂,水煎服,1个月为1个疗程。

【功 效】清热利湿,固涩摄精。

【临床运用】临床治疗36例,治愈24例,有效9例,无效3例,总有效率为91.7\%。

【经验心得】遗精为未婚青年较常见的男性疾病,已婚青年亦可发生,中医辨证分为阴虚火旺、湿热下注、心脾两虚、肾虚不固等证型。近年随着社会的发展,人们生活方式的不断变化,如酗酒成癖、过食肥甘及烧烤香燥之品等,致湿热之邪内蕴,下扰精室而成遗精,所以,湿热下注型遗精并不少见,正如《古今医鉴遗精篇》所言:“夫梦遗滑泄者,世人多作肾虚治,特不知此证多属脾肾,饮食厚味,痰火湿热之人多有之。”因此,针对此类型遗精进行辨证治疗,拟加味四妙汤以清热利湿固精。在治疗过程中,应嘱患者同时解除紧张和恐惧心理,勿过度思虑及妄想色欲,节房室,心寡神安,被褥不宜过软、过暖,夜卧侧身屈足,内裤不宜过紧,并戒除不良习惯,才能使疗效得以巩固。

【方剂出处】段登志.加味四妙汤治疗遗精36例疗效观察.云南中医中药杂志,2004,25(1):47

6.固 精 汤

【药物组成】龟甲20g(先煎),沙苑子15g,金樱子20g,山药20g,芡实20g,桑螵蛸20g,山茱萸15g,知母8g,黄柏8g,石菖蒲9g,牛膝9g。

【随症加减】伴心火偏亢者加黄连、莲子心、肝火偏亢者加龙胆草10g,栀子10g;湿热偏盛者加萆薢10g,碧玉散1包;阳痿早泄者加菟丝子20g,刺猬皮10g;前列腺炎伴尿频、尿后余沥不尽者加益智仁10g,乌药10g。

【治疗方法】每日1剂,水煎服,早、晚各1次,15天为1个疗程。

【功 效】滋阴降火,补肾固精。

【临床运用】临床治疗89例。治愈58例,好转28例,未愈3例,总有效率为96.63\%。

治验:刘某,男,38岁。于2005年5月29日初诊。患者遗精3年,近5个月以来,遗精频繁,每周2~3次,伴有形体消瘦,精神委靡,头晕耳鸣,腰膝酸软,寐少健忘,心烦易汗,面部潮热,口干咽燥,阳痿早泄,脉象细数,舌红苔薄黄,证属阴虚火旺,精关不固,治以滋阴泻火,固摄精关。方选固精汤加味,药用:龟甲20g(先煎),沙苑子15g,金樱子20g,山药20g,芡实20g,桑螵蛸20g,山茱萸15g,知母8g,黄柏8g,石菖蒲9g,牛膝9g,菟丝子20g,刺猬皮15g。每日1剂,水煎服,早、晚各1次。服上方25剂,诸恙皆愈,随访半年,未再复发。

【经验心得】《医贯·梦遗滑精》云:“肾之阴虚则精不藏,肝之阳强则火不秘,以不秘之火临不藏之精,有不梦,梦而泄矣。”从而在治疗上采用滋阴泻火、益肾固精,佐以清利导浊方法。方中龟甲大补肾阴,育精增液,入肾经补水以制火;沙苑子、金樱子为治泄精之要药,最能补肾益精固精;山药、芡实为涩精秘气之品,固肾补脾,其润能滋阴,涩能止脱;桑螵蛸、山茱萸补而不峻,温而不燥,收敛固摄,能调心气,益脾气、补肾育精固精关;知母、黄柏苦寒泻火,以保真阴;石菖蒲宣窍导浊,牛膝通下焦涩秘兼引诸药直趋精室。诸药配伍,不专于治肾,也治心、治肝、治脾;不专于固精,也益精潜阳,实为标本兼顾之法,共奏滋阴降火、补肾固精作用。

【方剂出处】李立凯.固精汤治疗遗精症89例临床观察.云南中医中药杂志,2008,29(12):82

7.知柏地黄汤

【药物组成】知母15g(盐炒),黄柏15g(盐炒),山茱萸25g,熟地黄30g,山药30g,牡丹皮9g,茯苓9g,怀牛膝10g。

【随症加减】夜寐差、心烦明显者加合欢皮30g,首乌藤30g;目眩、腰膝酸软重者加枸杞子10g,菟丝子10g;乏力明显者加生黄芪30g,炙柴胡10g,炙升麻10g;伴有形寒肢冷者加制附子10g,肉桂5g。

【治疗方法】每日1剂,水煎,分2次服。15日为1个疗程。

【功 效】滋阴降火,固精止遗。

【临床运用】临床治疗15例,治愈4例,显效7例,好转3例,无效1例。

【经验心得】遗精过频的病机为肾阴亏虚,肾阴不济,相火妄动,扰动精室,肾阴亏竭,肾阳虚弱,故而肾封藏失职,精气不固。治当滋肾阴而抑相火。知柏地黄丸方中知母、黄柏性苦寒而降肾火,坚肾阴,两药相须为用,盐炒制后功效大增,且能直达病所。山茱萸补肾阴并能涩精,熟地黄补肾阴而填精益髓,山药补脾阴(后天之阴)亦能固精,牡丹皮清泄相火,茯苓助脾运化水湿以资助肾阴,怀牛膝壮骨补肾。诸药合用,共奏滋阴降火、固精止遗之效。

【方剂出处】柴科远.知柏地黄汤加减治疗遗精15例.实用中医药杂志,2007,23(6):362

8.固精灵丸

【药物组成】鹿角胶12g,龟甲胶12g,枸杞子15g,山茱萸9g,生地黄15g,西洋参15g,远志15g,酸枣仁15g,茯神15g,五味子15g,金樱子12g,芡实12g,甘草6g。

【治疗方法】粉碎炼蜜为丸,每次2丸,每日2次,淡盐水送服,1个月为1个疗程。此类患者应注意不能单纯靠药物,更重要的是调摄精神,清心寡欲,排除杂念,勿食辛辣刺激之品。

【功 效】滋阴清热,交通心肾,固精止遗。

【临床运用】临床治疗150例患者,治愈129例,显效9例,有效6例,无效6例,有效率96\%。

【经验心得】本病的记载,首见于《内经》《灵枢·本神》称为“精时自下”;《金匮要略·血痹虚劳病脉证并治》称为“失精”,并有“有梦为心火,无梦为肾虚”之说。通过多年临床观察,遗精多见于君相火旺,心肾不交。心主藏神,气交于肾,凡人劳神过度,心阴暗耗,心阳独亢,心火久动,必伤肾水,则心火不能下交于肾,肾水不能上济于心,心肾不交,于是君火动越于上,肝肾相火应之于下,水亏火旺,扰动精室而遗。且本型肾阴亏损是其根本,心火偏亢为其标,故方中鹿角胶、龟甲胶、枸杞子、山茱萸大补其阴,生地黄滋阴清热;远志、茯神、酸枣仁养心宁神;金樱子、芡实、五味子清心,摄精固精;西洋参、甘草宁心益气。本方溶清滋补涩四者为一炉,使水火相济,阴平阳秘,精室得以固秘,故遗精自止。经临床观察,临证尽量不要黄连之属,以防苦寒太过,影响其疗效,个别患者可用莲子肉以配合之,效果更佳。

【方剂出处】常松山.固精灵丸治疗心肾不交型遗精150例.河南中医学院学报,2004,19(111):60

9.冬虫夏草

【药物组成】冬虫夏草25~30g。

【治疗方法】将冬虫夏草置1只鸡腹内,炖熟食用。每只鸡连吃3~4天,隔3~4天再按上法吃1只鸡,连吃4只鸡为1个疗程,一般治疗1个疗程。治疗期间,嘱患者注意适当的体育锻炼,严禁手淫及性生活,注意营养,严禁饮酒及食用辛辣之品。

【功 效】补益肾精。

【临床运用】临床治疗15例。临床治愈2例,显效6例,有效5例,无效2例,总有效率为86.67\%。

【经验心得】遗精是指精液自动外泄的一种疾患,成年男子未婚或久旷者,偶有遗精,次日并无任何不适者,属生理现象,不是病态。若3~5天遗精1次,甚至昼夜遗精次数无定,并有头晕神疲、腰酸腿软、心慌气短等症状者,则为病态,必须及时治疗。《金匮要略》称本病为“失精”“梦失精”“精自出”,并认为这是虚劳病证的一个症状。中医学认为,肾藏精,其气通于阴,劳伤肾虚,不能藏于精,故精液出。冬虫夏草能补益肾精,故能用于遗精的治疗。

冬虫夏草的虫体似蚕,长3~5mm,直径3~8cm,表面深黄色至黄棕色,有斑纹20~30个,头部红棕色,脚8对,中部4对明显;质脆以折断,断面略平坦,淡黄白色。子实体细长圆形,长4~7cm,直径3mm,表面棕褐色,有细纵皱纹,上部稍膨大,在放大镜下可见具粗糙的凸出点,为“子囊瓶”;质柔韧,断面类白色,气微腥,味淡。以完整虫体色泽黄亮,丰满肥大、断面黄白色、菌座虫体短小者为佳。冬虫夏草是一味强壮滋补药,有保肺补肾,止咳化痰,补虚损,治虚喘、咯血等功能,是治虚劳咯血、阳痿遗精、腰膝酸痛等症的良药。

【方剂出处】袁红芬,等.冬虫夏草治疗遗精15例临床观察.当代医学,2008,(6):153


\subsection{第4章 不射精}

性交时阴茎能勃起插入阴道内,但不能达到性高潮和射精,称为不射精,是由中枢和周围神经系统、内分泌系统及生殖器官等共同参与的复杂的生理反射过程中,某个环节功能障碍使性兴奋的刺激不足以产生射精反射。其病因有功能性与器质性两类,功能性更为多见。

在清醒状态下,从未发生过射精现象为原发性不射精。曾有在阴道内射精经历,以后在性交时不射精为继发性不射精。临床上能够看到的不射精患者,大多为原发性绝对(或严重)不射精的患者,他们在每次性活动中虽然做了各种努力,如延长阴茎勃起时间,进行富于刺激的性幻想,使用一些辅助的保健品或药物,阴茎持续勃起至体力耗竭仍未能射精。虽然性伴侣的女方可满足,但自己往往需靠抽出阴茎后手淫才能达到高潮、快感和射精。

1.加味四逆散

【药物组成】柴胡、枳壳、白芍、甘草、郁金、香附各10g,石菖蒲、远志各6g,茯神、枸杞子、熟地黄各12g,丹参、王不留行各15g。

【随症加减】射精无力者加黄芪、党参;肾阴虚者加山茱萸、桑椹;精液量少者加菟丝子、女贞子、韭菜子;阳痿者加仙茅、淫羊藿、巴戟天;睾丸胀痛者加橘核、荔枝核;瘀阻甚者加牛膝、三七;夹湿热者加萆薢、黄柏;心神不宁、遗精者加珍珠母、首乌藤、黄连、肉桂。

【治疗方法】每日1剂,水煎,早晚分服,30日为1个疗程。治疗期间配合适宜的心理治疗。

【功 效】解郁活血,畅通精道。

【临床运用】临床治疗38例,治愈22例;有效10例;无效6例。总有效率为84.2\%。

【经验心得】中医对本病早有认识,《诸病源候论》有“精不射出,但聚阴头,亦无子”的记载。肝之经脉绕阴器、过小腹,阴器为宗筋之会,肝的疏泄功能对精液藏泄正常与否有密切的关系;肾藏精,主生殖,肾的开合正常则精液藏泄有度。本病责于肝失条达,疏泄不及,气滞血瘀,宗筋失养,精窍失灵。有部分医家认为治疗本病应以滋补肝肾为法,然临床上如单用此法,则见效甚微,甚至适得其反。肝郁瘀血阻滞精道是本病的主要病机,治疗本病只有在疏肝活血的前提下,疏通其经脉,使精道通畅,然后补肾以改善性功能,促进性高潮,方能达到射精目的。故采用加味四逆散为主方化裁治疗,先疏肝理气、解郁活血,以调通经脉、畅通精道,然后再补肾填精、益和阴阳,如此标本兼治,自可收到较为理想的疗效。

【方剂出处】彭汉光,等.加味四逆散治疗功能性不射精症38例.湖北中医杂志,2004,26(12):36

2.逍遥散加味

【药物组成】柴胡9g,当归12g,白芍9g,炒白术15g,茯苓15g,炙甘草6g,薄荷6g,王不留行15g,香附6g,郁金12g,石菖蒲12g,穿山甲3g(研末冲服)。

【随症加减】①肝郁化火者加牡丹皮9g,栀子9g,生地黄12g;②气滞血瘀者加桃仁9g,红花9g,川牛膝12g;③兼有湿热者加苍术9g,黄柏9g,车前子12g;④兼有痰浊者加陈皮9g,半夏12g,地龙9g;⑤肾精亏虚者加菟丝子9g,紫河车9g,枸杞子12g,淫羊藿12g。

【治疗方法】每日1剂,水煎,分3次温服,1个月为1个疗程。

配合针灸治疗。第一组主穴:关元、曲骨、太冲(双)。配穴:期门(双)、足三里(双)。第二组主穴:肾俞(双)、次髎(双);配穴:内关(双)、三阴交(双)。上述两组穴位交替使用。操作方法:先令患者排净小便,局部常规消毒后,关元、曲骨用2寸30号毫针快速进针,针尖略向会阴方向倾斜,使针感传至龟头;肾俞、次髎进针1.5寸,肾俞用补法,次髎平补平泻;期门进针1寸,用泻法;足三里进针2寸,用补法;内关进针1.5寸,平补平泻;三阴交进针1.5寸,平补平泻。留针30分钟,中间行针1次,12次为1个疗程,休息3天开始第2个疗程。

【功 效】疏肝解郁,通利精关。

【临床运用】临床治疗28例。经治1~3个疗程。痊愈26例,占92.86\%,无效2例,占7.14\%。

【经验心得】以上病例属于功能性病变,并经辨证是中医肝气郁结型,治以疏肝解郁、通利精关。方中柴胡、香附、薄荷、郁金疏肝理气,解郁化瘀;当归、白芍养血柔肝;炒白术、茯苓、炙甘草健脾补中,穿山甲、王不留行活血化瘀,通经活络,利精窍;石菖蒲化痰湿。

关元为培补元气之要穴,可激发元气,温肾壮阳,培元固本;肾俞为肾之背俞穴,可补肾之虚,并振奋肾之阳气;曲骨、次髎为局部取穴,用之可促进局部气血运行,以通利下焦之经络;期门、太冲疏肝理气解郁;内关、足三里、三阴交疏肝理气、健脾补中。

【方剂出处】曹永贺,等.逍遥散加味联合针灸治疗肝郁型不射精症.医药论坛杂志,2007,28(23):86

3.解郁通精汤

【药物组成】柴胡6g,白芍1g,当归6g,香附10g,王不留行10g,石菖蒲10g,枸杞子15g,急性子6g,车前子10g。

【随症加减】肝郁明显加郁金、路路通;阴虚火旺加女贞子、墨旱莲,或另服知柏地黄丸;心脾两虚另服归脾丸;湿热瘀阻加木通、龙胆草,或另服龙胆泻肝丸;瘀血明显或久治不愈者可选加蜈蚣、水蛭、炮穿山甲等。

【治疗方法】水煎服,每日1剂,7日为1个疗程,连服1~4个疗程。

【功 效】疏肝解郁,和营通精。

【临床运用】临床治疗56例患者。治愈49例,好转3例,无效4例,总有效率为92.86\%。治愈病例服药时间7~28日。

【经验心得】影响射精活动有两项重要的因素,一是性器官局部触觉性刺激的“阈值”未能达到足够的强度,不能激发起射精反射;二是精神因素,由于不正常的心理活动,干扰了大脑皮质中枢的活动,会产生一种抑制性反射,继而阻断正常射精活动的神经反射的发生。这种精神因素致病,属于中医情志致病范畴。阴器乃肝经所绕,为宗筋所聚,肝主疏泄,肾藏精、开窍于二阴。肝主疏泄与肾主封藏存在着相互制约、相反相成的关系,肾精的排泄有赖肝气的疏泄。由于情志内伤,肝气郁结,疏泄失职,精关失灵则不能射精,不能射精反而加剧肝郁症状,日久则气滞血瘀。或化火伤阴,或败精瘀血阻滞精道,缠绵难愈。基于对以上病因病理的认识,采用疏肝解郁、和营通精为主法。方中柴胡、白芍、香附疏肝解郁,当归配白芍养血和营,枸杞子益精,王不留行、石菖蒲、车前子通利精窍,急性子破瘀通窍,结合辨证加减,配合性知识教育及心理疗法,取得较为满意的疗效,且未见明显的不良反应。

【方剂出处】郑国珍,等.解郁通精汤治疗功能性不射精症56例.中国中医药科技,2000,7(1):15

4.开窍通关汤

【药物组成】生麻黄5g,石菖蒲10g,冰片1g(冲服),蜈蚣1条(生用、研粉吞服),白芍、当归、路路通,川牛膝各15g,生甘草6g。

【随症加减】阴虚火旺者,加用知柏地黄丸,每次6g,每日2次;肝郁者加柴胡10g,香附10g;湿热阻滞者,加用龙胆泻肝丸,每次6g,每日2次;瘀阻精道者,加桃仁、红花、水蛭各10g。

【治疗方法】每日1剂,水煎,分2次服,10日为1个疗程。

【功 效】通窍疏精。

【临床运用】临床治疗30例,治愈16例,好转5例,无效9例,总有效率为70\%。

【经验心得】本病病因虽多,但不外虚实两端,虚者正虚无力开启精关,实者实邪阻滞精关。临床所见实者居多,先实后虚,因实致虚是本病发展的一般规律,故治疗以开启精关为主,随症加减。方中麻黄性温味辛,能开关通闭;石菖蒲味辛性温,功能化浊开窍,《本经》谓“开心窍,利五藏,通九窍”;冰片性味辛苦微寒,功能开窍醒神,《本草纲目》:“通诸窍、散郁火。”《医学衷中参西录》谓:“蜈蚣走窜之力最速,内而脏腑,外而经络,凡气血凝之处皆能开之。”路路通通行十二经,川牛膝引气下行,当归养血和血,白芍柔肝养筋,甘草和中解毒,监制麻黄、石菖蒲、蜈蚣走窜伤阴之弊。药理研究证实,麻黄中含有麻黄碱、伪麻黄碱,是肾上腺素能受体兴奋药,可使交感神经节后纤维释放儿茶酚胺,能增强输精道的平滑肌收缩,对射精有促进作用。

【方剂出处】赵土亮.开窍通关法治疗功能性不射精30例.四川中医,2003,21(8):43

5.补肾通精汤

【药物组成】菟丝子30g,肉苁蓉、巴戟天、远志、漏芦、石菖蒲、路路通、王不留行各10g,车前子9g(包煎)。

【随症加减】伴遗精、腰酸加生龙骨、牡蛎各15g,续断、牛膝各10g;伴性欲亢进加玄参12g,麦冬15g,黄柏10g,肉桂4g;伴性欲低下加淫羊藿、仙茅、枸杞子、蛇床子各10g;伴前列腺炎、精囊腺炎加苍术9g,薏苡仁15g,黄柏、穿山甲各10g;伴输精管扩张加三棱、莪术各9g,牛膝、地龙各10g。

【治疗方法】水煎服,每日1剂,早、晚各服1次,服15剂停药观察10~15日,再进入下1个疗程,连用1~3个疗程。同时给予必要的性知识指导及心理疏导,戒烟酒。

【功 效】补肾为主,启关通精。

【临床运用】临床治疗67例患者,治愈46例,好转14例,无效7例,治愈者最短1个疗程。

【经验心得】临床中,不射精症占男性不育的10\%~30\%,不射精症需与逆行射精相鉴别,逆行射精在性交中虽也无精液射出,但有性高潮或明显的射精动作,性交后第1次小便镜下检查有大量精子。中医学认为不射精症主要与肝肾功能失调有关,究其病因,仍以瘀浊阻滞,房室劳伤为本。现代医学认为可能与大脑中枢神经系统性兴奋的调节障碍有关。本病患者往往均伴有不同程度的心理性障碍,如性无知、少年手淫过度戕伤、性抑制等。补肾通精汤以补肾为主,“通”“补”结合,标本兼治。菟丝子、肉苁蓉、巴戟天补肾壮阳填精,远志、石菖蒲调心宁志、化浊开窍,路路通、王不留行、漏芦、车前子祛瘀浊通精窍,菟丝子更有雄激素样作用,诸药合用,瘀者去,精道通。同时给患者介绍有关性生理、性技巧知识,解除患者焦虑的心理情绪,要求患者妻子主动配合性治疗,往往能收到事半功倍的治疗效果。

【方剂出处】刘辉.补肾通精汤治疗不射精症67例.安徽中医学院学报,1999,18(6):31

6.益气通关射精汤

【药物组成】当归15g,白芍20g,黄芪30g,柴胡10g,威灵仙15g,淫羊藿15g,露蜂房10g,蜈蚣2条,水蛭3g,急性子10g,王不留行10g,路路通10g,远志12g,石菖蒲10g,车前子10g,牛膝15g,甘草3g。

【随症加减】肾阳虚衰加制附子、肉桂、肉苁蓉、巴戟天;肾阴不足加知母、黄柏、枸杞子、山茱萸;瘀血内阻加桃仁、红花、丹参、穿山甲;湿热下注加龙胆草、栀子、金银花、蒲公英。

【治疗方法】每日1剂,早、晚分服,蜈蚣、水蛭研末分吞。10剂为1个疗程。口服左旋多巴0.25g,每日3次。性知识缺乏者,对夫妻双方进行性心理指导教育以缓解其心理负担。

【功 效】益气养血,活血疏肝,通关射精。

【临床运用】临床治疗36例,治愈33例,好转2例,无效1例,治愈率91.67\%,总有效率97.22\%。

【经验心得】中医学认为,功能性不射精症与心、肝、肾有关。肾为先天之本,先天禀赋不足,肾阳虚衰则无力射精;肾精亏耗则精液内枯,不能射精。肝主气血及精液疏泄,肝气郁积,疏泄失常则气血瘀滞,精道阻滞不畅;肝经湿热下注则阴部瘀热,精窍阻塞。精藏于肾,其主在心,心肾不交,精关不开,故交而不泄,反见梦遗滑泄。益气通关射精汤方中当归、白芍、黄芪补肝肾,益气养血柔肝,气血充足,气血调和则精易生发,精满则易出也;柴胡疏肝解郁;淫羊藿、露蜂房温肾壮阳;蜈蚣、水蛭、急性子、王不留行、路路通等均为活血化瘀通络之品,起活血疏肝、通关开窍作用,其中蜈蚣助阳通关,“走窜力最速,内而脏腑,外而经络,凡气血凝聚之处皆能开之”;水蛭走血分,破血逐瘀通经,瘀血败精可搜剔;远志、石菖蒲开心窍,宁心神,交通心肾,尤其石菖蒲通窍化浊,如《本经》所说:“开心窍、补五脏、利九窍”,能荡涤邪秽则九窍灵通;车前子、泽泻通利精窍,清利下焦湿热;牛膝入肾,引药下行直达病所;甘草和中解毒,调和诸药。全方共奏益气养血、活血疏肝、通关射精之功。

【方剂出处】徐生荣.理疗加益气通关射精汤治疗功能性不射精症36例.河北中医,2001,23(2):138

7.射 精 汤

【药物组成】知母10g,黄柏10g,熟地黄15g,石菖蒲12g,麻黄5g。

【随症加减】阴虚火旺加生地黄、鳖甲、龟甲等;心肾两虚减知母、黄柏,加淫羊藿、阳起石、菟丝子、远志、茯苓、酸枣仁等;湿热下注加金银花、蒲公英、桃仁、当归等;肝郁气滞加柴胡、郁金、白芍、枳壳等。

【治疗方法】每日1剂,水煎服。同时,性交前用麻黄适量泡茶饮,并配合自编性知识指导,即性激发训练、性兴奋训练、性高潮训练三步训练法和心理疏导。

【功 效】养阴清火,交通心肾。

【临床运用】临床治疗86例患者,经过3个月的治疗,治愈58例,好转18例,无效10例。

【经验心得】不射精症按中医辨证分型,临床常见有阴虚火旺、心肝两虚、肝郁气滞、湿热下注四型,分别以滋阴降火、补益心肾、疏肝理气、清泄湿热治疗,能平衡阴阳,滋水涵木,气血充盛,肝肾舒达,经通络畅,从而达到肾阳振,肾精充,气机畅,痰瘀化,精关通,精乃泄的效果。正常射精动作的完成还有赖于交感、副交感神经的正常调节,特别是附睾、输精管、精囊腺、前列腺及膀胱内括约肌、会阴部肌肉的有效收缩及膀胱外括约肌的有节律松弛,才能使后尿道内精液经尿道外口喷出而完成射精。生麻黄性温味辛,发散宣通,能通利九窍,其有效成分麻黄碱具有肾上腺素样作用,能直接激动α和β受体,兴奋射精中枢并促使交感神经节后纤维释放儿茶酚胺,增强精路平滑肌的收缩而有利于射精。知母、黄柏、熟地黄养阴清火,石菖蒲交通心肾,辅助麻黄完成射精功能。

【方剂出处】范栋贤.射精汤为主治疗不射精不育症86例.河南中医,2003,23(11):26

8.六味地黄丸

【药物组成】熟地黄20g,山药15g,山茱萸15g,牡丹皮15g,茯苓15g,泽泻15g,淫羊藿15g,石菖蒲10g,萹蓄10g,瞿麦10g,桃仁10g,红花10g,杏仁10g。

【随症加减】肾阳虚加制附子、肉桂;阴虚火旺加知母、黄柏;肝郁气滞加柴胡、青皮、枳壳;湿热下注加蒲公英;由惊恐等精神因素导致者加酸枣仁、远志。

【治疗方法】水煎服,每日1剂,早、晚各服1次,15日为1个疗程。

【功 效】滋阴补肾。

【临床运用】治疗12例患者,治愈8例,好转3例,无效1例,总有效率91.7\%,服药最少1个疗程,最多4个疗程。

【经验心得】不射精是由大脑皮质对射精中枢抑制加强或射精中枢功能低下所致。中医学认为,本病的发病机制与肝肾两脏关系密切,多由先天禀赋不足,肾气虚弱,生化乏源,导致肾精亏乏,无精可泄;或饮食不节,过食肥甘及辛热之品导致湿热内生,流注于下,湿性黏滞,蕴结精道,以致精窍不通;或情志所伤以致肝失条达,疏泄不利,气机郁滞,导致精窍开启不利;或瘀血内阻,闭阻精窍;或由惊恐等精神因素导致射精不利。所以,本病虚实夹杂,治宜攻补兼施,方取六味地黄丸滋阴补肾。因本方补中有泻,泻中有补,三补三泻,故为首选,诸药合用,补其不足,泻其有余,调其平衡,疗效显著。

【方剂出处】张彦清.六味地黄丸加味治疗不射精症12例.内蒙古中医药,2000,10(6):42


\subsection{第5章 精液、精子异常}

一、血 精

血精是青壮年男性常见病症之一,多因精囊炎所造成。由于精囊与前列腺、泌尿道、直肠等器官相邻,当这些器官有炎症时,细菌极易蔓延到精囊,引起发炎,造成精囊壁肿胀、充血、微血管的损伤引起出血,故随着射精动作精囊腺收缩,形成血精现象。

1.小蓟饮子

【药物组成】小蓟、薏苡仁各30g,生地黄、石韦各15g,生蒲黄、干藕节、栀子各12g,淡竹叶、木通、血余炭各9g。

【治疗方法】每日1剂,水煎口服。15日为1个疗程。

【功 效】清热,通利,止血。

【临床运用】临床治疗31例患者。治愈22例,好转6例,无效3例,总有效率90\%。

【经验心得】血精在男性疾病中并非少见,一般认为大多由精囊炎、前列腺炎等疾病所致。现代医学多采用抗生素治疗,或配合坐浴疗法,但其临床疗效不理想。加之其发病后临床症状不明显,也不影响工作,不少患者就放弃了治疗。小蓟饮子是治疗血精的常用方剂。临床所见,血精发生在青壮年者居多,分型以下焦湿热和阴虚火旺居多,就其发病机制而言,与“血淋”有相似之处,故应用小蓟饮子治疗取得了较好疗效。治疗期间避免性生活是不可忽视的一个方面,过去治疗的失败与没有强调此点有关,应引起高度重视。

【方剂出处】李伯.小蓟饮子治疗血精症31例.安徽中医学院学报,1999,18(4):30

2.安 精 汤

【药物组成】生地黄30g,山茱萸6g,知母10g,黄柏10g,当归10g,紫草10g,牡丹皮6g,苎麻根25g,白茅根30g。

【随症加减】①兼五心烦热,性情急躁,口干喜饮,夜寐盗汗,舌红少苔,以阴虚偏重者,加知母、黄柏、女贞子、墨旱莲;②有肉眼血精颜色鲜红,排尿不适,少腹部坠胀,会阴部隐痛,舌苔黄脉滑数,伴有湿热者,加车前子、萆薢、黄柏;③肉眼血精颜色淡,周身乏力,腰膝痠软,性欲淡漠或阳痿早泄,舌质淡脉细弱伴有脾肾虚者,加芡实、杜仲、续断、淫羊藿;④兼有少腹会阴部刺痛,舌质紫黯或边有瘀斑者,加桃仁、红花。

【治疗方法】每日1剂,水煎,分2次服。7日为1个疗程。

【功 效】滋阴降火,凉血止血。

【临床运用】临床治疗102例,治愈43例,占84.31\%;有效8例,占15.19\%;无效0例。

【经验心得】血精的主要病因是房劳过度,肾虚是主要病机,正如隋代巢元方《诸病源候论·虚劳精血出》所指出的那样“虚劳精血之候。此劳伤肾气故也。肾藏精,精者血之所成也。虚劳则生七伤六极,气血俱损,肾家偏虚,不能藏精,故精血俱出矣。”房劳过度伤精,精损则伤肾,肾阴不足或久服壮阳之品,灼伤阴精,阴虚火旺,精室被扰,络伤动血,而致血精;或素体气血虚弱,无力固涩,血从内溢乃成血精;或久病成瘀或邪热伤精,熬精成瘀,酿成瘀精败浊,此时瘀热与败精合化为腐浊,瘀滞精室,与精俱出而为血精;或性交不洁,湿热之邪从尿道口而入,循经上延,熏蒸精室,迫血妄行而成血精。总而言之,血精患者多属虚证,阴虚火旺是基本,所以临床上立法滋补肝肾,滋阴降火,凉血止血,以求达到标本同治,提高临床疗效。取生地黄、山茱萸滋补肝肾,知母、黄柏苦寒入肾经以益肾阴降火,生地黄、紫草、牡丹皮清热凉血,白茅根、苎麻根以增强止血之效,当归养血活血化瘀以防邪滞难除。诸药合用以滋阴而不助湿,清热而不伤阴,止血而不留瘀。

【方剂出处】鲍身涛,等.安精汤治疗血精症疗效观察(附102例临床分析).中国性科学,2007,16(7):25

3.化瘀利湿方

【药物组成】制香附、乌药、茜草、炒蒲黄、黄柏各10g,海螵蛸、车前子各15g,丹参20g,败酱草30g,牛膝12g,生甘草6g。

【随症加减】会阴部、下腹坠胀,加柴胡,川楝子;睾丸胀痛不适加橘核、荔枝核、小茴香、马鞭草;射精疼痛加石菖蒲、路路通、穿山甲;腰膝酸软加续断、杜仲、狗脊、青风藤;面赤目胀、口干苦加龙胆草、炒栀子;气虚加黄芪、党参、升麻;血虚加当归、制何首乌;目昏、眩晕加女贞子、墨旱莲;血尿加白茅根、大蓟、小蓟;大便干结加制大黄;尿后余沥加薏苡仁、泽兰、瞿麦。

【治疗方法】水煎服,每日1剂。15天为1个疗程,连续服药2个疗程后,观察疗效。并嘱治疗期间禁忌房事、饮酒和辛辣刺激性食物,避免剧烈运动,辅以温水坐浴,每日1次,每次20分钟。同时保持心情愉快、舒畅。

【功 效】行气化瘀止血,清化湿热利窍。

【临床运用】临床治疗35例。治愈28例,有效5例,无效2例,总有效率94.3\%。

【经验心得】本病发病呈年轻化趋势,因前列腺、精囊与排精密切相关,随着人们的生活、物质水平的提高和动物蛋白的摄入过多,或贪食过量刺激性食物,平素又房劳过度,或忍精不泄,酒色劳倦,或过度手淫,禁欲等,劳伤肾精;感受外邪或内外交织,以致湿热毒邪蓄积于下焦,日久导致气血瘀滞不畅,败精浊液,瘀阻精室,血瘀脉外,迫血妄行而血随精流。目前,治疗本病一般常应用大量抗生素,但疗效不能令人满意。由于传统的认识及近代医家一直认为:血精一症,概云以肾虚为主,治疗滥用补益之剂,然临床上用此法,见效甚微,甚至适得其反。根据观察,血精症临床上以气滞血瘀、湿热瘀阻证型较为多见,故治疗上要重视“清化”二字,即行气化瘀止血而不凝滞,化湿利窍而不伤阴,清化湿热导热下行,才能相得益彰。故此,只有在行气化瘀、清热利湿的前提下,疏通其气血,使精道畅通,然后再辅以滋肾之品。方中香附、乌药行气解郁,畅达气血;茜草、炒蒲黄、丹参祛瘀清热凉血;同时茜草配蒲黄又为化瘀止血之良药;海螵蛸收敛止血兼化瘀止血,尤善治泌尿、生殖系统器官出血症;川黄柏、车前子、败酱草清热解毒、利湿通淋;牛膝既能化瘀补肝肾、通淋,又能引药下行,直达病所;生甘草泻火解毒,调和诸药。全方合用,使瘀血得清,湿毒得解,气血畅达,精室得利,血精自愈。

【方剂出处】徐惠华.化瘀利湿法治疗血精35例.中国中医药科技,2008,15(2):153

4.仙鹤地黄汤

【药物组成】仙鹤草30g,生地黄15g,山茱萸9g,黄柏炭12g,牡丹皮炭12g,墨旱莲12g,大蓟9g,小蓟9g,血余炭15g,白茅根15g。

【治疗方法】每日1剂,煎煮2次,取汁400ml,早、晚各服200ml。

【功 效】滋阴益肾,凉血涩血。

【临床运用】临床治疗30例。治愈22例,有效7例,无效1例,总有效率96.7\%。

【经验心得】血精是由精囊炎引起的非特异性感染性疾病,精囊通过射精管与后尿道相通。由于精囊腺的囊壁菲薄,一旦发炎充血,或者出现结石等病变,布满血管的囊壁就容易出血。精囊腺的分泌液是精液的主要成分之一,精囊出血使其分泌液染上血迹,当随着精液排出体外时,就发生了血精。由于精囊与前列腺炎在解剖上紧密的毗邻关系,这两个器官在发炎时也常相互影响,即精囊炎往往与慢性前列腺炎合并发生。血精属于中医学“血证”范围。中医学认为,房劳过度则伤肾,肾阴不足,虚火自炎,梦交或性交之时,欲火更旺,或因外感或因内生湿热,下注精室,精室被扰,血从内溢乃成血精,故以仙鹤地黄汤治之。方中仙鹤草收敛止血,生地黄、山茱萸、墨旱莲滋阴补肾涩血,大蓟、小蓟、白茅根清热凉血止血,黄柏炭、牡丹皮炭、血余炭清热凉血涩血。诸药合用共奏滋阴益肾、清热凉血涩血之效,药力直达精室,湿热得清,肾虚得复,诸症自愈。

【方剂出处】黄向阳.仙鹤地黄汤治疗血精症30例.江西中医药,2008,39(11):48

5.加味知柏地黄汤

【药物组成】生地黄、山药、山茱萸各12g,枸杞子15g,牡丹皮6g,茯苓、泽泻、知母、黄柏、墨旱莲、菟丝子各10g,白茅根、生地榆、仙鹤草各30g。

【随症加减】睾丸痛加荔枝核、延胡索;寐差加首乌藤、远志;出血量多加血余炭、阿胶等。

【治疗方法】每日1剂,水煎2次,分2次服。2周为1个疗程。1个疗程未愈者,休息2日后可再服第2个疗程。服药期间停用其他药物。

【功 效】滋阴补肾,降火利湿,凉血止血。

【临床运用】临床治疗58例,治愈41例,治愈率70.7\%;对照组45例,治愈14例,治愈率31.1\%。

【经验心得】血精是男性生殖系统的疾病之一,多由精囊炎、前列腺炎引起,临床上以青壮年为多见,易反复发作,缠绵难治。近年来其发病率呈明显上升趋势,本病在给人体造成一定损害的同时,也往往给患者带来较大的精神负担,目前对血精的治疗大多采用以抗生素为主的西药,有一定的临床疗效,但不良反应较多,疗效不稳定。本病应属中医学精浊中的赤浊证范畴。《诸病源候论·虚劳精血出候》云:“肾藏精,精者血之所成也,虚劳则生七伤六极。气血俱损,肾家偏虚,不能藏精,故精血俱出也。”《景岳全书·杂证谟·血证》也云:“精道之血,必自精宫血海而出于命门”“多因房劳,以致阴虚火旺,营血妄行而然。”临床所见本病多由肾阴不足,阴虚火旺,下焦湿热,精室被扰所致。因此,治疗应以滋阴降火、凉血止血为基本原则。方中知柏地黄汤滋阴降火,菟丝子、枸杞子补肾益精;墨旱莲益肾养阴,凉血止血;白茅根、生地榆、仙鹤草凉血止血、收敛。诸药合用,共奏滋阴补肾、降火利湿、凉血止血之功。

【方剂出处】李军.加味知柏地黄汤治疗血精58例.四川中医,2006,24(6):56

6.凉血活血汤

【药物组成】姜黄15g,水地丁30g,水牛角30g,琥珀粉5g,苎麻根30g,藕节15g,栀子10g,桑寄生15g,葎草15g,大蓟、小蓟各10g,墨旱莲20g,地榆20g,血余炭15g,白茅根30g。

【随症加减】阴虚火旺者,加知母、黄柏、生地黄、茜草、槐花、白芍、牡丹皮等;湿热内蕴者,加苍术、黄柏、萆薢、土茯苓、龙胆草、木通、薏苡仁、车前子、泽泻等;肾虚不固、气血不足者,加熟地黄、当归、党参、黄芪、白芍、阿胶、菟丝子、杜仲、金樱子等;血精兼有瘀血者,加茜草、三七、桃仁、三棱、莪术等;睾丸或会阴部胀痛者,加川楝子、延胡索、橘核、乌药等;其他如血余炭、藕节炭、蒲黄炭、侧柏炭、白及、白薇等止血治标之药均可随症选用。

【治疗方法】每日1剂,水煎服,早、晚分2次服用,7日为1个疗程。服药期间,禁用其他药物,忌食燥热刺激食物以及饮酒。

【功 效】凉血活血。

【临床运用】临床治疗18例患者,治愈16例,显效2例。

【经验心得】《诸病源候论·虚劳血精出候》说:“此劳伤肾故也,肾藏精,精者血之所成也。虚劳则七情亦极,气血俱虚,肾家偏虚,不能藏精,故精血俱出矣。”另外不洁性交,湿热内蕴或火邪内扰,血热妄行,封藏不固,气血不摄等可导致血精。血精的病机是由于精室脉络受损引起出血,但仍以热伤血络,迫血妄行为多见,瘀血阻络、血不循经也可引起血精。血精的治疗应标本兼施,对阴虚火旺者应滋肾清火,佐以凉血,湿热下注者,先清湿热,后理血调气;虚中夹实者,补肾佐以活血化瘀。本组病例以凉血、清湿热,佐以活血化瘀为基本方,再结合病情辨证加减。其生药水牛角苦咸寒,有清热凉血、解毒之功,可治高热、吐血、鼻出血、血精、尿血等,本品煎剂对动物有强心、缩短凝血时间及镇静作用。琥珀苦寒,有止血化瘀的作用。苎麻根甘寒,有止血利尿、清热作用,治尿血、便血、崩漏、吐血、咯血,经实验证实本品对小白鼠可缩短凝血时间,并有局部止血作用。

【方剂出处】刘日生.凉血活血法治血精18例.临床医学,2003,23(1):61

7.理 血 汤

【药物组成】山药30g,生龙骨18g,牡蛎18g,海螵蛸12g,茜草6g,白芍9g,白头翁9g,阿胶9g。

【随症加减】阴虚重者加生地黄;热重者加龙胆草;大便干结者加大黄;湿热重者加黄柏、苍术、薏苡仁。

【治疗方法】每日1剂,水煎服。

【功 效】清热解毒,凉血活血。

【临床运用】临床治疗32例患者,治愈21例,好转9例,无效2例。

【经验心得】临床观察,血精以热证多见,或因湿热下注,或因阴虚火旺,或因脾肾气虚、瘀血阻滞,或因久病体虚、房室过度等,致热入精室,血络受损,迫血妄行所致。理血汤能补肾泻热、凉血化瘀止血,是治疗血精较理想的方药。张锡纯谓其“治血淋,及溺血、大便下血证之由于热者。”并认为:“血淋之症,大抵出自精道也,其人或纵欲过度,而失于调摄,则肾脏阴虚生热;或欲盛强制而妄言采补,则相火动无所泻,亦能生热,以至血室(男女皆有,男以化精,女以系胞)中血热妄动……”。方中山药、阿胶滋补肝肾;白头翁泻热凉血止血;茜草、海螵蛸凉血化瘀止血;白芍滋阴清热兼利小便,共奏益肾泻热、凉血化瘀止血之效,故疗效满意。

【方剂出处】赵锦令.理血汤治血精32例.中国中医药信息杂志,2002,9(7):51

二、死精子症

死精子症是指排精后1小时内死精子总数超过40\%,6小时内超过80\%。死精子不具备正常的活动能力和受孕能力,是男性不育的常见原因之一。男子因精液异常引起的不育占不育因素的20.5\%,而死精子症占精液异常中的13.23\%。

精子的死亡与精浆的质量有很大关系。生精功能缺陷,发育不良,精子内部结构异常;生殖道有炎症,精子通过有炎症的精囊腺、前列腺;机体营养不良,维生素A、维生素E缺乏,果糖减少,锌含量改变;精索静脉曲张,睾丸缺血缺氧,均是造成死精过多的主要原因。

1.清热利湿活精汤

【药物组成】紫花地丁、蒲公英、川萆薢、炒白术、山药、生地黄、车前子、瞿麦、土茯苓、代赭石、生甘草(以上均为常用量)。

【随症加减】偏于湿胜者重用萆薢;精浆白细胞增多者,重用紫花地丁、蒲公英;阴虚重用生地黄、山药;阳虚重用炒白术、菟丝子;病久多瘀,宜活血化瘀,加丹参、三棱、莪术;睾丸坠胀加川楝子、乌药。

【治疗方法】每日1剂,水煎,分2次服。10天为1个疗程,一般1~3个疗程。

【功 效】清热利湿,平衡阴阳。

【临床运用】临床治疗66例。治愈48例,显效16例,无效2例,总有效率96.96\%。

治验:倪某,男,32岁。2003年11月7日初诊。主诉:婚后6年未孕,体格健壮,无明显不适。脉弦紧,舌质红,苔薄黄。精液量5ml,液化时间40分钟,pH7.3,精子密度111.0×106/ml,a级3\%,b级22\%,c级10\%,d级65\%,伊红染色活率:64\%,顶体完整率48\%,白细胞数2~10/HP。诊断:湿热下注型死精子症。用清热利湿活精汤加减:萆薢15g,紫花地丁15g,蒲公英20g,炒白术15g,山药20g,生地黄10g,车前子15g,瞿麦12g,土茯苓15g,代赭石12g,生甘草5g。连用7剂,11月14日复诊:自述服药后,自觉身体轻松,小便次数增多,脉弦紧,舌质红,苔薄白。精液分析:量2.5ml,液化时间:30分钟,pH:7.9,精子密度:57×106/ml,a级3\%,b级39\%,伊红染色活率:72\%,顶体完整率50\%,白细胞数2~3个/HP。守上方,加丹参连续用药20天,嘱其禁烟酒、辛辣食物、肥甘油腻。2个月后,其妻怀孕,次年生1男孩,身体健康。

【经验心得】死精子症是男性不育的一种类型,特别是湿热内蕴型死精子症,因湿热内蕴精室,致伤肾阴,耗损肾阳,而出现死精子症。实际上死亡前的精子,虽然仍存活,但却完全丧失了受精能力。通俗地说,死精子一定不动,而不动的精子并不是已发生死亡。但通过伊红染色来判断死精子症给我们提供了一个科学的判认死精、活精的有力证据。多年来用清热利湿活精汤加减治疗死精子症,疗效显著,虽有肾无实证之说,但一定要牢记审证求因,辨证论治,才是科学之法,临床上要视其兼证灵活用药,清热利湿,平衡阴阳,气血互生,提高精子活率,达到繁衍后代的目的。

【方剂出处】周建华,等.清热利湿活精汤治疗死精子症66例临床观察.四川中医,2008,26(1):69

2.龙胆泻肝汤合四妙丸

【药物组成】苍术12g,黄柏12g,薏苡仁20g,牛膝15g,萆薢20g,车前草15g,龙胆草12g,栀子12g,赤芍10g,牡丹皮、丹参各15g,蒲公英20g,金银花20g,土茯苓20g,王不留行12g,生地黄10g,山茱萸15g,白花蛇舌草30g。

【随症加减】重用萆薢;热甚者重用黄柏;白细胞升高者,重用白花蛇舌草30g,土茯苓20g。脾湿热得除,精室自安,生精得到复常,可加用露蜂房、菟丝子、枸杞子、蛇床子、丹参配加五子壮阳汤巩固疗效,使精子向正常方面转变。

【治疗方法】每日1剂,水煎,分2次服。7日为1个疗程。

【功 效】清热利湿,活血化瘀。

【临床运用】临床治疗102例,治愈80例,显效15例,无效7例,总有效率93.14\%。

【经验心得】死精症为男性不育症的一种类型。特别是湿热内蕴,郁结下焦,致伤肾阴,耗损肾阳,而出现死精。湿热熏蒸精室,致伤精子,故而死精,湿热下注,宗筋弛纵,导致阳痿,湿热迫精外泄而成早泄。胸脘满闷,纳呆,口中黏腻,大便黏滞不爽,小便黄,为肠胃湿热之表现。舌红、苔黄腻、脉象滑数为湿热内蕴之象。临床上表现腰骶部疼痛,会阴部及下腹部疼痛,尿道灼热感,多属湿热下注膀胱,气滞血瘀,是慢性前列腺炎之象。临床上用龙胆泻肝汤、四妙丸加味清热利湿,活血化瘀,消炎,杀菌,解毒,治死精症效果显著。方中苍术、黄柏、车前草、薏苡仁、龙胆草、萆薢清热利湿,泻火、解毒;牛膝、山茱萸补肾填精;甘草解毒、调和诸药;丹参、王不留行、牡丹皮活血化瘀,改变微循环;金银花、蒲公英、白花蛇舌草消炎抗菌;生地黄滋肾养阴,提高免疫功能。

【方剂出处】符贤才,等.清热利湿解毒法治死精102例体会.中国民族民间医药杂志,2007,(85):92

3.益肾生精汤1

【药物组成】熟地黄15g,山茱萸10g,枸杞子15g,菟丝子15g,山药20g,巴戟天10g,淫羊藿10g,鹿角胶10g(烊冲),制附子10g(先煎),肉桂6g(后下),紫石英10g,五味子6g,覆盆子15g,党参15g,黄芪20g。

【随症加减】阴虚内热者去附片、肉桂,加知母10g,黄柏10g;湿热内蕴者去附片、肉桂,加龙胆草6g,栀子10g,黄芩10g,土茯苓15g,车前草15g。

【治疗方法】水煎,每日1剂,连服30天为1个疗程。

【功 效】温阳补肾,健脾益精。

【临床运用】临床治疗68例。其妻怀孕者30例,显效者22例,有效者8例,无效8例,总有效率88.3\%。

【经验心得】精液是由前列腺液、精囊液、附睾液、尿道球腺和尿道旁腺液组成。根据精气属火为阳,精液属水为阴之阴阳学说,将精液视为阴中之阴,精子则视为阳中之阳。精子本身又可分为阴阳,即精体为阴,阴中之阴;精子存活率及活动力为阳,阳中之阳。根据阳化气,阴成形的理论,精子数量的多少,多责之于肾阴的盈亏;精子活力的强弱,取决于肾阳的盛衰。故治疗精子数量少,主要以滋肾阴为主;治疗精子存活率低,精子活动力差,以温肾阳为主。又由于阴阳之间互相依存,互相制约的特点,往往阴损及阳,阳损及阴,临床出现阴阳两虚的表现,即精子数量少合并精子存活率低,精子活动力差,此时则应该阴阳双补。基于上述思路,自拟益肾生精汤,其中鹿角胶、巴戟天、菟丝子、淫羊藿补肾助阳益精,附子、肉桂温阳补中,更有益于扶助肾阳,此“益火之源,以消阴翳”也;熟地黄、山茱萸、枸杞子、山药滋养肾阴;此“善补阳者,必于阴中求阳,则阳得阴助而生化无穷”也。黄芪、党参补元气、益脾肾;紫石英镇惊暖精宫,利于生精;五味子、覆盆子滋肾涩精。全方配伍,先天后天互补,共奏温阳补肾、健脾益精之功。该方以温为主,使精宫得暖,可祛寒益精,故精子活跃,死精子过多症获得比较好的疗效。

【方剂出处】周瑞芝.益肾生精汤治疗死精子过多症.男科医学,2007,(1):21

4.益肾生精汤2

【药物组成】淫羊藿、熟地黄、肉苁蓉各15g,菟丝子、枸杞子各20g,黄芪30g,当归10g。

【随症加减】辨证属阴虚火旺者,熟地黄改生地黄,加知母、赤芍各15g,蒲公英30g;属湿热下注者,加萆薢、车前子各15g,土茯苓30g;属肝郁血瘀者,加柴胡、赤芍、白芍各10g,郁金15g;属肾气亏虚者,加巴戟天15g,山药30g。

【治疗方法】水煎服,每日1剂,1个月为1个疗程。

【功 效】温阳,补肾,益精。

【临床运用】治疗18例,治愈7例,有效8例,无效3例。

【经验心得】引起死精子症的原因,除了生精功能障碍之外,与精子所处的微环境异常(如男性附属性腺炎症以及附睾炎症、精索静脉曲张、营养不良、微量元素失调)有关。此外,精子活动不良亦可导致不育。中医学认为,精子的产生与脏腑(尤其是肾)、气血功能密切相关。精子的活动有赖阳气的旺盛,而精子的死亡则与生存环境不良(如湿热、气滞、虚火等)有关。因此,治宜采用温阳补肾益精之法,以充精子之源,激活精子活力,并运用祛湿、清热、疏肝、活血、养阴法则,改善精子所在的微环境,保证精子存活。益肾生精汤方中,淫羊藿、菟丝子、黄芪温肾益精、益气补虚为主药;枸杞子平补肝肾而益精血;肉苁蓉温补肾阳而益精血;当归补血并能活血;熟地黄养血滋阴,是为精血互生而设。

【方剂出处】章恪.益肾生精汤治疗死精症.湖北中医杂志,2004,24(8):46

5.五子补阳益气汤

【药物组成】枸杞子15g,车前子15g,五味子10g,覆盆子15g,菟丝子20g,人参10g,黄芪30g,淫羊藿15g,锁阳10g,山茱萸15g,熟地黄15g,沉香10g。

【随症加减】伴湿热者,去熟地黄、人参、山茱萸,加蒲公英、黄柏、白花蛇舌草等;伴肝郁者,去熟地黄、山茱萸、人参,加柴胡、香附、郁金等。

【治疗方法】每日1剂,水煎服。1个月为1个疗程,一般可服3~4个疗程。

【功 效】补阳,益气。

【临床运用】临床治疗300例,治愈255例,有效24例,无效21例,总有效率为93\%。

【经验心得】死精症是男性不育的主要原因之一。精子成活率低于40\%即可诊断为死精症。根据死精症的临床表现及中医“阳主升主动”“气主温煦和推动”“肾主生殖”的理论,认为死精症患者多为肾阳不足和气虚。

【方剂出处】高振东.五子补阳益气汤治疗死精症300例.中国民间疗法,2005,13(12):35

三、无 精 症

无精症是指经过3次以上精液常规检查均末发现精子,此症约占生育期男性人群的1\%,有8.3\%~30\%的男性不育症是由无精子所致,是男性不育症中最严重、最主要的原因之一。

无精症分真假两种,真无精症是因睾丸生精细胞萎缩退化,不能产生精子,又称“先天性无精症”;假无精症是指睾丸能产生精子,但由于输精管阻塞,精子不能排出,故又称阻塞性无精症。无精症临床不多见,但患者基本或完全丧失生育能力。通过中西医结合治疗,也有奏效而怀孕者,但绝大部分是不可逆的,故属绝对不育症的范畴。精液无精子,不能使女方怀孕,是男性不育的根本性疾病。

1.四君生精汤

【药物组成】人参20g,茯苓15g,白术15g,甘草5g,熟地黄50g,山药10g,白芍10g,枸杞子25g,当归30g,制附子10g,泽泻10g,柴胡10g,牡丹皮10g。

【随症加减】少腹胀痛者加延胡索12g,白芷12g;腰膝酸软者加杜仲12g,怀牛膝12g;气虚明显者加黄芪30g;肝肾阴亏,相火炽盛者加龟甲24g,牡蛎30g;心肾不交,肝虚火盛者加酸枣仁24g,炒柏子仁9g,钩藤9g,生龙齿9g,胆南星3g。

【治疗方法】以上药为汤剂,每剂日煎2次,取汁约500ml,分2次饭后服。每个疗程30日,一般服药1~3个疗程。治疗期间忌酒和辛辣之品,停用其他任何治疗本病的药物。

【功 效】滋阴平肝,补肾生精。

【临床运用】临床治疗86例,显效62例,有效18例,无效6例,总有效率93.02\%。

【经验心得】先天肾气亏损,后天脾气不健,肝脏失其生发之令,而致精虫不生。饮食营养之精称为后天之精。肾为先天,脾为后天。先天生殖之精有赖于后天饮食之精滋养才能不断滋养化育,而后天饮食营养,其水谷化为精尤必赖于肾气之温养,肾气充足能使命门相火寄于肝胆,则能助脾胃之消化,故此病宜用补益脾肾平肝法治疗。自拟四味生精汤方中人参、茯苓、白术、甘草为经典四君子汤的组方,为君药;人参补气健脾,脾虚易生湿,配白术、茯苓健脾渗湿,甘草甘温补中。熟地黄滋肾育阴,与山药同用,有滋肾养肝作用,为臣药;枸杞子、当归滋阴补血,制附子温补脾肾,与泽泻同用,增强健脾温肾,化气行水之功效,为佐药;柴胡疏肝解郁,升阳举陷,牡丹皮清热凉血,活血散瘀,以防温补过正,为使药。以上诸药合用有补后天之脾胃,养先天之肾气,滋阴平肝之功效,使精子生成有源,存活有养,为中医治疗无精症之良方。

【方剂出处】韩晓峰.四君生精汤治疗无精子症疗效观察.社区中医药,2005,11(179):35

2.化瘀通精汤

【药物组成】当归20g,桃仁15g,红花15g,赤芍15g,枳壳12g,川牛膝30g,枸杞子30g,菟丝子20g,车前子15g(布包),王不留行15g,穿山甲12g,木通10g,桂枝12g,郁金12g,白芥子12g,甘草9g。

【随症加减】睾丸坠胀疼痛者加川楝子、延胡索;小便浑浊或精液镜检有脓细胞及白细胞者加黄柏、萆薢、金银花、蒲公英;阳痿早泄,性功能减退者,加制附子、锁阳;气虚者加党参、黄精;阴虚者加黄柏、知母;素嗜肥甘、痰浊内盛者加半夏、陈皮、白芥子。

【治疗方法】水煎服,隔日1剂,15剂为1个疗程。

【功 效】先通后补,生精通窍。

【临床运用】临床治疗13例,本组经1~3个疗程治疗,9例痊愈(女方怀孕7例);3例有效[精子数目达(30~50)×109/L,活动率达50\%~60\%];无效1例(经服药2个疗程精子数目≦10×109/L。总有效率92.3\%。

【经验心得】中医学认为,肝主疏泄,性喜条达。情绪抑郁,致肝气郁结,气滞血瘀;或形体肥胖,嗜食肥甘,致痰湿内生下注;或外伤损及生殖器,使瘀血败浊阻塞精道。特别是外伤或精索静脉曲张后,阴囊局部静脉扩张,血液瘀积,增加了阴囊内的温度,使睾丸缺血,造成生精及精子排出障碍,因此,精道瘀阻是导致无精症的重要原因之一,化瘀通精则是治疗无精子症的关键环节。本组病例以先通为主通其精道,后补为辅补其精源,故立化瘀通精汤。方中以血府逐瘀汤活血化瘀,疏肝解郁;辅以枸杞子、菟丝子、山茱萸滋补肝肾;王不留行、穿山甲、白芥子、木通化瘀通精,且木通能宣通气血,穿山甲能走窜经络,无处不到。精道既通,则用补肾益精汤以培补肝肾,助精血化生之源。临证时,不可一见无精症,即言先天不足而大补肝肾,使瘀者更瘀。只有审证求因,通中寓补,补中寓通,或先通后补,才能收到理想效果。

【方剂出处】黄全法.先通后补治疗阻塞性无精子症13例.河北中医,2000,22(1):60

3.茱杞参归汤加味

【药物组成】山茱萸12g,枸杞子12g,党参10g,当归10g,甘草2g。

【随症加减】①肾阳虚,加附片、肉桂、肉苁蓉、淫羊藿、菟丝子。②肾阴虚,加女贞子、墨旱莲、何首乌、生地黄、熟地黄。③肾气虚,加菟丝子、覆盆子、韭菜子、五味子。④气虚,加黄芪、白术、茯苓、山药。⑤兼湿热,加白术、栀子、黄柏、黄芩。⑥兼血瘀者,加牡丹皮、丹参、路路通。⑦精液量少者,加麦冬、天冬、生地黄、熟地黄。

【治疗方法】每日1剂,水煎300ml,分2次口服。根据情况,治疗1个月可停药1周。

同时根据病情和病人的意愿,先后或同时选用了以下的1~5种治疗方法:①ATP片,每日120mg,分3次口服,1~9个月(60例)。②维生素E,每日200mg,分2次口服,1~9个月(58例)。③HCG针,500~2 000U/次,肌内注射,2~3次/周,2万~4万U(53例)。④十一酸睾酮注射液,25mg/次,1个月一次,3~6次(50例)。⑤胰激肽释放酶片,每日720U,分3次口服,1~3个月(43例)。⑥对少精子症者有12例加用了克罗米酚胶囊,50mg/日,口服,3个月。⑦对精子凝集的10例患者加用地塞米松片,1~3天1mg/次,2次/日;4~5天0.5mg/次,2次/日;6~7天,0.5mg/次,每日1次。按此方法服2~4周。⑧对有慢性前列腺炎(脓精)的7例患者采用左氧氟沙星注射液0.4,每日1次,静脉滴注,加阿奇霉素片0.25g/次,2次/日,口服,7天为1个疗程,1~3个疗程。⑨嘱戒烟酒,并对症治疗。

【功 效】温补肾阳,补肝肾阴。

【临床运用】临床治疗85例。经治疗,其妻妊娠35例,其中原发不育28例,继发不育7例。35例中有32例(占91.43\%)用茱杞参归汤加味治疗了4~9个月,另外3例治疗了3个月。治疗不足3个月或超过10个月者无1例其妻获得妊娠。妊娠3个月后失访者25例(包含6例先兆性流产保胎成功),得之顺产健康婴儿10例。

【经验心得】中医学认为,肾主藏精,主发育,为生殖之本;小睾丸性非无精子症首当责之于先天禀赋不足,肾虚为基本病因病理,所以治疗首先应补肾益精;其次,因精血同源,精与气血又互生互化,除补肾益精外,还应补气养血;第三,肾为水火之脏,补肾须阴阳并补;第四,治疗过程中,病情证候也在不断的发生变化,随症加减至关重要;第五,考虑本症治疗时间较长,患者要有较好的依从性,选方用药应本着平和、味少、量中,作用肯定而又无明显毒副作用的原则。方中山茱萸既能温补肾阳,又能补肝肾之阴,为平补阴阳的补肝肾益精血的要药,枸杞子为养阴补血益精良药,二药合用共达补肾益精养血之功,以提高生精功能;党参补中益气,当归补血,二药与山茱萸、枸杞子合用共显补肾益精与补气养血互用之功能;甘草和中益气,兼调和诸药;更有随症加减和结合西药治疗,以使部分患者获得生育的目的。

【方剂出处】程良伟.茱杞参归汤加味治疗小睾丸性非无精子症分析.中国性科学,2008,17(4):33

4.活血化瘀通络汤

【药物组成】桃仁、红花各12g,丹参、路路通、王不留行各25g,皂角刺、刘寄奴、炒穿山甲各10g,川牛膝15g,水蛭6g,当归15g,川芎、甘草各12g。

【治疗方法】上药水煎服,每日1剂,分2次服。15剂为1个疗程。

【功 效】活血,化瘀,通络。

【临床运用】临床治疗95例,治愈14例,显效34例,有效15例,无效32例,总有效率为66.32\%。

【经验心得】本方所治精道瘀阻型无精子症,此证是体内离经之血未能消失,形成瘀血,瘀血影响,气机不畅,经气不利导致精道瘀阻所致。因为两胁与少腹是足厥阴肝经所过之处,经气不利,故见腰痛。足厥阴肝经绕阴器,经气不利故会阴部疼痛,睾丸胀痛,小便余沥。舌黯红,脉涩皆为瘀血之征象。所以治以活血化瘀通络为治则。施以活血化瘀通络汤。方中桃仁辛、苦,具有活血祛瘀之功效;红花辛、微温,具有活血祛瘀、通经之功效,以解瘀血之急。故为君药。丹参苦、微寒,具有活血祛瘀之功效;路路通辛、苦,具有通经络之功效;王不留行辛、甘、平,具有活血通经之功效;皂角刺辛、温,具有托毒排脓、活血消痈之功效;刘寄奴辛、苦、温,具有散瘀、破血通络之功效;炒穿山甲咸、微寒,具有通经、祛瘀、引血下行之功效;水蛭辛、咸,具有破血逐瘀、通经消癥之功效;川芎辛、温,具有活血行气之功,其乃血中之行气药,善于行气,用治血瘀气滞效果颇佳,为使药。甘草甘、平,具有益气补中、清热解毒、缓急止痛、缓和药性之功效,为使药。本方活血祛瘀,活血而不耗血,祛瘀又能生新,合而用之,使瘀祛气行,经气利,精道自通矣,则诸症可愈。

【方剂出处】马存亮.活血化瘀通络汤治疗精道瘀阻型无精子症.中医药学刊,2006,24(6):1164

四、精子活力低下症

对精子活力程度的判断,一般为5级分类法,即0级,精子无活力;1级,活力差;2级,活力中等;3级,活力好;4级,活力很好。精子活力低下症是反映精子活力在2级以下,活动精子数(精子活率)不足50\%者。

1.加味五子衍宗汤

【药物组成】枸杞子20g,覆盆子15g,菟丝子20g,五味子5g,车前子15g,巴戟天10g,淫羊藿10g,山茱萸12g,芡实15g。

【随症加减】偏于气虚加黄芪、党参;偏于阴虚加女贞子、龟甲;偏于精量少加制何首乌、补骨脂、淫羊藿滋阴填髓;偏于精子密度低加熟地黄、山茱萸、山药、附子培补肾阴、肾阳,益精;偏于精子活率低加肉苁蓉、巴戟天、韭菜子以壮阳生精;偏于活力低加鱼鳔、鹿角胶以血肉有情之物大补精血;活力差,有凝集,白细胞增多者,要先清热祛湿,加紫花地丁、瞿麦、土茯苓、黄柏、大黄。

【治疗方法】每日1剂,水煎,分2次服。精子生成周期为74天,故用药一疗程80天,一般用1~3个疗程。

【功 效】益肾填精,平补阴阳。

【临床运用】临床治疗156例。治愈97例,显效40例,有效9例,无效10例,总有效率93.5\%。

治验:汤某,男,24岁。1999年7月13日初诊。主诉:婚后1年零3个月未孕,腰膝酸软,纳呆倦意重,小便次数多,面色白无华,四肢末端冰冷,舌质淡,苔薄白,脉沉细尺弱。精液分析:量1ml,液化时间30分钟,pH7.4,精子密度13×106/ml,a级精子5\%,b级精子4\%,精子顶体完整率25\%,果糖210mg。3天后复查精液,精子密度14×106/ml。诊断:脾肾阳虚型少弱精症。用加味五子衍宗汤加减:太子参15g,黄芪15g,枸杞子15g,覆盆子12g,五味子5g,菟丝子15g,女贞子12g,巴戟天10g,肉苁蓉10g,鹿角胶15g,山药20g。连续用10剂,7月24日复诊:倦意重稍轻,小便次数减少,无其他不适。守原方继续用药,并嘱其禁烟酒、辛辣食物,适当加强营养,节制房事。其妻月经不调并痛经,用中药调理周期,活血理气通经止疼,月经规则,痛经症状消失。此时用药一疗程,复查精液:量3ml,液化时间30分钟,pH7.6,精子密度70×106/ml,a级精子22\%,b级精子32\%,精子顶体完整60\%。当月妻子怀孕,次年生一女孩,聪明健康。

【经验心得】根据临床及现代科学研究五子衍宗汤可明显促进幼鼠睾丸生长发育,提高附性腺的重量;巴戟天、肉苁蓉、山茱萸温肾壮阳,暖精祛寒,提高精子活力;淫羊藿补命门而兴阳道,现代研究可提高精囊腺分泌,可使睾酮分泌增高;枸杞子、覆盆子滋补肾精,养阴益血;精液黏稠,液化时间长,有白细胞及脓细胞,要先清泄湿热,清热解毒。自古以来,虽然有肾无实证之说,但必须辨证施治,灵活用药,临床要视其兼证加减权变,益肾填精,平补阴阳,气血互生,生化无穷,提高精子活力,以达续嗣助育之目的。

【方剂出处】周建华,等.加味五子衍宗汤治疗少弱精症156例临床观察.四川中医,2007,25(11):68

2.升 精 灵

【药物组成】黄芪15g,淫羊藿12g,制何首乌15g,枸杞子12g,菟丝子15g,覆盆子12g,当归12g,熟地黄12g,补骨脂9g,鹿角胶12g,龟甲胶12g,山茱萸12g,炙甘草9g。

【随症加减】偏阳虚者加肉桂3g,附子9g;偏阴虚者加制黄精15g,制玉竹15g。

【治疗方法】每日1剂,煎2次,取汁400ml,早、晚各服200ml,3个月为1个疗程。

【功 效】滋肾育精。

【临床运用】临床治疗62例。治愈24例,有效31例,无效7例,总有效率88.7\%。

【经验心得】《素问·上古天真论》曰:“丈夫……二八,肾气盛,天癸至,精气溢泻,阴阳和,故能有子……七八,天癸竭,精少,肾脏衰。”肾是先天之本,是发育生殖之源。滋肾育精为治疗弱精子症之要旨,据此拟升精灵治疗特发性弱精子症。方中黄芪补中益气、健脾益精;淫羊藿、菟丝子、覆盆子、补骨脂补阳益肾固精;制何首乌补益精血;枸杞子滋补肾阴;山茱萸补益肝肾;当归活血化瘀、通畅经络;熟地黄养血滋阴补精;鹿角胶滋补肾阳益精;龟甲胶滋补肾阴益精;炙甘草补中益气缓急。诸药合用,温阳补气,生精益髓,养血强精,阴阳互补,则精满气充,天癸充盈,精满溢泻,阴阳调和,故能生子。

【方剂出处】黄向阳.升精灵治疗特发性弱精子症62例.山东中医杂志,2009,28(2):89

3.生 精 汤

【药物组成】淫羊藿12g,肉苁蓉12g,沙苑子12g,熟地黄15g,枸杞子15g,制黄精12g,当归12g,菟丝子10g,覆盆子15g,车前子10g。

【随症加减】腰膝酸软、面色萎黄等肾精亏损者,加鹿角霜、桑椹、紫河车;精神抑郁、胸闷不舒、两胁胀痛、嗳气吞酸、不思饮食等肝气郁滞者,加柴胡、陈皮、枳壳、川芎、芍药;精神疲惫无神、性欲低下、少气无力等肾气不足者,加人参、黄芪、鹿角胶;畏寒肢冷、小便清长等肾阳虚者,加制附子、肉桂、巴戟天;倦怠乏力、心悸、失眠、不思饮食等心脾两虚者,加黄芪、山药、龙眼肉、茯苓。

【治疗方法】每日1剂,水煎服,连续3个月。

【功 效】补肾填精。

【临床运用】临床治疗71例患者。在服药结束6个月内,有12例的女方妊娠,成功率为31.59\%。弱精组33例,服药前精子直线运动率11.18\%±3.24\%,服药后精子直线运动率23.24\%±5.38\%(P<0.05);弱精组33例,服药结束以后6个月内,有11例的女方受孕,成功率为33.34\%。

【经验心得】中医理论认为精子的生成依赖于肾阴的营养加肾阳的温煦,精子的多少取决于肾中真阳的盛衰。基于这个理论作者自拟了生精汤,以补肾填精为主要目的,取得良好效果。对其方解如下:淫羊藿、肉苁蓉、沙苑子温肾助阳,益气暖精,提高生精能力;熟地黄、枸杞子滋肾水,生精血,阴阳调和,肾精充盈;制黄精补虚填精,益气健脾,同补先天后天;菟丝子补肾固精,不温不燥,益阴扶阳;当归养血活血,使本方补而不滞,动静相宜;覆盆子固肾涩精,填精补肾,疏利肾气,防其外泄;车前子泄肾中虚火,善行通利。诸药共济,温阳助肾,补肾生精,阴阳互调,肾精充盈,故能助孕生育。

【方剂出处】邱永生.生精汤治少精、弱精症71例报告.江西中医药,1999,30(4):31

4.先清后补法

【药物组成】清法用消炎壮精汤:金银花18g,蒲公英10g,丹参30g,赤芍9g,白芍9g,当归12g,续断15g,生地黄12g,黄柏9g,知母9g,茯苓15g,甘草9g。阳虚者加制附子5g,舌淡红苔白腻者加法半夏9g,面色灰青者加吴茱萸3g,薏苡仁12g,车前子10g。

补法用赵氏育精丸:熟地黄,制何首乌,枸杞子,菟丝子,车前子,女贞子,紫河车,淫羊藿,当归,茯苓,山药,肉苁蓉,丹参,山茱萸(各等份)。按比例量研末,制蜜丸如梧桐子大。

【治疗方法】消炎壮精汤:每日1剂,分2次服,连服10~15剂。然后用赵氏育精丸,每服9g,1日3次,连服15天。先用清法后用补法,共30天。

【功 效】热清湿除,毒去瘀散;补肾生精,益气活血。

【临床运用】临床治疗31例。治愈14例,有效13例,无效4例,总有效率87.10\%。

治验:陈某,男,32岁,2004年11月5日初诊。结婚8年,同居未孕,性生活正常,每周2~3次,查精液常规示密度47×106/ml,a级5\%,b级8\%。有高血压病史,一直服用尼群地平、卡托普利,血压维持在18.62/12.64kPa(140/95mmHg)左右。查两侧精索静脉增粗,女方检查正常。否认有肝炎、结核、腮腺炎病史,有吸烟嗜好10年。无任何不适,舌红苔薄白,脉弦细。诊为原发性不育,治宜清热解毒、补肾活血。用消炎壮精汤10剂,之后用赵氏育精丸9g,每日3次。前后服1个半月中药,2005年5月8日诉:妻子月经已2个月未潮,已怀孕。

【经验心得】弱、少精症病机当属本虚标实;本虚为肾之阴阳偏虚、肾精不足,标实为湿、热、瘀、毒,治疗当宜先清后补,清补结合,而一味滋补治疗,势必导致滋邪留寇。

消炎壮精汤针对湿、热、瘀、毒而立方,具有清热解毒、活血祛湿之效;方中金银花、蒲公英、黄柏、知母清热解毒,生地黄、白芍养阴凉血,丹参、赤芍、当归、续断活血补肾,茯苓健脾利湿,甘草调和诸药。全方可使热清湿除、毒去瘀散。

赵氏育精丸系治疗男性不育症经验方,方中熟地黄、制何首乌、枸杞子、菟丝子、女贞子、紫河车、淫羊藿、肉苁蓉、山茱萸滋补肾之元阴元阳,山药、茯苓益气健脾利湿,当归、丹参活血化瘀。全方气、血、阴、阳兼顾,且补中寓清,滋而不腻,有补肾生精、益气活血之效,能明显提高精子数量和活率活力。

【方剂出处】叶脉延.先清后补法治疗弱少精子症31例.实用中医药杂志,2008,24(12):769

5.人子生精汤

【药物组成】人参30g,五味子10g,菟丝子15g,牡丹皮10g,丹参10g,紫河车15g,附子5g,肉桂10g,仙茅5g,覆盆子15g,何首乌30g,淫羊藿10g,山药20g,枸杞子15g,山茱萸20g,炙甘草3g。

【治疗方法】煎取300ml,每日2次,早、晚分服,每日1剂,3个月为1个疗程。

【功 效】补肾益精。

【临床运用】临床治疗30例。临床痊愈8例,显效12例,有效8例,无效2例,总有效率93.3\%。

【经验心得】人子生精汤用于治疗肾阳虚少精症、弱精症,其中五味子、覆盆子、枸杞子可以提高大鼠下丘脑肾上腺和多巴胺含量,降低5-羟色胺含量,提高血浆睾丸酮含量,降低雌二醇/睾酮比值,从而具有提高大鼠性激素水平和生育力的作用。菟丝子具有人绒毛膜促性腺激素(HCG)样作用,促进生殖系统的发育,促进离体培养的睾丸间质细胞睾酮的基础分泌及HCG刺激的分泌。淫羊藿可能通过调节细胞核的亚显微结构,而调节核的DNA复制和RNA、蛋白质的合成,也可能调节线粒体的亚显微结构而改善细胞能量代谢,从而实现其温阳作用。人参具有HCG样的LH样活性,人参皂苷能作用于睾丸的间质细胞并合成、分泌睾酮,增加阴茎勃起功能和促进精子发生、发育和成熟。紫河车有激素样作用,且能产生促绒毛膜性腺激素,对睾丸有兴奋作用,可促进精子生成,其中所含钙、磷等元素亦可促进精子生成,提高精子成活率及活动力。

【方剂出处】姚文亮,等.人子生精汤治疗肾阳虚型男性少精症、弱精症30例临床观察.中医杂志,2008,49(8):709

6.生精种玉汤

【药物组成】黄芪30g,淫羊藿15g,续断15g,何首乌12g,当归12g,桑椹9g,枸杞子9g,菟丝子30g,五味子10g,覆盆子15g,车前子10g,川牛膝10g。

【治疗方法】每日1剂,水煎取汁500ml,分2次温服。3个月为1个疗程。

【功 效】补肾生精,益气养血。

【临床运用】临床治疗76例。治愈25例,显效23例,有效20例,无效8例,总有效率90.7\%。

【经验心得】少弱精症与中医古医籍中“精少”“精清”“精 冷”相似。中医学对本病的成因、治疗原则和方法早有记载,并有相应的论述。如《内经》云:“丈夫二八,肾气盛,精气溢泻,阴阳和,故能有子。”明确提出了肾气盛衰与男子生育息息相关。生精种玉汤具有补肾生精、益气养血之效。药用淫羊藿、续断、菟丝子温肾壮阳,鼓动肾气,激发生精功能;何首乌、枸杞子、桑椹滋补肝肾,填精化源;覆盆子、五味子固肾涩精,寓养精蓄锐之意;车前子性寒有下降利窍之功,且能泄肾浊、补肾阴而生精液;本方又以黄芪益气,“气行血行”,当归养血,冀气血旺盛,循精血互生之途而益肾精不足;用川牛膝以补肾通经,引诸药同入肾经。方中肾阴阳双补,养血活血同用,益气理气共使,使全方补而不留邪,活而不伤正。

【方剂出处】李轩,等.生精种玉汤治疗少弱精子症76例.中外健康文摘,2008,(5):167

7.补肾清热方

【药物组成】补肾基本方:枸杞子15g,菟丝子15g,女贞子15g,覆盆子15g,金樱子12g,五味子10g。清虚热基本方(白皮饮):地骨皮15g,牡丹皮12g,鸭脚皮15,野菊花15g,金银花叶12g,雪莲花15g,倒扣草15g,青蒿12g。

【随症加减】兼湿虚者加四妙散(苍术、黄柏、薏苡仁、牛膝);兼气滞者加四逆散(柴胡、枳壳、白芍、甘草);气虚者加四君子汤(党参、白术、茯苓、甘草);阴虚者加熟地黄、黄精、何首乌、女贞子、墨旱莲;阳虚者加山茱萸、菟丝子、鹿角胶;血瘀者加丹参、赤芍、毛冬青。虚热内扰重者先以白皮饮加味清其虚热,后补肾益精;虚热不甚者以六子汤合白皮饮加味,或六子汤加味与白皮饮加味隔日交替服用。

【治疗方法】用冷水浸泡30分钟,煎煮30分钟后隔渣顿服,复渣再煎,隔渣再服,每日1剂。

【功 效】补肾生精,清虚热毒。

【临床运用】临床治疗40例患者,治愈12例,有效23例,无效5例。总有效率为87.5\%。

【经验心得】男性不育,并非全由肾虚所致,脾虚、痰湿、火盛、气郁亦是常见的原因。临床所见,相当多患者有咽干口渴、手心热、舌红少苔、脉细数等虚热内扰的证候,现代人由于生活节奏快,工作压力大,精神紧张,夜生活丰富,常致“阳常有余,阴常不足”而出现阴虚火旺证候。前列腺炎、精囊炎等生殖道炎症也是引起男性不育最常见的原因之一。患者常服用抗生素,造成正虚邪恋,虚热内扰。现代男性身患不育之症,备受家庭、社会压力,郁郁寡欢,肝郁化热,灼伤阴液,导致阴的不足,从而虚热内生。因此少、弱精症患者常见虚热内扰的证候。虚热不清,障碍不除,补肾难以奏效且有助邪之弊,所以临床上治疗少精、弱精症患者,补肾配合清虚热法应用,疗效更为显著。但在清虚热过程中要避免损伤肾精,辅以益精之药。在补肾生精的过程中要避免加重虚热之证,佐以清热之药。孰轻孰重,要根据具体情况协调比例,分清主次。白皮饮以清虚热为主,兼清热解毒,活血祛瘀。其中金银花叶、野菊花清热解毒,祛除外邪;雪莲花、地骨皮、鸭脚皮、青蒿、倒扣草、白薇清虚热;牡丹皮既清虚热又活血化瘀。六子汤以枸杞子、菟丝子、女贞子补肾益精;以覆盆子、金樱子固肾涩精。白皮饮与六子汤合用,清补兼施,则虚热清,阴阳平,肾精足,肾气充,故治疗少精、弱精症能取得较好疗效。

【方剂出处】高洪寿,等.补肾清虚热法治疗少精弱精症40例总结.湖南中医杂志,2003,19(6):16

8.温肾育精汤

【药物组成】熟地黄15g,当归20g,巴戟天15g,蛇床子15g,淫羊藿15g,枸杞子15g,仙茅10g,肉桂10g,制附子6g,山茱萸15g,菟丝子15g,鹿角胶10g(烊化)。

【治疗方法】每日1剂,水煎分2次服。随症加用维生素E 0.3g,氯米芬(克罗米芬)25mg,罗红霉素0.1g,天方罗欣0.3g,每日1次,口服,1个月为1个疗程,一般1~2个疗程可治愈。

【功 效】温肾壮阳,补肾益精。

【临床运用】临床治疗282例患者。治愈234例,有效35例,无效13例,总有效率达95.39\%。

【经验心得】男性弱精不育的原因大多是由肾精衰退或命门火衰,功能失常所致。其治法为温肾壮阳,补肾益精。方选自拟温肾育精汤。方中淫羊藿补肾助阳,实验证明淫羊藿含淫羊藿苷等黄酮苷,并含维生素E等成分,对狗有促进精液分泌的作用;能使动物交尾力亢进,给小鼠注射淫羊藿制剂后,通过前列腺、精囊、举肛肌的重量增加法测定,说明本品具有雄性激素样的作用。仙茅、蛇床子、巴戟天、菟丝子,都有补肾壮阳的作用。制附子、肉桂温肾助阳,增强人体精液的活力;熟地黄、枸杞子、当归、山茱萸、鹿角胶,补肾益精。在服中药的基础上,针对不同的病因,配合相应的西药治疗,增加中药的疗效,达到了更加满意的效果。

【方剂出处】忽中乾,等.中西医结合治疗男性弱精不育症282例.四川中医,2003,21(9):52

9.自拟生精汤

【药物组成】熟地黄、山茱萸、肉苁蓉、巴戟天各10~12g;菟丝子、当归、茯苓各12~15g;韭菜子、山药、淫羊藿各15~20g;枸杞子、黄芪各20~30g;紫河车6~10g(冲服)。

【随症加减】伴阴虚火旺者,加盐黄柏、盐知母、生地黄;伴湿热下注者,加黄柏、车前、苍术;伴气虚脾弱者,加黄芪、白术;伴血瘀阻滞者,加丹参、王不留行、川牛膝;伴失眠多梦者,加酸枣仁、远志、五味子;伴食欲不振者,加焦三仙、莱菔子、陈皮;伴阳虚阳痿者,加制附块、肉桂、淫羊藿。

【治疗方法】每日1剂,水煎3次,合液分2次温服。30天为1个疗程,连续服用3个疗程,同时嘱患者禁烟酒。

【功 效】生精气,温阳气,充血气,健脾胃。

【临床运用】临床治疗90例。治愈52例,有效32例,无效6例,总有效率93.3\%。

治验:王某,男,28岁。2005年5月初诊,结婚3年未育,婚后夫妻俩未进行任何避孕措施而妻子未能怀孕。妻子曾在市妇幼保健院进行妇科检查,妇科正常。患者少年时体弱,曾有手淫习惯。诊见;腰膝酸软无力,健忘失眠,勃起不坚,性欲低下,畏寒喜暖,夜尿频多,面色黄白,舌有齿印,脉沉细。3年来患者曾在外院肌注绒促性素,口服克罗米芬,维生素E等西药,疗效不佳。又自服金匮肾气丸、男宝等中成药,疗效甚微,故来我院就诊。治疗前经在我院做3次精液常规化验示:色灰白,质稀薄,量约3ml,液化正常,活动率:42\%,活动力:b级12\%、c级30\%、d级58\%,畸形率:18\%,精子计数16×109/L。诊断:不育症;(①精子活力低下症,②少精子症)。中医证属:肾阳亏虚型。给予补肾填精、温阳促育的自拟生精汤加肉桂、附块、淫羊藿治疗。服药1个疗程(30天)后,患者临床症状明显改善,性功能明显好转,腰膝有力,夜尿减少,舌质淡红,脉沉有力。继续上方治疗2个疗程后临床症状痊愈,经复查精液常规示:色灰白,量约3ml,液化正常,活动率75\%,d级15\%,活动力:a级:40\%,b级:20\%,c级15\%。精子计数:65×109/L,后随访妻子已正常怀孕。

【经验心得】方中熟地黄、山茱萸、肉苁蓉、枸杞子填充肾精,增加精气生化之源;菟丝子、巴戟天、韭菜子、淫羊藿温补肾中之阳气,温而不燥;茯苓、山药、黄芪健脾益气以补后天,加强运化而助先天;黄芪、当归益气补血,使血气充盛;紫河车为血肉有精之品,温阳生精。全方共具生精气,温阳气,充血气,健脾胃之功效。正如《景岳全书》所曰:“故善补阳者,必于阴中求阳,阳得阴助而生化无穷;善补阴者,必于阳中求阴,则阴得阳升而泉源不竭。”故本方在临床上运用,多有效验,不但能够显著增加精子数量,提高精子活力,改善精液质量,同时也可以有效地改善患者的性功能,造福广大男性患者。

【方剂出处】杨德放.自拟生精汤治疗男性少弱精子症90例.现代中医药,2008,28(3):33

10.五味消毒饮

【药物组成】金银花15g,菊花10g,蒲公英20g,紫花地丁15g,丹参25g,黄芩12g,牛膝15g,大黄3g,甘草3g。

【随症加减】阳虚者去金银花、菊花、黄芩,加淫羊藿、巴戟天;肾阴虚者加女贞子、墨旱莲;遗精加金樱子、芡实、五味子。

【治疗方法】每日1剂,治疗30日为1个疗程,连续服药1~3个疗程,从服药开始,每疗程查精液常规1次。

【功 效】清热解毒,燥湿化瘀。

【临床运用】临床治疗43例患者,精子活动率上升至正常者21例;活动率上升,但未达到正常者17例;无效者5例,总有效率88.4\%。

【经验心得】五味消毒饮出自《医宗金鉴》,主治各种瘀毒、痈疮、疖肿,方中金银花、菊花、蒲公英、紫花地丁清热解毒;黄芩燥湿解毒;丹参、大黄活血化瘀,使血瘀生新、又泻火解毒;牛膝引药下行,甘草清热解毒。药理研究证明,金银花、菊花、蒲公英、紫花地丁、黄芩、大黄等分别有抗菌、抗病毒、抗支原体、抗衣原体等作用,可有效地清除生殖道感染;牛膝益肝肾之精气,以增强精子活力,从而使邪去正安,精有所养,而达到育子目的。

【方剂出处】高洪寿,等.五味消毒饮加味治疗男性弱精子症.中国男科学杂志,2001,15(4):281

11.强精玉液

【药物组成】熟地黄30g,何首乌60g,枸杞子60g,菟丝子60g,鹿角胶30g,龟甲胶30g,茯苓30g,补骨脂30g,当归30g,紫河车1具,香附15g,淫羊藿30g,黄芪30g,小茴香15g,三七15g,续断30g,红参30g,鹿茸30g,五味子30g,覆盆子30g,山茱萸60g。

【治疗方法】每次20ml,每日3次,餐前30分钟服,3个月为1个疗程,服用2个疗程。服药期间,加强营养,保持心情舒畅,节制性生活,忌食辛辣肥腻、芹菜,戒除烟酒。

【功 效】补肾固精,活血化瘀。

【临床运用】临床治疗72例患者,治愈38例,好转24例,未愈10例,有效率为86.11\%。

【经验心得】强精玉液参照戚广崇医师强精冲剂组方,经过蒸馏浓缩分离提纯等一系列工艺制成。方中以熟地黄、何首乌、枸杞子、菟丝子、当归、续断、鹿角胶、紫河车、鹿茸等味厚有情之品,以峻补肾精,使其化源充沛;淫羊藿、红参、黄芪、补骨脂以温肾益气;覆盆子、五味子补肾固精助闭藏,又辅以小茴香、三七、香附疏肝理气,活血化瘀,改善循环。

现代医学研究表明:何首乌、当归、续断、淫羊藿等均含有丰富的精子生成和代谢所需的营养物质,如微量元素锌、维生素E等。且补肾药物的作用靶点在下丘脑-垂体,下丘脑-垂体分泌的性腺激素调节睾丸的生精和内分泌功能。强精玉液组方,以滋肾填精为主,为精子的生成提供物质基础,以益气温肾为辅,“少火生气”,为精子运动提供动力,复又以固精活血,改善精子生成及生存的内环境,阳化气,阴成形,代谢旺盛,则少弱精子症均可得到满意改善。

【方剂出处】王国营.强精玉液治疗少弱精子症72例.河南中医学院学报,2003,18(9):56

12.生 精 丹

【药物组成】紫河车30g,山药250g,枸杞子120g,菟丝子60g,覆盆子60g,蛤蚧50g,淫羊藿120g,巴戟天120g,蒿本12g。

【治疗方法】以上碾细过筛,炼蜜为丸,每次9g,每日2次,口服,连用3个月。同时给予注射用人绒毛膜促性腺激素(HCG)2 000U肌注,每周2次,1个月后加用注射用尿人类绝经期促性腺激素(HMG)75U肌注,每周2次。总疗程3~6个月。

【功 效】补肾填精,固本培元。

【临床运用】临床治疗80例患者。治愈44例(其中30例配偶怀孕):精液常规检查,各项指标大致正常,占55\%;有效30例:精子数目及活动力有明显提高,占37.5\%;无效6例;精液无明显改善,占7.5\%。总有效率92.5\%。

【经验心得】肾藏精,主生殖,肾气不足则生精功能低下,故精子数少,活动力差。治疗以补肾填精,固本培元为原则。方中紫河车为无病妇人之胎盘,可补肾精而益气血,配以山药、枸杞子、菟丝子补肾益气填精。淫羊藿、巴戟天、蛤蚧、覆盆子温肾壮阳,固本止泄,蒿本引药入肝经。全方具有温阳益气、填精益髓、温肾固本之作用。据药理研究,淫羊藿有雄性激素样作用,对性腺的发育和提高精子生成率及精子活动率具有一定作用。HCG是一种含有30\%糖类的糖蛋白激素,主要具有黄体生成素(LH)样作用,注射后可直接刺激睾丸间质细胞分泌睾酮(T),促进睾丸精曲小管的生精上皮发育和成熟。HMG主要具有卵泡刺激素(FSH)样作用,与HCG合用可起协同作用。采用中西医结合疗法治疗男性少、弱精症,比单纯用中药或西药效果显著,疗效肯定。

【方剂出处】王旭初,等.中西医结合治疗少弱精症80例.现代中西医结合杂志,2002,11(1):53

五、精液不液化症

精液不液化症,是指精液排出体外后在30分钟以内不自行液化,是影响男子生育力的另一种变化,射出之离体精液黏稠度高,长时间不液化,使陷入精液凝块网络中的精子无法移动,女方难以受孕。促使精液排出体外液化的是前列腺液中所含的纤维蛋白溶酶,如这种酶减少或缺乏,则精液液化时间延长或不液化,这样大大束缚了精子的活动能力,较长时间在阴道内停留而死亡,与卵子不易结合,从而造成不育。

1.归芍地黄汤加味

【药物组成】当归10g,白芍15g,熟地黄12g,山药12g,山茱萸12g,茯苓15g,泽泻12g,牡丹皮15g,黄芪15g。

【随症加减】遗精者加金樱子15g,芡实15g;血瘀加川牛膝12g,赤芍易白芍。

【治疗方法】每日1剂,水煎服,1个月为1个疗程。

【功 效】活血化瘀,补益肝肾。

【临床运用】临床治疗32例。治愈22例,有效7例,无效3例,总有效率90.63\%。

治验:王某,男,35岁。2003年5月初诊。婚后5年不育,女方各项检查正常。近来见腰膝酸软、小便短赤、心烦盗汗、神疲肢倦等,舌红、苔黄腻,脉弦细数。精液常规检查:精液液化时间超过4小时,前列腺镜检查:卵磷脂小体减少,白细胞增多。诊断:不育。证属肾阴不足,脉络瘀滞。方用归芍地黄汤加味。处方:当归、熟地黄、山药、山茱萸、茯苓、泽泻、牡丹皮、金樱子。每日1剂,水煎服。治疗1个月后精液常规检查:精液量3.5ml,液化时间小于40分钟,其妻于2003年9月妊娠,次年顺产一男婴。

【经验心得】精液不液化症是导致男性不育症的主要原因之一,据统计占男子不育症的10\%左右。精液不液化症属中医学精浊范畴。本病多发生于青壮年,正值精力旺盛之时,有手淫、烟酒过度、房事不节、嗜食肥甘厚腻之品,或求子心切、思虑过度,皆可导致肾阴不足,湿热蕴结,精脉瘀滞,致使精液不液化。根据临床症状,拟方以当归(活血化瘀)、白芍(平肝敛阴止汗)、熟地黄(补血滋阴、益精填髓)、山药(益气养阴固精)、山茱萸(补益肝肾、收敛固涩)、茯苓(健脾利水渗湿)、泽泻(利水泄热渗湿)、牡丹皮(清热凉血、活血散瘀)、金樱子(收敛固精,涩肠止泻),诸药合用,疗效肯定。

【方剂出处】万水,等.归芍地黄汤加味治疗精液不液化症32例.中外健康文摘·医药学刊,2008,5(2):96

2.化 精 汤

【药物组成】黄柏12g,败酱草30g,白花蛇舌草30g,虎杖15g,皂角刺10g,赤芍15g,水蛭6g,穿山甲6g,菟丝子15g,淫羊藿15g,甘草6g。

【治疗方法】每日1剂,水煎,分2次服。4周为1个疗程。

【功 效】清热利湿,活血解毒,化瘀行滞。

【临床运用】临床治疗80例。有效70例,无效10例,总有效率87.5\%。

【经验心得】方中败酱草《新修本草》曰:“本品味苦、辛,性微寒,既能清热利湿,又能活血散瘀,善疗内痈”;黄柏味苦,性寒,长于清泻下焦湿热,二者共为君药。白花蛇舌草、虎杖、水蛭、穿山甲、皂角刺、赤芍为臣药,共奏清热利湿、活血祛瘀之效。菟丝子补肾填精;淫羊藿温肾助阳,可防诸药寒凉之弊;淫羊藿补而不滞,温而不燥,甘草调和药性,共为佐使。诸药合用,使湿热去,瘀滞消,精液液化正常,达到治疗目的。现代药理研究表明,败酱草有降低神经系统的兴奋作用,解除前列腺局部肌肉血管痉挛,增加前列腺分泌。黄柏的主要成分之一小檗碱,在前列腺炎组织中也有一定的渗透趋势,从而达到改善前列腺功能的作用。与白花蛇舌草、虎杖共奏清热解毒、化瘀行滞之功效。

活血化瘀药能改善毛细血管通透性,方中水蛭、穿山甲、皂角刺、赤芍活血解毒,化瘀行滞,能减轻炎症反应,促进炎症病灶的消退和吸收,改善结缔组织代谢,促进增生病变的转化和吸收,使腺体微环境改善、腺体分泌功能恢复,促进精液液化。现代药理研究表明:水蛭富含水蛭素、组胺物质、肝素、抗血栓素等,能阻止血液凝固,促进血管扩张,改善微循环,参与精液凝固和液化的调节作用,改善精液的黏稠度和理化特性。现代医学认为,睾酮含量不仅影响精液的成分,而且可以改变精液的凝固和液化。菟丝子提取物能明显促进小鼠睾丸及附睾的发育,实验证明菟丝子具有促性腺激素样作用。

【方剂出处】宋师光,等.化精汤治疗精液不液化症80例临床观察.深圳中西医结合杂志,2009,19(1):46

3.化 液 汤

【药物组成】生地黄、赤芍、白芍、莪术、玄参、桃仁、三七、地骨皮各15g,泽兰、当归、黄柏各10g,黄连6g,黄芪15g。

【治疗方法】每日1剂,水煎2次取汁混合,每次服150g,每日2次,连服8周为1个疗程。

【功 效】活血祛瘀,清热化湿,滋阴补肾,调和阴阳。

【临床运用】临床治疗50例。治愈19例,好转23例,无效8例,总有效率84.0\%。

治验:黄某,男,29岁。2005年3月17日初诊。婚后不育2年余,性生活正常,女方经妇科检查未发现异常。屡次检查精液常规液化时间均2小时以上,无法计数,白细胞(++)。平素喜饮酒,喜肥甘厚味,会阴、腹股沟时有胀坠不适感,尿频,尿灼热感,舌质红苔白微黄腻,脉弦滑。证属湿热内蕴,精脉瘀滞。治以清热化湿,活血祛瘀。用化液汤连服1个月,复查精液液化时间缩短为1小时,活率45\%,活力差,精子计数39×109/L,白细胞(+),症状缓解。续服上方1个月,复查精液常规诸项均正常。再服半个月以巩固疗效,随访其妻怀孕。

【经验心得】精液不液化症多发于青壮年,婚前有手淫不良习惯或婚后房事不节、性欲频繁,或烟酒过度、嗜食肥甘厚味之品,或求子心切、思虑过度,皆可致肾阴不足、肾阳虚衰、湿热壅结、精脉瘀滞,致使精液不液化。治疗当以清热除湿,活血化瘀为主。化液汤方中生地黄、地骨皮、玄参清热养阴,黄柏、黄连泻火坚阴,黄芪补中益气、托毒排脓,当归、三七补血养心、活血祛瘀,莪术、赤芍、泽兰行气破瘀。诸药合用,共奏活血祛瘀、清热化湿、滋阴补肾、调和阴阳之功,药证相符,故疗效较好。

【方剂出处】周伟强,等.化液汤治疗精液不液化症疗效观察.实用中医药杂志,2007,23(3):145

4.清痰利湿汤

【药物组成】瓜蒌、红藤、益母草、白茅根各30g,郁金、三棱、莪术各15g,穿山甲、王不留行、苍术、黄柏、丹参、香附、山楂、路路通各10g。

【随症加减】腰酸膝软者加杜仲、怀牛膝、枸杞子、女贞子、山茱萸;性功能减退者加益智仁、锁阳;早泄者加芡实、金樱子、煅龙骨、煅牡蛎;胃脘胀者加佛手、砂仁;纳食不振者加鸡内金、谷芽、麦芽、神曲;舌苔黄腻者加黄连;舌苔白腻者加厚朴;大便秘结者加制大黄。

【治疗方法】每日1剂,水煎,分2次服。1个月为1个疗程,一般治疗3个疗程。

【功 效】涤痰化浊,清热化湿,活血化瘀。

【临床运用】临床治疗30例,显效11例,有效15例,无效4例。总有效率为87\%。

【经验心得】精液不液化症,其精液黏稠成团成块,精液黏度较高,包裹并限制了精子的活动,使其难以穿透卵子受精。其实质是痰、湿、瘀作祟。痰湿之邪,重浊黏腻,其性凝积,阻碍气机,气行不畅,不能推动血液,便成瘀证。痰、湿、瘀互结,致使精液黏稠不化,乃致不育。本组患者有嗜酒、喜甜食及油脂厚味者,易酿湿生痰,抑郁不化,流注精囊而致病。尿路感染,邪毒内犯,精道瘀浊不畅,久之亦致湿浊阻滞,瘀血内生。瓜蒌涤痰化浊;红藤、益母草、白茅根清热化瘀;郁金、三棱、莪术、穿山甲、王不留行、丹参活血化瘀;苍术、黄柏化湿清热。全方共奏涤痰化浊、清热化湿、活血化瘀之效。但临证还应遵循辨证施治的法则,灵活加减。

【方剂出处】牟吉荣,等.从痰湿瘀论治精液不液化症30例.四川中医,2002,20(3):41

5.清火滋肾汤

【药物组成】生地黄、山茱萸、女贞子、麦冬、玄参、金银花、蒲公英各15g,牡丹皮、知母、墨旱莲、五味子、牛膝各10g,山药20g,黄柏6g。

【随症加减】肾阴虚久则损阳,阴阳两虚伴阳痿者,去黄柏、知母、蒲公英,加菟丝子、淫羊藿、肉苁蓉、肉桂等;伴下焦湿热,尿道灼热者,加紫花地丁、滑石、车前子、瞿麦、土茯苓等;伴失眠多梦,夜卧不安者,加远志、炒酸枣仁、生龙骨等。

【治疗方法】每日1剂,共煎500ml,分早、晚2次口服,连服30剂为1个疗程。

【功 效】清火,滋肾。

【临床运用】临床治疗48例,治愈20例,有效25例,无效3例,总有效率为93.7\%。

【经验心得】本病临床上可分为虚、实两个方面。实证多因湿热下注精室,湿热内蕴,熏灼津液而致精液不能液化。虚证多因恣情纵欲或五志化火,耗伤肾阴,阴虚阳亢,虚火灼耗津液而致精液黏稠不能液化;肾阳亏虚,气化不利,而致精液不能液化。在临床辨证治疗中观察,本病以肾精肾阴亏虚,火热内扰,灼津耗液,使精液黏稠而不液化者居多,故采用清火滋肾法治疗,取得了满意疗效。本法以知母、黄柏、牡丹皮清热泻火为主药,以生地黄、山茱萸茱、女贞子、墨旱莲、麦冬、玄参滋阴补肾为辅药,以金银花、蒲公英清热解毒为佐药,用牛膝引药下行为使药,诸药合用,具有清火滋肾、促使液化的作用。在临床治疗的同时,应让患者禁烟酒,禁食辛辣刺激之品,以及适当节制性生活,并保持心情愉快,方能取得好的疗效。

【方剂出处】阴勇,等.清火滋肾法治疗精液不液化症48例. 陕西中医,2004,25(7):611

6.桂枝茯苓丸加味

【药物组成】桂枝6g,茯苓10g,赤芍10g,牡丹皮10g,桃仁6g,水蛭3g(冲服),地龙10g,夏枯草10g,蒲公英10g,生麦芽30g,败酱草10g。

【治疗方法】每日1剂,水煎,分2次服。7日为1个疗程。

【功 效】活血化瘀,祛痰散结。

【临床运用】临床治疗54例,治愈7例,好转22例,无效7例,总有效率80.5\%。

【经验心得】精液不液化与肾虚、湿热等病理变化有关。因此,在治疗精液不液化时常常以滋阴降火、温补肾阳、清热祛湿为法,而应用活血化瘀、化痰散结之法治疗甚少。精液不液化症表现为精液黏稠或凝结成块,形态现象也与血瘀发生的病理变化极其相似。过食肥甘,超过消化吸收功能或由于脾胃脏腑功能失调,体液代谢紊乱,聚湿成痰,痰随气流,踞于精室,痰精互结,也是导致精液不能液化的重要原因。桂枝茯苓丸加味中选桂枝、桃仁通血脉、行瘀滞;牡丹皮、赤芍凉血清热、散瘀;茯苓淡渗利湿祛痰;水蛭、地龙破血散结,活血通络,化凝精、通精窍;夏枯草、蒲公英清热解毒、散热结、化痰浊、助精液液化;生麦芽助脾胃以运化痰浊。

【方剂出处】贾睿.桂枝茯苓丸加味治疗精液不液化症54例.实用中医内科杂志,2006,20(6):625

7.赛葵水蛭三仁汤

【药物组成】赛葵30g,水蛭10g,薏苡仁20g,杏仁15g,白豆蔻10g,半夏15g,厚朴10g,通草9g,竹叶9g,滑石18g,菟丝子15g,玄参15g,黄芪30g,牛膝15g,萆薢20g。

【随症加减】肾阳虚加淫羊藿、紫河车;肾阴虚加枸杞子、龟甲;气血两虚加红参、当归;生殖道感染加冬葵子、马齿苋、土茯苓。

【治疗方法】每日1剂,水煎,分2次服。20日为1个疗程。

【功 效】清热利湿,祛痰化瘀。

【临床运用】临床治疗56例,治愈46例,有效8例,无效2例。1个疗程治愈31例,2个疗程治愈23例,总有效率96.44\%。

【经验心得】临床观察到精液不液化症,以湿热、痰瘀为主,有伴阴虚者而阳虚等则少见,故治疗以清热利湿、祛痰化瘀为主,辅以滋阴润燥,故投以“葵蛭三仁汤加减”。方中草药赛葵(别名黄花如意、黄花草)具有清热解毒、活血行气、去瘀生新之功;水蛭破血逐瘀,据药理研究,具有抗精液不液化作用,是治疗精液不液化症的首选。取杏仁宣利上焦肺气,盖肺主一身之气,气化则湿亦化,白豆蔻芳香化湿、行气宽中;薏苡仁渗利湿热而健脾,加入滑石、通草、竹叶甘寒淡渗,增强利湿清热之功;以半夏、厚朴、行气化痰湿,散结除痞,萆薢利湿而分清去浊;取玄参养阴清热解毒之功;取黄芪大补脾胃之元气,使气旺、促血行,祛痰瘀而不伤正,并助诸药之力,据药理研究黄芪含有大量锌;取菟丝子既补肾阳又补肾阴之功。诸药相伍,相得益彰,使湿热清、痰瘀化,而兼症除,故可获得较好疗效。

【方剂出处】邓平荟.赛葵水蛭三仁汤治疗精液不液化症56例观察.中国性科学,2007,16(11):27

8.四逆散加味

【药物组成】柴胡20g,白芍20g,炙甘草10g,枳实15g,瓜蒌15g,香附15g,黄芩15g,虎杖20g,夏枯草20g,丹参24g,蛇床子15g,水蛭10g(研末装胶囊)。

【随症加减】血虚加枸杞子、鹿角胶;血瘀加桃仁、红花;肾阳亏虚加巴戟天、淫羊藿等。

【治疗方法】水煎服,每日1剂。15日为1个疗程。

【功 效】理气清热。

【临床运用】临床治疗36例患者。治愈22例,好转11例,无效3例,总有效率为91.7\%。

【经验心得】《灵枢·经脉篇》云:“肝足厥阴之脉……循股阴,入毛中,过阴器……阴器者筋之汇。”肝者,藏血之脏,体阴而用阳,性喜条达而恶抑郁,否则致气机不畅,壅阻下焦,而致湿热,以此立法,从肝论治,兼顾血虚、血瘀、肾虚等证型。四逆散加味方中以柴胡、枳实、白芍、甘草、黄芩、虎杖为君药,疏肝理气,清热利湿;瓜蒌、香附、夏枯草化痰散结为臣药;佐以丹参、水蛭活血化瘀。全方以理气清热为主,攻补兼施,故收良效。

【方剂出处】庞宏永,等.四逆散加味治疗精液不液化症36例.河北中医,2002,24(1):45

9.血府逐瘀汤加减

【药物组成】桃仁、红花、川芎、赤芍、当归、丹参各15g,生地黄、牛膝、柴胡、枳壳各10g,桔梗、甘草各5g,蜈蚣2条。

【随症加减】湿热者加土茯苓;肾阴虚者加枸杞子;肾阳虚者加淫羊藿。

【治疗方法】每日1剂,水煎,分2次服。20天为1个疗程。

【功 效】活血化瘀。

【临床运用】临床治疗30例。显效18例,有效9例,无效3例,总有效率90\%。

【经验心得】精液不液化症属于中医“精稠”的范畴,其病因病机一般认为是外感湿热之邪,或酗酒,过食肥甘,湿热内生,灼伤阴液,致精液不液化;或禀赋不足,肾阴亏虚,大病久病,耗伤肾阴,虚火煎熬精液,故精液不液化;或平素肾阳不足,肾气虚亏,房劳过度,耗伤肾气,气化失司,故精液不液化。治疗多采用滋阴降火、清利湿热、温阳化气之法,方投知柏地黄汤、四妙散、金匮肾气丸治疗,但用之于临床,仍有部分患者效果不好,原因就在于忽视了瘀血的存在。探究原因,精液不液化,病程较长,久病多瘀,并且精血同源,瘀阻精室,也可致精液不液化。

血府逐瘀汤出自清代王清任所著《医林改错》,由桃仁、红花、川芎、赤芍、当归、生地黄、牛膝、柴胡、枳壳、桔梗、甘草组成。该方气血同治,既重在活血化瘀,又能解气分之郁滞,用于精液不液化症,药证相符,方中加入丹参、蜈蚣更添活血化瘀之力。现代医学研究证明,血府逐瘀汤具有良好的改善血液流变性、改善微循环、抗缺氧、抗凝、解除病灶炎性梗阻、促进炎性分泌物排泄的作用,因此,应用血府逐瘀汤治疗精液不液化症有一定的实验基础。

【方剂出处】常建国,等.加味血府逐瘀汤治疗精液不液化症.四川中医,2008,26(5):21

10.液 化 汤

【药物组成】熟地黄20g,山药15g,山茱萸15g,茯苓12g,泽泻12g,牡丹皮12g,枸杞子15g,玄参15g,陈皮12g,菟丝子12g,麦冬15g。

【治疗方法】每日1剂,分2次早、晚空腹服用,每20日为1个疗程,共治疗3个疗程。

【功 效】滋阴降火,益气填精。

【临床运用】临床治疗46例。显效39例,有效3例,无效4例,总有效率91.3\%。

治验:李某,男,29岁。2003年9月12日就诊。主诉:婚后5年未育,性生活正常。同年9月曾两次进行精液常规检查,均为室温60分钟内不液化。平素腰酸无力、头昏、手足心热、舌质红、苔薄、脉细数,证属阴虚火旺。方用液化汤,服药39剂,精液常规检验,室温20分钟内液化完全,精子计数110×109/L,活动率75\%,活动力Ⅲ级,其妻一年后生下一男婴。

【经验心得】精液的正常液化有赖于阳气的气化作用,而阳气的气化又必须依赖阴气的协调,精液为肾所属,故精液的液化与肾的气化有关。因此,凡肾气不足、阴虚火旺、湿热郁滞均可引起气化失常,从而导致精液液化异常。液化汤中,以熟地黄滋肾填精为主,辅以山茱萸养肝益肾、固涩精气,山药益气健脾、助运填精,三药合用,可补肾肝脾之阴精;茯苓淡渗脾湿,可制山药补脾之壅,泽泻泻肾火而利水,可制熟地黄之滋腻,牡丹皮清泄肝火,又制山茱萸之温,三者共为佐药,制三补之偏;玄参、麦冬益阴而壮肾水,枸杞子滋肾补肝,“补益精气,强盛阴道”,菟丝子益三阴而强卫气,陈皮理气调中,燥湿化痰。诸药合用,具有滋肾水而降虚火,益气填精之功。肾气足则气化正常,精液则能液化。现代药理研究表明,熟地黄、山药、山茱萸、茯苓、泽泻、牡丹皮六药合用,具有升高外周血白细胞、改善机体免疫、强心利尿、抗菌抗感染、增强消化功能。玄参、麦冬二药具有抗感染作用,其中麦冬对伤寒杆菌、大肠埃希菌、枯草杆菌、白色葡萄球菌具有较强的抗菌作用;枸杞子所含的枸杞多糖,能显著增强实验动物单核-巨噬细胞系统的吞噬功能,能显著提高吞噬细胞的吞噬功能;菟丝子有雌激素样作用,能增强巨噬细胞吞噬功能,促进抗体生成;陈皮具有抗感染、扩张血管、改善其局部的血液循环的功能。从现代药理学看,本方具有较强的抗菌、抗感染作用。由于前列腺炎是造成精液不液化的主要原因之一,所以用本方治疗精液不液化可收到较好疗效。

【方剂出处】刘增柱,等.液化汤治疗精液不液化46例.中国医药导报,2007,4(26):82

11.温 化 汤

【药物组成】制附子6g,淫羊藿15g,巴戟天12g,生龙骨、生牡蛎各30g,桂枝12g,小茴香6g,生山楂15g,车前子12g,山药21g,炒白术15g,茯苓15g,枸杞子15g,水蛭2g。

【治疗方法】每日1剂,水煎分2次温服,其中水蛭须研末,分2次冲服;生龙骨,生牡蛎须先煎30分钟,然后放入其他药物再煎20分钟即可。服药30日为1个疗程。

【功 效】温肾壮阳,健脾利湿,化痰通络。

【临床运用】临床治疗34例,经3个疗程的治疗后,治愈13例,显效18例,无效3例,有效率91.18\%。

【经验心得】本病的病机有寒热之别,临床表现以肾阴虚或肾阳虚为主。由于肾阳虚导致的精液不液化屡见不鲜,故采用温化汤治疗。方中制附子补肾壮阳散寒;淫羊藿、巴戟天补肾壮阳,药理研究:淫羊藿有雄性激素样作用,能促进性腺功能,增加精液的分泌;生龙骨、生牡蛎化痰湿,软坚散结;桂枝、小茴香温通下焦之气,助气化;山药、炒白术、茯苓、车前子补肾健脾,利湿浊,祛肾浊;枸杞子滋补肾阴而助肾阳;生山楂活血消积导滞;水蛭活血化瘀,通经络。诸药合用,共奏温肾壮阳、健脾利湿、理气化痰通络之效,以促进精液的液化。运用该方药治疗肾阳虚型精液不液化症,疗效可靠。

【方剂出处】曹永贺.温化汤治疗精液不液化症34例.河南中医学院学报,2006,21(1):59

12.滋肾化精汤

【药物组成】山茱萸、知母、牡丹皮、茯苓、女贞子、墨旱莲、黄柏各10g,山楂、枸杞子各15g,龟甲、鳖甲、生地黄、何首乌各20g,黑芝麻30g,麦芽50g,水蛭粉3g(另包冲服)。

【随症加减】兼气虚血瘀者加黄芪、党参、三七、当归、赤芍;肾气虚、性功能减退者加杜仲、巴戟天、补骨脂、菟丝子;肝气郁结者加柴胡、郁金、川楝子;痰瘀互结者加桃仁、丹参、红花、浙贝母、土鳖虫;湿热下注者加蒲公英、薏苡仁、萆薢、车前子。

【治疗方法】每日1剂,加水600~800ml,小火煎至200ml,早、晚分2次服。每月复查精液常规1次。

【功 效】滋肾化精。

【临床运用】临床治疗38例患者,治愈29例,有效5例,无效4例,治愈率76.3\%,总有效率89.5\%。

【经验心得】精液不液化症在中医经典中没有类似的记载,但有相关内容的论述。中医学认为,肾藏精,主生殖,男子不育,精常不足。精液为肾中之阴,阴精不足则内热生,热灼精室则精少而浊,乃致精液黏稠不化。临床观察发现,肾阴虚损证在精液不液化性不育症中最为常见。所用方药,山茱萸、牡丹皮、知母、黄柏、女贞子、墨旱莲、枸杞子、龟甲、鳖甲、生地黄、黑芝麻、何首乌合用滋肾填精,养血和阴,同时又可遏制虚火,故能调节肾脏阴阳平衡,促进精液液化,有利于精子的生长发育和提高性功能;水蛭、土鳖虫善于破积逐瘀,祛瘀生新。茯苓、山楂、麦芽为酸甘化阴之品,药理研究证明能酸化血液,稳定精浆酸碱度,改善液化内环境,不仅可促进精液液化,而且可有效地提高精子的数量和质量;杜仲、巴戟天长于强筋骨,补肝肾,温肾壮阳,增强性功能,调节因阴损阳的功能状态,促进气和精的相互转化,正所谓善补阴者,阳中求阴;蒲公英配合茯苓、泽泻能清化下焦湿热。诸药合用,切合病因病机和临床证型的施治需要。

【方剂出处】张宗圣,等.滋肾化精汤治疗精液不液化性不育症.江苏中医药,2002,23(5):21

13.育阴化精汤

【药物组成】生地黄、熟地黄、玄参、麦冬、菟丝子、蒲公英各15g,地骨皮、白薇、枸杞子、山茱萸各10g。

【治疗方法】水煎服,每日1剂,服药1个月为1个疗程,1个疗程复查1次精液常规,可治疗3个疗程。服药期间忌烟酒,勿纵欲。

【功 效】补肾益精,育阴清热。

【临床运用】临床治疗57例,配偶怀孕者23例,占40.4\%;精液在25℃室温下1小时内液化良好,但配偶未怀孕者25例,占43.9\%;精液液化情况无明显改善者9例,占15.8\%。总有效率为84.2\%。

【经验心得】精液不液化是男性不育症的常见原因之一,液化过程的延迟,能使精子发生凝集或制动,减缓或抑制精子正常通过宫颈而造成不孕。根据精液不液化患者临床上多表现为阴虚火旺的特点,自拟育阴化精汤,方中生地黄、玄参、麦冬滋阴;地骨皮、白薇清虚热;山药、枸杞子、山茱萸、菟丝子健脾补肾、调补肾之阴阳;蒲公英清热解毒。诸药合用,共奏补肾益精、育阴清热之功。临床证明,育阴化精汤治疗精液不液化症疗效较好。

【方剂出处】张宗圣,等.育阴化精汤治疗精液不液化症57例.实用中医药杂志,2002,18(12):17

六、抗精子抗体阳性

抗精子抗体阳性病症,又成为免疫性不育。在男性不育症患者当中,有少数人精液常规检查均在正常范围之内,进行其他检查,例如性激素水平的测定等亦无异常发现。而往往被认为是原因不明之不育症,在临床中发现有一部分不育之原因与患者的自身免疫反应有关。

精子是一种抗原物质。当精子在男性生殖道里时,由于受睾丸中的“隔离小室”、附睾中罩在精子外的“隔离衣”,以及生殖管道保护性屏障之掩护,这种抗原性并没反映出来。但是当输精道有损伤或炎症病变,精液泄漏或渗出到外面组织时,精子就成为机体里的一种“异物”,免疫系统即产生一种对抗本身精子的抗体,将精子破坏或杀灭。此称男子自身精子免疫性不育。大约10\%不育男子发现有抗精子抗体,其发病率占所有不育夫妇病因的3\%左右。我国开展免疫学诊断尚不普及,实际情况可能更多。

1.白皮饮加味

【药物组成】金银花15g,野菊花15g,雪莲花12g,牡丹皮10g,地骨皮15g,青蒿6g,倒扣草15g,白薇15g。

【随症加减】肝肾阴虚者加女贞子15g,墨旱莲15g,白芍12g;阳虚者加山茱萸12g,菟丝子12g,鹿角胶10g(烊化);湿热者加车前子15g,土茯苓15g,龙胆草6g;血瘀者加水蛭10g,毛冬青20g,重楼30g,莪术10g。

【治疗方法】每日1剂,水煎服。1个月为1个疗程。每月复查血清1次。

【功 效】清热解毒,活血祛瘀。

【临床运用】临床治疗43例患者。痊愈36例,无效7例。其中1个疗程痊愈者28例,2个疗程痊愈者33例;另有4例1个月后复查为阴性,但2个月后复查为阳性。

【经验心得】方中金银花、野菊花清热解毒,祛除外邪;雪莲花、地骨皮、青蒿、倒扣草、白薇清虚热;牡丹皮既清虚热又活血化瘀;再分清症状的主次、缓急,辨证用药,从而调节人体阴阳平衡,扶正祛邪,达到标本兼治之效。治标以清除有害的免疫产物,抑制新的抗体继续产生;治本调节阴阳以抑制亢进的免疫功能,并提高机体已被削弱的免疫稳定功能;扶正祛邪以消除炎症、出血、损伤等病因。

【方剂出处】沈坚华,等.白皮饮加味治疗男性血清抗精子抗体阳性43例.湖南中医杂志,2002,18(2):32

2.扶 正 汤

【药物组成】熟地黄、山茱萸、女贞子、菟丝子、黄柏各15g,山药20g,何首乌、皂角刺、覆盆子各12g,丹参、赤芍各10g,红花6g。

【治疗方法】水煎,每日1剂,分2次服。

【功 效】扶正祛邪。

【临床运用】临床治疗27例患者,平均年龄36岁;原发不育18例,继发不育9例,不育年限平均5年;合并有慢性前列腺炎2例,曾患过腮腺炎1例,解脲支原体(UU)阳性9例。治疗结果,治愈5例,显效14例,有效5例,无效3例,总有效率88.9\%。

【经验心得】中医学认为:“正气存内,邪不可干;邪之所凑,其气必虚。”免疫性功能紊乱是一种正虚邪实的表现,虚在肾、脾,邪为湿、热、瘀之邪。脾、肾与免疫功能关系密切,通过补益脾肾,能扶持正气,调动机体抗病能力,提高免疫力,增强稳定性。再辅以清热解毒、祛湿化痰之品,可达到扶正祛邪的功效。本法在治疗中重用补肾健脾之熟地黄、山药、山茱萸、菟丝子、覆盆子、甘草等扶正固本,通过调节下丘脑-垂体-性腺轴的功能而增强睾丸的生精功能,并通过促进微循环来消除覆盖在精子膜上的抗体,从而达到治疗免疫性不育的目的。用黄柏等清热燥湿,何首乌补益精血,丹参、红花活血化瘀,提高人体淋巴细胞的转化率,增强细胞免疫功能,消除脉道阻滞。通过观察,扶正祛邪法治疗UU阳性亦取得了较好的疗效。

【方剂出处】韩兰英,等.扶正祛邪法治疗男性抗精子抗体阳性不育症27例疗效观察.新中医,2002,34(12):23

3.精 宁 汤

【药物组成】生地黄、丹参、益母草、黄芪各30g,山药、枸杞子、赤芍、蒲公英、车前子各20g,牡丹皮15g,桃仁、红花各10g。

【治疗方法】每日1剂,水煎2次,早、晚分服。

【功 效】滋阴,活血,解毒,利湿。

【临床运用】临床治疗60例患者,其中伴前列腺炎22例,慢性附睾炎3例,支原体或衣原体感染6例,精索静脉曲张7例,睾丸外伤史1例。治疗组AsAb转阴53例,转阴率为88.33\%;配偶妊娠19例,妊娠率为31.67\%。

【经验心得】男性免疫性不育症的基本病机是本虚标实。本虚即肝肾阴虚,标实乃湿热、邪毒、瘀血壅阻精宫。肾藏精、主生殖,肝藏血、主疏泄。肝肾精血同源,阴阳互补,共同激发、推动和协调全身组织器官的生理活动,对免疫功能起稳定和调节作用。肝肾阴虚,阴阳违和,则机体免疫调节失衡;复加内郁、外伤、感染邪毒等原因而致瘀血阻滞经络,湿热邪毒扰乱精宫,影响生殖之精的生长化藏,导致液化不良或弱精、死精、畸精等,从而引起不育。治宜滋补肝肾、活血祛瘀、利湿解毒为法。精宁汤中生地黄、山药、枸杞子补益肝肾阴血;丹参、益母草、赤芍、牡丹皮、桃仁、红花活血化瘀;车前子清肝利湿,蒲公英清热解毒;黄芪补益后天脾胃,以滋精血生化之源,又能补气以助血行。全方共奏滋阴、活血、解毒、利湿之功。现代中药药理学研究证明:生地黄能调节生殖轴活动并有免疫抑制作用;赤芍能阻止抗体的形成;益母草有抑制体液免疫和细胞免疫的双重功能;黄芪具有很好的免疫双向调节作用;蒲公英、牡丹皮能广谱抗菌,抑制病原微生物,并能清除炎性代谢产物;桃仁、红花、丹参、益母草等能改善微循环及血流动力学,消除组织缺血缺氧,改善精子生长发育的内环境,提高精子质量。

【方剂出处】徐丹,等.精宁汤治疗男性免疫性不育症60例临床观察——附西药治疗30例对照.浙江中医杂志,2002,(3):114

4.黄芪二仙汤

【药物组成】黄芪15g,仙鹤草30g,淫羊藿15g,丹参15g,白花蛇舌草15g,当归10g,红花10g,黄柏10g,生地黄15g,山茱萸10g,牛膝15g,车前子10g。

【治疗方法】每日1剂,水煎服,3个月为1个疗程。治疗中每月查AsAb一次,在未转阴前,采用避孕套过性生活,已转阴者,继续服药,同时过正常性生活。女方怀孕后即停药,未怀孕者继续治疗至2个疗程结束并随访。

【功 效】活血化瘀,解毒益肾。

【临床运用】临床治疗72例患者。经2个疗程治疗后,治疗组中有69例AsAb转阴,占96\%;在治疗第1个月转阴者13例,第2个月转阴者43例,第3个月转阴者9例,第6个月转阴者4例。

【经验心得】中医学认为,肾主生殖,病久必瘀,故应标本兼治,方以解毒、活血、益肾为主,方中仙鹤草、黄柏、白花蛇舌草、车前子清利湿热而祛浊毒,清除炎性代谢产物,抑制致病微生物,给组织修复创造基础条件;丹参、红花、当归活血散瘀通经络,改善血流动力学,消除组织缺血缺氧状态,促进炎症吸收;黄芪、淫羊藿、生地黄、山茱萸、牛膝补肾填精,扶助正气,增强抗体免疫细胞作用。药理研究表明,生地黄能调节生殖轴活动,并有抑制抗体的作用,丹参可清除血液中过剩抗原,防止免疫复合物产生;黄芪、丹参、白花蛇舌草可以增强机体免疫力。诸药合用,在消除病因的同时能迅速改善精子产生的微环境,消除AsAb,消除精子凝集,提高妊娠率。

【方剂出处】娄灿荣.黄芪二仙汤治疗男性免疫性不育72例.中国民间疗法,2002,10(9):43

5.虎杖丹参饮

【药物组成】虎杖、蒲公英、紫草、黄芪、丹参、赤芍、当归、何首乌、女贞子、生地黄、淫羊藿各15g,红花10g。

【随症加减】湿热偏重者加败酱草20g,黄柏9g;血瘀偏重者加三七粉(冲服)3g;肾虚偏重者加巴戟天15g。

【治疗方法】每日1剂,水煎,分2次服。

【功 效】调节免疫,活血化瘀。

【临床运用】临床治疗50例,痊愈41例,无效9例,总有效率为82\%。

【经验心得】本病治疗应以清利活血补肾为治疗大法,以虎杖丹参饮为主方。方中虎杖、蒲公英、紫草清利湿热、活血解毒;丹参、赤芍、当归、红花活血化瘀;淫羊藿、何首乌、女贞子补肾填精益髓。全方既能调节免疫,又能抗菌消炎,改善微循环,从而改善精子运动参数,提高妊娠率。

【方剂出处】邹强.虎杖丹参饮治疗男性抗精子抗体阳性不育症50例疗效观察.新中医,2005,37(3):43

6.化痰祛瘀汤

【药物组成】丹参、赤芍、三棱、莪术、茯苓各15g,穿山甲、皂角刺各10g,川芎9g,胆南星、柴胡各6g。

【治疗方法】每日1剂,水煎,早、晚分2次服。30日为1个疗程。

【功 效】活血化瘀,健脾化痰。

【临床运用】临床治疗30例,治愈12例,有效13例,无效5例,总有效率为83.3\%。

【经验心得】临床上发现,多数患者除不育外,多伴有眩晕、口干、胸腹痞满、舌下静脉曲张、脉涩或滑等症状,外观精液黏稠凝聚,与中医辨证痰浊瘀阻型的表现颇为吻合。故治以化痰祛瘀法可取良效。基本方中丹参、赤芍、川芎、莪术、穿山甲、皂角刺活血化瘀,软坚散结,并可降低精液的黏稠凝聚;胆南星、茯苓健脾化痰浊,促使AaAb滴度下降;柴胡为引经药,可使药物直达病所。诸药合用共奏活血化瘀、健脾化痰之效。

【方剂出处】李凯英,等.化痰祛瘀法治疗男性抗精子抗体不育症30例疗效观察.新中医,2005,37(4):48

7.抑抗转阴汤

【药物组成】淫羊藿15g,肉苁蓉12g,菟丝子12g,女贞子12g,枸杞子15g,丹参15g,益母草15g,鸡血藤20g,红花6g。

【随症加减】伴有湿热下注,症见阴部坠胀,小便黄赤或涩痛,舌质红,苔薄黄,脉滑数者,加黄柏12g,蒲公英15g,白花蛇舌草15g。

【治疗方法】每日1剂,水煎服,早、晚各1次,15个月为1个疗程。服用1个疗程后AsAb仍为阳性者,继续服用第2个疗程。总疗程为3个月。

【功 效】以补肾为主,以活血为辅。

【临床运用】临床治疗106例患者,转阴率达90.6\%,96例转阴病例停药6个月内,随访配偶怀孕者59例,受孕率61.5\%。

【经验心得】中医多认为AsAb与瘀血阻滞、湿热下注有关。活血化瘀和清热解毒类中药有抑制免疫反应,减少抗精子抗体生成的作用。黄宇烽以滋肾阴壮肾阳,活血化瘀为治则,治疗男性血清AsAb阳性者62例,转阴率93\%;治疗男性精浆AsAb阳性者38例,转阴率89.5\%。医者自拟抑抗转阴汤,以补肾为主,以活血为辅,伴湿热下注者加用清热利湿药,治疗男性免疫性不育症。抑抗转阴汤中淫羊藿、肉苁蓉、菟丝子补肾壮阳,能显著地促进机体免疫功能;滋阴补肾的女贞子、枸杞子也具有相同的作用。女贞子通过增强细胞表面受体活性,促进T细胞活动,发挥免疫作用;枸杞子能提高抗体效价和增强抗体形成细胞数量,其提取物枸杞子多糖对免疫有双向调节作用。方中丹参、益母草、鸡血藤、红花为活血化瘀药,对体液免疫和细胞免疫的不同环节均有抑制作用。如红花、鸡血藤、丹参对已沉积的AsAb复合物有促进吸收和消除作用,益母草可抑制抗原抗体免疫反应的病理损害。黄柏、蒲公英、白花蛇舌草为清热解毒药,这些药物一方面对生殖道有较强的抗菌消炎作用,另一方面能抑制异常免疫反应。

【方剂出处】郑文华.抑抗转阴汤治疗男性免疫性不育症106例.广西中医药,1999,22(3):24

8.滋肝肾消抗汤

【药物组成】生地黄、熟地黄、制何首乌各20g,山茱萸、牡丹皮、墨旱莲、沙苑子各12g,生山药、女贞子、黄精、桑椹子各15g。

【治疗方法】水煎服,每日1剂,分2次服。

【功 效】滋肝肾,消抗体。

【临床运用】临床治疗101例。男性101例,血抗精子抗体转阴61例,精液抗精子抗体转阴60例,总转阴率60.24\%。

【经验心得】免疫性不孕,首在肝肾,正虚为肝肾之虚。因此,肝肾阴虚型抗精子抗体阳性免疫性不孕应以滋补肝肾消抗为治则,以滋补肝肾消抗汤为方剂。方中选用;生地黄、熟地黄、生地黄性味甘、苦、寒具有清热凉血,养阴生津之功效;熟地黄性味甘、微温具有补血,滋阴之功效,善于滋阴,常用于肝肾阴虚;山药、山茱萸配伍治疗肝肾阴虚,山茱萸性味甘、酸、温,具有补益肝肾,收敛固涩之功效,生山药性味甘、平具有补脾胃,益肺肾之功效;牡丹皮性味苦、辛、微寒,具有清热凉血,活血散瘀之功效;女贞子性味甘、苦、凉,具有补养肝肾之功效,《证治准绳》二至丸,以本品与墨旱莲合用,《简便方》加入桑椹,效力尤著;墨旱莲性味甘、酸、寒,具有养肝益肾,凉血止血之功效;制何首乌性味甘、苦、涩、微温,具有补肝肾,益精血之功效;黄精性味甘、平,具有润肺、滋阴、补脾之功效,《奇效良方》补虚益精血,与熟地黄配伍以增加滋补阴血的作用;桑椹子性味甘、寒微,具有滋阴补血之功效;沙苑子性味苦、辛、平,具有平肝疏肝之功效。方中重用生地黄、熟地黄为主,滋肾填精,滋阴养血以补肝肾;辅以山茱萸、女贞子、墨旱莲、黄精、桑椹子滋阴养肝肾,生山药补脾阴,多药合用,以达到三阴并补之功效,这是补一面。牡丹皮、沙苑子清泄肝火,并制山茱萸之温,这是泻一面。诸药合用,使之滋补而不留邪,降泄而不伤正,补中有泻,寓泻于补。肾气充盛,肝阴得养,肝气条达,故正气存内,邪不可干,则抗精子抗体转阴。

【方剂出处】马存亮.滋肝肾消抗汤治疗抗精子抗体101例.陕西中医,2004;25(6):512

9.补肾化瘀方

【药物组成】党参、菟丝子、女贞子、地黄、淫羊藿各15g,丹参、当归、桃仁、炮穿山甲、王不留行各10g,炙甘草6g。

【治疗方法】每日1剂,加水700ml,煎成150ml分早、晚空腹口服。疗程3个月。

【功 效】补肾化瘀。

【临床运用】临床治疗50例。痊愈14例,好转28例,无效8例,总有效率为84.0\%。

【经验心得】肾藏精,精生髓,主生殖,其中骨髓是免疫系统的中枢免疫器官,在免疫应答及免疫调节过程中起重要作用。肾虚则孕育无能,免疫功能异常。临床补肾化瘀法通过整体性的调节作用,既可以提高机体已被削弱的免疫稳定功能,又可清除有害的免疫产物。本方有促进机体免疫作用,如党参有促肾上腺皮质功能及增强网状内皮质系统的吞噬作用;菟丝子能增加T细胞的比值;女贞子、地黄可抑制免疫功能亢进;当归、桃仁、炮穿山甲、王不留行等对体液免疫与细胞免疫有一定的抑制作用,不仅能减少已成的抗体,而且能抑制抗体形成;丹参对已沉积的抗原抗体复合物有促进吸收和消除的作用;甘草有类激素样作用,甘草的粗提物多糖体是抗体抑制因子,能抑制抗体的产生。补肾化瘀法使机体阴阳平衡而改善机体的免疫功能,清除已形成的抗体并抑制新抗体产生,从而使血清或精浆中的AsAb消失。

【方剂出处】李永生.补肾化瘀法治疗男性抗精子抗体阳性不育症50例.陕西中医,2007,28(11):1506

10.消抗种子方

【药物组成】金银花20g,蒲公英30g,紫花地丁30g,黄芩20g,制大黄5g,淫羊藿15g,丹参20g,当归10g,川芎10g,生黄芪20g,生地黄20g,生甘草5g。

【治疗方法】每天1剂,水煎分2次温服。

【功 效】清热解毒,调节免疫。

【临床运用】临床治疗55例。痊愈38例,有效14例,无效3例(其中2例因工作外出中断治疗),总有效率94.55\%。

【经验心得】方中金银花、蒲公英、紫花地丁清热解毒;黄芩燥湿解毒;大黄泻火解毒;丹参、当归、川芎活血化瘀,祛瘀生新;生黄芪托毒生肌;生地黄凉血解毒生肌;生甘草清热解毒生肌;淫羊藿温肾助阳,矫正凉药偏性。所选的中药可能具有清除抗原,抑制抗体,调节免疫,修复免疫屏障的作用,所以取得较好的疗效。

【方剂出处】祁天寿,等.自拟消抗种子方治疗抗精子抗体阳性男性不育的临床观察.中国中西医结合杂志,2007,27(11):983

11.理精消抗汤

【药物组成】丹参15g,桃仁10g,当归10g,川牛膝10g,柴胡10g,黄芪15g,淫羊藿10g,牡蛎30g(先煎),甘草5g。

【随症加减】伴尿急尿痛、小便黄赤、阴部湿痒、舌质红、苔黄、脉滑数或弦数,湿热下注者,加黄柏10g,白花蛇舌草30g,萆薢15g,车前子10g;精索静脉曲张,阴囊坠胀,气滞血瘀甚者,加莪术10g,王不留行10g,荔枝核15g;性欲减退、精少阳痿,肾虚精亏者,加菟丝子10g,枸杞子10g,熟地黄12g,蜈蚣2条。

【治疗方法】每日1剂,水煎服,早、晚各1次。6周为1个疗程,1个疗程AsAb仍未阴转者,再续治1个疗程,最多治疗2个疗程。

【功 效】理气活血,养血益肾,通精消抗。

【临床运用】临床治疗64例患者。不育时间为45.6±27.4个月。治疗结果,总转阴率达到82.8\%。

【经验心得】中医学认为,肾为先天之本,藏精,主生殖。肾气能激发和推动机体组织器官的生理活动,有类似于现代医学中“下丘脑-垂体-肾上腺皮质”系统的功能,对免疫功能起到稳定调节作用。方中丹参、桃仁、当归、川牛膝,既可活血化瘀,疏通脉络,又能养血濡精,使瘀血去,新血生。现代药理研究证明,活血祛瘀药物具有调整全身或局部血液循环,特别是微循环,改善血液物理化学性状,抑制病原体及炎症反应,调节免疫功能等作用。柴胡疏肝理气;黄芪益气补虚,固护藩篱,又可“逐五脏间恶血”(《别录》)。二者相合,既能理气和血,又可益气行血。黄柏、白花蛇舌草、萆薢、车前子等清热解毒利湿,这些药物一方面能对生殖道有较强的抗菌消炎作用,另一方面能抑制异常的免疫反应。甘草调和诸药,且具肾上腺皮质激素样作用,能抑制炎症反应及免疫抑制作用。诸药配合,契合病机,相得益彰,共奏理气活血、养血益肾、通精消抗之效。

【方剂出处】李其信,等.“理精消抗汤”治疗男性免疫性不育症的临床研究.江苏中医药,2003,24(7):14

12.抑抗促育汤

【药物组成】当归10g,丹参15g,赤芍24g,牡丹皮12g,桃仁10g,川牛膝15g,徐长卿10g,萆薢10g,黄柏10g,薏苡仁20g,益母草18g,淫羊藿10g菟丝子15g,枸杞子15g,甘草6g。

【治疗方法】每日1剂,水煎服,1个月为1个疗程,一般用药2个疗程。

【功 效】活血祛瘀,疏通经络。

【临床运用】临床治疗56例患者,经2个疗程治疗,AsAb转阴性且治疗后其配偶6个月内怀孕者24例,AsAb转阴性但其配偶未受孕者22例,AsAb仍阳性10例。

【经验心得】男性免疫性不育症的治疗,西医多从抗感染、手术、免疫抑制、睾酮的应用及精子洗涤和宫腔内人工授精等方法着手,但疗效不甚满意,而且有一定的不良反应。根据男性免疫性不育症的病理特点,本病系瘀血内阻、湿浊蕴结下焦、肾不生精所致,故自拟抑抗促育汤,取当归、丹参、赤芍、牡丹皮、桃仁、川牛膝、徐长卿等活血祛瘀、疏通经络,促进局部新陈代谢;萆薢、黄柏、薏苡仁、益母草等除湿降浊,改善生精环境;菟丝子、枸杞子、淫羊藿益火生精,阴阳并补,改善生精功能,从而使瘀阻消散,湿除浊降,肾之生精功能恢复正常。从现代医学的角度来看,活血化瘀类药物如牡丹皮、丹参、当归、赤芍、桃仁等均有抗炎作用,能降低毛细血管的通透性,减少炎性渗出和促进吸收,有抑制细胞和体液免疫作用;而徐长卿有广泛的抗免疫作用,从而为本病的中医治疗提供了依据。

【方剂出处】左恒.化瘀降浊法治疗男性自身免疫性不育症56例.安徽中医临床杂志,2003,15(2):119


\subsection{第6章 附睾炎}

附睾炎是常见的男性生殖系统疾病之一,附睾炎多由邻近器官感染蔓延所致,表现为阴囊部位突然性疼痛,附睾肿胀,触痛明显,可伴有发热,附睾硬结等。附睾炎有急性和慢性之分。急性附睾炎多见于青壮年,患者多处于气血充沛,精力旺盛时期,若所欲不遂,手淫过度,或意外损伤可使气血瘀滞;若房事不节(不洁)可使虚火内炽或感染毒邪。慢性附睾炎是由于细菌感染等因素引起附睾局部炎症,本病常继发于前列腺炎、精囊炎,或伴发睾丸炎等病,发病年龄以青壮年多见,大多为单侧发病,亦可有双侧同时发病,其病复杂,病程长,顽固难愈,极易复发。

1.三黄二香散

【药物组成】大黄、黄连、黄柏各20g,乳香、没药各10g。

【治疗方法】共研极细末,加米醋适量调为糊状,涂敷于患侧阴囊,厚为0.3~0.5cm,以纱布覆盖,每日换药1次。同时结合病情轻重,适量给予静脉输液或口服抗生素。并嘱患者垫高阴囊。

【功 效】清热利湿,理气活血。

【临床运用】适用于急性附睾炎。临床治疗37例患者,显效14例,有效23例。所有病例均在5~7日附睾炎性浸润消退,10~14日附睾恢复正常。

【经验心得】急性附睾炎是常见的男性生殖系统非特异性感染性疾病,隶属于中医学“子痈”范畴。《外科证治全书》载:“肾子作痈,下坠不能升上,外观红色者,子痈也。或左或右,故俗名偏坠,迟则溃烂莫治。”本病由于湿热下注厥阴之络,以致气血凝滞而成。药理研究报告,大黄蒽醌衍生物有强大的抗菌作用,其中以大黄酸、大黄素和芦荟大黄素抗菌作用最强,对金黄色葡萄球菌、链球菌、大肠埃希菌等有较强的抗菌作用;黄连、黄柏的主要成分为小檗碱,有广谱抗菌作用,对痢疾杆菌、金黄色葡萄球菌及溶血性链球菌作用最强。《本草纲目》则载有:“乳香活血、没药散血,皆能消肿生肌,故二药每每相兼而用。”方中以大黄泻火解毒、祛瘀通络;黄连、黄柏清热燥湿,泻火解毒;乳香、没药活血行气,消肿止痛。且阴囊皮肤血液循环丰富,药物通过皮肤吸收直达病所。故以三黄二香散外敷治疗急性附睾炎,能取得较为满意的效果。

【方剂出处】刘建国.三黄二香散外敷治疗急性附睾炎37例.中医外治杂志,2002,11(2):25

2.四逆散加味

【药物组成】柴胡10g,白芍15g,枳实10g,炙甘草6g,荔枝核10g,橘核10g,浙贝母15g,郁金10g,桃仁10g,蒲公英20g。

【随症加减】疼痛明显,以气滞为主者加川楝子、延胡索;以血瘀为主者加乳香、没药;热毒重者加败酱草、白花蛇舌草;湿重者加薏苡仁、土茯苓;附睾硬肿消退缓慢者加玄参、牡蛎;肾阳虚者加仙茅、淫羊藿;肾阴虚者加女贞子、墨旱莲;若偏寒凝经脉者,去蒲公英,加吴茱萸、肉桂、小茴香。

【治疗方法】每日1剂,煎服2次,20剂为1个疗程,每疗程间隔10日,治疗1~3个疗程。治疗期间忌饮酒及过冷过辣食物。

【功 效】疏肝解郁。

【临床运用】适用于慢性附睾炎。临床治疗32例患者,治愈25例,好转7例。

【经验心得】慢性附睾炎是男科病中比较常见的疾病之一。其病因大多为纵欲过度、不洁性交、淋证失治、性压抑、驾车劳累、嗜烟酗酒等,其病机主要在于足厥阴肝经气郁血滞、湿热或湿浊结聚成痰,阻塞经络。故采用以疏肝解郁为主要功效的四逆散加味进行治疗,方中柴胡、白芍疏肝柔肝,宣畅气机;枳实、荔枝核、橘核、浙贝母行气化痰散结;郁金、桃仁开郁化瘀通络;蒲公英解毒消肿;炙甘草缓急和中。诸药合用,共达病所而获满意疗效。

【方剂出处】蒋政余.四逆散加味治疗慢性附睾炎32例观察.湖南中医杂志,2001,17(2):29

3.桃核承气汤

【药物组成】桃仁15g,桂枝10g,大黄10g(后下),芒硝20g(冲服),甘草10g,金银花30g,蒲公英30g,赤芍20g,延胡索12g,香附12g,乌药12g。

【随症加减】红肿消退后,仍有触痛及结节者,原方去芒硝,加荔枝核15g,橘核15g,发热者加苦参30g,赤小豆30g,龙胆草30g;痛甚者加全蝎10g,小茴香10g;下坠感明显者加炙升麻10g。

【治疗方法】每日1剂,水煎2次,取900ml,分3次服。

【功 效】通下泄热,化瘀开结。

【临床运用】临床治疗32例患者,治愈30例,有效2例,无效0例,有效率为100\%。

【经验心得】急性附睾炎多见于青壮年,患者多处于气血充沛,精力旺盛时期,若所欲不遂,手淫过度,或意外损伤可使气血瘀滞;若房室不节(不洁)可使虚火内炽或感染毒邪。气滞血瘀,热毒内积,凝聚不散,结于下焦而成本病。热毒内积为其病因,瘀阻不行是其病机,结聚下焦是其病理结果。治疗关键是开启瘀结,荡涤热毒。桃核承气汤出自《伤寒论·太阳篇》,用于治疗热结少腹,其人如狂的蓄血证。用其通下泄热,化瘀开结,又加金银花、蒲公英清热解毒;延胡索、乌药、香附理气止痛。热毒得清,瘀血得祛,气血通畅,故而临床可见患者大便得下之后疼痛立减,继而加用通络散结之品,巩固疗效,竟收全功。

【方剂出处】陆保磊.桃核承气汤治疗急性附睾炎32例.河南中医,2003,23(7):38

4.橘核汤加味

【药物组成】橘核、荔枝核各15g,川芎、赤芍、车前子、当归、桃仁、地龙、小茴香各10g,生黄芪、生牡蛎各30g,肉桂6g。

【治疗方法】每日1剂,水煎,分2次服。4周为1个疗程。

【功 效】活血化瘀,理气散结。

【临床运用】临床治疗48例,治愈8例,显效15例,好转21例,无效4例,总有效率91.67\%。

【经验心得】慢性附睾炎发病机制多为本虚标实,内因为素体肝肾不足或饮食所伤,造成湿热内蕴,日久湿热下注睾络而发为本病;再者房室所伤,如房事无度、气血壅滞,瘀血与湿热之邪相搏,日久而发为本病;或为跌打损伤,睾丸络伤血瘀,若瘀血不能消散,兼感邪毒亦可发为本病;或为起居不慎、外感风寒湿之邪,如久居湿地、冒雨涉水等致湿邪侵犯人体,造成机体气血运行不畅,肝脉受阻,发为本病。加味橘核汤中取橘核、荔枝核疏泄厥阴,生黄芪、赤芍、当归、桃仁补气理血、活血化瘀;小茴香、肉桂温通下焦,牡蛎软坚散结;地龙性善走窜,通达血脉,可引诸药直达病所。

【方剂出处】黄健,等.加味橘核汤治疗慢性附睾炎48例.四川中医,2006,24(12):67

5.子痈消散汤

【药物组成】柴胡10g,黄芩10g,党参30g,煅牡蛎30g,夏枯草30g,王不留行60g,紫苏子30g,炮穿山甲10g,浙贝母10g,车前子10g。

【治疗方法】每日1剂,按照常规方法煎煮,分2次服,连服30天。

【功 效】软坚散结,行气活血。

【临床运用】临床治疗20例。治愈4例,显效11例,有效3例,无效1例,总有效率90\%。

【经验心得】附睾硬结因瘢痕增生,西医抗生素治疗不易发挥作用,一般效果均不佳,对反复发作的疾病有时只能手术治疗,而中药在治疗慢性病方面则优势明显。附睾炎的急性期,“湿热下注”为医家一致公认的基本病机,在慢性期,附睾组织纤维增生及硬结与中医的“气滞痰瘀证”十分吻合,而肝脉循会阴,络阴器,睾丸属肾,脾主运化,故慢性期中医治疗应当从肝、脾、肾着手,软坚散结,行气活血。子痈消散汤由柴胡、黄芩、党参、煅牡蛎、夏枯草、王不留行、紫苏子、炮穿山甲、浙贝母、车前子等药物组成。方中柴胡、黄芩、党参三味药物取自“小柴胡汤”,因“小柴胡汤”能调和少阳和太阴,疏肝理脾,顺畅气血;浙贝母、煅牡蛎、夏枯草、紫苏子四味药物命名为“软坚散”,四味合用能清火散结,化痰软坚,对中医“积聚”类疾病效果良好;车前子清利湿热;王不留行、炮穿山甲活血、破瘀散结。本研究中有4例双侧附睾结节同时伴有梗阻性无精子症患者,经治疗1个月后复查精液常规,有1例出现精子。多年临床经验体会到,对慢性附睾炎的治疗,煅牡蛎、夏枯草、紫苏子、王不留行等药物必须剂量特别大,临床上只有当剂量较大时才能获得较显著的 疗效。

【方剂出处】倪良玉.子痈消散汤治疗慢性附睾炎20例.湖南中医杂志,2008,24(1):45

6.子 舒 汤

【药物组成】川芎、丹参、柏子仁、海藻、黄药子、虎杖、桃仁、昆布各15g,海蛤壳30g,生大黄10g,黄连20g。

【治疗方法】加水煎至300ml,候温,以38~40℃为宜,坐浴。每次15~20分钟,每日2次。同时配合口服阿奇霉素0.25g,每日2次。14天为1个疗程。

【功 效】活血化瘀,清热解毒。

【临床运用】适用于慢性附睾炎。临床治疗27例患者,年龄最小19岁,最大33岁,平均25.8岁;病程最短3个月,最长3年,平均1.5年。治愈15例,好转11例,无效1例,总有效率达96.3\%。

【经验心得】中医学认为,本病属“子痈”范畴,其发病主要因为饮食不节,湿热内生,外受寒湿化生湿热,及房室不洁,跌打外伤等因素所致。总病机乃湿热下注厥阴之络,以致气血凝滞而成。该病急性期多以湿热辨治,慢性附睾炎因附睾硬结形成,多与痰瘀互结密切相关。自拟子舒汤方中川芎、丹参、虎杖、桃仁具有活血行气化瘀之功;海蛤壳、海藻、黄药子、昆布具有化痰软坚散结之效;大黄、黄连具有清热解毒化湿之功。以上诸药合用,能改善附睾局部微循环,抑制硬结增生,抗炎抗菌镇痛,配合局部坐浴和服用西药起到标本兼治的作用,值得临床推广。

【方剂出处】翁剑飞,等.自拟子舒汤坐浴治疗慢性附睾炎27例.福建中医药,2002,33(3):49

7.桂枝茯苓丸加味

【药物组成】桂枝6g,茯苓15g,桃仁9g,牡丹皮9g,赤芍12g,连翘20g,败酱草30g,生薏苡仁30g,穿山甲6g,皂角刺12g,路路通15g,丹参30g,黄芪30g,牛膝15g,荔枝核12g,橘核12g。

【随症加减】下坠明显,加党参、升麻、柴胡;胀痛明显,加延胡索、川楝子;疼痛明显,加三棱、莪术、制乳香、制没药;寒湿盛,去连翘、败酱草,加乌药、小茴香。

【治疗方法】每日1剂,水煎,分2次服。30日为1个疗程。

【功 效】消肿散络,化瘀止痛。

【临床运用】临床治疗68例,治愈50例,显效15例,好转3例,治愈率73.8\%。

【经验心得】桂枝茯苓丸为仲景名方,用于瘀血留结胞宫而致的妊娠胎动不安,漏下不止,血色紫黑晦暗,腹痛拒按等。本病用之以活血化瘀,缓消癥块。方以茯苓、薏苡仁利其湿,牡丹皮、连翘、败酱草清其余毒,黄芪补其虚,延胡索、川楝子理其气,桃仁、赤芍、丹参化其瘀,穿山甲、皂角刺、路路通消肿散结通络。坠者气虚,加党参、升麻、柴胡配合黄芪仿补中益气汤方义补气升提;胀者气郁,用延胡索、川楝子、荔枝核、橘核理气除胀;痛者血瘀,用桃仁、赤芍、丹参合三棱、莪术活血化瘀而止痛。

【方剂出处】王祖龙.桂枝茯苓丸加味治疗慢性附睾炎68例.河南中医,2007,27(5):17

8.疏肝活血清利汤

【药物组成】橘核10g,荔枝核10g,柴胡20g,当归10g,赤芍10g,青皮6g,甘草梢6g,川楝子10g,白芷10g,小茴香10g,王不留行15g,丹参30g,桃仁10g,红花15g,鸡血藤30g,败酱草30g,泽泻15g,黄柏10g。

【随症加减】下坠明显者,加炙升麻10g,生黄芪15g;刺痛甚者,加三七粉6g;有结节形成者,加穿山甲6~10g,鸡内金30g,或三棱10g,海藻30g。

【治疗方法】每日1剂,2次煎液混合,分早、晚口服;每晚睡前取第3次煎液约3000ml,局部坐浴热敷,温度35~45℃,以手指拭液不烫为度(未婚、未育者忌用)。10日为1个疗程,观察1~6个疗程。

【功 效】疏肝理气,行气活血,软坚散结,清利湿热。

【临床运用】临床治疗58例。治愈34例,显效13例,好转6例,无效5例,总有效率为91.4\%。一般治疗10天即获效。

【经验心得】慢性附睾炎与阴部不洁、外染湿热秽毒、精神紧张、压力过重、过食肥甘厚腻、嗜好烟酒及体质因素等有关。其病变脏腑主要在奇恒之腑——精室,与肝、肾二经密切相关。本病病程长,主要表现为阴囊部坠胀、疼痛、硬结(局部纤维化),为中医瘀血之象,肝经瘀血、气滞血瘀为本病的主要病机,且多夹未清之湿热余邪,因此治当疏肝活血为主,兼以清利湿热。疏肝活血清利汤中用橘核、荔枝核、川楝子疏肝行气止痛;柴胡与赤芍相合,能疏肝活血,畅达宗筋;当归、赤芍、丹参调肝,养血活血;青皮破气平肝,引诸药至肝经,与橘核、川楝子、小茴香等配用能治疝痛;芍药合甘草酸甘化阴,缓急止痛;王不留行通经活血消肿;桃仁、红花祛瘀止痛;穿山甲配王不留行,通血脉、通精窍,二者为通肝经血脉之要药;泽泻、黄柏坚阴清利湿热;白芷散风除湿,解毒止痛;小茴香温肾行气,为治疝气疼痛之要药,配黄柏疏肝脾而泄湿热,清膀胱而排瘀浊。现代药理研究证明,大多数活血化瘀药具有改善附睾局部微循环、促进纤维蛋白溶解以及抗纤维化作用,能促使炎症吸收,并可改善附睾管腔闭塞梗阻;黄柏、败酱草有抑菌、抗病毒作用,可消除炎性水肿。诸药共奏疏肝理气、行气活血、软坚散结、清利湿热之效,使气行瘀去结散,药证合拍,故取得满意疗效。

【方剂出处】包立振,等.疏肝活血清利汤治疗慢性附睾炎58例.上海中医药杂志,2008,42(3):44

9.仙方活命饮

【药物组成】金银花3g,皂角刺12g,连翘15g,浙贝母10g,玄参15g,土茯苓15g,生大黄10g,赤芍12g,炒穿山甲5g,制乳香5g,制没药5g,川楝子10g,生甘草10g。

【治疗方法】治疗前7日,每日1剂,水煎服,药渣待凉后再敷患处,每次20~30分钟,后7日上药方加三棱12g,莪术12g,牡蛎30g,每日1剂,水煎服,并用药渣热敷患处,疗程为14日,治疗1个疗程观察结果。

【功 效】解毒散结,化瘀消肿。

【临床运用】适用于急性附睾炎。临床治疗42例患者。痊愈31例,显效9例,有效2例,总有效率100\%。

【经验心得】急性附睾炎多为细菌经尿道逆行感染所致,常见致病菌为金黄色葡萄球菌、大肠埃希菌等,虽然及时足量应用抗生素,能较快改善症状和体征,但有相当一部分患者急性期后易形成附睾结节,久不消散。中医学认为,本病乃外感湿热毒邪,侵犯肝经,循经下注,结于阴部而成。治疗以解毒散结、化瘀消肿为大法。本方中金银花、连翘、土茯苓等清热解毒;湿热毒邪内侵,必致气血壅滞,故以赤芍、炒穿山甲、乳香、没药以活血消肿散结止痛,生大黄既可清泻热毒,又可化瘀散结,川楝子既可清热疏肝理气,又可引药归经,诸药并用共奏解毒散结消肿之功。由于“外治之理即内治之理,外治之药即内治之药“(《理瀹骈 文》),故在内服中药的同时,根据疾病发展的不同阶段,予以冷敷、热敷,可进一步提高疗效。仙方活命饮加减结合西药,可明显降低附睾结节形成,提高痊愈率,值得临床推广应用。

【方剂出处】孙自学.仙方活命饮加减结合西药治疗急性附睾炎性42例.中国中西医结合杂志,2003,23(3):234

10.荔 核 丸

【药物组成】荔枝核、橘核、瓦楞子、昆布、海藻各30g,川楝子、玄参各20g,赤芍15g,柴胡、夏枯草、虎杖、三棱、小茴香各10g。

【随症加减】湿热较著加龙胆草、败酱草各15g,湿热兼瘀加川牛膝、炮穿山甲各10g,湿热兼虚加枸杞子15g,仙茅10g,少腹痛加延胡索10g。

【治疗方法】每日1剂,水煎,分2次服。配合针灸治疗:通经活络,活血止痛,取足厥阴经及任脉腧穴。处方:中极、关元、三阴交、太冲、蠡沟。刺灸法:中极、关元用标准艾炷施灸5壮,余穴针刺泻法,每日1次,留针30分钟。10日为1个疗程。

【功 效】通经活络,活血止痛。

【临床运用】临床治疗32例,治愈22例,好转8例,无效2例,总有效率93.75\%。

【经验心得】慢性附睾炎属中医学“子痈”范畴,为痰瘀互结所致。足厥阴经环绕阴器,抵达小腹,选足厥阴经之络蠡沟,原穴太冲配合足三阴经会穴三阴交,采用泻法,能活血祛瘀,通络止痛。中极、关元为任脉要穴,根据“经脉所过,主治所及”的原则,用之于治疗前阴部疾病,施灸能扶正培元,温通经脉,活血化瘀。中药方剂中,荔枝核、橘核、川楝子行气散结止痛,赤芍、川芎、三棱活血化瘀止痛;而海藻、昆布、瓦楞子、夏枯草可化痰散结;虎杖清热利湿;诸药合用,软坚散结,活血止痛,清热利湿。使用针灸与中药治疗相结合,疗效显著。自拟方药荔核丸,可作为汤药服用,亦可制成丸药服用,丸药每日服2次,每次9~18g。

【方剂出处】管小虎,等.针灸合用自拟荔核丸治疗慢性附睾炎32例.陕西中医,2007,28(9):1226

11.五味龙虎丹

【药物组成】全蝎、蜈蚣、土鳖虫、血竭、三七各10g。

【治疗方法】烘干碾末,装胶囊,每粒0.3g。服法:每次3粒,每日2次。

【功 效】化痰解毒,散瘀活血。

【临床运用】适用于慢性附睾炎。临床治疗50例患者,治愈34例,好转11例,无效5例。总有效率为90\%。

【经验心得】附睾炎是一种感染性炎症,可由细菌、支原体、衣原体等引起,各种年龄均可发生,尤其好发于20—40岁的青壮年。急性附睾炎如及时治疗,肿胀可很快消退,但大多数容易形成硬结。为什么附睾炎易产生纤维性增生,机制尚不清楚,可能与附睾吸收睾丸液、分泌卡尼汀功能受到破坏有关。附睾硬结因瘢痕增生,抗生素治疗不易发挥作用,因而考虑中药治疗。五味龙虎丹是经验方,功效化痰毒,散瘀血。方中全蝎、蜈蚣解毒散结通络;土鳖虫、血竭、三七破血逐瘀。

【方剂出处】蔡国芳.五味龙虎丹治疗附睾炎性硬结50例.河南中医,2002,23(1):27


\subsection{第7章 鞘膜积液}

鞘膜积液是指鞘膜囊内积聚的液体超过正常量而形成囊液。本病的主要症状多为单侧性阴囊内肿块逐渐增大。肿块大小不一,小者无不适,肿块较大者则有阴囊下坠感。睾丸鞘膜积液阴囊肿块为卵圆形,表面光滑有波动感,与阴囊皮肤不粘连,睾丸、附睾不易摸到。婴儿型鞘膜积液肿块呈梨形,在腹股沟逐渐变细。精索鞘膜积液常于精索上扪及囊肿样肿块,牵拉睾丸或精索,肿块随之下移。

1.补脾活血方

【药物组成】党参20g,黄芪20g,白术6g,桃仁6g,泽兰9g,香附6g,青皮6g,吴茱萸6g,小茴香6g,橘核4g,荔枝核6g,泽漆6g,车前子20g,甘草3g。

【治疗方法】每日1剂,水煎3次,去渣兑匀,每日分3次服,7天为1个疗程,治疗期间嘱患者禁食寒冷食物。

【功 效】补脾益气,活血利水,疏肝理气,温经散寒。

【临床运用】临床治疗30例患者,治愈19例(63.9\%),好转9例(30\%),无效2例(6.7\%),总有效率为93.3\%。

【经验心得】方中党参、黄芪、白术补脾益气,使津液输布正常;桃仁、泽兰活血祛瘀、疏通血脉,以利积液消除;香附、青皮疏肝理气;小茴香、吴茱萸入肝经,温通散寒;橘核、荔枝核入厥阴气分以行气中之滞;车前子、泽漆利水消肿以助积液消失;甘草和中。诸药合用,具有补脾益气、活血利水、疏肝散寒之功,用于临床取得良效。

【方剂出处】廖志香.补脾活血利水法治疗睾丸鞘膜积液30例.中国中医药信息杂志,2002,9(2):53

2.水 疝 汤

【药物组成】党参、黄芪各20g,山茱萸、泽泻、巴戟天、茯苓各10g,青皮、柴胡、小茴香、紫苏梗、吴茱萸各6g,白术、山药、车前子各15g,甘草3g。

【治疗方法】每日1剂,水煎,分2次服。10天为1个疗程。

【功 效】健脾补肾,疏肝利水。

【临床运用】临床治疗70例,3—5岁21例,6—10岁39例,11—15岁10例。其中最大15岁,最小3岁,病程3天~4个月。治愈44例,占62.9\%;有效21例,占30\%;无效5例,占7.1\%。总有效率为92.9\%。本组用药最少者10剂,最多者30剂,

【经验心得】睾丸鞘膜积液属中医“水疝”范畴,本病病变在阴囊,为肝经所系。小儿脏腑娇嫩,脾虚健运失司,水湿内生,流注阴囊;小儿肾气未充,气化不利,水湿内停,流蓄囊中;小儿肝木未旺,肝之疏泄障碍,气化不利,水湿不化,积聚阴囊;小儿卫外功能未固,易感寒湿之邪,以致寒湿凝滞,聚于阴器而发生“水疝”之症,故立健脾补肾、疏肝散寒、利水之法,用自拟水疝汤治疗。方中黄芪、党参、白术健脾益气,使津液输布正常;巴戟天、山茱萸、山药补益肾气,使肾能气化,减少鞘膜间的渗出;紫苏梗、青皮、柴胡疏肝理气;小茴香、吴茱萸入肝经,温经散寒;泽泻、茯苓、车前子利水消肿;甘草调和诸药。因本方具有健脾补肾、疏肝散寒、利水之功,故对治疗“水疝”有效。

【方剂出处】周和平.水疝汤治疗睾丸鞘膜积液70例.四川中医,2004,22(10):46

3.疏肝渗湿汤

【药物组成】川楝子10g,小茴香6g,橘核10g,木香7g,吴茱萸3g,青皮10g,泽泻10g,茯苓12g,猪苓10g,桂枝5g,荔枝核10g。

【治疗方法】每日1剂,水煎2次分服,上、下午各1次,空腹温服。10天为1个疗程。

【功 效】健脾温肾,活血通络,利水消肿。

【临床运用】临床治疗36例,经1个疗程治愈者3例,2个疗程治愈者8例,3~4个疗程治愈者19例,好转4例,无效2例,总有效率94.4\%。

【经验心得】肾主二阴,前阴为肝肾两经所主,肝脉循少腹络阴器。若肝失疏泄,脾阳不足致健运失司,肾虚又气化不利,水湿因而停聚于前阴,或为寒邪外侵,客于肝肾两经,阳气被遏,致寒凝气滞,生湿积聚阴囊而为水疝。因此治疗主要以疏肝健脾温肾治其本,活血通络、利水消肿治其标。根据以上治则,自拟疏肝渗湿汤。该方用导气汤合五苓散两方加味而成。导气汤出自《沈氏尊生书》,原方由川楝子、木香、茴香、吴茱萸四味药组成,功能行气散寒止痛,主治寒疝痛。五苓散出自《伤寒论》,由猪苓、泽泻、白术、茯苓、桂枝组成,功效利水渗湿,温阳化气。《伤寒论》原用本方治太阳表邪未解,内传太阳膀胱腑,致膀胱气化不利,水蓄下焦,而成太阳经腑同病。方中以川楝子疏肝理气而止痛,小茴香、橘核均入肝肾两经,前者温和下焦、行气散寒,后者疏肝行气而散结。吴茱萸燥湿散寒、破结散郁,更入青皮入肝胆气分,气胜而峻烈,疏肝破气,散结除坚,行气而通络;荔枝核入肝经血分,软坚散滞祛寒,行血中之气,再加木香功专行气,而通调诸气,使气滞而通,气结而散。再合五苓散之泽泻、猪苓、茯苓以利水湿,桂枝通阳温化水气,且有温经通络之功,现代药理亦认为桂枝可扩张血管,增强和调整血液循环,促使血液流向体表,有利于发汗和散热,佐以白术健脾以助运化水湿之力。两方诸药相合共奏疏肝理气、温肾散寒除湿、利水消胀之功能,使肾之气化及脾之健运水湿之功恢复正常,积存过量之浆液及水湿得以吸收及分利,恢复其疏调、气化、健运之功能,从而取得良好疗效。

【方剂出处】潘正平.疏肝渗湿汤治疗睾丸鞘膜积液36例.江苏中医,1999,20(3):29

4.五苓散加味

【药物组成】白术9g,泽泻15g,猪苓9g,茯苓9g,桂枝6g,黑附子6g,干姜9g,小茴香9g,车前子9g,川牛膝6g,益母草9g,青皮9g,白芍9g。

【治疗方法】水煎服,每日1剂,分3次温服。

【功 效】利水渗湿,温肾化气。

【临床运用】临床治疗147例患者。治愈126例,显效19例,好转2例。

【经验心得】睾丸鞘膜积液属中医学“水疝”病范畴,发生多与寒邪内侵、禀赋不足等有关,其病机为肾阳不足,寒邪内侵,寒湿搏结致肾气不化,膀胱失司,水蓄下焦,聚于阴器发为“水疝”。临床表现为阴囊肿大,光亮湿冷,触之如囊裹水,少腹拘急,有下坠感,小便不利,畏寒肢冷,舌淡,苔白脉沉迟。治法为利水渗湿,温肾化气。方中泽泻、车前子直达膀胱,利水渗湿;猪苓、茯苓以增其效,白术健脾以化水湿,桂枝一药既可外解表寒,又可内助膀胱气化,附子、干姜、小茴香温肾祛寒,青皮、牛膝、益母草化瘀通路,引水归经,诸药合用则寒去表解,水行气化,肿痛自消。由于本病多因先天不足或过劳伤肾,汗出当风而遇寒湿之气,寒滞肝脉,水蓄阴囊中,气机失调,发展为水疝。本病属本虚标实之证,不可清利,不可妄下攻逐,否则变化丛生。

【方剂出处】李强,等.加味五苓散治疗睾丸鞘膜积液.齐齐哈尔医学院学报,2000,21(4):412

5.乌药茴香汤

【药物组成】乌药10g,小茴香30g。

【治疗方法】文火水煎取汁150~250ml,日服1剂,分早、中、晚及睡前4次服完,10天为1个疗程。

【功 效】温肾,行水,利湿。

【临床运用】临床治疗30例,年龄1—12岁,年龄分布:1—5岁25例:6—12岁5例。其中交通性鞘膜积液12例:精索性鞘膜积液16例:精索睾丸鞘膜积液2例。治愈21例,好转7例,无效2例。

【经验心得】本病又有“水疝”之说,属中医疝证之讨论范畴,《诸病源候论·疝病诸候·诸疝候》曰:“诸疝者,阴气积于内,复为寒邪所加,使荣卫不调,气血虚弱,故风冷入其腹内而成疝也。”可见疝证多为寒所致。其病位主要是足厥阴肝经及足少阴肾经。小茴香辛、温,具有祛寒理气之功效,可入肝肾经,故可温化积液。乌药辛、温,归胃、肾、膀胱经,具有行气、散寒、止痛之功效,与小茴香合用,可增强小茴香祛寒、理气之功效。此法简便易行,患儿易于接受。

【方剂出处】张国丽.乌药茴香汤治疗小儿鞘膜积液30例.

新中医,2004,(2):45

6.自 拟 方

【药物组成】乌药10g,萹蓄15g,薏苡仁30g,吴茱萸6g,小茴香10g,青皮6g,香附10g,柴胡10g,桂枝5g,泽泻10g,炒川楝子10g。

【随症加减】合并睾丸炎、附睾炎者加荔核15g,橘核15g,牡蛎20g。

【治疗方法】每日1剂。凉水浸泡30分钟后煎,水开后文火煎30分钟,煎3次所得药液混匀,分一日3次服用。

【功 效】暖肝行气,健脾渗湿,温经消液。

【临床运用】临床治疗12例。经治疗,所有病例均有效,积液消失。服药6剂积液消失2例,服9剂积液消失6例,服12剂积液消失3例,服18剂积液消失1例。

治验:骆某,男,6岁,右侧睾丸鞘膜积液3年余,多方医治无效,不愿手术,故求治于中医。刻诊:右侧阴囊肿大,下垂明显,透光试验(+),面色萎黄,舌淡苔腻,脉迟弦,辨证为“寒滞肝脉,脾虚湿聚。”以天台蓄薏吴茴汤加减治之。药用:乌药10g,萹蓄15g,薏苡仁30g,吴茱萸6g,小茴香10g,青皮6g,香附10g,柴胡10g,桂枝5g,泽泻10g,炒川楝子10g,半夏8g,茯苓15g,苍术10g,陈皮6g,生姜4g。每日1剂。凉水浸泡30分钟,水开后文火煎30分钟,煎3次所得药液混匀,分1日3次服用。服9剂后,积液消失,再给苓桂术甘汤5剂善后。

【经验心得】睾丸鞘膜积液中医多从肝胆经络论治。因儿童多穿开裆裤,容易引起外阴受凉,且患儿患处发凉、无红肿热象,故辨证为肝经风寒阻滞经络,气滞、湿阻、水停。天台蓄薏吴茴煎是在已故四川名医“小儿王”王静安之温经消液汤基础上变化而来。方中小茴香温肾暖肝,治疗寒滞肝脉,疝气腹痛牵引睾丸;吴茱萸入肝、脾、胃经,温中散寒、下气开郁,二药为治疝要药。乌药辛开温通,宣畅气机,温肾顺气、散寒止痛,配香附最善调理下焦气机;小茴香、乌药同用治疗寒疝腹痛。柴胡、川楝子、香附、青皮疏肝理气,桂枝温经活血、通阳化气行水。薏苡仁渗湿利水,湿盛在下者最宜用,泽泻利水。薏苡仁合萹蓄为一民间验方,专治水疝。诸药合用,共奏暖肝行气、健脾渗湿、温经消液之功。方证合拍,故显桴鼓之效。并编一歌诀:“天台蓄薏吴茴煎,桂附青柴泽川楝”,以便记忆。

【方剂出处】曲志中.自拟方治疗儿童慢性睾丸鞘膜积液12例.中国乡村医药,2009,16(3):45

7.复方五倍子汤

【药物组成】五倍子30g,白矾30g,金樱子30g,川楝子15g。

【治疗方法】每剂加水800ml,煎取药液300ml,待温后将肿大之阴囊浸入药液洗浴。每次10分钟,每日2次,每剂药可连用2天。治疗12天若无效停止治疗,药浴期间停服其他药物。

【功 效】收敛固涩,化痰除湿,行气利水。

【临床运用】临床治疗16例,年龄最小4个月,最大2周岁,平均8个月。均为原发性睾丸鞘膜积液。其中1侧睾丸积液者13例,两侧积液者3例。16例患者除1例无效外,其余15例均全部治愈,治愈率达93.75\%。治愈时间最短者6天,最长者12天,平均8.4天。1例治疗12天后因无效即停药,此系一3岁患儿,在本组中年龄最大。

【经验心得】婴儿睾丸鞘膜积液,属中医学“水疝”“偏气”症,《儒门事亲》云:“水疝,其状肾囊肿痛,阴汗时出,或囊肿而状如水晶。”其病因病机与任脉和足厥阴肝经之经气失常有关。所谓:“任脉为病,男子内结七疝。”“疝乃足厥阴肝经病是也。”关于疝气的治疗,中医有药物、针灸、外治等方法。洗浴法即为外治法之一。明代医家张景岳曾在《千金翼方》中介绍用雄黄、白矾、甘草等水煎外浴治疗阴核肿大。根据中医传统理论,结合临床实践自拟复方五倍子汤,外浴治疗小儿睾丸鞘膜积液效果良好。方中五倍子、白矾、金樱子、川楝子等诸药均可入肝、肾经,分别具有收敛固涩、化痰除湿、行气利水等功效。诸药合用能除湿利水,减少黏液分泌,故疗效显著。

【方剂出处】吴农荣.中药外浴治疗婴幼儿鞘膜积液16例报告.甘肃中医,2000,(4):23

8.水疝汤外洗

【药物组成】小茴香12g,肉桂6g,煅龙骨2g,五倍子15g,乌药12g,枯矾15g。

【治疗方法】将上药捣碎,置于煎药器皿内加水800ml,浸泡30分钟后置于火上煎煮,水沸后30分钟,移火并滤出煎煮液。待药液冷却与皮肤温度相近时,将阴囊全部置入盛药液的容器内,浸洗约30分钟。每晚浸洗1次,每剂用1次,10剂为1个疗程。

同时配合艾灸疗法:选取患者双侧阳池穴(伏掌,腕背横纹上,指总伸肌腱尺侧缘凹陷中),以75\%医用酒精常规消毒,点燃艾条一端,待火力温和时,置于穴位2cm之上悬灸,并根据火力调整悬灸距离,待施灸处皮肤泛红并有烘热感时,移灸对侧穴位,每日晚艾灸1次,10次为1个疗程。

【功 效】温经通络,收敛制水。

【临床运用】临床治疗33例。经1个疗程治疗后治愈者18例,2个疗程治愈者10例,3个疗程治愈者4例,无效1例,转手术治疗。本疗法治愈率97\%。

治验:某某,男,59岁。2002年6月就诊。阴囊肿大1周,如梨子,行走不便,垂坠疼痛,曾在三级医院门诊就诊。诊断为睾丸鞘膜积液。建议行手术治疗,患者因惧怕手术转入我院外科门诊治疗。经检查符合睾丸鞘膜积液诊断标准,遂给予水疝汤煎剂外洗合艾灸阳池穴治疗1个疗程,积液吸收,阴囊缩小,症状基本消失。用第2个疗程巩固疗效。3个月后随访未复发。

【经验心得】睾丸鞘膜积液属于中医学“水疝”范畴,主因寒凝气滞,肝脉受阻,三焦水液代谢失常,水气偏渗于下而成。水疝汤温经通络,收敛制水;艾灸阳池穴温运三焦,化气行水。方中小茴香入归肝、肾,经散寒止痛,主治寒疝腹痛,睾丸偏坠胀痛;肉桂含挥发油(桂皮醛)及鞣质,具有扩张血管的作用,并能刺激黏膜吸收,排除积气积液,鞣质有收敛作用;煅龙骨具有较强的收敛燥湿作用,可促进血液凝固,降低血管壁的通透性;五倍子性味酸涩寒,有收敛作用,含有磨石子鞣质,对蛋白质有沉淀作用,可制止黏膜渗出;乌药行气止痛、温肾散寒,其所含生物碱(乌药酸、乌药内酯)可促进血液凝固,减轻组织渗出;枯矾性味酸涩,有较强的收敛燥湿作用,其所含硫酸铝钾有制止黏膜分泌作用,能和蛋白化合物结合成难溶的蛋白化合物而沉淀于鞘膜的黏膜层表面。诸药合用有较强的收敛燥湿,制止鞘膜分泌,促进积液吸收的功能。《灵枢·九针十二原》:“原者,主治五脏六腑之有疾者也。”艾灸手少阳三焦经之原穴阳池,可以调畅三焦气机,促进水液代谢,使其行于道而不溢于外。

【方剂出处】张文光.水疝汤外洗加艾灸阳池穴治疗睾丸鞘膜积液33例.中国民间疗法,2008,16(12):20

9.丁香粉敷脐

【药物组成】干丁香适量。

【治疗方法】丁香粉制作:把干丁香于厚铜皿中反复用铜棒锤打研压成细末,去粗末,瓶装备用。操作:先行患侧阴囊常规消毒后用注射器穿刺,吸去睾丸鞘膜积液。当日开始,用丁香粉2g敷于神阙穴,十字胶布外固定,每2日换药1次。每晚睡前用20g车前草加入300ml水煎成约100ml药液,将小毛巾湿透后外敷积液患处。

【功 效】通经活络,利水渗湿。

【临床运用】临床治疗72例,均经B超等检查确诊为睾丸鞘膜积液,其中3—8岁50例,9—14岁22例;病程2~8年52例,8~12年20例。积液10~40ml 35例,41~60ml47例,均未手术治疗。经治疗72例中痊愈(积液消失,1年后随访无复发)65例,其中1个疗程治愈55例,2个疗程治愈10例;显效(积液消失,1年内复发)4例;无效(积液无明显减少)3例。总有效率95.8\%。

【经验心得】睾丸鞘膜积液是水湿升降失常,下注聚于鞘膜内所成。神阙穴位于任脉,任脉为阴脉之海,具有调节全身诸阴经经气作用。外敷丁香粉可芳香化湿,通经活络;辅以车前草渗湿利水,使鞘膜对渗出和吸收保持平衡,是针对病因治疗,故能收到良效。

【方剂出处】黄绍波,等.神阙穴外敷治疗小儿睾丸鞘膜积液72例.中国针灸,2000,(2):108

10.牡 苏 散

【药物组成】牡蛎、紫苏叶各15g。

【治疗方法】研为细末,用泡热茶水调匀适量药末,再加用米醋调成黏糊状,加醋时便发出“喳”之声,立即涂敷患处,即调即涂,每日1~5次。复涂前可用温茶水洗净患处,候干再涂药,1周为1个疗程,同时内服加减橘核丸。

【功 效】理气散结。

【临床运用】临床治疗32例患者。其中年龄1个月内者5例,2~12个月者15例,1—3岁者9例,4—7岁者3例;偏侧者17例,双侧者12例,并发腹股沟斜疝者3例。治疗结果,显效21例,占65.63\%;有效9例,占28.13\%;无效2例,占6.25\%。

【经验心得】牡苏散系经验方,配制加工成散剂供临床应用。本病虽有自愈者,但据临床观察,用中药治疗可以缩短病程,从验、便、廉言仍有可取之处。所用外涂散剂需注意即调即用,调匀后微温涂之为佳;患儿若为先天不足者,内服方用橘核丸加黄芪、胡芦巴、小茴香、巴戟天;若为肝胆湿热者宜用橘核丸合龙胆泻肝汤加减。

【方剂出处】陈先泽,等.牡苏散外用治疗小儿鞘膜积液32例临床体会.中医外治杂志,2002,11(3):54

11.温化凝滞方

【药物组成】肉桂3g,牛膝5g,当归5g,赤芍5g,红花5g,牵牛3g,猪苓10g,泽泻10g,小茴香5g,橘核5g,甘草3g。

【随症加减】脾气虚者加黄芪、党参、山药;阴囊坠胀,少腹抽痛者加升麻、木香、川楝子;腰膝冷痛者加制附子、菟丝子、枸杞子;便结者加大黄。

【治疗方法】水煎服,每日1剂。第三煎汤药用毛巾浸药液外敷。

【功 效】散寒利湿,活血化瘀。

【临床运用】临床治疗128例患者。治愈126例,治愈率98.4\%,好转2例,总有效率100\%。

【经验心得】鞘膜积液属于中医“水疝”范畴。本病多因先天不足,脾失健运,或肾虚气化不利,三焦水道气机不畅,外感寒湿之邪,水湿内停,寒湿瘀结所致。采用温化凝滞,辨证施治,配合外治法,温寒化湿散结,取得了较好的疗效。方中小茴香、肉桂、橘核温暖下焦,散寒止痛;泽泻、猪苓、当归、赤芍利水消肿,活血散结;槟榔、牵牛逐水利湿,驱邪外出,诸药合用共奏散寒利湿、活血化滞之功。临床应用,屡获效验。

【方剂出处】邱崇怡.温化凝滞法治疗鞘膜积液128例.湖南中医杂志,2000,16(6):73

12.健脾温肾活血汤

【药物组成】生黄芪、白术、泽泻、茯苓、车前子、泽兰、煅牡蛎各3~20g,肉桂、青皮、香附各3~6g,川牛膝、小茴香、橘核、荔枝核、山楂核、马鞭草、川楝子各3~12g。

【治疗方法】以水没药浸泡0.5小时,煎2次取汁80~300ml,早、晚饭后0.5小时各服40~150ml,服药困难者也可分3~4次服。每日1剂,连续服用4周。配合外洗:将2次煎后的内服药渣分别加入蝉蜕、枯矾各3~20g,醋50ml,每日1剂;加水300ml,小火煎煮0.5小时取汁,待温,嘱患儿取坐位,将阴囊全部浸泡入药液中20~30分钟,每日2~3次,下次浸泡时需将药液加温;连续用药4周。若患儿哭闹不配合者,可用外敷法。外敷用橘核、肉桂、小茴香、红花、芒硝、枯矾、五倍子、青盐各20g,研成粉末,加醋。

【功 效】健脾温肾,活血通经,利水消肿。

【临床运用】临床治疗26例,治愈23例,好转2例,无效1例。总有效率96.15\%。

【经验心得】本病为本虚标实之证,本虚为脾肾不足,标实即湿蕴、血瘀、气滞,病变与肝、脾、肾有关。方中生黄芪、茯苓、白术益气补脾,脾健水去;肉桂、小茴香温肾化气,疏通水道;川牛膝、泽兰、马鞭草、红花活血利水,通利血脉,以达欲治其水当活其血的目的;药理研究证明:肉桂有扩张人体下部和内部血管的作用,辅以牛膝等活血药,使水液吸收入于血脉,经肾以尿的形式排出,从而增强了利水消肿的作用。古人云:“治疝必先治气”,川楝子、香附、青皮疏肝理气,橘核、荔枝核入厥阴,行滞气,并配合山楂核、煅牡蛎、石决明、芒硝软坚散结;蝉蜕、泽泻、茯苓、车前子利水消肿;枯矾、五倍子、青盐、醋收敛、燥湿、消炎。小儿阴部皮肤薄嫩,为宗脉之所聚,血运丰富,配合外治,药物易于渗透和吸收,起到见效快,提高疗效的作用。


\subsection{第8章 精索静脉曲张}

精索静脉曲张是男子不育症中最常见的原因,男子不育症患者发病率可达21\%~39\%,对于精索静脉曲张合并不育症之患者一般均主张手术治疗。本病为精索静脉蔓丛发生扩张、伸长、纡曲。多发于20—30岁之成年人,多数在左侧。其主要原因是左侧精索内静脉长而无瓣膜,且垂直进入肾静脉,血流受阻较大之缘故。本病也可继发于肾肿瘤、肾积水等病,这种继发症,临床上称为症状性精索静脉曲张。

1.伸曲助育汤

【药物组成】香附10g,荔枝核10g,当归15g,赤芍12g,白芍15g,枳实10g,青皮10g,陈皮10g,炙甘草6g。

【治疗方法】每日1剂,水煎,分2次服。治疗15日为1个疗程。观察治疗1~4个疗程。

【功 效】疏肝解郁,理气定痛。

【临床运用】临床治疗106例,治愈88例,有效14例,无效4例,有效率达88.2\%。

【经验心得】精索静脉曲张的病位在肝,足厥阴肝经循阴股,入毛中,过阴器,抵小腹。足厥阴之筋则上循阴股,结于阴器,络诸筋,肝经及其经筋的走向及肝主筋的理论决定了精索静脉曲张与肝有着极其密切的关系。肝主疏泄,气血正常运行有赖于肝正常的疏泄功能,肝失疏泄,气血运行受阻,血行不畅,瘀阻脉络发为“筋疝”。因此,肝气郁结于下才是精索静脉曲张的基本病机。在此认识基础上,拟方组成了以香附、荔枝核、当归、赤白芍、枳实、青皮、陈皮、炙甘草为主要药物的伸曲助育汤。该方以香附、荔枝核为君,《本草纲目》言:“香附之气平而不寒,香而能窜。其味多辛能散,微苦能降,微甘能和……熟则下走肝肾,外彻腰足。”香附不失为疏肝理气之要药,更配合荔枝核理气止痛为君药,共奏疏肝理气之功效。当归、赤芍、白芍养血柔肝为臣,枳实、青皮、陈皮为佐助之品,加强疏肝理气之功。炙甘草调和诸药,全方共奏疏肝解郁、理气定痛之功。

【方剂出处】刘建荣,等.伸曲助育汤治疗精索静脉曲张及其不育的临床研究.上海中医药杂志,2005,39(3):33

2.生精化瘀汤

【药物组成】淫羊藿30g,仙茅15g,熟地黄30g,龟甲30g,菟丝子20g,知母15g,肉苁蓉15g,巴戟天15g,桃仁10g,红花10g。

【随症加减】气滞血瘀加丹参15g,莪术15g,牛膝15g,当归10g;肾虚络阻加黄芪10g,桂枝10g,鸡血藤30g,路路通10g;湿热夹瘀加防己10g,泽兰10g,荔枝核15g,丹参10g,柴胡6g,牡丹皮10g,赤芍10g。

【治疗方法】每日1剂。煎后分2次温服。3个月为1个疗程,治疗2个疗程。

【功 效】化瘀通络,调补阴阳。

【临床运用】临床治疗42例患者。1个疗程临床痊愈6例,显效9例,有效11例,无效16例,总有效率61.90\%;2个疗程临床痊愈14例,显效10例,有效7例,无效11例,总有效率73.81\%。

【经验心得】精索静脉曲张是精索内蔓状静脉因回流不畅而形成局部静脉扩张的一种病理现象。精索静脉曲张不仅直接损害睾丸造精功能,影响精子的生成,而且影响性激素的分泌,同时,还产生一定毒素损害睾丸实质,造成精液异常。医者运用二仙汤加减,以活血化瘀为基础,酌情配以补肾益精。方中淫羊藿、仙茅、菟丝子补肾填精,鼓动肾气,提高生精功能;肉苁蓉、龟甲、熟地黄填补肾精,为生精提供物质基础;桃仁、红花活血化瘀。全方化瘀通络,调补阴阳,既可作为临床非手术患者治疗,也可作为手术后的恢复性治疗。

【方剂出处】徐新建,等.生精化瘀汤(二仙汤)治疗精索静脉曲张性不育症42例报告.中国中西医结合外科杂志,2001,7(4):269

3.乌鸡肉菟汤

【药物组成】乌药、菟丝子、当归各25g,鸡血藤、淫羊藿各20g,肉桂、小茴香、橘核、川楝子、赤芍、香附各15g,桂枝、延胡索各10g。

【随症加减】如阴囊肿胀甚者,加厚朴15g,枳实、海藻、昆布各10g,以行气破坚积,软散除肿胀;阴囊坠胀者,加黄芪30g,升麻15g,以益气升提;早泄阳痿、神疲倦怠乏力加桑椹20g,巴戟天10g,肉苁蓉15g,温肾助阳;阴囊质较硬加木香15g,桃仁10g,以行气血,通瘀结。

【治疗方法】每日1剂,水煎,分2次服。配合外用法:取红椒汤外洗患处,方组:红花15g,花椒10g,艾叶15g,防己25g,吴茱萸15g,防风10g,木香15g,水煎1 500~3 000ml,熏洗患处,每晚睡前外洗30分钟,洗后避风寒,禁房室;忌食辛辣、生凉油腻及刺激性食品。症状减轻可每周外熏洗2次。

【功 效】散寒祛湿,化瘀通络,消肿。

【临床运用】临床治疗56例,治愈38例,有效16例,无效2例。

【经验心得】精索静脉曲张属于中医的“筋疝”“寒疝”范畴,由于寒湿之邪瘀阻足厥阴肝经,或多因阳虚之体,外受寒凉侵袭,又因足下保暖不够,致使肝经失养,寒凝气滞,血瘀阻络所形成,取乌药、香附、桂枝,理气通经脉;鸡血藤、当归、赤芍活血化瘀;肉桂、橘核、川楝子、小茴香、菟丝子、淫羊藿温经散寒,益肾壮阳;延胡索行气止痛;全方共奏行气化瘀、散寒止痛之功;又以“红椒汤”外洗,可增强散寒祛湿、化瘀通络、消肿的作用,故临床疗效显著。

【方剂出处】王凤智,等.中药内服外洗治疗精索静脉曲张56例.中医药信息,2004,21(2):39

4.鸡血藤汤

【药物组成】鸡血藤25g,红花15g,肉桂10g,小茴香10g,乌药15g,当归20g,丹参25g,淫羊藿20g,菟丝子20g,香附15g,赤芍15g,橘核10g,川楝子10g。

【随症加减】体虚阴囊肿坠重者,加黄芪、升麻;睾丸、阴囊肿痛重者,加延胡索;阳痿、早泄、神疲乏力重者,加淫羊藿叶、桑椹、枸杞子。

【治疗方法】每日1剂,水煎服。忌食辛辣、生冷油腻、酒及刺激性强的物品。椒艾己香汤外敷:花椒5g,艾叶15g,防己20g,木香10g。每日1剂,将其兑入煎好的内服药渣内,用纱布包成2包,加水适量,再煎数滚,趁热轮换外敷患处30分钟左右,每日1~2次。

【功 效】温元,祛瘀,散结。

【临床运用】临床治疗22例患者,治愈18例,有效2例,无效2例。

【经验心得】精索静脉曲张相当于中医的“筋疝”范畴,多属体虚或过劳受凉,寒湿之邪瘀阻足厥阴肝经,进而寒凝气滞,血瘀络阻而成。故本方取鸡血藤、红花、赤芍、当归、丹参活血化瘀,通经活络,消肿止痛;以香附、乌药理气散滞;以小茴香、肉桂、橘核、川楝子温经散寒,逐瘀止痛;以淫羊藿、菟丝子益肾壮阳,温通肾气。更以椒艾己香合内服之药渣煎后热敷,可增强祛湿散寒、理气开瘀、通经活络、消肿止痛而调治筋疝之功效。

【方剂出处】张林,等.中药内外合治治疗精索静脉曲张22例.中国临床医生,2002,30(11):55

5.通补结合方

【药物组成】当归10g,生地黄10g,熟地黄10g,川芎10g,丹参15g,莪术10g,王不留行15g,何首乌10g,黄精10g,菟丝子10g,枸杞子10g,淫羊藿10g,五味子10g。

【随症加减】有痛引少腹、畏冷喜暖等寒凝血瘀症状者加小茴香6g,肉桂3~6g;有阴囊坠胀、烦躁易怒等肝郁气滞表现者,加柴胡10g,枳实10g,川楝子10g;有倦怠少气、性欲低下等气虚不足者,加黄芪15g,山茱萸10g,山药10g;有阴囊潮湿、舌苔黄腻等湿热症状者,加盐黄柏10g,萆薢10g,薏苡仁10g,车前草10g。

【治疗方法】隔日1剂,水煎分服,3个月为1个疗程。治疗期间忌酒及辛辣饮食。

【功 效】补肾生精,活血化瘀。

【临床运用】临床治疗89例患者。痊愈31例(34.83\%),有效42例(47.19\%),无效16例(17.98\%),总有较率82.02\%。

【经验心得】本病属于中医学“筋瘤”“筋疝”的范畴。其病因病机是先天禀赋不足,脉络畸形,瘀血内阻,而致新血不布,外肾(睾丸)失养,精亏无子。瘀血阻络是本病的始发病因,而外肾失养则是其继发的病理变化,所以通与补应为本病的治疗大法。本文基本方中当归、生地黄、熟地黄、川芎、丹参、莪术、王不留行等活血化瘀,破积通络,使瘀血祛,新血布,外肾得养;何首乌、黄精、菟丝子、枸杞子、淫羊藿、五味子等皆有补肾生精之功。两法相辅相成,则改善血液循环,提高精液质量。

【方剂出处】杨家辉,等.通补结合治疗精索静脉曲张合并不育症.中华男科学,2002,8(4):310

6.活血通脉汤

【药物组成】桃仁10g,川芎10g,丹参30g,赤芍20g,牛膝30g,黄芪30g,木香10g,苏木10g,刘寄奴10g,水蛭(冲)6g,益母草30g,生地黄10g,当归10g。

【治疗方法】每日1剂,水煎服。1个月为1个疗程,一般服药1~3疗程。

【功 效】活血化瘀,益气通脉。

【临床运用】临床治疗60例,治愈35例(58.3\%),好转15例(25\%),无效10例(16.7\%),总有效率83.3\%。

【经验心得】精索静脉曲张是精索静脉血流瘀滞,使精索静脉血管丛扩张、纡曲所致。中医学认为本病属于血瘀凝滞所致。现代医学认为,精索静脉曲张患者由于睾丸组织的生精细胞有不同程度的脱落,造成曲细精管狭窄或阻塞,由此引起患者少精或无精,畸形精子增多,精子活力或活率下降,另外,患者曲张的静脉内血液滞留,阴囊温度调节出现紊乱,阴囊和睾丸的局部温度升高,也会影响精子的生成,最终导致男性不育。目前,治疗精索静脉曲张,多采用手术治疗,但有许多患者不愿接受手术治疗,而求助中医治疗。活血通脉汤方中桃仁、川芎、丹参、牛膝活血祛瘀;黄芪补脾胃之气,气足以促血行而不滞,活血而不伤正;木香行气止痛,气行则血行而祛瘀血;苏木、刘寄奴、水蛭、益母草破血化瘀,行气止痛;生地黄、当归养血和血,祛瘀而不伤血。诸药合用,共奏活血化瘀、益气通脉之功效。

【方剂出处】刘广程.活血通脉汤治疗精索静脉曲张所致不育60例.河北中医,2005,27(2):138

7.活血补肾汤

【药物组成】丹参20g,王不留行15g,乌药15g,荔枝核15g,川楝子15g,延胡索15g,乳香10g,没药10g,牛膝15g,当归15g,黄芪20g,赤芍15g,枸杞子20g,菟丝子20g,韭菜子15g,牡蛎30g(先煎)。

【随症加减】湿热蕴结者,加苍术、黄柏、薏苡仁、连翘、虎杖等;脾肾阳虚者,加制附子、肉桂、鹿角胶、巴戟天、肉苁蓉、续断、党参、白术等;肝肾阴虚或阴虚火旺者,加知母、黄柏、龟甲、鳖甲、生地黄、玄参、麦冬等;肝郁气滞者,加柴胡、川芎、香附、青皮、橘核等;血瘀明显者,加桃仁、红花、三棱、莪术、土鳖虫、穿山甲、三七等。

【治疗方法】每日1剂,水煎2次,取汁约400ml,早、晚温服。30日为1个疗程。

【功 效】活血补肾。

【临床运用】临床治疗76例患者,治愈38例,好转32例,无效6例,总有效率为92.1\%。

【经验心得】临床可先手术,再配合中医中药,辨证治疗,不仅能迅速消除自觉症状,还能明显改善局部血液循环,恢复睾丸生精功能,提高精子质量如妊娠率,缩短治疗时间等。方中以丹参、王不留行、当归、赤芍、乳香、没药为主,活血化瘀通络;以枸杞子、韭菜子、菟丝子为辅,补肾填精;更用黄芪益气扶正,乌药、川楝子、延胡索、荔枝核疏肝理气止痛,牡蛎软坚散结;用牛膝滋补肝肾,并能引药下行、直达病所。诸药相合,共奏益气活血、补肾填精、疏肝理气之功。倘能辨证施治、加减用药,则能收桴鼓之效。

【方剂出处】李君强,等.活血补肾汤治疗精索静脉曲张合并不育症76例.中国中医药科技,2000,7(1):58

8.温元祛瘀散结方

【药物组成】当归15g,青皮12g,红花10g,苏木10g,牛黄3g,穿山甲10g,制大黄6g,荔枝核12g,桃仁10g,丹参30g,土鳖虫10g,蜈蚣2条,壁虎4条,小茴香12g,黄芪30g,巴戟天10g。

【治疗方法】上药共研细末,炼蜜为丸,每丸10g,每次服2丸,每日2次,早、晚分服,3个月为1个疗程。

【功 效】温元补肾,祛瘀散结。

【临床运用】临床治疗30例患者。治愈10例,好转15例,无效5例,总有效率83.3\%。

【经验心得】本方从中医理论出发采用温元祛瘀散结的方法施治。方中荔枝核、小茴香、巴戟天、制附子补肾助阳,散寒止痛治本;桃仁、红花、苏木、制大黄、穿山甲、土鳖虫、丹参、壁虎活血祛瘀散结治标;蜈蚣、牛黄解毒;青皮理气;当归、黄芪补益气血,助正祛邪,其中黄芪重用可达60g,更能补益正气;荔枝核、小茴香作为全方的引经药,使药力直达病所。全方配伍,攻补兼施,标本并治,即可祛除病因,改善局部血液循环,达到治疗不育的目的。

【方剂出处】吴玉芙,等.温元祛瘀散结法治疗精索静脉曲张致不育30例.河北中医,1999,21(6):369

9.桂枝茯苓丸加味

【药物组成】桂枝、茯苓、牡丹皮、芍药、桃仁各10g,当归12g,黄芪、何首乌各15g,枸杞子、川牛膝各20g,甘草6g。

【随症加减】睾丸偏坠,胀痛不舒,脉弦,属肝经郁滞者,加橘核、乌药各10g;阴囊湿痒,尿黄,口苦,舌苔黄腻,属湿热下注者,加车前子、黄柏各10g;阴囊下坠不收,倦怠乏力,脉细弱,属气虚者,加党参、白术各10g;畏寒肢冷,阴部发凉,脉沉迟,属阳虚者,加吴茱萸3g,制附子6g;舌红口干,五心烦热,盗汗,脉细数,属阴虚者,加知母、鳖甲各10g。

【治疗方法】每日1剂,水煎服。每月复查1次精液,3个月为1个疗程,一般治疗1~2个疗程。

【功 效】活血化瘀。

【临床运用】临床治疗269例患者,治愈97例,占36.06\%;显效101例,有效34例,无效37例。总有效率86.25\%。

【经验心得】桂枝茯苓丸出自《金匮要略》,系张仲景为妇人宿有癥疾、妊娠漏下所设,由桂枝、茯苓、牡丹皮、芍药、桃仁各等份组成,方中桂枝善气化、消本寒、温通经脉;芍药调营血、散恶血、疏肝安脾,桂芍相伍能调和营卫使血脉通畅;桃仁破瘀血消癥癖,利于生新;瘀久则化热,巧用牡丹皮意有化瘀清热之妙;更有茯苓渗湿健脾而安护正气,共奏通血脉调营安正,化癥积祛瘀生新之功效而为主药,以改善精索静脉曲张对生精功能的影响,但缺乏生精作用。故加淫羊藿、何首乌、枸杞子补肝肾、调阴阳而益精血,且淫羊藿具有雄激素样作用,何首乌含有大量的卵磷脂,能保护精子细胞膜的完整性;加当归、黄芪(即当归补血汤)补气生血,能明显的提高精子活动率和精子活动力。共为辅药。牛膝补肾活血,引药下行,直达病所;甘草解百毒、调和诸药,同为佐使。符合精索静脉曲张并发不育症之病机,用以治疗本症能促进其局部血液循环,改善生精环境,明显的提高精液及精子质量。提示本方有祛瘀生新,补肾强精之功效,可使患者免除手术治疗之苦,同样也适用于术后的辅助治疗。

【方剂出处】徐吉祥.桂枝茯苓丸加味治疗精索静脉曲张型不育症269例.陕西中医,2003,24(9):783

10.补中益气汤合槐榆煎

【药物组成】炙黄芪20g,丹参20g,柴胡10g,升麻5g,当归10g,槐花15g,地榆10g,煅牡蛎10g,枸杞子10g,女贞子10g。

【随症加减】兼有少精症者,加制何首乌、黄精、淫羊藿等;阴囊不适、疼痛明显者,加荔枝核、橘核、延胡索等;食欲不振、乏力倦怠、射精无力者,加党参、茯苓、白术等。

【治疗方法】每日1剂,早、晚水煎服。1个月为1个疗程,共3个疗程。

【功 效】益气活血,填精生髓。

【临床运用】临床治疗72例。显效41例,有效24例,无效7例,总有效率90.28\%。

【经验心得】针对气虚血瘀证型,根据“形不足者,温之以气;精不足者,补之以味”及“血不自生,须得生阳气之药,血自旺矣”等中医经典理论,治以益气升提、活血化瘀,佐以补肾,宜用补中益气汤合槐榆煎加减治疗。全方以黄芪为君,补脾肺之气,合少量升麻以加强升阳举陷之功;柴胡疏肝理气,气行则血行,兼引经之用;气虚血瘀,久病致瘀,故以丹参、当归以养血活血,补中有行,行中有补;瘀久化热,又以槐花、地榆清热凉血;肾为先天之本,脾为后天之本,精血同源,故以枸杞子、女贞子以补肝肾益精血;用牡蛎主要取其软坚散结又富含锌等微量元素,为提高精子活力之要药。纵览全方,重在益气活血,兼以填精生髓,达到气血双调,阴阳并举之效。

【方剂出处】王鹏,等.补中益气汤合槐榆煎加减治疗精索静脉曲张并弱精症临床观察.医学综述,2007,13(24):2030

11.暖 肝 煎

【药物组成】当归、肉桂、乌药、桃仁、延胡索各9g,枸杞子12g,小茴香15g,沉香、橘核、川楝子、通草各6g。

【随症加减】冷痛、肿胀严重者加吴茱萸9g,干姜6g,甚者可加炮附子6~9g;静脉曲张累及阴囊皮肤和大腿内侧时可加丹参15g,赤芍12g,川芎9g;全身有热象如口苦、咽干、小便黄者可加黄芩、栀子各10g,竹叶3g。

【治疗方法】每日1剂,水煎,分2次服。

【功 效】行气血,散寒湿。

【临床运用】临床治疗52例,治疗组1度者治愈16例,显效1例;2度者治愈27例,显效2例,无效1例;3度者治愈3例,显效1例,无效1例;总治愈率88.4\%,总有效率96\%。

【经验心得】本病临床症状应属中医“疝气”范畴。病位在肾,病变在肝,肝脉络于阴器,上抵少腹,寒湿内侵,留滞于厥阴肝脉,气血瘀滞而发为本病。初起睾丸肿大,坠胀疼痛,久则气滞血瘀,痛引少腹。治当从肝肾入手,以当归、枸杞子温补肝肾,肉桂、小茴香温肾散寒,乌药、沉香行气止痛,桃仁、延胡索活血散瘀,通草通利血脉而除湿,橘核行气散结专治疝痛,川楝子助橘核延胡索行气止痛。寒痛甚者加吴茱萸、干姜、炮附子以暖肝壮阳、促进肝脉血液循环,静脉曲张面积增大累及阴囊皮肤和大腿内侧时加用丹参、赤芍、川芎以活血散瘀。全身有热象者加黄芩、栀子、竹叶以清热泻火。诸药合用,可直达厥阴肝经行气血、散寒湿而消肿胀。须注意的是,如为湿热下注引起的炎性阴囊红、肿、热、痛者,不可误用。

【方剂出处】李高旗,等.暖肝煎加减治疗精索静脉曲张52例.实用中医药杂志,2004,20(12):687


\subsection{第9章 前列腺炎}

前列腺炎是青壮年男性的常见疾病,占泌尿外科门诊及男科门诊患者的1/4左右。有研究认为,前列腺炎是因感染、充血以及不明原因引起的包括局部症状、全身症状、精神-神经症状的一种症候群。临床上通常将其分为急性细菌性前列腺炎、慢性细菌性前列腺炎、非细菌性前列腺炎和前列腺痛四类。其中以非细菌性前列腺炎最为多见。

1.利湿排毒散

【药物组成】败酱草30g,泽兰15g,石韦12g,车前子10g,灯心草6g,橘核15g,丹参15g,延胡索15g,淫羊藿15g。

【随症加减】小便热涩疼痛者选加瞿麦、萹蓄、琥珀、滑石、竹叶、白茅根;湿热盛者选加黄柏、大黄、生栀子、龙胆草;疼痛属气滞者选加川楝子、乌药、小茴香、沉香;血瘀者选加桃仁、赤芍、红花、穿山甲;兼肾阴虚者选加熟地黄、女贞子、枸杞子;肾阳虚者选加巴戟天、锁阳、仙茅、鹿角霜、蜈蚣;早泄者选加芡实、覆盆子、金樱子;腰痛甚者选加桑寄生、续断、炒杜仲;失眠多梦者选加炒酸枣仁、五味子、合欢皮、远志。

【治疗方法】水煎2次,取汁1 000ml,分早、晚2次空腹温服,每日1剂,30天为1个疗程,忌烟酒、海鲜等辛辣刺激性食物。

【功 效】祛湿排毒,疏通气机。

【临床运用】临床治疗561例。治愈356例,好转122例,无效83例,总有效率85.2\%。

治验:张某,男,28岁。2006年3月29日以“尿频、尿无力、尿痛近2年”就诊。主要症状:尿频、尿无力、尿痛,尿不尽、尿滴沥,尿白,小腹部、耻骨内、睾丸、会阴部间歇性坠胀不适或疼痛,阴囊潮湿,勃起不坚,早泄,性欲减退,苔薄腻,脉滑,多方治疗不愈。彩色多普勒超声检查:前列腺49mm×31mm×30mm,腺体回声不均质;鞘膜腔积液:左41mm×9mm,右26mm×13mm;前列腺液常规:卵磷脂小体(++),白细胞(+),细菌培养葡萄球菌生长。诊断:慢性细菌性前列腺炎,慢性附睾炎,鞘膜腔积液。治疗:利湿排毒散,每日1剂,每次500ml,早、晚2次空腹温服,连服30天。前列腺仪药物透入治疗1个疗程,症状全部消失。3个月后复查,无不适,性功能正常,前列腺液检查无异常。随访半年无复发。

【经验心得】慢性前列腺炎属中医学之“淋证”“精浊”“肾虚”范畴,治疗颇为棘手。本病主要临床表现为“淋”,湿郁蕴毒,或寒或热,下注膀胱,小便异常;“痛”,湿阻气机,或气滞或血瘀;“虚”,湿性黏滞,阻遏气机,宗筋弛缓,肾阳失展。互为因果,根本在湿。祛湿排毒,疏通气机,则阳气舒展。利湿排毒散专司祛湿,力排内毒,补肾固本。方中败酱草、泽兰、车前子清热利湿、解毒排毒;石韦、灯心草利湿通淋;橘核理气通络,延胡索、丹参活血化瘀止痛;淫羊藿补肾助阳。湿祛毒清,气血通畅,肾气旺盛。若见虚单纯滋补,病因不除,即便取效,终归难愈。

【方剂出处】王振洲.利湿排毒散治疗慢性前列腺炎561例临床观察.四川中医,2008,26(6):65

2.萆薢化浊汤

【药物组成】萆薢、白花蛇舌草、桃仁、黄柏、败酱草、土茯苓、苦参、蒲公英、海金沙、败酱草、丹参各10g,王不留行、泽泻、石莲子各9g。

【随症加减】若辨证属肝经湿热加龙胆草、柴胡、栀子;肾阴亏加生地黄、龟甲、山茱萸;肾阳虚加补肾脂、覆盆子、菟丝子;尿道涩痛加车前子、瞿麦、石韦;精血者加大蓟、小蓟、墨旱莲、阿胶;会阴部痛加川楝子、延胡索、白芍;尿道口痛加地榆、萹蓄;小腹坠胀加乌药、沉香;前列腺触痛加蒲黄、五灵脂、三七;前列腺坚硬加莪术、乳香、没药。

【治疗方法】 每日1剂,水煎2次分服,连服4周为1个疗程。

【功 效】清热解毒,利湿通淋,活血化瘀。

【临床运用】临床治疗34例,治愈22例,有效8例,无效4例,总有效率为88.3\%。

【经验心得】慢性前列腺炎属于中医淋证、白浊范畴。主要病机是湿热蕴结于下焦,膀胱气化不利所致。采用萆薢化浊汤,清热解毒,利湿通淋,活血化瘀。因活血化瘀药能和利湿通淋药改善前列腺血流循环,消肿利尿的效果,并可解除前列腺屏障,有利于清热解毒药(抗生素)进入腺体发挥抗菌消炎的作用。有资料证明,败酱草、黄柏、白花蛇舌草、蒲公英具有抑菌、消炎,增强免疫力的功效;萆薢、土茯苓、苦参、泽泻、石莲子、海金沙均有清热解毒,利湿消肿的作用。同时丹参、桃仁、茜草、王不留行有活血化瘀,扩张末梢血管循环,抑制结缔组织增生,软化前列腺组织,既降低后尿道压力,减少前列腺内尿液反流,又可促进前列腺组织血液循环,提高组织中的药物度,增强抗药 效应。

【方剂出处】游峰.萆薢化浊汤治疗慢性前列腺炎34例.辽宁中医学院学报,2004,6(4):302

3.瓜蒌瞿麦汤

【药物组成】天花粉、瞿麦、山药、浙贝母各12g,茯苓15g,制附子10g。

【随症加减】小便黄赤加木通12g,车前子18g,蒲公英15g。小便清长,性功能低下者加淫羊藿12g;少腹胀痛加乌药15g,川楝子12g;伴前列腺肥大者加炮穿山甲、莪术各10g,王不留行12g。

【治疗方法】每日1剂,水煎,分2次服。10日为1个疗程,治疗2个疗程观察疗效。

【功 效】健脾益肾,清热通淋,化瘀散结。

【临床运用】临床治疗66例,治愈10例,显效36例,有效17例,无效3例,治愈率15.2\%,总有效率95.5\%。

【经验心得】慢性前列腺炎属中医“淋证”范畴。病机为肾元亏虚、浊瘀阻塞或热结下焦,膀胱气化不利。病位在膀胱,涉及肺、脾、肾。瓜蒌瞿麦汤加减方以制附子、淫羊藿补肾温阳化气,山药健脾补肾扶正治其本,瞿麦化瘀通淋祛浊利尿,茯苓健脾利尿,淫羊藿温肾利尿,瓜蒌、木通、车前子、蒲公英清热通淋解毒,浙贝母宣肺、开提肺气以提壶揭盖,穿山甲、王不留行、莪术活血化瘀、软坚散结,乌药、川楝子行气止痛。诸药合用,共奏健脾益肾、清热通淋、化瘀散结之功。

【方剂出处】刘杰,等.瓜蒌瞿麦汤治疗慢性前列腺炎66例.实用中医药杂志,2005,21(8):468

4.前 列 丹

【药物组成】金银花50g,板蓝根50g,黄芪50g,山楂50g,玄参20g,泽泻25g,赤芍25g,前胡20g,猪苓20g,萆15g,牡丹皮20g,瞿麦20g,莪术15g,双皮15g,延胡索10g,黄柏10g,连翘15g,甘草10g。

【随症加减】急性前列腺炎,尿急、频,混有白浊,尿道灼热刺痛重者,前列腺液体检验、白细胞(++)以上伴有脓细胞及脓团者,加量用金银花、板蓝根、连翘,另加龙胆草及柴胡;尿浊濒而不畅,重用泽泻、猪苓,加灯心草;尿痛、尿白浊加白茅根;小腹坠痛、前列腺肿大、坚硬者重用山楂、莪术、三棱、丹参、洋兰;肺胃气虚、尤其老年肾气虚而尿频短涩、小腹膨隆排尿不畅者,重用黄芪、玄参,加山药、何首乌、党参;腰痛腿软、阳痿、早泄者加补骨脂、枸杞子。

【治疗方法】每日1剂,煎2次,取煎液300ml,分早、晚温服,服药期间忌食辛辣、肥腻之品,或将其原方加大数倍,洗净烘干、共研面以100目筛过筛,炼蜜为丸,每丸9g,每日3次,1丸/次,或装胶囊,每服5g,白开水送服。

【功 效】清热解毒,通经散瘀,补气强肾。

【临床运用】临床治疗200例。治愈168例,有效23例,无效9例,总有效率95.5\%。

治验:某某,男,50岁。1996年4月8日就诊,自述患前列腺炎5年余,曾多次住院,中西药治疗效果不佳,近1周加重。自觉尿频急,排尿不畅,尿有白浊,时有小腹坠胀,会阴部不适,伴有膝酸软、阳痿等症。诊见:发育正常、体瘦面红,脉细数,前列腺液检查:白细胞(+++),卵磷脂小体少许。诊为“慢性前列腺炎”,治以前列丹内服,金银花50g,板蓝根50g,黄芪50g,山楂50g,玄参20g,泽泻25g,赤芍25g,前胡10g,猪苓20g,萆薢15g,牡丹皮20g,瞿麦20g,莪术15g,延胡索10g,黄柏10g,枸杞子15g,淫羊藿15g,甘草20g。每日1剂,早、晚水煎服,5剂后自觉排尿通畅、白浊减少,连服10剂,病症大减,尚有腰痛、尿频乏力、阳痿,脉沉细,按上方减猪苓、黄柏,加何首乌、远志,又进25剂诸症消失,前列腺检验正常。病告痊愈。

【经验心得】本病相当于中医病证较广,病因复杂,分型较多,但其证候却又多为相兼而见。故此,综合病机,以清热解毒、通经散瘀、补气强肾为主法的前列丹汤剂治疗,效果显著。方中以金银花、连翘、板蓝根清热解毒,泽泻、猪苓清热利水,肺主气、主降,又为肾之母、水之上源,故选前胡清热宣肺,玄参补肾清热、利水消肿,山楂化滞消食、活血散瘀,更以黄芪补气圣药调理升降气机,萆薢、瞿麦利水通淋,黄柏清热利湿,莪术、延胡索散热化瘀,桑白皮宣肺,甘草调合诸药,另按各种证候加减运用更为妥当。总之本病以虚为本、以标为实,故采用清热解毒、通经散瘀治标,以补气强肾治本,标本兼治之前列腺丹内服,灵活辨证加减,收效显著。

【方剂出处】江希春,等.前列丹治疗前列腺炎200例.中国现代药物应用,2008,2(9):53

5.清 化 汤

【药物组成】红藤30g,丹参12g,赤芍12g,红花12g,杜仲12g,延胡索10g,怀牛膝25g,蒲公英30g,桃仁12g。

【随症加减】湿热型,加用黄柏、栀子、萹蓄、车前草、白花蛇舌草、萆薢、甘草;血瘀型,加用制乳香、制没药、川楝子、三棱、莪术、乌药、蜈蚣;肾精亏虚型,加用淫羊藿、肉苁蓉、益智仁、菟丝子、金樱子、芡实;阴虚火旺型,加用黄柏、知母、生地黄、山药、泽泻、车前子。

【治疗方法】 每日1剂,水煎2次,分2次口服,每次约200ml。每晚用上药余渣加黄酒250g,煎汤坐浴30分钟。服药期间忌食生冷、油腻、辛辣之品,保持大便通畅及合理的性生活。治疗30天为1个疗程。

【功 效】补肾填精,清利湿热,活血化瘀。

【临床运用】临床治疗198例。治愈150例,有效45例,无效3例,有效率占98.48\%。

治验:患者,男,33岁。2005年11月5日初诊。前列腺炎病史1年余。1年前在外地打工期间,曾有不洁性交史,3天后尿道口有较多的脓性分泌物流出,伴尿急、尿痛、尿频等,自服抗生素,症状减轻,分泌物消失。10余日后尿道再次出现黄白色分泌物,量少,晨起较明显,时有尿痛,耻骨上部、腹股沟及会阴部有腹痛,有排尿不尽感,经肌内注射头孢曲松、大观霉素等,口服药(药名不详)及前列腺体内注射液,疗效不佳。刻诊:少腹及会阴部胀痛,小便频数,终末尿白,尿道灼热,遗精,早泄,性欲减退,舌质偏红,苔黄腻,脉弦细。直肠指诊:前列腺体增大,中央沟变浅,稍有压触痛。前列腺常规化验:卵磷脂小体少许,脓细胞(++)。证属肾虚不固、湿热下注、血瘀阻络。治宜滋阴益肾、清利湿热,佐以活血通络。药用:红藤30g,丹参30g,杜仲15g,赤芍12g,延胡索10g,怀牛膝25g,红花12g,桃仁12g,当归15g,黄柏15g,蒲公英30g。每日1剂,水煎2次混匀,分早、晚2次口服。每晚用上药余渣加黄酒250g煎汤坐浴30分钟。经连续用上法加减治疗45天,诸症明显减轻,查前列腺增大不明显,压触痛消失,前列腺常规化验,结果正常,惟早泄未愈。故原方去延胡索、蒲公英、黄柏,加芡实、金樱子,煎服法及坐浴法同前。连续治疗1个月后,性生活恢复正常,并服六味地黄丸巩固疗效。

【经验心得】如阴精亏虚,气化功能减弱,造成气、血、精、液等体内物质的代谢障碍,出现水、湿、瘀、浊留下焦;降低机体卫外防御和祛病能力,使湿热之邪乘机入侵,湿热蕴结下焦可直接耗伤肾精,又可扰动精室而阴精外流。如此,湿热加重肾虚,肾虚产生湿热,互为因果,使病情加重。故治宜补肾填精、清利湿热、活血化瘀。清化汤方中红藤、蒲公英清利湿热;杜仲、怀牛膝补肾强精,祛湿热;丹参、赤芍、延胡索、红花、桃仁活血化瘀。慢性前列腺炎的疼痛多表现在会阴部、耻骨上、腹股沟、腰骶部。中医学认为,肝主筋,会阴为宗筋所聚之处,肝经绕阴器一周,循少腹上行。如果肝失调达,肝气郁结,气滞血瘀,经络不通,致使不通则痛。前列腺炎为腺泡、腺管炎症梗阻,腺叶内纤维组织增生,肛门指诊发现有前列腺体肿硬波及后尿道室,致使膀胱颈部、精阜和射精管纤维化,引起小便不畅、射精不适等症状。这些都与瘀血有关,同时前列腺充血在病理状态下,也表现为血瘀。此病缠绵,病久必瘀,肾虚、湿热也可致瘀,因此,本病更要重视血瘀的存在,故方中用丹参、赤芍、延胡索、红花、桃仁等活血化瘀之品。

【方剂出处】杨俊.清化汤治疗慢性前列腺炎198例.中医研究,2008,21(9):52

6.清利活血汤

【药物组成】 生地黄15g,牡丹皮12g,怀牛膝10g,黄柏10g,败酱草15g,土茯苓15g,三棱10g,莪术10g,赤芍15g,玄参10g,木通6g,泽泻10g。

【随症加减】阴囊潮湿明显者加苦参15g;小腹胀痛伴睾丸牵扯痛明显者加川楝子12g,橘核10g;腰酸痛乏力者加菟丝子10g,枸杞子10g,桑寄生10g。

【治疗方法】上方加水400ml浸泡15分钟后,煎煮30分钟,取汁150ml;再加水300ml煎煮取汁150ml,把两煎汁液混合,分2次早、晚温服。取药渣三煎熏洗会阴。1个月为1个疗程。嘱患者在服药期间忌食辛辣及肥甘厚味且禁坐湿地。

【功 效】清热利湿,活血化瘀,散结止痛。

【临床运用】临床治疗30例。显效8例,占26\%;好转16例,占53\%;总有效率80\%。

治验:患者,男,31岁。2001年3月12日初诊。自诉小腹胀痛,时窜至睾丸3月余,且伴有尿道口潮湿,尿末有白色黏液排出,有拉丝现象,阴囊及会阴部有潮湿感,轻度口干,大便尚可。舌质淡苔薄黄,脉弦滑。曾服氧氟沙星片和普乐安片治疗,未见明显改善。遂来我处就诊。体格检查:前列腺触诊轻度压痛。取前列腺液检查示:白细胞20~30个/HP。诊断为慢性前列腺炎。证属:湿热下注,血脉瘀阻。给予中药清利活血汤加苦参15g,川楝子12g,7剂,水煎服。药后小腹及睾丸疼痛减轻,继服14剂后,疼痛消失且尿后无滴白现象,阴囊及会阴部无潮湿感。因患者不愿再服汤药,故给予中药知柏地黄丸加沈阳红药片以巩固疗效。半年后随访无复发,临床定为显效。

【经验心得】《诸病源候论·淋病》进一步指出"诸淋者,由肾虚而膀胱热故也。"后世医家认为本病多因膀胱湿热,血瘀痰凝,肝郁气滞,肾气亏虚而发。此病的发病与嗜食辛辣、肥甘厚味之品及久坐湿地关系最为密切;与男性的不良习惯(过度手淫致前列腺反复充血而致血脉瘀阻)有关。其病因复杂,证型交错,多是湿、热、瘀相互夹杂。故治疗中针对其病机,自拟清利活血汤,治以清热利湿、活血化瘀、散结止痛;方用生地黄、黄柏、牡丹皮、泽泻、木通清热利湿;三棱、莪术、赤芍活血祛瘀,通络止痛;败酱草、土茯苓清热解毒;玄参滋阴清热、解毒散结;牛膝引药下行,且利尿行瘀以通淋。

【方剂出处】刘锦森,等.清利活血汤治疗慢性前列腺炎30例.中国民康医学,2008,20(12):1287

7.通 淋 汤

【药物组成】 黄柏20g,败酱草30g,蒲公英30g,土茯苓30g,车前子15g,赤芍15g,怀牛膝15g,乌药12g,甘草6g,王不留行20g,穿山甲15g。

【随症加减】下焦湿热偏盛者加萆薢、石韦、金银花;伴血尿或血精者加焦栀子、大蓟、小蓟;阳痿早泄加巴戟天、芡实、山茱萸;前列腺质地较硬或有结节者加皂角刺、地龙、蜈蚣。

【治疗方法】每日1剂,水煎服,早、晚各1次。30天为1个疗程。

【功 效】消瘀,攻坚,散结。

【临床运用】临床治疗68例。治愈38例,好转25例,无效5例,总有效率为92.7\%。

治验:金某,男,35岁。于2004年6月9日初诊。主诉患慢性前列腺炎4年余,用中西药治疗,效果不显。刻诊:尿频、尿急、尿痛,小便余沥不尽,尿道口有黏液,大便努挣后滴白,下腹及会阴部疼痛,并向睾丸放射,舌质紫,苔黄,脉弦细。检查:前列腺左侧有压痛和结节;前列腺液常规:卵磷脂小体25\%/HP,脓细胞(+++),红细胞(+);B超检查前列腺稍大,内部见回声光斑,包膜不光整。病机属湿热留于下焦,阻滞经络。采用清利湿热,化瘀消肿法治疗。自拟通淋汤加减:黄柏20g,败酱草30g,蒲公英30g,土茯苓30g,车前子15g,赤芍15g,怀牛膝15g,乌药12g,甘草6g,焦栀子15g,小蓟15g,皂角刺20g,地龙10g。服10剂,尿末滴白已少,尿频、尿急、尿痛及下腹会阴胀痛明显好转。再拟原方巩固治疗1个月。临床症状基本消失。复查前列腺不肿,无压痛,结节已消失。B超检查前列腺未见异常。随访1年,未见复发。

【经验心得】针对本病由于感染→炎症→瘀血的病理变化,采用清利湿热、化瘀消肿法治疗。方中黄柏清湿热,泻实火,偏治于下;赤芍清热凉血,祛瘀消肿;败酱草、蒲公英、土茯苓、车前子清热解毒,利湿化浊逐邪外出;牛膝活血通经,促进血液通行;乌药长于疏肝气温肾,治小腹痛而兼缩溺;甘草善走诸经,治阴茎中痛;王不留行功专通利,走而不守,通血脉,疏通管道而消肿;穿山甲善窜经络,直达病所,消瘀、攻坚、散结,改善前列腺局部循环,提高代谢,促进腺体纤维组织软化而消除局部肿痛,诸药合用,从而达到消除前列腺炎慢性充血、肿胀、瘀血的病灶。

【方剂出处】李立凯.通淋汤治疗慢性前列腺炎68例.云南中医中药杂志,2008,29(4):16

8.消 癥 饮

【药物组成】生薏苡仁30g,败酱草30g,红藤20g,牡丹皮15g,赤芍15g,桃仁10g,王不留行15g,桂枝6g,黄芪30g,茯苓15g,延胡索12g,川牛膝15g。

【随症加减】少腹、会阴部胀痛不适者,加橘核15g,荔枝核15g;会阴、肛门部下坠者,加柴胡6g,升麻5g;失眠者加酸枣仁6g。

【治疗方法】每日1剂,水煎2次,取汁500ml,分2次温服。

【功 效】清热利湿,祛瘀止痛。

【临床运用】临床治疗200例。治愈72例,显效60例,有效49例,无效19例,总有效率90.5\%。

【经验心得】方中薏苡仁健脾利水渗湿,清热排脓消痈,此处用之,一可清热利湿除湿热之标,二可强健脾胃除生湿之源,三可排脓消痈治疗局部炎症,为君药。败酱草配红藤既清热解毒、消痈排脓,又活血祛瘀止痛;牡丹皮、赤芍味苦而微寒,能活血化瘀,又能凉血以清退瘀久所化之热,并能缓急止痛;桃仁善泄血分之壅滞,治疗热毒壅聚、气血凝滞之痈;王不留行具有通淋、通经、通乳的“三通”作用,共为臣药;桂枝辛甘而温,可温通血脉以行瘀滞,取“结者非温不行”之义。血得温而行,遇寒则凝,凡痈肿瘀结之症有热者,过用清热,则热清而瘀结难散,此方在大量清凉药中佐桂枝辛散使热清瘀消;茯苓健脾益胃、渗湿祛痰;黄芪益气,既可助行瘀,又防辛散药物久用伤气;延胡索理气止痛;川牛膝引药下行,共为佐使药,奏清热利湿、祛瘀止痛之功。

【方剂出处】王祖龙,等.消癥饮治疗湿热瘀阻型慢性前列腺炎200例临床观察.中医杂志,2008,49(8):701

9.复元活血汤

【药物组成】柴胡、红花各6g,当归、穿山甲、桃仁、天花粉、黄柏、制大黄各9g,败酱草、山药、淫羊藿、肉苁蓉各15g,甘草3g。

【随症加减】湿热重者加蒲公英、马鞭草,瘀血明显者加三棱、莪术,气虚加党参、黄芪,腰膝酸软明显者加菟丝子、怀牛膝。

【治疗方法】每日1剂,水煎,分2次服。20日为1个疗程,连服2个疗程。

【功 效】清利湿热,化浊通窍。

【临床运用】临床治疗178例,临床治愈67例,显效72例,有效24例,无效15例,总有效率91.6\%。

【经验心得】慢性前列腺炎的基本病机是湿、热、瘀、虚,以肾虚为本,湿热为标,且有气血凝结、脉络瘀阻贯穿本病的始终。《医学发明》之复元活血汤,化瘀与通络并举,清热与疏解共进,与慢性前列腺炎的病机较为契合。方中之穿山甲,《本草从新》称“善窜,专能行散,通经络,达病所。”针对药物难及病所(前列腺),穿山甲的应用尤为重要。大黄通泄兼散瘀,对瘀热重者尤为适宜。淫羊藿、肉苁蓉、山药缓进顾本。慢性前列腺炎患者因病而郁者较多,故用郁金助柴胡清解舒郁。蒲公英、黄柏、败酱草清热祛湿。诸药合用,则热能清,湿能化,郁结能散,瘀阻能通,从而使前列腺症状明显改善。

【方剂出处】卢伟.复元活血汤为主治疗慢性前列腺炎178例.实用中医药杂志,2006,22(3):137

10.化瘀导浊汤

【药物组成】败酱草、丹参、赤芍、牛膝、萆薢、枸杞子、菟丝子各15g,王不留行、益智仁、皂角刺各10g,红花6~10g,泽兰10~15g,甘草6g。

【随症加减】尿黄、尿道灼热疼痛者加黄柏、知母各10g,白茅根30g;小腹、会阴、睾丸、精索胀痛明显者加川楝子、延胡索各10g;腰膝酸软者加杜仲、续断各15g;遗滑精者加莲须15g,煅龙骨、煅牡蛎各30g;性功能减退者加淫羊藿15g,阳起石10g;前列腺质地偏硬,高低不平或有结节者,加三棱、莪术各10g;湿热明显者,开始治疗时可减益智仁、枸杞子等补肾之品,加白茅根15g,车前子、泽泻各10g等,待湿热渐退后,再适当加补肾之品。

【治疗方法】每日1剂,水煎,分2次服。15日为1个疗程。

【功 效】利湿化浊,活血化瘀,补肾通淋。

【临床运用】临床治疗70例,临床近期治愈43例,有效21例,无效6例,总有效率90.1\%。治疗期间未发现明显不良反应。

【经验心得】肾虚、精关不固为发病之本,下焦湿热蕴结,气滞血瘀为致病之标。化瘀导浊汤中败酱草具清热解毒,消痈排脓,祛瘀止痛作用。王不留行既能利尿,又能活血,配合利水通淋,活血消肿之品,如车前子、白茅根、红花、丹参、泽兰、败酱草等治疗前列腺炎。赤芍、皂角刺活血消肿。萆薢利湿,分清去浊。益智仁、枸杞子、菟丝子补肾固精。牛膝引药下行,且活血祛瘀、利尿通淋、补益肝肾。全方具有消中有补,不致克伐正气,补中有消,无虑留滞湿热,标本兼顾的优点。有资料证明败酱草、黄柏、丹参具有抑菌、消炎、增强免疫力作用,丹参、王不留行、牛膝具有改善末梢循环,抑制结缔组织增生,软化前列腺组织,既降低了后尿道压力,减少前列腺内尿液反流,又促进前列腺组织的血液循环,提高组织中的药物浓度,增强抗菌效应。诸药配合可起到利湿化浊,活血化瘀,补肾通淋的作用。

【方剂出处】李瑾,等.化瘀导浊汤治疗慢性前列腺炎70例.陕西中医,2004,25(8):682

11.十味淋浊汤加味

【药物组成】土茯苓30g,金钱草15g,丹参、穿山甲、刘寄奴各10g,乳香、没药、乌药、炒川楝子各6g,牵牛子3g。

【随症加减】肾阴虚者加墨旱莲、鳖甲(先煎);肾阳虚者加胡芦巴、肉桂;气虚者加黄芪;血虚者加当归;兼便秘者加大黄;兼血尿、血精者加蒲黄、怀牛膝。

【治疗方法】每日1剂,水煎2次,取汁300ml,早、晚分服,药渣煎汤加水再煎,于睡前趁热坐浴30分钟。服药期间忌食辛辣及刺激性食物,清心寡欲,节制房室。15日为1个疗程。

【功 效】清热解毒,活血化瘀。

【临床运用】临床治疗81例,治愈36例,好转42例,无效2例。总有效率97.5\%。

【经验心得】慢性前列腺炎是男性常见的生殖系统疾病,属中医学精浊、劳淋范畴。其病程较长,缠绵难愈,易反复。多为相火久遏不泄,湿热长期不清,精道气血瘀滞所致。十味淋浊汤中土茯苓、金钱草清热利湿解毒为君。《本草正义》谓:“土茯苓,利湿去热,能入络,搜剔湿热之蕴毒。”前列腺内常有感染之微结石,这些结石贮于腺体,使感染很难控制,金钱草善消结石,与土茯苓相伍,可提高疗效。丹参、穿山甲、乳香、没药、刘寄奴活血祛瘀止痛为臣。现代药理学证实,活血祛瘀药可改善慢性充血,改善微循环,发挥抗缺氧、抗凝、解聚、纤溶等作用。另外炮穿山甲有较强穿透力,有利于药物进入前列腺病灶杀灭致病菌,恢复前列腺功能。“气为血之帅”,佐以乌药、炒川楝子行气,使气行则血行。《本草纲目》曰:“牵牛能达右肾命门,走精隧。”《本草正义》谓:“牵牛,善泄湿热。”十味淋浊汤以牵牛子为使(因本品有毒,用量不宜大)。本方除早、晚内服外,药渣煎液趁热坐浴,既起到局部热敷作用,又能通过直肠吸收,使药物直接进入前列腺发挥作用,内外方法合用,疗效更佳。但该方毕竟以祛邪通络(清热利湿解毒、理气活血化瘀)为主,扶正力弱,遇体虚之人应辨证酌加补虚之品,方为妥当。

【方剂出处】白正学.十味淋浊汤加味治疗慢性前列腺炎80例.河北中医,2006,28(3):227

12.知柏地黄丸加味

【药物组成】知母15g,黄柏6g,熟地黄30g,山药15g,大枣皮12g,土茯苓15g,牡丹皮12g,泽泻15g,牛膝12g,败酱草30g,虎杖10g。

【随症加减】尿后空痛加淫羊藿;神疲乏力加黄芪、当归;尿道灼热明显倍黄柏,加蒲公英、鱼腥草;少腹胀加乌药、香附;局部潮湿加生薏苡仁、生牡蛎;舌黯加红花、丹参、赤芍;腰酸腿软加续断、杜仲。

【治疗方法】每日1剂,水煎取汁200ml,早、晚各服1次,1个月为1个疗程,服3~6个疗程。

【功 效】补肾,化瘀,通淋。

【临床运用】临床治疗43例,治愈25例,显效11例,有效6例,无效1例,总有效率为96.67\%。

【经验心得】《类证治裁》说:“肾虚则小便数,膀胱热则水下涩。”本病由于湿热之邪或情志不畅致湿热留滞下焦,久病不去,肾气不足,气滞血瘀,虚实夹杂。加味知柏地黄丸具有清热利湿,活血化瘀,滋补肝肾之功。方中熟地黄、山药、大枣皮健脾固肾;淫羊藿补肾助阳;土茯苓、泽泻、虎杖、败酱草利湿去浊解毒;黄柏泻火坚肾,能改善前列腺分泌功能,抗菌消炎;赤芍、牡丹皮、牛膝化瘀、引药下行直达病所,能促进血液循环,抑制和消除纤维结缔组织增生,有效提高药物的抗菌效果。诸药合用,补肾化瘀通淋,故收良效。并嘱进行适当体育锻炼,节制性生活,忌辛辣,忌饮酒,保持乐观心态,消除精神症状(紧张、焦虑、抑郁等)。

【方剂出处】向玉华,等.加味知柏地黄丸治疗慢性前列腺炎43例.内蒙古中医药,2005,(5):8

13.温 针 法

【穴位选择】次髎、肾俞、中极、归来、三阴交。

【随症加减】小便不利加阴陵泉;神经衰弱加百会、神门、通里;阳痿、遗精加命门、太溪。

【治疗方法】肾俞针1.5寸,次髎针2寸,针感至腰骶部。归来针1.5寸,中极针3寸,针感向下放射。三阴交,酸麻向上放射即可。后将长1cm左右的艾段置于针柄上点燃。每次约30分钟,10次为1个疗程。疗程间间隔3~4天,4个疗程为1周期。

【功 效】温经散寒,扶阳固脱,消瘀散结。

【临床运用】临床治疗72例。痊愈46例,显效18例,好转5例,无效3例,总有效率95.83\%。

治验:常某,男,30岁,公务员,已婚。主诉:会阴部隐痛4年,伴阳痿、神经衰弱2年。病史:4年前患者婚后因工作繁忙,生活不规律,出现会阴部隐痛并有滴白现象,尿道有异常感。曾先后在多家医院服用环丙沙星等药物效果不显。2年前症状较前加重,时有小腹胀痛,尿道口有少量分泌物,出现阳痿,神经衰弱。遂来就诊。治疗:结合症状取肾俞、次髎、归来、中极、三阴交、命门、神门、通里、足三里、阴陵泉等,如上法治疗。经治疗3个疗程后,无会阴部隐痛感,但阳痿、神经衰弱仍存在,又行2个疗程后,阳痿及失眠症状明显改善,1年后随访,症状未复发。

【经验心得】肾为先天之本,肾俞是肾脏之气输注的部位,主肾脏之病证,故取之。次髎属膀胱经,位于骶部,据“经脉所过,主治所及”的治疗原理,可用于治疗膀胱和肾的病证。中极为膀胱的募穴,有调节膀胱功能的作用。三阴交为三阴经交会穴,可治疗肝、脾、肾三经之病,《百症赋》:“针三阴交于气海,专司白浊,久遗精。”阴陵泉为脾经合穴,配五行属水,应于肾,主治肾虚疾病和膀胱疾病。艾灸有温经散寒、扶阳固脱、消瘀散结的作用。根据本病不同的兼症及配穴,再运用温针可改善前列腺体局部血液循环,有利于炎症吸收。

【方剂出处】岳燕琼,等.温针治疗慢性前列腺炎72例.针灸临床杂志,2009,(1):24


\subsection{第10章 前列腺增生}

前列腺增生是老年男性的疾病。在病理上,30岁以上的男性,其前列腺就可发生增生性改变,以后随年龄增长,发病率也随之增加。我国的发病率大大低于国外,但近年来,由于生活水平逐步提高,平均寿命有所延长,其发病率也有逐渐上升的趋势。

前列腺增生临床主要表现为排尿困难,但在病变过程中并可出现尿潴留、充盈性尿失禁、血尿等病理变化。

1.龙胆桃夏汤

【药物组成】龙胆草6g,桃仁、红花、大贝母、黄芪、桔梗各10g,夏枯草、车前子各15g,土茯苓、萆薢各30g,赤芍20g。

【随症加减】尿道有灼热感,尿道口有白色分泌物,下焦湿热甚者,加蒲公英、黄柏;小便涩滞不畅,舌黯或舌有瘀点、瘀斑、脉弦涩者,加丹参、王不留行;腰膝酸软、阳痿、早泄、神疲纳差者,加熟地黄、杜仲、山茱萸、菟丝子;伴血尿、血精者,加茜草、白茅根、三七粉;大便秘结者,加酒大黄、玄明粉;小腹坠胀者加升麻;胸胁胀满、情志抑郁、多烦易怒者,加沉香、乌药。

【治疗方法】每日1剂,连煎3次,取汁约1 000ml,分早、中、晚3次温服。30天为1个疗程。

【功 效】利湿,活血,化痰。

【临床运用】临床治疗34例。显效22例,有效9例,无效3例,总有效率91.1\%。

治验:陈某,男,68岁。2001年11月12日就诊。主诉:小便滴沥不尽、尿细、夜尿多、小腹坠胀,反复发作2年。多家医院诊断为前列腺增生症,曾服前列康等药,症状时轻时重。3天前饮酒后再次发病,小便点滴而出,小腹坠胀,烦躁不安,舌黯有瘀点,脉弦。直肠指检;前列腺如鸡蛋大小,中央沟变浅。B超显示:前列腺53mm×35mm×32mm。中医诊断为癃闭;西医诊断为前列腺增生症。治以利湿活血化痰。药用龙胆桃夏汤加丹参、王不留行各15g,沉香、乌药、升麻各10g。服药4剂后,小便通畅,小腹坠胀,烦躁症状明显减轻,精神好转。服用1个疗程后,诸症消失。继予1个疗程巩固疗效,复查B超:前列腺45mm×27mm×24mm。

【经验心得】本病多因感受湿热之邪,或嗜食肥甘厚味,中焦湿热浊毒不解,下注膀胱,膀胱气化不利,致小便不通。故用龙胆草、土茯苓、萆薢等清热利湿之品清利下焦之湿热;湿热流注下焦,日久则气郁血滞、脉络瘀阻,致使病情加重或缠绵难愈,故用桃仁、红花、赤芍活血化瘀,并改善局部血液循环,消炎止痛,抑制纤维母细胞增生,使前列腺变软、回缩,对血瘀阻滞之痞块有良好疗效。老年人肾阳不足、脾失健运,体内津液输布失常、聚而为痰;肾阴不足,相火妄动,煎熬津液,凝而为痰;或因肝气不舒、升降失常,三焦气机不利,聚津为痰。痰浊凝聚,阻碍气血运行,湿热、瘀血、痰浊互结,日久不散,凝结成块,阻滞尿道而致小便不利,故用车前子、夏枯草、大贝母化痰通闭、软坚散结;黄芪、桔梗有“下病治上、欲降先升”之意,为“提壶揭盖”之法。诸药合用,有利湿、活血、化痰之功,再根据病症主次和兼症多少的不同,分别增加清热利湿、活血化痰、滋阴温阳、凉血止血、疏肝利气之药。用药切中病机,方能获满意疗效。

【方剂出处】刘松,等.龙胆桃夏汤治疗前列腺增生症34例.陕西中医,2008,29(4):423

2.穿甲八正散

【药物组成】穿山甲20g,瞿麦15g,萹蓄15g,王不留行15g,石韦15g,牛膝15g,车前子15g,黄芪20g,柴胡10g,冬葵子15g,升麻10g,红花10g,白花蛇舌草20g,栀子10g。

【随症加减】肾阳虚明显者去栀子,加制附子30g(散剂,改用鹿角胶15g),肉桂12g;阴虚加生地黄15g,大枣皮12g;湿热重加苍术12g,黄柏10g。

【治疗方法】上药水煎取汁,每日3次,每日1剂,连服半个月,之后再用上药研粉混合,每次10g,每日3次,连服2个半月后观察。

【功 效】活血化瘀,清热利尿,升清降浊。

【临床运用】临床治疗38例患者,显效15例,好转18例,无效5例,总有效率86.8\%。

【经验心得】方中用穿山甲为主药,配合王不留行、红花活血化瘀散结,黄芪、升麻、柴胡升清降浊;瞿麦、萹蓄、石韦、车前子、冬葵子、栀子、黄柏、苍术、白花蛇舌草、牛膝清热化湿利尿。共奏利尿通闭、活血祛瘀散结之功。由于增生的前列腺血循环较差,很多药物难以进入前列腺内发挥作用,故选用穿山甲活血化瘀、消癥为主药,以期直达病所,提高治疗效果。张锡纯在《医学衷中参西录》中说:“穿山甲善行五脏六腑,凡血凝血聚之病,皆能开之,至癥瘕积聚疼痛麻痹,两便闭塞诸证,用药不效者,皆可以穿山甲作向导。”

【方剂出处】郭晓云.穿甲八正散加减治疗前列腺增生38例.云南中医中药杂志,2001,22(4):25

3.倒换散加味

【药物组成】荆芥20g,大黄15g,瞿麦、石韦、冬葵子、茯苓各12g,青皮、陈皮各6g,泽泻、丹参、车前子各15g。

【随症加减】气虚者加生黄芪30g,党参18g,升麻6g;合并膀胱湿热者加黄柏12g,龙胆草6g;血瘀甚者加桃仁10g,牛膝20g。

【治疗方法】每日1剂,水煎,分2次服。7日为1个疗程。

【功 效】清热通腑,行气活血,通利小便。

【临床运用】临床治疗78例,78例患者经治疗3~7日后,拔除尿管后未再见尿潴留。继续服用原方加减或改服八正散30g/日,补中益气丸(浓缩)24粒/日,每日分3次服,连服2~5个月巩固疗效。

【经验心得】前列腺增生发展到急性尿潴留,属于中医学由癃至闭的证候演变,虽留置尿管排尿可暂时解除尿闭之苦,然其三焦气化功能终未能恢复。倒换散出自《医方考》,治疗内热小便不通。方中荆芥之轻清以升其阳,大黄之重浊以降其阴,清阳即出上窍,则浊阴自归下窍,而小便随泄矣。方名倒换者,小便不通,倍用荆芥,大便不通,倍用大黄,颠倒而用,故曰倒换,为主药;青皮、陈皮、丹参行气活血;瞿麦、石韦、冬葵子、茯苓、泽泻、炒车前子利水通淋。诸药合用,共奏清热通腑、行气活血、通利小便之功效。

【方剂出处】陈忠伟.倒换散加味治疗前列腺增生致急性尿潴留78例.河北中医,2007,29(5):433

4.抵当汤加减

【药物组成】水蛭10g,土鳖虫15g,炮穿山甲10g,制乳香10g,制没药10g,沉香末3g,川牛膝15g,大黄10g,琥珀3g,薏苡仁20g,黄芪20g,肉桂6g。

【治疗方法】以上各药研末,每次5g,每日2~3次,蜂蜜水调服或胶囊装服,90日为1个疗程,连服2个疗程。体质较好、合并尿潴留者,每次用量可增至8~10g;症状较轻或体质较差、年龄较大者,减为每日2次。疗程结束、症状基本消失后,可继续每日1次服用,每次3~5g,以巩固疗效。同时,在发病早期要根据患者具体情况,配合使用利湿、益气、清热、理气等中药方剂煎服,以增强疗效。

【功 效】祛瘀血,补中气,助气化,利三焦。

【临床运用】临床治疗49例,治愈3例,显效27例,有效19例,无效0例,总有效率为100\%。

【经验心得】前列腺增生症患者的发病年龄多在50岁以上,年龄越大,发病率越高,年老体虚、命门火衰、气化失常、气滞血瘀、三焦失利是其发病主要原因,单纯使用西医治疗,患者症状解除较快,但前列腺难于缩小,且较易复发,仅达治标作用。在治疗上采用中西医结合方法,西医以解除急性症状、缓解患者痛苦、减轻前列腺充血及尿路炎症为主,中医根据“久病必瘀”“血瘀水道不利”的理论,使用以活血祛瘀药为主的中药标本兼治,疗效明显提高。自拟前列增生散方用水蛭、土鳖虫破血逐瘀,炮穿山甲善走窜、散结消肿,制乳香、制没药活血消肿止痛、软化增生组织,大黄泻下通滞、活血化瘀,川牛膝破血通经、引药下行,且补肝肾强筋骨,沉香温肾导滞、引药达下焦,薏苡仁益脾渗湿、疏导下焦,黄芪具有益气健脾固中、攻逐不伤正的作用。药理研究,黄芪具有增强机体免疫功能、利尿、抗衰老作用,肉桂温阳化气、扶助元火,琥珀散结利尿通淋,诸药合用共奏祛瘀血、补中气、助气化、利三焦之功,达尿窍通畅、诸症自除之效。因该药制成散剂或胶囊,服用方便且费用较低而受患者欢迎。

【方剂出处】张仁良.自拟前列增生散治疗前列腺增生症临床体会(附49例报告).实用中西医结合临床,2006,6(3):60

5.通 关 丸

【药物组成】炒知母、炒黄柏、王不留行、川牛膝、萹蓄各15g,红藤、黄芪各20g,肉桂、升麻各3g,虎杖30g,当归10g,穿山甲5g。

【随症加减】伴尿脓者加白花蛇舌草、生薏苡仁各15g,蒲公英30g;尿血加白茅根30g,地榆、大蓟、小蓟各20g;尿痛加海金沙、石韦各15g;便秘加桃仁、大黄各10g。

【治疗方法】每日1剂,水煎服分2次,15日为1个疗程。服药期间禁烟酒,忌食生冷、辛辣、油腻之物,禁房事,戒恚怒。

【功 效】滋阴补肾,清热利湿,软坚散结。

【临床运用】临床治疗100例。显效59例,有效27例,无效14例,总有效率86\%。

治验:钟某,男,79岁。2001年3月10日初诊。夜尿增多8年余,近1年出现尿急、尿线细、淋漓不尽、排尿等待。经服中西药罔效。刻诊:小便不畅,夜五六次,色黄且痛,眼脸浮肿,足肿按之如泥,凹陷不起,面色灰滞,舌边黯红,苔白干燥,脉沉细涩。B超示:前列腺I度增生并膀胱积液500ml。证属脾肾气阴两虚,湿热瘀阻下焦,膀胱宣化失司。治宜滋阴清热,益气利尿,化瘀通淋。方药:炒知母、炒黄柏、王不留行、川牛膝、萹蓄各15g,红藤、黄芪各20g,虎杖30g,防己12g,当归10g,穿山甲5g,生升麻3g,肉桂2g。服药5剂后小溲转利,量亦增多,尿痛亦止,足肿已去三分之二。苔薄白,脉沉细。B超示:膀胱余尿仅100ml。服药1个疗程后,足肿消退,小便次数减至每夜一二次,色深黄且浑,味臭,纳呆,乏力,脉沉细,苔薄,舌质黯红。患者高年体弱,脾肾两亏,阳不足则阴无以化,正气虚则湿热未彻之故,当标本兼顾,重用萹蓄30g,加淡附子10g,继服1个疗程后,诸症消失,B超示:膀胱无积液。为巩固疗效,继服1个疗程善后。随访1年,病未复发。

【经验心得】前列腺增生症多为年高后三焦气化失职,尤其下焦气虚推动无力,瘀阻尿路,使水液排泄障碍。其中虚是瘀的基础,瘀是虚的结果,所以活血化瘀、软坚散结贯穿本病治疗全过程。临床上应滋阴补肾,益气健脾,清热利湿,化瘀消症,软坚散结,方选通关丸加味而治之。方中知母、黄芪滋阴益气,肉桂补火通脉,黄柏、萹蓄清热利湿,红藤、川牛膝、王不留行、当归清热活血消肿,穿山甲软坚散结,虎杖有散精而通利之效,生升麻能使气机通调而小便自利。诸药合用,阴阳(脾肾)得补,气机通畅,湿热得清,瘀浊得除,癥瘕得消,窍道通畅,开合有度,而获佳效。

【方剂出处】赵现朝,等.通关丸加味治疗前列腺增生症100例.浙江中医杂志,2008,43(5):275

6.通塞克癃汤

【药物组成】生牡蛎30g(先煎),海藻15g,地龙15g,虎杖15g,炮穿山甲15g,皂角刺10g,王不留行15g,丹参15g,当归12g,生黄芪30g,川牛膝10g。

【随症加减】伴小腹胀痛、大便作坠者,加柴胡6g,乌药10g;小便短赤涩者,加瞿麦10g,车前子15g;尿道溢脓者,加蒲公英15g;伴尿血者,加小蓟30g,生蒲黄12g。

【治疗方法】水煎服,每日1剂,10剂为1个疗程。

【功 效】软坚化结,解除梗阻,通利水道。

【临床运用】临床治疗47例。治愈32例,有效11例,无效4例,总有效率91.5\%。

【经验心得】《素问·宣明五气篇》说:“膀胱不利为癃,不约为遗弱。”故其病因虽众,然终将归于膀胱不利,尿路受阻,小便难以排出。正如张景岳所说:“或以败精,或以搞血,阻塞水道而不通也。”现代医学认为,前列腺增生症主要是因为膀胱颈部阻塞而引起梗阻性排尿症状。古今之说,如出一辙。由此,软坚化结、解除梗阻、通利水道方为治癃之要。方中重用牡蛎、海藻为先,软坚散结;虎杖、丹参、当归、皂角刺、王不留行诸药以炮穿山甲为统帅,势如破竹,直指病所,通瘀化结,透达关窍,近代张锡纯曾言:“二便闭塞诸症,用药不效者,皆可加穿山甲作导向”;佐以地龙、川牛膝清热利尿通淋,活血化瘀通络,黄芪升清降浊,鼓舞正气,以强三焦气化之功。综观全方,刚柔并济,使瘀结得化,湿热得清,尿路无阻,则癃闭自除矣。

【方剂出处】侯生芹.通塞克癃汤治疗前列腺增生症47例.中医药临床杂志,2007,19(6):610

7.温肾益气汤

【药物组成】熟地黄24g,山药、山茱萸各12g,牡丹皮、泽泻、茯苓各9g,附子、肉桂各6g,淫羊藿、巴戟天、海金沙各15g,沉香5g。

【随症加减】合并泌尿系感染者减附子、肉桂,加金钱草、白茅根、车前子或八正散。

【治疗方法】每日1剂,水煎2次,取汁400ml,分2次温服,15日为1个疗程。

【功 效】温补肾阳,化气行水。

【临床运用】临床治疗31例。显效12例,有效17例,无效2例,总有效率93.5\%。

治验:刘某,男,68岁。2005年7月3日初诊。半个月前出现小便淋漓不畅,夜间加重,B超示前列腺增生。曾用中西药物治疗不效,尿点滴而出,由癃成闭,遂插尿管准备手术治疗,因患者年高畏惧手术,故而求诊中医。症见:面色无华,精神委靡,腰腿酸软,卧床不起,小腹微胀,手足不温,畏寒怕冷,舌淡苔白,脉沉细而弱。辨证为癃闭证。治宜峻补肾气,助其气化。药用温肾益气汤:熟地黄24g,山药、山茱萸各12g,牡丹皮、泽泻、茯苓各9g,附子、肉桂各6g,淫羊藿、巴戟天、海金沙各15g,水煎服,日1剂。6剂后,精神稍振,觉有尿意仍不能自解,脉象稍复,守原方加沉香5g。9剂后,导尿管被尿液推出,此后小便已能自解,唯夜尿频数。又服5剂,余症均除。嘱继服金匮肾气丸1个月,巩固疗效,随访2年未发。

【经验心得】正常人小便的通畅,有赖于三焦气化的正常,但究其三焦气化之本,则源于肾所藏的精气。《素问·上古天真论》云:“男子五八肾气衰,发堕齿槁……八八天癸竭,精少,肾脏衰,形体皆极。”男性年老体弱,肾阳不足,命门火衰,所谓:“无阳则阴无以生”,致膀胱气化无权,而尿不得出。温补肾阳,化气行水,使小便得以通利。故投金匮肾气丸温补肾气以助气化,加淫羊藿、巴戟天温阳而不燥,佐海金沙意在消尿道之塞,再加沉香取其纳气归原,药合病机,收功必然。

【方剂出处】杨运池.温肾益气汤治疗前列腺增生31例.河北中医药学报,2008,23(4):26

8.仙 甲 汤

【药物组成】淫羊藿10g,枸杞子15g,巴戟天10g,炮穿山甲10g,丹参20g,莪术10g,败酱草20g,连翘20g,川芎15g,海藻15g,甘草6g。

【随症加减】症见尿频急、疼痛,尿道灼热,口苦咽干,烦闷呕恶,大便干结或黏滞不爽,舌质红,苔黄腻,脉滑数或弦数,治以基本方合四妙散、八正散方;症见小便滴淋不畅,时发时止,经久不愈,五心烦热,夜寐不安,口干咽燥,头晕耳鸣,舌红苔少,脉细数,治宜滋阴降火,基本方合知柏地黄汤;症见小便不畅或点滴而下,胸闷气短,咳嗽喘逆,或伴恶寒发热,呕恶痰涎,舌淡苔白,脉浮或滑,治宜宣畅肺气,祛邪利窍,合三拗汤;症见神疲倦怠,尿出无力,滴沥不畅,尿液澄清,畏寒肢冷,腰膝酸困,舌淡苔白,脉沉细弱,治宜温肾助阳,配合金匮肾气汤;症见小腹坠胀,时欲小便而不得出,或量少而不畅,神疲气短,倦怠乏力,舌淡苔薄,脉沉细弱,治宜健脾益气,升清降浊,合补中益气汤;症见小便点滴而下或闭塞不通,或尿时涩痛,或小腹胀痛,面色晦暗,舌质紫黯,或有瘀点,舌体胖大,有齿痕,脉涩,治宜化痰软坚、消瘀散结、通利水道,合抵挡丸。

【治疗方法】中药每日1剂,水煎取汁分3次服,15天为1个疗程。服药期间禁烟酒,忌食生冷、辛辣、油腻之物,节制房事。

【功 效】补益肾气,活血化瘀,化痰散结,清热利湿。

【临床运用】临床治疗57例。显效23例,好转26例,无效8例,总有效率为85.96\%。

治验:某患者,68岁。小便频数、尿线变细、排尿等待、淋漓不尽10余年,近2个月加重,曾多方治疗,病情无好转而要求中医治疗。B超检查:前列腺Ⅱ度增生并尿潴留150ml。现症见小便频数,滴沥刺痛,欲出不尽,畏寒肢冷,腰膝酸困,舌淡苔白,脉沉细弱。证属肾阳虚衰型,用仙甲汤基本方合金匮肾气汤加减治疗:制附子10g,淫羊藿10g,枸杞子15g,巴戟天10g,炮穿山甲10g,丹参20g,莪术10g,败酱草20g,连翘20g,海藻15g,茯苓15g,熟地黄15g,山药15g,鹿角霜15g,黄芪15g,肉桂6g,甘草6g。治疗2个疗程,尿流基本正常,B超复查前列腺Ⅰ度增生,残余尿为50ml。继续治疗2个疗程,诸症消失,B超检查前列腺稍大于正常,无残余尿,随访1年,病未反复。

【经验心得】仙甲汤以补肾泻肝立法,由淫羊藿、枸杞子、穿山甲、丹参、连翘、败酱草等组成。方中淫羊藿归肝、肾经,补肾壮阳,枸杞子归肝、肾经,滋补肾阴,同时制约淫羊藿之温燥,淫羊藿与枸杞子,二者一阴一阳,一味温而不伤阴,一味滋而不碍阳,一味促肾气之温煦、激发以助肾阳,一味充肾气之滋润、营养以协肾阴,两者阴阳互补,相须为用,共奏补益肾气之功效;穿山甲为血肉有情之品,归肝经,性味咸寒,善走窜,性专行散,活血化瘀,能引药直达病所,为破瘀消癥要药;走窜之穿山甲祛瘀生新,可助淫羊藿、枸杞子充养肾气,内守之淫羊藿、枸杞子补益肾气,又助穿山甲祛瘀血以消癥积;丹参归肝经,活血祛瘀,与穿山甲合用,加强活血破瘀消癥之力;败酱草味辛苦性微寒,入大肠、肝经,清热解毒;连翘味苦性微寒,入肺、心、小肠经,清热解毒,消肿散结;败酱草、连翘还兼有活血化瘀、消癥散结之作用,与穿山甲、丹参等配伍使用,可增强其活血化瘀之力。诸药配伍,肝肾同治,补泻并举,补攻兼施,守走相备,相得益彰,扶正而不留邪,祛邪而不伤正,共奏补益肾气、活血化瘀、化痰散结、清热利湿之功。

【方剂出处】曹继刚,等.仙甲汤加减治疗前列腺增生57例临床观察.云南中医中药杂志,2008,29(7):31

9.血府逐瘀汤加减

【药物组成】熟地黄20g,柴胡15g,枳壳15g,桃仁12g,红花10g,当归12g,赤芍15g,川芎15g,川牛膝12g,穿山甲10g,三棱10g,莪术10g。

【随症加减】伴尿频、尿急、尿痛等下焦湿热者加萹蓄、栀子各15g,黄柏20g;伴血尿者加白茅根15g;排尿不畅者加地龙10g;夜尿频加覆盆子15g。

【治疗方法】 每日1剂,水煎取汁300ml,上、下午各服150ml,30天为1个疗程。

【功 效】软坚散结,活血消瘀,补肾益精。

【临床运用】临床治疗56例。显效38例,有效15例,无效3例,总有效率94.65\%。

治验:李某,男,62岁。2002年3月6日初诊,患者自1996年以来,反复出现排尿滴沥不尽,经多次保留导尿管及中西药治疗效果欠佳,上级医院建议手术治疗,患者不愿手术,经病友介绍前来就诊。刻诊:尿频,尿急,小便点滴而下,小腹胀满,腰酸乏力。直肠指检前列腺大如鸡蛋,中央沟消失,轻度压痛。B超示:膀胱中度充盈,残余尿>120ml,前列腺大小4.9cm×5.5cm,边界欠清,回声欠均。舌黯红有瘀斑,苔薄,脉细涩。中医诊断为癃闭(肾虚血瘀)。治宜补肾活血化瘀。方药:熟地黄20g,柴胡15g,枳壳15g,桃仁12g,红花10g,当归12g,赤芍15g,川芎15g,川牛膝12g,山茱萸12g,地龙10g,穿山甲10g,三棱10g,莪术10g。8剂后,排尿通畅,但排尿无力,夜尿5~6次。前方加减调理3个月诸症消失,直肠指诊前列腺明显减小,中央沟表浅。B超检查前列腺2.9cm×3.2cm。随访至今未复发。

【经验心得】《内经》曰:“膀胱不利为癃”。膀胱与肾相表里,膀胱的功能有赖于肾的气化,肾主气化而司开合,年老体弱,肾气不足,日久影响气血运行,瘀血内停,或气滞血瘀,或因嗜食辛辣,或饮酒嗜欲导致前列腺急性充血,使瘀血凝滞,逐渐形成包块,在直肠指检时,可触及增生的前列腺。其阻塞于尿道膀胱之间,致使排尿困难,滴沥不尽,甚至塞闭不通。《内经》云:“坚者削之”“结者散之”“虚者补之”。故组方以软坚散结、活血消瘀为主,佐以补肾益精。方中枳壳、柴胡行气导滞,气行则血行;当归、桃仁、红花、赤芍、川芎、三棱、莪术活血祛瘀;熟地黄、川牛膝补肾益精;穿山甲作为引经之药,能活血消肿,性善走窜,通经络直达病所。中药药理研究表明,当归、川芎、桃仁、红花、赤芍、三棱、莪术活血通络,可以改善血循环及血液流变学的性质,从而改变病灶的血液供应;川牛膝抗炎消肿及扩张血管、改善微循环,促进炎性病变的吸收;熟地黄、穿山甲升高外周白细胞,增强免疫功能。

【方剂出处】饶应良.血府逐瘀汤加减治疗前列腺增生症56例.吉林中医药,2008,28(3):197

10.软坚化癥散

【药物组成】生黄芪、肉苁蓉、淫羊藿、刘寄奴各120g,穿山甲、莪术、水蛭、牡蛎、贝母、白芥子、生半夏、夏枯草、王不留行、川牛膝各60g,琥珀、沉香各30g。

【随症加减】肾阴虚配服六味地黄丸;肾阳虚配服八味地黄丸;中气不足配服补中益气丸;湿热盛加半枝莲、蒲公英、白花蛇舌草;血尿加白茅根、茜草根、三七;并发泌尿系结石加金钱草、冬葵子、石韦;夜尿频多加菟丝子、覆盆子、巴戟天、桑螵蛸。

【治疗方法】诸药共为细末,每次服20g,早、晚各服1次。1个月为1个疗程。共3个疗程。

【功 效】祛瘀化痰,软坚化癥。

【临床运用】临床治疗42例患者。治愈35例,好转5例,无效2例。总有效率95.3\%。

【经验心得】本方中生黄芪、肉苁蓉、淫羊藿补气温肾以培本扶正,以防克伐正气;刘寄奴、穿山甲、莪术、水蛭、王不留行、琥珀活血化瘀利水,软坚散结消癥;牡蛎、贝母、白芥子、生半夏、夏枯草化痰软坚散结,沉香行下焦气滞,牛膝引药下行。穿山甲可增加前列腺的通透性。诸药合用,共奏化痰、祛瘀、软坚化癥之功。

【方剂出处】张振卿.化痰祛瘀软坚化癥法治疗前列腺增生症的体会.四川中医,2003,21(5):25

11.真武汤加减

【药物组成】炮附子15g(先煎),黄芪15g,白芍15g,白术12g,车前子12g,生姜12g,滑石15g,丹参15g,升麻10g,橘核10g,淫羊藿12g,琥珀3g(冲)。

【治疗方法】水煎服,每日1剂,早、晚各服1次,10天为1个疗程。

【功 效】温阳化气,利水通闭。

【临床运用】临床治疗54例。治愈29例,好转23例,无效2例,总有效率为96.3\%。

治验:张某,68岁,小腹坠胀,夜尿增多,8~10次/晚,排尿费力,尿流变细,排尿时间延长已超过10日,服用酚苄明、己烯雌酚及其他抗生素治疗效果欠佳,并逐渐出现排尿困难,小便欲解不得出,有时呈点滴状排尿,伴有腰膝酸软,精神委靡,食欲不振,面色白,舌淡,苔薄白,脉沉细弱,于2001年11月18日急诊入院,查体:膀胱区胀满,叩诊明显浊音,直肠指检见:前列腺明显增大,中央沟消失,质中等,表面光滑无结节,触痛明显。B超检查示前列腺增生,突入膀胱。经服黄芪甘草汤加减,每日1剂,早、晚分服,连服10日排尿症状改善,前列腺指检较前缩小。

【经验心得】肾对尿液的生成具有决定性意义,故有“肾主水”之说。肾与膀胱相表里,肾气在参与膀胱气化和司膀胱开阖方面有着重要的作用,肾中精气的蒸腾气化促使尿液生成,膀胱贮存、气化尿液,司尿液的随意排泄。故《素问·灵兰秘典论篇》说:“膀胱者,州都之官,津液藏焉,气化则能出矣。”老年人由于肾气渐衰,肾阳气不足,固摄无权,膀胱气化乏力,开阖失控,再加上其他疾病的多元影响,临床虚证者居多,实证者居少;良性前列腺增生症是由于患者年事已高,肾气衰退、中气不足,经脉循行不畅,水瘀互阻化热,更由于耗伤精气,而形成恶性循环,其病机本质以虚为本,以实为标。中医药治疗良性前列腺增生症报道很多,但辨证分型太过繁琐,临床医生难以抓住主问题的关键。所以,根据其疾病本质,以虚为特征,重用炮附子为君温煦少阴之阳,恢复肾脏气化,黄芪与升麻共举中气,白术健脾运湿,生姜温胃散水,车前子主气癃、利水道,下走膀胱以行水。全方提升中气、上开肺气,以升清降浊,补肾阳以助蒸腾气化之功能,上气升则下窍自通,乃下病上取之法。

【方剂出处】杨楠.真武汤加减治疗前列腺增生症54例.甘肃中医,2009,22(2):12

12.滋 肾 丸

【药物组成】肉桂10g,知母15g,黄柏15g,黄芪30g,大黄10g(后下),芒硝8g(溶服),茯苓20g,桃仁12g,牛膝15g,穿山甲10g,金钱草30g,滑石20g。

【随症加减】偏寒加制附子9g,肉桂用至12g,减去知母;肿甚加薏苡仁20g,防己10g,大便燥结加天花粉30g,芒硝用量可至10g;尿道灼热疼痛,加重滑石,加生地黄;睾丸、精索腹痛明显加延胡索、荔枝核,腰骶痛加杜仲、续断;性功能减退加淫羊藿、黄精;前列腺液中脓细胞多者加金银花、蒲公英;前列腺液或精液中有红细胞者加墨旱莲、白茅根;前列腺质偏硬、高低不平或有结节者加莪术、鳖甲、龟甲。

【治疗方法】每日1剂,水煎,分2次服。药渣煎水,温水坐浴,并定期给患者做前列腺按摩。

【功 效】补肾益气,化瘀开结。

【临床运用】临床治疗50例,显效18例,有效29例,无效3例,总有效率为94\%。

【经验心得】滋肾丸又名通关丸、滋肾通关丸,系李东垣所著《兰室秘藏》方,《中国医学大辞典》:“此方以黄柏之苦寒清肾中之伏热,补水润燥,故以为君。以知母之苦寒,滋肺经之化原,泻肾火,故以为佐。而以肉桂之辛温引之,服后觉前阴若刀刺火烧,溺如涌泉,则膀胱之气化矣。”医圣仲景曰:“是瘀血也,当下之。”久病多瘀、年老多瘀的血瘀学说,用大黄、滑石、芒硝通关;以穿山甲、桃仁行血破瘀,通经散结;黄芪补气而利尿;茯芩补心脾益肺,利水通窍除邪热;金钱草通淋利水;牛膝祛瘀通淋,引药下达,本方合用具有升清、降浊、祛瘀散结、化气利水的作用,此病治疗的原则是:“腑以通为用。”择重于通,治宜补脾肾助气化以散瘀结。通水道而达到气化得行,则小便自通的目的,方药熨帖,故有良效。

【方剂出处】钟明平.加味滋肾丸治疗前列腺增生症50例.光明中医,2002,17(4):52

13.海底玉壶丸

【药物组成】 何首乌50g,炒杜仲25g,当归25g,沙苑子25g,菟丝子25g,补骨脂25g,巴戟天25g,淫羊藿12g,茯苓12g,莲须12g,枸杞子18g,怀牛膝18g,草河车18g,板蓝根18g,大青叶18g,蜈蚣40条,肉桂15g,知母18g,川黄柏18g。

【治疗方法】炼蜜为丸,每丸10g,每服2丸,日服2次,淡盐水送下,30日为1个疗程。共治疗3个疗程。

【功 效】温阳补肾,祛痰通络。

【临床运用】临床治疗53例,经治1个疗程后有效51例,有效率达96\%;3个疗程后全部达到显效。

【经验心得】临床常见患者不是单纯的阳虚,而是痰浊阻滞经络并发,两者相互影响而成病。临床治疗,如用化痰祛浊、消导之法可使阳气更虚,如用温补元阳之法可使痰浊胶滞化热。故治之之法,当温阳化痰法并施。自拟海底玉壶丸中,何首乌温肾养肝、延缓衰老;杜仲补肝肾,强筋骨,当归补血活血,调经止痛,治血虚结聚;沙苑子补肾固经,益肝明目,治尿频、遗尿;菟丝子养肌中阴,坚筋骨,主茎中寒、尿有余沥,可增强性腺功能;补骨脂补肾助阳,治尿频、遗尿;巴戟天补肾强肝,治尿不尽,有强壮、抗疲劳之作用,改善男性性腺器官的活力;淫羊藿辛、甘,温,补肾壮阳,治尿频失禁,可改善阳虚体质,增加性功能;茯苓宁心安神,健脾和胃,通利水湿,可使群药补而不滞;莲须可清心益肾,治尿频、遗尿,使诸温养之药不扰心神;枸杞子滋肾养肝润肺,可补益经气,强盛阴道;怀牛膝补肝肾、强筋骨,草河车、板蓝根、大青叶三药散结消肿,蜈蚣入肝,直入阴器,攻毒散结,四药合用,攻痰通络,导邪出巢;川黄柏、知母、肉桂三药乃东垣通关丸,引邪出巢,使邪顺小水而出。

【方剂出处】赵学勤.海底玉壶丸治疗阳虚痰阻型前列腺增生53例.北京中医,2007,26(9):622


\subsection{第11章 男性乳房发育症}

男性乳房发育症是以男性乳房一侧或两侧呈女性化发育肥大,甚可分泌乳汁样液体,局部胀痛,乳晕周围色素沉着加深,有压痛、触痛,触之有肿块硬节为主要表现的一类疾病,属内分泌疾病,与体内雄雌激素比例失调,或雌激素受体增加,或乳腺组织对雌激素敏感性增加等有关。肝功能失代偿期,甲状腺功能亢进,某些自身免疫性疾病,精神系统疾病,营养失衡致肥胖,使用己烯雌酚、西咪替丁等药物,服用含性激素样营养品等原因均可诱发、加重本病。临床多见于青春期、中老年期患者。

1.二 仙 汤

【药物组成】仙茅、淫羊藿、知母、黄柏、巴戟天、穿山甲各10g,当归、茯苓、夏枯草各20g,甘草5g。

【治疗方法】加水煎至250ml,每日1剂,早、晚分服。10剂为1个疗程,连服3个疗程。

配合外敷:将麝香、没药、乳香、血竭,按比例(为1︰20︰20︰10)研粉加入消炎止痛膏中外敷,48小时更换1次。

【功 效】止痛散结。

【临床运用】临床治疗100例。痊愈90例,好转8例,无效2例,总有效率为98\%。

治验:乔某,男,61岁。1992年10月20日初诊。右乳疼痛3月余前来就诊。左乳疼痛有包块,在某医院已手术切除,术后2个月右乳又痛,触之有硬结。患者不愿再手术,曾在私人诊所服中药(内容不详),效果不佳,后听别人介绍到医院诊治。局部检查,左乳头缺无,右乳肿胀、触压痛,可扪及3cm×3cm片状块,质韧,活动,界限清楚,包块与皮肤及基底部无粘连,舌红苔薄白,脉细弦,无烟酒嗜好。诊为肾阳虚,肾精不充,痰瘀互结乳疬。治法:温肾阳,滋肾阴,调冲任,散结块,内服汤药二仙汤加减。仙茅、淫羊藿、知母、穿山甲、巴戟天、黄柏各10g,当归、茯苓、夏枯草各20g,甘草5g,10剂;配合外敷,48小时更换1次。服10剂后,疼痛明显减轻,乳房包块变软;上方治疗1个月后,右乳包块及疼痛消失,1年后随访:右乳无恙。

【经验心得】陈实功《外科正宗·乳痈论》中探讨的本病的病因病机,认为“男子乳节与妇女微异,女损肝胃,男损肝肾,盖怒火,房欲过度,以此肝虚血燥,肾虚精怯,血脉不得上行,肝经无以荣养,遂结肿痛。”

治疗上以补气血,调肝肾为本之法。男子乳头属肝,乳房属肾,若情志不调,或年老体虚,久病及肾;或先天禀赋不足,冲任失调,或外邪伤肝,肝失柔养,皆可导致经络失养,气血不畅,从而出现瘀血、痰浊,阻滞经脉而成乳疬。冲任隶属肝肾,肾阳虚,肝血不足,肾精不充,冲任失调是本病之本,瘀血、痰浊积聚乳房为本病之标。治疗上当以补其病本,治其病标为法,方能取得疗效。方中仙茅、淫羊藿、巴戟天温补肾阳,温化痰浊;知母、黄柏滋肾阴,填精髓;当归、穿山甲养血活血、止疼散结;茯苓、夏枯草健脾化痰散结;甘草调和诸药,诸药合用能温肾阳、补肾精、养肝血、调冲任、散瘀血、化痰浊,共奏止痛散结之效。

【方剂出处】何凤贤.二仙汤配合外敷药治疗男性乳房异常发育症100例.陕西中医,2007,28(12):1630

2.化瘀通络散

【药物组成】露蜂房、重楼、乳香、没药、炒穿山甲、牡蛎、夏枯草、海藻、全蝎各30g,蜈蚣15g。

【治疗方法】以上各药干燥、研细、过筛、拌匀,装瓶盖紧备用。每次5g,饭后开水冲服,每日2次,治疗期间保持心情舒畅,忌辛辣、腥膻发物。

【功 效】解毒化瘀,软坚化积,消肿定痛。

【临床运用】临床治疗20例,治疗结果:服药2剂,肿块消散,症状解除14例;服药3剂,肿块消散,症状解除6例。

【经验心得】乳房为肝经所过,乳络气滞,痰凝血瘀,聚液生痰,痰结成核,发为乳癖。方中夏枯草解郁清热散结,为治肝热痰火郁结之要药;露蜂房散肿、定痛、攻毒;牡蛎、海藻软坚化痰;乳香、没药为宣通脏腑,疏通经络之要药,气血凝滞者,此二味皆能流通,蜈蚣、全蝎、穿山甲相伍,攻坚散结之功最卓著;诸药合用,解毒化瘀,软坚化积,消肿定痛,理气通络,功效尤强。

【方剂出处】杨光能.化瘀通络散治疗成年男性乳房发育症20例.实用中医药杂志,1999,15(3):18

3.消 疬 汤

【药物组成】仙茅、淫羊藿、巴戟天各12g,鹿角粉3g(另包吞服),柴胡、青皮各10g,白芍、当归、贝母、王不留行各15g,瓜蒌30g。

【随症加减】胸胁胀痛者加川楝子、郁金各10g;脘腹痞闷、纳呆、苔腻者加制半夏、陈皮各10g;烦躁易怒、面赤升火者加牡丹皮、栀子各10g;口干、目涩、舌红少苔者加熟地黄、枸杞子各15g;肿块结硬难消者加炮穿山甲片10g。

【治疗方法】每日1剂,水煎,每日2次,温服,2个月为1个疗程。1个疗程后观察疗效。

【功 效】温肾壮阳,理气缓急。

【临床运用】临床治疗30例,治愈18例,显效8例,好转2例,无效2例,总有效率93.3\%。

【经验心得】本病是因内分泌紊乱,雄性激素分泌减少,而雌性激素分泌相对增多;或下丘脑-垂体泌乳素分泌增加,从而对乳房的发育失去了应有的控制作用,使乳房呈女性化增大。基于以上认识,消疬汤中用仙茅、淫羊藿、巴戟天、鹿角粉温补肾阳,提高雄性激素、类皮质激素水平,拮抗雌激素,抑制、降低泌乳素的分泌;柴胡、青皮疏肝理气;白芍柔肝缓急;当归养血和营,具有调节神经系统,改善内分泌紊乱的功能;贝母、瓜蒌化痰散结,清除病理产物;王不留行宣通乳络,改善微循环,且为乳病专用药。

【方剂出处】钱小强.消疬汤治疗男性乳房发育症30例.实用中医药杂志,1999,15(6):38

4.阳 和 汤

【药物组成】熟地黄30g,鹿角胶、生甘草各15g,白芥子、生姜炭、麻黄各10g,肉桂5g。

【随症加减】肿痛明显者加郁金、延胡索各15g,丹参40g;硬结较大且质地较硬者加夏枯草30g,天花粉15g;有分泌物者加麦芽40g。

【治疗方法】每日1剂,水煎饭后温服,日2次,每服250ml,30日为1个疗程。

【功 效】温散寒凝,宣通散结。

【临床运用】临床治疗28例,治愈4例,显效8例,好转14例,无效2例,总有效率为92.86\%。

【经验心得】男性乳房发育症属于中医“乳疬”范畴,根据“乳头属肝经,乳房属胃经”的理论,多与肝气郁滞、痰湿凝结有关。两乳乃肝经所主,肝气不舒,脾失健运,痰湿内生,痰气互结则变生此疾。阳和汤出自清代著名外科学家王洪绪的《外科证治全生集》,其功效为温阳补血、散寒通滞。方中熟地黄温补营血;鹿角胶养血助阳,强筋壮骨;肉桂、生姜炭破阴和阳,温经通脉;麻黄、白芥子通阳止痛;甘草调和诸药。熟地黄、麻黄同用于一方,熟地黄补血而不腻,麻黄通络而不散,一走一守,相得益彰。根据此病特点,采用阳和汤温散寒凝,补而不滞,温而不燥,宣通散结,收到明显效果,特别是疏肝解郁法治疗无效者效果更佳。除个别病例服药后有口干、口中发热外,未见其他不良反应。

【方剂出处】张锐,等.阳和汤治疗男性乳房发育症28例.

山西中医,2003,19(6):20

5.疏肝化痰方

【药物组成】柴胡10g,香附10g,橘核10g,淫羊藿10g,鹿角霜10g,陈皮10g,半夏10g,海藻21g,昆布21g,浙贝母15g,牡蛎20g,穿山甲10g,山楂10g,麦芽10g。

【随症加减】若疼痛较重加山慈菇10g,黄药子10g;乳头溢液加炒薏苡仁10g,伴阴虚者加白芍20g。

【治疗方法】上药加水300ml,煎30分钟,取汁约200ml;二煎加水200ml,煎取汁约100ml,2煎混合,每日1剂,早、晚分服。15日为1个疗程。

【功 效】疏肝温肾,化痰散结。

【临床运用】临床治疗60例,治疗结果,治愈44例,占73\%;显效15例,占25\%;无效1例,占2\%。总有效率98\%。

【经验心得】 中医学认为,本病多由先天禀赋不足或年老肾虚,复因恚怒伤肝,肝气郁结,乘脾虚生湿成痰,痰气互结,顺经结于乳房而成为本病。男子乳头属肝,乳房属肾,故肝郁肾虚是主要病机,治疗以疏肝温肾、化痰散结为主要治法。方中柴胡、香附、橘核、陈皮以疏肝理气,淫羊藿、鹿角霜以温补肾阳,半夏、浙贝母、陈皮、海藻、昆布、牡蛎、穿山甲、山楂、麦芽以化痰软坚散结。

【方剂出处】张宗建,等.疏肝化痰法治疗男性乳房发育症.山东中医杂志,1999,18(2):72

6.淫羊藿消瘰汤

【药物组成】淫羊藿10g,玄参20g,川贝母10g,牡蛎20g。

【随症加减】胀痛明显,酌加柴胡、橘叶、白芷、当归、赤芍、川芎;肿块较大较硬,酌加橘核、瓦楞子、海蛤壳、夏枯草。

【治疗方法】每日1剂,水煎服。

【功 效】调补阴阳,化痰活血,软坚散结。

【临床运用】临床治疗16例,治疗结果:16例均痊愈(肿块消失,胀痛消除,乳房恢复正常)。

【经验心得】消瘰丸为《医学心悟》方,由玄参、牡蛎、贝母组成。玄参滋阴降火,软坚散结;贝母解郁散结,化痰消肿;牡蛎益阴潜阳,化痰软坚。诸药合用,共奏清热化痰,软坚散结之功。淫羊藿为助阳益精,强筋健骨之品,据现代研究具雄激素样作用。本方用淫羊藿与消瘰丸组成汤剂,以淫羊藿、玄参调整肾阴阳的偏盛偏衰。患者舌质偏淡白者重用淫羊藿,最大用量可至30g。舌质偏红者,重用玄参。并随患者痰凝气滞血瘀的孰轻孰重,辨证加味用药,以达调补阴阳,化痰活血,软坚散结功效。

【方剂出处】周兴忠.淫羊藿消瘰汤治疗男性乳房发育症疗效观察.河北中医,2001,23(9):685

7.自拟柴甲汤

【药物组成】柴胡10g,穿山甲15g,僵蚕10g,橘核10g,香附10g,王不留行10g,白芥子10g,当归10g,法半夏10g,郁金10g。

【随症加减】肾阳虚加仙茅10g,淫羊藿10g,肾阴虚加枸杞子20g,熟地黄10g。

【治疗方法】每2日1剂,水煎3次,混匀药液1 800ml,分6次内服,1日3次。30天为1个疗程。用药1~4个疗程观察疗效。

【功 效】疏肝解郁,化痰软坚,活血散结。

【临床运用】临床治疗21例。治愈18例,占85.7\%;好转2例,占9.5\%;无效1例,占4.8\%。总有效率95.2\%。

【经验心得】男性乳房肥大症,中医学称之为乳疬。肾气不足,肝失所养,气滞、痰凝、血瘀为主要病机。现代医学认为,本病的发生主要是由于体内雌激素水平绝对或相对增高,或是由于乳腺组织对雌激素敏感性增加所致。中医以疏肝解郁、化痰软坚、活血散结为治则。方中柴胡、郁金、香附、橘核行气化痰,穿山甲、王不留行、当归活血散结,法半夏、僵蚕、白芥子化痰软坚,肾阳虚者,加仙茅、淫羊藿益肾阳,肾阴虚者,加枸杞子、熟地黄补肾阴,诸药合用,治疗本病,疗效显著。

【方剂出处】洪国平,等.自拟柴甲汤治疗男性乳房肥大症21例.光明中医,2008,23(8):1152


\subsection{第12章 阴囊湿疹}

阴囊湿疹,中医学称为“肾囊风”、绣球风,是一种阴囊变应性皮肤炎症,以阴囊剧烈瘙痒,出现红斑、丘疹、水疱、脓胞、糜烂、渗出、结痂、肥厚、鳞屑等多形性皮损,反复发作,迁延不愈为特征,是一种难治的过敏性、炎症性皮肤病,其病程反复,常给患者在精神情绪上带来压力和焦虑。

阴囊湿疹多因风湿热邪客于肌肤和血虚生风、生燥,肌肤失养所致。临床上大多为慢性或亚急性鳞屑性癣花肥厚皮疹,剧痒。由于奇痒难忍,不少患者用手抓,或用热水烫,引起红肿糜烂,皮肤湿润肥厚及苔藓样变,色素沉着或色素减退。饮酒吃辣或睡眠不足,可诱发或加重病情。阴囊湿疹由于发病部位特殊,不易清洁,加上阴毛的刺激,皮肤极易破损。

1.当归贝母苦参丸

【药物组成】当归、苦参各15g,浙贝母、生地黄、黄柏、徐长卿各12g,苍术9g,薏苡仁30g,甘草6g。

【随症加减】渗出明显者加龙胆草、车前子各12g,茯苓15g;瘙痒甚者加地肤子、白鲜皮各15g,龙骨、牡蛎各30g;皮疹潮红热盛者加牡丹皮、赤芍、栀子各12g,紫草15g;皮损粗糙肥厚者加丹参、鸡血藤各30g,桃仁、红花各9g。

【治疗方法】每日1剂,水煎,分2次服。同时配合外用治疗,有渗出者用自制湿疹溶液湿敷,皮损粗糙肥厚者外用激素软膏。10天为1个疗程,2个疗程后统计疗效。治疗期间避免局部烫洗、搔抓,忌食辛辣刺激性食物。

【功 效】清热除湿,疏通脉络,滋养阴血。

【临床运用】临床治疗60例,痊愈55例,占90.17\%;好转5例,占9.83\%。

【经验心得】当归贝母苦参丸出自《金匮要略》,其原方主治血虚夹有湿热,病在下焦的妊娠小便不通之症。方中当归养血润燥、活血通络,贝母排浊祛湿、清热解毒,苦参清热除湿、通络止痒,全方具有养血润燥、清热除湿、通络散结之功。阴囊湿疹属中医“肾囊风”,中医认为多由于禀赋不耐,风、湿、热邪阻于会阴部肌肤所致。因湿性黏滞,缠绵难愈,其病迁延日久,渗水过多,又致伤阴耗血,血燥生风,亦可因长期服用苦寒燥湿或淡渗利湿之品,造成阴血亏虚。治疗如单利其湿则更伤其阴,单滋其阴血则有碍于湿,故具有养血润燥清热利湿之当归贝母苦参丸加味正合其病机,方中当归无滋腻碍湿之嫌,且能减少皮疹的复发;贝母功专治肺,肺气得降,则小便畅利,水湿浊邪自除;生地黄滋阴清热,薏苡仁健脾除湿,二妙散清热利湿,徐长卿祛风止痒,甘草调和诸药,诸药合用,可清热除湿,疏通脉络,滋养阴血,从而使皮疹消退。

【方剂出处】程晓春,等.当归贝母苦参丸加味为主治疗阴囊湿疹60例.实用中医药杂志,2004,20(4):181

2.当归饮子加味

【药物组成】当归、白芍、地肤子各15g,徐长卿、丹参各20g,生地黄、防风、白蒺藜、荆芥、何首乌、川芎、生黄芪、生甘草各10g。

【治疗方法】每日1剂,水煎服。配合中药外用:苦参30g,白鲜皮20g,蛇床子、川黄柏、明矾各15g,水煎汁150ml外洗和湿敷患处,早、晚各1次。

【功 效】补气养血,祛风止痒。

【临床运用】临床治疗48例,痊愈29例,显效10例,有效6例,无效3例,总有效率为93.8\%。

【经验心得】阴囊湿疹病因复杂,迁延难愈。中医学认为,本病多因风、湿、热、燥诸邪客于肌肤,或因饮食失节,过食辛腥助湿动风之品,脾失健运,湿热内生,充于腠理,蕴结于皮肤而发病,同时由于病久伤阴耗血,血虚风燥,皮肤失所养而致迁延难愈,反复发作。当归饮子加味方中,当归、川芎、丹参、何首乌、白芍、生地黄活血通经,养血润燥,意在欲去其风,先活其血,盖“治风先治血,血行风自灭”;防风、荆芥、白蒺藜、徐长卿、地肤子祛风止痒;黄芪益气固正,且使气行血行;甘草和中润燥。合而用之,共奏补气养血、润燥祛风止痒之功。从现代医学的观点来看,徐长卿、地肤子、防风、荆芥、甘草等祛风止痒、和中润燥之品具有抗过敏止痒作用,能明显缓解临床症状,而当归、川芎、丹参、生地黄、何首乌等活血化瘀药物可改善病变部位的血液循环,可能是由于改变了皮肤组织细胞的超微结构,使病变皮肤转为正常。

【方剂出处】汪卫平.当归饮子加味治疗慢性阴囊湿疹48例.浙江中医杂志,2002,(2):296

3.二蛇木鳖液

【药物组成】苦参30g,花椒10g,蛇床子30g,蛇蜕6g,木鳖子4个(去壳切片),苍术15g,五倍子10g,黄柏15g,百部15g,鬼针草20g。

【治疗方法】上药加水煎至2 000ml,将药汁倒入清洁盆内,先熏洗后坐浴20分钟左右,7天为1个疗程。

【功 效】清热燥湿,收敛止痒。

【临床运用】临床治疗58例患者。病程最短4日,最长26日。58例通过治疗后,显效39例,阴囊瘙痒、皮疹、疼痛完全消失;有效18例,部分皮疹消失,无淫脂水、瘙痒、疼痛;无效1例,皮疹流脂水,疼痛未减轻。总有效率98.45\%。

【经验心得】阴囊湿疹多数是由肝经湿热下注,风邪外袭而成,初起阴囊干燥作痒,喜浴,甚则起疙瘩,形如粟米,色红,搔破浸淫脂水,或热痛如火燎,经久不愈。方中苦参、花椒、蛇床子、苍术、黄柏、鬼针草有清热燥湿止痒作用;木鳖、百部杀虫;蛇蜕既能祛风,又有润肤作用;五倍子有收敛止痒功能,如果加服龙胆泻肝丸内外同治效果更好。本方法作用迅速,易学易用,容易推广,使用安全,毒副作用少,患者乐于接受,值得推广应用。

【方剂出处】戴明喜.二蛇木鳖液外洗治疗阴囊湿疹58例.中医外治杂志,2002,11(1):44

4.龙胆泻肝汤

【药物组成】龙胆草15g,栀子12g,柴胡10g,黄柏12g,甘草6g。

【治疗方法】每日1剂,水煎,分2次服。10天为1个疗程。龙胆泻肝汤药渣第3次煎后,取汤温热外洗,每日2次;黄连、滑石1∶3共碾为粉外敷。

【功 效】清热燥湿,祛风止痒。

【临床运用】临床治疗32例,治愈19例,有效11例,无效2例。有效率94\%。

【经验心得】阴囊湿疹又称“绣球风”“肾囊风”,类属中医的浸淫疮。此病病因病机相当复杂,除风热外袭、血虚风燥、阳虚风乘等证型外,湿热下注型阴囊湿疹临床上多见。其机制是饮食失节、恣食辛辣厚味、多食鱼腥海味、滥服苦寒药物,而伤及脾胃,运化失常,湿邪内停,蕴久化热,循经下注前阴,蕴郁肌肤,终致阴囊潮红作痒、糜烂、滋水。本方以龙胆泻肝汤为基本方加减。方中重用龙胆草、栀子清热燥湿,柴胡清肝解郁(用以引经),佐以车前子、泽泻、木通导湿下行,使湿热有所出。配以地肤子、浮萍草、士茯苓等诸药合用,共起清热燥湿、祛风止痒之效。

【方剂出处】高勇.湿热下注型阴囊湿疹的论治32例.宁夏医学杂志,2002,24(5):281

5.止痒洗剂

【药物组成】土茯苓30g,仙鹤草20g,明矾10g,花椒10g,苦参20g,蛇床子30g,百部10g,当归10g。

【治疗方法】取上药1剂,加水3 000ml煮沸10~20分钟,取汁先熏后洗,早、晚各1次,每次30~60分钟。同时配合服用当归苦参丸(北京同仁堂制药厂生产),每日2次,每次1丸,饭后服。10天为1个疗程,治疗2个疗程后判定疗效。在服药的同时嘱患者应忌食辛辣鱼腥,勿过劳累。

【功 效】清热燥湿,益气活血,祛风止痒。

【临床运用】临床治疗30例,本组病例经治疗1~2个疗程全部获效,其中痊愈27例,显效3例。

【经验心得】中医学认为,本病多由先天禀赋不足,风湿热邪阻于肌肤所致。止痒合剂的主要作用是清热燥湿、益气活血、祛风止痒。方中用大剂量苦参、蛇床子清热燥湿;土茯苓性凉味甘,与花椒相伍,有清热祛风止痒之效;明矾酸寒,寒能清热,酸能收敛,仙鹤草味苦、涩,性平,与百部相合,有收湿杀虫止痒之效;当归辛、温,活血养血与仙鹤草相伍益气活血,取其“治风先治血,血行风自灭”之意。以上诸药寒热并用,攻补兼施,共奏清热燥湿、益气活血、祛风止痒之效。

【方剂出处】杜付祥,等.止痒洗剂合当归苦参丸治疗阴囊湿疹30例.中国民间疗法,2004,12(6):29

6.止痒洗药

【药物组成】苦参60g,蛇床子60g,鹤虱30g,大枫子30g,地肤子30g,白鲜皮30g,黄柏30g,大黄30g,徐长卿30g,仙鹤草30g,生杏仁13g,百部13g,硫黄10g,露蜂房15g。

【治疗方法】治疗时取上药1剂,加水煎沸10~20分钟,先熏后洗,早、晚各1次,每次熏洗30~60分钟;并用丹参酊外涂患处,每日5~6次。治疗10天为1个疗程,2个疗程后判定疗效。嘱患者忌辛辣刺激性食物,避免过劳。

【功 效】清热燥湿,杀虫止痒。

【临床运用】临床治疗30例,经治疗1~2个疗程全部获效,其中痊愈28例,显效2例。

【经验心得】中医学认为,本病多由风、湿、热阻于肌肤所致。止痒洗药具有较好的清热燥湿、杀虫止痒作用,可抗炎、止痒、抗过敏,能迅速有效地缓解症状。丹参酊由丹参、黄芩等浸泡70\%酒精制成,具有很好的预防感染、活血化瘀、消肿止痛、促进皮损修复的作用,二药联用,可减轻病变部位的炎症和瘙痒,起到协同作用,提高疗效,缩短病程,减少复发。

【方剂出处】李兆军.止痒洗药并丹参酊治疗阴囊湿疹30例.中国民间疗法,2003,11(6):30

7.紫 苏 方

【药物组成】鲜紫苏叶或干紫苏。鲜紫苏每次250g,干紫苏50g左右。

【治疗方法】加水500ml,煎沸后10煮分钟(干紫苏煎12分钟左右),倒在干净的洗盆里,凉到40℃左右,用干净纱布浸湿后轻轻拍打患处,轻者每日1次,重者每日早、晚各1次。洗后局部皮肤擦干,保持清洁干燥,并卧床休息0.5~1小时,仰卧屈膝两腿分开。

【功 效】散热止痒,收敛除湿。

【临床运用】临床治疗19例患者,病程最短3个月,最长20年。均曾应用激素软膏外涂后症状暂时缓解,但不能根治。全部患者经3~5日治疗后症状消失,巩固治疗1周后痊愈,未再复发,无任何其他不良反应。

【经验心得】阴囊湿疹是多种内外因素引起的一种具有明显渗出倾向的皮肤炎性反应。由于奇痒难忍,不少患者用手抓,或用热水烫,引起红肿糜烂,皮肤湿润肥厚及苔藓样变,色素沉着或色素减退。饮酒吃辣或睡眠不足,可诱发或加重病情。阴囊湿疹由于发病部位特殊,不易清洁,加上阴毛的刺激,皮肤极易破损。而用激素软膏外涂只能暂时缓解症状,且可致局部皮肤色素沉着加重,甚至轻微角化,加重皮损。紫苏叶能发表散寒,行气宽中,清热解毒,外洗可散热止痒,收敛除湿,故治疗阴囊湿疹可收到较好效果。

【方剂出处】陈耀珍.紫苏治疗阴囊湿疹及护理.实用中医药杂志,2002,18(4):23


\subsection{第13章 男性更年期综合征}

男性更年期综合征是指男子在更年期内所出现的非器质性疾病引起的,以自主神经功能紊乱、精神、心理障碍和性功能减退为主体症状的一组症候群。主要是由于男子到中老年,机体代谢和内分泌功能减退,睾酮产生水平下降导致本病。

1.补肾调和汤

【药物组成】熟地黄20g,山药、枸杞子各15g,山茱萸、泽泻、茯神、淫羊藿、龟甲胶(烊化)、鹿角胶(烊化)各10g,牡丹皮、人参、酸枣仁各9g。

【随症加减】伴畏寒、阳痿、早泄明显者加锁阳、仙茅、芡实;阴虚火旺及盗汗者去鹿角胶,加地骨皮、黄精、龟甲胶;夜尿频多者可加缩泉丸;神志异常、喜哭笑者合以甘麦大枣汤;肝阳上亢血压高者去鹿角胶,加钩藤、桑寄生、川牛膝等。

【治疗方法】水煎服,日1剂。

【功 效】补肾填精,调和阴阳。

【临床运用】临床治疗39例。显效19例,有效15例,无效5例,总有效率87.2\%。

【经验心得】中医学认为,男性更年期正是“七八肝气衰,天癸竭,精少,肾藏衰,形体皆极,八八则齿发去”的阶段,肾气逐渐衰少,经血日趋不足,导致肾精亏虚和阴阳失调。肾精亏损,则导致天癸早竭,髓失化源,骨失所养,脑海空虚,故而出现发脱齿摇、眩晕耳鸣、健忘恍惚、精神呆钝、足痿无力、动作迟缓、性功能减退等早衰症状;肾阳虚则生内寒,故可出现精神委靡,腰膝酸痛,畏寒肢冷,性欲减退,阳痿或早泄,阴冷囊缩,或轻度水肿,肾阴亏虚,则腰酸膝软,失眠多梦,早泄遗精,阴虚生内热,故有潮热盗汗,五心烦热,咽干颧红等症状;肾阴虚日久不能上济心火,以致心肾不交,故可见失眠健忘,心悸怔忡,悲喜无常等症状。治法当补肾填精、调和阴阳,用自拟补肾调和汤加减治疗。方中龟甲胶、鹿角胶填补肾精;熟地黄、枸杞子、牡丹皮、泽泻补肾滋阴降火;山茱萸、淫羊藿、人参、山药温补肾阳;酸枣仁、茯神养心安神。诸药合用,共奏补肾填精、阴阳调和之功。临床观察发现,该方药能减轻患者紧张及焦虑情绪,减轻心慌失眠症状,增强性欲和体力,明显改善男性更年期综合征的症状。同时与“雄激素+对症治疗”方法比较,具有不良反应少、复发率低、治愈率高的优点。

【方剂出处】马钢.补肾调和汤加减治疗男性更年期综合征39例.吉林中医药,2008,28(9):667

2.二 仙 汤

【药物组成】仙茅、淫羊藿各20g,当归、巴戟天各10g,知母、黄柏各6g。

【随症加减】头晕头痛者加天麻、菊花;失眠抑郁者加首乌藤、酸枣仁、柴胡、郁金;腰痛明显者加杜仲、狗脊;阳痿早泄者加蜈蚣、锁阳、金樱子、芡实。

【治疗方法】水煎,每日1剂,早、晚分服。1个月为1个疗程,连服2个疗程。

【功 效】调理冲任,平衡阴阳。

【临床运用】临床治疗52例。治愈24例,显效12例,有效14例,无效2例,总有效率92.3\%。

【经验心得】中医无男性更年期综合征这一病名,多将其归属于“心悸”“不寐”“郁证”“眩晕”“脏躁”等的范畴。本病的病机基础是肾气衰少,天癸将竭,精血不足,肾阴阳失调所致。如《素问·上古天真论》曰:“丈夫……五八,肾气衰,发堕齿槁;六八,阳气衰竭于上,面焦,发鬓斑白;七八,肝气衰,筋不能动;八八,天癸竭,精少,肾脏衰,形体皆极,则齿发去。”因此,对男性更年期综合征的临床辨证,应抓住肾虚这一特点,以治肾虚为主,根据治病求本的原则,围绕肾虚来调整肾与其他脏腑的关系。

二仙汤是已故名医张伯讷创制的方剂,具有辛温与苦寒共用、壮阳与滋阴并举、温补与寒泻同施之特征,尤其以温肾阳、补肾精、泻相火、滋肾阴、调理冲任、平衡阴阳见长。方肇勤等研究证明,二仙汤及其两个拆方均能分别提高老龄雄性大鼠血浆睾酮和老龄雌性大鼠血浆雌二醇的含量,降低两者血浆黄体生成激素的含量,且观察到泻火组在这两个方面均作用突出。因此,二仙汤加味治疗本病,切中病机,疗效显著。

【方剂出处】刘强.二仙汤结合心理疏导治疗男性更年期综合征例疗效观察52.深圳中西医结合杂志,2008,18(1):55

3.男 更 汤

【药物组成】百合30g,炒酸枣仁60g,柴胡12g,白芍10g,桑寄生10g,郁金10g,当归15g,莲子心10g,山药10g,鹿衔草10g,巴戟天10g,浮小麦10g,大枣5枚,甘草6g。

【治疗方法】每日1剂,水煎,分2次服。10日为1个疗程。

【功 效】疏肝解郁,补肾益精,养心安神。

【临床运用】临床治疗35例,治愈9例,显效20例,好转4例,无效2例,总有效率94.7\%。

【经验心得】方中白芍养血柔肝,合柴胡疏肝理气解郁,更可防柴胡劫肝阴之弊;郁金既入气分又入血分,又能行气解郁兼凉血清心,合当归养血活血而解郁之功更强,柴胡、当归、白芍功能疏肝解郁以除郁闷不舒、焦虑抑郁;百合、合欢皮等共奏滋阴安神、解郁之效,治疗潮热、盗汗;甘草、浮小麦、大枣组成治疗脏躁的甘麦大枣汤,以养心安神、敛汗,加炒酸枣仁治疗心悸失眠多梦;桑寄生、山药、鹿衔草、巴戟天补肾壮阳,治疗性欲减退、腰膝酸软、白发或脱发、胆怯、嫉妒、猜疑。

【方剂出处】魏绪华,等.男更汤治疗男性更年期综合征35例临床观察.四川中医,2006,24(12):50

4.六味地黄汤

【药物组成】熟地黄40g,山茱萸、山药各20g,牡丹皮、泽泻、茯苓各15g。

【随症加减】阳虚者加附子、桂枝各15g;阴虚火旺明显者加墨旱莲、女贞子各15g,知母、黄柏各10g;气虚者加党参、黄芪、白术各10g;血虚者加当归、黄精、何首乌各15g;头晕头痛者加菊花、枸杞子、牛膝各10g;心悸失眠者加麦冬、五味子、酸枣仁、首乌藤各10g;记忆力下降者加益智仁、五味子、远志各10g;性欲减退、阳痿者加淫羊藿、巴戟天、菟丝子各10g;猜疑、忧虑者加郁金、石菖蒲各10g;易怒者加柴胡、白芍、淡竹叶各10g。

【治疗方法】每日1剂,水煎服,30天为1个疗程。

【功 效】滋阴补肾。

【临床运用】临床治疗124例,治愈35例,好转67例,无效22例,总有效率为82.8\%。

【经验心得】中医对本病的治疗从整体平衡出发,着重于治本。中医学认为,人到更年期,肾气日衰,天癸将竭,肝阴亏损,脾失健运,心肾不交,脑失所养,从而出现阴阳平衡失调,脏腑功能紊乱的一系列症状。根据“阴平阳秘,精神乃治”的原则,立滋养肝肾、平衡阴阳的方法,选用六味地黄汤加味对男性更年期综合征有较好的治疗作用。

【方剂出处】张静.六味地黄汤治疗男性更年期综合征124例疗效观察.信阳农业高等专科学校学报,1999,9(3):95

5.仙 茅 汤

【药物组成】仙茅10g,淫羊藿12g,当归10g,巴戟天10g,黄柏10g,知母10g,黄精10g,熟地黄15g,炙甘草10g。

【随症加减】脾肾阳虚者加白术、山药、茯苓;肾虚肝郁者加柴胡、白芍、香附、合欢花;心肾不交者加牡丹皮、浮小麦、大枣;阴虚阳亢者加龙骨、牡蛎、生龟甲。

【治疗方法】每日1剂,水煎服,7天为1个疗程。

【功 效】滋阴壮阳。

【临床运用】临床治疗48例,显效23例,好转18例,无效7例,总有效率为85.4\%。

【经验心得】仙茅汤的配伍特点是壮阳药与滋阴药同用,以针对阴阳俱虚于下,而又有虚火上炎的证候。仙茅辛、温,有小毒,入肾经,温肾壮阳;淫羊藿辛、温,入肝、肾经,温肾助阳;巴戟天补肾阳,活筋骨;知母、黄柏泻相火,坚肾阴;熟地黄、黄精滋补肾体,以阴中求阳;当归养血和血,共成调补肾阴肾阳之剂。

【方剂出处】杨晓勇.仙茅汤加味治疗男性更年期综合征48例.湖南中医杂志,2002,18(5):32

6.滋水清肝饮

【药物组成】 熟地黄25g,山药15g,山茱萸15g,牡丹皮10g,茯苓15g,泽泻10g,柴胡12g,白芍20g,栀子10g,酸枣仁10g,当归12g。

【随症加减】若五心烦热,加黄柏、知母、竹茹;头部胀麻疼痛者,加川芎、蔓荆子、牛膝;情绪紧张易激动,加龙骨、牡蛎、磁石、石菖蒲;夜难入眠,加首乌藤、茯神;身体潮热者,加银柴胡、地骨皮;性欲低下、阳痿者,加锁阳、巴戟天、鹿角片;夜尿频多,加益智仁、桑螵蛸;神疲乏力明显者,加党参、黄芪。

【治疗方法】每日1剂,水煎,分2次服。30剂为1个疗程。

【功 效】滋补肝肾,调理阴阳。

【临床运用】临床治疗56例,痊愈32例,有效18例,无效6例,总有效率为89.28\%。

【经验心得】本症患者大多肾阴不足,水不涵木,肝经郁热,热扰心神。治疗当以补肾调肝为主。滋水清肝饮主治燥火生风,发热胁痛,耳聋口干,手足头面似觉肿起。以滋水清肝饮治疗男性更年期综合征,切中病机,而临床再根据兼症适当加减更可提高疗效。

【方剂出处】黄凌.滋水清肝饮加减治疗男性更年期综合征56例.福建中医药,2006,37(1):41

7.丹栀逍遥散合二至丸

【药物组成】当归、白芍、柴胡、白术、茯苓、牡丹皮、栀子、生姜、薄荷、女贞子、墨旱莲各适量。

【随症加减】肾阴不足加山茱萸、枸杞子、生地黄;肾阴不足,阴损及阳加淫羊藿、巴戟天、仙茅;气虚加生黄芪、党参;血虚加熟地黄、何首乌;失眠加酸枣仁、柏子仁;肝阳上亢,血压升高加钩藤、天麻。

【治疗方法】每日1剂,水煎,分2次服。病情好转后改为隔日1剂。

【功 效】疏肝解郁,滋阴补肾。

【临床运用】临床治疗58例,痊愈38例,有效18例,无效2例,总有效率为97\%。

【经验心得】一般认为,男性更年期综合征是由于内分泌腺的腺体萎缩、功能减退所致,其发病年龄多在55—65岁,这与《素问·上古天真论》男子七八“肝气衰,天癸竭,精少,肾藏衰”的论述有暗合之处。男性到此段年龄,由于肾气渐衰,精血不足,天癸将竭,可致阴阳平衡失调。根据观察,本组病例大多因肾阴不足,水不涵木,肝经郁热,热扰心神所致。丹栀逍遥散合二至丸具有疏肝解郁、滋阴补肾之功,因而十分切合病证,而临床再根据兼夹证适当加减更可提高疗效。

【方剂出处】左松青.丹栀逍遥散合二至丸治疗男性“更年期综合征”58例.北京中医,2001,(5):19

8.耳穴贴压法

【穴位选择】耳穴贴压:取穴神门、交感、心、肾、肝、脾、内生殖器、皮质下、内分泌。

【治疗方法】每次取一侧耳穴,以王不留行贴压,双耳交替,隔日换贴1次。每日按压所贴耳穴3~6次,每次5分钟,以耳郭微有胀、麻、痛或灼热感为度。15次为1个疗程,连续治疗2个疗程。向患者耐心分析解释病情,使其消除紧张、焦虑等不良情绪;指导患者良好的生活习惯,避免过度劳累。

【功 效】滋补肾精,宁心安神,健脾运湿。

【临床运用】临床治疗93例,经治疗2个疗程。治愈48例,显效29例,有效16例,总有效率100\%。

【经验心得】耳和全身五脏六腑都有密切联系,《灵枢·口问》篇云:“耳者,宗脉之所聚也。”有研究表明,耳穴贴压的作用机制在于对神经内分泌系统的整体调节,可调整自主神经系统,促使紊乱的自主神经功能恢复正常,还可调整和改善中枢-下丘脑-垂体-性腺轴的功能状态,促使机体内分泌环境达到相对平衡的状态,从而纠正内分泌紊乱。本组选取肝、肾、心、神门、脾穴用以疏肝解郁,调节情志,滋补肾精,宁心安神,健脾运湿。取肾穴配内生殖器、内分泌以调节性激素水平,交感穴调节自主神经功能,皮质下穴补髓益脑,调整大脑皮质的兴奋与抑制。诸穴配合,标本兼治,调和气血,协调脏腑经络,平衡阴阳,使阴阳平复,疾患得愈。

【方剂出处】庄田畋.耳穴贴压结合心理疗法治疗男性更年期综合征93例.陕西中医,2006,27(7):859


\subsection{第14章 常用中药}

1.淫 羊 藿

为小蘖科植物淫羊藿及同属其他植物的全草。

【性味归经】辛、甘,温。归肝、肾经。

【功效主治】温肾壮阳。本品有温肾壮阳、益精起痿之效。用于肾阳虚的阳痿,不孕及尿频等证。可单味浸酒服,如《食医心镜》淫羊藿酒;亦可配伍熟地黄、枸杞子、巴戟天等同用,如赞育丸。

【用法用量】煎服,3~15g。亦可浸酒、熬膏或入丸、散。

【注意事项】阴虚火旺及有热者不宜用。

2.枸 杞 子

为茄科植物宁夏枸杞的成熟果实。

【性味归经】甘,平。归肝、肾经。

【功效主治】滋补肝肾,益精明目。用于肝肾不足,腰酸遗精,及头晕目眩,视力减退,内障目昏,消渴等。有补肝肾,益精血,明目,止渴之效。治肾虚遗精,常配熟地黄、沙苑子、菟丝子等;治肝肾阴虚,视物模糊,常配菊花、地黄等,如杞菊地黄丸;治消渴,可配生地黄、麦冬、天花粉等同用。

【用法用量】煎服,6~12g。

3.熟 地 黄

为玄参科草本植物地黄的块根,经加黄酒拌蒸至内外色黑、油润,或直接蒸至黑润而成。

【性味归经】甘,微温。归肝、肾经。

【功效主治】①补血。本品为补血要药。用于血虚萎黄,眩晕,心悸失眠,月经不调,崩漏等症。常与当归、白芍同用,并随证配伍相应的药物。②滋阴。本品为滋阴主药。用于肾阴不足的潮热骨蒸、盗汗、遗精、消渴等。常与山茱萸、山药等同用,如六味地黄丸。③益精填髓。用于肝肾精血亏虚的腰膝酸软,眩晕耳鸣,须发早白等。常与制何首乌、枸杞子、菟丝子等补精血、乌须发药同用。

【用法用量】煎服,10~30g。

【注意事项】本品性质滋腻,有碍消化,凡气滞痰多、脘腹胀痛、食少便溏者忌服。重用久服宜于陈皮、砂仁同用。

4.牛 膝

为苋科植物牛膝(怀牛膝)和川牛膝(甜牛膝)的根。以栽培品为主,也有野生者。怀牛膝主产河南;川牛膝主产四川、云南、贵州等地。

【性味归经】苦、甘、酸,平。归肝、肾经。

【功效主治】①补肝肾,强筋骨。本品制用能补肝肾,强筋骨,兼祛风湿,尤以怀牛膝为佳。用治肾虚腰痛及久痹腰膝酸痛乏力等。可配伍杜仲、续断、补骨脂等同用。②利水通淋。本品性善下行,能利水通淋。用于淋证,水肿,小便不利等。

【用法用量】煎服,6~15g。活血通经、利水通淋、引火下行宜生用;补肝肾、强筋骨宜酒炙用。

【注意事项】孕妇及月经过多者忌用。

【补充说明】牛膝有怀牛膝和川牛膝之分,两者功效基本相同,但怀牛膝偏于补肝肾、强筋骨,川牛膝偏于活血祛瘀。此外,另有一种土牛膝为怀牛膝野生品种及柳叶牛膝的根。性味、功效与牛膝相似,而长于清热利咽,活血通淋。主治咽喉肿痛,白喉,口舌生疮,痈肿丹毒及淋病等。用量10~30g。鲜品可至60g,煎服或捣汁饮。

5.黄 柏

为芸香科落叶乔木黄檗(关黄柏)和黄皮树(川黄柏)的干燥树皮。

【性味归经】苦,寒。归肾、膀胱、大肠经。

【功效主治】①清热燥湿。本品苦寒沉降,清热燥湿,长于清泻下焦湿热。用于湿热带下,热淋脚气,泻痢黄疸。②退热除蒸。本品长于清相火,退虚热。用于阴虚发热,盗汗遗精。

【用法用量】煎服,3~12g。或入丸、散。外用适量。

【注意事项】清热燥湿解毒多生用;泻火除蒸退热多盐水炙用;止血多炒炭用。本品苦寒,容易损伤胃气,故脾胃虚寒者 忌用。

6.菟 丝 子

为旋花科植物菟丝子或大菟丝子的成熟种子。

【性味归经】辛、甘,平。归肝、肾、脾经。

【功效主治】①补肾益精。本品既能补肾阳肾阴,又有固精、缩尿、止带之效。用于肾虚腰痛,阳痿遗精,尿频,带下等证。②养肝明目。本品能益肾养肝,使精血上注而明目而达明目之效。用于肝肾不足,目失所养而致目昏目暗,视力减退之证。

【用法用量】煎服,10~20g。外用适量。

【注意事项】阴虚火旺、大便燥结、小便短赤者不宜用。

7.车 前 子

为车前科植物车前或平车前的干燥成熟种子。

【性味归经】甘、微寒。归肾、肝、肺、小肠经。

【功效主治】①利尿通淋。本品甘而滑利,寒凉清热,有利尿通淋之功。对温热下注于膀胱而致小便淋漓涩痛者尤为适宜。用治小便淋涩。常与木通、滑石、萹蓄等清热利湿药同用,如八正散。②渗湿止泻。本品能利水湿,分清浊而止泻,即利小便以实大便。用于暑湿泄泻。

【用法用量】煎服,9~15g。宜布包。

【注意事项】肾虚精滑者慎用。

8.山 茱 萸

为山茱萸科植物山茱萸除去果核的成熟果肉。

【性味归经】酸、涩,微温。归肝、肾经。

【功效主治】①补益肝肾。山茱萸酸微温质润,其性温而不燥,补而不峻,既能补肾益精,又能温肾助阳;既能补阴,又能补阳,为补益肝肾之要药。用于肝肾亏虚,头晕目眩,腰膝酸软,阳痿等证。本品又能补肝肾,固冲任。亦可用于妇女因肝肾亏损,冲任不固所致的崩漏下血及月经过多之证。②收敛固涩。本品既能补肾益精,又能固精止遗。用于遗精,遗尿。本品能敛汗固脱。用于大汗不止,体虚欲脱证。治大汗虚脱者,常与人参、附子、龙骨等同用。此外,本品亦治消渴证,多与生地黄、天花粉等同用。

【用法用量】煎服,5~10g,急救固脱20~30g。

【注意事项】素有湿热、小便淋涩者,不宜应用。

9.山 药

为薯蓣科植物薯蓣的块茎。主产于河南省,湖南、江南等地亦产,习惯认为河南怀庆地区者最佳,称怀山药。

【性味归经】甘,平。归脾、肺、肾经。

【功效主治】①补脾养胃。本品性味甘平,能补脾益气,滋养脾阴,多用于脾胃虚弱或气阴两虚证。亦可用于脾虚不运,湿浊下注之妇人带下。唯其亦食亦药,力量不足,常需与人参、白术等补气健脾药配伍使用。②补肾涩精。用治肾虚不固的遗精、尿频等,常配熟地黄、山茱萸、菟丝子、金樱子等同用。用治肾虚不固,带下清稀,绵绵不止,可与熟地黄、山茱萸、五味子等同用。此外,本品有益气养阴、生津止渴之效。用治阴虚内热,口渴多饮,小便频数的消渴证。常配黄芪、生地黄、天花粉等同用。

【用法用量】煎服,15~30g,大量60~250g。研末吞服,每次6~10g。

【注意事项】补阴生津宜生用;健脾止泻宜炒用。

10.生 地 黄

为玄参科植物怀庆地黄或地黄的新鲜或干燥块根。

【性味归经】甘、苦,寒。归心、肝、肾经。

【功效主治】①清热凉血,养阴生津。本品甘寒质润,苦寒清热,入营分、血分,为清热凉血、养阴生津之要药。用于热入营血,口干舌绛。多配玄参、连翘、丹参等药同用。如《温病条辨》之清营汤。②凉血止血。本品清热泻火,凉血止血。用于血热妄行,斑疹吐衄。常与大黄、地榆、益母草等同用。③滋阴降火。本品甘苦性寒,入肾经而滋阴降火,养阴津而泄伏热。用治阴虚内热,骨蒸潮热等证可配知母、地骨皮等同用。

【用法用量】煎服,10~15g。鲜品用量加倍,或以鲜品捣汁入药。

【注意事项】鲜生地味甘、苦,性大寒,作用与干地黄相似,滋阴之力稍逊,但清热生津,凉血止血之力较强。本品性寒而滞,脾虚湿滞、腹满便溏者不宜使用。

11.桃 仁

为蔷薇科植物桃或山桃的成熟种子。

【性味归经】苦、甘,平。有小毒。归心、肝、大肠经。

【功效主治】①活血祛瘀。本品味苦而入心肝血分,善泄血滞,祛瘀力较强,又称破血药。用治多种瘀血证,如经闭、痛经、产后瘀滞腹痛,癥积及跌打损伤等。治跌打损伤,瘀肿疼痛,常配当归、红花、大黄等,如复元活血汤。②祛瘀消痈。痈之成者,热毒壅聚、气血凝滞所致。桃仁善泄血分之壅滞,本品常配清热药同用,以清热解毒活血消痈。用治肺痈、肠痈。

【用法用量】煎服,5~10g,宜捣碎入煎。

【注意事项】孕妇忌服;便溏者慎用。有毒,不可过量,过量可出现头痛、目眩、心悸,甚至因呼吸衰竭而死亡。

12.王不留行

为石竹科植物麦蓝菜的成熟种子。

【性味归经】苦,平。归肝、胃经。

【功效主治】利尿通淋。本品有利尿通淋作用。用治热淋、血淋、石淋等证。常配石韦、瞿麦等相须而用。近年治前列腺炎,亦常用本品,配红花、败酱草等同用。

【用法用量】煎服,5~10g。

【注意事项】孕妇慎用。

13.牡 蛎

为牡蛎科动物长牡蛎、大连湾牡蛎或浙江牡蛎的贝壳。

【性味归经】咸、涩,微寒。归肝、胆、肾经。

【功效主治】①重镇安神。本品质重能镇,有安神之效,用治心神不安,心悸怔忡,失眠多梦等症,常与龙骨相须为用,亦可配伍朱砂、琥珀、酸枣仁等同用。②平肝潜阳。本品咸寒质重,有类似石决明之平肝潜阳作用。用治阴虚阳亢之头晕目眩、烦躁不安、耳鸣者,常与龙骨、龟甲、白芍等同用。③收敛固涩。本品味涩,煅用有与煅龙骨相似的收敛固涩作用。随证配伍可用治滑脱诸证。如用治自汗、盗汗可与麻黄根、浮小麦等配伍;用治遗精、滑精可与沙苑子、龙骨、芡实等配伍;用治崩漏、带下又常与海螵蛸、山茱萸、山药、龙骨等同用。此外,煅牡蛎有收敛制酸作用,可治胃痛泛酸,以之与乌贼骨、浙贝母共为细末,内服取效。

【用法用量】煎服,9~30g;宜打碎先煎。除收敛固涩煅用外,余皆生用。

14.穿 山 甲

为鲮鲤科动物鲮鲤的鳞甲。

【性味归经】咸,微寒,归肝、胃经。

【功效主治】消肿排脓。本品能活血消痈,消肿排脓,可使未成脓者消散,已成脓者速溃。用于痈肿疮毒,瘰疬等。此外,近来以本品治外伤出血、手术切口渗血,及白细胞减少症,有止血和升白细胞作用。

【用法用量】煎服,3~10g;研末服,1~1.5g。

【注意事项】孕妇及痈肿已溃者忌用。

15.巴 戟 天

为茜草科植物巴戟天的干燥根。

【性味归经】辛、甘,微温。归肾、肝经。

【功效主治】①补肾助阳。本品甘润不燥,能温肾壮阳。用于肾阳虚弱的阳痿,不孕,月经不调,少腹冷痛等。②强筋骨,祛风湿。本品既可补阳益精而强筋骨,又兼辛温能除风湿。用于肝肾不足的筋骨痿软,腰膝疼痛,或风湿久痹,步履艰难。常配杜仲、萆薢等同用。

【用法用量】煎服,5~15g。

【注意事项】阴虚火旺及有热者不宜用。

16.肉 桂

为樟科植物肉桂的干燥树皮。

【性味归经】辛、甘,大热。归肾、脾、心、肝经。

【功效主治】①补火助阳。本品甘热助阳补火,为治命门火衰之要药。用治肾阳衰弱的阳痿,宫冷,虚喘,心悸等。②温经通脉。本品辛香,温通力强,温经通脉功胜,用治冲任虚寒,寒凝血滞的闭经、痛经等证,可与当归、川芎、小茴香等同用,如少腹逐瘀汤。③引火归原。本品大热入于肝肾,能使因下元虚衰所致上浮之虚阳回归故里,名曰引火归原,用治元阳亏虚,虚阳上浮之证,常与人参、山茱萸、五味子、牡蛎等同用。此外,久病体虚气血不足者,在补气益血方中,适当加入肉桂,能鼓舞气血生长。

【用法用量】煎服,1~4.5g,宜后下或焗服;研末冲服,每次1~2g。

【注意事项】阴虚火旺,里有实热,血热妄行出血及孕妇忌用。畏赤石脂。

17.五 味 子

为木兰科植物五味子或华中五味子的成熟果实。

【性味归经】酸,甘,温。归肺、心、肾经。

【功效主治】①收敛固涩。五味子酸能收敛,性温而润,上能敛肺气,下能滋肾阴,适用于肺虚久咳及肺肾两虚之喘咳。本品能敛肺止汗。治自汗、盗汗者,可与麻黄根、牡蛎等同用。本品能补肾涩精。用于遗精、滑精。本品又能涩肠止泻。用于久泻不止。②补肾宁心。本品既能补益心肾,又能宁心安神。用于心悸,失眠,多梦。此外,本品研末内服,对慢性肝炎转氨酶升高者,亦有治疗作用。

【用法用量】煎服,3~6g;研末服,每次1~3g。

【注意事项】凡表邪未解,内有实热,咳嗽初起,麻疹初期,均不宜用。

18.败 酱 草

为败酱科植物黄花败酱、白花败酱的干燥全草。

【性味归经】辛、苦,微寒。归脾、大肠、肝经。

【功效主治】①清热解毒,消痈排脓。本品苦辛寒凉,既清热解毒,又消痈排脓,且能活血止痛,适用于治疗肠痈、肺痈及痈肿疮毒等证,尤为肠痈腹痛的首选药物。常与金银花、蒲公英、桃仁、薏苡仁等药物配伍。②祛瘀止痛。本品辛散行滞,有破血行瘀,通经止痛之功。单用本品煎服即可用治产后瘀阻腹痛,或与当归、香附、五灵脂等同用。

【用法用量】煎服,6~15g。外用适量。

【注意事项】脾胃虚弱、食少泄泻者忌服。

19.肉 苁 蓉

为列当科植物肉苁蓉的干燥带鳞叶的肉质茎。

【性味归经】甘、咸,温。归肾、大肠经。

【功效主治】①补肾阳,益精血。用于肾阳不足,精血亏虚的阳痿,不孕,腰膝酸软,筋骨无力。②润肠通便。本品能润燥滑肠。用治肠燥便秘,对老人肾阳不足,精血亏虚者尤宜。常配当归、枳壳等同用,如《景岳全书》之济川煎。

【用法用量】煎服,10~15g;单用大剂量煎服,可用至30g。

【注意事项】本品能助阳、滑肠,故阴虚火旺及大便泄泻者不宜用。肠胃实热、大便秘结者亦不宜用。


\end{document}
