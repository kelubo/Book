\documentclass{book}
\usepackage{ctex}
\usepackage{graphicx}
\usepackage{hyperref}

\begin{document}

图书在版编目(CIP)数据

临床性医学/徐晓阳,马晓年主编.—北京:人民卫生出版社,2013

ISBN 978-7-117-17784-9

Ⅰ.①临… Ⅱ.①徐…②马… Ⅲ.①性医学 Ⅳ.①R167

中国版本图书馆CIP数据核字(2013)第224814号

人卫社官网 www.pmph.com 出版物查询,在线购书

人卫医学网 www.ipmph.com 医学考试辅导,医学数据库服务,医学教育资源,大众健康资讯

版权所有,侵权必究!

临床性医学

主 编:徐晓阳 马晓年

出版发行:人民卫生出版社有限公司 人民卫生电子音像出版社有限公司

地 址:北京市朝阳区潘家园南里19号

邮 编:100021

E - mail:ipmph@pmph.com

购书热线:4006-300-567

制作单位:人民卫生电子音像出版社有限公司

排 版:北京易成轩科技有限责任公司

制作时间:2016年8月

版 本 号:V1.0

容 量:31167 KB

格 式:mobi

标准书号:ISBN 978-7-117-17784-9

策划编辑:肖军

责任编辑:杨帆


\chapter{编委成员}

(按姓氏笔画排序)

万远廉 北京大学第一医院

马 乐 北京妇产医院

马晓年 清华大学玉泉医院

龙 云 北京大学深圳医院整形外科

卢存国 北京大学第三医院

关仁龙 重庆计生医院性医学专家会诊中心

刘 云 重庆医科大学附属第二医院

刘 勇 北京妇产医院

刘 捷 深圳市计划生育服务中心

刘华清 北京回龙观医院

刘明矾 江西师范大学心理学院

孙 伟 山东中医药大学第二附院

孙中义 第三军医大学大坪医院

庄礼大 北京大学深圳医院整形外科

朱 辉 北京大学深圳医院整形外科

张 渺 北京大学第一医院

张 滨 中山大学附属三院

张觇宇 重庆医科大学附属二院

李 铮 上海第二医科大学附属仁济医院

李微立 北京大学第一医院

杨大中 清华大学玉泉医院

杨华渝 北京安定医院

邸晓兰 北京回龙观医院

陈 乐 重庆医科大学公共卫生学院

陈焕然 中国医学科学院整形医院

周 旭 重庆计生医院性医学专家会诊中心

周 红 北京五洲妇儿医院

罗少波 中医研究院广安门医院

贺占举 北京大学第一医院

钟友彬 北京首钢医院古城门诊部

唐良萏 重庆医科大学附属第一医院

徐晓阳 重庆医科大学附属第二医院

贾金铭 中医研究院广安门医院

赵良运 云南省人民医院

郭公社 清华大学玉泉医院

陶 林 深圳市计划生育服务中心

陶 欣 中山大学附属三院

董晓静 重庆医科大学附属第二医院

董佳晨 中医研究院广安门医院

靳风硕 第三军医大学大坪医院泌尿科

彭 宇 北京回龙观医院

廖秦平 北京大学第一医院

编写组秘书:杨 莉 杨熔烨 杨 娇


\chapter{前 言}

2013年,也是国家进入“十二五”的第三年,《临床性医学》一书在众多专家的共同努力下终于问世了。它既反映了国际上性医学领域的最新进展,也总结了30多年来我国性医学发展的艰辛历程和积累的宝贵经验。近年来,性健康问题已日益受到广大民众的密切关注,成为民生工程里的一个重要组成部分。

性医学的研究内容较为广泛,涉及性生理学、性心理学、性药理学和性临床学等相关内容,其中主要包括以下几方面:

1)人类性器官的解剖学与生理学基础(如激素、神经递质等);

2)人类性心理发育与性心理健康,包括性别角色、性取向及相关问题;异常性心理表现及治疗;

3)人类性行为、性反应及随年龄的变化;

4)男女性功能障碍及预防与治疗;

5)疾病对人类性行为的影响;

6)生育与性(含避孕、节育或不孕不育与性);

7)药物与性(包括性药学);

8)性传播疾病;

9)性治疗的理论与实践;

10)传统性医学。

根据通常标准,我们可以把性问题做如下分类:

一、性焦虑是指人们对自己的生殖器、性功能和生殖能力等是否处于正常状态表现出的过分的关心和不安,尤其是青少年更为常见,是性心理和性行为方面的一种消极反应,在缺乏自信心的人中间更易产生。

二、男女性功能障碍(略):

三、性心理障碍

(1)性身份障碍(性别转换症);

(2)性偏好障碍(性欲倒错);

(3)性取向障碍(同性恋中自我适应不良的少数人);

(4)脑器质性疾病和精神疾病引起的继发性性行为障碍。

具有性焦虑的人求助的目标广泛(包括网络、报纸杂志、书籍、音像制品),而患性功能障碍的病人往往求助于泌尿科、妇科及新兴的男科、性医学科的临床医生。患性心理障碍的病人则在精神科求治。

回顾改革开放30多年的历史,我们不难发现性医学事业和性健康教育的发展仍然如履薄冰、举步维艰,尤其是成年人的性科学、正规的性教育几乎还是空白。所以从事性教育、性咨询、性治疗和性医学领域工作的相关科室的广大医务工作者、计划生育工作者、心理学工作者、教育工作者、司法战线工作者、婚姻家庭工作者、社会工作者,包括工青妇各行各业的工作者都可能涉及这一领域的问题,都应该适当地、不同程度地掌握一些这方面的科学信息。性健康教育应该授人以渔,而不是授人以鱼。一位优秀的从事性健康保健事业的工作者除了要充分掌握性医学知识,还必须了解更为广泛的相关知识。纵观过去、现在和未来,我们充满信心,一定一步一个脚印地走下去,不断努力,以最渊博的专业知识、最先进的科学技术、最能代表历史前进方向的文化素养造福广大人民。愿借此机会衷心感谢改革开放以来一直在这一领域披荆斩棘、不辞辛苦、热心奉献的各级专业人员,正是由于他们的高尚品德、严谨学风和渊博知识才能确保性医学事业的发展,让性医学专业的队伍不断壮大,本书的问世也得益于他们宝贵的临床经验。我们也无限怀念已经离开我们的性医学事业的开拓者如钟友彬教授和关人龙教授,他们的精彩论述仍然会给我们的读者带来无限欣慰和启迪。还要借此机会衷心感谢为本书付出辛勤劳动的多位专家学者。本书虽然付诸印梓,但存在的问题和不足仍难避免,期待和欢迎各位读者的批评与指正。

主编 徐晓阳 马晓年

2013年8月12日


\chapter{目 录}

编委成员

前 言

第一章 绪 论

第一节 人类性行为与性进化

第二节 人类性历史的演变

第三节 我国性文化

第四节 国外性文化

第五节 性医学发展史

第六节 性的道德伦理、社会文化和教育

第二章 性生物学基础

第一节 性发育生物学

第二节 男性性器官的结构与功能

第三节 女性性器官的结构与功能

第四节 性反应周期

第三章 性心理学基础

第一节 人类性心理的概念与基本成分

第二节 双重性态度

第三节 性心理发展的基本理论

第四章 生命周期各阶段的性

第一节 性分化的关键期理论

第二节 儿童的性

第三节 青春期

第四节 成年期

第五节 中年及老年

第五章 性咨询与性治疗

第一节 性治疗的基础理论和常用方法

第二节 性咨询及性治疗的病史采集和评估

第三节 性治疗的基本模式和实施

第四节 性障碍的分类原则

第六章 性欲异常

第一节 概 述

第二节 男性性欲亢进

第三节 男性性欲低下

第四节 男性性厌恶

第五节 女性性欲亢进

第六节 女性性欲低下

第七节 女性性厌恶

第七章 勃起功能障碍

第一节 阴茎勃起功能的生理

第二节 ED的流行病学

第三节 ED的病因和分类

第四节 ED的诊断

第五节 勃起功能障碍的治疗

第八章 射精障碍

第一节 早泄概论

第二节 早泄的病因及发病率

第三节 早泄的诊断与分类

第四节 早泄的临床应用

第五节 早泄的治疗与咨询

第六节 早泄诊断工具的研究

第七节 不射精症

第八节 逆行射精

第九章 女性性唤起障碍

第一节 女性性唤起的解剖和生理

第二节 女性性唤起的研究测定

第三节 性唤起障碍的定义与分类

第四节 性唤起障碍的病因分析

第五节 女性性唤起障碍的临床诊断与治疗

第十章 女性性高潮障碍

第一节 概 论

第二节 女性性高潮障碍的定义和流行率

第三节 女性性高潮障碍病因学

第四节 女性性高潮障碍临床评价

第五节 女性性高潮障碍治疗方法:理论探讨

第六节 女性性高潮障碍治疗实践

第十一章 阴道痉挛

第一节 概 论

第二节 阴道痉挛的定义与诊断

第三节 阴道痉挛的发病机制

第四节 病史采集和体检技术

第五节 阴道痉挛的治疗

第六节 个案报道

第十二章 女性性交疼痛

第一节 概 论

第二节 性交疼痛的病因

第三节 性交疼痛的分类诊断

第四节 预防和治疗

第五节 个案报告

第十三章 药物与性

第一节 有可能降低性欲或削弱性功能的药物

第二节 有可能增强性欲或改善性功能的药物

第三节 成瘾药物与性功能

第四节 中药与性功能

第五节 春 药

第六节 女性性功能障碍的药物研究

第十四章 常见疾病与性

第一节 内科疾病与性

第二节 妇科疾病与性

第三节 外科术后医源性性功能障碍

第四节 阴茎疾病与性

第五节 不孕不育对性功能或性生活质量的影响

第六节 精神障碍与性

第七节 残疾人的性问题

第十五章 男女生殖系统先天性畸形

第一节 性分化和男性生殖系统的发生

第二节 男性生殖系统先天性畸形

第三节 女性生殖器官先天性疾病

第四节 两性畸形

第十六章 性身份障碍

第一节 概 述

第二节 流行病学和病因

第三节 性别身份障碍的诊断

第四节 易性癖病的治疗方法

第十七章 性偏好障碍

第一节 概 述

第二节 病理本质和发病原理

第三节 临床类型

第四节 治 疗

第五节 性偏好与法律

第十八章 同性恋

第一节 概 述

第二节 国内外研究的现状

第三节 同性恋现象形成原因

第四节 同性恋的社会交往和性交特点

第五节 同性恋的现实表现

第六节 鉴别诊断

第七节 性取向的改变和同性恋者的心理调节

第八节 同性恋生活

第九节 同性恋的跨文化观点

第十节 美国互联网抽样中男男性行为的性功能障碍

第十九章 易性症与性别重塑手术

第一节 概 述

第二节 发病机制及病因

第三节 临床表现

第四节 诊断与鉴别诊断

第五节 手术适应证

第六节 手术方法

第七节 性别重塑手术的预后

第八节 性别重塑手术的伦理学问题

第九节 易性症的预防

第二十章 乳房整形与美容

第一节 小乳症

第二节 巨乳症(巨乳缩小成形术)

第三节 乳头内陷矫正

第四节 乳头乳晕重建

第五节 乳房再造术

第六节 男性乳房肥大症

第二十一章 生殖器官整形与再造

第一节 阴茎整形手术

第二节 阴茎延长术

第三节 阴茎增粗成形术

第四节 阴茎包皮过短矫正术

第五节 女性外阴整形手术

第二十二章 性传播疾病

第一节 概 论

第二节 梅 毒

第三节 淋 病

第四节 非淋菌性尿道炎

第五节 尖锐湿疣

第六节 生殖器疱疹

第七节 艾 滋 病

第二十三章 性功能障碍的中医治疗

第一节 勃起功能障碍

第二节 早 泄

第三节 女性性功能障碍

第二十四章 性学研究评定量表

第一节 临床性医学的研究内容与研究对象

第二节 性学研究评定量表

第三节 女性性问题量表汇总

第二十五章 性行为及其相关问题

第一节 何为健康的性行为

第二节 与医学有关的性问题的处理原则


\chapter{第一章 绪 论}

早在古老的原始社会,“性”就在母系制向父系制的转化中起到很大的催化作用,所以,人类性的存在远远超前于人类文明史的诞生。在那个社会里,由于社会科技发展水平有限,人们对很多自然现象无法认清,对自身的许多现象,特别是性欲与性行为的产生机制一无所知,把性看成是一种非常神秘的东西,于是就产生了一系列的性崇拜现象,如生殖崇拜、生殖器崇拜和性交崇拜等。

远古时期人类的性行为与其他动物没有什么区别,都是为了繁衍后代而进行的。随着时代的进步,人类性行为也在发生着许许多多本质性的变化,逐渐从动物性行为独立为人类性行为。

人类的性爱活动既是生理活动又是心理活动,不但能使人获得肉体满足,也能使其获得精神和情感需求上的满足,包括了建立在性欲基础上的一切性行为。所以爱情只是性爱的一部分,是满足精神需要的一个组成部分,是人类性欲的花朵。对于人类性行为的解释,不同学科有不同的看法:生物学家认为性行为是哺乳类动物(包括人类)具有的最重要、最基本的功能,因为有了它,才会有物种的永恒和生命的延续;心理学家认为性行为是人类理性行为的源泉,它是驱使男男女女承担日常生活的推动力;社会学家强调性的整合和凝聚功能,它对家庭的稳定和社会的安定起着重要作用;道学家则认为,在把性追求与道德标准和社会理想规范协调起来的过程中仍然存在不少问题;法学家认为:一切违背社会道德、文化、规范的性行为,包括卖淫、嫖娼、乱伦、强奸等,都应被禁止和受到制裁。性爱小说激发读者的性唤起,而性教科书则不应激起性唤起,而应激起人们大脑的思维。由此可见,人类性行为几乎涉及人类社会的各个方面。

当人类祖先认识到性在各族繁衍中起到的重要作用后,便把性行为奉若神明,给它赋予了无比荣耀的色彩。过去的生物学家把性行为定义为“任何增加配子(精子和卵子)结合可能性的行为称为性行为”,这个定义强调了性的生育功能。又如有些百科全书把性行为定义为“动物种内的异性个体之间互相展示的吸引、抚爱和交配的行为”。广义的性行为还包括同性个体之间的模拟交配和摆弄自身的生殖器官来满足性欲的行为,其条目中还指出:“有效的性行为必然要有异性个体的配合才能完成”。其定义是很注重性与生育功能之间的联系的。现代社会的快速进步无疑在不断冲击着过去的性观念,特别是人口爆炸的严峻局面和人们对高质量的小家庭生活的追求,使人们对性观念发生了深刻的变化,人们不再把性看做是生育的手段,而是把它视为享乐和愉悦的方式。特别是近几十年来,医学技术的飞跃发展已使我们能将性与生育分割开来,避孕技术的迅速进步使人们成功地做到了这一点。从另一方面来说,生殖本身也可能不具有任何性的含义,比如采用供精者的精液进行人工授精,采用丈夫的精子进行体外受精和胚胎移植(即“试管婴儿”)或配子输卵管内移植术都成功地解决了不育夫妇的生育问题,而在这些过程中,妇女都没有参与任何意义上的性行为。实际上,如果倒退150年,那时每对夫妇平均要生10个子女,按最保守的估计,他们将享有30年的性生活且每周仅性交一次,那么他们一生中至少性交1500次,与生育有关的还不到1\%。现在生育次数大大减少了,而性生活次数却相对增加了一些,所以性与生育的关系的确是越来越小了。

金西曾将性行为定义为以达到性高潮为目标的行为,这包括性交在内,也包括作用于自己或对方身体的其他行为方式。金西的研究以性高潮为中心和标准,实际上他把性高潮视为唯一客观正确的标准,以它来区别具体性活动是否属于性释放,甚至相对忽略了传统心理学所注重的心理和情绪。金西的定义具有一定的价值,因为它并不意味着性行为必定与生育有关,但它仍然存在着一定的问题,比如妻子与丈夫发生性交,而她并未获得性高潮及性紧张的充分释放,那对她来说还算不算是性行为呢?

性行为的定义若从单纯的行为方式上来界定似乎也有些不妥,例如,如果定义为通过生殖器的,那么还有通过语言的、手的、口的或工具的其他性活动方式,它们也能带来性高潮;如果定义为性交,那么还有大量并不以性交为最终目的的拥抱、接吻、爱抚等性活动方式;如果定义为男女之间的,那么还有同性之间的;如果定义为两个个体之间的,那么还有一个个体的自我刺激,如手淫或性幻想;甚至连“人与人之间的”说法都不严格或不全面,因为还有少量的人与动物之间的性行为。英国《不列颠百科全书》把性行为划分为“个人的”与“社会的”两大类,凡一个人独自进行的性行为多属于“个人的性行为”;凡是涉及另一个人,不以自己为性对象的,称为“社会的性行为”。但从心理学观点分析,如果个人在手淫时伴有与他人进行性交活动的性幻想时,这究竟算“个人的”还是“社会的”性行为呢?

为了避免出现上述各类问题,有人把性行为定义为:能产生性唤起并增加性高潮机会的行为,或者说凡能带来性满足的行为都是性行为,这样更为广义、笼统的定义也许更能被大多数人所接受。上海金德初提出性行为应该定义为:“凡是受性吸引力驱使的行为。”他进一步把性行为划分为三种类型:目的性行为,过程性行为和边缘性行为。他把性交称为“目的性行为”,认为这是符合物种世代传递的“含目的性”的本能行为,性交在人类性行为中的地位很高,算作是有目的性行为,简称“目的性行为”。完成这一行为后好像达到“目的”了,具有一定的终端性质,可以告一段落了,尤其是男子,随后便进入不适应期。他把为了顺利地达到“目的性行为”而采取的一系列能使配偶双方逐步达到性唤起和性兴奋的辅助性性活动称之为“过程性行为”,它实际上指事前爱抚或者前戏,事后爱抚或者后戏等为“目的性行为”服务的活动,也是一种求爱的过程。他把日常生活中,两性交往或夫妇相处时,受性吸引的驱使而出现的大量的行为与活动都列入“边缘性行为”的范畴,因为它们或多或少地具有性行为的性质,比如社交中广泛存在的“异性效应”,跳交谊舞等,其定位是比较模糊的,这一概念拓宽了对人类千姿百态的性行为的认知。没有日常大量的边缘性行为做铺垫,成不了恩爱的夫妻;没有过程性行为,性生活不可能协调。边缘性行为一旦具有明确的目的就会发展为边缘性行为→过程性行为→目的性行为的全过程。总之,他认为凡有性爱意味的举动都应该算作是性行为。

潘绥铭在他翻译的《金西报告———人类男性性行为》一书序言中指出,在性学术语中,性交、性行为、性活动这三个词有着严格的区别与界定,不仅不同于人们的日常用语,往往也不同于司法用语。他根据自己的研究提出如下定义:

性交———双方外生殖器直接插入交合,以及一方外生殖器插入另一方体腔开口处的行为。

性行为———包括性交在内,也包括作用于自己或对方身体,以达到性高潮为目标的行为。

性活动———包括性行为,也包括非直接肉体接触,但仍以获得性高潮或性感受为目标的行为。

金西报告在大多数情况下都是按这种标准进行划分的。虽然对性行为的定义是一件很困难的事,不过通过上面的介绍相信读者已能得出一个明确的答案,有了初步的认识,在进行深入的探讨时我们就会有共同的语言和理解。下面将从不同学派的观点对性行为进行理论剖析和评估。可以预见,对于研究性交等种种性行为的不同学派会有截然不同的观点。

在人类进化的整个历史长河中,性进化是其中一个重要组成部分。性进化必然导致性行为演变,而性行为的演变也必然推进性的进化,两者相辅相成,共同推进了人类社会的发展。

最早的生物繁殖是无性繁殖,无雌雄之分的低等单细胞生物,主要是通过细胞分裂来完成繁殖的。每个细胞体在生长到一定阶段时便开始分裂成两个全新的细胞,分裂出的两个子细胞与原母细胞体几乎一样,也就是现在所说的克隆。而分裂出的这两个子细胞又分别按照同样的方式分裂出下一代细胞体,这样循环往复,使物种得以延续。这种无性繁殖虽然简单、直接、安全和节能,但因其遗传特性的单一性,在遇到环境变化时,可能会遭到毁灭性的打击。

当生物进化为多细胞生物时,出现了两种繁殖方式(无性繁殖和有性繁殖)并存的局面。如水螅就同时具备这两套繁殖能力,它既可以通过发芽或生枝等无性繁殖方式产生新个体,又可以通过体腔内两侧出现的暂时性生殖腺产生精子和卵子,具有有性繁殖能力,此时它是雌雄同体,属自身受精。而对于高等生物来说,只有有性繁殖这一种生殖方式,必须要通过一对雌雄个体的交配,使精卵结合才能完成生殖功能,这种繁殖方式很快就以其遗传上的优势而在生物界占据了统治地位。因此,性成为地球上许多物种延续的共同基础保障。

古人类学和考古学的研究成果显示,约在600万年前人类祖先从动物界中分离出来,出现了最古老的人科生物;约在300万年前后,人类祖先开始狩猎并学会制造工具来进行各种简单生产,进而促进了古老人科生物朝着“正在形成中的人”① 方向发展;又大约在4万年前后,人类进入了成熟阶段,“正在形成中的人”逐渐进化成了具有现代体态的人类,形成了真正意义上的人。在这数百万年漫长的人类进化史上,我们的祖先经历了从树上到树下、从动物联合体到人类集团、从爬行动物到直立人种的伟大而又痛苦,充满着残酷与血腥的艰难转化历程,其中两性分化、雌雄动物和人类男女之间的性行为在这一转化过程中起到了必不可少的保障作用,使人类得以延续,人种得以优化。

在这漫长的转变历程中,人类性器官构造、生理功能和性行为也都发生了巨大的变化。这种变化包括两个方面:一方面是性行为的自然进化。人类祖先并不是一开始就会用脚进行直立行走,用手来从事各种活动的,他们为了生存,从树上到地面,为了获取食物,用前肢来拿取工具和食物,所以就由后肢来独立承担支撑身体的全部重量。渐渐地前肢完全离开地面,形成了人类所独有的上肢,而后肢也摆脱了依靠前肢才能支撑的局面,成为能够直立在地面上蹒跚行走的人类独有的下肢,至此直立人就基本形成了。这是人类为了自我生存,为了适应自然环境,四肢进一步分工的结果,同时这也使人类的体质形态发生了巨大的变化。由于重心和重量的影响,人类的下肢股骨、腓骨和胫骨变粗、变长,出现了与其功能相适应的变化。而对于女性来讲,其在体质形态方面的变化遇到了比男性更大的麻烦,其进化也要复杂和痛苦得多。由于运动姿势的改变和身体重量的下压,女性的骨盆上缘部分进一步张开,导致骨盆底部(耻骨联合部)相对地缩小,生产小孩的通道(医学上叫“产道”)相对地狭窄,造成了女性难产率和死亡率始终居高不下,导致成年男女之间的性比例失衡。在这种男多于女的情况下,每当到了狩猎生产的间隙期,处于任何情期的女性不管是自愿或不自愿,甚至是强迫地接受男性的性行为。女性怀孕和生育则成了她们必须通过的最大难关,只有那些耻骨弓角度小、产道宽、身体相对健壮的女性才能较顺利地渡过难关生存下来,这就是优胜劣汰的自然法则。从整个人类进化的历程来看,凡是能顺利通过分娩的女性其子女的骨骼体质都发生了很大的变化,因为只有那些头部、躯体比较小的婴儿才能通过母亲那狭小的产道而降临人世。这一点可以通过古老人科生物的头颅变化得到印证,过去人类祖先的头颅是呈基部馒头形状,随着进化的演变,人类的头颅呈现出现代人的半球形状,从而有利于通过母亲的产道顺利生产。

在人类漫长的进化历程中,女性后代不断从其前代那里继承优秀的遗传基因,使人种能够不断得以优化,能够在非常恶劣的环境中生存下来。其中有两种逐渐延长、增强的遗传基因与人类的性行为改变密切相关,一种是后代女婴从母体内得到了发情期相对延长直至逐渐消失的遗传基因,从而形成了稳定而有规律的月经;另一种是后代女婴从其母体那里逐渐继承了越来越强壮的身躯,使其体质形态都发生了根本变化,耻骨弓的角度相对变小、产道相对变宽,从而为人种数量的增多奠定了基础,推动了人类社会的不断发展与繁荣,最终从体态上将人类与其他灵长类动物彻底分离。

人类的社会进化与社会的发生、发展是密不可分的。由于自然环境恶劣、食物短缺,人类祖先不得不借助于一些工具来猎取、生产、加工食物,由此物质资料的生产拉开序幕,然后又在原有生产的基础上发展生产、扩大生产。就人类自身的生产来说,其社会结构是从动物联合体的生物学结构中逐步脱离出来,经过原始群体到母权制,再到父权制的氏族公社等几种结构而转变来的,人类性行为也是从血亲杂交、群体婚姻,经过对偶婚姻而过渡到现代的一夫一妻制婚姻的。

随着女性发情期的消失,人类性行为不再限制于某一特定时间内,而是几乎所有时间均可发生,这就使女性在性行为发生的时间和对象上具有了选择的余地。这一变化也迫使男性必须改变过去强行交媾的性行为方式,转为争相在女性面前展示其男性的机智、勇敢与健壮体魄等特性,以期能引起异性对他的青睐,从而引起她们的性响应,这也是人类最初产生爱情的生理心理基础。由于在原始群体中,每个成员都有爱与不爱、接受爱和拒绝爱的权力,而在当时还没产生能够约束两性之间情爱问题上的道德规范,于是整个群体就陷入了情爱纠葛的纷争之中。在人类最早的采集活动中,这种冲突对群体的影响还不大,但到狩猎时期,因狩猎活动讲究相互配合,这种冲突往往会导致一个群体的分裂,甚至瓦解、崩溃。因此,为了群体的生存、发展,人们不得不规定在狩猎时期禁止男女性交,这就是当时的狩猎性禁忌。除狩猎性禁忌外,随着社会的不断发展,人们思维能力的不断提高,对人与人之间的性行为也有了更进一步的认识,由此又产生了许多诸如乱伦禁忌、月经禁忌、生产性禁忌等多种禁忌规定,这些性禁忌又因社会生物学的自然选择作用而得以推广。这些性禁忌的产生,使人类的婚姻方式由群婚逐渐过渡到对偶婚,再逐渐过渡到个体婚姻,并由此改变了诸多社会分配关系和人与人之间的关系,极大地推动了人类的社会进化(详见第二节)。

由于生存和劳动的需要,人类开始逐渐地由四肢爬行到直立行走,其物种的进化和体位的改变对人类性行为的影响具有深远的意义。

所谓发情期(oestrus period)是指除人之外所有动物都具有的在某一时期才能产生性欲、进行交配的性生理现象。在动物体内分泌性激素的内分泌系统只在一年中的某些特定季节或时期内启动,这时动物在性激素的作用下,才能产生性欲即发情,并进行实际交配。而在这以外的其他季节或时期里,动物的性激素内分泌系统处于停歇状态,此时的动物没有性欲,也不会性交,如此往复,就形成了发情期和非发情期这样两个周期。动物发情期延续得越短,性交与怀孕的机会就越少,致使物种繁衍与进化的可能性也越小。

随着物种的进化,雌性猿类开始出现月经并呈周期性变化,于是发情期就开始根据月经周期的出现和循环而出现,受季节的影响日益减少。狒狒类动物的发情期就已相当长了,在每个月经周期(约30天)内可有19天的时间能够产生性欲并进行性交。而到人类发情期则已完全消失,也就是说人类女性可以在全年的任何时候产生性欲并进行性交,而不受月经周期的影响。发情期消失(aoestrus)被称为人类性进化史上的一次革命,它的意义在于:①使得人类成为自然界中唯一专注于性爱、性交的数量和形式最多的物种;②使得人类性交需要在体位上变化才能够突破时间上的限制,更快更充分地发展起来,成为人类持久的性行为模式,成为促成配偶家庭结构形成的重要因素,并为婚姻与家庭的维系打下坚实的基础;③增加了受孕机会,使得人类以其他动物所没有的速度扩大数量,100万年前世界人口约为50万,而到1万年前的农业社会世界人口已达300万,这在当时已经是一个非常可观的数字,因为当时人类的存活率是极低的。相比之下,在这段时期里其他猿类和哺乳动物却增加不多。除人口数量的增长外,人口质量也得到了迅速提高,这是因为人类发情期消失后,男女都可以更自由地按更高的标准来选择自己的性伴侣,从而生育出超过父母的儿女,这就是达尔文生物进化论中著名的“性选择”理论② 。

哺乳类动物包括灵长类动物性交时的姿势通常都是雌雄两性相互接近后,由雌性动物反转身来以其背部朝向雄性,然后利用上半身的自然弯曲向前俯曲,翘起臀部,雄性动物则在身后,将其阴茎由后方插入雌性的阴道进行交媾,在性学上将这种性交方式称为“后入位”(rear-entry)。这种后入位的性交姿势是短暂、残暴的,是只为达到泄欲和生殖目的而进行的性行为方式,即使在古老人科生物中也不例外。当人类使用这种性交姿势时,男性只能看到女性的背部,此时女性丰满的臀部成为其欲望满足的主要对象,而前面的女性除了来自臀部性感区的刺激外,其他部位则毫无感受。人类从直立行走起,在生产劳动的不断影响下,脊椎逐渐伸直,头从脊椎的前上方移至上方,此时无论男女都可以十分自然地平趴或躺下,这就给人类性交姿势的改变奠定了基础。人类直立行走既解放了双手,也使许多解剖结构得以改变,如男性的阴茎位置由朝下变为朝前;女性的阴道和骨盆位置也由朝后变为朝前等,这样人类就可以面对面地进行性交,男性阴茎能够顺利地从女性正面插入阴道,性学将其称为“前入位”,这种性交姿势的出现成为人类最后脱离动物界的一个伟大转折。人类这种区别于动物的性交方式,对于促进人类社会的发展,促进人与人之间的交流和沟通,促进人类性进化具有重大的意义。具体表现在:①人格化的结偶要求。在这种面对面的性交里,人类才能把某个特定的对方与整个异性群体严格地区分开来,才能更充分地对某个特定的对象产生最为强烈的吸引、爱慕和依恋感。这一切都强化了人类所独有的对性交对象的选择能力,那种动物性的、纯生理的发泄行为,变成了对人格的选择要求。因此,人类才有可能出现一对一的爱情和后来的专偶制婚姻,并为人类家庭的诞生奠定了基础。②心理情感的交流。在这种面对面的性交里,男女双方才能够更多、更充分地运用表情、神态、举止和语言来进行心理和情感上的交流,有利于双方共同体验性的快乐,使性交时间延长,性交方式多样化,创造出人类所独有的爱情和对于性活动的审美,以及性文化与性观念等。③性敏感区扩大。在这种面对面的性交里,男女双方才能运用自己的双手和口舌,充分地爱抚和刺激对方的身体,这在客观上促使人类体表性感区和敏感部位更加发达,也促使人类性感区的数量更多,分布更广,类型更加多变。④感受器官的多样化。在这种面对面的性交里,人类接受性刺激和输出性刺激的途径也发生了很大的变化,从以嗅觉为主转变为以视觉和综合心理感受为主的刺激途径,这也是人与人之间除了性关系以外,能够产生独有爱情的基础之一。⑤女性体验性高潮。在这种面对面的性交中,女性也能够积极参与性活动,享受性快乐,体验性高潮,这种不同于动物的性体验,对人类两性关系的发展以及家庭的产生都有重要意义。⑥性审美的产生。在这种面对面的性交中,男女开始关注对方,心怀情感的交流产生了性审美,原始时代的女性人体雕像,突出表现的是肥胖的体态和硕大的臀部,没有面部五官,而一些稍晚的艺术作品如古希腊的人体雕塑则从原始时代突出表现的臀部转移到了头面部、胸乳部和全身,展示了人类审美情趣的演变。⑦女性人格的独立。在这种面对面的性交中,人类女性才能与男性一样,解放出自己的双手。因此,在客观上具有了抗拒违背自己意愿的性交的可能性,这标志着女性在性生活中开始具有了独立的人格和意志。正是由于这种可能性的客观存在,在人类发展的各个时期里,虽然对强奸行为的处罚有着不同的规定,但是对强奸行为的定义却是一样的:凡是违背女性的主观意愿,强制(迫)进行性交的行为,都属强奸行为。

人类性行为从进行的形式上看,似乎与动物性行为无明显差异,但由于它更多地融入了人类的思维和情感,使性行为产生了一种质的飞跃,并与人类社会生活的各个层面产生了千丝万缕的联系,使得人类性行为具有以下特征:

世界上没有一个人与性行为无关,每个人都是性行为的产物,同时,在正常情况下,每个人都会有性要求,都要结婚、发生性行为、然后生儿育女。

性行为与人类生活有很大关系,因为性是生物繁衍的基础,性活动是人类的基本活动之一。“食色,性也。”除了吃饭以外,性要求是人类第二自然本能,人们在满足性要求的过程中获得了极大的快乐。不仅如此,性行为还是延续后代的必要手段,它和每一代人的身体健康、社会经济发展都有密切联系。因此,从古到今人们都十分重视性这个问题。

性问题伴随着人的一生而存在,表现形式也是错综复杂的。现代科学证明:人从幼儿时期开始就有性意识,这种性意识到青春期迅速增强,以至付诸行动,由此引发了许多性问题。即使到了七八十岁,多数人还是有性意识、性要求和性行为的,只是表现形式和中青年时期稍有不同而已,这时所面临的性问题也有所不同。

性行为的隐蔽性是由性行为的排他性引起的。在原始社会,实行群婚杂交,性行为说不上隐蔽,但随着私有制和一夫一妻制家庭的出现,妇女隶属于男子,而且只能归一个男子所享有,于是,对妇女的贞操要求严格得无以复加。性行为不仅不能有第三者参与,而且不能让第三者看见,这样就表现出性行为的排他性与隐蔽性。当性行为的隐蔽性发展到畸形程度,就会产生性神秘感。

性行为虽然源自人类本能的性要求,但绝不只是个人的私事,绝不能只满足于个人的寻欢作乐。它必须对对方负责,对后代负责,对社会负责,其责任是重大的,也是极为严肃的。

表现为同样一种性行为,在有的社会里是高尚、合乎道德的;而在有的社会里则是低下、不合乎道德的,甚至是严重的犯罪行为。同一行为具有如此尖锐而对立的性质,实属罕见。

正因为性行为具有以上特征,人们对普及性科学和性知识的学科———性教育学就抱着十分矛盾的心态。有的人认为这是个人的私欲,是不能公开探讨的,甚至把正常的性行为混同于黄色、低级和下流的行为。于是性教育学研究就成为一个“禁区”,在性问题上存在并流行着许多荒谬而愚昧的观念,导致我国出现众多的“性盲”。我们现在就是要大力推广和普及性知识和性科学,从各个方面深入研究性教育学,研究性与社会经济发展之间的各种关系,以提高人们的“性商”(即性健康水平的测量标准,如同智商、情商等),使人民群众在生活水平达到小康的同时,性生活水平也能达到“性小康”(即指性健康水平和性知识水平达到小康状态)。

最早对人类性行为进行分类的当属德国人Wilhelm Von Humboldt,他在公元1826~1827年间撰写的《人种信赖史》中就提出了对人类性行为的自然分类法,按照他的划分标准可以将性行为划分为:①自我性行为。②异性间性行为。③同性间性行为。④动物的性行为。美国著名性学家金赛(Alfred C kinsey)认为人类性行为表现为自我刺激(自慰)、性梦和梦遗(梦中性行为)、异性性交、同性恋、与动物的性行为等多种形式分类。这些分类更多的是考虑性行为对自身的影响,而对人类性行为与社会之间的关系考虑较少,如将这些关系进行全方位分析,可对性行为进行如下分类。

(1)自身性行为:如手淫、梦遗、意淫、性幻想、白日梦,还包括变通性对象的兽奸、代用性器等。

(2)社会的性行动:凡是以人为对象(包括尸体)的性行为,因牵涉一系列社会伦理问题,故属于社会的性行为,包括异性恋、同性恋、群居杂交、交换配偶等行为。

(1)正常的性行为:指被社会文化认可的性行为。它要求性对象必须是成年异性,关系是法定的配偶,否则就可归为异常性行为。对性行为的认可有些地方或国家也很严苛,如仅限于阴茎—阴道性交,甚至认为只有指向生育的性交才是正常的,妻子怀孕后的性生活也是不正常的。目前一般认为,只要是夫妇之间自愿的、无伤害的性行为,都属于正常的性交行为。

(2)异常的性行为:目前一般将性变异以及有害健康与违反社会道德的性行为视为异常性行为。

(1)目的性性行为:指能够达到较大性满足的性交或相当于完成性交(如能达到性高潮的手淫)的行为。在这里性交是性行为的直接目的和最高体现,一般来说,人们在性交之后,就满足了性的要求。

(2)过程性性行为:指导目的性性行为的调情或围绕实现目的性性行为的动作,即性交前的准备行为,如接吻、爱抚或其他调情动作。这些动作的目的是激发性欲,实现性交。性交后还需通过一些动作作为尾声,使性欲逐渐消退,这也属于过程性性行为。

(3)边缘性性行为:指日常生活中界于性行为与非性行为之间的行为。这种性行为的范围较广,它的目的是表示爱慕,或仅仅是爱慕之心的自然流露,而不是为了性交。边缘性性行为有时很隐晦,如表现为眉目传情、一丝微笑,这眼神、微笑有时只有两个当事人自己感觉得到,其他人则茫然无知。还有些少数民族为了挑选意中人,采用的某些风俗习惯,如丢荷包、抛绣球等行为,也属边缘性性行为。至于拥抱、亲吻,如果只是爱情的自然流露,不以性交为目的,就属于边缘性性行为,而边缘性性行为一旦具有明确的目的,就会发生边缘性性行为→过程性性行为→目的性性行为转变的全过程。

(1)社会性性行为:在婚姻形式下的一夫一妻、一夫多妻、一妻多夫间的性行为。

(2)非社会性性行为:如兽奸等。

(3)违规违法的性行为:如婚外性行为、乱伦、强奸、卖淫、嫖娼等。

既然手淫是普遍存在的一种有效和广为传播的能给人们带来身心愉快的生理现象和行为,那么人们为什么硬是要给它笼罩上一层神秘、压抑、罪孽、羞愧的面纱呢?为什么会对它深恶痛绝、严加禁锢呢?对手淫的批判始自西方的《圣经·旧约》时代,由于古代性观念的影响———性的目的就是为了繁衍子孙,所以当时的宗教把非生殖目的的手淫、性交中断和同性恋等性行为视为罪恶。中世纪的教义把性犯罪分为自然的和非自然的两种,例如通奸、乱伦和私通都属于自然的,因为其可以导致妊娠;而凡是手淫等不能制造生命的统统是不自然的,就像杀人一样严重。当时竟有人赞成娼妓制度,因为它可以避免更严重的性犯罪,这些人认为以妓女作为泄欲工具要远比让一个人手淫为好。中世纪的教皇圣·奥古斯丁更是宣称:“性生活是罪恶的起源,性欲是传播罪行的通道”。违反生育目的的手淫当然“使主不悦”,“罪莫大焉”。

到了18~19世纪,旧礼教让位于医学科学,对手淫的批判也出现了新的变化。一些学者和庸医在经济利益的驱动下开始喋喋不休地宣传手淫和梦遗对人体健康的危害,如“手淫者的后果是可怕的,他们要经受很多痛苦,最终死于脑衰竭”;也有人把手淫看做是一种满足性冲动的病理形式,认为它对身体健康和心理健康都极为有害,如造成脊髓痨、贫血病、内心冲突、犯罪感、神经衰弱、精神神经病等;还有的医生指出精液的损失会带来疲倦、虚弱、头痛、感觉模糊、愚蠢等,甚至把神经症、精神分裂症统统和手淫联系起来。于是人们只能找他们寻求“救助”和“治疗”。人们给医院里成千的精神患者穿上紧身衣或采取其他捆绑手段,以使他们无法手淫,据说这样才能控制和治疗他们。这些广为流传的荒诞言论世代沿袭,不知坑害了多少青少年,使他们对手淫产生极度恐惧的心理,视之为“洪水猛曾”,“自我摧残”。20世纪初一位叫斯利瓦纳斯·斯托尔的牧师写了一本名为《男孩必读———他们应该了解些什么》的书,书中大谈手淫的后果如何可怕,连犯科者的后代都不能从这一罪恶中幸免。他还举例说如果种子不好,就不会长出好庄稼,所以当男孩子不自重而沉溺于手淫的话,他的后代便会出现身体、精神和道德水准的低下。1950年一篇报告指出,这本书的影响是广泛而深刻的,它使许多青少年受到良心的谴责,由于惧怕书中宣扬的恐怖后果,不少人试图或真的自杀了。

直到1976年,一些宗教会议还顽固地坚持:“手淫构成了重大的道德混乱……手淫是一种天生的和严重的失常行为。”有些教派近来是这样谴责手淫的:因为它不是利他主义的,所以是错误的。他们认为只要不是为了上帝和一个人的左邻右舍而只是为了自己做的事,都是有罪的,他们认为手淫只能给自己带来快乐,违背了上帝为大众的意志,所以是不正常的。

手淫在中国历史上遭受的压抑虽然不如西方那么严重,但中国古代房术家们提倡的“御而不泄”“还精补脑”“采补术”等说教,都是强调精液宝贵的,笃信精液是“天地之精气”,“精华”,“一滴精,十滴血”,生怕流失了几亳升精液就伤了“元气”,他们把手淫称为“非法出精”,视之为异端,认为手淫和遗精可以使人精力耗尽,等于坐以待毙。尤其是西方对手淫的偏见在清朝末年开始传入中国之后,这些人更是如获至宝,把西方的毒素糅合到他们的邪说里,变本加厉地加大对手淫的抨击,至今仍有一些专业人员或持类似观点的人动辄扣上“肾亏”的大帽子吓唬青少年,例如1992年有家报纸还载文认为“手淫有害”,造成恶劣影响,近年来甚至有网站专门肆意宣传这类反科学的谬论。笔者对中学教师的一次调查中发现,75\%的中学教师认为手淫是有害的。

对手淫观念的转变可以说自弗洛伊德开始,他首先揭示了手淫和性幻想问题,但他对手淫的观念也是自相矛盾的。弗洛伊德承认手淫在婴儿期是普遍发生的,也注意到它在根本上属自体情欲的性质。他鉴别了手淫活动的三个阶段:①婴儿期,自体动情活动到为性满足服务的总目的。②儿童期,手淫固定在特殊的动情区。③青春期,一方面是儿童手淫的继续,也可能是与之不同的潜伏期表现。

以发展的眼光来看,手淫本身不再是病理性的行为,但不得不评价为性心理发育的面对面阶段。弗洛伊德认为到了更成熟的年龄,若仍把手淫作为性满足的主要方式就应考虑为“幼稚的性活动”,是过于神经质的。弗洛伊德对手淫活动的概念化也越来越具“心理”性质,他有效地把手淫与阉割焦虑联系起来,并注意到对它的抑制可能产生的心理功能。最后,弗洛伊德对手淫伴有的幻想的重要观察可能是他对手淫心理研究最有意义的贡献。实际上,手淫中的幻想作用在当代心理分析学说中始终作为一个中心重要性的问题,有些人还以此来区分生殖器自我刺激和手淫。总结弗洛伊德在建立对手淫和心理功能表现之间的理解方面的贡献,可以说他熟练地把手淫活动同化到性心理理论的整体结构之中,在这一方面,弗洛伊德的观点是新颖的,并使其成为他理论中的一个符合逻辑的重要成分,即异性间生殖器性活动是性发育的最终状态。然而,弗洛伊德从未完全放弃他对手淫纯粹有害作用的观点,一种缺乏心理学内涵的观点。所以可以认为弗洛伊德对手淫的观点是自相矛盾的。

金西报告(1948,1953)的统计学分析结果首次打破了因宗教虚伪宣传而造成的对手淫的社会否定,他指出58\%的女性和92\%的男性曾通过手淫方式达到过性高潮。金西报告极大地影响了人们的性观念,他认为没有任何特殊的性行为是错误的,性行为本身是应该受到尊重的,越是精力旺盛就越值得尊重。该报告反复指出,一位经常手淫的律师“在他的同事中”将是“出类拔萃的”,女人比男人更经常进行不带想入非非的手淫,而且她们往往都处在引人注目的社会地位上的最有能力、精力最充沛的女子。他的报告指出,人类性行为因社会阶级、教育程度的不同而存在显著差别。作为一种文化,西方人认为人的能动性趋向于幸福和成功。金西认为从生物学观点看,人们为发泄情感而采取的任何方式都是自然的,有些虽然离奇,但只要没害处,为什么不能效法?何况这些离奇的念头和做法也往往表现在人们的吃、穿、住、行等各个领域之中,人类性行为和其他行为一样,变化范围都在逐渐扩展。金西还指出:“虽然我们掌握的几千份病例报告说明,许多男孩因为对手淫感到忧心忡忡而痛苦不堪,但我们多年的研究并未发现手淫给人带来任何明显的伤害。”

马斯特斯和约翰逊的实验室研究证实手淫与性交引起的基本生理反应并没有什么区别,说明它并没有什么对身体的特殊影响,这就从根本上推翻了手淫对身体种种危害的说法。亨特在20世纪70年代的调查里发现,94\%的男子和63\%的妇女曾经手淫过,这表明人们对手淫的态度已有了进一步改变,大多数人特别是年轻人不再相信手淫是错误的观念。手淫在两性年轻人中的发生率都有所增高,但仍然存在明显的性别差异。受教育较少的男女仍然相信手淫的种种民间传说,受过良好教育的人更为相信关于手淫的事实证明,他们所在社会群体的性价值观不同,他们不太惧怕手淫危害健康,他们在婚前性交的自由程度较低并容易受到更多的限制,所以也更倾向于手淫。按照亨特的观点,手淫发生率不是性趋动力的指数,但却反映了通过对性活动和性伴侣性幻想的追求变异的心理需要。有73\%的男子和80\%的女子在手淫时产生与所钟爱的人进行性交的幻想。美国性知识和性教育委员会则针对20世纪70年代以后仍然顽强抵抗手淫的有些宗教会议作出强有力的反应并坚决指出:“手淫是一个正常的、自然的和使所有人都能得到美好乐趣的技巧,从本质上来说它是无辜的,纯属个人的行为”。

1992年6月18~22日在荷兰首都阿姆斯特丹召开的第十届世界性科学大会上,荷兰卫生、文化和社会福利部长在开幕式上讲到:“手淫以前曾被认为是一种病态,但现在被看做是无害,甚至是健康的行为。如果某人有性问题,往往是那些不能手淫的人!”这句话博得了来自58个国家和地区的850多名专家、学者的热烈掌声,因为许多国家的调查证实手淫在青少年中是普遍存在的现象。

统计表明有17\%的青少年时时为自己的手淫行为感到严重内疚和自责,有32\%的青少年有时感到内疚和自责,很少内疚的有32\%,从不对此感到内疚的只占19\%,也就是说有1/6的青少年始终生活在沉重的心理和社会压力之下,有1/3的青少年仍不能正确对待这一问题。他们的焦虑和畏惧是什么呢?可以说其畏惧的原因是五花八门的,他们总是道听途说,又特别喜欢听这方面的种种议论,每当听到些什么就又紧张又害怕。有人在举不出任何实例的情况下说手淫会导致同性恋,因为手淫得不到满足就会去找同性发泄;还有的人说手淫就是从同性伙伴那里学来的,因此,手淫会导致同性恋。其实,这些推测都是站不住脚的,关键是没有事实依据。这种想当然的推断是十分荒谬的,但其影响却不可忽视。

有的青少年害怕婚后不育,甚至精液里出现小凝块就说是死精子或畸形精子,便对过去染上手淫懊恼不已,想到将来可能会不育,甚至连女朋友也不敢交了,不少人还想到轻生。其实,手淫又怎么可能影响生育力呢?96\%的男子都有过手淫,95\%以上的男子都具有生育能力,可见手淫不会造成不育。手淫还是医学上应用最广泛的收集精液的方法,其简便易行,收集的精液标本完全、洁净、不会影响化验结果。

更多的青少年害怕婚后出现性功能障碍。随着时代的进步和性治疗技术的发展,人们认识到手淫不仅不会造成性功能障碍,相反,它还成为性欲抑制、性高潮障碍、早泄、勃起功能障碍和阴道痉挛等的治疗措施之一。研究表明,女性婚前是否手淫与其婚后能否达到性高潮显著相关,婚前感受到手淫高潮的妇女,婚后很容易达到性高潮。这是因为女子通过手淫学会了自我刺激的有效技术,并把它带到婚后性生活之中,这对于维持融洽的夫妻关系是十分重要的。

一名大学生说:“作为一名能到首都念大学的天之骄子,本应积极向上,奋力拼搏去攀登科学技术的高峰,以报效祖国和父亲的养育之恩,但我却长期陷入痛苦而不能自拔。我自幼失去母爱,总是感到孤独,缺乏母爱常使我悄然流泪。小学时趴在地上写作业,身体的重压触动了自己的那玩意儿,产生了快感,久而久之养成了习惯,我把它叫做‘五个对一个’。初中后才知道叫手淫,是坏习惯,对身体有害,但我改不了。随着年龄的增长和身体的成熟,当手淫时那里第一次流出黏液,我的确吓坏了,以为是血,一看是乳灰色的,是淋巴液?后来才知道是精液。在射精之前的半小时左右我有一种莫名的躁烦和不安,后来不知怎么的又压、又摩擦才射的精。慢慢地,我的阴毛长出来了,但我以为是手淫的恶果,于是洗澡前总要把它们剪去,以免爸爸发现我有过手淫。不知什么时候我又突然发现自己的阴茎短小而且睾丸一大一小的,更认为是手淫所致,每次洗澡时总是那么不自然,怀疑别人在注视我那过小的阴茎和一大一小的睾丸。从此我不再在有人的情况下站着小便,即使公厕里无人,我也要边解边侧耳旁听,一旦有风吹草动,尿就会被吓得憋了回去,然后假装大便蹲下去继续排尿。有时我拼命少喝水,或早早赶到学校以提前解好小便,或下学后飞跑回家再小便。由于满脑子的自卑和内疚,我几乎快崩溃了。我日益变得孤僻,没有朋友,不敢接触异性,只能把对性的向往停留在幻想和手淫之中。我渴望阳光雨露,渴望友情和爱情,可是我怎样才能够摆脱手淫给我带来的阴影呢?”

我们发现咨询性问题的孩子中很多是三好学生或学习尖子,越是受家长、老师宠爱的孩子,越容易遇到性生理和性发育带来的问题。他们像生活在真空的玻璃罩内,什么课外信息都得不到,只埋头于书卷之中,两耳不闻窗外事,一旦遇到事情就不知所措了,而自尊心又阻碍他们向别人请教,只有把性困惑甚至痛苦深深埋在心中,成为挥之不去的烦恼,最后变得心灵脆弱,好像是温室里硕大无茎的花朵,看上去不错,却经不得任何风吹雨打,缺乏野花那样顽强的生命力。他们在过分纯净的人工环境里长大,但要面对的却是遭到破坏的生态环境,因为平时接触不到各种抗原(包括有害的刺激),也就产生不了相应的抗体,最后免疫能力大大下降,一旦遇到不良刺激只有夭折。其实他们遇到的只不过是像手淫、遗精、生殖器官发育等几乎每个人都会遇到的问题,可是这些“天之骄子”却一筹莫展,在这些小问题面前败下阵来,可见全面素质教育的重要性,千万不能只抓分数。

羞耻感的实质是畏惧,害怕社会的非难、嘲笑和惩罚。羞耻感是自我评价的产物,内疚或负罪感则是自我评价的另一种产物,它与一个人的自尊(也即自我评价)有着密切关联。内疚或负罪感不是畏惧而是愤怒,是对自己生气,它是一种自我厌恶、自我定罪、自我惩罚的情绪。这种自我评价往往同一个人的道德价值和在他生活中占据特别重要地位的人的道德价值有密切关联。羞耻感一般仅发生在特定的场合和环境,如被陌生人看到自己的身体而出现的羞耻感;负罪感则比较泛化,个体将对自己给予全盘否定,认为自己不仅在这一特别行动中有罪,而且整个人就是有罪的,他们往往无法逃脱自我谴责。所以要解决好上述青少年的性焦虑,首先应纠正其过分的羞耻感和内疚感。

对许多人来说,性幻想是一种重要的色情刺激,是在清醒状态下或似睡非睡状态下通过思维活动而自我刺激或自我进行的一种性行为方式。性幻想的内容无疑充满色情色彩,它可以自行产生,也可以在手淫或与他人性交时产生。性幻想的能力对人类的自然性反应有着重要影响,所以有人说性就是摩擦加性幻想。有人指出将近80\%的男女在手淫时伴有不同程度的性幻想体验,如72\%的男性手淫时几乎都伴有性幻想,17\%的男子有时伴有性幻想,手淫时无性幻想的男子约占11\%。女性手淫时总存在性幻想的有50\%,14\%的女性有时伴有性幻想,而36\%的女性从未有过性幻想。男女性幻想的内容虽说有不少相似之处,但男性的性幻想多偏重于想象自己是强有力的,或主动追求一个想象中的女性;而女性的性幻想则常常带有浪漫色彩,或想象从未尝试过的性活动。这也充分反映出文化传统对男女两性性行为的深刻影响。过去一直把性幻想看做是思想不健康和道德格调低下的表现而加以否定和批判,而事实确是几乎所有人的头脑里都会存在不同程度、不同内容的性幻想,在意识中普遍存在动情的想象,美好性经历的回忆或期望。幻想可以在手淫或性交时伴发,也可以单独存在,青年人对爱情与性交场面的想入非非可以称作白日梦。

性幻想是从青少年时期开始,一直到老都存在的自然行为,但随着年龄的增长,它的频率将不断下降。性幻想的内容可以是离奇古怪的,也可以是现实生活中存在的人或事,这种想象是永无止境的,也是变化多端的。性幻想并不会造成任何伤害,甚至有一定的积极意义。但事情都有个限度,青年人若是整天沉溺于性幻想而影响了学习和工作,影响了正常的人际交往,那就会造成对心理发育的伤害。当然什么是正常和适度,什么是适应不良和过度,并不存在客观的定量标准,关键是看它对具体人的客观效果。在性咨询中常常遇到此类问题,有必要进行正确的引导,要帮助青少年正确对待性发育过程中萌发的“白日梦”问题,但又不能恫吓和指责他们,给他们造成无谓的精神压力,以免适得其反,影响他们的身心健康发育。

至于性幻想的内容是否健康,比如对一些出格和违禁的性对象或性情境的幻想是不是错误的,这显然是一个难以定性的问题。如一个女孩可能想象让心目中的白马王子或偶像轻轻吻了一下额部,而另一些人则对再离奇和荒诞的幻想也能耐受,所以个人的反应很重要;二是绝大多数性幻想仅仅是想想而已,根本不会付诸实施。所以,问题不在想什么而是实际干了什么,这种情形不仅在性幻想中会遇到,在生活的其他方面也会遇到。

与性幻想类似的是性梦,性梦指与性活动、性爱刺激、性信号或性关系有关的梦境,做梦者醒后可以回忆起梦的内容。性梦是人类一种完全正常的、无意识或潜意识的性心理活动,也是人们健康、普遍、常见的自身性行为或性自慰的行为方式之一,可以把它看做是机体调节性紧张和性冲动的自发措施,或是一种心理防卫机制。做性梦绝不是什么见不得人的事,更不是在道德或身体上出了问题。人的许多梦境都是为了替代性地满足某些需求而发生的,当一个人达到性成熟而又无法得到性满足时,往往就会以性梦的方式表达出来并从中获得一定的满足,所谓“日有所思,夜有所梦”就是这个道理。无论是未婚的年轻人还是已婚但性生活并不如意的中青年都可以出现频繁的性梦。男子性梦时可伴有遗精,女子可伴有性高潮。但性梦不一定都伴有遗精或高潮,遗精或高潮的发生也不一定都有性梦相伴随。性梦是潜意识里自发出现的,人们无法控制它,因为性梦多是代表一个人最原始的性欲的“本我”的自然释放,而平时的“本我”都被主宰非判断的“自我”和主宰道德判断的“超我”压抑到潜意识中了,于是在“自我”和“超我”暂停对一个人的原始性欲进行监控的性梦中,常常出现平时受压抑或禁忌的性内容就毫不奇怪了,从浮现出的这些禁忌内容可以分析做梦者的性心理或性功能障碍的潜在原因。梦中出现性禁忌不必引起做梦者的慌张和不安,因为它们毕竟只是梦,并非真的现实或真的想要实现那些禁忌的性活动,而且那些情节或对象往往是模糊的,所以不存在所谓的道德评价问题。对于有些年轻人存在的对性幻想和性梦的一些错误认识,考虑到这种消极观念和态度是在长期生活中形成的,所以要想一下子让它们消退得干干净净也是不现实的,需要医生长期耐心的帮助。由此也可以看出科学的性教育是多么重要,必须通过经常性的、科学的性教育,才能消除人们的种种困惑,防患于未然。

性交之前存在一个求爱的过程,事前爱抚就是性爱活动中表示情爱关系的一种行为方式。性交之后仍需爱抚,人们称为事后爱抚,爱抚是人类性行为的重要形式之一。即使是动物也不例外,如雄孔雀开屏围着雌孔雀翩翩起舞,就是一种典型的求偶行为。人类的求爱过程对保证性交质量和性满意程度十分重要。亲吻、拥抱、触摸、爱抚都是一些为达到性兴奋而必不可少的基本活动,它们不仅是性交交响乐的前奏曲,也是增进双方情感交流的必要方式。这种事前爱抚主要通过抚摸以乳房和生殖器为代表的性敏感区而达到性唤起的目的,一般应从温柔的爱抚开始,然后视对方的反应而加强。事前爱抚对大多数女性,对性感比较迟钝、具有种种性困难的人是很有用的手段。在夫妻生活中爱抚或情戏的次数要多于性交的次数,有时兴之所至,相拥热吻,情意绵绵,然后停止在这一阶段;也有时再向下发展,进入性交过程。事前爱抚为性交的顺利进行提供了生理和心理上的准备和保障,使双方能暂时摆脱与性生活无关的杂念,充分放松,把精力高度集中到自己的躯体感受上来。当然这也包括准备充满温馨情调的环境气氛,如把卧室布置成朦朦胧胧的光照,听听优雅浪漫的轻音乐,读读动人的言情小说,诉说娓娓动听的情话等。性爱包含了性与爱这两个互相融和的成分,情感是一种身心的全面感受,必须在事前爱抚中主动地投入,待性感的紧张程度达到一定水平之后,才能一步步使性反应顺利地发展下去。要完全给予才会有丰富的获取,先予后取的原则在这里是完全适用的。只有在完完全全地沉浸于他或她对自己身体的感受之后才会自然而然地实现,而不可能由一个人有意识地事先安排在某一既定的时候达到性高潮,事前爱抚恰恰是促使夫妻性生活幸福美满的前驱过程。

这是许多咨询信件和电话关心的问题,由于长期以来对性知识的封锁使不少人对此一无所知,一旦了解之后则获益非浅。不过,由于传统性观念的束缚,人们对此还是有些担忧。口—生殖器性交在西文中分为两个名词:一是Cunnilinctus,指以舌舔女子的阴部;二是Fellatio,指以口刺激男子的阴茎。中国旧时的性爱小说中分别称为品玉、吹箫、舌耕之道、樱口之技,这些雅称虽不能作为学术名词,但作为翻译用语却简练别致。口交是人类正常、普遍、健康的性行为方式之一,无可指责,谈不上变态。

附:肛交。通常指男性将阴茎插入性对象的肛门进行性交以获得快感。肛交又称鸡奸或希腊式性交,主要发生在男同性恋之间,也可发生在异性恋之间,并不流行和普遍,肛交过程中容易造成局部组织的裂伤、擦伤,从而引起一系列医学问题。如肛门括约肌失禁、脱肛、肛周肌肉损伤、肛周感染、男女生殖器炎症及性病,尤其艾滋病(因肛交是艾滋病的主要传播途径之一),直接威胁到肛交者的健康。

在性活动中所采取的性交姿势或称性交体位,对性反应的性质和强度有很大的影响,在配偶之间如果忽视了采用最适宜的体位,就常常会造成性调节过程的障碍,事实上并不存在一种能够适合所有配偶的性交体位。因此,在把性生活当作一种科学来探讨时都必须正视这一问题,并给予特殊的考虑。如果夫妇双方的性器官大小不是十分吻合、贴切,那就要采取不同体位来补救并找出最容易达到性高潮的体位。

需要指出的是在特殊情况下的性交体位应该有所选择,对大多数处女来说,初次性交应采用标准的男上位,因为这种体位便于事前爱抚,并容易使阴道产生润滑作用。女方可以尽量分开双腿并使双腿向腹部收回,以促使阴道口尽量扩张,在插入的瞬间女方屏气向下用力,对房事的顺利进行也会有所帮助。

孕期:性交时应避免对腹部的直接重压。女方仰卧时,男方可取蹲位或坐位,或采用后进入式,如侧位后进入式等;或女方仰卧将臀部置于床边,男方站于床边性交。插入应尽量浅些,动作应轻柔,有习惯性流产者应避免性交活动。

腰痛:当男方腰痛时,可坐在椅子上,女方骑跨坐在男方腿上,并由女方抽动;或女方膝胸卧位,男方站位;或女方仰卧置臀于床沿,男方取站位。女方腰痛时,女方仰卧,男方要以双膝双肘负重,避免给女方增加重量负担。男方背痛时也可仰卧,采用女上式。

阴道过紧:采用女上式有助于成功地插入。事先可用手指或不同型号阴道扩张器扩张阴道,然后女方蹲跨在男方躯干上,使阴茎插入即可。

1.马克思,恩格斯.《马克思恩格斯选集》人民出版社,1975(8).

2.达尔文.人类起源与性选择.1871.


\section{第二节 人类性历史的演变}

人类从远古时期到原始社会跨入古人科生物时代,为了生存与发展,开始捕猎和采集食物,并制造各种石器和工具以提高劳动生产效率。但由于狩猎讲究团体内的分工合作,尤其是在捕捉一些大型动物时,需要大伙分工、侦察地形、准备工具、挖掘陷阱和进行围猎,其中每一个步骤都离不开集体成员间的合作。为使狩猎生产正常进行,避免集体成员之间的冲突,保持团结的合作关系就显得至关重要,而原始群体内部冲突的主要原因是男性争夺与女性性交的权利,这种冲突往往造成大量的伤亡。人类的这种习性虽然源自动物,但其所造成的伤害程度却远远超过动物。其一是因为人类是喜群居的物种,群体的范围往往超过其他动物;其二是人类女性发情期的消失,使得人类的性行为可以在任何时候进行;其三是人类会制造和使用工具,因而其杀伤力也远远超过其他动物。以上三个原因便使不受控制的性行为对人类的生存造成了巨大威胁。当狩猎生产成为主要的生产方式时,在其准备阶段,停止群体内部的一切冲突便成为一种迫切需要,而禁止性交就成为制止冲突发生的关键,这种客观上的生存要求被逐渐认识到后,就产生了狩猎生产上的性禁忌。狩猎性禁忌除为维系人类生存起到重要作用,还在社会劳动方面进一步促进了男女的分工,对人类早期社会结构的改变产生了巨大影响,也为以后的社会大分工奠定了基础。在狩猎时期的劳动分工是以性别为划分标志的,男人从事狩猎,女人从事采集,这种分工并非完全是由男女在体力或智力上的差别而致。有关学研究发现,男人在估计远近距离及投掷石器的准确性上优于女性,而女性则在暗处的视觉和听觉上优于男性,这样男女分工,各取所长,也可看出这种男人狩猎、女人采集的社会分工与性禁忌是密切相关的。随着狩猎性禁忌的推行,狩猎队伍便由清一色的男人组成,当这些男人遇上其他群体的女人时,就会发生狂热的性行为,这种性行为是不受性禁忌约束的,也不会受到人们的谴责。这种超范围的性行为却为优化人类遗传基因起到了意想不到的作用,有力地推动了人类素质的优良化。但这种作用当时并没有被人们意识到,也没有把这种方式推广下去,而是逐渐地被发展起来的图腾制度所替代。

当社会发展到一定阶段,人们的思维能力达到一定程度时,对许多事物和现象的认识也会有许多改变,但由于科技文明程度的限制,这些认识也是比较肤浅的。图腾制度就是在这种情况下产生的,它是一种比性禁忌还要严格、周密的群体共同意志的体现。由于原始社团的不断分化,使得群体在数量上增长迅速,也使相互间的矛盾不断升级。彼此间为了争夺女性、猎物以及活动区域等而打斗的事件时常发生,有时甚至酿成严重的流血冲突。因而,对每个群体内部的成员来说,都必须要有共同的追求、信仰,此时群体意识得到了空前地增强,相互间都以动物、植物或无生物为标记,来区分其他群体的成员,这就为图腾制度的产生打下了基础。随着图腾制度的产生与普及,群体与群体之间的界限日益分明,不同群体间男女性接触与性行为在社会规范的压抑下逐渐终止,每个群体在物质资料和人类自身的两种生产上,都形成了自我封闭式的社会有机体,这就迫使男女间的性行为只能在血缘关系很近的亲族间进行,出现了被人类学家称之为“血缘家族”的婚姻形式。然而,这种血缘家族的婚姻形式不久就给人类带来了灾难性的打击,在几乎是完全相同的种族里,不育、早产、痴呆和畸形儿等情况越来越多,出生率、成活率不断下降,人口质量越来越低,逐渐将人类物种的繁衍导向濒临灭绝的边缘。在此情况下,人类面临了十分严峻的局面,不得不重新审视自己的行为,不得不选择有利于群体人丁兴旺的性行为,而那些阻碍群体发展的禁忌则必然被淘汰。因此开始实行图腾群体的外婚制,允许不同图腾群体的男女发生性行为,开创出新的男女性关系,使人类物种得以不断优化,不断绵延。

当人类历史发展到氏族公社时,狩猎业、采集业已有了长足发展,并从中分化出捕鱼业,极大地丰富了人类物质资料的生产内容。同时,由于氏族联姻使人类的遗传基因得到了极大的改善,种内杂交发挥了它应有的作用,将人类的体质形态、智力发育由古人推向了新人(即现代人),最终形成了人类自身生产的社会结构———婚姻形式。此外,由于氏族社会的人们居住地相当固定,这就为曾经发生过性行为的男女提供了相会的地方,再不会像过去那样分手后就很难再见面了,男人们经常会到邻近的氏族公社驻地去会见自己喜欢的女人。这种男女间性行为交往形式上的变化,为氏族婚姻联盟中的男女双方提供了较为充裕的性选择机会,让人类在女性发情期消失后曾萌芽过的爱情重新成长起来,并为从群体婚姻过渡到对偶婚姻打下了基础。群体婚姻是最原始的婚姻形式,即整个一群男子与整个一群女子互为所有,无论有无血缘关系均可互为夫妻,没有乱伦禁忌,只有辈分限制。对偶婚姻是一种由一男一女组成的配偶婚姻制度,可以是一夫多妻制,也可以是一妻多夫制,但无论哪种形式,在许多妻子或丈夫中都有一个主妻或主夫,其特点是仍实行了共产制的家庭经济制度,妇女在家庭中居统治地位。然而对偶婚姻的出现不是一开始就得到社会承认并推广的,而是一种没有任何保障,男女双方可以随意解除的、极不稳定的婚姻形式,在经历了漫长的历史演变后,男女之间的这种性关系才逐渐地由随意转为正式,由秘密转为公开,由临时转为长久,并以这三种转变形式得到了社会的普遍承认、尊重,得到了各种保障,从而也就出现了两性关系的社会组织形式———个体婚姻。在对偶婚姻时期,实行的是氏族公社内部每个成员的劳动产品归集体所有,每个人只能就地消费,分给自己的那份产品,不得带走,也不得占用他人份额的平均分配原则。但是随着个体婚姻的出现,人们对这种分配原则提出了新的要求,他们并不满足于这种共产制的分配方式,而是希望能得到部分多出于自己消费的产品,用来转送给氏族联盟中的异性配偶,以获得对方的欢心,从而确立和稳定双方的个体婚姻关系。当然,这种需求也只有在物质生产发展到一定程度,出现了少量的剩余产品后,即在经济状况所允许的基础上才能得以产生。产品分配原则的改变,最先是在产品生产者身上体现出来的,即只有参加狩猎或采集的劳动者才有权力分享他们的产品,而且他们总希望能分配到更多的剩余产品,以用来在男女之间的性行为和性活动中“慰藉”对方。而出现这种分配方式的最大受害者是那些没有劳动力的成员,尤其是尚待哺育的小孩。而哺育孩子的任务自然落到了母亲的身上,但在生产力十分落后的氏族公社时期,母亲并没有能力一人去哺育一个或多个孩子,于是在血缘纽带上最为接近母亲的同胞兄弟及长大但尚未成年的儿子,便经常去接济他们那些负担过重的母亲或姐妹,母亲或姐妹便给予他们生活上的照顾作为报答。这样久而久之,这种生活上的照顾与经济上的援助,便自然地在氏族公社内部形成了一个个由同母所生的兄弟姐妹及其子女所组成的最小消费单位———血缘亲族。虽然产品分配方式改变了,但成年男女的生活负担却相对加重,但他们仍然会挤出少量的生活消费品来取悦异性,这种在异性交往活动中互赠消费品或纪念品的活动,已经出现了性质上的变化,即渐渐地由过去基本上的等价交换变成了男子对女子经济上的馈赠。这是因为对于成年女子来说,基本上都是哺育有孩子的母亲,因此,她们无论是在经济上还是精力上都要比其他的亲族付出得更多,孩子们的拖累使她们没有更多的能力拿出多余的产品去回报男方,反之她们还要求与自己发生过性行为的配偶与其共同哺育孩子,在这种情况下,逐渐地就产生了个体家族。而个体家族的出现,无疑是给氏族公社内部注入了一道催化剂,促使这种社会结构迅速解体,从而使人类社会得到了长足发展。

个体家庭在其初始阶段曾与血缘亲族并存了一段时间,此时在这类家庭中孩子的供养除靠母亲外,还要靠母亲的同胞兄弟,而孩子的父亲并没有和他们居住在一起,这是母系氏族公社最初的居住形式,也是所谓“只知其母不知其父”的由来。对于人类社会早期来说,女性和男性在生活资料的生产上并没有明显的差别,只存在自然上的分工,男性狩猎、捕鱼、防御野兽入侵,女性则负责采集野果、制作衣食,男女处于原始的平衡状态。随着自然界的变化,女性渐渐占据了生活资料生产上的优势,这是由于男性所从事的渔猎活动存在着相当大的盲目性和偶然性,收获极为有限,不能成为原始群体生活资料的主要来源。相对来讲,女性所从事的采集劳动相对稳定,收获也比较丰富,是氏族成员们生活资料的主要来源途径。生产成果的差异无疑使女性的优势得到了初步确立。随后,经过女性长期的采集劳动,逐渐发现了农作物的生长规律,从而发展了原始农业。这时社会生产力已得到不断发展,同时在地理生存环境的影响下,农业逐渐从采集业中分离出来并迅速成为当时主要的物质生产行业之一,而女性也随之成为了这个生产领域的主要劳动者,这更加巩固了女性在生活资料生产中的优势地位。但因原始、粗放的农业劳动是一项极其繁重的生产,加上供养子女的沉重负担,妇女们早已被压得喘不过气来,特别是对于那些缺少男性(指母亲的同胞兄弟)的亲缘家庭来说,女性更难维持子女和家庭的生活。为了生存和家族的发展,一些女性开始把曾和自己发生过性关系的其他氏族的男子,即她的丈夫拉到自己家族中来一起生活生产,形成了在女方居住的个体家庭,这是一种典型的与母居制度相适应的社会形态,即以女性为中心的母系社会。

社会在不断进步,社会生产力也在不断发展,此时的社会生产领域中又有一种新的因素在萌芽,最终导致了畜牧业和狩猎业的分离。随着农业、畜牧业的进一步发展,所需的劳动强度也越来越大,自然赋予男性的强壮体魄逐渐显示出优势,此时男性逐渐取代了女性,成为社会生产的主力军,占据了人类生活资料生产上的优势地位。男性依靠新取得的经济地位来努力改变自己在社会中的地位,其最突出的做法就是把其他氏族中曾与自己有过性行为的女性拉回到自己本氏族中来共同生活与生产,从而形成了在男方居住的个体家庭。这是母系社会向父系社会的转变,这种制度的彻底改变实质是男女性关系交往向一种更新、更高的婚姻形式的过渡,由此渐渐产生了现代意义上的一夫一妻制婚姻,从而也出现了真正意义上的个体家庭。这种个体家庭的最终出现,完全打破了原来氏族公社之间的界限,氏族公社本身所具有的各种特征也被彻底颠覆,社会生产结构从此掀开了崭新的一页。


\section{第三节 我国性文化}

在充满着各种恐惧和生存威胁的原始社会里,人种的延续是最为宝贵和难得的,对于当时的社会来说,有了人就等于有了一切。先祖们在维持自身基本生存需要的同时,把希望寄托于人的自身生产上。由于被人们的认识所限,他们并不知道男性在生殖行为中的真正意义。在他们看来,生殖是一种神秘的、神圣的事情,而且只有女性才能承担。女性在生活资料生产以及人类再生产中的主导地位,使全社会对其产生了一种普遍的崇拜心理,即女性崇拜心理,此时的女性成为人们崇拜的偶像,就连女性自己也自我崇拜。此外,由于原始时代的人们对自身性欲与性行为的产生机制一无所知,觉得阴茎的勃起和阴户的张弛都不受人的意志所控制,生殖器似乎是个独立于人体之外的怪物,于是产生了对生殖器的崇拜;同时他们不了解性交与生育之间的关系,也没有对性交进行任何限制,只体会到性冲动驱使下的紧张感和伴随性交而产生的快感,以及在性交结束时的放松和倦怠,继而产生了对性交的崇拜;再加上原始人对人类生育毫无所知,认为是神赋予了女性这种魔力,使女性才有生育的本事,于是又形成了生殖崇拜。这三类崇拜就构成了我国古代性崇拜的主要内容。正如我国著名的性社会学家刘达临曾说过:“如果说性交崇拜只是基于男女在性交时所产生的高度快乐,生殖器崇拜只是对性器官的构造与功能不了解而产生的一种愚昧,那么生殖崇拜却涉及原始人的生存、发展和延续的一个实实在在的利益:添人进口才能兴旺发达。”

原始的生殖器崇拜是由于原始人对人类的器官构造和生理机能无法理解,认为长在人身上的生殖器如阴茎和阴户都是不属于人体内部的东西,是决定和控制着人类性行为的东西。人无法抗拒它的魔力,就好像人无法抗拒神的力量一样。所以,人们只要顺从它就能得到性的快乐,如果违抗它就会伤害自己的身体。同时在当时社会还有这样一种看法:认为只有男子与女子性交后,女子才能怀孕,于是断定男子是创造生命的主宰。因为万物的本源是种子,而男子才能提供人类物种的种子并将种子植入女子体内,女子体内则只是这个种子发育的场所,胎儿完全是由父亲的种子所形成的,于是对生殖器加以崇拜。据说人类祖先的“祖”字就蕴含有男性生殖器的象征。另外,在我国广东省韶关市丹霞山至今还保存有完好的、形态逼真的人类生殖器图腾的自然景观———阳元石和阴元洞。性崇拜则主要表现为原始人对在性交达到高潮时所出现的一系列表现与心理变化无法理解,继而认为是神灵所赋予的某种魔力,是神灵赐予他们的最大快乐。这既是原始人强烈的享乐需要,也是他们的审美情感,通过性交这唯一渠道,人就可以和神灵相互沟通。因此形成了性交是他们生活中最重要的乐趣和安慰的观念,故而对性交大肆加以崇拜,对性交活动大肆庆祝。这些可通过一些出土的文物发现,如贺兰山岩画、汉代画像砖女娲与伏羲交媾图等,画中男女不仅裸体形象,其边上画的虎、猴等动物也无一例外地或勃起阳具或作交媾状。而生殖崇拜主要表现为对妇女的崇拜,即对妇女分娩行为的崇拜。由于当时科技文化水平有限,原始人不懂得人类生殖的原理,认为是神的力量使妇女能生出一个小生命,同时由于当时原始人的平均寿命短,婴儿死亡率高,必须以高生育率才能保持种族的生存与发展,加上恶劣的生活环境,需要大量的劳动力,人口多少与体质强弱就决定了该氏族或部落的兴衰。所以他们非常重视妇女的分娩行为,每当这时总要举行一些隆重的祝祷仪式,如果妇女在分娩时死亡,她将被视作部落的英雄而举行隆重的葬礼。这些崇拜现象可以通过先人遗留下来的一些性文物体现出来,如在一些画像中常出现女性生殖器,或通过一些夸张的手法把妇女怀孕的体态表现出来,又比如中国历史上有女娲造人的神话,在法国、奥地利等欧洲国家还出土了许多原始女性偶像,这些偶像的共同特点是:不注意面部的刻画,主要强调肥大的躯干,突出表现的是硕大的乳房、大肚子和生殖器,体现出原始人对生殖的巨大热情。

上述性崇拜现象无一不折射出原始人类质朴的性文化,进而影响到人们对性的观察与思考,当母系社会向父系社会转变后,这些古老的哲学思想又得到进一步完善,于是出现了天人感应论、阴阳论、七损八益等多种学说。

该理论观点主要由汉代儒家董仲舒提出。中国古代哲学认为,人事活动会从“天”得到反映,这种神秘学说的理论基础是“天人合一”。古代认为“天道”和“人道”、“自然”和“人为”是合一的,战国时周公、孟子就提出了这一理论。如从天人关系角度看,周公主要表现为“敬天保民”的德政思想;孟子继承、发展了周公的政治思想,他的“天民合一”是儒家仁政思想和民本论的重要体现;而汉儒董仲舒则强调“天人之际,合而为一”,一方面继承了先秦儒家的人文精神,是仁政思想和民本论的一种新的表现形式,另一方面又带有宗教神学色彩。宋儒程颢则说:“天人本无二,不必有合。”正因为这样,古人认为,天能干预人事,人的行为也能感应上天,自然界的灾异和祥瑞表现了天对人的谴责或嘉奖,在这些感应之中,人类的性行为和天的感应尤其密切。《老子》说:“玄牝之门,是谓天地之根。”这就从女阴的生育功能引申出天地的起源。《易经》中说:“天尊地卑,乾坤定矣。卑高以陈,贵贱位矣。动静有常,刚柔断矣。方以类聚,物以群分,吉凶生矣。在天成象,在地成形,变化见矣。是故刚柔相摩,八卦相荡,鼓之以雷霆,润之以风雨,日月运行,一寒一暑,乾道成男,坤道成女。乾知大始,坤作成物。”意即男与女是宇宙中“乾”与“坤”的缩影,是自然的一对物体,进一步说明了天和人的关系。于是就有了“男女构精,万物化生”(1) “夫乾,其静也专,其动也直,是以大生焉。夫坤,其静也翕,其动也辟,是以广生焉”之说。这些阐述赞美了宇宙生成万物的伟大,把两性的交媾推及天地交合的广阔领域,并把对人类性和生殖行为的赞颂普及到对社会、政治、道德的产生与运动中。《易经》中“云行雨施,品物流形”“天地感而万物化生”“天地不交而万物不兴”也是这个意思,其中的“云雨”、“感”、“交”等都是男女性交的术语。

在卜封中,既卜天,也卜人。例如《左传·昭公元年》中医和解释“蛊”卦的卦象,“蛊”卦是上艮下巽,艮为山,巽为风,“蛊”卦是风吹山木之象;同时,艮为少男,巽为长女,所以“巽”卦又是女惑男之象。在中国古代,如遇旱涝之灾,人们常以祭典或性交舞蹈以避灾,还有把自然灾害看成是旷男怨女多而致天地不顺,所以有大遣宫女的历史记载。

在我国远古时期即有“一阴一阳之谓道”(意为阴阳是世界上万事万物的自然法则)等学说的概括,后在儒家的“阴阳学说”中解释男女之间的关系。中国古人认为:阴阳是宇宙间相反相成的两种根本力量。阴阳最初的含义是指日光的向背,背日为阴,向日为阳,可引申为气候的寒暖。把阴阳引申到两性关系后,认为男女的交合不仅是单纯的欲望发泄,更是阴阳两种宇宙力量在人类身上的具体体现。天地相交而生万物,男女交合而生子女,这样才有了世界的一切。这种思想集中地反映在古代儒家重要性学经典之一《易经》里,该书通过八卦形式(即乾、坤、震、巽、坎、离、艮、兑)代表天、地、雷、风、水、火、山、泽8种物质现象,亦分别寓像人之父母、长男、长女、中男、中女、少男、少女。八卦中的每相连两卦都是对立的,由于阴(—)、阳(—)是八卦的根本,它说明由阴、阳两种气体互相结合交感而产生万物,一阴和一阳间的交互作用叫做“道”,其所产生的生生不息的过程叫做“易”(变化)。《易经》又认为,乾卦和坤卦是其中最基本的两个卦,即所谓“天尊地卑,乾坤定矣”。“乾知大始,坤作成物”。同时,又把乾、坤和男女结合起来,认为“乾坤成男,坤道成女”。中国学者周予同、郭沫若等都考证说,乾、坤二卦的形状恰似男女两性的生殖器,如1928年郭沫若曾说过:“八卦的根柢我们很鲜明地可以看出是古代生殖器崇拜的孑遗。画—以像男根,分=以像女阴,所以由此而演出男女、父母、阴阳、刚柔、天地的观念。”(1) 《易经》把男女两性视为自然的一部分,以男女两性的相交来联系自然,重点阐述自然和人变化(易者,变也)的原理。例如,在自然现象中,月亮与冬天属于阴,太阳与夏天属于阳;对人类来说,女子属于阴,男子属于阳,阴阳需要互补,即所谓“刚柔相摩,屈伸相感而利生”。这都强调了阴阳结合,阴阳互补,男女要“相摩”、“相感”,才能“利生焉”。这种思想广见于古代书籍,如长沙马王堆汉墓出土的简《合阴阳》,更是一本纯粹的性学著作。

以《易经》为代表的阴阳学说,系统地体现了中国古代的生殖文化,并把生殖文化升华到一个新的阶段。《易经》在阐述阴阳变化万物的哲学观念时,也有许多有关性的描写,如“生生之谓易”。这就是说,生生不已的生殖运动是宇宙万物以及人类发展的根本规律,这一重视生命延续的思想对后来的中国文化影响极为深远。

我国古代认为,在男女交合过程中有7种行为方式对健康有害,有8种行为方式对健康有利。这种理论初见于湖北长沙马王堆出土的竹简《天下至道谈》,后世不少性学书籍多有引用,并有所发展。在《天下至道谈》中强调:“气有八益,又有七损。不能用八益、去七损,则行年四十而阴气自半也,五十而起居衰,六十而耳目不聪明,七十下枯上脱,阴气不用,淉泣流出。令之夏壮有道,去七损以振其病,用八益以贰其气,是故老者复壮,壮者不衰。”所谓“八益”:一曰治气,二曰致沫,三曰知时,四曰畜气,五曰和沫,立曰窃气,七曰待高,八曰定倾。这意思是所谓八益,一是调治精气,二是吞下津液,三是知道交合的最佳时机,四是蓄养精气,五是调和阴液,六是聚积精气,七是保持满盈,八是防止阳痿以及实施的具体办法。所谓“七损”:为之而疾痛,曰内闭;为之出汗,曰外泄;为之不已,曰竭;臻欲之而不能,曰 ;为之喘息中乱,曰烦;弗欲强之,曰绝;为之臻疾,曰费;此谓七损。故善用八益,去七损,耳目聪明,身体轻利,阴气益强,延年益寿,居处乐长。上述意思即性交时阴茎疼痛,叫内闭;性交时出汗多,叫走泄精气;房事没有节制,叫精液耗竭;到了想性交时却不能,叫阳痿;性交时喘息并心烦意乱,叫烦;女方无性交要求时而男方勉强她,对女方的身心健康有害,叫绝;性交过于急速图快,叫浪费精力。以上就是七损。善于用八益而除七损的人会耳聪目明,身体灵活轻便,心理功能日益增强,就能延年益寿,生活快乐长久。在这部学说中还论及了男女性生理、性心理和性养生之道的具体原则和措施。故而我国著名医学专家吴阶平教授认为它是中国古代性医学即房中术的雏形,代表了早期性医学实践,并已形成了一套集生殖、养生和保健为一体的综合理论。

当历史发展到奴隶社会时,社会生产力和科学技术已发展到一定水平,人类的文明程度也得到一定提高,其标志就是在这一时期逐步建立起了人类文明之初的性道德,反过来又时时作用于社会,促使社会的不断变革。最明显的就是男女婚姻在形态上的变化,促使个体家庭冲破了对偶家族的羁绊,成为当时社会的主流形式,使原本就根基不牢的对偶婚姻让位给一夫一妻制。这种新型婚姻的出现标志着文明社会的开始,它最先摒弃了原始社会的群婚制,排除了血缘婚姻,继而演化成了从伦理和生理两方面考虑的“同姓不结婚”的性道德原则,使人类物种在变革的婚姻形态中得到不断优化,人口素质得到不断提高。但由于受父系社会和男权制的影响,性道德的内容渐渐地发生了变化,演变成压迫、屈辱女性的一种工具,使之成为男性奴役女性的开始。男子凭借其在社会上的优势地位,在一夫一妻制的形成过程中,逐渐将女子的财产继承权和世系制转变为男子,也就是说这时的世系和继承权全归男子所有,女子的权力丧失,地位下降。同时男子为了确保能把自己的财产传给具有他的遗传基因的子女,开始要求妻子必须严守贞操,再加上血缘婚、族外婚、对偶婚时代的那种性自由已被家庭、被专偶制的夫妻关系所取代,这时男女之间的性行为亦不再是一种公开的行为,而是个人家庭中的私事,旁人无权过问,也不允许向旁人公开,更不允许和丈夫以外的人进行性活动,于是性崇拜现象消失,取而代之的是性神秘、性禁锢。而且这种一夫一妻制婚姻并不是对等的,妻子只有一个丈夫,而丈夫却可以拥有多个妻子。对于社会的强者即氏族酋长、奴隶主或其他有身份地位的人来说,主要是通过掠夺或霸占来获得妻妾,而对于平民百姓来说,则是通过到集市上去买回女奴作为自己的妻子。恩格斯在《家庭、私有制和国家的起源》一书中说道:“从群婚到对偶婚到一夫一妻制,这种性关系的逐步纯洁,是以剥夺妇女的性自由为代价的。”由于君权、父权和夫权制度的影响,这时的一夫一妻制并不是真正意义上的一夫一妻,而是一夫多妻,是以牺牲妇女的人格权力为代价的。同时女性的私有化使人们的伦理观念也随之发生了变化,最初那种允许别人与自己一起来共享从妻子那里牟欢乐和满足的现象亦不复存在,就是妻子在外人面前裸露身体也是绝对不能容忍的,这就使得过去那种带有自我炫耀的性器官崇拜蒙上了羞耻的社会情感,即产生了人类最初的裸露羞耻感。在造型艺术和文学描写上,人们也越来越多地把对女性的描述作为其审美的标准,将女性及其裸体形象暴露在一个被人“欣赏”的位置上,这充分体现了在阶级社会形成初期,男性对女性的奴役、压迫和性偏见。

通过封建社会对性文化的传承与发展,人们的性伦理道德观念得到不断进步和完善,但另一方面,随着封建社会经济的兴盛,封建礼教的逐步形成,人们在性观念上出现了急剧的变化。对于统治阶级而言,物质经济发展产生的腐蚀性使其在生活上变得荒淫糜烂,封建礼教的宽容又使其对女性的压迫更加放纵无度。从奴隶社会中解放出来的奴隶,许多都迅速上升为自耕小农,其经济地位上的变化,使得以男子为中心的社会地位更加巩固。而只有女性在这一社会性质的变革中沉入社会的最底层,仍然承受着精神和物质上的双重压迫。于是产生了影响我国社会将近2000年之久的“三从四德”、“三纲五常”、“男尊女卑”等封建礼教,其代表人物就是董仲舒,他的这一套学说成为当时封建社会普遍遵守的道德规范,也成了男性用来压迫和桎酷女性的有力工具。这一封建礼教的内容确立后,统治者便开始千方百计地引导人们去遵守,除了效仿前代刻碑立传以倡导“妇女贞操”,还采取褒奖为恩、酷刑为威、恩威并施的手段以加强效果。所以自西汉以来,有关婚姻的法令日渐增多,对于夫妻双方在家庭中的地位,汉律中亦作了进一步的规定:男子可以随意打骂妻子,而妻子只有“敬顺”的义务;男以强为贵,女以弱为美;丈夫可以休妻再娶,而女子却不能休夫再嫁,如有违反者当以酌情处以刑律。在施以刑威的同时,还大肆宣扬妇女守节,并以皇帝诏赐“贞妇顺女帛”的形式褒奖妇女的守节行为。在此影响之下,崇尚妇女贞节的性文化氛围在当时社会中越来越浓厚,一些文人墨客也开始围绕妇女守节问题著书立说,如从刘向的《烈女传》到班昭的《女诫》七篇等,无不再三强调封建礼教,崇尚贞节烈女,鼓吹男尊女卑,这些都是以限制妇女自由,压迫妇女地位为代价的,是对女性人权的剥夺,对女性人格的污辱。除此之外,封建统治者还通过政治与经济上的强制手段,把一部分妇女转化为姬妾、娼妓而供淫乐,如由于秦汉统治者荒淫生活而导致后宫姬妾成群的空前盛况;上层社会狎妓风行使城市娼妓日增;官府“三十倍于古”的苛税重赋使民间出现了溺婴和拒不生子的现象等。在当时封建礼教的影响下,社会的种种性文化都呈现为沉疴病态。

当发展到唐朝时期,性文化出现了空前繁荣的景象,不仅女教学说得到进一步发展,官办娼妓业不断兴盛,而且婚姻结合中门第观念盛行,言情文学泛滥等。门第观念的形成受政治势力的支配,是一些统治阶级和达官贵人为结成庞大的势力集团而采取的联姻方式,是血统、财产、地位、声望等方面的重新组合在性文化上的具体表现形式。大姓世族在性结合中注重门第观念的风气,十分自然地影响到下层社会,人们都用势利的眼光去寻找婚配的对象,或是攀龙附凤以图跻身高门,或是多索财物企盼丰富自家,使得贫穷家庭的女子出嫁难成为当时社会的一大顽疾。这种观念自出现后,就一直影响着中国社会,直至今天人们仍然没有完全从这一性文化氛围中脱离出来。除了上述门第观念的盛行,在唐朝娼妓业也异常兴盛,那时在皇室中有“宫妓”,在达官贵人府有“家妓”,在军旅驻地有“营妓”,城市会有“官妓”,茶楼酒肆有“歌妓”以及专司陪酒的“饮妓”等许多种类。当朝对各级官吏没有“宿娼”禁令,因此上至朝中权臣新贵,下至城乡富商大贾、文人墨客都把狎妓冶游作为一种时尚、一种风流雅趣。同时朝廷为了便于对娼妓的管理,甚至对其造册,并制定了各种规章予以束缚,逐渐形成了一套娼妓管理制度。而对于较高级妓院里的妓女来说,必须要精通音乐、歌舞、诗词、绘画等多方面的技艺与知识,正是由于妓女们的能诗善赋,诗人们便经常到妓院去饮美酒,食佳肴,与妓女们争诗斗赋,形成了唐代诗人的狎妓风气。诗人与妓女之间或出于逢场作戏,或出于真情外露,出现了一大批赞美妓女与男女爱慕的言情作品,同时也通过这些作品揭露了女性被压迫、蹂躏、抛弃的种种处境。但唐代诗人狎妓的社会风气和妓女们能歌善舞的技艺,却对唐朝乃至我国文学发展产生了重大影响。

关于我国古代妇女缠足制度的起源,归纳起来大致有南北朝、唐代和五代这三种说法,对起源于五代这种说法有史书为证,比较可信。从一些史书和诗集的记载中可以看出,在五代以前女子虽不缠足,但人们已经开始崇尚女子小脚,认为女子以娇弱、步履迟缓为楚贵动人,而且此风在文人墨客的描写下愈演愈烈。于是到了五代就逐渐形成了人为的缠足现象,在此以后的千余年里缠足成了女性优雅高贵的象征,除了偏远地区或少数需干粗活的劳动妇女,几乎所有的女子都难逃缠足的厄运。这是一种强迫女性肢体畸形发育的残酷行为,对女性来说是一个极其痛苦的过程。女性要将双脚缠到“小瘦尖弯香软正”才符合男人的欣赏标准,俗话说:“小脚一双,眼泪一缸”,其痛苦的滋味可见一斑。对于父母来说,为了让女儿长大后能够找到一个好婆家,也只好狠下心来把女儿一双健康的脚裹得尽量小;对于女子本身来说,由于身处以男子为中心的封建社会,女性一直被统治阶级视为玩物,她们本身的抵制对当时社会来说是微乎其微的,因而只有无声息地忍受,这便促进了男子们病态审判观的发展;同时,女子为了奢觅一绅士之家作为自己的归宿,也主动地去承受其肉体上的痛苦以追赶时髦。据日本性学家研究认为:“缠足的妇女,为了能够好好地站立行走,两腿及骨盆肌肉需要经常绷紧。这种摧残女性的恶习,从更深层、更隐晦的意义上,是为了满足男人的性快感。”③

到了两宋时期我国性文化进入一个相对宽泛的局面,开始容许妇女再嫁和离异。以李觏的“人性三品论”为代表的封建伦理纲常体系,承认“利欲之求”是人之本性,把传统中“礼”这一抽象化的理论教条通俗化,归还到现实生活中,对社会产生了很大影响。然后出现了王安石的“性情论”,他认为人性是指人的心理活动能力,是出于自然的“天资之材”,这种“生之谓性”的天生资质是性之固有的,人人相同。在他的这些理论中蕴涵着唯物主义的因素,因而他能够把其学术思想较为自然地放到现实生活中去,对贞操观念的理解和对涉及此类的事物都能较为合理地去处理。但此后一些文人学士又开始在男尊女卑上大做文章,使性文化氛围又比先前的宽泛局面严格了许多。在宋朝对我国性文化影响最深的当数程宋理学,他们主张无欲、禁欲,提倡“饿死事小、失节事大”,宁可饿死,也不能作出妇女改嫁、再嫁这类失节之事,但对男性来讲则允许休妻,允许其在妻死后再娶的双重道德标准。在朱熹“存天理,灭人欲”的理学思想影响下,我国女性进入了一个思想禁锢、行为束缚、宗教神化、礼教吃人的羁縻桎梏日益严重的性文化时代。在妇女贞操观念的作祟下,人们将能够证明女子贞操的唯一标志———处女去与男子婚配,以免遭社会的非议。再加上文人墨客或公开或隐讳地描述男子对处女的喜好,以及道家宣扬的“采阴补阳”的性医学理论,都不同程度地改变了人们的性心理,在人们的价值观上肯定了离异、丧偶妇女与处女之间的差异,诱导或加剧了人们的偏态陋习,久而久之就形成了整个男子社会偏执追求处女的嗜好。这样一来对女子贞操的要求也更加严格,玩弄女性和封建礼教并行发展,把女性置入了一个被玩弄后又遭遗弃的悲惨境地。

到了封建社会后期明清时代,性文化经历了各方面的冲击,处于一个保守与变革、桎梏与解放的曲折环境中。这一时代是我国历史上娼妓盛行、宦官横行、女子贞节观念宗教化及狎玩名伶戏子等性文化的畸形发展时期。虽然世界其他国家也有宦官这种人为创造出来的第三性存在,但却没有赋予他们类似于我国宦官那么多的性寓意。最初被逼作宦官的大多是在战争中被抓的俘虏,统治者为了灭其种族,即对其施以宫刑。我国宦官制度虽然形成很早,但“自宫”成为潮流却是源起于宋代。有些人羡慕宦官的显贵跋扈,而有些人则是受生活所迫,主动到兵部报道审核后,择吉日阉割,以致到后来宦官人数越来越多,据推测最多时达12 000~13 000人。④ 满民族在未入关之前是没有宦官的,建立清朝政府后,在汉文化的影响下仿效明制,设立了宦官,但在数量和规模上有所减少。到慈禧太后听政后,宦官又渐渐地兴盛起来,被阉割后的宦官失去了种种男子特征,失去了常人的信念与道德,但其性心理活动却无法因阉割而消失。在“心理补偿”的作用下,或为泄欲,或为追求心理满足,或是想以此来恢复性机能,他们娶妻纳妾、对性的追求甚至盛于常人,其性行为方式有同性恋和异性恋两种,从中表现出种种畸形的宦官性文化。此外,对处女的嗜好和对小脚的崇拜在这个时期更是得到了空前发展,男子们除在心理与兴趣上注重女性贞操,在其内容上更是出现了处女的检查方法。到清中期,女性贞节观念的宗教化发展到了巅峰阶段,妇女们将贞操名节奉为第一生命,夫亡殉夫、未嫁夫亡者亦须尽节,似乎只有一死才能明节,否则被视为偷生。在封建礼教的长期熏陶下,广大妇女在心灵上受到了无法估量的伤害,严重扭曲的文化氛围,造成了她们性格和心理上的残缺,以至于她们只有逆来顺受,默默地承受着肉体和精神上的双重折磨。到了晚清时期,在“西学东渐”维新变法形势的推动下,妇女问题被维新志士作为一个社会问题而提上议程。康有为、谭嗣同等把妇女解放和挽救国家危亡结合在一起,作为变法运动中组成部分,不仅在理论上对封建礼教予以批判,而且在行动上也作出了一些有益于妇女解放的工作,如组织女子不缠足运动,批判“女子无才便是德”的观点等。在他们的努力下,女子受教育的机会渐渐增多,从此一批新女性出现在中国大地上,并继续为妇女解放、为争取自由平等而不断努力。

新中国成立以后,在共产党的领导下,开展了一系列性革命,颁布了《婚姻法》:反对封建礼教,提倡婚姻自由;反对包办和买卖婚姻;主张男女平等和夫妻平等;严禁纳妾和卖淫嫖娼,提高了妇女在政治、经济上的地位,逐渐形成了一套新的社会道德规范。但是,性革命的道路并不是一帆风顺的,在“左”思想的影响下,被倡导的性自由、性平等又重新被否定,领导者大肆宣扬艰苦奋斗的精神,禁止追求性快乐和性满足,又重新回到了另一种方式的性封闭、性禁锢上,走上了程朱理学的“存天理,灭人欲”的老路,形成了新的禁欲主义。那时的人们忌讳一切与性有关的东西,如画、文字、符号等,甚至到了谈“性”色变的程度。到了20世纪后期,改革开放给各行各业带来了新的生机,思想也得到空前解放,开始较多地接触西方性文化,同时随着经济条件的改善,人们开始思考如何幸福快乐地生活,如何提高生活质量,包括性生活质量。这使千百年来的性禁锢终于被打破了,性不再是一件隐秘和肮脏的事情。但另一方面西方的“性解放”、“性开放”思想也严重腐蚀着一些人的头脑,过去较少出现的性问题逐渐暴露出来,如青少年的性早熟与早恋问题、婚前或婚外性行为、夫妻关系不睦、离婚、卖淫、性变态甚至性犯罪等,使性的自然性压倒其社会性,社会上出现不少性紊乱现象,为此我国政府特在《中华人民共和国刑法》中对性犯罪作出了明确规定,以法律的形式来进一步保护妇女的合法权益,维护社会的安定团结,为创造和谐社会打下良好的基础。

1.刘达临.世界古代性文化.上海:上海三联书店出版社,1998年.

2.彭晓辉.性科学概论,北京:科学出版社,2002:9.

3.郭沫若.郭沫若全集·历史篇.第1卷.北京:人民出版社,1982:33.

4.〔美〕蕾伊·唐娜希尔.人类性爱史.北京:中国文联出版公司,1988:103.

5.〔台〕施克宽.中国宦官秘史.台湾:宝文堂书店,1988:20.


\section{第四节 国外性文化}

在人类的起源、进化等问题上,世界各国的情况都基本相同,国外同样经历了图腾制、图腾外婚制、氏族公社等种内杂交,从群婚到对偶婚,再到个体婚姻的演变。但自国家成立后,各国因其社会背景、经济发展、风俗人情的不同而有所差异。

公元前8世纪~公元5世纪是古希腊奴隶制城邦从建立、发展到衰败、崩溃的时期,但这一时期古希腊人在政治、经济、文化思想领域的发展,在文学、艺术和哲学等方面的成就却为西方近代文明奠定了良好的基础,当然,在这一时期产生的独特的性生活方式和性观念体系,也成为现代西方人在这方面的重要传承来源。自古以来,希腊人都崇尚和追求美,他们认为自然界创造的一切都是美丽的,而人间最美的应属人体,只有人的健美裸体才兼具匀称、和谐、庄重和优美等特色,所以他们允许在公开场合展示自己矫健的裸体,把这一举动看做是对神的崇敬。因此在古代奥林匹克运动会上“裸体运动”曾风靡一时,不但青年男子在参加角斗、竞走或其他比赛时赤裸全身,在斯巴达甚至青年女子在参加运动会时也几乎全裸。在一些盛大的节日庆典上自然也少不了裸体男女,他们把这当做炫耀自己健美身躯的好时机。这种性观念和性道德的大环境,为学者和艺术家提供了研究题材和创作源泉,一些学者专门研究人体姿态和动作,创立了“奥盖斯底克”学说,美术家们则创造出了“断臂维纳斯”、“掷铁饼者”和“胜利女神”等不朽之作。对于性生活,古希腊人也是持非常开明的态度,他们把性活动从神学观念中解放出来,倡导一种轻松愉快,能给人带来肉体和精神享受的性观念和性行为模式。在他们看来,性不应该与堕落、野蛮、轻浮和玩弄感情等画上等号,应该是像做游戏或运动一样平常。一些希腊神话也对这方面进行了描述,既有性崇拜,又有自恋、同性恋、父恋、母恋等情节,并创造出“雌雄同体”、“色情狂”、“性欲亢进”、“鸡奸”、“兽奸”、“阴茎异常勃起”等性学专用语。虽然古希腊对性事看得轻松、开明,但法律和公众舆论都严厉禁止和强烈谴责强奸、性攻击、通奸、性虐待、露阴癖和奸淫幼女等行为。

古罗马人对性的态度和方式与古希腊人恰好相反,他们普遍缺乏抽象思维和审美情趣,只重视人的生理意义。由于长期的战争生活使罗马人的性行为、性生活比较集中单一、粗俗残暴,他们的处女情结非常严重。根据罗马帝国法律:任何处女无论是平民还是奴隶,是基督教还是异教徒,都不得处死,若非要处死则必须先强奸后再执行。按罗马的风俗,新娘在婚礼上必须接受来客的仪式性奸污,将此视为把处女膜献给生殖神所做的牺牲,而沾染了“处女红”的布则要公开展示,以作为新娘童贞的证物。另一方面,罗马人对性采取放纵的态度,他们热衷于性享乐,不仅通过一些文艺作品表现出来,而且无论在什么场所,只要是他们认为合适的地方都可以约会甚至进行性交。比如古罗马时期的瓶画、雕塑、诗歌和戏剧等都有以男女性爱为主题的,其刻画已不再停留在一般性爱的表现上,而是更注重揭示在性活动中人的心理感受,并达到了相当高的水平;而文人学者则在其作品中大量描写、研究“希腊式做爱”和充满杂技动作的性交活动。此外,在古罗马格斗场中除表演罗马人非常喜欢的奴隶间血腥格斗外,也常公开表演性交和性虐待,他们对性的追求完全是为了满足感官上的刺激,他们允许同性恋,这一切造成人口急剧减少,成为罗马帝国衰亡的原因之一。

中世纪是指从罗马帝国衰亡到15世纪欧洲文艺复兴时期。由于人们把罗马帝国衰亡的原因归结为纵欲、滥欲,加上经济衰退、生产力落后,人们又重新回到愚昧、野蛮的时期,在此情形下罗马天主教和基督教才得以盛行,其信奉的教条才能成为社会的道德行为准则。基督教推崇禁欲主义和贞洁观,反对性放纵,提倡性道德,禁止婚外性行为和一切不利于生殖的性行为。教会虽然不否定性是人类的本能,也不否定其价值,但却反对性享乐,反对表达激情与快乐的性行为,认为即使性交也完全是为了生育目的而进行的,而由此产生的快意是应该遭到人们鄙视的,并倡导采用“男上女下”的性交姿势,认为这是唯一正确、标准的姿势,而“后入式”性交则被认为是“恶魔式”,应遭唾弃。在这个世纪妇女普遍处于生活的最底层,备受轻视,娶妻的目的就是生育,没有任何其他用处,所以贵族们常把下层妇女当做奴隶和性工具使用。妇女得不到法律和社会舆论的保护,她们被随意地残酷对待,任意奸淫而不会受到处罚,甚至被杀死也无人问津。而为了防止妻子和别的男人性交,中世纪的欧洲还发明了能锁住妇女生殖器的贞操带,以表示妻子是丈夫的私有财产。虽然这一时期教会主张禁欲,但他们只禁止做,而不禁止说,不忌讳更不禁止对性器官和性行为的描写,在《圣经》的边页上或唱诗班的坐椅上以及在教堂的建筑装饰上都可见性的艺术作品。他们认为生殖器可以驱邪,而且教士们还担负有宣传和指导“男上位”(后又称“传教士位”)这种性交方式的任务,所以他们信奉“上帝不羞于创造,我们不羞于宣讲”的信条。对于教士们来说,其他人应该禁欲,而他们可以不禁欲甚至还可以纵欲,他们可以吸收大量的罗马妓女入教,为其服务,由此可见其道德的虚伪性。

14~16世纪,西方世界经历了一场巨大的变革,在学术研究上出现了以人为研究对象的人文主义,一些大胆的学者开始摆脱中世纪教会的统治和禁欲主义的桎梏,敢于怀疑千古不变的教条。人类的自然属性被赋予了新的意义,人们能够把自己作为一个个体看待,把自身拥有的感情、愿望、愉悦和冲动看成是自然本性,倡导世俗化和个性的解放。至此,性禁区被打破,在文学和艺术创作中有关性的主题开始重新出现,在日常生活中性的限制得到放松,人们可以公开追求性快乐,并赋予它理想的色彩。但由于此时正处于新旧两种思想、观念交替的时期,有的人口头上喊禁欲,背地里却在做着纵欲的事情,致使社会风气一片混乱,通奸、乱交、同性恋、强奸等行为司空见惯,“梅毒”等性病在欧洲大肆流行;生活中人人以享乐为人生目的,比如在女性服装方面,当时流行低领、袒露双乳,以增加女性的性感与时尚感。另外,在欧洲文艺复兴时期还有一大变化就是色情艺术空前繁荣,这使古典时期追求感官美的艺术趣味性复活,也使对性表达的坦率性风尚出现。在当时最具影响的性文艺作品有薄伽丘的《十日谈》、波提切利的《维纳斯的诞生》、米开朗基诺的《大卫》、达·芬奇的《丽达与鹅》以及由朱里奥·罗马诺、马尔康托尼奥·雷蒙和阿雷提诺合作创作的《性交大全》等,这些作品深深影响了以后西方世界色情艺术的发展。

19世纪中叶英国女王维多利亚统治了英国,并迅速向外扩张,建立了庞大的殖民地,号称“日不落帝国”。此时人们奉行维多利亚主义,在性道德方面否定性冲动,强调家庭是神圣不可侵犯的,认为妻子最重要的责任就是侍候丈夫。重视婚姻中的社会和经济因素,而爱情则屈居第二位,教育并鼓励男人在功成名就后再考虑婚事,而理想的妻子应该是纯洁、娇柔、可爱但没有性欲的。当时人们普遍认为男性只有在青春期后才会出现自发的性冲动;女性的性欲是潜伏的、被动的,只有通过外来的爱抚才会被激发,如果妇女不受到外来性的引诱,就可以在没有性欲的情况下度过一生;儿童则更是纯真、善良、没有性欲的,只有在受到外界的不良影响下才会失去他们天生纯洁的品质。在这种信念的指导下,社会上便出现了许多保护妇女和儿童的规范,规定尽量避免在书籍和公共场合提到性以及与性有关的问题,因此有人把这一时期称为“伟大的婴儿时代”或“无性文化时代”。当时的雕塑或美术作品中妇女有手无腿,有胸无乳,而文学作品中不得有“子宫”、“乳头”等字眼,凡是涉及性的内容一律删去。不但文学作品要受到严格的审查,即使是科学书籍或杂志上如有性的内容,同样被认为是宣传下流、腐化思想而遭到取缔或查封,如1899年霭理士写的《性心理研究》被英国当局查禁,并以“假借科学之名贩卖淫秽出版物”的罪名送到法庭审判。尽管维多利亚政府为对性严加控制,采取了一系列性禁锢政策,但并未真正消除不正当的男女关系,致使整个欧洲娼妓盛行、私生子剧增,无不暴露出性道德的虚伪性。

女权运动是指推翻父权统治,为妇女争取平等权利,使妇女具有和男性同等地位的社会运动,又称“妇女解放运动”,它的出现标志着妇女自我意识的觉醒。现代女权运动包括两个阶段:第一阶段在20世纪60年代初期,其主要目标是争取妇女的选举权,以取得在政治上的平等,后因相应权力已经争取到,加上第一次世界大战后的经济萧条、法西斯主义反女权主义的影响而渐入低潮。到20世纪60年代中后期,美国女权主义活动家贝蒂·弗里丹(Betty Friedan)发表了名为《女性的奥秘》一书,此书的出版标志着第二阶段女权运动的开始,其主要强调进一步争取妇女解放,争论的重点由注重妇女的权益转向注重妇女的“经历”,以及男女在性别上的差异,并带有强烈的政治和意识形态色彩。因为第一阶段女权运动跌入低潮,“妇女返回家庭”的呼声日益高涨,专职的贤妻良母又一次成为了妇女崇拜的偶像。而陷入家庭生活中的妇女们对现实生活普遍感到失望,她们希望女性能够真正得到在法律、经济、工作、婚姻和性等方面的平等与独立。在性爱方面,女权主义者认为性爱对女人来说应该是一种享受,而不应受到鄙视,女人的肉体是属于她本身的,应当由她自己来安排和控制,包括生育、避孕、流产等一切方面。在女权运动的推动下,传统泾渭分明的“男性气质”和“女性气质”开始变得模糊起来。

20世纪60年代到70年代,美国经历了一场迅猛的性革命,导致人们在性行为和性观念方面发生了巨大的改变,这场性革命对当今美国以致整个欧洲的性态度都产生了极其重大的影响。此次性革命最初是针对性压抑和性禁忌而兴起的,由于战争导致了夫妻分离、心理焦虑和渴望寻求感情慰藉,还有战争后经济的崛起,妇女大量参加社会劳动,妇女解放运动发展的心脏—避孕药的发明和普遍使用,使生育和性分离等原因,使妇女婚内和婚外的性行为发生了剧烈变化。他们在性方面极其直率,兴趣浓厚并较为满足,社会对妇女有权获得性满足的观点也无异议,导致结婚率下降,离婚率上升,各种婚姻形式和非婚性关系并存。同时也愿意接纳同性恋、提倡性自由,致社会性泛滥,少女孕妇的数量急剧增加,这也促使包括避孕、绝育和人工流产在内的生育技术得到迅速发展。性革命给这些国家带来的如性的商品化、色情泛滥、家庭解体、性犯罪居高不下、性病重新大流行,艾滋病被发现并流行等社会问题,促使美国人重新反思性革命的得失,总结经验教训,重新审视和确立爱情、婚姻、家庭的价值和地位。

(陈乐)

1.潘绥铭.性的社会史.河南:河南人民出版社,1998

2.樊民胜.性学辞典.上海:上海辞书出版社,1998

3.刘达临.20世纪中国性文化.上海:上海三联书店,2000

4.彭晓辉.性科学概论.北京:科学出版社,2002

5.石方.中国性文化史.黑龙江:黑龙江人民出版社,1993

6.董珊,王进鑫.大学生性健康教程.四川:四川科学技术出版社,2004

7.中国性科学百科全书编委会.中国性科学百科全书.北京:中国大百科全书出版社,2000

8.刘达临.20世纪世界性文化.上海:上海三联书店,2000


\section{第五节 性医学发展史}

在19世纪末之前,世界各国和民族都没有形成独立的性医学体系。

从猿到人的漫长进化过程中,人类的性系统和性功能在逐渐发生着变化,这主要表现为性交的面对面方式、动情期的消失和人类所特有的性活动方式。灵长类之前的所有动物和非灵长类都采取后进入性交方式,它阻碍了两性的亲昵行为,并削弱了雌性的性高潮。当人类能直立行走,把双手解放出来之后,便逐渐步入面对面的性交方式,加上人类大脑的日益发达,人们的性活动方式和追求也日新月异。

原始女性发情期的消失促进了家庭的形成。原始人类对性既敬又畏,产生了神话和神秘主义的性崇拜(性器官崇拜、性交崇拜和生殖崇拜)与性禁忌。农业社会的四条基本性道德是:由婚姻制造的一切性活动;生殖成为性的首要甚至唯一的目的;男性主宰性活动;由财产和地位决定性的特权大小。由此产生了“精液宝贵论”,反对性快乐的各种禁欲主义,男女性心理的差异。在农业社会中,为保护财产式婚姻形成性羞耻和性嫉妒,产生性信息扭曲与封闭;为保护人口增长和社会秩序,形成对非生殖目的性行为和性变异行为的制裁,产生对性行为方式的硬性规定。

生物医学科学一直是性学发展的核心。医学与文明一样古老,西方医哲希波克拉底(460—373BC)和亚里士多德(384—322BC)均在不同程度上对性医学的发展起到一定的作用。前者提出的“行为-体液”概念可以说是现代“性行为-激素”概念的前身;后者提出的“天人合一”则体现了自然医学模式的核心,由此而发源了现代的“生物—心理—社会医学模式”。

进入文艺复兴时期后,对解剖学的研究进入性器官领域,伟大的艺术家达·芬奇(1452—1518)留下了不朽之作,虽然其中不乏谬误之处,但它毕竟是人类对性器官和性交进行相对准确的解剖学描述的先祖。

在哥伦布发现新大陆后,梅毒由拉丁美洲传到欧洲,亚麻制的阴茎套于1564年问世。据说这是英王查尔斯二世的御医康德姆(Condom)首先发明的,以后人们便把阴茎套称为Condom。那时的阴茎套并不是为了避孕,而是为了防止梅毒传染,因为列文·胡克发明显微镜并用显微镜看到人和动物的精子是一个世纪以后的事了(1677)。在1843—1844年间硫化橡胶问世,阴茎套的使用目的才转变为今天的避孕,这也是性医学早期发展中的一个小插曲。

今天的性学研究可以追溯到启蒙运动时期的思想家们,如卢梭(1712—1778)等人博大的思考。他们在通信中谈到性关系及其恰当的社会地位问题,但这只能算是对人类本性及行为的诸多探索中的一小部分而已。如卢梭说:“人的本性是好的,是因为我们的制度使人变坏了。”马尔萨斯在《人口理论分析》(1798)中提出一个令人震惊的理论,提醒人们注意控制人口的增长,世界将负担不了过多的人口。他提倡人到30岁再结婚,这样可以少生孩子,他提倡的实际是“道德节欲”。当时的社会鼓励男子在功成名就时再考虑婚事,一个男子过了30岁才得到性满足,这“象征”着这个男子具有令人称道的气概。达尔文的《物种起源》(1889)发表后,其进化理论不仅使人们对肉体形式的变化加深了理解,也使人们对行为形式的进化加深了理解。

在18~19世纪,人们对手淫的谴责达到登峰造极的地步,认为性能量的不恰当释放会造成神经功能的紊乱。于是社会滋长了对性的种种神秘感、内疚感、罪恶感和肮脏感;性愚昧、性禁锢的影响既广泛又深远,流毒至今不能肃清。总之,在19世纪之前,各种文明和民族都没有形成严格独立的性学,人们对性问题的探讨多停留在价值观的哲学和伦理学范围,而性医学科学的发展则仅仅局限于生殖医学范畴。民间的性技巧和药物也是三分医术加七分巫术的混合体,缺乏独立的分析与验证。

性学囊括了性研究、性教育和性治疗的广博知识领域,跨度很大。性学的三大支柱是性医学、性心理学和性社会学,不过现代性学的创始期却是以性心理学的建立和发展为代表的。有人把1844年德国医生卡安出版的第一部性学著作《性心理病》看做是近代性学开始创立的萌芽,它意味着性学从生殖医学向性心理学的过渡。当时达尔文的进化理论、马尔萨斯的人口论及性病治疗学都已出现,尤其是心理学已经创始,叔本华和尼采的哲学为性研究呼吁,性学才得以破土而出。但初创期的学者仍带着禁欲主义的愚昧观点,以“反性主义道德观”看待性问题,存在着种种谬见。

国际上普遍认为克拉夫特·埃宾(1840—1902)于1866年出版的《性心理病》一书是现代性学的奠基之作。他是一位出生于德国的奥地利精神病学家,一生著作颇多,但《性心理病》无疑是影响最深远的一本。虽然受当时舆论的影响和限制,书的许多段落以拉丁文写成,但却仍然广为传播,大受欢迎,在作者逝世之前便已增订出版到第12版。克拉夫特·埃宾在《性心理病》一书中概括了早期医学,尤其是精神病学对性的研究,第一次把性疾患独立出来讨论。由于当时的性生理学和性医学还没有巨大突破,当时的社会急需辨别种种性心理现象是否属于道德败坏或犯罪行为,因此性学主要集中于性心理方面,尤其是变态性心理和心理病理学。又由于他是警方雇佣的医生,在书中选择的往往是一些极端病例,带有恐怖色彩,读起来令人毛骨悚然。读者可能由此得出他所描述的每个人都受着性困扰的结论,认为性行为是令人可怕的。虽然他是备受尊重的学者,但因涉及禁区,仍难逃脱种种批评和指责。受时代的局限,他对性问题持怀疑和忧虑的态度。他还支持性道德的双重标准,既强调对女性贞洁的严格要求,同时又容忍男子的不轨行为。这种态度和观念曾影响了几代医生,也给众多的读者造成困扰。

作为一门独立学科,性学的建立与当时三位德国犹太医生的努力是分不开的,他们是被誉为“性科学的爱因斯坦”的赫希菲尔德、摩尔,以及被誉为“性学之父”的现代性学奠基人之一的布洛赫。赫希菲尔德(1868—1935)是德国医学家,早期性学界最有影响的人物,他最大的兴趣在于对同性恋的研究。早在1896年赫希菲尔德就用笔名写了《怎样解释男人或女人爱同性的人》一书,1914年著《同性恋》,他认为同性恋是一种自然差异。1919年,他在柏林成立了世界上第一个性学研究所,其下设有性生物学、性医学、性社会学和性人类文化学4个研究室。该研究所有3项特别引人注目的服务项目:①婚前咨询中心(在德国系首家这样的中心);②每周一次的公共学术交流和讨论;③医学-法律服务,提供专家证明,特别是犯罪案例。该所90\%的服务项目是免费的,其收入主要依靠性治疗门诊和出版书刊。在成立1年之内,他们的免费咨询服务便积累了近2万份病例。1908年,他主编出版了世界上第一种性学杂志,独立发行了1年共12期,以后与其他杂志合并出版发行。1921年他组织了人类历史上第一次国际性的性学会议“在性学基础上的性改革国际大会”,大会上介绍了38篇论文,包括性内分泌学、性和法律、生育控制和性教育4个方面的内容。之后,他又组织了4次这样的会议。1928年他和别人合作组织了“性改革世界同盟”,在哥本哈根召开的首届会议上他当选为主席(第二、三届主席为霭理士和福勒尔)。1928年他出版了涉及整个性学领域的5卷本《性学》。1930年,赫希菲尔德因受纳粹迫害被迫离开德国四处流亡,足迹遍及美国、日本、东南亚、中国、印度、埃及、法国等国家和地区。他在中国停留期间访问了北京、天津、汉口、南京、上海、广州、香港、澳门等城市。1933年5月6日,希特勒上台刚过3个月,纳粹暴徒便捣毁了他的性学研究所,4天后纳粹分子在柏林歌剧院广场公开焚烧了赫氏研究所收藏的全部论文、图片和2万多册珍贵书籍,借口是他宣扬了“反德精神”,纳粹分子还把他的半身塑像抬到焚烧现场示众。以后他曾试图在法国重建性学研究所,但终以失败告终,没过多久他就客死法国。摩尔(1862—1939)是德国柏林的神经精神病医生,他虽然不如布洛赫博学,但组织能力很强。他的三部早期重要的性学著作是:1891年出版的《相反的性感受(同性性行为)》;1897年著书《性欲调查》,讨论了性欲的本质,本书对弗洛伊德有重要影响;1909年发表了第一本有关儿童性生活的书《儿童的性生活》,首次阐述了幼儿性欲的概念,此书对后来的弗洛伊德可能产生了一定的影响。1913年他领头成立了“实验心理学学会”和“国际性学研究会”,并于1926年10月10日在柏林议会大会议厅组织召开了第一次“纯科学的”国际性学研究大会。出席大会的性科学家包括马林诺夫斯基(著有《两性社会学》,1927),马库塞(著有《性禁欲对健康的危险》,1910;《性欲与性行为指南ABC》,1963),本杰明、海尔和拉塞尔等著名性学界先驱。布洛赫(1872—1922)是德国著名的皮肤性病学家,他受过良好的教育,知识面很宽,在社会科学、人文科学等领域尤有造诣。他首先把社会科学引入性学研究领域,用历史学、民族学和人类学的知识和方法研究性的演变和现状。为了反映性学的多学科方法研究的特点,他于1906年杜撰了德文词汇“性的科学”,即“性学”(sexology)。1907年著《我们时代的性生活》,1912年他开始主编《性学手册大全》,实际上出版了3卷:《妓女》(2卷,布洛赫,1912,1915);《同性恋》(赫希菲尔德,1914)。他和赫希菲尔德、摩尔等人共同提出性变态不是罪恶,而是心理疾病,甚至只是一种变异。他们一起为性教育、性改革而斗争。由于他们的努力,性学在世纪转折之际建立起来了。

除了上述性学家在变态性心理学方面作出了杰出贡献外,另一学术思潮则研究正常人的非变态性心理,代表人物有布洛赫、霭理士和弗洛伊德。西格蒙德·弗洛伊德(1856—1939)是与马克思、爱因斯坦齐名的三位著名的犹太学者之一。他对20世纪整个西方文化和社会都具有不可磨灭的影响。他从医学院校毕业后,曾在神经解剖学和生理学实验室工作,后因经济困难改行从事临床工作。由于他勤奋好学,思维活跃,善于钻研,判断力十分敏锐,工作认真,很快就取得了巨大进步。弗洛伊德大胆解剖自己,不断研究自己的记忆、梦幻、反应等,从中得出儿童性行为等一系列新理论,为未来的性研究人员留下了宝贵遗产。他提出了著名的幼儿性欲理论、性本能学说和人格结构论等。他在实践中创立了前所未有的精神分析疗法,这一学说风行西方世界并统治了性学领域达半个世纪之久,至今仍可见其顽强的生命力。精神分析学说讨论得最多的是性本能,受到指责最多的也是关于性本能的内容。弗洛伊德精神分析疗法的理论核心是泛性论,他认为以性本能为核心的本能冲动一直受到“超自我”原则的压制。它们总想冲破种种压制去实现满足,其主要方法就是通过各种玄妙的潜意识过程(如人的做梦、精神失常、性错乱等),变相地向外宣泄。虽然从生物本能出发,他用泛性论来解释一切人类精神活动与社会现象的主张缺乏科学和实践的依据,但却扩展了心理学研究领域,加深了心理学研究层次,使其与医疗实践更加紧密相连。他提出的许多独特概念和方法均得到了人们的认可。他的性心理学理论冲击了传统的、陈旧的性观念,促使人们对性、对“反常性行为”表现出更为开明的态度,不再把性问题视为神秘的事,这些都有力地推动了性科学的飞速发展。他为什么对性问题如此感兴趣呢?这与他当时所治疗的心理疾病的临床实践密切相关。当时的性禁锢十分严重,人们的性欲宣泄和满足受到严重压抑,于是人们在性方面遇到的心理障碍和挫折必然会以各种形式表现出来。弗洛伊德的论著颇丰,如《精神分析导论》、《梦的释义》、《两性社会关系》等著作均具代表性。特别是《爱情心理学》一书汇集了许多性学论述,而他在1905年发表的《性学三论》就是他自认为自己所有著作中最有生命力的、集中反映性心理学问题的论著,他在书中大胆地提出恋父恋母情结、诱惑论、阉割焦虑、两性同体、升华作用、压抑感等,他认为性功能障碍是由无意识内心冲突造成的,这往往是童年经历的折射。经他总结、说明和运用,无意识和幼儿性欲这两个术语成为西方现代极具影响的概念。

霭理士(1859—1939)是19世纪末、20世纪初英国文坛上的一颗明星,曾被誉为“最文明的英国人。”他生于伦敦,曾在澳大利亚上学和教学,后回英国学习医学,是一个受过严格训练的科学家,最有功于世的是他对性心理学的研究和为性教育奠定了科学基础。他在性压制最甚、清教徒之风盛行的维多利亚女王时代(1837~1901)勇敢地、始终不渝地同传统的清规戒律作斗争,终于突破了传统愚昧设下的重重障碍,在西方奠定了人类性学的基础。他的第一部著作《性反常:相反的性感受》在英国因遇到检查方面的问题,翻译为德文后才在德国出版(1896),并在德国产生了重大影响。霭理士从1896—1928年根据个案分析等写出的《性心理研究录》共七册,对人类性行为作了客观和系统的介绍,成为性心理学的创始者之一。但第二卷于1899年遭英国当局的查禁,认为其宣扬了淫荡、腐化、下流、邪恶的思想,直至1935年才得到解禁。他的理论在西方国家广为流传,人们通过他的书了解到其他社会并不支持英国维多利亚式限制种种性活动的态度。霭理士认为正是当时所谓的文明阻碍了正常的性表达和性活动,例如连钢琴腿都要用布包起来,以免人们会想入非非地联想到女人的大腿。他认为没有一个能适应全人类的性标准,而当时社会上流行的有关性行为的观点大多是错误的,因此必须了解英国以外的其他社会的情况。他揭示出一些令当时人们感到震惊的观点:如妇女的性欲在月经期最强烈;几乎所有人(包括女性)都有手淫经历;女性性反应的缺乏是童年受压抑的结果,也是男方的无知。《性心理研究录》对医师、心理学家、学术界是一部不可多得的好书,对于一般人来说却过于冗长。因而他在1933年所著的《性心理学:学生指南》更为普及,一方面把新的研究成果补充进去,另一方面把“研究录”中的内容精简化,使该书成为以后研究性心理学理论的重要论典。我国学者潘光旦早在1946年就把霭理士的《性心理学》介绍给中国读者,但是这本具有丰富注释的译本被打入冷宫几十年之久,直到1987年才重见天日。霭理士在周游各地及与读者的交流中搜集了数百份第一手的个案资料,并专于研究,创立了性卫生学。

玛格丽特·桑戈尔被誉为美国计划生育事业的先驱,她早先是一名护士,后来热衷于妇女运动。因为目睹了贫穷妇女由于接二连三地妊娠分娩,个个骨瘦如柴或体弱多病的状况,所以她立志要降低出生率,帮助这些妇女免遭反复妊娠之苦,避免过多婴儿出生,使每个孩子都能具有良好的生活条件。她创立的美国第一个计划生育诊所,第9天就被当局查封了,本人也被拘禁一个多月,但这并没有打垮她,反而激起了她与包括美国政府在内的反对势力斗争的热情,最后以胜利告终。1921年在纽约召开了全美国第一届计划生育学术会议,她向美国妇女大力宣传计划生育知识,并通过不懈的努力和奋斗使之合法化。她的奋斗精神不仅赢得了美国妇女,也赢得了世界各国妇女的信赖和尊敬。赖希(1897—1957)是维也纳的一位医生,后来成为早期的心理分析学派的杰出人物。他因不满于弗洛伊德对社会、政治因素的忽略,而成为一名马克思主义者,他认为马克思主义的异化概念应该延伸到性,因为资本主义制度强加给人们的生活方式削弱了性的自由的健康表达。他还认为所有的神经症及人格问题都是由于表达和释放受阻而累积起来的性能量造成的。1922年起,他陆续发表著作,提出性高潮能量、性高潮反射、性高潮辐射等一系列新理论。1927年发表《性高潮的功能》,首次确定“性高潮是性现象和性研究的中心”的原则,从而使性学开始着重研究正常人的正常性生活。为了弥补心理分析在政治思考方面的不足,他于1928年正式脱离弗洛伊德派,并于1929年组织了“社会主义性指导与性研究会”,有的激进分子称他为性与政治自由的斗士。

随着人类科学探索的迅猛发展,性反应的实验室研究禁区终于被打破了,他就是美国著名心理学家华生(1878~1958),在37岁就当上了美国心理学会主席。1914年,华生最早开始对性反应过程进行实验室研究,创立了心理学中的实验主义学派和行为主义学派。他和女秘书合作进行的人类性行为方面的研究由于超越了时代所能接受的范围,未能得到其妻子和世人的理解,妻子不择手段地破坏了他的实验室和资料,随之与他离婚;社会上的人们斥责他是伪君子、淫棍,他本人也被法院判为“行为很坏的专家”,研究资料被洗劫一空,研究结果未能发表而不为世人所知。华生的研究成了科学史上“被迫流产”的又一典型事例。以后汉密尔顿和戴维斯继续这方面的工作,分别发表了专著,使性学中的统计学发展起来。此后,赖希理论与行为主义结合起来产生了性行为学。20世纪20年代涌现出不少性知识手册,其中的佼佼者是荷兰妇产科医生范·德·维尔德于1928年出版的《理想的婚姻》,成为当时的时髦读物。他强调和谐的性生活是促成双方感情交流的重要因素,并宣扬性交前的爱抚、变换性交体位、吻生殖器等性行为方式。

美国性学家迪金森博士(1861—1950)既是医生,又是画家,曾描绘了许多女性和男性生殖器官的图像资料。1933年他出版了《人类性解剖学》一书,1949年该书在修订后改名为《人类性解剖学图谱》,并成为这方面的世界著名的权威性专著,时至今日,仍不失为一本非常有价值的参考书。在摹绘人体绘图之余,他还采取面对面交谈的方式收集了1200份性生活的个案资料。由于他在患者心目中树立了很高的威望,患者也能坦诚地向他敞开心扉。虽然他的书引起社会轰动,但医学界却对此反应冷淡并怀有敌意,使他遭到惨痛失败。

尽管在性学研究的道路上荆棘遍地,但一代代性科学家并没有被吓倒,他们前仆后继,终于在第二次世界大战后为现代性学的发展创造了柳暗花明的新局面。金西(1894~1956)是美国印第安纳州大学生物学教授,原系昆虫生态学家。他创造了一套特殊的面对面调查和记录的方法,取代了过去门诊积累和实验室观察的方法,最详尽、最广泛、最系统、最客观地研究了17 000多例美国不同肤色、不同年龄、不同教育程度、不同职业、不同地区男女的性生活的各个方面,调查问题多达350个,521项,他的调查采用严格保密的形式,至今人们仍不知道他所用的密码的含义。1948年和1953年他和同行发表了两大册共800页的专著《人类男性性行为》和《人类女性性行为》。人们把金西的工作称之为现代性学研究的第一座里程碑。他的研究报告阐明了有关人类性行为的精确定义,并总结归纳了美国社会存在的多种多样的性活动模式。虽然这不是一项随机抽样调查项目,但由于资料来源于正常人,所以还是很有代表性的。如经过对同性恋、异性恋的调查后,他把所有人群分为0~6这7个连续的等级,把人们分布于不同的等级位置上。他的调查报告指出美国白人中有37\%的男子和13\%女子在青春期以后,有过能达到性高潮的同性性行为;92\%的男子和62\%的女子有过手淫;大多数男子和半数的女子承认有过婚前性行为;半数已婚男子和1/4已婚女子至少有过一次婚外性行为;半数以上的人有过口交。他的报告极大地影响了人们的性观念,他崇尚性行为,认为性行为本身是一件好事,没有任何特殊的性行为是错误的。他说:“从生物学观点出发,我认为人们为发泄情感而采取的任何方式都是自然的。”正如清代大学士纪晓岚所云:“夫妇之间何事不可为。”金西工作报告的发表极大地推动了性知识的普及,使性教育成为一项群众运动,开阔了人们封闭已久的眼界。他的调查工作至今仍是世界上规模最大、最详细的性调查。

马斯特斯和约翰逊夫妇是美国乃至世界著名的性学家,他们继承了华生所开创的性实验室研究的事业,在人类性反应实验中大获成功。马斯特斯在1938年刚刚大学毕业时就对性行为研究产生了兴趣,但他的导师劝告他,“等一下,等你成熟了,等你在性科学以外的领域里取得相当声誉后再加考虑”。1954年,38岁的马斯特斯已是一名杰出的妇产科教授,获准在大学医院主持妇产科研究,他谨慎地迈出第一步,设立了生殖生物学研究室,从走访和调查妓女着手。然而他发现妓女在生理上都是不太正常的,性器官扩张明显且总处于慢性充血状态,让她们作为受试对象是不恰当的。于是他开始征募志愿受试者,消息传开后,许多男女纷纷报名参加,足以让他挑选到足够数量的,身体健康的,有相当表达能力的,能在性实验过程中详细报告所产生的体验和感觉的志愿者。随后他又找到学过音乐和社会学、作过广告和管理研究、有商业写作经验的约翰逊当助手,之后两人结成夫妻,在这项极富冒险性的新事业中携手奋斗。他们采用应变计测量阴茎周长等的变化,采用内置照相机的透明塑料阴茎模拟物来观察阴道壁的各种变化,他们测量了312位男子和382位女子在性交、手淫、模拟性交等三种性活动方式中的心率、呼吸、血压、肌肉收缩等基本生理变化,不再像过去那样只记录受试者的主观叙述,所以他们得到的资料更具科学性和客观性。1966年出版的《人类性反应》,以轰动世界的性实验室研究成果作为人类性研究从黑暗走向黎明的一颗启明星(该书于1989年由马晓年等人翻译并在中国大百科全书出版社出版)。这本书详细总结了他们在十几年里研究过的1万多个性反应周期的资料,提出了人类性反应的四阶段划分法,从而使全世界对性反应的描述有了共同语言。他们所做的工作纠正了广为流传的“手淫有害论”;纠正了弗洛伊德关于妇女反应高潮分为“阴蒂高潮”和“阴道高潮”两种类型并极力贬低阴蒂高潮的错误认识;延长了人们认可的性生活年龄,即衰老并不意味着性欲的必然减退和性高潮能力的必然丧失;首次提出男子在高潮射精后具有性不应期和女子具有多次高潮的能力;观察到哺乳可以引起妇女的性反应,实行母乳喂养的妇女的性欲和体力的恢复比不哺乳的母亲更快;提出性功能障碍的治疗需要夫妻双方共同参与;促使大多数医学院设立了有关性行为的课程。他们从1959年起又开始对人类性功能障碍进行研究,经过多年的努力,创立和总结出性感集中训练等一整套的性行为疗法于《人类性功能障碍》(1970)一书中,使众多患者摆脱了性的烦恼。这两本书矗立起现代性学研究的第二座里程碑,以后他们又发表了有关同性恋治疗中的伦理问题、性医学教科书等一系列著作,为性研究作出了重大贡献。

海伦·卡普兰是美国著名的女精神病学家和性治疗专家。她发现男女两性在发育过程中敏感区的形成上有明显不同,男性集中在生殖器区域,过了中年发生泛化,扩展到全身;女性则相反,青春期时敏感区泛化,到中年后才集中于生殖器区域。她把性反应划分为三个独立的时期:性欲期,探讨了性欲形成时期的性障碍;充血期,指生殖器官等部位的血管充血;收缩期,指性高潮期肌肉的收缩反应。她在1974年出版的《新性治疗学》中把心理分析治疗和行为治疗有机地结合起来,继承和发展了马斯特斯和约翰逊的工作,并给予了理性的总结,开创了性治疗的新局面。如性欲障碍偏重于性心理治疗,而无性高潮偏重于性行为治疗。性治疗的主要目标限于缓解患者的性症状,而不更多地涉及内心冲突和人际问题,只有当心理动力学问题成为治疗性功能障碍的阻抗时才给予适当处理。1987年她的《性厌恶、性恐惧和恐惧症》出版,专门探讨了性恐怖状态这一常见而又很难处理的问题,其临床表现就是性厌恶和恐惧性性回避。到20世纪90年代,她又陆续把海绵体血管活性药物注射和西地那非的使用融入她的性治疗实践中。

齐勃格尔德和艾力森于1980年提出了性行为的五期划分法,他们认为马氏夫妇忽视了性反应中认识(想什么)和主观(情绪如何)方面的内容,感到这两个特别重要的主观因素被忽略了,即性欲和唤起,前者指一个人想要性交的频率,后者指在性接触中所能兴奋起来的次数。

他们提出五期分类法:①兴趣或性欲;②唤起;③生理准备(阴道润滑、肿胀和勃起);④高潮;⑤满意(一个人对所发生的经历的评价或感受)。这种分类对于理解和处理性功能障碍比马斯特斯和约翰逊的理论更具实际意义,因为不少病例用马氏模型无法解释。

海特(又译为海蒂)是一位获得历史学硕士学位的社会学工作者。她在20世纪70年代指导了一项关于女性性反应的调查,这项调查不是采用选择题方式,而是填充题,回答者可以尽自己所能详细回答每一个问题。海特的研究发展了性调查研究,它的重要价值在于获得了一批约3000名妇女的第一人称自述的性感受资料,这些材料可以作为教科书中重要部分的直接引语。通过阅读海特报告,女性可以认识自我,男性则可以更多地了解女性性反应。其后她又作了男性性反应的调查,她的两本调查报告在西方产生了轰动。20世纪80年代后期,她又完成了妇女与爱情的调查报告。这三本调查报告均已先后翻译出版。

美国性心理学与性别研究的鼻祖约翰·莫尼在美国约翰·霍普金斯大学从事性医学的开拓性研究近50年,也是一位性学新术语的杜撰者。他认为性学(sexology)是研究性分化与性二态(即两性差异)和研究伴侣之间情爱与性爱相结合的科学。他还提出性哲学(sexosophy)的概念,其内容主要是根据人们个人或共同对性的体验而产生的禁忌、观念和信念;个人的和众人共享的性价值,及文化所传播的性价值体系。他与人合作创办了世界上第一家性别自认门诊,并首先提出性别角色的概念(1955);他为医学院学生设计了第一个性医学课程表;他与外科医生合作成功地完成了世界上第一例性别转换手术(1965);他第一个提出雄激素不仅是男性性欲的决定因素,也是女性性欲的动力;他第一个使用激素治疗性犯罪的行为自控能力,并第一个探索行为细胞遗传学。他与莫沙夫主编了《性学手册》(共7卷,前5卷1978年出版,第6、7卷分别于1988、1990年出版)等一系列专著,并不断杜撰新的性学词汇,如“性别自认”、“情爱图式”、“性别图式”等反映性心理成熟的术语;他的工作赢得广泛的尊重;他顽强地与各种性禁忌和流行的错误观念作斗争;他尤其痛恨所有假装正经的道德维护者,他说:“每当我看到某些人伪善到了过分的地步,就禁不住要说,如果撕掉他的面皮,看到的必定是罪恶”。2002年6月,因在性学界的终生杰出贡献,他荣获德国赫希菲尔德性学大奖。

当今性学否定了“性的中心是生殖器”的概念,提出“性是以大脑为中心,以皮肤为终端器官”的新概念,这受益于性生理学和医学几千年来的发展。在这一新的性系统概念的影响下,人们对性的认识与处理又有了新的变化,生理功能和器质性因素又受到极大重视。20世纪80年代初,性医学的飞速发展表现在对ED(勃起功能障碍)的病因、病理机制、诊断方法、治疗措施等的不断深入和提高。丹麦生理学家乔姆·瓦格纳给男性性行为科学研究带来了革命性的变化,他设计的研究方法极其巧妙,首先在制造阴茎血管塑胶铸型模型上下功夫,因为这对研究阴茎血管的走行和分支是绝对必要的。通过对阴茎血管的正常与病理变化的对比研究,提出血管性ED的新见解,这极大地推动了男性性功能障碍诊断和治疗的进程。过去人们一直认为90\%以上的ED患者是由于种种精神因素所致的,但20世纪80年代以后性医学的发展使人们认清至少有一半ED是由于神经、内分泌、血管等器质性因素所致。于是数十种新的诊断技术和手术治疗方法得以建立和发展,为众多的患者带来新的希望。他还提出盆腔截血综合征、静脉血管漏和海绵体漏的概念,创立了视觉色情刺激法鉴定心理性ED和器质性ED,并通过阴茎血管铸型研究发现阴茎动脉与静脉系统之间存在的交通支,对解释阴茎血管天然液压系统的机制十分重要。

近20年来性学之所以能取得令人瞩目的进展,是由于性研究由原先的纯性学的一条腿走路变成性学与交叉学科交互渗透、共同进步的两条腿走路的方式。如性社会文化学、社会生物学、行为内分泌学、生理行为学、社会生态学及社会心理学等都与性医学有交叉渗透。当今性学中还出现了作为性行为学和性治疗学对立面而出现的价值主义思想,强调人的价值和主观意志,被认为是“第三思潮”的人本主义心理学。他们认为心理学研究应着重向对人的价值和人格发展方面进行,他们既反对弗洛伊德把意识经验还原为基本驱动力或防御机制的精神分析,也反对行为主义把意识看做是行为的副现象。关于人的价值问题,他们大都同意柏拉图和卢梭的理想主义观点,认为人的本性是善良的,恶是环境影响下的派生现象,因而人是可以通过教育提高的,理想社会也是可能的。人本主义心理学的代表人物是马斯洛(代表著作为《人类动机论》,1943),罗杰斯(代表著作为《患者中心疗法》,1951;《论人的成长》,1961),梅(代表著作为《存在:精神病学和生理学中的新角度》,1959)。马斯洛提出,人的需要和动机是一种层级结构,高级动机的出现有赖于低级需要的满足。不论低级或高级的基本需要和动机都具有本能的或类似本能的性质,即都有自发追求满足的倾向,而高级的需要和动机如友爱、认知、审美和创造的满足,即人的价值的实现或人性的自我实现。罗杰斯则认为人的内在建设性倾向虽会受到环境条件的作用而发生障碍,但能通过医生对患者的无条件关怀、移情理解和积极诱导使障碍消除而恢复心理健康。在教育中则强调建立亲密的师生关系和学生自我指导能力的重要性。所以性治疗也要以患者为中心,要给他们自己选择的机会,给他们自主权,让他们最终能自己教育自己。医生只是教练,只能促进、鼓励、帮助,而不能强迫、硬性规定和空洞说教。治疗的基本前提是尊重患者的自主权和发展患者的自身人格,医生应以平等的身份,理解和同情的态度,深入了解患者的病因,为其提供咨询意见。班克罗夫特是英国的性治疗专家,他不信奉马斯特斯和约翰逊的或卡普兰的性学理论,相反,他只倾向于讨论两个概念:性唤起和性高潮。性唤起包括了性欲、中枢和外周的性唤起能力。他认为性欲是经验的产物,包括认知、情爱和神经生理的因素,而这三种因素又受某种特定机制的影响。如认知过程从属于社会心理学;神经生理过程从属于生物化学;情爱从属于心理和生物化学的作用。中枢和外周的性唤起能力是更复杂的问题,班克罗夫特把它们看做是两个相互联系而又各自保持独立的过程。外周唤起容易得到证实,而中枢唤起能力尚无理想的证实办法。性高潮与中枢唤起能力都容易在主观水平上得到认识,而难以作客观描述,往往只能从它的神经生理基础和男女经验的差别上进行推测。班克罗夫特更强调性的心身模式,他对近年来在医学生物学方面的进展十分重视,他认为性兴奋涉及某种中枢的神经生理唤起,除了生殖器官的反应之外,它也包括能够影响人们表现和主观体验的认知成分。由此看来,性的生理与心理因素是相互影响的,性功能障碍也与影响身心反应中某环节的消极因素有关联。虽然他也承认过去的经验和现在的感受同等重要,但他更注重双方的相互关系,认为它是婚姻不和谐和功能障碍的温床。因此他为性治疗设置了更广阔的目标,不仅减轻病症,还将更多的注意力转移到相互的关系上,提出在感情上拥有充分的安全感,才能表达并享受性感受。班克罗夫特把他的治疗方法称为行为、教育和心理治疗技术结合的产物,实际上是“操作和理解的结合”,我们可以把他看做是一个折中治疗学家或综合治疗学家。即他不依附于任何一套理论,而是在实践中把两种或更多的技术结合在一起加以运用。现代性治疗正是经历了从单一治疗理论(心理分析)向二元(心理分析与行为治疗)乃至多元的治疗理论的发展过程,折中治疗或综合治疗始自马斯特斯和约翰逊。卡普兰使之系统化和理论化,班克罗夫特则更加完善了这一理论与实践。

开发和研制有效药物是几千年来人们梦寐以求的目标,然而有效者寥寥无几。如人们曾对雄激素寄予很大希望,但对激素水平正常而又有男子勃起功能障碍(ED)的患者几乎无效,育亨宾于废弃多年后再次用于临床ED的治疗,因其疗效太低又遭淘汰。20世纪80年代开始,流行用直接向阴茎海绵体内注射血管活性药物的方法来治疗男性ED,其疗效可达70\%~80\%。但大多数男子从心理上不能接受这种方法,自己注射更是困难,何况长期注射易致海绵体组织纤维化或硬结形成。经尿道给予前列腺素E1的制剂称妙士(Muse),先于西地那非(万艾可,伟哥)于1998年12月上市,它既不会因注射而损伤海绵体组织,又以局部应用的方式使局部问题得以解决,用药后没有明显全身作用,是一种很好的治疗措施。1998年3月,西地那非(万艾可,伟哥)在美国公开上市并掀起空前的轰动效应,终于令百年性学进程迈上了新台阶。其上市仅9年时间,就为辉瑞公司创下了数百亿美元的经济效益,社会效益更是不可估量。它的巨大成功不仅仅是商业上的,也为解决好历史性难题提供了有力武器,推动了ED治疗领域的发展。在此之前,由于缺乏有效的治疗方法,美国ED患者的就诊率只有7\%,而在西地那非上市之后很快增长到40\%。西地那非给现代性医学带来了重大突破,是性治疗方法的又一场革命,所以人们毫无疑问地将它的问世称为现代性学的第三座里程碑。其实西地那非的问世只是偶然,它原来被作为一种治疗心血管疾病的药物而进入临床试验,由于疗效不佳,医生们打算撤回该药,然而服药的志愿者却不愿意,经仔细询问才发现其原因是它能使男子的勃起功能发生明显改善,而这又是心血管疾病的常见并发症。开始时人们对此只是当成一种笑谈,但医学专家们立即敏感地认识到它可能具有一种新的用途,于是在后来引起轰动效应的治疗ED的新药便诞生了。当年发现西地那非所涉及的作用原理的三位科学家均在1998年荣获诺贝尔医学奖。随后,伐地那非和他达拉非也随之问世,开辟了药物治疗的新时代。

无论对于临床医生还是研究人员,哪怕是普通老百姓,都需要对女性生殖器的解剖结构有更为详细、确切的了解,而在过去这却是遥不可及的事情。虽然人们一直在努力,但直到近年,才稍稍有了一些眉目,当然,离谜底彻底揭开的日子终究是越来越近了。下面是对女性性解剖,特别是围绕近年来发掘出的更多的女性前列腺等结构的探索历史进行简略的回顾。

古希腊和罗马的大师都能够接受女性射液,认为这是正常的现象,并认为其与男人射精一样可以给女性带来愉悦,只是那时人们还不知道精子与卵子的存在,所以对这些液体是否也像男性精液那样具有生育能力的疑问产生了争议。亚里士多德不相信女性射液中含有生命种子,他根据对动物的观察,认为女性液体有助于愉悦,但与生殖无关;也有人认为女性射液之所以不含生命种子是因为不同女性的射液体积悬殊很大,不像男性那样彼此体积相近且液量有限。公元2世纪的格兰则相信女性射液含有生命的种子,她认同两种精液的理论,主张女性有两种“精液”,一种与生育有关,另一种液体则来源于“前列腺”,与性交的愉悦有关,与生育毫不相干。两种精液理论不仅在西方医学界盛行,对阿拉伯文化也产生了重大影响。希波克拉底则认为男女的精液是一起运行的。

16~18世纪,英国医生(Laevinius Lemnius)提到一个女人如何“吸出男人的种子并且把自己的内容糅合进去”。17世纪Francois Mauriceau描述了“尿道口的腺体在性交之际喷涌出大量盐水样液体,增加了女性的温热感和兴奋”。这个世纪对女性的解剖和功能有了进一步了解,特别是丹麦的巴塞氏家族(发现巴氏腺的人)。

荷兰解剖学家瑞格涅·格拉夫(Regnier de Graaf)在早年的争论中站在亚里士多德一边,而且验证了液体是来源于围绕尿道黏膜和肉样纤维之间的腺体结构和管道。在整个尿道管的非常薄的黏膜上可以见到许多大的腺管开口,偶尔可以见到从那里分泌出相当数量的液体。这些组织的厚薄有一指左右,可以称之为“前列腺”或“腺样体”之类的名称。他认为“前列腺”的功能是分泌这样的液体,让妇女更富性欲、更加润滑以便在性交时表现出更易接受的方式,女性因此也更加兴奋,就像男性前列腺受到刺激时所表现得那样。但是他没有很好地区分润滑液和射液,也没有区分液体是否与生育有关。

1880年亚历山大·斯基尼率先提供解剖学证据,证实了围绕尿道的多寡不一的腺体,称之为尿道旁腺,这是射出液体的来源之一,现在不少人把它称之为斯基尼腺。19世纪,性心理病理学家克拉夫特·埃宾用有关性变态的研究描述了女性射液,列于标题“女性先天性性反转”之下,作为与神经麻痹和同性恋有关的性反转(1886)。“妇女同性之间的性满足似乎减弱了接吻和抚摸的作用(这些似乎只能满足她们微弱的性本能),但是产生了性的神经麻痹,即女性射液”。

弗洛伊德(1905)在多拉案例的研究中赋予它一个病理称谓———“歇斯底里”,而当时的女性则对此给予了积极的评价。20世纪早期的婚姻手册把射液看做是正常的现象,如TH Van de Velde's在他的(1926)《理想的婚姻》中就有所认识,并记录了妇女对此的不同体验,“似乎大多数非专业人士都相信女性在高潮时就像正常男性所做的那样,会强有力地从身体射出什么东西。对女性而言,射与不射其实都是正常的”。

不过20世纪的前几十年,这个问题还是没有受到人们的普遍重视。1948年,美国妇产科医生霍夫曼(Huffman,J.W.)在美国妇产科杂志第55卷86~101页上发表了题为《人类成年女性尿道旁腺腺管的详细解剖学结构》的研究报告,书中清楚地阐述了由斯基尼确认的尿道口的腺体与之后陆续收集到的直接排放到尿道的那些腺管的区别。“尿道就像一棵树,从它的主干上会发出许多最终是盲端的分支———尿道旁腺和腺管”。然而,人们最感兴趣的不是腺体的功能,而是它们的结构和所产生的物质,于是在两年之后,一个更为确切的有关射液的现代考虑便出现了。格拉夫伯格(Gräfenberg)在国际性学杂志第3卷145~148页上发表了《女性尿道在高潮中的作用》一文,这是基于他对女性高潮的观察得出的研究论证。“在女性阴道前壁沿着尿道的走向,总能证实一个动情区的存在。……作为男性尿道的同源物,女性尿道似乎也被勃起组织所包绕,……在性刺激过程中,女性尿道会扩展变大,很容易感觉到。在高潮结束之际,尿道会明显地向外肿胀。……偶尔,产生的液体会……毫不吝啬的……”。“如果有机会观察这些女性的高潮,一个人可以看到大量清澈透明的液体从尿道口而不是从阴裂喷涌而出”。“我最初的印象是强烈的高潮让女性的膀胱括约肌出现残缺。性学文献里介绍过高潮时尿液的不随意喷涌。但是在我们的观察过程中,对液体的检查结果表明其并无尿液的特征。我倾向于女性在高潮时排出的这些液体并非尿液,而是沿着阴道前壁的性敏感区的尿道内的腺体的分泌物。而且这些喷涌而出的液体并无润滑意义,润滑液应该在性交之初分泌而不是在高潮的巅峰时刻才排出”。

可惜这篇文章在当时没有激起任何反响就销声匿迹了,被金西的《人类女性性调查报告》、马斯特斯和约翰逊的《人类性反应》等名著所淹没。直到1982年,马斯特斯与约翰逊仍然认为射液等同于压力性尿失禁。

进入20世纪70年代后期,有关G点和射液的研究逐渐步入正轨,如1978年塞维利和班内特在《性研究杂志》上发表的论文《关注女性射液和女性前列腺》,回顾和追踪了历史上有关射液的争论。1981年,惠普尔等根据对临床主诉尿失禁妇女的观察和研究发表了3篇系列文章,肯定了女性前列腺和射液的存在,把射液与尿失禁加以明确区分。1982年,《G点》一书出版,终于成功地向人们证实了这些新发现。2010年1月去世的斯洛伐克病理学家米兰·扎维克,30年里废寝忘食,在经费等极其困难的情况下一直致力于女性G点、尿道旁腺、前列腺的研究,并认识到它决非胚胎发育时男性的残存物,而是一个实实在在的具有独特功能的女性前列腺,是女性重要的性敏感区,对于女性性交高潮而言尤为重要,同时也是女性射液现象的重要参与者。他在科学出版物和写给医学杂志的信中强调这个解剖学上的全新发现,即女性前列腺是围绕尿道的勃起组织。女性阴道内的敏感区也许很分散,也许仅仅局限于某个很小的区域。人们总是热衷于硕大的阴茎,怎么没有人在意女性前列腺的大小呢?而且他认为女性前列腺和男性前列腺一样会发生炎症、良性增生和癌症,只不过其发生率较低而已。

从1998年至2008年,澳大利亚墨尔本大学皇家墨尔本医院神经泌尿外科主任Helen O'Connell博士等的解剖学研究论文详细解释了射液所涉及的解剖结构和功能的关系,她观察到会阴尿道是包埋在阴道前壁里的,除了与阴道壁紧密贴合的尿道后壁,尿道周围所有方向上都包绕着勃起组织。而阴道远端、阴蒂和尿道形成一个整合体,共同覆盖着具有上皮特征的会阴皮肤,它们具有共同的神经支配和血管分布,在性刺激之下,作为一个整体而作出反应。

目前有关女性性反应的争论主要集中在3个问题上:一是射液是否存在;二是它的来源和组成;三是射液在性行为建构理论中的作用。这些争论不可避免地带有某种政治色彩,既关系到人们的信念,又受到大众文化和色情观的影响,难以得出一个真正确定的答案。其中争论最强烈的是1984—1987年,如一些女权主义者强调不要把女性的性器官割裂成一个个小单位,而应该把阴蒂、阴道和尿道整合为一个整体,这对于传统的解剖学来说是一种挑战。她们还认为射液不过是一种信念而非生理现象。2001年,纽约的一位心理学家Dr.Terence Hines在美国妇产科杂志上发表文章“G点:现代妇产科的一个神话”,认为支持G点实际存在的证据太少,因此是不可信的。由此而引发了一场新的争论,直到2007年还有人提出“是事实还是幻想?”

性治疗在21世纪中会衰败甚至消亡吗?答案自然是否定的,但它必须脱胎换骨才会有出路。这就需要与高科技结合起来,把计算机技术引入性治疗领域。近几年来,计算机技术的发展令人瞩目,虚拟技术的出现更是为性治疗提供了一片广阔的天地,性治疗必将走出“山穷水尽”的困境,迎来“柳暗花明”的未来。虚拟性治疗的基本方法是让患者带上特制的头盔和眼镜,穿上布满传感器的紧身治疗衫,坐在能固定四肢的治疗椅上,然后计算机就会按事先设计好的程序运转,把相应的电生理信号输入头盔、眼镜和治疗衫,患者便可以接受到如同实际生活中的但却是模拟的触摸刺激,这样的性刺激相当于给患者提供了舒适的爱抚,并让患者享受到充分甚至是超常的性感受。计算机还可以让患者超越时空的限制,重新经历往昔的生活,并在关键的时刻纠正错误和过失,消除不愉快的经历,重新开始新的、积极的性体验。计算机还可以实现和满足患者的一切性要求或性幻想,并且不会给人带来任何不利影响,更不会产生任何伦理或道德问题。所以虚拟性治疗不仅可以解决过去性治疗中所遇到的种种困难,大大提高其治疗效率,它还有可以解决不少社会问题,如遏制娼妓的盛行和降低性犯罪率。

虚拟性治疗无疑是一种可以明显提高身体感受力、唤起性反应的有效治疗方法,是解决无数性问题的最先进的手段。但由于每套几十万美元的高昂造价,它的推广还有待时日,我们相信这一天不会等得太久。

目前存在的若干问题:

1.人们对于女性性问题的认识还相当肤浅,至今仍然没有找到解决这些问题的有效方法;男子的早泄和不射精同样面临缺乏有效治疗手段的问题。

2.性治疗师队伍的建设和培养问题,特别要考虑到和生殖健康咨询师、心理治疗师与婚姻家庭咨询师的交叉问题。

3.如何处理好性治疗中的伦理冲突问题。

4.因为骗人的、恐吓人的、反性的伪科学宣传仍然铺天盖地地充斥在各种媒体里,毒害着各个年龄阶段的患者,给他们制造的巨大精神压力或把他们引导到所谓治疗的陷阱里,所以面向大众的性科普教育仍是我们的首要任务,可谓任重而道远,我们仍须努力、努力、再努力。

(马晓年)


\section{第六节 性的道德伦理、社会文化和教育}

从伦理学角度来看,道德是指调整和指导人与人之间、人与社会之间行为关系的社会准则规范。对于人类社会生活来说,一般可以划分为三大领域,即职业生活、家庭生活和公共生活。与此相适应的道德规范也可分成三大部分,即职业道德、婚姻家庭道德和社会公德。人类的性关系是婚姻家庭关系的一个重要内容,因此性道德也成为婚姻家庭道德中的重要组成部分。人之所以不同于禽兽,其中一个重要标志就是道德,道德是约束人们行为的内在力量,法律是约束人们行为的外在力量。人的性道德意识一经建立,对其性行为和性活动就有了指导和制约作用。对每个人来讲,其性行为都必须遵守性道德,否则将会受到道德的谴责,甚至法律的制裁。在伦理学中,“伦理”是指关于人伦道德的理论,其中“伦”指人与人之间的关系,“理”反映的是道理和规则,合起来看就是协调处理人与人之间关系所应遵循的道理和规则。与“道德”相比,它主要是以理论的形式来反映对人们的道德现象和道德关系的思考。⑤ 一般而言,性道德与性伦理作为性存在的两种反映形式,有其共同的内容,都是对性行为和性活动的调节规范。然而,性道德对人们的调控手段是靠社会舆论、传统习俗和内心信念来维系并发挥作用的;性伦理则是在性道德的基础之上,人们对于两性关系和性道德现象的一种理性思考与分析,揭示其本质并将其上升到理论或学说的高度。

在现代社会中,两性关系和性行为的道德伦理评价原则主要包括男女双方结合必须以爱作基础的原则;男女双方自愿的原则;男女性交必须以婚姻缔约为基础的原则;对个人、他人和社会无伤害的原则。性道德伦理对人们性行为的调节和制约,主要通过义务、责任、良心、信仰、追求、贞洁、羞耻和嫉妒等几方面的内心活动或手段来发挥作用。

在充满着各种恐惧和生存威胁的原始社会里,先祖们在维持自身基本生存需要的同时,把希望寄托于人的自身生产上。然而,由于认识的限制,他们并不知道男性在生殖行为中的真正意义。在他们看来生殖是一种神秘的、神圣的、不可理解的事情,而且只有女性才能承担,女性在生活资料生产以及人类再生产中的主导地位,使全社会产生了一种最原始的崇拜心理,即女性崇拜心理,此时的女性成为人们的崇拜偶像,就连女性自己也自我崇拜。此外,由于原始时代的人们需要付出极大的代价甚至生命,才能在与自然搏斗的劳动中求得生存,然而他们却很容易从性交中得到快乐和满足,他们对自身性欲与性行为的产生机制一无所知,觉得阴茎的勃起和阴户的张弛都不受人的意志所控制,生殖器似乎是个独立于人体之外的怪物,于是又对生殖器产生了崇拜;同时,他们不了解性交与生育之间的关系,也没有对性交进行任何限制,只体会到了性冲动驱使下的一种紧张状态和伴随性交而产生的快感,以及在性交结束时产生的一种放松和倦怠,继而开始崇拜性交;再加上原始人对人类生育毫无知晓,认为是神赋予了女性这种魔力,使女性具有生育的本事,于是又形成了生殖崇拜。这三类崇拜就构成了我国古代性崇拜的主要内容。正如我国著名的性社会学家刘达临所说:“如果说性交崇拜只是基于男女在性交时所产生的高度快乐,生殖器崇拜只是对性器官的构造与功能不了解而产生的一种愚昧,那么生殖崇拜却涉及原始人的生存、发展和延续的一个实实在在的利益:添人进口才能兴旺发达。”(1)

上述这些性崇拜无不折射出原始人类质朴的性文化现象,进而影响到人们对性的观察与思考,当母系社会向父系社会转变后,这些古老的哲学思想又得到了进一步完善,出现了天人感应论、阴阳论、七损八益说等多种学说。

奴隶社会,社会生产力和科学技术已发展到一定水平,人类的文明程度也得到一定提高,其标志就是在这一时期逐步建立起人类文明之初的性道德,反过来又时时作用于社会,促使社会的不断变革。最明显的就是男女婚姻在形态上的变化,促使个体家庭冲破了对偶家族的羁绊,成为当时社会的主流形式,使原本就根基不牢的对偶婚姻让位给一夫一妻制。这种新型婚姻的出现标志着文明社会的开始,它最先摈弃了原始社会的群婚制,排除了血缘婚姻,继而演化成了从伦理和生理两方面考虑的“同姓不结婚”的性道德原则,使人类物种在变革的婚姻形态中得以不断优化,人口素质得到不断提高。然而,在当时男权社会里,男子为了确保自己的财产能够传给具有他的血缘的子女,开始要求妻子严守贞操,再加上血缘婚、族外婚、对偶婚时代的性自由已被家庭、专偶制的夫妻关系所取代,这时男女之间的性行为不再是一种公开的行为,而成为个人家庭中的私事,旁人无权过问,不允许向旁人公开,更不允许和丈夫以外的人进行性活动,于是性崇拜现象消失,取而代之的是性禁锢、性神秘。

通过封建社会对性文化的传承与发展,人们的性伦理道德观念得到不断进步和完善。但另一方面,随着封建社会经济的发展兴盛,封建礼教的逐步形成,人们在性观念上出现了急剧的变化。对于统治阶级而言,物质经济发展所产生的腐蚀性使其在生活上变得荒淫糜烂,封建礼教对其的宽容性又使其对女性的压迫蹂躏变得放纵无度。从奴隶社会中解放出来的奴隶,许多都迅速上升为自耕小农,其在经济地位上的变化,使得以男子为中心的社会地位更加巩固。而只有女性在这一社会性质的变革中更加沉沦,仍继续承受着精神和物质上的双重压迫。于是形成了影响我国社会近2000年之久的“三从四德”、“三纲五常”、“男尊女卑”等封建礼教。其代表人物就是董仲舒,他的这一套学说成为当时封建社会普遍遵守的道德规范,也成为男性用来压迫和桎酷女性的有力工具。在此影响之下,崇尚妇女贞节的性文化氛围在当时社会中越来越浓厚,一些文人墨客也开始围绕妇女守节问题著书立说,如从刘向的《烈女传》到班昭的《女诫》七篇等,无不再三强调封建礼教,崇尚贞节烈女,鼓吹男尊女卑,这些都是以限制妇女自由,剥夺女性人权,污辱女性人格为代价的。

从五代开始,逐渐形成了我国特有的缠足文化,在此以后的千余年里缠足成了女性优雅高贵的象征,除了偏远地区或少数需干粗活的妇女外,几乎所有的女子都难逃缠足的厄运。它是一种强迫女性肢体畸形发育的残酷行为,因为要将双脚缠到“小瘦尖弯香软正”才符合男人的欣赏标准,俗话说:“小脚一双,眼泪一缸。”其痛苦的滋味可见一斑。

民国时期,由于半封建半殖民所依赖的社会经济基础和思想基础仍然存在,封建礼教仍在蹂躏着我国妇女,呈现出十分丑恶的性现象。遗老遗少、军阀官僚们道貌岸然,公开或半公开地实行性剥削,娶妻纳妾是合理合法的,玩弄名伶、嫖妓宿娼也是随处可见的。然而也有许多开明人士,他们受到西方先进科学思想的熏陶,一次又一次地在中国这片沉睡的大地上掀起思想解放运动,宣传以民主思想,反对封建礼教,提倡光复人性为主旨。当时的主要代表人物鲁迅,以其敏锐的思想,博大的胸怀,犀利的笔锋给封建礼教、专制制度多次致命的打击,他发表的杂文《我之节烈观》,猛烈地抨击了我国封建伦理道德中最腐朽的封建节烈观点。这些运动必然为推动男女平等,粉碎性禁锢的人性解放起到了重要作用。

新中国成立后,在新政府和共产党的领导下,开展了一系列性革命,颁布了《婚姻法》,反对封建礼教,提倡婚姻自由;反对包办和买卖婚姻;主张男女平等和夫妻平等;严禁纳妾和卖淫嫖娼,提高妇女在政治、经济上的地位,逐渐形成了一套新的社会道德规范。到了20世纪后期,改革开放给各行各业带来了新的生机,思想也得到空前解放,性已不再是隐秘和肮脏事情。但另一方面,西方的“性解放”、“性开放”思想也严重腐蚀着一些人的头脑,过去较少出现的性问题也逐渐暴露出来,如青少年的性早熟与早恋、婚前或婚外性行为、夫妻关系不睦、离婚、卖淫、性变态甚至性犯罪等问题,使性的自然性压倒了性的社会性,在社会上造成不少性紊乱现象。为此,我国政府特在《中华人民共和国刑法》中对性犯罪作出了明确规定,以法律的形式来进一步保护妇女的合法权益,维护社会的安定团结,为创造和谐社会打下良好的基础。

所谓性教育,是指结合教育学来传授各种性科学知识,以学会成为一个符合他(她)角色的合格的男人或女人;学会正常的异性交往,形成社会和道德所能接受的态度和行为。因此,性教育是一个范围广、内容杂的系统工程,它既包括性生理卫生知识的教育,又包括两性人际交往、伦理道德和性意识观念的教育。性教育的目的是用科学的知识武装人们的头脑,摒弃愚昧和谎言,明智地把握自己的情感,懂得恋爱、婚姻的意义,掌握性生理、性心理知识,以社会道德伦理和法律为准则,以健康的心理状态为基础,以最能获得愉悦的行为方式,满足双方的性欲,实现两性生活的和谐,繁衍身心健康的后代。开展性教育,尤其是针对青少年的性教育,不仅关系到他们的健康成长,还关系到国家的前途,民族的兴衰,具有重大的现实意义。

性教育的基本原则包括知识、态度和行为教育统一的原则;启发、诱导、尊重和交流并存的原则;适时、适量、适宜的“三适”原则。性教育的分期一般划分为婴儿期性教育、幼儿期性教育、儿童期性教育、少年期性教育和青年期(青春期)性教育。重点是青春期的性教育,因为这一时期是青少年人生观、道德观形成的关键时期,而且随着身体发育的变化,这一时期的孩子心理会变得非常敏感,他们有许多问题需要得到解答、指导和帮助,他们的好奇心也促使他们迫切地寻找答案。因此通过青春期性教育,不仅可以让他们掌握性生理、性心理方面的知识,更能够培养他们对性的道德约束能力,帮助他们理解和遵守社会规范。

性教育的具体内容:①性生理知识:这是性教育的基础,包括生殖系统的解剖与生理;男女生殖器的构造与功能;第一、第二性征;女性的月经、男性的遗精;性反应的特点;妊娠过程;性医学保健等。②性心理知识:包括性生理发育心理;性别角色心理;性爱发展心理;择偶心理;恋爱心理;性变态心理;性犯罪心理等。③性道德伦理知识:包括恋爱道德;婚姻道德;家庭道德;性道德原则等。④性法律知识:包括婚姻法;宪法中男女平等、家庭等方面的法律;妇女、儿童权益保护法;刑法中对于妨碍家庭罪等法律规章。⑤性病防治知识:包括目前性病与艾滋病的发病情况;临床症状、处理以及预防知识等。⑥人格教育:世界各国的性教育表明单一的性知识教育是远远不够的,因为人的行为涉及个人的价值选择、父母的态度、男女关系的社会规范、家庭的社会责任和个人事业历程等。所以性教育要贯穿于整个人格教育之中。⑦性美育教育:重点是让青少年了解如何塑造自己美的性别特征,其中应包括体型美、容貌美、肤色美、服装美、仪表美以及气质、性格美等。(陈乐)

原则上应该是“弘扬性文明,普及性教育,提高性素质,享有性健康”。

人类的性爱意识是不断增长的,他们不再像动物那样只在发情期才有性要求,他们时时刻刻需要爱。但人们又不得不约束自己的性要求以符合社会文化和伦理道德的规范,如果人们把性享乐作为性活动的唯一目的,那么人们往往会放纵自己的性行为,所以我们必须弘扬性文明。文明是相对野蛮和愚昧而言的,文明是人类在改造主客观世界实践活动中所创造的积极和肯定的成果的总和。它并非一成不变的,而是随着历史的发展在传统的基础上不断进步的,它呈现出由简单到复杂、由低级向高级的发展轨迹和趋势,既要继承人类历史进程中积累的精华,又要不断提高、创新和超越。人类文明包括物质文明、精神文明和性文明三个范畴,它们不是完全独立的,恰恰相反,它们共同构成了社会文明的基石,彼此之间相互影响、相互制约、相互渗透着,但它们又有各自相对确定的内涵、特征和发展规律。如果作为实践主体的人们在实践中发生种种失误,比如在政治经济发生重大变革时忽视了这三种文明中的某一方面,那么人类文明在发展进程中就会出现不协调或不平衡的矛盾,这时人们必须在社会实践中不断去调整,使之达到新的动态平衡。

物质文明是人类在改造客观世界过程中为人们提供物质生活的条件,以满足其衣食住行等基本的物质需求。它为精神文明和性文明的发展提供了动力,同时也从精神文明和性文明的发展中得到了强大动力。物质产品必须具备很大的文化含量才能适应社会发展的需求,并具有竞争力。

精神文明包括科学文化素质、思想道德素质、价值取向、情感信念等各个方面的内容,它是人们在不断改造自己的主观世界过程中不断发展、进步和创造的结果;为人们提供价值导向,以达到本质和现实水平上的高水准;它也造就了社会新人,因此也是对人的建设。随着社会的发展,精神文明受制于社会经济运行机制的现象日益明显,物质生产、经济活动的规律和特点不断渗透到精神生产之中,如文化产业和文化市场的形成。

性文明需要物质文明为其提供基本条件和动力,也需要精神文明提供价值导向、知识结构和发展动力。同样,人们对性健康和性生活质量的追求既刺激了性产业的产生和发展,也向精神文明提出了更新更高的要求,希望为其提供正确的价值导向和判断,提供更新、更多的科学信息和服务。

性市场主要包括公开的正规的与性相关的书刊音像制品和其他各种形式的信息服务、性保健用品的开发与销售、性咨询和性治疗。但是我们不得不承认,现实生活中还存在着屡禁不止的半公开的或地下的性市场,它主要指含有色情内容的非法出版物(包括网上的性信息)、各种形式的软性色情服务或三陪、卖淫嫖娼等。这就要求社会对现阶段,即社会主义初级阶段的性文明建设提出具体建议和要求,对性市场的形成、发展和管理提出具体要求,即如何界定和体现性文明,对若干敏感问题展开讨论。在目前经济发展还很不平衡、贫富差别和性别歧视仍相当严重的情况下,既然我们无法完全禁止和消除这些落后甚至是腐朽的社会现象,那么我们就必须拿出一些切实可行的、适合我国现阶段社会经济发展水平的,也就是符合我国目前市场需求的对策和管理办法,该打击的绝不手软,该查禁的坚决取缔,允许存在的则给予正确引导,绝不能放任自流、置之不理或有意回避,只有认真贯彻执行有关法律法规才是严谨、科学和负责任的态度。

我国在实现物质生产极大飞跃的30多年来,一直忽视精神文明建设,导致精神文明明显滞后于物质文明,形成一个低压区,使不少人缺乏信仰和活力、丧失了人生追求,往往为金钱和物欲所控制,甚至让种种封建迷信活动和资本主义的腐朽生活方式卷土重来。但如果像十年动乱期间一样,不允许人们生活富裕,离开物质生活的基本保障而空洞、抽象地讲什么思想政治工作也是徒劳的。30多年来,人们的性观念和性行为模式已发生了巨大变化,但这并不意味着那些消极保守的性观念和种种错误的信息便会自动消失,相反,种种不良风气和性行为方式还乘虚而入。一方面,许多青少年由于缺乏科学的性知识而对自己的性发育和性生理充满困惑、内疚甚至罪恶感,严重影响身心健康;许多成年人由于缺乏科学的性知识或存在各种性偏见而影响了他们的性和谐。另一方面,色情文化的侵入,搅乱了社会秩序,冲击着我国传统文化的优良美德。涉足色情场所并接受色情服务似乎成了公关和应酬的必要手段,而且还是公款消费。

现代性文明应该继承和发扬人类性文明的优良基因,并不断把它推向新的高度,使之成为人类历史上最高尚、最先进、最科学的新型性行为模式。这就要求它既要继承和发扬中华民族的传统美德,又要充分展示时代的进步;抵制和摒弃一切愚昧、禁锢、庸俗、腐朽没落的封建或资本主义的性习俗和性行为模式,提倡男女双方在爱情的基础上体贴和关爱对方,共同学习、交流、探讨和分享性的科学知识和乐趣,并通过性活动增进相互之间的感情,以达到性的安全、和谐与美满。性文明的实现需要物质文明和精神文明的保障,需要充分贯彻执行当前社会主义初级阶段中执行的各项方针政策,促进精神文明和性文明建设自身的制度化,将它们纳入政治、经济和法制的管理轨道。文明的产生、继承和发展都要靠社会制度的保障,社会制度在一定程度上规范了为社会所接受和承认的行为方式,没有道德法律的约束和监督(硬性约束),仅靠道德自律(软性约束)往往难以奏效。目前的问题是旧价值体系虽已打破,但新的价值体系却未建立,新旧脱节,于是道德失范。因此应该先建立硬性约束再提高和扩大软性约束,以达到二者之间的协调统一。

性教育是健康的保证,它应该成为国民素质教育的重要内容。要提高全民的性健康意识和自我保健能力,教育和引导群众移风易俗、破除迷信、摒弃陋习,建立起适合国情的科学文明的现代性观念和培养健康的性心理素质,养成良好的性卫生习惯和文明的性行为方式是极其重要的。性教育要回答“性是什么”、“性的目的”和“如何获得健康的性”这三个问题。性教育应该始自婴幼儿期,并一直延续到生命的尾声;它是一件关系到国家前途、民族兴衰的大事,必须坚持各相关部门加深理解、密切配合和加强协调的策略,并持之以恒地探索、创新和实践;它也要贯彻防治结合、预防为主的方针,不断发挥各基层的作用。从事性教育的专业队伍中既有医务人员、计划生育工作人员、教育人员、社区工作人员、民政部门人员,也有传媒界的人士,当然作为家长,他们也应参与对儿童和青少年的性教育,可以说性教育在人生不同阶段将面临不同的任务和要求,并具备不同的特点。

目前我国性教育已引进安全性教育、贞洁性教育、同伴性教育等多种形式,但在所谓贞操教育的背后总是存在西方右翼宗教的反动活动,我们主张在深入开展调查研究的基础上,探索和总结出更符合中国国情的性教育模式。总之,性教育要兼顾全面,符合当前社会各界群众的需求,使他们能够理解、接受,并落实到行动中。

素质是一个整体、多维的概念,它泛指人的综合能力和表现,是由人们在社会生活中所处的地位和参加的社会实践所决定的,需要通过不断的自我修养和锻炼以达到更高的水平。性素质泛指人们在性方面的综合能力和表现,它涵盖了与性相关的所有内容。人的性素质不是天生的,需要在后天的实践中不断形成、逐步发展和日趋完善;而性本能是天生的,它代表了人类性活动的生物学属性。性素质受不同文化背景、历史环境、社会活动等因素的影响,所以说它还代表人类性活动的社会学属性。性素质既指一个人在性本能的驱使下追求性乐趣和性快感的欲望、活动、能量和体验,也包含了人在性本能的驱使下表达彼此间爱情、温柔、体贴、承诺的欲望、活动、能量和体验。性素质的内涵很丰富,它包括:①性生物学成分———生物学性别、性反应能力、性功能状况、生殖功能和生育控制能力、性传播疾病的预防、性功能障碍的预防和治疗等。②性心理学成分———拥有正确的性别概念、性取向和性偏好,性欲正常,拥有健康和良好的性心理和性感受,拥有高度的自信心和安全感,能防止或克服种种性焦虑。③性社会学成分———拥有健康的社会性别(性角色),具有处理好两性交往、恋爱、成婚问题的能力,能建立良好成功的性关系,有调整好夫妻间交流的能力,熟练掌握各种健康的、正常的、普遍的性行为方式,遵守性法规法纪。④性伦理学成分———适时接受性教育、充分掌握性科学知识,要有科学的、积极的性态度和性信仰,具有进步的、有时代特色的性价值观念,能破除种种性迷信和性禁忌,性关系应当专一、彼此忠诚。性素质的教育、训练和形成的过程也就是一个人的性别特征、性角色和性行为准则形成的过程,我们也可以称之为性修养(sexulization),它是一个漫长、反复和多变的过程,贯穿于整个人生历程当中。虽然性素质在不同的时期、文化、社会、民族中有不同的标准,人们对它的好与坏、对与错、正常与异常难以作出一个明确的界定,但性的文明与进步、和谐与美满却是历史发展的必然趋势,这不仅是每个人、每对夫妻,也是每个民族、每个国家乃至整个人类的追求和奋斗的目标。

健康与教育不可分离,只有接受充分的性教育和加强长期的性修养并不断提高其性素质后,才能够保证性健康。性健康意味着人们对性生活采取积极的态度,从躯体、情感、精神、社会等方面都得到满足,能增进与改善性生活质量和人际关系。它包括以下内容:①根据社会道德和个人道德准则享受性行为和控制生殖行为的能力。②消除能抑制性反应、削弱性能力、损害性关系的消极心理因素,如恐惧、羞愧、罪恶感和虚伪的信仰。③没有器质性障碍、各种生殖系统疾病及妨碍性行为与生殖能力的躯体缺陷。④具有抵御性传播疾病和艾滋病感染的能力,具有防止意外妊娠的能力。

性健康要求人们以上诸方面的内容都应该是健康的,而不仅是没有疾病。人的性健康或生殖健康的命运由人的性素质所决定,一个人的未来和出路也掌握在自己的手中,只有认清自己、把握自己、战胜自己,才能维护自己的权利和健康。当然,性健康的实现也取决于政府关注程度,取决于其能否为性教育、性研究和性治疗等提供宽松的学术气氛和工作环境,还取决于相关专业技术人员的业务素质和技术水平的高低。为此,世界性学会已经更名为世界性健康学会。总之,实现人人性健康是一个宏伟的目标,需要我们的共同努力和不懈奋斗。

(马晓年)

1.曾钊新,吕耀怀.伦理社会学.湖南:中南大学出版社,2002:5.

2.刘达临.世界古代性文化.上海:上海三联书店出版社,1998.


\chapter{第二章 性生物学基础}

自然界的事物丰富多彩,千变万化,宇宙间的物质包括生物和非生物。生物与非生物最根本的区别在于生物具有繁殖功能,能够延续种族。低等生物如细菌,繁殖简便迅速,只要自身一分为二即可,这是最原始的生殖方式———无性生殖,这样的繁殖只是简单的复制。随着生物的进化,繁殖方式出现了新的飞跃———有性生殖。先是雌雄同体的低级有性生殖,渐渐进化到雌雄异体的高级形式,这样就产生了两套不同的生殖器官。

人类的生殖器官则是其进化的最高级形式,结构复杂,功能完善,男女两性的生殖器官结构尽管不同,但主要组成部分和性能却有着共同之处。男女生殖器官可分为两类:一类是性腺,一类是附性器官。男女生殖器官中的性腺都有双重使命,一是产生精子或卵子;二是分泌性激素,性激素有助于附性器官的发育与生长。附性器官的功能是参与完成性行为和新生命的形成与孕育。


\section{第一节 性发育生物学}

生殖腺来自体腔上皮、上皮下间质及原始生殖细胞三个不同的组成部分。

在人胚第4周时,位于卵黄囊后壁近尿囊处出现许多源于背内胚层的大圆形细胞,称原始生殖细胞。第5周时,两侧中肾嵴内侧的表面上皮下方的间质细胞增殖,形成一对纵行的生殖腺嵴。生殖腺嵴的表面上皮不久后向其下方的间质生出许多不同的细胞索,称初级性索。原始生殖细胞于第6周经背侧肠系膜陆续向生殖腺嵴迁移,约在1周内完成迁移,进入初级性索内。

原始生殖腺的自然分化趋势是朝向卵巢方向的,若原始生殖细胞及生殖腺嵴细胞膜表面均具有组织相容性Y抗原时,原始生殖腺就向睾丸方向发育。性染色体为XY的体细胞胞膜上一般具有H-Y抗原,而性染色体为XX的体细胞胞膜上则无H-Y抗原,故具有Y染色体的体细胞,对未分化生殖腺向睾丸方向的分化起决定性作用。目前认为,编码H-Y抗原的基因位于Y染色体的短臂近着丝点部位。人胚第7周,在H-Y抗原的影响下,初级性索增殖,并与表面上皮分离,向生殖腺嵴深部生长,分化为细长弯曲的袢状生精小管,其末端则相互连接形成睾丸网。第8周时,表面上皮下间质形成厚厚的白膜,分散在生精小管之间的间质细胞分化为睾丸间质细胞,并分泌雄激素。在人胚14~18周,间质细胞占睾丸体积一半以上,随后数目迅速下降,出生后睾丸内几乎见不到间质细胞,直至青春期才重现。胚胎时期的生精小管为实心细胞索,内含两类细胞,即由初级性索分化来的支持细胞和原始生殖细胞分化的精原细胞。生精小管的这种结构状态持续至青春期前。

若体细胞和原始生殖细胞的膜上不存在H-Y抗原,未分化性腺则会自然地向卵巢方向分化,卵巢的形成要比睾丸晚。人胚第10周后,初级性索向深部生长,在该处形成不完善的卵巢网;随后,初级性索与卵巢网都退化,被血管和基质所替代,成为卵巢髓质;此后,生殖腺表面上皮又形成新的细胞索,称次级性索或皮质索,它们较短,分散于皮质内。人胚约从第12周开始,皮质索就断裂成许多孤立的细胞团,即为原始卵泡。原始卵泡由中央的卵原细胞和周围小而扁平的卵泡细胞构成,卵泡之间的间质组成卵巢基质。胚胎时期的卵原细胞可分裂、增生并分化为初级卵母细胞。足月胎儿的卵巢内约有100万个初级卵泡,尽管在母体促性腺激素的刺激下,有部分卵泡可以生长发育,但它们很快退化,大多数的初级卵泡一直持续至青春期前。

生殖腺最初位于后腹壁的上方,在其尾侧有一条由中胚层形成的索状结构,称引带。它的末端与阴唇阴囊隆起相连,随着胚体长大,引带相对缩短,导致生殖腺的下降。第3个月时,生殖腺已位于盆腔,卵巢即停留在骨盆缘下方,睾丸则继续下降,于第7~8个月时抵达阴囊。当睾丸下降通过腹股沟管时,腹膜形成鞘突包于睾丸的周围,随同睾丸进入阴囊,鞘突成为鞘膜腔。随后,鞘膜腔与腹膜腔之间的通道逐渐封闭。

人胚第6周时,男女两性胚胎都具有两套生殖管,即中肾管和中肾旁管(又称Müller管)。中肾旁管由体腔上皮内陷卷褶而成,上段位于中肾管的外侧,两者相互平行;中段弯向内侧,越过中肾管的腹面,到达中肾管的内侧;下段的左、右中肾旁管在中线合并。中肾旁管上端呈漏斗形开口于腹腔,下端是盲端,突入尿生殖窦的背侧壁,在窦腔内形成一隆起,称窦结节(又称Müller结节)。中肾管开口于窦结节的两侧。

若生殖腺分化为卵巢,因缺乏睾丸间质细胞分泌的雄激素的作用,中肾管逐渐退化;同时,因缺乏睾丸支持细胞分泌的抗中肾旁管激素的抑制作用,中肾旁管则充分发育。中肾旁上段和中段分化形成输卵管,两侧的下段在中央愈合形成子宫及阴道穹隆部,阴道的其余部分则由尿生殖窦后壁的窦结节增生而成的阴道板形成。阴道板起初为实心结构,在胚胎第5个月时,演变成管道,内端与子宫相通,外端与尿生殖窦腔之间有处女膜相隔。

若生殖腺分化为睾丸,间质细胞分泌的雄激素促进中肾管发育,同时支持细胞产生的抗中肾旁管激素抑制中肾旁管的发育,使其逐渐退化。雄激素促使与睾丸相邻的十几条中肾小管发育为附睾的输出小管,中肾管头端增长弯曲成附睾管,中段变直形成输精管,尾端成为射精管的精囊。

1.未分化期 人胚第9周前,外生殖器尚不能识别性别之分。第5周初,尿生殖膜的头侧形成一隆起,称生殖结节。尿生殖膜的两侧各有两条隆起,内侧的较小,为尿生殖褶,外侧的较大,为阴唇阴囊隆起。尿生殖褶之间的侧陷为尿道沟,沟底覆有尿生殖膜。第7周时,尿生殖膜破裂。

2.生殖结节伸长形成阴茎;两侧的尿生殖褶沿阴茎的腹侧面,从后向前合并成管,形成尿道海绵体部;左右阴唇阴囊隆起移向尾侧,并相互靠拢,在中线处愈合成阴囊。

3.女性外生殖器分化因无雄激素的作用,外生殖器自然向女性分化。生殖结节略增大,形成阴蒂;两侧的尿生殖褶不合并,形成小阴唇;左右阴唇阴囊隆起在阴蒂前方愈合,形成阴阜;后方愈合形成阴唇后连合,大部分不愈合成为大阴唇;尿道沟扩展,并与尿生殖窦下段共同形成阴道前庭。

睾丸未下降至阴囊而停留在腹腔或腹股沟等处称隐睾。据统计,约有30\%的早产儿及3\%的新生儿睾丸未降入阴囊,其中大部分在1岁末可降入阴囊,仍有约1\%为单侧或双侧隐睾。因腹腔温度高于阻囊,故隐睾会影响精子发生,双侧隐睾可造成不育。

多见于男性。若腹腔与鞘突间的通道没有闭合,当腹压增大时,部分肠袢可突入鞘膜腔,形成先天性腹股沟疝。

是因左右中肾旁管的下段未愈合所致。较常见的是上半部未全愈合,形成双角子宫。若同时伴有阴道纵隔,则为双子宫双阴道。

因窦结节未形成阴道板,或因阴道板未形成管腔。有的为处女膜未穿通,外观不见阴道。

因左右尿生殖褶未能在正中愈合,造成阴茎腹侧面有尿道开口,称尿道下裂,发病率为1‰~3.3‰。

又称半阴阳,是因性分化异常导致不同程度的性别畸形,患者的外生殖器常不能辨别男女。按生殖腺结构的不同,两性畸形可分为两类:

(1)真两性畸形。极为罕见,患者体内同时有睾丸和卵巢,性染色体属嵌合型,即具有46,xy和46,xx两套染色体组型,第二性征可呈男性或女性,但外生殖器男女分辨不清。

(2)假两性畸形。患者体内只有一种生殖腺,按所含睾丸或卵巢的不同,又分为男性假两性畸形和女性假两性畸形。前者虽具睾丸,但外生殖器似女性,染色体组型为46,xy,主要由于雄激素分泌不足所致;后者具有卵巢,但外生殖器似男性,染色体组型为46,xx,由于雄激素分泌过多所致。常见有先天性男性化肾上腺增生症、肾上腺皮质分泌过多雄激素等原因,使外生殖器男性化。

患者虽有睾丸,也能分泌雄激素,染色体组型为46,xy,但因体细胞和中肾细胞缺乏雄激素受体,使中肾管未能发育为男性生殖管道,外生殖器也未向男性方向分化,而睾丸支持细胞产生的抗中肾旁管激素仍能抑制中肾旁管的发育,故输卵管与子宫也未能发育,患者外阴呈女性,且具有女性第二性征。

为了更清楚地了解男性的性发育和性生理与病理问题,我们有必要详细回顾阴茎的胚胎发生与发育过程。

在胚胎发育的第2个月,尿生殖窦为尿生殖窦膜所封闭。此膜周围出现一组隆起,发育成为外生殖器。它包括膜前方的生殖结节,以后伸长称初阴体,不分性别。初阴体基部在正中线上凹陷形成尿道沟,其两侧隆起为生殖襞,其外又出现一对隆起称生殖隆起。男子的性分化始于生殖襞的伸长,从而形成了阴茎,刚刚分化好的阴茎只有3.5mm长。两侧生殖襞并拢融合使尿道沟合成尿道管,最终形成尿道的海绵体部。在第3个月尿道沟闭合,而生殖隆起则向尾部移行至阴茎的基部,然后两侧隆起在中线上融合形成阴囊。至此,男性整个外生殖器的发育就暂告一段落。人们在很久以前就认识到雄激素的分泌(由胎盘分泌的人绒毛膜促性腺激素即hCG刺激而产生的雄激素)对于男性胎儿外生殖器的发育起着重要作用。如以阉割子宫内男胎后,男性外生殖器就不能正常发育为依据。后来的实验显示,胎儿睾丸生成和分泌睾酮的内分泌功能在第8周时加速,到12~20周后消退。然而,临床上曾介绍过的一组男性假阴阳人的病例,使人们对睾酮在男性正常发育过程中行使着的外周激素作用产生了怀疑。由于缺乏5α-还原酶,这些男子不能像正常人那样在外周组织中把睾酮转化为双氢睾酮。初生时,缺陷仅见于外生殖器,睾丸多呈现出腹股沟或阴唇内肿块,具有阴唇样的阴囊,一个尿生殖窦(阴道陷凹)和一个类似阴蒂的生殖器。因此,人们往往把这些男孩当做“女孩”来抚养,但是当他们到青春期时,就可以观察到戏剧性的变化:生殖器变大为一个具有正常功能的阴茎,精子开始形成,发声变得深沉粗重。有些病例从“女性角色”转变为男性行为模式,具有对女性的异性性兴趣。通过对先天发育不全的观察,人们确定双氢睾酮是子宫内性分化为男性表现型所必需的雄激素,而睾酮则似乎在青春期对第二性征的发育起着重要作用。因此也就解释了为什么这些患者在出生时性发育异常而到青春期又从解剖和功能状态上恢复为正常成年男子的现象。

阴茎生长速度很快,出生时已达2.5~3.5cm,这时胎儿垂体已开始分泌促性腺激素LH和FSH。

阴茎生长缓慢,这是由于下丘脑、垂体发育慢、睾酮水平也低的缘故。

垂体和睾丸激素分泌旺盛,阴茎迅速生长,仅5年左右就达到成人水平,增粗增长。

一般来说,只要有Y染色体,就存在决定男性的基因,也就能分泌雄激素,胎儿必定发育成男性表现型。否则,就会发育成女性表现型。但也有极少数例外,例如当受精卵内除了Y染色体外,还有两条X染色体,由于多余染色体的干扰,使男性的正常表现型受到影响,出现高个子、小睾丸、无精子的临床症状。随着科学技术的飞跃发展,人们对性分化又有了进一步的认识。近十年来,有些学者提出了新的学说。他们发现,在XY胚胎细胞膜上有一种蛋白质,是男性个体所特有的组织相容性抗原,叫H-Y抗原。它在胚胎发育过程中,对生殖腺向男性方向发展起着定向作用。这是分子胚胎学领域划时代的发现。但是,这种操纵性分化大权的H-Y抗原的遗传基因究竟在哪条染色体的哪一部位上,目前还不清楚,不少学者认为Y染色体短臂上存在H-Y抗原基因。但是,不论H-Y抗原基因位于哪条染色体上,一致的看法是H-Y抗原对性分化起着决定性作用。

总之,阴茎的分化与发育全靠体内的双氢睾酮或睾酮水平,如果妊娠第6~9个月期间睾酮分泌或转化为双氢睾酮的能力不足,阴茎发育低于正常速度,出生时阴茎便较小。但只要阴茎在青春期前不短于2.5cm,青春期后非勃起时不短于5cm,形态发育正常,勃起功能正常,且第二性征发育良好,就应该认为其阴茎正常,性发育正常。那些阴茎偏小,但属于正常差异范围的男子完全不要背包袱,要对自己有信心。因为从性功能角度讲,关键看勃起的能力而不是阴茎的大小,有些阴茎虽然很大,但无勃起功能那也属于异常;再者,从女性角度讲,阴道的神经末梢全部集中在外1/3段,性兴奋后充血肿胀的也是外1/3段,所以使女方获得性满足的关键不是阴茎的大小长短,而在于对女性性解剖和性生理的正确认识,及双方的情感、技巧等多方面因素。


\section{第二节 男性性器官的结构与功能}

男性生殖系统还可以按内、外生殖器来分类,这样介绍起来更方便些。

外生殖器官包括阴茎和阴囊,内生殖器包括睾丸、生殖管道及附属腺体。睾丸是男性的生殖腺,担负着生成精子和分泌雄激素的双重任务。生殖管道则包括附睾、输精管、射精管和尿道等。附睾具有储存精子和完成精子最终成熟的功能。输精管是附睾管的延续,承担着运输和储存精子的作用。射精时,精子通过上述管道后,经尿道最终排出体外。附属腺体包括精囊腺、前列腺、尿道球腺和尿道旁腺。可别小看这些配角的作用,精液的液体成分大部分来自它们,其中包含供养精子的大量营养成分,缺了它们,精子就会失去活力和受精能力。结合本书宗旨,我们主要介绍男性生殖器的结构和功能,其余性腺、副性腺、生殖管道等将不再赘述。

阴茎具有三大重要功能:它是男性性交的交媾器官;通过尿道把尿液由膀胱排出体外;在射精之前,它是聚集精子和精液液体成分的场所,也是排出精液的器官。要想真正了解阴茎的生理功能,就需先探讨其解剖结构。阴茎分为阴茎根、阴茎干及阴茎头(又称龟头)。阴茎根固定于耻骨,在尿生殖三角浅袋内,表面覆盖有会阴部和阴囊的皮肤,阴茎根是阴茎的固定部;根的前方为阴茎干,呈圆柱状;阴茎头为阴茎末端蕈状膨大部。阴茎干和头为阴茎的可动部。男子站立时,松弛的阴茎是悬吊下垂的。

人们正视的是阴茎疲软时的前面,又称阴茎背面;用手将阴茎翻起后是阴茎的腹侧面,又称阴茎尿道面,平时腹面与阴囊相贴。阴茎勃起后与躯体形成的角度称为勃起角,站立位时,阴茎勃起角在90°或90°以上时为正常;小于90°为勃起不全;有时阴茎完全不能勃起,呈下垂状态。勃起角主要与海绵体内压力的大小有关,在20岁左右最大,中年以后逐渐减小,可能与海绵体内维持压力降低和(或)悬韧带逐渐松弛有关。阴茎勃起时呈向前向上的方向,恰与女性向后向上的阴道管斜行匹配。

阴茎主要由3根结构相似的平行圆柱体构成。其中有阴茎海绵体两根,形似两端尖锐的圆锥体,两阴茎海绵体并行排列于人体中线的两侧并紧密相接,从功能上说,它们完全是一个单位。它们位于阴茎的背部和侧部,占据了阴茎的大部分体积,构成阴茎干的基础。每条阴茎海绵体除彼此相贴的内侧面呈扁平状外,横断面均似圆形,像两根圆棒并列于阴茎的背面,使阴茎干的背面成为扁平状。在阴茎背部中线表面有一纵行浅沟,这是两条阴茎海绵体在发育早期融合的痕迹。在两条并列的阴茎海绵体尿道侧表面平线上存在一条更深更宽的凹槽,称尿道沟,尿道海绵体恰好镶嵌其中。两条阴茎海绵体只有在根部才相互分离形成两只阴茎海绵体脚,它们分开后先稍肿大,然后逐渐变细,经由一根坚固的纤维带分别附着于每侧的耻骨、坐骨下支的骨膜和会阴筋膜上,每只阴茎脚均为坐骨海绵体肌所覆盖。阴茎的第3根圆柱体是不成对的尿道海绵体,它位于阴茎腹面中线上的尿道沟内,恰在两条阴茎海绵体的下方,因尿道从贯通其中而得名。它比阴茎海绵体小得多,所在的部位就是阴茎的尿道面。勃起时,尿道海绵体在阴茎腹侧突出,形成特征性的隆起部。在阴茎的远端尿道海绵体扩张并向背面延伸,形成一个前圆锥形皇冠样游离末端,即龟头。龟头覆盖在并联成一体的阴茎海绵体尖锐的圆锥形末端,也可以说阴茎海绵体的尖锐前端嵌入龟头底面的陷凹内。阴茎头的底边向外向后方突出,阔于阴茎干,形成游离凸隆的阴茎头冠。头冠边缘下方浅沟为冠状沟,头冠后的阴茎干较细,形成一条横沟,称为阴茎颈,又称龟头颈。尿道末端穿过龟头终止于龟头顶点处,形成一个矢状位裂隙样开口———尿道外口。在包皮环切术前,包皮覆盖龟头,环切术后,龟头可部分或完全裸露。同阴蒂一样,阴茎头对于性感觉有着特殊意义。阴茎头有着丰富的神经支配,对外界刺激十分敏感,但其神经纤维末梢的数量约比阴蒂少一半。尿道海绵体的始端膨大成球块状,称尿道球或阴茎球,它黏附于尿生殖膈下的会阴筋膜下方,使得阴茎的3条海绵体在阴茎根部相互分离。尿道球的大小变化很大,其后下方的中线上有一凹陷,深处可见一发育很差的中线纤维膈,从外观可见其是由双侧对称的各半融合而形成的。尿道穿过会阴筋膜,在离尿道球后端很近的地方斜行进入尿道球,尿道球的表面覆盖有球海绵体肌。人类覆盖于海绵体的骨骼肌(横纹肌),即球海绵体肌和坐骨海绵体肌均与勃起无关,但在射精时它们均作有节律的收缩,帮助精液射出体外。也有人认为这两组肌肉的收缩有助于遏止静脉血的回流,从而使阴茎勃起得到增强。反之,若它们过于薄弱则会成为ED的促成因素之一。所以有人提出男子锻炼耻骨尾骨肌等肌肉,将有利于其勃起功能的维持,特别是有助于阴茎根部勃起不坚的问题。

阴茎干的皮肤薄而柔软,可自由滑动,且具有良好的延展性,适于阴茎勃起时伸展。除了靠近耻骨处的根部有稀疏的阴毛生长外,阴茎干伸出部位的皮肤没有阴毛生长。在阴茎皮肤上有小汗腺和不伴有毛囊的皮脂腺分布,特别是在其两侧和尿道面。阴茎尿道面皮肤正中有阴茎缝,它与阴囊中缝相接,是胚胎发育期两侧生殖襞融合的标记。阴茎干皮肤向前延伸至阴茎颈,然后继续游离向前形成皱襞,包绕阴茎头,称为阴茎包皮。它由内外两层皮肤构成,内层皮肤润滑细薄,未角化,更像黏膜,并在阴茎颈处返折延续覆盖阴茎头。在阴茎头的腹侧中线上,有一条束状皮肤皱褶由包皮深处延附至尿道外口,称包皮系带,在包皮切除术中决不能将其切断。包皮内外两层返折游离缘围成包皮,由包皮口向内,在包皮内层与阴茎头皮肤之间,形成一个裂隙状包皮腔。包皮之下的分泌物在包皮腔内积聚,形成包皮垢,其来源包括龟头和包皮脱落或破碎的上皮细胞、包皮腺、阴茎头冠和颈部小腺体产生的松软豆渣样分泌物、尿液残渣等。包皮垢有特殊气味,但无生理作用。幼儿包皮包裹整个阴茎头,随着年龄的增长,包皮逐渐退缩至阴茎头冠之后,如成年人阴茎头还完全被包皮包裹,称包皮过长;若包皮口狭窄,不能使包皮向阴茎头后面翻转至阴茎颈部甚至与龟头粘连时称为包茎。包皮过长或包茎易引起包皮阴茎头炎和湿疣等并发症,甚至发生阴茎癌,所以包皮过长或包茎者,应早行包皮环切术将其切除。龟头表面的皮肤看上去像黏膜,其颜色变化很大,由浅红至紫色不等。龟头皮肤很薄,并紧密地与其下的勃起组织固着在一起,不能自由滑动。阴茎的被膜由浅入深,依次为会阴浅筋膜、阴茎筋膜和白膜。阴茎通过纤维结缔组织和肌肉把其根部牢牢地固着在骨盆上:阴茎脚将其附着于耻骨分支和尿道球部附着于三角韧带;悬韧带则是一束三角形的强韧的纤维组织,它起自下腹部的白线和耻骨联合前缘,走向阴茎根部,并呈扇形分布,广泛地与阴茎根部及阴茎海绵体的深筋膜鞘相融合;此外,阴茎的悬垂还依靠位置较浅、富于弹性的阴茎系韧带,它从直肠鞘和腹白线向下延伸分为左右两个系带,环状包绕阴茎,散布于阴茎的浅筋膜,并固定于阴囊肉膜的膈组织中。它在解剖上不十分明显,但对阴茎根部起着一定的悬吊作用。阴茎浅筋膜即Colles筋膜,由一层极为疏松、富含小空隙的结缔组织构成,位于阴茎柔嫩的皮肤之下,是阴囊肉膜的延续。阴茎浅筋膜缺乏脂肪,却有许多平滑肌纤维,此层内有阴茎背浅动、静脉走形。在浅筋膜的深处可见阴茎筋膜,其薄而致密,由弹性纤维所组成。此层结缔组织向前延续至阴茎颈部,逐渐变薄固着于龟头的基底。阴茎筋膜包绕3条海绵体,在这层鞘膜的近端,发出许多纤维插入坐骨海绵体肌和球海绵体肌的纤维,在阴茎根部紧紧地连结和附着在耻骨联合或悬韧带上。阴茎筋膜虽然与阴茎海绵体的白膜相结合,但它提供了血管和神经走行的空隙。阴茎的主要勃起器官阴茎海绵体封闭在白膜之中。当阴茎处于松弛状态时,白膜厚2~4mm,呈波浪形或皱缩形;当海绵体充盈伸展时,白膜便伸直展平。每根阴茎海绵体都有这样一层厚韧致密的白色胶原纤维包裹,两根阴茎海绵体邻面的白膜相融合,形成阴茎中膈。在阴茎的可活动部分,中膈是相当不完整的,特别是在靠近阴茎末端处,中膈几乎被许多平行的裂缝所中断。通过这些裂隙,两个阴茎海绵体的勃起组织就成为相互延续的整体。

包裹尿道海绵体的白膜比阴茎海绵体的要薄,且更富于弹性,它含有较少的胶原纤维而含有较多的弹力纤维。尿道球和尿道海绵体的尿道部分及阴茎头在勃起时只发生体积的扩张,而不会像阴茎海绵体那样因为具有厚韧的白膜而提供勃起时所具有的硬度。阴茎头也是由勃起组织组成,它通过丰富的静脉丛与尿道海绵体相通,与其他勃起组织不同之处是它不具有强韧有力的纤维鞘膜。在龟头之内的尿道部分扩张,并向两侧压缩,形成一个裂隙样通道,称作舟状窝,相当于女性的阴道前庭。围绕它的是一团在龟头形成的中膈样的纤维弹性组织,称为舟状窝襞,它持续向后加入阴茎海绵体锥形末端的白膜,并垂直地与包皮系带相结合。它不完全地把龟头的勃起组织分裂成左右两部分,然而,在背侧它们是自由交通的。从中膈起,纤维小梁向外向所有方向发出进入龟头的组织。龟头在勃起时是很软的,其海绵样的外表是由于存在众多血管吻合的丰富的静脉丛,其静脉的特征是具有非常厚的肌肉管壁。有些患者把双方性不满足的原因归咎于龟头不硬,甚至说过去一直很硬,现在不硬了,这是一种误解。认清在阴茎海绵体、尿道海绵体、龟头之间存在的被膜及内部结构的差别,也就可以理解它们的功能差异和在勃起中的不同表现了,因为这些结构差别正是它们参与勃起和射精生理的解剖基础。

近年来对进入与离开阴茎并引起勃起功能障碍的血运系统功能不全的介绍,已经引起人们对阴茎动脉供应的浓厚兴趣。腹主动脉首先分成左右两侧的髂总动脉,从髂总动脉又分出了髂内动脉。

大多数男子的阴部内动脉都来自髂内动脉的坐骨阴部干的最低点,即骶髂关节水平处的终末分支。在这之前,髂内动脉先分支出一些供应臀部肌肉的较大动脉分支。阴部内动脉在阴部管发出最后一个分支会阴支以后,即称为阴茎动脉,两条阴茎动脉平行走行并穿过尿生殖膈。阴茎动脉是一对非常短的动脉干,它们沿耻骨下支内缘走行于坐骨海绵体肌和球海绵体肌之间的裂缝中,然后在阴茎脚处接近尿道球处迅速分为4对终末分支:阴茎背动脉,尿道(海绵体)动脉,阴茎深动脉(阴茎海绵体动脉)和(尿道)球动脉。有时可见副阴部内动脉,沿膀胱下部和前列腺前侧方到达阴茎根部。

除此之外,血管吻合支在腹壁下动脉内支与阴茎动脉之间搭起桥梁,但是只有在正常动脉途径被阻断时,这根称为腹壁下阴茎动脉吻合支的血管才会行使其功能。阴茎动脉的分支都是成对分布的,其个体差异很大,可有种种变异,如它们的分支和数量,它们是在什么部位穿过白膜的,它们又是如何相互交通的。尿道球动脉短而口径较大,在穿过尿生殖膈下筋膜后常见的走行是在分支后很快相互吻合,然后交叉并穿进尿道海绵体的球部,在尿道球组织内向前走行,在一定程度上还与阴茎海绵体相交通,向尿道球腺和尿道海绵体近端供应血液。尿道动脉穿入尿道海绵体,然后向远端走行于阴茎龟头。阴茎背动脉在骨盆横韧带下穿出尿生殖膈下筋膜,再越经阴茎的悬韧带进入阴茎,其形状曲折蜿蜒,在阴茎深筋膜与白膜之间继续走行。阴茎背动脉始终与位于它们之间的单根(有时2根)阴茎背深静脉伴行,在它们的外侧则有成对的阴茎背神经。阴茎背动脉在随后沿阴茎海绵体走行过程中不断发出小分支,并以45°角向前斜行环绕阴茎海绵体,称旋动脉,与旋静脉伴行。它们的部分分支穿进白膜,进一步分成小的末梢动脉到达勃起组织,这些旋动脉负责向阴茎海绵体供应血液。最后,阴茎背动脉成为阴茎龟头的主要动脉供应者,此处可见该动脉与尿道海绵体动脉的吻合支。阴茎的筋膜和皮肤的血液供应有一部分来自阴茎背动脉的周边动脉,其余则来自阴部外动脉。阴茎背动脉也向下发出分支通过白膜进入阴茎海绵体,在那里与阴茎深动脉的分支相吻合。在阴茎松弛时阴茎海绵体血窦中的血流量很少,主要来自阴茎背动脉。

阴茎海绵体勃起组织的主要动脉供应者是成对的阴茎深动脉,又称阴茎海绵体动脉,它们的来源与数目有较大差异。阴茎深动脉是在穿过尿生殖膈之后才分出的,先沿阴茎海绵体背中线走行,在海绵体静脉的外侧,是一对极细的动脉血管,在称为阴茎门的两条阴茎脚汇合之处的海绵体组织中心点进入阴茎海绵体,然后在作为中央(深)动脉几乎一直向远端走行到海绵体勃起组织的尖端处。它们不断向海绵体勃起组织发出分支,以向组织供应营养血液。在海绵体非常纤细的小梁内,阴茎深动脉分出许多根蜿蜒盘曲的终末小动脉分支,根据其在阴茎松弛时呈现的螺旋形外观称之为螺旋动脉,它们的血管中层非常厚,就像未进入勃起组织之前的较大的动脉分支,其内膜下有纵行分布的结缔组织和平滑肌纤维组成的皱襞。螺旋动脉在阴茎海绵体的近端比远端更丰富,它们树枝样的分支直接弥散于海绵体血窦的窦状隙空间内。阴茎勃起时,纡曲的螺旋动脉伸直、扩张。中央深动脉还跨过中膈与对侧阴茎海绵体形成吻合支。此外,众多的解剖结构上的通道把阴茎的所有动脉互相沟通起来,如小动脉之间的吻合支和血管网络。瓦格纳等(1982)介绍了连接阴茎深动脉和尿道海绵体内动脉的“交通支”。大多数阴茎海绵体动脉在进入阴茎门之前没有分支,少数可有海绵体外分支走行于海绵体外,除供应同侧阴茎海绵体,也有少量分支穿过白膜供应对侧阴茎海绵体。来自阴茎动脉的1~3条小分支沿侧面进入两条阴茎海绵体脚的背中线表面处,称为阴茎脚动脉。由于阴茎背动脉和阴茎海绵体动脉在海绵体外没有交通支,因此,所有阴茎血管再建手术均需利用阴茎背动脉。由于海绵体动脉在海绵体外部分较长,可达20mm以上,因此手术再建血管时可以利用这一部分进行,而不必分离海绵体内的那部分血管。

19世纪中叶以来,不断有证据表明在阴茎所有动脉分支水平上存在着丰富的动静脉交通支,动静脉交通支的吻合不只发生在白膜外,也发生在白膜之内。有人向阴部内动脉注入直径为100μm的微球,而后在阴部静脉中找到了这些微球,于是在白膜外的阴茎背部空隙中发现了少数这样的动静脉之间的直接连接。在阴茎海绵体中央动脉和与之平行走行于尿道海绵体背侧表面的静脉血管之间分布着另外一种假定为A-V的血管吻合类型。目前认为大多数动静脉交通支应发生在阴茎海绵体之外,这样可为交通支中血液的流通提供最短的距离。

阴茎静脉回流的解剖学很是复杂,尚存在着不少争议,其命名也未能标准化。阴茎的静脉血流共分为3组:表浅的、中层的和深部的。

阴茎背浅静脉汇集阴茎干皮肤和阴茎筋膜以外的皮下组织中大量静脉网的静脉血,它位于阴茎浅筋膜与深筋膜之间,以多根分支形式沿阴茎背侧方向近端走行,这些皮下静脉在耻骨联合附近的阴茎根部汇合成单一的阴茎背浅静脉主干,主要注入左侧大隐静脉,或注入右侧大隐静脉。该静脉也可以向近端分成左右两条主干,分别注入两侧的阴部外浅静脉,继而进入大隐静脉或股静脉。

中层的静脉系统位于阴茎筋膜和阴茎海绵体白膜之间,称为阴茎背深静脉。其主干沿两条阴茎海绵体之间背中线浅沟内向近端走行,在它的外侧与其平行但逆向走行的是一对阴茎背动脉,再向外侧又各有一根阴茎背神经。背深静脉及其背部主要分支的壁较厚,直径3~5mm,管壁外包有肌纤维鞘,在某些区域为单根,有时也呈具有丰富血管吻合的血管网。背深静脉主干穿过两层悬韧带,经骨盆横韧带前界与耻骨弓状韧带之间空隙中的弓静脉进入盆腔,而后分成两根主干注入膀胱前列腺周围静脉丛。背深静脉有时有两条主干,有时主干旁有一小分支伴行,也注入前列腺周围静脉丛。该静脉丛由几根薄壁静脉血管组成,与膀胱和直肠的静脉相似,这些血管与邻近血管之间自由吻合,形成复杂的血管网。然后沿前列腺的前侧面上升,其上有盆内筋膜覆盖,再沿膀胱侧壁走行并汇入膀胱丛,最终汇入髂内静脉。通过海绵体造影有时可发现正常人背深静脉中有1~3个瓣膜,最近的尸解材料指出瓣膜数可多达3~8个,但它们与下肢静脉中所见到的有所不同。其呈双尖状,在静脉扩张时呈漏斗状,这显然与功能需要有关。有1~2个瓣膜位于静脉近端邻近向前列腺周围静脉丛开口处。在远端,静脉瓣的个数和位置的变异较大。静脉瓣的主要结构是纵向分布的肌肉层,但同样存在环行的肌肉层,从而形成一个非常显著的向血管腔内膨出的血管内层。阴茎背深静脉主要引流来自阴茎龟头、尿道海绵体和部分阴茎海绵体(主要是远侧2/3)的血液,它起源于5~8根或更多的来自阴茎龟头的弯弯曲曲的小静脉汇合成的冠后丛。旋静脉来自底面的尿道海绵体的纤细而菲薄的静脉管,仅见于阴茎远端2/3,其数目为3~10支。旋静脉穿出白膜,在白膜外表面沿阴茎侧面斜行汇入背深静脉,主要收集来自勃起组织外周部分的血液。有些旋静脉有由背动脉的小分支形成的同名旋动脉伴行。旋静脉在阴茎背侧面通过两根纵行的细静脉管(阴茎侧静脉)相互交通,侧静脉来自阴茎龟头,沿着阴茎背动脉的两侧走行。旋静脉除与同侧交通,其对侧之间也相互交通。阴茎海绵体的血液通过一系列斜行或垂直穿出白膜的导静脉引流。导静脉大小不一,小的管壁仅有内皮细胞,较大的有纵行排列的平滑肌分部于管壁上,其直径范围约在100μm至数毫米之间。在阴茎远端2/3的导静脉加入并与阴部内静脉有交通支,也与来自深静脉和背深静脉的血管相交通。

前些年报道的通过把阴茎背深静脉动脉化而恢复患者勃起能力的手术就是通过旋静脉和导静脉直接把动脉血注入阴茎海绵体内,其成败的关键取决于这些静脉的数目和直径。此外,由于旋静脉和导静脉与海绵体间隙的直接交通,可能成为阴茎海绵体和背深静脉之间静脉漏的原因,因此它们可能在静脉性ED中具有一定的临床意义。

深静脉系统引流来自3条海绵体的血液。所有深静脉的主要分支都具有较厚的肌性管壁,往往被误认为是动脉血管,所以在结扎之前一定要用超声多普勒探测器反复认真地予以鉴别。深静脉内膜下纵行的平滑肌纤维索向腔内伸出,使内膜外表呈不规则状。这些血管来自白膜内,恰好位于白膜之下,多见于勃起组织的后区,由壁上不含平滑肌的薄壁海绵体后静脉分支汇合而成。位于阴茎海绵体近端1/3区域海绵体间隙和海绵体组织内毛细血管的血液,将通过几条导静脉在阴茎脚分叉处的阴茎门汇合为1~2根较短的海绵体静脉主干(参见前述),在阴茎门中线深处走行于尿道球和阴茎海绵体脚之间,各向两侧走行2~3cm后注入阴部内静脉。其也可注入薄壁的背深静脉,或与前列腺周围静脉丛形成交通支。来自尿道球的小静脉注入海绵体静脉。阴茎海绵体静脉在海绵体外与将进入阴茎海绵体的深动脉及神经伴行,但在海绵体内的深动脉则无静脉与之伴行。两条海绵体静脉之间也存在交通支。在海绵体静脉中存在1~4个双尖形静脉瓣。除了海绵体静脉之外,还有3~4条小的阴茎脚静脉来自阴茎脚的背侧方表面,它们汇成一条稍大的静脉血管进入阴部内静脉。在大多数情况下,存在两条阴部内静脉,它们与阴部内动脉和阴部神经伴行。关于阴茎海绵体静脉排放问题一直是争论的焦点,直到最近,大多数研究人员仍然相信静脉回流主要通过背深静脉。而新方法的研究一直把更多的焦点集中在阴茎海绵体静脉上,尽管这些仅仅是根据放射学检查而发现的。由于阴茎海绵体的主要静脉回流通过海绵体静脉,也由于近来报道它们在静脉性ED患者的静脉漏中占据着重要作用,所以手术结扎海绵体静脉是治疗成功的关键。阴茎海绵体静脉是阴茎门最深最中央的结构,海绵体神经和海绵体动脉都在其表面走行,因此,外科暴露和结扎主要的海绵体静脉时可能危及其他重要的阴茎门的结构。然而,它们在结扎和切除背深静脉之后便会浮现出来,因此从外科上说,接近和结扎阴茎海绵体静脉还是可能的。另一种可供选择的手术方法是结扎阴茎脚,但它不能纠正因海绵体静脉漏造成的ED,只有海绵体造影显示出血管漏发生在阴茎脚静脉时,这一手术才有意义。此外,由于阴茎海绵体脚十分靠近阴茎动脉,所以盲目结扎阴茎脚可能会威胁到阴茎的动脉供血。除了这三组静脉系统外,位于深层的尿道球静脉和尿道静脉回流尿道海绵体近端的血液。尿道球静脉可与深静脉汇合成会阴内静脉。认识到上述主要静脉系统之间存在着无数血管吻合,我们就容易理解阴茎静脉回流的复杂性。此外,互相冲突的文献资料也使我们难以明确回答所有静脉回流的问题。例如,从解剖上说阴茎海绵体的血液将汇入阴茎海绵体(深)静脉,也将通过穿出白膜的导静脉加入到旋静脉或直接进入背深静脉。在对阴茎异常勃起患者的一项临床研究中,阴茎海绵体造影证实背深静脉阻塞,而尿道海绵体造影显示位于阴茎筋膜之外的浅静脉是畅通的,这就进一步表明在背浅静脉和背深静脉之间存在丰富的血管吻合支。要想对正常的和病理条件下的阴茎静脉回流系统具有更明确的认识和了解,就必须更深入、更科学、更广泛地探讨与此有关的临床现象。

阴茎内的勃起组织是由不规则的血窦或窦状隙系统组成的迷路样结构。阴茎三条海绵体内都含有丰富的窦状隙,又称海绵体血窦,它是由海绵体小梁所围绕成的不规则间隙,小梁则形成相邻血窦之间的隔板。小梁及其周围的间隙构成海绵体组织,小梁由平滑肌纤维束、弹力纤维束、胶原纤维和松散而含有小洞的组织构成,各种细胞(包括大量成纤维细胞)掺和起来固着于稀疏的结缔组织上,结缔组织内含许多小动脉、毛细血管和神经。小梁与海绵体组织实质部分的不同之处在于它们的颜色发白,定向有序。白膜的内面固着一些弹力纤维,它们横行或轻度斜行组成丰富的圆柱形纤维组织,通过阴茎深表面进入下面的海绵体内,形成一个精细的海绵体骨架,这些就是所谓的小梁。当阴茎松弛时,它们松散地悬垂于海绵体组织中。海绵体血窦壁表面铺衬着一层由传入小动脉和传出小静脉延续而来的与静脉内皮相似的扁平内皮细胞。窦状壁的厚度变化很大,在中心处较厚,而外周部较薄,据估计在阴茎松弛时平均为1mm左右,而勃起时可达6mm。其中占据数量优势的平滑肌肌束是固定在白膜内表面的。正是这些与海绵体组织有关的组成部分(如平滑肌束、纤维束、白膜、中膈)和结构上的相互关系决定了海绵体在松弛或勃起状态下的自然形状和性能,也决定了它们对勃起反应的依从性和维持勃起的能力。窦状隙内充满血液,该间隙的体积变动范围很大,平时体积很小,勃起时可扩张许多倍,以容纳大量涌入并储积的血液。人们普遍认为这些窦状隙系统是固有的对神经刺激敏感的主动收缩单位,因此,它们在阴茎勃起和肿胀的生理反应中起着十分重要的作用。海绵体间隙赖以回流血液的静脉系统或是直接汇入前列腺丛,或是先进入背深静脉,然后再进入前列腺丛。左右阴茎海绵体均充满血液,血液可以在彼此间通过中膈上的间隙自由来往,所以向一侧阴茎海绵体注入血管活性药物诱发勃起时,药物会很快透入另一侧而发挥同样作用。阴茎的大小可随阴茎勃起组织中血量的多寡而变。阴茎海绵体白膜对于阴茎的可伸展部分如小动脉和窦状隙系统来说是一个屏障,在勃起时它压迫导静脉以减少静脉的流出,这就使阴茎勃起程度逐渐加强,阴茎也由软变硬,最后达到坚挺的状态。尿道海绵体的基本组织形式与阴茎海绵体相似,只是其窦状隙更大、血窦小梁更少,而且小梁更纤细、排列得更规律,其平滑肌细胞要少得多,并填补有丰富的弹力纤维。它为尿道中液体的流动提供了方便,在液体已经排出之后又能迅速地关闭尿道。尿道海绵体的结构有助于“压力室”的形成,在射精第一期,即精液分别自输精管、精囊和前列腺向尿道球部作体内排放时,它可以充分扩张,使精液在向外射出之前的几秒钟内汇集和波动于尿道球部。

阴茎的淋巴分深浅两组。浅淋巴管收集包皮、阴茎皮肤、皮下组织及阴茎筋膜的淋巴,构成了背侧淋巴管,沿阴茎背浅静脉分别流向腹股沟线、深淋巴结。深淋巴管负责收集阴茎龟头、海绵体的淋巴液,流入腹股沟深淋巴结和髂外淋巴结。

当发生性功能障碍时,人们能明确证实的两个最重要的生理系统是循环系统和神经系统。尽管人们早已对海绵体的神经支配(自主神经和躯体神经)有所了解,但近年来的研究使我们对阴茎神经的大体解剖和显微解剖有了更深入的认识。支配阴茎的躯体神经主要来自骶丛短神经之一的阴部神经,它含有运动和感觉两种神经成分。阴部神经自骶髓2~4 节段发出的骶2~4 神经起,不经过盆腔丛,而沿坐骨直肠凹的侧面与阴部动脉伴行,进入阴部管并继续向前,然后分为3对分支:先发出支配会阴区域的肛门神经和会阴神经,其终支即为阴茎背神经,分布至阴茎,阴茎的感觉传入主要经由这一神经;阴部神经也发出纤维支配球海绵体肌、坐骨海绵体肌、尿生殖膈肌和尿道括约肌等会阴部横纹肌;阴茎背神经先沿腹中线在闭孔内肌之上、肛提肌之下走行,经由会阴部穿过会阴骨骼肌,然后从耻骨弓状韧带和阴茎悬韧带之间穿过到达阴茎背部,走行于阴茎背动脉的外侧。除向阴茎龟头、包皮、包皮系带及阴茎其他皮肤供应感觉神经纤维外,阴茎背神经也向阴茎海绵体、尿道海绵体和尿道发出小分支。故包皮环切术时多在阴茎根部对阴茎背神经施行阻滞麻醉。阴茎皮肤和龟头的感觉受体是无分支的神经末梢,对伸展极为敏感,这些感觉受体突触的传入神经将到达骶髓中调节勃起的副交感神经元。会阴和阴囊皮肤的感觉主要由阴部神经的会阴支调节,也有少量纤维来自髂腹股沟神经、生殖股神经和股外侧皮神经的后支。阴茎的自主神经,也就是发出运动神经的自盆神经丛,还发出神经纤维支配其他盆腔器官。盆神经丛位于直肠两侧面的前列腺水平处,既含交感神经纤维,也含有副交感神经纤维。与阴茎有关的交感神经成分来自中枢,在离开胸12 ~腰2 节段的背神经根后,通过以下两条途径走行:①通过交感干的神经节,即椎旁神经节,这部分纤维的突触转换基本上在椎旁神经节进行,节后纤维通过阴部神经和盆神经到达盆腔脏器或仅仅沿通往这一区域的血管走行;②通过腹腔神经丛,即椎前神经节,也包括来自肠系膜下神经节的节前纤维。这部分纤维的突触转换靠近靶器官,也即通过末梢神经节,但也有部分在椎旁或椎前神经节转换,到脊髓的胸12 ~腰2 水平神经节。交感神经成分主要通过腹下丛跨越骨盆缘与动脉血管紧密伴行,通过直肠、邻近器官膀胱和精囊腺侧面到达盆神经丛。而支配阴茎的副交感神经成分则来自骶2~4 节段,通过盆神经(勃起神经)也到达盆神经丛。盆神经丛与前列腺丛和位于盆神经丛最前端的阴茎海绵丛(其中含有一些小神经节)相连接,并沿直肠和前列腺与尿道结合部位的腹侧向前上方走行,恰好位于前列腺囊外方的盆腔侧凹之中。副交感神经的突触转换在局部,即末梢神经节。以上的这些神经紧邻尿道的背侧穿过尿生殖膈,大约在10点和2点位置上由背中线两侧进入同侧的阴茎海绵体。过去认为海绵体神经只是超微结构,在解剖上并不能予以辨别,实际上它们分为海绵体神经支配阴茎海绵体勃起组织和血管系统的平滑肌纤维。这些神经纤维在阴茎门处形成纤维状的神经纤维网,然后汇合形成1~3束沿阴茎海绵体动、静脉走行。躯体神经(骶2~4 )为阴部神经,其中含支配会阴横纹肌的传出运动神经纤维和来自阴茎等皮肤的传入感觉神经纤维。交感神经沿内脏神经的传入纤维径路是自盆丛→交感干→胸11 ~腰3 脊髓后角。节前运动纤维起自胸11 ~腰2 脊髓侧角,经交通支→交感干→腹腔丛→腹下丛(神经)→盆丛,或在交感干下行至交感干骶部。节后纤维起自腰骶交感节和肠系膜下节,经盆丛→前列腺丛→盆部生殖器;或从腰节发支沿精索内动脉到睾丸。其功能包括盆部生殖器平滑肌收缩配合射精;膀胱三角肌同时收缩,关闭尿道内口,防止精液反流;血管收缩。副交感神经的节前纤维起自骶2~4 脊髓骶部副交感核,经骶神经→盆(内脏神经)→盆丛、前列腺丛。节后纤维起自盆丛和前列腺丛的神经节,到前列腺和海绵体的血管。其功能为促进海绵体血管舒张,与会阴神经配合使阴茎勃起。阴茎血管平滑肌和阴茎海绵体组织的运动神经支配都来自腹下神经和盆(内脏)神经,形成盆神经丛。骶髓和腰髓中枢之间的突触传递与联系及与皮质中枢间的联系不明。负责阴茎勃起的主要是副交感神经成分的盆(内脏)神经(是骶丛的另一短神经,其中枢部分在脊髓的骶节),过去有人称之为勃起神经,它们的突触转换发生在终末神经节,如该神经损伤则发生ED。盆内脏神经中含有来自骶2~4 的节前纤维,穿过盆筋膜加入盆神经丛。阴茎的这些运动神经纤维伴随动脉进入所有海绵体中,调节阴茎的勃起反应。人们发现阴茎上有丰富的神经受体,在阴茎龟头上皮、包皮、皮肤、黏膜及尿道结缔组织中发现了游离的神经末梢。除了触觉小体,其他类型的末梢小体包括阴茎龟头真皮和尿道黏膜的生殖小体,以及沿着阴茎背静脉和背神经、特别是龟头结缔组织和阴茎海绵体白膜下方分布的环层小体。人们尚未阐明包含于阴茎海绵体基质中的特殊神经结构,也未鉴别出交感神经、副交感神经和其他调节勃起系统的确切分布和作用机制。

阴茎大小一直是男子十分关注的问题,尤其是青少年,常为自认为过小的阴茎而烦恼,不敢去公共浴池洗澡或游泳,有少数人甚至产生悲观厌世的情绪,丧失生活的勇气。所以我们有必要对阴茎的大小进行探讨。史成礼(1963)曾测定1 412例中国健康男青年在松弛时阴茎的大小及126例健康男性勃起时阴茎的长度。结果表明常态时阴茎平均长度为8.37cm,其范围为4.0~14.5cm,其中20岁组阴茎平均长度7.1cm;平均周长为8.3cm,其范围为4.5~12.0cm;勃起时平均长度10.75cm,其范围为8~14cm。我国一直采用这些数据作为避孕套生产规格的参考依据。刘国振、曹坚(1981)调查了1000例正常成年男子,其松弛状态下阴茎长度平均为6.55cm,最长为10.6cm,最短3.7cm。吴伟成等(1992)报道了2547例16~40岁男性大学新生(包括研究生)阴茎测量数据,其中长度、周径、牵长为实测,勃起长度和周径均自经验公式求出。结果表明,平均长度7.43±1.04,牵长13.33±1.19,勃起长度13.09±1.09。我国男子到16岁时阴茎发育已基本成熟,但在20岁后长度仍有所增加,25岁后则无明显差异。成人阴茎大小虽受地区及年龄因素影响,但差别不大,临床上可以忽略不计。阴茎大小与身高、体重、睾丸大小都呈正相关。胖瘦超过一定程度即影响阴茎发育,如体重低于相应身高计算体重的10\%时阴茎发育差,体重超重20\%时阴茎较短。包茎者阴茎发育差,精索静脉曲张者阴茎较正常人大。阴茎越小,勃起后增大的比率、阴茎长度的增加尤为明显。经台湾地区医生简邦平(2003)报告,台湾地区男子阴茎勃起后平均长11.39cm,而美国男子为12.89cm,巴西为14.50cm,土耳其12.73cm,他还得出勃起长度的计算公式为[身高(cm)×0.06+7.41]×0.65。

综上所述,阴茎的大小受年龄、发育、地区、民族、人种、遗传、身体状况等种种因素影响,从实用观点看,只要它的勃起长度足以插入阴道并完成在阴道内的射精任务,满足生殖与性交的需要就是正常的,否则这一家族便无法延续。


\section{第三节 女性性器官的结构与功能}

女性性器官的功能是繁殖后代以及显示出两性在形态、功能甚至性格、行为等方面的种种特征。女性性器官分为生殖器和乳房,生殖器官分为外生殖器和内生殖器。国内史建、马晓年、潘绥铭等于1993年首次测量了356例妇女的外阴。受试者中68\%为工人,12\%为干部,13\%为职员,5\%为知识分子,其他为2\%,平均年龄37.53岁,分布为16~17岁。已婚者占97.8\%,未婚1\%,离异1\%,丧偶0.2\%。结果表明:阴蒂头长6.73±1.29mm(均值±SD);阴蒂头宽4.73±0.85mm;阴蒂体长21.04±4.55mm;阴蒂体宽3.79±0.87mm(卡尺松松地置于阴蒂包皮两侧外缘处测量,比实际宽度要宽);由前联合至后联合间的小阴唇长71.19±8.80mm;小阴唇宽24.44±7.99mm;小阴唇厚2.06±0.68mm;阴阜至会阴的大阴唇长83.28±10.65mm;大阴唇宽20.20±3.85mm;由后联合至肛门缘的会阴长22.57±4.38mm;阴阜长66.62±10.99mm;阴蒂头至尿道口间距20.36±4.99mm;阴蒂头至阴道口间距32.92±5.93mm;阴蒂根至耻骨联合上缘间距23.45±6.72mm。调查结果还表明,阴蒂头平时能裸露的占25.3\%,不能裸露但能用手推出的占70.5\%,其余4.2\%属包蒂。

女性外生殖器一般指位于耻骨联合下缘与会阴之间所能见到的生殖器官的外露部分,包括阴阜、大阴唇、小阴唇、阴蒂、阴道前庭等结构,统称为外阴(图2-1)。外生殖器的神经分布:大阴唇、小阴唇和阴蒂都含有纤细的神经末梢网和触觉盘,生殖神经小体则多见于小阴唇,特别多见于阴蒂的包皮和阴蒂头。

阴阜为耻骨联合前方以脂肪组织为主的垫子样结构,性交时可起缓冲作用,进入青春期后此处的皮肤上长出卷曲状的毛发,呈盾式分布,称为阴毛。阴毛为第二性征之一,男女的分布范围有所不同,女性阴毛分布呈尖端向下的三角形,其疏密、粗细、色泽可因人或种族而异。

大阴唇是阴道口两旁的一对纵长隆起的结构,前起阴阜,后达会阴,两侧大阴唇前端为子宫圆韧带的终点,后端在会阴体前相触合,形成大阴唇的后联合。大阴唇外侧面与皮肤相同,有皮脂腺和汗腺。青春期长出阴毛,其内侧面皮肤湿润似黏膜。大阴唇有很厚的皮下脂肪层,其内含有丰富的血管、淋巴管和神经。一般女性的大阴唇长7~8cm,宽2~3cm,厚1~1.5cm。未婚妇女的两侧大阴唇自然合拢,遮盖阴道口及尿道口,对之起保护作用;经产妇的大阴唇由于受分娩的影响而向两侧分开;绝经后的大阴唇呈萎缩状,阴毛也减少。女性的大阴唇在解剖上相当于男性的阴囊。

图2-1 女性外生殖器

分开大阴唇后即可看到小阴唇。左右小阴唇在外阴的前上方互相靠拢结合形成阴蒂包皮,包绕阴蒂体。小阴唇表面光滑无阴毛,富含神经末梢,对刺激极为敏感,是性刺激和唤起的重要器官。性兴奋时,小阴唇颜色逐渐加深,充血并外翻,便于阴茎的插入。如在非性兴奋时强行性交,未充血的小阴唇可被阴茎挤入阴道而引起不适,严重时可导致性交疼痛。小阴唇的后端在阴道口后方中线处与大阴唇后端相结合形成一条横皱襞,称阴唇系带或小阴唇后联合。小阴唇的大小和形状因人而异,未产妇的小阴唇往往被大阴唇遮盖,而经产妇的小阴唇可伸展至大阴唇之外。在阴道口和肛门之间的皮肤称为会阴。

阴蒂位于两侧小阴唇的顶端,由海绵体组成,相当于男性阴茎海绵体,具有勃起功能。阴蒂由阴蒂头、阴蒂体和两只阴蒂脚组成。阴蒂头位于阴蒂的包皮和系带之间,属阴蒂显露可见部分,黄豆大小,富有神经末梢,感觉敏感,是女性最敏感的性器官,在性反应方面极为重要;阴蒂体位于阴蒂头后端,包括两个海绵体,在它们的壁中间有平滑肌纤维,表面盖以阴蒂包皮而被包埋,肉眼看不见;阴蒂脚分别起源于左右两侧坐耻支的下面。

阴蒂的结构类似阴茎,两者均有头部与体部;两者大小均因人而异;阴蒂也像阴茎一样有性勃起,这由于阴蒂所含的海绵体充盈血液。阴蒂被富有神经末梢的复层上皮所覆盖,有丰富的神经末梢,对触摸敏感,它对性爱的刺激反应比女性身体任何部位都强烈,为重要的性敏感区,是与人类性欲激发和性感受有关的器官,其唯一的生理功能就是激发女性的性欲和性快感。

阴道前庭位于两侧小阴唇之间,呈菱形。上端为阴蒂,下端为阴蒂系带。中央有较大的阴道口,其前方有较小的尿道外口。在处女的阴道口有半月状或环形状的黏膜皱襞称为处女膜。在尿道口的后方有一对腺体,称尿道球腺,其分泌物有润滑作用,但也是病原微生物易潜伏的部位。阴道前庭内有以下结构:

前庭球位于两侧大阴唇皮下,相当于男性的尿道海绵体,在性刺激和性兴奋时会充血、肿胀。前庭球前端与阴蒂相接,后端膨大,与同侧前庭大腺相邻,表面被球海绵体肌覆盖。

前庭大腺位于前庭球后端,阴道括约肌的深面,形如豌豆,相当于男性尿道球腺,以细小的导管开口于阴道口与小阴唇之间的沟内,相当于小阴唇中下1/3交界处,性兴奋时分泌黄白色黏液起润滑阴道口的作用。正常情况下不能触及此腺体,若腺管口闭塞,可形成前庭大腺囊肿或前庭大腺脓肿。

阴道口是排出月经的出口,也是性交时阴茎进入阴道的入口,其大小和形状各不相同。阴道口位于尿道外口后方的前庭大后部,其周缘覆有一层很薄的膜样组织,称为处女膜,其厚约2mm。处女膜内含结缔组织、血管及神经末梢。处女膜的中央有一裂口,能使经血排出。处女膜的形态、宽窄和伸展性各不相同,其形态有唇状、伞状、半月状、环状和筛状等。在初次性交时,阴茎进入阴道会使处女膜破裂,可引起出血或轻微的疼痛。过去,有人将处女膜是否完整作为该女子是否为处女的标志。但由于组成处女膜的黏膜组织很薄弱,可因剧烈运动如骑马、骑车或外阴受到碰撞、损伤等而破裂,所以部分女性初次性交时也未必出血。分娩后的妇女,由于胎儿的通过,处女膜会出现很深的裂伤,环状结构消失,仅残存结节样处女膜残痕,称为处女膜痕。

尿道口位于阴蒂头后下方,其后壁有一对并列腺体,称为尿道旁腺。尿道旁腺开口小,容易有细菌潜伏。

女性内生殖器位于真骨盆内,包括阴道、子宫、输卵管和卵巢(图2-2)。

图2-2 女性内生殖器

阴道位于真骨盆下部的中央,为一上宽下窄的肌性管道,是性交器官,在性交时接纳阴茎并存放精液,也是月经排出和胎儿娩出的通道。阴道前壁长7~9cm,与膀胱和尿道相邻;后壁长10~12cm,与直肠贴近。上端包围着子宫颈阴道部,下端开口于阴道前庭后部。环绕子宫颈周围的部分叫阴道穹隆,可分前、后、左、右四部分,后穹隆较深,与盆腔最低的直肠子宫陷凹紧密相邻,临床上可经此穿刺或引流。在男上位性交时精液可储存于此处,形成精液池,由于正常位置的宫颈外口常位于后穹隆处,因此这种解剖关系有利于精子进入宫腔。

阴道壁由内向外由黏膜、肌层和纤维组织膜构成,富有伸展性,妊娠时肌纤维长度可增加4~5倍,便于生产时胎儿娩出,以及性交时容纳勃起的阴茎。在静息状态下,阴道前后壁相贴,在性兴奋时扩张、膨胀,以便容纳勃起的阴茎。阴道在性交时可依照阴茎的大小恰好地贴附着阴茎,感受阴茎的刺激。阴道肌层在性交达高潮时可产生节律性收缩,导致性交时的快感。

阴道黏膜由复层扁平上皮覆盖,无腺体,淡红色,有很多横行皱襞,有较大的伸展性。阴道黏膜上皮受性激素的影响有周期性变化。肌层由内环外纵两层平滑肌构成,纤维组织膜与肌层紧密粘贴。在性兴奋时阴道周围的小血管高度充血,可从血管内渗出较多的液体,使阴道黏膜润滑,避免性交时的摩擦而损伤阴道壁。性兴奋过程中,阴道下1/3可发生显著充血,引起阴道口缩窄;阴道上2/3段的神经支配来自自主神经系统,对痛觉与触觉不敏感;阴道下1/3段则受阴部神经支配,而阴道口神经分布丰富,故对刺激极为敏感。女性G点位于阴道下1/3的区域,为阴道内最重要的动情区。

子宫位于直肠和膀胱之间,呈一倒置的梨形,重约50g,长7~8cm,宽4~5cm,厚2~3cm,容量5ml,是一个有腔壁厚的肌性器官,是孕胚胎、胎儿和产生经血的部位。子宫可分底、体、颈三部分。子宫上部较宽,为子宫体,子宫体的顶部为子宫底,子宫底的两侧为子宫角,与输卵管相通。宫体下部较窄,呈圆柱状,称子宫颈,颈与子宫体相连的狭细部称子宫峡,长约1cm。子宫颈可分上下两部分,上2/3位于阴道以上,称为子宫颈阴道上部;下1/3则伸入阴道腔内,称为子宫颈阴道部。

子宫体内部称为子宫腔。呈倒置三角形,基底向上,尖角向下。基底的两角有输卵管的子宫口。下部位于子宫颈内称子宫颈管。子宫颈管呈梭形,上口为颈管内口,通子宫体腔;下口称颈管外口,即子宫口,通阴道。子宫口在未产妇为圆形,分娩后呈横裂形。

子宫体的壁由三层组成,由内向外分为别由子宫内膜、肌层和浆膜层。子宫内膜分为三层:致密层、海绵层和基底层。内膜表面2/3为致密层、海绵层,统称为功能层,受卵巢性激素影响,发生周期性脱落和流血,即月经。基底层为靠近子宫肌层的1/3内膜,不受卵巢性激素影响,不发生周期性变化。子宫肌层较厚,在非孕时厚约0.8cm,由大量平滑肌束和少量弹力纤维组成。肌纤维纵横交错排列,含有丰富的血管,分娩时,子宫平滑肌的收缩成为胎儿娩出的动力,产后子宫逐渐复原。子宫浆膜层又称子宫外膜,即在子宫表面的薄层腹膜。在子宫前面,近子宫峡部处的腹膜向前反折覆盖膀胱,形成膀胱子宫凹陷。在子宫后面,腹膜沿子宫壁向下,至宫颈后方及阴道后穹隆再折向直肠,形成直肠子宫凹陷。

子宫颈主要由结缔组织构成,含有平滑肌与血管,黏膜层内有腺体,能分泌黏液,与子宫体的黏膜不同,宫颈黏膜在月经周期中不会发生周期性脱落现象。

子宫位于盆腔的中央,前为膀胱,后为直肠,下接阴道,两侧有输卵管和卵巢。宫底位于骨盆入口平面以下,宫颈外口位于坐骨棘水平稍上方。当膀胱充盈时,成人子宫的正常位置呈轻度前倾前屈位,主要靠子宫韧带及骨盆底肌肉和筋膜的支托作用。如果子宫韧带、骨盆底肌肉和筋膜薄弱或受损伤,可导致子宫脱垂。

输卵管位于阔韧带上缘,内侧与子宫角相通,外侧游离呈伞状开口于腹腔内,全长8~14cm,为一对细长弯曲的肌性管道。输卵管的管腔较窄,其内膜层有纤毛,卵子在输卵管的壶腹部受精,受精后再经输卵管向子宫腔运行。根据输卵管的形态,由内向外可分为四部:①输卵管间质部:为贯穿子宫壁的一段,经输卵管子宫口开口于子宫腔,长约1cm。②输卵管峡部:在间质部外侧,是紧接子宫壁外面的一段,细而直,管腔较窄,长2~3cm。③输卵管壶腹:在输卵管峡部的外侧,管径粗而较弯曲,内含丰富的皱襞,卵细胞受精即在此进行,长5~8cm。④输卵管伞部:为输卵管外侧端的扩大部分,长1~1.5cm,其末端有许多指状突起伸向卵巢表面,有“拾卵”作用,手术时常以此作为识别输卵管的标志。

输卵管管壁由外到内分为浆膜、肌层和黏膜。浆膜层为腹膜的一部分;肌层为平滑肌,该层肌肉的收缩有协助拾卵、运送受精卵及一定程度地阻止经血逆流和宫腔内感染向腹腔内扩散的作用;黏膜层由单层高柱状上皮覆盖。上皮细胞分为纤毛细胞、无纤毛细胞、楔形细胞和未分化细胞。纤毛细胞的纤毛摆动,能协助运送卵子;无纤毛细胞有分泌作用,又称为分泌细胞;楔形细胞可能是无纤毛细胞的前身;未分化细胞又称为游走细胞,是上皮的储备细胞。输卵管肌肉的收缩和黏膜上皮的形态、分泌及纤毛摆动等均受性激素影响而存在周期性变化。在雌激素的作用下,肌层可作节律性蠕动,加上黏膜纤毛摆动而将卵子或受精卵推送到子宫腔。

卵巢为一对扁卵圆形的实性器官,由外侧的骨盆漏斗韧带和内测的卵巢固有韧带悬于盆壁与子宫之间,借卵巢系膜与阔韧带相连。卵巢前缘中部有卵巢门,神经血管通过骨盆漏斗韧带经卵巢系膜在此出入卵巢,后缘游离。卵巢大小随年龄而不同。幼女卵巢较小,表面光滑,性成熟期最大;青春期开始排卵后,其表面逐渐凹凸不平;35~40岁开始缩小;40~50岁随月经停止而逐渐萎缩。成年女性卵巢平均大小约为4cm×3cm×1cm,重5~6g。卵巢的主要功能为产生卵子及合成、分泌性激素。

卵巢表面无腹膜,由单层立方上皮覆盖,称为生发上皮。上皮之内有一薄层致密结缔组织,称为白膜。卵巢的切面可分为皮质和髓质,皮质在外,髓质在内。皮质中有处于不同发育阶段的各级卵泡,成熟卵泡排卵后,卵泡细胞形成黄体,黄体退化萎缩后形成白体及一些生长卵泡退化成的闭锁卵泡等。髓质无卵泡,由疏松结缔组织、血管、淋巴管和神经组成。

乳房是女性第二性征之一,于青春期开始发育生长,尽管它不属于生殖器官,但在性活动中很有意义。对男女性来说乳房都是重要的性敏感区,在性反应中亦有变化。

未生育成年女性的乳房形态呈扁圆形、半球形或圆锥形。乳房中央的突起为乳头,其顶端有输乳管的开口。乳头由平滑肌纤维组成,当平滑肌收缩与局部血管充血时可引起乳头竖起。乳头富含神经末梢,刺激它对性欲的唤起起着重要作用。乳头周围有色素沉着的区域叫乳晕。乳房由腺体组织、结缔组织与脂肪组织构成,其中结缔组织与脂肪组织所占的比例很大,乳腺多半位于乳房的中心区,外周为脂肪。人的乳房约由15或20个乳房小叶组成,叶间有致密结缔组织隔开,周围有脂肪组织包围。每个小叶由10~100个腺泡组成。每个小叶均有一小导管,许多根小导管汇成一根大导管即输乳管,每个乳腺小叶都有一根输乳管,它独立开口于乳头顶端。

儿童时期,男女的乳房没有什么差别,此时的乳腺几乎只有导管而没有腺泡。进入青春期后,女性在卵巢分泌的雌、孕激素的作用下,乳房变得丰满、成熟、富有弹性。雌激素可以促进乳腺导管增生,孕激素促进乳腺腺泡发育。由于卵巢激素在月经周期中呈周期性变化,因此非孕妇女乳房的体积也可能有周期性变化,表现为经前期感到乳房稍增大,这与乳腺增生和局部充血有关。在妊娠期和哺乳期,乳腺增生,乳房体积明显增大,为哺乳做好了准备。停止哺乳以后,乳腺萎缩,乳房变小,并失去了一部分弹性。即使没有妊娠,30岁以后妇女的乳房也开始慢慢地变小,弹性下降。更年期后,妇女的乳房变得更小、更为松弛。

在心理学方面,乳房有着重要的意义;在两性接触中,女性乳房具有极大的异性吸引力,男性往往渴求抚摩女性的乳房,而女性也渴求被抚摩。有些女性仅仅由于对乳房,特别是对乳头的刺激即可性唤起或产生性高潮。乳房在性反应过程中也有很明显的变化。在性兴奋期,乳房会充血肿胀;在出现性高潮后,充血迅速消退,乳房回复到原状。如果女性的性欲长期受抑制,达不到性高潮,乳房便会充血消退得很慢,长时间的充血就会造成了乳房胀痛。

阴蒂在古代医学文献中早有描述,称之为“阴核”、“玉台”、“臭鼠”、“琴弦”、“雏先”、“鸡舌”、“阴豆”等,清末有人称之为“挺孔”,似男子之阴茎,而无尿道无海绵体,其物甚易感,空洞而勃起者也。又其位置,恰在于尿翅(小阴唇)相合之处”。虽知其功能“甚易感”,但对它的功能描述却不是很多。阴蒂在不同历史时期和不同社会中的遭遇截然不同,这与社会文明及女性在社会中的地位密切相关。如在英国维多利亚女皇时代,西方一些国家或地区对性行为尤其是女性性行为完全持否定态度,曾经采用阴蒂切除术以公开治疗“强迫手淫”的妇女,实际上这是一种残酷的性虐待、性迫害行为。而在非洲、拉丁美洲、大洋洲和中东的一部分地区,至今仍盛行女性环切手术(即阴蒂及部分阴唇切除术,多在女性进入青春期时由部落中的长者用土法实施,甚至还把女阴缝合起来,待结婚时再由丈夫打开),为的是防止女性出现性放纵行为,并保证女性能稳定于家庭中不断地生儿育女,目前全球约有8000万妇女做过这种惨无人道的手术。虽然世界卫生组织早已禁止这种对女性的残害行为,但却很难得以完全控制。

介于阴蒂在女性性高潮中的作用,要想有效治疗女性性功能障碍或使女性获得性满足,我们就不得不对它进行详细描述和深入探讨,不得不进一步了解和评价阴蒂在性反应中的作用。过去流传着种种有关阴蒂的误解,使人们对阴蒂的作用始终认识不足,例如在英国文豪莎士比亚时代和以后一段时间里,阴蒂这个词就从未被使用过。我们首先应该把对阴蒂作用的准确描述归功于性解剖学家迪金森和皮尔森在20世纪30年代初的努力,他们使用了100名成年妇女的测量数据大胆提出了一系列有关阴蒂的问题,并根据实测的统计作出回答。前面的介绍已回答了阴蒂对性刺激反应的一些根本性问题,如:①在给予性刺激时阴蒂发生了哪些解剖改变?②阴蒂的性反应是否具有统一的模式,并能与性反应周期的四个阶段对应起来?③不同刺激方法是否导致不同的性反应?

阴蒂富有感觉神经末梢,感觉敏锐,是女性最敏感的性器官,在性反应方面极为重要,它也是人类唯一只与性功能有关的器官。阴蒂位于两侧小阴唇会合处的顶端,分为三部分:前端为阴蒂头,中间为阴蒂体,后部为阴蒂脚。阴蒂的位置与阴蒂脚在耻骨联合前界的起始点和尿道口之间的距离有关,但从解剖学角度很难准确指出阴蒂脚在耻骨联合前界上的准确附着点,也不可能描述该点与尿道口的关系,故其临床意义尚待证实。阴蒂头很小,位于阴蒂前部,多露于阴蒂包皮之外,属阴蒂的可显露部分,其长宽均2~4mm,但个体差异很大,即使直径达10mm也属正常范围。阴蒂头和阴蒂体相加长20~25mm,头与体的比例多成正比,即细长的体具有纤小的头,或短粗的体有较大的头,但彼此的大小也可能成反比例。阴蒂体位于阴蒂头后端,包埋在阴蒂包皮之下,有时,阴蒂包皮还将阴蒂头也包裹起来。小阴唇的前端各形成内外两个小皱襞,外侧者在阴蒂背侧汇合形成阴蒂包皮,内侧者在阴蒂下方与对侧汇合形成阴蒂系带,向上连于阴蒂。阴蒂体由两条不完全分离的阴蒂海绵体组成,其外面像阴茎海绵体一样包裹有一层白膜。阴蒂海绵体的白膜含有一定量的弹性纤维和平滑肌束,纤维膜沿着海绵体中央面的表面聚集并形成一个梳状中膈,两侧海绵体间可以有血液交通。阴蒂也像阴茎一样可有性勃起,这是因为海绵体充盈血液所致。阴蒂海绵体向后延伸形成相互分离的两条粗大的阴蒂脚,它们分别附着于左右坐骨支和耻骨下支上,但固着点的位置有个体差异。阴蒂的固着除阴蒂脚的作用外,尚有阴蒂悬韧带,其上端沿中线中膈附着在耻骨联合前表面,位点可在上界与下界之间变动,因人而异;另一端与阴蒂体的基部和阴蒂脚的中间部的白膜相连。阴蒂脚的长度约为体的2倍,直径也粗得多,阴蒂脚是阴蒂勃起的重要部分。阴蒂脚上覆盖有一对小肌肉———坐骨海绵体肌,它们起自两侧的坐骨支,止于阴蒂脚的下方。另外还要注意到女性尿道口是在阴蒂和阴道口之间,而男性尿道口是在阴茎龟头顶端,这点男性与女性不同。女性排尿不是通过阴蒂也不是通过阴道,而且通过一个把阴蒂及阴道隔开的通道———尿道,尿道具有自己单独的开口———尿道口。

阴蒂的神经支配与阴茎相似。不过,阴蒂背神经非常纤细,是阴蒂神经最深的分支,它终止于阴蒂头和阴蒂海绵体本体内的神经末梢丛。贯穿阴蒂头和海绵体的躯体神经纤维上不规则地分布着环层小体,环层小体主要与本体感受刺激有关,阴蒂头上环层小体的密度最大。一般来讲,供应感觉末端器官的有髓神经纤维的直径与终末器官的大小有关,纤维传导速度一般也与其直径密切相关。阴蒂的血液供应来自阴部内动脉的分支:有阴蒂背动脉,它主要供应阴蒂头和体部,还有球动脉供应前庭球,深动脉供应阴蒂脚。进入阴蒂海绵体的动脉包括螺旋动脉,它控制流入的血液,经过曲折途径直接进入静脉血窦隙;此外还包括营养动脉分支,直径细而血流少,它们供应小梁的自身所需。动静脉之间像阴茎一样存在交通支:球静脉、海绵体静脉和背深静脉负责汇集和运送来自前庭球和阴蒂海绵体的静脉血液至阴部内静脉或邻近的一些静脉丛。阴蒂的肿胀、勃起和变硬都会造成动脉血流(尤其是螺旋动脉)的增加、动静脉交通支的关闭及静脉回流的阻断。阴蒂血液的流入与流出的控制机制十分复杂,也非常独特,与其他内脏器官皆不同,唯独与阴茎相似。因为阴蒂在每晚也有数次的夜间勃起现象,很多妇女可以在晨醒或睡眠中体验到阴蒂肿胀、勃起和抽动的感觉,这样的高潮反应大量见诸于报告之中。阴蒂背动脉和背深静脉及其分支中的平滑肌成分比其他血管中多,血管的环行肌和纵行肌的分布也与众不同。值得注意的是,阴蒂动脉分支血管内有一种血管垫样物质存在,它包括内皮下纵行的肌纤维条束,其上覆盖有一层长形纤维细胞层,外面包被有肌上皮细胞,看上去像小囊,但缺乏肌纤维,最外面是具有环形纤维细胞的基质。这种结构与男性阴茎动脉内的血管垫相似,又称球瓣。球瓣往往位于动脉血管壁的两侧,彼此对应,有时排列稀疏孤立存在,有时排列密集彼此相邻。当基质中的环行纤维和这种球瓣结构收缩时,血管腔变细,流入的血流则锐减,反之,当它们松弛时,流入的血流则增加。

阴蒂海绵体的基本组织结构与阴茎海绵体有雷同之处,无非是由血窦和小梁组成,神经纤维放射分布于小梁之中,起支配与调节作用。如果在扫描显微镜下把海绵体组织放大了观察,可见小梁错综复杂,粗细不等,简直像进了热带丛林。小梁有肌性与纤维性之分,可含平滑肌纤维、结缔组织纤维、弹力纤维等,小梁是海绵体的支架和勃起组织。海绵体外周的小梁呈短粗束状,其中含有更纤细的纤维束,方向多为纵行。小梁的壁多由肌纤维组成,周围包绕有网状结缔组织,阴蒂海绵体的弹力纤维多集中于外周小梁,因此外周的收缩力强。在中央区,主束主要由纤维和结缔组织组成,小梁的数目和大小增加,可见皮下纤维细胞。中央区小梁的方向各异,小梁数目和大小增加。勃起组织在人生中会经历从幼年到成年的增长增粗过程,也将经历衰老时血窦间隙不断增大而小梁不断变细、肌肉成分不断减少的变化,这时小梁中的网状结缔组织和弹力纤维将被纤维结缔组织和瘢痕组织所取代,粗大的纤维将成为散乱的纤维网络,它们的勃起能力自然逐渐消退。

阴蒂的大小可能受遗传及内分泌因素的影响,它的外表形态与邻近组织的发育状况相关。虽然阴蒂头的皮肤、包皮及小阴唇可能因频繁的性活动刺激而出现皱褶、增生和慢性充血,但阴蒂海绵体组织不会出现这种反应性增生。有人总认为妇女手淫可导致阴蒂肥大,这是毫无根据的,同样,男性总顾虑自己手淫会导致阴茎萎缩或影响了其正常生长也是毫无道理的。有些年轻女性常常担心自己的外生殖器发育异常,喜欢在洗澡时与同龄人相比较,从而产生了严重的心理负担。如一个女孩说:“我一直是一个发育较好、成熟较早的女孩,后来在洗澡的时候,惊讶地发现我的外生殖器与别人不同。我偷着查看了许多书,仍不能明白,只知道我的阴蒂好像大一些,翻出阴唇外,但不知是什么病?有的书谈到有‘阴蒂肥大’这种病会影响以后的夫妻生活和生育,我很害怕……”当然,这种情况究竟是不是病态,还需对其进行体检后再作具体分析。

女性阴蒂的大小、形态和位置的个体差异很大。少数人因长期过量服用雄激素或体内雄激素分泌过多而造成异常肥大,个别阴蒂头还分成两半,只是在根部相连。凡是能够影响身体其他器官系统的病理过程也同样会影响到阴蒂。这些病理改变包括感染、肿瘤、炎症、萎缩、瘢痕等。少女的生殖器官异常并不少见,如果染色体,特别是性染色体基因在排列上发生异常,则会出现病态或畸形。此外,在胚胎发育过程中,受到内外环境的干扰,如母亲患病、错服药物、X线照射、吸入有毒有害气体或粉尘等,也可以引起胚胎发育异常而出现畸形。这种由于胚胎发育的变异而引起的畸形如果表现在生殖器官上,则表现为生殖器官的异常。有时在胚胎发育时受到母亲过多性激素的影响也会产生生殖器官畸形。所以有些少女的生殖器官出现无阴道、直肠或尿道开口异常、两性畸形、处女膜闭锁、子宫发育不全、卵巢发育不全等。当内分泌器官发生异常时,可以出现女性男性化。性腺是决定性别的主要器官,它们分泌的性激素在性别决定上起主要作用。另外,肾上腺皮质的网状带也会产生一些与性别相反的性激素。若肾上腺患肿瘤或组织增生等病变时,便会不受脑垂体的控制,分泌大量的性激素。当患有先天性肾上腺疾病,如先天性肾上腺皮质增生、皮质激素合成酶缺乏、常染色体遗传病等,也会引起性激素的大量分泌。在女性,肾上腺皮质分泌的大量雄激素进入血液,使卵巢分泌的雌激素相对较少,可出现女性男性化的现象,表现为女性特征消失,出现男性特征。最早的改变,往往是长出胡须、四肢汗毛加重、阴毛增多而浓密,并且向上蔓延到脐孔下面,头发脱落甚至秃顶。最后相继出现乳房萎缩、月经停止、喉结增大、声调变得低沉、肌肉发达、子宫缩小、阴蒂增大和性欲消失或亢进等症状。心理上也会出现反常,易激怒、爱冲动、对男性的态度由喜爱逐渐变为讨厌直至憎恨。若是由先天性肾上腺疾病引起的女性男性化,青春期就直接表现为以上男性化特征。

外生殖器异常,也有两性畸形的可能性。两性畸形可分为真、假两性畸形。真两性畸形患者体内既有睾丸,又有卵巢,分泌的雄、雌激素数量都不少,两者势均力敌。外生殖器长得不男不女,前面有一个小阴茎,后面又有两片分开的大阴唇,中间还有一个小阴道开口,连接里面发育的子宫。假两性畸形是指身体里只有一种性腺,并非二者并存,只是生殖器官长得不男不女。如果体内有卵巢,称为女性假两性畸形,表示此人应属女性,但外生殖器不像女性。

综上所述,阴蒂过大等外生殖器异常究竟是否算病态,还要通过对女性其他第二性征进行检查,以及对体内性腺和肾上腺的检查才能作出正确的判断。如果仅仅是阴蒂较大,而第二性征无明显异常,月经正常、规律,则属阴蒂的正常变异,不会影响生育。若诊断发现有肾上腺疾病或两性畸形,则需进行治疗。

阴蒂的性生理反应机制和行为表现具有以下特点:

(1)阴蒂的两种生理状态:一般来讲,手指尖或口唇等部位的感觉和敏感性不论在什么时候都是相对恒定的,而生殖器官,特别是阴茎和阴蒂则与此相反,具有定性和定量两种不同的感觉状态,即唤起状态和非唤起状态。非唤起状态指没有受到任何性刺激,或性刺激仅有很小的性意义或价值;二是性唤起状态,当性刺激的强度和有效性明显增强时,阴蒂进入性唤起状态,除接受外部刺激之外,阴蒂也受自身肿胀的刺激。阴蒂的自身肿胀在性唤起到性高潮的进程中起正反馈作用,改变了特异神经末梢的感受性。

非唤起的性器官皮肤敏感性与周围皮肤的区别很小,除非局部有特别丰富的神经末梢集结。然而一旦肿胀之后,这些器官的感觉传入手段和敏感性均显著改变,这种突然发生的改变带有浓厚的动情和性特征色彩。带有性意义的感觉性质是不好定义的,一种可能是分布到皮肤表面小纤维的外阴感觉小体对肿胀的组织压力作出反应,成为感觉模式的放大器。值得令人注意的是,儿童期的消极性教育和惩罚可以抑制最初的性唤起感觉,阻止唤起—肿胀—唤起的环形反馈作用,性反应反射弧不能建立。这就提示,对无高潮妇女来说,应进行符合神经生理反应模式的技巧训练,以早日打通这一反射通道。

国外有人曾观察到,如果以振荡器作为人工模拟性交的手段,可以实验了解阴蒂功能与完美性行为之间的关系。当振荡频率达到60/s时,可以在不直接接触阴蒂的情况下给阴蒂带来强烈刺激。刺激主要作用于有被囊的感觉神经末梢而不是一般的传入神经纤维,但个体差异十分显著。一般刺激10s,间歇10s,再刺激10s,如此反复进行。刺激部位多选在阴阜或阴蒂干,用敏感的传感器记录阴道口和阴道内深处的压力变化,以此作为生理反应信号记录下来。在实验中的直接观察和手指触诊能使信息量增加。

(2)正反馈的重要性:过去的行为学研究往往忽视了正反馈的作用,如在吃饭、交配、战斗等活动中,行为的起动往往激发了一个正反馈机制,它使所选择的行为得到强化,这对行为的完成十分重要。性唤起的第一个信号就是生殖器的充血肿胀,男子对阴茎勃起肿胀的主观感受显著强于女性阴蒂肿胀带来的感受,因此男子的正反馈作用显然强于女性。阴茎勃起是最具有性意义重要感觉,即使在缺乏其他刺激的情况下也能感受到它本身的变化。而阴蒂则只有用较弱的感觉来进一步支持性反应的发展,这也恰恰是造成男女性反应差异的一个重要因素。

(3)在性唤起中的正反馈机制:上述正反馈机制趋使男女进一步坚持性行为的进展,来自靶器官的这种反馈机制能不断强化行为顺序,这种行为或行动顺序不是严格选定的、刻板的,它可以有数种变化。阴蒂这种较弱的反馈作用致使人们对它缺乏必要的关注,例如马斯特斯和约翰逊的前驱工作只对阴蒂性反应的客观特征作了详细描述,而没有讨论妇女在性活动肿胀后所知觉的感觉类型,即他们没有把主观感觉理解为生理性的。我们应该考虑到涉及神经机制的所有现象是神经生理性的,有些感受只能通过询问受试者才能了解到。有些可靠的受试者在回答问题时谈到,性唤起的第一个信号是阴蒂的震颤感觉,这对阴蒂本身又是一个高度有效的动情刺激,这实际上有助于消除她对投身性情境的焦虑,除非有其他刺激或形势使她分心。当性活动进一步深入时,这种跳动的感觉会被一种电流掠过的感觉所取代,这又反过来强化了震颤带来的感觉。个人反应的强度差别很大,每个人在不同的时间和环境里反应的强度也有所不同。有的人可能出现痛性震颤的感觉,“我突然觉得会阴部的肌肉全部挛缩了”。阴蒂的这种跳动成了生殖器肿胀过程产生的感觉强度的指征和检测指标,这时虽然没有实际的性交,但她们势必对性进入关注和投入的阶段。对阴蒂反应的测量对谁来说都是极其困难的任务。可以假设勃起器官的压力对有被囊的阴蒂神经末梢来说是一种刺激,同时周围组织充血后的压力和动脉搏动加重更加重了报告人的这种印象。

阴道内2/3的部位对疼痛与触觉都不敏感,因为它的神经支配来自自主神经系统,而且这些神经只通过其环层小体参与局部的血循环控制和深感觉,对其他刺激很不敏感。相反,阴道外1/3和尿道下方由于受到阴部神经敏感性的支配,可以感受到机械性刺激。G点恰恰位于这一区域,于是成了阴道内最重要的动情区。从整体上看,阴蒂虽是女性性兴奋的中心器官,但在不受刺激的情况下,性反应仍可以一直发展为性高潮。性高潮部分是生理性的,部分是心理性的,甚至截瘫妇女也可以获得性高潮,有些妇女仅仅通过性幻想也能达到高潮,即使在供应阴蒂等部位的神经支配完全中断的情况下。高潮的决定器官既不是阴道也不是阴蒂,既不是G点也不是别的什么部位,而是大脑。由男性变为女性的变性人,尽管可肿胀的部分已切除,假阴蒂也没有背神经的支配,但由于皮肤中具有压力感受器,其心理准备也十分充分,所以仍然能够通过手淫或阴道内性交达到高潮。

在女性性唤起能力和高潮能力的形态与组织学研究中,妇产科医生发现早期妊娠的一个体征具有某种启示,即从妊娠第6周起,尿道口下方至阴道前壁之上会出现一条青黑色的宽带,这就是解剖学上的阴道尿道隆凸,它是阴道前壁下方的皱褶柱,紧贴于尿道后壁,似乎与格拉夫伯格等描述的G点是一回事。不过人们对这一区域的描述极不准确,隆凸向两侧转变为阴道侧壁的陷凹并向下延伸到阴道口。大体上说,尿道隆凸附近没有凹陷,其下方也没有膨出。虽然妇女在未妊娠时的隆凸具有一定大小,但在妊娠时可增生肥大到橄榄大小,当妇女受到性刺激时它们也将肿胀。这些隆凸的形态和大小变异甚大,只有半数妇女相当明显,其余则并不明显,有时仅有些迹象而已。组织学检查表明隆凸上覆盖着角化的复层上皮,其组织由间质构成,多无皱褶,纤维分布稀疏。值得注意的是,平滑肌纤维不规则地分布其间。除具有陷窝性质的血管外,还有相对广泛的较大血管,部分血管具有很厚的肌肉管壁。因此血液供应很丰富,血管壁相对较厚是可肿胀体的特征。神经束在进行多种结缔组织染色时很常见,但并未发现神经末梢等其他神经结构的明显增加,邻近它们的是尿道旁和尿道周腺体,这些腺体的导管开口于尿道。

从性生活的角度而言,人们比较关心阴道,因为它与性生活直接关联。人们谑称其为“洞”或“桶”,实际上它并非一个洞穴,而是一个封闭性空间,它的功能有点像气球,平时阴道壁是彼此紧贴的,需要时却可以充分扩展。阴道实际上是一个极富弹力的肌肉器官,能收缩也能舒张,收缩时连一根小手指头也插不进去,而舒张时阴茎可以随意插入或抽动,甚至可以容纳胎儿和允许胎儿从中通过。女性阴道的松紧程度大不一样,未婚女子的阴道比较紧,肌肉富于弹性,生过孩子之后,阴道壁变得松弛,阴道也就比较宽松了。许多人对阴道的舒缩潜能缺乏了解,男性出于对情侣的关心,害怕阴茎的插入会对女方的身体造成损害,然而女性阴道所具有的伸展调节潜能有时是有相当惊人,弹性极强的阴道在性交时可以依照阴茎的大小恰好贴附于阴茎。也有人担心阴道太狭小会限制阴茎的插入,其实真正的小阴道和小阴茎一样少见,关键在于女方在插入时的反应状态,只要女方高度兴奋,那么再大的阴茎都不会成问题,相反,如果女方根本没有兴奋,哪怕很小的阴茎也会遭到麻烦。长期禁欲或绝经后的妇女,在重新开始性生活时其性反应是缓慢的,因此在女方高度兴奋之前不要贸然插入,否则只会给她带来痛苦而非快乐。

弗洛伊德提出的女性性成熟是以女性性高潮从幼年的阴茎型向成年的阴道型的过渡为标志,也即阴道究竟在女子的性高潮和情感受中究竟起什么作用的观点引发了一场长久的争议。早先的研究人员对阴道的性感受能力表示怀疑,因为阴道内壁缺乏神经末梢,故对触摸并不敏感。相反,会阴部特别是阴蒂却分布着密集的数种不同类型的神经末梢。它们中有些对触摸十分敏感,比如触觉小体、触盘、默克尔触觉小板神经末梢;也有对机械性变形敏感的环层小体;对温度敏感的鲁菲尼小体和多纪尔—克劳泽小体;及对痛觉敏感的游离神经末梢。

在这些解剖和组织学观察的基础上,人们总是把阴蒂看做是成熟妇女性感觉的主要来源。通过多项调查证实:50\%以上的妇女不能在单纯阴道性交(不刺激阴蒂)的情况下获得高潮,这一事实显然支持上述观点。然而,那些能够对阴道刺激作出反应并获得性高潮的妇女的确能够区分阴道高潮与阴蒂高潮的不同,这显然与马斯特斯和约翰逊的主张不相符合。阴道各段对性刺激的反应不同,如阴道外段1/3系由外胚层分化而来,富含神经纤维,所以对于触摸有反应的神经末梢只集中在阴道口附近;阴道内段2/3来自中胚层,没有神经末梢分布,所以阴道外段1/3要比内段2/3更富有性感觉。就阴茎而言,虽然松弛时大小差异较大,但勃起后的差别便减小,一般来说勃起后都能大大超过阴道外段1/3这个深度,所以女性性满足的决定因素决不是阴茎的大小粗细。当然,阴道内段2/3的挤压感觉还是较强的,所以当阴茎深入时,女性是有所感觉的。有人发现,在用手指对阴道前壁进行揉压刺激时,通过受试者下腹部向下施加轻压则可持续产生达到高水平性唤起时的动情感受。另一项研究则在48名能在性交中获得高潮的妇女中进行,研究人员通过用手指对阴道不同区域进行刺激来了解阴道的动情机能。平均刺激时间为20分钟,其中45人(占94\%左右)主诉阴道有性敏感性,30人达到性高潮或是在达到性高潮之前就要求停止刺激。实验证实敏感区多集中在阴道前壁的偏上端。在另一项中,研究人员向美国2350名从事卫生和咨询工作的职业妇女发出有关阴道敏感性问题的调查问卷,有1289名妇女寄回答卷,应答率为55\%。84.3\%的应答者认为阴道内存在性敏感区域,只要刺激它们就能引起性快感,不过有70\%的应答者认为它们只存在于部分妇女中,65.9\%的应答者认为自己的阴道内存在这样的敏感区域,其中72.6\%的妇女在接受对这个敏感区域的刺激时能经历性高潮。知觉到的敏感点的位置在前壁的占55.1\%,在后壁的占7.3\%。应答者最普遍的回答是,不论敏感点位于前壁还是后壁,它主要位于阴道上段(46.1\%),而认为在阴道下段的占21.5\%。

刺激阴道内敏感区不仅可以产生性感觉、激发性高潮,还能引起痛觉缺失产生镇痛作用,使妇女更能耐受手指刺激的压力,在获得快感时能使疼痛耐受阈值提高40.3\%,疼痛检测阈值提高47.4\%。当阴道刺激出现快感时,疼痛耐受和疼痛检测阈值分别增加36.8\%和53\%;当阴道刺激导致性高潮时,疼痛耐受和疼痛检测阈值分别增高74.6\%和106.7\%。然而,当对膝部或阴道后壁提供类似的机械刺激时,疼痛阈值却没有明显改变,也没有引起骨盆肌肉的随意收缩。实验还表明阴道刺激伴或不伴高潮时都不会影响触觉阈值。以下是惠普尔等在20世纪80年代中期完成的阴道机械刺激对检测疼痛阈值和耐受疼痛阈值影响的临床试验,刺激方式分为两种:一种是能造成疼痛的手指压迫;另一种是不刺激人的触觉刺激。试验使用了一个特殊制作的振荡装置,志愿者在不同的时间里接受对阴道前壁或后壁施加不超过0.06kg/cm2 的机械刺激。此外,这组研究人员在另一项研究中还以类似的方式对生殖器区域进行刺激以求引起愉悦的快感。调查了施加于阴道前壁、后壁和阴蒂的“压力”和自我“娱乐”刺激的痛觉缺失作用,对这三个区域施加的能带来快感的刺激将提高疼痛阈值,但对触觉阈值没有影响。相反,若对这些区域给予“压力”刺激仅提高阴道前壁的疼痛阈值,而且也不改变触觉阈值。通过志愿者的描述性说明和其他研究结果,说明阴道刺激似乎并没有引起刺激步骤的分散,而是使致痛步骤的致痛性质发生了转变,是痛觉丧失而不是麻醉。“压力”刺激的结果表明阴道前壁在引起痛觉丧失作用时要比阴道后壁或阴蒂更为敏感。动物研究表明阴道刺激的痛觉丧失作用是通过α-肾上腺素能和5-羟色胺机制调节的,大概涉及脊髓以上和本体的脊髓内啡肽神经递质系统。

这种镇痛作用的生理意义何在呢?阴道压力刺激引起的痛觉丧失能减少插入时的疼痛,这对动物比对人更有意义,因为动物具有多次插入(这对激活神经内分泌活动来说是必要的)。有人认为胎儿分娩时胎头对阴道前壁的压迫作用可以激发这种内源性镇痛作用,从而减轻分娩时的疼痛。这一发现也有助于解释一个常见的体验:为什么某些慢性关节炎患者的躯体疼痛症状在性交之后可以缓解数小时之久呢。

一项研究表明,阴道刺激对性满足有重要影响:当妇女在接受阴道敏感区域刺激后体验到愉悦快感时,她们会比那些缺乏这种感觉的妇女更容易感到全身生理上的性满足;若妇女在接受阴道刺激后获得性高潮,她们就会感到完全的心理上的满足。她们比在阴道刺激后体验不到高潮的妇女更渴望频繁的性交和高潮,也会体验到更多的性高潮。在传统的男上位性交中,阴茎只是沿阴道壁的表面移动摩擦,而不能提供为最佳刺激所必需的与阴道前壁成一定角度的压力刺激,这大概也是众多妇女感受不到性高潮的重要原因吧!一些善于动脑的女性发现,当后位性交时把小腹部倚在一个硬物上或用手按压小腹部,那种感觉就会很快出现,说明她们的做法确实有助于加强对阴道前壁的有效刺激,类似的做法还包括在女下位时用一个较硬的枕头把臀部垫高等。

正确评估女性性功能障碍需要客观的测定技术,由于缺乏对女性性生理的充分理解,这种客观测量技术也是不成熟的。目前已发现的结果包括:放置于阴道壁的热敏电阻记录到性唤起时阴道温度的增加;光体积描记法记录到手淫中和手淫高潮后阴道血管充血、血流量增加;在视觉刺激或自我手法刺激时,阴道周围血流量增高,达到高潮时可能暂时下降,随之又会回升到高潮水平;对阴蒂进行振荡刺激可引起阴道光电反射的变化。但这一方法的人工痕迹较重,也不能直接测定血流量,故对它的意义尚存争议。上述生理反应中的血运改变可能是由于阴道周围血流的增加,血量脉运幅度的增加和总体血量的增加所致。

1992年,国外有人首先围绕性刺激对阴蒂动脉血管的作用展开了研究。他们用多普勒超声技术监测阴蒂血管对阴道内压力刺激所产生的反应。连续的多普勒技术测量表明血流速增加,这与血管阻力下降、血流量增加密切相关。他们对10名健康的志愿妇女进行了性生理测定,她们的年龄为20~28岁,均有积极的性生活,只有一人是未经产妇。测定时令妇女仰卧,充分放松,采用一个压力探制仪做刺激器,它是一个可膨胀的圆柱体,其内装有一个管状压力传感器。不充气时圆柱体大小为2.5cm×13cm,在放入阴道前将压力探测器充气到13.3kPa(100mmHg),这样可以在没有疼痛或不适的情况下很容易地把它插入阴道腔。提供压力刺激的方法有两种:一是使压力控测仪充气以增强压力;二是通过插入/抽出的移位活动产生刺激。使用连续的多普勒超声仪(8MHz)来记录阴道刺激导致的血液流速的改变。血液流速与多普勒频率的变化相关,可用下列公式予以表达:F/F=2V/CcosO。其中F是发射频率,V是血液流速,C是超声传播速率,O是探头与动脉形成的角度。多普勒探头放置在阴蒂动脉之上,注意在每次实验中保持探头角度的一致,多普勒信号稳定时再提供压力刺激。在压力刺激开始之前、刺激过程中、刺激结束之后均同时收集并放大压力传感器和多普勒信号(-10~+10V输出压),在50Hz的采样频率下将信号转换到数字处理系统(A/D板),这些信号将储存到计算机硬盘上供以后显示和分析。每一次反应的潜伏和持续时间,以及多普勒曲线下的整体面积均用他们自己建立的软件来判定。当刺激开始后,收缩期和舒张期速率均将增加并超越基线,当反应结束时多普勒曲线返回基线水平,这一段的曲线下面积为S2 ;而静止无刺激时的相同时间的曲线下面积为S1 。只要增加压力探测仪的压力2.67~21.3kPa(20~160mmHg),就可引出多普勒反应,刺激后的潜伏期不超过2s。反应图形呈正弦波状,先逐渐缓慢上升,达到顶峰后再缓慢下降。当降低压力时只能引起微弱的多普勒反应。当压力持续维持而不改变时,无多普勒反应可以测出。当进行插入和抽出刺激时,插入时引起的多普勒反应明显高于抽出时的反应,个别人例外。反应潜伏时间在0.1~1.6s之间,平均0.53±0.49s;持续时间在3.2~9.5s之间,平均5.8±1.2s;与反应相关的多普勒曲线下面积为12.2~53.4数字单位之间;S2 / S1 的比值在3.2~11.5之间,持续时间与S2 面积之间没有直接关系。唯一未能测出血流速率变化反应的是那位未产妇,目前尚不能解释这一结果。由于阴道外1/3压力刺激的增加,阴蒂动脉血流速度和血流量均增加,这反映在多普勒曲线幅度的增高和曲线下面积的增加。这两种刺激方法均引出多普勒反应的变化,而且S2 基本上都大于S1 。他们还发现阴道内2/3和外1/3的自然压力是不同的,它们给予阴茎龟头的相应的压力和激发的血管反应也自然有所不同,这种差别可以反映在流入阴茎海绵体动脉内血流量的增加。此外,当给阴茎龟头类似的压力刺激时,也能激发涉及阴茎变硬过程的坐骨海绵体肌反射。此外,临床试验还表明对阴蒂的振荡刺激可以引起盆底肌肉的反射性收缩,也可能增加阴道周围的血流量。这些观察均提示在性交过程中,男女都可以区分出两种生理成分———血管和肌肉反应。在抽动过程中,阴茎龟头受到的压力刺激导致进入海绵体动脉的血流增加和盆底肌肉的反射性收缩,从而导致了海绵体内压力和阴茎硬度的增加。同时,阴道刺激也引起女性血管和肌肉的相应反应,增强了阴道的弹性和张力,增加了阴道周围的血流量。以上现象证明:在性交过程中存在性协同作用,男女两性的血管和肌肉反应相互影响并得到共同的增强。

有人认为女性性反应并不是过去所想象的那样,是一种反射或一个单纯的反应,只要给予规定的刺激,就可以按照预期的频率发生。这一反应看上去很简单,但实际上涉及个体的整个运动行为。涉及女性性反应的肌群主要分为四组:①女性阴道口周围的表浅肌肉群,肌纤维也分布到阴蒂脚和干。这组肌群包括会阴横肌、坐骨海绵体肌、球海绵体肌和会阴体等,肌纤维起自坐骨支,并插入阴道周围的结缔组织。②尿生殖膈和肌肉,它们的收缩使阴道下端感到环状突起,并组成鞘状物。③耻骨尾骨肌,在阴道的两侧壁上可以分别触到束状的肌肉,然后二者在阴道后面重新聚合为一个吊带样结构。④阴道最下端的内在的肌肉,这些平滑肌的性质与骨骼肌不同。这四组肌肉在性高潮中所起的作用是不同的,妇女性能力的好坏取决于它们的作用。

性高潮的形态学表现是阴道周围括约肌和子宫肌肉的收缩,这些都可以客观测量到,但未必能感觉到。阴道的一个重要解剖特点是它的周围具有丰富的肌肉,它们由两条肛提肌组成,又称为耻骨尾骨肌(简称PC肌)。它们来自耻骨,包绕尿道和阴道外侧部,并在直肠后形成一个袢固定于尾骨。PC肌的一部分肌纤维呈放射状进入阴道壁成为耻骨阴道肌。在两条肛提肌之前是围绕阴道口,也被称为阴道括约肌的球海绵体肌,其起自会阴中腱,与坐骨海绵体肌一起呈放射状进入围绕阴蒂体的筋膜。坐骨海绵体肌较薄弱,又称阴蒂勃起肌。阴道周围肌肉的良好张力与知觉对高潮极其重要,这些肌肉由阴部神经的分支之一会阴神经支配,于是兼具运动和感觉功能。在性交抽动过程中,PC肌的微微收缩使阴茎感受到“紧握”的感觉,在性高潮时它将产生有力的节律性收缩,对男方阴茎的紧握感显著增强。

德国医生艾彻和诺克测量了130位妇女的阴道周围压力。当妇女收缩阴道时,压力超过1.3kPa(10mmHg)的有85例,不足这一压力的有45例。在第一组超过1.3kPa的85例中84\%总能或经常能达到性高潮,而第二组的45例中只有29\%的妇女能达到这种状态,两组之间存在显著的统计学差异。很显然,当妇女的性高潮能力损害或低下时常常表现为阴道周围肌肉的收缩能力降低。出现这种问题的原因包括解剖缺陷(体质的或损伤的)或抑制。另一方面,他们还发现随着年龄的增长和分娩次数的增多,高潮能力可以得到增强。也可以理解为,性经验的增加对性高潮能力有重大的改进作用。当然,在个别情况下,分娩时过度伸展所造成的阴道裂伤和松弛,可以明显减弱性高潮能力,至少会影响到阴道性交。以耻骨尾骨肌为主的阴道周围肌肉在性高潮过程中产生的不随意节律性收缩总伴有十分欣快的感觉。由于阴道入口对压力和痛刺激均高度敏感,而且其周围肌肉都可以收缩,所以它可以反映一个妇女对性的心理反应。如她对性唤起畏惧或焦虑,肌肉则收缩使阴道口变紧变小,阴茎插入会遇到困难。

用压力传感器记录从性反应逐步建立到释放的20分钟内阴道口的反应,可以记录到在开始后的3~4分钟出现一系列慢波,其中每个慢波是由几个快波迭加在一起而形成的,压力反应的基线水平则不断升高;待到刺激后7~8分钟,反应突然强烈起来,这时患者的主观感觉是阴蒂的震颤;到高潮时,肌内收缩频率可达每秒1次。若要求受试者主动收缩阴道周围肌肉3次,无论是唤起5分钟还是性高潮之后,均可记录到压力变化的3个尖锐的峰状波形。实验还通过触诊记录到围绕阴道外1/3处的尿生殖膈区域有每分钟3~4次慢波,其中迭加着每分钟15~20次的快波,代表了耻骨尾骨肌的收缩。测定阴道内2.5,5.0,7.5,10.0cm几个不同位置的压力,发现阴道内10cm处,也就是后穹隆处的反应最弱。即使受试者达到性高潮,在这一位置上也几乎记录不到下段所记录到的快速收缩。这时记录到的只是耻骨尾骨肌的收缩,而所谓尿生殖膈收缩的慢波已不清楚了,然而,在阴道5和7.5cm处的收缩仍很清楚有力。记录中压力曲线基线水平的不断上升反映了阴道下段平滑肌的收缩或张力,也反映了阴道壁的充血和前庭球的肿胀。令人感兴趣的是,表层肌肉在开始时并未参加这一系列反应,直到高潮开始之际才出现大约每秒1次的收缩,它们仅代表了表浅肌群的活动,这些收缩具有相当的强度,也反映了阴蒂的迅速升高和下降及小阴唇的抽搐。另一个引人注目的高潮特征是:持续12~15s的阴道下段的压力突然升高,开始于表浅肌群收缩之前。正如马斯特斯和约翰逊所指出的,高潮期开始的主观感觉发生在肌肉收缩之前。

总之,神经分布的不同,肌肉结构的复杂,心理和生理刺激的多变,激素平衡的波动,个体差异等都会使肌肉群收缩的次数、强度、各期持续时间等反应参数发生很大变化。

阴茎周长在阴茎夜间勃起试验中增加1.6~2.0cm,从周长的变化转换为对硬度的判断(即能否插入)可发生1/3误差。其中77\%的误差发生在判断为不能插入者,因为它们的周长仅增加1.5cm或更少,而硬度却是正常的。那么阴道插入究竟需要多大压力呢?

有人设计了一种阴道插入压力测量仪(VPP),测量了男上位、女上位和后位时的插入压力。该仪器是在一个60ml的塑料注射器针嘴上接上压力计,在针栓上接上一个模拟阴茎(大号直径3.49cm,小号直径2.86cm),大小号可以换用。使用时在模拟阴茎上套上避孕套并涂上水性润滑剂,以插入5cm处为记录压力大小。把受试者分为三组,测量时体位顺序不同,每种体位都用大小两号分别测定,单数者先用大号,双数者先用小号,每个受试者的6次测定要由同一医生和同一护士进行。第一组:男上、后入、女上;第二组:后入、女上、男上;第三组:女上、男上、后入。结果发现:男上式:小号6.27kPa(47mmHg),大号11.86kPa(89mmHg);后进式:小号9.86kPa(74mmHg),大号20.13kPa(151mmHg);女上式:小号13.20kPa(99mmHg),大号22.53kPa(169mmHg)。所以结论是:插入压力与阴茎探头的粗细明显相关;经产妇平均插入压力为91±60mmHg(n=11),未产妇平均插入压力为124±81mmHg(n=9),两组有显著差异(P=0.01);经产妇中,VPP与产次无关;VPP与年龄、会阴切开术、性交频率、性交要求、性乐趣、性交疼痛、性交体验历史等均无统计学相关性(注:1mmHg=0.133kPa)。这一调查初步了解到性交活跃妇女VPP的正常值,证实了该仪器在测量VPP中的用途和有效性。如果测量到的VPP和阴茎硬度测量相关,就可以更准确地制造固体的或可弯曲的阴茎假体,这样能从功能和美学两方面提高ED患者对植入假体的接受性。如果检查时发现VPP异常高,就应安排相应的治疗措施,如有意的耻骨尾骨肌训练,以便掌握控制该肌群活动的能力,阴道扩张练习,必要时采用激素替代疗法,短期的行为治疗或心理咨询。如果VPP异常低,就应安排耻骨尾骨肌锻炼或外科整形手术。

女上位时测得的插入压力在这三种体位中最高(无论大号、小号均如此),但女上位恰恰是性治疗学家推荐的治疗心理性勃起障碍的体位,这又如何解释呢?可能是由于女上位时女性可以用手接触男性生殖器,同时把阴茎塞入阴道并超过其所需要的VPP压力。如果因个体原因造成女上位插入困难时,就需要更换侧位等其他体位。

第一位描述G点的人实际上并不是格拉夫伯格,而是一位17世纪荷兰籍解剖学家雷尼尔·德·格拉夫,他是对男女生殖器作出现代解释的第一人。雷尼尔·德·格拉夫不仅着眼于女性性器官的构造,而且重视其对性爱的重要性。他贴切地把包绕尿道的那层薄薄的组织称为女性前列腺或旁腺体,刺激它会使女人拥有更加强烈的性欲并可带来强烈快感。亚历山大·斯基恩医学博士(1880)曾谈到女性阴道周围各种腺体管道染上淋病双球菌后的引流问题,并用图示说明感染病状,使人们一目了然。时至今日,人们仍然把女性尿道腺体称为斯基恩腺体。后来的研究者认为胚胎期的女性尿道腺体可能和男性前列腺体源于相同的胚胎组织,女性尿道腺体具有与5~6个月男性胎儿前列腺体相似的结构。医学博士乔治·考德威尔(1941)报告说:“女性尿道腺体与男性前列腺体的结构和特有分泌物非常相似……”对女性来说,恰当的性刺激能够引起明显的性反应,女性尿道腺体内不断出现的分泌物就证明了这一点。几乎半个世纪,没有人对斯基恩的研究成果进行进一步探索,直到1943年,医学博士妇产科专家约翰·W·赫夫曼才开始研究斯基恩腺体,并且下结论说:就连斯基恩本人也在很大程度上低估了女性尿道周围腺体管道的范围。并于同年在《美国医学协会杂志》刊登的一篇文章中报道,女性“前列腺”肥大症必须通过外科手术治疗。

早在1944年,德国妇产科专家厄恩斯特·格拉夫伯格与美国著名解剖学和妇产科专家罗伯特·L·迪金森(许多美国人把他视为第一位美国性专家)就曾合作描写过位于阴道前壁尿道表层下方的“动情区”。格拉夫伯格于1950年撰写的文章首次详细描述了阴道前壁内动情区的存在:“……沿尿道阴道前壁有一个动情区,……似乎在它周围有一层类似阴茎海绵体状的能够勃起的组织……在性刺激过程中,女性尿道开始膨胀,很容易被摸到。在性兴奋到达高潮期末,这个动情区膨胀到最大限度。最敏感的刺激部位位于尿道后部,与膀胱颈相毗连的地方”。格拉夫伯格认为这个动情区非常重要,“因为女性自己总能意识到对方的手指或阴茎已经离开这个紧靠尿道的阴道部位,于是她会通过调整改变自己的体位以达到重新接触的目的”。当了解到处于屈从地位的妇女的性问题并未能引起由男性占主导地位的医疗部门的重视,格拉夫伯格于1953年指出,“对于女性性问题的低估简直到了令人吃惊的程度,以至于连她们连性高潮的含义和性敏感区的位置都不知道”。他进一步强调指出,“阴道前壁性敏感区的位置证明人跟其他行走动物是一样的。西方最普遍采用的性交体位,是女性仰卧,除非勃起的阴茎角度陡峭或者将女性的两条腿搭在男人肩头,否则,阴茎的抽动触及不到阴道的尿道旁部位。我赞成莱蒙·克拉克的意见,人类是设计为四足动物的,因而正常的性交体位应该为背位(男人从背后插入阴道)”。鉴于格拉夫伯格对阴道敏感区与女性性快感之间密切关系的深刻认识,佩里和惠普尔在20世纪80年代初把该敏感区命名为“G点”,以肯定和纪念格拉夫伯格早年的工作。

从医学角度分析,G点不如阴道弹性肌肉组织那么重要,可是G点的发现却让我们彻底摆脱了过去几十年的传统观念。因为它证明了女性生殖器部位的性唤起敏感点不止一个,而在马斯特斯和约翰逊时期性学家们一直没法引导人们相信女性只有一个敏感点。其实女性至少有两个性敏感点,就像男性的阴茎和前列腺一样。女子不难发现位于体外阴道入口处的阴蒂,并经常地抚弄它。而G点,则位于阴道前壁内侧,该区位于11点和13点之间,其大小范围2~4cm,硬度描述为肿胀。虽然伴侣的合作并非必不可少,但如果没有伴侣的合作,妇女自己难以发现。对于男性来说,也有同样的情况。阴茎易被男人自己发现,抚弄,然而,如果没有性伴侣的帮助,想通过直肠前壁触摸到前列腺的位置,也是很困难的。

例如一位丈夫这样描述他寻找并刺激妻子G点的过程:“我调整了她的身体体位,腹部朝上,两条腿支在床边。我开始轻轻地将手指探入阴道……我很快就确定了医生曾经给我们描述的那个卵形点的位置,并开始使劲地用手指刺激它。妻子不停地呻吟着,小声催促我‘别停下来。’不一会儿,她开始舒适地喘粗气,她的阴道吸着我的手指,好像要把手指从阴道里挤出去似的,这种感觉特别令人激动。”他的妻子则叙述了她刺激丈夫前列腺的经过:“我下意识地轻轻把手指插入他的肛门,使用自己的体液作为润滑剂。他发出舒适的呻吟声,暗示我可以继续进行这种新探索。我一边吻他,一边用手指按抚他的前列腺,我本想在他高潮来临之前,改变体位,坐在他身上,可是,时间来不及了,他已经达到高潮。”就这样,两个人都尝试了通过刺激新的性感点产生性快感的试验。

G点位于阴道前壁耻骨后面,其大小与准确的位置因人而异。一般说来,其位置大约在耻骨后面与子宫颈前部间的半途中,顺着尿道,靠近膀胱颈,与尿道相毗连。假定阴道似一个小闹钟,以12点指向前壁中央,6点指向后壁和肛门,大多数妇女都能在11点与1点之间找到G点的位置。它不像阴蒂,从周围的组织凸起,而是深深地藏在阴道壁内,要想找到处于非刺激状的G点,往往需要用力按压。G点大体上是由血管和血管丛、尿道旁腺腺体及腺管、神经末梢和膀胱颈四周其他组织组成的复杂的网状结构。在对女性进行性学检查时可以发现,她们的性敏感区因刺激而膨胀,本来是柔软的组织,但刺激后就会变得很硬,还带有一定的轮廓。如果处于适合性唤起的条件它会迅速膨胀。G点在未受刺激的状态下体积相当小,而且很难确定其位置,特别是不能被肉眼所见。正如在常规体检时一般不会见到阴茎勃起一样,G点也不会肿胀。

在美国妇女女权主义者联合会妇女保健中心编辑的《女性躯体新观察》杂志中,G点被称为“尿道海绵体”。作者自己给它命名是因为她们无法在医学书籍中查阅这个结构的名称。她们解释说,它在性兴奋和性交过程中通过血液充盈来包绕和保护尿道,并起到阴茎和尿道之间缓冲器的作用。另据报道,有些女权主义者把阴道周围血管丛、前庭球、G点等统称为女性的尿道海绵体,还把阴蒂的头、体、脚(相当于男性的阴茎海绵体)和她们所称的女性尿道海绵体合在一起统称之为阴蒂,也就是说她们给阴蒂赋予了一个与过去解剖学截然不同的更为广义的新概念。

女性如何发现自己的G点呢?仰卧体位是难以发现的,因为重力的作用会把内部器官向下拉,使其远离阴道口,因此需要手指长或者阴道短才能奏效。发现G点的最佳体位是下蹲式,由于G点受到刺激后,女人产生的第一个感觉通常是想小便,发现G点的方法之一就是蹲厕所。由于女性下面的举动会给充盈的膀胱发出信号,引起尿意,所以在想确定G点的位置之前需先排尿。可以用手指使劲地向上压阴道前壁的方法探索G点(有些女性发现用另一只手同时在腹部耻骨上方向下按压很有助于找到G点)。G点一旦受到刺激,就开始肿胀,在内外两只手之间,往往能够摸到一个小小的硬块。当女性把按压的手指松开,就会感受到手指从阴道内向外推压的感觉。如果继续用力触动G点区域,女性会感到轻微的快感,可能还感觉到子宫有阵痛或收缩的现象。当女性学会了手淫,便可以用G点做试验。如果更加用力地压迫G点,可能获得比手淫刺激阴蒂更强烈的内在感觉。当进展到超越膀胱的充盈感时,女性可能想转移到床上或者另一个更舒适的场所。如果仍有排尿感,可随手准备一块毛巾,以两腿跪地,或两腿盘曲的姿势继续刺激G点。如果出现了性高潮,请注意体会这与刺激阴蒂引起的性高潮是否有区别,区别在哪里。有些女性出现这种性高潮时可能会射液,此液比尿的色泽更清澈、更白、无任何异味。

如果女性有性伴侣,和他在一起时会感到舒适,女性可能向往与他共享发现G点的乐趣。采用俯卧位,两腿叉开,臀部轻微向上倾斜,让你的性伴侣把两个手指(手掌朝下)插入阴道,用力按压阴道前壁(最靠近床的部位),移动骨盆,会更容易发现G点,在这种体位下用阴茎刺激同样能产生最佳效果。如果采取仰卧位,性伴侣将一个或两个手指(手掌心朝上)插入阴道,压迫阴道上壁通常可以摸到G点。G点位于耻骨后侧与子宫颈所在的阴道壁末端连线之间的中点处。然后将另一只手置于腹部的耻骨缘上方,径直往下按压,有时更有助于刺激G点。对许多夫妻来说,还有一种可取的体位是男人仰卧,女人坐在男方的身上,将坚挺的阴茎插入阴道。这种体位要求女方上下抽动,使阴茎接触其G点,这样便会导致频繁的性高潮。

众所周知,几乎所有哺乳动物的性交体位都是从躯体背后插入(背位性交)。伊莱恩·摩根曾于1972年撰文描写过G点性敏感区,尽管她没有称之为G点。她在一篇题为《女性的祖先》的文章中这样写道:“性高潮的到来简单地说是由短促而又剧烈的节律性性器官摩擦而激发的。理想的摩擦通常是从背后插入阴道刺激该区的阴道前壁……在这里,我们需要注意的另外一个问题是,像人、猿、猴这样的最高级哺乳动物和其他四足动物,对于阴道的压力不仅来自身后,而且压力自上而下,因此压力施加于阴道腹侧即前壁。”她说:“女性结构机能在某种程度上出现萎缩是由于其进化和文化历史发展的必然结果。但是,可以说明这样一个事实,有些女性难以触及G点,而另外一些女性却很关心自己在性交时的正确体位,她们完全依赖于性伴侣的特定的身体作用。”虽然摩根通过人类行为学和进化论从完全不同的角度研究了这个课题,但其得出的结论却与有实践经验的医学专家格拉夫柏格的结论颇为相似。她写道,从背后插入是使包括女人在内的一切雌性哺乳动物达到性满足的最佳体位。事实上许多男女的性生活实践都能证实格拉夫伯格和摩根的结论。一名女权运动者听到G点和背后插入这个说法时非常恼火,她认为背后插入会使女性处于劣势的位置。不管人们把此性交体位视为“低下”也好,还是“享乐”也好,都是极为主观的看法。性娱乐活动从未有优劣之别,重要的是,我们应认识到不同的妇女有着不同的需要。女性应具有各种选择的权力,积极地参与到性交活动中并自由选择性交体位。

格拉夫伯格认为不仅性交的体位对刺激G点具有重要意义,性伴侣的身体结构和相互间的配合也至关重要:“阴茎与身体构成的角度具有重要性,对此应加以考虑。完美的情人形象应以此生理特征为基础。”另外有些夫妻反映女上位也是刺激G点的最佳性交体位,以这种体位性交,有时较小的阴茎往往比较大的阴茎更奏效。正如一位30岁的妻子写道:“我总能获得性高潮,但是当阴茎完全插入阴道内,我却感觉不到任何刺激了,我的性兴奋和性唤起因阴茎的完全插入而突然消逝。一般情况下,当阴茎插入阴道一半或1/3的深度,总能给我带来性兴奋。现在我才明白,这是阴茎触到了我那‘魔点’的缘故。”

对试图确定G点位置的医生来说,让患者仰卧体位,双手检查也是个有效的方法。刚开始用手指触摸G点时,它宛如一粒小蚕豆。但一旦受到刺激,体积就会膨胀得像2分硬币,甚至像5分硬币那么大。一些妇女的G点会更大些,正如有些妇女的乳房会比别人更大,有些男人阴茎比别人更长一样,但身体这些部位的大小决不会影响刺激带来的反应。刺激G点使妇女产生的兴奋程度不同,同样,有些妇女对刺激乳头所产生的反应也不尽相同。通过体检表明,绝经期后妇女的G点体积会变小,可是她们对刺激G点的敏感程度与育龄期的妇女相比较时,并没有什么区别。

人们很早就注意到,刺激G点虽然没有导致女性射液,可往往会使女性感觉有尿意,这可能引起严重的后果。因为她们会感到害羞和不安,害怕真的要排尿,从而使她们的感情受到压抑,甚至对性行为采取克制态度,最终必然阻碍性高潮的产生。

有些妇女在分娩过程中,会产生性高潮的感觉。这是由于在分娩时胎儿沿产道出生,向上的压力可能刺激到G点的缘故。但拥有强烈反应能力的G点也可能出现一些问题,如妇女在接受妇科检查时,因为阴道窥镜压在G点上,导致其性兴奋突然从平静状态进展到“爆发”状态,达到不可控制的程度,所以必须集中精力避免它的发生。

在手术时,G点的存在位置是特别值得外科医生注意的,因为错误地切除就会夺去女性一生的性快感。根据访谈和来往的信件分析,外科手术对性功能产生的积极或消极的影响取决于外科手术的类型和对神经组织的干扰程度。例如,有些妇女在子宫切除以后表现为性快感有增无减。其中一位女性谈道:“由于对那个点的刺激,我一直有性高潮。我的G点摸起来体积很大,我和丈夫都喜爱它。9年前,在我30岁时,被迫做了子宫切除术,但保留了卵巢,可是我的性快感却增强了。”另一方面,有些妇女反映了与之相反的结果。我们不再认为阴蒂是所有妇女性快感的主要源泉,而认为影响阴道周围的组织会撕裂或损伤G点,因此会削弱或者破坏女性的性快感。一位妇女曾经提出这样一个问题:“我现年42岁,正面临子宫切除的问题。在22年的婚姻生活中,我的性生活是美满愉快的。我有过人们描述的那种子宫向下推出引起深在性高潮的经历。我一直想不通的问题是如果没有了子宫还会跟有子宫时的感觉一样吗?”格拉夫伯格的确注意到如果妇女在子宫切除手术时损伤了阴道前壁的性敏感区,可能会影响以后性高潮的出现。可是,不少切除了子宫的妇女反映,尽管她们不再有子宫了,可是她们确实仍伴有向下推压的感觉和“强烈的性高潮”。人们把产生这个问题的原因解释为子宫和G点的神经支配仍然完整无损,因而,阴道上部的肌肉功能未受到任何影响。

马来西亚怡保的性学专家蔡志安医生,先后花了4年时间在271名妇女身上,研究出女性的A点。据他介绍,只要找到女性阴道内的A点,她们的性欲就会在短短5至10秒钟内被挑起,阴道立马湿润起来。而这些女性当中,有1/3因为老公后来找到了A点,而享受到多次性高潮。他在第7届亚洲性学大会(莱佛士城会议中心)上的报告,令人耳目一新,反应空前强烈。蔡医生说很多妇女是因为阴道干燥导致性交疼痛,无法好好享受性爱而上门求诊的,他决定对此进行研究,解决女性的困扰。所谓A点,其实就是指越过G点更向里,接近阴道前穹隆的敏感区,“Anterior Fornix Erogenous Zone”的简写,指的就是与子宫颈稍有点距离,在阴道壁上的敏感地带,也就是阴道内促使性欲加强的前穹隆。G点的兴奋程度绝对赶不上A点,过去A点之所以被忽视,是因为它比G点更隐秘。在阴道未完全兴奋的时候,A点是很不明显的,很难用手摸到。只有先刺激了G点一段时间后,更深处的A点才会翘出脑袋。A点的发现,让我们的性上升到一个崭新的台阶。

U点是尿道口上方及两侧的很小的区域,位于阴道入口处2.5cm左右。刺激它时会产生排尿的欲望,其实也不是真的有尿要排,因为停止刺激时,排尿感也随之消失。如果忍住这样的尿意继续享受,有时会出现通常所说的女性射精。即使伴有尿意,排出的液体也并不是尿液,而是一种清白色液体。在刺激G点时可能会同时刺激到U点,但也不是每次都能刺激到,有时顺势刺激到U点则产生尿意,有时却没有。

由于阴道内部十分脆弱,指交时一定要修剪指甲,避免划伤阴道内壁。进入前,一定要做好清洗消毒的工作,预防感染。男士务必保护好自己的手指,过分粗糙、干硬的手指必将给女方带来不适而并非愉悦的感觉。

佩里和惠普尔(1982)指出,一些妇女因为在每次愉悦的性经历中“排尿”而感到苦恼,甚至因此而引起对方的严重不满。但她们却不知道这究竟是怎么回事,总认为是自己的身体出了毛病,所以在性高潮到来时膀胱失控。因此,她们尽可能地回避性生活,或者花费大量时间和费用去看精神科医生。当然也有些女性对此持不同态度,如一个妇女写道:“我与爱人的性生活很和谐美满,尤其是当爱人以口刺激我的阴蒂并且以一或两个手指伸入我的阴道时,性高潮很容易来到,我不知道这是怎么发生的,全身会突然出现痉挛性收缩,感到特别痛快,并突然射出一些液体。开始时我以为是因刺激而排出的尿夜,但在我们12年的性生活中只要这样刺激,射液就会出现。我肯定这是一种新的发现,做爱如此有趣,虽然有些污秽,但我们对此并不在意。”由于这类案例层出不穷,促使学者们不得不就此展开深入的研究。

西奥多·德维尔德(1926)在《理想的婚姻》一书中就提到一些妇女在性高潮期间会排出一些液体。1950年,格拉夫伯格详细描述了女性射液与愉悦的关系,解释了这种液体的排出是在性高潮的时候。他写道:“这种痉挛后液体的排出总是伴随着高潮的极点同时发生的。如果你有机会观察一些妇女的性高潮,你可以看到大量的清亮、透明的液体被排出,不是从外生殖器,而是从尿道口喷射而出……这种大量的分泌是性高潮到来的结果,而没有润滑的意义,否则应该产生于性交的开始而不是在高潮的高峰期。”虽然格拉夫伯格表示已经检查了这些液体,但是他没有阐明所采用的步骤。金西(1953)对这个问题的关注要稍稍多一些。他认为由于前列腺和精囊腺在女性仅是一种痕迹结构,它实际上并没有射液。随着性高潮的到来,强烈的肌肉收缩可以排出少许生殖器分泌物,在某些情况下,某些力量可使之射出来。这种现象常常作为一种女性射液现象,特别是在某些色情文学中更常见到,但严格地说这个字眼不能用在性交关系中。1966年马斯特斯和约翰逊写道,女性射液是一个“错误的和广泛流传的概念”。自1958年以来,一些具有这方面知识的医生对女性射液现象采取理解、支持和帮助的态度,如泌尿科医学博士伯纳德·海曼尔就一直拒绝给这些“尿失禁”的妇女手术,而且以他的经验判断这是性高潮的液体喷射。他读过格拉夫伯格的原著,而且了解G点及女性射液,他向他的同事再三提出自己的观点,多数人都以为他发疯了,他觉得自己是正确的,但很孤立,直到遇到惠普尔,他勇敢的态度才得以证实。从有关女性射液的历史文献中可以看出,人们其实早已认识它,但为什么所有人都将它忽视了呢?曾描述G点的某些历史性人物,同样也讨论女性射液,但他们对这两种现象几乎完全不知。如杰曼·格里尔(1970)发表的《女太监》中所叙述的关于妇女的各种虚假观念,“尽管多少年前就证明是假的,但许多人都拒绝放弃女性射液的观念,尽管它有长期的历史影响但最终纯属幻想”。人们拒绝接受女性射液的一个原因可能属语言问题。在古代,术语“精液”习惯上用于描写两性的“种子”或“射精”,而女性射液中显然没有精子,所以历史上并没有哪个词用来描述这些液体。目前英文中射液与射精仍完全是同一个词,不少人把它译为中文时竟推理般地翻译为“女性射精”,这显然不妥。

生理学家认为男子高潮运动中的排尿几乎是不可能的,即使肌肉薄弱也不应在高潮时尿失禁,除非这个男子的膀胱有严重的器质性问题。由于女性射液仅仅为其带来性愉悦,对生育毫无意义,所以很多人并不推崇这样一种没有生殖目的的液体。女性射液并非膀胱括约肌在性高潮时松弛而发生的,所谓尿失禁实际上是女性高潮时排出的,仅仅是阴道前壁与尿道周围有关的动情区尿道内腺体的分泌物,而并不是真正的尿液。尽管人们发现了女性在高潮期间自尿道射液的现象,但毕竟只出现在少数妇女中,如今承认有过射液经历的女性仅40\%,少数妇女讲述她们经过阴蒂刺激便有射液。

女性射液就是在性活动过程中体内的液体通过尿道口排出体外的过程。女性性高潮时所射液体多是无色无味的,从床单等染色看也不像尿,所以对它的描述各不相同。当排出量较少时,它可能是一种无色、灰白色或泛浅黄色的,清澈的黏液样液体;当排出量增加时,它可能变成一种清亮的水性液体;当排出量很少时,它可能具有一种麝香样或刺鼻性的气味;但当排出量显著增加时,它就不再具有任何气味了。一位30多岁的妇女不相信医生给她下的结论———“尿失禁”,而且需要手术治疗,她设计了一个巧妙的试验来检验医生说的是否正确。她服用一种可以使尿液呈蓝色的药片,再检验多次高潮射液以后床单上的“湿点”是否会出现相应的蓝颜色。有的时候液体没有任何颜色,而在其他检查时稍有一点浅淡的蓝色,因此她有意地在床单上撒了一些尿,这次染色呈明显的深蓝色。这位职业妇女因此得出结论,她的高潮射液不是来自膀胱。

女性所报告的高潮射液的颜色、气味、稠度等也存在时间和个体差异(一个人品尝自己所射液体和已经具有体液交换的伴侣品尝所射液体的味道是安全的,而对于那些需要采取安全措施的伴侣则不建议这样做)。有人尝过它的味道,它可能是苦的、咸的、酸的、辛的甚至略带甜味,这可能是随女性饮食和代谢变动而发生的变化。有人发现她们的月经周期会影响所射液体的性质,饮食和液体的入量也可能对此产生影响。如有人可以明显感觉到没喝水、少喝水与大量喝水时射液量的差别。可以有把握地说,大多数女性在不同时间和一次性活动的各次射液之间都会存在变化。有些妇女说有时所射液体无色无味,有时却稠厚而刺鼻。也有些人说,所射液体外观和气味都像尿,这也可能是对的,但往往仅见于女性达到一系列高潮之后的特别时刻,无法自己控制收缩过程了,所以用力屏气射液之时,结果尿液也连带着一下子喷了出来,颜色也自然而然地由先前的无色透明变成清澈浅黄的射液与尿液的混合液。也就是说,如果之前的性刺激不太强烈,还在她们的控制范围之内时,即使自我用力都不会真的将尿液挤出来。这应该是一种很细微的体会吧,特别值得一提的是,当女性意识到就要射液的时刻,无论是手指还是器具,都必须撤出阴道。尤其是有些外国影片里所表现的女人的射液量非常大,其实那肯定就是排尿而非所谓的射液了,这与她们的肌肉强度和肌肉控制能力关系不大。

不仅每个妇女射液量的多少会因人而异、因时而异,它的出现频率也会不断变化变化。有些妇女说她们每次做爱都有射液,有些则表示偶尔发生,她们认为这可能与月经周期存在着某种关联。报道中女性高潮射液量的差异之大确实令人吃惊:不足1毫升~444毫升。真的会有400毫升之多?要知道女性前列腺的大小平均只有3.3cm×1.9cm×1cm,这样大小的椭圆形装置大概只可以容纳5ml液体,那么如此多的液体又从何而来呢?一项研究发现女性每秒钟可以产生1ml液体,那么其在性兴奋时前列腺会明显肿胀,从而容纳更多的液体。如果真能射出那么多液体,前列腺至少要增加多少倍呢?而我们发现,女性只要处在性唤起状态,就会不停地产生液体,在连续多次射液的情况下,她们可以射出多达60毫升的液体,这就需要女性前列腺以非常快的速率不断充盈和排空液体。也就是说,女性高潮的持续时间越长,她们所能射出的液体就越多。如果确实如此,那么女性就可以在不依靠膀胱“补给”液体的情况下,射出相当多的液体。要想证实这一结论,显然需要进行科学的实验室研究,也许可通过经阴道B超来观察女性前列腺在性兴奋和性高潮时的变化。

可以做这样的假设:女人的射液和她的性兴奋期也有关系,射液量和分泌物的量都应该与兴奋程度成正比。在她们最兴奋的那段时间,射液量也会有所增加,越兴奋,越容易分泌,因此射液量和分泌物的量也成正比。射液与否和体能也有一定关系,身体状态不好的时候根本无法射液。

总之,那个特殊部位的肌肉力量和掌控能力是很重要的,也许和长久的夹腿习惯或游泳所锻炼出的强健腹肌与PC肌有关。不过,一切都有待进一步的分析调查才能得到正确的结论。

既然女性射液的体积和外观不尽相同,那么它们的组成成分也会有所不同。射液和排尿时所排出的液体当然是不相同的,不过它们可能具有某些相同的物质。射液时的液体一般含有女性前列腺分泌的前列腺特异抗原(PSA)、前列腺酸性磷酸酶或前列腺特异酸性磷酸酶(PAP or PSAP),但也发现一些尿液里存在的成分,如尿素和肉碱,不过二者比常规尿液中的浓度要低得多。当女性尿液里发现前列腺液特有成分时,其浓度也比射液里低得多。爱德温·贝尔兹、惠普尔和佩里(1981)分析了志愿者的尿液和射液样本,他们要求受试者在收集尿液和射液样本之前至少禁欲48小时,且不得与男性的精液相接触。液体由志愿者在家中私下收集,然后立即冷冻送检。以下则是两种成分的对比结果:除葡萄糖外,其他化学成分均存在显著差异。高潮射液中的前列腺酸性磷酸酶和葡萄糖分别比尿液高166和2倍;尿液中来自蛋白代谢的终端产物尿素和肌酸酐则分别比高潮射液的高3和6倍;尿素和肌酸酐在射液中的浓度显著低于尿液;射液中果糖水平显著高于尿液,但与男性精液相比则低10~15倍。以上结果表明射液并非尿液,但尿液也是射液的规律性成分之一;女性G点可分泌前列腺酸性磷酸酶,人们过去从未在其体内检验出这一男性所特有的酶;射出的液体中从未见前列腺素类物质。

除了由尿道射出液体之外,女性射液高潮与马斯特斯和约翰逊夫妇所描述的性高潮仍有其他方面的区别。马斯特斯夫妇描述子宫在兴奋晚期和平台早期具有一种帐篷作用,子宫抬高升入盆腔,阴道上部则膨胀如球状。然而,在射液高潮中子宫降至阴道开口处,阴道上部则显著紧缩成惠普尔和佩里所说的“A框架作用”,在这类高潮中妇女的阴道前壁好像在进行一种特别的前凸运动(向外突出或推出)。他们推测这两种不同类型的高潮是受两类不同的神经所支配的:即阴蒂高潮是由阴部神经所激发的,而以射液高潮为特点的子宫或深层性高潮是由盆神经,特别是腹下神经丛所支配的。

尸解发现G点区域存在着许多开口于尿道的腺体组织,但由于尸解数目太少,结论尚存在争议。组织学研究发现G点区域中存在富含前列腺酸性磷酸酶的腺体组织,此外还包括有复杂的血管网络、尿道及腺体、神经末梢及围绕膀胱颈的组织。电镜研究结果证明男女两性的前列腺酸性磷酸酶存在一定的差异。这一发现很可能对法医学诊断产生重大影响,因为人们以往总是根据是否存在前列腺酸性磷酸酶来判断是否发生过性袭击或性伤害,如今这种检查显然已失去了判断价值。

有时,妇女是射液还是患压力性尿失禁是难以区分的。后者可发生于打喷嚏、咳嗽、大笑、跳跃或性高潮期间,它们可以同时发生。尿失禁更多发生于盆底肌肉薄弱的妇女,而射液则多发生在盆底肌肉强有力的妇女身上。即使是诊断为压力性尿失禁,在手术前对盆底肌肉进行全面评价也是相当重要的,因为压力性尿失禁的纠正仅仅通过肌肉训练就能实现。有些妇女在达到高潮时不是从尿道排出液体,而是将液体排入膀胱,这种现象称为逆向射液。这是由于一些妇女因为阴道刺激而导致高潮后需要立即排尿,然而这时仅能排出一些清亮的、稍带白色的液体而绝不是尿,当检查这些妇女时发现,她们的尿道口往往紧邻阴道口,这就可以说明为什么这些妇女认为液体是从阴道排出的。一些妇女反映如同男人一样,有“遗精”现象,睡醒后发现床单或内裤被浸湿一片,液体没有味道,其污染的颜色也不像尿液。睡间少数人有过动情的梦幻或感觉,但另一些则回忆不起有什么明显的梦。

女性射液并不只发生在异性恋的妇女中,许多双性恋或同性恋妇女同样能经历高潮时的射液现象,甚至比异性恋妇女的射液频率还要高。调查结果确是如此,至于是什么原因尚待研究。或许是因为手指比阴茎更容易接触和刺激到敏感区域G点,又或许是女性比男性更容易接受这种液体的排出。

附录:(女网友雨薇的经验之谈)阴蒂高潮像是春风吹起湖面一阵阵的波动,而G点高潮的感受则像银龙在湖底引发翻江倒海般的巨澜,人也不由自主地像出了水的大鱼拼命摆动着身躯。终于可以自己刺激G点射液了,像没有关紧的水龙头在不停地冒水,真奇妙!喜欢爬到巅峰的感觉,感到自己是真正的女人,做女人感觉真好!

与生殖器相联系的神经有若干对,每对神经负责联系生殖器的不同部位,刺激不同组合的神经将产生不同的感觉。所以,每次的高潮感受取决于对生殖器不同部位、不同强度、不同方式的刺激。

感觉区:阴蒂,会阴皮肤,阴道,子宫颈,子宫。

神经:阴部神经,盆神经,腹下神经,迷走神经。

女性生殖系统的神经联系支配图:有4对神经(左右各一)负责女性生殖器区域与大脑之间的信息联系,生殖器的不同部位分布着不同的神经,生殖器的有些区域与大脑之间的联系涉及一对以上的神经。前三对神经通向脊髓,各种感觉经此传到大脑,而迷走神经绕过脊髓,直通大脑。因此,子宫颈的感觉通过盆神经、腹下神经和迷走神经这三条通路来传递,而盆神经则传递会阴皮肤(包括生殖器皮肤)、阴道和子宫颈的感觉。

对女性而言,高潮的质量取决于刺激的部位:阴蒂、阴道、子宫颈。阴蒂主要联系着阴部神经,阴道主要是盆神经,子宫颈则联系着盆神经、腹下神经和迷走神经。虽然单独刺激生殖器某个部位时引发的高潮是其独自产生的,但若是同时刺激一个以上的部位,高潮的产生就不是一己之力了,这种额外的作用就产生了一种集合的高潮,即所谓的“混合高潮”。

在女性的性感觉系统中具有一些独特的性质,包括在躯体感觉中具有典型特性的感觉辨别作用;生殖器感觉中所具有的情感特性和动机特性(如同样的刺激会因当时的情感与动机差异产生完全相反的效果);此外还有为生殖所必需的各种行为和内分泌控制机能。生殖器感觉功能的独特性与两性生殖器官的差别极其相关,所以两性的大脑也存在明显差异,这与两性生殖器和体表差异明显不同是一致的。

与前所述,女性性器官外部结构的感觉神经分布很密,包括阴蒂、阴唇和阴道前庭。除宫颈有较多的传入神经外,阴道的神经支配则较少,阴道传入神经对阴道的扩张或宫颈的压力显示出紧张性反应。女性的外生殖器和阴蒂的传入神经与男性一样,起自成对的阴蒂背神经和邻近组织的辅支,通过阴部神经,然后大多汇入骶2神经背根;有髓鞘或无髓鞘的大多数阴道传入神经通过腹下神经上行,在胸12~腰2处进入脊髓;少数阴道传入神经并入盆神经,进入骶2和邻近的背根;也有一些阴道传入神经经腹根进入脊髓。

女性会阴肌肉系统和男性一样也接受来自有髓鞘传入纤维的广泛神经支配,这些纤维通过阴部神经等进入骶丛。

能感受女性性器官表皮感觉的脊髓背角神经元主要位于骶1~2节段。这些神经元表现出快适应的机械感受性质,其感受野有的较狭窄,仅局限于同侧生殖器结构;有的较广泛,包括邻近的非生殖器皮肤区。对阴道或宫颈刺激敏感的脊髓神经元一般比皮肤感受器的细胞位于背角的更深处,这些神经元呈长时程的反应形式,并兼具皮肤和内脏的低和高阈值机械感受特性。

生殖器在皮质上的代表区主要在第一体感投射区(灵长类为背中线的中央后回),也可在第二体感皮质区。丘脑和皮质体感结构中生殖器感觉神经元的皮肤感受野常包括腰皮肤区,但也有小感受野局限于外生殖器的神经元。脊髓丘脑系统的皮质下的代表区位于延脑薄束核和腹后侧丘脑核(VPL)的最外侧。刺激阴道激括VPL骶骨投射区内的神经元,其中某些可包括外生殖器的触觉感受野。然而宫颈代表区是在VPL的躯干投射区、第一体感皮质区以及眶皮质。

除了躯体局部的脊髓内侧丘系统外,还有一个对生殖器刺激有反应的外丘系神经元系统。由于阴道刺激诱发区反应的神经元广泛分布于脑干网状结构、某些脑神经运动核、中央灰质和顶盖,这些脑干生殖器感觉神经元的突出反应特征包括:明显而快速的反应,常随刺激的重复而增强;子宫颈刺激比阴道刺激引起的反应更强;刺激后反应时程长;形式多样化;这些反应受内脏、生殖器以及皮肤区的伤害或非伤害刺激的影响。对阴道刺激反应的神经元也见于正中和外侧下丘脑、丘脑底中、正中核、板状内核和后丘脑核,还有边缘系统(中膈核、杏仁核和扣带回)。边缘系统神经元的反应比较简单,且潜伏期较长。

对女性进行的心理生理学研究表明,刺激阴道可减弱或阻滞机体对痛觉的反应,而雌激素可以在性感觉系统的不同水平上易化对皮肤触觉刺激的神经反应。

性刺激是指作用于人体性感受器的任何刺激,包括心理刺激、感官刺激和生殖器刺激等。性刺激可以分为内部刺激和外部刺激:①内部刺激:指自己的内心活动和机体内部状态的种种变化,如性幻想时出现的心理刺激和附性腺胀满引起的局部性刺激;②外部刺激:无论是作用于人体视觉、嗅觉、听觉、触觉、味觉等感官系统的间接刺激还是直接针对性器官的刺激,都能激发机体对这些刺激作出一定的反应,包括消极的和积极的反应,但也有不能引起机体相应反应的情况。这不仅与刺激的性质、强度、持续时间等有关,也与机体的心理状态、感受器本身的状况、躯体的健康状况等有关。性刺激还可以分为适宜刺激和不适宜刺激,比如柔和的灯光、温馨的气氛、优雅的音乐就是适宜的性刺激;相反,若是明亮的灯光、粗暴的举动、夫妻吵闹不宁就是不适宜的性刺激。要想有和谐的性生活和性关系,就必须营造出适宜的性刺激。

在解剖上,子宫受部分盆腔交感神经系统的支配,并与来自骶2~4的神经一起形成子宫阴道神经丛。该神经丛位于子宫颈和阴道上部侧面的子宫旁组织,并发出与血管相伴行的分支,在阴道前方组成阴蒂海绵体神经丛,主宰阴蒂及周围组织的神经分布。通过刺激女性生殖器神经来了解女性性反应过程的生理实验尚未开展,也没有得到有关女性脊髓损伤而影响反射和生殖器反应等方面的资料。

阴道对性刺激的即刻反应是血流增加,随后是阴道表面的液体渗出和阴蒂肿胀。尽管该渗出液的生成是依赖于雌激素,但在月经周期中似乎并没有发生定量的改变,唯有雌激素剥夺能显著地改变这一类型。这是皮下血管减少以及阴道内皮细胞减少的结果。然而,已经有人观察到绝经后的妇女在接受雌激素的初期对振荡刺激的阈值下降。在一项研究中,给予女性志愿者大治疗剂量的阿托品和甲基阿托品,研究其对阴道血流以及对主观报告的性自我刺激所致的性高潮反应的作用。结果表明,这类药物对刺激阴蒂引起的阴道血流反应没有任何影响,而且对所报告的性高潮也没有任何改变。这可以解释为即使胆碱能调解的神经支配确实存在且有意义,它也是耐阿托品的,并且在行为上不像典型的毒蕈碱受体那样的效应器系统。

另一项研究用光学体积描记的方法记录了阴道血管对阴蒂振荡性刺激的反应。他们发现这一反应很容易重复,并且在整个阴道都可发生,但最大振幅在其前壁下段,并且具有不同的潜伏期(2~3s),累积可超过30s或更久。阴蒂的振荡刺激(80Hz)也引起了盆腔底部横纹肌持久的反射性收缩。该研究者把这一反应当做男性周期性球海绵体肌反射的紧张性的对应反应,作为躯体敏感传入通路具有躯体系统和自主系统方面作用的一个例证。

在静止状态期间,几乎没有阴道平滑肌的收缩,在性高潮期间所观察到的压力增加是由环绕阴道的骨骼肌收缩所致。静脉注射500μg促甲状腺素释放激素(TRH),从而引起尿道和阴道内的压力升高,均由这些器官的平滑肌收缩所致。在另一项有关TRH的研究中,9例受试者中有7例体验到短暂的阴道温热感,并伴随有阴道血流量明显的增加。为证实这一现象是由于中枢还是外周机制所控制,他们将母羊麻醉,一次性将TRH注入供养阴道的动脉。结果发现给药后导致阴道血流和排尿量明显增多。

在一项对绵羊阴蒂神经的神经纤维数量的组织学研究中发现,母羊每侧能计数出4000根以上的纤维,而雄性绵羊仅能计数出2000根,这表明阴蒂的感觉神经分布是十分丰富的。人们还用双极电极刺激感觉末梢器官的方法来研究猫和绵羊的来自阴蒂和邻近前庭区域的传往骶2~4脊神经根的传导神经冲动,结果在与生殖器功能有关的骶3~4水平发现了交叉反射。

下面介绍一下女性性器官的多肽能神经分布。众所周知,生殖道内的功能控制并不是胆碱或肾上腺素能神经,在泌尿生殖系统组织中起到局部非胆碱能、非肾上腺素能作用的是前列腺素和近来日益受到重视的具有免疫活性的多肽类物质。

舒血管肠肽(VIP)就是最引人注目的一个,它于1972年在猪小肠中发现,能松弛平滑肌,具有潜在的扩血管作用,人的VIP含有170个氨基酸前体。VIP免疫活性(VIP能)神经纤维的分布方式在不同种属是相似的。人们以免疫组织化学方法证实了在女性生殖道内,VIP免疫反应活性细胞定位在神经纤维内,但不同种属所含的VIP神经纤维数目有所不同。VIP能神经纤维在阴道、宫颈和阴蒂内最丰富,其次是子宫体和输卵管,卵巢内罕见。在自然的括约肌部位(如宫颈内外口),VIP能神经分布很丰富。切除腹下神经不影响VIP能神经分布;切除宫颈旁神经节则减少卵巢以外的生殖器官VIP能神经纤维的数目;而切除宫颈,子宫体和输卵管VIP纤维消失,但卵巢内这些纤维仍然存在。这些发现表明女性生殖道大部分VIP神经是本体内供应的,来源于局部神经节。VIP能神经纤维似乎分布到上皮细胞、血管和平滑肌细胞,它们之间的密切联系说明VIP对血流、平滑肌收缩和黏液产生具有重要作用。VIP还对所有血管床具有相似的与剂量相关的血管平滑肌松弛作用。在分子水平上,VIP比其他子宫血管扩张剂如乙酰胆碱、舒缓激肽、前列腺素、P物质和雌激素等的作用更强。VIP对女性生殖道所有非血管平滑肌的机械和电活动都具有剂量依赖性的抑制作用,在体研究也证实了VIP的这种松弛作用,例如在子宫平滑肌内存在VIP的特殊受体。在阿托品化和阻断肾上腺素能神经之后刺激腹下神经引起VIP释放的同时伴随有子宫静脉血流的增加,这与外源性VIP的作用相似。性兴奋伴有阴道血流增加(能对抗阿托品作用)、阴蒂肿胀和阴道平滑肌活性增强。由于阴道和阴蒂含VIP神经纤维的血管和平滑肌致密的神经分布,加上VIP在非胆碱能、非肾上腺素能血管扩张和平滑肌松弛中的神经递质作用,人们提出这样一个假设,即VIP在性唤起中起着生理作用。有些试验支持这一观念,如静脉灌注VIP时可导致与剂量相关的阴道血流量的增加,在性唤起时,外周静脉血中VIP浓度有了显著增高。

P物质免疫活性神经纤维定位于阴道和阴蒂之中,可能代表着初级感觉神经元的外周末端。这些神经纤维可能是导致平滑肌活动、血流和分泌增加的神经反射弧中的一部分。免疫组织化学研究首先在大脑,而后在周围和自主神经,特别是主要的传入感觉神经之中证实了P物质的存在。P物质由11个氨基酸组成,分子量为1347,属于舒缓激肽一族,是人们发现的第一个具有脑、肠道双重分布的多肽物质。它的主要作用是影响血流和平滑肌活动,其次对初级感觉神经也具有一定影响。


\section{第四节 性反应周期}

美国著名妇产科教授W·H·马斯特斯于1954年开始了对人类性反应的解剖学和生理学的研究。1957年他与V·约翰逊确定了对正常自愿受试人群的询问、观察和生理指标的记录等一系列技术问题,开始了人类正常性反应的研究。1966年他们根据对男女性活动期间生理改变的研究结果,首次提出了性反应周期,成为性医学史上最重要的发现之一。其将女性和男性的性兴奋过程分为四个阶段:兴奋期、平台期、高潮期和消退期。目前有学者认为马斯特斯和约翰逊的周期模式虽然有助于理解性反应时所发生的解剖学和生理学方面的变化,但忽视了性欲和性唤起这两个人对于人类来说极为重要的主观感受,建议综合各种周期模式,并将性反应周期划分为性欲期、性兴奋期、性持续期、性高潮期和性消退期。但本章节仍按照马斯特斯和约翰逊的周期模式进行阐述,应该注意到性反应周期只是人为划分的四个阶段,它们是连贯的、不可分割的、完整的动态过程。

性反应是指人体在受到刺激后,身体上出现的可以感觉、观察并能测量到的变化,这些变化不仅发生在生殖器官,也可发生在身体其他部位。人类的性反应是极其复杂的过程,男女双方的性欲因性刺激而被唤起,进而发生性兴奋,性兴奋积蓄到一定强度通过性高潮使性能量释放,并同时表现出行为、生理和心理的阶段性变化模式和周期性变化规律,即性反应周期。

男性性反应的特点:男性的性反应模式只有一种。男性性反应存在许多可以识别的不同改变,这些变化通常只与持续的时间有关,与反应强度无关。男性在继不断增强的性唤起后仅有一个单一的性高潮期。在性高潮后,绝大多数男性进入一个不应期,此时他们对进一步的性刺激无反应,甚至感到厌恶。即在射精后,尽管一部分人的阴茎可以部分或完全勃起一段时间,但几乎所有的男子,对性刺激不能再次唤起性兴奋,也不会再次射精。在可能发生第二次性高潮反应前,他们一般必须恢复到性唤起的兴奋水平。所以对于多数男性来说,只有在不应期结束后,才能重新勃起和射精。而不应期的长短受许多因素的影响,尤其是年龄。处于青年期的男性不应期可以短至数分钟,而在老年期可长达数小时及以上。

女性性反应可归纳为三种不同的反应模式。第一种模式:为女性完整的性反应周期,性唤起缓慢,但在性高潮后没有不应期,如果继续给予有效的刺激,可以获得不止一次的性高潮。第二种模式:未能获得性高潮,性紧张水平仅仅波动于平台水平,由于未能达到高潮释放,其消退期缓慢,持续时间长。第三种模式:性唤起迅速,很快达到性高潮又很快消退。女性这三种不同的性反应模式是女性复杂的性反应变化过程的简化和代表。对女性性反应的评价,反应强度和持续时间都是应该考虑到的因素。

和男性的性反应特点不同,女性在性高潮之后,没有不应期,如果继续给予有效的性刺激,可以获得不止一次的性高潮;女性在性反应周期中,消退期过程比男性缓慢,尤其是性器官的充血消退时间较长,可达10~20分钟。

兴奋期:指性欲发动,性器官及全身都进入性紧张和兴奋阶段。对于男女两性来说,无论来自肉体还是精神的性刺激都能引起性兴奋。唤起性兴奋所需的时间受心理状态、环境、体力和刺激的有效性等因素的影响,兴奋期越长则消退期也越长。男女两性在性兴奋期的主要区别在于:男性一般能迅速达到性兴奋,从一开始就渴望性交。女性的性唤起则受社会心理的影响,女性更渴望得到爱抚、拥抱和温存,一般不会希望马上进行性交,只有等她们调理好心态,才能产生性唤起。

(1)阴茎:男性对有效性刺激的第一个生理反应是阴茎勃起。阴茎勃起是因为阴茎海绵体和尿道海绵体内血管充血所致。由于海绵体内的腔隙与螺旋动脉相通,性兴奋时,螺旋动脉开放,引起海绵体充血肿胀,使阴茎变得粗大而坚硬,造成勃起。男性可以有意延长阴茎勃起时间,在兴奋期中,阴茎勃起可部分消失,随后又很快地获得,如此反复许多次。年轻男性勃起较老年男性快,中枢神经系统的高级部位对勃起有明显影响,如精神性刺激可引起勃起。而在性唤起的兴奋期,如果性刺激中断,或介入非性刺激,如突然来临的响声、灯光或温度的突然变化等,可致勃起消退,再次给予性刺激或去除非性刺激,年轻男性可很快重新勃起,而50岁以上的男性再次勃起较困难。

(2)阴囊和睾丸:男性的阴囊和睾丸像其他生殖器官一样,也能对性刺激作出反应。既有局部血管的充血,又有肌张力的增加。当性紧张升高时,在兴奋期阴囊皮肤和肉膜伴随局部血管的充血和肉膜平滑肌纤维的收缩而拉紧增厚,阴囊由平时的皱褶状态开始变得平滑,内部直径缩小。如果长时间维持兴奋期性紧张水平,而未达到向平台期进展的强度,紧缩和充血的阴囊可以松弛,这时阴囊的皱褶重新出现。此外,因提睾肌收缩,使精索缩短,睾丸提升并因充血而开始增大,尿道外口出现少量由尿道球腺或前列腺分泌的液体。

(3)生殖器官外反应:全身肌紧张增高、心率增加、血压升高和呼吸加强等。男性的乳房反应不像女性那样明显,有少数的男性会发生性红晕,但到性唤起的平台期才会出现。

(1)阴道:在兴奋期女性的阴道反应有:阴道滑润、阴道扩张、阴道壁颜色变化三种主要的变化。女性对性刺激发生反应的第一个生理学征象是阴道滑润。一般在性刺激开始后10~30秒,由于阴道组织的血管充血,使液体从阴道壁渗出而使阴道湿润,老龄妇女的阴道润滑作用减弱。未受性刺激时女性的阴道前后壁基本是紧贴在一起的,当兴奋期性紧张度增加时,阴道的解剖学变化是阴道壁的扩张和阴道长度的增加。在高度兴奋期或平台期,阴道的长度平均可加长2.5cm。兴奋时,由于阴道壁内血管充血,阴道壁的颜色由淡紫色缓慢变为深紫色。在兴奋早期变色特征是斑块状的,但接近平台期时,由于骨盆充血加剧,整个阴道壁颜色加深。

(2)阴蒂:在兴奋期初期,阴蒂的变化不明显,而到了兴奋期晚期,由于阴蒂血管充血引起阴蒂头和阴蒂体的肿胀而导致阴蒂体的直径增大,这种变化增加了阴阜和小阴唇运动时对阴蒂刺激的可能性。

(3)大阴唇:无性刺激时,女性两侧大阴唇在中线合拢,覆盖住小阴唇、阴道口和尿道外口。在兴奋期,由于大阴唇的血管充血,使大阴唇肿胀,两侧大阴唇分开,贴靠于会阴部,同时伴有阴唇向上和向外离开阴道口。

(4)小阴唇:在兴奋期,小阴唇由于血管充血而增大,其厚度至少增加2倍,有时可达到3倍。增大的小阴唇从已分开的大阴唇开口中伸出,使阴道的长度增加,同时小阴唇的充血可避免阴茎插入时对外生殖器的损伤。

(5)生殖器官外反应

1)乳房:在兴奋期,乳房对性紧张的反应是乳头勃起和乳房肿胀。在性紧张的早期,由于乳头结构中的肌纤维不随意收缩引起乳头的勃起。随着性紧张度的增加,乳房内血管充血使皮下静脉扩张,乳晕肿胀,色泽加深。到了兴奋期晚期,因乳房深部组织的血管充血反应,使整个乳房体积增大。

2)性红晕:性红晕的强度和分布类型在个体之间是有差异的。但作为一个规律,性红晕反应的强烈程度可以作为评价反应中妇女所体验到的性紧张程度的直接指征。到兴奋晚期,部分女性由于皮肤组织的浅表血管充血,皮肤上出现斑丘疹样红色皮疹,首先出现在上腹部,然后迅速扩展到乳房。最初出现在乳房表面的前方或上方,而后延至胸前壁,并涉及乳房侧面和下表面。

3)肌强直:在性紧张的性兴奋晚期,由于四肢和腹部肌肉的紧张性增高,出现全身肌肉不随意地收缩,其特点是既延及全身又有特定部位。通常肌肉不随意的发作方式有规律地收缩和痉挛,但肌肉的收缩往往受性交体位的控制。兴奋期可伴心率加快,每分钟可达100~120次,呼吸加快及血压轻度增加。

平台期又称性持续期、高涨期,是指性兴奋不断积聚、性紧张持续稳定在较高水平的阶段。平台期实际上是性交抽动的时期。与兴奋期相比,平台期没有突出的生理变化,而是生理反应在兴奋基础上的持续和加强,使在心理上进入明显兴奋和激动状态。

(1)阴茎:由于血管充血的轻微增加使阴茎完全勃起,阴茎体较兴奋期更为坚硬。在性反应周期的平台期,尿道内径扩张,可为性唤起前的2~3倍。平台晚期,大约20\%的男性出现阴茎龟头的颜色变化,呈深紫红色,这是由于静脉淤积造成的。平台末期,位于阴茎基底部的尿道球发生明显肿胀,阴茎有一个较小的充血性直径增加。

(2)阴囊和睾丸:因性紧张性增高而引起的阴囊反应在兴奋期就已完成,偶尔会有阴囊被膜的显著增厚和绷紧,但这只发生在兴奋期异常短的情况下。阴囊在平台期的高潮期缺乏特异性反应。然而,睾丸进一步抬高、旋转、肿胀直至达到最终的射精前紧贴在男性的会阴位置。当睾丸确实上升到紧贴于男性会阴位置时,如果有效性刺激继续存在,高潮期会随之而来。完全的睾丸提高是射精即将来临的特殊指征。约85\%的男性左侧睾丸在阴囊内的位置较右侧低,所以阴囊抬高时,左侧的睾丸必须进一步上移,而往往到射精时才完全抬高和旋转。由于血管充血,睾丸大小的变化通常要到兴奋期或平台期才明显。睾丸在高潮期到来之前,体积较刺激前增加50\%,有的男性可增加100\%。一般而言,在没有高潮期释放的情况下,平台期性紧张水平维持越长,则睾丸深部血管充血越严重,睾丸体积的增加越明显。

(3)生殖器官外反应:在此期,约50\%以上的男性出现一定程度的乳头勃起,约25\%的男子有性红晕。性红晕从上腹部开始,逐渐扩散到前胸壁,进而波及颈部、面部和前额。尽管这种性红晕可能出现在性紧张快速升高状态下的兴奋晚期,但它通常发生在平台期,而后随着男性性紧张增高而迅速扩散。男性在性唤起的平台期也会出现肌肉紧张,对大多数男性来说,脸部和颈部肌肉的扭曲,腹部、背部、大腿、臀部肌肉的紧张,都标志着高水平的性唤起。在此期,心跳明显加快,每分钟可达100~175次,呼吸增加到30~40次/分,血压增高,呼吸进一步加快。

(1)阴道:由于阴道局部血管充血反应更加明显,使阴道更湿润。在此期,阴道外1/3段及前庭腺静脉充血而致阴道外1/3的内径缩小,引起阴道口缩窄,阴道内2/3段扩张。在平台期,阴道的宽度和深度进一步增加。

(2)阴蒂:在平台期,阴蒂对性刺激表现出明显的变化,阴蒂体从正常的阴部悬垂位置后退,并回缩紧贴在耻骨联合的前界。在临近高潮时,阴蒂长度至少减少一半,这一退缩反应在直接对阴阜区域进行手法刺激时发生于平台早期,而在性交或乳房手法刺激时,阴蒂的退缩发生在平台期末。阴蒂的这一回缩反应可逆转,在延长的平台期内,阴蒂可回缩和重新露出数次。应该强调的是,阴蒂回缩意味着性唤起的提高,而不能被误解为性紧张的释放性丧失。如果减慢或撤销刺激,使性紧张度水平下降,退缩的阴蒂体和头将返回到正常的阴部悬垂位置。当重新开始有效性刺激时,阴蒂体的退缩将再度出现。

(3)大阴唇:进入平台期,大阴唇充血更明显,并向上移位。在性周期的平台晚期和高潮期大阴唇无明显变化。

(4)小阴唇:在此期小阴唇出现鲜明的颜色变化。末产妇的小阴唇从粉色变为鲜红色;经产妇的小阴唇从鲜红色变为深紫色。由于这种特异性,小阴唇在性反应中的称为“性皮肤”。因此,“性皮肤”反应的出现是高潮期即将来临的特有临床征象。

(5)生殖器外反应

1)乳房:在性紧张刺激下乳房显著增大、饱满,乳晕显著充血,乳晕开始肿胀。哺乳过的女性乳房增大不明显,甚至不增大。

2)性红晕:在平台期晚期,性红晕反应达到高潮,在乳房前侧面经常出现一种粉红色的斑点,即“性红晕”。这对有此反应的女性来说,是性唤起程度很高的标志。随着性兴奋的增加,性红晕可以迅速扩展至下腹部、肩部甚至肘前窝。

3)肌紧张:随意和不随意的肌紧张进一步增加,面部肌肉出现半痉挛性收缩,颈部肌肉越来越硬,腹部和背部肌肉自主或不自主的收缩,使背部呈弓状。由于臀部肌紧张增加,大腿强直性伸直并挤在一起。出现腕足痉挛,即一种手和足部肌肉的痉挛性收缩,并无意识地处于抓握反应中。

4)心率显著加快,每分钟可达100~175次;血压升高,收缩压升高2.67~8.00KPa,舒张压升高1.33~2.67Kpa;呼吸进一步加快;神经系统的兴奋也达到更高程度;无明显的排汗反应。

高潮期:指性反应的顶峰,性紧张已到了一触即发的程度。是在性平台期基础上迅速发展起来的身心极度欣快的阶段,是性反应周期中最关键、最短暂的阶段。

(1)阴茎:男性性高潮的主要特征是射精,即精液从阴茎口喷射而出。射精是一种反射活动,男性在性高潮即将来临时,会体验到一种强大的压力,使射精迫不及待,尽管距高峰还有1~3秒时间,但要想控制几乎是不可能的。所以此过程一旦启动,就不可能停止,直到射精完成。在人类性反应周期中,男性的性高潮与女性的根本差别在于男性有精液排出。来自附性器官(前列腺、精囊腺、射精管)的精液成分排向尿道前列腺部,并在压力下通过整个阴茎尿道,最终射出尿道口,这一过程就是男性性高潮的生理表达。男性的射精过程可划分为两个阶段:第一阶段是精液溢出,即输精管、精囊腺和前列腺收缩,将各自的内容物释放到尿道的前列腺部。同时位于阴茎尿道基部的尿道球肿胀,体积增大2~3倍。膀胱内括约肌紧闭,可有效防止逆行射精和精液与尿液混合的发生。膀胱外括约肌放松,使精液沿着尿道前列腺部流入肿胀的尿道球部。第二阶段是正式射精,即性高潮期。阴茎的射精反应来自尿道括约肌、尿道海绵体肌、坐骨海绵体肌和会阴肌发生有节律的收缩,将精液沿着尿道阴茎段挤出,并使之从尿道口连续喷射而出。这些肌肉收缩的间隔时间非常接近,开始时收缩间隔时间为0.8秒,前后收缩3~4次,以后收缩频率与幅度减小,直到停止。在射精后的短暂时间里,绝大多数男性对进一步的性刺激呈不反应性,即不应期。不应期的长短在不同男性中及同一男性的不同时期差异很大。一般而言,年龄越大,不应期越长。30岁以下的男子,精力充沛,不应期短;而60岁以上的男性在一次性高潮后,数小时甚至数天内不能再射精。

(2)阴囊和睾丸:射精时,阴囊皮肤也伴随着出现高度收缩与增厚,睾丸继续处于上升位置。

(3)生殖器官外反应:乳头竖起。在有性红晕的男性中,性红晕充分发展,其程度与性高潮程度相平行。随意肌与不随意肌发生不同程度的收缩,有时皮肤竖毛肌收缩,皮肤上会引起一瞬间“鸡皮疙瘩”。直肠与肛门括约肌也出现间隔0.8秒的与射精时间同步的不随意收缩,但一般不超过3~5次,常常在阴茎部尿道收缩完成前结束。心率可增加到每分钟110~180次,血压升高,呼吸加快,每分钟超过40次。手心、足底有出汗现象。

一般而言,男性的性高潮时间较女性为短,精液射出以后即告结束,心率、呼吸、血压以及肌紧张也很快下降。

(1)阴道:性高潮生理反应的起始点以生殖器的收缩为标志。阴道外1/3在性高潮时出现节律性收缩,收缩次数3~12次,初始是强而有力的收缩,每次时间隔0.8秒。随着收缩次数的增加,收缩间隔延长而性紧张下降。

(2)子宫:在性高潮期子宫发生收缩和提升。子宫收缩从底部开始,进行性通过子宫中段,并终止于子宫下段。收缩程度平行于性高潮体验的强度,经产妇子宫大小增加可达50\%。

(3)阴蒂及大小阴唇:阴蒂及大小阴唇的变化与平台期相仿,没有其他进一步的表现。但在阴道口周围肌肉收缩之前,有一种以阴蒂为中心向外周呈波浪式扩散或一种悬置的感觉。

(4)生殖器官外反应

1)乳房:与平台期相似,乳头的勃起及乳晕肿胀已经建立,血管束醒目地显露,吸吮过的乳头显著增大,超过未受刺激时的基线。

2)性红晕:在性高潮时最明显,分布最广,身体许多部位都出现性红晕,尤其以面、颈、胸及上腹部为甚。

3)肌强直:全身肌肉发生不随意地收缩和轻度痉挛。性高潮时肌收缩不只限于骨盆和生殖器,平台期处于高水平的肌紧张在性高潮时达到了顶峰,女性的整个身体强直呈弓状。四肢肌肉痉挛收缩,手指和足趾呈爪样痉挛收缩。皮肤竖毛肌收缩,出现皮肤“鸡皮疙瘩”,或喉部肌肉痉挛,出现呻吟声等。

4)肛门外括约肌:强烈的性高潮还可伴有2~3次肛门外括约肌的不自主收缩,这种收缩一旦出现,则与性高潮平台收缩同步开始,但结束一般早于高潮平台收缩。

5)心率加快到每分钟110~180次,女性比男性更快。血压升高,收缩压升高4.00~10.7KPa,舒张压升高2.67~5.33Kpa,呼吸频率可达每分钟40次以上。

整个性高潮一般持续数秒,但有的女性可持续12~30秒之久。性高潮的强度取决于性刺激的方式与有效程度,也取决于体力、心理承受力和男女双方的亲密程度。Masters与Johnson曾观察到,女性的性高潮强度,在性伴侣为其进行手淫或刺激阴蒂时最为强烈。与男性相比,女性在一次性活动中可能出现多次性高潮。如果有效刺激持续存在,或在回复到平台期水平前,重新给予有效的性刺激,多数女性能紧接着一次性高潮后很快再次出现性高潮。

消退期:指身体的肌紧张得到逐步放松,性能量得到充分释放,血管充血得到逐渐消退的过程。此期一般需要5~10分钟,性紧张逐步松弛并恢复到性唤起前状态的阶段。从解剖和生理角度看,消退期的生理变化正好是兴奋期的反向过程。

(1)阴茎:在消退期,阴茎的勃起开始消失,并分两个阶段来完成。第一阶段:血管充血迅速消退期,阴茎比平时仅大50\%。第二阶段:阴茎肿胀消退,完全恢复到未受刺激时的状态。

(2)阴囊:阴囊充血和绷紧的外表迅速消失,阴囊缩小,双侧睾丸恢复到充分松弛的阴囊深部位置的时间延长5~20分钟。阴囊皱褶样外表重新出现。25\%男性呈缓慢消退模式。

(3)睾丸:睾丸体积的充血性增大现象在消退期消失,睾丸完全降至松弛的阴囊深部。该过程可迅速或缓慢发生,取决于平台期的长短。

(4)生殖器官外反应:乳头竖起消退。性红晕以出现时的相反顺序迅速消退。全身性肌肉放松,进入本期以后,肌强直很少延伸到5分钟以上。心率、血压和呼吸等变化逐渐趋于平静。多数男子的出汗限于手心与足底,但也可出现在躯干,偶尔涉及头面及颈部。

(1)阴道:继性高潮之后,阴道外1/3的血管充血消退,阴道内径回复到未受刺激时的状态,阴道壁松弛。阴道内2/3缓慢退缩,重新呈现合拢状态,需10~15分钟才能完成。阴道壁的深紫色渐渐恢复到未受刺激时的浅紫红色。

(2)子宫:回到正常位置,子宫颈外口轻度展张开,膨胀的子宫在性高潮后10~15分钟内恢复至未受刺激时的状态。

(3)阴蒂:返回到正常的阴部悬垂位置。阴蒂体和阴蒂头的肿胀消退较缓慢,有时需要5~10分钟。阴蒂血管充血的消退更为缓慢。

(4)大阴唇:未产妇大阴唇恢复到平常的厚度和中线的位置,经产妇大阴唇血管充血消退。

(5)小阴唇:性紧张引起的性皮肤的颜色改变,常在性高潮后10~15秒内从深红色或鲜红色迅速变为浅粉色。小阴唇颜色消退的第二阶段,从粉红色到未受刺激时的少色状态相当迅速。

(6)生殖器官外反应

1)乳房:晕充血和乳头竖起迅速消退,乳房肿胀消退,体积缓慢复原并慢慢恢复到正常状态时乳房的静脉树走行。

2)性红晕:以性红晕出现时的相反顺序从身体的不同部位迅速消退。首先从背部、臀部、腹部和四肢迅速消退,然后从胸部、乳房、颈部消退,最后从上肢消退。

3)肌张力恢复正常,在此期肌强直很少延续至5分钟以上,但它的消失没有血管充血消失得那样快。心率、血压与呼吸均恢复平稳,少数女性局部或全身会出汗。

自1974年起,又有一些性学专家对人类性反应周期提出新的见解。这里介绍两种不同的模式。

海伦·辛格·卡普兰(Helen Singer Kaplan)是美国的性学专家,也是性行为治疗学的重要奠基人之一。她根据自己在性治疗工作中积累的经验,于1974年提出:男女两性在发育过程中性反应的发生顺序是相反的,男性由青春期集中于生殖器区域到中年后发生泛化,扩展到全身;而女性则相反,青春期时性敏感区比较泛化,中年后才集中于生殖器区域。她认为性反应不是一个连续的过程,而是两个相对独立的期或组成部分,即首先是生殖器的血管充血,然后是高潮期反射性的肌肉收缩。这种划分的理由是:

1.两期由不同的神经系统支配:男性阴茎的勃起和女性阴道的润滑都是由自主神经系统的副交感神经支配的,而男性射精或女性高潮则是由交感神经支配的。

2.两期所涉及的解剖结构不同:兴奋期涉及的是血管(充血),高潮期涉及的是肌肉(收缩)。

3.两期对损伤、药物(吸毒)或年龄等因素的敏感性不同:如男性的勃起能力受年龄影响较小,而性高潮射精后的不应期则明显受年龄的影响。

4.大部分男性能随意控制射精反射,但是不能控制勃起反射。

5.血管充血的损伤和高潮反应的损伤所造成的性功能障碍是不同的:勃起障碍是血管充血损伤的结果,而早泄与不射精是高潮反应受损伤的结果。同样,女性可以有强烈的性唤起和阴道润滑,但却达不到性高潮。

卡普兰的双相模式有助于了解性反应的性质及采取针对性强的性功能障碍的治疗措施。1979年,卡普兰又进一步修改了她的性反应模式,提出性反应的三相模式,认为在兴奋期之前还应有一“性欲期”,并对性欲期的障碍及其治疗方法进行了探讨。

齐勃盖德(Zilbergeld)和艾力森(Ellison)于1980年对马斯特斯和约翰逊的性反应模式提出异议,他们认为马斯特斯和约翰逊几乎把全部注意力集中在性生理反应上,而忽视两个特别重要的主观因素,即性的欲望和性唤起。他们给性欲下的定义是人们希望过性生活的频繁程度,即每天1次,还是每周1次,还是每半年1次。而性唤起则指在性接触中所能兴奋起来的次数,指激发或发动人们进入性反应阶段的次数。

根据这样的认识,齐勃盖德和艾力森提出性反应的五期划分法,并用这一概念去理解性反应过程,这五期中的每一期不但互相关联而且各具独立性。这五期是:①兴趣和性欲;②性唤起;③生理准备(阴道润滑,阴茎勃起);④性高潮;⑤满足(一个人对所发生的经历的评价或感受)。

齐勃盖德和艾力森的五期划分法在治疗性功能障碍方面比马斯特斯和约翰逊的性反应模式更有用和更有实际意义。他们认为性欲发生的主要部位是在大脑,性的主观方面更为重要。

关于性反应的实验室研究始自半个世纪之前,学术界在不断完善对它的认识,美国精神病学会诊断学与统计学手册第四版(1994年)根据神经生理反应对人类性反应周期作了如下划分:

性欲指有发起性活动的主观愿望和兴趣或对性刺激作出反应的意识。性欲分为背景性性欲和应激性性欲,背景性性欲是由性激素决定的,应激性性欲则由个人受到内生的或外来的性刺激所致。

在兴奋期,生殖器及相关器官明显充血,这主要受副交感神经控制。具体反应为阴蒂直径和长度的增加,阴道周围小动脉的扩张,由于血管渗出物透过阴道上皮导致阴道分泌物增加,阴道内2/3扩张并形成帐篷样表现。

一系列反射性、痉挛性收缩是通过交感神经系统调控的。高潮期仅持续5~15秒,是性反应周期中最短的阶段。其生殖器外反应包括全身诸多肌群的收缩,不随意呻吟,舌头冰凉,性红晕达到最大限度,心率、呼吸和血压的增高都达到最大限度。

指高潮之后身体由兴奋到平静的复原或恢复过程,全身各种性生理反应缓慢消退,如生殖器充血逐渐消失,包括性能量和性紧张释放后带来的肌肉松弛的感觉和总体的幸福感。消退期将持续数分钟到数小时。男性在消退期有不应期,女性则能对再次刺激立即作出反应。女性若能达到性高潮,消退期便很短,可以很快平静下来并安然入睡;但若不能达到性高潮,消退期便会很长,使其处于性兴奋状态,生殖器等的充血不能及时消退,女性会感到小腹及会阴部位有憋胀感或不适,从而出现烦躁不安及失眠等问题。

(张觇宇 马晓年)


\chapter{第三章 性心理学基础}

性,人人都有。每个人都能塑造生命,性是我们生命的一个组成部分。人类的性是一种多维价值系统,如果没有性的生物学基础,也就谈不上它的心理成分,所以人类的性心理不仅仅是一个单纯的神经精神活动及其伴随的生理和生化过程,如各种激素、内啡肽与苯乙胺等神经化学物质,肾上腺素与5羟色胺等神经递质的生成、释放、代谢和重吸收等,它也是一种生命感受,受意识形态、道德、伦理等加工和控制,这些文化沉淀的注入使性心理更加复杂化,成为人类隐蔽得最深、最神秘的心理活动。许多来咨询的人都不约而同地说,“这是我和任何人都不曾透露过的隐私”。

一般的教科书和性学著作在谈到性反应时都着重于性生理反应,而不太谈性心理反应。一方面是因为其十分复杂,另一方面也因为它不像性生理反应那么容易客观观察和测定,比如心率、呼吸、血压、皮肤颜色和温度、润滑、充血等生理反应都可以客观地予以测定和记录,有说服性,而性心理活动则不然,多靠受试者的自述,除了反应程度和性质的不同外,它还与受试者的教育层次、表述能力、理解能力等多种因素相关。同一句话可能表达出不同的感受,而不同的描述却可能指的是同一种感受,这就给性心理反应的研究带来极大困难。现代性科学的发展告诉我们:性及性行为不单纯具有生物学含义,更具有心理学和社会学含义,是一种生理—心理—社会现象。也就是说,性有生物、心理和社会三重属性。

综合国内外学者的观点:人类的心理活动就是脑的机能活动,心理也即精神活动,它是以大脑对客观现实的反映为基础的,所以脑是产生心理活动的器官,而客观存在的现实世界是人们产生心理活动的源泉。精神现象是被移植到人脑中的客观世界的映象,按其形式来说是主观的,按其内容来说则是客观的。大脑虽然是精神活动的器官,但是只有在外界刺激的作用下,大脑才产生活动。脱离了客观物质世界,大脑便不可能有独立的精神活动。人的心理活动是在具有健康的脑组织活动的生理基础上,在社会生活实践中通过认知活动对外界客观世界作出积极、能动的应答反应,这种反应活动系统是多水平、多层次、多功能的,涉及神经系统内的机械分子运动(如位移)、脑电活动、神经生物化学活动(如激素和神经递质)及神经生理活动。心理活动包括认知活动,情感活动和意志、意向行为活动。一般来说,人的心理活动和行为表现应该是一致的,行为就是心理活动的外在表现形式。

1.认知心理活动指认识外界事物的心理活动,它包括:

(1)感觉:感觉是一切精神活动的最初环节与源泉,一切感觉来源于物质,来源于实践过程中与客观世界的接触。人们通过眼、耳、鼻、舌、身等感觉器官对外界事物的颜色、声音、气味、形状、大小、软硬、远近等各个属性进行辨认,这是初级的认识过程。

(2)知觉:即在感觉的基础上,通过对事物属性的归纳、综合,得出对外界事物的整体感觉。对同一外界事物,可以通过正确的感知,形成与外界事物相符合的正确映象;也可以通过错误的感知,形成与外界事物不相符合的错误映象,即错觉。这里说的映象是指当时感知的东西在大脑中的成像。

(3)记忆:指以往事物经验的重现,或者说记忆是人脑对过去经验中发生过的事物的反映。这就是大脑所具有的铭记、回忆(再生)和再认(认知)功能。大脑铭记、回忆(再生)和再认(认知)的连续过程中可以呈现正确的或错误的表象,如记忆正确、记忆恍惚、记忆错误。表象即以往感知过的东西在大脑中留下来的映象痕记的再现。映象是恒定的,不随意志而改变,表象则是游移不定的,随意志而唤起或消失。表象往往不是映象的简单再现,而是对原先直观感受的概括,有时甚至是恍惚、错构或虚构的。

(4)思维:人们把感觉到的大量事物加以整理、综合,进行分析比较,并且总结出规律,进行推理和判断,找出外界事物的本质及互相的联系等,再把它们总结成文字加以记载和交流,这就是思维活动。在同类映象、表象等感性材料的基础上,人们既可以通过正确的抽象概括形成正确的概念,也可以通过错误的抽象概括而形成错误的概念。在此基础上还可以分别形成正确的或错误的推理或判断。可见,认识过程发展的每一步都有产生错误的可能性。鉴别主观世界的东西是正确的还是错误的,即是否与客观世界相符合,唯一的标准是实践,只有通过反复实践,才能使反映比较准确。按说客观世界事物本身是没有正确与错误之分的,正确与错误发生于反映过程中。遇到这类问题时我们必须区分两种不同的情况:一种是健全的大脑出现了错误的反映;另一种是大脑不健全,发生大量歪曲的反映。健全的反映器官出现错误的反映有两个原因:一是反映者本身的局限性,对某一事物的认识仍停留在片面的、非本质的、感性认识的阶段;二是事物本身的发展变化还停留在早期阶段,本质特征并未得到充分显露,反映者只能掌握一些表面现象。不健全的反映器官自然不能真实反映客观世界,于是出现大量的幻觉、妄想、胡言乱语、行为紊乱。这时的错误反映是疾病造成的,在大脑机能障碍恢复之前,实践是不能纠正这类错误认识的。所以,错误思想、偏见成见、宗教迷信等都是正常人脑对于客观现实的歪曲反映,而精神症状如幻觉与妄想则是不健全的人脑对于客观现实的歪曲反映。

2.情感活动情绪是人对客观事物的态度或人在认识和接触外界事物时产生的心理体验,又称情绪体验,指人的喜、怒、哀、乐、惊、恐、悲等反应,也称情感活动。情绪和情感活动是复杂的问题,即使对同一事物,不同的人或同一个人在不同的境遇下所产生的情感反应的性质、强度和持续时间也是不同的。

根据客观事物是否符合主体的需要,可以把情绪划分为两类:满意的与不满意的,或愉快的与不愉快的。如新婚之际男女双方总想长相厮守,当然会产生愉快情绪,但随着时间的推移,特别是双方不再继续培养和深化其情感时,就为挣脱对方的束缚而难以享受愉快的情绪,甚至一见面就产生厌恶情绪。可见情绪不是由客观事物本身的属性所决定,而是根据客观事物与主体需要之间的关系所决定的。这样,在满意情绪下人们会产生趋向反应,在不满意情绪下会产生躲避反应,也就是说性活动的频率将受情绪的极大影响。健康的情绪:一是稳定、平和;二是心情愉快。情绪稳定的人将是开朗、豁达、自信、乐观、遇事冷静、承认现实的。满意的情绪有愉快、喜爱、幸福感和狂喜之分,也有尊敬、崇拜、赞美、感激、信赖、自豪之分。当一个人情绪不稳定,忽高忽低,变化莫测,或经常心情忧郁,缺乏兴趣,则是心理不健康的表现。不满意的情绪包括忧郁、焦虑、愤怒、恐惧、妒忌、悔恨、委屈、憎恨、怀疑、敌对、困惑、羞愧、自卑等。情绪反应伴随各种心理过程而出现,如性生活和谐美满引起愉快情绪,若相反则引起不愉快情绪。而不同的情绪状态(心境)又影响人们的日常活动,决定他或她当时愿意做什么,拒绝做什么。比如当一个人因未评定上理想的职称或未分到期望的住房而沮丧、委屈、愤怒时,哪有心思做爱;相反,这些目标都顺利达到的话,在狂喜、自豪的心境下,夫妻往往以做爱来表示庆祝,这时的情绪反应可以称作激情,往往伴有强烈的体内生理变化,如皮肤温度、心率、血压、呼吸频率、竖毛反应、肌肉震颤与肌张力、肌电图与脑电图、血糖等的变化。同时体内机能的变化或内脏疾患也可以引起情绪的改变,如很多疾病都使人感到忧郁,而甲状腺功能亢进有时却引起愉快的情绪。此外,情绪反应还受思维或认识过程的制约。因此,一个人的性价值观很可能影响一个人在性情境中的情绪反应,从而影响其性行为和性功能表现。

3.意志行为活动 意志是指人们根据自己的意愿和需求来认知和变革世界,为了达到自己的目的,就要展开思考、计划,并付诸行动,不达目标不罢休。所以说意志是自觉地,有动机、有目的地调节自己行为的能力,从而达到变革客观现实的目的。意志与一个人的世界观、情感及个性等密切相关。人的食欲、性欲及防御自身安全的基本生理需要都属于保存人类个体和繁衍种族目的的行为活动,都是低级的意向活动。而以改造世界为目的的行为才是高级的,并通过自己的行动达到一定的目的,称意志行为活动。人的活动主要是在社会生活中进行的,人的行为主要指社会行为,性行为也不例外。所以对人类性心理的形成和发展,以及性活动与性行为的考察离不开社会。意志行为活动包括以下三种主要表现:一是克制自己(高级神经活动的内抑制过程),即不采取行动的能力,这时意志与欲望处于对立的地位。二是坚韧不拔地克服重重阻力,以实现最终目标(高级神经活动的兴奋过程)。三是在同时存在相互矛盾的欲望和行为动机时,意志表现为作出选择的能力(高级神经活动的分化抑制过程),它既与神经生理特征相关,也与后天社会实践的锻炼有关。

健全的意志表现为自觉性、果断性和自制力,如果一个人在处理问题时总是懒散、被动、强迫、草率、束手无策、缺乏自控甚至过于冲动,则是意志不健全的表现。行为是有一定目的的一系列动作,是在意志支配下发生的,如果出现意志障碍,行为也将出现异常。行为的协调表现在意志与行为的一致和行为的反应适度。如果一个人的思想、语言和行为自相矛盾,做事盲目、反复无常,那么他的行为肯定是不协调的,不会达到预期效果。行为也受欲望左右,如果欲望异常,行为也会出现障碍。例如性欲倒错看上去是性行为的障碍,反映出来的却是性心理障碍,其性欲往往是幼年性欲的再现。

心理健康应包括如下内容:

1.正常的心理状态,即每个人应该有健康的情绪、健全的意志和协调的行动。

2.良好的人际关系 人与人要通过思想、语言、行为相互交往、相互影响、相互制约并伴有一定的心理体验和心理反应,故人际交往要态度诚恳、情感真挚。社会心理学研究发现,人际间能否建立良好关系往往与最初形成的观感有关,也即所谓的第一印象。比如一个人的外表、气质、衣着、姿态、神情、语调、谈吐内容,甚至发式和修饰等,特别是人的可接近程度、性格相似性或互补性及相容性等都将影响人际关系。有了良好的人际关系,才能有自身的安全感和满足感,相反,若处处斤斤计较、胆小、多疑、孤僻则反映出心理上的不健全。为培养健全的心态,有时必须放下架子,设法与他人平等相处才行,克服盲目的优越感也很重要。

3.完善的社会适应能力 人在社会中生活就必须遵循社会行为规范,对自己的性角色、性能力作出适度的估价,才能搞好自己的家庭小环境,才能适应社会,充分发挥自己的聪明才智,为社会贡献自己的力量。

性是人的自然属性与社会属性的统一。作为自然属性的性,是指男女在生理构造上的差异和人生来具有的性欲望和本能,它是人类生存和繁衍后代的必要基础条件。从生物的形态学和生理学来理解,性是伴随着性生殖而出现的。人的基因与性器官的差异形成了雄性和雌性之分,性征便是两性差异的表达。作为社会属性的性,是性的本质体现。人的性需要,不仅包括生理性需要,也包括社会性需要。恩格斯指出,人类婚姻家庭从群婚到一夫一妻,再到现代性爱的发展,完全是由生产力的发展状况和生产关系所有制的性质所决定的。

性是人的自然属性和社会属性的统一体,这说明性既受人类发展的生物规律的支配,又受人类社会文化发展条件和各种社会需要的制约,两者是有机统一体,密不可分的。

性别是人类的基本特征之一。性别不仅仅具有生物学意义,还具有社会学意义,它决定了一个人的社会地位、行为处世的社会规范,这就是性别生理现象的社会化。一个新生命诞生时,人们首先关心的往往是婴儿的性别:“是男孩还是女孩?”人们依据婴儿外生殖器的不同形状来分辨是男是女。也就是说,从这一时刻起,这个新生命就有了性别之分,就被赋予了男性或女性的特殊身份。婴儿这种性别上的差异,是由父母精卵细胞结合时的性染色体所决定的。性染色体的不同造成了男女在生理构造上的差异,尤其是生殖系统的不同。尽管男女在出生时身体上已经有了不同的构造,但这并不意味着一个男婴或女婴必定会长成一个真正的男人或女人。一个男婴或女婴要达到符合他(她)所处的社会文化体系中的标准,还要经过一个漫长的过程,这个过程就是“性角色的发展”。人一生经历的种种大事和他(她)在社会、家庭中的地位都与性别有关。让我们先来看看以下几个概念:由于性是人的自然属性和社会属性的统一,因此性在生理、心理、社会各方面有不同的含义。“性别”、“性身份”或“性别角色”分别从性的三个构成方面反映了性的特质,它们的区分涉及生物学、心理学和社会学的知识。

(1)性别:是生物学词汇,常指男女两性在解剖、生理或生物学上的差异。它包括男女两性染色体不同,性腺不同,性激素不同,生殖道与外生殖器不同和第二性征不同。

(2)性身份:即心理性别,可以理解为心理学概念,是心理学上的词汇,又称性别自认或性认同或性别同一性。它是指男女两性在生理差别基础上的心理差异,主要表现在与性别有关的性格、气质、感觉、思想感情、智能和行为等方面。初生婴儿的心理性别为中性,他们的心理性别是父母按其生物学性别,依照社会文化标准、民族和民俗的要求培养出来的。研究者认为,男女在心理学上存在差别的特点多达50种以上。美国学者马可比(E.E.Maccoby)和杰可林(C.N.Jaklin)根据大量的客观研究指出,男女确实存在的心理差异有四项:1)女孩的语言表达能力较好。2)男孩的视觉、平衡觉能力较强。3)男孩的数学能力较强。4)男孩更为好斗。

(3)性别角色:即社会性别,是社会学概念,用于表现男女差异的社会行为模式。它指的是一个人在生物学性别基础上,在一定的社会地位、文化环境下逐步形成的特有的性别表现形式、行为模式或身份,是为社会所认可的男人或女人的行为方式。一般来说,一个人的性角色由其生物学性别所决定,并与之一致,因此其性别自认(心理性别)也是没有问题的。男女在家庭和社会生活中扮演什么角色,主要由社会的伦理、道德、风俗、传统等社会文化所决定。性别角色是一个人心理上成熟与否的重要组成部分,并受到许多因素的影响,主要有以下三方面的因素:①生物学因素,包括激素、解剖结构、生理功能、身长速度和发育早晚等;②文化因素,包括家庭的传统观念,在家庭中男女的地位,社会对男女两种性别的态度等;③心理内部因素,包括人格类型、内心冲突、早年经验、生活事件等。而早年的经验也相当重要,它往往决定一个人对待处理生物、文化和心理事件的方式。婴幼儿虽然对语言懵懂无知,但从此期就应开始进行“潜移默化”的性教育,让婴幼儿形成正确的“性别自认”和充当恰当的“性别角色”,使其心理成长与自身性别相一致。从古至今,绝大多数社会都认为男女在言行举止、生活习性、待人处事、思维方式、价值观念等各种方面都应该有自己不同的标准,都应该遵循不同的基本原则,也就是所谓的“男女有别”。人的整个一生都受所处社会的种种准则所影响,人们往往用这些准则来衡量一个男人或一个女人的行为方式,并以此来要求一个男人或者一个女人应该怎么做,不应该怎么做。不同性别的人在社会上要扮演不同的角色,如传统观念认为女孩应该文静,感性,依附于男子等等,而男孩应该自信、好强、不为感情所左右,甚至认为调皮的男孩今后“有出息”,他们所受的教育与女孩显然不同。传统的性角色观念在西方与中国大不相同。在西方,男孩成年之后都知道自己长大是要干事业的,男人应当外出挣钱养家,而女孩将来都要结婚、生孩子、操持家务等。这种观念往往使男女青少年感到迷惑不解,因为它鼓励男孩子把性的成熟当做青春期的标志,而要求女孩子尽量使自己符合社会要求,要做一个好女人,既要在性方面有所压抑,又要求她们具有一定的性吸引力。在国内,由于妇女解放,男女平等,这种性价值观念的差别要小得多。尽管青少年在青春期经常显示出对性明显的好奇心,但大多数仍然相信性并非爱情中最重要的事。在一次对美国青年的调查中得到如下结果:在回答“在爱情关系中最重要的事是性吗?”的问题时,13~15岁的男孩中有69\%不同意,31\%同意;在13~15岁女孩中,77\%不同意,3\%不知道,20\%同意;在16~19岁男孩中,不同意的占84\%,同意的占16\%;在同龄女孩中,不同意的占64\%,32\%不知道,4\%同意。性教育就应纠正这些错误观念。青春期是每个人发展自己的性准则和性价值观念的重要时期,开始时最主要的影响来自父母,而后逐渐被学校、传媒、伙伴等取代。社会发展的趋势要求男女两性在社会生活中各自展现出不同的形象,使性别角色更合理地发展和完善。柯尔伯格认为儿童时期是性别角色形成的关键时期。研究表明,儿童正确认识性别、性取向性别和性别发育的能力必须在双性别的环境中,在孩子和两种不同性别成员的相互关系中,在孩子与异性性别的不断比较中逐步发育并成熟。

性角色和性身份一般来说应是一回事,但有时也有例外,如花木兰替父从军时的社会性别(或说社会角色)就暂时为男性,与其生物学性别不一致,但她的性别自认并没有问题。一旦打完仗,她又重着女儿装,恢复了自己的性角色,她不存性身份问题;又如电视剧《蓝花豹》里的女主人公蓝花豹就是一个典型的例子,母亲生下她就去世了,她父亲害怕别人说他绝后,对外一直说孩他妈给他留下个儿子,从而把她当儿子抚养,甚至给她娶亲,“借种”生子,她的社会性别,即性角色便成了男性,即使在户口本和结婚证上也不例外,但蓝花豹的性身份和性别是女性这一点是谁也改变不了的事实。主人公长大后也始终认为自己是女性,是不能娶妻的,而且她心中早已埋藏有对男性的爱情,只是不敢也不愿反抗愚昧的父亲。因此她并不存在性身份障碍或性认同障碍。另一些人,明明是男性,却偏偏要做手术把自己改为女性,他们讨厌做男性,这就是性别自认出了差错,即他们的心理性别与生物学性别不一致,称性身份障碍或性别同一性障碍,又称易性症。发生这种病态表现与婴幼儿时期所接受的性教育有关。如有的父母为了好玩或盼望有个女孩,于是给男孩涂脂抹粉,穿花裙子,还扎上一对小辫,从上到下把他打扮成一个十足的女孩,这固然满足了他们“儿女双全”的梦想,但实际上却害了他们的儿子。当然,这种性别转换症或易性症也可能存在先天性的生物学因素。

现代性学对性行为的定义很广泛,凡是可以带来性愉悦、获得性快感、引起性高潮、达到性满足的行为都是性行为。人与动物的性行为有很大的不同,动物性行为是一种在体内性激素的作用下以繁殖为目的的本能行为,其主要方式是交媾。有的研究者还认为性行为包括自身性行为和与异性的性行为。自身性行为就是不需要其他对象,只通过自娱活动就可以达到性满足的行为,包括性幻想、性梦、手淫等;与异性的性行为包括边缘性性行为和核心性性行为。人类的性行为从目的、对象、方式和结果诸方面来考察,具有远比繁殖后代更为广泛的内涵和外衍。从儿童期的手淫、性游戏,青春期对异性的追求,成年期伴随婚姻和婚外的性生活,直到老年期的性满足,可以说性行为伴随人的一生,并对个人和社会都产生了重要影响。

在与性行为有关的行为中,有四个特别重要的领域:生育、欢愉、性别和情感。性行为与这四方面交叠相联,但又不可与任何一方面等同。生育与性行为是密切相关的,但有大量的性行为与生育无关,同时生育也不一定非要借助性行为,例如人工授精;性别差异在产生性行为中具有极大的意义,但与性别有关的行为并不就是性行为,而同性之间也可以发生性行为;情感对性行为也具有重要意义,但大量的性行为与情感并没有关系,而且许多爱的形式是无性欲的;伴随性行为而产生的欢愉是性的一个关键成分,但并不是说性体验都是欢愉的,同样并不是所有欢愉的体验都是有性欲的。

长期以来,人类对性的认识局限于生物学意义上,判断男女性别亦仅限于外生殖器,对人类性行为的认识局限在如同动物一样的生物本能反应上。随着现代性科学的发展,人们日渐意识到:性及性行为不单纯具有生物学含义,更具有心理学和社会学意义,其是一种生物、心理、社会现象。换言之,人类的性有生物、心理和社会三重属性。心理学家认为,人的心理活动是人脑对客观世界的主观反映,性心理是人脑对性的主观反映,其中包括了对人类自身性现象和各种性活动的认识,这就是人类的性心理,是人类复杂心理活动的重要组成部分。性心理包括人从婴幼儿时期开始的、一生中不同发展阶段的、与性有关或以性为主要内容的心理过程,以及与其人格特质、个性相联系的心理特征、发展规律、性心理体验、变态性心理的产生根源及其特征等综合心理因素。总的说来,性心理是指在性生理的基础上,与性征、性欲、性行为有关的心理状态与心理过程,也包括与异性交往和婚恋等心理状态。

人们以往只把性活动简单地看做是男女之间的一种肉体结合和繁衍后代的手段,而随着人类对性科学的研究,在生物、心理、社会、伦理、美学、哲学等多个领域开展了对性活动的科学分析,加强了对人类性活动的全面认识。从性行为来看,以往所认为的本能行为,现在被看做是具有高度发展的社会性行为。在人类性行为由生物本能向社会化发展的进程中,性心理活动成为联结的纽带,起着决定和促进作用。个体的性心理过程规律和性人格特征决定了其独特的性心理内容,如性观念和性态度,同时决定了其性行为方式。

一直以来,人类受不同时代、不同文化、不同伦理道德规范等因素影响,性在人的大脑意识里具有隐秘性,人们对谈论性抱有羞耻感和罪恶感。因此,与其他人类心理活动的研究相比,对人类性心理的研究显得尤其困难,甚至一度成为科学领域的研究“禁区”。

最早涉及性心理研究的是精神分析学派的创始人弗洛伊德。他在《性学三论》中提到了Libido,提出幼儿性欲、青春期性改变、性变态等性心理动力理论,最早揭示了性心理对人类心理活动的重要影响。弗洛伊德认为Libido是性欲的原始动力,即性欲的内驱力,是支配人们一切心理活动的“心理动力”。

英国最著名的性科学先驱Havelock Ellis说过:“性是一个通体的现象,我们说一个人浑身是性也不为过,一个人的性的气质是融贯他全部体气的一部分,分不开的。”

Lief(1971)提出,人类的性应当从五方面来认识:①生物学上的性;②心理上的性统一性;③性别统一性;④性行为;⑤性角色。其中除第一方面外,其他四个方面均属于性心理学范畴。

性心理过程是人类对待性及其相关事件的具有开端、进程和终端的完整的心理活动过程,是人类普遍存在的行为。按照传统心理学的知、情、意三分法,通常将性心理过程分为性认知、性情绪情感和性意志三个基本方面。

指人脑对性的现象、本质的理性反映过程,即个体对性的认识和察觉。现代认知神经科学的研究表明,人类性认知过程包括对性的感觉、知觉、记忆、思维、想象、注意等心理活动,这些活动分别由不同的高级神经结构以及其间的相互联系完成其功能。当性及其相关事物成为认知的对象时,人们往往先通过对性的感知察觉到性及与性相关的事物,并对它们进行加工处理,逐步产生性记忆,随着性感知和性记忆资料的丰富,形成了对性本质的深刻认识,产生了性思维及性想象等高级的性心理活动。在处理这些特殊的性信息时,不仅大脑皮层等高级神经结构参与了性认知过程,一些相对低级的神经结构如脊髓等也发挥了不可替代的作用。同时,高级神经结构对低级神经结构的抑制、开放等复杂调节过程使性认知过程变得更加精确。人与动物共有的性感觉、性知觉、性注意等低级性认知功能逐渐与性记忆、性思维、性想象等高级性认知功能发生联系,后者正是人类优于动物之处。

性认知过程是每个个体产生成熟的性认识和性行为的基础,是个体在生长发育过程中随着性生理发育的成熟和社会经验的积累逐步形成的性心理活动。在漫长的性认知过程中,性经验发挥着重要的作用,它指个体在性生活实践中所获得、积累的性感受、记忆和行为模式等性认知层面的知识。性经验对个体性行为动机的强弱和性满意度有潜在的影响,同时也影响着个体性能力的形成。性行为经验的积累是丰富性生活的源泉。

指人类对性对象、性行为是否符合自身需要所产生的态度体验。其中,直接和性生理需要相联系的体验为性情绪,主要包括性快感、性反感等;和性社会需要相联系的体验为性情感,主要包括性爱、性美感、性道德感等。性情绪表现为短暂而剧烈,性情感则持久而稳定。性情感的形成主要受人的生理需要、社会条件、文化背景、民族传统、价值观念、年龄等因素影响。性情绪情感具有3个方面的特点:①两极性。既表现出两性之间的爱恋与思念,又表现出仇恨与嫉妒;②复杂性。既可以深藏不露,也可以表露过度或向相反方向表现;③动力性。既可成为人们追求性满足、采取性行为的动力,又可成为人们厌倦、憎恨性行为,消极、悲观、犯罪的驱动力。

意志是人自觉地确定目的,并且根据这一目的来支配和调节自己的行为,经过克服困难,以实现预定目的的心理过程。性意志专指在性及其相关事件中的意志表现,以实现预定的性目的的心理过程。性意志对性行为的调节作用表现为两个方面:①推动作用。个体为了达到一定的性目的,将性需要转化为性动机,推动个体为达到预定目的,克服困难,采取所必需的行动。如在生物学研究中经常可以观察到,发情期的动物为了争取和异性交配的权利与同性展开殊死搏斗,代价则是自己的伤残和或对方的生命。人类生活中更是如此,如俗语“牡丹花下死,做鬼也风流”,典型地描述了性意志过程的推动作用。历史上,很多战争是因为统治者的婚姻、爱情或性而引发的,如古希腊名著《荷马史诗》中的特洛伊战争就是为争夺美女海伦而爆发的;中国清代吴三桂叛乱的理由也是“冲冠一怒为红颜”。另一方面,性也可以作为消弭战争,换取和平的手段,如中国历代中原统治者以“和亲”的方式与周边民族结成“秦晋之好”来避免军事冲突。②制止作用:个体性意志可以抑制不符合预定目的的行为,在性生活中实行性控制和性调适,约束或克制某些性行为,或者通过性升华转化某些过度的或不良的性行为。这种作用在某些以禁欲为教义的宗教中表现得尤为明显。当个体性意志过程的制止作用被广泛认同并作为制度时,就成为该文化背景之下人们所遵循的戒律、法规、伦理、道德等,使性心理过程与性社会范畴产生了联系。

人类的心理活动分为无意识活动和意识活动。相应地,人类性心理的内容也包括性无意识和性意识。

指不知不觉、未意识到的性心理内容。尽管不为意识所知,它却是性心理的主要部分,也是性心理学家研究的重要课题。人类对自己的性了解甚少,说明大部分性还潜藏在无意识中。性无意识包括:

(1)个体觉醒状态时,在感受阈限之下的性思维和性行为,如某些性意识行为经过多次重复之后转化为自动化的性习惯行为。

(2)睡眠、催眠等特殊意识状态时,健康的个体没有意识到的性心理活动,如性梦、梦遗、催眠时的性心理反应等。

(3)病态或特殊精神状态时,个体未意识到的性心理现象,如性幻觉、性妄想等。

指人意识到的一切性心理活动的总和,即个体自觉的性认知、性情感体验和性意志行为的统一体。性意识对性行为起着导向作用。目前研究所知的性心理的基本成分主要指在性意识层面的内容,从宏观和微观的两个不同方面,意识层面的性心理包括具有普遍意义的性心理过程和具有个性特征的性人格。

性人格(sexual personality)是指个体在长期的社会化过程中逐渐形成的对性及与性有关的事件所表现出的独特、稳定、统一的反应模式,也称之为“性个性”。在性人格系统中,个体的人生观、价值观、道德观在其性心理活动产生过程中起着意识导向的作用。同时,人类性心理和性行为受到社会文化、历史传统、风俗习惯等各种因素的影响,于是造成性心理活动的个体差异性很大。性人格表现在性自我意识、性需要、性动机、性能力、性兴趣、性差异程度、性行为等方面,代表一个人有关性的所有内部倾向性和心理特征的总和,具体论述如下:

指个人对自身的性、性别、性别角色的认识和态度。其中,自身的性属于生理层面,性别属于心理层面,性别角色属于社会层面。当个体能认识、体验这三个层面的不同内容,并有意识地自觉调控自己的心理和行为以与自身的性相符合时,性自我意识就成熟了。性自我意识的感受、作用、价值等主观感知过程的体验与认识,主要有三个层次:性别意识、性欲意识和性观念。婴儿能感受到性器官部位皮肤的高度触觉敏感,标志着性欲意识的神经生理基础。孩子到18个月至3岁时,自我性别的辨认已经确认。进入情窦初开的青春期,性意识发生精神动力学上的巨大变化,萌生性欲意识,产生越来越强烈的亲近异性的欲望。因此,性自我意识是个体社会化的结果,是性人格的重要组成部分,是人类性行为区别于动物的主要特征之一。

是个体对性需求在头脑中的反映。它的根本特征是具有实施性行为的动力性,即为引发外在的性行为提供内在的能量和源泉。人的需要分为自然性需要和社会性需要。性需要归属于自然性需要,也称为生理性需要或生物性需要,如对空气、食物、水、睡眠等的需要一样,为人类和动物所共同具有的需要。但人类这些包括性需要在内的自然需要同时具有社会性。社会性需要则为人类所独有的,表现在两个方面:一方面人类性需要的满足受到风俗、习惯、法律、经济等社会因素的制约,表现为内外的矛盾性;另一方面为了合情、合理、合法地满足性需要,人会积极地寻求交友、恋爱、婚姻等正常的社会行为以实现性目的,表现为内外的一致性。

因为人性需要的对象和满足方式受社会条件和社会文化的制约,正常人的性需要通过爱情实现。个体对生活中必不可少的性生活、性对象的渴望和追求是性需要的表现,但正常人能够根据客观条件和社会行为的道德规范有意识地调节自己的性需要。

性动机的产生需要两个必要条件:一是以性需要为前提,一是以外部刺激为诱因。性动机是一种直接推动个体进行性活动和满足性需要的内驱力。最易引起性动机的外界刺激是感觉刺激,即与眼、耳、鼻、舌、皮肤等感官对应的与性内容有关的视、听、嗅、味、触觉信息。如妻子身着漂亮性感的内衣,是引起丈夫性动机常见的感官刺激。性动机也受到个体对刺激的主观反应能力和自控能力的影响。与性需要类似,性动机可分为生理性动机和社会性动机。单纯满足生理性或生物性的性需要的动机为生理性动机,如夫妻间的性行为,性冲动下引发的性行为;出于非生理性或生物性性需要的动机,为了获得经济的、政治的、社会关系等方面的利益的性动机为社会性动机,如卖淫、性贿赂等。

个体为顺利完成性活动所必备的心理特征,即包括性感知能力、性模仿能力、性学习能力、性记忆能力、性幻想能力、性观念、性记忆、性经验等心理品质,也包括言语沟通、身体爱抚、性交技巧等技能。性能力的形成和发展来源于生活实践,并在性活动中表现出来,直接影响个体的性生活的质量。因此,每一位成年人应注重性能力的学习和培养,不仅可以提高性生活质量,还能提高心身健康的水平。

指个体对性及其相关事件产生的接近、探索、从事和坚持活动的态度和倾向性。性兴趣对于个体性心理和性行为有重要影响,具有导向作用。如个体对性感兴趣,那么他的心理活动和行为就会主动指向性;如个体对性缺乏兴趣,其心理活动和行为就会回避性,或在性活动中处于被动地位。同时,早期社会生活经历也对性兴趣有决定性的影响。有研究表明,一些民族的母亲习惯用怀抱的方式抚育婴儿,则会造成男性对女性的胸部和乳房较感兴趣,女性身体的这些部位更容易引起其性兴奋和性幻想;相反,在母亲用后背背着孩子长大的民族中,最容易引起男性性兴奋和性幻想的女性身体部位则是脖子和后背。

是一个性心理学的概念,指个体男性化或女性化的程度,反映某性别的个体在性格、气质、行为方式等人格特征方面的性别倾向性。性差异程度从个体的性格特征、气质类型以及行为特征上区分男性化、女性化或中性化,而抛开了两性解剖学和生理学的差异。例如,男性女性化(feminization)就是指男性在行为举止、性格、气质、角色身份等方面具有女性特点;女性男性化(masculinization)与之相反。男性化或女性化的结果,一方面可能使个体在保持自身性别优秀特征的同时兼具异性的优秀特征,使其成为社会适应性更强的双性化人;另一方面也可能使其自身性别特征削弱或消失。个体的性差异程度主要受社会因素、心理因素和生理因素影响。

性人格特征在性心理的结构中具有重要位置。个体的性人格特征具有独特性、社会性、整体性和稳定性,其形成受生物因素、家庭教育、社会环境、实践活动等因素影响,在个体社会化进程中逐步形成。个体性人格的独特性决定了他们对性的认知模式和行为方式。因此,性心理障碍者的发病几乎都与其变态的性人格有关。

(邸晓兰 彭旭 刘明矾)


\section{第二节 双重性态度}

性态度是人的一种稳定的心理状态,是一个由性认知、性情感和性行为取向三种因素构成的较为持久的系统。由于人们可能会对与性有关的想法和行为产生不洁感、神秘感及压抑感,因此很多时候会表现出与真实想法不一致的外显性态度。20世纪中期,美国心理学家Greenwald和Banaji在内隐性社会认知这一新研究领域的基础之上,提出了一种关于态度的新概念———内隐态度,即过去经验和已有态度积淀下来的一种潜在的影响个体对社会客体对象的情感倾向。Wilson和Lindsey等人进一步提出了双重态度模型,认为人们对于同一态度客体能同时作出两种不同的评价,一种是能被人们所意识到、所承认的外显的态度,另一种则是无意识的、自动激活的内隐的态度。这一理论的提出引发了人们对态度的形成、态度的改变、态度的测量和态度与行为一致性的思考。根据Wilson和Lindsey的观点,双重性态度有两个特点:第一,在内隐性态度形成后,当人们一遇到态度客体时它就会被自动激活,不需要心理能量与动机。而外显性态度还没有成为自动化的态度,因而需要较多的能量和动机去检索。第二,在自动化的程度上,不同的双重性态度是有区别的,主要表现为内隐性态度被激活后能否被人们意识到,以及当人们从记忆中检索到外显性态度时,是否有足够的心理能量与动机超越原有的内隐性态度。根据人们对内隐性态度的意识程度和是否具备超越内隐性态度的动机与能量,双重性态度可分为压抑、独立系统、主动压制和自动压制四种。压抑是由于某种性态度会唤醒焦虑,能通过动机驱力将内隐性态度排挤出意识之外;独立系统表现为人们既有一种内隐的、无意识的系统,也有一种外显的、有意识的独立存在的系统。主动压制与自动压制都是指个体能完全意识到他们的内隐性态度,然而这种内隐性态度通常是不符合社会逻辑规范的或个人所不想要的,于是人们要用一种完全不同的态度去压倒和战胜它。两者之间的区别在于超越和压制内隐性态度所需的心理能量,对于主动压制而言,人们是主动、积极地,需要一定的努力与能量去发起对内隐性态度的压制;而对于自动压制,当人们从记忆中检索到外显性态度,他们并不能感知到内隐性态度,因为外显态度自动压制了内隐性态度。

由于性态度的测量涉及较多个人隐私,同时作答者意识到在面对小组内部成员时偏好态度是可以自动激活的,为了保持良好的公众形象,他们便会主动去隐藏和掩饰这种态度。因此,在作答问题时便会超越自己的内隐态度,从记忆中检索出外显性态度,从而出现对外显性态度测量上的反应,表现出内隐性态度与外显性态度的不一致。纸笔式的性态度自我报告量表意味着性态度的测量只是对社会客体的一种有意识的、外显的报告。双重性态度理论提示我们:人们的性态度中还应包括那些自动操作的、不需要意识努力的心理成分。因此,要对个体性态度进行准确的测量,应该同时兼顾内隐测量和外显测量两种方法。

如何准确测量和评价内隐性态度一直是国内外性态度研究领域的难点之一。然而,近几年Greenwald根据神经网络模型设计的内隐联想测验方法的发展与完善为这一难题提供了新的思路。这种方法针对内隐社会认知的研究方法,采用一种计算机化的辨别分类任务,以反应时为指标,通过对概念词和属性词之间的自动化联系的评估而对个体的内隐性态度进行间接测量。当概念词与属性词之间的联系与被试的内隐态度一致时,由于内隐性态度自动激活的特点,被试反应快,反应时短;反之则为不相容,被试的反应就慢,反应时就会加长。由于这种方法运用了反应时这一基本范式,利用被试的快速反应,有效地降低了意识的监控作用。内隐联想测验这一方法,虽然能了解个体的内隐性态度及其与外显性态度的联系,然而这种方法也有一定的局限性,如测验程序复杂,难以获得较大样本的资料。“电脑问卷调查法”则较好地解决了这一问题,虽然在内隐性态度的准确性上不如内隐联想测验,但由于操作简单,且电脑存在“即选即消”的特点,作答者无须为了公众形象而去掩饰自己真实的性态度,与自填问卷法相比,更能反应作答者的真实性态度。因此,此方法被认为是目前国际上调查较大样本敏感问题的较佳方法。1998年,美国《全国男性青少年调查》把这种方法与“自填问卷法”进行对比,发现各种敏感行为(包括性行为、吸毒和暴力行为)的报告率都有所上升。也就是说,在面对电脑回答敏感问题的时候,人们更容易说实话。笔者在1999年对男性性功能障碍患者进行研究时也采用了这一方法,效果很好。潘绥铭在1999年8月到2000年8月也采用这一方法对全国进行性调查,最后有3824人全部回答,有效应答率高达76.3\%,使他们得以最大限度地了解中国的性国情。


\section{第三节 性心理发展的基本理论}

弗洛伊德在他的名著《精神分析引论》中开宗明义地宣布,精神分析理论有两个基本命题,一个是潜意识,一个是性本能。他指出“坚持性欲乃是一切人类成就之泉源,以及性欲观念的扩展,自始便是精神分析学的重要动机”。由此可见,坚持把性欲作为人类活动的普遍动机因素,是弗洛伊德的基本分析原则,而性欲观念的扩展则是他性理论的主要内容。这两个方面的思想是不可分割的,因为坚持以性欲作为普遍动机因素的分析原则,正是通过性欲观念的扩展得以充分说明和强化的。这两种思想的统一,从总体上构成了弗洛伊德性理论的基本观点。所谓“性欲观念的扩展”,即弗洛伊德对性的概念、范围和功能的解释与一般传统的观念不同。弗洛伊德的性观念可以概括为以下三点:一、从概念来看,弗洛伊德认为在“性”与“生殖器”的概念之间,必须作出明确的区分。前者是更为广泛的概念,它也包括许多涉及生殖器方面的活动。二、从范围来看,性并不以个体成熟为标志,它不仅仅开始于青春发育期,而是在婴儿出生后不久便开始了。三、从功能来看,性功能不仅表现为生殖欲的快感,还包括从身体的其他区域获得的快感,而这两种功能经常是不一致的。总之,弗洛伊德认为,如果仅仅把“性”看做是两性的差异,或者是从异性身体上获得快感的满足,特别是只理解为生殖器的结合和性动作的完成,那就过于狭窄了。他说:“据一般的见解,‘性’的含义兼扮两性的差别,快感的刺激和满足,生活的机能,不正当而必须隐匿的观念等,这个见解在一般生活上虽然适用,但在科学上就不够了。”正是因为不满足于这“一般见解”,弗洛伊德沿着一种新的思维方式建构着他的精神分析的性理论。在具体内容上,弗洛伊德首先提出“幼儿性欲”说,包括“奥狄浦斯情结”,即‘恋母情结”、“性欲发展”等理论。他认为,人在幼儿早期就有性活动的躯体标志,并一直有规律地发展和变化着。在发展的每一阶段,都有一个特殊的身体部位成为性兴奋中心,被称作“性感带”或“动欲区”。

有人认为,弗洛伊德对儿童性心理发展的年龄阶段划分的依据或标准,对儿童性心理发展各阶段特点的分析,都是不科学的,尤其是对恋母情结的阐述,是他的泛性论观点在儿童心理学上的表现。因为他仅仅将出现在个别儿童身上的一些现象,错误地上升为代表儿童心理发展本质的东西。人类学家对弗洛伊德理论中的恋母情结也提出了异议。马利洛斯基(Malinowski)等人指出,恋母情结远不是弗洛伊德想象得那么普遍,而且并不是所有文化中都出现以恋母情结为依据的家庭。在屈比安得(Trobriand)岛,教育孩子纪律的人不是父亲,而是舅舅。而且,最严格的乱伦禁忌,不在孩子和父母之间,而在兄弟姐妹之间。马利洛斯基指出,在这种情况下,压抑的忧虑与渴望和一般的文化是极不相同的。“我们可能说在恋母情结的时候,有一种压抑的想杀死父亲,和母亲结婚的愿望,而在屈比安得岛上,这种愿望是杀死舅舅,和姐妹结婚”。

也有些心理学家认为,弗洛伊德精神分析理论的科学价值不大,因为它是那样的复杂而费解。如舒尔茨(Schultz)所指出的那样,“似乎在浩瀚的词海中,没有精神分析理论这种含义的东西!有大量的一般化和假设,但是似乎没有原理、必要的条件、精确的关系、有次序地构成必需的科学理论”。有时,弗洛伊德的理论似乎预言了某些问题,但是,实际结果可能是自相矛盾的。例如在口唇阶段体验到挫折的孩子,可能发展为守纪律,爱整洁、顺从的习惯,或者发展为和这些习惯相反的特性,如反抗和肮脏等。据此,人们如何能够预言孩子的特性,它又将以哪种趋势发展呢?

霭理士在他所著的《性心理学》一书中把性心理学定义为:“本书所论的性心理学,指的是性冲动或性能的心理学。”即完成从“积欲”到“解欲”的圆满过程。在对这一过程的解析中,“积欲与解欲不是两个分明的过程,而是一个过程的两个段落。这是造化的不二法门,一边教生物个体多多地把力量积蓄起来,一边又紧接着教它快快地把这力量解放出去……此种力量的由张而弛,对于个体的身心健康,亦有其维护与培养的功用”。这即是人的生理变化与心理活动的相互交织作用停止在情绪体验之上的性心理过程。

霭理士的儿童性心理观与弗洛伊德有很大的不同之处,具体表现在以下几个方面:

霭理士认为,“阳具、手指、脚趾是儿童身体上最‘奇特’的部分,是最可以供他玩弄的部分,玩弄的结果可能引起愉快的感觉”,这种感觉只“逗留在性领域的边缘之上,其在成人,便是引进到真正的性感觉的一种准备感觉”。“儿童对性的兴趣或类似性的兴趣自有它们的特点和领域,这领域是在成人的性领域以外的。一则因为在体格方面,生殖器官还没有发育成熟,再则,在心理方面,对于所谓异性还没有清楚的认识,一直到青春启动期过去以后,这种发展与认识才将到来”。

霭理士认为,阉割情结对于神经脆弱的人自有其强烈的影响,对其心理也的确有一番不良的影响,“不过若说这种感想很普遍,很有力,凡属寻常的儿童都有,那我怕是言过其实”。

“自动恋”是儿童独处的时候所自然涌现的性活动。“自动恋”既不是异性恋,也不是同性恋。霭理士认为,一切不由旁人刺激而自发的性情绪都可以叫做自动恋。同时他还认为,自动恋在有的领域还是比较广泛的,大致包括:①性爱的白日梦;②性爱的睡梦;③影恋,包括由顾影自怜或自我冥想引起的性爱情绪;④自慰(手淫)。

霭理士认为,儿童心理活动的方式和成人很不一样,儿童不一定能了解成人的心理活动,成人也不一定总能了解儿童的心理活动。但是我们也不要认为童年时代就没有性的变态。不过和成人比较,这些变态更是一种数量与程度的问题,而不是品质与种类的问题。在分析“童年性生活的变态”与成人之后的关系时,霭氏认为,童年性生活的变态往往可以分两类,而在不良的境遇之下,这两类变态又有维持到壮年的趋势:一是不足与缺陷的倾向,二是过渡与流放的倾向。在霭氏看来,在童年时代不足的倾向(性感不足与性兴奋性不足)比过度的倾向(性感过度与性兴奋性过度)的危险性小,而且不足的倾向或发育迟缓者“在壮年时期的性生活,说不定更有力量,更为幸福”。

性别图式(gendermap)和情爱图式(lovemap)是约翰·莫尼提出的新概念。两种图式都像天生的语言图式一样具有多元变量,且有先后顺序,它们都是同步存在于大脑和思维意识中的模式。性别图式描述一个人的“性别—认同/角色”(GI/R)的细节;而情爱图式则描述理想的情侣、理想的爱情、在想象中或与情侣实际发生的理想的性和情爱活动。性别图式包括情爱图式,但范围更广泛,情爱图式则以性别编码。性别图式不仅包容了性别编码的一切如一个人的职业、受教育程度、消遣娱乐、服装打扮以及社会合法地位等,还包容了表现在礼仪、修饰、身体语言以反语音语调等方面的内容。性别图式的编码形成男性气质、女性气质或双性化气质,并形成异性恋、同性恋或双性恋性取向。情爱图式的编码则形成正常性爱或性欲低下、亢进、倒错等。当婴儿渡过生命的最初阶段,大脑的编码便从双向化转向男性或女性气质的单向化。这时不再借助于激素或大脑中的其他神经递质的直接调节,取而代之的是触、嗅、味、听和视觉等感觉,从社会环境中输入信息而调节。儿童使性别图式与情爱图式达成一致的准则及定式是按照两大原则进行的,即认同和互补。认同是指模仿、重复与自己指定性别相同的人的行为,使自己的性角色与他们的行为相一致。互补则指通过与自己指定性别相反的人的交互作用而肯定自己的性角色。在童年性别图式编码发展过程中,双向化发展的潜在可能性将转化为单向性。肯定的一极编码为“我”,我是男孩或女孩;否定的一极则编码为“非我”,我不是女孩或男孩。在每一极都有标定为某性别气质的编码图式,两极的编码图式恰恰相反。一个人的性别图式既可能置换为另一个,即成为性别转换症;也可能同时存在,即出现双重人格的现象,既有男性的阳刚,也有女性的温柔。在正常情况下,一个人的性别图式编码在出生前和出生后是一致的,而且与其外在生殖器性别也一致。当出生前或出生后的性别图式编码发生障碍时,性别图式将与出生性别不一致。

情爱图式的发展有可能发生错误,如性欲倒错或性偏好障碍,男女均可发生,但以男性属多。它是对异常的、个人或社会都不能接受的刺激方式作出的强烈反应,或不可避免地依赖这种刺激。这种刺激是从观念或幻想的意象中感知到的,它们成为激发与保持情欲并促成性高潮的最佳刺激。有时可以从早年的发育历史中寻找情爱图式倒错的早期线索。通常可以看到8岁左右遭到情爱图式发展的早期创伤,如虐待、侮辱、暴力致伤,无论这种打击是否涉及性问题都可能成为其诱因。还可能是受到了超越其性发育年龄阶段的性诱惑或性接触。有时越是被禁止的、不准许的、要给予惩罚的、而且是极度危险的事,越会变成其嗜好而不能自拔,不论自己如何试图控制都无济于事。这就是情爱图式转化为倒错爱的常见形式之一。当然性欲倒错受不同文化的影响,某些民族可以接受的性偏好到了另一些民族可能就成了禁忌,认为是变态,反之亦然。

美国心理学家赫罗克认为,青少年性心理的发展应经历四个阶段,每个阶段都有其明显的特征,并对此做了详细的解释。

这一阶段的少男少女总想远远地避开异性,表现为对异性的否定倾向特征。由于生理发育成熟的变化而导致他们心理上的变化,产生对性的害羞、不安甚至反感,男女学生建立起各自的“独立王国”。

这一阶段的少男少女以崇拜年长者为主要特点。尤其是稍为年长些的同性,无论在容貌、学习能力、体育运动、人格或其他方面,都对他们具有强烈的吸引力。当他们向往的对象为同性时,把这种崇拜叫做迷恋;如果是异性则叫做英雄崇拜。

这一时期的少男少女对年长者的崇拜已经结束,开始将“注意”的对象转移到年龄相当的异性。在集体活动中,少男少女都努力设法引起异性对自己的注意,但是交往的对象多不稳定。

这一时期的青年男女开始把爱情对象集中到一个人身上,他们开始与自己选择的异性朋友单独交往,不愿意与集体在一起活动,而且开始谈论对未来美好的设想,甚至到了“山盟海誓”的地步。

赫罗克的性心理发展阶段理论与青少年发展的实际较为一致,与我国学者一般采用的阶段划分观点也极为相似。

在性意识萌芽的初期,人们常常以否定的形式表现出来。小学四五年级,儿童开始关注异性,但表现为男女之间的对抗、排斥,与异性关系密切者会受到同伴的嘲讽。这时并没有萌生真正的性意识。进入青春期,少男少女的性意识开始觉醒,他们对两性差别特别敏感,开始产生性不安与羞涩心理。近年来,受过良好性教育的中学生虽然并不以冷漠和厌恶的态度对待异性,但与异性交往中仍然以排斥和疏远的方式居多。

随着进入青年期,少男少女们情窦初开,与异性疏远逐渐缩小。表现为性心理的本能性和朦胧性,但这种本能性和朦胧性缺乏深刻的社会内容,基本上还是生理急剧变化带来的本能作用。他们对异性的认识还披着一层朦胧的面纱,对异性的兴趣、好感和爱慕主要是异性间的吸引,然后便对异性表现出好奇心,并以善意、友好、欣赏的态度对待异性,也愿意与异性同学一起学习,一起参加社会劳动,并发展友谊。这个时期是青少年性意识发展的重要阶段,其主要是由于青春期发育高潮的到来而引起的。

恋爱期处于青春晚期,是性亲近的自然延续。这一时期,男女在交际中逐步将异性作为结婚对象而加以选择,男女交往的标准也逐渐从外在转向内在,高尚的思想品德、广博的文化知识、健康的兴趣爱好、丰富细腻的情感等对异性都有着巨大的吸引力。

(刘明矾)


\chapter{第四章 生命周期各阶段的性}

性是伴随人一生的重要功能,它能给人以欢乐,也能给人以痛苦,它可以带人进入崇高的情操境界,也能使之身败名裂。因此,人的性心理发展决定着他一生的幸福。要使人们获得最大的性欢乐,要使性最大限度地造福于人类,并预防和避免各种性问题、性障碍的发生,都有赖于性心理的健康发展。

约翰·莫尼提出性发育过程中要经历一系列关键期,或者说关口或岔路口,也就是说在性分化与性成熟过程中具有一定的可变性。他的性发育关键期理论认为先天的生物学与后天抚养过程中的因素是相辅相成、互相影响并共同作用的。这种作用将在某些特定时期起作用,故称之为关键期,即所谓“本性/关键期/教养”(nature/critical period/nurture)模式。它强调在性心理成熟过程的关键时期中存在的生物学因素(本性或先天)和文化环境因素(教养或后天)的共同作用与相互作用,以致对人的行为或性取向产生持久的记忆痕迹或印记作用。

关键期始自受精之日,一直持续到青春期之后。研究表明,出生前或刚刚出生的经历有可能助长某些行为趋势和性取向,而在成长和社会化过程中使之细化与固定,把出生前的因素都说成是生物学因素或把出生后的因素都说成是社会环境影响的结果都是不可取的。当一个人步入性分化的每道关口或岔路口时,他或她都具有双向选择。但是当一个人完全通过这个关口后,该关口就将关闭,于是其性分化或性成熟将固定于先前发展的方向。例如,一个人的性腺在胚胎发育期既可以向睾丸方向,也可以向卵巢方向发展,但一旦发展为睾丸就不可能再退回去发展为卵巢。在下一个关口关闭之前,其在这一阶段的发展便有可塑性。这些关口或关键期包括出生前和出生后的各6个。出生前有:①染色体性别;②性腺性别;③激素性别;④内生殖器形态性别;⑤外生殖器形态性别;⑥神经系统特别是大脑内编码。出生后有:⑦出生时的指定性别;⑧性脚本的形成;⑨性角色学习;⑩性别同一性(即心理性别或性别认同); 性取向;青春期。

由于向出生后的学习和社会化过程中输入大脑的信息是来自各种感觉的,是经由大脑的学习和记忆功能予以加工的,据有些学者的观点,人类的大多数行为特别是性行为,就像在剧院上演的戏剧一样有一个脚本。脚本实际上是一个人在成长过程中把他或她认为适合的行为、态度、价值观和期望接受下来并予以内化的观念,它将成为其之后奉行的行为准则,所以脚本随性别的不同而有所不同,也随个体所处的文化环境而不同。一个性脚本具有5个关键变量:即一个人应该与谁发生性关系,一个人应该从事哪些性活动,性在什么时候是适宜的,什么是合适的性环境,一个人为什么要有性活动。

性别和性征的正常发育顺序为:①染色体:女性是46,XX,男性是46,XY;②性腺:Y染色体的性别决定基因影响原始性腺嵴向睾丸方向分化(6周半左右即可辨认),否则向卵巢方向分化;③胎儿内分泌:睾丸分泌抗副中肾管因子即苗勒氏管抑制因子,使苗勒氏管(副中肾管)退化,而睾丸分泌的睾酮则刺激男性内外性器官的发育(7周左右)。若胚胎中缺乏上述两种物质,那么无论遗传因子或性腺表现出的是男性或女性,最后均发育为形态上的女性;④内部形态:男女胎儿在7周左右分别开始形成完善的男女内生殖器结构;⑤外部形态:9周前胎儿外生殖器尚未分化出性别差异,外观可见尿生殖膜两侧 的阴唇阴囊隆起和尿生殖褶,以及位于其头端的生殖结节(初阴)。从第10周外阴开始向男性化发展,如初阴增长明显成阴茎,左右尿生殖褶在中线互相合并。12周后可认出形成的阴囊。女性外阴可见形成较小的阴蒂,大小阴唇开始形成。两性外生殖器的主要解剖部分均对应,如阴蒂与阴茎相对应,小阴唇和阴蒂包皮与阴茎皮肤相对应,大阴唇与阴囊相对应;⑥神经:胎儿内分泌可影响下丘脑神经核团的组成,睾酮使神经核团向男性化发展;⑦性别鉴定:出生时按其外生殖器判断和指定新生儿的性别;⑧抚养:从1.5岁~2岁是幼儿正常心理性别发育的最初阶段。大脑的男性化包括增强男性特征,抑制女性行为和神经内分泌特征(去女性化)的双重过程,而大脑的女性化则只需增强女性行为和神经内分泌特征。换句话说,大脑的女性化是自然过程,而大脑的男性化则需要外来因素的干扰,就是男性胎儿睾丸分泌出的睾酮。这种激素对大脑男性化的影响只会发生在某一特定阶段,即“关键期”。虽然女性胎儿体内也有睾酮,但其升高水平已过了“关键期”,故大脑不会向男性化方向发展。女胎在出生前的胎儿时期若受睾酮的影响,出生后可能出现男孩样性格;相反,若男性胎儿母亲患糖尿病使雌激素水平增高,受过高雌激素的影响,男孩出生后可能缺少冒险性和决断力,不喜欢户外活动也不易产生对异性的爱。

性别分化中的内分泌影响还可以从其他一些研究得到验证:例如多米尼加有些农村存在家族性的5α还原酶缺乏综合征,因此在对新生儿进行性别鉴定时,往往以他们发育缺陷的外生殖器指定为女性,但是到了青春期后,随着体内睾酮水平的增高,生殖器又向男性化发育,他们同时还经历了性心理的改变过程。因7岁以前都按女性抚养,故曾明认自己是女性,从7~12岁则开始出现性别问题上的烦恼,逐渐感到自己不像女性,而后认识到像男性,最终明确自己本来就应该是男性。由于这些男孩体内缺乏5α还原酶,不能将睾酮转化为双氢睾酮,而胎儿时期性器官发育实际上是依赖双氢睾酮的,于是新生儿出生时便出现外生殖器异常而被错认为女孩。等到青春期生殖器等可以对睾酮作出反应时,他们又迅速恢复男性的本来面目。由此可见,先天因素的影响远远超过后天抚养经历的影响(环境因素),但这些争论至今并无明确结论。


\section{第二节 儿童的性}

如果有人说幼儿也有性感觉,也许你会嗤之以鼻,在精神病学家弗洛伊德对儿童期的性问题细致研究之前,大多数人也是如此。弗洛伊德论述道:“公众的观点是儿童期没有性感觉,这种本能最初出现的时间似乎是在青春期。这虽是一种普遍的错误,但其后果却是严重的。这主要是由于我们对性活动的基本原理的无知所造成的。”现在都已接受极年幼的儿童已有了性感觉这一观点。细心的母亲可以发现,新生男婴常有自发性勃起,新生女婴可有阴道分泌物。性器官的这些早期生理功能的机制还没有弄清楚,但它们代表了一种先天性的反射活动,犹如呼吸、排尿、出汗、消化一样。下面我们简单回顾一下弗洛伊德的性心理发育学说:

性欲发展可以按照“奥狄浦斯情结”出现的先后为标志,分为三个时期。

第一个时期称为“前奥狄浦斯期”,分为以下三个发展阶段:第一阶段,从婴儿出生到1岁左右的“口欲期”;第二阶段,从1岁左右到两岁多,叫做“肛欲期”;第三阶段,从3岁到4岁,叫做“性蕾期”或“阳具欲期”。总之,这一时期各阶段性发展的特点,只是表现为“自体性欲满足”,属于原始性的自恋时期。

第二个时期,即“奥狄浦斯情结期”,从三四岁到五六岁。这一时期的特点是儿童的性需要已开始从自体转向外界,即表现为原始性的“他恋”。此时,男孩子表现为一种潜抑的“恋母仇父”情结,女孩子则表现为一种潜抑的“恋父嫌母”情结。弗洛伊德认为,这一时期如何应付奥狄浦斯情结,将影响着以后人格发展的各方面特点。

第三个时期,即“后奥狄浦斯期”,这一时期又可分为三个阶段。第一阶段叫做“潜伏期”,为6岁到11岁。由于此时社会和文化观念开始逐步渗透到儿童的意识之中,已萌发着羞耻感,因此幼儿性欲被暂时“冻结”。第二阶段为“青春发育期”,从11岁到14岁,又称“生殖欲期”。在经过潜伏期的低潮之后,这时性欲又开始朝着生殖这个生物学目标正常发展。第三阶段是“青年期”,从14岁到18岁或稍后。这是随着青春发育期相继到来的婚恋、建立家庭、养育子女等阶段,从而走向社会化的成熟期。

是婴儿对自身性别认识的关键时期,有以下几个方面的活动直接与性心理的发展有关。

1.婴儿学习感受自己以及别人的身体,这主要由母亲抚育婴儿的态度而定,也就是由母亲对婴儿抚摸、搂抱等身体接触及体温传递的质和量而决定的。可以产生愉快的、可信赖的体验,也可以形成不愉快的、危险的体验。在婴儿早期母子之间身体互相接触的这种行为会使婴儿产生信任感,进而影响到日后对别人的信任感,从而影响到与人能否和睦相处。

弗洛伊德称此期为“口阶段或口欲期”。因为此期间婴儿的需要和快感乐趣来自吮吸乳汁的动作,敏感带主要集中在口唇。从吸吮母亲的奶头及被抚抱,婴儿不仅解决了饥饿之欲,也从肉体的接触得到情感上的满足。在没有母亲的奶头时,有时给婴儿吮吸空奶嘴或婴儿自己吸吮手指、脚趾,他们同样可以获得高潮般快感,即使在吃饱之后也会这样做。这种对自体性敏感区进行刺激称为自我性乐(autoerotic)。因此,此期也可称为自我性乐阶段。由于这些欲望的满足,婴儿建立了对世界的安全感,以及对人的基本信赖感。假如此阶段的发展受到挫折,长大之后,其在人格上将发生偏离,变成不信任别人,缺乏安全感,喋喋不休及贪嘴好吃的性格。

小孩在自己的手能灵活活动后会发现他自己的性器官,这时手握生殖器玩的感觉就像他们握住自己的脚趾和手指一样自然,毫无特殊感受。据统计,1岁的婴儿36\%有手淫行为,男孩为55\%,女孩为16\%。Spitz(1949)研究母亲抚育态度与婴儿出现性行为的时间关系,发现母亲照顾得好的婴儿在一岁时出现手淫行为,而母亲照顾得差的婴儿此时则不出现手淫行为,即性心理发育出现迟滞的表现。

2.婴儿学习感受不同性别的态度。新生儿自呱呱落地起,在行为上就显示出性别的差异:男孩的肌肉较为发达且多动;女孩常“啊啊”发声,微笑(自发地,甚至在睡眠时),其触觉较男婴更为敏感。人们对待男婴及女婴态度以及照顾方式往往不尽相同。比如把男婴抱举得高高地逗弄,对女婴逗弄的方式则十分轻柔。对婴儿不同行为所表现出的态度及耐心程度,将塑造出不同行为特点的婴儿。

自出生起,男女婴儿在称呼上也被授予不同的性别标志,如称男孩为“大儿子”,往往起“刚”、“强”之类的名字;女孩则被称为“小丫头”,起的名字也往往是“花”、“芳”之类。随后,他们又学到“爸”、“妈”、“叔”、“姨”等具有性别标志的称呼,逐渐学会区分不同的性别。

3.婴儿通过自己的身体领会到男女的不同。女性的性器官大多隐藏在体内,从外面看不到,而且感觉弥漫;男性的性器官大多显露在身体外面,看得见,摸得着,感觉也较局限。

从出生的第一天起,父母就用不同的态度来看待男婴和女婴。对男婴,盼望能长得虎头虎脑,壮实一些,个子大一些;对女婴,则盼望长得眉清目秀,容貌好看一些。父母亲的态度,无意中在孩子身上建立起了对性角色规范的“条件反射”,加强了婴儿对自己身体的领会。

婴儿喜欢身体上的接触,是建立信任及和睦相处的最佳时期,也是获得自身性别身份的最佳时期。这种自身性别身份的认识过程,即认识自己是男孩还是女孩的过程,虽然会一直延续到幼儿的早期,但通常在2岁左右,幼儿已能认识到自己是个男孩或者是个女孩,而且一旦认定了自己的性别身份,就会深深地印入脑海,难以逆转。

这是性心理发展的关键时期。此期间儿童能够自主活动,并学着按照社会的要求来控制自己的行为,这些就成为幼儿性心理发展的主要内容。

1.大小便训练是这一阶段带有性色彩的冲突之一。此时儿童必须学会在适当的时间和地点排尿和排便,而排便时的感受与性感受有相似之处。如果父母处理这个问题不得当,就可能使儿童将排便的感受与性的感受不适当地联系在一起,认为性的感受与排便一样是“脏”的事情。这在情绪上对女孩所造成的不良影响比男孩要大得多。因此,在训练幼儿形成良好的大小便习惯时,不要让儿童形成这种错误的印象。

2.在此期间,大多数儿童会发现两性在解剖结构上的差异,激起他们极大的好奇。又由于周围环境不断地出现如“怀孕”、“生孩子”之类的事情,就更驱动了他们这种好奇心理。

幼儿的早期对自己到底属于哪一个性别已有认识。但是,这个性别到底意味着什么,他们并不清楚。与婴儿相比,幼儿的身体活动更加自如了,活动量大大增加。由于活动范围扩大,与别人(主要在家庭里)的交往增多,他们获得信息的能力也增加了,在好奇心的驱动下,他们对自己及周围的探索明显增加。他们探索两性外生殖器的差异,他们有目的地手淫,甚至还与小伙伴们玩性游戏。

3.幼儿开始与同性父母认同,开始“学爸爸那样”或者“学妈妈那样”,模仿同性父母的言谈举止,与小伙伴们一起玩“过家家”。他们有时会出现跨性别认同,若为暂时性的,似属正常的范畴;若持续地表现为跨性别认同,则应考虑为异常情况,并加以矫正。

某些单亲家庭或夫妻两地分居,儿童与母亲生活在一起,或父亲每天工作早出晚归,平时与子女待在一起的时间很短,孩子大部分时间与母亲在一起,这些都可能会造成男童与母亲的认同,影响男童的性心理发展。

由于婴儿在出生后的最初几个月,不论男女,均与母样接触最多,从母亲那里得到赖以生存的乳汁、躯体上的温暖和抚爱,母亲就是他(她)们的一切。因此,这个时期的男孩女孩都与母亲认同。嗣后,作为男孩,必须逐渐从与母亲的认同中解脱出来,以便形成男性的气质;女孩就不必费此过程,而继续保持与母亲的认同。这也就是为什么男人比女人更容易发生性身份障碍的原因之一。

4.幼儿期,父母不断地把自己关于男性或女性品质特点的观念灌输给子女。比如教育男孩应该勇敢、刚强;教育女孩应当温柔、文静。还通过给孩子梳妆打扮(女孩子穿花衣,扎蝴蝶结、涂胭脂抹口红)、赠送礼物(男孩送长枪、大刀,女孩送布娃娃等)、对孩子进行指数(如男孩摔破了皮流血、哭泣,大人会说:“男孩子要勇敢,疼了也不哭。”女孩子大声讲话时,大人会说:“女孩子家家的,不许大嗓门讲话。”)、鼓励他们参加某些活动(男孩玩球,女孩做手工)等,对儿童产生了潜移默化的作用。如果儿童的行为符合他(她)的性别,父母就给予鼓励和奖赏;如果儿童的行为发生偏离,不符合他(她)的性别,父母就不予赞同或予以惩罚。儿童在父母及亲属这种态度的影响下,按性角色的规范行事,不断地向某种性角色发展。

5.幼儿期是语言能力发展的最佳时期。在这些年头里,儿童努力学习如何有效地用语言进行交流。

与语言发展的同时,幼儿对性的好奇不光只表现在身体和视觉上,也表现在语言和认识上。他们会提出各种各样的问题询问大人,如“为什么要结婚?”、“我是从哪儿生出来的?”遗憾的是,对孩子提出的有关两性的问题,大人们往往不是不回答,就是胡说一通,甚至还有的人斥责孩子不该问这些问题,说什么“大人的事,孩子不要问。”给儿童幼小的心灵注进了错误的性观念。

由上可见,幼儿期是性心理发展的关键时期。如果能成功地进行引导,儿童将愉快地接受自己生理上的性别,乐意接受社会对自己性别行为规范的要求,也就是在认定自己性别身份的基础上,遵循性角色的规范行事。同时,他们往后还敢于坦率地与别人,尤其是自己的父母谈论性问题,以便获取更多的性知识。

弗洛伊德称此期为“肛门阶段或肛欲期”。他认为婴儿的口腔快感转移到幼儿排便引起的肛门快感上,由排便时肛门黏膜的兴奋而得到满足。儿童在储留和排出大便时拥有快感,而且他们体会到在长时间憋便后、排便量越大越舒适,会有一种特别的快感。此期儿童爬上爬下、东摸西摸,同时也一天一天、一项一项地学习怎样做人。他们第一件需要学习的是控制自己的大小便,要在一定的时间,一定的处所排尿、排便。然而,父母往往在这一时期开始训练婴幼儿排便习惯,也就是说,儿童必须开始学习如何控制自己的欲望,去接受外在的约束力,这与他们总愿意多积攒大便以图更大快感的需求发生冲突,使婴幼儿出现第一次逆反心理期。于是在情感上有了苦(禁制)和乐(释放)的体验,这种苦乐的情感体验与排泄关系密切,使此期的儿童对排泄物发生极大的兴趣,产生玩屎玩尿的现象。弗洛伊德认为,这种兴趣是将来产生艺术创造的原动力。顺利通过肛欲期的儿童,会逐渐养成自治自立的能力,果断行事,并能与别人和睦相处、合作共事。如果成年后人格固结在肛欲期,对粪便的兴趣转化为对金钱的爱好,则表现为洁癖、吝啬、任性和固执。

小孩在此期发现触摸性器官会带来快感,于是开始手淫。男孩手淫时可发生一系列生理变化,随着阴茎明显的震颤和挺伸,身体发生节律性的运动,感觉能力明显改变,最后肌肉紧张,一阵痉挛,有节律的收缩停止,接着所有症状消失。女婴手淫同样可出现性高潮的生理表现,包括全身突然松弛和出汗。然而,婴儿通过手淫达到性高潮并非常见,但是随着儿童的成长,就有可能把刺激生殖器当成一种快感,并且重复而有目的地寻求性满足。

此时期,神经生理学上发生重要的变化。其一,在3岁时,通向阴茎和阴蒂的感觉神经髓鞘已完全形成,使得这些部位有了更为独立、强烈的性感体验。其二,此期不论男女,雄激素的产生均有轻度增加。雄激素会使男女两性的性欲增高,并使阴蒂和阴茎对触觉刺激敏感化。在以上两种因素的作用下,此时期儿童的性要求也会相应增高。其三,某些研究表明,大约在3岁时,神经生理学上的成熟使脑的嗅区对人类性的信息素(外激素的一种)敏感化,这就跟某些动物的中枢神经系统成熟过程有相似之处。比如一只先前对信息素感觉迟钝的小公狗,当中枢神经系统成熟之后,会对母狗发情时的气味作出反应。有证据表明,人类同样也会对自己亲生父母身上所散发出来的气味有独特的感觉,并作出不同的情绪反应。

以上变化导致儿童性欲的不断增强。与幼儿期相比,学前期儿童的手淫活动大大增多了。除了量的增多,此期最重要的变化是手淫往往伴有性幻想,而且目标明确,直接指向异性。

对此年龄阶段的儿童来说,这种愿望的满足,不可避免地要指向异性中他(她)所最爱的那个人———通常是父亲或母亲。也许不少父母能回想起孩子曾天真地对他们说“等我长大了,爸爸一死,我就跟妈妈结婚”或“我跟妈妈一样也是女的,我也想跟爸爸结婚,生一个小弟弟”这一类的话。这无疑反映出儿童恋父或恋母的情绪,这种现象也可出现于睡梦中及手淫时的幻想之中。

在此年龄阶段,多数儿童似乎知道了性交对生殖所起的作用。但是在他们的想象之中,性交是一种痛苦的、令人害怕的事情,而不是一件愉快的事。如果亲眼见到成年人的性交活动,他们更会认为那是一种野蛮粗暴的行为,毫无乐趣可言。加上成年人此时的反应又是尴尬窘迫,含糊其辞,使得儿童进一步认为那是一件丢脸的事情。

在一般的家庭里,随着年龄的增长,儿童认识到自己与父母悬殊太大,没有可能取代他们的父亲或母亲,于是放弃了那些不现实的愿望。他们开始以同性家长为认同对象,竭力模仿同性家长,以达到“像爸爸一样”或“像妈妈一样”。至此,所谓“恋父”或“恋母”的过程就结束了。

可见,这个时期使儿童第一次体验到异性恋的滋味,对今后形成异性恋倾向十分关键。因此,父母适宜的反应,对成功解决儿童“恋父”或“恋母”过程及其性心理发展至关重要。

什么才是适宜的态度呢?应该坦率地欢迎和保护儿童此时所萌发出来的异性恋嫩芽,进而应该让儿童知道父母之间的亲密行为是正常的,父母的相亲相爱丝毫无碍于对子女的疼爱。这样一来,生活在这种家庭气氛中的儿童就不会再有妒忌,并且将这一份异性恋的爱心保留下来,长大之后,将性的要求指向异性伴侣。

弗洛伊德称此期为“阴茎阶段或性蕾期”(阴茎欲期,阴茎崇拜期),意味着儿童的性兴趣开始萌芽。儿童开始注意到男女两性生殖器之间的差别,或对两性生殖器官的不同认识更清楚了,并有意识地窥视异性的性器官。男孩因具有阴茎而感到骄傲,对母亲产生性爱情绪,想占有母亲,把父亲看成是与他争夺性爱对象的对手而产生记恨,产生了恋母现象。但此时男孩还对父亲产生钦佩之情,想模仿父亲,即所谓矛盾(ambivalence)情感。由于恋母,男孩害怕惹怒比他们强大有力的父亲,惧怕被父亲发现而将他阴茎割掉。男孩的阴茎外在而柔软细小,加上所见到的女性生殖器的形状,更“证明”了被割掉的可能性,遂产生了阉割焦虑。在这种威胁之下,男孩只得放弃对母亲的性爱,加强对父亲的模仿,将父亲的形象内化(internalization)成他自己人格的一部分,那种恐惧性幻想也就逐渐消失了。

男女孩之间开始进行一些性游戏。大多数孩子在这一时期会对“生命事实的真相”产生好奇心,他们可能会问妈妈:小孩是怎样生下来的?

恋父恋母期也发生在3~5岁,孩子出现对异性父母的过分依恋,而对同性父母则怀有抗拒心情。性的游戏或温柔的表示可能持续到7岁。

女孩在口欲期及肛欲期的性爱对象也是母亲。在阴茎欲期,男孩产生阉割焦虑的同时,女孩发现男孩有阴茎而自己没有,则羡慕或嫉妒男孩的阴茎,认为是自己过去不听话或犯了错误,所以被阉割掉了,出现自罪心理,并且怨恨母亲。以后,她们发现母亲也没有阴茎,而父亲长有此物,转而与父亲亲近,产生了恋父现象。再后,当她们意识到这种乱伦的关系行不通时,就专心模仿母亲,将性力转向无乱伦关系的其他男性。

弗洛伊德认为,如果儿童时对父母的这种乱伦性爱关系不能得到正确的解决,这种带有强烈感情的观念被压抑到无意识之中,日后则会由此而产生精神上的障碍。他还认为,如果男孩恋父恨母,女孩恋母恨父,日后则会发展成为性的变态。

弗洛伊德认为在此年龄阶段,性的能力大大减低了,儿童的性心理发展进入一个安静的时期,性力升华为学习、体育和游戏等活动能力,称之为潜伏阶段或潜伏期。这一时期男孩与女孩界限分明,互相排斥。这是性发展的停顿和颠倒。但是许多跨文化的调查研究材料并不支持这种观点,后来的学者认为,这一年龄的儿童尽管公开的性好奇会渐渐减弱,但儿童的性兴趣和性活动实际上从未停止,仍然不断地稳步发展。性游戏或稍许显露生殖器的活动可能持续到7岁左右,而对性的兴趣会继续趋向一种不同的、更秘密的形式。从活动的人数上看,有的呈单人形式、有的呈双人形式,有的则集体活动;从活动的性别上看,有的为同性,有的为异性。从8~9岁起,儿童开始对说脏话、看与性有关的图片感兴趣,性问题成为小朋友们之间经常讨论的话题。他们也开始注意发生在父母卧室里的事,窥探父母的亲昵举动。他们专心地模仿成年人的言谈举止,观点看法,在体力与智力上成长与发展。更为重要的是,儿童从家庭这个较为单纯的环境进入了学校,学习如何在更大的范围里及更复杂的情况下完成他们的性角色。

这时男孩所遇到的性别认同要比女孩困难得多。因为正当男孩努力摆脱与母亲的认同,转向与父亲的认同之时,又进入满是女教师的小学环境。老师们总是喜欢规规矩矩、文静听话的“好孩子”,而这种“好孩子”又以女孩子居多,于是老师们往往以这样的女孩作为男孩,尤其是“顽皮”多动的男孩学习的榜样。老师的这种要求与男孩内心的愿望是相违背的。因为此时的男孩已不愿再与女性认同了,并极力想摆脱与女性的认同。这种逆反心理,使得他们之中的有些人因为拒绝成为老师所要求的“女孩子气十足的听话的学生”而不愿去学校,甚至逃学,影响了他们的智力发展和社会能力。

虽然在以后的年龄阶段里,生活经验仍然会继续作用于个体的性角色倾向,但此期儿童作为男性或女性的切身体验,对他们日后对自己的性别满意与否起着重要的作用。因此,认识到家庭及社会对不同性别所持的态度,对此期儿童的性心理发展影响极大。比如生活在“重男轻女”“男尊女卑”的家庭或社会中,会形成男孩的优越感和女孩的自卑感,造成女孩不喜欢自己性别的心理。

如果不严厉予以制止,性游戏会从幼儿期一直延续到学龄期。在我们的社会里,儿童个人或相互手淫,好奇地观看或触摸别人的身体,以及模仿成人的性活动,都是不被允许的,并会遭到斥责和惩罚。可是在某些社会里(并非所有未开化的社会),人们允许儿童有这种行为。其结果很有趣,在这两种不同文化背景下的成年人表现出明显的差异。在开放的社会里,几乎不存在同性恋和性偏好障碍。而在严格限制的社会里,男女之间分界清楚,男孩只能跟男孩玩,女孩只能跟女孩玩,甚至男女分校,结果成年人之中性偏好障碍、性冲突、对别人的性侵犯,以及性功能障碍频频发生。由此可以看出,限制儿童期的性游戏和好奇心只是基于某些社会的道德价值,而不是为儿童的利益着想。这种做法阻碍了儿童性心理的正常发展,并将在成人之后产生不良的后果。

幼儿对性有强烈的好奇心。弗洛伊德在研究时发现2岁的幼儿即在观察成年人,聆听周围发生的事情,并且把那些支离破碎的片段联系起来。儿童对他们父母的亲密行为感兴趣,他们想知道父母睡房中发生的事情。然而对父母的性作直接观察,弗洛伊德称之为最初的性的舞台,常扰乱着儿童的心灵。他们缺乏真正的知识以解释他们所看到的一切。他们常常把性认为是暴力行为。如有的孩子看到父母同房后大为发火,认为父亲竟像日本鬼子那样欺负母亲,于是恨不得立即为母亲报仇。还有的母亲为了掩饰自己的尴尬也说自己受到欺负,这对孩子更是心理打击。

成年人的性活动质量在很大程度上取决于儿童时期是否接受了科学和正确的性教育并形成积极的、正确的性观念。传统观念的清规戒律和错误信念等复杂因素对性功能的天赋起着潜在、有害的影响,父母不正当地非难和指责常常会对其个性的发展甚至一生的幸福产生相当程度的消极影响。因而,我们有必要正确认识儿童时期的性问题,开展适时的早期性教育十分重要。现代性科学认为,儿童也会有某种性的冲动,成人对此不必大惊小怪,严加斥责。如要干预,最好采取巧妙的方式,如做游戏、讲故事等,转移其注意力,切不可使孩子从小就认为生殖器是“肮脏的、见不得人的”。同时,不要怕幼儿看到同性或异性的裸体,不要阻止幼儿同异性小朋友一起玩耍,如果孩子在看裸体的同时询问生殖器的名称,就应自然地告诉他。从小就应鼓励孩子同异性交朋友,防止人为地将男女儿童分开,或对同异性玩的小朋友施以嘲弄和斥责。

孩子对世间的一切都是好奇的,当他们询问有关性的问题时,成人应坦诚相告,既不回避,也不说谎,更不可嘲笑和斥责。否则他们不仅将失去对父母的信任和尊重,还会形成性可耻、不净的观念,不敢正视自身的性冲动,以后当自身出现明显的性变化时,也不敢询问,此时,如遇上坏人,得到一些淫秽、荒诞的知识,将贻害终生。

在回答儿童有关性的问题时,既要注意科学性,又可把人的生育过程与他们所熟悉的动物生育过程作比较,以加深其认识。但不必主动去给儿童讲,只要做到“有问必答”即可,顺其自然,见机行事。有关性的具体知识,一般要到青春期再讲。

家庭教育不得法的另一悲剧是性抑制教育。小男孩玩弄、抚摸外生殖器,父母往往恐吓说,“再玩就割了它”,“脏得很”等。其实,这种年龄的孩子绝无歹念或非分之想,他们的行动决不具有强烈的性色彩,这就像他们摸摸自己的鼻子一样。对此,当父母的千万不要惩罚孩子。儿童的行为只是求知欲的反映,而不是对性本身感兴趣。严厉斥责是不对的,它只会扼杀孩子们的好奇心。在这一时期,最重要的是培养孩子学习和求知的兴趣,一方面可以采取适当的正面教育,另一方面可以用讲故事,做游戏等方式把孩子的注意力吸引开,而对他们抚摸等行为佯作不知,视而不见,不提最好。不然会使孩子们牢固树立起强烈的性抑制心理,认为生殖器又脏又丑,见不得人,神秘可怕,动了就要挨打挨骂,这种消极影响严重时可以影响到成年后的性表现能力。

要尽可能正确回答婴幼儿的问题,不要骗他们。与性有关的最常见的问题是:我是怎么生出来的?这是几乎每个孩子都会问的问题,可是父母的答案往往是哄他们说是:“捡来的”,“石头里蹦出来的”,或斥责他们“讨厌,羞不羞”,“再瞎问就揍你”。这种种回答或处理都会破坏孩子的求知欲望,使他们认为性是见不得人的丑事,是不能说实话的。而他们从其他途径得来的“知识”、听来的“答案”却往往带有荒诞或淫秽色彩,这种错误概念如果在他们心目中深深扎根,很有可能害其终生。因此,对孩子们提出的问题不要回避,要简单而生动地讲解,可用动植物等为例。这一阶段孩子们感兴趣的是知其然,而并不想知其所以然,因此讲解时点到即止,没必要过多解释。

弗洛伊德认为,性角色的学习过程对男孩来说比对女孩更困难些,因为女孩一开始就认同于母亲,与成年期一致。男孩则必须离开母亲,认同父亲,放弃旧有的联系,形成新的联系。这个转变过程的第二个困难在于,母亲一般总是给孩子大量的感情和照料,而父亲则是感情较少,权威更多一些,尤其是中国传统文化总是提倡严父慈母,于是更加重了这种困难。在恋母情结这一关键阶段(生殖器阶段),男性必须学会对女性产生性欲望,认同父亲,成为一个男性。所以男孩在性认同方面同女孩相比要难得多。孩子们的性别角色的第一学习对象就是与他们性别相同的父母亲,如女孩跟妈妈学家务活如搞卫生,男孩则跟爸爸学力气活如干木工和泥瓦匠活儿。所以父母成为性角色教育的示范,孩子们总是以父母的行为举止为楷模、作标准,来衡量自己的行为。在西方国家中,近年来在两性性角色方面的观念有所更新,如更多的女性外出工作,更多的男性担负家务劳动。这无疑也会反映到家庭教育中,孩子们也学习和模仿他们异性父母的行为举止。


\section{第三节 青春期}

弗洛伊德称此期为生殖器阶段或生殖器期。青春期是性生理的成熟阶段,青少年开始对异性产生好奇、爱慕,并逐渐有目的地追求异性,向成年人活跃的异性交往阶段发展,最终建立与异性正常的性关系。从此,生殖器在性生活中占据主导地位。

男女两性的第三性征,又称性别程度或性度,指的是男性气质和女性气质的明朗化,也即男女在性格、气质、感觉、感情、智力等方面的性别差异,如男孩子以阳刚为本,他们多表现为直率、勇敢、雄心勃勃、好斗、对爱的要求强烈而且主动;女孩子则以温柔为贵,她们呈羞涩、虚腼、多愁善感、温文尔雅,对爱的要求被动。虽然性度是历史上劳动分工的产物,不过随着时代的进步,人们的观念和意识已发生明显变化,于是提出第四性征的概念,即男女两性的气质应该双性化,相互取长补短,男子应多一点儿温柔少一些毛躁鲁莽,女子应多一点儿热情泼辣少一些娇柔做作,这样的刚柔并济、感情丰富、才华出众等双性化特征将使两性更有竞争力,自尊感更强烈,事业也更为成功。取长补短可以充分克服各自的缺点,但若取了对方的缺点就糟了,如男孩子羞羞答答,见了生人特别是女孩子就满脸通红、欲言又止、词不达意;女孩子又过于轻率、主动,往往导致人际交往的失败,或带来种种性问题。男女性别差异和动物的雌雄差异一样是天生的,这种性差不仅决定了两性鲜明的特征并使各自的优势得到充分体现。比如阳刚之气使男子能够对付自然灾害、外敌侵袭、保卫家园,就像“男”字的组成那样,在“田”里干“力”气活的人;而阴柔之美则使女子能更好地抚养子女并成为家庭的凝聚力,恰像“女”字,是盘膝而坐守家中的人。第四性征强调的是两性的互补,这就促进了家庭的和谐与稳定,互补型的婚姻使两性更容易交流,从而达到心理上的愉悦,也更容易激发相互的吸引力,从而使爱情之树常绿。男女双性化、男女合一的提法确实动摇了传统的性观念,而传统性观念中恰恰拥有许多性别歧视和性别压抑。性角色的改变有助于给予两性相同的尊重、理解和评价,有助于两性平等和友好相处。第四性征的提出并非要抹杀两性的性差,性差是人类永恒的特性,但在承认性差的同时,寻求互补却是利大于弊的事。男女应该共享智慧、天赋、胆识、进取心、慈爱心及各种情感,不要重蹈历史上歧视女性的覆辙,应逐渐创造更加融洽的环境,促使世界和人生更加美好。

女孩的青春期通常开始于9~11岁,较男孩要早两年。卵巢分泌的雌激素逐渐增多,使得乳房及子宫发育,脂肪的分布促使女性体型的形成。女性也分泌雄激素,使阴毛及腋毛生长、阴蒂及大阴唇增大。雄激素还影响男女两性的性欲及生殖器的敏感度。在青春期的中期(11~13岁,峰值为12岁),丘脑下部开始周期性地释放性激素,于是出现了月经。

第一次月经来潮被看做是女性从儿童期向成熟转变的重要标志之一,月经初潮对少女性心理的发展有着重要的影响。如果家长事先将这方面的知识传授给她们,使她们有思想准备,并帮助她们处理好经期的事宜,少女就会对月经来潮持积极的态度。

遗憾的是事实往往不尽如人意。大多数家长避讳谈论这方面的事情,少女缺乏这方面的知识,思想上缺乏准备,加上第一次月经量多并带有血凝块,腰腹部疼痛不适等现象,使她们不知所措,甚至紧张恐惧。有些少女也许正在准备着、期待着月经的来临,但她们万万没有想到会遇到如此尴尬的困境。于是从此对女人的处境不满,对自己的性别身份不满,影响了性心理的正常发展。少女要知道月经本身会带给她们这样那样的不适,学会在行经期保护自己。

在行经期,多数女孩会感受到不同程度的下腹胀痛或腰部酸痛。这是由于盆腔中的脏器在每次月经来潮之前就开始充血肿胀,并且一直持续到本次月经结束;再加上为排出萎缩脱落的子宫内膜和经血,行经期子宫壁的肌肉不断收缩。在行经期的第一二天,腹痛较重,此后的几天里,伴随着子宫内充盈物排出体外,盆腔内的脏器充血减轻,腹部不适自然缓解。如果在月经期间,下腹疼痛非常剧烈,以至于影响自己的日常生活和学习,这就是痛经。青春期女孩的痛经多是功能性痛经,即生殖器官无器质性病变,常在月经初潮开始就有的痛经。引起这种痛经的确切原因不详,但研究发现,除了内分泌和子宫的功能外,心理因素也有一定的影响。严重的忧郁、敏感、紧张,或由于对月经生理的错误认识而产生的恐惧,或因对月经来潮有“倒霉”“痛苦”等心情而产生的憎恶、怨恨,或在经期情绪不稳定、有心理冲突等都会引起或加重痛经。因此,除了及时请教医生,调剂自己的情绪非常重要。

一般来说,女孩在行经期会比平时更容易疲劳,有的还会感到头痛。这是由于女性的体力随着月经周期而呈周期性变化。月经来潮前8~9天体力逐渐上升,直到前1~3天体力达到高峰,月经来潮后体力急剧下降。等到月经结束后体力逐渐恢复:月经结束后3~5天体力比平时稍高,此后又很快回复到平常水平,直至下一次月经的到来。知道了这个规律,你就不要在月经期间对自己的体力有太高的要求,不要过分劳累,要适当地休息。如果条件容许,不妨把那些需要高体力、动脑力的事情放在自己体力最好的时间去做。

有的女孩发现自己在月经期前的几天里情绪会突然发生变化,表现为烦躁易怒、焦虑不安、精神紧张、缺乏自信、注意力不能集中,甚至还会出现一些躯体症状,如失眠、乳房胀痛、头痛、腹胀、水肿等,月经来过以后,症状自然消失,医学上称之为经前期紧张综合征。

虽然来月经会给少女增添不少麻烦,但它毕竟是一个成熟女性所必须拥有的正常生理现象。只要逐渐习惯并注意经期卫生,调剂情绪,适当休息,月经并不会影响她们的正常生活。千万别把自己当做一个患者,否则会加重经期的身心不适。

12岁时,男孩的睾丸开始增大。约一年之后,睾酮的产生逐渐增多,使得阴茎增大、阴毛生长、前列腺增大、声音变粗、形成男性的肌肉和骨骼。通常在12~13岁时,就像女性月经初潮一样,少年男子开始射出带有活精子的精液。男孩的第一次射精对性心理发展极为重要,因为这种经历会伴有不同的情感反应。不少男孩表现为紧张、羞涩、害怕、困惑,而这种与射精伴随的情感体验会长期保存在一个人的性活动中。因此,在此之前对青少年进行性教育,让他们了解这方面的知识就十分必要了。一位青少年这样写信向专家求教:“‘通常遗精是一种正常的生理现象。只有频繁遗精(例如一两天一次)才对健康不利。为了减少或者消除非疾病引起的频繁遗精,我们应该把精力都用在学习和工作上。’这是我初二生物书上面的原话,我很相信,请问这里所说的对健康不利到底指什么?既然一天遗精一次对健康不利,那么一天手淫一次会不会对健康不利呢?这里所说的“不利”是不是说精液是人体的精华,丢失多了伤害身体?我很焦虑,因为我无法把自己的手淫频率降低到一天一次以下,我又很崇拜武打明星,希望自己拥有强壮的身体,我实在是不能忍受自己的精华全都这样被自己射出去。拯救我!深度抑郁中!拯救我!”

“我们应该把精力都用在学习和工作上”,多么虚伪的说教呀!先听听一位老师说的话:一位中年男教师在给学生上性教育课时坦然地谈起自己的第一次遗精:“那是在我15岁上初三那年,不知怎么搞的,我在那个记忆犹新的特殊夜晚突然感到一阵阵莫名的慌乱,坐立不安,浑身燥热,躺下后也不像往常那样很快入睡,翻来覆去地把床压得吱吱乱响,好像还因此受到父母的呵斥。迷迷糊糊地不知睡到什么时候,我突然惊醒了,还没等我反应过来,勃起的硬邦邦的阴茎突然一阵阵地急剧抽动起来,紧跟着便排出一股股热乎乎的液体,我还以为是流出尿来了,赶紧用手去捂,怎么黏糊糊的,那就不是尿,难道是出血了?这是什么要命的病呢?当时吓坏了,但又不敢开灯看看究竟是怎么回事,反正紧张得没觉出一点点快感,只好忐忑不安地等着天亮。天亮之后一看,内裤和床单上根本没有想象中的血迹,由于没睡好觉,上课时还昏沉沉的,脑子里不住地想,这是怎么回事呢?后来查了半天书总算‘明白’了,原来是‘淋巴液’,因为书上讲到淋巴液是无色的。等我弄明白那是精液已是大半年后的事了。那时根本没有性教育,除了政治还是政治,根本没有任何书刊杂志可以查阅,有事又从来不敢问父母,只好憋在心里挥之不去,现在想起来我们那时的孩子真是可怜。后来家长终于发现我‘画’在床单和内裤上的‘地图’了,他们总训斥我,说如果我能控制自己,别去胡想‘那类事’,床单就不会弄脏了。我不知道他们说的是什么———为什么我要控制自己?别去想什么事?可我什么也没想呀,我在睡觉啊。到了我念高中时,他们更加关注这件事了,总劝告我要注意身体,千万别早早就把身体搞垮了,将来可就完了。当时我总不明白,它怎么就和我的健康和前途有关呢?”

中国传统观念过分强调精液宝贵,所以始终把遗精视为精关不固、肾亏或肾虚,会导致元气大伤,因此存留了大量专治遗精的方药。如果说手淫尚可戒除,然而结婚前的遗精却是不可避免的,所以有人宣扬的什么“根治遗精”的药物简直是无稽之谈。现代医学研究早已证实精液的成分既不复杂,也不神奇,过去的错误宣传必须及早纠正。但是一些以讹传讹的恐吓宣传和错误观点仍然阴魂不散,再加上不恰当的宣传,给青少年带来巨大的精神压力。所以青少年务必掌握足够的科学知识,正确面对自己的身体和生理现象,不要被那些可恶的宣传吓趴下。性成熟之后,下丘脑开始分泌一系列多肽释放激素作用于脑垂体,脑垂体虽然小如豌豆,只有0.6克重,但它在接到下丘脑指令后便立即分泌多种促激素,其中的促性腺激素可促进男子睾丸成熟,生成精子和分泌雄性激素。而前列腺和精囊腺则分泌精浆,它们共同组成精液,达到一定量后必然以遗精或手淫方式排出体外,即所谓“满则溢”。

性意识是人对性的认识和态度,青少年性意识的健康发展包括意识到性别内涵、两性差异、两性关系,以及对待两性的态度和行为规范。在封闭社会文化环境中成长的青少年容易对性征变化持惊奇、恐慌等消极的心理状态;而在开放社会成长的青少年则因处于性信息发达的氛围中,会对自己的性征感到高兴和自豪。我国城乡青少年的性心理发展过程可能存在相当的差异。性意识的发展可以划分为以下几期:7~10岁———性别对立期,同性间的交往很多,两性形成对立的群体,男女界限很清楚;11~14岁———接近异性期,开始对异性产生兴趣,喜欢在异性面前自我炫耀,愿意并频繁接触异性,但这时往往是集体交流;15~18岁———初恋期,性意识更加成熟,心中往往有明确的白马王子或白雪公主的形象,这一形象虽不一定具体到人,但对自己想往的目标有了标准。有的已开始发展一对一的异性间交流,甚至进入初恋。

孩子进入青春期后,开始把目光更多地投向同龄人是好事,这标志着他们开始成熟,敢于怀疑家长的权威,包括从小就给他(她)们树立的本性别的榜样,而去寻找比父母亲更为符合自己理想的性别榜样。青少年只有与同龄人打成一片,才能获得从父母那里得不到的信息、成长经验和自主意识。中学生喜欢成帮结伙,他们的友谊来得快,但很盲目;交情深,却又说不清有什么具体意义。他们中还常常有一个领袖式的人物,这个人除了别的优点外,还往往是青少年心中最像本性别楷模的那个人。尽管他(她)可能并不受老师或家长们的赏识,但对同龄人却有着不可低估的魅力。不少家长因怕孩子被带坏,便施行“保护性禁闭”,这是无济于事的。倒不如花点时间精力去熟悉孩子的朋友们,运用自己的人生经验,帮助孩子挑选一个各方面比较成熟的好朋友。家长和老师要正确引导孩子在不同时期处理好与异性的正确交往,既不要严加限制,剥夺他们自由交往的权利,侵犯他们的隐私,也不能不闻不问任其自由发展,乃至陷入盲目的为生理冲动所操纵的两性交流,甚至造成意外妊娠、感染性病的后果。

青少年除了需要在同性中学习,还需要到异性那里去验证,才能逐渐对自己的性身份和性角色产生自信和自律。在青少年的异性交往和友谊中,会不可避免地带有多多少少的性成分,但也绝不是只有性内容。他(她)们往往只是想试探一下自己在异性心目中是不是一个合格的男性或女性。如果他(她)们得到满意的回答,就会产生自信,并逐渐学会像成人那样的自律和自制。因此,作为父母应该明白这个道理,对孩子的异性交往不必那么害怕,应尽力创造条件,使他(她)们无论是在生理还是心理上都能够健康成长。目前社会性观念的逐步开放必将对青少年产生种种影响,关键是应该在理解的基础上予以正确的引导,使他们懂得如何面对和处理两性交往问题,应该记住高压手段往往产生逆反心理,适得其反。

孩子发生早恋是一件让家长十分头疼的事,为此许多家长询问该怎么办?其实,早恋的提法似乎并不科学,因为孩子在这一时期表现出来的仅仅是对异性的好感和向往,而不是真正的恋爱。有些父母受升学压力和固有错误性观念的影响,对少年男女的正常交往过于敏感,动辄打骂训斥,千方百计地压制和遏止这种交往,殊不知在逆反心理的作用下,往往适得其反。一位高二女生的母亲发现她和丈夫出差期间女儿曾把一个男孩领回家,于是撬了女儿的抽屉翻看她的日记,并没收了女儿的家门钥匙,不许女儿有单独在家的机会,还动辄训斥女儿。最后女儿也认可了母亲的说法,觉得自己“太肮脏、太下流了”,竟会做出这么见不得人的事,从此患上强迫症嗜洁癖,整天把自己关在洗手间里没完没了地洗手。家长正确的做法应是鼓励男女青少年相互尊重,平等相待,友好相处。这是破除性神秘感、建立健全性观念的基础。要向孩子讲明在这种年龄涉入爱情为时过早。爱情可以分为几个层次,层次最低的是对两性关系的朦胧意识及对异性的兴趣和爱慕,歌德说过:“青年男子哪个不善钟情,妙龄少女谁不善怀春。”以此为标准来谈情说爱,符合条件者简直太多了;另一层次是外在美,即一个人的外貌带来的视觉美感,如美丽的外表、迷人的风度都能形成对异性的吸引力,俗话说情人眼里出西施,仅以外在美作为标准,被挑中者将不止十个八个;而最高层次的爱情则是建立在内在美的基础之上,它包括品德、才华、性格、情趣、气质等,这一层次的条件千差万别,综合这些条件的最终结果才能选中一个中意者。而对这些条件的选择,无疑要等一个人成熟了才能有所认识和判断,作为中学生尚缺乏这些能力。中学是接触社会的实习期,青少年在这个阶段应努力培养自己处理人际关系、特别是处理与异性关系的能力,在广泛的交往中建立纯洁的友情,认识自我,完善自我,促使自己人格的健全发展。要知道人生道路是漫长的,在不同阶段有不同的任务,中学生处于知识与身体全面发展的时期,主要任务是学习,错过这段黄金时期将悔之晚矣。如果过早地把注意力集中在一个人身上,过早脱离集体生活,把自己局限在二人天地中,容易导致人格发展的偏颇。

青少年在与异性接触时会产生愉悦的感受。这种体验称为性情感,青少年的性情感体验像天上的云,说变就变,来得快去得也快,而且往往产生不少错觉,一厢情愿地把对方的无意表现解释为爱情来临。当青少年不断成熟时,他们的性情感才能趋于稳定、成熟,避免无节制地滥用或盲目追求。

青少年的性适应包括对身体性征和性心理与生理变化的适应,对和异性相处的适应,对社会道德约束与行为规范的适应。刚进入青春期的青少年大多适应不良,随着年龄的增长和人生阅历的丰富自然会有所进步,不断成熟,最后达到性适应良好的状态。对那些少数过早发生了不适当的性接触的中学生,作为父母应和风细雨,循循善诱,绝不能简单粗暴。要让他们懂得这涉及社会道德规范、伦理关系、法律约束等多方面的问题。因多数孩子属于缺乏自控能力,不要统统扣上道德败坏的帽子,学校也不要急于开除、处分,要本着爱护、保护的原则,妥善处理。青春期少年男女们之间的性活动方式主要是同性之间的性游戏。根据金西等人的调查,女孩同性之间的性游戏在9~12岁达到高峰,一直到13岁这种活动仍然多于异性之间的性游戏;男孩的顶峰在12岁,一直到15岁这种同性之间的性游戏仍然多于异性之间的性游戏。不可轻视这种同性之间的性活动,因为它对性心理发展起着重要的作用。因为此时期的少男少女心理尚不成熟,情绪亦不稳定,不知如何处理与异性之间的关系,而与他们所熟悉的同性伙伴一起活动,会容易、自然得多,由此逐渐过渡到与异性伙伴来往。这表明同性之间的性活动实际是为今后与异性进行性活动的铺垫。

情绪不稳定,易激惹,行为捉摸不定是少男少女的特点。这种“青春期躁动”的心理特点是由于性激素分泌增加所造成的。性激素的作用甚至可以发生在少年少女身体明显变化之前,并影响其行为,以致这些少年不知自己为什么会变得那样躁动不安,控制不住自己。即使在身体外部的变化出现之后,这些少男少女仍然表现得幼稚浮躁,情绪变化不定。他们还往往自以为是、与人顶撞、与人格格不入等。

青春期,男女少年开始有意识地手淫,尤其男孩更为多见。通过手淫,他们的性欲得以释放,并获得在身体及情感上的愉悦体验。根据学习理论,成熟的性行为是在逐渐强化的过程中获得的,而手淫中体验到的性快感将会促进少年男女性活动的成熟。手淫时,通过手与生殖器接触所引起的快感,还可以逐渐改变人们对生殖器的态度,扭转对生殖器肮脏的错误看法(尤其是女性)。

每个人对手淫的反应不同,这由对手淫的认识和态度所决定。不少人多多少少会有些内疚、焦虑不安、恐惧,担心会损害身体。尤其是那些听信不科学宣传的少年,思想上的负担更大。在未能找到性伴侣时,手淫无疑是一种解决性欲要求安全而有效的途径。对于少女来说,这种自我刺激的方式提供了一个积极了解自己生殖器官感觉的途径。手淫时,阴道分泌物的气味、身体在性反应时的微妙变化,插入阴道时的体验等,都能使她们更好地认识自己的身体,认识到性活动并不像有些人宣传的那样神秘,那样令人恐惧,或者那样令人神魂颠倒。因此可以说,只要不是过度,手淫不失为一种促进性心理发展的积极因素。

在青春期的早期阶段,少男少女对于性的注意力主要还是集中于自己。虽然他们也会想象得很浪漫,但在幻想的世界中,仍然是以自我为中心,他们考虑自己的感受会远远多于顾及别人。此时男孩会想到的是:“吻她,我会有什么感觉?”、“摸她,我会有什么感觉?”,而不是“她会有什么感觉?”。女孩的想法也与此相似。

少女与异性的社会活动要比少男来得早。先前的“假小子”,此时大多变得温柔娴静,越来越像个“大姑娘”。她们更注意自己的装束打扮,自觉不自觉地在异性面前忸忸怩怩、娇嗔作态,以唤起异性的注意。在此时期,男女集体交往的形式有时会被成双成对的活动所代替,少男少女们更加注意与异性伙伴关系。

青春期的后期阶段,女孩大约在16岁,男孩在17~18岁时,也就是相当于上高中的时代,个体的性发育日趋成熟,青年男女的接触从集体的形式渐渐发展为只有两个人的活动。开始时通常是双方互相注意,或是一方有意地寻找与对方接触的机会,然后便是早期的约会———虽然此时他们大多数的时间仍是参加集体活动。

通过约会,男女成双成对的活动,两个人频繁地接触、交谈,谈话的内容逐渐扩延、深化,于是青少年们第一次有意识地去了解别人,理解别人,确认自己在他人内心世界中的位置;同时他们开始努力塑造自己,以博取对方的好感。因此,我们说,男女青年的约会可以为人格的形成和性心理的发展提供一个新的途径和积极学习的机会。

与青春期的早期不同,这时他们对性的注意力大都不局限于自己,而是转向异性伙伴。他们更多地考虑对方的感情和利益,心甘情愿地照顾自己的异性伙伴,表现出对异性伙伴关怀备至,不少人还考虑到结婚的问题。但是,由于心理上的不成熟,情绪上的不稳定,行为上的冲动性,两个人的关系发展到一定程度之后散伙的屡见不鲜。

虽然目前社会上有早恋的趋向,少男少女之间发生性关系亦非罕见,但手淫仍然是这一时期解决性欲的主要手段,尤其是男性。到底在青春期的哪个年龄两性开始互相之间的性活动,则由于地理位置、社会文化、风俗习惯等而大不相同。


\section{第四节 成年期}

不受父母的管束以及经济上的独立,使青年们有更多的自由去谈情说爱。随着约会的发展,两性逐渐变得亲密和不拘形式,这时,双方都希望肉体上的进一步亲近,有了拥抱、接吻、抚摸等行为。

爱抚反映了一个人对性活动的态度,对性心理的发展有积极作用。拒绝爱抚或不能体验到爱抚带来的愉快是心理不正常的表现,往往表明该人对性活动感到羞涩、自卑或是有罪恶感。只有消除个人在心理和身体上的这种防线,才能真正体验到爱抚所带来的愉快,进一步认识自己和对方身体不同部位在性活动中的感受和作用,增加双方的快感。

性交是两性关系中情爱的高级表现形式,也是男女双方精神融合的过程。性交成功与否取决于两个人的感情、欲望、和谐程度以及身体状况。如果双方都能尊重对方,尊重对方的感情,在性交过程中经常交流体验,互相配合,就能使双方在性交时达到心身融合的程度。但成人健康的性爱活动应该能够专一(并非指个体一生中只能有一个性伴侣)。在这里面,青春期性心理的健康发展,以及青年人最初性关系的体验,对个体与性伴侣成功地结合有着重要的作用。而今,婚前性生活并不少见。值得注意的是,从1997年到2001年,大学生婚前性行为的比例在增加,特别是男生,在5年中性交的比例增加了一倍;还需指出的是,有12.8\%的男生、2.6\%的女生、总体的7.3\%在上大学前就有过性交。据2003年2月11日《中国青年报》报道,在重庆大学生中,男生的23.5\%、女生的10.2\%、总体的12.8\%有过性交行为。最有代表性者为2004年中英性病艾滋病防治合作项目专家公布的调查结果,该报告称我国大学生对婚前性行为持赞同或无所谓态度者已占90\%以上,与美国大学生的相应指标值相似。

结婚和生儿育女只是成年人解决性问题的许多方式中的一种。不能说结婚是解决成年人性欲唯一适当的、最健康的途径;也不能说结了婚当了父母的就一定比不结婚的,无子女的人更健康些。只能说结婚、为人父母有它的问题;而不结婚、无子女的又有另外一些问题,孰优孰劣,恐怕在今后相当长的时间内都难以得出定论。

但是,结婚毫无疑问地从法律上保证了夫妻在性生活上的合作关系,是取得性满足和生活幸福的有效途径。性活动在婚姻中占有非常重要的地位,脱离了性活动的婚姻是难以维持的。研究表明,夫妻间对婚姻感到满足与获得性满足的水平之间存在着相互作用。在婚姻中,性不满足却不破坏婚姻幸福感的情况是极少的;而婚姻不幸福却不破坏性满足的情况同样少见。在良好的婚姻状态中,为了取得性满足,夫妻双方应该经常对他们之间的关系以及性活动进行协调,达到真正的情感交流,从而保持夫妻之间的爱情关系,使之不至于枯萎。

在第一个孩子出生之后,夫妻充当起父母的角色,即“双亲身份”。他们进一步与自己的父亲或母亲认同,在家庭生活中抚养孩子、获取经济收入和分摊家务劳动的安排里,完成自己的“婚姻角色”。

潜在的性快感似乎从出生之时就开始了,除非生命终结,否则它不会结束。年龄对性行为的影响很大,在不同的年龄阶段里性功能和性反应的激烈程度和质量将会发生相当大的改变,年龄对男女两性的性活动周期有着迥然不同的影响,而人类的其他功能如学习能力、体力、身体调节能力等受年龄影响则无明显的性别差异。比如男子的性反应和性能力在性成熟后的17~20岁迅速达到高峰期,随后逐渐稳定地减弱,而女子的性反应和性能力在性成熟后缓慢增强,直到35~40岁才达到性的高峰,随后以比男子更为缓慢的速度减弱。

一个20多岁的正常青年即使在忙于求学、运动等其他追求之时,也时刻想着寻觅女友,也会以手淫或梦遗的方式来获得性高潮,在日常生活中对各种可能的性刺激反应极为敏感。如一位年轻男子会对因汽车颠簸而引起的勃起感到尴尬。此外,高潮后的消退速度很慢,他可能在射精后半小时左右还会有一定程度的勃起,他们的高潮反应很强烈且射精力量强劲有力。

30来岁时,性冲动的急迫程度减弱。这个年龄的男子仍对性追求极感兴趣,但是他们的高潮次数不那么频繁。大多数男子满足于较为完美、和谐的做爱。一般来说,30多岁的男子不会在没有性刺激的条件下过分沉溺于对性的想象和性幻想。但他们对性刺激的反应仍然很快,并积极寻求性机会。从机体上讲,他们的勃起快,消退慢。等男子接近40岁时,除在异常的刺激条件下,一般不能很快地进行连续射精。

在20多岁的已婚生活早期,女性的性交频率较高,这可能主要是由年轻丈夫的强烈性欲所激起的。但这一时期的女性常常会因丈夫的速战速决及过高频率的要求而苦恼。一般情况下,妇女在35~40岁前后达到她们性反应的高峰期,阴道润滑频繁发生,多次性高潮经常出现。国外报告指出这一时期也正是妇女发生婚外恋的高峰期。许多妇女说她们对性的兴趣比早年时更浓厚,但这往往不是生理因素决定的,更可能是心理压抑感和障碍一扫而光,不再故作矜持的缘故。由于男子早已走下坡路,所以这一时期是性生活的不协调阶段。这些妇女对性已有一定程度的自主权,学会向她们的丈夫要求一种容易使她们唤起、但又不会像以前那样令她们感到羞涩和恐惧的刺激。


\section{第五节 中年及老年}

我们社会或含蓄或明确地认为,只有产育年龄的人才应有性活动,年纪老了就不应该有性活动了。这种看法以及心态是错误的。40岁以后,男子的性快感开始发生明显的变化,所以有人说40岁是青春期的老年,50岁是老年时的青春。男性此时的性反应特点由年轻时集中于性器官的强烈感觉转变为扩散泛化、延及全身的感受,他们对性高潮的追求逐渐变得不那么迫切。50岁以后,对男子年龄增大反应最敏感的两项指标是:①性高潮的频率降低;②两次射精间隔的时间延长。50岁以后妇女的性反应同样因人而异,她们的性表达往往取决于其性需要和性能力日益减退的丈夫的影响。一位有正常性交机会的妇女一般能保持她的性反应性,缺乏这种机会,性欲则会明显下降。

女性更年期的概念已逐渐被废止,改而称之为围绝经期,即从开始月经不规律到绝经后12个月这段年龄范围,也就是说它有一个明确的标志———绝经。妇女的中年,起始于丧失生殖能力之时,即所谓的更年期。月经紊乱及绝经使许多妇女的心理状态发生了或轻或重的变化。有的人想到前半生大好时光已过,结果是碌碌无为,往后精力不足,更是黯淡无望;加之子女长大离家独立,一种无可名状的孤独感油然而生,甚至心灰意懒,情绪低沉,焦虑不安。激素的变化,还会造成躯体上许多不适以及阴道干燥等症状,进而影响性反应和快感。好在只有15\%的绝经期妇女上述表现较为明显,而且可以用雌激素替代治疗(HRT)预防阴道的变化。

更年期妇女的性功能差异很大,而且还由妇女们的总体心理状况及与丈夫的关系所决定。排卵功能的突然停止,血中雌激素和孕激素水平的突然下降,使许多(但绝对不是所有)妇女出现沮丧、易激动、易怒等情绪波动。这些变化就像一小部分妇女在每月的“小更年期”(即血循环中雌激素和孕激素暂时下降的经前期4天和经期4天共8天阶段)中所经历的那种情感危机或“紧张”一样。在这8天之中,精神病、暴力犯罪事件、事故及发病率均明显增长。女性性甾体激素的撤退对性欲的影响也不是一成不变的。如果一个女子感到沮丧、易怒、不安时,她是不会对性感兴趣的。与此相反,也有许多妇女觉得在更年期的性欲增强了,从物质基础上说,这是女性体内的雄激素(对女性性欲起重要作用)不再受雌激素的排斥而引起的。此外,这一时期的性功能还受包括心理因素等在内的许多因素的综合影响。但如果中年丈夫因沮丧、不安全感等回避性生活时,愤怒的妇女将视此为自己肉体吸引力的减退,她也同样回避性生活,也和丈夫一样经受这种挫败感和冷落感的痛苦。我们必须把中年女性出现的这些生物学变化理解为日后性心理发展的动力,而不能因为少数人出现的躯体和/或精神症状就武断地认为这些生物学变化只能带来麻烦。

比如一直采取避孕措施的妇女,在绝经期之后再也不用担心怀孕生育。这样,她们在性交时就会感到身心放松,充分地享受性生活所带来的乐趣。又由于这时子女们都成长自立,家务事大大地减少了,生活节奏放慢了,夫妻俩有更多的闲暇时间在一起交谈、散步、欣赏生活,从而增进彼此间的感情。研究证明,此期间不少夫妻谈到他们的婚姻满足感并不是随年龄而下降,反而上升了。这时,妇女从繁杂的家庭事务中解脱出来,不少人又重新走上工作岗位或开始新的职业生涯。由于看到自己继续对社会作出的贡献,提高了她们的自尊心以及自信心,从而克服了她们抑郁的心情。

近年来越来越多的学者倾向于接受男性从中年到老年的转变过程中也会经历一段不平常的时期,过去叫更年期,后来称为中老年男子雄激素部分缺乏综合征,不过科学家们最终确定把中老年男子因为雄激素减少造成的一系列相关症状称之为迟发性性腺功能低减。但研究人员说,男性的生物学变化没有妇女绝经期那么明显,男性不会因为年纪大而出现性功能和生育能力的急剧减退或丧失。不像女性更年期会影响所有女性,男性更年期是比较少见的,只影响5\%左右的老年男子,并经常伴有一般健康状况差和肥胖。

英国曼彻斯特大学、伦敦帝国学院、伦敦大学学院和欧洲其他的研究人员一起,从欧洲的8个研究中心测量了3369名40和79岁男子的睾酮水平和详细询问了他们的性、身体和心理健康。研究小组发现,具有症状的32名男子中只有9名实际的睾酮激素水平低,最重要的3种症状是:清晨勃起次数减少、性的向往减少、勃起功能障碍频繁发生。这项研究的结论是,所有这3个性症状连同睾酮激素水平低,是诊断迟发性性腺功能低减所必需的,但其他非性症状也可能存在。其他症状包括3个躯体症状:不能从事剧烈运动如跑步或提举重物,无力行走超过1公里,以及无法弯腰或跪下;3个心理症状:精力不济,悲伤和疲劳。既然所谓男性更年期是一个不争的事实,那么它有哪些主要表现呢?

(1)精神与心理症状:神经过敏(约占就诊者的90\%);健忘、效率低、注意力不集中(76\%)、焦虑、急躁、爱发脾气(80\%);倦怠(80\%);抑郁或压抑感(97\%);睡眠减少、失眠(60\%);兴趣减低(60\%);麻木、刺痛感(44\%);恐怖感(40\%);缺乏自信心、常有孤独感、易纠缠琐事等。

(2)血管调节失常症状:更年期男子有时会像孩子一样浑身发热、不安、蹬被子、躁动不安、易出汗;此外,还有头痛、眩晕、心悸、眼前有黑点;也有的人自觉四肢冰凉。

(3)生理体能症状:失眠、食欲减退、骨骼与关节疼痛。

(4)性功能减退:常见的有性兴趣减少、性欲低下和勃起功能障碍(阳痿)。更年期勃起功能障碍是最常见的自我感觉症状,这里既有机体生理方面的原因,如中枢神经系统介质传递能力的改变,即脑内多巴胺和5羟色胺水平降低;更有精神方面的原因,如社会地位、生活地位、家庭因素等变化带来的影响。总之,更年期男子的性功能衰减就是向老年和随后各器官衰退过渡的征象。

出现上述症状并不奇怪,因为睾酮分泌的减少确实会在不同程度上对男性的体力和精力产生一定影响。如果对雄激素部分缺乏缺少心理和精神准备,还像年轻时那样事事跃跃欲试,处处奋力拼搏,直到事态发展到力不从心时就会突然出现失落感,心绪不宁,忧心忡忡。这种紧张不安的心理状态反过来又可能加重上述生理变化对机体的影响,变得疲惫不堪、未老先衰,尤其是知识分子更要处理好这一时期的保健问题。

值得注意的是,男性迟发性性腺功能减退(俗称男性更年期综合征)的诊断并不是轻易能下的,必须先排除这段时间内各脏器常见的、易发的各种器质性疾病,如抑郁症、神经衰弱、脑动脉硬化、高血压病等。如有一位老年男子拿着讲述男子更年期的报纸由老伴陪同前来就诊,医生听完病情介绍后作出初步判断,其实他患的是精神分裂症中典型的迫害妄想,每天坐立不安,总说有人要整他,决不是什么更年期的表现。必要时可测血睾酮和LH水平,如果前者明显下降有助于诊断,后者则往往偏高。中老年男子也可进行自我判断,如果经常出现某些症状,说明有部分雄激素缺乏的可能,如果经常出现大部分症状或多数症状持续存在,便可初步诊断,应找医生治疗。那么我们的对策又是什么呢?

(1)有无必要治疗:一般来说,只有1/3~1/2的更年期男子需要治疗。关键是能自我控制情绪,像林则徐的座右铭那样“制怒”,像郑板桥那样“难得糊涂”,并注意保持愉快和稳定的情绪,减少精神创伤;控制工作量,但又要保持一定的运动量,切忌在办公桌边度过你的白天和夜晚,因为过度疲劳后不但休息不好,还会因异常兴奋而不能入睡;在饮食上要限制脂肪和糖类食物;对于性生活则需要多情,相互体贴,而不可纵欲,但也不必惜精如命,过分抑制生物本能同样对健康不利。最好把性生活控制在最低生理需要的限度之内,但平时又要多一些亲昵和爱抚,这种体肤之亲对保持双方身心健康都十分有利。

(2)如果雄激素水平确实低下,症状明显时可以长期给予雄激素替代治疗,这一方法特别适用于那些性功能障碍和性欲都应得到治疗的中老年男子,对单纯因其他器质性因素所致的性功能障碍疗效较差。应该记住激素不能自行滥用,要由医生指导服用。一般可以先行试验治疗,比较一下治疗前、治疗期间(4周)、治疗后4周左右的症状变化情况。如果患者用药后有进步,说明需要治疗,应适当补充激素,每3个月为一疗程。目前使用的是十一酸睾酮系经淋巴吸收,不经肝脏直接入血,所以没有肝脏首过作用对雄激素的破坏,服用后生物利用度很高,也不增加肝脏解毒、代谢的负担,商品名为安特尔。过去使用的甲基睾丸素对肝脏毒性作用大,不应再用。在进行睾酮治疗前应作直肠指检和前列腺特异抗原的测定。轻度前列腺增生是睾酮治疗的相对禁忌证,而不是绝对禁忌证,如果不存在或偶有泌尿道梗阻的性腺功能低下患者仍可接受睾酮治疗,如果梗阻症状还在发展则不应服用激素。直肠指检或前列腺特异抗原检查怀疑有前列腺癌或已确诊为前列腺癌则是睾酮治疗的禁忌证。由于服用睾酮后可使体内雌二醇水平增高,因此,男子患乳癌时也禁忌睾酮治疗。长期服用者应每半年进行直肠指检,前列腺特异抗原复查,并注意患者的主观反映如尿路梗阻。每年还应查一次血脂、血红蛋白和血钙。长期服用睾酮,应根据症状的改善调整剂量,使之尽量保持在最低维持量水平。经过10年长期使用安特尔的临床观察,没有发现它能诱发新的前列腺癌,也未发现其对肝功能有任何影响,所以它应该是安全、可靠、有效的。

老年人血液中的睾酮逐渐下降是不可否认的事实,但这并不意味着生育能力就一定减退。有不少例子证明,八旬、九旬的老翁仍有生育能力。男子性高潮后不应期时间随年龄的增长而延长;阴茎勃起角度随年龄增长而减小;勃起功能障碍发生率随年龄增长而增长,但65岁以后,年龄因素才起重要作用。老年人如果长期中断性生活,其性能力会如上所述发生不可逆的损害。

此外,一些年老男子常经历着一种“自相矛盾的不应期”,是指老年男子在拖延较久的性嬉戏过程中如果失去勃起,他就要再过12~24小时后才能再次勃起,从而使这次性活动半途而废,这就像他已经经历过实际上的性高潮后一样,有人把这称之为继发性不应期。此外,在60岁或年龄更大时,射精时的喷射力量也大大减弱,甚至减到缓缓涌出的程度。在这一年龄阶段性高潮的消退过程是十分迅速的,射精后阴茎很快完全疲软。55岁左右的男子体内的雄性激素生成逐渐减少,但不同于女性卵巢功能的突然衰退。有些男子在这一阶段的生活中出现沮丧、容易激动、无精力、性生活遭遇困难等。所以,男子更年期的含义是对性功能减退的心理、生理反应的集合,而且这种年龄的男子普遍都会经历这种变化。这种生理改变除了对雄激素水平的影响之外,还有多种心理影响,而给予睾酮补充治疗也并不是总能使男子的沮丧程度减轻,或使其精力有所改善。相反,如果他们对这种变化早有思想准备的话,可以通过各种防御和适应性活动来有效地处理和度过这种危机。

老年男性性欲的减退,也表现在对性兴趣、想象及性幻想的减退上。几乎所有传统文化和社会习俗都想当然地认为,男子步入老年之时必然是出现继发性勃起功能障碍之日,这种谬误的影响大概比其他任何一种的影响都来得深远、广泛和严重。事实上,尽管年龄增长确实伴随着身体功能的退化,但是年长的男子对性刺激仍有很强的反应力,尤其在他积极地保持着性活动的情况下。改进性技巧能调节这些变化,其中包括加强肉体刺激的强度和延长性刺激的时间。但继发性勃起功能障碍并非衰老的必然结果,即使出现也是可以经治疗而改善的。在多数情况下,即使是器质性原因造成的继发性勃起功能障碍(手术、创伤、血管等),不论其年龄如何,都是可以得到相应治疗的。当然,性能力也肯定随着年龄的增长而逐渐减弱。

经过上述变化的健康男子,是能够在晚年保持享受性乐趣的能力的。的确,摆脱了强烈的高潮发泄需要和青年时代的压抑之后,年长的男子与其性伴侣将能经常享受到更为满足及充满想象的性爱。对生活安逸的男子来说,如果健康情况好且又有机会,年龄从不会对性的乐趣构成障碍。一个80岁的男子即使不能再产生年轻时强烈的多次高潮,也能体会到明显的射精感受,而且他还完全能够经历偶尔的高潮和更多愉快的勃起,享受更多的体肤之亲,这就是性与男性的一生。性在男性的一生中从不会停止,即使到了老年,也不会像水龙头那样一下子就完全关闭了。在性生活中,男性在没有患什么疾病的情况下,其勃起功能绝不会因年纪大而丧失。60岁以上的男性所分泌的睾酮足够维持他们的性行为。随着年龄变老,只不过是性唤起所需的时间、达到性高潮所需的时间以及性高潮过后的不应期时间延长,射精不如以前有劲,也不会每次性交都射精。千万不要误认为这是男子阳刚之气衰退的表现,应当看到这些变化所带来的积极作用。

这些变化避免了性活动中激烈鲁莽的行为———这些行为对于老年夫妻来说已没有什么意思。这些变化导致更长时间的爱抚活动和性交。在这种较长时间的性爱活动中,男方更加注意女方的感受,力图带给其更多的乐趣,显得体贴入微。许多女性爱上年纪比她们大得多的男性,恐怕也就是出于这个缘故吧。

多数研究表明,老年妇女的性活动略低于老年男性,这很可能是由于老年妇女缺少性伴侣的缘故。65岁以后,她们的性兴趣开始下降,但仍可以寻找和对性机会作出反应。这一时期中,她们的手淫或与男性的性活动并非少见。这时女性阴道润滑的发生趋于缓慢,高潮期阴道高潮阵挛性收缩的激烈程度、次数减少,与此同时,她们对性感觉的强烈程度也减弱。与男性明显不同的是女性到老年后仍可保持多次性高潮的能力。许多妇女在50~60岁后便停止性交,这种性禁忌的出现不是生理因素所决定的,而主要受社会和心理的影响,特别是她们的性观念的影响。

研究表明,能将性活动保持到晚年的夫妇,一般都是那些在初期性活动中即能共同体验到性快感的人。因此,除了身体健康情况之外,年轻时期的性快感对今后一生的性反应有很大的影响。

年老、慢性病以及配偶的去世,会影响老年人的性功能。65岁时,大约20\%的已婚男性和50\%的已婚女性丧偶(美国)。许多老人的配偶虽然还活着,却因为慢性病而损害了他们的性功能———男性多于女性。但是大部分老年人的性功能依然存在,并能从中得到乐趣。许多研究证明,那些“老年人没有性要求”、“老年人丧失性功能”、“老年人不应过性生活”等观念纯属无稽之谈。老年人有权力、也有能力享受性生活。

老年人应该有活跃的性生活,而活跃的性生活反过来又可促进老年人健康长寿。即使由于躯体的情况而损害了性功能,他们也可以用其他的方法来弥补功能上的不足。如果阴茎不能勃起或勃起不坚,无法进行性交,男人可以采用其他身体接触的方式对妻子爱抚,使双方都得到快感,并保持自己的男性尊严。在没有配偶的情况下,不妨用手淫的方式来达到性满足,切不能将其当做是一种“可耻”的行为。也就是说,衰老和疾患不应该妨碍一个人完整而有尊严的生活。

总之,年龄会使性反应的性质发生变化,老年夫妇对此也有一定认识,从而重建他们的性活动方式。这种方式不应当只限于采取性交这一种形式。应当意识到他们两个人不可能永远不分离,从而应当越加珍重相互之间的感情,越加柔情相待。实际上,高龄人群期望每一次性交时都达到性高潮是不现实的,这样的要求反而会抑制双方在性活动中获得乐趣。老年人的性活动乐趣更多地来自性活动的娱乐性质,而不是那种强烈的身体宣泄。因此,从某种意义上来说,老年人性活动的方式转向了另一类同样丰富多彩、具有生气的形式。

(杨华渝 徐晓阳 杨大中)


\chapter{第五章 性咨询与性治疗}

性治疗(sex therapy)是指由马斯特斯和约翰逊于1970年首创的一种行为治疗的特殊技术。它融合了前人精神和心理学家的成就,开创了行为疗法的新模式,主要应用于男女性功能障碍和性变态的治疗。在历经卡普兰(1974)等的认同和发展后,性治疗已成为当代除药物和外科手术以外的主要治疗手段。

我国性治疗的研究始于20世纪80年代初期。在吴阶平等编译的《性医学》首度把性治疗介绍到国内后,老一辈专家都从不同角度开始了对性治疗的研究,1988年10月18日国家计生委科学技术研究所等单位率先举办了“全国首届性医学培训班”,260多名学员接受了专业培养。1994年中国性学会成立并于同年在重庆市成立首个《性感集中训练实验室》,1995年薛兆英、许又新、马晓年主编的《现代性医学》及其他相关专著的出版等,都积极推动了我国性治疗研究的发展。2003年《适合中国国情的行为疗法临床研究》课题在重庆市结题,标志着我国性治疗进入一个相对规范的时期。但是由于受到传统观念和性医学尚未在医疗机构广泛开展的影响,目前性治疗在我国还处于一个有待大力发展的阶段。本章将立足于中国国情和临床适用情况,着重讨论性治疗的基础理论和实施方案。


\section{第一节 性治疗的基础理论和常用方法}

精神分析理论(psychoanalysis)由弗洛伊德(Freud S,1856—1939)于19世纪末创立,是西方最早的心理治疗理论,对其后心理治疗的发展具有非常重要的影响。该理论主要由五大体系组成:

弗洛伊德把人的心理活动分为意识、潜意识和前意识三个层次:

(1)意识(conscious):是指与语言有关、与现实相联系并能被自我意识所知觉的心理活动,如感知觉、情绪、意志、思维等。使个体保持对环境和自身状态的知觉,对人体的适应性有重要作用。

(2)潜意识(unconscious):又称无意识。是指个体无法直接感知到的那一部分心理活功,通常是不被外部现实、道德、理智所接受的各种本能冲动、需求和欲望,或明显导致精神痛苦的既往事件。如已被意识遗忘了的童年不愉快的经历,心理上的创伤等。弗洛伊德认为,无意识几乎是各种精神活动的原动力,这种本能的精神能量长期受到过分压抑是导致一些心理和行为障碍的因素之一。

(3)前意识(preconscious):介于前两者之间,有警戒并阻止潜意识的本能冲动到达意识中去的功能。如目前未被注意到或不在意识之中,但通过自己集中注意力或经他人提醒又能被带到意识区域的心理活动和过程。

精神分析理论认为人格(personality)是由本我、自我和超我三部分组成。

(1)本我(id):又称伊的或它我(兽我、幼年的我),存在于无意识深处,是人格中最原始的部分。主要是指性本能和攻击本能,其中性本能对人格的发展尤为重要。本我的心理过程是人类非理性心理活动的部分,即不遵循现实的逻辑思维和推理,可超越时空的限制,不顾后果,任意的否定,这表现在人类的梦,游戏和幻想甚至艺术创作中。

(2)自我(ego):大部分存在于意识中,小部分是无意识的。一方面自我的动力来自本我,是各种本能、冲动和欲望得以实现的执行者;另一方面自我又在超我的要求下,顺应外在的环境现实,采取社会所允许的方式指导行为,保护个体的安全。因此,自我可以说是人格的执行部门,它设法在外环境允许的情况下满足本我的欲求,使两者保持平衡,其发育状况及功能决定着个体心理健康的水平。

(3)超我(superego):类似于良心与道德,具有良知、理性等含义,大部分属于意识泛围。其特点是能按照社会法律、规范、伦理、习俗来辨明是非,分清善恶,因而能对个人的行为动机进行监督管制,使人格达到社会要求的完善程度。超我按“至善原则”(principle of ideal)行事。

弗洛伊德认为“自我”在“本我”和“超我”中间起协调作用,使两者之间保持平衡,如果二者间的矛盾冲突达到“自我”无法调节时,就会产生各种心理或精神障碍的病态行为。

人格的发展实际上是性心理的发展过程。它要经历若干不同的发展阶段(时期),在每个阶段,个体都被迫处理由内驱力和社会环境强加的限制之间的冲突中。弗洛伊德强调个人早期生活经验对人格发展的影响,他认为一个人的人格形成要经过五个时期:从出生到一岁半左右称为口腔期,主要从口腔部位的刺激中得到快感;一岁半至三岁时称为肛门期,从自身控制大小便中得到快感;三至五岁时称为性蕾期,开始注意两性之间的差别;六至十二岁时由于儿童的注意力从自身转移到外界的各种活动中,因此称为潜伏期;之后到青春期时称为生殖器性期。弗洛伊德认为,在每个时期都可能发生人格三部分的冲突,解决得不好就可能产生人格障碍或心理疾病。

梦的解析是弗洛伊德创立的精神分析学说的重要理论支柱之一。弗洛伊德认为,梦是一种合理而富有深刻含义的心理现象,是对受到压抑愿望的伪装的满足。梦与神经症中的症状都是潜意识欲望的替代性满足,有着共同的机制。因此,精神分析可以通过对梦的分析来了解患者的潜意识,为诊断神经症并为神经症的治疗提供有价值的信息。

心理防御机制(mental defense mechanism)是一个人为了保护自我、避免精神上的痛苦、缓解各种矛盾冲突以达到心理平衡而表现的一种心理反应体系。这一机制使本我得到一定的表现而不触犯超我,使行为表现被现实所接受,从而不引起自我的焦虑反应和心理矛盾,或不使心理矛盾激化。心理防御机制不良便会导致现实性或神经症性等各种焦虑反应。

行为疗法又称学习疗法,其理论基础来自实验心理学和学习理论。现代学习理论认为行为不仅指可观察到的行为动作,还包括潜意识在内的心理活动,如欲望、动机、思维、恐惧等。行为疗法就是学习理论在临床上的实际运用。

巴甫洛夫在对狗进行消化生理研究时发现有两种反射,一种是食物引起唾液分泌的自然反射,称无条件反射。另一种是伴随食物出现的刺激———铃声,经反复训练后也可单独引起狗的唾液分泌称为条件反射。进一步的研究证实,这种条件反射形成后并不是一成不变的,而是具有消退、抑制和辨别等特性。所以巴甫洛夫认为,人类的一切行为方式都是习得的,而学习就是在刺激和反应之间建立起的神经联结,并因建立起新的条件反射而改变。这种对学习的见解,至今为大多数心理学家所接受,也为行为疗法奠定了最基本的理论基础。

华生在他的代表著作《行为主义观点的心理学》一书中,把有机体应答环境的一切活动都称为行为。而行为的基本要素是刺激和反应,进而他将一切心理学的问题都纳入到刺激和反应的范围中。通过深入的研究,华生得出了两个重要的心理学定律:①频因律:指对某种刺激发生反应的次数越多,就越有可能对该刺激再度发生反应;②近因律:指某一反应对某种刺激在时间上发生得越快,就越有可能对这种刺激再度发生反应。他认为人的一切不良行为和心理疾病都是后天习得的,因此也都可以通过行为治疗而得到纠正。

桑代克在反复训练一只关进笼子的猫为获得笼子外的鱼而设法打开门的实验中得到两个基本学习定律:①练习律:实验证明刺激和反应之间的联结可以通过重复得到加强,也即练习导致熟练。②效果律:是指刺激和反应之间的联系可以通过奖励或满足而加强,反之也可因惩罚而削弱或中断。这一理论为行为重塑提供了理论依据。

斯氏认为一切刺激和反应均可视作反射。在他设计的问题箱中有一个操作器(如开关、把手、杠杆等)和一个传递强化物(如食物等)。通过训练,动物可对操作器而不是强化物作出反应,斯氏将此称为操作性条件反射。并通过进一步研究得出了行为是学习得来的,条件强化中有机体决定着它自己的需要,因而只有相应的刺激才能强化它。此外还有泛化和消退作用等。据此,斯氏创造了一套起塑造作用的矫正程序技术,为现代行为疗法提供了最直接的理论依据和方法。

示范作用(modeling)是社会学习理论(social learning theory)的基本概念。该理论认为,人们通过对具体模型(model)行为活动的观察和模仿,能够学会一种新的行为。示范作用包括四个过程:①注意阶段,学习者反复观看某一榜样,接受其中的特征性信息,以此为学习的依据;②保持阶段,观察对象的行为特征并有意无意地记住,成为日后自己行为的模型;③行动阶段,学习者表现出观察对象的特征性行为;④强化阶段,增加或减少这种特征性行为的再发生次数。由于受到奖赏的行为比受到惩罚的行为更易被模仿,因此临床上对患者进行指导时,更应发挥积极的示范作用。

认知疗法是认知心理学在临床的应用。认知心理学认为思维和情绪及行为是密切相关的,情绪和行为都受到思维和认知的制约和调节。当一个人的认知出现障碍或思维活动不正常时,就可能导致“非理性的思维和行为”,也就是说异常的情绪和行为并非直接由激发事件引起,而是由错误的认知和思维造成。那么只要纠正了他的认知障碍和异常思维,就可以使其情绪和行为恢复正常。

埃里斯(Ellis A)在认知与行为关系理论的基础上创立了著名的“A—B—C理论”。“A”是指某种激发事件(如一件事、一件东西、某个人、一种态度或行为等),“B”是指当事人对“A”的思维、认知和评价(包括正确的或错误的),“C”是指由此产生的情绪、行为或某种倾向(可能是正常的,也可能是不恰当和不正常的)。认知疗法认为“A”并不直接导致“C”,而“B”才是造成“C”的关键因素。也就是说一个人的情绪和行为障碍完全是由于自己错误的思维和认知所造成,只有改变了患者的非理性思维和认知后,不良的情绪和行为才能得到改善。临床上因此就产生了“理性情绪疗法”,即用科学的知识和思维方法去改变患者的认知活动,以达到治疗的目的。治疗医生常用的办法是:①通过启发、说服、鼓励或命令的手段使患者明白什么是正确的思维逻辑。②向患者说明导致痛苦情绪和行为障碍的真正原因是自己错误认知引起的自毁观念,治疗也必须依靠自己去纠正这种非逻辑思维造成的错误认知。③在帮助患者分析出所有的问题后,用理性思维和正确的人生哲学去教育患者。此时要特别注意患者的态度,认知的改变必须靠患者自己实现。治疗医生仅似为一个教师,整个治疗过程也就是一个再教育的过程。

贝克认为情绪障碍者有独特的认知模式,并开辟了认知-行为理论和相应的认知-行为疗法。该理论认为各种生活事件导致情绪和行为反应时要经过个体的认知中介。情绪和行为不是由事件直接引起的,而是经由个体接受、评价、赋予事件以意义才产生的。每个人的情感和行为在很大程度上是由其自身认识外部世界、处世的方式方法所决定的,也就是说一个人的想法决定了他的内心体验和反应。因此我们说,是人的认知而不是事件本身激发了人的情绪,思想和信念才是情绪状态和行为表现的原因。

心理生物学主要从事心理行为变量与生物学变量二者之间关系的研究。一方面,可以以心理和行为因素作为自变量,以生理指标为因变量,观察不同个性和行为状态下的各种生理变化(如脑电、心电、血液中激素及其代谢物的含量等);另一方面,也可以以生物干预为自变量(如损毁、电刺激、药物干预等),以心理变量为因变量,研究由脑和躯体的生理状况改变所引起的心理行为改变。心理生物学研究最突出的优点是采用了严格的实验设计、客观的测量手段和可靠的数理统计,因而能准确地揭示心身之间的相互关系。同时,由于心理生物学研究能及时地采用各种新技术,因此更具有前沿性,已是当今心身医学形成和发展的主要方向之一。

精神分析又称为深层心理学,其基本技术主要有:(1)自由联想(free association),即受治疗者在完全放松的情况下,使其内心体验及想法任意涌出,并讲出正在想什么,不必考虑是否有逻辑,是否符合道德标准,甚至是否有意义等,治疗师则可对这些资料进行分析。(2)释梦(dream interpretation),梦被理解为一种形象的语言,不仅可以通过它揭示一般情况下潜意识的心理内容,还能了解到某些被压抑或被排斥于意识之外的、在自我防御活动时才出现的心理过程和内容。治疗师需对梦境进行解释,也即运用自由联想等方法进行释梦,从而揭示梦境的真正含义,找出影响患者的潜意识活动内容。(3)阻抗(resistance),是一种无意识的心理过程,其目的是阻止受压抑的冲突意识化。因此,对阻抗进行分析并解除阻抗是精神分析治疗的中心任务之一。(4)移情(transference),受治者将对某人(如他的父母)早期的体验、态度或行为方式自觉地转移到其他人身上的心理现象称为移情。由于有些问题只有在移情中才能表现出来,而这种移情是患者没有意识到的。它使受治疗者重新经历,并在与治疗师的移情关系中重新获得处理早期未能解决的冲突的机会,使问题有可能得到积极有利的解决。因此移情也是精神分析治疗的重要环节。

由弗洛伊德创立的以精神分析疗法为代表的深层次心理治疗,以及他提出的系统人格理论对人性的解释都是具有划时代意义的。它的适应证包括各种神经症、心境障碍、各种心身疾病以及某些人格障碍等。在性治疗中,对某些早期受过创伤性性经历的患者来说,仍是重要的治疗手段。但是这种疗法耗时较长(少则半年,长者2到4年),费用较高,且需患者有较强的领悟和配合能力,同时对治疗师的要求也较高,因此在性治疗中已很少使用。不过随着分析性心理治疗,特别是短期焦点精神分析性治疗的发展,其在性治疗中的意义仍在探索中。

由于产生恐惧、焦虑的条件反射可用产生该反射同样的方法来治疗,所以系统脱敏疗法就是在产生原焦虑的境遇中,施以温柔、舒适的刺激,从而产生比较愉快的情感来取代恐惧、焦虑等不良情感的一种方法。在临床上它要求医生谨慎操纵患者的想象和体验,逐步以新的情感取代不良情感。因此,这种取代是渐进式的。医生可将引起患者焦虑的刺激按轻重列表,一般可分5个层次,每个层次有一个主题。患者从最轻的层次开始,应用放松疗法在完全松弛的情况下想象一个主题,等到不再引起焦虑后,再进入下一个层次的主题,直到引起最大反应的主题不再引起焦虑或恐惧为止。本疗法在治疗期间不仅对行为本身,还往往导致患者认知的改变。正因如此,系统脱敏疗法已被广泛的应用在性功能障碍的治疗中。

该疗法是反复多次将一种恶性刺激与患者的病态行为相结合,使患者产生一种新的条件反射,诱导患者对不良行为产生恶感并逐渐使之消退的方法。作为恶性刺激的刺激物可以是电击、药物或难闻的气味等,在性治疗中常用于性变态如恋物症中。治疗早泄的挤捏技术也属一种温和的厌恶疗法。

又称快速暴露法,即在医生的控制下使患者处于造成其焦虑、恐惧的最大刺激中,即使患者惊恐万状也不让其离开,使其快速地产生对抗性条件反射,直至焦虑、恐惧和引起焦虑、恐惧的刺激间的联结终止为止。该疗法适合患者有一定的自控能力,病程短和恐惧对象明确的病例。性治疗中偶用于有适应证的患者。

是一种按操作条件反射原理设计的行为疗法。当患者出现医生期望的行为时就给予一种标记(如卡片、筹码、红星等)以代替钱币给予奖励,出现不良行为时就不给奖励,故本疗法又称代币法。患者积累一定数量的标记后就可换得某种物质或权利,如食品、看电影、优诊待遇等。由此激发患者为了获得这些物资或权利而改正不良的行为。这种疗法常用于治疗某些有不良行为的儿童和建立精神患者在病房的生活制度等。

(1)注意和采集病史相结合:应通过病史的采集和心理学检查,找到患者心理和行为失调的特征和由此引起的相应后果,进而尽可能了解引起这种心理失调行为的时间和决定因素,如环境、个人因素、社会因素等。由于这些因素中可能包含患者的个人隐私,因此不能操之过急,有时需要患者多次门诊随访才能搞清问题之所在,而这是开展行为治疗的前提条件,在此基础上才能制定出行为疗法的具体方案。

(2)应考虑患者的经济承受能力和时间:虽然缩短疗程和减少费用是医生和患者共同追求的目标,但行为疗法毕竟是一种心理治疗,要按一定的疗程来进行。因此要使患者打消急于求成的念头,作好必要的时间和经济准备,这一点在中国的患者中显得尤其重要。

(3)行为治疗强调控制刺激和环境,并直接面对症状,相对不那么注重患者的认知和意识反应,和传统的精神分析疗法的区别开来。治疗中医生起主导作用,因此从事行为治疗的医生必须经过专门训练,并对医生就有关学科的专业知识提出了较高的要求。

(4)应根据个案的不同情况设计行为疗法的具体方案。由于每个患者心理失调行为的特征、症状和决定因素各异,选用的治疗方法也就不一样了。即使是同一方法,根据患者的具体情况在一些实施的操作技巧上也应有所不同。所以绝不能把行为治疗作为一种固定的模式来机械地照搬。因此,有针对性地设计好行为疗法的具体方案,是取得高疗效的关键因素之一。

由学习理论造就出来的行为治疗方法在性治疗学中的价值是无法估量的,它将治疗集中在特定行为可改变的机制上,而不是集中在普遍的行为模式上。这种可改变的机制表现为通过强化和条件反射的形式使行为实际获得或受到排除,临床上它通过系统脱敏和放松等疗法而具有很强的操作性。医生可通过训练患者体验性活动中产生的动情乐趣,及放松紧张情绪来逐步消除对性行为本身的紧张或焦虑,从而使性功能障碍的疗效有了显著的提高。但是,它也不是完美无缺的,过分依赖行为治疗往往会忽视性功能障碍中更深层的问题,这也是少部分患者应用行为疗法效果不佳的主要原因。而心理分析疗法又恰能揭示性功能障碍的深层问题,如以恋父或恋母情结为代表的潜意识障碍等。因此,两种疗法都是不可替代且相互补充的。众多的心理学家仍然把行为治疗纳入心理动力学范围内,认为它只是治疗手段的一种发展,和心理分析无冲突可言。

不过在临床实践中,由于行为疗法具有令人信服的较高疗效,且所需时间和费用也相对较低,所以它成为世界各国性治疗机构应用最广泛的性治疗手段。

1.埃里斯的“A—B—C理论”认为情绪来自思考,所以改变情绪或行为要从改变思考着手,从而创立了“合理情绪疗法”(rational-emotive therapy)。贝克则归纳了认知过程中常见的几种歪曲形式,如任意推断、选择性概括、过度引申、夸大或缩小以及“全或无”的思维等,这在性治疗学中有很高的指导意义。大多数心理性性功能障碍患者都存在着认知歪曲方面的问题,如男科学领域中勃起功能障碍、早泄的病例,本来男性一生中偶尔出现一二次勃起失败或早泄完全属正常现象,但由于当事人对这种现象产生错误的思维和认知,误以为自己性功能出现了问题,由此产生的紧张情绪就可能诱发第二次勃起失败或早泄,这种恶性循环是心理性性功能障碍的常见原因。在女性学领域中,认知疗法也有广泛的指导意义。如同样对离婚这样一个激发事件,有的妻子因为解脱了一桩痛苦的婚姻而对未来充满了希望,另外一些却因此而受到严重伤害,并将抑郁的情绪带进下一次婚姻中,甚至诱发性功能障碍。其他如因阳具崇拜和精液宝贵等错误观念而引起性应答或高潮障碍的患者,其实也是一种错误认知的结果。类似情况,通过认知疗法进行耐心的、科学的性生理解剖知识和理性思维的教育,常能收到很好的效果。

在对患者进行认知治疗时,对具体方法和时间的选择十分重要,治疗师要根据患者的具体情况来作出决定。对患者进行认知重建一般在治疗师同患者进行早期性咨询时就已经开始了,其后在整个治疗进程中都要不断强化患者的认知。这个过程需要不断地对患者建立在不合理或错误逻辑基础上的信念进行分析和反驳。同时把认知重建和性治疗的其他方法结合起来效果会更加。

2.认知疗法的适应证和局限性:认知疗法可广泛应用于因抑郁或焦虑引发的各种性功能障碍,以及情绪问题、婚姻家庭问题等方面的治疗,因此在国外的心理治疗中,约有60\%的受治者接受认知治疗,而在我国性治疗的门诊中,其应用率在80\%以上。该疗法经过30多年的发展已日趋成熟,它关注受治者的当前情况,治疗目标包括缓解症状、解决患者面临的紧迫问题并教给防止复发的方法,耗时较少。但该疗法无论是在理论上还是操作上都存在局限性:如研究方法不够科学,在理论形成过程中更多地依靠推理,认知因素与心理障碍之间因果关系的推断缺乏科学依据,以及在实际操作中对其疗效的评价尚存在某些不足等。

夫妇共同性治疗是马斯特斯和约翰逊(1970)创立的一种综合性的性治疗技术,与传统的心理分析疗法完全相反,它的基础是行为学和学习理论。这一疗法的宗旨是改变过去形成的错误的性行为方式,性治疗就是性科学的再教育过程,行为矫正是建立在学习理论基础上的。

马斯特斯和约翰逊强调,性功能是婚姻关系的一部分,因此必须在婚姻关系的基础上来考察某一方的性功能。性功能障碍可能是以婚姻关系紧张或破裂为因,也可能为果,只有夫妇共同参与治疗,把婚姻关系作为一个整体来处理,才可能取得疗效和巩固疗效。因此,夫妇共同疗法不接受夫妇某一方单独求诊,这与以前的个别心理治疗有所不同。为了更深入地了解婚姻关系,强调男女治疗学家共同参与治疗。于是他们创立了合作治疗小组的形式,即由男女两位治疗师组成小组,与患者夫妇一起参与整个治疗过程。

他们认为性功能障碍本质上是一种性操作焦虑,而解决操作焦虑应从认知和行为两个方面入手。其治疗过程的第一阶段称为圆桌会议,即由男女治疗师与夫妇一起讨论他们所面临的性问题。这既是了解病情的过程,也是给予性指导、纠正性观念、重建夫妇性交流和调整婚姻关系的过程。第二阶段是行为治疗,由两类技术组成:即性感集中训练和其他特殊治疗技术。性感集中训练广泛适用于各类性功能障碍,是基本的非针对性的训练课程,旨在帮助患者从“旁观心理”、操作焦虑中摆脱出来,逐步学会全身心地投入到性生活的性前爱抚活动中,尽情体验性的愉悦。特殊技术指专门适用于某种障碍的技术,如用于早泄治疗的挤捏术和“停—动—停”技术,用于治疗射精障碍的“配偶手淫法”等。在这些病例中,妻子对丈夫偶尔出现勃起失败或早泄的不理解甚至抱怨常常是加重丈夫紧张、焦虑和负疚感的重要因素。所以性活动是由夫妇共同参与的,错误的认知所导致的后果不仅来自本身,还可因配偶的错误认知而加重,因此治疗需对夫妇双方同时进行。

夫妇共同性治疗的治疗的目的不是创造新的性功能,而是帮助患者恢复原有的、被非自然化了的性功能。由于夫妇共同性治疗主要是对夫妻在家庭环境下进行各种行为训练的指导,所以更容易取得较好的疗效。

夫妇共同疗法自创立以来,迅速传播至全世界,是目前国际上最流行的性治疗技术之一。这一技术与行为治疗技术一起,对我国现代性治疗学产生了深远的影响。

由美国纽约医院的精神病学专家卡普兰博士创立。1974年她出版的《新性治疗学》一书是继马斯特斯和约翰逊之后对性治疗学的又一重大贡献。她把心理治疗和行为治疗有机地结合起来,创立了新性治疗理论。

传统的心理分析和婚姻治疗专家认为,性的问题是潜在的内心冲突或人际关系破坏的一种表现形式。因此,治疗时首先考虑的不是性症状本身,而是强调解决压抑在患者的内心深处或人际关系上的困扰。治疗目标也不仅仅是性问题的改善,还需解决患者潜意识中的冲突和根本改善婚姻状况后才能结束治疗。新性治疗理论的一大特点就是将治疗局限于一个有限的目标,即以解除性功能障碍的症状为目的,而不随意扩大解决潜在的内心或婚姻冲突。只有在治疗中遇到顽固的防御反应,必须用心理动力学改变才能解释时,才考虑作深层病因的分析和治疗。

另一方面,卡普兰认为性治疗应在心理动力学概念的体制里进行。单靠性知识教育、性咨询和包括性感集中训练在内的行为疗法来达到治疗目的是远远不够的,还必须以心理动力学作为心理治疗的基础,兼顾直接和深远的原因。因此,对每对夫妇都要注意有无内心冲突、无意识阻抗和婚姻关系恶化等因素的存在,并在治疗过程中予以充分的考虑和妥善对待。

在这种理论的指导下,新性治疗就是把行为疗法的性体验训练和心理分析有机地结合起来,以取得优势互补。当患者快速和直接投身性体验的训练时,完全可以使相当部分患者的性功能障碍得到明显和持久的改善。而对另外一些有深层心理障碍的患者却难以奏效,但通过这种训练仍然比单纯的晤谈能更快、更鲜明地揭示出婚姻和个人内心深处的冲突,并且这些性治疗的作业还为迅速解决这些深层冲突提供了机会。所以,二者是可以紧密结合的。因此新性治疗的治疗方案强调因人而异,更富针对性。在制定治疗计划时,不强调双重治疗组,也不需患者夫妇集中两周,只需每周晤谈1~2次,以便对患者完成家庭性作业进行指导。至于性作业的内容、顺序则因人而异。对在晤谈或性作业的训练过程中发现有内心冲突的患者,应随时介入心理分析治疗。在这种以心理动力学为基础,将系统的性体验训练和心理治疗相结合的基本原则的指导下,根据患者的具体情况,可以灵活多变地设计出各种行之有效的治疗计划。正因如此,卡普兰的新性治疗理论比以往的性治疗更有创造性,也是继马斯特斯和约翰逊之后影响最大的一个性治疗学派。卡普兰(1997)曾报导新性治疗法1000例的疗效,治愈63\%,改善7\%,无效和失访30\%。

新性治疗理论的许多观念已为我国的性治疗学家所接受。医师针对患者的具体情况,灵活多变地设计治疗计划,而不是把以性感集中训练为代表的行为疗法作为一种机械的训练程序。同时根据需要有机结合心理分析治疗以解决患者深层的问题,已成为当代大多数性治疗学家一致推崇的治疗原则。

可贵的是卡普兰的新性治疗法与时俱进,先后把阴茎海绵体注射和口服药物西地那非结合进来,目的就是最大限度地、更灵活、更有效地帮助患者解决性问题。

该模式是国际上最新的治疗模式,简便易行,步骤清晰。在性治疗的临床工作中,可依据患者的具体情况和病情的轻重,逐级采用不同的治疗形式和内容。

提供承诺。对许多问题仅仅提供一种保证,即可收到效果,也即性咨询的开始阶段,目的在于消除患者的紧张、焦虑情绪,建立医患互信,给患者一个释放情绪的空间。例如许多青年男子常怀疑自己阴茎的长度,经检查后,发现正常,即可给予肯定的承诺。

有限信息。有些患者的问题仅仅需要提供一定的信息,即可使其放下心来。例如告知患者阴茎的长短、粗细在性活动中并非起决定性因素,以及一些规范、健康的基础性知识,如性器官只要清洗后并不特别肮脏,只要身体状况允许每周2~3次的性生活不会因精液的丧失而“肾虚”等。

特殊建议。有些患者的问题需要提供一些特殊的建议,即可使其放下心来。例如包茎的问题可以通过很简单和安全的手术解决。对另外一些患者则需提供有关性技巧的指导,如:强调性前爱抚的重要性,自慰的合理性及方法,性幻想的合理性及意义,盆底PC肌训练方法(凯格氏法)等。

强化治疗。大多数性的问题,特别是来医院就诊的患者需要接受系统的心理治疗,这是一个强化的过程。临床上进入第四级的患者多数需进行夫妻共同参与的性感集中训练治疗。

催眠疗法是运用心理暗示的方法,使患者进入一种特殊的意识状态———催眠状态,进而进行心理治疗的疗法。

从15~16世纪西方医生提出的“磁气学说”开始,对催眠术的应用和研究已有数百年的历史。随着民族、信仰、文化传统及科技发展水平的改变,各国的心理学家创造出了丰富多彩的催眠技术。如以语言暗示为主的催眠法:其中又可分为语言暗示加视觉刺激,语言暗示加听觉刺激以及语言暗示加皮肤感觉刺激等方法。快速催眠疗法。药物暗示催眠疗法:又分非麻醉药物的暗示催眠和麻醉药物的暗示催眠等。由于在催眠状态下人的分析、认识和判断能力下降,在治疗医生暗示性语言或刺激的诱导下,潜意识中的经验能够再现,压抑的感情可获得释放,内心深层的冲突也就可以表露出来。此时治疗医生对被催眠者已处于一种支配性地位,所作的安慰、劝告和疏导对患者来说将是一种难以抗拒的力量,从而达到排解患者内心冲突和治疗疾病的目的。

催眠疗法曾广泛应用于精神性疾病、神经系统疾病和各种心身疾病的治疗中。在性治疗中常用于以下情况:

(1)性欲障碍:通过催眠可了解到患者内心潜在的心理矛盾。此时若加以相应的疏导,同时利用患者的顺应现象,指导患者反复体验和想象愉快的性经历,控制有碍性反应的心理活动,常可取得很好的疗效。所以通过催眠既可查找病因又可进行有效的治疗。

(2)性唤起障碍:催眠术可增强患者的信心,缓解紧张、焦虑的情绪,因此可作为行为疗法的辅助治疗手段。

(3)性高潮障碍:催眠疗法有利于探知引起高潮障碍的病因和帮助女患者把她们自由的性幻想重新定位到自己的丈夫身上。

(4)对阴道痉挛和性交疼痛的治疗:对于单纯心理性的阴道痉挛和性交疼痛,催眠术可有效地消除患者对性交的恐惧和紧张心理,因此有很高的疗效,是此类疾病最有价值的疗法之一。

(5)性犯罪所致的创伤后紧张综合征:女性在遭到乱伦、强奸等性犯罪后,常表现出性欲、性兴奋和高潮障碍,临床上称之为创伤后紧张综合征。在这种情况下,患者往往会不自觉重温创伤的经历,在反复体验内心痛苦的同时把自己封闭起来,疏远他人甚至整个社会。催眠术将有助于她们从痛苦经历和封闭状态中解脱出来。虽然心灵的创伤仍将存在,但患者将学会对自己情绪的控制和逐步恢复对自身的性感觉。

催眠术在女性性功能障碍治疗学中有相应的价值是肯定的。但由于对施术者的要求较高,而且要耗费较大的人力物力,因此在我国的性诊疗机构中应用并不普遍,多数是在设有精神科或心理卫生科的医院开展。

一个人若长期处于焦虑、紧张或应激状态下,就有可能导致各种身心疾病。但若在引起焦虑、紧张刺激存在的同时,加入一个能抑制焦虑、紧张的因素,就能起到削弱这种刺激、焦虑和紧张之间联系的作用。而松弛就恰能对焦虑和紧张起到条件性的抑制作用。松弛疗法在性治疗的临床应用上可分为紧张与松弛和直接松弛两种方法。

(1)紧张与松弛疗法:由于紧张或焦虑时必定伴有肌肉紧张,本疗法就是通过训练患者体验身体各部位肌肉先紧张后放松的轻松感觉,来打断焦虑—肌肉紧张—进一步焦虑的恶性循环,以达到治疗的目的。

(2)直接松弛法:主要是在平缓、轻松和舒适的前提下,训练患者去体会身体内部的各种感受,或回想、感受过去令其轻松、愉快和喜悦的事情,以达到削弱不良刺激并抑制焦虑情绪的目的。

集体疗法在性治疗方面的运用称为集体性治疗。它包括同病种组、同性别组和混合组三种形式。主要是为心理性功能障碍患者在调整心理情绪时提供一种有效的支持系统。

集中性治疗在女性学中最常用于性高潮障碍的治疗,也可用于性欲或性唤起障碍的妇女。此类患者多为离婚、丧偶或仅有性伴侣的单身女性。治疗时一般以6~8人为一组,由医院组织她们定期集中,每次活动约2小时。在活动中主要由患者自己交流性经验和发泄内心的抑郁,治疗医生只起诱导和回答有关疑问的作用。患者可对自己存在的性问题进行评价并在医生的指导下决定自己的治疗计划。由于是在一种集体讨论的民主气氛中进行,患者将感到自己是“决策者”,并因此而增强实施治疗计划的信心和积极性。

也可组织几位已开始执行性感集中训练的妻子定期聚会,在一起共同讨论在家庭作业时遇到的困难和解决办法。通过一些成功病例的自述,对在家庭训练中信心不足的患者会起到很好的鼓励和促进作用。

在中国施行集体性治疗的普遍问题仍受传统文化的影响,患者不愿在众人面前公开讨论自己的性问题,因此集体性治疗并不是适合每一位单身女患者。治疗医生应选择一些性格比较外向和善于交流的患者,在作出参加集体性治疗的建议后要充分尊重患者的意见,或让她“见习”一次,以调动她的兴趣。参加的人数也可减为3~4人,以减轻面对众多陌生人的压力。如何使集体性治疗的讨论生动有趣,则是治疗医生在每次活动前要考虑的问题。应特别注意若非本人意愿,应避免触及患者的隐私或敏感的婚姻关系及社会问题,而将讨论集中在性感受和压抑感情的释放上。让参加这种集体讨论的患者既能抒发和排解个人情绪的压抑,又能在回家进行个体或夫妇共同训练时,得到有效的心理支持,也就达到了集体性治疗的目的。

用手或其他器具刺激自己的性器官从而获得性快感,这是人类与生俱来的一种生理能力,就如多数男孩都有手淫的经历。临床上本疗法主要应用于女性性快感缺乏或性高潮障碍。该疗法包括指导患者和夫妇正确认识自我刺激的合理性及女性的性敏感区,教会正确的刺激方法,必要时加用润滑剂和振荡器辅助患者体验性快感。在通过自我刺激取得快感后,再鼓励丈夫加入治疗中,最终建立起满意的夫妻性生活。Lopiccolo(1972)报告该疗法对原发性女性性高潮障碍的疗效接近100\%。

这是西方少数性学家为解决在配偶不愿合作或患者已经离异的情况下,为执行夫妇共同参与治疗计划而设计的一种疗法———代理性伴侣疗法(Surrogate Sexual Partners),也即由医院雇请一位临时的“丈夫”或“妻子”来协助完成治疗。马斯特斯和约翰逊雇用的代理性伴侣均是医生或护士中的自愿者,由于他们具有一定的专业知识,更能有效领会和贯彻治疗医生的意图,故也取得了一定的疗效。但即使在性观念相对开放,甚至妓女成为一种合法职业的西方国家,也同样要面临法律和伦理道德的问题。很难想象一个不愿配合妻子治疗的丈夫会同意妻子和另一个男人发生性关系———即使是为了治疗。另外,就算患者通过代理性伴侣疗法解决了性功能障碍,却往往无法把所学到的东西应用到夫妇实际的性生活中去,特别是对促进夫妇双方在性活动中的合作精神并无帮助。因此,即使在西方性学界也遭到广泛的批评,目前已很少采用。代理性伴侣目前已改称为身体工作治疗师。

至于在中国,由于国情和有关法律的明确规定,代替性伴侣是绝对不允许的。因此各性治疗机构要严格把关,以确保我国性学研究的健康发展。对于配偶不愿合作要分析其具体原因,医生可采取配偶先不到场而仅仅通过电话咨询的方式进行交流,以笔者的经验,只要配偶愿意讨论有关性的问题,那么大多数均能被说服配合丈夫或妻子的治疗。因为性治疗的结局,必将是双方受益。


\section{第二节 性咨询及性治疗的病史采集和评估}

性咨询及性治疗室的规划与配置应注意到患者特殊的心理需求,同时应具备性咨询、性教育、性心理和常见器质性疾病诊疗的能力。因此,性诊疗室在医院门诊应作独立配置,最好具有单独的通道和相对安静的环境,以便尽可能地减少环境因素对患者的干扰。

宜稍宽大。接诊护士态度应亲切、诚挚,可简短回答询问和介绍有关情况,特别强调严谨的工作作风和保密制度。尽可能满足初诊患者的各种心理需求,如要求知名专家接诊,要求女医生接诊,要求隐私保密以及在时间安排和治疗经费等方面的希望等。待诊室内可提供一些性生理、解剖、心理和常见性生殖疾病的科普读物供患者读阅。另外,待诊室可以允许丈夫或亲友陪伴,目的在于为患者在接触治疗医生前,提供一个舒缓情绪的过渡空间。

是性治疗医生采集病史,进行性咨询、性教育和一般心理治疗的地方。因此应相对隔音和避免打扰,包括护士在内的其他医务人员若无特殊必要亦不宜随便进出。诊断室最好布置成客厅式而不是一般医院的诊断室式,尽量创造出一种轻松、舒缓、不会令人紧张的和谐环境,以利于患者消除疑虑并在短时间里和医生建立起相互信任和理解的关系。

基本设施同男科或妇科检查室。此外应配备身高体重测量仪,测量用的硬尺、软尺,观察用放大镜以及用于微生物病原学检测留取标本的各种条件。在女性学专用设备中妇科B超是必备的,因其对常见妇科疾病和监测卵泡发育情况是不可缺少的检查手段。阴道镜也是必备设备,最好还能配备阴道压力测试仪。

依据其功能又可分为:

应按护理的要求配备,并相对无菌。

用于阴道冲洗、上药等用。

如妇科微波、激光及波姆光治疗仪,用于阴道炎、宫颈糜烂、尖锐湿疣及盆腔炎等的治疗,还可配备乳腺疾病的诊疗仪等。

是开展行为疗法的专用场所,所以也是一种心理治疗室。该室应设在相对隐蔽和不受外界干扰的地方。室内光线应柔和隐蔽、色调淡雅,温度应在24~28℃间,尽可能营造一种温柔、放松的氛围。

设施首先是具备一张较宽大的床,头侧应有软靠背以利患者半坐卧位。应配备一面手持的小镜子,以供女性观察自身的会阴部。面对床头需配置电视机、DVD和音响设备,以便患者(或夫妇)观看有关行为疗法的影像教材。此外,还需配备一间稍宽大的洗浴室,除淋浴设施外,应专门安放一面150cm×50cm大小的立式镜子,以供患者对自身裸体的观察和欣赏。

另外,性感集中训练室内至少需从两个角度安置摄像监视器,并与旁边的治疗师监控室相连,治疗师可隔墙观察,并通过麦克风对患者进行指导。除非有必要并获得治疗者的同意,治疗师一般不进入性感集中训练室,这样可减少因医生直接进入室内而对患者心理和情绪产生的干扰。特别是对单个异性患者尤需慎重,必须有同性别医护人员的陪同,这也是在我国开展性感集中训练工作时一定要遵循的原则之一。

性治疗是一个专业性很强的工作,对性治疗师的要求主要包括专业知识、心理素质和职业道德这三个方面。

对性治疗师的知识层面要求十分广泛,除与泌尿外科、妇产科、心血管内科、内分泌科等相关的性生理、病理、解剖等医学知识外,还需要掌握性心理、性伦理、性法学、性社会学以及婚姻家庭治疗学等相关的科学知识,尤其是需要熟练地掌握和运用与性功能障碍和性心理障碍有关的基础知识,这样才能对患者的性问题进行全面的评估,以给予最恰当的治疗。

对性治疗师心理素质的要求包括以下三个方面。

(1)观察能力:在人际沟通过程中,非言语方式往往大于言语方式所传达的信息量,因此性治疗师需要具备良好的观察能力。比如患者的面部表情、目光接触、肢体动作、语音、语调、语速等非言语信息,均对心理评估有着重要的意义,这也是强调性治疗师观察能力的重要原因。

(2)人际沟通能力:性治疗主要依靠治疗师和患者的人际交往完成。如果治疗师缺乏沟通能力,就可能在评估和治疗的过程中遗漏某些重要信息,影响性治疗的疗效。良好的沟通能力取决于治疗师对待患者的态度以及沟通技巧的训练和实践经验的积累。性治疗师应当平等、尊重、诚恳、热情地对待患者,要站在患者的角度去体会、理解和分享他们的情感体验,让患者感到被理解和接纳。患者只有感到安全、舒适时才会敞开心扉,畅所欲言。

(3)自我认识能力:性治疗师应对自己有比较客观、明确的认识,清楚了解自己的价值取向、道德标准、宗教信仰、情绪状态、兴趣爱好等因素。当患者与治疗师的观点不一致时,治疗师应尊重他们,并且注意不要把自己的价值观念强加给他们。只有具备良好的自我认识能力,性治疗师才有可能对患者有一个客观的认识和评价,才能够做到真诚地接纳患者,提高性治疗的疗效。

性治疗涉及的都是患者绝对隐私的东西,因此性治疗师首先要尊重患者的隐私,特别是尊重异性患者。在性治疗的过程中,治疗师有可能会受到来自患者的性吸引,这就更加需要治疗师具备良好的自我控制能力,严格遵守在性治疗过程中的一切道德规范,必要时可建议患者转诊以防止诸如移情现象的发生。

随着我国物质和文化生活水平的不断提高,越来越多的人希望改善性生活质量或解决他们的性问题,都积极向医生咨询以寻求帮助。这时医生所起的作用就是接受患者倾诉和回答相关性的问题。通常性咨询的内容包括与配偶在性交频率、偏好和感受方面的冲突,各种性功能障碍,婚姻情感方面的不和谐,对子女的性教育方式,以及对避孕、不育和防治性传播疾病的知识等。

由于性咨询往往是性治疗的前期工作,所以此时的医生只需掌握心理疗法和性治疗中的主要内容和常用方法,而不必要求他们具备处理所有类型的性问题的能力。因此即使在非专业的性治疗机构如心理诊所等都可进行性咨询。当然在初步评价患者与性有关的问题之后,医生应客观判断自己是否有能力胜任和有效处理这些问题,若有困难就应及时转诊到有经验的性治疗机构。

由于我国医科院校尚未广泛开展性医学专科教育,因此多数医生未受过心理治疗、婚姻咨询或性治疗方面的专业培训,但他们却有能力、有义务成为一名健康性知识的宣传教育者。如对男性应破除“缩阴症恐惧”以及“阴茎大小决定了性能力”等谬误观念,而在女性除宣传经期、孕期的性保健知识外,还应特别强调阴蒂、G点等性敏感区功能的重要性和适度自慰的合理性等。对于有经验的治疗师而言,就可根据患者的具体情况有针对性地作出更深入和广泛的性知识和性技巧的教育。

要想成为一个心理治疗专家就必须具备临床心理学、精神病学的专业知识,在处理婚姻和性问题等内心冲突时,常常要上溯到早期生活中发生的事,此类经历往往是创伤性的,如乱伦(事实上的或幻想中的)、强奸等。也可能是对父母中异性一方的过度依恋(恋父或恋母情结)而产生的罪恶感,这些均需要深层次的精神分析的帮助。因此,对一个心理治疗专家来说,必须具备较高的心理和精神病学专业素养和实践经验,才能运用心理动力学的手段来实施治疗,以取得良好的效果。

在开始性治疗之前,应区分婚姻关系失调和性问题的因果关系。若性问题是引起婚姻关系不和谐的原因,那么性治疗痊愈后婚姻状况自然就会改善。如果性问题是婚姻关系不和谐而引起的,就必须先处理婚姻关系。婚姻治疗在心理治疗学中已形成一门新兴的学科,因此处理婚姻问题也需经过专门的训练。对多数医务人员而言,若缺乏这方面的专业训练,往往需求助于有经验的婚姻治疗学家。

性治疗是指包括个人的性问题和婚姻治疗技巧在内的心理疗法的统称,因此对性治疗师的要求是比较高的。除应具备较全面的专业知识和实践技能外,还应正确分析自身的性观念、感受和对某些性行为所持的道德评判。由于患者对性治疗师的态度十分敏感,性治疗师的某些固有观念可能对治疗产生消极的破坏作用。因此,性治疗师必须不断地提高自己,在使性治疗技巧日臻成熟的同时,通过不断地自身剖析来去除某些固有的偏执观念,才能更坦然地面对各种性行为,处理好患者所面临的各种性问题。

性咨询(sex counseling)是指性咨询师给那些对性有困惑或苦恼的来访者以指导和帮助的过程。通过咨询,性咨询师能够帮助来访者解决其对性问题的困惑或苦恼,减轻其病痛,从而促进其身心健康的发展。性咨询在我国尤其强调性知识的教育,因为相当一部分来访者的问题仅仅是由于缺乏正确的性知识所造成的。因此,给予必要的性知识教育,帮助他们学会放松与沟通,树立自信和消除自卑,以及改变不良的性行为方式等,是性咨询的关键。

(1)门诊咨询:在综合医院、精神卫生中心、心理诊所和卫生保健部门均可设置性咨询门诊,接待来访者。这种形式可与来访者进行面对面的对话,所以咨询能够深入,一般效果较好。

(2)信函咨询:多为外地性咨询者。这种方式可以克服“羞于启齿”的弊端,但也同样存在不能面谈的诸多不便,难以深入探讨问题和给予具体指导。

(3)电话咨询:对一些不愿面谈和怕暴露身份的人,或因交通不便无法到门诊咨询的人,电话咨询也是一种及时有效的咨询方式。

(4)专题咨询:专家在报刊、杂志、电台、电视台等进行性专题讨论和答疑。这种形式的咨询,还具有对健康性知识进行宣传的科普教育作用。

(5)网上咨询:现在许多网站开始提供网上性咨询服务,通过论坛或在线咨询的形式解答网友的性困惑。随着电脑网络的普及,这一咨询方式有着广阔的发展前景。北京回龙观医院邸晓兰应用这种网上咨询形式进行心理治疗,取得了很好的效果。有的专家则应用博客(blog)来宣传健康性知识和解难答疑,如我国马晓年的博客就拥有很高的点击率。

(1)理解与支持:来访者希望通过专业人士的咨询和帮助,解决他们的性问题。他们对咨询师抱有很大的希望,同时也可能存在某些担忧和疑虑,担心咨询师不能理解他们的苦衷,也担心不能解决他们的问题。因此,咨询师要热情诚恳地接待来访者,向他们讲明性咨询的基本精神和原则,鼓励他们消除顾虑、畅所欲言,咨询师还必须给予他们必要的心理支持。

(2)耐心倾听,鼓励疏泄:倾听是性咨询的重要步骤,只有认真倾听,才能了解来访者存在的问题,而且认真倾听本身就能缓解来访者的心理压力,有一定的治疗意义。在倾听的过程中,咨询师一定要专注、耐心,不要随意打断对方的谈话,并及时给以鼓励,使其能够宣泄自己内心的困惑和痛苦。

(3)解释得当,提供知识:来访者的痛苦常常是因为性知识缺乏造成的,咨询师可以通过自己掌握的性科学专业知识来解除或减轻来访者对性的困惑和烦恼。但在来访者的问题没有明确之前,不要轻易回答。咨询师的解释要言之有理、分寸恰当,不要简单草率地敷衍来访者,切忌发表模棱两可、没有根据的咨询意见。

(4)尊重来访者,严守秘密:性咨询的内容都涉及来访者的隐私,很多人不希望被其他人知道。因此咨询师应尊重来访者的意愿,对与来访者会谈的内容严格保密,不得随便谈论。在未得到来访者允许的情况下,也不得随便透露给他/她的家人。这也是性咨询业健康发展的重要保证。

良好的医患关系是性咨询成功的关键因素。咨询师要满腔热情,同情关心来访者,还要有精湛的技术和高尚的职业道德。熟练运用倾听和晤谈技巧对尽快建立相互信任的医患关系是十分必要的。

进一步了解与核实来访者的具体问题以及心理社会背景,查清问题的来龙去脉,评定症状的严重程度。要注意三个区别:其一是区别属于一般的性问题还是病理的性问题;其二是区别由心理因素所致还是器质性因素所致;其三是区别来访者有无主动求治的愿望。在明确了这三点之后才能着手制订和实施咨询计划。

性咨询需要确定咨询目标并制订计划和策略以达到目标。要做到这一点,必须详尽地收集可靠的材料,经过分析比较,找出关键问题。为了帮助来访者分析和认识问题,常用的方法有询问、提出问题并要求来访者自我解释、对来访者的诉说进行准确的有重点的复述,提醒来访者注意可能与其有关但易被忽略的因素等。咨询的目标要与来访者协商确定,借此可以调动来访者的积极性。

这一阶段是性咨询中最有影响的环节。在这一阶段,咨询师根据来访者的具体问题确定方案,通过分析、解释、指导、训练等方式来影响来访者。来访者积极参与这一活动,开始理解、领悟、模仿、学习新的认识方式和行为方式,向目标方向取得积极的进步。

对于一般的性问题,主要是对来访者进行有关性生理、心理知识教育和行为指导,咨询后的效果是比较明显的,如对手淫、遗精等问题的疑惑。有少数问题,如老年期性问题、性病来访者的心理压力等,则需要家庭及社会的配合,并需要帮助来访者协调人际关系。

患性功能障碍的来访者除了一般咨询外,还需要进行特殊的性治疗。当然这要考虑他们本身有无求治的要求。如果来访者能够配合治疗,适当选用一些简单的性治疗方法就能取得一定效果。对较重的性功能障碍或性心理障碍的来访者的处理,已超出了一般性咨询的范围而属性治疗的范畴。因此咨询师如果感到处理困难,可以建议来访者找有经验的性治疗师进行治疗。

在这方面我国和西方发达国家存在很大差距,不少性功能障碍患者就仅仅是因为性知识缺乏而造成的。所以,性知识教育也是性治疗的重要组成部分,而且有些知识和观念是患者在开始其他治疗前必须获知和具备的。

(1)树立性是以大脑性中枢为中心,而以皮肤为终末器官的观念。这是在否定了几千年来认为“性的中心是生殖器”的错误观念后,当代性学的一大发展。在广义的性观念中,生殖器仅是最敏感的部位之一,而在生殖器之外还存在其他的性敏感区,性行为也并不仅仅限于生殖器的接触。这就为性交前、后爱抚的必要性和丰富多彩的性技巧提供了理论基础。在这种观念的指导下,人类性生活的质量有了很大的提高。遗憾的是,在我国仍有大部分夫妇认为生殖器接触是性活动的唯一方式,加之由男性主宰性活动的传统观念,使众多中国妇女失去了应有的性乐趣。因此,在性知识教育中首先应树立起正确的性观念。

(2)了解男、女性心理的特点:婚姻虽使一对男女结合在一起,但彼此对男女间存在的性心理、生理差异都往往不了解。如在感觉器官对性刺激的敏感性上,男性对视觉比较敏感,对与性有关的影视、照片、图画和外露的女性酮体等有较强的性刺激感受,这在临床上已作为判断器质性或心理性勃起功能障碍的手段之一。而女性则更侧重于听觉,缠绵的情话,示爱的语言。再如对性的感受上,男性侧重于性器官的刺激,往往以追求性交和射精为目的,所以有嫖妓现象的存在。而女性更注重感情,做爱除生理上的需求外,更是一种情感表达的方式。因此,女性的性唤起必须具有相应的感情基础,每次做爱丈夫都应先培养起应有的情调,尽可能满足妻子情感上的需求。相反,丈夫若只顾满足自己性器官刺激的需要而忽视了妻子的正当情感需求,就有可能构成导致妻子性功能障碍的原因。

(3)纠正一些常见的错误认识:①性器官肮脏论:女性和男性不一样,除去白带、月经、尿液等异味造成的生物学因素外,还有来自童年与小男孩相比缺少阳具而产生被“阉割”的自悲,所以很少有女人会自觉地去欣赏自己的外生殖器。事实上清洁后的女性外阴和身体的其他部位并无什么两样,阴道因为有自洁作用更是相对干净的。②手淫有害论:男性有手淫经历是普遍的现象,但我国女性有手淫经历的仍是少数。从性治疗学角度来看,女性手淫比男性更富有积极的意义。因此,从某种意义上讲它是每个已婚妇女都应掌握的技巧。特别是在进行性治疗前,必须彻底纠正手淫有害的错误观念。③阳具崇拜:这是源自原始人类对性崇拜的一种延伸。在女性,由于自身缺少阳具,所以对丈夫的阳具十分珍视,虽然在正常解剖范围内的阴茎长短和粗细在性生理上并无特殊意义,但对部分妇女却有着心理上的影响。这种抱怨丈夫阴茎短小的现象在临床上绝非少见,几乎都发生在有性经验的女性身上,有时单凭性知识教育很难说服此类患者,必须找出深层次的心理问题并通过行为疗法,调整好性交姿势和技巧后才能解决。如一离婚后再嫁的妇女来院就诊的主诉即是丈夫阴茎短小,经检查其丈夫阴茎完全正常。经深入晤谈后了解到该妇女还存在着对前夫感情上明显的眷恋,在赞美前夫身材魁梧、性欲强烈、阴茎粗壮、性能力超强的同时,对现丈夫相对柔弱、温文尔雅、欲望不强、持续时间太短的性行为采取了不认同并归咎为阴茎短小的说法。

总之,健康的性知识教育是心理咨询的主要手段之一,也是开展其他治疗方法前必经的过程。性咨询师应通过耐心地倾听和提问,找到患者存在的主要心理问题,有针对性地进行强化和教育,为制定进一步的治疗计划打下基础。

倾听是一项十分重要的技术,然而并不是所有医生都懂得如何去倾听。掌握倾听技术是对每一位性咨询师的基本要求。

(1)保密承诺:尽快和患者建立起相互信任的医患关系,这是性咨询在采集病史、倾听患者陈述之前首先要做到的。为达到这个目的,性治疗医生在开场白中可以简单介绍一下科室和自己的情况。如科室雄厚的技术力量以及自己的学位、从事性治疗的经验和学术成就等,让患者知道医生有能力和信心来处理好他/她的性问题。同时要对科室严谨保密的规定作充分的说明,如性医学科的病历是不归入医院病案室,而是由科内专柜保管的,任何人不得随意调阅。甚至强调即使是夫妻间存在的隐私,也会受到充分的尊重和保密等。在此基础上,性咨询师通过端庄亲切的仪表,富于同情和理解的语言,是完全可以在短时间里和患者建立起相互信任并倾心交谈的合作关系。在以后的晤谈和接触中,医生要随时注意自己的形象,始终维护患者对自己的信任和对治疗的信心,这将是取得满意疗效的保障。

(2)倾听的主要程序:常用的四种倾听技术是澄清、释义、情感反映和总结。澄清是在来访者发出模棱两可的信息后,对来访者所提问题的反应。它开始于“你的意思是……”或“你是说……”这样的问句,然后重复来访者先前的信息。释义是把来访者信息中与情境、事件、人物和想法有关的内容进行逻辑组合,找出与性问题相关的因素。情感反映是对来访者的感受或信息中的情感内容重新加以编排。信息中的情感成分通常揭示出来访者对有关内容的感受,比如来访者可能对自己在性生活中的表现(内容)感到失望(情感)。而总结则是释义和情感反映这两种反应的进一步延伸,它把信息的不同内容或多个不同的信息加以链接,并重新编排。

(3)细心观察、耐心地倾听:在咨询会谈中,咨询师除了要倾听来访者的谈话之外,还要注意观察来访者的非言语行为,以协助了解来访者的情绪、情感和内心的真实思想。非言语行为包括面部表情、躯体动作和副语言等。在咨询过程中,咨询师需要对这些非言语行为给予仔细观察和理解。当患者开始陈述自己的性问题时,医生要专注倾听,不要轻易打断,可以用点头或偶尔重复一句患者的话来表示自己是听懂了。在回答问题之前,应让患者把他/她所有的问题和想法都讲清楚,很多女患者在谈到因性问题(自己的或丈夫的)而受到的困扰或委曲时都会哭泣,这时医生应在理解的基础上以关切的态度让她休息一下,待平静后再继续谈。切莫在患者情绪不稳定时急切提问,更不要在未倾听完患者的陈述时就主观下结论或作出判断。这和内外科以问诊为主的采集病史不一样,医生首先要十分耐心的听取来访者对性历史的陈述,这是性咨询的一大特点。

(4)尊重来访者,不要强加于人:性咨询师自己首先要树立正确的性观念,以科学和客观的态度来看待患者的性问题。当来访者谈到自己的隐私时,医生要以落落大方、从容不迫的态度仔细听取,绝不能表现出惊异、拘谨、冷漠或轻浮,这样才能使来访者放心地进一步谈出内心深处的问题或感受。

有的来访者在陈述中谈了很多和配偶的生活琐事而始终未谈到性问题的实质,使医生听后不得要领,这时医生也可作一些提示。但要避免打断话头,先入为主的高谈阔论。即使发现来访者存在一些错误的性观念也不要急于纠正,可留待以后的晤谈和治疗中去解决。更不要试着用自己的性价值观去强加给来访者,甚至要求他\\textbackslash\{\}她以此来评价和表述自己的性问题。

总之,倾听技术要求性咨询师把自己摆在和来访者平等的位置上,在相互信任和轻松和谐的气氛中,耐心倾听患者的陈述。除非很有必要的提示,医生要避免对来访者的打扰。这种仔细倾听是作出正确评价的基础。

晤谈是继倾听后采集病史的第二项必备技术。它是一种以患者自述为中心的谈话技巧。

(1)晤谈的主要内容:晤谈首先要听取患者介绍目前存在的问题和通过问诊找出隐蔽的问题。一旦确认患者确实存在性问题后,就应依次了解下列情况:①患者主要的性功能障碍表现是什么?它是突然发生的还是逐步加重的?已持续了多长时间?②导致性功能障碍的病因可能是什么?如意外事件、吵架、婚外恋、工作压力等。还应注意有无引起器质性性功能障碍的因素,如阴道炎、附件炎及经绝期后内分泌改变的影响等。③患者对性生理、解剖和心理知识了解多少?这在中国是有普遍意义的问题。如对性器官的认识,以及对手淫的态度等。④除性功能障碍外,还有什么性方面的问题?如手淫的方式、频率,婚前或婚外性生活的情况?有无同性恋倾向等。⑤配偶的性功能状况,是否存在交流不够或婚姻问题?⑥患者夫妇间还存在哪些非性方面的问题?如经济问题、性格问题和与家庭其他成员的关系等。⑦是否在它院作过诊治?用过何种药物或治疗措施?疗效怎样?⑧患者来院就诊的真正动机是什么?对治疗的信心和耐心有多大?⑨患者要想达到的治疗目标是什么?这些目标是否现实?⑩患者是否存在病态心理或精神疾病?当然,上述问题只是采集病史时应了解的基本情况,对每例患者的具体情况,性咨询师都应作出有针对性的深入晤谈,包括对有深层心理问题的来访者进行耐心的心理分析。

(2)常用的晤谈技术包括提问、解释、提供信息、即时化、自我暴露和对质。提问是提出开放性或闭合性的问题,以便从来访者那里寻求详细的解释或信息。解释是在来访者的讲述中找出主题和模式,使来访者隐含的信息更清晰地显现出来的一种反应技术。提供信息是指与来访者交流有关经验、事件、行动选择或人物的资料和事实,即时化是在咨询面谈中对当前正在发生的事情作出的言语反应。自我暴露是与来访者分享个人的信息或经验。对质技术指的是指出来访者行为和言语表达中的矛盾或不一致之处。

(3)分阶段进行:马斯特斯和约翰逊等西方性学家一般把晤谈分成逐步深入的三个单元来进行,也即每周一次,分三次门诊来完成。这既是病史采集过程,又是进行心理辅导、性知识教育和对性问题作出评价并制定治疗计划的过程。在每个单元实施前最好先拟定一个晤谈计划。首先通过倾听为患者存在的问题作一个假设,而后为晤谈确定一个目标,并安排出晤谈的内容和顺序。由于每次晤谈的时间最好控制在45分钟,这种预先拟定的计划将有助于性咨询师控制进程和达到预期的效果。三个单元的具体安排是:第一单元首先请患者夫妇尽可能地提出他们的问题和希望,而后留下一位进行单独晤谈。另一位则留作第二单元进行单独的晤谈。第三单元由夫妇双方共同参加,在治疗医生的参与下鼓励他们相互交流、讨论双方性问题之所在和可能的诱因,确定治疗目标和制定治疗计划。

不过在中国由于文化和经济等因素的差异,西方的这种晤谈模式不完全适用。很多来访者不能坚持三周的门诊随访,特别是远道而来的夫妇或配偶对性治疗并不热衷时,往往在初次门诊后就因“收获不大”而放弃治疗或改往它处就诊。针对这种情况,笔者在基本参照三单元晤谈原则的基础上对晤谈次数作了压缩。若是患者夫妇同来就诊,就先用半小时左右了解一般概况并和患者建立初步的医患合作关系(第一单元)。而后留下妻子单独晤谈约半小时,丈夫此时作男性学体检。接下来由丈夫单独晤谈约半小时,妻子作女性学体检,必要时夫妇双方在专科护士指导下填写有关自我报告调查问卷(第二单元)。最后待性咨询师/性治疗师汇总已掌握的资料后,再请夫妇双方一起讨论有关问题,并由医生作出初步诊断和治疗建议(第三单元)。总共需2~3小时。在来访者夫妇下一次就诊前,性治疗师可根据需要,先单独安排妻子或丈夫来院进行深入的晤谈,以便对初次就诊采集的病史作出补充。这种相对集中的采史方法虽一次性耗时较长,但为多数患者夫妇所接受。事实上我国多数患者的性功能障碍都和婚后家庭及社会因素有关,并无每例均需花很多时间去找寻深层心理问题的必要。因此,为了患者的利益,尽快抓住问题的核心,建立起患者的治疗信心才是最重要的,完全没有必要拘泥于国外性治疗晤谈的典型模式。

(4)注意提问的技巧:晤谈和倾听不一样,它是在医生的提问下和患者进行交流的过程。提问时一定不能采取刨根问底式的问诊,而应充分尊重和保护患者已经很脆弱的心理,并遵循以患者自述为中心的晤谈原则。因此提问最好以委婉或诱导的方式进行,让患者以比较轻松的节奏来陈述他们的问题。

很多患者在表达性问题时是比较隐晦的。医生为搞清情况提问时可遵循由“学习”到“态度”再到“行为”的程序。如要想了解手淫情况就可先问:“你是从哪里知道有关手淫行为的?”而后问:“你觉得男孩子(女孩子)手淫有什么问题吗?”最后问:“你是否尝试过?”而不要一开始就问:“你手淫过吗?”使得一些敏感的患者难以启齿。

医生在提问前也可先作些陈述,表明自己对一些性问题的看法,以消除患者的疑虑和窘迫感。如医生可以说:“现在婚外性行为已相当常见,你对此有何看法?有没有这方面的经历?”或者问:“很多自慰器具对促进性快感和体验性高潮是有帮助的,你同意这种看法吗?是否有这方面的体会?”

对一些涉及个人隐私或可能危及夫妻关系的敏感问题,在询问前一是要先作出慎重的保密承诺,并在能使患者放心的单独晤谈中提出来。如对乱伦的回忆,婚外恋情及卖淫的经历等。

(5)注意不同晤谈对象的特点:性咨询师/性治疗师在和来访者进行晤谈时要考虑到患者的智力、学历和处世经历的不同而有所区别,不恰当的使用医学或心理学术语不仅不能使问题澄清,还可能使问题混淆。如对只有初中文化程度的来访者使用“性游戏”、“潜意识阻抗”或“性应答功能失调”等术语或诊断,就只能造成患者的困惑甚至误解。对这样的来访者一定要使用通俗易懂但不失科学的语言才能达到沟通和交流的目的。

不同年龄段的来访者也有其相应的特点。如对青、中年女性来说,晤谈的内容往往会集中在婚姻上,因为性与婚姻是分不开的。由婚姻中产生的怨恨、内疚和畏惧常是造成女性性功能障碍的原因。因此对这一年龄段的女性患者,必须认真探讨婚姻和性的相互关系,并找出明显的或潜伏的影响因素。当然,由于妊娠和孩子的出生影响到性的调整,以及工作和家务的矛盾,婚外关系等可能影响到性表现能力的问题,都可能成为晤谈的内容。此年龄段的女性,晤谈内容所涉及的领域相对要广泛些。而对老年女性来说寻求性治疗的唯一目的几乎都是对性功能减退的焦虑,当然也包括对丈夫过强性欲的不理解。在这一年龄段,老年女性来就诊的人数远低于老年男性,而且很多老年女性就诊的原因不是自身对性生活的追求,而是为维系或重建婚姻以获取安全感。老年妇女因受陈腐观念的影响,往往比年轻妇女有更多的顾虑,性咨询师必须抱以十分尊重、理解和耐心的心情来进行晤谈,在没有获得来访者充分信任前,她们将不会暴露真实的想法和感受。

另外,来访者的社会地位、民族和宗教信仰等都会影响到医患关系和晤谈的方式。性咨询师必须综合考虑到这些因素,拟定出来访者能够接受的晤谈内容和进程安排,才能收到预期的效果。

在性咨询的进程中,常常会出现来访者或咨询师对某些特定咨询表现出阻抗的现象,如来访者咨询时无故迟到、推迟或取消约定的咨询等,或治疗师提早结束某次咨询以及阻止来访者要说出的话。一般把阻抗界定为来访者或咨询师干扰咨询过程和结果,或降低其成功可能性的任何行为。为消除阻抗应注意以下几个方面。

(1)鼓励来访者积极参与。在开始性咨询的初期应多给予患者关怀与鼓励,肯定他们的某些观点和进步,让来访者自己选择那些更适合他们生活的咨询计划,从而调动来访者的参与积极性,有助于减少心理阻抗的影响。一旦发现来访者有阻抗现象,咨询师应减少强调自己的影响,相反,要充分肯定来访者的贡献以及尽量要来访者表达自己的意见等。

(2)咨询师要把握好咨询的节奏,也就是咨询的进展不宜太快也不能太慢。如果咨询师前进得太快,来访者对咨询师的某些建议尚难接受,就可能发生阻抗。而太慢又会降低咨询的效率,使来访者感到咨询效果不明显也容易发生阻抗。因此要把握好咨询的节奏,注意避免过早地进行干预。

(3)对于缺乏经验的咨询师,由于担心自己的能力不足而可能出现焦虑,或因对来访者过强的责任感而使来访者的阻抗表现得更为明显。处理咨询师焦虑的第一个有效步骤就是承认它,向同事或导师倾诉出自己的焦虑,并在工作中尽量减少对自身的关注,这将能有效降低焦虑的程度。咨询师还要明确自己在咨询中的界限,区分出哪些是自己应该负责的事情,哪些是应该由访者负责的事情,避免替来访者做他应该做的事情。对咨询的一个常见的误解就是,咨询师总是要为产生改变的各种结果而负全责,这种强加给自己的“责任感”是没有必要也不合理的。

(4)关怀自己,避免同情的疲劳;如果咨询师忙于关心他人的生活而忽略了自己的需要,甚至过度投入工作而导致身心疲惫,那么咨询师本身的阻抗就会变得严重。这种现象,称为“同情的疲劳”,即由于对他人比对自己有更多的关怀而产生的心理和身体精疲力竭的现象。解决这种阻抗的最佳方式是预防发生。可通过合理的安排工作强度和丰富多彩的休息方式,使疲劳和耗竭比较不容易发生。尤需强调的是,不要把工作带回家或者在休息时间仍然沉湎于工作中发生的事情上,这一点对初学者来说是尤其需要注意的。

性咨询或性治疗采集病史的另一主要手段就是来访者填写的自我报告调查问卷。这种问卷是一种标准化的量表,在各类量表中均要求患者应用数字或等级来对自己的心理和行为进行客观和标准化的描述,从而成为倾听和晤谈的重要补充。在精神科它被广泛地应用于对患者的智力、人格和心理状态的评价。因此,它不仅是采集病史的重要组成部分,也是对患者进行心理学检查的主要手段之一。

量表是经专家周密设计和统计处理后拟定的,在特定的范围内提问比较全面、规范,而且经过定量评分就可通过统计学处理使经验性评价上得到定量性评价,从而使性学研究进入一个更加科学和可对不同样本进行比较研究的阶段。但是,由于性学研究涉及的范围十分广泛,不可能用一个量表来概括所有的问题。因此,各国学者针对不同的性学问题设计了大量的量表,并正在不断完善中。这些量表具体可分为三类:

(1)性功能症状评定量表:主要用于对性功能障碍的病状直接打分。这是采集病史时普遍采用的。如:国际勃起功能指数问卷调查表(IIEF),症状自评量表(SCL-90)等。

(2)与性有关的心理社会状态评定量表:此类量表又可分成五种,性治疗常用的是评价婚姻关系的量表。如:Olson婚姻质量问卷(ENRICH),婚姻调适量表(dyadic adyustment scale,DAS),性相互作用调查表(sexual Interaction Inventory,SII)等。

(3)用以评价情绪的量表,如抑郁自评量表(SDS),焦虑自评量表(SAS)和用以评价与性有关的单项能力或特质的量表等。

(4)人格测验用量表:用于对患者较稳定的心理特质打分,如明尼苏达多相人格测验(MMPI),以及对外界环境的反应方式和行为模式测验等。临床上可根据需要选用相应的量表进行测试。

量表在我国性咨询或治疗机构中使用还不够广泛,主要原因是这些量表受西方文化影响很大,不全适用于中国国情,加之各类量表普遍设置的项目繁多,对文化程度较低的来访者来说,存在着理解和判断上的问题。而指导来访者填写自我报告调查问卷又往往给工作繁忙的性咨询或治疗机构增加了额外的负担。因此目前主要用于临床研究或认为有必要进行心理或人格测试时才使用。我国性学家已着手研究适合中国国情的量表,如李学谦设计的“性功能评定量表”和“性满意量表”等,为量表在我国的推广和应用起到了积极的推动作用。

体格检查的主要目的是发现是否有与性功能障碍有关的、病理性改变导致的器质性原因。因此,对通过性咨询明确需作性治疗的来访者进行全面的体检是十分必要的。由于导致或诱发性功能障碍的病变甚多,体检时应按系统有侧重地进行。与性关系密切的常见疾病有:

1.性腺功能低下 如Turner's综合征可出现身材矮小,乳腺发育差乳头间距大,以及盾型胸等畸形,此类患者的性欲明显减退。

2.糖尿病 主要影响男性勃起功能和女性达到性高潮。

3.甲状腺功能亢进或减退 可导致部分患者的性欲增高或减退。

4.肾上腺皮质功能减退 如爱迪森氏病患者大部分都有性欲下降。

5.垂体功能的低下或亢进都可以造成性欲的降低。

因此,体检时应根据病史和一般体检而有针对性地作内分泌系统检查。

心血管疾病对男性性功能的影响已比较明确,而对女性的影响尚缺乏深入研究。但由于冠心病、心肌梗死、心绞痛的患者在性交中或性交后发生猝死的病案时有报导,因此这类疾病对女性性活动在心理上的影响是肯定的。其他如高血压、心力衰竭、主动脉病变等所使用的药物也都可能对性功能产生损害。

肿瘤引起的性功能障碍既可由肿瘤本身所造成,也可由化疗、放疗或手术造成。同时患者患肿瘤后的忧虑心理也是形成性功能障碍的重要因素。因此各类恶性肿瘤对性功能都有不同程度的影响。

各种慢性、消耗性疾病对性功能都有不同程度的影响,体检时应给予应有的重视。如慢性肾衰竭的患者约有80\%发生性欲减退,回肠或结肠道造痿后部分患者也出现性交困难。神经和肌肉疾病对性功能影响也是明显的,如脑卒中、截瘫、多发性硬化、重症肌无力等症都会给性生活带来困难。另外,一些少见的关节和结缔组织疾病往往导致女性生殖道溃疡或阴道润滑障碍(干燥综合征),对性功能的影响也十分明显。

因此,性治疗的体检既应是全面系统的,也应是具有针对性的,并且要特别注意与性功能有关的异常发现。由于人类对性学研究的历史尚短,对很多躯体疾病对性功能的影响尚缺乏深入的研究,正如男性学中早期认为勃起功能障碍大多是心理性的,而随着研究的深入现已明确大多数勃起功能障碍是与器质性因素有关一样,在今后的女性学研究中,也应重视器质性因素的研究。所以,临床体检时绝不能因为当前认为大多数女性性功能障碍属心理性的,就放松了对各种器质性因素的探索和研究。

性医学检查主要包括性心理和性器官检查两部分。性心理检查又主要以倾听、晤谈和填写自我报告调查量表的形式进行,下面我们着重讨论性器官的检查。

性器官检查是和性评价以及性治疗结合在一起的,其和泌尿科、妇科检查最大的区别在于它不仅是发现与性功能障碍有关的器质性变化,同时还是进行性知识教育和促进夫妇间公开交流的开始。因此,在一般情况下,只要患者允许,其配偶可以在场。应鼓励双方在体检时交流和提问,性治疗师应给予耐心讲解,并纠正各种认识上的误区和偏见。如正确了解包皮、阴茎长短和粗细在性活动中的意义,以及女性生殖器官的解剖结构、生理功能,消除女性生殖器官肮脏、难看或神秘莫测的误解等。这种由夫妇共同参加的性器官检查,将对以后开展其他性治疗起到奠基作用。

由于男性外生殖器外露,其检查程序基本同于泌尿外科。但要侧重检查第二性征发育情况,阴茎的长短(包括牵拉长)、粗细,有无包皮过长或包茎,有无尿道下裂等畸形,睪丸和附睾的体积大小、硬度,是否有隐睾或精索静脉曲张以及有无炎症、肿瘤、赘生物等。而女性的性医学检查稍区别于妇科检查,现重点介绍如下:

患者以膀胱截石位仰卧于检查床,双腿分开并用脚架托起,头部可垫高,以便患者向下观察。

性治疗师首先要观察阴毛的分布,阴阜、阴蒂、大阴唇、小阴唇及尿道口、处女膜残痕有无异常。在作出初步判断后,可用较明亮的灯光照射外阴部,然后指导患者手持一面镜子观察自己的外生殖器,正确了解各部位的解剖名称和功能。通过这种观察将有助于缓解患者因性无知而带来的性压抑。

除和妇产科检查一样外,对有性交疼痛或阴道痉挛的患者来说,阴道指检和置窥阴器检查很可能让患者难受。遇到这种情况先要对患者作好心理疏导,检查时动作尽可能轻柔并时间要短、动作要快。对多数合作的患者来说,在调整光源后可通过小镜子观察自己的阴道和子宫颈,了解性交射精后精液贮存于后窟窿的部位。指检时可对阴道前壁距阴道口约5cm处的“G”点作加压刺激,以了解“G”点是否明显存在,并告知患者该部位可能存在性敏感区,以及性交时包含“G”点在内的整个阴道外1/3因充血而隆起形成的环状垫,是阴道性交时快感和性高潮体验的解剖基础。对声称缺乏快感的患者来说,性交时集中精力去体验该部位所受到的刺激,并通过心理暗示去认同它为“快感”,是行为疗法的重要手段之一。另外,让患者正确了解自身的阴道,也是纠正一些患者抱怨丈夫的阴茎不够粗长的错误观点的好机会,事实上阴茎勃起长度达8cm以上,就能完成性生活。

此外,性治疗师在检查时若怀疑阴道或白带有异常,可取白带标本作清洁度和细菌学检查。

首先应了解会阴体是否完整,有的经产妇会阴可呈不同程度的撕裂。关于阴道压力(松紧度)可通过专用设备进行测试,阴道压力低于10mmHg提示阴道较松弛有可能影响性感受,高于16mmHg且伴有性交疼痛者可能有阴道痉挛存在。在检查的同时可指导患者学习以耻骨尾骨肌为主的盆底肌群锻炼法(凯格尔法),即在完全放松后让患者收缩阴道周围的肌肉(相当于排尿时突然终止排尿的动作),放松后再次作上述收缩动作,如此反复20次以上,每天至少坚持2~3组锻炼。虽然阴道压力和性满意度的关系尚缺乏深入的研究,但这种锻炼将有助于增加患者在性交时的体验和信心。

由于阴蒂在女性性反应中的特殊地位,检查时应特别注意。医生应轻轻将阴蒂包皮向上翻起观察阴蒂头和体,若有丈夫陪同可邀请其一起观察,并告诉患者和丈夫该部位是女性最敏感的动欲器官,通过刺激阴蒂可使女性获得性高潮。而后,让患者自己用手指触摸阴蒂体会这种敏感性,并鼓励他们回家后自行练习,找到自己最喜欢的触摸方式。

阴蒂常见的问题有炎症、肥大、包皮过长或粘连等。

虽不多见,但一旦出现问题往往需要先通过手术治疗才能进入心理治疗。如两性畸形、无阴道、阴蒂包皮过长或粘连、无孔处女膜或处女膜增厚症,以及影响性生活的阴道横膈、纵隔等。

包括外阴感染、尿道感染、阴道炎、附件炎等。既可是一般化脓性感染,又可是性传播疾病如尖锐湿疣、梅毒、淋病、生殖器疱疹等,还可能是真菌、滴虫等造成的炎症。由于女性性器官在解剖结构上相对隐匿和深藏,特异性和非特异性感染成为女性最常见的器质性因素之一。只要这些感染性炎症引起了局部的疼痛、烧灼或搔痒就会导致性生活困难。当然,由此带来的心理压力也是不容忽视的。此类疾病常需通过实验室检查才能确诊。

生殖器官的肿瘤并非罕见,如外阴肿瘤、子宫颈癌、子宫肌瘤、卵巢肿瘤等,它们既可对性器官造成直接损害,又可因对性腺的破坏而造成性激素的分泌失调(如卵巢癌的广泛盆腔清扫术后)。另外,作为性敏感区的乳腺癌对性功能的影响也是明显的,和全身其他部位的肿瘤不同,性生殖器官的肿瘤对患者的性心理有更加直接的压抑作用。

如尿瘘(膀胱阴道瘘)、粪瘘(直肠阴道瘘)、子宫脱垂等病变,性医学检查时也应予以重视。

总之,性医学检查兼具诊断和治疗的双重意义,处置得当可缓解患者紧张或压抑的情绪,并为今后治疗计划的拟订打下基础。不过在实践中,是否由配偶共同参加要尊重患者的意见,并且最好由有丰富经验的性治疗医生来进行检查和指导。一旦发现或怀疑患者有器质性病变时,不要急于下结论,可考虑适时地让配偶回避,待一切问题搞清楚后再选择适当的方式和患者及家属沟通。性治疗医生通过大方和熟练的性医学检查,特别是引导患者夫妇对性体验的交流,是建立良好医患关系的重要手段。

由于导致男、女性功能障碍的器质性因素众多,临床上应根据患者的具体情况选择性地作超声或实验室检查,常用的有:

B超在男性常用于准确测量睾丸、附睾及前列腺的大小,彩色双功能多普勒超声在血管性ED的诊断中,被认为是诊断动脉性勃起功能障碍的“金标准”。在女性,特别是经阴道B超已广泛应用于对各种妇科疾病的诊断中。

在男性,泌尿外科已做了大量研究,如睾酮低下可能为性腺发育不良,也是中老年部分雄性激素缺乏综合征(迟发性性腺功能减退)的主要确诊依据。对女性有两方面的意义:

(1)对性功能的影响:国外性学家就雌激素对女性性功能的影响曾做了大量研究,由于女性性激素随月经周期的变化而变化,加之各研究样本的非同一性,因此尚无明确结论。不过大多数研究表明除经绝期妇女因雌激素降低而导致阴道润滑力下降外,雌激素水平的降低与性活动之间没有显著的关系。而雌激素水平若有增高,则似有降低性反应的倾向。至于雄激素虽然在女性性行为中的作用和机制仍有待进一步明确,但多数学者观察到当女性雄激素水平升高时,阴道对色情刺激的反应也随之增强,并且有助于“性满足”。因此,在临床上对雄激素偏低的患者适量补给雄激素已是一种常用的治疗手段。

(2)对不育患者的意义:在女性不育症中,下丘脑—垂体—卵巢性腺轴功能的失调是常见病因之一。而不孕症是可能影响女性性功能的心理因素之一。

(1)甲状腺功能测定:常用的检查是T3、T4以及甲状腺I131吸收率测定,可用于诊断甲状腺功能亢进或减退。

(2)肾上腺皮质机能测定:常用的有17-羟类固醇,17-酮类固醇,游离皮质醇及醛固酮等的测定,主要用于诊断肾上腺皮质机能减退症等。

可通过细胞培养或活体组织标本直接制片等方法检查,通过对染色体和性染色体的观察可对遗传性的单基因或多基因病作出诊断,同时也可用于对炎症和肿瘤的鉴别诊断。

主要用于不孕症中各种抗体的检测。如有抗精子抗体存在,表现为血中的IgA与IgG增高,偶尔也有IgM和IgE的增高,若有IgE增高则患者将出现过敏症状,这是少数妇女对精液过敏的原因。

白细胞>10/HP提示有前列腺炎症的可能,从而影响到男性的性心理。卵磷脂小体<++提示勃起功能可能减弱。

(1)镜检:清洁度可分Ⅰ~Ⅳ度,Ⅰ~Ⅱ度属正常,Ⅲ~Ⅳ度提示有炎症。同时可检查出真菌,滴虫等致病因素。

(2)细菌培养及药敏试验:可检测出导致阴道感染的致病性细菌菌种及对常见抗菌药物的敏感性。而阴道、宫颈、子宫和附件的炎症,是导致女性性交疼痛的常见器质性原因。

世界卫生组织提出的性传播疾病(STD)已多达29种之多。我国法定有艾滋病、淋病、梅毒、尖锐湿疣、非淋菌性尿道炎、生殖器疱疹、性病性淋巴肉芽肿、软下疳等8种。由于病原体涉及细菌、螺旋体、支源体、衣原体、病毒和昆虫等,其检测手段也包括镜检(光学显微镜、荧光显微镜、相差显微镜、电镜等),厌氧菌培养、抗原、抗体检测,基因增扩以及动物接种等多种方法。由于艾滋病和其他性病的传播在全球范围内呈上升趋势,因此,性传播疾病对男女性功能的影响已日益显现,其治疗措施主要是针对病因治疗和必要的心理干预,这将在专门的章节中详细描述。

包括血液分析、血生化检查(肝功、肾功、电解质等)、尿液分析等,可由医院中心实验室承担。对怀疑有其他躯体疾病的患者,可根据病情选择性地应用。

为方便记录和收集资料的相对完整,笔者在重庆市第五人民医院工作期间制作的病历模式已应用了15年,有一定的参考价值,现介绍如下:

性医学科病历

1.性知识状况:(1)自评:较好□ 一般□ 差□ (2)精液宝贵□ (3)手淫有害□ (4)阳具崇拜□ (5)性器官肮脏□ (6)乱伦恐惧□ (7)怕怀孕□ (8)其他:

2.性欲:(1)正常□ (2)亢进□ (3)淡漠□ (4)消失□

3.清晨或夜间勃起情况:(1)明显□ (2)偶尔□ (3)有勃起但硬度差□ (4)不明显□

4.初次性行为:(1)年龄 岁 (2)性伴侣(现配偶□ 前妻□ 情人□)

5.性交环境:(1)较好□ (2)不受干扰□ (3)易受干扰□ (4)受过干扰□

6.性前爱抚:(1)0~5分钟□ (2)5~10分钟□ (3)10~20分钟□ (4)20~30分钟□ (5)>30分钟□(6)无□

7.勃起情况:(1)正常□ (2)不坚但可插入□ (3)未射精即疲软□ (4)不能插入□ (5)其他:

8.射精潜伏时间:(1)<1分钟□ (2)1~5分钟□ (3)6~10分钟□ (4)11~20分钟□ (5)21~30分钟□(6)>30分钟□ (7)不射精□

9.连续抽动次数:(1)<15次□ (2)15~30次□ (3)31~60次□ (4)>60次□ (5)未插入即射精□

10.性交次数:(1) 次/周 (2) 次/月 (3) 月一次 (4)间断同居□ (5)其他:

11.性交方式:(1)阴道□ (2)口交□ (3)肛交□ (4)手淫□ (5)其他:

12.性交姿势:(1)男上位□ (2)女上位□ (3)侧位□ (4)后进式□ (5)变换姿势□ (6)性虐(自虐)方式:

13.避孕方式:(1)安全期□ (2)性交中断□ (3)避孕套□ (4)体外射精□ (5)安环□ (6)避孕药□ (7)未采取避孕□ (8)其他:

14.婚前性行为:(1)无□ (2)有□ (3)性伴侣 位 (4)维持 年 月

15.与前妻或女友性生活情况:(1)满意□ (2)一般□ (3)不满意□ (4)无□

16.手淫史:(1) 岁开始 (2)婚前 次/周、月 (3)婚后 次/周、月 (4)偶尔□ (5)无□ (6)其他:

17.婚外性生活:(1)无□ (2)有□ (3)性感受:(满意□ 不满意□) (4)已断交□

18.影响性生活的其他情况:\_\_\_\_\_\_\_\_\_\_\_\_\_\_\_\_\_\_\_\_\_\_\_\_\_\_\_\_\_\_\_\_\_\_\_\_\_\_\_\_\_\_\_\_\_\_\_\_\_\_\_\_\_\_\_\_\_\_\_\_\_\_\_\_\_\_

\_\_\_\_\_\_\_\_\_\_\_\_\_\_\_\_\_\_\_\_\_\_\_\_\_\_\_\_\_\_\_\_\_\_\_\_\_\_\_\_\_\_\_\_\_\_\_\_\_\_\_\_\_\_\_\_\_\_\_\_\_\_\_\_\_\_

\_\_\_\_\_\_\_\_\_\_\_\_\_\_\_\_\_\_\_\_\_\_\_\_\_\_\_\_\_\_\_\_\_\_\_\_\_\_\_\_\_\_\_\_\_\_\_\_\_\_\_\_\_\_\_\_\_\_\_\_\_\_\_\_\_\_

\_\_\_\_\_\_\_\_\_\_\_\_\_\_\_\_\_\_\_\_\_\_\_\_\_\_\_\_\_\_\_\_\_\_\_\_\_\_\_\_\_\_\_\_\_\_\_\_\_\_\_\_\_\_\_\_\_\_\_\_\_\_\_\_\_\_

19.对现有性生活的评价:\_\_\_\_\_\_\_\_\_\_\_\_\_\_\_\_\_\_\_\_\_\_\_\_\_\_\_\_\_\_\_\_\_\_\_\_\_\_\_\_\_\_\_\_\_\_\_\_\_\_\_\_\_\_\_\_\_\_\_\_\_\_\_\_\_\_

\_\_\_\_\_\_\_\_\_\_\_\_\_\_\_\_\_\_\_\_\_\_\_\_\_\_\_\_\_\_\_\_\_\_\_\_\_\_\_\_\_\_\_\_\_\_\_\_\_\_\_\_\_\_\_\_\_\_\_\_\_\_\_\_\_\_

\_\_\_\_\_\_\_\_\_\_\_\_\_\_\_\_\_\_\_\_\_\_\_\_\_\_\_\_\_\_\_\_\_\_\_\_\_\_\_\_\_\_\_\_\_\_\_\_\_\_\_\_\_\_\_\_\_\_\_\_\_\_\_\_\_\_

\_\_\_\_\_\_\_\_\_\_\_\_\_\_\_\_\_\_\_\_\_\_\_\_\_\_\_\_\_\_\_\_\_\_\_\_\_\_\_\_\_\_\_\_\_\_\_\_\_\_\_\_\_\_\_\_\_\_\_\_\_\_\_\_\_\_

1.无□ 2.时间及医院名称:\_\_\_\_\_\_\_\_\_\_\_\_\_\_\_\_\_\_\_\_\_

3.特殊检查:\_\_\_\_\_\_\_\_\_\_\_\_\_\_\_\_\_\_\_\_\_ 4.诊断:\_\_\_\_\_\_\_\_\_\_\_\_\_\_\_\_\_\_\_\_\_

5.治疗情况:(1)药物□ (2)手术□ (3)其他:

6.疗效:(1)有改善□ (2)无效□ (3)其他:

1.结婚(同居)时间: 年 月 2.性经验时间: 年 月

3.生育情况:(1)长子 岁(2)次子 岁(3)长女 岁(4)次女 岁(3)无□

4.夫妻感情:(1)恩爱□ (2)一般□ (3)冷漠□ (4)已趋破裂□

1.疾病:(1)无□ (2)生殖系结核□ (3)性传播疾病□ (4)前列腺炎□ (5)附睾炎□ (6)糖尿病□ (7)甲亢□ (8)其他:

2.手术:(1)无□ (2)名称: (3)时间:

3.药物史:(1)无□ (2)性激素□ (3)抗肿瘤药□ (4)镇静剂□ (5)麻醉剂□ (6)毒品□ (7)其他:

1.工作情况:(1)一般□ (2)轻松□ (3)劳累□ (4)经常离家出差□ (5)失业□ (6)与领导或同事关系紧张□ (7)其他:

2.经济状况:(1)较宽裕□ (2)一般□ (3)较紧张□ (4)十分困难□

3.嗜好:(1)无□ (2)吸烟 支/日 (3)饮酒 两/日 (4)其他:

4.与家庭成员的关系:(1)融洽□ (2)一般□ (3)紧张□

1.父系母系三代以内有否精神疾病:(1)无□ (2)有□:\_\_\_\_\_\_\_\_\_(关系)

2.父系母系三代以内有否有癫痫:(1)无□ (2)有□:\_\_\_\_\_\_\_\_\_(关系)

3.父系母系三代以内有否严重心理问题:(1)无□ (2)有□:\_\_\_\_\_\_\_\_\_(关系)

1.血压:/ mmHg 2.脉搏: 次/分 3.呼吸: 次/分

4.精神状况:(1)正常□ (2)焦虑□ (3)激动□ (4)紧张□ (5)抑郁□ (6)冷漠□ (7)其他:(言语和行为特征)

5.体型:(1)瘦弱□ (2)一般□ (3)强壮□ (4)肥胖□ (5)第二性征异常□

6.神经系:(1)深反射:存在□ 减弱□ (2)浅反射:存在□ 减弱□

7.其他:

1.阴茎:(1)长度 cm (2)周径 cm (3)牵拉长 cm (4)包皮过长□ (5)包茎□ (6)海绵体硬结□ (7)尿道下裂□ (8)尿道口:正常□ 充血□ 脓性分泌□ (9)其他:

2.阴囊:正常□ 湿疹□ 癣□ 炎症□ 其他:

3.睾丸:(1)位置:左(阴囊内□ 阴囊外□) 右(阴囊内□ 阴囊外□)

(2)大小:左( ml) 右( ml)

(3)硬度:左(较坚硬□ 中□ 软□) 右(较坚硬□ 中□ 软□)

(4)触痛:左(有□ 无□) 右(有□ 无□)

(5)鞘膜积液(有□ 无□)

(6)其他:

4.附睾:(1)正常□ (2)胀大:左( cm) 右( cm)

(3)触痛:左(有□ 无□) 右(有□ 无□)

5.精索静脉曲张:(1)有:左(轻□ 中□ 重□) 右(轻□ 中□ 重□) (2)无□

6.前列腺:(1)大小:cm (2)硬度:(坚硬□ 正常□) (3)结节:(有□ 无□) (4)中央沟:(消失□ 存在□)(5)触痛:(有□ 无□) (6)其他:

1.尿糖: 2.血糖:

3.前列腺液常规: 4.T3、T4:

5.性激素: 6.其他:

1.性知识状况:(1)较好□ (2)一般□ (3)差□ (4)精液宝贵□ (5)手淫有害□ (6)阳具崇拜□ (7)性器官肮脏□ (8)乱伦恐惧□ (9)怕怀孕□ (10)其他:

2.性欲:(1)正常□ (2)亢进□ (3)淡漠□ (4)消失□ (5)经期性欲:强□一般□弱□ (6)有无经期性交:有□无□ (7)其他:

3.性敏感部位:(1)口唇□ (2)耳根□ (3)颈部□ (4)乳房□ (5)臀部□ (6)下腹部□ (7)大腿内侧□(8)阴蒂□ (9)其他部位:

4.月经史:(1)初潮 岁 (2)周期 天(3)每次持续 天(4)经期紊乱□

5.性交疼痛:(1)轻度□ (2)剧烈□ (3)偶尔□ (4)无□

6.性高潮:(1)经常有□ (2)偶尔有□ (3)无高潮但有快感□ (4)无快感□

7.婚前性行为:(1)无□ (2)性伴侣 位 (3)性经验时间: 年 月

8.与前夫或男友性生活情况:(1)满意□ (2)一般□ (3)不满意□ (4)无□

9.手淫史:(1)无□(2) 岁开始 (3)婚前 次/周、月 (4)婚后 次/周、月 (5)偶尔□ (6)手淫性高潮(有□ 无□) (7)其他:

10.婚外性交:(1)无□ (2)有□ (3)已断交□ (4)满意度(满意□ 不满意□)

11.既往疾病:(1)无□ (2)阴道炎□ (3)宫颈糜烂□ (4)子宫附件炎□ (5)性传播疾病□ (6)其他:

12.影响性生活的其他情况:\_\_\_\_\_\_\_\_\_\_\_\_\_\_\_\_\_\_\_\_\_\_\_\_\_\_\_\_\_\_\_\_\_\_\_\_\_\_\_\_\_\_\_\_\_\_\_\_\_\_\_\_\_\_\_\_\_\_\_\_\_\_\_\_\_\_

\_\_\_\_\_\_\_\_\_\_\_\_\_\_\_\_\_\_\_\_\_\_\_\_\_\_\_\_\_\_\_\_\_\_\_\_\_\_\_\_\_\_\_\_\_\_\_\_\_\_\_\_\_\_\_\_\_\_\_\_\_\_\_\_\_\_

\_\_\_\_\_\_\_\_\_\_\_\_\_\_\_\_\_\_\_\_\_\_\_\_\_\_\_\_\_\_\_\_\_\_\_\_\_\_\_\_\_\_\_\_\_\_\_\_\_\_\_\_\_\_\_\_\_\_\_\_\_\_\_\_\_\_

\_\_\_\_\_\_\_\_\_\_\_\_\_\_\_\_\_\_\_\_\_\_\_\_\_\_\_\_\_\_\_\_\_\_\_\_\_\_\_\_\_\_\_\_\_\_\_\_\_\_\_\_\_\_\_\_\_\_\_\_\_\_\_\_\_\_

\_\_\_\_\_\_\_\_\_\_\_\_\_\_\_\_\_\_\_\_\_\_\_\_\_\_\_\_\_\_\_\_\_\_\_\_\_\_\_\_\_\_\_\_\_\_\_\_\_\_\_\_\_\_\_\_\_\_\_\_\_\_\_\_\_\_

13.对现有性生活的评价:\_\_\_\_\_\_\_\_\_\_\_\_\_\_\_\_\_\_\_\_\_\_\_\_\_\_\_\_\_\_\_\_\_\_\_\_\_\_\_\_\_\_\_\_\_\_\_\_\_\_\_\_\_\_\_\_\_\_\_\_\_\_\_\_\_\_

\_\_\_\_\_\_\_\_\_\_\_\_\_\_\_\_\_\_\_\_\_\_\_\_\_\_\_\_\_\_\_\_\_\_\_\_\_\_\_\_\_\_\_\_\_\_\_\_\_\_\_\_\_\_\_\_\_\_\_\_\_\_\_\_\_\_

\_\_\_\_\_\_\_\_\_\_\_\_\_\_\_\_\_\_\_\_\_\_\_\_\_\_\_\_\_\_\_\_\_\_\_\_\_\_\_\_\_\_\_\_\_\_\_\_\_\_\_\_\_\_\_\_\_\_\_\_\_\_\_\_\_\_

\_\_\_\_\_\_\_\_\_\_\_\_\_\_\_\_\_\_\_\_\_\_\_\_\_\_\_\_\_\_\_\_\_\_\_\_\_\_\_\_\_\_\_\_\_\_\_\_\_\_\_\_\_\_\_\_\_\_\_\_\_\_\_\_\_\_

14.性器官检查:

(1)外阴:①正常□ ②炎症□ ③溃疡□ ④赘生物□ ⑤其他:

(2)阴蒂:①正常□ ②包皮过长□ ③炎症□ ④粘连□ ⑤其他:

(3)阴道口:①正常□ ②松弛□ ③痉挛□ ④瘢痕触痛□ ⑤其他:

(4)尿道口:①正常□ ②炎症□ ③肉阜□ ④其他:

(5)阴道:①正常□ ②炎症□ ③赘生物□ ④G点(明显□ 不明显□) ⑤其他:

(6)宫颈:①正常□ ②糜烂(轻□ 中□ 重□) ③触血□ ④触痛□ ⑤其他:

(7)白带:①正常□ ②黄稠□ ③异味□ ④奶酪样□ ⑤其他:

(8)子宫:①正常□ ②压痛□ ③增大□ ④包块□ ⑤其他:

(9)附件:①正常□ ②压痛□ ③增厚□ ④包块□ ⑤其他:

15.实验室检查:

(1)白带常规: (2)T3、T4:

(3)性激素: (4)B超检查:

(5)其他:

16.阴道压力:(1)静态压力: mmHg (2)自主收缩压力: mmHg

(3)结论:

1.焦虑自评量表:

2.抑郁自评量表:

3.婚姻调节量表:

4.性相互作用调查表:

5.其他量表:

诊断与评估

\_\_\_\_\_\_\_\_\_\_\_\_\_\_\_\_\_\_\_\_\_\_\_\_\_\_\_\_\_\_\_\_\_\_\_\_\_\_\_\_\_\_\_\_\_\_\_\_\_\_\_\_\_\_\_\_\_\_\_\_\_\_\_\_\_\_

\_\_\_\_\_\_\_\_\_\_\_\_\_\_\_\_\_\_\_\_\_\_\_\_\_\_\_\_\_\_\_\_\_\_\_\_\_\_\_\_\_\_\_\_\_\_\_\_\_\_\_\_\_\_\_\_\_\_\_\_\_\_\_\_\_\_

\_\_\_\_\_\_\_\_\_\_\_\_\_\_\_\_\_\_\_\_\_\_\_\_\_\_\_\_\_\_\_\_\_\_\_\_\_\_\_\_\_\_\_\_\_\_\_\_\_\_\_\_\_\_\_\_\_\_\_\_\_\_\_\_\_\_

\_\_\_\_\_\_\_\_\_\_\_\_\_\_\_\_\_\_\_\_\_\_\_\_\_\_\_\_\_\_\_\_\_\_\_\_\_\_\_\_\_\_\_\_\_\_\_\_\_\_\_\_\_\_\_\_\_\_\_\_\_\_\_\_\_\_

\_\_\_\_\_\_\_\_\_\_\_\_\_\_\_\_\_\_\_\_\_\_\_\_\_\_\_\_\_\_\_\_\_\_\_\_\_\_\_\_\_\_\_\_\_\_\_\_\_\_\_\_\_\_\_\_\_\_\_\_\_\_\_\_\_\_

\_\_\_\_\_\_\_\_\_\_\_\_\_\_\_\_\_\_\_\_\_\_\_\_\_\_\_\_\_\_\_\_\_\_\_\_\_\_\_\_\_\_\_\_\_\_\_\_\_\_\_\_\_\_\_\_\_\_\_\_\_\_\_\_\_\_

\_\_\_\_\_\_\_\_\_\_\_\_\_\_\_\_\_\_\_\_\_\_\_\_\_\_\_\_\_\_\_\_\_\_\_\_\_\_\_\_\_\_\_\_\_\_\_\_\_\_\_\_\_\_\_\_\_\_\_\_\_\_\_\_\_\_

\_\_\_\_\_\_\_\_\_\_\_\_\_\_\_\_\_\_\_\_\_\_\_\_\_\_\_\_\_\_\_\_\_\_\_\_\_\_\_\_\_\_\_\_\_\_\_\_\_\_\_\_\_\_\_\_\_\_\_\_\_\_\_\_\_\_

\_\_\_\_\_\_\_\_\_\_\_\_\_\_\_\_\_\_\_\_\_\_\_\_\_\_\_\_\_\_\_\_\_\_\_\_\_\_\_\_\_\_\_\_\_\_\_\_\_\_\_\_\_\_\_\_\_\_\_\_\_\_\_\_\_\_

\_\_\_\_\_\_\_\_\_\_\_\_\_\_\_\_\_\_\_\_\_\_\_\_\_\_\_\_\_\_\_\_\_\_\_\_\_\_\_\_\_\_\_\_\_\_\_\_\_\_\_\_\_\_\_\_\_\_\_\_\_\_\_\_\_\_

医师:\_\_\_\_\_\_\_\_\_\_\_\_\_\_\_\_\_\_


\section{第三节 性治疗的基本模式和实施}

男性患者最常见的问题是怀疑因性交频率较高、手淫或遗精对身体可能造成的伤害,以及中老年男性因雄性激素分泌下降而出现夫妻生活不协调所产生的焦虑。女性患者常见的是经期、妊娠期和经绝期性交的问题。一般妇产科医生都告诫患者不能在经期性交,以免造成生殖道感染而危害身体健康。但这种情况仍可能发生在经期较长或经期出现性欲增强的患者身上,一旦发生经期性交后,有些妇女就会产生紧张情绪并把其后发生的各种不适均归罪于此。性治疗学家认为经期应避免性交,但也并不是性交的绝对禁忌期。只要避开出血较多的头2~3天并使用避孕套就不会带来不良后果。相反,对经期较长的妇女,若强行压抑性欲和拒绝丈夫的性爱,反而有可能损害夫妻关系和导致自身的性欲障碍。

妊娠早期的性生活虽不是禁忌,但丈夫的动作应轻柔,对有习惯性流产史的孕妇应暂停性交。妊娠晚期的性交可采用女上位或后进式。分娩后若产妇恢复正常可在产后6~8周恢复性交。

绝经后的性交不仅正常而且值得鼓励。这一时期的问题主要出现在传统观念的束缚和社会问题,在受封建文化孕育的大男子主义眼中,女性是传宗接代的生育工具,一旦停经结束了生育能力,其性能力也应随之停止。这一观念也被很多老年妇女认同,甚至担心继续保持性行为会被社会或子女看做是老不正经。因此导致了我国近一半老年夫妇分床的现象,急需得到纠正。

婚姻关系是导致性问题最直接的原因之一。一对貌合神离,甚至对配偶的心理或肉体十分厌恶的夫妇,在性生活中必然表现出形形色色的问题。这也是马斯特斯和约翰逊研究所治疗失败的常见原因之一。因此,在开始性治疗前一定要搞清楚该夫妇的婚姻状况,特别是现存的感情基础。若仅是因为缺乏满意的性生活而导致的公开或隐蔽的不满情绪,是可以通过满意的性治疗结果来解决的。如一位36岁的男子,因勃起功能障碍已离婚3次,本次在妻子的督促下来性医学中心就诊,妻子认为丈夫欺骗了她,坦言若不能治疗好就立即离婚。该男子也痛心疾首地表示这次是最后的希望,若不能治愈今后再也不结婚了。经过性感集中训练治疗6周,患者的勃起功能障碍有了明显改善,一个即将破裂的家庭得到了保全,夫妻间的感情问题也随之得到解决。

婚姻中更为常见的问题是夫妇间互不信任的情绪,这在女性患者身上表现得尤为突出。如一位28岁具有大学本科文化程度的女患者单独来院就医,述及一年前患上阴虱,而其丈夫因工作性质经常出入夜总会,故怀疑丈夫有性滥交并将阴虱传给自己。但又苦于无法得到证实,只得将猜疑藏于心中,从此对丈夫的一切社交活动都表示怀疑,性生活更是从回避到拒绝,甚至性交后出现恶心、呕吐,已有半年多无法过性生活。丈夫也因不理解而对妻子产生了猜忌。显然,如果不先解决夫妇间的这种不信任状态,任何性治疗都将是徒劳的。

另一种表现形式是对婚姻的失望,也是引发性问题的常见原因。一个妻子对丈夫最大的期望是责任感,这种责任感表现为对爱情的忠贞,对自己和孩子的关爱以及经济能力和应付各种困难时的安全感。因此,引发对婚姻失望的原因是多方面的。一旦这种失望发展到一定的程度就会敏感地反映到性关系上。妻子通常会用拒绝性关系作为惩罚丈夫的手段。而这种惩罚往往不能解决丈夫存在的根本问题,若超过一定限度将会使丈夫产生敌意,反过来损害婚姻关系。因此,在进行性治疗前应尽可能地消除这种敌意。至于为达离婚目的而蓄意破坏性关系的情况则常见于丈夫方面。有的妻子为挽回婚姻会表现出强烈的求治欲望,而丈夫扮演着虚假的勃起功能障碍,甚至公开要求医生协助,促使其达到离婚的目的,显然,这种情况已超出了单纯性治疗的范围。

缺乏交流是导致男女性功能障碍的常见病因。这和传统文化的影响有十分明显的关系,很多女子从小就受到父母和长辈的教诲,中国传统的“淑女”形象就应是矜持、害羞和对性讳莫言深的。因此很多妻子在性活动中压抑自己,即使没有得到快感或性高潮也从不和丈夫交流。加之性生活又多在黑暗中进行,丈夫难以观察妻子甚至得到错误的信息。一对因妻子“性冷淡”而来就诊的中年夫妇,结婚初期性生活十分和谐,后来丈夫因工作性质养成夜间工作的习惯,10多年来每次性生活都是丈夫将妻子从睡梦中推醒而后做爱。妻子早已厌倦这种影响睡眠的做爱,但却默默忍受以尽妻子的义务,而丈夫却误认为这是妻子乐于接受的方式。直到丈夫对妻子提出更高要求时才发现妻子对做爱早已失去了兴趣。其实只要他们夫妇间就做爱的时间和方式进行坦诚的交流,就可避免妻子10多年的无奈。这仅是传统文化对中国女性束缚的表现之一。

至于西方性学家通过精神分析发现的,因小女孩童年时被“阉割”的潜意识而导致成年后希望他们的丈夫也“阉割”的心理障碍在我国妇女中并不多见。因恋父情结而出现移情现象导致性功能障碍的偶见于知识女性,此类患者必须经过一定的精神分析治疗后才能转入其他治疗。

在我国,实施性治疗经常要遭到传统观念的阻碍,比如对手淫的看法,认为这是”正派”人不应作的下流、变态的行为,这在女性患者中尤为常见。对曾有手淫习惯的男性又常把它看做是造成“肾虚”或性功能障碍的病因。而在我们性治疗中无论是ED、早泄、不射精或各类女性性功能障碍,都离不开对手淫的有关指导和在家庭训练计划中的应用,因此在开始性治疗前必须树立对手淫的正确认识。另外对口交的坚决排斥,以及某些妇女对夫妻双方应平等参与性活动建议的不认同,都是源于传统性观念的阻碍。

要想纠正由传统性观念造成的错误认识并非易事。不过传统性观念的影响在我国40岁以下的青年人中已日趋减少,通过理性的性教育大部都能接受健康的性观念。对于年龄偏大或固执己见的夫妇,不能急于求成,而应循序渐进的分次讨论有关问题,尽力更新他们的观念,并在制订性治疗计划中充分考虑他们所能接受的程度,这也是性治疗必须切合实际并做到个案化的原因之一。以笔者的经验,只要患者夫妇愿意接受以性感集中训练为代表的行为疗法,很多传统的错误性观念都能在治疗进程中得到改变,这些改变将使患者夫妇切身受益。因此,性感集中训练本身就是一种最快捷而有效的性教育方式。

社会因素的影响是性治疗学家在着手拟订治疗计划前必须搞清的问题。性功能障碍本身就是个体和其所在环境之间相互作用的产物。因此社会因素可表现在多方面,如:婆媳关系、工作中的挫折、经济问题、宗教信仰甚至对子女教养的争执等,职业女性因性骚扰而导致的性心理障碍也是社会因素的一种表现。如某女性患者,26岁,容貌姣好,职业为公关文秘。其上司平均每月对其提出1~2次性要求,为了保住高薪职位,患者只得应付。但在长达2年的时间里,每次被迫性交后均感心情压抑,更谈不上获得快感。待交上男友后,虽尽力投入仍无性乐趣可言,并因阴道干涩而经常出现性交疼痛。类似病例在性治疗中心是十分常见的。

按照弗洛伊德理论,潜意识是指个体无法感知到的那一部分心理活功。干扰性治疗的常见潜意识活动,是乱伦恐惧和明显导致精神痛苦的既往事件。如早年受过性暴力侵犯的女孩潜意识中往往存在性厌恶甚至性恐惧情绪,这种患者对爱情和婚姻的向往并无问题,但直接面对性行为时潜意识中的恐惧就会激发出来,在这种情况下患者很难接受性治疗,即使进行了性感集中训练也不可能取得好的疗效。因此在采集病史时要注意有无恋父、恋母、乱伦恐惧等问题的存在,对女性患者尤其要注意是否有早年被性侵犯的经历。

如果发现有潜意识干扰问题的存在,应暂时终止性治疗。而改用心理动力学的心理分析方法对患者潜意识中的问题进行治疗。其方法就是把患者压抑的情绪识别、释放出来,进而引导到理性的意识领域中来。事实上,处理这种对性的潜意识干扰比其他精神疾病中的潜意识问题更为容易,只要患者能重述并面对既往的问题,同时又能得到配偶的理解与支持就可试行性治疗,因为性感集中训练本身就是促进夫妻感情并对患者摆脱潜意识干扰的有力支撑。

人格是指一个人的思维、情绪和行为的特征模式,以及这些模式背后隐藏或外显的心理机制。人格特征在形成过程中受到先天生物因素(占30\%~50\%)、后天自然和社会环境因素(占50\%~70\%)以及个人需要和动机因素的综合作用,它可分为正常人格(normal personality)和异常人格(disordered personality)。有人格障碍的患者,往往会表现出自私、任性、自卑、孤僻、易激惹甚至攻击性强等特点,这很可能就是导致夫妻和社会关系紧张的原因,在性治疗中也很难与治疗师进行交流和配合。因此,在初步接触发现患者有异常人格倾向时,可应用自陈量表做人格测验,常用的有明尼苏达多项人格调查表(MMPI)、艾森克人格问卷(EPQ)、卡特尔16项人格因素问卷(16PF)等。对确有明显人格障碍的患者不宜急于进行性治疗,而应建议首先获得精神科医生的帮助,待情况改善后再考虑有无作性治疗的必要。

情绪问题可以发生在性交的前、中、后,因为即使夫妻关系很好也难免会发生口角、争执,其中性交前和性交中的情绪变化对性生活质量影响较大。如果不愉快发生在性生活之前,原则上应中止这次计划中的性生活。现实生活中妻子也常用这种方式来惩戒丈夫,不过这种方式应适度并照顾到配偶的承受度,不能由此造成夫妻关系的长期损害。如某高级职称夫妇,因一点生活琐事争吵,妻子就以拒绝做爱来惩治丈夫,其后逐渐发展到各自找情人分居8年。待双方愿回头重组家庭时,却发现已失去了对性爱的激情和良好的性感受。在接受了长达7个月的婚姻治疗和3个月的性治疗后,这对夫妇才恢复了正常的性生活。另一种情况是一方有激情,而另一方有情绪不愿做爱,勉强的性生活实际上是一种“侵犯”,必使不愿做爱一方产生厌恶甚至敌意,这也是家庭暴力的常见诱因。

发生在性生活中的情绪变化也十分常见。如居住条件差怕性交响动传到室外,或正在性交时父母或孩子来敲门;性生活时突然接到不愉快的电话;因动作粗暴或配偶不愿接受的性交方式(如口交、肛交)所产生的不愉快等。若已在性生活中出现情绪问题,不必勉强继续进行,而在下次性生活时应尽量避免上次发生的不愉快情绪的教训。只有在这些情绪问题累积成某种病理性的条件反射,并以性功能障碍的形式表现出来时才需要性治疗的介入。

性变态是指性唤起、性满足方式及对象异于常人。如恋童症、恋物症、偷窥症等。以恋物症为例,患者嗅到偷来的女人胸罩、内裤时会获得性兴奋和快感,而直接面对女人时却不感兴趣。这种婚前长期形成的性反应模式,必然会抑制其夫妻性生活时的性反应。因此,由性变态导致的性功能障碍不宜作性治疗,而应首先处理性变态。一旦性变态获得纠正,其性功能障碍大多能随之解决。

治疗目标是在开始治疗前必须由性治疗医生和患者达成共识的一个问题。医生应首先倾听患者的意见,防止把自己的观点强加于人。但也要注意患者提出的目标是否错误或不切实际。如一位来就诊的妻子提出,丈夫的阴茎太短小是自己不能满足的主要原因,因此治疗的目标就是希望医生能将丈夫的阴茎加粗加长。经医生检查发现丈夫阴茎完全正常,深入晤谈得知问题在妻子对前夫相对粗暴的性行为的认同,反而对现丈夫温柔的爱抚不适应。经医生耐心的疏导并改进丈夫的性交姿势和性技巧后,患者放弃了原有的治疗目标并取得了很好的疗效。

在制订性治疗计划时,我们强调的是有针对性和可行性。虽然行为疗法的基本程序有一定的规范,但每对求治的夫妇都有其特殊性,因此必须做到个案设计。这不仅针对不同病种,还要考虑到患者的年龄、文化水平、职业、民族、宗教信仰、人格特征、性经验、是否合并有其他身心疾病等因素。

在可行性方面最常遇到的问题是夫妻双方是否能共同参与以性感集中训练为代表的行为治疗,马斯特斯和约翰逊的研究十分重视这一点,甚至认为夫妻共同参与是开展性治疗的前提。但在我国,由于多数女性仍然受到传统观念的影响,在以男性性功能障碍为主的夫妇中,只有约30\%的妻子愿意来医院参与行为治疗。所以对多数患者来说,性治疗师即使设计出“理想”的性治疗计划却无可行性。

面对这一国情,笔者通过“适合中国国情的行为疗法临床研究”课题作出了重大修订。对于配偶不愿意来医院参与行为治疗的病例,首先力劝配偶能来医院晤谈一次,这对完善病史和制定治疗计划十分重要,同时可进行必要的心理辅导和性知识教育,同时还可以争取到一部分配偶改变想法,愿意到医院来参与性治疗。若配偶坚决不愿来医院,就建议通过电话晤谈,至少要取得其理解并愿意在家中积极配合患者治疗的承诺。在此基础上,我们针对不同病种设计了单独进行的行为疗法治疗计划,患者可先在医院性治疗室接受除夫妻相互爱抚以外的行为训练。如松弛训练、意念集中训练、自慰技巧训练以及挤捏技巧训练等,并在治疗师指导下观看有针对性的影视教学片,让其初步掌握夫妻间爱抚的技巧,而后按计划回家作有配偶参与的、相互爱抚的家庭训练。在整个家庭训练计划执行期间,患者还需定期接受性治疗师的指导。这和夫妻共同来医院接受由性治疗师直接指导的性治疗相比,显然是事倍功半的无奈之举,但只要家庭训练能按计划坚持完成,仍可取得较好疗效。至于时间方面,由于我们实行以家庭训练为主的改良方案,患者夫妇只需集中来医院2~3次,每次半天(约4小时),这对绝大多数患者而言均不会构成问题。

性治疗通常可分为三个主要阶段,并依次分步进行。

通常包括获得性知识、了解男女间的性差异、患者承认具有操作焦虑、明确对治疗的期望和合理的治疗目标等。其中性知识教育和明确治疗动机及目标最为重要,至于避免患者的性操作焦虑,最有效的办法就是在性治疗初期暂时停止性交,并且不以成功性交作为近期治疗的首要目标。只有达成了这种共识,患者才能抛开对性交不成功(或不理想)的各种顾虑,为进入第二阶段的治疗做好准备。

是为努力实现第一阶段达成的特定性治疗目标。主要包括实践减少造成性焦虑的新行为;应用松弛训练、性幻想训练等增进性唤起;增进双方的性交流以减少对性的误解;以及通过认知疗法改善不良的认知和信念等。换言之,这是行为疗法的操作阶段,要点是鼓励患者夫妻间的语言交流,增进他们之间的理解和信任,只有建立起了巩固的和睦关系,才能进行针对不同性功能障碍的特殊行为训练治疗。

本阶段应是随第二阶段治疗获得初步成效后,随着治疗进程而逐步进行的。在肯定已取得疗效的基础上分析哪些目标已经达到并及时巩固,还存在哪些缺陷和不足并就处理这些问题的策略进行讨论,进而加强相应的治疗措施,以及在完成整个治疗计划后进行总结和评述,也包括对预后的评估。

性感集中训练是一种依据系统脱敏理论设计的行为疗法。它由美国马斯特斯和约翰逊夫妇创建于20世纪60年代,并于1970年在他们出版的《人类性功能障碍》一书中正式发表。该疗法的出现对人类性治疗学起到了划时代的推动作用,也可以说是对前人性治疗技术的总结和发展。他们应用这种技术创造了仅2周的最短疗程和5年随访失败率仅20\%的最佳疗效。因此,它一公布就立即受到世界各国性治疗学家的重视和欢迎。虽然性感集中训练还存在着局限性和缺陷,但至今仍然是各国性治疗机构采用的主要治疗手段之一。

性感集中训练是一种性治疗的技术,其适应证主要是心理性性功能障碍。如男性心理性勃起功能障碍、早泄、不射精以及女性性欲障碍、性唤起障碍、高潮障碍、阴道痉挛、性交疼痛、性厌恶等都属于这一范围。然而所谓器质性或心理性性功能障碍在临床上有时是很难截然分开的。因此有的性学家认为,凡有性功能障碍者,都有不同程度的心理问题存在,只是这种心理因素在性功能障碍的发生中是起直接和决定性作用,还是只起次要的诱发作用,或者是性功能障碍发生后继发或伴随出现的问题。因此,性感集中训练在性功能障碍的治疗中有着广泛的适应证,只不过对单纯心理性的性功能障碍起主要治疗作用,而对器质性的则是起辅助治疗的作用而已。

在中国,性感集中训练还有一个十分突出的价值,即对受治疗者性知识和性技巧的显著提高。当前我国性知识的教育水平和普及程度与发达国家相比还存在着很大的差距,连高等医学教育也少有开设性医学教育的课程,即使在其他的性治疗中(如心理分析、药物、外科手术等)也很难得到性行为的具体指导。因此,性感集中训练就成为一种最有效的性知识和技能的教育方式,它不仅能起到治疗的作用,而且还能有效地提高患者夫妇的性生活质量,并且终身受益。从这个观念出发,除外精神疾病患者,性感集中训练几乎没有什么禁忌证。

进行性感集中训练的一个重要原则是让患者夫妇建立一种强化意识,即在治疗中集中精力去体验愉快的性感受,而不是注意性表现的好坏,更不是追求某种难以达到的目的(如高潮的出现)。在患者夫妇开始治疗前必须对这一点有清楚的认识。该疗法所谓的“集中”就是指这种意念的集中,也即抛开为解决某种导致性功能障碍的不良行为而带来的紧张或焦虑情绪,只集中意念去体会渐渐增强的性感受,这将贯穿于整个治疗过程中。性感集中训练的目的仅仅是恢复患者本来具有的、自然的性能力。随着紧张焦虑情绪的克服,各种压抑自然性能力的不良行为也就会随之消失。医院的性感集中训练治疗室,就是创造一个安全的、不受任何干扰的环境,为患者夫妇提供一个学会以更自由、更愉快的方式做爱的机会。他们将在此学会给予和接受各种各样的性刺激,进行相对不受压抑的性行为。

当然,患者在强化意识和集中精力于良好性感受的体验过程中会受到多种因素的阻抗。如女性中常见的对婚姻的失望导致对发生性行为的矛盾心理,并因此产生对性乐趣的阻抗等。特别是性感集中的原则不仅要表现在医院治疗室,而且要带回到家庭作业中,这就需要事前通过心理疏导使患者夫妇达成愿积极治疗的共识,以及具有按照疗程的安排坚持治疗的耐心和信心。

马斯特斯和约翰逊创立的性感集中训练又称两周强化治疗法,必须由夫妇双方共同参与。即患者夫妇到医院来过两周几乎与外界隔绝的封闭生活,集中接受性治疗师的性感集中训练治疗。由于最大限度的排除了外界干扰和整个治疗计划均在性治疗医师的直接指导和控制下进行,这种两周强化疗法的优越性是显而易见的。但是,多数患者夫妇很难达到集中两周时间的要求,因此治疗也可变为在医院接受治疗指导后,回家依据性治疗医师布置的家庭性作业进行自我训练的形式进行。在中国,绝大多数患者都愿意接受这种家庭性作业式的治疗计划。

在治疗的内容上又可分为常规的、典型的四步训练和针对各种不同的性功能障碍病种,专门设计的特殊训练两类。现将基础的四步训练程序介绍如下:

按照广义的性观念———全身皮肤都能接受性刺激的理论,本阶段的目的就是通过夫妇对彼此皮肤的抚摸来获得性乐趣,并以此缓解患者对性刺激的紧张、焦虑情绪。

为松弛紧张情绪,患者可先洗一个热水澡,而后抛开工作、学习和其他一切杂念的干扰,专注地进入训练状态。性感集中训练的房间布置、色调和光线应简洁、柔和,尽量避免视觉上的不良刺激,室内温度以26~28℃为宜,过于寒冷不利性敏感区浅表毛细血管的扩张和情欲的调动。

常用的抚摸方式是妻子俯卧,丈夫以指腹、指背或唇,尽可能轻柔的从颈部开始抚摸,包括耳朵、耳背后、背部,然后经臀、大腿、小腿至足部,继之仰卧,从上肢、胸部、腹部到大小腿。抚摸胸部时应跳过乳房,决不准许接触外生殖器官。而后夫妇交换位置,由妻子按同样程序对丈夫进行除外阴茎和阴囊的抚摸。由谁开始,抚摸的顺序、频率,手法的轻重和抚摸的时间都不是一成不变的,这需要性治疗医师事前和患者夫妇进行充分的讨论而作出安排。其目的是在夫妇双方都能接受的肉体接触水平上开始训练,并逐步增强亲密感和愉快的性体验。本阶段应注意的问题是:

(1)由于我国传统的性生活都是在夜间黑暗中进行,有的女性对在灯光下裸露会感到羞涩和紧张,因而难以进行从容的爱抚。遇此情况可先穿着宽松舒适的衣服和丈夫进行不会引起任何惊恐或反感的亲昵接触,而后除去衣服,彼此注视裸露的胴体,直到紧张情绪消失后才开始抚摸训练。

(2)对于女性患者来说,本阶段训练除了专注地去体会抚摸带来的性乐趣外,还有一个注意体验和开发自身性敏感区的意义。由于传统文化和道德观的约束,众多的中国女性对自身除生殖器官外还存在性敏感区处于无知状态。常见的性敏感区包括唇、耳根、颈部、乳房、下腹部、臀部和大腿内侧等部位。发现和体会这些性敏感区带来的较强性刺激,有利于性唤起的调动和增加抚摸带来的舒适享受。因此,性治疗医生在开始第一阶段训练前应向患者介绍有关知识,并鼓励他们去发现和体验这些生殖器官以外的性敏感区。

(3)由于男性倾向于追求性器官的直接刺激,因此丈夫常有嫌这种抚摸太慢,太间接和不够刺激的情况。对此,性治疗医生要作出耐心的劝解,因为哪怕是隐晦的不耐烦情绪都会影响妻子的情绪。本阶段的目标仅仅是集中在感觉上而不是任何性能力的表现上。正所谓欲速则不达,只有认真耐心地配合训练,通过本阶段的时间才越快。

在第一阶段抚摸的基础上把范围扩大到乳房和生殖器官。本阶段的目的仍是让患者体会性刺激的乐趣而不是把注意力放在实现某种目标上。对妻子而言,首先要鼓励配偶观察、欣赏自己的外生殖器官。丈夫的手指应随着观察进程十分轻柔地从阴阜往下抚摸大阴唇、小阴唇、会阴体,并可试行用洗洁的手指探入阴道口,体验妻子盆底肌肉收缩的感觉。最后向上翻起阴蒂包皮,观察阴蒂头、体,抚摸阴蒂时要十分小心,以免妻子紧张,可由妻子握住丈夫的手指自己施压,教会丈夫抚摸刺激阴蒂的手法和力度。此后妻子也应训练对丈夫生殖器官的抚摸。男子的性敏感区包括臀部、大腿内侧、腹股沟、阴囊和肛门周围等部位。在抚摸阴囊时可轻柔地抚摸睾丸、精索。阴茎的敏感区在腹侧系带、冠状沟和尿道口。在对阴茎抚摸时也应征询丈夫的意见,以丈夫希望的方式和刺激强度来进行。抚摸生殖器,将有利于缓解某些女性患者对阳具插入阴道的“防御性”紧张,以及缓解某些男性患者面对女性外阴时产生的“境遇性焦虑”。总之在生殖器的抚摸中,一定要充分照顾到配偶的意愿和承受力,不要希望一次就能给对方带来令其满意的刺激,而是需要多次训练才能达到默契的配合。

本阶段训练对取得较好的疗效有十分重要的意义,是消除患者紧张焦虑的情绪,和改善性唤起能力的关键步骤。性治疗医生应协助分析能促进或抑制实现治疗目的各种因素,鼓励患者战胜初期的不适应或失望情绪。对于少数在训练中暴露出来的深层次心理问题,应适时地介入心理分析治疗,对已取得的成绩(如ED患者成功的勃起)给予充分的肯定和鼓励,适时的介入和指导是帮助患者完成本阶段训练的关键。

本阶段是在愉快地完成生殖器抚摸后向性交的过渡,即允许阴茎插入阴道,但不要作抽动性交,故又称“阴道包容”。对于男性勃起功能障碍或早泄的患者,主导权应主要控制在妻子身上。初期可采用女上位,由妻子来控制阴茎插入的方式,并通过训练来逐渐增强插入的深度和时间。而后交换体位由丈夫在妻子的指导下插入,但不能抽动,如此反复练习直到患者完全适应插入并能维持足够的时间为止。在此期间丈夫不能急于求成,只有在得到性治疗医生许可的情况下才能转入性交阶段。对于女性性功能障碍患者的性感集中训练治疗,此阶段可尽量缩短或跳过。

即允许抽动和性交的阶段。其要点仍是把注意力放在体验性交的感受上,而不要追求某种期望的目标。当性活动恢复了其自然性后,勃起或高潮会随之到来。在试行性交的过程中,若患者又出现高度紧张,涌现出大量起干扰作用的念头,甚至回避做爱就应立即停止性交训练。性治疗医生应重新对患者存在的心理问题进行评估,必要时可重复第一、二、三阶段的训练。

马斯特斯和约翰逊的原型设计中,夫妻共同参与是基本原则之一。这是因为夫妇中任何一方的性功能障碍都必将累及到另一方,特别是在病程较长的病例中这种情况就更为明显。因此,如果只治疗其中的一方是很难获得满意疗效的。在这个原则的指导下,就得出当夫妇中的一方不愿参与或不配合时,这种治疗就应暂时终止的结论。不过我们在临床实践中发现这并不是绝对的,如前所述,我们可以用患者单独来院参加性感训练而后回家执行有配偶参与的家庭训练计划,也收到了一定的疗效。特别是对女性患者,在进行典型的夫妇共同参与的性感集中训练之前,先进行单独的自我训练反而有着重要的意义。这是因为女性性功能障碍大多数是心理性的,其成因除和男性相似的以外,还带有更浓厚的传统文化和情感因素的影响,同时与女性的性感觉是后天习得的特性有关。很多原发性女性性功能障碍都源自早期性行为过程未能正常渡过,由于受到各种干扰意念或不良刺激的影响,始终未能把阴道插入后所产生的刺激与愉快的心理享受建立起条件反射式的联系,甚至在潜意识里对这种被插入后的刺激产生“自闭”式的阻抗,因此即使面对做爱时的强烈刺激也坚称“没感觉”、“没快感”。而解决此类问题往往需要患者在不受丈夫干扰的情况下,通过单独的性治疗来去除各种干扰因素,重建正常的性交过程。若非如此,让一个性知识缺乏,性反应尚未成熟或已出现某种性功能障碍的妻子和丈夫在一起进行性感集中训练,就难免会出现事倍功半,甚至妻子始终难以跨越某些心理障碍的效果。因此,在她们和丈夫共同训练前,先进行自我训练是十分必要的。

双重性治疗组是指由一男一女两位不同性别的性治疗医生共同组成一个治疗组来对患者夫妇进行治疗。这也是马斯特斯和约翰逊十分强调的保障治疗效果的先决条件之一。由于异性医生很难和患者进行深入的性问题讨论,特别是涉及患者的隐私时,可能会给患者带来紧张和压力。同时医生对异性患者抱怨的性问题和感受,由于自身性别的差异,有时很难准确地理解,甚至会出现判断上的失误。而这种弊端恰可在双重性治疗组中得到解决。一般做法是先由同性别的性治疗医生和夫妇双方的一方单独晤谈,然后由男、女两位性治疗医生各自充当自己晤谈者的“代言人”,进行病情分析和讨论,最后请患者夫妇加入进行四人的“圆桌会议”,进一步讨论有关问题并初步拟订出患者夫妇能够接受的治疗方案。治疗小组将介入整个治疗过程,当发现某一方出现操作焦虑或意念无法集中等影响治疗计划执行的情况时,治疗小组要随时作出分析、判断,并进行适当的心理疏导,而这一工作最好也由同一性别的性治疗医生去完成。

双重性治疗组在性治疗特别是在性感集中训练中的优越性是显而易见的,但单独一位性治疗医生也不是不可以进行类似治疗。特别是在我国,专业的性治疗学家数量有限,很多刚开展性治疗的诊疗机构还不具备可合作的双重性治疗组。在这种情况下,独立进行治疗的性治疗医生应尽量避免偏袒患者夫妇的任何一方,并将这一观念明确地向患者夫妇双方表明。同时对单独进行治疗的性治疗医生本身也提出了较高的要求,最好由资历较深,有相当性治疗经验和能获得患者夫妇双方共同信任的性治疗医生来担任。只要处理得当,仍然可以获得很好的疗效。事实上,目前我国的性治疗机构多数采用的是单个性治疗医生治疗的形式。

性功能障碍患者的治疗动机多种多样,如来自农村的性功能障碍夫妇,其治疗动机往往不是性功能本身,而是为了生育。还有不少陪同患有勃起功能障碍、不射精等症的丈夫前来就诊的妻子,对自身性生活无快感,无性高潮根本不提治疗要求,只是在需要夫妇共同参与的性感集中训练中才表现出来,而妻子的“性冷淡”恰是导致丈夫性功能障碍的主要原因之一。

更有一种伪装的性功能障碍,其目的是通过蓄意破坏性关系来达到破坏婚姻的目的。如一对中年夫妇因ED前来性治疗中心就诊,妻子表现出了强烈的求治欲望。而在单独晤谈时丈夫表示:“我根本没有ED,和妻子在一起不行是装出来的,我已另有心上人。就请医生您下一个ED而且治不好的诊断,我好和妻子尽快离婚”。这种情况显然已超出了性治疗的范围,应建议他们请婚姻治疗学家处理。

此外,有些患者讲的治疗动机前后不一致,甚至自相矛盾。如一28岁结婚二年多的妻子,因丈夫勃起不坚和自己缺乏性高潮而来就诊,初诊时积极敦促丈夫治疗并表示夫妇感情无问题。但经性感集中训练丈夫症状有明显改善后,却单独来院找医生哭述,对过去2年多所受的委屈愤愤不平,认为现在还要辛苦地配合丈夫做家庭训练,即使治好了自己的付出也太大了,倒不如就此分手,去找寻一个女人应得到的幸福,并因此提出不愿再配合治疗。经性治疗医生与之深入晤谈后得知该妻子早已有离婚的动机,但自己不愿公开承认,积极支持丈夫治疗的另一原因是潜意识里希望治不好,找到自己离婚的理由。现看到丈夫疗效显著,这种由潜意识所支配的焦躁情绪终于爆发。所以,有时要搞清患者来治疗的真实动机并不是一件容易的事。

在我国需要性治疗的病例中,约有一半的配偶不愿或不能陪同前来就医。究其原因大致可分为:

(1)婚姻状况十分紧张或已趋破裂,配偶已不再关心患者的性问题。

(2)性功能障碍主要表现在丈夫一方,而妻子因羞涩不愿陪同就诊。

(3)性功能障碍的主要表现在妻子,丈夫认为自己的性功能很好而无视妻子性问题的存在。

(4)配偶因工作问题(在外地工作或经常出差等)无法陪同患者就诊。

其中第一种情况较为复杂。如前所述,当婚姻状况已十分紧张或趋于破裂时,就必然要表现出形形色色的性问题。此时如果某一方想通过解决性问题来挽救婚姻,那显然是犯了因果倒置的错误,遇到这种情况应当先从婚姻治疗入手。至于无视妻子的性问题既可能是夫权思想在作怪,也可能是性无知造成的,性治疗医生可通过单独约见晤谈或通过电话沟通来促使其配合。

婚姻问题对性治疗的影响已在前面多处提到,由于受到“嫁鸡随鸡”、“嫁狗随狗”等传统文化的影响,婚姻问题对中国妇女性问题的影响远大于西方。在农村或经济落后地区,稳定的婚姻关系常可掩盖妇女存在的各种性问题。另一方面,性功能障碍不一定是严重的心理或病理改变造成,也可能仅是严重婚姻冲突造成的后果。马斯特斯和约翰逊创立的性感集中训练基础就是通过行为语言的交流,引发夫妻自然的性反应和建立起密切合作的关系,而要达此目的的前提必须是夫妻间具有信任和精诚合作的精神。如果婚姻关系失调,这种信任和合作必将受到干扰或破坏,性感集中训练也就失去了实施的基础。如一46岁的女患者因“性欲低下”“缺乏高潮”就诊。5年前其丈夫有了外遇,并生了一个女孩,夫妻已分居3年多,现虽有修好的意向,但双方仍存在很深的猜忌,并难以处置私生女。患者首先询问医生的是:“我们这段婚姻究竟该怎么办?”遇到这种情况,医生若未受过婚姻治疗的专门训练,就最好尽早将患者转给婚姻治疗学家,待患者夫妇恢复了合作关系再进行性治疗。

由于我国绝大多数患者都不能接受集中两周的训练,因此家庭训练计划的执行就显得尤为重要。影响家庭计划执行的因素是多方面的,尤其在女性性功能障碍的治疗中,最常见的原因是丈夫重视程度不够,缺乏耐心。这和男性性功能障碍妻子在家庭训练中的耐心配合形成鲜明反差。所以我们强调女性性功能障碍应先从事个人训练的必要性,这样可以缩短第二阶段抚摸生殖器和第三阶段阴道容纳训练的时间,使丈夫比较容易接受家庭训练计划。另一种情况是医生在布置家庭训练计划前未作必要的心理治疗,或计划本身缺乏针对性和合理性,使患者带着问题进行训练或根本就对训练缺乏信心,以致无法完成家庭训练计划。这往往又是医生屈从于患者急于求成的心理而缩短疗程,或因性治疗医生本身缺乏性治疗经验而造成的。所以,家庭训练计划一定要在搞清楚患者存在的心理问题,家庭、社会和经济条件,真实的治疗动机和可行的目标后,才能慎重而有针对性的作出安排。心理问题不能奢望在短期内就能解决,在作完每个阶段的训练后,患者最好能到医院来和性治疗医生一起作出小结并接受对下一阶段训练的指导,性治疗医生可根据具体情况对训练计划进行适当的调整。至少每个阶段的训练情况都要通过电话和医生取得联系并获得指导。当然,还有诸如配偶出差等客观原因影响家庭训练的。因此,在安排家庭训练时应要求患者夫妇作好相应的时间安排,尽量避免社会因素的干扰。一旦治疗中断,不仅影响疗效,还可能打击患者的治疗信心,造成对配偶怨恨等新的心理问题。所以,如果基本条件不具备,就宁可暂缓家庭训练计划,切不可草率行事。

这是一个受到社会和性治疗学家广泛关注的问题。对任何一位患者来讲,愿意和一位异性医生讨论私生活中最隐秘的部分———性生活问题,已充分显示了他/她对这位性治疗医生的信任和“接受”。尤其在男性治疗医生和女患者的接触中,由于约有70\%的女性患者和丈夫婚姻或感情上存在或多或少的问题,因此在进行晤谈和性治疗中,女患者很容易让能理解其性问题的男性治疗医生代替丈夫的形象。这将是一种潜在的使患者疏远丈夫的危险倾向,是与通过性感集中训练促使患者夫妇建立亲密关系的初衷背道而驰的,这种移情现象十分有害。在对单个女患者进行性治疗时,要注意患者的移情可表现为一种诱惑。当男医生和女患者就性问题进行亲切讨论时,就可能进入性幻想和相互性吸引的阶段。作为一名性治疗医生必须注意到这种潜在的危险,避免使医患关系人格化。对初次接触性治疗的年轻医生来说,可能会出现激动甚至性唤起,这属正常生理反应。类似情况同样可发生在女性治疗医生为男患者作性治疗的过程中。一旦出现这种情况,性治疗医生就必须严格控制自己的情绪和行为,须知在这种情况下和患者的任何亲昵行为都是极其错误的,是违背道德和法纪的。关于这一点,即使是西方性学界也采取了同样的看法,在我国将会导致更为严重的后果。所以一旦发现患者有移情倾向,性治疗医生就应立即采取反移情措施,如以双重治疗组介入、将患者转给同性别治疗医生或暂时终止治疗。

以笔者的经验,移情既不常见,也不可怕。关键在于早期防范和性治疗医生对患者的正确诱导,必要时性治疗医生可以公开提出和患者讨论移情问题,指出有可能造成移情的原因和其对治疗的危害。所以,只要时刻警惕移情的问题,它并不会构成性治疗中特别严重的障碍。当然,若用同性别的性治疗医生来接诊是防止移情现象最有效的手段。

大多数性功能障碍患者都存在心理问题。如果这种问题是来自于深层次的内心冲突,就不能单靠行为疗法来解决。这种冲突常要追溯到早期甚至童年时期发生的创伤性经历,如乱伦(事实上的或幻想的)和强奸等。也可能来自传统观念地反复灌输或潜意识中的恋父情结带来的负疚感等。对这类患者必须通过耐心的心理分析治疗才能奏效。或者在解决了深层次的内心冲突后再考虑作性感集中训练等治疗。心理分析和行为疗法应该相互协调而不能截然分开,更不是对立的。因此,性治疗学家又必然是心理治疗专家。

性功能障碍也可发生在已知的精神病与人格障碍的患者身上。马斯特斯和约翰逊研究所于1971—1977年间,在接受6个月以上心理治疗的夫妇中发现有8\%可诊断为精神性疾病,并在每300名接受行为疗法治疗的患者中发现约有1例会在治疗中发生精神病危象,而需立即接受精神心理治疗。性功能障碍患者合并的常见精神疾病包括:

(1)抑郁症:这是一种性治疗中最常见的情感障碍,可明显损害性欲和性唤起的反应能力。对轻、中度的抑郁症,性治疗仍可谨慎地进行,但对严重,特别是具有悲观厌世甚至自杀倾向的患者,就必须尽早转给精神科医生治疗。

(2)神经症:又称神经官能症,是一种轻型的精神障碍。它包括焦虑性神经症、强迫性神经症、抑郁性神经症、神经衰弱、疑病性神经症和恐怖性神经症等。无意识的内心冲突是神经症的根源,而其核心仍是焦虑。对神经症患者的性治疗和对无精神疾病患者的基本一致,但要困难得多,因为患者对治疗的阻抗和障碍要顽固得多。因此性治疗医生在治疗所需的时间和对疗效的估计上必须有充分的心理准备。

(3)人格障碍:此类患者的内心体验是背离生活常情的,外在行为则是违反社会准则的。所以常给社会和他人造成伤害,同时又给自己带来痛苦。如人际关系紧张、怨天尤人而从不自责,对自己的言行执意的偏袒,缺乏道德观,对旁人猜忌和仇视等。和神经症一样,人格障碍和性功能障碍有着密切的关系,它也给性治疗带来了困难,需要结合精神治疗来进行。因此,应用卡普兰的新性治疗法要比单纯性感集中训练更为理想。不过在通常情况下即使性治疗能改善性功能障碍,却不能改善人格或神经症的症状。所以,精神治疗仍是不可或缺的。

(4)精神分裂症:主要表现为精神活动与现实相脱离,思维、情感和意志行为不协调,相互分裂。其中与性有关的症状十分多见,如钟情妄想、嫉妒妄想、性被害妄想、性色彩幻听等。由于其发病率很高(约占人口的1\%),因此不难遇到在发病初期或两次发作之间因性功能障碍而来就诊的精神分裂患者,性治疗医生必须对此有足够的警惕。一旦发现此类患者或疑似者,应首先转精神科医生诊治并听从他们的建议,对精神状况已稳定,患者的性症状不是作为自我防御形式出现的,并预计治疗过程不会威胁其病情恶化的患者,仍可谨慎进行性治疗。治疗时要注意保护患者的防御心理,千万不能威胁或羞辱患者,以免因发怒或畏惧而激发病情,所以此类患者应由经验丰富的性治疗学家来承担治疗。另一需注意的治疗矛盾是精神分裂症患者所用的药物治疗。因为多数抗精神病药物本身就可引起性功能障碍,性治疗医生需和精神科医生协商来调整二者间的矛盾。因此,取得精神科医生的监护和合作是治疗此类患者的明智之举。

性感集中训练从表面看来似乎比较简单,因此常被初学者误解和滥用,结果并未收到想象中的疗效。事实上性感集中训练对性治疗师的要求是比较高的。要想取得好的疗效,首先是自身要具备较强的专业知识基础和控制整个治疗进程、解决各种干扰因素的能力,而后是选择愿积极配合并有一定理解能力的合适病例,并且要向患者夫妇详尽地解释练习步骤和要领,特别是要让伴侣在参与这一训练过程时发挥主动配合的作用,都是取得高疗效的必备条件。当在治疗过程中患者出现干扰意念或不依从现象时,不应轻言放弃,因过早地中止性感集中训练往往会起到加重双方回避性接触的作用。当然,如果经性治疗师的干预和努力后,患者或其伴侣的不依从行为仍然持续在4周以上,就应考虑改换治疗方案和步骤。

因此,性感集中训练的实际操作不仅不简单,相较于药物和手术治疗还更加繁琐费时,这也是导致其推广困难和众多妇科、泌尿科医师无法介入的重要原因。虽然“适合中国国情的行为疗法”已对马斯特斯的治疗原型、就来医院的时间作了重大的简化,但总疗程仍在2—3个月以上。事实上任何一种心理治疗都不可能一蹴而就,不仅患者需具备信心和耐心,性治疗师也必须具备这种坚韧的素质和奉献精神。

由于性感集中训练直接涉及性行为本身,所以在性治疗中一直是比较敏感的话题。由于我国传统文化根深蒂固的影响,健康性知识的传播受到很大的限制。直到1982年吴阶平教授等编译的《性医学》一书出版后,我国现代性医学的研究才有了快速发展,但对性感集中训练这种治疗方式,社会各界甚至包括部分未从事性学研究医生都持怀疑态度。如:这种疗法的实用价值究竟有多大?它是否适合中国国情?患者能接受吗?它会不会对社会秩序构成不安定因素等。总之,即使性感集中训练在性治疗中有着很高的实用价值,但在实际应用中也要考虑到中国国情的具体情况。为此,笔者提出开展此项疗法必须具备的三个基本条件:第一必须是有资质的正规医疗机构才能开展,而且最好能获当地卫生主管部门的批准。第二是必须具备经专门训练的心理学家和性治疗学家来承担治疗医生。由于性感集中训练并不是一套机械的行为程序,而是一种灵活的心理治疗运作方式,加之考虑到性治疗医生可能遇到涉及范围十分广泛的各种问题,所以从事性感集中训练的医生必须具备性生理、性解剖、性心理、性医学、性药学和性社会学、伦理学、婚姻治疗学等广博的知识。实践表明,患者疗效的高低和从事性治疗医生的专业水平和素质修养有着密切的关系。第三是不能搞配偶替代。如前所述,即使西方性学家绝大多数也对替代配偶持否定态度。在我国更是法律和道德所不允许,这是社会关心的焦点之一,所以在我国应绝对予以禁止。

由于性学研究是一门十分严肃的新兴学科,所以有必要提出上述三个条件以便对性感集中训练疗法的研究和推广进行规范和约束,从而保障我国的性治疗学研究得到更加健康、顺利的发展。

树立女人在性活动中获得快感或高潮是靠自己争取而不是靠男人给予的观念,在女性治疗学中有着重要的意义。由于传统“淑女”意识的束缚和对男性特权的认同,大多数中国妇女在性活动中将自己置于一个从属、被动和压抑的地位,这就极大地限制了女性在性活动中的主动性。因此,女性在性活动中究竟应扮演什么样的角色和处于什么样的地位是在女性性知识教育中必须解决的问题。很多性功能障碍的女性患者在性活动中一方面自己不能放松和把注意力集中在体验性刺激的快感上,另一方面又把自己置于一种审视丈夫性能力,甚至是旁观者的地位上,这种不能全身心投入的状态既会造成自身的问题,又往往是导致丈夫性功能障碍的重要原因。所以必须让患者明白在性活动中男女是“平等”的。女性只有积极主动地参与到性活动中才可能获得高质量的性生活。丈夫应尊重、鼓励妻子的这种积极性和权利,要把夫妻间的正当性行为和所谓的淫荡、风骚严格区别开来。从广义的角度来讲,不仅是对性功能障碍患者,更涉及如何让广大的中国妇女从传统的、封建的性观念桎梏中走出来,这也是摆在性学家和社会学家面前的共同课题。

对于女性性功能障碍的治疗,我们首先强调进行自我性感集中训练,在不受丈夫和其他外界因素干扰的情况下可按以下步骤进行:

患者首先可进行温热水浴,仔细清洁会阴和全身皮肤。然后全身裸露地站在大穿衣镜前,仔细欣赏自己的胴体,特别是挺拔的乳房和后翘的臀部所构成的女性曲线,尽量发现自身的美,忽略可能存在的瑕疵,以增强对自身魅力的信心。同时轻轻抚摸乳房、腹部和大腿,自行体会性敏感区的存在。而后来到床上,半坐卧靠在床头并分开两腿,用一面小镜子照看自己的外生殖器。认清阴蒂包皮、阴蒂头,大、小阴唇、尿道口和阴道口。并用中指试行探入阴道口,同时模拟“排尿”和“急停排尿”的动作,以感受阴道耻骨尾骨肌群的收缩,这种收缩训练将提高性交时阴道感觉的敏感性。通过自我欣赏的训练,将大大提高夫妇共同参与的性感集中训练的承受能力。

患者可取一舒适的侧卧体位,手可轻抚会阴,而后闭目遐想可构成愉悦性想象的任何内容。由于性想象具有不受时空和法律道德观念限制的随意性,所以在调动性兴奋方面有时比局部刺激更富有创造性。幻想的内容可以是小说或电影中有关性行为的描述,也可以是自己的异性偶像或梦中情人,这和对丈夫的不忠有着本质的区别。当然也包括假想和自己深爱的丈夫做爱的愉快场景。总之,幻想可使患者完全放松地去自由想象。从而提高自己对性的兴趣和缓解对真实性活动的紧张情绪。

马斯特斯和约翰逊的实验室研究证实手淫与性交引起的基本生理反应没有什么区别,因此对身体和精神是无害的。这里所谓的手淫是指用手指对阴蒂的刺激。对女性而言,阴蒂是女性最敏感的性刺激感受器,因而手淫是女性除性交之外获得性乐趣和性宣泄的主要途径。在美国,有手淫史的已婚妇女可达60\%,而在中国尚缺乏确切的大量调查报告,但在来笔者性医学门诊就诊的女性中,大多数无手淫经历是肯定的。训练手淫技巧有助于增强性感觉和体验性高潮,在治疗女性性功能障碍中也发挥重要作用。患者可以先将它当做一种自慰技巧来掌握,而后运用到丈夫参与的训练或实际性活动中。在操作中,医生应指导女性患者如何暴露阴蒂头和鼓励其战胜初期对阴蒂过分敏感的不适应,可使用红霉素眼膏等润滑剂来减轻对阴蒂的过强刺激。需要注意的是包括手淫训练在内的女性自我训练的目的是让患者去体验这种刺激带来的愉快感受,而不是为达到某种预期的目标。因为约有10\%的妇女即使通过正规的手淫训练也不会出现性高潮,但这并不影响患者在日后的性活动中得到满意的快感。

如前所述,女性的性感觉迟于性反应,而且是后天获得的。因此在临床上常常见到一种现象:对同一种刺激,有的女性感受强烈,而有的确毫无感觉。这一现象揭示了后者对公认的性刺激和愉快的性感受之间缺乏应有的条件反射联系,所以必须按照学习理论加以训练。关于这一点不少性治疗论著中均有提到,这种自我心理暗示性的训练在女性性功能障碍的治疗中有着重要的地位。如很多女患者可有明显的自慰性高潮,但与丈夫做爱时阴道虽有大量分泌物出现,却仍称“无感觉”。这种灵与肉分离的现象常被称之为“性冷淡”。性治疗医生在对这类患者作心理治疗或性感集中训练时应注重心理暗示效应的训练,即让患者注意体会阴茎对阴道口摩擦和对子宫颈撞击的感觉,并把这种感觉认同为愉快的性刺激,可以结合性幻想尽力使自己兴奋,在集中意念去体验这种刺激的同时,在心理上有意“放大”这些以往并不重视的感觉。如此反复训练,患者将逐步提高对阴道刺激的敏感性,并将这种感受认同为快感。这在原发性性功能失调的治疗中显得尤为重要。

如对阴道痉挛症可作盆底肌群训练,并可在性治疗师的指导下,应用阴道扩张器(一套5具)由小到大地作阴道适应性练习,也可以采用简单的手指插入的训练方法。一般经6~8周的训练后,非器质性原因的阴道痉挛均可得到明显改善。

女性患者通过自我训练基本建立起良好的性反应后,就可将学到的知识和技巧应用到夫妻性生活中去。如有可能,尽量接受夫妻共同参与的性感集中训练,此时将会收到事半而功倍的效果。

性功能障碍不一定是因为夫妻双方或某一方具有严重的心理或生理疾病,而可能仅仅是由于严重的婚姻冲突所造成。造成婚姻冲突的常见原因有:夫妻双方缺乏交流,长期以来的争吵不休,不能满足情感的需求,性生活不和谐,经济方面的分歧,与家庭中其他主要成员的矛盾,婚外恋,有关子女问题的冲突,一方过于好强企图主宰另一方,一方总是疑神疑鬼地怀疑对方等。其中对性生活不满意是十分常见的原因,在一项非专业性的调查中发现,在存在婚姻冲突并需要帮助的夫妇中,有75\%的人同时具有性生活方面的问题。

与婚姻纠纷有关的性功能障碍大体可分成三类。第一类患者是因为性功能障碍造成婚姻关系中继发性的不和谐或冲突,性治疗通常是这组患者的治疗选择;第二类患者是曾有过对性功能具有破坏作用的婚姻纠纷的夫妇,这是最为常见的。总的来看进行性治疗是对其恰当的安排,这是因为夫妻的积极感受及要求改善他们性功能的愿望大于他们相互关系中的消极方面;第三类患者是指那些具有严重婚姻纠纷甚至互相敌视的夫妇,他们不能在性功能障碍的短期治疗中进行起码的合作。

对于第一二类患者,在开始性治疗之前,夫妻双方必须满足某些要求:首先,停止争吵和敌意状态数周,这样可使消极成分不再起主导作用;其次,相互视对方为性伴侣,不能回避应有的性接触;第三,要有真心诚意帮助对方和相互促进的愿望。第四,要求丈夫或妻子将自己的性满足要求暂搁置数周,以便为有问题的一方提供更多的性体验机会;第五,积极参与,但又保持一种无性需求的气氛。能够达到前三条要求的夫妇常常很容易达到第四、五条。如果夫妻不能按这些要求合作,性功能障碍的治疗是不可能取得成功的。

至于第三类患者,则应当首先进行婚姻治疗,性治疗必须在婚姻状况改善后才能进行。

绝大多数性功能障碍患者均同时伴有不同程度的焦虑症状,因此处理好患者的性焦虑成为性治疗的重要组成部分。由于我国的性健康启蒙教育十分薄弱,青少年常常对青春发育中所遇到的生理、心理以及行为方面的变化不知所措,从而因各种性困惑导致性焦虑。临床上最常见的诱因是对手淫的依赖和对其可能损害健康的恐惧;其次是对性器官及第二性征发育的敏感和困惑,如男孩常关心阴茎的大小会对性功能和生育能力产生影响,女孩对经期不规律、面部长痤疮等生理现象感到惶恐和烦恼,以及由于性好奇和性饥渴而对异性产生性幻想甚至单相思等,都可能对其生活和学习产生重要影响。这类问题大多可以通过性咨询、性教育得以解决。

成年人的性焦虑主要表现为对自己性能力的怀疑,如过分夸大自己在性活动中的挫折,并由此产生焦虑或抑郁,可在性活动中表现为冷漠或恐惧,甚至在日常生活和工作中表现为脾气暴躁、灰心丧气、疑神疑鬼等。成人的性困难常源自于配偶间对性活动的不同认识和接纳程度上的差异,如对性交频率、性交时间、性交前爱抚及性交方式、性交体位、插入时间的长短、有无性高潮、有无性交后爱抚等性活动的不协调。另一方面由于性生活不和谐导致的感情伤害,又成为加重性焦虑的重要因素。临床上除通过倾听、晤谈等询诊技术加以判断外,常可借助各种焦虑、抑郁量表来协助作出诊断。部分性焦虑及性困难可通过性咨询和心理疏导加以解决,但大多数需作以松弛疗法和系统脱敏疗法为代表的性行为治疗。而渐进式的系统脱敏正是性感集中训练的基础理论,因此,有夫妻共同参与的性感集中训练应是最佳的治疗方式。对于焦虑或抑郁症状较重者,笔者的经验是在开始性治疗前先用抭焦虑或抗抑郁的药物进行短期治疗,待焦虑或抑郁症状有所缓解并停药后再开始性治疗,减少此类药物因镇静而可能减弱性欲的副作用,性治疗能收到事半功倍的效果。

影像制品在性教育及行为疗法中均有广泛应用。其对象包括医科院校的学生、新婚夫妻、育龄夫妻、老年夫妻以及因性功能障碍或婚姻问题而接受治疗的夫妻等。美国20世纪80年代曾出版过七盘性教育录像带作为医科院校性医学教育的辅助教材,其内容虽丰富,但由于许多方面不符合中国国情而未被采用。由北京高教音像出版社发行的“新婚性指南”和“中老年性和谐”等在普及性教育中曾发挥了积极的作用。

在性感集中训练等行为疗法中,指导患者观看有针对性的治疗光盘已是一个重要的辅助手段。因为通过比讲解更为直观的视听示范,患者更易获得相应的性知识和性技巧。1994年由中国性学会有关专家编译的一套性治疗光盘,包括了ED、早泄、女子性功能障碍和性和谐共计4盘,曾得到我国性治疗机构的广泛应用。遗憾的是,我国尚无自行制作的既符合国情又能满足性治疗需求的影像制品,这将是我国性治疗学家今后努力实现的目标之一。需要明确的是,在性治疗中切不可应用三级片等黄色淫秽影像制品,这不仅违反有关法规,更可能对患者产生误导,加重或诱发新的症状和病情,同时对我国方兴未艾的性治疗事业也将产生严重的负面影响,必须严格禁止。

阴茎负压助勃器是一种已使用了半个多世纪的为帮助患者克服勃起功能障碍而设计的性辅助工具。它由一透明的塑料筒、抽气的负压球和有弹性的压缩圈组成。使用时将塑料筒套在阴茎上并紧抵阴茎根部的皮肤,反复挤捏负压球抽出塑料筒中的空气,阴茎便因负压吸引而逐渐充盈勃起,然后将压缩环迅速从塑料筒底端翻下套住阴茎根部以维持阴茎的勃起状态,撤去塑料筒后即可性交,性交结束后去除阴茎根部的压缩环即可。阴茎负压助勃器的缺点在于其充盈阴茎的原理主要是阻止静脉回流,因此其皮温远不及正常勃起,且阴茎根部仍然是疲软的,同样会给性交带来困难。利用相同原理的还有硅胶助勃套,但也因有降低性感觉等弊端而很少有人采用。

当前,在众多男科治疗室可以看到一种电动的负压吸引助勃装置,用于作为治疗勃起功能障碍的手段。由于可在套筒中放入温水及丹参等带色的制剂,患者在感觉和视觉上将优于自用的负压吸引助勃器。当看到自己膨大勃起的阴茎后,对部分怀疑自己无勃起功能的患者而言,有一定的心理暗示作用。但把它夸大为治疗各种ED的妙方,甚至有增长增粗阴茎的功效就纯粹是一种炒作和误导,一旦患者治无效反而会增加心理负担,甚至降低接受其他性治疗的信心。

电动振荡按摩在性治疗中的应用已相当普遍。在男性性治疗中主要用于不射精症的阴茎刺激。对于多数原发性不射精症患者的治疗,重要的是使其能在清醒的状态下体验初次射精的快感,进而建立起正常的射精反射。此类患者诱发射精反射的刺激阈值通常较高,很难通过阴茎在阴道内的摩擦达到。而电动振荡按摩器的刺激强度则由治疗师控制,当看到自己具备射精能力后,对缓解患者的焦虑症状和增强自信心有很好的功效。大多数不射精患者通过电动振荡按摩治疗后都能实现阴道内射精。因此,电动振荡按摩已成为不射精症的主要治疗手段之一。

在女性性治疗中,跳弹或仿真性具之类的振荡器主要用作自慰或性生活时的情趣用品。美国1983年对15000名妇女进行过调查,其中有半数以上的人承认使用过各种型号的振荡器来帮助获得性高潮,但在我国却受到大多数女性的排斥,因此在性活动中很少应用。但近几年以“摩奇”为代表的电动振荡按摩器的出现改变了这种状况。该按摩器的原设计是一种用于美容院作按摩、消除多余脂肪的设备。在为女性作按摩的实践中,意外发现其能诱发女性性高潮,从而引起性治疗学家的兴趣。该按摩器可变频,最高振频可达4000次/分钟,因此无需直接接触阴蒂头,只需振动阴阜或会阴体即可诱发高潮。在临床应用中高潮发生率在90\%以上,且对原发性和继发性高潮障碍均有效。这就为部分不善或不愿做手淫训练的患者提供了一种器械辅助的治疗手段。目前这种按摩器(或类似的按摩器)已广泛应用于女性美容院,甚至被作为吸引顾客的一种手段。但如果产生了对这种强烈刺激的依赖,将存在疏远其与配偶做爱兴趣的风险。因此,我们认为对女性性功能障碍的患者而言,仍应到医院接受正规的性治疗,因为即使在美容院获得了性高潮却并不能解决心理上、感情上和与丈夫性生活中出现的诸多问题。但是,由于我国目前开展女性性治疗的机构有限,可以预见在相当长一段时间内,仍会有部分患者求助于女性美容按摩院,如何对其进行规范和管理,尚需进一步探讨。

性治疗的疗效受到多种因素的影响,由于缺乏公认的、科学的规范研究方案,因此不同的学者、不同的研究方法和研究对象其研究结果也就存在很大差异。比较权威的是由英国Hawton主持的由62位性治疗学家参与的对154位性功能障碍患者所做的标准化前瞻性研究,研究对象包括:男性勃起功能障碍34例,早泄14例,不射精5例,性欲低下2例;女性性欲低下57例,阴道痉挛26例,性高潮障碍8例,性交疼痛4例,其他4例。治疗结束时总疗效为:满意42例(27\%),基本满意50例(32\%),有进步27例(18\%),无变化32例(21\%),变得更糟3例(2\%)。也即成功率(满意和基本满意)为59\%,无效率(无变化和变得更糟)为23\%。这虽然比马斯特斯报道的成功率要低,但却客观可信。而按病种分类统计成功率,阴道痉挛为81\%,勃起功能障碍为68\%,早泄为64\%,女性性欲低下为56\%,性高潮障碍为50\%,性交疼痛为50\%,不射精为20\%。在我国尚缺乏类似Hawton所做的标准化前瞻性研究,笔者应用“适合中国国情的性感集中训练”,在重庆市性治疗中心所取得的近期疗效(隨访1~5年)有效率,男性勃起功能障碍为54/60(90\%),早泄为109/120(90.83\%),不射精为41/52(78.85\%),女性性功能障碍疗效如表:

女性性感集中训练疗效

分病种统计有效率以性交恐惧、阴道干涩、高潮障碍、阴道痉挛为高,性欲亢进、性欲低下疗效较差。本组统计的有效率较Hawton的研究普遍为高,这可能与我国因性知识教育缺乏所致的性功能障碍明显高于西方发达国家有关,其中在女性高潮障碍和男性不射精症的疗效方面表现得尤其突出。5年以上的远期随访疗效在我国尚鲜见报道。为此,胡佩诚等设计的“城市男女性治疗质量问卷”问世后,相信将有助于加强这方面的工作。

从患者的条件来看,影响性治疗效果的因素主要有三方面:婚姻整体状况;性关系总体状况;对性治疗的积极性。性伴侣间婚姻整体状况好的,治疗效果显著强于整体关系差的,而且在评价整体关系时女性的意见和考虑将更为重要。性关系总体状况直接反映出性功能障碍的性质和严重程度,这也和婚姻状况相联系,如治疗前有过分居经历的人治疗效果明显低于从未分居的夫妇。在参加性治疗的积极性方面,特别是在能否坚持执行家庭训练计划时表现出来。为此,争取在4—6周内取得较明显的进步,对鼓励患者坚持执行家庭训练计划和提高治疗效果有重要意义。

在中国开展性治疗所面临的困难,除传统文化的束缚和性知识教育普及的不足外,时间和经费承受力也是常遇到的问题。很多患者鼓足勇气来医院就诊是抱着试一试的态度,希望通过1~2次就诊就能解决问题。正如一位女计算机工程师所说:“我以为就像一层窗纸,被您一点破就迎刃而解了”。因此多数患者没有按计划接受治疗的思想准备,加之部分患者远道而来经济承受能力有限,一旦经过1~2次门诊未见明显效果就可能放弃治疗。在某些文化程度不高而又迷恋药物的患者中,即使通过行为治疗性功能有了明显进步,但却始终对未吃到药而不能释怀,总希望能得到“灵药”的帮助。面对这些现实情况,笔者认为除非是开展学术研究,否则不必在治疗中拘泥于某一种疗法。如对勃起功能障碍或早泄患者,在通过性感集中训练第1、2、3阶段的练习后,第4阶段(性交训练)就可辅以阴茎血管活性药物注射或口服西地那非(万艾可)、伐地那非等,以帮助阴茎的勃起和维持足够的时间,从而帮助患者尽快建立起治疗的信心,待较满意的疗效显现后,再根据具体情况考虑逐步减量或终止药物辅助治疗。又如单身女子作自我刺激训练时,电磁振荡等自慰器具常是有效的辅助措施。在性感集中训练期间,不少患者提出是否可同时服用滋阴补肾的中药———这也许是中国患者特有的问题。笔者认为只要有利于增强患者的治疗信心而不会过多增加经济负担就可,中西医结合本身就是我国性学研究的发展方向,完全可以兼容之。总之,从临床治疗学的角度来看,只要能缩短疗程,提高疗效,根据患者的具体情况采取综合性的治疗措施都是可取的。只是在性感集中训练的基础上如何较规范的应用其他治疗措施,以及滋阴补肾或壮阳的中药对提高性治疗的疗效究竟有多大的帮助,目前还缺乏系统的、随机的、多中心、双盲对照性研究。这也是需要我国性治疗学者今后进行深入研究的课题。

(关仁龙 徐晓阳)


\section{第四节 性障碍的分类原则}

根据通常标准,性障碍的分类原则如下:

性欲低下、性欲亢进、性厌恶。

男性:勃起功能障碍(ED)。

女性:性唤起障碍(阴道干燥)、阴道痉挛、性交疼痛。

男性:早泄、射精迟缓、不射精、逆行射精、射精痛。

女性:性高潮障碍。

1.性身份障碍(性别转换症)。

2.性偏好障碍(性欲倒错)。

3.性取向障碍(同性恋中极少数自我适应不良的人)。

4.脑器质性疾病和精神疾病引起的性行为障碍(继发性,如老年痴呆猥亵幼女往往发生于性欲已消失多年之后,突然对女孩感兴趣)。

患性功能障碍的患者往往求助于泌尿科、妇科及新兴的男科、性医学科临床医生。患性心理障碍的患者则在精神科求治,国内少有医生专攻性心理障碍。

鉴别诊断要注意,如喜穿异性服装有几种情况:异装症者可从中得到性的满足;同性恋者自己并无快感,主要是吸引同性的注意力;性别转换症者主要是厌恶自己的生殖器,仅穿异性服装并不能给他们带来满足。

治疗性心理障碍的原则为自愿求治,成功取决于真诚的合作。患者往往在反复被抓、劳教、离婚等不得已情况下才求治,所以惩罚和处理是必要的,可以督促他们求治,这些患者不仅仅是追求性的满足,而且以此对抗孤独感和性压抑,就像借酒浇愁。治疗医生要与患者讨论治疗计划,坦诚相待,交代各阶段的治疗目标,取得患者的信任。除医生外还应取得单位、家庭、亲朋等多方面的帮助,调动各种积极因素。采取领悟心理疗法、行为(厌恶)治疗并配合心理治疗、药物治疗,如安定或三环类抗抑郁药等可解除患者的烦恼。

如公开的性功能障碍可包括下列几种情形:有患者自己知晓的,如ED;有从现有主诉中推断出来的,如患者明知有ED却以不育为主诉;有从采集病史时得到的与性无关的主诉中获得的启示,如患者明知症结的所在却否认它的重要性,作为自我防卫机制而掩盖事实或轻视它。当患者与医生相处自如后,这些性功能障碍都会被发现。一旦症状或综合征状公开之后,医生的治疗处理就轻松多了。

隐蔽的性功能障碍指那些并未在患者头脑中与他现有的症状联系起来的性问题。这些症状包括疲劳、头痛、背痛、胃肠不适、月经周期紊乱或痛经。查明并确认患者的某些症状与性挫折之间的联系就成为治疗学家的任务。

如果先前有过正常的性功能表现,然后出现性功能障碍就称之为继发性的;如果从来没有达到能插入阴道程度的勃起,那就是原发性的。这一划分对于性功能障碍原因的诊断评价是十分重要的。有些原发性的是终身的,无法治疗,有些是可以得到纠正治疗的。

如果ED或无高潮是境遇性的(在特定情况下或与特定对象时发生),就可以肯定这种障碍无疑是心理性的(除非单纯因乙醇或药物引起);如果性功能障碍发生在所有情况下,即完全性的,它可能是心理性的,也可能是器质性的,或两者兼而有之。最典型的情况是性功能障碍仅局限于婚姻内部,因此,医生要检查婚姻关系的本质如何。如果在夫妻之间存在经济、情趣、志向等其他非性方面的冲突,迟早会影响到夫妻间性生活的和谐,这是非性因素所致性功能障碍中常见的病因之一。

由于焦虑或其他消极情绪所导致的性功能障碍是内心因素引起的,因此,所有的性功能障碍都具有内心世界的成分,在这种情况下,内心冲突和人际关系的区分意味着可以追踪占据主导地位的因素。如果妇女在婚前的性关系中就存在无性高潮的消极表现,其病因很明显是原发性内心因素的。如果曾有过一段良好的性功能行使时期,由于婚姻冲突而导致的障碍无疑是人际关系占据主导地位,尽管如此,它也必然有内心因素的成分存在。

不论是心因的或人际的,性功能障碍的原因可以是久远的,也可以是即刻的。即刻因素诸如行为焦虑、监视自己的性表现或作为一个旁观者、对性伴侣的反应程度过分敏感,他们往往回避性接触,也回避谈论性问题。这种影响往往具有短期效果。而久远影响比较恒定,一般难于逆转,如过去创伤性性经历造成的性功能障碍。

前者如糖尿病,是非常典型的医源性原因;后者如抑郁症,还有其他精神疾患虽不直接影响性功能,但因所服药物具有中枢抑制作用,所以也间接造成性功能障碍。

非器质性的多指性知识缺乏、婚姻关系紧张、环境不良因素影响等所致;器质性的则由于器官病变等所致。心理性与器质性因素还可共同存在,称为混合性性功能障碍。

(马晓年)

1.Masters WH,JohnsonVE.Human Sexual Response.London:Churchill,1966

2.吴阶平.性医学.北京:科学技术文献出版社,1983

3.姚树桥.医学心理学与精神病学.第2版.北京:人民卫生出版社,2007

4.马晓年.现代性医学.北京:人民军医出版社,1995

5.徐晓阳.性医学.北京:人民卫生出版社,2007

6.胡蕾,胡佩诚.女性性生活质量问卷的编制和信效度检验.中国心理卫生杂志,2008,22(6):447-450

7.邸晓兰.网上的性心理咨询.中国性科学,2003,12(12)增刊

8.关仁龙.行为疗法治疗早泄的临床研究(附120例报告).中国男科学杂志,2003,5(17):311,313

9.关仁龙.行为疗法治疗勃起功能障碍的临床研究(附60例报告).中华综合临床医学杂志,2003,3(5):10-11

10.关仁龙.行为疗法治疗女性性功能障碍56例报告.中华男科学杂志,2003,5(9):335-337


\chapter{第六章 性欲异常}

性欲(sexual desire,sexuality)是指由机体内外的各种刺激引起性兴奋,进而产生企图主动通过与异性(或性对象)完成身心结合(即性活动)而达到满足和获得乐趣的一种主观欲望和兴趣。性欲是人的本能,也是整个性活动过程的启动因素和中心要素。不难看出,性欲首先应定性为一种主观欲望和兴趣,属于精神活动。诊断和治疗性欲异常,必须先明确何谓性欲正常。通常将同一历史阶段、相同社会环境、同一人种中同一年龄阶段的大多数人所具备和认可的范围定义为正常,而超出此范围的定义为异常。但临床工作中定义性欲异常时还必须充分考虑到超出正常范围的性欲变异所持续的时间、影响和危害程度,同时患者及其性伴侣的个体差异性也必须考虑在内。

同所有精神活动一样,性欲的产生必须具备其客观物质或称生理基础,人类的性欲产生于青春期生殖器官及第二性征发育之后。现代医学研究证实:人的性欲受到大脑灰质、下丘脑等部位的“性中枢”的控制。性欲的生理基础就是机体对性刺激作出反应而产生的一系列生理变化,多巴胺、催乳素、5-羟色胺、α-肾上腺素能阻滞剂、睾酮等激素对性欲的维持和调节至关重要。性欲的体现———性兴奋不仅需要性敏感区(也称发欲带,指产生性兴奋时机体特别容易接受性刺激的敏感区域,多位于生殖器、胸部、大腿内侧等部位)的存在,而且需要性刺激(sexual stimulate,指能导致和唤起性欲的各种因素、刺激方式或作用)。不论是直接的触觉、视觉刺激,还是更高级、抽象的嗅觉、听觉,其产生都不能脱离相关的身体感受器官。而且只要性刺激感受器、传入神经、性中枢、传出神经和性效应器官(生殖器)这一反射弧健全,有时即使主观意愿(性兴趣)低下甚至反对时,性兴奋及其实践———性活动仍可以完成。所以我们可以观察到部分有智力障碍的患者其性欲(并非性心理)仍可能正常。

当我们谈及影响性欲的因素时,不可避免地涉及以下因素:①影响性欲的生理性因素如:遗传因素、年龄因素、体质因素、内分泌因素和相关疾病等因素;②直接影响性兴奋和性兴趣的因素:如性欲带(即性刺激敏感区域)因素、性敏感度因素、性伴侣的配合程度及性活动获得的满意程度等因素;③影响性心理的因素:精神情感因素、对性相关知识的认知程度等意识形态因素以及宗教文化因素等社会环境因素。此外,某些食物和药物因素也可造成人为的影响因素。对性欲影响因素的认识有助于对性欲异常的诊断和治疗。

同其他物种相比,人类的繁衍无疑是极为成功和高效的,并为其成为地球的主宰提供了最基础的数量保证;大多数物种都有相对规律的性兴奋增强时期,即发情期,而人类一年之中性欲几乎无明显的季节性差异;与其他以生殖为主要目的的物种不同,人类多数的性活动发生在并不宜受孕的时段,可以说人类的性活动主要是为了获得乐趣、满足,同时也是为了满足性伴侣的要求,甚至在特定环境下可成为一种谋生手段。与其他高等动物相比,人类(尤其男性)的性欲可从性成熟初期一直维持到衰老期,并可随人类寿命的增加而延长。

性兴趣,这一希望通过性行为获得乐趣和满足而非纯粹生殖目的的主观欲望无疑为人类所特有。性幻想,多数成年男女都有单纯依靠主观性幻想而获得性兴奋甚至进而通过手淫获得性满足的经历,多数性幻想由视觉、听觉、嗅觉和触觉诱发,但有时性幻想在缺乏外界客观性刺激时也可由单纯的主观臆想产生。人类大脑甚至可以在不依赖任何生殖器刺激的情况下自行产生性高潮,男性表现为梦遗,女性则可以通过仪器检测记录到心率、呼吸频率明显增快、阴道血流量增加等。此外,性幻想不仅可以增强或减弱听觉、嗅觉、视觉甚至触觉等性刺激对性兴奋产生、发展的影响,还能够影响人们从性活动获得乐趣和满足的程度,例如用肠道构建阴道的患者在性交时依然可以产生快感甚至体会到性高潮。

性欲本身由多因素构成,主要划分为性兴趣(sexual interest)和性兴奋(sexual excitation)。性兴趣,或称性欲望,是指其希望与异性完成身心结合来获得乐趣和满足而非纯粹生殖目的的主观欲望,是性欲的本质、动力,为具有思维能力的人类所特有。性兴奋,又称性冲动、性唤起(sexual impulse),是指由机体内外的各种性刺激而引起的机体的一种特殊的有效反应,进而可激发起机体进行性活动的强烈欲望和冲动,属性欲的体现。

性兴趣和性兴奋二者既相互独立,又互相依赖、影响、调节和转化。进一步分析,性兴趣受性心理的支配、影响,可引起并调节性兴奋,进而将这一影响表现到具体的性活动中;而性活动中所获得的满足、乐趣的程度又转而反馈、影响性兴趣,进而反馈、影响甚至改变性心理。在“性心理———性兴趣———性兴奋———性活动”这一过程中,上一级的活动可引起并调节下一级的活动,其作用如同电路中的二极管,可将下一级的传出信号放大或减小,而下一级的活动同样可通过反馈影响上一级的活动,从而完成其循环。

应当说明,性兴趣和性兴奋二者虽然是相互依赖并影响的关系,但也有可能发生分离现象。例如:绝大多数男性青少年在青春期因身体发育而自发产生强烈的性兴奋(或称性冲动),并导致手淫或遗精,由此可获得性快感,但不少青少年由于对性知识的缺乏和某些错误观念的引导(如认定手淫会导致“肾亏”或所有的性活动都是羞耻、肮脏的等等)而产生恐惧甚至罪恶感,此时其性兴趣降低甚至缺乏,但性兴奋却增强甚至难以控制。而临床上常有中年男、女性患者虽然希望与性伴侣完成身心结合,获得乐趣和满足的愿望(性兴趣)正常甚至强烈,但因某些生理、心理的原因(如疾病、体质较差、工作和社会压力过大等,甚至无明确的原因)却往往无法通过正常性刺激产生应有的性兴奋(或称性冲动),或许阴茎勃起反应或阴道充血、润滑良好,但性交时觉得“没兴趣”,更无法获得足够的性快感。

根据性欲的来源,部分学者又将性欲分为:背景性性欲———是由性激素所产生、决定的,它维持性欲的持续张力、紧张性和兴奋性;应激性性欲———指由受到内生的或外来的性刺激而引起,能保持机体对环境性刺激的有效反应,激发或形成突出的性欲冲动并导致性行为。而根据其目的,又有学者将性欲分为:解欲的目的(又称性交欲)———即指促使狭义的生殖器官发挥功能、排出积聚的性腺分泌物的目的;接触或称厮磨的目的(又称接触欲)———促使两性发生身体和精神上的密切接触。二者往往都通过性交来获得最终满足。

一般而言,男性的性活动比女性更有主动性或者更具决定性,所以一旦男性出现性欲异常,对两性关系的危害性要比女性患者的影响大。

有关妇女性功能的最新构想强调了女性性欲的互动成分。图6-1描述的循环模式解释了对女性性欲的现时理解,即性欲如何在性约会过程中得到激活并增进了任何原有的欲望。研究证实妇女提出各种各样的理由和诱因参与到性活动中。具有性含意的刺激会整合到性反应之中,而且将在性功能障碍的诊断和治疗方案中加以评估。

图6-1 在任何一次性机遇中,由相互重叠的性反应各阶段组成的循环的性反应周期可以体验许多次。最初可以有、也可以没有性欲:它是由性刺激引起的唤起所激发的。性的和非性的结果影响将来的性积极性。

本章节讨论的性欲异常主要有:男女性欲亢进、性欲低下和性厌恶。而性心理异常(或称为性变态)将在其他相关章节阐述。


\section{第二节 男性性欲亢进}

男性性欲亢进(male hypersexuality)是指男性以持久地对性活动要求过于强烈为主要特征的疾病,又称性欲过盛、性欲过旺,中医常称为阳事易举。

性欲亢进可分为:性兴趣亢进和性兴奋亢进(即性冲动过度强烈)。二者常具体表现为:对性行为迫切要求、性交次数增加、性交时间延长等,甚至不考虑条件和场合去寻求性接触,严重时必将影响患者的生活、工作、社会交往,乃至发生犯罪行为。

性欲亢进的发生率很低,约占普通人群的1\%,男性稍高于女性。单纯、原发性性欲亢进者更为少见。在神经、精神性疾病的患者中其发病率较高。

现代医学认为,性欲亢进的主要发病机制是性中枢兴奋过程增强。绝大多数是由于精神心理失调或对性知识认识不足而产生的焦虑所导致,少部分是由于病理改变而引起的器质性病变。其常见病因为:

受某些性文化,尤其是色情小说、黄色录像、色情服务等的影响,过度刺激导致患者性欲失常而长期处于亢奋状态;存在精神疾病或认知障碍,如躁狂症、精神分裂症等。值得注意的是,由于性活动能给人带来乐趣和性高潮后的身心放松,部分患者将性生活作为一种减压、发泄情绪甚至逃避现实的方式,即使性兴趣没有那么强烈,也会寻找额外的性刺激来激发高昂的性兴奋,现实生活中将性活动当成缓解压力和情绪调节手段的人并不少见。如有些中年男性尽管工作紧张、思想压力大,但还是热衷于“找女人”,情感受挫时频频寻找“一夜情”。但这种阶段性的性欲亢进常因持续性心理压力过大或身体健康状况的恶化最终转为性欲低下。此外,部分人会把性交视为掌控别人的工具,防止爱侣变心出轨、驱使对方臣服。

性欲亢进多继发于各类疾病引起的神经内分泌失调,如垂体肿瘤、睾丸间质细胞瘤、颅内肿瘤等;或起源于垂体前叶促性腺激素或雄激素分泌过多、缺少抑制;或因大脑、下丘脑对性激素的敏感性增强所致;此外,也可因性欲带对性刺激的过度敏感或其传入的性信号被过度增强而导致。人体大脑的工作是最精密的,更需要良好的协调性,任何脑部的疾患都可能对其功能产生不同程度的影响,而其中对社会规范的适应是人成长过程中通过学习建立的,也是老年痴呆患者最先、最容易失去的,最典型的表现就是对性方面的抑制解除。例如前叶脑白质切除后,部分患者会性欲亢进,理智性行为减少,患者能感觉、理解自己的“过分”行为不被社会允许,但并不以为然,更拒绝承担责任。有的患者喜欢动用暴力,也有人为配偶的暴力而对性活动收敛,但也可转为在外寻求快感。

如直接使用促性腺激素类、睾酮类药物,长期服用可以导致体内此类激素浓度升高、代谢下降的药物、食物,均可导致性兴趣的亢进。使用可提高神经兴奋性和发欲带敏感度的药物,如服用某些“壮阳药”、“春药”引起体内多巴胺、催乳素、5-羟色胺等神经肽的紊乱则可导致性欲亢进。既往的临床研究发现,电刺激或注入乙酰胆碱等刺激大脑部分区域(如:隔区)可以直接引发类似性高潮的感觉、体质和情绪上的明显变化,部分患者可表现为性欲亢进。

男性性欲亢进常表现为男性患者对性行为迫切要求、出现频繁的性兴奋、性交次数增加、性交时间延长等,盲目、随意地选择性伴侣,甚至不考虑条件和场合去寻求性接触,对患者的生活、工作甚至社会交往都会产生不良影响。

因男性在性活动中多数处于主动地位,且体格大多较女性强壮,容易对女性造成严重的身体创伤。部分患者为满足失控的性欲甚至不顾法律、道德约束而犯罪。

根据性欲构成可将性欲亢进细分为:①性兴趣亢进:表现为对性活动有超常的兴趣、呈现一种强迫性的需要,不考虑任何条件和环境的约束,不断有性交欲望或频繁出现性幻想;②性兴奋亢进:频繁出现性兴奋和性冲动现象,对非性刺激的正常举止、言谈反应强烈并迅速将其转化为性活动的原始推动力,稍有外界刺激(如看到女性正常暴露的肢体、躯干部位,与异性正常的言谈或轻微的肢体碰撞、接触)或内源性刺激(如脑海里突然出现的任何与“性”相关或不相关的事务)阴茎便会勃起、有迫切的性交和宣泄欲望。患者夏天不敢去游泳、不敢把上衣束起来甚至不愿到女性较多的场合中等。而且这类冲动不易抑制,高潮后迅速再发,性欲与理智的冲突给患者本人带来精神痛苦。上述性兴趣亢进和性兴奋亢进既可以同时存在,也可以单独出现。有些男孩子在青春发育期出现阶段性的频繁阴茎勃起属于生理表现。

此外,临床上部分癫痫患者在癫痫发作前可体验到性高潮样感觉,称为“性高潮型先兆(预兆)”,患者甚至因此拒绝接受癫痫治疗。有些癫痫患者平时是“性欲低下”表现,一年性高潮可少于一次,但是在接受颞叶手术治疗癫痫后反而表现为“性欲亢进”,因此手术前应当考虑这种情况的发生。

性欲亢进的诊断应当建立在耐心的病史询问基础上,询问中医生应当注意态度的严肃性和话题的灵活性,对不愿主动诉说或难以把握主题的患者可以从“你总是为异性而注重打扮吗?”“你觉得和异性之间很难维持纯粹的精神恋爱吗?”“你认为没有爱情的性行为是不合适的吗?”等问题开始。进而可以问“你曾经担心自己的性欲太强吗?”“你常故意压抑自己的性欲吗?”“你经常梦到性爱的情景吗?”“你会整天都对性爱产生遐想吗?”“没有性生活(或手淫)你仍能过得愉快(或感到难受)吗?”再循序渐进地进入“你喜欢阅读色情杂志吗?”“性爱是你人生最大的乐趣吗?”“你希望与陌生人发生性行为吗?”“你愿意参加性杂交的聚会吗?”等。在询问过程中要了解患者是主动就诊还是被迫就诊,对答是否合理,反应是否敏捷,逻辑思维是否健全等,以及初步判断性欲亢进属于原发还是继发,譬如继发于某次头颅外伤等。必要时须对患者家属、性伴侣进行耐心询问。

体格检查的重点是第二性征的发育情况,必要时行前列腺指检及前列腺液常规检查。

基本实验室检查为性激素,必要时行染色体检查,其他神经内分泌等实验室检查,头颅CT等影像学检查。

1.性兴奋增强,性欲要求强烈,性交或性幻想过度频繁,有与其年龄不相适应的性要求,虽有性交的全过程但难以满足。

2.性要求不考虑任何条件、女方意愿和环境情况。

3.因性欲亢进导致精神痛苦,影响正常生活和身心健康,造成人际关系不和谐,甚至影响社会交往。

4.上述症状必须持续一段时间,多定为3个月以上。

5.往往伴有内分泌失调,性激素检测异常或某些性欲带敏感性过高。

6.有躁狂症、精神分裂症等精神性疾病或有垂体肿瘤、睾丸间质细胞瘤等器质性原发病史。

以上1至4项为性欲亢进的必备症状诊断依据,而5、6项为患者的常见病因学诊断。应当注重其诱因、病因的诊断,以期能达到彻底治愈本病的目的。

性欲亢进应和性欲旺盛相区别:性欲亢进并不同于性欲旺盛,性欲旺盛是中性甚至褒义词,但性欲亢进则是病态的。二者都以对性活动要求强烈为主要特征,而不同之处在于前者具有强迫性、难以抑制、难以用理智调控,即使有高潮也得不到满足。而性欲旺盛多为青春期、接触性活动初期、长期禁欲后的一时性改变,随性要求的满足可逐步缓解、恢复正常,持续时间较短且理智调控良好、对性对象和社会危害性小。

性欲亢进的诊断必须排除青春期、接触性活动初期、长期禁欲后等正常性欲增强的表现。

性欲亢进的治疗原则除缓解症状、控制病情外,更强调注重病因治疗、预防复发。应当根据患者的具体情况,提出针对性的个体化治疗方案。

必须详细询问病情,分析患者致病原因,有针对性地作正面引导,纠正其错误认识,解除患者思想上的种种焦虑,帮助患者树立正确的人生观和道德观,努力提高文化素养,建立高尚情操,减少色情事物的刺激。研究发现,当把精力集中于某种活动时,会在大脑相应区域形成兴奋灶,而其他区域的兴奋则受到抑制,可建议患者将精力放在工作和学习上或培养可替代的兴趣、爱好。通过行为疗法减少患者自发刺激性欲带的行为。加强全民性知识普及教育,使得患者一旦出现某些性方面的问题,能够及时发现并找有关医生咨询,争取早诊断、早治疗。对精神疾病或认知障碍的治疗详见相关章节。

当患者症状稳定后,可考虑调查其发育过程中对疾病发生有影响的因素:①家庭方面的问题,如评价家庭成员中的精神疾患及其影响。②查询有无肉体或性虐待、从小遭双亲忽视、过早发生性活动等在内的能影响早期性行为的事件。③建立和塑造“假自我”,以便把非性欲倒错性性欲亢进和痛苦情愫的处理区分开来。④探查使性症状形成和持续的可能的心理动力学或行为背景。⑤评价并发的精神疾患对其性心理发育和形成的影响。

有针对性地治疗各种原发病,包括药物、手术等。对单纯内分泌失调、睾酮含量明显偏高者,可以使用雌激素等抗雄激素治疗,如:己烯雌酚片1mg,口服,3次/天。同时可加服镇静剂对症治疗,如安定片2.5~5mg,口服,l~3次/天。曲普瑞林是促性腺激素释放激素类似物,有报道用于治疗非倒错性性欲亢进,肌肉注射3.75mg/次,治疗两次以后LH、FSH开始降低,65\%患者血睾酮水平减少50\%。

防止服用可使促性腺激素、睾酮浓度升高的药物、食物。若处于治疗用药阶段,可通过减少药物的剂量或改用其他药物治疗,以减少产生性欲亢进的机会。对须长期服药治疗精神疾病而引起体内多巴胺、催乳素、5-羟色胺等神经肽类紊乱,导致性欲亢进的患者,应在精神专科医生的监测下减少药物的剂量或改用其他药物代替治疗,必要时加用镇静类药物拮抗。

对待性欲亢进,“疏导”远重于“禁锢”、“对抗”,尤其应通过与其家属、配偶的沟通,增进了解,切忌对患者冠以“色情狂”、“花痴”、“强奸犯”等侮辱性称呼。长期的压抑可导致以下情况:①逆反心理加重的性欲亢进,进而失控,导致道德败坏,甚至性犯罪;②产生巨大心理冲突、痛苦,最终转为性欲低下甚至性厌恶;③诱发或加重精神疾病。

普及科学的两性教育,树立正确的人生观和道德观,提高文化素养,建立高尚情操,减少与色情事物的接触,集中精力于工作和学习,培养良好的兴趣、爱好无疑是最好的预防方式。此外,应当避免滥用“壮阳药”、“春药”,服用可导致促性腺激素类、睾酮类激素升高的药物需严格遵从医嘱、定期复查。对须行中枢神经系统手术、治疗或直接刺激的患者,应在治疗前充分考虑到治疗后的并发症、副作用并向患者配偶、家属详细解释,减少误解。


\section{第三节 男性性欲低下}

男性性欲低下(male hyposexuality)是指成年男子持续或反复地对性幻想和性活动不感兴趣,出现与其自身年龄不相符的性欲望和性兴趣淡漠,进而表现为性行为表达水平降低和性活动能力减弱,甚至完全缺乏。

现实社会中,男性性欲低下较性欲亢进更常见。如上所述,因绝大多数男性在性活动中处于主动地位,男性性欲低下的后果及对双方的危害远高于女性性欲低下,严重时可直接导致婚姻的破裂。而且历来性功能良好甚至性欲旺盛是男性值得标榜的资本,性活动能力的减弱使男性性欲低下的患者极易感到自卑、耻辱,引起抑郁等心理障碍,陷入“性欲低下→抑郁→性欲低下加重→更抑郁”的恶性循环。所以对男性患者而言,性欲低下更容易危害其生活质量和社会交往。

男性性欲低下可以是独立的性医学疾病,也可以继发于其他性问题,性欲低下病因多以心因性为主,常见致病原因有:

精神心理状态和社会、人际、环境关系的异常抑制性欲的产生,是最常见的病理因素。心理素质较为脆弱、紧张者,更易受外界影响,从而产生焦虑与压抑交织的心理紊乱状态,干扰大脑皮层的功能,导致性欲低下。而现代生活日益增强的竞争机制更使得性欲低下的发生率呈上升趋势。常见因素有:①缺乏性教育或受到错误的性教育,存在对性生活的恐惧心理,如对性卫生状态感到忧虑、害怕感染性病等;②既往有因性生活不成功或不和谐而被对方责怪、嘲弄的经历,这在青年男性患者中尤其多见;③宗教戒律和民族、社会传统的束缚;④夫妻感情不和、家庭生活不和谐,甚至对长期同一性生活方式(如体位等)厌倦、缺乏激情;⑤有婚外情或婚外性生活史,从而产生压抑和罪恶感;⑥工作压力大,工作受挫折或打击;⑦人际关系不协调、安全无保障等社会问题诱发的抑郁、焦虑已成为目前成年男性性欲低下的主要原因;⑧居住条件差,杂乱无章、通风不良、过于拥挤的环境不仅会引起心绪不佳,而且由于室内新鲜空气不足,导致大脑供氧不足,影响性功能,使性欲降低。若是几代人同居一室,或与子女同睡一床更易导致性欲低下。

此外,肿瘤患者性欲低下的问题正日益受到关注,大部分患者并无与性功能相关的器官性损害,仔细分析其原因可能为:①对肿瘤的恐惧使患者精神压力很重;②肿瘤伴随症状的干扰,患者常有疼痛、出血、发热等影响性欲;③担心性活动损伤“元气”、促进肿瘤复发;④顾虑性活动干扰治疗;⑤顾虑肿瘤具有传染性或配偶有这种担心又不便说明,夫妇双方自觉或不自觉地企图通过避免接触而减少“传染性”;⑥因肿瘤或治疗而损害形象,以致从视觉、触觉、听觉、嗅觉等方面不同程度地干扰性活动;⑦医务人员关心患者的躯体健康,但对患者的性要求和性问题常避而不提,使部分患者认为性是不允许或不重要的。

这类患者除了表现出非常规的性行为之外,并无其他方面的人格缺陷。他们的性行为和性行为对象与一般人不一样,如恋物癖、露阴癖、恋童癖等,该类患者常常对正常的性行为方式不感兴趣,一般的性刺激也难以引起其性兴奋,但对其所认定的特殊性对象却有正常的甚至是亢进的主观欲望和兴趣,即患者可有正常的性兴趣和性兴奋,但性对象却异于常人。

几乎所有严重的全身性急、慢性疾病或长期疲劳都可导致男性性欲低下,即由于体力和精力的降低而导致。甲状腺功能低下、慢性活动性肝炎、肝硬化、慢性肾衰竭、充血性心力衰竭等不仅可破坏正常的激素代谢过程,导致患者生理上相应功能的减退甚至衰竭,患者对性生活时可能出现相关脏器功能损害加重的认识,使患者对性活动的主观欲望和兴趣下降甚至消失。值得注意的是,甲状腺功能亢进者可表现出多种形式的性功能紊乱,10\%~20\%的患者(尤其是轻度甲亢患者)早期有性欲亢进表现,而多数中重度甲亢患者则因血睾酮含量下降、睾酮储备功能降低表现为性欲低下。

本病可直接导致器质性性欲下降甚至无性欲。常见于各类内分泌系统疾病:许多疾病都可影响雄激素的产生,而雄激素是维持男性生精功能、男性第二性征和性行为能力的重要物质,同时雄性激素能提高大脑皮层性中枢的兴奋性,激发性欲,产生性兴奋。睾酮水平低下可直接导致男性性欲低下。小阴茎与体内睾酮缺乏有关,这是因为多种疾病最终都要通过体内睾酮分泌的减少而影响阴茎的发育。一般将这些导致体内睾酮分泌减少的疾病分为三类:①促性腺激素分泌不足的性腺机能减退;②促性腺激素分泌过多的性机能减退;③原发性小阴茎,这类患者下丘脑-垂体-睾丸轴正常,到了青春期多数人阴茎还能增长,具体原因不清,推测可能是胚胎后期促性腺激素延迟而引起的一过性睾酮分泌下降或雄激素受体异常。虽然睾酮等性激素对性欲的维持和调节至关重要,但并非决定性因素,部分睾丸切除后患者的性欲(包括性兴趣和性兴奋)仍可以正常,而且临床工作中常可以发现大多数性欲低下的患者性激素检查属于正常范围。

中枢或盆腔会阴神经的病变或损伤,视觉、听觉、嗅觉及性欲带的触觉的原发或继发功能下降甚至缺失,都可以直接或间接降低性兴奋,影响性活动中达到高潮。如肿瘤患者长期接受放射治疗,损伤相应内脏、中枢神经系统都可导致性欲降低。

可分为:①降低神经兴奋性的药物和部分治疗精神疾病的药物,如:镇静药等。②可使雄激素、促性腺激素、睾酮浓度降低的药物、食物,抗肿瘤药物、抗雄激素治疗的药物(治疗前列腺癌)等。③心血管、降压药物:利血平、降压灵、安体舒通等。④抗过敏药:非那根、扑尔敏、安其敏、苯海拉明等。⑤胃肠道药物:西咪替丁、雷尼替丁等。⑥部分精神兴奋剂和麻醉剂:最常见的是可卡因、乙醇等,民间习俗认为乙醇可以增强性欲,但实际上乙醇是降低了判断、自控能力而增加了对性的放纵,或是因为减轻了大脑理智对性欲的抑制作用,长期大量饮酒将影响阴茎的勃起功能。

男性性欲低下患者大多既往性欲正常,身体素质及性心理也多正常。但表现与其自身年龄不相匹配、不和谐的性欲淡漠,继而表现性行为表达水平降低和性活动能力减弱,性活动频率降低。如性生活每月不足1次或更少,有的虽然次数稍多,但并不是主动要求,而是在性伴侣的压力下不得已而为之,即主动性生活减少。

男性性欲低下同样可分为:①性兴趣低下:表现为对性生活的兴趣淡漠,性幻想明显减少,即使对性刺激反应正常,但通过性交获得的乐趣明显下降;②性兴奋低下:表现为在对性活动的要求、主观欲望(性兴趣)正常甚至强烈的情况下难以产生性兴奋和性冲动,对各种强烈的性刺激、爱抚反应低下,或无法在性活动中维持足够的兴奋度以完成性交(在未达到性高潮情况下)。上述两种情况可同时存在,或其中一项低下,但都可导致性活动的减少、从性活动中获得的乐趣减低,甚至直接导致男性勃起功能障碍。

此外,部分性欲下降的男性为了满足女方要求或维护自身尊严也可以参与性生活,即被动性生活(passive sexual life)。长期被动性生活可能诱发抑郁。

性欲低下的诊断应建立在耐心的病史询问和细致的体格检查、神经内分泌等实验室检查的基础上,适当辅以影像学检查。根据1993年马晓年等提出的性欲低下的诊断标准可将其划分为以下4级:

Ⅰ级:性欲较正常减弱,但可接受配偶的性要求。

Ⅱ级:性欲在某一阶段后出现减弱或只在特定境遇下才出现减弱。

Ⅲ级:性欲一贯低下,每月性生活不足2次或虽然超过但系被动服从。

Ⅳ级:性欲一贯低下,中断性活动6个月以上。

诊断性欲低下时不仅应当区分其属于心因性还是器质性,而且在临床上,性欲低下要与自然性性欲降低(特别是进入老年期后)相鉴别。随着年龄的增长(一般来说,50岁以后),男性睾丸功能自然减退,雄激素分泌减少,性欲逐渐减弱,这不属于病态。

同性欲亢进的治疗原则相似,性欲低下的治疗原则除缓解症状、控制病情外,也应注重根除病因、预防复发。其治疗方法包括:心理辅导、行为疗法、药物治疗等。

大多数性欲低下由精神因素引起。最主要是采用咨询和指导为主的精神心理疗法,即根据高级神经中枢的条件反射机制,对曾经有性生活经历的人,向后天获得的条件反射,通过视、听、回忆等刺激,引起大脑皮层性中枢的兴奋。加速性教育的全民普及,解除不必要的性顾虑;加强社会道德建设及家庭责任观念;对有不成功性经历的患者应剖析其原因,树立和加强其自信;心理压力过大或有抑郁等心理障碍者应先解除其心理障碍;夫妻感情融洽、相互体贴很重要,一般无特殊情况,夫妻不要长期分居。男方性欲低下阶段女方如能主动加强性刺激和爱抚,有明显治疗作用。对部分患者可以阶段性给予小剂量磷酸二酯酶抑制剂,如西地那非、他达拉非等药物,可提高性功能、增进性感受。应当强调,夫妻协同配合治疗至关重要。

首先是针对全身性器质性病变进行治疗,在身体机能不允许的情况下,不必急于治疗性欲低下,只有等待全身功能改善后,根据年龄、家庭、夫妻状态、个人情况等选择治疗方法。对于睾丸功能减退、雄激素分泌减少的患者可以给予雄激素辅助治疗。如:庚酸睾酮肌肉内注射,300mg/3w;长效油剂睾酮酯(丙酸睾酮)肌肉内注射,250mg/3~4w。对睾酮、FSH、LH水平降低,促性腺激素分泌不足导致的性机能减退者最常用的治疗方式为HCG皮下注射,1000~2000IU/3d。但必须注意:激素替代治疗应严格掌握剂量,定期复查性激素水平,及时调整剂量和适时停药,防止血红蛋白升高、肝功能损害、低氯血症等并发症。此外,还应适度加强体质锻炼,生活规律,做到劳逸结合,消除精神抑郁的体质性因素。

防止服用降低神经兴奋性及促性腺激素、睾酮的药物、食物,或减少药物的剂量,或改用其他替代药物治疗。减少吸烟,拒绝毒品。

治疗性欲低下时应避免滥用“壮阳药”、“春药”,以防止转为性欲亢进,部分该类药物使用过量甚至可导致阴茎异常勃起。我国民间、中医养生学一直推崇“药补不如食补”,强调合理饮食对性欲的调理作用,近年国内外的科研、调查也得出一致的结论,如鱼类、海鲜类食物、大豆、韭菜、花生、核桃类坚果核有促进性欲的作用,而烟草、大量饮酒、芹菜、含大量脂肪的肉食对性欲乃至性功能都不利。

普及科学的全民两性教育,解除不必要的性顾虑;面对社会、工作中的各种压力,应学会“取舍”,通过适当方式放松、减压;培养夫妻感情,增进两性沟通、交流等,“爱情”无疑是预防性欲低下最好的方式。此外,长期服用镇静药、心血管药物、降压药物以及其他可导致促性腺激素类、睾酮类激素降低的药物的患者需严格遵从医嘱、定期复查。值得注意的是,有人认为黄色影像资料是防止性欲低下的最佳方式,这并不正确。因为此类影像的内容通常都是夸大的,好像每个女性有了性刺激就会兴奋起来,每个男性都会很持久,这些容易让人产生错误的认识,青少年更应警惕由此引发的性犯罪。


\section{第四节 男性性厌恶}

性厌恶(sexual aversion)指对与性伴侣所有的或几乎所有的生殖器接触具有持续的或反复的极度不适和回避。性接触时,甚至只是在想象将要发生性接触时,都会产生强烈的消极情绪、极度紧张或焦虑,有时还伴有强烈的躯体反应,患者会极力回避一切性接触,有时也会出现惊恐反应。人们称之为“性厌恶”、“恐惧性性回避”、“性恐惧”。因女性性厌恶远较男性常见,部分问题将在女性性欲异常章节中详细论述。

产生性厌恶的主要原因为精神心理因素,如畏惧感、罪恶感、牺牲感、被动感、恐惧心理反应等,其致病机制及病因如下:

青春发育期面对突然而至的身体变化,如阴茎勃起、遗精等生理变化,缺乏思想准备而又得不到正确的教育、疏导时,极易产生罪恶感。有的患者在与同伴比较中自觉身体不如别人,如阴茎短小、胡须稀少等而产生自卑感。有的人经历过来自性生活、性犯罪、怀孕等的惊吓,最后发展为性厌恶。有的人从小受宗教迷信思想影响,认为性是邪恶、肮脏或有损健康的,从而对性活动产生厌恶感。

如体质差、射精过快而受到配偶的埋怨、冷待,多次受挫折发展成性厌恶。或性活动中受到身体、精神心理上的创伤,即有创伤性性经历史,或性生活受到令患者极度厌恶、恐惧等不良情绪反应的外界刺激(声音、气味、他人侵入等)而形成厌恶的“条件反射”,表现焦虑、出汗、心悸、恶心、呕吐、抽搐等。

夫妻长期感情不和,性生活不和谐,如妻子将性生活作为控制丈夫的武器,在丈夫提出性要求时妻子提出苛刻条件作为回报,或者丈夫在不能承受性生活的情况下被强求而为之,久而久之就会导致男方从内心逃避性生活,进而产生厌恶。

一些偶发因素可以扭曲男性的性心理,如有些丈夫观看妻子分娩过程之后,对女性生殖器产生不可遏制的厌恶情绪,从此拒绝性生活。

焦虑症、强迫症及恐惧症等精神疾病中,部分患者合并性厌恶。部分癫痫患者对性高潮的预兆产生极度恐惧或厌恶的情绪。有的患者认定其神经衰弱、性器官疾病与性生活有关,从而产生性厌恶表现。

性厌恶患者多表现为丧失了正常性生活起始阶段的性冲动或拒绝接受性刺激。较轻症状的性厌恶患者仅表现为性活动次数减少或缺乏性生活兴趣,典型重症患者则对正常性欲的各种现象,如接吻、拥抱、抚摸等均表现出焦虑、出汗、心悸、恶心、呕吐、腹泻等病态性反应,出现迅速中断性活动或逃离性活动现场等现象。性厌恶患者除回避性生活之外,有的还对自己及异性的生殖器、女性乳房、异性或同性性活动均产生厌恶感。

性厌恶以对性生活或与性活动相关的思想的持续性憎恶反应为特征。根据疾病特点可将其分为:原发性性厌恶和继发性性厌恶;完全性性厌恶和境遇性性厌恶。

原发性性厌恶指从具备性意识开始始终对性怀有畏惧和排斥心理,多具有严重的心理病理问题;继发性性厌恶是指曾经有正常性经历的历史,但因某种继发因素发展成性厌恶,常与感情或性生活创伤有关,明确病因后其预后相对较好。

完全性性厌恶指完全厌恶性活动,在对任何的动情感觉、感受、想法和机遇作出反应时都会产生恐惧或厌恶。境遇性性厌恶指只限于性的某一特殊方面有厌恶感,多对某种形状、气味、声音、言行或体位有厌恶、恐惧感,严重时常表现为强迫性性回避,但当特定诱因去除后性活动可恢复正常。

性厌恶应与性欲低下相鉴别:二者都可具体表现为性活动主动性差,常常处于被动的应付状态,甚至处于一种“无性”的状态。但性厌恶的特点是畏惧和回避;而性欲低下的典型表现是抑郁和缺乏反应的主动性,性欲低下患者对于性活动中肉体接触持中立态度。

消除性厌恶的心理因素,通过系统暴露先前造成患者性回避的情境从而减轻对患者的性回避反应。治疗的关键是劝说夫妻共同参与性行为疗法的治疗。丈夫应当认识到性生活是夫妻感情的重要纽带,更是一项重要的家庭职责。另一方面,妻子的积极配合和充分的爱抚对丈夫性厌恶的治疗有积极意义。

行为治疗常用脱敏疗法,即去除患者精神防卫趋势,探求、暴露患者最大的不安全感和脆弱性,激发有意义的梦或记忆等,使其认识到致病的根源及疾病的可控制性、可治愈性,主动、逐步克服和纠正其不当的心理及行为。

性感集中训练:先禁欲一段时期,在此期间夫妻间开展情感交流,不谈“性”问题,只为使感情更加融洽。进入治疗阶段,妻子对丈夫身体进行试探性抚摸,其路径是身体非敏感部位—性器官—刺激使性器官勃起—最后转入性交。这个阶段的时间长短因人而异,一般需要2~4周,应避免对患者造成不适。期间妻子对丈夫某些性欲改善的现象要热情鼓励,丈夫也可告诉妻子一些自己的感受和性要求。当丈夫出现不适症状时,妻子不应勉强继续,而须停下来安慰丈夫。

部分有如焦虑症、强迫症及恐惧症等精神疾病的患者,在原发病未得到有效控制的情况下应暂时避免性活动、性诱导,这可能是处理焦虑情绪的最好办法。

性厌恶的药物治疗主要是抗抑郁药。

三环类抗恐惧药:如丙咪嗪、阿米替林、去甲替林等为治疗首选药,能阻断恐惧,干扰知觉。

米氮平:不仅能改善焦虑,还能改善性功能,促进睡眠。

有人报道使用丁氨苯丙酮225~450mg/d,12周后63\%患者性功能明显改善。

同性欲低下相似,预防性厌恶强调普及科学的全民两性教育,解除不必要的性顾虑;培养夫妻感情,增进两性沟通、交流,共同探讨合适的性生活方式;避免不良嗜好;面对既往的各种“创伤”,应学会淡忘,相信明天会更好,通过适当方式放松、减压;停止使用影响性激素水平、性功能的药物,如雌激素、镇静剂等。


\section{第五节 女性性欲亢进}

女性性欲亢进(female hypersexuality)指女性的性欲持续异常旺盛,处于一种强烈、持续的性冲动状态,远远超出正常水平。其表现为性欲要求强烈;性兴奋出现过多、过快、过剧;性反应超常地迅速、强烈,甚至拥抱、接吻、轻触阴部也能产生强烈的性高潮。中医称之为“花痴”。

由于性欲强弱在正常人之间存在明显差别,而且在不同年龄段甚至在不同环境下都可有很大变化,因而很难为性欲亢进作出明确的界定。典型的性欲亢进表现为整天沉湎于性冲动之中,从各方面都表现出对性的渴求,为了获得性感满足寻找一切可能的性交对象和一切可能性交的机会。当这种欲望强烈又无处宣泄时,患者便出现焦虑、激惹、心慌、失眠等症状,甚至痛苦不堪。人们把性欲亢进看做是一种像酗酒、赌博或吸毒一样成瘾的表现或一种受性冲动控制的障碍,属于性欲或性高潮能力病理性增强的性功能障碍,临床上比较少见。

女性性功能与下丘脑-垂体-卵巢轴(HPO轴)的运转密切相关,只要其中某一环节出现异常,就可以导致性功能异常,少数表现为性欲亢进。性活动除受HPO轴影响外,也受大脑皮质的影响,如果大脑皮质或下丘脑的性欲中枢处于持续兴奋状态,垂体前叶促性腺激素分泌过多也可引起性欲亢进。

性欲亢进病因可分为器质性和功能性。近年研究认为器质性因素占主要地位,而内分泌失调是女性性欲亢进最常见的病因,另外生殖、神经等系统的某些疾病也与女性性欲亢进有关。

1)雌激素(estrogen):雌激素被认为是保持女性基本性欲所必需的性激素。雌二醇(E2)水平影响中枢及周围神经的功能及信号传导。研究发现,雌激素可增加脑内、脊髓内性中枢的兴奋性,促进女性会阴、外生殖器神经末梢的发育和敏感性,从生理层面提高性欲及促进与性伴侣的关系。有人认为雌激素能提高整个神经系统中雌激素敏感细胞及神经递质的传递,能增加阴道一氧化氮合酶的表达,增加外周血管对感觉刺激的反应,从而提高女性性欲、性反应、性高潮和性活动频率。

2)雄激素(androgen):雄激素过高可导致性欲亢进。睾酮能增强阴道近端NOS活性,减弱精氨酸酶活性,能使电性刺激阈值降低和血管活性肠肽(VIP)阴道平滑肌松弛。正常女性卵巢、肾上腺分泌少量雄激素,对维持女性的生理代谢有一定作用。当肾上腺、垂体、卵巢等生长某些肿瘤则会产生过多的雄激素,使女性生理心理发生改变,如体毛过多增长、情绪急躁和性欲亢进。

3)多巴胺(dopamine,DA):DA是中枢神经系统中重要的神经递质,通过多巴胺受体(DR)、阿片样受体的介导,在尾状核、豆核及边缘系统中,产生生理状态如性需求满足等驱动。中脑-边缘叶(伏隔核、嗅节结、杏仁核、隔区)及中脑大脑皮层(额叶、扣带回)的DA系统,在生理功能上主要是调控精神活动,包括性欲、情绪、思维、感觉、理解、判断等。下丘脑-垂体DA系统控制黄体激素和卵泡激素释放因子的分泌,致使黄体生成素(LH)、促卵泡激素(FSH)、孕酮(P)、雌二醇(E2)水平增高。阿朴吗啡是DA受体激动剂,可明显提高人的性欲。

4)5-羟色胺(5-HT):脑内大部分5-HT能神经元分布于中缝核群。5-HT降低除可引起抑郁症患者心情抑郁、食欲减退、失眠、昼夜节律紊乱、内分泌功能紊乱、焦虑不安等症状,还抑制患者性反应,如性欲低下。精神分裂症、躁狂症、忧郁症、情感障碍所致的性欲亢进和中枢5-HT功能受损有关。然而,5-HT受体的类型和亚型非常多,动物实验提示:5-HT受体亚型的5-HT1A 激动剂抑制雌鼠的性行为,5-HT2 激动剂促进雌鼠的性行为。

1)颅内肿瘤、炎症:导致下丘脑-垂体器质性病变可引起青春期前儿童性早熟或成年人性欲亢进。如垂体生长激素分泌瘤,早期可反射性引起腺体分泌过多的生长激素,出现性欲亢进,晚期则表现为性欲低下或丧失。

2)卵巢肿瘤:卵泡膜细胞瘤可使体内性激素分泌增加,表现出性欲亢进和思维紊乱等精神症状,手术切除肿瘤后相关症状消失,性欲也可恢复正常。

3)肾上腺肿瘤、Cushing综合征:可出现肾上腺皮质网状带增生,雄激素分泌过多导致性欲亢进。这已经通过抗雄激素类药环丙孕酮的临床实验证实,环丙孕酮是17-羟基孕酮衍生物,直接抑制雄激素的作用不强,主要是作为雄激素受体阻滞剂而抑制雄激素生物效应。

4)甲状腺功能亢进:表现为多种形式的性功能和性行为紊乱,10\%~20\%的患者早期有性欲亢进表现,特别是轻度甲亢患者。但大多数甲亢患者最终表现为性欲低下。

5)更年期性欲亢进:由于更年期女性卵巢雌激素分泌减少,使脑垂体促性腺激素反馈性分泌过多,雄激素与雌激素比例失调而引起性欲增强。但要注意与更年期精神病症状相鉴别。

1)某些精神-神经疾病是女性性欲亢进的重要原因,躁狂症、精神分裂症、老年痴呆等精神疾患、颅脑外伤、颅脑手术等疾病均可导致患者对性兴奋、性行为的抑制能力下降,表现为性欲亢进倾向。

2)产后精神抑郁(postpartum depression):产后精神抑郁病因复杂,与遗传、环境、经历等因素有关。症状较轻的表现为哭泣、忧郁、烦闷、不安、易疲乏等情绪障碍,症状较重的出现焦虑、易怒、紧张、性欲亢进等不良反应。

3)克-列二氏综合征(Kleine-Levin syndrome,KLS):又称周期性睡眠过度(hypersomnia periodica),较为少见,男女均可发病。表现为睡眠增多、饮食过量和行为改变,如性欲增强,尤其在月经前后发病。其病因不详,考虑与下丘脑免疫性病变有关,如病毒性疾病或其他感染。

4)克-布二氏综合征(Kluver-Bucy syndrome):又称双侧颞叶切除综合征,是一种较罕见的神经精神障碍引起的综合征,其特征为视觉失认、饮食障碍如贪食或嗜食异物、思维变化过速、同性恋倾向、性欲亢进等。其病因见于切除双侧癫痫病灶、脑炎后、脑挫伤、脑动脉硬化等。

5)其他:有报道称帕金森病(Parkinson disease,PD)手术治疗后、多发性硬化症(multiple sclerosis,MS)、中风(apoplexy)患者康复治疗中可表现出性欲亢进。

心理、社会因素形式多样化,青少年时代的环境影响在发病中有一定作用。包括:过早接触色情物品、色情小说、淫秽读物和录像等,或反复接受大量性刺激;受双亲或朋友公开或隐蔽的性诱导,自身过早的性体验;性行为无抑制、纵欲过度等。

性欲亢进表现为整天沉溺于性兴奋之中,从各方面持续地表现出对性的强烈渴求,为了获取性满足而寻找一切可能的性交对象和性交机会。具体表现为性兴奋出现过多、过快、过于剧烈,当达不到要求时便出现焦虑、激动,甚至心慌、头昏、失眠、四肢无力、发呆等症状。患者对性活动的反应也超常的强烈,甚至拥抱、接吻也能产生强烈的性高潮。

该病起病常较缓慢。轻度的患者有性幻想或错觉,可以通过性交来达到缓解心中强烈的性需求目的;重度患者每天性交可达数次至十几次,甚至达到不避亲疏的程度。部分患者常伴有多疑症状、思维混乱、精神恍惚。过于亢进的性行为常包括:过于热衷于谈论性话题,不适合地触摸他人(如过多的拥抱、亲吻等),性冲动强烈,当众手淫,有性幻想或亲昵的错觉,试图性引诱他人,想脱光他人衣物,挑逗他人,暴露他人生殖器,抓他人的阴部等。

女性性欲亢进根据其构成可细分为:①性兴趣亢进,表现为对性生活有超常的兴趣,呈现出一种强迫性的需求,不考虑任何条件和环境的约束,不断有性交欲望或频繁出现性幻想;②性兴奋亢进,患者频繁出现性兴奋和性冲动现象,并对非性刺激的正常举止、言谈反应强烈并迅速将其转化为性活动的原始推动力,而且这类冲动不易抑制,高潮后迅速再发,甚至给患者带来性欲与理智冲突的痛苦。性兴趣和性兴奋亢进既可以同时存在,也可以单独出现。

孤立性的性欲亢进在临床上很少见,常常伴有其他精神情感障碍,如重度抑郁症、焦虑、恐惧症的社交恐惧或其他强迫症等。

女性性欲亢进较男性少见。女性在性活动中又常常处于被动、接受地位,很难单凭性生活频率多少来作出诊断。而且个人所处的环境、认知、心理、人际关系等都存在巨大差异,所以各人性欲强弱表现差异甚大,临床上很难准确和简单地去定量性欲程度。有些人认为不是自己性欲亢进而是伴侣性兴奋减弱;有些患者与伴侣性活动完全丧失兴趣而宁愿自己频繁手淫。由于女性性欲亢进人群相对隐蔽,患者有强烈的羞愧感,医生在病史询问时不仅要详尽,避免遗漏,更应具有一定的策略性,应注意保护患者的个人隐私,一些特定的相关问题可在配偶等人不在场的情况下询问,避免给患者增加压力。

应当重视对性欲亢进患者是否具有其他精神情感问题进行系统的评价,临床上大多数此类患者可能伴有情感障碍,特别是重度抑郁症、焦虑、恐惧症中的社交恐惧、对精神活性药物等的依赖(如乙醇依赖)、非特异性的冲动控制障碍(如强迫性购物欲或不规范驾驶)、注意力不集中、高度易动症等。焦虑、抑郁、强迫行为、性欲亢进等可以享有共同的病理变化。精神障碍可能对性欲亢进的发生、增强,对社会有危害的变异行为等具有促进作用。

性欲亢进症状多来源于器质性疾病,诊断女性性欲亢进除必须对患者进行全面问诊外,体格检查也十分重要,特别是针对内分泌和精神神经系统的检查,妇科检查要注意女性生殖器是否正常。

血性激素检查,特别是血睾酮水平和甲状腺功能的评估尤为重要。

脑部MRI扫描T2加权显像,单光子放射计算机成像(SPECT)扫描,检测脑部血流状态。

根据病史、体格检查和实验室检查一般可作出诊断,但应与以下情况相鉴别:

1.新婚蜜月性生活过多:新婚蜜月期间年轻人性生活每天一次甚至数次都不算过多过频,但是倘若减少一次性生活便觉得十分痛苦的话,则是性欲亢进的重要特征之一。

2.性心理异常(又称性欲倒错):其特点是满足性欲的方式不同于正常的性活动。一般是以偏离正常的方式来获得性满足,如恋物癖、露阴癖等。

3.性强迫症:反复出现性要求的痛苦的认知体验。强迫行为是通过外显的性行为和隐蔽的精神活动来减轻不愉快的强迫性行为的强度,其表现为不相关联的性行为,并受到精神活动的支配,其性质和强度与其诱发性强迫观念的外部环境不相称。

4.原发疾病诊断:孤立的性欲亢进在临床上很少见的,常常伴有其他疾病,例如持续的认知障碍或精神病。青春型精神分裂症的主要症状为言语增多,内容荒谬离奇,言语杂乱,动作及行为幼稚、愚蠢,情感变化莫测,喜怒无常,易冲动、性欲亢进等。多发性硬化症可伴发一时性的性欲亢进。克-布二氏综合征的患者嗜睡、多食和情感障碍(典型的性欲亢进、易激惹和冲动行为),症状持续几天至几周可完全缓解。

性欲亢进的治疗包括病因治疗、心理治疗、行为治疗和药物治疗。对于患有器质性疾病的患者无疑应给予积极的、针对病因的治疗。因为几乎所有性欲异常的患者都同时伴有某些心理方面的异常,在去除其具体病因后,应同时配合心理治疗。

对患有器质性疾病的患者应给予积极的原发疾病的治疗。

认知行为疗法是对患者进行必要的性知识教育,使患者建立起对性生活的合理认识,消除其不良的情绪反应,以正常的性心理去规范性行为,树立起治疗信心。根据患者的具体情况适当配合药物治疗,综合调整造成性欲亢进的认知信念及行为。

1.对性欲过强症状的控制:医生可以和患者讨论可行的治疗策略,共同设计一个能够使患者遵守的高发性行为频率和时间的底线,如:每天最多幻想半小时,就是说“控制”要循序渐进,开始要求不要太高,否则不仅达不到目标,反而会使患者失去信心。

2.协商采取一些客观措施来限制患者的行为实施:如通过取消色情物的刺激,限制其经济消费,由伴侣、家人或医生等控制监督。

3.合理安排生活节奏:如夫妻适当分居一段时间,以减少性刺激,多参加文娱体育活动,将更多的精力用于学习和生活中,使性神经有适当的休息机会。

以上治疗措施应在取得患者充分信任的基础上实施,以便能让医生、伴侣等帮助患者彻底暴露其相关高发行为的细节。

4.行为治疗技术:目前常用的有厌恶疗法,如氨气嗅觉厌恶等。

药物治疗应针对不同的发病原因合理用药,在其他治疗失败或仅部分有效时可给予药物治疗。治疗性欲亢进的主要药物有抗精神病类药物和内分泌类药物。

1)盐酸曲唑酮:初始剂量为50~100mg/日(分次服用),然后每三至四天剂量可增加50mg/日,一般以200mg/日(分次服用)为宜。一旦有足够的疗效,可逐渐减量,长期维持的剂量应保持在最低有效剂量。疗程一般持续数月。常见不良反应为嗜睡、疲乏、头晕、头疼、失眠、紧张和震颤等,以及视物模糊、口干、便秘。少见体位性低血压和心动过速、恶心、呕吐和腹部不适。极少数患者出现肌肉骨骼疼痛和多梦。

2)氟西汀:其抗抑郁疗效与三环类相似,而抗胆碱能及心血管副反应则比三环类小。10mg/d,逐渐增加至60mg/d,3~4月后性压迫和情感紊乱症状减轻。不良反应:常有胃肠不适:厌食,恶心,腹泻;神经失调:头痛,焦虑,神经质,失眠,昏昏欲睡,倦怠或虚弱,流汗,颤抖,目眩,头重脚轻。发疹或出现荨麻疹应立即停药,约1\%患者发生狂躁或轻躁症。

1)激素类:每个月经周期内用环丙孕酮50~25mg/d,月经第15~25天;雌二醇0.5~1mg/d,月经第5~25天;也可用黄体酮20mg/d,肌注,与月经第5天开始,连用14天,用药第1周期性冲动明显减少,第3周期性欲亢进完全缓解。

2)非激素类:甲氰咪胍是一种抗H2受体拮抗剂,600~1600mg/d,副作用有恶心、头晕、头痛。酮康唑100~200mg/d,副作用包括可逆性的肝脏损害、恶心、乏力、皮肤黏膜干燥等。螺内酯75mg/d等。

可解除患者的性冲动,用于更年期、内分泌失调等原因的性欲亢进。

1)三溴片0.1~0.3g,3次/天。

2)氯氮平2.5mg,3次/天。

3)地西泮(安定)2.5mg,3次/天。

4)甲丙氨酯(眠尔通)0.2g,3次/天。

5)氯丙嗪25mg,3次/天。


\section{第六节 女性性欲低下}

女性性欲低下(female hypoactive sexual desire disorder,HSDD)是指女性持续或反复缺乏性幻想和性活动的接受性,出现与其自身年龄不相符的性欲望和性兴趣淡漠,从而导致精神忧虑。当性表达机会遭到剥夺时也没有挫折感,没有寻求刺激和减少挫折的动机。个体通常不会主动发起性活动,只是在伴侣的发动之下不情愿地参与性活动。性欲低下并不排除女性在被动接受性活动时达到性唤起和获得性快感的可能性。

性欲低下可以是独立的性问题,也可以继发于其他性问题。由于缺乏有关性欲频率或程度的与年龄或性别相关的正常值资料,性欲低下的判断应由医生根据其年龄、人格特征、人际关系的决定因素、生活背景、文化环境等因素作出。

女性在性活动中绝大多数处于被动地位,且多数女性羞于谈及性生活、性欲话题。为维护家庭、感情,多数性欲低下的女性通常“牺牲自己”来取悦、满足男性。女性性欲低下的后果及对双方的危害远低于男性性欲低下,多数婚姻能在女性性欲低下中得以维持。被动性生活(passive sexual life)即为满足男方要求或维护女性自身尊严,非享受性乐趣而被动进行的性生活;主动性生活(initiative sexual life)即为满足女性自身要求和享受性乐趣而主动进行的性生活。

据20世纪70年代英、美、丹麦等国家的统计显示,性欲抑制的疾患可占性治疗患者的37\%~46\%,他们同期所进行的社区调查表明将近34\%的女性和16\%的男性有性欲低下。2007年调查欧洲女性性欲低下的发生率为6\%~13\%,美国女性为12\%~19\%。

人类性行为是一个生物、心理和社会因素相互联系的整体,女性正常的性活动依赖于体内正常的性欲中枢、正常的性激素水平、良好的血管反应、和谐的夫妻关系及正常的性观念。性欲低下的原因非常复杂,是多种因素共同造成的。目前认为其病因可分为功能性和器质性两大类,以功能性性欲低下较为常见。

性欲低下以心因性为主,社会心理因素是导致性欲低下的最主要原因,它包括许多方面的因素:①错误的性信念和性信息。②心理障碍,失恋、婚前过多性交往等带来的性挫折、心理创伤;自身心理冲突如对性能力的过分焦虑、对不能满足伴侣性要求的内疚感等;潜在的性偏好障碍、人格障碍、性取向冲突等也会影响性欲的正常表达。③婚姻冲突。夫妻之间性需求、性感受的交流不够,把非性问题的冲突带进性生活中。性欲低下与成年后的性虐待具有相关性(Laumann,1999)。④性技巧贫乏,缺乏新鲜感的性生活方式,使性生活成了索然无味的例行公事,缺乏激情,缺乏动力,缺乏乐趣,最终导致性欲低下。⑤生活方式,紧张而充满压力的工作环境,家庭居住条件太差,缺乏隐蔽和安全的条件,夫妻工作时间冲突,两地分居。⑥年龄因素,取决于两性的性观念、身体状况、过去的性经历、社会文化影响等因素。随着年龄的增加,性欲低下的发生率也随之增加(Richard D,2007)。

对于所有的性功能障碍患者而言,都强烈推荐全面地了解医学和心理社会历史(表6-1)。包括对抑郁的筛选,不论是否使用抗抑郁剂,抑郁总是与性功能障碍相关的,特别是与性欲低下相关。如果妇女透露曾有过性虐待的经历,建议对她做进一步的评估。包括评估妇女从性虐待的康复(过去是否接受过治疗),确定任何重症抑郁症,药物依赖障碍,严重焦虑,自残或滥交的历史;完全不能相信他人,特别是那些同性别的同案犯;或对控制他人或对取悦他人(不能说不)的过度需求。当注意到遗留着与性虐待相关的症状时,对性功能障碍的评估就要暂时推迟。

表6-1 一个综合的、性的、医学的、心理社会的、历史的组成成分

心、脑血管疾病,呼吸系统疾病,神经精神疾病,性传播疾病,内分泌疾病,妇科疾病等均可导致性欲低下;营养过剩、过度肥胖等也可导致性欲低下;糖尿病、高血压可能与微循环和神经血管的变化相关。

下丘脑、垂体、卵巢功能障碍导致内分泌失调可以造成性腺功能低下或完全无性欲。

降低性欲的药物有:①抗雄激素药物,如醋酸甲羟孕酮、醋酸环丙酮;②抗精神病药物;③镇静安眠类药物,如苯二氮 类、巴比妥类;④抗高血压药物,如利尿剂螺内酯;⑤双重作用的药物,如乙醇、苯二氮 类、巴比妥类、可卡因、苯丙胺等,低剂量有提高性欲的作用,高剂量或长期使用可降低性欲;⑥皮质醇药物及扰乱体内性激素的药物,如促肾上腺皮质激素(ACTH)和皮质醇,孕激素等;⑦其他:如尼古丁、氟他胺、亮氨酰脯氨酸醋酸盐、甲氰咪胍、化疗药物等。

主要表现为持续或反复地对性不感兴趣,缺乏性幻想,参与性活动的主观愿望和意识缺失以及主动性行为的要求减少,由于患者或双方对性活动的频率不满可导致夫妻关系紧张。

功能性性欲低下可伴有心理障碍;器质性性欲低下可伴随相应的器质性病变症状,如全身慢性疾病、生殖器官疾病、性传播疾病、内分泌疾病等。境遇性性欲低下只发生在特定的对象或特定的环境,如在某种环境下性生活很正常,而在另一环境下没有兴趣;完全性性欲低下是在任何环境下长期或持续存在的性欲减退。

诊断性欲低下必须对患者日常生活中与性功能有关的各种因素进行综合分析,如年龄、身体状况、爱好等。对于女性性欲低下的诊断至今尚无统一标准,有人提出可以用女性参加性活动的频率来评价女性性欲低下,但这种机械的数字诊断方法并不恰当。因为它忽视了女性屈从于配偶的某种压力而被动接受性活动这一因素。

包括姓名、年龄、受教育程度、职业、患者和配偶与性相关的情况、性历史、月经生育史及接受过的治疗、婚姻关系、彼此感情、精神病史及其他全身性疾病情况,注意有无使用影响性功能的药物。

用女性性功能积分表对性功能进行评估,包括患者的性交次数、性欲强度、性高潮次数、阴蒂感觉及性交不适等问题,分数越高性功能状况越好。

国外有性治疗专家评价方法,其内容包括患者的形象、与性伴侣的相互关系、与性伴侣交流性需求的能力、性的态度、性反应和婚姻关系等。

主要是对性激素、甲状腺素,糖尿病的内分泌检查。

包括女性生殖道血流、阴道pH值、阴道顺应性及生殖道震动感应阈值等的检查。彩色超声测定阴蒂、阴唇、尿道、阴道和子宫血流流速(最大收缩期流速)和静脉池(舒张期末流速)。阴道pH值是阴道润滑的间接指标,可用数字式pH测量探头测量。阴道压力/流量变化可用顺应性测量仪测定。阴蒂和阴唇震动阈值可用标准的生物震感阈值测量器记录。

参见本章第三节男性性欲低下的诊断。

有明确的可导致性欲低下的躯体疾病、药物或其他不良嗜好等原因。常见的疾病有慢性疾病如糖尿病、心血管疾病、激素水平下降、生殖器官手术、损伤、炎症以及与性活动有关的神经系统疾病;常见的药物如抗高血压、抗精神病、镇静类药物及某些激素等;不良嗜好如酗酒、吸烟、吸毒等。

性欲低下的治疗应根据不同病因进行针对性治疗。若为心因性因素引起的,一方面要学会自我调节,也可多和亲友、同事交流,一方面应向精神科医生进行心理咨询。若为某些环境因素导致的,要积极改善性生活环境。倘若因其他疾病引起,应首先治疗原发疾病。因药物引起者,要征求专科医生意见,在允许的情况下更换其他药物或停药。

医生应全面掌握患者患病经历与病情特点,综合分析,准确判断性心理障碍的类型和程度,结合其生活背景、行为模式、病因及影响因素、个性特征,制订出有针对性的治疗法案,确定治疗时间和治疗进度,强调夫妻共同治疗。要消除背景因素,并指导患者观看动情材料,手淫训练,增加交流,积极参与性体验,通过协商形成一种夫妻双方都能接受和满意的性生活方式。具体方法有:

即对性欲低下的形成原因作深入的回顾和分析,向患者分析其不合理的信念及错误的思维方法,作出有针对性的疏导治疗。

夫妻感情是治疗成功的基础,只有先解决夫妻间的矛盾,使关系和谐,才可能使治疗成功。探索良好的夫妻性生活模式,强调共同的活动和培养共同的乐趣。婚姻治疗的目的就是要纠正双方错误的生活模式,建立良好的伴侣生活模式,帮助就诊夫妇消除婚姻及性生活中存在的心理障碍问题。

通过强调夫妇间自尊、亲昵关系、满意和乐趣的分享来获得良好的性状态和性感受。治疗成功的标志包括自我感觉良好、对方感觉良好、能共度美好时光。通过一起欣赏音乐、旅游、散步等重温昔日的美好时光,来增加性外的亲昵感受。

对于性知识缺乏的患者进行性知识教育,观摩男性及女性生殖器官解剖生理教材录像和有关治疗录像,学习性生活技巧,让患者模仿行事,注意性活动过程中的爱抚。丈夫要了解妻子的性敏感区,摸索出双方都满意的性爱方式。

训练的顺序是由非性敏感区到性敏感区,经过一段时间的训练,使患者对性行为紧张和焦虑的反应随自身的耐受力增强而明显减弱,甚至消失。治疗中要求配偶注意妻子的性冲动和情欲,注意语音或行为交流中的细节。当不存在阻碍性感集中体验的障碍时,配偶便可以在愉悦、相亲相爱和充满希望的气氛下进入下一阶段的训练。

该疗法通过对性高潮的体验来提高性欲,有助于建立起自信心和增强性欲。振荡器用于增强性快感,主要通过高频振荡促进性反应过程,适当使用可使女性达到性高潮,对因长期性高潮缺乏而导致的性欲低下有效。手淫疗法可达到类似的效果,但少部分人由此产生对手淫的依赖而回避夫妻性生活。

耻骨尾骨肌是类似阔韧带的肌肉,支撑着骨盆内器官及环绕阴道,经常进行耻骨尾骨肌锻炼有助于增强该肌的力量,增加性交时对阴茎的紧握作用,增强快感,从而提高性欲。

(1)雌激素

更年期雌激素替代治疗可增加阴蒂的敏感性和性欲,减轻性交疼痛,防止骨质疏松及降低心脏病的危险性。雌激素局部应用可以解除阴道干涩、灼热感以及尿频、尿急感。Kokcu等对25例绝经妇女分别用7-甲异炔诺酮及雌激素加醋酸甲孕酮进行治疗,发现其能改善绝经后症状群,特别是能有效地提高性欲及性活动。

(2)雄激素

将甲基睾酮与雌激素合用可以缓解绝经期妇女的性欲低下、性交痛和阴道干涩;而对绝经期前女性的性欲低下,甲基睾酮的治疗效果不明确。有报道称去氢表雄酮(DHEA)50mg~100mg/d或睾酮0.25~1.0mg/d对其有效。局部应用睾酮霜剂可以增加阴蒂的敏感性、阴道的润滑作用,可提高性欲和性唤起能力。

雄激素替代治疗可用于卵巢功能或(和)肾上腺功能不足,而对雌激素水平正常的患者只能作为辅助治疗。甲基睾酮可导致体重增加,阴蒂增大,面部毛发增多及高胆固醇血症等副作用。

丁胺苯丙酮(bupropion)具有加强多巴胺和抑制催乳素的作用,可增强性欲。氯哌三唑酮(trazodone)具有多种潜在的促进性行为的药理作用:能抑制5-羟色胺摄取、阻断α1受体,对节后β受体有脱敏作用,降低血中催乳素水平及多巴胺刺激作用。氟西汀(fluoxetine)为选择性5-羟色胺再摄取抑制剂,可改善性功能。以上药物对抑郁伴性功能障碍者有用。

左旋多巴为多巴胺的前体物质,通过血脑屏障进入中枢,经多巴脱梭酶作用转化成多巴胺而改善性功能。溴隐亭能降低催乳素,改善性功能。司来吉兰(Deprenyl,Selepiline)为单胺氧化酶选择性抑制剂,可增加多巴胺在脑内的活动及多巴胺神经的敏感性,副作用较少。

作为选择性V型磷酸二酯酶抑制剂,西地那非可以减少第二信使cGMP的降解,增强NO介导的阴蒂海绵体和阴道平滑肌的舒张作用。西地那非单用或与其他血管活性药物合用,均可有效治疗女性性唤起障碍。有关该药治疗女性性唤起障碍有效性和安全性评价的临床试验正在进行中。左旋精氨酸是NO合成的前体物质,在NO合成酶(NOS)的作用下分解为NO和左旋瓜氨酸。在大鼠试验中已证实左旋精氨酸可改善勃起功能,其在男性性功能障碍治疗上取得的初步结果也令人鼓舞,标准剂量是1500mg/d。该药对女性性功能障碍治疗的研究工作正在进行中。

前列腺素E1 最初为男性使用的一种0.01\%凝胶制剂,尿道内使用,经尿道黏膜吸收。前列腺素E1 制成女性阴道用制剂,可用来提高女性性器官反应能力,该药已进入临床试验阶段。

酚妥拉明为非选择性α肾上腺素能受体阻滞剂,可引起阴蒂海绵体、血管平滑肌舒张。一项对绝经后女性的研究结果显示,酚妥拉明可以增加阴道血流,改善性唤起能力。

有报告用含银杏树叶的草药合剂治疗202例性欲低下的女性患者,其中131例服药后自述性欲、性交、性幻想及对性交的满意度等均显著提高。该药的双盲对照试验正在进行中。

苯丙胺、可卡因、致幻药、美散痛等短期内可提高性欲,但长期使用会导致性欲低下,一般不用作性欲低下的治疗。

由躯体疾病引起的性欲低下首先应积极治疗原发病。各种慢性疾病、内分泌疾病、妇科疾病等应积极控制。注意患者是否使用了影响性欲的药物,如果为此种原因所致的性欲低下,可在允许的情况下考虑停用。

有报道称阴蒂包皮环切术可增强快感,提高性欲,但也有人认为阴蒂包皮实际上是小阴唇的延续,切除后使性交过程中对阴蒂的牵拉刺激减弱。其真实效果尚需作进一步研究。

有不良嗜好,如吸烟、酗酒、吸毒等应戒除。


\section{第七节 女性性厌恶}

性厌恶(sexual aversion disorder)是指持续或反复对性伴侣正常的性接触具有恐惧厌恶感并回避性接触,从而导致精神忧虑。又称“恐惧性性回避”或“性恐惧”。

女性性厌恶患者厌恶正常性行为及性观念,但性欲和性反应均存在,可以经历正常的性欲感受,她们能在手淫时产生性幻想、性唤起和性高潮,但对性行为有憎恶性反应,所以常常厌恶性伴侣的触摸。性厌恶是抑郁的重要原因之一,它可以严重限制个体行使性活动的能力。性厌恶一般为继发的,多见于女性,常由创伤性性经历所激发。

女性性厌恶是在性心理异常的基础上发展起来的,是患者将性的恐惧、焦虑、厌恶等情绪与性活动联系起来的结果。产生性厌恶反应的基础是性心理异常,她们一想到性交就莫名其妙地感到恐惧和忧虑,甚至是普通的接吻、拥抱、抚摸等只要与性产生思想联系的行为都会诱发性厌恶反应。但其性功能是正常的,实际上她们对性交行为的厌恶感一般与性交行为中实际产生的生理反应或具体动作无关。因此有些性厌恶患者表现出对性想象的忧虑比实际性行为引起的忧虑更强烈,在性生活中显露身体和触摸伴侣比性交过程更困难。

与童年和青春期遭受创伤性虐待经历有关,如强奸、乱伦等性虐待;或担心怀孕、性病或艾滋病而将性生活想象得很可怕。

接受错误的思想,认为性生活是淫荡、可耻和肮脏的,是取悦男性、为男性服务的低贱行为。这些错误认识导致对性行为产生憎恶反应,进而诱发性厌恶。

性生活中的挫折或被耻笑,可导致继发性性厌恶。性生活不协调或期望值过高也是性厌恶发生的原因之一。

1.更年期或卵巢功能早衰导致体内雌激素水平下降。

2.会阴部外伤、手术、残疾等原因导致的性交疼痛而出现性厌恶。

常见的临床表现为:回避性行为,对性行为或性想象过分焦虑、厌恶,甚至惊恐。对性活动产生一种持续性厌恶反应,这种厌恶反应在某些女性只局限于心理方面,表现为对性行为或性想象的恐惧或焦虑;有的同时表现为心理和生理两方面的问题,轻者可能有性高潮反应,但也不能在随意的性接触中使厌恶情绪得到缓解,重者即使是男方轻柔的爱抚、拥抱也可产生恶心,呕吐反应,性器官接触更会产生情绪紧张、恐惧不安、心慌、全身颤抖、大汗淋漓、面色苍白,有大祸临头感,有人称之为性交恐惧症。

性厌恶分类:参见本章第四节男性性厌恶

包括原发性性厌恶及继发性性厌恶;完全性性厌恶及境遇性性厌恶。

性厌恶分级:根据病情轻重,女性性厌恶可分为以下四级:

Ⅰ级:一般情况下尚能勉强接受性接触和性活动,只是在特定环境下才发生性厌恶,即性厌恶只是对特定的人或在特定的环境或特定的性生活方式时才产生。

Ⅱ级:从来就对性生活持强烈反对的态度,不愿参加性活动,对性活动从来没有主动要求。在一般情况下,对性生活有紧张和焦虑感,产生惊恐反应,并尽力回避性生活。但在特定条件下,如在爱人的强烈要求下和非常温静安全的环境下,经过长时间的安抚和性刺激,尚能被动地接受性生活。

Ⅲ级:在态度上反对任何性活动和性接触,在行动上竭力排斥任何性活动和性接触,回避任何性活动,根本不可能接受性生活。

Ⅳ级:对性活动和性接触持反对态度,在行为上进行排斥,而且在实际性生活过程中出现各种变态性机体反应,一旦发生性活动,就出现心悸、气短、恶心、呕吐、冷汗、颤抖、僵直、晕厥等性恐惧症状。

可根据患者对性一贯的厌恶反应对性厌恶作出诊断,既往的性经历、性创伤史,夫妻关系、宗教信仰等情况也有助于诊断。对因过度疲劳、工作繁重、睡眠过少、对性生活不感兴趣或性兴奋降低的不能诊断为性厌恶。只有在性生活中产生强烈消极情绪而排斥和憎恶性活动,竭力回避性接触的才能诊断为性厌恶。性厌恶可能单独存在,也可能继发于某些精神疾病或与其他性功能障碍同时存在。

按照DSM-Ⅳ诊断标准,诊断女性性厌恶的标准为:

1.持续反复出现过度厌恶或回避与男性伴侣的所有(几乎所有)生殖器的性接触;

2.给患者带来明显烦恼和人际关系上的困难。

有些原发性性厌恶患者并不一定存在心理病理问题,性厌恶的出现总是不知不觉的。在夫妻双方具有一段美满的性生活之后,出现问题的一方往往会发现自己逐渐回避性生活并日益加重。在患病早期,她们总是企图否认这一点或作出种种解释,而健康的一方最终会施加压力,不允许她们回避性关系,这种做法必将加重患者的恐惧,拒绝性接触。有时患者也强迫自己和丈夫发生性交,但情况总是每况愈下。

特点是忧郁和恐惧不限于性行为或性活动思想,而是表现在日常生活中的各个方面。

如同性爱,详细询问病史即可鉴别。

如阴道痉挛和性高潮障碍。性功能障碍多是原发性的,继发于性功能障碍的性厌恶可在性功能障碍治愈后自行缓解或消失。

低血糖,嗜铬细胞瘤,甲状腺功能低下,和某些中枢神经系统疾患都可表现出性厌恶症状,必须对原发病进行治疗。

性欲低下的女性对于性活动中的肉体接触可能持中性的、无所谓的态度,她们处于一种“无性欲”的心态,可以心平气和地回避性接触或者会喜欢某些性举动。

性厌恶在心理治疗和行为治疗相结合的基础上,可配合药物治疗。对于有明确发病因素的性厌恶应根据其不同的病因进行治疗。

对于存在严重的神经症障碍、性内疚感和婚姻问题的性厌恶患者,她们常常抵制对她们的性厌恶症状的行为矫正,这时就应采取心理治疗。

仔细分析病情,帮助患者挖掘引起性厌恶的病因。

解除对男性的敌意和偏见,尽量去除内心的阴影,使患者有要求治疗的愿望,使夫妻双方均参与到治疗过程中。

从解剖、生理知识开始,让患者对自己的生殖器官和性生理有一些了解,消除患者畸形的性观念。

性治疗的行为学疗法目的是消除患者不合理的性恐惧,并通过系统地暴露于先前造成性回避的情境而减轻她对性的回避。

在进行心理治疗后逐步进入性行为治疗。克服长期存在的性畏惧心理,其关键步骤是劝说夫妇共同参与到可能带来一定紧张的性行为治疗中。医生要耐心地向患病夫妇解释有关他们性问题的准确性质,并向他们说明治疗的步骤和原则,使患者及伴侣从原先的动摇、犹豫和回避,转变成乐意参与性行为治疗。在治疗中应遵循消除厌恶反应和改善性行为方式的原则,逐步进行一些限制性活动的缓慢治疗计划,使患者在有限的性活动中产生舒适感而不是厌恶感,使患者增加治疗信心。临床上常用的性行为疗法有:

这种方法的优点是行为干预可以促进无意识冲突的解决,因为治疗过程是从心理动力学角度加以高度概括的。用于治疗性厌恶的脱敏方法有去除患者精神防卫的趋势,并且会暴露他最大的不安全感和最大的脆弱性。这一过程常常激发出有意义的梦、记忆和联想。

脱敏疗法分四步:第一步:患者着少量衣服,对着镜子欣赏自己的体形,从各个方位欣赏自己。如果对自己的裸体无反感,再对着镜子观察自己的外生殖器结构,直到非常熟悉自己生殖器结构。第二步:在熟悉自己生殖器结构的基础上,每天两次用温水冲洗外阴、大小阴唇、阴道口,直到冲洗时毫无紧张感。第三步:每天冲洗外阴后,用手指对外生殖器进行自我抚摸,由外向内,抚摸大小阴唇、阴蒂、阴道口等,再用1个手指伸入阴道,直到2个手指伸入阴道不再紧张为止。第四步:采用女上位姿势进行性交,成功后再用其他姿势进行性交,直到能进行正常的性生活。

详见本章第六节。

凯格尔练习可恢复骨盆肌肉的张力,增加生殖器部位的血流,从而改善性功能。该训练分为四个基本步骤:①收缩耻骨尾骨肌,保持3秒,放松,重复练习。②收缩耻骨尾骨肌,收缩后立即放松,重复练习。③吸气,从阴道口开始上提,再沿着阴道上升,使盆底上举,重复练习。④呼气,使阴道下降,重复练习。这四种基本训练方法可以改善尿失禁,提高阴道润滑,增强性高潮的感受。

大部分性厌恶的患者需要进行药物治疗,以减轻焦虑和使性恐惧消失。

是治疗恐惧症的首选药物,它们是经过广泛研究的第一类既能阻断恐惧又不会干扰知觉过程或不会使清醒意识模糊的制剂。最常用药物有丙咪嗪、地昔帕明、阿米替林、去甲替林等。

该类药物属于第二线的抗恐惧药物,用于三环类抗抑郁药后效果不佳或患者不能耐受其副作用时。

对性功能的副作用很小,它比三环类和单胺氧化酶抑制剂有更优越的特点。

药物的副作用:第二线的抗恐惧药物和单胺氧化酶抑制剂都可能影响性反应周期中的性欲、性兴奋和性高潮,所以在治疗中只使用小剂量,即使出现副作用也是很轻微的。阿普唑仑对性功能影响的副作用很小,在服用抗恐惧药物后可使大约85\%的恐惧症患者的病情得到控制,因此,即使患者属于难于医治的类型,医生也有必要花费时间和精力去耐心寻找适当的药物,探索有效的剂量水平。

普及科学的两性教育,解除不必要的性顾虑,学会淡忘过去的“创伤”,勇敢迎接现在的幸福,培养夫妻感情,增进两性沟通、交流等,爱情的培养无疑是性厌恶最好的预防方式。

(赵良运 陶欣 张滨)

1.郭应禄,胡礼泉.男科学.北京:人民卫生出版社,2004

2.中华医学会精神科学会.中国精神疾病分类方案与诊断标准.南京:东南大学出社,1995

3.Robert Berkow,M.D.默克诊疗手册.北京:人民卫生出版社,1992

4.爱理斯.幸福密码.喀什:喀什维吾尔文出版社,2004

5.陈学明.爱情、爱欲与性欲———评“西方马克思主义”性伦理学.江苏行政学院学报,2004,6:10

6.Bodenmann G,Ledermann T,Blattner D,et al.Associations among everyday stress,critical life events,and sexual problems.J Nerv Ment Dis,2006,194(7):494-501

7.Levine SB.Reexploring the concept of sexual desire.J Sex Marital Ther,2002,28(1):39-51

8.Reece R.Causes and treatments of sexual desire discrepancies in male couples.J Homosex,1987,14(1-2):157-172

9.马晓年.现代性医学.第2版.北京:人民军医出版社,2004

10.刘继红,熊承良.性功能障碍学.北京:中国医药科技出版社,2004

11.Davis SR,Guay AT,Shifren JL,et al.Endocrine aspects of female sexual dysfunction.J Sex Med,2004,1(1):82-86

12.Traish AM,Noel Kim,Kweonsik Min,et al.Roles of androgens in female genital sexual arousal:receptor expression,structure,and function.Fertil Steril,2002,77(Suppl 4):S11-18

13.Richard D.Hayes B.Sc,Lorraine Dennerstein M.B.B.S.,Bennett,et al.Alessandra Graziottin.Relationship between hypoactive sexual desire disorder and aging.Fertil Steril,2007,87(1):107-112

14.Xin ZC,Tian ZJ,Ym XR,et al.Studies of sexual dysfunction in young and middle-aged Chinese women with BISF(Brief Index of Sexual Function for Women).Int J Import Res,2000,(Supple 2):S17

15.Erdogan Aslan,Michelle Fynes.Female sexual dysfunction.Int Urogynecol J,2008,19(2):293-305

16.T.Stadler,M.Bader,S.Ückert,et al.Adverse effects of drug therapies on male and female sexual function.World J Urol,2006,24:623-629

17.Laumann EO,Paik A,Rosen RC.Sexual dysfunction in the United States:Prevalence and predictors.JAMA,1999,281:537-544


\chapter{第七章 勃起功能障碍}

勃起功能障碍(erectile dysfunction,ED)是中老年男性的常见疾病,严重影响患者的生活质量。目前ED的定义是“阴茎不能达到和(或)维持足够的勃起以顺利完成性生活”。本章就ED的相关临床问题进行简要介绍。


\section{第一节 阴茎勃起功能的生理}

控制阴茎勃起的脊髓中枢位于T13 ~L2 和S2 ~S4 ,这两个部位分别发出的交感神经纤维和副交感神经纤维汇合形成下腹下丛和骶神经丛,再发出神经纤维支配盆腔器官。支配阴茎的神经(海绵体神经)沿精囊和前列腺的后外侧走行,与尿道膜部一起穿过尿生殖膈。其中的一部分神经纤维与海绵体动脉和尿道球动脉伴行,支配阴茎海绵体和尿道海绵体;另一部分神经纤维走向阴茎的远端,支配阴茎中部和远端的阴茎海绵体和尿道海绵体。海绵体神经支配螺旋动脉和海绵体小梁平滑肌,负责调控阴茎勃起和疲软时的血管反应。控制阴茎的躯体运动神经中枢位于S2 ~S4 (Onuf核),其发出的运动神经纤维加入阴部神经,控制球海绵体肌和耻骨海绵体肌。阴茎的躯体感觉神经负责传导阴茎的疼痛、温度、触觉和振动觉。大脑对阴茎勃起的脊髓通路具有调控作用,尤其是下丘脑的视前内侧区和室旁核,以及海马等部位。

人体的阴茎勃起分为三种类型:生殖器直接刺激引起的勃起(接触或反射性),中枢刺激引起的勃起(非接触性或精神心理性),以及中枢来源的勃起(夜间阴茎勃起)。生殖器直接刺激引起的勃起来源与生殖器区域的触觉刺激,这种勃起由上脊髓控制,勃起的持续时间较短,并且不易被意识控制。引起勃起的中枢刺激更为复杂,包括记忆、性幻想、视觉和听觉刺激。中枢来源的勃起可以在没有刺激和睡眠时出现,睡眠时出现的阴茎勃起多发生在睡眠的快速动眼相。

阴茎海绵体白膜为由弹性纤维组织和胶原纤维组成的不规则网状结构,形成许多亚层组成的双层结构,内层呈环行包裹支持阴茎海绵体组织。从纵隔腹侧内层呈放射状分布到白膜背外侧束梁,对阴茎海绵体起支持作用,并维持阴茎环型结构。外层纵行分布于束带,从阴茎头延伸到阴茎脚,与耻骨下支相连,但是在5~7点间接近尿道海绵体处并没有外层分布。尿道海绵体没有外层和纵隔束梁,在阴茎勃起时尿道海绵体仍然保持柔软的结构状态。导静脉穿行于白膜内外层之间,其间距离很短,常斜行穿过外层。阴茎海绵体动脉和阴茎背动脉分支在阴茎勃起时提供海绵体额外的血液供应,这种阴茎海绵体外血液供应更为直接,血管被周围纤维鞘所包绕,防止阴茎在勃起时受白膜压迫而被阻断。

成对的阴部内动脉是阴茎的主要血供动脉,它有三个分支:阴茎背动脉、尿道球动脉和海绵体动脉。海绵体动脉供应阴茎海绵体,阴茎背动脉供应阴茎的皮肤、皮下组织和阴茎头部,尿道球动脉供应尿道海绵体。然而很多情况下存在起于髂外动脉、闭孔动脉、膀胱动脉和股动脉的副动脉。有的还可出现副阴部动脉,来源于髂外动脉和闭孔动脉,可以供应大部分阴茎的血供,与上述的三支动脉之间有侧支相吻合。阴茎的静脉回流主要通过阴茎背深静脉;尿道海绵体的静脉回流则通过螺旋静脉、尿道静脉和球静脉;而阴茎海绵体静脉的回流较为复杂:中远端主要通过阴茎背深静脉进入前列腺前静脉丛,而近端则通过海绵体静脉和脚静脉汇入前列腺前静脉丛和阴部内静脉。阴茎头静脉与背深静脉相交通,阴茎皮肤和皮下组织通过阴茎背浅静脉引流入隐静脉。

自主神经的活动能够使海绵体充血从而产生勃起。当阴茎充分勃起后,体神经的活动使耻骨海绵体肌收缩,压迫近端海绵体,使海绵体内的压力升高超过循环动脉压,使得阴茎勃起变硬。阴茎从勃起到疲软的全过程分为六期:1.萎软期———动静脉血流很少,阴茎血气分析与静脉相似;2.潜伏(注入)期———阴部内动脉的压力降低,血流在收缩期和舒张期都增加,海绵体内的压力不变,阴茎略延长;3.肿胀期———海绵体内的压力上升直至阴茎完全勃起,动脉血流相应下降,当海绵体内的压力超过动脉舒张期的压力时,就只有收缩期血流,阴茎变得更粗、更长,并且可以有搏动;4.完全勃起期———海绵体内的压力上升至动脉收缩期压力的80\%~90\%,阴部内动脉的压力也上升,但仍低于循环收缩期血压,动脉血流少于潜伏期,高于疲软期,尽管海绵体的静脉回流因阴茎勃起而受阻,但仍高于疲软期,血气分析接近动脉血的情况;5.坚硬勃起期———耻骨海绵体肌收缩,海绵体内的压力升高超过循环的动脉压力,使得勃起的阴茎变硬,在这一期,没有血流通过动脉,但这一期持续时间较短,这样可以避免缺氧或组织损伤的发生;6.消肿期———射精后或结束性刺激后,交感神经重新活跃,使得海绵体和动脉的平滑肌收缩,动脉血流降至疲软期水平,海绵体内的血液排出,静脉重新开放,阴茎恢复至疲软期状态。阴茎头部的血流动力学较为复杂,在完全勃起期,由于阴茎头缺乏白膜,使得头部在功能上形成动静脉瘘,因而无法充血。而在坚硬勃起期,由于回流的静脉受到压迫,阴茎头部也会出现充血。

阴茎勃起组织,尤其是海绵体、小动脉和动脉壁平滑肌,在勃起过程中起关键作用。在疲软期,这些部位的平滑肌因自身的张力和肾上腺素能神经的控制,只允许少量血流通过,血液中的氧分压为35mmHg。当神经冲动使得平滑肌松弛时,动脉和小动脉就舒张,血流量增加,海绵体窦扩张接受更多的血流,导致阴茎变长变粗。海绵体窦的扩张使得白膜压迫静脉,减少静脉的回流,海绵体内的压力和氧分压增加至100mmHg和90mmHg,阴茎呈勃起状态,随后耻骨海绵体肌收缩使得阴茎勃起变得坚硬(见图7-1)。

图7-1 阴茎勃起的机制

在阴茎疲软状态时(A),动脉、小动脉以及海绵体窦的平滑肌呈收缩状态。海绵体窦间隙和白膜下静脉丛开放,血液可自由回流至导静脉。在勃起状态时(B),海绵体窦壁和小动脉的平滑肌松弛,更多的血液流入海绵体窦。静脉被扩张的海绵体窦和白膜所挤压,使得其血流量减至最少。(改编自Campell Urology 9th ed)

雄激素对男性性成熟而言是必不可少的。睾酮能够调节促性腺激素的分泌和肌肉的发育;双氢睾酮调节男性性成熟,包括毛发的生长、痤疮、男性的秃顶和精子的生成。当男性发生性激素缺乏时,就会出现性欲减退和射精功能障碍。随着年龄的增大,睾酮、脱羟表雄酮、甲状腺素、退黑素和生长激素成进行性下降,性激素结合蛋白、垂体促性腺激素和泌乳素则增加。睾酮的水平与ED病情的严重程度无显著正相关。尽管睾酮水平降低时,患者夜间勃起的频率、幅度和持续时间都相应减少,但视觉性刺激时,阴茎仍能勃起,提示雄激素不是阴茎勃起所必需的。高泌乳素血症能够抑制中枢多巴胺神经的活动,并导致促性腺激素释放减少,患者出现生殖和性功能的障碍。

控制阴茎勃起的神经元包括肾上腺素能、胆碱能、非肾上腺素非胆碱能。肾上腺素能神经调控海绵体平滑肌的张力,通过释放去甲肾上腺素,保持阴茎的疲软状态。目前认为,交感神经介导的平滑肌收缩是通过突触后的α-1a和α-1d肾上腺素受体完成的,并受突触前的α-2受体调节。胆碱能神经能够通过神经元间的抑制作用抑制肾上腺素能神经的冲动,并通过乙酰胆碱促使血管内皮释放一氧化氮(NO),使得平滑肌舒张。而血管内皮释放的内皮素能使血管收缩,也参与了阴茎勃起后疲软的调控。促使阴茎勃起的主要神经递质是来源于副交感神经的非肾上腺素非胆碱能神经元—NO。一旦海绵体窦内的血流增加,血管内皮即释放NO,进一步松弛平滑肌,促进勃起的发生。另外,血液中的氧分压和内皮释放的一些其他物质,如前列腺素、血管紧张素等,也都参与了阴茎勃起的过程。

平滑肌的收缩是由Ca2+ 介导的。当细胞内游离Ca2+ 从静息时的120~270nM增加至500~700nM时,钙调蛋白-4与Ca2+ 的复合物和肌球蛋白轻链激酶结合,活化的激酶使得轻链磷酸化,减少其对肌动蛋白的抑制作用,从而引起平滑肌的收缩。当胞浆内游离钙离子水平恢复到静息水平,钙敏途径就被激活,与G蛋白偶联的兴奋性受体激活后能够在不改变胞浆内游离钙离子浓度的情况下通过增加对钙的敏感性而引起平滑肌收缩。这一途径包含一种小的单体G蛋白RhoA,它能激活Rho激酶。激活的Rho激酶磷酸化可抑制平滑肌肌球蛋白磷酸酶,阻止肌丝的去磷酸化,维持平滑肌的张力。因而,阴茎平滑肌的收缩由细胞内的Ca2+ 水平所调控,而张力是由钙敏途径控制的。

性刺激能够使神经末梢和内皮细胞释放一氧化氮进入海绵体和血管平滑肌细胞,激活鸟苷酸环化酶,产生第二信使cGMP。cGMP随后激活蛋白激酶G,使得钾离子和钙离子通道蛋白磷酸化,导致超极化,细胞内的钙离子浓度下降,肌球蛋白的头部与肌动蛋白解离,从而使得平滑肌松弛。cAMP是参与平滑肌松弛的另一个第二信使,其信号通路涉及的分子包括腺苷、降钙素基因相关蛋白、前列腺素。这两种第二信使都能激活相应的蛋白激酶,它们降低细胞内游离钙离子浓度和松弛平滑肌的机制包括钾离子通道开放和超极化,细胞内的钙离子进入内质网,抑制电压依赖的钙通道减少钙内流。与之相反的是去甲肾上腺素、苯肾上腺素、内皮素能激活磷脂酶C,产生三磷酸腺苷和二酰基甘油,从而增加了细胞内钙离子的浓度,使得平滑肌收缩。当cGMP、cAMP被相应的磷酸二酯酶降解为GMP、AMP后,就会导致勃起的阴茎疲软。目前已知的磷酸二酯酶有11种亚型,在阴茎中以磷酸二酯酶5为主,主要降解cGMP。因此在ED患者中,采用磷酸二酯酶5抑制剂能够改善勃起功能。


\section{第二节 ED的流行病学}

许多作者对ED的发病率进行了观察,由于采用的调查方法和研究人群不同,其ED的发病率也有一定差异。MMAS(The Massachusetts Male Study,MMAS)是基于社区人群的ED发病率的调查研究,随机抽取1047名40~70岁的男性,采用调查问卷的方式对研究对象的勃起功能进行横断面和纵向研究。结果发现,ED总发病率为52\%,其中轻度ED的发病率为17.2\%,中度为25.2\%,重度完全性为9.6\%;40岁和70岁年龄段相比较,严重ED的发病率从5.1\%增加到15\%,中度ED的发病率从17\%增加到34\%,而轻度ED的发病率相似,约为17\%。随访中发现,ED的发病率为25.9/1000人·年(95\%CI:22.5~29.9),年发病例数每隔10年就有所增加:40~49岁12.4例,50~59岁29.8例,60~69岁46.4例。我国学者对上海人群的ED发病率进行了调查,各年龄段的发病率为:40~49岁为32.8\%,50~59岁36.4\%,60~69岁为74.2\%,70岁以上86.3\%。Prins等对有关ED发病率的23项研究进行了系统分析,总结出各年龄组ED的发病率为:19~29岁为7\%,30~39岁为2\%~9\%,40~49岁为9\%~11\%,50~59岁为16\%~18\%,60~69岁为34\%,70~80岁为53\%。尽管各研究ED发病率的具体数值存在差异,但从整体上来看,ED是与年龄相关的疾病,其发病率随着年龄的增长而相应增加。

与ED有关的常见危险因素包括:总体身体健康状况、糖尿病、心血管疾病、并发其他的泌尿生殖系疾病、精神和心理障碍、其他的慢性疾病和社会经济状况。吸烟也是ED的危险因素,并且随着吸烟量的增加,ED的发病率也相应增加,存在着剂量依赖的关系。在前列腺癌和其他盆腔肿瘤进行手术治疗或放疗后的患者ED发生率较高。


\section{第三节 ED的病因和分类}

ED的发病与机体器质性和心理性因素密切相关。依据其病因和发病机制的不同,国际勃起功能障碍研究协会建立了ED的分类方法(见表7-1)。既往曾认为90\%的ED是心理性的,但目前多数学者认为心理性和器质性的混合性因素是发生ED的最常见病因。

表7-1 男性勃起功能障碍的分类

心理性ED通常是由于紧张、压力、抑郁、焦虑或感情不和等精神心理因素引起的。性行为和阴茎勃起由下丘脑、边缘系统和大脑皮质控制。因此,兴奋性和抑制性的信息被传递到脊髓勃起中枢,进而诱发或抑制勃起。心理性ED的可能机制包括:中枢神经递质的失衡、大脑对脊髓勃起中枢的过度抑制、一氧化氮的释放减少、交感神经过度兴奋。

任何影响大脑、脊髓、海绵体和阴部神经功能的疾病或功能障碍都可引起ED。据估计,20\%的ED患者病因是神经源性的。脊髓损伤的性质、部位和程度决定了勃起功能的状况。有95\%的高位脊髓完全损伤患者保留有反射性勃起,但低位损伤者仅25\%。与ED相关的脑部病变包括:帕金森氏病,中风,创伤,肿瘤,Shy-Drager综合征,外周神经病变,如糖尿病、长期嗜酒、维生素缺乏等,都可能导致神经纤维的神经递质释放减少,从而引起ED的发生。当控制阴茎勃起的神经受到创伤、手术、放疗损伤时,也会导致ED的出现。

许多内分泌疾病与ED的发生密切相关,如糖尿病、垂体功能低下、高泌乳素血症、甲状腺功能亢进或低下、垂体肿瘤、皮质醇增多症和性腺功能低下等。任何下丘脑-垂体轴的功能障碍都会导致性腺机能低下。促性腺激素分泌不足引起的性腺功能低下可以是先天性的,也可以由肿瘤、外伤引起;促性腺激素分泌增加引起的性腺功能低下与肿瘤、外伤、手术和腮腺炎后的睾丸炎相关。高泌乳素血症,不管由垂体腺瘤还是药物引起,常发生血循环中睾酮水平降低,与泌乳素水平增高抑制促性腺激素释放激素的分泌有关,能导致ED和性欲减退。

髂内动脉-海绵体动脉-螺旋动脉出现病变(如外伤或动脉硬化)时,就可能导致动脉阻塞,减少海绵体动脉血流和降低压力,从而延长达到最大勃起限度所需的时间,降低了勃起阴茎的硬度,导致ED的发生。大多数动脉性ED的患者,阴茎血管硬化是全身粥样硬化过程的一部分。Michael等发现冠状动脉硬化性心脏病和ED的发病率和发病年龄相一致。ED和心血管疾病具有相同的危险因素,例如高血压、糖尿病、高脂血症和吸烟。长距离骑车也是发生血管性和神经性ED的危险因素。

动脉病变据病变部位可以分为阴茎外和阴茎内动脉病变。阴茎外动脉病变有些可以采用手术治疗,包括阴部动脉、髂内动脉、髂总动脉和主动脉的病变、盆腔窃血综合征、盆腔创伤。阴茎外动脉病变可由动脉硬化、糖尿病等引起,目前尚无理想的手术治疗方法。

海绵体静脉闭塞不全是血管性ED的最常见原因。退行性病变(Peyronie病、老龄、糖尿病)或者白膜外伤,如阴茎折断,能破坏对白膜下静脉和导静脉的压迫。静脉闭塞不全可由多种病理生理过程引起:白膜退化改变、纤维弹性结构改变、小梁平滑肌松弛缺陷和静脉分流。糖尿病和动脉硬化的患者更易出现平滑肌萎缩、纤维增生和内皮功能缺陷,从而促进ED的发病。

海绵体静脉闭塞不全按病因可分为5类:1型,阴茎海绵体外的较大静脉病变(常见于先天性疾病);2型,因阴茎白膜的弹性降低导致静脉的闭合不全(常见于Peyronie病和老化);3型,因海绵体平滑肌的纤维化、退行性变等导致的松弛障碍,影响静脉闭合;4型,神经递质释放减少(常见于神经源性和心理性ED,内皮功能障碍);5型,阴茎海绵体与尿道海绵体、阴茎头之间存在异常分流(常见于先天性疾病、创伤,或者阴茎异常勃起获得性静脉分流)。

许多药物都能导致ED,但其发病的确切机制尚不清楚,并且很少有专门针对药物引起ED的随机对照研究。与药物有关的性功能症状不仅仅限于对勃起功能的损害,可能还包含关于性欲、性兴奋和性欲高潮方面的主诉。通常来说,可能干扰神经中枢的神经内分泌功能或影响局部阴茎平滑肌的血管神经的药物都可能导致ED。参与调控性功能的中枢神经递质通路包括5-羟色胺能、肾上腺素能和多巴胺能神经。而抗抑郁药、抗精神病药、中枢作用降压药都可能影响这些神经,从而引起性功能的改变。选择性5羟色胺再摄取抑制剂是目前常用的一类治疗抑郁的药物,应用这些药物的患者约有50\%的性功能发生了改变。β受体阻滞剂可以通过激活阴茎的α1受体而引起ED,相反,α1受体阻滞剂和血管紧张素Ⅱ的受体拮抗剂则可以改善勃起功能。噻嗪类利尿剂和非噻嗪类利尿剂都可能导致ED。其他可能导致ED的药物还有阿片样物质、抗反转录病毒药物和H2受体拮抗剂。

抗雄激素药物通过抑制雄激素的产生或拮抗雄激素受体的作用而发挥雄激素阻滞作用。雄激素被认为是通过调控中枢神经系统的雄激素受体而调控性行为的。雄激素缺乏对性活动的影响程度存在差异,可以从完全丧失到完全正常。5α还原酶抑制剂非那雄胺是临床上用于治疗良性前列腺增生的常用药物,它是对循环中睾酮水平和性功能影响最小的抗雄激素制剂。采用5mg/d非那雄胺治疗的患者约5\%主诉有性欲下降和ED。雌激素和具有抗雄激素作用的药物,如酮康唑、LHRH激动剂、非甾体类抗雄激素药(比卡鲁胺)和甾体类抗雄激素药(环丙孕酮),都可能引起性功能的改变。

手术和创伤可以在不同平面影响控制勃起的神经和阴茎的血流,如脑和脊髓的损伤和手术,后腹膜淋巴结清扫术,前列腺癌根治术,腹会阴直肠切除术等,这些都可能导致ED的发生。Cunsolo等报告腹会阴直肠切除术后ED的发生率为59\%;前列腺癌根治术后完全性ED和部分性ED的发生率分别为26\%~100\%和16\%~48\%;McDermott等发现179例因尿道狭窄而行尿道内切开术的患者,术后有4例(2.2\%)发生ED;骨盆骨折后,有高达61\%的患者出现不同程度的ED表现;脊髓损伤的患者中,骶髓上平面损伤后ED的发生率为5\%~8\%,而骶髓及其下平面损伤后的ED发生率高达60\%~80\%。

研究显示,吸烟与ED的发生密切相关。Mannino调查了4462例男性,年龄均在31~49岁,吸烟者ED发生率为3.7\%,而非吸烟者为2.2\%。MMAS研究发现,在心脏病患者中,吸烟者出现完全性ED的比率为56\%,而未吸烟者为21\%。在ED患者中,吸烟者和曾经吸烟者的比率为58.4\%和81\%,这一比例显著高于没有ED的人群中吸烟者的比例。研究发现,吸烟者的阴部内动脉、海绵体动脉有动脉硬化形成。吸烟对海绵体平滑肌的收缩作用可诱发血管收缩和阴茎静脉漏。另外,吸烟也是心血管疾病、糖尿病等与ED发生密切相关的疾病的危险因素。

小剂量的乙醇由于其扩张血管作用和抑制焦虑作用可改善勃起和提高性欲;但是大量的乙醇可引起中枢镇静作用,使性欲降低,甚至引起短暂的勃起功能障碍。在嗜酒者中,ED的发生率高达54\%。慢性乙醇中毒导致肝功能障碍,使睾酮水平降低,雌二醇水平增加,还可引起多发神经病变,也可影响阴茎神经。

随着年龄的增长,男性的性功能逐渐降低,出现一系列改变,包括勃起潜伏期延长、硬度减低、缺少强有力的射精、精量减少和不应期较长。另有研究发现,勃起功能减退的老年男性睾酮水平降低而促性腺激素相对正常,表明下丘脑垂体轴功能障碍。

糖尿病与ED的发病密切相关。1型和2型糖尿病患者中,ED的发病率分别为32\%和46\%,并且糖尿病患者中ED的发病年龄较早,在30岁时ED的发病率约15\%,60岁时上升到55\%。MMAS研究发现,糖尿病患者ED发病率是无糖尿病者的3倍(分别为28\%和9.6\%)。研究显示,糖尿病可能通过影响下列因素引发ED:心理作用、中枢神经系统作用、雄激素分泌、外周神经活动、内皮细胞功能和平滑肌的收缩力。

心血管疾病,如高血压、动脉粥样硬化、高脂血症等常伴发ED。MMAS研究发现,在心脏病患者中ED的发病率为39\%,未治疗的高血压患者的发病率为15\%,都显著高于研究总体的ED发病率(9\%)。在急性心梗的患者中,64\%的患者存在不同程度的ED,在行冠脉手术的患者中,ED的发生率为57\%。在高血压患者中发现,ED的发生率显著高于没有高血压的患者,并且前者ED的病情更重,随访研究发现,伴有ED的高血压患者出现心脑血管事件的风险更高。心血管疾病与ED发生密切相关的可能机制是血管病变导致阴茎的血流动力学异常。

研究显示,慢性肾衰竭患者中的ED发病率为20\%~50\%。接受透析治疗的慢性肾衰竭患者中严重ED的发生率是45\%,并且随年龄的增长、合并糖尿病和未应用血管紧张素转换酶抑制剂而相应增加。对肾移植患者的研究发现,移植术后多数患者的勃起功能获得显著改善,这一研究证实了慢性肾衰竭与ED的发病密切相关。慢性肾衰竭引发ED的机制包括下丘脑垂体性腺轴的功能紊乱,高催乳素血症,动脉粥样硬化疾病和心理性因素等。

其他的疾病,如慢性肺疾病、肝硬化等也都能导致ED的发生。


\section{第四节 ED的诊断}

ED的诊断和治疗以患者为中心,证据为原则。需要详细采集患者的病史、合并疾病、心理状况、全面的体格检查及辅助检查,这些均有助于ED的诊断和鉴别诊断。必要时需要性伴侣配合,更好地明确病情,制订治疗方案。

病史的采集应包括详细的合并疾病,药物使用情况,性生活和精神心理状况。采集病史可以了解合并存在的疾病状况(如糖尿病)、手术治疗(如前列腺癌根治性切除术)情况及其与ED发生的可能关系,有助于鉴别器质性和心理性ED,评价治疗药物对性功能的影响以及ED治疗药物对合并疾病的影响(如治疗ED的磷酸二酯酶抑制剂对某些冠心病患者应慎用)。详细了解患者性生活的状况,有助于ED的确诊,了解ED的发病时间,持续时间,病情的严重程度,以及与合并疾病和精神心理状况之间的关系。必须明确患者除了ED之外,是否还合并有其他性功能障碍的情况(如性欲、射精和性高潮)。必要时需要患者和性伴侣一起接受评估,以便更准确全面地了解患者的性生活状况。应该仔细评价患者与性伴侣的关系。性功能障碍和社会交往、工作情况一样都会影响到患者的自信、社交能力和工作表现。

ED患者应进行全面的体格检查,重点检查泌尿生殖系统、内分泌系统、心血管系统。体格检查中有时会发现性功能障碍的明显诱因,如阴茎硬结症,性腺功能低下(睾丸体积较小,第二性征不全)。神经病变以及糖尿病的患者可能出现外周神经系统病变的体征,生殖器、会阴部感觉以及阴茎海绵体反射试验有助于评价神经性ED。

男性性功能障碍的实验室检查应包括空腹血糖、血脂以及睾酮的检测。糖尿病患者还应检测糖化血红蛋白。如果患者的睾酮水平下降,必要时可以检测泌乳素,FSH和LH。一些患者应了解甲状腺机能。50岁以上的以及有前列腺癌家族史的患者要测量前列腺特异抗原(PSA)。

有关勃起功能评价的最常用的问卷是15项的国际勃起功能指数(International Index of Erectile Dysfunction,IIEF)。IIEF评价了男性性功能的5个方面:勃起功能、性高潮状况、性欲、性交满意度,以及整体的满意度。采用IIEF评分可以将患者的勃起功能进行一定的量化,有助于了解病情的严重程度,以及治疗后病情改善的程度。从IIEF拣选出5个项目,其中4个是关于勃起功能方面的,就构成了IIEF-5(见表7-2)。根据IIEF-5的评分,ED被分为5度:严重(5~7分),中度(8~11分),中低度(12~16分),轻度(17~21分),以及没有ED(22~25分)。

根据过去6个月内的情况进行评估

表7-2 男性勃起功能问卷IIEF-5

对于病情较为复杂的患者,如阴茎畸形,盆腔或会阴外伤,ED病因不清等,在基本的病史、体格检查和实验室检查的基础之上,需要采用一些特殊的检查,了解患者的神经、血管、内分泌系统的情况以及精神心理状况,有助于了解ED的病因及制订治疗方案。

阴茎血管功能评估的目的是诊断动脉功能或静脉闭塞功能的障碍。常用的检查包括:海绵体内注射与刺激的结合试验,多普勒超声,阴茎海绵体灌注动力和造影检查,选择性阴茎血管造影。

海绵体内注射与刺激的结合试验是最常用的ED诊断试验,该实验采用血管扩张剂注射入海绵体内,并对生殖器或视听进行性刺激,然后对患者的阴茎勃起状况进行评估。在进行试验之前,应详细告知患者进行该检查的目的、意义及风险。该实验可以对阴茎血管的功能进行直接而客观的评估。常用的血管扩张剂有前列腺素E1、联用罂粟碱和苯妥拉明。如果试验发现阴茎勃起的硬度较好,且持续时间超过10分钟,则认为阴茎的静脉功能是正常的。但由于动脉功能在轻度受损时,阴茎也会出现上述勃起反应,因而检查结果不能除外动脉功能的异常。

多普勒超声检查是评估ED血管病变最可靠、损伤最小的方法。它包括海绵体内注射血管活性药物和彩色多普勒血管超声检查。多普勒超声采用的是高频(7—10MHz)彩色脉冲探头,可以进行实时超声显像,观察阴茎背动脉以及阴茎海绵体动脉,进行血流动力学观察。性刺激后,多普勒超声观察到的动脉的正常反应是阴茎根部动脉血流的峰值>30cm/s。超声检查还可以发现阴茎的病变,如阴茎硬结症,钙化,增厚的血管壁,海绵体纤维化。当多普勒超声发现流速较高的收缩期血流(峰值>30cm/s)和持续存在的舒张末期血流(>5cm/s)伴有自慰后阴茎快速疲软时,则考虑存在静脉性ED。用于诊断静脉漏的指标有:舒张期静脉血流流速>5cm/s和(或)阻力指数<0.75。

阴茎海绵体灌注动力检查包括海绵体内注射血管活性药物(罂粟碱+酚妥拉明+前列腺素E1),随后采用生理盐水进行灌注,测量海绵体内的压力以评价阴茎静脉的功能。当静脉功能正常,海绵体内的压力为100mmHg时,海绵体的灌注流速<10mL/min,而且灌注的流速下降后,海绵体内的压力应在30s时<50mmHg。静脉闭塞功能不全的表现为盐水注入时海绵体内压力不能升高至平均收缩期血压水平,或停止注射后海绵体内压力迅速下降。

海绵体造影检查可用于发现静脉漏的部位。血管舒张剂诱导勃起后,向阴茎海绵体内注射造影剂,从而显示静脉泄漏的部位。若静脉闭合功能正常,则很少或没有造影剂出现在海绵体之外,如患者因先天性或创伤性因素导致静脉漏,则可以在造影时发现病变部位存在造影剂外漏。常见的静脉漏的部位是龟头、球海绵体、阴茎背部表浅静脉和海绵体及阴茎脚部静脉,大部分患者可以看到不止一处泄漏。

动脉血管造影检查通过向海绵体内注射一种血管扩张剂后在阴部内动脉选择性X线造影检查,观察海绵体动脉,髂血管、阴茎背血管及腹壁下血管的解剖结构和影像学表现。血管造影检查主要用于对复杂的ED患者的评估,最好的指征就是继发于创伤性动脉断裂的年轻的ED患者或有会阴挤压伤的患者。在这些患者中,血管造影检查对于计划外科重建手术是必不可少的。

海绵体动脉闭合压是一个反映阴茎血压变化的参数,其具体测量方法是在海绵体内注入血管活性药物后,向海绵体内持续灌注生理盐水,使海绵体内的压力超过动脉收缩压,采用多普勒超声探头检测阴茎根部的动脉血流,停止生理盐水的灌注,海绵体内的压力开始下降,当压力降至超声刚好能够发现海绵体动脉血流时,这时的压力称为海绵体动脉闭合压。若海绵体和分支动脉压力之间的梯度大于35mmHg,并且右侧和左侧海绵体动脉之间压力相等,则被定义为正常。该结果与动脉血管造影检查和多普勒超声血管检查的一致性较好。

生理学上,勃起有三种类型:夜间性、心因性和反射性。神经系统检查用于评价外周、脊髓、脊髓上中枢,还有与勃起类型和性唤起相联系的体神经及自主神经通路。然而,在阴茎勃起中的神经功能异常是一种复杂的现象,而且除了一些特殊情况,神经学检查很少会改变治疗方案。神经学检查主要用于研究或法医鉴定(如创伤和手术并发症),也用于发现可逆的神经病变(如骑长途自行车导致的阴茎背神经的病变),评价糖尿病或盆腔损伤导致的神经病变损伤的范围,决定是否需要神经科的专科处理(如可疑脊髓肿瘤)。但神经学检查对常规临床诊断仍缺乏足够的敏感性和特异性。

体神经通过测试神经传导速度和诱发电位来评价,这些测试的重复性和有效性较高。自主神经系统检查的可靠性较低是因为它需要同时测量受体、小纤维和靶器官的一连串事件或反应。中枢和外周的交感和副交感神经系统中的复杂的相互反应使得自主神经系统的检查变得困难。目前,自主神经系统的检查尚缺乏标准,并且其重复性、有效性和可比性较差。

这是简单检查阴茎背神经传入通路的方法。采用特定频率而振幅可调的电磁振动装置,刺激阴茎体两侧和阴茎头表面,测定患者对特定频率、不同振幅的知觉敏感阈值。

该检查能客观反映骶髓、马尾神经和阴部神经的功能完整性。这项测试通过环绕阴茎放置两个环形电极来进行,一个位于冠状沟,另一个位于距冠状沟3cm处。同心的针状电极被放置在球海绵体肌左右侧记录,通过直流电刺激发送方波脉冲。每个刺激反应的潜伏期是从开始刺激时到开始反应的时间。一个异常潜伏期的定义为:长于平均值(30~40毫秒)三个标准差以上。表示有较高的神经病理学改变的可能性。

温度阈值测试通过检测细小感觉神经纤维的电导系数,间接反映自主神经功能状况。阴茎温度感觉测试与勃起功能的相关性较强,是一种诊断神经源性ED的新型工具。

80\%的夜间阴茎勃起发生在快速动眼睡眠期,他可以发生于几乎所有年龄的男性,并且不受精神因素的影响。平均的夜间勃起次数为3~5次,持续时间平均30~60分钟。随着年龄的增长,夜间勃起的总时间也相应减少。夜间阴茎勃起可以通过很多方法检测:邮票试验;测量箍带;睡眠实验室夜间阴茎勃起和硬度测定(NPTR)等。目前较为常用的是Rigiscan NPTR,该装置便于携带,可用于门诊患者。它可以记录勃起事件次数、勃起度(由张力量表测量周径变化)、阴茎硬度和夜间勃起持续时间。NPTR正常标准为:每晚4~5次勃起事件;平均持续时间超过30分钟;阴茎周径增加在根部超过3cm,末端超过2cm;最大硬度在阴茎根部和末端均超过70\%。夜间阴茎勃起测试的优点主要是它相对脱离了心理影响,可用于鉴别心理性ED和器质性ED。另外联合睡眠监测的仪器,还能发现与睡眠相关的异常。Heaton等提出NPTR的指征如下:①可疑睡眠失调;②ED原因不明;③对治疗无反应;④计划行外科治疗;⑤合法的敏感病例;⑥在安慰剂对照的药物试验中测定药物的影响;⑦可疑有精神性病因。

心因性ED的现代定义是持久地不能达到或保持满意的勃起以成功地进行性行为,主要或仅仅因为精神性的或伴侣间关系的因素造成。心理性ED可以分为广泛性的和特定条件下的;终生的(原发的)和获得性的(继发的,包括物质滥用或精神疾病)。由于精神因素,以及性功能正常或不正常的性伴侣之间的关系衡量是较为复杂的,因此,在诊断时与患者的沟通需要较高的技巧,并且以此为精神心理评价的主要方式。心理性ED的病史采集包括ED发生的时间,选择性ED的情况(如对某一性伴侣时,勃起功能正常,而对另一性伴侣时,勃起功能较差,或者在手淫或性幻想时勃起功能正常,而在性交时却出现勃起功能障碍),需要了解患者是否在夜间睡眠时阴茎勃起功能正常,而在清醒时出现功能障碍。通常心理性ED的患者可能伴随焦虑、恐惧、负罪感、精神紧张,或可能导致ED的宗教信仰或父母亲传输的信念。如果患者的病史提示存在心理性及器质性因素,应诊断为混合性ED,治疗时两方面都需要关注。

大部分ED患者没有必要进行心理评估,但其对评价和治疗有深层心理问题的患者非常有用。目前常用的可被用于评价ED的心理测量工具有三组:(1)人格问卷;(2)抑郁量表;(3)性功能障碍和人际关系因素问卷。明尼苏达多项人格量表(MMPI)-2是评定患者人格和相关性功能障碍的有利工具。贝克抑郁量表是自评测试,当得分超过18则提示为有临床意义的抑郁。简易的婚姻调节测试(用于已婚夫妻)即双值的调节量表(用于未婚者)可被用于确定整体人际关系质量。


\section{第五节 勃起功能障碍的治疗}

ED治疗历史中有三个标志性飞跃,即1973年的可膨胀性阴茎起勃器植入手术疗法,1982年的阴茎海绵体内药物注射疗法和1998年的口服药物疗法-枸橼酸西地那非。今天,大部分ED患者首先可以选择口服药物磷酸二酯酶-5抑制剂,如果治疗失败或者希望得到进一步治疗,则可考虑第二线(真空缩窄装置,经尿道给药)和第三线(阴茎海绵体药物注射)治疗方法。对上述治疗失败或拒绝上述治疗的患者,通常可选择阴茎起勃器植入手术治疗。

Esposito等在2004年报道了改变生活方式(通过减少热量摄入和增加身体锻炼使得体重下降超过10\%)对勃起功能的影响,研究对象为患有ED(IIEF为21或更低)的110例肥胖患者(体重指数≥30),年龄在35岁到55岁之间,没有糖尿病、心脏病和高血压。改变生活方式组的IIEF评分由13.9±4.0增加至17±5,而对照组的IIEF评分无显著改善。这一研究证实了生活方式的改变有助于勃起功能的改善。

ED的重要病因是血管硬化。血管硬化与代谢综合征(血脂异常、高血压、中心型肥胖、高胰岛素血症)密切相关,而这些也是ED的危险因素。因此,规律锻炼,健康饮食,戒烟和控制饮酒可以减少ED发生的危险性,从而改善勃起功能。

长时间的自行车运动也是ED的危险因素,因为这项运动可能压迫会阴的阴茎动脉。Schwarzer等发现自行车爱好者ED的发病率达到4\%,而相同年龄的游泳爱好者的患病率为2\%。因此,如果发现存在阴茎血管受压,则需要改变自行车坐椅或改变骑自行车的方式。

当患者使用某一药物后出现性功能障碍时,需要考虑该药物是否与性欲减退、勃起障碍、射精加快或延迟等性功能障碍相关。通常,换用不同种类的药物是首选的处理方法。降压药能够降低血压,但又可能影响阴茎的血管,从而影响勃起功能。高血压患者在服降压药后出现性功能障碍时,可换用α-受体阻滞剂、离子通道阻断剂和血管紧张素转化酶抑制剂等来逆转一些患者的勃起功能。采用抗抑郁药治疗的患者出现性功能障碍时,可以考虑采用等待观察、换用其他药物(丁氨苯丙酮、聚酰胺纤维、丁螺环酮、米氮平)、暂时停药、选择性5羟色胺再吸收抑制剂减量,和(或)使用PDE-5抑制剂。

性心理治疗主要适用于具有导致ED的潜在的精神心理因素的患者,与磷酸二酯酶抑制剂相比,其治疗的时间较长,但对于一些具有特定病因的患者,性心理治疗有可能使患者治愈。目前性心理治疗的主要手段包括:感知行为的纠正以改变不适应的感觉,行为疗法(脱敏治疗和建立自信心),挖掘患者过去的行为以便进行系统的治疗,以及配偶共同治疗。在一些混合性ED患者中,性心理治疗能够缓解紧张情绪,清除对药物和手术治疗不切实际的期望。

与ED相关的甲状腺、肾上腺、垂体后叶和下丘脑等功能障碍的患者,其评估及治疗主要由内分泌医师进行。患有ED的老年男性具有不同程度的性腺功能低下。对磷酸二酯酶抑制剂反应较差的患者,如伴有性腺功能低下,可以考虑进行雄激素补充治疗,这有可能改善磷酸二酯酶抑制剂的疗效。

用于注射的睾酮制剂,包括环戊丙酸盐睾酮和庚酸睾酮,注射后可使血中的睾酮达到正常水平。但这些药物不能复制睾酮在人体内的生理节律。相反,这些药物通过深部肌肉注射后(200~250mg,每两周一次),导致血睾酮在72小时内超出生理水平,并且激素水平在10~12天内呈指数下降。早期高水平的激素会使一些患者有不适的反应,而另外一些患者的整体自我感觉和性欲都得到改善。

睾酮透皮吸收贴剂是通过皮肤吸收睾酮的方式补充睾酮。患者在早晨使用,能够模拟人睾酮水平的生理变化。睾酮透皮吸收贴剂可用于阴囊局部皮肤、手臂、后背及臀部。用于阴囊皮肤贴剂(4到6mg)能够使睾酮达到65\%的生理浓度,贴剂不易固定,因为阴囊皮肤中有高浓度的5α还原酶,能够产生高浓度的双氢睾酮。而用于手臂等部位的睾酮经皮治疗系统每天用5mg剂量,避免了经阴囊皮肤治疗的不便,使用较为方便,是目前主要的使用方式。睾酮贴剂的主要副作用有瘙痒、慢性皮肤刺激和过敏性接触皮炎。

睾酮也可以采用凝胶的方式通过皮肤吸收使用。1\%睾酮凝胶剂可以含有50mg,75mg,或100mg的睾酮。每天早晨在肩、上臂或腹部清洁干燥的皮肤用药。由于睾酮可以通过皮肤扩散,因此使用后必须将手彻底洗净。另一种局部使用的睾酮凝胶剂含1\%的睾酮,1次应用能够提供持续24小时的经皮肤吸收。

因为存在肝脏的首过效应,通过口服补充睾酮时,睾酮的剂量应较大,每天口服剂量需要超过200mg才能保持正常的睾酮浓度。但口服大剂量的睾酮有肝毒性,可能导致肝炎、胆汁淤积性黄疸、肝癌、出血性肝循环和肝癌。唯一能保持口服药活性且安全的形式是十一酸酯睾酮,这是由于它有亲酯性长链,能够部分通过淋巴系统吸收,从而减少了肝脏对药物的灭活。服用十一酸酯睾酮2~3小时后血中的浓度达到高峰,6~8小时后恢复到服药前的水平。饮食对药物吸收影响较大,具体的服用剂量因人而异。睾酮还可以通过舌下含服药物,口腔黏膜贴剂等方式使用。

在高泌乳素血症伴或不伴性腺功能低下的患者中,单纯睾酮治疗不能改善性功能。治疗前首先要排除刺激性药物,如:雌激素、吗啡、镇静剂和精神类抑制药物。溴麦角环肽是一种多巴胺激动剂,它能够降低泌乳素在血中的浓度和恢复睾酮的正常水平,还可用来缩小分泌泌乳素的腺瘤的大小。如果药物治疗的疗效欠佳或因视神经受到挤压出现视觉障碍时,可考虑神经外科手术。

睾酮补充治疗对没有禁忌证的性腺功能低下的年轻男性有明确的治疗指征。然而,在某些患者中可能弊大于利。超过生理浓度的睾酮水平抑制了LH和FSH的生成,导致不育、男性女化。长期治疗最常见的实验室检查异常就是红细胞增多。其他的异常还包括:红细胞压积升高,血栓素-A2的升高和血小板的聚集力增加,而这些是心血管疾病的危险因素。雄激素能诱发或恶化睡眠呼吸暂停。

由于雄激素与前列腺癌的发生密切相关,因此雄激素补充治疗对前列腺的安全性在临床上就显得十分重要。目前许多研究证实,在前列腺正常的患者中,雄激素补充治疗不会诱发前列腺癌,安慰剂对照研究表明,与对照组比较,补充雄激素的患者的前列腺体积、PSA和下尿路症状方面无显著差异。尽管雄激素补充治疗可能恶化潜在的前列腺癌,但由于很多老年患者在激素补充治疗后性欲和勃起功能得到了改善,因此不能完全对其进行否定。与患者详细沟通,使患者了解激素补充治疗的利弊,如果患者要求激素补充治疗,在激素治疗之前,患者应行前列腺指诊和PSA的检测。如果有怀疑,应该做超声引导下的前列腺穿刺活检。已经确诊的乳腺癌和前列腺癌是激素补充治疗的绝对禁忌证。在激素治疗过程中,应每6个月给患者做前列腺指诊和PSA检测,实验室检查应包括:血红蛋白水平、红细胞压积、肝功能、胆固醇和脂蛋白水平。雄激素补充治疗的效果评价主要依据临床疗效而不是血中的睾酮水平。

西地那非、伐地那非和他达拉非都是目前用于治疗ED的磷酸二酯酶-5抑制剂。由于这类药物的疗效较好、安全且使用方便,因而是目前用于ED的首选治疗方法。

(1)作用机制:性刺激使阴茎神经末梢和血管内皮释放一氧化氮(NO),NO作用于血管和海绵体平滑肌细胞,激活鸟苷酸环化酶,使cGMP水平升高,降低细胞内Ca2+的水平,使平滑肌细胞松弛,阴茎勃起。cGMP的降解是通过PDE-5进行的,因此PDE-5抑制剂通过竞争性抑制PDE-5的活性,从而减少cGMP的降解,改善阴茎的勃起功能。但如果没有性刺激通过神经纤维造成NO的释放,PDE-5抑制剂则不能生效。因此,对于双侧性神经都破坏的盆腔手术,PDE-5抑制剂是无效的。

(2)临床疗效:许多临床研究都证实了西地那非、伐地那非和他达拉非这三种药物治疗ED的有效性和安全性。服用25~100mg西地那非后,勃起功能改善的比例为56\%~84\%,而安慰剂组只有25\%。西地那非的剂量超过100mg,进一步改善勃起功能的作用就很小,而副作用则明显增加。整体上,70\%~80\%的ED患者在服用西地那非后勃起功能显著改善,不同亚组治疗的有效率分别为:高血压79\%,糖尿病56\%,根治性前列腺切除术后43\%,脊髓损伤80\%。伐地那非也具有相似的疗效:服用10mg和20mg的伐地那非后,勃起功能改善的比例分别为73\%和81\%。服用12周后,平均IIEF评分由基线的12.8增加至21,而安慰剂组只从13.6增加至15.0。汇总他达拉非的三期研究的1112例患者的资料显示,服用20mg的他达拉非的患者的IIEF评分为24,而安慰剂组为15。服药后半小时到36小时,70\%的患者成功完成性交。对于大多数的难治性ED患者,包括糖尿病、重度ED及前列腺根治性切除术后,3种PDE-5抑制剂都是有效的。目前尚缺乏3种PDE-5抑制剂直接对比的临床研究,但从已有的研究来看,3种药物的疗效相似。

(3)起效时间:研究显示,3种PDE-5抑制剂起效时间分别为:西地那非为14分钟,伐地那非为10分钟,他达拉非为16分钟。然而,使用药物后20分钟的性交成功率要少于服药1小时后的。因此,如果患者在服药后没有快速起效,应建议在服药后1小时(西地那非、伐地那非)或2小时(他达拉非)后,待药物浓度达到高峰时再进行性交。高脂饮食能够延缓西地那非和伐地那非的吸收,而不影响他达拉非的吸收。

(4)有效时间:与西地那非和伐地那非的半衰期(4.5小时)相比,他达拉非的半衰期更长(17.5小时),因此,他达那非作用的有效时间更长,可以达到36小时,这使得患者安排性生活更加从容和方便。

(5)副作用:决定药物副作用的关键是药物的生化选择性。对于PDE-5抑制剂来说,选择性是通过与其他PDE或其他蛋白相比的PDE-5的IC50表现出来的。已经知道了PDE有11个家族成员(PDE-1到PDE-11),他们在细胞内有复杂的相互作用。阴茎海绵体平滑肌细胞内有高浓度的PDE-5。西地那非和伐地那非对PDE-6有轻微的作用,它们作用于PDE-6的IC50要比PDE-5高4到10倍,因为存在对PDE-6的作用,这就导致了一些患者出现视觉障碍的副作用。他达拉非可以微弱作用于PDE-11,但其效果尚不清楚。PDE-5抑制剂的副作用多与药物作用于其他器官或组织的PDE有关。在随机对照研究中,服用西地那非和伐地那非的患者更多出现面部潮红(10\%)和视觉障碍,而他达拉非的患者后背痛/肌肉痛更常见(1\%~4\%)。通常这些副作用都较轻微,2~4周内都能消退,只有一小部分患者需要停止使用药物。除了视觉障碍外,PDE-5抑制剂作用于血管或胃肠道平滑肌,导致头痛(15\%)、面部充血、鼻炎(伐地那非和西地那非中为5\%~10\%)、轻微的低血压、消化不良等。

ED在动脉粥样硬化性冠心病患者中很常见。在对FDA批准进入市场的三种PDE-5抑制剂进行的对照研究和上市后研究发现,与对照相比较,PDE-5抑制剂并没有显著增加心梗发生率和死亡率。服用PDE-5抑制剂的患者在性生活中发生心绞痛,应及时停止性活动,休息5至10分钟,如果疼痛不缓解,则应去医院急诊治疗,并告知急救人员你服用了PDE-5抑制剂。服用PDE-5抑制剂出现急性心梗则给予常规治疗(除了硝酸酯药物)。如果患者在服用西地那非或伐地那非后出现胸痛,则至少24小时内不要服用硝酸甘油,如果是他达拉非则应至少48小时。如患者服用硝酸酯药物和PDE-5抑制剂后出现低血压,则患者应采用Trendelenburg体位,必要时静脉给予α-肾上腺素激动剂(酚妥拉明)。对于顽固性低血压,ACC/AHA的指南建议给予主动脉球囊反搏治疗。目前还没有一种专门的药物能够拮抗PDE-5抑制剂和硝酸酯药物的相互作用。

PDE-5抑制剂对QTc略有影响,但只有伐地那非不被推荐用于服用IA型抗心律失常药物(如奎尼丁或普鲁卡因胺)、3型抗心律失常药物(如索他洛尔或胺碘酮),以及先天性长QT间期综合征的患者。

近年来发现,服用PDE-5抑制剂的患者有出现非动脉炎性前部缺血性视神经病变(nonarteritic anterior ischemic optic neuropathy,NAION)的报告。流行病学上,NAION是50岁以上男性第二位常见的视神经病变。NAION与心血管疾病、ED具有相似的危险因素,如老年、血脂异常、糖尿病、高血压、吸烟等。迄今为止,FDA收到的与PDE-5抑制剂相关的NAION例数不超过50例,其中西地那非有38例,他达拉非4例,伐地那非1例。由于服用PDE-5抑制剂的患者众多,而NAION例数又太少,因此很难鉴定PDE-5抑制剂与NAION发生之间的直接因果关系。PDE-5抑制剂不建议用于既往有NAION发作的患者。如果患者在服用PDE-5抑制剂后出现视野缩小或尚失,无论有无疼痛,都需要急诊救治,并停用PDE-5抑制剂。

(6)注意事项和药物的相互作用:由于可能导致致命性的低血压,PDE-5抑制剂禁止与硝酸酯药物合用。对于严重的冠心病、左心室输出道梗阻、临床研究没有入选的遗传性视网膜退行性变疾病和阴茎异常勃起风险较高的患者(如镰状红细胞贫血、白血病等)也应禁止使用PDE-5抑制剂。PDE-5抑制剂不推荐或慎用于不稳定心绞痛、心衰、新近发生心梗、未控制的或致命性的心律失常、血压控制不理想(血压低于90/50mmHg或高于170/(100~110)mmHg)的患者。一些药物如酮康唑、伊曲康唑、蛋白酶抑制剂(如利托那伟)等通过阻断CYP3A4通路,减少PDE-5抑制剂的降解,因此服用这些药物的患者应减少PDE-5抑制剂的剂量。而另外一些药物如利福平,能够诱导CYP3A4的表达,增强PDE-5抑制剂的降解,因此患者需要增加PDE-5抑制剂的剂量。年龄超过65岁,肝功能受损,肾功能严重不全的患者,PDE-5抑制剂的血药浓度会升高。因此,这些患者应适当减少药物的剂量。α-受体阻滞剂和PDE-5抑制剂也能相互作用,导致患者血管过度扩张和低血压。

(7)起始剂量:推荐的西地那非起始剂量为50mg,伐地那非和他达拉非为10mg。根据患者的疗效和耐受性,药物范围可以是25~100mg和5~20mg。患者刚服药时疗效欠佳,但并不意味着患者对药物就没有反应。研究发现,西地那非治疗的成功率与服用次数相关,在起初的一到九次服药,成功率依次增加,到十次服药后,成功率才维持稳定。

阿朴吗啡是多巴胺能拮抗药,可激动D1和D2受体,引起阴茎勃起。Heaton等发现阿朴吗啡通过颊部黏膜吸收,能使67\%的精神心理性ED患者的勃起功能获得改善。阿朴吗啡不是阿片类药物,在化学结构上与吗啡也无相关性,它通过作用于大脑的室旁核(哺乳动物中室旁核是性驱动中枢)而达到疗效。性觉醒对于增强阿朴吗啡的效果是必要的。阿朴吗啡起效迅速,使用后引起阴茎勃起的平均时间为12分钟,在用药后的2小时内,患者可以进行性生活。50分钟内达到最大血药浓度。随机双盲的安慰剂对照研究发现,使用2mg和4mg阿朴吗啡的患者产生足以性交的坚硬勃起的比例分别为45\%和55\%,而安慰剂组只有35\%和36\%。患者自我评价成功的占47\%和59.9\%。不良反应包括恶心(16.9\%)、头晕(8.3\%)、流汗(5\%)、瞌睡(5.8\%)、呵欠(7.9\%)和呕吐(3.7\%)。在使用最大推荐剂量时,0.6\%的患者出现晕厥并伴有明显的血管迷走神经反应的前驱症状:恶心、呕吐、流汗、头晕和头重脚轻感。临床试验中未见药物食物存在相互反应(乙醇除外),且与硝酸酯药物也无相互作用。

育亨宾是一种中枢作用的α2肾上腺素能受体拮抗药。随机对照的研究发现,患器质性勃起功能障碍的患者在接受口服育亨宾6mg/天,一天3次,连续10周的治疗后与服用安慰剂的患者无显著差异。2005年美国泌尿协会指南中不推荐其用于ED的治疗。副反应包括胃肠道反应、心悸、头痛、激动、焦虑和血压升高(在心血管疾病患者中应慎用)。曲唑酮也不推荐用于ED的治疗。荟萃分析显示曲唑酮对ED的疗效与安慰剂组无显著差异。副反应包括瞌睡、恶心、呕吐、血压变化(低血压和高血压)、尿潴留和阴茎异常勃起(特别是治疗抑郁症的剂量水平)。

前列地尔是PGE1 的合成形式,是FDA核准用于海绵体内注射和经尿道治疗ED的唯一药剂。当注入尿道后,药物由尿道吸收转运至阴茎海绵体,刺激腺苷酸环化酶,使细胞内的钙离子水平降低,舒张血管及海绵体平滑肌细胞,导致阴茎勃起。经尿道给药勃起治疗系统(medicated urethral system for erection,MUSE)是通过专门的给药器将一颗非常小的半固体药丸(3mm×1mm)注入尿道远端(3cm)。临床研究发现,用药后66\%的患者勃起功能获得改善。但上市后的临床观察,有效率只有50\%。阴茎和(或)阴囊不适是经尿道治疗常见的副反应,并与剂量有明确的相关性,其发生率为33\%,低血压和晕厥的发生率为1\%~5.8\%。故要求初次应在医院内给药,患者的性伴侣在射精后出现阴道不适的比例为10\%。

海绵体内注射血管活性药物是ED最有效的非手术疗法。它起效较快,局部给药,减少了药物的全身副作用,以及与其他药物的相互作用,对各种类型的ED具有较好的疗效,适用于对PDE-5抑制剂不敏感或者不能耐受口服药物的副作用的患者。在PDE-5抑制剂不敏感的ED患者中,超过85\%的患者在使用海绵体内注射治疗后获得较好的疗效。常见的注射药物有罂粟碱、酚妥拉明、前列地尔及复合制剂。

罂粟碱是由罂粟花分离出来的一种生物碱。它能非特异性地抑制PDE的活性,增加cAMP水平,阻滞电压依赖钙通道,从而减少钙内流,使得海绵体平滑肌松弛及阴茎血管舒张。罂粟碱在肝脏代谢,血浆半衰期为1~2小时。单次治疗剂量为15~60mg。主要的副作用是阴茎异常勃起(发生率多达6\%)、海绵体纤维化(6\%~30\%,与注射技巧差、按压注射部位过短、注射剂量大于1ml和pH 3~4的低酸度有关)以及偶见肝酶升高。全身副反应包括头晕、脸色苍白和出冷汗,这些可能由血管迷走反射引起。

酚妥拉明是一种对α1和α2-肾上腺素能受体具有同样亲和力的竞争性α-肾上腺受体阻滞剂,血浆半衰期短(30分钟)。当单独在海绵体内注射时,它能增加海绵体血流而不显著地引起海绵体内压升高。最常见的全身副反应有低血压、反射性心动过速、鼻腔充血和胃肠道不适。

前列地尔通过升高细胞内cAMP水平,使平滑肌松弛,血管扩张,并抑制血小板聚集。前列地尔由前列腺素-15-羟基脱氢酶代谢,该酶被证实在人类阴茎海绵体内也具有活性。海绵体内注射后,96\%的前列地尔在60分钟内于阴茎局部代谢,而外周血中药物浓度没有改变。Linet等发现前列地尔剂量在10~20μg时能使70\%~80\%的勃起功能障碍患者完全勃起。全身副作用罕见,最多见的是注射部位疼痛或勃起时疼痛(11\%~15\%),血肿/瘀斑,海绵体纤维化(1\%~3\%),注射时烧灼感,阴茎异常勃起(1\%~3\%)。

常用的药物联合使用方案有二联(罂粟碱和酚妥拉明)和三联(罂粟碱、酚妥拉明和前列地尔)。各种药物以不同浓度混合使用,有效率超过75\%,而阴茎异常勃起和海绵体纤维化的发生率较低。由于每种药物的使用剂量较低,因此,副作用的发生率降低,不同药物对各种途径的联合治疗使得药物剂量虽然较低,但疗效较好。

(1)二联治疗:最常用的二联治疗包括罂粟碱(30mg/mL)和酚妥拉明(1mg/mL)。Armstrong等观察了共160例ED患者的13 030次阴茎海绵体内注射罂粟碱和酚妥拉明的治疗,统计结果显示,115例患者(72\%)治疗后能成功进行性交,不同病因的ED治疗的成功率分别为:血管性为48\%、精神性为93\%、神经源性为92\%、糖尿病为68\%、特发性为63\%、创伤性为60\%、乙醇相关性为80\%、药物相关性为75\%。平均经过14.1个月的随访,55例患者(48\%)仍坚持使用海绵体内注射治疗。副作用方面,16例患者共出现22次阴茎异常勃起,1例患者出现海绵体纤维化。

(2)三联治疗:常用的三联治疗方案包括罂粟碱(30mg/mL)、酚妥拉明(1mg/mL)和前列地尔(10μg/mL)。与二联治疗相比,其疗效更强,有效率约为90\%。McMahon等对228例ED患者进行了随机交叉研究,并比较三联治疗、二联治疗和单一前列地尔治疗的疗效差异,结果发现与二联治疗相比,三联治疗对严重动脉性ED和轻度静脉闭合功能不全性ED患者的疗效更好,而阴茎异常勃起的发生率低于二联治疗,与前列地尔单一治疗的发生率相似。与前列地尔单一治疗相比,三联治疗具有相似甚至更佳的疗效(成功率为89\%),而痛性勃起发生率更低。目前,三联治疗常用于前列地尔或二联治疗失败,或使用前列地尔后阴茎疼痛显著的患者。

在一些研究中,患者在医院内接受注射治疗的比例为49\%~84\%。然而,在长期治疗中,停止采用注射治疗的比例高达20\%~60\%。常见的停止治疗的原因包括性兴趣丧失、花费、配偶的去世或对治疗后的勃起反应不满意等。

阴茎异常勃起和海绵体纤维化是海绵体内注射治疗两个较为严重的副反应。Linet等[75]荟萃分析了48项研究中入选的总共8090名注射前列地尔的患者,采用前列地尔治疗的患者阴茎异常勃起发生率占1.3\%,显著低于罂粟碱单一治疗(10\%)和二联治疗(7\%)的患者。海绵体纤维化可表现为单个硬结、弥漫性疤痕、斑块或阴茎弯曲。使用前列地尔后出现海绵体纤维化的比例(1\%)也显著低于罂粟碱单一治疗(12\%)或二联治疗(9\%)。而采用三联治疗时这两个副反应的发生率与前列地尔单一治疗相似,甚至更低。

患者首次注射必须由医疗人员执行,在家中注射前应接受适当的教育与训练。前列地尔的推荐起始剂量为2.5μg。如果疗效不好,可以增加剂量,每次2.5μg,直到获得较好的疗效或达到60μg的最大剂量。对于联合治疗,应从小剂量开始(如0.1mL的混合制剂),依据阴茎的反应,逐渐增加剂量。治疗目标是达到足以性交的勃起,持续时间应少于1小时,以避免阴茎异常勃起的发生。注射后,应压迫局部针眼至少5分钟,以避免血肿的形成和减少海绵体纤维化的发生。

最好的处理是预防。注射治疗过程中出现的阴茎勃起时间过长多见于患者药物剂量增加太快,忘记药物注射后短时间内重复给药导致药物剂量过大,以及神经源性和(或)年轻患者。逐步渐进地增加剂量可以避免绝大多数阴茎异常勃起的发生。患者在采用海绵体内注射疗法前,应明确被告知阴茎异常勃起是需要急诊副作用,勃起超过4小时应于医院急诊治疗,治疗方法可以采用新福林(去氧肾上腺素)25~500μg海绵体内注射,每隔3~5分钟重复,直至阴茎疲软,治疗室时患者如出现心血管疾病,应密切监测生命体征。

海绵体内注射治疗的禁忌证包括镰状红细胞贫血、精神分裂症或其他严重精神疾病、严重的静脉闭合功能不全。对于正在服用抗凝药物或阿司匹林的患者,应在注射后压迫注射部位7~10分钟。对手部活动不变的患者,可由其性伴侣在指导下完成注射。

真空缩窄装置的原理是使用真空负压装置产生的负压提高海绵体的血流量,从而使阴茎产生勃起,再采用缩窄环环扎在阴茎根部,阻断阴茎的静脉回流以达到延长阴茎勃起时间的目的。通常的真空缩窄装置由套在阴茎的塑料圆筒、真空产生泵组成,两者可以直接相连或者采用导管相连,当阴茎因负压勃起后,于阴茎根部的缩窄环环扎维持勃起。缩窄环可能引起不适或疼痛,为了避免损伤,压缩环放置不应超过30分钟。需要注意的是,负压装置产生的负压应有一定的限制,避免过高的负压损伤阴茎。由真空缩窄装置产生的勃起与生理性勃起或海绵体内注射产生的勃起不同,其阴茎海绵体内血氧水平较低,且压缩环近端的阴茎部分不会坚硬,可导致勃起的阴茎出现绕轴旋转。该装置能产生与正常相近并且硬度足以性交的勃起,同时也能使阴茎头充血,故对阴茎头机能不足患者有益。大部分使用该装置的患者诉对阴茎硬度、长度和周径满意,性伴侣同样感到满意。对患有严重近端静脉漏或动脉关闭不全、阴茎异常勃起继发纤维化或假体感染的患者,该装置的疗效欠佳。该治疗的并发症包括阴茎疼痛和麻木、射精困难、瘀点、瘀斑。服用阿司匹林和华法林的患者使用真空缩窄装置时应格外慎重。

阴茎假体植入手术是ED手术治疗的主要方式。目前阴茎假体主要分为三类:半硬性可屈性假体,机械性假体,可膨胀性假体。半硬性可屈性阴茎假体是一种带有轴心的半硬式装置,阴茎在日常生活状态时可以向下弯曲,需要时则向上直立进行性生活,此装置优点是机器故障率非常低和容易使用,缺点是阴茎持续处于僵硬状态;机械性阴茎假体由近心端、远心端和中间体构成的一对圆柱体构成,不同部分之间的轴心安装有弹簧,能够更好地调节阴茎向上及向下的角度,可屈性和隐蔽性比半硬性可屈性假体更好;可膨胀性假体由圆柱体、泵和储水囊构成,目前以双件套和三件套为主,双件套的水囊和泵一起植入阴囊,圆柱体植入海绵体内,两者用细管相连,而三件套的水囊单独置入膀胱前间隙,泵植入阴囊,三件套假体可达到类似正常勃起的阴茎勃起周径和硬度,是目前最常用的假体类型。

患者在进行假体植入术前应充分了解可供使用的假体类型及优缺点、疗效、可能的并发症(包括感染,机械故障,圆柱体或管路漏液,假体穿出阴茎,阴茎短缩和假体自发膨胀等)。对于可膨胀性假体而言,在术后5年内的失败率为5\%~15\%,术后10~15年,绝大多数假体失效,需要重新植入。三件套假体植入术后高达85\%~90\%的患者对疗效满意。

阴部外动脉的单一狭窄或闭塞可以考虑进行手术治疗。动脉重建手术通常用于年龄不超过55岁,局部动脉病变导致的ED,患者无高脂血症、糖尿病、高血压导致的全身动脉病变,无海绵体缺血引起的海绵体肌肌源性损伤。阴茎血运重建的常用方法是腹壁下动脉与阴茎背动脉的旁路手术。

阴茎静脉手术只适用于先天性或创伤性静脉漏的年轻患者。由阴茎脚静脉或背浅静脉功能不全导致的先天性静脉漏,采用阴茎静脉手术可能治愈。创伤性静脉漏多由海绵体白膜局部损伤导致,或者在阴茎海绵体和尿道海绵体之间形成“瘘”,因此,修复白膜的局部损伤和将瘘管闭合可显著改善勃起功能。在老年患者中,全身疾病导致的静脉漏多由海绵体平滑肌萎缩和海绵体纤维化导致,静脉手术只能暂时改善勃起功能,因而不建议这些患者进行此类手术。

(贺占举)

在2009年7月召开的第三次性医学国际咨询委员会中,性医学临床评估测量委员会回顾了在男性勃起功能障碍心理测试领域的当前实践及新发展。基于对Pubmed数据库的调查、委员会的回顾和证据的等级分类,为男性ED生理检测领域中现行检测手段和新发展提供了一个关键性评估。

选择性阴部血管造影法、双功能超声波扫描和海绵体内压力测定是评估海绵体动脉流入通道和静脉关闭机制的功能和解剖的经典“金标准”(表7-3);

为使这些测试能够准确,必须执行遵循2009年7月召集综述的有关海绵体平滑肌松弛的当前实践的归纳。也就是说,海绵体流入通道和静脉闭塞机制的状态是最大的挑战。为了达到这个目的,通常不仅采用血管活性药物海绵体内注射,同时还采用诸如视觉色情片、阴茎振荡刺激或口服具有助勃作用的药物。使用选择性动脉造影可以清楚地看到海绵体流入通道,包括阴部动脉、海绵体动脉和阴茎背动脉。它最初广泛应用于怀疑“动脉性ED”的男性,而如今仅局限于疑似孤立的阴部动脉狭窄的ED男性,或者计划做海绵体动脉瘘管栓塞的高血流量阴茎异常勃起的男性。两组都是遭遇骨盆或会阴损伤的年轻男性。

对比-增强磁共振在功能性阴茎血管成像模块中是一个新生儿。它赋予了解剖细节的精细可视化和骨盆及阴茎微循环的评估,可以用于动脉功能不全的分级、静脉闭合功能的评估和可视化、阴茎折断的确定、阴茎假体的评估和阴茎硬结症斑块的识别。

表7-3 最高水平证据的诊断测试

阴茎海绵体测压是提供海绵体静脉闭合功能定量和图示的主要手段,其基本原理是在盐水注入海绵体以维持其内压力的流率与平均动脉系统血压流率之间建立相关关系。

对于阴茎海绵体造影术,注入具有放射性对照物质的溶液以确定阴茎海绵体泄漏的具体位置。为准确起见,阴茎测压必须在海绵体平滑肌完全松弛的状态下进行。在临床测试的情况下,难以达到这种状态。大多数案例中,有必要重复向海绵体窦内注射血管活性药物以达到阴茎平滑肌的完全松弛,这样才能在灌流率与海绵体内压之间建立线性关系并加以记录。

近来,假定海绵体静电位可以为海绵体平滑肌完全松弛提供一种替代的测量。因为它的有创性,海绵体测压保留于原发性年轻ED患者和罕见患者,即怀疑有特殊位点静脉漏、伴有硬度低下的阴茎硬结症、阴茎折断史、会阴或盆骨损伤史的患者或考虑进行手术的患者。

阴茎双功能超声波比以往的技术具有更明确的优势:它是创伤性最小的选择,且可在诊室环境下进行。双功能超声探头可以选择性地呈现各根阴茎动脉的图像,同时测量血流速率。用于推断阴茎流入血管通道完整性的参数是阴茎动脉的直径、血流方向、收缩期最大血流速度和加速时间。

为评估阴茎静脉闭合机制,可在阴茎平滑肌完全松弛的状态下测量扩张末期的血流速率和阻力指数。在评估静脉闭合功能方面,双功能超声图像是不同于阴茎海绵体测压的,其缺陷在于不能记录平滑肌完全松弛情况下的功能状态。

还没有达到广泛临床应用的其他的血管测量方法有:近红外线光度法和放射性同位素阴茎造影。

衰老所致男性ED常常是盆腔灌注受损症状,以及代谢综合征和病理生理学导致的性腺机能低减的概念,将血管测试的视野转移到心血管药物可靠性检测所证实的常规检测技术中,如颈动脉内膜中层厚度测量(cIMT),肱动脉血流介导的血管扩张功能检测(bFMD)、组织氧合、激光多普勒血流测量。

已经发现超声波测量cIMT为全身动脉粥样硬化程度和进展的可靠参数,以及作为判断药物介入效果的替代端点,例如他汀类药物或血管紧张素转换酶抑制剂。然而,ED男性的股动脉内膜中层厚度要大于颈动脉内膜中层厚度,这就导致对cIMT作为盆动脉硬化参数的有效性产生了怀疑。

建立了在高频率超声波肱动脉成像技术辅助下的由肱动脉血流介导的血管扩张功能检测(bFMD),以检测血管内皮功能。这项技术激活了氧化氮的释放,导致血管扩张,其结果能够量化为内皮功能的一个指数。这种无创性的技术允许超时的重复测定,以研究可能影响血管内皮健康的各种干预措施的有效性。

在组织血氧定量计的帮助下测定了阴茎在疲软和勃起状态下的氧气饱和度,提供了实时、无创和无痛的局部氧气饱和度测量,它与血气数据高度相关。男性ED患者阴茎海绵体氧气饱和度显著低下。

这种方法学也许有助于进一步阐明海绵体组织缺氧与ED的发生、发展,以及对它的治疗和预防的相关关系。这种血氧定量方法是基于测量组织中光子的散射和吸收而展开的。

它依靠的是向组织发送光学信号(持续的波动、脉动或强度调制调换波)并测量相应的扩散反射或组织表面的透光率。组织的光吸收、光单位、发色团的相互作用导致吸收。血红蛋白是组织中的主要发色团之一,为反射光的评估提供了有关这些发色团生化状态的临床信息。

LDF是一种探索皮肤微循环规律的方法。LDF的理论基于这一概念:交感神经放电造成连续的小动脉收缩和扩张,导致分布于皮肤的血流节律变化。这些节律变化表现在几个频率带之一,包括涉及心脏、呼吸、内皮和肌源性周期的频率带。把这些知识(信息)转换而应用于阴茎血流动力学研究,还有待进一步实验。可以推测的是:测量阴茎龟头微循环的波动也许有助于评估阴茎自主神经系统的功能。

血管测试的最新发展:以测量循环内皮前体细胞的数量来作为海绵体血管健康的参数。

ED鉴别诊断的问题之一是大脑和阴茎海绵体轴之间的轴是否完整。为此目的,可采用精神生理学的评估。有必要在睡眠或外来性刺激的情况下,观察阴茎海绵体的反应(阴茎勃起)。

1970年推出测量与快动眼睡眠相关联的阴茎反复勃起,它的优势在于避免了心理调节作用的影响,因此有助于区分ED案例的心理性和器质性因素。作为经验惯例,如果证明存在完全的勃起,表明神经与血管轴的功能是完整的,那么导致ED的原因是心理性的。

然而,诸如抑郁、消极梦境内容、睡眠紊乱和吸烟等因素都有显示导致假阴性的结果,而假阳性结果也许来自盆腔窃血综合征的患者,和某些外周神经病理的个案。

可采用阴茎硬度计(Timm医学技术有限公司)测量阴茎硬度的质量。这个仪器有两个环,一个环放置于阴茎的根部,另一个放置于冠状沟后缘,它们每15或30秒钟收紧一次,测量放射状的压缩力。长期来看,以测量放射状的压缩力来评估阴茎(轴向)硬度的有效性一直存在争议。

一项近期的生物力学研究支持轴向硬度测量的有效性,从而结束了这个争议。至今,与睡眠相关的勃起测量主要用于研究如量化口服药物引起的勃起作用、药物介入以及环境因素引起的潜在的抗勃起作用。其临床价值尚未充分确立或被接受。

视觉色情刺激可以作为引发勃起反应的一个手段。一个充分的视觉色情刺激勃起反应基本上排除了躯体性原因的可能性。其缺点在于在实验室进行的VES特别容易受到心理因素的影响,并且与年龄呈负相关,限制了在年龄较大男性中的应用。至今,它最重要的应用是在临床药理学研究药物的(抗勃起)或引起勃起的作用和法医学领域,而没有常规用于ED的临床评估。

在20世纪80年代和90年代初期,一系列测量神经传导速度、诱发电位,以及自主神经功能的神经生理学测试方法应用于男性ED的诊断测试。而如今它们的应用仅限于特殊的研究方案,因为它们之中没有一个能单独地或组合地表现出达到测量与阴茎勃起相关神经网络的所有部位的(自主的和躯体的、传入和传出的)功能状况的目标。

自主功能的测量看似悬而未决,有几个原因:首先,它们要测量涉及一系列事件的链,包括感受器、微纤维和靶器官。混淆因素如药物、咖啡因、温度、低血管容积和高血管容积、精神情绪,以及感受器或靶器官功能障碍都可能对每个个别的组分产生不同的影响。其次,与其说它们是在评估阴茎海绵体,还不如说它们是在评估器官的自主神经分布(诸如,心率变异的测试),而它们并不是受到神经病变的同等影响。最后,在实验室之间,自主(神经)测试并没有制定很好的标准,缺乏重复性和可比性。如此说来,若要使自主神经测试具有临床意义,必须明确地针对阴茎海绵体量身定做。

在这一方面,阴茎海绵体肌电描记法(ccEMC)表现为一项有前景的技术。在ccEMC,在阴茎表面的电极记录阴茎海绵体内部的电活动。然而,仍然须把这个测试看做是试验性的,根本问题在于仍然无法合理解释记录到的结果。

近来,使用功能MRI和正电子发射断层技术(PET)扫描大脑,实验性评价涉及性行为和性反应周期不同阶段的大脑中枢。这些扫描可能提供与ED相关的大脑功能定位的探索性信息。尽管如此,尚未确立其临床效用。

结论:自从引进不用考虑病因的口服药治疗以来,已经把“经典”的专门检测的应用减少到最低限度,只局限于极少数男性;而且已经把生理检测的范围从测试男性的性终末器官转换为利用心血管药物可靠性检测所证实的检测技术。

附:

颈动脉内膜中层厚度的测量:应用二维B型血管超声(HP 5500型超声,配有10MHz线性超声探头)同时获取颈总动脉、颈动脉分叉处和颈内动脉起始段至少1.5cm的优质图像。管腔内膜和颈动脉中层膜交界面之间的距离即为颈动脉内膜中层厚度(IMT)。笔者分别在颈总动脉远端、球部和颈内动脉起始部等3个部位,各取1cm长,对颈动脉进行测量,取其平均厚度。左右两侧的颈动脉内膜中层厚度的均值为最终值。将颈动脉内膜中层厚度≥0.9cm定义为增厚。

肱动脉内皮功能的检测:在检测之前,患者取卧位,用水银血压测量仪测量该研究对象的左测肱动脉血压。然后应用二维B型血管超声,获取右侧肱动脉的纵轴图像和血液信号,测量管腔直径的基础值。而后将袖带置于肘弯以下2~3cm处,加压,加压压力为收缩压+50mmHg,截断血流5分钟。并于放气后60~90s内快速对右侧肱动脉管腔直径进行测量,以评价血流介导的肱动脉扩张(flow mediated dilation,FMD)。当肱动脉扩张率≥6\%,定义为肱动脉内皮功能正常,反之则为受损。患者休息10分钟,以重新建立基础值。然后给予0.5mg硝酸甘油舌下含服,4分钟后再次测量管腔直径,以评估硝酸甘油介导的肱动脉扩张(nitroglycerin mediated dilation,NMD)。

(周旭编 译)


\chapter{第八章 射精障碍}

早泄(Premature Ejaculation,PE)是男性性功能障碍之中最复杂的一个疾病单元,早泄与其他性功能障碍相比具有几个非常显著的特点:首先,是早泄的发病率最高,无论采用什么样的早泄诊断标准,其发病率都是男性性功能障碍中最高的疾病,依据目前国际上比较认可的诊断标准,早泄的发病率已经达到29\%(美国劳曼等人的研究),由于早泄影响夫妻性生活,所以同样数量的女性也会因为早泄而降低性生活质量,这无疑使早泄成为最重要的社会和医学问题;其二,早泄是目前所有疾病中唯一一个要以患者性伴侣的性反应作为诊断指标的一种疾病,所以早泄已经超出了个人的疾病范畴,无论是诊断还是治疗都更加复杂;其三,除了少部分继发性早泄外,大多数早泄患者都是原发性的而且持续终生。所以,早泄是比勃起功能障碍对夫妻性生活影响更大的疾病,而且至今没有最佳的治疗方法;其四,在勃起功能障碍的原理已经比较明确,且万艾投入临床使用已经取得显著疗效之后,早泄成为男科学专家们努力攻克的最后堡垒。早泄作为一个特殊的疾病单元,其诊断标准仍然在不断地发展和变化之中,所以本文主要将重点放在诊断标准的演变、临床治疗及存在的需要的进一步研究方面。

国际上有关早泄的研究和文献浩如烟海,但真正有价值的研究基本是近代学者们的贡献。目前人们普遍接受的早泄研究和治疗的历史分为四个主要阶段:

第一阶段(1887~1917,概念形成):Gross首次在医学文献上提出快速射精的概念,后来克拉夫特·埃宾又接着在1901年发表的文章中提出“早泄”。

第二阶段(1917~1950,精神动力学病因):阿勃拉汉姆提出“早泄”(Ejaculatio praecox),将注意力从早熟(premature)转向快速(rapid)。他提出了早泄病因的精神动力学理论,认为早泄是一种在婴儿期没有解决的、过分自恋的倾向在成人期的表现,是对自己阴茎的过分看重。这显然受到佛洛伊德精神分析理论的影响。在1943年,Schapiro将早泄定义为身心紊乱,他认为其发生主要是过度焦虑的心理素质和虚弱的射精系统共同的作用。

第三阶段(1950~1990,行为医学):美国著名的妇产科专家马斯特斯和心理学家约翰逊在性学研究中开创了近代性医学,他们开发了“性反应周期的模式”并且创造了挤压技巧来治疗早泄,他们认为早泄是习得性的行为,明显受到当时流行的行为医学的影响。他们认为青少年早期在精神紧张的情况下快速手淫或快速性交导致了成人期的早泄。他们开发的性感集中训练法治疗早泄取得了一定程度的成功。这种方法就是行为医学的行为治疗,属于去条件化反射训练。

第四阶段(1990至今,精神药理学):随着精神药理学的发展,氯丙咪嗪和五羟色胺再摄取抑制剂(SSRIs)能够治疗早泄的报道和临床研究导致5-HT作为中枢神经递质在早泄的病因中得到了关注,由这些药物治疗早泄取得的实际疗效而推断出了5-HT早泄病因模式。学者们认为五羟色胺神经递质的功能失调是导致早泄的病因。这一理论不但指导进一步的病因研究,也用于指导临床治疗和药物开发。

尽管早泄作为疾病存在的历史与人类的历史一样漫长,但有关早泄的研究却不尽如人意。有关早泄的诊断至今仍然没有定论,药物的研究与勃起功能障碍相比更为逊色,导致这种状况的原因有很多,主要有以下几项:

早泄不同于其他性功能障碍,早泄的特点在于发作持续的时间太短,只有几分钟。比如勃起功能障碍和性欲低下至少是持续存在一段时间,而早泄无论轻重都是暂时性存在,不在性行为过程中就不存在早泄的问题。所以研究早泄必须在性交过程中而不是在其他时间,由于时间短,无论是观察还是干预都十分困难。

早泄只是临床表现,而导致早泄的关键是射精机制,但目前有关射精的机制还不是十分清楚,所以多数针对早泄的治疗还是经验性的,中枢和外周神经系统都有参与射精活动,所以主要是内在的射精机制,这样,就为进一步研究增添了难度。

由于学者们对于早泄的认识还没有达成共识,如何看待早泄的疾病归属问题就更为复杂,早泄到底是一种疾病,还是人们为追求快感而人为造成的这样一种所谓的“疾病”?目前还不能得出肯定的结论。由于缺乏客观的生理学基础,所以有关早泄的病因学理论就更加难以令人信服。

虽然早泄都是表现在射精过快这个主要问题上,但具体的表现形式还是有差异的。从阴茎插入阴道到射精称为射精潜伏期。在潜伏期内男性有无抽动(阴茎完全撤出阴道重新插入,尽可能深入,甚至龟头可以触及宫颈,古代医学书籍中称为深;或者半撤出再插入,以冠状沟与阴道的外三分之一狭窄部位摩擦为主,古代医学书籍中称为浅)和抽动的幅度大小以及频率快慢都与是否发生早泄有关。与男性的插入相反,女性也可以主动套入阴茎并且进行夹缩,这种情况多数是在女上位或臀位以及侧卧位性交中使用。女性套入的深度、频率和夹缩的力度都与早泄有密切关系。

早泄历来没有统一的诊断标准,这与人们对早泄的认识过程有关。中国古代书籍中历来没有对早泄采用严格的时间标准,在接触女性后没有插入就发生射精称为严重的早泄,其他都称为早泄,看来是以性生活的满意度为标准,但主要是以男性为主。西方医学最严格的早泄诊断标准是从插入到射精为十五秒,多数是两分钟。其实这反映了早期的早泄诊断标准与性生活目的一致性,即性生活的目的是为了生育,如果从生育的目的出发,射精潜伏期有十五秒已经足够了,当然两分钟更加没有问题,这就是最早的以生育为中心的早泄诊断标准;封建社会到资本主义社会都是男权社会,性生活的满意度当然就要考虑男性的需求,所以早泄的诊断标准又从以生育为中心向以男性享乐为中心转变。虽然时间不是最主要的因素,但没有涉及女性的快感体验。到了性医学发展的人本主义时代,妇女的地位明显提高,马斯特斯和约翰逊夫妇首次提出男性和女性达到性高潮的比例低于50\%就是早泄这样的诊断标准,这是从以男性为中心向以男性和女性共同享乐为中心的又一次飞跃,具有划时代的意义。

影响男性射精和诊断早泄的因素过多。环境因素与早泄有密切关系,环境缺乏安全感容易加剧早泄;性兴奋程度过高容易加剧早泄,最常见的现象是遇到新的性伴侣,或初次发生性交行为;性生活频率与早泄关系密切,性生活频率越高早泄的发生率越低,反之亦然,因此很多人利用不应期的原理,通过多次性交而达到改善早泄的目的;酒、阿片类药物以及某些精神药物都具有延迟射精的作用,在停止使用这些物质后会出现反跳性的早泄。


\section{第二节 早泄的病因及发病率}

射精是和许多神经生理过程相关的,整个射精过程可以用四个英文单词来表达:Excitation(兴奋);Erection(勃起);Emission(泌精);Ejaculation(射精),四个英文单词的第一个字母都是E,简写为4E。性兴奋是性行为的起始阶段,也是性反应周期的第一个期,性兴奋期自然出现明显的阴茎勃起。勃起是指引起阴茎膨胀和增加硬度的神经和血流动力学活动过程,主要由来自骶部(盆腔内脏神经或勃起神经)的副交感神经控制。泌精(emission)是指从前列腺、精囊、输精管收集并且传运到尿道而准备射精的过程,这些综合的分泌物称为精液。伴随着膀胱颈部和远端的尿道括约肌的收缩和关闭,进入后尿道和精阜的精液形成一个压力池。射精是迫使精液通过尿道排出体外的过程。性高潮是与射精有关的主观体验,被认为是大脑皮层的感受。虽然泌精、射精和性高潮是一个整体过程,但由于不同的神经生理学机制的临床证据表明,勃起并不一定需要射精,而射精也不一定需要有性高潮,但射精的神经生理学机制还没有完全明了。另外,研究表明多巴胺系统促进射精而五羟色胺系统抑制射精。目前广泛接受的观点是:男性射精器官的神经分布包括体神经,主要是阴部神经和它的分支(一种感觉和运动混合的神经),以及自主神经系统的两个分支。副交感自主神经的分支主要负责勃起功能,而交感神经和体神经的分支主要负责溢精和射精。在性活动时,感觉刺激从生殖器通过阴部神经的感觉分支(一条体神经)传到骶丛,再从骶丛传到脊索。当冲动传导到脊索之后,这些神经冲动刺激脊索水平的神经反射,并进一步向脑部的高级中枢传导。虽然大脑射精中枢的准确部位还没有确定,但下丘脑被认为起到重要作用。在适当地整合和翻译信息之后,大脑不是扩大就是抑制这种反应,大脑的这些中枢就返回一个传出信息冲动到下丘脑、脑干和脊索的中间和外侧束到交感链,在胸10和腰3传出神经节段部位发生突触活动。神经冲动从这里通过上腹下丛和下腹神经到突触并对附近器官的肾上腺素神经元传递冲动,这些器官包括精囊、输精管、前列腺和男性射精系统的其他部分。从这个第二级传出神经元来的神经冲动通过神经通道继续刺激输精管、精囊、前列腺、膀胱颈并引起泌精。到膀胱颈部的交感神经分布引起射精时的膀胱颈部的高度闭锁而防止逆行射精。同时,来自高级中枢或通过脊索反射的神经冲动通过骶尾膀胱部位并且通过体神经的运动支来刺激会阴部的横纹肌产生射精时典型的节律性收缩。这些肌肉包括球海绵体肌、坐骨海绵体肌、耻骨尾骨肌。

虽然关于早泄的理论有多种,但是几乎没有一种有关于病因的对照组的研究。一些推断来自于生活的、心理的和医学的观察。例如,一些人根据金西博士的调查,即75\%的男性有PE,因此认为PE是正常的,从生理学上可以说是根源于人类的进化。也有人根据简单的联系提出病因,例如根据性交的频率越低,越容易发生PE,因此认为早泄是性交频率过低所致。然而斯特拉思勃格等认为性生活的逃避是早泄造成的结果,而不是早泄的病因。尽管目前的资料还不完全一致,但越来越多的资料表明多种原发性和终生性早泄的病例是有生理学、神经病学和素质反射方面的生物学病因的。正如前面提到的,多种研究表明:早泄的男性和无早泄的男性球海绵体肌的反射是不同的,早泄的男性球海绵体肌的反射更快,这也使一些人推论患有早泄的人比无早泄的人在生物学上具有更加明显的阴茎敏感性。

神经药理学根据药理学模式提出了可能的早泄机制和治疗方法。凯特尼和赛路波发现三氟拉嗪和某些麻醉剂,如阿片戒断时可能出现早泄,但机制目前不清楚。其他如盐酸假麻黄硷、硫酸麻黄硷和盐酸去甲麻黄硷都被说成是能够改善由于增加副交感神经活动而导致的射精迟钝或抑制的药物。能够延迟射精的药物可以分为三类:多巴胺拮抗剂、抗抑郁剂和β- 氨基丁酸(JABA)增效剂。

这是一组能够阻断中枢多巴胺受体的药物,通常称为神经松弛剂或抗精神病药。通过阻断中枢多巴胺受体可以防止早泄。多巴胺拮抗剂表明了对人类抑或动物的特殊的抑制射精作用,这包括哌咪清、胃复安、舒必利、氟哌啶醇和强力催产素拮抗剂vasotocin。

另一类被认为能够延迟射精的药物是抗抑郁剂,这类药物能够提高五羟色胺的水平。新的非典型性药物包括氟西汀、色曲啉、帕罗西汀、氟伏草胺和顽发克星。这些药物有很弱的抗组织胺作用,通过阻断五羟色胺的回收而提高五羟色胺的水平。五羟色胺水平的升高被认为是抑制射精的机制之一。

许多苯二氮 类药物对治疗广泛性焦虑和惊恐发作有效,同时也发现对某些男性有抑制射精的作用,推断其机制为提高β-氨基丁酸(GABA)。这些药物包括利眠宁、氯羟安定和阿普唑仑。然而抑制射精的作用没有五羟色胺回收阻断剂那么明显和广泛。对照研究表明,服用抗焦虑剂引起射精抑制的病例小于10\%。一项有安慰剂对照的女性服用安定的研究表明,其存在明显的与剂量相关的性唤起和性高潮抑制。

许多研究表明,苯氧苄胺(降压药)可以抑制射精。在开始剂量从每天10mg到30mg的增加过程中,苯氧苄胺是强有力的α-肾上腺素能阻断剂,目前认为抑制射精的机理是干扰了射精反射的交感神经活动。射精也许不只受到交感神经控制,还受到交感神经和副交感神经的调节,交感神经负责精液进入后尿道的泌精过程,而副交感神经主管最后的射精。西龙·帕泽德·厚牟奈的发现支持上述双重调节的理论,即苯氧苄胺废止了精液排出,使其进入后尿道,并且延迟射精,而患者能够体验射精的感觉(性高潮)。这是一种无精液射出的干性性高潮。一些报告也提到典型的麻醉剂可以通过减少感觉而抑制射精,如乙基氨基苯甲酸盐和地布卡因(3\%溶液)。其他受益的报告来自颠茄止血素、一溴樟脑、左旋多巴、盐酸罂粟碱。

尽管任何推论都是暂时的,但药理学研究表明,一些机制在射精过程中还是起到了重要作用:1)起到中枢神经系统多巴胺受体阻断作用的神经松弛剂或抗精神病药能够抑制射精;2)通过阻断五羟色胺的回收提高五羟色胺的水平能够抑制射精。然而,某些轶事性的报告表明偶然的情况之下五羟色胺回收阻断剂能够引起自发的性高潮。3)具有抗胆碱能作用和α-肾上腺素能受体拮抗作用的三环类抗抑郁剂能够通过阻断五羟色胺和去甲肾上腺素的回收而提高其水平,起到抑制射精使用。4)使用提高β-氨基丁酸水平的抗焦虑剂,如苯二氮 类能够轻度地抑制射精,这种作用与药物剂量明显相关。5)通过α-肾上腺素能受体阻断剂干扰射精反射弧的副交感神经而抑制射精,如可以用苯苄胺α阻断来解释这一发现。射精过程是多种因素决定的,既受交感神经又受到副交感神经的调节。然而,药理学的研究仅提供一些线索:1)脑以及周围神经系统是起作用的;2)许多神经病学机制在射精过程中是有意义的,如果是这样,多种早泄在生理学上应该是存在的,每一种都受到不同的生理化学机制的调节。

医学文献强调早泄的生理学、神经病学、药理学和泌尿外科学的病因,而心理学文献则提出早泄病史、各种因素相互作用、行为和心理社会学的解释。然而,许多心理学解释依赖于病例印象,而不是科学的依据。精神分析将早泄归因于广泛性焦虑、阉割焦虑、被动侵犯的人格障碍、自恋、对女性的无意识情感。然而卡普兰明确地驳斥了这种观点,认为在早泄中根本不存在这种神经症的特点。按照圣劳伦斯和玛达卡期拉的观点,没有对照研究支持精神动力学的可靠性。格式塔的观点强调早泄是伴侣之间缺乏相互了解的结果。横向的和其他人际关系的解释指出:具有破坏性的相互影响的模式在性交过程中孕育了无意识的力量,导致男性快速射精。这种假设受到一般临床病例的挑战,因为婚姻关系的改善并不一定能够带来性功能的改善,同时也缺乏对照科研的支持。行为科学的方法无论在早泄的病因还是治疗方面都已取得了实质性的进步。行为治疗家坚持早泄的病因是原发性焦虑,它可以缩短射精的潜伏期。例如沃尔佩(wolpe)提出早泄可能是焦虑的结果,因为早泄和焦虑都是由交感神经调节的。一些研究使个体心理因素与性功能联系起来,与早泄关系最密切的心理因素是焦虑和抑郁,尽管焦虑和抑郁都已被认为是性功能障碍的起始原因,但它们通常也被作为结果发生。因此,确定因果关系是复杂和困难的。作为一般的规律,临床医师往往将严重的焦虑和抑郁看成是病因,而将轻微的焦虑和抑郁看成是性生活失败的结果。而在多数心理学家的印象中,主要的病因仍然是焦虑,只是还缺乏具有对照研究的结果支持。事实上,考克考特等认为焦虑将影响男性性唤起的主观和客观评价之间的相互关系,发现早泄患者比患有勃起障碍的患者这些评价之间的联系更加密切:缺乏自尊、受到挫折、感到内疚、疑病观念、性交恐怖、敌意与愤怒、非现实的愿望和完美主义、内心冲突(例如悲伤、没有解决的性别定向、担心变态的性唤起方式等)以及严重的精神病理学问题都有可能造成早泄。在具有心理测量的有对照组的早泄研究中,使用明尼苏达多项人格测验,症状检查量表(SCL-90-R)已经产生混合的,有时是矛盾的结果。法甘等发现有性功能障碍的男性有正常的心理功能,许多其他的研究没有能够提供有力的论据来支持操作性焦虑或人际关系冲突与早泄之间的联系。另一方面,在对比研究具有不同性功能障碍的男性时发现具有早泄的男性有精神病理学性质(但是没有急性的临床表现),而其他患者则没有。一些研究表明早泄患者比没有早泄的其他性功能障碍的患者和正常人更容易焦虑,早泄的男性比没有早泄的其他性功能障碍者在抑郁和躯体化量表评定时分数为低,而在敌意和社会隔离方面的得分较高。另一些研究表明患有早泄的患者比患有勃起障碍的患者更健康。在一项最近使用的MMPI-2临床量表的研究中,麦耶斯发现量表分数在正常范围,而且在30个早泄患者,35个勃起障碍患者和25个性欲低下的患者之间没有显著性差异。

最常见的有关早泄的因果关系由那些治疗早泄的临床医师提出,他们虽然没有严格的科学的对照观察,但近几十年来却得到普遍的接受。例如,早泄者往往是因为性生理知识的缺乏和性技巧的不足,对性行为过于理想化,不能处理性刺激或性交困难。马斯特思和约翰逊将早泄归因于对性交过程特定的焦虑(称为操作性焦虑)和在情感、认知方面的双重压力,这种压力导致男性将自己从性行为的相互作用中分离出来,如同他本人对自己的性行为进行自我批评,同时他们还坚持早泄是在引起快速射精的具体情况下的条件反射和经验学习的结果,如仓促的第一次性交,与妓女性交或缺乏亲昵的性交。其他学者也非常强调条件反射,早期的性交失败(经常是第一次性交),长期的性行为不连贯,反对学习性技巧和对性持否定态度都可能成为早泄的病因。有了这些特点,一种快速射精的模式就成了男性的一种习惯。

始终存在争议的是手淫与早泄之间的关系,临床医师提出手淫可以使射精时间条件化,男性手淫时往往出于紧张,担心被人发现,以及内疚心理等而反复快速地射精,时间久了形成快速射精的状态,也就是早泄。但是一些人同样有手淫,他们不出现早泄,反而出现射精延迟,性交后还要手淫才能射精,医师的解释是因为手淫使射精的阈值提高,正常的性交刺激达不到手淫的刺激强度,还要靠手淫才能射精,因此发生射精延迟。而有一些男性从来没有手淫,也出现早泄,看来早泄与手淫有关的条件反射论仍然不是唯一的解释。

其他能够在临床上观察到的早泄的特点还有男性的内疚感和羞耻感,对性行为期望过高,严格的宗教观念,来源于女性性交要求的操作性焦虑,理智性的保护(如否认性唤起和快感的独立存在),性别定向问题以及具有冲突性质的亲子关系等。引起早泄的非心理特征包括性交频率过低,性刺激的质量问题,性伴侣的否定态度。临床医师还注意到性的因素和人际关系因素经常相互作用,人际关系问题可以引起早泄,而躯体和心理的问题可以加剧人际关系的紧张,两种因素相互作用或独立存在。

另外需要注意的是,性功能问题和婚姻问题之间没有严格的区别。一些有明显婚姻问题的夫妇可能有满意的性生活,而没有满意性生活的夫妇也可能有良好的婚姻关系。引起早泄的最常见的人际关系因素来源于对婚姻的不满、意见不一或冲突,这在继发性早泄更为明显。认为早泄是来源于对人际关系不满的理论根植于许多特性,如缺乏交流,非现实的婚姻期望(例如相信性是完美的或冲突是不好的),不能建设性地解决冲突问题,误信对方,害怕解决冲突问题,父母的不良婚姻模式影响到自己的婚姻,家庭系统的压力(照管伴侣的老年父母或学龄前儿童),性角色冲突,变异的性爱好或性价值观,职业问题,引起人际关系紧张的其他实质性问题。不幸的是这些理论缺乏科学的对照研究支持。很明显,许多早泄的男性有一个同时具有性功能障碍的伴侣。通常这些妇女的性唤起受抑制,由于丈夫有早泄而失去了许多性交的机会,而其他妇女只是难以达到性高潮。相反,继发性的早泄可能是对女性性唤起受抑制,性高潮受抑制,性交困难的反应。对某些夫妇而言,要辨认最初的起因即使并非不可能,也是非常困难的,因为许多情况下各种因素是相互影响,相互作用的。总之,文献揭示许多生理因素可以引起原发性或继发性早泄,没有什么心理学的现象或心理社会特点是永恒不变的。

目前有关早泄病因的理论倾向于将心理和器质的因素结合起来。尽管到目前为止还没有一种病因学理论得到广泛的认可,但人们认为心理因素加剧或恶化了潜在的器质性因素。原发性和继发性早泄可能有不同的病因。

早泄的心理因素包括性行为的条件化,焦虑也起到举足轻重的作用。当自主神经系统在射精过程中起到重要作用的时候,焦虑增加了交感神经的活动。在增加交感神经活动和降低射精阈值之间存在一个plausible的生理联接。然而,男性的射精问题与焦虑之间的关系还没有得到真正的证明。有学者认为焦虑更可能是早泄的结果而不是早泄的原因。

支持早泄器质性病因的依据是发现早泄有遗传性。Schapiro在1943年就报告早泄有家族的素质。最近Waldinger等报告射精潜伏期小于1分钟的早泄者的一级亲属中,14人中有10人患有早泄。根据遗传因素以及中枢神经系统5-HT功能紊乱,Waldinger推论早泄就是代表了人口射精潜伏期分布的一个时点。早泄的病因主要是以遗传素质为基础的,尽管认知和情绪也影响着这些原有的缺陷,但心理紊乱与早泄的关系是继发性的现象。然而,这种早泄的遗传素质理论也可以巩固其他早泄的器质性理论,包括阴茎的敏感性增强,射精反射弧过度兴奋,中枢5-HT受体过度敏感。

尽管已经存在多种多样有关早泄的病因学理论,但并没有一种理论能够让学者们完全信服,而且随着有关早泄诊断标准的放宽,尤其是参考女性的性生活满意度之后,早泄的比例已经上升到30\%左右,那么就有一个非常重要的问题浮现出来,我们社会上有近三分之一的男性是性功能障碍吗?这么多的早泄患者会是一种病因吗?自然答案是否定的。对于严重的素质型早泄可能有比较一致的病因,素质型的遗传学病因尤其值得重视。而针对不能让配偶达到性高潮的不严重的早泄,本人提出错配的病因学理论,尤其适用于那些射精潜伏期多于三分钟以上的男性。男性射精潜伏期有短有长,女性的性高潮潜伏期也有短有长,那么,我们所谓的早泄,其中一部分就是男性射精过快与女性性高潮延迟婚配造成的。例如,一个男性的射精潜伏期是5分钟,他的配偶的性高潮潜伏期是3分钟,他们就有和谐的性生活,男性的射精潜伏期还有2分钟的剩余,在性生活中可以游刃有余。如果他的配偶性高潮潜伏期是10分钟,那么他就是早泄。这样的例子在临床上屡见不鲜,尤其是那些夫妻都有过其他婚前性伴侣时这种感受尤为突出。如果男性是严重的素质性早泄,射精潜伏期在2分钟以内,而配偶的性高潮潜伏期是20分钟(我们可以称为性高潮延迟),他们的性生活一定不会和谐,无论如何治疗男性的早泄也不会达到理想的程度。所以早泄与性高潮延迟应该一同治疗。

意大利的Francesco Montorsi博士对多个国家进行了有关早泄发病率的研究,男性平均射精潜伏期是7~14分钟。德国男性射精潜伏期最短为7分钟,美国最长超过13分钟,英国、法国和意大利相似为9.6分钟左右。总体看女性定义的正常与男性比较接近,只有美国的男女之间的差异比较明显,男性高估了射精潜伏期(认为是13.6分钟)而女性则低估了射精潜伏期(认为是11.2分钟)。德国妇女正相反,高估了射精潜伏期,女性认为是7.4分钟,而男性认为是6.9分钟。形成广泛差异的主要原因是没有统一的标准和定义。目前在世界范围内,应用最广泛的早泄诊断标准就是ICD-10和DSM-Ⅳ。两个诊断标准都含有三个核心内容:①射精潜伏期短;②缺乏射精控制力;③性交满意度低。早泄诊断标准的局限性在于诊断的主观性太大,这样对于诊断和治疗都不能提供有力的基础数据。

Nolazno等报告:阿根廷男性2715人填写性健康问卷(参加前列腺保护活动),结果早泄的发病率为28.3\%(自我报告)。

全球研究的性态度和性行为包括29个国家,主要调查民众在性关系方面的态度、信念和整体健康状况。研究样本包括了27000个(年龄40~80岁)研究对象(其中女性13882名,男性13618名)。拉丁美洲的早泄发病率为28.3\%,接近三分之一的男性患有早泄。早泄发病率存在很明显的地区差异,中东最低为12.4\%,南亚最高为30.5\%,报告中包括偶发的早泄和持续存在的早泄两种情况。

1992年Lauman E牵头完成的全美健康和社会生活调查(使用问卷当面访问3432个美国市民),结果有29\%的男性报告患有早泄(PE)。美国国家健康和生活问卷于1992年完成,调查对象包括1410名18到59岁的男性,其中caucasion早泄是19\%;黑人早泄是34\%;西班牙人早泄是27\%。而最近的另外一项研究得出不同的结论。caucasion早泄是16\%;黑人早泄是21\%;西班牙人早泄是29\%。但该研究的人群年龄为40~80岁。

有研究表明,50\%的男性难以控制射精,23\%对性交不满意,30\%对性关系不满意,28\%对性交缺乏兴趣,34\%性唤起困难,31\%在性交中不能放松。

McCullough研究分析1158名网络男性(大约21岁,异性恋),按照DSM-Ⅳ早泄诊断标准有32\%符合早泄的诊断。

在性医学杂志五月刊:1587名男性的4周研究中,患PE的男性射精潜伏期(IELT)是1.8分钟,而正常的对照组是7.3分钟。PE者有明显苦恼,与性伴侣都有明显人际关系困难,他们缺乏对射精的控制并且对性交不满意。其中207名PE,1380名正常。

有关早泄的发病率没有统一的结果,高低差异悬殊。主要是早泄的诊断标准和类型不同,所以得出的结论也参差不齐。如果将一生中有过一次早泄就认定是患早泄,那么早泄的发病率可能很高,而早泄的诊断标准就是没有病程诊断标准,因为很多境遇型早泄只发生一两次或几次,而且与正常的非早泄状态同时存在。严格地讲,偶尔发生的早泄是没有诊断意义的,只有持续和反复发生的早泄才真正具有病理性意义(表8-1)。

表8-1 有关早泄发生率的文献选摘


\section{第三节 早泄的诊断与分类}

尽管近代对于早泄的研究只有一百多年的历史,但是在诊断与分类方面却取得了巨大进展。随着时代的进步,早泄的诊断越来越国际化,以美国为中心的有关早泄的研究逐渐成为主流并且对世界各国有明显影响。下面我们将简述早泄的诊断演变并且重点介绍美国早泄诊断系统的变化。

有关早泄的分类和诊断标准有很多,不同的学者根据不同的理论提出各自的标准,应该说各有各的道理,关于早泄(PE)的诊断标准差异很大,一个全球普遍认可的早泄定义仍未达成。常见的早泄诊断标准及理论有以下几种:

PE可以降低男性的自尊心,并且损害情侣之间的性关系。这种紧张的性关系导致马斯特思和约翰逊为早泄制定了这样一个定义:“男性在性生活过程中有一半的时候不能延长射精时间以便使配偶达到性高潮。”当这种性功能障碍是男性个人的问题时,它可能已经对伴侣双方都产生了不利影响。有关性交时间越长,女性达到性高潮的比例越高的研究强化了对PE研究的重要性。费瑟的研究表明:只有27\%的妇女在少于一分钟的性交中能够经常达到性高潮,而66\%的妇女要性交12分钟以上才能经常达到性高潮。费瑟主张女性平均需要8分钟才能经常达到性高潮。有人对马斯特思和约翰逊的PE定义提出批评,认为以女性性生活的满意度或性高潮来决定男性的早泄,其变异性太大,不够科学。

有人认为早泄的定义应该取决于插入和射精之间的一个标准的时间值。瓦西尔赛克经过调查确定:在健康的男性,性交平均持续20分钟。然而,盖勃哈德调查了上千对夫妻,他们的性交持续时间平均为4到7分钟之间,翰特发现美国夫妻的平均性交持续时间为10分钟,施奈波尔调查结果表明50\%的德国夫妇性交持续时间为5分钟多一点。为了测定男性的射精时间,豆纳德、波莱克做了一项男性射精时间的测定实验,具体步骤如下:首先向所有的受试者讲解实验的方法和全部过程,在实验开始之前取得实验对象的书面同意,并要求受试者至少在实验前几天要停止性交和手淫等性活动。在他们进入实验室之前,已经准备好接受测定射精时间的局部按摩刺激。他们被蒙上眼睛,仰卧在床上。在蒙上眼睛之后,女工作人员开始进行按摩,使用按摩油只对阴茎进行局部按摩。按摩师均是经过了职业训练,了解实验目的并且同意参加实验研究。所有参加实验的男性全部接受同一女性的局部阴茎按摩刺激。按摩是连续的,射精后停止。从开始刺激(从勃起时计算,因为有一些人在进入实验室时还没有勃起)到射精,定义为射精时间。达到勃起的时间,射精时间以及射精力量均由一名研究人员在隔壁的房间内记录下来,对受试者保密。同样的实验连续进行4次,每次间隔一周,实验对象和按摩师不变。结果所有男性4次射精时间的范围是45~470秒,平均数和标准差(SD)为156.5±80.7秒(N等于116)。116个实验样本的射精时间的频率呈正态分布。这为我们提供了一个较为可靠的男性射精时间的标准值。

又有人认为插入后的抽动次数很重要,因此,对PE提出不同的定义,如插入阴道后的抽动次数(例如8次或15次)和随意控制的能力。卡普兰将PE定义为缺乏对射精的控制,而拉舍斯对此有不同看法。在性学家最近最常使用的PE标准定义中,爱萨里安引用了“随意控制”。有关PE的定义很多,例如将时间与随意控制结合,将时间与随意控制看成一个连续过程,以及加入劳丽的早泄严重程度指数的多重标准(Laurie-PESI)。目前早泄的诊断标准中很少使用抽动次数的概念,也很少使用射精控制力的概念,在诊断标准中经常会出现射精的控制力,但并非必要的条款。

1.罗皮克洛认为插入后维持5分钟以上,或双方都同意他们的性生活美满,不因努力延迟射精而有所影响,即为正常。

2.我国著名的男科和泌尿外科学专家吴阶平教授认为,在壮年健康成人,在性交2~6分钟时射精或更短时间内射精仍属正常。

3.我国中华医学会精神科分会制定的早泄诊断标准为 射精过早致使性交不满意,严重者阴茎插入阴道时就射精。继发于勃起延迟者诊断为ED。诊断标准:①性交时射精过早,致使性交不满意。②持续至少三个月。③不是器质性原因所致。中国精神疾病诊断标准延用其他精神障碍的病程概念,如抑郁症、焦虑症,首次将病程纳入早泄的诊断标准中。但偶发的早泄怎么会持续三个月呢?如果新婚后三个月内一直是早泄,他们也不符合诊断,看来这个标准存在细节问题,未能与国际接轨,更不符合自己的国情。

4.ICD-10早泄诊断标准 无法控制射精,以使性交双方都不能享受性快感。在严重的病例中,未进入阴道或还未勃起时就出现射精。早泄多不是器质性的,但可作为器质性损害(如勃起不能或疼痛)的一种心理反应而出现。如勃起所需要的刺激时间较长,射精也会显得过早,这是由于充分的勃起与射精之间的间隔被缩短了。造成这种情况的根本原因是射精延迟。

北美的性健康专家开发了一个以循证为基础(evidence-based)的PE诊断标准,在2008年5月19日的103届美国泌尿外科科学年会上发表。他们将PE定义为:PE是一种男性性功能障碍,具有以下特征:总是或几乎总是以阴茎插入阴道之前或插入后大约一分钟以内就射精,总是或几乎总是不能在阴茎插入后延迟射精,而且产生负面的人际关系(比如苦恼)。泌尿外科学的PE诊断非常简洁,而精神科的PE诊断比较繁琐,但这个诊断也引入了射精潜伏期的时间概念,与即将颁布的DSM-Ⅴ同步,看来世界范围内早泄的诊断标准经过有时间概念到无时间概念,然后又回到了有时间概念,专家又有达成共识的趋势,诊断从宽松又回归严谨,而早泄的发病率也就又会从高向低变化。

美国精神病学会(APA)制定的《美国精神疾病诊断和统计手册》(简称为DSM)。DSM-Ⅰ1952年出版、DSM-Ⅱ1968年出版、DSM-Ⅲ1980年出版、DSM-Ⅲ-R 1987年出版、DSM-Ⅳ1994年出版、DSM-Ⅳ-TR 2000年出版、DSM-Ⅴ2012年出版。

对早泄研究与治疗影响最大的是美国的DSM-Ⅳ早泄诊断标准系统,因为这个早泄诊断标准完全采用模糊的指标,没有具体的时间概念,如:在受到轻微的性刺激之后,插入前、中、后很短时间,比本人意愿的时间提前,引起本人痛苦或人际关系紧张,考虑到年龄、伴侣、环境和频率的变化。这种采取模糊概念的方式恰恰反映出学者们认识到早泄是一个量变的过程,不是一个质变的过程,早泄与非早泄之间没有一个明确的分水岭。

自美国的精神疾病诊断和统计手册第四版(DSM-Ⅳ)发表以来,几乎全世界都使用这个诊断分类和标准,但依据这个诊断标准作出的PE诊断也引发了一些诸如诊断范围扩大化、就诊率低以及分类过于简单不利于指导临床研究和治疗等问题,2000年美国精神病学会又出版了精神疾病诊断和统计手册第四版教科书修改版(DSM-Ⅳ-TR:Diagnostic and Statistical Manual of Mental Disorders Text Reversion),对于PE的诊断标准做了较大的修改。该版本就是即将在2012年问世的美国精神疾病诊断和统计手册第五版(DSM-Ⅴ)的前身或者基础,将对世界范围内PE的诊断和治疗趋势产生重大的影响。

我们先阐述DSM-Ⅳ中PE诊断标准的历史意义,并且分析其存在的问题,再介绍DSM-Ⅳ-TR。

美国精神疾病诊断和统计手册第四版对PE的诊断标准是:A.长期或反复出现,在受到轻微的性刺激后、在阴茎插入之前、刚插入时或插入后很短的时间内即发生射精,比本人的意愿明显提前。医生判断时应考虑影响性兴奋持续时间的各种因素,如年龄、新的性伴侣、新的环境和近期性交的频率。B.这种功能紊乱明显引起了本人的痛苦或人际关系的紧张。C.这种PE不是由于某种物质(例如鸦片)的直接作用引起。亚型又可以分为:终生型(life-long type):从有性功能开始就发生PE。获得型(acquired type):PE发生在有一段正常的性功能即非PE状态之后。广泛型:PE的发生不局限于某种特殊刺激、环境或伴侣。境遇型:PE只局限于某种特殊的刺激、环境或性伴侣。

在DSM-Ⅳ的PE诊断标准问世以前,虽然已经有一些PE的诊断标准在临床上使用,但没有达成较为一致的共识,所以影响不大。而DSM-Ⅳ的PE诊断标准问世后,由于对于PE诊断的概念发生巨大变化,所以对医学界产生了广泛而深远的影响。该诊断标准的重要意义主要有以下几个方面:

自从马斯特斯和约翰逊提出他们的PE诊断标准之后,这种以能够满足女性或考虑到女性性生活满意度的PE诊断标准就大行其道。美国DSM-Ⅳ中的PE诊断显然受到影响,反映出对于性生活双方感受的同等重视程度。这一点由人际关系紧张的诊断条目得到证实。

以往性行为的主要功能集中在生育功能方面,而随着社会的进步和避孕节育技术的广泛应用,性行为的主要任务是享乐而不再是生育,所以PE诊断标准的放宽自然考虑到享乐因素所占的比重,而生育因素则是从属地位,在发达国家这个问题更为突出。

按照DSM-Ⅳ的PE诊断标准,有29\%的男性是罹患PE的,从生物学的角度看,除传染病之外,发病率如此之高是根本不可能的。因为DSM-Ⅳ是精神疾病的诊断标准,显然应用了生物、心理和社会医学模式作为制定诊断的依据,故意淡化时间概念而强调人际关系困难或主观感受,是心理社会因素主宰了诊断标准。

在诊断标准中提出人际关系,就是将男性独自的性功能障碍转变为男女双方共同的性问题,不但诊断需要考虑双方的感受,而且治疗也需要参考双方的感受。从而,将男性PE这种性功能障的治疗引向了夫妻共同参与的心理或药物治疗。PE是唯一一种诊断标准需要考虑女性性感受的男性性功能障碍。

DSM-Ⅳ的PE诊断标准已经使用了近15年,对世界各国的医学界有广泛影响,但经过多年的应用也发现了很多问题,主要有以下几项:

由于PE的诊断没有具体的时间概念,只要配偶双方中有一方对射精时间不满意或引起人际关系紧张就可以诊断为PE,这样符合PE的人数就剧增,研究PE的发病率几乎高达男性的三分之一。这样高的PE发病率能够真正反应男性性功能障碍的问题吗?这么高的发病率还是不是一种疾病或障碍呢?发病率过高对于问题男性的压力只会增加而不会减少。

虽然诊断PE的标准放宽了,符合PE诊断的人数增加了,但是,在现实生活中前来就诊的男性并没有显著增加,就诊率低与发病率高之间的矛盾非常突出。诊断范围过宽并没有促进PE问题的解决,这样的诊断标准也就失去了实际意义。这似乎有一些法不责众的味道,当然也与目前并没有最为有效的PE治疗方法有关,只能诊断不能治疗也没有实际意义。

DSM-Ⅳ的PE诊断标准只有终生型和获得型,以及广泛型和境遇型的简单分类,虽然诊断和使用方便,但由于过分宽松的诊断和过于简单的分类,该诊断系统对于治疗没有什么指导意义。罗兰德的PE分类就非常细致,例如:生理型PE(生物学因素致病的):神经素质型PE、躯体疾病型PE、躯体损伤型PE、药物副作用型PE;心理型PE(心理因素致病的):心理素质型PE、心理应激型PE、伴侣关系紧张型PE、性心理技巧缺乏型PE以及共病型PE。

由于DSM-Ⅳ发表之前,对于信度和效度没有做过深入的、适当的验证,所以诊断标准具有一定的局限性。ICD-10中有6个月的期限,而DSM-Ⅳ没有。该诊断标准也不是操作性的诊断标准(operational criteria),有些概念比较模糊,如,插入后很短时间内(shortly after penetration),到底是多么短的时间,应该有具体的时间概念。比自己的意愿提前,自己的意愿是多长时间,如果自己的意愿是60分钟呢?由于该诊断标准过于宽泛,没有具体的时间概念,主要依靠当事人的主观感受,分型也相对简单,所以不利于对PE进行深入研究。

2000年美国精神病学会又出版了精神疾病诊断和统计手册第四版教科书修改版(DSM-Ⅳ-TR)。其中对PE诊断标准做了较大的修订。提出终生型PE、获得型PE、自然变异型PE和PE样射精功能障碍四种类型。简介如下:

终生型PE是以下列一组核心症状为特点的综合征:

(1)几乎每次交媾都射精过早;

(2)几乎与每个女性都是如此;

(3)从第一次交媾开始就发生;

(4)大多数情况下(80\%)射精在30~60秒,或者在1~2分钟之间(20\%);

(5)一生都将处于快速射精(70\%)或随着衰老加剧(30\%);

(6)控制射精的能力减弱或缺失,对于诊断并非必要。

这是一类最为严重、最难以治疗的PE,应该说是典型的PE,或者是唯一必须治疗的PE。这类PE一定影响伴侣之间的感情,不能让配偶达到性高潮,而自身也一定感受到精神压力,由于射精太快虽然有性高潮,但不够典型和强烈。尤其是缺乏性生活中的愉快感受,他们挫折感明显,有些人否认PE或者忽视配偶的感受,但内心却十分痛苦。诊断标准的第一和第二条说明他们无论与什么人性交,无论什么时间性交都出现PE,这表明PE的稳定性。第三条第一次性交就发生PE和第五条一生都将处于快速射精(70\%)或随着衰老加剧(30\%)表明了PE的持续性。第四条在大多数情况下(80\%)30~60秒射精,或者在1~2分钟之间(20\%),引进了时间和比率概念,表明了PE的严重性。该类型的PE是PE治疗和研究的重点。

获得型PE的主诉与潜在的身心或心理问题不同,有以下特征:

(1)PE发生在男性生活中的一些不同的时期;

(2)开始有PE主诉之前通常有正常射精体验;

(3)这种PE如果不是突然出现就是缓慢起病;

(4)PE可能源于泌尿科的功能障碍(如勃起功能障碍或前列腺炎)、甲状腺功能障碍、心理或人际关系问题。

与终生型PE不同,获得型PE可以通过治疗潜在的病因而得到治愈。该类型的PE没有很大的变化,获得型PE就是继发性PE。以往曾有过正常的射精潜伏期,之后由于躯体疾病而产生PE。最常见的原因就是泌尿系的炎症。治疗原发病为主要治疗方法。

在自然变异型PE中,男性只是巧合的或境遇性的发生PE。这种PE的类型不能看成是真正病理性的症状或现象,而只是性活动的正常变异。这种综合征以下列症状为特点:

(1)PE不是持续发生或是不规律发生;

(2)控制射精的能力(感受到快要射精时自己主动控制射精)减弱或缺失,对于诊断并非必要;

(3)控制射精能力减弱的体验与射精时间短或正常同时存在(小于或大于1.5分钟)。

这种类型的PE以往称为境遇型PE,主要是在特殊的环境和心理因素作用之下才发生。比如与新的性伴侣发生性行为就PE,性兴奋程度过高,熟悉之后不再发生PE。新婚时期发生PE而以后并不PE也属于这一类,原因包括受到手淫导致PE误解的暗示作用和性兴奋程度较高。适应之后可以恢复正常,及时的咨询或指导更为重要。精神紧张或焦虑是重要因素,由于PE的发生具有非常明显的心理原因,诊断并不困难。

有早泄样射精功能障碍的男性体验或主诉PE是在射精时间正常(3~6分钟)或超长(5~25分钟)情况之下。这种类型的PE不能真正看成是病理性的症状或现象,心理抑或伴侣关系问题可能是主诉的原因。这种综合征以下列症状为特点:

(1)交媾中持续或非持续快速射精的主观感知;

(2)想象的PE或射精控制能力缺失的先占观念;

(3)实际的射精潜伏期(IELT)正常或超长(5~25分钟);

(4)控制射精的能力减弱或缺失,并非诊断所必须;

(5)这种先占观念不能用其他精神障碍合理解释。

早泄样射精功能障碍诊断要点:①主观感受为主;②客观时间不短;③并非精神障碍。

早泄样射精功能障碍实际上是采用了精神病学分类之中的一个术语,如同精神分裂样精神障碍,实际上就不是精神分裂症,只是有些类似。而早泄样射精功能障碍就如同是早泄样早泄,不是早泄而类似早泄。诊断中突出的是主观感受。先占观念是一个精神病学名词,是指受情绪影响而产生的与现实不符的难以改变的想法,类似一种坚定不移的信念。无论事实如何,患者就认定自己是PE。客观上射精潜伏期并不短,甚至还比较长是诊断本型PE的一个最重要的依据。主观的感受短与客观的射精潜伏期长形成了矛盾。第三个特点是并非精神障碍。这是一个排除用的诊断条目。如果一个精神患者认为自己是PE,那么可能是妄想所致,但在排除严重的精神障碍之后就可以诊断为PE样射精功能障碍,这类似于神经症类别中的疑病症。

表8-2 DSM-Ⅳ-TR提出的PE分型和诊疗特点

精神疾病的诊断标准与其他器质性疾病不同,由于功能性疾病没有病理改变或实验室检查作为辅助诊断依据,多数是根据临床表现作出诊断的,所以诊断的准确性就难以把握。但与DSM-Ⅳ相比,DSM-Ⅳ-TR有其成功之处,当然也存在许多问题。

主要包括以下几个方面:

(1)有PE的详细分类和相对明确的时间概念:由于DSM-Ⅳ没有时间概念,也就是取消了射精潜伏期的概念,难于进行操作性诊断。DSM-Ⅳ-TR分类更加详细,又重新引入时间概念,而且突出不同类型的PE时间的差异。发病比例也反映在诊断之中。

(2)取消描述性诊断,引入综合征更加科学:由于精神科疾病多数是功能性疾病,没有器质性改变作为病理依据,也没有相应实验室检查作为辅助诊断工具,所以临床表现就成为主要的诊断依据。DSM-Ⅳ采用的是描述性定义,具体概念比较模糊,主观性强。而DSM-Ⅳ-TR采用临床综合征作为诊断依据,更加具有科学性。

(3)减少高发病率和解释治疗成功率的差异:DSM-Ⅳ-TR的PE诊断标准更为严格,所以会适当减少PE的发病率,这样也避免了许多男性带上PE的帽子,而生活并没有得到相应的改善或者只是增加了精神压力而已。以往有学者报道采用性感集中训练法治疗PE取得骄人的成效,但后人使用同样的方法却难以获得同样的效果,问题在于没有亚型的诊断。不同类型的PE对于同样的方法疗效是完全不同的。终生PE是难于治疗的,而自然变异型PE是容易治疗的,所以不分组的和诊断不明确的研究没有实际价值。

(4)PE亚型分类为进一步研究拓展了空间:由于能够引起PE的原因很多,PE不是同质性的性功能障碍,对PE进行详细的分类有利于对其进行深入的研究。比如遗传学研究,实验室研究以及药物治疗研究。也就是说,PE是一大类综合征而不是一种障碍。每一亚型的PE都有特定的临床表现,无论是诊断还是治疗都有必要进行深入研究。

(5)将射精潜伏期正常或超常患者单独归类:一些射精潜伏期正常或超常的“PE者”也被纳入了PE诊断标准,这样就解决了主观愿望与客观指标不一致的问题。这主要针对PE样PE的患者,他们的射精潜伏期可能是3~6分钟,甚至是5~25分钟,只要他们认为是PE,仍然可以诊断为PE,主观的感受作为诊断的重要依据,而不是具体的时间概念。这样的单独归类更能够体现该亚型的心理疾病特点。

尽管DSM-Ⅳ-TR在原有的DSM-Ⅳ基础上做了很大幅度的修改,但是对于PE的诊断标准仍然不能说是取得了重大突破,或者达到金标准。当然,由于PE的特殊性,目前彻底解决这些问题还不是时机,我们提出这些问题主要是让大家理解作为诊断标准的利弊关系或者不足之处,有待各位专家在今后的工作之中不断研究,探索出新的解决办法。

(1)仍然有特殊类型难于归类:虽然该诊断标准有了四种类型的PE亚型,但是不能包括一些常见的亚型,比如共病型PE,或者自己不认为是PE,但不能让配偶达到性高潮。生理与心理因素混合类型广泛存在,生理因素导致的PE也对于心理因素有明显影响,而心理因素引起的PE也会对生理因素产生作用,两者相互影响,最后难于彻底区分。对于一个PE患者而言,不能完全彻底地区分出心理和生理型的PE。

(2)没有考虑性伴侣错配问题:一些PE不是男性本身的问题,是错配而导致的。比如,男性的射精潜伏期是6分钟,他的配偶的性高潮潜伏期是4分钟,他们的性生活是和谐的,男性不是PE。如果同样是这位男性,他的性伴侣的性高潮潜伏期是15分钟,那么这位男士就由原来的两分钟优势变成了9分钟的欠缺,由于女性不能达到性高潮,他就是PE了。这就是PE的错配理论。所以,PE不但是要考虑性伴侣(女性)的感受,还要考虑她们的性高潮潜伏期与男性的匹配情况。治疗男性的PE同时应该治疗女性性高潮延迟。

(3)诊断PE的指标仍然不足:影响PE的因素很多,但分类之中并没有采用更多的诊断条目或指标,该诊断仍然没有解决抽动数量和时间关系的概念。比如,一个男性说射精潜伏期是2分钟,但是没有抽动动作,另一个男性也说射精潜伏期是2分钟,他连续抽动60次,这样的性生活的实质性差异不能单纯用时间判断,而应该加入抽动次数的概念,如果要全面反映性生活的实质,甚至还应该考虑到有无女性的躯体摆动、性生活体位对射精的影响、女性主动的阴道套入滑动以及性高潮期间的收缩等情况。该诊断没有纳入性生活频率的指标,但是性生活频率对于PE有明显的影响。

(4)PE样射精功能障碍意义有多大?男性射精潜伏期的问题比较复杂,很多人认为射精潜伏期越长越好,这样的先占观念与阳具越大越好如出一辙,是男性对自身性能力的一种炫耀性的期盼,也是受传统社会文化习俗影响的结果。PE样射精功能障碍的实际射精潜伏期并不短,可能还比较长,但这样的诊断标准到底会有多少人符合呢?按照该诊断标准研究该类型的发病率还比较高,但实际的意义并不大,主要不是男性的主观感受,而是男性考虑或者担心女性的感受。男性会因为不能让女性达到性高潮而感到自卑或挫折感。

在官方或国家的早泄分类和诊断系统之外,最具有价值和对临床具有指导作用的是罗兰德早泄分类治疗指南。除了按照一般学者将PE分为生理型和心理型之外,还可以将PE分为不同的亚型,这些亚型是建立在对PE的病因假说和严谨的临床观察基础之上的,所以更加科学、合理、简便和实用,尤其适用于临床治疗和研究工作的指导。在各类学术文献中有4种生物学或生理学类型的PE,即神经素质型、躯体疾病型、身体损伤型和药物副作用型PE。心理型PE也是4种,即心理素质型、急性心理应激型、伴侣关系紧张型和性心理技巧缺乏型。PE分类见附录表8-3。

(1)神经素质型PE:神经素质型PE很可能是由于内在的生物学的“反射”或神经系统中的生理学素质造成的射精过快,并且划分为终生型和广泛型(在什么情况下都早泄,比如性生活中和手淫时一样)。

(2)心理素质型PE:一般认为心理素质型PE是由人格造成的,快速射精是慢性或长期心理障碍的结果,诸如环性心境障碍、广泛型焦虑、强迫性障碍和各种人格障碍。

(3)性心理技巧缺乏型PE:性心理技巧缺乏型PE是由于男性在性交过程中缺乏对身体感觉的控制技巧而造成的。当最初的病史提示是终生型PE时,无论是广泛型还是境遇型,临床医生都应该考虑性心理技巧缺乏型PE。

(1)躯体疾病型PE:躯体疾病型PE往往是由许多急性的疾病所引起,也就是说继发于其他的躯体疾病,例如尿路感染,躯体疾病型PE要使用广泛型和获得型来进行特殊的限定和标明,使诊断更加完善。躯体疾病型PE要通过观察来确定,当疾病引起PE时要进行适当的治疗(例如,对前列腺炎的抗菌治疗),这样,PE的问题就解决了。

(2)躯体损伤型PE:躯体损伤型PE是由对身体的不可逆损伤而引起的,比如脊髓的损伤、脑损伤、腹主动脉瘤手术对交感神经的损伤、其他躯体疾病损伤以及局部的感觉神经损害等。病史一般可以提示引起PE的躯体损伤。

(3)药物副作用型PE:药物副作用型PE是一种在戒断使用某种药物或毒品的过程中出现的独特的PE。例如,使用氯丙咪嗪治疗强迫症,在强迫症治疗康复后停药时,原来PE出现了。其实是氯丙咪嗪治疗强迫症的同时也控制了患者原来的PE,但停药后这种PE又反跳或叫做复发了。滥用阿片类毒品的患者也会在停止使用毒品后出现射精过快的现象,所以在诊断标准中要排除药物的滥用。

(4)心理应激型PE:心理应激型PE是由暂时的心理问题,如适应障碍、急性应激障碍、严重的抑郁、心理社会性的应激(例如,职业性的应激性交焦虑,权利纷争、居丧反应、经济负担问题以及文化适应问题等)都能加重PE。

(5)伴侣关系紧张型PE:伴侣关系紧张型PE来源于人际关系的精神动力学的情感关系冲突,对于性伴侣的过度敏感或不忠行为的反应。传统的性治疗师都相信人际关系的精神动力学在PE的致病因素中充当了重要角色,快速射精可能是一种潜意识的对于妻子不忠行为的愤怒或者是不再喜爱对方的一种敷衍了事,但是在这方面的临床研究报告数量相当有限。

(6)共病型PE:最后我们要讨论的是PE与其他性功能障碍同时存在(coexists)的问题。伴随其他性功能障碍的PE也称为共病型PE。这在目前的精神医学中是一个时髦的话题,我们称之为共病(comorbidity)或并存疾病。共病的概念是两种同类的疾病同时或先后出现,相互影响。PE是一种性功能障碍,如果与另外一种性功能障碍同时出现在一个人身上,就是共病。比如,早泄的患者同时患有勃起功能障碍。有些患者在还没有充分勃起的时候就已经射精了,早泄和勃起功能障碍无论是先后还是同时发生,仍然是两个病,不能混为一谈,共病就是共同存在。这种情况会使治疗复杂化,增大了治疗的难度。

启始阶段的鉴别诊断:

(1)确定是终生型还是获得型PE。

(2)确定是广泛型还是境遇型PE。

鉴别诊断:

终生型PE:如果确诊为终生型和广泛型PE。

查找是否存在快速射精反射(如尿道球海面体肌反射测试)的生理素质依据和病史,如果为阳性结果→诊断为神经素质型PE。

治疗:药物治疗和认知行为治疗。

如果是阴性结果→第二步。

查找是否存在构成PE的长期心理素质或精神病理学依据(例如,强迫症、抑郁症、广泛性焦虑和人格障碍等)。

阳性结果→诊断为心理素质型PE。

治疗:个体治疗、精神药物治疗和认知行为治疗。

阴性结果→第三步。

终生型PE:只要确定为终生型PE,无论是广泛型还是境遇型PE。

查找男性在性交过程中,在认知和行为方面是否存在不懂得进行生理性放松的情况,也就是说不具有进行生理性放松的能力。注意力要集中在男性对性唤起和射精的体验方面,看他是否采用了能够快速引发性唤起的途径。

阳性结果→性心理技巧缺乏型PE。

治疗:标准的认知行为治疗。

阴性结果→再复查终生型PE,再考虑第一步和第二步,然后进入第四步。

获得型PE:如果确诊为获得型和广泛型PE。

查找是否存在能够引起PE的躯体疾病。

阳性结果→躯体疾病型PE。

治疗:针对现存疾病进行治疗。

阴性结果→第五步。

查找是否存在能够引起PE的躯体损伤、骨盆手术或神经系统损害。

阳性结果→躯体损伤型PE。

治疗:药物治疗,也可以考虑认知行为治疗。

查找是否使用过能够引起PE的药物或正在戒断能够引起PE的药物或毒品。

阳性结果→药物副作用型PE。

治疗:戒除能够引起PE的药物或毒品。

阴性结果→进入第七步。

获得型PE:如果确诊为获得型PE,无论是广泛型还是境遇型PE。

查找是否存在支持患者目前处于应激状态的病史和心理测验的结果。

阳性结果→心理应激型PE。

治疗:心理治疗和认知行为治疗。

阴性结果→第八步。

通过访谈和询问病史以及心理测验,查找是否存在伴侣关系紧张的证据。

阳性结果→伴侣关系紧张型PE。

治疗:心理治疗和认知行为性治疗。

阴性结果→第九步。

无论是什么类型和状态的PE。

查找是否同时存在其他性功能障碍,如勃起障碍(ED)、性欲低下等。

阳性结果→共病型PE。

治疗:药物与心理治疗。

(如果其他性功能障碍与PE有因果关系(如先ED后PE),先治疗ED,然后治疗PE,也可以同时治疗几种性功能障碍),再考虑第一步、第二步、第七步和第八步。

阴性结果→重新考虑和再评价该病例(例如,ED的表现是勃起持续的时间缩短,而PE能够掩盖ED的存在等)。

表8-3 PE的分类


\section{第四节 早泄的临床应用}

早泄患者的临床表现主要来自患者的自诉或性伴侣的表述。早泄患者多数具有正常的性欲与良好的勃起功能,只是在性交时过快射精。射精过快的表现方式复杂,与环境、对象、时间和情感等因素有着千丝万缕的联系。主要有下列几种表现:

最为严重的病例是阴茎还没有真正插入阴道时就已经发生射精。这种情况往往是以视觉和听觉因素为主,在强烈的性兴奋之下导致射精。最为典型的是有些男性在观看色情影视作品的时候射精;也有在公共汽车或拥挤的时候,阴茎接触到女性的身体时控制不住而射精;新婚或与新的性伴侣在准备性交的时候,阴茎没有插入时发生射精。古代书籍中将这种状态描述为“见花泄”,如果这种情况持续存在,那么就是最为严重的早泄类别。

有些男性害怕女性用身体,尤其是手抚摸他的阴茎,轻微的抚摸即可引起射精。这样的男性往往不喜欢前戏,他们害怕或回避真正性交前的身体接触或情感交流,他们希望一步到位,直接性交。但女性在没有达到充分性唤起的时候,阴道的润化不足,情绪也处于启动状态,不会有愉悦的感受。当然,这样的男性即使可以插入,也是很快射精,多数女性不会达到性高潮,而男性最多是避免了插入阴道之前的射精。

这类患者具有良好的勃起功能,性交过程也顺利,只是不能延长射精潜伏期,在任何情况下都是一样的快速射精。一般射精的时间在2分钟之内,有时这些患者的表现会有一些微妙的变化,比如在插入阴道后保持不动,也就是没有抽动或推送动作,这样能够保持相对长一些的性交持续时间。有些患者性交过程的射精潜伏期是2分钟,抽动80次,有些是抽动4次。减少抽动次数多会延长性交,而增加抽动会缩短性交。也有患者述说女性的主动活动或阴道的收缩会加速他们的射精。这种情况是早泄的主体。

这类患者的主要特点是射精潜伏期呈波动状态,时间长短可以变化,而且主要是与性交对象或场所的变化直接相关。他们在熟悉的地点或与熟悉的性伴侣发生性行为时并不出现早泄,而当遇到新的性伴侣或在陌生或不安全的场所时才会发生。最为典型的是婚外性行为,如嫖妓行为或与偶遇的情人发生性行为。

这种类型的早泄患者往往不是射精潜伏期过短,也许他们的性交时间可以持续5分钟或者更长,但他们内心并不满足。主要有两种表现:一种是自己觉得是早泄,因为很多男性可以持续10分钟,20分钟,甚至更长时间,所以希望治疗。另一类是以配偶达到性高潮为标准,这是属于相对性早泄,也是应该治疗的类别,错配的早泄主要是指这种类型。

能够导致射精过快的因素有很多,简述如下:许多两地分居的夫妻在节日相会时会发生早泄;普通夫妻外出后(禁欲一段时间)也会有早泄出现;过度疲劳、夫妻情感冲突或有明显精神压力时也会出现早泄;停止使用能够延长射精的药物后会出现早泄(最为典型的是阿片类药物戒断后和抗抑郁剂停止使用后的射精过快);某些泌尿生殖系统的炎症或疾病也会引起射精过快,如前列腺炎等。

在早泄诊断和治疗中病史的采集非常重要。因为早泄的诊断主要是射精潜伏期过短,而且是在性生活中发生,所以相关因素都需要详细了解。主要有以下几个方面:

患者什么时候开始手淫,手淫的方式和频率,手淫对于患者的影响和有无精神压力。在手淫过程中患者的射精时间,是否存在初期手淫射精潜伏期较长,后来射精潜伏期较短。患者是否听说手淫可以导致早泄,有无在紧张恐惧或不利环境中快速手淫的经验。

患者第一次性交是否发生早泄,而后来早泄是否持续。患者是第一次性生活就发生早泄,还是有了一段正常的性生活后出现早泄。特殊的时间和地点以及更换性伴侣对于早泄有无影响。有无特殊的性生活经历与早泄有关。

性伴侣对于早泄的态度,性伴侣能不能宽容早泄或者给予患者安慰和支持。性伴侣的性生活经历,是否有过其他性伴侣,性生活质量,性高潮潜伏期长短以及是否可以通过手淫达到性高潮等。

很多早泄患者都有一些治疗早泄的经验,他们服用过哪些药物,哪些药物有效,有无副作用。还有什么治疗早泄的方法,是否有效。

有没有什么疾病与早泄有内在的联系,包括生殖道感染和慢性疾病。有无吸毒或使用抗抑郁剂的经历,目前患者的精神状态。

患者有无其他性心理障碍,是否合并其他性功能障碍,患者的性取向如何,是否为同性恋,患者自己对于早泄的看法和治疗需求以及与性伴侣的愿望的差异等。

辅助检查对于早泄诊断的意义不仅是对早泄作出了判别,更为重要的是对其进行了分类,为其治疗提供了依据。

尽管患者已经有早泄的主诉,诊断并不困难,但仍然需严禁的体格检查。体格检查的重点是全身发育情况和第二性征的发育情况。关注阴茎的发育、长度、有无包皮过长、嵌顿、狭窄、包皮炎、龟头炎、阴茎弯曲、尿道下裂等,注意双侧睾丸发育、大小、是否有硬结、输精管缺如、是否做过包皮手术或输精管结扎手术等,详细检查前列腺大小、是否触痛或硬节等。

一般的辅助检查对于诊断早泄并非必不可少,但是在用药治疗方面具有重要意义。一般需要进行血常规和尿常规检查。血生化检查包括肝肾功能和尿糖水平。心电图检查排除心血管疾病,泌尿科常规检查排除泌尿系炎症引发的早泄。检查血的性激素(LH、FSH、PRL、E2、T)主要是了解性腺的发育情况,如果有阳性结果要进行进一步的相关学科检查。

心理量表评估或检查主要包括SCL-90检查、焦虑抑郁量表检查、人格测验、早泄诊断问卷和抗抑郁剂副作用调查问卷等。SCL-90检查、焦虑和抑郁量表目前主要用于判断早泄者的精神健康状态,排除严重的精神疾病。但焦虑和抑郁往往不是早泄的原因,而是早泄的结果,这对治疗有参考意义。人格测验中,使用最多的是明尼苏达人格测验,但人格测验的结果只能够作为参考,对于诊断不具有实质性的意义,因为早泄的发病率大约是男性的30\%左右,这些男性不都是有人格问题的,但却是有严重人格障碍的人,在诊断和治疗时要格外谨慎和特殊对待。早泄的筛查和抗抑郁剂评估问卷是非常实用的,一方面可以辅助诊断,另一方面可以指导治疗,而且有利于保存临床治疗的研究资料(量表附后)。

(1)阴部诱发电位(pudendal evoked potential):该方法是通过电刺激阴部神经末梢来记录脑电图波形,分析躯体性感觉诱发电位潜伏期和振幅,来评价阴茎背神经向心性传导功能及脑中枢神经兴奋性。一般认为,阴部诱发电位可以对神经检查中有微小异常的患者的阴茎传入感受障碍的存在、定位和性质作出客观的评价。

(2)阴茎震动感感觉度测定法(penile biothesiometry):素质型早泄的部分致病因素是阴茎皮肤过度敏感、传入神经的兴奋性增高或射精中枢的兴奋阈值过低。可以利用阴茎震动感感觉度测量仪检测阴茎的龟头、体部、阴囊的震动感觉阈值(vibration threshold)。早泄患者的阴茎感觉度比正常人明显降低。该检查方法可以为早泄诊断提供重要的客观依据。

(3)球海绵体反射潜伏期测定法(bulbocavernous evoked potential):球海绵体反射测定的方法是通过电刺激阴茎体,然后记录海绵体肌电图,如果海绵体肌电图潜伏期延长,提示躯体神经反射弧病变,这种检测有助于勃起功能障碍的诊断,对于早泄有一定的参考意义,但特异性较差。

根据临床表现和病史对早泄作出诊断并不难,关键是采用什么诊断标准。目前国际上比较通用的早泄诊断标准有世界卫生组织的诊断标准(IDC-10)、美国泌尿外科早泄诊断标准和美国精神病协会的诊断标准。而国内有泌尿外科协会的诊断标准和精神病学会的诊断标准。目前多数国家以美国精神病学会的早泄诊断标准为重要参考依据。与其他疾病不同,早泄的诊断主要是依据临床表现、病史和辅助检查以及患者自己和性伴侣的治疗愿望综合判断后作出的。也就是说,尽管患者的射精潜伏期很短,完全符合早泄的诊断,但患者自己和配偶都不认为这种情况有什么不妥,他们的感情也不会因此而受到影响,那么就无需作出早泄的诊断。相反,他们的性交时间可能超过10分钟,其中有一方或双方都认为射精过快,影响性生活质量和夫妻感情,那么也需要给予帮助,仍然可以诊断为早泄,这就是美国精神病学会即将在2010年出版的(DSM-Ⅴ)最新早泄诊断标准中发表的早泄样射精功能障碍。

在性兴奋时,尿道球腺和尿道旁腺的分泌液体可以从尿道口流出,这些液体对尿道起润滑作用,便于精液的射出,有些患者误认为这是射精。射精的量要比分泌出的黏液多很多,而且黏液是呈透明的液体,而精液比较黏稠。经过检查,里面含有大量的精子。

通常勃起功能障碍与早泄是很容易鉴别的,但有的患者早泄与勃起功能障碍会同时存在。正常人在射精后很快会发生阴茎疲软,勃起功能障碍也会使阴茎很快疲软,如果早泄发生在有勃起功能障碍的男性,会掩盖勃起功能障碍的存在。

有些男性对于射精有很好的控制能力,对于不喜爱的性伴侣可以表现为早泄,以此作为借口而回避性生活,而在与自己喜欢的性伴侣发生性关系时并不发生早泄。这种情况在夫妻共同治疗早泄的情况中偶见,需要通过婚姻关系治疗来解决。

这种类型的早泄可以诊断为早泄,也可以不诊断为早泄,关键看使用哪种早泄的诊断标准。其实这些男性不是真正意义上的早泄患者,只是男性的射精潜伏期相对较短,而性伴侣的性高潮潜伏期相对较长,形成时间差。比如一个男性的射精潜伏期是5分钟,前妻性高潮潜伏期是3分钟,他们的性生活很和谐。而离婚后再婚,现任妻子的性高潮潜伏期是15分钟,而她的前夫射精潜伏期是20分钟,他们的性生活也很和谐。可是这对再婚的夫妻性生活就出现了问题,这位男士成了早泄,妻子认为他的性功能有问题。严格来说这种情况可以不诊断为早泄,但从夫妻性生活和谐的需要角度应该给予治疗,那么就可以诊断为早泄。


\section{第五节 早泄的治疗与咨询}

由于早泄的发病率高达30\%左右,是男性最为常见的性功能障碍,除继发性早泄之外,原发性早泄往往是从新婚期开始发病,而且持续终生,早泄对于男女双方的情感和性生活质量都具有明显的影响。所以早泄的治疗具有举足轻重的作用。下面我们从认知行为治疗、抗抑郁剂治疗、经验性应对措施、其他治疗早泄的方法和早泄咨询方法几个方面进行介绍。

我们现在将所有能够治疗PE的方法都罗列出来,因为单独使用一种方法来治疗PE是很难奏效的。几乎在所有成功治疗PE的病例中,都需要将多种方法有机地结合起来使用。多数男性同时需要4~5种方法的联合治疗。

满意的性功能是生理性放松的结果。因为身体要做到操作行为的理想化状态,就必须做到生理性的放松。这通常与我们的直觉相反,因为多数男性都知道,性唤起是建立在兴奋或焦虑的基础之上而不是放松的基础之上。然而,要想成功地治疗PE,就必须对特殊的生理性放松进行详尽的指导,将其作为其他治疗方法的基础。在治疗PE过程中有很多简单和容易掌握的使用技巧。每天10~20分钟的呼吸训练、身体知觉训练、肌肉放松训练等,这些都应该得到支持和鼓励。这样做的目的是将注意力集中在躯体感觉上,因而消除身体的紧张。

在感觉体验训练中,患者应该将注意力集中到对自己身体的视觉和触觉的探索方面。这样做的目的是从视觉和触觉两个方面更加熟悉自己的身体,并且对触觉的反应也有更清晰的知觉体验。我们要让PE患者学会建立在催眠理论上的性唤起模式。我们观察到有PE的男性本能地使用了涉及性伴侣的来自“自己身体以外”的性唤起模式,他完全是通过将注意力集中在身体外部的性刺激之上而产生性唤起,当然主要的关注对象是性伴侣的身体。对于大多数有PE的人来说,这是很“自然的”并且是他们能够产生性唤起的唯一方式。虽然这样做对于产生性唤起是一种享乐技巧,但对于性唤起的控制却没有什么帮助。很多有PE的人在射精时感到很惊讶,因为他们通常没有将注意力集中在自己的身体和感觉上。

要想学会控制射精,就必须教会这些患有PE的男性掌握另外一种性唤起模式,即自我关注的性唤起。在这种性唤起之中,注意力的焦点是在他自己的躯体感觉之上。大多数的治疗练习(见下面)都以将注意力集中在自己的感觉方面作为一个客观的目标。让他们通过练习学会通过自我关注而达到性唤起,而不是通过“性伴侣的互相影响”而达到性唤起,要向他们提供一些基本的改善对性唤起控制能力的方法。通过这种方法的练习,他们就会更精细地将注意力集中在躯体感觉方面,并且能够和谐地从认知和行为两个方面调控性唤起。将注意力集中在身体方面是一种有助于放松、能够帮助饱受PE之苦的男性克服走神和分心的技巧。这种技巧能够让他们更好地体会控制自己身体感觉的乐趣。例如,男性不要将注意力集中在性伴侣的乳房、性幻想上,但也不要有注意力分散,诸如运动形象等引起的注意力分散,要尽量将注意力集中到阴茎龟头的快感上。

有效的治疗是要让患者学会有意识地放松骨盆部位的PC肌,这个PC肌既是一个监视躯体放松水平的指标,也是一种帮助控制射精的生理方法。PC肌的训练是由凯格尔(Kegel)的女性训练技巧演变而来,这种方法是要让男性在产生性唤起的时候将PC肌(球海绵体肌和坐骨海绵体肌)放松下来。只有当PE的男性完全按要求放松并且与其他认知和行为治疗技巧结合起来时才能发挥对PE的治疗作用。这种技巧成功的关键是抓住了这些肌肉的放松,这种肌肉放松作用是一种对射精的自然控制力量。而多数男性在手淫过程中学会的阴茎勃起是肌肉的极度紧张,好比枪的子弹已经上膛而且随时准备发射,这样的状态就容易导致射精。有患者的射精潜伏期是1分钟,抽动次数是60次,其实完全将性交等同于手淫,丝毫没有停歇,如果能够放松,早泄就消除了。

用于治疗二便失禁的改良的骨盆血流的恢复训练计划有3种技巧:骨盆血流的生理动力学治疗、电刺激和生物反馈。这样的治疗程序要分20次,每周3次。这种技巧是要让患者学会利用骨盆肌肉的放松和收缩来调整骨盆区的血流,改善对自主活动的认知。大约有60\%的PE男性在经过治疗之后感到有疗效。这种方法主要涉及射精功能的肌肉训练,是物理治疗的内容。

此法由西蒙(Seman)于1956年首次提出。长期手淫可以导致快速射精,此法则反其道而行之。这种技巧是通过调节生理的性唤起强度来达到控制射精的目的。是要男性通过私下的自娱(手淫)练习而达到延长与性伴侣做爱的时间。动-停技巧要求男性通过一系列的练习,也就是集中注意自己的身体感受而激发性唤起,然后再减弱性唤起,这样他就了解了自己即将射精的感受,也就学会了控制射精反射的能力。当一个人自己能够完全掌握这种方法之后,就要配偶共同使用同样的方法。动-停技巧包括循序渐进地通过手淫练习来掌握和熟悉在性唤起之中的感受,并且对射精有控制能力。通过4~5次成功的训练,配偶就可以将阴茎插入阴道。当男性放松PC肌的时候,最好采用女上位,要教会女性使自己的身体慢慢地起伏,以便使阴茎能够有抽动的感觉。当男性有要射精的感觉时,他要向女性发出停止活动的信号而自己依然尽情地感受阴茎的感觉,直到射精的欲望减退为止。这样的练习在射精之前要多次练习。当这种水平刺激的控制能力已经达到之后,配偶就可以练习阴道内插入的抽动练习。性治疗师要鼓励配偶每周坚持练习动-停方法,尝试使用不同的性交体位,直到完全能够在性交中控制射精为止。

性交采取男性仰卧位和女方半坐式上位,对患者阴茎进行按摩刺激或性交,当患者有临近射精的感觉时停止刺激或性交,女性用拇指和其他指头夹住阴茎的龟头部,拇指压在阴茎的系带处,时间为3~4秒,半分钟后射精反射消失;或用拇指和食指夹住阴茎的冠状沟处,力量的大小要适中,能够引起轻度的疼痛,以达到男性的射精反射自动消退为标准。然后再继续进行性交或挤压法练习,每天练习4~5次,可以提高射精阈值,达到治疗早泄的目的。这种方法及其演变的其他方法都被认为是有效的,特别是将其与其他控制射精的方法结合起来的效果更为突出。

这是一种通过改变性唤起水平来调节性唤起(刺激控制)和控制射精的一种认知或思维配合的技巧。有了性唤起连续谱技巧,PE的男性就会系统地观察、思考、区别那些形成他个人性唤起模式的思维细节(性幻想)、动作、感受、情节和结果。通过确定那些对性唤起作用强弱的具体项目,他就会以一种渐进的方式将这些项目排列起来,形成一个连续谱。如同光谱一样,连续谱是一个逐渐过渡的过程。然后在做爱的过程中,他就能够按照自己的需要来和谐地调节自己的性唤起水平。他们之间即使已经发生了实际的性行为,但仍然可以通过严格控制注意力来调节刺激的水平。

“自我关注的性唤起”是在经典的性治疗中对于性伴侣传统的感觉集中训练,是使用强化在个体治疗阶段中已经建立起来的自我关注的性唤起。自我关注的性唤起需要高度集中的注意力,因为当爱人就在眼前时,他很容易就又利用原来的、通过性爱对象的身体形象的性唤起模式。家庭作业的任务就是要让一对伴侣互相爱抚并且让男性能够在女性面前做到完全的生理放松。然后男性仰卧,女性抚摩他的阴茎,他全神贯注地体验被抚摩的感觉。通过练习身体放松的性唤起来做到有勃起又处于平静的性唤起状态。这就是治疗所要达到的目的。

在阴道适应技巧里面,当女性将自己的阴道套入男性的阴茎时,男性将注意力集中在射精肌肉PC肌放松方面并且轻松地躺着。他要耐心等待,期待达到一种生理性的快感饱和状态———阴茎适应阴道的温暖和对阴茎的压力和摩擦。对于大多数男性来说,这种阴道的适应开始于插入阴道后10分钟左右。他可以适当进行小幅度的抽动以保持阴茎的充分勃起,也可以指挥女性适当地增加运动的幅度。最长可以达到30分钟的适应训练。在适应训练完成之后,男性的阴茎可以经历更加强烈的刺激,他可以体验在阴道内的抽动而保持延长的射精状态和快感体验。这样不仅使他增加了性交的时间,并且通过性唤起模式的转变而提高了性生活的质量。

性治疗是由马斯特思和约翰逊(1970年)创造的,是一种心理行为疗法。性治疗以夫妇或性关系为对象,放映了对传统的心理治疗中重视个别患者倾向的反叛。

(1)性治疗的目的大致可以定义为:①帮助个体接受自身的性活动,并从中得到满足;②帮助夫妻改善性关系。

(2)性治疗的原则为:①给患者规定适当和明确的任务,要求患者在下一次治疗前完成;②详细考察任务完成的情况和遇到的困难;③确认使任务难以完成的态度、情感和阻力;④改变上述的态度、情感、阻力等,使任务能够完成;⑤制定下一步的任务(或称家庭作业),然后重新经历上述各步骤。

治疗中夫妻双方都有责任,而治疗中也包含了性教育的成分。

(3)具体的治疗分六个阶段

1)第一阶段:为了自己的快乐抚摸配偶,但不能触摸生殖器。

夫妻大约利用1周的时间来完成这个阶段的练习。这个练习的主要目的是通过抚摸对方来让自己获得性快感。通常情况下,人们都需要接受抚摸才感到愉快,但抚摸别人同样可以得到愉快的感受。治疗师在布置家庭作业的时候可以让夫妻一起做一个用右手掌心抚摸左手掌背的练习,在抚摸之后问他们夫妇有什么样的感受,多数夫妇是谈论掌背被抚摸的感觉,这时治疗师要询问右手的掌心有什么感觉,他们可能已经忽视了右手还有什么感觉。治疗师要求他们重新做这个练习。他们这次将注意力集中在右手掌心,其实也得到非常愉快的感觉,只是我们没有注意这个问题,好像抚摸别人就是给予而没有收获,其实在给予的过程中一样也有收获。让夫妻回家后制定一个练习计划,每天练习一次(特殊情况除外),一次40~60分钟,事先约定不能有口角或怨言,断绝对外联系(如关闭手机或电话),避免一切干扰。由丈夫抚摸妻子10分钟或15分钟,然后交换,在抚摸的过程中要求赤身裸体。这个阶段严禁对双方的生殖器和乳房进行抚摸,在抚摸过程中,如果有性唤起也要尽量按治疗计划的要求加以控制。该练习其实就是前戏过程,目的就是通过互相抚摸来获取愉快的感受,从而能够在配偶面前减少紧张和要做爱的压力(操作性焦虑),达到一种从容不迫的状态。

2)第二阶段:为了自己和配偶的快乐触摸对方,不触及生殖器。

这个阶段的练习基本要求与第一个阶段一致,但目的不但是为自己的快乐而且为了对方的快乐而进行抚摸。这个阶段可以互相抚摸,但仍然不能触及生殖器。如果进展顺利可以进行乳房的抚摸,人们往往认为只有女性的乳房才有被抚摸的快感,其实男性的乳头一样也有被抚摸的快感。在这个阶段除了用传统的双手之外,还可以用吻,舔或乳房摩擦等方法。在练习中男女之间要有言语的交流,尤其是关于抚摸和被抚摸的感觉,要给予指导,鼓励正确或感觉舒适的抚摸方法,改进不能产生愉快感觉的抚摸方法。练习中很可能会造成男性的生殖器勃起或女性阴道的润滑,也就是产生了性唤起,但仍然要遵守事先的治疗约定,不能进行生殖器抚摸,更不能尝试性交,只有在非性交目标的练习中才能够减轻操作性焦虑,以便达到治疗早泄的目的。夫妻双方要在日常生活中和性治疗练习过程中不断地互相鼓励。

3)第三阶段:包括生殖器在内的触摸。

这个阶段已经是标准的前戏过程了,双方可以进行包括生殖器在内的抚摸以及除性交外的各种各样的性刺激。在这个阶段的练习中,夫妻首先还是应该重复以往做的非生殖器触摸的练习,也就是说将所有的抚摸有机地联接起来,前戏应该有一个循序渐进的过程,从低级性感部位开始逐步向高级性感部位过渡,最后才是男女的生殖器。男性对于女性的生殖器刺激应该从大阴唇、小阴唇逐步向阴蒂过渡,抚摸过程中应该利用女性分泌出的阴道液体,这样可以减少摩擦带来的不适而产生快感。男性应该通过单纯抚摸女性的生殖器而让女性产生性高潮。能够接受和给予口与生殖器接触的夫妇,这时可以使用口和舌刺激女性的生殖器,但是女性不能对男性口交。在女性对男性的生殖器抚摸过程的练习中,男性应该给予指导,并且注意是否能够引起强烈的性兴奋和有临近射精的感觉。如果有临近射精的感觉,就要停止刺激和抚摸,让射精感觉降低后再重复练习。在射精反射消失后可能会产生阴茎的疲软,这没有任何关系,男性就是要通过练习达到反复勃起而不射精的能力,这样才能够克服早泄的毛病。万一兴奋过度发生射精也不要紧张,休息后继续练习,这并不意味着治疗的失败,是治疗中经常经历的情况。

4)第四阶段:相互同时触摸对方包括生殖器。

通过第三阶段的练习,夫妻双方已经能够控制射精和获得抚摸的快感,男性也能够让女性通过抚摸而达到性高潮,那么双方就可以进行互相抚摸,包括双方的生殖器在内。这一阶段夫妻应该将练习的重点逐步转移到生殖器的抚摸与感受上。女性对男性生殖器的抚摸应该集中在冠状沟和阴茎系带等部位,男性的阴囊和会阴部也是性敏感区。其实就是女性要学会为男性手淫,因为这样才能让男性精确地感受射精临界点,学会更好地控制射精。不但为了治疗早泄,而且在今后不适合性交的时候,也可以帮助男性达到性高潮。

5)第五阶段:阴道容纳阴茎(女上位,男女双方无骨盆前倾动作,体会感觉)。

该阶段的练习实质上就是无抽动的性交,因为很多男性在阴茎没有抽动的情况下是不会射精的,但在抽动时产生强烈的摩擦就会容易射精。这种插入后停止不动的状态一方面可以减轻由于摩擦引起的射精紧迫感,另一方面也可以让男性体验被阴道包容的感受。男性虽然可以保持阴茎插入后的不动状态,但仍然可能有强烈的性兴奋,或害怕产生射精的紧张情绪,练习的主要目的就是将关注射精的紧迫感转移到体验阴道的包容感受。也可以在阴茎插入后与伴侣进行言语交流。女性在这个练习中可能会产生挫败感,因为在强烈的性兴奋之下,最后的结果是男性的抽动伴有女性的性高潮,但不能忘记这是治疗阶段,为了能够治愈早泄,只有克服眼前的困难,放弃性高潮的要求,而集中精力去体验单纯的插入所带来的快感,而且应该尽量用言语安慰和鼓励伴侣,让他感到轻松自在。如果女性希望达到性高潮,可以通过抚摸进行而不是性交。如果女性有不可克制的性兴奋而产生阴道收缩,男性可以任其自然,尽量延长阴茎插入阴道的时间。男性在插入后如果产生射精临近感,就可以将阴茎撤出阴道,等射精反射消退后再次插入,当阴茎撤出时,可以用抚摸替代插入,以减少性交中断为女性带来的不适感。由于早泄问题的主要原因在于男性,而在治疗过程中女性的配合却极为重要,所以男性应该在情感方面给予支持和奖励,包括给予礼物和关爱。这样可以缓冲在性治疗过程中情感抑制所带来的伤害,同时也为进一步的治疗奠定了基础。

6)第六阶段:正式性交。

这个阶段是前五个阶段顺利通过后的最后练习,也是对性治疗结果的最后检验。在这个练习中,夫妻仍然要相互鼓励,积极地看待在以往练习中所取得的成绩,不要将注意力集中在到底能够性交多长时间,而应该将注意力集中在愉快的感受和夫妻情感的改善方面。在具体的插入过程中,应该选用男性认为比较容易控制射精感受的性交体位。有些男性面对女性更加容易产生性兴奋,那么就可以采用面对女性背部的插入方式;女性也可以仰卧,男性跪或站立,这样的性交方式男性比较容易控制射精;也可以采用不断更换性交体位的办法来延缓射精。总之,该阶段的练习要夫妻增加交流,尽量减少性交时间本身为夫妻带来的压力。即使在练习中射精也不要紧张,在阴茎疲软前不要忙于将阴茎撤出阴道。如果在练习中使用了安全套,那么最好另外使用一个带有小刺的胶圈(也是促进性快感的小性用品),这样阴茎在射精后不会马上疲软,也不会让精液溢出安全套之外。在情绪方面能够让男性更加自信,也能够让女性体验到更长时间的插入感。

以上六个阶段的练习大致是每个阶段一周,但如果哪一个阶段出现问题或者不能完成,那么就重复这个阶段的练习,直到能够完成任务为止,所以治疗的整个阶段需要6~8周。治疗师每周应该对夫妻进行指导,布置治疗的家庭作业,并且每周对于夫妻的练习给予评估。如果一次性将所有的练习作业交给夫妇,两个月后做疗效评定,那么治疗的效果多数不会理想。因为夫妻要面对太多的压力,而又缺乏及时的指导,于是对过多的作业感到力不从心,产生畏难情绪,很可能回家之后根本就放弃治疗练习。性感集中训练的目的虽然是治疗早泄,但治疗过程中的真正目的是调节夫妻关系,也就是要得到女性的配合、宽容以及鼓励,让女性分担男性早泄的责任和压力,所以对于女性的心理治疗更加重要,必要时可以单独进行。治疗前最好能够签订一个知情同意的治疗协议,对于治疗任务以及基本要求给予充分考虑,让夫妻能够进一步达成治疗共识。

早泄患者在性生活过程中也有一些自己的经验积累,一些甚至成为医生指导患者克服早泄的方法。虽然未必具有突出疗效,但是仍然有一些治疗作用,对于不同类型的早泄疗效也不尽相同。主要有以下几种:

分散注意力的目的是降低性兴奋,减少主观的和客观的性刺激因素,例如,患者在黑暗的环境中做爱,减少直视配偶,主动思考与性生活无关的内容,甚至谈论无关的话题等。

早泄者的性生活往往不能让配偶获得性高潮,而且性生活草草了事更是令人尴尬,无论是男性还是女性,采用幽默的方式来缓解尴尬的局面都是明智之举。

在做爱过程中有人喜欢自己心中数数字,这其实是一种转移注意力的方式,主要是避免将注意力集中到阴茎的感觉上。

这种方式的目的仍然是减少阴道对于阴茎的刺激,也就是在男性感到射精潜伏期临近时主动中断性交,等待射精反射消失后再进行性交,但这种情况往往让女性感到失望或难受。有些患者在临近射精的时候改变性交体位,可以达到同样的目的,更能够掩饰真正的目的而且让配偶也可以接受。

有些男性能够在性生活中逐步认识射精潜伏期的规律以及自己对射精控制力的感觉,这样就可以准确分清什么时候要射精,能够经过主动控制而延长射精时间。有人从性交开始就保持阴茎肌肉的紧缩状态,以为这样能够控制射精,但却事与愿违,而做到完全放松的状态才能够保持更长时间的勃起,只是在真正要射精的时候才保持高度的阴茎肌肉收缩状态。

对于早泄这个问题,越是回避内心压力就越大。性生活是两个人的事情,无法真正回避和掩饰,最好的办法是配偶之间进行交流,如果女性能够安慰男性或者主动配合男性来改善或治疗早泄,效果肯定是非常理想的。女性的宽容和主动帮助是最好的治疗药物。

性兴奋过程中阴茎充分勃起,此时的睾丸上提贴近阴茎,如果向会阴后部位牵拉睾丸会产生延迟射精的作用。轻轻地拍打睾丸或者向下牵拉甚至能够让一些男性不射精。其机制除了与性兴奋时睾丸的位置相反有关外,轻微的疼痛和注意力的转移也起到协同作用。

背式站立性交和站立式性交能够让一些男性对射精具有一定的控制力,背入式性交能够让部分男士延长性交,也许是因为减少了视觉刺激和性伴侣表情带来的压力。

安全套是一种屏障,通过乳胶隔断了阴茎与阴道的直接接触,这样会减弱男性的性兴奋。大多数人会通过安全套而降低阴茎的敏感性,但并不是所有人都会因安全套而降低兴奋性。目前一些安全套内使用了麻醉剂或药物乳膏,具有明显的延迟射精的作用。

耻骨尾骨肌对于控制射精具有一定的作用,男性如果耻骨尾骨肌无力就会失去对射精的控制,为此要加强该肌肉的锻炼。具体方法是在排尿时主动中途停止,然后再继续排尿。这种中断排尿的方法即使用了耻骨尾骨肌。自己也可以像跳芭蕾舞一样脚尖站起,用力将双腿和臀部以及排尿的肌肉同时紧缩,一次可以跳十几次,每天锻炼三到五次,慢慢就会提高对射精的控制能力。

男性在性生理上有一个特殊的不应期,也就是性交过后再即使有新的性刺激其兴奋程度也会降低。男性一般在短时间内重复性交的能力往往不足,而且精液量一次比一次明显减少,而射精潜伏期不断延长。正是因为这样一种生理机制,有些轻微的早泄患者可以通过重复性交而达到延长性交时间的目的。多数人第二次性交会比第一次射精潜伏期明显延长。

针灸能够治疗早泄的原理还不明确,但针灸肯定是有效的方法之一。麻醉作用、暗示作用以及局部神经的阻断作用都可能是针灸治疗早泄的有效机制。

有些国家开展部分阴茎背神经切断术来治疗早泄,韩国尤为突出。但过一段时间早泄又恢复,长期疗效有待观察。

1.对于有PE同时又患有ED的患者,可使用万艾可。

2.外用软膏 主要含有麻醉剂,表面麻醉剂如1\%的达克罗宁冷霜或3\%的氨基苯甲酸乙脂,或25\%的苯唑卡因可用于治疗早泄;也有学者用中药制成外用药软膏来治疗早泄,比如,韩国的SS ICECREAM等。前列腺素E1阴茎海面体注射或尿道口滴注。有些患者的PE得到缓解,因为他们对ED的焦虑和恐惧已经通过治疗而减轻,所以对PE也产生了良好的影响。

3.外用中药。

4.口服中药 很多中药或中成药都能够治疗早泄,比如锁阳固精丸等。但最能够治疗早泄的中药是罂粟,以往中药中可以加入少量罂粟。罂粟的主要成分是阿片,阿片进入体内就转变成吗啡,吗啡具有强烈的延迟射精的作用,但由于其成瘾性而不能广泛应用于临床。乙醇具有一定延迟射精的作用。

5.阿片类药物抑制射精作用完美,致命缺点是成瘾性,对于性欲没有影响,口服无任何不适。SSRIs类药物抑制射精作用中等,有一定副作用,大剂量影响性欲,最大优点是没有成瘾性。目前已经有厂家研制了一种治疗早泄的新药,称为LI301,SSRI和阿片样(opiate-like dulling)药物的混合剂。患者在性生活前2小时服用药物。如果这种药物能够成功,早泄的药物市场是29\%,而ED的市场是10\%。早泄药丸的潜在市场可能比万艾可还要好。

负压吸引是用来治疗勃起功能障碍的一种治疗方式,有在医院使用的大型仪器和个人家庭使用的袖珍型设备,对于勃起功能障碍有肯定的疗效,但在万艾可问世后使用率明显减少。对其早泄治疗机制的深入研究较少,但有一定的治疗作用,除了暗示性之外,反复的勃起练习起到循序渐进的系统脱敏作用。

新型的抗抑郁剂(五羟色胺回收阻断剂SSRIs)适用于治疗各种类型的早泄,但对于紧张等心理因素引起的早泄则可以在短期内彻底治愈,对于长期的难治性早泄,服药后肯定有效,但停药后50\%又恢复常态。在目前治疗早泄的方法中,SSRIs已经基本达到临床治疗的目的。

舍曲林最大量可以达到200mg/d(平均141mg/d)。帕罗西叮剂量可以是40mg/d,但20mg/d也能够起到明显的作用。氟西叮20mg/d,以后可以每两天服20mg。

抗抑郁剂对射精的影响包括射精延迟、痛性射精和逆行射精。1973年Beaumont就报告了氯丙咪嗪能引起射精延迟和射精不能。Mendels等认为SSRIs能够延长射精潜伏期的机制可能是促进了五羟色胺神经递质的转运。

很明显,目前对抗抑郁剂能够延长射精潜伏期的作用机制还没有完全弄清楚。中枢的五羟色胺能通路和周围的α-肾上腺素能系统都可能包括在内,而在周围神经系统中的胆碱能系统也可能有一定的作用。

抗抑郁剂绝不是唯一能够延长射精的药物。抗精神病药、抗焦虑药、肌肉松弛剂、锂盐和其他药物也有过能够延长射精的报告。但是,考虑到药物副作用,药物滥用的可能性和能够延长射精潜伏期的连续不断的报告,抗抑郁剂是最适合也是最容易耐受的治疗早泄的药物。

陶林等用氯丙咪嗪治疗50例早泄患者,其中有10例无原因脱落,剩余的40例中39例明显有效,射精潜伏期平均延长6.4分钟,约7天起效,平均剂量为30mg/d。副作用很小,患者容易耐受。

Angel等编写的性问题调查问卷,对344名门诊患者(192例女性,152例男性)进行了调查。结果发现与药物有关的性功能障碍副作用占58.14\%,这明显高于患者自报的14.2\%。引起不良性功能反应的药物由高到低的顺序是帕罗西叮(64.71\%),氟伏草胺(58.94\%),舍曲林(56.4\%)和氟西叮(54.38\%)。这个研究进一步表明抗抑郁剂在治疗早泄方面是大有可为的,同时从反面看,对抗抑郁剂引起副作用的研究是治疗早泄的潜力研究,当然对药物引起的副作用也应该引起重视,这可能是影响抗抑郁剂治疗早泄依从性的原因。

近年来,世界各国的医生们通过大量的临床研究证明了这些药物确实可以治疗早泄,并且达到比较满意的疗效。这类药物的共同的特点为:①使用方便(口服,1日1次或3次);②价格适中;③无明显的副作用,比治疗焦虑症需要的剂量低;④长期服药无成瘾性;⑤可以随时停药;⑥起效快(2~4天);⑦不需要女性配合,可以增强男性的自信心;⑧对性生活频率不高的夫妻,可以间断服药;⑨如果可以将SSRIs与行为治疗方法合用,能够达到巩固疗效的作用。

SSRIs适用于治疗各种类型的早泄,由紧张焦虑等心理因素引起的早泄则可以在短期内彻底治愈,对于素质型早泄服药时有效,停药后又逐步恢复原来的早泄状态。在目前治疗早泄的方法中,SSRIs已经基本能够达到治疗的目的。

笔者用帕罗西汀治疗符合DSM-Ⅳ早泄诊断标准的患者,常规治疗量20mg/d,治疗前平均射精潜伏期为1.67±0.85秒,治疗后为11.73±7分钟,平均延长了10.13分钟,起效时间为6.88±3.42天。国外学者Waldinger M等使用帕罗西汀和安慰剂对照治疗早泄患者,结果是治疗三周时射精潜伏期延长为7.5分钟,六周时延长为10分钟,起效时间为一周,这与我们的研究结果很相近。

抗抑郁剂治疗早泄的机制是利用抗抑郁剂引起射精延迟的药物副作用,但与此同时会不会出现其他性功能方面的药物副作用,如性高潮和性欲损害呢?我们的研究对此得出结论:大多数患者的性欲会有所升高,只有个别人性欲会降低,而对射精时性高潮的体验没有什么影响,很多患者还会觉得提高了性高潮的感受,这可能是以往射精过快而体验不明显的结果。帕罗西汀起效的时间是一周左右,这与其抗抑郁作用的起效时间1~2周很接近,这也表明帕罗西汀治疗早泄的作用是通过中枢神经系统,也支持早泄的关键部位是在大脑的观点。我们的研究方案中包括性生活次数在内,这样就避免了因为没有过性生活而无法评价疗效的情况,多数患者每周有一两次性生活。

Balon L认为射精潜伏期、性交满意度和配偶的性交满意度是三个重要的抗抑郁剂治疗早泄疗效的评价指标。而在临床治疗早泄患者的时候,经常要回答的也有三个问题:一是抗抑郁剂治疗剂量多少合适?抗抑郁剂治疗早泄的药物剂量一般低于治疗焦虑症和抑郁症的药物剂量,所以一般采用半量或常规治疗剂量,比较安全有效的方式是从小剂量逐渐增加,不但可以摸索出适宜剂量,还可以减少药物副作用。在不同的研究中,氯丙咪嗪的剂量通常达到25~50mg/d;舍曲林最高可以达到200mg/d(平均141mg/d);帕罗西叮剂量可以是40mg/d,但20mg/d也能够起到明显的作用;氟西叮是20mg/d,以后可以每两天20mg。二是有什么药物副作用?目前使用的抗抑郁剂都比较安全,而且副作用较小,患者对于药物副作用容易耐受,不需要特殊处理,一周左右就可以自然消失或耐受了。医生不要夸大药物的副作用,应当指出治疗早泄的作用与副作用是同时出现的,开始出现药物副作用就等于对早泄的治疗产生了作用,这样能够暗示药物副作用与早泄治疗作用之间的关系,患者也将出现的药物副作用当成治疗早泄有效性的指标,心情由恐惧转为喜悦,对于治疗很有帮助。抗抑郁剂的治疗剂量与药物副作用成正比,剂量越大副作用自然越大,治疗早泄的疗效当然也越好,因为治疗早泄就是利用了抗抑郁剂的药物副作用,没有任何副作用疗效也就不会明显。抗抑郁剂副作用相对少,而且与剂量呈明显的正相关。使用氯丙咪嗪的副作用包括口干、便秘、疲倦、眩晕、恶心、困倦、腹泻和偶然出现的性功能障碍;舍曲林的副作用有腹泻、口干、疲倦和偶然出现的性功能障碍;帕罗西叮的副作用包括疲倦和哈欠;氟西叮会引起焦虑并且改善性功能。三是治疗时间应该多久。抗抑郁剂治疗早泄需要多长时间呢?这是一个复杂的问题。有些患者服用短暂的药物后,早泄彻底治愈了,他们自然无须继续用药。有些患者在停药后射精潜伏期又在很短的时间内缩短,直至恢复原来的水平,这说明要达到长期延长射精的目的需要持续用药。考虑到药物的耐受性和副作用等,最好能够在服用药物一两个月后,停用一到两周,这样可以保持疗效。在没有更令人满意的治疗早泄的方法问世之前,尽管抗抑郁剂治疗早泄仍然存在一定的问题,如药物副作用和起效较慢等,但它仍然是最简便和最有效的治疗方法。多数新型抗抑郁剂没有镇静作用,头晕等症状也许会出现,开始服药的时候最好不要开汽车或从事高空作业等具有一定危险性的活动。

很多男性认为早泄是一个严重的问题,他们常将早泄的原因归咎于自己的手淫,并为自己的“过错行为”感到极度的懊悔。很多医生的看法是,既然患者自己都说是手淫引起的,那也没有必要反对,但真实的情况到底如何呢?经过研究发现,6\%的早泄者从来没有手淫,25\%很少手淫,即使是有手淫的早泄者,他们开始手淫的年龄也比普通人晚三年左右。我们还了解到,男性中一般有90\%以上的人都有过手淫,但很多有过手淫的人并没有早泄。一些手淫者在性交时感到射精困难,无论时间长短都达不到射精的阈值,但通过手淫却可以射精,显然手淫过程的刺激强度和部位的准确性都胜过性交本身。西方广泛流行的治疗早泄的方法其实就是手淫,由此可以得出结论:不能轻易认为手淫可引起早泄。事实上,现代很多性学家也都认为,手淫不但不会引起早泄,而且还能够预防和治疗早泄。当然,部分有严重手淫的人在看到手淫可以引起早泄的书籍之后,就产生了自己一定会罹患早泄的忧虑,这种忧虑在不能戒除手淫的情况下得到了强化。在第一次性生活的时候,这种担忧和性兴奋都达到极点,这样就产生了第一次早泄,这种本来主要是性兴奋导致的早泄更加证实了患者原来的担忧,他完全接受了手淫导致早泄的事实,只对自己过去的手淫行为感到懊悔和无奈,这样就形成了后来早泄的恶性循环。有学者认为青少年手淫时担心被发现或者希望快速达到性高潮而形成快速射精的毛病就是早泄的原因,但这不能解释大多数早泄患者的状况,但改善精神紧张和快速射精的习惯有助于改善早泄。

早泄能不能去根,关键要看患了哪一种类型的早泄,是否伴随性欲性低下、勃起障碍等。根据早泄的分类,我们可以判断哪些是能够彻底治愈的,而哪些是不能完全治愈的。与性生活技巧缺陷有关的早泄,新婚夫妇因为紧张而发生的早泄,随着性技巧的熟练掌握或经验的增加,对射精潜伏期单方面的不满足会逐渐减少,也就比较容易治疗。对于继发性早泄,只要治疗好原发病就能够去根,比如前列腺炎后出现早泄,只要前列腺炎治好了,性功能也就会随之恢复,而原发性早泄的治疗比较困难。很多学者一致认为有部分人的早泄是天生的,即素质型早泄,就好像人个子的高矮一样,这样的早泄并不是什么病,当然也不存在彻底治疗的问题。不了解所患早泄的类型,其根治问题也就无从谈起。

很多早泄者认为是因为自己太敏感才发生早泄,于是在性生活时尽量避免前戏或抚摩过程,直接进入插入阶段,结果不但让女性感到不满,其结果是仍然发生早泄。其实,在性兴奋程度过高、神经敏感和精神紧张的情况下很容易发生早泄,但越是减少性接触,越是避免前戏,碰到女性身体时性兴奋就越强烈,也就越容易出现早泄。目前西方治疗早泄主要的行为疗法就是采用了增加抚摸练习,以减少夫妻前戏时的性兴奋程度来达到治疗早泄的目的。所以治疗早泄反而应该增加性接触。

事实上,早泄与其他疾病不同,是一种相对的疾病,与女性紧密相关。笔者的研究发现,男性平均的射精潜伏期大约是12分钟,而女性希望男性的射精潜伏期约为11分钟,差别并不明显,但有34\%的女性希望男性延长射精潜伏期,这就出现了错配问题。也就是说,射精潜伏期短的男性与性高潮潜伏期长的女性结合,结果导致早泄问题的发生。所以说,早泄不仅是男性的问题,也是女性的问题。更科学地说:是男性的射精潜伏期与女性的性高潮潜伏期之间错误搭配的问题。当然,结婚不能只考虑性生活,除了努力延长男性的射精潜伏期外,缩短女性的性高潮潜伏期也等于延长了男性的射精潜伏期,所以,早泄的治疗不仅是男性的问题,女性也有促进和谐性生活的义务。

其实,我们从第一个误解中可以得到信息,早泄与女性也有关,治疗早泄应该是男女双方的事。更重要的是,早泄是在性生活中发生的,而性生活又是夫妻双方共同进行的,所以治疗早泄要夫妻配合。目前比较流行的行为治疗方法要求夫妻共同完成治疗早泄的任务,妻子的理解和配合能够减少丈夫的紧张和焦虑,这是治疗任务的一部分,如果妻子采取冷淡或歧视态度,甚至在言语和行为方面给男性施加压力,那么早泄就会进一步加重。

多数早泄并不影响生育,很多人也没有将早泄当做什么病,甚至一些社会学者和民间人士也认为早泄不是病,慢慢就会好,尤其鼓吹早泄到了老年自然会慢慢缓解,不需要治疗。或者为了维护丈夫的权威和面子,女性从不表示对性生活的不满,而男性则若无其事。这样的情况更加糟糕,对任何类型的早泄患者来说,都应该提早治疗,而且越早治疗效果越好。因为早泄会形成条件反射,越怕早泄就越紧张,越紧张就越容易出现早泄,夫妻对性生活质量的不满也加重各自的心理负担,这样就慢慢地形成了恶性循环,对早泄治疗没有益处。


\section{第六节 早泄诊断工具的研究}

请根据您过去6个月的性生活实际情况回答下列问题,选择适当的编号标记(√)

1.您平时产生性欲望或性兴趣的频度如何?

A.几乎没有 B.少数几次 C.约一半左右

D.多数时候 E.几乎总是

2.性生活时阴茎勃起硬度足以插入阴道的频度如何?

A.几乎没有 B.少数几次 C.约一半左右

D.多数时候 E.几乎总是

3.性生活时,能够维持阴茎勃起直到完成性生活的频度如何?

A.几乎没有 B.少数几次 C.约一半左右

D.多数时候 E.几乎总是

4.性生活时,从阴茎插入阴道直到射精的时间有多久?

A.极短(<30s) B.很短(1分钟) C.短(2分钟)

D.比较短(3分钟) E.不短(>3分钟) F.4~5分钟

G.6~10分钟 H.11~20分钟 I.21~30分钟

J.>45分钟

5.性生活时,您试图延长性交时间的困难程度如何?

A.很困难 B.困难 C.有些困难

D.一般 E.没有困难

6.总体而言,您对性生活的满意程度如何?

A.很不满意 B.不满意 C.一般

D.满意 E.非常满意

7.总体而言,您的配偶对性生活的满意程度如何?

A.很不满意 B.不满意 C.一般

D.满意 E.非常满意

8.性生活时,您的配偶达到性高潮的频度如何?

A.几乎没有 B.少数几次 C.约一半左右

D.多数时候 E.几乎总是

9.您对圆满地完成性生活的自信程度如何?

A.很低 B.低 C.一般

D.自信 E.很自信

10.性生活时,有多少次感到焦虑、紧张或不安感?

A.几乎总是 B.多数时候 C.一般

D.少数几次 E.几乎没有

Tara Symonds,Michael.A.Perelman,Stanley Althof等认为大多数PE的诊断都是依赖射精潜伏期(IELT),但单纯的时间维度(因素)不能囊括所有PE诊断应该包含的多个维度(因素),所以,他们力图开发一个简便的、多维度的、心理测量有效的PE诊断工具。研究人员将研究对象分为三组,一组研究对象的IELT小于或等于2分钟(人数是292人),按照DSM-Ⅳ-TR的临床诊断,临床医生确诊他们为PE患者;第二组是自我报告是PE的人(309);第三组是自我报告没有PE的人(701)。按照标准的心理测量分析结果,用5个项目囊括了DSM-Ⅳ-TR的核心内容,即射精控制力(control);频度(frequency);较小的刺激(minimal stimulation);痛苦(distress);人际关系困难(interpersonal difficulty)。临床诊断最终开发出PE的诊断问卷。

该PE评估问卷的优点是简便和实用。完全可以用于任何PE研究或治疗的筛查工具。由于是在参考了DSM-Ⅳ-TR诊断标准之后编写和修订的版本,所以可以广泛应用,而且也完全可以和即将问世的DSM-Ⅴ接轨,或者作为先期研究的资料。

PE评估问卷

这是一个帮助男性确认在性生活中有无射精过快问题的问卷。即使你没有这方面的困扰,也请你完成问卷。请注意以下填写问卷的要求:

1.请将每个问题下最能代表你现状的答案涂黑;

2.每个问题你只能涂黑一个答案;

3.请注意这里的问题没有正确或错误的区别;

4.如果你在不同的时候对这些感受有所不同,那么我们这里需要的是你总体上的主要感受。

射精的定义:这里指的射精是当你的阴茎已经进入你性伴侣阴道之后的精液射出。

问题:

1.延长射精时间对你来说困难吗?

How difficult is it for you to delay ejaculation?

答案:根本不困难 有些困难 中等困难 很困难 极度困难

(0) (1) (2) (3) (4)

2.你射精的时间早于你自己的意愿吗?

Do you ejaculate before you want to?

答案:没有或几乎没有 少于一半时间 大约一半时间 多于一半时间 总是或几乎总是

(0) (1) (2) (3) (4)

3.在很小的刺激下你就能射精吗?

Do you ejaculate with very little stimulation?

答案:没有或几乎没有 少于一半时间 大约一半时间 多于一半时间 总是或几乎总是

(0) (1) (2) (3) (4)

4.因为射精时间早于你的愿望,你会有挫败感吗?

Do you feel frustrated because of ejaculation before you want to?

答案:根本没有 有点 中等 严重 极度严重

(0) (1) (2) (3) (4)

5.你过早的射精时间让你的性伴侣感到意犹未尽,你感到担忧吗?

How concerned are you that your time to ejaculation leaves your partner sexually unfulfilled?

答案:根本没有 有点 中等 严重 极度严重

(0) (1) (2) (3) (4)

评分方法:总共有5个问题,每个问题得分是从0分到4分,所有问题回答后分数最少是0分,最多是20分。按照信度和效度分析和研究的结果,小于或等于8分可以排除PE,9或10分可能是PE,等于或大于11分可以诊断PE。

1.服药后出现性功能障碍吗?(是 否)

2.是否由患者主动提出?(是 否)

3.性欲是否减退?(0=没有 1=轻度 2=严重)

4.射精或性高潮是否延迟?(0=没有 1=偶尔有 2=中度 3=严重)

5.性高潮是否缺失?(0=没有 1=偶尔有 2=经常有 3=一直有)

6.是否有阴茎勃起困难?(0=没有 1=偶尔有 2=经常有 3=一直有)

7.服药之前的性功能状况:(正常 不正常)

8.随着服药时间的延长,性功能障碍是否自然缓解?(0=完全 1=中度 2=没有缓解)

9.患者本人对性功能障碍的忍耐性:(0=好 1=很好 2=不好)

10.最近药物剂量是否有减少?(是 否) 新的剂量:( )mg/d

11.减药后性功能障碍是否缓解?

(0=完全缓解 1=明显缓解 2=有所缓解 3=略有缓解 4=没有缓解)

12.是否改换药物?(是 否) 新药名称:( ) 剂量:( )mg/d

13.换药后性功能障碍缓解程度如何?(0=完全 1=明显 2=有所 3=略有 4=没有)

14.换药后多长时间性功能障碍得到缓解:( 天)

15.是否因性生活而短暂停药?(是 否) 停药间隔时间为( )小时

16.停药后性功能障碍缓解程度如何?(0=完全 1=明显 2=有所 3=略有 4=没有)

17.是否因服药使原来的性功能障碍好转,如早泄等?(是 否)

18.患者配偶是否发现在服药后患者的性功能有障碍?(是 否)

19.患者配偶是否要求患者治疗性功能有障碍?(是 否)

20.患者认为与性功能有障碍有关的问题请详述。

1.张滨主编.性医学.广东教育出版社.2008

2.世界卫生组织.ICD-10精神与行为障碍分类.范肖冬,汪向东,于欣,刘平,译.人民卫生出版社,1993

3.刘继红,熊承良.性功能障碍学.北京:中国医药科技出版社,2004

4.陶林,王春华.PE的诊断和分型.中国男科学杂志,2002,2:142-143

5.Balon R.Antidepressants in the Treatment of Premature Ejaculation.Journal of Sex &Marital Therapy,1996,22(2):85

6.郭应禄.阴茎勃起功能障碍.北京:北京医科大学出版社,1999:122

7.马晓年.现代性医学.第2版.北京:人民军医出版社,2004

8.LoPiccolo J PH.D.Premature Ejaculation and Male Orgasmic Disorder.1994—1997,American Psychiatric Press,Inc

9.merican Psychiatric Association.Diagnostic Criteria from DSM-ⅤI.1994:P236-237

10.Edward O.Laumann,John H.Gagnon,Robert T.Michael,and Stuart Michaels.The social organization of Sexuality.Chicago:The University of Chicago Press,1994:368-375

11.American Psychiatric Association.The Diagnostic and Statistical Manual(4th ).Washington,DC:Author,1994

12.Diagnostic and statistical manual of mental disorders,fourth edition,text revision:DSM-Ⅳ-TR.Washington(DC):American Psychiatric Association,2000:554

13.Tara Symonds,Michael A.Perelman,Stanley Althof et al.Development and Validation of a Premature Ejaculation Diagnostic Tool.European Association of Urology,52(2007):565-573

14.Metz M,Pryor J,Nesvacil L,et al.Premature Ejaculation:A Psychophysiological Review.J of Sex &Marital Therapy,1997,23(1):3

15.Waldinger M,Hengeveld M,Zwinderman A,et al.Effect of SSRI Antidepressants on Ejaculation:A Double-Blind,Randomized,Placebo-Controlled Study With Fluoxetion,Fluvoxamine,Paroxetine,and Sertraline.Journal of Clinical Psychopharmacology,1998,18(4):274

16.Diagnostic Criteria from DSM—Ⅳ.Whshington,DC:American Psychiatric Assosiation,1994

17.Waldinger M,Hengeveld M,Zwinderman A,et al.Paroxetine Treatment of Premature Ejaculation:A Double-Blind,Randomized,Placebo-Controlled Study.Am J of Psychiatry,1994,151:1377

(陶林 刘捷 王春华)

简介:在过去的20多年里,关于早泄(PE)的知识已经取得显著进步,尤其是对于早泄的生理学理解、基于人群统计学对PE真实患病率的澄清、重新确定这一障碍的定义和诊断标准、评估社会心理对患者和伴侣的影响、构想有效的诊断和治疗结果评估、提出新的药物策略和有效性的检测、这些新建疗法的安全性与满意度等,都取得了有目共睹的实质性进展。考虑到这些大量的高水平研究,和我们已经确认的早泄(PE)诊断和治疗的3套临床实用指南《美国泌尿学科学会2004年PE药物治疗指南》、《欧洲泌尿外科学会关于男性性功能障碍指南:勃起功能障碍和早泄》以及修正于2009年的《泛阿拉伯性医学学会实用指南(射精异常)》没有足够的广泛性,并未能详细阐述心理干预和药物干预,而且非常重要的新证据并未能纳入这些指南。现在正是国际性医学学会(ISSM)颁布一套当代的、措辞严谨的、循证的、综合的和实用的PE诊断和治疗临床指南的适当时机,其针对的首要目标是一线临床医生,其次才是性医学专家。

指南建立进程(摘要)

2009年9月,ISSM的PE指南委员会在伦敦召开为期3天的会议。26个委员会成员采取同辈推荐和进一步审查筛选的方式,以提供学科的多样化、不同观点、学识、性别(22位男性和4位女性)和地域的平衡。所有成员都必须事先申报他们参与委员会工作的任何潜在的利益冲突。委员会按照“循证医学牛津中心”制定的质量、任何提议的效力划分等级。表3包含了相关委员会循证提议的摘要。会议由达帕西汀制造厂商强生公司提供无限制津贴支持。在指南和相关资源开发期间,ISSM要求(会议)从企业中完全独立出来。会议中没有企业代表,没有企业试图在(指南)建立和书写过程的任何部分或任何时间施加影响。

PE定义

现存的几个PE定义已经由各类专业组织和(或)个人整理(见表1),大多数包括终生的(原发性)和获得的(继发性,指PE症状之前有过射精功能正常的阶段)子类型。对其的主要批评和不满包括:并非循证、缺乏专业操作标准、过分含糊以及依赖诊断专家的主观判断。其共同点是:(i)射精潜伏期短;(ii)知觉的自我能力或对射精时机控制的缺乏;(iii)痛苦和人际困难(涉及射精功能障碍的)。

ISSM在2007年召开专家会议仔细回顾了循证资料,委员会在明确的科学标准基础上提出终生性PE的定义是:具有“总是或几乎总是在插入阴道前或插入后大约1分钟之内发生射精”特征的一种男性性功能障碍。不能在所有或几乎所有阴道插入中延迟射精,以及带来诸如痛苦、烦恼、沮丧和(或)回避性行为的消极的个人后果(发表于2008)(LOE 1a)。

这个定义仅适用于插入(阴道)的性活动,它没有对其他的或男-男的性行为给以PE的定义。目前尚无适用于其他一些群体的充足的临床资料,也没有足够的已经发表的客观数据为获得性PE提出新的循证定义,终生性PE的标准也许同样适用于获得性PE。(LOE 5d)

表1 早泄的定义

续表

把男性早于阴道插入的射精称作产前射精,是PE类型中最为严重的。典型的情况是这些男人/情侣出现在不育不孕门诊。据估计,5\%的终生PE男性属于产前PE。

针对许多不够早泄诊断标准、但对自己射精功能感到痛苦而请求帮助的男性,提出了自然变异型PE和早泄样射精功能障碍这两个亚型。自然变异型PE指不规律的过早射精,往往是主观上感觉射精控制能力减弱。这个亚型不属于性功能障碍或心理病理,只是性表现的正常变化。早泄样射精功能障碍指:(i)主观认为性交射精一贯过快;(ii)对射精过快或缺乏控制持先占为主的想象;(iii)实际的IELT在正常范围内或时间更长(超过5分钟才射精);(iv)在即将射精之际控制射精的能力也许降低或缺乏;(v)并不能由其他心理异常来解释。(LOE 5d)

患病率

由于PE定义的变化、患病率数据搜集方式(基于人群、自我报告或基于临床)的不同,尚缺乏一般男性终生或获得的PE患病率的可靠资,而且还受不同地理、文化环境、宗教信仰、种族和社会地位、政治影响力等背景的影响。患者自我报告的患病率明显高于根据ISSMPE定义的临床评估。根据“环球性态度和性行为研究”(GSSAB)对27,500名40~80岁的男女关于态度、行为、信念和性满意的国际调查数据显示,跨越各年龄组的全球PE患病率约为30\%(接近患者的自我报告)。而按ISSMPE定义的时间参数(IELT约为1分钟),在以人群为基础的500名男子中用秒表测量其IELT时间,仅有1\%~3\%符合诊断条件,但没有评估他们是否痛苦。这样低的患病率更接近于参加PE治疗的人数。对于那些可能不符合ISSM诊断标准的男子,也应予以谨慎评估并考虑适当的治疗。(LOE 3b)

临床实践中PE的患病率

临床中的PE常来自患者主诉,诊断更依靠其痛苦程度,症状多反复无常,所以临床实践很难评估其真正患病率。只有9\%左右的患者求治,其中81.9\%是患者主动求治,91.5\%的报告疗效极差或没有改善。26\%的患者要求处方SSRI药物,22\%每天服用SSRIs和11\%使用局部麻醉剂。

平均射精潜伏时间

近期的跨国(荷兰、英语、西班牙和土耳其)IELT调查报告表明:其中位数是5.4分钟(介于0.55—44.1分钟),随年龄增加而显著降低,从18~30岁年龄组的6.5分钟到51岁以上组的4.3分钟。中位数在不同国家之间也不同,一般与避孕套和包皮手术无关。几年后进行的相似研究结果与此吻合,IELT中位数为6分钟(为0.1-15.2分钟)。(LOE 2a)

病因学

先前认为PE主要是心理和人际因素所致,近年研究表明PE也许是躯体异常或神经生理紊乱所致。这些生物学因素如龟头高度敏感、阴部神经在大脑皮层的定位、中枢5羟色胺能神经递质紊乱、勃起困难、前列腺炎、处方药物的解毒作用(如抗抑郁药瑞波西汀、西酞普兰或滋补药)、慢性盆腔疼痛综合征、精索静脉曲张和甲状腺功能异常等。但没有哪个因素得到综合性和大范围研究的支持。

近期研究提示一些神经生物和遗传变异因素可能是符合ISSM定义的终生性PE的病理生理原因,而心理/环境因素可能维持或强化这种状态。终生性PE可以用5-羟色胺浓度低、5-羟色胺2C受体敏感度低和(或)5-羟色胺1A受体过度敏感来解释,推测也许与遗传因素有关,因此在最小刺激下快速射精,但这些只能作为2\%~5\%的PE的假设病因。

PE的遗传学

有些男性的IELT和终生性PE都是由遗传因素决定的,PE患者的一些直系家庭成员也患有PE。对芬兰男性双胞胎进行的调查表明,遗传影响可能使一些早泄男性产生特异质和易染病体质,而不支持所有终生性PE男性都具有潜在遗传影响的结论。(LOE 2a)

甲状腺激素

射精反射的激素控制尚未完全澄清。证据表明抑郁、5羟色胺和甲状腺激素之间存在一定关联,50\%的甲亢男性患PE,而甲亢成功治愈后PE降低到15\%。

前列腺炎

考虑到前列腺在射精机制中的角色和26\%到77\%的慢性前列腺炎或慢性盆腔疼痛综合征的男子报告有PE,因此推论局部感染的直接影响可能是有些获得性PE的病因。在慢性前列腺炎、ED和PE之间的准确病理生理学联系尚不清楚。(LOE 3a)

心理因素

存在一系列促成和维持PE的心理因素,可以划分为易患或历史因素(如性虐待、家庭内性态度)、个人心理因素(如体像感,抑郁、操作焦虑、情感表达障碍)或相互关系因素(如亲昵、愤怒)。有关这些变量与PE之间相关关系的研究是非常有限的,存在横断面性质,只能确认各变量之间的联系,而不是病因学的相关关系。要慎重解释这些相关关系,每种心理因素都可能导致PE,反之亦然,很可能存在事件的交互关系和影响,如操作焦虑导致PE,PE进一步恶化了操作焦虑。(LOE 5d)

PE对男性和伴侣生活质量的影响

回顾从1997年到2007年使用不同方法和结果测量的有关PE对男性、其伴侣及其相互关系,对心理和生活质量的影响的定性和定量研究报告。一致认为PE男性及其女伴存在与早泄相关的消极影响、相互关系困难、抑郁和生活质量的整体下降。PE男性容易自卑和缺乏自信,报告称1/3的PE男性存在与性情境相关的焦虑。PE对单身男子的消极影响更大,而且成为阻碍寻找新关系的障碍。PE对女性的性体验具有直接的消极影响。(LOE 1a-3a)

临床医生应该特意筛选PE吗

筛选涉及对无症状人群的测试,以发现处于疾病早期阶段的案例。受PE影响的男性不是没有症状。委员会同意,无论对一般人群还是最可能的亚人群,没有足够证据对PE的筛选提出建议。建议对ED男性要筛选一下有无PE。(LOE 5d)

需要提高公众有关性健康问题的意识,包括PE,这样的话,受性忧虑影响的个体才能意识到对他们开放的选择和干预介入。临床医生应该扮演性健康教育的重要角色,并且PE也应该包含在性健康教育的项目中。

PE的评估

历史

如果在医务人员诊室谈论性主诉,患者常会尴尬、害羞、吞吞吐吐。患者期待临床医生主动询问他们的性健康,这就允许患者自由地讨论他们的性忧虑,并有助于筛选相关的健康风险(如心血管风险和ED)。

表2列举了推荐的和可选择的应该询问主诉PE患者的问题,以便确立诊断和实施恰当治疗的细节。(LOE 5d)

图1是由罗兰德等设计的流程表,为主诉PE的患者详细设计了评估和治疗的选择办法。

射精潜伏时间的评估

射精潜伏时间(IELT)的秒表评估

在PE的临床试验和观察研究中,广泛利用秒表测量IELT,但没有推荐到常规临床使用。秒表测量也存在干扰性愉悦或破坏自然性的缺点。患者和伴侣自我报告的IELT与秒表测量具有很好的相关性,可以替代秒表测量。自我报告是寻求治疗和满意的决定因素,建议把患者和伴侣对IELT的自我评估作为临床决定IELT的方法。(LOE 2b)

身体检查

对于终生性PE,尽可能实行身体检查但并非强制性的,有些患者觉得检查具有安慰作用。对于获得性PE,必须进行定向的身体检查,以评估与ED、甲状腺机能失调和前列腺炎等相关联/或有因果关系的疾病。(LOE 5d)

表2 为建立早泄诊断和恰当治疗推荐和可选择的问题

仪器的评估

秒表评估

PE的标准评估测量包括使用经过验证的问卷和患者报告的结果(PRO),再加上IELT的秒表测量。这些新的测量主要作为研究工具而建立。一些量表已经显示出良好的心理测量性能,对于临床筛查和评估具有潜在的辅助价值。相反,IELT的秒表测量虽然广泛运用于临床测试和观察研究,但是一般不建议运用在临床实践中。

至今已经建立和发表了5个可用的问卷。目前,已有两个经过广泛心理测量学的测试和确认,是当前评估PE的首选问卷量表,特别是能够满足大多数开发和治疗反应监测的标准:早泄量表(The Premature Ejaculation Pro le,PEP)和早泄指数量表(Index of Premature Ejaculation,IPE),它们已具有海量的数据库。第三个简短诊断量表也已建立并有效运用于临床。所有上述三个测量都可在附录1.中找到。一个量表(PEDT)具有中等规模的数据库,另外两个量表(阿拉伯,中国PE问卷)只得到最低限度的效度或临床验证数据。不推荐后面这些量表使用于临床。总之,这些量表可以当做有用的辅助,但不应代替有资历的临床医生所采集的详细性历史。(LOE 2b)

治疗

药物治疗

以下概括了所有推荐的治疗早泄的药物,不包含仍在临床试验中的化合物。使用局部麻醉以减少龟头敏感度可能是PE治疗的最熟知的模式。选择性5羟色胺再摄取抑制剂(SSRIs)、帕罗西丁、舍曲林、氟西汀、西他罗仑和三环类抗抑郁剂(TCA)氯丙咪嗪等的引入,给PE治疗带来了革命性的变化。这些药物通过5-羟色胺转运者阻止了轴突由中枢5羟色胺能神经元的突触间隙再摄取5羟色胺,结果导致5-羟色胺能神经传递的增强并刺激突触后膜的5-羟色胺自身受体。

选择性5羟色胺再摄取抑制剂和三环类抗抑郁剂治疗

早泄治疗可以使用按需服用SSRIs如达帕西汀,或药品核准标示外使用(或“非常规性使用”,offlabel)的氯丙咪嗪、帕罗西丁、舍曲林、氟西汀和西他罗仑。

达帕西汀(Dapoxetine)

达帕西汀已经在奥地利、德国、意大利、芬兰、墨西哥、新西兰、葡萄牙、韩国、西班牙和瑞典等地获准用于早泄的治疗。它是一个起效迅速、半衰期短的SSRI药物,其药代动力学特点提示可以作为按需服用的PE治疗药物。没有达帕西汀与其他药物相互作用的报告,包括磷酸二酯酶抑制剂。在RCTs中,从第一次给药开始,在性交前1~2小时服用30或60mg的达帕西汀明显比安慰剂有效,IELT增加2.5至3.0倍,射精控制增强,降低郁闷并增强愉悦感。对于终生性或获得性PE都非常有效。副作用是罕见的,具有剂量依赖性,包括恶心、腹泻、头痛和头晕。4\%(30mg)和10\%(60mg)的受试者因此而退出临床试验。没有自杀观念或自杀企图上升的迹象,突然停用达帕西汀基本上也没有撤退症状。

有1a的证据支持按需服用达帕西汀治疗终生性或获得性PE的有效性与安全性。(LOE 1a)

药品核准标示外使用的SSRIs和TCAs

每日服用帕罗西汀10~40mg、氯丙咪嗪12.5~50mg、舍曲林50~200mg、氟西汀20~40mg、西他罗仑20~40mg常常能有效延迟射精。发表的数据元分析建议,帕罗西汀延迟射精的作用最强,基线水平比IELT提高约8.8倍。

射精延迟常在开始治疗5~10天内发生,但也许2~3周后才能充分起效,要求长时期坚持服用。不良反应通常很轻微,开始出现在治疗的第1周,可能在2~3周内逐渐消失。它们包括疲劳、困倦、轻微恶心、腹泻或出汗。很少有性欲低下和ED的报告,在无抑郁的PE男子中发生率较低,而在服用SSRIs治疗抑郁的男子中发生率更高。神经认知模式的不良反应包括少量患者的明显焦躁不安和轻度躁狂。有双向抑郁病史的男子应避免采用SSRIs治疗。

对抑郁症和(或)焦虑症患者的RCTs全面分析表明,在年轻人中,自杀观念或自杀企图的风险会有所增加。相反,在无抑郁症的PE男子的SSRIs临床试验中并未发现自杀观念的风险。对18岁或以下的PE青少年,以及合并抑郁症的PE男子,尤其是具有自杀观念的男子,仍然建议慎开SSRIs处方。应建议患者避免突然中断或迅速减少每日服用的SSRIs用量,否则也许会伴发SSRI撤退综合征。

按需服用氯丙咪嗪、帕罗西汀、舍曲林和氟西汀应该在性交前3~6小时服用,中度有效及容易耐受,但其射精延迟作用明显不如每日服药的治疗效果。按需治疗既可以配合每日用药的最初试验,也可以配合低剂量的每日用药。SSRIs或TCAs治疗PE的最大局限性是停药所导致的复发。再者,患者不愿意服用SSRIs来开始PE的药品核准标示外的治疗。据报告,30\%的患者拒绝开始治疗(帕罗西汀每天10mg共21天,接着按需服用20mg),而另外有30\%开始治疗的患者中止了治疗。其原因包括:不想服用抗抑郁药;治疗效果低于预期;因为关系问题,短时间失去性兴趣;以及一些不良反应。

具有1a水平的证据支持SSRIs药品核准标示外每日服用帕罗西汀、舍曲林、西他罗仑、氟西汀和5羟色胺能三环类氯丙咪嗪和按需服用氯丙咪嗪、帕罗西汀、舍曲林治疗终生性或获得性PE的有效性和安全性。(LOE 1a)

采用达帕西汀(如果有药可用的话)按需给药或每日服用药品核准标示外SSRIs治疗PE的方案,应基于医生对各个患者具体需求的评估。性交机会很少的PE男子,可能更喜欢按需服药,而已经建立稳定性关系的男子可能更喜欢每日用药。

在一些国家,因为监管机构对未注册/批准的药品核准标示外适应证治疗的强烈反对,医生开处方会存在困难,那么早泄治疗将是复杂和困难的。

局部麻醉剂

局部麻醉剂的使用已经得到很好发展,如乳状、凝胶或喷雾状的利多卡因和(或)丙胺卡因,对延迟射精具有中等效力。PSD502是正在临床试验中的利多卡因和(或)丙胺卡因喷雾剂。试验结果表明治疗组的IELT提高了6.3倍,并伴有与控制和性满足相关的PRO测量的改善。因为化合物的独特配方,只有极少报告说阴茎感觉减退会转移给伴侣。其他局部麻醉剂伴有显著的阴茎感觉缺失和可能的经阴道吸收,除非带上避孕套,否则将导致阴道麻木和女性性高潮缺失。

有1b证据支持在药品核准标示外使用局部麻醉剂治疗终生性PE的有效性和安全性。(LOE 1b)

磷酸二酯酶5型抑制剂(PDE5i)

对服用PDE5i治疗PE的14项研究进行综述,不过未能提供充分证据支持单独或配合SSRIs使用PDE5i在治疗PE中的作用,除非PE男子合并有ED。近期设计的合理研究确实支持那些药物的潜在作用,需要和鼓励更进一步循证研究的证据。

只有4d水平的证据支持在药品核准标示外使用按需或每日口服PDE5i治疗勃起功能正常的终生性PE的有效性和安全性。不推荐使用PDE5i治疗勃起功能正常的终生性PE。

其他药物治疗

文献已经报告过按需服用中枢止痛药物曲马多,或海绵体内血管活性药物注射的方法。按需服用曲马多25mg可以把IELT从基线的1.17分钟提高到7.37分钟。另一研究表明,按需服用曲马多50mg,能在治疗结束时使IELT从19秒的基线值提高到243秒。曲马多组报告的与治疗相关的不良反应为28\%,而安慰剂组为15\%,包括恶心、呕吐和头晕。

有2d证据支持这些治疗的有效性和安全性的,但并不推荐使用它们治疗PE。

手术

有些作者报告在行为和(或)药物治疗难以治愈的终生性PE治疗中,有选择性地进行阴茎背神经阻断术、透明质酸凝胶阴茎龟头增大术引致阴茎感觉减退。直到有进一步的研究报告后,才能清楚外科手术对PE处理的作用。

只有4级水平的证据,也就是说目前没有证据提示选择性阴茎背神经阻断术或透明质酸凝胶阴茎龟头增大术是治疗PE的有效方法。外科手术也许与性功能永久丧失有关,当前并不推荐用于PE的处理。

心理/行为,综合的医学、心理和教育的干预

目前已建立起广泛针对PE的心理干预方法。大多数对心理治疗结果的研究是没有对照、非双盲的试验,没有一个符合高水平的循证研究要求。文献报告中参与人群数量很少或规模不够大,他们接受不同形式的心理干预,只有有限的随访或没有随访。在大多数研究中,积极的治疗组未与安慰组、对照组或候诊小组相比较。经常使用的行为治疗方法大多是挤捏技术和“停-动-停”技术。这两种治疗都是用于教育男性去识别中度程度的兴奋。男性通过一系列循序渐进的训练,掌握能够识别中度兴奋的技术:从自我刺激开始,变换为伴侣手法刺激,然后是不抽动的性交,最后采取“停-动-停”的抽动。这些循序渐进步骤的结果是增加了IELT、性自信心和自尊,不过只有极少数的对照研究支持这一主张。

较早的无对照组的挤捏技术研究报告说治疗结束时的失败率为2.2\%,5年随访时为2.7\%。其他研究结果发现成功率在60\%~90\%之间。近期一项研究证明,行为技术组的IELT比候诊对照组提高了8倍。

心理干预的设计目标不是单纯为提高IELT,而是要达到另外的结果。目标因素集中在男性、他的伴侣和他们的两性关系。其目标特别包括:①提高男性性表现的自信心,也要提高其整体自信心;②减轻操作焦虑;③加强与伴侣的沟通;④解决可能促成或维持PE的人际关系问题。

有关心理/行为干预治疗PE有效性的证据为2b水平。

应该对具有自然变异性PE的男性(伴有射精主观控制能力感觉减弱的,不规则的和反复发生的快速射精)进行教育并恢复其自信心。早泄样射精功能障碍(那些IELT在正常值范围,但先占为主认为射精控制有问题)的男性需要转诊接受心理治疗。同时需要更多的研究以更好地确定对这些暂时性PE亚型恢复信心、教育和心理疗法的有效性。

医疗保健机构和心理健康专业人员在治疗PE方面的兴趣和训练水平是不同的。一般来说,所有临床医生都应该能够作出诊断、提供支持并指导行为练习。当情况复杂和(或)患者对初级的干预缺乏反应时,医生应该考虑转诊给性健康专家处理。

伴侣的重要性

把伴侣包含在治疗过程中是极其重要的,但并非成功治疗的强制组成部分。有些患者不理解为什么医生希望其伴侣也参与进来,有些伴侣则不愿意加入到患者的治疗中。然而,如果伴侣不参与治疗,她们也许抵制改变她们性生活方式的性合作或配合。有一个愿意合作的伴侣,能增强男性的自信、技能、自尊和男子汉气概,以及更广泛协助男性建立射精控制能力。反过来,这样又能促使伴侣性关系的改善、乃至他们整体关系的广泛改善。虽然没有对伴侣是否参与PE治疗对治疗效果影响的对照研究,但一篇对ED治疗研究的综述证明,在成功治疗的案例中,性伴侣的配合包括人际因素都起到重要作用。

医学和心理综合治疗的益处

有3项研究报告了PE的药物和行为综合治疗,1项研究报告连续的药物治疗和随后的行为治疗。每项研究报告的药物是不同的,在所有3项研究中,无论是IELT和(或)中国早泄指数,综合治疗都优于单独的药物治疗。

对于ED,同样发现综合治疗既比单独的心理治疗有效,也比单独的药物治疗有效。单独药物治疗所不能涵盖的因素可以用心理方法处理,包括:①患者因素(操作焦虑、自信);②伴侣因素(伴侣性功能障碍);③人际因素(冲突、缺乏沟通);④在两性关系中的性因素(性脚本、性满足);⑤背景因素(生活紧张因素)。

对于有清楚的心理积淀的获得性PE,或者有个体或双方问题干扰医学治疗的终生性PE,综合的药物和心理方法也许特别有用。同样,对于PE合并ED的男性,综合治疗可能也有助于处理这些性功能障碍的心理社会因素。(LOE 2a)

教育和教练的角色

对于那些没有接受药物治疗的PE男性,对其进行PE的教育或许有助于改善PE。提供PE流行率及一般人群IELT的教育,也许能消除关于PE的神话。额外教育对于回避性活动的PE男性也有所帮助,他们因为害怕性兴奋而不愿和伴侣讨论PE,或限制他们的全部性活动。教育策略的目的是给予男性自信心,去尝试医疗干预、减少操作焦虑和调整他适应不良的性脚本。(LOE 5d)

终生性PE 因为终生性PE很可能具有器质性病因,首先推荐结合基本心理教育的医学介入。然而,如果PE已经导致心理或关系的问题,患者程度的评分、伴侣咨询、指导和(或)关系治疗也许是医学干预有用的辅助手段。(LOE 1a)

获得性PE 推荐HCPs采用可行的医学和心理结合的方法。男性渴望治疗得到立竿见影的效果,因此,对于ED这样的相关疾病因素的医学治疗及改善,会非常有用。

此外,对PE性质的教育,帮助男性通过行为练习提高射精控制能力,识别受约束的/狭隘的性行为模式,及化解人际关系的问题,也许会对获得性PE有显著帮助。一旦男子自信心和控制感得以改善,就有可能减少或中止医疗干预。(LOE 5d)

特殊患者群体

PE合并ED

近期数据证明,30\%~50\%的ED患者也经历过PE。男性患ED时可能要求更强烈的手法刺激以达到勃起,或尽快性交以避免勃起的消退,结果导致快速射精。而伴随存在的与ED相关的严重操作焦虑,只能使早泄程度进一步恶化。证据提示,单独的PDE5i或联合使用SSRI也许会对合并ED的获得性PE产生一定作用。PDE5i改善勃起功能和提高IELTs之间的高度相关性表明,改善勃起功能能够减轻PE的严重程度。尽管轻度的ED和PE男性确实得益于SSRIs的治疗,可是与没有ED的男性相比,他们的治疗反应肯定降低了。另外,具有终生性PE合并轻度ED的男性,对治疗的反应性肯定不如具有获得性PE合并轻度ED的男性。

用ED药物治疗合并ED的PE,得到1A水平的证据支持。用ED药物和PE药物组合的治疗合并ED的PE,得到3C水平的证据支持。鼓励进一步的循证研究。

PE和甲亢

许多甲低患者具有ED,只有极少数经历PE,而50\%的甲亢男性有获得性PE。采用抗甲状腺药物、放射活性碘或甲状腺切除术等甲亢治疗,使35\%的人射精功能正常化。委员会不推荐让获得性PE男子做常规的TSH筛选。

PE和慢性前列腺炎

慢性前列腺炎的抗生素治疗减轻了下尿道症状(LUTS),但几乎没有资料提示其能改善PE和其他性功能障碍症状。虽然对射精疼痛或LUTS男子进行了强制性的前列腺身体和微生物学检查,但没有足够的证据支持对PE男子进行慢性前列腺炎的常规筛查。

结果

这篇文章包含了ISSM的PE指南委员会的报告,肯定了ISSM关于PE的定义,并提出其患病率远远低于先前的估计。介绍了基于人群统计学的正常射精潜伏时间和关于PE的生物与心理病因的循证数据。描述和证实了简短的评估程序,回顾了有效的诊断和治疗问卷。最后,介绍了最实际的治疗提议指导那些熟悉或不熟悉PE的临床医生,以便能给他们的患者提供便利的治疗。纵观所有治疗,包括每天服药和按需服药治疗计划在内的患者的IELT,预期会比基线水平提高3到8倍。治疗也显著改善了患者可以知觉的对他们射精时间的控制能力(自我效能感)。来自患者的观点和可以知觉的射精控制能力是比单独的IELT更为重要的结果变量。

通过一个直观、简短和得到确认的问卷就能判断治疗的结果,这就是称作“临床改变的整体印象”(Clinical Global Impression of Change,CGIC)。它询问患者,“与开始治疗之前相比,请你描述你治疗后的PE是:严重恶化、恶化、轻微恶化、没有变化,轻微好转、好转或大大好转”。(LOE 1b)

在随机的、有安慰剂对照的、双盲的帕罗西汀的临床研究中,包括CGIC改善、IELT增加幅度、IELT增加与CGIC得分的关系等范畴都得到改善。另外,注意到在CGIC得分和射精控制与性满足之间的正相关,而CGIC得分与性抑郁之间呈负相关。对照化解了CGIC绝大部分的差异。

初级保健医生(PCP)的角色

初级保健提供者通常是健康保健系统接触患者的一线人员,包括性问题的诊断和治疗。这个角色包括:(i)对任何未确诊的体征、症状或健康忧虑(“未分化的患者”)的最初的识别和评估;(ii)包括疾病预防、健康管理、咨询、患者教育、慢性疾病处理和患者支持在内的健康促进;(iii)协调与患者保健促进的有效沟通,和鼓励患者成为健康的伙伴。这样的保健模式不局限于问题来源、器官系统或诊断。

基于几个原因,初级保健医生是为性困难患者提供帮助的理想群体,这些原因包括:(i)与患者讨论和解决性问题时的纵向的和个人之间关系的价值;(ii)一个全科医生可能对围绕性问题的多因素问题作出更恰当的评估;(iii)对于已确诊的性功能障碍患者非常适合于在初级保健做常规的长期随访。PCP的主要任务是识别PE,并使患者对获得帮助感到舒适。应该由具有与患者就性活动顺利畅沟通能力的,以及对一线治疗有广博学识的PCP来规划最初的引导和治疗。泌尿科医生有助于处理困难和复杂的形式。具有处理性问题经验的精神健康专业人员与临床医生协作,可以提高成功治疗的可能性:(i)解决性困难;(ii)梳理出重要病史;(iii)患者和伴侣的教育;(iv)建议增强性能力的技巧;(v)帮助伴侣解决个体问题,也解决相互关系问题。

对于在处理男子PE时遇到的不同时间段的不同变化,最好得到性健康专家的帮助。作出需要会诊决定的主要因素包括:(i)初级保健医生在讨论和处理治疗选择时的舒适问题;(ii)所涉及的心理社会和性问题的深度;(iii)最初干预努力的成功或失败。在特别情况下,更为特别的专业帮助常常是有用的,它包括:(i)治疗失败;(ii)解剖的或复杂的激素问题;(iii)围绕性和(或)伴侣关系的复杂问题;(iv)严重的心理问题;(v)在治疗医生需要帮助的任何时间。

表3 有关早泄的循证建议

续表

处理原则“ALLOW(容许)”是“PLISSIT”原则(允许、有限信息、特殊建议和强化治疗)的副产品,能够帮助PCPs就性健康问题与患者交谈,以及在适当时转诊患者。医生从“询问(Asking)”患者的性生活开始,并“合理(法)地(Legitimizes)”关注和处理性功能障碍,也许要把患者转诊给一位性健康专家以进行进一步的评价和处理。相反地,如果医生对处理患者的问题感到自信无忧,他/她接着会“开放地(Opens):就性问题展开进一步的讨论”;并且医生和患者“一起工作(Work together)去建立处理计划”。这种方法允许患者在讨论其性问题时能随医生的具体情况而有所伸展。

当初级保健医生在临床工作中遇到棘手问题时,可以有两个选择:一是把患者送到性健康专家处寻求“咨询”。考虑到额外投入,进一步的处理仍打算由初级保健医生执行。第二个选项是“转诊”患者给另一位医生,通常是一位性健康专家去做进一步的处理。在这种情况下,患者应随身携带转诊理由的小结、充分的病史、相关的诊断研究。性健康专家接着将做进一步适当的处理,也保持初级保健医生考虑过的进展。在“转诊”的情况下,患者的保健直接转给性健康专家。初级保健医生通过告知性健康专家所涉及的特定请求的水平,可以增强彼此的沟通。

由缺乏经验的初级保健医生与适合的性健康专家一道合作处理,这是处理患者问题和提高初级保健医生对PE理解的极好方式。临床医生之间的良好沟通能够增进患者的治疗结果,并理解治疗计划和监督措施。如果初级保健医生与性健康专家之间就谁将做什么的问题上达成明确共识,即可优化共同处理的问题。通过达成使用一个具体转诊指南的“转诊协议”就得以确保这类沟通,每个医生的职责和活动就得以清晰阐述。

结论:

这些指南已由跨学科的PE领域公认的专家组成的国际专题组慎重颁布,目的是为诊断和治疗PE的家庭医生和性医学专家构建一个措词清晰、实用、循证的推荐。意识到不是所有的证据都有同等的效力,委员会根据循证医学牛津中心制订的排名,经过评审和辩论,推出了他们所推荐的不同级别。

上述表格列举了所有与PE指南委员会相关的提议。这些指南确认了PE的ISSM定义并提议,其患病率明显低于先前的想象。介绍了PE的有关生物和心理病因的循证资料,就像正常射精潜伏时间的以人群为基础的统计学数字。描述了简短的评估程序,评审了经过验证的诊断和治疗问卷。最后,介绍了指导医生的最实际的治疗提议,包括熟悉或不熟悉PE的临床医生,使他们能够便利地治疗他们的患者。

指南的制定是一个持续评审资料和吸收最新研究结果的循序渐进的过程。我们期待不间断的研究为这个性功能障碍提供更完整的病理生理学理解、也使得性功能障碍的治疗更有效更安全。因此,强烈建议ISSM每隔四年重新评价和更新这些指南。

最后,重要的是记住PE可以引起男性、他的伴侣和伴侣双方显著的个人和人际关系的痛苦。我们热切期待这些指南能够协助医生准确诊断和处理那些存在PE主诉的患者。

早泄指数(IPE)

这些问题是:在过去4周里,在你性生活中遇到的性问题对你的影响。请尽可能诚实和清晰地回答提问。在回答这些问题中,适用下列定义:

性交:是定义为阴道插入(你的性器官进入你伴侣的阴道)。

射精:精液自阴茎射出。

控制:当你准备好时才射精。

痛苦:意思是早泄如何让你沮丧、失望或烦恼。

每个问题只选择一个答案(标注其中一个方格)

1)在过去4周的性交中,你的射精时有多少时候是能够控制的?

□没有性交 □几乎总是或总是 □一半以上时间 □大概一半时间 □少于一半时间 □几乎从未或从未

2)在过去4周的性交中,当你射精时你有多自信?

□没有性交 □高度自信 □中高度的自信 □自信心不高不低 □中低度的不自信 □自信心低

3)在过去4周的性交中,当你射精时有多少时候是满意的?

□没有性交 □几乎总是或总是 □一半以上时间 □大概一半时间 □少于一半时间 □几乎从未或从未

4)在过去4周的性交中,当你射精时你对控制感有多满意?

□没有性交 □非常满意 □有些满意 □谈不上满意不满意 □有些不满意 □非常不满意

5)在过去4周的性交中,当你射精时对你的性交时间长度有多满意?

□没有性交 □非常满意 □有些满意 □谈不上满意不满意 □有些不满意 □非常不满意

6)在过去4周中,你对你的性生活整体上有多满意?

□没有性交 □非常满意 □有些满意 □谈不上满意不满意 □有些不满意 □非常不满意

7)在过去4周中,你对你与性伴侣的性关系有多满意?

□没有性交 □非常满意 □有些满意 □谈不上满意不满意 □有些不满意 □非常不满意

8)在过去4周中,性交给予你多大的愉悦?

□没有性交 □高度愉悦 □中高度愉悦 □谈不上高度愉悦或不愉悦 □中低度愉悦 □很不愉悦

9)在过去4周中,你对射精前持续时间多长有多痛苦(挫折)?

□没有性交 □极其痛苦 □非常痛苦 □中度痛苦 □轻度痛苦 □根本不痛苦

10)在过去4周中,你对自己的射精控制有多痛苦(挫折)?

□没有性交 □极其痛苦 □非常痛苦 □中度痛苦 □轻度痛苦 □根本不痛苦

早泄象(PEP Items)(这个早泄象,翻译有没有问题?请提供原文及作者核对)

在过去的一个月中,你在性交期间能够控制射精吗?

非常差 差 一般 好 非常好

在过去的一个月中,你对性交满意吗?

非常不满意 不满意 一般 满意 非常满意

在过去的一个月中,你对性交时射精快有多痛苦?

根本不痛苦 有点儿痛苦 中度痛苦 相当痛苦 极度痛苦

在过去的一个月的性交中,射精快对你和伴侣的关系造成多大程度的困难?

根本没有困难 有点儿困难 中度困难 相当困难 极度困难

早泄诊断工具(PEDT)

患者指导

后面的问卷是帮助鉴别男子在性活动中可能射精太快问题的。即使你没有困难,也请回答以下所有问题:

对于下面的每个问题,选择最符合你自己情况的答案,请在其方框中画×。每个问题请只选择一个答案。

虽然你的经历可能随时间而有所发生变化,请报告你一般的性交经历。

定义:射精在这里指的是插入后(你进入你的伴侣后)精液的排出。

(周旭编 译)


\section{第七节 不射精症}

不射精(Unejaculation)或称不射精症、射精不全、射精不能等。不射精是射精障碍的一种,是男性生殖生理过程中的一种病理表现,为一种或多种病因引发的病理过程,男性患者有正常的性兴奋和阴茎勃起,阴茎插人阴道并作抽动,达不到性高潮和获得性快感,不能产生节律的射精动作,没有精液射出尿道外口的一种异常现象;或是在其他情况下可射出精液,而在阴道内不射精,两者都称为不射精症。由于这种病主要见于青壮年,处理不当还会影响夫妻感情,甚至导致家庭破裂,给患者带来精神和肉体上的痛苦。

马斯特斯和约翰逊于1970年总结了约450名性功能障碍的男性患者,其中不射精患者仅有17名,还不到4\%。因此,国外学者认为不射精是一种少见疾病。在国内,泌尿外科、男性病专科或不育专科的不射精患者并不少见,国内报道不射精的发病率约占男性性功能障碍的28\%,亦有报道称不射精患者占男性不育症者的20\%。统计上的差异可能与国内外专科患者的集中比例差别有关。

在不射精患者中,功能性不射精占多数,而器质性不射精则较少。这一点和国外并无差别。大多数器质性不射精是由原发疾病引起或者属于医源性疾病,有些患者虽然患有不射精,但就诊时并不将不射精作为第一主诉;有些垂体肿瘤、截瘫患者、前列腺肿瘤或盆腔肿瘤术后的病者,因神经损伤致不射精,影响了他们的生活质量。医生应详细询问病史即可预知和发现这些患者合并ED或不射精,实际上器质性不射精的患病率比以此症状就诊的要多些。过去曾认为不射精症的发病率与其他男性性功能障碍相比较低,由于盆部术后的患者在术前已知道术后性功能障碍的结果,这类性功能障碍的发病率并不像有关学者报道的那么低。该类障碍在中国人中似乎比西方更为多见,例如,国内江鱼教授等(1986)报道的1107名男性不育症患者中,不射精者竟占39.3\%。不射精所占的比例就更高了,远超出国外报道的水平。70\%以上的不射精症是因为性知识缺乏及房事方法不当所引起。另外,中国人对于精液“一滴精,十滴血”之类的观念,深入人心,容易引发心理问题导致的不射精症。在临床观察上,不射精是早泄的另一个极端,从射精阈值来看,不射精患者的射精阈值太高,对一般人来说是足够和有效的性刺激,也无法使他们射精,而早泄患者的射精阈值则大低。不射精症的主要后果是不孕,如仅仅在性交时不射精,怀孕问题是不难解决的。但某些患者的潜意识中出现抑制射精的倾向,怀孕问题就难以解决。不射精患者在性生活时,阴茎可以勃起,也渴望性高潮的释放,但即使性交很长时间,都会使双方感到十分劳累和不快。男方不能达到性高潮,突出特点是只要患者的阴茎位于阴道之内时就无法射精。不射精的处境与勃起功能障碍相反,ED的抑制发生在勃起阶段,而不射精的抑制发生在射精阶段。如受到足够刺激(如按摩器),勃起功能障碍患者可以再射精。

患者在性生活时,阴茎可以毫无困难地勃起而且很硬实,性交时间也能维持很久而不疲软,但不能达到性高潮及射精,快感明显减弱。这类患者并不少见,他们羞于启齿,讳疾忌医,默默忍受因性问题及不育带来的烦恼。射精过程属于一种精神、神经反应。精神方面的性刺激,神经方面包括神经通路的完整及脊髓射精中枢反应正常的阈值;对色情影像、感官性接触、性幻想、性梦等刺激大脑性中枢兴奋引致性行为并能产生性高潮、射精。各种心理障碍和病理因素导致大脑性兴奋中枢抑制,传出神经损伤,以及脊髓射精中枢阈值提高均可导致性高潮缺失致不射精。外界各种性刺激诱导的神经信号强度达不到射精反射所需的阈值水平。用电刺激射精获取精液的方法就是避开以上的精神神经因素,能利用较强度的电流直接对截瘫患者的前列腺和精囊腺进行电刺激而诱导射精发生。

与其他性功能障碍一样,不射精也可分为原发性、继发性和偶发性三种类型。

原发性不射精系指初次性交就出现不射精,若在任何状态下从未射过精,包括自慰、性梦中从未有过射精,则称原发性完全性不射精症,与女性原发性性高潮障碍是一致的。原发性完全性不射精症多存在严重的神经病理问题(如青春期前就发生的截瘫),这种情况不列入不射精症的范畴之内。

但是大多数原发性不射精患者在遗精、手淫或由女方用手或口等进行非性交刺激时仍能射精,称为原发性选择性不射精症,也有人称之为境遇性不射精症。原发性选择性不射精症的严重程度有相当大的差别,轻者仅限于特殊的能激起焦虑的情境。患者如处于能明显激起犯罪感或抵触的情境中或只有与特殊的性伴侣性交时不能射精,但与一个他喜欢的女性或在他认为的安悦情境中,他也能顺利地进行阴道内射精,这些轻度的境遇性不射精症患者很少求医。临床多见的是抑制程度更为严重的不射精症。例如,很多患者主诉,虽然他们在每次性活动中都竭尽能力,但在性交过程中却从未达到过性高潮。尽管性交时间长达几十分钟甚至超过一小时、富于刺激的性幻想、拼命喝酒和调情类饮料等,但均无济于事。但在同一性伴侣用手或口刺激时却能顺利地达到性高潮,完成射精。另一些具有更严重症状的患者主诉,他们的射精反射会仅仅由于与伴侣的接触而受到抑制,这些患者为了使他们的性伴侣达到性高潮而进行性交,也能从中获得一定程度的欢愉,但他们只有在女伴满足之后,抽出阴茎,在性伴面前手淫才能达到高潮。“甚至连这种行为都会激起相当程度的焦虑,以至于它们不得不利用性幻想来分散焦虑,以便能够完成射精反射。当射精抑制更严重时,他们根本不能在女性面前射精。为了缓解性紧张,他们必须离开卧室背着女方手淫”。有些患者的焦虑性绪需等一或数小时,直到与异性相遇的兴奋彻底消除后,才能缓解。本病中的许多人甚至不愿意性交,只靠手淫释放和舒缓其性紧张。

原发性选择性不射精症多由性无知、性抑制或各种内心与关系冲突(意识性抵触,对阴道内射精有恐惧心理)等因素所引起,或性交时刺激性神经兴奋的程度不够,没有达到能引起射精反射的强度,使清醒状态下的射精反射弧未“接通”。

(1)性无知:新婚时双方均缺乏基本的性知识,不知道性交是怎么回事,不知道性交时阴茎要在阴道内进行频率快、幅度大的持续摩擦,甚至对性关系、处女膜破裂等怀有畏惧心理。

(2)精神及感情因素:由于从小接受的“正统”教育,把性斥为下流、肮脏、淫秽、见不得人的事,把性生活视为洪水猛兽,怀有畏惧与犯罪心理;对现配偶不满意,仍念念不忘旧日情人;由于受配偶的排斥,不愿意让配偶分享他达到性高潮时的愉悦;敌视配偶,怀疑妻子婚前不贞或有外遇;无性交欲望或为采取避孕措施,羞于说“不”,不敢表示自己的不情愿;结婚负债多,思想压力大,性欲受抑制;新婚性交失败遭妻子冷遇和呵斥,逐渐丧失性欲和回避性交;对生育问题存在不同看法,根本不愿意让妻子怀孕。如果男性因上述理由不愿意射精,应排除在不射精症之外,不过由于妻子的要求,治疗仍需安排,甚至相当武断地强加给他们(让人联想到性虐待)。有的男性对婚姻不情愿,这时,治疗重点不是帮助他在性交中达到高潮,而应该帮助他更直接地表达自己对妻子的拒绝,由他们双方决定婚姻的断或续。

(3)女方因素:害怕性交疼痛,怕损伤阴道和内脏器官而限制男方抽动;女方体质差、厌烦性活动,而使男方性冲动屡受挫折。

(4)客观因素:如居室条件差,形成性压抑;双方工作时间不同,性生活不协调;男方过度疲劳或酗酒。

(5)局部解剖因素:包皮过长,在阴道内摩擦时龟头奇痒难忍;包皮嵌顿、疼痛,性交被迫中断;严重精阜炎以致发生萎缩性变化,不能有效参与射精过程。

(6)错误的手淫方式:有些人的手淫方式与众不同,不是摩擦阴茎而是压迫阴茎,如坐位时,把阴茎压在大腿下面,通过压迫方式最终达到射精;或卧位时,把阴茎压在躯体下面,通过压迫方式最终达到射精。长期以来形成的条件反射使得他们在婚后性生活中适应不了摩擦刺激,结果性交不能射精。

指过去曾经有过射精,现在由于某种原因出现不射精。

继发性不射精的患者,有过正常的射精功能。这类病例常见,发病前他们有过特殊的精神创伤;或某次性交时被外人发现而突然中断;或未婚或婚外性行为受到责罚;或对方畏惧妊娠;或双方感情不和或对妻子怀有敌意等均可导致阴道内射精能力的丧失。

指在某种场合下或某种条件下出现的境遇性不射精,在其他的环境及场合和恢复原来的做爱环境后会正常射精。

功能性不射精是射精障碍中最多见的一型,约占病例数的90\%以上,它的主要特点是性交时不能射精,但在睡梦中可出现遗精,或者在清醒状态下采用较强烈的自慰刺激能诱发射精。功能性不射精的原因可以归类为:

(1)各种外来的不利因素使大脑皮质对脑部和脊髓射精中枢产生抑制。常见的不利因素有:长期受不正确性观念的影响,认为性交是不道德不卫生的行为,对配偶有敌视态度,担心配偶对性交不满足,担心怀孕,恶劣的性交环境,夫妻关系不和等均可导致不射精。

(2)脊髓射精中枢兴奋阈值增高。常见于长期自慰患者,由于长时间高强度刺激阴茎使脊髓射精中枢长期处于疲劳状态而使兴奋阈值增高。

(3)对阴茎的刺激程度不足而不能诱导脊髓射精中枢反应。具体原因有:阴茎短小、包皮过长、错误的手淫方式、阴道宽松、性交姿势不正确、女性处女膜伞形成或处女膜坚韧使阴茎不能完全插入阴道等均可由于对阴茎的刺激程度不足而引起不射精。长期婚前习惯于自慰,通过自慰体会欣快感,而阴道中的阴茎抽动所感受的刺激强度不如自慰;患者心理负担加重或者来自性伴侣的压力均可影响射精,有些患者婚后依然利用自慰代替性交。

(4)阴茎本身的器质性病变(阴茎硬结、严重尿道下裂、阴茎系带过短)导致性交疼痛引起的不射精,这类患者可有遗精。

器质性不射精常由垂体肿瘤、脊神经损伤、先天性输精管及附属性腺器官发育不良以及手术、外伤等造成腹膜后交感神经损伤等器质性病变或医源性损伤所引起,这类患者无论在性交中还是睡梦中均无射精现象。

产生的原因有:①大脑侧叶病变(中枢性射精障碍)。性欲虽正常,但性交不能射精。主要起因于大脑功能异常,对性兴奋的抑制性加强,尤其是对射精中枢抑制性增强,患者无性欲高潮和射精动作。也有少数病例是因为感受器官、感情、智力障碍等造成不射精。②脊髓肿瘤、损伤(脊髓性中枢性射精障碍)第十二胸椎到第一腰椎段及骶髓段损伤,射精中枢和勃起中枢功能紊乱或衰竭,会导致阴茎勃起减弱以至完全不能勃起,射精迟缓以至完全不能射精。③传导神经障碍。胸腰交感神经切除术,腹膜后淋巴结清扫术都能损伤神经,引起不射精。④局部病变,膀胱颈松弛、精阜肥大、阴茎外伤、硬结、疤痕、纤维化、极度弯曲、严重尿道下裂、前列腺钙化和射精管阻塞也可造成精液量的显著减少。⑤垂体功能低下、甲亢、肢端肥大症、黏液水肿等也可引起射精障碍。

这种类型的射精障碍没有生殖系统的器质性疾患,纯属心理性因素所致。

部分药物可以影响射精功能。肾上腺素能受体阻滞剂、镇静剂(利眠定、甲硫哒嗪)、安眠药、抗抑郁剂等药物均可对射精功能产生抑制作用,引起不射精的副作用。具有抗交感活性的治疗高血压(胍乙啶、酚噻嗪、利血平类)、神经症(如抑郁症)、精神病的药物常常引起损伤射精功能的不良反应,如服用硫利达嗪的患者中有半数主诉精液量减少或高潮时无精液射出。根据某些学者报道,苯二甲基和胍乙啶(肾上腺素能神经阻断剂)可导致75\%的服药者出现不射精,甲基多巴(神经节阻断剂)可导致14\%的人不射精。目前,这类降压药已很少应用。引起射精障碍的药物还有乙醇、阿米替林、氯米帕明(氯丙咪嗪)、盐酸丙咪嗪、美沙酮(美散痛)、奋乃静、盐酸三氟啦嚓、利血平、酚妥拉明及盐酸氯苯苄胺。但它们的作用机制尚未完全阐明。大多数药物的作用与自主神经病理改变的影响相似,小剂量时主要影响膀胱颈的关闭,大剂量时抑制精囊等的收缩。有些药物可以抑制精液进入尿道,还有些药物则使射精反应延迟,使开始性交到射精之间的时间间隔明显延长。

混合性不射精可以由多种致病因素引起,如正在接受5-羟色胺(5-HT)再吸收抑制剂治疗的抑郁症患者,一方面抑郁症患者可能出现的性冷淡、畏惧与他人交往或者对疾病的过分关注等均是性功能障碍的易患因素,另一方面5-HT再吸收抑制剂对射精中枢产生抑制而易产生不射精。

患者有正常的性欲,性兴奋状态下阴茎能够完全勃起并能维持,阴茎在阴道中抽动始终达不到性高潮及不能从尿道外口射出精液。

功能性不射精者有梦遗或者经自慰能够射精,器质性不射精者无梦遗并且自慰也不能射精。有些患者曾经有正常的射精功能,但为了避孕长期采用体外射精方法,久而久之成为习惯,等到妻子希望怀孕时才发现其不能再在阴道中射精。

有些患者则必须采用某些特殊性交方式(性交姿势)才能射精;有些患者对性交条件有特殊要求(境遇性不射精),否则不能射精;有的患者在性生活中不习惯妻子说话,妻子总是在接近性高潮时情不自禁地笑出声来,导致他不射精。

医源性不射精并不少见,如大脑侧叶病变或切除、脊髓损伤、腰交感神经损伤或切除、盆腔手术,又如睾丸肿瘤行腹膜后淋巴清扫术等均可导致神经传导中断。损伤较轻的患者可在术后一段时间恢复射精功能。

临床上有性欲低下合并勃起困难和不射精情况,对此应作详细分析,判断真正病因及主要矛盾。

不射精的诊断主要依据患者性交中不能达到性高潮和无精液射出的病史,性交后的尿液检查中基本无精液成分者则可诊断。对不射精患者,医生除详细询问病史外,还要进行相关检查,这对不射精的诊断很有必要。医生应仔细检查阴茎发育状况,睾丸的大小、质地和有无触痛。为排除泌尿生殖系统疾病、内分泌疾病、精神病、神经系统及手术等原因引起的射精痛,应进行相应检查,如前列腺液常规、尿液常规或细菌培养,以排除前列腺及生殖道炎症。对于原发性不射精者,还应检查双侧附睾、输精管及精囊,进行排泄性尿路造影和输精管、精囊造影,甚至CT或螺旋CT检查,明确是否有先天畸形存在。

正常的性交行为中不能达到性高潮,且无射精动作,无精液流出。对于没有性交经历而从未梦遗或者无自慰射精行为的应该引起重视,但不能轻易诊断为不射精。由于性交中阴茎勃起不能维持而导致的性交失败也不列为不射精。

通过体格检查可以发现部分不射精的发病原因及协助鉴别诊断。体格检查应注意全身发育状况和第二性征的发育情况。特别要注意阴茎是否短小,是否有包皮过长,有无阴茎弯曲、尿道下裂等病变,而双侧睾丸发育不良、附睾结节、输精管缺如、前列腺发育不良、前列腺炎症引起的前列腺结节形成、前列腺增生等都可以影响精液的形成及排出,在体检时要尤其注意。

检查血生化以了解肝、肾功能和血糖水平;检查血的黄体生成素(LH)、卵泡刺激素(FSH)、催乳素(PRL)、雌二醇(EZ)、睾酮(T)水平以判断性腺功能,如结果异常必须进一步采取相应检查,如催乳素水平异常增高伴继发性不射精要注意是否有垂体肿瘤存在,应该增加垂体影像学检查。

输精管造影可了解输精管是否畸形或闭锁;通过顺行性膀胱尿道造影,可以了解膀胱颈是否增宽、前尿道是否狭窄;膀胱镜检查可以了解精阜状况;多普勒超声波检查可以了解精囊是否扩张、缺如。

检验性交后的尿液中是否有精子和大量果糖,以鉴别逆行性射精与不射精。

阴部诱发电位测定和阴茎震动感觉度测定评价阴茎背神经向心性传导功能及脑神经中枢兴奋性。阴茎勃起功能障碍可以影响射精,所以评价勃起功能也很重要。常用的检查方法有夜间阴茎勃起测定、阴茎海绵体血管活性药物注射试验。

并非每个患者都需要作所有的辅助检查,应避免对功能性不射精患者进行繁杂检查,以减少患者不必要的负担。

性欲的高低直接影响射精中枢的兴奋度,性交过程中射精中枢的反应阈值比勃起中枢的反应阈值高,在阴茎勃起之后通过增加性刺激的时间和强度才能引起射精。有些患者对待特定性伴侣可以射精或者不能射精的现象反映性欲程度与外部条件有关。

心理性阴茎勃起和性交活动首先必须有性神经中枢的兴奋,随后进人性高潮、射精等阶段。对于主诉不射精的患者,首先要了解他的性欲、勃起状况、性交动作、阴茎在阴道里的勃起维持时间;对于主诉勃起功能障碍合并不射精的患者,其诊断和治疗的重点应该放在勃起功能障碍。

逆行射精也表现为性交过程中无精液从尿道外口排出,所以被列为不射精的常规鉴别疾病。正常射精时尿道内口闭锁以防止精液向膀胱逆流,虽然逆行射精患者性交过程也有泄精、性高潮、射精、欣快感,但由于尿道内口关闭不全,导致精液逆行射入膀胱,所以性交后的中段尿液检查可以发现精子和果糖。原发性逆行射精较为罕见,继发性逆行射精可见于前列腺电切术后、经尿道膀胱颈切开术后、尿道外伤、糖尿病等。不射精在性交时无性高潮、射精动作和射精快感,性交后尿液中无精子和果糖存在,可与逆行射精鉴别。

干性射精有别于不射精和逆行射精,干性射精在性交中仍有射精动作和欣快感,但是在性兴奋时无精液排泄到尿道,性交后的尿液检查不能发现精子和果糖。其常见原因有:前列腺摘除术后、经尿道前列腺电切术后、慢性前列腺炎、前列腺结核、前列腺肿瘤、前列腺囊肿等造成精道堵塞,或者射精管的变异。也有人将逆行射精纳入干性射精。

射精疼痛有时合并阴茎勃起疼痛,当性兴奋或射精时阴茎根部或会阴部出现疼痛,因此被迫中止性交,有些患者在梦精时痛醒。射精疼痛的病因有精囊炎、前列腺炎、前列腺结石、附睾炎、尿道狭窄等,由于射精时疼痛,患者畏惧射精而可能进一步发展成心理性勃起功能障碍或功能性不射精。

射精迟缓是由于药物、手术、外伤、神经病变等原因造成脊髓射精中枢兴奋性减弱、神经传导功能障碍所引起,性交时可以射精但需要较长时间和较强的刺激方能诱发射精,其症状较不射精为轻。射精迟缓也可出现在不射精的初始阶段或恢复阶段。

射精无力即射精时精液似流出而非射出,缺乏欣快感,此症多发生于精囊炎、前列腺炎、尿道炎、疲劳及其他慢性疾患。射精无力的原因系射精前精囊腺、前列腺、尿道处未能积储较高的压力或者会阴部的横纹肌收缩无力而致。

即每次性交射精时,后尿道的精液未能完全排出,而致射精不完全,其病变多与精神心理因素有关,故多为功能性。

性交时有性欲高潮出现,也可有射精动作,但无精液排出,亦无遗精史。

性交姿势不当或者阴道狭窄导致阴茎无法插入阴道所引起的不射精。

慢性前列腺炎是目前比较常见也较难根治的男性泌尿生殖系疾患,偶尔也可以是不射精的病因,也可以发生于不射精之后,对同时合并有这两种疾患的患者来说,要确认两者的因果关系有一定困难,临床上易将不射精诊断为慢性前列腺炎。对这类患者的诊断首先要详细了解不射精的发生时间,是原发性还是继发性,如果不射精症为慢性前列腺炎引起,则一般情况下不射精的症状将随着慢性前列腺炎的轻重缓急而变化。

不射精治疗的关键在于明确不射精原因和原发疾病。由于性功能障碍涉及患者的隐私,患者就诊时可能难以启齿,也可能对性交过程表现的各种功能障碍概念认识模糊而在主诉时表达不清,医生在接诊时要耐心听取患者叙述,以取得患者的信任,必要时也要听取患者配偶对性交细节的描述。对于某些功能性不射精病例而言,接诊开始就是治疗的开始。

心理治疗主要是消除心理障碍,明确告诉患者功能性不射精只是心理上许多抑制因素造成的,有梦遗说明自身并无器质性病变,使患者相信自己的射精功能是正常的,从而增加患者的治疗信心。有些患者内向腼腆,面对医生不敢讲出内心困惑而增加医生详细询问的难度,因此患者妻子的主动配合显得尤为重要。要让患者妻子了解一些性知识,包括性交过程中的性心理和可能表现的性行为,如性交中男女双方都希望对方能够了解自己的需求,但常常又不愿直接将自己的需求明确地告诉对方,这样会影响性交质量。医生应协助患者夫妻建立性心理沟通机制,对治疗包括不射精在内的性功能障碍很有益处。

从临床角度讲,无论是心理分析疗法还是性行为疗法,都对心理性不射精症的治疗存在某种偏见和误解。不射精症的治疗是整个性治疗领域中最具有强制性的,需要直接针对具体症状,其标准治疗措施是以妻子对患者阴茎施加强有力的、强制性的、快节奏的手淫摩擦刺激开始,在临近射精阈值时插入阴道,继续抽动直至射精为止。安排治疗步骤的理由是:既然高潮反射受到抑制,那么就需要高强度刺激打破僵局,不要试图去调节不射精症患者“开小差”的想法。手淫达到高潮的能力相对来说是比较简单、标准的行为治疗的基础。若手淫射精也困难,症状就多了一个,其治疗方法也就大不相同了,在对不射精症的治疗时需要把手淫不射精排除在外。因此,不射精症的诊断应该是“患者只能在自我刺激时(手淫)才能达到性唤起,而在异性或同性性行为时不能兴奋和达到性高潮,这是鉴别诊断的重要指征。不射精症患者几乎全都喜欢手淫,而且手淫给他们带来的乐趣要远远强于伴侣间的性活动。

治疗目的是消除高级神经中枢对射精反射的抑制,消除过去形成的与不良意外事件相联系的消极条件反射。其基本原理是应用治疗恐惧症的“系统脱敏”的基本策略,但从技术上说它又有明显不同的特点,不射精治疗中采用的情境脱敏方法是由夫妻双方在自己家中执行,而治疗恐惧症则在医生的诊室内执行。性治疗中患者获得充分放松的方法是性幻想,其他治疗则利用肌肉放松的方法。治疗的首要目的是揭示出患者出现消极抑制的特殊因素,然后逐渐系统地缓解维持抑制的消极意外事件。为了达到这一目的,就要指导患者和他的妻子进行一系列旨在使患者逐步去条件或“脱敏”而设计的性治疗作业。要想指导患者获得成功,就要先详细了解患者的性历史:

(1)找出影响其射精或干扰他性表达的任何具体原因;

(2)如果患者对性活动很感兴趣的话,试图找出他所喜欢的性爱方式和性幻想内容是什么性质的,以便在脱敏练习中加以利用;

(3)他目前和过去能成功射精的环境是什么?大多患者是很少完全受抑制的,在治疗中必须依赖和利用他现有的任何射精能力。首先指导患者在性生活中注意在他最适宜的环境和方式下达到射精。从这个基本点出发,逐步使他的行为实现向阴道内射精的方式转移。治疗安排的目的在于促使患者洞察任何非理性的恐惧、创伤记忆、能增强各种抑制作用的破坏性人际作用,这些问题可能在首次就诊时就已了解到,有些则是在完成家庭作业时暴露出来的。了解配偶双方对性治疗作业的反应可以揭示出应该立即作为治疗调停指征的对性治疗的心理障碍与阻抗。在治疗疗程的安排中,要努力解决成为阻抗的内心的和交流方面的问题。除这些基本原则外,治疗方案不应遵循刻板的程序,要使之符合患者的特定情况。这种因人而异的安排有助于治疗的进展和成功。

在治疗的前二三天内指导患者与妻子一起进行任何形式的性活动,原则是不要试图使男方射精,也不要插入阴道。他们可以互相爱抚,互相进行手或口的刺激,当然可以使妻子产生性高潮。如果这些无射精体验导致性唤起和性爱乐趣的增强,则指导他们重复这些体验,不过不再限制患者的射精,但这一阶段的射精应在他最易于成功的情境下进行。例如,如果他只能在自己独处时才能射精,那么就让他在与妻子嬉戏爱抚达到高度性唤起后,离开卧室自己通过手淫获得高潮,但他应努力使与妻子做爱和射精之间建立起联想,而排除任何消极的无意识期望。如果他的抑制程度有所减轻,便可当着妻子的面或与妻子一起进行手淫刺激,当他感到紧张或性唤起程度有所降低时,建议他在进行刺激时应用性幻想以分散自己的紧张与焦虑。当上述步骤都进展顺利,便可进入下一治疗安排:由妻子用手或口刺激他射精。马斯特斯和约翰逊,以及卡普兰等其他性治疗学家一致认为:在对不射精症的治疗中,由妻子以各种方式引起的射精是治疗的一个决定性的里程碑。开始时,用性幻想来分散注意力是有益的。然后,借助阴道分泌物作润滑,在靠近阴道口处进行阴茎刺激,当妻子刺激他的阴茎使他临近性高潮时,指导他把阴茎插入阴道放置一段时间。如果他并未立即射精或失去勃起就退出阴道,则由妻子重新进行手淫或口刺激,直到高潮再次迫近时再度插入,到能在阴道内射精为止。下一步的治疗安排是性交与手淫刺激的结合,即在插入后的抽动过程中由妻子伸手在阴茎下面配合进行手对阴茎的刺激,这样能增强对阴茎的刺激,并逐渐使患者适应在阴道内通过抽动时的摩擦刺激而达到性高潮。如果不能奏效,说明需要进行更广泛的心理、社会、生理的干预。

在治疗初期,妻子的直接性需要可能暂时受到忽视,因为性活动必须首先服从消除丈夫性抑制的治疗需要。如果女性性心理相当成熟,能够抵御暂时的挫折而没有敌意,且夫妻关系原来就很好,足以使她的慷慨行为得到心理的补偿,这样,性治疗就能顺利地进行下去。但是,由于丈夫性功能的进行性改善和她需要为丈夫“服务”的事实可能引起妻子内心的抵触,这些抵触可以表现为不易为人们所觉察的对性治疗的破坏企图。治疗学家必须对此保持足够警惕,一旦有苗头,就应集中精力解除这些阻力和障碍。患者也可能对治疗产生一定的焦虑和阻抗,表现为性唤起到平台期晚期时出现的惶恐和疑惑,担心自己能否“表演得出色和成功”,这就需要让他进行性幻想或驱散这些分心的念头,集中精力使自己沉浸在性感觉中。此外,还要注意解决夫妻关系中存在的严重问题,以免成为治疗的阻力。

尽管性交是一种无师自通的本能活动,但是并非人人均能操作完美,医生可让不射精患者或性伴侣了解性交过程的相关细节。对性知识缺乏的患者,要向他们介绍有关男女生殖系统的解剖生理知识,告诉患者在阴茎插入阴道后,必须通过抽动阴茎来提高射精中枢兴奋度以引起射精反射。要求患者戒除不良的自慰习惯。夫妻在性交前可先通过语言交流,互相抚摸敏感部位,待双方较为兴奋时再行阴茎插入。如果射精有困难则不要勉强延长性交时间,以免产生负面的心理影响。性幻想偶像的方法可以提高射精中枢的兴奋性。指导患者尝试合适的性交体位,如阴茎短小、阴茎弯曲、阴道解剖异常的人可能需要某些特殊体位才能获得满意的感受。

利用电动按摩器刺激阴茎头和系带诱发射精对部分病例有效:由于电动按摩器对阴茎的机械刺激强度低于自慰而高于阴道摩擦,所以对于有自慰习惯的患者,通过停止自慰使用电动按摩器诱导患者体会低强度的刺激,同时结合幻想可以收到较好效果。

对于阴茎本身病变如包皮过长、包茎引起的功能性不射精,可采用包皮环切手术,使阴茎头外露以感受性刺激而促进射精反射。

(l)麻黄碱(Ephedrine):麻黄碱是从中药麻黄中提取的生物碱,现可人工合成,其作用与肾上腺素相似:麻黄碱能够兴奋大脑皮质和皮质下中枢,可使部分性欲低下患者提高性欲,由于麻黄碱是肾上腺素能受体兴奋剂,可以使交感神经节后纤维释放儿茶酚胺,增加输精管道平滑肌收缩,促进射精。此外麻黄碱可以提高膀胱颈括约肌张力,可用于某些逆行性射精。用法为每次25~50mg,临睡前口服。可以配用左旋多巴以提高疗效。高血压及冠心病患者禁用。

(2)左旋多巴(1evodopa):左旋多巴是体内合成多巴胺的直接前体,也是合成去甲肾上腺素、肾上腺素等的前体:左旋多巴可以透过血脑屏障,在脑内转化为多巴胺。左旋多巴作用于性功能方面能激发性欲,促进射精,临床上用于治疗高位射精中枢异常和无性欲高潮的射精障碍。左旋多巴可以单独使用或与溴隐亭合用以加强作用。用法:开始每次0.1~0.259g,每日2~4次,每隔一周增加。每日0.1~0.759g,可根据患者的耐受情况增减剂量,成人维持量为每日2.5~5g,分4~6次服用。禁用于急性精神病、溶血性贫血、孕妇等。肝肾功能损害、心肺疾病、惊厥病史者慎用。

(3)中医中药(为辅助治疗,略)。

新斯的明(普鲁斯的明,Neostigmine)注射液1mg,肌注,每天1次,连用5天治疗不射精。

器质性不射精中比较多见的是医源性不射精,由于损伤部位、程度的不同,其临床表现及预后也有所不同:凡与射精有关的中枢神经、末梢神经、效应器等的外科损伤或某些药物副作用都可以引起不射精。对医源性不射精的防治最重要的是在医疗行为(手术治疗)中如何有效地保护各类神经和精路,特别是腹膜后淋巴结清扫术,术中要保护和尽量保留胸腰部交感神经干、内脏神经、上下神经丛、下腹神经、骨盆神经丛、精路和下尿路。

对于由垂体病变等引起的易复性器质性不射精的治疗,首先要治疗原发性疾病。可首先试行电动按摩阴茎体和阴茎头,这种治疗的副作用很少,但是效果欠佳。另外可向蛛网膜下腔缓慢注射新斯的明0.25~0.5mg,大约3小时后诱发射精,然后收集精液行人工授精。这种治疗的副作用是有可能出现一过性高血压、头痛、恶心、呕吐,治疗过程要保留静脉点滴管道、心电监护、准备心脏急救用药。由于蛛网膜下腔注射新斯的明可引起较多副作用,所以有人采用电器刺激获取精液,方法是将一双极电极贴在术者食指末节指腹,在食指引导下插入患者肛门,电刺激前列腺和精囊腺,引起射精:电流频率100Hz,刺激幅2毫秒,电压2~15V。通过电刺激获取精液后行人工授精。

在医生指导下,无论心理性或器质性不射精均可使用电动按摩器诱发射精,常可获得较好的效果。据报道,有半数左右的患者在首次治疗中即可恢复正常,而其余的人通过十余次治疗也能痊愈。开始时可能需要持续刺激10~15分钟,以后每次只要5分钟即可达到射精目的。刺激部位以龟头、系带处为主,也可沿阴茎干上下移动,市售的各种型号的保健按摩器均可选用。

对症治疗包括包皮环切术、戒烟酒、改善居室环境和增强体质等。

禁用噻嗪类安定剂和某些降压药,如胍乙啶、利舍平、α-甲基多巴等。国内江鱼等报道,采用左旋多巴0.25g/次,每日3次,能抑制泌乳素水平并增加血循环中的生长激素和肾上腺素水平,从而达到兴奋大脑皮质的作用,常能收到一定效果。另据报道,性交前1小时口服50~75mg盐酸麻黄碱,可使肌肉张力增加,由于中枢神经系统兴奋,射精也容易些,常能取得一定疗效,但高血压或冠心病者禁用。此外,HCG注射或口服右旋苯丙胺也可试用。一时无法治愈,而又急于生育者,可通过手淫采精或收集遗精时的精液注入阴道内,这一切应在女方排卵期进行,可有一定疗效;或采用辅助生育技术助孕。

脊髓损伤患者的射精刺激技术。脊髓损伤的男性往往不能射精,或者不能在预想的时间射精,所以需要建立人工刺激射精的方法,以便在需要时射出精液。这些技术包括电射精、鞘内注射新斯的明、振荡器刺激和皮下注射新斯的明。

1.电刺激射精,使用电射精器(具体略)。

2.药物刺激射精(鞘内注射新斯的明、麻黄碱、左旋多巴。具体略)。

3.刺激方法的选择,刺激的研究往往存在种种设计上的缺陷:①研究对象的特征;②所用刺激和技术的详尽描述;③成功的定义和描述;④观察到的并发症的描述。没有任何一种刺激射精的方法能取得普遍成功,或在运用中不遇到困难,所以,在将来的研究工作中应努力解决这些问题。

不射精患者多以不育症就诊,取出精液后可选择辅助生殖技术帮助受孕,以精液参数选择不同的辅助生殖技术。如果精液的参数值偏低则选用体外受精(IVF)和显微注射法受精(ICSI)治疗;如果精子的数量和运动性能较好,则可进行多次子宫内人工授精。

1.可诱发勃起功能障碍 性交中断,但大脑皮质的性中枢、各路性活动控制神经以及各个性器官依然处于兴奋状态,使其快速松弛无形中就加重了神经系统与性器官的负担,造成一种“过度疲劳”现象。

2.可诱发射精异常 经常发生性交中断强烈地抑制不射精,会酿成射精障碍。轻者出现射精时间延迟、射精不畅,重者就不再射精。

3.正常性行为高潮射精后,阴茎勃起很快消退,几分钟内阴茎的血液减少50\%~60\%,随后10~20分钟,性器官血流状况恢复常态。无菌性前列腺炎患者如果是性交中断,性器官血液的复原速度大大减慢,性器官会处于长久的充血状态,易形成血栓。

4.可诱发血精 与上述性器官充血消退较慢同样的道理,精囊广泛与持久充血,精囊壁上的毛细血管会扩张破裂,导致血精。

5.可诱发频繁遗精 性交过程中,随着性冲动的发生,各附属性腺分泌增多,精液量骤增。如果中断性交,这些精液没有去处,必然会通过遗精方式排出体外,易产生精液与诱发频繁遗精,对身体健康不利。

不射精对男性身体造成的危害是不可估量的,建议有此病症者及时到医院进行治疗。


\section{第八节 逆行射精}

逆行射精(Retrograde ejaculation)是指男性性交达到性高潮,出现射精动作和感觉,精液不循常道外射,而从后尿道逆向后流入膀胱中,不从尿道口向前射出的病症,是男性不育的原因之一。性交后第一次排尿的中段尿液化验发现精子和果糖。有报道,本症占不育人群的0.3\%~2.0\%,在无精子症患者中可高达18\%。现代医学认为,逆行射精病因多见于神经损伤、泌尿生殖道损伤、糖尿病等内分泌疾病及阻滞肾上腺素能神经作用的药物。逆行射精并不少见,特别是近年来各类经尿道前列腺手术的开展和普及,在迅速解除患者排尿困难的同时也增加逆行射精的发生率。

国内报道经尿道前列腺电气化术(Transurethral Vaporization of Prostate,简称TUVP)后逆行射精发生率高达50.5\%,国外报道经尿道前列腺切开术(Transurethral incision of prostate,简称TUIP)后逆行射精的发生率为6.6\%。逆行射精对于老年患者除在心理上难以接受外,不会造成身体的严重不适,但对年轻患者来说主要是影响生育和精神上的打击。因此,对年轻患者施行经尿道前列腺手术应更谨慎。

在性交过程中随着性欲递增,即有精液泄入后尿道,在达到性欲高潮时出现射精。正常射精时,尿道内括约肌在神经系统控制下紧紧关闭,而尿道外括约肌开放,其目的是防止精液逆行进入膀胱,顺利射出尿道。各种异常原因引起的膀胱颈部在交感神经支配下关闭功能障碍,尿道膜部阻力增高,当坐骨海绵体肌、球海绵体肌及盆腔横纹肌节律性收缩时,精液则流向压力低的膀胱内而造成逆行射精。

先天性宽膀胱颈,先天性尿道瓣膜或尿道憩室,先天性脊柱裂等疾病使得膀胱颈关闭不全及尿道膜部阻力增加,造成逆行射精。

创伤或外科手术损伤交感神经,如腹膜后淋巴结清扫、腹主动脉瘤切除、直肠切除等,可导致膀胱颈部尿道内括约肌功能丧失。糖尿病累及交感神经病变,导致尿道内、外括约肌功能共济失调,射精时尿道内压高于膀胱内压,导致精液排入膀胱。

主要包括各种膀胱颈部和前列腺手术,前列腺手术或经尿道膀胱颈阻塞切开术等损伤了膀胱颈部的肌肉和弹性纤维,膀胱颈部松弛,可致逆行性射精。另外,胸腰部交感神经切除术,腹膜后广泛淋巴结清除术及其他的盆腔手术,导致了神经根切除或损伤,使膀胱颈部关闭不全,发生逆行射精。前列腺切除术、膀胱手术或创伤改变了膀胱颈的完整性,特别是前列腺增生患者接受经尿道前列腺电切手术,术后大部分患者可出现不同程度的逆行射精。尿道外伤、手术创伤、尿道炎症等原因导致的尿道狭窄、精阜囊肿、先天性宽膀胱颈、先天性尿道瓣膜、先天性射精管开口异常等解剖学改变均可出现逆行射精。外伤性及炎症性尿道狭窄由于尿道阻力增加,导致射精时精液受阻。外伤性骨盆骨折常可引起后尿道损伤导致狭窄,同时骨折片又可破坏膀胱颈部的结构,致膀胱颈关闭功能不良造成逆行射精。另外,长期排尿困难亦可使膀胱颈部张力下降,导致关闭无力。

糖尿病可并发逆行射精,严重尿道狭窄、膀胱结石、膀胱炎、尿道炎等,也可以引起逆行性射精。

服用a-肾上腺素能受体阻滞剂,如利血平、胍乙啶、苯甲呱及溴苄胺等都可引起平滑肌收缩无力而出现逆行射精。抗高血压药物、抗精神病药物,如利血平、胍乙啶、溴苄胺、盐酸甲硫达嗪等,为肾上腺素能神经抑制剂,使交感神经兴奋性降低,妨碍了支配膀胱颈的交感神经冲动而影响尿道内括约肌收缩,导致精液逆向流入膀胱,产生逆行射精。

部分原因不明。中医认为是先天不足,后天失养,肾气亏虚,阴阳失调,司精无权;或久病体虚,脾肾亏损,固摄失调,膀胱不约,精液不循常道排泄,即是体质虚弱和神经功能调控紊乱所致。

性生活、梦交或者自慰中,患者尽管能够达到性高潮和体验射精的欣快感,但是没有或几乎没有精液射出:有的患者在一段时间里可觉察到精液量随性生活次数的增多而逐渐减少直到消失;有些缺乏经验的患者误认为性兴奋时,尿道外口不自主流出的黏稠清亮的尿道旁腺和尿道球腺液是精液,因不育症而就诊时才被诊断为逆行射精。因此在病史询问时要详细,并注意判定患者的表达能力,同时向患者性伴侣了解性生活情况。大多数患者能发现性交后第一次中段尿液比较混浊和尿停顿感。并详细了解患者的既往史(糖尿病、神经系统疾病)、外伤史、手术史(尿道、膀胱手术史)和服用药物史。

一般在采用手淫方法获得性高潮、射精动作和感觉后,立即取中段尿液镜检,本症可作初步诊断,这种检查一般医院均可进行。

正常性交或手淫出现性高潮和射精感后,未见精液射出尿道外口,在随后的第一次排尿中发现尿液中有精液状浊物。本症明确诊断容易,关键是确定病因。在正常的夫妇性生活中或男性在其他性行为(如手淫、口交等)的性刺激达到性高潮,有正常的射精感觉,但几乎没有或没有精液射出尿道外口;或结婚后妻子长时间没有怀孕,就有可能发生了逆行射精或其他方面的毛病。

逆行射精患者体格检查一般无特殊征象,应注意患者全身发育状况和第二性征的发育情况。注意阴茎是否有外伤畸形、包皮过长、包皮炎、阴茎头炎,有无阴茎弯曲、尿道下裂等;注意双侧睾丸大小、副睾是否结节、输精管是否缺如;前列腺检查了解前列腺大小、有无触痛、结节等。

检查性交或者自慰有性高潮和射精动作后的首次中段尿液,可见大量的精子,必要时离心尿液找精子。如在性交后的中段尿液中发现精子可考虑为逆行射精,但应排除少精液者。

如尿液中无精子,可测定尿液的果糖浓度。嘱患者自慰或性交前排空小便,射精后即检测中段尿液的果糖浓度,必要时与第二次排尿的果糖浓度比较。如明确无精子射出,且在性交后第一次中段尿液中的果糖浓度明显高于第二次尿液可诊断逆行射精。

前列腺液常规或培养,了解膀胱、尿道、前列腺等是否有炎症或感染,对于以往曾有过射精者具有重要的意义。性交后的前列腺按摩是否能取得前列腺液,并注意前列腺液的多少,对判断患者是否无精症与鉴别诊断有一定的帮助。

多普勒超声、彩色B超、CT、MR、放射线检查或其他影像学检查,进一步明确疾病的性质,了解膀胱、尿道及有无泌尿系结石等,有无泌尿生殖系感染。明确生殖道的解剖有无改变,找出无精液射出的原因及疾病性质。需注意射精管是否松弛扩大或缩窄,精囊腺和前列腺是否正常,了解膀胱颈的结构及有无生殖道肿物。对有过正常射精者具有重要的意义,并对本症诊断、鉴别诊断及采取何种措施治疗和预后估计具有指导作用。

血生化检查,特别注意患者是否有糖尿病。检查血性激素了解性腺功能。

逆行射精同时伴有睾丸发育不良、输精管缺如或堵塞、射精管堵塞等疾患,患者射精后的尿液中无精子。这种情况可以通过尿流动力学检查尿液中果糖浓度鉴别。

有的患者射出体外的精液量极少,误认为无精液排出,不能单凭在射精后的尿液中发现精子就诊断为逆行射精。由于首次排尿时前尿道内残余的精液可能引起误诊,因此诊断逆行射精前最好向性伴侣了解性交情况,或者建议使用避孕套了解是否无精液排出。

干性射精是指患者有射精感觉但是没有或者很少有精液产生。多见于前列腺摘除术、前列腺电切术后的患者。尿液精子检查和果糖检查可做鉴别。

性兴奋时,尿道旁腺、尿道球腺分泌黏液,加之阴茎勃起对海绵体的挤压,这些腺液可以在射精前流出尿道外口。逆行射精患者性兴奋时也可能有腺液流出。

逆行射精的治疗包括:药物治疗、手术治疗、心理治疗、人工授精。前两种治疗方法是为了恢复射精功能,后一种方法用于助孕。逆行射精治疗可根据发病原因选择一种或者联合几种治疗方法。

药物治疗适应膀胱颈部解剖结构完整,非梗阻因素的神经、肌肉控制障碍,包括因糖尿病引起的逆行射精。患有慢性膀胱炎、慢性尿道炎、慢性精阜炎者,应注意个人的性行为卫生,在医嘱下使用抗生素。选用具有刺激膀胱颈部α肾上腺素能受体,增加膀胱颈部收缩关闭能力,防止精液逆行射入膀胱的药物。

麻黄碱为肾上腺素能受体兴奋剂,可增强精道平滑肌的收缩,对射精有促进作用;可改善膀胱颈和后尿道平滑肌功能,治疗糖尿病患者的逆行射精。用法:每次25~50mg,每日3次,口服。

盐酸米多君的活性物质可选择性地刺激外周。肾上腺素能受体,刺激膀胱颈部的肾上腺素受体,导致膀胱出口阻力增加,治疗逆行射精。用法:每次2.5mg,每日3次,口服。

为三环类抗抑郁药物的代表药物,可阻止神经末梢对去甲肾上腺素的重吸收,从而增强肾上腺素能活性。常用量为每日口服25~50mg。

国外有用甲氧明治疗糖尿病性逆行射精的成功个案报道。用法:每次性交前30分钟患者自己肌肉注射甲氧明5~10mg,3个月后妻子怀孕。

解剖异常导致逆行射精的,可采用手术治疗方法。

该手术适应于先天性膀胱颈过宽、曾经施行过膀胱颈手术导致关闭障碍的患者,不适应于神经性膀胱颈关闭障碍症者。手术前必须全面考察患者的排尿功能,衡量手术利弊。

尿道狭窄的处理包括定期尿道扩展术、尿道疤痕电切术。

由于精阜增大引起的逆行射精可将增大的精阜切除。

逆行射精对患者构成不同程度的心理压力,同样也会对患者妻子造成打击,年轻夫妇甚之。如患者夫妻对逆行射精对其生育能力影响很在意的话,产生的心理影响更为明显。有的会出现焦虑抑郁、有些出现家庭纠纷或夫妻感情破裂、有的出现其他性功能障碍。精神的创伤也能影响睾丸生精功能。医生应重视患者的咨询,详细询问病史,对患者采用各种药物和手术治疗的同时应该对患者夫妻进行心理疏导,明确告诉患者逆行射精依然可以保持夫妻正常性生活,也通过辅助生殖技术受孕,尽量减少或控制患者的精神症状,减少家庭纠纷。

对于育龄患者在药物和手术治疗未能见效的情况下,应该考虑如何从尿液中收集精子进行人工授精。由于精子与尿液接触5分钟后其活力降低50\%左右,减少精子与尿液接触时间、调整尿液渗透压和pH值及减少尿液中氨的含量是回收高质量精子及提高受孕率的关键。

(刘捷 陶林)


\chapter{第九章 女性性唤起障碍}

性唤起是指男女两性分别为准备性活动而发生的生理变化。男子性唤起时以阴茎勃起和睾丸提升为特征,俗话说“变硬”;而女子性唤起则经历一个更广泛和更复杂的生理变化模式,它包括盆腔充血、阴道润滑、外生殖器肿胀、阴道管外1/3变窄及内2/3增长和扩宽、乳房肿胀和乳头勃起,通俗而概括地说是“肿胀”和“变湿”。在所有困扰女性的性功能障碍中,女性性唤起障碍(female sexual arousal disorder,FSAD)是最不受重视的。事实上,在临床实践中的确有不少人认为FSAD是很难作为一个单独的、孤立的性功能障碍来看待的,在他们的门诊中有时仅占不足2\%的比例。性学研究给予的关注,大多数集中到更容易定量研究的性欲和性高潮问题。虽然我们尚不清楚FSAD的实际发生率,但它无疑困惑着各个年龄阶段中相当比例的妇女。美国芝加哥1994年性调查结果表明在18~59岁的妇女中有19\%的人存在阴道润滑困难,而绝经后妇女中更高达44\%。

尽管如此,仍有少数学者在近十年来对其潜在的生理或心理因素进行过一些研究,人们曾把FSAD归咎于雌激素缺乏、心理抑制或其共同作用。所以对于国外那些可能没有医学背景的性治疗学家或全科家庭医生来说,要么把患者转给妇科医生以接受雌激素替代治疗,要么转给可能根本不懂得性问题的心理医生。既然人们对FSAD的病理生理机制并不清楚,所以临床上几乎没有可供医生和患者选择的、有效解决FSAD的药物或其他治疗手段也就可以理解了。人们对FSAD的兴趣不高也是必然的。性唤起障碍的危害在于可以导致性交疼痛、性回避、婚姻或性关系障碍。

因为与男性勃起功能障碍同属性兴奋期的问题,所以也有人把它简称为女性ED。只是男性勃起功能障碍(ED)却在近1/4世纪备受关注,成了性医学研究和性干预的主要目标、始终占据着中心位置,并在近十几年里取得重大突破。西地那非(万艾可)等磷酸二酯酶抑制剂的成功使人们开始对FSAD抱有日益增长的兴趣和希望,若能在这一问题上取得进展必将造福于许多女性、夫妻和家庭。


\section{第一节 女性性唤起的解剖和生理}

FSAD与当前热门的男性ED是一个对等的医学问题。过去曾有人质询为什么女性性功能与男性相比会有那么大的差别,现在可以看出实际上二者相差无几,非常相似。有人把FSAD称为阴道肿胀和阴蒂勃起不足综合征不是没有道理的。涉及FSAD的解剖学的最关键问题是:在生殖器/盆腔的神经、血管、肌肉群这一大环境之中阴蒂与阴道的相互关系。由于向内延伸的阴蒂脚会通过阴道周围结构而内陷并隐藏起来,这就使得人们很难对阴蒂和阴道的生理反应作分别测定,这也使女性性研究变得更为错综复杂。阴蒂头和体加起来才2~4cm长,而阴蒂脚或基部足有5~9cm长。阴蒂体由两条阴蒂海绵体组成,外有纤维膜包裹,就像男性一样,海绵体的40\%~45\%是由平滑肌组成的。近年的尸体解剖研究表明阴蒂海绵体平滑肌的数量将随着年龄的增长和心血管功能的减退而减少。

国外近年来的初步研究提示,在男性性反应中起重要作用的天然血管扩张物质血管活性肠多肽(VIP)和一氧化氮(NO)在女性性反应中同样具有重要作用。女性阴蒂肿胀也涉及NO,就像NO整合到男性勃起进程中一样,在阴蒂组织中能分离出NO合成酶免疫活性神经原。把来自接受肾上腺增生外科手术妇女的阴蒂海绵体平滑肌细胞进行培养并用cGMP磷酸二酯酶抑制剂西地那非或Zaprinast处理。结果经如此后cGMP水平增高3倍,平滑肌松弛增强,血管扩张。

曾认为阴蒂和阴道外前壁的神经支配同样是来自阴部神经的,性唤起和性高潮具有相同的生理机制。与此相似的是曾认为阴蒂与阴唇的扩张是作为一个统一的功能单位的。小阴唇中血管化的海绵体组织的延伸上至阴蒂、下至前庭的会阴部,包括了尿道开口和阴道口处。阴道口是被海绵体组织束紧的,在性唤起时会充血肿胀。小阴唇是被大阴唇所束紧的,在唤起时也会肿胀。

据估计,妇女单位体重所拥有的勃起组织与男性是一样多的。然而,女性的勃起是更分散、更内在,并不像男性那么外显和突出。刺激阴蒂、前庭、阴道前壁外1/3可以激发环阴道肌肉收缩的性高潮;然而,与性高潮有关的神经关联却几乎没有进行过测试。在有些妇女,阴道前壁中还普遍地验证到一个称为G点的区域和位置,而且近年来又分别发现还可能有U点和A点的存在。假如推测由阴蒂到阴道是一个连续体的话,那么总是引起争议的去人为区分阴蒂高潮与阴道高潮(或G点高潮)似乎就没有必要了。然而,之所以造成很有说服力的争议,是因为以插入为主所致的深在的对子宫阴道部位和宫颈施加的压力可以产生一种不同类型的性高潮。这种类型性高潮的特征是涉及整个身体的反应,如剧烈窒息,它不仅仅等同或完全替代阴道周围肌肉的收缩,而且可能涉及另外的神经通路。尽管存在这些区别,对短暂性高潮的关注不应该脱离开生殖器唤起的感觉和特征,这是提前出现的并是导致性高潮的性反应高峰的来源,甚至在高潮未发生时它就已经存在了。对于性唤起和性高潮的阴蒂和阴道感觉而言,它们是有着个体差异的。

尽管女性性唤起解剖的神经和脊髓成分仅在动物和骨髓损伤患者身上检测过,研究人员还是发现了一个很有前途的神经环路。从事雌大鼠研究的人员认为盆腔肌肉反射(阴部—阴部反射)与节律性阴道神经活动有关,导致生殖器唤起的增强。当给尿道周围施加压力时,这些反射沿脊髓向下传导。脊髓出现的这些反应可以受到来自大脑的肾上腺素能神经刺激而促进,和来自大脑的5-羟色胺能神经活动而抑制。通过迷走神经刺激而诱发的神经环路系统对女性生殖器性唤起具有另一种潜在的兴奋作用。由于这一从生殖器到大脑的传入通道是通过脊髓的,通过迷走神经的生殖器刺激可能允许脊髓损伤妇女性唤起与正常妇女额外的性唤起。

当前的性唤起定义是集中在生殖器事件上的。它会随着与妇女主观唤起重要的性迹象而变化,它与生殖器充血的相关性是很差的。在主诉唤起障碍的妇女中,缺乏主观体验和测量到的充血增加之间的相关性,她们在观看视觉色情刺激时记录到的阴道充血甚至比配对研究的性健康妇女的反应更为显著。只有性健康妇女发现色情刺激光盘具有主观唤起的作用。这一证据来自阴道光体积描记计研究结果,它测量了妇女在观看色情录像时围绕阴道周围的血管充血的增加,并在随后评价了主观的性兴奋。但研究一般认为许多妇女并没有在任何广泛程度上从她们的生殖器感觉去评价唤起的体验———似乎特别是在低水平或中等程度的唤起之时。对妇女而言,只有在高水平的性唤起时生殖器反馈才会变得更加重要。与生殖器反馈相比,情绪和思想似乎在更大程度上以一种不断进展中的方式调节着唤起的体验。

如果性刺激能够提供恰当的背景条件的话,那么在接受性刺激若干秒钟之后生殖器就会出现反射性的充血肿胀,这时将会出现与之密切相关的、强劲的、主观性唤起。除非支配会阴区域结构和阴道黏膜下血管丛血管充血扩张的自主神经系统受到已知的损伤,例如不保留神经的子宫切除术。许多因素会调节大脑对性刺激的加工和处理过程,从而很可能改变主观的性唤起程度。生物学因素包括疲劳、抑郁、药物不良反应、性激素活性下降,也可能包括高泌乳素血症和甲状腺功能低下。心理学因素包括日常生活造成的分心、害怕性交疼痛等不良后果、畏惧意外妊娠或性传播疾病等引起的不安全感、对不育的担忧、过去消极的性经历、对性的不安和羞愧等。


\section{第二节 女性性唤起的研究测定}

女性性唤起之所以比男性更复杂和很难于测定,完全是由其性质所决定的。当男性性唤起研究几乎始终全部集中在阴茎勃起质量之上时,过去近30年里对女性生殖器唤起的特定测量主要是利用阴道血管充血程度或阴道组织温度变化为指标的。虽然应该对阴蒂血管充血程度和敏感性进行测量,但几乎没有人作过这类测量研究。Berman等(1999)把阴蒂血流、由于润滑导致的阴道pH变化、多普勒超声等都列入了他们的研究计划。他们创造性地利用这些测量技术检测诸如西地那非等血管扩张药治疗FSAD的有效性。他们的报告表明阴道润滑和宫颈分泌会增高阴道pH值。吸水纸和棉栓也偶尔用来测量阴道润滑变化。单独测量阴道润滑,对于评价躯体性唤起程度当然是不充分的。这些阴道润滑和阴蒂血流测量的有效性和意义尚未完全明确,但这些手段的建立对于治疗绝经期FSAD妇女可能是很必要的,因为通常的阴道血管充血测量并不能显示出绝经期妇女与正常绝经前妇女反应功能的区别。

遗憾的是,通过阴道光体积描记技术(光反射)和热敏电阻这样的女性生殖器唤起标准的生理测定所得到的结果,与她们自我报告的性唤起程度不存在显著的或仅有有限的相关关系。基本上没有证据能说明客观生理测定是女性躯体性反应的真实指征或反映。此外,许多研究人员表明妇女并不能像男性那样准确地知晓她们的性反应。在客观测定显示阴道血流体积呈现最大限度生理改变的妇女中,有42\%的人声称没有感到身体反应,54\%说没有阴道感觉,63\%没有阴道润滑迹象。在女性样本中,光体积描记测定和自我报告的性唤起程度之间的相关系数只有0.4~0.7,常常是没有显著意义的。

阴道光体积描记器只是一个阴道卫生栓大小的传感器,当插入阴道后,它将发射光线至包绕阴道的毛细血管床。由于血液可以吸收光线,所以反射回来的光量是与血液体积成反比例的。阴道血流体积(VBV)的灌注梯度变化是通过一个直流信号耦合测定的。随着心脏搏动变化而发生的阴道充血的节律性改变称为阴道脉搏幅度(VPA)。VPA是通过一个交流电信号耦合测定的。虽然一般认为VPA能更好地与自我报告的性唤起相关,不过,VBV仍在研究中使用,VPA和VBV常常是一起予以报告的。令人惊奇的是,实际上是VBV测定、而不是VPA测定能显示出性功能正常和性功能障碍妇女的明显差异。

当女性受试者在暴露于色情刺激的情况下,甚至当她们对该刺激持消极、否定的评价时或她们因此而受到惊吓时,VPA在给予刺激后的10~20s内仍有可能增高。此外,有报道表明在绝经前和绝经后妇女中VPA的增加是相等的,甚至在绝经后妇女中没有使用雌激素替代治疗,并预期其阴道润滑会很差时也不例外。然而,在通过一个能测定阴道内热量改变和氧扩散的仪器评定阴道血流量时,可以发现给予那些具有阴道萎缩、性抑制和性交疼痛的已绝经的性功能障碍妇女使用雌激素替代治疗后,阴道血流量可以恢复到绝经前、性功能正常的对照组妇女的水平。显然,为了更好地诊断和治疗绝经后FSAD妇女,必须对她们进行认真和准确的生殖器唤起测定。

阴道光体积描记术的主要替代测定方法是通过夹在小阴唇上的热敏电阻来测量小阴唇的表面温度变化。利用热传导装置测量表明女性也具有阴道夜间唤起,这与男性睡眠眼快动期中的阴茎夜间勃起相类似。对女性性反应的夜间睡眠测定将能在将来运用于由于神经、血管或绝经/激素缺乏等因素所致的FSAD的诊断。与VBV相比,阴唇温度测定与自我报告的性唤起具有更高的相关关系。然而,温度测定的使用尚未普及,也未做过性功能正常和性功能障碍妇女之间的比较研究。肌电图可以用于高潮中肌肉收缩的测量但不适用于生殖器唤起的研究。


\section{第三节 性唤起障碍的定义与分类}

虽然,性唤起的生理变化不仅发生在生殖器区域之内、也发生在生殖器之外,但驱使患者前来求治的典型主诉却仅仅是生殖器区域之内性反应的减弱。美国精神疾病诊断与统计学手册第四版(DSM-Ⅳ,1994)对FSAD作了如下定义:

①女性在性活动的激发过程中乃至到性活动完成之时,仍持续地或反复地、部分或完全地不能获得或维持性兴奋期的阴道润滑和肿胀反应。性唤起反应包括盆腔血管充血、阴道润滑和扩张、外生殖器的肿胀。②这一障碍可以引起显著的痛苦和人际关系的困难。③这一问题显然主要不是由与精神症状有关的其他疾患(除了其他性功能障碍)引起的,也不单纯由于某种物质或全身性疾病的直接生理作用所致(如药物或毒品)。

DSM-Ⅳ标准把注意力集中在以生理反应“润滑/肿胀反应”不足为其特征是非常成问题的,它极少成为激励女性寻求治疗的主诉。此外,这也很容易解释成为常规上需要雌激素替代治疗以解决润滑不足的问题。“肿胀”的概念和性质与润滑相反,仅仅是在实验室里予以研究的,一般是与常规临床实践相分离的。人们推测肿胀仅仅与其引起润滑的程度有关,按照这一推论,它与唤起反应很基本的感觉(抽动、紧张、胀满)和感觉的反馈往往是受到忽视的。考虑到“适当性刺激”对性唤起和性欲的重要性,当评价是否存在一个唤起或高潮障碍时就要对此作出临床评估。事实上,对唤起来说有关的感觉和感觉反馈是否充分的调查是很少的。对与性功能障碍有关的躯体感觉的注意力一般都集中在了高潮障碍上。

即使在美国修订DSM-Ⅳ的过程中,也曾认为它很难作为一个独立的性功能障碍而存在,似乎只应把它限定于润滑不足,否则很难与性欲或性高潮障碍相区别。尽管临床已表明存在这样一种性功能障碍达20年之久,但特定的诊断标准却始终没有特别明确。1978年《新英格兰医学杂志》曾发表一篇调查报告,在100对正常夫妻中,有48\%的妇女主诉“难于达到性兴奋”和33\%的妇女主诉“难于维持性兴奋”。然而,尽管有这么多的妇女说她们经常存在与性兴奋或唤起有关的性困难,但这些妇女中86\%却评定她们的性关系“非常满意”或“相当满意”。报告结果之所以出现这种明显矛盾的重要原因是:尽管妇女事实上基本未能达到性唤起但性交仍可发生,因此她们觉得还算满意;而若男子不能充分唤起或勃起,性交则根本不能发生。值得一提的是,这项调查中只有15\%的丈夫认为或觉察到他们的妻子存在这一问题。

虽然人们总是把唤起当做是润滑的同义词,有些研究却表明在测量阴道/阴蒂唤起反应和润滑反应时存在的差异。阴道反应水平的测量研究已经发现,在测量到的血管充血程度和主观对躯体性唤起评价(自我报告)之间的重大差别。此外,当注意力集中到血管充血和(或)润滑时,往往容易忽略阴蒂唤起的作用,其实,它才是大多数妇女主观躯体性兴奋/唤起的主要来源。一个尚不能满足的技术需求是如何准确测量阴蒂反应,它与男性勃起反应测定是相互对应的,我们需要了解阴蒂自身的反应和它与阴道反应之间的相互关系。

在DSM-Ⅲ-R(1987)中曾经包括“缺乏对性唤起的主观感觉”(性感缺失)。也就是说无论个体在性接触过程中是否发生正常性反应,可以完全或几乎完全、持续地或反复地缺乏性兴奋和性快感的主观感受和感觉,有些妇女即使出现性高潮反应,也声称自己缺乏性快感或性满足。当时人们认为性唤起不仅仅是生理变化,也有认知和情感的变化。性唤起的认知和情感变化常常比生理成分更微妙,认知成分涉及把注意力集中或局限于动情刺激、幻想和性的暗示;情感成分涉及与上述生理变化和认知焦点同时发生的一个人对性兴奋、浪漫情调、性乐趣的主观感觉。女性对性唤起伴随的认知与情感变化的意识要比男性更明确和更清楚,但人们很难客观地、准确地判断主观感觉的减弱。她们往往形容说“性交时根本兴奋不起来”、“我一点感觉也没有”、“有没有性生活无所谓”,其实就是说没有性感觉,卡普兰(1974)称之为“全身性性功能障碍”,也有人称生殖器麻痹或性麻痹。主诉性感缺失的不仅仅是女性,也有不少男性抱怨这一问题,性交似乎成了他们的一个巨大精神负担,因为他们不仅没有从中得到快感反而让他们倍感失望。目前对于性感缺失问题的认识、诊断和划归尚有争议。

马晓年等(1993)曾提出女性性唤起障碍可以划分为:

Ⅰ级:女性在性活动中可有正常性生理反应,以往也有性快感,但目前性感减退或在某些特定的境遇下,性感缺失。有时有阴道润滑不足或反应较慢的表现。

Ⅱ级:女性一直具有性感度中度缺失。经常有阴道润滑不足或反应过慢。

Ⅲ级:过去曾具有正常的阴道润滑等性生理反应,但目前性兴奋反应缺失,阴道润滑不足或严重不足。常伴性感缺失。

Ⅳ级:一直缺乏性兴奋期性生理反应,阴道润滑不足或严重不足。性感严重缺失。

目前看来应将其修正如下:

Ⅰ级:女性在性活动中有时或在某些特定境遇下出现阴道润滑不足或反应较慢的表现。

Ⅱ级:女性经常出现阴道润滑不足或反应过慢的现象,对性生活有一定影响。

Ⅲ级:阴道润滑不足或反应很慢导致明显焦虑、不安或不适。

Ⅳ级:阴道润滑严重不足或几乎没有润滑反应,给性生活造成很大困难,也令个人和对方感到极大不满。

女性性唤起障碍同样可以划分为原发性与继发性、完全性与境遇性等不同类型。原发性指患者从性生活一开始就从未能获得满意的性唤起生理反应,始终缺乏阴道润滑反应;继发性指过去曾有正常的阴道润滑反应而现在却丧失了这种性反应能力。完全性指患者在任何情境或与任何伴侣始终不能获得满意的性唤起生理反应也即完全缺乏阴道润滑反应。境遇性指患者在某些情境或与某些伴侣能获得满意的性唤起生理反应,但有时或与有些人在一起时却缺乏阴道润滑反应。

性唤起障碍还意味着几乎或完全缺乏性的感受或性的乐趣,基本上没有性感觉,也有人称之为性感缺失、性麻痹或生殖器麻醉。一般认为这在女性尤为多见,据称这是较严重的女性性抑制。许多无反应性的妇女视性经历为一种考验,她们之所以能够忍受仅仅是为了维持她们的婚姻,实在厌烦时只好寻找种种借口尽量回避性接触。《国际疾病分类》第10版中把性感缺失与性高潮障碍划分为两个独立的临床病种,虽然性感缺失的患者也经常缺乏性高潮。主诉性感缺失的妇女或缺少性唤起生理特征性的反应,但可以在没有明显润滑的情况下通过刺激阴蒂、性交、振荡器而相当容易地达到性高潮,在某种意义上说,这种反应与勃起困难的男子可以在依然疲软的状态下射精是类似的。

由于在实验室测试中可供使用的性唤起刺激(如影片、幻想或情境)种类繁多且性质不同,它们引起的生理反应程度和受试女性对这些测试的反应都会有很大的个体差异,人们将很难设置什么客观指标并不可避免地主要依赖于她们的主观报告。不论FSAD的生理不足是否能够得到确证,都要考虑和安排认知治疗、加强躯体性刺激、性敏感区开发、动情念头训练、心理诱导等类似的行为敏感技术。

诊断标准中的引起“明显痛苦”或“人际关系困难”的条件有时会使诊断发生困惑,因为单凭其伴侣提出的这样的主诉对诊断就足够了,不必再考虑女性自己是否感到痛苦。这样的诊断标准也使FSAD与性欲低下或性厌恶的诊断出现混淆。此外,伴侣的问题或性刺激不充分也常常导致FSAD或性高潮问题。

在实际应用中判断是否与另外的和精神症状有关的疾患或疾病状况有关联时,可能会遇到一定困难。抑制、焦虑、惊恐、恐惧症或心理障碍常常与性功能障碍有关,而这些问题的成功治疗并不一定意味着治愈,这些障碍的残存影响仍可能足以阻止充分的性唤起。在这种情况下,必须运用明确针对FSAD的特定性治疗,而未接受过性治疗训练的普通精神健康工作者提供的、一般意义上的健康咨询或心理治疗就不会奏效了。此外,与精神症状有关疾患的经典治疗是离不开使用诸如选择性5-羟色胺再摄取抑制剂、抗惊厥剂或安定剂等的,这些药物本身就可以引起性唤起不足。这样,与精神症状有关的其他疾患和药物不良反应都可以与特定的性功能障碍相混淆。除此之外,这些药物的过量使用或误用也并非少见的现象。

涉及全身性疾病的严格说来应该得到治疗。因为尽管缺乏经研究证实的可靠证据,性功能障碍还是有可能由这些疾病所致,疾病状况对性功能很可能产生不利影响。例如,我们明确知道糖尿病、高血压和大量吸烟能够直接引起男性ED,但这些疾病与FSAD之间的关系至今仍不清楚。更让人困惑的是对绝经或激素缺乏是性功能障碍原因的考虑,因为性能力本身就受年龄的影响。

性唤起是复杂的,诊断为性唤起障碍的躯体健康的妇女在一个控制的环境下,对性刺激作出反应时通常表现出生殖器的正常血管充血反应。这样,造成她们不幸的关键不是生殖器充血的失败,而是缺乏主观的唤起。现在推荐的是得到认可的性唤起障碍可能存在的若干亚型:主观性唤起障碍、生殖器性唤起障碍、复合的生殖器的和主观的性唤起障碍,以及持续性唤起的障碍。能够区别唤起障碍不同亚型的一个主要的临床标准是集中在性刺激(如果有的话)是否是有效的。值得注意的是,这些定义的基础是妇女自我报告的生殖器充血和润滑的缺乏或损害,需要进行心理社会的测试来确认其潜在的生理病理情况。

具有唤起关切的大多数妇女报告说在经受任何类型的性刺激后仍缺乏主观的性唤起的感受(性兴奋和性快感)或显著减弱,她们对任何看上去应该是恰当的性刺激都缺乏唤起,如阅读一些色情材料,取悦其伴侣,接受口的、乳房的、生殖器的刺激,或直接进行性交。她们可能特别关注于她们所缺乏的主观唤起,不过承认依然发生一定程度的阴道润滑或其他躯体反应的征兆。一个重大的困难是妇女与生殖器事件的相对分离,实质上是常常缺乏对生殖器充血和润滑的知觉而不是真的有证据说明生理反应的缺乏或显著减少。

主诉生殖器唤起受到伤害。自我报告可能包括经受任何类型性刺激后仅有极少的会阴肿胀或阴道润滑,或抚摩生殖器后性感觉减弱。在非生殖器刺激之下仍然可以发生主观的性兴奋。

虽然一位妇女诊断为生殖器唤起障碍亚型,她仍然可以在观看情色影片时、挑逗她的伴侣时、接受亲吻时或接受乳房刺激时表现出主观的唤起。她主诉包括高潮在内的任何生殖器反应的强度显著减轻。悸动/肿胀/润滑的意识缺乏或显著减弱。此外,尽管可以发生似乎充分的充血肿胀,生殖器感觉的性内涵却减少了,这实在令人费解。

在经受任何类型的性刺激后性唤起的感受(性兴奋和性快感)缺乏或显著减弱,而且主诉生殖器性唤起缺乏或明显减少,(会阴肿胀,阴道润滑)。自我报告可能包括经受任何类型性刺激后仅有极少的,或抚摩生殖器后性感觉减弱。在非生殖器刺激之下仍然可以发生主观的性兴奋。

与生殖器性唤起障碍妇女不同的是,在任何性刺激下都缺乏主观的性兴奋。

在没有性兴趣和欲望的情况下,一种无端自发的、令人困扰的、让人讨厌的生殖器唤起,也即麻刺感、悸动和搏动感。对于主观唤起的任何意识都很典型,倒不都是令人讨厌的。这种唤起是不能通过一次或多次高潮就能缓解的,唤起的感觉会持续数小时至数天。人们对此尚缺乏起码的了解,但它已经成为一种公认的越来越常见的综合征。需要就其流行率、病因和有效治疗开展更多的研究。

性唤起障碍可以单独发生也可以伴有性欲低下或性高潮障碍,有必要加以区分或鉴别。性唤起障碍与性欲低下有时很难区分,关键是要看其性生理反应的程度和进展如何。例如,具有正常性欲的个体急切希望进入性情境,但她们又发现不能让性反应水平达到她们向往的程度,即充分润滑肿胀,这属于性唤起问题。性欲是对性活动向往的频率,性欲低下指女性对性活动根本不感兴趣,它与一个人能否完成这项任务的能力或曰性唤起能力无关。又如,性欲低下问题可以是很表面化的,利用一些现成的问卷就能测量到主观的打分及对包括性交、手淫、性念头和幻想、性活动满意度等在内的性行为质量的整体评价。然而,这些外围性的指标并不能直接运用于躯体性唤起本身。另一方面,如果将注意力主要集中在高潮障碍上,就会过度关切躯体性唤起反应中一个很局限的方面。何况对高潮障碍本身的描述本来就不那么容易,而且它在躯体性唤起整个连续过程中的作用也不是很清楚的(也即高潮与先前唤起增强之间的关系)。主诉性感缺失或阴道不能润滑的女性仍然可以在振荡器刺激下轻松达到性高潮,这和男子也能在阴茎疲软的情况下射精是一个道理。此外,作为女性测量与检测的目的,性高潮是能够通过准确询问而得到确定的“一个事件”和一个能与男性达到及维持足够硬度勃起相对应的“成绩”。性唤起障碍也可以导致性交疼痛、性回避、婚姻或性关系障碍。

女性性唤起是一种内脏反应,这是在自主神经系统控制下的生殖器官和性器官的充血肿胀反应。由于它受自主神经系统的调节,这种调节对情绪波动等的破坏作用极度脆弱。换句话说,正常的性功能要求人体处于相对平静的精神状态,如果一个人心烦意乱,自主神经系统就要对付这些消极情绪而不是性活动过程所需要的平衡调节,于是当妇女处于这样的惊恐状态时,性反应同样会因此而出现相应的生理反应损害,如生殖器血流量明显减少,这样构成兴奋期和平台期性反应生理基础的反射性血管充血肿胀将不能出现,她很可能丧失应有的性反应。同时女性还会出现其他一些生理变化如胃酸分泌过多、平滑肌出现痉挛、呼吸不匀或加深加快、血压升高、心率加快。情绪破坏性反应的作用与其来源和性质无关,只与它的强度有关。消极情绪可能与性有关,也可能与性毫无关联,如工作过于紧张、经济上的巨大压力或情感方面的曲折。心理性阴道干燥并非少见,这从这次的调查结果也可以看出。当妇女预期性交会带来疼痛或男方会发生早泄时就会发生这种情况,它可以抑制导致正常润滑的充血反应。


\section{第四节 性唤起障碍的病因分析}

与男性勃起功能障碍病因学研究相比,人们对女性性唤起障碍的病因学研究还刚刚起步,特别是实验室和鉴别诊断方法的研究还很落后,差距很大。对于年轻妇女的FSAD,我们首先要了解她的性发育史、性心理发育情况、性观念、夫妻关系如何等,试图寻找出有无心理的、社会的、人际的因素的影响;此外,还需了解双方事先有无充分的爱抚,女方有无夜间自发分泌现象,还是仅在性交时才出现分泌障碍。内分泌状况如何(即卵巢等的功能如何)、月经是否正常、有无生育等,如果缺乏这些信息,我们是不能对性唤起障碍的病因进行妄加的分析。

女性性唤起的生理反应既然是依赖于血管和神经系统的完整性,那么,这两个系统的任何损伤都会造成性唤起困难。造成阴道分泌不足有哪些常见的器质性原因呢?如盆腔血管疾患可以导致阴道润滑减少。同样,糖尿病和多发性硬化的外周神经损伤也可以影响性唤起进程。凡是能降低雌激素水平而增高孕激素水平的因素均可导致阴道干燥。激素水平会影响阴道组织的血流量、阴道内pH值、阴道上皮细胞的电位等,进而影响分泌物的多寡,而这些因素都具有雌激素依赖性。卵巢功能早衰、绝经期激素水平下降(特别是雌激素水平的下降)是造成阴道干燥的最主要原因,或者使阴道干燥更加严重,有可能导致性交疼痛。这种变化在哺乳期、更年期、绝经期或卵巢切除术后均会出现,如激素水平降低需要及时补充雌激素和孕激素,必然有助于改善症状。至于口服避孕药是否会影响阴道润滑,人们尚无明确结论,可能应因药而异,如某些孕激素含量高的口服避孕药也趋于减少阴道分泌物。类似的孕酮升高的现象也见于正常育龄妇女的排卵后的黄体期,相反,在排卵前的分泌期妇女将感觉到阴道分泌比较充分。

有些药物如抗组织胺药、抗胆碱药、降压药和镇静剂等也能造成阴道干燥。如果阴道有炎症则要积极治疗。

女性性唤起是一种由自主神经系统控制的内脏反应,其实质是生殖器血管扩张和充血,随后将发生阴唇、会阴、阴道周围组织的肿胀,这是控制生殖器血管口径的平滑肌的舒张所引起的。此外,阴道和子宫的平滑肌反应引起的阴道内2/3扩张和子宫的升高也是女性性唤起的特征。性与生殖系统的这些内脏反应(包括消化、呼吸和血压调节等与生命息息相关的其他重要生理功能)受自主神经系统的调节,那么凡能影响自主神经系统功能的因素都会影响相关的内脏反应。有些妇女诉说在性唤起阶段能有充分的阴道润滑,但在阴茎插入后不久润滑便停止了,这样妇女在抽动几分钟后便会感到不适。这往往反映了她们对性交的接受程度和对性的整体观念,心理和态度上的任何排斥都会影响妇女对阴道润滑反应的客观评价。因此,这些主诉反映出她们在插入后的内心感受的变化和抽动并未带来乐趣和舒适感的实际状况。当消极心理因素居主导地位时可以影响和干扰阴道下段的充分反应和润滑物的渗出。这些因素包括对性交不适和疼痛的畏惧,对伴侣作为合格情人的消极感受,对自己作为令人满意的性伴侣的消极感受,或对性交本身的错误认识。所以,没有健康的心理和情感,性反应很难不受影响和干扰。

众所周知,自主神经系统功能对情绪变化的影响是很脆弱的,如充分的食物消化要求人体处于相对平静的精神状态,而一个人心烦意乱、焦躁不安时,自主神经系统就要对付这些消极情绪变化,消化过程需要的平稳调节就将失去,从而影响消化并出现腹胀、腹痛等症状。又如当人惊恐时,胃酸分泌过多,平滑肌出现痉挛,血液从消化道转移出去,从而破坏消化作用。两性性反应同样易受消极情绪的不良影响,如恐惧或愤怒都会减少做爱过程中的生殖器充血肿胀反应,由于阴道润滑受到严重影响,做爱时出现干燥甚至性交疼痛也就不可避免了。情绪破坏性反应的能力与其来源和性质无关,而只与它的强度有关。妇女出现消极情绪的诱因是多方面的,包括性的以及非性的因素。损害性反应的焦虑不一定总是无意识的、未解决的、内心冲突的折射。这些焦虑常有近期的、直接的、往往不甚复杂的根源。妇女对不能达到高潮的恐惧、她的判断式的自我观望、不情愿与伴侣就自己的性需求进行交流以及未建立性的自主性或主观能动性,凡此种种因素单独或结合存在均可触发消极的情绪反应,从而损伤性反应所必需的松弛、恣意和放任。这些情绪与心理因素包括:疑虑、内疚、畏惧、焦虑、羞涩、冲突、不安、紧张、讨厌、激惹、憎恨、悲痛、对伴侣的敌意、成长过程中接受的严厉说教、性虐待或创伤史等。

人们不难发现性功能障碍妇女常常表现出神经系统对性刺激的全身自主交感神经唤起不充分,这可能是生殖器唤起反应减低的一个因素。按照过去推测,交感神经唤起将会减少生殖器血管充血反应,这是直接通过诱发血管收缩和间接通过对抗副交感神经功能潜在的生殖器反应的结果,它可能预示着,如果在检测女性生殖器血管充血反应之前让她们暴露于交感神经唤起刺激,就将影响并减弱生殖器反应。然而事实并非如此,佩雷斯(1995)发现若让妇女在检测前观看一些没有暴力或性情节、只是包括危难临头之类恐怖情节的影片,那么无论性功能正常还是性欲低下、性唤起障碍或性高潮障碍的妇女都将随着交感神经的激活而显示出生殖器唤起反应的增强。她认为这是全身交感神经唤起“推动”了女性生殖器血管的充血反应。也有人指出交感神经刺激剂肾上腺素能增强性功能正常妇女的生殖器唤起,例如当妇女剧烈锻炼引起交感神经兴奋后约15分钟能加速她们的生殖器唤起。然而,尽管测量到生殖器血管充血反应的增强,但这种交感神经激活却没有影响妇女对性唤起感受的主观打分。这样,佩雷斯提出在治疗患有性功能障碍妇女时,应告知她们如果通过性活动前给予交感神经刺激和其他促进手段,就可以不断增强生殖器反应,这样她们就可以达到正常的性唤起,从而加快对性唤起障碍的性治疗进程。全身交感神经激活至少也可以部分地支持其他刺激手段对性功能障碍妇女的治疗作用。

虽然在现实生活中人们总会预料各种焦虑或畏惧必然会干扰性唤起,可是在实验室的相对安全条件之下,这些焦虑或畏惧却通过诱导交感神经的激活促成事实上的性唤起。通过在不同情境之下直接测量女性性唤起反应得出的这些发现,可能有助于澄清FSAD的发生原因,也可导致研究人员在不受当前对FSAD先入为主概念的影响下建立起革新的治疗方法。


\section{第五节 女性性唤起障碍的临床诊断与治疗}

性唤起生理反应受到损害的女性常常同时经历其他形式的性功能障碍,如性欲低下、无高潮、性交疼痛等。所以必须采集完全的性历史以评价其性反应周期各个阶段的表现,其中要特别注意唤起问题,评价可能在一或几个单元中进行,取决于问题的复杂性。对于生理损伤性性唤起障碍妇女来说,最好在会见她们的伴侣之前先单独约见她们几次,然后单独约见她们的伴侣,最后再同时接见夫妻双方。

医生要获得有关性问题进展过程的信息,例如是突然发生的还是逐渐发生的?是在所有情况下都有问题还是偶尔遇到问题?是否存在能够改善或加剧性问题的因素?注意有无性兴趣、性兴奋、性行为、性乐趣、焦虑或整体满意度等方面的改变。

医生要获得有关患者当前的或典型的性经历的详细描述,这样对其问题便有了一个清楚的了解。这包括从性接触开始到结束的具体行动、想法和感受。既包括自己的手淫也包括与伴侣的性活动。因为患者往往倾向于迅速陈述和概括她们的经历,绕过或跳过一些重要的细节,医生常常需要追问她们并要求她们准确解释她们在每一步骤中做什么、想什么和感受到什么,好像让她们在描述一部电影:谁发起的?如何交流的?对她们有什么帮助?如何操作的?问题是何时和如何发生的?存在愤怒或焦虑吗?这些是如何把握的?是否存在任何产生相反作用的行为?引起不适或疼痛的原因是什么?

重要的是要了解女性是否曾经达到过性高潮?如果是这样,又是在什么情况下发生的?她自己能够让自己达到性高潮吗?当她手淫时她触摸自己的什么部位和如何刺激?使用手指、振荡器或其他手段?手淫时她是否伴有性幻想?她发现最容易让她性唤起的念头或幻想是什么?在与伴侣做爱时她是否利用这些幻想?如果没有,为什么?她和伴侣共享这些幻想吗?还是根本不愿和伴侣分享这些幻想?当她和伴侣在一起时,她接受的是哪些刺激?有多长时间?这些足够吗?还是想要更多一些?阴道和阴蒂经历了愉快的感觉吗(温暖、震颤、敏感性提高)?如果是,来自哪些感觉?当唤起时她经历心率和呼吸次数的改变吗?她的生殖器组织肿胀、充血、变硬了吗?她的阴道润滑充分吗?她是否遇到阴道干燥问题?她使用润滑剂吗?她发现什么样的性活动能带来性唤起呢?她经常这样做吗?她经常能达到性高潮吗?如果达到过,又是经由那些活动达到性高潮的呢?她害怕性高潮时的充分性释放吗?如果害怕,为什么?她在性活动中集中精力有困难吗?如果是这样,是什么在干扰她呢?她是否精神贯注于不安或不愉快的念头呢?她的伴侣如何令她兴奋或令她扫兴?她是否发现自己或对方具有性吸引力?她在性交前是否感到紧张或焦虑?在做爱之前直至之后她的脑子里在想什么呢?她的性欲、性念头或幻想经常发生吗?她经常进行性活动吗?独自或与伴侣在一起?

医生应该询问是否存在可能导致性功能障碍的任何健康问题。如果有的话,在过去或最近诊断过哪些疾病或接受过哪些手术?在过去或最近服用过哪些药物?她抽烟吗?酗酒吗?滥用药物或其他物质?月经正常和规律吗?她服用可能损害性唤起的避孕药吗?在围绕妊娠、分娩或产后时她存在任何问题吗?对于绝经后妇女或那些卵巢功能受损的妇女,医生可能希望获得更多实验室检查结果,如游离睾酮和总睾酮,雌二醇(E2),孕酮(P),泌乳素(PRL),黄体生成素(LH)及卵泡刺激素(FSH)。此外,还可能要检查甲状腺、血常规、血生化、血糖或糖化血红蛋白AIC等。有阴道不适或疼痛的妇女应转诊给妇产科医生。

临床也需要决定性症状是否继发于精神障碍如情感性障碍或焦虑障碍。以下几个筛选问题可能是有帮助的:过去曾经滥用药物或其他物质吗?妇女曾接受过精神病治疗或因精神原因而住过院吗?她服用过精神科药物(包括镇静剂、抗抑郁药或安眠药)吗?她是否经历过抑郁、恐惧或恐慌发作?

如果精神疾患似乎确实存在,就应该阐明其诊断和治疗。重要的是要了解是否有任何药物会造成性功能障碍。重要的是还应了解究竟是性症状先发生还是精神问题率先出现,是否存在严重的紧张因素?就像惯常的做法那样,也要询问患者是否有过自杀念头?具有严重精神病患的患者是很脆弱的,不可能耐受性心理治疗或至少不能从中获益。

医生应从患者阐述中获知其性发育的历史资料,包括关于她们最早的性感受的记忆,早期性经历,初潮年龄,青春期成长过程中性梦、性幻想、初次性交情况、特别是有无性创伤或性虐待等历史,过去及目前的性生活史,两次婚姻之间的性生活,绝经后性生活状况等。例如一位妇女曾受到她父亲的抚弄式性虐待,后来她对接吻和性交均有反应,就是对拥抱和事前爱抚没有反应,说明过去心灵创伤与特殊性行为方式之间的联系。

患者家庭本身的、文化的、宗教的背景是很重要的,因为它将深刻影响其性观念和性素质的形成。恰恰是这些影响将成为引起其成年期性问题的冲突来源。家庭的组成成员,成员间与患者关系质量,家庭关切的主题,家庭内性信息的传递等都是重要问题。

医生必须评价患者个人过去浪漫关系的质量。例如,是否长期存在性吸引力本身或相关的问题,是否存在接纳异性或轻易拒绝等问题。

医生必须评价患者目前的伴侣关系并决定是否具有能损害性唤起的问题。伴侣间相互作用往往成为性问题的潜在根源。常见的障碍包括交流太差、性技巧贫乏、期望值过高得不到满足、关系冲突、伴侣的性功能障碍、伴侣的病理心理问题、害怕亲昵、不相容的性幻想、常见的移情、未解决的冲突和权力斗争。

在临床评价和讨论时,医生应具备充分的知识以便能作出准确诊断并正确判断该问题究竟主要是器质性问题还是心理性问题。医生还应确认是否存在关系问题,不现实的期待,或不充分的技巧等诱发因素或决定性因素。只要可能,医生就要给予患者充分的鼓励,不一定十分准确地承诺其治疗结果如何。应该提出治疗的推荐方案,包括对预后、所需时间和治疗难易程度的预测。

尽管我们支持循征医学的实践,可是按照“良好医疗实践”的规则认真处理妇女的性问题时,我们不得不在没有一致有效性证据的情况下坚持工作。FSAD的治疗既可以针对患者个人进行,也可以让双方一起来就诊,或单独就诊与同时就诊相结合。一般多至每周就诊一次,少至每月就诊一次,这取决于患者的需求、可行性、经济状况和对医生的信任程度。性唤起障碍的治疗应力求把医学治疗和性咨询、性治疗结合起来同时进行。对于绝经前妇女是很少需要医学治疗的;而对于具有雌激素缺乏、性唤起受损害和性交疼痛的绝经期妇女而言,只需给予激素替代治疗和性咨询指导,不必再安排特殊的性治疗。不过,这种激素替代治疗方法虽然能增进润滑、减少不适,但对于真正性唤起障碍治疗来讲却仍是不充分的。对于这些病例来讲最有效的治疗是在激素替代或其他医学治疗基础上的针对性唤起障碍的特定性治疗。

以交流电或电池为动力的振荡器在增进女性性欲、性唤起和性高潮方面是非常有效的。它们提供的强烈刺激有助于克服心理抑制和唤起抑制的。它们可以运用于针对个人或伴侣的治疗练习中。振荡器具有不同地形状、大小和功能。除了传统的仿阴茎型,现在还有强有力的、但只有唇膏大小的掌上型模式,它们是设计来专门刺激阴蒂的,及专门刺激G点的曲柄型。现代的许多振荡器都包上了硅胶的可伸缩的外套,这样可以带来更多舒适感和易于保持清洁。

对于同时患有的心理性性欲和唤起障碍的女性,心理处理包括认知行为治疗(CBT)和传统的性治疗两方面的技巧。CBT的理论基础是各种想法、感受和行为相互作用,彼此互相影响。以负面的或适应不良的想法作为目标,无论是行为和情感都能改进和中断功能障碍的周期。行为治疗内容包括双方都参加到有问题的性环境和行为之中,减少相互吸引力或信赖,并能够集中在性刺激和感受上。CBT的认知内容把目标指向适应不良的想法,这些想法培养了负面情绪,并且使得回避这类问题的行为继续下去。这可能包括识别和挑战那些信念,如她没有吸引力,把有关描绘得不现实的性反应模式理想化,或者挑战那些信念,除非她在所有的时间里总是感受高水平的性欲,否则她就有功能障碍。当更深层次的情感问题和对治疗的阻抗日益突出时,这些技巧还应与心理动力学为主的心理治疗相结合。卡普兰(1995)介绍了这样一些技巧:努力识别对性反应起消极作用的想法、情绪和行为,不断强化对它们的领悟,包括性幻想训练、色情艺术品的使用、手淫作业、性技巧训练、改进交流、性感集中训练、转换以及改进亲昵等。有些妇女发现亲昵或与特定他人共享感受的行动,是达到性唤起所必不可少的前奏。许多性治疗家庭作业涉及的训练内容其实从本质上讲都不是性训练,它们仅仅适应于改善亲昵关系。

最近,在一项没有对照组的、针对继发于妇科癌症的性欲和唤起障碍妇女的小组治疗中,尝试探索了一个以专注力为基础的CBT方案。专注力是一个扎根于佛教冥想的东方人的实践,其专注于当前的时刻,无偏见的意识。性治疗师不仅在培训课上教给妇女提高专注力的练习,而且鼓励她们在两次课之间的两周内自己进行大约5小时的练习,这种培训共进行四次(注:共计4次6周)。这一方法使性欲、唤起、性反应和情绪的其他方面得到明显的改善。然而,在这一领域十分需要对照的有效性研究。从我们的回顾得出结论,妇女的大多数性唤起问题与生殖器反应性的损伤无关,并且可以遵照我们推荐的唤起障碍的心理治疗方法进行治疗。

治疗措施仍以性感集中疗法为主,辅以必要的心理治疗。性治疗的基本原则是通过改变妇女发挥性功能的性系统,促进妇女在性体验之时充分放松,这样才能达到预期的治疗目的。根据经验水平,性治疗学家试图在做爱之际通过造成一种允许性反应自然展开的松弛的、无需求的和肉感的气氛,以完成性治疗的目的。鼓励配偶公开地、坦率地交流他们的性感受和愿望,将有助于促进上述气氛。此外,也已证实系统地规定和安排各种感官的和性感的体验或家庭练习,在消除某些影响性功能发挥的直接障碍中,是高度有效的。

此外,还应根据月经和第二性征情况判断一下激素水平是否正常,否则可以安排人工周期治疗。最简单的治标的办法是建议她们使用润滑剂来解决润滑不足的问题,以缓解燃眉之急。市面上已有很多产品在销售,如美国强生公司的K-Y人体润滑剂等。

下面简单介绍针对性唤起障碍或性感缺失这一类女性性功能障碍的指定性作业的通常程序。它包括性感集中训练、无需求快感体验、生殖器刺激和无需求性交等实际的治疗过程。

性感集中训练,是由马斯特斯和约翰逊首先提出的治疗性功能障碍的一种巧妙和宝贵的方法。简而言之,它要求伴侣在性交和高潮之前,进行一段时间的准备活动,暂时把她们的性活动限制在动情的拥抱、轻柔、触摸和爱抚彼此的身体。在治疗女性性功能障碍中,采用这一训练方法时,通常指导妻子首先爱抚她的丈夫,然后再交换角色进行。这有助于中和她为自己接受某种帮助而感到的内疚,她往往担心她若总是需要对方的帮助,对方将会拒绝或抛弃她。因此,如果进展顺利的话,这将允许她集中精力于丈夫的爱抚所激发和带来的感觉,而不会为上述的情感上的内疚而分心。

这种表面上简单的相互作用的影响,可能是非常富于戏剧性的。由于妇女消除了对产生性高潮和服务于她的丈夫的压力,她常常能首次体验到性快感的和感官方面的感觉,她还可能发现丈夫并未因她的问题而拒绝她。相反,她丈夫会很喜欢这种能产生意想不到的乐趣的机会,他往往会欢迎这一机会并为能给妻子提供性乐趣而感到自豪。性感集中训练的成功实现必须得要求丈夫推迟他的性高潮愿望的迫切冲动和情欲。他喜欢这样做的话则证明他是挂念和关心妻子的性乐趣的。妻子对丈夫的这种反应常常是激动与感到放心的,这可使她更充分地集中于自己的性感受之上。

当不存在阻碍性感集中体验的障碍时,配偶便可以在愉悦、相亲相爱和充满希望的气氛下进入下一阶段的训练。但是,情况并非总是这么顺利。有时患者还是觉得她的性感受和性感觉方面仍有困难。她可能诉说在丈夫爱抚她时,她感到紧张不安、怕痒甚至会生气,或抱怨丈夫笨拙或急躁。如果出现这些情况,就要与患者讨论存在这些障碍的问题在哪里,并指导配偶重复这一阶段的体验,直到产生更积极的反应为止。

生殖器刺激,当患者诉说性感集中训练引起她感觉和性的快感时,练习可扩展到包括轻柔地抚摸生殖器的举动。在对妇女的整个身体进行爱抚之后,丈夫要轻轻地触摸她的乳头、阴蒂区域和阴道入口,告诫他不能试图提供强求的并以性高潮为目标的刺激方式。这时的生殖器抚摸务必十分轻柔缓慢,如果需要的话,可使用少许凡士林或其他润滑剂帮助润滑,并应该在妻子口头或非口头表达的指引下进行。如果抚摸生殖器对于丈夫来说太刺激或令他感到沮丧时,可指导患者用手或口刺激的方式促使他达到性高潮,但这只是在她已经得到没有压力的和集中精力的生殖器爱抚的机会之后,才可这样做。

马斯特斯和约翰逊在这个治疗阶段中,指导他们的患者采用一种特殊的无需求体位。应该鼓励患病夫妇去实验,直到发现对他们最为舒适最能带来乐趣的体位为止。

在典型的病例中,如妇女的问题不是由于复杂的内心冲突造成的,而生殖器的刺激体验对于她的性反应性就会有明显的促进作用,使她感受到性的唤起与愉悦,并渴望进入下一步骤的练习。而且,如果她的丈夫很钟情于她并对这种体验不感到任何威胁或厌倦,他可以从这些“练习”中分享妻子的热情与愉悦。因此,在这一治疗阶段,配偶的关系易于保持密切和浪漫。

无需求性交,如果生殖器的爱抚能激发出良好的反应,下一步骤便是性交。在开始时还是要根据妇女的感觉和情感反应进行引导,并要排除对达到性高潮的任何压力。特别要注意的是作为经过性感集中训练和轻柔的生殖器刺激训练,使患者能达到高度性唤起之后,要由患者自己发起性交。应该提醒患病夫妇注意的是,开始抽动时要缓慢地探索着进行,而不能快速、猛烈和强求。进一步指导她通过勃起的阴茎慢慢地抽动而集中精力体验来自阴道的肉体的感觉。此外,还应指导妇女在作抽动动作的同时收缩她的耻骨尾骨肌,以体验该肌肉的存在和功能。在典型情况下,当性交抽动是在她自己控制之下进行并以增强她感觉意识为唯一目的时,妇女对舒适的阴道感觉的意识会有相当大的提高。在这个治疗阶段中,医生一般都会指导他们的患者采用女上式体位或侧位,借此方式伴侣双方能够自由地从事骨盆的抽动。关于体位的选择倒不必刻板地规定什么,而应鼓励患者去实验或探索,并使用她们发现的最有助于双方性乐趣感受的体位。

如果在妻子试探性抽动期间,男方对射精的欲望变得太强烈时,要劝告配偶暂时分开,这时丈夫预感到的性高潮感觉会迅速消除,当丈夫休息时,可以指导他以手来刺激妻子。这种训练过程重复几次后,如果妻子感到愿意向性高潮发展,她应向丈夫传送这种信息和要求。如果她不想这样做,经过一段合理的间歇后,配偶可进行导向男方性高潮释放的性交。许多妇女发现这种挑逗、悠闲、愉悦的做爱过程能极大限度地使她达到性唤起。

这些无需求的、以调整性愉悦为定向的性体验的优点可概括如下:第一,因为可缓解妇女对必须出现性反应的巨大压力,所以这些练习不易促成她的防卫和焦虑,相反地,这样她还可以常常体验无障碍的性快感。第二,已有大量的事实证实,这些特殊的安排在激起女性性兴奋的训练中是有相当疗效的。生殖器的刺激,即男人轻轻触动她的阴蒂、阴唇和阴道入口的周围,不仅仅使男子得到兴奋,其实这也是她过去的体验,而刺激的目的主要是为女方提供性乐趣。在性感集中训练中,男子不应按自己的好恶来加速或停止爱抚,而恰当的做法是,他要根据女方的要求决定是否应该继续爱抚她的身体和他究竟应该采取何种步调。第三,在完成这些性作业的过程中,患者和她的丈夫对彼此的性需求和反应必然会变得更明智和敏感。她开始认识到丈夫并未把这种训练看做是讨厌的事,他会因为给她带来幸福而感到高兴,并且当她主动寻求性快感时,丈夫并没有像她所预料的那样反对或拒绝她。这样,患者所建立的对性体验的防卫,如回避有效性刺激,以及她针对她自己动情感觉所建立的防卫,在这种放松的性环境、这些指定的相互作用的作业和公开可靠的交流中,统统开始消融。

这些性治疗体验常常能激发患者的感受和阻抗,这就使我们能够找出阻碍她性反应的特殊障碍。如有的患者会抱怨爱抚某些部位时“感到不舒服”、“发痒”,这是临床实践中常常遇到的现实问题。据推测这很可能是一种防卫性的反应,以此来对抗由这些刺激引起的对性感受的知觉。根据详细询问,一些患者指出这是因为刺激了其最强和最敏感的动情区后,才引起这种不良感觉的。这很可能反映出她潜在的内心冲突,一方面她头脑中的陈腐观念对性持消极和否定的态度,另一方面她又渴求性的快感和满足,于是产生不易调和的矛盾。

性治疗医生应根据自己的临床经验和采用以心理动力学为宗旨的解释来处理这些障碍。指导患者采取一些策略并设计一些能使她自由地进行性表达和体验性感受与性乐趣的性作业。把治疗中遇到的障碍和阻抗看做是心理治疗探索的靶标。例如,如果患者很明显地拒绝让自己对性感集中训练作出晌应,医生则应让他们在女方乐意和喜欢男方时或女方愿意接受男方爱抚的时候再试一下,积极说服她通过这种体验的不同阶段激发自己的性欲感受,努力把自己的注意力着重集中在自己的性感觉上,而不是对方的满意或反应上,这时她必须排除一切杂念并且充分放松。同时,在此疗程中,医生应尝试帮助患者识别和消除她的抑制其性反应的在潜意识里起作用的内心冲突。

面对这种现象的治疗处理办法包括劝告患病夫妇,告诉他们这种反应可能是由于内心冲突所引起的,并向他们说明这种反应往往是短暂存在的。患者可以发现,她丈夫的第一次爱抚可能令她发痒或寒战,但下一次时,则可能引起她的性愉悦的感觉。身体潜在的性敏感区的逐步“脱敏”技术将在这种情况下发挥作用,它常可促进在性欲要求和对有效唤起的畏惧感之间的冲突的消除。此外,在这种共同参加的疗程中,也应以言语交流的方式处理导致这种冲突的内心的和相互关系方面的不利因素。这种不良感受也可能只是皮肤对异性触摸的不适应,这也恰恰是过去的患病夫妇所忽视的并成为性感缺失原因之一的根源。他们过去的性活动中缺乏的也正是医生现在所安排要集中精力体验的,因此经过短期地适应过程,这种不良感受自然就会消失。

许多患者诉说,一旦他们开始体验到动情感觉时,她们的精神立即开始恍惚。一般认为,这种“注意力分散”现象可能构成一种固执的防御,它是足以“关闭”妇女性感受的一种决定性的致病机制。医生可以首先尝试简单地指导患者有意识地再度集中其注意力;然后,如果问题继续存在,则以性幻想作为一种保护措施。在进行性作业过程中和在心理治疗过程中,都要注意探察表现出这种冲突的根源。平时患者显现的性反应常常因为她对丈夫反应的固执偏见所阻止,恰在妇女开始出现性反应之时,她的反应会因激发上述的念头而流产,必然也会因为她极度的不安和加重的对遗弃的恐惧感而流产。她往往会这样想:“他是不是太疲劳了?”“他一定认为我是个真正的神经质的女人”,“这不是他想干的事,我不能要求他为我做这种说不出口的事”,“我确信他希望他能与一个更性感的女人在一起”,等等。显然,这些念头会熄灭她对性活动的热情,会阻碍她应有的性反应和性感受。

看来,这种常见临床现象的有效治疗需要联合运用指定的性感集中训练等行为疗法和心理治疗法。因而,指导患者在性活动中变得“自私些”,仅仅想她自己的性感受和性感觉,正像她爱抚丈夫时丈夫也是集中精力于他自己的感受那样。如果需要的话,在此疗程期间,可尝试帮助患者与对被抛弃的畏惧作斗争,当然这种畏惧的感觉也可能来源于她早年与双亲关系的变迁或文化传统引起的无可救药的感觉,以及在我们社会妇女中,流行的被动屈从的错误意识。

在性感缺失治疗中,必须结合使用性练习作业和具有特殊技术特点的性治疗的心理干预。性练习不是机械的动情设计,除其固有的治疗意义外,还能在运用过程中,显露和阐明患者问题的心理动力学变化,这样恰恰能更有效地进行针对性很强的心理治疗疗程。这一疗程也揭示了配偶关系中的问题。例如,由于在妻子能作出性反应之前,往往需要延长的性刺激,丈夫可能对此感到烦恼和生气,甚至感到被妻子所拒绝。如果以女上位作为促进无需求性交体验的方式时,配偶可能对女上位感到害怕和威胁。

正如男性患者的症状即将缓解时也常常动员起的焦虑和阻抗一样,当妻子将开始初次与丈夫一起体验性愉悦时,她们经常出现这些心理障碍。在这一点上,临床医师在处理时,要具有特别的技巧,因为这不仅仅是性练习,它同样决定着治疗的成功或失败。

虽然治疗可能由于造成妇女性唤起障碍或性感缺失的更深在的原因引起的防卫和阻抗变得更为复杂化,但从本节提出的治疗程序看来,可以使许多无反应性的妇女的由近因障碍引起性反应缺失的现象得到改善。通过这些简单的以经验主义为主的治疗方法,没有获得帮助的主要是那些由于深在的敌意或冲突而阻断性反应的妇女。

这种治疗安排和类似的治疗程序,对于促进性乐趣、性反应性和性接触频率的增加,常常是有效的。但是,正如先前所指出的,这些无反应性的妇女也可能是无性高潮的,所以,性感缺失的成功治疗也是对她们性高潮困难的缓解。然而有的时候,虽然患者感觉到了性乐趣的增强和性欲望的加深,以及对性刺激表现出阴道充分的润滑,但她可能仍未能达到性高潮。这说明她们存在着针对性反应中性高潮成分的一种独立和特殊的抑制,根据临床的经验,这需要给予另外的处理,这需要一种稍微有所变更的治疗方式。

(1)睾酮:人们知道睾酮对女性性欲和性唤起是必不可少的因素已有40年之久,然而这一知识并未能融合到常规的临床实践之中。国外经皮睾酮治疗的大多数是以外科的、经历某些形式的化学治疗的和自然绝经的妇女为主。这几种情况都导致卵巢功能急剧下降、卵巢雄激素的突然丧失和肾上腺性激素前体的不可避免的下降,这些妇女中有相当一部分人在一定程度上丧失了性欲或性唤起功能的表现能力,但医生很少告诉她们存在这种潜在的危险。许多具有睾酮缺乏的妇女,其性欲和性幻想的减少,她们的乳头、阴道、阴蒂对刺激也不那么敏感。此外,缺乏睾酮的妇女会出现性反应周期各期(性欲、唤起、高潮和消退)的性反应丧失,特别引人注目的是对性欲和性高潮有不良反应。如果仍能达到性高潮,其感觉也减弱、时间缩短、不再那么令人激动、仅局限于生殖器局部。睾酮缺乏也会导致精力不足、容易疲劳、肌张力减弱及生殖器萎缩,而且对雌激素替代治疗毫无反应。

她们可以接受睾酮替代疗法。但女性使用的剂量要远远低于男性(女性睾酮只相当于男性1/12)。由于生殖器组织萎缩或缺乏受体,最初对口服睾酮可能没有反应,所以开始时最好每天一次在会阴局部直接涂抹以乳膏为基质的睾酮。等一、二周之后乳膏涂抹可以扩展到大腿内侧或腰,每周5次,会阴处每周2次。剂量应尽可能低至为恢复生理水平所必需的剂量。对于妇女,局部应用为每次0.25~0.8mg。会阴局部使用睾酮乳膏后半小时到数小时,女性会感到性欲提高。经双盲交叉对照试验证实,给予单剂量睾酮之后3~4.5小时女性生殖器的反应性迅速增加。睾酮水平在15~90分钟达到高峰,反应峰在其之后1小时左右,滞后的原因是负责女性对性唤起的大脑机制减退。另一个便利的选择是使用贴皮形式经皮给予睾酮,通过贴皮释放达到生理睾酮水平可以把不良反应降低到最低程度。但经皮贴剂是供男人使用的剂量,那么妇女每天使用的时间可缩短到4小时。这样5mg的贴皮剂能通过皮肤吸收达通常每天剂量的1/6,约0.8mg。最近回顾了在接受雌激素替代治疗的外科的和自然绝经的妇女中使用经皮睾酮的试验,发现这些妇女在接受300μg/d贴片后,每月的性满意事件有所增加,达1.9次,而空白对照为0.9次,同时性欲也显著增加。对睾酮的反应有时可迟至几周、几个月都不出现,可能是因为受体已变得不那么敏感、数量也减少的缘故。停药两周激素的有效性就可能恢复。在欧洲,已经批准睾酮贴片在外科绝经妇女中的使用。现有的资料不支持在绝经前和围绝经期妇女中使用睾酮。

治疗决定必须是个体化的,应该充分告知患者治疗方法的利与弊,目前为止,尚缺乏长期安全性资料。在使用适于妇女处方的低剂量睾酮时是很少见不良反应的:如体重增加、肝损伤、高密度胆固醇降低、阴蒂增大、痤疮、易激惹,人们也很关心给女性睾酮后会不会使女性呈现男性化,出现男性的第二性征(如胡须、低音及秃顶趋势)。已知睾酮可造成对心血管有好处的高密度胆固醇的轻度下降,这将使她们对心血管疾病更为敏感,但这会被它对心血管的其他好处所抵消。实际上,低剂量睾酮对女性心脏是有保护的作用。而乳癌和心血管疾病患者是否因此而增加尚不得而知。

相对禁忌证包括雄激素性脱发、痤疮、高脂血症多毛症和肝功能障碍等。绝对禁忌证包括存在乳癌或具有乳癌的高度危险、内膜癌、静脉血栓发作、心血管疾病等。

监测应当包括每年的乳房和盆腔检查,每年的乳腺X线检查,异常出血的评价,痤疮的评价,多毛症,雄激素性脱发。采用质谱法监控睾酮水平(性激素结合球蛋白,计算得出的游离睾酮)的目的是不要超越正常值。还应考虑血脂谱,全血计数和肝功能检查。使用6个月以上可能发生完全的改善,而且没有不良反应。

(2)雌激素:国外有证据表明局部或全身使用雌激素治疗有助于外阴阴道萎缩的缓解和减轻阴道干燥和性交疼痛。有人反映它们具有增强性欲和性功能行使的积极作用。正向的性作用可能部分来自雌激素对皮肤、乳房和阴道的女性化作用的心理影响,还可能带来自信心和情绪的改善,从而能够间接导致性欲的增强。当然,这时纠正继发于雌激素缺乏的性交疼痛是必要的,而性交疼痛对性唤起是一个重要的抑制因素。雌激素对性唤起可能具有一种直接的积极作用,它也可能激活大脑和阴道组织中的后叶缩宫素,而后叶缩宫素可能促进女性躯体性唤起,就像它能刺激男性勃起一样。它也可能增强与伴侣的放松和增加对触摸的敏感性。此外,雌激素可以刺激为血管扩张所必需的一氧化氮(NO)的产生。

另一方面,许多接受外源性雌激素的妇女(含或不含孕激素)主诉性欲和性唤起能力降低。这可能是由于甲地孕酮使血蛋白水平增高的缘故,这些蛋白会与激素相结合(包括睾酮)并使之灭活。这样服用外源性雌激素实际上可能通过突然降低游离睾酮水平而导致性功能障碍。

看来,女性对外源性雌激素的性反应差异是相当大的,甚至是相反的。雌激素即使有性强化作用,似乎也不如睾酮的作用更强更有力。现在,雌激素具有多种给药途径,包括口服、经皮贴剂、阴道栓剂和可以在阴道内保留3个月之久的含药阴道环。

(3)局部脱氢表雄酮(DHEA):脱氢表雄酮(DHEA)替代治疗似乎能导致肾上腺功能低下妇女性反应周期所有阶段的性功能有所改善。从轶事角度看,观察到DHEA(10mg/d)能在治疗热红晕时恢复绝经期妇女的性欲,用法包括口服、经皮、经阴道。有些男性报告显示可以有效改善性欲和性表现能力。

国外最近一项阴道使用DHEA随机对照试验(RCT)表明有益于阴道萎缩和性功能的其他方面。没有检测到血清DHEA,睾酮,雌二醇,或雄激素代谢物的增高。因此在得到其他人的验证之前,还不能作出推荐。

(4)选择性雌激素受体调节剂(SERMs):最初,SERM是定义为与雌激素受体(ER)具有高度亲和力并与之结合的一个化合物,而与任何其他核受体没有显著的结合活性;它有在一些组织诱发“雌激素激动剂”的活性,而在另一些组织则呈现“雌激素拮抗剂”的活性。国外新近浮现的资料表明在一个特定的SERM和ER之间的交互作用,导致在一个既定的组织中的反应不能必然地、简单地、特定为要么是“激动剂”,要么是“拮抗剂”。每个SERM可以具有一套独特的临床反应,它们并不总是可以从其他SERM的表现预见出来。Ospemifene是一个正在试验中的、用来治疗绝经后妇女阴道萎缩的、新的SERM。

(5)替勃龙:替勃龙是一个19-去甲睾酮衍生物,它代谢为3个主要代谢产物:3α、3β-羟基化合物(雌激素性质)以及本品δ-4异构体(孕激素性质和雄激素性质)。替勃龙是一个选择性的组织雌激素活性调节剂。在绝经后妇女中,它像雌激素一样作用于大脑、阴道和骨骼,但是不会作用于子宫内膜和乳房。在因为非性功能障碍原因征募的大多数受试者中,记录到替勃龙使性功能得到改善。替勃龙在欧洲已经上市,但是在美国遭到拒绝。

(1)中枢作用制剂:阻断去甲肾上腺素和多巴胺再摄取的抗抑郁剂安非他酮,能够显著改善没有抑郁症的HSDD妇女的性唤起和高潮,但是不影响其性欲。服用选择性5-羟色胺再摄取抑制剂(SSRI)并伴有混合的性症状的妇女,在增加安非他酮治疗4周后,导致自我报告的性欲和性活动的显著增加,但是对性的念头没有显著影响。Flibanserin是一个5-HT1A激动剂/5-HT2A拮抗剂,最初是作为一个抗抑郁剂来试验的。在一项未发表的涉及333位广义的、获得性HSDD妇女的试验中,与空白对照相比,flibanserin表现出对FSFI所有方面明显的改善。

中枢作用制剂在以性欲低下妇女为治疗目标时显示出一定的希望,但是需要发表的随机对照试验(RCTs)和对它们安全性的评价仍然有待进一步的研究。

(2)磷酸二酯酶抑制剂:西地那非———从理论上讲应通过局部生殖器的血管扩张而增强女性阴道的充血和润滑反应。英国伯曼等完成了一项随机、双盲、空白对照临床研究以评价枸橼酸西地那非治疗绝经期(52名)或子宫切除术后(150名)诊断为原发性女性性唤起障碍(FSAD)的有效性和患者的耐受性。他们将这些患者随机分为空白对照和枸橼酸西地那非(剂量可灵活调整,25~100mg)两组,参加为期12周的双盲治疗。所有妇女的雌二醇和睾酮水平都在生理范围之内和(或)接受雌激素和(或)雄激素替代治疗。使用标准化的性史问卷进行性功能障碍的诊断和性心理的评价。主要疗效参数为女性干预有效性指数(FIEI)中的问题2(在性交或性刺激时生殖器的感觉增强)和问题4(在性交和(或)事前爱抚时满意度增强)。结果表明枸橼酸西地那非组报告生殖器感觉增强和整体性满意度增强的妇女明显多于空白对照组。子宫切除术对其性反应没有影响。枸橼酸西地那非治疗能显著增强具有性唤起障碍但没有继发性性欲低下妇女的阴道润滑(p=0.0154)、提高获得性高潮的能力(p=0.0116)和整体的性体验(p=0.0001)。大多数不良反应为轻度的或中度的,大多数表现为头痛、潮红、鼻塞和视觉障碍。小样本研究表明西地那非能使因服用抗抑郁药所致的女性性功能障碍的性功能有所改善。看来西地那非能增强某些妇女的性反应性,包括原发性性高潮障碍、抗抑郁药引致的高潮延迟、与子宫切除相关的性功能障碍。有些情况下,西地那非导致妇女性欲的增强和对性高潮后再次性要求的愿望。有患者局部使用相当于口服一半剂量的西地那非制剂后也有效。有些妇女原则上拒绝通过口服药物的方式来增进性功能,她们还是愿意接受局部用药。但2004年2月27日辉瑞公司宣布终止自1996年开始的西地那非治疗女性性功能障碍的临床研究,因为得不出一个肯定有效的结果。看来人们对女性性反应和性功能障碍的理解还相对肤浅,有待进一步的深入研究。

在一项涉及具有不同诊断的性功能障碍妇女的大样本研究中,使用50~100mg西地那非没有显示预期效果。总的来说,诸多文献的发现是相互冲突的,一项研究是否显示受益取决于心理生理测评的性唤起的损伤。在一项针对脊髓损伤或糖尿病导致的生殖器唤起受损妇女的小样本研究中,西地那非呈现出明显的积极作用。

研究5型磷酸二酯酶抑制剂(PDE5)有效性的一个新途径是,与药物睾酮联合应用。这种设想的原理是中枢神经的性机制的激活,这在解释性刺激作为性邀请时是必不可少的。这样,这些刺激可以引起(行为的)性反应(也即性欲和动机的增加,引起性接近的行为)。激活中枢的“性”机制,是激活一氧化氮途径的必需的条件,该途径的激活又是PDE5抑制剂发挥作用的前提条件。中枢作用药物将会增加对性刺激的敏感性,可能诱导出PDE5抑制剂能够发挥作用所需要的条件。在一个叫色测试(Stroop test)里色彩命名潜伏时间(Color naming latency times)是测量性线索的前意识的注意偏差的。在一个最初低注意力的小组里,睾酮使前意识的注意偏差增加,在另一个最初高注意力的小组里,睾酮使前意识的注意偏差减少。只有前面那个组在联合使用了0.5mg舌下给药的睾酮(产生高超生理水平)和10mg伐地那非之后,引起生殖器反应的改进(阴道脉动振幅)和性功能主观指数的改进,支持睾酮使大脑敏感化的观点,这恰好为伐地那非发挥作用铺平了道路。这项研究使用的高剂量睾酮限制了这一发现的普遍应用的能力。

总而言之,具有不同医学情况的妇女,不是医学上健康的妇女,可能具有生殖器反应的损伤,所以诸如PDE5抑制剂的生殖器唤起增强制剂更可能让她们具有更多收获。相似的是,其他短暂研究过的增强生殖器充血的药物,例如l-精氨酸,前列腺素,酚妥拉明,是不可能让具有唤起障碍的健康妇女受惠的。PDE5抑制剂结合高剂量睾酮在一些小组里可能比对照组具有有益的作用;然而,高剂量睾酮是需要安全性评价的。

(3)前列腺素E1: —种天然的血管扩张剂,一直用来通过海绵体内注射和经尿道给药刺激阴茎血流,曾局部用于女性生殖器,取得令人鼓舞的结果。尼克美等几家药物公司正在研制供女性局部使用的乳剂或胶冻剂。

(4)润滑剂:它们不仅对绝经期和产后妇女有效,对于性激素水平完全正常但仍存在阴道润滑问题的妇女也适用。这些年来许多新型水性润滑剂发生了改进,柔软光润,持续时间也更长久。它们比油性的润滑剂更可取,它们将不会损伤乳胶避孕套,也不容易引起感染。以硅胶为基质的润滑剂更适合于在水中做爱,或特别剧烈的活动,然而,它们可能更昂贵些。虽然证明它们是安全的,但润滑中所含的抗病毒的杀精剂壬苯醇醚仍能使某些妇女出现刺激症状。一般来说,润滑剂里添加剂数量越少,引起变态反应的危险就越小。

K-Y润滑剂———无化学活性的润滑剂是解决润滑问题的首选方法。

近些年来,抗抑郁药物的性不良反应已受到广泛关注。除了安非他酮、曲唑酮等之外,所有抗抑郁药常常具有抑制性功能行使的作用。能对抗抗抑郁药性不良的反应的药物包括金刚烷胺、赛庚啶、白果叶、萘发扎酮、育亨宾和低剂量心理刺激剂哌甲酯、右苯丙胺。人们抱有很大兴趣的问题是能否证实西地那非能有效逆转抗抑郁药的性抑制作用。因为西地那非是特定地增强性唤起/兴奋阶段的血管扩张,它可能不会改善SSRIs引起的性欲降低和性高潮延迟。事实上,西地那非甚至可能具有延迟性高潮的作用,有些临床医生已经注意到它能有效地治疗早泄。至今尚无双盲、对照研究来证实哪些药物能够逆转性不良反应,而这样的研究是绝对需要的。

(1)安非他酮(丁氨苯丙酮):是一种不常用的抗抑郁药,具有重要的多巴胺再摄取作用和继发性去甲肾上腺素再摄取作用,但对5-羟色胺再摄取或5-羟色胺受体活性没有任何作用。人们期待它具有积极的性作用,这是因为多巴胺刺激作用均与快感和雌雄动物实验研究的性刺激作用有关。虽然麻黄碱化学衍生物具有轻度多巴胺能刺激作用,它不会代谢为苯丙胺(安非他明),它不是一个需要按时服用的药物,也不会成瘾。它已被美国FDA批准为戒烟药,商品名为Zyban。小样本双盲、对照临床观察结果表明60\%以上的非抑郁所致性欲低下和其他性反应困难患者的性功能具有显著改善,而对照组不足10\%。治疗女性性功能障碍和/或抑郁症的最适剂量为225~300mg/d。性反应研究使用的是立即释放的方式,但目前经常使用的是恒定释放的配方。目前正进行恒定释放配方治疗性欲低下的临床对照研究。由于它的刺激特性带来的、成问题的不良反应包括情绪紧张、失眠、战栗和罕见的眩晕等。正在给具有性关切患者的处方是把SSRIs和安非他酮结合起来使用,显然,这一领域尚需要开展进一步研究。安非他酮的化学作用与哌甲酯是相似的,但其去甲肾上腺素活性则弱得多。它的刺激潜能比哌甲酯或苯丙胺都弱,所以它作为治疗性功能障碍刺激剂的可能性并不大。

其他多巴胺能药物———现在正在研究治疗可卡因药瘾的新的多巴胺刺激剂,但是仅有的、可供使用的其他多巴胺激动剂是非苯丙胺司来吉兰,它是作为治疗帕金森病的一种辅助剂而处方的。司来吉兰对老龄雄大鼠具有强烈的春药作用,它能使性懒怠的大鼠像年轻时那样频繁交配和射精,但尚未检测它对雌动物和女性的性作用。临床印象是它可能具有轻度增进女性性欲的作用,但不如安非他酮。然而,它是一个安全的药物,只伴有微弱的致焦虑或干扰睡眠的不良反应。特别要注意它的大多数化学作用是由于明显增加内源性身体刺激剂苯丙胺(PEA),过去10多年内,人们把它广泛宣传为一种就像巧克力那样具有春药般作用的“爱情化学”成分(没有任何研究或临床证据)。

正在进行ED治疗临床试验的一个经舌下形式给药、直接的多巴胺受体激动剂是阿扑吗啡。这个多巴胺激动剂是需要时才服用而且起效时间很快的药物。由于它具有明显的生殖器局部血管扩张作用和对大脑性反应中枢的诱导作用,它对刺激女性性高潮应该特别有效。然而,尚未进行这方面的临床研究,甚至调查性质的尝试。尽管认为阿扑吗啡仅仅是多巴胺能激动剂,它对刺激阴茎勃起的有效性表明它也涉及对NO的直接刺激作用。初期研究结果显示在多巴胺、NO和睾酮之间具有很强的协同作用。这就提示西地那非和多巴胺能制剂(就像阿扑吗啡和/或睾酮)的混合物可能最终用于治疗严重的女性性障碍。

(2)5-羟色胺能药物:有些妇女报告说因治疗抑郁症服用曲唑酮期间性欲出现不寻常增加或性唤起增强。新抗抑郁药萘发扎酮与曲唑酮在化学结构上是相近的,其特征为相对地缺乏消极性作用。这可能由于萘发扎酮是5-羟色胺2受体拮抗剂,而这种受体与SSRI诱导的性功能障碍是相关的。这就提示萘发扎酮是一个可能用于绝经后妇女的非常好的抗抑郁剂,特别是那些具有继发于内源性睾酮下降而出现的性反应下降。这些SSRIs,可能进一步加重这些患者的性功能障碍。丁螺环酮是5-羟色胺(5-HTIA)激动剂,一个治疗全身性焦虑障碍的处方药。用它治疗全身性焦虑和性唤起低下时(每天用药45mg,共4周),多数患者反映服药期间恢复正常性唤起和功能,没有任何证据表明性欲亢进。有理由对丁螺环酮的性作用作进一步调查研究。

(3)心理刺激剂:低剂量中枢兴奋药盐酸哌甲酯(5~15mg)和右苯丙胺,可以缓解继发于服用SSRIs和其他抗抑郁药的性高潮延迟或抑制。女性服用这些药物的作用比男性要强,这大概因为女性大脑对多巴胺具有更高的敏感性。苯丙胺和哌甲酯合剂的作用比右苯丙胺和哌甲酯要强。虽然这些刺激剂使女性容易达到高潮,但也可能具有潜在的减少性唤起和性兴奋的作用,这是继发于它们的血管收缩和潜在的诱发焦虑的作用。此外,因为心理刺激剂尚未批准用于这一目的,并可能具有心律失常、成瘾和其他不良反应,在处方时必须特别谨慎。

(4)替代品和非处方药:越来越多的人会服用一些非处方营养保健品以期增强性欲和性表现能力。然而,没有哪个保健品已通过以化验为依据的研究验证,它们的“证据”都是来自传闻轶事的。此外,关于它们的安全性,特别是在妊娠期间尚不得而知。银杏叶也许能逆转抗抑郁药引起的性功能障碍,作用机制包括增加生殖器血流、胆碱能刺激、血小板激活因子的抑制、增强前列腺素活性。一项开放试验表明它能。人参因含有植物雌激素和其他化学物质,能增进两性性能力。L精氨酸是NO的前体,NO是男子阴茎勃起的主要调节物。在许多非处方复方壮阳药中都会有精氨酸,已在人类和实验动物中进行过验证研究,并得到令人鼓舞的结果。大剂量口服L精氨酸能刺激老龄雄大鼠阴茎勃起。它能改善器质性ED男子的性表现能力,但仅当血浆和尿中硝酸盐和硝酸酯处于低基线水平时有效。给性功能障碍妇女局部应用L精氨酸与己酮可可碱或孕酮的外用合剂,可作为非处方使用。在性活动前30分钟局部用于会阴,轶事报告表明对有些妇女是有效的。麻黄是能提炼麻黄碱的植物,对性有积极作用,可能如上所述具有刺激外周交感神经系统的作用。蜂王浆是工蜂分泌的仅供蜂后服用的物质,据说具有春药作用。菝契含有孕酮和睾酮的前体。据说能治疗ED和治愈性病。

阴蒂治疗仪是设计来治疗女性性唤起障碍的。女性性唤起障碍是常见的女性性功能障碍之一,它的影响包括阴道润滑减少或缺乏、阴蒂不能充血肿胀、性满意程度减低及达到性高潮的能力减弱。阴蒂治疗仪适用于绝经前和绝经后妇女,它是一种使用简便、无创伤、无痛、高度有效和不良反应很小的女性性治疗仪器。它的上市给许许多多妇女带来了希望和极大的帮助。

在某些情况下,阴道和阴蒂的动脉血管会出现狭窄,从而干扰或影响阴蒂的明显肿胀,也因此而减弱了阴蒂的敏感性,如多发性硬化、肝硬化、慢性肾衰竭或酗酒。人们相信阴蒂难于达到或缺乏肿胀,将给女性带来一系列性问题或直接与女性性功能障碍有关,诸如性欲低下、难于达到性高潮、阴道润滑缺乏和性交疼痛。阴蒂治疗仪是设计来增加阴蒂血流并帮助女性阴蒂海绵体充血肿胀的,其使阴蒂充血肿胀的原理与男性真空助勃装置是一样的,它通过增加阴蒂周围的负压而促使流向阴蒂的血流增加。由于它可以增强阴蒂的敏感性,因此它可以用来治疗女性性功能障碍,它的作用还包括改善阴道润滑减少、增强达到性高潮的能力,在很大程度上提高整体性满意程度。

阴蒂治疗仪包括电池控制的真空装置和一次性的负压杯,使用时将负压杯扣在阴蒂之上,然后开机以使杯内产生负压,这样就能把血液吸入阴蒂之内而导致阴蒂充血肿胀。

如果患者存在以下任何情况就不能使用爱神阴蒂治疗仪:疼痛感降低;手的灵巧程度差;慢性的或复杂的泌尿道或阴道感染(过去一年之内);盆腔炎(过去一年之内);非阴道干燥所致的性交疼痛(过去一年之内);各种心理因素;性虐待史,抑郁等;影响有关性器官的手术史;服用可能影响性功能的药物;药瘾或药物依赖;未经治疗的萎缩性阴道炎;子宫脱垂;阴道痉挛;会阴前庭炎。

医生要充分、清楚、确切地向患者介绍如何使用、保养爱神阴蒂治疗仪。医生要向患者讲明下列注意事项:性交之后中断使用,或在30分钟的间断使用之后中断使用,连续使用不得超过4.4分钟。至少在1小时之内不得再次使用爱神阴蒂治疗仪进行治疗。在使用爱神阴蒂治疗仪进行治疗的过程中不得饮酒或服用其他药物。当打开爱神阴蒂治疗仪进行治疗而且负压杯扣在阴蒂上时,不得入睡。超时持续使用爱神阴蒂治疗仪有可能导致持久损伤。爱神阴蒂治疗仪不是用来进行避孕的工具。使用者在治疗过程中应坚持使用最小程度的负压来达到充血的目的。如果出现阴蒂的持续性疼痛就应停止使用爱神阴蒂治疗仪。不要在水中或靠近水的地方使用爱神阴蒂治疗仪。使用爱神阴蒂治疗仪时禁用油性润滑剂。

医生应向患者讲明使用爱神阴蒂治疗仪具有以下危险:过度使用或负压杯的使用位置不当时,有可能发生皮肤红斑、淤血或轻度不适。患者应在事前爱抚中先从低度负压开始使用爱神阴蒂治疗仪,对于第一次使用爱神阴蒂治疗仪的患者来说,在低度负压的情况下让阴蒂充血的时间不要超过60秒。如果还没有到60秒就已出现不适,应放掉负压并记住在开始治疗后多少秒钟出现不适的,那么在休息60秒之后可以重复在低度负压的情况下让阴蒂充血多少秒钟(或者说负压充血的时间在到不了60秒就引起不适时,应以不引起阴蒂不适为准)。她可以在间隔休息不少于60秒的情况下,重复在低度负压的情况下让阴蒂充血4次。在到不了60秒就引起不适的妇女应逐步延长负压充血的时间,直至可以达到60秒。

如果负压杯置于一处伤口之上,可能导致创伤的加重。虽然在临床研究过程中并未见到,但使用爱神阴蒂治疗仪过程中仍有可能出现青肿、淤血、血肿、疼痛或持久损伤。滥用爱神阴蒂治疗仪有可能加重某些原先就已存在的临床问题。滥用爱神阴蒂治疗仪有可能导致阴蒂肿胀或给阴蒂带来永久性损伤。如果患者在使用过程中出现上述问题,应停止滥用爱神阴蒂治疗仪并向医生咨询。

[案例]性唤起障碍

“今年29岁,结婚几年来每次同丈夫做爱时下身总是缺少润滑。只有前半年尚可,虽说同房时没有润滑液从阴道流出,但勉强还可以同房做爱,后来简直一点也没有。由于做爱时没有润滑液的分泌,丈夫得不到满足事小,而我的阴道由于总没有润滑,表面全被磨破了,每次做爱我都非常疼痛。同时,丈夫的阴茎也会被磨痛,由于这种原因,我做爱时别说有快感,只要能不疼就是蛮好了。因此我非常讨厌性交,而丈夫又不能克制。最难以忍受的是丈夫由于从我身上得不到满足在外寻花问柳,我一干涉就要和我离婚。另外丈夫还嫌我所要求的做爱姿势与别人不同,他说别人都是上下起伏抽动,而我总希望丈夫紧贴在身上做上下平行滑动,希望能给我顺便介绍一些姿势。唉,羞死人了,请不要耻笑。万望你能为我解决没有爱液的问题,让我像其他女人那样能在做爱时爱液外流而又欢乐无比。”

对于年轻妇女的性唤起障碍,我们首先要了解她的内分泌状况如何,即卵巢等的功能如何,月经是否正常,有无卵巢功能早衰的问题,有无生育等。由于缺乏这些信息,我们不能妄加分析,如果有卵巢功能早衰的问题需要补充雌激素和孕激素;如果有炎症要积极治疗。此外,还需了解双方事先有无充分的爱抚,女方有无夜间自发分泌现象还是仅在性交时才出现分泌障碍。最简单的办法是使用润滑剂来解决润滑不足的问题,市面上已有很多产品在销售,其实使用避孕药膏也可起到同样作用,价格也便宜得多。至于她对性交姿势的喜好可能与她的阴道口位置偏低有关,这样若想做上下起伏的抽动,势必造成不适,而平行滑动则避免了不适,但使男方倍感吃力。解决办法是用薄枕头将臀部垫起,这样就能使双方都感到舒适,而且有助于刺激阴道前壁中点的性敏感区(G点)。

一位云南的农村妇女背着丈夫写来这样一封信:“我是今年结的婚,但性生活一直不成功。我23岁,对丈夫也有性要求,可是我们在接触时我的心总是很平静,不管怎么抚摸阴道总是干干的,润滑不起来,使他很难进入,即使进入后,运动起来阴道里总是火辣辣的并有轻微疼痛感,很少有快感。医生,为什么我有性要求阴道总没有分泌物,你们一定要帮我改变这种状况,否则我就只好和丈夫分手了,我已经鼓足了很大勇气才写了这封信,你们一定要在百忙之中替我解答,我这里很偏僻,专心等候你们的来信。”这是一例典型的性唤起障碍,但她的来信太简单,只讲了症状———性欲正常而阴道不能润滑,虽然有必要的抚摸但仍没有起码的性反应。所以当即回信询问她的性发育史、性心理发育的情况、性观念、夫妻关系如何等,试图寻找出有无心理与人际因素的影响。此外,还应根据月经和第二性征情况判断一下激素水平是否正常,否则可以安排人工周期治疗。治标的办法是建议她使用阴道润滑剂以解燃眉之急。

(马晓年)


\chapter{第十章 女性性高潮障碍}

在不同的历史时期,女性性高潮一直被人们赋以不同的意义和重要性。究竟性高潮对女性及其性伴侣以及对她的婚姻来说是好事还是坏事?即使到了21世纪,人们还在争论,不过,主流认识还是更多地把女性性高潮视为具有积极意义的事。所以,人们会把女性性高潮缺乏看做是需要给予临床关注的一个问题。

西方在17世纪末至18世纪初认为女性要想怀孕就必须达到高潮,而到了19世纪30年代,人们开始相信即使没有乐趣、没有高潮甚至没有知觉仍有可能妊娠。从临床视角来看,19世纪后半期传统的、维多利亚时代的、极其消极保守的性价值观念把性问题看做是需要治疗控制和社会抑制的,而更赞成性的20世纪的现代性学则认为性是需要刺激和促进的,性行为和性感受的抑制或丧失已变成性治疗界和社会所关切的事。

首先给20世纪性观念带来重大转变的是西格蒙德·弗洛伊德的开创性工作。他以人格成熟为基础对女性性高潮作了分类,他把阴蒂高潮看做是肤浅的和不成熟的,而把阴道高潮看做是可信的和成熟的。尽管现在看起来有些荒谬,但是当时这种以解剖学为基础的、对高潮不同类型的区分,仍代表了从传统认识向更科学、更合理方向的重要改变。它承认女性性高潮的存在和力量,挑战了所谓“为生殖进行的性活动才是更有价值的和正确表达的”旧文化范式。与弗洛伊德同时代的性学家哈夫洛克·霭理士,不仅指出女性的性反应和男性性反应是等同的对应物,而且女性性能力可能比男性更强,女性性行为更神秘、更广泛,因为女性不仅有阴蒂而且还有子宫,那么女性对性刺激作出反应时具有更大的、更弥散的解剖学基础。阿尔弗雷德·金西和他的同事们(1953)从完全不同的视角提出了女性性高潮的新概念,他们把性高潮作为是人类性行为的若干标记物之一,把性高潮频率看做是个人的、人群的、社会的性人口资料的关键变量之一。金西还报告说,女性更容易在手淫时而不是在与伴侣性交中达到性高潮。马斯特斯和约翰逊(1966)则根据他们的实验室研究资料指出,所有形式的女性性高潮具有同样的神经生理反应过程,女性性高潮与男性性高潮的相同之处要远远多于不同之处。这时的社会大气候比之前的金西年代更直率、更有利于性平等,女性开始提出要求生育权利和高潮权利,她们已经能够享受到刚刚研制成功并投放市场的口服避孕药的帮助。马斯特斯和约翰逊的《人类性功能障碍》(1970)提出一种非病理模式的训练计划,以改变很多女性从未或极少经历性高潮的状况:只要消除焦虑,就会出现自然表达的性反应。

回顾历史时,会发现这样一个事实:女性性高潮问题往往不是纯粹的医学问题、而很可能是一个人际的或社会文化的问题。而在100多年前,人们是不可能这样看问题的。于是我们很难把高潮问题简单地分类为纯躯体的或纯心理的问题。一位女性自己的历史经验、性关系、成长和生活文化环境都持续地与她的性神经生理学发生相互作用。人们仍需致力于探讨和研究躯体、心理和社会文化因素的错综复杂的相互作用的方式。


\section{第二节 女性性高潮障碍的定义和流行率}

关于女性性高潮的通常定义是阴道和盆腔区域的主观体验和生理变化的总和。妇女使用的主观描述包括“达到高峰”,感到肌紧张的逐渐建立并在生殖器区域逐渐增强的肌肉收缩感受,和(或)一段时间的高度兴奋及随之发生的相当突然的释放而达到完全松弛。性紧张的释放总会伴有肉体上和情绪上的变化,这就表现为性高潮。这些生理变化就像周期性因饥饿而进食一样、其心理变化就像进食后饥饿感消失并得到满足。对于性高潮来说,并不需要明确的主观体验;例如,有些妇女报告她们了经历性高潮而没有伴随肌肉节律性收缩的感觉。性高潮的另外的主观感觉是仿佛丧失了时间在消逝的感觉。她们似乎低估了高潮经历的持续时间,她们报告的只是实验室里测量到的时间的50\%。关于性高潮的心理和行为指数是马斯特斯和约翰逊率先研究测定的。他们记录了发生性高潮时涉及整个身体的多种生理表现方式:如很多肌群的节律性收缩、因面部表情肌收缩而显示的痛苦状面容、全身肌肉紧张……,但这些现象并不一定是性高潮体验所必须发生的。

就男性而言,尽管射精基本上总是与性高潮相伴随,但它们并不是同义词。性高潮可以在不射精的情况下发生,如少数男子能在一次性活动中达到多次并不伴随射精的性高潮;反之,射精也不一定伴有性高潮,如不少男子声称虽然能射精但没有感觉。所以不能以射精等性高潮伴随成分的出现来推断性高潮是否发生。就女性而言,目前并没有能与男性射精相对应的女性性高潮识别标志,如我们并不能以阴道收缩的有无来界定女性是否达到了性高潮,因为太多的女性并没有在性高潮体验中经历或者说主观感受到阴道的收缩。有人根据对936名已婚育龄女性的性高潮调查资料把女性性高潮区分为以下8种较为单纯的类型:阴道收缩型;周身暖流型;全身抖动型;电流通过型;飘飘升空型;呻吟不安型;醉酒朦胧型;嬉笑狂欢型。其实还应该具有角弓反张型等更多的更加剧烈的表现形式。除了这些单纯类型的典型表现外,女性还可以同时表现出上述8种单纯类型中的两种、三种或四种的混合表现,具有单纯类型和两种混合类型的居多,可占到2/3左右,其中88\%的人在高潮时有抱紧对方的表现。总计有12\%的人从未达到过性高潮(5\%只有快感而无高潮,7\%既无快感也无高潮)。由于人们更多地把注意力集于阴道收缩这类生理反应的有无,而往往忽视了她们的主观和认知方面的乐趣。所以女性性高潮的表现是复杂的(类型和强度的多样化)、多变的(即使同一个体对相同刺激的反应不尽相同)、迟发的(比男性慢得多)和多发的(女性多没有高潮后的不应期,具有在一次性活动中达到多次性高潮的能力)。

高潮反应过早意味着高潮发生在个人所想往的、合理限度的时间之前。换句话说,高潮反应过早表明一个人理想的期望和实际表现之间出现明显差异,于是他自己、尤其是伴侣对达到高潮的时间产生抱怨。出现的原因包括焦虑、禁欲过久、糖尿病等。虽然高潮反应过早,但排精、膀胱颈机制和射精过程均完整。

高潮反应延迟意味着超出个人期待的、性活动拖延过久,但最终还是能够达到性高潮,排精、膀胱颈机制和射精均完整。对于一个完善的诊断来说,高潮伴随成分的任何缺乏或改变都需要加以详细说明。其原因包括:性唤起不充分、焦虑、神经病理改变、肌肉病理改变、药物作用等。

高潮反应损伤指性高潮反应强度和快感显著减弱,患者仍维持前向排精,表明排精是正常的,精液在没有推动力的情况下经尿道口溢出,表明横纹肌活动不足或受抑制。

缺乏高潮反应指未能经历性高潮。有些病例完全缺乏伴随的快感,但排精、膀胱颈机制和射精是正常和完整的,称之为射精麻痹或性高潮麻痹。如果个体无论在清醒或睡眠时均不能达到性高潮,也没有排精和射精的证据,则可称之为完全性不伴释放的缺乏高潮反应。

张力增大感(增涨、向顶峰发展、肿胀);

张力释放(解脱感、释放感、爆发似的、突然爆发的);

快感的散布感(涌现、散布、放射感);

整体投入(战栗的、阵挛的);

射精感觉(冒出、射出、爆发);

节律性感觉(颤动、律动、悸动);

热感(感觉发冷、温暖感、燥热感)

其他感觉(痒感、刺痛、红晕)。

时间评价(太快、较长、无尽);

强度(轻微、中度、强、强烈);

躯体作用(放松、吞没、耗竭感);

深度(深在、充实、十全十美);

完全释放(不可避免感、舒适、愉快感、享受感、高兴、难以置信);

肉体释放(平和、情欲、极度);

满意度(满意、令人满足、完成任务似的、精神发泄);

兴奋(振奋、兴奋、狂热)。

情感亲昵(亲密、爱情);

享受平静(平静、祝福、温柔);

享受兴高采烈(兴高采烈、狂喜、欣快、欢天喜地、激昂);

情绪兴奋(纵情、不能控制);

情感融合(沉浸、合二为一);

幻想(轻浮、茫然、头晕目眩);

意识暂时丧失(超然、不真实);

悬吊感(悬浮、飞翔);

其他感觉(融化、脆弱)。

可见女性性高潮反应形式之复杂和难以确认,况且她们还可能伴有性欲或性唤起方面的问题,这就会使其临床表现或个人感受出现更复杂的局面。

在新的诊断术语中,诊断仅表明主观报告的高潮障碍,因为性高潮的伴随成分不像男性那么容易得以确认。新的分类是描述性的,它的系统框架拥有足够的空间,一旦将来验明这些高潮伴随成分后便能把它们安排进去。

女性高潮反应过早并没有形成一个诊断名目,只有极少数女性会因为高潮反应过早而阴道分泌停止,乃至性交无法继续进行并导致男方的抱怨。

人们往往对男子的早泄概念精雕细刻,却草率地对待女性高潮反应延迟问题。也就是说,如果女性能够克服她们自身性能力不足和高潮阈值过高的问题,她们就不会需求太长的性交时间,男子也就不存在那么多早泄问题了。新分类和治疗战略着重解决增强女性性能力、调整女性性唤起模式和如何降低她们的高潮阈值。

与某些男性一样,女性也可能主诉其高潮强度的降低,这是由于深在的内心冲突、服用药物或使用违禁药和毒品、性内疚或传统消极性观念导致性欲低下所致,她们仍可以在缺乏快感的情况下达到高潮。性学文献对此尚少有报道,在实践中应注重对此的专门调查。

人们推测不能经历性高潮是性抑制的结果,不过其具体临床表现是很难准确叙述或判断的。女性可以有阴道收缩而没有高潮快感,而阴道收缩并不是高潮的绝对必要的伴随成分。也很少有女性主诉睡梦中的高潮。

性高潮障碍指性高潮反应出现的时机过早或过迟、反应程度的损伤乃至缺乏。主要包括:男性的早泄与不射精与女性的性高潮障碍(又称无高潮)。国际定义委员会专家指出,对于性高潮障碍的定义,始终存在的困难是许多仅仅有最低限度唤起的妇女(因此可以预期是没有高潮的)一直被误诊为性高潮障碍。因此需要作出这样的澄清,报告有高水平的唤起和兴奋的妇女,已经激发了对高潮释放的一种需求,但又不能达到这一目标。

女性性高潮障碍系指女性虽有性兴趣和要求,性欲正常甚至较强,但在性活动时即使受到足够强度和有效性刺激并出现正常性兴奋期反应(如生殖器肿胀和阴道充分润滑)之后,性高潮仍反复地或持续地延迟或缺乏,她们只能获得低水平的性快感,因此很少或很难达到性满足。它与性欲或性唤起困难是不同的,女性性高潮障碍属于一种独立的综合征。人们目前对女性性功能障碍的认识远远赶不上对男性性功能障碍的认识。女性在激发性高潮的刺激类型或强度上存在广泛的变异。诊断要由医生作出,妇女的性高潮能力确实低于按她的年龄、性经验和她接受的性刺激所应该达到的水平。这一障碍确实引起显著的痛苦和人际关系的困难,应除外其他可能作出的任何诊断。根据目前的临床研究和长期观察没有发现在特定的人格特征或心理病理状况与性高潮障碍之间的联系。性高潮障碍可以影响体像、自尊或关系满意程度。按照对照研究,高潮能力与阴道大小和阴道口的位置高低无关。盆腔肌肉强度一般不影响女性性高潮能力。

马晓年等(1993)提出性高潮障碍可分级为:

Ⅰ级:既往有过性高潮史,但目前性高潮缺失。

Ⅱ级:性高潮延迟,指在足够强度和时间的有效性刺激下,女性在兴奋期性反应出现20分钟以上,仍难出现性高潮。

Ⅲ级:从未获得性高潮。或除性高潮障碍外,还同时具有性欲低下、性唤起障碍及性感缺失,呈全程式性功能障碍。

Ⅳ级:从未获得性高潮,并经多种治疗仍无改善(难治性性高潮障碍)。

性高潮障碍又可分为完全性和境遇性两类:前者指女性在任何情况下、与任何性伴侣都从来不能达到性高潮;后者指女性在有些条件下能达到性高潮,而在某些特定条件下又不能达到性高潮,如不少妇女在自慰时、口交时、性梦中、使用振荡器时、性交同时由自己或丈夫给予阴蒂直接刺激时能够达到性高潮,而在单纯性交时则不能,这就说明个体的身心因素确实在起作用。这些妇女也会寻求治疗,因为她们认为自己是有缺憾的,甚至感到痛苦。同时,这也令男方倍感压抑。金西曾指出,大多数妇女在仅仅进行阴茎阴道性交时是不能经常达到性高潮的,如果夫妻双方都懂得这一点,问题就容易解决了。换句话说,性交无高潮究竟是一种正常的变异,一种误解或不现实的期待?还是病理性的或能构成性功能障碍诊断的?到现在仍是仁者见仁智者见智。但由于其发生率尚不得而知,只有经过对深入、彻底调查的结果进行详尽、科学分析之后,才能判断它到底属于什么性质的问题。当性高潮障碍为境遇性的,妇女也往往存在性欲或性唤起障碍。

有些女性能够享受到性高潮、但却把性交时男女不能同时达到性高潮看成是严重问题,这是受了某些文人墨客过分渲染或某些婚姻手册不负责任介绍的影响,这种为性活动预先设置既定的、过高目标的作法总是趋于削弱而不是增进性兴奋过程。因此,同时的性高潮是可遇而不可求的事,同时的性高潮实际上必然要浪费其中一方的性高潮,因为性高潮是在性紧张水平达到一定程度后才发生的,它需要一个人完全沉浸于自己的身体感受,一心焉能二用。

性高潮障碍可以划分为原发性和继发性两类。原发性性高潮障碍又可称为高潮前状态,因为她们从未在任何知觉状态下、以任何方式达到过性高潮。她们往往很沮丧,特别是一些中年女性,大有命乖运蹇之叹,感慨已虚度大半生。继发性性高潮障碍指过去规律性或间断性获得过性高潮,而现在不能以她们所推崇的方式达到性高潮,或者是高潮频率减低或需要一些限定条件(例如只能通过手淫)。性高潮障碍大多数是终生性的,因为一旦获得性高潮之后很少会丢失这种能力。继发性性高潮障碍很可能与交流障碍导致的人际关系紧张、性创伤经历、年龄变化带来的性紧张度的降低、身体状况差或患有某些疾病等有关。临床和流行病调查研究都没有检查原发性和继发性的区别。同时也没有区别单纯性高潮障碍的女性和兼患其他性功能障碍的女性。

这两种类型的妇女通常都试图促成一个不随意反应,而自发地抑制所有感知到的、有可能导致性高潮的美好感受,这样就无形中忽视或破坏了性反应向高潮阶段的发展。她们容易自认为肉体吸引力不足,对自己的身体缺乏认识,她们不能接受或总是拒绝任何积极刺激,哪怕是非性的接吻、拥抱和抚摸。她们在男子性唤起时往往不能产生相应的对等的唤起就是因为上述消极感受的影响,同时也反映出她们缺乏一定的性经验。她们有时相信自己可能像火箭发射一样突然出现强烈的高潮反应,而且连发射之前的倒计时的时间也不给。她们往往有意识地显示自己的能力,但往往不容易体验具体的通向性高潮的种种细节。妇女总是丈夫高潮的见证人,但却难得或根本没有机会见到其他女性的性高潮,尽管她们可以与女性深入讨论性问题,但往往不尽如人意。如果她想象自己能和男人一样有神奇的勃起和射精,她有可能寻求更多的肌肉和骨盆运动、有力的抽动、发声和急促的呼吸。有些妇女自认为自己的责任就是取悦男方,这是婚内不得不尽的义务,她们便会从自己的感受分心或勉强自己做不情愿的事。由于她们本身就是男子性反应的一种强有力的刺激物,她们所激发的男子的渴求和反应速度往往会导致男性的早泄并使性爱活动过早告终。她们可能总是寄希望于丈夫给她们带来性高潮,而她又不需付出任何努力,实际上这种努力恰恰是她达到性高潮的必备条件。她好像对自己性行为的理解和满足不负任何责任,好像不必对爱抚和其他刺激给予任何正确的引导。

虽然人们普遍印象中女性性高潮障碍是常见的,是女性最普遍的抱怨,但过去几乎没有什么严格的对照研究来认真调查其发生率。唯一的例外是美国芝加哥大学完成的美国国家社会和健康生活调查(1994),在这项随机抽样概率样本中包括了年龄在18~59岁之间的1749名美国女性。样本中包括了不同肤色、不同人种、不同教育、经济、宗教背景、及不同性取向的人群。性高潮问题是第二位最常见的能确认的性问题,约有24\%的女性主诉在过去一年中,至少也有几个月或更多的时间缺乏性高潮。Johannes &Avis(1997)报道在以美国马萨诸塞州人群为基础的随机样本调查中,349名年龄在51~61岁的女性中有10.3\%声称完全或大多数时间具有高潮困难。Osborn,Hawton,&Gath(1988)报道英国一项涉及35~59岁521名女性的随机社区样本调查中,有16\%的人主诉很少达到性高潮。还是这个样本,若仅仅统计那些有伴侣的女性,在过去3个月中就有15.8\%的女性在和伴侣在一起时没有性高潮,还有22.2\%的女性达到性高潮的频率不足一半。以临床为基础的资料显示出女性性功能障碍的比例更高,一项在妇科门诊患者中作的调查表明她们中的29\%有高潮问题,11\%有经常性性交疼痛,38\%在性生活中有焦虑或抑郁。重要的是要注意到性高潮问题并非总要引起女性性痛苦或婚姻不幸福。一项非随机样本表明63\%的女性主诉存在性唤起和高潮问题,但她们又声称婚姻是幸福的,85\%的人主诉她们对性关系是满意的。


\section{第三节 女性性高潮障碍病因学}

虽然有些女性声称自己能够获得性高潮,然而经过医生更细致的问诊后却了解到她们根本没有达到过性高潮,甚至根本不知道性高潮究竟是怎么回事。必须通过学习,她们才可能识别并且体验性高潮。也有些女性实际上已经经历了性高潮但她们自己却没有识别出来,自以为她们存在性高潮问题,因为她们错误地认为性高潮是一种令人震撼的、天惊地破般的神奇经历,于是产生过高期望,殊不知这些往往是小说、电影中过分渲染的不良影响。当女性体验更广泛的性刺激方式和获得有关她们身体的更广泛知识时,性高潮能力就会增强,可以说如果你对自己有无高潮缺乏把握,那很可能是没有高潮。对女性性高潮障碍而言似乎没有绝对的诊断标准,因为女性对此所持的观点有很大差别:有的女性满足于与丈夫沟通感情和一般的肉体亲昵,不一定要求性高潮;有的则要求经常、次次甚至一夜不止一次的高潮;有些女性能够获得性高潮、但却把男女不能同时达到性高潮看成是严重问题,这是受了某些错误宣传的消极影响,这种为性活动预先设置了既定的、过高目标的作法总是趋于削弱而不是增进性反应过程。客观地、科学地讲,同步性高潮是可遇而不可求的事,因为性高潮是在性紧张水平达到一定程度后才发生的,它需要一个人完全沉浸于自己的身体感受,不能一心二用。

如果女性希望获得性高潮而未能得到满足,临床上就可以把她划归进性高潮障碍范畴。无高潮女性多分为两种类型:一种是毫无感受或感受很弱型,她们会说:“我讨厌触摸”,“我从来没兴趣”,她们往往没有主观层面的快乐的感受和肉体层面的感觉;她们会有浪漫的白日梦,但没有动情的主动投入的性念头;在治疗过程中会觉得是父母过分严厉的管制毁了她们的一生,使自己处于性压抑状态。另一种是感受强烈型,性的感觉虽然很强,性心理探索也处于较高水平,只是无法达到顶峰,这与总是预见到失败、有意回避性高潮反应、畏惧性反应中的失控等心理因素有关。因为上述消极感受的影响,这两种类型的女性都可能不知不觉地抑制了性接触过程中的美好感受,无形中忽视或破坏了性反应向高潮阶段的发展。有些女性自认为自己的责任就是取悦男方,这是婚内不得不尽的义务,她们会从自己的感受中分心或勉强自己做不情愿的事,她们总认为丈夫必然能给她们带来性高潮,而不需自己付出什么努力,实际上这种努力恰恰是她达到性高潮的必备条件,并且她们还应该对丈夫的爱抚和其他刺激给予正确的引导。

性高潮障碍可以由器质性因素(如破坏调节高潮反射的脊髓中枢的退行性疾病和肿瘤,糖尿病或内分泌病,还有酗酒和药物的影响)、心理性因素或二者共同引起。在评价影响性功能表现的可能的心理因素之前,必须首先评价生理、疾病和药物的影响。然而,生理因素的存在并不能排除引起性功能障碍的心理因素的存在,心理因素往往会使器质性问题变得更糟。很有可能的是单纯的器质性因素或心理因素都不足以引起性高潮障碍,而它们的共同作用则导致性功能障碍的发生。因此,即使是器质性性功能障碍者也不意味着其心理治疗在解决性困难时显得无能为力。而且,如果不积极处理心理因素,治疗性功能障碍器质性因素的医疗措施也可能解决不好性困难和性满足问题。

根据对照研究,高潮能力与阴道松紧和阴道口的位置高低无关,但是和有些解剖部位(即性敏感区)密切相关,如阴蒂、G点、阴道口、阴唇、乳头。尤其是阴蒂,其神经分布是很丰富的,对触觉是非常敏感的,而且也具有血管充血能力。如果患有糖尿病等导致这些神经末梢有缺陷时,可造成感觉减退或称生殖器麻痹,相对而言,阴道对触觉就不那么敏感了,只有对G点实施足够强度的按摩、施压和滚动刺激才能成为有效的本体感觉刺激。另外,盆腔肌肉强度会在一定程度上影响女性性高潮能力,在位于阴道4点和8点处的耻骨尾骨肌(简称PC肌)上也具有丰富的本体感觉神经末梢,它们能够在压力下或抽动过程中产生愉悦的感受。这样PC肌的强度和灵活运用也是影响性高潮能力的重要因素之一,二者之间不仅仅是特异性的协调关系,进一步的证据还表明阴道刺激能够提高女性的疼痛阈值。有关对G点的存在与否及其在性高潮中所起的作用的争论至今尚未结束,但它无疑是阴道刺激作用的一个有力证据。

大脑也是女性性唤起和性高潮的重要来源,虽然还缺乏直接的研究证据,但的确有些女性可以在不对生殖器进行直接刺激的情况下获得性高潮,如截瘫患者能通过幻觉而达到性高潮,催眠术也可引发性高潮,在实验室条件下直接刺激大脑区域也能引起性高潮,此外,也有报道指出接受阴蒂和阴唇切除及阴道再造的女性仍能获得性高潮。Whipple,Ogden &Komisaruk(1992)研究了10位自述能单凭性幻想而引发性高潮的女性,她们无论在自我想象时还是在自我刺激生殖器并伴有同样想象时都能达到性高潮,这一证据提示并指出:为性高潮经历所必需的包括大脑在内的性解剖部位都是重要的。Whipple&Mc Greer(1997)提出涉及性反应的几个不同的神经通路:阴部神经接受阴蒂刺激、腹下神经丛和盆神经接受阴道刺激、迷走神经直接从宫颈通向大脑。

人们至今并不完全了解与女性性反应直接相关的神经解剖和神经生理,现在的文献都是从男性研究中推得来的。女性生殖器的神经支配是经由躯体的和自主的两种神经系统所调节的。躯体神经分布是通过来自骶2~4节段的阴部神经而传导的,它们向两侧行进分布到盆腔;自主神经分布的神经纤维分别来自交感和副交感系统。交感纤维来自脊髓的胸10~腰2节段,副交感纤维来自骶2~4节段,而这些资料主要来自男性研究。Alexander &Rosen(1995)完成的一项对照研究表明具有脊髓胸10或腰1以上节段脊髓损伤的女性中有52\%能获得高潮,而躯体能正常运动的女性则100\%能达到性高潮。当生殖器—脊髓反射中枢兴奋性减弱时,女性对外界性刺激作出反应的能力减退,性感觉和高潮反应也自然受到影响。有报道表明具有脊髓损伤、会阴切除术、阴道切开和重建手术者易伴有性高潮障碍,但是慢性疾病如糖尿病或盆腔肿瘤更容易损害性反应的唤起阶段,而高潮能力却相对完整。尽管高潮的生理的和主观的反应都是重要的,但它们却并非总是相关的。与性唤起、性体验相似的是,人们尚未发现在呼吸率、心率或平均盆腔收缩次数与报告的高潮主观强度或满意度之间的相关关系。此外,人们也没有发现在阴道血流增加、高潮潜伏时间或持续时间与自我感受的高潮强度之间的相关关系。Goldstein &Berman(1998)提出生殖器血管生成不充分是女性性唤起障碍和潜在的性高潮障碍的可能原因之一。与男性勃起功能障碍相平行,动脉血管硬化是一个常见病因。目前正在进行有关这一推测的研究,但尚无结论性结果。

内分泌因素可能也会对女性性高潮能力产生一定影响,如甲状腺功能低下或亢进会造成血睾酮转化水平降低或女性性激素减低的负面影响;但也有矛盾的或相反的报道,如女性在双侧卵巢切除后仍可有满意的性欲和性高潮。

女性生殖器的神经分布包括胆碱能的和肾上腺素能的。此外,也发现有些多肽涉及神经传导或神经调节作用并定位在女性生殖器组织中,并且发现这些多肽中有3个涉及女性性唤起过程。希望在今后的10年间能对神经多肽及神经递质物质进行深入研究。与此同时,通过药物对女性性行为的影响已清楚地了解到5-羟色胺能因子在唤起、高潮潜伏时间和性欲方面扮演重要角色,重要的例子是增加可利用的5-羟色胺和降低多巴胺水平会对性反应具有消极作用,不过,其作用机制尚不完全清楚。此外,肾上腺素能阻断剂或外周抗胆碱活性可以延迟或抑制性高潮。

有人综述了精神治疗药物引起女性性高潮抑制的文献之后指出,这些女性在接受治疗前具有正常的性高潮,一旦停药后性高潮能力恢复,而且像单胺氧化酶抑制剂(MAOIs)对性的不良反应呈剂量相关性,剂量过低时原发疾病则得不到控制。类似的抑制作用也见于男性,表现为射精抑制。有人提出MAOIs和三环类抗抑郁药的抗胆碱能副作用是女性性高潮抑制的可能原因,但考虑到实验证实阿托品不能抑制女性性高潮的研究结果,所以人们对这些药物的阿托品样作用造成女性性高潮抑制的观点表示怀疑。另一方面,尽管MAOIs的确抑制女性性高潮,但它也的确不具有明显的抗毒扁豆碱样乙酰胆碱能作用。已经有人提出精神治疗药物所致肾上腺素能和胆碱能神经之间的不平衡是造成女性性高潮抑制的另一个原因。然而,它不能解释为什么硫利达嗪(甲硫达嗪)比其他吩噻嗪更容易引起性功能障碍。不同神经递质系统的受体特异性和不同的精神治疗药物的分子结构,可能是不同药物具有不同副作用的原因。有关女性性高潮抑制的药物原因还有待于更多的临床和药理研究,医生在工作中应注意询问患者的性困难是否自服用某些药物后才出现的,通常可尝试换用另一种类似药物的办法,既不影响对原发疾病的治疗又不至于造成对女性性高潮的抑制。

女性经历性高潮的能力在很大程度上似乎受很多因素的影响,除解剖与生理、性知识与技巧之外,社会文化与性价值观念等因素的作用也不容忽视。尽管人们一直推测宗教、教育、年龄、社会阶层和其他社会文化因素等对性有种种影响,不过,可能影响性高潮难易程度的心理因素确很难得到证实。在评价可能造成性功能障碍的危险预测因子中包括:家庭收入少、性生活贫乏、性念头很少、青春期前性接触史、受性骚扰史、遭强奸史。健康和生活方式问题包括:是否患过性传播疾病、尿道感染症状、健康状态差、情感问题或紧张等,它们也是相当重要的预测因素。

Laumann等(1999)指出,与女性无高潮相关的重要因素只有受教育程度、婚姻状态和年龄。能够经常达到性高潮的女性包括:年轻、婚姻整体关系和谐和婚龄短的女性,女运动员,有创造性的女性,舞蹈演员,活跃的女性,小时候带男孩子气的女性,能够充分放松的女性,对她们的身体和感觉有所认识的女性,把性看做是生活重要组成部分的女性,富于挑战的女性,有决断能力的女性,生气勃勃的女性,敢于明确表示可否的女性,自尊心强的女性。当然也有少数女性属于害羞和被动型。他们发现没有宗教信仰的女性是最难于与最初的性伴侣有性高潮的,而宗教信仰浓厚的女性的性高潮发生率更高。这与典型的临床经验恰恰相反,这可能反映出参加随机抽样调查的人群与临床就诊人群有所不同。当前的精神状态与高潮缺乏也有关联,与无高潮女性相比,性功能正常的女性较少与伴侣有情感纠葛,她们在第一次性交时具有更强的唤起和更多的乐趣,对她们伴侣的身体具有更积极的反应。但是至今没有证明哪种影响因素是把性反应阻断于兴奋期的主要或绝对原因。

目前看来,从来没有达到高潮或难于达到高潮并寻求治疗以解脱这种困境的女性,往往容易受心理因素、文化因素和社会条件对生理活动的干扰与影响。但她们并非都具有早年消极性教育或性创伤历史,而且具有这样历史的女性也不会都具有性困难。问题是“有高潮”、“无高潮”或“境遇性高潮”等标签本身就太笼统并很难产生可靠的区分因素。如果高潮生成和无高潮生成的模式仅仅是以早年历史因素、个人的(身、心)因素和相互关系因素为基础而建立起来的话,则我们可能发现更满意的结果。研究性高潮问题的困难在于人们尚缺乏以下知识:①如何通过简便易行的神经生理的客观测定来证实性高潮何时发生;②探讨性高潮客观神经生理测定与主观感受的一致性。后者对性治疗学更是尤为重要的。

下面将讨论性领域常见的社会因素及它们是如何导致性高潮障碍的发生。

很多人存在根深蒂固的错误信念和错误信息,如不懂得阴蒂的作用,不懂得男女性反应的差异,对性行为的正常与变异的认识存在争论,对中老年性活动的错误认识等。

[无高潮案例1]

“我今年35岁,我们结婚已有7年,但婚后这么久我还没有过一次满意的性生活。在结婚前的几年里我们有每月1次左右的性生活,由于害怕怀孕,所以都是采取体外射精,每次既紧张又匆忙的性生活根本没给我们带来多少欢乐,大概也是因为这种原因使丈夫习惯于快节奏。婚后虽然不再紧张了,但仍不理想。我们常常先用阴茎摩擦阴蒂,少则一、二分钟多则十几分钟,待我觉得阴蒂深处有一种快感,随后便有一种强烈的性欲时,就要求丈夫快速强有力的进行性交抽动,但在这种情形下往往十几秒钟就射精了,我自然也没能达到高潮,而且觉得很难受。再以后,我因为怕难受就不再摩擦阴蒂了,虽然丈夫也能把性交时间拖长到十几分钟,可时间一长我只感到整个阴部麻木、大脑疲劳,只想快点结束,结果时间延长仍不能令我满意。1991年春我生了孩子,得了一种怪病,大脑疲劳,入睡困难,多梦易醒,现在每晚靠安定入睡。由于身体不好,那年基本上未性交,就在这时我开始手淫。每当性欲强烈时,我常常忍耐不住用手去摩擦阴蒂,刚开始时能得到一种短暂的快感,可随之而来的是一种难以忍受的滋味,这种滋味我说不太清楚。觉得那会不会是一种性欲的极点或是书上讲的阴蒂高潮?起初我怕不卫生,即使难受也不去触及阴道,以后想得到满足就用手快速摩擦阴道,可仍不能满足,最后总因为身体和大脑都疲惫而失望地停下来,并难受得乱滚,那痛苦真是无法言喻。它曾不止一次让我想到死,想到毁掉我自己的性能力。不光手淫,可恨的是还有梦交,梦交时常常出现这种清醒时长时间摩擦阴蒂才出现的阴蒂性高潮,有时不出现,醒来后就再用手去摩擦。

其实我内心并不愿意手淫,而且想改掉它,因为它并没有给我带来满足,只是在梦醒之后难受之时才这样做的。另外,我看到书上讲的达到性高潮时可能暂时意识丧失之类的话也让我感到很紧张,像我这样有严重神经衰弱、大脑经常疲倦的人能承受得起性高潮吗?”

这位女性描述的是她多年来的性困惑,其实她不是无高潮,手淫时完全可以达到阴蒂高潮,只不过常常觉得它还不够过瘾或不够强有力而已。女性性高潮确实是一个复杂的生理心理现象,但决不会像她担心的那样,有神经衰弱的人就不能承受。其实她的阴蒂高潮早已达到,而阴道高潮可能在手法上还有些问题而未能达到,比如她讲的总差那么一点点,可有些成功往往就在再坚持一下的努力之中。临近性高潮时可以通过绷紧全身肌肉、用力屏气的办法促进它的到来。因为高潮是肌肉高度紧张后的突然松弛,所以有意增加其紧张性是有好处的。这时由于前面已达到高潮,机体往往处于高潮后的超敏状态,一方面还有要求,另一方面机体却不能正常作出反应,所以会出现这种缺憾并给她们带来烦躁不安的感觉。如果经过一定的努力还无法达到,则可放弃,通过与伴侣搂抱和事后爱抚而逐渐恢复平静,最终渡过这次性经历,这样既不会令她很失望,也能很快平静下来。至于时间延长仍不能达到性高潮可能反映出以下几方面的问题:①药物作用,安定的镇静作用限制了性反应的发生;②过分焦虑和紧张,她的注意力全都放在能不能达到高潮上,而不是悉心体验性活动所带来的感觉和感受,当然不容易使性反应得以自发发展;③性交刺激有效性,这是多次讨论过的问题,简单的抽动的效果较差,何不把手淫刺激阴蒂配合在一起呢?④体质的原因,如果体质差,心理素质又不好,机体的性反应甚至其他功能也不会太健康。

“三从四德”及“人生莫作妇人身,百年苦乐由他人!”等陈词滥调充分反映了封建社会男女的不平等,女人一生的命运总是由男子决定,先是父亲,后是丈夫,再是儿子。性是男性对女性的榨取而女性只是男子的性工具,这样她们往往对性持消极态度。长期保守的性教育使不少人认为“好女孩”不应对性发生兴趣,如父母总把性视为丑恶、禁忌或令人畏惧的事,或在家里根本不触及这一话题,她们成年后便很难持有积极的性态度。作为女孩,父母尤其禁止她们辨别和触摸自己的身体隐私部位,一旦被大人发现她们这种举动,便会遭到惩罚;到了青春期,父母的这种控制就更加严格,这就逐渐使她们讨厌自己身体的新变化如阴毛的生长、乳房的发育、月经的来源。她们还必须随时警惕背上个坏名声。因此她们对生殖器解剖构造一无所知,不到万不得已更不得轻易触碰它,总认为它是又脏、又丑、又臭烘烘的,而这种自卑心理在男性则较少见到。女性的心理状况将明白无误地反映其所接受的教育和训练,可以说是从儿童期起形成的性观念的翻版。这种种因素均可导致性活动时的焦虑或其他消极感受并进一步引起性高潮障碍的发生。

有一些女孩在10岁以前有过与男孩进行性游戏的历史,或有过遭诱骗而受到性猥亵或性骚扰的经历。童年的这种创伤性经历会深深地印刻在孩子的脑子中,再加上父母当时会作出一些过激的反应,小题大做,这就更加使少不经事的孩子在传统性道德观念“万恶淫为首”、“失节臭一生”的沉重枷锁压迫下抬不起头来,如果社会偏见再给她们添加一些冷言冷语,他们就会更软弱、无主见、缺乏信心、孤独、万念俱灰、冷酷、对异性充满敌意或怨恨。这样,童年创伤的经历就很难抹去了。她们在成年后往往因为自我防御机制而产生性交恐惧等性障碍,这是为了摆脱内心冲突和对性欲的罪恶感所致。要想从痛苦的记忆中走出来,关键在于自己,不论外部环境如何,只要自己能振作起来,尽早忘记不幸往事,忘记那童年无知时发生的不论任何事情,注重现在的生活,多寄厚望于明天更美好的生活,那么她们将能战胜自己,拥有积极向上的心态和美好幸福的明天。要知道,“除非你自己想受到伤害,否则别人是伤害不了你的。”千万不要拿别人的错误来折磨自己。

性治疗学家常常把各种不同形式的焦虑当做是导致性功能障碍的原因,然而,焦虑的类型是不同的,研究人员和临床医生常常忽视对这些焦虑的区分。能够高度性唤起、但只有振荡器才能诱发性高潮的女性的焦虑与极难唤起或从未有过性高潮的女性的焦虑应该是不同的。如果某些患有性高潮障碍的女性对达到性唤起感到不安,那么全身唤起本身只会对她们产生消极影响,所以在接受性刺激时应充分放松,这样才能达到性唤起,而在面对唤起和高潮作出反应时又必须有足够的紧张。与性高潮障碍有关的主要焦虑形式有两种:一是在参加性活动时常常回忆起过去的性创伤经历,从而出现焦虑;二是操作焦虑,在性活动中不是专注于动情的感受和感觉,相反却过度关心和评价自己的性表现,于是成为旁观者,这种分心显然干扰了性活动。

抑郁本身及抗抑郁药都可能引起性高潮问题,抑郁常常伴有性欲低下,也可以间接引起性高潮障碍。

除了来自成长家庭和先前两性关系经历中的人际影响外,女性对性关系的满意程度与当前婚姻的和谐程度密切相关,而且对性高潮反应也有影响,如伴侣是否具有充分性兴奋的能力和能提供充分、有效的性刺激。性高潮频率与婚姻和谐美满呈正相关,但满意的婚姻关系并非女性获得高潮的必要条件。无论伴侣对她们好或不好,情境舒适或不舒适,事前爱抚充分或根本没有,女性往往也能够在不同的环境条件下作出性反应。婚姻中的非性部分存在问题,如人生乐在相知心,而彼此交流不够,缺乏共同兴趣,缺乏彼此间的信任,互不满意,都会造成性欲低下之类的性功能障碍。有些女性则为自己设置了很高的人生追求目标,把事业看得重于一切,成功是最主要的,而性变成可有可无的事。性和非性困难可以单独存在,也可以互为因果,不过它们一旦出现总是使问题加重或复杂化。角色期待也使她们感到无权追求性享受,她们想象中的女性化就是“答应”,是全身心的付出,是反高潮的,在内她们要服从父母、丈夫、子女、家庭,在外服从上司和工作。即使有任何不满,也不愿说出来,生怕显得自己要求过分或显是太轻浮,总之怕得罪所有人。结果,虽然有许多丈夫总是对妻子大打出手,但她们也不愿寻求外界帮助,生怕丢了家庭和自己的面子,这就更让无理的丈夫得逞。尽管如此,她们也很少能下决心离婚跳出火坑。

紧张而充满压力的工作环境、长时间的工作、家庭环境缺乏隐私、酗酒、嗜烟等均可导致性高潮障碍,而人们对此往往不够重视。

性技巧不是起决定性作用的因素,但其确实有助于给女性提供针对性强的、有效的性刺激。如果双方没有坦诚、深入、充分的性交流,他们又如何能了解对方的需求和感受并及时调节自己的行为方式?

心理冲突可能是性功能障碍的原因,但并非所有心理冲突均会导致性功能障碍。心理冲突也可能是对性功能表现的过分关切的结果而并非是性功能障碍的原因,二者虽然可能共存,但彼此并非存在必然的因果关系。性高潮障碍可能是更深层的心理困惑的信号。成长过程中的消极体验会使某些女性在成年生活中对男子产生种种畏惧,如乱伦史、强奸史、性交疼痛、非法流产的创伤、失恋、失身,因此她们不可能想象任何能带来乐趣的情境,她们很可能缺乏任何积极的性感受。具有性虐待史的女性主诉性高潮问题的比例会高于没有性虐待史的女性,她们还会主诉性反应性低下和性唤起能力减低,当然具有性高潮困难的人未必都有性虐待史。无高潮女性幼年的爱恋对象(特别是父亲),以及后来的爱恋对象都不可信赖。这些女性既需要增强对涉及高度唤起的情境的控制,也需要增强丧失控制的潜能,如一位34岁女性早年遭受父亲抛弃,后来的初次性交伴侣又抛弃了她,之后虽遇到钟爱的男子却患有严重性高潮问题。正因为对性活动存在偏见和错误认识,有些女性会对性高潮的剧烈程度感到害怕并引起无意识心理冲突,因为她们害怕男方不能接受这些生理反应并因此而抛弃她们,所以她们要努力控制性反应的进程,而过分控制的防卫机制也就成了造成性高潮障碍的关键因素。

男方若射精太快,则自然减少了女性达到性满足的机会。但主流文化往往对男子的性表现存在过分要求,认为男子天生就对性问题无师自通、始终占据主导地位、理所当然地能给女性带来性满足,这显然让男子背上了灾难性的精神包袱和心理负担。如早泄男子一旦对自己的性功能健全产生怀疑,他必然忧虑自己的射精控制和妻子的性满意问题,从而设法转移注意力来延长时间,殊不知这种钝化和削弱肉体刺激的办法不仅使双方的亲昵和情感交流受到影响,也无形中伤害到其勃起能力。

[无高潮案例2]

“我和丈夫都是银行干部,结婚近9年,生有一位聪明可爱的女儿,自认为夫妻感情基础尚好。自从我看到您在《健与美》杂志发表的《女性性高潮障碍纵横谈》一文后,便鼓起勇气将这困扰我9年之久的难以启齿的痛苦向您倾诉。丈夫身高1.79m,体格强壮,篮球、足球都玩,又没什么病,不知为何阴茎总是不那么坚硬,插入阴道后我都感觉不到有快感。而且常常是阴茎刚进入阴道,不等抽动就射精了,为此事我们总闹别扭,渐渐地他不再主动要求了,我也觉得厌恶了,有时一个月同房一次也是刚插入就射精。丈夫偶尔也能持久一会儿,可是我却很紧张,很担心,一动不敢动,只怕自己稍一动,丈夫就控制不住了。结婚9年来,性生活一直很不和谐,我从未体会到性高潮,这也严重影响了我们的夫妻感情,为此,我感到很苦恼。

我俩都是大专毕业,书读了不少,心理调适、药物治疗、性交体位的变换都尝试过,毫无起色。我们都是30多岁的人,真的能不过性生活吗?有时我想干脆离婚算了,可是想想女儿都上学了,为了孩子,只好忍一忍,这么多年不也就过来了吗?况且丈夫除了这一点以外,还算是个好丈夫,我舍不得把这个自己熬了这么多年才建起来的家给毁了。但有时想想,凭自己的身材、长相和过人的聪明才智要找一个情人也不是件难事,可我又为什么要干那种偷偷摸摸的事呢?我很苦恼,曾多次主动找丈夫谈怎样协调性生活,可他就是不重视,要么简单、粗暴地拒绝我,要么应付式地玩笑两句。我很痛苦,有一种被抛弃、被冷落的感觉。为此,我不知躲起来哭过多少回,有几回被他看见了,他生气地说:“哭什么哭,大不了不要就是了”。说完就自己睡觉去了。我娘家是外地的,本地没有亲人,这种事,叫我向谁去倾诉?希望得到您的指点,帮我脱离苦海。”

在性反应之中妻子不能进入性角色的原因在于丈夫只顾自己而不顾及女方的需求和满足,所以说没有性冷淡的妻子,只有不合格、不称职的丈夫。如果丈夫总是坚持这种态度并简单、粗暴地拒绝交流,也不积极主动地配合咨询与治疗,那么离婚姻破裂的日子就不会太远。问题是许多男子的表现与其所受教育的程度不成比例,甚至达到令人难以置信地步,可见学校教育不仅要教知识、教文化、教科学技术,更要教修养、教品德、教如何做人和待人。性教育就是爱的教育,而爱又是丰富多彩的。我们已习惯于“与人斗,其乐无穷”,什么时候能转变为“与人爱,其乐无穷”,那才能说明我们社会的文明程度具有相当水平。如果世界上充满爱,那还有什么困难不能克服呢?类似这对夫妻的问题离不开对丈夫进行爱的再教育,当然也不要忽视女性的自我觉醒问题。

总之,至今尚没有发现哪些经过大量研究一致认定的共同因素,而成为划分有高潮或无高潮女性的标准。


\section{第四节 女性性高潮障碍临床评价}

当面对无高潮主诉时,性治疗师该如何认真倾听和思考并把上面讨论过的理论和实践经验结合起来运用于临床评价呢?虽然没有现成的公式可供套用,但在评价之时还是需要重视一些重要因素。

患者在第一次就诊时的前15~30分钟向医生讲述的内容都是发人深省的,并且这些内容将提供有关高潮缺乏的有用的、初步的线索,也能够看出他们是如何看待这一性困难的。因为其内容和背景都是重要的,如“我不喜欢性生活”、“我发现性对我是一种羞辱”、“我妻子对性毫无兴趣”、“我丈夫认为我应该有高潮”,他们对性问题的起源、维持以及与他们生活其他部分的联系的陈述都是非常重要的。

患者在表述其性问题时的情感变化也是重要的:如悲伤、自我贬低、愤怒、易激动、焦虑和抑郁。对原发性性高潮障碍女性来说,明显缺乏这样的情感变化或超然是很常见的。还要注意观察他们双方关系中的情感气氛是紧张、挫折感、彼此界限划分明显还是双方地位的高低?伴侣的非语言表达诸如姿势、癖好、眼光接触、语言及非语言的反驳、总体能量水平等也可能显示出未暴露的冲突和畏惧。

性治疗师首先需要备有一个严谨的、详细的性问题问卷来描述和记录性欲、唤起、高潮、生殖器痛、性满意度的水平和频率,以及所存在问题带来的痛苦。对于有性唤起而无性高潮的女性的治疗干预显然不同于那些连性唤起都没有的女性。作为一个自然的开场白,性治疗师要了解更多的历史资料,所以应询问性症状是境遇性的或完全性的?一直如此还是最近才发生的?

认知—行为和系统治疗主要关注当前的因素和冲突,他们认为双方过去的经历往往不会继续影响患者。不过应该了解患者父母对性的态度及患者对父母关系的知觉程度,一个直接而又无伤害的问题是“你从父母的两性关系中学到哪些有关男女如何相处的技巧?”,而这些影响对患者的身心健康发展可能是很重要的。

性治疗师必须评价性高潮问题是否继发于器质性的或药物的因素。抗精神病药物会损伤性高潮功能。所有类型的抗抑郁药都有可能影响女性性功能的行使。SSRIs造成女性高潮的延迟或性欲的改变的程度是不同的,有的影响很小,有的可致80\%的服药者出现更大的影响。5-羟色胺能受体拮抗剂(如赛庚啶)可以用来对抗SSRIs抗抑郁药的抑制作用。治疗严重强迫症的氯米帕明(氯丙咪嗪)也是一个α肾上腺素能阻滞剂,能阻止性高潮所需的交感神经系统的激活。降低氯米帕明(氯丙咪嗪)的剂量或用地昔帕明(盐酸去甲丙咪嗪)取而代之可以恢复患者的高潮能力。应该检查一下现代的处方药和非处方药物,以及身心调剂药和乙醇等是否同时具有对性功能的相应影响。

其他医学处置也可能影响性高潮反应,背痛问题或神经损伤有时可使女性出现与损伤相关的改变,她们能够耐受长时间的强有力的振荡器刺激,或者缺乏生殖器区域的任何感觉。多发性硬化患者早期就会出现性功能障碍表现,同时还会伴有直肠或膀胱问题,此外具有外周神经损伤的糖尿病女性也会存在性问题。腹部手术切断血管丛和切除淋巴结时可能导致植物神经损伤而影响性高潮,如接受子宫切除手术(不论是否保留卵巢)的女性中有33\%~46\%会丧失性高潮能力,但这些研究并无对照组。有报道指出子宫切除时若保留宫颈可以防止性交疼痛并保存性高潮能力。

性高潮障碍女性现有症状的背后可能隐藏着未作诊断的临床的或亚临床的抑郁症,特别是当她们伴有性欲低下时。季节性抑郁也需要加以诊断,当然这些女性很少会讲述自己的性问题。

对于治疗干预的决定和如何摆脱治疗中的困境,一个关键的任务是了解性高潮对患者而言意味着什么?女性的答案各有不同:“我将觉得很能干”,“我会变得易受诱惑”,“我们关系会更密切”,“太性感了”等。这些含义通常是有冲突的,渴望和畏惧,寻求和回避。一些女性在解决了高潮问题并使夫妻关系更加亲密之后,可能会默默羡慕那些存在高潮困难的夫妻,因为那些夫妻要不断付出努力以寻求进一步的亲近,而自己则无事可做。个别女性甚至这样说:“高潮伴随着丧失、依赖甚至死亡”。

性治疗师的作用是认真倾听患者所言,不要随意打断患者的叙述并在没有掌握完整信息的情况下轻率下结论,不能为了有助于构建或重新构建解释性问题存在的原因,从而起到某种有价值的作用而主观地提醒、诱导、删节、粉饰或强调什么。如果患者的观点是把性高潮看做是失控,而她的家庭背景又充满暴力和酗酒问题,那么这些因素对于无高潮的发生显然有负面影响,而治疗目标也就转变成能否纠正把性高潮看成是失去控制的观念。

当性高潮问题能够确诊时,初期治疗安排要考虑的问题不是个人的而是夫妻双方的。如果双方都能积极响应则无疑是很好的信号,他们可以开诚布公地去解决问题;如果伴侣之一存在某种创伤性历史时,如儿童或成年后的性虐待史,这种情况下,性治疗师将面临进退两难的处境。是否应该单独地会见伴侣双方?单独会见有利于获得更多信息;不单独会见,那么有些性秘密会永远保密和压抑下去,这就使性治疗师不能了解全部真相,于是处于被动的不利状态。实践中还是应该有一次单独的会见,只不过要向双方讲清楚,单独会见的目的是要以另一种方式来了解更多的情况,这也是对双方的一种保护措施。

从最开始就需要选择和确立一个治疗方案:究竟是单独治疗、夫妻治疗还是小组治疗?而哪种方法会更为有效?但单凭治疗结果来看,很难判断当时的决定是否恰当。如果这位女性有固定伴侣而且目前也没有婚外恋,那么夫妻治疗对她最合适,特别是当她也渴望和这位伴侣共享性高潮的到来。如果这位女性只有暂时的性伴侣或处在无感情的婚姻之中或有性虐待的两性关系史,那么单独治疗可能更合适,起码在开始治疗时应该这样安排。如果女性没有性伴侣,那么安排只有女性参与的小组治疗是更合适的。


\section{第五节 女性性高潮障碍治疗方法:理论探讨}

本节主要介绍三种针对女性性高潮障碍主要治疗理论的突出特征:心理分析、认知—行为及系统治疗。应该指出只有认知—行为方法具有实质性研究成果,其他方法只是一种概念和阐释。

传统上,人们把性功能障碍看做是一种表现人格发展病理过程的症候群,认为是由于阉割焦虑幻想、对恋父情结的内疚、无意识畏惧等导致的人格发展停滞。现代观念已不再把所谓性心理成熟作为治疗目标(即由儿时阴蒂高潮向成年阴道高潮的转变),注意力已由对性唤起生理原因的争论转变为阴道高潮和阴蒂高潮心理体验的差异上。从本质上讲,女性在性交中经历高潮和乐趣的能力,与女性和其他人建立亲昵关系的能力相关。

从客体—关系理论的视角,某个体与其他人相处的能力是从出生时就具备的,是一种内在的或本能的能力,这种能力部分地取决于如何内化他人正确形象的能力。在儿童成长过程中,与看护人有关的无论是消极的还是积极的印象都将内化于儿童的内心,这些将左右其在以后生活中形成与他人亲昵关系的能力。一个人能否维持对另一个人的兴趣和亲昵,将取决于他能否耐受并承认他或她所爱之人的缺点这种矛盾现象。有关女性心理发育的较新模式强调,她与母亲的关系将明显影响她成年后在异性恋中的表现。一位女性对两性关系的信赖程度是她对自己基本素质自我界定的一个正常组成部分,而不是人们通常所说的病态的依赖综合征。

作为成年性功能行使好坏标志的、对两性关系信赖的自我界定不足以使亲昵关系受到威胁,因为它们调节一种与母亲融合的感受,这种感受与获得语言能力前的、母婴之间无明显分化的体验是相似的。性亲昵关系可能重新产生了对拥有与其他人情感交流需求的需要,先前是与母亲的,目前则是与性伴侣的。对分离的内在感觉的清楚界定是必要的,它能使一个人忍受亲昵关系;在亲近时的冲突可能导致敌意、愤怒、缺乏信任、高潮抑制。一位女性需要在性中感受到足够的安全感,才能接受他人的肉体亲近并感受到乐趣,而且不会畏惧与伴侣的融合或放纵自己去体验性的感受。

无意识处理焦虑的手段即防卫中的若干种,是与性高潮障碍相关的。①否认,可能表达为最轻微程度的躯体感觉。②压抑,从意识中消除一种经历过的感受,允许个体在痛苦得不能忍受的情况下能继续生活下去。③创伤,例如儿童期遭遇的肉体或性的虐待,可能是某些形式压抑的先兆。④投射自居作用,无意识地把一个人的感受投射到有意义的其他人身上,是处理不能接受的感受或冲动的另一种方式。例如,一位对自己想要控制他人的欲望感到不舒服的无高潮女性,可以以她丈夫对她实施控制时的方式来对待她的丈夫,并按这种方式处理双方的相互关系。在心理分析学派的心理治疗中,强调的不是症状的去除,而是力求解决导致其症状的各种冲突。有必要探究症状的象征性内容和功能效用。其他常见特征包括检查和改正在相关关系中的儿童早期经历的记忆,解释和解决在治疗中变化的阻抗,关心、治疗关系中的移情和反移情问题。治疗集中于患者和治疗学家的二元关系上,要使用比其他方法更频繁、更多的治疗单元。传统心理分析治疗已经发生变化,把个别治疗与夫妻治疗相结合,而且还结合了另外的性治疗。然而,性治疗还是引人注目的事,因为它需要在治疗学家的中立性和患者为性功能恢复的投入之间作出选择。心理分析治疗结果的资料是很少见到的,主要限于临床报告。一般认为心理动力学治疗与症状消除对性功能的长期恢复是有作用的。

认知—行为方法是治疗学家通过学习和认知过程理论来帮助解决性高潮问题的缘由。伴随性经历的焦虑可能干扰放松,阻止性唤起和抑制性高潮反应。相似的是,在性念头和行为与积极感受之间若缺乏联系,能够导致女性回避那些有可能引起性唤起和性高潮的行为。在无高潮的临床样本中会伴有这样的相关关系,如自我体象感差,总感觉与她人不同,感觉到不能自如地交流性欲望。后者可能是因为缺乏自信或交流技巧有限。认知—行为治疗的目的是促进认知改变、态度转换、减少焦虑、增加高潮频率、增强积极感受和性行为之间的联系。已经证明不会影响性和婚姻结果的治疗因素包括:性治疗学家的性别、一位或同时有男女两性治疗家队伍、治疗单元之间的间隔时间。认知—行为治疗的特征是安排私下进行的行为训练,根据患者汇报训练的结果来制定新的训练内容以符合患者的需要。总的来说治疗宜简短,平均15~20个治疗单元,情况复杂时当然应花费更多时间。

指导下的手淫是治疗原发性性高潮障碍女性最常使用的技巧。治疗程序包括一段时间的教育和学习,随之安排视觉的和动态感觉的对自己身体的自我探究。指导下的手淫在多种治疗模式中都取得成功:包括每月一次共4次的最少的治疗单元安排,包括观看影片和阅读手册,小组治疗、个体治疗或夫妻治疗。大多数研究表明80\%的女性可以通过指导下的手淫获得性高潮,但能在与伴侣一起时达到性高潮的比例较低,占20\%~60\%。大多数女性主诉乐趣增加后对性交活动的满意度增加,对性生活能有更放松的态度,对自己身体的接受程度也增加。小组治疗帮助每位女性对自己条件的知觉正常化,它强调的是共享、提高自信、自我触摸和手淫,以及教育、松弛、阅读和处方的性活动。小组模式使用指导下手淫的效果不尽如人意,因为它强调自体性行为和反应,也因为这些体验未必能转化并融进伴侣关系活动中。在治疗之后能实现这种转化的比例也可能与年龄有关,比如35岁以下可能有80\%的人能转化,而年龄较大的女性则只有约60\%的人能够转化。总的治疗成功率可能有较大差别,治疗后的性交高潮率也没有明显的改善。这时,需要对伴侣进行教育,对许多女性来讲在性交时要想经历高潮,需要额外性刺激是很正常的事。

另一个矛盾是耻骨尾骨肌锻炼在产生性高潮中的有效性。国外有的调查发现虽然经过耻骨尾骨肌锻炼后女性们的肌张力提高了,但她们的性高潮频率与其他未经训练者并没有差别。绷紧阴道肌肉群对女性性唤起的作用不如性幻想的作用强,这一比较包括客观的生理测定结果和主观评定,然而,绷紧加幻想的作用是最强的。耻骨尾骨肌锻炼可能具有促进性唤起反应的价值,而且也可能以此增强女性对自身生殖器的清醒认识和舒适感而加速性高潮的到来。不过根据我们自己的经验,有些女性能够在性交过程中通过有意识地收缩耻骨尾骨肌而加速性高潮的到来,也有些女性能够单纯通过自己收缩耻骨尾骨肌而达到性高潮,如果把一个类似阴茎的物件事先插入阴道,然后再来收缩耻骨尾骨肌那就更容易达到性高潮,因为耻骨尾骨肌更容易用上劲儿,在这些过程中若伴有性幻想则效果更佳。插入阴道的物件,她们采用的包括有仿真器具、火腿肠加避孕套等,有些物件颇富创造性。

无论是原发性或继发性性高潮障碍,治疗计划都是非常重要的。指导下的手淫治疗对原发性的治疗是最有效的,成功率在80\%~90\%,继发性的治疗成功率在10\%~75\%之间。年轻的、心理健康的、婚姻幸福的女性更有可能治疗成功。在大多数研究中,虽然性和两性关系的满意度增加了,但现存性症状并没有显著改善。

一般的系统理论主张采用一种模式来考虑多因素现象,如医学实践主要运用于那些身心疾患者,有人认为马斯特斯和约翰逊的工作恰恰是采用了系统理论方法的视角。尽管他们把性功能障碍看做是“关系”出了问题并利用夫妻配合来治疗性问题,其实性功能障碍的概念仍是集中在个人独特的致病因素上的。

当婚姻治疗师因为自身观念的保守而回避性功能障碍时,性治疗师也很典型地忽视婚姻家庭的问题,之所以出现这种局面与这两种学派均比较年轻有很大的关系。婚姻治疗刚出现时太在意对心理分析理论的反抗,所以把性仅仅看做是一种症状,主要处理成人或已付诸性行为的青少年与生殖有关的性行为后果。性治疗师出于要肯定20世纪60年代文化和生育自由,很固执地把目标对准“性的目的是乐趣”而不是“生殖”,结果导致它们与婚姻治疗总是格格不入。其实性治疗与婚姻治疗应该整合在一起去对付性功能障碍,所以一定要使用系统的概念思考并要回避单纯处理“症状”的方法。

系统概念是很重要的:家庭存在于一个社会文化背景之中,每个家庭构成一个独立的系统。每个人既是家庭之内亚系统的一个组成部分(比如中年夫妻与子女构成两个亚系统),他或她也是一个独立的系统,其生存系统是具有安全的疆界的,不能轻易越界。不过,系统也是开放的,因为每个人的社会和家庭角色总会发生一些变化。从个人系统角度看:应该在个人系统内保持身心稳定性的自我调整、随情境改变而更好适应的能力、关注一种因素的改变对其他因素的连带影响。在夫妻系统里则具有其特别的特性,既包含又超越了夫妻各自带进系统的特性。具有性问题的伴侣要在夫妻系统内扭转“彼此对抗”的局面,淡化“有病一方”与“无病一方”的矛盾,以便建立起平等的和谐关系。例如,我们可能对女方讲:“当你处理与同事、老板的关系时挺自信、游刃有余的,而在丈夫面前却那么胆怯、不能自立”;而我们对男方则讲:“其他人觉得你很容易接近、宽宏大量,而你却总是和妻子斤斤计较”。随着治疗不断进展的启发,夫妻将更加看重彼此的关系,而不再是只关注自己。

从伴侣角度看,在性活动时,两个人之间会发生在不同水平上的相互作用的交流。

第一个水平是象征性相互作用:比如双方对两性关系、性行为和生儿育女的象征意义究竟如何认识,他们应该具有相似的、共享的理解和认识。一般认为两个人如果来自相似的文化背景时,彼此才容易拥有较多的共同语言。又如最基本的象征性问题是性高潮对双方的意义是什么?女方可能说:“如果我有性高潮,我们的性关系就会很完美,”而男方则可能说:“如果她有性高潮,我们的婚姻会更完美。”这些信息将有助于在象征性相互作用水平上安排可能有效的治疗方案。

第二个水平是自觉情感调节作用:它在两性关系中占支配地位,自觉情感调节作用可以分为4种不同类型,它们将引导夫妻是趋向或远离性的背景:

(1)(这些数码合适否?能独断成断吗)依恋:即在建立、保持和强化彼此亲密关系时的亲和力;

(2)探索:即通过感官接触寻求亲近;

(3)防卫:即给自己划定身心的安全疆界。当女性认为性高潮对自己是一种侵犯时,被占有的感觉就会压倒肉体快感,尤其是对于有性虐待史的女性来说,当对方触摸自己身体(特别是生殖器)时会表现得毫无感觉。问题的关键是身体安全疆界干扰了性高潮的发生。对于那些过度防卫的女性,可以把手淫作为她们的治疗方案;

(4)家庭地位:夫妻都想获取和保卫自己在家庭内的支配地位,家庭地位之争如何体现在性问题上呢?女方会说:“我之所以迫切希望达到高潮,是怕伤害丈夫的自尊心”,男方则说:“我要尽力让她高兴,这样她才能兴奋起来。”谁都想控制主动权。在进行性感集中训练时,性治疗师应鼓励丈夫按他所喜欢的方式去做,而当女方感到有压力时,则让女方主动并按照让她感到舒适的方式去做,这样,就可以避免家庭地位之争对治疗训练的影响,就能够通过依恋和探索的相互作用帮助夫妻体验并实现性乐化。所以,自觉情感调节模式既可以是性相互作用的入场券也可以对其有削弱和损害性。

第三个水平的相互作用是感觉交换,是关于每个伴侣受对方影响得出的感觉模式、神经生理反应、运动神经反射。夫妻常常感到难于促进性唤起和难于通过性活动时的交流得到她们所向往的有效性刺激,总觉得彼此不够协调,所以要特别注意协调感官刺激。例如,当性活动开始时女方过度兴奋或紧张并且紧紧收缩耻骨尾骨肌,那么这种夹持作用必然给男方带来过分强烈的刺激,男方自然一泄如注,女方则出现被抛弃的感觉。也可能是男方在事前爱抚调动女方情绪时,使自己过度兴奋,从而难以控制自己的射精时间。男方若使用降温理疗袋或凉毛巾给外生殖器局部降温,则有助于控制自己的过快兴奋。

象征性相互作用、自觉情感调节作用和感觉交换这三个水平的交互作用可以分别发挥作用,但它们也需要一些类型的相互作用以填补个体之间的需求。其实,对于每个个体来说,它们彼此间又是相互的影响。采用系统治疗方法处理性功能问题的文章并不多,也不能因此而估计治疗时间的长度和成功率。一般来说,性治疗师若从这一视角出发,而希望治疗时间尽可能短一些的话,就要强调治疗干预的综合运用,治疗期也将控制在数月左右。

性高潮障碍的其他治疗方法也具有一定的成功率。正在研究开发的药物无非这么两大类:一类是通过增加血流而改善性唤起,这与男性用药的原则是一致的;另一类是增强性欲的药物,其中的许多药物是基于男性性激素的。它们采用的治疗方法包括雌激素替代药7-甲异炔诺酮(利维爱,欧加农)、雌激素与雄激素复合制剂(estratest,solvay)、阿扑吗啡(pentech)以及西地那非(万艾可)等。

关于利维爱问题,它是一个创新的、有组织特异性的雌激素替代物,能有效缓解阴道干燥和性交疼痛,消除或改善泌尿生殖系统症状,但又不刺激子宫内膜也不引起乳房胀痛,而且服药后女性的健康水平、体质、性欲和性生活质量明显增强或改善。服药后可观察到性刺激引起的阴道光学体积扫描改变有明显增强。一些调查表明利维爱对女性主观性唤起、性反应及性感受的各个方面都有改善,这可能是其具有的弱雄激素活性直接增强性欲的作用,在这一点上它优于传统的雌激素制剂,故有人把它称为女性伟哥也不是没有道理的。人们尚未能证实安非他酮(盐酸丁氨苯丙酮)药物治疗是否能有效增进特定的性反应,补充睾酮也是如此。卡普兰等(1999)的无对照研究表明绝经后女性服用西地那非后对性高潮或性满意并无作用。而性教育和书籍疗法显示有积极治疗效果,催眠疗法也显示出一定疗效。

总而言之,自从2003年国际性医学咨询会议直至2010年以来,有关女性高潮障碍问题并没有涌现出任何有意义的新资料,唯一的例外是近年在美国7个治疗中心征募了98位高选择性妇女样本即SSRI引起的女性高潮障碍妇女,进行了随机对照(RCT)的50~100mg的西地那非临床试验。治疗组的高潮功能得到了显著的改善。然而,这种高度特别的入选标准,也引起该发现是否能普遍推广的问题。


\section{第六节 女性性高潮障碍治疗实践}

在海曼和梅斯顿1997年的综述中,她们得出结论说对于从来没有经历过性高潮的有高潮障碍的妇女,只有有指导的手淫治疗才能符合“得到确认”的治疗标准。有指导的手淫研究对于获得性的障碍也可能有效。这一结论至今仍然有效。有指导的手淫治疗加上性教育、焦虑缓解技术和认知行为治疗是最主要的治疗手段。至今没有发现针对高潮障碍者有效的药物治疗方法。

当女士主诉从未体验过高潮时,性治疗医生的首要任务是获得有关她性活动的详细信息以确定她是真正的性高潮困难,还是仅仅因为她从未得到足够的刺激而导致的困难?事实上,很多女性之所以从未体验过性高潮,是因为她们从未得到过足够的、有效的性刺激。如果问题仅是如此,那么治疗将十分简单,预后也会令人感到十分满意。治疗包括努力解除罪恶感等偏见、纠正对性的错误认识、增进夫妻间的性交流以及努力学习并熟练掌握各种性技巧如阴蒂和G点等性敏感区对于女性性高潮的重要性。锻炼耻骨尾骨肌等有助于提高女性性反应,这些措施将很快带来明显改善和积极效果。

如果了解到她曾尝试过手淫但毫无作用,而且她的丈夫也曾在性交前专注地刺激她的阴蒂但始终没能奏效,这种情况下,在她已经得到了充分地、有效地性刺激后还是不能高潮,我们称之为原发性性高潮障碍。医生在处理原发性性高潮障碍时,治疗将变得更为棘手。

我们治疗这种患者的第一个目标是帮助她们获得初次性高潮,这是治疗极其重要的一步,因为首次性高潮的经历将驱散女性对于她不能获得高潮反应的忧虑。而且,一旦她有过这样一次高潮,她就向消除她的约束抑制迈出了成功的第一步。关于治疗有着这样一个至关重要的前提:患者的高潮反射只不过一直受到抑制,它没有被摧毁,如果提供的刺激强烈到足以克服抑制,那么还是可以获得高潮的。为了达到这一目的,必须作出各种努力来削弱抑制力量,同时以最大的强度进行刺激。鼓励患者努力去满足自己的性欲,建立起女性有权利、也有能力享受性乐趣的信念。

在简短的治疗阶段,要探究和证实患者受抑制的深远根源是什么,调整患者的心理状态使其高潮反应不再受到伤害。由于患者往往意识不到这些问题,因此在评价的最初过程中,它们可能不会暴露出来。对于患者来说,她们用来抑制其高潮反应的手段和自我防卫手段,可能只会在训练过程中表现出来,只要她通过手淫或阅读色情文学而唤起一定程度的性欲时,她的大脑就开始恍惚或者就感到困倦。如果不和治疗医生讨论,患者可能不会将焦虑促成的自我防卫和她的高潮抑制必然地联系起来。治疗医生常常需要主动进行干预,具有这样的领悟,使患者自己意识到有哪些因素可以诱发这些抑制,这是治疗中重要的一步。治疗医生要利用这种在领悟过程中造成的暂时痛苦,让患者正视现实,即使她由于各种各样的原因在压抑自己。

治疗医生要打破患者为抵御她的性感觉而在深层和表层水平上建立起来的防卫。医生会得出一些关于这些防卫的潜在原因的推论,他还将提出专门的行为建议来抵制这些防卫所造成的破坏性结果。例如,一些患者只有在高度分散其“抑制”倾向的情况下,譬如在受到刺激的同时,还得醉心于阅读色情读物,才能达到她们的初次性高潮。

治疗手段可分述如下:

性治疗学家指出,解决原发性女性性高潮障碍的最有成效的办法就是鼓励她们以手淫的方式来获得自己的首次高潮,研究表明,大多数患者都能通过刺激阴蒂而达到这一目的。但这一治疗方案常常引起她们的焦虑,因为从儿童时代起接受的消极性教育一直把手淫视为洪水猛兽,并且把它看做是危险和可耻的。所以,在治疗的最初阶段患者必须彻底改变这种错误地认识,打消一切顾虑,消除各种抑制因素。第一次手淫的尝试最好是自己私下进行,应该在不受外界任何干扰的时间和地点、在完全放松、毫无顾虑的情况下,采用患者自己喜欢的手淫方式来体会性高潮所带来的快感,在手淫的时候最好伴有一定的、积极的性幻想。虽然这与一般性治疗时要求伴侣双方共同合作的原则相悖,但确实是十分必要的举措。当然,她们往往也不愿意伴侣在场,否则会让她们分心、担心不成功、受到压抑或引起对方反感。

患者在治疗的初期阶段往往能够达到一定水平的性兴奋,但是很快便感到紧张和不适,于是停止刺激。在这种情况下,性治疗师就要指导患者与这些不舒服的感觉“并驾齐驱”,并告知她们要逐步学会驾驭和跨越这种感受。要达到高潮,首先就要允许自己把性能量彻底释放出来,以减少抑制高潮释放的焦虑。所以,在这一时刻决不要停止,而是继续原有的刺激。此外,还应指导患者在达到高水平的性兴奋时,通过收缩其腹部和会阴部的肌肉,甚至绷紧全身的肌肉,可以促进高潮的释放,这大概是因为肌紧张是性反应主要成分之一,并且它具有分散注意力的作用。

如果患者选择以手淫的方式作为治疗女性无高潮的突破口,那就应该告诉她们具体该怎么做,如果不进行技术上的辅助,她们很可能不懂得应该如何进行。还应强调手淫达到的高潮与性交或口交时达到的高潮同样正常、同样女性化、同样刺激。无高潮女性可能发现在她最初尝试以手淫来达到性高潮时需要很长的时间,甚至超过1小时。手淫可能令人十分疲劳,尤其是手臂,这时使用振荡器将是一个很有效的手段,它将更快更富于快感地帮助女性达到高潮。

当她们取得完全成功、又经过双方充分地交流达到认识上的统一和共识之后,女方就可以邀请男方参加进来。而男方要具有足够的敏感性,知道如何在她的建议和指导下来刺激她的会阴和阴蒂区域,所以女方仍是处于控制的地位。男方先是旁观妻子自己手淫,然后男方给女方手淫。这样就能使双方都感到能为共同的性关系作出有意义的贡献,并从中获得应有的乐趣。双方必须彼此充分信任,具有足够的勇气和信心,能够积极交流,因为亲昵感受只有在十分乐意的情况下才能保持,它是很脆弱的,任何勉强、不情愿或拒绝都会干扰它的发生。有时男方会出现内疚感或感到很悲观,认为是自己的性无能才造成女方无高潮,这时应鼓励他并向他讲明男女性反应的差异,这是正常的和常见的现象。同时也要注意有些男子的大男子主义或封建守旧思想会作怪,认为妻子的要求太过分,太令他难堪,这就应让他懂得妻子的性要求是正当的、合理的,他应满腔热忱地帮助妻子获得性满足,这样他也可以从对方的满足中获得更多更强的乐趣。男方若想帮助妻子解决这一长久存在的困难,就必须具有极大的耐心,做好充分的心理准备。只要具有耐心,女性就可以通过以上办法达到性高潮。当女性通过刺激阴蒂而达到性高潮时,夫妻就可以进入“无需求”的性交活动,这将有助于寻求增强女性的性唤起的方法。之所以称后一技术无需求,是因为让女方充分放松时,她们将不会感到有“实施操作”的巨大压力,相反,她们能集中精力体验性活动带来的快感。在后期采用综合治疗的措施可使性高潮障碍的治愈率提高到85\%~90\%,而性欲抑制的治愈率则提高到60\%~75\%。

提到思想的认识,要先从我们对自慰(手淫)的看法来说起。所谓手淫,就是独自一人去爱抚自己的性器官,以便得到性的刺激和快感。从现代医学和心理学观点来看,手淫是在性冲动时自我发泄性欲的举动。狭义的说是指个人用手来抚摸刺激自己的外生殖器,使心理上得到满足,达到自慰的一种现象。广义的则泛指采用各种手段,来刺激生殖器官或其他性敏感部位而达到性快感和高潮的各种技巧。所以,福柯把自慰比喻做“不过是出于好奇心”的跟自己玩的游戏。

今天的我们,是否可以扔掉那些所谓的枷锁,重新来认识性的独舞(自慰),证实它存在的价值呢?我想,否定自慰也就意味着否定自我价值的体现。事实上,大多数人都是以自慰作为正式性交的演习,自慰其实就是一个人独处时,自己对自己的关爱与照顾。手淫可使性器官处于良好的状态,是对自己身体最自然的亲近,是一种很正常的行为,人们通过自慰来认识自己、了解自己的性敏感区域。在手淫过程中体验到的性快感,将成为影响人们获得成熟性经验的过程,并使过去形成的对生殖器的厌恶情感潜移默化地向积极方向转化。自慰对保持个人的身心健康非常重要。任何人都不可能随时有伴侣随侍在旁,而欲望则会随时出现。

手淫对儿童来讲是一种求知欲、自我认识的积极表现,由于偶然发现以手或物触碰性器官后带来的快感,使她(他)们对此产生浓厚的兴趣,以后就有意识地这样做了。有些小女孩从小就喜欢夹腿,在夹腿扭动的过程中,就能产生一种性冲动和快感,两腿发直就达到了高潮。儿童从2~3岁就会产生这种本能的生理反应。

如果你是个从小就爱夹腿的女孩,那么恭喜你了!你得到了比中大奖还要棒的东西,那就是经过长期这样的锻炼,你可能拥有了强健的PC肌。也就是说,你可能比其他女性更容易获得高潮,更容易达到射液的极致状态……从小没有夹腿经历和自慰的女孩,一般不懂得怎样收缩阴道。

手淫会伴随你走过童年、青春期、恋爱、婚姻生活和整个晚年。自我性爱是性的一种基本形式,它不仅仅是年轻人、准情人或伴侣死去的老人的事,它更是贯穿于我们每一个人一生的事。在人生的漫长历程中,我们将不断获得性的启发与性的乐趣,而手淫可以说是这种体验的第一站,是一个人性心理发育的标志,是性认知感受、自信心和安全感形成的基础。

手淫能够让你对自己的身体懂得更多,它的教育作用远远超过你看过的任何书本、录像和影片。自慰是一种自然的人类性行为,世上没有唯一的或者正确的享有性高潮的方式。我相信自慰把握着解除性控制与压抑的钥匙,特别是对于性冷淡、性高潮缺失的人群。自慰是一种途径,我们可以通过它来读懂自己的性反应从而让我们了解,怎样才能更好地得到富有生命力的、快乐的性!

每一次自我给予的性高潮,都是一次对自我的肯定。女人的身体,就像神奇的乐器,需要用爱抚摸,只有这样才能使更多的性敏感点越来越敏感,才能奏出美妙的爱的旋律。自慰,不仅能让自身愉悦,更能开发自己的性敏感点。当一个女人懂得了自慰,她就懂得了喜欢自己的性器官,享受自己的性高潮,而且对性可以胸有成竹。我们所谈论的是对性的一种尊重和对快乐的开发,是对上帝赋予我们身体的美好本能的一种怜惜。性技能跟任何别的技能一样,它不会自然地遗传,它必须要通过学习才能得到。

性爱在我们的心中不该是一件例行的公事,而应该把它看做是一件可以冒险、可以创新、可以学习、可以研究、可以改变、可以实验……乐趣无穷的事情。

因为我们对性有着错误的认知,所以有必要进行修正。快乐是无罪的,为什么要让快乐也套上无形的枷锁呢?国人就是太爱做这样的枷锁,所以不快乐的人才多!爱的世界里没有尊卑,只有平等的付出与给予,这样你和她才能快乐的进入伊甸园,享受上帝赋予人的快乐!性,一直都是纯洁神圣的,但却有许多人把它扭曲了……

如果经过系统的、有针对性的手淫刺激仍不能引起高潮,那么建议患者使用振荡器来达到这一目的。众所周知,振荡器能够不知疲倦地提供频率很高的剧烈刺激,所以它们肯定能胜任这个治疗任务,振荡器代表了性技术的一个重要进步。然而,有些患者十分担心她们今后可能会依赖振荡器,进而影响夫妻关系,乃至婚姻。为了防止女性可能沉溺于振荡器引发的高潮,我们只是在相当剧烈的阴蒂手法刺激仍无效时,才建议使用振荡器。此外,我们不鼓励只依赖于振荡器而产生高潮,相反,我们一直鼓励探索其他刺激手段。实际上,工具再好,也是冷冰冰的器具而已,它不可能取代两性关系中能够达到充分身心交融的性爱过程。

尽管振荡器是性刺激的强有力的来源,但是使用振荡器并不能确保高潮到来。有些女性第一次试用振荡器就很快、很容易地达到性高潮,但是也有少数女性在刺激几十分钟后也达不到首次高潮,她们在接近高潮时总感到紧张和不舒服,于是抑制她们自己“平静下来”,然后再重新刺激直到再次接近高潮,如此反反复复地也只是体验到一系列的兴奋期反应。当患者感到很难达到首次高潮时,性治疗师要积极鼓励和指导,让她懂得在接受剧烈的阴蒂刺激并濒临高潮时如何“释放”高潮,她必须学习如何使她的肌肉紧张起来、如何呼吸、如何性幻想,以便分散其习惯性的防御性抑制倾向,这种治疗策略能使大多数无高潮女性迅速达到高潮。

大人与小孩的不同,就在于玩具的不同,大人有自己的世界、自己的视角。那么对于成年人来说,最有趣的玩具是什么呢?我想,当然是成人玩具啦!

就像小女孩需要娃娃来当做玩具一样,成年的女性难道内心就不需要这样呵护自己的玩具了吗?其实,大人也是需要游戏的,就像是性的快乐游戏!因为好奇、因为种种不同的原因,我试用过不同类型的一些玩具,现就一些玩具所给我的感触,总结如下:

首先,我想每个想要购买玩具的成年女性都会有个梦想,希望它能替代爱人所有全部的能力,这,其实是个幻想,是不可能实现的。要是自己试出哪个玩具很可爱、很好用,并且可以替代男人的三个头,那我一定高兴地扔下男人,直奔西天的极乐世界!可惜,到现在为止,我还没有发现任何一个能让我有这样感觉的玩具,每件玩具都有各自的缺陷,很难有十全十美的!

所以对于我这样的老手来说,现在不会轻易地购买这些被描述得天花乱坠的玩具。我会凭着我的经验精挑细选,从中筛选出比较适合自己的玩具来玩。

有种振动乳峰夹,夹在乳头上挺刺激的,就是有点疼!乳头多敏感呀,哪能受得了那种酷刑!所以现在这个玩意也被淘汰了。夹子的部分不太好用,因为不是乳胶或柔软些的材料,而是金属质感硬硬凉凉的铁,用起来肯定影响感觉。紧了夹着特疼,如果掰一掰,虽然不那么紧了,可是一振动就会松了掉下来,又起不到刺激的效果。你说设计时,他们有没有考虑过这些问题呢?我想应该没有。都说顾客是上帝,但是,不知道在设计时他们有没有考虑过上帝的感受呢?上帝也是会疼的!

瑞典天使之翼振动棒(紫色),精致风情的外型设计,双震子、快速充电、多频多速震动的功能配备,表面丝质材质,手感很是不错。这款是我比较喜欢的,不仅因为其外形设计美观大方手感好,而且从质量上来说,比国内的同类品牌质量强很多。但它也有美中不足,在身体里面的时候会感觉到有点硬,不过不影响快乐的使用。我感觉因为震荡的频率很强,所以可以释放很大的能量。也就是说,当欲望很强烈的时候,可以使用这个玩具,它尤其可以释放G点的高潮。

但副作用是如果玩的频率太高,有可能近几天功能失灵,几天都不再眼馋“食物”了。所以我觉得这个产品不适宜天天用、常规用,那样可能会提高你的高潮阈值。一旦养成了习惯,就难于达到高潮,如果是那样的话玩具的作用就适得其反。所以,我认为这款玩具要控制用量。就像大餐,隔三差五的吃一次就好。

举个例子,同类的产品,国内的品牌和国外的品牌,从震动的频率、变换的档位上来说都是不能相提并论的。而且最关键的一个问题是,国内品牌的这类产品,一旦你开的时间久了,就会产生很大的热量。别小看这种隔热效果的处理差距,如果用在私处,那种差异性就会被放大,甚至产生致命性的伤害。

曾用过一个仿真器具,前面带个小振动器用来刺激阴蒂,但开得久了,前面那个小振动器就会慢慢地变得很烫。起初,还是身体感觉到了不舒服的热点才发现的。之后,每次用它都会避让开那个部位,我相信没有谁想烫坏自己娇嫩的肌肤,更何况是那个特殊神奇的部位。

(漏电元凶:导线的接口处存在很大隐患问题,导致接触不良还算是好的,会电你才是厉害的呢!)

也许因为用到的玩具不算很少,所以总能碰上怪异的状况。比如这个三头跳蛋,买来不到一天就坏掉了一个头,然后就改成双头震动;之后没过多久另一个头也坏掉,就改为单头震动;最后,没想到我所喜爱的这个玩具因为漏电电了我一下,才使我对国产玩具彻底的失去了信心。我的天!幸亏用的是电池,如果用的是交流电或电源充电的,那还不要了我的小命啦!本想快乐的放松心情,结果让漏电闹得心慌慌,只有把它打入冷宫才算踏实!真想说:哎,国产玩具,想爱你真的很难呀!

有的款式属于小型振荡器,随身携带很方便的一种。但总体感觉震动的频率不是很强也不够刺激强度,所以更适合初级使用者或者作为一种调情的小道具,应该是个不错的选择。

我不建议女性朋友去选择有些号称能模拟口交状况的玩具,其实介绍里说的都是假的,夸大其词,根本就没有那种所谓的功效。这种类型的玩具,很难做到可以替代两性的宠爱方式,也许是技术的问题,很带给你那种类似真空或被亲吻的感觉,如果你听信了花言巧语而买来,也只能是拿来当摆设或等着扔掉。许多有关介绍刺激G点的工具,未必有那些如何如何地功效,这些所谓的G点棒并不适合于自己的口味。我觉得还是看个人的喜好,最好先对自己有个全面的了解,再选择不同的玩具。最好选择自己觉得很顺手的玩具,这样也利于你从中找到快乐。

推荐女性朋友使用能增强你的爱肌力量的工具,用的时候虽然不能给你高潮的快乐,但锻炼还是很有用处的。它是比“凯格尔”运动更富有乐趣的产品。它是专门针对女性的阴道和骨盆肌肉而设计,可用作骨盆腔肌底运动,预防阴道松弛。但有一个缺点,就是球的直径有些大,可能不太适合亚洲女性。至少,我觉得有些大,放进去时不会很顺利。如果遇到润滑有问题的女性,可能还会害怕这个东西呢?所以,如果国产的厂家能够设计出一款接近这个的产品,直径稍微小一些也许销路会很好。它是利用自然活动而产生震动,两个球中均藏有滚球,当滚球滚动时会撞来撞去产生震动从而制造刺激。轻松实现持续锻炼,紧致私处部位肌肉。而且很有趣的是,它里面配有两个重量不等的滚球(粉红色滚球较轻28g,蓝红色滚球较重37g)。这个是我在用的时候才发现的,锻炼的时候感觉可以抓紧它,不同于那些凭空的锻炼既感觉不到也体会不到,这个你可以感觉到实实在在的掌握着一个东西,并且在不停地玩转它。

玩具的魔力,就在于使用者可以从中得到快乐。如果你快乐了,它就实现了它玩具的价值!每个人大概都需要这样一个磨合体会的过程,才能真正的找寻到一件适合自己的好玩的玩具!

“其实,男人本身就是女人的玩具,可惜多数男人认识不到这一点,总以为女人是男人的玩具。结果,惹得女人不能满意,甚至被女人所淘汰。要知道,今天的时代,是女人做主的时代。所以,要想永远成为某一女性的宠物,首先就要做好她的玩具。这是当今两性关系的绝对真理。”这段话出自一个男性之口,是他读了我这篇文章之后的感慨,让我想起了龙应台说过的一句话:“女人并不一定就有女性意识,男人不一定不是女权主义者,差别在头脑,而不在性器官。”很欣赏她的这句话,也很欣赏具有这样女权思维的男人。我觉得只有男人抛弃了男权的偏见,才能帮助女人,让女人达到极致的高潮。当然,男人从中同样可以得到很大的快乐,这些是我在思考的过程中感受到的。

这样的男人才是值得女人尊重的好男人,我很喜欢也很欣赏这样的男人。因为平等,因为女士优先的原则,让男女从性爱中得到了高于性爱快乐的东西。可惜许多男人不懂,连许多女人自己都不懂,你又怎么能要求她变成女权主义去追求自己的权利呢?这就是为什么国人的性那么可悲、那么可怜的原因了,希望我们伟大的祖国能多些这样思维的男性,也许世界会更美!

在某些病例中,各种有意识的和无意识的恐惧将构成潜在的或实际的高潮抑制,如一些女性担心,她们达到高潮时她们会耗竭或死去;一些女性把高潮看做是失去控制;另一些女性害怕一旦她们获得高潮,她们将过分迷恋和沉湎于性;还有些女性害怕她们体验到高潮后,丈夫会鄙视自己。问题是这些患者往往没有认识到这些抑制背后的无意识恐惧及其根源,所以治疗医生必须使患者正视事实,帮助她们理解这些恐惧的起源,患者必须认知即使她获得高潮,那些所谓的“灾难”实际上决不会发生。可以采用以上专门针对患者的无意识高潮恐惧的心理治疗作为补充治疗,打消患者对成功的焦虑。我们强调这一事实:高潮仅仅是一种反射。它并不具有受抑制的女性常常误认为它所具有的那种象征性的性质;同时,医生要对付构成这种象征主义基础的心理动力学力量。

一般来说,在患者达到她的第一次高潮之后,她逐渐要求在更短的时间内通过刺激产生以后的高潮。当她能够多次达到高潮并且毫不怀疑她达到高潮的能力之后,让她“戒掉”振荡器并且进入手指刺激。在某些病例中,这是突然做到的———告诉患者丢掉振荡器,重新开始进行手指刺激。在另一些病例中,最初指导患者使用振荡器达到一种高水平的性欲唤起,然后用她的手指来“结束”,这样她能够逐渐变得习惯于在不太强烈的刺激下(如果她愿意这样)达到高潮。

在刺激过程中使用分散注意力的方法,这在高潮障碍的治疗中起着极其重要的作用。这种方法普遍适用于释放任何受抑制的反射。例如,神经病学家知道,当一位患者的膝跳反射受抑制时,通过让他在受到刺激时把注意力集中在两手交叉紧握的十指上,分散他的注意力,就能引出他的膝跳反射。再有,当患者一边大便一边阅读时,他的注意力从排便行为上转移了,这时便秘常常得到缓解。当摆脱对自我注意的压抑控制时,直肠的膨胀会自动引起排便反射。同样,如果受抑制患者有意识地将其注意力集中在自己的性体验上,如果她是一位“高潮旁观者”,如果她没有全身心的投入反而对自己进行评判,那么她常常不能体验到高潮,尽管这时她受到很充分的强烈性刺激。因此,在性交或阴蒂刺激过程中让她集中精力作性幻想或作阴道肌肉的收缩或把注意力集中在性交抽动或呼吸或集中在她的性伴侣上,这都是非常有用的。卡普兰学派把这种治疗手段称之为“使分心者的注意力分散”。

性活动过程中伴随着性幻想是一种出色的分散注意力的方法,它也是克服高潮障碍的法宝。然而患者常常对她们的性幻想感到内疚,这就需要治疗医生鼓励她们在刺激过程中采用能够唤起其性欲的性幻想。同使用振荡器存在的危害一样,这些患者也可能变得习惯于性幻想并因此不再会真正体验性行为了。尽管如此,情爱幻想仍是理想的分散注意力的方法,因为它既能够分散注意力,同时又是一种刺激源。在正常的治疗过程中,患者对于性幻想的使用将随着她对性欲渴望的加强而逐渐减少;另外,值得注意的是,性幻想和振荡器不仅是正当治疗的关键,它们也可以用来克服患者性经验的局限并提升她的性快感。

正如前面提到的那样,高潮释放包括耻骨尾骨肌和阴道外肌肉的阵挛性挛缩。高潮反射可以由于肌肉过分松弛,或者由于骨盆肌肉痉挛性收缩而受到抑制。同样,高潮反射类似于膝反射:一位患者可以放松大腿肌肉,从而使进行检查的神经科医生灰心丧气,因为这些肌肉也包括在这种反射的表达之中。与此类似的是,在询问许多无高潮患者时,她们会说阴道肌肉在动情刺激过程中非常松弛。在这样的病例中,如果女性在有高潮来临的感觉时主动绷紧并且收缩她的阴道和腰腹部肌肉,那么在高度性兴奋时就能促进高潮反射的释放。在性交过程中,会阴部肌肉主动的、有节奏的收缩和放松对于促进性交高潮是十分重要的。许多女性并不是这样做的,在这一点上,她们需要专门的指导。有些女性可以通过自己有力地收缩局部肌肉而达到性高潮,甚至一次次地射液。

首先,要避免外伤。妊娠、分娩及房事过度,对PC肌都会有不同程度的损伤。在产后应加强锻炼,做产后保健操。其次,要防治便秘。大便秘结可对PC肌形成慢性刺激,阻碍PC肌的新陈代谢。另外,盆腔淤血综合征、盆腔炎等疾病也会削弱肌肉的功能,应加强预防,一旦发生要及时治疗。最有效的办法是进行提肛运动。

验证PC肌的功能有一个比较好的方法:上厕所时,在排尿过程中有意收缩肌肉阻止尿液排出(能终止排尿的这些肌肉交错在你生殖器官的底部),做类似憋尿动作时就是在主动收缩PC肌,假如PC肌强度和控制能力都很好的话,它们就能主动地中断及重新开始排尿。如果第一次尝试失败也不要灰心,学会反复地收缩—放松本身就是锻炼。整个练习的过程就这么简单,收紧肌肉阻止排尿,然后放松肌肉就可以重新排尿。想要增强这组肌肉群的话,那就收紧你的括约肌数三下,然后放松数三下。

因为PC肌的强度与女性在性交时达到性高潮的能力有直接关系,所以通过锻炼可以加强PC肌的强度,提高PC肌张弛的程度,并逐步达到随意控制的地步,这样就可以大大改善性生活的质量,这是改善性和谐的一条捷径。当练习持续6~8周时,不但阴道肌肉会呈现较为紧绷的状态,阴道的敏感度也会有所增强。另外,还应坚持加强腰部和腿部力量的训练,这对保持旺盛的性能力有很重要的作用。

一些女孩子从六七岁开始,就会不自觉地刺激自己的阴部获得快感,但大部分家长认为是不好的习惯,于是将其抹杀掉。爱夹腿的女孩,其阴道周围的肌肉群长期处于强烈地收缩状态,从而使得阴道入口附近的肌肉特别发达,同时可形成W阴道壁内摺纹。所以,从小有夹腿习惯的女孩,一般都会无师自通的收缩那里的肌群。我就是这样的,从小就会这些。加上后天从小就开始游泳,并且一直保持着游泳的习惯,所以可以将腹肌、PC肌群锻炼得很强大。

我的锻炼方法一

我凭自己的感觉总结出来,游泳是让我具有超强性能量的主要原因。常年的游泳让我的腹肌很发达,能自如的收缩,自如的进行肌肉的控制,不必在兴奋后才能迎来那种不自主的收缩。我认为,人是可以通过不断地练习从而提高对自己身体的掌控的!

在游泳过程中,在潜进水里之前,深吸气、收缩腹肌、提肛以及屏气收缩是锻炼PC肌的最好状态。基本上是一样的动作,一样的效果。而且,因为需要潜入水中很长时间,所以能增强肺活量的锻炼,这样也更利于性爱中的身体体能的需求。因此,我想到一个更便捷的锻炼PC肌和腹肌的方法,那就是在平时的状态下模拟潜水之前的状态,深吸气,然后屏住气息提肛数秒。在吸气的同时收缩腹肌,感觉到阴道内壁好像在贴紧皱缩般,越是能感觉到这样效果会越好。然后放松,交替进行数次。开始时可以坚持收缩3秒,之后放松3秒。无论是站立或是坐着,只要熟练后任何的姿势其实都是可以做这个练习的。

当锻炼的有信心之后,可以逐渐延长收缩时间到10秒。当然,放松时间也要相应延长到同样长的时间。之后,可以练习快速短促的运动PC肌,要尽可能快地反复收缩和放松该肌肉数分钟,开始时你可能很难分清是收缩还是放松,但逐渐就容易区别了。锻炼时可采用慢速收缩、快速收缩或两者交叉进行。

根据个人的身体状态来练习,不要一下子就加很大的量,要量力而行,慢慢加量。当然,坚持的时间越长效果越好。在初次尝试时可以短些时间,之后慢慢的延长时间。我想这个能增加整个腹肌、PC肌群的整体力量,属于大部分肌肉的协调训练。之后你也可以加上局部停止排尿的练习,这样双管齐下,既能掌握、区分各部位肌肉的力量,又能有协调全部肌群。我想这样的练习效果,比单做局部PC肌收缩见效更快,各个部位锻炼的效果也会更加准确。

我的锻炼方法二

当你的锻炼成果达到了一定的阶段,这时,你可以选择用玩具来参与练习。这样的效果会比之前的更好,当然你必须要完全熟练的掌握了上面的方法后,才能做这个“器械”的练习。这就好像我们平日练习手劲一样,手里攥着练习用的小圈状握力器,不断的松攥活动。只有这样才能更清楚那种感觉是什么,更能正确使用力量来练习,也能更清楚地知道自己练习后的效果。这与玩具的练习是同样的道理。

选择合适自己的玩具很重要,从阳具倒模到振荡器等,均可利用。大小要适中,最好不要选择太大的东西。硬度要适中并有弹性,太硬会使女性感到不适。当然手指也是一种很好的替代物。只有练到一定的功力,才可能有强大的力量来做那个“推”的动作。

你可以在等红灯的时候做,也可以在收拾衣服的时候做,还可以在看电视的时候做,甚至你在工作的时候,你也可以这样做,除了你自己,又有谁会知道呢?不用玩具的方法更为方便,因为你可以在任何时间和任何场合进行,而且是不为人知的,这样,白天和晚上的任意时间都能得到锻炼。

性的快感和乐趣,可以说是无限的。其实阴道收缩训练很简单也很容易,关键是要持之以恒,经常练习才能出现明显的效果。女性需要有轻松的心态,慢慢地找到那种感觉。最重要的一点就是,不要和旁人比,只和自己比。制定练习的目标,你要多长时间做一次,每次做多久……整套练习计划的安排,完全取决于你自己。

一旦用自我手法刺激达到高潮所需要的时间缩短到合理的范围,即它不会破坏夫妻做爱的节奏,那么就需要丈夫参加下一步的治疗,帮助患者体验与伴侣在一起时的高潮。实现这一点所使用的方法是不同的,主要取决于夫妻的活动特点以及他们的特殊需要。典型的方法是,医生指导他们以通常的方法做爱,并且告诉妻子在性交中不要作出获得高潮的任何特殊企图。丈夫在射精后,患者不再存在快速“表演”的压力,这时丈夫可以使用振荡器,按照她的手的指导使她达到高潮。应告诫患者千万不要“旁观”———“我奇怪我是否将达到高潮?”,“花的时间已经太久;这样他会厌烦的”,等。应告诉她要完全“自私”,完全集中于自己的感觉。如果尽管她作出努力,她仍发现她在“失去兴趣”,那么就鼓励她采用她喜欢的色情幻想来“分散她的分心”,以便抵消她的抑制她自己的无意识条件反射的倾向。

实际上,所有无高潮的女性通过前面描述的方法进行练习后都能够达到高潮。其中,许多患者变得更容易达到高潮并且得到与她们的丈夫之间良好的性关系。然而,有些患者发现从手淫高潮进展到有伴侣在场的高潮更为困难,她们在能够处于这样条件下发挥功能之前,还需要额外治疗。下面介绍的是卡普兰的两个病例,说明以前无高潮的女性在通过手淫获得高潮过程中的两种不同反应。

这里介绍性高潮障碍的一个病例,说明从手淫高潮向性交高潮的困难的过渡。

患者是一位29岁的心理学家,与一位40岁的经商的男子结婚7年,他们有一个2岁女儿,这个婚姻看似和谐美满。

妻子主诉从未获得性高潮,尽管她在实际上受到了足够的刺激,但她在性生活中从未有过高潮。她尝试过手淫,没有成功;丈夫一直耐心地、长时间地来刺激她,也没有成功。尽管没有达到高潮,但在做爱时有性快感,并且阴道非常湿润,不存在性交困难。

治疗:最初几次是一起会见这对夫妻,以便可以评价他们的性关系。然后单独会见患者,确定她除有高潮抑制外,其他性反应均为正常,治疗的首要目的在于试图帮助她通过手淫达到她的第一次高潮。

尽管患者多次使用过振荡器,但仍然很难达到高潮,主要是她认为如果她变得有高潮,就有某种说不清的灾难,如“患心脏功能代偿失调”。因此随着探索她的根本性的高潮恐惧,治疗医生则积极支持和鼓励她认识到:高潮是在无意识水平上的一种简单反射。医生使她确信如果她达到高潮,她的生活不会改变。经过4周治疗,最后她借助振荡器和利用性幻想(被一个男人压倒)作为分散注意力(以前总是担心如何如何)的手段达到她的第一次高潮。此后,她使自己达到高潮需要的刺激时间越来越短,但是继续需要振荡器。

从这时起让患者和她的丈夫一起见医生。最初试图让她的丈夫用振荡器刺激她达到高潮,但是失败了。她很苦恼,担心他觉得自己比其他女人低下,因为“她需要的时间太长”,这个过程对他冗长乏味。这种想法也不断地干扰她放纵自己的能力。在这一时期,她和丈夫在一起时变得越来越容易烦躁。

在治疗期间变得很明显,患者有很强的受性交情景激发的妄想狂倾向。异常敏感的自尊心和难于应付受侵犯是这个患者的主要问题。在她感到毫无忧虑之前,她处于高度竞争状态并且需要控制别人,但是她感到表达这些是有罪的,并为此恐惧,结果发展成对她获胜愿望的反作用。因此,“约束”她的敌意冲动是一种非常重要的防卫。十分明显,她持续的高潮困难在很大程度上来源于这种冲突。由于这种情绪问题不属于简单治疗过程之列,因此应该建议患者去进行个人心理治疗。

当她感到治疗医生真正尊重她的智慧和她的热情的时候,她就逐渐地对她的潜在问题产生有建设性的认知。她不再感到自己是个功能不健全的弱者,她能够在一定程度上摆脱这些问题,逐渐促进了她的性功能调节。首先,她能够获得由她丈夫用振荡器诱发的高潮。经过近一年深入细致的领悟治疗,她能够摆脱她对于振荡器的依赖并在异性恋情景中达到高潮。

这位女性对治疗的反应,在那些高潮困难与情感心理问题联系在一起的患者中是很典型的。然而,正像下面的病例说明的那样,许多女性能够仅在伴侣短时间附加的干预之后,从最初减轻其高潮抑制过渡到获得高潮。这些患者能够比较容易地实现从手淫高潮向性交高潮的过渡。

这对夫妻是一位28岁的公务员和她的34岁当医生的丈夫。她们均没有任何重大精神障碍问题,他们的婚姻非常美满。

主诉是尽管她们做爱频繁并且充满柔情蜜意,患者在性生活中却从未能获得和体验过性高潮。这对夫妻爱情深厚,患者对丈夫的性吸引力无话可说,无可挑剔,并且他轻而易举地就能把她带到强烈的性唤起状态。在婚后头几年,她模仿高潮,因为她担心如果他发现自己不能达到高潮,他会感到受伤和有罪。在他们决定来治疗的前一年,她暴露出她没有高潮,此后当医生的丈夫尽力通过阴蒂使她达到高潮,但是没有成功,她感到阴蒂刺激只能令人不愉快。

治疗:第一次,患者自己来了。她说丈夫“工作很忙”,不愿意耽误他,这是由于担心被抛弃而对丈夫的一种过分保护。治疗医生告诉患者,将来某些疗程会要求她和她丈夫都参加。治疗定于两周后开始。首先指导她试着通过手淫达到高潮,而不要试图在性生活中达到高潮。当两周后,夫妻一起就诊时,妻子说她自己用振荡器刺激阴蒂很容易达到高潮了。然而,她羞于让丈夫这样做,她认为这会使他反感并使他感到无能。这对夫妻在医生办公室公开讨论了这些问题,丈夫表示渴望参与训练治疗的态度使她大为放心。

医生进一步发现患者的高潮抑制不是严重的心理病态或婚姻困难,而是从未完全放纵自己的性欲感受,妻子的性需要从来没有主动表达过,因为她考虑的只是如何满足和取悦丈夫:“当我看到他勃起时,我觉得我必须立即满足它”,否则感到不安全,害怕被丈夫抛弃的危机感。这种情景并非由于丈夫自私引起,相反,丈夫总是试图刺激她,但是她从未想到自己的性感受问题,甚至只要她闪过这些想法就会引起焦虑。这是主流文化渲染的女性应该充分自信和实际上对男人的过分依赖的冲突造成的。因此,只要她被做爱高度唤起并达到较高的性紧张时,她不是鼓励丈夫继续刺激她,而是想,“够了,他一定累了。”在长期这样的性活动过程中她学会了控制和抑制自己的高潮反应,而丈夫并没有鼓励她的性依从和被动性,也不知道“她走到哪一步了”,他将她的结束性交的信号曲解为她已经有了高潮并从她的满足感中得到快感。

治疗主要通过增进他们之间的交流,集中改变他们的性价值观念,使她感到更安全、更积极、更敏感和更有责任感。在治疗医生和她丈夫共同的帮助下使她确信,她的性自主性和性自信行为不会削弱丈夫的性享受或损害她们之间的关系。她必须接受丈夫的爱抚刺激,讲出她受到刺激的感受,必须学会采用女上位控制性交。尤其是指导她仅仅对自己的感受作出反应而抽动,直到他出现要射精的感觉就停下来,让丈夫对她进行手法刺激,待丈夫射精感充分消退之后,再继续进行性交。在丈夫射精之后,让丈夫用振荡器刺激她直至达到高潮。即使需要很长时间,这种训练也应该坚持下去。要帮助患者学会在接受刺激时,不要在意自己是否在向高潮发展而分散自己的注意力。

丈夫很快就能使患者在振荡器的刺激下达到高潮,并且不久就能放弃振荡器,并且在很大程度上不再依赖性幻想。在治疗过程中,患者看到她的性功能的进步使他们更亲密,她从总体上变得更自信,并且更幸福。

她们的治疗共计12个疗程,其中4个疗程是夫妻共同参加的,治疗一直延续了9周以上。当治疗结束时,患者不仅能够很容易地经阴蒂刺激达到高潮,而且也能体验到性交高潮。

境遇性性高潮障碍的发生率远远高于原发性完全性性高潮障碍,这些患者在不太紧张的情景下可以获得高潮,但是当她们产生轻度焦虑的情况下,就不能获得性高潮。因此,她独自一个人时,也许能够通过手淫达到高潮,而与伴侣在一起时却不能达到高潮。或者她可能与她的丈夫在一起时能达到高潮,因为和他在一起时,她有一种安全感,所以没有任何心理的抑制和负担;但是她与情人在一起时,则不能达到高潮,因为她把情人过分理想化,并且担心情人会抛弃她。

一些境遇性无高潮女性,只有在剧烈和长时间刺激之后,才能达到高潮,一些女性只有在使用振荡器或进行口刺激时,才能达到高潮;有的女性不能因阴蒂刺激达到高潮,她们只有经长时间性交,才能有高潮,这些女性尤其值得临床注意。有证据说明,这种依赖于性交的女人表现出一种特殊的高潮抑制形式,它可能就像性交高潮障碍一样,对夫妻的性机能调节是有害的,因为它对男人的勃起产生压力。

治疗境遇性高潮障碍的目的在于发现和解决在“无高潮”境况下,对患者产生抑制作用的特殊冲突。

应该说只有那些高潮阈值相当低的女性,才能在相对温和的性交刺激下达到高潮。性交“冷淡”代表着一种特殊的境遇性性高潮障碍。千百万女性具有性反应性,并且常常可以多次高潮,但是她们在性交过程中只有同时得到阴蒂刺激才能有高潮。这些女性中的大多数都能享受性交所带来的快感,也能享受阴茎插入状态下接受阴蒂刺激达到高潮的快感,但是,性交刺激本身并不足以使她们达到高潮。

这些女性给治疗医生提了一个难题。她们有神经症吗?她们的丈夫是性无能吗?婚姻关系中有根深蒂固的干扰因素吗?或者这是女性性行为的一种正常表现吗?这种令人极其苦恼的问题预示着治疗的困境,也折磨着无数对夫妻,破坏了许多很好的婚姻关系。许多男人对于妻子不能在性交过程中达到高潮很困惑,要么自叹无能,要么觉得妻子“有病”或“性冷淡”。到现在为止也没有确切的证据支持下面这个观点:在性交过程中不能达到高潮是病态,或者相反地这样的反应属于正常范围。因此,在没有完满地解决这个问题之前,应该按照以下步骤进行考虑。

有些情况下,性交无高潮属于一种特殊的高潮抑制,或属于做爱技术贫乏,或兼而有之,如果是这些因素在起作用,那么性交无高潮可以经治疗而好转。但是,在有些情况下,有些女性并没有这些特殊病因因素,似乎就是需要比性交所产生的更强烈的刺激来使自己达到高潮。目前的治疗方法可以解决女性的与高潮有关的冲突,但是不能降低她的性高潮生理阈值,所以性治疗对性高潮阈值过高的女性是不会奏效的,她们必须得到强烈的阴蒂刺激才能获得性高潮体验。因此,没有必要把数以百万计的没有性交高潮却有其他形式高潮的女性统统诊断为性高潮障碍者。

治疗医生要查明造成性交高潮障碍的冲突是否存在,即这些女性对于性交是否具有一种特殊的冲突,如一些女性害怕怀孕,另一些女性担心受到男性生殖器的伤害,还有一些女性会因性交引起罪恶感或敌对情绪等。实际上,大多数性交无高潮女性似乎没有受到这种特殊冲突的伤害,她们对于插入没有矛盾心理,而且大多数人能享受其中的快乐并被容纳男性生殖器的感觉所唤起。事实上,在阴蒂刺激和手淫过程中,这种性交“冷淡”的女性在高潮时刻常常想性交或插入。所以一旦治疗无效,就可以认为是女性性反应的一种正常变异,其性高潮阈值很高而且不可能经治疗改变,于是放弃继续治疗的努力。医生对此要抱有充分的思想准备。医生应劝慰她和她的丈夫,她们依赖于阴蒂刺激达到高潮完全是一种正常和可靠的反应,没有性交高潮的做爱并不是有“病”或“失败”。应该强调,对于要么“依次”有高潮、要么在性交过程中使用特殊技术来刺激阴蒂体验高潮的伴侣来说,只要双方都不觉得这是一种“低下”的性表达方式,那么他们就可以享受极其丰富和美满的性生活。围绕“阴道”高潮的神话包括:①必须同时获得高潮;②女性在性交中每次都应该达到高潮。性教育应该致力于消除这些神话给人们造成的危害。

在另一些情况下,当夫妻性关系受到干扰时,则产生性高潮障碍。患者也许不能达到高潮释放所必需的性放任,因为她在“旁观”并且将注意力集中在取悦她的丈夫,或集中在不要阻挠他的射精驱动力。这样的女性对她们从性交获得快感的兴趣远远不如对性生活给她的丈夫带来的诱惑影响更感兴趣。毫不奇怪,嫁给早泄者的女性在性交中常常不能达到高潮。当丈夫难以控制射精时,他不能调整性交速度来适应他妻子较慢的速度。

对失败的预感可能也是性交高潮障碍的一个重要影响因素。在多次失望以后,女性觉得自己到在性交过程中根本不能达到高潮,而这种消极的“趋向”使她的困难得以固化。在治疗中,医生应努力让夫妻创造一种趋向快乐和无需求的环境,以此克服这种困难。例如,女性不能达到性交高潮常常由于夫妻性交进行的太早———女性刚开始湿润(它发生在兴奋的开始阶段),或者当她丈夫乐意时就立即开始性交,而此时她还未达到足够的兴奋,并且她还要抵制她预感不能达到高潮的消极力量,因此性交抽动提供的进一步刺激不足以使她达到高潮。由于女性在接近高度兴奋期时,更容易达到性交高潮,因此在开始性交以前,先建立规定的性作业,以便将女性的性紧张提高到高度兴奋期水平。首先指导夫妻应该尝试增加性交前的爱抚行为,而不是只依赖于性交提供的刺激来达到高潮。

因此,应该给所有性交无高潮女性参加尝试治疗的机会。治疗的目的是以此来降低高潮阈值,提高患者的整体性反应水平:一是证实并且尽可能消除确实存在的任何神经抑制和心理抑制的双重抑制因素,因为这些冲突将破坏女性的性反应;二是规定性感集中作业,它们旨在①提高性唤起,使得在性交开始时,女性已接近高潮;②增进她对自己阴道感觉的意识和快感;③最大限度地刺激阴蒂。我们发现,规定性感集中作业对性交与阴蒂刺激相结合的技术很有帮助。

无论造成患者问题的原因是什么,在试图进行性交之际,要规定做爱技术,这种技术可以提高女性身体的性反应能力和性唤起反应,它在阴茎插入之前降低了她的高潮阈值,并且有益于降低为达到性交高潮所必需的机械刺激的量和强度。具体训练任务则根据夫妻性关系的质量而制定。

对于两性关系良好而女性只存在一般性性抑制的情况,可以从一开始就直接进行指导下的性交。首先建议他们采用能够让她感觉高兴和注意力集中的事前爱抚方法,如采用几种缓慢的、挑逗的、间歇的、无需求的爱抚技术,直到妻子达到高度唤起后再行阴茎插入,开始时的来回摩擦可采用旋转、左突右冲、戏弄的方式,也可以抽动一阵后拔出来休息一会儿,然后再插入阴道并继续抽动。这种“停止—启动”的性交挑逗方式对妻子极其刺激,而且它本身也可能产生高潮。当它并未引起高潮时,丈夫可在性交间歇阶段刺激阴蒂,等到妻子被剧烈唤起之后再重新开始性交。他的抽动仍然应该是挑逗性的。一旦丈夫达到高度兴奋则停止抽动。这次不要拔出阴茎,他边刺激阴蒂,边调整呼吸以便加强对射精的控制。在女方临近高潮时停止阴蒂刺激,重新开始抽动,这时就可能引起高潮。“休息”期间的阴蒂刺激有助于保持女方的高水平唤起状态。因为与男人不同,如果刺激中断,大多数女性将很快回归到零点。如果从零点重新开始,唤起的过程将冗长而乏味。

女性在高潮即将来临时,必须加快自主的抽动或迎合动作,有时就能以这种方式达到高潮。如果由更激烈的阴蒂刺激向较弱的性交抽动刺激转变时患者出现情绪跌落,这时她必须无忧无虑地向伴侣说明,然后中断性交并重新进行阴蒂刺激,使她再次达到高水平性唤起;如果顺利,这种方法常常在一两个星期的实践之后奏效。这种技术有助于女性将注意力集中在自己的动情感觉上,减少任何分神的念头,主动对自己的性要求承担责任。

尽管阴道刺激和阴蒂刺激都能带来性快感,但二者却有相当大的区别。性交无高潮女性往往更喜欢阴蒂刺激产生的感觉,而不能清楚地感受阴道刺激的感觉,因此她们在性交过程中很少体验到阴道快感。

阴道有两种感觉输入来源,而且这些来源产生极其不同的性欲感觉。阴道上皮外1/3段和毗连的小阴唇的触觉刺激能产生特殊的性感觉,是由躯体神经支配;相比之下,阴道上皮内2/3段缺少触觉纤维,而只有感受压力觉的本体感受器,受自主神经支配,因此这一部分缺乏触觉和痛觉。扩张阴道和收缩并深压阴道四周肌肉时产生的是本体感觉,尤其是当性刺激过程中阴道的渗出物、血管的充血以及容纳阴茎而使得阴道更加充实时,这时可以给予女性无可比拟的、极度的快感,这与阴蒂刺激带来的感觉是截然不同的。阴道感觉对整个性反应的快感有重要作用,但一般不产生高潮。往往是阴蒂感觉的高涨才引起女性高潮。

为性交无高潮女性所建议的性练习的目的在于增进她们的阴道感觉。如果女性诉说她在性交过程中,她的阴道没有什么感觉,那么就建议夫妻用手爱抚和扩展阴道口。当深压施加于阴道入口4点和8点位置时,许多女性都有性欲感觉。还建议女性把注意力集中在由收缩耻骨尾骨肌肉而产生的愉悦感觉上。当开始性交时,要告诫夫妻,性交活动的主要目的不是获得高潮而是增进女性的快感,所以,不应给女性施加任何必须“达到”高潮的压力。因此,此刻的抽动应该在女性的控制之下,暂时“自私”一些,仅仅根据她的需求进行抽动。她要集中精力体验勃起的阴茎是如何在阴道内运动,又是如何给她带来快感,这是一种无需求的、女性自己控制的性交抽动。

虽然这个过程有时足以使女性产生高潮,但它也常常给女性带来产生新的焦虑。因女性之前一直受到抑制,若此时直接要求性快感让她担心自己可能被抛弃,她担心丈夫将对“迎合”她感到厌倦。如果患者在性交中采取女上位,她担心丈夫会觉得她在坐立时的姿势时不吸引人。事实上,如果丈夫感到他的性角色被先入为主地占领了,他很可能产生焦虑、不耐烦或拒绝,他可能采取反抗和恐吓妻子的行为方式来抵御这种焦虑。当这些反应成为治疗阻碍时,就要对此予以讨论并在共同就诊时解决它们,同时要坚持这种无需求的、由女性控制的抽动,直到女性对阴道快感有了明显的意识。

在阴道容纳阴茎的性交过程中同时对阴蒂提供可行的、舒适的、有效的刺激,如女上位时使阴蒂压迫在丈夫耻骨上,男子在女上位、侧位或后位性交抽动的同时用一只手或双手刺激阴蒂,是促成性交无高潮女性达到高潮的最有用的刺激方式,常常也能使她们如愿以偿。而女性的主动抽动、运动臀部,往往更能增加她们的性紧张。事实上,许多女性只有处于女上位时才能达到性交高潮。除非女性患有特殊的性交抑制,否则她几乎总能在这种刺激下获得高潮反应。不过,这种方法很少导致真正意义上的性交高潮。如果偶尔使用这种方法,它也是很有效的,因为它使女性能够体验到在阴茎容纳于阴道之际而同时获得高潮的巨大快感。如果每次性交都必须使用这种方法,那么,它很容易使男方感到乏味。

如果用阴道和阴蒂刺激相结合的方法作为一种过渡,那么女性有可能逐渐达到不伴阴蒂刺激的性交高潮。阴茎插入阴道后,基本不抽动或极缓慢抽动,同时以手刺激阴蒂,待妻子感到临近高潮时,让男方停止阴蒂刺激而女方主动抽动,她常常能够借助性交抽动达到高潮。如果这样没有达到高潮,那么她应立即停止抽动并再次让男方用手刺激阴蒂,直到高潮迫近,她再重新开始主动抽动。这个过程可能需要重复几次,除性交抽动外,不“允许”以其他任何方式达到高潮。当她体验这种方式几周后,她就越来越不依赖阴蒂刺激,反而更依赖通过性交抽动而促成的高潮。

简单提一下,关于结合阴茎插入和阴蒂刺激的其他两种不同的方法:振荡器和自我刺激。一些很容易抵触性交高潮的女子,当在性交过程中,受到一个振荡器的刺激时,她们也能够达到高潮。对于那些对手淫没有抵触的女性来说,在进行性交时可以自己刺激自己,同样,当她接近高潮时,她停止刺激自己并且可以以抽动来“结束”。自我刺激有以下几个优点:首先,一些女性认为在性交中需要男人额外的刺激,这对于男人来说可能是一种负担并且可能降低他们的快感,因此,她们不能集中精力达到高潮释放;其次,对于许多男人来说,看着一位女性并刺激她也会令他们自己高度唤起。

对于继发性无高潮患者,医生要把治疗重点更多地放在指导双方的情感交流和提高性交技术上,使伴侣间建立起良好的交流关系。有时还需要处理夫妻间存在的一般关系问题,如工作对他们关系的影响或者导致其感情疏远的某些因素等,在这些问题没有解决或改善之前,应先停止一段时间的性交。还要纠正男方错误的性观念,如他们总认为他们应该处于主导地位,只有他们才能决定性交的时间和频率,他们不理会女方的要求,不懂得体贴女方、尊重女方。要让他们懂得在性问题上如果能彼此关照,为对方着想,单凭这种和谐和亲密的气氛,就能使性交高潮的频率大大地提高。此外,还应让夫妻一起回味过去美好的时光,重温旧梦,重游故地,找回过去的感觉,检讨自己这些年来因为哪些因素影响了过去的甜蜜关系,如把精力完全集中在了工作、孩子和家务上,从而使得双方在一起的时间越来越少。回忆过去,找出根源,这本身就会重新点燃过去强烈的情欲,恢复正常的性体验和感受。

临床观察表明性幻想可以使夫妻双方保持更强烈、更亲密的性关系。需要说明一个问题,其中一方可能出现对性幻想的内容感到内疚,这是完全没有必要的。男子可能更多地使用性幻想来提高性紧张度,因为他们的躯体感觉强度和敏感性较差;女子可能觉得性幻想会干扰她对躯体感觉的感受,所以较少性幻想。如果需要的话,采用性幻想也是完全有必要的。

在性紧张达到一定水平后,与伴侣的语言交流往往会停顿下来,因为这会分散她们的注意力,而且性兴奋越强烈,呼吸也就越急促,即使讲话也是断断续续的,甚至词不达意。为了让女性消除外界的影响及摆脱任何的控制,进而达到充分放松并惬意和舒适地享受自己身体所带来的美妙感觉,这时就应尽可能地延长爱抚时间,以促使她达到性高潮。这个过程中,她必须信任她的伴侣,要相信不论她在性交过程中说些什么、做些什么,都不会降低她在伴侣眼中的形象和地位。因为生活中充满了种种约束,使得性伴侣离心离德,这也可能成为逃避性行为的一种合法方式。最令人烦恼的事情是对方一开始就直接指向生殖器。其实人体的许多部位都是十分敏感的,特别是脚、手、面、颈、肩、乳房和大腿内侧,如果忽略了这些区域,也就等于浪费了伴侣的宝贵资源。使用油剂和润滑剂等将有助于增强爱抚时的感受,其实,最重要的是发现其最敏感的区域和最容易诱发快感的最佳刺激方式。刺激时务必循序渐进,应该从不敏感区域向敏感区域逐渐过渡。

女性环切术即阴蒂包皮切除术,是治疗性高潮障碍的一种特殊治疗方法。它与北非、南美和大洋洲的一些落后部族中所盛行的将阴蒂与阴唇等残酷切除的割礼习俗不同。这种环切术是通过楔形切除阴蒂包皮从而达到使阴蒂更充分暴露于性刺激之下为目的,有人认为这样可以增进高潮的到来,特别是在原先存在粘连的情况下。反对普及这一手术的专家们则指出,由于阴蒂包皮是阴唇向上的延续,它能在性交时使阴蒂受到间接刺激,此外,阴蒂在平台期和高潮期将退缩到包皮之下,这将对此时业已十分敏感的阴蒂起到一个保护的作用,而过度地刺激会令阴蒂感到疼痛,从而破坏性反应的进展和女性的情绪。即使在外伤、感染和粘连的情况下,也很少需要切除阴蒂包皮。尽管这一手术步骤十分简单,几乎谈不上死亡率,但为数不多的阴蒂包皮环切术的结果并不能证实这种手术是十分得当的。这些女性往往主诉阴蒂敏感性有过暂时性增强,然后便恢复到术前水平。这与男性包皮环切术有相似之处,术后短时间内龟头很敏感,但时间一长龟头的敏感性与周围皮肤就没有太大差别了,而且,阴茎包皮的存在与否对男性性感并无重要影响。在进行女性环切术前,应详尽了解其病史和性历史,它往往可以揭示出更适合的性咨询和性心理治疗方案。

第一步:告诉患者不要触摸她自己的身体,因为她可能从未真正的了解过自己的身体,也未曾学会欣赏她的性器官。因此,最先给她安排的作业是设法增加她的自身觉醒过程,告诉她如何细心地查看自己的裸体,并试着全面观赏自己的胴体;然后用一面手镜来详细检查她的生殖器,对照有关图谱来辨明不同的区域及其构造。推荐患者在刚刚结束洗浴之际来进行这一生殖器的探索,其原因是清洁;另外也可利用洗澡时的温暖带来放松的特点便于进行生殖器的探索。好多人在进行这一阶段的练习之后,表达出无比惊异的感受,觉得怎么在过去这么久的时间中竟对自己的身体都一无所知或不甚了解呢?在这段时间,患者还要开始做凯格的锻炼方法来增加骨盆肌肉的弹性和血管分布,推测起来这将有助于增加她性欲高潮的可能性或开发其高潮的潜力。建议患者每天重复做3次,每次收缩和放松骨盆肌肉10次。

第二步:指导患者不只是靠视觉而且也靠触觉来探索她的生殖器。首先,思想上要解除顾虑,要避免使她感受到任何履行义务的压力,不要要求她在这一时刻就应达到性唤起(或曰冲动)。在这前两步中,我们仅仅要求患者对她的生殖器的形状和感觉不要过于敏感,并且应该对手淫变得习惯起来。但在治疗过程中也正是在这前两步中遇到来自患者的种种阻力。常常见到女性申诉她试过此种方法,但却不能做到观赏自己或触摸自己。治疗学家和丈夫应帮助她克服这种顾虑。治疗学家必须把要求患者这样做的真实目的和要求统统告知患者,要让她们懂得和理解她们将要做什么,而且为什么要做,这将有助于帮助患者充分发挥她们自己的主观能动性从而使她们采取主动合作的态度,这样才能取得积极的成功的进展。尽管她在这一阶段感觉到恐惧或厌恶,但是,一旦她按照上面所说的程序开始练习之后,她的这些感觉就会慢慢地消失,“犯罪感”、“神秘感”统统在扫除之列。

第三步:指导患者不断地对她的生殖器进行视觉与触觉的探究,但应强调让她们找出能产生快感的敏感区域。她也不要太局限于某个特定的区域,而应彻底地探究阴蒂及其包皮、大阴唇、小阴唇、阴道开口、阴阜以及整个会阴部,特别是那些毗邻阴蒂的区域。大多数女性将会把阴蒂定为最敏感、最令人满足的点,只有少数人会指出阴道或其他一些部位更敏感。

第四步:当产生快感的区域确定之后,则告诉患者集中精力地以手来刺激这些区域。女治疗学家应该在这个时候与患者来交流手淫技术。由于大多数患者把阴蒂定为最能引起快感的地方,所以一般情况下,应讨论对阴蒂的刺激手法,涉及的内容包括抚摸和施加压力等种种技巧的合理运用,在刺激时注意使用消毒的润滑剂以增强快感并防止出现酸痛感觉。

第五步:假如在第四步并未达到性高潮,这时就要告知患者需要增加手淫的强度和持续时间,直到“发生了特殊感受”或者在她感到疲倦或酸楚时再停下来。可以假定最长不超过30分钟或45分钟,这是一个合理的手淫持续时间上限,当然有些人需要的时间可能更久。可以推荐患者阅读一些关于描写爱情的小说来帮助性的唤起;还可推荐患者在手淫时进行富于刺激的性幻想,这往往也有助于性唤起。可是女患者们在手淫时性幻想的概念似乎并不能自发地产生。当然,这也是男女性生理的不同点之一。

第六步:假如在第五步时仍未能达到高潮,就建议患者购买一只在商店可以买到的供面部按摩或保健用的振荡器,它们在对性的唤起作用上是十分有效的。它有两种类型,一种是用皮带戴在手上,靠手指的活动来进行被动按摩;另一种可直接按摩治疗部位。这两种类型都有效,也具有各自的优点。

值得注意的是,女性可能会对达到高潮感到不安或害怕,她们害怕伴随的肌肉阵挛及不由自主的呻吟或尖叫会有失尊严,对这些失去控制的行为认为是“失态”或“色情”。为了使患者不必对失控产生恐惧或太敏感,就建议她们在家中没人时来进行有关训练。

第七步:就是在丈夫在场的情况下进行上述锻炼。在开始时女性可能不愿意,认为这是隐私,也害怕男方有反感情绪,这种种因素导致患者不放松。因此,要对患者进行劝说,让她们解除顾虑,解放思想。

第八步:让丈夫来进行上述的全部操作过程,既是训练丈夫如何理解自己的妻子,也能增进双方的依赖与合作,并可消除生活中的种种顾虑,去除隐患,安排好家庭生活。

第九步:一旦在第八步时达到高潮,这时可建议夫妻双方转入真正的性交,同时男方仍应坚持用手或振荡器来刺激女性生殖器。在性交过程中,我们推荐女上式(坐位)、侧位或后进入式,因为这三种姿势可以使得男方在插入后抽动时易于用手来接触女性的生殖器。一旦发生高潮,就可以认为是治愈了。最后,如果经过上述一切措施都不能达到高潮,那么建议患者在阴道内放置一个形状与男性生殖器相仿的类似物,最好选择大小、形态、硬度都很相像的再进行治疗。

因为女性的性敏感地带多而复杂,所以其手淫的手法也变化多端。根据海特调查报告(1976)介绍,女性手淫的方式主要有以下几类种:

Ⅰ型:为绝大多数的女性所喜爱,即在仰卧时以手来刺激阴蒂和会阴区域,其中47\%的女性系直接对阴蒂施加刺激;另外17\%的女性除了经典的对阴蒂刺激外,还有多种翻新的花样;还有9\%的女性不仅对阴蒂而且对整个会阴区域施加刺激。其实,最好的手淫方式以从轻柔的抚摸或轻巧的按摩刺激整个阴蒂区域开始。整个操作过程需要保持两腿稍分开的姿势,然后一只手用指尖作上下的推拉动作,待兴奋程度增加后,开始在阴蒂之上用力地摩擦或挤捏,最后对阴蒂包皮给以迅速的、急促的绕圈样刺激;同时,另一手可刺激乳头,或向下按压下腹部。由于手往往会感到疲劳,必要时换手以便休息;当然,也可以采用电池驱动的按摩器代替手的活动。高潮时身体会出现种种剧烈的翻滚、扭动、上下屈伸等运动。要注意保持阴蒂的润滑,适当使用凡士林、按摩乳等。要善于在手淫时幻想,从而在精神上达到唤起的状态,这也是十分重要的。在穿衣镜前,从镜子中边欣赏边刺激自己可以增加兴致。在Ⅰ型中还有一些新的变化,比如大部分女性用手刺激阴蒂时,总会把手指插入阴道;少数女性则用手掌来抚摸阴蒂和阴阜,而用手指刺激阴道口或插入阴道。当然,有些人只是为了得到润滑而把手伸进阴道之内。有不少女性除了直接刺激阴蒂之外,还对会阴的其他部分(外生殖器)用摩擦和按摩等各种手法进行刺激。

Ⅱ型:5.5\%的女性喜欢在俯卧位时用手刺激阴蒂会阴区域。Ⅱ型与Ⅰ型的动作和分类基本一致,不同的只是换成了俯卧位。这时女性可以对着手通过移动身体来进行摩擦,也可以手动身体不动。这一型中更多的女性喜欢把腿并拢,而其余花样与Ⅰ型接近。

Ⅲ型:4\%的女性喜欢用一个软的物体来压迫和摩擦阴蒂或会阴区域,如采用枕头或其他软物件来摩擦阴蒂和会阴区,特别是耻骨区。这时采用的体位也多为俯卧、并腿。Ⅲ型中并不利用手,而是通过对各种物件的摩擦产生手法所不能提供的美好感觉。也有的女性同时向阴道内插入手指或某种小物件行乐,有时可能造成阴道异物,故不提倡这种行为。

Ⅳ型:此型系通过紧紧地交叉双腿,并且有规律地挤压两腿而得到刺激感。约3\%的女性喜欢这种方式。通过这种方式女性可以选择坐着、躺着或者侧卧而不断剧烈地交替、紧缩和松弛腿部肌肉,特别是大腿的肌肉。或像剪刀样交叉摆动两腿,而两手则可抚摸双乳。有些女性甚至想用手臂等部位支持自己,使自己的身体完全离开地面,从而把全身重量通过会阴部压在相应的物体上寻找刺激。

Ⅴ型:2\%的女性喜欢温热水流按摩阴蒂。

Ⅵ型:1\%的女性喜欢采用阴道插入的方式,一般用手指、振荡器或阴茎类似物(如用性工具阴道哑铃或避孕套包裹的火腿肠)手淫。

在手淫过程中有一个很重要但难以回答的问题,就是腿的位置问题。有的女性喜欢把腿分开,而有的则喜欢并拢,还有的喜欢屈膝并且把腿伸向空中。当然大多数是把腿分开的。腿的位置的选择原因仍旧是个神奇的谜,可能取决于第一次得到高潮体验时的启发,也可是因为生殖器解剖的复杂多变造成的。总之,还没有一个确切的答案。

喜欢并腿的人认为这样可以使整个生殖器区域变得紧张起来,振荡器就能更快、更好、更容易地传送;还有的人认为并腿使高潮变得更强烈,而分腿则有削弱高潮的作用;有些人觉得如果自己不紧绷着腿和臂部等就不能达到高潮,或者需要刺激更久的时间才能高潮。

值得注意的是,女性在性活动中为了达到高潮而采用的技巧与女性在手淫过程中使用的技巧有关。这是由于已建立的反射和她们已习惯于接受这种刺激的缘故。我们可以人为地把手淫技巧分为两大类:一类是骨盆被动活动者,所有这些女性在手淫时均对她们的生殖器给予了一定的外来刺激,比如利用自己的手指、依靠外来的水流或是插入某些物品等。在性交中,不论她们是否意识到,她们的伴侣在性交中或性交外给她们提供的各种刺激总要比采用其他手段得到的高潮来得更容易些。如她喜欢用手指按摩阴蒂而获得高潮,若是其伙伴对她提供相同的刺激,那么,她更容易达到高潮;如果她喜欢通过淋浴时的流水来刺激自己兴奋,那么,当其伴侣以口来刺激其阴蒂时她也会更容易达到高潮,因为这不仅舒服而且也同样潮湿。而喜欢在手淫时向阴道内插入物品者,则阴茎的抽动将更易使她们达到高潮。另一类是骨盆主动活动者,这些女性在手淫时为了达到高潮多采用主动活动下半身肌肉的办法,或是对着软硬不一的物体摩擦自己的生殖器,或是交替挤压大腿以刺激生殖器。不论她们是否觉察到,在性交时或性交外,她们总会设法倚靠着伴侣的身体如大腿、臀部、生殖器等来摩擦自己的生殖器。

有关性高潮障碍的分类和定义显然是历史和文化的产物。认知—行为疗法是有效的,尤其是对原发性性高潮障碍更是如此。如果能充分调动起主观能动性就比单纯对症治疗的治疗效果更好。此外,为了理解和实施有效的性治疗模式,通过一个系统的框架提供了理论的前瞻性分析。希望在今后的十年中,人们能对女性性生理、性功能障碍、药物治疗等方面取得不断地新的突破。而针对目前单独的或联合的心理和生理治疗的有效性还需要作进一步的观察,这将使患有这一问题的女性从中受益。

(马晓年)


\chapter{第十一章 阴道痉挛}

阴道痉挛是一个令人困惑且引人注目的问题,多见于年轻女性,尤其是具有消极性观念者和具有性虐待或性创伤历史者。事实上,结婚多年而未成功性交的案例并非少见,其发生率可能远远超过人们的预料。许多妇女因性交困难而寻求治疗,但一些妇科医生确未能准确作出诊断,导致很多患者不能得到相应的治疗。在前来就诊的女性性功能障碍患者中可占12\%~14\%,占婚姻咨询的8\%。经历部分的或境遇性的阴道痉挛妇女的数目并不清楚,不过在一般人群中的发生率可能要高于临床样本。阴道痉挛发生率也受人种、民族、文化等社会心理因素的影响,如爱尔兰曾报道过他们的阴道痉挛患者竟占女性性功能障碍的40\%。


\section{第二节 阴道痉挛的定义与诊断}

阴道痉挛是影响女性性反应能力的一种身心疾病,临床上常常表现为性交恐惧。尽管女性明确希望阴茎、一个手指和(或)任何物体能够成功插入,但事实上,如果在持续或反复地试图进入阴道之时出现困难,她们将很快地出现强烈畏惧及严重焦虑,这是由于围绕阴道外1/3的肌肉发生不随意的痉挛反射,导致阴道口的关闭,使性交根本不能进行,甚至连常规的妇科指诊也无法进行。她们经常出现(恐惧性)回避、不随意的盆腔肌肉的收缩和对疼痛的预感/畏惧/体验。必须排除或处理身体结构的或其他的身体畸形。有关阴道痉挛反应的概念有相当大的出入,它可以被描述为条件性畏惧反应、性交恐怖综合征、转换反应,更新的说法是对反复会阴痛的条件化反应。当痉挛涉及大腿内侧肌群时,患者会紧并双腿死死护住会阴部,在西方有人把这种不随意痉挛现象称为“处女石柱”。

肌肉的收缩和阴道口的关闭多发生在性交之前,所以其性反应(如性欲)大多是正常的,并且患者也能有一定程度的性唤起和阴道润滑,其性高潮能力也可能并未受到损害,甚至有30\%~60\%的患者能在性梦中、爱抚时、手或口的刺激下体验到性高潮。所以,阴道痉挛并不是必然与整体性抑制或高潮抑制相联系。正因为如此,她们的不安和挫折感有时反而会更强烈。

那么,男女双方对阴道痉挛有何不同反应呢?这首先取决于他们的心理和性能力易受伤害的程度。由于这一障碍使得性交不能正常进行,所以人们很少对此置之不理。不过也有些“处女”妻子及其伴侣能够忍耐,甚至可能避开这种性交困难而拥有丰富的非性交性技能和性体验长达10年或更久。当然,尽管她们能通过非性交的交流方式带来相互和谐和满意,但丈夫还是会因为不能性交而感到挫折感甚至时不时地爆发激愤,他们会因破坏性心理而倍感失望或耻辱,而丈夫也可能把妻子的这一性问题看做是妻子对他的反抗。所以,这一性困难将限制性关系的发展或者破坏现有性关系。如果从新婚开始就存在这一问题,有可能造成婚姻失败和不育,它也严重影响妇科的保健(即常规的宫颈刮片检查)。由于面临离婚的威胁最终驱使她们前来求治。当然,对许多中国女性患者而言,更重要的就诊原因出于想要子女的强烈愿望,她们也是因为不育而就诊的。

尽管在DSM-Ⅳ(1994)中性交疼痛和阴道痉挛这两种性功能障碍是有所区别的,但在临床实践中有时却很难把它们区分开。阴道痉挛的诊断标准为“反复地或持续地发生阻碍性交进行的阴道外1/3部分肌肉群的不随意痉挛性收缩”,它不是由于躯体疾患,也不是由于其他具有精神症状的精神疾患所引起,如强行插入阴茎时往往造成严重疼痛,有时也可能在疼痛发生之前完成部分插入。然而,许多患者诉说这两种情形是同时发生的。事实上,表浅性交疼痛的最直接原因是阴道入口处的痉挛性收缩及伴随的性唤起与润滑的不充分。由于引起性交疼痛或阴道痉挛或二者同时存在的致病因素几乎是相似的,那么,在这种情况下可以命名为“混合性性交疼痛障碍”。

除了阴道口不随意痉挛之外,患者也常常伴有严重恐惧,由此而产生的恐惧性回避又使进一步的性交尝试受到干扰,并造成严重痛楚。虽然这种恐惧往往继发于最初的阴道痉挛,但是对插入的畏惧也可能先于阴道痉挛而存在。

从解剖学角度来说,阴道痉挛妇女的生殖器是正常的;从生理学角度讲,这时引起的肌群痉挛性收缩与性高潮经历中的节律性收缩截然不同。诊断常常是在常规妇科检查时作出的,指诊将难于进行或观察到明显的阴道口的收缩。有时,由于收缩强度很严重和收缩时间很长可造成疼痛,甚至妇科检查要在麻醉情况下进行。有些医生认为阴道痉挛是唯一的可以不经直接盆腔检查就能作出明确诊断的女性性功能障碍,他不能单凭任何已得到的确认的问诊和咨询技术就作出判断,即使病史很支持也不能轻易作出诊断。不过有些妇女的阴道痉挛仅见于性交之时而非妇科检查之时,此外也有部分妇女发生在性交过程之中。

无论是阴道痉挛和性交疼痛的哪种主诉,诊断时都应除外由于单纯的恐惧性性交回避和器官性因素而引起的痉挛或阻止插入,更重要的是需要评价是否存在可能引起或是促进这一问题持续存在的器质性因素。所以必要的病史采集、实验室检查和医学检查是不可缺少的。临床上经常遇到这样的情况,即使引起疼痛的躯体本身或生物学的原因得到治愈,而痉挛性阴道收缩仍可发生并引起部分性阴道痉挛。如果医学问题与阴道痉挛并存,诊断应记录为:阴道痉挛,因某种原因所致;如果存在其他精神障碍则不诊断阴道痉挛。

马晓年等(1993)从阴道痉挛严重程度上将其划分为:

Ⅰ级:痉挛的发生仅限于会阴部肌肉和肛提肌群,或痉挛仅在特定的境遇下发生。

Ⅱ级:痉挛不仅限于会阴部,而且包括整个骨盆的肌群,或痉挛在多种境遇下均会发生。

Ⅲ级:臀部肌肉也发生不随意痉挛,整个臀部可不由自主地抬起,痉挛频繁发生,性交很难完成。

Ⅳ级:患者双腿内收并极力向后撤退整个躯体,甚至出现大喊大叫等惊恐反应。这种反应往往不是实际行动所引起,而是对伴侣或医生的靠近和预感的反应,痉挛系原发性的,性交从未完成过。

阴道痉挛也可分为原发性和继发性,完全性和境遇性。原发性阴道痉挛占大多数,系指从开始建立性关系时起就发生的阴道痉挛,她们排斥任何种类的阴道插入,在性接触时,阴道口的紧闭使插入完全不可能,润滑也可能受到影响。原发性阴道痉挛妇女在她们预期或试图被插入时形容其感受或体验到的疼痛或感觉为“撕裂般”、“烧灼般”或“刺痛”,毫无疑问,妇女对阴茎接近时的疼痛和不适体验,会随着时间的推移而阻止她伴侣的阴茎勃起,并且可能对阴茎或手指插入的尝试不再感兴趣。许多患有阴道痉挛的妇女也从未能满意地完成过妇科检查或置入月经栓。少数情况属继发性,系指先前有过正常的性生活史,后来因种种因素发生阴道痉挛。完全性阴道痉挛系指在任何场合下都不能完成阴茎或类似物的插入。而境遇性阴道痉挛则指女性可以耐受某些类型的插入如月经栓或窥阴器,只是在阴茎插入时才紧张;也指女性可以耐受某些场合、某些伴侣的插入,却不能耐受另外一些场合或伴侣的插入,此时,患者的丈夫会很形象地描述“正常时很顺利,一犯病时阴道里就像有个肉球堵在那里,或在阴道内2cm左右地方好像墙一样堵在那里”。

[案例]境遇性阴道痉挛

“马老师,您好!最近见到您主编的《女性学咨询》,真是难得,它应该更早一些出版,我又意外地见到前言上说欢迎读者来信,这是我以前见过的所有书中都没有的话,使我很感动。我觉得您真是一位学者,不只是出书,还继续和读者沟通、交流。更多的是您人格的魅力,我特别注意到了您说的医德,我想您正是那样的人,这才使我斗胆写信(因为我碰过壁)。我丈夫有时说,在性生活过程中,他的阴茎会感到我阴道内有一个大肉球似的,于是阴茎根本插不进去,最后只好罢手。他说真纳闷,怎么有时好好的,有时却进不去,问我是不是有什么病,所以他特别担心,会不会有时候把他的阴茎卡在阴道之内呢?!这对生儿育女有什么妨碍吗?”

这位妇女介绍的情况属于境遇阴道痉挛,当不犯病时,阴茎插入没问题,当犯病时阴茎无法插入,这是由于阴道口周围肌肉痉挛性收缩的结果。如果是完全性阴道痉挛,性交无法发生,生儿育女自然会受影响,否则,只要有时能插入,生儿育女该不会成问题。


\section{第三节 阴道痉挛的发病机制}

阴道痉挛的病因始终是一个复杂问题,它既可以是精神性的,也可以是器质性的,有不少情况则是混合性的。

在把阴道痉挛正确地识别为一个条件反射性的消极反应之前,曾经一直认为它是一种转换性歇斯底里症状,把它概念化为一种特定的、无意识内心冲突的象征性表达。他们倾向于把阴道痉挛解释为女性对自身角色的反抗,是想挫败男性性欲的肉体表达及对男性性特权的持续反抗,女性会无意识地对自己说:“这个又大又危险的家伙就要进入我的身体了,我那儿会撕裂并大量出血,我会遇到不能忍受的特大痛苦,我面临的伤害简直太可怕了。”心理分析学派的治疗原则旨在鼓励患者从她假想的对男子的无意识冲突中清醒过来,并解决由此而产生的内心冲突,但他们治疗成功的案例却很有限,而且遗憾的是从来没有得到过有关这种治疗结果的正式的临床报道。相反,性治疗医生们却常常在没有探求或无意识的情况下取得成功,而且当阴道痉挛妇女治愈后,她们会对自己能正常性交并给予伴侣性乐趣而感到欢欣鼓舞。所以说,阴道痉挛未必就是所谓内心冲突造成的。这样,性治疗就可以通过系统脱敏技术把那种消极条件刺激与阴道不随意痉挛反应解脱开,治疗也就成功了。

学习理论普遍把阴道痉挛看做是一种条件化的畏惧反应,一种学习恐惧症。由于对插入必然伴随着极大困难及疼痛和不适的认知信念,这将不断强化条件化的畏惧反应。要想克服对性交的回避,就必须向认知信念和恐惧因素同时挑战。患者一方面渴望得到帮助并彻底治愈其性问题,否则婚姻将破裂并遭到男方的抛弃,而且自己也有不健全的挫折感;另一方面又害怕治愈,因为她们往往存在对性的偏见或潜意识中的某些严重心理冲突。为了避免插入可能激起的严重焦虑、强烈恐惧和极大痛苦,她们总是采取恐惧性回避矛盾的行为模式,虽然回避行为缓解了她们的预期焦虑,但强化了其回避行为。这只能使阴道痉挛长期持续下去得不到解决,形成一种恶性循环并成为将来接受治疗的主要障碍。

并非所有阴道痉挛都是典型的条件反射,这种身心障碍还会有其他社会心理致病因素,包括文献报道的:暴虐父亲留给女儿的是男性霸道的印象,使女性从小缺乏对男性的信任;正统的宗教传统信念常常造成严重性心理抑制;成长家庭中封建的消极保守理念经常使女性对性抱有偏见和反感;患者母亲自身的性冲突及向患者传递的错误性信息和对男人的偏见常常起到消极的干扰作用,她们往往认为性活动即使不是有伤害的起码也是令人失望或不愉快的事;对先前同性恋身份的反应。但上述这些因素并非必然的原因,具有同样问题的大多数女性仍会具有正常的性功能。在处理和治疗阴道痉挛时并不一定要明确找出其决定性的致病原因,因为在处理和治疗手段上的区别并不大,而治疗措施有它自身的共性和普遍性。

如果患者在过去具有儿童期性虐待、痛心的失恋经历或经受过残暴的性攻击等严重的身心或性创伤经历,她们往往会对所有男性产生偏见、恐惧或憎恨心理。尽管她们后来也约会、恋爱、结婚并与丈夫有着深深的感情,但在结婚后却常常遇到包括阴道痉挛在内的各种各样的性问题,因为她们一旦有性接触时就会预感到又要有一次性伤害了。人格因素如个性特别胆小、事事无主见、完全被动、依赖性强和自卑感强都会妨碍正常性功能的实现。

有些女性可能出于对自身解剖结构的无知,从而产生恐惧心理。如有的女孩存在处女膜焦虑:一位自行车运动员曾因外伤致外阴出血怀疑处女膜破裂使她的自信心受到重创,认为自己无法在婚姻中证实自己的纯洁性,从而导致婚后性交失败。建议她们通过阅读《我们的身体,我们自己》之类的自我教育的图书或一些科普著作了解自己的身体,打消种种错误认识。

值得注意的是阴道痉挛有时会漏诊,以致夫妻双方的性交失败并因此而长期持续下去。而在我们的传统意识中往往把性生活的成功与否归咎于男性性能力的强弱,认为性交不成功肯定是男方有问题如“硬度不够”,有时甚至给男子扣上“ED”的帽子,由于一直按男子有性问题而求治,所以,其根本问题始终得不到解决,甚至使男子出现严重的心理障碍。

笔者曾经接待过这样一对因不育而就诊的年轻夫妇,妻子(是一位护士)理直气壮地指责丈夫患有ED,导致结婚半年来性生活始终不能成功。因为她已年近30岁,特别想要孩子,也十分焦急,因此气急败坏地问医生,“他的ED能治好吗?要不然我就打算离婚了,我可不能再等了”。男方则嗫嗫喏喏地承认自己确实不行,表情十分紧张,脉搏竟高达160次/分钟,反复说“我过去手淫,所以现在有ED”。笔者在详尽了解他们的问题后发现这决不是所谓的ED,因为妻子所描述的男子勃起角度和持续时间完全正常,具有足够长时间的、充分的贴腹能力。当笔者根据问诊结果告诉他们男方没有勃起问题时,他们都愣住了,“那为什么插不进去呢?”笔者告诉他们问题大概出在女方身上,女方必须接受妇科检查。这时女方有些反感,因为她不相信自己有问题,并争论道:“我是学医的,我知道自己不会有事的”,这当然也出于对自我隐私的一个保护。在诊断和治疗安排上,笔者先向患者及其配偶证实阴道不随意痉挛的存在,最后经妇科医生检查证实女方确实患有阴道痉挛,因为医生刚刚接触患者身体时,患者便马上惊叫起来,而妇科医生根本没有办法插入一个手指以完成妇科检查。

通过深入了解发现,这位女护士曾因结肠炎接受过乙状结肠镜检查,而那次检查的巨大疼痛给她留下极其深刻的印象,于是她想当然地认为初次性交也会让她痛苦不堪。再加上她结婚较晚,先结婚的女同学们向她讲述了新婚之夜的疼痛难忍,更加剧了她对性交疼痛的畏惧,导致在新婚的前几个夜晚她一直以怕疼为由而拒绝丈夫碰她,因儒弱文雅的工程师丈夫心疼妻子,也只好忍着,直到第五个夜晚,当他们决定真的尝试性交之时,妻子马上又喊疼又哆嗦起来,男方只好再一次放弃。以后女方虽然不再那么紧张和不安,表面上也能够合作,但始终未能完成插入和性交。而女方则根据自己的点滴医学知识,开始抱怨是因为男方的勃起不够坚硬才导致性交的失败,此时男方也怀疑起自己的性能力了。于是男方开始“自查”,回想过去的手淫经历,女方一听则“恍然大悟”,“看!都是你过去瞎折腾闹的”,从此男方的压力和痛苦与日俱增。在此之前,有一家医院建议女方做处女膜切开术,说可能是处女膜太肥厚,但他们不同意,因为他们还是希望把破膜的“神圣一刻”留给自己。笔者接着给他们讲了阴道痉挛究竟是怎么回事,重要的是让他们双方理解这种收缩并非故意的,而是不随意的、反射性的。之后的主要治疗内容是指导他们回去后做一些行为训练,在充分放松的情况做自我插入手指,从一个手指到两个,实际上是系统脱敏治疗。并且鼓励他们要不了多少时间问题就可以得到彻底解决,只要他们按照医生的指导、双方密切配合就行,结果他们很快就能成功性交,并且在不久之后妻子就成功怀孕。一方面采用尺码不同的阴道扩张器进行阴道插入训练,另一方面为了减轻妇女对阴道插入的畏惧性条件反应,有必要教会妇女作进行性放松练习并鼓励她们自我探索阴道的结构。首先仔细辨认阴道的解剖结构,然后在充分放松的情况下自行插入手指,并感受阴道内置入手指后的感觉,经过逐渐增加手指的数目和让手指在阴道内作一系列的转动、张开和闭合等动作练习,慢慢地她的畏惧心理和条件反射就会逐渐消失。最终成功地转化为正式性交。

有些妇女不喜欢或畏惧性交,她们宁愿经过生物学技术帮助其得到一个孩子,那么治疗就不应采用扩张器训练或全身性脱敏,因为这些方法都可能以失败而告终。有些人认为没有必要迫使妇女接受性交,因为这是以男性生殖器为中心的性行为。存在主义与经验主义治疗是加强双方交流、密切结合、充分尊重双方意愿进行选择和行为体现,目的是尽量减少或淡化男性对不能性交的失望或不满,而不是简单地完成阴道容纳。如果双方能接受没有性交的性生活,那么人工授精则可以解决其对子女的需求。这也充分体现了西方多元价值观和对个人主观愿望的尊重。相信将来在国内临床中也会遇到类似的问题。有些女性虽然偶然成功并且经阴道分娩,之后仍然存在阴道痉挛。

任何造成现实的或过去的插入或性交疼痛的盆腔器官病理变化,都可以成为引起阴道痉挛的基本原因。这些器质性因素也是妇科教科书中所列举的最常见的引起阴道痉挛和性交疼痛的原因,例如,处女膜坚韧、致痛的处女膜痕、子宫内膜异位症、盆腔或阴道感染性疾患、经阴道子宫切除术、老年性阴道萎缩、子宫后倾或其支持组织的损伤或松弛、盆腔肿瘤、会阴切开术、分娩引起的病理损伤、阴道狭窄、尿道息肉、痔疮等。另一个重要的原因是会阴前庭炎,在这种炎症的影响下,即使是最轻微的触摸也会引起两侧小阴唇之间区域的剧烈疼痛,女性将主诉在试图插入时感觉异常疼痛。妇科检查可发现该部位不同程度的红肿,明显触痛,烧灼感等。创伤性妇科检查也是其原因之一。实际上,大多数器质性因素并不直接涉及或影响到阴道入口,但由于它们能够引起插入和性交疼痛,说明这些病理条件引起阴道痉挛的反应也并非偶然。

引起阴道痉挛的原因有很多,它们可以单独存在,也可以共同存在。无论是现实的或想象中的不良刺激,无论是患者主观意识到的还是没有意识到的,它们都可以成为引起阴道痉挛的原因。总之,最初造成阴道痉挛的直接原因可能是特定的,而该不良刺激可以引起肉体疼痛或精神上的压抑,一旦形成这样的消极条件反射之后,当初导致疼痛和痉挛的原因可能仍起作用,也可能早就不复存在了。不过,也有些时候是根本找不到明显的、有说服力的、可以充分证实的社会心理或器质性因素的特定的引起阴道痉挛的原因。


\section{第四节 病史采集和体检技术}

如上所述,病史采集很关键,而体检的阳性发现更是诊断阴道痉挛必不可少的依据,然而这两方面却恰恰容易为某些医生所忽视。例如有位著名男科专家仅仅用手捏捏阴茎便告诉患者:“你肯定没有血管性ED,完全是心理问题”。其实这是一例典型的血管性ED,患者已经在许多医院求治,只是下不了决心作一系列的检查和手术。如果我们不详细的采集病史和认真地作体检,又如何得到正确的诊断呢?

由于对具体的病史采集程序和方法在其他章节中已有专门论述。这里只针对阴道痉挛强调一下通过问诊应该了解哪些内容:从医生角度看,在问诊之后应对患者的以下情况有充分了解并作出评价。①简况:如年龄、职业、婚次、婚龄;目前的性问题和伴侣的性问题;何时开始,突然发生还是逐渐加重,双方性反应状况,性生活能力以及所能达到的最好反应是什么;②性史:手淫史、先前性经历、有无同性性兴趣和同性性行为、有无乱伦史;③与伴侣的总的关系:喜欢不喜欢伴侣、亲密程度、问题持续时间、针对这一性问题先前接受过何种治疗;④精神病史:目前的精神障碍、躯体疾患;⑤宗教信仰或对性持何种观念;双方性知识水平;各方寻求治疗的动机。根据上述资料对夫妻间的整个关系和性关系的质量作出基本判断。

从患者角度看,应该让双方对自己在处理下列问题时的表现作出评价:如能否承担义务,能否就需要和感受进行交流(一般和性方面),是否喜欢性活动,是否感到满足,是否紧张,想往的性活动频率,社交活动频率,业余时间的爱好,上个月的交流和性活动,各方估计伴侣对这些活动所向往的频率。

具有性交恐惧和阴道痉挛的妇女,常常拒绝体检,而能不能完成首次体检则与治疗能否取得较快进展有密切的关系。比如一位外地患者曾经接受某顶级医院心理学教授每周一次的心理治疗达9个月之久,尽管每次来回路途需耗时10个小时,但是其治疗效果也一直没有进展,而到性医学科接受一次体检后,取得明显进展。她们也许在之前有盆腔检查时造成严重疼痛的历史,但她们很少自愿地向医生讲明这一事实,有时只是偶然地在系统回顾的时候提到这一情况。因为她们认为盆腔检查必然会造成疼痛,所以,强忍着疼痛和不适来显示自己是个“好患者”。她们对性交也抱着类似的态度,如果医生不问性交时是否感到疼痛,她们很可能缄口不语,于是造成漏诊。因为疼痛不严重并且可以忍受,尤其是对于那些轻度阴道痉挛者来说更是如此,这一状况逐渐加剧并破坏婚姻关系为止,这时的性交次数往往已逐步降低到对方不可接受的低水平。妇女自己常用“撕裂感”、“烧灼感”、“刺痛”,来形容阴道痉挛的不适。不过阴道检查时很难发现有任何异常,因为阴道痉挛症本来就不会造成任何可见的组织损伤。而体检时让丈夫在场可以对其起到教育作用。

首先向患者阐述诊断性骨盆检查的目的和方法,向患者保证检查动作将尽可能轻柔,检查的进程将由患者自己来决定,也就是说只有当患者能够耐受时才能向下一步检查过渡。反之,一旦患者不能耐受时则暂时中断或放慢检查速度。任何带强迫性质的检查都是达不到目的的,相反它还会进一步的加强性心理损伤,从而使阴道痉挛的治疗变得更加困难。体验最好由女医生执行,以减轻患者的紧张情绪,每做一个动作之前都应预先通知患者即将进行的步骤和可能发生的不适或感受。

尽量使患者舒适地躺在妇科检查床上,可以用膝支持器则尽量不用足跟镫,不要许诺患者一点儿也不疼,而应说:“我的动作尽量轻柔。”

医生在对外阴进行望诊时常常可以发现阴道痉挛和股内侧或会阴部肌肉强直。这时要确定肌收缩是否是一种随意性收缩,因为有随意收缩时医生将无法确定有无不随意痉挛的存在。方法是让患者放松,做深呼吸,与之交谈转移其高度集中所带来的紧张。患者之所以出现随意收缩的原因包括紧张、内收双股或试图离开检查台。

首先向患者出示戴好手套的手,告诉患者检查时将分开她的阴唇、检查她的前庭、尿道口等外部结构。然后缓慢地分开阴唇、随时提醒患者无须紧张和畏惧,注意检查阴唇、阴蒂、尿道口、阴道口,注意有无痉挛现象发生。

先向患者出示涂好润滑剂的手指,然后把手指轻轻放在阴道口,稍施压力,询问患者有无不适,能否忍受等等。边检查边交谈可以起到松弛作用。待患者已不紧张、也无明显不适时,将手指缓缓插入阴道2~5cm。要向后稍施压力而不要直接向前插,若发现围绕阴道口之外的肌肉出现不随意痉挛性收缩或缩窄时,即可作出诊断,这一发现将使患者十分惊讶,因为她过去可能根本没有意识到这一问题,往往误认为性交不成功是男方ED或自己的处女膜太厚所致。这样就以可见的和动态的形式向伴侣双方证实了阴道关闭的实质是什么。如果能够成功地进行使患者毫无痛苦地、不适地、完全能够耐受的盆腔检查,就能使妇女的重重顾虑和担忧开始解除,这样,在阴道痉挛的治疗中便迈出了成功的第一步。

更深入细致的体检包括深部触诊,窥器检查,巴氏染色涂片,阴道分泌物的培养等。若在阴道浅部检查中已可作出阴道痉挛的诊断时,进一步的检查可以不必再做,或者在为了寻找和排除器质性病变时,待治疗开始后的一、两天内再安排这些检查。

当阴道痉挛十分严重、患者极度畏惧盆腔检查时,有时不得不采取在麻醉状态下进行盆腔检查。虽然这种方法有助于在这种特殊情况下确定有无器质性病变,但由于麻醉会引起肌肉的松弛,从而使本应存在的疼痛或痉挛消失,造成“正常”的假象,所以不能根据这种方法的检查结果而轻易排除阴道痉挛的存在,故麻醉无助于阴道痉挛的诊断。

值得注意的是有些妇女偶尔能够耐受妇科医生的检查,但在性交时的确存在阴道痉挛。如果没有预见和直接提出这种可能性,那么在体检时邀请丈夫的参加就可能就事与愿违了。相反,这时丈夫会认为女方在性交时出现阴道痉挛完全是拒绝自己的一种表现,是不爱自己了,反而造成双方更激烈的冲突。所以医生在这时要及时提醒伴侣双方,有这种情况也是可能的。因为医生的检查和丈夫的性交企图并不完全是对等的事,所以女方可能出现不同的反应,当然这种情况并不多见。所以光凭主诉就对阴道痉挛作出诊断仍是不可取的。


\section{第五节 阴道痉挛的治疗}

阴道痉挛的基本治疗原则应在采集病史和体检过程中就陆续向患者讲明,应该强调她们的生殖器的解剖结构是正常的,性交的失败不是因为她们的阴道太狭小,而是她们的心情太紧张或存在潜意识里的性抑制。当然,器质性阴道痉挛除外。治疗的最主要的和最重要的步骤是从肉体上证明阴道存在不随意痉挛,正是由于这种痉挛导致婚姻的不美满。治疗阴道痉挛的主要目的是改善其发生的直接原因—条件反射性反应,也就是要通过夫妻双方的通力合作来达到去条件反射作用。对于心理分析学派所说的无意识内心冲突等暂时置之不理,只有表现为脱敏治疗过程中发生障碍时才予以处理。只要她们在充分放松的情况下或在治疗取得一定进展后能够成功地向阴道内插入一个手指或类似物,治愈就是可能的。此外,让患者建立起一种由自己控制的、不会造成疼痛和严重不适的阴道周围肌肉的收缩和松弛方式,有助于她们从视觉(借助于一面手镜)和触觉两方面证实自己阴道的调节和容纳能力,即她们能随意地收缩和松弛阴道口,并自由地插入自己的手指。然后再把这些方法合理地运用于她们的家庭作业中,当然也可以在温水浴或淋浴时进行这些练习。阴道痉挛的基本治疗策略其实很简单,当然首先得解决因器质性因素而导致的肉体疼痛,只有这样下一步的治疗才不会太困难。从根本上说,治疗的完成就是成功地、进行性地解除体内早先存在的或形成的“保护”阴道口的不随意肌肉痉挛的条件反射。然而,在达到这一步之前,患者对进入阴道的恐惧性回避必须得到纠正。如今,已发现各种各样的能完成以上目的的有效治疗技术和方案。

人们假定阴道痉挛的发生属于一种条件反射性反应,旨在消除这一条件反射的治疗步骤将比较容易地并迅速地取得程度不等的治疗效果。使用阴道扩张器是阴道痉挛中经常采用的一种特殊治疗方法,这是性治疗权威们极力主张的能够使阴道口不随意痉挛得到缓解的系统脱敏治疗方法之一。这一常见的方法是妇产科医生发明的,它也得到马斯特斯、约翰逊和卡普兰等权威专家的推荐。这是一种必须在充分放松的、不会造成任何进一步伤害的条件下进行的治疗方法,通过从小到大的扩张器使紧张性极强的阴道逐渐松弛下来。因为这一步骤不会使患者产生畏惧和反感,所以治疗的最初目标便是缓解患者的焦虑,而此种方法也会减轻她对扩张器练习和性交的恐惧性回避反应。

进行第二次盆腔检查,目的是向患者展示各种规格的阴道扩张器,它们有橡胶的,有玻璃的,但它们的治疗效果没有任何区别。其型号分为1、1.5、2、3、4、5共6种。应向患者讲清这一系统脱敏疗法主要涉及这套扩张器,通过扩张器的使用证实她的阴道容纳能力,而并非利用扩张器来扩张她的阴道。这一扩张练习可以让患者自己参加,也可征得患者意见后邀请其丈夫在场,并使用手镜让患者自己观察。由于这一检查或治疗过程主要是对患者进行再教育的过程,所以要调动患者的积极性。因此要求患者必须接受自我治疗方法,也就是自己作交替的收缩和放松联系。如果在练习时使用一面小镜子的话,患者既能亲眼看到也能直接感受到这种一张一弛的变化,从原则上说,经过这种训练的患者其阴道周围的肌肉群会达到一定程度的松弛。正是因为许多妇女对自己的会阴和生殖器解剖结构的无知或不甚了解,所以在咨询或体检时必须向她们进行耐心讲解,补上这一课。令患者自己做绷紧—松弛练习时,要求其尽量绷紧骨盆肌肉(作憋尿的动作),维持数秒钟后再松弛(即完全松弛),如此反复练习1~2分钟。其目的在于让患者先主动地、有力地收缩骨盆肌肉,而后因不能持续收缩才进入相对松弛状态。这就在某种程度上使骨盆肌肉主动地得到松弛。然后可以让患者用手夹在两腿之间并达到完全放松。完成这一练习后就可以开始进行阴道扩张器的插入练习。为患者作第一次扩张练习时应使用最小号的扩张器,它通常只有筷子粗细。让患者自己好好看看扩张器的粗细,并告诉患者下一步将试着把扩张器插入阴道,插入的速度和进度由她自己掌握。医生将涂有润滑油的手指放在阴道口,嘱患者再次收缩肌肉,然后放松。趁患者放松时医生将手指末端缓缓插入阴道并在适当的位置上停下来,让患者体会一下感觉,若有不适就退出来,或把手指轻轻向下压向直肠,休息一段时间后再放慢插入速度,或调整一下插入的方向。再次显示扩张器,并让患者明白扩张器比手指细多了。再嘱咐患者绷紧肌肉数秒钟后放松,趁其放松时抽出手指。将涂有润滑剂的扩张器经指端缓慢插入阴道,插入时应向后方略施旋转的压力,遇到困难时再让患者重复绷紧—松弛过程,趁松弛时再插深一些,直到把扩张器完全插入阴道为止。扩张器插入阴道时应使其顶端指向尾骨方向,动作务必轻柔、缓慢。只要患者感到有任何不适就暂停操作,与之交谈,让其放松,即使是最轻微的不适也应等其消除后,才向下进行。插入后让患者拿住扩张器底部并轻轻在阴道内滑动扩张器,让其重复几次绷紧—松弛的练习,最后让患者自己取出扩张器。

下一步就是让患者自己使用扩张器,让其取得舒适的体位,自己把涂好润滑剂的扩张器插入阴道,具体操作方法与上述相似,注意先收缩肌肉,再松弛并缓缓插入扩张器。让患者重复操作数次,以建立信心和取得经验。当患者能够轻松地耐受时,再试用大一号的扩张器。

在结束这次治疗指导时应给患者布置家庭作业:包括一般性感集中练习和每天数次的收缩与松弛练习。医生要随时掌握患者的使用情况,若顺利则换用大1号的扩张器。随着治疗的进展,扩张器的号也不断扩大,若有困难则退回小1号的扩张器。一般5~6天就可结束1~4号扩张器的练习。这时可安排患者回家进行扩张器练习,习惯之后还可以让扩张器留在阴道内入睡,这样有助于促进和加快这一去条件反射过程(也可以用月经栓取代),若难以入睡再取出。切记不要在治疗开始时就让患者带上一套阴道扩张器回家练习,因为她在得不到及时鼓励和帮助的情况下面对这些扩张器,并且当她看到最大号的扩张器时,就使其精神紧张导致肌肉不能够松弛下来。

当使用到4号扩张器时,若无不适感即说明患者已具有能熟练地控制肌肉活动的能力,便可以尝试进入阴道容纳练习,即以达到高潮为目的的性交练习。这时一般选择女上位姿势,以便患者充分活动并对整个练习过程加以自我控制,如果喜欢也可采用侧位或男上位。要说明一点在她们试图开始实际的阴道容纳练习之前,她们已使用过周径与阴茎勃起时相同的4号扩张器,而且扩张器是金属等很硬的材料所制成的。事实上,阴茎即使充分勃起也有较大程度的柔韧性,这就会增加夫妻间过好性生活的信心。这时,患者就像使用扩张器那样把阴茎插入阴道内,若患者需要,可在阴茎上涂润滑剂。尝试阴道容纳的时候一定要有一个周密的安排,不要让妇女感到情感上和肉体上的伤害。如果一时不能耐受就暂停并推迟这种尝试。一般来说,经过一段时间性感集中练习和阴道扩张练习后,患者会达到一定程度的松弛,从而向正常性生活过渡。阴道容纳阶段是关键的,但也可能面临困难。如遇困难则多加鼓励,接受自己已有控制能力。在容纳练习时,应该告诉患者,以她的手指来指导丈夫的阴茎向阴道口的插入,及时指导丈夫在进行初次插入试验时注意抑制自己的主动的、迫切的要求,让抽动动作尽量缓解、轻柔、幅度小。

当患者使用2号扩张器无不适感觉后,可安排进一步的盆腔检查以发现潜在的器质性因素,如果发现器质性因素应作积极治疗。

对于是否让丈夫积极参与这一治疗过程,医学界存在不同的见解。有人主张让丈夫从一开始就参加进来,包括让丈夫来放置扩张器,这一方面使妻子容易解除顾虑,增加安全感,另一方面也让丈夫受到教育和启发。另外一些医生则主张直到患者的恐惧性焦虑有了显著缓解之后再让丈夫参加进来,他们认为在治疗阴道痉挛的开始阶段就让丈夫参加进来是不成熟的,它可以激起患者的自卫和焦虑。其实没有必要事先就具有先入之见,而应采取有弹性的策略,更多地尊重患者的意见,目的只有一个,那就是怎样才能更好地解除妻子的思想顾虑。一般来说,在治疗阶段可以让夫妻共同参加,而在扩张练习时多让妻子私下进行,或安排女方单独参加治疗。共同参加可以作为消除顾虑和达到充分放松的第一个步骤,这也是使用脱敏技术治疗手段中的一个关键点。

也有人主张尽量用两人的手指、月经栓等代替扩张器,因为阴道扩张器往往使患者产生不良的情绪反应,甚至激发情感损伤和强奸幻想,从而对治疗起到阻抗作用,所以用手指取代扩张器使患者更能接受。如果患者对月经栓较熟悉,就使用月经栓,否则最好用手指。此外,国内难以购到阴道扩张器。

如前所述,患者的担忧、畏惧及她随后对插入的恐惧性回避,可以成为完成这一简单的但又必须的治疗策略的重大障碍。因此,在治疗阴道痉挛中必需达到的第一个目标就是消除患者对阴道插入的恐惧性回避。一旦达到这一个目标,就会很容易地在几天之内完成彻底消除这一症状的治疗步骤。显然,治疗的成功最终取决于临床医师处理这一恐惧因素的技术。

对于一系列治疗技术来说,特定恐惧能相当迅速地退让和屈服。心理分析、行为治疗、催眠治疗以及药物治疗都已宣称了治疗的成功。过去,针对恐惧性条件反射的治疗所考虑的是心理分析,至今许多临床医生仍提倡使用这一方法。简单地说,治疗涉及阐释和消除构成患者畏惧基础的无意识内心冲突。然而,使用传统的心理分析治疗技术来诱导患者领悟是一个漫长的治疗过程。因此,基于促成无意识觉醒的心理分析战略的诱导领悟技术,若采用更主动的方法,则能够在短期性治疗安排中起到迅速减少恐惧性回避的作用。

行为治疗学家报道了利用“系统脱敏”的方法治疗性畏惧并且取得了良好的结果。这一高度有效地消除荒谬畏惧的方法,是让阴道痉挛的女子在放松的情况下反复地唤起对引起性畏惧情境的想象,即畏惧性阴道痉挛的妇女在她充分放松的状况下以逐步介入的方式幻想她所害怕的性情景。首先,她“勾画”出她丈夫靠近她的场面;然后,在她能够毫不紧张地耐受这一幻想时,她可以想象她与丈夫一起躺在床上,想象丈夫在勃起之后靠拢她,等等。当她能够在治疗医生的办公室里平静地假想和丈夫一起性交的场面并发生插入交媾时,可以认为她已准备好开始进行阴道扩张练习了,她通常能够在不太焦虑的情形下进行这些阴道扩张练习。

“幻想消除”法的种种变异方案也已采用过,例如,与逐步使用增加恐惧的性想象的系统脱敏法相反。“冲击疗法(满灌疗法)”旨在让患者想象她所能想得起的最令人恐惧的情境。这样,当以冲击疗法治疗一个阴道痉挛的患者时,行为治疗学家可能要求他的患者假想她正被丈夫的阴茎所撕破,按照这一理论,一旦她能够耐受她的这种无意识期望的幻想时,她就能够耐受实际的性交了。

在迅速克服患者的畏惧方面,也使用了一些其他方法,并取得一定的成功。这些方法包括催眠、使用镇静剂、止痛剂、鼓励和消除患者顾虑。

当患者的畏惧没有得到解决时,治疗不能继续下去。像上面所说的,现在可以利用的能够迅速缓解畏惧的技术有很多。但它们似乎都必须让妇女在事实上或幻想中反复暴露于曾经造成恐惧的情境之下。此外,我们相信当一个敏感的和能够胜任的治疗学家通过合理使用已推荐的能够缓解患者对插入的恐惧性回避的方法时,其所有技术都将证明是有效的。然而,我们已经发现,很少需要采用更复杂的技术来克服上面所罗列的种种恐惧。通过努力解除患者的顾虑,激励患者的信心,对引起患者畏惧的无意识成分进行详细的阐释等方法,让患者面对这一事实,如果她还不能主动地、乐意地往阴道里插入某种适当的物体的话,那么她就不能治愈。采取上述有效措施后通常可以减轻她们的畏惧程度,使她们能够继续进行阴道扩张练习。有时,医生也可采用药物治疗来促使其放松。

根据以上治疗原则,针对每位患者的特殊情况来安排治疗单元和性家庭作业都是为达到治疗的初期目标而服务的,即尽量减轻患者与插入相关的焦虑,以使她能够继续进行扩张练习。家庭作业的安排越早开始,治疗效果也就越好。家庭作业的最初阶段可以由患者自己单独练习,也可以建议患者和丈夫一起参加练习,可以让他们两人在自己的卧室里在充足的光照之下用镜子来检查女方的外生殖器。下一步,建议夫妇双方通过对照解剖图找出阴道开口的准确位置并进行轻柔的触摸,因妻子患有阴道痉挛,常常因为恐惧而回避性的接触。而夫妻俩往往对女性生殖器的解剖结构一无所知,这一经验对于脱敏治疗来说可以起到一个恰如其分的序幕作用,它对于患病夫妇来说具有指导意义,同时还能促进双方间的公开交流。在家里进行的体内脱敏是完全按照患者的节奏而逐步地完成的。临床已经证实如果患者自己控制练习过程,对治疗的疗效是有促进作用的。因此,通常告诉患者用她自己的手指,或让她丈夫用手指轻轻地向阴道内插入,或使用最小号的扩张器,并让手指或扩张器在插入之后在阴道内多停留一会儿直到出现的不舒服的感觉消失为止,此时便可以感受到阴道舒张和收缩动作给予的反馈信息,而这种感觉是这些妇女在向她们阴道内插入任何物体时的典型体验。至于这种扩张练习是应由妻子单独进行,还是应由夫妻双方共同参加,就像“合资经营”一样,应该由治疗学家审时度势地作出判断,主要原则是考虑哪一种方法使患者产生的焦虑程度达到最低。当然,它随病例的不同而有所变化,这取决于所采用的方法。如果这一最初阶段的任务能够在没有过分困难的情况下完成,那么就可以指导患者在她的阴道内进行反复地里外移动手指,或者让丈夫听从她的信号指挥来做这种手法操作,直到她能够在没有任何不适的情况下耐受这一操作。在这一时期,应该特别注重性咨询,对患者给予鼓励,打消她们的顾虑,正如上面所述,作为减轻她的畏惧和焦虑的手段之一,应该让女方来控制形势。当出现焦虑指征时,为了增进患者的放松,可以给予抗焦虑药物治疗,让患者在脱敏试验之前在家中服用。

当妇女能够耐受她自己的或丈夫的一个手指之后,然后让她试着插入两个手指。如果这一努力成功了,该夫妇或患者在下一个晚上可以试着在阴道内进行旋转手指的运动,并用不止一个手指的方式轻轻扩展阴道。这一练习要在实际的性交尝试之前进行。等到患者能够耐受插入她自己或丈夫的一个以上的手指,或在某些情况下向她阴道内塞入月经栓而不造成不适感时就能提议进行阴茎插入的阴道容纳练习。

为了促进阴道的松弛,还要指导患者随意地收紧和放松她的阴道肌肉,要让她感受到她可以通过锻炼对她的阴道口实行一定程度的随意控制。此外,指导女方在被阴茎插入的片刻应当随意地放松她的阴道肌肉,这时可以建议她采取双腿外展屈曲的姿势,这样可以反射性松弛盆腔肌肉。研究发现她在插入的片刻向下屏劲往往也会有所帮助。要等到患者能够耐受向阴道内插入物体而不造成不适时再试着性交。采取每一个可能的预防措施以保证最初性交体验的成功是很重要的。有15\%~20\%的患者在治疗过程中会遇到困难,但只要再教育、再鼓励也是可以很快地解决问题。治疗步骤与造成阴道痉挛的起因和维持因素无关。性交疼痛的治疗取决于病因,而局部状况确很容易得到改善。但阴道痉挛常需要为时较久的治疗,偶尔需要在麻醉之下扩张阴道,直至能插入3个手指为止,这样才能使阴部肌肉伸展开来,从而消除其痉挛。扩张之后,有些患者还可以通过阴道扩张器来获得帮助,但它不是为了保持阴道的扩张,而是为了给予心理上的支持,直到再次尝试性交时为止。应该记住最好的阴道扩张器是勃起的阴茎。绝经期后的妇女将发生会阴、阴道的萎缩,使用雌激素乳剂将有助于使组织获得更大的柔韧性。但是对行经正常的育龄妇女来说,她可以分泌产生足够量的内源性雌激素,外源性的雌激素对她是毫无价值的。只有极少数情况下需要作阴道整形手术以扩大瘢痕造成的狭窄的阴道入口,但单纯的阴道痉挛则不需要手术治疗,实际上还会因为造成阴道入口处的敏感的瘢痕而加重这一问题。

一般可以建议夫妇们在阴茎插入后先停留一段时间,丈夫在妻子的信号指引下开始作缓慢的轻轻的抽动,如果妻子希望他离开,他应马上抽出阴茎。总的来说,必须经抽动达到高潮的性交应延迟到随后的某个时间。

随着这些性体验逐步进展之际,患者对扩张和插入的恐惧及其他障碍也应在治疗过程中予以处理。在有些情况下,简单的教育和鼓励、逐步的、无压力的治疗方法使患者能勇敢地面对阻力和障碍。另外,把患者推到控制性练习的位置上,也能非常有效地化解患者的焦虑和对阴道插入的恐惧性回避。在这一过程中,她将在一定程度上愿意并且能够耐受不适的紧张,而这些在阴道痉挛消退步骤的开始阶段是必然伴随的现象。

然而,有时因为面对恐惧情境激发的焦虑太严重,以致无法经由这些步骤来得到缓解,必须采用另外的办法来克服患者对治疗的抵制。在这种情形下,经常用的办法是使患者勇敢地正视这一事实,即当治疗医生认识到向患者阴道插入物体时会产生不适和遇到困难时,除非患者从身心两方面都准备好这样做时,不然患者就无法治愈。作为一个一般规律,当经过简单的对抗和解除顾虑也没能战胜患者的阻抗时,治疗学家必须准备好在更深入的心理治疗水平上做工作。有时这种必要性能帮助患者度过她对性的无意识的内心冲突和负罪感。如上所述,无意识地害怕高潮、恋母情结的冲突、性别同一性问题等都能在阴道痉挛的病因学中起到一定作用。其他情况下,必须鉴别出夫妻之间相互关系上的深重的障碍,在性治疗进行之前至少能够解决部分障碍。然而,当畏惧性回避主要是继发于阴道痉挛时,可以使用行为疗法进行治疗。

在治疗阴道痉挛患者中已经证实,最有用的治疗策略是建议和鼓励“与你的不愉快感觉并驾齐驱”。改编自“完形心理治疗”的这一治疗策略是以回避不愉快情感和激发它的情境为基础的,它是一个重要的病因机制。因此,完形心理学治疗学家鼓励患者体验这种不愉快的情感,使它作为一个促进问题解决的手段,也从根本上对局面的处理不当进行改善。回避机制似乎在阴道痉挛的病理起因上起着重要作用。这些患者的报告显示其具最初的性经历都具有伤害性,但在这一群人中的表现却并不一致,而且并非所有暴露于这种伤害经历的妇女都发展为阴道痉挛。可以推测出许多姑娘的这种最初的性交体验除了造成预料到的损伤之外,还激起恐惧和紧张的感受。这就使得这些妇女中的大多数总是怀有这种不愉快的感受,当她们把性交变成一种具有高度乐趣的经验之后,她们就能够克服这种不愉快的感受了。相反,阴道痉挛的妇女似乎不惜付出任何代价来避免这种令人不快的感受,这就使她们控制和解决冲突的机会被剥夺掉了。

相似的是,在治疗过程中,一旦试图向她的阴道内插入一个物体便可以激发她的恐惧,阴道痉挛的患者想从插入引起的焦虑中解脱出来,因此回避扩张练习。然而,治疗的成功与否主要取决于患者实际上向她阴道内插入某种物体的能力,所以这些练习更是显得十分关键。如果要尽可能多地解除患者的顾虑,那么就要努力安排好治疗时的情境,因为这样才能使患者的焦虑变得尽可能的小。不过,如果患者想克服自己的问题的话,她就必须耐受某种程度的、暂时的、心理上的(而不是肉体上的)不适。治疗学家要提前让她作好思想准备,医生也要预先告知她,她会经历一定程度的、暂时的畏惧和紧张,并且在一开始时必须得耐受暂时的不适,而且得作出一个自愿的决定。还要向她讲明,她的治愈取决于她能在她的恐惧和紧张感觉时挺住的能力。当她确实有能力驾驭自己并在这些感觉面前挺住时,治疗学家的称赞和她丈夫带给她的欢乐将使她的行为得到加强。


\section{第六节 个案报道}

一位妇女提出的问题是:“马老师,我在1989年6月27日的《健康报》上看到您写的关于“阴道痉挛及其治疗”的文章,然后对照我结婚5年来的性生活方面的困惑,我认为自己患有阴道痉挛。结婚前我与一男子在恋爱的时候发生了性关系,那时我只有19岁,因为我的单纯、软弱,所以当他对我施行强行插入时我没敢反抗,而经过这件事情之后,性行为留给我所有的感觉就是恐惧和疼痛,我想那次被动性体验给我留下的正是您所说的不良刺激。在婚后,我的丈夫即使再温柔、再体谅,我也抹不去那个人留给我的阴影。丈夫了解我的这段历史,也常常劝我要放松,不要像还债似的对待性,每次他想同房之前总是先征求我的意见,而我的无能确使得他只能尽量克制自己。我深爱自己的丈夫,也总为自己的无能感到不安,越是想给丈夫性的满足就越是紧张,而阴茎就更不能插入。偶尔遇上我比较放松的时候也能插,一旦完全插入,稍停片刻,随着丈夫由缓到急的抽动,我也能体会到兴奋、满足,有时也能达到性高潮,激动之至甚至会不自主地喊出来。但每当丈夫有性交企图时,我又要下定决心才能投入,否则就会找借口回避,结果就导致造成丈夫偶尔不能勃起。我问过本地的妇产科医生有关我的性问题,她们或是不屑一顾,或是敷衍几句了事。尽管丈夫按报上讲的动作要温柔,要先爱抚敏感部位,但我总像一朝被蛇咬,十年怕井绳,丈夫的努力只能加深我对那次创伤的回忆,有时越摸越想睡,根本兴奋不起来。我生女儿的时候选择的是剖宫产,我有时候想如果当时是经阴道分娩是不是就可以缓解我的性问题了,直到看了您的文章后,我才知道我的问题并非我原先想象的是阴道狭窄所致。我真想早日消除这些性的阴影,创建一个完美的爱情生活。”

当时作者给她的回信是这样写的:“从来信中可以看出你的病因比较明确,而且幸运的是你有一个通情达理、十分体贴你的丈夫,那么你何苦继续背负着多年前的那个沉重的十字架呢?我看问题不在于如何让你丈夫更温柔、更体贴,而在于你自己要从心理障碍中尽快跳出来,因为你有正常的性反应,也能达到高潮,说明不存在器质性问题。要知道光是忧虑本身就能影响正常性反应,就像有些运动员一样,技术没问题,水平也不低,但总是紧张,结果发挥不好,甚至一败涂地。想要克服这种紧张的心情并不是嘴里念叨念叨就能解决的,下面介绍你如何做阴道插入练习,最好能找到一套阴道扩张器,(长7~10cm,粗1~3cm)如果找不到用自己的手指也可以。在仰卧位等情况下,让自己充分放松,排除杂念,然后插入最小号的扩张器,或一个手指,只要不产生不适,就可往深插,因为你有过成功性交的经验,按说这是不成问题的,每天练几次,每次10分钟,从小号到大号(或从1个手指到2~3个手指)。如果紧张,插不进,就努力使自己放松,再试。成功后便可以在想象有不良刺激的情况下做插入练习,慢慢地再想恶性刺激时做插入练习。如用扩张器,还可在插入后停留几个小时或者直接插着扩张器过夜,然后慢慢地适应。治疗的关键是放松,自己要有信心,摆脱一切顾虑,畏惧等情绪的束缚。你其实未必是性欲低下,只是重重顾虑抑制了自己正常的性要求。阴道痉挛是最好治疗的性功能障碍,几乎100\%可以治疗。所以务必充满信心。相信你会早日恢复正常的性反应。

当你能够成功地做手指插入练习或扩张器练习后,就可让你丈夫作同样的事情来帮助你,再经过一段时间,你的问题就可以得到彻底地解决。其实,只要成功几次,以后就不会再有问题。

你丈夫的问题可能是性欲得不到正常发泄而造成的,长期压制着性欲自然会抑制正常的性反应,勃起就是男子性兴奋的第一步反应。如果你实在不能耐受插入时,也可以帮助他手淫或寻找其他刺激方法,让他达到高潮(射精),这样对于保持他正常的性反应是有益的,没有害处的。当他勃起困难之时,就要靠你来刺激他、帮助他,同时,在刺激他的过程中也有助于你自己的性唤起”。

因为性交恐惧而出现阴道痉挛,根本无法过正常性生活的一位28岁的女网友,在网上接受我持续几个月的多次指导(方法同前述)后,来信叙述了她最近的情况。

马老师,您好!好久没和您联系了,本想等成功了直接告诉你这个好消息的,但是我这两天又有点难过了,前段时间我一直按您指点的方法训练,也用火腿肠(稍微粗点的,比实际生殖器稍微细一点的那种)试着练习,并且能成功地进入,虽然阴道口处不好进,但如果找对了位置,进去以后空间还是蛮大的。当时自己也挺高兴的,觉得快要成功了,然而前几天晚上我们试了两次,但都没能进去,还是感觉阴道口处根本进不去,是我还有些紧张吗?我有点失望了,通过训练我感觉自己的恐惧程度已经大大降低了,但是真试的时候还是不好进去。我当时是按照女上男下的姿势试的,这样我自己可以控制进去的幅度,但好像很用力还是进不去。唉,您说是我训练得还不够,还是心理还不够放松啊?请您再指点我。对了,我曾经给您说过我们群里的那两个女孩,她们都成功了,一个就在前两天,是按您所教的方法训练,然后真的就进去了。她们跟我说也不觉得疼就是稍微有些胀,总之是真的进去了,真替她们感到高兴,而且其中一个女孩说已经怀孕了,通过您的教导让我们深深地感受到你给予我们的莫大的帮助……。我什么时候才能成功呢?真的好期盼啊!最后,谢谢马老师在这段时间以来给予我的帮助和支持!

我鼓励她:“成功总在风雨后,试试男上位如何?既然火腿肠练习已经成功,那么也就差不多能进行性交了”(因为初次性交还是男上位好!)。没过两天,她紧接着又来了一封信:

“马老师,您好!向您汇报一个好消息,昨天我们又试了一下,感觉这次是真成功了,就是我在床边那种,比较传统的好进入的姿势,而且现在我对这些动作也不大恐惧了。真是感谢您的开导和鼓励啊,我太激动了……对了,马老师,我总怀疑进去后是不是到位了?反正就是感觉胀胀的,被撑的感觉,也没有特别深入到里面的感觉,但看着阴茎就差一点点就顶到头了,应该是进去了吧!现在比原来放得开了,而且通过这一次,我已经不怕了,我感觉真的不疼了。治疗的关键就是在于勇气和方法,我觉得这种循序渐进的方法起到了根本的作用,我逐步地脱敏了。再次谢谢马老师!”

我回信说:听到这个好消息,我为你们感到高兴,治疗的关键的确就在于勇气和方法,我觉得增强性幻想将有助于提高性兴奋和性冲动,当冲动和兴奋到达一定程度时,那么插入就更加容易些。不久,她再次来信:

“嗯,谢谢马老师!我现在有过一次勇敢的尝试后已经慢慢消除了心理障碍,我决定多试几次,逐步放松再放松,相信一定会达到好的效果,下回再向您报告好消息。练习过程中要是有什么感觉不对的地方,我还得向您请教。我感觉现在已经成功地进行了第一步,敢于让阴茎进入,接下来还有功课要做,要去找方法……您工作一定挺忙的,也要多保重身体啊,希望您也每天开开心心的!”

既然已经入门,相信她会很快取得更大进展。已经记不清这是通过远程指导而成功的第几个案例了,其实数字不重要,重要的是有一个这么好的平台让我发挥余热。感谢网络带来的便利,也感谢辛辛苦苦为我们提供技术支持的人们,你们辛苦了!

(马晓年)


\chapter{第十二章 女性性交疼痛}

性交疼痛是妇科就诊的最常见的性功能障碍,多见于生育期女性,也见于少数有性生活的绝经后期妇女。19世纪的妇科学往往把它归结为外科原因所致,多采取各种外科手术来处理这一问题。20世纪以来,随着心理学科的发展及其与多学科的交叉渗透,使得人们对这种性功能障碍有了全新的了解和认识。但由于大多数妇科医生没有受过心理学或身心医学的规范训练,以至于不能有效地解决此类问题。大多数性交疼痛的患者可以通过治疗得到痊愈或缓解,但若患者得不到医疗机构的有效诊治,或女性患者因怕羞等观念原因不去就诊,最终会因夫妻间长期的性生活不和谐造成婚姻破裂。

性交疼痛的实际发病率尚不清楚,因为前来就诊的性交疼痛患者往往是因为自己在性交中得不到快感,但又不得不满足男方的性需求,饱受煎熬而来就诊。在我国,其总发病率至少是就诊率的数十倍,甚至更高。国外有关调查表明,妇女性交疼痛的社区流行率在8\%~35\%。来自性治疗机构的性交疼痛发生率为3\%~5\%,与社区流行率调查结果相比较低的原因可能是许多妇女羞于或不愿因性问题而求医,或认为性交疼痛主要是妇科病所致而直接去妇科就诊。有关男性性交疼痛发生率的报道很少,多在1\%~2\%,多与包皮口过紧或粘连、泌尿系感染、阴茎畸形、阴茎硬结症等有关。

性交疼痛泛指在性交时伴有的急性或反复发生的生殖器或盆腔疼痛。性交疼痛的特点是性交时经常伴有下腹部疼痛,疼痛剧烈且反复发作,往往在性交之后数小时仍不能消失,有时不得不拒绝性交,从而使双方的性关系和婚姻关系蒙上阴影。它与初次性交时处女膜破裂造成的暂时性疼痛不同,也与很多妇女有时可能出现的性交不适不同,不应混淆。

大多数性学专家曾把性交疼痛看做是一种特定的器质性病理改变,传统的习惯是试图减少器质病理引起的性交疼痛;而如果检查不出任何器质性问题则认为它是性心理冲突的反映;剩下的小部分病例可以通过心理学方法得到解决。这些区分性交疼痛种类的观点及其反对意见是历史上长期存在的争论。比如马斯特斯和约翰逊(1970)是这样总结这一问题的:“一个常常令治疗医生感到讨厌和困难的问题是妇女这样的主诉,‘当我性交时它让我受伤害’,即使经过充分的盆腔检查,医生也不能断定患者的主诉确凿无疑,这究竟是因为存在一个检查不出来的盆腔病理改变,还是一位性功能障碍的妇女利用疼痛的复合症状作为逃避她婚姻当中不受欢迎的性活动的借口呢?而这种逃避的确是非常普遍的现象”。又如卡普兰(1974)的权威性著作《新性治疗学》等反映的基本上还是马斯特斯和约翰逊的二元论观点,顶多更明确地提到“器质性”和“心理性”(功能性)的划分。一般来说,对性交疼痛的讨论总是紧跟着对阴道痉挛的讨论。Smith和Buck(1983)是这样总结的:“功能性性交疼痛是一个排除诊断。它是一个更少见的问题,无论如何也不能诊断为器质性性交疼痛。它更可能归属于心理问题,或是在最初的妇科会诊之后要特殊的按性功能障碍来处理的问题。在霍普金斯医学院性行为会诊中心10年的历史中,仅仅诊断了不足20例这样的病例。

性交疼痛的定义是指在性活动时,由于阴茎向阴道内插入或在阴道内抽动或性交之后所出现的经常的或反复的阴部局部或下腹部等部位轻重不等的疼痛。性交疼痛包括两组症状:一是性交疼痛,指性交引起的阴道局部或下腹部疼痛;二是性交不能,指阴茎不能到达前庭及进入阴道,性交疼痛严重时往往出现性交不能。性交疼痛可分为原发性和继发性:原发性的系指性生活刚开始时症状即存在,多意味着存在某种解剖缺陷或顽固的心理因素;而继发性性交疼痛者,曾有过美满的性生活,后因种种因素出现性交疼痛。完全性性交疼痛指的是在任何场合下疼痛均持续存在,往往有器质性因素的影响;而境遇性性交疼痛则指在某些情境下出现疼痛,而在某些情况下却一切正常。显然,境遇性者多为心理因素所致。只发生在性交之后数小时的疼痛多为心理因素在作怪。马晓年等从严重程度将性交疼痛划分为:

Ⅰ级:性交时不适感或轻度疼痛。

Ⅱ级:性交插入时或抽动时阴道浅部疼痛。

Ⅲ级:性交时阴道深处疼痛或疼痛在性交结束后仍持续存在。

Ⅳ级:性交疼痛严重乃至性交不能进行,且性交疼痛的症状存在时间较久。

值得注意的是,这一诊断已经受到新的挑战,人们最近开始重新思考像性交疼痛这样的复杂问题。Maurice(1998)等开始把性和疼痛的研究与治疗结合起来考虑,试图建立一种更全面、临床上更有效的治疗方法。Binik等(2000)指出过去对性交疼痛的认识缺乏科学依据,从患者角度来讲是对她们的不尊重。他们认为性交疼痛应该是一个全新的概念,即应该把它看做是一种疼痛障碍而不再像传统那样把它看做是一种性功能障碍。他们列举了几个简短的病例摘要:①一位40岁妇女在性交插入较深并抽动时会出现右下小腹部的钝痛,酸痛感;②一位21岁妇女在阴茎插入时感到阴道口有灼伤感或切割样痛;③一位55岁绝经后妇女在充分性唤起后的插入和性交过程中经历阴道的酸痛和锐痛。

大多数知识阶层人士经过思考都会将上述表现判断为截然不同的问题,但在DSM-Ⅳ(1994)等许多疾病分类中却把它们划为一个相同的诊断。在DSM-Ⅳ分类中,无疑把女性性交疼痛的核心理解为性问题或各种冲突的产物。虽然在某些情况下这种判断是对的,但大多数时间却未必如此。因为性交中所经历的这种疼痛在其他一些特定的非性交情况下也能够引起,如妇科检查或放置月经栓时。

性交疼痛有时是多种因素交叉所致,无论是药物的或社会心理的单一干预都不可能解决好或治愈这些问题。当前的临床模式强调治疗队伍应包括各专科医师,特别是妇科医师和心理医师,婚姻治疗师,社会工作者等多学科工作人员,或者由各学科人员共同承担以疼痛为治疗目标的各个不同方面的治疗工作(也即神经的,肌肉的,器质病理的,心理情感的,人际的等)。这种生物—心理—社会医学模式也应成为性功能障碍治疗的原则。


\section{第二节 性交疼痛的病因}

在临床实践中,妇女在性交过程中伴有疼痛,大多数会去妇科就诊。有的可以通过妇科诊治找到病理性原因,如药物,外科手术等,从而得到成功治疗;但在大多数情况中疼痛只能部分或很少得到缓解。那些因为性交疼痛而反复去妇科就诊的妇女常常得到这样的回答:“妇科检查没有问题,问题出在你的心里”,然后建议她们去找心理医师或精神科医师。妇科医生这种回答虽然无可厚非,但对这些妇女来讲却不是一个积极的回答,她们往往就此而不再去诊治,于是备受疼痛的困扰。而当那些依从性较好的患者真正求助于某些心理医师或精神科医师时,却常常感到灰心。因为医生会详详细细地把她们的社会、心理情况问清楚,就是不问疼痛这回事。另有一些疼痛并非病理性和心理性原因所致,暂且把它归类为由性不匹配而造成的物理机械性病因。由此可以看出,学科间团队协作共同处理诸多因素致病原因的重要性。

妇女因性交疼痛而苦恼是一个非常普遍的现象,但只有小部分存在非常严重乃至影响性功能的女性性交疼痛。这种疼痛可能发生于阴茎插入阴道过程中,也可能发生于性交或性交后。如果发生于前者,需要区分是疼痛还是怕痛。如果发生于后者,疼痛和怕痛就不易混淆。此外,性交后疼痛通常属于深部骨盆痛。每当疼病时,就会不可避免地出现一定程度的肌肉紧张。

性功能障碍可由急性或慢性、全身或局部疾病引起。消耗性疾病如癌、退行性疾病、重症感染以及机体各脏器功能紊乱等都可间接影响性功能;局部紊乱也可较为直接地影响它。这些疾病的共同点是诱发疼痛,后者对性唤起、性兴奋和性爱都有抑制或衰减作用。在女性,诱发性交痛的原因很多,包括炎症、深部骨盆感染、阴蒂粘连以及各种阴道炎等等;可能损伤女性性器官的外科手术如产科损伤、难复性会阴切开术常常也会引起性交痛;此外,影响生殖血液供应的疾病,如血栓形成等,也可导致性交疼痛。

从单纯性生理学观点来看,阴道润滑不足会影响性生活。因此,凡影响这一润滑过程的疾患都会造成性交疼痛。如药物的影响,口服避孕药及抗组胺药物吩噻嗪就是两个典型,它们可造成插入或抽动时的灼痛感。其他造成阴道润滑缺乏或不足的问题包括干燥综合征、糖尿病。

由于人们观察到性兴奋期和平台期阴道内2/3段的扩张,以及子宫和阴道顶端的升高,因此对影响这种扩张和升高的器质性因素所致的阴道深部性交疼痛的发病机制有了新的认识,于是就采取一些措施来防止性交疼痛的发生。生殖道畸形如阴道下段的各种解剖变异、处女膜肥厚或闭锁、阴道口先天性狭窄、睾丸女性化或未经治疗的特纳综合征的阴道短小,子宫后倾、卵巢脱垂或囊肿、子宫内膜炎、盆腔炎等慢性感染或手术后等造成的盆腔内软组织的粘连,均可加剧阴茎深插入或抽动时的深部疼痛。子宫肌炎造成的深部抽动时的疼痛较少见,宫颈炎虽然很少造成性交疼痛,但在口服避孕药使用者中的发生率增高。

未很好愈合的外阴切开术造成较锐利的插入疼痛;因不熟练的盆腔底部修补术或子宫脱垂手术造成的软瘢痕、皮赘或挛缩均可引起性交疼痛;会阴皮肤及阴部软组织的暴力撕裂,擦伤,脱皮等损伤;处女膜痕,处女膜伞的裂伤或水肿使插入困难,抽动时加剧;尿道肉阜、膀胱膨出等均会因抽动过度加重插入带来的烧灼感;阴唇或阴道脓肿可造成局限性锐痛或灼痛;阴蒂包茎、阴蒂神经炎、阴蒂包皮炎等偶可引起阴道外部疼痛。白塞综合征又称眼—口—生殖器综合征,其临床症状不一,常见阴唇或阴道单独或同时存在溃疡并伴有性交疼痛,通常伴有口部损伤,眼部炎症,罕见关节痛或多发性关节炎。外阴营养不良性白斑病变也是阴道口疼痛的原因之一。子宫内膜异位症所致者临床症状变化很大,有时几年都不能确诊,疼痛特点是经前期较严重,子宫后壁及阴道后穹隆处可触及痛性结节是本病所特有的症状,痛经和不育是其常见并发症。值得注意的是膀胱炎或尿道炎造成抽动过程中的阴道前壁疼痛;女阴前庭炎近来受到越来越多的关注,它指女阴前庭部位出现多个细小红斑样溃疡或疮,其病因不明,可致表浅性交疼痛;感染性或化学性外阴炎可致阴部外表面灼痛;性传播疾病如淋病、尖锐湿疣、生殖器疱疹等可造成会阴部或插入时灼痛,往往可追问出不洁性交往史;阴道感染如滴虫或念珠菌所致者较常见,合并盆腔炎时会造成抽动引起的下腹部疼痛等疾患,性交疼痛因病情的间歇性波动而呈间歇性变化。萎缩性阴道炎造成的性交疼痛多见于卵巢切除,卵巢癌化学或放射治疗,绝经后,垂体功能低下,尿毒症及未经治疗的特纳综合征等雌激素缺乏的情况,产后使用某些化学药品欲使阴道恢复产前的紧张度而造成的瘢痕或其他损伤,阴道肿瘤也会导致性交疼痛。疝气常常造成发作性疼痛。肠炎,严重痔疮,直肠阴道瘘,直肠肿瘤等肛门直肠均可造成局部或弥散性疼痛。

一种较少见且尚未被国内医务人员重视的特殊器质性性交疼痛是锐性耻骨缘所致,它往往造成阴茎的插入困难,在插入时女方会出现锐痛。妇科检查时可发现妇女的耻骨很宽,且有或大或小的突出边缘伸向阴道之内,相当锐利,如果以指诊的手指向耻骨边缘的中心点略施压力的话,妇女就会体验到性交时感受到的那种疼痛。

精神因素同生物因素比较起来,更难识别和分类,尤其是在它们根探蒂固地从属于内心苦恼的情况下。过去对性功能障碍的治疗主要集中在基本的心理问题上,最近发展起来的性疗法则是分析引起性功能障碍的较直接的精神因索。过去所经历的内在矛盾是性功能障碍的一个根深蒂固的原因。如果这些矛盾在一定程度上控制着人的性生活,并使性功能发生障碍,我们称之为“内心因素”;另一方面,如果这种性问题只是两个人较大的矛盾方式的一部分,我们习惯称之为“人际间因素”。“内心因素”必须针对其自身进行处理。一个多次离婚的人,尽管配偶变换了,同样的矛盾可能依然存在。“人际间因素”也可能由于“内心因素”所引起,但后者可能是有限的,并不要求任何特殊处理。因此,“人际间因素”的处理重点在于“双方关系”,而不是单独某方。

同样的,假如某妇人曾经被人强奸,那么她以后总会觉得性交疼痛和困难,甚至同她最最亲密的伴侣在一起时也是如此。一连串复杂的、很久以前就忘记了的阅历也常常成为性交疼痛的原因。例如,某些性态度(如认为性行为是肮脏和卑鄙的)和孩子价值的认识也是导致心理性性交疼痛的原因。

对“失控”的恐惧是性交疼痛的又一个重要因素,常见于第一次性交或与不同男性的第一次性交。如果性高潮意味着自我放弃,那么“进攻性冲动”被触发的可能性就会随之出现。担心阴茎勃起损伤阴道或者阴茎被阴道陷住和噎塞的想法并不是没有。这种担心可能是有意识的,但更多的是无意识的,表现为男方不能勃起或女方性交疼痛而不能达到性快感,而双方都不知道其原因所在。对第一次性交将破坏“珍贵”的处女膜而伴随的失落感加重了性交疼痛的症状。

具体来讲,造成女性性交疼痛的原因包括消极性观念,性知识缺乏,缺乏经验及其他种种心理(焦虑,抑郁,畏惧,自卑,体象障碍,性虐待或创伤史),社会因素(伴侣间不信任,交流差等人际因素)及各种因素间的相互作用。从患者个人角度来说,应该注意获得与她们有关的整个精神心理状态的详尽信息,以寻找可能存在的诸如焦虑、抑郁等情感障碍的踪迹。例如,一位缺乏基本性知识的妇女这样说道:“我和丈夫新婚半年,性交却没有一丝快感,爱人的阴茎在我的某个管道中抽动,我一点都不舒服,我暗自怀疑别是进入尿道了吧?但如果进入尿道一定很疼吧,因为我每次做爱时都有一种不适感,尽管是阴部很湿润的情况下,阴茎往深处捅时仍很疼痛,我真要掉眼泪了。我该怎么办呢?如果我阴道太紧窄怎么办呢?另外,怎么确认阴茎是进入阴道了呢?”。我们发现,有时很难用单一的因素来解释这些现象,特别是当病程较为持久时。在确定性交疼痛的原因是精神性的之前,首先必须寻找和排除器质性因素。

既往的很多论著对这一因素给予否认,研究资料稀少,近期关于“性匹配”的问题再次由专家们提出,并给予关注。大多数专家认为:虽然阴茎和阴道间可能存在很多差异,但大小、形状上的差别从来不是性功能障碍的原因。松弛、润滑的阴道作为一个富于弹性的器官在性交时可自动扩张至一定程度以适应勃起阴茎的需要。尽管如此,客观存在的因“性不匹配”造成的性交疼痛的实例在妇科门诊随处可见。“性匹配”是指男女双方在性活动时,由于个体差异的广泛性,表现为男女双方的健康状况(生理和病理)、社会心理状况的差异性,而导致性欲强度和性能力不相适应,一方或双方长期得不到性满足的现象。大多数较难通过自我调节得到解决,部分可以通过专业机构的帮助消除不适应因素,从而获得适应。

物理因素是指在性活动时,由于女性阴道的大小、长度、弹性和扩张度,阴道的方向或叫软产道的方向(受骨盆倾斜度和盆底软组织影响),以及女性的外阴阴道口的相对位置(前后)和暴露程度(受软组织影响)与男性生殖器接触时出现暂时性过分扩张,顺应性差,或不能得到有效的接触,男女双方因骨盆形态差异、脂肪的厚薄造成在性交过程中发生“不匹配”或性交不适,出现直接的性交疼痛、性焦虑和性紧张,造成一方或双方长期性交疼痛,得不到性的满足。通常在同种同族差异相对较小,反之这种客观存在的差异会造成“性不匹配”。不排除在性焦虑和性紧张后,出现躯体形式障碍样的性交疼痛。尽管此类因素造成的性交疼痛多由后置因素变为混合因素,但一开始是以单纯物理因素出现的,所以往往为性治疗增加了难度。因此有必要将其列为单独的因素加以区别,以便更早发现问题而得到有效快速的解决。

指一开始就存在两种或两种以上上述因素而导致的性交疼痛。若患者自认为在性交开始时限制阴茎的插入或过深的插入有助于缓解和减轻性交疼痛的发生,这就揭示患者可能存在心理方面或婚姻关系方面问题的可能性,它们的影响可能叠加于原先存在的某些器质性因素的影响之上,这时除治疗器质性问题外,还应安排性感集中训练等行为治疗,以求解决并存的其他问题。上述心理性或器质性因素既可单独起作用,也可相互作用。

如果性交时下腹双侧疼痛并伴有中线部位的不适,而又找不出明显的器质性问题时,很可能是盆腔充血综合征或盆腔交感神经综合征所致。腹腔镜检查往往可以证实盆腔静脉的扩张。通过近年阔韧带造影得到的输卵管结扎后出现静脉曲张状态的证据证实,扩张的盆腔血管组织对性交疼痛是有一定影响的。当伴有性功能障碍时,这一综合征往往存在显著的心理因素影响,所以必须强调采集弥散性深部性交疼痛患者详细病史的重要性。因静脉曲张、撕裂,子宫内膜异位症造成的阔韧带操作常导致阴道深部疼痛,性交后常持续数小时。

此外,症状持续越久,就越容易引起双方更多的性问题,如丈夫会继发出现早泄甚至ED,这将给治疗带来更大困难。假如病史调查及体检证实造成性交疼痛的原因既不是男方因素引起,也不是女方器质性因素所致,就很可能是社会心理性因素所造成的。与其他性问题相似的是问题也可能出自男方,如阴茎严重畸形造成的性交疼痛,或男性在女方阴道尚未润滑时就粗暴地插入。也可能由于男性自卑或过分迁就女方。


\section{第三节 性交疼痛的分类诊断}

按照医学专科的方法,诊断步骤应包括病史询问,现病史中描述出性问题的现状,其中应增加详细的经历及伴侣情况、成长过程、家庭社会背景及出现过的身体或心理伤害事件;体检除按照常规体检外,应增加对性器官解剖生理功能检查,以及与性相关的内分泌、性腺的检查,常规心理CT检查或心理专项测量;根据病史及检查情况初步得出结论,确立病因,以便进一步制定出性治疗方案。

很多妇女会很有理由,也很合逻辑地要求对其可能存在的器质性因素作详细检查,临床医生也应该对性交疼痛作多学科评价,包括详细的妇科检查,有时甚至是有创性的检查,如阴道分泌物培养、超声检查、阴道镜和腹腔镜检查。安排阴道镜、宫腔镜或腹腔镜检查要慎重,因其可能给患者带来疼痛或创伤,当有足够临床指征支持时才做此类检查。鉴别诊断必须由一位能胜任的妇科医生担当,妇检应彻底检查盆底肌肉的弹性和张力,感染,萎缩等。由于患者可能存在潜在的妇科器质性问题,所以检查时可能伴有疼痛,因此检查手法务必轻柔,耐心,操作正规。即使进行了彻底的盆腔检查也未必能确认或排除是否存在器质性问题。

对性交疼痛的生理、心理、社会评价应首先集中在对疼痛的详细描述上,然后是对性行为、相互关系和个人幸福感的干扰程度。对性交疼痛的描述应包括疼痛部位、性质、强度、诱发因素、持续时间、具有的意义等特定信息。详细问诊除能提供重要的诊断信息之外,搜集信息过程本身也起到重要的治疗作用,即证明现有的主诉是合情合理的。一般来讲,很少有人对这一点作过认真考虑。对于性交疼痛后果的评价应集中在它是如何破坏性反应周期和性活动频率的,集中在对亲昵或两性关系的非性方面的破坏,集中在如何评价克服疼痛的策略方法和它们对个人调整的影响上。由精神健康专业人员单方面去检测和评价性交疼痛是不妥的,而见多识广和富有同情心的妇科医生同时进行评定显得极为重要,能对耻骨尾骨肌进行评价也会发挥很大的功效。

这一评价策略是基于若干推测的。首先,应把疼痛这一现存问题作为评价的重要出发点。第二,应该假设任何性心理冲突,社会关系问题或个人抑郁,所有可能成为导致疼痛的原因。第三,人们恐怕很难把器质性和心理性性交疼痛完全区别开来。根据临床经验,这两方面的作用势必相互影响并共同成为性问题的诱发因素,对任何性功能障碍都是如此。

下面从疼痛发生的部位来分析其发病原因:就像询问所有具有疼痛的患者那样,首先要询问“疼痛确切发生在哪个部位?”有些妇女能够立即在解剖图上或她们身体上指出发生疼痛的确切位置,而对另外一些妇女来说这却是一个很难说清楚的问题。当患者不能准确描述出疼痛部位时,往往反映出疼痛部位并非限定性的而是游走性的,或者说患者对自己的盆腔解剖并不熟悉。有些妇女因性交疼痛而就诊多次或多处,但从未有人问过她们这个问题。一般可以问疼痛是在会阴处还是在阴道内部?是表浅还是深在?是比较弥散还是局限?如果患者说不清楚疼痛部位,还可以询问她们是否在接受妇科检查时遇到类似的疼痛?

围绕阴道口的一处、多处疼痛或不适,对这一病因的分析,主要由于对性生理缺乏基本认识,消极条件反射的形成或焦虑等情感障碍造成的抑制性影响,也可能源于局部的炎症或其他病变。

患者主诉当阴茎深深插入阴道之时有中线部位上的疼痛,医生们往往把这种疼痛看做是器质性疾患的信号。但是如果全面考虑性生理反应过程,就会发现心理因素必然也起着重要作用。因为在缺乏充分的性唤起时,由于盆腔还处于正常的休息状态,妇女们就会感受到阴道深处的撞击感。虽然在大多数情况下这种感觉不会造成疼痛,但一旦同时存在某些局部疾患,过度焦虑之类的心理因素或既往遭受过痛苦的经历时,便给使女性带来疼痛感。其病理生理问题和阴道痉挛相似,也就是说先是具有致痛的问题或消极的经历,然后对所发生的问题越来越失去控制,性反应于是逐渐减弱,焦虑则不断加重。

如果双侧疼痛并伴有中线部位的不适而又找不出明显的器质性问题,则很可能属于盆腔充血综合征或盆腔交感神经综合征。没有明显的病理证据而出现单侧疼痛时,除了可能存在的心理因素之外,还应考虑是否与单侧卵巢的生理活动变化有关。

指妇女在进行阴道性交后出现直肠疼痛,这并不少见。直肠、子宫、阴道和膀胱的神经支配是相同的,它们都来自骶髓神经中枢。阴道性交涉及提肛肌、耻骨尾骨肌、耻骨直肠肌、梨状肌和直肠括约肌。疼痛的传入刺激通常来源于直肠黏膜和皮肤交界处的病变,激发肌肉的各种反射性痉挛。性交时发生肌肉收缩,在性兴奋的不同时期,外括约肌将发生不规律的收缩。在高潮时,肛门括约肌收缩呈规律性,持续8~10秒,并在收缩3~4次后逐渐消失。严重疼痛可发生于以下两种情况:①存在急性或慢性肛门直肠病理改变,如肛门直肠损害产生压迫造成的急性疼痛;②直肠存在轻微或慢性病理改变,如隐窝炎。当性高潮时,括约肌的活动提供了传入刺激,以致引起提肛肌、尾骨肌、梨状肌的痉挛,通常十分严重,造成位置固定的极端致痛性疼痛,称“游走性直肠痛。”耻骨直肠悬韧带也将受累。热水浴有助于缓解疼痛,但治本的办法莫过于消除或缓解肛门直肠原发疾病,如肛瘘切除。痔疮不会导致这类疼痛,除非是栓塞性的。如上所述的疼痛常常引起复杂的心理问题,男性常常为此感到委屈或内疚。其对女性的心理影响也是很明显的。治疗的关键是找出和解除病因,不要低估严重的不适合疼痛。可以安慰患者,讲明问题是能够得到纠正和改善的。

评价工作的第二个关键是设法确定疼痛的性质。由于性交疼痛是一种主观体验,所以难以客观和准确地加以描述。从性交疼痛现象学的角度出发可以对其性质作如下描述:它是“一种烧灼感”,“能觉察到的程度不等的,瞬息间的锐痛”,“断断续续的刺痛或剧痛”,“反复的高度不适”,“钝痛”或“放射性”等。

医生会询问患者在性交之前的事前爱抚手段是否有创造性?是否已充分唤起?对性活动是否有厌恶或抵触情绪?除了性交之外还有哪些情况可以导致疼痛的发生,一旦出现疼痛之后它会持续多久?例如,一位30岁妇女主诉性交时只有轻度不适,而性交之后的严重烧灼痛却可持续36小时之久。这种灼痛干扰她的睡眠,经常使她不能上班或放弃第二天的活动安排。在妇科检查、排尿、放置卫生栓、手或口刺激时、与衣物摩擦、运动等情况下同样可以引发与性交时所经历的相似的疼痛。然而,尚不清楚上述疼痛诱导因素在各种活动的行为水平上的具体作用如何。例如,性交活动将涉及压力、摩擦和温度的增加,尚不清楚这些诱导因素中哪一个能单独地或哪几个能共同地诱发性交疼痛。有些妇女主诉在性交或其他活动时出现疼痛,但同时也主诉不进行这类活动时仍会出现同样的疼痛。大多数临床医生和研究人员坚信,区分反复发作的急性疼痛和慢性疼痛是很重要的,但相关资料几乎没有。一旦产生疼痛,其持续时间的长短也很重要,持续数秒、数小时、数天,或持续时间不规律。若性交疼痛仅在性交之后数小时才出现,则多为心理因素在作怪。还应注意有哪些因素可以增加或缓解疼痛。

要求患者对性交疼痛和其他时间的疼痛进行对比并分别打分。临床可以采用两套有用的记分法,一是疼痛的感觉方面(无痛,轻度疼痛,中度疼痛,重度疼痛,我感受过的最剧烈的疼痛);二是疼痛的情感方面(毫无抑郁,轻度抑郁,中度抑郁,重度抑郁,十分严重的抑郁)。具有性交疼痛的妇女常能系统和可靠地在两套记分系统里打出合适的分数。让妇女坚持写与疼痛相关的日记,即要求妇女在每次疼痛发作后记日记。除了用打分法定量记录疼痛之外,还要求描述疼痛部位和性质,导致疼痛的情境和活动,在疼痛之前、之中和之后的想法和感受,是否作任何努力以减轻疼痛,这些努力的效果如何等。如果引起疼痛的情境与性有关,则应要求妇女记录主观和肉体(润滑)的性唤起程度如何,她们的伴侣对疼痛有何反应。

通过体检及专科检查发现明确的器质性疾患所导致的性交疼痛。一般来说,器质性因素容易使疼痛集中于某个固定的区域,而心理病理因素造成的疼痛往往具有飘忽不定、时轻时重的特点,当然,心理因素有时也可使其疼痛点固定,这需要加以鉴别。

在排除器质性疾患所引起的性交疼痛症状后,通过心理询问及专项测量而确立。可分为原发性和继发性心理性性交疼痛。

症状出现初期就诊者或通过病史发现,可排除器质性疾患,特别是排除症状初期所出现的心理性性交疼痛,通过男女双方的专科检查,可明确为物理性性交疼痛。其循证诊断标准尚需大量实证数据研究。

当通过病史及相关检查证实上述两种或两种以上致病因素同时存在时,可诊断为混合性性交疼痛。也可分为原发性和继发性的混合性性交疼痛。

对疼痛意义的评价也很重要。许多妇女愿意按她们自己的观点对性交疼痛的原因或意义作出解释。尽管这些解释并不完善,但其对于评价和治疗性交疼痛仍具重要意义。例如,一位35岁妇女相信她的性交疼痛是性病所致,而性病是丈夫有了婚外恋之后传给她的,但她丈夫却否认有婚外恋,而且双方检查均排除患有任何性病的可能性。妇女对疼痛的描述可能受以下一些因素影响:①疼痛的强度和标签;②时间过程;③后果;④原因;⑤能否控制。这些因素都会严重影响患者对待疾病的态度和对治疗安排的配合。对于性交疼痛的妇女来说更是如此。例如,妇女更倾向于把性交疼痛归因于器质性疾病而不愿意把它看做是一种性问题。把疼痛看做是器质性问题将会使心理干预变得很困难,当然这样可以减少妇女的心理痛苦。Meana等(1999)指出,认为自己的性交疼痛是心理因素所致的妇女给自己疼痛打的分要更高些,心理痛苦的程度也更高,给婚姻调节打的分则较低,更多地把它看做是一种性问题;认为是器质性问题的妇女打分情况则相反。有时,夫妻频繁争吵意味着他们之间隐藏着严重的性问题,争吵是婚姻冲突的外在表现形式,而性交疼痛则成了回避冲突的自我保护措施;也有些夫妻声称除了双方性关系有问题,其婚姻的各个方面都很好,这时要警惕是否存在尚未被他们认识到的其他方面的冲突以性问题的方式表现出来了。所以要重视性交疼痛对患者夫妇究竟意味着什么。

性反应周期和频率:性交疼痛对妇女性反应周期的破坏程度因人而异,有的妇女严重,而有的妇女性反应过程并未受到干扰。在标准妇科检查或非性交活动时的疼痛计分结果与性交中的疼痛强度并不是高度相关的。此外,在性交时所经历的疼痛与性欲、性唤起、性高潮及性交频率之间的关系也十分复杂,人们对此了解甚少。具有网球肘的人尽管肘部很疼,仍常常继续打网球并能自得其乐;但也有些人在出现网球肘后就再也不打网球了。所以人们很难从躯体的病理变化程度,从疼痛报道,从先前的行为来预见行为受干扰的程度。例如,一位27岁的大学毕业患子宫内膜异位症和会阴瘘,在妇科检查时医生触摸的所有地方几乎都有严重的疼痛,但她在患病的2年期间里一直保持着良好的性关系,平均每周2次,几乎每次都能达到性高潮。

一些妇女虽然长时间存在性交疼痛,她们会主诉性欲低下或性唤起差并期望得到帮助,但却从不提及疼痛问题。很难理解她们为什么不愿谈及自己性交时的疼痛呢?有些妇女在最初具有充分性欲或性唤起时的确伴有疼痛,之后她们便以为只要性交就应该伴随有疼痛,所以不以为然。如果不是后来性交疼痛影响她们的唤起或出现其他性问题,她们还不会知道这是一种性问题。所以,当面对妇女性问题时,询问她们在性交时有无疼痛就显得尤为重要。DSM-Ⅳ(1994)规定,一旦存在润滑不足的情况,就不再考虑性交疼痛的诊断。这似乎表明,对于那些绝经后润滑缺乏的妇女来说缺乏疼痛是很自然的事,而性交疼痛能够导致性唤起减弱和润滑减少。详细询问病史有助于解释问题的发展过程和前因后果,但有时仍不能得出明确结论。Wouda等(1998)想在实验的条件下证实性交疼痛妇女与对照组妇女在观看色情影片时阴道血管充血反应会有显著差别,但是结果并未证实他们的假设。从这一点看,将阴道润滑或唤起不足作为性交疼痛的优先排除标准或致病原因还为时尚早。

女性性交疼痛会很自然地影响到两性亲昵关系的维持和发展,我们要评价疼痛性质对此的影响程度。如果性交疼痛的妇女对疼痛程度的打分并不高,那么她们伴侣调节的得分就会明显处于较高水平。有些性交疼痛妇女并无稳定的性伴侣,她们期待这一问题解决之后再建立稳定的性伴侣关系;有些妇女则自暴自弃,认为反正也不指望有更好的伴侣关系,还不如将就目前的并不满意的两性关系。有些证据表明,女性性交疼痛的经历能够影响她们伴侣的性反应,所以必须认真对待。

一般来说,器质性因素容易使疼痛集中于某个固定的区域,而心理病理因素造成的疼痛往往具有飘忽不定和时轻时重的特点,当然,心理因素有时也可使其疼痛点固定。物理性因素往往在第一次性交时就会出现疼痛或焦虑,并伴有性交困难为特征,要与其后出现的心因性疼痛加以区别。混合性因素所致性交疼痛在一开始就证实有多种因素的存在,其鉴别诊断的意义是为性治疗的预期治疗效果的判断提供帮助。

特别提出与阴道痉挛的鉴别,早在DSM-Ⅳ(1994)和大多数性治疗学家认为性交疼痛与阴道痉挛完全是两回事,尽管它们可以是互为因果的关系并形成恶性循环。原则上,只有妇科医生才能观察到具有诊断价值和代表性的阴道痉挛的阴道/盆腔肌肉痉挛性收缩,这是二者鉴别诊断的唯一体征。实际上,许多妇科医生进行鉴别诊断时是以阴茎是否能插入为基础的,她们并未进行具有性学目的的妇科检查。不少文献报道这种鉴别的确很困难,而且二者还有相当比例的共同患病率。


\section{第四节 预防和治疗}

造成性交疼痛的因素自然成为预防性交疼痛的关注重点,也正是这些因素交叉所致的症状性疾痛给诊断和治疗带来困难,它要求治疗师具有多学科的专业知识,并在咨询和治疗中具有相当的人格魅力和信任感,在其身后有一个配合良好的专业团队。预防往需要医疗机构、心理干预机构和文化教育机构的共同介入,这种发挥有效作用的专业机构必将给众多的个人与家庭带来和谐与稳定,促进社会的进步。

应付疼痛的策略:妇女如何应付她们在性交时的疼痛存在明显的个体差异。临床上可以从不同的角度对疼痛进行描述,如被动的或主动的,回避的或倍加关注的,情绪所致的或有问题基础的。了解一位妇女如何应付性交疼痛,有助于有针对性治疗方案的设计。如果她持被动和回避的态度,那么在采取以解决问题为重点的治疗方案之前,还需要努力安排更多的前期治疗。

性交疼痛所致的对性和非性关系的破坏,以及对自尊、亲昵性、幸福感等产生的严重消极影响,是以引起痛苦为原发问题的相对“正常”的反应。发生性交疼痛并不意味着患者先前就存在其他心理问题或障碍。应该认真评价其焦虑与抑郁水平,因为它们与性交疼痛可能有相当显著的相关关系。抑郁的加重也能干扰治疗进程或对治疗的依从性。没有证据表明性交疼痛妇女就一定具有更高的性或肉体受虐史或创伤史的发生率。

性交疼痛的病因是复杂的,治疗的选择也是多样化的。心理医生或性治疗师偏重选择与心理或行为治疗有关的技术,如心理动力学治疗对疼痛的认知行为干预,夫妻治疗,催眠疗法,松弛疗法,生物反馈,性感集中治疗,耻骨尾骨肌训练,阴道扩张;而临床医生可能更关注药物的或外科的治疗如坐浴,口服药物,局部用药,干扰素注射,激光外科,前庭大腺造口或切除术等;社会学工作者则利用现今的文化习俗和道德规范去调解夫妻间不愿公开的这一矛盾因素。各种治疗方法的个案报道和非对照研究很多,但很难对其治疗效果作出结论性评价。近几年国外开始有随机对照研究,他们比较了针对会阴前庭炎患者小组的认知行为治疗、表面肌电图生物反馈、前庭大腺切除术等治疗方法的效果。生物反馈治疗等需要8~12周的时间,治疗过程涉及一系列的肌肉松弛———收缩训练,并通过仪器来监视训练状况,每天一次,每次20分钟。结果发现各组在治疗后的自我问卷打分均为有效,尤其是前庭大腺切除组,在6个月后随诊时仍然有65\%的人认为治疗成功,另两组分别为30\%和40\%。

治疗队伍应包括妇科医生,临床心理医生,性治疗师和社会学工作者等,甚至包括研究疼痛的基础专业人员。因为单靠任何一个专业的干预将很难彻底解决性交疼痛问题。让多学科齐心协力从各自专业的特点,针对疼痛及其影响的不同侧面去解决问题才能取得理想效果。治疗应直接从这三个方面努力:①减少或控制疼痛;②处理疼痛经历带来的负面影响;③重建愉悦的性生活。对于某些妇女来讲,减轻或控制疼痛基本上就能明显改善她们现有的性关系和生活质量;而对另一些妇女来讲,解决了疼痛问题后还遗留着发生疼痛之前就存在的或同时存在的其他方面的问题。有时,要在采取缓解疼痛措施之前或同时积极处理疼痛带来的消极后果(如阉割恐惧和关系困难)。对某些妇女来讲,性交时发生的疼痛伴有依赖感,失败,遭遗弃或创伤经历,有时深入的心理治疗对于顺利进行控制疼痛的治疗来讲是一个先决条件,或必不可少的步骤。所以要向患者解释清楚,以目前的医学检查手段还很难在种种治疗措施中选择出一种对她最有针对性或效果最好的措施或治疗安排。通常可向她们解释各种治疗措施的利弊和效果,而第一次则建议采取创伤最小的治疗方法,不过治疗的选择权掌握在患者手中。一般先由某位专科医生开始进行治疗,如果无效,再与其他专业人员共同合作,制定出有效的治疗方案。

事实上,这种多学科参与治疗的情况还是很少见的。国外的性治疗师往往是在治疗不成功时才找妇科医生帮助,而国内则不存在一支非医学的性治疗队伍,因此在学科合作之间应该更加便利。患者如果被推来推去而得不到及时治疗或改善,将会产生焦虑、抑郁、愤怒和挫折感。所以必须强调学科间的合作与协调,并成立专业治疗团队。

要向患者讲明性交时发生的疼痛可能是多种原因造成的,这些原因的发生时间不同,彼此间又可造成负性反馈或恶性循环。比如有些器质性因素与会阴区的神经相关,这就使会阴对触摸和摩擦十分敏感,这种原发性损伤还可引起继发感染或其他相关问题。随着患病时间的延长,周围肌肉就会处于收缩和紧张状态以努力保护这一区域。如果是短期疼痛,这种反应是积极的,有帮助的;如果疼痛时间长久,肌肉的这种反应就会使症状加重。肌肉张力的增加可以使插入更加困难,并减少流向这一区域的血流,从而减少阴道润滑和性唤起的感受。所以药物或其他医学治疗是必要的,如外科清除受损的神经。

如果疼痛持续存在,性生活肯定受影响,患者对疼痛的反应也会受到重要影响,她们会感到无法忍受,甚至说“简直让人活不下去了“,这种想法可称作阉割恐惧,她们所报告的疼痛水平自然要比实际发生的高。这种情形下自然会激发不愉快的情绪反应,如焦虑、抑郁及不断增强的紧张。这种情绪状态又会影响身体和大脑内的化学平衡,进一步增强了疼痛感。而性兴趣的减少又会反过来加重疼痛,因为患者总会很自然地预期疼痛的发生,并把注意力完全集中在疼痛感觉上而不是性感觉上。此外,伴侣对患者疼痛的反应及患者对对方可能出现的反应的担心都可以影响患者对疼痛的知觉。伴侣发起性活动的次数会减少,或在触摸时表现得更加体贴,这其实使患者更加内疚,增加了患者的紧张和压力,并进一步降低其唤起和兴趣。

接受性治疗,采取减缓疼痛的措施和给予认知干预等都可以减少患者种种消极念头,情绪,与伴侣相互作用等对性交疼痛的影响。事实上,这些干预能够使各种积极想法发挥其最大良性影响,高质量的性生活和密切的两性关系都能起到消除或减轻疼痛的作用。

以上这些解释是必要的,也能让患者更清楚地找到自己的问题所在,并动员她们投入和配合治疗。现在还不清楚究竟是个体治疗、夫妻治疗还是小组治疗对性交疼痛的缓解或治愈更有效,这恐怕要根据情况灵活掌握。另外也应该强调引起疼痛的最初原因可能已经解决(如阴道炎),但它的影响(性交疼痛)却可能持续下去,所以最初原因与临床症状可能无关。这种现象是疼痛机制的一种模式,但无论是患者还是治疗师可能都不了解这一点。所以我们要在尽力减轻或消除疼痛的同时,进一步解决后发的心因性疼痛问题。

对于性知识缺乏和消极的条件反射等引起的焦虑,可使用释放缓解交感神经紧张的药物进行治疗。一旦获得一定程度的舒适与缓解,就可以采用治疗阴道痉挛所采用的阴道松弛练习方法:即让患者在充分松弛的情况下向阴道插入手指,如果插入疼痛,就停下来继续放松,然后再插入;如果训练顺利,可尝试插入2指或3指,直至性交时也不会引起疼痛为止。由于临床发现器质性问题本身并不足以造成对性交的完全破坏,所以处理原发疾病后一个必不可少的步骤是重新安排性感集中等行为训练以解决同时存在的非器质性因素。即便存在可以治疗的器质性问题,也不能忽视心理治疗。如对于盆腔充血综合征,必要时可以适当限制盐的摄入,进行循序渐进的体能锻炼(如膝胸卧位训练),在未充分唤起时不要插入过深等,这些措施都是有益的。

由病理性因素所致的性交疼痛,专科医学对原发病的治愈程度直接影响到性交疼痛症状的改变。有些病理因素包括慢性盆腔炎、生殖道结核、易复发的感染性外阴阴道炎以及复发率较高的盆腔子宫内膜异位症等,要达到有效的治愈较为困难,因此对症状的控制和心理治疗发挥了主要作用。对物理性因素所致的性交疼痛患者,一部分通过对男女双方性知识的宣传教育、形体训练达到治疗效果,特别是首次性交由物理性因素穿破处女膜所带来的复杂心理问题,最终表现为以后的性交疼痛症状,人群发生率高,应高度重视;对生理解剖因素和文化心理的“不匹配”造成性交疼痛的治疗需要花费更长的时间,有专业的性治疗机构及心理学、社会学家的参与能起到更好的治疗效果。

在心理性性交疼痛治疗过程中常常要最大限度地减少并发的器质性因素,如使用包括局部麻醉剂在内的使交感神经紧张释放缓解的药物来对抗组织胺控制瘙痒。一旦获得一定程度的舒适与缓解就可以采用治疗阴道痉挛时经常用的阴道松弛训练方法。还应告诉患者心理因素在性反应中的重要作用,比如阴道的润滑和阴道管的扩张与延伸都不可以避免地受到心理因素的影响。如果妇女尚未唤起就发生性交,自然会给她们带来某种不适或疼痛,但大多数情况下尚不会造成严重疼痛。一旦同时存在某种局部疾患,过度焦虑之类的心理因素,或既往遭受过消极或痛苦的经历,过早过深的插入便会给女性带来疼痛和心理伤害。也可以说先有致痛的病患或消极的经历,然后对发生的问题越来越失去控制,于是性反应减弱,焦虑则不断加重,最后形成恶性循环。

性交疼痛往往代表着一种严重的心理情感障碍,虽然性欲低下也会造成性交疼痛,但更多的恐怕还是内心和人际因素的复合产物。在这种情形下,治疗显然应以性治疗为主(即心理治疗加上行为治疗),兼顾妇科检查和专科的特殊治疗,此外,婚姻治疗也有其特别的效果。从人际关系角度来说,需要检查性交疼痛给双方和婚姻关系带来的影响。所以单纯依靠标准的妇科治疗措施往往不容易收到明显的效果,至于那些声称除了性关系问题之外别的什么都好的夫妇,特别要警惕其中潜在的尚未为他们自己所认识到的其他方面的冲突,反之也需要注意整天争吵的夫妻间潜在的种种难言的性问题,争吵正是性冲突的外在表现形式,只是双方都不敢也不愿正视这一问题罢了。有关性交疼痛的主诉往往会使医生忽视或误解她们存在的其他方面的真正困难。

性交后热水浴有助于缓解直肠疼痛,但治本的办法莫过于消除或缓解肛门直肠原发疾病,它们可能是在高潮时受到压迫,并刺激痉挛而产生的,如肛瘘切除。痔疮不会产生这类疼痛,除非是栓塞性的。如上所述的疼痛可以引起复杂的心理问题,男性常常为此感到内疚或委屈,而对女性的心理影响也是很明显的。最重要的是找出和消除病因,不要低估严重的不适和疼痛。可以安慰患者,讲明问题是容易得到纠正和改善的。

一旦尖锐耻骨缘得到明确诊断,只能尽可能地选用一些减少对锐性耻骨造成直接压力的性交体位,如女方用枕头垫高臀部或让女方在男上位时抬高双腿,有助于阴道口抬平、扩张和暴露,疼痛将趋于缓解;或采取更多的非性交刺激技术来解决双方的性需求。因为解剖变异无法克服,能够改变的只是开发更多替代的性刺激技术。这些措施可能短期起效,但仍需不断努力并变换措施以解决好这一长期问题。


\section{第五节 个案报告}

女性,25岁,已婚1年,因“婚后性交疼痛,渐进行加重,中断性生活2个月”到妇科就诊。婚前无性生活史,月经周期、经期时间及经量均正常,12岁月经初潮,2年后开始出现痛经,经前1天和经期第1天为重,第2天开始明显缓解,以后痛经呈渐进性加重,经痛时间延长,疼痛加重,由其母亲带至医院妇科就诊,诊断为“痛经”,给予对症治疗处理,痛经好转。婚前两年开始出现经期延长,经量增多,长期服用痛经镇痛等对症药物,疗效欠差,再到妇科门诊,B超检查发现右侧附件囊性包块约3+ cm,诊为“右卵巢巧克力囊肿”,继续服用镇痛药及中成药治疗控制症状。

恋爱2年后与男友结婚,受到丈夫倍加爱护,婚后第一次性生活后不久,发现性交时下腹疼痛,在丈夫用力和射精时明显加重,性交完成后疼痛持续5至10分钟,性交中未能得到欣快感,随后逐渐减少性交次数,尽管丈夫细心呵护,性交疼痛仍未能得到明显改善,内心痛苦,不愿自己的问题影响到丈夫的性满足,观念保守的她在性生活中主动采用口交,帮助其手淫等方式来代偿其因性交疼痛而产生的不愿性交状态,此次因自感心理负担严重再次就诊于妇科门诊。根据病史、妇检与B超检查诊为“盆腔子宫内膜异位症,双侧卵巢巧克力囊肿”,建议其行腹腔镜手术治疗。

随后采纳妇科医生的建议,住院行腹腔镜诊治,术中剥除双侧卵巢巧克力囊肿,分离盆腔粘连及输卵管粘连,术后采用药物连续治疗2月控制其复发。术后3月恢复阴道性交,感性交痛有减轻,痛经有明显缓解,但仍没有欣快感及性高潮的产生。对此感到十分痛苦,希望得到专业机构的治疗,以获得性快感及性高潮。

对于这种类似的病例,目前常见于妇科门诊。盆腔子宫内膜异位症就以性交疼痛、不孕和月经改变为其主要症状表现,不管采用手术治疗、药物治疗或联合治疗方案,其轻症患者复发率约为37\%,重症患者复发率仍达到了74\%。绝经后可自愈,可采用手术切除较大病灶加子宫切除、一侧或双侧卵巢切除的方法来达到完全控制症状,显著降低复发率的效果。但手术治疗、抑制雌激素产生的药物治疗方法均对正常性功能有不小的影响,尽管其或多或少的缓解了性交疼痛,但治疗过程中必然对卵巢产生了损伤和功能抑制,又不可避免地发生了新的性功能障碍,包括性唤起障碍、心理性或器质性性欲望障碍等,仍然不能达到治疗的目的。因此,这类由严重的病理性因素所致的性交疼痛在最后都演变为混合性因素所致的性交疼痛,应得到性治疗机构主诊,在如何治疗器质性病变方面与妇科医生联合会诊,制定一个合理的治疗方案,以求取得最佳治疗效果。当通过手术和药物治疗不能完全缓解症状时,镇痛治疗和心理治疗变成了治疗的主体。外周性止痛剂包括阿司匹林、非类固醇抗炎药和对乙酰氨基酚、中枢性止痛剂等。极少数严重患者需要采取神经破坏性疗法,包括骶前神经切除术、神经毒性化学物质的注射或应用足够的能量(热能、激光等)破坏神经组织等方法。心理治疗中的认知疗法及对患者的放松训练也能得到不错的效果。

“我今年36岁。26岁时结婚,一年后丈夫病逝,几个月后因梦到与他性交并达到性高潮,感觉非常舒服,此后,不知不觉中我染上了手淫,大约每周一次,起到了缓解性要求的作用。”当时我认为在没有正常性生活的情况下,适当手淫不但不会损害身体反而对身体有益,在情绪上也有安抚作用,所以就没有进行控制。这样过了将近两年的时间,我的手淫方式主要是刺激阴蒂,没动过阴道。随着时间的推移,我开始出现了一些症状,即每次手淫开始时大阴唇及会阴部酸胀,性高潮小腹中间(子宫处)有轻微疼痛,因症状越来越重,手淫次数减为每月一次。以后发展到一有性幻想或看到书中有关描述时,小腹中间就产生痉挛性疼痛,并牵扯着两侧的附件痛,如果马上终止性幻想或停止看书,疼痛也很快缓解。如果性兴奋到一定程度,又通过手淫达到性高潮时,几乎总会伴有子宫收缩及双侧附件的牵扯疼痛,并放射到肛门,疼痛时不知手放在哪里好,恨不得插进肛门,大约经过20分钟疼痛慢慢缓解。

4年前,我再婚了,原指望正常的性生活会使我好起来,但结果让我失望。如果在性交中我不动情,只是应付应付,即使他抽动和射精,我一点也不疼,只要我一有性幻想,性兴奋,马上就会出现上述疼痛。几年来,我进行过内诊,B超等检查,没有任何阳性发现,不知服用过多少药,做过针灸、封闭,但症状丝毫没有减轻,而且性生活间隔时间越长,疼痛越重。婚后我未手淫,丈夫很有经验,几乎每次都能使我达到高潮,但却是快感与疼痛同在,我既有性欲望,又害怕性生活,难道我的后半生就只能这样了吗?就在我写信时,因为想到性交,小腹及两侧又不时有不适和疼痛感了,渴望医生能给我解开疑团,盼您早日复信。”

这种性交疼痛显然属于心理性的,而非器质性的,它与手淫史本身无关,但又系对手淫的错误观念产生的消极心理因素带来的躯体症状。此外她的临床检查并无任何阳性发现,经心理暗示好转。可以说,她的问题属于操作性条件反射对性行为的消极影响。下面分析一下这里的奥秘:

我们说条件反射不是与生俱来的,是个体经后天学习获得的。例如向动物同时提供两种刺激,一种是非条件刺激如食物,一种是无关刺激如灯光或铃声,二者多次结合后,单独给灯光或铃声也可引起动物分泌唾液,灯光或铃声便成为一种信号刺激或条件刺激。动物实验形成了对无关动因(条件刺激)的条件反射(分泌唾液),这种形成条件反射的方式是经典的,它是使个体更好适应环境、与环境取得平衡的重要基础和条件。下面要介绍的则是与之不同的操作性条件反射,它与人类性活动的关系密切。操作性条件反射是另一种学习过程,其特点是用奖励性的手段来强化某种反应方式。这种学习往往是学会一种操作过程,故称操作性条件反射。它指的是通过一些措施来改变行为发生的频率,给予奖赏(强化剂)就使得将来的行为频率增高,给予惩罚就使得将来的行为频率减少。如果一个人在实施某种特定行为(操作)后获得奖励,那么他以后还愿意再做;如果受到惩罚,这个人就不愿意再干下去了。如果某一种行为反复得到奖励,它就会频繁重复;反之若受到惩罚,这一行为就不会频繁发生,以致逐渐消失。奖励就是一种原始的强化剂,也就是说有的事情本来就有好处。食物就是这样一种强化剂,性行为则是另一种强化剂。例如:当迷宫出口处有一只动情的性伴侣时,老鼠可以更快地学会穿过迷宫。所以性行为对学习理论来说有两方面的作用,它本身就是一种奖励,它也可以是一种受奖励或受惩罚的行为。

另一种奖励则是条件强化剂。如果你做了某件工作而得到100元报酬,这不是条件强化剂,只有通过学习或条件反射使你认识到得到这张货币是值得的并感觉到它是一种奖励才行。另一个范畴是社会强化剂,如由于工作出色而得到来自领导或社会的赞扬与尊敬。操作性条件反射理论的简单原则有助于解释性生活方面的某些现象。例如,一位妇女在性交时总是出现疼痛(如阴道炎所致),她也许就会减少或完全回避性交。按照操作性条件反射理论的原则,性交疼痛的惩罚性质使性行为减少。操作性条件反射的其他哪些原则对理解性行为有用呢?一个原则是一次行为之后如果立即发生又一次行为,则有较强的强化作用;一次行为之后如果延迟另一次行为,则无强化作用。这个原则的一个例子就是前夫去世后连续的手淫行为,不顾被他人发现而受鄙视和自我的谴责。既然她的这种行为有可能受到惩罚,为什么她还要坚持干呢?原来,每当她手淫时总感到愉快,发现有乐趣时又立即重复发生,这种眼前的乐趣便刺激她一直坚持干下去;而来自他人和自我的谴责或惩罚则迟迟没有发生,因而它没有有效地遏制这一行为。但是随着时光的推移,她受手淫有害观念的影响渐渐发生作用了,从潜意识里给她带来不良反应,因此手淫时或性幻想时伴有疼痛也就容易理解了。

操作性条件反射理论中的另一原则是,与奖励相比惩罚对行为的塑造作用较小。例如,有的年轻人婚前性行为受到不适当的惩罚,这种惩罚反而促使他继续偷偷摸摸地保持这一性关系。又如当大人采取惩罚措施来制止孩子手淫时,往往适得其反,大多数孩子仍会继续手淫,而且还学会在不会被捉住的情况下进行手淫。所以在纠正某些不良习惯时决不应一味惩罚,相反鼓励他们的点滴进步却可以收到良好成效。其实只要她能正确认识手淫问题,鼓励她积极参与性活动,她的操作条件反射手淫和性交疼痛就会逐渐消失。

我在妇科门诊中,不时会遇到被强奸的女性,她们有的是来检查强奸后身体的损伤(并不准备告发),有的则要求在阴道里取精虫检查等取证(可惜都被我建议去报案后由相关司法部门来调查取证),直到建立了女性性学咨询门诊后,才有了机会和时间来详细了解女性被强奸后造成的性功能障碍和性心理障碍等。一个面临高考的女生,周末的一天独自在家,因有数学题解不出来,便邀请同班的一个男同学到家指导,在相处的过程中,性压抑的男生开始触摸该女生的暴露部分,在遭到拒绝和反对后,不顾女生的反抗,强行剥光女生衣服,女生当场被突如其来的状况吓昏了,男生对其实施了强奸后离开。之后该女生说:“我的外阴红肿疼痛了8天,阴道流血5天才干净,而由于害怕,没有去医院诊治,也没有去控告该男生。从此,我不再相信任何人了,甚至包括我自己。我很痛苦,感到非常恐惧,我在任何地方都不能感到安全。我恨我的卧室,这个地方曾经是我觉得最安全的地方,但就是在那个地方,我被强奸了。我恨他强奸我,在我的床上强奸我。我没能考上大学,而且在那张床上,我又睡了四年。后来找到工作,恋爱后搬出去的时候,我没有带走那个房间里面的任何东西。所有的美好记忆都化为乌有,我远离了伤害自己的地方。”

恋爱期间,当男友接触到她身体时,她就感到害怕,更不愿生殖器的接触,男友对其行为又喜又恨,但仍细心地呵护,并不强行与其发生任何性接触。这样的恋爱在持续两年后结婚了,丈夫对妻子的性接触小心翼翼,作为妻子的她在婚后的性生活中感到非常痛苦,每当性交前会害怕,性交时会感到外阴阴道疼痛,不能享受性快乐,性交后总感到外阴阴道红肿,有损伤,因此常到妇科门诊,希望医生能够治愈其性交疼痛。

世界卫生组织对强奸的描述性定义为:“使用身体的强迫或其他强制方式,用阴茎、身体的其他部位或物体对阴部或肛门的强迫插入,不论程度如何。”这个女生在被强奸后,出现创伤后应激障碍(PTSD)中的症状,在其他强奸案中女性受害者所出现的症状相类似,大多数在婚后也会出现或多或少的性功能障碍,只是症状的程度不同。随着时间的推移,大多数PTSD的症状会自行缓解或自愈,而随后出现的性交疼痛症状却成为生活的痛苦和障碍,难以自愈。从症状的发病因素来分析,可以明确看到,强奸伤害的开始就伴随着身体伤害(主要是处女膜破裂出血,暴力致外阴阴道的擦伤甚至撕裂伤等物理性伤害)、心理伤害,以及随后可能出现的生殖道感染所致的器质性病变等混合性因素。当事件发生后,及时的关怀和专业的创伤性医学治疗,心理诊治PTSD,以及性观念的改变是防止以后性交疼痛症状发生的最积极的预防措施,这样的机构需要多学科人才参与合作。当妇科医生检查这个主诉性交疼痛女性的身体时,并未发现所述的病理性状态,因此容易确定其为心理性性交疼痛,但治疗往往较为困难。通过对这个患者较长时间的专业心理咨询,采用认知治疗和阴道性交恐惧性疼痛的放松训练等行为治疗,最终治愈了患者,但其进一步的性愉悦性体验还需家庭的帮助配合,以促进其享受性健康的快乐。

(刘云)


\chapter{第十三章 药物与性}

药物与男女性之间存在着密切的关系。自古以来人类就发现药物可以影响性功能,并不断寻找那些能够明显改善或增强性功能的药物,如:古时的“春药”、现代的助性剂等。由于个体性反应的差异以及药物作用的复杂机制,始终未能找到一类既安全又普遍适用的性兴奋剂。相反,随着各类药物在疾病治疗中的广泛应用,以及对其药理作用的深入了解,人们逐渐认识到临床上许多常用的药物可能对人的性功能产生不良影响,严重者不但影响患者的性生活质量,而且在一定程度上会影响到患者对药物治疗的依从性。因此,充分了解药物对性功能的不良作用,在临床治疗过程中既能充分发挥药物的有效治疗作用,又能尽量避免不良反应的发生,对于提高药物治疗效果和患者依从性具有重要意义。

正常的性功能与个体的激素水平、神经冲动、精神状况以及生殖器血流的调节等密切相关,任何一个环节发生改变,均能对性功能产生不良影响。药物可以从以下几个方面对性功能产生影响:①对中枢神经系统起镇静或抑制作用;②增高患者血液中泌乳素水平;③直接拮抗患者体内性激素的作用;④影响患者脑内儿茶酚胺系统和5-羟色胺(5-hydroxytryptamine,5-HT)系统;⑤抗胆碱作用及抗交感作用等。

需要引起注意的是,药物对性功能影响的机制相当复杂,与摄入药物的种类、剂量、个体对药物的反应等多种因素有关,因此某类药物具体作用往往因人而异。同一类药物可以影响到性反应的不同环节而产生不同的症状,有时甚至是相反的效果。例如:小剂量的雄激素可以改善因激素水平低下所致的男性性功能障碍,而大剂量的雄激素摄入体内后,因其抑制了下丘脑-垂体-性腺轴,反而会引起或加重性功能障碍。一些抗抑郁药可引起男性性欲下降、射精延迟,但在合适剂量下可用于治疗男性早泄,反而可以取得良好疗效。

由于男性性功能改变较女性更容易进行观察和评定,因此目前药物对性功能影响的研究多局限于男性。


\section{第一节 有可能降低性欲或削弱性功能的药物}

高血压病是危害人类健康的常见病。流行病学调查显示,我国2002年的患病率已达到18.8\%,估计全国约有1.6亿高血压病患者,并且有逐年上升的趋势,由高血压病引起的心脑血管疾病在一些城市和地区已成为居民死亡的最主要原因。高血压病中90\%以上为原发性,具体发病机制尚未完全明了,这意味着高血压病目前仍是一种只能控制症状而无法彻底治愈的疾病,患者往往需要终身服药。

抗高血压药物种类繁多,药物的作用范围涉及中枢神经、自主神经节、神经末梢、神经递质受体、血管平滑肌等多个部位,这些部位均与性功能的调控有密切关系。加之高血压患者多为老年人,常常合并其他疾病需要同时服用多种药物,因此对判断高血压患者的性功能下降和所服降压药的关系造成困难,临床上要注意具体分析。

临床上常用的药物为可乐定和甲基多巴,通过作用于中枢抑制外周交感神经的活性,从而达到降压效果。可乐定可以激动延髓α2 肾上腺素受体,强烈抑制血中儿茶酚胺和胰岛素的作用,在降压的同时可导致血糖升高和糖耐量降低,进而影响性功能。服用可乐定的男性患者中,约10\%~20\%发生勃起功能下降或性欲减低。甲基多巴与可乐定具有相似的中枢降压机制,其对性功能的抑制作用与药量成正比,随着用药量的增加,性欲减低和勃起功能下降的发生率也逐步增加。当每日剂量≥2.0g时,会有一半的患者出现显著的性功能下降,表现为男性性欲降低、射精延迟,女性性高潮丧失和性兴奋降低。同时,甲基多巴能增加体内催乳激素水平,故有可能导致男子出现女性型乳房或溢乳。

常见药物有美加明、安血定等,通过阻滞交感神经节,扩张外周血管达到降压目的。由于腹胀、尿潴留等药物副作用发生率较高,现已不常应用。该类药物明显抑制了交感神经系统,故性欲降低、射精困难等性问题的发生率很高。

主要有利血平和胍乙啶类药物,作用机制是耗竭中枢和周围神经末梢中的儿茶酚胺和5-HT,或抑制其吸收与释放。利血平对性功能的影响有以下三个方面:其一,通过耗竭大脑中的兴奋性神经递质儿茶酚胺,从而具有较强的镇静作用,甚至可使高血压患者产生抑郁表现,因此会降低性欲,即使剂量相当小,对性欲的影响也很明显。其二,中枢神经受到抑制后,阴茎海绵体内的肾上腺素能β2受体功能受抑,从而引起勃起功能障碍。长期使用利血平的高血压患者中,有30\%~40\%出现阴茎勃起能力下降或不能勃起。其三,利血平具有潜在的内分泌样作用,可升高体内泌乳素水平,长期使用会引起男子女性型乳房和溢乳。胍乙啶能阻滞交感神经末梢释放去甲肾上腺素,引起男性射精抑制。胍乙啶对性功能的抑制程度与药物剂量有关,每日服用剂量超过25mg的男性患者,一半以上可出现射精延迟或不能射精,约25\%同时出现勃起功能障碍。胍乙啶的同类药物苄甲胍、异喹胍、胍生与胍氯酚等对性功能的副作用与胍乙啶类似。

β2 受体阻滞剂是治疗高血压及心律失常最常用的药物之一,通过阻滞交感神经对心脏的激动作用达到降低血压和恢复正常心率的作用。普遍认为β2 受体阻滞剂对性功能影响很小。临床应用较为广泛的β2 受体阻滞剂普萘洛尔(心得安),每日剂量低于160mg时,一般不会引起性功能障碍;每日剂量超过320mg时,少数高血压患者出现阴茎勃起功能障碍。β2 受体阻滞剂引起性功能下降的作用机制目前尚不清楚,可能与干扰肾上腺素能神经活性、减少阴茎血液供应有关。其他β2 受体阻滞剂,如吲哚洛尔、阿替洛尔、拉贝洛尔等药物,偶而也有引起男性性欲减退、阴茎勃起功能障碍的报道。

此类药物通过直接松弛小动脉壁内的平滑肌而扩张血管,达到降压目的。代表药物是肼苯哒嗪(肼屈嗪)。肼苯哒嗪对性功能影响较小,除非服用剂量相当大,一般不会引起性功能损害。每日口服剂量超过200mg时,有少数患者出现性欲减退,有时可伴有勃起障碍。停药后,阴茎可以恢复正常勃起。因此,当高血压患者在治疗过程中服用其他降压药影响性功能时,可以考虑换用肼苯哒嗪。

利尿剂通过降低血容量达到降压目的。利尿剂对性功能的影响与剂量和用药时间密切相关。长期应用噻嗪类利尿剂均可出现不同程度的性功能障碍。

噻嗪类排钾性利尿剂具有引起血糖升高的副作用,而高血糖对性功能有明显影响。部分患者的性功能下降也可能与药物所致的低血钾有关。血钾浓度降低,导致神经肌肉电活动异常,诱发勃起功能障碍。当低血钾得到纠正后,患者的性功能也会随之恢复。

保钾利尿剂中的螺内酯具有对抗雄激素的作用,可抑制与合成雄激素有关的细胞色素P450 酶活性,降低体内雄激素水平,并抑制雄激素在外周的作用。另外,螺内酯具有拟黄体酮样作用,容易引起男性性欲减退和勃起功能障碍,对于女性则可出现月经失调。螺内酯对性功能的影响也与剂量有关,用药量越大,男性性欲降低和勃起功能障碍、女性月经紊乱的发生率就越高。当停止用药后,症状可得到迅速纠正。

精神失常是指由于各种原因导致的精神活动障碍,包括精神分裂症、焦虑症、抑郁症和躁狂症。根据治疗目的,抗精神失常药可以分为:抗精神分裂症药、抗焦虑药、抗抑郁药和抗躁狂药。抗精神失常药对患者的性功能具有不同程度的影响,能导致患者治疗依从性下降,增加精神失常的复发率。因此,避免和纠正抗精神失常药引起的性功能障碍,将有助于提高患者的药物依从性和治疗效果。但对于精神失常患者的性功能障碍评价存在很大难度。首先,许多精神失常患者存在认知障碍,医患沟通困难;其次,在鉴别患者的性功能障碍究竟是原发精神疾病的症状之一,还是药物所致的副作用常有一定困难。例如,抑郁症患者由于情绪低落常伴有性欲低下、勃起功能障碍、性兴奋降低等表现,随着精神疾病的好转,患者性功能障碍也会有所改善;而抗抑郁药有时也会降低患者的性功能。临床上对此类现象要注意区分并作出相应的处理。抗精神失常药物可通过多种机制影响患者的性功能,但药物所引起的性功能障碍大部分是可逆性的,停药、换药均可取得较好效果。

精神分裂症是一类常见的精神疾病,存在认知、情感、行为等多方面的障碍和精神活动与环境的不协调。精神分裂症的病因未明,被认为与遗传、心理、大脑结构改变和神经生化异常等多种因素有关。目前神经递质假说受到广泛接受,例如多巴胺(dopamine,DA)功能亢进假说、5-HT代谢障碍假说等,常用的抗精神分裂症药物与这些递质或受体的功能密切相关。

典型抗精神病药按照分子结构分为三类。分别为:以氯丙嗪为代表的吩噻嗪类药物,以氟哌啶醇为代表的丁酰苯类药物,以及以泰尔登为代表的硫杂蒽类药物。这些药物主要通过作用于大脑的多巴胺受体,阻断多巴胺的功能而发挥抗精神病作用,能有效地缓解精神分裂症幻觉、妄想、兴奋冲动等症状。

又称冬眠灵、可乐静、氯普马嗪,是目前应用最广泛的一种抗精神失常药,对其药理机制和不良反应的了解也较为深入。氯丙嗪可作用于机体多个系统,产生多种多样的药物作用。对于中枢神经系统,可通过阻断中脑-边缘系统以及中脑-皮质通路中的多巴胺受体而发挥强大的抗精神病作用,用药者产生安定、镇静、感情淡漠等效果;对于植物神经系统,氯丙嗪具有明显的阻断α受体的作用,可扩张血管、降低血压,并且有轻度的阻断M受体的作用,出现口干、排尿困难、便秘及视力模糊等症状;对于内分泌系统,氯丙嗪可以阻断结节-漏斗通路中的多巴胺受体,减少下丘脑和垂体中泌乳素抑制因子、促性腺激素的释放,导致机体泌乳素分泌增加,引起高泌乳素血症。

氯丙嗪对性功能影响明显,在治疗剂量时患者即可出现性功能障碍,而且性功能障碍的表现多种多样。例如,患者最常出现的性功能障碍为性欲低下,与氯丙嗪强大的中枢镇静作用有关。勃起功能障碍的发生与服药剂量有关,当每日剂量达到400mg以上时容易出现,停药后可很快恢复正常。机体血液中泌乳素水平过高,可直接抑制性功能,并且高泌乳素能抑制女性排卵,引起月经不调和溢乳,男性患者则出现乳房女性化以及睾丸缩小等副反应。此外,氯丙嗪具有轻度抗组胺作用,可引起性兴奋时阴道分泌不足,润滑性下降,降低性交时的快感。

亦属于吩噻嗪类抗精神分裂症药物,是最早被报道的可影响性功能的抗精神病药物,服药者性功能障碍的发生率很高,可达一半以上。男性现自杀倾向。抑郁症在普通人群中的发病率约200/10万每年。抑郁症的病因学尚不清楚,但有大量研究资料显示遗传因素、神经生化因素、心理因素对抑郁症的发生有明显的影响。其中,有关中枢内单胺类神经递质(5-HT、去甲肾上腺素、多巴胺等)发生变化和相应受体功能改变的假说最受重视,大部分抗抑郁药是以此学说为依据筛选出来的,而且药理作用和副作用也基本类似。按照化学结构和药理作用特点,抗抑郁症药物可分为四类:①三环类抗抑郁药,包括在此基础上开发出来的四环类和杂环类抗抑郁药;②单胺氧化酶抑制剂,例如异烟肼、苯乙肼、苯环丙胺等;③选择性5-HT再摄取抑制剂(selective serotonin reuptake inhibitor,SSRI),例如米胺色林、曲唑酮、氟西汀等;④作用于其他递质的抗抑郁药,如色氨酸等。前两类属于传统抗抑郁药,后两类为新型抗抑郁药。

与抗精神分裂症相似,对抑郁症患者性功能障碍的评价是一个颇为棘手的问题,常难以鉴别患者的性功能障碍到底是疾病本身的症状还是抗抑郁药所致的副作用。

是最早用于治疗抑郁症的药物,疗效肯定。临床常见的有丙米嗪、去甲米嗪、阿米替林、多虑平等。这些药物结构中都有2个苯环和1个杂环。药理作用上都属于单胺摄取抑制剂,通过阻断去甲肾上腺素和5-HT的再摄取,减少这些递质被再摄取进入神经元末梢降解,从而增加了突触间隙递质的浓度,长期应用也可降低受体的敏感性。除了阻断肾上腺素和5-HT再摄取起到治疗作用外,三环类抗抑郁药还具有M1 、α1 和H1 受体阻断作用,可导致口干、便秘、视物模糊、体位性低血压、镇静、嗜睡、性功能障碍等副作用。性欲低下、射精障碍是服用丙米嗪等三环类抗抑郁药后常见的性功能方面的副作用,与此类药物具有明显的抗胆碱作用有关。除此以外,三环类抗抑郁药还可影响机体的内分泌系统,导致男性乳房发育和睾丸胀痛、女性乳房肿胀和溢乳。三环类抗抑郁药对患者性功能的影响与服用剂量呈正相关,减量或撤药可降低不良反应的发生率。

属非三环类抗抑郁药,如异烟肼和异丙肼。早期曾主要用于治疗结核,后来发现它们具有抗抑郁作用。其药理机制为抑制单胺氧化酶的活性,减少单胺类神经递质的降解,使突触间隙的5-HT、去甲肾上腺素等神经递质的浓度增加,从而产生抗抑郁的作用。由于受饮食和其他药物影响大,高血压和肝功能损害等严重不良反应发生率较高,故目前临床应用并不广泛。服用单胺氧化酶抑制剂的男性中有近30\%发生射精延迟或不射精,10\%~15\%的男性出现勃起功能障碍。停药后,不良反应即可消失。

是一类新型抗抑郁药,通过选择性抑制突触前神经元对5-HT的再摄取,从而增加突触间隙中的5-HT含量。其不良反应较少。SSRI对性功能的影响在于它明显增加了突触间隙中5-HT的浓度,可引起男性勃起功能障碍、射精延迟、不射精以及性高潮障碍等。近年来,利用SSRI延缓射精的副作用来治疗早泄,取得了良好的效果。

抗焦虑药是指用来减轻焦虑、缓解紧张和恐惧、稳定情绪的药物;镇静催眠药是指能引起镇静和近似生理性睡眠的药物。在实际应用中,这两类药物效果具有交叉性,所以常在同一章节内进行介绍。随着非苯二氮 类抗焦虑药和新的镇静催眠药的不断问世,这两类药在精神科领域有进一步区分的趋势。与抗精神失常药和抗焦虑药类似,临床上评价抗焦虑药和镇静催眠药对性功能的影响存在一定的难度,有关这些药物在性功能方面的副作用,目前尚缺乏系统性的研究。

是苯二氮 类药物的典型代表,其主要作用部位在调节情绪反应的边缘系统,通过作用于苯二氮 类受体,抑制大脑边缘系统中海马和杏仁核神经元电活动的发放和传递。安定的抗焦虑作用效果较好,小剂量即可明显改善患者恐惧、紧张、忧虑等症状。随着使用剂量的增加,安定具有镇静和催眠作用,可明显缩短入睡时间,显著延长睡眠持续时间,减少睡眠中觉醒的次数。若剂量进一步加大,则产生抗惊厥、抗癫痫及中枢性肌肉松弛作用。通常安定可通过减轻焦虑症状而改善焦虑患者的性功能,但在大剂量情况下会引起勃起功能障碍,偶有服药者出现溢乳、月经不调、排卵障碍等副反应。

是20世纪50年代最常使用的镇静催眠药物,其中枢抑制作用与激活GABA受体、阻断谷氨酸作用于相应的受体有关。小剂量巴比妥类药物可镇静情绪,缓解焦虑、烦躁不安的状态;中等剂量可缩短入睡时间,减少睡眠中觉醒的次数和延长睡眠时间;大剂量对心血管系统有明显的抑制作用;过量可因呼吸中枢麻痹引起死亡。由于这类药易产生耐药性及药物依赖性,现已很少使用。与苯二氮 类药物相似,巴比妥类药可通过镇静、抗焦虑作用解除性抑制状态,从而改善服用者的性功能。但是临床应用中发现此类药物导致服药者性欲减退、勃起功能障碍或性高潮障碍等性功能障碍的现象更常见,原因可能与巴比妥类药物抑制垂体促性腺激素的释放,促进肝脏对血中睾酮和雌二醇的灭活有关。

肾上腺是人体最重要的内分泌器官之一,分为肾上腺皮质和髓质两部分,肾上腺皮质由外向内依次分为球状带、束状带及网状带三层。球状带合成醛固酮等盐皮质激素,束状带合成氢化可的松等糖皮质激素,网状带主要合成性激素。髓质合成肾上腺素和去甲肾上腺素。临床上所谓的皮质激素通常系指糖皮质激素。本节主要讲述糖皮质激素对性功能的影响。

无论是基础分泌还是应激时的分泌,机体的糖皮质激素均受腺垂体产生的促肾上腺皮质激素(adrenocorticotropic hormone,ACTH)调控,ACTH的分泌又受到下丘脑促肾上腺皮质激素释放激素(corticotropin releasing hormone,CRH)的调控,糖皮质激素反过来对下丘脑和垂体分泌CRH、ACTH具有负反馈作用。因此,下丘脑、垂体和肾上腺皮质组成一个密切而又协调的功能活动轴,维持着机体糖皮质激素浓度的相对稳定和在不同状态下的适应性变化。

糖皮质激素对机体的作用广泛而复杂。在生理剂量时主要影响机体对物质的代谢,可促进糖异生,减少机体对葡萄糖的利用,促使血糖升高;促进机体蛋白质分解,导致肌肉消瘦、皮肤变薄;促进体内脂肪重新分布,使机体出现四肢消瘦、躯干发胖的向心性肥胖体形。缺乏糖皮质激素时,将引起代谢失调甚至死亡。当应激状态时,机体分泌大量糖皮质激素,通过允许作用等,使机体适应内外环境的强烈变化。超生理剂量应用时,糖皮质激素除影响物质代谢外,还有情绪改变、抗炎、抗休克和免疫抑制等作用。

由于糖皮质激素在临床上的广泛应用,它的副作用,包括对性功能的影响受到广泛关注。尽管目前缺乏糖皮质激素对性功能影响的针对性研究,但已有其对生殖和性功能的负面影响的报道,包括其引起男性的性欲降低、勃起功能障碍、女性的月经异常,甚至闭经等。这些副作用的出现,一方面是因为滥用皮质激素可诱发糖尿病、增加泌尿生殖系感染机会、引起精神抑郁甚至紊乱;另一方面,大量外源性皮质激素进入体内后可抑制下丘脑和垂体功能,干扰了下丘脑-垂体-性腺轴,直接或间接降低了体内性激素水平,对患者性功能产生负面影响。皮质激素对性功能产生影响的剂量因人而异,有些患者局部应用皮质激素软膏治疗皮肤病也可能发生性功能障碍。需要注意的是,在分析应用皮质激素患者的性方面的有关问题时,必须考虑患者的原发病对性功能的影响,因为慢性病本身以及由慢性病产生的焦虑和抑郁状态都会对性功能产生不良影响。

主要由睾丸间质细胞合成,肾上腺及卵巢也可合成少量雄激素,这些少量雄激素在正常情况下无生理意义。雄激素属于甾体类激素,主要有睾酮、双氢睾酮、脱氢异雄酮和雄烯二酮,其中以双氢睾酮活性最强,其次为睾酮,其余的雄激素活性都很弱。睾酮主要在肝脏内被灭活,少量睾酮在睾丸内被支持细胞转化为雌激素。

睾丸间质细胞分泌睾酮受到下丘脑-垂体的调控,睾酮对下丘脑-垂体又具有负反馈调节,并且在睾丸内部也存在着复杂的局部调节,通过一系列的调节维持着机体全身和睾丸局部的激素水平稳定。

雄激素对体内多个系统都有重要影响。雄激素可促进蛋白质合成,使肌肉发达;促进骨基质合成、钙盐沉积增加,使骨骼生长;刺激骨髓生成红细胞、增强机体的免疫能力和抗感染能力,并促进神经系统的发育和成熟;在雄性生殖系统及第二性征的发生发育中起着决定性作用;在性分化过程中,只有在胚胎睾丸分泌的雄激素刺激下,胚胎生殖器才会向雄性方向发展,若缺乏雄激素,则生殖器官向雌性方向发展;从青春期开始,雄激素对睾丸、精囊腺、阴茎、阴囊等雄性器官的进一步发育产生直接刺激作用,并且个体出现喉结增大、声带变厚、体毛增多等第二性征。若青春期前缺乏雄激素,机体性器官会始终处于幼稚状态,青春期后缺乏性激素,则已发育的睾丸、附睾、前列腺、阴茎、阴囊等性器官会发生萎缩。

雄激素对于促进和维持雄性功能是必不可少的。幼小动物切除睾丸后,由于缺乏睾酮,会始终处于性幼稚阶段,不会出现性行为;成年者切除睾丸,性欲和性能力丧失的发生率较高,但部分患者仍会有性行为,可能与肾上腺分泌的少量雄激素有关。

需要注意的是,对于血浆睾酮低下导致的性欲低下等性功能障碍的男性患者,补充适量的雄激素可以取得良好的效果,使其性欲和性功能恢复正常;血浆睾酮含量正常的男性,雄激素并不能增强其性欲和性功能。正常男性长期大量应用反而会出现副作用,原因在于大量雄激素除干扰下丘脑-垂体-睾丸轴外,在睾丸内被转化为雌激素的量也相应增多,可导致睾丸萎缩和男性乳房女性化,停药后一般可以恢复。有报道称正常人给予睾酮后,可有性欲增强的感觉,但经严格设计的研究并未证实这种现象,因此受试者性欲增强的感觉可能是由于心理作用所致。应用雄激素的另一个危险是引起前列腺增生,或者加重前列腺癌。女性应用大剂量雄激素,可增强性欲,但多毛、痤疮、阴蒂肥大和尿潴留等副作用带来的负面影响,常抵消其治疗效果。如果怀孕早期服用雄激素,则存在女性胎儿男性化的危险。

雄激素的疗效与其使用方式密切相关。天然雄激素口服易被消化道酶降解或吸收入血后被肝脏破坏,故口服无效。一般将其油溶液肌肉注射或植入皮下。人工合成的睾酮衍生物,如甲基睾酮不易被肝脏破坏,口服有效。民间常将雄激素或雄性动物的生殖器官作为健康人的“壮阳药”、“助性剂”,这种做法是没有根据的。

具有对抗雄激素的作用。按照分子结构中是否含有类固醇结构,雄激素拮抗剂分为甾体类抗雄激素和非甾体类抗雄激素药两大类。安宫黄体酮和氟化酰胺分别为两大类雄激素拮抗剂的代表药物,它们通过减少雄激素生成、阻断雄激素前体活化为活性成分、竞争性结合雄激素受体、反馈性抑制性腺轴等方式拮抗雄激素,致使男性性欲减退、勃起功能损害和性高潮障碍。临床上常用来治疗性欲亢进。

天然雌激素主要是雌二醇。它在女性卵巢中合成,经肝脏代谢生成雌酮和雌三醇。应用最广泛的人工合成的雌激素是己烯雌酚,男性肾上腺可产生少量雌激素。此外,睾丸中部分雄激素可在支持细胞的作用下转变为雌激素。

雌激素对于促进雌性生殖器官发育成熟具有重要意义。当女性缺乏雌激素时,将会发生子宫及阴道上皮萎缩、阴道分泌功能下降,导致性交时阴道干涩、性交不适。雌激素对于雌性动物有催情、增强性欲的作用,但对于人类,此项作用似乎并不明显。如果男性患者摄入过多雌激素,由于雌激素可抑制睾酮的生成,可很快出现性欲下降、勃起功能障碍、射精功能损害和精液量减少等症状。

以阿托品为代表,临床常用来解除平滑肌痉挛、抑制腺体分泌、散瞳、提高心率、改善微循环和解救有机磷农药中毒。应用该类药物后,女性常常出现阴道分泌功能下降、阴道润滑性减弱、性兴奋障碍等副作用。

可作用于突触后膜的N胆碱受体,产生阻断神经兴奋传导、肌肉松弛的作用。应用此类药物的男性,由于植物神经受到干扰、阴茎血管反射性舒张受到抑制,可出现勃起功能障碍。

以苯海拉明、异丙嗪等为代表,常用来镇静催眠和防晕止吐。明显的中枢镇静作用可降低用药者的性欲。

西咪替丁等H2 受体拮抗剂有抗雄激素的作用,应用西咪替丁治疗胃溃疡的患者,常出现男性乳腺增生、女性溢乳等不良反应。


\section{第二节 有可能增强性欲或改善性功能的药物}

能够增强性欲或改善性功能的药物称为性兴奋剂。完整的性反应有赖于机体神经、循环、内分泌等各个系统的精确调控,性兴奋剂可通过作用于相应的神经系统及效应器官而对性功能产生影响。一般而言,作用于中枢的性兴奋剂具有催发性欲、增强勃起、提高性快感等多方面效果,作用于效应器官的外周性兴奋剂在治疗勃起障碍方面效果更明显。

育亨宾,俗名萎必治,是从非洲植物育亨宾树皮中提取的生物碱,我国植物蛇根木和光亮萝芙木的根中也含有少量。非洲当地人自古就以水煎煮育亨宾树皮作为催欲剂和致幻剂。20世纪初哥维和苗勒等人将其用于治疗勃起功能障碍,取得一定疗效,并开始广泛应用于临床。

育亨宾是α2 肾上腺素能受体阻滞剂,可提高外周交感神经兴奋性,收缩外生殖器中回流静脉,使生殖器充血,从而增加勃起硬度和延长勃起时间。在中枢可作用于与性反应有关的室旁核和视前叶,增强延脑-脊髓神经元及其以下的交感神经通路,有利于性冲动的形成与传递,因此可产生较强的催欲作用。育亨宾治疗心理性ED的效果明显优于器质性ED,对严重器质性ED几乎无效,所以并不主张用育亨宾治疗器质性ED。其不良反应主要有焦虑、心率加快、头痛等交感神经兴奋症状,一般尚可耐受,严重过量后可出现流涎、痉挛甚至中枢麻痹症状。

士的宁是从植物番木鳖和云南马钱子种子中提取出来的一种生物碱,能对抗中枢中抑制性递质甘氨酸的作用,具有强烈的脊髓兴奋作用。因为能兴奋脊髓的勃起中枢,对大脑皮层也有一定的兴奋作用,所以曾被用来治疗勃起功能障碍。由于士的宁安全范围窄,治疗剂量和中毒剂量很接近,中毒后可出现全身骨骼肌痉挛、角弓反张,甚至呼吸停止、延髓生命中枢麻痹等症状,所以目前基本限用于科研,也很少用于临床治疗。

溴隐亭是人工合成的麦角生物碱溴代衍生物,可直接抑制垂体前叶合成和分泌泌乳素,降低血中泌乳素水平,恢复睾丸或卵巢的功能。临床上主要用于由于泌乳素升高所致的勃起功能障碍及性欲下降。

阿扑吗啡系吗啡的衍生物,麻醉镇痛效应显著低于吗啡,对中枢神经系统的兴奋性明显强于吗啡,尤其对延脑后区催吐化学感受区有强烈的刺激作用,临床最常用于催吐。它可兴奋丘脑性活动中枢内与性有关的多巴胺受体,扩大性刺激的效应,还可通过骶副交感神经丛扩张阴茎海绵体血管,调节勃起功能。阿扑吗啡用于治疗心理性和轻度器质性勃起障碍效果较好。目前报道主要使用片剂3mg舌下含化,或与西地那非等药物合用治疗勃起功能障碍。

左旋多巴是多巴胺的前体物质。多巴胺口服不能通过血脑屏障,而左旋多巴容易通过血脑屏障,在脑内转化为多巴胺,临床上最常用于治疗帕金森氏病。在使用左旋多巴治疗帕金森氏病过程中,人们发现绝大部分患者表现出性欲增强。进一步研究证实,左旋多巴的催发性欲作用也是因为脑内多巴胺水平增高所致。左旋多巴可激活射精的高位中枢,单用左旋多巴或与5-HT拮抗剂合用治疗射精障碍,可取得良好疗效。需要注意的是,左旋多巴的不良反应较多,这是因为口服的左旋多巴在通过血脑屏障前绝大部分已经被转化为多巴胺,这些多巴胺大量蓄积在体内可引起恶心、呕吐、血压降低、心悸甚至是精神障碍。

曲唑酮最初用于治疗抑郁症,后来发现其对性功能有作用。主要表现为能改善ED患者的勃起,延长健康男性阴茎勃起时间及增强性欲。药理机制与选择性抑制中枢的5-HT及外周α-肾上腺素能递质的作用有关。该药对心理性勃起障碍,尤其伴性欲低下、忧郁、焦虑的患者效果更为明显。

代表药物是酚妥拉明、酚苄明等。阴茎的勃起有赖于血流的增加和血量的维持。阴茎内小动脉壁平滑肌主要以α1 受体为主,酚妥拉明、酚苄明等α受体阻滞剂可以阻滞肾上腺素和去甲肾上腺素的α受体样作用,降低交感神经张力,促进勃起。口服α受体阻滞剂的生物利用度明显低于静脉应用,但静脉应用易出现心悸、体位性低血压等不良反应。目前多为小剂量阴茎海绵体局部注射,或浸润滤纸片局部外用,据报道可收到较满意的疗效。不良反应除血压降低外,常见的有注射部位疼痛、出血、海绵体纤维化,异常勃起的发生率也相对较高,使用时需要密切监护。α受体阻滞剂对于心理性和神经性勃起功能障碍具有良好疗效,对于血管性勃起功能障碍疗效较差。与罂粟碱、前列腺素E1 等平滑肌松弛剂合用可提高疗效并降低并发症的发生率。

罂粟碱、前列腺素E1 (prostaglandin E1 ,PGE1 )是最常用于局部注射的平滑肌松弛剂。罂粟碱是非特异性磷酸二酯酶抑制剂,通过抑制环磷酸鸟苷(cyclic guanosine monophosphate,cGMP)和环磷酸腺苷(cyclic Adenosine monophosphate,cAMP)的降解,舒张阴茎海绵体平滑肌。PGE1 是一种强有力的平滑肌松弛药,通过增加细胞内cAMP和抑制交感神经末梢释放的去甲肾上腺素的活性而扩张血管,降低海绵体阻力及增加动脉血流量,使阴茎勃起。罂粟碱、PGE1 与酚妥拉明联合应用比单独应用效果好。对心理性和神经性勃起功能障碍疗效显著。主要不良反应是单用时异常勃起发生率高,长期应用可出现阴茎海绵体纤维化和肝功能受损。

目前上市的磷酸二酯酶抑制剂有西地那非(sildenafil citrate)、伐地那非(vardenafil hydrochloride)和他达拉非(tadalafil)。磷酸磷酸二酯酶抑制剂被用于治疗勃起功能障碍,打破了口服药物治疗ED疗效很低的状况,是ED治疗史上的一次革命,具有划时代的意义,它已成为治疗勃起功能障碍患者的一线药物。

磷酸二酯酶抑制剂选择性作用于阴茎,通过抑制局部血管的磷酸二酯酶对cGMP的降解,使海绵体中cGMP水平升高,导致平滑肌松弛,进入海绵体的血流增加,产生或增强阴茎勃起(详细内容见男性勃起功能障碍一章)。磷酸二酯酶抑制剂对大部分心理性、器质性及混合性ED均有效,药效持续时间为4~36小时。需要强调的是,西地那非本身对平滑肌无舒张作用,也无催欲作用,其作用的前提是要有足够的性刺激,方可有改善勃起的功能。

磷酸二酯酶抑制剂作为一种治疗ED的口服药物,使用方便,耐受性好,无阴茎异常勃起。副作用有头痛、面部潮红、消化不良、鼻塞、皮疹、视觉改变等。但大多是暂时性的,程度较轻,易为使用者接受。此类药物可增强硝酸酯的降压作用,因此严禁与任何剂型或剂量的硝酸酯类药物合用,以防血压过低。


\section{第三节 成瘾药物与性功能}

人类有目的地酿造酒类饮料已有数千年历史,酒对性功能的影响自古以来就受到人们的关注。“酒助人性”、“酒后乱性”是人们对酒类饮品的普遍看法,认为乙醇具有催欲和增强性功能的作用,我国中医和民间也有饮用药酒“壮阳”(改善男性性功能)的习惯。而现代医学深入的研究却得出相反的结论:乙醇对于人类的性功能具有负面影响。酗酒女性常见的性功能障碍包括性欲抑制、性高潮障碍、性交疼痛和阴道痉挛。男性则多出现阴茎勃起障碍、早泄、性欲减退。大约40\%的男性乙醇中毒患者患有ED,约5\%~10\%有射精障碍。在戒酒后的数月或数年内,性功能恢复至正常者仅占这些病例的半数。

1.乙醇对于中枢神经系统具有抑制作用。当少量饮酒、血液中乙醇浓度不高时,大脑皮层受到抑制,解除了大脑皮层对皮层下中枢的抑制作用,饮酒者可出现欣快、言行增多等表现。此时在合适的场景诱导下,饮酒者会有性欲提高、性交快感增强等主观感受,对一些紧张和压抑引起的早泄患者,可以解除压抑和紧张感,延长性交时间。但当饮酒过量、血液中乙醇浓度明显升高时,大脑皮层及皮层下中枢均受到抑制,因此性功能也受到抑制。

2.长期过量饮酒可造成体内性激素水平紊乱。乙醇可以增加人体内的儿茶酚胺水平,它能减少睾丸的血流量,使睾丸受损,严重损害合成雄激素的睾丸间质细胞,降低血中睾酮浓度;酗酒者肝脏对体内雌激素的灭活能力下降,导致血中雌激素水平升高。

3.长期饮酒可造成肝功能下降、体质衰弱等,这些均对性功能产生不良影响。

处理乙醇中毒患者的性问题,最重要的是帮助酗酒者戒酒。应该注意的是,有些顽固的性功能障碍,或者医护人员对他们性方面的康复没有足够重视,可使患者丧失信心,致使处于康复状态的嗜酒者再重新酗酒。

在治疗中最好要求嗜酒夫妇双方一起参加,提供必要的性教育,以减轻顾虑,同时给予双方充分的机会来讨论情感问题和存在的性问题。应让患者渐渐适应治疗,恢复机体正常的营养状态,及时处理已经存在的和将会出现的种种问题。这些问题如不能得到及时解决,也会影响到对性功能障碍的治疗。通常酗酒者的性功能障碍在半年以后才会有明显改善。

吸烟是当今普遍存在的社会现象,它与许多疾病和健康问题密切相关。吸烟是否可降低机体睾酮水平还存在争议,但吸烟者小动脉硬化和狭窄发生率高于不吸烟者,因此存在降低阴茎勃起时动脉血流量、引起勃起功能障碍的可能。

茶、咖啡和可可是世界最著名的三大饮品,都具有兴奋、提神功效。茶大多由茶树叶制成,含有较多量的咖啡因、少量的茶碱和极少量的可可碱;咖啡是由咖啡树的种子———咖啡豆烤制而成,含有较多的咖啡因;可可是由可可树的种子———可可豆制成,含有较多的可可碱及少量咖啡因。三大饮品所含兴奋性物质种类基本一致,只是含量和制作方法不同。茶碱、咖啡因和可可碱的化学结构和药理作用也基本类似,都属于甲基黄嘌呤类药物,具有兴奋中枢神经系统、松弛平滑肌、利尿的功效,但各自的药理效果又略有不同。茶碱松弛平滑肌、利尿的效果更显著,咖啡因兴奋中枢神经系统的药效更确切,而可可碱的各种效果均较弱。由于这三大饮品可以兴奋中枢、振作精神、消除疲劳、提高机体对性刺激的感受能力和反应能力,因此对于机体的性功能具有一定的益处。

阿片是未成熟的罂粟果实划破后渗出液的干燥物。它具有强大的镇痛、镇静、解痉、镇咳等功效,人类有意识地使用阿片治疗疾病的历史已超过千余年。需要注意的是,滥用阿片及其提纯物或合成产品会产生严重的心理和生理依赖性,对个人和社会造成极大危害。

阿片中含有数十种天然生物碱,大家较为熟悉的提纯品有吗啡、可待因和罂粟碱,现广泛用于镇痛、镇咳、解痉和扩血管治疗。常见的人工合成的阿片类生物碱有海洛因、度冷丁和美沙酮。海洛因的镇痛和镇咳效应明显强于吗啡,但由于对呼吸中枢抑制作用更强,并且较吗啡更容易引起成瘾性,因此基本不用于临床治疗。度冷丁镇痛疗效是吗啡二十分之一,且体内代谢产物会对神经系统产生不良影响,目前临床应用逐渐减少。美沙酮的药理作用与吗啡相似,成瘾性和戒断症状相对较轻,故常用于阿片类药物成瘾性的戒断治疗。

国外研究机构报道,毒品依赖者中,以催情作为吸毒的最初动机者高达九成左右。国内这种比例大约是五成。阿片类毒品对成瘾者性欲的影响与使用时间密切相关。滥用初期,可产生类似性快感的感觉,性行为的强度及持续时间等方面也有明显提高。一旦形成药物依赖性,性功能和性心理则出现相反变化,表现为性欲减退或消失、勃起功能明显下降。有资料表明,长期使用海洛因的男性,其性欲的损害高达100\%,女性亦达60\%。海洛因可引起射精延迟,延长的时间从30分钟到120分钟,甚至更长。这使一些人产生了海洛因能提高性能力的错觉,随着继续吸食海洛因,最终将出现不射精。射精延迟和不射精实质都是性高潮障碍,虽然能始终保持插入姿势,但仅是机械运动,缺乏性交的快感。使用海洛因15周之后,男性阴茎勃起硬度降低或发生ED者高达61\%,女性则出现阴道分泌物减少、性交疼痛。

阿片类毒品降低性欲的主要原因是此类毒品及代谢产物可作用于垂体和下丘脑,通过抑制LHRH和LH的分泌而影响性腺轴的功能,导致泌乳素升高,LHRH、LH和睾酮分泌下降。另外,人类的性高潮与中枢神经系统脑内啡肽释放有关,反复使用阿片类物质后,可通过某种反馈机制使脑内啡肽的释放减少或停止,从而导致成瘾者射精延迟或不射精,性交时性快感降低性、高潮障碍或丧失。

戒毒治疗后,性功能可有所恢复,但部分成瘾者的性功能可能终身无法恢复正常。

成瘾品大麻通常指从植物大麻中提取的有效成分,包括大麻酚、大麻二酚、四氢大麻酚、大麻环酚等。吸食大麻后会出现一种心理上的欣快和幸福感,因此大麻最初被西方僧侣用于宗教活动,作为致幻剂使吸食者减轻痛苦和焦虑。大麻的成瘾性和戒断症状均弱于阿片类毒品。二十世纪中后期,为了放松精神、解除烦恼,大麻在西方社会的年轻人中被广泛滥用。由于缺乏严格而科学的调查和研究,大麻对吸食者性功能的影响,始终存在着争议,有报道称吸食大麻可增强性功能,但更多的报道倾向于抑制吸食者的性功能,引起吸食者睾丸或卵巢萎缩。严格的动物实验已经证实大麻的活性成分能够抑制小鼠LH、FSH和血中睾酮浓度,减少小鼠的性行为。

新型毒品包括甲基安非他明(又称为去氧麻黄碱、甲基苯丙胺、冰毒等)、3,4亚甲二氧基甲基苯丙胺(摇头丸)、麦角酰二乙胺(lysergic acid diethylamide,LSD)、氯胺酮(K粉)等,目前在国际和国内广为滥用,并有上升趋势。与阿片类传统毒品相比,新型毒品具有更强大的致幻性和成瘾性,长期或大量滥用可造成机体不可逆性的损伤。由于出现时间相对较短,以及伦理学的限制,这些新型毒品对性功能的影响尚缺乏系统研究。部分使用者反映使用初期确能增强性快感、延长性交时间,推测是由于使用毒品的目的和环境对促使性行为的发生有一定作用,这些毒品还能明显兴奋大脑边缘系统,增强使用者对性行为的主观感受。大量使用后,可导致内分泌和神经系统不可逆损伤,可能会对性功能产生不良影响。


\section{第四节 中药与性功能}

在我国,用中药治疗性功能障碍和增强性功能已有上千年历史。传统医学认为性功能障碍多为肾虚所致,具体辨证为肾阳虚、肾阴虚、肾阴阳两虚等证,分别采用不同的方剂进行补肾治疗。现代药理学研究证实,众多的补肾中药通过增强体质、强化中枢神经系统的调节机制、刺激性激素释放或增强激素功能而产生改善性功能的作用。目前尚缺乏大规模临床药物实验来验证中药的性功能改善作用。


\section{第五节 春 药}

能够促进性欲产生、增强勃起程度、延长性交时间以及提高性快感的药物统称为“春药”,也被称为壮阳药、性兴奋剂、助性剂等。尽管人类千百年来都在对春药进行寻找和研究,民间传说和文学作品中也常常提到一些神奇的春药,但目前尚无一种既安全、又适用于所有个体的性兴奋剂。

由于认识上的局限性,人们过去认为雄性动物的生殖器(阴茎、睾丸等)或与生殖器外形相似的动植物,如牡蛎、泥鳅、香蕉、犀牛角等,具有强大的改善性功能作用,现代研究并未证实这些观点。它们所产生的改善性功能效果更多是因为心理作用所致。

海洛因、大麻、LSD等成瘾品可使吸食者产生富有刺激性的幻想,增强包括性在内的各种精神感觉,这些成瘾品对机体内分泌和神经系统有损伤作用,最终会严重损害吸食者的性功能。磷酸二酯酶抑制剂(如西地那非等)和硝酸酯类血管扩张剂对一些勃起功能障碍的患者有较好疗效,但不具有诱发性欲的作用,也不能明显增强正常人的勃起强度、延长性交时间。所谓的“印度神油”实质是一种表面麻醉剂,通过降低龟头和阴茎的感觉而延缓射精,过量应用可由于局部感觉丧失导致勃起障碍或不射精。“西班牙苍蝇水”是斑蝥(又称为“西班牙苍蝇”)的浸出液,含有对机体致敏的物质。这些物质经泌尿系统排泄时,可引起膀胱和尿道严重的过敏反应,使阴茎充血,产生痛性勃起,服用者会试图通过性交以缓解疼痛,因此使用者当时并无多少愉悦的感觉。此药安全范围窄,对于人类,其安全使用剂量和中毒剂量很接近,易出现局部疼痛、血尿甚至过敏性休克等严重后果。对于体重远超过人类的大型动物,使用时安全性相对较高,因此常作为催情药用于牛、马等大型家畜的配种。

(靳风烁 孙中义)


\section{第六节 女性性功能障碍的药物研究}

自西地那非的问世给男性性功能障碍的治疗带来划时代的变化后,女性性功能障碍治疗药物的研究从未停止,但迄今为止,全球公认的专门用于治疗女性性功能障碍(FSD)的上市药物仍处于空白阶段。在研发治疗FSD药物的临床试验时,一方面需要遵循药物临床试验的GCP(Good Clinical Practice)原则,另一方面应注意女性性功能研究的特殊性,才能获得较为规范的研究结果。

任何一种新药从诞生到上市,都需要经过实验室试验、临床前动物试验和临床研究。药物临床研究是指“任何在人体(患者或者健康志愿者)进行的药品的系统性研究,以证实或揭示药品的作用、不良反应及/或研究药品的吸收、分布、代谢和排泄,目的是确定研究药品的疗效与安全性”。我国国家药品监督管理局于1999年7月23日颁布施行了《药品临床试验管理规范》,对我国的药物临床研究进行规范化和法制化管理。根据新药研究的不同阶段和目的,人体药物临床试验分为Ⅰ期、Ⅱ期、Ⅲ期、Ⅳ期临床研究。

主要是了解药物的毒性反应。试验在正常成年健康志愿者中进行,受试对象较少,一般10~100例,观察药物在人体的作用机制,尝试确定药物耐受范围和最大耐受剂量,据此拟定合理的给药方案。一般在药物研究病房进行。

完成Ⅰ期试验后进入Ⅱ期临床试验。主要目的是寻找最佳的临床药物剂量和疗程。受试人数较Ⅰ期增加,25~100人,甚至500人或根据需求増至更多人,但仍属少量例数的临床前期试验。受试者为第一批接受研究药物的患者,因此挑选受试者标准应严谨,选择仅有适应证而无其他疾病的患者,以避免其他同期疾病影响疗效评估。研究设计严格对照,通常公开,目的是观察最大耐受剂量以下的剂量范围,找出最佳疗效、没有或可接受的副作用的剂量范围,进而确定最佳剂量以及评价疗效,试验设计要有评价治疗的主要指标和(或)次要指标,使研究具有可行性。Ⅱ期是探索性研究药物治疗剂量和疗程的研究过程,同时也是继续观察短期毒副作用的研究过程,并为Ⅲ期临床研究提供了基础。

目的是在Ⅱ期临床结果上扩大样本量进一步确定药物的有效性和安全性。研究设计更加严格,须采用盲法对照以减少偏倚。通常应设置标准疗法及安慰剂对照。如用市场上已公认的标准药物或疗法作为对照,如尚未确立标准疗法,则只能用安慰剂进行空白对照。样本量应较Ⅱ期大,根据疾病的发病率、该种药物和治疗该种疾病的估计治愈率进行统计学计算和根据新药审批标准增加患者人数,需要正在接受治疗的患者志愿受试者,也可包括特殊人群如一些可以接受的老年人,以观察药物的普遍耐受性。(ⅢB期临床研究:Ⅲ期临床研究结束后研究数据和结果送呈药监部门申请注册上市等待过程中,继续进行的Ⅲ期临床研究,目的是继续扩大受试者样本,累积更多的安全性数据,观察尚未发现的情况和附加的适应证。)

新药获准上市后的大型研究,观察投入临床普遍使用时少见的不良反应和毒性,进一步了解药物的安全性和疗效的详细资料、收集长期安全性数据、附加剂量信息和其他未被发现的适应证等。研究设计多采用对照研究,或者对不同的给药方法、剂量、疗程和其他药物的相互作用进行评价。

Ⅰ期、Ⅱ期、Ⅲ期、Ⅳ期临床研究是新药的研究阶段,一个全新的药物从开发到上市,一般需要10年以上的时间,从发现药物到临床前阶段能够成功的比例估计大约占Ⅰ期临床试验13\%、Ⅱ期临床40\%、Ⅲ期临床80\%,甚至有些在完成了所有临床研究后,在向药监部门申请新药注册时由于种种原因仍不能成功上市。在这一整个研究过程中,临床医生一般涉及并参的是Ⅱ期~Ⅳ期的临床研究,在这一过程中,“必须符合《赫尔辛基宣言》和国际医学科学组织委员会颁布的《人体生物医学研究国际道德指南》的道德原则,即公正、尊重人格、力求使受试者最大限度受益和尽可能避免伤害。参加临床试验的各方都必须充分了解和遵循这些原则,并遵守中国有关药品管理的法律法规(《药品临床试验管理规范》第二章第四条),即必须把受试者/患者利益放在首位,对药品临床试验的全过程进行严格的质量控制,确保受试者/患者的权益受到保护。因此,从试验设计、试验过程和试验结束的整个研究阶段,都应贯穿执行这一原则。

临床研究设计阶段

(1)研究者

临床药物研究须在临床药理基地进行,尤其是Ⅱ期、Ⅲ期研究,由药物研发机构委托发起。作为研究员,必须具备承担研究的专业特长和能力,接受《药品临床试验管理规范》的培训,并有足够的和具有资格的人员协助等条件。如为多中心研究,则要由一位研究人员主要负责,并负责各个分中心间的协调,各分中心应各指定一名分中心负责人,负责本中心内的研究质量控制。

研究者职责包括:负责知情同意书和研究方案的各种批准申请和申报,取得伦理委员会批准后方可开始进行临床研究;执行和督导研究按照方案进行;搜集资料和准确填写病例报告表(CRF)。

(2)试验设计原则

基本试验设计有基本平行设计(受试者编入不同治疗组别,各组同时进行研究)、基本交叉设计(每个受试者先后接受研究中要比较的各个疗法),需要遵循对照、设盲、随机的原则。

(3)确定研究方案

1)研究对象:根据研究目的确定样本含量,制定纳入标准和排除标准。应根据国家诊断标准作为受试者纳入标准,以选择符合适应证的相似基线的受试者,排除可能影响研究结果的因素。

2)疗效指标:应在研究开始前设定疗效评定标准,包括主要指标和次要指标,

评定参数的方法、观察时间、记录与分析。

3)安全性指标:在研究前后评价研究对受试者安全的影响,包括体格检查、实验室检查、心电图检查、药物不良反应、药物可能产生影响的特殊检查。在设定的安全性指标中,应规定在何种情况下需要紧急揭盲。预期可能出现的不良事件。不良事件是指在临床研究中接受药物治疗的受试者出现的任何不良医学事件,包括已存在状况的加重,不良事件不一定与药物治疗存在因果关系。严重不良事件是指:“有致命或危及生命的可能,导致永久性伤害,需要住院或延长受试者住院时间,或导致先天畸形或出生缺陷,或导致其他被视为严重的医学事件”。但均应上报并记录在案。

i.知情同意书

应根据药物特性制定符合《赫尔辛基宣言》和国家法规、道德规范知情同意书,报伦理委员会批准,内容须包括:

●研究的目的和性质;

●有关研究药物的信息,但注意不能暗示药物的绝对安全和有效、不能答应研究结束后继续提供研究药物、避免承诺一定可加入任何有关的延续研究;

●研究的大致受试总例数,预期每个受试者参加研究的最长时限;

●需参加的研究步骤,包括纳入/排除标准、随访/复诊次数、给药方案等;

●受试者可能获得的益处,不能暗示药物的绝对安全和有效或保证极好的治疗疗效,仅能说“可能有良好效应”,如受试者可获得经济报酬必须说明;

●受试者可能面临的预期风险和不适,包括药物可能的副反应、所有研究使用的一般副反应等;

●受试者未知的风险,如对胎儿、胚胎的影响等;

●如因参加研究而出现危害时的治疗责任和赔偿;

●受试者声明自愿参加,并有充分的时间考虑,如拒绝不会受到惩罚,可随时退出研究而不会受到歧视和报复,医疗权益不受影响,研究机构提供可以解答受试者权利问题和医学疑问的咨询人;

●需要退出/终止研究的情况;

●承诺受试者资料保密,其身份和个人信息不被泄露;

●受试者是否需要额外付出费用。

i.病例报告表

病例报告表(Case Report Form,CRF)指按试验方案所规定设计的一种文件,用以记录每一名受试者试验过程中的数据。内容包括研究方案、管理和有关规定的信息:受试者身份代码或编号、治疗处理方案、受试者接受治疗前后和治疗期间的情况、检验数据、不良事件、研究中退出时间及原因等。

设计中应注意使CRF满足以下要求:

a.表格设计简单不累赘,只收集方案所需的全部必要数据;

b.易填写,以空格、方框或一览表形式。如为选择,则须注明单选或多选;如为相对评估,应注明相对比较的基点;尽量不采用开放性问题,如需要采用,也只要求简短回答,不必详述或评价;

c.一般以研究机构的统一专用表为基础,根据研究需要增减改编;

d.每页均带有页码编码;

e.问题简单易懂,避免外文缩写、简称、用字深奥难懂、意思含糊、否定式问题;

f.用统一标准格式搜集数据,最好只要求研究者在方框内打“√”,而不要求研究者填写为输入计算机程序而设计的编码,因为编码在填写时需要花时间专注阅读,而且易产生内容混淆导致错误;

g.问题准确,对可能产生理解分歧的关键词设下定义,以免不同研究者之间理解不同;

h.提问次序合乎逻辑,根据随访时间、发生先后次序、操作步骤系统性地列出,避免追溯事件;

i.为帮助研究者顺利执行研究方案,重要步骤可在适当处附加提示;

j.CRF一式三份(或四份,根据需要而定),分别是正本(申办者存档)、数据处理、研究者存档;

k.根据需要制作装订方式,如受试者不能通过筛选期可能性较大,可将CRF印制成筛选期和研究期分册,以免浪费;

l.最后需有研究者签署声明页。

4)临床试验研究过程

a)受试者入组:在招募受试者时,为便于操作,可将入选标准和排除标准列成表格式样,逐项筛选,只有符合入选标准无排除标准方可参与临床试验。在任何有关研究的步骤和用药前,必须首先取得受试者同意,签署自愿参加研究的知情同意书,在签署知情同意书时,研究者要用受试者明白的简单日常用语向受试者详细解释知情同意书的内容,并给予足够的时间考虑,受试者亲自签名,并同时亲笔注明签署日期。知情同意书复印件交给受试者。为减少失访,入组时争取不纳入合作性欠佳、不安定或随访困难者。

b)随访:研究设计应使整个研究时间和随访期间隔适中,尤其是耗时较长的研究,间隔太短受试者生活将受到影响,而间隔太长则容易忘记,因此2~4周比较适宜。研究者在受试者入组时除应详细解释研究期限、研究步骤外,还应强调依从性的重要性。研究进行期间,尽量与受试者保持联络,每次随访鼓励和强调下一次随访,如过了复诊时间,应主动联络受试者,询问原因并要求返回随访。如果受试者主动要求退出,应查询并记录原因,如受试者不愿透露原因,应尊重其意愿,允许退出。如出现不良反应或其他治疗更利于受试者利益时,应让受试者退出研究。

c)原始资料和CRF:原始资料是指对与下述事项之所有记录及经证明无误的拷贝资料:原始观察、临床发现或对该试验的重建和评估必要性的其他活动,是记录受试者信息,证实受试者真实存在的最原始版本,是除CRF和研究报告以外,证实CRF内容真实的独立记录文件。原始记录不得由CRF代替。记录原始记录时应注意所有有关研究的信息均已记录清楚。

CRF是核实研究数据真实性的重要文件,研究完整与否取决于CRF的真实和正确。在填写CRF时应注意:

●用黑色圆珠笔或签字笔填写;

●必须按照研究要求设计的格式填写,不得做任何改动;

●所有空格必须填写,采用事先一致统一的填写标准;

●做更正时不得改变、涂污、遮盖最初记录,只能在错误处划上一线删除,保证能够辨认,在旁边填上更正数据并附加理由说明,由作出更改的研究者签名并注明修改日期;

●如有疑问或不清楚如何填写,应暂留空白,询问研究监察员,清楚后再做填写。

d)监察员查阅:监察员是指由申办者任命并对申办者负责的具备相关知识的人员,其任务是监察和报告试验的进行情况和核实数据。在研究过程中核实研究者研究执行是否依据方案;查阅原始资料,与CRF数据核对,核实CRF内资料数据的准确性和完整性;确保出现不良反应尤其严重不良反应后按要求迅速处理和报告;对研究用药协调供应和及时补给,检查药物管理记录;解答CRF中的问题。

e)报告不良事件:任何不良事件均应记录在CRF中。

严重不良事件一般包括:①死亡;②危及生命的情况;③需住院或住院时间延长;④持续或明显丧失功能;⑤先天异常或出现缺陷;⑥需要采取措施预防前述任何事件发生;⑦癌症;⑧药物过量。出现严重不良事件后,研究员要在获悉事件发生后24小时内通知伦理委员会和申办者/监察员,申办者的安全监察委员会应迅速跟进解决和随访,并判断该事件是否与本次临床药物试验有关。

f)研究药物管理:研究药物经申办者准备并按照国家规定适当包装和标签后交付研究机构,但不包括随机表(随机表绝对保密,仅在关系受试者安全的紧急情况下,急需知道曾服用何种药物而决定治疗方法时,方可破盲)。研究机构接到药物后,由研究员和(或)药剂师全权负责药物的使用和管理,一切步骤均须准确书面记录,与研究文件一同保存,应注意:

●根据药物特性存放在适当的保存条;

●药物上锁,确保非授权研究人员不能接触;

●保持药物的管理和记录,内容包括:受试者编号、日期和数量(发出、使用、回收),保证所有药品仅用于合格的临床研究受试者、剂量用法均按照方案使用;

●将未发放和受试者退回的药物交还申办者或协商处理,如在院内销毁必须准确记录;

●在研究员档案中,保存下列文件的完整性:研究药物管理指引、药检证书、药物收据、药物派发记录、药物销毁/申办者回收药物清单等。

5)临床试验分析阶段:在所有受试者完成随访,全部CRF填妥后,进入分析阶段,即数据库处理和统计学分析。分析疗效指标和安全性指标。在分析时如果发现与预期结果不相符的阴性结果,应小心总结分析其真实性,如发现研究设计中的不足之处,只能作出讨论并建议再作研究确定。还可尝试从方案设定以外的其他方向分析,其中的某一亚群或组可能有显著差异,但总结时不能将此项发现作为研究结论,只能提议下次研究中跟进。安全性指标除分析用药前后体格和实验室检查有临床意义外,还应对试验过程中的不良事件进行分析。

研究者应对不良事件和试验药物及合并用药之间可能存在的关联性按照以下5级分类标准作出评估:

肯定有关:用药与不良事件发生的时间有合理关系;不良事件的发生较容易解释为因用此药而引起,而非其他原因;用本药及重复用药可见同样不良事件;不良事件发生类型与前次用药相同。

可能有关:不良事件的发生与用药有着合理的相关关系;不良事件也可能为其他类似原因所致。

可能无关:另有更可能导致不良事件发生的原因;重复用药不再出现不良事件或两者关系不明。

肯定无关:不良事件的发生与用药无合理的相关关系;或存在另一个明显引起不良事件的原因,不良事件随着该原因的消失而消除,随着该原因的重现而再现。

无法判定:不良事件的出现与用药的时间顺序无明确关系,不良事件与此药已知的反应类型相似,同时使用的其他药物也可能引起相同的反应。

在进行安全性分析时“肯定有关”、“可能有关”和“无法判定”作为相关不良事件。

6)临床研究后阶段

撰写研究总结报告:总结报告是指试验完成后的一份详尽总结,包括试验方法与材料、结果的描述和评估、统计分析以及最终所获鉴定性的、合乎道德的统计学和临床评价报告。是临床研究的最后一个步骤,用以送呈药物注册上市申请。内容应针对方案中设定的假设和立题,简要总结结果。总结报告应和研究计划一致,内容按逻辑性顺序编排,适当地方加入图表易于阅读理解,前后内容注意一致,无相互矛盾,尤其是多于一个作者时(如研究者和统计学家),更应特别注意。

资料保存:研究结束后,研究机构需将临床试验的原始资料、CRF副本、研究文件一起存放,我国规定研究结束后应该保存原始资料5年。

发表研究结果:在发表临床药物试验研究结果时,除与其他医学论文撰写的要求相同,根据所投医学期刊要求格式撰写,需要详述研究设计、受试者要求、药物来源、研究指标、统计方法、研究结果和讨论外,在文章投稿时还应附政府和伦理委员会批文,以证实新药临床研究的合法性。文章的人员署名一般是由临床研究的主要牵头者(PI)作为第一作者或者责任作者,作者顺序根据参加研究的单位或个人在研究中所作的贡献顺序排列,或研究之前研究者团队有协议。

在进行药物研究时需要考虑“性”和“女性”的特殊性。

(1)入选标准

1)根据不同药物研究阶段,在符合一般药物研究原则的基础上制定一般入选标准;

2)在Ⅱ期以后的临床试验制定入选标准时,应严格掌握性功能障碍的诊断标准,除症状、病程时间外,确定是否“引起自身精神痛苦”,可以应用女性性痛苦等级量表(The Female Sexual Distress Scale,FSDS)进行评价(详见第二十章第十九节);

3)由于性功能研究常需要以能否获得满意阴道性交作为主要或次要的疗效指标,因此具有稳定单一的性伴侣关系并取得男方同意是入组的必备条件;同时应由受试者评价男方性功能情况,根据药物目标疗效配偶应符合“无勃起功能障碍”/“轻度勃起功能障碍”、不能患有早泄,并在研究期间至少有一半以上的时间与受试者共同生活;也有学者认为夫妻双方共同作为受试组一起入选观察研究;

4)受试者及配偶能够承诺达到一定的阴道性交频率方可入选(根据研究药物标准不同);

5)为保证研究的顺利进行,受试者应做好避孕措施,在研究期间避免妊娠;

6)简易精神状况检查显示无严重的焦虑症或抑郁症;

7)身体状况符合参加临床药物研究的基本要求。

(2)排除标准

1)由研究药物本身药理特点决定的特殊禁忌或慎用情况;

2)精神异常包括抑郁、焦虑或药物依赖;

3)性伴侣患有可能干扰受试者对疗效评价的器质性或性心理障碍;

4)与性伴侣存在情感障碍;

5)未经治疗的内分泌疾病(如垂体功能低下,甲状腺功能低下或糖尿病)或手术(如肾上腺切除,垂体切除)引起的女性性功能障碍;

6)需要使用可能影响性功能的治疗躯体疾病的药物;

7)妊娠期或哺乳期;

8)急性感染期阴道、泌尿系统感染;

9)与其他药物临床试验一样,存在心、肝、肾、脑等疾病或近期异常病史,参加试验可能对患者安全造成影响的各种异常状况;

10)症状或精神痛苦极其严重或轻微,或症状表现间断性出现,由于临床试验时间有限,难以达到预期治疗效果,均不宜参加药物临床试验。

(3)退出研究标准:由于女性性功能障碍治疗药物试验过程多需要数月,因此应在设计阶段制定退出标准,内容一般包括:

1)发生严重不良事件;

2)患者要求从研究中退出;

3)患者不按方案用药或填写表格,无法提供准确信息;

4)健康原因需要使用研究方案中禁止使用的药物;

5)研究过程中受孕并拟继续妊娠。

(4)疗效指标:由于女性性功能完全依靠主观评价,因此应根据药物治疗目标效果确定主要疗效指标和次要疗效指标。疗效指标可包括受试者性功能评价量表的改善、性活动指标的改善、总体性满意度评价,但不应将配偶评价作为疗效指标。美国FDA和一些国外学者推荐将满意性事件(SSE)计数作为基本的主要疗效指标(即满意性活动次数/性活动总次数,满意与否由受试者自我评价)。由于影响女性性感受的因素较多,且存在女性主观感知与生理反应分离的客观现象,因此以药物预期作用的观察指标作为主要疗效指标,是否确定SSE作为主要疗效指标可根据药物区分。在国外的研究中发现,以有效的自我问卷评价如女性性功能问卷(the Female Sexual Function Index,FSFI)是最为敏感的治疗变化指标。性焦虑程度作为入选的必要条件,其改善一般只能作为次要疗效指标。也有研究将男女双方关系作为评价指标以衡量女性治疗后变化。如果采用其他语言已有的量表工具,在语言翻译时应注意直译可能影响评价指标的可靠性,可先行进行小规模的量表准确性评价。

性活动指标主要依靠受试者日记,记录受试者性活动情况。设计应做到记录内容可靠、准确、可反映前后变化,日记应简单易填写,避免手续复杂影响受试者依从性,问题通俗易懂,避免因理解差异出现信息收集误差。一般研究设计中应考虑受试者能够接受的记录和回访时间。疗效指标中的受试者记录部分可以是每1天、7天或1月进行总结填写。尽管美国FDA的指南中认为应每日填写日记,但可根据研究药物的不同而规定记录时间,更便于操作,如对性欲进行评价,一般受试者接受每7天或每4周进行一次评价,如果是评价每一次性交情况,则应每次性交后总结。而对一些量表的评价,则以每4周一次为宜。

(5)安全性指标:在确定安全性指标时,除与其他临床试验的药物一样需要比较用药前后的体格和实验室检查外,如为局部用药,还应关注局部安全,比如对宫颈细胞学有无影响。尽管有效性研究需3~6个月,但对安全性指标则需要更久,即便是在临床研究结束后,仍需长期观察。

(6)研究时间:由于多种因素可能导致女性性功能障碍,参加临床试验这一事件本身对女性参与性活动有积极作用,因此为消除试验事件的影响,一方面在入组后的一段时间暂不用药,而作为了解性现状的基线期,一般建议4~8周,另一方面研究设计应注意使受试者有足够试验用药时间,以观察研究持续进行后性功能改善情况。一般经过3个月即可显现药物作用,但不足以起到改善男女双方关系的作用。故如为Ⅱ期临床研究可为3个月。美国FDA推荐的时间标准为除基线时间外药物使用时间6个月。

如为确定剂量的药物研究,应考虑受试者依从性问题,如果开始用药后1个月最多2个月未见到治疗效果,则受试者一般无法继续依从。为避免受试者脱失,增加临床可操作性,可为3月(12周)。

(7)其他:研究入组人数根据统计学需要设定。

对照组可以采用安慰剂空白对照。在选择研究剂量时,应根据前期研究选择最小有效剂量。也可在满足统计学要求的前提下,设置剂量平行对照组。

根据药物本身特点确定用药时间,晨起、睡前、性交前使用。用药间隔时间应根据药代动力学结果制定,对半衰期长的药物,应注意叠加效应。

所有的FSD药物研究中,必须进行女性激素水平的测定,除为内分泌异常外,应根据激素水平将受试者至少分为绝经前组、围绝经组和绝经组,还可以细分至自然绝经组、手术绝经组、激素治疗组、口服避孕药组,在进行疗效和安全性分析时分别讨论。

(1)受试者入组:女性性功能障碍的诊断主要依据患者的症状。但当受试者要求参加药物临床试验时,受试者本身对诊断标准并不清楚,自行以某种性功能障碍为主诉,或为解决自身问题用药心切,根据猜测的入选标准描述性功能现状。由于各种女性性功能障碍互为因果,因此要求研究者在入选过程中不仅仅依据受试者提供的异常表现,而应同其他疾病一样,分析出现性功能异常的原因,筛选出符合入组标准、无排除标准的患者。

(2)知情同意:在签署知情同意书时,同其他临床研究一样,需要告知受试者其权利和义务,同时应告知研究药物可能存在的风险和用药后的注意事项。由于目标作用有阴道性交成功,因此应提醒受试者避免妊娠,必要时应提供避孕措施;根据药物的不同,研究药物的使用时间和方法的差异,如果是性交前使用的药物,尤其是外用药物,应考虑在特殊性行为方式如口交时药物是否会对配偶的安全产生影响;应保证受试者与医生联系方式的畅通,以便在出现急性不良事件时能够在第一时间得到研究者的指导。

(3)鼓励受试者认真填写性交日记:在女性性功能障碍的药物临床试验过程中,多需要受试者填写性交日记。及时填写日记是反应真实情况的保证。因此在研究过程中应提醒鼓励受试者及时认真填写性交日记,及时记录当时的药物反应和感受,避免出现在返院访视前突击填写,造成记录内容的不真实。

(1)疗效分析:在进行疗效分析时应首先对未用药的空白基线进行了解和分析,因为可能存在参与试验的作用引发受试者对性现状的关注,自身的积极参与已使受试者的性功能得到改善,主要疗效观察指标在未用药的情况下已显著提高甚至达到良好程度,此类受试者的试验数据应从意向治疗(Intention-to-treat,ITT)人群和符合方案病例(Per-Protocol,PP)人群中去除。由于FSD受到心理因素影响,因此安慰剂也会产生一定的治疗效应,但如果安慰剂效应超过50\%,说明研究失败。

(2)安全性分析:在进行安全性分析过程中,研究者应在揭盲前通过会议讨论分析所有不良事件,按照5级原则确定不良事件与研究药物相关性,对部分存在疑问的不良事件,如受试者可能将药物作用如生殖器官的充血作为不良反应上报,此类由于受试者缺乏性知识误报的不良事件应在分析中去除。在基线期过后从ITT人群和PP人群去除参加疗效分析的病例,如应用药物应纳入安全性分析人群。如果女性生殖器外用药物,性交过程中配偶接触药物,则男方出现的不良事件也应一并纳入安全事件进入分析。

尽管存在争议,但女性性功能障碍治疗药物的临床试验一直都在进行,包括新药的研发、上市药物对改善女性性功能的扩大适应证研究(由于资料所限,以下列出研究结果可能资料不全,仅供参考)。

(1)氟班色林(Flibanserin):德国Boehringer Ingelheim药厂生产,5-羟色胺1A受体激动剂和5-羟色胺2A受体拮抗剂。在美国进行880人和1581人的两项Ⅲ期临床试验,受试者为18岁以上绝经前性欲低下(HSDD)患者,接受氟班色林和安慰剂的对照研究,研究组接受氟班色林治疗,每名受试者参加研究时间6月。结果显示每日100mg氟班色林,用药前女性平均每月只有2.8次满意的性生活(SSE),但服用氟班色林后,SSE增至每月4.6和4.4次,SSE率为47.6\%和44.2\%,为而服用安慰剂的女性,SSE只增加至3.7次,SSE率为33.0\%和34.1\%;但是用日记每天记录性欲情况,用药前后性欲评分变化与安慰剂相比无统计学差异;FSDS评分药物组由治疗前的30.6和30.7降至21.6和22.9,安慰剂由30.1和30.2降至25.2和25.3(P<0.05)。安全性方面未出现严重安全事件,常见的不良反应包括头晕、嗜睡、疲倦、恶心、失眠、口干和焦虑。同样在欧洲进行的Ⅲ期临床试验中,Flibanserin组的SSE由每月2.4次上升至3.9次,但较安慰剂组2.3上升至3.1次,P=0.140,无统计学差异。

在2010年6月17日,美国FDA未通过flibanserin治疗女性性欲低下投入临床使用。

(2)雄激素(Testosterone,androgen)

1)英特丽莎(Intrinsa):德国宝洁公司(Procter &Gamble)生产的一款雄激素缓慢释放的皮肤贴剂,每周使用两贴,相当于每日释放剂量为300μg。适用于手术或自然绝经妇女。在英国进行的814例随机、双盲、安慰剂对照、多中心研究中,HSDD志愿者随机分组至300μg、150μg和安慰剂组,疗效观察至24周,安全性观察至52周。研究结果显示24周时,300μg组SSE上升了2.1次/4周,较安慰剂的0.7次/4周有显著差异(P<0.001),而150μg组上升了1.2次/4周,与安慰剂相比无统计学差异(P=0.11);300μg组、150μg组和安慰剂组比较,均有提高性欲(300μg,P<0.001;150μg P=0.04)、降低痛苦(300μg P<0.001;150μg P=0.04)的作用;安全性观察中不良反应多数轻微,主要为非意愿性毛发增多,300μg组(19.9\%)高于150μg组(11.6\%)和安慰剂组(10.3\%),其他如痤疮等各组无差异,治疗组中在参加试验的4~12周间发现3例乳癌。另有一项英国272例随机、双盲、安慰剂对照、多中心研究中,治疗组接受每日300μg的释放量,血清游离雄激素和总雄激素水平分别升高至5.65pg/ml和67.8ng/dl,但仍在育龄妇女正常范围。

通过临床试验认为Intrinsa可有效提高绝经妇女的性欲,因此在欧洲部分国家获准上市,但美国在2004年未予通过,原因是担心滥用及增加妇女心血管意外和肿瘤的风险。

2)利比胶(LibiGel):美国BioSante制药公司生产,为雄激素凝胶制剂,每日涂于上臂皮肤一次,药物通过透皮技术进入体内。正在美国进行随机、双盲、安慰剂对照、多中心的Ⅲ期临床试验,据称其SSE提高达238\%。现主要进行长期安全性研究,主要观察用药后的心血管疾病和乳腺癌发生情况。

治疗男性勃起功能障碍的上市药物包括磷酸二酯酶-5(PDE-5)抑制剂(西地那非、伐地那非,他达拉非)、前列腺素E1乳膏(比法尔),其对于男性勃起功能障碍的作用得到公认,考虑到女性阴蒂海绵体与男性的阴茎海绵体在组织起源、解剖学和生理学上的相似性,因此分析此药物可能对女性性唤起障碍(FSAD)有治疗作用。

(1)磷酸二酯酶-5(PDE-5)抑制剂:其中西地那非(sildenafil)最早投入临床试验。曾对781例FSAD患者进行的西地那非临床试验和安慰剂比较未显示出其治疗作用。另有其他11项用西地那非治疗女性性功能障碍的临床研究,但样本量多较小,且结论不同,其中8项研究认为西地那非对女性性功能障碍有治疗作用,另外3项研究未发现显著的治疗作用。但研究表明,西地那非可能对由多发性硬化症、Ⅰ型糖尿病、脊髓损伤或抗抑郁治疗导致的女性性功能障碍有益。伐地那非和他达拉非未用于女性性功能障碍的大样本的临床试验。

(2)前列腺素E1乳膏(alprostadil):在我国进行的400例前列腺素E1乳膏受试者参加的治疗女性性唤起障碍多中心、随机、双盲、剂量组平行对照临床试验显示,安慰剂和3种不同剂量的试验药物(500μg、700μg、900μg)性唤起满意率相对基线(治疗前)分别提高了22.63\%、36.67\%、34.01\%、44.29\%(P<0.05),FSFI评分在结束时相对基线提高14.68、20.71、21.69、22.89(P<0.05),FSDS评分在结束时相对基线降低17.60、20.27、21.98、25.97(P<0.05),研究中主要不良反应为局部刺激症状,未发现与药物有关的体格检查、血生化检查、心电图、宫颈TCT异常,认为前列腺素E1乳膏治疗女性性唤起障碍有效,其中900μg作用最佳,无明显全身副作用,不良反应以局部刺激症状为主。在其他国家进行的4个小样本前列腺素E1的研究中,尽管剂量略有差别,3个显示有效,另1个研究尽管较安慰剂有改善但无统计学差异。

女性性功能障碍的药物研发由于种种原因尚处于研发的初级阶段,临床药物实验进行得较为困难,尤其是在我国和一些对性比较保守的国家。但随着人们对生活质量要求的提高、女性性观念的觉醒和进步,该类药物会不断的研发并投入社会。在进行临床实验时,需要注意其性学和社会观念上的特点,才能设计执行出符合GCP和国情的临床药物试验。

1.马晓年.现代性医学(第2版).北京:人民军医出版社,2004

2.沈渔邨.精神病学(第4版).北京:人民卫生出版社,2002

3.江开达.精神病学.北京:人民卫生出版社,2005

4.魏莎莉,杨戎.性医学.重庆:重庆出版社,2005

5.刘继红,熊承良.性功能障碍学.北京:中国医药科技出版社,2004

6.许士凯,李东.新编性药物学.天津:天津科学技术出版社,2005

7.性医学.马斯特斯,等.吴阶平,等译.北京:科学技术文献出版社,1998

8.陈灏珠.实用内科学.(第11版).北京:人民卫生出版社,2002

9.金有豫.药理学(第5版).北京:人民卫生出版社,2002

(张渺 廖秦平)


\chapter{第十四章 常见疾病与性}

一般来说,任何疾病均可影响性兴趣和性能力,但这种影响在多数情况下是短暂和可逆的。不过,很多疾病对性功能具有更为严重和长期的影响。


\section{第一节 内科疾病与性}

疾病影响性功能的原因包括:①造成直接的身体障碍(器质性原因);②触发导致性障碍的精神心理反应(心理性原因);③造成夫妻关系的改变,后者反过来对性功能产生影响(人际关系原因)。上述因素可以同时存在,也可以相互影响,比如疾病引起器质性损害,随之又引起精神反应导致完全障碍。

如肝肾疾患导致对代谢产物的解毒和排泄障碍,特别容易降低性兴趣。相反,虽然患者饱尝肺结核的痛苦,但性兴趣和性功能仍可能保持良好。凡影响垂体、性腺,或同时影响它们的内分泌疾患都会造成性问题。任何影响来自生殖器的外周感觉神经、分布到生殖器的内脏感觉神经、交感神经、副交感神经因素和躯体运动神经因素,或影响控制血管充血和高潮的脊髓反射中枢的因素均能损伤性反应。

患者得知自己患病后经常发生对疾病和死亡的恐惧、震惊、愤怒、忧虑、否认、逃避、抑郁以及退化等情绪反应,这些可改变或抑制性兴趣。患者的情绪反应通常表现在他们应付压力的性格方式上。此外,身体疾病引起的忧虑经常导致患者对目前和未来性功能的关注,这样既可使性欲增加,也可能导致性欲丧失,或可使性行为变得更为主动或被动。如:愤怒可导致更为积极的性行为,而忧虑则使性行为减少。疾病的另一个重要方面是疾病也可使其配偶感到孤独、被遗弃、愤怒和不耐烦,患者对疾病的焦虑,愤怒或退缩的反应,可使许多配偶感觉被拒绝,无助和迷惑。在考虑精神因素对性功能的影响时,另一个不容忽视的问题是它与器质性因素的相互作用。某些诉说勃起困难的男性糖尿病患者,阴茎夜间勃起试验可表现为部分或没有器质性损害。很明显,医生不能仅仅把性障碍当成是功能性的,而应把它同器质和精神因素一同考虑。

患病后处于危机中的人常有的抑郁、焦虑和紧张状态会损伤性功能,也有人认为严重的抑郁、焦虑和紧张可影响到中枢神经系统和神经递质,减少具有生物活性的雄激素供应而造成生殖和内分泌功能显著变化。近年来已证实不论是在昼夜间还是在更长的周期中,血睾酮水平都有显著波动,所以,它很可能与身心状态的变化有关。长期处于紧张状态的男性血睾酮水平会特异、持久和显著降低,当紧张解除之后,则迅速恢复正常。据推测,当人处于紧张状态时,下丘脑促性腺激素释放激素的分泌减少,垂体卵泡刺激素和黄体生成素分泌也相应减少,从而导致睾酮的产生受到抑制,进而使血睾酮水平明显下降,大脑性中枢的兴奋性降低、性欲减退、性反应减弱。紧张或抑郁也会造成血肾上腺皮质激素增高,其抗雄激素作用造成性欲减退。5-羟色胺等神经递质可抑制性兴趣和性反应,而去甲肾上腺素则有相反的作用。这些物质对人的情感和行为有重要作用,反过来,这些物质又受情感变化的影响。当人处于严重情感障碍(如抑郁)时,大脑儿茶酚胺水平下降并抑制性欲和性行为。当然上述心理生理研究的结果尚不是结论性的,还有待深入和系统的研究。从上面的叙述可以看到,疾病对性功能的影响是复杂多变的。

高血压病是以体循环动脉压增高为主要表现的临床综合征,其确切的发病原因尚不清楚。遗传作用只是可能的因素之一,行为因素如钠盐摄入过多,精神因素如长期的紧张压抑焦虑烦躁是最可疑的因素。高血压患者常伴有性功能障碍,尚不清楚究竟是高血压病本身,还是治疗高血压所用药物引起性功能障碍。有人对新诊断并且未接受治疗的男性高血压患者和经长期降压治疗的患者以及血压正常的男性作为对照进行了调查,结果发现正常血压组、高血压未治疗组以及降压治疗组的阳痿、勃起不坚以及不射精的患病率依次增高。未治疗高血压组与正常血压组比较,性功能障碍患病率增高1倍,高血压组患者接受降压治疗后,患病率没有改变。值得注意的是,给正常血压者冠以高血压病的诊断后不接受药物治疗,勃起功能障碍的发生率与高血压患者相似,提示高血压的诊断本身对患者性功能障碍有确切的影响。

当患者得知患高血压病时,他需要面对的现实是自己患了一种慢性疾病,并需要终生服药控制,情况可因激动而变坏,还可能伴发心肌梗死、卒中(中风)或肾衰竭等对生命产生严重威胁的疾病,因而引起某些患者不同程度的严重焦虑抑郁,这些多重的精神因素势必损害性功能。

正常的勃起功能需要健全的中枢及外周自主神经系统。原发性高血压患者的中枢神经系统单胺通道受累(或受影响),可能是未治疗的高血压患者性功能障碍的原因。这也可以说明为什么通过改变中枢或周围的神经递质而起药理作用的降压药物,具有影响性欲和勃起的副作用。有趣的是,血压正常的健康志愿者,在短期服用拟交感神经药物后,并未影响阴茎的血流及勃起能力。这是由于短期服药对勃起功能无影响,还是由于这类药物仅特异地影响高血压患者的阴茎勃起功能?目前尚不清楚。

阴茎勃起是一个动脉血流量增加,静脉血回流减少的血流动力学过程。阴茎勃起时,海绵体内压力接近平均动脉压(11.3~14.0kPa)。未治疗高血压患者的性障碍,可能有高血压本身存在的血流动力学方面异常改变的原因。降压治疗在降低系统血压的同时也可能影响阴茎海绵体内压力,从而导致阴茎勃起困难。如果上述观点成立,则血流动力学改变(原发的或药物引起的)就是某些或全部高血压患者性功能障碍的发病基础,那么对男性高血压人群进行降压治疗而又不影响勃起几乎是不可能的。动态心率血压记录表明在性活动过程中,心率和血压的波动幅度很大,尤其是高血压病。如果降压药物阻止了这种心血管系统的反应,就可能影响到阴茎勃起硬度的维持。然而,研究发现β-受体阻滞剂并不影响性交过程中心率血压的波动幅度。尚不清楚上述哪种因素起关键作用,只能说血管性、神经性以及精神性因素在这方面的作用较为重要。

螺内酯(安体舒通)除具有利尿作用外,还有抗雄激素作用,并能导致性功能障碍。利尿剂噻嗪类和氯噻酮也可引起勃起功能障碍,发生率为3\%~32\%不等。由于有这方面的副作用,有些患者便放弃治疗。利尿剂与其他降压药物联用造成的性功能障碍比单纯服用利尿剂更为多见。抗交感药物可造成勃起障碍和射精抑制延迟。α-甲基多巴、可乐定、普萘洛尔、胍乙啶、利血平均可有这方面副作用。

性功能障碍的诊治,应以对其发病机制的分析为依据,需遵循以下几条原则(1)新发病的高血压患者,在开始药物治疗前,应对患者的性生活史进行详细调查。(2)接受药物治疗的患者出现性功能障碍,不一定是药物不良反应所致。(3)高血压患者的药物治疗采取最小数量和最低剂量控制血压的原则。(4)对那些因降压药物的不良反应而停药的患者应定期随访。(5)对患者性生活史的询问,可参考以下几项进行:①勃起不坚:性交时、清晨、手淫时、在所有与性有关或无关的情况下;②性交准备阶段有充分勃起,而在插入前又软缩;③存在其他疾病;④射精障碍。(6)对上述问题肯定的回答,提示为器质性原因:①充分勃起出现于:性交准备阶段、手淫时、睡眠及清晨、无性要求时;②发病突然;③性欲减退(除外与内分泌有关的原因)。对上述总是肯定回答,提示有精神方面的原因。获取这些资料可为发现造成性功能障碍的原因提供线索,有可能的话,还可发现各种与高血压无关的导致性功能障碍的因素。下一步对患者体检时,尤其要注意有无外周血管疾患,睾丸大小、阴茎上有无纤维斑以及性欲低的征象。

参见“勃起功能障碍”章内论述。

冠心病可引起心绞痛,严重时可导致心律不齐、心肌梗死和心力衰竭,这些情况可严重影响患者的生活质量并可危及生命。据报道,1/3到1/2的男性患者在患心肌梗死后,性欲降低,性交次数减少,有人还发生阳痿,甚至在病后几年都是这样。然而也有些患者病后性欲增加。女性患者的情况与男性患者类似。

据报告心肌梗死恢复期的中年男子24小时Holter监测结果显示:在家与熟悉的伴侣进行性活动时,平均最大心率是97.5次/分钟,在性高潮时则达到117次/分钟。一般行走、蹬楼梯等活动的平均最大心率是120次/分钟。上述患者性生活时心电图的改变也与日常活动相似。对这组患者进行踏车试验,同时监测心率和血压,结果发现,在心率相当于过性生活的两倍时,患者的血压仍未上升到危险程度(22.5/11.9kPa)。这些资料清楚地表明,性活动并不需要比日常活动更多的能量,多数患心肌梗死但无心力衰竭或心律失常等并发症的患者,可以放心地恢复性生活。然而,有些学者对此持谨慎态度,他们发现尤其是在与不熟悉的性伴侣性交或在手淫时多见严重的心电异常,但他们观察的例数不足以得出可信的结论。日本一项研究报告发现,5559例突然死亡的病例中,有0.6\%(34人)发生在性交时,这34人中18人死于心脏病,并且有27人死于非夫妇间的性交或性交后。

目前更多地认为冠心病患者性功能的变化与患者情绪改变以及与配偶的相互影响有关。当患者患心肌梗死后,首先感到自己的生命受到严重威胁,时刻担心自己会就此死去。当急性期过去后,又会对自己未来的生活、工作以及性功能产生这样或那样的忧虑。患者的配偶同样会出现相似的畏惧、压抑和焦虑情形,他们尤其担心性交会加重病情,引起心脏病再次发作。因此变得过度谨慎,结果导致欲望丧失、性唤起障碍。有的医生对这些知识不甚了解,会让患者避免性生活;也有的医生知道禁止性生活会给患者带来更为紧张的情绪,而督促患者恢复性生活,事实上这也使医生成为患者性生活的观察者,造成患者的压力。有些患者为证明自己身体完好、强壮而更多地进行性活动,以此来满足自己的心理需要。然而一个力图证明自己功能完整的患者,可能变得对自己的性行为过于关注,从而感到压力并进入操作焦虑状态,结果引起勃起、阴道湿润和性高潮问题。

医生在治疗患者冠心病的同时,应及时发现患者对性的关心、恐惧以及功能失调的情况。对患者性问题的任何讨论,最好有配偶参加。有人对100例心肌梗死患者的妻子做过调查,几乎所有患者的妻子表示她们愿意与丈夫一起接受性指导,并希望有机会暴露她们对性的关心和害怕。基于上述原因,患者以及配偶的性需要均应予以足够的重视和考虑。事实上,对疾病的恐惧并不能阻止他(她)恢复性活动,更多的是影响到性生活质量和频率。在患心肌梗死后,一般应禁止性生活4~8周。在建议患者恢复性生活之前,首先应对患者进行心脏耐力检查。如果患者在踏车试验中能达到5~6km/h,或者经过Master二阶梯运动试验并没有出现心绞痛、心电图改变或血压异常升高的情况,那么患者的心脏便完全能够承受性交的活动量。此外,还应让患者了解在性活动中实际上消耗的能量与登上二层楼梯或轻松散步的差不多,同时向患者保证因性交而发生突然死亡的可能性很小。有研究表明,突然死亡者中仅有不到1\%的与性交有关。而其中大多数与婚外性生活、新奇的性生活方式,以及在此之前暴饮暴食有关。如果患者身体状况差,或在性交时发生持续心绞痛或心律失常,性活动应受限制。冠心病患者伴阳痿时的治疗参见“男性勃起功能障碍”,此外若患者平时需服用硝酸酯类药物,禁忌服用万艾可。

支气管———肺疾病是一类常见的疾病,每年死于这类疾病的人数已上升到继肿瘤和心血管疾病之后的第三位,这类疾病对性功能的影响也不容忽视。几乎半数呼吸系统疾病的患者主诉疾病影响了他们的性功能。呼吸困难或害怕发生呼吸困难,就像治疗呼吸疾病药物的副作用一样明显地阻碍了他们性功能的发挥。

性唤起和性活动将引起呼吸频率和深度的改变,有时可造成患者呼吸困难,容易被误认为是哮喘发作。性活动前使用香水或剃须用的乳液均可诱发哮喘。同样,性交后抽烟也能激发哮喘。从心血管功能消耗的观点看,性活动相当于在街上轻快地散步,这一用力水平可以使容易因锻炼而引起支气管痉挛的患者出现哮喘发作。在某些情况下,肺功能状况与性损害之间并没有显著联系,对于具有慢性阻塞性肺疾患的患者来说,不论其肺部功能状况如何,性功能不全总可能存在。当肺功能恶化时,性功能障碍的发生率增加,还可能引起阳痿。另一个容易为人们所忽视的生理因素是缺氧时血睾酮的降低。

有人认为精神因素在患者性障碍中较器质性因素更为重要。如哮喘可以使患者产生自卑、抑郁、焦虑及出现性功能障碍。目前尚不清楚这些表现和原发疾病的因果关系,很可能是相互影响,形成恶性循环。

呼吸病常用药物治疗,如茶碱、交感神经兴奋性支气管扩张剂、糖皮质激素等,可使患者失去吸引力,变得易于激惹。长期应用皮质激素以及慢性缺氧可使患者识别力下降,记忆能力受损。疾病本身及药物治疗的副作用,也可造成更多的性障碍。如果药物引起阳痿、逆行射精和性欲低下等副作用,调整药物剂量和给药次数可能会减轻有害作用。有些心理治疗对于终止焦虑与性功能损害之间的恶性循环是必要的,对于促进性伴侣之间的情感交流也是有益的。抑郁和焦虑引起的性功能障碍可以进行抗抑郁治疗,同时应认真监测药物对性功能潜在的副作用,必要时辅以心理治疗以缩短药物治疗时间。

据报道,1/2的男性以及1/4~1/2的女性在患慢性肾衰竭后出现性失调加重、性欲降低、性交频率减少、性高潮缺乏,而透析常不能减轻这些变化。事实上,1/3的男性患者和1/4的女性患者在尿毒症得到控制后,性障碍增加。家中透析对性功能的改善作用优于门诊透析,但许多患者的性问题依然存在。许多患者的性功能在肾移植后得以改善,但仍有20\%~30\%的男性患者没有恢复。一些研究提示,肾移植对于女性患者性功能的改进同样是有限的。

在男性肾衰患者,血睾酮降低而黄体生成素和促卵泡成熟激素水平则明显升高,患者睾丸萎缩、精子生成功能严重障碍。将血液透析患者与肾移植的男性患者进行对比发现,长期的透析并不能逆转上述血中激素水平的变化。慢性尿毒症患者血泌乳素水平持续增高,而且不能被血液透析所矫正。泌乳素可能通过其中枢作用,或影响睾酮的产生和周围利用,导致性唤起障碍、性欲降低,进而影响睾酮产生。尿毒症还常影响自主神经系统。副交感神经受损可影响勃起能力。早期的肾移植可逆转神经系统的病理改变。

慢性肾衰竭患者与其他患慢性病的患者一样,有抑郁、焦虑、自卑等心理改变。即使对患者进行透析或肾移植,解决了器质方面的问题,患者的精神因素仍可持续存在,器质性因素和精神性因素的相互作用与终期肾衰的性问题有关。

除对尿毒症患者进行积极有效的治疗以减轻疾病本身对性功能的有害作用外,对于血睾酮低下和缺锌的患者,可试用替代补充治疗。有人报道,长期服用枸橼酸氯米芬可纠正男性尿毒症患者的睾酮水平,并能增加性欲,改善性交能力。如果患者肾移植后,性功能仍无改善,心理治疗以及性治疗则更为重要。

尽管治愈率增加,但肿瘤仍是人类最常见的死因之一。肿瘤可以造成长期慢性疼痛和身体虚弱,对于肿瘤的放射、化学药物及外科治疗均可造成性功能方面的不良结果。患肿瘤的患者首先必须面对的是死亡,这对于大多数患者无疑是一种巨大的精神压力,恐惧、忧虑、绝望时刻袭扰着他们。即使治疗成功,许多患者仍担心复发。如果治疗仅仅减轻了病情或延缓了疾病的进展,患者仍可能生活在相当程度的疼痛之中,并承受着继续治疗过程中的不良作用。在疾病晚期,失去运动能力、长期住院以及死亡逼近的感觉可以完全地破坏性功能。肿瘤患者的配偶常有新的责任和忧虑,并可能经历性困难,特别是配偶在与患者进行性活动时有内疚感,更容易造成性欲的压抑,反过来,又影响了患者的性兴趣。肿瘤对特定的组织器官的侵袭和破坏,可造成直接的性损害。生殖器、前列腺、睾丸、会阴部和骨盆区域的肿瘤都可能影响性能力。有些颅内和神经的肿瘤由于直接影响性活动的高级中枢和神经反射,也可能造成性障碍。此外,垂体、睾丸、肾上腺肿瘤,也可引起性功能的继发改变。为了对患者的性活动给予切实可行的指导,医生应了解患者病前的性行为和性反应类型,正确估计肿瘤本身以及对其进行的放疗、化疗对性功能造成的影响,并与患者讨论。

一半以上的男性糖尿病患者最后出现性功能障碍,如阳痿、射精延迟、逆向射精及性高潮损害。原有正常性高潮的女性,35\%在患糖尿病后出现性高潮缺乏,或频率及强度的下降,性交疼痛也不少见。但无论男女,多数人患糖尿病后都没有性欲变化。多数患者在患糖尿病数年后才出现性功能障碍,并呈进行性加重。

糖尿病造成性功能障碍与阴茎背动脉等血管病变有关:动脉造影显示阴茎背动脉阻塞、搏动减弱或消失,女性阴蒂海绵体血管也可发生硬化性改变。糖尿病损害了阴茎海绵体的自主神经纤维、感觉神经和运动神经。糖尿病代谢紊乱一方面导致山梨醇旁路代谢改变而直接损害神经系统,另一方面可使供应神经的小动脉发生硬化,继而造成神经的营养供应发生障碍。神经的病理损害主要表现为神经脱髓鞘、糖原沉积、神经鞘膜细胞基底膜增厚以及轴索崩解。中枢神经、周围神经以及自主神经均可受累。女性糖尿病也可伴发神经病变,并对性功能产生影响,但不会发生男性那样严重的性问题。女性糖尿病患者的性高潮异常(包括需要更广泛更强烈的刺激才能诱发性高潮)和糖尿病的严重程度、病程、胰岛素用量、并发症如周围神经病的严重程度,都不成正比。糖尿病性周围神经病变引起的阳痿是不可逆转的,糖尿病患者多无性激素水平的明显改变。

糖尿病患者的性功能障碍除有明显的器质性因素参与外,精神作用也是一个不容忽视的因素。糖尿病患者可由于器质性原因发生性功能障碍,对此的忧虑担心又使性功能障碍进一步恶化。

性功能障碍一般不是糖尿病的首发症状,关键是确认性问题是继发于糖尿病还是与糖尿病无关。如果与糖尿病有关,则应该有周围神经受损的其他证据,如末梢性感觉障碍、足部振动觉消失、深感觉性共济失调、跟腱反射消失、双下肢无力等等。

糖尿病性周围神经病变对许多治疗基本上是无效的,对此,正如对糖尿病的其他并发症一样,严格控制血糖似乎有点效果,但效果不大,甚至需要几年才有一点疗效。具体的药物治疗参见“勃起功能障碍”章内论述。

10\%~20\%男性甲状腺功能亢进患者发生性欲亢进,特别是在病情较轻的患者。若青春期患本病同时伴性欲亢进,有可能被误诊为精神疾病。30\%~40\%的患者发生阳痿。5\%~10\%的女性患者出现性高潮和性唤起反应增强,15\%有轻度减弱。甲状腺功能亢进的一个显著特征是引起患者精神情绪的明显变化,这也可能是患者发生性功能变化的一个原因。甲状腺功能亢进患者的焦虑和抑郁也会影响到性能力,使其性欲低下。本病还可使妇女的月经量和月经周期发生改变。积极治疗甲状腺功能亢进是治疗性功能障碍的基础,必要时进行心理干预和支持。

80\%的男性病患者性欲减退,40\%~50\%有不同程度的阳痿。约80\%的女性患者性唤起困难。

垂体肿瘤(尤其是泌乳素瘤)以及垂体肿瘤的垂体、性腺以及其他内分泌腺的功能改变,对男女患者的性功能有很大影响。据报道,76\%的男性垂体瘤患者会发生性欲和性交能力的降低或丧失。性能力受损的程度与肿瘤体积有关,女性患垂体瘤也可出现性欲及性高潮反应减退。垂体瘤患者常有血泌乳素增高、睾酮低下及其他性腺激素的变化,给予溴隐亭治疗常可纠正上述激素异常并改善患者的性功能。雄激素替代治疗也有一定疗效。

大多数肥胖本身并不是病态,但它与许多病症的发生有密切关联,如心脏病、糖尿病、高血压等。肥胖者一般不愿活动、嗜睡、易疲劳,稍事运动即感心慌气短。男性肥胖者血清睾酮水平较正常人为低。相反,女性肥胖者血睾酮水平增高并常伴月经不调、多毛。肥胖者常对自己臃肿的身体感到不安、自卑甚至抑郁,因此而影响到社交。长久下去,青春期肥胖的患者就难以在人际交往中学会所需的行为技巧,尤其是与异性交往的障碍,将严重损害其性心理的发育,在客观上大大限制了他(她)们与伴侣的接触选择范围,并可能对后来的婚姻及性生活的满意程度造成影响。目前对肥胖症的治疗,主要是通过节食和增加锻炼,减肥药物、针灸、气功治疗也有一定疗效。对于上述方法无效而又强烈要求减肥的患者,可以考虑肠道改造手术和胃气球手术。对于肥胖症患者精神方面的问题以及性问题,也应给予必要的咨询和治疗。

在1987—1991年进行的我国城乡大规模调查中发现,脑血管病在城市死亡病因排序中占第二位,仅次于恶性肿瘤。北方很多城市中已经上升为第一位,例如北京和哈尔滨。大规模流行病学调查表明,我国脑卒中发病率为120~180/10万,病死率为60~120/10万,幸存者中3/4不同程度地丧失生活能力和劳动能力,中度致残者高达40\%,遗留后遗症的患者性能力或多或少地受到限制。

脑梗死或出血可导致一过性的性欲低下,但这些器质性损害不一定会永久影响性能力。卒中后最常见的性行为变化是男性与女性的性活动减少及性兴趣降低,性交频率、持续时间以及性前嬉戏均减少或缩短。男性卒中后,59\%~86\%面临勃起问题。女性卒中后,55\%面临阴道干涩。卒中前后,男性对性行为的满意程度由84\%降至34\%,女性由60\%降为31\%。由于右侧半身麻痹损害了左侧大脑优势半球,因而性欲降低更明显。性活动受到影响最少的是右侧病灶(左侧偏瘫)的女性。但是,一般人群中以右侧为主,右侧偏瘫使患者性活动更加不便,这很难说性欲的改变完全是由于左侧优势半球的器质性损伤引起的。

卒中偶尔导致性兴趣增加和性活动活跃,与之有关的区域是丘脑、下丘脑、扁桃体及杏仁体附近的颞叶底部前内侧区域。性欲亢进可伴随强制进食。曾有一男子的额叶底部内侧脑梗死,表现为公开手淫,并试图公开与男性及女性交合。颅内出血偶尔产生性欲亢进。前交通动脉瘤破裂,出血进入额叶基底部也可引起上述表现,即公开手淫及试图公开与男性及女性交合。

(郭公社)


\section{第二节 妇科疾病与性}

女性生殖器官是由不同的胚胎组织经过一系列的发育演变而成,如果在胚胎发育过程中受到某些内在或外来因素的干扰,均可导致发育异常,且常合并泌尿系统畸形。常见的生殖器官发育异常有:①正常管道形成受阻所致异常。包括处女膜闭锁、阴道横隔、阴道纵隔、阴道闭锁和宫颈闭锁;②副中肾管衍化物发育不全所致异常。包括无子宫、无阴道、痕迹子宫、子宫发育不良、单角子宫、始基子宫、输卵管发育异常;③副中肾管衍化物融合障碍所致异常。包括双子宫、双角子宫、鞍状子宫和纵隔子宫等。女性生殖器官发育异常很少在青春期前发现,患者常是在青春期因原发性闭经、腹痛或婚后因性生活困难、流产或早产就医时被确诊。

女性生殖系统包括生殖腺、生殖管道和外生殖器。

在胚胎第4~5周时,体腔背面肠系膜基底部两侧各出现2个由体腔上皮增生所形成的隆起,称泌尿生殖嵴(urogenital ridge),外侧隆起为中肾,内侧隆起为生殖嵴。胚胎第3~4周时,在卵黄囊内胚层内出现许多比体细胞稍大的生殖细胞,称为原始生殖细胞(primordial germ cell)。在胚胎第4~6周末,原始生殖细胞沿肠系膜迁移到生殖嵴,并被性索包围,形成原始生殖腺。原始生殖腺具有向睾丸或卵巢分化的双向潜能,其进一步分化取决于有无睾丸决定因子(testis-determining factor,TDY)的存在。目前研究认为,Y染色体短臂性决定区即为睾丸决定因子所在。如果不存在睾丸决定因子,在胚胎第8周时,原始生殖腺即分化为卵巢,故卵巢及其生殖细胞的发育和形成不是由于两条x染色体的存在,而是由于缺乏Y染色体短臂上性决定区基因所致。因此,Y染色体短臂性决定区在生殖腺分化中起着决定性的作用。

生殖嵴外侧的中肾有两对纵形管道,一是中肾管(也称午非管,Wolffian duct),为男性生殖管道始基;另一是副中肾管(也称苗勒管,Müllerian duct),为女性生殖管道始基。当生殖腺发育为睾丸后,在HCG刺激下,睾丸中的间质细胞产生睾酮,促使同侧胚胎中肾管发育为副睾、输精管和精囊;而睾丸中的支持细胞则分泌副中肾管抑制因子抑制同侧副中肾管的发育,从而使生殖管道向男性分化。如果生殖腺发育为卵巢,则中肾管出现退化。两侧副中肾管的头段形成两侧输卵管,两侧中段和尾段开始合并,构成子宫及阴道上段。在合并早期仍有中隔,使之分为两个腔,约在胎儿12周末中隔消失,成为单一内腔。副中肾管最尾端与尿生殖窦(urogenital sinus)相连,并同时分裂增殖,形成一圆柱状实体,称阴道板(vaginal plat)。阴道板随后由上向下穿过,形成阴道腔。阴道腔与尿生殖窦之间有一层薄膜即为处女膜。

胚胎初期的生殖腔分化为后方的直肠与前方的尿生殖窦,尿生殖窦两侧隆起为泌尿生殖褶(urogenital fold)。褶的前方左右相会合呈结节形隆起,称生殖结节,以后长大称初阴。褶外侧隆起为左右阴唇阴囊隆起。生殖腺为卵巢时,约在第12周末生殖结节发育成阴蒂。两侧的尿生殖褶不合并,形成小阴唇,左右阴唇阴囊隆起发育成大阴唇。尿道沟扩展,并与尿生殖窦下段共同形成阴道前庭。生殖腺为睾丸时,在雄激素的作用下,初阴伸长形成阴茎,两侧的尿生殖褶沿阴茎的腹侧面,从后向前合并成管,形成尿道海绵体部,左右阴唇阴囊隆起,移向尾侧,并相互靠拢,在中线处连接呈阴囊。外生殖器的分化虽受性染色体支配,但若在其分化以前切除胚胎生殖腺,则胚胎不受睾丸或卵巢所产生的激素影响,其外生殖器必然向雌性分化;反之,若给予雄激素,则向雄性分化。以上事实说明外生殖器向雌性分化是胚胎发育的自然规律,它不需雌激素的作用,而向雄性方向分化则必须有雄激素即睾酮的作用。虽然外生殖器向雄性分化依赖睾酮的存在,但睾酮还必须通过外阴局部靶器官组织中5α-还原酶的作用,衍化为二氢睾酮,再与外阴细胞中相应的二氢睾酮受体相结合后,才能使外阴向雄性分化。因此,即使睾丸分泌睾酮,但外阴局部组织中缺乏5α-还原酶或无二氢睾酮受体,外生殖器仍将向女性分化,表现为两性畸形。

处女膜闭锁(imperforate hymen)又称无孔处女膜,临床上较常见,系尿生殖窦上皮未能贯穿前庭部所致。处女膜闭锁的女婴在新生儿期多漏诊,如果患者子宫阴道发育正常,无孔处女膜女婴在出生时,阴道内分泌物无法排出,故以外阴部清洁,无分泌物为特征,但很少有助产人员注意到。偶见幼女因阴道内过量的黏液潴留,以致处女膜向外凸出而被发现,而绝大多数患者在青春期才被发现。由于处女膜闭锁,少女至青春期初潮时,经血无法排出而不见月经来潮,同时出现逐渐加重的周期性下腹部疼痛,最初血积在阴道内,反复多次月经来潮后,逐渐发展至子宫积血、输卵管积血甚至腹腔内积血。由于输卵管伞端多因积血而粘连闭锁,故月经血进入腹腔者较少见,可并发上行感染。患者自己可在下腹正中部扪及增大的包块,严重者因包块压迫伴便秘、肛门坠胀、尿频或尿潴留。如已结婚,则发生性交困难。妇科检查时可见阴道口外有一层膜样组织覆盖,向外膨隆,表面呈紫蓝色,无阴道开口。当用食指放入肛门内,可扪及因积血而扩张的阴道形成一球状包块向直肠前壁突出。行直肠腹部扪诊可在下腹部触及位于阴道包块上方的另一较小包块(为经血潴留的子宫),压痛明显。如果用手往下按压此包块时,可见处女膜向外膨隆更明显。盆腔B型超声检查可发现子宫及阴道内有积液。在处女膜处进行穿刺,若抽出黏稠深褐色或暗红色血液即可确诊。

确诊后应即在静脉麻醉或骶麻下手术。先用粗针穿刺处女膜正中膨隆部,抽出褐色积血后,即将处女膜作“X”形切开,引流积血,同时切除多余的处女膜瓣,使切口呈圆形,再用3-0肠线缝合切口边缘黏膜,以保持引流通畅和防止创缘粘连,术中注意认清解剖关系。积血基本排净后,常规检查宫颈是否正常,但不宜进一步探查宫腔,以免引起上行性感染。术后置导尿管l~2日,外阴部置消毒会阴垫,每日擦洗外阴1~2次,直至积血排净为止。术后给予抗感染药物。

处女膜闭锁的患者多由于青春期月经不能排出,伴有周期性腹痛而就医,经过简单的手术治疗后即可痊愈,一般不会对性生活造成不利影响。对于青春期延迟的患者,月经来潮推迟,可能会因处女膜闭锁而导致性生活困难。

先天性无阴道(congenital absence of vagina)为双侧副中肾管发育不全的结果,双侧副中肾管会合后未能向尾端伸展成阴道,使直肠与膀胱、尿道间无空隙,从而导致阴道缺如。先天性无阴道患者几乎均合并无子宫或仅有始基子宫,但卵巢、输卵管一般发育正常。患者多因青春期后无月经来潮,或因性交困难而就诊。检查可见外阴和第二性征发育正常,但无阴道口或仅在阴道外口处见一浅凹陷,有时可见到由尿生殖窦内陷所形成的约2cm的短浅阴道盲端。肛查和盆腔B型超声检查无子宫,约15\%合并有泌尿道畸形。

治疗先天性无阴道的唯一措施是通过手术或非手术方法制造人工阴道,恢复性交功能。人工阴道成形术一般在结婚前半年进行,对少数子宫发育正常者,应该在初潮后尽早行阴道成形术,引流宫腔积血,并将人工阴道与子宫相连,以建立经血排出通道,保留生育功能,无法保留者应切除子宫。

非手术方法是用压迫法形成人工阴道,安全、简便,适合于有短浅阴道盲端的患者,对阴道外口处无明显浅凹隐窝的患者效果较差。操作时患者取仰卧位,双腿屈曲略分开,垫高头部,在阴道前庭处女膜凹陷处,用直径0.8~1cm的光滑木质或玻璃圆棒,向后、向内伸入及顶压,每次保持约30分钟,每日2~3次。大约1周后可形成较深凹陷,改变顶压方向为稍向下接近水平方向顶压,每次保持约30分钟,每日2~3次,3~4周后可形成6~7cm腔道。此后开始夜间佩戴阴道模具,外端用丁字带固定。模具先由小至大,直径可由1.5cm逐渐增加至3.5cm,长度亦逐渐延长。一般经过6~8周后,顶压形成的前庭凹陷可达8cm左右,即能满足性生活的需要。对于已婚患者,可指导用丈夫阴茎进行顶压,经过较长时间后也可以使前庭凹陷变深。顶压法形成的人工阴道虽然可以满足性生活的需要,但分泌润滑功能差,性生活时需使用润滑剂,性快感也较差。

手术方法即阴道成形术,术式繁多,各有利弊,各种手术的共同原则是:①在直肠和膀胱间分离出足够的空间;②在穴腔创面镶嵌各种移植物(包括:羊膜、腹膜、生物补片、皮片、皮瓣、结肠、回肠等);③需要在愈合的收缩期持续扩张阴道。常见的术式有:

(1)游离皮瓣阴道成形术:手术简单,成功率高,但人工阴道没有分泌物,干涩,性生活时需要使用润滑剂,满意度稍差,需要损伤正常组织,供皮区瘢痕影响美观,术后需放置模具,目前已少用。

(2)羊膜法阴道成形术:手术简单,不损伤脏器或重要组织,但上皮形成慢,完全愈合需1年左右,羊膜仅为暂时支架,容易感染,术后需长时间佩戴模具,现已少用。

(3)肠管(乙状结肠、回肠)代阴道成形术:人工阴道的位置与正常阴道相同,肠管的分泌物可以保持人工阴道湿润,性生活时不需要使用润滑剂,术后阴道不会狭窄,不需要使用模具,术后1个月即可开始性生活,性生活满意度很高。但分泌物有时会有异味,条件致病菌在黏膜破损或机体抵抗力低下时致病,需注意清洁,避免损伤。

(4)外阴阴道成形术(又称Williams手术):利用患者阴唇,在外阴部形成一袋鼠样皮瓣袋。手术极简单,安全、成功率高,术后短期即可有性生活,无需阴道模具。但人工阴道分泌物少,性生活时需使用润滑剂。仅适用于外阴发育良好的妇女,术后易患尿路感染。

(5)腹膜法阴道成形术:1933年由Ksido首创,经不断改进,采用腹阴联合手术,用盆腔壁腹膜覆盖穴腔壁。操作简单、成功率高,成形后的阴道壁柔软湿润,弹性好。最近采用腹腔镜下腹膜代阴道术式取代传统的开腹游离腹膜,使手术更微创,目前已成为主要术式,但也存在术后需佩戴模具的缺点。

阴道成形术是在尿道和直肠之间建立一人工阴道以解决先天性无阴道患者的性生活问题。马斯特斯和约翰逊(1966年)研究了7例阴道成形术后的人工阴道在性反应周期中的变化,发现无论人工阴道壁是怎样组成的,它们的反应类型都一样。并以人类女性性反应周期的四个时期作为描述构架来阐述人工阴道在性反应时的变化。

(1)兴奋期:在躯体性或心理性的性刺激发动30~40s后,人工阴道壁上出现黏液样润滑物,与正常阴道相比,人工阴道产生对性交足够量的润滑物质通常要花更长的时间。随着活跃的性交,润滑物质生成同样达到充分和迅速的润滑作用。7例妇女中有2例在对性刺激的反应中,其润滑程度比许多正常阴道更广泛、更迅速。润滑物质的来源与正常阴道一致。当产生性紧张时,环绕人工阴道管的静脉丛显著扩张,其上皮与正常阴道上皮一样具有双向膜的功能。体液通过上皮渗透至阴道,形成润滑物,润滑物的成分很可能与正常阴道润滑物相似。当兴奋期持续发展时,和正常阴道一样,人工阴道的上2/3部分出现延长、扩张。阴道可由6cm延长至8cm,阴道中段以上宽度由1.5cm扩张到3.5cm。相继以缓慢、无张力的方式松弛下来。由于没有正常阴道壁良好的移动性,人工阴道扩张及增长的能力都发展较慢。与正常阴道一样,人工阴道除了在受到明显性紧张刺激的情况下,其前后壁总是贴合在一起的。

(2)平台期:当性反应达到平台期水平时,阴道局部的充血很明显。其反应相当强烈,以致近50\%的阴道管中心腔被堵塞。如同正常阴道管外1/3建立的高潮平台,人工阴道的外1/3也持续建立高潮平台。在平台期,阴道润滑物仍继续显著增加,而正常阴道润滑物质的最高生成率仅限于在性反应周期的兴奋期,平台期润滑物质的生成稍稍减慢。与正常阴道相比,人工阴道管的中心直径和长度在平台期增加很少。当性兴奋进展到平台期后期时,具有人工阴道的妇女小阴唇性皮质变为鲜红色。当出现这一征象时(假如有效的性刺激持续下去),表明具有人工阴道的妇女的性高潮即将到来,这无异于具有正常阴道的妇女。

(3)高潮期:无论是人工阴道还是正常阴道,特征性性高潮的生理表达都是从规律性的由血管充血所形成的高潮平台节律性收缩的发动开始的。在起始时收缩频率大约为0.8秒/次。当性高潮体验进展时,收缩间隔变慢。具有人工阴道个体的性高潮反应包括整个会阴体的不随意收缩。不仅阴道的外1/3,而且直肠的外括约肌及下腹部的肌肉都在性高潮表达时收缩。浅表和深部的横向会阴部的肌肉,如球海绵体肌、提肛肌和腹直肌下部是首要反应的肌肉。人工阴道亦具有随意的和不随意的肌肉收缩控制能力。人工阴道性高潮的一个十分独特的反应是整个阴道壁颜色的显著变化,这一特征不会在正常阴道中出现。阴道壁强烈的变色似乎是在性高潮中突然发生的,人工阴道在性高潮中颜色变化的现象已用电影摄影术记录下来。在未经性刺激的状态下,人工阴道上皮颜色变化呈灰色到淡紫红色。在性高潮体验中,上皮颜色变成鲜红色。这种红色的迸发在其发生的突然性和颜色的鲜明性上都是令人吃惊的。颜色改变越显著,性高潮体验越强烈。这种性高潮的阴道壁颜色反应还从未在正常阴道的性反应周期的直接观察中确认过。这种性高潮中血管显著充血反应的触发机制尚未阐明。

(4)消退期:根据已建立的正常阴道的复原模式,人工阴道消退期的退行性改变也以原来发展过程的顺序反向进行。第一个反应是阴道外1/3(高潮平台)局部血管充血消失和小阴唇性皮肤变色的复原。实际上,在具有人工阴道的妇女中,性皮肤变色的消失比高潮平台的消失更快。人工阴道壁缓慢地回缩到未受刺激的基准状态。人工阴道壁回复到未受刺激的贴合位置的速率及阴道扩张反应的恢复速率要比正常阴道慢得多。性高潮颜色改变将缓慢消失。在具有人工阴道的妇女中,残存的变色常常在经历由性紧张到高潮释放后10~15分钟仍可见到。一般而言,对有效性刺激的反应与正常阴道比较,人工阴道在速度和强度方面都有所延迟。但是,成功的人工阴道能迅速而有效地进行性生活。人工阴道患者往往存在一定的性心理问题,需要进一步关注及研究。

阴道横隔(transverse vaginal septum)是由于双侧融合的副中肾管尾端与尿生殖窦相连处未贯通或仅部分贯通所致。横隔可位于阴道内任何部位,但多位于阴道上1/3与中2/3的交界处,亦有位于阴道其他部位者。横隔厚度差异很大,有的薄如纸张,有的较厚,可达1~1.5cm。阴道横隔有无临床症状,首先取决于横隔上有无小孔,其次为横隔位置的高低。多数横隔的中央或侧方有一小孔,月经血可自小孔排出,多无临床症状。少数为完全性横隔,中央没有小孔,青春期后经血潴留在横隔之上的阴道内,表现为原发性闭经,周期性下腹疼痛且进行性加重,下腹包块,检查发现第二性征发育正常,外阴正常,肛腹诊可扪及经血潴留形成的包块。不全横隔位于阴道上段者多无症状,由于不影响性生活,故发现较晚,常系偶然或其他疾病检查时不能窥见宫颈而发现。位置较低者少见,多因性生活不满意而就医。

一旦确诊后可行手术切开并切除其多余部分。切缘糙面缝合,术后短期放置模型,以免再度产生粘连。临产时发现横隔阻碍先露下降,可在子宫颈口近开全或横隔被胎头撑得较薄时再行切开,分娩后应检查切口处有无撕裂、出血以决定是否需行缝合;若横隔较厚,且部位较高时,可行剖宫产术,以后再处理阴道横隔。

双侧副中肾管会合后,若中隔未消失或未完全消失便形成阴道纵隔(longitudinal vaginal septum),根据中隔消失的程度可分完全纵隔及不完全纵隔两种。中隔未消失会导致完全纵隔,形成双阴道并常伴双子宫和双宫颈。中隔部分消失则导致不完全纵隔,有时纵隔偏向一侧形成斜隔,导致该侧阴道完全闭锁,因经血潴留可形成阴道侧方包块。绝大多数阴道纵隔无症状.有些是婚后性交困难才被发现,另一些可能晚至分娩时产程进展缓慢才确诊。当一侧阴道较宽时能满足性生活的要求,一般不发生性交困难。如两侧阴道均狭窄,则会产生性交困难或性交疼痛。

当纵隔影响性交或斜隔妨碍经血排出时,应将其切除。切除纵隔时要达到其顶端,防止术后形成镰状瘢痕,引起配偶性交疼痛。不完全性纵隔可影响性生活、妊娠或分娩,应在非孕时及早手术,切除纵隔,缝合创面,以防粘连。如在临产时发现纵隔阻碍先露下降,可沿隔的中央部及时切断,分娩后再将多余黏膜瓣切除,缝合黏膜边缘,以防伤口出血。

可使性生活发生困难,轻度强直者可行机械扩张术,逐步扩大阴道口,使强直的处女膜松弛。少数人处女膜组织坚韧,扩张困难,需要在麻醉下作处女膜切开,切断强硬组织,此手术宜在发育成熟后进行。

阴道闭锁或阴道狭窄系由于双侧副中肾管会合后最末端未贯通或仅部分贯通所致,多发生于阴道下段,其上为正常阴道。前者称为阴道闭锁(atresia of vagina),后者称为阴道狭窄。阴道闭锁者常在青春期后出现周期性腹痛,但无月经来潮,肛查时处女膜有孔,可扪及因积血形成的包块,其位置较处女膜闭锁时为高。治疗时应尽早手术。切开闭锁段阴道,排净积血后,阴道创面利用外阴皮瓣覆盖,术后应定期扩张阴道以防挛缩。阴道狭窄者无经血排出困难,但性交困难,治疗可根据狭窄的程度行阴道成形术。

阴蒂是女性外生殖器的重要组成部分,它位于两小阴唇顶端的联合处,与男性阴茎海绵体相似,具有勃起功能。它分为三部分:前端为阴蒂头,富含神经末梢,极敏感;中部为阴蒂体;后部分为两个阴蒂脚,附着于各侧的耻骨支上。仅可见阴蒂头,其直径6~8mm。阴蒂常见的疾病有:炎症、粘连、瘢痕、白色病变、萎缩及增生肥大、肿瘤浸润等。阴蒂疾病常见的病因有:外阴部创伤、感染、肿瘤、萎缩性疾病及全身各种内分泌功能障碍,如肾上腺皮质功能亢进、卵巢功能低下、卵巢含睾丸细胞瘤及卵巢门细胞瘤等卵巢男性化肿瘤等,当阴蒂包皮粘连或过长时,可行部分包皮切除术以暴露其头部。

阴蒂疾病与性功能是现代性医学重要的研究课题之一。阴蒂是女性外生殖器具有独特功能的独特性器官。阴蒂既是性刺激的感受器,也是性刺激的换能器。阴蒂的存在使人类女性具有从启动到提高性紧张水平的生理功能的器官系统。阴蒂的各种病变可直接影响性功能,发生性功能障碍。阴蒂或阴唇肥大也是女性外生殖器男性化的表现,与外源性或内源性雄激素有关,常见于药物作用或两性畸形、肾上腺皮质增生或肿瘤、卵巢男性细胞瘤等。卵巢内肿瘤细胞都能分泌过多的雄激素而导致阴蒂肥大,可影响一侧或两侧小阴唇,一般在青春期后才表现出来。处理时应先找出病因,对症治疗,如阴唇过长影响行动或因摩擦发生水肿和溃疡可行部分切除术。有人认为经常手淫的妇女会因反复地、长时间地对阴蒂进行机械刺激,导致局部长期充血水肿及结缔组织增生,结果也出现阴蒂肥大。临床工作中观察到的手淫导致阴蒂明显肥大的病例,她们每日都有反复长时间的自我刺激行为。阴蒂肥大患者往往性欲亢进、性冲动明显、性交频率较高,常常主动积极的追求性满足。但也有不少专家持不同观点。因为人们总说手淫使男子阴茎短小,为何反而使阴蒂肥大呢?上述临床病例是否排除了激素的影响呢?人们无法得到证实。近年来还大量研究了阴蒂包皮过长的问题,认为这可能是影响女性性反应的一个原因。有人指出此症较为常见。但马斯特斯与约翰逊的报道称,2000例接受检查的妇女中阴蒂包皮过长者仅有2例。亦有资料提出阴蒂包皮切除术后能增强女性性兴奋,但迄今没有类似资料证实这一点。然而,国外一些妇女杂志把阴蒂环形切开术描述为一种对女性性行为具有奇迹般效果的方法。有许多妇女怀着体验一种新形式的性反应及领会新的性感的欲望而要求手术。外阴部白色病变如萎缩性硬化性苔藓使外阴皮肤干痒、变白、皲裂、伴有疼痛或烧灼感。由于萎缩性病变使阴道口缩小,引起性交疼痛及性交困难,甚至排尿障碍。当病变累及阴蒂,使阴蒂包皮增厚或萎缩,表皮皲裂、弹性消失、敏感性减低,从而影响其作为性刺激的感受器和换能器的功能。往往需较强性刺激方能达到与患病前同等的性兴奋及性紧张状态。如因阴蒂皲裂感染疼痛,则性欲低下,性反应减弱。长期久治不愈的外阴白色病变,常常造成配偶性关系紧张,感情不和,甚至离异。

女性生殖器损伤,主要发生于外阴、阴道及子宫体。外阴、阴道因各种原因损伤后,如未经适当处理,正常解剖关系受到破坏,作为女性性表达功能、性工具的阴道,则产生形态及功能变化,直接影响性功能。同时,性交又是外阴、阴道损伤的重要原因之一,两者关系紧密,互相影响。现将常见的外阴、阴道损伤及损伤引起的泌尿生殖瘘分述于下。

包括处女膜、阴蒂、大小阴唇及会阴体损伤。外阴损伤的主要原因是产伤。当产妇会阴体厚而宽、组织水肿、胎儿过大或胎儿娩出迅速而外阴组织扩张欠充分,施行产钳、胎头吸引器、臀位牵引时未做适当会阴切开,则可引起会阴撕裂伤。会阴撕裂伤分为Ⅰ度、Ⅱ度及Ⅲ度。当会阴Ⅲ度裂伤,肛门括约肌断裂未修补,造成陈旧性会阴Ⅲ度裂伤,则不能控制排便,生活十分困难。因不慎跌伤或外伤,外阴部骤然触于有棱角的硬物上,或在骑自行车途中发生车祸,自行车座垫冲撞外阴部,也会发生严重的外阴及阴道损伤,甚至阴道、尿道完全断裂,大量出血,需要抢救处理。初次性交使处女膜破裂,多能自愈。当处女膜厚而紧,或性交粗暴,处女膜撕裂伤大而深,引起多量出血,应缝合止血。据国际计生联(IPPF)资料报道,在非洲及亚洲部分地区、欧洲、澳洲及北美的部分移民人群中,一百多万妇女及女孩遭受着女性生殖器官的摧残。人们为了宗教、文化及社会原因,给1岁至青春期少女行女性环切术(PGM)。手术包括不同程度的切除阴蒂、大阴唇、小阴唇,有时甚至将阴道口闭合起来,仅留一小孔以供排尿及排出经血。手术通常在没有麻醉及用未消毒的剃刀、刀子及碎玻璃的条件下进行。受术者除要承受巨大疼痛、出血及感染的危险外,均残留外阴部严重瘢痕及持续性泌尿系统感染,造成终身外生殖器残疾。

因分娩造成会阴裂伤,如果能够及时修补,对分娩后性功能无不良影响。会阴Ⅱ度裂伤、提肛肌撕裂而未缝合,会导致产后盆底支持组织功能减弱、阴道前后壁脱垂、子宫脱垂、张力性尿失禁等。会阴Ⅲ度裂伤,肛门括约肌完全断裂,如未恢复其正常解剖关系,形成陈旧性会阴Ⅲ度裂伤,则发生大便失禁、直肠阴道瘘等。此种患者因不能控制排便,特别在腹泻时,衣物污染,臭气难闻,在心理上产生极大压力,往往表现为性格孤僻、烦躁易怒、少言寡欢。为了减少排便,食量减少,面目清瘦、两眼无神。她们有严重的自卑心理,不愿与配偶相处,回避性生活,性欲抑制,性反应低下,配偶亦因性格不合,性生活频率稀少,无法得到性满足而逐渐疏远,影响夫妻感情,继而分居及离婚。此种情况主要发生在医疗条件差的偏僻农村,如发现此类患者,应尽早手术,术后患者体力恢复、精神状态与术前相比将判若两人,性生活、夫妻关系恢复正常。初婚性交造成处女膜损伤大出血的患者,除了要适应新婚之夜的变化外,疼痛及大出血使她们在精神上及肉体上受到较大刺激。来医院时往往表现为表情呆滞、沉默寡言、面色苍白,反应迟钝。在手术缝合止血后,医生还需要给患者进行性知识咨询及必要的精神安慰,防止因初次性交的遭遇,产生焦虑而发生性功能障碍。女性环切术对妇女在精神及身体是一种摧残,是破坏人的完整性的暴行,给受术者在心理及性心理的发展方面造成深远影响,对其成年后的性功能造成沉重打击。国际计生联坚决反对实施女性环切术。女性环切术对性功能的影响程度,取决于手术的范围、切除组织部位、是否有感染、瘢痕大小及残疾的程度。如切除组织较少,瘢痕软化后在成年期不致影响性生活,如切除范围广、感染严重、瘢痕较大,则造成严重性交困难或性交疼痛。必须进行整形手术,松解瘢痕,尽力恢复局部解剖关系,才能建立较正常性生活,解除终身痛苦。

由于阴道难产及手术助产造成阴道裂伤而未缝合者,在阴道穹隆部或阴道部产生环形瘢痕,可致阴道狭窄、变短,有时会形成纵形瘢痕。性交粗暴、产后或老年妇女雌激素水平低致阴道黏膜菲薄或组织弹性差,或因手术后阴道缩短,在性交时引起阴道损伤,其损伤部位常见于阴道后穹隆。伤口多呈新月形,环绕于子宫颈后方。损伤能造成严重出血及感染。有时穿透腹膜,造成腹腔内大出血。分娩过程中若发现阴道撕裂或性交引起的阴道穹隆裂伤,应立即缝合止血,及抗感染治疗。产后及老年妇女宜同时补充雌激素,促进上皮增生愈合。

在阴道穹隆或阴道部形成环形瘢痕,可使阴道变短,发生性交困难及性交疼痛。在阴道穹隆部有纵行质硬瘢痕,性交过程中会令配偶感觉疼痛或不适。严重的阴道瘢痕需在麻醉下分离粘连。手术后填塞阴道纱布,必要时安放阴道模型,防止术后阴道壁再度粘连及瘢痕挛缩,引起阴道狭窄。

造成泌尿生殖瘘最主要的原因是产伤。有人报道,在357例尿瘘患者中,产伤占95.8\%。在分娩过程中,如果存在骨盆狭窄、胎头长时间压迫、局部组织坏死、急产、胎儿偏大等情况都有可能导致泌尿生殖道的损伤,如果产时缝合不及时或者产后伤口愈合不良,均可造成泌尿生殖瘘。在妇科手术中,由于操作不当也可损伤膀胱及输尿管。广泛性子宫切除术损伤输尿管机会较多,占0.38\%~1.4\%。子宫颈癌或阴道癌放疗时,也会造成局部组织坏死而形成尿瘘。一般多在放疗后1~2年发生,也有晚至20年后才发生者。由于尿瘘解剖部位不同,可分为尿道阴道瘘和膀胱阴道瘘等。泌尿生殖瘘的主要症状为漏尿,由于瘘孔的位置及大小不同,漏尿亦表现不同。瘘孔在膀胱三角区或在膀胱颈部,尿液日夜外溢。位于膀胱三角区以上的高位膀胱阴道瘘站立时可暂时无漏尿,平位则漏尿不止。膀胱内瘘孔小,则于膀胱充盈时方出现不自主的漏尿。位于尿道下1/3段的尿道阴道瘘,一般能控制排尿,但排尿时,尿液大部均经阴道流出。输尿管阴道瘘除能自主排尿外,同时有尿液不自主地阵性自阴道漏出。因为长期漏尿,外阴部、大腿内侧及臀部皮肤发红、增厚,并有丘疹及表浅溃疡,呈现湿疹,发生外阴瘙痒及烧灼痛。经过询问病史、妇科检查、亚甲蓝试验、膀胱镜、肾图及泌尿系统造影检查,明确诊断后,应早期进行手术,修补瘘孔。手术后要注意护理,禁性生活3个月,泌尿生殖瘘手术后容易复发,如复发则会进一步增加再次修补术的难度。

泌尿生殖瘘患者由于日夜尿液淋漓,尿臭四溢,昼间羞于与人共处,闭门不出。夜间床褥潮湿,难以安眠,以致痛苦万分,抑郁寡欢,自卑心理严重。对性生活全然不感兴趣,性欲低下,性高潮障碍。夫妻关系受到不良影响,无法进行性生活及相处,以致分居或离婚。个别患者因不堪长期肉体上的折磨和精神上的打击而萌轻生念头。手术治愈尿瘘后,患者精神及健康状况好转,性生活恢复正常。

子宫处于正常位置有赖于盆底肌肉、筋膜以及附着于子宫的韧带支持,站立时子宫呈前倾位,子宫纵轴与阴道纵轴呈90~100°的夹角。子宫颈外口位于坐骨棘水平以上。阴道前壁邻膀胱、尿道、中间隔以耻骨膀胱子宫颈筋膜,阴道外口与直肠间为会阴体。如因分娩、产伤或产后过早参加体力劳动或慢性咳嗽,体质虚弱,子宫圆韧带、主韧带、子宫骶骨韧带松弛、薄弱,提肛肌及耻骨尾骨肌过度伸展或撕裂,耻骨膀胱子宫颈筋膜及直肠阴道筋膜受损,子宫、膀胱、尿道下垂均可造成子宫脱垂,膀胱阴道前壁和直肠阴道后壁膨出。如尿道内括约肌松弛,可在大笑、咳嗽用力等增加腹压的情况下有尿漏出,称压力性尿失禁。子宫脱垂临床分Ⅲ度,Ⅰ度时子宫颈在坐骨棘水平以下,阴道口内;Ⅱ度时子宫颈在阴道口外,但子宫未完全脱出;Ⅲ度时子宫颈及宫体完全脱出于阴道口外。子宫脱垂的临床表现是劳累后阴道口脱出肿物,卧位时回缩入阴道。因局部摩擦,宫颈部产生溃疡、糜烂、感染及出血。阴道壁角化增厚,患者不能自由活动,十分痛苦。有膀胱及前阴道壁膨出者可发生排尿困难、尿潴留,需还纳子宫后才能排空小便。子宫脱垂患者常有腰部酸痛及下坠感。治疗方法分为非手术治疗及手术治疗。Ⅰ度、Ⅱ度子宫脱垂,病情较轻者注意防止劳累,多锻炼身体及盆部肌肉,病情会逐渐好转。病情较重及Ⅲ度子宫脱垂者,宜行阴道前后壁修补术、阴道前后壁修补术及宫颈切除术(Manchester手术)或阴道子宫切除术。体质弱、年龄大或有内、外科疾病不适于手术者,可放置子宫托治疗。

子宫脱垂Ⅰ度、Ⅱ度,轻度膀胱阴道前壁或直肠阴道后壁膨出时,性功能没有明显改变。子宫脱垂Ⅲ度,重度阴道前后壁膨出,外阴脱出物及排尿困难等给患者精神上造成压抑而避免或拒绝性生活。宫颈部溃疡及分泌物增多,出血使配偶双方对性生活产生顾虑。阴道壁高度角化增厚,在性唤起过程中阴道润滑作用明显减低,虽然阴道宽松却有干涩感。在性反应过程中,由于子宫及盆底的筋膜、韧带、肌肉松弛及陈旧性损伤,使性紧张达到平台期时,子宫不能上升至假骨盆,由阴道壁扩张在宫颈横断面形成的帐篷现象则不明显。阴道外1/3充血、肿胀虽然存在,但陈旧性裂伤的提肛肌不能满意地协助高潮平台完成紧握样动作,使性交时配偶双方产生不满足感。于高潮期,提肛肌的薄弱无力造成不自主的、规律的强烈收缩不显著,从而影响性关系,甚至离婚。中老年子宫脱垂患者,虽然病情并不十分严重,不妨碍健康及日常生活,但亦有夫妻双方坚决要求阴道及会阴修补者,术后患者均产生外阴部较紧的舒适感,性生活明显改善,家庭美满。

1.为了证明阴道在女性性功能及性反应中的作用,国内、外进行了广泛研究。组织学观察发现,在阴道上皮内及上皮下有神经末梢分布,并围绕着阴道血管周围形成神经丛,尤其在膀胱与阴道前壁之间神经纤维的分布更加丰富。1959~1989年格拉夫伯格(Grafen-berg)等先后证实,14\%的妇女在没有情欲的情况下能感觉到器械对阴道的触觉刺激,90\%能感觉到器械对阴道的压力。在有情欲的情况下,则更为敏感。1989年魏吉曼(Weijman)等用电刺激的方法进行研究,证明了以前在组织学上或临床上提出的关于阴道前壁极为敏感的观点,并且提出,阴道壁并不是只有某一点比较敏感,而是沿着阴道前壁中线的黏膜和黏膜下的深层组织均比其他部位的阴道壁敏感。1981~1990年爱迪哥(Addiego)等指出,10\%的妇女在性高潮时有排液现象,虽然不了解液体究竟是来源于阴道、巴氏腺、尿道内腺体(intraurethral glands)还是尿失禁,但认为排液是由于刺激阴道前壁中、下1/3交界处的G点或阴蒂所致,刺激阴道前壁G点产生的性高潮生理模式与刺激阴蒂产生的模式并不一致。

2.阴道前壁修补术,要分离阴道前壁与膀胱筋膜间的疏松组织,上推膀胱,切除部分阴道前壁,无疑会损伤局部神经末梢。术后在阴道前壁正中线上形成一条瘢痕,此时带有瘢痕的阴道前壁已不是术前敏感的阴道前壁,对术后性刺激的感受产生一定影响。

3.阴道后壁及会阴修补术,能可靠地纠正直肠膨出,使陈旧裂伤的提肛肌并拢,阴道口缩小,会阴体适当增高,有效解决盆底的支持力,改善性生活,使配偶双方得到性满足,从而缓和了阴道前壁修补术后敏感性下降的影响。手术时要注意会阴体不能修补得过高,以防阴道口狭窄,造成术后性交困难。如修补不足,会造成阴体扁平,则难以达到治疗效果。

4.阴道前、后壁修补加宫颈切除术(Man-chester手术),主要治疗Ⅱ度以上子宫脱垂,病情较重、宫颈延长者。此手术的优点是缩短双侧子宫主韧带,加强盆底组织支持力,并使子宫由后位变为前位,有利于承受腹腔和盆腔的压力,防止手术后子宫脱垂及阴道前后壁膨出复发。手术后对性功能的影响与阴道前、后壁修补术一致。

5.经阴道子宫切除术是治疗重度子宫脱垂的传统方法之一。其优点为手术后腹部皮肤无瘢痕,与腹部子宫切除术相比,性生活恢复较早。但是目前认为,在阴道子宫切除术后,作为盆腔重要支撑的组织———子宫被切去,各有关韧带被切断,盆腔相对更为空虚,盆底缺乏支持组织,需要注意术后盆底松弛,膀胱、直肠膨出的复发。

女性泌尿生殖器炎症是妇女常见病。主要有膀胱炎、尿道炎、外阴炎、阴道炎及盆腔炎等。临床表现有发热、腰痛、尿频、尿急、尿痛,或局部红、肿、疼痛和分泌物增多。可影响性生活,对患者造成心理压力。经积极抗生素治疗,均可取得满意效果。

女性尿道长3~5.5cm,管腔直径6~10mm,走行基本呈直线,尿道口松弛,在尿道与膀胱的后方有阴道相邻,其解剖结构是女性泌尿系感染发生率较高的原因。膀胱炎、尿道炎较多见,主要致病菌为大肠杆菌。表现为尿频、尿急、尿痛或下腹痛,有时会出现高热或血象增高,尿常规检查发现大量白细胞或脓球、红细胞、尿蛋白(+)。积极抗感染治疗效果显著,患病后应注意休息,多饮水。如治疗不彻底,易形成慢性膀胱炎,长期尿中带菌,反复发作,甚至发展为肾盂肾炎。

女性泌尿系统感染与性活动有明显关系。临床发现已婚妇女泌尿系统感染发生率高于未婚妇女。1978年伯克里(Buckley)等报道,性交可以导致尿内致病菌和非致病菌计数的暂时性增加。性交引起泌尿系统感染,可以通过不洁性交,或性活动中手、口及器具的刺激将细菌直接带至尿道口,逆行入尿道、膀胱。亦可通过性交动作对尿道的挤压作用而发生。当性生活频繁,性交动作粗暴,或产后、围绝经期阴道黏膜脆弱、萎缩,局部可发生微小损伤,细菌通过血管及淋巴循环直达膀胱及尿道,引起局部充血、水肿,产生尿频、尿急、尿痛等不适,从而造成性欲低下,性功能受到影响。慢性膀胱炎及尿道炎急性发作的原因,往往是频繁的或不洁的性交活动。

外阴与尿道、肛门临近,经常受到经血、阴道分泌物、尿液、粪便的刺激.若不注意皮肤清洁易引起外阴炎;其次糖尿病患者糖尿的刺激、粪瘘患者粪便的刺激以及尿瘘患者尿液的长期浸渍等;此外,穿紧身化纤内裤,导致局部通透性差,局部潮湿以及经期使用卫生巾的刺激,均可引起非特异性外阴炎。外阴炎常常表现为外阴皮肤瘙痒、疼痛、烧灼感,于活动、性交、排尿、排便时加重。检查见局部充血、肿胀、糜烂,常有抓痕,严重者形成溃疡或湿疹。慢性炎症可使皮肤增厚、粗糙、皲裂,甚至苔藓样变。治疗上首先要积极寻找病因,若发现糖尿病应治疗糖尿病,若有尿瘘、粪瘘,应及时行修补术。局部治疗可用1∶5000高锰酸钾液坐浴,每日2次,若有破溃可涂抗生素软膏或紫草油。此外可选用中药苦参、蛇床子、白癣皮、土茯苓、黄柏各15g,川椒6g,水煎熏洗外阴部,每日1~2次。预防上首先要注意个人卫生,经常换内裤,穿纯棉内裤,保持外阴清洁、干燥。

轻度外阴炎对性功能无明显影响,如局部水肿、充血、疼痛或有溃疡,可发生严重性交困难及性交疼痛,甚至阴道痉挛。同时,性交活动使局部病变加重,患者十分痛苦。严重外阴炎伴溃疡形成者,应适当节制或停止性生活,以利于炎症消退,溃疡愈合。外阴尖锐湿疣由乳头瘤病毒感染所致,通常经性交传染,男方可在包皮、龟头或阴茎冠状沟部发生湿疣,甚至男女双方在手指或舌部出现病灶。外阴尖锐湿疣可引起性交不适,性交疼痛。病灶较大而广泛者,往往产生性交困难。性活动亦使局部病情加重,因此,在患病期间应节制性生活,积极治疗。疱疹性外阴炎由单纯疱疹病毒Ⅰ型及Ⅱ型感染所致。Ⅰ型主要表现为口唇周围的小水疱,通过唾液传染。Ⅱ型主要表现为生殖器皮肤小水疱,可发展为阴蒂部、大小阴唇部的圆形或椭圆形溃疡,周围有弥散的红斑,腹股沟淋巴结肿大,甚至体温升高。近年来发现,Ⅰ、Ⅱ型病毒感染有交叉现象,加拿大的调查表明,外生殖器疱疹的病原体中,Ⅱ型病毒占78\%,Ⅰ型病毒占22\%。男性生殖器亦可同时发生感染,引起水疱及溃疡。本病主要通过性交,或口-生殖器接触传染。发病时有性交疼痛及性交困难。妊娠期感染疱疹病毒,能经过胎盘传染,引起流产、早产和胎儿畸形。分娩时经宫颈及阴道分泌物引起新生儿中枢神经系统及眼部损害,甚至死亡。

女性阴道呈管状,上接子宫,下连处女膜,前后扁平,通过阴道前庭与大小阴唇与外界相通,前为膀胱、尿道及尿道口,后为直肠及肛门,所以可因外环境、尿道口、肛门口的改变而影响阴道的环境。阴道内存在G—和G+球菌、杆菌及其他微生物(如人型支原体、解脲支原体、酵母菌等),共计29种之多,一般至少可检出7~8种。正常阴道菌群中数量最多的是乳酸杆菌,占95\%,可达8×107 /mL,与其他菌群共生和拮抗。健康阴道内寄生的菌群检出率:乳杆菌68\%,表皮葡萄球菌53\%,类杆菌52\%,非溶血性链球菌37\%,棒状杆菌31\%,肠球菌及消化链球菌26\%,肠道加德纳菌25\%,梭杆菌21\%,大肠杆菌20\%,梭状芽胞杆菌12\%,金黄色葡萄球菌8\%,其他如乳链球菌、肠球菌、变形杆菌等也在阴道下段常见,少数人上有少量念珠菌。一般健康妇女阴道内有如此多的病菌,但并不致病,处在微生态平衡状态,如果这种平衡被破坏,互相制约作用消失,氢离子浓度下降,乳杆菌失去优势,致病菌得以繁殖,则产生相应的炎症或临床综合征。正常情况下,阴道的微生态和免疫功能处于平衡状态,阴道pH值在正常范围之内不易发生炎症。影响阴道微生态的因素有:阴道pH值、激素活性、女性不同时期菌群的变迁、月经周期、妊娠、阴道壁生物电势、避孕工具(避孕套、子宫帽、阴道隔膜、阴道橡皮圈、宫内节育器的尾丝等)、月经用品、性交(由于精液润滑黏液的强碱性,性交后8小时阴道pH值不能恢复正常,性交频度等)、阴道或宫颈手术、阴道消毒剂和阴道冲洗、药物(全身或阴道局部使用抗生素、免疫抑制剂等)、维持阴道平衡的乳杆菌减少、贫血、甲状腺功能减退、感染因素等,均可影响阴道微生态和免疫功能。阴道是全身免疫系统的组成部分,也受全身免疫和局部免疫功能的相互影响。生殖道黏膜下层有淋巴组织及散在淋巴细胞,包括T细胞、B细胞,此外中性粒细胞、补体以及一些细胞因子均在局部有重要的免疫功能,可发挥抗感染作用。也形成免疫应答网络,一但阴道黏膜受病原体攻击,除受攻击部位产生免疫反应外,血液和其他黏膜部位很快产生免疫应答。月经周期变化、外源性雌激素和孕激素,其他疾病或AIDS等对阴道免疫功能均有影响。

阴道炎是常见病和多发病,是不同疾病引起的多种阴道黏膜炎性疾病的总称。阴道炎的主要表现为白带增多、外阴瘙痒及性交疼痛,由于炎症的病因,分泌物的特点、性质及瘙痒的轻重不同,阴道炎的种类有十余种之多:滴虫性阴道炎、假丝酵母菌(原称念珠菌)性阴道炎、老年性(或萎缩性)阴道炎、细菌性阴道病、婴幼儿性阴道炎、病毒性(人乳头瘤病毒HPV、单纯疱疹病毒HSV)阴道炎、淋菌性阴道炎、阿米巴性阴道炎、脱屑性阴道炎、药物性阴道炎、气体性阴道炎、结核性阴道炎、非特异性阴道炎、阴道溃疡(急性、慢性)、放射性阴道炎、过敏性阴道炎、蛲虫性阴道炎等,临床上50\%以上的阴道炎为混合感染。

(1)日常生活习惯:保持良好的情绪,不动怒、不忧郁;不要有吸烟、酗酒等不良嗜好;规律的作息时间,保证睡眠充足;适当运动,有一个健康的体质;注意保暖。

(2)正确的清洁:以少量中性无香精肥皂,轻拍会阴取代揉搓。清洁后可涂上薄薄的婴儿油,保持会阴皮肤的正常湿润。清洗会阴的水,宜冷或微温的开水,每次5~10分钟。用手清洗会阴最好,尽量不要用布等粗糙物品。不要用强力的清洁剂过度清洗会阴皮肤,更不能用杀菌洗液、碱性液体如高锰酸钾溶液等清洗外阴。清洗时只清洗阴道口外,切勿冲洗阴道。有过敏体质者,避免使用非棉制品的月经护垫或尿垫。不穿紧身内裤,内裤被汗渍或水浸湿后要及时更换,内裤应用专盆单独洗涤,洗后应在有阳光的地方晒干。大便后要从前向后擦拭肛门。

(3)性交:性交是造成会阴皮肤、阴道黏膜挫伤,也是造成阴道内pH值明显升高,从而破坏阴道内“生态平衡”的主要原因。所以性交不宜过频,时间不宜过长,一般建议每周1~2次,每次不超过30分钟。性交前要清洗外阴和男性龟头、阴茎,避免不洁性生活,以免带来意外的病原体感染。经期禁止性交。

(4)用药:切勿滥用药物如抗生素、激素、免疫抑制剂或同位素等,以免使正常微生物群和人体内的生态平衡遭受严重破坏。不能用碱性液体、杀菌洗液(如高锰酸钾溶液、碘伏溶液、洁尔阴等)清洗外阴。围绝经期妇女外阴、阴道干燥萎缩可以用激素替代治疗。

(5)经期的护理:月经垫宜用质地柔软、吸水性能好的消毒棉垫或卫生巾,及时更换,不穿紧身内衣。保持外阴清洁,不盆浴。注意保暖,避免受凉,尤其是下半身的保暖更为重要。突然或过强的冷刺激有可能使子宫及盆腔内血管挛缩而引起痛经或月经骤停。此外,经期身体抵抗力降低,受凉后更易感染疾病。应保持心情舒畅,因为精神紧张或情绪波动都能影响中枢神经系统的调节功能,从而引起月经失调或加重经期反应。饮食适当,避免辛辣,多饮水,多食蔬菜,保持大便通畅。规律充足的睡眠,注意劳逸结合,调整作息时间,夜间早睡。避免剧烈运动,经期一般体力劳动和户外活动可促进盆腔血液循环,从而减轻腰酸和坠胀等感觉。但重体力劳动和剧烈运动可使盆腔血流过速,引起经量过多或经期延长。禁性生活。

(6)产后护理:产后护理类似经期护理,另外应及时做产后恢复训练,如适当活动、缩肛运动、仰卧起坐等促进盆底肌肉和韧带恢复。

性活动往往是阴道炎的致病原因。性生活不当,卫生不良,可造成感染性阴道炎。包括滴虫性阴道炎在内的多种阴道炎都可以在配偶间相互传染,因此,需要双方同时治疗,以避免复发。在1972年、1977年欧雷尔(Oriel)及史因(Thin)等发现,分别有26\%及35.3\%的性病门诊妇女患有假丝酵母菌性阴道炎。因此,提出对性生活活跃的假丝酵母菌性阴道炎妇女进行隔离,以防止可能并发的性病传播。最近在79\%的患有嗜血杆菌性阴道炎患者配偶的尿道中发现了嗜血杆菌,而且治愈阴道炎后出现重复感染,说明本病可通过性交传染。支原体阴道炎能引起尿道炎和不孕。1972年麦柯马克(McCormack)等证明,阴道支原体的生长同性生活有关。他们指出,没有性生活史的妇女,阴道内没有支原体。仅有一个性交对象的妇女,阴道内有支原体者占37.5\%。有3个或3个以上性交对象的妇女,阴道支原体阳性者占75\%。阴道炎病原体不同,但临床症状均表现白带增多,外阴瘙痒及灼热感和性交疼痛。上述不适使患者苦恼和烦躁,表现出性活动减少,性欲低下,有性交疼痛者常常回避性生活。性交能加重阴道局部病变,因此,在患病期间应适当节制性生活,减少局部刺激及创伤。阴道炎亦可能产生性心理异常,导致性功能障碍。1例妇科门诊患者,40岁,经产妇,职员,主诉严重性交疼痛1年。检查:阴道黏膜充血、水肿明显,但白带不多,子宫及双侧附件正常,无压痛。详细追问病史,了解到患者于1年前曾因白带增多就诊,医生诊断为阴道炎。患者自认为阴道炎系男方带入细菌所致,从此开始节制性生活,并于每次性交前用乙醇涂抹配偶外生殖器消毒,其后逐渐发生性交疼痛及性交困难,病情日益加重。同时出现尿频、尿急、尿痛等症状。了解病情后,给其咨询指导,纠正错误观念,指出乙醇有脱水及干燥作用,对性交动作不利。患者接受指导后恍然大悟。经对阴道炎及膀胱炎适当治疗,纠正错误做法后,阴道灼热感消失,尿路刺激症状明显好转,未再发生性交疼痛及性交困难等情况,性功能恢复正常,夫妻关系美满。

宫颈炎症是妇科最常见的疾病。正常情况下,宫颈具有多种防御功能:①宫颈阴道部表面覆以复层鳞状上皮,具有较强的抗感染能力;②宫颈内口紧闭,宫颈管黏膜为分泌黏液的高柱状上皮所覆盖,黏膜形成皱褶、嵴突或陷窝,从而增加黏膜表面积。宫颈管分泌大量黏液形成黏液栓,内含溶菌酶、局部抗体(抗白细胞蛋白酶),这对保持内生殖器的无菌状态非常重要。据报道,宫颈黏液栓的下1/3能查出细菌,而上2/3查不到细菌。但宫颈易受分娩、宫腔操作的损伤,宫颈管单层柱状上皮抗感染能力较差,并且由于宫颈管黏膜皱襞多,一旦发生感染,很难将病原体完全清除,而导致慢性宫颈炎症。

(1)急性宫颈炎:急性宫颈炎主要见于感染性流产、产褥期感染、宫颈损伤或阴道异物并发感染等情况。常见的病原体为葡萄球菌、链球菌、肠球菌等。随着性传播疾病的增加,近年来急性宫颈炎已成为常见疾病。目前急性宫颈炎最常见的病原体为淋病奈氏菌、沙眼衣原体。淋病奈氏菌及沙眼衣原体均感染宫颈管柱状上皮,沿黏膜面扩散引起浅层感染,引起黏液脓性宫颈黏膜炎。除宫颈管柱状上皮外,淋病奈氏菌还常侵袭尿道移行上皮、尿道旁腺及前庭大腺。沙眼衣原体感染只发生在宫颈管柱状上皮,不感染鳞状上皮,故不引起阴道炎,仅形成急性宫颈炎症。急性宫颈炎主要表现为阴道分泌物增多,呈黏液脓性,阴道分泌物的刺激可引起外阴瘙痒,伴有腰酸及下腹部坠痛。此外,常有下泌尿道症状,如尿急、尿频、尿痛。沙眼衣原体感染还可出现经量增多、经间期出血、性交后出血等症状。妇科检查见宫颈充血、水肿、靡烂,有黏液脓性分泌物从宫颈管流出。衣原体宫颈炎可见宫颈红肿、黏膜外翻、宫颈触痛,且常有接触性出血。淋病奈氏菌感染还可见到尿道口、阴道口黏膜充血、水肿以及多量脓性分泌物。

(2)慢性宫颈炎:慢性宫颈炎多由急性宫颈炎转变而来,常因急性宫颈炎治疗不彻底,病原体隐藏于宫颈黏膜内而形成,多见于分娩、流产或手术损伤宫颈后,病原体侵入而引起感染。也有的患者无急性宫颈炎症状,直接发生慢性宫颈炎。慢性宫颈炎的病原体主要为葡萄球菌、链球菌、大肠杆菌及厌氧菌。目前沙眼衣原体及淋病奈氏菌感染引起的慢性宫颈炎日益增多,已引起广泛关注。此外,单纯疱疹病毒也可能与慢性宫颈炎有关。病原体侵入宫颈黏膜,并在此处潜藏,由于宫颈黏膜皱襞多,感染不易彻底清除,往往形成慢性宫颈炎。慢性宫颈炎的主要表现是阴道分泌物增多。由于病原体、炎症的范围及程度不同,分泌物的量、性质、颜色及气味也不同。分泌物呈乳白色黏液状,有时呈淡黄色脓性,伴息肉时易出现血性白带或性交后出血。当炎症沿宫骶韧带扩散到盆腔时,可有腰骶部疼痛、盆腔部下坠痛等。宫颈黏稠脓性分泌物不利于精子穿过,可造成不孕。妇科检查时可见宫颈有不同程度糜烂、肥大,有时质较硬,有时可见息肉、裂伤、外翻及宫颈腺体囊肿。

宫颈糜烂表面的单层柱状上皮菲薄脆弱,很难抵御各种致病微生物的侵袭,在加上糜烂表面血管增生,经常出现白带增多、性交出血甚至不育,宫颈息肉组织血管丰富而柔软,常并发感染,出现少量不规则阴道出血或性交出血,这将会给配偶双方造成精神上的压力,唯恐罹患妇科癌症而节制性生活或到妇科就诊,经过规范的诊治,通常可以完全治愈,症状消失,不致影响性功能。

女性内生殖器及其周围的结缔组织、盆腔腹膜发生炎症时称盆腔炎(pelvic inflammatory disease.PID)。盆腔炎大多发生在性活跃期,有月经的妇女。初潮前、绝经后或未婚者很少发生盆腔炎,若发生盆腔炎也往往是邻近器官炎症的扩散。炎症可局限于一个部位,也可同时累及几个部位,最常见的是输卵管炎及输卵管卵巢炎,单纯的子宫内膜炎或卵巢炎较少见。盆腔炎有急性和慢性两类。急性盆腔炎发展可引起弥漫性腹膜炎、败血症、感染性休克,严重者可危及生命。若在急性期未能得到彻底治愈,则转为慢性盆腔炎,往往经久不愈,并可反复发作,不仅严重影响妇女健康、生活及工作,也会造成家庭与社会的负担。随着医疗条件及水平的提高,妇女卫生保健知识的普及和广谱抗生素的应用,严重危及生命的急性盆腔炎及久治不愈的慢性盆腔炎在临床已不多见。女性生殖道在解剖、生理上的特点使其有比较完善的自然防御功能,增强了其对感染的防御能力。除了外阴、阴道、宫颈的自然防御功能外,孕龄妇女子宫内膜的周期性剥脱也是消除宫腔感染的有利条件。此外,输卵管黏膜上皮细胞的纤毛向子宫腔方向摆动以及输卵管的蠕动,均有利于阻止病原体的侵入。但当机体免疫功能下降、内分泌发生变化或外源性致病菌侵入时,均可导致炎症的发生。引起盆腔炎的病原体有两个来源,来自寄居于阴道内的菌群包括需氧菌及厌氧菌,来自外界的病原体如淋病奈瑟氏菌、沙眼衣原体、结核杆菌、绿脓杆菌等。

(1)急性盆腔炎:可因炎症轻重及范围大小而有不同的临床表现。发病时下腹痛伴发热,若病情严重可有寒战、高热、头痛、食欲不振。月经期发病可出现经量增多、经期延长,非月经期发病可有白带增多。若有腹膜炎,则出现消化系统症状如恶心、呕吐、腹胀、腹泻等。若有脓肿形成,可有下腹包块及局部压迫刺激症状。包块位于前方可出现膀胱刺激症状,如排尿困难、尿频,若引起膀胱肌炎还可有尿痛等;包块位于后方可有直肠刺激症状,若在腹膜外可致腹泻、里急后重感和排便困难。根据感染的病原体不同,临床表现也有差异。淋病奈氏菌感染起病急,多在48小时内出现高热、腹膜刺激征及阴道脓性分泌物。非淋病奈氏菌性盆腔炎起病较缓慢,高热及腹膜刺激征不明显,常伴有脓肿形成。若为厌氧菌感染,则容易有多次复发,脓肿形成,患者的年龄偏大,往往大于30岁。沙眼衣原体感染病程较长,高热不明显,长期持续低热,主要表现为轻微下腹痛,久治不愈,阴道不规则出血。患者呈急性病容,体温升高,心率加快,腹胀,下腹部有压痛、反跳痛及肌紧张,肠鸣音减弱或消失。盆腔检查:阴道可能充血,并有大量脓性分泌物,将宫颈表面的分泌物拭净,若见脓性分泌物从宫颈口外流,说明宫颈黏膜或宫腔有急性炎症。穹隆有明显触痛,须注意是否饱满;宫颈充血、水肿、举痛明显;宫体稍大,有压痛,活动受限;子宫两侧压痛明显,若为单纯输卵营炎,可触及增粗的输卵管,有明显压痛;若为输卵管积脓或输卵管卵巢脓肿,则可触及包块且压痛明显;宫旁结缔组织炎时,可扪到宫旁一侧或两侧有片状增厚,或两侧宫骶韧带高度水肿、增粗,压痛明显;若有脓肿形成且位置较低,可扪及后穹隆或侧穹隆有肿块且有波动感,三合诊常能协助进一步了解盆腔情况。急性淋菌性盆腔炎的治疗效果良好。如发生盆腔脓肿,体温不降,白细胞增高,应及时开腹探查,切除输卵管卵巢脓肿,或放置引流。

(2)慢性盆腔炎:慢性盆腔炎病情较顽固,当机体抵抗力较差时,可有急性发作。全身炎症症状多不明显,有时仅有低热,易感疲倦。由于病程较长,部分患者可出现神经衰弱症状,如精神不振、周身不适、失眠等。当患者抵抗力差时,易有急性或亚急性发作。慢性炎症形成的瘢痕粘连以及盆腔充血,常引起下腹部坠胀、疼痛及腰骶部酸痛。常在劳累、性交后及月经前后加剧。慢性炎症导致盆腔淤血,患者常有经量增多;卵巢功能损害时可致月经失调;输卵管粘连阻塞时可致不孕。子宫常呈后倾后屈,活动受限或粘连固定。若为输卵管炎,则在子宫一侧或两侧触到呈索条状增粗的输卵管,并有轻度压痛。若为输卵管积水或输卵管卵巢囊肿,则可在盆腔一侧或两侧触及囊性肿物,活动多受限。若为盆腔结缔组织炎。子宫一侧或两侧有片状增厚、压痛,宫骶韧带常增粗、变硬、有触痛。慢性盆腔炎用抗生素控制感染,或用清热解毒中药红藤、败酱草、蒲公英、鸭跖草、地丁各30g煎汤灌肠,或下腹超短波等透热疗法。确诊为结核性盆腔炎者,应抗结核治疗。

急性和慢性盆腔炎与性生活有密切关系,互为因果,互相影响。不洁性交、流产、早产、分娩后或月经期性交是盆腔炎重要的发病原因。反过来,盆腔炎亦可导致性交疼痛、性交困难等性功能障碍。急性盆腔炎患者在发病期间,病情重、体力不支,不可能顾及性生活。如被动进行,因性交动作刺激,可引起严重性交疼痛,甚至阴道痉挛,加重病情。急性期停止性活动能减少盆腔充血、水肿及炎性浸润。慢性盆腔炎患者子宫骶骨韧带增粗,有触痛,可引起深部性交疼痛及性交后不适,影响性快感,导致感情上及行为上的变化,严重影响性关系。因此,应积极治疗盆腔炎,以减少性功能障碍。

多囊卵巢综合征(polycystic ovary syndrome,PCOS)是临床常见的妇科内分泌疾病,在我国有着庞大的患者群。PCOS临床表现异质性,不但严重影响患者的生殖功能,而且会增加雌激素依赖性肿瘤如子宫内膜癌的发病率,相关的代谢失调包括高雄激素血症、胰岛素抵抗、糖代谢异常、脂代谢异常、心血管疾病危险也增加。PCOS占生育年龄妇女的5\%~10\%,占无排卵性不孕症患者的30\%~60\%。PCOS的病因至今尚不明确,有研究认为,其可能是由于某些遗传基因与环境因素相互作用引起的。遗传因素:PCOS有家族聚集现象,被推测为一种多基因病,目前的候选基因研究涉及胰岛素作用相关基因、高雄激素相关基因和慢性炎症因子等。环境因素:宫内高雄激素、抗癫痫药物、地域、营养和生活方式等。但尚需进行流行病学调查后,方能完善环境与PCOS关系的认识。

初潮2~3年后仍不能建立规律月经;闭经(停经时间超过3个以往月经周期或≥6个月);月经稀发,即周期≥35天及每年≥3个月不排卵者;注意,月经规律并不能作为判断有排卵的证据,判断是否排卵的方法有:基础体温(BBT)、B超监测排卵、月经后半期孕酮测定等。

痤疮(复发性痤疮,常位于额、双颊、鼻及下颌等部位)、多毛(上唇、下颌、乳晕周围、下腹正中线等部位出现粗硬毛发)。

一侧或双侧卵巢中直径2~9mm的卵泡≥12个,和(或)卵巢体积≥10ml:

先天性肾上腺皮质增生、库欣综合征、分泌雄激素的肿瘤等,以及其他引起排卵障碍的疾病如:高泌乳素血症,卵巢早衰,垂体或下丘脑性闭经,以及甲状腺功能异常。

多囊卵巢综合征患者的性功能受到体内激素水平变化的影响。由于血中雄激素水平比正常妇女高,表现为男性化倾向,多毛、痤疮、阴蒂肥大。此类患者性欲增强,性交次数增多、迫切要求自身的性满足。最近一项研究表明,多囊卵巢综合征患者与那些事业心强爱好体育运动的妇女相比,有更高的性欲、更强的主动性和更明显的性冲动。这项结果与马斯特斯和约翰逊对多囊卵巢综合征不孕患者的临床观察是一致的。但是,多毛、肥胖及不孕使许多患者产生自卑心理,引起性的问题。虽然她们因盼子心切而明确或含蓄地要求频繁的性生活,但表现出性欲低下,性兴奋障碍,性高潮频率减少或消失。焦虑使她们丧失信心,而不能尽情享受性的乐趣,只是呆板地完成程序,性反应亦随之降低。

子宫内膜异位症(内异症)是指子宫内膜组织(腺体和间质)在子宫腔被覆内膜及子宫肌层以外的部位生长﹑浸润﹑反复出血,可形成结节及包块,引起疼痛﹑不育等。是生育年龄妇女的多发病、常见病,子宫内膜异位症多发生于30~40岁的妇女。以痛经、慢性盆腔痛和不育为主要表现,可以形成盆腔肿物,见于卵巢内异症。其病变广泛,形态多样,极具侵袭和复发性,呈现恶性临床表现。内异症发病机制复杂不清,近几年注意到“在位内膜”的重要作用。临床病理分类为腹膜型、卵巢型、盆腔深部浸润型及其他部位型。腹腔镜检查是诊断的通用方法,亦可依据疼痛、不育、盆腔检查、影像学及血清CA125检测等重要指标诊断。治疗根据疼痛、包块和不育,采用手术与药物的联合方法。子宫内膜异位症(内异症)特点如下:①生育年龄妇女的多发病,主要引起疼痛及不育;②发病率有明显上升趋势;③症状与体征及疾病的严重性不成比例;④病变广泛﹑形态多样;⑤极具浸润性,可形成广泛、严重的粘连;⑥激素依赖性,易于复发。内异症的临床病理类型分为4型:①腹膜型子宫内膜异位症(peritoneal endometriosis,PEM);②卵巢型子宫内膜异位症(ovarian endometriosis,OEM);③深部浸润型子宫内膜异位症(deep infiltrating endometriosis,DIE),包括宫骶韧带﹑阴道直肠窝﹑直肠结肠壁﹑阴道穹隆等;④其他部位的子宫内膜异位症(other endometriosis,OtEM),如消化(I)、泌尿(U)、呼吸(R)﹑瘢痕(S)等。

1.疼痛 70\%~80\%的患者有不同程度的盆腔疼痛,与病变程度不完全平行。①痛经:典型者为继发性,并渐进性加重;②非经期腹痛:慢性盆腔痛(chronic pelvic pain,CPP);③性交痛以及排便疼痛等;④卵巢内异症囊肿破裂可引起急性腹痛。

2.不育 约50\%的患者合并不育。

3.月经异常。

4.盆腔包块。

5.特殊部位内异症 各种症状常有周期性变化,可合并盆腔内异症的临床表现:①消化道内异症,大便次数增多或便秘﹑便血﹑排便痛等症状。②泌尿道内异症,尿频﹑尿痛﹑血尿及腰痛,甚至造成泌尿系梗阻及肾功能障碍。③呼吸道内异症,经期咯血及气胸。④瘢痕内异症:腹壁剖宫产等手术后切口瘢痕处结节,经期增大,疼痛加重;会阴切口或伤口瘢痕结节,经期增大,疼痛加重。

6.妇科检查 典型病例子宫常为后位,活动度差。宫骶韧带﹑子宫直肠窝或后穹隆触痛结节,可同时有附件囊性不活动的包块。

7.血CA125检查 CA125水平多为轻中度升高。

8.影像学检查 超声扫描主要对卵巢内异症囊肿诊断有意义。典型的超声影像为附件区无回声包块,内有强光点。MRI对卵巢内膜异位囊肿﹑盆腔外内异症以及深部浸润病变的诊断和评估有意义。

9.其他 必要时可行其他辅助检查,如IVP、膀胱镜、结肠镜等。

子宫内膜异位症是引起性交疼痛的重要妇科病原因之一。由于性交动作的冲撞,使质硬、缺乏弹性的子宫骶骨韧带张力增加、移位,产生深部性交疼痛及性交后疼痛。月经来潮前盆腔充血,性交疼痛更为明显。长期患病可出现性心理障碍,恐惧及回避性生活。表现为性欲低下,性高潮频率减少。配偶间性关系不协调,亦有发展为感情冲突及婚姻破裂者。在临床工作中可以遇到少数年龄较长、富有性经验、知识水平较高的患者,为了解除性交疼痛的烦恼,将习惯采用的男上式体位主动改为侧卧后进式性交体位。由于体位的变化,性交动作冲撞的方向由阴道后穹隆转为阴道前壁或前穹隆,从而减轻或避免性交疼痛的发生。患者表示,配偶双方均能适应上述体位,并能得到性的满足。

绝经是每一妇女生命进程中必然发生的生理过程。绝经提示卵巢功能衰退,生殖能力终止。卵巢功能衰退呈渐进性,人们一直用“更年期”来形容这一渐进的变更时期。由于更年期定义模糊,1994年WHO提出废弃“更年期”,推荐采用“围绝经期”一词。围绝经期指卵巢功能开始减退,直至绝经后1年内的一段时间。可始于35岁~45岁。此期内卵泡数量明显减少,残留卵泡对促性腺激素的反应性降低,或完全丧失反应,因此,最早的改变是排卵频率减少,继而停止排卵,最终卵泡不再发育,由发育卵泡合成的性激素减少甚至消失,由此引起月经改变,如月经频发、月经过少、月经不规则及闭经等。同时身体及心理上也出现很多变化,临床上出现各种症状,表现为内分泌、躯体和心理方面不同程度的变化,严重地影响围绝经期妇女的健康及生活质量。我国城市妇女的平均绝经年龄为49.5岁,农村妇女为47.5岁。约1/3的围绝经期妇女能通过神经内分泌的自我调节达到新的平衡而无自觉症状,2/3的则可出现一系列性激素减少所致的症状,称为围绝经期综合征。围绝经期对健康的危害主要是由于卵巢功能衰退,雌激素分泌相应减少所致。由于雌激素缺乏,引起全身骨量的迅速下降,血脂代谢紊乱,动脉硬化逐渐形成,最终导致骨质疏松骨折和心血管疾病的发生。月经失调是围绝经期的基本症状,除此之外,七成多的女性还会出现不同程度的围绝经期症状,表现出潮热多汗、脾气暴躁、抑郁焦虑、皮肤萎缩和关节疼痛等。对于围绝经期症状,应给予足够的重视与合理的治疗,避免给骨折、高血压、心脑血管疾病、糖尿病、肿瘤和老年痴呆等老年疾病埋下隐患。

1.月经改变

(1)月经频发(polymenorrhea):指月经周期短于21天,常伴有经前点滴出血致出血时间延长。其发生原因多为黄体功能不足或黄体期缩短(<20天)无排卵。

(2)月经稀发(oligomenorrhea):指月经周期延长至35天~6个月,常因排卵稀少引起,也可能为不排卵性出血,常伴经血量减少。

(3)不规律子宫出血:因停止排卵而发生无排卵型功能失调性子官出血。

(4)闭经(amenorrhea):卵巢合成性激素大幅度减少后,子宫内膜失去雌、孕激素的影响而处于静止状态,因而不再增殖及脱落,此时发生闭经(月经停止6个月以上)。多数妇女经历不同类型和时期的月经改变后,逐渐进入闭经,而少数妇女可能突然闭经。闭经超过1年时,即为绝经。

2.血管舒缩功能不稳定症状 即潮红、出汗、心悸、眩晕等症状,其中潮热及出汗是围绝经期及绝经后妇女有特征性的症状,发生率占妇女的75\%~85\%,严重者占10\%~20\%。典型表现:突发上半身发热,由胸部冲向头部,或伴头痛、头胀、眩晕或无力,持续数秒至数十分钟不等,症状消失前常大量出汗或畏寒。体征可见面、颈及胸部潮红,上肢温度升高,躯体温度正常或稍降低,血压不变.手指血流量增加。轻者数日发作一次,重者日夜发作几十次。10\%为每日发作,或每日频繁发作;50\%感到苦恼,因常发生在夜间而影响睡眠,由此又引起疲乏、注意力不集中、记忆力下降等症状。症状发生的时间不定,约41\%发生在绝经过渡期,以绝经前1~2年最严重,少数发生于绝经以后。症状持续1年以上者占50\%~75\%,持续5年者占25\%~50\%,随着停经时间的延长,症状可减轻或自然消失,10\%~15\%的妇女持续10~15年或更长。

3.自主神经系统功能不稳定症状 如心悸、眩晕、失眠、皮肤感觉异常等。常伴随潮热症状,少数妇女无潮热发作,只表现为此类症状的一种或数种。

4.精神、心理症状 如抑郁、焦虑、多疑、自信心降低、注意力不集中、易激动、恐怖感甚至癔症发作样症状。

5.心血管系统的症状 血压升高或血压波动、心悸或心律不齐。

6.泌尿生殖器官萎缩 外阴及阴道发干或伴瘙痒,合并感染时阴道分泌物增多,或有臭味;血性阴道分泌物,或绝经后阴道出血;性交痛;尿急、尿失禁,约40\%绝经后妇女出现压力性尿失禁。因尿道黏膜萎缩而使管腔变宽,同时盆底肌肉张力下降,当加腹压时即不能控制而漏尿;由于尿道变宽,上行感染的机会增加,容易并发泌尿系感染。

7.皮肤、体型的改变 皮肤明显变薄,弹性下降,易出现皱褶、红润消失,干燥,出现瘙痒或烧灼感,阴毛减少,头发脱落、变细;乳房松软、下垂;腹部及臀部增大;合并骨质疏松症时,可致身材变短、驼背及腹部前挺,行走时则取足尖向外,步态蹒跚,严重时可出现骨折,以及由骨折引起的一系列问题

8.冠状动脉粥样硬化性心脏病。

由于雌激素缺乏,围绝经期妇女在性功能及性反应上均有不同程度的减退。在《人类性反应》一书中,马斯特斯和约翰逊研究了34名绝经后妇女,发现在性兴奋中乳房增大不明显或缺乏,性红晕不常见,而且分布局限,延及全身的肌强直作用减弱,阴蒂反应时间后延,阴道润滑出现缓慢、作用减弱,巴氏腺分泌缓慢和减少,宫颈横径扩展能力低下,阴道外1/3充血不明显,从而减低了性高潮平台的反应。且较少发生高潮时的子宫收缩感。上述变化均与雌激素水平下降,引起性器官萎缩,局部血流减少,感觉功能减退有关。性欲与雌激素水平亦有关系。Halstron报道50岁以上的绝经妇女,其性欲低下占68\%,而同年龄绝经前妇女的性欲低下占52\%,Sarrel研究154例绝经期妇女,其性欲低下及性厌恶(sexual aversion)占45\%,触觉障碍占36\%,阴蒂不敏感占20\%,阴道干燥占36\%,尿失禁占10\%。其原发性性高潮障碍为10\%,继发性性高潮障碍27\%,另有性交疼痛及性交困难占43\%,对性交困难者进行妇科检查时发生阴道痉挛者50\%。Philip等研究93例绝经妇女,68\%有性问题,其中性欲低下77\%、阴蒂迟钝36\%、阴道中度干燥以上者58\%、高潮障碍29\%、高潮强度下降35\%、性交困难39\%、每月性交1次以下者50\%。性交频率与血浆雌激素水平有关。Mc-Coy等报道,绝经后早期血浆雌二醇水平在91.8pg/ml者比绝经后晚期血浆雌二醇水平在41.6pg/ml者性交频率要高,而两组血浆睾酮水平没有差异。Cutler等试验证明,血浆雌二醇水平低于35pg/ml,则性交频率减少,高于35pg/ml的妇女,一般每周性交1次,如果性交次数少于一周1次,可预示血浆雌二醇水平低于35pg/ml。Philip等研究发现,血浆雌二醇少于50pg/ml时,则主诉阴道干燥,插入疼痛及灼热感,如用雌激素治疗,血浆雌二醇水平高于50pg/ml,则症状消失。雌激素是性原动力的决定因素。同时亦应充分认识社会心理因素是与雌激素同等重要的性原动力。由于传统的观念、文化上的抑制,对妇女性观念、性兴趣均有明显影响,其他的心理因素,如过去性行为史、对性功能正常减退的态度、对围绝经期生理变化的反应、对周围态度变化的反应、对围绝经期疾病的态度,均对围绝经期妇女性心理状态有明确影响。绝经使妇女意识到自身已开始衰老。性欲减退、阴道分泌物减少、性交疼痛及缺乏高潮,使围绝经期妇女不注意尽力保持性功能,长期避免性生活,形成性器官失用性萎缩,产生性功能障碍。

围绝经期妇女常常自述性欲下降,但并没有性交痛及性交困难,少数妇女性欲亢进。追究性欲减退的原因,多数认为“年纪大了,不需要了”,其实质是一种认识上的错误及心理障碍,认为自己已老,性能力已减退;或者认为性生活只是为了生育,因而没有必要了,从而对性生活缺乏主动及兴趣,甚至怀疑自己已经没有性的吸引力,而增加对性生活的干扰与不和谐。性医学认为性行为是一种生理的和心理的综合产物,人类的性行为完全可以不与性激素水平相平行,因此,不少妇女在50岁后性欲反而增强。从医学的观点出发,老年人应该有规律的性生活,这对男、女双方的身心健康都是有益的。围绝经期综合征的妇女由于忧虑、烦躁、失眠、潮热、夜汗及焦虑,较正常围绝经期妇女更容易出现性心理障碍及性问题。要认识到围绝经期妇女的性观念、性兴趣和性能力有极大的个体差异。金西等报道,绝经对女性的性反应没有什么直接影响,其性活动减少主要与男方性兴趣下降有关。还有作者注意到,许多妇女在绝经后对性的兴趣更浓,这种兴趣的增加可能与不再担心怀孕有关。围绝经期妇女性功能状况与配偶性功能有关,如配偶性兴趣浓厚,性能力很强。但相当一部分男性配偶同时出现性问题。其表现为睾丸缩小,血浆睾酮水平下降及阳痿发生率上升。Sarrel报道的50对绝经期配偶中,11对仅女方有性问题(5例高潮障碍、6例性交困难合并高潮障碍),9对仅男方有性问题(8例继发性勃起困难,1例原发性勃起困难),30对男女双方均有问题(19对女方性交困难,男方勃起障碍,11对女方高潮障碍,男方继发勃起困难)。围绝经期综合征妇女,精神神经系统及心血管系统症状较多,必然对配偶产生较大的心理压力,而影响其性兴趣及性功能。围绝经期综合征妇女内分泌、神经系统、生殖系统及身体的变化,使她们感到自己已衰老,而影响性生活及夫妻关系。因此,除了对其进行一般治疗及激素治疗外,还需要进行相应的心理治疗及性治疗。首先应当详细了解其全身健康状况、生殖器萎缩程度以及性功能状态,了解存在的性问题,和需要医生解决的困难。向其解释围绝经期妇女随着年龄的增长,性功能逐渐衰退是自然规律,不必十分惊讶和沮丧,鼓励她们正确对待围绝经期的生理改变及产生的疾病,正确对待周围态度的变化,增强自尊心和自信心,更尽力维持性生活,以减缓生殖器官衰老的进程。应用激素替代治疗也能有效缓解泌尿生殖道的萎缩症状,消除性生活过程中的不适感,提高性生活的质量。此外,还要注意到围绝经期妇女在感情上的需要,她们不仅仅需要简单的阴道性交,还需要拥抱,需要向别人倾诉,需要表达感情和接受他人的感情,以达到自己感情的满足和心理平衡。在调适围绝经期妇女性生活的过程中重视配偶的作用,使配偶能正确认识女方围绝经期的生理变化,正确认识配偶本身存在的性问题,多进行感情及性信息的相互交流,重视对性行为进行调适,以保持围绝经期性功能。此外,要选择舒适的环境及适当的时间进行活动,热水浴可增加全身血液循环,增强性欲及舒适感,书籍及影视亦有促进作用。阴道萎缩及干燥者亦可用水或油性润滑剂缓解性交困难及性交疼痛。有严重性功能障碍者,可到性治疗门诊进行咨询及治疗。

妇科肿瘤是妇女的常见疾病,无论是良性肿瘤还是恶性肿瘤,其对妇女身心健康均可能产生不良影响。妇科医生主要研究肿瘤的发生、发展、诊断及治疗,而很少关注妇科肿瘤与性的关系。实际上,性的问题与某些妇科肿瘤的发病有一定关系。由于妇科肿瘤所引起的阴道出血、疼痛及肿块,以及治疗肿瘤所需的手术、放射治疗及化疗均可能影响正常的性反应,甚至产生性交疼痛、性交困难或性交停止。妇科肿瘤亦可能影响妇女的心理及性心理状态,因此,应对妇科肿瘤与性功能有关的生理、病理、解剖、内分泌及社会心理及性心理的变化进行深入广泛的研究,以利于提出预防及治疗性功能障碍的措施和咨询指导。

外阴良性肿瘤较少见,主要有平滑肌瘤、纤维瘤、脂肪瘤、乳头瘤、汗腺瘤等。外阴上皮内瘤样病变(vulva intraepithelial neoplasia,VIN),指肿瘤局限于外阴表皮内,未向周围间质浸润及转移,包括外阴鳞状细胞上皮内瘤样病变和外阴非鳞状细胞上皮内瘤样病变,发病多见于中老年或伴有免疫抑制的妇女。V1N的症状无特异性,与外阴营养不良一样,主要为瘙痒、皮肤破损、烧灼感、溃疡等。体征有时表现为丘疹或斑点,单个或多个,融合或分散,灰白或粉红色;少数为略高出表面的色素沉着。

外阴鳞状细胞癌是最常见的外阴恶性肿瘤,占外阴恶性肿瘤的80\%~90\%。多见于60岁以上妇女,症状主要为不易治愈的外阴瘙痒和各种不同形态的肿物,如结节状、菜花状、溃疡状。肿物合并感染或癌症晚期可出现疼痛、渗液和出血。癌灶可生长在外阴任何部位,大阴唇最多见,其次为小阴唇、阴蒂、会阴、尿道口、肛门周围等。早期可见局部丘疹、结节或小溃疡;晚期见不规则肿块,伴或不伴破溃或呈乳头样肿瘤,若癌灶已转移至腹股沟淋巴结,可扪及一侧或双侧腹股沟淋巴结增大、质硬、固定。

外阴良性肿瘤一般对性功能没有影响。外阴癌前病变、原位癌及早期外阴癌可出现局部瘙痒、浅表溃疡及渗出增加。晚期局部硬结增大,可呈菜花状,组织坏死脱落形成溃疡、感染,排出脓性或血性分泌物,使患者感到不适。健康状况下降,并在心理上造成压力,而影响正常的性欲、性反应和性生理功能。

单纯外阴切除术手术范围包括部分大阴唇、小阴唇、阴蒂及部分会阴体,切除组织深度为皮肤及部分皮下组织,不需达会阴部筋膜。单纯外阴切除术后阴蒂消失,从而使其作为女性性感受体及传感器的功能不复存在。在性反应的兴奋期没有阴蒂海绵体的充血及勃起,亦不能接受达到性紧张所必需的性刺激,而影响性高潮出现的频率和强度。由于切除了部分大阴唇及小阴唇,阴道口组织张力增加,缺乏弹性。术后阴道口及尿道口由于没有小阴唇保护而外露,容易受到摩擦产生阴道炎、尿道炎及膀胱炎,患者有一种“敞开”感和不适。单纯外阴切除术后,外阴形态的变化都会使患者及其配偶在心理上产生异常的感觉,从而影响到性功能,以致产生性功能障碍。但是,术后1年均能逐渐恢复性生活。性生活恢复的早晚及程度与手术范围有一定关系。妇科医生在施行手术时,应在病情允许的前提下,适当保留部分阴蒂海绵体,减少正常组织的创伤。

广泛性外阴切除术的手术范围包括阴蒂、大阴唇、小阴唇及部分会阴体,切除组织深度为皮肤、脂肪组织、直达筋膜,腹股沟淋巴结摘除术的切口始于髂前上棘内3cm,下达股三角尖端,切除范围为腹股沟深浅淋巴结及其周围脂肪组织。手术后,外阴组织有较大缺损,张力大,且因皮下脂肪已剔除,局部血液循环不良,切口容易裂开及感染,愈合差,甚至需要植皮。腹股沟部切口因缺乏皮下脂肪而出血、感染,愈合不佳,需要二期处理。切口愈合后往往形成较大的瘢痕,外阴变形,阴毛移位而稀少,可严重损伤患者的心理状态及体型感。阴阜及外阴部因缺乏脂肪组织而变得扁平、僵硬,在男上位性生活时双方均有不适感。由于瘢痕挛缩,阴道缩短,阴道口狭窄,缺乏弹性,会出现性交困难或性交疼痛。手术后,尿道口外露,因瘢痕牵扯而使尿道缩短,尿流偏向侧方,偶尔因结缔组织收缩而贴近耻骨时,尿流可进入阴道,也可产生尿失禁,反复膀胱炎等。上述改变均可使配偶关系疏远,影响性生活。较大的瘢痕能产生外阴及大腿部持续性麻木感,摘除腹股沟深、浅淋巴结后产生的下肢淋巴肿均可减低性反应。由于阴蒂、阴唇及阴道下部感觉神经末梢的损伤,爱抚对患者并不起作用。1983—1988年Anderson Moth、Stellman等报道有50\%~80\%的妇女在广泛性外阴切除术后完全停止性生活,但她们的性欲正常。1986年Weijmar对10对配偶进行回顾性研究,发现她们的性欲均正常,且8例在手术后性功能部分或全部恢复正常,双方对性关系满意。1990年Weijmar又对广泛性外阴切除术进行前瞻性研究,发现术前性生活正常者,在术后1年均恢复了性生活,但是,生殖器接受性刺激的感觉较差。同时发现术后患者对双方感情上的满足比对生理上的性满足更加迫切,也表现出感情的平衡对性功能的恢复有促进作用。Merrill认为,广泛性外阴切除术虽然损伤了阴部神经末梢,产生感觉障碍,然而并未涉及支配盆底肌肉的阴部神经纤维,因此,手术后高潮功能的障碍,主要是感觉障碍及中枢性唤起障碍,为心理因素所致。

阴道良性肿瘤有平滑肌瘤、纤维肌瘤、乳头状瘤、神经纤维瘤、血管瘤等。其中以平滑肌瘤最为常见,阴道平滑肌瘤在任何阶段的成年妇女中均可发生,可发生于阴道任何部位,大小不等,平均直径3cm,通常见于阴道前壁。肿瘤较小时可无症状,随肿瘤逐渐长大,可出现白带增多,下坠感,发现阴道肿块。出现膀胱、直肠压迫症状如尿频、尿急、大小便困难或性交困难。当肿瘤有溃疡、坏死时,可出现白带增多、阴道出血。临床检查可发现阴道壁上有实性、质硬、边界清楚的肿块,并可向阴道内或外阴突出。肿瘤常为单个,诊断时应注意与膀胱、直肠膨出,阴道壁囊肿或阴道壁包涵囊肿鉴别。

阴道上皮内肿瘤(vaginal intraepithelial neoplasia.VAIN)是指局限于阴道上皮层内的不同程度的非典型增生病灶,它是阴道浸润性癌的癌前病变。阴道上皮内肿瘤可无症状,偶有性交后出现血性白带、反复阴道排液和不正常阴道出血,体征上阴道黏膜可无异常,或仅呈糜烂。

阴道鳞状上皮癌是最常见的阴道恶性肿瘤,发病高峰期在50~70岁,60岁以上者占半数。阴道上皮内肿瘤或早期浸润癌可无明显症状,或仅有阴道分泌物增多和接触性出血。随着病程的进展,阴道癌灶的增大、坏死,可出现阴道排恶臭液、无痛性阴道出血。当肿瘤向周围器官和组织扩展时,可出现相应的症状。累及尿道或膀胱可出现尿频、尿急、血尿和排尿困难;累及直肠可出现排便困难或里急后重;阴道旁、主韧带、宫骶韧带受侵犯时,可出现腰骶部的疼痛等。

阴道良性肿瘤一般不引起性功能障碍,少见较大的良性肿瘤或囊肿能产生性交困难。阴道恶性肿瘤出现阴道分泌物增多、恶臭、不规则阴道出血、性交出血时均可抑制性欲,产生性反应低下。瘤体占位较大,亦可引起性交困难。阴道良性肿瘤一般可行肿瘤切除术,阴道腺病用冷冻、激光及手术治疗时应注意防止阴道狭窄及粘连,以免引起性交困难、性交疼痛或分娩困难。

阴道癌除了行广泛性子宫切除或广泛性外阴切除外,由于癌肿向膀胱、直肠浸润,往往需要行前盆腔脏器切除术、后盆腔脏器切除术或全盆腔脏器切除术。此类手术范围广,创伤大,能严重影响患者的性功能。如手术后采用放疗或单纯采用放疗,可引起放射性阴道炎、放射性膀胱炎及放射性直肠炎,或由于组织坏死脱落而形成膀胱阴道瘘和直肠阴道瘘,使病情进一步复杂化,导致性生活困难及障碍。如手术及放疗后恢复较好,则可以应用润滑剂或阴道扩张器辅以消炎药治疗,以尽早恢复性交的功能。

(1)子宫肌瘤:子宫肌瘤是女性生殖器官最常见的良性肿瘤,多发生于35~45岁的妇女,发生率为20\%~30\%,其中30\%~50\%发生于育龄女性,不孕女性中肌瘤则更为常见。根据瘤体的数量、位置不同,子宫肌瘤分为多发性子宫肌瘤、浆膜下子宫肌瘤、黏膜下子宫肌瘤及宫颈肌瘤。子宫肌瘤的症状与肌瘤生长部位、生长速度及肌瘤变性有关,与肌瘤数目、大小关系不大。多数患者无症状,仅于妇科检查或B超检查时偶被发现。常见症状有:①阴道流血:为最常见症状,肌壁间肌瘤表现为月经量增多,经期延长或周期缩短,黏膜下肌瘤表现为不规则阴道流血、月经过多,浆膜下肌瘤常无月经改变。②腹块:下腹触及实质性肿块,不规则,质硬,特别是在清晨膀胱充盈时更为明显。③白带增多:肌壁间肌瘤可有白带增多,黏膜下肌瘤更为明显,当其感染坏死时,可产生大量脓血性排液,伴有臭味。④压迫症状:肌瘤增大时常可压迫周围邻近器官而产生压迫症状,尤多见于子宫体下段及宫颈部肌瘤。压迫膀胱则产生尿频、尿急,甚至尿潴留;压迫直肠产生排便困难。⑤腰酸、下腹坠胀、腹痛:一般患者无腹痛,常诉有下腹坠胀、腰背酸痛。浆膜下肌瘤蒂扭转时可出现急腹痛。肌瘤红色变性时,腹痛剧烈且伴发热。⑥其他症状:患者可伴不孕,继发性贫血等。

(2)宫颈癌:子宫颈癌的发病率在女性恶性肿瘤中居第二位,仅次于乳腺癌。全世界每年估计有46.6万子宫颈癌新发病例,其中80\%发生在发展中国家,子宫颈癌的年轻病例有逐年增加的趋势。近年来,子宫颈癌在病因研究方面取得了重大进展,特别在生物学病因方面取得了突破。与子宫颈癌最为密切的相关因素是性行为,人们很早就怀疑某些感染因子的作用。在20世纪60~70年代,人们将目光重点投向单纯疱疹病毒(herpes simplex virus,HSV)Ⅱ型,因为HSV在体外被证实具有一定的致癌性,且在子宫颈癌标本中有一定的检出率。但临床活体标本能检出HSV的始终仅占一小部分,流行病学调查也不支持HSV与子宫颈癌相关。而其他的因子如巨细胞病毒、EB病毒、衣原体等至今未能发现有力的证据。1977年Laverty在电镜中观察到子宫颈癌活检组织中存在人乳头瘤病毒(human papilloma virus,HPV)颗粒,在Zur Hausen提出HPV与子宫颈癌发病可能有关的假设后,国内外学者就HPV感染与子宫颈癌的关系进行了大量的研究,并获取了许多证据。几乎所有的流行病学调查和实验室研究数据均显示,HPV感染是子宫颈癌的主要流行因素,HPV与子宫颈癌高度相关,其相对危险度或危险度比值高达250。这些流行病学资料结合实验室的证据都强有力地支持HPV感染与子宫颈癌之间的因果关系,均表明HPV感染是子宫颈癌的病因,即HPV感染是子宫颈癌发生的必要条件。1995年IARC专题讨论会认为:HPV感染是子宫颈癌的主要病因。

早期宫颈癌常无症状,也无明显体征,很难与慢性宫颈炎相区别。有时甚至见宫颈光滑,尤其是宫颈已萎缩的老年妇女。有些宫颈癌患者,病灶位于宫颈管内,宫颈阴道部外观正常,易被忽略而漏诊或误诊。患者一旦出现症状,主要表现为:①阴道流血:年轻患者常表现为接触性出血,发生在性生活后或妇科检查后出血。出血量可多可少,根据病灶大小、侵及间质内血管的情况而定。早期流血量少,晚期病灶较大,表现为大量出血,一旦侵蚀较大血管可能引起致命性大出血。年轻患者也可表现为经期延长、周期缩短、经量增多等。老年患者常主诉绝经后不规则阴道流血。一般外生型癌出血较早,血量也多;内生型癌出血较晚。②阴道排液:患者常诉阴道排液增多,白色或血性,稀薄如水样或米泔状,有腥臭。晚期因癌组织破溃,组织坏死,继发感染有大量脓性或米汤样恶臭白带。③晚期癌的症状:根据病灶侵犯范围出现继发性症状。病灶波及盆腔结缔组织、骨盆壁、压迫输尿管或直肠、坐骨神经时,患者诉尿痛、尿急、肛门坠胀、大便秘结、里急后重、下肢肿痛等,严重时导致输尿管梗阻、肾盂积水,最后引起尿毒症。到了疾病末期,患者出现恶病质。

(3)子宫内膜癌:子宫内膜癌又称子宫体癌,绝大多数为腺癌,为女性生殖道常见的三大恶性肿瘤之一,高发年龄为58~61岁,约占女性癌症总数的7\%,占女性生殖道恶性肿瘤的20\%~30\%,近年发病率有上升趋势,已趋于接近甚至超过宫颈癌。

子宫内膜癌早期无明显症状,仅在普查或因其他原因检查时偶然发现,一旦出现症状则多表现为:①阴道流血:主要表现为绝经后阴道流血,量一般不多,大量出血者少见,或为持续性或间歇性流血;尚未绝经者则诉经量增多、经期延长或经间期出血。②阴道排液:少数患者诉排液增多,早期多为浆液性或浆液血性排液,晚期合并感染则有脓血性排液,并有恶臭。③疼痛:通常不引起疼痛。晚期癌瘤浸润周围组织或压迫神经引起下腹及腰骶部疼痛,并向下肢及足部放射。癌灶侵犯宫颈,堵塞宫颈管导致宫腔积脓时,出现下腹胀痛及痉挛样疼痛。④全身症状:晚期患者常伴全身症状,如贫血、消瘦、恶病质、发热及全身衰竭等。

(1)子宫良性肿瘤:子宫良性肿瘤早期往往没有症状,对性功能没有影响。如果出现不规则阴道流血,由于女方害怕感染及对疾病的恐惧,而主动减少性交频率,性反应受到抑制。黏膜下子宫肌瘤患者,因瘤体缺血、坏死、脱落、感染而伴较多量不规则阴道出血,往往导致性生活终止。较大的黏膜下子宫肌瘤脱出于阴道内产生性交困难,此类患者中常见因性生活不满意而分居甚至离婚者。

(2)子宫恶性肿瘤诊断前、后的性功能:在诊断子宫恶性肿瘤或其他妇科恶性肿瘤前,患者往往有不规则阴道出血、月经过多、出现肿块或疲劳等情况,均可引起性功能障碍。Anderson于1986年观察了41例早期宫颈癌及子宫内膜腺癌患者,她们在妇科癌症状发生前性功能正常,在出现症状,尚未明确诊断前有75\%发生性功能障碍,其中包括性欲抑制56\%、性唤起抑制49\%、性高潮障碍37\%及性交困难37\%。1983年Cain等研究了诊断为妇科癌后对患者性功能的影响。他们观察了60例诊断为子宫颈癌、子宫体癌或卵巢癌后一个月的患者,40例心理学门诊妇女及272例正常妇女,对照分析中发现诊断妇癌后的患者有轻度和中度的精神压抑和焦虑,职业和料理家务能力下降,性功能损伤。虽然她们的婚姻关系正常,但性关系发生了显著变化。在60例中29例配偶关系美满者,于妇癌诊断后停止了性生活。因此,Cain指出,诊断妇癌后工作能力、进行家务劳动能力的减退,性活动的停止,是直接由患者对癌症及其治疗的心理反应造成的,医生语言的暗示也有关系,使患者认为性交可能使阴道出血及癌症加重,而产生性功能障碍。1985年Anderson的研究结果亦指出,诊断妇科癌症后,患者感到恐惧、悲哀、焦虑和消沉,在此种情况下,不可能对性活动产生兴趣,均表现出性欲及性唤起抑制,性高潮功能障碍。

(1)宫颈锥切术与性功能:宫颈锥切是一个古老的妇科手术,已有近180年的历史。1815年Lisfranc首次描述了宫颈锥切术,当时的指征包括了感染和癌症。1861年MarionνSims用银丝线缝合锥切后的宫颈。1916年Sturmdorf报告了宫颈锥切缝合方法。1938年Martzloff第一次提出宫颈锥切术的名称,并用于诊断宫颈原位癌。随着锥切方法的不断改进,其在宫颈病变的诊断和治疗中越来越显示出它特有的临床价值。常用的锥切术方法有:冷刀法(Cold Knife cone)、激光法(Laser Conization)、电刀法(Electrodiathermy)、环行电切法(LEEP conization)。其中环形电切法应用最广,宫颈环型电切术(loop electrosurgical excisional procedure,LEEP)是近年发展起来的一种新技术。该方法采用低电压、高电流以及细小的环型电刀切除宫颈病变,可在门诊进行并可提供标本进行病理学检查,简便、易行,且不影响将来妊娠,是宫颈病变安全、有效的诊治方法。

1982年Kikku研究了因宫颈非典型增生和原位癌而行宫颈锥形切除的患者,发现在性欲、性高潮、性交频率、性满足方面没有任何变化,从而指出:宫颈锥形切除术不会引起性功能障碍。

(2)子宫切除术与性功能:子宫切除术可用于治疗各种子宫疾病,如子宫肌瘤、子宫内膜异位症、严重的功能失调性子宫出血、子宫颈重度非典型增生或原位癌、子宫内膜癌前病变等。全子宫切除术系在阴道穹隆部切除子宫和宫颈。次全子宫切除术是在子宫颈峡部切除子宫体,保留宫颈及完整的阴道穹隆。

子宫位于阴道的顶端,在排卵前性兴奋时宫颈黏液的分泌明显增加,在性高潮时子宫位置升高,是参与性兴奋反应的组成部分之一。1963~1987年Patterson及Gijsbers等的众多研究指出,单纯子宫切除术本身不会对性功能产生机械的和生理的影响,这类患者的性功能障碍往往是心理源性。患者在子宫切除术后仍能完成正常的性交,也会出现满意的性高潮反应,这可能是因为子宫切除术后并不影响阴蒂、阴唇、阴道等性敏感区的反应,且性快感中心在大脑中枢,多年的性生活体验在大脑中枢已形成精神神经反射。在这方面,夫妇深厚的感情也起一定的作用。部分患者在子宫切除术后可出现不同程度的忧郁感、眩晕、失眠、精神异常等症状,称为子宫切除术后综合征。患者普遍对子官切除存在紧张忧虑和恐惧心理,担心手术操作者的技术,担心术时术后疼痛,担心要输血,更担心术后的性功能,知识贫乏者还会担心子宫切除术后腹腔内是否留下一个空洞、术后会变性等。医生术前未能较好地帮助患者解除思想负担,患者术后往往会出现各种不适反应,术前思想负担越重,术后不适反应发生率也越高。术前心理准备越充分,术后出现子宫切除术后综合征者越少。子宫切除虽保留了双侧卵巢,但据学者们报道,保留下来的卵巢功能可能是由于血运问题而较未手术妇女早几年衰退。张春风随访了37例子宫切除术后患者,探讨她们的心理、精神反应和术后性生活情况,本组病例中16例术后无性欲,13例性欲减退,性功能障碍表现较明显。这组病例在术前的忧虑情绪也较严重。因此,对待子宫切除术采取审慎态度,应有明确的手术指征,术前必须了解并帮助患者认识手术的必要性及消除顾虑,术中应注意保持阴道的长度,避免引起盆腔内感染及粘连。术后指导患者夫妇适时恢复性生活有良好效果,恢复性生活是消除患者畏惧术后变性或影响性功能最好的办法。

但是,对于那些在性反应中子宫起到积极作用的妇女,性高潮时有明确的子宫收缩感,在子宫切除术后,则会影响到她们对此种收缩感的感受。此外,子宫切除术后的妇女,由于子宫动、静脉已结扎,在性反应中盆腔充血减少。兴奋期不再有子宫抬高,阴道穹隆膨胀、扩张的帐篷现象产生。由于双侧主韧带、骶韧带及圆韧带已切断,子宫已切除,在性反应的兴奋期和平台期则感受不到高度的性紧张。对于那些体验以外阴性高潮为主的妇女,子宫切除术后则没有上述影响。

手术后阴道断端的瘢痕可引起双方性交时的不适感,3个月后瘢痕逐渐软化,不适感渐渐消失。通常,子宫切除术后2~3个月开始逐渐恢复性生活,术后4个月恢复至术前的性交频率,不致发生性功能障碍。部分妇女误认为子宫切除术后不来月经、不能生育,性功能也会自然减退,甚至认为自己已变成“中性人”,感到性生活没有意思,于是发生性欲低下,性感缺乏,性交次数减少。另外一些妇女则不同,手术纠正了严重的子宫出血、贫血及脏器压迫感,消除了患肿瘤的思想压力或对生育的顾虑,体验到一种轻松和解脱,以及出自内心的自豪和充实,反而使性欲提高,性反应增强,配偶关系更加协调。

(3)子宫广泛切除术与性功能:子宫广泛切除术是全子宫切除的一种,包括切除主韧带至少3cm以上,切除阴道上段至少3cm以上。每个病例切除主韧带及阴道壁的多寡须根据病灶浸润的范围而定,切缘至少距离病灶2cm以上。适应证:①子宫颈癌Ⅰb期(Ⅰb1及Ⅰb2期)。②子宫内膜癌Ⅱ期以上(包括Ⅱ期)或Ⅰ期有深肌层浸润或癌灶已达子宫颈内口水平或病理分级G3者。盆腔淋巴结清扫术是宫颈癌手术治疗的重要部分,它关系到手术的彻底性和手术效果。盆腔淋巴结清扫术的范围应包括髂总、髂外、髂内、闭孔、腹股沟深、骶前淋巴结等,有时还需清除腹主动脉旁淋巴结。手术可经腹膜内或腹膜外进行。

盆腔脏器的神经支配主要为交感神经及副交感神经。盆腔神经丛主要位于阔韧带底部的子宫颈旁组织内,分布于子宫体、子宫颈及膀胱上部,盆腔神经丛中有来自第Ⅱ、Ⅲ、Ⅳ骶神经的副交感神经纤维,并含有向心传导的感觉神经纤维。广泛性子宫切除术中均可能损伤上述神经纤维,影响膀胱功能及性功能。广泛性子宫切除术及盆腔淋巴结清扫术可以产生各种并发症,如膀胱、输尿管及直肠损伤,术后产生排尿功能障碍,膀胱麻痹及尿潴留,并发严重泌尿系感染,或膀胱阴道瘘和直肠阴道瘘。由于盆腔淋巴结已摘除而产生下肢淋巴结循环障碍及腹股沟部淋巴囊肿,均可引起性功能障碍。1980、1982年Forney和Sasaki对宫颈癌行广泛性子宫切除术后的膀胱功能及泌尿生殖系统的神经生理进行了研究,发现确有部分交感神经及副交感神经损伤,认为损伤程度与宫旁组织、阴道旁组织切除的多少,以及阴道切除的长短有关。盆腔神经损伤引起的术后膀胱功能障碍通常是暂时性的,主要表现为排尿感消失或迟钝,排尿困难及压力性尿失禁,亦有永久性膀胱功能障碍者。1983年Kadar资料显示,广泛性子宫切除术后的严重排尿障碍者占5\%,由于排尿不畅及困难能引起顽固的泌尿系感染,反复发热、腰痛、尿频、尿急、尿痛,造成较大的精神压力及身体不适感,明显抑制性欲及性反应,以致完全停止性交。盆腔神经丛自主神经及感觉神经纤维的损伤,以及子宫动静脉的结扎,使性唤起和性交时阴道的充血、润滑功能减退、阴道感觉迟钝。阴道断端与膀胱三角、直肠形成粘连,引起性交疼痛。广泛性子宫切除术及盆腔淋巴结清扫术中可能损伤坐骨神经、闭孔神经、生殖股神经、髂腹股沟神经、腹下神经及侧股皮神经。1987年Hoffman报道,其总体发生率为1.06\%。上述神经损伤后可产生下肢肌肉运动功能障碍及大腿部麻木感和感觉消失。损伤生殖股神经则在阴唇部发生麻木感及感觉消失、异常,如损伤髂腹股沟神经和髂腹下神经能引起大腿上1/2及阴阜部、大阴唇部麻木及感觉异常。这些感觉功能障碍对患者全身健康的恢复不利,亦能影响患者接受必要的性刺激,产生性功能障碍。

有学者观察了子宫良性肿瘤行单纯子宫切除术与妇科恶性肿瘤行广泛性子宫切除术或放射治疗后性功能障碍的差异。1989年Anderson等研究了47例早期卵巢癌、宫颈癌及子宫内膜腺癌患者,其中广泛性子宫切除23例、放射及手术治疗11例,手术及放射治疗13例,与良性肿瘤行单纯子宫切除术18例和健康妇女57例对比观察,了解治疗前、治疗后4个月、8个月及12个月的性功能状况,发现在治疗后4个月时良性与恶性肿瘤组的性交频率均下降,治疗后8个月及12个月时两组性交频率与健康妇女无显著差异。但是,恶性肿瘤组表现出性反应障碍。其中性兴奋明显下降,且往往为持续性。作者认为这是由于广泛性子宫切除术及放射治疗后引起的生理及病理的改变,导致性交疼痛、性交后疼痛、过早绝经及各种治疗的副作用及并发症所致,与焦虑没有显著关系。

对于晚期妇科癌症进行盆腔脏器切除术后的性功能亦有一定研究。1975年Demp-sey对16例复发宫颈癌、晚期宫颈癌和外阴癌进行全盆腔脏器切除术患者进行了随访,发现87.5\%的存活者均有性功能障碍。10例术前性生活正常的患者,只有3例术后有性生活,这3例均未进行外生殖器切除术。1978年Lamont等观察12例盆腔恶性肿瘤行盆腔脏器切除术的患者,其中10例同时行阴道重建术,在术前、术后6周及12个月随诊中,发现8例在术前对性生活态度正常及调适较好的妇女,其中7例在术后12个月恢复了性生活,此7例中有6例术后6个月恢复了性高潮。此研究说明,虽然赖于产生性反应的盆腔组织受到严重损伤,仍可保持性高潮功能。不过,需要给患者更多的爱抚和刺激,才能达到术前的性兴奋水平。

随着宫颈癌发病的年轻化和人们生活水平的提高,患者迫切要求在手术时尽量减少对手术后性生活的影响。有效的解决方法是在手术过程中根据适应证行阴道延长和卵巢移位术。临床资料显示卵巢内分泌的性激素与宫颈癌的发生没有确切的关系,早期宫颈癌的卵巢转移率很低,Ⅰa~Ⅲb期宫颈鳞癌的卵巢转移率为0~2.5\%,腺癌则为1.7\%~28.6\%。故除宫颈腺癌外,45岁以下的早期宫颈癌者,只要卵巢外观正常,可保留1侧或2侧卵巢。Ⅱb期以上采用放疗者,如患者年轻,放疗前可先用腹腔镜进行卵巢移位术,保留的卵巢需移位到结肠旁沟中部并标记。对宫颈腺癌患者行卵巢移位术应持慎重态度。另外,Ⅲ型以及Ⅲ型以上子宫切除者,术后阴道较短,将对性生活造成一定影响。对年轻患者可在手术同时行阴道延长术,可采用腹膜反折阴道延长术或者乙状结肠阴道延长术,前者操作简单,后者效果更好。

(4)卵巢切除术与性功能:卵巢是维持女性内分泌活动的重要器官,卵巢功能的保持,也使下丘脑—垂体—卵巢轴处于正常的反馈调节状态。如卵巢缺失,除直接造成雌激素水平低下外,还产生由雌激素低下带来的系列症状和并发症,如更年期综合征、骨质疏松、脂代谢紊乱、心血管疾病发生率增高、性器官萎缩、性功能衰退等。对患者本人健康、夫妻关系、工作和学习都会产生严重的影响。因此,应慎重对待卵巢切除术。

对子宫的良性疾患应保留双侧卵巢。曾有人认为保留单侧卵巢已能满足内分泌的需要,但有不少研究报道,保留双侧卵巢较单侧卵巢对维持卵巢正常功能的效果更好。也有人认为,若年已近绝经,卵巢功能已衰退,手术时可一并切除,以防日后发生卵巢肿瘤,这是一种预防性切除的观点。近年来,人们已逐渐认识到卵巢在维持老年妇女身心健康的价值。据报道,绝经期后妇女的卵巢并非无内分泌作用,卵巢静脉血中睾酮与雄烯二酮较外周血高12倍和4倍,在外周脂肪组织中可转化成雌激素。我们曾研究绝经后妇女的卵巢功能,也得出同样的结论。因此,对手术中即使已过绝经年龄患者,只要卵巢无病理情况,均应当保留,实行预防性切除有欠妥之处。

虽然保留卵巢可能会使其功能较快衰退,但可避免功能突然消失带来的更为明显的更年期症状。卵巢切除虽然可用激素替代治疗,外源性补充雌激素,可改善由于低雌激素带来的一些问题。但仿周期给药始终不如保留卵巢对维持下丘脑-垂体-卵巢轴功能那样协调,因为卵巢产生的激素是多方面的,而且是个量变的过程,在它们的正、负反馈调节下,下丘脑-垂体-卵巢轴才能处于正常生理状态,仿周期激素替代治疗并不能完全取代正常卵巢的功能。

(5)放射治疗与性功能:放射治疗是妇科癌症的重要治疗方法,可用于手术前、手术后或单独采用。其对宫颈癌的治疗效果与手术治疗效果相近。放射治疗后的宫颈癌患者的性功能变化受到普遍重视。1974~1983年Abit-bol、Davenport及Bertelsen等指出,阴道上皮的基底层对电离辐射十分敏感,其小血管的内皮细胞及结缔组织的成纤维细胞为中度敏感。放射治疗后阴道的小血管狭窄、闭塞,结缔组织增生,血管扩张能力丧失,使性唤起时的阴道充血、润滑作用和高潮功能受到抑制。放疗后的结缔组织增生,使阴道弹性消失和变形,损害了在性活动中阴道延长、扩张等生理反应。他们认为放疗造成性交困难的原因是:①阴道上皮变薄,润滑功能下降;②阴道狭窄、变形、弹性减低;③阴道易于损伤及感染;④卵巢功能衰退、阴道萎缩;⑤性交次数减少。认为只要没有对卵巢进行特殊的保护措施,腹部及盆腔放疗后均产生永久性的卵巢功能衰竭。1980年Poma估计,放疗后生殖系统的组织学变化可持续3年以上,其后逐渐稳定,并不可逆转。

宫颈癌放射治疗后患者可产生比手术后患者更严重的性功能障碍,1980年Marf观察67例宫颈癌Ⅱ、Ⅲ期放疗后7~13年的妇女,其中性交停止者占34\%,因解剖因素不能性交者占43\%,性交次数减少者占40\%。宫颈癌放疗后可引起皮肤、皮下组织、直肠、膀胱及输尿管损伤。一般在宫颈癌术前或术后放疗的患者仅产生轻度的全身乏力、尿频、尿痛、大便次数增多、下坠感及暂时性性功抑制,严重放疗后并发症仅占1\%~2\%。而单纯用放射治疗的宫颈癌患者,放射剂量比手术加放疗者为多,其性功能障碍发生率高,常为持续性,放疗并发症达17\%。以直肠病变为主,很多患者有腹泻、肠痉挛等症状。放疗后6~12个月产生里急后重、便血、直肠部疼痛或尿频、尿急、尿痛、尿血等放射性直肠炎及放射性膀胱炎等症状。放疗后数年,能发生输尿管狭窄,输尿管及肾盂积水,或直肠阴道瘘和膀胱阴道瘘。放射治疗后能产生严重放射性阴道炎,阴道萎缩狭窄、变形,润滑作用明显减退。

Anderson BL.(1989,1992)报道,妇科恶性肿瘤治疗后出现性功能障碍的患者中,根治术后患者占78\%,放疗后患者占44\%~79\%,而多数专家认为放射治疗比手术对治疗后性功能的影响更明显。但有时由于治疗中的神经损伤或结构变化,可使患者不能感到性唤起和性兴奋。如性交疼痛是放疗后的常见问题,特别是腔内放疗后,直接损伤阴道上皮而使上皮血管不能充盈、肿胀而减少阴道漏出液,因此,阴道干燥、性交困难、疼痛。阴道变短、狭窄也是造成性交困难的原因,这种情况下的性交,不容易达到性兴奋,更不可能获得性高潮了。这种情况就需要医生的帮助和指导。

(6)化学治疗与性功能:化学治疗是治疗恶性肿瘤的主要手段之一,在卵巢癌、子宫颈癌及子宫体癌的治疗中起着重要的作用。根据细胞增殖动力学,选择合适的药物,根据肿瘤的生长规律和生物学特性,选择适当的给药途径,充分发挥药物的抗肿瘤作用,显著提高疗效。常用的药物有氟尿嘧啶、环磷酰胺、长春新碱、多柔比星(阿霉素)、博来霉素、顺铂等。

化疗对性功能的影响亦受到重视。单纯、连续化疗的患者可引起卵巢功能衰竭及低下,影响到性功能。如病情允许,外源性雌激素治疗能缓解阴道干燥、性交疼痛和性欲、性唤起功能低下。化疗引起脱发,一般持续6个月后恢复。脱发可以严重改变患者的体像,产生恐惧感,影响自尊,觉得自己失去吸引力而不愿见人,更不愿与配偶接触,严重抑制性功能。目前,大多数妇女均使用假发套以维持自身的体像,保持心理平衡和正常的社交活动。一般更生霉素可引起较严重脱发,而患绒毛膜上皮癌及恶性葡萄胎的年轻妇女使用较多。因此,在治疗前要给患者做好宣传教育,使她们有所准备,不致在化疗中因每日产生大量脱发而不知所措,还要鼓励她们,说更黑更亮的头发很快会长出来,这样才不致使其丧失信心,甚至影响配偶间的关系。长春新碱是卵巢癌常用的化疗药物,可引起周围神经病变,四肢麻木、感觉异常,影响性功能。化疗不仅有上述影响性功能的副作用,对骨髓亦有抑制作用,导致白细胞下降及血小板减少。常有恶心、呕吐及腹泻等消化道症状,有时影响肝脏功能,转氨酶上升。患者感到全身虚弱、乏力、食欲不振。在这种健康状况下,患者的性欲、性唤起、性高潮均受到抑制。

(1)卵巢良性肿瘤:卵巢肿瘤是妇科的常见肿瘤,其组织学分类繁多,占全身肿瘤之首位。常见的卵巢良性肿瘤有发生于上皮的浆液性囊腺瘤、黏液性囊腺瘤,和发生于生殖细胞的良性畸胎瘤,以及来自卵巢非特异性间质的纤维瘤、血管瘤、平滑肌瘤及脂肪瘤等。卵巢良性肿瘤还需与卵巢非赘生性囊肿鉴别,如卵泡囊肿、黄体囊肿、多囊卵巢及卵巢子宫内膜异位症等。卵巢良性肿瘤的主要症状是腹部包块及腹痛,有时出现尿频、尿急和下坠感等膀胱、直肠压迫症状。肿瘤蒂扭转可引起腹痛。通常妇科检查及B型超声波检查能早期明确诊断。

(2)卵巢恶性肿瘤:卵巢恶性肿瘤居妇科癌症发病率的第3位,近年来有增加趋势,由于其早期多无症状,有60\%的病例于诊断时已为Ⅲ或Ⅳ级(FIGO临床分期),其病死率占妇科癌症首位,5年存活率仅13.0\%~63.0\%。卵巢原发性恶性肿瘤的组织分型繁多,有上皮性浆液性囊腺癌、黏液性囊腺癌和来自生殖细胞的实性畸胎瘤、无性细胞瘤及内胚窦瘤等。发生于性索间质的有颗粒细胞瘤、非特异间质的纤维肉瘤、平滑肌肉瘤等。另有来源于胃肠、乳腺及子宫的转移瘤,如库肯勃瘤(Krukenberg tumor)等。主要症状为腹部包块,腹痛,腹部胀满及膀胱、直肠压迫症状,有腹水时产生下肢浮肿、呼吸困难。晚期患者肿瘤压迫神经而生产下肢疼痛。根据病史、妇科检查、B型超声波检查、腹水脱落细胞检查及腹部CT检查能明确诊断。

卵巢良性肿瘤一般对性功能无明显影响,如腹部肿块较大,影响患者行动及产生压迫感,或有卵巢囊肿扭转引起慢性或急性腹痛,则可能抑制性欲,导致性反应降低,性交频率下降。但手术后上述性功能障碍均能自愈。卵巢恶性肿瘤早期并无症状,当出现腹部包块、腹水引起上腹部不适、呼吸困难、下肢浮肿或肿瘤蒂扭转、破裂、感染产生慢性或急性腹痛时,则严重影响患者的性功能,不仅使性欲低下,性反应抑制,甚至导致性交停止。卵巢颗粒细胞瘤为性索间质的低度恶性肿瘤,瘤细胞能分泌雌激素,使老年妇女出现乳胀、子宫出血及性功能亢进。也有一些肿瘤产生雄激素,使患者多毛、阴蒂肥大、乳房萎缩、性欲亢进。手术后症状消失。

卵巢恶性肿瘤的手术治疗、化学治疗、放射治疗对性功能的影响与子宫颈癌一致。

妇科恶性肿瘤进行广泛性外阴切除术、广泛性子宫切除术、盆腔脏器切除术及放射治疗后全身及泌尿生殖系统产生一系列生理、解剖、内分泌及心理方面的变化,这些变化均能引起患者器质性和心理性的性功能障碍,因此,性功能的恢复是妇科癌症治疗的组成部分。妇科恶性肿瘤及其治疗对妇女影响很大,它关系到患者的生命、婚姻及家庭的稳定,关系到她的体像、吸引力,术后性生活以及作为一个人的社会及精神方面的价值。所以,为了使患者顺利适应妇科癌症带来的变化,提高治疗后妇女的生活质量,医务人员从治疗前,在治疗中及治疗后均应予以恢复性功能的指导。

(1)年龄因素:年轻患者,未曾体验过生育、抚养及较长稳定婚姻者比年龄较长,曾经有生育、抚养和美满家庭者较易发生性功能障碍。

(2)手术前心理调适状态不好:性心理不稳定,遇到威胁即表现抑制者,为高危人群。

(3)顾虑手术会影响性功能者亦为高危因素。1977年Dennerstein资料证明,手术前即考虑可能影响性功能的妇女,在手术后阴道润滑功能下降,性欲减退及性交困难的发生率比没有顾虑的患者要高。

(4)性角色的作用(Gender role defini-tion)。Latorre根据大量资料指出,那些特别女气的妇女(highly feminine female)常常容易产生性功能障碍,因为她们的个性趋向刻板(tend to be rigid),对丈夫的期望不易变。女气的男性或女性(feminine individual-swhether male or female),其调适障碍评分最高。这一发现与Bem's提出的男性作用(an-drogenous gender role definition)能提供较佳调适能力的观点一致。

生殖系统癌症对妇女有双重影响,首先是癌症威胁生命,其次,癌症影响她的体像、自尊、吸引力、性生活及在文化及精神、社会方面的作用。我们应当了解妇科癌症及癌症治疗给患者带来的生理病理变化及其对性功能的影响,亦要十分重视社会心理学及性心理的动态。切除子宫的人常常认为自己是不完整的人,而对广泛性外阴及子宫切除术的看法更是如此。手术不仅影响患者本身,而且使配偶亦感到恐惧、悲伤和失望,手术将影响他们的共同生活。正常和谐的性生活,是调节、维持机体正常内分泌功能的重要因素,而正常的内分泌功能对机体免疫功能的发挥是十分必要的。因此,妇科癌症患者在康复后恢复正常的性生活,不仅对保持家庭、夫妻关系是必要的.而且对患者自身的远期康复也十分重要。

陈振东(1997)认为,性欲低下和无性欲是肿瘤患者中常见的性功能障碍,常见原因有:对肿瘤的恐惧导致精神压力过重,难以产生性兴趣;夫妻双方或一方认为肿瘤具有传染性;担心性交会“伤元气”,对身体康复不利;害怕性交会引起出血、疼痛并影响治疗;切除子宫、卵巢等器官有一种被“阉割”的感觉,认为自己不再具有性功能等。而这些原因中有许多是由于患者对疾病及手术所造成的影响认识不清、心理负担过重造成的,需要妇科医生在术前、术后给予患者适当的性指导。

Stead(2003)的研究指出,从诊断出癌症到长期的随访过程中,性功能减退是导致女性生殖道和乳腺恶性肿瘤患者抑郁的重要原因。年轻患者在性生活及生育能力方面的问题会比老年更多。

另外,妇科癌症本身的病变结果如贫血、腹胀、食欲不振及机体尚未恢复等,均会引起明显的性功能障碍。一些治疗的副作用,妇脱发、恶心、呕吐等常常使患者对性生活感到厌恶。手术或放疗本身也可致阴道缩短、狭窄、疤痕或炎症等,造成性交困难、疼痛。切除双侧卵巢的患者还会产生绝经后症状,但并不引起性欲的明显改变。

美国妇科肿瘤杂志1994编辑的文章指出,回顾性研究过去50年和近10年来前瞻性研究妇科恶性肿瘤患者治疗后存活者的生活质量,都说明性功能障碍是这些患者主要关心的问题。

Leenhouts(2002)指出,改善早期妇科恶性肿瘤患者的管理,加强患者的性知识教育,有利于改善患者性满意度。

Molassiotis(2002)研究发现,在香港,几乎所有病情稳定的妇科恶性肿瘤患者都有性生活方面的问题,对性功能存在误解,同时,她们关注自己的性欲及女性特征的变化。因此需要医务人员给予这些患者长期的心理辅导,帮助她们适应肿瘤后的生活,特别是性心理和性生活的恢复。

刘朝晖(2000)调查妇科恶性肿瘤妇女术后性生活情况,结果发现60.17\%表现为性欲下降。卵巢癌、子宫内膜癌、宫颈癌三组妇女在性生活恢复时间、性生活频度以及性生活满意度方面并无显著性差异。只有18.15\%的患者咨询过医生性方面的问题,而87.17\%的患者希望医生在性问题上给予帮助,并表示希望开设此类门诊。认为手术范围不是决定术后性功能状态的唯一因素,心理上的压抑与恐惧常常是导致术后性功能障碍的主要原因。因此重视妇科恶性肿瘤患者的心理状态,将对改善这些妇女术后的性功能有重要的帮助。

Barhara(1994)指出,多数患者对性生活的愿望在各期中都是减少的,只是理性上认为应该有性愿望,实际上,如果雄激素是引起女性性要求的内分泌基础,那么放射或手术治疗切除卵巢后造成更年期提前到来,并不影响性生活的继续,所以关键在于患者对性生活的态度。只要患者有这种愿望,就同样可以达到性兴奋,甚至性高潮。但是对这个问题,多数患者亲属不了解其重要性和必要性,有的还有一些严重错误的理解,造成患者新的精神压抑和负担,甚至不良后果。

王浩(1990)的研究表明,相当多的妇科癌症患者不了解女性生殖器官的结构,缺乏性知识。有的患者认为切除子宫就失去了女性特征,性生活会造成癌症的复发,担心生殖道癌症会传染、会因性交而传染给丈夫;有的丈夫对妻子接受生殖器官切除手术心情沉闷,对以后的性生活感到忧虑或自责,这些思想负担都可使性欲降低,性感缺乏,甚至终止性生活。

Thranov等(1994)报告,妇科恶性肿瘤治愈患者中,35\%有性交困难,15\%完全丧失性生活,40\%明显减少性生活频率,25\%~50\%缺乏性激情,难于达到性兴奋并缺少性高潮。尽管如此,但Schultz等(1992)的研究表明,多数妇科癌症患者有要求恢复生理性功能的愿望,并认为这才是真正的健康恢复。Jankins等(1988)发现,75\%的妇科癌症患者在术前、木中及术后没有得到有关性功能变化的信息,而80\%的患者需要获得这方面的信息,并且不希望由自己提出,而希望由医生提出。最近的一项研究调查了1700多名妇科癌症患者,仅有23\%的患者得到过有关手术可能使她们性功能产生某些后果的信息,而其中只有55\%认为这些信息是足够的。因此,医生应在治疗的各个不同阶段给患者以充分的解释和指导,帮助患者进行性康复。

(1)对妇科癌症患者在手术前、手术后进行宣传教育,对促进其性功能的恢复十分重要。首先了解患者患病前的性生活史,向患者讲解女性生殖器官的解剖结构,女性内分泌功能以及正确的性生活知识,预测手术后可能发生的性生活问题,并给予相应的指导。

1980年Sewell等对46例妇科癌症患者进行广泛性外阴切除术、广泛性子宫切除术和盆腔脏器切除术后患者的社交情况、个性、自尊、一般健康情况、体像、性关系及性交次数的变化进行了研究。研究方法以评分法为标准,各评分法为:一般健康状况评分(general well-being scale),自尊评分(Bach-man self esteem scale),社交调节评分(weissman social adjustment scale),内、外控制评分(Rotter internal vs external con-trol scale),体像评分(body image scale)。资料表明手术前及手术后体像评分,性关系(sexual relationship)和性交频率有明显差异(P<0.01)。社交调节能力评分明显下降(P<0.01),且与患者的个性关系显著。但是,体像、性关系和性交次数的变化对个性没有影响。资料还说一般健康状况评分与术后体像评分密切相关,自尊与体像关系不明显。在人际关系方面,他们研究的46例患者中,38例有配偶,其中29\%(11/38例)术后与配偶关系疏远,13.2\%(5/38例)断绝关系。另一重要发现是,年龄平均在42.2岁与53.5岁的两组患者中,年轻组比年长组配偶关系变化明显。且发现配偶相处关系长短与其的关系,有稳定关系9.9年者比24.9年者关系疏远发生率高(P<0.01),这种现象在广泛性外阴切除术、广泛性子宫切除术及盆腔脏器切除术三组中结果一致。说明配偶关系疏远与手术类型无关。因此,妇科癌症手术后的患者,必须了解她的社会心理、性心理、人际关系方面的变化,掌握她们存在的具体问题,才能有目的地进行宣教及咨询,使患者在术后尽可能建立较高的生活质量。

1980年Capone提出给妇科癌症病房的患者设立“适应转折的咨询干预计划(crisis-oriented counseling inter-vention program)”,此课程从明确诊断恶性肿瘤时即开始进行,发现在没有进行咨询干预的患者中,治疗后3~6个月时,性交次数减少或停止者占3/4,治疗后1年占1/2以上。而接受咨询干预者,在3个月时性交次数减少或停止者仅占1/2,6个月时为1/3,12个月时为1/5。且妇科癌症治疗后1年性交次数恢复至治疗前者,受教育者比未受教育者多两倍。其咨询干预计划的主要内容为首先认真估计患者的健康状况、心理状态及性功能,存在的性困难和要求。了解患者对疾病的知识和性知识水平,以及对口交、肛交及手淫等性活动的态度,因为这些情况十分重要,它可能成为妇科癌症手术后暂时的或永久的性交代替方式。其次,要纠正患者的错误认识,如认为手术后性功能不可能恢复、性交可以引起出血或癌症复发等。对患者治疗后的性要求必须正确估计。1980年David等报道,妇科癌症手术后患者中有性伴侣者70\%有性交愿望,57\%希望性交次数多一些。因此,医生不能认为患者对性生活是漠不关心的。此外,在实施咨询干预计划时要注意创造医生和患者交流的宽松环境,有利于相互对话及提出她们的要求,促进患者产生对性生活的积极愿望。要通过热情的语言、关怀的表情及手的触摸与患者建立感情,减少患者的孤立感,使她们从癌症的恐怖中解脱出来。

(2)医生应该对患者夫妇提出的性问题作出明确的回答和具体的指导,不能含糊不清,避免误解和产生不良后果。比如:如何时性交为好,性交前必要的准备,治疗后初期性交的方式、方法等。根据不同患者及其丈夫的文化、受教育程度、健康水平、对性生活的态度,在所有治疗过程中对涉及的性问题及时给予个性化的答复。告诉那些器官切除的患者:这并不意味着降低了性敏感或丧失了女性特征,应该将患者的注意力从被切除的子宫、卵巢和外阴方面转移到自身仍然存在的其他女性特征上,这种特征包括各种精神状态及肉体特征,特别是女性性征,如面孔、乳房及体形的吸引力等,以消除患者的自卑感和不完整感。刘朝晖(2000)调查发现,子宫在妇女心中占有重要地位,子宫切除术前许多女性对手术呈焦虑心态,最担心的是今后的性生活,医生对此普遍解释不够。子宫切除术对性生活本身并无明显影响,但许多妇女会在术后出现各种各样的性问题,其中近一半为心理因素,几乎所有的妇女都渴望与医生进行性方面的交流。揭示目前我国医生在此方面的解释工作远远未能满足患者需要。

(3)重视患者丈夫的思想工作,医务人员向患者提供教育时,在手术前、手术后亦要对配偶进行性咨询,使其正确认识妇科癌症及其治疗的影响,配偶双方同时参加咨询干预计划,对患者性功能的恢复有促进作用。一方面,鼓励患者和丈夫进行坦率、明确地讨论,交流性活动的有关问题。Wiel van de等(1991)报道,70\%的盆腔廓清术患者术后停止性活动,但那些对性活动态度正确且愿意进行性调节的妇女,如果术前决定手术重整阴部、重建阴道,就有可能达到术后和谐的性生活。另一方面,同时对其丈夫教育、说明,使丈夫接受并对这种做法感兴趣,是促使患者完全恢复性功能的最重要因素。所以,夫妇双方,特别是丈夫对术后性器官变化的认识和态度,是决定术后性生活是否美满的关键因素。还要强调丈夫在双方心理适应和性适应过程中的重要地位。如有的患者认为自己失去对丈夫的吸引力,表现为没有自信心也是性生活失败的原因。单纯强调在性交中的性高潮或双方同时达到性高潮,都是不合适的,即使健康人也不一定都能达到。性活动包括拥抱、接吻、抚摸、互相亲昵的动作,这对于不能进行满意性生活的夫妇来说,也是十分重要的。

(4)对某些广泛性手术的患者,医生应考虑到术后性功能的恢复而采取同期或二期手术,以重建外阴或阴道,并于术后告诉患者早期使用阴道模具。Anderaen(1990)报道,对放疗后阴道上皮变薄、萎缩、狭窄、阴道变短的患者应鼓励尽早性交,性交困难者可局部应用雌激素类药物霜剂、乳剂,以利正常性生活,GOG protocol 137(1994)决定采用雌激素替代治疗用于治愈后的子宫内膜癌和卵巢癌。盆腔廓清术后腹部留有造口和结肠造口,也可在医生的指导下避开造口发挥功能的时间,从而进行正常的性交活动。Cerny JA(1978)认为对于缺乏性高潮的患者,可用手淫的方法来训练她们重新获得性高潮而建立信心,目前还有很多妇产科的技术和器材可以帮助患者更好地解决这一问题。因此Schover(1989)指出,所有治疗的副作用都是可以预防和治疗的,从而可使性生活正常进行。要在身体条件允许的情况下尽早恢复性生活,性生活恢复愈晚,性功能障碍发生愈多,甚至完全停止性交。

(周红)


\section{第三节 外科术后医源性性功能障碍}

部分医源性性功能障碍是由药物、医生言谈举止所引起,其形成的另一更重要原因是手术,它是盆腔手术的一种重要并发症。随着盆腔手术范围的不断扩大,该并发症发生率有增高的趋势,同时,随着原发病手术疗效的提高及人们对术后生活质量的重视,如何减少或预防术后性功能障碍已日益引起医学界的普遍关注。

由于盆腔自主神经、血管与直肠、膀胱、精囊、前列腺及尿道的密切关系,各种盆腔手术均有可能导致医源性性功能障碍。其主要原因有:

为术后性功能障碍最主要原因。神经完全离断便造成永久性性功能障碍,若神经受到过度牵拉甚至撕脱,常导致其麻痹而发生暂时性性功能障碍。一般而言,盆腔自主神经系统只有在双侧神经均受损时,才出现性功能障碍,一侧神经损伤时仍可维持其正常功能。交感神经如上腹下或下腹下神经、盆神经丛(交感神经部分)损伤则导致射精障碍。副交感神经如盆神经、盆神经丛、海绵体神经损伤可造成阳痿。神经损伤常见于直肠切除术,特别是直肠癌根治术,其术后性功能障碍发生率高达33\%~100\%,及泌尿系手术,如膀胱、前列腺切除术、尿道修补术等。前列腺癌根治性切除术阳痿发生率达43\%~100\%,前列腺膜部尿道修补术后则为51\%。此外,下段腹主动脉瘤手术也易损伤该动脉旁的胸腰交感神经,腰交感神经节切除术将腰1交感神经节切除,均可导致射精障碍。

阴茎勃起依赖于动脉灌注血量的增加。阴茎的血供主要来自阴部内动脉,该动脉是髂内动脉的终末支。这些动脉的损伤同样可造成阳痿。这种血管源性阳痿较神经损伤者少见,主要见于低位直肠癌经腹会阴联合切除术中,为了减少出血而结扎双侧髂内动脉或在会阴部手术中偶尔损伤位于坐骨直肠窝侧壁的阴部内动脉。在膀胱、前列腺切除术中,上述情况罕见,虽然背静脉常被结扎,但不影响性功能。有人报道,前列腺癌根治术后出现性功能障碍的患者阴茎血流量正常,表明动脉供血不足不是一个重要因素。

由于患者对手术不理解,或手术带来不便如乙状结肠造瘘或回肠代膀胱等造成精神心理障碍而影响性功能。

手术造成的血肿、炎症、感染或纤维化等均可能影响神经,造成性功能障碍。

一般认为老年患者术后性功能障碍的可能性较年轻患者大,特别是60岁以上患者。如男性直肠癌经腹会阴联合切除术后,青年组(41~48岁)、中年组(49~57岁),老年组(58~65岁)性功能障碍发生率分别为14\%,55\%与70\%。前列腺癌根治切除术后性功能正常者在60岁以下与60岁以上分别为31\%与6\%。

大量证据表明,良性病变切除术后性功能障碍发生率较恶性病变低得多。因单纯炎性疾患而行直肠切除的术后阳痿、射精障碍发生率分别为4.7\%与6.2\%,远低于直肠癌根治切除术后的45\%与32\%。前列腺增生切除术后阳痿发生率比前列腺癌切除术后也低得多。其原因是良性病变手术切除范围较恶性病变者小,神经损伤的机会也相应少些。对于恶性病变,肿瘤侵犯深度、范围及淋巴结转移等对于术后性功能障碍的发生有一定影响。对于前列癌,前列腺被膜侵犯与否对术后性功能障碍的发生有重要影响,癌症限于前列腺被膜内与穿透被膜的前列腺癌经传统根治术后性功能恢复正常者分别为33\%与5\%。采用经改进的保留神经根治术者,性功能正常的比例明显提高。

对直肠癌,距肛门同一水平上的肿瘤,无论位于直肠前壁、后壁还是侧壁,均不影响术后性功能障碍的发生率。但是直肠肿瘤部位高低却与之有密切关系,位置越低,其发生率越高。一般而言,距肛门10cm以上的直肠癌根治术后极少发生性功能障碍,而位于10cm以下,特别是5cm以下者其发生率很高。距肛门>10cm、5~10cm、<5cm三组直肠癌经骶前切除术后性功能障碍发生率分别为0、25\%与33\%。距肛门10cm以下的直肠癌采用骶前切除术或经腹会阴联合切除术,该并发症发生率多无明显差异。对于前列腺增生行前列腺切除,手术途径不同,术后性功能障碍发生率则各异。经尿道前列腺切除术后很少发生阳痿,性功能障碍发生率仅5\%。耻骨上经膀胱前列腺切除术后阳痿发生率10\%~20\%,经会阴前列腺切除术后阳痿发生率可高达40\%~50\%。

为了降低直肠癌等盆腔肿瘤的局部复发率,提高5年生存率,常常在传统根治术基础上附加盆侧壁淋巴结清扫,而术后性功能障碍发生率则明显增高,传统根治术与扩大根治术后性功能障碍发生率分别为37\%与76\%。显然盆侧壁淋巴结清扫大大增加了神经损伤的可能。

术后性功能障碍主要表现为部分性或完全性阳痿、射精不能或逆行射精。术后性功能障碍可分为暂时性与永久性两种,前者与手术造成的精神心理障碍有关,或因神经牵拉、压迫等因素造成暂时性神经麻痹引起,多在术后1年逐渐恢复正常,少数患者症状可持续2年。超过2年仍不能恢复正常者应考虑为永久性障碍,这常是手术造成的器质性病变所致。不同的术式对性功能的影响不尽相同,如直肠癌根治术中,经腹会阴联合切除等术后阳痿、射精障碍较常见,平均发生率分别为45\%、32\%;而经骶前切除术后阳痿者较少,仅15\%,主要发生射精障碍,平均达42\%。泌尿生殖系统手术如前列腺癌根治性切除术后阳痿发生率达43\%~100\%,经会阴前列腺单纯切除和耻骨上经膀胱前列腺单纯切除术后阳痿发生率分别达40\%~50\%和10\%~20\%;但因切开膀胱前壁及其颈部周围黏膜,75\%~80\%的患者发生逆行射精。

可通过询问患者有无夜间勃起或晨勃初步判断是精神性还是器质性因素所致阳痿。若有性高潮及遗精,但性交时不能射精者,说明是术后精神因素造成。通过询问病史仍难作出结论时,需进行夜间阴茎勃起测定,减弱或消失者常为器质性阳痿。

对于术后性功能障碍者,以口服西地那非(万艾可)等磷酸二酯酶5型抑制剂为治疗原则,经正确口服仍无效的阳痿患者,可考虑阴茎假体植入术。术后器质性不射精的治疗较为困难。

了解并熟悉盆腔自主神经的走行及部位,与盆腔内脏的关系,术中予以辨认及保护,或远离神经操作避免损伤,是预防术后性功能障碍的关键。对于恶性肿瘤,应考虑到手术的主要目的是根治肿瘤,其次是保留性功能。保留海绵神经的根治性膀胱切除术后性功能正常者达83\%,根治性前列腺切除术后为74\%,性功能状况较传统手术有了大大改善。需要注意的是在盆腔会阴手术中,应尽可能避免结扎双侧髂内动脉或双侧阴部内动脉,以免造成血管源性性功能障碍。

(万远廉)


\section{第四节 阴茎疾病与性}

具有良好弹性的阴茎筋膜组织是维系正常勃起功能的重要条件。痛性勃起是由于其伸缩性减弱或丧失,导致勃起后阴茎弯曲,并引起疼痛。阴茎的弯曲可朝向背侧、腹侧或者侧方等任何方向。它可能是某些先天或后天的因素所致。若是先天性的,阴茎弯曲也许还伴随扭转,当然,阴茎扭转也可以单独出现。如果阴茎各组织结构的形成过程准确而同步,那么,阴茎诸层的弹性是对称的,否则就是阴茎发育不良,更谈不上其诸层组织的伸展性是否对称。

指阴茎发育不良引起的勃起阴茎的弯曲和疼痛,通常伴有尿道下裂或者尿道上裂。一般认为,尿道下裂是男性激素分泌不足、未能满足阴茎发育的需求,结果使阴茎发育没有充分完成而致。尿道下裂患者的尿道开口于阴茎体的下部,间质也未形成海绵体等包绕尿道的结构,而只能在尿道下裂口的周围构成扇形的非伸缩性的囊状组织。尿道上裂的情况与此类似。临床上可以观察到各种不同病理解剖结构所导致的外形各异的尿道下裂和阴茎弯曲。(参见第二十五章内容)

指因外伤或一种特殊疾病———阴茎硬结病所引起。

指因意外造成的阴茎外伤瘢痕,或者因医源性原因,如经受过治疗遗尿症的尿道手术或尿道下裂修复手术后的瘢痕所引起。由于修补处的瘢痕没有弹性,或者皮瓣移植修补尿道缺乏弹性,导致尿道狭窄,而修整尿道狭窄时无论是直视手术还是经尿道手术均会累及海绵体,留下瘢痕也会导致阴茎的痛性勃起。纠正外伤性痛性勃起必须采用手术治疗。由于手术方法与阴茎硬结病相同,因此在下文中一并介绍。

法国医生Peyronie于1743年首先描述了本病,故称之为Peyronie病。本病的名称较多,又称阴茎纤维性海绵体炎,慢性海绵体炎,海绵体硬化病或纤维化等。已经将本病定义为在白膜和海绵体之间因血管炎症导致白膜内层形成瘢痕。阴茎硬结病以中年人多见,40~50岁为发病高峰,但也有19岁未婚男青年及80岁老人患病的报道。本病的主要特点为阴茎上有结节状或条索状硬结,阴茎勃起时弯曲并且疼痛,可影响性交。

(1)病因:尚不十分明确。阴茎的损伤,如手术、骑跨伤、频繁性交、手淫所致的小损伤;全身胶原性疾病,动脉粥样硬化,糖尿病,高血压,痛风,维生素E缺乏,酗酒等,都与病症的发生有关。另外,有人发现大约10\%左右的患者与Dupuytren挛缩(掌挛缩症)有关,因此,本症又可能有一定的遗传因素。

(2)病理:既往对病变的位置是在阴茎筋膜与白膜之间还是在白膜与海绵体之间存在分歧。近年来的病理学检查证明,硬化开始于白膜与阴茎海绵体之间的疏松结缔组织,初期表现为血管炎,镜下可见淋巴细胞和浆细胞浸润,结缔组织增生,胶原纤维及纤维母细胞增多,继而在白膜内层及阴茎海绵体间隔形成以胶原纤维为主的斑块。大约80\%的病例为单一斑块,其余病例可有两个或两个以上的斑块。病变呈自限性,开始几个月发展较快,而后减慢,一般不侵犯尿道海绵体,因此很少发生排尿困难。大多数患者阴茎病灶中可出现局限性钙化或骨化,但从不溃破或癌变。有些病例活检或手术切除后可见到骨、软骨和骨髓组织。

(3)诊断:患者多因阴茎硬结、勃起弯曲、疼痛,影响生活而就诊。在大多数患者阴茎背侧冠状沟的后方,可触到单个或多个斑块状或条索状硬结,硬度犹如软骨,界限清晰,可有轻微疼痛。阴茎勃起后多向背侧弯曲并伴有疼痛。如果硬结位于一侧海绵体上,阴茎勃起则弯向患侧。因此,凡阴茎背侧有慢性硬结的中年人,平时无不适,阴茎勃起疼痛并向背侧或侧方弯曲,X线平片偶见钙化影,可诊断为阴茎硬结病。海绵体造影法(cavernosography)可以清楚显示硬结的范围和大小,克服了既往仅凭触诊难以了解病变全貌和不易触清阴茎海绵体间隔附近硬结的缺点。具体方法为:患者仰卧于XTV检查台上,常规消毒后行局麻或全麻。以较粗头皮静脉针刺入一侧海绵体内,先注入生理盐水10ml,若局部不肿胀证实针头已在海绵体内,将60\%urogra-fin加压注入,并在电视监视下摄片,可清晰显示与病损相应的充盈缺损,并明确病变的大小,为手术切除范围提供准确依据。

(4)治疗:阴茎硬结病的治疗方法甚多,但由于病因尚未完全明确,因此非手术疗法的效果欠佳。但给患者做必要的解释,尤其是告诉他们疾病的良性性质,消除其恐惧心理是非常必要的。

1)非手术治疗药物治疗:维生素E 200mg,3次/天,连服3个月以上,被认为是疗效较为肯定的内服药物疗法。服药后,大多数患者可有不同程度的改善。也可用作其他治疗的辅助疗法。丙卡巴肼(procarhazine)为抗肿瘤药,服50mg,3次/天;对氨基苯甲酸钾(potassium paraamino-benzoate)为溶蛋白剂,服4g,3次/天;泼尼松为肾上腺皮质激素,服5mg,3次/天。但由于服药时间长,副作用大,而且疗效并不肯定,因此应用价值不大。中草药可选用柴胡、赤白芍、白芷、夏枯草、皂角刺、桃仁泥、制乳没、莪术、生薏仁(郭应禄方),据报道,有一定疗效。局部封闭治疗:醋酸氢化可的松25mg溶于1\%普鲁卡因1~2ml,硬结内注射,每周1次,连续使用6~8可见一定效果。但使用不当易损伤阴茎背动脉,可并发皮肤溃疡或尿道口周围局限性坏死。亦有报道导致进一步纤维化和组织萎缩者,因此应持慎重态度。药物离子导入疗法:将类固醇或组织胺等药物通过直流电经皮肤透入组织内和病变区,较注射用药相对安全合理。放射治疗:有X线、镭及钴-60等治疗,历史亦比较久远。据报道,放射治疗仅对疼痛有所改善,但鲜有硬结消失者。因此,对疼痛严重者方考虑使用。另外,音频、超声波等理疗方法也有应用,可作为综合治疗的一部分。

2)手术治疗阴茎硬结切除皮片移植术:由于非手术治疗效果不理想,近年来许多学者主张手术切除硬结,游离皮片移植修补缺损处,疗效良好。麻醉:硬脊膜外麻醉、腰麻或全身麻醉。体位:仰卧位。手术步骤:首先安好留置导尿管。在距冠状沟0.5cm处作环形切口,沿皮下组织分离,使阴茎皮肤似退下的袖套,卷在阴茎根部。于阴茎根部扎止血带并作人工勃起。分离阴茎背侧血管神经束,将其牵开予以保护,并显露阴茎硬结。然后将纤维硬结分离切除,应避免切除不完全,并防止损伤其深面的阴茎深动脉。按缺损的形态和大小在下腹部取全层皮片,切缘应整齐,并稍大于组织缺损。将皮下脂肪剪除,利刀割除表皮,再移植到缺损处。作人工勃起以观察植皮处有无漏水及阴茎伸直情况,必要时予以修正。松开止血带,将根部皮肤退回原处并缝合皮肤切口。为彻底纠正弯曲和防止术后阳痿,有人主张同时植入阴茎假体。术后处理:阴茎外用弹力绷带轻轻加压包扎,用抗生素预防感染,使用雌激素防止阴茎勃起,术后留置尿管于1周后拔除。若术后阴茎头感觉减退,多与术中牵拉阴茎背神经有关,一般数月可恢复。

阴茎异常勃起是指在无性欲的情况下,阴茎长时间持续性勃起。阴茎异常勃起依据病因和血流动力学的不同,可以分为静脉闭塞性阴茎异常勃起和高血流性阴茎异常勃起。

闭塞性阴茎异常勃起是指在无性欲的情况下,阴茎长时间持续性的痛性勃起。本症在临床上比较少见,据统计约占泌尿外科住院患者的0.4\%。静脉闭塞性阴茎异常勃起可见于任何年龄,但以已婚成年男子常见,30~40岁为发病高峰。静脉闭塞性阴茎异常勃起多起病突然,阴茎海绵体呈高度勃起状态,而尿道海绵体和阴茎头仍较松软。本症系泌尿外科的急症,其病因学、病理学表现特殊,治疗亦比较困难,容易导致永久性阳痿。

1.病理生理 正常阴茎的勃起受脊髓副交感神经中枢(S2 ~S4 )、胸腰中枢(T12 ~L1 )的支配和大脑皮质的调控。当脊髓或者大脑皮质的反射作用引起阴茎动脉扩张时,流入阴茎海绵体内的血流量便增加,这时,阴茎海绵体内小梁的平滑肌松弛,螺旋动脉开放,使大量血液由阴茎深动脉流经螺旋动脉注入小梁间腔隙,阴茎海绵体即充血胀大。与此同时,阴茎静脉回流的血液量也会有所增加,但较注入量少。这样,由于受到阴茎白膜的限制,阴茎的体积就增大、变硬。当勃起达到一定程度时,动脉流入量与静脉流出量就达到了一种足以维持勃起的动态平衡状态。但是,静脉闭塞性阴茎异常勃起的情况就不同了。有的学者通过阴茎海绵体造影发现,阴茎动脉血注入增加,但阴茎背深静脉却是阻塞的,这就使得海绵体内血液回流受阻,引起其血压增高,挤压白膜而产生疼痛。此时,外界任何轻微的刺激都会使疼痛加剧。而海绵体间隙中血液的阻滞,会使血氧饱和度和pH下降、CO2 增加和乳酸堆积,导致酸中毒和血液黏滞度增加,直至血栓形成,机化和纤维化,最终阻碍海绵体的重新扩张而产生阳痿。尿道海绵体造影显示,阴茎异常勃起时,位于阴茎筋膜之上的阴茎背浅静脉是开放的,浅静脉系统未受损害。

2.病因 既往认为本病多为特发性,并未发现潜在的原发疾病。部分仅与延长性交时间和反复手淫有关。近年发现继发性阴茎异常勃起的发病率有所增加,占所有病例的50\%左右,甚至更多。常见的病因有:

(1)血液系统疾病:①镰状细胞贫血:黑种人,尤其黑人儿童多见。由于阴茎勃起时静脉淤血,血氧饱和度下降,致使更多的红细胞变成镰刀状,进一步引起血液滞留,最终形成血栓。许多患者开始时曾有过持续几分钟至数小时的勃起,然后过渡为不缓解的、疼痛的勃起。有的在就诊时阴茎即已失去了柔软的质地,代之为坚硬的纤维组织结构。②白血病:与镰状细胞贫血相似,大量白细胞沉积在阴茎海绵体腔隙中,引起血液滞留可形成细胞性血栓。Moure通过解剖证明,白血病所致异常勃起的阴茎海绵体血栓,是由纤维素与白细胞所形成的。③其他血液因素:包括血红蛋白C病、地中海贫血、红细胞增多症、血小板增多症、血小板减少症以及磷酸葡萄糖异构酶缺乏症等。

(2)恶性肿瘤:恶性肿瘤转移到阴茎海绵体可以导致阴茎异常勃起,原因为转移病灶使海绵体中血液流出受阻。转移灶多在膀胱、前列腺、肾、直肠或肺。压迫盆腔内晚期肿瘤也可引起阴茎痛性勃起。

(3)神经性疾病:以脊髓损伤、炎症、脑干病变、多发性硬化等较多见,确切机制尚不清楚,推测可能与病变抑制交感神经、降低血管紧张性、增加海绵体的血流量有关。其可能是由于脊髓的勃起中枢过度兴奋引起的。据统计,高位或完全性脊髓损伤患者的发病率比低位或不完全损伤者为高。另外,脊髓动脉瘤破裂,腰疝的形成并突出,自缢的损伤,甚至癫痫都可能引起阴茎异常勃起。这些少见的致病因素,很可能与对中枢神经系统的压迫或者持续刺激有关。

(4)外伤:骑跨伤、阴茎外伤导致栓塞或阴茎根部受压造成血液回流障碍,均可引起阴茎异常勃起。

(5)感染:前列腺炎、尿道炎、包茎均可为局部因素。而腮腺炎、梅毒等疾病有时也与异常勃起有关。其致病原因可能是局部淋巴细胞增多,可能是脱水及血浆黏滞度过高。前列腺炎等后尿道炎症的致病因素还可能与炎症导致前列腺静脉丛栓塞,影响深静脉回流有关。

(6)药物:许多药物可以诱发本病。以降压药如胍乙啶、利舍平(利血平)、肼屈嗪(肼苯哒嗪)、复方降压片等较为多见。其致病原因可能与血压突然下降,导致血流缓慢、在阴茎海绵体内淤滞有关。另外,应用氨基乙酸等止血药,肝素等抗凝药,地西泮等抗焦虑药,以及静脉营养的脂肪乳剂滴注等,偶尔也可引起阴茎异常勃起。需要指出的是,近年来国内使用阴茎海绵体内注射罂粟碱等药物治疗阳痿而引起阴茎异常勃起的病例时有发生,应引起临床医师的足够重视。由于神经源性阳痿患者对血管舒张药物非常敏感,因此对这类患者施以化学假体治疗时应格外小心,宜从小剂量开始试用。

3.临床表现和诊断 阴茎海绵体极度肿胀、坚硬伴有疼痛,触诊坚硬、疼痛明显,大多伴有排尿困难。若持续时间超过24小时,则可能伴有表皮水肿、发亮。绝大多数病例不累及阴茎头和尿道海绵体。静脉闭塞性阴茎异常勃起症状较重,阴茎坚硬、无弹性、疼痛剧烈,预后较差。本病诊断简单,视诊即可确诊。

4.治疗阴茎异常勃起是泌尿外科的急症,治疗要兼顾以下几个方面:①减轻疼痛;②明确病因及决定相应的治疗方法;③迅速恢复阴茎正常的血液循环,使阴茎软化,恢复其正常勃起功能,预防阴茎纤维化,防止阳痿发生。

因此,在采集病史和体检后,通常应进行下列各项化验检查:白细胞计数,血小板计数,血液黏滞度,肌酐,镰状细胞及血红蛋白因子,尿常规,胸透等。当然,明确病因并进行去因治疗是必不可少的。不过,临床上更为紧迫的是尽早对症处理,尤其对病因不明或者不能迅速去除病因的病例更是如此。近年多数人主张采用手术疗法。

(1)非手术治疗:①药物:镇静止痛剂,雌激素,抗凝剂,纤维蛋白溶解剂,降压药,肌松剂及抗生素均可使用。②物理疗法:冷敷,冷、热盐水灌肠。X线照射对恶性肿瘤浸润引起的阴茎异常勃起可能有效。近来有报告用深部X线照射胸腰勃起中枢取得较好疗效。③麻醉:如用氯胺酮全身麻醉,T8以下的腰麻或硬膜外麻醉可能对神经因素引起的阴茎异常勃起有效。④穿刺抽吸冲洗或注药:用粗针头穿刺阴茎海绵体,尽量抽吸其中积存的黏稠血液,并用含肝素或肾上腺素的生理盐水反复冲洗,直到有新鲜血液流出,阴茎松软为止。松软阴茎可加压包扎,但应注意观察,避免阴茎缺血坏死。必要时间隔12小时左右再次抽吸冲洗,可重复2~3次。据报道,经此法治疗痊愈的患者,约50\%可恢复正常的勃起功能。对于因行化学假体治疗导致阴茎异常勃起的患者,还可试用抽吸淤血20ml后,向海绵体内注射阿拉明2mg,间隔1小时可重复注射1次,据说有较好的疗效。

(2)手术治疗:由于阴茎异常勃起的发病原因和病理生理尚不十分清楚,因此,在手术治疗的选择上也存在一些分歧。不过,鉴于非手术治疗的效果多不肯定,而手术治疗效果颇佳,因此,多数学者主张经短期(24小时以内)非手术治疗无效的患者尽早行手术治疗,以避免阴茎纤维化和预防永久性阳痿的发生。手术主要基于解除阴茎海绵体的回流受阻和将阴茎海绵体的血液分流至另一静脉系,穿刺抽吸冲洗或给药而设计。常用的手术方法有:①大隐静脉阴茎海绵体分流术:在大腿根部切口,从卵圆窝向下分离大隐静脉,近足端结扎,另一端通过皮下隧道牵至阴茎根部,切除一小块海绵体组织和阴茎白膜,挤出淤血并用肝素盐水灌洗,至阴茎完全萎缩并流出鲜血,然后将大隐静脉断端剪成斜面与海绵体切口吻合,关闭切口。虽然这种手术的疗效肯定,但因静脉回流过多,术后有发生阳痿的可能。不过,若出现阳痿可通过结扎吻合的大隐静脉而恢复阴茎勃起。②阴茎海绵体尿道海绵体分流术:本手术操作简易,分流效果好,已为多数医生所接受和采用,故对其作较详细介绍。麻醉:硬膜外阻滞、腰麻或全身麻醉。体位:膀胱截石位。手术步骤:在阴囊阴茎交界正中处作纵切口,长约4cm,切开并分离皮下组织至阴茎海绵体白膜,将白膜切下长1.0~1.5cm的椭圆形片,从切口挤出淤积血液并用肝素盐水反复灌洗,直至新鲜血液流出及阴茎松软,然后插导尿管,在切口对应的尿道海绵体上作同形等大切口,可见鲜血流出。用5-0尼龙线间断或连续缝合两海绵体切口的前壁和后壁。缝合时应谨慎,切勿伤及尿道壁以防形成海绵体尿道瘘。仔细止血后逐层缝合切口,皮下置橡皮片引流条。术后处理:①卧床1周;②抗生素预防切口感染;③口服己烯雌酚防阴茎勃起;④术后注意有无血液或脓液从尿道流出,若出现应进行尿道造影,证实出现海绵体尿道瘘需行耻骨上膀胱造瘘;⑤切口愈合良好,1周后拆除缝线。

(3)阴茎背静脉阴茎海绵体分流术:阴茎异常勃起后阴茎背静脉回流正常,因此可采用此手术解除阴茎海绵体的淤积。在阴茎根背侧作长约4cm的切口,切开皮肤、皮下及阴茎筋膜。暴露阴茎背浅或深静脉,远侧结扎,近侧剪成斜面,牵开阴茎背动脉和背神经,切开并用肝素盐水充分灌洗一侧阴茎海绵体,用6-0尼龙线连接缝合阴茎背静脉与海绵体的吻合口,仔细止血并逐层缝合切口。

(4)阴茎头阴茎海绵体分流术:局麻或鞍麻后,用尖刀或18号针头刺入阴茎头,注意避开尿道,并刺开阴茎海绵体白膜,总深度约5cm。用尖刀时可抽出,旋转90°后再刺入,使白膜切口呈“十”字,以利引流,挤压阴茎尽量排除淤血。为彻底清除淤血,还可在阴茎海绵体根部外侧另刺一粗针头,用肝素盐水冲洗。用细丝线缝合阴茎头刺口,术毕。

与静脉闭塞、阴茎海绵体内血流受阻引起的无性欲要求的海绵体持续性痛性勃起不同。临床上还有一种更少见的高血流量性阴茎异常勃起,主要因外伤引起海绵体动脉撕裂,从而发生高血流量性、非痛性的阴茎异常勃起。

1.病理生理高血流量阴茎勃起可在受伤后即时发生或延迟发生。

(1)即时发生:由于外伤将海绵体动脉撕裂,其血管内的血液迅速涌入海绵体窦状间隙内而产生阴茎勃起。

(2)延迟发生:外伤虽然将海绵体动脉撕裂,但体内正常的止血机制迅速发挥作用,动脉收缩、血小板聚集暂时阻止了动脉血液迅速和大量流入海绵体窦状间隙内,终止了异常勃起的发生。但是在随后的日常生活中,如性欲刺激、阴茎夜间勃起或者口服某些药物,可使海绵体动脉扩张致动脉壁拉长,从而使尚未牢固愈合的动脉破损处再次裂开,动脉血液不经螺旋动脉直接从受损处流入阴茎海绵体。与此同时,瘘口处持续性高内流状态,可导致血管内皮释放内皮衍化物质,如前列腺素和血管舒张因子,这两种物质不仅扩张动脉,同时也抑制血小板的凝集,从而改变了正常的止血机制,加之受伤初始的小动脉痉挛止血等机制的消退,因此海绵体动脉内血液从受损处直接流入海绵体窦状间隙内,并且永久性储存。另一种因素是海绵体动脉壁呈不完全损伤,在受伤后的几天内,部分受损的血管壁逐渐坏死、脱落,导致血管壁缺损,致使动脉瘘形成。而高血流的冲击使该区域产生应力,可刺激局部的内皮释发出一氧化氮。近年的研究已经证明,一氧化氮对阴茎平滑肌松弛有重要作用,它能激活腺甘酸环化酶,而高流量动脉血进入窦状间隙有高氧张力的作用,也可以刺激内皮一氧化氮的合成和释放。因此,阴茎窦状间隙内皮所受的应力和高氧张力的联合作用,共同刺激内皮一氧化氮的合成和释放,引起动脉和整个阴茎小梁平滑肌的松弛,导致阴茎持续性勃起。应该指出的是,高血流性阴茎异常勃起对阴茎静脉的回流并无影响,既无阴茎静脉的闭塞,也无阴茎被膜下静脉的明显受压,因此保证了阴茎内静脉血液无限制地畅通外流,从而防止了窦状间隙内的血液淤滞,更不会阻塞动脉血流,引起缺血和疼痛。一般来说,高血流量性阴茎持续勃起时,阴茎多呈现半垂直或垂直状态,无痛、局部温度也无明显降低。

2.病因高血流量阴茎异常勃起均发生在会阴部或阴茎外伤后。由于外伤导致海绵体动脉撕裂,使动脉内血液直接流入海绵体窦状间隙内,形成海绵体动脉———海绵体窦状隙瘘,从而引发高血流阴茎异常勃起。

3.临床表现 高血流量阴茎异常勃起与静脉闭塞性阴茎异常勃起的临床表现迥然不同。阴茎或会阴部外伤即刻或延迟发病是重要的临床特征。一般为无痛性,阴茎呈半垂或垂直状异常勃起,可持续数天、数月或者数年。若有性刺激可达到完全勃起,并能正常性交。检查时,可见阴茎呈半垂或垂直状,皮肤颜色无异常,皮温正常,无触痛。应用α受体阻滞剂海绵体内灌洗只能暂时疲软,停止用药后迅速恢复,阴茎海绵体和尿道海绵体分流术也不能取得满意疗效。

4.诊断高血流量阴茎异常勃起的诊断依据

(1)病史:会阴部或阴茎的外伤史。

(2)体检:阴茎半垂或垂直状异常勃起、皮肤颜色正常、皮温正常、无触痛。

(3)特殊检查:海绵体内抽吸为鲜红色血液、氧张力正常。①彩色多普勒超声:图像扫描可分辨阴茎海绵体、尿道海绵体、海绵体中隔、海绵体动脉、阴茎背静脉、螺旋静脉等;能够测量阴部内动脉、海绵体动脉的内径、相应血管中血流速度和血管收缩与舒张的变化,并且能观察血管壁是否完整及破损处。介于彩超无损伤和高清晰度的特点,对可疑高血流量阴茎异常勃起患者,应该作为首选检查,其可清晰显示血管的损伤部位和损伤程度,并能测出高血流或湍流;在动脉损伤的远端,可测到血流明显减少,同时可见动脉—窦状隙瘘周围的假包膜,可作为手术定位的标志。②选择性阴部内动脉造影:经一侧股动脉穿刺插入动脉导管,经腹垂动脉进入对侧髂动脉并伸入髂内动脉,注入造影剂并摄片,然后再进行同侧髂内动脉造影。X线片可显示阴部内动脉、阴茎背动脉、海绵体动脉及其分出的小动脉。若见到造影剂从小动脉外溢、从海绵体动脉流入海绵体窦状间隙中,或者从海绵体动脉直接流入静脉系统中形成海绵体动脉—静脉瘘,则诊断成立。

5.高血流量性阴茎异常勃起的治疗原则 暂时性闭塞撕裂动脉的血流,使撕裂的血管愈合,而后恢复生理控制的阴茎海绵体血流,保护勃起组织的活力和正常的勃起功能。

(1)机械压迫:在会阴部或阴茎部外伤处加压,在不妨碍加压的前提下可试用冰袋冷敷,尝试闭塞损伤动脉,使撕裂的血管愈合。由于血管损伤定位上的欠缺和压力大小掌握的困难,仅对少数损伤较轻的病例有效。

(2)药物:①α肾上腺素能药物,如间羟胺等。该类药物使血管平滑肌收缩,有助破裂血管止血。但事实上,用药后药物会随高血流状态的血液迅速离开病灶处,不但难以在局部发挥作用,反而可能导致全身性并发症———持续性高血压。因此,一定要慎重使用α肾上腺素能药物,并在注药前用止血带扎住阴茎根部。②亚甲蓝:亚甲蓝是腺苷酸环化酶的抑制剂,可阻止环化鸟嘌呤—磷酸盐的形成,从而使平滑肌收缩,减少动脉血流,使阴茎疲软。一般取10mg用于阴茎海绵体内注射即可使阴茎消肿,但几小时后阴茎会再次勃起,很难达到治愈的目的。

(3)手术:定位明确的高血流量阴茎勃起可采用手术治疗,阴部内动脉结扎术、海绵体动脉结扎术、海绵体动脉—窦状隙瘘切除术均可永久性闭塞受损动脉。该治疗虽疗效肯定,但可导致永久性阳痿。

(4)介入治疗:这是一种既能达到暂时阻断血流、促使损伤血管愈合,又可保持勃起组织活力和勃起生理功能的方法,值得推荐。建议在进行选择性阴部内动脉造影或海绵体动脉造影时同时进行。可选用自体血凝块、明胶海绵等作栓塞物质,经1~3次栓塞多能治愈。当动脉损伤较重、瘘口较大时,可选择氰丙烯酸异丁酯,是一种常用于精索静脉曲张栓塞治疗的组织黏合剂,注射后有快速诱导组织黏附的作用,效果可靠。

(卢存国)


\section{第五节 不孕不育对性功能或性生活质量的影响}

不孕不育症已被列入21世纪人类三大疾病(心脑血管、肿瘤、不孕症)之一,不孕症人群逐渐增多,年龄也趋向年轻化。在众多不孕不育的原因里,不可忽视的一点就是:心理因素也同样会导致不孕不育症的发生。正是由于心理因素的影响,从而导致了不孕不育夫妇性功能和性生活质量的降低,所以在治疗不孕不育的同时,应该对不孕症夫妇进行有效的心理引导、生育知识教育的普及和性医学方面的治疗,由此可以大大提高不孕不育症的治愈率。

从大量临床病历中可以看到,很多不孕不育症夫妇的性生活频率都很低,性生活质量也不会高,很多人完全是为怀孕而同房,为了赶在排卵日前后的受孕时机而机械性的完成任务。从统计数字上可以看到,每周同房1~2次的夫妻只占10\%左右,射精无力、时间过短现象也常有发生,有些人到了需要同房日,根本就无法勃起或射精,而导致性生活失败,错过最佳受孕时间,双方都很焦虑。

这种现象普遍存在于众多不孕症患者之中,由于他们对生育知识的缺乏,又自认为很懂,对下一代要求过高,无意中破坏了自然规律,而导致不孕不育。另外很多年龄偏大、婚龄较长的夫妻,受盼子心切、望子成龙的传统观念影响,越着急,越怀不上,到处求医问药,只要谁说能治都要去尝试,最终是劳民伤财、精疲力竭、毫无结果,从而失去信心,生活没有任何激情。导致这些问题出现的原因大致可归为以下几点:

很多城市白领族的夫妻,为了学业、事业,计划性都很强,一定要功成名就时才考虑要小孩儿,而一旦想怀孕时又自认为很懂,每月、每周、每天计算着排卵期,要么每天测基础体温、要么用排卵试纸天天测试,一定要用“指定”的时间来指导同房,如果这样持续半年还没能怀孕,两个人的精神都会高度紧张起来,开始怀疑自己是否有病,开始计划看医生,此时由于双方的心情都不好,已经开始影响到夫妻生活,性生活由自然规律变为无规律,质量明显降低。

正是由于有了上面的计划,等到真正想怀孕时,夫妻双方的年龄基本都到了三十岁以上,有的甚至四十岁以上,还有一些是再婚夫妻。到了这个年龄,男性和女性在生理上都会发生不同程度的变化,性器官、性功能都会有不同程度的减退,而有些疾病的发生,也会随着年龄的增长显现出来,在这个年龄段发病率更高些,如:肥胖、高血压、心脏病、糖尿病、痛经、子宫肌瘤、卵巢囊肿等。在就医的过程中发现,心理负担过大,加之某些疾病的影响,往往直接影响到性功能和性生活质量,从而影响到受孕。

试孕一年半载之后还没有怀孕,周围同事要问、家里亲戚要问、同学朋友要问,这些对于不孕症夫妇来说都成了“过分的关心”,让他们非常着急和不安,不愿与同学或亲戚好友见面,怕人提及此事,怕看到别人的孩子,工作没有了热情,生活没有了激情、情趣,有的夫妻几个月或半年以上才有一两次性生活,性生活成了为完成“任务”而不得已才做的事,与此同时也极大影响了夫妻感情,认为是在为他人而活,心理感觉很累很累!

有些人由于工作性质的需要,经常出差或在外应酬,饮食和作息时间都非常不规律,各忙各的,夫妻双方聚少离多,不能同步生活,交流越来越少,这样下去不但影响了受孕率,还大大影响了夫妻感情和夫妻生活,使得性生活也很不正常,双方渐渐对性生活冷淡,长此以往,双方的性功能都在减退,没有了性欲望、性需求越来越少,有些人甚至失去了性能力,而促使一方出现外遇。

现在很多人在年轻时或婚前就开始了性生活,而且是多个性伴侣,有些人采取避孕措施,有些人根本就不采取任何避孕措施,性体验时间过长、过多已很普遍,这种性行为在某种程度上也会影响到夫妻的性生活。

婚前性行为过多,对自己现在的妻子或丈夫都没有了新鲜感,有些人完全是为了传宗接代而选择婚姻。有些男女在与过去的异性朋友交往中均有过怀孕史,但婚后都不想让对方知道,这样也会加重自己的心理负担,特别是女性,由于在婚前多次做过人流或药流,而导致不孕症发生的几率明显升高,以致内心有很强的愧疚感和负罪感,这种心理状态直接影响了婚后的性生活。

在过去的性交往或婚后与异性交往中,可能不经意染上了性传播疾病,其中有些可能会直接影响到今后的性生活和生育,但由于对方都不知道,所以这块“心病”很难去除,有些疾病很可能会复发或根本就无法根治,直接影响到生育和下一代的健康,这些人每天都生活在极度的紧张和恐慌之中,生理和心理都受到了很大影响,而直接影响到生育、健康和性生活的质量。

因为不孕,夫妻双方有病乱投医,中药西药双方都没少吃,看男科说男方有问题,看妇科说女方有问题,有些甚至还做了一些相关的手术治疗。

随着辅助生育技术的广泛推广,人工授精、试管婴儿普遍应用在了不孕不育的治疗领域中,但因成功率并不高,很多患者在不明原因地失败多次后,仍然执著地做下去,但始终没能怀孕,最后落得人财两空。

长时间精神和经济上的双重压力,让他们处在了崩溃的边缘,这种长期痛苦的经历和一些药物的副作用,使原本很简单的问题越治越复杂、越治病越多,原本正常的月经不正常了,原本正常的精子水平不正常了,原本正常的夫妻生活也不正常了,这对双方的性功能影响都很大,有些甚至再也无法过性生活,最后只有采取人工授精的方式来试孕。

有些夫妻因一方或双方的原因造成多年不孕,治疗又没有效果,双方从感情上都受到很大影响,过去的爱情已转变到只是普通的亲情,没有了过去的激情和浪漫,性生活很不和谐,觉得很没意思。

受我国“不孝有三,无后为大”的传统观念影响,很多夫妻觉得没有孩子无法向家中老人交代,觉得对不起父母,所以常年外出打工或多年不回家看望父母,好一些的还能相互理解、相互安慰,差一些的就相互埋怨,感情生活每况愈下,根本就谈不上夫妻生活了。

因为一直没有生育,使夫妻双方在心理和生理上都受到很大的创伤,感情越来越疏远,有些人还有了外遇,在第三者怀孕或另找到新欢后,最终导致离婚的结局,目前由于不孕症而离婚的比例呈上升趋势。

夫妻一方或双方确实有病,而影响了正常的夫妻生活,导致不孕,性生活质量越来越差,最终不能过性生活,双方都很痛苦,但有些疾病经过治疗,痊愈后还是可以生育的,有些性功能障碍者,经过性医学科医生治疗后,恢复了性功能,生活还是过得美满幸福。

这些影响因素表现在不孕不育的生理及心理问题上,但实际上根源还是由于人体的性腺轴紊乱所致。人体性腺轴由下丘脑、垂体、卵巢(睾丸)所组成,下丘脑分泌的两个重要激素:促卵泡生成素(FSH)和促黄体生成素(LH)直接与情绪有关,如果心情不好,这两个激素分泌不正常,就会直接影响到脑垂体激素的分泌(PRL),还会通过卵巢或睾丸而影响到雌性激素(E2 )、孕激素(P)、雄性激素(T)的正常分泌,从而影响到卵子和精子、性功能和性生活的质量。

而不孕不育症是一个综合性问题,它是集妇产科、泌尿科、性医学科、心理学科为一体的综合性学科,在此学科的研究中,还有很多有待解决的问题,大量的基础和临床结合的研究项目有待开发,所以要求治疗不孕不育的专业人员应该具备一定的综合素质,在工作中不断学习、研究、总结经验教训,为提高人们的生育生活质量做一些有益的工作。

(李潭 孙伟)


\section{第六节 精神障碍与性}

精神分裂症(schizophrenia)是一组常见的病因尚未明确的精神病。多起病于青壮年,常缓慢起病,具有思维、情感、行为等多方面障碍,及精神活动与环境的不协调。通常意识清晰,智能尚好,有的患者在疾病过程中可出现认知功能损害。自然病程多迁延,呈反复加重或恶化,但部分患者可保持痊愈或基本痊愈状态。

持久性妄想性障碍是指一组以系统妄想为主要症状,而病因未明的精神障碍,其特点是出现一种或一整套相互关联的妄想,妄想往往持久,有时持续终生,妄想内容有一定的现实性,不经了解,有时难辨真伪。妄想的内容变异很大,常为被害、疑病或夸大性的,但也可与诉讼或嫉妒有关;或表现为坚信其身体畸形,或确信他人认为自己有异味或是同性恋者。典型病例缺乏其他精神病理改变,但可间断地出现抑郁症状,某些患者可出现幻嗅和幻味。起病常在中年,但有时可在成年早期(尤其是确信身体畸形的病例)。妄想的内容及出现时间常与患者的生活处境有关,如少数患者出现被害妄想。除了与妄想或妄想系统直接相关的行为和态度外,情感、言语和行为均正常。

许多研究已经证实,性功能障碍在精神分裂症患者中的确是一种常见现象,其发生率在不同的研究中有不同的报导,为16\%~78\%。精神分裂症患者中性功能障碍的类型是有性别差异的。男性患者多表现为性欲降低或亢进、勃起障碍、射精障碍、性高潮障碍,包括勃起困难、勃起维持困难、延迟或抑制射精、逆行性射精和自发性射精;女性患者常出现性欲降低、阴道润滑困难、性高潮障碍(包括性高潮困难、性高潮的质量改变和性快感缺失)。

精神分裂症患者较多表现为性欲减退。近年来的研究发现,精神分裂症患者体内的激素和某些代谢物质有改变,这是造成其性欲降低的原因。也有部分患者表现为性欲亢进,临床上青春型精神分裂症患者常出现追逐异性,言语轻浮、污秽,表现为低级意向活动的亢进,考虑与皮层下过度释放有关。

临床研究表明,与性有关的精神症状在精神分裂症患者中是很常见的,约占25.8\%。其中以钟情妄想(35\%)、嫉妒妄想(26\%)、性色彩幻听(22\%)、性被害妄想(14\%)等症状多见(王锦霞等,1993)。性色彩触幻觉女性多见,患者常在黑暗中凭空感受到自己的生殖器被杵弄,产生触幻觉,在此基础上常继发性被害妄想,坚信有异性强奸自己,或继发钟情妄想,认为异性暗恋自己,甚至想方设法和自己发生了性关系,还有的患者在此基础上坚信自己有了身孕,要求某某人(妄想所涉及的异性对象)对她要负责。部分患者受钟情妄想的影响,认为某异性爱他,因此要与现配偶离婚,出现心因性性功能障碍;有的患者受幻听影响,听到有人说妻子与许多人发生了性关系,继发了嫉妒妄想而不能与妻子进行正常性生活;有的患者感到夜间有人吸他的精液,生殖器已被吸空,因此出现心因性阳痿。当然,有些精神分裂症者具有阳痿、无性高潮和早泄。在这种情况下,患者的性症状可能相对地独立于精神分裂症过程中,但也可能与他精神分裂症的心理异常或心理防御有密切联系。

目前业内学者多认为精神分裂症患者的临床症状主要包括阳性症状、阴性症状、认知症状、情绪症状、攻击敌意五维症状群。其中阴性症状是指精神活动功能的减退或缺失,可表现为思维贫乏、情感淡漠、意志缺乏及行为退缩。精神分裂症急性期患者的阳性症状往往突出,阴性症状被掩盖,但随着疾病慢性化的进程,阴性症状逐渐缓慢加重,占据主要临床相,阳性症状反而不明显。严重者出现显著的精神活动全面衰退,包括高级意向活动和低级意向活动的明显减退或缺乏。阴性症状突出的精神分裂症患者往往表现性欲缺乏、性唤起障碍及性能力丧失等。

抗精神病药物引起性功能障碍的机制 抗精神病药物可能通过以下几个途径影响性功能:①通过对中枢多巴胺(D2)受体的阻滞,抑制多巴胺的释放,中枢多巴胺递质的减少使性欲降低、勃起功能减退。②多巴胺的阻滞导致血清催乳素水平提高,从而对体内激素水平产生影响,出现男性乳房发育、闭经、溢乳等;血清催乳素水平的提高,还能够抑制性活动的各个水平(如:性欲望、勃起、性高潮等),导致性功能障碍的发生;另外,血清催乳素水平升高还能降低体内睾酮的水平,导致性行为的减少。③抗精神病药引起的镇静作用和体重增加,会降低性兴趣。④外周胆碱能受体拮抗、a-肾上腺素能受体阻滞,可以引起性高潮障碍、射精障碍。⑤5-羟色胺的作用机制也是产生性功能障碍的一个重要因素。5-羟色胺在中枢神经系统中是一种神经递质,而在外周神经系统中却有收缩血管和舒张血管的作用。5-羟色胺可能是通过调节血管的舒张和收缩参与性唤起的过程,同时它还作用于泌尿生殖系统的平滑肌,而且在支配性器官的神经中已经发现了它的存在。以上这些都表明,外周5-羟色胺能的活性与正常的性反应周期有关,因此改变外周5-羟色胺能活性的药物都可以影响性功能。⑥锥体外系副反应(EPS)和迟发性运动障碍可以降低性功能的灵活性。

1.针对原发疾病的治疗,比如通过使用抗精神病药物控制症状,从而改善与性有关的精神病性症状。

2.尽早发现、诊断、治疗精神疾病,以防止因阴性症状出现而造成的性功能障碍。

3.合理选择抗精神病药物,注意药物性方面的副作用,特别是注意高催乳素血症、锥体外系反应、抗胆碱样反应、体重增加等副作用的发生。必要时可以换用抗精神病药物,在专业医生指导下使用改善性功能障碍的药物,包括性激素、溴隐亭、PDE5 抑制剂、中药等。

4.积极进行精神疾病的康复训练,避免长期在精神病医院住院,最大限度地保持与社会的密切交流、社区生活、家庭生活。

情感性精神障碍(affective disorder),又称心境障碍(mood disorder),是以情感或心境显著而持久的改变(高涨或低落)为基本临床特征,并伴有相应认知和行为异常的一类精神障碍。大多数患者有反复发作倾向,每次发病常与应激事件或处境有关,发作多可缓解,部分可有残留症状或转为慢性。

根据ICD-10分类,情感性精神障碍包括躁狂发作(manic episode)、双相情感障碍(bipolar disorder)、抑郁发作(depressive episode)、复发性抑郁障碍(recurrent persistent disorder)、持续性心境(情感)障碍[persistent mood(affective)disorder]等。

情感障碍发病机制不明。研究最多、了解领域最深的是生物胺水平、通路和结构的异常。基本一致的观点是儿茶酚胺,特别是去甲肾上腺素(NE)在重要脑区相对或绝对缺乏与抑郁相关,而躁狂是儿茶酚胺过多。五羟色胺(5-HT)系统功能低下与情绪低落、自杀相关,而且在5-HT功能低下的基础上,NE功能低下出现抑郁,NE功能亢进出现躁狂。其次,γ-氨基丁酸(GABA)、神经肽类如血管紧张素(vasopressin)和内源性阿片样物质在情感障碍发病中有一定作用,如抗抑郁药物和电抽搐(ECT)治疗可以改善GABAβ受体数目,起到抗抑郁效果。中枢谷氨酸(兴奋性氨基酸)受体5个亚型中的代谢性谷氨酸受体(mGluR2 )与抑郁的发病有一定关联。下丘脑-垂体-肾上腺轴(HPA)和下丘脑-垂体-甲状腺轴(HPA)的神经内分泌激素水平、节律以及激素在突触前释放增加、突触后受体功能下调,与情感障碍相关。应激可以造成神经免疫学改变,趋炎细胞因子如白细胞介素(IL)2和3、肿瘤坏死因子、干扰素-α/β等水平升高,临床则表现出衰弱、疲乏、快乐缺失、厌食、注意力不集中,并可能导致5-HT水平下降和HPA轴功能亢进。此外,脑影像学显示:双相情感障碍(男性为甚)有脑室扩大,重性抑郁症患者尾状核体积缩小、额叶萎缩,双相Ⅰ型别人细胞膜磷脂代谢异常,部分抑郁症患者额叶皮层血流量减少。

遗传因素对双相情感障碍的影响比抑郁症大,调查发现:一级亲属双相情感障碍患病率是普通人群的8~18倍,抑郁症患病率是普通人群的1.5~2.5倍;单卵孪生双相情感障碍同病率为33\%~90\%,双卵孪生双相情感障碍同病率5~25\%,抑郁症同病率为10\%~25\%。血缘关系越近,遗传风险越大。

社会心理因素在易感人群中起到了诱发作用。

2005年底,世界卫生组织披露:目前全球抑郁人口多达1.2亿。而中国心理卫生协会的有关统计显示:中国有超过2600万的人患有不同程度的抑郁症,其中90\%的抑郁症患者没有意识到自己可能患有抑郁症,并及时就医。西方发达国家20世纪七八十年代流行病调查显示,双相障碍终生患病率3.0\%~3.4\%,90年代上升为5.5\%~7.8\%;20世纪90年代中国香港及台湾地区的流行病学调查资料显示,双相障碍的终生患病率为1.6\%,较同期(1982)大陆12地区调查(0.042\%)高出35倍。分析除了与经济社会状况差异有关外,更重要的是与大陆的流行病调查方法学的差异有关。

关于情感性精神障碍患者的性功能障碍问题,目前研究得不是很多,而且主要集中在抑郁发作以及抗抑郁药对性功能的影响方面,而对躁狂发作或双相障碍患者的性功能状态研究得比较少。随着社会和医学的进步以及性观念的开放、人们对性生活质量追求的不断提高,情感性精神障碍引起的性功能障碍问题越来越受到人们的重视。

多项研究报道,25\%~75\%的单相抑郁(unipolar depression)患者出现性欲缺乏,其患病率与抑郁的严重程度有关。性唤起障碍在重性抑郁(major depression)患者中也较常见,将近25\%的此类患者存在勃起或润滑方面的问题。根据有限的研究资料显示,未经治疗的抑郁症患者其性高潮障碍的发生率也较普通人群高。至于抑郁与性功能障碍之间的关系以及其内在的联系机制尚未明确。目前研究的焦点主要从现象学的角度以负性情绪与性功能障碍、负性认知与性功能障碍作为切入点展开。

(1)负性情绪与性功能障碍:根据传统的精神病学常识,抑郁状态下,人的性兴趣、性反应降低,而在情绪高涨的状态如轻躁狂状态下,性兴趣增加。而近来的研究发现,情感障碍与“强迫性性行为(compulsive sexual behavior)”和“性瘾(sexual addictions)”之间存在一定的同病率,说明负性情绪不总是在同一个方向对性功能产生影响。早在1967年Beck报道,61\%的严重抑郁患者性兴趣减退,而非抑郁对照组只有27\%性兴趣减退。Beck还发现,性兴趣缺失与疲劳感、食欲下降、体重下降和失眠相关,并由此认为性兴趣减退是抑郁症生物性症状的一部分。此外,Cassidy(1957)等也有相似报道。Schreiner-Engel和Schiavi(1986)在一项研究中发现,绝大多数患者在先前的抑郁发作中出现性欲减退,甚至持续至发作后。Araujo等(1998)在一项社区研究中在控制诸如年龄、生理健康状况等混杂因素后发现,勃起障碍与抑郁症状有关。尽管他们发现勃起功能障碍与性兴趣减退有一定关联,但是他们惊讶地发现抑郁与性欲减退无明显相关性。苏黎世的一项纵向队列研究把“抑郁”的范围扩大至涵盖重性抑郁症、恶劣心境(dysthymia)和反复发作的短暂性抑郁(recurrent brief depression)。结果发现,性欲减退与抑郁有关,但在女性更明显。然而,抑郁的患者,其性欲不全都减退。Mathewand Weinman(1982)发现在57名抑郁症患者中,31\%诉性欲减退,而22\%诉性欲增加。Angst(1998)发现抑郁对性欲的改变与性别有关,在男性抑郁症患者中,25.7\%诉性欲减退,23.3\%诉性欲增加(非抑郁对照组,其比例分别为11.1\%和6.9\%)。对于女性抑郁患者,只有8.8\%在抑郁时诉性欲增加,35.3\%诉性欲减退(非抑郁对照组,其比例分别为1.7\%和31.6\%)。Nofzinger等(1993)在对接受认知-行为治疗的男性抑郁患者进行研究时发现,那些治疗效果欠佳的患者,其性欲水平较病情缓解组和非抑郁对照组要高,而且他们的焦虑水平以及抑郁发作次数也较病情缓解组和非抑郁对照组高。根据这一发现,Nofzinger等认为,性欲的改变或许可以作为情感障碍分类的一个参考指标。John Bancroft等的进一步研究发现,那些在抑郁状态下性欲减退的患者,往往想独处、想弄清是什么原因导致他(她)抑郁,而不把性作为情绪调节的工具;而那些抑郁状态下性欲增强的患者,往往把性作为寻求亲密、自我证明、情绪调节的工具。无论是性欲减退还是性欲增强,均说明负性情绪是性行为“失控”的高危因素。

(2)负性认知与性功能障碍:根据认知理论,人的行为是由人的“认知图式(cognitive schemas)”决定的。所谓“图式”,是指一个相对稳定的认知模式,它是根据过去的经验发展起来的,并决定了对将来经历的感受和理解。当一个人面对某一特定情景时,与之相关的“图式”被激活,个体以此为基础对此情景中的刺激信息进行筛选、辨别或编码,最后形成判断、作出决定并付诸行动。根据这一理论,如果一个个体具有正性的与性有关的图式,一个性刺激就有可能激活其记忆系统中的性意义,并由此激起生殖系统/主观感觉的性反应;而如果一个个体具有的是负性的与性有关的图式,一个性刺激就有可能以非性方面的或者是负性的方式被个体感受、理解,其生殖系统/主观感觉的性反应就无法激活,其性唤起就有可能被抑制。对于抑郁症患者而言,其负性情绪和关于其个人总体方面的负性评价有可能促成了患者关于性方面的负性图式,进而抑制了患者的性反应。为了验证这一理论假设,Wtephanie等对28名女性抑郁患者以及28名非抑郁女性进行研究,结果支持上述理论观点。

目前对患者性功能状况的研究甚少。国内田峰等对58名躁狂症患者进行调查研究,发现性欲亢进、性生活频度增加者35例,占60.34\%;早泄者22例,占37.93\%,其中与性欲亢进并存者19例,占性欲亢进总数的54.28\%;性欲减退6例,占1O.34\%;另有24例表现为其他种种性行为的异常,为追逐异性、赤身裸体、口出秽言等,占41.37\%。有学者认为,躁狂患者性欲减退可能为不切合实际的期待产生阴茎勃起功能或性活动能力未能达到患者预想效果而“先扬后抑”所致,而早泄被视为对性冲动的控制能力降低或对性刺激的反应阈值下降所致。

目前尚没有关于双相障碍与性功能障碍之间关系的系统研究。从双相障碍的发作形式看,患者的性功能状况要较单纯的抑郁或躁狂发作患者的性功能状况要复杂得多。笔者曾遇到一位年轻的未婚女性双相障碍患者,当其处于躁狂状态的时候,性活动明显增强,衣着性感,当街勾引异性,并与之发生一夜情,可一旦转到抑郁状态下,性欲明显减退,并对之前轻率的性行为十分后悔,觉得自己十分堕落。还有一位属于双相障碍(混合状态)的未婚男性,大部分时间其情绪处于抑郁状态,而其性欲则处于持续亢奋状态,一见到陌生异性即有性冲动,曾找过妓女泄欲,但性欲亢进的状况无明显改善,反而加重了自己的思想负担,常常为自己的性问题感到懊悔,因此情绪更加低落。由此可见,双相障碍与性功能障碍之间的关系是十分复杂的,而且有时两者可互为因果,形成恶性循环。

(1)抗抑郁药物的性问题

1)抗抑郁药与性高潮缺乏或延迟:Harrisn(1985)使用性功能问卷对服用丙米嗪、苯乙肼及安慰剂6周以后的男性患者进行了调查,结果发现,有21\%的男性患者服用丙米嗪后出现性高潮延迟,服用苯乙肼的男性患者为36\%,服用安慰剂仅为11\%。Monteiro等(1987年)采取直接询问患者的方式调查了氯丙咪嗪在治疗强迫症患者时引起的性功能障碍,研究发现,96\%服药前能达到性高潮的患者在服药后出现性高潮缺乏,而Segraves(1993)、Althof(1995)则利用氯丙咪嗪所致的性高潮延迟治疗早泄。大量的回顾性研究提示,SSRI类中氟西汀、帕罗西汀与舍曲林引起性高潮延迟和射精延迟的发生率相近,大约在30\%左右,氟伏沙明对射精的影响相对最小。Wendiner等(1994年)的双盲对照研究表明,帕罗西汀能有效治疗早泄,这也反证了帕罗西汀能引起射精延迟。Feiger等(1996年)发现,使用尼法唑酮(nefazodone)的女性比服用舍曲林的女性更容易达到性高潮,47\%的女性在服用舍曲林后很难达到性高潮,67\%的男性出现射精困难;而服用尼法唑酮的患者则与用药前无明显变化。Mendels等(1995)则用舍曲林成功地治疗了早泄的患者。Segraves(1993)的一篇综述中还提到,阿米替林、阿莫沙平(amoxapine)、去甲丙咪嗪(desipramine)、多虑平、马普替林、去甲替林(notriptyline)、普罗替林(proteiptyline)、三甲丙咪嗪(trimipramine)、曲唑酮(trazodone)等均能引起性高潮缺乏或延迟。去甲肾上腺素(NE)和DA促进射精和性高潮的到来。氯丙眯嗪和SSRI激动5-HT2受体,抑制NE和DA释放,从而可以抑制射精和性高潮,其中氟伏草胺还能激动5-HT1A受体,后者衰减5-HT2受体抑制NE和DA释放的效应,故抑制射精比其他SSRI为轻。萘法唑酮和米氮平的5-HT回收阻断效应较弱,丁氨苯丙酮无5-HT回收阻断效应,故很少发生射精和性高潮功能障碍。其中萘法唑酮、米氮平和赛庚啶还能拮抗5-HT2受体,使NE和DA脱抑制性增加,丁氨苯丙酮和育亨宾促进NE释放,金刚胺和溴隐亭增加DA能活性,均能促进射精和性高潮到来。

2)抗抑郁药与勃起障碍:在中枢,中脑边缘系统的DA能加强性唤醒。丁氨苯丙酮拟DA能,改善性唤醒;SSRI拟5-HT2受体,从而抑制DA释放,抑制性唤醒。在外周,骶2、3副交感神经纤维兴奋释放乙酰胆碱(Ach),引起阴茎勃起困难。

刺激腰交感神经纤维引起阴茎弛缓,心因性勃起是通过阻断腰交感神经纤维释放NE,导致α1-肾上腺素受体阻断和阴茎勃起。TCA、单胺氧化酶抑制剂(MAOI,如苯乙肼和反苯环丙胺)和不典型抗抑郁药(如麦普替林)均拟NE能,可引起阳痿。

(2)抗躁狂药物的性问题:用于躁狂发作的药物以碳酸锂为代表。目前尚无有说服力的关于锂盐与性功能障碍关系的报告。但是,由于锂盐本身的药理特点及副作用,在少数患者身上可能会出现反应迟钝、认知障碍(在对正常志愿者做实验时亦可发现识记和信息加工障碍)、周围神经病变、甲状腺功能减低(女性相对多见)、消化道反应、体重增加以及锂中毒造成的急性脑病综合征,这些都会间接造成性问题。

抗癫痫药物如卡马西平、丙戊酸钠、拉莫三嗪、妥泰等被当作心境稳定剂广泛应用于躁狂发作及双相情感障碍治疗中。由于自身的药理特点,这些药物可以诱发一些性问题。比如,卡马西平可引起低钠血症,往往使体内的水潴留,导致性欲低下;丙戊酸钠的致畸率在1\%~1.5\%;多种抗癫痫药物的消化道反应,继发躯体性不适,包括性问题。

目前,各种新型抗精神病药物,比如奥氮平、喹硫平、利培酮、齐拉西酮等,都作为治疗躁狂发作或心境稳定剂,被广泛用于临床。其可能引起的药源性性问题可以参见本章第一节中相关内容。

1.积极治疗原发情感障碍,比如通过使用抗抑郁药物、情感稳定剂、抗躁狂药物控制症状,从而改善与性有关的情感症状。

2.通过有针对性的性治疗、心理治疗,改变负性认知,调节人际关系,学习性技巧,改变因情感障碍和其他因素共同作用而导致的性心理问题。

3.尽量回避抗抑郁药物对性功能影响的副作用。轻度的抑郁障碍可以暂时不使用药物治疗。必须使用抗抑郁药物时要特别关注性方面的副作用,一旦出现要积极处理。可以合并改善性功能的药物,如使用PDE5抑制剂改善勃起障碍、使用润滑剂改善性交疼痛;还可以辅以性工具、性感集中训练、物理治疗等。

4.注意利用某些抗抑郁药物的特点,既改善抑郁,同时也改善早泄、性欲低下等性问题。

(1)神经症(neurosis)旧称神经官能症,是一组轻型精神障碍的总称,主要表现为焦虑、抑郁、恐惧,强迫、疑病症状,或神经衰弱症状的精神障碍。神经症患者常自觉其精神活动能力受损,产生焦虑和烦扰,或为各种躯体不适感所苦;体格检查不能发现脑器质性病变或躯体疾病作为其临床症状的基础;自知力大都良好,无持久的精神病性症状;通常不会把自己的病态体验与客观现实相混淆,患者现实检验能力未受损害;行为一般保持在社会规范容许的范围内,可为他人理解和接受;常迫切要求治疗。起病多与素质、人格特征或精神刺激有关,病程多迁延或呈发作性。临床症状至少有下列1项:①恐惧;②强迫症状;③惊恐发作;④焦虑;⑤躯体形式症状;⑥躯体化症状;⑦疑病症状;⑧神经衰弱症状。

(2)应激相关障碍(reaction to severe stress)又称心因性精神障碍(psychogenic mental disorders),是一组心理社会因素所致的精神障碍。导致此类精神障碍发生、转归、临床症状的相关因素大致有三方面:一是应激性生活事件或不愉快的处境;二是患者个体易感性;三是文化传统、教育水平及生活信仰等。应激性生活事件常引起情绪反应或某些精神异常,但其严重程度并未达到抑郁症或焦虑症的诊断标准。主要临床分类有:急性应激相关障碍(acute stress reaction)、创伤后应激障碍(post traumatic stress disorder)、适应性障碍(adjustment disorder)。

(3)分离〔转换〕障碍【dissociative(conversion)disorders】旧称癔症(hysteria),是由精神因素如生活事件、内心冲突、暗示或自我暗示,作用为易病个体引起的精神障碍。分离,是指对过去经历与当今环境和自我身份的认知完全或部分不相符合。转换,是指精神刺激引起情绪反应,接着出现躯体症状,一旦躯体症状出现,情绪反应便褪色或消失,这时的躯体症状便叫做转换症状。有时癔症可以表现为精神病状态,此时称之为癔症性精神病。

人类性活动的功能状况不仅仅是一种生理现象,也与心理活动、人格特点紧密相关。神经症、应激相关障碍、分离〔转换〕障碍三类精神障碍的共同点就是其发病与精神因素、人格素质相关。这些患者多具有不良人格素质基础,一旦遇到应激性生活事件,就会出现各种适应障碍、情绪反应或精神病性障碍。他们的婚姻质量、性生活质量较正常人低,性功能障碍发生率较正常人群更高。下面仅以神经症为例,解释性问题在上述各种精神障碍患者中广泛存在的原因。

弗洛伊德认为神经症是由于压抑和性放纵所致。压抑性欲使性驱动力积聚过多而转化为焦虑;纵欲则使性驱动力耗竭,酿成神经衰弱和疑病性神经症。无意识内心冲突是一切神经症的根源,一切神经症的核心则是焦虑。巴浦洛夫则以人的高级神经类型学说来解释神经症的生理基础,认为神经症是持续的高级神经活动紊乱,即兴奋和抑制这两种基本神经过程的失调。他以神经过程的强弱、灵活性、均衡性以及第二信号系统的特征解释患者对神经症的易感性和病理反应形式的个体差异。心理学家们已经意识到心理冲突是神经症发病的重要根源。心理冲突凡人皆不可免,指的是两种相互对立的情感或欲望同时并存于一个人的心里,当事人既不能放弃其中之一,又无法在更高的水平上将二者整合或统一协调起来,从而体验着紧张、不安或其他不快甚至痛苦。神经症性心理防御机制的过度使用,则使得这些神经症性心理冲突妨碍了正常的心理功能(如集中注意,良好的记忆,有条理和有效的思考,相对稳定的心情,作出抉择并付之于有效行动,等),或导致社会功能受损(如不能上学,不能操持家务,不能正常工作,造成人际关系不和或不能与人正常交往等),可以使人陷于非建设性甚至破坏性的疾病状态之中,因而表现出各种各样的临床征象。按照精神分析的观点,神经症症状的形成过程为:幼年时期未能解决的冲突→现实冲突→焦虑→退行到幼年的某种体验或行为方式中→幼年冲突的现实化→冲突性焦虑加重→不恰当的防御→在冲突的双方之间达成“妥协”(让步)→最终以症状的形式表现出来。

神经症患者常见的性问题包括性欲障碍(性冷淡、性厌恶、性欲亢进等)、阴茎勃起障碍(阳痿、阴茎勃起不坚、阴茎异常勃起等)、性交障碍(性交疼痛、性交昏厥、性交失语、性交癔症、性交恐惧症等)、射精障碍(早泄、遗精、不射精、射精疼痛、性高潮缺乏等)。有文献报道:在惊恐障碍患者出现的性问题中,性厌恶障碍是最常见的类型,约有36\%的男性患者和50\%的女性患者有此障碍。在社交恐怖症男性患者中,早泄是最常见的类型。

癔症患者可出现性交一过性癔症性躯体障碍,指患者在性交时发生的躯体感觉障碍和(或)躯体运动障碍,其表现多种多样:可表现为突然失明、失聪,或一只手,或前臂,或以中线为界半身感觉丧失,但经神经系统检查,未发现器质性病变;或表现为突然肢体瘫痪,或单瘫,或截瘫,或偏瘫,但经神经系统检查、各种生理反射正常,无锥体束征,电刺激反应也正常。患者在每次性交时都发生,症状基本上相同,在大脑皮层形成抑制性“病灶”,重复出现则形成了条件反射,即出现反射弧:性交→大脑皮层抑制反应→躯体障碍(感觉或运动)→自行缓解。

除对原发病的积极治疗外,同时要特别注意对人格缺陷的矫正,尽量消除容易引起内心冲突的应激事件负面影响,避免过度使用消极的心理防御机制,从而避免性问题。在这类精神障碍继发的性问题处理上,心理治疗的价值远远高于简单直接的性治疗、药物治疗。

人格障碍(personality disorder)是指人格特征明显偏离正常,使患者形成了一贯的、反映个人生活风格和人际关系的异常行为模式。这种模式极端或明显偏离特定文化背景、一般认知方式(尤其在对待他人方面),对社会环境适应不良,明显影响其社会功能与职业功能,并已具有临床意义,常自感精神痛苦。患者虽然无认知功能缺损,但适应不良的行为模式难以矫正。通常开始于童年或青少年期,并长期持续发展至成年或终生,仅少数患者成年后可能在程度上有所改善。由各种疾病,如躯体疾病(例如脑病、脑外伤、一氧化碳中毒、慢性酒中毒)、精神障碍导致的人格特征偏离正常,应作为原发疾病的症状,称为人格改变。常见的人格障碍有偏执型人格障碍、分裂样人格障碍、反社会型人格障碍、表演型人格障碍及强迫型人格障碍等。

人格是指由遗传决定的个人先天素质以及后天发育与习得性有机结合形成的总体精神活动(思维、情感和行为)模式。人格特征可在社会活动、人际关系中表现出来,也可在社会生活实践中塑造和发展。人格障碍往往由以下因素综合形成,但幼年期心理因素起主要作用。

(1)生物学因素:有学者发现人格障碍的亲属中患人格障碍的比率显著高于正常人群,由此提示人格障碍的遗传因素不能忽略。也有报告人格障碍者脑电图异常比率高于正常人群,因而生物学因素对人格障碍有一定的影响。

(2)心理发育影响:幼儿心理发展过程中受到精神创伤,对人格的发育有着重大的影响,是未来形成人格障碍的主要因素。婴幼儿被剥夺母爱或父爱,被遗弃或受歧视;其他亲人过分溺爱,使其以自我为中心的思想恶性膨胀,异常地发展为无视父母、校规与社会纪律,这为发展成反社会性人格障碍提供了温床。研究表明,人格障碍和犯罪者其植物神经功能异常,植物神经反应性低下,皮肤电恢复缓慢,作为罪犯和人格障碍的一种易病素质特征。幼儿与青少年期受虐待导致产生仇恨与敌视社会或人类的心理。父母或其他抚养者、幼儿园或小学老师教育方法失当或期望过高,过分强迫、训斥易造成精神压力或逆反心理,形成不良人格。父母本人品行或行为不良,对儿童的人格发育影响极大。

(3)社会环境影响:社会上的不良风气、不合理现象、拜金主义等都会影响青少年的道德价值观,产生对抗、愤怒、压抑、自暴自弃等不良心理而发展至人格障碍。目前一般认为人格障碍与精神疾病间的关系为:人格特征可成为精神疾病的易感因素或诱因;某些人格特征是精神疾病的潜隐或残留表现;人格障碍和临床综合征可有共同的素质与环境背景,两者可共存,但不一定有病因联系。

1982年和1993年我国部分地区精神疾病调查结果显示,人格障碍患病率为0.1\%,而国外的调查结果是2\%~10\%。除表演型人格障碍男女患病比例相近外,偏执型人格障碍、反社会型人格障碍、分裂样人格障碍、强迫型人格障碍等,男性患病均高于女性。

以猜疑和偏执为特点。始于成年早期,男性多于女性。表现为对周围的人或事物敏感、多疑、不信任,过分警惕与抱有敌意;遇挫折或失败时,强调自己有理,夸大对方缺点或失误,易与他人发生争辩、对抗;常有病理性嫉妒观念,怀疑恋人有新欢或伴侣不忠;易记恨,自认为受到轻视、侮辱、不公平待遇而耿耿于怀,引起强烈的敌意,常有回击、报复之心。

以观念、行为、外貌装饰的奇特,情感冷漠,人际关系明显缺陷为特点,男性略多于女性。由于与家庭和社会疏远、独来独往,爱幻想,独出心裁,脱离现实,有奇异信念,部分患者伴有性变态,例如恋兽癖、恋童癖、恋物癖等;性行为怪异,离奇。另外,患者对人冷漠,缺乏情感体验,可伴有性冷淡。

以行为不符合社会规范,具有经常违法乱纪,对人冷酷无情为特点,男性多于女性。本组患者往往在少儿期就出现品行问题,无视家教、校规、社会道德礼仪,甚至出现性犯罪行为,成年后(指18岁后)习性不改,对家庭亲属缺乏爱和责任心,不抚养子女或不赡养父母,待人冷酷无情,甚至出现乱伦;经常撒谎、欺骗,以获私利或取乐,无羞耻感,性行为极不检点,令其家属、亲友、同事、邻居感到深恶痛绝。

以过分感情用事或夸张言行以吸引他人注意为特点,患病率两性无明显差异。性欲亢进,情感体验较肤浅,情感反应强烈易变,常感情用事,暗示性强,意志较薄弱,容易受他人影响或诱惑,常有多个性伴侣,性关系不稳定,爱幻想,不切合实际,夸大其词,可掺杂幻想情节,喜欢寻求刺激而过分地参加各种社交活动。

以过分严格要求与完美无缺为特征,男性是女性的3倍。临床特征为循规蹈矩,按部就班,不容改变,否则感到焦虑不安,并影响其工作效率;拘泥细节,甚至对生活小节也要程序化,清洁成癖;若不按照要求做就感到不安,要求重做;常有不安全感,往往穷思竭虑,对计划实施反复检查、核对,唯恐有疏忽或差错,因此患者可能会出现性功能障碍;由于怕脏,产生错误的性观念,认为性是肮脏的,出现心因性的性欲低下;由于做事过于循规蹈矩,主观、固执,比较专制,要求别人也要按照他的方式办事,否则即感不愉快;可能会对每次性行为的实施要求刻板,不仅令患者本人感到痛苦,而且令配偶对性生活产生厌烦情绪,久而久之,性欲日渐低下;伴随有焦虑和抑郁时,性欲低下更为明显,甚至会出现性欲消失;为追求完美的性生活,获取性高潮,部分患者可能有性欲旺盛,性交时间延长等情况。

尤其是认知行为治疗,通过深入接触,同障碍者建立良好关系,帮助其认识个性的弱点,鼓励重建一个更为健全的行为情感模式。尽可能创造一个健康的生活和学习环境,让人格障碍者通过在团体中的有益活动,控制和改善自己的偏离行为,纠正既往习得的不良习惯。

抗精神病药、MAOI、锂盐、卡马西平、BZ类药物、抗癫痫药、受体阻滞剂、5-HT类药物等对人格障碍有疗效。目前发现,使用抗精神病药对分裂型人格障碍有效,可以改善牵连观念、兴奋冲动等,而抑郁、焦虑等症状可以应用抗焦虑药、抗抑郁剂以及情感稳定剂等进行治疗。对冲动性人格障碍伴有脑电图改变者可试用苯妥英纳或卡马西平等抗癫痫药物,可合并β受体阻滞剂治疗。

脑器质性精神障碍是指由于各种原因产生的脑部外伤、感染、肿瘤及代谢异常导致脑功能紊乱,可能出现意识障碍、注意障碍、认知功能障碍(涉及记忆、智能及学习功能)、幻觉妄想、抑郁、焦虑、情感高涨和人格改变以及行为异常。痴呆是指一种脑部疾病所致的综合征,通常具有慢性或进行性的特点,常常在意识清晰地情况下,出现多种高级皮层功能的紊乱,其中包括记忆、思维、定向、理解、计算、学习能力、语言和分析综合能力,以智能明显减退为特征。临床上常见有阿尔茨海默病(AD)、脑血管病所致精神障碍、癫痫所致精神障碍以及颅脑外伤感染所致精神障碍。

AD与遗传关系密切,患者存在广泛的胆碱能神经元变性和脱失,皮层和海马NE和5-HT含量减少。也有研究提示,水中铝含量较高的地区其居民患AD的危险增高,另外,脑外伤、免疫系统机能异常等均与AD的发生有关。其他一些脑器质性疾病造成脑代谢和功能紊乱,即会产生不同精神障碍。

英国Martin Roth在1978年研究发现AD的患病率为6.2\%,AD是一个与年龄相关的疾病,Evans等在1989年报道,65~74岁的患病率为3\%,75~84岁为19\%,超过85岁为47\%。女性患病率是男性的2~3倍。

痴呆以及其他脑器质性精神障碍患者在额叶、颞叶受累,患者常有明显人格改变,表现为懒散退缩,以自我为中心;出现性的脱抑制,不知羞耻,当众裸体、大小便或公开手淫;食欲性欲亢进,随意与女性过分亲近,与以往判若两人;有明显抑郁焦虑的患者,性欲减退或消失。

对于病因不明的痴呆或脑器质性疾病,无法实施对因治疗,对于轻症患者采取支持性心理治疗,加强行为训练。另外,针对焦虑、抑郁及精神病性症状可以相应地采取抗抑郁剂、抗焦虑及抗精神病药物治疗。总体来说,应采用促大脑代谢治疗以改善脑循环。针对AD患者,可以采用增加乙酰胆碱合成和释放、抑制乙酰胆碱降解的药物,具有一定的疗效。使用清除氧自由基的药物,也可减少脑细胞的死亡,改善脑功能,进一步改善性问题。对于病因明确的脑外伤、感染、肿瘤等疾病,应积极治疗原发病,同时加强行为训练。心理治疗以支持治疗为主,以减轻不良情绪,减少不良刺激。

精神活性物质(psychoactive drug or substance)是指摄入人体后会对思维、情感、意志行为等心理过程产生影响的物质。较为常见的精神活性物质有乙醇、阿片类物质,而在ICD-10中,精神活性物质滥用(psychoactive substance abuse)指一种适应不良的物质使用方式,特征是:①尽管认识到对身体有害仍使用;②已造成社交、职业、心理、躯体问题或使之恶化,仍继续使用。诊断前一般应优先考虑“成瘾”的诊断。1969年WHO提出滥用是指持续或间歇性精神活性物质(酒或药物)过度使用,与可接受的医疗措施不相称或无关。CCMD-3采纳两者结合的描述性定义,即:指持续或间歇性精神活性物质(酒或药物)过度使用,与可接受的医疗措施不相称或无关。

药物依赖(drug dependence)是指对药物强烈的渴求。患者为了谋求服药后的精神效应以及避免断药而产生的痛苦,强制性地长期慢性或周期性地服用。耐药性(tolerance)是指重复使用某种药物后,其应用逐渐减低,如欲得到与用药初期的同等效应,必须加大剂量。交叉耐药性是指某种药物形成的耐药性,是对开始用其他药物时出现耐药性而言,在吗啡及其他镇静剂、乙醇和许多镇痛安眠药之间可见。药物依赖性有精神依赖(psychological dependence)和躯体依赖(physical de-pendence)之分。精神依赖是指患者对药物的渴求,以期获得服瘾药后的特殊快感。精神依赖的产生与药物种类和个性特点有关。可产生依赖的药物很多,ICD-10将其分为十大类,即:乙醇类、鸦片类、大麻类、镇静催眠剂、可卡因类。其他兴奋剂包括咖啡因、致幻剂类;烟草、挥发性溶剂;其他精神活性物质。容易引起精神依赖的药物有:吗啡、海洛因、可待因、度冷丁及巴比妥类、乙醇、苯丙胺、大麻等。躯体依赖是指反复使用药物使中枢神经系统发生了某种生化或生理变化,以致需要药物持续存在于体内,以避免出现戒断综合征(withdrawal syndrome)的症状。轻者全身不适,重者出现抽搐,可威胁生命。可引起躯体依赖的典型药物有:吗啡类、巴比妥类和乙醇。也有些药物只引起精神依赖,而不引起躯体依赖者,如尼古丁。

引起物质依赖的因素不是单一的,与该种物质的可获得性、遗传素质、人格的易感性以及社会文化因素有关。部分成瘾者,特别是青年人,在服药前有某种程度的性格、品德障碍,如学习成绩差、逃学或违纪。家庭中有精神病或人格障碍者,童年有不愉快经历,社会文化等均对药瘾的发生有影响。社会对成瘾物质的应用呈宽容态度,因而容易泛滥,如大麻广泛流行于北美。群体内的社会压力如亲密伙伴间的压力,也会对药瘾产生影响。医护和药剂人员晚获得药物,可成为其好发阶层。

药物依赖形成的机制:①代谢耐药性和细胞耐药性。代谢耐药性是指因药代谢过程加快,在组织内浓度降低、作用减弱、有效时间缩短而言。细胞耐药性是指因神经细胞有了某种适应性的改变而引起,使神经细胞只有血液中含有高浓度药物的情况下才能正常工作。这种细胞适应性的改变机制尚不清楚。②受体学说,脑内发现了对吗啡类药物有特殊亲和力的吗啡受体以及内源性吗啡受体激动剂。因此推测药物依赖性的迅速形成可能与外源性吗啡与吗啡受体的特殊亲和力有关,后者被阻断后,造成耐药性的急剧增高。③戒断综合征的废用性增敏假说。吗啡受体长期被吗啡阻断后出现耐药性增高的同时,也可由于瘾药阻断了受体,出现废用性增敏,以致在停药过程中出现戒断综合征。④生物胺学说。研究资料证明单胺类神经递质参与镇痛和成瘾机制。自第二次世界大战以来,欧美,亚洲许多国家中,海洛因、鸦片等麻醉毒品成瘾者急速增加,业已在许多国家形成重大公共卫生和社会问题。在国外,药物依赖问题主要是青少年海洛因成瘾者居多,以享乐快感和娱乐为目的。此类成瘾者具有迅速扩散或流行性质。海洛因成瘾者,由于药物毒性大,引起自我中毒、自杀大量增加,以致这类人群的死亡率很高。英国资料显示,这类人群的死亡率高出一般人群20倍以上。我国在建国后基本控制了鸦片烟毒的流行,但自20世纪80年代后期,有死灰复燃之势,并呈迅速蔓延的趋势。药物依赖的临床常见类型,20世纪70年代以镇静安眠药为主;80年代以来抗焦虑药物成瘾者居多,吗啡类药物成瘾主要是度冷丁,多为医药滥用所致;90年代以来,临床以海洛因成瘾多见。

1.鸦片类、巴比妥类、乙醇、抗焦虑药类物质成瘾后,性欲明显减退甚至消失。男性患者出现阳痿、性欲丧失,女性月经紊乱、闭经。有时为满足性刺激和亢奋的需要,出现攻击型恋童癖,这种恋童癖是最具危险性和反社会性的恋童行为,往往伴有虐待和暴力。这种攻击行为本质上是一种恶意的侵犯。攻击的方式多种多样,往往造成攻击对象严重损伤,甚至死亡。此类虐待型的恋童癖者不仅给儿童造成痛苦,而且其行为方式极其色情。他们在虐待行为中常使用枪、刀子、铁管或皮带作为威胁伤害的工具。另外患者人格改变突出,此时,患者生活中所渴求的只有一件事:千方百计搞到毒品。部分患者为换取毒品,不惜从事性服务。部分患者可能被迫成为同性恋的性伴侣,性病发生率明显增加。

2.其他易成瘾药物:例如苯丙胺、印度大麻和可卡因等中枢神经兴奋剂,可减少嗜睡及疲劳感,使用该类物质后精力充沛,欣喜若狂,充满自信。可出现错觉和感知综合障碍,短期可出现性欲亢进,性生活增加,部分患者可出现性滥交,一般作用时间维持4小时,继之出现疲劳嗜睡。每日小量服用,很快产生耐药性。戒断综合征中以抑郁最常见,此时会出现性欲减退或消失。长期、大量服用苯丙胺,可出现苯丙胺性精神病,临床症状与偏执型精神分裂症十分相似,在意识清晰的背景下出现被害妄想,援引观念或是嫉妒妄想,无故坚信配偶或伴侣有外遇,对配偶进行跟踪盘查,实施暴力,例如强奸配偶,破坏配偶的性器官等性虐待行为。但持续时间短,停药数天、最多数周即可消失。

精神活性物质导致的性症状多与精神活性物质的使用有关,因而如何治疗精神活性物质成瘾,对于改善由此引起的性症状至关重要。多数情况下,药物成瘾后患者很难自动戒药。为确保疗效,应该杜绝患者再次得到成瘾物质。治疗中或是逐渐撤药,或是采用替代治疗的办法,同时积极改善患者营养不良状况;减轻物质戒断时产生的痛苦、焦虑、抑郁或失眠等,如果有幻觉妄想等精神病性症状,可以短期给予小剂量抗精神病药治疗;倘若发生癫痫发作,可选择抗癫痫药物进行治疗。

支持性心理治疗十分重要。患者大多意志薄弱,对治疗缺乏信心,必须经常鼓励和支持患者坚持治疗,鼓励患者参加各项文体活动,转移其成瘾物质的注意力。家庭社会支持,对患者出院后的巩固疗效十分关键。在康复阶段必须取得家庭和工作单位的支持和监督,切断成瘾物质的来源。

精神发育迟滞(mental retardation,MR)是指个体在发育阶段(通常指18岁以前)精神发育迟滞或受阻。临床表现为认知、语言、情感意志和社会化等方面的缺陷或不足,在成熟和功能水平上显著落后于同龄人。ICD-10对精神发育迟滞的诊断标准为:①起病于18岁以前;②智力明显低于同龄人的平均水平,一般智商低于70;③社会适应能力不良,表现为个人生活能力和履行社会职责有明显的缺陷。根据智能水平、适应能力缺陷程度及训练后达到的水平,可分为4级,即轻度(IQ值为50~70);中度(IQ值为35~49);重度(IQ值为20~34);和极重度(IQ值小于20)。精神发育迟滞可单独出现,也可以同时伴有某种精神或躯体疾病。

古代医书中“惽塞”、“五迟”、“五软”、“解颅”等词就是对此病的描述。国外一百多年前就开始用智力测验来筛选精神发育迟滞儿童,根据其病因和表现的不同,一般可以分为遗传因素和环境因素。

(1)遗传学因素:原因不明的MR与遗传有关。近亲结婚是原因不明的MR的一个危险因素。可能与轻微的染色体异常有关,群体中某种遗传病患者虽为数不多,但致病基因的携带者为患者的数十至数百倍,近亲大大增加了隐性基因成为纯合子的机会。

1)染色体畸变:包括染色体数目和结构的改变、性染色体畸变等。例如第21对常染色体为三体可引起Down综合征,而先天性卵巢发育不全(Turner综合征)则是因为女性丢失一个性染色体造成的,编码甲状腺激素核受体相关蛋白2(TRAP2)o P ROSIT240是人们在研究Noonan综合征疑似患者平衡易位t(2;12)(g37;q24)时新发现的,在12号染色体断位附近的人类基因,参与TRAP复合物在转录水平上调控生理过程的基因表达。PROSIT240在胚胎期和新生儿期起着至关重要的作用。PROSf1240突变可能与Noonan综合征、先天性心脏病和精神发育迟缓有密切关系,是弄清Noonan综合征、先天性心脏病及精神发育迟缓发病机制的一个非常好的候选基因。

2)单基因疾病:单基因遗传疾病比较常见,常因生化代谢异常进而导致脑功能受损,苯丙酮尿症、半乳糖血症和神经纤维瘤等均为单基因遗传性代谢缺陷病。

(2)环境因素:母孕期接触有害因素、罹患病毒性感染性疾病、服用有害药物、出生时父母年龄过大,以及胎次和产次等均与原发性精神发育迟滞有关,其机制可能与母孕期患病毒性疾病,病毒直接对胎儿的中枢神经系统的发育产生影响,出生窒息可引起脑组织充血、水肿、变性坏死,而导致智力发育障碍有关。出生后前2年脑发育最快,致病因素在此期内造成的脑损害也很严重;学龄前至小学年龄期是智力发展关键时期,此期的感染(特别是中枢神经系统的感染)、颅脑外伤、中毒、癫痫、营养不良、内分泌或代谢疾病及疫苗接种后脑炎等也同样会造成不可逆的智力低下。

精神发育迟滞是一种比较常见的精神障碍,也是导致残疾的原因之一。WHO(1985)报导本病轻度障碍患者的患病率约为3\%,中度及重度患者为3‰~4‰。轻度精神发育迟滞患者在学龄期的患病率高于成年期,可能是因为患儿在入学后其智力活动较其他儿童明显落后才被发现,而部分患者在无特殊情况下,能适应某些简单工作,因而在一般人群中难以被识别。轻度患者约占全部患者的85\%以上,而中度、重度和极重度患者分别为10\%、3\%~4\%和1\%~2\%,而大多数极重度患儿常因合并多种严重躯体疾病或是照顾不当而过早夭折。

受智力水平影响,中度和重度精神发育迟滞患者对性知识的了解程度有限,对月经、射精与受孕、性冲动与性高潮、爱情与婚姻、性行为的后果等都懵懵懂懂,不懂得如何避孕,当有性冲动时,就会有本能的性行为。部分精神发育患者性行为不够检点,缺乏道德感和社会责任感,其性行为常带有原始性,缺乏性前戏、性技巧,性行为也有愚蠢笨拙的特点。另外,对性行为实质性辨认能力,如什么是合法性行为、非婚性行为对自己在社会声誉上的影响、性不可侵犯性的法律规定等的了解及程度有限,继而也会影响患者的性活动。由于智能水平缺陷,女性性自卫能力有不同程度的减退,常常是受到性侵犯的对象。在受到性侵犯时,她们往往不知呼救、逃跑或有效反抗,受侵犯后不会主动告诉亲友或是报案,受性侵犯后的躯体及心理损害更加明显,往往在意外受孕后才被家人发现,故必须关注女性精神发育迟滞者的性保护,包括立法保护。男性患者常常很难有固定伴侣,有可能出现恋童或兽奸,有可能当众手淫、暴露生殖器、侵犯女性,而且还可能成为他人利用、攻击的对象,因而也必须加强保护、管理。

由于精神发育迟滞患者的性问题与其智能情况以及社会适应能力、性知识、辨认力和自我保护能力等密切相关,因而,治疗方面既要尽力改善或延缓其智能缺陷,又要注意管理和教育。关于精神发育迟滞本身,可以考虑如下治疗手段:

(1)病因治疗:对于遗传代谢性疾病,如苯丙酮尿症、半乳糖血症、枫糖尿症、肝豆状核变性等,如果能够早期诊断,应及早进行饮食治疗,可以避免发生严重的智能障碍。某些先天性颅脑畸形,如先天性脑积水、狭颅症,应进行手术治疗,这样可以减轻大脑压迫,有助于患儿的智力发育。以上疾病只占精神发育迟滞的少数,多数患儿不能进行病因治疗。

(2)药物治疗:多年来,医学界尝试过很多药物治疗,企图帮助MR儿童的脑发育,增强智力,如谷氨酸、脑磷脂等,但都没有肯定的效果。对于脆性x综合征患儿,可以采用叶酸治疗;对于先天性克汀病,应给予甲状腺素治疗。但绝大多数病因不明或无法对因治疗的MR还是无有效的特异性药物,

(3)脑移植:最近十多年来,国外有用脑移植来治疗MR。近几年来,国内也有少数地方开展此治疗,但疗效尚待评定。

(4)基因治疗:对于一些单基因遗传性、代谢性疾病,国外已在开展基因治疗,理论上应有前景。

(5)教育与训练:因为目前绝大多数MR没有可治愈的办法,但是通过特殊教育和训练,可以在一定程度上改善结局。对于大量智商(IQ)为50~70的精神发育迟滞的患儿,随着年龄的增长,其脑功能也有缓慢的改善,所以特殊教育及耐心辅导尤为重要。通过提高其社会适应能力、性知识、辨认力和自我保护能力,减少性问题的出现,以及降低遭受性侵犯后的不良后果。比如帮助其了解性知识,告知性交与受孕的关系,指导避孕知识,教患者了解什么是合法性行为、非婚性行为对自己声誉上的不良影响,学会在受性侵犯时要呼救、逃跑,受性侵犯后要告诉亲友。另外,患者监护人要尽职尽责行使监护职责,减少患者受到性侵犯的可能,并及时处理其发生的性问题,处理不幸遭受性侵犯后的生理、心理问题及法律问题。

防患于未然对于精神发育迟滞者的性问题来说,尤为贴切。加强遗传咨询、禁止近亲结婚、加强孕期保健、提倡优生优育,对预防精神发育迟滞具有重要意义。

1.沈渔邨.精神病学.第5版.北京:人民卫生出版社,2009

2.杨春霞,刘力克,张思霖,等.不明原因的精神发育迟滞的多因素分析.现代预防医学,2007,27(3):308

3.世界卫生组织.ICD-10精神与行为障碍分类临床描述与诊断要点.北京:人民卫生出版社,1994

(邸晓兰 过斌 王宁)


\section{第七节 残疾人的性问题}

残疾人是社会中平等的一员,他们应与健全人一样享有全面参与社会生活的权利,同时共享由社会发展所带来的物质文化成果。既然性是人类与生俱来的权利,在性的生理和心理支配下的性问题是客观需要,不能回避的。传统的观念往往否认残疾人还会有性问题,其实残疾人也必然有正常的性要求,于是健全人的各种性问题也必然会在残疾人中有所反映,加之他们有不同于健全人的生理和心理方面的变化,甚至在感情方面的需求更为强烈,因此在性问题上所遇到的困难也就更为突出和复杂。鉴于这个缘由,应当重视和关心残疾人的性问题。

残疾人的性问题与健全人相比,反映在临床上的性要求和性功能障碍的表现有下列特点:

1.由于躯体的伤残有明显与不明显、进展性与非进展性、青春期前与青春期后的不同,在性问题上的表现则不尽一致。一般来讲,伤残的明显程度与性功能障碍程度并非成正比,例如有些伤残明显者不一定有性功能障碍,反而伤残不明显者(肠造瘘、心脏病、盆腔手术后)可以有严重的性功能障碍。对于青春期前致残而无性经验者,往往会影响到他的性态度和对性的要求。

2.伤残后导致自身生理和心理的异常变化所造成的性功能障碍,尤以男性为突出,但是女性的性问题也并非少见。例如一位青年女性主诉,“我是一个善良又富有感情的人,渴望得到爱情和幸福,可是在下肢伤残后没有一个男子爱我,我可以参加许多社会公益活动,也有一定的知名度,他们只知道利用我来达到他们各自的目的,可是从来没有人问过我感情和性的需求,这些话我能够和谁说说呢?”

3.夫妻中一方因伤残导致了性功能障碍,需要正确认识,进行适当的治疗,否则会出现双方的性问题。例如一位妇女截瘫后不仅本人情绪不好,其丈夫由于看到妻子整日躺在床上忧心忡忡的样子,精神心理上也受到影响而导致性欲减低。

显然,残疾人的性问题不仅是医学问题,也与社会相关,包括婚姻、家庭、道德观念等。为此,在诊治残疾人性功能障碍之前,必须从多方面了解情况,诸如:①一般状况,包括性别、年龄、文化程度、职业和婚姻状况等;②伤残原因;③伤残时间,包括先天的还是后天的,青春期前还是青春期后,恋爱之前还是恋爱之后,有性经验之前还是之后;④伤残后造成的活动功能障碍的程度,即肢体行动能力如何、感觉器官如何;⑤伤残后的康复治疗情况以及心理状况如何;⑥有无其他疾病;⑦有关的性经验;⑧社会背景以及家庭和社会的态度。临床医生在较为详尽地了解了上述情况后才能客观地对其性问题进行判断,并提出诊断和治疗意见。

残疾人的性功能障碍临床表现为:男子的性欲降低、阳痿、早泄、不射精等。在女子可以表现为性欲低下、性唤起障碍、阴道痉挛和性高潮障碍。从而出现性交困难、性交疼痛、性感缺失等各种问题。此外,残疾人的性功能障碍还有下列特点:

1.先天发育畸形或损伤造成的外生殖器结构功能异常 在男性最常见的是脊髓损伤的截瘫患者,阴茎先天发育畸形和骨盆、会阴部严重损伤造成的阴茎勃起功能障碍,由于支配阴茎勃起的神经系统和血循环系统的病变导致不同程度的阳痿,此种为器质性阳痿。在女性最常见为性交困难。

2.外生殖器结构正常的盲、聋哑、肢残者由于精神心理上的创伤所导致的精神抑郁和焦虑,从而产生大脑皮质功能紊乱性的神经衰弱症状,男性表现为性欲低下、阴茎勃起不满意、不能维持正常性交,称之为功能性阳痿。女子主要表现为性欲低下。

3.先天性智力发育异常者的性功能障碍 只占智力低下者的一小部分,往往由于习惯性呆板行为造成性行为的异常。

4.医源性性功能障碍 由于外科或内科疾病造成的外生殖器结构功能异常导致的性功能障碍,在临床较为常见,例如内分泌疾病(糖尿病、甲状腺疾病、肾上腺疾病、垂体疾病、性腺功能低下等),慢性疾病(慢性肾衰竭、神经疾患、心血管疾病等),骨关节疾病导致的性活动受限,癌症病变以及在手术治疗、化疗和放疗过程中造成的影响,甚至癌症给患者带来的忧虑心理均可导致性功能障碍。

5.药源性性功能障碍 临床上有许多药物对性功能产生很强的抑制作用,无论是男性还是女性,长期、大剂量使用这些药物后,都可造成不可恢复的性功能障碍。

脊髓损伤分为闭合性和开放性两种。由于受损伤的部位和严重程度不同,损伤可发生在颈髓、胸髓、腰髓和骶尾髓,如果损伤平面以下的躯体感觉和运动能力全部丧失称为完全性脊髓损伤,亦叫做截瘫;如果损伤平面以下仍保存有一定的功能,则为不完全脊髓损伤。根据损伤平面及其病理变化,可将导致性功能障碍的临床表现分为下列几种:

1.脊髓休克阶段的性功能障碍 脊髓受损伤,由于脊神经传导和反射功能受到暂时抑制而产生暂时性的脊髓功能丧失,历时2~4周,在此期间,男性的阴茎勃起和射精功能全部丧失。有些患者也可能出现阴茎异常勃起,这是由于血管麻痹性扩张所致的阴茎静脉淤血。在脊髓休克阶段的最初几周内,尚难确定患者的性功能障碍有无恢复的可能,一般在6个月或更久时间后仍无恢复,则认为这种完全性的脊髓损伤将是永久性的,性功能障碍亦不可恢复。

2.颈髓、胸髓、腰髓平面的上运动神经元损伤的男性患者,可以有反射性阴茎勃起,即刺激生殖器和周围皮肤,通过生殖器与骶髓之间的神经反射出现勃起;而缺乏精神性阴茎勃起,即由意识上的刺激受大脑皮质控制的主动勃起,不是直接的生理刺激所引起的勃起。此种患者仍具有正常的肛门反射(刺激会阴部出现肛门括约肌收缩)和龟头至球海绵体反射(刺激龟头引起阴茎球海绵体肌收缩)。据报道,不完全性上运动神经元损伤的患者,出现反射性勃起的为99\%,其中约有1/3的患者具有射精能力;完全性上运动神经元损伤的患者,出现反射性勃起的为91\%,其中仅有4\%的患者还能够射精。

3.骶髓和马尾神经平面的下运动神经元损伤的男性患者,缺乏反射性阴茎勃起,只有精神性阴茎勃起,此种患者的肛门反射和龟头至球海绵体反射亦消失。据报道,约有90\%的不完全性下运动神经元损伤的患者可以出现阴茎勃起,其中70\%可以射精;约有26\%的完全性下运动神经元损伤的患者可以出现阴茎勃起,其中有18\%可以射精。

4.T12 至S2 平面的脊髓损伤的男性患者可以有精神性与反射性混合的阴茎勃起。

5.T4 或T5 平面脊髓损伤的男性患者,在性冲动时可能有自主神经系统(植物神经系统)过度活动的表现和性欲增强的感觉,其机制尚不清楚。

6.性交能力的评价。由于反射性勃起的持续时间很短暂,而且也缺少生理上的快感,精神性勃起的持续时间虽较反射性勃起长,但是勃起总是不完全的,所以脊髓损伤的男性患者在性交能力方面并非完全如愿。据报道,在不完全性下运动神经元损伤后阴茎能勃起的患者中有80\%可以进行性交,其中70\%可以射精。下列依次递减:不完全上运动神经元损伤者为63\%,完全性上运动神经元损伤者为53\%,其中只有1\%可以射精,完全性下运动神经元损伤者仅为23\%。尽管如此,对于大多数脊髓损伤的男患者,即使经过相当水平的实践,也仅有15\%~25\%可以获得满意的性交。关于脊髓损伤后女患者的性功能障碍问题,至今研究的甚少,已知女患者在脊髓损伤后可以有规律的排卵和月经周期,也可以受孕并妊娠。临床观察性兴奋时阴道渗出液和盆腔充血程度减少,部分人可以丧失获得性高潮的能力,但是对性生活则无多大影响。

视觉对于性感觉和性行为十分重要,无论是先天失明还是后天失明,如果缺乏来自眼睛的感觉冲动,则可影响性活动。在青春期以前发生失明的妇女,有半数会出现月经紊乱,但是性欲并不受影响。一些失明的男性因为精神抑郁而减少或停止了性生活。先天性失明合并其他发育异常时,由于接触社会受到影响,可能使性关系的发展受到限制,同时双亲的态度和教育方式的不妥当,也会对其带来不良影响。成年人失明后由于抑郁、自尊心受损害、孤独感和社会的冷遇等,均可因精神心理作用影响性活动。所以盲人的性功能异常,除极少数原发性的器质性的疾病(如糖尿病、多发性神经硬化症或脑肿瘤),不仅造成失明、损伤性器官功能外,大多数是精神心理因素引起的功能性性功能异常。

聋哑一般并不影响性欲或性生理活动,然而,许多聋哑人由于缺乏必要的性知识和社交活动而造成婚姻和性生活方面的问题。尽管他们没有躯体上的残疾,但是往往因个人的孤独与抑郁而发生精神心理性的性功能异常。

残疾人出现了性功能障碍,能否进行性生活,不能一概而论。应当对性功能障碍的性质(是器质性障碍还是精神心理性障碍)进行正确的估价。例如,估价男性阳痿的性质,可依据凡是阻断影响阴茎海绵体充血膨胀的因素,均可导致器质性阳痿。诸如截瘫患者由于脊髓损伤使阴茎勃起神经中枢受到损害,严重骨盆骨折,会阴部尿道损伤由于支配阴茎的神经和血管受到损害均使阴茎的勃起功能遭受破坏而发生器质性阳痿。这些一般不能进行正常的性生活,需要予以特殊治疗。关于功能性阳痿,目前仍认为是由于大脑皮质功能紊乱导致的性功能抑制作用的增强所致。其特点是:在不性交时可以有正常的阴茎勃起,在睡眠时也可勃起,甚至在接受直接、间接性刺激时亦可勃起,唯有在要性交时或开始性交时勃起则消失。最常见的精神心理诱发因素是:来自本身致残后产生的疑虑、恐惧和抑郁,通常怀疑自己的性器官有异常,怕影响夫妻生活;再有是来自对方的不理解,厌倦和各种形式的压力,使患者产生自悲、失望,无信心维持正常的性生活。这是残疾人比健全人更为突出的精神心理障碍。经过咨询、耐心疏导和必要的精神心理治疗,此种患者有可能恢复正常的性功能。为此,在进行咨询时应让患者正确理解下列几点:①泌尿道的病变导致了小便失禁,并不意味着性器官的无能;②缺乏性知识并不意味着缺乏性的感受;③生理上的畸形并不意味着性欲有缺陷;④无能力性交并不意味着无能力得到性享受;⑤失去了生殖器并不意味着失去了性。正确认识、理解上述概念,有助于患者解除其精神心理障碍。

1.患有器质性阳痿的残疾男性,并非全无治疗的可能,其中勃起神经中枢尚未受损,血循环供给正常者仍有治疗的机会。可以通过服用营养、兴奋神经的药物,针灸,电刺激等提高勃起中枢和勃起神经的兴奋性,恢复阴茎勃起功能。

2.阴茎勃起恢复后,若勃起不满意,甚至性欲低下者,仍可辅以男性激素(睾酮类药物)治疗。虽然男性激素并非直接作用于性勃起中枢导致兴奋,但是它可以提高性勃起中枢对性刺激的反应能力,更容易对外界的性刺激产生性兴奋。

3.对于无法恢复阴茎勃起功能者,可以通过手术治疗,在阴茎海绵体内植入阴茎假体,达到支撑阴茎勃起作用,能够进行性交。

人们常常认为残疾人不会像健全人那样具有性兴趣,甚至根本没有行使性功能的能力。实际并非如此,一项临床调查表明:在中风存活者中,虽然躯体运动能力尚未完全恢复,但73\%的妇女和88\%的男性仍有性欲,46\%的男性仍能勃起。中风者如此,更何况那些血气方刚的年轻的残疾人呢。对残疾人性问题的另一种错误观念是,特别容易把与慢性疾病并存的性功能障碍归咎于残疾,推测为器质性问题,其实这一推测是不合理的。残疾人可能面临更多的性问题,虽然性问题可能与残疾状况有关,但性障碍往往是由更多的其他因素造成的。这些因素有对性表现能力的焦虑、对性解剖和性生理知识的缺乏、存在错误的性观念、担忧达不到男性化或女性化的理想标准、难于与伴侣就性问题进行交流等,这与健全人群存在的问题并无区别。残疾人自然还会有自己独特的一些问题,例如残疾带来的社会适应与心理因素的障碍往往具有破坏性作用。因为人的性唤起和性器官的生理反应本来就对来自大脑的消极心理的抑制作用十分敏感或脆弱,对于残疾患者来说,经历灾难性的疾病打击之后出现的多种焦虑就更会雪上加霜,进一步阻断或完全抑制性唤起和性反应过程。如对性表现能力衰退的焦虑、对丧失工作能力和职业的焦虑、对体态外貌改变和生活自理能力的焦虑,甚至对疾病复发造成死亡的焦虑等,都会引起性唤起和性反应过程的异常。残疾人性心理还会因伴侣的感受和反应,以及一些特殊的畏惧而进一步复杂化。如伴侣厌倦或不愿意与残疾人同床共眠,怕承担性生活时发生意外的责任,或害怕自己也染上疾病(如有人竟担心肿瘤会传染),对尿失禁或大便失禁反感等。家属和亲朋的不良影响会直接加重残疾患者的性障碍。如认为残疾人有性要求是“不正常的”、“过分的”,他们便用“防止疾病复发”或“要注意保养”等借口暗示甚至要求残疾患者放弃性活动。于是,有的残疾人本身也以这种借口来结束自己在性表现方面承受的压力。总之,在残疾人面临的性问题中很少是完全因疾病或残疾本身引起的,更多的是与自己或他人的消极和传统的观念有关,与上述种种偏见的影响有关。既然性问题大量存在于残疾人与他们的伴侣之间,因此,承认、正视并积极处理好这类问题是十分重要的。医生在处理这类问题时,要详细采集患者的病史和性发育史,并进行认真全面的体检,包括生殖器和有关神经支配的功能状况,之后就可以判定性主诉与躯体因素或社会心理因素在多大程度上有关,然后策划出一个有效的康复或治疗计划。各种行之有效的治疗机体健全者性功能障碍的措施在残疾人中同样适用,应该鼓励在残疾人整体康复计划中综合利用现代的性教育、性咨询和性治疗技术。假如一个患者不能正常行走,那么人们可以给他提供拐杖、轮椅或其他助行工具,并指导他们如何利用这些新工具以尽可能发挥它们的最大功效。同样,当生殖器的性反应受到损伤时,治疗原则也应始终集中在发挥其潜在的功能上,而不是恢复它们所失去的能力。还应该强调,并使患者夫妻认识到,在表达爱情和性欲的技术方面是没有什么正确与错误或正常与变态的区别的。残疾人应尽可能利用力所能及的方式来表达自己的爱情和性欲。例如,当以男上位完成性交有很大困难或完全不可能时,他并不应该就此放弃所有的性体验。他们可以采用女上位或其他可能的性技巧,这就像一个人不能独立行走就连拐杖和轮椅也不愿利用一样,他们试图树立或维持一个“完美”的自我形象,这种做法是不科学的,也是不近情理的。现实地对待性问题,调动自己身体的一切潜能是十分重要的。当你摆脱了那些陈腐的偏见,性享乐仍然会经常光顾你的。可以预计,在性领域中若能尽早采取性康复计划、给予早期干预,必然有助于最大限度地恢复各类患者的性功能状况,并把继发的后果减少到最低程度,这与运动员伤残后的康复原则是一致的。无论是在综合性的康复中心或是在独立的性治疗诊所,对残疾人性问题的治疗原则应该是一致的。

1.强调性是一个很广义的概念,性爱决非仅仅局限于性交本身,性活动和性行为的方式很多,残疾人可以根据自己的身体条件找到最适合自己的有效方式,身体力行,以获得最大限度的性满足。

2.强调生殖器完全或部分失去反应并不意味着性生活的结束。需要指出,截瘫或四肢瘫痪的男性患者如经一定训练仍能达到相当水平的性反应,当然有些人的性高潮不再是以射精为标志。因为性高潮实际上是生理上和心理上的整体反应,并非仅仅局限于生殖器部位,甚至没有生殖器的参与也能发生。在性反应研究中可以见到残疾人所表现出的心血管、呼吸和神经肌肉等系统的一系列性高潮反应,虽然他们并未发生射精这一生理反应,但身体其他部位确实出现了高潮反应。从另一方面看,当患者没有出现事实上的性高潮生理表现时,只要经过适当的训练,如“感觉放大技术”,即想象一种生理刺激和反应,加以集中和放大,使这种感觉达到十分强烈的程度,这样可以促进性高潮的到来。经过训练并取得一定成效后,尽管性交中未出现射精这样的典型性高潮反应,但他们也常常能获得性高潮的体验。研究还发现,有些女性在脊髓损伤截瘫后(第12胸椎完全性下运动神经元损伤),虽然失去了所有的骨盆感觉和性刺激带来的明显的骨盆充血与阴道润滑作用,然而患者的乳房对性刺激的敏感性代偿性增强了。因此在脊髓损伤数月后,她们可以通过乳房刺激达到性高潮。刺激乳房可使患者很快出现心肺等的正常性生理反应和乳房、乳头的生理变化,同时还发现在平台期末出现口唇充血、肿胀等不寻常的生理反应。这说明性敏感区或曰动情区具有从身体某特定部位向其他部位转移的可能性,而且性生理反应也能从原有部位转移到另一在生理上具有类似功能特点的部位。例如,转移到与各种开口有关的区域像口、眼周、耳周、乳头周、鼻周、脐周等。

3.强调集中精力挖掘身体的潜力,而不是集中在恢复受到损伤部分的功能,这一难度往往较大。

4.调节临床使用药物的剂量和给药时间,以便在性生活时提供对疾病的最佳控制。

5.提供有关其病情和预后的准确信息,允许患者作出现实主义的期望,打消种种顾虑。当医生作出诊断或安排某种治疗措施时,事先往往会给患者安排一些疏导工作和帮助。当然,患者自己全身心投入性康复过程是最重要的先决条件,他们若不最大限度地调动自己的积极性,光靠外界的支持帮助仍很难奏效。

6.介绍使用适当的性工具。所谓性工具是指能够帮助男女克服他们的性困难,而能像健全人那样完成性反应的一些器具,但它们也不是万能的,需要由医生来帮助选择,而且它们只能起到拐杖、轮椅的作用,并不能根治原有的性问题。由于残疾人的婚姻问题较困难,不少残疾人过着独身生活,对他们来讲,对性工具的需求就更迫切了。性工具分男用和女用两大类。供男用的有假阴道,假阴道内含有振荡器或蠕动装置,对阴茎可起到一定的按摩、刺激作用,既有娱乐功能,还对阳痿患者有一定的康复功能。现代的假阴道由高级塑胶制成,做工精美,有了功能齐全的假阴道,男性残疾人能够获得充分的性满足。女用振荡器或女用性功能康复器有多种形式:一是做成阴茎状的,有的逼真,有的形似,一般都带有振荡刺激阴蒂的小叉;二是G点棒,像香蕉状;三是蝴蝶状的扁平装置,仅供刺激外阴用;四是多功能的三叉状,可以同时刺激阴蒂、G点和肛门;其他五花八门的工具。这几类性工具所用的电源既可以是直流电,也可以是交流电,若是交流电的,应注意不要在淋浴或洗澡时使用,以防因触电而发生事故。由于振荡器可以提供80次/秒的高频率振荡刺激,所以它们可以不断增强女性的性兴奋和性反应。有些女用振荡器也可供男性使用。女用器具目前多具有振荡、蠕动和旋转等功能。最后要提醒的是,性工具毕竟只是一种机械装置,它代替不了人的感情投入和需求,所以建立美满的姻缘,发挥婚内性潜力仍然十分重要。对于难觅伴侣者,性工具则是不得已而求之的一种替代选择。另外,目前性工具的销售专卖店很多,各式各样的产品也很多,残疾人在选购时要选择有产地、厂家名称和生产批号的产品,并要选择售后服务好的销售单位。

人们早就注意到严重创伤后会导致受伤者的行为异常,然而只是近些年人们才注意到这一现象的非器质性后果。值得注意的是创伤后紧张综合征也会造成性功能障碍,而这一问题却很少受到大家的注意。无论是地震(如汶川、玉树)、车祸(频繁发生)、罪犯袭击、火灾(上海)、矿难(频繁出现)或洪水等自然的或人为的灾害,都会使人们突然发生创伤后紧张综合征,即使患者的身体完好无损也不一定能幸免。当然,若是存在持久的身体损害或症状时,问题定会更加严重。创伤后紧张综合征患者的主要临床特征为:①紧张:患者忧郁,紧张,不安,神经过敏,易受惊吓和恐惧;②过分专注于创伤:患者总是谈论事故经过,常常推测会发生更严重的伤害,甚至死亡;③夸大疼痛或躯体的不适;④失眠及其引起的疲倦;⑤不断在倒叙回忆或梦魇中重新体验伤害的打击,并存在与事故发生时相似的心理情感反应,于是加重心理创伤;⑥表现能力退步:患者不能进行正常的生活活动,如工作、学习、家庭责任心、娱乐及创伤前所喜爱的种种活动;⑦恐惧:一朝被蛇咬,十年怕井绳就是典型的写照,患者会回避创伤发生的环境与地点;⑧个性改变:患者变得伤感、易发火、退缩、健忘、心烦意乱、自卑;⑨爱发脾气:怒火冲天,动辄与人争吵,毫不掩饰内心的不满,丝毫不顾及他人的反应,患者也会变得沮丧、忧郁、失去自信、悲痛、悲观失望。

患有创伤后紧张综合征的患者往往乐于详细讲述他们受伤害时的情节,而不愿意诉说他们的性问题,如男性的性欲低下和偶尔的阳痿、女性的性感缺失和性高潮障碍。当医生向患者询问有关亲昵和性行为方面的问题时,常常可以发现以前一直为夫妻双方所忽略的问题。例如,一位男性在目睹好朋友死于事故的全过程后一直极度紧张,备受失眠、噩梦、噪声的折磨,整天头昏眼花,摆脱不了死亡和受害感。他总是不停地倒叙当时的险境和自己无力解救朋友的懊悔,反复回顾那血淋淋的细节。由于强烈的焦虑,他在事故发生3周之后仍不能上班,在妻子的催促下他接受了精神科医生的检查,确诊为创伤后紧张综合征。经过进行性肌肉放松的行为疗法和镇静剂药物治疗后,他更换了工作单位,病情逐渐得到好转,医生告诉他,总是思考和谈论事故实际上是不断地重复创伤过程,这将严重影响他的康复,所以换个环境是有好处的。开始时虽然仍会造成一定焦虑,但新的环境和活动会加快他对恶性刺激的脱敏过程,并最终使他消除焦虑。然而他的问题并未完全解决,因为仍存在严重的婚姻危机。这位患者还主诉性欲低下,勃起次数和时间减少,妻子对此采取讽刺挖苦的做法,并怀疑他另有所爱。医生向她解释这些问题很有可能是创伤后综合征的表现之一,所以她应该积极帮助丈夫度过康复期。丈夫也再次向妻子表达了爱意,并诚恳地告诉她,他容易发火和生气并不是针对她,而是冲着他朋友的粗心大意和马虎造成的悲剧。他的妻子释然了,马上愉快地表现出理解和体谅。在医生的帮助下,他们的婚姻关系和性关系很快恢复正常。另一位女性在值夜班时遇到歹徒袭击和猥亵,虽然没有受到过于严重的创伤,但她在好长一段时间内不能自控,脑海中总是反复出现暴力事件场面,包括歹徒狰狞的面孔和威胁她的手枪,她无法入睡,整天昏昏沉沉的,打不起精神,对歹徒欺负她这样的弱女子的暴行感到极度气愤。她害怕袭击再次发生,工作时非常紧张,看见每个顾客都要打个问号,同时伴有头痛、胃痛,负伤和死亡念头总是萦绕在心,种种干扰削弱了她的工作能力。经过一段休假情况并未好转。除此之外,这位妇女还有意疏远丈夫,她对性的兴趣几乎完全丧失,她拒绝一切性活动,甚至讨厌一般的肉体接触如亲吻。因为她在这种情形下总会想起在事件中受到的侮辱,她痛恨自己的软弱无力并存在深深的羞愧感。医生向这对伴侣指出,创伤确实给这位妇女带来严重的身心创伤,由于不断回想痛苦的经历,又激起她的焦虑和气愤,这种情绪必将抑制和抵消性的欢愉。医生给他们一些轻松的录音磁带,让他们边听音乐边松弛肌肉,然后进行性感集中训练,逐步恢复对亲昵和性的积极感受。经过几个月的调整,她重返工作岗位,夫妻关系和性关系也恢复到原有水平。许多越战、伊战归来的美国老兵总不能从思想中抹去对越战、伊战的记忆,战争不仅使许多家庭解体,大屠杀和战争残酷的噩梦还常常使他们突然惊醒,只有酗酒能给他们带来暂时的解脱。然而,创伤的瘢痕和酗酒的影响使他们的人际关系恶化,使他们的性欲低下,性功能衰退,使他们的婚姻关系失败。医生很清楚他们酗酒的目的是对抗其焦虑,可是患者并不知道焦虑和酗酒都干扰了他的性功能。因此,解决这类问题时要综合处理,建议戒酒,服用抗焦虑药物,学习控制思想的行为调整技术,指导他们进行肌肉松弛练习和性感集中训练。这样往往可以奏效。有位男子在车祸中造成手腕、胸骨和颅骨多处骨折后昏迷了两天,虽然他恢复得很好,没有遗留任何神经病学方面的问题,但他在受伤后就丧失了勃起能力,对性的兴趣也消失了。虽然身心的任何伤害都有可能造成性功能障碍,但头部受伤后继发性问题者尚少见。有人统计了66名在车祸中头部受伤男性的数据,他们在事故前均无性功能障碍,然而有52\%的人在事故后出现性问题:性欲有显著变化的占2/3,射精障碍占1/5。经临床内分泌检查发现,性功能完全丧失或严重障碍者往往存在下丘脑-垂体功能的损伤。在处理这类问题时,除了需要排除内分泌因素外,尚应全面采集病史,了解他们在事故前的性生活质量,此外,还须了解是否存在创伤后紧张综合征等精神心理方面的问题。短期使用阴茎海绵体内罂粟碱等血管活性药物注射的方法,可以帮助患者战胜其心理障碍。俗话说:“大难不死,必有后福”,然而情况并非总是如此,若忽视了上述创伤后紧张综合征的存在及其对性功能的影响,那么他们的生活往往要遇到麻烦。在意外后的早期应注意他们的躯体健康问题,但急性影响一旦过去了,就要注意他们的婚姻与性问题。保持对这一综合征的高度警惕,发现问题后及时采取各种积极措施往往有利于患者的康复。若发现病情复杂,可转诊给精神科医生诊断处理。

(马晓年)


\chapter{第十五章 男女生殖系统先天性畸形}

男性生殖器分为内、外生殖器两部分,内生殖器有睾丸、输精管道及附属腺。睾丸是男性的生殖腺,是产生精子的器官,也是男性性腺,具有分泌男性激素、刺激男性性征发育并维持第二性征的功能。输精管道包括附睾、输精管和射精管等。附睾有暂时储存精子并使之进一步发育成熟的作用。精子通过上述管道后,再经尿道排出体外。附属腺有精囊腺、前列腺和尿道球腺。这些腺体的分泌物参与精液的组成,对精子具有营养和增强其活力的作用。外生殖器有阴阜、阴茎和阴囊。

人类性分化需经过三个连续的阶段:遗传性别或者染色体性别、性腺性别和表型性别。遗传性别是母亲的X染色体配子和父亲的X或Y配子结合形成XX或XY合子。遗传性别指导原始性腺分化成卵巢或睾丸,这就是性腺性别。表型性别是最后一个重要环节。人类胚胎的生殖系统除性腺外还包括生殖管和外阴部,它们的分化成为两性的表型。生殖系统在胚胎期都需经过一个男女不分的未分化期。当原始性腺分化为睾丸后,它分泌睾酮和苗勒管抑制物质(muellerian inhibiting substance,MIS),分别作用于中肾管(Wolff管)和苗勒管(副中肾管,Mueller管)这两对生殖管。睾酮促使中肾管分化成输精管、附睾和精囊;MIS使苗勒管退化消失。卵巢缺乏睾酮的作用,在胚胎期也不分泌MIS,因此中肾管退化、消失,苗勒管则分化成子宫、输卵管和上部阴道。在男性,由于睾酮的还原产物双氢睾酮的作用,原始会阴和尿生殖窦一起分化成阴茎和阴囊,在女性,由于没有双氢睾酮的作用,分化成下部阴道和前庭。

从性别表型的发生机制可想象睾丸和卵巢所起的作用不是等同的。男性表型的分化完全是睾丸的作用,而女性的分化和卵巢无关,因为胚胎的生殖管和外阴具有自发地向女性分化的能力。从细胞遗传角度上观察睾丸的重要功能,认为是通过Y染色体来实现的。很早就推测Y上存在一个睾丸决定因子(testis-determining factor,TDF),促使原始性腺分化成睾丸。未分化性腺发育成睾丸的第一迹象就是Sertoli细胞。他们分裂并繁殖,聚集在性细胞周围形成原始精细管。如果在关键时期没有Sertoli细胞,那么性细胞继续分裂一段时间,逐渐发育成为卵巢。性细胞对形成睾丸或者卵巢没有什么作用,但对保持卵巢的发育起了重要作用(Berkovitz,1991)。随着分子遗传学的迅猛发展,20世纪90年代初在Y的短臂上发现了一个决定睾丸发育的基因,其表达使原始性腺分化成睾丸。

早在20世纪50年代末就认定人和小鼠的Y决定个体是男(雄)性。但至1966年方开始把TDF定位在Y的短臂上。然而找寻TDF的正确座位的经历却很曲折,其间曾把组织相容性Y抗原(histocompatibility Y antigen,H-Y抗原)和Y的锌指蛋白(zinc-finger containing region on Y,ZFY)错认为TDF。直至1990年Sinclair等在其他学者成果的基础上深入研究,发现Y短臂的IA1亚区,在与假常染色体区相邻的35Kb内存在着一个睾丸分化基因,命名为SRY(小鼠的同样基因为Sry)(图15-1)。

图15-1 寻找人类性别决定基因———睾丸决定因子TDF的历程

(黑色区域为TDF所在的位置)

MIS又名抗苗勒管激素(anti-muellerian hormone,AMH)。Jost早就观察到牛的雌雄同体中的雌胎没有苗勒管的衍化物,并影响卵巢的发育:卵巢内细胞消失,出现精细管样结构,这种现象称为生殖器不全牛牝犊现象(freemartinism),是由于雄胎睾丸产生的一种抑制苗勒管分化的物质通过血液经由共用的胎盘作用于雌胎的生殖系统(图15-2)。

图15-2 MIS是睾丸间质细胞前体增殖和分化的负向调节因子,也是成人间质细胞雄性激素生物合成的抑制剂。(来源:麻省理工学院医学院网站)

MIS由胎儿睾丸的Sertoli细胞分泌,而胎儿卵巢并不分泌,卵巢的颗粒细胞在出生后就开始分泌MIS,直至绝经期。刚生下的男婴血清MIS值很低,但很快攀升,在生下3~4天后增加4倍。1~4岁时血清最高值达10~70ng/ml,以后逐渐下降,至青春期为1~5ng/ml。女婴在生下1年内MIS值始终很低,在0.5ng/ml或更低,在儿童期开始上升,直至青春期MIS值为1~5ng/ml,和男性接近。

MIS的作用除苗勒管退化外,还有助于生殖细胞发育、性器官的形成、肺的发育,同时对睾丸下降也有影响。

生殖腺来自体腔上皮、上皮下方的间充质及原始生殖细胞三个不同的部分。

胚胎第5周时,在尿生殖嵴的内侧出现体腔上皮的增殖区,称作表面上皮或“生殖”上皮,是生殖腺发生的开始。人胚第7周时,由于生殖上皮细胞及其下的间充质细胞的增殖,在左右中肾的内侧分别形成一隆起,称作生殖腺嵴。未分化的生殖腺由两部分组成:表层的皮质和深层的髓质。左、右中肾嵴内侧的表面上皮下方间充质细胞增殖,形成一对纵行的生殖腺嵴。不久,生殖腺嵴的表面上皮向其下方的间充质生出许多不规则的细胞索,称初级性索(primary sex cord)。胚胎第4周时,位于卵黄囊后壁近尿囊处有许多源于内胚层的大圆形细胞,称原始生殖细胞(primordial germ cell)。它们于第6周经背侧肠系膜陆续向生殖腺嵴迁移,约在1周内完成,原始生殖细胞进入初级性索内(图15-3)。原始生殖细胞移到生殖腺后,在此分化为卵原细胞或精原细胞。

图15-3 原始生殖细胞及其迁移示意图

性染色体为XX的胚胎,其皮质通常分化为卵巢,而髓质退化。性染色体为XY的胚胎,其髓质通常分化为睾丸,而皮质退化。原始生殖腺有向卵巢方向分化的自然趋势。当原始生殖细胞及生殖腺嵴细胞膜表面均具有H-Y抗原时,原始生殖腺才向睾丸方向发育。一般情况下,性染色体为XY的体细胞胞膜上有H-Y抗原,对生殖腺向睾丸方向分化起决定性作用。目前认为,编码H-Y抗原的基因位于Y染色体的短臂近着丝点的部位。人胚第7周,在H-Y抗原的影响下,初级性索增殖,并与表面上皮分离,向生殖腺嵴深部生长,分化为细长弯曲的袢状生精小管,其末端相互连接形成睾丸网。第8周时,表面上皮下方的间充质形成一层白膜,白膜的发生是睾丸发生的特征,也是判断生殖腺为睾丸的指征。分布在生精小管之间的间充质细胞分化为睾丸间质细胞,并分泌雄激素。在人胚第14~18周,间质细胞占睾丸体积一半以上,随后数目迅即下降,出生后睾丸内几乎见不到间质细胞,直至青春期才重现。生精索发育成曲细精管、直细精管和睾丸网。胚胎时期的生精小管为实心细胞索,内含两类细胞,即由初级性索分化来的支持细胞和原始生殖细胞分化的精原细胞。生精小管的这种结构状态持续至青春期前(图15-4)。曲细精管的管壁由两种细胞组成:来自生殖上皮的Sertoli细胞和来自原始生殖细胞的精原细胞。在胎儿的睾丸中,曲细精管的大部分细胞是支持细胞。

生殖腺最初位于后腹壁的上方,在其尾侧有一条由中胚层形成的索状结构,称引带(gubernaculum),它的末端与阴唇阴囊隆起相连,随着胚体长大,引带相对缩短,导致生殖腺的下降。第3个月时,生殖腺已位于盆腔,卵巢即停留在骨盆缘稍下方,睾丸则继续下降,于第7~8个月时抵达阴囊。当睾丸下降通过腹股沟管时,腹膜形成鞘突包于睾丸的周围,随同睾丸进入阴囊,鞘突成为鞘膜腔。然后,鞘膜腔与腹膜腔之间的通道逐渐封闭(图15-5)。睾丸下降过程中出现的问题在临床很常见,了解睾丸的下降过程在临床上对隐睾的诊断治疗非常有意义。

图15-4 睾丸与卵巢的分化

图15-5 睾丸下降

人胚第6周时,男女两性胚胎都具有两套生殖管,即中肾管和中肾旁管(paramesonephric duct,又称Muller管)。中肾旁管由体腔上皮内陷卷折而成,上段位于中肾管的外侧,两者相互平行;中段弯向内侧,越过中肾管的腹面,到达中肾管的内侧;下段的左、右中肾管在中线合并。中肾旁管上端呈漏斗形,开口于腹腔,下端是盲端,突入尿生殖窦的后侧壁,在窦腔内形成一隆起,称窦结节(sinus tubercle,又称Muller结节)。中肾管开口于窦结节的两侧。

若生殖腺分化为睾丸,胎儿的睾丸至少产生两种激素:一种激素刺激中肾管,间质细胞分泌的雄激素促进中肾管发育,使其发育成男性生殖管道;另一种是支持细胞产生的抗中肾旁管激素,抑制中肾旁管的发育,使其逐渐分泌激素,则抑制副中肾管的发育。同时。雄激素促使与睾丸相邻的十几条中肾小管发育为附睾的输出小管,中肾管头端增长弯曲成附睾管,中段变直形成输精管,尾端成为射精管和精囊(图15-6)。

未分化期外生殖器也经历一段不能区别男女的时期。在第4周初,在泄殖腔膜的颅侧发生生殖结节(genital tubercle)。阴唇阴囊隆突(起)和尿生殖褶也随即发生于泄殖腔膜的两侧。尿生殖膜的两侧各有两条隆起,内侧的较小,为尿生殖褶(urogenital fold),外侧的较大,为阴唇阴囊隆起(labioscrotal swelling)。尿生殖褶之间的陷为尿道沟,沟底覆有尿生殖膜。第7周时,尿生殖膜破裂。不久,生殖结节伸长,称作初阴,它在男性和女性都一样大。第6周末尿直肠隔与泄殖腔膜相连接时,隔便将泄殖腔膜分为背侧的肛膜和腹侧的尿生殖膜。经一周或更长时间,这些膜破裂,分别形成肛门和尿生殖孔。在初阴腹侧面(下面)形成一条尿道沟,与尿生殖孔相通。

图15-6 男性生殖管道的演变

A示第3个月 B示睾丸下降以后

在睾丸产生的雄激素的作用下,促使外生殖器向男性发育。当初阴伸长形成阴茎时,它向前牵拉尿生殖褶。生殖结节伸长形成阴茎,两侧的尿生殖褶沿阴茎的腹侧面,从后向前合并成管,形成尿道海绵体部。尿生殖褶形成阴茎腹侧面(下面)的尿道沟的侧壁。尿道沟表面有一层内胚层,它是由尿生殖窦初阴部的内胚层延伸而来。左右阴唇阴囊隆起移向尾侧,并相互靠拢,在中线处愈合成阴囊(图15-7)。

图15-7 男、女外生殖器的发育


\section{第二节 男性生殖系统先天性畸形}

尿道缺如及尿道闭锁(urethral agenesis and atresia)极少见。尿道闭锁可呈完全性、部分性或膜状。尿道完全闭锁者,尿道呈一索状;部分闭锁者,多发生于阴茎头部或阴茎部尿道;膜状闭锁者,男性常发生在阴茎头部尿道外口或后尿道,女性多发生在尿道外口。

其真正原因不明,可能是胚胎时尿道的上皮组织未及时反折于尿道内、阴茎头部的尿道上皮发育受到障碍,也可能是尿生殖窦膜未能穿破,而形成膜样闭锁所致。此病男性多于女性。

由于这两种病使产前胎儿在宫内排出的尿液潴留于膀胱内,致使膀胱扩张,进而压迫脐动脉,引起胎儿循环障碍;或因尿道梗阻发生肾积水,肾脏萎缩,在胚胎或出生时多已死亡,常合并其他严重畸形。极少数存活者大部分胎尿有排出的路径,如膀胱外翻、脐尿管未闭、泄殖腔残存、膀胱直肠或阴道瘘、尿道直肠或阴道瘘等。有的出生时虽然存活,但不久亦死亡。

尿道闭锁的预后决定于闭锁部位,如为后尿道闭锁,与尿道缺如相同,多于产前或生后不久死亡。前尿道闭锁尤其靠近尿道外口者,上尿路受影响相对较轻。本病一经发现,应立即处理。尿道缺如者应立即行耻骨上膀胱造瘘。有肾衰竭者,需行双侧肾造瘘术。并发尿外渗者行尿外渗引流,并加用抗生素。待肾功能好转,小孩长大后,行尿道成形术及畸形矫正术。如为膜样闭锁,可用尿道探子将隔膜捅破,留置导尿。

尿道下裂(hypospadias)是由于生殖结节腹侧纵行的尿生殖沟自后向前闭合过程停止所致,部分病例并发阴茎下弯。尿道外口的远端、尿道与周围组织发育不全,形成纤维索牵扯阴茎,使阴茎弯向腹侧。先天性阴茎下弯者并不全有尿道下裂,但尿道下裂者都伴不同程度的阴茎下弯。尿道下裂是男性泌尿生殖系较常见的先天畸形,出生男婴中发病率可高达1/125~250。

近年来有关尿道下裂的病因学研究概括起来包括以下几个方面:

部分病例雄激素受体和5α-还原酶缺陷,也有发现在人绒毛膜促性腺激素(human chorionic gonadotropin,HCG)刺激后,尿道下裂患者的雄激素增高反应明显低于正常对照组人群,提示尿道下裂患者的下丘脑-垂体-性腺轴不正常。

研究发现在妊娠早期用过黄体酮保胎的新生儿中尿道下裂的发生率较高,同时有研究表明,尿道下裂患者的雌二醇和雌酮水平增加。这些研究提示雌性激素有拮抗雄激素作用。

在尿道下裂患者中的染色体畸变率较正常人群有明显增高,其中包括常染色体畸变及性染色体畸变。

发现尿道下裂患者可存在雄激素受体基因、性别决定基因、5α-还原酶基因、抗苗勒管激素基因、CYP21B基因的突变。

正常情况下,当胚胎第7周后尿道皱襞自尿道近端逐渐向龟头端融合成一管形,即尿道,这一过程有赖于胚胎性腺分泌的雄性激素,也取决于胚胎尿道沟及皱襞对睾酮的反应。当尿道皱襞形成管形发生障碍时即导致尿道下裂。另外,尿道开口处的间质组织不发育,形成一扇形的纤维索,围绕尿道外口并延伸和嵌入龟头。尿道下裂发病有明显的家族倾向,有报道8\%的患儿父亲及14\%患者兄弟中也有尿道下裂,可能与基因遗传有关。

典型的尿道下裂的特点:①尿道开口异常,尿道口可出现在正常尿道口近端至会阴部尿道的任何部位。部分尿道口有狭窄,其远端为尿道板。②阴茎下弯,即阴茎向腹侧弯曲。目前认为尿道下裂有明显阴茎下弯者只占35\%,而且往往是轻度下弯。③包皮异常分布。阴茎头腹侧包皮因未能在中线融合,故成V形缺损,包皮系带缺如,全部包皮转至阴茎头背侧呈帽状堆积。

临床上按尿道口位置不同,可将尿道下裂分为阴茎头型、阴茎型、阴茎阴囊型及会阴型等4型,其中以阴茎头型及阴茎型占多数(图15-8)。

图15-8 A示尿道下裂Barrcat分型,B示尿道下裂的类型

最常见,尿道口位于冠状沟腹侧,常呈裂隙状,有的可并发尿道狭窄,背侧包皮长,腹侧无包皮及系带。阴茎头裸露,较细小且稍扁宽,呈球状。阴茎向腹侧弯曲,但程度较轻,多不影响性交及排尿,可不手术治疗。国外强调美容,主张手术将尿道外口前移至正常位置。

尿道口位于冠状沟至阴茎阴囊交界处的任何部位的腹侧,尿道口远侧端的尿道板分开,不形成管状,阴茎向腹侧弯曲。尿道口愈靠近侧,弯曲愈严重,影响性交及排尿,也影响生育。阴茎头及包皮形状与阴茎头型尿道下裂相同,需手术矫正。

尿道开口于阴茎根部与阴囊交界处的正中线上,阴囊常未闭合呈分裂状,若同时并发隐睾,颇似女性阴唇。纤维变性的尿道海绵体形成一根粗硬的纤维带,从系带部伸向阴茎根部。阴茎发育不良并向腹侧严重弯曲,阴茎短小而扁平,有的甚似女性阴蒂,有的睾丸未降入分裂的阴囊或形成阴茎阴囊转位。

尿道外口位于会阴部,外生殖器发育极差,阴茎短小而严重下曲,极似肥大的阴蒂。阴囊对裂,发育不全,形如女性外阴,以致被不少父母误认为女性而需蹲位排尿。

尿道下裂越严重,伴发畸形率越高。尿道下裂伴发上尿路畸形的机会不多,因为肾脏的胚胎发育早于外生殖器。

尿道下裂最常见的并发畸形是隐睾和腹股沟斜疝,其发生率为7\%~13\%,尿道开口越靠近阴囊其发生率越高。

后型尿道下裂者其发生率为1\%~5\%,合并其他系统畸形者发生率高,合并其他一个系统畸形发生率7\%,两系统畸形发生率13\%,三个系统畸形发生率37\%。

前列腺囊是胚胎发育过程中苗勒管抑制不全或尿生殖窦男性化不全的一种表现,在后型尿道下裂患者中的发生率为10\%~15\%,前列腺囊可造成尿路梗阻,囊内结石形成和感染等。影响插导尿管,可经排泄性膀胱尿道造影检出,治疗方法为手术切除。

严重的尿道下裂合并有外生殖器的性别特征模糊,如睾丸下降不全、小阴茎、阴茎阴囊转位、阴囊分裂等表现,则应注意两性畸形的情况,应行染色体检查及有关内分泌功能检查。少见的畸形有肛门闭锁,脊膜膨出等。

阴茎阴囊型和会阴型尿道下裂应注意与女性假两性畸形及真两性畸形相鉴别,特别是合并隐睾者。不论何种类型的尿道下裂,其性染色质应为阴性,性染色体为XY,尿内17-酮类固醇正常,性腺为睾丸。

其病因是由于肾上腺皮质某些酶先天缺乏,致使肾上腺皮质的激素合成及代谢异常,使雄激素性质的中间代谢产物增加,女性胎儿外生殖器男性化。出生后外阴继续向男性方向发展,阴蒂肥大酷似阴茎,尿道口位于肥大的阴蒂根部而极似会阴型尿道下裂,阴道狭小。其主要鉴别要点如下:(1)认真检查外阴:除尿道口外,尚有阴道开口;肥大的大阴唇内无睾丸;(2)尿17-酮类固醇检查:数值升高;(3)性染色质检查:用口腔黏膜上皮或阴道黏膜上皮、皮肤或白细胞经特殊染色后检查性染色质的阳性率。本病为女性,其阳性率应高于10\%;(4)性染色体检查:应为XX。必要时可做肾上腺影像学检查,除外肾上腺皮质肿瘤。

真两性畸形的生殖腺既有睾丸又有卵巢,或为卵睾。故外生殖器可表现出两种性别同时存在的外观,也可呈典型的尿道下裂外观。其性染色质可为阳性,也可为阴性;性染色体2/3为XX,1/3为XY。若性染色质为阳性,性染色体为XX,可排除尿道下裂。如不能确定性别或最后确诊为真两性畸形,则以性腺活体组织检查为依据。

尿道下裂主要采取手术治疗,已发表的手术方法多达300多种。

(1)当性别确定为男性后,应根据尿道下裂的类型,结合有无女性生殖道、睾丸发育状况,制订全面治疗方案。分阶段进行,应保持各个阶段治疗方案的连续性。

(2)如小儿阴茎发育差,可于术前用1~2个疗程绒毛膜促性腺激素治疗,待阴茎发育后,再行手术。

(3)手术目的是矫正阴茎下弯,使尿道口恢复或接近正常阴茎头的位置,阴茎外观接近正常,能站立排尿,成年后能进行正常的性生活。

(4)有尿路感染者,术前必须严格控制感染。

(5)尿道成形术应暂行尿液分流术,根据尿道下裂类型,选择耻骨上膀胱造瘘或会阴部尿道造瘘。

(6)倾向于早期治疗。手术年龄既往多偏重于学龄期儿童,实际上1岁小儿阴茎发育的大小与5~6岁小儿相近,且幼儿手术后反应轻,早做手术能解除家属及小儿的精神压力,故目前以1岁后手术为宜,至少应于入学前或入幼儿园前完成。

是治疗尿道下裂的第1个重要环节。根据阴茎下弯程度及尿道下裂的有无确定手术方法。

(1)阴茎下弯无尿道下裂:一般主张不切断尿道,仔细将尿道周围的纤维组织切除,将阴茎头背部的包皮转至腹侧,覆盖于尿道上面。如果用此法不能将阴茎伸直,则切断尿道,伸直阴茎,中间缺损的尿道行尿道成形术。

(2)阴茎下弯合并尿道下裂:尿道下裂合并阴茎下弯,尿道周围组织缺损的程度多为Ⅰ级。因此一般要求切开阴茎筋膜至白膜的外面,勿伤及白膜,将白膜外的索状纤维组织完全切除,同时游离尿道口,切除尿道口周围与阴茎粘连的纤维组织,使尿道口后移,这样才能将阴茎伸直。有的学者认为纤维索周围的皮肤常无弹力,其对阴茎的牵扯亦影响阴茎伸直,强调应将这些皮肤切除,缺损的皮肤可用包皮转移至腹侧覆盖。

这是治疗尿道下裂的第2个重要环节。尿道成形术失败率较高,概括起来分为2类。

(1)一期手术:矫正阴茎畸形与尿道成形术一次完成,多用于阴茎型尿道下裂。尿道大部分选用包皮成形。包皮优点是皮肤菲薄、有弹性、无毛、距离近、血运好,因此形成的尿道不致坏死,成功率高。也有选用膀胱黏膜皮条形成尿道者,优点是合乎尿道生理,缺点是一旦失败,形成的整个尿道就会坏死,无法弥补。不论用什么组织形成尿道,一期手术的共同优点是一次性完成手术,痛苦少,治疗周期短。

(2)分期手术:分期手术是阴茎畸形矫正与尿道成形术分期进行。手术方法很多,不少是由Thietsch、Denis Browne及Cecil 3种手术方法演变而来的,这3种方法仍为尿道成形术的基本方法。

尿道下裂手术失败常见原因有以下4点:

(1)阴茎下弯畸形矫正不彻底:是手术失败的重要原因,多因手术切除尿道纤维索不彻底,无弹力的皮肤未曾切除,尿道外口未分离,以及发生血肿、感染等。为了在手术中确定阴茎下弯是否已完全矫正,可于阴茎根部用橡皮带扎紧,阴茎海绵体内注入无菌生理盐水,人为勃起,观察阴茎是否仍有下弯。手术时切忌切除白膜,否则可发生阴茎下弯,这种阴茎下弯很难矫正。

(2)尿道瘘的形成:是手术失败的另一重要原因,有的报道高达55\%。尿道瘘的发生与手术方法有一定关系,一期手术发生率高,发生原因是由于新形成的尿道或尿道口有狭窄、止血不彻底导致血肿形成、止血结扎线过多及切口感染等。为此,形成尿道的皮瓣越向远端越要宽些,防止尿道狭窄,用电凝器止血,可防止异物存留。止血必须彻底,防止血肿形成。

(3)皮肤坏死及裂开:形成的尿道完全坏死者,多见于膀胱黏膜移植、游离皮管移植。用包皮行尿道成形,如果血运不佳也可发生尿道坏死。其他手术多见于皮肤部分裂开。常见的原因有感染、皮肤缝合过密,边缘缺血坏死。术后敷料包扎过紧也可影响血运,使皮肤坏死。在Denis Browne手术,阴茎背部减张切口长度不足,也是伤口裂开的一个重要原因。

(4)尿道外口狭窄:一旦发现应立即进行整形,扩张外口。

尿道上裂(epispadias)是指尿道背侧壁部分或全部缺如,尿道开口于阴茎背侧,尿道口的远端呈沟状,是一种极为少见的先天性尿道畸形。

在胚胎第8周,前腹壁下部形成阴茎的生殖结节始基向后移位过多,尿生殖窦末端连接的尿生殖沟的位置靠前,使以后形成的尿道位于阴茎背侧,如尿生殖沟不在中线汇合,就形成尿道上裂。根据尿道在阴茎背侧的开口位置,男性尿道上裂可分为3型:

尿道开口于阴茎头的背侧冠状沟之前(图15-9)。

图15-9 阴茎头型

尿道开口于阴茎体背侧阴茎根部至冠状沟之间的任何位置上,这一类型最为常见(图15-10)。

尿道开口于膀胱颈,有的甚至合并不同程度的膀胱外翻和耻骨联合分离,尿道外括约肌及膀胱颈部肌肉发育不全(图15-11)。

1.尿道位置反常及阴茎畸形。

2.尿失禁 约50\%以上的尿道上裂患者有尿失禁。尿失禁的轻重主要取决于后尿道前壁组织的缺损程度。完全型者均有尿失禁,阴茎体型开口于耻骨联合下时可发生压力性尿失禁。

图15-10 阴茎体型

图15-11 完全型

3.性功能障碍 阴茎虽可勃起,但多弯向背侧,并可伴勃起疼痛,大都不能性交。

4.耻骨联合分离 正常耻骨联合在X射线片上宽约5mm,为软骨性连接。完全型尿道上裂耻骨联合则分离变宽,左右耻骨之间仅有一些纤维组织相连,坐骨结节之间的距离亦变宽。

5.女性尿道上裂 表现为阴蒂分裂、阴唇广阔性分开、耻骨联合分离和尿失禁。此外病员多有尿路感染。

单凭视诊即可确诊。对尿失禁者,应注意检查膀胱颈肌肉及尿道括约肌功能。

女性尿漏:输尿管异位开口、膀胱阴道瘘、输尿管阴道瘘等疾病因有尿漏症状,需要与女性尿道上裂伴尿失禁相鉴别。一般通过询问病史、详细体检、寻找漏尿的具体部位,两者不难鉴别。

治疗尿道上裂需达到2个目的:达到正常排尿、维持正常性交及生殖功能。为了达到这些目的必须进行:(1)膀胱颈成形;(2)矫正阴茎畸形;(3)尿道重建。

这是治疗尿失禁的主要手段,也是一个非常棘手的问题。在2~3岁以前由于尿道括约肌发育尚不十分完善,并存在着自然遗尿现象,很难确定尿失禁程度,膀胱颈成形后亦难观察疗效。因此膀胱颈成形应待病儿稍长,能有意识进行排尿后施行为好。可用Leadbetter术式做膀胱颈缩紧,延长尿道。用膀胱三角区的全层肌层,做成平滑肌管。切除原有膀胱颈及后尿道背侧的纤维组织,再将其合成管状,使管腔缩小,以形成新的膀胱颈及后尿道。由于90\%的小儿术后都会发生膀胱输尿管反流,故于缩窄膀胱颈的同时做防反流的输尿管膀胱再吻合术,也有的用膀胱前壁行膀胱颈尿道成形术,可避免行输尿管膀胱再吻合术。膀胱颈成形术后,仍需经过一段时间的排尿训练,方能达到正常排尿。

一般主张在性成熟后再行阴茎畸形矫正术。矫正阴茎畸形主要包括2个方面的问题:(1)伸长阴茎;(2)矫正阴茎下弯。如阴茎发育不好,应首先用雄激素治疗,待其发育后再行手术。手术时将左右两侧的阴茎悬韧带切断,阴茎方能伸长。将尿道黏膜与阴茎完全分离,直达精阜,阴茎才能伸直。保留尿道黏膜,包皮不得切除,留待行尿道成形术用。

大部分学者用尿道沟的黏膜条或尿道沟周围的皮肤行尿道成形术,也有的主张用包皮皮肤。无论用什么方法,新形成的尿道应置于腹侧两个阴茎海绵体之间,并应将2个分离的阴茎海绵体缝拢,在行尿道成形术时,需行耻骨上膀胱造瘘。

先天性后尿道瓣膜(congenital posterior urethral valve)在临床上非常重要。因为严重的尿流梗阻会导致整个尿路的功能障碍,包括肾小球滤过、输尿管及膀胱平滑肌的功能以及排尿的控制。后尿道瓣膜是导致肾衰竭的重要原因。

后尿道瓣膜症的病因不清楚,因偶有家族史,有人认为是中肾管的发育异常所致,也可能是多因素的结果,也有人认为是尿生殖窦发育异常造成的(图15-12)。

图15-12 胚胎期后尿道瓣膜模拟图

对于后尿道瓣膜的形成有4种学说:

1.正常精阜的远近端均有几条黏膜皱襞,如果这些黏膜皱襞肥大,突入尿道,即形成第Ⅰ型或第Ⅱ型后尿道瓣膜。

2.胚胎时期的尿生殖膜没有完全消退,尿生殖膜的残留,即形成了第Ⅲ型后尿道瓣膜。

3.中肾管或苗勒管先天畸形。

4.精阜的黏膜与尿道黏膜粘连融合。有报道,同卵双生兄弟全有后尿道瓣膜,与遗传有何关系,尚难肯定。

后尿道瓣膜症造成下尿路梗阻,在胎儿期所引起的主要危害是原肾组织在腔内高压环境下势必导致尿路发育的异常,包括膀胱、输尿管平滑肌和肾实质的结构及功能的损害。

约25\%的后尿道瓣膜症患儿有不同程度的膀胱功能异常,大多表现为尿失禁,过去曾认为与括约肌障碍有关(由于瓣膜位于括约肌水平,膜部尿道及膀胱颈发育异常)。尿流动力学检查的出现使人们认识到后尿道瓣膜症多合并原发性膀胱功能障碍,而且不会随着后尿道梗阻的解除而有所缓解,并会影响患儿的预后。从不同的有关后尿道瓣膜症患儿的研究报道发现,该症患儿往往并发有几种膀胱逼尿肌功能的异常,包括原发性肌源性障碍、无抑制性膀胱、低顺应性膀胱。膀胱功能障碍不仅表现为尿失禁,还表现为膀胱和输尿管内压增高,导致肾功能损害。同时有尿失禁患儿的膀胱功能障碍比能正常控制排尿的患儿要更严重。治疗应从膀胱功能障碍和排空能力方面为切入点,先用胆碱能药物以减少无抑制性逼尿肌收缩,间歇导尿治疗以提供满意的膀胱排空,必要时行膀胱增大手术来改善膀胱容量和顺应性。定时排尿以维持膀胱内低压,对降低上尿路腔内压力和改善肾功能有利。

是后尿道瓣膜症的常见并发症之一,在该症最初诊断时有1/3至1/2的患儿合并有膀胱输尿管反流。其中大多数是继发于膀胱内压增高、输尿管憩室和膀胱输尿管连接部功能丧失。有一些膀胱输尿管反流病例是原发的,是由于输尿管胚芽发育异常所致。后尿道瓣膜合并双侧膀胱输尿管反流者病死率大于57\%,一侧反流者为17\%,无反流者为9\%。在经过治疗的后尿道瓣膜症患儿中,约1/3的膀胱输尿管反流病例在后尿道梗阻解除后自行缓解;另外1/3的病例反流可继续存在,但在药物治疗下反流不引起问题;其余患儿在随访过程中如发现膀胱输尿管反流加重,应尽早行输尿管膀胱再植术。

在尿道明显梗阻、输尿管不同程度明显扩张的病例,经过内镜下切除后尿道瓣膜或膀胱造瘘术,梗阻一旦解除,肾积水将逐渐减轻。如果肾积水未减轻,要想到是否存在输尿管膀胱连接部梗阻,输尿管动力性梗阻不能产生有效的蠕动,肾积水是否继发于膀胱内压增高或尿流量增高,这些因素将在不同的病例中产生不一样的影响。有学者指出在后尿道瓣膜切除术后,输尿管管径减小、肾积水缓解将需几年时间,并建议在肾功能稳定和尿路感染能够控制时,输尿管的进一步外科治疗可暂缓考虑。膀胱尿动力学检查提示患儿有无低顺应性膀胱合并膀胱内压增高,若有则必须在行输尿管成形术前予以纠正。上尿路的尿动力学检查有利于输尿管非梗阻性扩张和输尿管膀胱连接部梗阻的鉴别。利尿性肾透对合并肾功能障碍、稀释性高流量尿、输尿管扩张等帮助不大。此时经皮穿刺行上尿路压力及尿流量测定(Whitaker试验)通常是非常必要的,对尿流量的测定也是很重要的。许多后尿道瓣膜症患儿合并有明显的低渗尿,像原发性尿崩症患儿一样有很高的尿流量,也将引起持续性输尿管扩张。

由于输尿管高压的影响,几乎一半的患儿有明显的尿浓缩功能障碍。后尿道瓣膜症的治疗目的是最大限度地保留肾脏功能。在后尿道瓣膜症的早期(30年前)约有25\%的患儿1岁内死亡,25\%于儿童期死亡,约50\%生存到青年期合并不同程度的肾功能障碍。现在新生儿因肾功能障碍和脓毒血症而死亡的病例很少见,其中患儿在新生儿期死亡的原因多是合并肺发育不良而导致呼吸衰竭。然而在后尿道瓣膜症患儿出生时可能合并有严重的肾功能障碍,即使在梗阻解除后,由于尿液潴留或反流而导致感染,也可能发展为肾功能受损合并高血压。肾功能的障碍可能是肾实质发育不良、肾积水、感染性肾萎缩或可能由于肾脏超滤而导致渐进性肾小球硬化的结果。后尿道瓣膜症多合并肾实质发育不良,其表现为肾实质微囊性变,特别是在肾皮质外周带比较明显,这可能是后肾胚基在腔内压增高的情况下发育而成的结果。也有学者指出后尿道瓣膜症合并肾发育不良的病因是原发性胚胎发育异常,表现为输尿管胚芽发育的位置异常。因此常见的VURD综合征(瓣膜、一侧膀胱输尿管反流、肾发育不良)也是输尿管胚芽发育异常的表现之一。一些合并症的发生也对肾脏的损害起到缓冲作用,如巨大的单侧膀胱输尿管反流、较大的膀胱憩室及尿性腹水。它们对降低腔内压力起到积极作用。由于后尿道瓣膜症造成尿路梗阻,输尿管压力增高首先影响最远端肾单位,一些患儿尿浓缩功能受损比肾小球滤过率受损程度要严重,导致尿流量增高,出现严重的脱水和电解质平衡失调。同时,尿流量增高导致输尿管和膀胱功能障碍已如前述。

稍年长儿童可被亲人发现有排尿困难症状,排尿时需加腹压,有尿频及尿流滴沥,甚至充溢性尿失禁,遗尿症状比较严重而顽固。但年龄幼小者自己不能陈述而易被亲人忽略。

此为常见体征。因排尿障碍致膀胱尿潴留及继发肾积水,又因儿童腹壁及腰部肌肉较薄弱,充盈的膀胱及积水的肾脏易于触及。排尿时腰部胀痛提示膀胱输尿管反流。

因肾功能障碍导致发育及营养不良,患儿身高、体重及智力发育均迟于实际年龄,常有贫血及低蛋白血症。

肾功能检查浓缩功能减退,严重者血BUN及Cr升高,有代谢性酸中毒及电解质紊乱表现。

常因继发肾盂肾炎而出现高热、寒战、脓尿及血尿等。

有些新生儿表现的呼吸窘迫综合征或不能解释的气胸或纵隔气肿常由后尿道瓣膜伴肺发育不良所致。

此症所致的尿道梗阻对泌尿系统损害极大,应早期诊断,及时治疗。凡有以下临床症状者,均应考虑到患有尿道瓣膜的可能性,应作进一步检查。

(1)有不能解释的纵隔气肿或气胸所并发的呼吸障碍。

(2)双侧腰部及(或)耻骨上区有囊性肿块。

(3)胎尿性腹水。

(4)羊水过少(因尿道梗阻,胎尿排出较少)。

(5)出生时即有明显脱水。

(1)持续尿频、尿线无力、排尿困难、尿失禁、遗尿。

(2)反复出现尿路感染、脓尿、血尿。

(3)腰部可触及包块,特别是双侧腰部包块。

(4)膀胱尿潴留。

(5)发育障碍。

(6)慢性肾功能不全。

多见于小儿因膀胱颈部肌肉、纤维组织增生及慢性炎症,导致膀胱颈部狭窄而发生尿路梗阻。有排尿困难、尿潴留、膀胱输尿管反流、肾输尿管积水、肾功能减退及反复发作的尿路感染。直肠指检可触及膀胱颈部硬块。排尿期膀胱尿道造影示膀胱出口抬高膀胱底部呈圆形,膀胱尿道镜检查可见膀胱颈部环状狭窄,后唇呈堤状隆起鶒三角区肥厚,膀胱底部凹陷。

系精阜先天性增大,突入尿道,形成阻塞所致的排尿障碍性疾病。可有排尿困难、尿线无力、尿频尿失禁、遗尿、肾功能不全、水电解质紊乱等表现。排尿期膀胱尿道造影可见后尿道充盈缺损,其上的尿道扩张,膀胱输尿管反流。尿道镜检可见隆起、肥大的精阜。

由先天性、炎症性、损伤性、医源性等原因所造成的尿道纤维组织增生,导致尿道管腔的狭窄。有排尿困难、尿潴留,甚至继发感染。尿道造影可显示狭窄段,用尿道探子探查时,可在狭窄段受阻。

系控制排尿的中枢或周围神经受到损害后所引起的排尿功能障碍。有排尿困难、尿失禁、尿潴留、双肾积水、肾功能减退及继发尿路感染等。一般多由外伤或手术所致的神经损伤或脊柱裂、脊膜膨出、骶骨发育不良等先天性畸形引起,也可由糖尿病,脊髓灰、白质炎等全身性疾病或某些药物引起。除排尿困难等症状外,尚有便秘、大便失禁、膀胱感觉减退或消失、会阴部皮肤感觉减退或消失、肛门括约肌张力减退、肢体瘫痪等表现。膀胱造影可见膀胱呈“圣诞树”样改变。尿动力学检查示膀胱顺应性增加、膀胱逼尿肌收缩力减退或丧失。

治疗措施的选择取决于肾功能的情况及患儿的年龄。对婴儿后尿道瓣膜症所引起的严重尿路梗阻的首要治疗是纠正水、电解质失衡,控制感染及经尿道或膀胱置管引流,应尽可能保护肾功能并使肾功能最大限度地得到恢复,改善一般情况。一般来讲,导管引流5~7天即可适当地恢复现存的肾脏功能。

近年来,内镜的应用使后尿道瓣膜症较易得到早期诊断及治疗。在肾功能改善后,可经尿道或膀胱电灼瓣膜。可用8F内镜或输尿管镜观察尿道,了解外括约肌部位。如经尿道放入内镜从膀胱内向外冲水则可见瓣膜向外张开,电灼5点、7点及中间12点部位的瓣膜。对不能经尿道放入内镜者可经膀胱造口处放入内镜,顺行电灼瓣膜,此法的优点是在扩张的尿道中能清楚看到瓣膜,对尿道创伤小。若后尿道过分伸长,膀胱尿道镜不能抵达瓣膜部位,可选用可弯曲性膀胱尿道镜,也可经输尿管镜用Nd-YAG激光切除后尿道瓣膜。

对一般情况较差的新生儿或早产儿可先行膀胱造口术(把膀胱前壁固定在腹壁上开窗,不带造瘘管)引流尿液,待一般情况好转后再电灼瓣膜,很少使用输尿管皮肤造口或肾造瘘。现已很少采用开放性后尿道瓣膜切除术和尿道扩张术治疗后尿瓣膜症。

凡经电灼瓣膜后应密切随访,观察膀胱是否能排空及肾功能恢复情况,有无复发性尿路感染。临床上,小儿的一般情况改善较快,但膀胱的恢复要慢得多,而扩张输尿管的恢复更慢。有些膀胱输尿管反流可能会缓解乃至消失。若仍有膀胱输尿管反流可作具有抗反流作用的输尿管膀胱再植术,使膀胱输尿管具有抗反流作用。若肾、输尿管积水无改善,仍持续有单侧严重反流,应鉴别输尿管有无梗阻,可考虑行输尿管成形及输尿管膀胱再植术。若肾脏无功能,可能是严重发育异常肾,则考虑行患侧肾切除术。在随访中,一小部分小儿经电灼瓣膜后仍持续有排尿困难,则需行尿流动力学检查,可能合并膀胱逼尿肌功能障碍、膀胱颈肥厚、膀胱容量减小等,可采用相应的药物治疗、间歇导尿或膀胱扩大术来改善排尿困难症状。

先天性前尿道瓣膜(congenital anterior urethral valve)可伴发或不伴发憩室。瓣膜位于阴茎阴囊交界处尿道的腹侧,不阻碍导尿管的插入,但阻碍尿液排出,导致近端尿道扩张,梗阻严重时与后尿道瓣膜所造成的损害相同。有颈的小口憩室一般不造成梗阻,但可并发结石而有症状。广口憩室被尿液充满时,其远侧唇会起到梗阻尿流的瓣膜作用。该先天性憩室可能是局部海绵体缺乏所致。

前尿道瓣膜及憩室的胚胎学病因尚不明确,有可能是尿道板在胚胎期某个阶段融合不全,也可能是尿道海绵体发育不全使局部尿道缺乏支持组织,尿道黏膜因而向外突出。

有排尿困难、尿滴沥,膀胱可有大量剩余尿。如憩室被尿液充满时,则可于阴茎阴囊交界处出现膨隆肿块,排尿后仍有滴沥,用手挤压肿块,可有尿排出。

手术切除瓣膜及憩室,如系单纯前尿道瓣膜,也可经尿道电灼瓣膜。在新生儿可先作憩室造口,日后再行憩室切除,修复尿道。对有水电解质失衡及尿路感染的婴儿,则需予以纠正,控制尿路感染,留置导尿管引流尿液,待情况好转后再行尿道瓣膜的处理。

重复尿道(duplication of urethra)是指除正常尿道外,在其背侧或腹侧还有一与膀胱相通或不通的副尿道,是极为少见的先天性尿道畸形。两个尿道可分别与膀胱相连,也可于膀胱下方汇合。可合并重复阴茎及重复膀胱。

重复尿道的真正原因尚不十分明确,有多种学说,但不能用其中任何一种解释所有类型的发生原因,主要学说有:

1.Das(1977)认为,在胚胎尿道发育过程中,阴茎及尿道板发育互相不平衡、不协调,而发生了以下3种情况:(1)尿生殖窦的阴茎已发育,而尿道嵴发育迟缓。因阴茎已经发育,尿生殖窦也随着阴茎向前发育,即形成了副尿道。当副尿道形成后尿道嵴才开始发育,形成了正尿道。结果成为第Ⅰ型重复尿道。(2)生殖皱襞畸形融合,使尿道交叉,结果形成了第Ⅱ型不完全性重复尿道。(3)乌非管及苗勒管对男女尿道的分化有重要作用。乌非管占优势,即发展成女性尿道。在尿道发育过程中,若开始时乌非管处于劣势,尿道即开口于会阴部,继而乌非管由劣势又转为优势,则尿道随着阴茎又发育成男性尿道,结果形成了重复尿道,一条尿道开口于会阴部,另一条尿道随着阴茎而发育,开口于阴茎头,即第Ⅲ型重复尿道。

2.Wilson认为(1971),尿直肠膈将穴肛分隔为前后两个腔,前面发育成膀胱及后尿道,后面发育成肛门直肠。尿生殖膈止于穴肛膜。若继续向下发育,则把尿道也分隔成前后两部分形成重复尿道,即第Ⅰ型重复尿道。

3.间质从尿道原基的侧面插入,将尿道隔成前后两个管腔(Tripathi,1969)。

4.胚胎时尿道沟交叉,即形成了重复尿道呈第Ⅱ型(Moog1968)。

这是重复尿道畸形中最多见的一种。主尿道位于阴茎腹侧,副尿道位于阴茎背侧,主要亚型有:

(1)完全型:两条尿道均与膀胱相通。

(2)不完全型:尿道出膀胱后在阴茎根部分叉,形成上下两支尿道。

(3)发育不全型:副尿道的后段极度发育不全而萎缩。

正副尿道均位于阴茎腹侧,其中一根尿道形成尿道下裂开口。主要亚型有:

(1)完全型:两根尿道完整进入膀胱,远端形成尿道下裂尿道口和正位尿道口。

(2)不完全型:尿道出膀胱后分叉,形成尿道下裂尿道口及正位尿道口。

(3)发育不全型:盲端尿道位于尿道下裂处的背侧位。

尿道在中部分裂成重复尿道。多在出膀胱颈后开始分裂(图15-13)。

图15-13 梭状重复尿道

尿道出膀胱后分叉,副尿道远端与肛管相通,形成肛前副尿道畸形。

尿道憩室(urethral diverticulum)是指尿道周围与尿道相通的囊性腔隙。可以是先天性的,也可以是后天性的。先天性尿道憩室以女性多见,多为单发,位于尿道与阴道之间;在男性则多位于阴茎阴囊交界处的尿道腹侧。憩室大小及颈部宽窄不同,造成的尿路梗阻程度和症状亦不同。

亦名原发性尿道憩室或真性憩室,真正原因尚不清,可能由下列4种原因引起:

(1)尿道海绵体先天性发育不良:尿道腹侧组织薄弱,尿流的压力使前壁扩张、突起,形成憩室。

(2)尿道沟未融合:像尿道下裂一样,尿道壁部分缺损但周围组织发育良好,形成憩室。

(3)胚胎时尿道旁残留的细胞团:发育成囊状,进而与尿道沟相通,即成为憩室。

(4)憩室的远端常有尿道狭窄:因此Campbell憩室远端尿道狭窄对憩室形成的作用。如果远端狭窄又同时有重复尿道,顶端呈盲管,则副尿道逐渐扩张形成憩室。

Shintaku(1996)报道了1例Cowper腺管扩张引起的先天性前尿道憩室。

又名继发性尿道憩室或假性憩室,其原因有以下3点:

(1)尿道外伤:这种原因最为常见。尿道损伤后周围血肿、尿外渗、感染、未能及时充分引流,周围组织机化,成为憩室壁。

(2)尿道结石:结石在尿道内停留,压迫尿道,引起局部坏死、穿破,形成憩室。

(3)尿道周围脓肿:尿道周围脓肿穿破尿道,形成憩室。病原菌为革兰阴性杆菌,埃及报道有因血吸虫病引起者。憩室位于前列腺者,多因前列腺脓肿引起,临床上较常见。

先天性尿道憩室,壁内有上皮细胞覆盖,憩室壁有平滑肌纤维。后天性憩室壁为机化的纤维组织。先天性尿道憩室多在阴茎部及球部尿道,位于尿道腹侧。后天性者在尿道任何部位均可发生(图15-14)。憩室口大小不一,先天性憩室口多宽大,后天性者一般较小。在憩室口边缘的远端,有的有瓣膜存在。有的认为瓣膜是尿道连接不良造成的,但也有可能是排尿时尿液进入憩室,憩室内的压力把憩室的前唇压入尿道,形成瓣膜,影响憩室引流。憩室内可继发感染、结石,穿破后形成尿瘘。

图15-14 尿道憩室发生部位

小的憩室无临床症状,不易被发现。憩室较大时,由于排尿时尿液灌入憩室内,可在尿道腹侧看到或触及肿块,肿块可压缩,压缩时可有尿液自尿道口滴出。

原则上应当完全切除尿道憩室。憩室口小者,切除后将尿道缝合;口宽大者,憩室切除后,尿道行Cecil尿道成形术,以弥补尿道的缺损。憩室切除有困难者,将憩室大部分切除,残余部分行内翻缝合。单纯切开引流,多形成反复发作的尿瘘。各种憩室切除术,均需行耻骨上膀胱造瘘术或会阴部尿道造瘘,待尿道完全愈合后,再拔除造瘘管。目前有用经尿道切开憩室口前后唇的方法治疗尿道憩室,可立即解除梗阻。

先天性巨尿道症(congenital megalourethra)是指先天性无梗阻的尿道扩张,一般发生于阴茎体部尿道。本病少见,合并有尿道海绵体发育异常,有时也有阴茎海绵体发育异常。巨尿道可为独立的畸形,也常并发不同程度的尿道下裂及上尿路异常,尤其在梅干腹综合征中常见。常呈舟状(scaphoid)及梭状(fusiform)两型,往往伴有上尿路先天畸形。1986年Appel等报道6例并复查35例,均伴有明显的上尿路畸形。但Sharma(1997)报道的4例均无其他畸形。(图15-15)

图15-15 先天性巨尿道症

A示排尿时外形,B示排泄性尿道造影所见

可能为胚胎期尿道皱褶处的中胚层发育不良引起,亦有认为是尿道板极度扩张所致。可分为两种类型:1.舟状巨尿道 合并尿道海绵体发育异常。2.梭形巨尿道 有阴茎、尿道海绵体发育不良。以上2种巨尿道均可伴肾发育不良、异常等并发症,而梭形巨尿道更可因并发其他严重畸形而致早期死亡。

与先天性尿道憩室症相似,主要表现为尿路梗阻及感染症状。这种巨大的尿道,实际是一个巨大的尿道憩室。

早期治疗并发的上尿路畸形,对扩张的巨尿道进行裁剪、紧缩,使其口径与正常尿道相符。如果有严重的阴茎海绵体缺乏,应考虑是否做早期变性手术。

阴茎头完全被包皮包裹,但能上翻露出尿道口及阴茎头,称为包皮过长(redundant prepuce)。包皮口狭小或包皮与阴茎头粘连,使包皮不能上翻露出尿道口和阴茎头,称为包茎(phimosis)。包茎分先天性和后天性,先天性包茎分为萎缩型和肥大型,后天性包茎系炎症、外伤等使包皮口粘连狭窄所致。

先天性包茎可见于每一个正常新生儿及婴幼儿。新生儿出生后包皮与阴茎头之间均有粘连,数月后粘连逐渐吸收,包皮与阴茎头分离。3~4岁后由于阴茎及阴茎头生长,阴茎勃起,包皮可向上退缩,外翻包皮可显露阴茎头。小儿3岁后有90\%的包茎自愈,17岁以后仅有不足1\%有包茎。包茎自愈后的小儿大部分有包皮过长,属正常现象。有些小儿的包皮口非常细小,使包皮不能退缩,妨碍阴茎头甚至整个阴茎的发育。其尿道外口亦常细小,有时包皮口小若针孔,以至发生排尿困难。有包茎的小儿,由于分泌物积留于包皮下,经常刺激黏膜,可造成阴茎头包皮炎。

后天性包茎多继发于阴茎头包皮炎和阴茎头损伤。后天性包茎多继发于包皮炎及包皮和阴茎头的损伤。包皮口有瘢痕性挛缩形成,失去皮肤的弹性和扩张能力,包皮不能向上退缩,并常伴有尿道口狭窄,这种包茎不会自愈。

包茎患儿由于包皮垢积留于包皮腔,可诱发阴茎头包皮炎。表现为排尿次数明显增多。急性发炎时包皮可出现红肿,严重者可产生脓性分泌物,伴发热等全身中毒症状。阴茎头包皮炎反复发作,由于阴茎痛痒,小儿易养成挤压阴茎的习惯,长期可形成手淫。积聚的包皮垢于冠状沟处隔着包皮显示略呈白色的小肿块,常被家长误认为肿瘤而就诊。有时包皮口小如针眼,排尿时尿液先积聚在包皮腔内,使包皮如囊肿样膨大,尿线细长,排尿困难。长期排尿困难可导致形成膀胱、尿道结石,出现膀胱输尿管反流,使肾脏受损。

婴幼儿期的先天性包茎如无症状可不必处理,如有症状,可将包皮反复试行上翻,以便扩大包皮口。手法要轻柔,不可过分急于把包皮退缩上去。当露出阴茎头后清洁包皮垢,涂抗生素药膏或液状石蜡使其润滑,然后将包皮复原,否则会造成嵌顿包茎。大部分小儿经此种方法治疗,随年龄增长均可治愈,只有少数需做包皮环切术。后天性包茎患者由于其包皮口呈纤维狭窄环,需行包皮环切术。对于包皮环切术的适应证说法不一,有些国家及地区因宗教或民族习惯,生后常规做包皮环切。有人认为包皮环切可减少阴茎癌与婚后女性宫颈癌的发病率。但有资料表明,在做常规包皮环切的以色列与包皮环切术不普及的北欧国家,这两种癌的发病率均很低,无显著差异。所以只要注意及时正确治疗包茎,讲究良好卫生习惯,可以预防阴茎癌。

包皮环切术的适应证:(1)包皮口有纤维性狭窄环;(2)反复发作阴茎头包皮炎;(3)5岁以后包皮口狭窄,包皮不能退缩而显露阴茎头。对于阴茎头包皮炎患儿,在急性期应用抗生素控制炎症,每天用温水或4\%硼酸水局部浸泡数次。待炎症消退后,先试行手法分离包皮,局部清洁治疗,无效时考虑做包皮环切术。炎症难以控制时,应做包皮背侧切开以利引流。

嵌顿包茎的治疗目的是将嵌顿的包皮恢复原位。早期嵌顿的包皮水肿较轻,可先应用手法使包皮复原。方法:局部消毒后用粗针头多处穿刺包皮,挤出水肿液,在冠状沟涂液状石蜡,用两手的中、食指夹住阴茎体包皮,两个拇指挤压阴茎头,将阴茎头推入包皮囊内,使之复位。若手法复位失败,则应做包皮狭窄环切开术。手术方法:在阴茎背侧正中用小头圆刀纵向切断狭窄环,长1.0~1.5cm,横向间断缝合,合并严重感染时,切口不予缝合。嵌顿包茎手术复位时,如情况允许应尽可能做包皮环切术。否则待局部炎症、水肿消退后再择期做包皮环切术。

阴茎阴囊转位(penoscrotal transposition)又称阴茎前阴囊(prepenile scrotum),指阴囊异位于阴茎上方,是一种十分少见的先天性畸形,分为完全性和部分性2类。阴茎前阴囊常合并会阴型、阴囊型尿道下裂,也有并发性染色体及骶尾部发育异常的报道。

阴茎前阴囊病因不明。可能是某种原因使生殖结节形成阴茎的发育过程延迟,而阴唇阴囊隆突在其前方继续生长发育所致。也可能在生殖结节和阴唇阴囊隆突同时发育的情况下,阴唇阴囊隆突向阴茎下方迁移发生障碍,从而造成阴茎阴囊相互移位。

完全性阴茎前阴囊者阴茎完全移位于阴囊后方或阴囊肛门之间,而阴茎本身发育正常。部分性阴茎前阴囊者的阴茎则位于阴囊中后部,常合并尿道下裂、阴茎弯曲、阴茎短小、肛门闭锁等畸形。

阴茎阴囊转位并不影响性生活,治疗只是解决外观异常。对于不太严重的部分性阴茎阴囊转位可不必治疗。手术是沿两侧阴囊翼上缘、阴茎阴囊交界处做“M”形切口,将阴囊转至阴茎下方。对于合并重度尿道下裂的病例,在完成尿道成形术后使用上述方法。为了保护包皮瓣血运,多主张在术后6个月修复阴茎阴囊转位。

阴茎阴囊融合(fusion of penis and scrotum)又称蹼状阴茎(webbed pennis),指阴囊中线皮肤与阴茎腹侧皮肤相融合,使阴茎阴囊未完全分离。多是先天性畸形,少数继发于包皮环切术后或其他手术切除阴茎腹侧皮肤过多所致。大多数无尿道发育异常。约3.5\%的尿道下裂并发本畸形。

先天性发病原因不明,可能为阴唇阴囊隆突相互靠拢发育成阴囊时,未与阴茎皮肤分离,而继续向阴茎延伸发育所致。阴茎通常无弯曲,尿道发育无异常。极少数患者可同时合并尿道下裂。

患儿除外观异常外无其他不适,但因蹼状皮肤伸展至阴茎头,在成人时可造成性交困难。

在阴茎阴囊之间的蹼状皮肤上做纵切横缝,可满意矫正外形,或加做V-Y、W等成形术。

隐匿阴茎(conceled penis)指原本正常的阴茎被埋藏于皮下,阴茎外观短小,包皮口与阴茎根距离短。包皮背侧短、腹侧长、内板多、外板少。包皮如鸟嘴般包住阴茎,与阴茎体不附着。如果用手将阴茎周围皮肤后推可显示正常的阴茎体。当查体时于阴茎头背侧触及一浅沟,应注意可能并发尿道上裂。

隐匿阴茎(concealed penis)是肉膜发育异常所致的先天性畸形,它与肥胖所致的阴阜阴囊基部脂肪堆积,阴茎深藏于皮下的情况不同,后者在发育成熟、脂肪组织减少后,阴茎可恢复正常状态。隐匿阴茎患者是由于阴茎部肉膜发育异常,疏松、富有弹性的肉膜变成没有弹性的、厚的纤维筋膜,有时还形成索条状物所致。这些发育异常的筋膜和索条,将阴茎拉向近侧,拘束在耻骨联合的下方。

患者多有包茎,覆盖阴茎的皮肤很短,但包皮内板却比较充足。隐匿阴茎患者若不施行阴茎松解术,会影响阴茎的发育,造成心理和生理障碍。

对隐匿阴茎的治疗及手术年龄有很大争论。如能上翻包皮暴露阴茎头,可不必手术,隐匿阴茎随年龄增长逐渐好转。手术目的是扩大包皮口,暴露阴茎头。应注意不要做简单的包皮环切术。手术方法:(1)沿包皮口环形切开包皮外板,沿2、6、10三点方向切开外板约1.5cm;(2)沿4、8、12三点纵向切开包皮内板1.5cm,外翻包皮,能顺利暴露阴茎头;(3)将包皮内外板呈嵌插皮瓣状缝合。该术式要点:分别错开纵向切开包皮内、外板,呈嵌插皮瓣状缝合切口,手术扩大了包皮口。

另一种手术方法:尽量外翻包皮,在10点及2点处纵向切开狭窄环,使包皮能外翻,露出阴茎头。距冠状沟0.8~1.0cm环形切开内板,修整缝合包皮。

由于前述隐匿阴茎的特点是外板少内板多,阴茎背侧皮肤短,腹侧皮肤长,所以有时须将腹侧带蒂包皮瓣转向背侧修复缺损的皮肤。

阴茎扭转(penile torsion)指阴茎头向一侧扭转,偏离中线,多呈逆时针方向,即向左扭转。患者阴茎发育一般正常,部分合并尿道下裂或包皮分布异常。阴茎腹侧中线偏向一侧。很多病例是在做包皮环切或外翻包皮时发现的。

先天性阴茎扭转的发生可能是生殖结节在形成阴茎的过程中,两侧阴茎海绵体发育不平衡所致。绝大多数呈逆时针方向扭转,尿道口和阴茎系带常同时扭向一侧。按阴茎头偏离中线的角度,阴茎扭转分为以下3类:

阴茎头偏离中线角度小于60°,排尿正常,无明显畸形外观。

阴茎头偏离中线角度在60°~90°之间,阴茎外观及排尿功能异常。

阴茎头偏离中线角度大于90°,阴茎外观明显畸形,可发生排尿困难。多数阴茎体及尿道海绵体也出现扭转。

阴茎扭转时,尿道口方向改变,少数患者勃起时可有隐痛症状。

第1类患者如果不影响阴茎的外观及功能,可不必治疗。大多数1、2类患者需要手术矫治,即在冠状沟上方环形切开阴茎皮肤,将皮肤分离脱套至阴茎根部,矫正扭转以中线为准,缝合阴茎皮肤。但对阴茎扭转大于90°的病例效果不佳。有的需要暴露并松解阴茎根部海绵体,切除引起扭转的纤维索带。若仍不满意,可用不吸收线将扭转对侧的阴茎海绵体白膜与耻骨联合固定,以达到整形目的。合并尿道下裂或其他畸形时应一并矫治。

重复阴茎(diphallia)的两个阴茎多是并列的,可分为分支阴茎(部分重复阴茎)和真性双阴茎(完全性重复阴茎)两种类型。重复阴茎的大小可从一个很小的附属体到正常大小的阴茎。大部分有重复尿道和独立的海绵体组织。多合并其他泌尿生殖系统畸形及肛门直肠、心血管畸形等。此病极少见,发生率约1∶500万。

胚胎发育中受特异环境或遗传因素的影响,导致泄殖腔膜的纵行重复,其头侧中胚层增多形成两个生殖结节,各发育成一个阴茎,或生殖结节延长形成阴茎时,发生融合缺陷形成分支阴茎。如果胚胎的缺陷发生在泄殖腔膜头部,可以仅发生阴茎头重复。如果缺陷发生在幼胚的后侧和尾侧,除有完全性重复阴茎外,常伴有腰骶部脊柱畸形、肛门闭锁、阴囊分裂和膀胱重复等。

有排尿、性交及射精等障碍。

①常有两个左右并列或前后排列的阴茎,通常为并列,同时有两个尿道及尿道口和独立的海绵体组织。②有时一阴茎在正常位置,而在其他部位发现另一阴茎,阴茎发育较小。

手术目的是切除相对发育不良的阴茎海绵体及尿道,同时对保留阴茎施行成形术,有尿道缺损者应行尿道成形术。分支型重复阴茎可一期或分期行阴茎、尿道成形术,真性双阴茎手术方式为切除发育较差的阴茎及尿道。在上述治疗过程中,应同时处理其他并发畸形。

小阴茎(micropenis)是指阴茎外观正常,长度与直径比值正常,但阴茎体的长度小于正常阴茎长度平均值2.5个标准差以上。阴茎的长度是指用手提阴茎头,使其尽量拉直,即相当于阴茎充分勃起时从阴茎顶到耻骨联合的距离。成人一般以阴茎松弛长度不足3cm为小阴茎。有的病例可有阴茎海绵体发育异常,睾丸发育差或下降不全。

小阴茎的病因有:①促性腺激素分泌不足的性腺功能减退(hypogonadism);②促性腺激素分泌过多的性腺功能减退;③原发性小阴茎。

(1)脑组织结构异常:无脑畸形患儿无下丘脑分泌功能,即使脑垂体发育正常,由于无促性腺激素释放激素,致使睾酮分泌少,造成小阴茎。先天性脑垂体不发育,部分脑胼胝体发育不良导致的下丘脑功能障碍,枕部脑膨出伴运动失调的小脑畸形等脑中线发育异常,均因促性腺激素分泌不足而引起小阴茎。此外还导致其他系统的多发畸形。

(2)无脑组织异常的先天性促性腺激素释放激素缺乏:此类原因引起的小阴茎比前者多见,具体病因不清,多为各种综合征,如Kallmann、帕德维利综合征(Prader-Willi)、Lawrence-Moon-Beidel综合征等,常伴有多发畸形。据研究,其与染色体、基因异常有关,还有因内分泌生化代谢异常导致的促性腺激素释放激素、黄体生成激素等缺乏症。

这类患者的下丘脑、垂体分泌功能均正常,只是睾丸在妊娠后期出现退行性变而致睾酮分泌减少,通过负反馈途径而致促性腺激素分泌过多。引起小阴茎的原因主要在睾丸本身,如先天性睾丸缺如、睾丸下降不全等。有的患者睾丸正常,但其黄体生成激素受体异常,以致不能分泌足量睾酮。此外需注意有无性别异常。

除上述原因外,还有少部分患者下丘脑-垂体-睾丸轴激素分泌正常,但有小阴茎畸形,到了青春期又多能增长。病因不清楚,推测可能是胚胎后期促性腺激素刺激延迟、一过性睾酮分泌下降等原因,也有少部分患者可能为雄激素受体异常所致。

小阴茎患者可有性染色体异常,如Klinefelter综合征(47,XXY)、多X综合征(48,XXXY及49,XXXXY)、多染色体(69,XXY三倍体)畸形。

正常男性的阴茎发育在胚胎期前12周完成,分3个阶段。第1阶段为生殖结节期,生殖结节逐渐延长类似小丘,长8~15mm。第2阶段为阴茎体期,在双氢睾酮(dihydrotestosterone,DHT)的作用下阴茎继续增长呈圆柱状,长16~38mm,尿道沟延伸至阴茎头。第3阶段阴茎长度为38~45mm,尿道发育完成。

胚胎发育到第7~8周时,生殖腺逐步分化成睾丸,并在胎盘产生的绒毛膜促性腺激素(HCG)的刺激下,睾丸间质细胞(Leydig's细胞)产生睾酮。妊娠4个月后胎儿下丘脑分泌促性腺激素释放激素(gonadotropin-nelasing hormone,GRH),刺激垂体前叶产生黄体生成素(LH)及卵泡刺激素(FSH)。在HCG、LH及FSH共同作用下,睾丸不断产生睾酮,睾酮又在5-α还原酶的作用下转化成DHT,刺激阴茎逐渐增长。小阴茎多因胚胎14周后激素缺乏所致。

询问有无家族遗传病史,注意母亲孕期情况。

有无与染色体、脑发育异常有关的畸形。检查外生殖器,测量阴茎长度、阴囊发育、睾丸的大小、质地及位置。

主要检查脑部有无下丘脑和垂体的畸形。

对小阴茎患者应该常规检查,先测定FSH、LH、T,6个月~14岁小儿的上述3个值偏低。如FSH、LH高而T低,则应做HCG刺激实验除外原发性睾丸功能低下。如FSH、LH、T均低,则先做HCG刺激实验鉴定睾丸功能,然后做促性腺激素释放激素刺激实验鉴定脑垂体前叶功能。如以上实验均正常,则考虑小阴茎的原因在下丘脑。对于脑垂体发育不良的患者进行脑垂体前叶筛查实验。如果通过检查发现激素分泌正常,应考虑是否为阴茎的受体对雄激素不敏感。

主要用于对未触及睾丸患者的探查。

对小阴茎者的治疗很困难,应根据病因及具体情况决定治疗方案。

(1)促性腺激素分泌不足性腺功能减退:最常用与FSH、LH有类似功能的HCG治疗。首次疗程即为HCG刺激试验,作为检查及治疗。若效果不明显,可用第2个疗程:每5天肌内注射1次500U的HCG,共3个月,疗程中间及完成后各复查1次。

对于下丘脑功能异常的患儿,直接用促性腺激素释放激素(如黄体生成素释放激素,LHRH)替代治疗效果最好,为了获得理想效果,给药应类似下丘脑分泌促性腺激素释放激素生理性脉冲式释放方式,每2小时增加给药量,每次25ng/kg,通过喷鼻或皮下注射给药。

(2)性腺功能异常:如单纯睾丸分泌睾酮异常,用睾酮替代疗法。可外用睾酮霜或肌内注射睾酮,每3周1次,每次25mg,共4次。治疗后阴茎、阴囊均可增长,有时有阴毛出现,有的患者可引起脊柱发育过快。

对合并睾丸下降不全患者,在内分泌治疗无效后尽早做睾丸固定术。

对于激素治疗无效的患者,可能为雄激素受体异常,要考虑手术整形。坚持做男性的可用阴茎再造成形、阴茎假体放置等方法。但应用最多的还是变性手术。

隐睾(cryptorchidism)系指一侧或双侧睾丸未能按照正常发育过程从腰部腹膜后下降至同侧阴囊内,又称睾丸下降不全(incomplete orchiocatabasis),是小儿最常见的男性生殖系统先天性疾病之一。

导致睾丸下降异常的因素很多,常见的有:

有学者通过内分泌功能测定,认为隐睾可能是青春期前下丘脑-垂体-性腺轴功能失衡,黄体生成素(LH)- 间质细胞(leydies cell)轴分泌不足,导致血浆睾酮降低,因为睾丸下降与睾酮水平密切相关,有学者测定隐睾患者睾酮水平正常,提出主要是5α-还原酶缺乏,使双氢睾酮产生障碍,或是靶器官雄激素受体不足或受体基因突变等因素,妨碍睾酮与靶细胞受体蛋白结合。某些垂体促性腺激素和雄激素紊乱疾病如Kallmann综合征(LH-RH不足)、无脑畸形垂体发育不全等多伴有隐睾症,也表明垂体促性腺激素及雄激素与睾丸下降有一定关系。近来有人在隐睾患者血中发现抗促性腺激素细胞抗体,提出隐睾可能是患者垂体自身免疫性疾病。

主要有:①睾丸引带缺如:睾丸引带在睾丸下降过程中起牵拉作用,引带末端主要分支附着于阴囊底,睾丸随引带的牵引而降入阴囊;②鞘状突未闭;③腹股沟部发育异常:内环过小或阴囊入口有机械性梗阻;④精索血管或输精管过短。

有些病例存在睾丸本身的缺陷,如睾丸在宫内扭转后萎缩,仅存有精索血管和输精管残端,睾丸与附睾分离,附睾缺如等先天性缺陷影响睾丸下降(图15-16)。

图15-16 双侧腹股沟管内隐睾

隐睾常伴不同程度的睾丸发育不全,体积较健侧明显缩小,质地松软,大部分患者伴有附睾、输精管发育异常,发生率为19\%~90\%。1\%~3\%的隐睾患者手术探查时睾丸已经缺如,仅见睾丸、附睾残迹和(或)精索血管、输精管残端。

隐睾的病理组织学特点为生殖细胞发育障碍,间质细胞数量减少。其改变随年龄增大而更加明显,成人隐睾曲细精管退行性变显著,几乎看不到正常精子。这些改变也和隐睾的位置有关,位置越低,越接近阴囊部位,病理损害就越轻微,反之病理损害越严重。

隐睾的病理组织学标志:①患儿1岁后仍持续出现生殖母细胞;②Ad型精原细胞数减少,正常睾丸曲细精管内生殖细胞的发育过程是:生殖母细胞→Ad型精原细胞→Ap型精原细胞→B型精原细胞→初级精母细胞→次级精原细胞→精子细胞→精子。正常男孩出生后60~90天血液中LH和FSH有一潮涌样分泌,刺激间质细胞增生,并分泌大量睾酮,形成睾酮峰波促使生殖母细胞发育成Ad型精原细胞。这一过程在出生后3~4个月时完成。由于隐睾患者生后60~90天的LH和FSH潮涌受挫,间质细胞数目减少,睾酮分泌量下降,不能形成睾酮峰波,从而使生殖母细胞转变成Ad型精原细胞发生障碍。

按睾丸所处位置,临床上将隐睾分为:①高位隐睾:指睾丸位于腹腔内或靠近腹股沟内环处,占隐睾的14\%~15\%;②低位隐睾:指睾丸位于腹股沟管或外环处(图15-17)。

图15-17 隐睾A睾丸下降不全可停留的位置;B异位睾丸位置

也有将隐睾分为4类:①腹腔内睾丸:睾丸位于内环上方;②腹股沟管内睾丸:睾丸位于内环和外环之间;③异位睾丸:睾丸偏离从腹腔至阴囊的正常下降路径;④回缩睾丸:睾丸可推挤或拉入阴囊内,松开后又缩上至腹股沟处。

没有并发症的隐睾患者一般无自觉症状,主要表现为患侧阴囊扁平,单侧者左、右侧阴囊不对称,双侧隐睾者阴囊空虚、瘪陷。若并发腹股沟斜疝时,活动后患侧出现包块,伴胀痛不适,严重时可出现阵发性腹痛、呕吐、发热。若隐睾发生扭转,如隐睾位于腹股沟管或外环处,则主要表现为局部疼痛性肿块,患侧阴囊内无正常睾丸,胃肠道症状较轻。如隐睾位于腹内,扭转后疼痛部位在下腹部靠近内环处,右侧腹内型隐睾扭转与急性阑尾炎的症状和体征颇为相似,主要区别是腹内隐睾扭转压痛点偏低,靠近内环处。此外,患侧阴囊内无睾丸时应高度怀疑腹内睾丸扭转。

隐睾的主要病理变化是生殖细胞发育障碍,因此可导致生育能力下降或不育。Lipshultz(1976)报道一组在青春期前治疗的单侧隐睾患者,约62\%在成年后有生育能力,没有手术治疗者仅有40\%~46\%有生育能力。即使在青春期前治疗,其精液密度也(2.68×104/ml)低,表明隐睾可以影响健侧睾丸的生育能力。Hecker对125例单侧隐睾患者做两侧睾丸活组织检查,仅40\%的健侧睾丸有正常成熟精子,并提出可能存在自身抗体,但尚未被证实。单侧隐睾患者的生育能力与手术年龄呈反比关系,即手术时年龄越大,术后生育能力越低。在1~2岁间手术者,成年后生育率为87.5\%,在3~4岁手术者为57\%,如延迟到13岁以后则仅仅为14\%。此外,单侧隐睾生育能力还受健侧睾丸与附睾的发育与成熟程度的影响,如附睾与睾丸附着变异,将阻碍成熟精子的向外输送而引起不育。

双侧隐睾患者生育能力显著下降,如睾丸位置较高,由于病理损害严重,生殖细胞发育严重障碍可致不育。但若隐睾位置较低,经适当治疗后残留部分可有生育能力。

隐睾者约65\%伴有先天性腹股沟斜疝。这是腹膜腔与睾丸鞘膜腔之间的鞘突管未闭合,肠襻降入阴囊内的鞘突腔内所致。几乎所有隐睾的腹膜腔与鞘突间的鞘突管均未闭合,当鞘状管口狭小时则不形成疝,有些患儿在生后几个月即发生较大疝,可压迫精索血管,使隐睾进一步萎缩,有些则发生斜疝嵌顿和绞窄,这些均须尽早采取手术治疗。一般情况下则待患儿稍大后与隐睾一并手术处理。

Wallenstein统计150例睾丸扭转患者中,有90例(60\%)为隐睾患者。未降睾丸发生扭转的机会是阴囊内睾丸的21~53倍。其发病原因和机制不明,可能与睾丸引带或提睾肌附着异常有关。

由于隐睾常位于腹股沟管内或耻骨结节附近,位置比较表浅、固定,容易受外界暴力创伤。随着年龄增长,隐睾患者活动范围增大,创伤机会也随之增多。睾丸伤后易发生纤维变性,加速其萎缩。

隐睾患者发生睾丸肿瘤的机会,比正常睾丸大20~40倍。高位隐睾,特别是腹内型隐睾,其恶变发生率比低位隐睾高4~6倍。腹内型隐睾的睾丸肿瘤发生率为22.7\%,Campbell的统计显示其恶变率高达48.5\%,而腹股沟或外环处隐睾仅为6.8\%。据临床观察,10岁以后手术者不能防止肿瘤发生,10岁以前手术可明显减少肿瘤的发生,3岁前手术则能避免肿瘤的发生。一般认为睾丸固定手术并不能预防恶性变的发生,即使早期手术,也不会逆转这种恶变倾向。但下降至阴囊的睾丸发生恶变后容易被早期发现。隐睾恶变的发病年龄多在30岁之后,其发病原因除睾丸自身因素外,还与局部温度、血运障碍、内分泌功能失调等有关。Sohval观察隐睾中的未分化曲细精管和阴囊内睾丸发生精原细胞瘤的变化相似,认为隐睾中先天性缺陷是使其更易发生恶变的原因。Batata等统计137例隐睾伴发恶变的资料,93例已行睾丸固定术,44例未手术。手术组年龄12岁,发生肿瘤平均年龄32岁,平均间隔时间20年。未手术组发病平均年龄也是32岁,认为隐睾纠正后应至少严密观察20年。

查体时应首先立位检查患者的睾丸。如果触诊不满意,还应再取仰卧位检查。应该避免室内过冷,并避免引起提睾反射的刺激。因为这些因素可能诱发提睾反射和睾丸的回缩。有时体检前洗个温水浴更有利于检查。大约20\%隐睾在体检时不能被扪及,这并不意味着这些隐睾都位于腹腔内。在手术中,约80\%不能扪及的隐睾可在腹股沟管内或内环附近被发现,在剩余的20\%中,单侧睾丸缺如占3\%~5\%,双侧睾丸缺如约占0.6\%。手术探查时,约65\%睾丸缺如者可在腹股沟内环处找到精索和输精管残端。因此,需要进行腹内探查寻找睾丸的病例仅占全部隐睾的2\%~3\%。

应与睾丸缺如、异位睾丸、回缩性睾丸等相鉴别。

回缩性睾丸多见于婴幼儿,是由于提睾肌过度收缩所致,随着年龄的增长、睾丸增大和提睾肌作用的减弱,这种现象将逐渐减少。用手轻柔地向下推挤可将回缩睾丸回纳至阴囊内,松手后睾丸可在阴囊内停留一段时间。其与滑动睾丸的区别是后者被推进阴囊后,一旦松手睾丸即退回原来位置,属隐睾范畴。

异位睾丸约占隐睾的1\%,可在耻骨联合上方、股部或会阴部找到睾丸。绝大多数情况下,正常睾丸下降时都沿着引带末端的阴囊分支而进入阴囊底部,如睾丸未降至阴囊底部而沿睾丸引带末端的其他分支下降至耻骨部、股部或会阴部则形成异位睾丸。

无论是单侧或双侧隐睾,在病理上都存在着退行性变,且随年龄增大而加重,因此隐睾确诊后,如生后1年仍不下降,即应开始治疗。

隐睾,尤其是双侧隐睾的病因可能与内分泌有关,因此1岁后可给予内分泌治疗,目前应用的内分泌治疗有:

(1)人绒毛膜促性腺激素(HCG):治疗目的是改善间质细胞(leydig's cell)和支持细胞(Sertoli cell)功能,促进睾丸发育,增加睾酮分泌,促使睾丸下降。有效率为14\%~50\%,剂量为1000~1500U,隔天肌内注射,1个月后随访。总量应>1万U,2万U并不增加疗效,相反会产生睾丸萎缩等不良反应。

(2)黄体生成素释放激素(LH-RH):有效率为30\%~40\%,剂量为1.2mg/d。每侧鼻孔200?g,3次/天,经鼻雾化吸入,4周为1疗程,3个月后随访。

(3)LH-RH+HCG:两者联合应用,可提高疗效,剂量LH-RH 1.2mg/d,分3次经鼻雾化吸入,持续4周后+HCG 1000~1500U,每周1次,共用3周。

睾丸固定术是治疗隐睾的主要方法,初诊时已超过6个月或激素治疗无效,1岁以后即可行手术治疗。腹股沟部斜切口的睾丸肉膜囊外固定术已被国内外广泛应用,对精索血管过短的隐睾可分两期手术,以充分保证睾丸的血供,但也有第2次手术误伤精索血管的可能,对长襻输精管高位隐睾可应用Fowler-Stephens术式,近来推荐此术式的改良方法,Fowler-Stephens分期手术,即初期手术仅高位切断精索血管蒂,不作睾丸固定,第2期有待丰富侧支循环建立后,将睾丸固定于阴囊内,减少了睾丸萎缩的机会。

经2名有经验的泌尿科医师对患者进行卧位、立位仔细检查后,未能触及睾丸者称未触及睾,约85\%以上位于腹股沟部位,Hali等对161个未触及睾丸者行腹股沟探查,仅6个未找到睾丸,其中5个不伴有疝囊,进一步作腹腔镜及腹内探查,证实为睾丸缺如,1例同侧有疝囊,于腹腔内找到睾丸,表明隐睾患者有无睾丸与疝囊有密切关系,因此Hali提出对体检、影像学及激素激发试验等项检查未扪及睾丸者,治疗方法当以腹股沟探查为首选,如腹股沟有疝囊、无睾丸,应进一步作腹腔镜或剖腹探查,如同侧无疝囊亦无睾丸,可诊断为睾丸缺如,没有必要进一步探查。如腹腔镜发现内环处有血管索状物,表明腹腔内可能有睾丸残余结节存在,应予切除,具体方案见图15-18。

图15-18 未触及睾丸的处理方案

腹腔镜在隐睾诊断与治疗中的应用:1901年,Kelling应用膀胱镜经人工造口进入狗的腹腔,通过过滤的空气造成气腹进行腹腔检查,首次提出了腹腔镜概念。至1910年,Jacobaeus才将这一技术应用于临床。此后许多学者在设备、技术等方面进行了大量探索和改进。1924年Zollikofer改用CO2 作气腹;1929年Kalk设计出135°镜及气腹针;1938年Veress制成弹簧气腹针,此针沿用至今并以他命名,即Veress气腹针。1960年Semm发明了自动气腹机,同年Hopkins改进了冷光源导光系统,使腹腔镜也可方便应用冷光源;1996年Hopkins又设计出柱状石英腹腔镜。此改进对现代腹腔镜的广泛应用起到了积极推动作用。

1976年Cortesi等在泌尿外科首次应用腹腔镜对隐睾进行了定位和活检,成为目前腹内隐睾和睾丸缺如的一种安全、准确的诊断方法。据1974年美国泌尿外科协会(AUA)统计,隐睾占整个泌尿外科腹腔镜手术的11\%,平均手术时间14分钟,准确率达98\%,并发症仅为1.6\%。腹腔镜不仅能对隐睾进行定位和评估,还能进行腹内隐睾固定或切除术。自1982年Scott及Cohen首次应用腹腔镜治疗高位隐睾以来,已有较多应用腹腔镜治疗隐睾的报道,并逐渐受到泌尿外科医生的重视。

无睾症又称先天性睾丸缺如(congenital anorchidism),患儿出生时单侧或双侧单纯性无睾丸,外生殖器表现为男性,性染色体无异常。临床上较少见,分为单侧和双侧睾丸缺如。无睾症应该区别于诸如精索扭转或是睾丸炎等引起的部分或完全性的睾丸萎缩,上述疾病至少能在组织学上发现退化残存的睾丸组织。无睾症可以是先天的,也可以是获得性的。

先天性睾丸缺如的发生原因迄今尚不确定,可能与以下2方面因素有关:(1)胚胎发育过程中,因某种因素干扰使性腺发育障碍。Lobacarro等认为Y染色体性别决定区上SRY基因异常可能导致无睾症;(2)妊娠期或出生前、后不久睾丸扭转、精索血管栓塞,致睾丸血流供应受阻而使睾丸萎缩,可能是最常见的原因。

单侧睾丸缺如者,其阴茎、阴囊发育正常,由于健侧睾丸代偿性增生,血中睾酮水平正常,青春期第二性征正常。双侧睾丸缺如者,患者无发育能力,呈宦官型发育,表现为皮下脂肪丰满、皮肤细腻、语调高尖。体格检查发现阴囊发育不良,阴囊内空虚无睾丸。阴茎小,无阴毛生长。睾丸缺如常在隐睾手术探察中被诊断,约为隐睾手术的3.3\%,约20\%的术前不能触及睾丸的患者在手术探察时不能找到睾丸。

由于单侧睾丸缺如者对侧睾丸功能正常,如无其他并发畸形,则不需治疗。从患者心理角度考虑,可将人造睾丸植入阴囊内作为假体,除无功能外,假体外形和感觉均较满意。

双侧睾丸缺如婴幼儿可考虑做变性手术,先做阴蒂成形术,待青春期再做阴道成形术。早期确定社会性别,按女孩抚养可减轻其父母的焦虑。若拒绝变性手术,则应早期应用性激素治疗。肌内注射睾酮,促进阴茎阴囊发育。

双侧睾丸缺如者可于青春期用性激素替代治疗,肌内注射睾酮以促使男性化。双侧无睾症的不育症无法治疗,为满足患者心理需要,可将人造睾丸植入阴囊内作为假体,此外还可选择同种异体睾丸移植。同种异体睾丸移植与其他脏器移植一样,术前组织配型,排除供者睾丸、附睾及输精管疾病。

多睾症(polyorchidism)十分罕见,系指阴囊内除有两个正常睾丸外,还有一个或一个以上额外睾丸在一侧阴囊内。多睾症的额外睾丸可具有正常的附睾和输精管,并有生成精子的能力,或与正常睾丸共同拥有一个附睾和输精管。

多睾丸可能是胚胎的生殖嵴在衍化成睾丸的过程中,因某种因素使胚胎早期生殖嵴内上皮细胞索分裂所致。

多数患者没有任何的症状,只是被偶然发现。生育力一般正常。然而,第3个睾丸发生恶变或是扭转时则需要及时处理。多数情况下会同时伴有疝或下降不全。有附睾和输精管的第3个睾丸也可能成为输精管结扎术后保持生育力的一个原因。

多睾症的治疗因人而异。肿瘤和扭转时需要手术切除,而其他情况则可以保留睾丸,但需要做包括超声检查在内的常规检查来排除早期肿瘤。如果额外睾丸发育良好,无病理改变可不切除。

本症又称曲细精管发育不全或原发小睾丸症或Klinefelter综合征(Klinefelter syndrome)。1942年由Klinefelter、Reifenstein及Albright描述,其特点是睾丸小、无精子及尿中促性腺激素增高等。1959年Jacobs等发现本病患者性染色体为47,XXY,即比正常男性多了1条X染色体,因此本病称为47XXY综合征。其根本缺陷是男性多一X染色体,常见的核型是47,XXY或46,XY/47,XXY。

本病的发病率相当高,在性别异常中占重要地位。男性新生儿中的发病率为1.3‰,即850人中有1名患者。根据白种人的资料,身高183cm的男性人群中其患病率为260∶1,在精神病患者或刑事收容机构人群中为100∶1,在因男性不育而就诊者中约20∶1。许多XXY患者除不育外无任何症状或不适,因而不会就诊。

先天性睾丸发育不全的形成可能是卵细胞在成熟分裂过程中,性染色体不分离,形成含有两个X的卵子,这种卵子若与Y精子相结合即形成47,XXY受精卵。如果生精细胞在第1次成熟分裂中XY不分离,则形成XY精子,这种精子与X卵相结合也可形成47,XXY的受精卵。一般认为大多数47,XXY的形成系卵子在成熟分裂过程中性染色体不分离所引起。

迄今发现本病染色体核型约有30余种。绝大多数患者的核型为47,XXY,占全部病例的80\%。大约15\%的患者有2个或更多细胞系的嵌合体,其中较为多见的是46,XY/47,XXY(约占7\%)及46,XY/48,XXXY嵌合型,前者的临床表现较典型的47,XXY轻微。另有1\%患者核型为46,XX/47,XXY,但表型与典型的先天性睾丸发育不全患者无异。还有一些具有典型临床表现的患者的核型是46,XX。对这一奇怪现象的解释是:患者在胚胎发育中曾是嵌合体46,XX/47,XXY,不过XXY细胞系以后消失或者后者比例极小,因而未被查出。

除嵌合体外,先天性睾丸发育不全还有许多变异类型,如48,XXYY、48XXXY、49,XXXXY、49,XXXYY等。患者的X染色体愈多,智力发育障碍愈严重,男性化障碍程度亦更明显,并伴有躯体畸形。一些XXXY患者有尺骨桡骨骨性联合,XXXXY患者还有颅面和四肢多发畸形。但无论X染色体的数目增加到多少,只要有Y染色体就决定其表现型为男性。

先天性睾丸发育不全患者在儿童期无异常,常于青春期或成年期时方出现异常。患者体型较高,下肢细长,皮肤细白,阴毛及胡须稀少,腋毛常常没有。呈类阉体型。约半数患者两侧乳房肥大。外生殖器常呈正常男性样,但阴茎较正常男性短小,两侧睾丸显著缩小,多小于3cm,质地坚硬,性功能较差,精液中无精子,患者常因不育或性功能低下求而治,智力发育正常或略低。

通常X性染色体数目愈多,则智力障碍愈严重,且常伴有一些躯干畸形如腭裂隐睾、斜颈等。

一般在发育期前难于作出诊断,不育或性功能障碍是患者就诊的主要原因。体型较高,双侧睾丸较小,两侧乳房肥大是典型病状。X小体阳性,染色体组型为47,XXY则可肯定诊断。

本病常需与低促性腺激素性性腺功能减退(hypogonadotropic hypogonadism,HH)相鉴别。后者睾丸小而软,外生殖器及第二性征发育较差,男性乳房发育较少见,身材亦较高,上下肢均过长,血浆促性腺激素减低,血清T明显降低,曲细精管和Leydig细胞均有明显异常,染色体核型正常。根据这些不同的特点即可与本病相鉴别。

长期补充男性激素以改善第二性征,但疗效并不理想。一般采用丙酸睾酮(丙酸睾酮)或甲睾酮(甲基睾酮)片舌下含服。较方便的是给以长效睾酮如庚酸睾酮或环戊丙酸睾酮。也可考虑同时给予绒毛膜促性腺激素药物,仅对男性化有一定帮助,但并不能改变女性型乳房,故对乳房肥大者,可将乳房内乳腺及脂肪组织切除。

附睾畸形(deformity of epididymis)临床上较常见。通常指附睾明显变长或与睾丸附着异常。至今尚无确切定义和统一的分类方法。

胚胎第6周中肾管和中肾旁管形成,这些管道将衍变成男、女生殖道。当胚胎生殖腺分化成睾丸并产生睾酮后,在雄激素的作用下中肾管逐渐衍变成男性生殖管道。中肾管头端部分变成附睾附件,由中肾小管衍变而来的睾丸输出管,与位于其下方的中肾管增长曲折盘绕形成的附睾管共同构成附睾头部,余下的附睾管则形成附睾体和尾部。

先天性附睾畸形病因尚不清楚。由于隐睾患者多合并附睾畸形,故其发生可能与胚胎发育过程中内分泌功能失调有关。因睾酮水平低下,中肾小管及中肾管不发育或发育不全,而形成各种类型的附睾畸形,如中肾管完全不发育,则可导致先天性附睾、输精管缺如。若发育在某一部位时终止,则形成该部闭锁。当附睾管曲折盘绕障碍时,可出现附睾明显延长,形成发生长襻形附睾畸形。

附睾为连接睾丸的排精管道,外形细长扁平,位于睾丸的后外侧。10~15根睾丸输出管迂曲成圆锥状,末端汇合成一根长4~6cm,且高度迂曲的附睾管。据Turck对94例非隐睾者如疝、鞘膜积液、精索静脉曲张等进行阴囊探查发现,83.9\%的附睾头与附睾尾附着于睾丸,而附睾体与睾丸之间有一定距离,一般可容纳一指尖(图15-19),附睾与睾丸完全紧贴者仅占12.5\%(图15-20)。

图15-19 附睾头、尾附着于睾丸

图15-20 附睾与睾丸完全紧贴

附睾畸形主要表现为附睾发育障碍和与睾丸附着异常。前者包括附睾缺如,头部囊性变,中部、尾部不发育,呈纤维索状闭锁,附睾明显变长呈长襻形等。附睾缺如又可分为:

1.中肾管完全不发育,输精管、精囊及射精管一并缺如。

2.中肾管发育不全,附睾体尾部缺如,同时伴有输精管缺如。

3.中肾管未发育成附睾管,而直接衍变成输精管、精囊和射精管,睾丸输出管与输精管相连。

4.无附睾,输精管不与睾丸连接,其近端呈盲端。附睾附着异常,包括附睾与睾丸完全分离和部分分离,后者指附睾头与睾丸不连接,附睾不附着于睾丸下极等。

自1971年Scorer及Farrington首次对附睾畸形进行分类以来,已报道了许多不同的分类方法。1990年Koff及Scaletscky在Scorer分类方法的基础上进行了一些修改,将附睾畸形分为5类:

Ⅰ型:长襻形附睾:附睾呈长襻状,与睾丸比较明显变长,又分4型:①较睾丸大小长2倍;②2~3倍;③3~4倍;④4倍以上。

Ⅱ型:附睾与睾丸分离:按分离的部位和程度,此型又可分为如下3种情况:①仅有尾部分离;②头尾部均与睾丸分离,但相距较近;③头尾部均与睾丸分离,相距较远。

Ⅲ型:附睾与睾丸成角:①单纯成角;②伴有附睾狭窄。

Ⅳ型:附睾闭锁或附睾、输精管的任何部位连续性中断。

Ⅴ型:较长的睾丸系膜。

上述不同类型的附睾畸形可在同一患者中同时存在,Koff的资料表明长襻形附睾畸形在隐睾及异位睾丸中最常见,约占79.3\%,附睾与睾丸分离为45.1\%,附睾与睾丸成角为8.5\%,附睾或输精管闭锁为3.7\%,较长的睾丸系膜为1.2\%。Koff分类方法尚不能包括所有附睾畸形,综合有关资料归纳附睾畸形的各种类型(图15-21)。

附睾畸形患者无任何不适,临床上常以隐睾或男性不育而就诊。

附睾畸形不影响生育时,无须治疗。节段性附睾闭锁可采取附睾管输精管吻合术进行治疗。在放大10~20倍手术显微镜下,选用Silber和Wagenknecht两种附睾输精管吻合方法。二者的主要区别是Silber法采用扩张的附睾管与输精管端端吻合,而Wagenknecht法则采用扩张的附睾管与输精管端侧吻合。切开扩张的附睾管,先接取附睾管断端流出的液体查找精子,确定存在精子后,用11-0无损伤尼龙线将附睾管与输精管黏膜吻合,一般间断缝合4~6针,然后用9-0无损伤尼龙线行附睾被膜与输精管肌层吻合。

图15-21 附睾畸形的分类

A.附睾、输精管均缺如:B.附睾发育不全;C.无附睾,输精管与睾丸直接相连;D.无附睾,输精管呈盲端与睾丸分离;E.附睾头分离,尾部与睾丸连接;F.附睾头与睾丸相连,尾部与睾丸分离;G.附睾头尾与睾丸分离,体部与睾丸相连接;H.附睾与睾丸完全分离:I.附睾体部闭锁:J.附睾尾部闭锁:K.长襻状附睾:L.附睾囊肿

附睾缺如者本身无法治疗,主要解决其生育问题。若患者睾丸生精功能正常,可行辅助生殖治疗。附睾畸形合并隐睾者,应及时行睾丸固定术。

附睾囊肿多位于附睾头部,罕有破溃、出血引起急性阴囊病变者。附睾囊肿虽然在小儿不常见,但近年注意到易发生于母亲妊娠期用过乙烯雌酚的小儿。据报告,正常男性中,有5\%的人有附睾囊肿,而曾暴露于乙烯雌酚中的男性,附睾囊肿的发生率可达21\%。

附睾囊肿经触诊、透光试验及超声可以确诊,除非囊肿增大,一般不必手术。本病是否影响生育尚无定论。附睾头囊肿可采取穿刺抽液注射硬化剂的治疗方法,但由于复发率较高,不如手术方法安全可靠,目前已较少使用。

阴囊鞘膜腔内液体增多而形成的囊肿称为鞘膜积液(hydrocele)。它是泌尿外科的常见病,可见于各个年龄阶段。

在胎儿发育过程中,睾丸从腹膜后下降,下腹部腹膜形成一突起,经腹膜沟管进入阴囊,即形成鞘状突,鞘状突为一双层腹膜构成的盲袋,一面紧贴睾丸和精索,婴儿出生前后除紧贴睾丸的鞘膜形成固有的鞘膜腔外,其余鞘膜均闭合,若闭合不全则形成鞘膜积液。

鞘膜积液有原发、继发两种。原发者无明显诱因,病程缓慢,可能与慢性炎症和创伤有关,积液为淡黄色清亮液;继发者可能是由于急性睾丸炎、急性附睾炎、创伤、丝虫病、血吸虫病等,积液多浑浊,甚至呈血性、脓性或乳糜性。

根据鞘膜积液所在的部位与鞘状突闭锁的情况分为以下类型(图15-22):

图15-22 鞘膜积液类型

最常见,鞘状突闭合正常,积液发生在睾丸鞘膜腔内,呈球形或卵圆形。

鞘状突的两端闭合,而中间的精索鞘膜囊未闭合而形成的囊性积液,又称精索囊肿。

鞘状突在内环处闭合,精索处未闭合,并与睾丸鞘膜囊相通,与腹腔不相通。

鞘状突完全未闭合,鞘膜腔与腹膜腔相通,鞘膜腔内积液为腹腔内液体,积液量随体位改变而变化,此型又称为先天性鞘膜积液。如鞘状突与腹腔的通道较大,可同时出现腹股沟斜疝。

一侧多见,一般无自觉症状,常在体检时偶然发现。当积液量大,囊肿增大,张力高时,站立位可有下垂感或牵扯痛,巨大鞘膜积液时,阴茎缩入包皮内,影响排尿、行走和劳动。睾丸鞘膜积液多呈卵圆形,位于阴囊内,表面光滑,无压痛,囊性感,触不到睾丸和附睾,透明试验阳性;精索鞘膜积液位于睾丸上方或腹股沟内,其下方可触及睾丸、附睾;婴儿型鞘膜积液,阴囊内有梨形肿物,睾丸亦触不清;交通性鞘膜积液与体位有关,站立位积液增多,阴囊增大,卧位时积液可减少或消失,亦可触及睾丸。

根据病史、体征,诊断鞘膜积液一般不难。应与腹股沟斜疝、睾丸肿瘤、精液囊肿鉴别。腹股沟斜疝透光试验阴性,咳嗽时内环处有冲击感,有时可见肠鸣或听到过肠鸣音,较易回纳入腹腔;睾丸肿瘤为实质性肿物,患侧有沉重感,质地硬,透光试验阴性,一般呈持续增长,B超检查有助于鉴别;精液囊肿通常发生于附睾头,透光试验阳性,可触及睾丸,如行囊肿穿刺,囊液为淡黄色微浊,镜检可见大量死精子。

鞘膜积液的治疗原则是对体积较小,张力不高且无症状的囊肿无须急于手术,尤其是1岁以内的婴幼儿尚有自行消退的可能。但对体积较大,张力很高,可影响睾丸血液循环,导致睾丸萎缩者,应及时手术治疗。目前治疗鞘膜积液的方法主要有如下两种:

早在18世纪,Guy首次注射布尔得葡萄酒以治疗鞘膜积液。此后陆续有用奎宁、尿素、鱼肝油酸钠等药物注射治疗鞘膜积液的报告,直到1933年Kilbotlrne及Murray报告他们应用硬化剂治疗鞘膜积液的经验后,临床上才开始广泛应用。1975年Maloney对手术与硬化疗法治疗鞘膜积液作了前瞻性对比研究,结果发现硬化疗法明显优于手术方法。1985年Baker首次应用四环素液作硬化剂治疗鞘膜积液,取得了显著疗效,其副作用少,随访无复发。1988年Levine对25例积液量在20~780ml的睾丸鞘膜积液患者应用四环素硬化治疗,成功率为93\%,其中一次成功率达75\%,个别患者需要治疗2~3次。1989年李磊报告41例应用四环素硬化治疗的鞘膜积液患者,成功率达97.6\%,无并发症,认为这是一种安全、有效的手术替代疗法。

一般认为积液量大于500ml或疑有睾丸肿瘤者不适合硬化疗法。交通性鞘膜积液切忌应用此法,以免四环素溶液流入腹腔而出现严重并发症。

用四环素治疗鞘膜积液所取得的成效,可能是由于溶液pH低引起明显的细胞反应,导致纤维增生及鞘膜粘连所致。这些组织反应是否会造成尚在发育中的小儿睾丸的远期损伤,目前尚不清楚。由于小儿鞘膜积液患者鞘状突管多未闭塞,单纯穿刺排液注药硬化治疗难以达到治愈目的。1995年阮元峰报告3例鞘膜内注射四环素致小儿睾丸坏死的病例,应引起警惕。

若囊肿大、张力高,有可能影响睾丸血运和睾丸发育者可行手术治疗。手术是治疗鞘膜积液的主要方法,术中要打开鞘膜囊,并将鞘膜翻转。对于交通性鞘膜积液,尚应高位结扎鞘状突。术中应仔细止血并置橡皮膜引流,以防阴囊血肿形成。

输精管缺如(absence of vas deferens)是男性生殖系统的一种先天性畸形,是阻塞性无精子症及男性不育的重要病因。该病早在18世纪中叶就已发现,但由于诊断手段的局限,直到20世纪前半叶全世界仅有报道25例。其后,随着男性不育诊治技术的提高,病例报道不断增加。从1985年至今,国内陆续报道病例及治疗研究已近170余例。尽管如此,该病的病因仍未阐明。近10余年来,随着对该病病因学探讨的逐渐深入,输精管缺如与囊性纤维化病(cystic fibrosis,CF)的关系得到广泛关注,对后者的深入研究使先天性输精管缺如的分子生物学基础得到了初步的揭示。

输精管缺如是输精管畸形中最常见的一种类型。自1755年John Hunter首先发现以来,报道例数逐年增多,迄今国内已报道约200余例。国外报道阻塞性无精子症占男性不育的6\%~14\%,而输精管缺如占阻塞性无精子症的18\%~50\%。其中,先天性双侧输精管缺如(CBAVD)占整个男性不育的1\%~2\%,先天性单侧输精管缺如(CUAVD)占男性不育0.5\%~1\%。1985年我国曾一次报道1310例男性不育患者中,输精管缺如占1.15\%,而在1989年报道的250例无精子症中输精管缺如占24\%,与国外报道相近。由此可见,先天性输精管缺如在临床上并非罕见。

先天性输精管缺如作为一种男性生殖系统的先天畸形,一直被怀疑与遗传因素有关。曾发现该病在一些家族中有聚集现象,而发现囊性纤维化病与先天性输精管缺如在临床上密切相关则为遗传因素的重要作用提供了有力的依据。自20世纪80年代末,囊性纤维化病的临床和分子遗传学研究取得了重大进展,对该病与先天性输精管缺如的关系在分子水平上的研究,使先天性输精管缺如的遗传基础得到了初步阐明。

囊性纤维化病是一种常见的致死性常染色体隐性遗传病,发病率在白种人中达1/2000活产儿,致病基因携带者频率高达1/22。其临床主要表现为慢性肺部疾患,胰腺外分泌功能不足,汗液电解质浓度增高及男性不育等。该病致病基因已于1989年被定为位于7号染色体长臂3区1带(7q31),并已克隆测序。基因全长250kb,有27个外显子,eDNA长6129bp,编码蛋白质称为囊性纤维化病跨膜转运调节物(cystic fibrosis transport regulator,CFTR),行使氯离子通道的功能。目前已发现600余种该基因的突变和变异,覆盖整个CFTR基因区域,其中白种人中70\%的突变△F508,即第10外显子第1653~l655位碱基对缺失了一个编码肽链第508位苯丙氨酸的密码子。

在对囊性纤维化病的研究中发现,绝大多数男性患者均因先天性输精管缺如而不育,这表明该基因的突变与输精管发育异常密切相关,研究已经证实第1类先天性输精管缺如患者由CFTR基因突变引起。然而,无典型囊性纤维化症状或身体貌似健康男性的先天性输精管缺如是否与CFTR基因异常有关,目前尚不清楚。1983年Petit等首先报道1例无典型囊性纤维化病症状的先天性输精管缺如患者,其伴有7号染色体异常即inv(7)(p15,q32),inv(9)(p11,q13)。由于当时CF基因尚未定位,未考虑到其与囊性纤维化的关系。现在知道,CFTR基因恰好定位于7q31。此例染色体臂间倒位可能破坏了7号染色体上与其相邻的CFTR基因结构,从而引起先天性输精管缺如的发生。此例提示第2类先天性输精管缺如的发生可能与CFTR基因异常存在相关关系。其次,随着囊性纤维化病的临床与分子遗传学的深入研究,日益显示出囊性纤维化病临床表现的多样化及突变基因型与表现型的密切相关,即不同的突变基因型可能导致不同的临床表现,同一突变基因型在不同个体中亦可能有不同的临床表现。许多研究者就此类CBAVD患者的CFTR基因外显子及外显子与内含子剪切位点进行了广泛的突变筛查,结果发现此类CBAVD确与CFTR基因突变密切相关。在这些患者中至少50\%~70\%,有时甚至高达86\%携带有一个CFTR突变基因,其中还有10\%是CFTR基因突变的复杂杂合子,即两条7号染色体各有一种CFTR基因突变。这与正常人群中仅4\%的CFTR基因突变携带频率及0.2\%的发病率相比有显著差异。此外还发现此类CBAVD患者CFTR基因中有许多新的或比较罕见的突变,与典型囊性纤维化患者携带的突变类型及频率有所不同,进一步证实了囊性纤维化病的突变基因型与表现型之间的关系,即较常见的突变引起了典型囊性纤维化病症状,而一些相对罕见的突变则更多地引起了CBAVD。

综上所述,目前比较一致的看法是:相当一部分无典型囊性纤维化病症状的CBAVD男性是囊性纤维化病的一个独特的遗传群体,CBAVD与肺部疾患、胰腺功能不足同是囊性纤维化基因突变的一种重要的表现型。CFTR基因编码区突变是第2类CBAVD最重要的遗传病因之一。

此外,在对CUAVD的研究中,研究者发现当患者单侧输精管缺如对侧的输精管在腹股沟或骨盆水平有非医源性闭锁时,其CFTR基因突变率常高达89\%,与CBAVD相似而与正常人群有极显著差异这证实了CFTR基因编码区突变也是此类CUAVD最重要的遗传病因之一。

目前对囊性纤维化病基因表达的研究不仅涉及基因编码区,还包含了少数非编码区,如第8内含子(introne 8)的poly(T)。在对呼吸上皮CFTR基因表达的研究中证实,等位基因5T能影响该基因第9外显子的正常剪切,使转录水平下降,导致CFTR基因不完全表达,引起CFTR蛋白水平的降低,并可引发一系列临床症状。故目前认为introne 8的5T突变是导致囊性纤维化病临床表现多样性的原因之一。为探讨CBAVD与5T突变的关系,有作者在对CBAVD的研究中,根据囊性纤维化基因突变情况将结果分为3组:第1组约占15\%,为CFTR基因突变的复杂杂合子,均无5T突变;第2组约占60\%,有一个CFTR基因突变,其中60\%以上伴有另一个CFTR基因上的5T突变;第3组约占25\%,未发现其他CFTR基因突变,而其5T突变携带者频率约为25\%,甚至出现5T纯合子。由此可见当2个CFTR基因均发生突变时,CBAVD的发生可能仅与前者有关而与5T突变无关;而后两组5T突变携带者频率均明显高于正常人群5\%的携带者频率,显示出极显著差异,提示CFTR基因introne 8上5T突变的存在可能是先天性输精管缺如发生的另一个遗传病因,且一个CFTR基因编码区突变加上另一CFTR基因非编码区的一个5T突变可能是导致先天性输精管缺如最常见的原因。

对CBAVD男性CFTR基因转录水平的研究还表明,当一个CFTR基因突变与一个5T突变组成杂合子时,其转录产物仅为正常的6\%~16\%,而5T突变纯合子转录产物仅为正常的24\%。这进一步为5T突变在先天性输精管缺如发生中的重要作用提供了理论依据,并在一定程度上揭示了作为常染色体隐性遗传的囊性纤维化病在仅有一个基因编码区突变时也是发生先天性输精管缺如的原因。此外,该基因的转录有一定的组织特异性,如呼吸上皮转录水平高于附睾上皮,这可能是携带CFTR基因突变的先天性输精管缺如患者无其他临床表现的原因之一。

相当一部分的单纯性先天性输精管缺如是CFTR基因突变的结果,是囊性纤维化病的一个特殊的表现型。但在另一部分CBAVD及CUAVD患者中并未发现CFTR基因突变,其原因可能是:①CFTR基因较大,目前的聚合酶链反应-单链构象多态性(polymerase chain Reaction-single strand Conformation polymorphism,PCR-SSCP)等技术尚难检出全部的突变;②迄今仅对CFTR基因的外显子、剪切位点和极少数内合子进行过突变分析,尚不能排除启动子区域或其他调节位点突变的存在,而后者的存在是可能的;③一些家系中父子、同胞携带同一突变时,仅其中之一发生CBAVD,这说明先天性输精管缺如发生过程中,除CFTR基因异常外,还可能存在其他遗传因素和环境因素的作用;④在先天性输精管缺如合并其他泌尿系统畸形(如肾脏畸形等)及一侧输精管基本正常的CUAVD患者中,均未发现CFTR基因的突变,提示此类先天性输精管缺如与CFTR基因无关,也支持尚有其他病因作用的观点。

鉴于对先天性输精管缺如的分子遗传学研究尚局限于基因编码及少数非编码区,因而除继续从CFTR基因编码区检出更多突变外,还应将筛查的区域逐渐扩展到基因启动子区及其他调控区,并研究CFTR基因外的其他与先天性输精管缺如发生可能有关的基因,从而更加充分地揭示先天性输精管缺如的分子遗传基础。

先天性输精管缺如可分为:

由于双侧中肾管均未发育或发育不全,可伴有附睾、精囊缺如,很少伴发有双肾畸形或缺如。

由单侧中肾管未发育或发育不全所致,常伴发同侧输尿管不发育而致肾不发育,出现同侧肾、输尿管输精管、附睾管均缺如。

又可分为输精管阴囊段缺如和输精管盆腔段缺如,可能是中肾管在衍变成输精管过程中突然中止所致。其他输精管畸形包括:输精管某一段呈纤维索状、管腔闭锁不通。中肾管分支发育成重复输精管,在所报告的重复输精管病例中,大多数重复侧有两个睾丸,输精管各自与一个睾丸相连。此外输精管可偏离精索,异位开口于其他部,1978年Kaplan报道8例输精管异位畸形患者,其中6例合并其他泌尿生殖器官畸形,3例伴有先天性肛门闭锁,由于睾丸自生殖嵴发育而来,故输精管畸形时睾丸一般无异常。

根据临床表现及与囊性纤维化病的关系,先天性输精管缺如可分为2类:第1类与囊性纤维化病明确相关,患者多以慢性肺部疾患、胰腺功能不足等就诊。检查时可发现汗液中电解质浓度升高等典型的囊性纤维化病症状;第2类病因不明,临床多以不育就诊,而体检时未见其他异常。

双侧输精管缺如常因婚后不育而就诊,患者身体健康,性生活正常,能射精。阴囊触诊不能扪及精索内输精管。由于对侧睾丸输精管正常,单侧输精管缺如可不影响正常生育,故不必治疗。重复输精管畸形无临床症状,性生活正常,多在阴囊探查手术中被发现。

研究表明先天性输精管缺如的主要遗传学病因———CFTR基因突变并不影响精子本身的功能及人工授精的成功率,但该病的治疗仍相当困难。以往可通过穿刺人工精子贮囊抽吸精子作人工授精,1955年Hanly首先用羊膜制作一个贮囊,并使患者妻子怀孕。随后一些学者在静脉、睾丸鞘膜以及动物试验中用硅橡胶制作贮囊,但均未能得到推广。Cruz(1980)报道25例有4例怀孕,1例自然流产。Kelaml(1982)报道23例中2例怀孕,1例自然流产。Silber等(1985)报道24例却无1例怀孕,实践证明成功率太低,难以在临床上推广。近年Tournaye等推荐显微外科附睾精子吸引术(microsurgical epididymal sperm aspiration,MESA)与卵胞浆内单精子注射(intracytoplamic sperm injection,ICSI)技术相结合进行治疗,可有效提高生育率。

针对囊性纤维化主要是对症处理,应用抗生素采用饮食疗法以利食物消化吸收。

先天性精囊囊肿(cyst of seminal vesicle)较罕见,多在青春期后始被发现,常合并射精管梗阻、血精和泌尿系感染,亦可因压迫膀胱或尿道而引起排尿障碍,少数是在直肠检查时偶被发现。

根据其发生的来源,先天性精囊囊肿可分为精囊本身和胚胎期中肾管发育异常形成的囊肿两类。男性胚胎在发育过程中,中肾旁管,中肾管以及存留的一些中肾小管衍变成有用结构或退化无用,它们的某些部分常形成一些管状或泡状残余结构存留在睾丸、附睾或精囊的组织中,有的形成囊肿;有的则在生后的长期生活中,因某些因素引起异常增生而形成囊肿。近来文献报告精囊囊肿的发生与常染色体显性遗传的成人多囊肾病(adult polcystic kidney disease,APKD)有关,亦有人(Varney,1954)认为此类囊肿相当于短缩的输尿管或认为是残留的输尿管芽发展成的一囊肿样憩室,故有人称之为“假性精囊囊肿”。

囊肿多为单发,大小不等,最大者容量数千毫升,有报道达5000ml者,国内国鸿钧报告1例囊液达2500ml,有些病例射精管狭窄或闭塞,可并发感染、结石等。国内报道合并结石7例,1例多达157枚。结石多为磷酸钙盐,内含果糖成分,王树森等利用能谱仪分析一例精囊结石,见各层化学成分均为水草酸钙石,实为罕见。镜检囊肿壁为胶原结缔组织,内衬一层立方上皮,可见炎性反应,巨大精囊囊肿可造成输尿管梗阻。

主要症状有血精,射精后会阴部疼痛,尿道血性分泌物,血尿、尿频、尿痛,较大者可有腹部疼痛、排尿障碍,直肠指诊或双合诊可触及囊性肿物。

精囊囊肿通常不需要治疗,如果患者症状明显,可经会阴引流囊肿。如果合并输尿管口异位,可行肾输尿管切除术,精囊肿大且症状明显时,可手术切除。

Lowley(1929)首先报告前列腺异位,提出在前列腺正常部位以外发现的前列腺组织称之为前列腺异位(ectopia of prostate)。前列腺异位可出现在不同部位,如膀胱三角区、膀胱颈、膀胱壁肌层、阴茎根部、残留脐尿管的末端,前列腺部尿道内。张正午等报告巨大盆腔前列腺异位(重400g),张定金等则报告1例包皮前列腺异位,实属少见。膀胱内的前列腺常被误诊为膀胱肿瘤,前列腺部尿道内的前列腺异位,往往以“尿道息肉”的形态出现。

胚胎的第12周,前列腺囊形成,尿道前列腺的内胚层细胞的外突形成前列腺原基,这些原基分成5组,构成前列腺的原始叶,这些原始叶至成年期才合并,其界限消失,除了上述前列腺原基以外,中肾旁管以及中肾管退化后的一些残余组织在膀胱颈等部位形成与前列腺性质相同但较小的尿道上皮旁支,这些较小的旁支通常无生理功能,不发育成前列腺,但在某一特定的环境条件下,它们可能进入发育状态,引起前列腺异位的发生。

前列腺异位在临床上常表现为血尿、尿频或排尿梗阻。Butterick(1971)总结报告的一组68例尿道内前列腺异位患者,96.5\%有血尿,年轻的男性血尿患者必须考虑该病的可能性。其次为尿频和排尿梗阻,血凝块堵塞有时可导致急性尿潴留。盆腔内的前列腺异位病史一般较长,体积大,多于压迫膀胱或膀胱颈部引起排尿梗阻后发现。

无症状的前列腺异位可保守观察,无须手术。由于尿道内前列腺异位多呈息肉样形态,单纯电灼治疗即可治愈,术后无复发和恶性变。膀胱内的前列腺异位亦可行电切或电灼,较大的膀胱内前列腺,可行开放摘除术。

前列腺囊肿(cyst of prostate)有先天性囊肿和后天性囊肿之分,而先天性囊肿又有两种情况,一种是前列腺的囊上发生的囊肿,称前列腺囊囊肿,另一种为前列腺本身存在的先天性囊肿,前者较后者多见。

前列腺囊肿先天性者发生于中肾管或副中肾管系统的残余部分,囊壁由纤维肌组织构成;后天性囊肿多由于前列腺腺泡不完全梗阻形成潴留性囊肿,腺泡上皮变薄,囊肿壁为由立方上皮覆盖的纤维肌组织,囊液为黏液,可有精子。

患者的症状依囊肿的大小而不同,轻者可完全无症状,重者可表现为尿急、尿频、排尿困难、尿线细、残余尿多及尿潴留,血尿极少见,囊肿内并发感染时,可出现脓尿。

囊肿较小无症状者可随诊观察,亦可经会阴或直肠穿刺抽吸,但易感染和复发;较大囊肿经耻骨后或经会阴手术切除,对于突入膀胱的囊肿,可经膀胱手术切除或经尿道电切,去除大部分囊肿顶部,使其充分引流。

(刘勇,马乐)


\section{第三节 女性生殖器官先天性疾病}

女性生殖器官在胚胎期发育形成过程中,其各部位均可受到某些内源性因素(生殖细胞染色体不分离、嵌合体、核型异常等)或外源性因素(使用性激素药物)的影响,导致发育异常。生殖器官发育异常常合并泌尿系统畸形。

女性生殖器官发育异常有时在出生时发现外生殖器异常而得到诊断,其余多在青春期因原发性闭经、腹痛或婚后性生活困难、流产或早产就医时得到诊治。

处女膜闭锁(imperforate hymen),又称无孔处女膜,临床上较常见。系发育过程中泌尿生殖窦上皮未能向前庭部分穿透所致。在青春期月经初潮前无任何症状。初潮后由于处女膜闭锁致使经血无法排出,形成阴道积血,多次月经来潮后,经血逐渐积聚,发展成为宫腔积血,输卵管积血,甚至腹腔内积血。输卵管伞端多因积血而粘连闭锁,故经血较少进入腹腔。

绝大多数患者临床表现为青春期无月经来潮,同时出现进行性加剧的周期性下腹痛。多次反复发作后,可形成下腹部包块,继续发展时可伴有肛门坠胀、便秘、尿潴留等。检查时见处女膜向外膨隆,表面呈紫蓝色,无阴道开口。肛诊可扪及阴道内有球状包块向直肠前壁突出,如扪及盆腔肿块,用手指按压肿块可见处女膜向外膨隆更明显,伴压痛。盆腔超声检查能发现子宫及阴道内有积液。确诊后应立即手术治疗。以手术切开处女膜为最佳,时机选择在青春期,宜早不宜晚,以避免并发症的发生。先用粗针穿刺处女膜正中膨隆部,抽出褐色积血证实诊断后,将处女膜做“×”形切开,引流积血。排除大部分积血后,常规检查宫颈是否有异常。吸尽积血后,切除多余的处女膜瓣,使开口呈圆形,切口边缘黏膜予以3-0可吸收线缝合,保持引流通畅和防止创缘粘连。术后给予广谱抗生素及外阴护理。

阴道上2/3起源于副中肾管、副中肾结节,下1/3起源于尿生殖窦,在发育融合的管腔化过程中,可形成不同形式的异常。

系因双侧副中肾管发育不全或双侧副中肾管尾端发育不良所致,病因不清楚。卵巢功能多正常。临床表现为青春期后无月经来潮,或无法进行性生活,外阴及第二性征发育一般正常,但无阴道开口,或仅有浅的凹陷,偶见短浅阴道盲端。多合并无子宫或仅有痕迹子宫,如子宫发育正常可有原发性闭经、周期性下腹痛、宫腔积血等。直肠-腹部扪及增大、压痛的子宫。临床应与完全型雄激素不敏感综合征相鉴别。后者染色体核型为46,XY,阴毛和腋毛极少,血睾酮值升高。

处理包括机械扩张法和阴道成型术。对准备结婚的先天性无阴道患者,有短浅阴道且组织疏松者可先用机械扩张法,即按顺序由小到大使用阴道模具局部加压扩张,直至能满足性生活要求为止。阴道模型夜间放置日间取出。不宜机械扩张或机械扩张无效者行阴道成型术。手术应在婚前为宜,对有发育正常子宫的患者,应尽早作阴道成型术,同时引流宫腔积血并将人工阴道与子宫相通,以保留生育功能。具体术式各有利弊,但以下几种方式可供选择:①乙状结肠代阴道成型术;②皮瓣阴道成型术;③羊膜阴道成型术;④盆腔腹膜阴道成型术等。

系因泌尿生殖窦未参与形成尿道下段所致。闭锁位于阴道下段,长约2~3cm,其上多为正常阴道。症状与处女膜闭锁相似,无阴道开口,但闭锁处黏膜表面色泽正常,亦不向外膨隆,肛查扪及向直肠凸出的阴道积血包块,其位置较处女膜闭锁高。治疗应尽早手术。术时应先切开闭锁段阴道,并游离积血下段的阴道黏膜,再切开积血包块,排净积血后,利用已游离的阴道黏膜覆盖创面。术后定期扩张阴道以防瘢痕挛缩。

系因两侧副中肾管融合后的尾端与尿生殖窦相连接处部分贯通或完全未贯通,可发生于阴道的任何部位,以中、上段交接处居多,由纤维肌组织组成,外覆鳞状上皮,厚薄不一,一般为1cm左右。阴道横膈无孔称完全性横膈,较少见,多数是隔中央或侧方有一小孔,称不全性横膈。位于阴道上端的横膈多为不全性横膈;阴道下部的横膈多为完全性横膈。横膈位于上段者不影响性生活,常系妇科检查时被偶然发现;位置较低者少见,多因性生活不满意而就医。阴道分娩时影响胎先露下降。完全性横膈有原发性闭经伴周期性腹痛,并呈进行性加剧。妇科检查见阴道较短或仅见盲端,横膈中部可见小孔。肛诊可扪及宫颈及宫体。完全性横膈由于经血潴留,可在相当于横膈上方部位触及块物。

一般将横膈作放射状切开并切除其多余部分,最后缝合切缘以防粘连形成。术后短期放置阴道模型,定期更换,防止瘢痕挛缩,直到上皮愈合。也可先用7、8号穿刺针定位,抽出积血后再行切开术。若系分娩时发现横膈阻碍胎先露下降,横膈薄者,当胎先露部下降至横膈处并将横膈撑得极薄时,将其切开后胎儿即经阴道分娩,横膈厚者应行剖宫产。

为两侧副中肾管融合后尾端中隔未消失或部分消失。分为完全纵隔或不全纵隔,偶见斜隔。完全纵隔形成双阴道,常合并双宫颈、双子宫。有时纵隔偏向一侧形成阴道斜隔(图15-2),导致该侧阴道完全闭锁,可出现因经血潴留所形成的阴道侧方包块。阴道完全纵隔者无症状,性生活和阴道分娩无影响;不完全纵隔者可有性交困难或不适;潴留在斜隔盲端的积血可继发感染;另一些可能至分娩时产程进展缓慢才确诊。阴道检查可见阴道被一纵行黏膜壁分成两条纵行通道,黏膜壁上端近宫颈,完全纵隔下端达阴道口,不全纵隔未达阴道口。若斜隔妨碍经血排出或纵隔影响性生活时,应将其切除,创面缝合以防粘连。若临产后发现纵隔阻碍胎先露部下降,可沿隔的中部切断,分娩后缝合切缘止血。因阴道纵隔不孕的患者切除纵隔可能提高受孕机会。

先天性宫颈闭锁(congenital atresia of cervix)罕见。若患者子宫内膜有功能,青春期后可因宫腔积血而出现周期性腹痛,经血还能经输卵管逆流入腹腔,引起盆腔子宫内膜异位症和子宫腺肌病。治疗时手术穿透宫颈,使子宫与阴道相通,若宫颈未发育,行子宫切除术。

系因两侧副中肾管中段及尾段未发育,常合并无阴道,但卵巢发育正常,第二性征不受影响。常因青春期后无月经就诊,肛诊不能扪及子宫。

又称为痕迹子宫,系因两侧副中肾管融合后不久即停止发育,常合并无阴道。子宫较小,仅长1~3cm,多数无宫腔或为一实体肌性子宫,偶见始基子宫有宫腔和内膜。卵巢发育可正常。多无临床症状,若宫腔闭锁或无阴道者可因月经血倒流出现周期性腹痛,需手术切除。

又称幼稚子宫(infantile uterus),系因副中肾管融合后短时期内即停止发育。卵巢发育正常。子宫较正常小,有时极度前屈或后屈。宫颈呈圆锥形,相对较长,宫体与宫颈之比为1∶1或2∶3。患者月经量较少,婚后不孕。肛诊可扪及小而活动的子宫。治疗方法用小剂量雌激素加孕激素序贯用药,一般可自月经第5日开始每晚口服结合雌激素0.625mg或戊酸雌二醇2mg,连服21日,服药后11日加服醋酸甲羟孕酮8~10mg,每日1次,连用10日,共服6~12周期,定期测子宫径线。

临床上较常见。常见类型见图15-23。

图15-23 子宫发育异常

系因两侧副中肾管完全未融合而形成的两个子宫及两个宫颈,各自有圆韧带、输卵管、卵巢。两个宫颈可分开或相连,宫颈之间有交通管,双子宫可伴有阴道纵隔或斜隔。临床多无自觉症状,通常在人工流产、产前检查甚至分娩时偶然发现。伴有阴道纵隔者可有性生活不适。检查可扪及子宫呈分叉状。宫腔探查或子宫输卵管碘油造影可见两个宫腔。妊娠者在妊娠晚期胎位异常率增加,分娩时未孕侧子宫可能阻碍胎先露下降,子宫收缩乏力较多见,使剖宫产率增加。偶见两侧子宫同时妊娠,属双卵受精。双子宫一般不予处理,伴阴道不全纵隔或斜隔应作隔切除术。

系因两侧副中肾管融合不全所致。子宫底部融合不全呈双角者,称为双角子宫;子宫底部稍下陷呈鞍状,称为鞍型子宫。双角子宫一般无症状,检查可扪及宫底部有凹陷,凹陷深者可能为双角子宫,浅者可能为鞍状子宫。B型超声、子宫输卵管碘油造影有助于诊断。一般不予处理。若双角子宫反复发生流产者,应行子宫整形术。

系因两侧副中肾管融合不全,在宫腔内形成中隔。从子宫底至宫颈内口将宫腔完全隔为两部分为完全中隔;仅部分隔开为不全中隔。一般无临床症状,但易发生不孕、流产、早产和胎位异常,若胎盘附着在隔上,可出现产后胎盘滞留。中隔子宫外形正常,经子宫输卵管造影或宫腔镜检查确诊。对有不孕和反复流产的中隔子宫患者,可在腹腔镜监视下通过宫腔镜切除中隔,术后宫腔内放置金属IUD,防止中隔创面形成粘连,数月后取出。

系因一侧副中肾管发育,另一侧副中肾管未发育或未形成管道。未发育侧的卵巢、输卵管、肾常同时缺如。检查可见单角子宫偏小、梭形,偏离中线。无临床症状时不予处理。妊娠可发生在单角子宫,但反复流产、早产较多见。

系因一侧副中肾管发育正常,另一侧副中肾管中下段发育不全形成残角子宫,可伴有该侧泌尿系发育畸形。65\%单角子宫合并残角子宫。根据残角子宫与单角子宫解剖上的关系,分为3类:①残角子宫有宫腔,并与单角子宫腔相通。②残角子宫有宫腔,但与单角子宫腔不相通。③实体残角子宫,仅以纤维带与单角子宫相连。若残角子宫内膜无功能,一般无症状,不需治疗;若内膜有功能且与正常宫腔不相通时,往往因宫腔积血而出现痛经,甚至并发子宫内膜异位症,需切除残角子宫。若妊娠发生在残角子宫内,人工流产时无法探及,至妊娠16~20周时破裂而出现典型输卵管妊娠破裂症状,若不及时手术切除破裂的残角子宫,患者可因大量出血而死亡。

输卵管发育异常有:①输卵管缺失或痕迹:输卵管痕迹(rudimentary fallopian tube)或单侧输卵管缺失系因该侧副中肾管未发育所致。常伴有该侧输尿管和肾脏的发育异常。双侧输卵管缺失常见于先天性无子宫或始基子宫患者。②单侧(偶尔双侧)副输卵管:为输卵管分支,具有伞部,内腔与输卵管相通或不通。③输卵管发育不全、闭塞或中段缺失:类似结扎术后的输卵管。

输卵管发育异常可能是不孕原因,亦可能导致输卵管妊娠,因临床罕见,几乎均在其他疾病手术时偶然发现。除输卵管部分节段缺失可整形吻合外,其他均无法手术。希望生育者需借助辅助生殖技术。

卵巢发育异常因原始生殖细胞迁移受阻或性腺形成移位异常所致,有以下几种情况:①卵巢未发育或发育不良:单侧卵巢缺失见于单角子宫。双侧卵巢缺失极罕见,一般为卵巢发育不全,卵巢外观细长条索状,白色质硬,又称条索状卵巢(streak ovary),切面仅见纤维组织,无卵泡,甚至仅为条状痕迹。临床表现为原发性闭经或初潮延迟,月经稀少,第二性征发育不良。常伴内生殖器或泌尿器官异常。多见于特纳(Turner's syndrome)综合征患者。B型超声、腹腔镜有助于诊断,必要时行活组织检查。②异位卵巢:卵巢形成后仍停留在原始生殖嵴部位,未下降至盆腔内。卵巢发育正常者无症状。③多余卵巢:罕见,多余卵巢一般远离卵巢部位,可位于腹膜后。无症状,多在其他疾病手术时发现。若条索状卵巢患者染色体核型为XY,卵巢发生恶变的频率较高,确诊后应予切除。


\section{第四节 两性畸形}

两性畸形(hermaphroditism)也称两性体,为胚胎期分化异常所致,患者同时具有男女两性器官,是一种先天性生殖器官发育畸形合并第二性征异常的特殊类型。两性畸形患者性别可根据性染色体、生殖腺结构、外生殖器形态以及第二性征加以区别。可分为真两性畸形(true hermaphroditism)和假两性畸形(pseudohermaphroditism),两性畸形的类型繁多、表现各异,可能对患儿的抚育、心理、生活、工作和婚姻等带来诸多困扰,必须及早确定性别畸形的实质,适时进行治疗。

外生殖器出现两性畸形,均是胚胎或胎儿在宫腔内接受过高或不足量雄激素刺激所致。根据其发病原因可分为:女性假两性畸形、男性假两性畸形和生殖腺发育异常3类。生殖腺发育异常包括真两性畸形、混合型生殖腺发育不全和单纯型生殖腺发育不全。

即女性男性化,患者染色体核型为46,XX,生殖腺为卵巢,内生殖器包括子宫、卵巢和阴道均存在,但外生殖器出现部分男性化。男性化程度取决于胚胎暴露于高雄激素时期早晚和雄激素剂量,可从阴蒂中度粗大直至阴唇后部融合和出现阴茎。雄激素过高是主要原因,或是先天性肾上腺皮质增生,或是非肾上腺来源。

(1)先天性肾上腺皮质增生(congenital adrenal hyperplasia,CAH):又称为肾上腺生殖综合征(adrenogenital syndrome),是一种常染色体隐性遗传病,是女性假两性畸形最常见的类型。其基本病变为胎儿肾上腺合成皮质醇的某些酶缺乏,不能将17α-羟孕酮转化为皮质醇,以21-羟化酶缺乏最常见。皮质醇合成量减少对下丘脑和垂体负反馈作用消失,导致垂体促肾上腺皮质激素(ACTH)分泌增加,刺激肾上腺增生,促使其分泌皮质醇量趋于正常,但同时也刺激肾上腺网状带分泌过多的雄激素,使女性胎儿外生殖器不同程度男性化。通常患者出生时即有阴蒂肥大,阴唇融合遮盖阴道口和尿道口,仅在阴蒂下方见一小孔,尿液由此排出。严重者两侧大阴唇肥厚,形成皱褶,并有程度不等的融合,状似阴囊,但其中无睾丸;子宫、卵巢、阴道均存在,但阴道下段狭窄,难以发现阴道口。随女婴发育,男性化日益明显,阴毛和腋毛出现较早,至青春期乳房不发育,内生殖器发育受抑制,常无月经。虽幼女期身高增长快,但因骨骺愈合早,至成年时反较正常妇女矮小。实验室检查:血雄激素含量增高,血皮质醇偏低,尿17-酮增高,血雌激素、FSH均降低,血清ACTH及17α-羟孕酮均显著升高。

(2)孕妇在妊娠早期服用具有雄激素作用的药物:人工合成的孕激素、达那唑或甲基睾酮等都有不同程度的雄激素作用,若用于妊娠早期保胎或服药过程中同时受孕,均可导致女胎外生殖器男性化,类似先天性肾上腺皮质增生所致畸形,但程度轻,且在出生后男性化不再加剧,至青春期可出现月经来潮,还可有正常生育。血雄激素和尿17-酮值均在正常范围。因出生后不再有雄激素的影响,除外生殖器明显畸形需及早矫治外,一般不需要治疗。

假两性畸形是遗传性别、性腺性别和表型性别的不均一性,即性腺性别与遗传性别一致,而生殖导管和尿生殖窦的发育却具有异性的成分或兼有两性的特征。以遗传性别和性腺性别为基础,假两性畸形又可再分为女性假两性畸形(female pseudohermaphroditism)和男假两性畸形(male pseudohermaphroditism)。

男性假两性畸形,也称雄激素不敏感综合征、睾丸女性化综合征,基本特点是核型为46,XY,如果能发现性腺,肯定是睾丸,但外阴男性化不全,模棱两可,或是完全女性化。由于在胎儿时正常男性化的过程非常复杂,男假两性畸形有许多种类型。

造成男性假两性畸形主要原因有:一是雄激素作用不全,其合成睾酮正常,但雄激素发挥作用出现异常;二是睾丸激素合成缺陷,当间质细胞分化障碍或酶有遗传性缺陷时,可导致46,XY的胎儿的中肾管男性生殖管道和外生殖器分化不完全,从而出现男假两性畸形;三是Y染色体结构异常或基因突变,或者其他染色体异常导致性腺发育不全。

男假两性畸形的发生机制与下列雄激素作用环节缺陷相关:①雄激素生成减少,如3β-HSD、17β-HSD和5α-还原酶缺陷;②雄激素受体(AR)基因突变,引起AR数量减少和功能障碍;③5α还原酶缺陷,不能将睾酮转化为二氢睾酮(DHT);④靶细胞内雄激素代谢异常等,其中AR突变是引起完全型和不完全型睾丸女性化的主要原因。

人类雄激素受体(AR)基因位于Xq11~12,现已发现AR突变基因200余种,其中基因单碱基突变占90\%以上。AR基因突变引起AR结合力降低和DNA结构异常,包括完全性基因缺失、编码雄激素结合区或DNA结合区的外显子缺失和点突变等。雄激素作用机制见图15-24。

睾丸女性化患者的染色体性别是46,XY,常呈家族性发病,家系分析为X连锁隐性遗传病。在X染色体长臂上近着丝粒处—雄激素受体基因发生突变,靶细胞缺乏与雄激素结合的特异蛋白,虽存有生物活性的雄激素,但不与之结合而失去反应性。下丘脑也对雄激素不敏感而失去负反馈机制,导致垂体分泌大量促性腺激素刺激间质细胞增生。现已证实雄激素受体基因位于Xq11~12,长度大于90kb,有8个外显子,外显子1占据整个氨基端,具有激发转录的功能。睾丸女性化患者大多数是基因点突变或碱基对的缺失,导致雄激素受体发生缺陷。致使男性外生殖器转化受阻,向女性表型分化而发病(图15-24)。此类患者是由于靶组织缺乏雄激素受体,故对睾酮不敏感所致,是男假两性畸形中最常见的类型,发生率在新生儿中约为1∶12万。有家族发病史。睾丸在显微镜下像发育期前的隐睾,曲细精管变细,管内充满支持细胞和未成熟的生精细胞,精子生成障碍,但间质细胞仍增生,这种睾丸易发生恶变。

图15-24 雄激素作用原理

又称5α-还原酶缺乏症,遗传学研究证实,患者染色体核型为46,XY。父母表型正常,发病随血缘通婚而增加。家系分析大都可追溯到共同祖先患病。两性都出现酶的异常缺陷,并且两性都有表型正常的基因携带者,支持常染色体隐性遗传。

已知在对雄激素敏感的靶组织中睾酮经5α-还原酶的作用,转变为双氢睾酮而起作用;而睾酮及双氢睾酮在内生殖器向男性分化过程中是不可缺少的。若男性胎儿在发育早期缺乏5α-还原酶,则靶组织中双氢睾酮缺乏而引起外生殖器发育异常,常呈女性或男女难分。

根据5α-还原酶活性测定的结果,本病可分为酶缺乏和酶不稳定两类。基因分析结果表明,酶活性缺乏的原因是基因的突变与缺失造成的,或者造成酶不能与睾酮结合,或者影响了酶的功能;或是酶编码基因以外的突变影响基因的表达所致(图15-25,图15-26)。

图15-25 雄激素受体缺陷病因示意图

图15-26 5α-还原酶缺乏症示意图

患者的外生殖器完全和正常女性相同,阴蒂不肥大,但阴道比较浅,呈盲端。无子宫颈,腹腔内没有子宫及输卵管。原发性闭经。两侧睾丸的位置可在腹腔内、腹股沟部及大阴唇中,以腹股沟部最常见(78\%)。两侧乳房发育肥大,但腺组织较少,乳头发育欠佳体型也呈女性样。本病的另一个特点是患者没有腋毛及阴毛(无毛女人)。

患者生殖道、性腺血浆性激素和促性腺激素、染色体组型、遗传方式等均与完全性睾丸女性化症相同,但外阴部有不同程度的男性化,伴阴毛及腋毛生长。这是因为靶组织雄激素受体的质或量低于正常水平而非完全缺乏,故有不同程度的男性化表现。

(1)Lubs综合征:患者部分中肾管发育,有性毛,呈男性体型后联合外生殖器倾向于女性。

(2)Gilbert-Dreyfus综合征:患者呈男性体型阴茎较小,伴有尿道下裂,部分中肾管发育,乳房肥大,男性化程度为中间型。

(3)Reifenstein综合征:男性外阴阴茎短小,有不同程度的尿道下裂,阴囊分叉,发育期乳房肥大,有性毛,不育。

(4)Rosewater综合征:男性外阴部及生殖管,发育期乳房肥大脂肪分布呈女性有性毛,不育。

(5)假阴道会阴阴囊型尿道下裂:这类患者由于5-α还原酶缺陷,睾酮不能转化为5-α双氢睾酮,其结果其阴茎像阴蒂,尿道开口在会阴部有一浅的阴道。中肾管像正常男性一样分化,即有精囊、输精管、附睾、射精管开口在生殖窦。睾丸在腹股沟或分叉的阴囊内。成年患者精液内可有成熟的精子及各类精细胞。在发育期患者出现男性第二性征:肌肉发达,声音低沉。乳房不发育,阴茎增大,能勃起和射精。激素测定示血睾酮值和正常男性相同而5-α双氢睾酮值降低。患者包皮、阴茎海绵体等组织中几乎不能测到5-α双氢睾酮,表明在这些组织中缺乏5-α还原酶,致使尿生殖窦及外阴部不能充分地向男性方向发育。根据患者家谱的分析,该患者的遗传方式可能是常染色体隐性遗传。

完全型睾丸女性化根据染色体核型46、XY,女性表型,子宫缺如,腹股沟疝无阴毛,隐睾和男性血清睾酮测定可确定完全型睾丸女性化诊断。

不完全性睾丸女性化与完全型睾丸女性化基本相同,不同点在于出现部分男性化症状和体征。

睾丸女性化综合征应予鉴别诊断的相关疾病包括:

即先天性无阴道畸形。MRH综合征染色体核型为46、XX,双侧性腺为卵巢,无阴道。但有阴毛和腋毛发育,内生殖器为女性,血清雌激素浓度正常。

染色体核型为46、XX或46、XY或嵌合型,双侧性腺为睾丸/卵巢/卵睾,两性内外生殖器发育,外阴畸形,表型为女性或男性或畸形。

即先天性睾丸发育不全,染色体核型为47,XXY可以确诊。

儿童患者易与17β-羟脱氢酶缺乏症相混淆,后者属睾酮生物合成缺陷,同样具有假阴道、会阴、阴囊型尿道下裂症的特征,但本病血浆睾酮正常,青春期有进行性男性发育,可排除睾酮生成不足。

因为患者的外生殖器为完善的女性,并按女性抚养,故治疗原则仍维持女性性别治疗。治疗上争论点是什么时候切除双侧睾丸,因为患者的睾丸易发生恶变,但这种睾丸在发育期能引起女性第二性征的形成。目前认为,在20岁以前睾丸发生恶变较罕见,20岁以后恶变率逐渐增高,年过30岁以后,睾丸恶变发生率可高达25\%左右。所以一般都主张在发育期后再将双侧睾丸切除,这样既可使女性化更为完善,又可避免睾丸的恶性变。由于切除睾丸后患者常出现类似女性绝经期的症状,如面色潮红、乳房缩小、阴道上皮萎缩等,因此应长期给予雌激素来替代治疗。对于阴道较浅或狭窄而妨碍性交者,应适当时延长或扩大。不宜告诉患者生殖腺为睾丸,只宜说不能生育,以免患者精神上遭受难以医治的创伤。

不完全性睾丸女性化症的治疗比完全性类型复杂,因其外阴部的变化比较大。如Reifenstein综合征及Rosewater综合征患者的外生殖器几乎完全为男性,而且都当做男性抚育治疗以纠正尿道下裂,切除发育的乳房组织补充男性激素即可。Lubs综合征因外生殖器倾向于女性常常当做女性抚育,应根据性分化异常治疗原则处理。对于假阴道会阴阴囊型尿道下裂,因患者在发育期出现明显的男性第二性征,诊断明确后,应当做男性给予相应的处置。

在孕期,特别是孕早期,孕妇如服用孕激素、雌激素及雄激素,均有可能使女性胚胎或胎儿男性化。

服用孕激素使女性胚胎或胎儿男性化的几率远较雌激素和雄激素高,这是因为孕期使用性激素多是因治疗先兆流产,在治疗时往往多首选孕激素。据Jchuzuk(1964)的统计,在2421例曾使用大量的黄体酮的孕妇中有27名女胎男性化,约为1.16\%,若按所生胎儿的一半为男性的比例计算其发生率在2\%左右。

母亲和外源性雄激素增多引起的性分化异常,包括第1孕季应用具有潜在性雄激素活性的孕激素(19-去甲基睾酮衍生物,包括炔孕酮、炔诺酮、异炔诺酮达那唑、孕三烯酮、左炔诺酮以及甲羟孕酮)和雄激素制剂,引起男性化第2孕季应用仅引起阴蒂肥大,尿生殖窦畸形。妊娠期母体患有男性化肿瘤包括男性细胞瘤(arrhenoblastomas)、转移性卵巢癌(Krükenberg tumors)、黄体瘤(luteomas)、类脂瘤(lipoid tumors)、间质细胞瘤(stromal cell tumors)和间质细胞增生症(hyperthecosis)也可引起女胎男性化。

在孕期服用性激素促使男性化的发生与下列因素有关:如性激素的种类及剂量;孕期的早晚;胎盘的渗透性及其代谢作用;孕妇对外源性激素的代谢及降解功能;以及胎儿生殖系统组织对性激素的敏感度和代谢功能障碍等。

治疗先兆流产时服用的孕激素多为人工合成制剂,其生物作用特性不仅有孕激素的作用,还有雄激素的作用,而且合成孕激素的作用较自然的黄体酮强10倍以上,其排泄缓慢,作用时间较长,故易导致女胎男性化。服用雌激素致女胎男性化是因为合成的雌激素可使黄体酮的代谢紊乱而产生雄激素所致,以及合成的雌激素可以刺激胎儿的肾上腺皮质,促其分泌雄激素。

在孕期的不同时期服用雌激素,对女胎男性化的影响亦不一致。如果在孕12周以前服用性激素,唇囊的合闭比较明显;如果使用性激素是在孕12周以后,阴蒂的增大比较明显,有时阴道口开口于尿道,但较少见。服雄激素后的女胎男性化的程度似乎较轻,而且生后增大的阴蒂可以逐渐缩小。

诊断依据:①女性婴儿出生后外生殖器两性畸形。②母亲在妊娠期间曾有应用性激素制剂的历史。③母亲有或无男性化表现。④影像学检查卵巢或肾上腺发现肿瘤。⑤女婴出生后的生长发育正常,并且有正常的青春期发育,没有E2合成障碍引起的体格和代谢异常。

外观与女性假两性畸形相似。但性染色质阴性,染色体组型为46,XY。

有时外生殖器像女性,与女性假两性畸形相似但其内外生殖器模棱两可,性腺活检既有睾丸又有卵巢组织,且24小时尿17-酮类固醇正常。

男性假两性畸形即男性女性化。患者染色体核型为46,XY。生殖腺为睾丸,无子宫,阴茎极小、生精功能异常,无生育能力。男性假两性畸形系因男性胚胎或胎儿在母体缺少雄激素刺激发育。发病机制:①促进生物合成睾酮的酶缺失或异常;②外周组织5α-还原酶缺乏;③外周组织和靶器官缺少雄激素受体或受体功能异常。男性假两性畸形者睾丸可分泌雄激素,但机体对雄激素不敏感,故临床将其称为雄激素不敏感综合征(androgen insensitivity syndrome),属于X连锁隐性遗传,常在同一家族中发生。根据外阴组织对雄激素不敏感程度,又分为完全型和不完全型两种。

(1)完全型:又称为睾丸女性化综合征(testicular feminization syndrome),出生时外生殖器为女性型,有睾丸存在者多为隐睾。患者体内睾酮经芳香化酶转化为雌激素,至青春期乳房发育丰满,但乳头小,乳晕较苍白,阴毛、腋毛多缺如,阴道为盲端,较短浅,无子宫。两侧睾丸正常大,位于腹腔内、腹股沟或偶在大阴唇内。血睾酮、FSH、尿17-酮均为男性正常值,血LH较正常男性增高,雄激素略高于正常男性。

(2)不完全型:较完全型少见,外阴多为两性畸形,表现为阴蒂肥大或阴茎短小,阴唇部分融合,阴道较短或仅有浅凹陷。至青春期可出现阴毛、腋毛增多和阴蒂继续增大等男性改变。血LH、睾酮水平增高,但也可能出现正常值水平。

真两性畸形也称性分化异常,患者体内同时具有睾丸和卵巢两种生殖腺,称为真两性畸形,是两性畸形最罕见一种。可能一侧生殖腺为卵巢,另一侧为睾丸;或每侧生殖腺内同时含卵巢及睾丸两种组织,称为卵睾(ovotestis);也可能一侧为卵睾,另一侧为卵巢或睾丸。染色体核型多为正常女性型46,XX;其次为嵌合型46,XX/46,XY;男性型46,XY较少见。内外生殖器可能具有男女两性特征,同时分泌雄激素及雌激素,而以其中一种占优势。临床表现与其他两性畸形相同,外生殖器多为混合型,但多有能勃起的阴茎,而乳房几乎均为女性型。核型为46,XX者,体内雌激素水平达正常男性两倍。多数患婴出生时阴茎较大,往往按男婴抚育。但若能及早确诊,绝大多数患者仍以按女婴抚育为宜。个别有子宫患者在切除睾丸组织后不但月经来潮,还具有正常生育能力。

生殖导管和外生殖器往往为两性畸形。真两性畸形生殖腺必须是完整的,即睾丸必须有正常的结构,有曲细精管、间质细胞及生殖细胞的迹象;卵巢必须有各种卵泡并有卵细胞生长的现象。至于仅有卵巢或睾丸的残遗组织,不属于真两性畸形。

真两性畸形有3种类型:

1.一侧为卵巢,另一侧为睾丸,称为单侧性真两性畸形,这种类型占40\%。

2.两侧均为卵睾(即在一个性腺内既有卵巢组织又有睾丸组织),卵巢组织与睾丸组织之间有纤维组织相隔称为双侧性真两性畸形,这种类型占20\%。

3.一侧为卵睾另一侧为卵巢或睾丸,这种类型占40\%。

真两性畸形是既有睾丸组织又有卵巢组织的患者,其发病情况各国不同,在西欧和北美少见,但在非洲却是最常见的两性体(Aaronson,1985)。其核型以46,XX最多见,占63\%,Aaronson的一组41例都是46,XX。其次为46,XY占13\%。嵌合体和非整倍体分别为31\%和3\%(Luks等,1988)。

真两性畸形的原因可能是:①单合子性染色体镶嵌,这是减数分裂或有丝分裂错误所致;②非单合子性染色体镶嵌,这往往是两个受精卵融合或两次受精的结果;③Y染色体向X染色体易位;④常染色体突变基因。家族性患者的遗传方式是常染色体隐性或显性遗传。

为常染色体隐性遗传,真两性畸形核型为46,XX占60\%;核型为46,XY占20\%;核型为嵌合体46,XX/46,XY约占20\%。在核型为46,XX的基因组织中,用Y特异性DNA探针未发现Y染色体,故不能用Y→X或Y→常染色体移位或通过性染色性嵌合解释其发病机制。已证明控制性发育和分化的基因可能位于常染色体,有报道46,XX核型两同胞H-Y均为阳性,外生殖器畸形,性腺均为卵睾,但性别为一男一女,据认为由父方传递而得,属常染色体显性遗传。

从XX真两性畸形卵睾的睾丸取细胞培养,可检出H-Y抗原阳性,而取自卵巢部分的培养为阴性,提示卵睾起自H-Y阳性/H-Y阳性的嵌合原基。

真两性畸形核型46,XY的病因学尚待进一步研究,其发病机制类同46,XY不完全性腺发育不良,在睾丸发育的早期,生殖嵴等区域与睾丸发育有关的基因变异,而另一区域则保留向卵巢分化的倾向,因缺乏双X染色体,卵巢组织中原始卵泡加速分化,若有些卵泡保留下来,这种病症叫真两性体,若仅有卵巢基质保留下来,则称为性腺发育不全。因此卵睾和性腺发育不全很可能是同一过程中的不同表现。

患者出生时外阴部男女难分,但比较倾向于男性,约3/4的患儿当作男孩抚育,而阴囊发育不良似大阴唇。性腺大多可在腹股沟部位或阴囊内摸到。患者在发育期一般都出现女性第二性征,如乳房肥大,女性体型,阴毛呈女性样分布,可有月经来潮。这是因为任何核型的真两性畸形都有卵巢组织,而卵巢的结构比较完善,所以大多数真两性畸形的卵巢在发育期可分泌雌激素,有排卵时还分泌孕激素,故可出现女性第二性征,但乳腺的发育较晚。患者大都有子宫及阴道,阴道开口在尿生殖窦,常见的子宫发育障碍有发育不良和子宫颈缺如。

如果性腺是卵巢,则显微镜下一般正常,而睾丸在显微镜下都无精子生成,因此患者可有正常卵巢功能,极少数患者甚至能怀孕。卵睾是最多见的性腺异常,约半数卵睾在正常卵巢位置上,其余一半或在腹股沟或在阴囊内。卵睾所在的部位与其成分有关,睾丸组织所占比例越大越易进入腹股沟或阴囊内。在卵巢一侧的生殖管总是输卵管,睾丸一侧的生殖管都是输精管,至于卵睾一侧的生殖管既可是输卵管也可是输精管,此与卵巢和睾丸组织的成分有关,一般以出现输卵管为多见。

常合并尿道下裂或阴茎系带半数患者并发腹股沟疝和隐睾。

患儿出生后若发现外生殖器异常,应立即请专科医师会诊,尽早作出诊断,不能简单地作出单纯性尿道下裂合并隐睾或阴囊分裂的错误诊断。应作性染色质检查,多数呈阳性,若此项检查不符合正常男性,作染色体核型分析,组织细胞染色体较血细胞染色体核型分析对发现嵌合体更有帮助。对核型为XX者应仔细寻找女性男性化表型的来源,测定各种肾上腺激素、17-酮类固醇、孕三醇、17-脱氢黄体酮以除外常见类型的先天性肾上腺增生。组织学检查发现兼有卵巢和睾丸组织即可明确诊断,但有时因性腺发育不正常造成诊断困难。

单纯从外生殖器难以确定性别,染色体组型亦为46,XX,与真两性畸形表现相似。但24小时尿17-酮类固醇及孕三醇增高,B超、CT检查常可见双侧肾上腺增大或有占位。

单纯从外生殖器难以确定性别,与真两性畸形表现相似。但5α-二氢睾酮偏低,性腺活检只有睾丸组织,无卵巢组织。

只从外生殖器难以确定性别,与真两性畸形表现相似。但染色体组型为47,XXY,性腺活检只有睾丸组织,无卵巢组织。

治疗时所取性别是否恰当对患者身心健康发育至关重要,一般认为2~3岁前确定性别可避免发生心理异常。以往对真两性畸形性别的取向主要根据外生殖器的外形和功能来决定是否行男性或女性矫形手术,而不是根据性腺、内生殖器结构或染色体组型来决定。近年来对真两性畸形,特别是核型为46,XX者,多倾向改造为女性较好。因为:①真两性畸形患者的卵巢组织切片,大多能观察到原始卵泡,50\%有排卵现象,而双侧睾丸曲细精管有精子发生者仅占1.2\%;②真两性畸形患者中70\%乳腺发育良好,24.5\%发育较差,不发育者仅5.5\%;③男性尿道修补外生殖器成型较为困难,且效果不理想,而女性成形术的成活率较男性为高;④核型为45,X/46,XY的患者的隐睾约30\%可发生恶变,睾丸需予以切除。

染色体核型为45,X与另含有至少一个Y的嵌合型,以45,X/46,XY多见。其他如45,X/47,XXY;45,X/46,XY,XXY亦有报道。混合型系指一侧为异常睾丸,多为腹内隐睾,另一侧为未分化生殖腺、生殖腺呈索状痕迹或生殖腺缺如。患者外阴部分男性化,表现为阴蒂增大,外阴不同程度融合、尿道下裂。睾丸侧有输精管,未分化生殖腺侧有输卵管、发育不良子宫和阴道,不少患者有Turner综合征的躯体特征。出生时多以女婴抚养,但至青春期往往出现男性化,女性化者极少。若出现女性化时,应考虑为生殖腺分泌雄激素肿瘤所致。

染色体核型为46,XY,但生殖腺未能分化为睾丸而呈索条状,故无雄激素分泌,副中肾管亦不退化。患者表现为女性,身材较高大,有发育不良子宫、输卵管,青春期乳房及毛发发育差,无月经来潮。

(董晓静)


\chapter{第十六章 性身份障碍}

性身份障碍又称为易性癖病、性别转换症等,是一种性心理障碍。患者幼年(2~3岁)由于性心理和人格发育出现偏差,导致性别的自我认知发生逆转,随着青春期的到来,这种心理状况得到加强,患者强烈地感觉到自己是异性并且深信不疑。他(她)们坚决要求通过手术改变自己的生理性别,并且一旦改变性别的要求得不到满足时,常常会因为心理和生理矛盾的激烈冲突而痛苦万分,最终导致自残和自杀。

我们首先必须关注一个最重要的基本事实—“性(sex)”和“性别(gender)”之间的差别。性是你所看到的,性别则是你所感觉到的。这两者之间的协调对人类的幸福是至关重要的。“性(sex)”是指生理学和解剖学上的性,它包括:性染色体、性腺、性激素和性生殖器官等。也可以称为解剖学性别、生理学性别或者生物学性别,也就是指解剖、生理或生物学上的男女之别。“性别(gender)”则是指心理上的性别或性别自认(gender identity),即一个人对自己是男性还是女性的自我认识,即心理性别。对于绝大多数人而言,其解剖学上的性别与其心理上的性别(性别自认)是协调一致的。但也有极少数人情况相反。一个生物学上的男性或女性个体,尽管他(她)们清楚地知道自己的生物学性别,但却在心理上感觉到自己是异性,并渴望改变自己的生物学性别,性身份识别障碍是指一个人怀有一种强烈而持久的改换性别身份的意识(他们之所以要改变自己的性别,大部分原因是为了作为另一性别生活时能获得社会生存的好处)。

如是儿童,表现为下列4项以上:

(1)反复申述自己想成为另一性别,或坚持认为自己是另一性别;

(2)男孩喜欢换穿女装或耀眼的女性盛装;女孩则坚持一直穿典型男性的服装;

(3)在假扮游戏中强烈而坚持地偏爱另一性别的角色,或坚持幻想成为另一性别;

(4)强烈地希望参加典型的另一性别的游戏及娱乐;

(5)强烈偏爱另一性别的游戏伙伴。如果青少年或成年人则表现为申述自己愿成为另一性别的愿望,往往发誓是另一性别,希望像另一性别那样地生活或要求他们如此对待,或深信自己具有另一性别的典型感受和反应。患者始终对自己的性别感到极不舒服,或者认为自己目前的性别角色很不合适。

如是儿童,表现为下列任一项:如是男孩,断言自己的阴茎或睾丸是令人厌恶的,即将消失,或者断言最好没有阴茎,或者厌恶莽撞性的游戏并拒绝典型男性的玩具、游戏和活动;如是女孩,拒绝坐在那里小便,断言自己有阴茎或会长出一个阴茎,或断言自己不会长乳房或来月经,或厌恶正式的女性服装。如是青少年或成年人,表现为沉湎于设法除去第一及第二性征的想法(例如要求注射性激素、进行手术,或用其他方法来改变现有的性征,以便更像另一性别)或深信自己生错了性别。此障碍并不与躯体上的两性人同时存在。此障碍产生了临床上明显的痛苦、烦恼,或在社交、职业或其他重要功能方面的功能缺损。

临床上绝大多数求治者想改变自己的性别的想法都是与生俱来的,并没有明显的后天的影响存在,而且经过较长时间的心理治疗均告无效。据考证,易性癖病现象在古希腊和古罗马时代的文献中即有记载。检索文献发现,1838年Esgurol即在精神医学文献中有过报道;1886年Kreft Ebing做了更为详细的介绍;到1916年Marcuse对这种现象从精神和性心理等方面进行了初步探讨。20世纪60年代,各国学者对易性癖病的研究已取得初步成果。1963年Edgerton等在Johns Hopkins大学医学院建立了第一个性别自认障碍门诊;1966年Harry Benjamin写出了第一本有关易性癖病现象的专著;1969年Green和Money首次论证了外科手段治疗易性癖病的必要性和科学性,并出版了《易性癖病与性别再赋于外科》一书。但是,对易性癖病的研究取得真正意义上的进展还是近20年的事情。全球第1例性别重塑外科手术(gender comfirmation surgery,简称GCS)始于1931年,当时,被世人视为大逆不道。而目前全球已经施行的性别重塑外科手术已有1万多例,手术后绝大部分患者都获得了新生。世界上一些主要发达国家已就性别重塑外科手术制定了一系列相应的法律法规,并将性别重塑外科手术看做一项常规医疗处理措施,一般由医疗保险公司提供费用。东方国家,尤其是我国周边国家,由于受东方文化影响而比较内敛,对性的问题一向讳莫如深。对易性癖病的研究起步较晚。据查,我国第一例性别重塑外科手术始于1986年(北京),限于国情,当时没有公开报道。公开报道的性别重塑外科手术始于1990年(上海)。随后我国对此领域的研究蓬勃开展,迅速追赶世界水平,有关的文献报道也日益增多。据不完全统计,我国目前已施行性别重塑外科手术将近300例。手术后绝大部分患者顺利融入社会并开始了新的人生。2001年10月,经卫生部和北京市卫生局批准,中国医学科学院整形外科医院成立了,是亚洲第一家,也是我国目前唯一一家专门研究性别畸形和性别障碍的治疗中心———中国性别重塑外科中心。目前,来医院门诊或经各种渠道咨询的患者每年逾千人。由于我国目前还没有有关性别重塑外科手术的法律法规,因此我国性别重塑外科手术的开展十分混乱,近年来,全国范围内(杭州、上海、武汉、成都等地)就出现了不少与性别重塑外科手术有关的刑事案件和法律诉讼。为此,司法部门和卫生部相关部门委托我院专家专门制定了《关于性别重塑外科手术的管理办法和规范(诊疗规程)》。其中有关易性癖病的诊断标准和严格的施行性别重塑外科手术的条件,以及一套完善的法律程序已经出台并报卫生部门备案。目前,这些“标准”和“条件”已经逐渐推广到全国范围并已经成为性别重塑外科手术这一特殊领域里的行业标准。另外,性别重塑外科手术本身是一个系统工程,需要多种学科(包括妇产科、泌尿外科、内分泌科、心理和精神科以及法律工作者等等)相互协作共同参与才能很好地完成,因此,国外把性别重塑外科手术又称为“性别工程”。


\section{第二节 流行病学和病因}

易性癖病现象普遍存在于世界各地,男女皆可发病。据文献检索,有关发病率各国学者的报道极不一致。在美国,男性大约为1∶100 000,女性大约为1∶400 000;在英格兰男性大约为1∶35 000,女性大约为1∶105 000;在新加坡,男性大约为35.2∶100 000,女性大约为12∶100 000。一般认为,男性发病率高于女性,但近年来,有统计资料表明两者基本相近。这里有必要强调,作为一个临床医师,对于要求变性的众多患者,应该怀有高度的社会责任感和一个临床医师的良知去对待他们。必须认真负责,极其严肃慎重地作出诊断并选择最合适的患者,确认万无一失时才可决定施行手术,不可为了其他的目的滥施此类手术,造成不可挽回的悲剧。国外不断有报道手术后后悔的患者。

很久以来,我国几乎没有对易性癖病进行研究的相关记载。人们一直认为易性癖病是一种思想道德败坏、违反自然的亵渎行为,是一种堕落的表现或恶习。其实,易性癖病患者并不完全是出自他(她)们主观上的愿望想改变自己的性别,易性癖病并不是单纯的性身份障碍性心理疾患,易性癖病现象有一定的生物学基础。只是我们人类目前对此种疾病还知之甚少而已。对易性癖病的确切发病原因至今尚不清楚。目前,尽管有多种学说试图解释这种现象,但几乎没有一种学说能够被学者们普遍接受。这里,将主要的几种学说简述如下:

生物学学说认为,易性癖病现象可能是和性别和性激素水平在性中枢分化的这一关键时期内相互脱节的结果。如果雄性激素不足,一个遗传学上的雄性将出现脑的雌性分化。相反,如果雄性激素过剩,一个遗传学上的雌性则出现脑的雄性分化。前者将产生雄-雌易性癖病现象,后者会导致雌-雄易性癖病现象。

两性体学说将易性癖病患者看做是两性体,但并未得到广泛的支持。

脑学说推测在易性癖病患者的大脑中可能存在着一个不受正常皮质控制而放电的中枢,该中枢产生一种像帕金森病和指痛症那样的神经释放现象,认为易性癖病的发生可能与此有关。有学者报道,通过使用抗痉挛药和颞叶切除术治疗易性癖病可获得成功。

心理学学说认为性别自认基本上是学得的,因它把易性癖病的发生归因于儿童时期有害的心理调节。父母对孩子解剖学性别的排斥[因他(她)们不希望孩子有这一性别]、按异性养育孩子、为了惩罚而迫使孩子穿异性服装的错误强化等,均被认为与易性癖病现象的发生有关。有些作者认为,不良的心理调节只有在“体质上的易感倾向”存在时才能导致易性癖病的发生。但是,这种“体质上的易感倾向”的本质是什么还不得而知。

精神分析学说对易性癖病的解释是基于弗洛伊德的观点。认为男-女易性癖病实际上是男性个体害怕遭受阉割的表现,为了克服阉割恐怖,他想象出了一个“带阴茎的女人”,并随后认同她。这种解释无法加以验证,因此,未被普遍承认。

以上是关于易性癖病成因的几种主要说法,除此之外,尚有“器官源性学说”(theory of organic origin)、“母-子情结早期缺陷学说”以及“HY抗原异常学说”等,但没有一种学说可单独用来解释所有易性癖病患者的发病机制。看来这种现象可能系多种因素所致。

一般认为,易性癖病是一种性心理疾患,他(她)们的性染色体和性征并无异常,其体内也无相应的器质性损害存在。值得注意的是,近年来随着分子生物学技术的飞速发展,有关研究已经达到分子水平。故而有学者在“染色体性别”的基础之上又提出了“基因性别”的概念———性别最终是由基因决定的,并以此来解释为什么有人执著地想要改变自己的性别。这也证明了易性癖病确是有一定的生物学基础。


\section{第三节 性别身份障碍的诊断}

古往今来,人们均按照社会约定俗成的用于表现男女差别的社会行为模式来完成自己的性角色。但那种背离自己性别身份的人在社会里亦不难见,他(她)们的性情和行为常常与众不同。日常生活中,人们称呼那些在性情和行为上不那么符合男人标准的男孩子为“娘娘腔”(英美国家称为Sissy);称呼那些在性情和行为上不那么符合女人标准的女孩子为“假小子”(英美国家称为Tomboy)。社会上对于那些行为超出性别范围的女孩子的态度比较容忍,甚至认为这是天真可爱的表现,是与众不同和有出息的表现。而那些“娘娘腔”的男孩子却没有这么幸运。他们的行为举止让人感到恶心难受,是没有出息的表现,往往成为人们讥讽嘲弄的靶子。从转归上看,女孩子虽然常有某种程度的女童男性化,但是,一般在青春期以后都能接受她由遗传或解剖特点所规定的女性身份。也就是说,“假小子”一般在来月经之后以及女性第二性征发育之后,都能面对现实,表现出几分温柔与稳重,并能成为“贤妻良母”。对男孩子来说,跨性别行为却是一个严重的问题,是一种严重的性别身份确定的紊乱,往往是同性恋或易性症的前兆,必须尽早地加以注意及矫治,切不可认为这是个会自行消失的现象而让它拖延下去。性别身份障碍主要有易性症(性别转换症)、双重角色异装症及儿童性别身份障碍等。由于国际分类与美国分类略有不同,现分别介绍如下。

渴望像异性一样生活,被异性接受为其中一员,通常伴有对自己解剖性别的苦恼感及不相称感,希望通过激素治疗和外科手术以使自己的身体尽可能地与所偏好的性别一致。诊断要点:转换性别身份至少应持续存在2年以上才能确立诊断。且不应是其他精神障碍如精神分裂症的症状,也不伴有雌雄同体、遗传或性染色体异常等情况。

个体生活中某一时刻穿着异性服装,以暂时享受作为异性成员的体验,但无永久改变性别的愿望,也不打算以外科手术改变性别。在穿着异性服装时,并不伴有性兴奋,这一点可与恋物性异装症相鉴别。包含:青春期或成年期性身份障碍,非性别转换型。不含:恋物性异装症。

这一障碍通常最早发生于童年早期(一般在青春期前已充分表现),其特征为对本身性别有持续的、强烈的痛苦感,同时渴望成为异性(或坚持本人就是异性)。持续地专注于异性的服装和(或)活动,而对本人的性别予以否认。通常认为这类障碍相对少见,较常见的是与程式化性角色行为不一致的状况,两者不应混淆。只有正常意义上的男性或女性概念出现了全面紊乱时,才可考虑儿童性别身份障碍的诊断。仅有女孩子像“假小子”、男孩子“女孩子气”是不够的。若已进入青春期,此诊断便不能成立。诊断要点:必要的诊断特征为,儿童出现根深蒂固的、持续的成为异性的渴望,伴有对自身性别的行为、特性和(或)衣着强烈的排斥。典型情况下,在入学前就首次出现;要想确立诊断,这一障碍必须在青春期前就已十分明显。在男女两性中,都可能会出现对本身性别的解剖结构的否认,然而上述表现较少见,也许很罕见。性别身份障碍的儿童有一个特点,即尽管他们因与家庭、好友的期望相冲突而苦恼,也因所受到的嘲笑和(或)排斥所痛苦,但他们却否认因性身份障碍而苦恼。对这种障碍的知识,男孩多于女孩。典型的情况是,从上学前数年,男孩就开始沉湎于那些通常属于女性的游戏和活动,而且常常偏爱穿女孩的服装或妇女的衣着。然而,这种换穿异性服装的举动并不会引起性兴奋(并不像成年人恋物性异装症)。他们会有极强的欲望,想参加女孩子的游戏和消遣,洋娃娃常是他们钟爱的玩具,而女孩子一般是他们偏爱的玩伴。上学的头几年,这些孩子会越来越被人孤立,这种情况在童年中期达到顶峰,其他男孩会羞辱、嘲笑他们。明显的女性化举止在青春期早期会有所减轻,但随访研究显示,性别身份障碍的男孩中,有1/3~2/3在青春期及青春期后显露出同性恋倾向。然而,在成年后表现为性别转换症的却极少(尽管有报道说,大多数成年性别转换症者在童年都有性别身份问题)。临床工作中所碰到的性别身份障碍的女孩少于男孩,但这一性别比率在总人口中是否适用尚不得而知。像男孩一样,女孩也通常较早表现出热衷于一般属于异性的一些行为。典型情况下,有性别身份障碍的女孩结交男伙伴,对体育运动和激烈争斗的游戏极为喜爱。她们对洋娃娃没有兴趣,对在假装的游戏如“爸爸和妈妈”或“过家家”中扮演女性角色也不屑一顾。有性别身份障碍的女孩在学校中并不像有性别身份障碍的男孩那样受到同等程度的孤立,然而她们在童年后期或青春期也会遭到嘲笑。大多数女孩在接近青春期时,会放弃对男性活动和服装过分张扬的追求,但是一些人会保留男性性别认同,并逐渐显露出同性恋的倾向。性身份障碍伴有对本身性别的解剖结构的持续排斥的情况是罕见的。不含:自我不和谐的性取向,性成熟障碍。

包含:性角色障碍,未特定。

(1)持久而强烈地为自己是个女孩而苦恼,诉说希望自己是个男孩(并非由于重男轻女的文化所影响),或坚持她是个男孩。

(2)具有以下①或②两者之一:①固执地厌恶标准女式服装,并坚持穿老古板的男式服装,如男孩的内衣裤及其他物品。②坚决否认女性的解剖结构,至少具有以下之一:a.断言她有阴茎,或将要长出阴茎;b.拒绝蹲着或坐着排尿;c.断言她的乳房不会发育,也不会有月经。

(3)此女孩未及青春期。

(1)喜爱女性活动,如喜爱女式服装或作女性装扮;特别愿意加入女孩子的游戏,并拒绝男孩的玩具、游戏及活动。

(2)坚决否认男性解剖结构,反复地断言至少以下之一:①他将来长大会变成一个女人(不仅只是女人的角色);②他的阴茎或睾丸极其讨厌,或将会消失;③如果没有阴茎或睾丸则更好。

(3)此男孩未及青春期。

1.持久地对自己解剖上的性别感到不舒服及不适应。

2.至少有两年坚持要去掉自己的第一、第二性征,并要得到异性的这些性征。

3.此人已达青春期。

1.持久或反复地对自己解剖上的性别感到不舒服及不适应。

2.固执或反复地以异性的角色而穿着打扮(或在幻想中,或在现实生活里),但其目的并不是像异性装扮症那样为了性兴奋。

3.至少有两年不像性别转换症那样坚持要去掉自己的第一、第二性征,以及得到异性的这些性征。

4.此人已达青春期。

1.坚持穿着异性服装的儿童,又不具备儿童期性别身份障碍的其他标准。

2.与紧张刺激有关的成人暂时性异装行为。

3.性别转换症临床相的成人,但时间少于两年。

4.坚持阉割或切除阴茎,却又不想得到异性的特征。

案例1 性别转换症

“我今年24岁,是一位地道的男同性恋者,我生在农村,姐弟三个我最小,在家里一直受宠爱,使我养成了依赖性强,胆小,害羞的女孩性格,上小学时总爱和女孩玩,上中学时尽管和女孩分开了,但是和男同学在一起时就总是觉得不自在,并且见了那些英俊的男子或老师就会耳热脸红、心跳加快,有时在梦中也时常梦见他们。大学没考上,但有机会在大学里进修了两年,我学会了织毛衣,别的男孩子都去会女朋友,我却从没有这种想法。我也从未去过公共浴池,因为男人的每一部位都令我情不自禁。现在我踏上岗位快三年了,有人介绍了一个女朋友,三个多月来只拉了一次手,还是她主动的,我没有一句情话,没有其他要求和动作,因为我一点也爱不起来,尽管如此,她对我还是很好的,我就更内疚了。我觉得对不起她,对不起父母,我也受不了社会的舆论,我有千言万语却无法向任何人倾诉,我觉得我活在世上的确太苦了,我期望有一天能过上真正的女人的生活,这就是我,一个患者的自白。我有一个强烈的愿望是改变我的性别,不知哪个地方能做这种变性手术?术后是否能和女人一样过性生活,以及生儿育女。当我写完这封信,我觉得您已是我最要好的朋友了,因为我的知心话只有向您倾诉。”

案例2 非性别转换症型性别身份障碍

女性,15岁,自述:“我今年15岁,是一个年纪轻轻的姑娘,不知怎的,每当我看见自己腋下和阴道外部的毛及稍稍突起的乳房时,就有一种不高兴,特别是月经期间。说真的,我是很希望自己变成男性的。我也曾多次梦到自己变成了男性。我申明:我从来没有产生过重男轻女的思想。这种心情不知是因啥产生的,但绝对不是重男轻女的封建思想在作怪。不骗你们说,我是很喜欢女同学的,但就不喜欢自己是个女性。”

案例3 双重角色异装症

男性,43岁,自幼年起,他母亲就把他打扮成女孩的模样。儿时随父亲学画人像,特别喜欢画美女像,养成女孩的性格,喜穿款式新颖、色彩鲜艳的女装,喜留长头发,并经常改变发型。在上学期间及当兵期间,虽白天不能穿着女装,每至夜间即从里到外全部换上女装,并抹口红、涂胭脂、学女人步态。由部队转业后,购置大量女式时装、乳罩、月经带、长袜、女式皮鞋等,又缝制丝绒小垫子垫于臀部,在乳罩里填塞海绵,涂脂抹粉,打扮得花枝招展,公开招摇过市。他还为自己拍摄了许多女妆时的相片,闲时自我欣赏。虽然家里人及同事们都反对他的装束,他却满不在乎,说:“这是个人爱好”。阴茎能勃起,并时有遗精。虽交了几个女友,均因不喜欢他的装束打扮而分手。他现仍希望能找到一个称心的女人而结婚。他从无同性恋行为,也没有转变性别的要求。

异装症和性别转换症有源于同性恋者,有主要是异性恋者,也有同时为两性恋者,因而可分为同性恋的异装症、同性恋的性别转换症、异性恋的异装症和异性恋的性别转换症几种。对异性恋的异装症与异性恋的性别转换的区分,有时十分困难。因为它们是个连续的过程,而不是互不相干的两种情况。在行为表现上和在性的取向上,它们很是相似,只不过后者强烈地希望变成异性,并要求做转变性别的手术而已。虽然它们被称为异性恋的性别转换症和异性恋的异装症,实际上其中大多数为两性恋者。他们照样能结婚,并过上家庭生活。只有很少数的一些人乔装打扮成女人之时,才惹得不明底细的男人心动。异性恋的异装症及性别转换症与同性恋者有下列不同。

异性恋的异装症者和性别转换症者虽喜作异性穿着打扮,但多穿戴如普通异性的日常便装。男性同性恋的异装症者和性别转换症者则往往浓妆艳抹,发式时髦,追求外围情调,以显示自己或吸引男人。

异性恋的异装症者和性别转换症者不着异性装束时,若是男人,在行为举止上与正常男人一样。在公共场合,若不注意观察,很难发现他们会有性身份障碍。即使他们着异性服装,大多数人也不会改变自己的行为举止。他们不会像异性装扮的同性恋男人那样模仿女人讲话的腔调和姿态。因而,粗略一看,某些异性恋的异装症者和性别转换症者只不过是个男子气很足的男人穿上女人的服装而已。同性恋的异装症者及性别转换症者则不同,即使他们不着异性装束,许多男人也表现得女里女气的,甚至扭捏作态。

许多异性恋的异装症者和性别转换症的男人只存在与他们生物学性别不同的性别感,而几乎不表现女性的行为特征。除了有异性的性取向之外,他们往往爱好体育活动,不喜欢跳舞和表演,不注意化妆打扮,而且他们还很难学会正确的分妆和挑选合适的服装。同性恋的异装症者和性别转换症的男人特别喜欢而且精于梳妆打扮。他们往往自己设计、自己裁制许许多多女装。

异性恋的异装症者和性别转换症的男人往往选择传统上由男性来做的工作,如工程师、机械工、推销员等。同性恋的异装症者和性别转换症的男人则往往做理发师、美容师、裁缝或娱乐圈里的工作。

异性恋的异装症者和性别转换症的男人通常不会像同性恋者那样公开跳舞或表演。同性恋的男人的女子气愈浓,则愈喜欢这些活动。目前有数以百计的同性恋的异装症的男人和性别转换症的男人在美国及欧洲的夜总会里作色相表演。

自恋症(autogynephilia):类似于恋物症及异装症。这种男性是通过幻想或认为自己是一个女人,以此来达到对性欲的唤起,但他实际上并不渴望通过手术改变性别而成为女人。性生活中通过幻想自己是一个女人来获得性高潮,并希望自己的伴侣作为男性处于主动地位来对待自己。

成人的性别转换症有原发性及继发性两种情况。

自幼年起即已开始。有人认为男性性别转换症与女性性别转换症在病因上有所不同。他们认为,男性性别转换症者未能完全与母亲区分开来,以发展一个男性的身份来看待自己。按照此种说法,母亲如果与男孩在身体上接触过多、过于密切、时间过长,给男孩过多的保护和照顾,就会造成男孩对母亲的性别身份接受得过多,从而妨碍了男孩对父亲的认同作用,使得男孩形成了女性的身份,早期的跨性别身份就由此造成了。此时,除了有性别转换之外,通过心理测验检查是不会发现有任何精神障碍的表现的。女性原发性性别转换症的发展则与上述情况不同。从逻辑上看,早年过长地与母亲共生,不会影响女性的性别身份。虽然对这种女孩的研究不多,但有证据表明,她们的母亲往往情绪抑郁,感情上难以沟通。于是,在早年,她们只得将感情投向照顾她们的父亲(通常为男子气十足的、不会照顾妻子的男人)。她们模仿父亲的角色,学着像父亲一样的做事,并且由此而得到鼓励和表扬。此外,她们的父母往往自她们降生之时起就不喜欢她们,不娇惯她们,并为她们起了一个不男不女的名字。

可出现在成人期的任何阶段,甚至在老年期。在此之前,他们过着符合自己解剖性别的生活,并结婚、生儿育女。嗣后,由异装症、同性恋、性受虐症、精神分裂症等继发产生性别转换症。某些十分女子气、好穿女装的同性恋男人特别想变成女人身。他们之中,大部分人不断地纠缠医生,要求给他们做性别转变手术;有些人用异性性激素;有些人因性别上的苦恼发展到企图自杀。原发性与继发性性别转换症的治疗处理:对性别转换症应进行以精神科为主的综合治疗,最好能在专门治疗性别身份障碍的诊所里进行。医生应理解这种情况,并督促这种人就治。但是,所有旨于纠正青年和成年人的原发性性别转换症的努力,终归几乎不可避免地要以失败而告终。只有极个别的原发性性别转换症者用行为疗法有效。矫治这种人特别困难之处,还在于这种人几乎没有一个愿意矫治他们的性别身份障碍。矫治继发性性别转换症要比矫治原发性的容易,但也会遇到相当的阻抗。当性别转换症继发于严重的精神疾病,或极可能发生自杀行为之时,不论这些人愿意与否,都应安排紧急住院治疗。这时,首先需要处理的是精神疾病和自杀观念,然后才能考虑到他们在性别要求上的烦恼。然而,如果一个人只是要求做转变性别的手术,而其他方面的功能都没有毛病,就只能针对性别转换进行安排了。为了鉴别原发性性别转换症和继发性性别转换症,需要做仔细的精神检查和躯体检查,以明确精神障碍的性质和程度,排除雌雄同体或其他器质性病变。

双重角色异装症者的性爱指向往往是正常的。他们之中,大多数有正常的异性恋关系。他们之所以作异性打扮,是源于性别身份障碍,感到只有这样才符合自己的性别身份。恋物性异装症是指男性穿着异性服装而取得性满足的一种癖好。他们穿着异性服装只是为了得到性满足(表16-1)。关于这三者的鉴别诊断,可根据表16-1的内容进行。

表16-1 男性易性癖病、异装症、同性恋三者之间的鉴别

案例4 恋物性异装症

男性,22岁,出生于干部家庭,由于父母感情不和,经常吵架,很少从父母那里得到温暖。只有姑姑非常疼爱他。他常幻想能有一个温暖的家庭。11岁上小学五年级时,不知为何一见女式圆口鞋就心慌意乱、不好意思,好像这种鞋代表着什么似的。越是不好意思,就越想多看几眼。在卖鞋的地方,常不由自主地停下来看圆口鞋。半年前,在一本时装杂志上看到一个脚穿圆口鞋、身着漂亮服装的妇女,服装美衬托出鞋更美,一下子便被迷住了。当时想,他若穿上这鞋和衣服也一定很美,一阵阵脑子发热、手心出汗,极想得到这些妇女衣物,而且是被女人用过的。从此,开始偷窃妇女用过的乳罩、短裤、月经带、长袜、裙子、鞋子等。到手后反复抚摸闻嗅,感到这些物品对他有特殊的吸引力。常常在家里一一穿上后照着镜子自我欣赏,觉得自己很美,此时心情异常的兴奋激动。有两次着女装在街上主动与女人攀谈,心里非常痛快。自述:“我对她们没有邪念,也没有非礼行为。我对女性一直没有兴趣,平日也不跟女的来往接触。可是,她们的东西使得我异常激动而且一冲动起来就控制不住,不干不行,既顾不得害怕,又不嫌脏”。在半年的期间,他共窃得女袜100余双,女鞋7双,裙子14条,另有不少裤衩、乳罩、月经带、连衣裙……。

(杨华渝)

某些精神病,如精神分裂症和抑郁症(depression)患者常有固定不变的性别幻想,但精神病患者一般均有行为、知觉、思维、情感和智能方面的障碍,通过详细的精神检查均能排除。

反社会病态是指各种异常社交行为表现的一种状况。反社会病态者是指那些自童年或青少年开始即有失职和犯罪行为而后又无自责悔恨之心、病理性说谎、不负责任、难以维持人与人之间关系的这样一群人。他(她)们当中也常有要求转性手术者,目的是企图通过这种方式成为所谓的“知名人士”,并不是真正想变为异性。通过调查不难发现,这些人一般都有劣迹。

有很多学者报道,异性装癖和要求改变性别者也常发生在苯异丙胺中毒的状态下和颞叶癫痫患者中。但经过详细询问病史和必要的临床检查即可排除。


\section{第四节 易性癖病的治疗方法}

对于易性癖病这种心理疾病是否应该用外科手段来治疗,曾经存在激烈的争论。直到20世纪90年代,绝大部分专家学者才达成共识:对于真正的易性癖病患者,整形外科手术是目前最好的治疗手段,而药物或精神治疗则均没有持久的帮助。一般来讲,治疗方法主要有:心理治疗、药物治疗和整形手术治疗。

心理治疗的目的是试图通过心理调节、精神干涉和行为修饰等手段使易性癖病患者的心理性别适应于其生理性别。Money(1988)认为,性别自认一旦形成便不可改变,因此试图通过心理学和精神病学专家使易性癖病患者的心理性别发生逆转,一般不能成功。甚至有些学者认为,易性癖病是一种对任何形式的心理治疗均不产生应答反应的心理状态,没有真正的易性癖病患者可以通过说服、威吓、药物、精神分析、羞辱、嘲弄或电休克等方式使其接受自己身体的现实。

药物治疗主要是指性激素治疗。一般认为,应用性激素只是对性别重塑外科手术的一种辅助性治疗措施。男性患者应用雌激素是有帮助的,它可产生一个暂时的化学性阉割作用,通过减少性欲产生镇静效果,雌激素的女性化作用还可带来感情上的欣慰。男性患者应用雄激素有害,因为雄激素有加强性驱动的作用,但并不改变其性定向,其男性化作用可加剧心理性别与解剖学性别的矛盾并引进更大的精神创伤,因此它的使用是禁忌的。女性患者应用雄激素,可使体毛增多、喉结突出、肌肉发达、嗓音变粗等,总之可使其在第二性征上接近男性。此外,长期应用雄激素可刺激阴蒂增大,以致一些患者不再希望施行阴茎成形术。

手术治疗即是通过整形外科手段(组织移植和器官再造等)使易性癖病患者的生物学性别与其心理性别相符。也就是实施一系列的整形外科手术重新塑造患者的第一、第二性征,使患者身体的形态结构和功能尽可能地与他(她)们想要的异性性别的形态结构和功能相似或相同,并能基本满足生活、工作和学习的各项生理需要。过去,人们一直将易性癖病的手术治疗称为“变性、改性、转性或性别再赋外科”(sex reassignment surgery,简称SRS),这是很不确切和极不尊重易性癖病患者的提法。因此,1982年Benjamin国际性焦虑学会主席Edgerton博士纠正了这一说法。他强调指出,对易性癖病患者所行的手术治疗事实上应该是“性别重塑或性别认同外科”(gender comfirmation surgery,简称GCS)。GCS手术是一个系统工程,它不仅需要很高的整形外科手术的技术和技巧,同时又需要多学科(包括妇产科、泌尿外科、内分泌科、心理和精神科以及法律工作者等等)的相互协作和紧密配合才能很好地完成。与此同时,它又不单纯是医学技术本身能完全解决的问题,总是会牵涉到诸如法律、道德、伦理、宗教及社会学等多个相关领域。我们认为,在科学日新月异,现代医学观念不断更新的今天,当易性癖病患者在心理、精神治疗无效时,如何运用整形外科技术进行GCS手术,使他(她)们尽早地在生理和心理上获得统一并为社会所接受,是我们整形外科医师的责任,也正是现代医学模式:生物—心理—社会医学模式的具体运用和体现。

(陈焕然)


\chapter{第十七章 性偏好障碍}

21世纪以来,人们对性偏好障碍的认识发生了很大变化,有些学者认为性偏好的种种表现都属于性多元化的不同表达方式,并无对错和是非之分,完全应该尊重每个人的特殊选择,不应该污名化、矮化和病理化这些行为方式。不过本书还是遵照过去的理解对此予以介绍,让人们对此有一个历史性的认知。


\section{第一节 概 述}

在人类社会生活中,两性间的性心理及性交往是有一定的基本常态或要遵从所在社会的行为规范的,可以说,正常性活动是男女两个成年人之间的亲昵活动,而且最终和生育相联系。性偏好是一种正常现象,是普遍存在的。正因为有了性偏好,大家在追求异性时才不会都去追求同一个人,撞车撞成一团糟。但性偏好又可分为健康的和病态的两种。健康的性心理和性偏好总是把性看做是美好生活的一部分,也是良好的人际关系和夫妻关系的一部分。他们不仅寻求自己的满足,也努力使对方得到满足,或至少尊重对方,在对方同意的情况下才发生性关系。也就是说他们的性爱总是指向完整的异性个体,异性间的爱慕最终总会发展为相同的行为模式,即以能导致生育的性交作为满足性欲的基本形式。

而病态的性偏好则是指性偏好障碍,又称性欲倒错。他们的性心理和性行为则大相径庭,他们常常由于各种原因而失去常态,变得使正常人不能接受甚至觉得不可思议。也就是说其性观念、情感反应、态度和行为呈病态表现,患者往往表现出多种多样、十分离奇古怪的性唤起、性对象和性满足方式,其性欲和追求目的不是指向完整的异性成年个体和正常性行为,而是分别表现为性对象的异常(把性对象象征化,如把异性身体的某一部分或异性的衣物作为性爱对象,以这些目标替代了完整的异性个体,如恋物症);和性行为方式的病态表现(把表达爱慕之情的求偶行为目的化为发泄性欲的目的,如露阴症)。这种把性对象象征化或把性行为目的化的人就属于性偏好障碍。他们的特点不在于其异常偏离常态有多远,而在于从不追求正常的性关系和缺乏正常的性行为,即万变不离其宗,是和生育脱离的,对正常的最终导致生育的性交不感兴趣。患者不仅对于正常的性活动没有要求,甚至心怀恐惧,他们的变态性行为常具有强迫性和反复性,他们的自我控制和自我保护能力往往较差,但并非时时发作。他们只是在歪曲的性冲动支配下,在特定的情景和处境下突然付诸行动,而当时怎么也控制不了自己,他们在事前并无周密准备,案发后又能供认不讳,常常感到痛心疾首,无限悔恨,有些人强烈要求医治,希望摆脱这种令人痛苦的状况,但也有人则不认为自己是病态。按照性心理障碍发生的不同阶段可以划分为以下层次:(1)寻找伴侣阶段的异常:如窥淫症。(2)触觉前阶段的异常:如露阴症、淫语症。(3)触觉相互作用阶段异常:如摩擦症。(4)生殖器结合阶段异常:如施虐狂。

目前,在诊断性心理障碍问题上尚缺乏明确、客观的指标,这种指标往往带有明显的社会文化烙印,并随着历史的变迁而变化。因此,只要一个人存在正常的性行为要求和表现,不论他还有什么奇特的嗜好,就应认为他的性心理是正常的,就算有变态性行为。性偏好障碍的基本特征是至少6个月以来,在一些非寻常的性对象或性行为方式下激起反复的、强烈的具有性唤起作用的幻想、性冲动或行为。只有对所迷恋的物件或行为等具有固结性(反复多次主动地“恋物”)和排他性(完全脱离或对两性生殖器性交不感兴趣或不当做满足性欲的主要方式),这时才是病态的。它通常涉及非人性对象,惩罚或侮辱自己,或伴侣,或涉及儿童或其他没有同意的人。对有些个体来讲,变态的性幻想或刺激是强制性的,旨在达到性的唤起,而且总是包括性活动。对另外一些人来讲,变态的偏好行为仅仅是间断发生的(如在感到紧张的时期),而在其他时间里,个体可以在不具有变态的性幻想或刺激的情况下正常行使性功能。这些性行为、性冲动或性幻想引起具有临床意义的明显痛苦或在社交、职业或其他重要功能领域造成损害。

性偏好障碍不是精神病,因为他们除了取得性满足的方式偏离正常外,其情感、理智、智能等其他方面均表现正常,只是性心理不正常,他们往往只在特定情境下不能控制自己的行为。这些现象与流氓或其他性犯罪分子的根本区别在于,患者从不谋求与异性发生肉体关系,流氓则会与受害者纠缠不清,往往有进一步的伤害女方的举动。现在,我们的社会已开始认识到这些人的异常性行为与性犯罪有所区别,虽然他们可能负有一定的行为责任,但他们是患者而非罪犯。许多性偏好障碍患者虽然被揭发、处罚甚至判刑,但其行为并得不到改正,而且大多数患者也不会主动要求改变他的这些行为。

在人类历史上,最早把上述这些异常的性活动统称为性变态(perversion),是贬义词,认为是亵渎神明的罪孽,是道德败坏的恶癖。20世纪初,性心理学家霭理士指出,研究性变态的目的在于了解和设法治疗它,不在于判断善恶,并主张把其改为性偏向(sexual deviation)。Meyer A则创用了性偏好(paraphilia)一词,不但没有鄙视的含义,还带有主动偏爱的意思。1980年修订美国精神疾患诊断和统计手册(DSM-Ⅲ)时,就是使用性偏好障碍一词,这也适应于给疾病下描述性定义的需要。大多数患者的性偏好性活动起自幼年,难以纠正,而且不主动要求改变,也不伴有人格异常。所以,各国疾病分类都把这些性心理和性行为的异常当做精神障碍的一个独立的类别。ICD-10(1992)中,在性偏好障碍(F65)项下,除正式列入7个亚型(恋物症、异性装扮症、露阴症、窥阴症、施虐与受虐症和恋童症)外,还提到一些相对少见属于个人特异性而无法命名的性偏好表现,但不包括同性恋和性别转换症。

变态的性想象会驱使患者与一个没有同意的伴侣发生实际的性行为并可能对该伴侣造成伤害,如施虐狂或恋童症,患者可能因此被逮捕并监禁起来。儿童的性犯罪在整个性犯罪中所占比例相当大。有自觉能力的性犯罪分子中有相当多的人属于露阴症、恋童症和窥淫症。在有些情况下,性欲倒错想象的实际实施会导致自我损伤,如受虐狂。如果其他的人发现这些不寻常的性行为令人难堪或厌恶,如果其性伴侣拒绝在这些不寻常的性偏好行为中合作,那么这些人在社会交往或性关系中将出现困难或痛苦。在某些情况下,不寻常的性行为(如露阴或收集所恋的物品)将占据他们一生中性活动的很大部分。他们很少会主动求医,只有他们与伴侣或社会发生显著冲突时才会求医或引起有关医疗保健工作者的关注。

当这些个体找不到一个能与之合作实现其性幻想的性伴侣时,他们会花钱与妓女实现他们的愿望或寻找和强迫不情愿的受害者。他们可能选择符合他们癖好的职业或志愿工作以便使他们能接触到向往的刺激,如恋物症者去卖女鞋或女内衣,也有的人专门从事美容工作以接近化妆品等;恋童症者寻找能接触儿童的工作;施虐症者去开救护车。在他们的阅读、观赏、购物、图片收集、影片、写作描述中无不集中到他们所偏爱的变态的性刺激类型上。具有这类性心理障碍的多数人否认自己感到痛苦,他们的唯一问题是其他人对他们的行为作出的反应造成了他们的社交功能障碍。有些人则为他们不得不从事社会所不能接受的不寻常的性活动感到内疚、羞愧和抑郁,认为这是不道德的。他们的成熟的、富有情感的性活动能力往往受到损害,也可能存在性功能障碍。他们也可能具有人格障碍,有时甚至很严重。具有性欲倒错者常出现抑郁症状并伴有变态行为的频率和强度的增加。

性欲倒错的发生率并不清楚。但从西方反映性欲倒错性行为的色情文艺品的广大商业市场来看,它的发生率势必不低,据估计可能在数十人中就会有一个。除了在受虐狂中女性可占到5\%左右之外,其余性欲倒错者几乎都是男性。约半数性欲倒错者仍会结婚。


\section{第二节 病理本质和发病原理}

这些五花八门的与性欲倒错有关的幻想和行为可以从儿童期或青春期早期就开始发生,但要在青春期和成年早期才能更好地确定和得到精雕细刻与不断修订,这个过程很可能还要在整个成年期持续进行。性行为并非无师自通,是要经过后天学习而掌握的,而一个人婚前长期的,尤其是十几岁之前性不成熟期的家庭生活经历及社会环境千差万别的影响,则塑造了变化多端的性意识并给他的性观念打上深深的烙印。因此,不同的性心理可以造就不同的性行为方式和不同的性障碍。按照定义,性欲倒错这种性心理障碍的幻想与冲动是反复发生的和持续存在的,只不过幻想的频率和冲动的强度在不同时期有不同的表现,变化很大。这种性心理障碍往往是慢性的和终生性的,但在成年之后,性幻想和性行为常随年龄的增长而逐渐减少。如果遇到心理社会压力,其不寻常行为会反应性增加。当合并其他精神疾患或从事性欲倒错的活动机会增加时,其不寻常行为的发生也将变得更频繁些。

原发性性心理障碍包括:性身份障碍和性偏好障碍;继发性性心理障碍多系脑器质性病变和精神疾患所引起的性行为障碍,如颞叶癫痫患者可发生窥淫症、恋物症、露阴症等多种类变态性行为;精神发育不全患者也往往存在这类变态性行为。这些变态性行为只是原发疾病的并发症状,应与单纯性性心理障碍加以鉴别。

性心理障碍病因尚不明确,它们包括生物遗传方面、心理学方面、环境和社会等方面因素的影响。下面将分析一下性心理障碍的各种可能的原因:

生物遗传学因素———某些性偏好行为可能有生物学基础,例如露阴行为可以在患有神经系统器质性疾病如癫痫、痴呆或智障的患者身上出现,但极为少见,容易鉴别诊断。多年来生物遗传方面研究的结果表明,大脑和神经系统中并未能证实有任何特别的化学物质与性心理障碍有关,而激素方面的研究结果始终是矛盾的,无法作出任何结论。专家们正致力寻找遗传基因方面的问题。

心理学因素———绝大多数性偏好障碍患者并没有躯体器质性疾病或相应的遗传倾向,而且除了这种性偏好行为以外,身体发育和精神生活的其他方面都是正常的,有着良好的人际关系。他们的性观念和表现甚至比一般人还要“严肃”和“保守”。行为主义学派用社会学习理论来解释性偏好障碍的本质和发病原理,他们认为人的性偏好行为和正常的其他行为一样,都是从儿童少年时期起潜移默化地学习得来的,若在性发育过程中遭受不良性教育或性经验的影响,就会按照条件反射形成的原理在特定性刺激与性快感之间建立起固定的联系。弗洛伊德把青春期以后成年人通过两性生殖器性交得到的性快感叫做终极快感(end pleasure),青春期以前各个阶段幼儿性兴奋得到的快感叫做前期快感(fore pleasure)。在正常情况下,随着发育成长,前期快感最后都隶属于终极快感。弗洛伊德指出过性心理发育中的两个危险:即固结和退行。在人的性心理发育过程中,幼儿性活动的各阶段都可停滞不前,即固结(fixation),其固结处多是性欲得到满足、感到快乐的阶段。即使一次纯粹偶然的经验也可能形成固结,在青春期前幼儿性欲发育过程中任一阶段如果形成固结,就会使性心理不能继续发展成熟,阻碍前期快感向终极快感迈进,使性活动在某一预备动作上滞留不前,形成性偏好行为。幼儿性欲的固结和上面提到的未妥善解决的情感症结都会影响性心理的发育和成熟。有些性偏好障碍患者自幼年起,即有“变态”性行为(实际上是较明显的幼儿性行为)直到成年而不变,就是因为这个原因。如一位患者年轻时在一次偶然机会里通过浴室门钥匙孔窥视妇女裸浴,视野很小,只能看到妇女的部分身体和鞋袜并引起第一次性兴奋。条件刺激(妇女的鞋袜)和非条件刺激(可引起性冲动和得到满足的对象)结合多次后,终于使这个性欲尚未成熟的人形成巩固的条件反射,恋物偏好。一位趴在长凳上遭受老师鞭打的中学生的视野里,突然出现穿着超短裙的、美丽女教师诱人的漂亮大腿时,他竟然勃起并第一次射精了,从此,没有鞭打的过程,他就不能达到勃起,即成为受虐症。

环境和社会因素———弗洛伊德指出过性心理发育中的第二个危险是退行(regression)。有些人的性发育中的固结并未能影响整个性心理的发展,安全度过青春期并发育到成熟阶段,之后也有了正常成年人的性生活,中年后一旦受到现时生活环境的限制、客观阻碍、遇到较强挫折、精神创伤、性欲压抑或个人人格的缺陷时,增高的性兴奋无法得到正常宣泄和满足,便只好另寻出路,即性心理会退行到早年已经过了时的固结点,避开自我控制,而直接表现为幼儿式性欲、性心理和性行为的方式,寻求已经放弃了的对象,形成和突然表现出性偏好障碍,这就是退行机制。弗洛伊德曾打比喻说,一条溪流的主流受到阻拦,溪流的水只好溢向干涸了的旁道。这就是成年人突然出现性偏好活动的原因。一般来说,反常的性行为是不合理的社会强制和压抑所造成的性心理冲突的后果,所以它也是一种复杂的社会问题。按照弗洛伊德的观点,“变态的性行为就是幼儿的性行为”。所以他认为性心理障碍是在成年人生活中持续地表现幼年性欲的成分并以幼年的方式获得满足。

性心理障碍的患者中常常存在不同程度的人格缺陷。如强迫性人格—刻板固执,又称执拗性人格。办事循规蹈矩,墨守成规,意识保守,缺乏随机应变能力,遇事优柔寡断不善决断,对自己要求高但缺乏自信,工作负责过于谨慎。这些人容易产生强迫性症状和焦虑、抑郁反应。此外,有些人可能具有分裂型(如窥淫症)、未成熟型或被动型人格(如露阴症)。幼年和早年性心理发展中的挫折或冲突与成年后的性心理障碍有着心理动力学上的因果关系。按照弗洛伊德的观点,儿童在3岁左右产生对异性父母的爱恋,即俄狄浦斯爱恋。男孩由于爱母亲而把父亲当作情敌而嫉恨,又怕父亲生气会割去他的阴茎而心怀恐惧,当其年龄再大些,男孩将放弃对母亲的爱恋而仿同其父,这样男孩的俄狄浦斯爱恋便得到正常解决。否则,这种爱恋就会凝固并在无意识中形成的俄狄浦斯情结,并对以后的性格形成严重的不良影响,成为日后性心理变态的根源。

弗洛伊德的幼儿性欲和性心理发展学说可以解释多数性变态的病理本质和发病原理。但是,正如ICD-10中所说的那样,“性欲活动变化万千”,还有许多罕见的类型,用心理分析学说也很难解释清楚。要想阐明所有性变态的发病原理,还有待今后的研究。


\section{第三节 临床类型}

性偏好是一种正常现象,是普遍存在的。正因为有了性偏好,大家在追求异性时才不会都去追求同一个人,撞车撞成一团糟。但性偏好又可分为健康的和病态的两种。健康的性心理和性偏好总是把性看做是美好生活的一部分,也是良好的人际关系和夫妻关系的一部分。他们不仅寻求自己的满足,也努力使对方得到满足,或至少尊重对方,在对方同意的情况下才发生性关系。也就是说他们的性爱总是指向完整的异性个体,异性间的爱慕最终总会发展为相同的行为模式,即以性爱作为满足性欲的基本形式。

性偏好障碍有许多表现类型,有的患者仅有一种,也有的患者兼有几种。常见以下几种表现:

①至少6个月以来,反复多次在事先毫无准备的陌生人面前突然显露自己外生殖器从而激起性幻想、性迫切愿望或行为。②这种幻想、愿望或行为,产生了临床上明显的痛苦、烦恼或在社交、职业或其他重要方面的功能缺损。如一位露阴症患者,男,35岁,司机。自婚前起的多年来,在不同僻静的场合或合住的单元房里有露阴行为,虽然尚未被人们抓获过,但无论是他自己还是他妻子整天提心吊胆地生怕出事,他们害怕出事后,将毁掉这个看上去还幸福的家庭。

露阴癖是1877年首次描述报道的。其主要表现是喜欢躲在阴暗处或林间小路上,突然向走过的陌生女孩或青年妇女暴露出勃起的阴茎,以取得性兴奋和性快感。不论对方表现恐惧、不理睬、惊恐、害羞或愤怒叫骂,患者都可得到性的满足。少数妇女表现好奇或稍稍停步在远处眺望时,患者更感到极其兴奋,可达射精和性高潮。如果被暴露对象表现出无动于衷,反倒令露阴者大为扫兴。也有的在公共场所如影院、商店或公共汽车上行动,但大多数露阴癖患者的行动最多到此为止。露阴症患者在发病时不一定具有色情欲念,也不一定达到勃起,但这一举动无疑会使他们得到放松和满意,之后总是匆匆溜走,决不会采取进一步的非礼举动。虽然如此,妇女们常因感到受了污辱而向有关方面控告,成为司法部门性侵犯罪中最常见的案例。

有人发现在强奸、纵火犯和抢劫谋杀犯中,有12\%在犯罪以前曾有过当众露阴行为,但在露阴的当时并没有上述犯罪行为。按行为严重程度可以对露阴癖进行分类:①简单地露出阴茎;②露出阴茎同时手淫;③除了露阴和手淫外,还与被侵犯的妇女说话,喊叫,吹口哨等;④除露阴外,还直接触摸妇女身体。由于男女解剖学的差异,露阴癖主要见于男性,露阴行为的根源可能是幼年时期留下的阉割焦虑。男孩在窥看女孩外阴时发现和自己不同,以为被割去了,怕自己也被割。这种恐惧留在潜意识中,成年后的露阴行为是向妇女显示,他有一个阴茎,没有被割去,并由此感到自豪。这当然是退行到儿童时期的心理活动,是无意识的,患者自己意识不到。已婚患者会诉说,露阴行动带来的快感比和妻子性交更强烈,更有“魅力”。因此,虽然他们很爱妻子,却更喜欢让妻子看他的阴茎,而对性交则表示冷淡。让患者回忆性经历时,大多数患者都可回忆起三四岁到11岁之间有过各种性游戏和性经历:①和同龄男女儿童玩耍,互相观看、抚摸外生殖器以取乐,当时感到新奇、激动并有快感;②幼年时喜欢裸体,周围成年人包括父母、姐姐等都喜欢观看他们的小阴茎以逗乐,当时有莫明其妙的自豪感。有些患者自幼年起的在人面前裸露阴茎的行为实际上并没有长期间断过,只是没有被人揭露而已,到初次被揭发时,已经是成年人了。对成年后才出现露阴行为的患者,仔细询问他们的发病过程,发现在初次出现露阴行为之前或当时,都有较为明显的精神创伤因素,这些因素有:①精神生活陷入严重困难,如政治上受到严重打击,情绪沮丧;亲人死亡异常悲痛;工作中受到批评处分,心情郁闷等;②感情的挫折或性欲长期不得满足,如突然失恋,心情苦闷;夫妻两地分居长期不能解决;夫妻不和而又无法调解;婚后性生活不和谐,妻子经常拒绝性交等。患者对这些困难处境和性欲挫折没有应付和排解能力,心情异常苦闷。正在这时,首次出现露阴行为。研究发现,不论是自幼年延续而来,还是成年后突然发病的露阴癖患者,在幼年儿童时期都曾有过主动参与的、可取得快感的性经历。这些曾经引起过兴奋激动的性经历渐渐被遗忘,但没有消失,而是作为儿童取乐方式被固结在无意识中。成年后遇到精神创伤或性的压抑无法应付时,便不自觉地用幼年儿童的取乐方式来排除成年人的困难,宣泄成年人的性欲。这就是露阴癖的病理心理本质和发病原理。另外,患者的个性特点有惊人的相似之处,而与他们的受教育程度或职业无关。他们大多数人性格内向,不善于和人交往,尤其和女性。在妇女面前表现腼腆,害羞,拘谨。作风较严肃,从来不和妇女开有关性的玩笑,更没有过分的举动。有的患者还表示对那些作风轻浮的男女青年很反感。工作都比较好,认真负责,循规蹈矩。受到上级和同事的夸奖。由于个性特点以及平时表现和他们的变态性行为极不相称,所以,在他们违反社会道德规范的行为被揭露后,起初人们感到惊讶,不相信他们会有这种行动。在不断被揭露后,便被批评为“伪君子”。患者拘谨怕羞的个性使他们不能用其他方式来排解现实的困境,宣泄被压抑的性欲,也是他们心理上不自觉地退行到幼年期的条件之一。

露阴症在西方性犯罪中是最常见的一种类型,约占整个性犯罪的1/3。人们最初推测露阴症的出现可能是对过去“假斯文”和性压抑的一种反抗。人们还预见随着社会的日益宽容和公众对性问题不再那么谈性色变,其定罪率将有稳定下降。

过去总认为露阴症多见于老年男性,其实这只是年轻人的一种偏见,往往把老年人的生活看得一塌糊涂。不过的确有一些老年患者和年轻人一样,这也是一种性的病态心理表现。到50多岁之后,性的生理功能自然减弱,但性的心理要求并没有减弱,他们不再能感受到成年人的性满足。这时幼年的性活动方式重新浮现在脑海中,不自觉地用幼年方式来解决衰老过程中的性欲,结果就呈现上述病态行为方式。另一类原因是继发于潜在的心理病理表现,如动脉硬化、老年痴呆、躁狂症或压抑性疾患。露阴行为很少预示老年痴呆。他们也许具有完整的认知能力,但可能具有潜在的历史问题,并寻求社会非难的发泄途径以发泄他们的性挫折。儿童往往更容易成为老年露阴症的受害者,因为他们更容易在“老爷爷”的可亲形象诱惑下上钩。即使是成年妇女在这种情况下也很少报案,因为她们认为老年人的威胁和带来的危险并不大,况且她们仅仅受到惊吓而非进一步的伤害。

是和露阴癖表现相对应的性变态。①至少6个月以来,反复多次观察一个事先毫无准备的裸体者在脱衣服或从事性活动,从而激起患者的性幻想、性迫切愿望或行为。②这种幻想、愿望或行为,产生了临床上明显的痛苦、烦恼或在社交、职业或其他重要方面的功能缺损。患者的主要表现是窥视别人的性活动、异性裸体或异性外阴部,从而获得最高的性兴奋和性满足。对正常的两性性交反而表现冷淡。窥阴癖主要见于男性,少数患者可兼有露阴或恋物行为。自己知道这样做有被发现和被捉的危险,仍然要去行动。诚如弗洛伊德所说,他们“很像一个可怜虫,不得不付出痛苦的代价,以换取不易求得的满足。”部分患者从幼年起就有这种幼稚的窥视行为,而且没有随年龄增长和其他心理成分成熟而有所改变。窥阴癖的病理心理本质及发病原理和露阴癖是一样的,是幼稚性行为表现的两个方面,即幼年期男女小伙伴性游戏中看和被看的愿望和行动在成年人身上的延续和再现。这也可能是这两种性偏好障碍最多见的原因。

这是一位患者家属写给专家的信:“我的丈夫在读书时品学兼优,下乡期间是先进知青。参军后在对越自卫反击战中光荣负伤,至今颅内尚有弹片未取出,因此,荣立三等功并光荣入党,因为伤残退伍回厂当工人。但1980年回厂后出现从未有过的令人费解的异常表现,即多次偷看女厕所和女浴室,拘留、批斗都无效。1983年8月,在全国从重从快打击刑事犯罪活动中被该厂送去劳动教养二年,党籍也被开除。刑满释放后依旧如故,在当地精神病院诊断为性心理变态:①战争致残有一定影响;②有责任能力;③建议从轻处理。我是1988年10月和他谈恋爱的,我发现他为人诚实,态度严谨,很尊重我的感情,从无轻浮之举,他也向我讲了他的一切,包括劳改。结婚后性生活正常,但我们只能两地分居。今年3月(1989年)他又故伎重演,在派出所里遭到毒打,伤痕累累。我是个工人,不懂医学,但我觉得丈夫的这种间歇性反常行为并非他随心所欲的,更不是有意的。他的同事讲,他发病时如有人喊他,他不能答应,要拍拍他的背,他才能神志清醒,恢复后也知道这是违法的。在发作之前,他的心情特别烦躁,不能做任何事情,头脑昏沉异常,有一种不可压制的冲动,一点儿也不会去考虑那样做后会发生的后果,非要去偷看女厕所后心情才能慢慢平静下来,过后又后悔不已。他的发作时间没有规律,不分昼夜,厂领导对他痛恨无比,使他在生活道路上背上了沉重的精神包袱,很难向前移步。我该怎么帮助他呢?他不是坏人,是一个真正的好人。我诚恳地希望得到您的指教。”

脑外伤可以引起癫痫样发作,其性心理障碍和发作时的表现可以认为是脑外伤所致。手术取出弹片能否治疗很难讲,即使弹片可以取出但大脑所受损伤却未必能恢复。可以肯定的是这位男子决非流氓之类的故意犯罪或扰乱社会治安,故给以劳教两年的处理是不妥当的,而且也不会有任何预防复发的作用,它并非单纯惩罚就能解决的问题。最好解决两地分居问题,以便妻子更好照顾患者,使患者的性能量能得到正常宣泄,以减少发作的次数,有助于病情的控制。战伤等心理因素或社会适应方面的因素也有可能使他产生失落感,这种心理刺激使他从对前途的过高期望一下子跌入极度悲观,从而出现性本能的异常宣泄,退回到幼儿期的性活动方式,受到惩罚后又进一步自暴自弃,对生活失去信心,反过来又促使性变态的恶化,形成恶性循环。不过,从他的病情分析,这种可能性较小,脑外伤所致可能性更大。

对窥阴症主要采用心理疗法,可通过行为技术来矫正。第一步是引导其求治心理。第二步是从两方面进行矫治,其一,用厌恶疗法抑制并消除其异常的性行为,其二是用系统脱敏疗法消除对正常性活动的不适感。这两种方法的结合运用也称为交互抑制。治疗阶段一般需要两个月。必要时还可辅以药物治疗,疗效比较理想。

①至少6个月以来,反复多次地与不同意此行为者作触碰及挨擦,从而激起性幻想、性迫切愿望或行为。②这种幻想、愿望或行为,产生了临床上明显的痛苦、烦恼或在社交、职业或其他重要方面的功能缺损。

挨擦症也是较常见的一种性偏好障碍,其主要表现是在人多拥挤的场所用阴茎隔衣挨擦妇女的臀部、手臂等处,以取得性快感。有时取出阴茎直接挨擦,可以达到性高潮并射精。患者几乎都是男性。他们往往在外观上服装整齐,彬彬有礼。被揭发批评后,大多能主动承认错误,但行动上并不改正。和露阴癖、窥阴癖一样,有的患者从幼年儿童性游戏的挨擦行为开始,一直延续到成年,没有长时期的间断。约有一半患者在成年后甚至到中年才初次发病,在这种情况下,大都有精神刺激或性压抑的经历。钟友彬教授介绍过一个自幼年期延续不断直到成年的例子:男性,28岁,已婚。受过中等教育。20岁高中毕业后即参加工作。23岁和一农村女青年相识,相处尚好,曾在拥抱接吻时达到射精。1年后,女友离开了他。自感到受刺激,心情郁闷,容易胡思乱想。有一次在公共汽车上,人多拥挤,阴茎勃起,隔衣用阴茎顶触前面女青年的臀部,感到兴奋以致排精。以后,在人多拥挤的场所同样行动多次。25岁结婚后两个月内,性交频繁,可以得到性满足,要“挤汽车”的念头大为减轻。不久,妻怀孕,有时拒绝和他性交,即感到心烦,郁闷,要用阴茎挨触妇女的念头很强烈。以后两年内又在公共汽车上多次行动,曾被发现、拘留两次。想戒掉,但时常有冲动,在家人催促下找医生看病。患者承认和妻子性交可以得到满足,但有时感到其兴奋满足的程度还不如挨擦妇女时那么强烈,那么“刺激”。由他成长的整个经历看,这个患者自四五岁开始到结婚后,用阴茎顶触妇女身体的行为并没有长时间地间断过,一直延续到成年,只是对象和行动的方式不同而已。这种儿童时期幼稚的性活动并未随年龄增长而完全发育成熟。即使成年后两性生殖器性交可以暂时得到满足,但幼稚的性活动似乎更有力量,即使受到惩罚也不能改正。这种挨擦行为在幼年儿童身上出现,应看作是正常的,但延续到成年,类似的行为就要受到社会的干预了。有些患者的挨擦行为是成年后才开始的,也常是在精神刺激和性的压抑下突然出现。不论变态性行为最初发生在成年以后或自幼年起不间断地延续到成年,患者在幼年期都有过可以取得快感的性经历。而且其变态性行为的表现和幼年性经历之间,在内容上还有一些相应的关系。

恋物症是至少6个月以来,反复多次以无生命的物体、主要是人体的延伸物如异性体表接触的物品(乳罩、内裤、长裤袜、高跟鞋等)或异性身体的一部分,作为性幻想、性迫切愿望或行为的刺激物并从中获得性兴奋及性满足的一种性偏好。恋物对象可以是任何东西,包括具有某类特殊质地的物品如橡胶、塑料或皮革。恋物症者所用物体往往不仅限于易装(如异装症)用的女性服装,也不是作外生殖器刺激用的器具(如某种振荡器)。恋物症者通常无法以一个实际存在的完整的异性人作为性爱中心,而是因所恋物品引起性联想,性兴奋。

恋物症者对物品的迷恋程度有强弱的不同,或者说迷恋物的重要性因人而异:在某些病例中仅作为提高以正常方式获得的性兴奋的一种手段(如要伴侣穿上特殊的衣服)。患者通常在抚摸或嗅闻所恋物品的同时自己手淫,或在性交时自己或性对象手持此物,以获得性满足。典型的恋物症需要视觉和触觉刺激。有时,仅视觉刺激如色情画等,即可引起内心一阵愉快的反应。患者会千方百计收集所恋物品,常不惜冒险偷窃女性乳罩、内裤或剪女性发辫等为乐。然后在背地里面对偷来的衣服产生性欲冲动,再通过手淫以求性感的满足。所以其诊断要点为:只有当迷恋物是性刺激的最重要的来源或达到满意的性反应的必备条件时,才能诊断为恋物症。恋物性的幻想很常见,但除非这种幻想、愿望或行为,引起了显著强制性、无法接受的仪式动作,以至干扰了性交,产生了个体明显的痛苦、烦恼或在社交、职业或其他重要方面的功能缺损,否则不足以诊断为此种障碍。恋物症几乎仅见于男性。其病因不明,患病率不详。

恋物症无特殊药物治疗,以心理治疗为主,当今应用认知疗法和行为疗法,特别是采用厌恶疗法或系统脱敏疗法,已成为治疗恋物癖的主要手段,疗效也较理想。

一位患者这样介绍他的情况:“我从初中时突然对袜子感兴趣了,一开始只对某种颜色的袜子有兴趣,到了高中我开始对女性长筒袜感兴趣,开始偷穿妈妈的长袜,穿时精液流了出来,我感到很兴奋,从此一发不可收拾。不知怎么地开始玩弄生殖器,直至射精,演变成了真正的手淫,有时边穿长筒袜边手淫。尽管我试图把日程安排得很满,但怎么也消除不了手淫习惯,而且我的兴趣又扩大到长筒靴、白手套和足球袜。我整天无精打采,昏昏沉沉,面色苍白,非常痛苦,甚至对生活失去信心。请问我为什么会出现这种奇怪的嗜好,它为什么和手淫有联系,我该怎样戒除这种毛病呢?”

他的特点恰恰是把手淫和恋物这样的非性交性活动联系起来,往往是通过这些幼稚的行为来发泄自己的性能量和性冲动。患者在幼年发育的关键时期缺乏母爱或母亲过分溺爱导致心理变态、缺项、空白或人格缺陷,随之超我能力减弱,使本我带着原始性冲动肆无忌惮地不断跃位。这又随之造成潜抑闸门的失调和意识管理网络失控,最后只有通过畸形宣泄渠道尝试性的满足。而每次恋物过程均使患者达到变态性满足,使倒错的性欲得到不断强化。恋物症:常以偷女性乳罩、内裤或剪女性发辫等为乐。然后在背地里对偷来的衣服产生性欲冲动,再通过手淫以求性感的满足。

恋物症初发于青少年性成熟期,有些家长反映,他们十几岁的男孩枕头下竟藏着女人的胸罩、内裤、尼龙袜等。当然,当这些家长发现这些问题后无不十分紧张,这是不是性变态呀,这可该怎么办呢?也有些家长见了这种现象只是一顿打骂,结果不但不见效还有日趋严重的倾向。其实孩子从幼儿起便会对性感兴趣,也就是说性欲自婴幼儿起就存在了,当然这种性欲与成人的性欲不一样。比如有很多小孩不到1岁就爱用手玩弄自己的生殖器,这就是因为他偶尔碰到生殖器后给他带来了与触摸身体其他部位时不同的感觉,而且这种感觉会让他觉得舒服,于是他便不断地去寻求这种感觉。所以有的男孩在4岁左右便对女性的腿、脚这些经常暴露的部位或尼龙袜感兴趣,表现出极强的触摸欲望,就不奇怪了,特别是它们的线条、光洁的皮肤或尼龙袜的触觉与视觉效应都会令他们激动。他们毕竟也逐渐懂得审美观,再加上电视中经常闪现的这类镜头的影响,他们便会出现这种看似早熟的心态,可见这是后天环境的影响了。

那么有没有先天因素的作用呢?为什么只有少数孩子才表现出这种好奇呢?这是一个尚不明确的问题,医学界尚无法判断其发生原因究竟是先天的,还是后天的。因为从成年性变态患者的资料看,他们不一定具有明显的后天环境的影响,但说这些患者是先天所致又缺乏科学根据。现在的问题是如何正确对待和认识孩子们的这些表现,应该说多数孩子的这些表现是自然的,不必过多干预,更不能责罚或打骂,因为这样只会强化这一倾向并激起逆反心理。可以带孩子到性医学或心理科去咨询一下,并建立一个病历,随访几年,观察今后若干年的发展变化情况。因为目前的行为表现只是从幼年心理出发的,他们还不懂得按成人的性需求去发展,待他们逐渐成长为成人后,就会把这种不全面的性兴趣转变为成人真正应该感兴趣的事情上去。只有少数人不能完成这种转变,这就需要专门的心理治疗了。应该说十几岁孩子的性心理正处于过渡或转变阶段,他们如能得到医生的及早帮助,他们就会顺利完成这种过渡或转变,从而形成健康、完善的性欲,否则将有可能形成性偏好障碍,但即使形成性偏好障碍也是能够得到改善、纠正或治愈的。

①至少6个月以来,一个异性恋的男人反复多次以更换女性服装来激起性幻想、性迫切愿望或行为。②这种幻想、愿望或行为,产生了临床上明显的痛苦、烦恼或在社交、职业或其他重要方面的功能缺损。

ICD-10对“异装症”的描述为:穿着异性服装主要是为了获得性兴奋。其诊断要点为:通常不止穿戴一种物品,常为全套装备,包括假发和化妆品等。恋物性异装症与异性装扮症不同,前者清楚地伴有性唤起,一旦达到性高潮,性唤起开始消退时,便强烈希望脱去异性服装。在易性症者中,早期阶段常有恋物性异装症的历史,这种病例可能为易性症的一个发展阶段。

异装症是恋物症的一种特殊形式,这一障碍与单纯的恋物症不同:他们所迷恋的衣物不仅是穿戴,而是打扮成异性的整个外表。反复多次出现穿戴异性服饰的强烈欲望并付诸行动,目的是由此可引起性兴奋和性快感,当这种行为受抑制时可引起明显的不安情绪。此症病因不明,多起病于青春期,开始时偶尔穿着一两样异性服装,以后逐渐增加异性服饰的数量。异装症表现程度不同,从在家中秘密穿戴到公开穿着异性服装外出,从部分异性着装如戴上胸罩、穿上内衣到全部女装并打扮成女性。这种情况可以叫做恋物异装症。有的患者在身着异性服饰时手淫,有的患者则在性交时部分或完全着异性服装。

有的男性同性恋者也常打扮成女性,目的是为了吸引同性,而异装症患者打扮成异性是为了唤起自己的性兴奋。一旦性兴奋消退,即不愿再扮为异性。异装症的发展可有三个过程:开始是为了好奇,穿上异性衣服,立即感到很强的性兴奋,是以前从来没有过的,即保留这些衣物作为“恋物”;以后便会全身穿上异性服装,打扮成异性,保持几分钟到几个小时,以唤起性兴奋,这时性的对象还是异性;如再发展,就会出现希望成为女人的愿望,但患者性身份识别没有问题。没有相应的性别转换症行为,如讨厌自己的阴茎,盼望手术改换自己的性别等,还不能说他们就是一个性别转换症患者,更没有同性恋倾向。真正的异装症患者的性爱对象仍然是异性。能和异性结婚并能和异性性交,但要先穿上异性服装才能引起性兴奋。

异装症也主要见于男性,少数见于女性。喜欢穿男服的妇女,各国各个时代都有,极少是真正的异装症患者。她们穿男性衣服打扮成男性往往不是为了唤起她们的性兴奋。异装症患者对异性服装感兴趣也多起于幼儿时期,自然会使人推想异装症可能是儿童被成年人打扮成异性促成的。偶尔也可见到由于父母的原因把男孩装扮成女孩。这样就可能形成了“条件反射”而发展成为异装症患者。如果真有此事,那也是浅层的、表面的原因。因为改变或塑造一个正常发育的儿童使之成为明显的异装症患者是不容易的。在群体中长大的男孩们自己早晚会发现自己有个阴茎而且能鉴别男女的不同,这时,把一个正常发育的男孩打扮成为女孩是会遭到他们反抗的。可以推测那些在幼年时被父母打扮成女孩而不加反抗反而高兴的男孩,必然还有其他尚未知道的深在原因。

异装症和单纯的恋物症不同,在后者,女服本身作为“恋物”是唤起性欲的对象。在异装症患者,唤起性兴奋的对象则是穿上女人服装的自己。提示患者在无意识中把自己和性的对象(妇女)等同起来,这个妇女常是母亲或母亲的替身。这种解释当然是自己不能知道的。

性虐待症是1899年首次提出的,包括主动的和被动的两种表现。主动向所爱的性对象施加肉体上的痛苦和心理上的折磨,从而获得性满足的行为叫施虐症,施虐狂患者可能在生活中遭受过挫折或欺凌,或遭受过异性的拒绝和侮辱,因而形成报复与反抗的心理,或系自卑感的过度补偿,以此反常行为作为性欲的发泄和表现男性的优越感。(1)至少6个月以来,反复多次以使对方受到心理或躯体痛楚(包括羞辱)而使患者感到性刺激的行为(实际行动,而且不是被激起的),来激起性幻想、性迫切愿望或行为。(2)这种幻想、愿望或行为,产生了临床上明显的痛苦、烦恼或在社交、职业或其他重要方面的功能缺损。主动要求性对象对自己施加心身的痛苦和折磨,即被动接受虐待,这样才能唤起他的性兴奋和性满足的行为叫受虐症,受虐症则多见于女性,它可能是害怕遗弃的恐惧心理的变态表现,也可能是内疚或罪恶感的自责自罚的表现。(1)至少6个月以来,反复多次以被羞辱、被捆绑、被殴打或其他受苦方式来激起性幻想、性迫切愿望或行为。(2)这种幻想、愿望或行为,产生了临床上明显的痛苦、烦恼或在社交、职业或其他重要方面的功能缺损。以肉体虐待取代正常的性行为,在摧残对方或体验痛楚中获得性的满足,这两种人往往结成伴侣,一个愿打,一个愿挨,许多患者还交替充当这两种角色。在正常成年人的性生活中,有时在达到性高潮时,双方都可出现轻度打骂、掐、咬等行为,如果没有过重的伤害而且不是靠这些行为唤起性兴奋,不属于性虐待症。即使真正的性虐待症患者,其行为动机也不在于故意使别人或自己受苦,而是这些虐待可使他们唤起性的激动情绪。仅有一方是性虐待症患者,不论患的是施虐症或受虐症,对方都会不能忍受,这种婚姻是难以持续的。施虐症患者的施虐行为可轻可重,一般是咬、掐或恶言辱骂。稍重的可能把性对象捆绑起来,辱骂、鞭打等。有人认为故意在公共场所偷偷地割破或玷污妇女的衣服,剪断女人的头发等以唤起性兴奋,也属于施虐行为。有的施虐症患者可作出严重的伤害行为。施虐症患者残酷行为的对象也可以是同性。残酷行为也可施加在动物身上,属于恋兽症的施虐症。和施虐症一样,受虐癖也多见于男性。

还有的患者在手淫时,勒紧颈部使自己处于轻度窒息状态,以增加性快感。认为这种情况也属于受虐癖的一种表现。一个人的性冲动走向虐待的路可有两种解释:第一是不论把痛苦加到别人身上或自己身上,这种虐待症倾向都是原始时代所有求爱过程的一部分,所以,这是一种返祖现象。

又称童奸,是指反复多次把青春期前未成年的儿童或少年作为性幻想和性活动的性对象,靠猥亵或奸污他们来引起性兴奋与获得性满足,而对成年异性对象和成年人性生活相对或完全缺乏性兴趣的病症。患者主要是男性。异性恋的恋童症多起始于成年以后,如果性对象是8~10岁的女童,其性活动常限于窥视或抚摸胸部或阴部。对年龄稍大接近青春期的女孩,可有性交行为。患者和所恋儿童年龄的差异和儿童未达完全性成熟,在诊断上有重要意义。同性恋的恋童症患者指向同性儿童,可有鸡奸行为。①至少6个月以来,反复多次以与未发育儿童(一般在12岁以下)的性活动来激起性幻想、性迫切愿望或行为。②这种幻想、愿望或行为,产生了临床上明显的痛苦、烦恼或在社交、职业或其他重要方面的功能缺损。③患者至少16岁,而且至少比A标准中所提及的儿童年长5岁。

注:不包括青年人与12或13岁儿童发生持久的性关系。对于年龄较大的青少年患者,未规定明确的年龄差距。确诊有赖于临床判断,如果受害儿童已届青春期后期,一般可称作儿童性骚扰或恋慕儿童。而不是恋童症。

恋童症患者偏好异性儿童和同性儿童的比例为2∶1,异性指向的男性偏爱8~10岁的女童;绝大多数案例中,患者均与受害儿童认识。患者较多采用眼观手摸而非生殖器接触。同性指向的男性患者偏爱10~13岁的男童,与受害儿童相识的比例远低于异性指向的男性患者。双性指向的成年患者选择小于8岁的儿童。专一恋童症患者只对儿童感兴趣,非专一恋童症患者也会对成人感兴趣。

恋童症患者的性活动可能局限于自己的孩子或近亲中的孩子(乱伦),也可能侵害其他儿童。玩弄儿童的患者可对受害儿童或其他宠物施以暴力或暴力威胁以阻止儿童泄露其暴行。恋童症的病程缓慢,可因药物依赖、抑郁、人格障碍而复杂化。

恋童症的病因不详。恋童症患者通常有不同的性问题,既可能是性心理发育不成熟,也可能是其他心理因素引起的性功能障碍。这些问题使患者无法与成年人进行正常的性交往。

恋童症是一种很难治疗的精神障碍。对不造成儿童身体直接伤害的恋童症患者,医学治疗似乎比法律惩处更可取。治疗主要采用厌恶疗法,同时辅以心理动力学治疗,并注意纠正个体与成年人性交往中存在的问题。对儿童造成伤害的恋童症患者,法律制裁是必要的。

指对动物眷恋并和动物发生性行为,是性变态中对人类社会危害最小的行为方式。多发生在农村或牧区。所用动物大多是各种家养动物,如猪、马、牛、羊、驴、狗等。妇女也可能与雄性宠物狗交媾。成年人与兽相恋并与之交媾大多是正常性欲受到禁制的结果,由于正常的性宣泄受到阻碍,便转向兽交,这时动物成了唤起性兴奋的主要对象。农村放羊娃等有机会和动物的密切接触,往往是境遇性的行为,故不属于此列。偶尔因性伴侣不方便如患病或分居,与动物试着交媾,或偶尔在酒醉后有这类性活动的,还不能算是恋兽症。兽交行为可能和幼年性欲有关,儿童不能清楚地区分对待动物和对待人应该有不同的感情。如果儿童时期对动物产生过深的感情,形成无意识的固结,成年后当性欲遭到挫折无法宣泄时,便不自觉地用幼年方式来满足性欲了。

以异性尸体为性对象并和尸体交媾叫作恋尸症,俗称“奸尸”。仅见于男性。一般认为恋尸症很少见。恋尸症偶尔可见于殡仪馆和医院停尸间等停放尸体的场所。但是,除了这种行为被揭发或判刑之外,别人是不可能知道的,而患者本人是不会因此找医生求助的。如一位农民,男性,27岁,只受过小学教育。17岁时和一有夫之妇同居,受到双方家庭批评,指责,曾被捆打致昏迷达数小时。被捆打后,增高的性兴奋无法宣泄时,即转向异常的渠道,掘墓,奸尸。患者自幼家贫,体弱,三四岁才开始学走路,由养父母抚育。小学功课常不及格,勉强毕业,即参加农业劳动,会干简单的农活。性格固执、任性,不服大人管教。平时话少,和人交往不多,一般人际关系尚好。检查患者没有发现明显智力障碍,并且知道自己的行为是错误的。自诉自从和某妇女姘居后,就难以控制自己。承认奸尸比和妇女性交能得到更大的性快感。

ICD-10对其描述为:有时在一个人身上可同时存在一种以上的性偏好障碍,其中无论哪种也不占优势。最常见的是恋物症、异装症和施虐受虐症的结合。

其他各种类型的性偏好与活动也可发生。但每种相对少见。包括以淫秽的语言打电话(淫语症)、以勒颈或缺氧的方式达到部分的自我窒息,增加性兴奋(性窒息症),或偏爱那些解剖上异常的性伴侣如截肢者(恋残肢症)。

总之,性欲活动变化万千,有许多很罕见或属个人的症性,以至无法一一命名。喝尿、涂抹粪便或刺穿阴茎包皮或乳头都可为施虐受虐症中的行为模式。各种类型的手淫方式更为常见,但更极端的手段,如将物体插入直肠或阴茎尿道,一旦这类行为取代了正常的性接触,便达到了异常的程度。性偏好障碍的实质是患者的性心理不成熟发育,性格过于羞怯,在社会上缺乏与异性交往的机会或能力,害怕以正常的方式去求爱或做爱,故表现出异常的或幼稚的行为方式。


\section{第四节 治 疗}

人们患了病感到痛苦,才主动要求治疗,并和医生合作。因此,患者对治疗的动机和主动性是治好病的首要条件。许多精神病例如精神分裂症,患者对他们的病态思想和行为没有自知力,不承认有病,当然不主动求治甚至拒绝治疗。但是,当代有了抗精神病药物,在必要的时候,可以对患者施行强制治疗,患者的情况照样可以好转。但对于各种变态的性心理和性行为,目前还没有有效的药物。对于少数有严重侵犯别人的性偏好障碍例如顽固的恋童症,在欧洲和美国已有人试用抗雄性激素如醋酸环丙孕酮(cyproterone acetate)等以压抑基本的性冲动,据说有效,但患者同样不愿主动接受。

对于多数常见的性偏好障碍患者,到目前为止,心理治疗是主要方法。然而,他们在很长时期内,都不能清楚地意识到自己的行为是病态的,虽然他们可以清醒地承认这些行为是错误的,是违反社会道德准则和法律的。又由于可以从这些行为中得到极大快感,他们多不主动求治。有些严重侵犯别人的性变态行为如施虐症或色情谋杀,患者一旦被发现,主要是要他服刑的问题,而不是给他们治疗。那些侵犯别人不严重但为社会所不容的变态性行为如露阴癖、挨擦症、窥阴癖及较重的异装症等,患者的行为也不会自动停止,直到有一天被发现,受到社会的非议和法律的制裁,这时,患者才不得不主动求治。有些恋物癖患者或异装症患者,如果无人揭发,其性配偶可以容忍,也不会主动求治,但恋物癖患者为了得到妇女用过的“恋物”,大多成为惯窃,在被发现受到惩罚时,才可能下决心求治。许多事实证明,对于有侵犯行为的性偏好障碍患者,用对成年人的教育方法如劝说、批评、降级、拘留甚至判刑来纠正他们的变态性行为,见效甚微。患者在服刑期间可以勉强克制自己,一旦恢复自由,仍寻机行动。我们的经验是,这些惩罚虽然直接效果不大,但却是非常必要的,只有这样才能促使他们下决心求治并和医生合作。

最常用的方法是行为矫正法,用厌恶技巧来破除变态性行为的“条件反射”,同时培养正常的性行为。由于恋物活动总是伴有手淫的高潮体验,所以应该在未达高潮之前给予患者一个恶性或不良刺激(如电流刺激、弹皮筋、催吐剂的应用),由于性反应遭阻断而骤然跌落至低谷,就截断了复制体验,强化了大脑建立警戒信号,形成了正常心理防卫。方法是用图片、实物(如女人内衣、鞋袜等)或在实际对象前,让患者想象他所迷恋的对象或情境,激起他们变态的性兴奋,同时用一定强度的电流刺激手腕部皮肤使之疼痛,或肌内注射催吐剂使之产生呕吐反应,也可以要求患者在遇到可作出变态性行为的情境时,想象自己被人捉住当众殴打、羞辱的场面。由于恶性刺激对已形成的条件反射有消退作用,这样反复几次,旧的条件联系也就消除,病态性行为也将随之消失。目前,人们对这些可能带来许多负面影响的厌恶疗法的使用越来越持保留态度。在进行这种治疗时,医生还要求家人和有关人员的合作,对患者持谅解态度。对已婚患者还要尽可能使他和配偶之间建立正常的性行为。必要时可服抗抑郁药,启动大脑正常意识部分管理的“隋性灶”,不再发生变态的性兴奋灶。使患者学会抵抗不良刺激,阻止和缓解倒错的性冲动,转化和控制自己的行为,最后摆脱不良刺激的侵扰,实现自我镇静,自我控制,不再怨天尤人,不再拿别人当出气筒,而是发奋工作,努力进取,从而以积极的心理反应战胜心理障碍。

钟友彬生前根据心理分析的原理设计创用了认识领悟心理疗法,用来治疗露阴癖、窥阴癖、挨擦症和恋物癖,并得到极好的效果。只要患者有求治愿望并和医生合作,不用求得家人或其他有关方面如单位领导和公安人员等的谅解和帮助,不用对患者进行正面教育,也不用叫患者忍受恶性刺激的痛苦,经过10次左右的会面和交谈(每次1小时),医生设法启发患者认识变态性行为的幼稚性,告诉他们这些行为实际上是幼年儿童的幼稚取乐行为,不是成年人的行为。同时让患者回忆幼年生活情况,几乎无例外地都可回忆起在幼年儿童期各种性游戏经历。患者通过医生的启发性谈话,联系幼年经历,对医生的解释由怀疑到相信,逐步有所领悟。当患者一旦认识到他们的行为原来是幼年儿童的性游戏的持续或再现后,常有大梦初醒的感觉,羞耻心立即恢复,感到以往的行为幼稚可笑,最后不但自愿放弃变态的性行为,连同要“干”的愿望都一起消失。

为了介绍认识领悟疗法,特地介绍钟友彬教授接诊的一个实际案例如下:患者男性,1944年生,高中毕业,机关职员。1988年9月由患者工作单位两位领导人陪同患者自外地来北京看病。领导人单独向医生介绍说,患者参加工作10余年,业务能力强,是工作中的骨干。平时为人“忠厚老实”,上级曾多次要重用他。但由于他曾多次在公共场所向妇女显示生殖器,群众影响很不好。和他熟悉的人以及他的妻子都认为他的这种行为和他平时的生活和工作作风极不相称。前些天在一个科普刊物上知道这种行为可能是病态,如果真是病,请医生给他治疗。医生没有要求领导人对他谅解,也没有要求他们帮助医生给他做“思想工作”。直接单独会见患者。患者身体发育正常,外貌和实际年龄相当。情绪表现较低沉,可以按医生要求叙述他多年来的“流氓”行为产生过程:来诊前16年,即1972年,当时28岁。在一次下班回家的路上,看到五六个女学生迎面走来。当时路上行人不多,他突然有一个冲动要把阴茎取出来,当即行动并直对她们走去。对方看了以后转身走开。事后被公安部门叫去,批评了一顿,回单位后又做了检讨,被记过一次,感到“丢了脸”。本来性格就孤僻,从这以后,更觉得在人前抬不起头来。以后时常有同样的冲动。有时想,对方人多总会有人告状,便只向单个的女青年显示。对方虽有时表示惧怕或气愤,但没有揭发他。有几次在商店里,乘人多之时,挤在人群中间,取出阴茎挨触妇女的手背,感到很舒服。1973年(29岁)夏天,有一次在商店里,刚取出阴茎便被值勤人员发现,送他到公安局拘留了两周。这一次真的害怕了,极力克制自己。除了上班以外,平时不敢外出,时常手淫。有二三年没有行动,内心里感到非常压抑。1977年(33岁)经人介绍和现在的妻子结了婚,第二年生了一个女孩。患者说,他对和妻子性交的兴趣并不大,愿意叫妻子摆弄他的阴茎,这种快感比性交还要强。另外,婚后一段时期,仍有时偷偷地外出用阴茎挨触妇女的手,幸未被发觉。妻子知道他以前有过这类行为,但能谅解他,时常劝他不要去人多的地方。1979年(35岁)有一次在极难克制的情况下,在商店里取出阴茎,正要挨触一青年女子时被发现,第二次被拘留半个月。患者说,他也知道自己的这种行为是人们特别憎恶的,几次被揭露、拘留,感到没有脸面和别人来往。除了上班时好好工作外,大部分时间待在家里。越是寂寞、苦闷,越想外出行动,非常悲观。看到上级、妻子甚至有些女同事对他都持原谅的态度,更感到无地自容。1986年和1988年又有过两次被公安部门拘留,以后被交给本单位“严加管制”。这时,公安人员和单位领导都怀疑他可能有病了,但他本人并没有意识到是病。医生要他谈谈1972年第一次出现这种行动以前的经历,患者回忆说,父亲在他出生后不久即死去,由外婆抚养。自幼性格孤僻,没有十分亲密的朋友。高中毕业后到农村插队,曾和一个女青年相爱,因对方回城市工作而分开。心中很苦闷。参加工作以后,想恋爱,经人介绍认识了几个女朋友,都没有成功。终日感到郁闷,时常手淫并幻想着和以前的女友性交。正在这时,出现第一次露阴行为。每次行动后都感到心情轻松,舒服,但之后又后悔、害怕。在确定诊断为露阴癖和挨擦症以后,医生和他进行了简单的谈话,医生问:“你自己是怎样评价你的这些行为的?”患者:“是流氓行为”。医生:“为什么说是流氓行为呢?”患者:“在公共场所叫妇女看我的生殖器,还用生殖器‘顶'她们,这种行为谁不说是流氓呢?”医生:“你既然知道这是流氓行为,而且多次受惩罚,为什么还要去干呢?”患者:“(停顿一会)我也不知道”。医生:“在作出这些行为的时候,你怎么想的?”患者:“我也说不清楚……当时‘糊里糊涂地就拿出来了',事后才后悔、害怕,但已经来不及了”。医生:“可是,据你的同事讲,你人还老实而且头脑很清楚,怎么能老是‘糊里糊涂'地作出这些坏事呢?”患者:(不回答),医生:“你认为你的这些行为是病态的吗?”患者:(低头不答),医生:“你认为你的这些行为别人能理解吗?”患者:“不能。(停了一会说)我也不能理解为什么到时候非做不可”。医生:“有谁强迫你做吗?”患者:“没有”。医生:“(重复以前的问话)既然你自己知道这是不好的行为,而且受过多次惩罚,为什么还要非做不可呢?”患者:(不答),医生:“假如我们带一个二三岁的小男孩到商店去,他突然说要撒尿,便拿出‘小鸡'在许多女顾客面前撒起来,人们会理解他的行为吗?会惩罚他吗?”患者:“人们不会惩罚他,会理解他的。弄脏了地,我们给人家擦干净就是了”。医生:“人们为什么不惩罚他,为什么能理解他呢?”患者:“因为小孩子不懂事,人们当然不会惩罚他”。医生:“你是说小孩子不懂社会道德,不懂这种行为是不礼貌的行为,对吗?”患者:“对”。医生:“回过头来说,你在公共场所取出阴茎向妇女们显示甚至挨触她们,就行为本身来说和上面提到的儿童行为表面上有什么不同呢?”患者:“(沉思一会)我从来没有这么想过……,就行为来说,好像没有什么不同”。医生接着向他解释说,就行为本身来说,两者没有什么不同,所不同的是这种行为小孩子作出来,人们能理解,能原谅,甚至还可能觉得他露出“小鸡”好玩,但你这个40多岁的成年人在同样场合也这么做,不管什么理由,不管是有意的或无意的,人们都不会理解,不能谅解,而必然会对你指责和惩罚。医生还指出,在一个身体和智力发育都正常的成年人身上作出这种违反社会道德的儿童式的行为,就是一种心理病态。这种病是可以治好的,但用药物和别的医疗方法都没有效,只有心理治疗才能治好。要在医生指导下认真思考,懂得了病的道理,才能放弃这种行为。必须和医生合作。医生还告诉他,他的这种行为,不是到20多岁才突然无故产生的,而是有它的幼年根源。希望他回去后,仔细想一想今天和医生谈话的内容,并写出书面体会。尽量回忆一下幼年有关的经历。医生在谈话中没有指责他,会见结束时,患者的情绪较初见面时表现轻松些,愿意照医生的嘱咐办。一周后第二次会见时,患者说,上次和医生谈话,医生没有用“大道理”批评他,只说他有儿童的行为,心里轻松了一些,不老是责备自己犯罪了。患者说,他回忆了过去的经历。在他出生后两个月,父亲便死去。5岁前后母亲改嫁,他便和外婆一起生活。大约6岁前后,邻居一个十七八岁的男孩让他玩弄对方的阴茎,只感到有趣,但对方没有看、摸他的阴茎。此外,想不起自己会有类似的经历。患者又说,今天在候诊期间和另外两个同类患者谈天,他两人都已经治好,而且都想起了幼年期和女孩们玩耍的经历。这时,他突然想起一件往事:大约在6岁前后,有一次外婆不在家,他和同村子里的一个比他大两岁的女孩一起玩,当时只有他两个人。女孩关上门,双方都脱下裤子,女孩要他把阴茎插到她的阴道里,但阴茎没有勃起,女孩用手玩弄他的阴茎到半勃起时,试了一下,也未成功。以后就罢了。记得当时感到很舒服,有些兴奋,但也有些怕羞。患者说,当他回忆起这个经历时,体会到上次医生说他的这种行为有幼年的根源是很对的。也明白了他为什么在结婚后老是希望让妻子摆弄他的阴茎,这样得到的快感比性交还强,其根源就是这次经历。医生称赞他能认真思考,和医生合作。鼓励他继续思考他的行为的幼稚性以及和幼年经历的关系。医生还解释说,儿童和成年人不同,儿童还不懂得真正的性交,但这些性的游戏也会使孩子们产生快感。性游戏经历表面上虽已忘掉,但未消失,这种快感在内心深处留下痕迹,成年后遇到挫折或困难处境无法解脱时,便会不自觉地用幼年的性取乐方式来解除成年人的苦闷。因此,一半是成年人的心理,另一半还有幼年的心理在支配着他的某些行为。这就是他病的本质。认识了这个道理,应经常想到自己已经是成年人了,不应再作出可笑的幼儿行为了。患者表示能理解,愿意多思考。以后两周内又进行了两次会见。患者写了他的生活经历和会谈后的体会。患者说,在来治疗以前,他不懂得病的道理,也从来没有想到这种行为是儿童式的取乐方式,只是认为自己的思想品质不好。但是,多年来多次下决心改正,就是改不掉。和医生谈了几次话以后,确实体会到以前好像有两个人在支配着自己的行动,一个大人,一个小孩。在平时,在工作的时候,是大人在支配,但在“冲动”的时候,又是小孩在支配着寻机去干,但并不“糊涂”。作出幼稚的蠢事后,大人又回来指挥了,即感到羞愧和后悔。经过医生的解释和自己的思考。明白这个道理以后,以前不理解的行为现在可以理解了。说也奇怪,要去干的冲动一下子轻多了,虽然有时还想去干一下,但可以用成年人的心理来控制了。因此,对改正这个毛病真正有了信心。医生问他,在他内心里是否认为妇女们愿意看、摸他的阴茎,欣赏他的露阴行为时,患者立即承认。并说,关于这一点以前也没有深入地想过,但内心里确实有这个想法。现在想起来,根源也是在幼年,记得当年那个小姑娘看、摸他的“小鸡”时,确是表现得很惊奇和喜爱,回忆起来印象还很深。在和医生讨论中,患者结合几次被揭发的事实认识到:“从道理上讲,一个作风良好的成年妇女,对我的行为是不能容忍的,即使是作风随便的妇女,也不可能在大庭广众下忍受我的污辱。看来,在我内心深处,还摆脱不了6岁时留下的印象”。医生让他继续思考幼年的性游戏和成年人正常性生活的区别。第五次会见时,患者说情况大见好转,同时交上书面体会,叙述了好转的过程:“在来治疗以前,我总是认为自己犯了流氓罪,很痛心。想改,努力控制自己。结果,越控制,要干的冲动越强烈,不知为什么。现在想起来,原因是没有找到病的根子,不能从内心里挖去这种坏行为的根源,不能打消产生这种行为的念头。一有机会就想去干。每次行动后也确实能得到短时的满足感,以致‘瘾头'越来越大……。通过和医生的几次谈话,情况有了很大转变。首先是心理压力减轻了,不再有犯罪感。认识到那是不懂事的孩子好奇、取乐的游戏行为,现在我已经是40多岁的成年人,怎么还可以作出这些幼稚可等的行为呢?明白了这一点以后,脑子里刚一冒出小孩的念头,就立即被‘我是成年人了'这个想法压了下去……,这些日子以来,小孩的想法越来越少。现在一个人到公共场所去,原来的那种冲动几乎没有了,也不用故意躲开人多的场合,更不用有意地控制自己了。”医生建议他继续思考,想想在性生活问题上还有哪些幼年经历的残余影响。患者很合作。1周后,第6次会见。患者很高兴。反复地说,在来治疗以前,认为自己前途暗淡,情绪悲观。后来领导叫他来看病,他自己起初感觉不到自己的行为是病态,更不相信医生的心理治疗会使他改过来。在半信半疑的心理状态下来接受治疗。经过了5次谈话,在医生的启发、解释、指导和讨论后,他明白了许多以前不清楚的问题,心里发生了根本性的改变,真正信服了。患者在书面体会中说:“近来,每当我想到如果有一个女人触摸我的阴茎该有多么舒服时,就自然而然地想到这是小孩子的行为,是小孩子时候留下的印象,马上感到这种想法和做法都是幼稚可笑的。这些想法也就自然而然地消失了,根本不需要控制它……。现在我可以随便到商店或其他人多的地方去。”在和医生谈论中,患者承认,有时看到妇女白嫩的手,还感到有些吸引力,但已不能引起太大的冲动了。医生指出,他之所以还残存这种感情是因为在他内心深处还以为有极少妇女愿意主动看、摸他的阴茎。当然,也不能完全排除这种可能,但可以肯定地说,在大庭广众下愿意这么做的妇女是绝对不可能有的,除非她患有精神病。患者同意医生的分析。1988年11月下旬,第七次会见,也是最后一次。患者自称病已经完全好了。写了接受心理治疗的长篇感想和体会。他说,几天来他回忆了结婚以后的夫妻性生活情况,他不是以成年人的性交来获得最后的满足,多数是要妻子抚摸他的阴茎而达到排精,也给妻子带来痛苦。联系幼年经历,充分认清了变态性行为的本质,感到非常荒唐、幼稚可笑。在书面体会中最后写道:“我的那些小孩子行为给我带来无限痛苦,也对不起我的妻子……,现在,我要作为一个真正的成年男人踏上归途,就要回家和妻子会面了,我产生了真正和妻子过成年性生活的愿望……”。1年后,患者来信,行为已完全正常。

性偏好障碍男子虽然具有性心理障碍,但他们基本上都会结婚成家,过上貌似正常的夫妻生活,然而,由于他们的性欲“低下”或者说“醉翁之意不在酒”,他们的妻子会作出何种反应呢?因为这些男子往往行事谨慎,循规蹈矩,所以在没有发现丈夫的变态行为之前,夫妻尚可相安无事,不会出现太大的问题。然而事情一旦败露之后,没有哪位妻子会情愿接受丈夫的变态行为,不过她们中的多数仍能容忍这一事实。据国外一项调查表明,60\%的患异装症患者的妻子仍把她们的丈夫描绘为“很敏感、很满意和很不拘束”或“很宁静、很软弱、很温柔和很可爱”,这是因为这些男子在他们的日常生活的其他领域中,仍能按照正常男子的方式行事,所以她们尚能够耐受丈夫的异装行为。

这些异装症男子的妻子对于丈夫竟是一个性偏好障碍患者的事实需要经历一个心理调整的时期,虽然她们作出的反应各不相同,但大体上总会按照下列方式发展的:

大多数妇女是在婚后发现丈夫的这一行为的,但也有大约1/3的妇女在决定结婚之前,就已发现男子的变态性行为。她们在发现之时的反应是大不相同的,有的人感到震惊、慌乱或责怪自己有什么地方做得不对,也有的人感到怨恨、愤怒和强烈反对。她们最终还是会接受这一事实,并把它保持为婚姻之内的一个秘密。

当她们发现这一秘密后,往往要通过阅读或咨询等手段了解与此有关的信息。这种迫切要求理解这一现象的动机可能是由于好奇而激发,也可能是出于对自己前途的担忧。这种理解可能缓解她们对自己的责备,也可以有助于实现她们心理上的平衡,因为她们总试图把它解释为一种丈夫自己也无能为力的“疾病”,这样就不必责备任何一方,只好默认这一事实。

在这一阶段的主要表现是为丈夫设置一个界限,以尽可能保守秘密,“家丑不可外扬”;忽略丈夫在穿着上的怪异举动,多想丈夫的优点,用“妇女可以穿牛仔裤,为什么男子就不能穿裙子呢”的想法给自己宽心。如果丈夫或男友超越所设置的界限,或使他的行动逐步升级,女方的反应也会升级,她可能收回自己的感情并对丈夫(或男友)充满敌意,最后可能导致分手。

当妻子对丈夫的异装症表现感到更适应和更宽容时,她可能开始帮助丈夫使用化妆品,帮助他梳妆打扮,甚至帮他选购女性化服装。妻子可能感到有必要保护丈夫,陪伴他外出旅游,甚至与他一起出现在公共场合。当然,妻子也会向医生或其他咨询中心寻求帮助,以便治愈或控制丈夫的举动。异装症患者的女友可能通过上述心理调节和适应过程而与男方结婚,患者的妻子也会通过调整和适应过程而与丈夫和睦相处。医生在无法治愈患者时,起码也可以帮助这些伴侣顺利地渡过这几个阶段,直到达到主动参与的过程。比如上述那位因丈夫脑部受伤后导致异装症的妻子还挺身而出为他辩解。

总之,像异装症这样的性偏好障碍患者,仍可以组织家庭并过上近似正常人的夫妻生活。


\section{第五节 性偏好与法律}

一个人的性活动除了自己手淫外,不论是正常的或变态的,都需要有客观对象。在现代文明社会里,正常成年人的性活动是异性双方自愿的。除了受虐癖、兽奸症以外,大多数性性偏好障碍患者都使他们的性对象受到污辱、侵犯或伤害。恋物癖和异装症患者为了得到他们的“恋物”也常有盗窃行为。这些都是触犯法律的行为。精神病学工作者在司法部门的委托下,应对这些触犯法律的当事人进行精神鉴定以确定他们有无责任能力。在国内外性犯罪案件中,性偏好障碍引起的犯罪占有不小的比例,尤其是露阴癖,可以占19.23\%~33.2\%。我国刑法第十五条规定:“精神患者在不能辨认或者不能控制自己行为的时候造成危害结果的,不负刑事责任……”。性变态虽被认为是一种精神障碍,但患者在作出危害行为的当时,意识是清醒的,对他们的危险行为没有完全丧失辨认能力和控制能力,因此,他们都具有对行为的责任能力。对那些作出严重的伤害行为的施虐症和奸尸症案例,大多数从事精神鉴定的精神病学家都认为他们具有完全的责任能力。至于露阴癖、窥阴癖和挨擦症等,患者都能认识到变态性行为的危害性,承认他们的行为是人们憎恶的,是社会不能容忍的。控制自己行为的能力虽然减弱但没有丧失。多数精神病学家认为他们至少应具有部分责任能力,由司法机关酌情给予的处罚,但由于所造成的危害多不严重,在实行应有的处罚后,要督促他们接受治疗。

(钟友彬)


\chapter{第十八章 同性恋}

中国古代就有用“男风”、“余桃”、“断袖”、“龙阳”、“安陵”等来描述这类群体的习惯。同性恋一词首先在19世纪60年代由一位匈牙利作家在文学作品中提出,该名词被德国学界采用并被翻译成英文“homosexuality”,是一个含义模糊的概念,既可以指一个人感受到同性吸引力的性取向,也可以指一个人是同性恋者。20世纪20年代它被译为中文“同性恋”,见于有关性教育著作中。虽然衍生出同性恋一词,但同性恋这一术语仅见于港台学术及部分非学术专著。20世纪80年代初,同性恋一词又常被引用于我国相关学术文献、专著和性学科普著作中。在现代社会,在对性伴侣的选择拥有充分自由的条件下,一个性成熟的个体如果具有明显或强烈的指向同性的性欲或同时存在主动的同性性行为,方可视为同性恋者。假如个体仅有偶然的同性性行为,但有关性取向的意识模糊,可视为同性性行为而不宜简单地判定其为同性恋者。同性恋者(homosexual)包括男同性恋者(gay,gay men,male homosexual)和女同性恋者(gay women,lesbian,Les),在我国及国际华人社区为避免歧视,男同性恋者常自称gay或“同志”,女同性恋者常自称“Les”或“拉拉”。习惯上,双性爱者常被一并归入同性恋者中。

科学界已认定同性恋不是性变态而是属于少数人的自然现象。


\section{第一节 概 述}

进入青春期后,几乎每个人都会出现在性生理变化上的性心理发育,继而感受到来自他人及物品的性吸引力。人类性取向(sexual orientation)又称性倾向、性定向、性欲指向等,主要指个体感受到的性引力或性爱对象是来自异性还是来自同性的倾向或心理。同性恋性取向是指心理上感受到的性引力完全或主要来自同性。异性恋(heterosexuality)是与同性恋有明显不同的性取向,这一性取向的个体感受到的性引力主要或完全来自异性。双性爱(bisexuality,Bi)指几乎相似地感受到男女两性性引力的性取向,一般被认为是位于同性恋与异性爱“两端”间的中间体。异性爱、双性爱和同性恋并不是孤立的三个“点”,它们是性学界为便于区分而给出的学术概念。有些人感受到的性引力处于这“三个点”之间,如较双性爱更偏向于异性爱或同性恋者。

由于种种原因,在某些环境下人类性行为与性取向并完全不一致。根据性行为主体一般分为男女性接触者、同性性接触者。后者包括男-男性接触者(men who have sex with men,MSM)和女-女性接触者(women who have sex with women,WSW)。同性性接触者所涉及的人群通常比同性恋者广泛得多。男男性接触者又称男同性性行为者、男-男性行为者,指与男性发生(过)性关系的男性,包括与男性发生(过)性关系的男同/双性爱者、尚未行变性手术的男变性欲(易性症)者(transgender,transgender people)和偶尔或长期参与男-男性活动的男异性爱者。科技部攻关课题“男同性恋人群艾滋病流行向一般人群的传播模式及阻断策略研究”(性病艾滋病专家张北川等,2006)调查发现,在2250位男-男性接触者中,男同性恋者占52.0\%、男双性爱者占38.0\%、男变性欲者占0.9\%、有男-男性接触史的男异性爱者占9.1\%。男-男性接触者中“最活跃”的亚人群是男同/双性爱者,变性欲者因占极少数常被忽略,男异性爱者在同性性活动中“活跃”程度一般较低。男-男性接触者这一术语是学术界在艾滋病(acquired immunodeficiency syndrome,AIDS)流行以来提出的行为学概念,并广为流传。女-女性接触者指与女性发生(过)性关系的女性,以女同性恋者为主体,也包括部分非同性恋女性。


\section{第二节 国内外研究的现状}

在西方,基督文化原来并不特别反对同性恋,“十诫”里唯一反对的性行为是男女通奸。在基督前时代的后期,犹太人愈来愈不能容忍同性恋,这可能有两种原因:1.同性恋在异教徒中较为多见。2.不以生殖为目的而射精似乎违反了上帝的旨意。旧约圣经利未记里明言,两个男人同床睡是邪恶的,两人都应处死。基督教义继承了希伯来人反对同性恋的传统,甚至更加仇视同性恋,西方文化对同性恋的态度就是这样发展来的。即使到了现代,虽然从法律上来说西方对同性恋的反对和敌视已经大大缓和甚至消失了,但社会舆论和一般老百姓仍对此持否定态度,甚至极为蔑视。

在中国,从奴隶社会开始同性恋现象便公开流行,并同许多人在异性恋活动中的放纵行为交织存在。直至秦到汉,同性恋一直未受到直接的谴责和批评;自三国到南北朝,人们对同性恋现象在道德方面亦未予以明显的歧视;隋唐至明清,社会主流性道德对同性恋现象也相当宽容。中国历史上不少小说中都有对同性恋现象的描写,如《红楼梦》、《金瓶梅》等,更有《品花宝鉴》一书,完全是以描写梨园界的同性恋为主题的。近代我国称同性恋风气为“男风”,又称“南风”,因为这一风气“闽广两越尤甚”。男同性恋者互称契哥契弟,女同性恋者则结拜金兰。明清两代南方女性同性恋盛行,她们互相结盟,亲如夫妻,居住在称为“故婆居”的房子里,这里是男子禁止入内的。高罗佩也注意到清代对同性恋宽容,对异性恋反而严厉的态度:当时社会对这些公开表现男人手拉手在街上走,戏剧表演中出现妾童等现象相当宽容,反而把异性恋严格限定在私人生活的范围内。有人对我国历史上各个朝代同性恋风气兴衰的看法与上述略有不同,这种观点认为:中国古代同性恋的存在状况是前后一致的,不能仅以古籍上对同性恋现象记载的多少来推测各朝代同性恋风气的兴衰。我们认为,用这种观点来推测明清以前的情况是错误的,但由于明清禁女娼而使社会上层人物中同性恋活动兴盛起来也是合乎逻辑的推理。因此,情况很可能是这样的:在前几千年,同性恋活动状况大致平稳,到明清达到一个小小的高潮,而这个高潮是由政府的禁娼规定所导致的。

目前中国多数百姓对同性恋行为持否定或鄙视的态度,从道德上的指责再到认为是病态行为的理解,虽然近几年对这些歧视有了明显改善,同性恋者也逐渐被周围人所理解和接纳,但对于公开自己的性取向仍旧比较慎重。

同性恋者在性成熟期(成年)人口中所占比例(百分率)的多少,一直是备受关注的问题。人类学研究表明,在任何社会不同的历史阶段,不同的政治、经济、文化及风习背景下,均存在同性恋现象。欧洲某些国家自20世纪初最先开始了这一方面的调查。在20世纪前中期,一些国家也进行过相似调查。由于一度对同性恋者的概念有某些争议,特别是调查方法的差异,早期对同性恋者所占成年人口比例的调查结果存在很大差异。直至艾滋病时代,许多国家的学界和政府才开始重视同性恋问题,特别是关注同性恋者的数量,尤其是男同性恋者的数量。在同一个国家不同时间进行的调查结果显示,同性恋者所占成年总人口的比例是相当恒定的。无论社会制度、文化背景如何,成年男性中2.0\%~5.0\%是男同/双性爱者,而女同/双性爱者在成年女性人口中所占比例大约为前者的一半。

罗蒙(1902)对荷兰600位大学生的调查发现,其中2.0\%是男同性恋者、4.0\%兼有两性性行为。赫希菲尔德(1920)调查了德国3600位成年男性,发现其中2.3\%是同性恋者、3.4\%兼有两性性行为。他结合其他学者的研究,推测各国3.0\%男性专意于男-男性行为。美国著名的性学专著《金赛报告》(1948)指出,男性中有10.0\%一生中维持过至少3年的同性性关系,37.0\%至少有一次同性性行为,4.0\%终身只有同性性行为。后来保罗·格普哈德对《金赛报告》中大量原始数据深入分析后估计,4.0\%成年男性和1.0\%成年女性中有同性恋倾向,平均比例为2.5\%。

金赛的七点连续谱包含了从完全异性恋到完全同性恋的范围。即“①完全异性恋,没有同性恋;②以异性恋为主,偶尔有同性恋;③以异性恋为主,同性恋占少数;④同性恋与异性恋相等;⑤以同性恋为主,异性恋占少数;⑥以同性恋为主,偶尔有异性恋;⑦完全同性恋,没有异性恋”。金赛将他的分类建立在公开的性行为和性欲吸引相结合的基础之上。根据美国国家健康与社会生活调查,估计大约有2.8\%的男性和1.4\%的女性认为自己是同性恋。双性恋则可以通过对男性和女性两者的公开性行为和(或)性欲吸引的反应来归类。由于既有同性恋又有异性恋,对双性恋的明确定义就很难建立。四种双性恋包括:真正双性恋、暂时双性恋、过度双性恋以及同性恋否定性双性恋。

当代美国经严格设计的大样本随机抽样调查(1994)发现,男性中2.8\%自认为是同/双性爱者,6.0\%的男性能感受到同性的性吸(引)力,9.0\%的男性青春期后有过同性性接触。其中,美国前12大城市中9.0\%的男性自认为是男同/双性爱者,第13~100位大中城市中为3.0\%~4.0\%,其他地区仅1.0\%。研究表明,这一悬殊差异首先是孤独感驱使同性恋者迁徙的结果。

法国的权威调查(1993)发现,有性行为史的男性中,4.1\%曾有同性性伴侣,与该国1972年的调查结果基本相同。

20世纪90年代,联合国艾滋病规划署(UNAIDS)组织的对多个国家的调查发现,与男女均有过性行为的男性所占男性总人口的比例分别为:泰国6.0\%~16.0\%,挪威3.0\%,博茨瓦纳和秘鲁15.0\%,墨西哥0.5\%~3.0\%,美国10.0\%~14.0\%,巴西5.0\%。

我国大量史籍表明,同性性行为自古有之。在当代,社会学家刘达临等(1992)调查发现,约0.5\%城市已婚男性、2.3\%农村已婚男性和7.5\%男大学生有同性性行为经历。他指出,这一调查未涵盖未婚的非大学生人口,而且“绝大多数”男同性恋者“迫于社会的压力”拒绝承认自己的同性性行为,估测我国已婚者中男同性恋者的“实际数字要比以上数字大得多”。后来,刘达临等(2005)根据多项调查估测,我国男女同性恋者人数不少于3000万。社会学家潘绥铭等(1995)随机抽样调查显示,男大学生中,4.2\%既有同性恋倾向又有过男-男性行为,约4.0\%只有某种同性恋心理倾向而无男-男性接触史。社会学家李银河(1998)根据一些人类社会学调查,提出我国人口的3.0\%~4.0\%,即3600万~4800万人是同性恋者和成年后会成为同性恋者。中华医学会精神科分会负责人陈彦方(2001)认为同性恋者占我国人口的2.0\%。张北川等(2002)根据我国2001年人口统计公报给出的15~60岁年龄组男性人口数量和多项国内外调查结果分析,我国该年龄组有1018万~2545万人,平均约1782万是男同性恋者。其中有524万~1024万,平均为774万男同性恋者生活在城市和由农村乡镇流入城市。潘绥铭等(2004)随机抽样调查发现,12.6\%男大学生有过同性性接触,在大学期间男生新发生同性性接触者占其总数的6.0\%,男生中既有同性恋倾向又有同性性行为者占4.3\%。同年,中国疾病预防控制中心公布的调查结果表明,男同性恋者约占我国性活跃期男性人口的2.0\%~4.0\%,据此估算我国15~49岁男性中有500万~1000万男同性恋者。

此外,张北川等对我国男-男性接触者占男性人口的比例进行了估测。他在1999—2001年主持的三次调查中发现,约半数男同/双性爱者与自认为是“异性爱”的男性发生过性行为,所有参与调查者与之发生过性行为的异性爱男性的数量稳定,人均4~5人;以男同性恋者占男性人口的2.0\%~3.0\%的4倍计算,男男性接触者总计占男性成年人口的10.0\%~15.0\%。

由于政治、文化及经济原因,特别是女同性恋者在性别和性取向两方面处于弱势,也由于女同性恋者间的性行为体液交换量很少,不属于艾滋病高危行为人群,因此女同性恋者是受到社会和学界较少关注的弱势人群之一。近几年,随着城市酒吧文化和互联网的发展,该人群开始受到关注。根据国际社会20世纪的大量调查和我国学者1990年公布的多项调查估测,我国女同性恋者人口约900万。

总之,以上研究提示,我国有约3000万人男女同性恋者,其中半数以上是男性。

通过当前对同性恋的研究—美国精神病学和心理学协会不再将同性恋归为精神疾病—大多数的治疗和咨询工作者已经改变了治疗的焦点。在国内,由许多精神科专家共同参与制定的《中国精神障碍分类与诊断标准<第三版>(CCMD-3)》在2001年4月正式出版。在编制该标准过程中,样本的收集均来自同性恋社区,而非精神科门诊,这项基础性的工作对同性恋的重新认识起到重要作用。结论认为同性恋作为一种性爱活动不一定是异常的。如果同性恋中有的人在个人性指向或性发育过程中,曾感到焦虑、抑郁,甚至痛苦,或者感到犹豫不决,甚至有的希望改变为异性恋。在这种情况下,提供精神科医学和心理学的服务是必要的。在《国际疾病分类(第10版)(ICD-10)》中也纳入了同性恋,其表述为:同性恋者中也确实存在非和谐性同性恋需要医学帮助的人。在制定新标准时,我们至少跟踪51位同性恋者一年以上的时间,其中有6例伴有心理问题的。值得注意的一个有趣现象是:在美国曾经有同性恋者,经过精神科或临床心理学家的医学帮助后,改变为异性恋者的,他们回顾过去的同性恋状态,认为自己过去是不正常的,所以,他们强烈抗议是为了从精神病诊断标准里取消同性恋名称。


\section{第三节 同性恋现象形成原因}

学界对同性恋现象成因的研究已有100多年历史。20世纪前期,由于西方政治、宗教文化的影响,医学界出于改变同性恋者性取向的目的,主要通过一般性调查和治疗等方法探索同性恋的成因。20世纪中晚期,特别是工业化国家同性恋社团公开化及艾滋病在男同性恋者间开始大流行以来,国际间开始更多地对同性恋的成因进行研究和探讨。随着生物学、心理学和社会科学等研究的发展,对同性恋性取向发生原因的解释日趋多样。生物学家认为同性恋是部分人生来具有的本性,他们的认识常被称作本质主义认识或本质论;心理学家强调童年和少年时期所受影响对性心理的作用;社会科学家一般认定社会的巨大作用在塑造性取向时起决定性作用,这类认识常被笼统地概括为社会建构主义认识或社会建构论。虽然对同性恋现象成因至今没有最终“定论”,但目前学界通常像解释许多人类现象一样,把同性恋现象看成是先天基础上多种因素综合作用的结果。

19世纪的学者多强调遗传。20世纪初克拉夫特·艾宾(1931)认为,在脑子里有男性和女性两个性中枢,何者占有优势便决定何种性取向占优势。胚胎学对性的生物学学说发生了重大影响。在胚胎发育的不同阶段,从男女性器官都有,逐渐演变成一套性器官逐渐占优势,而另一套只剩下些残余。有关遗传因素决定性取向的研究,主要集中在对孪生子、基因表达和偏手性等的研究方面,研究对象主要是男同性恋者。

同卵双生、异卵双生和领养的兄弟,发现当兄弟中的一人是同性恋时,另一人也为同性恋的比例在同卵双生子为52\%,异卵双生子为22\%,领养的兄弟为11\%。一个与此类似的在女性中的研究发现了相同的模式。这些结果证明同性恋的产生的原因可能有一部分是环境因素造成的,但同卵双生子组与另外两组之间表现出的巨大差异,有力地证明了基因组成可以导致性倾向的产生。

对114个有男同性恋者的家庭进行家系遗传学研究发现,男同性恋者具有家族性聚集的特点。其亲族同性恋性取向在母系的上辈男性亲属和外甥中的发生率明显高于无男同性恋的家系,而父系亲属则无此现象。母系和性取向的巨大相关性揭示出母亲为“同性恋性取向基因”的携带者,而性取向在人类存在性联遗传的可能性,但尚不能确认男女同性恋者是否有共同的家庭性聚集。

已确认偏手性由遗传所决定,它与某些特殊心理状况和免疫异常性疾病有关。研究发现,男同性恋者中左利手的发生率明显高于异性爱者。但左利手的男同性恋者中免疫异常性疾病,如哮喘、枯草热的发病率、在心理测试中出现异常的比例,均不高于异性爱者。

研究者指出,虽然不能除外家庭环境、所受教育和社会影响等后天因素对同性恋性取向形成的作用,但就遗传因素与环境因素对他们性取向形成作用的进一步比较分析显示,性取向在更大程度上是由遗传因素所确定的。今后研究重点是发现和识别这一个或一组基因的存在。

出生前激素水平研究:胎儿期母亲体内不同性激素水平的作用,可能对胎儿未来的性取向产生影响。在异孪双生子中,较高的同性恋比率也可能是他们享有相同胎儿环境的结果。内分泌学家道尔纳通过动物实验研究证明,性激素对垂体和下丘脑的性别差异分化具有重要作用,这种作用有可能影响个体未来的性取向。据推测,如果人类雄性(男性)胎儿得不到高浓度的睾丸激素的影响,而主要受到来自母体卵巢中的雌激素的影响,胎儿大脑会女性化,从而使胎儿成为同性恋类型;而在高浓度睾丸激素影响下的雄性(男性)胎儿则发育成异性爱类型;雌性(女性)胎儿如果受到高浓度睾丸激素的影响,胎儿也会发育成同性恋类型。维兰尼等分别测量了同性恋者和异性爱者的激素水平,并将两者加以比较后发现,男同性恋者尿中的睾酮较异性爱对照组为少,而女同性恋者尿中的睾酮较异性爱对照组为多。这似乎可以解释为性取向与激素水平有关。但目前仍难以确定,究竟是激素水平的变化导致了同性恋者发生,还是同性恋者的心理及行为引起了继发的激素水平变化。

成年后激素水平研究:一些研究者考虑到激素可能会导致同性恋。然而尚没有好的对照实验研究能够发现,在血液中异性恋和同性恋的成年男性的激素水平有什么不同。并且即使发现有固有的差异,也很难判断它们是性倾向的原因还是结果。由于受到了社会的压力,很多同性恋者经历的压力和焦虑本身就可能会影响到激素水平。许多研究者相信,成年人的激素水平与性倾向的成因无关,因为性倾向是在成年之前就已经确立了。

新的研究表明,性取向有少部分是生物性的。研究者利维通过研究部分已故同性恋和异性恋死者的下丘脑时发现了这一点。作为一个男同性科学家,利维想要做“一些与男同性恋身份认同有关的事情”。为了避免结果出现偏差,该项研究采用“盲实验”,即并不知道哪些捐献者是男同性恋。9个月来,他通过显微镜将对他认为可能很重要的细胞团进行比对。研究发现,异性恋男子的这个细胞团比女子和同性恋男子的更大。

性取向不同,大脑也不同,这恰恰迎合了“任何心理现象同时都有相应的生理基础”的观点。

利维没有把这个神经中枢看作是性取向中枢,相反,把它看作是参与性行为的神经通络的一个重要部分。他认为,很可能是性行为模式影响大脑的解剖结构。所谓的性研究者布里德洛夫在研究报告中指出,在鱼、鸟、鼠和人类中,大脑结构随着经历不同而不同—包括性爱经历。但是,利维认为更可能是大脑解剖结构影响性取向。他的预感似乎可以从这个发现中得到证实,即6\%~10\%的表现为同性吸引的雄性绵羊和90\%以上异性吸引的雄性绵羊之间具有相似的小丘脑差异。

艾伦和戈尔斯基也得出结论,认为大脑解剖结构影响性取向,因为他们发现同性恋男子的胼胝体的一部分比异性恋男子的大1/3。格拉多指出:“从现有的神经解剖图上可以发现,在某些脑区,同性恋男子比异性恋男子可能具有女性类型的神经解剖结构。”

已有多项研究发现,男同性恋者在多子女家庭中排行通常更小,兄长数目与男同性恋者的发生呈正相关关系,即兄长(哥哥)越多,男孩越倾向成为同性恋者。男性性取向则与姐妹、弟弟的数目无关,与出生时父母的年龄也无关联。

解释兄弟排序与男性性取向相关的是一种免疫假说:母亲每生育一个男性婴儿,体内的y连锁次要组织相容性抗原H-Y抗原的免疫性就增强一次,产生的抗H-Y抗原抗体就增多一次,即母体产生的抗H-Y抗原的抗体随所生男孩数目的增加而逐渐增多,此抗体通过胎盘进入胎儿体内并对胎儿未来的性取向产生影响。在女性同性恋者,未发现以上现象和差异,这也与免疫假说相一致。

部分学者认为,人类的同性性行为不过是动物间同性性行为本能的反映之一,是人类与其他动物的相同之处。在大多数智商接近人类的灵长类动物中,同性性活动作为异性间性行为的对等物得到了充分的表现。这些动物的同性性行为提示,人脑中可能潜存着性的二元化机制。动物(不包括人类)对同性性行为的绝对偏好仅在某些环境里才能见到,而动物中绝对的异性性活动也不常见。绝对的异性爱和绝对的同性恋是只有在人类社会才存在的性现象。

由于同性恋的起因尚不能完全用生物学因素最终解释,特别是某些个体同性恋倾向似乎与童年期遭受过某种印象强烈的冲击、受到过某种激励或压抑性外来影响等存在一定关系,一些学者提出了同性恋起因于某些特殊社会心理因素的理论,即所谓性取向由社会构建的认识。

早期的调查发现,母亲对男孩的过度关爱、父亲对男孩的过度严厉、父子分离、及父子关系恶劣等,均可使男孩的气质、行为女性化,儿童和少年的同性恋冲动反映了对抗孤独和自我怜悯的幼稚渴望。虽然这种理论已被当代学者普遍否定,但一些男同性恋者仍用此解释自己性取向的成因。

20世纪60年代开始,国际间对“性别认同障碍”儿童的较多研究发现,一些男孩自出生后至建立性别意识时,有一个阶段被家长当作女孩抚养,自己也认同为女性。这种男性成年后,成为男同性恋者及变性欲者的比例非常高。张北川等(1998)通过问卷形式进行调查,发现我国18.3\%男同性恋者在儿童期曾被当作女孩抚养。

一些心理学家强调,个体出生后18~36个月的养育因素是对性取向形成起到重大作用的时间段。已发现一些幼儿在这一阶段的性别自我认定并不等同于大众对其性别的认定。这种儿童成年后,高比例的人成为同性恋者,还有个别人成为变性欲者和有异装嗜好者。

一些学者认为,青春期及其后偶然际遇的影响,如被女性“伤害”或被男性“诱惑”等,有可能成为男同性恋形成的原因之一。这也是许多男同性恋者自我解释性取向形成的原因。

Tomeo等(2001)调查了924位同性恋者,并以异性爱者作对比,发现同性恋者儿童期遭受性骚扰的概率明显高于异性爱者,36.0\%男同性恋者有在儿童期遭受性骚扰经历,而仅7.0\%男异性爱者儿童期遭受过性骚扰;22.0\%女同性恋者儿童期受到性骚扰,而女异性爱者儿童期仅有1.0\%遭受过性骚扰。Dolezal等(2002)对100例儿时与年长者有性经历的男同性恋者进行了调查,发现其中59例经历过儿童期性虐待。

在同性恋成因的后天说即心理社会成因说中,存在着两大流派,一是精神分析学派的观点,一是行为学派的观点。

从弗洛伊德开始,精神分析学派在同性恋成因问题上做过大量研究,其核心论点是所谓“异性恋恐怖”说,这种观点认为,儿时的遭遇在潜意识中种下了异性恐怖的种子,因此成年以后会害怕与异性作性的接触。弗洛伊德指出,同性恋是性心理发展中某个阶段的抑制或停顿。

弗罗伊德坚持认为儿时的经历和与父母的关系均是同性恋产生的原因,即与父亲或母亲的关系是同性恋产生的一个重要因素。他认为在“正常”的发育过程中,我们都经历了一个“同性恋的”阶段,如果男孩子与他们的父亲关系恶劣而与母亲非常亲近,他们会固定在这个同性恋阶段;而如果一个女人羡慕阴茎,那么相同的事情可能会发生在她身上。尽管这些模式在一些实例中存在,许多同性恋者不适合这个模式确实是一个事实—就是说,他们的母亲的作用不是那么突出,他们的父亲也不是在感情上远离他们。而且同时,大量的异性恋者生长也确实生长在这种广泛存在的家庭模式中。

同性恋后天形成说中的另一大流派是行为学派。按照这一学派的学习理论,同性恋行为是受环境的影响而习得的。如果一个人在与异性交往中受挫,有过不愉快的经验,异性恋感情得不到正常的发展,而同时又受到同性的诱导,就会产生同性恋倾向。行为学派特别注重的是伙伴群关系,偶然的机遇,以及特殊的性经历,如童年时受到同性恋者的诱惑等事实。

对许多在儿童期并没有遭受过所谓不利于异性爱性取向形成因素影响的青少年来说,好奇心理也被推测为同性恋性取向形成的重大因素。据行为学派认为作为“被引诱者”,接受对方的同性性行为并得到积极强化,有可能导致同性恋性取向形成。对同性恋性取向形成原因的其他解释还有母亲精神创伤因素等。

心理分析学派和行为学派这两大流派各自形成了对同性恋倾向的心理疗法和行为疗法。前者运用规范的心理分析方法治疗“患者”;后者则采用掏同性恋倾向的电击疗法或呕吐疗法强行校正性取向。据说心理疗法的治疗成功率大约在三分之一到四分之一;经过行为治疗,有百分之五十的“患者”可以被治愈。

在造成同性恋倾向的先天生理因素的先天社会心理诸因素当中,我们的调查发现,早年的性经验,尤其是首次性经验,有着非同一般的重要意义。如果按照个人性格发展史的时间顺序看,童年环境的影响当然在前,而青春期经验在后,但是二者相比,后者在同性恋形成过程中所起的作用绝不比前者小,在许多个案中,甚至比童年家庭环境的影响更强烈。


\section{第四节 同性恋的社会交往和性交特点}

一般地说,我们很难有把握地根据一个人的外观举止来判断他是否是同性恋者。男子的所谓“女性化”,和女子所谓的“男性化”,都有许多不同的程度和表现形式,因而在概念上并不明确。这两个术语在文献中有趋于滥用的情况。

性身份和性取向二者之间有重叠,但并非完全相同。据估计,无论从他们的自我认知、兴趣爱好还是外观举止上看,有一半的男子同性恋者和3/4的男子异性恋者有典型的男子汉气,另一半的男子同性恋者和1/4的男子异性恋者却有程度不等和形式不一的偏离。估计女子同性恋者和异性恋者的典型女子气的百分比较男子相应情况要低一些。

然而,性身份主要指一个人内心的觉察和认知而言。同性恋者有时用显眼的外在标志如服饰和特殊语言、表情和手势等表现自己,那是同性恋者交往的一种信号,跟性身份并不是一回事,不能混为一谈。

非常随便的和漫不经心的性行为,即与碰巧遇上的多个陌生人发生性行为,不论同性还是异性,在大多数文化里都被认为是不健康的。这意味着性的非人化、纯冲动的性行为,从某种意义上说只不过是自恋的一种表现,也是缺乏与人发展亲密关系的共情能力的表现。

调查发现,随便的性行为在同性恋中,尤其是男子同性恋中,比异性恋要常见的多。没有结婚、怀孕和养育子女等的制约,对同性性行为缺乏社会同情和赞赏,大概是同性性行为随便的重要原因。

贝尔和温伯格对同性恋者与异性恋者进行了详细的比较性研究,发现男子同性恋者在心理内在调节、人际关系和社会适应等方面都比异性恋者要差得多。女子同性恋者在上述方面也略差于异性恋者,但没有男子那样显著。对两种群体的个人进行比较时,同性恋者的个人差异比异性恋者还要大。他们设计了一种类型划分,可以把70\%的同性恋者归于某一类型。总共研究了686位男子同性恋者,其中70\%的人可以分为下述5种类型之一:

(1)封闭配对型:该类人群占9.8\%。他们愿意过一对一的配偶生活,愿意双方又经常的密切的接触,不允许对方与第三者有密切来往,嫉妒心很强烈。

(2)开放配对型:该类人群占17.5\%。他们认为恒定的性伴侣是可取合有好处的,但同意对方另有所欢。

(3)实用型:该类人群占14.9\%。他们性生活水平很高,关系不稳定,一个人经常与不止一个甚至很多人有性行为,对自己的性行为没有羞耻悔疚心理,几乎不觉得有什么性问题。

(4)功能障碍型:该类人群占12.5\%。他们经常发生各种性问题,对自己的同性性行为常有悔疚心理,情绪问题多。

(5)性欲低下型:该类人群占16\%。这种人对性的兴趣不强烈,性生活也少。性伴侣少,似乎很难找到,是他们不主动还是过于挑剔,很难说。这种人也有不少性问题。

女子同性恋者也有大约70\%可以归入上述五种类型中的任何一型,但构成的百分率与男子明显不同。封闭配对型的占大约30\%,明显高于男子。实用型的,功能障碍型的和性欲低下型的都明显地比男子少。女子同性恋者开放配对型的,功能障碍性的和性欲低下型的精神卫生问题都多,但另两类的人与异性恋者对照组在精神卫生上相近似。

男女同性恋者都有30\%的人无法分类分型,他们个人差异太多,情况各不相同,报告中也没有提到他们与异性恋者对照比较的数据。

我国同性恋者的社会交往和性交往,在历史渊源和社会文化基础上,与某些国家,特别是文化差异很大的欧美、中东国家(这些国家的文化深受宗教影响)有所不同。相比而言,女同性恋社区的发展和进程均迟于男同性恋社区。

1989年,丹麦成为第一个立法认可同性结合并允许同性伴侣进行登记的国家,但当时该国的立法不允许同性伴侣领养子女和举行宗教婚姻仪式。2001年,荷兰成为第一个通过法律认可同性婚姻的国家,规定同性婚姻家庭享有传统婚姻家庭所享有的一切待遇。至2006年,已有部分西北欧国家、加拿大、南非立法承认同性婚姻。在我国,根据民政部对2003年开始实施的新婚姻法的解释,禁止同性结婚。

由于通常没有法律、婚姻的支持与制约,以及社会存在对同性恋者的歧视等,包括我国在内的许多国家的同性恋者,他们的同性伴侣关系一般处于地下隐蔽状态,通常短暂而不稳定,并导致绝大多数同性恋者因找不到合适的同性伴侣,而频繁更换性伴侣或同时拥有多个性伴侣。

男同性恋社区通常被该人群称“同志”社区,或“gay”社区,目前泛指有亚文化特点的男同性恋者的多种社会交往网络。社区的出现,为大量同性恋者提供了心理和现实的支持,使得部分同性恋者能够进行较好的交流。

我国男同性恋者的交往方式,随着男同性恋社区不同活动场所的出现有所不同。这种社区活动,其萌芽出现于20世纪70年代后期的少数大城市(如北京)。最初是以隐秘的户外场所,如公厕、街头、不收费公园等为约定俗成的活动场所。不同地方的“圈内人”对此类场所有不同称谓,如“渔场”、“公司”或“点”。男同性恋者通常在晚间到这类场所聚集,一般通过极简单的交谈,甚至不交谈便发生性关系,随后分手。性行为方式以手交、口交和肛交为主。这种交往最大限度上保护自己不被辨识进而保护隐私和安全,同时得到性生理的满足。

至20世纪90年代,随着我国社会开放、经济发展和多元化消费市场的出现,许多大中城市出现了同性恋酒吧、茶吧、演艺吧、迪厅及浴池等消费场所。大城市及部分中等城市公开存在的同性恋酒吧,是目前男同性恋者最活跃的活动场所,常有数以百计的同性恋者聚集,观看有浓郁亚文化特点的娱乐表演,交流社区信息,谈论与社区有关的现象,结交朋友,包括寻找性伴侣。很多同性恋吧有向男性提供性服务的男性性工作者存在。而在这类浴池,性活动混杂,很多人在浴池进行口交、肛交。虽然警方发现此类浴池后常常进行打击,但由于市场经济的导向,一家浴池关闭后,通常在同一城市很快又有新的这类浴池出现。

20世纪90年代后期,在一些大中城市,有相似的经济文化和社会地位背景的男同性恋者建立了自己的“小圈子”。在圈内举行小型聚会,一同打牌、外出游乐或开展体育活动等,同时交流有关社区的信息。这样“小圈子”中的男同性恋者和已经形成相对固定“一对一”情侣关系并同居的男同性恋者,很少到公开的同性恋场所活动。

目前,一些城市的男同性恋者经常利用节假日,组织纯粹文化娱乐活动和一般社会性交往,发展富有情感的人际关系。

互联网的地域性和全球性,使同性恋者中的网民更易于与同性恋者,尤其是其他无法直接结识的异地同性恋者建立联系,乃至性交往关系。同时,互联网及聊天室等虚拟性的安全保障和所提供的信息,对同性恋社区文化建设,起到了重大积极作用,有效地促进了同性恋者良好的自我认同和群体内的交流,而且为同性恋者的健康教育提供了较理想的平台。

Elford等(2000)对英国伦敦的743位男同性恋者进行了调查,发现80.9\%上网,上网者中34.4\%通过互联网寻找性伴侣。Benotsch等(2002)对亚特兰大的609位男同性恋者的调查发现,75.0\%访问过同性恋网站,34.0\%通过互联网寻找性伴侣,通过互联网寻找性伴侣的男同性恋者中高比例的人使用甲基苯丙胺(一种中枢兴奋药),高比例的人有无保护性被动肛交和主动肛交。Ross等(2003)调查了716位未曾上网的男-男性接触者和678位上网的男-男性接触者,发现前者1/4和后者1/3希望通过网络咨询有关艾滋病信息。Lau等(2003)对香港地区283位在过去6个月内曾有过同性性行为的男-男性接触者的调查,发现17.7\%在过去6个月内曾通过互联网寻找性伴侣。张北川等(2006)对2250位男-男性接触者调查发现,72.3\%登录过同性恋网站及有关聊天室,其中52.0\%近一年内通过以上途径寻找过性伴侣。

由于东西方社会文化的不同,不同国家男同性恋者在与女性性关系方面存在很大差异。发达国家仅有少数男同性恋者与女性间有性行为,与女性结婚者更少。瑞典1993年的一项大样本调查发现,约有10.0\%男同性恋者与女性发生过性行为,仅3.0\%与女性配偶一同生活。Kumar和Ross(1991)对印度和澳大利亚男同性恋者的跨文化比较研究发现,印度男同性恋者结婚的比例明显为高。

我国传统文化崇尚生育,经济落后使得大众普遍把养育子女作为晚年生活的保障。男同性恋者普遍存在与女性的性关系,多数同性恋者随着年龄的增长,迫于父母和社会要求他们结婚的压力,以及自己不愿意保持单身成为社会舆论非议的对象,绝大多数最终选择结婚。刘达临等(2005)的报告指出,我国男同性恋者中90.0\%以上最终会结婚。张北川等(2006)调查发现,平均约30岁的男同性恋者,42.9\%已婚(含再婚),57.1\%未婚。未婚者中19.9\%目前有固定女性性伴侣,26.4\%未来准备结婚,39.4\%还没有确定是否结婚,只有30.5\%自己确定未来独身。对其中已婚者的调查发现,他们结婚的最重要原因依次是,53.3\%满足父母愿望,13.5\%证明自己是“正常人”,6.9\%孤独需要有家庭的温暖作为支持,6.1\%女性喜欢并追求自己,5.4\%希望老年时有依靠,3.4\%希望通过婚姻改变自己等。

在社会严重歧视同性恋的地区,有大量男同性恋者从不或基本不去同性恋活动场所。其中社会地位高、有良好职业的同性恋者,为保护自己的个人隐私和免受伤害,更注意掩饰其性取向,并力求表面上与“主流”生活方式保持一致,因此往往生活在孤独之中,常通过性幻想得到低质量的性心理平衡。

我国女同性恋者的社会交往,最初常与男同性恋者的社区活动结合在一起,出现在20世纪90年代北京的酒吧,以娱乐性聚会为主要表现形式。21世纪初,我国少数大城市出现了由女同性恋者经营的主要向该人群提供服务的酒吧。个别大城市目前形成女同性恋社区网络,所进行的活动主要是娱乐性交往,其次是组织的讨论该人群所面临的生活问题,通过交流经验,彼此提供心理支持。

目前,我国已有40多个由女同性恋者主办的和面向当地,或覆盖全国的网站。网站通常设有论坛、服饰、聊天室和交友服务等。较多女同性恋者通过网络进行交流和交友,寻求心理安慰,以及提供相互帮助。

由于受我国传统文化影响,女性与亲长的联系更为密切。社会要求女性具有温柔、贤惠的品质,使女同性恋者倾向于比男同性恋者更多地进入传统婚姻。仅极少数经济上独立且收入较高、生活在大城市及父母对女儿的婚姻采取不干涉态度的女同性恋者,与同性伴侣一起过着相对理想的生活。此外,在大城市,有少数女同性恋者与社会地位相近和情趣相投的男同性恋者,为了减缓家庭要求自己结婚的压力或满足亲长的愿望,而彼此结合成无性婚姻。这种婚姻常被称为“形式婚姻”或“互助婚姻”。


\section{第五节 同性恋的现实表现}

成人同性恋有多种性行为方式,如:接吻,拥抱,抚摸,相互用手玩弄生殖器,口腔刺激生殖器,肛门性交(俗称鸡奸)等。据调查,美国女子同性恋者很少使用人造阴茎或阴茎代用物品。

与异性恋者一样,同性恋者也各有偏好,个人差异多而且大。有人偏好主动,有人偏好被动;插入者或被插入者也是各有所好。

很多人喜欢变换行为方式甚至性对象。大多数同性恋者常改变角色。也有些同性恋者没有什么特殊偏好,似乎怎么干都行。

调查观察未发现同性恋者有什么特征性的自我显示方式。个别人或小群体之间的联络暗号,像黑社会组织成员那样,不应该说成是同性恋的特征性自我显示。实际上,从外观举止上我们一般很难看出谁是谁不是同性恋者。据调查,大约20\%的男子同性恋者是过分女性化的,有类似舞台表演式的夸张的女性外貌和举止。同样,大约20\%的女子同性恋者也是过分男子化的,她们头发剪得很短,或留着典型的男子发型,走路办事雄赳赳气昂昂的,举止粗鲁,说话的语调和内容都是典型男子式的。

同性恋里有各种职业的人都有,即使是那些传统的由异性恋者从事的职业里也有同性恋者。人际关系模式也是各种各样的。笔者在1999年为《中国精神疾病分类方案与诊断标准》修订过程中,对51名同性恋者进行了调查,这51名研究对象均来自北京市居民及外地来京务工人员。行业涉及:编辑、医生、艺术工作者、学生、工人、个体户等,文化程度覆盖从博士到初中及下文化程度不等。单从这些方面来看,我们很难区分同性恋与异性恋之间的差异。

马斯特斯和约翰逊发现,讲求实际和有献身精神的同性恋配偶比异性恋配偶更加平等待人,他们更自由地沟通双方的性感觉和性情感,花更多的时间去体验双方肉体的一般接触(性器官以外的)感受和所谓地“性交前爱抚”,对对方的需要和反应更加敏感。

相当一部分同性恋者有异性性交经验,百分率随研究报告不同而异;男同性恋者1/3~2/3有异性性交经验,女子同性恋者更高,有60\%~85\%有过异性性交经验。很多同性恋者是已婚,尽管不少已分局或离婚,50\%~75\%的已婚同性恋者有孩子。

据金西报告,女孩13岁以下,男孩15岁以下,同性刺激引起的性兴奋多于异性的刺激。报告在15岁前有过同性激起的性兴奋者妇女占33\%,男人占50\%。

因此,应该用事实去帮助父母和孩子解除对同性恋的焦虑。其实,上述那些有同性性行为的未成年人只有少数长大成人后成为同性恋者。

假同性恋是Lovesey在1955年首先加以描述的。对一个个体就某方面特征作出界定,内在的心理因素比外显的行为因素更为重要,因而在对假性同性恋的界定上,笔者便比较多的从性心理方面进行定义。在笔者看来,所谓假性同性恋是指性心理指向异性,但性行为随外在境遇的变化而经历指向异性、指向同性、自发矫正为异性的性取向的变动。具有假性同性恋这样一种性取向的同性恋者就是假性同性恋者。假性同性恋的具体发生机制大体是这样的:单性环境引起了异性情感的缺失或冷漠,导致了以同性为对象的性爱行为;在单性环境发生变化后,个体的性行为在性心理的驱动下自发地矫正,重新指向了异性。

一个同性恋者是怎样由非同性恋者变成同性恋者,又怎样再变成非同性恋者的,这其中的影响因素和他们的生活状况都值得我们好好研究。

Lovesey从研究表面上的同性恋焦虑出发,发现这类焦虑可以分解为三种不同的动机成分:性欲成分、依赖成分和权力成分。假同性恋主要指男人,同性并不唤起性欲,而是唤起统治支配欲和依赖服从欲,伴之以男子气或女子气。

他们通常以依赖或权力作为被满足的需要而并不以追求性为最终目标,但可以使用性器官作为手段以达到非性的目的。只有对同性有性唤起或性欲才会引起真正的同性恋焦虑。而依赖或权力动机引起的焦虑有可能被当事人误以为是由同性恋引起,尤其是当焦虑与性行为偶然地发生了联系的时候,这实际上是假同性恋焦虑。

有些男人自认性格软弱,服从性强,没有本事,可能误认自己是同性恋者。他们意识中害怕成为同性恋者,可有持续的焦虑不安,可能发作同性恋惊恐,有时产生暴力或自杀冲动,需精神科急诊处理。假同性恋可以表现为神经症性心理冲突的形式,临床表现主要是焦虑状态,也可以由惊恐发作或恐怖症。这种假同性恋者的人格主要以依赖、自卑等特征为主。

尽管笔者在此提出了假性同性恋者和非假性同性恋者的概念,但两者的区分并非是非常绝对的,否则就有简单化倾向的危险。非假性同性恋者的成因中虽然以性心理的同性指向为主因,但并不排除外界环境的诱导和激发。同样,对于假性同性恋者而言,他们之所以发生同性性行为其主要原因是由于外界环境导致的,但同样存在着一点潜在的性心理偏离,不然就难以解释为什么在同样的境遇下,只有少部分人成为同性恋。当然,这里少部分人是指假性同性恋者与全部个体成员的比重,如果将假性同性恋者与全部同性恋者相比较,其比例是巨大的。

另外,在金西调查中,尽管表明了“有过同性恋行为者”和“终身同性恋者”之间存在着巨大的差异,但我们却无从探寻导致这种差异的原因,本文尝试着运用“假性同性恋”这个新概念做一点阐释。

在单性寄宿学校,监狱,军营,寺院以及远洋轮船等处境下,由于没有异性,原来的异性恋者可能出现同性性行为。有人认为这是一种性适应。当然,如果双方同意,对双方大概都至少是无害的。但是,有些不安全的人格,或者性身份不大确定的人,或者有未充分意识到道德禁忌的人,同性性行为可以诱发强烈的悔疚和心理冲突。

正像男子对女子实行强奸一样,性行为可以成为攻击的外衣,或者两者混合,男子使用阴茎作为武器对另一男子发动攻击,使对方屈服,这种情况并不限于暴力成风的监狱。通常的表现形式是,将阴茎插入另一男子的肛门,或者强求另一男子用口含蹂躏攻击者的阴茎。性释放并不是情欲上的主要目的,而是要迫使对方屈从。这种情况在妇女监狱里较少见。蹂躏攻击者,尤其是男子,并不认为自己是同性恋,但这种人可能存在严重的人格问题,很可能是人格障碍者。受害者是被迫的另一方,他们感到屈辱、恐惧、心理冲突,甚至抑郁自罪。有些被迫进行同性恋者,似乎是一种不得已的适应,但照例有各式各样的情绪障碍。

前面已提到过,同性恋者大多有异性性行为,这种情况被称为双性恋。仔细了解就会清楚,他们的性幻想多涉及同性,他们实际上更容易被同性激起性欲。

真正的两性等价取向指,对同性和异性同样地喜欢,同样地容易被激发性感和性欲。但这种情况较为罕见。

所谓意识形态的同性恋,是指那些好斗的女权主义者,她们对女人需要男人有强烈的愤怒、反感和蔑视。她们常极力否认男女的差别和两性的互补性。这种特殊的观点或信念可以视为性心理冲突的理智化或合理化。这种女人矢口否认自己有任何不正常,相反,她们声称在为某种理想而奋斗。实际上,她们私下有时要求同性性伙伴担任她们所看不起的“妻子”的角色。有报道说,有极少数这种女人求助于精神病学家或心理治疗家,因为她们感到在处理人际关系上遇到了无法克服的困难。男人和女人都不喜欢她们,甚至疏远、反对她们,以致她们对自己的性身份也产生了根本的惶惑。


\section{第六节 鉴别诊断}

在本节内容中,分别截取《美国精神障碍诊断与统计手册(第四版)》(DSM-Ⅳ)、《中国精神障碍分类与诊断标准(第三版)》(CCMD-3),以及《国际疾病分类(第十版)》(ICD-10)中有关性别障碍及相关内容的阐述,目的是更清晰的认识到它们彼此之间的区别以及鉴别要点。

美国精神病学会(APA)从1952年起制订《诊断与统计手册:精神障碍》(Diagnostic and StatisticalManual of Mental Disorders),后来称这份手册为DSM-Ⅰ。现在《诊断与统计手册:精神障碍》已经发展到DSM-Ⅳ,DSM-Ⅳ(美国精神疾病诊断标准)对“性身份识别障碍”的定义设定了四个标准:

1.一种强烈而持久的交换性别的身份认识(不仅仅是想以作为另一性别而获得社会文化上的好处的这种欲望)患者“强烈而持久”地渴望交换自己的性别。这种渴望源自于一种根深蒂固的对性别的“自我感觉”,即患者自身的生理性别与这种“自我感觉”完全相反。虽然有些学术调查发现部分女生认为自己如果是男生,将获得比作为女生更多的社会荣誉、尊重、成就等。但是,我们所提到的这种强烈而持久的交换性别的身份认识不是指为了作为另一性别而获得上面所提到的社会文化上的好处,而是内心中对自己生理性别激烈的冲突。DSM-Ⅳ同样对这种渴望的外在表现进行了细分,儿童和青少年或成年人对于性身份识别障碍的外在表现是不尽相同的。一个儿童,如果表现为下列4项以上,则说明这个儿童具有DSM-Ⅳ所规定的“性身份识别障碍”:(1)反复申述自己想成为另一性别,或坚持认为自己是另一性别;(2)男孩喜欢换穿女装或耀眼的女性盛装,女孩则坚持一直穿典型男性的服装;(3)在假扮游戏中强烈而坚持地偏爱另一性别的角色,或坚持幻想成为另一性别;(4)强烈地希望参加典型的另一性别的游戏及娱乐;(5)强烈偏爱另一性别的游戏伙伴。青少年或成年人的外在表现没有像儿童那样直观和外显,他们则表现为申述自己愿成为另一性别的愿望,往往发誓是另一性别,希望像另一性别那样地生活或要求他人如此对待,或深信自己具有另一性别的典型感受和反应。

2.为患者自己的性别感到持久的不舒服,或者认为自己目前的性别角色很不合适。身患“性身份识别障碍”的儿童、青少年或成年人由于自身的生理性别与自我对性别的认同存在深深的矛盾,他们会感到自己陷在一个错误性别的躯体里不能自拔,从而体验到深深的不适和痛苦。这也是“性身份识别障碍”与“异装癖”、“同性恋”的区别之一。“异装癖”的患者常常是男性只有在穿着异性的衣服物品的时候才能引起性唤起。“异装癖”的基本目的是性唤起和性满足,对自己的性别身份没有任何的不适和痛苦。

“性身份识别障碍”患者的目的不是与性有关,而是希望自己以另外一种性别的身份去生活、去爱。同性恋中,也有男性表现出女性化或者女性表现出男性化的一面,但是“同性恋”的个体不会像“性身份识别障碍”那样“感到自己陷在一个错误性别的躯体里不能自拔,从而体验到深深的不适和痛苦”,他们不会感觉自己是一个陷在女人身体的男人或者陷在男人身体的女人,也不会想改变性别,没有不适和痛苦的体验。

3.此障碍并不与躯体上的两性人同时存在。“性身份识别障碍”患者不与躯体上的“两性人”同时存在。现实存在的“两性人”生来就有着模棱两可的生殖器,激素的产生上或者是生理上的变异。“两性人”生来往往就根据他们自己不同的混和特征,被“指定”为一定的性别,日后再通过手术或者激素治疗的方法来改变他们的性别解剖特征。现在越来越多的医生建议一旦观察到出生后的个体具有双性化的解剖特征,就应该作为急诊处理,立刻实施手术。“性身份识别障碍”没有“两性人”所表现出来的这种激素上或者是生理上的变异,没有模棱两可的生殖器。所以说“性身份识别障碍”患者不与躯体上的“两性人”同时存在。

4.此障碍产生了临床上明显的痛苦烦恼,或在社交、职业或其他重要功能方面的功能缺损。“性身份识别障碍”患者由于产生了临床上明显的不适和痛苦。他们往往在社交、职业或其他重要功能方面的功能有所缺损。

《中国精神障碍分类与诊断标准(第3版)》(CCMD-3)是属于我们自己的精神障碍分类与诊断标准,是由卫生部科学研究基金资助,通过41家精神卫生机构负责对24种精神障碍的分类与诊断标准完成了前瞻性随访测试之后编写而成。

由于具有浓厚的本土气息,所以《中国精神障碍分类与诊断标准(第3版)》(CCMD-3)是更加适合中国人的精神障碍分类与诊断标准。CCMD-3共分9大类,其中“性心理障碍(性变态)”属于第6大类“人格障碍、习惯与冲动控制障碍、性心理”。

CCMD-3把“性心理障碍(性变态)”分为3大类:性身份障碍、性偏好障碍和性指向障碍。其中,有异常性行为的性心理障碍特征是有变换自身性别的强烈欲望(性身份障碍);采用与常人不同的异常性行为满足性欲(性偏好障碍);不引起正常人性兴奋的人或物,对这些人有强烈的性兴奋作用(性指向障碍)。除此之外,与之无关的精神活动均无明显障碍,不包括单纯性欲减退、性欲亢进及性生理功能障碍。

对于“性身份障碍”的诊断标准,CCMD-3是如此阐述的:

1.对于男性而言,持久而强烈地为自己是男性而痛苦,渴望自己是女性(并非因看到任何文化或社会方面的好处,而希望成为女性)或坚持自己是女性,并至少有下列1项:

(1)专注于女性常规活动,表现为偏爱女性着装或强烈渴望参加女性的游戏或娱乐活动,拒绝参加男性的常规活动。

(2)固执地否定男性解剖结构,至少可由下列1项证实:断言将长成女人(不仅是角色方面);明确表示阴茎或睾丸令人厌恶;认为阴茎或睾丸即将消失,或最好没有。

(3)上述障碍至少已持续6个月。

2.对于女性而言,持久和强烈地因自己是女性而感到痛苦,渴望自己是男性(并非因看到任何文化或社会方面的好处,而希望成为男性)或坚持自己是男性,并至少有下列1项:

(1)固执地表明厌恶女装,并坚持穿男装。

(2)固执地否定女性解剖结构,至少可由下列1项证实:明确表示已经有或将长出阴茎;不愿取蹲位排尿;明确表示不愿意乳房发育或月经来潮。

(3)上述障碍至少已持续6个月。从上面的诊断标准可以看出,CCMD-3对“性身份障碍”的诊断标准从两个方面着手,一是行为上的表现,二是时间的延续性。男性或者女性在行为上的表现都具有一定的规律:那就是在着装、参加活动趋向于异性,拒绝参加同性的活动;都固执而且强烈的否定自身的生理解剖特征,排斥作为自己生理性别的个体的行为;同时,这些表现持续6个月以上才能被定为具有“性身份障碍”。

CCMD-3把“易性症”隶属于“性身份障碍”。之前我们讨论过“性身份障碍”和“易性症”的区别和联系。那就是性别认同障碍者如因不接纳自己生理性别所引起的心理困扰演变恶化到不能忍受,而不得不求助医师改变其生理性别的地步时,就成了易性症。

CCMD-3对“易性症”的阐述是这样的:对自身性别的认定与解剖生理上的性别特征呈逆反心理,持续存在厌恶和改变本身性别的解剖生理特征以达到转换性别的强烈愿望,并要求变换为异性的解剖生理特征(如使用手术或异性激素),其性爱倾向为纯粹同性恋。已排除其他精神疾病所致的类似表现,无生殖器解剖生理畸变与内分泌异常。

从CCMD-3对“易性症”的阐述中我们可以看出来,CCMD-3将性身份障碍、性偏好障碍以及性指向障碍进行了详细的阐述和对比,在对易性症的描述中将其与性偏好障碍结合起来,认为“易性症”的“性爱倾向为纯粹同性恋”。“易性症”的这种表现“排除其他精神疾病所致”,纯粹是由性身份障碍引起。CCMD-3对“易性症”的诊断标准是这样的:期望成为异性并被别人接受,常希望通过外科手术或激素治疗而使自己的躯体尽可能与自己所偏爱的性别一致;转换性别的认同至少已持续2年;不是其他精神障碍(如精神分裂症)的症状,或与染色体异常有关的症状。

CCMD-3对“易性症”的诊断阐述了同DSM-4和ICD-10基本一致的诊断标准,那就是:“希望通过外科手术或激素治疗而使自己的躯体尽可能与自己所偏爱的性别一致”。这一基本点达到了广泛的共识。值得一提的是,CCMD-3规定了“易性症”持续的时间标准为“2年”。同时,避免了其他精神障碍(如精神分裂症)的症状导致的“易性”倾向,也与“双性化个体”以及“染色体异常个体”区别开来。

ICD-10是国际上用来临床描述与诊断精神障碍的诊断要点。ICD-10对每一种障碍的主要临床特征,以及任何重要的、但特异性较差的有关特征均进行了描述。为大多数障碍提供了“诊断要点”,指明确诊断所需症状的数量和比重。诊断要点的措词使临床工作中作出诊断决定时有一定程度的变通,尤其在临床表现不充分或资料不完整的情况下,医生不得不作出临时性诊断时。ICD-10把“性身份障碍”放在“F60~F69成人人格与行为障碍”。认为“性身份障碍”是一种持续性的,是个人特征性的生活风格的表现,也是对待自己及他人的一种模式。这些行为状况及模式有的在个体发育的早期阶段,作为体质因素和社会经历的双重结果而出现,其他一些则在生活后期获得。ICD-10认为“性身份障碍”在临床上可以分为性别改变症、双重异装症和童年性身份障碍。

1.在性别改变症中,ICD-10对其的阐述是:渴望像异性一样生活,被异性接受为其中一员,通常伴有对自己的解剖性别的苦恼感及不相称感,希望通过激素治疗和外科手术使自己的身体尽可能地与所偏爱的性别一致。ICD-10对其的规定诊断标准为:转换性别身份至少应该持续存在2年以上,才能确立诊断,且不应是其他精神障碍如精神分裂症的症状,也不伴有雌雄同体、遗传或性染色体异常等情况。

2.在双重异装症中,ICD-10对其的阐述和规定的诊断标准是:个体生活中某一时刻穿着异性服装,以暂时享受作为异性成员的体验,但并无永久改变性别的愿望,也不打算以外科手术改变性别。在穿着异性服装时并不伴有性兴奋,,这一点可与恋物性异装症相鉴别(F65.1)。双重异装症包含:青春期或成年期性身份障碍,非易性型,不含:恋物性异装症(F65.1)。“双重异装症”患者“某一时刻穿着异性服装”,“暂时享受作为异性成员的体验”,“并无永久改变性别的愿望”,“也不打算以外科手术改变性别”,最重要的是他们“在穿着异性服装时并不伴有性兴奋”,这就与恋物性异装症区别开来。

3.在童年性身份障碍中,ICD-10对其的阐述是:童年性身份障碍通常最早发生于童年早期(一般在青春期前已充分表现),其特征为对本身性别有持续的、强烈的痛苦感,同时渴望成为异性(或坚持本人就是异性)。持续地专注于异性的服装和/或活动,而对本人的性别予以否认。典型情况下,在入学前就首次出现;要想确立诊断,这一障碍必须在青春期前就已十分显著。在男女两性中,都可能会出现对本身性别的解剖结构的否认,然而上述表现较少见,也许很罕见。患有性身份障碍的儿童有一个特点,即尽管他们因与家庭、好友的期望相冲突而苦恼,也因所受到的嘲笑和/或排斥所痛苦,但他们却否认因性身份障碍而苦恼。通常情况下,童年性身份障碍相对少见,较常见的是与程式化性角色行为不一致的状况,二者不应混淆。只有正常意义上的男性或女性概念出现了全面紊乱时,才可考虑童年性身份障碍的诊断,仅有女孩子像“假小子”、男孩子“女孩子气”是不够的。如果已经进入青春期,此诊断便不能成立。

在ICD-10中,由于“童年性身份障碍”与“成人人格与行为障碍”中其他身份障碍有许多共同特征,所以把它放在“成人人格与行为障碍”中,而不再放在“童年与少年期的行为与情绪”。ICD-10对“童年性身份障碍”的规定诊断标准为:对这种障碍的知识,男孩多于女孩。典型情况是,从上学前数年,男孩就开始沉湎于那些通常属于女性的游戏和活动,而且常常偏爱穿女孩的服装或妇女的衣着。然而,这种换穿异性服装的举动并不会引起性兴奋,这就与成年人恋物性异装症区别开来,成年人恋物性异装症纯粹是为了性唤起。他们往往会讨厌男孩子,他们会有极强的欲望,想参加女孩子的游戏和消遣,洋娃娃常是他们钟爱的玩具,而女孩一般是他们偏爱的玩伴,他们也感到和女孩子在一起时更加愉快和安全。上学的头几年,这些孩子会越来越被人孤立,这些情况在童年中期达到顶峰,其他男孩会羞辱、嘲笑他们,孤立他们,更有甚者以暴力欺负他们。明显的女性化举止在青春期早期会有所减轻,但随访研究显示,患有童年性身份障碍的男孩中,有1/3~2/3在青春期及青春期后显露出同性恋倾向。然而,在成年后表现为易性症的却极少(尽管有报告说大多数成年易性症者在童年都有性身份问题)。这就与“性别改变症”区别开来。“性别改变症”患者希望通过激素治疗和外科手术使自己的身体尽可能地与所偏爱的性别一致。而“童年性身份障碍”患者在成年后表现为易性症的却极少。临床工作中所碰到的性身份障碍女孩少于男孩,但这一性别比率在总人口中是否适用尚不得而知。像男孩一样,女孩也通常较早表现出热衷于一般属于异性的一些行为。典型情况下,患有此障碍的女孩交结男伙伴,对体育运动和激烈争斗的游戏极为喜爱。她们对洋娃娃没兴趣,对在假装的游戏如“爸爸和妈妈”或“过家家”中扮演女性角色也不屑一顾。患有性身份障碍的女孩在学校中并不像男孩患者那样受到同等程度的孤立,然而她们在童年后期或青春期也会遭到嘲笑。大多数女孩在接近青春期时,会放弃对男性活动和服装过分张扬的追求,但是一些人会保留男性性别认同,并逐渐显露出同性恋的倾向。性身份障碍伴有对本身性别解剖结构持续排斥的情况是罕见的。在女孩,会表现为反复声称她们有或将要长出阴茎来,拒绝以蹲位姿势排尿,声称她们不愿乳房发育或来月经。在男孩,会表现为反复声称他们的身体将发育成为女人,阴茎和睾丸令人讨厌或将消失,最好没有阴茎或睾丸。

像异性性行为一样,同性性行为的表现形式也是多种多样的。但性取向是异性还是同性这一点并不难区分。当然,跟对任何一位可疑的精神障碍者进行诊断性交谈一样,要求医生持理解和尊重的态度,愿意花时间倾听患者的叙述,善于启发患者畅所欲言,这样才能弄清楚事实和精神现象的重要细节。为了全面的临床评定,详细的背景材料和完整的病史是不可缺少的。

当事人对自己的性取向、性身份和性行为持什么态度和观点,必须通过真诚而深入的交谈把它弄清楚。仅仅限于“自我失谐的”或“自我和谐的”这样的二分法,是过于简单化的。这类现象的澄清通常是心理治疗的一部分。

关于鉴别诊断,首先要弄清楚同性性行为是不是继发于其他某种精神障碍,如精神分裂症,情感性障碍等。性取向障碍很可能与人格障碍、神经症性障碍等同时并存,只着眼于患者诉述的性取向问题,诊断是不完全的,这对治疗会产生不利影响。


\section{第七节 性取向的改变和同性恋者的心理调节}

在20世纪中叶,对同性恋的社会态度有了一个转变。认为同性恋者是罪人的想法在一定程度上被认为他们是“病态”的想法所代替。医学和心理学工作者使用了很多激烈的治疗方法来努力治愈同性恋这种“疾病”。在19世纪,诸如切除生殖器这样的外科手术得到开展。后来到1951年,脑白质切除术(分离大脑前叶的神经纤维的脑外科手术)作为对同性恋的一种“治疗方法”开展起来。心理治疗、药物、激素、催眠、电击疗法和厌恶疗法(在同性恋刺激的同时给予使人呕吐的药物或电击)也都曾被应用过。如今,通过几十年的研究推翻了同性恋是“病态”的观念。第一个主要的研究是对比非病态的异性恋者和同性恋者的适应性并没有发现两组人群有明显的差异。除了抑郁的发生率和自杀企图的危险性升高之外,大部分男女同性恋者并没有心理障碍。

有一些关于改变性取向的报道。在性重新定向的倡导者搭理宣传的一项报告中,斯皮策指出,在专门研究性转变的“前男同性恋部门和治疗家的帮助下,200名被试者报告了由同性恋取向到异性恋取向的变化。”然而,在接受采访时,仅有54\%的女性和17\%的男性取得“只被异性吸引”的成功。批评家们质疑这些事后追溯性记忆和证据的准确性(因为没有对性反应之前和之后进行检测)。即使如此,斯皮策明确地说,难以寻找那些自我报告性取向已转变的人这一事实表明,重新确定性取向“可能确实相当罕见”。

然而,这个研究提示了最近由其他研究者报告的—女性的性取向没有男性性取向的感觉强烈,前者可能比后者更不确定且易变。这个发现符合另一个关于男性的“目的更明确”的性唤起研究。在他们的自我报告和测量的性反应中,异性恋男子更容易被女性色情刺激物唤起,同性恋男子则更多被男性色情刺激物唤起。女性则不管性取向如何,对男性和女性刺激都产生反应。

鲍迈斯特指出,男性性方面较少变化在其他许多方面也有明显表现。由于时间、文化、处境、教育水平、宗教信仰和同伴影响等方面的不同,成年女性的性驱力和兴趣比成年男性更加灵活多变—这是鲍迈斯特称为“性爱可塑性”的性别差异现象。比如,女人比男人更喜欢几乎不分时段地更替高性欲活动期。

目前已经基本不再提性取向的改变问题。

大多数人,无论是异性恋还是同性恋,都接受他们的性取向—通过选择独身生活,或者性乱交(通常男同恋者比女同性恋者更多作出这样的选择),或通过建立一种承担义务的、长期的爱情关系(与男同性恋相比,女同性恋者更多地作出这样的选择)。贝尔和韦恩伯格总结道:“那些最终对同性恋妥协的,不后悔同性恋性倾向的,和可以有效发挥性和社会作用的同性恋者,并不比异性恋男女有更多的心理压力”。

美国精神病学会于1973年把同性恋从“心理疾病”的名单中划掉了。世界卫生组织于1993年、日本和中国的精神病学会分别于1995年和2001年把同性恋排除在“心理疾病”之外。

根据以上章节的介绍,无论是从生理、心理和精神健康状况等方面,都很难把同性恋者从异性恋者中清楚的划分出来。这可以推出另一个观点:同性恋者的生活方式和异性恋者的生活方式一样多姿多彩。所有社会阶层、职业、种族、宗教以及政党的信仰在同性恋者中普遍存在。同性恋者唯一共通的特征是他们对同性别的人有情感和性满足的需求,还有他们都经历过来自于敌对他们的社会环境中的压力。

尽管同性恋者与异性恋者有许多的相似之处,并且他们的生活方式也是多种多样的,但是人们对他们依然存在着典型模式化的看法。大部分的典型模式与同性恋者个人的穿着打扮有关。确实有一些同性恋的人按照一些固有的形式穿着打扮和为人处世。能够辨认出是同性恋男性的特征通常包括:夸张的“女性”姿势以及紧身且华丽的衣着;相对应地,能够典型地认出是同性恋女性的典型特征包括:短发和高度“男性”化的衣着和手势。

意识到自己同性恋性倾向的重要步骤就是接受它。自我接受通常是困难的,因为它牵涉到必须克服内心对同性恋的抵触以及社会上对同性恋的憎恶的观点,由于这些困难,男性同性恋中青少年尝试自杀的可能性较男性异性恋青少年高七倍。十几岁的女性同性恋者试图自杀的可能性只比正常的女孩高一点儿。孤独、缺乏自尊以及身体和语言的羞辱通常是试图自杀的原因。对于同性恋的男女青少年来说,至少找一个支持他们、没有偏见的成年人谈谈心,是很有帮助的,家庭的支持也是非常有价值的。

在目前,想通过医学、心理治疗或者其他手段来改变一个人的性取向还是相当困难的,这是针对素质性同性恋而言的,一是帮助接受自己的性取向,明白同性恋者的智力和能力并不比异性恋者差,一样可以实现自我价值可以为社会作贡献,而且现在越来越多的专业人士认为同性恋并不等同于“性心理障碍”和“性变态”。另外在接受自己是同性恋的同时,也要接受社会上许多人对同性恋的态度,因为人们对与自己不一样的人都会有好奇和不理解,这也是正常的。寻求归属的需求是人类发自内心的特征。对于许多同性恋者来说,要安排好自己的生活。越来越多的心理学工作者们不再认为必须通过改变同性恋者的性倾向来“治愈”他们,而是开始努力帮助他们去爱、去生活、去工作。

总之,同性恋现象是一种客观存在的,尽管原因未明。同性恋不是当事人故意所为,也不是习得的,因此同性恋者对自己的性指向本身不负责任,与不道德或罪错无关。

因同性恋问题来求助的人,往往因为在精神上非常苦恼。这些苦恼通常来自父母、配偶、朋友、同事、邻居、法庭等社会因素并伴有内疚感、担心被歧视和责难的情绪体验,同时由于很多前来求助的同性恋者对于自己的身份存在一定程度的不接纳,这也造成其内心痛苦的来源之一。当然,若其痛苦确系植根于内心世界,改变性取向的努力便不会有太明显的效果,也不恰当。但帮其建立良好的人际关系,自我身份的接纳,也可以帮助其减轻内心的苦恼。

因此,在与同性恋求助者的治疗过程中,并不一定要改变其性取向。帮助其正视自己的同性性取向,宽容的接纳它,在行为上采取适当的社会适应的约束,可使求助的同性恋者得到更切实的好处。

除了一些来自社会的外源性苦恼以外,部分同性恋者还存有内在的痛苦。所谓内在的痛苦也不是生来就有或直接由同性性取向本身所决定。家庭教养的模式在很大程度上对于个人成长以及性取向方面有着不可避免的影响。案例(1):求助者为男性,30岁左右,硕士学历。因为要面对父母催促结婚一事而感到焦虑,因为该男子知道自己是同性恋者。来访者自述小时候父母经常吵架,父亲暴躁易怒,常常无辜殴打母亲和自己。很多次梦中醒来看到的就是父母厮打的场面,内心感到非常害怕。在该男子出生之前,已经有了一个哥哥和姐姐,但是姐姐夭折。母亲多次在该男子面前表示过希望有个女儿,如果姐姐活着就不会再要他。该男子从小乖巧听话,也曾多次被亲戚朋友们称为像丫头一样,小时候认为这是一种赞美。由于父亲经常施暴,因此也给他留下了很深的烙印。大学期间发现自己爱上同班的一个男生,但因该男生并非是同性恋者所以这段关系并没有发展。大学毕业后该男子曾与一位同性恋者同居3年,在这期间多次发生性关系,并且该男子始终扮演插入者角色。虽然那几年生活平静,但该男子总感觉这不是正常的生活,没有结果,没有未来。从内心来说自己也并不是那么愿意接受自己的同性恋生活。随着年龄的增长,父母常催促其婚事,加之自己也感觉这样下去并非真正的生活,因此希望通过心理咨询改变其性取向。案例:(2)女性,20岁上下,着装男性化特征明显。每天都穿那种有军队风格的野战夹克,搭配工装裤子。脚上穿着一双军靴。无论是穿戴还是行为举止都像个男人。从小到大好打抱不平,从来不服输,连喝酒都一定要超过其他男生。后来得知,该女子的父母一直盼望能生个男孩,却始终未能如愿。她是家中的长女,父亲对她特别疼爱。小时候就常常在与父亲相处的过程中听到父亲伤感的说:“你要是个男孩子该多好!”从此,她的行动越来越像个男孩,这些变化也许是她希望自己可以成为父亲心中的儿子。随着年龄的增长,该女子也曾对异性产生恋情,但是很难与异性接近,感觉无法忍受异性与自己亲密接触,因而产生忧郁的情绪前来求助。以上案例显示出的家庭环境、教育方式以及父母对子女性别的期待等,都会成为影响个人性取向发展的关键性因素。与此同时家庭和社会对同性恋的否定性道德评价可以在童年就内在化,从而成为同性恋者价值系统的一个组成部分,这种内疚感是“深埋”的,不容易改变的。在这一点上,治疗师应该在开始心理治疗的初期就与求助的同性恋者取得共识。

因同性性行为被人觉察或当面指责而感到羞耻,或一个人独处时想起自己的同性恋而感到内疚,这两者的区别往往很不容易。心理治疗师对同性恋者的性行为持中立态度,而对其内心痛苦和现实处境困难则持理解、同情、接纳的态度,而这种态度是弄清楚有关性行为的各种内心活动和情感体验的必要条件。

事实上,同性恋者常伴有情绪问题或人际关系(包括与性伴侣的关系)问题,这些问题的发生率要比异性恋者高。这种现象并不难理解,因为文化和社会环境对同性恋都是不利的。

在笔者多年的临床经验和研究中发现,更多的同性恋者寻求医生并非要求治疗或改变他们的性取向,而是由于情感、情绪上出现困难或苦恼才寻求心理医生的帮助。但在初期他们很可能既不谈性问题,也不谈情绪问题,而只谈一些躯体的症状,比如:失眠、头痛之类的生理困扰。有经验的心理治疗师不会仅停留在其生理性症状的表面,带着耐心、专注的接纳态度倾听求助的同性恋者带来的问题,要给他们足够的时间表达自己,在这样陪伴的过程中不难发现他们带来的生理性症状背后存在着心理上的问题,而逐渐深入的交谈便有可能触及求助者的性心理,这也为最后的确诊提供了事实依据。

来求助的同性恋者往往对相关方面的信息了解很有限,在初步建立关系以后,治疗师可以向他们提供相关的信息,这对于缓解他们内心冲突可能有好处。例如:20世纪70年代初美国出现过同性恋者为自己权利而斗争的运动以及所取得的进展,瑞典已经通过立法允许同性者结婚并得到法律保护等。

对于那些真心实意的想成为异性恋者的人,治疗的有效性比人们想象的要好得多。大约1/3或更多一些的人经过治疗可以达到能被异性引起性反应和性兴奋的程度,甚至有令人满意的异性性生活。

目前治疗的方法有多种:精神分析;精神分析取向的各种不同的心理治疗技术;分析性小组(集体)治疗;行为矫正技术包括条件化、脱敏、再条件化、厌恶技术等。但据说不同方法报道的疗效大致相同。

决定疗效和预后最重要的因素是求助的同性恋者有无改变性取向的强烈动机。但是,光有动机还是不够的。他们的人格或自我整合的水平,自我力量(对挫折和冲突的承受能力)的强度也很重要。如果有原始的防御机制起作用,例如倾向于产生精神病性反应,那么,治疗成功的可能性就很小。例如:在1981年,有一名叫苏珊的女子尸体在自己的寓所里被发现,令人困惑的是该名女子在九个月之前刚刚做完变性手术。在此之前,她一直是一个男人,名叫瓦特。讲述这个人的离奇经历要从她还是瓦特的时候讲起。1925年瓦特出生在北卡罗来纳州,父亲是某著名神学院的院长,祖父则是美以美教派的教主,家里一直充斥着浓厚的宗教气氛。在瓦特出生的前三年,这个家里曾有一名聪颖非凡的男孩降生。导致瓦特自幼在哥哥的影子下长大。但哥哥在11岁的时候死去,父母悲痛欲绝。然而瓦特对哥哥的死产生了一种“解脱”与“罪恶”的复杂感觉,但这种感觉使他发奋想要超越哥哥。后来,瓦特以优秀的成绩赢得了以纪念哥哥而设立的奖学金。欣慰的父亲写信嘉许瓦特,但却不慎将瓦特的名字写成其哥哥的名字,这件事让瓦特心理非常难过,并产生急于逃离这个家庭的想法。后来在其16岁的时候,进入著名的普林斯顿大学,在这个男孩子的世界里瓦特开始了所谓的“情境性的同性恋”行为。然而,当周末来临那些曾与他发生关系的同学都到纽约去和女孩子约会的时候,瓦特才感受到深深的孤独感,他发现自己对女生根本没有兴趣。带着这样的感觉,他不敢与任何人倾诉。然而他也逐渐的认为自己是一个男女混合体,虽然拥有“男性的身体”,但却“被当做女人来使用”。二次大战期间,瓦特也参军服役,在这个阶段他仍旧持续认为自己是女人,也一再地在同性恋行为中扮演着女性的角色。退役后,瓦特到哈佛大学攻读博士学位,后又分别在麻省理工学院和一些著名的院校任教。表面上,他是一个拘谨、学有专长的学者,和同事保持礼貌而不亲密的关系,但私底下却和同性恋者搞在一起,常常过着危险的生活。与此同时还经常参加异装癖的社交活动。随着年龄的增长,瓦特也越来越大胆的在同事们面前穿着女人味的衣服,佩戴女人的装饰物等,渴望变成女人的欲望也越来越强烈,瓦特先进行了荷尔蒙疗法,又在1981年不顾医生的劝阻依然进行变性手术。在变成真正的女人以后,她曾经的背部疾患也再度加剧,最终因过度服用可卡因致死。在死之前,最后和她见面的是一个女人,而且是个女同性恋者。在她最后的日记中写道:“现在(手术后)我觉得我是个女同性恋者”。当他还是一个男人的时候,他无法与女人发生亲密的肉体关系,但在成为女人以后,却又无法和男人发生肉体关系,而变的渴望和女人做爱。

这是一个较为极端的案例,根据瓦特的成长以及人生经历似乎真的无法判断其究竟属于那种类型导致的性取向障碍。可以明确的是,瓦特的内心非常排斥从小就被当做哥哥替身来抚养的经历,这可能是造成他对男性性别认同的混淆原因之一,然而在念大学期间清一色男性环境下,和同学发生同性性行为也可能是其最后选择同性性取向的诱因。然而,这样的情况在绝大多数有着同样经验的人来说似乎并不像他那样排斥女性,完全对女性没有兴趣,因此瓦特的性取向的另一个促进因素可能源自于生物学上的原因。环境仅仅是起到催化的作用。不过令人错愕的也许是当其真正拥有女人的身体以后,她仍旧是一个同性恋者,这又使我们不得不认为他的问题并非是单纯的生物学上的问题。也许其核心问题是他是一个女性化的同性恋者。这个案例给我们从事心理治疗工作者一个很好的提醒,同性恋的成因复杂并不是可以简单的根据一些信息就可以评估出治疗效果的。在治疗的过程中,可以采用心理测量术评估一下求助者的人格和自我力量,也可以通过与求助者轻松自在的交谈了解他的个人心理发育史和生活经历,以及他的兴趣爱好,成就,人际关系水平,人际交往的动机和技巧等。这些都可以帮助我们评估治疗的可能效果和困难。当然一些有过异性激起性兴奋的经验,是预后良好的一个指征。反之,根本从来没有过这种经验预示着治疗的困难。一般地说,年纪轻的比年纪大的对治疗的反应要好一些,但年龄并非最重要的因素。改变引起性兴奋的“客体”是治疗最困难的部分。即使是所谓自我失谐的同性恋,“客体”引起的直接反应仍然总是伴有性的快感。性兴奋和性高潮是行为极有力的强化者。

治疗师应该帮助前来求助的同性恋者对治疗可能达到的效果采取现实主义的态度。追求的目标太高,结果适得其反,强烈的挫折感会冲淡甚至抵消已取得的初步疗效。这是必须事先想到而加以防止的。如果激起了求助者的羞耻或内疚,那就有可能使治疗失败。因此,治疗的一个重要内容或方面,是帮助他们接受自己(同性恋是自我的一部分),不带羞耻和内疚去看待自己。不论人们对造成同性恋的原因取何种科学观点,有一点是毫无疑问的,那就是:同性恋既非邪恶,也不是堕落或衰退的表现,它是丰富多彩的人性的变异形式之一。当然,这种变异形式也跟其他变异形式一样,是有可能改变的。治疗师必须在这个基本观点上与其取得共识,这既是治疗成功的条件,也是治疗的目标之一。

在对同性恋者各种情绪问题和人际关系问题的处理上,跟异性恋者并没有什么原则上的不同。对于常见的焦虑或抑郁这两种情绪障碍并发症,可以考虑用抗焦虑剂或抗抑郁剂作对症治疗,以减轻患者的痛苦。除非出现精神病性发作,抗精神病药不宜使用,因为有害无益。

总的原则是,治疗师不能单纯着眼于同性恋,而应该把治疗的对象看作一个完整的、活生生的人,是和周围环境有相互作用的人。这是取得患者主动合作的条件,也是防止治疗走向片面而削弱效果的战略原则。

作为医生,我们最好把性取向本身究竟算不算“病”看作一个学院式的问题,留给学者去从容探讨。而对于求助的同性恋者,我们有义务给他们提供帮助,减轻他们的痛苦,帮助他们提高自己的生活质量。医生的业务实践与其说是一种科学活动,不如说是一种服务。

通过笔者等人的研究显示:“自我不和谐同性恋对待自己性倾向的认识与认同是一个过程,他们在青春期因为和常人不同而自卑,认为自己不同寻常、有缺陷,感到十分痛苦。在我们研究中发现由自我不和谐同性恋到自我和谐同性恋是一个过程,时间长短因人而异。”

最初,一个人可能假定自己是异性恋身份,因为在我们的社会里异性恋才是正常的。当同性吸引或同性性行为出现的时候,就出现了混乱。

一个人这时会想:“我可能是同性恋。”这是其可能会存在异化的感觉,因为符合理性的异性恋身份消失了。

在这个阶段,一个人会想:“我大概是个同性恋”。然后,开始接触同性恋者及同性恋亚文化,希望能得到确认。最初接触到得这些亚文化特性是至关重要的。

一个人这时候会说:“我是同性恋”,并且接受而不仅仅是忍受这种认同。

一个人将世界分为同性恋者和异性恋者。对同性恋者有着强烈的认同,并有很强烈地向周围人承人自己是同性恋的愿望。

一个人不再用“我们和他们”的观点来看待异性恋者和同性恋者,而是承认有一些异性恋者是友好的,并且是支持他的。在这最后的阶段,他能够综合公众的和个人的性别认同。

在有同性恋行为的人中可以分为素质性同性恋、从根本上排斥异性恋、双性恋者心理两性人、特定环境下短期同性性行为者如囚犯、海员、民工等流动人口和特殊群体中和偶尔为之的同性性游戏行为者。

若把这几类人都统计为同性恋者,显然是不科学的,数字必然扩大。但有些心理医生却不加区分地把后几种情况简单地判断为同性恋,并告之“治不好”,而且此种情况不是个别。尤其后两类人的存在,表明人类在缓解性焦虑方面拥有多方面的能力,但绝不证明他们就是同性恋者,以及同性恋行为的存在就是正确的。

对于处于自我不和谐状态的素质性同性恋,虽然不具备矫正性取向的可能性,但心理咨询工作应该帮助他们面对自己的性取向,主动适应社会环境。

对于其他情况,心理咨询工作应该帮助他们了解自己性取向变化的可能因素,客观地指出心理调节调整性取向的痛苦和困难程度,尊重他们的选择,并帮助他们自我适应和适应环境。但是这种帮助与矫正都必须建立在咨客自我选择的基础之上。笔者多年的临床工作中了解到对于前来求助的同性恋者中的30\%左右可改变其性取向或松动其性取向。

一位高中二年级的男生,因为渴望与本校的“校花”“谈恋爱”受到了挫折,迅速出现了排斥异性的倾向,并出现了同性恋性行为。在最初的咨询中,他坚持自己是同性恋而且表示同性恋取向很好。

一位30岁的男青年,自幼丧父,从小就渴望中老年男士的关爱。成年后非常欣赏有文化、有教养的中老年男士,出现了“老年同性恋”的倾向,迟迟不肯组成家庭。

一位中学生随父母来到大城市,一时没有融入同学之中,看到别人都有好朋友,自己感到孤单。于是想通过网络寻找朋友,结果与一同性恋者相识。在对方的诱惑下很快出现了同性恋性行为,持续多年后,出现了严重的自我不和谐状态。

以上几例,都不是素质性同性恋,有的根本就算不上是同性恋,可是他们都紧张焦虑地认为自己是同性恋。面对这样的来访者,我认为首先应该向他们介绍什么是同性恋和同性恋群体在社会生活中的边缘处境,其次帮助他们寻找导致他们同性恋取向的可能因素,最后根据他们的选择帮助他们解决现实的问题。

问题出现的顺序虽不能简单地说就是因果关系,但解决他们的心理问题,就有助于他们自己矫正已经出现的性取向问题,而选择主流文化,却是可以肯定的。

有些人会提出上述情况并非素质性同性恋,或者不是严格意义上的同性恋,因此,不足以说明心理咨询干预同性恋问题的必要性。但是,必须看到,他们已经承认自己是同性恋了,在一些学者的调查统计中类似的一些情况已经被统计在同性恋的群体中,也被一些心理门诊确定为同性恋了,而且,其中的一些人继续发展下去,他们的同性恋倾向会进一步强化和固定下来。在这一点上尤其表现出心理咨询工作介入同性恋问题具有非常积极的意义。

从发生学角度,存在着一些不同的认识,但目前更多的只是假说。而且,客观上同性恋已经形成了自己的群体,绝对数字亦不可低估。不可否认,有些同性恋者除了性取向不同外,不存在其他任何心理和行为问题,一些人甚至非常出众。他们的公民权和平等权利不容忽视,这体现了社会的进步和综合生活质量的提高。我认为在咨询环境中不应该鼓励、支持存在性取向变化的非素质性同性恋者进一步向性偏离方向发展,但必须尊重他们的选择。


\section{第八节 同性恋生活}

我们在前面章节已经得知,不能依据激素的水平和精神健康状况,把同性恋者从异性恋者中清楚地划分出来。这可以推出另一个观点:同性恋者的生活方式和异性恋者的生活方式一样多姿多彩。所有社会阶层、职业、种族、宗教以及政党的信仰在同性恋中普遍存在。同性恋者唯一共同的特征是他们对同性别的人有情感和性满足的需要,还有他们都经历过来自于敌对他们的社会环境中的压力。

尽管同性恋者与异性恋者有许多的相似之处,并且他们的生活方式也是多种多样的,但是人们对他们依然存在着典型模式化的看法。大部分的典型模式与同性恋者个人的穿着打扮有关。确实有一些同性恋的人按照一些固有的形式穿着打扮和为人处世。能够辨认出是同性恋男性的特征通常包括:夸张的“女性”姿势以及紧身且华丽的衣着;相对应地,能够典型地认出是同性恋女性的典型特征包括:短发和高度“男性”化的衣着和手势。

尽管符合这种模式的人为数很少,但这种典型模式持久存在。这种现象的部分原因在于,那些认为同性恋者有固定的模式的人留心观察那些看起来符合这种想象的同性恋者并对他们加以分类,当然有时这样的分类是错误的。绝大多数同性恋者可能根本不适合这个模式的事实通常被忽略。一项研究发现,异性恋者和同性恋者区分访谈录像中表现出同性恋男女和异性恋男女的水平,都不会超过随即选择水平。

同性恋个体决定隐藏或公开性倾向的程度对他们的生活方式有重大的影响。“关在私室中”的程度有多种,而坦白真相的过程涉及一些步骤,即承认、接受、并公开承认同性恋。尽管这些决定对每个人和在不同情况下不尽相同,但通常会有一些相同的组成部分。

个体知道自己同性恋的情感发生在生命的不同阶段,非常“隐蔽的”同性恋男女会努力压抑自己的性倾向,甚至是从他们自己的思想意识方面进行抑制。

这些人会主动地寻找异性性伙伴,并且为了努力证明他们自己的“正常状态”,他们中结婚的人也并不少见。一些已经结了婚的同性恋者这样做的目的是为了避免公开地面对他们的性倾向。

坦白真相的第一步通常是,一个人意识到他或她对异性恋模式有不同的感情。一些人报告说当他们还是小孩子的时候就知道自己是同性恋者。许多人在青春期意识到(或到大学期间才意识到),他们的异性恋中缺少点什么东西,并发现相同性别的同伴更具有性方面的吸引力。一旦个体意识到自己的同性恋情感,他们就必须面对自己内心中对同性恋的憎恶态度,以应付自己成为被指责的少数人中的一员的现实。

意识到自己同性恋倾向的下一个重要步骤就是接受它。自我接受通常是困难的,因为它牵涉到必须克服内心对同性恋的抵触以及社会上对同性恋的憎恶。

当个体意识到他们属于在社会中被指责的团体时,自我接受就成为一个困难的但必须挑战的事情。一项研究发现,美国黑人男女同性恋者抑郁的痛苦的发生率比白人要高,这可能是由于黑人受到了人中和性倾向的双重指责。

几乎没有一个学校有公开的同性恋学生组织;大多数同性恋的男生女生都经历了情感的困惑,并且他们没有地方去获得支持和指导。他们通常都认同了关于同性恋的错误且消极的典型模式,并且可能没有认清自己同性恋的感情,或者由于他们这种感情以及看上去不可避免的不幸、迫害和拒绝而使他们感到特别的糟糕。并且他们好像还认为自己是唯一的有这种感觉的人,从而感觉到强烈的孤独并与其他同龄伙伴格格不入,而后者在这一生活阶段里是非常重要的。

那些严格的、好说教的、恪守性别典型模式的家庭给同性恋者增加了压力,一些父母把他们同性恋的孩子赶出家门。

由于这些困难,也使得男性同性恋者在青春期阶段尝试自杀的可能性比男性异性恋者高七倍。十几岁的女性同性恋者试图自杀的可能性只比异性恋的女性高一点。孤独、缺乏自尊以及身体和语言的羞辱通常是试图自杀的原因。对于同性恋的男女青少年来说,至少找一个支持他们、没有偏见的成人谈谈心,是很有帮助的;家庭的支持也是非常有价值的。

在承认和自我接受之后,接下来就是决定保密还是公开。做一个同性恋者需要随时在新的关系和环境出现时作出决定,是继续保密还是对外公开。隐藏自己的性取向会增加同性恋者的社会孤立感和个人孤独感。所有通过隐蔽而获得的安全感都会被任何时候的发掘所破坏。“作假”是一个术语,有时用来说明维持异性恋假象的行为。装作是异性恋十分容易,因为大多数人假定所有人都应该是异性恋者。

在一些日常交往中,性取向是无关紧要的。但同性恋和异性恋之间总有一种暗流在涌动,冲击着彼此生活的许多部分。一位男同性恋者这样说:“我同性恋的生活本身并不是重要的事,它只是发生在我的卧室里和私事上。但它却会影响很多方面:如谁是我的朋友,我选择谁与我共同生活,我做什么工作,我属于哪个组织,我读哪些杂志,去哪里度假和谁聊天等”想象一下这样的情景,做一个秘密的同性恋者,听朋友轻蔑的提到“fags”或“dykes”;当被问及:“你打算什么时候结婚成家?”等。因此承认或否认同性恋是一个重要的行动,它会给人们的生活方式带来重要的影响。

“作假”同样会给男性和女性同性恋关系的质量产生不良的影响:每天不得不过着双重生活———一个生活在公共场合中,另一个生活在私生活里———对我们的性表达有不良的影响。与那些你在整个星期的工作都排斥的人有性关系是非常困难的。在外是“朋友”和在家是充满激情的爱人之间来回变换,会对我们性自由的能力产生破坏性的影响。

还有一些例外,就是在“这个体系”中一个人做的或渴望做得越多,暴露性倾向的危险性就越大。工作、社会地位和朋友都有可能陷入危机之中。周围人群的保守程度会进一步地影响这个人是否暴露和向谁暴露的决定。例如,一个同性恋的医学生可能在上大学的时候公开了自己同性恋身份,但在医学院中又因为害怕在他的职业生涯中受到歧视而转为隐蔽。生活在大城市中的同性恋者相对于生活在郊区和小乡镇的同性恋者来说,有更多的人公开了自己的身份。公开自己是个同性恋者所得到的相对较多的放松使人们向往过城市的生活。

对于已经为人父母的同性恋成人来说,公开自己是个同性恋是一个特别困难的问题。事实上,一些人由于这个原因而继续维持婚姻。差不多60\%的已婚同性恋男女至少有一个孩子。同性恋父母在试图获得监护权或探视权的努力方面面临的困难也会相当大。不管他们多么适合做父母,但是由于他们的同性恋倾向而失去这些权力的情况并不少见。

很多同性恋者因为预期到父母对自己性取向的不接纳和对各种风险的评估,使得他们选择永远不告诉父母。父母因此被隔绝在真相之外,永远无法真正的与孩子一起成长,分享他/她的快乐和忧伤。

在一次家庭治疗的过程中,心理医生发现那个曾因多次自杀未遂而被父母带进治疗室的孩子始终表现着无发言说的痛苦,当心理医生单独与其谈话的时候,孩子说:“之所以痛苦,是因为我是同性恋。但我不能告诉我的父母,因为我们家是一个虔诚信仰基督教的家庭。”心理医生在得到孩子同意后将这一发现告诉了父母。然而,父母听到后的反应并不是愤怒而是留下了伤心的泪水,他们说:“如果你的痛苦源自于此,我们宁愿接受自己的孩子是个同性恋,也不希望就此失去你!”

同性恋者的父母曾经认为自己与寻常的父母一样,直到某一天,他们突然发现自己的孩子是同志。这些父母在中年甚至老年才开始学习做同志子女的父母,这是一个痛苦的然而却更深刻的成长过程。

一位父亲说:“我家三代单传,原指望儿子可以完成传宗接代的任务,可谁知到他竟然是一个同性恋!起初知道这件事后,我觉得天都塌下来了,绝望极了。恨不得杀了他,然后自杀。但是后来通过心理医生的指导,我才知道这不是儿子的选择,不是他的错……”

知道孩子是同性恋这对于一个家庭来说是件颠覆性的事件。因为父母是异性恋,所以会将对子女的期待和养育框定在异性恋的社会制度中,但是当得知自己的孩子是同性恋的时候,他们痛苦、不解、哭泣、抗拒甚至伴有罪恶感和毁灭感。当得知真相的初期他们的情绪和感觉是复杂的,无法言说。

当然,很多父母不仅仅因为意识到自己作为人类的一分子将不再有基因在这个世上传播,这种深深印刻在内心对死亡的恐惧让他们焦虑以外,同时也因为对孩子真挚的爱,使得同性恋者的父母不得不去替孩子们思考未来或检讨自己的人生。

一位拉拉的父亲这样表达了对女儿的担心:“你现在可以依照自己的意愿这样做,但是当你老了以后,孤苦伶仃的一个人住在养老院里,无依无靠,没有亲人。听着养老院的工作人员在背后议论你是个老变态,到那个时候你该怎么办?”。

当一位单亲母亲发现自己儿子是同性恋的时候,她感到非常痛苦和后悔,一遍又一遍地问自己“是不是因为我和他父亲的离婚,让他走上这条路?”、“是不是我个性太强,对他太严厉,让他对异性产生抗拒?”、“是不是因为他很小就没有父亲的陪伴,使他用这样的方式来填补父爱的缺失?”她在不停地自责中生活,在自己失败的婚姻和儿子是同性恋两者之间画上的等号。社会学家李银河说:“这些父母很无助,很痛苦,很自责,好像孩子是同志就是得了一场大病。”难以接受孩子不是自己所期待的样子,这样的困难在同志父母中很普遍。

也是因为这样,很多父母会提出一个关键性的问题:孩子,你是否可以变成正常人?

变正常是很多同志父母的最大愿望,这里常常隐藏着一个误区或成为一个超级障碍,即误以为同性恋是一种选择。在笔者的临床工作中接触的同性恋者及其父母通常是因为父母发现子女是同性恋因而强行将他们带到心理医生这里,并要求心理医生将其性取向转变过来。或是有些同性恋子女带着父母来到咨询室,希望通过心理医生来转告父母他/她是无法改变的事实。作为心理医生,我们只能解决由性取向引起的情绪问题,而无法将治疗目标定位在改变性取向上。无法改变同性恋倾向,这种观点既让父母绝望,但同时也可以令很多父母获得慰藉,因为他们逐渐明白这不是他们的错,也不是孩子的错,从而减轻了负罪感。

当发现自己的孩子是同性恋时,父母应该认识到这是学习一个重新做父母的机会,父母对孩子是同性恋的反应是完全可以理解的,是因为整个社会没有给父母提供一个友善的环境和资源。但这并不意味着父母不会接受不能成长。

同性恋就像人类的左撇子一样是自然的。当父母们了解到这一点,就会理解孩子不与异性结婚是不愿意伤害一个无辜的男人或女人,那么他们就会尊重孩子的爱,给予他们接纳、认同和支持。对于同性恋者,来自父母的接纳是非常重要的。这让他们感到不孤立,感到被接纳。

寻求归属的需求是人类发自内心的特征。对于许多同性恋者来说,社区提供了一种归属、确认和接受的感觉,而这感觉正在大的文化背景中逐渐消失。与其他同性恋者共同融入社会和政治之中是公开身份过程的另一步骤。在一些大的城市中,同性恋酒吧咖啡厅适合不同的人群或顾客,像异性恋的酒吧一样,这样的聚集地点分布在从高级到低级的不同环境中。尤其是在过去的几年中,同性恋酒吧或同性恋聚集的社区就像一些特别设计的娱乐场所担负着一项重要的功能:它们往往是唯一让同性恋者卸下它们异性恋面孔的地方。随着社会群体对同性恋人群态度的变化,在国外同性恋人群在帮助下建立了相关的服务组织、教育中心和职业组织,还有面向同性恋者的宗教团体也已建立,包括在19个国家中有400个圣会4万名成员的大都市社区教会以及像罗马天主教会与圣公会那样的宗教团体等,这些都不在不同程度上满足了生活在大城市中的同性恋者对于归属感、认同感的需要。

不仅如此,这样的同性恋社区还针对艾滋病危机发动了有教育意义的运动,并开展有创新性的项目来照顾艾滋病患者,建成自愿者网络为艾滋病患者提供必要的帮助,并且经常公开地游说义演以加强艾滋病意识和筹集医学研究基金。

即便如此,对于那些居住在大城市以外的同性恋者仍很少有机会参加同性恋的社区活动。

互联网以前所未有的方式提供了一个同性恋的“虚拟社区”。无论身在何方,只要接入互联网就可以随意与其他同性恋者产生私人的联系。与异性恋相同,互相联网也为同性恋者提供了虚拟的性、性信息和各种服务。更重要的是,它们和同性恋男女隐秘的相互交谈所获得的信息一样。在网上他们可以通过个人广告找同性恋伙伴、对同性恋者友好的度假胜地进行研究,联系青少年自杀防止热线,得到艾滋病治疗的最新消息,加入支持少数民族的团体或从事网上政治等。


\section{第九节 同性恋的跨文化观点}

不同的国家、民族对同性恋的态度有很大差异。大量对其他文化的研究显示了对同性恋活动的广泛接受情况。一项对190个社会的研究发现,三分之二的社会可以接受一些特定的同性恋个体,或在特殊情况下可以接受同性恋。在许多较早期的文化中同性恋被广泛接受。例如,在225个美洲土著部落中,超过50\%接受男性同性恋,17\%接受女性同性恋。

在拉美文化中,强调男子气概,通常会导致男性同性恋对他们的性倾向保密。在美国的黑人社区中社会经济地位较低的人群中,强调坚强的男子气盖世理想的性别规范,使不遵从性别规范的男性同性恋者陷入特别的困境。

亚洲文化体系尤其认为对家庭的忠实和遵从是很重要的,并认为个人的需要和渴望是不重要的。亚洲人通常被看做是他或她家庭的代表,而不是个体的代表。另外,人们从不谈论性,并且对此保密。公开自己是个同性恋,或即使是在家里讨论性,是与亚裔美国人的传统价值观相违背的。公开同性恋身份被认为会使家庭蒙羞,并且威胁到家庭的未来。没有结婚,或没有为家族传宗接代,被认为是造成了家族延续的失败。然而,秘密从事同性恋的行为同时又能满足家庭的期望,则不会有任何罪行。

在20世纪中叶,对同性恋的社会态度有了一个转变。认为同性恋者是罪人的想法在一定程度上被认为他们是“病态”的想法所代替。医学和心理学工作者使用了很多激烈的治疗方法来努力治愈同性恋这种“疾病”。在19世纪,诸如切除生殖器这样的外科手术得到开展。后来到1951年,脑白质切除术(分离大脑前叶的神经纤维的脑外科手术)作为对同性恋的一种“治疗方法”开展起来。心理治疗、药物、激素、催眠、电击疗法和厌恶疗法(在同性恋刺激的同时给予使人呕吐的药物或电击)也都曾被应用过。

如今,通过几十年的研究推翻了同性恋是“病态”的观念。第一个主要的研究是对比非病态的异性恋者和同性恋者的适应性并没有发现两组人群有明显的差异。进一步的研究支持这些发现。贝尔总结道“那些最终对同性恋妥协的,不后悔同性恋性倾向的,和可以有效发挥性和社会作用的同性恋者,并不比异性恋男女有更多的心理压力”。

(刘华清)


\section{第十节 美国互联网抽样中男男性行为的性功能障碍}

关于男男性行为(MSM)中的性功能障碍(SD),知之甚少。大多数已发表的研究主要关注在HIV的传播风险背景下的性行为和性功能障碍。

美国在2004—2005年间实施了一项网络匿名调查,通过美国和加拿大的8个以同性恋为指向的性网站、聊天室和新闻网站主页横幅,链接到有关最近的和过去1年内邂逅中的性、毒品和乙醇滥用行为的问卷,征募了一份MSM的互联网抽样样本,分析样本限于拥有终生的和过去1年内有男性性伴侣的18岁及以上的7001名美国男子,其男性伴侣目前具有活跃的男男性行为。他们在过去1年内最近期的邂逅中与男性伴侣有口交或肛交。参与者定居于美国各州、加拿大各省或其领土或国外。完成调查需耗时10~15分钟,研究人员没有提供金钱刺激来完成调查,这次MSM调查参与者和其他在线调查参与者一样,显然愿意报告他们的健康和性功能状况。牵头机构的评审委员会批准了所有程序并允许放弃获得知情同意的要求。

这次MSM研究中SD症状自我报告采用的是美国国家健康和社会生活调查(NHSLS)的基于马斯特斯和约翰逊性反应4阶段理论的7个问题、答案为“是/否”。询问在过去的12个月中,是否有“一段时间”发生SD症状:(i)在性活动中缺乏兴趣(低性欲);(ii)达到或维持勃起有困难(勃起问题);(iii)不能达到高潮(无达到高潮的能力);(iv)恰好在性活动前感到焦虑,担心性操作的能力(操作焦虑);(v)太快达到高潮(早泄);(vi)在性活动期间经历疼痛或不舒适(性交疼痛);以及(vii)即使没有疼痛,也不感到性愉悦(无性愉悦)。研究目的是更好地理解过去1年中MSM报告的SD症状的基本模式及相关的人口统计学和行为的特征。

年龄为连续变量。关系状况编码为单身(未婚/无家庭伴侣)对所有其他状况(已婚/男性家庭伴侣;已婚/女性家庭伴侣;与一男子离异、分居或鳏夫;与一女子离异、分居或鳏夫)。收入分类为低收入(低于$10 000至$29 999),中等收入($30 000至$49 999),以及高收入($50 000到超过$100 000)。过去1年中的任何性传播的感染(STI)由医生或护士界定为新的或复发的诊断,并按以下分类:疱疹、人类乳头瘤病毒、衣原体、淋病、梅毒、软下疳以及非淋菌性尿道炎。HIV测试变量是自我报告的,分成HIV阳性对HIV阴性/未测试(是,否)。

过去1年中在性活动之前或期间经常使用药(毒)品的大约占公共方差的60\%。分别标记为药(毒)品因素1(“俱乐部毒品”克他命、透明的脱氧麻黄碱[熏吸、喷鼻、吞服或注射]迷幻药,或γ羟基酪酸盐);药(毒)品因素2(乙醇、大麻、亚硝酸戊酯、可卡因[熏吸、喷鼻或吞服])或镇定剂);药(毒)品因素3(西地那非、伐地那非,或他地那非[所有处方药]);以及药(毒)品因素4(西地那非、伐地那非,或他地那非[所有处方药])。药物分类并不相互排斥。此外,因为样本小,从药物因素中排除可卡因注射(N=32),以及海洛因(熏吸、喷鼻、吞服或注射,N=29)。

结果介绍了样本的人口统计学和行为特征,86\%的男子报告他们在过去1个月内与男性伴侣邂逅。应答者主要是高收入的白人和中位数年龄为38岁(范围18~85岁)。多数(89\%)男性自我认同为同性恋,10\%自我认同为双性恋以及1\%自我认同为异性恋。在那些回答HIV测试问题(N=6466)的男性中,14\%报告HIV测试为阳性。HIV阳性男性显著比HIV阴性(平均相差3.9岁)和未测试的男性(平均相差8.1岁)年龄更大。自我报告终生诊断的患病率包含精神健康和躯体健康因素的:焦虑(23\%)、抑郁(13\%)、双相障碍(躁郁症,5\%)、高胆固醇(17\%)、高血压(13\%)、心脏病(3\%)以及糖尿病(3\%)。年龄和每个身体健康因素以及终生抑郁之间存在线性相关。在过去1年男性口交或肛交性伴侣的中位数在6~10个之间,其中13\%报告过去1年有若干次STI。

总计79\%的男性报告在过去1年中“有过一段时间”遇到1个或更多的SD症状,最普遍的是性欲低下、勃起问题和操作焦虑。按患病率顺序,报告低性欲的男子(57\%)、随后是勃起问题(45\%)、操作焦虑(44\%)、无性愉悦感(37\%)、不能达到高潮(36\%)、早泄(34\%)以及性活动期间疼痛(14\%)。采用潜在类别分析方法可以把性功能障碍分为4个截然不同的SD症状基本模式:无/轻度SD(32\%)、勃起问题/操作焦虑(24\%)、低性欲/愉悦(23\%),以及严重的SD/性交疼痛(21\%)。无/轻度SD类别在多元逻辑回归中充当其他3个类别的对照组。在性问题患病率的研究中,MSM中报告的勃起功能障碍和性交疼痛发生率比男女性活动(MSW)时更高。

此项研究发现在总体上SD症状的比例高于过去的采用相似标准的MSM研究。其原因是:(i)此项研究包括在过去1年中“有过一段时间”的任何SD症状,而过去研究的时间框架是“至少一个月”或“连续3个月”的SD症状;(ii)也许因为这次是年龄相对较大的MSM样本(也即有健康问题);(iii)这项在线研究的匿名和缺乏面谈的偏差,在电脑上比人工面对面调查更容易得到真实的敏感信息。

过去1年里报告的SD症状因人口统计学和行为特征而有所不同。凡年龄低于30岁的、收入低于$50,000的、HIV呈阳性的、单身的、在过去1年里的性行为之前使用俱乐部毒品或勃起功能障碍药物的、在过去1年中诊断出患有STI的男性,以及报告终身精神的或躯体的健康问题诊断的男性明显地比他们的对等的人更有可能报告在过去1年里的SD症状。在过去1年性伴侣数少于6位的男子报告过去1年SD症状的可能性要小于性伴侣更多的男子。

以来自NHSLS的资料作为过去1年中MSW中SD患病率的参考点,总的来说,31\%的异性恋报告过去1年的SD症状:21\%报告早泄、5\%低性欲和5\%勃起功能障碍。这些差别提示问卷中的这些SD问题也许不适合于MSM的SD测量、MSM中SD也许真的很高、也许两者都是原因。无法比较MSW和MSM的SD症状,因为这两组人群具有各自不同的文化规范、SD问题解释、性期望和性伴侣性别。MSM也许比MSW更倾向于性操作焦虑。在MSM和MSW样本中进行的一项对勃起和射精功能的研究,年龄和操作焦虑强劲地预测着无论MSM还是MSW中的勃起问题,然而,无论他们是否报告勃起问题,MSM比MSW都有着更强烈的操作焦虑。

在症状水平多元分析中发现在年龄、关系和HIV状况方面的显著区别。年轻男子显然比50岁和更年长男子性更有可能报告在过去1年中的低性欲、早泄、性交疼痛和无性愉悦。相反,年轻男性显然很少会报告达到高潮的麻烦、操作焦虑或勃起问题。性交疼痛的可能原因是由于缺乏经验或STI所导致的直肠炎。50岁和年龄更大的男子更有可能报告达到高潮、操作焦虑和勃起问题方面的烦恼,他们经历勃起问题的可能性是30岁以下男子的两倍。症状也许预示着躯体或精神健康差和(或)与年龄相关。

单身男子与不论与男性或女性结婚或离婚/分居/鳏居的男子相比,显然更可能报告过去1年中除早泄和性活动疼痛之外的SD症状。可能是在性关系上缺乏经验或抑制男子进入相互关系的SD。

HIV阳性的男性比HIV阴性和未测试男性明显地更有可能报告过去1年中除早泄以外的所有SD症状。几项研究已经发现HIV阳性MSM比HIV阴性和未测试男性更有可能报告SD症状。报告中这一差别的原因也许由抗反转录病毒的药物副作用或HIV本身所导致,也包括社会心理问题,诸如抑郁、操作焦虑或担忧传染给其他人。

勃起问题/操作焦虑类别与大龄、单身、有躯体健康问题以及在性活动之前采用处方和非处方勃起功能障碍治疗药物显著相关,对照为无/轻度SD类别。有些报告这一行为模式的男子可能使用勃起功能障碍的处方药物来对付躯体健康问题,而另一些男性则可能使用非处方的勃起功能障碍药物以降低操作焦虑,以服药来作为后盾。低性欲/愉悦类别与年轻、单身、精神和躯体健康差、在过去1年中性活动之前不大可能使用勃起功能障碍药物显著相关,类别对照无/低SD类别。严重SD/性交疼痛类别与过去1年中性活动之前使用俱乐部毒品和处方或非处方治疗勃起功能障碍药物的其他类别相区别。与无/轻度SD类别的男性相比,与此类别相关的男性明显年轻、单身、精神和躯体健康更可能欠佳、更可能在过去1年中诊断患有STI。当模型中其他人口统计学和行为特征具有对照时,人种/种族、收入、HIV状况以及药(毒)物因素2的跨潜在类别分析无助于区分潜在类别回归分析中的类别成员,因而是不显著的。

低性欲和愉悦类别与年轻、单身和有精神和躯体健康问题相关联。发现年轻男性报告低性欲和低愉悦真令人感到意外。然而,极少研究专门去关注和处理男子的低性欲问题。我们无从知道在这项研究中报告低性欲和低愉悦的年轻男子是否能够得到性欲低下障碍(HSDD)的临床诊断。原发性男子HSDD是罕见并常常发生在性秘密的背景下,例如有性欲倒错、喜欢手淫超过喜欢与其他人发生性行为、性虐待史、对性身份有内心冲突。继发性HSDD更常见并最常被视为对性功能障碍的反应,诸如勃起功能障碍或早泄。继发性HSDD也许由精神健康问题(抑郁或焦虑)、药物不良反应或源自与伴侣相关的情况。最后,“完美性交表演”的文化也会很严重地导致继发性HSDD。

严重SD和性交疼痛关系到年轻、单身、居有精神和躯体健康问题、过去1年内的STI诊断、过去1年在性活动前使用俱乐部毒品,以及过去1年在性活动之前服用治疗勃起功能障碍的处方或非处方药物。MSM中的SD与躯体和精神健康问题和STIs相关联。与患有STI相关联的抑郁也许促使了SD发病,而且在STI已得到治愈之后SD仍长期持续存在。在某些MSM中的SD已经归因于在性交之前的物质滥用。在采用物质手段提高性愉悦的背景下,一些男性经历了性问题并把其定义为SD。诸如透明脱氧麻黄碱和迷幻药的俱乐部毒品能够抑制勃起。协同俱乐部毒品使用的勃起功能障碍药物的研究以抵消性的副作用与HIV传播风险相关联。在两份MSM和HIV传播风险的网络研究中,与性活动前使用透明脱氧麻黄碱相关联的风险因素,包括年轻、患有STI和HIV阳性。

关于MSM和SD的研究是有限的,还需要更多正规的工作以便理解MSM如何认识他们的性问题和出现那些问题的背景。研究者和临床医生已经注意到评估SD的诊断工具并没有承认MSM的性问题与MSW是有所不同的。例如在DSM中没有纳入与肛交相关联的疼痛,应该扩展一个类别称之为肛交疼痛(anodyspareunia),涉及与男性或女性肛交期间的疼痛。

群集的SD症状诸如勃起问题和操作焦虑、低性欲和低愉悦、严重SD和性交疼痛等表明,不同的子群体具有不同的功能障碍和相关特征。增进对MSM中SD症状的社会、文化和身体因素的理解有助于研究者和临床医生更准确评估和完善与MSM相关的SD标准。对于寻求SD治疗的男性,确定可能显示其SD的精神和躯体健康问题、药物或物质滥用或关系问题,对于MSM的SD的评估和治疗是重要的。使用互联网充当SD治疗的一个平台尚处于早期阶段,若干在线研究已经发现通过在线干预有希望减少MSW的SD症状,未来研究应该调查这些模式以发展可能为MSM的SD的在线针对性治疗。

(刘华清)


\chapter{第十九章 易性症与性别重塑手术}

期望作为异性来生活和被人们接受,通常伴有对自己解剖性别的不适感和不恰当感,并希望通过外科手术和激素治疗使自己的身体尽可能和所偏爱的性别一致。这种现象称为易性症(Transsexualism)。本病最早由Esquirol报道(1838),由Caldwell(1949)正式予以命名。易性症是一种少见的性身份障碍(Gender identity Disorders),国外有统计发病率为1/5万~10万,男女比例约为3∶1。在我国,易性症患者由于受到社会生活环境及伦理道德的约束,主动求医者很少,但这种状况随着社会的进步已有很大改变。

性别重塑手术是针对易性症患者在心理、药物等手段治疗无效的情况下而实施的外科治疗,即通过整形外科手段(组织移植和器官再造)使易性症患者的生物学性别与心理性别相符。20世纪30年代起,西方逐渐建立了一门新兴学科———性别重塑外科(Sex reassignment surgery,SRS),并在50年代后得以迅速发展。到70年代中期,西半球建立的SRS中心就达到40个。在不少国家,性别重塑手术被认为是常规手术,并纳入医疗保险范畴。不少国家成立了由整形外科学、心理学、泌尿外科学、妇科学、内分泌学、伦理学、社会学、法学等专业的专家组成的易性症治疗小组。我国第一例性别重塑手术是在1984年完成的。对于真正的易性症患者,性别重塑手术是最好的治疗手段,而药物和心理治疗并没有持久的帮助。国际性焦虑协会(Harry Benjamin International Gender DysphoriaAssociation,HBIGDA)于2001年2月在其最新版的《性身份障碍诊疗标准———第六版》(The standards of care for gender identity disorders—sixth version)中提出:“在任何意义上,性别重塑手术都不是一个‘实验性’、‘研究性’、‘选择性’、‘美容性’的手术。它对易性症和严重的性身份障碍是一项非常有效和适当的治疗方法。”


\section{第二节 发病机制及病因}

正常的性别身份发展过程的研究仍处于初级阶段。目前,已经比较明确的是:儿童的性别身份确定是在一个很短的时期内完成的。这个时期在2.5~3岁之间。这个时间一结束,儿童就能体验到自己的性别身份,而不论他(她)的生理性别发展到什么程度。对于这种现象尚无理想的解释。易性症的病因学研究目前也同样尚无定论。推测其发生与以下因素有关:

从胚胎学的角度看,胎儿的性腺结构在发生的初期是倾向于形成女性性器官—卵巢的,只是由于Y染色体的原因,才引起男性性腺—睾丸和雄激素的产生。胚胎6周时,在胎儿雄激素的作用下,女性生殖管道的前身—苗勒氏管退化,同时,外生殖器向男性发育,生殖结节伸长成为阴茎,两侧的尿生殖褶沿阴茎的腹侧面从后向前合并成管,形成尿道海绵体部,左右阴唇阴囊隆起移向尾侧,并相互靠拢,在中线处形成阴囊。在缺乏胎儿雄激素的情况下,胎儿的女性化倾向就保留下来,形成女性化表型。如果脑中通过激素而接受男性信息的部位发生障碍,不能接受这种信息,在男性的躯体内就会保留女性的成分。国外有学者用此观点解释性取向障碍和性别转换症的发生机制。

Sipova和Starka发现女性易性症患者的雄激素分泌过多,可能为宫内发育时激素异常而造成性身份障碍。在出生之后,雄激素改变不了正常成年女性的性取向,就会使她们中许多人增加性欲或使躯体男性化。

母子接触的持续,母亲的温柔,对儿童心理的健康发育是必不可少的。但是,如果接触过多,过于强烈,对男孩的心理发育是不利的;患者在心理上缺乏母亲的关爱,接受过强的父女接触,过多的父女联系,导致女孩发生大幅度的男性化并否认自己的女性身份;此外,患者的家长偏爱男孩,从小将患者以男孩打扮,抚养,从言谈举止到感情都倾向男性,久而久之患者便习惯于扮演男性角色;反之,父母或家庭把男孩当女孩对待,也会使此男孩产生女性变的倾向。目前,很多学者都认为:模仿混乱和童年异性角色行为的长期强化,是多数易性症患者发病的主要因素。

如患者的长相、性格、行为等方面自幼与异性相像,喜欢做异性做的事,周围的人往往称她为“娘娘腔”或“假小子”。从而产生变性思想;患者经历了情感挫折,特别是长时间无夫妻生活而产生性别转换的逆反心理,该患者一般年龄偏大,一但产生变性念头都较为强烈;也有人认为,成年后由于严重的病理变化或外伤破坏了生殖器官,或造成性激素分泌的紊乱,也影响到男子气或女子气,但不会使童年时已牢固建立的性别身份发生变化。


\section{第三节 临床表现}

幼儿期:男性患者均不喜欢粗野、攻击性游戏,表现胆小、温顺。女性患者则喜爱男性的粗野、攻击性游戏。

儿童期(6~12岁):男女患者均不能进入正常的“同性阶段”,即不喜欢与同性玩耍,其后也不能进入“仿同阶段”,即不愿模仿同性的长者,如同性的父、兄或母、姐的性别角色,相反却模仿异性长者的行为,如男孩学母亲做饭、做家务等,女孩则学父、兄耕地种田等。患者的气质、形象向异性发展。

青春发育前:男女患者开始持续、强烈地为自己的性别所苦恼,希望自己成为异性,穿着异性服装,关注异性的生活及活动方式,呈现所谓的“儿童性身份障碍”。

青春发育期:患者多从14~15岁开始形成异性性心理,为自己的心理与解剖生理、性征的不一致而十分痛苦,厌恶自己的性器官及性征发育,性心理指向同性。患者开始以异性心理、角色爱慕同性,以明显的异性角色进行社交活动及生活。

青春期后患者的性心理指向同性,爱慕同性,以致出现了同性恋行为。在同性恋行为中,患者均为异性角色。

由于患者对自己的性器官有持久而强烈的厌恶和不适感,导致焦虑和抑郁情绪。

第二性征及内、外生殖器均发育正常;男性有过遗精,女性月经来潮;除非服用激素,性激素检查也在正常范围。染色体均为正常46、XY或46、XX。

明尼苏达多项人格(MMPI)测查:男女患者均有突出的性心理异性化的特点,男性更为强烈,这与易性症患者的性变态心理相一致。男女性患者的D、Pd、Pa、Pt、Si五个临床量表分值均比常模分值高,且差异有显著性,说明这些患者都存在多相心理问题及程度不同的人格偏离。


\section{第四节 诊断与鉴别诊断}

(一)强烈而持久的相反性别识别(不仅为了想从成为相反性别获得文化上可得到的利益)。

儿童至少有如下4项表现:

1.反复说想成为相反性别或坚持认为自己是相反性别;

2.男孩爱穿女性服装或模仿女性打扮;女孩爱穿男孩惯有的服装;

3.在假装游戏中强烈地坚持扮演相反性别的角色,或坚持幻想成为相反的性别;

4.强烈地希望参与相反性别惯有的游戏和娱乐;

5.强烈喜爱相反性别的游戏伙伴;

少年或成人有如下表现:说自己希望成为相反性别,往往仿效相反性别排小便,希望像相反性别那样生活或他人将自己看做相反性别,或深信自己具有相反性别的典型感受和反应。

(二)持久地对自己的性别感到不舒服,或感到自己的性别角色不适当。

儿童有如下表现:男孩声称自己的阴茎或睾丸令人讨厌,它即将消失,或者声称最好没有阴茎,或者厌恶打闹的游戏并拒绝男孩惯有的玩具、游戏和活动;女孩拒绝蹲着小便,声称自己有阴茎或会长出一个阴茎,或者声称自己不想长乳房或来月经,或者厌恶规范的女性服装。

少年或成人有如下表现:全神贯注于去掉自己的第一及第二性征(例如,要求使用激素、手术或其他方法改变性征以模拟相反性别),或深信其生错了性别。

(三)身体并不同时存在雌雄同体。

(四)障碍引起具有临床意义的苦恼或者社交、职业或其他重要功能的损害。

1.期望作为异性的一员生活并被别人接受,常常伴有想通过外科手术或激素治疗而使自己的身体尽可能与自己所偏爱的性别相一致的愿望;

2.转换性别的心理持续存在至少两年;

3.不是其他精神障碍,如精神分裂症的症状,也不伴有染色体异常。

1.持久、强烈地为自己是女孩而痛苦,并宣称渴望是一个男孩(不仅仅是一种看到任何文化上的好处而成为男孩的愿望),或坚持她就是一个男孩。

2.具备(1)或(2)

(1)固执地表明厌恶标准的女性服装,并坚持穿着常规男性服装,如男孩的内衣和其他附属用品;

(2)固执地否定女性解剖结构,证据为下列至少一条:

1)断言她有,或将长出阴茎;

2)拒绝取蹲位排尿;

3)断言她不想乳房发育或月经来潮。

3.女孩还没到达青春期。

4.此障碍必须至少持续存在6个月。

1.持久、强烈地为自己是男孩而痛苦,并强烈渴望是一个女孩,或更为罕见地坚持他就是女孩。

2.具备(1)或(2)

(1)专注于女性常规活动,表现为偏爱女性服装或模仿女性装饰,或表现为强烈渴望参加女孩的游戏和娱乐活动,而拒绝男孩的常规玩具、游戏和活动;

(2)固执地否定男性解剖结构,必须为重复主张下列的至少一条:

1)他将长成为女人(不仅是角色方面);

2)他的阴茎或睾丸令人厌恶或将要消失;

3)最好没有阴茎或睾丸。

3.该男孩还没到达青春期。

4.此障碍必须至少持续存在6个月。

对自身性别的认定与解剖生理上的性别特征呈逆反心理,持续存在厌恶和改变自身性别的解剖生理特征以达到转换性别的强烈愿望,并要求变换为异性的解剖生理特征(如使用手术或异性激素),其性爱倾向为纯粹同性恋。已排除其他精神病所致的类似表现,无生殖器解剖生理畸变与内分泌异常。

诊断标准:

(一)期望成为异性并被别人接受,常希望通过外科手术或激素治疗而使自己的躯体尽可能与自己所偏爱的性别一致;

(二)转换性别的认同至少已持续2年;

(三)不是其他精神障碍(如精神分裂症)的症状,或与染色体异常有关的症状。

1.在衣着装束上,易性症虽喜作异性穿着打扮,但多穿戴如普通异性的日常便装。

男同性恋者往往浓妆艳抹,发式时髦,追求外围情调,以显示自己或吸引男人。

2.行为举止上,易性症患者不着异性装束时,若是男人,在行为举止上与正常男人一样。在公共场合,若不注意观察,很难发现他们会有性身份障碍。即使他们着异性服装,大多数人也不会改变自己的行为举止。他们不会像同性恋男人那样模仿女人的讲话和姿态,而许多同性恋男人不着异性装束时也表现得女里女气的,甚至扭捏作态。

3.在性别感上,易性症患者只存在与他们生物学性别不同的性别感,而几乎不表现异性的行为特征,除了有异性的性取向之外,他们往往爱好体育活动,不喜欢跳舞和表演,不注意化妆打扮,而且他们还很难学会正确的化妆和挑选合适的服装。同性恋的男人特别喜欢而且精于梳妆打扮,他们往往自己设计、自己裁制许多女装。

4.在职业选择上,易性症患者男人往往选择传统上由男性来做的工作,如工程师、机械工等,男同性恋者往往做理发师、美容师等。易性症患者通常不会像同性恋者那样公开跳舞或表演,而男同性恋的女子气愈浓,则愈喜欢这些活动。

5.同性恋者并不厌恶自己的女性解剖结构,不要求通过手术转换性别。

6.同性恋者会坚持上女厕所,也不厌恶每月的月经。

女性异装症患者主要是为了体验穿着男装所引起的强烈的性兴奋,其性意向仍为女性,并不要求手术变性。

(一)会同精神科医生与患者及家属会谈,以了解患者发病以来的性别认同与性别角色扮演的情况,评估其是否患有其他精神疾病或重大压力引起性变态的可能,了解生活状态、交友状况等社会适应情况,以及父母家人的态度。

(二)请妇产科和泌尿外科会诊,并检查性激素和染色体来筛选生理性别及性生理是否正常。

(三)安排相关的问卷调查或心理测试来做更加审慎的分析、评估与诊断,以了解其智力状况、人格结构、性格特征等。填写问卷与量表,包括:性知识量表、性态度量表、性偏好问卷、性别形容词检核表、性认同量表、多向度性量表、中国人健康问卷,以广泛了解其性别认同、性别角色、性欲取向等是否特殊,性偏好、性知识与态度并筛选其精神健康状态。

(四)精神科医生会诊—评估其精神状态,了解是否患有精神疾患。

(五)以变性欲症为架构,进行多次个别会谈,广泛了解家庭背景、发展过程、性发展史与性别困扰的历程,并对其社会适应进行深入探讨。

(六)分析评估会谈、问卷与检测结果,并与小组成员讨论,作出确切诊断。

(七)与患者和家属一起会谈,以家属的观点来澄清患者的发病史,并了解家属对易性症和性别转换手术的看法。当家属同意性别重塑手术时,和患者会谈,探讨其对性别重塑手术的了解程度、期待程度以及可能发生的问题和解决途径,并进一步明确诊断。


\section{第五节 手术适应证}

易性症的治疗十分棘手,一般应治疗患者的变态心理,使其能发挥正常的生理功能,患者往往自己服用或在医生指导下已服用性激素治疗,但心理治疗与药物治疗往往难以产生满意的疗效。患者在接受激素治疗一段时间后,会出现不同程度的异性化表现,如:女性会出现闭经、声音变粗、阴蒂肥大、乳房萎缩、皮肤粗糙、性欲增强以及髋部脂肪减少等。男性则出现乳房发育、胡须及体毛消退、皮肤细嫩等。药物副作用包括:长期服用睾酮类激素会导致患者不孕、痤疮、情绪不稳定、性欲增强、心血管疾病增加、肝功能异常、肝肿瘤发病率上升等。一但确诊为易性症,往往最后的转归是性别重塑手术。Hage对200名易性症女性患者进行问卷调查发现患者对男性性器官的需求期望值极高。Smith等人通过对20位手术变性患者和21位通过药物治疗的易性症患者1~7年的随访,发现手术治疗的患者的性心境恶劣状况得以逆转,而且精神与社会功能良好;没有进行手术的患者仍存在强烈的变性要求。

性别转换手术是不可逆的,因而施行手术之前要做详尽的检查,并严格掌握手术适应证。

国际性焦虑协会(HBIGDA)推荐的手术适应证为:

1.法定年龄的成年患者;

2.曾经连续接受激素治疗一年以上;

3.对自己的解剖生理性别感到不适应,并连续以异性身份在社会上成功地生活了一年以上;

4.如果心理医生要求,患者可以接受由患者和心理医生共同参与的贯穿于整个真实社会生活的精神治疗,但这种精神治疗并不是手术的前提;

5.充分了解和认同手术的费用、住院时间、术后并发症、术后康复等诸多问题;

6.由多学科医生共同讨论并达成共识。

由于变性手术涉及重大伦理问题,风险较高,其安全性、有效性尚需经规范的临床试验研究进一步验证,卫生部将其列为“第三类医疗技术”即“需要卫生行政部门加以严格控制管理的医疗技术”,并于2009年11月正式颁布了技术管理规范:

卫办医政发〔2009〕185号

变性手术技术管理规范(试行)

为规范变性手术技术审核和临床应用,保证医疗质量和医疗安全,制定本规范。本规范为技术审核机构对医疗机构申请临床应用变性手术技术进行技术审核的依据,是医疗机构及其医师开展变性手术的最低要求。

本规范所称变性手术,是指通过整形外科手段(组织移植和器官再造)使易性癖病患者的生理性别与其心理性别相符,即切除其原有的性器官并重建新性别的体表性器官和第二性征。

(一)医疗机构开展变性手术技术应当与其功能、任务相适应。

(二)三级甲等综合医院或整形外科医院,有卫生行政部门核准登记的整形外科诊疗科目。

(三)医院设有管理规范、运作正常的由医学、法学、伦理学等方面专家组成的变性手术技术临床应用伦理委员会。

(四)整形外科。

1.设置整形外科10年以上+,床位20张以上,有较强的整形外科工作基础。

2.能独立完成整形外科各种手术,包括器官再造和组织移植。

3.病房设施便于保护变性手术患者隐私和进行心理治疗等。

(五)有至少2名具备变性手术技术临床应用能力的本院在职医师,有经过变性手术相关知识和技能培训并考核合格的、与开展的变性手术相适应的其他专业技术人员。

(一)手术组由整形外科医师为主组成,必要时可有其他相关科室医师参与。

(二)手术者 取得《医师执业证书》的本院在职医师,执业范围为整形外科,具有副主任医师及以上专业技术职务任职资格;从事整形外科临床工作10年以上,其中有5年以上参与变性手术临床工作的经验,曾独立完成10例以上的生殖器再造术。

(三)第一助手 从事整形外科临床工作5年以上的整形外科医师,或者其他相关科室具有主治医师以上专业技术职务任职资格的医师。

(一)遵循整形外科以及相关学科诊疗规范和技术操作常规。

(二)变性手术的实施顺序:生殖器的切除、成形是变性手术的主体手术,任何改变第二性征的手术必须在性腺切除之后或与性腺切除术同期进行。

(三)手术前要求患者必须提供的材料和应当满足的条件:

1.手术前患者必须提交的材料

(1)当地公安部门出具的患者无在案犯罪记录证明。

(2)有精神科医师开具的易性癖病诊断证明,同时证明未见其他精神状态异常;经心理学专家测试,证明其心理上性取向的指向为异性,无其他心理变态。

(3)患者本人要求手术的书面报告并进行公证。

(4)患者提供已告知直系亲属拟行变性手术的相关证明。

上述材料须纳入病历资料。

2.手术前患者必须满足的条件

(1)对变性的要求至少持续5年以上,且无反复过程。

(2)术前接受心理、精神治疗1年以上且无效。

(3)未在婚姻状态。

(4)年龄大于20岁,是完全民事行为能力人。

(5)无手术禁忌证。

(四)实施变性手术前,应当由手术者向患者充分告知手术目的、手术风险、手术后的后续治疗、注意事项、可能发生的并发症及预防措施、变性手术的后果,并签署知情同意书。

(五)医院管理。

1.实施变性手术前须经过医院和伦理委员会同意,获准后方可施行。

2.完成每例次变性手术的一期手术后,将有关信息按规定报送至相应卫生行政部门。

3.性腺切除后,送病理检查,其他组织视情况送病理检查。

4.变性手术后,医院为患者出具有关诊疗证明,以便患者办理相关法律手续。

5.医务人员应尊重患者隐私权。

(六)开展变性手术的医疗机构应建立健全变性手术后随访制度,按规定进行随访、记录。

(七)医疗机构和医师按照规定定期接受变性手术技术临床应用能力审核,包括病例选择、手术成功率、严重并发症、死亡病例、医疗事故发生情况、术后患者管理、患者生存质量、随访情况和病历质量等。


\section{第六节 手术方法}

男性易性症的性别重塑手术包括三个主要步骤:

1.睾丸、阴茎切除;

2.阴道成形:方法包括应用肠袢或腹膜作腔壁;皮片或羊膜植入作衬里;皮瓣移植作衬里等。还有一种简易阴道成形术,即利用外阴局部皮肤相对缝合构成阴道,但宽度与深度均难以满足生理要求。

3.隆乳、喉结整形等:隆乳目前采用的方法为胸大肌后硅胶囊假体置入。切除过多的甲状软骨来求得颈部女性外观。

女性易性症的性别重塑手术包括以下几个主要步骤:

1.子宫及双侧卵巢切除术;由妇科医师协助完成。

2.乳房过多组织切除术;睾丸假体植入术;阴囊成形;嘴唇增厚术;减少髋部、大腿、臀部脂肪等整形手术等。

3.阴茎成形术;阴茎成形是这一性别重塑手术中最为复杂和关键的步骤。成形后的阴茎需满足以下要求:

(1)外观上提供一个美的男性特征;

(2)阴茎内尿道的延续;

(3)为适应术后性生活所需的阴茎挺拔;

(4)感觉存在。

目前采用的阴茎成形术多用腹部皮管、腹部岛状皮瓣、游离皮瓣(前臂皮瓣、背阔肌皮瓣等)和腹股沟皮瓣等方法,手术通常需分几期完成(具体参见其他章节)。由于再造的阴茎并无性感觉和自主勃起功能,与正常阴茎相比,其外形也差强人意,所以,国外有报道,相当一部分患者完成了第1期手术后,并不要求行阴茎再造术,而采用其他替代方法(如性玩具等)。我们在第1期手术时,利用阴道前壁和尿道板重建、提升尿道至耻骨联合下方,并同时将阴蒂转移至其上方,此期手术,即可达到站立排尿的目的,而且也为将来可能进行的阴茎再造作好了受区准备。

4.阴蒂阴茎化手术(Metoidioplasty) 对于某些易性症患者不想接受阴茎再造这种复杂手术,只是想形成一个男性化的外观,并达到站立排尿的目的,则可施行“阴蒂阴茎化手术”。需要指出的是:这种手术形成的“阴茎”对于完成真正意义上的性交是不足够的,有的还需要行阴茎再造术。手术将阴蒂向前、上方移位,用大阴唇再造阴囊,延长尿道来完成,最终的结果是把女性的外生殖器转变成更像男性的外观。具体方法介绍如下:

(1)术前准备

1)常规的病史采集、物理检查和心理测验。

2)术前进行1年以上的睾酮替代治疗,使现有的肥大阴蒂,在以后可利用局部组织转移来形成新阴茎。

3)妇科手术可以和阴蒂阴茎化手术同时实行。阴道前壁可用于尿道成形。

(2)麻醉和体位:手术通常采用全身麻醉,膀胱截石位。

(3)主要手术步骤

1)围绕尿道外口和阴蒂切开,分离并切断阴蒂悬韧带,使阴蒂延长(图19-1(1))。

2)将阴蒂掀起,分离阴蒂与尿道口之间尿道板(图19-1(2))。

3)同时行妇科手术时,切除的子宫可由阴道拖出,保留部分阴道前壁,将阴道前壁与尿道板缝合形成部分尿道(图19-1(3))。

图19-1 (1)

图q19-1 (2)

图w19-1 (3)

4)沿阴蒂背侧平行切开,形成阴蒂背侧皮瓣,游离阴蒂周围皮肤,使阴蒂完全伸直(图19-1(4))。

5)在皮瓣根部筋膜打孔,将延长的阴蒂由孔中穿出,此时背侧皮瓣即被移转至腹侧(图19-1(5))。

6)取颊黏膜游离移植至阴蒂腹侧,然后将皮瓣皮面向内与颊黏膜缝合形成第二部分尿道,吻合重建的两段尿道(图19-1(6))。

图e19-1 (4)

图r19-1 (5)

图t19-1 (6)

7)切开阴蒂,将尿道包埋其中。将阴蒂周围的皮瓣进行适当剪裁、转移环行包绕延长的阴蒂体部形成“新阴茎”(图19-1(7))。

8)利用两侧大阴唇作阴囊成形(图19-1(8))。

图y19-1 (7)

图u19-1 (8)


\section{第七节 性别重塑手术的预后}

男性转变为女性的手术效果,每个人都不太相同。虽然所有的人都能体会到色欲感,某些人体验到的性乐高潮与过去差不多;某些人感到的性乐高潮更为弥漫;也有的人性乐高潮更差了,影响了性的满足。女性转变为男性手术更复杂,她们不可能完全隐瞒自己原先的性别,尤其是术后不可能有一个良好功能的阴茎,但随着阴茎再造技术的日趋完善和生物材料的进步,如三件套阴茎植入体等,已给将来可能的婚姻生活带来希望。因国内性别重塑手术起步较晚,目前尚无大样本、较长时间的随访研究。国外有研究表明:即使术后有很多问题,易性症患者仍觉得新的生活给他(她)带来了欢乐,有些新的家庭还相当幸福美满。

Zielinski报道了位于Lodz的SRS中心从1983年到1997年里252例女性易性症,其中209例进行手术治疗,分别采用双蒂腹部皮管、单蒂脐下皮瓣、带蒂股薄肌肌皮瓣和带蒂腹股沟皮瓣四种皮瓣。从整体上看,69.8\%患者手术结果良好。Barrett和Kuipe对40例已行性别重塑手术的患者进行随访表明97\%的患者对术后外观表示满意。Rakic等人对32例性别转换术后患者随访6个月到4年,分别从自己对生活的态度、与他人的关系、性生活情况和工作情况四个方面进行调查,表明这些患者术后生活质量均得到明显改善。


\section{第八节 性别重塑手术的伦理学问题}

性别重塑手术作为对易性癖患者采取的一种特殊治疗手段,常引发其是否违背医学目的的争议。

性别重塑手术最大的利就是能满足易性癖患者的心理需求,符合医学伦理对患者有利的原则。随着医学模式从单纯生物医学模式到生物-心理-社会医学模式的转变和生命伦理学的发展,医学的目的不仅仅是防治疾病,延长人类寿命,更注重患者生命质量的提高,即不仅使患者拥有健全的躯体,而且保持良好的心理和社会适应。易性症又称性身份转换症,是性变态中的一种性心理障碍,更确切地说属于性身份认同障碍。正常人都有一个明确的性别,心身一致地相信和认同自己是男或女。易性症患者的躯体上也有明确的性别,而且发育正常,却不认同躯体生就的性别,并在此基础上萌发易性欲望。目前对其病因、发病机制尚无公论,通常5~6岁萌发易性心理,以后欲望逐渐强烈,有的乔装打扮,模仿异性行为方式;患者自感痛苦万分,不能自拔,不仅影响求学、就业、成家,甚至出现自残或自杀现象。虽然一般医学学者主张用心理治疗配合行为疗法,但多无持久疗效,有的虽经过了6、7年的心理治疗,易性者仍矢志不移。因此国外大多数专家认为,对于真正的易性症者,性别重塑手术是目前最好的治疗手段。易性症患者术后常具有极大的满足感和喜悦,感激地将手术医师视为再生父母,对生活和未来也充满信心和希望。

性别重塑手术毕竟是一种创伤性而且不可逆的手术,不仅患者身体上要受到可能不止一次的创伤,而且变性后患者也不是真正意义上的异性,由于没有产生相应异性性激素的性腺及输精管、子宫,不可能有月经或遗精等现象;再造的性器官也不可能有正常人的性感受;除非使用克隆技术,变性人也不可能拥有自己遗传学后代;另外,变性后患者为了从外观上更接近其要求的性别,必须长期使用外源性激素,常引起恶心、头昏甚至血栓形成、乳房肿瘤等问题,并且对内分泌产生影响。即使移植异性内生殖器官,术后患者必须长时间使用免疫抑制剂,而且,此方法的远期效果尚未得到证实。

性别重塑手术因治疗周期长、需多次手术、医疗费用较高,一般家庭难以承受。从患者角度考虑,以这种昂贵的经济、躯体损伤甚至残缺为代价换取一种单纯的心理满足,是否得不偿失?从社会角色考虑,在我国目前卫生资源普遍紧缺的情况下,此手术是否合适?

随着现代医学技术的进步,医学不仅可攻克许多疑难损伤病症,对人类自身的生命控制也趋于随心所欲,性别重塑手术对传统的生命、生活家庭、亲情观念、国家的人口法规造成了极大的冲击,进而引发了许多社会伦理甚至法律问题,使得变形人的社会适应不容乐观。

性别、容貌是重要的个人特征,具有重要的社会意义。人类性别是由分别来自父母双亲的一对染色体决定,XX型为女性,XY型为男性,胎儿期即分别具有女性、男性内外生殖器官,出生时一般按外生殖器确认性别,进入青春期后受性激素影响,逐渐长出女性、男性的第二性征,体态也有明显性别差异,因此从外形、声音多可辨认性别。变性人虽可具有异性部分甚至全部外形,但遗传基因毕竟难以改变。那么变性人到底属于男性还是女性?或第三性?我国目前尚无法律承认性别重塑手术后的性别。一些地方的公安部门只是根据实际情况,出于保护变性人的隐私权,尊重他们的个人意愿、方便他们今后的生活,对其必备的证件上的性别做了必要的修正,但周围亲友和社会各部门未必能适应他(她)们,以致变性者随时可陷入尴尬、受排斥和歧视的孤立困境。

目前尚无法律对变性人的法律权益作专门规定,如变性人术前术后从法律意义上讲是否同一人:性别重塑手术受医学技术水平、医疗条件限制及患者自身生理状况、经济承受能力影响,不能保障每例手术都能达到患者要求,一旦产生分歧,是否算医疗事故?如果是,又如何补偿?从医学上讲,变性人可以以异性身份寻找生活伴侣,但变性人组成家庭后,如有纠纷或离异等,将给法院等部门带来许多新的问题。

上述问题,必然使变性人肉体上、精神上衍生更多的不幸,最终可能完全失去性别重塑手术解除易性症患者精神痛苦的医学意义。所以性别重塑手术并不单纯是一般医学整形手术,其社会学、伦理学、宗教学、法学意义和问题,远远超出医学家们的设计和想象。诸多问题都有待各学科通力合作,深入进行探讨研究,那么,是否就让易性症患者继续忍受心灵的煎熬,而不进行有效的干预呢?临床学家确为此陷入尴尬的境地。


\section{第九节 易性症的预防}

因为易性症的病因不明,而且一经发现,即已进入青春期,对于如何预防,目前也只有以下几点可以认同:

人在婴儿阶段,便面临了一个重要的性问题:性别身份的确定。这个问题在人生的头两、三年里,同母子关系的性质密切相关。为了防止性身份障碍的发生,必须建立起恰当的母子关系,尤其是在这个关键性的成长阶段,以保证性心理的正常发展。

在与婴儿的接触过程中,既要避免接触过少,也要避免接触过分。“过犹不及”这个道理应当牢记。正如需要建立母子联系来完成孩子的性别认同一样,逐步地解除这种联系也十分必要。只有这样,才能使孩子能够顺利地脱离母亲的直接庇护,建立自己独立的人格。因此,孩子稍大一些,就应当适当地少抱一些,多让孩子自己活动一些,以便孩子独立化和正常成长。

对男孩来说,这种与母亲相对脱离的状态可以为他提供对父亲进行认同的机会。如果男孩与母亲之间的“共生”关系延续过长,则会阻止这种认同过程。假如家庭中没有父亲,应当为男孩提供一个替代性的男性进行接触,使男孩有一个性别认同的条件。

除了家庭,整个社会环境对儿童性别身份的确定也有影响。但社会对儿童的影响基本上是通过父母对待孩子的态度及处理问题的一贯方法来实现的。孩子的健康成长要以双亲能提供健康的指导为前提。因此父母应该以一种始终如一的方式来保证孩子的行为与其性别相适应。孩子只有在和谐的家庭气氛中才能不断地调整自己的行为特征,以符合自己的性别身份。

(朱辉)

1.张庆国,李琳,陈志瑾.易性癖患者性别重塑手术前后的心理活动与行为浅析.中华医学美学美容杂志,2001,7:256

2.崔玉华,任桂英,方明昭,等.易性癖54例MMPI测试结果分析.中国心理卫生杂志,1998,12:138

3.Harry Benjamin International Gender Dysphoria Association.The standards of care for gender identity disorders(6th version).http://www.hbigda.org/soc.c

4.马晓年.现代性医学.第2版.北京:人民军医出版社,2004:734-755

5.American Psychiatric Association.Diagnostic and Statistical Manual of Mental Disorders(4th ed.):Washington,DC:1994

6.中国精神障碍分类与诊断标准.第3版.济南:山东科学技术出版社,2001:135-136

7.Smith YL,van Goozen SH,Cohen-Kettenis PT.Adolescents with gender identity disorder who were accepted or rejected for sex reassignment surgery:aprospective follow-up study.J Am Acad Child Adolesc Psychiatry,2001,40:472-481

8.Hage JJ,Amber Y.Goedkoop,Refaat B.Karim,Robert C.J.Kanhai.Secondary corrections of the vulva in male-to-female transsexuals.Plast Reconstr Surg,2000,106:350-359

9.Jamil Rehman,Arnold Melman.Formation of neoclitoris from glans penis by reduction glansplasty with preservation of neruovascular bundle in Male-to-female gender surgery:function and cosmetic outcome.J Urol,1999,161:200

10.Robert C.J.Kanhai,J.Joris Hage,Henk Asscheman,J.Wiebe Mulder.Augmentation mammaplasty in male-to-female transsexuals.Plast Reconstr Surg,1999,104:542-549

11.Robert C.J.Kanhai,J.Joris Hage,Paul J.van Diest,Elisabeth Bloemena,J.Wiebe Mulder.Castration and estrogen treatment on breast tissue of 14male-to-female transsexuals in comparison with two chemically castrated men.The American Journal of Surgical Pathology,2000,24:74

12.Akoz T,Kargi E.Phalloplasty in a female-to-male transsexual using a double-pedicle composite groin flap.Ann Plast Surg,2002,48:423-427;discussion 427

13.Chang TS,Huang WY.Forearm flap one-stage reconstruction of the peins.Plast Recinstr Surg,1986,74:251-258

14.Meyer R,Daverio PJ.One stage phalloplasty without sensory deprivation in famale transsexuls.World J Urol,1987:9-13

15.Van Borsel J,De Cuypere G,Rubens R,Destaerke B.Voice problems in female-to-male transsexuals.Int J Lang Commun Disord,2000,35:427-429

16.Hage JJ,Dekker JJ,Karim RB,Verheijen RH,Bloemena E.Ovarian cancer in female-to-male transsexuals:report of two cases.Gynecol Oncol,2000,76:413-415

17.Lebovic GS,Laub,DR,Ozek G,et al.Metoidioplasty//Ehrlich RM,Alter G.Reconstructive and plastic surgery of the external genitalia.Philadelphia,W.B.Saunders,1999,355-360

18.Hage JJ,Metaidoioplasty.an alternative phalloplasty technique in transsexuals.Plast Rconstr Surg,1996,97:161-167

19.Perovic SV,Djordjevic,Metoioplasty.a variant of phalloplasty in female transsexuals.BJU Int.2003,92:981-985

20.Zielinski T.Evaluation of surgical flaps used for creation of an artificial penis in female-male type transsexuals.Pol Merkuriusz Lek,2001,10:27-30

21.Barrett,J.Psychological and social function before and after phalloplasty.The International Journal of Transgenderism,1998,1:28

22.Rakic,-Z;Starcevic,-V;Maric,-J;Kelin,-K.The outcome of sex reassignment surgery in Belgrade:32patients of both sexes.Arch-Sex-Behav,1996,25:515-525

23.Chiland C.The psychoanalyst and the transsexual patient.Int J Psychoanal,2000,81:21-35

24.Gallarda-T,Amado-I,Coussinoux-S.The transsexualism syndrome:Clinical aspects and therapeutic prospects.Encephale,1997,23:321-326

25.Justine M,Schober.Sexual behaviors,Sexual Orientation and Gender identity in adult intersexuals:A Pilot study.J Urol,2001,165:2350-2 353

26.Kuiper,B.&Cohen-Kettenis,P.Gender role reversal among postoperative transsexuals.International Journal of Transgenderism,1998,2:1-16

27.Mary M.Moebius.Gender identity disorder and psychosexual problems in children and adults.Journal of the American Academy of Child &Adolescent Psychiatry,1998,37:337-338

28.Meningaud JP,Descamps MA,Herve C.Sex Reassignment surgery in France:Analysis of the legal framework and current procedures and its consequences for transsexuals.Med Law,2000,19:827-837

29.Peggy T.Cohen-Kettenis,PhD;Stephanie H.M.van Goozen,PhD.Sex reassignment of adolescent transsexuals:a follow-up study.Journal of the American Academy of Child &Adolescent Psychiatry,1997,36:263-271

30.Yolanda L.S.Smith,Stephanie H.M.Vangoozen,Peggy T.Cohen-Kettenis.Adolescents with gender identity disorder who were accepted or rejected for sex reassignment surgery:A prospective follow-up study.Journal of the American Academy of Child &Adolescent Psychiatry,2001,40:472-481

31.舒玲华,李文刚.性别重塑手术的伦理学思考.中国医学伦理学,2001,5:22

(朱辉)


\chapter{第二十章 乳房整形与美容}

女性乳房是一功能器官,它被喻为生命之源泉;女性乳房也是一形体器官,是女性形态美最显著的标志。丰满而有弹性的女性乳房,是女性妩媚、青春、活力、自信的象征。对女性乳房美的追求,是人的天性,也是人的本能。

女性的乳房为半球形或水滴形,位于上胸部,由乳腺的皮肤、乳腺、筋膜、乳头及乳晕所构成。乳房在锁骨中线上位于第3~6肋骨之间,或是第2~6肋间隙之间,内起胸骨旁,外达腋前线。

乳房的血液供应主要有三个:胸廓内动脉、胸外侧动脉和肋间动脉的穿支。乳房的感觉神经分布:第3~6肋间神经的外侧支,为乳房的支配神经。分布到乳头乳晕的是第4肋间神经的外侧皮神经前支。


\section{第一节 小乳症}

小乳症特指由于先天发育不良或不发育所导致的乳房过小。

胸部平坦,乳房欠丰满,缺乏弹性,无乳沟,缺乏曲线美,失去正常的形态。

根据临床表现和体征可明确诊断。需与哺乳后乳腺萎缩、外伤、炎症及腺体的破坏所导致的乳房过小相鉴别。

可通过药物、锻炼、局部按摩及手术进行治疗。手术具有见效快,立竿见影的效果。

手术适应证 小乳症及单纯为美容目的要求做隆乳手术者,乳房不对称,单侧乳房发育不全,乳房缺损等。

手术禁忌证 乳房有炎症、肿瘤;受术者心理不正常;精神病患者;其他同常规手术禁忌证。

一般采用硬膜外麻醉,也可选用全身麻醉。隆乳术如选用乳晕或乳房下皱襞切口可应用局部麻醉。

站立位与平卧位时乳房有明显的移动,且乳房内、外侧皮肤的弹性差异较大,因此乳房假体植入的定位设计要考虑到这些变化因素。具体设计方法:首先于立位时标记出两侧乳房下皱襞连线相交于胸骨中线的点;麻醉后患者仰卧、双臂外展90度,再自此点垂直于胸骨中线向两侧划一水平线,以此线作为乳房假体植入的下界,胸骨中线旁2cm为其内界,标记出一大于乳房假体直径3~4cm的圆形乳房假体植入区。

(1)真皮脂肪瓣移植:近年来通常用于乳癌根治术后以及乳腺腺病行乳腺单纯切除的患者。

(2)游离真皮脂肪瓣移植:因其吸收率可高达30\%~50\%,且易发生脂肪液化、组织坏死等诸多并发症,近年来很少有人使用。

(3)颗粒脂肪注射移植隆乳术。

是一种有机硅化物聚合体,为黏度极高的聚二甲基硅氧烷,它具有以下性质:①不受软组织干扰;②无化学活性;③不产生炎症反应或异物反应;④无致癌性;⑤无变态反应或过敏反应;⑥能形成理想的形状;⑦可消毒,耐高温高压。医用硅凝胶乳房假体的表面为一完整的硅橡胶囊,按其内囊中内容物不同可分为硅凝胶充填型和盐水充填型等;依照使用方法可分为注入用型和植入用型(一般硅凝胶充填型称植入型,盐水充填型称注入型);按照硅凝胶囊表面特性分为:光滑面型和毛面型;按照硅橡胶囊囊腔多少分为:单囊型、双囊型以及多囊型。

常用的切口有腋窝横皱襞切口、腋窝前皱襞切口、乳晕下切口及乳房下皱襞切口。过去尚有腋前线切口,目前已很少采用。隆乳术的手术切口一般有如下三种:(见图20-1)

图20-1 手术切口

切口位于腋窝顶部,平行于腋窝皮肤皱纹,该切口最为隐蔽。切口的长度根据不同种类的乳房假体而定。一般为3~5cm。

沿切口切开皮肤及皮下组织,顺乳头方向剥离深面的结缔组织至胸大肌后间隙(患者有乳腺萎缩及轻度下垂时,可在乳腺与胸大肌之间隙剥离)。此时可用隆乳术剥离器按设计的乳房假体植入范围在胸大肌后间隙处分离出一腔隙。如剥离到胸大肌在Ⅴ,Ⅵ,Ⅶ和Ⅷ肋的起点处受阻,可将这部分胸大肌起点离断。

充分止血后,植入乳房假体于胸大肌后腔隙(或将假体置于乳腺与胸大肌之间隙)。

缝合切口时一定要在直视下确定不会扎破假体后逐层缝合切口。术后加压包扎。

切口设定在乳晕的内下方与皮肤的交界处以避免损伤乳头乳晕的感觉神经。切口长度3~4cm。该切口小,而乳晕皮肤呈褐色,又有结节状乳晕皮脂腺掩饰,所以瘢痕不明显。

沿切口切开皮肤及皮下组织至乳腺表面。此时有两种选择:第一,放射状切开乳腺达胸大肌表面,切口长度4cm。第二,沿乳腺表面剥离,绕过乳腺到胸大肌表面。

后沿胸大肌肌纤维方向分开胸大肌至胸大肌后间隙,再用隆乳剥离器按设计的乳房假体植入范围分离胸大肌后间隙。其他操作同腋窝入路。

切口位于乳房下皱襞中央。切口长度可较其他切口略短,一般为2~4cm。该切口较隐蔽,与皮肤纹理基本一致,切口瘢痕不明显,不损伤乳腺组织及重要神经血管;切口入路手术操作比较简单,切开皮肤、皮下组织后可直接与胸大肌后间隙剥离。其他手术操作同腋窝入路。

脂肪颗粒注射移植隆乳术

用脂肪抽取器从腹部或大腿抽取脂肪,将抽取的脂肪收集到经消毒的容器中,将较大块的脂肪剪碎,再经漂浮沉淀法过滤、分离、去除杂质,即可提出纯净的脂肪。再用2cm口径的注射针头将脂肪颗粒均匀地注射到乳房腺体表面和基底。脂肪颗粒注射移植术后6~12个月吸收率为30\%~60\%,故过度矫正50\%是必要的。但移植的脂肪颗粒过多,会导致脂肪细胞液化,坏死。故通常每侧乳房每次脂肪颗粒注射的量为100ml左右,术后3个月可再次注射,可反复注射3~4次。

隆乳术后切口用纱布包扎,乳房四周垫敷料,使乳房固定并塑形,再适当加压包扎。包扎的压力及方向可根据手术的具体情况加以调整。一般经腋窝入路的手术包扎的压力可偏重于乳房上部,其他的入路方法应均匀包扎。包扎的目的是为了压迫止血和防止假体移位,因此包扎可持续3~7天。隆乳术后可放置引流管,于术后48小时拔除,手术后7天拆线。隆乳术后应做乳房按摩(具体时间根据假体性质、患者体质及手术医生经验等而定)。

(1)病因:剥离的腔隙不充分。

乳房假体纤维囊挛缩。术后血肿、感染、异物、患者的特异性体质、病毒或无临床症状的细菌感染、迟发性乳房假体周围小血管破裂等,都可能是潜在的乳房假体纤维囊挛缩的病因。

术后无有效的乳房按摩。

(2)诊断:隆乳术后乳房硬化的分级标准:

Ⅰ级:不能扪及乳房假体,乳房柔软,如同没有手术的正常乳房。

Ⅱ级:略可扪及乳房假体,外形正常,患者无不适感。

Ⅲ级:可扪及乳房假体,乳房假体中等硬度,患者有感觉,乳房形态有变化。

Ⅳ级:乳房高度硬化,双侧不对称,移位。乳房外形明显异常。

(3)处理方法

预防:手术要严格执行无菌操作程序;严格按解剖层次剥离;剥离的腔隙必须充分;严格止血;术后常规抗感染治疗;均匀的加压包扎;应用创面引流措施;术后按摩。

治疗:乳房硬化一旦发生,手术治疗是唯一有效的方法。手术切口采用经乳晕下切口,通常需要反射状切开挛缩的纤维囊,扩大假体植入腔隙,必要时切除纤维囊及更换假体。术后常规放置皮片引流。加压包扎需持续7天。术后加强按摩。

常见乳房假体在乳房的上极、乳房外侧和乳房的下极。偶见在乳房的内上方。

(1)病因:乳房假体植入位置设计有误;剥离腔隙的范围有误;加压包扎方法有误;假体纤维囊挛缩等等。

(2)诊断:视诊即可作出诊断。病因诊断需结合病史具体分析。

(3)预防:正确设计乳房假体植入位置;严格按照设计剥离腔隙;均匀加压包扎;严格执行预防纤维囊挛缩的措施;限制上肢过度活动2周。

(4)治疗:术后早期发现可通过局部加压调整等手段修正。如无效需再次手术处理。根据情况可取出或不取出假体。手术的内容是要按正确的假体植入位置重新剥离腔隙。已形成纤维囊的需切开松解或切除,按需要扩大假体植入腔隙。术后常规引流。特别需要对那些异常的植入部位做限制性加压包扎。

(1)病因:术中损伤血管,创面止血不彻底,术后加压包扎不妥,患者凝血系统有异常,术后上肢过度活动,过早按摩乳房等。

(2)诊断:切口持续渗血,切口周围皮肤有淤血斑。患侧乳房逐渐肿大,局部皮肤有明显青紫色淤血斑,患者有胀痛或跳痛感等。

(3)预防:避免月经期手术,严格检查受术者凝血功能,术中严格止血,术后加压包扎,限制受术者上肢活动,禁止早期按摩乳房等。

(4)治疗:即刻清创止血,创面引流,加压包扎,术后应用止血药,应用抗生素防止继发感染。

临床常见有急性感染,通常在术后几周内发生,偶见迟发性感染,可以在手术几个月后出现。

(1)病因:污染是急性感染常见的病因,它可以来自术前、术中和术后各种途径。迟发性感染的病因则比较复杂,可能与病毒或条件致病菌的存在有关。

(2)诊断:急性感染术后几周内手术区域出现红、肿、热、痛,即可明确诊断,可以伴有体温升高。迟发性感染一般在术后几个月后逐渐出现手术区域的红、肿、热、痛,通常不伴有体温升高,或仅有全身低热。

(3)预防:严格掌握手术适应证,一切与手术区域接触的物品、器械和乳房假体均应严格消毒灭菌,严格术前、术中和术后的无菌操作,术后使用抗生素预防感染等。

(4)治疗:感染早期应立即使用大剂量广谱抗生素1~2周。如局部皮肤出现波动感则需立即切开引流。严重的感染通常需取出乳房假体,清创引流,3~6个月后可再次行隆乳手术。

(1)病因:切口感染裂开是假体外露的常见病因。由于切口缝合层次少或因其他原因引起的切口愈合不良也可导致乳房假体外露。

(2)治疗:因感染造成的假体外露,应按抗感染措施治疗。因切口愈合不良导致的假体外露除对症治疗外,可试行重新缝合创口。如重新缝合创口失败,则需取出假体,待创口愈合3个月后,再次手术修复。

(1)病因:各种原因可导致假体损伤,如乳房假体的质量。假体破漏的高危因素有假体充盈不足、假体囊内使用抗生素或激素、假体的品牌等。

(2)诊断:早期2~7天内患侧乳房体积迅速减小,数周或数月,甚至数年后患侧乳房体积在数天内逐渐减小。盐水充注式乳房假体渗漏,无明显的临床症状。硅胶假体破裂临床上常有患侧乳房红、肿、热、痛等刺激症状,并可形成多个团块。

(3)预防:术前严格检查假体质量,避免术中任何尖锐器械刺破假体,适当的假体充盈度,剥离腔隙要充分,使假体充分舒展。

(4)治疗:一旦发生假体渗漏应立即取出假体。盐水充注式乳房假体取出后可以即刻行乳房假体再植入手术。如是硅凝胶的漏出,需要尽可能地清除干净,创面引流,创口愈合至少3~6个月后考虑再次行乳房假体植入手术。


\section{第二节 巨乳症(巨乳缩小成形术)}

乳房的过度发育使乳房的体积过度增大,形成乳房肥大症(mammary hypertrophy,macromastia),俗称巨乳症。乳房肥大的确切病因目前还不十分清楚。乳房肥大常常在不同程度上伴有乳房下垂,也给许多女性带来精神上及肉体上的痛苦。

乳房肥大可分为3类:乳腺过度增生性乳房肥大、肥胖型乳房肥大及青春型乳房肥大。

表现为乳腺组织过度增生,肥大的乳房坚实,乳腺小叶增生明显,常有压痛。在月经周期期间,常常有自发性疼痛,并伴有乳房下垂,较多发生于已婚育的妇女。严重的病例,由于乳房的赘生及经久的胀痛,给患者带来心理上及肉体上的痛苦。

表现为整个乳房匀称的肥大。在组织结构上,是以乳房中的脂肪匀称增生、脂肪细胞肥大为主;在手术中可发现乳房皮下有脂肪增生,在乳腺组织之间,也有脂肪增生及浸润。这类乳房肥大的患者常伴有全身性肥胖,肥大的乳房虽可能伴有不同程度的乳房下垂,但较乳腺过度增生性乳房肥大为轻。

是一种在青春发育期发现的乳房渐进性增大,并过度发育,乳腺组织增生、肥大。乳房表现为匀称性肥大,乳房下垂不明显,这类患者有时有家族史。

乳房肥大常见于以下情况,一种是伴发于全身性肥胖,另一种是青春期乳房肥大,偶见于多次生产后继发的巨乳症。此外乳腺腺病也可导致乳房肥大。上述情况均为手术适应证。

原则是使巨大及下垂的乳房经过手术以后,达到外形及功能良好的目的。

图20-2

巨乳缩小成形术通常选用硬膜外麻醉或全身麻醉。手术范围较小的可用局部麻醉。

新乳头乳晕的定位设计

患者取坐位,设计乳头、乳晕的上移位置,用美蓝标出标。定新乳头位置主要有两种方法:第一种:乳房下皱襞水平位置参考法。患者于坐位或站立位,此时新乳头的水平位置应位于乳房下皱襞在乳房表面的投影上2~3cm,并于锁骨中点与乳头的连线相交。第二种:锁乳线及胸乳线参考法。患者坐位或站立位,从锁骨中点向下至乳头作一连线,再从胸骨上切迹中点作一长20~24cm的直线,此直线远端与锁乳线相交,相交点即为新乳头位置。新乳头的水平位置大致相当于患者上臂中点的位置。(图20-2、图20-3、图20-4)

图20-3

图20-4

此切口因不能切除水平方向的多余皮肤,目前很少使用,一般用于轻、中度乳房肥大的乳房基底盘形切除术。

可切除水平和垂直方向多余的皮肤,切口缝合后形成倒“T”形,是目前广泛采用的切口。其中Wise模板和Duke钢丝环可作为倒“T”形切口设计的辅助工具确定所需切除的乳房组织量,确定缝合新乳晕的切口周长和弧度,确定新乳晕至乳房下皱襞的长度。

术后切口缝合形成“L”形。其优点是避免了倒“T”形切口胸骨侧的瘢痕,切口设计隐蔽。但因设计复杂,未被广泛采用。

手术设计及手术操作

1.定位新乳头 患者站立位,以胸骨上切迹为中心,半径19~24cm,向下方划弧。再以剑突为中心,半径11~13cm划弧,两弧相交点为新乳头位置。以原乳头为中心,直径3~5cm划圆,作为新乳晕的界限。

2.切口 自新乳头定点向下,绕过原乳头划线垂直向下至乳房下皱襞中点。切开皮肤至皮下组织,完全暴露乳房腺体,仅乳房后壁与胸大肌相连。

3.切除乳腺组织 右侧乳腺的外侧做“S”形切除,左侧乳腺的外侧做反“S”形切除,将乳头乳晕保留在剩余的腺体上。

4.将剩余的腺体向外上方旋转,腺体间相互缝合成一圆锥形。

5.将乳房皮肤向下牵拉,自然覆盖新塑形的腺体,切除多余的皮肤,使垂直的切口位于乳晕的正下方,水平切口位于新的乳房下皱襞。

6.将新乳头乳晕自垂直切口的顶端拉出,切除多余的皮肤,创缘与乳晕缝合。

7.术后放置引流,24~48小时后拔除。术后加压包扎,9~10天拆线。

手术设计及手术操作

图20-5

改进术式(Pitanguy):考虑到Strǒmbeck氏手术乳房上极空虚的缺点,改进术式(Pitanguy)保留了乳房下极的脂肪组织。在形成的水平双向真皮腺体蒂携带乳头乳晕向上移转时,将脂肪组织置于真皮腺体蒂的下部,改善了乳房上极空虚的缺点。

1.定位新乳头乳晕 按锁乳线和胸乳线法确定新乳头乳晕的位置。

2.切口设计 须用Wise模板设计切口,并将模板两臂的长度定为5cm。

3.将新乳头乳晕部位的皮肤、皮下组织和腺体切除。

4.将新乳头乳晕下两臂至乳头乳晕间的皮肤去表皮。

5.切除去表皮以下至乳房下皱襞间的全部皮肤、皮下组织和腺体。保留直径4~4.5cm的乳晕。

6.将所形成的水平双向真皮腺体蒂携带乳头乳晕向上移转至新乳头乳晕位置。

7.缝合切口。所形成的缝合口为倒“T”形。

8.术后处理同前。

在手术过程中,为了使乳晕、乳头顺利移至适当的位置,经常需切开该蒂的近中和侧方的真皮层。而支配乳晕区感觉及乳头勃起功能的第4肋间神经纤维即在真皮深层呈网状分布。手术切断真皮层时,则部分或全部切断该神经纤维。致使术后相当一段时间内乳晕、乳头感觉功能以及乳头的勃起功能丧失,恢复期长,这是该法的缺点。

手术设计及手术操作

1.新乳头乳晕的设计采用乳房下皱襞参考法,切口用Wise模板法。

2.去除下部蒂范围的表皮,保留4.5cm直径的乳晕。

3.按设计去除内、外和上三部分的乳房组织。

4.将下部蒂携带乳庆乳晕移转至新的位置。缝合切口。

该术式是在垂直双蒂瓣的设计基础上去除双蒂的乳头乳晕以上部分,并包括新乳头、乳晕位置的皮肤、皮下、乳腺组织。对于乳房重度下垂,体积较大的巨乳,手术选择下蒂瓣法比较有效。这一术式操作简单,容易掌握,术中只需维持两侧乳房下蒂瓣的组织量一致,即可获得术后两侧乳房对称,并且该术式在切除乳房组织的量上有很大的灵活性,可以使乳头乳晕上移幅度很大且血供较为可靠,移至新位置的乳头乳晕松弛自然、感觉功能不受破坏或能早期恢复。

图20-6

改进术式:其蒂的基底为锥形,厚度为10cm,乳头乳晕部位的腺体厚度为5cm,成为下部锥形真皮腺体蒂(Robbins).这样的改进使乳头乳晕的血运更丰富,术后乳头乳晕的凸度更良好。

图20-7

垂直双蒂乳房缩小整形(McKissock法)是由McKissock(1972)报告的。其手术内容是:乳头、乳晕皮下蒂移植,其蒂设计在垂直上、下方;乳房肥大的中下部皮肤切除,应用乳房外侧皮瓣及内侧皮瓣转移,修复缺损;乳腺部分作中下部楔形切除,然后作乳腺在胸肌筋膜下部分游离,旋转作对合,使乳腺成锥体形组织块。该手术采用了Wise氏的标准模型作术前设计,可操作性强,易于掌握。术中只切除垂直双蒂瓣下部的两侧过多的腺体组织,没有过多的皮瓣、腺体的游离和推移,操作容易、简单。由于此种术式容易掌握、精通和灵活运用,扬长避短,此种术式成为适合术者的最佳术式,并能取得满意的疗效。该术式符合乳房解剖学原则,乳头乳晕的血供可靠。手术中保留了乳腺的上部和两侧的大部分及垂直双蒂的形成,充分保证了乳腺乳头、乳晕的血供。设计具有可靠的解剖学基础,即保证了乳头乳晕的血供,手术操作也较简便,是治疗乳房重度肥大下垂的较理想选择术式,遗留垂直切口疤痕为其不足。目前多采用改良术式,其乳头和乳晕垂直上、下蒂的术式被多种乳房缩小整形术所采用。

手术设计及手术操作

1.新乳头乳晕的定位与切口设计同Strombeck手术。

2.从新乳头乳晕两侧垂直向下至乳房下皱襞划两条线,作为垂直双向蒂瓣的界线。蒂的宽度不得少于5cm。

3.将垂直双向蒂瓣的部位去表皮,保留4.5cm直径的乳头乳晕。

4.将垂直双向蒂瓣两侧的乳房组织包括皮肤、皮下组织和腺体全部切除。

5.将垂直双向真皮腺体瓣与胸大肌分离,携带乳头乳晕向上移转至新乳头乳晕位置。

6.缝合乳头乳晕和其他切口。缝合切口呈倒“T”形。

7.术后处理同前。

是一种切口瘢痕较小的乳房缩小整形技术,手术后仅留有乳房下半一条直线瘢痕及一条短横瘢痕。该术式还设计了一种乳房肥大组织量切除的预测方法。这是综合了Strombeck法及乳晕周乳房缩小整形手术优点的一种术式。

Marchac法可用于轻、中度乳房肥大及下垂的整形,对于重度乳房肥大也可应用此法,但乳头、乳晕移植应采用垂直双蒂移植。熟练掌握此方法即掌握了一种良好的手术操作技术。

手术设计

受术者取坐位,绘制手术切口设计。

在离胸骨中线10cm处,通过乳头中点,绘制一垂直线,上达乳房上方,向下超越乳房下皱襞中点到季肋缘。

将乳房上推,绘出乳房上皱襞的界限。

将乳房推向外侧,在乳房内侧绘出与季肋部乳房中轴相连的垂直连线;将乳房推向内侧,在乳房外侧绘出与季肋部乳房中轴相连的垂直线。乳房内侧及外侧的垂直连线之间,是乳房多余皮肤切除的界限。

乳房内、外侧切口线平行下降至乳房下皱襞上方约5cm处,画一平行于乳房下皱襞的弧线,与内、外侧两线相交。将内、外侧切口线靠拢,确定切除的范围不会切除其张力。

在内、外垂直线间画一圆,作为乳头、乳晕周去上皮组织的范围。乳房内侧及外侧垂直线与乳房下方弧线的距离约为5cm,这是乳房下方直线瘢痕较短的依据。

手术操作

1.作乳头、乳晕周围皮肤切口,并去上皮。行下部皮肤切除,作乳房缩小整形。遇有乳房下垂的病例,则宜保留下部皮肤,仅作去上皮处理。

2.在乳房下部水平线外侧,作深部乳房下部组织切除,深处可达胸肌筋膜表面,深层上界可达原先确定的乳房上界。

3.在内侧及外侧垂直切口处,切除内侧及外侧乳腺。上界切除线是乳晕下方2cm的水平线,在服发明家下向上延伸,类同Pitanguy技术,使中部乳腺组织切除达到预先估测的切除量。

对于巨大的乳房肥大的病例,在保留乳腺中柱部分周围作乳腺组织的切除,预防切除过多的乳腺组织。

4.用2-0可吸收缝线在胸大肌筋膜表面及乳腺组织后方作乳房悬吊,直达原先确定的乳房上界(位于乳晕的稍上方)。乳腺组织固定后使两侧的乳腺下半部分游离、缝合,一个半球形的乳房体即形成。

5.缝合乳晕周围的皮肤。对已设计的5cm的垂直皮肤切口作皮下缝合,即可见形成新的乳房下皱襞线。切除乳房下方过多的组织及脂肪,矫正猫耳畸形。用可吸收缝线缝合皮下及皮内,乳晕周围用5-0可吸收缝线缝合,用5-0不吸收缝线作皮内连续缝合。

是一种术后乳房下方为直线瘢痕的乳房缩小整形技术,是Dar-tigues的改良术式。

本手术在欧洲受到广泛推荐。采取乳房蘑菇形切口,以乳头、乳晕上方为去上皮的皮瓣蒂,行中部乳腺组织切除、乳房下部及乳晕周围的皮肤皮下组织切除。乳房形态的整形主要靠乳腺组织的再塑形,手术后显示乳房下方不平整,但手术后远期效果良好。Lejour有一千余例的临床经验报告。

该手术方法既可用于轻、中度乳房肥大,也可用于重度乳房肥大。乳房缩小整形效果的评价有4方面:形态良好,两侧对称;瘢痕细小;乳头感觉良好;尽可能于术后保留泌乳功能。Lejour曾随访了170例手术患者,仅有1例乳头感觉丧失、7例感觉减退。

该手术的原则有3点:①广泛的乳房下部皮肤及皮下组织分离,减少皮肤缝合张力,减少瘢痕;②畸形矫枉过正,以便取得较好形态;③作脂肪抽吸,去除不必要的组织,便于乳房缩小的塑形。

对于乳房的脂肪抽吸,Lejour描述了好几种优点:使乳房软化便于成形,有利于乳头、乳晕长蒂移植;有利于保护乳房的血管、神经及实质组织;能减少缝合皮肤张力;当术后患者减肥时,不致发生乳房下垂等。

手术步骤包括:手术设计;皮下浸润注射血管收缩药物;乳头、乳晕皮瓣蒂部去上皮;脂肪抽吸;手术切除乳房皮肤和部分乳腺组织及再塑形。

手术设计及手术操作

1.绘制乳房中轴,同Marchac法;乳房上界的确定,同Marchac法;乳房切除范围的预测,同Marchac法。

在乳房内侧及外垂直线确定后,将两线在乳房下皱襞上方相交成一弧线。

乳晕上方的切口设计线位于新乳头上方2cm处,新乳头位置确定方法参见前述的“乳房缩不整形的基本技术”。从此点出发,在乳房内、外侧各绘一弧线,相交于两垂直线,相交点的位置,根据乳房大小而变化。

2.皮下浸润 取半卧位手术,麻醉后在乳房下部作0.5\%利多卡因加1∶10万肾上腺素20ml局部浸润,以减少手术过程中出血,对巨大乳房则用40ml的0.5\%利多卡因浸润。笔者认为采用0.25\%的利多卡因浸润较妥。

3.乳头、乳晕皮瓣蒂去上皮 从乳头、乳晕上部设计线到其下2cm区域去上皮。

4.脂肪抽吸 在乳房下部切口线上方作一小切口,用6mm三孔钝头脂肪抽吸管,在乳房上部、内侧及外侧进行抽吸。

5.手术切除及再塑形 沿着切口线切开皮肤,必须保护乳头、乳晕蒂不受破坏。为此,宜在乳头、乳晕去上皮皮瓣的皮肤留0.5cm厚的脂肪,乳腺部分切除如同皮下乳房切除的进路一样。切除乳房下中部的乳腺组织。切除范围向上方到第3肋间处,即乳房上界画线处,在胸肌筋膜表面切除。如果是巨大的乳房,乳腺实质的切除包括乳头、乳晕下的乳腺组织,其乳头、乳晕带蒂移植的蒂可长达10~12cm,将乳腺组织悬吊缝合,用缓慢吸收缝线,缝合两侧及上部乳腺组织,矫正乳房下垂。行皮下、皮内皮肤缝合,矫正乳房下皱襞的猫耳畸形(图20-8)。

图20-8

图20-9

手术设计及手术操作

1.在新设计的乳晕周围去表皮。

2.自乳晕去表皮的部位将皮肤与腺体分离,皮瓣的厚度至少要有2cm。并且在距胸壁2cm处停止剥离。

3.自乳晕去表皮的部位开始在乳腺的周围切除多余的腺体,需注意的是,腺体切除应距胸壁3~4cm开始,目的是保护进乳腺的主要动脉。

4.腺体切除后再根据实际需要切除内、外和下方多余的皮肤和皮下组织。

5.乳头乳晕最后整形。

吸脂乳房缩小术适用于乳房肥大的患者,特别适用于术后欲生育的妇女及男性乳房肥大。吸脂乳房缩小术的手术方法一般是使用传统的负压吸脂器。手术切口通常选用乳房下皱襞或乳晕切口。有文献报道,从乳房吸出的脂肪量可达350~2250ml,乳头乳晕可上提3~11cm。术后没有乳房感觉障碍,术后瘢痕细小。乳房皮肤的弹性回缩在术后3个月基本稳定,其术后效果良好。

新乳头乳晕的定位与切口设计同Strombeck手术和Mckissock手术一样,同样使用了Wise模板技术。其他的手术方法也同Mckissock手术相似。所不同的是该术式切除了乳晕以下的腺体,仅保留了乳晕以上的真皮腺体蒂瓣,成为单向上部真皮腺体蒂。蒂瓣的宽度限制在6~7cm。该方法改善了Mckissock手术乳房下极臃肿的缺点。

图20-10

图20-11

环形切口乳房缩小整形是一种以矫正乳房下垂为主的乳房缩小整形技术,包括Himderer及Hester Bostwick法。该手术方法切口瘢痕小而隐蔽。Himderer法是环乳晕切口、乳房下垂的矫正术式,也可用于轻、中度乳房肥大;而Hester Bostwick法,则是环乳晕切口及类似Lejour法切口,该手术方法切口瘢痕小而隐蔽。

手术切口:在乳晕外围,乳头、乳晕的血供在位于中央部分的乳腺上,或包括乳晕周围相连的环形去上皮的皮肤蒂上,所以手术后乳晕周的去上皮区域应作荷包样缝合。

手术设计:先作乳头定位,估计乳房皮肤需切除的范围及乳房上界的位置(参见前述的“乳房缩小整形的基本技术”及“Marchac法”)。手术切口形为环形,或Lejour法切口。

手术操作:在乳晕周围做皮肤切口,乳晕周围去上皮,或乳晕保留在中央乳腺上,周围由皮下进入乳腺,切除过多的乳腺,周围作较广泛的皮肤分离。作下垂乳腺组织悬吊,然后缝合皮肤。

该术式采用乳晕周双环切口,外环可根据情况设计成圆形或椭圆形,切口行荷包收拢缝合,减少了切口张力,基本保留乳房浅、深层血管供应,并保证乳晕形态大小稳定,对乳房单纯下垂及轻中度巨乳效果较好。本方法的优点有:切口设计简单,该方法较其他三种方法均为简单,且手术步骤简化、损伤小、不影响泌乳功能,手术切口隐蔽,瘢痕不明显,使部分巨乳情况不是很严重且审美要求高的患者易于接受。但它不适合于重度巨乳的修复,且由于组织切除量不多可能导致术后乳房呈圆盘形,乳晕增宽凹陷或乳晕皮肤周围留有皱褶。

此种方法术后瘢痕在乳晕较隐蔽,但是在去除皮肤及腺体有一定局限性,因此不太适合于重度乳房肥大特别是下垂较严重的患者。同时由于双环法主要去除乳房内、外、上象限的腺体而不能去除相应的皮肤,因此外形不如垂直双蒂丰满挺拔。此法较适合于轻中度乳房肥大且以腺体增生为主且皮肤松弛不明显的年轻患者。

图20-12

“真皮帽”技术(dermal voult technique),其主要理论之一是:皮肤是悬吊乳房腺体唯一有效的组织结构,而Cooper韧带是联系皮肤与腺体的基本解剖结构。因此Ladardrie手术主要方法是保留乳头乳晕周围的真皮组织以及真皮下至少1cm厚的腺体,按需要切除深层的腺体,靠真皮间的相互粘连悬吊乳房腺体(保留Cooper韧带)。该术式主要适用于轻度乳房肥大及单纯乳房悬吊。

上述是国内外常用的乳房缩小术,尽管乳房缩小术有几十种术式,每种术式有其各自的特点,但是,直到目前还没有一种手术方法能够解决乳房肥大的所有问题。对乳房缩小术而言,重要的是需要美容外科医生领会每种术式的要领,尽可能熟练地掌握几种术式,并能够灵活运用于各种条件的乳房肥大患者。在此基础上才能不断发现新的问题,提出新的解决办法。


\section{第三节 乳头内陷矫正}

乳头内陷的病因有先天性和获得性两种。先天性乳头内陷主要是因为乳头乳晕的平滑肌和乳腺导管发育不良,缺乏乳头搏起的动力和结构所导致。临床上常表现为双侧乳头内陷,且程度较重。获得性乳头内陷常因感染、创伤等导致乳腺导管周围瘢痕形成而出现程度不同的乳头内陷。乳头内陷多半是先天性畸形,也可能因外伤、炎症、肿瘤及手术造成乳头内陷。严重乳头内陷难以让婴儿吮吸乳汁,从而给患者带来心理上的压抑或生活上的不便。乳头内陷,常为双侧性,两侧凹陷程度不一,也可以是单侧性的。

从病情可以分类为:

轻度乳头内陷:乳头颈存在,在受到冷或触摸刺激时乳头可搏起突出于乳晕表面,能轻易用手使内陷乳头挤出,挤出后乳头大小与常人相似。

中度乳头内陷:在向外牵拉乳头时乳头可短暂维持凸起状态。乳头较正常的小,且多半没有乳头颈部;

重度乳头内陷:乳头完全埋在乳晕下方,向外牵拉乳头,松开后乳头立即回缩。或无法使内陷乳头挤出。

松解引起乳头内陷的纤维束,必要时切断部分或大部分短缩的乳腺导管;乳头颈部紧缩;乳头基底支持组织重建;术后作一定时间的乳头牵引,防止乳头内陷的复发。

采用1\%的利多卡因加1∶100 000的肾上腺素局部浸润麻醉。

在乳头至乳晕的中部对称地设计4个菱形切口。每个菱形切口最宽处应位于所设计乳头的基底处。4个菱形切除缝合后所形成的乳头基底周长要等于或略大于乳头周长。在局部浸润麻醉下按设计切除乳晕皮肤,去除乳头内乳腺导管周围皮下组织。用3-0丝线缝合切口。该方法适用于轻度乳头内陷。(图20-13)

图20-13

与乳头基底部乳晕菱形切除的原理相似,仅乳晕皮肤去除的形状不同。必要时乳晕外围的正常皮肤也需做三角形切除,意在避免产生皮肤皱褶。该方法适用于乳晕较大的轻、中度乳头内陷。

其主要特点是需彻底地松解牵拉乳头的纤维结缔组织,切除乳头下乳腺导管外周的所有结缔组织,用乳晕皮瓣重塑乳头颈部。主要适用于中、重度的乳头内陷。也适用于乳晕较小者。(图20-14)

图20-14

将内陷乳头挤出,用1号线在乳头上方缝合两针,使乳头牵引出体表。沿乳头中央横轴处切开乳头,切口线部分进入乳晕区。沿上半及下半个乳头的乳腺管周围彻底分离乳头纤维束,使之剪断,放置乳头牵引线,测试内陷的乳头是否得以矫正。对Ⅰ型乳头内陷,经过这样分离、切断纤维束后,再加上外翻的缝合,常能矫正乳头内陷畸形;但对Ⅱ、Ⅲ型乳头内陷,则还需进行下面的几种改良。分别在2点、8点部位的乳晕下方设计乳腺组织瓣,约0.8cm×0.8cm×1.5cm蒂在乳头部位,切断一半或大部分乳腺导管,其中一块乳腺组织瓣翻转充填空虚的乳头,另一组织瓣充填剪断乳腺导管后留下的乳头颈空虚区。在乳腺瓣供区的空虚处作乳头颈部的紧缩缝合,防止移植的乳腺瓣回纳;或采用乳头颈部荷包口缝合,缩窄乳头颈部。用胶布固定乳头牵引线,防止乳头回纳内陷。术后1周拆线,去除乳头牵引线。(图20-15)

图20-15


\section{第四节 乳头乳晕重建}

采用1\%的利多卡因加1∶200 000的肾上腺素局部浸润麻醉。

皮瓣的蒂在中央,供瓣区创面自行上皮化形成乳晕。三瓣法术再造的乳头凸度不足。

图20-16

(1)手术设计:以新乳头中心O点为圆心,画直径40mm的圆,以O点为中心做一2cm长的直线。在此直线两端各做一舌形瓣为C瓣和D瓣,其底宽1cm,高1cm。再在舌形瓣外侧各做一长2cm,宽1cm的矩形瓣为A瓣和B瓣。

图20-17

(2)操作:按设计线切开皮肤及皮下组织,然后将两个皮瓣完全掀起,此时将(1)中所述直线可分成为L和L'线。将两个皮瓣旋转对合,两瓣的基底缝合后形成乳头的基底。A瓣与L'线对接缝合;B瓣与对位皮瓣的L线缝合固定形成乳头柱,C瓣与A瓣顶端缝合,D瓣与B瓣顶端缝合,形成圆锥形乳头的顶部。在上述乳头柱形成过程中,原设计的圆已经变形,再以直径40mm重画新圆。将乳头基底与新圆周边之间的表皮全部切除,取阴股沟部全厚皮片移植修复创面,形成乳晕。

小阴唇游离移植。不足之处是小阴唇组织有限,所以移植的乳头小而平坦。

健侧乳头乳晕复合组织平分,将健侧乳头乳晕分为左右两个半圆,取一个半圆游离移植到缺损再造乳头乳晕,余下半圆剥离后拉拢缝合。将两个半圆重塑为两个全圆,外形稍差,供区剥离过多易出现皮肤坏死。

其他游离组织移植再造乳头的方法还有耳垂,足趾趾腹等方法。

腋下皮管法和带脐的腹部皮瓣一期乳房乳头再造法。在某些特定情况下,这两种方法可以采用。

作为一种辅助性措施,一些作者曾采用真皮、瘢痕组织、软骨植入、硅胶块置入等方法增加乳头的凸度。真皮和瘢痕组织植入后,后期收缩,质地硬韧,外形亦不佳。硅胶块置入,远期渐向基底深方陷入,较大的软骨块植入后也将发生相似的变化,凸度终难保持。

以皮片(大腿内侧断层皮片,耳后全层皮片以及腹股沟部全层皮片)、小阴唇和健侧乳晕游离移植三者应用最多,效果亦较好。

乳晕再造主要在于颜色的近似。在再造乳头及其周围刺入棕褐色颜料形成乳晕。

皮片移植后,再造的乳晕颜色深度不够,采用紫外线照射两周,皮片颜色明显加深,能获得预想的效果。


\section{第五节 乳房再造术}

乳房形态不良可造成女性心理上的压抑和缺陷,而乳房的缺失,则更易导致女性形体、精神上的创伤。这类精神上的创伤,通过乳房再造(breast reconstruction)可以给予不同程度地弥补。

乳房再造及乳房缺失的原因,最多见的是乳房良性或恶性肿瘤切除后乳房缺失;也可能因为外伤及烧伤,造成乳房缺失;亦有先天性发育不良造成一侧乳房缺失或两侧乳房缺失需进行乳房再造;异性癖的患者,由男性变为女性,也需进行女性乳房再造成。

乳房缺损常见于乳腺肿瘤行乳腺切除胸大肌根治术后,也见于乳房的烧伤、外伤、感染以及乳腺先天发育不良等。上述病因导致的乳房缺损均可为手术适应证。

术区准备同常规外科手术的准备。

外伤性乳房缺失、先天性乳房发育不良性乳房缺失,宜等待女孩至发育年龄时进行再造;变性术后乳房再造时机的选择,随受术者身体及心理准备的情况而定。

乳腺癌乳房切除后的乳房再造可即刻施行,也可在第一次手术后3~6个月后进行二期乳房再造,即在完成化疗后进行。如果是乳腺癌手术后需进行放射治疗的患者,则宜在停止放疗后6~12个月后进行,待放疗后皮肤及皮下瘢痕软化后,或“趋于软化”时进行。

所有乳房再造的患者,特别是乳房癌术后的患者,必须是身体健康、情绪稳定,没有精神及心理障碍,没有癌症复发的危险,而且对侧乳房是健康的,没有恶性肿瘤的情况下进行乳房再造。

1.乳房再造首先要解决皮肤缺失的修复。皮肤缺失的修复方法可应用组织扩张器,使皮肤扩张,增加皮肤的面积;采用局部皮瓣转移修复,包括上腹部逆行或旋转皮瓣移植;采用腹部皮瓣或皮管转移、背阔肌肌皮瓣移植、腹直肌肌皮瓣移植,以及显微外科游离皮瓣移植等。

2.在乳房皮肤修复的同时,或修复之后的一定时期要进行乳房半球形形态的塑造,包括应用肌皮瓣移植、假体移植等。

3.乳腺癌根治术后常伴有腋窝前壁缺失及锁骨下空虚区域,需进行畸形的整形,常可用肌皮瓣移植进行修复。

4.乳头及乳晕的再造。

5.修正双侧乳房的不对称性。

背阔肌肌皮瓣再造乳房的优点:它能提供足够的体积和厚度以维持植入的乳房假体的稳定。由于背阔肌肌皮瓣的丰富血液供应,且背阔肌肌皮瓣本身容易形成,也不会遗留形态和功能方面的问题,所以背阔肌肌皮瓣特别适用于乳癌根治术后患者的乳房再造,这些患者多数接受过放射治疗,常伴有深度的溃疡或肋骨坏死。某些乳癌根治术后的患者需要用乳房假体植入再造乳房,而背阔肌肌皮瓣能提供足够体积的厚度以维持植入乳房假体的稳定。但是利用背阔肌肌皮瓣时,尤其是对乳癌根治术后的患者,术前最好要检查是否保留了背阔肌的营养血管胸背动静脉。对乳癌根治术后立即进行乳房再造,或接受过大剂量放射治疗的病例,由于侧支循环不好,可能出现意外的失败。

背阔肌肌皮瓣乳房再造,背阔肌因其具有血供恒定,可供切取面积大,易于切取,供区隐蔽等优点,被广泛应用于外科乳房再造。研究表明,背阔肌肌皮瓣乳房再造的患者,一期乳房再造的乳腺癌患者的5年生存率同未行乳房再造者相似。经随访结果统计显示,再造组和根治组患者的5年生存率、局部复发率及远处转移率无显著性差异。

背阔肌是人体最大的阔约肌,主要血供是胸背动静脉,有同名神经伴行。背阔肌因其具有血液供应丰富、血管分支恒定、容易切取、可供切取面积大、切取转移后不影响背部的外形和功能、切口隐蔽、易于患者接受等优点,所以在外科乳房再造中得以广泛应用。背阔肌肌皮瓣虽然具有操作简单、安全、背部疤痕隐蔽、能填充锁骨下缺损及形成乳房腋皱襞的优点,但不能提供足够组织量,因此与假体结合使用比较常用,背阔肌肌皮瓣的缺点是背部有新的瘢痕,部分患者同侧举手力量降低,这是使用假体带来的一系列并发症。

应用解剖(图20-18)

图20-18

背阔肌起自胸腰筋膜浅叶,4~8胸椎及全部腰骶椎的棘突、髂嵴,斜向外上方行走,最后止于肱骨结节间沟,血供为多源,但主要来自于肩胛骨下动脉的胸背动脉,该动脉向下越过大圆肌沿背阔肌前缘深面前锯肌浅层向下行走,于肩胛下角平面上方入肌,入肌后血管分为内、外两支,外侧支距该肌前缘2~3cm处下行,内侧支分出后横行与肌肉上缘平行向内行走,其他血供则来自于相应的肋间动脉背侧支。

背阔肌的神经支配主要为胸背神经,支配肌肉运动,其皮肤感觉主要为胸背神经的背侧支。

手术设计与操作

(1)体位:侧卧,上臂外展,肘关节屈曲。

(2)设计:可按设计要求设计成轴型肌皮瓣、岛状瓣,一般多以前支动脉为轴心,或按解剖标志点即:以胸肱联合下方1.5cm处肩胛下动脉为上点,以背阔肌在髂嵴附着处为下点,两点之连线为该肌皮瓣的轴线;肌皮瓣的前缘应当于背阔肌前缘,后缘可按需确定,一般肌皮瓣的长度可达10~12cm,宽度可达18~20cm。

(3)操作:先自腋下沿背阔肌前缘切开皮肤、皮下组织,显露腋窝脂肪组织及背阔肌外缘。用手指钝性分离出前锯肌的肌肉间隙,向上分离开脂肪组织,显露出位于血管神经门内的血管神经束。而后自背阔肌前缘后钝性分离背阔肌。按设计需要切取肌皮瓣。

按设计切除受区的瘢痕、分离皮肤,将背阔肌肌皮瓣移转到胸前,如移植的背阔肌肌皮瓣不够丰满,可在背阔肌肌皮瓣下植入乳房假体。受区创缘与背阔肌肌皮瓣缝合,供区缺损可直接拉拢缝合或游离植皮。(图20-19)

应用解剖(图20-20)

图20-19

图20-20

腹直肌起于剑突及第5~7肋软骨,止于耻骨联合及耻骨嵴。腹直肌有前后鞘在腹直肌正中融合形成白线,将腹直肌分成左右两部分。腹直肌前鞘完整,后鞘在脐下5.8cm处形成半环线,半环线以下无后鞘。腹直肌有3~4个腱划。

腹直肌的主要营养血管有腹壁上动静脉和腹壁下动静脉。腹壁上动脉为胸廓内动脉的乳房内动脉的分支在剑突与肋缘间自腹直肌后鞘进入腹直肌。腹壁下动脉为髂外动脉的分支,血管外径2.5~3.4mm,有同名静脉伴行。腹壁上动静脉和腹壁下动静脉发出肌皮穿支分布到腹直肌表面的皮肤。此外腹壁浅血管、第一腰动脉的前支皮支等血管也供应、营养腹直肌表面皮肤。腹壁上动静脉和腹壁下动静脉在腹直肌肌体的后方相互吻合。以其任意一方为蒂就能够营养腹直肌全体。

腹直肌肌皮瓣乳房再造,因具有携带组织量大,腹壁供区瘢痕隐蔽,同时具有腹壁整形功能的优点,已经逐渐成为国内外多数整形外科医师进行乳房再造的首选术式。以其组织量大、血运良好,且同时有腹壁整形的效果,特别适合于中年、腹部已膨隆的患者。不足的是由于血供有限,一般需要去除IV区皮瓣,否则易出现局部皮瓣坏死、脂肪液化;另外,由于要带走部分腹直肌,所以对肌肉的损伤很大,术后易出现腹壁薄弱、腹壁疝等并发症。

禁忌证:①消瘦或未生育过的患者;②没有足够腹壁组织的患者;③以前腹部做过手术、胸廓内动脉或腹壁上动脉进入腹直肌肌皮瓣的血供受阻断者;④肋缘下和旁正中切口是带蒂腹直肌肌皮瓣转移的禁忌证;⑤以前的阑尾切除术切口,下腹部横行切口和腹股沟部的手术切口使游离的腹直肌肌皮瓣的应用成为禁忌;⑥腹壁脂肪抽吸术后。

手术设计与操作

(1)单侧单蒂腹直肌肌皮瓣(图20-21)

图20-21

1)设计:首先按健侧乳房的位置和皮肤表面设计供瓣区皮肤的表面积和形状以及受区的移植部位。然后设计对侧腹直肌为蒂携带下腹壁皮肤脂肪瓣的腹直肌肌皮瓣。

2)操作:沿皮瓣设计线切开皮肤皮下组织显露腹直肌前鞘和腹外斜肌浅层,后自皮瓣上缘腹壁深筋膜层向上分离到剑突下5cm,显露双侧腹直肌前鞘和腹外斜肌浅层。在对侧腹直肌前鞘外缘和腹白线外侧1cm处平行于腹直肌长轴切开腹直肌前鞘,并保留部分腹直肌前鞘于腹直肌上。游离下腹部皮肤脂肪瓣,保持对侧腹直肌与皮瓣的接触。在半环线处试阻断腹壁下动脉,验证皮瓣血供良好后切断腹壁下动静脉,自腹直肌后鞘游离腹直肌蒂至剑突下5cm。形成完整的腹直肌肌皮瓣。

自剑突皮下做一隧道至对侧乳房缺损区。将腹直肌肌皮瓣自隧道穿出,将腹直肌固定在胸壁上,皮瓣与乳房缺损区周围皮肤缝合。用7号丝线缝合剩余的腹直肌前鞘,缝合腹壁切口,重建脐孔。

如有乳头乳晕的缺损,可同期行乳头乳晕再造。

3)术后术区应置引流条48小时。加压包扎。术后10天拆线。

(2)双侧双蒂腹直肌肌皮瓣:单侧腹直肌蒂携带的皮肤脂肪瓣的面积是有限的,特别是跨越腹壁中线的皮瓣常出现血液循环障碍。如在再造乳房的同时需修复大面积胸壁瘢痕,可选用双侧双蒂腹直肌肌皮瓣,以携带更大面积的皮瓣。手术方法基本同单侧单蒂腹直肌肌皮瓣的方法。但要特别注意腹壁成形,防止腹壁疝。(图20-22)

(3)双侧单蒂腹直肌肌真皮脂肪瓣:双侧单蒂腹直肌肌真皮脂肪瓣主要用于同时填充因双侧乳腺腺病单纯切除乳腺后的腺体缺损。其手术方法基本同单侧单蒂腹直肌肌皮瓣的方法。所不同的是同侧的单蒂腹直肌肌皮瓣填充同侧的乳房,所携带的皮肤脂肪瓣需去表皮。

臀大肌的解剖 臀大肌位于臀部的后面,起于髂骨外面的后部及骶骨、尾骨、腰背筋膜等处,肌纤维斜向外下,至股骨大转子附近形成腱膜,部分止于大转子下方,部分借髂胫束移行于阔筋膜,再向下达胫骨上端。

臀大肌的血供主要来自臀上动脉和臀下动脉。臀上动脉是髂内动脉最大的分支,自梨状肌上缘穿出,分为两支:浅支直径2.3mm,在臀大肌深面入肌,供给臀大肌及其表面的脂肪和臀上部的皮肤。深支在臀中肌深面行走,供应该肌。臀上动脉的浅支有两条伴行静脉,其血管外径约2~3mm。

臀大肌肌皮瓣再造乳房是以臀上动脉浅支为血管蒂,携带上部臀大肌及其皮肤皮下组织游离移植。手术需将臀大肌肌皮瓣的臀上动、静脉与受区的胸廓内动、静脉吻合。

臀大肌肌皮瓣游离移植法要求技术较高,目前较少应用。(图20-23)

图20-22

图20-23

适应证 一侧乳房缺失,健侧乳房有足够的体积和一定程度的下垂,乳头乳晕较大,且健侧乳房无器质性病变。

手术设计与操作 以健侧乳房的内侧1/2为供瓣区,其主要的血供为胸廓内动脉。沿设计画线将健侧乳房切开成内外两部分,形成乳房腺体皮肤组织瓣。而后外侧瓣向内上旋转180度,分层缝合腺体、皮下组织和皮肤形成缩小的乳房;内侧瓣以乳房上端皮肤和腺体基底为蒂向对侧乳房缺损区移转,逐层与制备好的缺损区创面缝合。术后3周断蒂,再次行乳房整形,乳头乳晕移转。(图20-24)

选择乳房癌改良根治术后的患者,其乳房有皮肤残留并存留有胸大肌、锁骨下区饱满,无畸形。由于皮肤量不足,宜采用组织扩张器,增加皮肤组织量,经过4~6个月后,再植入硅凝胶假体。手术在全麻或高位硬膜外麻醉下进行,由于手术过程中出血很少,不必为乳房再造术而备血。

将组织扩张器置于胸大肌下,以及前锯肌、腹外斜肌、腹直肌筋膜下。组织扩张器种植囊的制备,内侧在胸骨旁线外1cm处,外侧达腋前线及腋中线之间,上方超过第3肋间,下至第6肋,在乳房下皱襞下1~2cm处。临床经验证明,下方应超过乳房下皱襞,以使术后假体纤维囊形成时,不致因为纤维囊挛缩而引起乳房位置上移的缺陷。这种组织扩张器对后来种植假体,较少形成纤维囊挛缩而造成再造乳房位置偏高。

图20-24

安放组织扩张器的容积视患者身高、体型及对侧乳房大小而定。国外一般安放400~800ml的组织扩张器作为乳房再造安放假体的囊腔。对于东方人,安放400~500ml的组织扩张器已足够。组织扩张器宜选择毛面组织扩张器,以防止扩张后纤维囊收缩。Mentor生产的Becker双腔组织扩张器,内腔注入盐水,外表面为毛面,外腔有硅凝胶。

组织扩张器宜安放在肌肉下,并对覆盖的肌肉进行修复。如在胸大肌下制成空隙后仍不足以覆盖放入的组织扩张器,则还应分离前锯肌、腹外斜肌及腹直肌筋膜,并将前锯肌边缘与胸大肌边缘作必要的缝合。如遇前锯肌不完整的病例,可取背阔肌肌瓣移植,覆盖组织扩张器囊。

为保证手术成功,手术过程中应在组织扩张器内注入盐水50~150ml。

不但要在皮下安放引流,而且在组织扩张器的肌肉下囊腔内也要安放引流。

根据乳房皮肤组织的松紧程度及组织愈合情况,给予注入盐水,在1~2周内注入盐水50~100ml。与一般组织扩张器使用方法一样,在注入盐水时应观察被扩张皮肤的血供状况,不要引起患者的疼痛;也需观察被扩张皮肤的张力,防止张力太高、皮肤血供不良、创口裂开等并发症。

需要进行二期手术埋入硅凝胶假体,替代组织扩张器。

更换假体的条件是:组织扩张器植入后的一侧乳房较健侧轻度下垂,组织松软,外形良好。

切开组织扩张器纤维囊腔,取出组织扩张器,修复乳房下皱襞的囊腔。因组织扩张器安放时下界超过乳房下皱襞,故应切开组织扩张器的下边囊腔壁,依据健侧乳房下皱襞的位置,重新固定腔壁到胸壁上。

选择与对侧乳房相称容量的、毛面的乳房硅凝胶假体植入。根据对侧乳房的形态,对再造乳房的假体囊腔作必要的修整,特别是外侧及下方宜进行必要的修整。

在安放组织扩张器期间,其他有关乳腺癌的化疗可同时进行。

在第二期植入硅凝胶假体的同时,行乳头及乳晕的再造(参见“乳头及乳晕的再造”)。


\section{第六节 男性乳房肥大症}

男性乳房肥大(gynecomastia)又称男子女性型乳房,表现为一侧或双侧乳房呈女性样发育、肥大,有时有乳汁样分泌物,多起始于男性青春期(12~17岁)或老年期(50~70岁)。男性乳房肥大症临床较为常见,是男性乳腺增生而引起的乳房增大。通常表现为单侧或双侧乳晕下区域乳房触痛性肿块,或乳房无痛性进行性增大。一般可分为生理性和病理性两大类。生理性男性乳房肥大症可发生在新生儿、青春期和老年期三个阶段。新生儿乳腺受胎盘高雌激素水平的影响,发生率可达60\%以上,以后逐渐消退,但可持续几个月;青春期部分男性出现乳房肥大,一般从13岁开始,持续数月至数年,有自限性。激素水平测定提示睾酮、二氢睾酮低下,雌激素、雄烯二酮比率和雌二醇/睾酮比率增高;老年人发生本症主要是由于睾丸退化导致雌/雄激素比值升高,还有一些老年期倾向性因素如性激素、黄体激素等反应性降低,另外与肥胖也有关。

肉眼观察

大体标本见乳腺肿块扁平,呈盘状,质韧,无完整包膜,切面呈灰白色,并可见孔状导管断面。

组织形态

常可见乳腺导管扩张,上皮增生,呈乳头状,基本上不形成腺泡和小叶结构,腺管间的纤维组织也有增生。

乳腺发育与许多内分泌激素作用有关,如雌酮能使乳腺导管增生,黄体酮在雌酮的协同作用下会使乳腺的腺泡增生,垂体生长素和催乳素也影响着乳腺的生长、发育。国内外学者研究结果表明,男性乳腺增生与这些激素失调有密切关系。如血液中雌激素增高或相对增高(即雄激素不足),或雄激素受体缺陷(在睾丸女性化中可见),亦或乳腺组织雌激素受体对雌激素敏感度增高等,均可导致各种类型的男性乳房肥大症的发生。

特发性乳房肥大只是乳腺体积增大,状如青春期少女乳房,其乳头、乳晕发育良好,生殖器官及其他器官不伴有发育异常及相关的病变。患者多为儿童期6~8岁男孩,肥大的乳房内除有乳腺导管增生,尚有许多腺泡,常为单侧、弥漫性,往往可自行消退。

发病者多为青春期男性(12~17岁),约3/4病例为双侧受累。常在乳晕下形成2~3cm的盘状肿块,并随时间推移而逐渐增生,可达到女性乳房大小程度。本病多无明显自觉症状,但少数患者可出现乳腺胀痛或压痛,常在1~2年内自然消退,偶有持续存在者。研究表明,在青春期男性乳房发育的病例中,血浆雌二醇比睾酮含量高,但随机体发育成熟睾酮含量逐渐达到正常成年男性的水平。

老年男性有时会发生不同程度的睾丸萎缩或功能衰竭,致使血液内总睾酮浓度和游离血清睾酮浓度降低,导致雌二醇含量相对增高,从而引起老年男性乳房肥大。患者年龄多在50~70岁,初起时为一侧乳房增大,继而对侧亦增大。临床上常可在乳晕下扪及一个3~4cm的肿块,边界清楚,与周围组织不粘连,推之移动,质地较硬,并有轻度压痛,常在1年内自然消失。少数病例在乳房肥大消退后,乳腺内还留有小硬结。

1.性腺功能减退引起男性乳房肥大 一般常见于原发性睾丸功能衰竭或减退的患者,亦见于继发性(垂体及下丘脑病变)性腺功能减退的患者。其原因在于血液中睾酮与雌二醇的比例下降。本病可见于大多数Klinefelter综合征患者。

2.全身性疾病引起的男性乳房肥大 迁延性疾病的恢复期 有些疾病使体重严重降低之后,在恢复期体重增加时,促性腺激素的分泌和性腺功能恢复正常,产生了一种类似第二青春期的现象,临床上称之为进食增加性乳房肥大,可在数月至1~2年内消退。

血液透析治疗后 肾衰患者血液透析后数周或数月后,由于饮食增加可引起乳房肥大。此外,尿毒症患者常出现激素水平异常,如血清睾酮浓度降低、血清雌激素及催乳素浓度升高等,亦可导致乳房肥大。

肝功能受损 男性乳房肥大也见于肝功能受损时。肝功能受损时,肝脏降解雌激素的过程发生障碍,但雄激素的降解过程未受影响,致使雌激素、雄激素比例失调,雌激素浓度相对增高,从而引起乳房肥大。

甲状腺功能亢进引起男性乳房肥大 在甲亢的男性病例中,有10\%~40\%的患者并发此症,这是由于血循环中甾体类蛋白增多,游离睾酮含量正常,而游离的雌二醇增多,从而刺激乳腺组织增生。

此外,男性乳房肥大常伴发于隐睾症、睾丸萎缩、睾丸炎、Reifenstesin综合征、肾有腺功能障碍、截瘫、糖尿病、麻风、结核、高血压、风湿性关节炎及溃疡性结肠炎等疾病。

3.肿瘤性男性乳房肥大、肾上腺肿瘤、肾上腺皮质增生及良、恶性肾上腺肿瘤,可直接分泌雌激素或产生过量的雌激素前体,后者在组织中转化为有效的雌激素。此类肿瘤多为恶性,在诊察时常可发现腹部肿块。

睾丸间质细胞肿瘤 较为罕见,大多数为良性肿瘤。睾丸间质细胞能产生过量雌二醇。过量的雌激素抑制了垂体的黄体分泌素,导致正常睾丸间质细胞的雄激素产量不足,从而促使男性乳房肥大。

此外,起源于睾丸生殖成分的恶性肿瘤,如睾丸良性或恶性畸胎瘤等,常可产生人体绒毛膜促性腺激素(HCG),HCG刺激睾丸间质细胞,从而产生大量的雌二醇,刺激乳腺增生。Gillbert(1940)报道102例睾丸畸胎瘤病例,均伴有男性乳房肥大。

4.药物性男性乳房肥大 临床长期服用雌激素(己烯雌酚等)类药物可直接刺激乳腺增生、肥大。另外,如螺内酯、甲腈咪胺与双氢睾酮竞争细胞表面受体,亦可引起乳腺增生、肥大。

青春期及老年男性无明显原因出现乳房肥大,或有上述全身性疾病、上述药物长期服用史而出现乳房肥大,或单纯出现乳晕下块状物,伴轻压痛,质地柔软,一般即可诊断。

B超表现:以乳头为中心或略偏向一侧的扇形或梭形低回声区,边缘较清晰,可达胸大肌的表面,两者之间界限清楚。内部回声不均匀.并可见向乳头方向聚拢的管状回声,彩色多普勒显示乳腺区内有1~2条彩色血流。

X线辅助诊断较为简便易行,该症X线表现可分为以下4种类型:

1.纤维型 又称腺体型,最为常见。乳晕下区呈现三角形或锥形致密阴影,有的尚合并有刷状或树枝状突起阴影,向下放射,伸入周围脂肪组织。所谓“单侧性”高度男性乳房发育症,常能在对侧“正常”的乳腺软组织摄影片上见到乳晕下较小的树枝状阴影。

2.大结节型 又称肿块型,表现为圆形或卵圆形、密度大致均匀的致密块影(真性男性乳腺发育症)。此型不仅有乳腺管增生,而且有腺小叶增生,亦有称其为假肿瘤型。有时尚可出现哑铃状改变。

3.分泌型 一般因雌激素治疗所致。乳腺管造影有助于本型与导管扩张症及乳头状瘤样病变相鉴别。

4.假性男性乳房发育症(脂肪性乳腺)因脂肪组织过多,可形成假性男性乳房肥大,一般多伴有全身性肥胖症。在乳腺软组织摄影片上,不能看到乳腺管增生或乳腺密度增高区,只能看到增多的脂肪组织。

1.男性乳腺癌 多见于老年男性,常为单侧乳房内肿块。肿块质地坚实,不规则,边界不清,常无触痛,早期可出现乳晕皮肤粘连及腋窝淋巴结肿大或转移,少数或可见乳头血性溢液。影像学上肿块多位于外上1/4部位,呈偏心性,边缘不清,呈毛刺状伸展;而男性乳房肥大的肿块位于乳晕后,呈中心性,边缘较光滑、整齐。

2.许多肥胖男性,随着躯体增胖,乳房脂肪也渐增多。双侧乳房呈对称性肥大、隆起。乳房内无肿块扪及,无触痛。

原发性男性乳房肥大多系暂时性,一般多逐渐自行消退。对于继发性男性乳房肥大、症状明显者,以及青春期乳房肥大经久不退,为改善外观,可采用以下治疗方法。

对睾丸肿瘤、甲状腺功能亢进及肝病等,应针对病因予以治疗。因外源性雌激素或药物引起的男性乳房肥大者,应停用相关药物。

应用三苯氧胺、甲睾酮及激素、中药等,部分患者的疼痛可缓解、肿块消退。

对于肿块疼痛明显、药物治疗无效或乳房肥大明显影响外观者,可采用手术治疗。该手术一般均可在局麻下进行。要求进行手术整形的男性乳房肥大患者,其乳腺增生一般均在100ml以上,因此患者常伴有不同程度的乳房皮肤松弛下垂。手术切口一般有3种选择:①乳晕下半圆形弧形切口。这是最常见的手术切口。切开皮肤,深入到乳腺周围筋膜,摘除肥大的乳腺组织块,置负压引流,缝合皮肤。②横乳晕、乳头切口,较少见。切开皮肤,深入到乳腺周围筋膜,摘除肥大的乳腺组织块,置负压引流,缝合皮肤。③在乳晕及乳房外侧作“L”形乳房缩小整形及切除肥大的乳腺组织块,置引流,缝合皮肤。术后胸廓加压包扎,根据伤口引流情况,一般2~3日后拔除引流。乳房肥大较轻者,小于50ml者,常采用第一种手术切口。近年来还出现了腔镜下切除及吸脂法治疗乳房肥大,吸脂法具有瘢痕小、恢复快等优点;但对于腺体性乳房肥大则疗效欠佳。腔镜下切除则需要术者有熟练的腔镜操作技术,另外,手术时间长,手术费用高等是其缺点。

1.朱洪荫,张涤生.整形外科·手术失误及处理.昆明:云南科技出版社,2000:159

2.王炜.整形外科学.杭州:浙江科技出版社,1999:1141-1149

(庄礼大 朱辉)


\chapter{第二十一章 生殖器官整形与再造}

隐匿性阴茎(concealed penis或buried penis)是指由于阴茎肉膜组织发育异常形成纤维条索或者耻骨前阴阜阴囊基部脂肪垫过厚,从而使正常大小的阴茎隐匿于耻骨前皮下,表现为阴茎外观短小,包皮口与阴茎根部距离较短。包皮似一鸟嘴包住阴茎,而与阴茎体并不附着,呈现出背侧短、腹侧长、内板多、外板少的情况。如用手握住阴茎的同时将阴茎周围皮肤后推,即可显示正常阴茎体。据国内报道发病率约为6.7‰。

引起隐匿性阴茎的原因较多,大多数学者认为先天性隐匿阴茎的主要病因是由于阴茎包皮肉膜层发育异常引起的。此外,包皮发育不良的患儿,因附着于阴茎体的皮肤不足也易导致隐匿阴茎的发生。有的鞘膜积液或腹股沟斜疝,以及蹼状阴茎的患儿,在施行包皮环切术后并发包皮口狭窄,导致包皮外板缩短,也可形成隐匿阴茎。而许多肥胖儿由于阴阜脂肪堆积,使得阴茎深藏于皮下,但当其脂肪组织减少,发育成熟后,阴茎通常可恢复正常状态,因此有的学者认为其并不属于隐匿性阴茎。

隐匿性阴茎可分为两种类型:

多见于肥胖患儿,由于下腹部特别是耻骨前脂肪较厚,使得阴茎深藏于皮下,而并无肉膜组织发育不良和皮下纤维条索存在,随着患儿年龄增长,脂肪组织逐渐减少和青春期发育成熟后,阴茎外形可自行恢复正常状态,无需手术治疗。因此有部分学者并不将此类情况归为隐匿性阴茎。

正常阴茎皮下有一层疏松而无脂肪的筋膜组织称为肉膜。肉膜在阴茎头部将包皮内外板相分隔开,并延续至会阴浅筋膜。肉膜在阴茎体部与皮肤及其深面的阴茎筋膜疏松附着,并向近端与腹壁浅筋膜相延续。正常发育的肉膜具有良好的弹性,能使阴茎皮肤在阴茎体上自由滑动。此类患儿由于阴茎肉膜组织发育不良,而在皮下常形成纤维条索,近侧附着于Scarpa筋膜上,远侧附着于从冠状沟到阴茎根部各部位的白膜上,由于纤维条索缺乏弹性,因此限制了阴茎的活动,使阴茎不能正常的在皮肤深层自由滑动,而隐匿在耻骨前的皮下层。因此纤维条索远端附着部位越靠近阴茎头,患儿阴茎的隐匿程度越严重。对于真性隐匿性阴茎的患儿需施行阴茎松解术,否则会影响其阴茎的发育,造成生理和心理上的障碍。

而根据Bergeson(1993)的隐匿性阴茎的分类标准,将隐匿性阴茎分为三类:

是指正常发育的阴茎隐藏于增厚的耻骨前脂肪内,由先天性发育异常所致。

是指阴囊的皮肤延伸到阴茎的腹侧,阴茎与阴囊之间连接异常所致。

束缚阴茎是一种后天性疾病,常见于包皮环切术后所形成的环状瘢痕,或是因蹼状阴茎、埋藏阴茎施行了简单的包皮环切术后导致阴茎的伸出限制,使其埋藏在耻骨前脂肪内。此类患者多伴有腹股沟斜疝和鞘膜积液,使阴囊明显肿胀。

也有学者将隐匿性阴茎分为:索带型,包茎型和肥胖型。

是指阴茎背侧挛缩的肉膜肌异常附着,术中可见肉膜呈索带状。表现为牵拉阴茎头后放开,阴茎很快回缩进去,恢复原状。

是指包皮口狭窄,且狭窄口接近阴茎根部,阴茎皮肤严重不足。表现为严重包茎,阴茎勃起时常将耻骨前皮肤、阴囊皮肤顶出。

是指阴茎体隐匿于耻骨前堆积的皮下脂肪中,并存在阴茎皮肤与阴茎体分离,常伴有包茎,亦可伴肉膜层发育异常。表现为:将阴茎两侧堆积的脂肪压向耻骨,可见阴茎显露正常,但松手后立即恢复原状,除伴有包茎者,阴茎皮肤通常足以遮盖阴茎体。

隐匿性阴茎患儿切勿将其当做一般的包茎患儿施行包皮环切术,否则术后不但不能恢复阴茎的正常解剖位置,而且切除了部分包皮,易导致包皮口的瘢痕性狭窄和皮肤缩短,给今后的治疗增加了很大的困难。

体检发现阴茎埋藏于皮下,阴茎短小,包皮似一鸟嘴状包住阴茎,耻骨前脂肪堆积,阴茎挤压试验可作出正确诊断。

阴茎挤压试验:用手握住阴茎,同时将阴茎周围皮肤向后推,能显示正常长度的阴茎为阳性。以此与小阴茎相鉴别。

对于存在阴茎肉膜发育异常和形成纤维条索的真性隐匿性阴茎通常采取手术治疗。对于手术年龄目前尚有争论,大多数学者认为以学龄前为宜,而就诊较晚者,也可在任何年龄手术。对于肥胖引起的假性隐匿性阴茎由于可随年龄增长而好转,尤其在减肥后更为明显,故凡包皮能上翻显露阴茎头者可不必手术。

隐匿性阴茎手术的目的是扩大包皮口,暴露阴茎头,切除肉膜层纤维索带,使阴茎体得到松解,从而使阴茎能够正常发育。

隐匿性阴茎手术方法较多,至今尚无统一的理想手术方法。最简单的方法是先尽量将包皮外翻,露出阴茎头,分别于阴茎背侧中线(12点)和阴茎两侧(3、9点)纵向切开狭窄环,再横向间断缝合,扩大包皮口。由于术中未松解阴茎体的纤维条索,且术后阴茎包皮水肿明显,因此隐匿性阴茎矫正手术并不理想。

目前较常用的手术方法有Shiraki包皮成形术和Devine隐匿阴茎矫正术。

术前行会阴部备皮,体检观察阴茎缩窄环及有无牵拉的纤维条索并作以标记。

根据患者的年龄来选择麻醉方式,通常采用硬膜外麻醉或者全身麻醉。

仰卧位。

(1)沿包皮外口环行切开包皮,并分离内、外板。

(2)先将包皮外板三等分后(2、6、10点)纵向切开1~1.5cm。形成三个皮瓣。

(3)再将包皮内板三等分(4、8、12点)纵向切开1~1.5cm,形成三个皮瓣。注意内板的切口须与外板的切口相互交错。

(4)松解和切除牵扯阴茎体的纤维条索组织,将阴茎皮肤向根部推移,并将两侧皮肤固定于阴茎根部白膜上。

(5)将包皮内、外板的皮瓣交错嵌插缝合(图21-1)。

(6)伤口适当加压包扎,留置导尿管。

图21-1 Shiraki包皮成形术

A 分离包皮内外板 B 内外板皮瓣嵌插缝合

(1)将包皮尽量上翻,显露狭窄的包皮口。沿阴茎背侧中线纵行切开包皮内外板,分离皮下组织,保护皮下的血管,切口不易过小。以阴茎头能够完全外露为宜,若包皮内板与阴茎头、冠状沟粘连,须进行彻底游离。当包皮完全翻转后,纵向切口已变成菱形。

(2)分离并显露出增厚的肉膜。将肉膜向耻骨方向剥离并横行剪开,显露出其深面的阴茎背血管和背神经。

(3)沿着菱形创口横轴向两侧横向延长切口,环形切口皮肤全层,显露出阴茎肉膜及深筋膜。

(4)剥离、切断肉膜,彻底切除环包裹牵拉阴茎的纤维条索组织,必要时可切断阴茎悬韧带,使得阴茎完全松解并充分伸直。

(5)当增厚挛缩的肉膜与腹壁筋膜完全离断后,腹壁筋膜退缩至耻骨区,原本被牵拉向下移位的脂肪组织亦将退回腹部,若阴茎周围仍有过于肥厚的脂肪垫,可同时将其切除。切除脂肪组织时,注意避免损伤精索。

(6)用可吸收缝线将下腹部皮肤缝合固定于耻骨区,再将包皮内板皮下弹性良好的会阴浅筋膜,缝合固定于阴茎根部的白膜上,为了避免损伤尿道海绵体、阴茎神经及阴茎血供,应在阴茎根部3~4点及8~9点处作阴茎根部皮肤和阴茎体固定。最后缝合阴茎的环形切口,缝线须穿过Buck筋膜,以防止阴茎退缩(图21-2)。

(7)伤口适当加压包扎,留置导尿管。

(1)在阴茎根部耻骨前皮肤上设计一倒“V”形皮瓣,蒂部位于耻骨前下皮肤、阴茎交界处,皮瓣大小为约长5cm,宽2.5cm大小,皮瓣的中轴与阴茎一致。

(2)在阴茎头部缝牵引线,向肢端方向牵引阴茎。

(3)切开皮肤及皮下层暴露皮下脂肪组织,在此层切除耻骨前大部分脂肪组织。牵拉阴茎,即可见纤维条索状增生组织限制阴茎向前延伸。切除此纤维条索后即可使阴茎外露。

图21-2 Devine法隐匿性阴茎矫正术

A 纵行切开阴茎皮肤并向两侧延伸为菱形 B切除包绕阴茎的纤维条索组织C 切断阴茎腹侧的肉膜 D 缝合切口

(4)为了增加阴茎长度,同时切断阴茎浅悬韧带及大部分深悬韧带,使阴茎向前延伸。

(5)利用“V-Y”成形的方法缝合皮肤,并将阴茎根部皮肤固定于耻骨联合的骨膜上,以防阴茎回缩(图21-3)。

图21-3 “V-Y”成形法矫正成人隐匿性阴茎

1.留置导尿管3~4天,避免排尿时弄湿敷料,引起感染。

2.术后注意观察阴茎,包皮切口的渗血、水肿情况,观察阴茎皮瓣的血运。

3.抑制阴茎勃起,成人可于术前二日开始口服或肌注己烯雌酚4mg,1次/日,或氯丙嗪12.5mg,1次/日,异丙嗪25mg,1次/日。

4.术后8~10天拆线。术后10天拆除阴茎的纱布敷料,继续保留弹力绷带包扎3~4周。过早解除包扎会发生顽固性的阴茎皮肤水肿。

这是由于没有充分认识和了解本病的发病机制误将隐匿性阴茎诊断为包皮过长或是包茎而进行简单的包皮环切术所致。因此术中应尽量保留全部阴茎皮肤。

阴茎皮肤水肿为阴茎手术后最为常见的并发症,一般于术后3~5天出现,持续5~10天不等。由于手术切断了部分阴茎背浅静脉、淋巴管甚至阴茎背深静脉,造成部分静脉回流和淋巴回流受阻所致。因此术中阴茎根部皮肤不宜行环形切口,否则宜导致术后环形瘢痕而产生回流障碍。术后阴茎异常勃起和患者早期下床活动过多也可加重阴茎的回流障碍。临床上主要表现为阴茎皮肤肿胀,包皮表面皮肤发亮,以阴茎远端包皮,尤其是腹侧系带处水肿最为明显。术后应以弹性绷带适当加压包扎。

阴茎血肿的形成通常是术中止血不彻底造成的,因此术中彻底止血是预防血肿发生的关键。另外由于会阴部的血供相对丰富,创面的渗血较多,术后引流不畅也是造成血肿形成的主要因素,所以常规应于伤口低位放置引流片或引流管充分引流。对活动性出血则应立即手术扩创,彻底止血并清除血肿。

先天性蹼状阴茎又称阴茎阴囊融合(webbed penis),是指阴茎腹侧皮肤与阴囊中缝皮肤呈蹼状融合,主要表现为阴囊皮肤向前扩展呈皱襞样达阴茎腹侧,有的与阴茎系带相连成蹼状。儿童期无影响,但成人后影响性生活。

蹼状阴茎多数为先天性畸形,其发病原因尚不明确,可能为阴唇阴囊隆突相互靠拢发育成阴囊时,未与阴茎皮肤分离,而继续向阴茎延伸发育所致。蹼状阴茎腹侧皮肤与阴囊中缝皮肤相连,形成蹼状,通常无阴茎弯曲,尿道也无发育异常。约3.5\%尿道下裂患者合并蹼状阴茎。

有少数患者,因阴茎发育不良行阴茎延长术后阴囊随阴茎皮肤向前延伸,在阴茎腹侧形成皮肤皱襞,在形态上和先天性蹼状阴茎相似,也影响性生活,我们将其称之为阴茎延长术后蹼状阴茎。也有极少数患者在包皮环切等手术中切除了过多的阴茎腹侧皮肤之后形成蹼状阴茎。

体检时可发现阴囊皮肤向前延伸至阴茎腹侧,使阴茎皮肤与阴囊中缝皮肤融合成蹼状的特殊外观,可作出正确诊断。

轻微的蹼状阴茎可不处理,但如果蹼状阴茎影响阴茎勃起和性生活者需行手术治疗。

手术目的是修复阴茎腹侧皮肤。手术方法包括简单的横切纵缝成形术和倒“V—Y”成形手术以及“W”成形术。

采用简单的横切纵缝成形术和倒“V—Y”成形术后易在阴茎腹侧形成直线疤痕,疤痕挛缩后易导致阴茎勃起时向腹侧弯曲,影响阴茎形态和性生活。而采用倒“V—Y”推进皮瓣结合“W”成形术,术后在阴茎腹侧形成连续“Z”形小切口,术后瘢痕不易挛缩,有助于阴茎勃起时的伸展,不会形成阴茎腹侧弯曲。

尿道下裂患者合并阴茎阴囊融合时,可在做尿道成形术的同时矫正蹼状阴茎。

术前行会阴部备皮,人工勃起阴茎,观察阴茎腹侧皮肤与阴囊皮肤交界位置。

根据患者的年龄来选择麻醉方式,通常采用硬膜外麻醉或者全身麻醉。

仰卧位。

(1)阴茎头留置牵引线。

(2)于阴茎阴囊之间的蹼状皮肤中段做横行切口,其宽度根据蹼状皮肤的大小决定。

(3)于皮下层广泛游离阴茎、阴囊皮肤后,纵向缝合创口(图21-4)。

(4)可向阴茎海绵体内注入生理盐水使阴茎完全勃起,判断蹼状阴茎是否完全矫正,必要时可延长横行切口。

(5)伤口适当加压包扎。

(1)阴茎头留置牵引线。

(2)于阴茎腹侧蹼状皮肤上设计一倒“V”形切口,切开皮肤、皮下组织及阴茎筋膜,暴露阴茎白膜层,在阴茎深筋膜和白膜之间分离,将“∧”形皮瓣,向阴囊方向推移,使阴茎皮肤和阴囊皮肤分开至阴茎能够完全伸直。

(3)创面止血后,将“∧”形皮瓣缝合成倒“Y”形(图21-5)。

(4)伤口适当加压包扎。

(1)阴茎头留置牵引线。

(2)先在蹼状皱襞皮肤侧设计“∧”形线,“∧”形角度及两臂长度根据蹼状皱襞皮肤的长度和宽度而定,以缝合后阴茎海绵体不受压为度,然后再以“∧”形线两臂为中轴线,设计多个“W”切口线,使三角形皮瓣相互交错,必要时在阴茎根部一侧加设一舌形皮瓣,以加深阴茎和阴囊之间的自然弧度。

图21-4 横切纵缝成形术

图21-5 倒“V—Y”成形术

(3)按照设计线切开皮肤、皮下组织及阴茎筋膜,暴露阴茎白膜层,在阴茎深筋膜和白膜之间分离。将“∧”形皮瓣向阴囊方向推移,两侧“W”形皮瓣作适当皮下分离,彻底止血,将两侧三角形皮瓣相互交错镶嵌缝合。缝合形成倒“Y”形(图21-6)。

(4)伤口适当加压包扎。

图21-6 “W”成形术

1.术后阴茎与阴囊之间保持一定的角度固定,以不牵拉阴茎影响皮瓣血运为宜。

2.术后注意观察阴茎,包皮切口的渗血、水肿情况,观察阴茎、阴囊皮瓣的血运。

3.抑制阴茎勃起,可于术前二日开始口服或肌注己烯雌酚4mg,1次/日,或氯丙嗪12.5mg,1次/日,异丙嗪25mg,1次/日。

4.术后8~10天拆线。

5.术后4周内禁止性生活。

术前清洁会阴

部,术中和术后合理应用抗生素以避免感染。

阴茎、阴囊皮肤血运丰富,皮瓣一般不易坏死,但“W”成形术中,设计三角形皮瓣夹角不应太小,以免尖端血运障碍、坏死,延迟切口愈合,影响疗效。

术后可采用弹力绷带包扎固定,减少因反复阴茎勃起引起切口出血形成的皮下血肿,弹力绷带压力以不影响排尿为宜。

阴茎阴囊转位(penoscrotal transposition)又称为阴囊分裂或阴茎前阴囊,是一种少见的先天性畸形。是指阴囊位于阴茎上方,阴囊两侧翼皱襞上方高于阴茎根部。分为完全性和部分性二类。常合并尿道下裂。

阴茎阴囊转位的病因目前尚不明确。有的学者认为可能是在胚胎期由于某种因素导致在生殖结节形成阴茎的发育过程受到延迟,而阴唇阴囊隆突则在其前方继续生长发育所致。也有学者认为可能在生殖结节和阴唇阴囊隆突同时发育的情况下,阴唇阴囊隆突向阴茎下方迁移不全,从而形成阴茎阴囊相互移位。

阴茎阴囊转位患者其阴囊呈分裂状,位于阴茎前上方,按阴囊与阴茎移位的程度分为两型:

是指部分阴囊延伸至阴茎两侧及背侧,呈半环形。

是指阴茎与阴囊位置上下颠倒,阴囊完全移位于阴茎上方(图21-7)。

阴茎阴囊转位常合并会阴型、阴囊型尿道下裂,也有报道指出有并发隐匿性阴茎和性染色体及骶尾部发育异常者。不合并尿道下裂者,其阴茎发育正常,无阴茎下弯和尿道开口移位。

图21-7 阴茎阴囊转位分类

根据阴囊部分或全部位于阴茎上方的典型临床表现,诊断并不困难。但应注意检查是否合并尿道下裂等畸形,也要与合并阴囊分裂的尿道下裂相鉴别。常规行染色体检查,排除染色体异常。

治疗阴茎阴囊转位的唯一方法是阴茎阴囊整形手术。手术通过消除或使阴茎根部两侧的阴囊皱襞移位来恢复阴茎阴囊的正常的解剖位置。最常应用的术式是“M”形皮瓣阴囊成形术。合并尿道下裂的患者可以在一期完成尿道成形的同时行阴囊成形术,术中应注意隐匿皮肤的血运情况,避免成形的尿道因血运障碍而坏死。为确保尿道成形术后尿道皮瓣的血液供应,也可分期手术,先完成尿道成形,术后6个月再修复阴茎阴囊转位。这样可避免由于先进行阴囊转位的矫治而过多的切除阴囊皮瓣,导致尿道成形,阴茎下曲矫治时阴囊皮瓣的缺少或阴囊转移皮瓣张力过大而影响术后伤口的愈合,从而影响尿道成形术的成功。

术前行会阴部备皮,观察阴茎根部与阴囊皮肤交界位置,如合并尿道下裂患者,术前设计一期或分期手术方案,尿道成形通常采用阴囊皮瓣,设计时注意保留阴囊皮瓣蒂部的血运不受阴茎阴囊移位矫正时切除阴囊皮肤的影响。

根据患者的年龄来选择麻醉方式,通常采用硬膜外麻醉或者全身麻醉。

截石位。

(1)阴茎头部牵引,插入导尿管。

(2)沿阴茎根部两侧阴囊翼上方、阴茎阴囊交界处,分别作倒“V”形切口,每侧阴囊切口至少包括阴囊的一半,两侧切口在阴茎腹侧根部合成“M”形。术中应保留的阴茎背侧皮肤宽度至少1.0cm以上,以免阴茎皮肤的血供受到影响。

(3)在阴茎腹侧游离并切除纤维条索组织,使阴茎弯曲得以完全矫正。

(4)以阴囊皮瓣转位行尿道成形术(详见尿道下裂章节)。

(5)沿阴囊皮瓣的两翼行皮下广泛分离后,两侧阴囊皮瓣向下推进到正常解剖位置。

(6)创面彻底止血后,利用“V-Y”成形的方法缝合两侧阴囊皮瓣(图21-8)。

(7)同时行尿道成形患者可作耻骨上膀胱造瘘。

1.术后阴茎与阴囊之间保持一定的角度固定,以不牵拉阴茎影响皮瓣血运为宜。

2.术后注意观察阴茎,阴囊切口的渗血、水肿情况,观察阴茎、阴囊皮瓣的血运。

3.抑制阴茎勃起,可于术前二日开始口服或肌注己烯雌酚4mg,1次/日,或氯丙嗪12.5mg,1次/日,异丙嗪25mg,1次/日。

4.术后保留导尿管3~4天,行尿道成形患者保留耻骨上膀胱造瘘管至少一周。

5.术后10~12天拆线。

图21-8 阴囊成形术

6.术后4周内禁止性生活。

术前清洁会阴部,术中和术后合理应用抗生素以避免感染。

阴囊皮肤血运丰富,皮瓣一般不易坏死,但“M”皮瓣成形术中,设计三角形皮瓣夹角不应太小,以免尖端血运障碍、坏死,延迟切口愈合,影响疗效。术中应于切口低位放置引流。

阴茎扭转(penile torsion)是指阴茎头偏离中线,呈逆时针方向转向一侧。多为先天性畸形,少数可继发于尿道成形术时,因阴茎皮肤处理不当或术后疤痕牵拉使得阴茎头扭转。先天性阴茎扭转有的可同时合并包茎或尿道下裂等泌尿生殖器畸形。阴茎扭转多在行包皮环切术或外翻包皮显露阴茎头时被发现。有的成年患者因为性生活障碍来就诊时发现。

先天性阴茎扭转的产生可能是生殖结节在形成阴茎的过程中,两侧阴茎海绵体发育不平衡所致。一般呈逆时针方向扭转,尿道口和阴茎系带常同时扭向一侧。

根据阴茎头偏离中线扭向一侧的体征可明确诊断。同时应仔细检查有无合并包茎或尿道下裂等泌尿生殖器畸形。

按阴茎头偏离中线的角度(以逆时针方向计算)将阴茎扭转分为三类:

阴茎头偏离中线角度小于60°,排尿正常,外观无明显畸形。

阴茎头偏离中线角度在60°~90°之间,排尿正常,外观无明显畸形。

阴茎头偏离中线角度大于90°,多伴阴茎弯曲,阴茎体及尿道海绵体可随之扭转,发生排尿困难及性生活障碍。

轻度阴茎扭转患者若不影响外观和排尿功能时,可不必治疗。轻、中度扭转患者可以通过阴茎皮肤脱套成形术来矫治。重度扭转患者在行阴茎皮肤脱套术的同时还应松解阴茎根部海绵体,切除牵拉的纤维条索,对较长侧阴茎海绵体白膜部分切除缩短,并将切除的白膜组织移植于较短侧阴茎海绵体上以矫正阴茎弯曲。必要时可将阴茎海绵体和尿道海绵体交界处的阴茎白膜固定在耻骨联合上。合并其他泌尿生殖器畸形的可同时矫正。

术前行会阴部备皮,人工勃起阴茎,观察阴茎扭转角度及有无阴茎海绵体发育不对称。

根据患者的年龄来选择麻醉方式,通常采用硬膜外麻醉或者全身麻醉。

仰卧位,同时矫正其他泌尿生殖器畸形时可采取截石位。

(1)阴茎头部牵引,插入导尿管。

(2)沿包皮环切切口及腹侧中线切开包皮至阴茎白膜,沿阴茎白膜分离,并将包皮完全脱套,以看到阴茎两侧的耻骨前脂肪垫(Scarpa's筋膜)为界。

(3)切断阴茎浅悬韧带层,彻底松解阴茎海绵体周围的纤维条索组织带。

(4)对于存在阴茎弯曲的患者,在较长侧阴茎海绵体弯曲最突出处适当切除约3.0×2.0cm椭圆形背侧白膜,注意保护阴茎海绵体,采用3-0薇乔线横行缝合切口。于较短侧阴茎海绵体腹侧做“S”形切口,横行切开阴茎白膜约2cm。适当游离海绵体组织,矫正阴茎弯曲后,将切下的白膜修剪后缝补于创口处。

(5)以阴茎背侧中线为基准,牵引阴茎体,将其向逆扭转方向保持在正常位置,将两侧的阴茎海绵体与尿道海绵体交界处阴茎白膜与耻骨联合分别对称固定,纠正阴茎体部扭转。

(6)将阴茎包皮背侧中点与阴茎头部包皮内板背侧中点对合后缝合,阴茎包皮腹侧中线与包皮内板系带侧中点缝合,纠正阴茎头部的扭转,修剪包皮,间断缝合切口(图21-9)。

(7)人工勃起试验,检查阴茎扭转矫正效果

(8)合并尿道下裂或其他畸形时应一并矫治。阴茎适当加压包扎。

图21-9 阴茎皮肤脱套术

1.术后阴茎向上固定。

2.术后注意观察阴茎,包皮切口的渗血、水肿情况,观察阴茎的血运。

3.抑制阴茎勃起,可于术前二日开始口服或肌注己烯雌酚4mg,1次/日,或氯丙嗪12.5mg,1次/日,异丙嗪25mg,1次/日。

4.术后3~4天拔除导尿管。

5.术后10~12天拆线。

6.术后6周内禁止性生活。

这与术前设计和术中处理不当有关。为了达到良好的手术效果,应当注意以下几方面:(1)阴茎包皮脱套必须彻底,彻底松解阴茎白膜周围的纤维条索组织,切断阴茎浅悬韧带,使得阴茎充分复位。(2)选择阴茎海绵体与尿道海绵体交界处阴茎白膜固定于耻骨联合,因为此处的白膜厚,可以承受较大的张力,固定牢靠。(3)以阴茎背侧中线为基准固定,在矫正了阴茎扭转后,使发育正常的尿道海绵体与耻骨联合保持垂直。(4)包皮背侧中点(对应耻骨联合中点)与阴茎头部背侧内板中点缝合,保持尿道外口的裂隙与阴茎腹侧系带在垂直水平。

术后可采用弹力绷带包扎固定,减少因反复阴茎勃起引起切口出血形成的皮下血肿,弹力绷带压力以不影响排尿为宜。如果包扎一周后,阴茎仍明显水肿,则继续加压包扎4~6周后会缓解。

阴茎弯曲(penile curvature of the penis,PC)可分为先天性阴茎弯曲和继发性阴茎弯曲。先天性阴茎弯曲非常罕见,其发病率约为37/100 000,通常都伴有尿道下裂,无尿道下裂的先天性阴茎弯曲,称先天性单纯阴茎弯曲,或称为原发性阴茎弯曲,占先天性阴茎弯曲患者的4\%~10\%。继发性阴茎弯曲是由阴茎硬结症(peyronie'S disease,PD)、创伤、感染,以及皮肤硬化症等疾病引起,其中以由阴茎硬结症引起者较为多见。

阴茎弯曲的病因至今尚不清楚,多数学者认为是在胚胎期因雄激素缺乏或不敏感而导致阴茎和尿道发育停顿或发育不良所致。可能与下列因素有关。

1.尿道发育不良,尿道海绵体缺乏 尿道海绵体在胚胎第10周左右,由左右尿生殖褶自生殖结节形成的阴茎根部向头端逐渐融合而成。来源于外胚层的阴茎头部尿道和来源于尿生殖窦初阴体部的尿道应在冠状沟处结合时,则不会出现弯曲。若在冠状沟近侧段结合时,由于外胚层所形成的尿道缺乏尿道海绵体,而由一种原始纤维替代产生牵拉从而使阴茎弯曲。

2.阴茎筋膜发育异常 尿道黏膜、尿道海绵体发育正常,但是Buck's筋膜和皮下肉膜发育异常,牵扯阴茎引起阴茎弯曲。

3.阴茎白膜发育异常 阴茎腹侧与背侧白膜发育不对称,背侧白膜过多而腹侧白膜相对较短致使阴茎弯曲。继发性阴茎弯曲,由于阴茎硬结症及反复创伤、感染导致白膜局部炎症、纤维化、斑块形成引起阴茎弯曲。而由皮肤硬化症引起的阴茎弯曲,可能由于机体免疫反应的异常而引起阴茎白膜广泛纤维化,致使阴茎弯曲。

4.阴茎皮肤发育异常:阴茎体部与皮肤粘连引起阴茎弯曲。

通常依据Devine分型将不伴尿道下裂的阴茎弯曲分为三型:

I型:尿道海绵体发育不良型,尿道仅由一层薄的黏膜管组成,直接位于皮下,而无尿道海绵体及其周围筋膜,阴茎被尿道下及两旁的纤维组织牵拉,向腹侧弯曲,病损最严重。

Ⅱ型:筋膜先天发育不良型,尿道海绵体发育正常,但Buck's筋膜及Dartos筋膜发育不良牵拉阴茎引起腹侧弯曲。

Ⅲ型:阴茎白膜发育不对称型,尿道海绵体及Buck's筋膜发育正常,阴茎弯曲由发育不良的Dartos筋膜牵拉引起。

Donnahoo等在Devine分型的基础上,将阴茎弯曲分为4型:Ⅰ型(皮肤挛缩型),Ⅱ型(筋膜先天发育不良型),Ⅲ型(阴茎海绵体发育不对称型),Ⅳ型(先天性短尿道型)。

Ⅰ型(皮肤挛缩型):主要表现为单纯的阴茎弯曲,无尿道缺损,尿道开口正常。

Ⅱ型(筋膜先天发育不良型),主要表现为尿道外口开口于阴茎龟头正常位置,阴茎弯曲,可有中线偏离但无头巾状覆盖。

Ⅲ型(阴茎海绵体发育不对称型),主要临床表现为尿道外口开口于阴茎龟头腹面,接近于正常位置;远端尿道不同程度呈膜状,紧贴附于阴茎腹面;阴茎包皮呈头巾状覆盖于阴茎背面;腹面包皮缺如。

无尿道下裂的阴茎弯曲经体格检查可明确诊断。但同时应注意与阴茎头型及冠状沟型尿道下裂相鉴别。

根据阴茎头与阴茎体纵轴的夹角将无尿道下裂的阴茎弯曲分为三度:夹角小于15°时为轻度弯曲,大于35°时为重度弯曲度,介于二者之间为中度弯曲。中度和重度阴茎弯曲在成年后会有性交困难。

中、重度阴茎下弯患者,由于成年后可发生性交困难,故应尽早施行下弯矫正术。

先天性单纯的阴茎弯曲手术应遵循三点基本原则:①切除尿道海绵体和阴茎海绵体之间的纤维条索,使尿道松解、延长;②适当缩短阴茎海绵体背侧的长度;③尿道延长或尿道成形。

(1)尿道海绵体发育不良型:矫正阴茎下弯,选用不同皮瓣修复尿道,如包皮瓣、阴囊皮瓣等。

(2)阴茎筋膜发育不良型:大部分患者应用阴茎皮肤脱套尿道松解方法可矫正阴茎弯曲,只有少数患者需做尿道成形术。术前留置导尿管避免损伤尿道,于冠状沟下方环形切开包皮,在浅筋膜下阴茎筋膜之间,将包皮分离至阴茎根部,腹侧包皮应在Buck's筋膜与尿道海绵体之间剥离。若阴茎仍不能完全伸直,则应游离尿道,切除尿道与阴茎海绵体之间的纤维组织,同时切除阴茎两侧的纤维带。必要时应考虑切断尿道,矫正阴茎弯曲后,再行尿道成形术。

(3)阴茎白膜发育不对称型:缩紧阴茎背侧白膜或扩大腹侧白膜矫正阴茎弯曲。手术方法有:Nebit法;Yachia法;Daskalopoulos法;单纯缝合阴茎白膜折叠法;阴茎腹侧白膜移植物植入法等。

(1)术前准备:术前行会阴部备皮,人工勃起阴茎,观察阴茎弯曲的程度及方向。术前谈话告知患者及家属术后阴茎海绵体长度可能缩短,阴茎头的敏感程度可能下降。

(2)麻醉:根据患者的年龄来选择麻醉方式,通常采用硬膜外麻醉或者全身麻醉。

(3)体位:仰卧位。

(4)各种手术方式及步骤———Nesbit法

1)插入导尿管,于阴茎冠状沟下方0.5~1.0cm处作包皮环形切口,切开包皮至Colles筋膜,分离Colles筋膜,将包皮脱套褪至阴茎根部。

2)于阴茎根部系一根橡皮带阻断阴茎海绵体静脉回流,然后用针头穿刺阴茎海绵体注入生理盐水,形成人工勃起状态。

3)测量并计算阴茎背侧及腹侧的长度及两者之差,用以判断阴茎背侧需缩短的长度。与阴茎背侧纵行切开Buck氏筋膜并向两侧分离,暴露出阴茎海绵体白膜及阴茎背深动静脉及神经,以阴茎弯曲弧度顶点为中点,分别设计数个小的梭形阴茎海绵体背侧白膜开窗切口,每个白膜窗口大小约为0.5×1.0cm,其数量和间隔的距离根据阴茎海绵体的长度和弯曲的程度决定。

图21-10 Nesbit法

4)切除设计的白膜窗口,间断缝合创口(图21-10),可采用抗张力强度大,吸收时间长的可吸收缝线,如3-0薇乔(吸收时间60~90天)。再次向阴茎海绵体内注入生理盐水使阴茎完全勃起,判断阴茎弯曲是否完全矫正,如果仍有部分弯曲,可根据弯曲的程度再次开窗。

5)分层缝合Buck氏筋膜、Colles筋膜和包皮。阴茎适当加压包扎。

(5)Yachia法

1)阴茎包皮切口,切开Colles筋膜,Buck氏筋膜,建立人工勃起,测量和计算阴茎海绵体折叠程度与Nesbit法基本相同。

2)以阴茎弯曲弧度顶点为中点,于阴茎海绵体背侧白膜分别设计数个小的间断纵向切口,每个切口长度0.6~1.0cm,每侧阴茎海绵体切口的数量和间隔的距离根据阴茎海绵体的长度和弯曲的程度决定。切开白膜全层后,将纵向切口横向缝合以缩短阴茎海绵体背侧长度,矫正阴茎弯曲(图21-11)。

3)分层缝合Buck氏筋膜、Colles筋膜和包皮。阴茎适当加压包扎。

(6)单纯缝合阴茎白膜折叠法

1)阴茎包皮切口,切开Colles筋膜,Buck氏筋膜,建立人工勃起,测量和计算阴茎海绵体折叠程度与Nesbit法基本相同。

2)在阴茎背侧白膜最突出处(10点和2点)直接将阴茎背侧白膜折叠后,外侧缘对合作褥氏缝合,矫正阴茎弯曲(图21-12)。

3)分层缝合Buck氏筋膜、Colles筋膜和包皮。阴茎适当加压包扎。

图21-11 Yachia法

图21-12 单纯缝合阴茎白膜折叠法

(7)Daskalopoulos法

1)阴茎包皮切口,切开Colles筋膜,Buck氏筋膜,建立人工勃起,测量和计算阴茎海绵体折叠程度与Nesbit法基本相同。

2)在阴茎背侧白膜最突出处(10点和2点)分别作二个长约0.8cm,相距0.5cm的平行的横行切口,将切口间的白膜折叠后,外侧缘对合缝合,矫正阴茎弯曲(图21-13)。

3)分层缝合Buck氏筋膜、Colles筋膜和包皮。阴茎适当加压包扎。

(8)阴茎腹侧白膜移植物植入法

1)阴茎包皮切口,切开Colles筋膜,Buck氏筋膜,建立人工勃起,测量和计算阴茎海绵体折叠程度与Nesbit法基本相同。

2)在阴茎背侧白膜最突出处适当切除约3.0×2.0cm条状白膜,注意保护阴茎海绵体,采用3-0薇荞线横行缝合切口。于阴茎腹侧做”S”形切口,横行切开阴茎白膜,切口约2cm。适当游离海绵体组织,矫正阴茎弯曲后,将切下的白膜修剪后缝补于创口处,检查有无明显出血。

3)分层缝合Buck氏筋膜、Colles筋膜和包皮。阴茎适当加压包扎(图21-14)。

图21-13 Daskalopoulos法

A 在阴茎背侧白膜最突出处作二个平行的横行切口B 将切口间的白膜折叠后,外侧缘对合缝合

图21-14 阴茎腹侧白膜移植物植入法

A 阴茎背侧白膜最突出处适当切除约3.0×2.0cm条状白膜B 将切下的白膜修剪后缝补于腹侧弯曲处

阴茎白膜补片可以采用阴茎海绵体自体白膜、口腔黏膜、睾丸鞘膜、牛心包膜、小肠黏膜下层和人工合成材料等材料。而理想白膜补片应具有:①取材简便,对机体损伤性小;②有良好血供及再生能力;③与白膜组织相容性好,无免疫排斥反应;④顺应性及组织弹性好;⑤易于缝合,价格低廉等特点。

1.术后阴茎向上固定,使阴茎根部皮瓣没有张力。

2.术后注意观察阴茎,包皮切口的渗血、水肿情况,观察阴茎的血运。

3.抑制阴茎勃起,可于术前二日开始口服或肌注己烯雌酚4mg,1次/日,或氯丙嗪12.5mg,1次/日,异丙嗪25mg,1次/日。

4.术后一周左右拔除导尿管。

5.术后10~12天拆线。

6.术后6周内禁止性生活。

主要由于导致弯曲的因素未完全去除,忽视了阴茎海绵体背侧和腹侧长度发育不平衡因素的存在。阴茎伸直后腹侧皮肤相对过短产生牵拉以及术后感染、瘢痕挛缩等因素导致矫正效果不满意,只要在术中建立人工勃起、白膜折叠、将多余的背侧皮肤转移覆盖于腹侧等措施,一般可获满意效果。

多为术中尿道损伤所致,尤其是尿道海绵体发育不良型患者。前尿道仅为一菲薄的黏膜管,极易被损伤,术中应先插入尿管作标志。游离尿道时必须非常仔细,从其侧面开始小心解剖,提拉尿道要使用宽胶片,切勿用丝线提拉或钳夹,局部出血点勿用电灼止血。如发现尿道有撕裂,可用5-0丝线缝补。如尿道无损伤,可留置导尿管,尿道损伤患者,须行膀胱造瘘。

可能与选择偏大的导尿管及术后加压包扎不当引起局部尿道黏膜坏死、感染有关。如尿管大小合适,注意包扎切勿过紧,可避免其发生。


\section{第二节 阴茎延长术}

阴茎延长术是治疗阴茎发育不良症、小阴茎畸形及外伤性阴茎部分缺损的主要术式。许多患者由于阴茎短小产生强烈的自卑感,有的阴茎短小的患者在性生活时难以让女方达到性高潮,因此他们都希望通过阴茎延长术来解决这些问题。

中国正常成人阴茎的长度在常态下为4.5~11.0cm,平均长度为7.1±1.5cm,周径约为5.5~11.0cm,平均周径为7.8±0.7cm;勃起时的长度约为10.7~16.5cm,平均为13.0±1.3cm,周径为8.5~13.5cm,平均为12.2±1.2cm。若阴茎勃起时的长度小于10cm,且性生活不和谐者,根据患者要求可行阴茎延长术。

小阴茎的病因包括低促性腺激素的性腺功能减退(hypogonadotropic hypogonadism)、高促性腺激素的性腺功能减退(hypergonadotropic hypogonadism)和原发性小阴茎。其中最常见的原因是低促性腺激素的性腺功能减退,患者同时可伴有某些综合征,如Kallmann、Prader-Willi、Lawrence-Moon-Biedl综合征等。正常男性外生殖器于胚胎期的前12周完成,阴茎发育分为3个阶段。第一阶段为生殖结节期。阴茎于会阴部类似小丘,长8~15mm;第二阶段为阴茎体期,阴茎拉长呈圆筒状,长16~38mm,尿道沟延至阴茎头;第三阶段于胚胎的第3个月,尿道发育完成,阴茎长38~45mm;胚胎第4个月后,阴茎逐渐增长。

阴茎的发育受到激素的调控。胎盘产生的绒毛膜促性腺激素(HCG)、腺垂体合成并分泌的黄体生成素(LH)和卵泡刺激素(FSH)能刺激睾丸间质细胞(Leydig细胞)产生睾酮(T),睾酮在5α-还原酶的作用下转化为双氢睾酮(DHT),双氢睾酮能刺激阴茎的发育。

若小阴茎患者尿道开口于阴茎头部,说明在妊娠12周内睾丸接受人绒毛膜促性腺激素(HCG)刺激的功能是正常的,但12周后由于胎儿促性腺激素缺乏,睾丸功能处于静止状态,如果同时伴有尿道下裂,可能是由于睾丸分泌睾酮不足(原发性性腺功能不全、酶异常)或靶器官利用睾酮不充分(雄激素受体缺乏、5α一还原酶缺乏)。相反,原始生殖结节的异常并非小阴茎的常见原因。

阴茎的发育分为三个阶段:胎儿分化发育期;幼儿发育期(0~6岁);青春发育期(自13岁左右进入青春期)。6~12岁间阴茎发育基本停滞,20岁以后阴茎已发育成熟。患者在阴茎发育期间,由于某些疾病而导致全身性营养不良,身体发育迟缓,阴茎的发育也受到抑制。有的患儿特别是肥胖儿由于血睾酮的含量较低,造成阴茎发育迟缓。

学龄前儿童若患包皮过长或包茎,包皮不能外翻,阴茎头无法外露,如未能及时行包皮环切术,也可能影响阴茎的发育。

由于外伤、阴茎肿瘤切除手术或包皮环切术后处理不当而导致阴茎部分缺损。

以往有的学者认为女性的性敏感区仅存在于阴蒂、阴唇和距阴道内口2/3区域内,所以阴茎的长度并不影响性生活的质量,但是根据近年来的研究显示,女性阴道远端1/3也存在性敏感区,而女性往往通过阴茎反复从宫颈滑到穹隆来获得强烈的性快感,这就为阴茎延长术提供了理论上的依据。

阴茎由两个阴茎海绵体和一个尿道海绵体组成,其根部由两个阴茎海绵体脚固定于耻骨弓上,阴茎体部借助阴茎悬韧带固定于耻骨联合和腹白线的下部。阴茎悬韧带分为阴茎浅悬韧带和阴茎深悬韧带(图21-15)。根据解剖学研究发现:阴茎浅悬韧带厚度为0.6~2.0cm;阴茎深悬韧带厚度为2.2~2.8cm。阴茎浅悬韧带前缘至阴茎深悬韧带后缘的长度为4.0~6.8cm,平均5.5cm。当阴茎浅、深悬韧带被完全切断并分离至耻骨弓时,原固定于耻骨联合和耻骨支下方前方的阴茎得以充分游离,从而增加了阴茎体的长度(图21-16)。由于阴茎海绵体脚附着于耻骨弓和两侧的坐骨支,并有坐骨海绵体肌和腱膜覆盖,从而在阴茎勃起时,仍能够保持阴茎的勃起强度和稳定性。通过大量的尸体解剖和临床实践发现通过切断阴茎浅、深悬韧带后可以使阴茎延伸4~6cm。

图21-15 阴茎悬韧带分为阴茎浅悬韧带

(A)和阴茎深悬韧带(B)

图21-16 切断阴茎浅、深悬韧带

由于阴茎的长度常受到患者的精神、体位、室温及外界环境的影响,因此准确的测量阴茎长度是衡量阴茎延长术的手术疗效的重要指标。测量的长度应以阴茎勃起时的长度为准,为了减少误差,在测量时应注意以下几个方面:

1.患者取站立位,室温在25℃左右,避免异性刺激。

2.反复测量,在患者熟悉的环境中逐渐适应测量操作,通过多次测量取测定数据的平均值。

3.测量时以从阴茎根部腹壁翻折处至阴茎头尿道外口的长度为阴茎长度。包皮过长患者应翻起包皮再进行测量。

4.一般应测量手术前后阴茎在常态下和勃起状态下阴茎的长度和周径。

根据测量阴茎的长度、周径进行诊断,若成年男性在勃起状态下阴茎长度小于10cm,即可作出诊断。但同时应仔细检查患者的第二性征、性腺发育情况、尿道开口、睾丸是否已完全下降至阴囊内等判断患者是否合并其他泌尿生殖系畸形,必要时可检查染色体。

成年男性若阴茎勃起状态下长度小于10cm,且性生活不和谐者,根据患者的要求可行阴茎延长术。

此类患者阴茎海绵体发育基本正常。可以先行阴茎延长术,切断全部阴茎浅、深悬韧带,甚至可以分离至阴茎海绵体脚,使得埋藏于耻骨联合前的阴茎海绵体游离出来,在利用腹部皮瓣、腹股沟皮瓣或阴囊皮瓣来包绕外露的阴茎海绵体创面。用这种方法替代阴茎再造术,不仅可以有效地延长阴茎的长度,还使得术后阴茎具有正常的感觉和勃起功能。

在幼儿期及青春期行内分泌治疗后阴茎的长度及周径仍大大低于正常,可行阴茎延长及增粗术来改善阴茎形态。

在行隐匿性阴茎矫正术或阴茎隐匿转位矫正术的同时切断阴茎浅、深悬韧带,能够使阴茎充分延伸,也使术后阴茎的形态更加满意。但同时应注意有无尿道的缺损或开口异常。

在行阴茎背浅、深静脉结扎的同时行阴茎延长术可以取得更好的效果。

术前行会阴部备皮,测量阴茎在常态下及勃起状态下阴茎的长度和周径。术前谈话告知患者及家属术后阴茎根部位置可能稍向下移,短期内阴茎根部可能出现凹陷。

根据患者的年龄来选择麻醉方式,通常采用硬膜外麻醉或者全身麻醉。

仰卧位。

适用于阴茎部分缺损而阴囊大部分保留患者。

(1)阴茎头部牵引,于阴茎残端的根部作环形切口。

(2)彻底切除和松解阴茎根部的疤痕和纤维条索组织,使得阴茎海绵体充分松解,延伸。

(3)于阴茎根部的两侧各设计一个大小相同而方向相反的三角形皮瓣,皮瓣的大小根据外露的阴茎海绵体创面大小来决定。

(4)游离两侧三角形皮瓣皮下,注意保护皮瓣的蒂部。

(5)将两侧的三角形皮瓣旋转包绕阴茎海绵体创面后,缝合切缘,皮瓣供区直接拉拢缝合(图21-17)。

此术式由龙道畴教授于1984年率先开展。此术式避开了阴茎背动脉和阴茎背神经,术后性功能不受影响,同时术中结扎了部分阴茎背浅静脉,使得某些阴茎静脉性勃起功能障碍患者的勃起功能得到改善。

图21-17 阴茎残端延伸法

A 阴茎残端根部切口 B 松解、切除阴茎根部瘢痕,设计两个大小相同而方向相反的三角形皮瓣 C 皮瓣转位 D 缝合切口

(1)切口设计

1)于耻骨联合处作“M”形切口,行“Y-Z”形缝合或“X”形缝合。

2)于耻骨联合处作“V”形切口,行“X”形缝合。

3)于耻骨联合处作倒“V”形切口,行倒“Y”形缝合(图21-18)。

(2)按照设计切口切开皮肤全层,逐层向下分离,暴露阴茎浅悬韧带。

(3)牵拉阴茎头部,使得阴茎悬韧带有一定张力,分离悬韧带两侧的浅筋膜和疏松结缔组织。

(4)剪断阴茎浅悬韧带后可以看到其下方的阴茎深悬韧带,剪断深悬韧带,将阴茎海绵体分离至耻骨弓,必要时也可继续分离部分海绵体脚,使原固定于耻骨下支的阴茎海绵体充分分离,以使阴茎更为延伸。

(5)创面彻底止血,必要时可缝扎阴茎背深静脉。游离耻骨弓两侧的脂肪组织及结缔组织形成两个组织瓣,将这两个组织瓣填塞于阴茎浅、深悬韧带切断后,耻骨弓前的空隙处,并缝合固定于耻骨联合的骨膜上,这样既消灭了死腔又能够防止韧带的再次粘连。

(6)按照切口设计分层缝合切缘。

1.术后阴茎向上固定,使阴茎根部皮瓣没有张力。

2.术后注意观察阴茎根部切口的渗血、水肿情况,观察阴茎根部皮瓣的血运。

3.抑制阴茎勃起,可于术前二日开始口服或肌注己烯雌酚4mg,1次/日,或氯丙嗪12.5mg,1次/日,异丙嗪25mg,1次/日。

4.术后3~4天拔除导尿管。

5.术后5天开始将阴茎头向前下方牵拉,开始轻拉。7天后逐渐加重,以避免被切断的浅悬韧带两断端间的粘连。

6.术后两周拆线。

7.术后应尽可能平卧,可用弹力绷带适当加压包扎阴茎,促进水肿消退。

8.术后6周内禁止性生活。

术后阴茎包皮水肿为阴茎延长术后最为常见的并发症,一般出现于术后3~5天,持续5~10天不等。表现为阴茎包皮肿胀,包皮表面皮肤发亮,常见阴茎远端包皮水肿较重,尤其是腹侧系带处水肿最为明显。由于术中切断了部分阴茎背浅静脉和部分淋巴管,有的甚至切断了背深静脉,造成部分淋巴回流和静脉回流受阻,常出现阴茎包皮水肿。另外,阴茎术后异常勃起也可加重阴茎的回流障碍,患者早期下床过多活动亦可由于重力的原因而引起水肿。防治措施:1.术后嘱患者尽可能平卧;2.应用活血化瘀药物行消肿治疗;3.严重者可用弹性绷带压迫法进行物理治疗。

图21-18 耻骨弓前海绵体延伸法

A “M”形切口,行“Y-Z”形缝合或“X”形缝合 B “V”形切口,行“X”形缝合C 倒“V”形切口,行倒“Y”形缝合

皮瓣尖端坏死表现为阴茎根部皮瓣在术后48~72小时内,皮瓣尖端出现发白、暗红现象,然后逐渐变黑坏死。预防措施:1.术后常规阴茎中立位外固定,以减轻阴茎悬垂对皮瓣的牵拉使皮瓣张力过大;2.术后抑制阴茎的勃起;3.应用保温、微波治疗、药物等方法促进皮瓣的血液循环。一旦发生此并发症,即应积极处理。首先,应采取积极的换药治疗,同时去除影响血供的不利因素,并辅以药物改善微循环,换药时应先将切口的血痂用过氧化氢溶液、抗生素盐水去除后再进行。如果仍不能改善,应采取手术清创,去除坏死的组织,重新缝合。如张力较大可设计辅助皮瓣修复。

血肿的发生通常是由于术中止血不彻底或术后引流不畅所造成。由于阴茎延长术手术野的暴露相对困难,给止血带来一定难度;另外会阴部的血供相对丰富,创面的渗血较多,术后引流不畅也会造成血肿的发生。预防措施:1.术中严格、彻底的止血,这是最关键的一点;2.手术区放置引流片或引流管进行引流,每天观察引流情况;3.术后常规应用止血药物。对并发症的治疗应在上述治疗的基础上采用局部加压包扎等方法。如穿刺发现存在活动性出血则应积极行手术切开,彻底止血并清除血肿。

伤口愈合时间一般为12~16天,故拆线时间应严格限制在2周之后,过早拆线不利于伤口愈合。如术后3周仍不能愈合,则视为伤口延迟愈合。一般认为与患者的机体状况和营养水平有关,术后护理不当也是原因之一。对于要求阴茎延长的患者,应严格把握手术的适应证,一些有严重器质性病变者应视为手术禁忌。所以,对于这种类型的患者除了常规的治疗之外,全身的支持治疗也是非常重要的。另外,一些患者术后由于各方面的原因导致伤口愈合缓慢,营养较差也是导致伤口延迟愈合的原因之一,因此,全面加强患者营养就显得非常重要。

感染的发生大多是由于无菌操作不严格,术后换药及护理不当所致。由于术后患者常常因为疼痛和包皮水肿致排尿相对困难,小便浸染切口的情况也会发生。因此,术后换药和护理应及时严格进行无菌操作,密切观察切口情况。因为阴茎延长术的手术部位比较特殊,术后应常规应用抗生素以预防感染的发生。一旦出现感染,则应加强术后护理和换药治疗,并根据药敏试验来选择敏感的抗生素。

严格意义上讲,术后阴茎蹼状畸形并不是真正的并发症,因为这种情况的出现原因为:

1.手术皮瓣的设计及进行主要是在阴茎的背侧;

2.手术后阴茎腹侧的皮肤没有相应的补充,而是由皮肤的弹性拉伸来满足;

3.当阴茎延长的长度超出腹侧皮肤的弹性极限时,阴茎腹侧形成皮肤皱襞与阴茎阴囊相连的蹼状畸形。手术者为保证皮瓣的成活,过多地保留皮瓣而使阴茎过于臃肿而出现蹼状畸形。对于此类患者,可以行二次手术进行蹼状阴茎矫正术,但一期手术与二期塑形之间一般要间隔3~6个月。


\section{第三节 阴茎增粗成形术}

以往阴茎延长增粗手术仅用于治疗各种泌尿生殖系疾病,包括尿道上裂、Peyronie病、外伤性阴茎缺损、脊髓损伤所致的阴茎短缩等。Austoni等将此技术用于小阴茎和畸形恐怖症患者。20世纪90年代以后阴茎延长增粗术才被用于正常阴茎的美容整形手术。目前,阴茎增粗成形术通常用于先天性或特发性阴茎发育不良而呈小阴茎者,或因妻子生育后阴道松弛、性生活不满意,而强烈要求行阴茎增粗并延长手术以改善夫妻性生活者。

阴茎增粗的方法很多,几十年前盛行注射液态硅胶以增粗阴茎,由于液态硅胶容易产生排异反应、结节形成、形态异常、腹股沟淋巴结肿大等并发症,现在已被禁用。20世纪80年代中期,开始应用自体脂肪注射增粗阴茎,但由于脂肪的准备、注射方法和注射量都不相同,效果也各不相同,因此争论也比较大。由于可能出现脂肪的吸收,脂肪结节的形成,外形的不满意,远端脂肪的堆积等,可能需要多次的注射,所以部分学者不赞同这种方法。近年来有的学者用真皮脂肪瓣条游离移植以及整片真皮脂肪瓣包裹的方法,行阴茎增粗术,认为这种方法的优点是自体组织无排异反应、脂肪细胞的破坏少、真皮可以快速存活、而且外形自然、无异物感、效果优于脂肪注射。缺点是供区有切口疤痕,有坏死钙化的可能性。还有学者采用自体大隐静脉加粗阴茎,优点是无排异反应、解剖恒定、操作简便,缺点是供区有切口疤痕。近年来随着人工合成材料、生物材料和组织工程技术的应用,如人工血管、膨体聚四氟乙烯(PTFE)、异体无细胞真皮基质、硅橡胶假体及透明质酸凝胶等材料均用于阴茎增粗术,这些材料中除硅橡胶假体外价格普遍偏贵。虽然有的方法操作比较复杂,但是由于不增加其他切口,近年来也越来越被大家接受和开展。

先天性或特发性阴茎发育不良而呈小阴茎者,或因妻子生育后阴道松弛、性生活不满意,而强烈要求行阴茎增粗并延长手术以改善夫妻性生活者。

对于先天性阴茎发育不良的成人,阴茎勃起长度小于9cm者。必须对患者情绪的稳定性、期望值和动机进行评估。高度怀疑、不情愿、期望过高、认为阴茎增大能解决包括婚姻生活、改善男性形象等所有问题的患者不宜接受手术。

由于同时行包皮环切术会增加术后伤口裂开、迟发性出血、水肿及术后的不适,故患者未行包皮环切者则应在阴茎增粗成形术前6周进行环切手术。

术前行会阴部备皮,对患者情绪的稳定性、期望值和动机进行评估。术前谈话告知患者及家属术后阴茎的形态改变,以及可能出现的并发症。

通常采用局部麻醉、硬膜外麻醉或全身麻醉。

仰卧位。

(1)腹部吸脂:选择脐部或双侧髂前上嵴吸脂切口,根据患者阴茎的大小和意愿吸取30~60ml脂肪。并置于带刻度漏斗状脂肪收集器或注射器内,静置一段时间后,出现脂肪层分离(最初吸出的脂肪损伤最轻、碎片最少,因而是最理想的供体)。一般不冲洗脂肪组织。

(2)脂肪注射:留置尿管,取脂肪层上层的脂肪组织于注射器中,牵拉阴茎头,在冠状沟处提起包皮皱褶。将注射针头于阴茎体背部中线刺入皱褶直至皮下浅筋膜层。针头在冠状沟与阴茎根部之间移动注射,动作缓慢小心,避免损伤尿道。注射脂肪的量需根据患者的意愿、吸取的质量、皮肤的松弛程度、阴茎的大小特别是阴茎头的大小而定。

(3)阴茎适度加压包扎。

(1)切取真皮脂肪瓣:于双侧腰部或臀下沟处作约2cm大小切口,以此切口为中心,按照皮肤皱褶的方向将周围10×2cm区域的皮肤去表皮,再切取其下方的真皮脂肪瓣,供区创面可直接缝合,也可将表皮回植,打包加压包扎。

(2)真皮脂肪瓣移植:作包皮环切切口,或在阴茎体四周做多个小切口,切口皮肤直至阴茎深筋膜,沿深筋膜分离。将真皮脂肪瓣修剪为3~4条,真皮面向阴茎海绵体,绕着阴茎背侧、内外侧植入,分段固定。

(3)缝合创口,阴茎适度加压包扎。

(1)从大隐静脉汇入股静脉处开始游离,根据阴茎海绵体长度,切取一段大隐静脉,其长度相当于阴茎海绵体长度的2倍,以供两侧海绵体移植使用。沿其纵轴剖开裁剪宽为1cm的静脉补片。

(2)在阴茎体两侧,阴茎海绵体与尿道海绵体间凹沟处分别做与阴茎体长轴平行的长2~3cm纵切口,切开皮肤,将皮肤创缘向上轻柔牵引并对准阴茎海绵体做纵切,逐层切开浅筋膜、Buck's筋膜,术中应小心避开阴茎背神经走向腹侧的分支并保护阴茎背动脉。

(3)结扎切断白膜表面3~4条环静脉。显露拟扩大的阴茎海绵体白膜。用橡皮筋紧束阴茎根部以止血,切开白膜,注意不要损伤下面的白膜下静脉和窦状隙外一层菲薄的膜状组织。向两侧锐性剥离,形成宽1cm的椭圆形白膜创口。

(4)将准备好的大隐静脉补片或ePTFE人工血管植入阴茎海绵体白膜创口处,使静脉内膜面朝向海绵体窦状隙,严密缝合补片与海绵体白膜创缘。两侧操作方法相同。

(5)松开阴茎根部止血橡皮筋,见两侧阴茎海绵体为血液充盈,移植静脉亦充盈饱满,阴茎海绵体体积增大。检查创缘有无出血。

(6)分层缝合阴茎深筋膜、浅筋膜及皮肤。留置导尿管,阴茎适度加压包扎(图21-19)。

(1)于阴茎体背侧距冠状沟约1cm处作约2cm的横形切口。依次切开皮肤、浅筋膜、深筋膜至白膜表面,向两侧分离腔隙。

(2)将人造材料如:膨体聚四氟乙烯(ePTFE)及硅橡胶假体雕刻制作成“C”形或条状后植入阴茎体近冠状沟处使之增粗。

(3)缝合切口,阴茎适度加压包扎。

图21-19 自体大隐静脉或膨体聚四氟乙烯(ePTFE)人工血管阴茎海绵体增粗术

A 切开阴茎海绵体白膜 B 大隐静脉补片或ePTFE人工血管植入阴茎海绵体白膜创口处

理想的生物材料应该是生物可降解材料,随着材料的降解和细胞的繁殖,形成新的、与自身功能和形态相适应的、有活力的组织或器官,以达到永久替代,不留异物。还应该具有好的生物相容性,以及合适的机械、物理特性。目前可应用于阴茎增粗术的生物材料有两类:一是无细胞真皮基质(AlloDerm),是无细胞的天然组织支架,在结构上除了无细胞外,均与真皮基质一样,是一种大分子的网状结构,可引导细胞迁移至移植物中。同种AlloDerm已通过了美国FDA的批准,优点是无抗原性、无毒性、吸收少、柔软易塑形、植入后有新生血管长入和成纤维细胞移入。缺点是一次移植不能太多,增厚的厚度有限、价格昂贵、来源有限等。二是细胞-生物复合体:聚乳酸-乙醇酸共聚物(PLGA),这是一种生物可降解材料,植入体内后随细胞的生长逐渐降解成水和二氧化碳而排出体外,其强度、柔韧性好,对机械外力耐受程度高。通过调节乳酸、乙醇酸两种单体的混入比例,可调节支架的降解周期和机械强度。

(1)先切取少许阴茎体软组织细胞,进行体外培养扩增约30天,将扩增的细胞移植无细胞真皮基质支架或人生物可降解高分子材料(PLGA)支架中,从而形成一种复合体,确认细胞成活。

(2)与阴茎冠状沟下0.3~0.5cm处环形切口,切开阴茎皮肤,至Buck筋膜表面。沿Buck筋膜表面向下游离直至阴茎根部,将阴茎包皮脱套。

(3)根据阴茎大小将AlloDerm或PLGA材料裁剪成合适的大小覆盖于Buck氏筋膜表面。

(4)分层缝合肉膜及阴茎皮肤,阴茎适度加压包扎。

1.术后阴茎向上适度加压固定。

2.术后注意观察阴茎,包皮切口的渗血、水肿情况,观察阴茎的血运。

3.抑制阴茎勃起,可于术前二日开始口服或肌注己烯雌酚4mg,1次/日,或氯丙嗪12.5mg,1次/日,异丙嗪25mg,1次/日。

4.术后8~10天拆线。

5.术后6周内禁止性生活。

水肿一般在2~3周逐渐消退,勿同时行包皮环切,这样可能会导致包皮顽固性水肿。

通常弯曲的程度并不影响性生活,可能是由于植入物放置不对称,或过早的进行性生活后使假体移位造成,一般3个月可自行恢复,若假体移位较多需再次手术矫正。

假体材质坚硬或机体产生排斥反应可能导致阴茎皮肤破溃,甚至性交疼痛,必要时需取出假体。


\section{第四节 阴茎包皮过短矫正术}

包皮过短通常见于包皮环切术中包皮切除过多,或者术后发生感染、处理不当等因素造成部分包皮坏死。

患者既往曾行包皮环切术,阴茎冠状沟下方见环形疤痕,疤痕牵拉致阴茎勃起时疼痛甚至阴茎弯曲。阴茎海绵体及尿道发育正常。

手术是唯一的解决方式。一般常用延迟阴囊皮瓣法。手术采用二期完成,一期切除包皮疤痕组织后,将阴茎体的包皮缺损创面埋置于阴囊皮瓣内。二期行阴囊皮瓣断蒂,修复缺损的包皮。

术前行会阴部备皮,人工勃起阴茎,估计包皮缺损的面积和所需阴囊皮瓣的大小。

根据患者的年龄来选择麻醉方式,通常采用硬膜外麻醉或者全身麻醉。

仰卧位。

1.牵引阴茎头,于阴茎冠状沟下方0.5~1.0cm处作一环形切口,切开阴茎皮肤。充分进行皮下分离松解,切除疤痕组织,形成一段环形创面。

2.将阴茎向腹侧牵拉,标记出阴囊前壁上相对位置。在阴囊前壁上作两条横行切口,切口的间距与阴茎创面的宽度相等,长度约为阴茎创面周径的一半。

3.分离切口间的皮下组织,形成一双蒂皮瓣。然后将阴茎头由皮瓣下穿过,分别将阴茎上下创缘与阴囊皮瓣的上下切口缘对合后缝合。

4.术后三周,通过血循环阻断实验检查皮瓣断蒂后阴茎皮肤的血运。在双蒂两端作延长切口(每侧切口距阴茎的距离约相当于阴茎周径的1/4),切断蒂部,分离阴茎、阴囊。

5.将阴茎背侧的阴囊皮瓣两端包绕阴茎创面后缝合。阴囊皮瓣创面直接拉拢缝合(图21-20)。

1.一期术后保持阴茎向腹侧弯曲。

图21-20 阴茎包皮过短矫正术

A 切除阴茎疤痕 B 将阴茎创面埋置于阴囊皮瓣下 C 切取阴囊皮瓣并断蒂 D 缝合皮瓣

2.术后注意观察阴茎,包皮切口的渗血、水肿情况,观察阴茎、阴囊皮瓣的血运。

3.抑制阴茎勃起,可于术前二日开始口服或肌注己烯雌酚4mg,1次/日,或氯丙嗪12.5mg,1次/日,异丙嗪25mg,1次/日。

4.术后8~10天拆线。

5.术后4周内禁止性生活。

术前清洁会阴部,术中和术后合理应用抗生素以避免感染。

阴茎、阴囊皮肤血运丰富,皮瓣一般不易坏死,但皮瓣形成过程中长宽不足,或断蒂时间过早均可造成皮瓣坏死。因此一期术后5~6天可开始进行血循环阻断训练,从5~10分钟开始,每天训练2~3次,时间逐渐延长。二期断蒂时,先切断一半,观察一段时间后,若皮瓣无缺血或淤血等血供不良表现,再完全切断蒂部。如有可疑,可暂缓断蒂,一周后再行手术。


\section{第五节 女性外阴整形手术}

年龄的增长,可使会阴部出现下垂,外生殖器官开始萎缩,皮下脂肪进行性减少,外阴干燥、黏膜萎缩。怀孕和体重增加也会加速脂肪积聚和皮下浅筋膜系统的松弛,导致会阴下垂。同时,大腿的脂肪代谢也常伴随筋膜松弛和变薄。当大阴唇过度增生或变得松弛时,皱褶就会产生,色素沉着,神经、皮脂腺及相关结构的增生也会相伴而来,这些改变会引起自净能力下降、性交痛、炎症等病理变化。过度下垂的阴唇可在坐位或穿紧身裤时产生疼痛。

同时,现在很多女性会参加各种高强度的锻炼项目,如跑步、减肥、健美操、骑自行车等,这些都会对外阴都造成很大的负担,而且这类人群常会喜欢比基尼内衣、紧身裤、塑身衣,进一步加重会阴组织的增生。在这类求术者当中,通常以中年妇女为主,尤其是生育、体重增加后已有外生殖器的改变征象的且希望恢复从前体态的患者。

另外,虽然社会进步,人们的“贞操观”已今非昔比,但还是有年轻女性因种种原因而要求行处女膜修补术。

随年龄增长,外阴部皮下浅筋膜变薄松弛。脂肪出现堆积。虽然会阴下垂和腹部下垂是两个独立的改变,但是这两个部位的抽脂常可同时进行,有时还会联合大腿提升术。阴阜和大阴唇的抽脂范围必须局限,单纯为增加阴蒂的性快感而行阴蒂周边区域的抽脂应十分谨慎,此区抽取的脂肪总量不应超过100g。

小阴唇位于尿道口与阴道口入口的两侧,是两条片状黏膜组织,具有保持阴道口湿润,防止外来污染的作用,因此可维持阴道的自净作用。小阴唇正常宽度一般为1.5~2.0cm,长度为3.0(1.2~6.2)cm,立位时两侧小阴唇贴拢于两侧大阴唇之间。小阴唇过度发育时可呈现肥大,高出大阴唇1cm以上时,可因行走摩擦引起不适、疼痛并影响性生活,造成性交疼痛,也会影响尿流方向等。

小阴唇肥大通常是先天的,如使用性激素、局部持续牵拉、慢性炎症长期刺激、过度手淫等,多次性交也会引起小阴唇肥大。过于肥大的小阴唇可能会给一些女性在心理上造成压力,影响其正常生活。

超过大阴唇1cm以上,行走时由于摩擦而感到不适可考虑手术。

直接切除小阴唇肥大的突出部位,直接缝合,操作方法简单易行。由于小阴唇肥大多是因局部慢性炎症长期刺激造成组织增生或淋巴管阻塞等病变引起,肥大的小阴唇部分组织发生明显的病理变化,并失去原有的柔软特性,采用直接切除缝合术在缩小小阴唇的同时又能去除病理组织,也是一种很好的选择,但会形成一条切口瘢痕线;另外,小阴唇边缘深褐色部位被切除后,小阴唇外形也会变得不美观。

手术采用局部浸润麻醉,切口一般设计在小阴唇外缘,长度与小阴唇一致,内侧缘切口高于侧缘切口线0.3~0.5cm,使切口线位于小阴唇外侧。去除适量小阴唇后,用3-0线拉拢缝合,外涂抗生素软膏,保持外阴部清洁。

此法具有术后小阴唇外缘不会被破坏,外形美观,缩小程度理想,瘢痕隐蔽等优点。

切口设计:小阴唇上方为蒂,蒂部宽度为1.2~1.5cm,向下画三角形去除区,舌形瓣远端宽0.5cm,单蒂舌形瓣长宽比例可达(3~4)∶1,形成上宽下窄的小阴唇舌形瓣,近阴道口处的切口线与小阴唇基底线平行。

手术操作:用0.5\%利多卡因局部浸润麻醉。用眼科剪剪除画线设计要去除的小阴唇组织,形成小阴唇舌形组织瓣,电凝止血,黏膜下层用4-0可吸收线间断缝合,使小阴唇舌形组织瓣远端固定在下端相应的位置,两侧切口用6-0可吸收缝线间断或连续缝合,力求两侧对称一致。术后小阴唇的高度应与大阴唇相平行或低于大阴唇。

术后处理:术后外涂抗生素软膏,无菌敷料加压包扎,必要时可用缝线打包加压包扎。每次排尿后应用0.1\%新洁尔灭液清洗外阴及阴道口,并用消毒纱布拭干,外涂抗生素软膏。术后3天复查换药。需拆线者一般7天后拆线。

术后并发症:切口渗血、局部血肿,缝线裂开等,预防主要是止血彻底,对一些渗血明显者,可采用植皮打包加压包扎法固定,可有效防止术后出血与血肿。

多由于全身或下身过度肥胖,少数为淋巴水肿引起,除全身应进行减肥治疗外,局部可实行美容整形手术。

切口设计:在大阴唇中最厚的部分,设计与大阴唇轴线一致的梭形切口线,两侧可以根据大阴唇肥厚的情况画出需要去除的组织。

部分切除术:按设计线切开皮肤全层及皮下组织,同时剪切大阴唇多余的皮下脂肪组织,要注意保留前庭大腺和圆韧带组织,严密止血后修整皮缘,间断缝合皮下组织与皮肤。术中不要过多去除大阴唇的皮肤组织,以防术后因牵拉而影响小阴唇的自然贴合。术后留置导尿管,厚敷料加压包扎,7~8天后拆线。

局部脂肪抽吸术:局部肿胀麻醉后,用2mm吸脂反复抽吸,可使大阴唇变薄变小。

多因身体过于消瘦,老化或内分泌失调引起。

非手术治疗主要包括加强营养,配合治疗内分泌疾病,局部外用丰乳霜或含雌激素的霜剂2~3个月;如非手术治疗无效或对大阴唇外形有较高要求者可考虑局部自体脂肪颗粒注射填充术,有较好的效果。

在处女时,分开小阴唇,可见处女膜封闭大部分阴道口,处女膜孔形态分为椭圆形(唇形)、伞形、环形与筛孔状等数种,处女膜的大小差异明显,从1mm到可容纳一个指尖。

处女膜多在第一次性交时破裂,一般不会再愈合。因性生活造成的处女膜破裂,破裂口常呈“梅花瓣”状,以4截石位点和8截石位点破裂口为较大,其他裂口较小,如花瓣状。外伤造成的处女膜破裂常不规则,有大有小,截石位6点多见(如骑跨伤等)。

手术方法

适用于裂口边缘整齐,无明显瘢痕增长,处女膜厚度2mm以上者。手术在裂口缘内外侧中央纵行剖开,适当向两侧分离,两侧切口在基底部相连。手术可用尖刀切开,或切开一小口后用小尖剪剪开,然后把处女内外层分别对缘缝合。

适用于处女膜较厚、边缘不整齐、瘢痕增生明显和再次破裂的裂口。术中用刀或小剪平行去除部分破裂缘的处女膜组织和增生的瘢痕组织,形成新的裂口缘创面,适当分离阴道内外层后分别缝合。为增加创面的接触面积,防止内外层缝合线的重叠,去除处女膜边缘组织时可考虑作斜行或梯形切除,但两侧缘形成的新鲜创面应相互对应。

适用于各种处女膜破裂者,尤其是处女膜较薄,破裂部位较多以及处女膜组织缺损者。术中在近破裂裂口一侧的内缘和对侧的外缘作纵向切口。在黏膜下层向裂口缘剖离,形成两侧的处女膜黏膜瓣,一层作为衬里,一层作为覆盖,将两侧黏膜瓣做瓦合重叠后,按阴道内外层缝合。

以上三种处女膜破裂修补的缝合可选用间断和连续缝合法。为防止创缘内翻影响创口愈合,间断使用褥式缝合法。可选用5-0丝线、7-0尼龙线或6-0可吸收线,但以6-0可吸收线为首选,术后无需拆线。缝合时遵循由内到外、由基地部到处女膜阴道缘的顺序,这样可确保伤口愈合的质量,达到以后性生活出血的目的。

术后每日用0.1\%新洁尔灭液清洗,保持阴道口及外阴部清洁,洗后用抗生素药膏以防感染。常规留置尿管。来月经后注意是否通畅,如流出不畅应及时处理。用尼龙线或丝线缝合者5~7天拆线。

阴蒂肥大是较常见的外阴先天性畸形,它可单独存在,也可与小阴唇融合并存。阴蒂肥大是女性假两性畸形的主要诊断依据之一。

凡两性畸形经诊断确定向女性方向矫治而存在肥大的阴蒂者;或发育不良的小阴茎者均可将此阴蒂或小阴茎切除。关于手术时机,大多数学者认为,青春期或以后的患者。经确诊的病例应及早手术,拖延日久对性发育不利。

截石位,局麻。于阴蒂背侧皮肤作“工”形切开,向两侧分离形成两个皮瓣,分别折叠缝合形成小阴唇上份。仔细分离阴蒂背侧神经血管束至阴蒂根部,切除部分阴蒂海绵体,楔形切除部分肥大阴蒂头,将其缝合于阴蒂根部,形成较为正常大小的阴蒂,同时,阴蒂头部的感觉功能能得以保留。

视情况留置导尿管2~3天,保持外阴清洁,创口每日换药,大小便后冲洗外阴。

术后局部敷冰袋2~4天,可减少局部出血和血肿,酌情用抗生素预防感染,术后5~7天拆线。

出血及血肿:处理好阴蒂背动静脉是关键。血管结扎要切实可靠。阴蒂头剥离时应轻巧、准确。局部水肿:术中操作粗暴,组织损伤过多所致。术后注意伤口疼痛及肿胀,用压迫法或热敷,可以消除。

其他还有阴蒂重建术、阴蒂包皮成形术等。

在阴道松弛、子宫脱垂的患者中,已婚经产妇占总数99.9\%,年龄以30~50岁为最多,占妇女总数9\%左右。阴道分娩会不同程度地损伤盆地组织,特别是随着年龄增长,卵巢功能减退,雌激素分泌减少,使筋膜等支持结构发生退行性变,肌肉张力下降,黏膜萎缩,因而使阴道变得松弛,缺少弹性,部分患者出现阴道或膀胱膨出,进而造成张力性尿失禁,排尿困难或反复泌尿系统感染,直肠膨出导致大便困难等,给患者造成极大的痛苦。阴道松弛致使女性在性生活时对刺激反应迟钝或不反应,很难达到性高潮,久之导致性冷淡,严重者引起夫妻感情破裂。

阴道紧缩术就是针对女性上述生理变化,为提高夫妻性生活质量而设计的一种妇科整形手术。根据患者的不同年龄、阴道松弛及会阴损伤的不同程度进行紧缩及修补,通过手术修复损伤和松弛的盆底组织,使阴道前后壁得到加强,阴道弹性得到改善,裂伤的会阴也会达到产前状态,同时,外观也得以改善,恢复女性的自信心。另外,阴道紧缩能起到预防和治疗因盆底组织松弛而导致的子宫脱垂和尿失禁等症状。

(1)因阴道松弛影响夫妻性生活质量者。

(2)子宫脱垂半阴道前壁和(或)阴道后壁膨出,有临床症状者。

(3)陈旧性会阴裂伤伴阴道松弛者。

(4)不再阴道分娩者。

(1)全身状况不良,如患有严重的心脏病、高血压、肾炎、肝功能损害、甲亢、糖尿病、肺功能不全、哮喘、各种恶性肿瘤、出血性疾病、严重贫血、精神病等不宜做该手术。

(2)各种阴道炎(滴虫、霉菌及老年性)和外阴炎、盆腔炎及重度宫颈糜烂等,治愈后方可手术。

(3)月经期、妊娠期、哺乳期及绝经2年以上者不宜做该手术。

月经干净后3~7天为最佳手术时间,手术时间最迟应距下次月经来潮前2周进行,以利伤口愈合和减少感染发生。

(1)手术前3天开始用0.1\%新洁尔灭溶液阴道灌洗,1次/天。

(2)手术前1天进食易消化食物。

(3)手术前1天备皮,做青霉素皮试。

(4)术前晚清洁灌肠,术晨禁食水。

取截石位,于6点处作一菱形切口,远端达阴道中段,切除一块菱形阴道黏膜和裂伤的部分会阴部瘢痕皮肤,分离出断裂的肛提肌、球海绵体肌,缝合撕裂的肌肉,以恢复这些肌肉的收缩力,同时缝合阴道后壁肌层组织。若肛门括约肌也见部分撕裂,亦应重新拉紧缝合,同时缝合撕裂的会阴联合,以增加阴道口的紧缚力。

(1)预防感染。

(2)每日测血压、脉搏、体温1~2次,注意观察尿色、尿量及阴道出血情况。

(3)术后3天内进半流质饮食,3天后改普通饮食。

(4)术后3天不能自解大便者,给予服用液体石蜡30ml/天,或服用果导片2片/次,1~2次/天。保持大便通畅。便后新洁尔灭擦洗外阴。

(5)术后24~48小时拔除尿管和阴道纱布,插导尿管期间,每日换无菌尿袋1次。

(6)术后第3~5天拆除会阴皮肤缝线。

(7)手术2月后方可同房。

(1)出血和血肿

原因:术野剥离面过大,止血不彻底,结扎不牢靠,或手术操作粗暴,组织损伤过多所致。

临床表现:术后阴道有鲜红色血液流出,或有下坠感和憋尿感,后壁血肿肛指检查可触及血肿包快。

处理:少量出血或小血肿可用纱布填塞阴道,局部压迫止血即可,大量出血或大血肿,应及时拆开缝线,清理积血,找到出血点重新缝扎,彻底止血,术后加强抗感染治疗。

(2)伤口感染

原因:主要因体质差,阴道炎症未治愈,术前阴道准备不充分,或术中无菌操作不严格所致。

临床表现:阴道异常分泌物,有臭味,重者出现体温升高,白细胞计数升高等全身症状。

处理:加强全身抗感染治疗,局部换药2次/天,保持外阴清洁,多可治愈。

(3)伤口裂开

原因:切除组织过多导致伤口张力过大,创缘对合不好,多早做剧烈活动,术后便秘及伤口感染等。

临床表现:多在剧烈活动或便秘时出现阴道不等量活动性出血。

处理:立刻重新缝合,加强抗感染操作,卧床休息。

(4)周围脏器损伤

原因:术者解剖层次不清,手术操作粗暴,钳夹组织过多,缝合过深等,导致膀胱、直肠或尿道损伤。

临床表现:术中尿液或粪便溢入阴道,或术后一段时间阴道内有异常分泌物,臭味等。

处理:术中发现损伤者立即手术修补。术后一段时间发现者择期修补。

(5)阴道狭窄

原因:手术设计不妥,切除黏膜范围过大,或感染后创面挛缩所致。

临床表现:性交痛,重者不能性交。

处理:轻度者用阴道扩张器扩张,重度者行松解术或植皮。

注意保持外阴清洁,大便后用0.1\%新洁尔灭纱布拭擦。术后用碘仿纱条填塞阴道,5~7天后换药。黏膜处不必拆线,会阴部缝线7天拆除。术后2月内禁止性生活,以免再次造成阴道撕裂。

由于手术后阴道的扩张性受到影响,故该手术后阴道分娩有一定的危险性,若再生产,以剖宫产为宜。

先天性无阴道(Congenital Absence of the Vagina)或阴道闭锁(Obliteration of the Vagina)是一种先天性畸形,其发生率约为1/5000,临床表现主要为青春期无月经初潮,或初潮后出现周期性下腹疼痛,或婚后性交不能。此外,还有部分两性畸形患者要求行女性化手术(Feminizing Surgery),或男性易性症患者要求行女-男(F-M)性别重塑手术(Sex Reassignment Surgery,SRS)等,均需行阴道再造术。手术主要包括在膀胱和直肠间形成腔穴和腔壁衬里的重建,衬里重建的材料一般有上皮组织自然长入,带蒂肠袢或腹膜转移,皮片或羊膜游离移植,邻近或游离皮瓣移植等。各种方法均有其优缺点,有的方法已淘汰,如上皮组织长入等。临床上应根据手术者的掌握程度和患者的实际情况决定采用何种方法。本章主要讨论整形外科常用的皮瓣移植法,其余方法见有关章节。

(1)术前准备

手术前按照肠道手术准备。

手术前三天每天清洁会阴部。

手术前一天开始进流质饮食,会阴部及供皮区备皮。

手术前一天服用广谱抗生素及肠道抑菌药物。

手术前晚和术晨作清洁灌肠。

必要时应用多普勒探测皮瓣供血动脉。

(2)术后处理

术后留置导尿管1~2周,并行膀胱冲洗。

术后三天进流质饮食,后可逐渐改为无渣半流质饮食。

术后保持大便通畅,必要时可服用缓泻剂及清洁灌肠。

术后10~18天可取出阴道内填塞的碘仿纱条,并及时更换敷料。

术后12~14天可根据皮瓣成活情况逐步拆线,并清洁阴道。

术后需长期佩戴阴道模具6~12个月直至有正常的性生活,以防止再造的阴道挛缩。

(1)阴股沟岛状皮瓣阴道再造术

(2)下腹壁岛状皮瓣阴道再造术

(3)脐旁皮瓣阴道再造术

(4)上腹壁岛状皮瓣阴道成形

(5)股薄肌肌皮瓣阴道再造术

(6)短股薄肌肌皮瓣阴道再造术

(7)腹直肌肌皮瓣阴道再造术

(朱辉)

1.Bergeson P S,Hopkin R J,Bailey R B J r,et al.The incon-spicuous penis.Pediatrics,1993,92:794-798

2.魏辉,梅骅.隐匿阴茎的分类和手术治疗.临床泌尿外科杂志,2003,18(2):102-103

3.余墨声,赵月强.成人隐匿阴茎的外科矫正.武汉大学学报:医学版,2003,24(3):286-288

4.张国强,杨波.隐匿阴茎的分型及处理原则.临床泌尿外科杂志,2003,18(1):34-35

5.马成海,袭燕.小儿隐匿阴茎的诊断与治疗.临床泌尿外科杂志,1999,14(12):528-531

6.吴阶平,裘法祖.黄家驷外科学.第5版.北京:人民卫生出版社,1996:1783

7.余墨声,龙道畴.蹼状阴茎的矫正.临床外科杂志,1998,6(3):147-148

8.罗洪,江晓海.阴茎阴囊转位3例报告.中华男科学杂志,2000,6(3):201-202

9.陆文奇,谭志忠.阴茎根部阴囊两侧翼上方开窗术治疗阴茎阴囊转位.广西医科大学学报,2001,18(6):862-862

10.陈安屏,刘毅东.阴茎固定术治疗阴茎扭转(附3例报告).中国男科学杂志,2007,21(5):56-56

11.Culp OS.Struggles and triumphs with hypospadias and associated anomalies:review of 400 cases.J Urol,1966,96(3):339,351

12.Ebbeh J,Metz P.Congenital penile angulation.BrJ Urol,1987,60(3):264-266

13.杨昕,金景平.先天性无尿道下裂阴茎弯曲症.综合临床医学,1995,11(2):84-85

14.梁建波.先天性阴茎弯曲(无尿道下裂)的外科治疗.临床泌尿外科杂志,1996,11(5):291-293

15.庄乾元,韩见知.先天性泌尿生殖系疾病.武汉:湖北科技出版社,2001:280-291

16.朱辉,宋博.重度先天性阴茎弯曲———附1例报告并文献复习.罕少疾病杂志,2002,9(4):12-13

17.刘继红.男科手术学.北京:北京科技出版社,2006,72-89

18.朱选文,钟达川.阴茎弯曲的诊断及外科治疗(附25例报告).中国男科学杂志,2008,22(3):19-21

19.张滨.性医学.广州:广东教育出版社,2008:458

王炜.整形外科学.杭州:浙江科学技术出版社,1999:1588-1599

20.辛时林,易传勋.整形外科手术图谱.第2版.武汉:湖北科学技术出版社,2001:694-697

21.王传家,陈茵.改良阴茎延长加增粗术临床应用分析.中国美容医学,2008,17(7):966-967

22.李建宁.阴茎延长整形手术450例临床分析.中华男科学杂志,2007,13(11):1037-1038

23.吴小蔚,程邦昌.阴茎大部分缺损的治疗.临床外科杂志,2005,13(3):178-180

24.陈小萱,张金明.阴茎延长术的临床应用体会.伤残医学杂志,2002,10(4):22-24

25.赵月强,陕声国.阴茎延长术后并发症的分析与防治.临床外科杂志,2003,11(6):409-410

26.丁自海,马全福.阴茎延长术的解剖学基础.中国临床解剖学杂志,1993,11(1):44-46

27.龙道畴,陕声国.阴茎延长术的临床研究.中华整形烧伤外科杂志,1990,6(1):17-19

28.Austoni E,Guarneri A,Garti G.Penile elongation and thickening a myth?Is there a cosmetic or medical indication.Andrologia,1999,31(Suppl1):4

29.WESSELLS H,LUE T F.MCANINCH J W.Complications of penile lengthening and augmentation seen at l referral center.J Urol,1996,l55(5):1617-1620

30.ALTER G J.Augmentation phalloplasty.Urol Clin North Am,1995,22(4):887-902

31.金东浩,尤子龙.硅橡胶假体植入阴茎增粗术.中国美容医学,2005,14(4):436-437

32.杨斌,刘小容.两种阴茎海绵体增粗手术方法的探讨.中国实用美容整形外科杂志,2004,15(4):171-173

33.杨斌,梁伟强.人工血管补片置入阴茎海绵体白膜阴茎增粗术.中华整形外科杂志,2003,19(4):309-310

34.汪灏,李森恺.经包皮内板切口阴茎延长增粗术的研究.中国实用美容整形外科杂志,2006,17(1):21-23

35.程开祥,刘阳.植入膨体聚四氟乙烯(e-PTFE)阴茎增粗术.中国美容医学,2007,16(9):1199-1201

(龙云 朱辉)


\chapter{第二十二章 性传播疾病}

传统的性病是指通过直接接触而传染的全身性传染性疾病,包括梅毒、淋病、软下疳、性病性淋巴肉芽肿和腹股沟肉芽肿5种疾病。1975年,世界卫生组织提出性传播疾病(Sexual transmitted Disease,STD)的新概念。性传播疾病是指以性接触为主要传播方式的一组疾病。国际上将20多种通过性行为或类似性行为引起的感染性疾病列入性传播疾病范畴,包括传统的性病、尖锐湿疣、生殖器疱疹、艾滋病等。

性传播疾病主要传染途径是通过性行为使病原体侵入人体传染的,通过此方式传播的感染占全部病例的95\%左右。性行为包括阴道性交、肛门性交、口-生殖器接触等所有方式。

性交时,性器官处于充血状态,组织压力孵加,性腺兴奋,分泌旺盛,腺口开放;性交动作导致皮肤、黏膜表面发生损伤或微小破损,若性交一方带有病原菌,则易导致对方被传染。

性传播疾病还可以通过间接途径传播,如污染的衣裤、被褥、浴盆、马桶、公共浴池、注射器针头、医疗器械等。通过此种方式传染机会较少,但淋病通过间接接触传播的可能性较高。

血液及血液制品是艾滋病及梅毒等的传播途径之一,尤其是艾滋病通过此种方式传播的病例较多。

通过胎盘或分娩传染给婴儿。先天性梅毒、艾滋病等可通过胎盘传染胎儿。淋病可通过分娩传染新生儿。

性传播疾病传染性很强,对人体健康威胁比其他传染病要大很多。其病变不仅仅限于生殖器,还可侵犯内脏器官,如心血管系统、免疫系统、血液系统等。艾滋病患者由于免疫功能低下易引起机会感染、恶性肿瘤、多器官多系统损害,最后大多数死亡。同时,部分患者出现严重心理问题,如抑郁、焦虑、强迫、失眠、性功能障碍等。

性传播疾病首先可传染给配偶及子女,导致家庭内感染。同时,易导致家庭矛盾,破坏家庭的和谐、美满。

该病的泛滥往往伴随社会犯罪率的增加。如艾滋病与注射毒品密切相关,毒品又与国际、国内贩毒集团相连。艾滋病高发国的兵员及战时血源供应均受到严重影响。因此,性传播疾病对社会影响相当严重,必须高度重视。

性传播疾病是主要通过性行为传播的疾病。近年来,各种性传播疾病在我国蔓延已是不争的事实。尤其是被称为世纪瘟疫的艾滋病,在我国已进入迅速发展期。防治性传播疾病已成为全社会的严峻任务。性病是一种传染病。除了艾滋病,只要及时进行正规的治疗,现代医学完全可以治好。性传播疾病的防治工作是一项复杂而艰巨的系统工程,只有在政府和社会的统一领导和具体规划下,组织各相关部门分工合作,综合治疗,并充分调动个人的积极性,才能取得成效。

社会和政府应大力做好性传播疾病的防治工作。①打击卖淫、嫖娼、取缔淫秽书刊网站,加强对宾馆、酒店、浴池等公共场所的管理。②大力开展对性传播疾病知识的宣传教育,充分利用各种宣传形式,使广大人民群众了解性传播疾病的传播途径、症状、防治方法以及对个人的身心健康、家庭、社会的严重危害性。③建立健全性传播疾病防治机构,培训医务人员和防治技术骨干。性病的防治与监测必须实行专业机构与基层医疗预防保健机构相结合,形成网络,扩大覆盖面,才能早期发现患者,及时掌握疫情,落实防治措施。同时,必须加强相关人员的业务培训工作,培养较高水平的从事性病防治工作和研究工作的专业技术骨干,在医学院校内增加有关性病的教学内容,不断提高性病防治工作的质量和科学技术水平。④早期发现患者,及时规范化治疗。除医院门诊外,对高危人群如暗娼、嫖客、多性伴者、同性恋等重点检查。一旦发现,正规治疗。⑤强化个人防护措施,实行一夫一妻制,提倡洁身自爱,避免婚外性行为。同时,注意避免通过间接途径传播。公共浴池应提供清洁衣物和纸袋,旅馆应及时清洁被褥和马桶。

对个人而言,最好是增加有关性病的知识,构筑预防性病的防线,把性病拒之身外。

第一,最好的预防是不接触病原。①既然性行为是性病传播的主要途径,那么只要坚守洁身自爱,避免婚前和婚外性行为,只与一个性伴侣发生性关系,就基本上能预防性病。这是预防性病最基本最有效的办法。性行为不仅指男女性交,肛交、口交,同性恋同样可以传染性病。甚至性器官皮肤的直接接触以及手和性器官的接触也属于性行为,也可以间接传染性病。②艾滋病和梅毒还可以通过血液传播,所以接受输血和血液制品要特别慎重。万不得已时也一定要在医生指导下使用,千万勿将人血白蛋白、丙种球蛋白等血制品当做补品而滥用。处理伤口、注射药物一定要到正规医院去,最好用一次性注射器。③一些容易被忽略的个人生活用品如牙刷、剃须刀和刮脸刀、浴巾必须自己独用。这是因为刮脸刀极容易造成皮肤细小破损而成为传染艾滋病、其他性病和肝炎的中介。还有最重要的一点就是不吸毒,坚决远离毒品。

第二,进行性行为前的预防。性欲是人的基本欲望,在某些情况下总会有些人发生婚前、婚外性行为和同性恋行为。从防止性病蔓延的医学角度出发,要正确使用合格安全套,可使传染病几率下降。这是因为各类性病最容易通过生殖器直接接触传染,安全套隔离了双方的生殖器皮肤黏膜,从而大大降低了大多数性病的传染几率。还要事先检查确认安全套有无漏洞,排除气泡,正确并全程使用。

第三,性行为之后采取的补救措施。①性交后要及时仔细、反复地清洗阴茎、阴道、阴部以及手与口。因为各种性病病原体传到对方身上后,需要相当的时间才能进入皮内繁殖和感染。②必要时可用杀菌剂冲洗阴道、阴茎以及可能沾上对方分泌物的部位。这也是一种比较有效的补救措施,对淋病、非淋菌性尿道炎和阴道炎有效。


\section{第二节 梅 毒}

梅毒(Syphilis)是由梅毒螺旋体引起的一种慢性性传播疾病,可侵犯皮肤、黏膜和心血管系统、神经系统等重要器官。可分为先天梅毒和后天梅毒两种。先天梅毒是由患梅毒的孕妇血中的梅毒螺旋体通过胎盘传染胎儿引起。后天梅毒多由性交直接传播,也可由血液或血液制品传播。

梅毒螺旋体亦称苍白螺旋体(Treponemiapallidum,TP),在分类学上属螺旋体体目(Spirochaetales),密螺旋体科(Treponemataceae),密螺旋体属(Genus Treponema)。菌体细长,带均匀排列的6~12个螺旋,长5~20μm,平均长6~10μm,横径0.15μm左右,运动较缓慢而有规律,实验室常用染料不易着色,可用暗视野显微镜或相差显微镜观察。

梅毒螺旋体感染后产生感染性免疫,感染2周后产生特异性IgM抗体,此型抗体不能通过胎盘;感染后4周出现特异性IgG抗体,可通过胎盘。完全治愈的早期梅毒可再感染。另外,TP破坏人体组织,使组织释放一种抗原性心脂酶,刺激机体产生反应素,用RPR、USR、VDRL等方法可检出,在感染TP后5~7周或出现硬下疳后2~3周转阳性。

人体外存活力低,40℃时失去传染力,56℃时煮沸3~5分钟立即死亡;耐低温;潮湿的生活用品上可存活数小时,不耐干燥。对肥皂水和常用消毒剂(70\%乙醇、0.1\%石碳酸、0.1\%升汞等)敏感。

梅毒在全世界流行。据WHO估计,全球每年约有1200万新发病例,主要集中在南亚、东南亚和非洲。新中国成立前是中国四大性病之首,20世纪60年代初基本被消灭,80年代再次发生和流行。1991年报告病例数为1870例,1995年11336例,1997年33668例。近年来呈明显增多趋势,临床经常可见一、二期梅毒,三期梅毒和先天梅毒也有发现。在义务献血员中发现隐性梅毒。

梅毒是人类独有的疾病,显性和隐性梅毒患者是传染源。感染TP的人的皮损分泌物、血液中含大量TP。感染后的前2年最具传染性,2年后基本不通过性传播。

性接触是梅毒的主要传播途径,约占其95\%以上。感染TP的早期传染性最强。如果是显性梅毒,可在性行为接触的任何部位发生硬下疳,如生殖器、直肠、肛周、乳头、舌、咽、手指等部位。随着病期的延长,传染性越来越小,一般认为感染2年以上性接触就不再有传染性。

患有梅毒的孕妇可通过胎盘传染给胎儿,起胎儿宫内感染,多发生在妊娠4个月以后,导致流产、早产、死胎或分娩感染梅毒的胎儿。一般认为孕妇梅毒病期越短,对胎儿感染的机会越大。感染后2年仍可通过胎盘传给胎儿。

梅毒螺旋体也可以间接接触传染,通过接吻、哺乳和被患者分泌物污染的衣裤、被褥等日常用品造成传播。

根据感染时间和临床特点,梅毒均可分为早期梅毒和晚期梅毒。早期先天梅毒发生在2岁以下的儿童,晚期先天梅毒发生在2岁以上的儿童。早期后天梅毒病程在2年以内,晚期后天梅毒病程在2年以上。

通常发生于性交后2~4周,主要症状为硬下疳与局部淋巴结肿大。螺旋体入侵部位出现黄豆大、浸润性、无痛性硬结,称为硬下疳。硬下疳为其特征性表现,通常单发,也可为2~3个,直径1~2cm,圆形或椭圆形。硬下疳出现不久后表现即糜烂或浅溃疡,周围隆起,基底清洁硬如软骨,组织液内有大量梅毒螺旋体,具有很强的传染性。如不经治疗,硬下疳一般在3~5周自然消失,不留痕迹或仅有色素沉着。

硬下疳好发于男性阴茎的包皮、冠状沟、系带或龟头上;女性则在大小阴唇、子宫颈;男性同性恋可出现在肛周、肛门部或直肠,也可在口唇、舌、手指等处。

硬下疳出现1~2周后,可出现腹股沟淋巴结肿大,可以是同侧,也可以是两侧。其特征是无痛性、较硬、彼此不融合,无红、肿、热,淋巴结穿刺液有大量梅毒螺旋体。淋巴结肿大消退较硬下疳晚。病程初期,梅毒血清反应呈阴性,以后逐渐升高,7~8周后大部分患者呈阳性反应。

梅毒螺旋体经血行播散引起,传染性很大。通常发生在感染后8~10周,或硬下疳出现后6~8周,患者梅毒血清反应100\%阳性。前驱症状呈流感样综合征,如低热、头痛、四肢酸痛等,也可出现全身淋巴结肿大。数日后出现皮疹,可表现为斑疹、丘疹、玫瑰糠疹、多形红斑样疹、毛囊疹、斑丘疹、丘疹鳞屑性梅毒疹、脓疹、蛎壳状疹等。皮疹边界清楚,呈铜红色,压之不退色,有轻度浸润,多对称分布。全身皮肤均可受累,以掌跖部铜红色斑有诊断价值。自觉症状较轻,破坏性较轻,但传染性较强。1~2周后皮疹消退,消退后可有色素脱失。

发生于外阴、肛周、腋下的丘疹呈扁平增殖样隆起,表面湿润,又称为扁平湿疣,其内含有大量梅毒螺旋体。生殖器部位、口腔、咽喉部黏膜也可受累,表现为红肿及糜烂,如累及声带可出现声音嘶哑。黏膜损害内有大量梅毒螺旋体,具有传染性。部分患者可出现梅毒性脱发,呈虫蚀状,为一过性,多发生在颞部。梅毒骨关节损害表现为骨膜炎及关节炎,晚间疼痛较重,白天较轻。眼部症状表现为虹膜炎、虹膜睫状体炎、脉络膜炎、视神经炎和视网膜炎等。神经系统通常无症状,但脑脊液有异常变化;也可有脑膜炎、脑血管梅毒等。

因患者免疫力低下或治疗剂量不足,二期损害消退后重新出现,典型表现为皮疹呈环形、弧形、花环状排列,可同时伴有黏膜、眼、骨等的复发。发生在感染后1~2年内,梅毒血清反应为阳性,也可只表现为血清复发,无其他症状。

多于感染2年后发病,主要原因是早期梅毒未经治疗或治疗剂量不足。三期梅毒无传染性,但组织器官破坏性较大。皮损于感染后5~10年发生,表现为结节性梅毒疹和树胶肿。结节梅毒疹特征为粟粒大小、棕红色、浸润性,数目较少不对称,集簇成群,呈环形、花环状、蛇形排列。好发于躯干、四肢及面部,可自行吸收,遗留浅癍痕。树胶肿多发生在前额、头、四肢伸侧、胸骨等处,初始为深达皮下的豌豆大的浸润性硬结,逐渐形成肾形或马蹄形溃疡,周围有色素沉着,愈后有萎缩性癍痕。

三期梅毒常累及黏膜,引起鼻中隔及硬腭穿孔,可侵犯骨质,死骨排出后形成鞍鼻。骨梅毒表现为骨膜炎、树胶肿性骨炎及骨髓炎。眼梅毒表现为虹膜睫状体炎、间质性角膜炎和视网膜炎等。

晚期心血管梅毒,发生于感染后15~30年,可同时合并神经梅毒。表现为梅毒性单纯性主动脉炎、梅毒性主动脉闭锁不全、梅毒性主动脉瘤、梅毒性冠状动脉狭窄,严重时可发生充血性心力衰竭,甚至死亡。梅毒主动脉瘤多发生于升主动脉及主动脉弓部,呈梭形,可有压迫症状,严重者可突然破裂,导致死亡。梅毒性冠状动脉狭窄常并发梅毒性主动脉闭锁不全。

晚期神经梅毒可发分无症状神经梅毒及脑血管梅毒、脑实质梅毒。无症状神经梅毒无临床症状和异常体征,脑脊液检查检查白细胞和蛋白增多,梅毒反应阳性。脑血管梅毒发生感染后5~12年,可发生灶性神经系统表现,尤其是偏瘫和失语;脑脊液表现同无症状神经梅毒。脑实质梅毒分为麻痹性痴呆、脊髓痨、视神经萎缩三类。麻痹性痴呆发生感染后15~20年,为脑膜脑炎并小脑受累。精神症状表现为性格变化,注意力不集中,智力及记忆力衰退,情绪变化无常,幻想,抑郁。还可有唇、舌、手震颤、口吃,癫痫,四肢瘫痪,阿罗瞳孔(对光反应消失,调节反应存在)、大小便失禁等。95\%~100\%患者梅毒血清试验阳性,大部分患者脑脊液VDRL(玻片法梅毒血清反应)试验阳性。脊髓痨发生于感染后20~25年,系脊髓后索发生变性所致,表现为闪电样痛,下肢感觉异常,腱反射减弱或消失,触痛觉及温度觉障碍,深感觉减退或消失,共济失调,关节炎,排尿困难,尿潴留,性欲减退,内脏危象(胃、肠)。70\%~80\%患者梅毒血清反应阳性,脑脊液检查VDRL阳性。视神经萎缩较为罕见,表现为进行性视力丧失,开始为一侧,以后对侧也可发生。

二期梅毒不经治疗而症状自然消失,进入潜伏状态,但梅毒血清试验阳性,称为潜伏梅毒。三期梅毒部分患者不出现晚期梅毒症状,仅表现为梅毒血清反应持续阳性,称为晚期潜伏梅毒。

2岁以内发病,有传染性。症状发生在出生后不久。患儿瘦小、发育差、声音嘶哑,常有梅毒性鼻炎,口腔有黏膜斑。皮疹与二期梅毒疹相似,以脓疮多见。口周、肛周、掌跖可有大片浸润性红斑,要有大疱或糜烂。梅毒新生儿皮肤表现干皱状;还可有脱发、甲沟炎、甲床炎,手指呈梭状肿胀;发生骨骺炎时可出现假性瘫痪。淋巴结、肝脾肿大,贫血,血小板减少。梅毒血清反应阳性。

2岁以后发病,极少侵犯心血管和神经系统,皮疹与后天晚期梅毒疹相似。晚期先天梅毒有三大特征:①实质性角膜炎,5~20岁出现,双侧角膜有较深浸润,可影响视力;②神经性耳聋,10岁左右出现,突然发病,多为双侧;③牙齿,恒齿短,稀疏,排列不齐,第一对上门齿切缘呈弧状凹陷。

先天梅毒未经治疗而无临床症状,梅毒血清试验阳性,2岁以内者为早期潜伏梅毒,2岁以上者为晚期先天潜伏梅毒。

梅毒患者妊娠或妊娠期间感染梅毒称作妊娠梅毒。20\%~40\%患活动梅毒的妇女不孕。妊娠可加重梅毒,易发生骨关节、心血管和神经梅毒,常有全身症状如发热、关节痛、肌无力、贫血等。患早期梅毒的孕妇,在妊娠4个月左右可传给胎儿,或导致死胎或流产、早产或死产。

病史中有无不洁性交史,配偶及性伴侣有无梅毒,已婚妇女有无早产、流产或死产史,本人是否患过性病,有无梅毒史,是否发生过硬下疳、二三期梅毒症状。如有梅毒史,是否按疗程规范治疗。

应全面检查,感染期较短患者注意检查皮肤、黏膜、外阴、肛门、口腔等处。感染较长的患者还应检查心血管、神经系统、眼、骨骼等。

暗视野显微镜检查,早期梅毒皮肤黏膜损害可发现梅毒螺旋体。梅毒血清试验分为非梅毒螺旋体抗原血清试验和梅毒螺旋体特异抗原血清试验两种。目前通用的螺旋体血清试验有TPHA、FTA—ABS和TPPA试验,非螺旋体梅毒血清试验有RPR、USR、TRUST。脑脊液检查对神经梅毒有意义,应包括细胞计数,蛋白测定,VDRL试验。

治疗前必须明确诊断,越早治疗效果越好。治疗剂量必须足够,疗程必须规范。治疗后要足够时间随访观察。一期、二期梅毒治疗应迅速使病损失去传染性,并达到临床治愈,血清反应转阴。部分三期梅毒患者虽经足量规范治疗,非螺旋体抗原试验也不转阴,不需继续抗梅毒治疗。传染源及配偶、性伴侣须接受检查或治疗,治疗前和治疗期间应禁止性交。

(1)青霉素疗法:①苄星青霉素G(长效西林),240万单位,分两侧臀部肌注,每周1次,共2~3次。②普鲁卡因青霉素G,80万单位/日,肌肉注射,连续10~15天,总量800万~1200万单位。

(2)对青霉素过敏者:①盐酸四环素500mg,4次/日,口服,连服15天。②强力霉素100mg,2次/日,连服15天。③红霉素,用法同四环素。

(1)青霉素:①苄星青霉素G,240万单位,1次/周,肌肉注射,共3次。②普鲁卡因青霉素G,80万单位/日,肌肉注射,连续20天。

(2)对青霉素过敏者:①盐酸四环素500mg,4次/日,口服,连服30天。②强力霉素100mg,2次/日,连服30天。

应住院治疗,如有心衰,首先治疗心衰;心功能代偿时,为避免吉海反应,小剂量开始注射青霉素,并在注射青霉素前一天口服强的松20mg,一天1次,连服3天。首日10万单位,1次/日,肌注;第二日10万单位,2次/日,肌注;第三日20万单位,2次/日,肌注;自第4日起按如下方案治疗:①普鲁卡因青霉素G,80万单位/日,肌注,连续15天为一疗程,共两个疗程,疗程间隔2周。②对青霉素过敏者:四环素500mg,4次/日,连服30天。

应住院治疗,为避免治疗中产生吉海反感,在注射青霉素前,口服强的松,每次20mg,1次/日,连续3天。

(1)水剂青霉素G,每天1200万~2400万单位。静脉点滴(每4小时200万单位),连续14天。然后苄星青霉素G,240万单位,1次/周,肌注,共3次。

(2)普鲁卡因青霉素G,每天240万单位,肌肉注射,同时口服丙磺舒每次0.5克,4次/日,共10~14天。必要时再用苄星青霉素G,240万单位,1次/周,肌注,连续3周。

(3)对青霉素过敏者:盐酸四环素500mg,4次/日,口服,连服30天。

(1)普鲁卡因青霉素G,80万单位/日,肌注,连续10天。妊娠初3个月内,注射一疗程,妊娠末3个月注射一个疗程。

(2)对青霉素过敏者,红霉素治疗,每次500mg,4次/日,早期梅毒连服15天,二期复发及晚期梅毒连服30天。妊娠初3个月与妊娠末3个月各进行一个疗程(禁用四环素)。但其所生婴儿应用青霉素补治。

(1)早期先天梅毒(2岁以内):脑脊液异常者:①水剂青霉素G,5万单位/公斤体重,每日分2次静脉点滴,共10~14天。②普鲁卡因青霉素G,每日5万单位/公斤体重,肌注,连续注射10~14天。脑脊液正常者:苄星青霉素G,5万单位/公斤体重,一次注射(分两侧臀肌)。如无条件检查脑脊液者,可按脑脊液异常者治疗。

(2)晚期先天梅毒(2岁以上):普鲁卡因青霉素G,每日5万单位/公斤体重,肌注,连续10天为一疗程(不超过成人剂量)。8岁以下儿童禁用四环素。

先天梅毒对青霉素过敏者可用红霉素治疗,每日7.5~12.6毫/公斤体重,分4次服,连服30天。

常发生首剂抗梅毒治疗后数小时,于24小时内消退。全身反应包括发热、全身不适、肌肉骨骼疼痛、恶心、心悸等。多见于早期梅毒,反应时硬下疳肿胀,二期梅毒疹加重或第一次出现二期梅毒损害。晚期梅毒发生率低,但反应较重,可有生命危险。治疗前口服强的松可减轻此反应。

经充分治疗,应随方2~3年。治疗后第一年每3个月复查一次,包括临床与血清试验,以后每半年复查一次。在此期间应密切观察血清反应滴度下降及临床改变情况,如无复发则可终止观察。

早期梅毒治疗后,如有血清复发或临床症状复发,应加倍剂量进行复治,还须做脑脊液检查排除中枢神经系统感染。血清复发指血清反应由阴转阳,或滴度升高两个稀释度。

与晚期潜伏梅毒治疗后血清反应固定,需随访3年以上判断是否终止观察。

早期梅毒治疗后在分娩前每月检查梅毒血清试验,如3个月内血清反应滴度不下降两个稀释度,应予复治。分娩后按一般梅毒患者进行随访。

治疗后3个月复查一次,包括临床、血清学及脑脊液;以后每6个月检查一次,直至脑脊液正常;此后每年复查一次,至少3年。


\section{第三节 淋 病}

淋病是通过性交传播的疾病,人类是淋球菌的唯一宿主,淋球菌对低等动物无致病能力。淋球菌不仅可产生破坏青霉素的酶,并通过染色体突变,产生耐药性,致使淋病的防治工作变得更加复杂。

淋病奈瑟氏菌又叫淋球菌,为革兰氏阴性双球菌。形态类似脑膜炎双球菌,能分解葡萄糖,不能利用乳糖、蔗糖。常用Thayer-Martin培养进行培养,部分需特殊氨基酸才能生长。

在急性期间,细菌多位于分叶核粒细胞内、外。急性期,淋球菌有排他性,不允许其他细菌存在。分泌物涂片通常只见淋球菌,很少有其他细菌。而在慢性期,淋球菌潜伏于前列腺或尿道球腺,涂片时不易见到。

淋球菌在35~37℃、pH为7.2~7.6、CO2 3\%~5\%、适当的湿度环境下容易生长。淋球菌最怕干燥,在干燥环境下或经风吹日晒后1~2小时内即死亡。

淋球菌表面有菌毛,根据菌毛的多少可分为P++ 或P+ 。在细菌衰老时,菌毛可消失,称为P-。菌毛由多肽组成,有抗原性,其终端氨基酸序列较固定,而中段及羧基端可发生变异,因而抗原性也不同。菌毛也有抑制白细胞吞噬的功能。通过菌毛,淋球菌可黏附于宿主黏膜上皮引起感染。

淋球菌的细胞膜由外、中、内三层组成。外膜含有多种蛋白,如蛋白Ⅰ、Ⅱ、Ⅲ和脂多糖。这些物质是淋球菌重要的毒力结构,也是淋球菌发病的关键。当细菌黏附于人体黏膜后,蛋白Ⅰ迅速转移至黏膜细胞膜,使黏膜细胞吞噬淋球菌后再将其排出,于黏膜下引起感染。蛋白Ⅱ则能使细菌与上皮、白细胞溶合。蛋白Ⅲ可能刺激封闭抗体,降低血清对淋球菌的杀菌作用。脂多糖为淋球菌的内毒素,是人体产生抗体杀灭淋球菌的主要抗原。

正常男性尿道由三种不同性质的细胞组成,他们对淋球菌的抵抗力不一致。舟状窝黏膜由复层鳞状上皮组成,对细菌抵抗力最强,感染发生在表面。前尿道黏膜由单层柱状细胞组成,抵抗力最弱。后尿道及膀胱黏膜由移行上皮组成,抵抗力较强,仅次于复层鳞状上皮。

淋球菌接种尿道后,借助菌毛、蛋白Ⅰ等,迅速与尿道上皮粘合,约1小时后即形成尿道黏膜表面感染。同时淋球菌被上皮细胞吞噬后再释放至黏膜下层。淋球菌也可通过细胞间隙抵达黏膜下层,在内毒素、IgM、补体等的协同下,引起炎症反应。也可引起尿道球腺感染,如导管堵塞;可引起小脓肿,脓肿破溃后引起尿道瘘;严重者侵及海绵体,引起尿道周围炎。合并淋巴管炎及血栓静脉炎时可出现阴茎水肿。尿道脓液涂片时通常淋球菌的分布不均,少数白细胞内有大量淋球菌,大多数白细胞无淋球菌。

黏膜固有层的感染可引起结缔组织增生,如感染轻微或治疗及时,可完全恢复;若感染严重或治疗不及时,增生的纤维结缔组织可引起尿道狭窄。

淋球菌可沿尿道黏膜向前后扩展,形成后尿道炎。主要侵及膀胱三角区、前列腺、精囊等开口,形成前列腺炎、精囊炎,进一步发展可导致附睾炎。

急性淋病潜伏期2~8天,通常为4天,潜伏期长短与细菌的毒性、数量及个体的抵抗力有关。早期表现为尿道外口红肿伴轻微疼痛。尿道分泌物开始为黏液性,1~2天后转为黄色脓液,红肿可发展到整个阴茎头。通常伴排尿次数增多,明显尿痛,并可出现阴茎痛性勃起。如患者有包皮过长或包茎,可出现包皮龟头炎。部分患者可伴低热、疲乏等症状,腹股沟淋巴结也可增大、红肿和疼痛。部分患者有淋球菌尿道感染,但无临床症状而成为重要的传染源。

起病1周后,症状逐渐减轻,阴茎头、尿道外口红肿减轻,分泌物转为稀薄黏液,排尿次数逐渐正常。4~8周后,症状完全消失,少数患者仍可伴有晨间尿道口黏液,并可查到淋球菌。约1\%患者可发生并发症,如前列腺炎、附睾炎等,全身严重感染较少见。

有肛交史的患者可患肛管直肠炎,表现为里急后重,有脓血便;肛镜检查发现肛管黏膜充血水肿,表面有脓性分泌物;涂片可见革兰氏阴性双球菌,培养淋球菌阳性。部分患者培养阳性,但无临床症状,该类细菌耐胆盐和脂肪酸,能在肛管繁殖,带有耐多种抗生素的基因,治疗较为困难。

有口-生殖器接触者可出现淋菌性咽炎。据国外统计,男性同性恋者10\%~25\%有咽部淋球菌感染。只有少数患者有临床表现,出现咽炎和颈部淋巴结肿大。感染一般于10~12周后自行消失。淋菌性咽炎可引起播散性淋病。

由淋球菌菌血症引起,典型表现为关节炎-皮肤综合征,通常无淋菌性尿道炎、咽炎、肛管直肠炎表现。急性期患者畏寒、发热,肘、腕以及手足部小关节多关节肿胀及腱鞘炎,关节背侧可见出血性或化脓性丘疹。血培养及关节渗出液的淋球菌培养阳性率约为50\%,而尿道、直肠、咽部培养阳性可达80\%。因此,怀疑为播散性淋病者,应作上述培养。心内膜炎、脑膜炎较少见,可与关节炎-皮肤综合征并存。

患者站立位,暴露全部外生殖器,观察内裤有无污染、脓痂,尿道口有无分泌物。如尿道口分泌物过少,嘱患者停止排尿2小时后再作检查,晨起第一次排尿前取标本检查可显著地提高细菌的检出率及诊断。

患者可合并其他性传播疾病,因此应仔细检查全部外生殖器以及会阴、阴囊、阴茎、腹股沟等的皮肤病变,观察腹股沟淋巴结有无红肿及压痛,睾丸、附睾、精索有无肿块。包皮过长者应上翻包皮露出阴茎头,包茎者如必要时可切开观察阴茎头有无溃疡或肿瘤,尿道口有无分泌物。如尿道外口无分泌物者可由后向前轻轻挤压尿道以获取分泌物。不能获得分泌物者可行尿沉渣检查。

尿道分泌物涂片染色查革兰氏阴性细胞内双球菌是诊断有症状的淋病最有效最可靠的方法,敏感性可达95\%~100\%。涂片阳性者可不作淋球菌培养。男性尿道外的淋球菌感染,如咽部、肛管直肠及播散性感染,则不能单凭涂片诊断,仍需作淋球菌培养。男性无症状者可通过培养确定诊断,并可确定细菌是否产生β-内酰胺酶及细菌的耐药程度。

尿道有脓性分泌物而不能确定为淋病时,应考虑结核、肿瘤、狭窄、结石的可能。

急性期患者应注意休息,多饮水,避免饮酒及辛辣食物。在未治愈前应禁止性生活,注意保持局部清洁,避免交叉感染。

自20世纪以来,淋病的治疗就几经起伏。随着耐药菌株的不断出现,对淋病的诊断与治疗提出了新的挑战。耐药性的产生是通过淋球菌的染色体突变引起的,可能降低了细菌细胞膜的青霉素结合蛋白与β-内酰胺抗生素之间的亲合力,也可能降低了细胞膜对青霉素的渗透性,导致青霉素不能进入细胞。染色体突变不仅引起对青霉素的药耐性,也可引起对四环素、壮观霉素、链霉素的耐药性。近年来世界各地的淋球菌耐药菌株(包括产青霉素的耐药菌株及染色体突变的菌株)均在成倍地增长。在我国,淋球菌的耐药菌株达21.6\%~81.4\%。当今防止、减少及治疗耐药菌株是防治淋病的关键。研究显示应常规采用对各类淋球菌菌株及各类淋球菌感染均有效的抗生素,并同时应用两种对大多数淋球菌均有效的抗生素以预防或延迟耐药菌株的出现。

头孢菌素(菌必治)曾被认为是第一线治疗淋病的药物,对各类耐药菌株及咽部、直肠的淋球菌感染均有效,对梅毒也有效,对孕妇无副作用。头孢三嗪250mg,肌肉注射,一次即能治愈淋病。但近年来,耐药菌株也开始出现。

壮观霉素(淋必治)对尿道、直肠淋球菌感染疗效较好,咽部感染疗效较差,对梅毒无效。常用剂量2g肌注1次,对头孢菌素、青霉素过敏者也可应用壮观霉素。

喹诺酮类药物对淋球菌有较强的杀菌作用,它能破坏细菌的细胞膜同时又是DNA旋转酶的阻滞剂,能使细菌的DNA的螺旋及超螺旋不能实现,DNA解链,导致细菌死亡。淋球菌对第一代喹诺酮类药物大多已产生耐药性,目前常用的是第二代、第三代药物,如氧氟沙星、环丙沙星、加替沙星等。喹诺酮类药物服用方便,毒性低,无交叉耐药,但对骨骺的发育有影响,不能用于孕妇及18岁以下患者。

氧氟沙星600mg口服每日1次,1~3日。环丙沙星500mg口服每日1次,1~3日。加替沙星首剂400mg口服,以后每日1次,7~10日。

随着耐药菌株的不断出现,淋球菌对大多数抗生素已产生耐药性,因此药物的治疗剂量加大,疗程延长。另外,约30\%的淋菌性尿道炎患者合并有衣原体感染,应给予四环素500mg,每日四次,疗程1周,以消除尿道内衣原体。淋病治愈4~7天后,无论症状是否消失,均应再次作培养,症状消失不能判断其治愈。

一般需住院治疗,原则上按耐药菌株治疗,可采用头孢三嗪(菌必治)1g/d静脉滴注,或壮观霉素2g静脉滴注每日2次,疗程7~10天。

急性附睾炎及前列腺炎较多见,症状与非特异性感染相似,诊断时应仔细询问病史,采取尿道标本涂片染色及进行细菌培养,以明确诊断。药物应选择对淋球菌及衣原体均有效的药物,通常联合用药,如菌必治联合四环素类药物。疗程应至少在10天以上。

尿道狭窄是淋病晚期严重的并发症,在使用抗生素后已不常见。对轻度狭窄者可行尿道扩张,重度者根据狭窄部位、长短及程度选择不同手术方案。局限性病变可行尿道内切开。顽固阴茎段狭窄可行尿道切开,形成尿道下裂,等炎症消退后再修补尿道下裂。


\section{第四节 非淋菌性尿道炎}

非淋菌性尿道炎是除淋菌外由其他病原体引起的尿道感染,发病率为淋病的4倍左右。病原体主要为衣原体和支原体。

沙眼衣原体为感染力很强的病原体,不易被人体杀灭。约有1/5为人类感染衣原体,感染后可长期不引起疾病。衣原体是女性盆腔炎、不育、宫外孕的常见原因。衣原体寄生于人体细胞内,不能在没有细胞的培养基上生存和繁殖。

衣原体有两种生活形态,原生小体和网状小体。原生小体呈圆形,直径约300nm,具有细胞壁和细胞外膜,表面有突起但无菌毛,在细胞外生活且具有传染性。原生小体细胞壁具有青霉素结合蛋白,青霉素在体外能抑制其生长,但对细胞内原生小体无作用。原生小体被人体柱状细胞吞噬进入细胞后进行分裂繁殖,体积增大,细胞壁消失,称为网状小体。网状小体在细胞内形成微波菌落,在数量达到100~1000个时,重新获得细胞壁,成为原生小体,然后突破宿主细胞,侵袭其他细胞。

近半数非淋菌性尿道炎的患者尿道内无衣原体,可找到解脲脲原体。虽然正常人尿道中也存在此类支原体,但目前认为支原体仍然为非淋菌性尿道炎的病原体之一。另有15\%~20\%患者既无衣原体又无支原体,为其他病原体,此类患者症状较易复发。

衣原体主要侵犯人体尿道、角膜、子宫内膜、宫颈及输卵管的柱状上皮。动物研究显示,衣原体接种于尿道5~14天后,出现分泌物,黏膜表面分叶核增多,黏膜上可见沙眼样结节,黏膜下淋巴细胞浸润。抗生素治疗后黏膜表面炎症消退,黏膜下病变继续存在,慢性感染可导致纤维化。

非淋菌性尿道炎潜伏期较长,2~5周,尿道分泌物量少呈白色或无色透明黏液,于晨间起床挤压尿道时明显,可伴轻微尿痛。衣原体感染也可引起附睾炎、肛管炎、Reiter综合征。

非淋菌性尿道炎诊断的关键是确定尿道炎存在并排除淋球菌感染及前列腺炎、膀胱炎等。尿道分泌物涂片检查可确定是否有尿道炎。如分泌物过少,可于晨起排尿前取标本,也可用尿道拭子在前尿道3cm处转动数秒采取标本,革兰氏染色后于1000倍下观察白细胞数>4,即可诊断。尿二杯实验也有助于诊断,第一杯前15ml尿沉渣检查,400倍下白细胞数>15,第二杯尿内白细胞明显少于第一杯,符合尿道炎;两杯白细胞数相近,符合尿路感染。

尿道炎诊断明确后,尿道分泌物涂片染色及培养均未能查到淋菌,即可诊断非淋菌性尿道炎。革兰氏染色诊断非淋菌性尿道炎准确度可达98\%。需要注意的是,淋菌感染可以合并衣原体感染。衣原体感染可以采用细胞培养以明确诊断,也可采用单克隆抗体染色及衣原体DNA探针进行诊断。

首选四环素类药物,沙眼衣原体及支原体对其敏感。治疗1周,治愈率可达60\%~95\%。常用四环素、强力霉素、二甲胺四环素,应避免用于孕妇及小儿。

盐酸四环素500mg,口服,每日4次,连服7天。

强力霉素100mg,口服,每日2次,连服7天。

二甲胺四环素100mg,口服,每日2次,连服7天。

红霉素与四环素效果接近,可用于孕妇及小儿。目前常用的有阿奇霉素、罗红霉素等,其疗效优于四环素。

红霉素250mg,口服,每日4次,连服7天。

罗红霉素150mg,口服,每日2次,连服10天。

阿奇霉素首剂0.5g,以后0.25g每日1次,连服5天;或单剂1g口服1次。

不经治疗,约70\%患者症状6个月内可自行消失,但并不意味治愈,需经培养确定。

约20\%的衣原体感染和30\%的支原体感染在药物治疗时症状消失,停药后症状复发。如果尿道分泌物未查到衣原体和支原体,患者复发比例更高。治疗期间应禁止性生活,有助于复发或再感染的鉴别诊断,也可预防疾病的传播。再感染者可按原方案治疗,复发者疗程应达3~6周甚至更长。

疗效不佳的患者应注意有无阴道滴虫、尿素分解类杆菌、生殖支原体感染等。反复发作长期不愈的患者应进一步了解下尿路解剖与功能情况,必要时可行尿道镜检查。

【附】 Reiter综合征

Reiter综合征主要症状是非淋菌性尿道炎、结膜炎、虹膜炎、关节炎及皮肤黏膜病变,于1961年由Hans Reiter首先描述而得名。1\%~2\%患者可合并此综合征,其发生可能与衣原体感染有关。

临床表现:非淋菌尿道炎症状出现较早但较轻,1~5天后出现其他症状。约80\%的患者在病后4周内出现关节炎症状,累及膝关节、骶髂关节、踝关节以及足部小关节。关节炎持续时间最长,其他症状消失后关节炎症状仍可持续数月。

眼部出现结膜炎、虹膜炎,并可引起化脓性角膜炎。症状可多次发作,与沙眼衣原体直接感染不同,结合膜无沙眼滤泡增生。

皮肤早期病变多见于手掌及足底,皮肤角化增厚,表现为脓溢性皮肤角化症,其次为阴囊、头皮及躯干。约40\%患者有环状阴茎头炎,舌及咽部有无痛性糜烂。

Reiter综合征也可只表现为尿道炎及关节炎,少数可合并心肌炎、心包炎。首次发病症状持续2~6月,完全恢复后仍有40\%~70\%的患者复发,病情重者可致残。

治疗:目前尚无有效治疗方法。四环素可治疗尿道炎,对关节炎及综合征全部病程无明显疗效。非类固醇抗炎药物、消炎痛有一定疗效,优于糖皮质激素及水杨酸制剂。


\section{第五节 尖锐湿疣}

尖锐湿疣又名生殖疣、尖圭湿疣,好发于生殖器、肛门的常见的性传播疾病。近年来,国内外的发病率都明显增加。

尖锐湿疣病原体为人类乳头状瘤病毒(Human papilloma virus,HPV),含双链、螺旋状、环状DNA,由72个病毒衣壳组成的20面体的衣壳包绕,无包膜。HPV基因组有8000个碱基对,衣壳直径约55nm,分子量5×106 ,其中88\%为病毒蛋白。该病毒可被乙醚灭活,耐干燥和低温,不能进行组织培养。

HPV可分60型,其中13型与尖锐湿疣有关,包括HPV6,11,18,30,31 等;HPV16,18,30,31,42,51~54 等与恶性肿瘤有关。

在本病的病程中,细胞与体液免疫均有作用,但主要是细胞免疫。

大多数患者检测不到抗体,少数患者可检测到低滴度的IgM型抗体。患者的细胞免疫反应降低,血中T淋巴细胞及CD4 数目明显减少。病损中的朗罕格细胞及T淋巴细胞数减少。体外细胞免疫阳性低于正常人,对HPV抗原特异的细胞免疫实验高于正常人。细胞免疫缺陷的患者(艾滋病患者)、接受免疫抑制治疗的患者(肾移植患者)易感染HPV且较重。妊娠期细胞免疫降低,可发生尖锐湿疣或加重。

通过性接触传播,HPV通过皮肤或黏膜的微小损伤进入接触者的皮肤黏膜,刺激表皮基底细胞,发生分裂,产生表皮增殖性损害。通常只在表皮颗粒层及其上部的细胞中可检测到HPV DNA及病毒颗粒。

本病多发生于性活动活跃的年轻人,以18~35岁居多。传染源主要为症状明显的患者,亚临床型也有传染性。传播途径:①性接触,最主要的方式;②母婴传播,通过分娩可被母亲传染;③家庭内密切的非性接触,如毛巾、内衣等。发生感染的相关因素:①病毒的数量;②与病损接触的程度;③宿主对病毒的抵抗力;④皮肤黏膜的损伤。

早期为局限性表皮增生,真皮乳头受压变扁平。成熟病损表现为表皮角化过度,角化不全,棘层肥厚,皮突增厚延长,呈乳头状瘤样增生,类似鳞癌。颗粒层和棘层细胞有空泡形成,空泡细胞体积较大,核浓缩,核周有透亮晕。真皮水肿,毛细血管充血扩张并有炎症细胞浸润。

潜伏期长短不一,2周~8月,平均3个月。

临床表现多样。刚起病时为小而柔软的淡红色疣状丘疹,以后逐渐增大,表面凹凸不平,继续发展可成乳头状、菜花状、鸡冠状增生物,通常无自觉症状。疣体表面粗糙,呈白色、红色或污秽状,部分疣体呈团块状,可破溃、渗出或继发感染。

亚临床损害表面无肉眼可见病损,但涂5\%醋酸后,用放大镜可见呈扁平状的淡红色丘疹。

男性可发生在阴茎的任何部位,首发于系带、冠状沟和包皮内板,这些部位在性交中易受伤。还可侵及男性尿道,发生尿道内尖锐湿疣,通常呈鲜红色,多发。如继发出血或感染,可出现尿道分泌物及排尿困难。偶有膀胱发病的报告。有肛交史的可发生肛周及直肠损害,男同性恋多见。

女性好发于大小阴唇、阴蒂、阴道、宫颈、肛周等,通常无自觉症状,可有白带增多、瘙痒、性交困难或性交后出血。

巨大尖锐湿疣:发生阴茎,疣体增大,破坏部分阴茎组织。

癌变:偶可发生癌变。生殖器和肛周的尖锐湿疣经过较长时间的间变和发展成原位癌或鳞癌,宫颈癌多见。用DNA杂交技术可于宫颈癌损害检测到HPV DNA,宫颈癌涂片也可见HPV感染特有的空泡细胞。

①溃疡;②感染;③出血;④梗阻,如产道和尿道。

根据典型临床表现即可诊断。部分病损不典型者,可借助于细胞学、免疫组化、核酸杂交及PCR等手段。

取疣组织作涂片作帕氏染色,可见空泡细胞与角化不良细胞;用抗HPV抗体作免疫组化检测,特异性较高。核酸杂交技术检测HPV DNA,可鉴定HPV DNA序列。PCR检测HPV DNA特异性及敏感性均很高。

①同时检查性伴侣;②检测是否有其他性传播疾病;③5\%醋酸检查亚临床表现;④排除恶性病变;⑤不要采用毒性大的或可能遗留癍痕的药物。

1.足叶草毒素:0.5g外用,每日2次,3天为一疗程。重复用药间隔4天。

2.5\%氟脲嘧啶(5-Fu)霜:用于治疗尿道内尖锐湿疣,每日4次,2~3周为一疗程。

3.冷冻疗法:液氮或二氧化碳干冰,用于疣体不大或较局限者。

4.激光:二氧化碳激光可用于任何部位的尖锐湿疣,钕-YAG激光可治疗尿道近端的疣。

5.电灼:用于较小或较局限的疣,禁用于带有心脏起搏器的患者和肛门边缘的疣。

6.干扰素及胸腺肽:局部应用有效但复发率高,全身应用有助于预防复发。干扰素皮下或肌肉注射,从1×106 到5×106 再到18×106 每日1次,10~14次后改为每周3次,连续4周。

本病较易复发。原因:①距病损2cm处的正常皮肤可检测到HPV DNA;②常有亚临床感染;③隐性感染,临床及组织学检查均正常的皮肤,可检测到HPV DNA。少数反复发作者,可发生恶变。


\section{第六节 生殖器疱疹}

生殖器疱疹(Genital Herples,GH)是由单纯疱疹病毒(Herpes Simplex Virus,HSV)引起的性传播疾病,发病率较高,较易复发,目前尚无有效治疗方法。本病可合并宫颈癌,并可引起新生儿、胎儿感染。

病原体为单纯疱疹病毒(HSV),直径150~200nm的DNA病毒,由75nm的DNA核心和直径150nm的蛋白质衣壳组成。HSV可分为两型:HSV-1和HSV-2。HSV-1常引起口、咽、眼及皮肤感染;HSV-2引起生殖器疱疹。

HSV感染人体后可长期存在,因其有逃避宿主防御机制:①HSV可在细胞间传播而不需进入细胞外环境;②逃避宿主的自然防御系统,即巨噬细胞、自然杀伤细胞及干扰素;③可使神经节发生潜伏性感染。宿主可通过抗体、特异性细胞毒性T细胞及特异性迟发超敏反应来消除病毒。

特异性抗体可阻止持续性感染的建立,也可阻止病毒侵犯神经系统。首次感染HSV后,4~6周抗HSV抗体升至高峰,并保持稳定,而IgM抗体可维持6~8周。复发或再感染者不出现IgM抗体。

HSV-1和HSV-2具有相同的抗原决定簇,因此二者可出现交叉保护。原发性HSV感染较非原发性感染的临床表现重。

在性接触过程中,HSV通过生殖器皮肤的微小损伤的裂隙进入皮肤黏膜的角朊细胞。病毒在细胞内复制,并播散到周围细胞,破坏受感染的表皮细胞,引起表皮损伤。

病毒进入宿主后,部分被免疫反应清除,部分因逃避宿主的防御反应而长期潜伏在神经节中(HSV-1在三叉神经节,HSV-2在骶神经节)。当机体宿主在受外伤、细菌感染、月经来潮、免疫抑制等抵抗力下降情况下,病毒复苏和激活,由神经节返回受累的皮肤黏膜而出现感染的复发。

国内外此病发病率呈急速增加趋势,主要发生在性活跃的年轻人。传染源为GH患者及无症状携带者。传播途径:①性接触传播,最主要的途径;②母婴传播,胎儿可在宫内受感染,也可因羊膜早破逆行感染,或在分娩中受感染;③间接接触,日常生活中的密切接触而感染,较少见。

基本病理表现为局部坏死,病变细胞出现细胞内水肿,基底上表皮内水疱形成,气球变性,核染色质边移,细胞核内有噬酸性包涵体。部分发生溃疡时见角朊细胞坏死及溶解。水疱形成时发生单核细胞浸润,水疱破溃时多核细胞浸润。

生殖器疱疹的临床表现与此病为原发性感染还是复发性感染有关,还与是否为首次感染有关。原发性感染指未受过HSV(1型或2型)感染,体内无抗体。首次感染是指第一次HSV感染,可以是原发性感染,也可以是非原发性感染(过去曾有另一型HSV感染,体内有抗体)。原发性首次感染症状重,病程长,排毒时间长;非原发性首次感染者,症状轻,病程及排毒时间短;复发感染者症状最轻,病程及排毒时间最短。

原发性GH特征为全身及局部症状发生率高,持续时间长。通常在病程早期即出现全身症状,发热、头痛、乏力、肌肉酸痛等,于3~4天达高潮,持续3~4天后消失。症状的严重程度有差异,部分可无症状。

在起病后的6~7天出现局部症状,如疼痛、瘙痒、排尿困难等;7~11天症状最重,2周后症状逐渐消失。部分患者于发病后的第2~3周出现腹股沟淋巴结肿大、触痛。

潜伏期通常为3~5日。损害常发生在阴茎头、冠状沟、尿道口、阴茎体及阴囊等部位。原发损害为散发或簇状小红丘疹,有痒感,迅速变成小水疱。3~5日后水疱破溃形成溃疡或糜烂,溃疡大小不一,并伴有较剧烈地疼痛。溃疡通常持续4~15日后结痂愈合。病损愈合通常无癍痕。约75\%患者可于病程中(4~10日)出现新损害。中位排毒时间(从发生损害到最后一次病毒培养阳性)为12日。从发生水疱到结痂约10.5日,从发生损害到上皮完全形成约16.5日。

1/3的患者可出现尿道分泌物及排尿困难。尿道分泌物呈清亮黏液状,排尿困难与分泌物不呈正比。尿道拭子和晨尿可分离到HSV。

通常较男性重,主要表现为外阴阴道炎、宫颈炎,阴道分泌物增多,偶可有排尿困难及尿潴留。损害可波及肛周及股部皮肤,可伴局部淋巴结肿大触痛。通常于1~2周愈合,病程长的可达6周,并出现新的病损。

约50\%患者复发前有前驱症状,局部瘙痒、烧灼和刺痛,多发生在生殖器。此后发生水疱,数量多少不等,数个至20个,溃疡形成时疼痛加剧。全身症状较少见。通常于7~10日愈合,中位排毒时间4日,从发生水疱至结痂4~5日,发生水疱至上皮形成10日。

表现为肛周、肛门与直肠发生水疱及溃疡,伴肛门直肠痛、里急后重、便秘、肛门瘙痒及骶部感觉异常等。发生直肠炎者可出现发热及腹股沟淋巴结肿大。

受HSV-2感染的孕妇,可于妊娠后期3个月出现宫内感染,虽罕见但后果严重,一旦发生死胎率很高。分娩时可通过产道或羊膜早破出现逆行感染,新生儿可出现病毒血症、局限性或播散性HSV-2感染,甚至脑炎。

原发性生殖器疱疹可以并发无菌脑膜炎、骶部自主神经功能障碍、急性尿潴留、生殖器外损害(皮肤、眼结膜等)、播散性感染等。非原发性生殖器疱疹并发症较少见,可有外生殖器部位色素减退及癍痕。

临床表现典型者,诊断不难。少数需要作实验室诊断。

水疱底部取材作帕氏(Papanicolaou)染色,可见特征性的核内包涵体及多核巨细胞。

水疱病损作组织培养48~72小时可见特征性的细胞致病作用,敏感性和特异性均较高,可确定诊断。

可用免疫荧光、酶联免疫吸附实验(ELISA)、放射免疫测定(RIA)等方法检测病毒抗原。

可用免疫荧光抗体实验、限制性内切酶分析法、微量中和实验和血清学实验等进行HSV分型。

应注意是否合并其他性传播疾病,如梅毒、淋病和衣原体感染等。应与以下疾病鉴别。

一期梅毒有生殖器糜烂,暗视野显微镜下可找到梅毒螺旋体,血清学实验阳性。

软下疳有生殖器溃疡,培养可查到杜克雷嗜血杆菌。

生殖器部位接触性皮炎有接触性过敏史,炎症范围超出水疱及糜烂范围,查不到HSV。

生殖器部位药疹有药物过敏史,通常发生在固定部位,水疱不成簇,病损消退后有色素沉着,查不到HSV。

1.预防继发感染,保持疱壁完整、清洁与干燥,可用生理盐水清洗3次并吸干。

2.继发感染者,选用敏感抗生素。

3.疼痛明显者,可口服止痛剂或外用利多卡因软膏。

4.妇女性复发者,应作妇科检查,排除宫颈癌。

首选阿昔洛韦(Acyclovir,ACV)。ACV可选择性地被HSV感染的细胞摄取,HSV特异的胸腺嘧啶激酶将其转化为单磷酸盐,然后宿主细胞的其他激酶又将其转为双磷酸和三磷酸盐,后者抑制HSV DNA合成。ACV对正常宿主细胞无毒性。

ACV可迅速改善GH症状,缩短排毒时间,促进病损愈合。用法与剂量:①病情较重或有并发症者:ACV静脉滴注5mg/Kg,每日三次,5~7天为一疗程。②病情较轻者,ACV 200mg口服,每日5次,7~10天为一疗程。③皮肤黏膜损害,5\%ACV霜剂,每日5次,连续5天。④肾功能不全者、孕妇慎用。

近来还可选择泛昔洛韦、更昔洛韦。

左旋咪唑、干扰素、胸腺肽等有一定效果。

生殖器疱疹患者应禁止性交,避孕套也不能完全防止病毒的传播。口服ACV可降低生殖器疱疹的复发率。对于妊娠后期仍有活动性的生殖器疱疹的孕妇,建议中止妊娠,避免新生儿感染。


\section{第七节 艾 滋 病}

艾滋病是获得性免疫缺陷综合征(Acquired immunodeficiency syndrome,AIDS)英文名称缩写的音译名。1981年首先在美国报道,1982年正式命名。近年来,世界各地发病率有上升趋势,主要经性接触、血液和母婴传播。

1986年国际病毒分类委员会将艾滋病的病原体正式命名为人类免疫缺陷病毒(Human Immunodeficiency Virus,HIV)。HIV系逆转录RNA病毒,细胞膜芽生。未成熟病毒颗粒有一新月状的核,成熟的病毒颗粒有一致密偏心的圆形或棒状的核,颗粒直径100~140nm。病毒蛋白包括:核蛋白、膜蛋白和与复制有关的酶蛋白。HIV具有典型的逆转录病毒的基因结构,两边是重复调节序列,中间有3个主要基因:①gag基因,编码三种核蛋白;②pol基因,编码多聚酶和蛋白酶;③env基因,编码膜蛋白。

HIV由皮肤或黏膜破口进入血液,主要攻击和破坏辅助性T细胞(T4 )。HIV对T4 有较强亲和力,进入T4 使之破裂、溶解、消失,从而使机体T4 数目减少,呈现免疫抑制状态。因此,易于发生条件致病性感染和Kaposi肉瘤。同时,病毒具有嗜神经性,能侵犯神经系统,感染脑和脊髓,引起神经系统症状。

1978年在美国纽约发现第1例病例以后,1979年发现7例,1980年12例,1981年204例,1982年750例,到1983年已累计发生1739例,呈逐年直线上升,世界卫生组织宣布到1992年7月底统计已达164个国家。1995年6月30日,世界卫生组织公布,全世界登记在册的艾滋病病例已接近117万,但实际人数要比此数高得多,估计全世界AIDS病例总数可能已超过500万,全世界目前HIV感染者的总数已超过2000万人,每天增加约600人。死亡患者数近100万人,故称之为世纪绝症。目前以美洲为最多,其次是亚洲,欧洲名列第三。近十年来,亚洲HIV感染人数正飞速上升,亚洲处于艾滋病扩散期。在泰国、印度,已从高危人群扩散到一般人群,成人已有20\%受感染。

我国1986年发现第一例AIDS,发患者数逐年上升。目前我国现有艾滋病病毒感染者约84万人,其中,艾滋病患者约8万例。根据世界卫生组织统计,目前我国艾滋病病毒感染者占总人口的比例虽然很低,但感染人数在亚洲位居第二位,在全球居第十四位。

艾滋患者及病毒携带者。

①性接触传播,主要的传播方式。包括同性恋、异性恋及双性恋的密切的性接触,非密切的接触如接吻、拥抱、握手等不引起传播。单次性接触传播机会较低1/1000~1/100;如伴有生殖溃疡的性传播疾病如软下疳、梅毒等传播机会将增加10~20倍;性伴侣越多,传染可能性越大;同性恋或异性恋被动方感染危险性大于主动方。②血液传播。输入HIV感染的血液和静脉药瘾者共用针具均可引起HIV传播和感染。③母婴传播:HIV感染的母亲在妊娠、分娩甚至产后哺乳均可传给婴儿,感染率可达25\%~50\%。

男性同性恋者、静脉药瘾者、接受输血及血液制品者、血友病患者、多性伴的异性恋、具有生殖器溃疡的性病患者。

潜伏期尚不完全清楚,可能为6个月~5年。

目前沿用1986年美国疾病控制中心的分类方法,分为四组。

急性HIV感染,表现为一过性传染性单核细胞增多症,HIV抗体阳性。

无症状HIV感染,无临床症状,血清HIV抗体阳性。

淋巴结病,表现为原因不明的持续全身淋巴结肿大,数目在3个以上,直径>3cm,时间超过3个月。

有其他的疾病或临床症状,分为5个亚型。

(1)A亚型:表现为非特异的全身症状,如持续1个月以上的发热、腹泻、体重减轻超过10\%而原因不明者。

(2)B亚型:表现为神经系统症状,如脊髓病、痴呆、末梢神经病而原因不明者。

(3)C亚型:二重感染。由于HIV感染后导致细胞免疫功能不全导致的二重感染。分为两类:①C1:常见感染有卡氏肺囊虫性肺炎、慢性、弓形体病、念珠菌病、类圆形虫病、组织胞浆菌病、隐球菌病、巨细胞病毒感染、慢性播散性疱疹病毒感染、鸟型结核分支杆菌感染及进行性多灶性脑白质病。②C2:其他感染:口腔毛状黏膜白斑、复发性沙门氏菌血症、播性带状疱疹、奴卡氏菌症、结核及口腔念珠菌病。

(4)D型:继发性肿瘤。主要为Kaposi肉瘤、非何杰金氏淋巴瘤及脑的原发性淋巴瘤。

(5)E型:其他并发症,由HIV感染引起的不属于其他亚型的并发症如慢性淋巴性间质性肺炎。

条件致病性感染是艾滋病最突出的特点,表现为范围广,病情重,发病率高,是引起死亡的主要原因。如卡氏肺囊虫性肺炎占艾滋病肺部感染的80\%。

最常见的肿瘤为Kaposi肉瘤和非何杰金氏淋巴瘤。超过30\%的艾滋病患者有Kaposi肉瘤,与非洲型Kaposi肉瘤相似,恶性程度高于经典型Kaposi肉瘤,多脏器受累并很快死亡。非何杰金氏淋巴瘤常累及中枢神经系统。

(一)流行病学及临床表现。

(二)实验室诊断。确诊艾滋病必须有实验室诊断。

T4 减少(正常为0.8×109 /L),T4 :T8 <1(正常为1.75~2.1)。

(1)组织学检查有无原虫及蠕虫感染、弓形体病、隐孢子虫病、类圆形虫病。

(2)真菌感染:白色念珠菌病可根据组织学、食管印片镜检或内窥镜检查确定。隐球菌病可根据培养、抗原测定、组织学及脑脊液的印度墨汁染色确定。

(3)病毒感染:根据组织学、培养或细胞学检查确定有无巨细胞病毒或α-疱疹病毒感染。

(4)细菌感染:根据培养确定有无鸟形分枝杆菌感染。

细胞培养分离病毒,病毒抗原检测,病毒核酸检测,逆转录酶检测四种方法。

分为筛查和确证实验。筛查实验包括:酶联免疫吸附实验(ELISA)、间接免疫荧光(IIF)、明胶凝集实验(PA)。确证实验包括:放射免疫沉淀实验(RIP)、蛋白印迹法(Weratern Blot),此类方法测定病毒结构蛋白,特异性强。

艾滋病病毒抗体检测是确定是否感染的最简便的方法,通常先用ELISA方法检测两次均为阳性后,才能确定是阳性,然后作确证实验方可确诊。

母婴传播和血液传播。母亲是HIV患者、HIV携带者或者是高危人群。患儿输入感染的血液或血液制品。

主要为学龄前儿童,大多数在1岁前确诊。

短于成年人。

新生儿期多无症状,逐渐出现发热、肝脾及淋巴结肿大、顽固性鹅口疮、反复腹泻、呼吸道感染、生长发育减慢或停止等。WHO关于儿童艾滋病的定义:在婴儿或儿童中,出现以下两种主要症状和两种次要症状,又无已知的免疫抑制原因(癌症、严重营养不良及其他已知病因),怀疑为艾滋病。

主要症状:①慢性腹泻1月;②体重下降或生长缓慢;③持续发热超过1月。

次要症状:①口-咽念珠菌病;②全身淋巴结病;③反复发作常见感染;④持续性咳嗽;⑤泛发性皮炎;⑥母亲HIV感染。

儿童艾滋病多无Kaposi肉瘤,有多克隆高丙球蛋白血症,而淋巴细胞计数大多正常。这些特点异于成年艾滋病。对于小儿艾滋病的诊断应慎重,应与其他引起免疫缺陷鉴别,如先天性感染、饥饿、遗传性免疫缺陷及使用免疫抑制剂等。

此类疾病是由皮质类固醇、放疗、化疗或已经存在的恶性肿瘤或严重的蛋白质-热能性营养不良引起。

由于艾滋病患者有发热、脾肿大、淋巴结肿大、白细胞减少、淋巴细胞减少等症状,应与血液病鉴别。可通过骨髓检查、HIV抗体鉴别。

艾滋病急性HIV感染期症状与传染性单核细胞增多症十分相似,与之鉴别主要手段是进行HIV抗体检测。

通过HIV抗体检测可与中枢神经系统病变鉴别。

白介素-2、γ-干扰素等治疗艾滋病有一定疗效,但疗效不明显。

根据不同的病原体选用相应的药物,见表。

艾滋病常见条件性感染的药物治疗

续表

目前抑制HIV的药物包括苏拉明、三氮唑核苷、α-干扰素、甲磷酸盐、叠氮胸苷(AZT)及利福霉素衍生物等。上述药物在体外试验有较明显的抑制病毒的效果,但均未观察到临床及免疫学的改善。AZT口服吸收好,能通过血脑屏障,临床应用显示AZT能延长艾滋病患者的生存时间。其作用机制是抑制逆转录酶,阻断HIV复制,但不能根除病毒。临床用药时病毒复制停止,停药后又恢复。因此,正在研究更有效的抗病毒药物,以求获得最大的功效和最低的毒性。

包括抗肿瘤治疗、支持治疗及对症治疗等。

1.卫生宣传教育是最重要的预防措施,目的使公众认识到艾滋病的临床表现、危害性及防护措施。

2.避免与艾滋病患者及携带者发生性接触,尤其是同性恋者的肛门性交。

3.提倡使用阴茎套。

4.对献血者进行HIV抗体检测,阳性者应禁止供血、血液制品及精液。

5.不共用注射器及针头,应尽量使用一次性注射器。

6.不共用牙刷、剃须刀或其他可能被血液污染的器具。

7.艾滋病患者及携带者应避免妊娠,防止母婴传播。

8.医务人员及实验工作者应注意被血液或其他体液所感染。

9.加强国境检疫,防止艾滋病的传入。


\chapter{第二十三章 性功能障碍的中医治疗}

中医没有勃起功能障碍的病名,依据勃起功能障碍的定义即男子在性刺激下持续或反复不能达到或维持足够的阴茎勃起以完成满意的性生活,祖国医学将其称为阴痿(萎)、阳痿(萎)。在认识本病及为本病赋名方面经历了几个阶段,现简述如下:

长沙马王堆医书中有关于本病的最初记载,《天下至道谈》云:“不能”,《养生方》称之为“不起”、“老不起”;《黄帝内经》中记载本病时命名为“阴痿”,如“年六十,阴痿,气大衰,九窍不利”、“阴痿气大衰而不起不用”、“阴器不用”、“阴不用”、“隐曲不利”、“不得隐曲”、“不起”等;《神农本草经》中称其为阴痿。另外这一时期的房中养生书籍如《素女经》、《玄女经》等将其称为“阴痿不起”、“茎不起”、“不怒”等。

这一时期的医家将本病多称为“阴痿”、“阴萎”。《脉经》中称为“阴萎不起”;《针灸甲乙经》中为“阴痿”、“阴痿不用”;《肘后备急方》中名为“阴萎”;《诸病源候论》中则有“阴痿”、“阴萎”、“阴不起”等称谓;孙思邈在《备急千金要方》和《千金翼方》中对本病的病名记载就有“阴痿”、“阴痿不用”、“阴痿不起”、“阳不起”等17种病名;而收集我国隋唐以前医籍内容的《医心方》中有“阴痿”、“阴痿不起”、“房内衰”等13种病名的记载。

这一时期,多数医家仍将本病称为“阴痿”、阴萎”、“庶事不兴”、“阳道衰弱”等。“阳痿”,病名首见于《扁鹊心书·神方》,“五福丹……又能壮阳治阳痿,于肾虚之人功效更多”。《三因极—病证方略》称其为“阳事不兴”;《圣济总录》称其为“阳气痿弱”、“阳道痿弱”;《太平圣惠方》中称为“阳道痿弱”、“阳气痿弱”、“阴气痿弱”。由此可以看出,元代以前虽然主要以“阴痿”为病名,但命名杂乱,而且在一些医籍中同时出现不同的病名。

明代《慎斋遗书》中首次将本病命名为“阳痿”,明代名医张景岳在《景岳全书·杂证谟·阳痿》中也使用“阳痿”病名,并将其定义为“阳不举也,阳道不振”。之后的大部分医家均称本病为“阳痿”,亦有部分著作如《济阳纲目》、《张氏医通》、《证治准绳》、《医碥》、《赤水玄珠》、《医宗必读》等仍以“阴痿”名之。不过,在明清医籍中,一般用“阳痿”者不用“阴痿”,用“阴痿”者不用“阳痿”,改变了宋元之前一书多名混用的状态。《杂证治要秘录》则明确指出“阴痿即阳痿”。

另外,由于本病是一种与房事有关的疾病,因此历代医家除了将本病称之为“阳痿(萎)或阴痿(萎)”外,也常常用对房事或阴茎的称谓词语描述该病,而侧重点则是对房事时阴茎勃起功能状态的描述。这也是古代医家对阳痿病病名认识的另一个重要特点。如“临事不举”、“阴茎不举”、“举而不坚”、“坚而不久”、“房事不举”、“男子绝阳”等。

民国时期的医著中沿用明清命名之法,多以“阳痿”名之,少数医籍如《中西验方新编》、《通俗内科学》等仍以“阴痿”为名。新中国成立之后,随着中医规范化工作的逐渐深入,“阴痿”这一病名逐渐被“阳痿”完全取代。

从《马王堆医书》中勃起“不大”、“不坚”、“不热”的病机为肌、筋、气三者不至,“三至乃入”为阳痿病机的最初阐述开始,历代医家对该病病机的认识逐渐深入。自先秦、秦汉时期对本病病机从肾气虚与肾阳虚认识以来,每经历一个时期,都有进步。晋隋唐时期也从肾虚认识本病的病机,而且比前代有所发挥,增加了从肾精亏虚、肾阴虚、肾阴阳两虚和气血不足,心神失养4种病机。而自金以后,医家不止是从肾虚认识本病的病机,而是从多个脏腑的虚实出发,因而出现了肾经火郁、脾阳虚、脾胃衰损、肝经湿热、肝气郁结、肝阳上亢、心气虚、心血虚、心火闭塞、心包火衰、阴虚阳亢、心脾精血不足、小肠虚寒、三焦雍滞、下焦阳虚、胆虚、少阳郁、心脾虚、心脾郁结、心肾不交、肝肾虚、肝肾实、脾肾阳虚、五脏阳气虚、脑虚、痰湿阻滞、淤血阻滞等其他近三十种病机。在发病学立论思路上有从肾虚立论、从肝立论、从多因立论、从阳明立论、从湿立论和从郁火立论等,但以肾虚论和多因论为主流。从医学发展史看,古代医学家仅从肾虚或一因立论的人数随着历史的不断延伸而逐渐减少,而从多脏多因立论者反而逐渐增多。总的认识是,病因多为七情所伤和房事不当,病机多为肾脏功能失调。

随着现代社会的发展进步,本病发病的因素及病机病理也发生了变化。现代社会,由于生活水平明显提高,医学技术逐渐进步,身体素质不断增强,以及婚姻制度的改革,房劳损伤所致本病者已显著减少。相反,由于生活节奏快,社会竞争强烈,工作压力大,致使精神紧张,情志内伤,肝气郁结引起的ED日见增多,夫妻关系的不和睦,偶尔几次性生活的失败,都会给男方以心理压力、情志抑郁而发病。环境污染,饮食结构改变,以及嗜食肥甘厚味,大量吸烟酗酒等,往往内聚痰浊或变生湿热瘀毒。大量的临床文献资料表明,当前ED的病机已由虚证向实证方面转化,痰、热、瘀、浊、湿、郁等则是构成实证的病理基础。总的来说ED的病因病机可以归纳为肝郁、肾虚、痰湿、血瘀。

从发病学理论上分析,肝郁既可是ED的发病原因,又可是ED既病后病机病理变化的结果。从肝郁与ED的类型关系来看,作为病因时,因肝郁而致的ED多属功能性ED;作为病理变化结果时,则普遍存在于功能性ED和器质性ED患者。即不论ED的类型如何患者都伴有不同程度的心理障碍。

肝郁之所以能够引起ED的发生,并可使其病情变得更加复杂,主要是肝郁能影响肝功能(主疏泄、主宗筋、主藏血)的正常发挥。肝与外肾(睾丸)联系密切,肝的生理功能的正常与否影响着外肾的功能。首先,肝脉上循股阴,入毛中,结于阴器,阴器的功能活动受到肝气的调节,肝主筋,阴茎以筋为木,肝气充于筋,肝气充盛是阴茎勃起的动力,肝之功能正常,则阴茎伸缩自如、勃起刚劲;其次,肝寄相火,具有鼓动阴器、启闭精窍而主司精液走泻的作用;再者,肝主藏血,具有调节血量之功能,肝血充盈是阴茎勃起的物质基础,肝血充足则宗筋振奋。因肝主疏泄,主司精神情志活动的调节,情志刺激最易伤肝。肝郁气滞,疏泄不及,房事时不能将血液迅速灌注阴茎,影响到宗筋功能的正常发挥,阴茎不能勃起或举而不坚或坚而不久,从而导致ED。

男子性功能除受肝之疏泄、调节血量参与外,还受肾的调节与控制。男子以肾为先天男子生长发育、性功能和生殖能力兴衰的过程,就是肾之功能强弱的反映。肾主阴器,为阴茎勃发坚举提供原动力,肾气终身处于亏损态势,加之“久病穷及必肾”,因而病在肾又以虚为主。男子以肾为先天,精气为其本,男子生长发育、性功能和生殖能力兴衰的过程,就是肾之精气盛衰的反映。然而,男性一生,肾气惟有亏耗而不会过胜,少儿时期肾气末盛而多肾气不足,或先天禀赋素弱或后天失养而致;青壮时期,肾之精气虽已充盛,但每因自持而恣情纵欲,不节房事,或习手淫,或形志过劳,而致精气亏乏;“五八”之后尤其是进入更年期及老年期后,肾之精气开始自然衰退,若加之调摄不法,则可加速肾气亏损。肾气一亏,启动功能不足,阴茎难以勃发。对于ED的中医基本病理变化来说,在当代社会环境条件下肾虚是ED发病的主要病理趋势。

心主血脉,藏神。《素问·痿论》说“心主身之血脉”。心气推动血液在脉中运行,流注全身,发挥营养和滋润作用。若心气血亏虚,血液不能充养阳道,气不至则无以令阴器振兴,血不充则难求其势壮,是故ED生矣。又心主神明,主宰人体的精神、意识思维活动,喻嘉言谓“心者情欲之府”。情志活动是“心神”的体现。《素问·痿论》:“思想无穷,所愿不得,意淫于外,入房太甚,宗筋弛纵,发为筋痿,及为白淫。”性生活是一个复杂的生理机能活动,性行为除了需要正常的性生理功能作为基础,同时也受到思想意识的支配。由于经济的飞速发展,新的生活方式使人们生活节奏明显加快,人们的社会心理、道德观念、处世哲学均发生着变化。工作责任重,压力太大,或者长期面临生活困境,经济出现危机等,精神长期处于高度紧张状态,长时间脑力劳动,思虑太过,暗耗心血,宗筋失养;或初次性交失败后的自卑心理,或幼年遭受精神创伤,性格孤僻,缺乏自信心,对性问题持消极态度而发为ED。

痰湿是脏腑功能失调的病理产物,即成之后又成为一种常见的致病因素。痰湿之源,是各种致病因素导致的五脏及气血功能失调,痰湿即成则贮留体内,一旦痰湿过盛,湿浊下注,困阻下焦,经络受阻,使阳气不能伸达阴器,血液不能充养阳道,气不至则无以令阴器振兴,血不充则难求其势壮,是故ED生矣。又痰湿与淤血同源,二者致病常相互凝结,使经脉阻滞,阳气不能通达阴器,气血不能荣养宗筋,也易引起ED的发生。由于体质不同,即成之痰湿可从寒化可从热化故痰湿致ED有寒热之别。

阴茎之勃起,必须有足量的血液充养宗筋,一旦脉络瘀滞,血液运行不畅,宗筋失于血液充养,则软而不举。瘀致ED之因多系肝失疏泄和外伤宗筋,属肝所主,以血为充养,肝主疏泄之功能正常,血液运行通畅,则宗筋受血而振奋;若疏泄失常,致气机不利,血运障碍,则宗筋血少不充,萎软不用。外伤损及外肾,气滞血瘀,阳气阻遏,则发为ED;此外,痰湿、寒凝、久病、败精留滞等,也可致血脉瘀阻而致ED;宗筋局部病损,如前列腺增生、慢性前列腺炎等,湿痰瘀阻日久,阻碍血液运行,宗筋失充而ED。不论何种原因致瘀,其瘀致ED之理,皆为血瘀络阻,气血运行不畅,阴茎充血障碍,久则阴茎痿而不举。淤血是一种病理产物,同时又是一种致病因素。从西医理论看,正常阴茎的勃起过程实际上是一系列的神经血管活动,勃起的程度取决于动脉流入血量和静脉流出量之间的平衡,也就是说需要充分的动脉血输入、有效阻断静脉血的回流和健全的神经反射通路配合,而更重要的是血液灌注系统综合作用的结果又必须通过血流动力学的变化来完成。从这个意义上说.任何原因只要影响到阴茎动脉血流灌注或静脉充盈障碍均可导致ED。国内学者研究发现,各种类型ED患者的血液流变血多数发生异常改变,如血液黏稠度增高等。可见,不论从中医角度还是从西医学角度看ED的发生都与淤血有关。

主证:阴茎勃起障碍,腰膝酸软而痛,畏寒肢冷,下肢尤甚。

次证:面色恍白或黧黑,精神萎靡,晕眩耳鸣,阴茎寒凉,精冷滑泄,小便清长,夜尿多,舌淡,苔薄白,脉沉细迟。

辨析:见于功能性、内分泌性ED。大病久病后,老年,下丘脑垂体肿瘤、甲亢或甲低、高PRL血症、原发性睾丸功能低下,以及皮质醇增多症、肾上腺功能不足等疾病。亦包括糖尿病后期,肝硬化、慢性肾功衰。

方药:右归丸加减。

中成药:右归丸、金匮肾气丸、全鹿丸、参茸片。

50岁以上患者,实验室检查雄激素偏低者,宜配合应用睾酮替代疗法(TST),但应注意监测前列腺,可配合使用VCD或ICI治疗。

主证:勃起功能障碍,阳器易兴却举而不坚,腰膝酸软,舌红少津,脉细数。

次证:形体消瘦,潮热盗汗,咽干颧红,五心烦热,眩晕耳鸣,失眠多梦,遗精,溲黄便干。

辨析:多见于大病久病后,老年体弱,慢性肾功衰,糖尿病,甲亢或甲低、高血压。

方药:滋水清肝饮加减,杞菊地黄汤加减。

中成药:杞菊地黄丸,三才封髓丹,知柏地黄丸。

主证:勃起功能障碍,发育迟缓或早衰,健忘恍惚,精神呆钝。

次证:发脱齿摇,足软无力,眩晕耳鸣,动作迟缓,舌淡红,脉沉细无力。

辨析:多见于内分泌性ED,如原发性睾丸功能低下,以及皮质醇增多症、肾上腺功能不足,PADAM等所致ED。

方药:六味地黄丸合龟鹿二仙胶加减。

中成药:五子衍宗丸、龟灵集散、全鹿丸。

主证:突发勃起功能障碍,有大惊卒恐史,或房事时被惊扰史。

次证:胆怯多疑,精神苦闷,心悸失眠,舌淡,苔薄,脉弦细。

辨析:多见于功能性ED,有大惊卒恐史,或房事时被惊扰史,以及害怕妊娠,害怕染上传染病,初次性交失败后的自卑心理等。

方药:启阳娱心丹加减。

中成药:天王补心丹或安神定志丸合金匮肾气丸、补中益气丸。

主证:勃起功能障碍,病程长,腰膝酸软,舌淡红暗,或有瘀斑、瘀点、苔薄。

次证:性欲低下,精神不振,足跟疼痛,脉沉细涩。

辨析:多见于器质性和混合型ED,老年,糖尿病,慢性肾功衰等所致ED。

方药:金匮肾气丸合桃红四物汤加减。

中成药:金匮肾气丸合小活络丹。

宜加用万艾可(用量4~10次),或配合使用VCD。

主证:勃起功能障碍,有精神因素的病因,精神抑郁,意志消沉喜猜疑,两胁胀闷或疼痛,舌淡红黯或红黯,苔薄,脉弦。

次证:失眠多梦,善叹息,咽干或咽中如有异物堵塞。

辨析:多见于功能性ED,如因夫妻关系不睦,或工作紧张,人际关系复杂,性交环境不良,精神压力过重等所致ED。也常见于内分泌性以及混合型ED。

方药:柴胡疏肝散加减;达郁汤加减。

中成药:柴胡疏肝丸,逍遥丸。

对精神压力过重或伴有器质性疾患者,可适当加用万艾可(用量1~2次),配合心理治疗;对性欲低下者可少量使用十一酸睾酮,以提高性欲。

主证:勃起功能障碍,口苦,小便短赤,舌红,苔黄腻,脉滑数或弦数。

次证:胸胁胀痛灼热,阴囊潮湿、瘙痒或臊臭坠胀,头晕身重,下肢酸困。

辨析:常见于伴有生殖道感染,或酗酒嗜烟患者。

方药:龙胆泻肝汤加减、固真汤加减。

中成药:龙胆泻肝丸(片)。

有明确证据的生殖系感染者,可酌情选用敏感抗生素。

主证:勃起功能障碍,伴少腹胀痛,遇冷加重得热痛减;或见痛势拘紧而睾丸、阴囊上缩。

次证:阴囊潮湿寒冷,舌淡紫或青,苔白,脉沉迟或沉弦。

辨析:常见于深井或水中作业者,慢性生殖系炎症久治不愈者所致ED。

方药:温经汤加减、暖肝煎加减。

中成药:小活络丹合右归丸或青娥丸。

主证:勃起功能障碍,每临交媾即恐惧,胆小多疑。

次证:日闻声而易惊惕,夜噩梦,舌淡,苔薄,脉细弱或结代。

辨析:多见于功能性ED,如初次性交失败后的自卑心理,或幼年遭受精神创伤,性格孤僻,缺乏自信心,对性问题持消极态度等所致ED。

方药:十味温胆汤加减。

中成药:归脾丸、天王补心丹合逍遥丸。

主证:勃起功能障碍,神疲气短,失眠多梦。

次证:怔忡健忘,少食腹胀,倦怠乏力,纳呆便溏,舌淡,脉细弱或结代。

辨析:多见于思虑太过,久病或大病后体虚,大量或较长时间脑力劳动后,或见于冠心病、肺心病等。

方药:归脾汤加减、七福饮加减。

中成药:归脾丸。

对性欲低下者可少量使用安特儿,以提高性欲。

主证:勃起功能障碍日久,有腹、腰、阴部刺痛或包块,舌紫黯、黯或有瘀斑、瘀点。

次证:胸闷不舒,口渴不喜饮,脉涩、动或结代。

辨析:多见于血管性、神经性ED,如大动脉炎、髂内动脉闭塞症、高血压、动脉硬化、盆腔外伤、盆腔或阴部手术损伤神经等疾病。也见于冠心病、糖尿病所致ED。

方药:血府逐瘀汤加减。

中成药:小活络丹合无比山药丸、金匮肾气丸。

对于神经性,宜首选针灸治疗;也可加用万艾可治疗(用量6~10次,如盆腔外伤、盆腔或阴部手术损伤神经等)。

主证:素体肥胖之人,阳痿不起,肢体困重,痰多,口中粘腻,舌淡苔厚腻,脉沉滑或弦滑。

次证:头晕目眩,胸闷恶心。

方药:二陈汤合健脾丸加减。

中成药:苍附导痰丸。


\section{第二节 早 泄}

早泄是指男子在阴茎勃起之后,未进入阴道之前,或正当纳入以及刚刚进入而尚未抽动时便已射精,阴茎也随之疲软并进入不应期的现象。其判断标准为男女双方中某一方对射精潜伏期不满意或企图延长射精潜伏期失败,即可认为是早泄。属中医滑精、遗泄、鸡精范畴。

祖国医学认为,本病虽与五脏有关,但以心、肝、肾尤为重要。

《证治概要》:“凡肝经郁勃之人,于欲事每迫不育,必待一泄,始得舒快。此肝阳不得宣达,下陷于肾,是怒之激其志气,使志气不得静也。肝以疏泄为性,既不得疏于上,而陷于下,遂不得不泄于下。”说明肝与性生活的调节及精液的排泄有关。

《辨证录·种嗣门》曰:“男子有精滑之极,一到妇女之门,即便泄精,欲勉强图欢不得,且泄精甚薄,人以为天分之弱,谁知心肾两虚乎。”强调心肾两虚是早泄的病机所在。

朱丹溪则进一步指出:“主闭藏者,肾也,司疏泄者,肝也,二脏皆有相火,而其系上属于心。心,君火也,为物所感则易动,心动则相火亦动,动则精自走。”由此可见,肾主藏精,心主神明,肝主疏泄,三脏与精液的闭藏和施泄密切相关,共司精关之开合。若肾气健旺,肝疏泄有度,心主得宣,阴平阳秘,精关开合有序,则精液藏泄有度。若屡犯手淫、房劳过度、惊恐伤肾或者劳心过度,耗伤心之阴血,或者愤怒伤肝,郁郁不得志,或者饮食起居等,均可影响肾之封藏、肝之疏泄和心之藏神,以致封藏失职,疏泄不利,神明失守,使精关约束无权,精关易开,精液外泄,而交则早泄。总之,早泄与心、肝、肾密切相关,其制在心,其藏在肾,其动在肝。

对早泄的治疗,当根据不同病机,采用虚则补之、实则泻之、虚实夹杂者宜消补兼施的治疗原则。早泄兼有虚证,补肾以平补一法最为得当,若滋水不宜过于滋腻,若补肾阳不宜过于温燥,在平补的基础上,或加健脾益气,或加滋肝养肝,或加益心养心。早泄兼有实证者,当辨其标本缓急,治标则以清利为主,惟甘淡一法最为得当。若利湿宜淡渗,若清火宜甘寒,在甘淡清利的基础上,或加清肝利胆,或加清肾坚阴,或加清心导赤诸法。例如,属于湿热者重在清利,慎用补涩,中病即止,以防伤正;阴虚火旺者,既要滋阴,又要清虚火;阴阳两虚者,应阴阳并补。总以调理精关,使精关开合有度,精泄得控。在实际的临证过程中,心、肝、肾脏腑的病症很少单独出现,而是往往多相互夹杂,应当灵活运用,不可拘泥一格。

证候:射精过快伴阴茎勃而不坚,精液清稀,性欲减退,遗精,腰膝酸软,尿频。舌淡苔白,脉沉细无力。

首选方剂:金匮肾气丸加减。

中成药:男宝,每次6粒,每日3次;或龟龄集,每次6粒,每日2次;或五子衍宗丸,每次9g,每日2次;或金锁固精丸每次9g,每日2次。

证候:射精过快,伴性欲亢进,头晕目眩,口苦咽干,急躁易怒,阴囊潮湿,小便黄赤,或淋浊。舌质红,苔黄腻,脉弦滑或弦数。

首选方剂:龙胆泻肝汤加减。

中成药:龙胆泻肝丸,每次9g,每日2次;或甘露消毒丹每次9g,每日2次。

证候:射精过快,阳强易举,遗精,伴五心烦热,虚烦不寐,头晕耳鸣,潮热盗汗,腰膝酸软。舌红少苔,脉细数。

首选方剂:知柏地黄汤加减。

中成药:知柏地黄丸,每次8粒,每日3次口服;或左金丸,每次9g,每日2次;或大补阴丸每次9g,每日2次。

证候:射精过快,肢体倦怠,面色无华,形体消瘦,心悸气短,乏力,健忘多梦,自汗,纳呆便溏。舌淡苔白,脉细。

首选方剂:归脾汤加减。

中成药:补中益气丸,每次8粒,每日3次口服;或归脾丸每次8粒,每日3次口服;或十全大补丸每次9g,每日3次口服。

证候:射精过快,时有心悸耳鸣,多梦易醒,腰膝酸软,每逢性事之后易疲乏,阴囊潮湿,舌淡苔白,脉沉细。

首选方剂:柴胡桂枝龙骨牡蛎汤加味

证候:初次性交时阳物易兴,但精神紧张,未交先泄,之后精神负担加重,虽能举阳,但合房早泄,不甚尽意,神志不安,心烦面赤,舌红苔少,脉细弦数。

首选方剂:三才封髓丹加减。

症候:射精快,伴心烦神乱,失眠多梦,怔忡惊悸,耳鸣耳聋,腰膝酸软,舌红脉细数。

首选方剂:酸枣仁汤合朱砂安神丸加减。

金樱子酊:金樱子100g,党参、续断、淫羊藿、蛇床子各50g,白酒2500ml,将药物放如酒中密封半个月后服用,每日两次,每次饮用20ml,10天为一个疗程。适用于肾气亏虚型早泄。

1.临睡前按摩,每晚睡觉前用手按压关元、气海、中极等穴位。点按会阴法,牵拉法等都有一定疗效,需要在女方的配合下治疗,疗效更佳。

2.取穴气海、曲骨、足三里、膀胱俞。以上诸穴均用手捻补法,行针15分钟,每日1次,连针10次为1疗程,一般做两个疗程。针曲骨向阴部方向刺,与气海二穴都有向阴茎放射感疗效最好。此法适于肾气亏虚型患者。

3.取穴以足太阴脾经为主:三阴交、阴陵泉、中极,用泻法。此法适于湿热阻滞型患者。

4.取肾俞、八髎穴。以毫针刺,用补法。每次留针20分钟,15次为一疗程。可分别与前几组穴位交替进行。适用于各型患者。

5.取穴肾俞、志室、命门、三阴交等,平补平泻。根据证型的不同进行加减配穴。肝经湿热加丰隆、陵泉、太冲、太溪,用泻法;阴虚火旺加涌泉、照海、太冲等,宜平补平泻;肾气不固加中极、关元等,用补法;心脾两虚,加脾俞、内关、神门等,用补法。每日1次,10天为1疗程。

运用龟头涂药法,中药熏洗法可以降低龟头的敏感性,以达到治疗目的。另外还有灌肠法,敷脐法,外洗法都可配合使用以提高疗效。

1.用五倍子适量煎汤,于性交前熏洗会阴部及阴茎数分钟,待水温下降至40℃左右,可将龟头浸泡在药液中5~15分钟。每晚一次,10~20天为一疗程,待龟头皮肤黏膜变厚变粗即可。

2.用细辛、丁香、五倍子各15g,蛇床子30g。上药共研粗粉,浸于95\%乙醇200ml内,15天后,将浸出液过滤,装入瓶中保藏。性交前以浸出液涂擦阴茎龟头部位,反复4~5次,然后同房。

3.蛇床子、石榴皮、乌梅、细辛各10g、菊花5g,水煎冷却后,于性交前浸泡会阴部及龟头。每次10~15分钟,10~20天为一疗程。

加强身体锻炼,可保持充沛的精力和体力行房,以免身体不支而养成快速射精的习惯。

保持规律的性生活,忌过度纵欲和长时间的禁欲。

因本病与精神因素密切相关,故保持良好心态,增强治疗信心,对本病的康复极其重要。

饮食要清淡而富有营养,禁辛辣、肥甘厚味,避免伤及脾胃,以防湿热内生加重病情。

金樱子15g,粳米100g。两者混合煮粥,早晚温热服用。用于肾气虚损之早泄。

莲子、茯苓、麦冬各等份,白糖适量,桂花适量。先将莲子肉、茯苓、麦冬共研细末,加入白糖、桂花拌匀,用水合面蒸糕,晨起作早餐服用,每次50~100g。用于心脾两虚之早泄。

五味子100g,蜜1000g,将五味子水浸后去核再用水洗,尽量洗去其味,过滤,加入上等蜜,在火上慢熬成膏,收存瓶中。每次食用1~2小勺。具有滋阴涩精之功。用于肾阴不足型早泄。

鲜车前草60g~90g,猪小肚200g。将猪小肚切成小块加水,与车前草煲汤,加适量盐。饮汤食肚,有清利湿热之功。用于湿热蕴结下焦之早泄。

心肾不交,相火妄动而遗精早泄者少吃葱、姜、蒜、辣椒、胡椒、肉桂、花椒、丁香、茴香、洋葱、雀肉、羊肉、狗肉以及烟酒等助火兴阳伤阴之品;肾虚不固遗精早泄者少吃生冷性寒,损伤阳气的各种冷饮、田螺、蟹、柿子、河蚌、鸭子、苦瓜、荸荠、薄荷、西瓜等食品。


\section{第三节 女性性功能障碍}

历代中医文献中没有“女性性功能障碍”的病名,其相应的临床表现散在记载于古代中医性学文献中。性欲低下、性唤起障碍在古代中医文献中称之为“阴冷”、“阴寒”、“女子阴痿”、“玉门冷”等。阴冷最早出现在《诸病源候论·卷三·虚劳病诸候》,指男女自觉阴器寒冷,甚至冷及下腹,性欲低下的一种疾病。《增补内经拾遗方论》有广嗣丸置入阴道治疗阴冷的记载:“情窦不开,阴阳背驰,则以本方纳之户内,以动其欲。庶子宫开,两情美,真元媾合,如鱼得水。”性欲亢进古代文献中称之为“花癫”,又叫“花旋风”、“花风”,最早见《石室秘录》:“如人病花癫,妇人忽然癫痫,见男子则抱住不肯放,以及思慕男子不可得,忽然病如暴风疾雨,罔识羞耻,见男子则以为情人。”此外《辩证奇闻》也有记载:“妇人一时发癫,全不识羞,见男子而始怡,遇女子而甚怒,往往有赤身裸体而不顾者…。”《左传》中也有记载:“欲男子不得”而致腰背痛,月事不下。性交痛称之为“嫁痛”,(《千金要方》)、“阴肿痛”,(《诸病源候论》)、“吊阴痛”,(《竹林女科证治》)。《各科证治选编》中还记载了治疗方法:“女人合房,阴中痛楚,没药一钱,末,空心好酒三盏调下”。而《潜斋医学丛书》记载的“玉门大”、“阴宽冷”、“阴宽大”则指女子阴道口松宽弛缓,交媾不快者。以现代医学观点看,部分阴宽患者可能与女性性欲高潮障碍、女性阴道缺乏间歇性收缩有关。

1.肾为先天之本,主生殖。《素问·上古天真论》:“女子七岁,肾气盛,齿更发长;二七而天癸至,任脉通,太冲脉盛,月事以时下,故有子;三七,肾气平均,故真牙生而长极;四七,筋骨坚,发长极,身体盛壮;五七,阳明脉衰,面始焦,发始堕;六七,三阳脉衰于上,面皆焦,发始白;七七,任脉虚,太冲脉衰少,天癸竭,地道不通,故形坏而无子也。………………肾者主水,受五藏六府之精而藏之,故五藏盛乃能写。今五藏皆衰,筋骨解堕,天癸尽矣,故发鬓白,身体重,行步不正,而无子耳。”肾藏精,乃生殖之本,女子成年后肾气充,精血旺则性欲自然正常。肾对生殖与性功能的调节,主要是精的作用,阴精是基础,肾阳是作用的原动力,若无肾阳的催动,则天癸不至,月事不来,阴痿、阴冷等。若先天禀赋不足,或早婚多产、房劳过度,耗伐肾气,耗伤肾精;或久病大病,或惊恐伤志、损伤肾气,冲任脉虚,气血不足,则性欲冷漠,不易兴奋,甚则厌恶房事;若肾阴虚,失其濡润作用.以致阴虚火旺,虚阳当令,则可导致性欲过强。

2.女子以肝为先天,以血为本,肝司血海,肝脉绕阴器,抵少腹,挟胃贯膈,布胁肋,经乳头上巅顶,故肝与会阴、少腹、乳部有密切生理关系。肝主疏泄,喜条达,肝的疏泄和肝血畅旺,直接调节着子宫、月经的通调,阴部的肌肤、毛际的充养,性功能的正常。肝藏血,主疏泄,司血海而为月经之本,肝脏对子宫的生理功能起着重要的调节作用。肝气郁滞,肝血不足,是导致月经不调、痛经、闭经的主要原因。而月经不调、痛经、闭经可直接导致性欲低下,性功能减退。肝为将军之官,性喜条达而恶抑郁,肝的生理变化直接与人的精神情志有着密切的关系,这些精神情志的变化又往往通过肝直接影响了女性正常的性生理活动,导致了性功能失常。如肝气郁结则人的精神郁闷、情志压抑、欲念难启,引起性欲遏抑、性高潮障碍;肝火旺盛则人的精神躁动、喜怒无常,引起性欲亢进、性功能失调;肝血虚则人的精神恍惚、情志无常,引起性欲减退、性功能低下。肝之经脉“循股阴,入毛中,过阴器,抵小腹”与任脉交会于“曲骨”穴,而任脉起于胞中,故肝经通过任脉与子宫相联系,肝脏对子宫的生理功能起着重要的调节作用。又肝主筋,外阴为宗筋所聚之处,当肝血不足则筋失所养,外阴弹性降低甚至萎缩,而引起性交疼痛、性交恐惧,久之则发展为性欲低下。

冲任督同起会阴,而胞宫系于肾,又连带脉,内通肝肾,为足厥阴肝和足少阴肾所管辖,与肝肾极为密切。肝肾同源,同居下焦,且同寄水火,体阴用阳。肾水与肝木为母子关系,肝气条达,有利于肾精的封藏,而肾精封藏则滋养肝阴,“肾水滋养肝木”,使肝疏泄条达,功能得以正常发挥。在肾的作用、肝的调节下,来维持正常的女性性功能。古曰:妇人病,三十六种,皆由冲任劳损所致。而调治冲任,皆离不开调补肝肾。只有肾的生理功能正常,肝的疏泄功能正常,冲任二脉才会经血盛满,流注畅通,而肝藏血,肾藏精,精血同源,精能生血,血亦能生精,精血相互化生。补肾即养肝,从临床表现看:房事淡漠:厌房事,无快感,阴中干涩,性交疼痛,腰酸腿软,心烦不宁等症状无不与肝肾有关系。因此肾虚肝郁是造成上述性功能障碍的根本原因。

3.脾胃为后天之本,气血生化之源。《医宗必读》:“一有此身,必资谷气,谷入于胃,洒陈于六腑而气至,和调于五脏而血生,而人资之以为生者也,故曰后天之本在脾。”若脾运化水谷的功能正常,才能为化生精、气、血、津液提供足够的养料,则脏腑、经络、四肢百骸,以致筋肉皮毛等组织都能得到充分的营养,而进行正常的生理功能。若脾失健运,水谷精微不能很好地吸收和输布,血脉不充,阴器不养,则出现倦怠乏力,性欲低下。脾又主运化水湿,若功能强健,既能使全身各组织器官得到水液的充分滋养,又能防止水液在体内部发生不正常的停留即痰湿生成。痰湿即成则贮留体内,一旦痰湿过盛,湿浊下注,困阻下焦,经络受阻,使阳气不能伸达阴器,血液不能充养阳道,气不至则无以令阴器振兴,血不充则失其濡养,而至性欲低下、性交痛。

性欲减退,性厌恶,性唤起障碍,性高潮障碍,腰膝酸冷,神疲乏力,形寒畏冷,阴冷阴痛,小便频数或夜尿频,面色晦暗,舌质淡,苔薄白,脉沉细无力,尺部尤甚。

治法:温肾助阳。

方药:右归丸加减。

性欲低下,性唤起障碍,阴中干涩,性交时涩痛,或厌恶性生活,头晕耳鸣,腰腿酸软,五心烦热,潮热盗汗,失眠健忘,午后颧红,舌质红,苔少或无苔,脉沉细数。

治法:滋补肾阴。

方药:左归丸加减。

性欲减退,性厌恶,性高潮障碍,腰腿酸软,头晕耳鸣,神疲乏力,小便频数而清,或尿后余沥不净,或夜尿频多,舌淡,苔薄白,脉沉弱。

治法:益气补肾。

方药:大补元煎加减。

性欲减退,性厌恶,性高潮障碍,早婚多产,发育迟缓或早衰,健忘恍惚,精神呆钝,发脱齿摇,足软无力,眩晕耳鸣,动作迟缓,舌淡红,脉沉细无力。

治法:益肾填精。

方药:六味地黄丸合龟鹿二仙胶加减。

性欲亢奋,欲交之望不能自制,交合过频,带下色黄或夹赤,或梦交,阴中灼热或拘急不适,口干咽燥,头胀头痛,形体消瘦,倦怠乏力,而色苍白,两颧浮红,舌质淡红,少苔或无苔,脉细数无力。

治法:滋阴降火。

方药:知柏地黄丸加减。

性欲低下,厌恶房事,或性交痛,胸胁、乳房胀痛,心烦易怒,善太息,或闷闷不乐,舌质正常或紫暗,脉弦或细弦。

治法:养血柔肝,理气解郁。

方药:柴胡疏肝散加减。

性欲亢进,欲交之望难忘,或梦交,头痛耳鸣,目赤昏花,口苦咽干,烦躁易怒,胸院满闷,乳房、小腹胀痛,带下量多,色黄稠粘,秽臭,外阴瘙痒,舌红,苔黄,脉滑数。

治法:清热化湿,理气通络。

方药:龙胆泻肝汤加减、固真汤加减。

性欲低下,厌恶房事,性交痛,月经不调、痛经、闭经,口干咽燥,心中烦热,两目干涩,头晕目眩,舌红少苔,脉弦细数。

治法:滋阴柔肝,养血通络。

方药:一贯煎加减。

性欲低下,厌恶房事,前阴寒冷,甚或阴缩,形寒肢冷,面色苍白,蜷卧,口淡不渴,痰涎清稀,小便清长,大便稀溏,舌质淡,苔白而润滑,脉迟或紧。

治法:暖肝散寒,理气通络。

方药:暖肝煎加减。

阴道痉挛,或性交疼痛、出血,伴虚烦不寐,心悸健忘,舌红少苔,脉沉细数。

治法:滋阴养血,交通心肾。

方药:黄连阿胶汤合交泰丸加减。

性欲低下,或性高潮缺失,甚则闭经,心悸怔忡,四肢欠温,舌淡苔白滑脉沉细。

治法:振奋心阳。

方药:二仙汤合四逆汤加减。

性交疼痛,或性欲低下高潮缺如,伴月经淋漓不尽,带下绵绵不断,心悸气短,舌质淡嫩,脉细弱。

治法:补益心气。

方药:归脾汤加减。

性欲低下,性交疼痛、出血,或性高潮缺如,阴道痉挛,伴月经量少色淡,甚则经闭,阴道干涩,心烦少寐,舌红苔薄,脉细数。

治法:补血养心。

方药:四物汤合一至丸加减。

性交疼痛,出血,性厌恶,伴阴痒灼热,赤自带下,月经紊乱,心烦易怒,舌尖红,苔黄腻,脉数。

治法:清心安神,滋阴清热。

方药:黄连清心饮合三才封髓丹。

性欲减退,性高潮障碍,或厌恶房事,体倦乏力,精神萎靡,气短嗜睡,而色萎黄,胃呆纳少,舌体胖大,质淡齿痕,苔薄白,脉细弱。

治法:健脾益气,和中化湿。

方药:六君子丸加减。

性欲低下,性高潮缺乏,或厌恶房事,健忘心悸,失眠,气短乏力,精神不易集中,或善虑多疑,心烦不安,面色苍白,舌质淡,苔薄白,脉细弱。

治法:补益心脾,益气养血。

方药:归脾汤加减。

(贾金明 罗少波 董佳晨)

1.郭应禄,胡礼泉.男科学.北京:人民卫生出版社,2004

2.贾金铭.中国中西医结合男科学.北京:中国医药科技出版社,2005

3.张元芳,吴登龙.男科治疗学.北京:科学技术文献出版社,2002

4.马永江,安崇辰.中西医结合男科学.北京:中国中医药出版社,2001

5.王琦.王琦男科学.郑州:河南科学技术出版社,1997

6.谢英彪,颜培增,刘光隆.食物营养与食疗宝典.北京:人民军医出版社,2007

7.陈剑飞,金保方,李相如,等.徐福松教授辨治早泄经验.南京中医药大学学报,2008,24(6):366-369

8.李兰群.清肝疏肝柔肝法治疗早泄临证心得.北京中医药大学学报(中医临床版),2008,15(1):42-43

9.王东坡.王琦教授治疗男科疾病经验介绍.新中医,2005,37(10):12-13

10.王劲松,曾庆琪,徐福松.早泄辨治七法.四川中医,2005,23(1):11-12

11.陈继明,张传涛.王久源治疗早泄经验.中医杂志,2007,48(2):123-124

12.王清任.医林改错.北京:人民卫生出版社,1976

13.朱丹溪.丹溪心法.上海:上海科技出版社,1983

14.傅青主.傅青主男科.福州:福建科技出版社,1984

15.季羡林.中国养生术.北京:中央编译出版社,2008

16.刘达临.中国历代房内考.北京:中国古籍出版社,1998

17.段成功,刘亚柱.中国古代房中养生秘笈.北京:中国古籍出版社,2000


\chapter{第二十四章 性学研究评定量表}

性医学的研究内容较为广泛,涉及性生理学、性心理学、性药学和性临床学等相关内容,其中主要包括以下几方面:

(1)人类性器官的解剖学与生理学基础(如激素、神经递质等);

(2)人类性发育的生物学与遗传优生学;

(3)人类性心理发育与性心理健康,异常性心理表现及治疗;

(4)人类性行为、性反应及随年龄的变化;

(5)性卫生与性保健;

(6)性别角色、性取向及相关问题;

(7)男女性功能障碍及预防与治疗;

(8)性传播疾病与艾滋病的预防与治疗;

(9)疾病对人类性行为的影响;

(10)生育与性(含避孕、节育或不孕不育与性);

(11)药物与性(包括性药学);

(12)性治疗的理论与实践;

(13)传统性医学。

性医学的研究对象主要是指与人类性行为有关的各类问题、困惑和障碍,可分为个体性、群体性和社会性。

由于人类性行为涉及的范围非常广泛,包括与性有关的思维活动、情感反应、言语举止、行动等。而每个人因其生活环境、接受教育的程度、生理心理和社会文化背景方面的差异,在对性刺激的反应、性情绪体验和性活动方式上也存在一定的差异。不同性别、不同年龄阶段的性问题反映到具体的个人身上时也不尽相同。如手淫是青春期较多见的一种性行为,可手淫发生的频度、刺激的方式和认识的态度则不尽相同,某些人可以频繁手淫,而有些人则只是偶尔为之;有些人认为手淫会损害健康,因而终日苦恼、困惑,有些人则对手淫持无所谓的态度,毫不在意,心态释然;有些人在手淫时常伴有性幻想,而有的人只偶尔伴有或不伴有性幻想。同时,性幻想的对象千奇百怪,可以是异性,也可以是同性,也可以是有生命的动物或其他无生命的物体等。又如在男性性功能障碍中勃起功能障碍是较常见的一类,但其发生的原因、频度、表现的形式在不同个体身上却可能存在很大的差异。性医学的宗旨是帮助人们解决生活中所遇到的各类性问题,而在性咨询和在性治疗中,所面对的都是某个具体的个人,因此针对不同的个体采取的干预、调适或治疗的手段、模式都不尽相同,所以需要制定一套适合该患者的具体方案。

性问题除存在个体差异外,有些则表现为普遍性和共性,如青春期性问题、同性恋性问题、妇女特殊时期的性问题、性犯罪群体的性问题等。对于这类问题需进行必要的群体性研究,寻找群体成员之间的共性,把群体研究作为个体研究的一个重要补充,以便能及时、有效地解决这些性问题。

由于人类性行为除具有自然属性外,还具有社会属性,因此许多性问题都必然会受到社会文化背景的广泛影响。如人们的思想观念、价值判断、道德水准等,无不受到其教育程度、宗教信仰、传统习俗、社会文化环境等的影响。反过来,性问题也会影响社会文明的发展,两者相辅相成,相互影响。所以,性医学也必须结合社会学的观点来研究和分析性问题,寻找解决的有效办法和途径。


\section{第二节 性学研究评定量表}

性学研究用诊断标准、定式检查和评定量表的发展,是20世纪60年代以来性学研究方法中较为重大的进展。这些都属于研究工具的范围,应用这些研究工具,使研究结论更具有客观性、可比性和可重复性。

本节着重介绍性学研究量表的基本性质,及其在性学临床和研究中的应用。

没有比较,就没有研究。研究工作的最基本的一条,就是进行比较。例如,把新药物和标准药物进行比较,把某一疾病和其他疾病或正常群体做比较,或者是把一病种的各个类型做比较,如此等等。这类比较可以是绝对的,就是谁高谁低,孰优孰劣,好似定性的比较;也可以是相对的,如几种疗法中,哪一种最优,哪一种次之,哪一种较逊,哪一种最差。这里便涉及一个概念,便是“等级”,这是量表中最基本的概念之一。

在日常生活中,我们经常应用各种表示等级的词汇。例如,我们在评价一个人好坏时,我们会把这个具体的人,和一般的人比较,并分成若干等级:最好、很好、较好、一般、较差和最差等。这便是从好到差系列的7级评定法。把这样的方法规范化,应用于性功能及性功能的心理社会因素的评定,便成为性学研究用的量表。

评定量表中量表一词的原文是“Scale”。这个词表示为“尺度”、“标度”、“刻度”、“等级”和“比例尺”等。换句话说,是表示数量的概念。那么,这里就有一个十分基本的问题,人的正常或异常心理活动,能不能正确的测量或估计,开始时人们是有怀疑的。然而,近十几年来的实践证明,人的心理活动是可以评估的,它广泛的运用于精神对智力、人格、心理状态的评价。

评定量表的分类,就其内容来分,可以分为诊断量表和其他量表;就其评定方式而言,可以分为大体评定量表和症状(分项)评定量表,或自评量表或他评量表,或观察量表或检查量表等。

量表的种类很多,它们有的较好有的较差,如何来评定量表的质量呢?换言之,如何来为自己选择有效的量表呢?

1.信度(reliability)又名可靠性,是指量表本身稳定性及可重复性。常用的检查方法是应用联合检测法和再检查法。

联合检查法,又名检查者观察者法。即由二位或更多的评定员,同时检查患者,其中一人作为检查者,其余为观察者。然后,分别独立评分,最后比较评分结果,统计分析各检查者间评分的一致性和相关性。如果量表评定的结果是可以重复的,那么,在同一场合,观察到相同的情况,应该得到相同的评分。在症状量表中,联合检测法是最常用的检验信度的方法,这也是训练评定员的重要方法之一。

检查和再检查法,又名重测法,用于检查量表的稳定性。即在相隔不长的时间内,由统一评定员或两名评定员,分别来做评定,然后比较评分结果的相关情况。这一方法具有其局限性。因为如果间隔时间太长,患者的症状已经起了变化;而时间过短,由同一评定者评定可能有意无意地受上次评定的影响;若有不同的评定员评定,则又加上不同的评定员这样一个变因。

2.效度(validity)又名真实性,指量表的评定结果能否符合编制目的,以及符合的良好程度。就症状量表而言,主要的是指评分结果能否反映病情的严重程度及变化。

常用的效度检验有:内容效度和平行效度两种。

所谓的内容效度即从其内容来看,是否符合量表所试图检查的要求。例如,以焦虑量表而言,它是否包括了精神性、运动性及躯体性焦虑这三个方面。在每一个方面是否包括了常见的重要的症状项目。内容效度通常无理想评估指标,主要通过专家对内容的评价和编制量表时严格按预定的定义、行为取样的范畴进行项目筛选来保证。每一项目的定义是否合理,应符合通行的学术观点。

平行效度,有两种方法,一是和临床判断相比较,又名经验效度;另一类是和公认的其他同类量表的评定结果相比。以某种性功能问卷为例,可比较性医学专家对性功能状况的评价和问卷评分的一致性;临床判断的疗效和治疗前后量表评分差值的相关性。

如果是诊断量表,则以敏感性和特异性作为效度指标。若与所谓“金标准”相比,所得结果称为标准效度或效标效度。

3.结构效度 反应编制的量表所依据理论的程度。如编制一个勃起障碍评定量表,必定依据有关勃起障碍的症状及理论。同时,所编制的量表是否符合原来依据的理论框架,也可用结构效度来检验。结构效度主要用于新编量表及量表本身的研究和分析。就临床应用而言,内容效度及效标效度就已经足够。

另外,评价量表的质量,还应考虑到量表的可行性(practicality),量表中的每一个选项对于回答者来说都应该容易理解,是否会因用词太专业化导致被试者产生误解而影响回答及测试结果。在临床使用中,笔者发现有的问卷所使用的词汇超出了大多数被试者所能理解的范围,一些用语使被试者产生模棱两可的感觉。可比性(comparability)也是选择量表应考虑的一个指标。被试者做答的数据应与标准化数据相比较才有意义。因此,必须考虑到标准化样本是否适合你的被试者。如果标准化的样本是50~70岁的男性,而你的被试是一个年轻的女性,则标准化被试对你的被试者来讲就失去了意义。同时也应考虑到许多量表或问卷的常来自大学生,而他们并不一定能代表你的被试者。最后应考虑到这次测试是否将花费您和您的被试者太多的时间及费用。

性功能状况评定量表的研制始于20世纪60年代,70年代达高峰,80~90年代仍不断有新量表产生,量表在性学研究领域中能蓬勃发展,除了它自身的原因(如性交行为的道德属性从而决定了它不能在实验室或第三者(医生)在场的情况下对其进行“客观”评价。况且,作为身心疾病,性功能障碍与心理学因素关系密切,而这些因素无法用现有生物学指标去评价)外,与量表本身的价值也有极大的关系。

作为客观标准,它并不是根据自己的意愿,随意把几个问题凑合在一起的问卷。性学研究所用的量表是经有关专业人员周密设计,对题项的反复斟酌,经临床验证及统计处理而成。其每一项目:①应具有被多数权威所认可的内容效度。②在不同的样本中具有一定的通过率。比如,一份合格的性功能评定量表的项目通过率,患者组和正常组得分之差≥0.30左右。③项目与总分、因子分或量表分之间内部结构具有一致性,相关值≥0.30。达不到统计要求的项目,需修改或删除。然后对初步定下的量表进行信度和效度检验,以确定其稳定性和准确性。诊断量表还需要报告其与临床诊断相比较而言的诊断“特异性”和“敏感性”。具有这种客观标准,不论是何人、何时、何条件下来评定受评者,根据这个客观标准来收集资料,作出等级评定,其所得的结果也比较客观,治疗评价也更准确,不容易发生与治疗医生相关的偏差。即使就他评量表而言,尽管评定者作出的评价是主观的,但其依据来源是真实的,从这种意义上来讲,同样具有相当的客观性。

对影响人们性功能状况的心理和社会因素描述,如果没有一定的数量,而只有文字描述,那么在不同地点,不同时间、不同观察结果便难以比较。评定量表使观察结果数量化,用数字语言代替文字描述,是研究样本较理想的入组指标和研究因素的变量形式,有助于分类研究,便于将观察结果做统计处理,更有利于计算机分析,研究的结果表达也更符合科学要求。

评定量表的内容全面而系统,等级清楚。用它来观察受评者,收集个体一般资料,评价心理状况各个方面,估计防治效果,一般不会遗漏重要内容。其功能相当于一份详尽的观察和晤谈大纲,并能协助评定者发现其他评估方法如观察、晤谈等方法的不足之处。在性医学门诊中,不少医生只注意患者的主诉,往往遗漏患者未提及的其他症状。比如,“忽略早泄患者的性高潮障碍;忽略患者的继发性性欲低下及心理社会文化因素;忽略患者妻子的性功能问题等等”。如使用性功能评价量表,就能全面了解在晤谈中被忽略或难以用语言表达而在评定量表中能暴露的一些信息。

评定量表能够广泛运用,一个重要原因在于各类人员较易学会操作方法,无须像心理测验那样的特殊器材和条件,一些测试工具(MMPI)的题项太多以致回答者产生厌烦情绪而抱以不合作态度,而部分量表价格昂贵也无法获得。一般来说,完成每一份量表评定通常只需10~30分钟,省时、省力、省钱,评定者和受评者都一致乐于接受。

评定量表具体的实施应按其使用手册规定的步骤严格进行。概括起来评定量表的实施过程有准备阶段、量表的填写、评定结果换算及结果解释报告这4个步骤。

在采用量表对某一人群实施评定之前,通常需要对评定者进行系统训练,选择适合的评定工具及评定场地。

评定者的训练就是组织量表的使用者(也称评定员)对有关所使用的量表理论基础进行全面的学习,并就量表的具体操作方法和结果解释进行反复练习,以达到熟练掌握所使用的量表评定方法,并能较准确的分析解释评定结果的目的。一般经过一定的训练后,正式使用量表前最好要求预试,进行一致性检验,一致性检验符合要求才能正式成为合格的评定者。训练方式有两种,一种为集中训练,这种方式受训者较多(一般每个组最好为5~10人),虽较经济,但受训者易相互干扰;另一种为个别指导,即一对一训练,虽然效果好,但不经济。因此,二种方式相互结合最为理想。

评定工具的准备就是要选择适合评定对象情况的评定量表,量表选择正确与否,直接影响了评定的质量。评定量表一般为纸笔形式,即一些表格和填表用笔,但少量评定量表有时还要求准备一些评定用的道具。

评定通常无需特别的场地,一般在安静的房间进行即可。有些特殊人群(住院患者等)的评定量表需要到受评者经常活动的地方(如病房)进行评定,以使评定结果更加准确。

性学研究用量表填写观察中,首先应特别强调对问卷的保密措施,并填写受评者的一般资料,如年龄、性别、职业等。如果说是临床用量表,尤其要注明病种,有的还要求对受评者申请评定的理由及健康或疾病史作一简要描述。量表的各项目记录或填写方法,自评量表与他评量表不同。

(1)自评量表:各项目填写前就有一简短指导语,说明评定主要目的、评定内容的范围、评定的时间界定(如评定1周内出现的现象,还是几个月内出现的现象)、频度和程度标准以及记录方法与其他要求等,这一指导语虽用文字写明,但评定者最好口头加以说明。量表的项目由受评者自己填写,独立完成填表过程。如果受评者文化教育程度低,对一些项目不理解,评定者可逐项念题,并以中性态度把项目本身意思告诉受评者。自评量表评价也称单盲式评价,当事患者可能因期望不同而对自身症状及疗效评价偏高或偏低。

(2)他评量表:评定者一般多为专业工作者。评定的依据来源,大多数通过知情者提供。所谓知情者是指最了解受评者日常生活及学习、工作情况的人,一般为受评者父母、兄弟姐妹或配偶等亲属;或是了解情况的邻里、同事、老师等。这种通过评定者自己的观察印象直接记录量表各项目的评分,这类评定者常对受评者进行过系统的观察,如病房医师、与受评者接受密切的心理学工作者等。他评量表评价,也属于双盲式评价,有研究表明,第三者他评最接近实际情况。目前我国性功能障碍症状和疗效评价大多由当事医生直接参与,这种评价掺杂了当事医生的个人动机和期望,因而,当事医生的疗效自评总是高于客观情况。

(3)结果换算:量表的项目评分需要累加因子分(或分量表分)和总分,这些分数均为原始分,很多量表要求进一步转换成各种形式的标准分或百分位,或者做加权处理,如Derogatis性功能问卷(DSFI)的结果,须将原始分转换成T分才有意义。有些量表使用手册上提供了各种转换表,使用者只需查表即可。

为了达到评定量表的使用目的,需要对各种评定结果进行分析综合,提出结论,并对其意义进行解释。量表的种类,功能不同和评定的原因不同,其解释的深度各异。一般而言,如果只打算了解受评者某方面总的状况,总分即可。如果还想了解某方面内部成分特征,则需在因子分(分量表分)水平甚至项目水平进行分析。有些量表结果解释复杂,如MMPI,应由专门心理测验工作者进行分析。

将评定主要结果、结论及解释用文字或口头形式表达即报告。对某一人群的评定结果报告类似于科研报告,比较复杂,需要进行大量的统计学处理,如果这类报告可靠,则对该项研究有指导性意义;对个体评定结果的报告,用语要精确明了,解释合理,才有科学性。一般提交报告的对象多为专业人员,报告中应采用专用术语,如均数、标准分、百分位等,结论和解释要适度,因为任何评估方法都有一定局限性,结果有程度不一的误差,故不能绝对化。有时受评者个人或家属(评定者认为有必要告知)需要了解评定结果,则多以口头形式报告,一般把专业性术语用较通俗化用语表达,但要注意其科学性。

(1)参照标准不统一 不同的评定者,限于本身专业知识角度,对一些专业名词有不同理解,或一些术语本身概念不能统一,而以自己熟悉的概念为参照,致使评定结果不一致。

(2)信息来源问题 评定者可能对受评者缺乏足够的了解,对某些症状或行为不能作出如实的判断,从而高估或低估受评者。他评量表有时采用直接评定法,而要评定的现象又不一定在当时就会出现,甚至某种心理或生理的病理特征也不一定是经常出现。

(3)集中趋势 一般人皆有集中趋势现象,为避免评定结果过于极端,而多选择中间答案,而使最终结果不真实。

(4)严格和宽容倾向 有些评定者过于挑剔,评分过严;或者喜欢选用较优级别,给分过宽;或计算登记分数不准确;或者评定者偏好不同等都难以保证评定结果的真实性和一致性。

(1)提高评定者素质 评定者必须有一定水平的专业知识和相关学科知识,通常评定者作评定之前要进行专业训练,要能够切实地把握评定目标,彻底了解所要评定的各种行为及症状的含义,充分掌握评定量表的使用方法。

(2)建立良好的主评与受评的关系 评定量表结果要精确,在进行评定时,主评者与受评者之间必须保持友好信任的关系。否则受评者可能因对评定不合作使结果不准。

(3)正确合理地使用评定量表 评定量表的作用是使心理品质或社会现象数量化的主要手段之一,但比物理的数量化的难度大,加上目前的评定量表尚在不断发展之中,并未达到尽善尽美的程度,所以在认识它们的作用时,还要认识它们的局限性。临床医生若过于依赖量表评定,在发现评定结果与临床所见不相符或不能解决自己的疑难时,往往走向反面甚至完全否定评定结果。临床医生应掌握心理测量技术,提高自己综合利用有关资料的能力,对评定结果作出符合实际的分析。

性功能评定症状量表的研制始于20世纪60年代,在那个年代,对于勃起障碍的病因,大多数学者认为一般为心因性的,因而量表内容庞杂,包括了过多的了解心因性病因的项目,如Deroatis等(1979)编制的《性功能问卷》(Deroatis sexual function inventory,DSFI)。随着医学技术的不断发展,揭示了许多勃起障碍为器质性,因而病因的探讨主要靠实验室、流行病调查研究和临床试验性治疗。性功能评定量表主要用于度量一种现时症状及受测者对性生活的满意程度。在20世纪80年代,性功能量表开始从综合量表走向评价目标单一,符合临床需要的小型量表,如Golombok&Rust(1986)编制的《性满意量表》(the golombok rust inventory of sexual satisfaction)。

1.性相互作用问卷(SII) 性相互作用问卷(sexual interaction inventory,SII)是Lopiccolo与Steger于1974年编制,用于评估夫妇双方性行为的满意感及频率。SII能有效的评估夫妇双方的交流问题及用于疗效评定。该量表由17种性行为组成(表24-1),每一种性行为由夫妇双方从6个维度来进行评价(表24-2),因此SII量表有102个问题。

表24-1 性相互作用问卷(SII)

表24-2 在SII中使用的打分表

续表

量表包括了11个分量表。量表1和2为男女频率不满意量表;量表3和4为男女自我接纳量表;量表5和6为男女性乐量表;量表7和8为男女感知正确度量表;量表9和10为男女双方接纳量表;量表11为总分,用于评价患者性功能及其夫妇满意度。大多数人完成SII不到30分钟。

研究表明SII具有良好的信度及效度(Lopiccolo Steger,1974)。重测信度:总分相关为0.81,内部一致性检验:a系数为0.88。收敛效率:总分r=0.35,p<0.01;判别效度为:患者和正常人症状判别为T=8064,p<0.01;治疗前后症状判别T=6.97,p<0.01。

2.Derogatis《性功能问卷》(DSFI)尽管许多评定者把重点放在夫妇双方的评估上,但是对于单独了解夫妇个人也是非常重要的。为了获得个人的详细信息,Derogatis与Melisaratos于1979年编制了性功能问卷(Derogatis sexual interaction inventory,DSFI)。该量表为一综合性量表,包括了10个基本自成系统的分量表。其设计与Derogatis的精神临床评定量表研制经验有关。在研制DSFI之前,他已经成功研制并发表了《主要症状问卷》(brief symptom inventory)、《情绪平衡量表》(affect balance scale)和《症状自评量表》(the self-report symptom inventory或symptom checklist90,简称SCL-90)。上述量表已经在精神病科广泛应用并获好评。Derogatis驾轻就熟地把上述内容与其他新设计量表放在一起组成了DSFI,DSFI不仅反映性功能,也反映继发于性功能障碍的心理问题,因此它是全面的综合性问卷。然而在临床上使用时,往往因题项太多,测试时间太长,问卷内容重复引起被试者厌烦而马虎作答,精神症状提问所占比例过多易使被试者反感。

DSFI的10个分量表包括:

(1)性知识量表(26项)

(2)性经验量表(24项):调查被试者已经发生过的性行为。

(3)性欲量表(7项):调查性交、手淫、接吻、爱抚、性幻想和理想中的性交频度等6个方面内容。

(4)性态度量表(30项):旨在反应被试者的保守性和开放性。

(5)心理症状量表(53项):即《主要症状问卷》(BSI),调查各种病例的精神症状。

(6)情绪量表(40项):即《情绪平衡量表》(ABS),反映被试者当时的负面情绪(焦虑、抑郁、负罪感和敌意)与正面情绪(愉快、满足、活跃和爱意)所占的比例。

(7)性别角色量表(30项):反映男性化和女性化的角色行为。

(8)性想象量表(20项):反映在各种患者头脑中出现的、尚未付诸行动的或无法付诸行动的性愿望。

(9)体象量表(男女各做其中10项):反映患者对自己的躯体现状的满意程度。

(10)性满足量表【10项性满意指数(9级自评)】:直接反映夫妇的性关系。

DSFI的信度、效度(1979):DSFI报道了两种类型的信度,一种是测量分量表项目的内部一致性信度;另一种是测量量表暂时稳定性的重测信度。DSFI各分量表的内部一致性(n=325)大多数都相当高,性知识量表的一致性检验最低(0.57),性经验量表的一致性检验最高(0.97)。重测信度也相当高(n=60;间隔14天),仍然是性知识量表最低(0.61),性态度量表的重测信度最高(0.96)。DSFI的结构效度:对380名作答者进行因素分析发现,7个因子组成DSFI的10个分量(如心理症状量表和情绪量表都由命名为心理苦恼的因子组成);DSFI鉴别效度:在一项有91例男性性功能障碍者与具有相似人口统计学资料的76名非性功能障碍者的对照研究中,9个分量表的得分(除性态度量表)及DSRI总分在两组分间具有极显著的统计学差异。在59例女性性功能障碍患者和类似人口的统计学资料的154名正常性功能的被试的对照研究中,性知识、症状、情绪、体像、性满足分量表的得分和总分在两组间均具有显著性差异。各分量表症状鉴别效度检验男性优于女性。如性欲量表,男性F=16.33,p<0.001;性经验量表,男性F=8.44,p<0.005;女性各为F=0.09和0.073,差异无统计意义。性满意指数男性为F=62.41,p<0.001;女性F=64.47,p<0.001。

在国外,DSFI被广泛应用于临床,编制新量表时,也常被研究者当做“金标准”使用。Moyer(1979)曾用该量表比较87例性功能障碍患者和200位同龄正常人。发现患者在性知识量表、性经验量表、性欲量表得分低,性态度趋向保守,心理症状量表和情绪量表得分明显升高。正常男人的男性化得分高于男患者,而正常女性的女性化和男性化得分均高于女患者。正常男性的性想象得分高于男患者,而女性相反。女患者组在性满意指数得分上也有统计学差异。DSFI是性功能障碍研究的有效工具之一。

3.Golombok和Rust编制的《性满意量表》(GRISS) 该量表是现今在欧洲大陆流行的主要的性功能障碍评定量表。此量表优点是:简明扼要又涉及临床所见的各主要性功能障碍症状,为临床应用提供了便利。

GRISS由男女独立的两个量表组成,每一量表共24项提问,可分别构成7个量表。男、女量表中,4个分量表是相似的。如缺乏性敏感量表(SM)、性回避量表(VM)、性生活不满意量表(DIS)和缺乏性交流量表(CO)。另两个分量表各不相同,如男性阳痿量表(IMP)、男性早泄量表(PE)、女性阴道痉挛量表(VAG)和女性性高潮缺乏量表(ANO)。最后一项是总分。

该量表信度分半法:男表r=0.94,女表r=0.87。内部一致性检验各分量表相关在0.69~0.83间,重测检验0.47~0.82间。效度检验(效标效度):症状严重程度评定相关:男表r=0.53(p<0.001),女表r=0.56(p<0.001)。治疗后疗效评定相关:男表r=0.54(p<0.005),女表r=0.43(p<0.005)。

国内学者李学谦在临床上曾用该量表对13位心因性勃起障碍患者进行疗效评定,结果表明该量表能适用于中国文化,并有效的反映出性治疗效果。

4.考虑到国内至今未有专门的评估勃起障碍的量表,仅有的一份评定性功能状况的综合量表也忽视了年龄对性功能的影响及诊断量表的敏感度和特异度等指标。而国外性学理论发展活跃,根据各自理论设计的问卷未必能反映我国临床的需要。因此,为获得一个较适于我国文化背景的评定勃起障碍症状的性功能量表,既可对就诊者的性功能作鉴别,又可作定量分析。刘明矾、马晓年在1998年编制了一个勃起障碍评定量表(Erectile Dysfunction Rating Scale,EDRS),初步问卷取自《实用男科杂志》上的一份讨论稿,此讨论稿为马晓年等性医学专家多年临床经验并经多位专家反复讨论及修订而成。作者在着重了解调查有关评定勃起障碍患者临床症状特点的基础上,对90个条目进行测试,并根据临床需要,取消讨论稿中涉及较多的各种病因检查项目,并修改意义不清的用语,初步确定了由22题组成的“勃起障碍评定量表”。由于电脑作答方式与纸笔方式不同,存在选项“即选即消”的特点,降低了外来因素的影响,而打消了被试作答时害怕别人窥视的心理。作者在Foxprow软件下把EDRS设计成“阳痿初步诊断系统”。EDRS所有题项为最近四周内阴茎勃起方面的问题,根据症状程度和对性满意的影响程度分为5个等级,从毫无影响到影响严重分别记1、2、3、4、5分。EDRS中的22题均与测定勃起障碍密切相关(见表24-3)。研究表明,EDRS具有良好的信度及效度。条目内部一致性:Alpha=0.96(P<0.01);分半相关:r=0.973;量表总分重测相关:r=0.857。因子分析是验证量表结构效度的主要方法,EDRS所获性唤起、性活动、性满意3个因子与DSM-Ⅳ中ED的诊断标准存在着对应关系,能较满意地解释勃起障碍患者的基本临床特征,与临床客观现象相吻合。在以症状为主的本量表中以年龄作为影响变量确定划界值,以临床诊断作校标,对ED患者与正常被试者以分界值54(20~49岁组)及59(≥50岁组)作诊断划界分,临床诊断试验结果表明总灵敏度为95.89\%,总特异度为95.49\%,总准确度为95.70\%,实证效度较好。

EDRS为一现时症状及性关系满意程度的评定量表。项目设计基于国际公认的分类诊断标准和满足我国临床实际需要的小型量表,有利于推广使用。EDRS的题目定式化,在测试答题之前,对性交、性活动、性刺激、性幻想等性学概念进行统一科学定义,使被试者不致因误解题意而影响作答,采用5级评分方法,易被评定者掌握。评定用时少,大多数被试在15分钟之内即可完成。从初步结果来看,EDRS能较好地反映勃起障碍患者临床症状的基本特点,有较强的鉴别效用和定量作用,能满足临床及科研的需要。

表24-3 勃起障碍评定量表(EDRS)

续表

续表

续表

这类量表均为专项量表,可根据不同的临床研究需要灵活选用。按这些量表的用途,大约可分为4种:①用以评价和性有关的单项能力和特质的量表。②用以评价婚姻关系的量表。③用以评价情绪的量表。④用以评价人格特征的量表。

它们重点研究性功能中的某一功能,如性欲,或反映性行为的某种社会文化特征,如性态度、性角色和控制源等。

(1)《性态度问卷》(the sexual attitudes questionnaire,SAQ):Goldfried Friednan(1982)认为,非表达性和高工具性是这个社会中男性的特征。它们非常明显的出现在男性与女性的交往中,男性经常寻找非情感投入,视亲昵、爱抚与依附为非男性气质。Jourard报告,这种非表达性消耗能量,增加压力,并使当事人缺少来自对方身体暗示。Backer(1987)为评定西方人性态度而设计了SAQ。该量表短小简明,其核心提问仅9个项目(见表24-4)。

表24-4 性态度量表(SAQ)

《性态度问卷》为4级评分,完全不同意0分,大部分不同意1分,大部分同意2分,完全同意3分。第5、7、8为反向记分,目的在于避免被试者受量表项目的暗示。总分得分范围在0~27,得分越高,性观念问题越多。该量表的重测信度为Kappa值为0.7。比较研究表明(控制年龄、文化背景、婚姻关系和宗教信仰等方面的变量),正常组(n=24)在该量表的总分为5.1±3.0,性功能障碍患者组(n=32)总分为10.6±3.7,具有显著统计差异。

(2)两人性生活控制源(the dyadic sex regulation scale,DSR)量表发:评价有关与伴侣之间性生活的控制观念,Catania Mc Dermott &Wood(1984)通过对异性和同性伙伴所进行的有关性生活态度的开放式访谈调查编制了DSR量表。该量表共有11个测题,由受测者自行填答,采用的是Likert式7点评尺(1代表非常不同意,7代表非常同意)。量表中有5题(第2、5、6、8、10题)为平衡起见,而以反向表述。在对这些测题进行反向掉转之后,就可以相加为总分,分数越高表示内控程度越强。本来并没有采取十分系统的测题筛选办法,但每个测题在概念上都经过精心的考虑,务求与控制源有关。量表分数分布范围为11(外控)至77(内控)。DSR的信度资料和表面效度都不错。Cronbacha值为0.83。就DSR测题进行主成分因素分析发现,除第一因素外,测题在其他因素上均没有显著负荷。第一因素的变异解释率为95\%。间隔两周的再测信度为0.77。效度主要为聚敛效度,DSR与Nowicki-Strickland成人内/外控量表(ANSIE,Nowicki&Duke,1974a)有显著相关(r为0.19,p<0.05,115人)。DSR与两人互报的性生活状况有相关,性生活的内控性与较高频率的性交、伴侣的口交、与伴侣一起达到性高潮、性关系、热情表达、及性满意感有关。它也与性生活中有较少的焦虑有关。相反,ANSIE与这些相关均较弱(表24-5)。

表24-5 两人性生活控制源量表

注:*为反向记分题

(3)Bem性别角色调查量表(Bem sexbol inventory,BSRI;Bem,1974):Bem性别角色调查量表(BSRI)根据受测者自述其是否具有社会认为是好的男性刻板和女性刻板性格特征,来分别评价其男性气质和女性气质。在作答BSRI时,受测者要在60个性格特质项目上表示自己是否具有此项目所述的特质。全部测验时间为15分钟左右。20题男性气质和20题女性气质涵盖了性格特点的广阔领域。两个量表在实证上被证明是相互独立的,而且每个分量表的因素不只一个,这与Bem认为性别角色本身就具有异质性的观点一致。BSRI有良好的信度,它的内在一致性信度:男性气质量表的a为0.86;女性气质量表a值为0.80;社会期许量表a值为0.75。它被广泛应用于各类问题的研究,施测来自各年龄、各个国家和文化的受测者,而且BSRI的效度也相当好。当被用于与编制量表的理论相似的研究中时,更是如此。BSRI的最常用的计分方法是用中位数分类法将受测者分为男性特质(男性化分数高于中位数,女性化分数低于中位数)、女性特质(男性化分数低于中位数,女性化分数高于中位数)、双性特质(男女性化分数都高于中位数)、未分化者(男女性化分数都低于中位数)四种性别类型。BSRI是性别角色研究中最常用的测量工具,也是其他测量工具精心比较的效标。

Bem认为兼具有男性与女性特质者的性别角色行为较具弹性,视情景的需要而表现男性或女性角色行为,因此较易获得满意的人际动力、人际吸引、感情表达与爱情关系。Walfish and Myetson(1980)的研究表明,男女双性化同其他类型的性别取向相比,更少有性行为的定型观念,双性化女性在性生活中比女性化气质女性更为舒适,而双性化男性在性生活中也比男性化气质的男性更为舒适。作者曾用此表对30例心因性勃起障碍进行测评,发现有近一半患者呈现女性化性别类型,并与44例对照组比较,卡方检验显著性差异(x2 =11.33,df=3,p<0.01)。然而,同国外一些研究相似,作者在临床上并未发现双性化的男性比传统定义的具有刻板观念的男性在性生活上更为满意,因此,我们对双性化这种性别类型仍不能期望过高,它可能并不是我们社会中所有性问题的灵丹妙药,也许性仍然是一个较为脆弱及敏感的领域,在这个领域中,放弃传统性别角色行为,接受新的性别角色行为类型的回报并不是显而易见的。

婚姻关系与性关系密切相关,马斯特斯与约翰逊估计在美国至少有半数的婚姻伴侣中存在着某种形式的性功能障碍。他们强调,性功能是婚姻关系的一部分,因此必须在婚姻关系的背景上来考察某一方的性功能。性功能障碍可能是婚姻关系紧张或破裂的原因,也可能是结果。把婚姻关系作为一个整体来处理,才可能在性治疗中取得疗效和巩固疗效。因此,评价婚姻关系一直是性学研究中的一项任务。

(1)Olson婚姻质量问卷(ENRICH):婚姻幸福与否受多维因素影响,它主要源于3个方面。①个体因素:包括文化教育背景、价值观、对婚姻的期望、在婚姻中承担的任务、个性等;②婚际因素:包括夫妻间权利与角色的分配、夫妻间的交流、夫妻间解决冲突的方式与能力、性生活等;③外界因素:包括经济状况、与子女、父母的关系,与亲人的关系等。因此,研究者们都试图编制一个从多维角度能准确判别某婚姻是否幸福,能测出婚姻不幸福的症结,能了解婚姻幸福原因的测评工具。据此,美国明尼苏达大学Olson等教授于1981年将已有较好信效度的“婚前预测问卷(PREPARE)”作为基础,编制了本问卷。目前,它主要用于婚姻咨询工作中,用以判断婚姻的满意程度,识别婚姻的冲突所在,以便有针对性地展开婚姻治疗和效果观察。

ENRICH共包含124个条目,内容包括过分理想化、婚姻满意度、性格相容性、夫妻交流、解决冲突的方式、经济安排、业余活动、性生活、子女和婚姻、与亲友的关系、角色平等、性及信仰一致性共12个因子,每一个条目均采用5级评分制,分为确实是这样、可能是这样、不同意也不反对、可能不是这样、确实不是这样。该问卷条目内部一致性平均相关系数为0.74。重测信度为0.87(样本1344名)。其判别婚姻满意不满意的准确性的判别效度为:85\%~90\%(样本数7621名受试者)。与其他婚姻问卷相比,它的信、效度检验具有样本量大、控制了背景因素干扰,多维度,夫妇双方评估等特点。

ENRICH的统计指标主要为总分和因子分。总分高揭示婚姻质量好。因子分着重反映受试者的婚姻某一方面的情况。各因子分别为:①过分理想化,②婚姻满意度,③性格相容性,④夫妻交流,⑤解决冲突的方式,⑥经济安排,⑦业余活动,⑧性生活,⑨子女和婚姻,⑩与亲友的关系, 角色平等性, 信仰一致性。

(2)Locke-Wallace婚姻调适测定:婚姻调适是指夫妻间在一定的时间内的相互适应。1959年,Locke和Wallace收集了所有测定婚姻调适的量表,共383个条目。从中筛选出了15个条目满足下列条件:①在原来的研究中有最好的判别水平,②不与所收集的其他条目重复,③研究者认为其能反映婚姻调适的主要方面。这15个条目便构成了Locke-Wallace婚姻调适测量。该问卷的目的是客观地、定量地对夫妻的婚姻调适进行评估,对所有已婚者的婚姻调适均可用该问卷予以评定。

该量表的信度系数为0.90(由split-half法计算,由Spearman-Brown formula校正);判别效度(critical ratio)为17.5,具有显著性差异。国内有人将该问卷翻译后在25名已婚个体中进行重测(间隔10天)相关系数为0.59,提示信度较好。将量表应用于118对婚龄在1~10年、文化程度均在大专以上的夫妻,其中自觉对婚姻满意者(191人)的问卷分为113.5分,自觉对婚姻不满意者(45人)的问卷分为83.5分,t检验两组有显著性(p<0.01),提示该问卷有效。

情绪因素既是一应激源,又可影响个体对应激刺激的认知评价和应付反映的程度,它与心因性性功能障碍的关系较为密切。

焦虑是性功能障碍患者中最常见的情绪问题。它可能成为性功能障碍的起因,也可能继发于心功能障碍而成为慢性化的因素。多数专家都赞成心因性功能障碍的主要中间变量是性行为焦虑,性行为焦虑作为一种状态焦虑,是人类有机体的一种暂时的情绪状态。其特点表现为对烦扰和紧张的、主观的、有意识的情感,并且也唤醒自主神经系统的活动性。

MartinB(1973)总结了焦虑伴随的心身反应,他把这些反应归结为:①自我感觉方面的:担忧,迫近的、即将来临的危险感受;紧张,不能集中注意力,即刻瓦解的感觉;逃避和摆脱现状的强烈愿望;②行为方面的:逃脱行为,回避行为,言语异常,动作协调异常,解决问题的复杂过程异常;③心理生理的:肌紧张或震颤,心率和呼吸变快或不规律,血压增加,手脚出汗,胃肠功能异常,腹泻,小便过多等。当性功能障碍患者处于性行为焦虑时,他不能表现出对自身性感受的专注,并且在必要时及时地改变这种定向的能力。他们迫切希望性唤起趋向不要受到外部的刺激的影响。然而,满足配偶的要求以及配偶期望的眼神,这些偶尔出现的无关刺激周期性地干扰患者的新感受,而外界信息的输入与加工要求有鉴别、有净化才能做到最为有效。这些额外的干扰阻碍了患者集中注意于自身的性感受,从而造成性唤起缓慢或丧失,由此引起患者性唤起能力异常。

(1)焦虑状态/特质焦虑问卷(STAL):特质焦虑(trait anxiety)的概念。前者描述一种不愉快的情绪体验,如紧张、恐惧、忧虑和神经质,伴有自主神经系统的功能亢进,一般为短暂性的。特质焦虑则用来描述相对稳定的,作为一种人格特质且具有个体差异的焦虑倾向。STAI由Charles Spielberger于1977年编制,并于1983年修订,为自我评价问卷,其特点是简便,并能相当直观地反映焦虑患者的主观感受,而且能将当前状态和一贯特性的焦虑症状区分开来,优于其他焦虑量表。STAI共40个项目。所有项目采用1~4分的四级评分法,前20个项目是询问焦虑状态的,各级标准为:1.完全没有,2.有些,3.中等程度,4.非常明显。后20个项目是询问焦虑特性的,各级的标准为:1.几乎从来没有,2.有时有,3.经常有,4.几乎总是如此。而1、2、5、8、10、11、15、16、19、20、21、23、24、26、27、30、33、34、36、39等20个项目为反向记分。即按上述顺序依次评为4、3、2、1分。原作者对该量表进行了测试再测试的信度检验,发现焦虑状态问卷的稳定性较低,相关系数为0.16~0.62。特质问卷的稳定性较高,二次评分相关系数为0.73~0.86。同时进行了效度检验,该量表的一致性(concurrent)、会聚性(convergent)、区分性(divergent)和结构性(construct)比较满意,国内北医大精神卫生研究所与长春第一汽车公司职工医院精神科合作在长春地区和北京分别对正常人和抑郁症患者进行了STAI中译版的测试。获得与原作者近似的结果:正常人群总样本状态焦虑问卷评分为39.71±8.89(男375例),38.97±8.45(女,443例);特质焦虑问卷评分为41.11±7.74(男),41.31±7.54(女)。抑郁症组(50例):状态焦虑问卷为57.22±10.48,特质焦虑问卷为46.22±26.22,明显高于正常人群。因子分析均可得出焦虑因子和非焦虑因子。笔者曾用状态焦虑问卷对30例心因性阳痿患者组进行性行为焦虑测评(31.27±7.95),并与44例正常对照组进行比较(55.5±10.3),两组具有极显著性差异(t=-11.383,p=0.001),心因性阳痿患者组得分明显高于正常人群。

(2)焦虑自评量表:焦虑自评量表(selfrating anxiety scale,SAS)由Zung于1971年编制,它是一个含有20个项目,用于评价记录患者的主观感受的自评量表,SAS采用4级评分,主要评定项目所定义的症状出现频度,其标准为:①表示没有或很少时间有,②是小部分时间有,③是相当多时间有,④是绝大部分时间或全部时间都有。注意评分时有5个项目是反向记分(5、9、13、17、19)。SAS的主要统计指标是总分。由自评者评定结束后,将20个项目的各个得分相加,即得粗分(raw score)经过下式换算,y=int(1.25x),即用粗分乘以1.25以后取整数部分,就得到标准分(index score,Y)。

国内有作者把该量表用于性功能障碍患者(男,32例;女,46例)的疗效评定,治疗前SAS得分为59.89±13.76,治疗后SAS得分为45.21±13.47,差异显著,p<0.01。表明该量表有效的反映治疗效果。

(3)汉密顿抑郁量表(HAMD):汉密顿抑郁量表(Hamilton depression scale,HAMD)由Hamilton于1960年编制,是临床上评定抑郁状态时应用的最为普遍的量表。HAMD大部分项目采用0~4的5级评分法。各级的标准为: 无,①轻度,②中度,③重度,④极重度。少数项目采用0~2级评分法,其分级的标准为: 无,①轻至中度,②重度。

(4)《明尼苏达多相人格调查表》(MMPI):MMPI含556个自陈形式的项目,其中16个重复出现的题目分散在不同的位置上,主要用于检验被试者反应态度的一致性如何。因此,MMPI实际上只有550个不同的题目。这些题目的内容涉及面很广。包括身体生理状况、精神状况以及家庭、婚姻、宗教、政治、法律社会等问题的态度。MMPI分成4个效度控制量表,10个临床用量表及其他研究用量表。他们分别是掩饰量表(L)、诈病量表(F)、自我防御量表(K)、疑病(Hs)、抑郁(D)、癔症(Hg)、病态人格(Pd)、性别化(Mf)、妄想(Pa)、精神衰弱(Pt)、精神分裂(Sc)、轻躁狂(Mf)和社会内向(Si)。这些量表设计时侧重于病理心理内容,但在实际经验主持者眼里,MMPI剖面图能反映众多非病理性的人格特征。由于MMPI测验时耗时较多,分析又较复杂,除精神科用得较多外,性门诊中大多尚未开展此项测评工作。国外有过少量的研究。如Beufler(1976)认为Mf和Si两量表在诊断心因性功能障碍上有特殊意义,其标准分大于60就提示有心因性阳痿的可能,检出率达80\%~90\%。从理论上讲,心因性阳痿的患者Mf分与Si分应比正常人要高,即具有和异性相同而与同性相异倾向,并且不善于社交活动与言谈,内心闭锁,行为孤僻、寡言少语的人。然而,国内有作者对此进行过研究并未发现确诊为心因性阳痿患者组的Mf分和Si分与对照组有何差异。

在引进量表时,应掌握足够的信息,找出同类量表中品质最优、最适合我们实际需要者。而且,最好是在国际上较有影响,应用的较广泛的那一种。量表不一定是最新者最好。新编量表的作者总对他的新产品情有独钟,总是说他的量表与其他量表相比有哪些优越性。然而,只有经得起实践考验,愈来愈多的人使用的量表才具有生命力。许多量表自编制以来,并未得到广泛的验证和认可,除作者本人使用外,很少有人问津。因此,在量表引进时不必赶时髦。

量表的版权在我国是新近才遇到的问题。大多数量表不涉及版权,作者将量表全文公布,希望他人应用。但有些量表则有版权,如DSFI(性功能问卷),其翻译及应用都有限制。使用者不了解这一点,草率从事,很可能卷入版权纠纷。量表翻译之前,最好是向量表作者发封信,征求意见,取得同意。

一般在书刊上发表的量表内容或项目,或者是附有一张简单的记分单。单纯将之译出是不够的,对于如何评分,有何注意事项、如何解释结果、有哪些统计指标、量表信、效度如何,样本的大小及源至何处、结果解释是否适合于使用者的结果都应作全面的了解。

应将量表及其工作手册全文译出,应尽量译出原意。有时量表的条文用词简练,又无上下文;有些词或短语是多义的,即使借助于词典也难以抉择。此时,应向精通外语的同行或专家请教。量表是一种测量工具,在临床、科研上具有一定的应用价值,翻译一定要严谨,以免以讹传讹。

要检验中译本是否与原文本相符。常用的经验方法有两种:一是回译法,即将量表的中译本,请精通该国文字者或以该国文字为母语的外国人,重新译成原文。译者应该是不知道量表原文者。比较量表原文及回译的文本,如果发现两文本中有些条目有出入,则应重译,重译部分再进行回译,直到回译文与原本意义相同为止。回译校对法,在跨国的合作研究中是必不可少的步骤。另一检验法是找几名精通两国语言者对检查对象,用原文本和中译本各检查一次,比较两次量表检查的结果,相当于量表信度检验中的常用的重测法,只是在译本检验时,一次用外文版,一次用中文版。

译成的并经过检验的中文版本,要推广应用之前,还需在小样本中进行预测。一方面通过预测了解量表中译本实际应用中的问题,并从检查者和被检查者那里征集意见。另一方面,通过预初研究,取得中译本的信度和效度的初步资料,特别是信度资料。有时,在预初实验以后,还需对译本的文字进行一定的修改。

即将量表的中译本在一定数量的目标人群中应用,以期得到有关量表品质和性能的全部有关指标。量表原文本的品质和性能指标,只适用于原文在该地区应用的情况,中译本则应建立自己的参数,真正反映它在我国应用的结果。

目前,国内外应用的量表并不尽善尽美,国内有学者曾参考诸多国外量表在临床小范围试用,感到它们仍存在某些不足之处,例如,性行为、性观念甚至问卷的提问设计,存在明显的西方文化色彩。一些综合量表内容庞杂,包括过多了解心因性病因的项目。国外性学理论发展活跃,根据各自理论设计的问卷和因子分并不能充分反映我国临床的需要。由于知识产权法的实施,量表的引进有了许多新的限制,这便涉及自己量表的编制。

性学量表的理论和方法,均借鉴了心理测试,因此,新量表的编制也同心理测验的编制相仿,以下简述其基本原则。

应视量表编制为一项研究课题,研究目的必须明确。新编量表的目标应有针对性,易小不宜大。

立题者要在开始工作之前针对课题进行详细的文献检索,了解国内外有关现状。既可避免“撞车”,又可吸取前人的经验和教训。

量表的编制不可能一蹴而就,在成为正式文本之前,应先有草案的形成并经过反复讨论、实践和修改。草案的形成有两种方式,一是专家讨论,组织一批专家,根据他们的知识和经验编制初稿;另一是借鉴法,利用若干现有量表重新确定设计原则,进行选择或组合;或者是两者结合。

量表应先在小样本中进行预初研究,仔细地分析具体项目及全文本的适用性,听取调查员和被试者两方面的反馈信息。通常最初的草案项目多一些,预初测试后,根据各项的适用率、阳性预测值等指标和综合分析,去粗存精,综合平衡。然后组成新的草案,再作测试。

经过多次的测试和修改,形成正式文本。正式文本应包括量表全文、评分指南、应用手册和记录单等有关文件。

新量表正式发表前,必须在较大的样本中进行考核。对新编量表的性能和品质考核,比对引进量表的要求高,除在量表的品质中所提到的信度、效度外,还需进行项目分析,包括项目的适合率,项目的通过率,项目的内部一致性检验,检查组成量表的各项目是否同源。有些量表目的是为了判别正常或异常,或者是判断疾病的轻、中、重,则需要确定量表的分界值。分界值的确定有多种方法,可以以正常人的均值加上或减2个、1.5个或1个标准差,也可以取正常人的第90个或95个百分位数。还有的则是根据敏感性和特异性确定。在决定分界值时,还应考虑量表的用途。筛查量表要求高敏感度,以期尽量减少遗漏可能的病例。诊断量表则要求高特异性,以免将正常误判为有病。

(刘明矾)


\section{第三节 女性性问题量表汇总}

这是一个国际通用的问卷,在药物临床观察中广泛使用,相当于男性国际勃起功能IIEF问卷。

患者编号:\_\_\_\_\_\_\_\_\_\_\_\_\_\_\_\_日期:\_\_\_\_\_\_\_\_\_\_\_\_\_\_\_\_年\_\_\_\_\_\_\_\_\_\_月\_\_\_\_\_\_\_\_\_\_日

提示:以下所列的19道问题会问及您在过去4个星期(即过去4周),对女性性唤起障碍(FSAD)的感觉及反应。请尽可能诚实、清楚地回答。在回答这些问题时,将涉及以下名词解释:

性活动:包括爱抚、性交前的一些性嬉戏、手淫和阴道性交。

性交:指性伴侣的阴茎插入阴道的行为。

性刺激:包括与性伴侣进行的性交前嬉戏、自我刺激(手淫)或性幻想。

性欲或性兴趣:是一种感觉,包括对性活动的需要,想接受性伴侣对性活动的引发,想或幻想有关性的事情。

每一个问题请选择一个答案,在您认为合适的答案前划“√”。

1.过去4周内,有多少次您感觉到有性欲或性兴趣?

□5=几乎总是或总是

□4=多数时候(超过一半时候)

□3=有时(约一半时候)

□2=少数时候(少于一半时候)

□1=几乎没有或完全没有

2.过去4周内,您的性欲或性兴趣的程度如何?

□5=很高

□4=高

□3=一般或中等

□2=低

□1=很低或根本没有

性唤起:一种感觉,包括身体和精神上的性兴奋。如生殖器的暖感、麻刺感、润滑或湿润、肌肉收缩

3.过去4周内,在您进行性活动或性交时,有多少次感到有性唤起或性兴奋?

□0=无性活动

□5=几乎总是或总是

□4=多数时候(超过一半时候)

□3=有时(约一半时候)

□2=少数时候(少于一半时候)

□1=几乎没有或完全没有

4.过去4周内,在您进行性活动或性交时,您性唤起(或性兴奋)的程度如何?

□0=无性活动

□5=很高

□4=高

□3=一般或中等

□2=低

□1=很低或根本没有

5.过去4周内,在您进行性活动或性交时,您对您引发性唤起的信心如何?

□0=无性活动

□5=很高

□4=高

□3=一般或中等

□2=低

□1=很低或没有信心

6.过去4周内,在您进行性活动或性交时,有多少次对您的性唤起(或性兴奋)感到满意?

□0=无性活动

□5=几乎总是或总是

□4=多数时候(超过一半时候)

□3=有时(约一半时候)

□2=少数时候(少于一半时候)

□1=几乎没有或完全没有

7.过去4周内,在您进行性活动或性交时,有多少次有阴道润滑(或湿润)?

□0=无性活动

□5=几乎总是或总是有

□4=多数时候(超过一半时候)

□3=有时(约一半时候)

□2=少数时候(少于一半时候)

□1=几乎没有或完全没有

8.过去4周内,在您进行性活动或性交时,您要达到阴道润滑(或湿润)有多大困难?

□0=无性活动

□1=极困难或不可能达到

□2=很困难

□3=困难

□4=不太困难

□5=没有困难

9.过去4周内,有多少次您能保持阴道润滑(或湿润)直到性活动或性交结束?

□0=无性活动

□5=几乎总是或总是

□4=多数时候(超过一半时候)

□3=有时(约一半时候)

□2=少数时候(少于一半时候)

□1=几乎没有或完全没有

10.过去4周内,您要保持阴道润滑(或湿润)直到性活动或性交结束有多大困难?

□0=无性活动

□1=极困难或不可能保持

□2=很困难

□3=困难

□4=不太困难

□5=没有困难

11.过去4周内,当您有性刺激或性交时,有多少次达到性高潮?

□0=无性活动

□5=几乎总是或总是

□4=多数时候(超过一半时候)

□3=有时(约一半时候)

□2=少数时候(少于一半时候)

□1=几乎没有或完全没有

12.过去4周内,当您有性刺激或性交时,您达到性高潮有多大困难?

□0=无性活动

□1=极困难或不可能达到

□2=很困难

□3=困难

□4=不太困难

□5=没有困难

13.过去4周内,在进行性活动或性交时,您对您达到性高潮的能力满意吗?

□0=无性活动

□1=很不满意

□2=较不满意

□3=一般或中等

□4=较满意

□5=很满意

14.过去4周内,与伴侣进行性活动时,您对您俩情感上的亲密程度满意吗?

□0=无性活动

□1=很不满意

□2=较不满意

□3=一般或中等

□4=较满意

□5=很满意

15.过去4周内,您对您与伴侣间的性关系满意吗?

□1=很不满意

□2=较不满意

□3=一般或中等

□4=较满意

□5=很满意

16.过去4周内,总的来说,您对您的性生活满意吗?

□1=很不满意

□2=较不满意

□3=一般或中等

□4=较满意

□5=很满意

17.过去4周内,在阴茎插入时,有多少次您感到有阴道不适或疼痛?

□0=没有尝试性交

□1=几乎总是或总是

□2=多数时候(超过一半时候)

□3=有时(约一半时候)

□4=少数时候(少于一半时候)

□5=几乎没有或完全没有

18.过去4周内,在阴茎插入后,有多少次您感到有阴道不适或疼痛?

□0=没有尝试性交

□1=几乎总是或总是

□2=多数时候(超过一半时候)

□3=有时(约一半时候)

□4=少数时候(少于一半时候)

□5=几乎没有或完全没有

19.过去4周内,在阴茎插入时或插入后,您感到阴道不适或疼痛的程度如何?

□0=没有尝试性交

□1=很高

□2=高

□3=一般或中等

□4=低

□5=很低或没有不适

女性性功能指数(FSFI)是一个用于女性性功能障碍的临床测试和流行病学研究的共有19个题目的性功能自陈量表。FSFI设计为多维度问卷量表,设置分量表以评估女性性功能主要组成部分,包括性欲、唤起、高潮、疼痛和满意。问卷的开发包括定性和定量研究,个别题目基于与患有或不患有性功能障碍的女性的定性面谈。各自的效度研究是使用独立妇女样本的一些作者报告的。FSFI已翻译为多种语言并已广泛应用于各种临床或非临床机构对女性性功能的评估。FSFI是一个很有特点的工具,它的心理测量的特性早前已做过介绍。

基于早期的效度研究,威杰尔等提出整体FSFI等级分数的割点(26.5),以划分女性患有或不患有性功能障碍。尽管采用整体得分割点来区分患有或不患有性功能障碍,它不能提供一个特定领域割点来评估女性是否存在性欲减退。性欲(SD)是男性和女性复杂的、由性反应的多方面组成,需要通过一个已经得到验证的自陈量表进行独立评估。

在考虑需要特定领域割点时,有两个重要原因。首先,可以对当前性不活跃女性的性欲减退作出评估,而其他领域和整体性功能的得分,是根据无论有无伴侣的性活跃程度来断定的。第二,因为性欲低下障碍(HSDD)是妇女中诊断得最频繁的性问题,常常与女性的其他性障碍同时存在,一个已得到验证的SD领域的诊断割点对于将来的临床和探讨研究,尤其是在评估与性欲相关的治疗效果时,具有宝贵价值。

结果表明使用诊断割点将SD领域的SD分数定为5分,等于或低于5分的女性视为患有HSDD,那些SD分数为6分或高于6分的女性视为不患有HSDD。

这是一个国际通用的问卷,也在药物临床观察中广泛使用。

你常感到:

1.对性生活感到痛苦 Distressed about your sex life

2.对性生活焦虑不安 Anxious about your sexuality

3.对性关系不满意 Unhappy about your sexual relationship

4.因性生活有困难而产生负疚感 Guilty about sexual difficulties

5.自己是不合适的性伴侣 That you are a poor sexual partner

6.因性生活问题产生挫折感 Frustrated by your sexual problems

7.对性行为有压力和紧张 Stressed about sex

8.自己已失去了性的吸引力 No longer sexually attractive

9.因性问题而觉得比不上别人(自卑?) Inferior because of sexual problems

10.对进行性行为有所担心 Worried about sex

11.在性生活上没有达到要求 Sexually inadequate性功能欠缺

12.对性生活感到遗憾 Regrets about your sexuality行为懊悔

13.在性生活上并未满足 Sexually unfulfilled

14.对性问题觉得尴尬 Embarrassed about sexual problems

15.对性生活不满 Dissatisfied with your sex life对性生活不满意

16.对性生活感到愤怒 Angry about your sex life

17.对性行为感到困惑或烦恼 Confused about sex对性感到困惑

18.对性生活失望 Disappointed about sex

19.陷入不幸的性关系 Trapped in a poor sexual relationship

20.因性问题而羞愧 Humiliated because of sex因性问题而受到羞辱

1.你和爱人多久做爱一次?

2.谁先有做爱动机?

3.你曾有过想做爱的心情,但却不愿表达出来吗?

4.如果是这样,为什么呢?

5.你曾有过哪些性幻想的内容呢?

6.你觉得配偶的身体对你有吸引力吗?

7.你是否想到过和别人而不是爱人做爱呢?

8.当很长一段时间没有做爱后,你会不会觉得有挫折感?是自己主动求治还是伴侣要求的?

9.当看到一些有直观性描写的电影或录像后,你有何反应呢?你常有性梦吗?

10.做爱会不会让你产生任何消极反应或情绪,如不安或恶心?

11.性欲是否突然降低?是否伴有其他性功能障碍或其他原因?是否伴侣提出主诉?

12.性欲低下是全面的还是境遇性的?是否包括所有性活动形式?

13.性欲低下是原发的还是继发的?

14.有无疾病、药物、情感方面的原因?

15.你过去是否手淫,手淫时有何想法,手淫能否给你带来高潮?

如果想作出简单的性欲水平是否低下的评价,也可以从性内疚水平、先前的性经历情况、对手淫的情感反应三方面作出判断。

C.Meston和P.Trapnell(2005)根据他们对886名妇女的调查和测试提出女性的性满意度量表(SSS-W),希望能为女性性功能障碍的临床诊断和治疗提供一个有力的工具。

据他们分析,影响女性性满意度的原因有很多:

1.社会因素如年龄、婚姻状态、收入水平;

2.人格/情感因素如自尊、性内疚、自私、易激惹、愤怒;

3.背景变量如躯体疾病、成长家庭内的性积极态度、性教育。

女性的性满意与对双方整体关系的满意度是密切相关的,如爱情、承诺及关系的稳定性等均是双方整体关系好坏的指征,而离异的可能性则与之呈负相关。

人们在文献中经常讨论到的、与性满意最为相关的因素就是伴侣间的交流。良好的性交流是性满意的婚姻变量中的最好的指标。如果一个人能够在向询问者报告时讲述出双方关系中更多的、非性方面的和性方面的细节的话,他或她将拥有更为满意的性生活。伴侣间的有效交流可以通过促进亲密无间和加强亲昵关系、通过伴侣们在性欲和性表现方面的信息而增强性的满意度,反过来也能增强性唤起和高潮的能力。事实上,不能就性欲问题展开交流是性高潮问题的一个常见因素,具有性自信的妇女报告更高的性欲水平、高潮能力和性满意度。

人们对妇女性满意的方式和程度是如何与性功能和性功能障碍联系在一起的尚缺乏充分的了解,问题在于对性功能障碍和/或性满意还缺乏综合的、有效的评价系统。近年来为了建立和发展新的女性性功能障碍治疗方法,人们正极力推动这方面的研究,而且对二者之间的关系已有了更好的了解。虽然治疗有效性最终目标的临床标准如性念头和行为频率的增加、生殖器和主观唤起的增强、高潮频率增加和强度的增强等无可置疑地将是很重要的考虑,但是如果没有相伴随的、整体性满意度的临床意义上的改进的话,那些改变的临床相关性将是有疑问的。能整合到大多数女性性功能障碍中的、能广泛为人们所接受的问题是个人的痛苦问题。据了解尚无研究以检查在性领域中什么将构成满意与什么将构成痛苦之间的关系。

我们认为要想理解什么构成妇女的性满意,就必须意识到其中包含有个人成分和关系成分,而二者都是必要的构成条件。

文献资料表明有两个重要的因素将影响关系中的性满意:一是坦诚和深入的性交流;二是整体的、可以测量的、双方的相容适配性。谈到相容适配性包括性欲、性信念、性价值观、性态度以及双方在行为等方面的相似性。

影响个人性满意的成分也将包括两个方面:一是对整体性满意的综合判断,如大多数人当前使用的女性性功能指数量表(FSFI)或女性性功能简表;二是主观的、有关特定性自我关切的痛苦水平。

SSS-W已经证实为一个能评价和理解女性性满意度和痛苦程度的有效、可靠、综合、多维、自我报告的量表,它包括五个范围(两个反映关系因素———交流、相容性;三个反映个人因素———满意、对关系的关切、对个人的关切),每个范围有6个问题。

1.我感觉对目前的性生活是满意的。

2.我总觉得目前的性生活中缺少点儿什么。

3.我总觉得目前的性生活中没有足够的激情。

4.我感觉满意目前性生活中的亲昵频率。

5.我感觉对目前的性生活没有什么特别要关注的事。

6.我对目前的性生活究竟满意不满意呢?

7.伴侣总在抵制我对讨论性问题的要求。

8.我和伴侣彼此之间没有公开地、充分地讨论性问题,或不经常讨论性问题。

9.不论伴侣何时有要求,我都觉得讨论性问题是完全自然的事。

10.不论我何时有要求,伴侣都觉得讨论性问题是完全自然的事。

11.当伴侣要求我的时候,我会毫不困难地谈起自己最深层的感受和情绪。

12.当我要求伴侣的时候,伴侣会毫不困难地谈起自己最深层的感受和情绪。

13.我常常觉得我的伴侣对于我的性喜好和性欲不够敏感或缺乏足够的意识。

14.我常常觉得伴侣和我在性方面不是充分匹配的。

15.我常常觉得我的伴侣在性的信念和态度上与我有着太大的不同。

16.我有时想我的伴侣和我在有关性亲昵方面的需求和欲望是错配了。

17.我有时想我的伴侣和我彼此间在肉体方面不具有足够的吸引力。

18.我有时想我的伴侣和我在我们的性模式和偏好上是错配了。

19.我担心我的伴侣将会因为我的性困难而遭受挫折感。

20.我担心我的伴侣将会因为我的性困难而对我们的相互关系产生不良影响。

21.我担心我的伴侣将会因为我的性困难而拥有一份婚外恋情。

22.我担心我的伴侣在性方面得不到满足。

23.我担心我的伴侣将会因为我的性困难而认为我不够女人味。

24.我感觉因为我的性困难一直让我的伴侣失望。

25.我的性困难让我倍感挫折。

26.我的性困难让我感到性的不满足。

27.我担心我的性困难会使我在婚外寻求性的满足。

28.我对我的性困难感到特别苦恼,它影响了我的自我感受的方式。

29.我对我的性困难感到特别苦恼,它影响了我自己的幸福。

30.我的性困难让我烦躁并让我对自己耿耿于怀。

(中国性学会性医学专业委员会中华医学妇产科分会绝经学组)

1.年龄:口口岁

2.职业:(1)工人 (2)农民 (3)专业技术 (4)干部职员 (5)商业服务 (6)军人武警等 (7)其他 (8)无业

3.文化程度:(1)小学 (2)初中 (3)高中 (4)大学 (5)研究生

4.性关系状态:(1)已婚 (2)已婚,有同性伴侣 (3)离异,有同性伴侣 (4)离异,有异性伴侣 (5)离异,无性伴侣 (6)丧偶,有同性伴侣 (7)丧偶,有异性伴侣 (8)丧偶,无性伴侣 (9)未婚,有同性伴侣 (10)未婚,有异性伴侣 (11)未婚,无性伴侣

5.你的性取向是:(1)同性 (2)异性 (3)皆可

6.生育史:孕次口产次口人工流产次数口

7.既往史(可选一项以上):(1)生殖道感染口 (2)阴道不规则出血口 (3)子宫切除术口 (4)泌尿系疾患口 (5)精神疾患口 (6)高血压口 (7)心脏病口 (8)糖尿病口

8.嗜好:(1)烟 (2)酒 (3)赌博 (4)网恋 (5)健身 (6)音乐 (7)读书

1.您最近一次性生活距今有多久?(1)1周内 (2)1月内 (3)半年内 (4)一年内

2.如果以周计算,您的性生活平均次数是:(1)0次 (2)1次 (3)2次 (4)3次 (5)4次 (6)5次 (7)6次 (8)7次

引导语:和谐美满的性生活对每个人的身心健康都是十分重要的,不过,每个人或每对夫妻对此的要求却有很大的差异,它取决于个人的或双方的感情基础、性欲水平、身体条件和生活环境等因素,因此,我们很难说什么样的频率是正常的或是异常的。

3.您和您的伴侣能互相交流对性的感受并表达需求吗?(1)经常交流,畅所欲言 (2)有时交流 (3)很少交流,难以启齿 (4)从来都不交流 (5)不需要交流

4.当您具有性要求时:(1)主动告诉对方 (2)向对方暗示 (3)不做任何表示 (4)没有性要求

5.当您不愿意过性生活时,而对方要求时:(1)顺从对方的要求 (2)敷衍了事 (3)找借口拒绝 (4)坚决拒绝 (5)没有发生过,所以不知道

引导语:和谐美满的性生活在很大程度上取决于双方是否能对此进行充分的、坦率的交流。既要大胆表达自己的性需求和性感受,也要尊重、理解和支持对方的性需求。掩饰问题、回避矛盾、激烈对抗甚至把性作为夫妻间相互控制的手段,那就大错特错了。

6.长时期不过性生活时您有何种感受:(1)很难耐受 (2)可以忍受 (3)如释重负 (4)无所谓 (5)没有经历过

引导语:长期不过性生活对自己的健康不利,对婚姻关系和伴侣关系也不利,当然若双方能心态平和地共同接受这一事实(如因一方或双方患病不得不终止性生活),这也不一定意味着肯定能伤害一方或双方的整体健康。

7.您在性生活中能获得性乐趣吗?(1)经常获得 (2)有时 (3)很少 (4)从无

8.您对您的性生活质量满意吗?(1)非常满意 (2)比较满意 (3)不太满意 (4)非常不满意

引导语:缺乏感情基础或激情的、公式般的性生活显然令人乏味,长久下去肯定会对双方的关系产生负面影响。

9.您有过婚外性关系吗?(1)经常有 (2)曾经有过 (3)从来没有

引导语:婚外性关系就像玩火,看上去烧得轰轰烈烈,不过其结局却未必美好,要么烧得焦头烂额,要么就剩一堆灰,被风吹得满世界都是。

10.您开始手淫的年龄是\_\_\_\_\_\_\_\_\_\_\_岁(记不清)。

11.您上次手淫的时间距今:(1)1周内 (2)1月内 (3)半年内 (4)1年内 (5)太久,记不清了

引导语:手淫是老生常谈的话题了,可仍有不少人对此存有偏见,什么不道德啦、什么有害健康啦、什么可以导致种种疾病或弊端啦,好像这些人真的活在真空之内,六根清净。其实手淫是我们正确认识自己、认识对方、认识人类的性素质的正常的、极其普遍的、健康的性行为,是性心理成熟的标志,是性生活的事前演练,完全是无害的。

12.您在性幻想时会想到谁?(可选一项以上)(1)伴侣口 (2)情人口 (3)偶像口 (4)其他人口 (5)从来没有口

引导语:性幻想是手淫时或性生活中的重要心理活动,有益无害,能起到催化剂的积极作用。

13.你们性交前的爱抚时间足够长吗?(1)足够,总是能准备得很好 (2)还可以,没觉着有什么特别 (3)不够,经常觉着感觉不好 (4)根本就没有爱抚

引导语:事前爱抚就像大赛前的热身,尤其对女性而言更是如此,因为女性的性反应就像电熨斗,通电后总也热不起来,而男性的性反应却像电灯泡,一拉开关就亮。同样重要的还有事后爱抚,因为女性的性反应就像热了之后的电熨斗,即使断电之后还需要很久时间才能恢复到正常温度,相反,男性的性反应则说停就停,断电后灯泡马上就灭。

14.你们在性活动中刺激女性乳房吗?(1)总是 (2)经常 (3)有时 (4)很少 (5)从不

15.你们相互以手刺激生殖器吗?(1)总是 (2)经常 (3)有时 (4)很少 (5)从不

16.你们相互以口刺激生殖器吗?(1)总是 (2)经常 (3)有时 (4)很少 (5)从不

17.你们在性活动中是否经常变换体位?(1)总是 (2)经常 (3)有时 (4)很少 (5)从不

引导语:性技巧不是万能的,没有性技巧却是万万不能的。尤其对女性而言,阴道往往不是她们最重要的性敏感区,所以,非性交刺激对她们就显得更为重要。然而不少女性认识不到这一点,一味指责对方时间太短、器官太小等。由于观念的过分保守,她们也可能拒绝对方的一切努力,指责对方的举动令人厌恶或让人恶心,其实受到损失的恰恰是她们自己。

18.阴道润滑不足的现象在何种情况下出现(可以多选)?(1)月经前后 (2)生完孩子以后 (3)使用避孕套的时侯 (4)情绪不好或者压力很大的时候 (5)更年期以后 (6)不知道什么原因 (7)从来没有润滑不足的现象

引导语:阴道润滑是性生活正常进行的重要物质基础,但它会受到多种因素的影响,没关系,医生会给你有效帮助的。

19.阴茎插入阴道后平均\_\_\_\_\_\_\_\_\_\_\_分钟射精。

引导语:抽动时间的长短并不是性生活和谐美满的决定性因素,当然时间太短也会让人感到遗憾,同样可以寻求医生的具体帮助。

20.您在性活动中能出现阴道节律性收缩吗?(1)总是 (2)经常 (3)有时 (4)很少 (5)从无

21.您的性高潮出现在下列哪种性行为方式?(1)性交 (2)手淫 (3)两者都有 (4)从来没有性高潮

22.您在手淫时需多长时间达到高潮?\_\_\_\_\_\_\_\_\_\_\_分钟。

引导语:女性性高潮是重要的,其表现形式又是多样化的,这往往让女性感到困惑。关于女性性高潮的通常定义是阴道和盆腔区域的主观体验和生理变化的总和。妇女使用的主观描述包括“达到高峰”,“悬吊感”,感到肌紧张的逐渐建立并在生殖器区域逐渐增强的肌肉收缩感受,和(或)一段时间的高度兴奋及随之发生的相当突然的释放而达到完全松弛。性紧张的释放总会伴有肉体上和情绪上的变化,这就表现为性高潮。这些生理变化就像周期性因饥饿而进食一样,其心理变化就像进食后饥饿感消失并得到满足。对于性高潮来说,并不需要明确的主观体验,例如,有些妇女报告她们了经历性高潮而没有伴随肌肉节律性收缩的感觉。性高潮的另外的主观感觉是仿佛丧失了时间在消逝的感觉。她们似乎低估了高潮经历的持续时间,她们报告的只是实验室里测量到的时间的50\%。

(第23题是跳问题,同性恋跳过此题)

23.您的伴侣存在下列哪些性困难?(可选一项以上)(1)勃起不足 (2)射精过快 (3)兴趣不大 (4)无

引导语:男性的性问题往往成为女性性障碍的诱因,不过有些男子却出于过分的自尊心理,拒绝承认自己的问题,反而指责对方已经没有吸引力,所以他们才缺乏反应。甚至说尼姑一辈子是怎么过来的,认为是对方在无理取闹。这种大男子主义或虚荣心实在要不得。

24.你们在遇到性问题后寻求过哪些帮助?(可选一项以上)(1)医生 (2)书籍 (3)家人 (4)朋友 (5)对方 (6)网络 (7)其他途径 (8)从未寻求过帮助

25.您曾使用过哪些性辅助用品?(1)润滑剂 (2)振荡器 (3)其他 (4)从未用过

26.你们用过下列哪种润滑剂:(1)KY润滑剂 (2)倍柔情润滑剂 (3)人初油 (4)印度神油 (5)都没有用过

27.你们用过下列哪种药物来辅助性生活:(1)伟哥 (2)肾宝 (3)女宝 (4)其他 (5)都没有用过

引导语:遇到性问题并不可怕,怕就怕不正视它,其实绝大多数性问题都会在专科医生的帮助之下获得解决。

28.您认为性生活在婚姻中的作用:(1)非常重要 (2)重要 (3)可有可无 (4)不重要

29.您对自己的性能力:(1)充满自信 (2)一般,觉着和别人差不多 (3)有自卑心理

引导语:性是婚姻的黏合剂,无性婚姻固然可以存在,但那种婚姻势必早已变得没有任何滋味了。对自己的性能力要具有充分的信心,但这种自信心是建立在能掌握充分的科学知识基础之上的。

如果您的年龄在40岁以上,请继续回答下列问题:

30.您目前月经的情况是怎样的?(1)正常 (2)月经已不规律 (3)绝经

31.您曾经出现过下列症状吗(可选一项以上)?(1)烦躁,心慌 (2)盗汗 (3)阴道干燥 (4)性交疼痛

引导语:女性在40岁之后就会陆续迎来自己的更年期,顾名思义这一时期对其人生来讲是一个十分重要的转折期,能够顺利渡过这一时期对女性的后半生健康与幸福十分关键。

32.当出现上述症状时您愿意接受雌激素替代治疗吗?(1)正在接受 (2)不了解 (3)不愿意接受

如果您选择愿意接受或者正在接受雌激素治疗,请继续回答:

33.你愿意接受雌激素治疗的原因是:(可以多选)(1)解决烦躁心慌的问题 (2)预防骨质疏松 (3)解决性生活困难的问题 (4)为了保持年轻 (5)因为医生的劝说

34.接受雌激素治疗以后,在哪些方面得到了改善?(1)不再烦躁心慌 (2)骨关节不再疼痛 (3)性生活质量得到了改善 (4)没什么改变

引导语:到了绝经期,女性卵巢不再产生雌激素,所谓的性唤起障碍也就多起来了,也就是说女性的性生理兴奋遇到困难,表现为阴道的分泌减少了,润滑过程变慢了,而且可能带来性交时的疼痛。雌激素替代治疗对缓解她们的症状大有帮助,但有不少人因为担心激素的副作用,宁可忍受疼痛也不愿寻求治疗。现在好了,杨森公司的K-Y人体润滑剂的正式推出和上市必将给女性带来极大福音,既能解决问题,又没有任何副作用,其安全、有效、经济实惠。

(刘明矾 周旭 马晓年)


\chapter{第二十五章 性行为及其相关问题}

性行为是人类的本能行为之一,是最自然的生理需要,是种族延续的基本纽带,性行为同时也具有享乐功能,是人类生活乐趣的重要形式。从最狭义的定义来说,性行为仅指异性生殖器之间的接触,即性交。从广义上来说,性行为是指任何以达到性满足为目的的行为,包括抚摸、亲吻、相互触摸生殖器、口交、自我刺激(自慰)、肛交等,甚至性梦和梦遗也可包含在性行为的范畴之内。

人类的性行为有生理学基础,即受体内性激素的调节和控制,也受躯体状况及药物等多种其他因素的影响。另一方面人类的性行为又受到心理、社会、文化等因素的影响和控制。在一定的社会历史背景下,人类的性行为通常要符合当时社会的性观念。纵观历史,社会对性态度就像一个连续谱,一端是以性禁锢、严格控制性行为的态度为主,而另一端是以性放纵、性解放、主张完全性自由的态度为主。这两种极端的态度对人的身心健康及社会的健康发展都是不利的。


\section{第一节 何为健康的性行为}

人们已经认识到性禁锢与性放纵均不利于人体健康及人类文明,但性行为及性观念本身也与社会文化传统,人们价值观念、道德标准等密切相关。对何为性健康、何为健康的性行为制定放之四海而皆准的标准是不切实际的。但某些基本的原则还是得到大多数人的认同。

世界卫生组织提出性健康是指有性欲的人在躯体上、感情上、知识上和社会方面等的整体表现,是积极的增进人际交往和情爱,正确的性知识和正常的性快感是性健康的基础。性健康包括三个方面的基本因素:一、根据社会道德和个人道德准则充分享有性行为和控制生殖行为的能力;二、消除抑制性反应和损害性关系的恐惧、羞耻、罪恶感和虚伪的信仰等心理因素;三、消除器质性紊乱、各种疾病及妨碍性行为与生殖功能的缺陷。

性健康最终通过性行为体现,虽然对关于什么是健康的性行为存在不少争论,但有些观点还是得到普遍的认同:①人类不能过分压抑自己的性欲望,也不能无限制地追求满足自己的性欲望;②人类的性欲望有很大的个体差异,不能硬性规定每个人要有多少或只能有多少;③性行为要在符合法律和社会道德的范围内进行;④健康的性行为必须以正确的性卫生知识为基础,要有意识地防止疾病的产生和传播;⑤健康性行为应有爱的栽培过程,性与爱应有机结合。


\section{第二节 与医学有关的性问题的处理原则}

与医学有关的性问题分为性功能障碍和性心理障碍两大类。

性功能障碍是一组与心理社会因素密切相关的性活动过程中的某些阶段发生的性生理功能障碍。包括个人在性欲上的障碍,或是达到性满足的能力上的障碍。除了某些例外,这种障碍没有解剖上或是生理上的病理基础,大致是建立在不当的心理适应和学习上。这些障碍在一定程度上有很大的变异,但无论哪一方面的功能障碍,性关系中的双方在性愉悦上都会受到不利影响。性功能障碍症状的表现必须是持续存在或反复存在的,并因此不能进行自己所希望的性生活、对日常生活或社会功能造成影响,给患者带来明显痛苦的。至于偶尔的、一过性的性功能出现的问题不能诊断为性功能障碍。常见的非器质性性功能障碍的类型有性欲减退、阳痿(勃起功能障碍)、阴冷、性乐高潮障碍、早泄、阴道痉挛、性交疼痛等。

性欲望和性器官的功能受到多种因素的影响,包括生殖器的疾病、疲倦、过量的乙醇和某些药物。但是大多数性功能障碍患者还是源于心理和社会因素。

在性压抑的社会文化背景下,早期的教育可能会告诉年轻人性关系是色情的、肮脏和邪恶的。这种态度和随之而产生的压抑可能会导致对性行为的严重焦虑和罪恶感。从而引起性功能障碍。害怕性交失败而产生的焦虑恐惧情绪常影响阴茎的勃起和阴道的润滑,会造成性交困难和疼痛。在性交中没有全身心地投入,或过分理智或性交过程程式化,都会影响性交的情感体验。夫妻双方缺乏性交体验的交流也是造成性功能障碍的重要原因之一。

此外,缺乏性生理、性心理及避孕有关的知识会造成对性生活的忧虑,也可能会引起性功能障碍。生活事件会影响人们的情绪和生活质量,如工作压力大、长期精神压抑等也会使人感到性生活力不从心,或不能达到满意的境界。

国外女性性功能障碍的发生率约为20\%~50\%。社区流行病学调查发现,女性婚后第一年未体验到性高潮者达81\%、性欲减退34\%、性兴奋障碍11\%~48\%、阴道痉挛12\%~14\%、性交疼痛8\%~15\%。国外统计男性性功能障碍35岁以下者患病率为1.3\%、50岁以下6.7\%、60岁以下18.4\%、75岁以下55\%。我国男性性功能障碍的患病率估计为10\%左右。

国际疾病分类第10版(ICD-10)中非器质性障碍或疾病引起的性功能障碍的描述是:性功能障碍有各种表现形式,即个体不能参与他/她所期望的性关系。包括兴趣缺乏,性感缺乏,不能产生为有效的性行为所必须的生理反应(如勃起),或不能控制或体验到高潮。

性反应是一种心身过程,心理及躯体过程通常都在性功能障碍的发病中起作用。尽管可能辨认出心因性或器质性病因,但是更为常见的,是那些诸如勃起不能或性交疼痛等问题,这就很难确定心理性和(或)器质性因素何者为重。类似这样的病例中,将这种状况分类为混合性或病因不明较为恰当。

有些类型的功能障碍(如性欲缺乏)男女都可发生。不过,女性主诉性的主观体验不满意的较多见(如快感或兴趣缺乏),而缺乏特异性反应的较少见。主诉高潮功能障碍的并非罕见。但是一旦女性的性反应的一个方面受到了影响,其他方面也很可能会受损。例如,如果一位妇女不能体验到性高潮,那么她也常会觉得无法享受情调的其他乐趣,并因此丧失大部分性欲。而男性尽管主诉无法产生特异性反应如勃起或射精,却常报告仍有性欲存在。因此,为了作出最恰当的诊断分类,就不能仅局限于所提供的主诉。

目前治疗性功能障碍的方法一般是以行为训练结合心理治疗为主要手段,即性治疗。治疗前首先应当明确一系列与治疗有关的问题。如患者是否有性功能障碍?患者存在哪一类性功能障碍?可能的原因有哪些?是否存在躯体因素?哪些心理因素对目前的性功能障碍有影响?患者的性关系有什么特点?性心理发育有什么特点?

性治疗的过程就是把心理治疗会谈和行为干预有机地融合在一起。行为干预是在治疗者指导下,通过患者或患者及其配偶在家进行练习来实现。性治疗的目的是确立并改变引起性功能障碍的直接原因,如操作焦虑、缺乏自信、缺乏技巧等。

性治疗的方法主要有:

(1)性知识教育:由于相当部分的性功能障碍起因是缺乏必要的性心理和生理卫生常识,因此,针对治疗所涉及的各种性问题向患者提供新的、正常的知识。

(2)系统脱敏法:通过消除患者的焦虑、恐怖情绪来实现的。因为临床上许多患者主要是担心自己的性功能,或是怕得不到愉快而容易在夫妻性活动中感到不安。采用系统脱敏法便可通过松弛肌肉来减轻这种焦虑。

(3)认知行为疗法:对于性欲减退患者,认知治疗的策略在于改变抑制患者性欲的那些不合理信念,如“性欲旺盛是邪恶的、下流的,这种人不是好人”。在治疗中针对患者的这种观点进行解释、指导和给予必要的性知识教育,然后再指导、告诫患者,人们的行为(包括性活动)绝大多数是受其自身控制的,他们往往对自己所做的事情能够承担责任。在此基础上再结合使用有关行为松弛、性交技巧指导等训练技术,可使相当一部分性功能障碍患者的问题得以解决。

(4)精神分析疗法:精神分析理论认为性功能障碍一般有两个原因。其一,在成长过程中特别成年后无可选择或没有适宜的性对象;其二,对于必须放弃的童年早期性对象至今仍在潜意识中迷恋。前者因为不爱或者无情,后者则会沉溺在性幻想的臆造中。这种性心理活动长期持续,势必压抑了性欲,在婚后的性生活中就出现了性功能障碍。

通过自由联想,虽然可以使患者早年体验从潜意识中呈现出来,但大多数患者不能意识到与目前的病症有关,因此心理医生需要进一步进行释梦和阐释,以使患者领悟。

在应用精神分析疗法的整个过程中,心理医生一定要把握住两个原则:第一,将患者的性爱和情爱、性行为和性欲结合起来;第二,使患者夫妻在性生活中处于平等的性角色地位。同时要特别注意,患者并非仅仅表现为性功能障碍,还有其他方面心理障碍,因此,在治疗中应综合运用各种心理治疗方法。

(5)催眠疗法:性功能障碍大多是由心理因素所引起,也有躯体因素。催眠治疗对心理因素引起的性功能障碍是一种较有力的治疗手段,收效较明显而且迅速。躯体因素引起的性功能障碍在实行躯体治疗的同时也可配合催眠治疗。

催眠治疗性功能障碍的步骤如下:

1)分析原因与类型。通过详尽了解病史剂检查,找出产生性功能障碍的原因,确定其类型。如有躯体疾病则予以相应的治疗。

2)解除忧虑。在催眠中首先应改善因性功能障碍导致的焦虑和忧郁情绪,提高并增强患者对治疗的信心。

3)疏导与暗示。在催眠状态中,针对心理问题进行分析和疏导,按照不同的病症,下达治疗性指令。

4)咨询服务。做好求治者配偶的咨询工作,必要时也对其进行催眠治疗,使之能够积极配合。建议在催眠治疗的初期阶段暂缓性生活,最好分居两室。

5)试行阶段。通过数次治疗后证明性功能确有改善,方可试行性生活。

催眠治疗的次数和间歇时间,一般为隔日进行一次,共10次左右。见效后,需再进行两三次治疗,以提高和巩固疗效。

(6)性感集中训练法:性感集中训练法20世纪70年代由美国性学专家马斯特斯和心理学家约翰逊夫妇创立,是治疗性功能障碍的一种快速而有效的心理疗法。

性感集中训练法的基本理论是:虽然性功能障碍的病因是多因素的综合作用,但从根本上来说,是由焦虑引起的,尤其是性活动失败而导致的操作性焦虑。在性生活中,由于害怕失败而产生的焦虑紧张情绪,压抑了性功能的自然性,性功能的压抑又使性交失败。“焦虑—失败—焦虑”,长久下去,这种恶性循环形成了一种错误的性行为模式,即出现性功能障碍。采用性感集中训练,可使夫妻在性生活中很快消除焦虑,在循序渐进学习正确性行为模式的过程中,性功能的自然性会逐渐恢复,其功能障碍也会逐渐消除。

性感集中训练法一般分为以下四步:

第一步,性认识的一致与焦虑的松弛(3~5天)

当进行了详细的检查而除外器质性病变后,对夫妻双方详细介绍性的解剖、生理和心理知识,重点要介绍男女性反应周期的特点,不同的性表达方式及如何唤起性兴奋等。在讲解过程中,应辅以图片和一定的模型帮助他们理解。同时,鼓励他们对性的有关问题进行讨论,以求得比较一致的意见。在这个阶段,夫妻之间要分开居住,禁止性交,其目的是为了消除对性活动的焦虑状态。同时,应进行一些简易的松弛训练,以进一步消除焦虑紧张。

第二步,非性器官的肉体及情感交流(3~5天)

上一步完成后,夫妻双方应赤身裸体地躺在一起,互相接吻、拥抱和抚摸全身,但注意不要抚摸乳房和性器官。在进行这些活动时,可以用一些亲昵的言语进行交流,并体会由此带来的皮肤快感和情感享受。要注意,这些活动是为了提高身体各部分的感受能力,而不是为了使性唤起或满足性交需要。虽然这个阶段往往出现性兴奋,但一定不要性交,应该把注意力集中到体会整个身体的快感上。在这一步最后的1~2天,可以开始抚摸乳房,但仍然不要接触性器官。

第三步,性器官的抚摸与手淫技术的应用(2~3天)

在继续进行上一步活动的基础上,夫妻双方都要寻找自身性器官的最佳性刺激点。一般而言,男性的最佳性刺激点多集中在阴茎系带而不是阴茎头,女性则多为阴蒂和阴道口。然而,其刺激部位、刺激时间和刺激强度因人而异。当通过自身对性器官的刺激而达到最佳性快感后,应彼此抚摸性器官。此时,可用“手把手”而使对方的操作恰到好处。当互相抚摸性器官时,双方可以轻轻地把手搭在一起,以便抚摸时进行非言语性暗示,避免因讲话而冲淡愉快的感受。非言语暗示的信号可自行设计。例如,手可以从一点移向另一点,以表示“我不喜欢”或者手搭着不移动,表示“我喜欢这样”。

这个阶段仍然不要性交,而在操作过程中尽量体会心身的欣快感,并逐渐把性感集中到性器官上。

第四步,治疗性性交活动(4~5天)

在上述三步完成以后,就可以进行性交活动了,但这并非完全的随意性交,应该针对不同的性功能障碍辅以特殊的操作方法。

从1971—1977年间,玛斯特斯—约翰逊研究所用性感集中训练法治疗原发性阳痿19例,成功率为78.9\%;治疗继发性阳痿228例,成功率为85.4\%;治疗早泄240例,成功率为95.1\%;治疗不射精58例,成功率为74.1\%;治疗阴道痉挛54例,成功率98.1\%;治疗性高潮功能障褥(包括性感缺乏)388例,成功率71.9\%;治疗性厌恶116例(男35例,女85例),成功率92.4\%。实为行之有效的一种性功能障碍治疗法。

(7)药物治疗:万艾可(viagra又称西地那非)、希爱力等药物对治疗勃起功能障碍有效。激素替代疗法也被用于治疗某些性功能障碍。

性欲减退是指成人持续存在性兴趣和性活动的降低,甚至丧失。表现为性欲望、性爱好及性幻想缺乏。患者对性生活无任何要求,也没有性欲冲动,对性表现出无所谓的态度。早期的资料认为女性性欲减退较男性多见,但现在发现男女并无明显差异。

性欲减退的病因是多方面的,包括心理因素、生物学因素以及心理和生物学因素交互作用的结果。心理因素较为重要,夫妻感情不和会造成性生活不协调,继而产生对性生活、性行为厌恶、反感的负性情绪,导致性欲减退。婚外性行为所造成的疏离或负罪感;害怕性传播疾病而对性生活产生的恐惧情绪;不正确的性观念、不良性经历的影响;自信心不足;性伴侣之间冷淡及彼此缺乏信任感;性欲较强者不断要求做爱的压力;生活中长期、沉重的应激压力造成的持续疲劳状态等诸多原因都可能导致性欲低下。许多慢性疾病伴随的痛苦沮丧的负性情绪也可影响性欲。

对于性欲减退的诊断需根据患者经常的表现,需考虑个体既往对性的兴趣如何,做纵向的自身对比。同时也要考虑患者对此问题的看法及对不同性活动的兴趣。性活动的频率也不是判断此症的可靠标准,因为女性可能在配偶的压力下被动服从。

ICD-10对“性欲减退或缺失”描述到:性欲缺失是本障碍的首要问题,它并不是继发于其他性问题如勃起不能或性交疼痛。性欲缺失并不排斥性的快感或唤起,只是使性活动不易起动,包括阴冷、性欲低下障碍。

非器质性性欲减退应注意以下鉴别诊断:

1)躯体疾病所致性欲减退:慢性风湿病、高催乳素血症、神经退行性变疾病、心血管疾病以及消化系统疾病、泌尿系统疾病等都可能影响人的性兴趣、性功能。通过详细了解既往病史可以提供依据。

2)药物因素所致性欲减退:许多药物如三环类抗抑郁药、5-羟色胺再摄取抑制剂、抗雄性激素作用的物质如醋酸环丙酮等都有降低性欲的副作用,用药后会出现性欲减退,停用这些药物后性欲通常会逐渐恢复正常。

我国性学专家马晓年建议将性功能减退分为4级:

Ⅰ级:性欲较正常情况减弱,但可接受配偶性交;

Ⅱ级:性欲原本正常,但在某阶段后出现减退,或凪在特定的境遇下才出现减退;

Ⅲ级:性欲一贯低下,每月性生活不足两次或虽然超过这一标准,但系在配偶压力下被动服从;

Ⅳ级:性欲一贯低下,中断性活动6个月以上。

由于性欲减退的原因通常不是单一的,因此治疗时必须全面综合分析,采取适当的策略。首先应消除引起性欲减退背景因素,并指导患者接触动情材料及手淫训练,鼓励积极性体验,增加交流,并通过协商形成一种能为对方接受和满意的性活动方式。一些药物也被用于治疗性欲减退,如睾酮、氯哌三唑酮,但疗效尚难肯定。

1)男性性唤起障碍:男性的主要问题是勃起障碍。男性勃起功能障碍也称勃起不能,旧称阳痿,是指成年男性在性活动的场合下有性欲,但阴茎不能勃起或勃起不充分或历时短暂,不能插入阴道完成性交过程。但在手淫时、睡眠中或造成醒来时可以有勃起。这是一种最常见的男性性功能障碍。美国一项调查显示,40~70岁男性中,勃起功能障碍的发病率超过30\%。几乎一半以上的男性都有过一过性或阶段性的勃起功能障碍。

依病情不同可分为原发性勃起功能障碍和继发性勃起功能障碍。前者是指从性活动出现,就存在阴茎勃起障碍;后者是指原先阴茎勃起功能与性活动均正常,后来因某种原因而出现勃起功能障碍。

阴茎勃起是一个由中枢神经系统、神经血管、内分泌系统及生殖器等多系统常与的复杂反射过程,很多因素都会对其造成影响,如心理因素,躯体疾病、药物影响、年龄影响等。过去认为由精神心理因素引起的勃起功能障碍约占90\%,近年来,大多数学者认为至少50\%勃起功能障碍由器质性原因引起。

心理因素方面,严重性压抑的成长环境,对性交失败的恐惧、被伴侣贬低、伴侣拒绝敌意、心情抑郁、特殊的创伤性体验等因素都可能导致男性勃起功能障碍。

ICD-10对“男性勃起障碍”描述到:难以产生或维持进行满意的性交所需要的动起。如果在某些特定的场会如手淫时或睡眠中或与另一个伴侣在一起时,可正常勃起,那么其原因便可能是心因性的。否则,为使非器质性勃起功能障碍的诊断成立,就需要依靠特殊的检查(如测量夜间阴茎膨胀度)或心理治疗的效果来定。

非器质性的勃起功能障碍需排除躯体疾病及药物所致的勃起功能障碍。

内分泌失调类的睾酮水平不足、各类影响阴茎功能的神经性疾病、影响阴茎动脉血流入、流出的血管性疾病都能引起勃起功能障碍,此时应诊断为躯体疾病所致的勃起功能障碍。

许多药物影响勃起功能,如:大量长期饮酒、吸入尼古丁、某些抗抑郁药物、具有多巴胺阻滞作用的抗精神病药物等。通过仔细询问用药史及停药后症状的缓解可予以鉴别。

治疗勃起功能障碍首先要去除可影响勃起的各种心理因素,纠正过去形成的错误观念与习惯。常用的治疗方法包括性感集中练习和生物反馈疗法。性感集中练习时女方可有节奏地抚弄阴茎,使其勃起,当阴茎勃起而坚硬时,女方停止抚弄,让勃起消退。这样可反复多次,待勃起持续时间较长后,方可进入性交过程。

万艾可、希爱力等能有效治疗勃起功能障碍,它们的作用是在有性欲及性刺激的情境下发挥的,不能增强性欲,也不能解决心理问题,所以只能是心理治疗的辅助方法。

2)女性性唤起障碍:女性性唤起障碍过去也称作阴冷,但因其带有蔑视的味道,现已逐渐被弃用。它是指成年女性有性欲,但难以产生或维持满意的性交所需要的性交生殖器的适当反应,以致性交时阴茎不能舒适地插入阴道。心理因素可能是女性性唤起障碍的主要原因。本病的症状有复发的倾向,有的病程迁延不愈,有的可能发展成性欲低下。

ICD-10对“女性性唤起障碍”描述到:女性的主要问题是阴道干燥,或缺乏滑润。其原因可能是心因性的也可能是病理性的(如感染)或雌激素缺乏(如绝经后)。女性以阴道干燥为主诉就医的情况是不多见的,除非是绝经后雌激素缺乏所致。

马晓年等将女性性唤起障碍按严重程度划分为4级:

Ⅰ级:女性在性活动中,可以有正常的性生理反应,以往也有性快感,但目前性感缺失,或在某些特定的境遇下性感缺失;

Ⅱ级:女性一直具有性感缺失;

Ⅲ级:过去曾具有正常的阴道润滑等生理反应,但目前性兴奋反应缺失,阴道润滑不足或严重不足;

Ⅳ级:一直缺乏性兴奋和性生理反应,阴道润滑不足或严重不足。

性唤起障碍的治疗除了进行性知识的教育外,增加夫妻或伴侣间的情感交流,尤其是性感体验的交流,以及增加爱抚,寻求敏感点,注意性前戏及性后戏等都是有益的。

对男性而言性高潮的主要表现是射精,对女性而言,性高潮是指性兴奋和性乐的顶点。性高潮障碍指持续地发生性交时缺乏性高潮的体验,不能从性交中获得足够的刺激以达到性高潮。女性较常见,男性往往同时伴有不射精或射精显著迟缓。

女性性高潮功能障碍率较高,据国外报道,新婚第一年仅19\%的女性出现性高潮,结婚5年内也只有约60\%出现性高潮。女性性高潮障碍的表现形式多样,根据程度不同可表现为从未能达到性高潮;有时可以有性高潮,但不能经常出现性高潮;根据发生的时段可分为原发性性高潮障碍(自有性生活开始就缺乏性乐高潮)和继发性性高潮障碍(曾经有一段时间出现过性高潮,而后又消失了)。有的女性在性交时不出现性高潮,而通过手淫可以出现性高潮,也有的在改变性交方式后才会出现性高潮。

缺乏安全感、过分完美主义、工作繁忙、精神高度紧张、夫妻关系不和、缺乏性经验交流、对高潮到来的过分关注、对异性的不满、对性行为的厌恶、对怀孕的恐惧等因素都会影响性高潮的出现。

ICD-10对“性高潮功能障碍”描述到:性高潮不出现或明显延迟。这可能是境通性的(如只见于某些特定环境),其中病因有可能是心因性的或恒定的,除非个体对心理治疗反应良好,否则不应轻易排除躯体或体质因素。性高潮障碍女性比男性多见,包括性高潮受抑(男性、女性)和心因性性高潮缺失。

鉴别诊断应与继发于躯体疾病或药物使用区别开来。

对于性乐高潮障碍主要是心理治疗。据临床分析,患有性感缺乏的女性,基本上缺乏科学的性知识,往往认为性交是“男性享乐、女性受罪”。因此,给予患者一些性科学方面的知识指导,矫正对性的错误认识,是治疗的基础。此外还要帮助患者消除造成障碍的精神心理因素,尽量减轻患者对高潮释放的焦虑。教会患者用自己手淫的方法来获得性高潮。

行为技术也是治疗性乐高潮障碍的主要手段。行为技术治疗性性乐高潮障碍的基本目的就是要使女性体会到性感,并进一步激发出性高潮。用行为技术治疗性乐高潮缺乏可大致分为以下几个步骤:①学习科学的性知识,了解行为治疗的原理和方法。需2~3天。②夫妻分居,不断熟悉自己的身体。需3~5天。③用按摩器及自身抚摸来逐渐体会性快感。需3~5天。④在双方抚弄、亲吻、拥抱的基础上,让另方抚摸其最佳性刺激点。需2~3天。⑤治疗性性交活动,需3~5天。⑥坚持③至⑤步的方法,巩固训练1~2月,以获得巩固的疗效。在整个治疗过程中,为了进一步观察疗效,调整治疗方法,应坚持治疗日记,重点记录训练前后的具体感受。这不仅可以进行自我总结,增强治疗的信心,还可以给临床心理医生的进一步指导提供切实依据。

早泄是指持续地发生性交时射精过早导致性交不满意,或阴茎未插入阴道时就射精。如果需要长时间刺激才能引起阴茎勃起,射精就会显得过早,这种情况属于勃起迟缓而不是早泄。偶尔出现早泄属于正常现象。

早泄的病因很复杂,主要与生物、心理、社会因素有关。射精控制困难与性交时出现的焦虑情绪密切相关。“早泄—性交焦虑—早泄”的恶性循环会使早泄症状不断恶化。女性伴侣对性交的态度也与男性的早泄密切相关,女性的激动与责备可能会导致男性早泄的出现。对婚姻的不满、性伴侣间缺乏了解和情感交流、性价值观不一致、工作紧张、事业受挫、性环境缺乏安全等都可能促使早泄的发生。

ICD-10对“早泄”描述到:无法控制射精,以使性交双方都能享受性快感。在严重的病例中,未进入阴道或还未勃起时就出现射精。早泄多不是器质性的,但可作为器质性损害(如勃起不能或疼痛)的一种心理反应而出现。如果勃起所需的刺激时间较长,射精也会显得过早,这是由于充分的勃起与射精之间的间隔被缩短了。这种情况下的根本问题是射精延迟。

对于早泄主要采取心理治疗和行为治疗。心理治疗主要是帮助患者消除影响早泄的不良心理因素,缓解患者焦虑的情绪,促进性伴侣间的沟通等。

治疗早泄的行为主义治疗方法主要有:

1)系统脱敏疗法:射精是由前列腺、会阴部肌肉、阴茎体一起有节律地收缩来协同完成的。如果主动有意识地放松会阴部肌肉,就可以延缓射精,延长性交的时间。这种放松关键是掌握好时间(每当快要射精时就停止性交动作立刻放松骨盆肌,特别是肛门括约肌和提肌),如此反复的练习就会提高射精所需要的刺激阈,直到男方达到能耐受很大的刺激而又不射精,当女方达到强烈的性高潮时,男方再主动射精,这种训练便告成功。

2)捏挤术法:当男方阴茎被刺激而快要射精时,女方把拇指放在阴茎系带的部位,食指与中指放在阴茎另一面的冠状沟上下方。由前向后稳捏、压迫4秒钟,然后突然放松。女方需注意用指腹而不要用指甲,施加压力的方向不要从一侧向另一侧。反复多次后,再进入性交。

3)牵拉阴襄法:当男方快要射精时,为了减轻其兴奋性,防止其提前射精,可用手向下方牵拉男方阴囊和睾丸。反复进行2~3次,来延缓射精时间,达到推迟射精。

4)手淫法:手淫已被用于男女性问题的不同方面。在男性方面,手淫可以通过条件化,改变性指向;另一方面手淫能作为早泄的治疗方法,即在性交之前先手淫一次,能减轻性交时的兴奋程度而防止早泄,这种方法适用于力强的青年人。

5)紧握术法:女方在男方出现性唤起时,紧握阴茎几秒钟,然后突然放开,如此反复紧握后再性交。男方欲要射精时便暂停性交,拔出阴茎,再紧握几秒钟然后放开。如此反复,直至勃起减退些,再性交,持续摩擦而射精。

非器质性阴道痉挛指性交时阴道肌肉强烈收缩,致使阴茎插入困难或引起疼痛。如勉强插入常可引起性交疼痛,所以常有回避行为。这是一种影响妇女性反应能力的心理生理综合征。阴道痉挛常与性交疼痛互为因果关系。性交疼痛导致阴道痉挛,而阴道痉挛又加重性交疼痛。其发病原因常源于对性生活的无知而产生的恐惧、紧张、担心、害怕的心理(如害怕怀孕、害怕受伤、害怕性传播疾病等),严厉的家庭教育、早期的性创伤、也可能是造成阴道痉挛的因素。

阴道痉挛可以分为原发性阴道痉挛和继发性阴道痉挛,前者是指从未有过正常反应,后者指曾一段时期性活动的反应相对正常,然后发生阴道痉挛。有些患者不进行性交时可有正常的性反应。有些则对任何性接触的企图都恐惧,并力图避免性交。

ICD-10对“非器质性阴道痉挛”的描述是:阴道周围的肌肉挛缩,导致阴道人口的封闭。使阴茎不能插入或引起疼痛。阴道痉挛可能是局部疼痛所致的继发性反应,在这种情况下,不应使用本类别。包含:心因性阴道痉挛。

诊断时应仔细采集病史,进行妇科检查,排除妇科疾病。

马晓年等建议将阴道痉挛按严重程度分为4级:

Ⅰ级:痉挛的发生仅限于会阴部肌肉和提肛肌群,或痉挛仅在某些特定的境遇下发生;

Ⅱ级:痉挛不仅限于会阴部,而且包括整个骨盆肌群,或痉挛在多种境遇下均会发生;

Ⅲ级:臀部肌肉也发生不经意痉挛,整个臀部可不由自主地抬起,痉挛频繁发生,性交很难完成;

Ⅳ级:患者双腿内收并竭力后撤整个身体,甚至出现大喊大叫等惊恐反应。这种反应往往不是由实际行为所引起,而是对临近行为的预感性反应。该痉挛系原发性,性交从未完成。

关于阴道痉挛的治疗,目前主张综合性治疗原则,以向患者讲授性科学知识、指导性技巧为主,还可采用松弛-收缩-松弛行为技术疗法和优势兴奋中心转移疗法。可以让患者用手指逐渐插入阴道而体会快感,并可适当用润滑剂。根据需还要考虑使用阴道扩张器,逐渐使阴道肌肉的收缩得以缓解。但后一种治疗必须在专科医生的指导下进行,选择不同型号的扩张器,逐渐由小号到大号进行扩张。患者自己熟悉后,由丈夫协助进行训练,再逐步使阴茎插入阴道。

性交疼痛指性交引起男性或女性生殖器疼痛。包括阴道、外阴及下腹部的疼痛。这种情况不是由于局部病变引起,也不是阴道干燥或阴道痉挛引起。

性交疼痛可以分为:①原发性性交疼痛,指从初次性交开始既有疼痛;②继发性性交疼痛,指曾经有过一个阶段满意的性生活,而后才发生性交疼痛;③完全性性交疼痛,指任何情况下进行性交都会产生疼痛;④境遇性性交疼痛,指仅在某些情况下产生性交疼痛。

引起性交疼痛的常见原因有缺乏性经验、性技巧、男女双方配合欠佳,性兴奋不足、性情粗暴、急躁等。童年期错误性知识、强烈的性压抑、性罪恶、性耻辱感导致的焦虑情绪、人际关系的麻烦、工作压力的重负、性对象缺乏性魅力等因素也是可以引起性交疼痛的原因。

ICD-10对“非器质性性交疼痛”的描述是:性交疼痛(性交时的疼痛感)在男性和女性都可见到。它常与局部的病理状况有关,因此应归入适当类别。然而在有些病例中,并无明显原因可见,而情绪因素显得重要。只有当不存在其他原发的性功能障碍(如阴道痉挛或阴道干燥)时,本类别才适用。包含:心因性性交疼痛。

性交疼痛需要与泌尿科或妇产科的疾病相鉴别,排除上述两科疾病后方可下诊断。

马晓年建议将性交疼痛按严重程度分为4级:

Ⅰ级:性交时不适感或轻度疼痛;

Ⅱ级:阴茎插入时或抽动时阴道浅部疼痛;

Ⅲ级:性交时阴道深部疼痛或疼痛在性交结束后扔持续存在;

Ⅳ级:性交疼痛严重,乃至性交不能进行。

性交疼痛的治疗可采用使交感神经紧张缓解的药物,包括局部麻醉药,一旦获得缓解就可以开始采取阴道松弛训练,同时针对心理原因进行心理治疗。

性心理障碍(psychosexual disorder)又称性变态或性欲倒错。泛指在两性行为方面的心理和行为明显偏离正常,并以这类偏离为性兴奋、性满足的主要或唯一方式,从而不同程度的影响和干扰正常性活动。此类性变态是指在有正常异性个体存在或可能获得的前提下出现的,而境遇性的类似情况不属性变态。

性行为是人类的基本活动,人类通过生殖功能使种族得到繁衍。社会的稳定发展也有赖性生理及性心理方面的成长和发展。人类性行为受社会文化的制约,不同国家、不同种族在不同历史阶段对性行为的观点也不相同。因此对性行为的正常与否作出一个明确的界定是有一定困难的,两者的区别只是相对的、有条件的。凡是符合某个社会公认的社会道德标准并符合生物学需要的即可看做是正常性行为,否则就视为不正常。

性心理障碍的共同特点是:患者产生性兴奋、性冲动及性行为的对象和一般常人不一样。他们对一般常人不引起性兴奋的某些物体或情境产生强烈的性兴奋,而对正常的性行为方式有不同程度的干扰或减低。

由于性心理障碍患者很少主动暴露疾病和主动求医,因此性心理障碍的患病率难以确切估计,现有资料多来自被拘留的患者,在性心理障碍各种类型中以露阴症最为多见。

性心理障碍的形成有多种原因,精神动力学理论强调恋母情结,近来又强调与不适当心理防御机制有关。行为主义学派则强调性变态是后天经条件反射习得的行为,近年来强调整合理论模式。幼年期的家庭环境不良,如父母的婚姻生活以离异家庭较多、父亲性格粗暴、虐待妻子及子女、对子女的培育不负责任,在学校中受到同学及老师的歧视或欺负、学习成绩不佳,能使人性格古怪、内向、不合群、不主动社交活动,在工作及婚姻等方面处处失败,从而在性心理方面产生明显障碍。也有的性心理障碍的人在性格及生活、工作等各方面比其他人相差并不悬殊,如某些同性恋者。

在生物学方面,有学者发现颞叶损伤可致恋物症、异装症、性施虐症、恋尸症。乙醇中毒时可出现露阴症、恋尸症。颅脑外伤后可产生露阴症。精神分裂症、精神发育迟滞、老年性精神病可伴发性变态行为。有人认为性心理障碍患者血浆睾酮可能存在异常,但这一推测并未被验证。性心理障碍的病因仍是个谜。

ICD-10中将性心理障碍分为性身份障碍(有变换自身性别的强烈欲望)、性偏好障碍(采用与常人不同的异常性行为满足性欲)和与性发育和性取向有关的心理及行为障碍(个体可能已成为问题的各种性发育或取向障碍)三部分。不包括单纯性欲减退、性欲亢进及性生理功能障碍。

对于性心理障碍各种类型的诊断,只要有详细的病情材料,特别是患者自己的性行为的体验及行为表现,诊断并不困难。只要把由于脑器质性疾病产生的性欲改变、人格异常鉴别出来,这种鉴别从年龄、脑器质性疾病的各种表现,以及脑电图、头颅CT等项检查结果都可以明确诊断。

对性心理障碍进行治疗与处理包括正面教育、心理治疗、认知行为矫正、药物治疗等。正面教育主要是明确指出患者异常行为的危害性。心理治疗主要是帮助患者回顾自己心理发展的过程,理解在何时、何阶段、由哪些因素导致心理的偏差,使患者正确领悟并进行自我心理纠正。

性心理障碍心理治疗的基本原则包括:①明确求治动机。治疗时要仔细询问患者的一贯性活动表现,了解患者的治疗动机。一般性心理障碍的患者较少主动愿意改变自己的性取向,往往是基于其行为被其他人发现的压力,或对其他人造成伤害,或感到抑郁内疚等原因。在治疗初期,必须强调患者要明确其求治动机,否则,治疗将难以进行,疗效也难以保证。有些性变态患者为了免于行政或司法处罚,摆脱当时面临的困境而前来求治,以后一旦危机解除便会中断治疗。②建立治疗性协议和良好的医患关系。心理治疗的成功与否在很大程度上依赖于治疗性协议的制定和良好的医患关系的建立。在治疗初期,医生与患者应共同讨论有关的治疗计划、治疗目标及双方各自应承担的义务和责任,明确治疗过程中的角色,相互监察各自的遵守情况。

认知行为矫正的技术包括:

1)认知矫正:应用认知三栏技术,让患者学会辨别自己错误的认知所在,诘难认知过程中的曲解,同时结合布置行为家庭作业,不断进行真实性检验。

2)放松训练:部分患者往往伴有焦虑、抑郁等负性情绪体验,指导他们学会怎样放松骨骼肌的紧张。

3)隐匿性强化伴嗅觉厌恶:先让患者列出所有会诱发其异常性行为活动的场所或情境(如恋物症者对某种物品,窥阴症者对女浴室、女厕所等);然后列出可能导致的不利后果(如受家人和朋友的蔑视,或被开除、拘留等)。最后让患者闭眼想象某一境遇,诱导其性兴奋;一旦表现出性兴奋,可令其深深呼气后对着氨水瓶深吸一口气,同时想象和讲出可能会发生的不利后果。

4)预防复发训练:有一系列技术,包括让患者学会识别诱发变态性行为的有关生活事件及情绪状态,识别看似无关实质的举措,以及允许小错、禁止违法等策略。

对某些性心理障碍也可适当采用药物治疗。如对那些异常性冲动非常强的男性可以短期使用雌激素,但雌激素有不少副作用,使用时应权衡利弊。而那些伴有严重焦虑、抑郁情绪的缓和则需要用一定的精神药物缓解其情绪焦虑和抑郁。

性身份是指与性别有关的性格、气质、思想感情和行为,是人对自身性属性的自我体验和辨析,也就是说一个人的内心对自己的性别的认识是否与其生物学性别相一致。一般来说,二者一致,但有极少数人却完全不一致,即性身份障碍。这些患者的解剖和生理完全正常,只是性心理不正常。性身份障碍可能与年幼时所受的教育不当、正常的异性活动受挫、社会不良文化的影响和个性因素有关。

(1)易性症:也称性别转换症,是性身份的严重颠倒。典型异性癖者性器官解剖结构通常没有什么异常,但对自身性别的认定与解剖生理上的性别特征持续存在厌恶,以强烈要求改变性别为特征,并采取各种措施或寻求医药帮助。这种人不仅自我深信并声称自己是异性,而且希望他人也按异性对待自己,可同时作异性装扮及伴有同性恋。非其他精神疾病所致的类似表现,无生殖器解剖生理畸变与内分泌异常。患者绝大多数是男性,一般起自青春期,如果不满足其性别转换的要求,其内心常感到十分痛苦,可导致心因性抑郁,具有强烈的自杀或自残倾向,甚至自己动手割去阴茎和睾丸。

ICD-10对“易性症”的描述是:渴望象异性一样生活,被异性接受为其中一员,通常伴有对自己解剖性别的苦恼感及不相称感,希望通过激素治疗和外科手术以使自己的身体尽可能地与所偏爱的性别一致。其诊断要点是:转换性别身份至少应持续存在2年以上,才能确立诊断,且不应是其他精神障碍如精神分裂症的症状,也不伴有雌雄同体、遗传或性染色体异常等情况。

性身份障碍需与异装症、同性恋、精神分裂症相鉴别。

异装症者以异性装扮来体验所引起的强烈性兴奋或性感满足,其对自己的生物学性别持肯定态度,性指向正常。易性症患者可有异性装扮行为史,但无特别的性兴奋及性快感,他对自己的生物学性别持否定态度,有改变自己性别的一贯性要求,其性定向通常指向同性。

同性恋者的性定向与大多数异性症者相同,都指向同性,但同性恋者心理性别与生物学性别统一;而异性症者心理性别与生物学性别处于对立的两极。易性症与同性恋者在性行为上也有一定差异,异性症者在衣着、举止和行为上自动模仿异性,在行为中是自觉的异性,而同性恋者在性行为和感觉中均是同性。

精神分裂症的性别改变妄想是荒诞离奇的,而且伴有其他情感、知觉和思维障碍。异性症者却是正常人出现性别意识的改变。

(2)双重异装症:在生活中的某一时刻穿着异性服装,以暂时享受作为异性成员的体验,但无永久改变性别愿望的病症。患者穿着异性服装时,并不伴有性兴奋,而是基于对异性性别的偏爱,认为只有着异性装扮才符合其性身份。双重异装症与恋物性异装症不同,后者穿着异性服装是为了唤起性欲以及得到性满足。属于性心理障碍中的性偏好障碍。双重异装症与同性恋者穿着异性服装不同,后者属于性取向障碍,穿着异性服装是为了取悦于性伴,或认为只有这样才符合他们的性取向。

双重异装症的产生原因由生物、心理、社会等因素的综合作用形成(见性身份障碍)。在治疗上,以心理治疗为主。可采用行为治疗,如厌恶疗法,以逐渐减少乃至消除穿着异性服装的行为。与此同时,还要培养并增强他们对本身性别的认同,鼓励参加各种社交活动,加强他们的社交能力和技巧,以增加其正常的性行为。

ICD-10对“双重异装症”的描述是:个体生活中某一时刻穿着异性服装,以暂时享受作为异性成员的体验,但并无永久改变性别的愿望,也不打算以外科手术改变性别。在穿着异性服装时并不伴有性兴奋,这一点可与恋物性异装症相鉴别。包含:青春期或成年期性身份障碍,非易性型。不含:恋物性异装症。

(3)童年性身份障碍:这一障碍通常最早发生于童年早期(一般在青春期前已充分表现),其特征为对本身性别有持续的、强烈的痛苦感,同时渴望成为异性(或坚持本人就是异性)。持续地专注于异性的服装和/或活动,而对患者本人的性别予以否认。通常认为这类障碍相对少见,较常见的是与程式化性角色行为不一致的状况,二者不应混淆。只有正常意义上的男性或女性概念出现了全面紊乱时,才可考虑童年性身份障碍的诊断。仅有女孩子像“假小子”、男孩子“女孩子气”是不够的。若已进入青春期,此诊断便不能成立。

性身份障碍伴有对本身性别的解剖结构的持续排斥的情况是罕见的。在女孩,会表现为反复声称她们有或将要长出阴茎来,拒绝以蹲位姿势排尿,或声称她们不愿乳房发育或来月经。在男孩,会表现为反复声称他们的身体将发育成为女人,阴茎和睾丸令人讨厌或将消失,最好没有阴茎或睾丸。

ICD-10中,“童年性身份障碍”的诊断要点是:儿童出现根深蒂固的、持续的成为异性的渴望,伴有对自身性别的行为、特性和(或)衣着强烈的排斥。典型情况下,在入学前就首次出现;要想确立诊断,这一障碍必须在青春期前就已十分显著。在男女两性中,都可能会出现对本身性别的解剖结构的否认,然而上述表现较少见,甚至很罕见。患有性身份障碍的儿童有一个特点,即尽管他们因与家庭、好友的期望相冲突而苦恼,也因所受到的嘲笑和/或排斥所痛苦,但他们却否认因性身分障碍而苦恼。

(4)其他性身份障碍和未特定的性身份障碍。

(5)性身份障碍的处理原则:异性症的纠正是十分困难的,故应强调预防为主。对新生儿进行正确的性指导和符合其生物学性别的行为训练有较大意义。在异性症早期,尤其是青春前期,行为疗法有助于控制其异性意识的发展。另外,由于异性症者常伴有抑郁情绪,因此需要医学心理干预。国外有的就满足患者要求给予异性激素或采取性别转换术。

性偏好障碍是指满足性欲望所选择的对象或方式异常。性偏好障碍的实质是患者的性心理发育不成熟,性格过于羞怯,在社会上缺乏与异性交往的机会或能力,害怕以正常的方式去求爱或做爱,故表现出异常的或幼稚的行为方式。性偏好障碍患者的病态心理表现多种多样、十分离奇,但万变不离其宗:他们对正常的性交并不感兴趣,只青睐于离奇的性偏好,往往表现出种种其目的不是指向异性完整个体和正常性行为性满足方式,而分别表现为性对象的异常和性行为方式的异常。他们的特点不在于异常的性行为和追求偏离常态有多远,而在于正常性行为的缺乏。那些从不追求正常的性关系,把性对象象征化或把性行为目的化的人就属于性偏好障碍,或称性欲倒错。

(1)恋物症:恋物症是直接从无生命的物体主要是异性体表接触的物品(乳罩、内裤、长裤袜、高跟鞋等)或异性身体的一部分获得性兴奋的一种性偏好。恋物症者通常无法以一个实际存在的完整的异性人作为性爱中心,而是因所恋物品引起性联想,性兴奋。患者通常抚摸或嗅闻所恋物品同时自己手淫,或在性交时自己或性对象手持此物,以获得性满足。恋物对象可以是任何东西,恋物症者对物品的迷恋程度有强弱的不同,典型的恋物症需要视觉和触觉刺激。有时,仅视觉刺激如色情画等,即可引起内心一阵愉快的反应。患者会千方百计收集所恋物品,不惜冒险偷窃。本症几乎仅见于男性,初发于青少年性成熟期,病因不明,患病率不详。

ICD-10对“恋物症”的描述为:以某些非生命物体作为性唤起及性满足的刺激物。恋物对象多为人体的延伸物,如衣物或鞋袜。其他常见的对象是具有某类特殊质地的物品如橡胶、塑料或皮革。迷恋物的重要性因人而异:在某些病例中仅作为提高以正常方式获得的性兴奋的一种手段(如要伴侣穿上特殊的衣服)。其诊断要点为:只有当迷恋物是性刺激的最重要的来源或达到满意的性反应的必备条件时,才能诊断为恋物症。恋物性的幻想很常见,但除非它们引起了显著强制性、无法接受的仪式动作,以至干扰了性交,造成了个体的痛苦,否则不足以诊断为此种障碍。恋物症几乎仅见于男性。

恋物症无特殊药物治疗,以心理治疗为主,当今应用认知疗法和行为疗法,特别是采用厌恶疗法或系统脱敏疗法,已成为治疗恋物癖的主要手段,疗效也较理想。

(2)异装症:异装症是恋物症的一种特殊形式,表现对异性衣着特别喜爱,反复出现穿戴异性服饰的强烈欲望并付诸行动,由此可引起性兴奋。其穿戴异性服饰主要是为了获得性兴奋,当这种行为受抑制时可引起明显的不安情绪。患者性身份辨识没有问题,而且其性指向也正常,只是一种性行为手段方式异常。此症病因不明,多起病于青春期,开始时偶尔穿着一两样异性服装,以后逐渐增加异性服饰的数量。有的患者在身着异性服饰时手淫,有的患者则在性交时部分或完全着异性服装。

ICD-10对“异装症”的描述为:穿着异性服装主要是为了获得性兴奋。其诊断要点为:这一障碍与单纯的恋物症不同,他们所迷恋的衣物不仅是穿戴,而是打扮成异性的整个外表。通常不止穿戴一种物品,常为全套装备,包括假发和化妆品等。恋物性异装症与异性装扮症不同,前者清楚地伴有性唤起,一旦达到性高潮,性唤起开始消退时,便强烈希望脱去异性服装。在易性症者中,早期阶段常有恋物性异装症的历史,这种病例可能为易性症的一个发展阶段。

(3)露阴症:是指在不适当的环境下暴露自己的生殖器,引起异性紧张性情绪反应,从中获取性快感的一种性偏离。几乎仅见于男性,多发生在青春期。这些患者在偏僻场所或黑暗角落处守候,当异性走进时突然露出生殖器,有的还伴有其他行为,如露阴时与视者说话或者大喊大叫,证明其生殖器是正常的,有的同时伴有手淫,还有的试图与视者有手的接触,但这只是一种象征行为。患者在引起对方惊恐反应后迅速离去,一般无进一步攻击行为。患者在露阴之前有逐渐增强的焦虑紧张体验。在陌生的异性面前暴露出通常是勃起的生殖器,当对方感到震惊、恐惧或对其辱骂时感到性满足。如果被暴露对象表现出无动于衷,反倒令露阴者大为扫兴。大部分露阴症患者个性内向,不善与人交往,性功能低下或缺乏正常的性功能,已婚的患者通常不能与妻子建立起满意的性关系。

ICD-10对“露阴症”的描述为:向陌生人(通常为异性)或公共场合的人群暴露生殖器的一种反复发作或持续存在的倾向,但无进一步勾引或接近的意图。在露阴时通常出现性兴奋并继以手淫,但也并非全都如此。这类行为也可在很长的间歇期不明显,只在情绪应激或危机时出现。

对于露阴症患者可采用厌恶疗法,如让患者面对一美女画像露阴,随即给予厌恶刺激(如电针)。

(4)窥阴症:是指一种反复窥视他人的性活动、裸体、异性下身以满足引起性兴奋的强烈欲望,可当场手淫或事后回忆窥视景象并手淫,获得性感满足的一种性偏离。窥阴症的患者常常千方百计偷看妇女如厕、女浴室或他人卧室,在较早的年代有的会携带镜子钻入粪池或伏于水沟中、屋梁上。一些文学作品对此有生动的描述。如余华的小说《兄弟》中李光头的父亲就是在窥阴时跌入粪池淹死的。现在有不少患者通过高倍望远镜窥视他人的卫生间或卧室。

窥阴症患者通常为男性,年轻者较多,多有孤僻、不善交际的人格特点。窥阴症者的性身份辨识正确,性定向及选择也无问题,只是以视觉性的性刺激作为性感满足的主要内容,而在客观上部分或全部排斥了异性恋的性交行为。观看淫秽音像制品,并获得性的满足,不属于本诊断。

ICD-10对“窥阴症”的描述为:一种反复出现或持续存在的窥视他人性生活或亲昵行为如脱衣的倾向。通常引起性兴奋和手淫,这些活动是在被窥视者觉察不到时进行的。

对窥阴症主要采用心理疗法,可通过行为技术来矫正。第一步是引导其求治心理。第二步是从两方面进行矫治,其一,用厌恶疗法抑制并消除其异常的性行为,其二是用系统脱敏疗法消除对正常性活动的不适感。这两种方法的结合运用也称为交互抑制。治疗阶段一般需要两个月。必要时还可辅以药物治疗,疗效比较理想。

(5)恋童症:恋童症又称童奸,是指反复多次把儿童或发育未成熟的少年作为性对象,靠猥亵或奸污他们来引起性兴奋与获得性满足,而对成年异性对象相对或完全缺乏性兴趣的病症。根据不成文的规定,患者年龄须大于等于16岁,患者与受害儿童的年龄差距大于等于5岁,受害儿童年龄一般小于等于12岁。对于年龄较大的青少年患者,未规定明确的年龄差距。确诊有赖于临床判断,如果受害儿童已届青春期后期,一般可称作儿童性骚扰或恋慕儿童。而不是恋童症。

恋童症患者偏好异性儿童和同性儿童的比例为2∶1,异性指向的男性偏爱8~10岁的女童;绝大多数案例中,患者均与受害儿童认识。患者较多采用眼观手摸而非生殖器接触。同性指向的男性患者偏爱10~13岁的男童,与受害儿童相识的比例远低于异性指向的男性患者。双性指向的成年患者选择小于8岁的儿童。专一恋童症患者只对儿童感兴趣,非专一恋童症患者也会对成人感兴趣。

恋童症患者的性活动可能局限于自己的孩子或近亲中的孩子(乱伦),也可能侵害其他儿童。玩弄儿童的患者可对受害儿童或其他宠物施以暴力或暴力威胁以阻止儿童泄露其暴行。恋童症的病程缓慢,可因药物依赖、抑郁、人格障碍而复杂化。

恋童症的病因不详。恋童症患者通常有不同的性问题,既可能是性心理发育不成熟,也可能是其他心理因素引起的性功能障碍。这些问题使患者无法与成年人进行正常的性交往。

恋童症是一种很难治疗的精神障碍。对不造成儿童身体直接伤害的恋童症患者,医学治疗似乎比法律惩处更可取。治疗主要采用厌恶疗法,同时辅以心理动力学治疗,并注意纠正个体与成年人性交往中存在的问题。对儿童造成伤害的恋童症患者,法律制裁是必要的。

(6)施虐受虐症:以向性爱对象施加虐待或接受对方虐待,作为性兴奋的主要手段。前者是指通过对异性对象身体上造成痛苦和屈辱(其手段为捆绑、鞭挞、侮辱甚至伤害)而产生性兴奋和性快感的一种性心理异常。后者是指在接受所爱对象或自己施行的虐待,通过痛楚和屈辱而发泄其情欲并获得性满足的一种性行为异常,他们往往捆绑、鞭挞自己,或伤害自己的乳房、阴茎、睾丸等部位。性窒息和性缢死者常伴有这种性心理障碍。性施虐症与性受虐症者常结成伴侣,一个愿打一个愿挨,许多患者还交替充当这两种角色,施虐狂患者可能在生活中遭受过挫折或欺凌,或遭受过异性的拒绝和侮辱,因而形成报复与反抗的心理,或系自卑感的过度补偿,以此反常行为作为性欲的发泄和表现男性的优越感;而受虐癖者则多见于女性,它可能是害怕遗弃的恐惧心理的变态表现,也可能是内疚或罪恶感的自责自罚的表现。其暴力程度和范围可明显不同,从不形成肉体损害或造成轻微痛苦到极端的残暴行为均可发生。

ICD-10对“施虐受虐症”的描述是:将捆绑、施加痛苦或侮辱带入性活动的一种偏好。如果个体乐于承受这种刺激,便称为受虐症;如果是施予者,便为施虐症。个体常常从施虐和受虐两种活动中都可获得性兴奋。

性施虐症与性受虐症尚无特殊药物治疗,以心理治疗为主。

(7)性偏好多相障碍:ICD-10对其描述为:有时在一个人身上可同时存在一种以上的性偏好障碍,其中无论哪种也不占优势。最常见的是恋物症、异装症和施虐受虐症的结合。

(8)其他性偏好障碍:其他各种类型的性偏好与活动也可发生。但每种相对少见。包括以淫秽的语言打电话、在拥挤的公共场所以摩擦别人的身体获得性刺激(摩擦症)、与动物发生性行为、以勒颈或缺氧的方式增加性兴奋,或偏爱那些解剖上异常的性伴侣如截肢者。

性欲活动变化万千,许多都很罕见或属个人症性,以至无法一一命名。喝尿、涂抹粪便或刺穿阴茎包皮或乳头都可为施虐受虐症中的行为模式。各种类型的手淫方式更为常见,但更极端的手段,如将物体插入直肠或阴茎尿道,或部分的自我窒息,一旦这类行为取代了正常的性接触,便达到了异常的程度。恋尸症、摩擦症均包含在其中。

本病无特殊药物治疗,以心理治疗及纠正治疗为主。

(9)其他性偏好障碍,未特定。包含性偏离NOS。

与性发育和性取向有关的心理及行为异常主要包括性成熟障碍、自我不和谐的性取向、性关系障碍及其他性心理发育障碍。

与性发育和性取向有关的心理及行为异常诊断时需注意,单纯的性取向问题不能被视为一种障碍。诊断要点分别为:

(1)性成熟障碍:ICD-10对“性成熟障碍”的描述是:个体为不能确定他/她的性身份或性取向而苦恼,从而产生焦虑或抑郁。最多见于少年,他们无法确定自己是同性恋、异性恋还是双性恋。有些个体常常已经有固定的性关系,却在一段时间的确定稳固的性取向之后,发现他们的性取向发生了变化。

(2)自我不和谐的性取向:ICD-10对“自我不和谐的性取向”的描述是:性身份或偏好是确定无疑的,但由于伴随有心理和行为障碍,个体希望它们并非如此,并可能寻求治疗试图加以改变。

现在国际上普遍认为同性恋是一种正常的变异,是一种正常的性取向,不是心理障碍,而仅仅是一种变异而已。目前我们国家的精神疾患诊断标准已经和国际接轨,开始纠正过去把同性恋看成是性心理障碍的认识,但仍有少数同性恋者在寻求医生的帮助,他们对同性恋的性取向感到很痛苦,来自父母、家庭、亲戚朋友和社会的压力使其苦不堪言,则属于自我不和谐的同性恋。

(3)性关系障碍:ICD-10对“性关系障碍”的描述是:由于性身份或性偏好异常,因而无法与性伴侣形成或维持正常关系的一种性心理障碍。

性身份是人对自身性别的自我体验和确信。正常个体均有自己的性别,包括基因、性别、生殖器性别以及在此基础上形成的性身份和性别角色。在正常情况下,性别与性身份是一致的,与社会生活中扮演的性别角色相适应。如果因某种原因出现了性身份异常,便可造成性别角色的错位或紊乱,因而无法与社会公认的异性性伴保持正常稳定的关系。例如,性别改变症可以按照自己的性别找到异性伴侣,亦可以按性身份与异性认同而以同性者为性伴。如果实施了性别转换手术,再造性体像,更愿意寻找原本与其同性别的人为性伴。许多个案报告表明,即使这些性别改变症者目的达到,其婚姻仍是痛苦的。因为社会不接纳他们另一类的所谓性偏好者。由于他们对常人不引起性兴奋的某些物体能产生强烈的性兴奋,或者采用与常人不同的异常性行为方式,如异装、露阴、窥阴、摩擦异性身体、虐待性对象或接受性对象的虐待等作为性兴奋的主要手段和满足性欲的主要途径,也无法与其性伴侣形成和维持正常和谐的关系。性关系障碍者有可能保持正常的工作能力和维持一定的人际关系,当然这种人际关系是指性关系以外的社会交往。不过,在很多情况下,患者自身都会感到十分痛苦或给家庭带来不少麻烦,甚至造成违纪违法的严重后果。

(陆峥 邸晓兰)

1.Gelder M.Concise oxford textbook of psychiatry.Oxford:Oxford university press,1994

2.刘达临.世界性文化图考.北京:中国友谊出版公司,2000

3.中华精神科学会.中国精神障碍与诊断标准(第3版)(CCMD-3).济南:山东科学技术出版社,2001

4.陆峥.性心理咨询.上海:同济大学出版社,2002

5.沈渔邨.精神病学.北京:人民卫生出版社,2002

6.王晓慧,孙家华.现代精神医学.北京:人民军医出版社,2002

7.徐兆寿.非常对话.北京:中国青年出版社,2003

8.潘绥铭,白维廉,王爱丽,劳曼.当代中国人的性行为与性关系.北京:社会科学文献出版社,2004

9.马晓年.现代性医学.北京:人民军医出版社,2004

10.江开达.精神病学.北京:人民卫生出版社,2005

11.刘文明,刘宇.性生活与社会规范.武昌:武汉大学出版社,2006

12.李凌江.行为医学.湖南:湖南科学技术出版社,2008


\end{document}
