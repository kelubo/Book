% 舒克和贝塔全传
% 舒克和贝塔全传.tex

\documentclass[a4paper,12pt,UTF8,twoside]{ctexbook}

% 设置纸张信息。
\RequirePackage[a4paper]{geometry}
\geometry{
	%textwidth=138mm,
	%textheight=215mm,
	%left=27mm,
	%right=27mm,
	%top=25.4mm, 
	%bottom=25.4mm,
	%headheight=2.17cm,
	%headsep=4mm,
	%footskip=12mm,
	%heightrounded,
	inner=1in,
	outer=1.25in
}

% 设置字体,并解决显示难检字问题。
\xeCJKsetup{AutoFallBack=true}
\setCJKmainfont{SimSun}[BoldFont=SimHei, ItalicFont=KaiTi, FallBack=SimSun-ExtB]

% 目录 chapter 级别加点(.)。
\usepackage{titletoc}
\titlecontents{chapter}[0pt]{\vspace{3mm}\bf\addvspace{2pt}\filright}{\contentspush{\thecontentslabel\hspace{0.8em}}}{}{\titlerule*[8pt]{.}\contentspage}

% 设置 part 和 chapter 标题格式。
\ctexset{
	chapter/name={},
	chapter/number={}
}

% 设置古文原文格式。
\newenvironment{yuanwen}{\bfseries\zihao{4}}

% 设置署名格式。
\newenvironment{shuming}{\hfill\bfseries\zihao{4}}

\title{\heiti\zihao{0} 舒克和贝塔全传}
\author{郑渊洁}
\date{}

\begin{document}
	
\maketitle
\tableofcontents
	
\frontmatter
\chapter{前言、序言}
	
\mainmatter
	
	% 增加空行
	~\\
	
	% 增加字间间隔,适用于三字经、诗文等。
	\qquad  
	
\chapter{第1集}

舒克生在一个名声不好的家庭里;
舒克吃了有生以来最香的一顿饭   
“舒克,你都大了,可以自己出去找东西吃了。”一天,妈妈对小老鼠舒克说。   
“真的吗?”舒克高兴了。  
舒克是一只生活在中国的小老鼠,他从生下来以后就一直憋在洞里,从来没有出去玩过。  
“今大晚上,我带你出去,先认认路,以后你就可以自己去了。”妈妈一边说,一边磨牙。  
舒克也学着妈妈的样子磨牙。他爱吃好东西。每次妈妈给他带回来好吃的.他都吃个没够。  
夜里,舒克跟在妈妈身后出了洞。  
“好大的屋子!”舒克惊叫道。  
“小声点儿!”妈妈告诫舒克。  
“为什么不能大声说话?”舒克问。  
“对于咱们老鼠来说,在外边小声说话安全。”妈妈说。  
妈妈告诉舒克,那是衣柜,那是写字台,那是电脑,那是床。舒克把眼睛都看累了,他觉得这个世界很有意思。  “这个柜子对咱们最有用,里面全是好吃的,叫冰箱。”妈妈把舒克带到一个柜子跟前。“可它的门总是关着,得找机会。现在,咱们到餐桌上去,那里有一盘花牛米。”  
一听有花生米,舒克的口水快流出来了,他跟着妈妈爬上了餐桌,果然,桌上有一盘香喷喷的花生米。  
舒克和妈妈大吃起来。  
“小偷!这么小就学偷东西!”黑暗里传来一个声音,吓了舒克一跳。  
“偷吃人家的东西,真不要脸!”又是一声。  
舒克借着月光一看,窗台上有一个鸟笼子,笼子里有两只鹦鹉,一蓝一绿,刚才的话,就是他俩说的。  
听人家管他叫“小偷”,舒克脸红了。他看看妈妈,妈妈就像没听见一样,继续吃着。  
“你吃饱了?”妈妈看见舒克不吃了,问。  
“妈妈,咱们这叫偷吗?”舒克小声问。  
“傻孩子,什么偷不偷的,咱们老鼠世世代代就是这样活下来的。别理他们,贩卖正直的人最不正直。快吃吧。”  舒克又吃了两颗花生米,他觉得,今天的花生米不如以往的香。  
第二天夜里,舒克自己出来找吃的了。他又来到写字台上,可那盘花生米不见了。舒克正准备下去,蓝鹦鹉喊起来:“小偷又来了!”  
“真是的,有什么样的妈妈就有什么样的儿子。”绿鹦鹉也跟着说。  
“胡说!我妈妈说,我们不是小偷!”舒克要争这口气,他大声对鹦鹉们说。  
“这些吃的东西是你劳动得来的么?”蓝鹦鹉问舒克。  
“这……”舒克说不出话来了。  
“不是你劳动换来的,就是偷!”绿鹦鹉耸耸鼻子。  
“哼,你妈妈不但偷,还净搞破坏,衣柜里的衣服就是被她咬坏的!”蓝鹦鹉说。  
舒克愣住了。  
“你出去打听打听,谁不知道你们老鼠是坏蛋!你敢大白天出去吗?人家都说,老鼠过街,人人喊打!”绿鹦鹉说。  
舒克没想到自己家的名声这么坏,他委屈极了,自己干吗生下来就是只老鼠呢!舒克哭了。  
舒克不愿意当小偷,他决定离开家,到外面去闯闯,通过劳动换取食物。  
舒克看中了床头柜上那架米黄色的电动直升机,它有一副红色的塑料螺旋桨。舒克曾经从洞口里看见直升机在屋里飞过,很酷。  
这天清晨,窗户大开着,直升机静静地停在床头柜上。舒克悄悄地钻进了飞机,这架直升机的机舱挺大,除了驾驶员坐的地方以外,后面还有两排皮椅子。  
舒克想起了“老鼠过街人人喊打”的话,他决定化装一下,让人家看不出他是老鼠。  
舒克忍着疼,把胡子都拔下来。他穿上飞行服.将尾巴缠在腰里。舒克看见床头柜上有一筒牙膏,他跑过去打开盖,挤出许多牙膏涂在脸上。  
一切都准备好了,舒克坐进驾驶舱,戴上飞行帽。  
“现在我已经不足老鼠了,足飞行员舒克。”舒克兴奋地想。他打开了启动器,红色的螺旋桨转了起来,它越转越快,不一会儿,直升机就离开了床头柜。  
舒克驾驶着直升机在屋里盘旋了一圈,他还故意擦着鸟笼飞过去,当他看见鹦鹉们认不出他时,得意极了。  
小老鼠舒克,不,飞行员舒克驾驶着直升机,从开着的窗户飞出了屋子。 
    外面是碧绿的田野,起伏的丘陵,还有宽阔的河流和盛开的花丛……舒克驾驶直升机尽情地在天上飞,他很兴奋。 
    舒克觉得肚子有点儿饿,他决定去找点儿吃的。舒克操纵直升机下降高度,他把头探出飞机,注意观察地面。 
    “救命!救命呀!’ 
    舒克忽然听到地面上传来呼救声。 
    舒克一看,是一只蚂蚁掉进水洼里,他正在拼命挣扎。 
    舒克急忙将直升机开到了水洼上空,然后操纵飞机垂直下    “我来救你!”舒克把头探出飞机,大声喊。他将飞机悬停在空中,离水面只有两寸远。可飞机上没有绳子,蚂蚁怎么上来呢? 
    眼看小蚂蚁不行了,舒克忽然想起了自己的尾巴。他急忙解开裤子,把尾巴从腰上解下来,打开飞机舱门,将尾巴伸向水面。 
    “你抓住绳子爬上来,快!”舒克大声喊。 
    小蚂蚁抓住舒克的尾巴,爬上了直升机。 
    舒克关上舱门,操纵直升机拉起了高度。 
    “谢谢你,谢谢你!'’小蚂蚁一边擦身上的水,一边感激地说。 
    活这么大,舒克头一次听到别人谢他。 
    “你叫什么名字?”小蚂蚁问。 
    “我叫飞行员舒克。”舒克说。 
    “这架直升机真漂亮。”小蚂蚁打量着机舱说。他忽然看见了舒克的尾巴,“哟,你的绳子真像老鼠的尾巴。” 
    “啊,是吗?”舒克一惊,这才想起忘了将尾巴藏起来,他一边把尾巴往裤子里塞,一边说:“直升机上的绳子都是这样的,有弹性。” 
    小蚂蚁仔细地打量舒克,笑了。 
    舒克担心小蚂蚁认出他是老鼠来.看样子没有,要不,小蚂蚁肯定不会再对他笑了。 
    “你家在哪儿?我进你回家。”舒克说。 
    小蚂蚁把头贴在玻璃上,给舒克指路:“就在那棵大树后面。对,再往前飞,绕过那个土坡。看见了吗?就是那个洞口。” 
    这是舒克第一次操纵直升机着陆,他聚精会神。 
    直升机平稳地降落在蚂蚁洞旁边。 
    舒克给小蚂蚁打开舱门,小蚂蚁跳了下去。 
    舒克赶紧把尾巴缠在腰里。不一会儿,小蚂蚁领着一大群蚂蚁走到飞机旁边。 
    “舒克,这是我们蚁王,她来谢你了。”小蚂蚁对舒克说。 
    一听是蚁王,舒克赶紧从飞机上下来。 
    “谢谢你救了我的孩子,我能为你做点儿什么吗?”蚁王和蔼地问舒克。 
    “不用谢,”舒克心里美滋滋的,“我,我有点儿…饿。” 
    “快去拿最高档的食物。”蚁千命令。 
    很快,几百只蚂蚁抬着许多米饭粒、面包渣放到舒克面前。舒克大吃起来,真怪,他觉得,这些东西比花生米还香。 
    “你们以后有什么事,就来找我帮忙,我经常在这一带飞。”舒克吃饱以后对蚁王说。 
    “我们也欢迎你经常来!”蚁王笑眯眯地回答。 
    “你可经常来呀!”小蚂蚁眼圈红了。 
    舒克心里也挺酸,可他不敢哭,要是眼泪把牙膏冲掉了,人家认出他是老鼠来,谁还理他! 
    舒克钻进直升机,冲大家招招手,起飞了。   

\chapter{第2集}
 
​    飞机多转了一个弯; 
​    舒克为自己的名声苦恼; 
​    舒克永远是大家的朋友   
​    舒克驾驶直升机来到一片花丛上空,他看见许多蜜蜂在采蜜。 
​    “今天的蜜真多,都运不回去了,怎么办呢?”一只蜜蜂对同伴说。 
​    “就是,怎么办呢?”大家都很着急。 
​    舒克把头探出窗外:  “我帮你们运吧?” 
​    蜜蜂们吓了一跳,抬头一看,是一架米黄色的大直升机悬在空中。 
​    “你是谁?” 
​    “我是飞行员舒克。” 
​    蜜蜂们一看有飞机帮他们空运蜜,高兴了。 
​    直升机在花丛中着陆,蜜蜂们把蜂蜜运进机舱。 
​    “你自己送去吧,我们还得采蜜。我们家就在小河对岸那棵最高的树上。”一只金翅膀蜜蜂对舒克说。 
​    一只白翅膀蜜蜂不放心,小声说:  “咱们又不认识他,要是他……” 
​    “你别把人都想得那么坏,我看不会。” 
​    舒克看见金翅膀蜜蜂这么相信他,很感动,说:“你们放心,我一定送到。” 
​    直升机起飞了。机舱里充满了蜂蜜的香昧儿。小时候妈妈给舒克吃过蜂蜜,很香。舒克回头看了几眼蜂蜜,咽了一下口水,心想人家这么相信自己,自己可不能偷吃。 
​    舒克看见了小河,他驾驶飞机转弯向小河对面飞去。 
​    飞机转弯的时候,盆里的蜜洒出来一点儿。 
​    舒克用手指蘸着一尝,味道不错。原来,这是没有加工过的花粉蜜。舒克想,这不算偷吃,是它自己洒出来的。这么想着,他又操纵飞机在小河上面做了一个更急的转弯,这回洒出来的蜜更多了。 
​    “这倒不错,既没偷,又能解馋。”舒克满意地想。 
​    舒克把蜂蜜安全地送到了蜜蜂的家。他来来回回帮助蜜蜂空运了十几次,蜜蜂们都很感谢他,收工时,蜜蜂给舒克搬来了一大盆蜂蜜。 
​    “我说他是好人吧!”金翅膀对白翅膀说。 
​    舒克想起自己在飞机上吃人家的蜜,有点儿后悔。 
​    “我不要蜂蜜了。”舒克说。 
​    “那不行,一定得留下。”蜜蜂们不容分说,将蜂蜜搬进了机舱。 
​    “你以后想吃蜂蜜就来,咱们是朋友了,我们对朋友一点儿不吝惜。可上次有只老鼠来偷蜜,我们就狠狠地教训了他一顿。”金翅膀说。 
​    舒克真怕蜜蜂看出他是老鼠,他向蜜蜂们告别后,急忙起飞了。 
​    舒克开着直升机在天上转悠,他知道,只要人家认不出他是老鼠,都会对他友好。可一旦人家知道他是老鼠,一定不会理他了。想到这儿,舒克把飞行服整了整,再摸摸腰里的尾巴缠得牢不牢,又将飞行帽戴好。 
​    “砰!”地面传来一声枪响。 
​    舒克往下一看,一个小男孩拿汽枪将一只麻雀从树上打下来。麻雀的翅膀被打伤了,在地上一蹦一蹦地跳着,小男孩从远处追过去。 
​    舒克驾驶直升机来了一个俯冲,落在小麻雀身旁。他打开舱门,喊:“快!快上来!” 
​    小麻雀来不及细想,上了直升机。 
​    好险!小男孩刚跑到跟前,米黄色的直升机腾空而起,小男孩愣在那里。 
​    “你很勇敢!”小麻雀望着舒克说。 
​    “伤得重吗?” 
​    “翅膀伤了,特疼。” 
​    “他干吗打你?” 
​    “我也不知道,他总拿枪打我们。妈妈就是让他们打死的。” 
​    “人比老鼠还坏吧?”舒克问。 
​    “老鼠?老鼠最坏。” 
​    “可老鼠没用枪打死别人呀!”舒克提醒小麻雀。 
​    “老鼠名声不好。” 
​    名声,就是这个名声!害得舒克整天穿着飞行服,戴着飞行帽,还把尾巴缠在腰里,热死了,也不敢脱。舒克恨死“名声”这个东西了。 
​    “你怎么了?”小麻雀看到舒克不吭气了,  “对了,我还忘了问你是谁呢?” 
​    “飞行员舒克。”舒克不大情愿地回答。他不明白,自己救了他为什么不能理直气壮地说真名字——小老鼠舒克! 
​    “你真好,谢谢你,飞行员舒克!” 
​    这次听到人家谢他,舒克心里不是滋味儿。他想听“谢谢你,小老鼠舒克”。 
​    不过,舒克一会儿就把不愉快的事忘了,他请小麻雀吃蜜,小麻雀说不喜欢吃蜜,喜欢吃虫子。舒克答应帮他抓。 
​    天快黑了,舒克将直升机停在一座楼房的房顶上。他让小麻雀在机舱中休息,自己跑出去给他抓虫子。 
​    舒克从来没抓过虫子,他费了九牛二虎之力,总算抓到了几条。 
​    小麻雀看到虫子,高兴地吃了起来。舒克笑了。 
​    第二天,舒克把小麻雀送回家。 
​    舒克经常为大家办事。渐渐地,谁都知道有位飞行员舒克开着米黄色的直升机,爱帮助别人。 
​    这天,经小麻雀提议,大家宴请飞行员舒克。主办宴会的是蚂蚁国王、蜜蜂皇后,还有许许多多受过舒克帮助的朋友都来了。 
​    宴会很丰盛,有很多好吃的食物。大家坐好了在等舒克。 
​    舒克开着直升机去参加宴会。这些日子,他通过自己的劳动,交了许多朋友。舒克看见下面有一片花丛,他操纵飞机在花丛中着陆,舒克要给朋友们带点鲜花。 
​    舒克摘了一朵红花。 
​    “这朵送给小麻雀,”他想。 
​    舒克又摘了一朵黄花。 
​    “这朵送给金翅膀小蜜蜂。” 
​    忽然,舒克身后刮来一阵急风,他感到一阵颤栗,浑身发软,他还没明白是怎么回事,就觉得自己的肩膀已经被牢牢地抓住了。 
​    “我当什么飞行员舒克,原来是只老鼠!”舒克身后传来一阵大笑。 
​    舒克回头一看,天哪,是只小花猫!小时候,舒克就听妈妈说过,猫是老鼠最大的敌人,祖祖辈辈是冤家。 
​    “你以为化装了,就能逃过我的眼睛?走,我要让小麻雀他们看看你的真面目,然后再处决你。” 
​    一听说要带他去见小麻雀他们,舒克急了,他哀求道: 
​    “我求求你,现在就把我处死吧,千万别让他们知道我是老鼠。” 
​    舒克宁可死了,也要保个好名声。 
​    “想得倒美!走!”小花猫不理会舒克的苦苦哀求,用手轻轻一提,就把舒克拎走了。 
​    “这下完了。”舒克闭上眼睛,想像着一会儿大家骂他的场面。 
​    “你们的飞行员是只老鼠,看看吧!”小花猫把舒克往地上一扔,大声宣布。 
​    舒克站起来,他不敢睁眼睛。 
​    大家都愣了。 
​    “我现在就去处决他!”小花猫像审判长一样宣布,他说完又抓起舒克。 
​    “住手!”小麻雀飞到小花猫跟前,“你干吗要处决他?” 
​    “因为他是老鼠!” 
​    “可他没干过坏事呀!” 
​    “老鼠都是坏蛋!” 
​    “不对,舒克就不是坏蛋!” 
​    “对,舒克不是坏蛋!”大家一起嚷道。 
​    “他是一只老鼠呀!”小花猫急了。 
​    “老鼠不老鼠我们不管,他是我们的朋友舒克!我们的朋友舒克!”小蚂蚁大声说。 
​    “对,他是我们的朋友舒克,不许你伤害他!”金翅膀蜜蜂飞起来,只要小花猫敢动舒克一根毫毛,他就要蜇他。 
​    舒克再也忍不住了,眼泪刷地流了下来,他不怕把脸上的牙膏冲掉了。 
​    小麻雀过去给他擦干眼泪。 
​    “舒克,来,宴会开始。”小麻雀宣布。 
​    舒克笑了,他把飞行帽摘掉,坐在餐桌正中央。 
​    “他们疯了!和老鼠一起会餐!”小花猫讨了个没趣,怏快地走了,他实在想不通。 
​    从那以后,舒克再也不怕别人知道他是老鼠了,他每天驾驶着米黄色的直升机,为朋友们做事。   

    第3集 
        贝塔用布口袋装香味儿; 
        咪丽不让贝塔吃饭; 
        贝塔学会驾驶坦克   
        贝塔也是一只小老鼠,从他降生的那天开始,就有一个可怕的影子始终跟踪着他,那影子是咪丽。 
        咪丽是一只猫。 
        咪丽害得贝塔两天没吃东西了,这天晚上,一股香味儿从洞外飘进来,贝塔忙拿出他的小布口袋,将香味儿装进去。这是贝塔想出的办法,每当香味儿飘进来时,就用口袋把它装起来,留着以后饿了时闻。 
        可今天贝塔实在太饿了,越闻香味儿就越想吃东西,他决定出去冒一次险。 
        贝塔先把头探出洞外,屋里静悄悄的。 
        “咪丽大概出去玩了吧?”他小心翼翼地出了洞。 
        冰柜旁边有只碗,那里边总足有好吃的,什么鱼呀、肉呀…贝塔就是饿死也不敢过去吃,那是咪丽的饭碗,主人每大往这个碗里放好吃的。 
        贝塔想在地上找点儿剩饭。就在这时,他忽然听见有响动。贝塔探头一看,好家伙,眯丽正盯着他呢! 
        他赶忙窜回洞去。吓得直喘粗气。 
        “小偷!你敢出来吗?”咪丽在洞口吓唬贝塔。 
        贝塔连答话都不敢。就这样,贝塔被咪丽一连堵了3天!他已经饿得全身无力,手脚发软了。 
        咪丽呢,每天放意当着贝塔大吃大喝。主人这几天似乎特别优待她。 
        看着咪丽大吃一气,贝塔咽口水。 
        “干吗她每天可以大模大样吃这么多东西?而我吃一点儿就是偷。要是主人每天也给我一点儿东西吃,哪怕比咪丽少得多,我就不会偷了,主人真是个怪东西。”贝塔想。 
        贝塔不想饿死,他得想办法活下去。 
        贝塔惟一的乐趣,就是每天晚上看电视。屋里的电视机正好对着贝塔的洞口,他不用出去就可以看电视。 
        这天,一部电视片吸引了贝塔,屏幕上的一群坦克在进攻,把敌人打得落花流水。坦克什么都不怕,连高大的墙都被它撞塌了。 
        “坦克这么厉害!”贝塔想起床底下有一辆绿色的电动坦克,他的眼睛闪出了奇异的光。 
        趁眯丽出去喝水的空儿,贝塔钻出洞,跑到床底下找到了那辆绿色的电动坦克。 
        贝塔学着电视上坦克驾驶员的样子,打开坦克上的舱盖,钻进坦克里边。 
        坦克里很宽敞,装几个贝塔都不成问题。贝塔关紧炮塔上的舱盖,从里面把插销插上,又使劲儿推了推,直到他确信咪丽从外面肯定打不开时,才松了一口气。 
        贝塔仔细打量坦克内部,他对这里的一切都很陌生。贝塔坐在驾驶员的座位上,他发现前面有一个小镜子,贝塔把脑袋凑过去,居然能看到外面! 
        贝塔想起来了,电视上说过,这叫潜望镜。 
        潜望镜下面有一排漂亮的电钮。贝塔试着按了一下红色的电钮,坦克启动了,飞快地向前冲去,贝塔又按了一下黄色电钮,坦克向后退去。 
        贝塔开心极了,他把所有电钮都按了一遍。有的能操纵炮塔转圈,有的能加大前进速度。有的能让坦克拐弯。不一会儿,贝塔就能熟练地操纵坦克了,现在,贝塔不怕咪丽了,他甚至盼着咪丽快点回来;这种心情贝塔还是头一次有。 
        贝塔把坦克隐蔽在床底下,焦急地盼望眯丽的出现。   第4集 
        贝塔驾驶坦克大败咪丽; 
        贝塔击退咪丽援兵; 
        咪丽奄奄一息; 
        贝塔出走   
        贝塔从潜望镜里看见咪丽回来了,他按了一下电钮,操纵坦克向咪丽冲过去。 
        眯丽看见一辆坦克从床底下开出来,她还没明白是怎么回事时,坦克已经撞到她身上,把她撞了一个跟头。她刚站稳,坦克又冲过来了,又是一个大跟头。咪丽急忙跳上桌子,气喘吁吁地看着这辆凶猛的坦克。 
        不一会儿,坦克上的盖子打开了,露出贝塔的头。 
        “喂,怎么样?害怕了吧?”贝塔嘲笑地说。 
        咪丽一看是贝塔,猛地从桌上扑下来。 
        贝塔连忙钻进坦克,等咪丽刚落地,坦克又把她撞了个跟头。这次撞在头上,咪丽两眼直冒金星。 
        咪丽傻眼了,忙逃回到桌子上。 
        这回,贝塔不理咪丽了,他开着坦克来到咪丽的饭碗旁边,把咪丽的食物都搬进了坦克。 
        贝塔在坦克里荡气回肠地狂吃。通过潜望镜,贝塔看见咪丽急得直跺脚,他得意地笑了。 
        这天咪丽没吃上东西。 
        贝塔决定以后就住在坦克里。他找来一些棉花,在坦克里铺了一张舒舒服服的软床。又找了一个纸盒子当贮藏食物的仓库。 
        白天,贝塔把坦克开到床底下隐蔽起来。晚上,他开着坦克出来吃主人给咪丽准备的食物。 
        一到夜里,整个屋子就成了贝塔的天下。他驾驶着坦克横冲直撞,追得咪丽满屋子乱蹿。 
       咪丽决定去搬援兵。 
      “臭贝塔,你等着,一会儿非把你的乌龟壳翻个底朝天不可,哼!”咪丽边说边跑出屋子。 
        “她要真叫来十几只猫,把坦克翻过来就糟,。”贝塔着急了。 
        他忽然看见了坦克上的大炮,对,用大炮打他们!可没有炮弹呀,贝塔眼珠一转,想出个主意。 
        床底下的篮子里有不少花生米,贝塔拿了个口袋,装了满满一口袋,搬进坦克里。他把一颗花生米塞进炮膛,一按电钮,“啪!”打出去一颗。 
        贝塔很快发现炮上也有一个小镜子,那是瞄准镜。他又装进一颗炮弹,瞄准挂在墙上的气球,一按电钮,“啪!”气球炸了。 
        现在贝塔什么都不怕了,他把炮口对准门口,装好炮弹,等着咪丽。果然,咪丽叫来了5只猫! 
        “他在哪儿?”一只黄猫刚进屋就说。他不相信一只老鼠能把猫治住。 
        话音未落,就昕“啪”的一声,黄猫的门牙被打掉了,疼得他“嗷嗷”直叫。 
        另一只灰猫朝着坦克冲过来。 
        贝塔瞄准他的鼻子又是一炮,炮弹打进灰猫的鼻孔里出不来了,疼得他掉头就跑。 
        另外几只猫都傻了眼,他们看见一辆绿色的坦克从床底下冲出来,一边开炮一边横冲直撞,猫们争先恐后逃出了屋子。 
        从此以后,不管白天晚上,整个屋子都成了贝塔的天下,就是咪丽跑到衣柜上,贝塔的炮弹也能打着她。 
        咪丽想了许多办法,可每次她都败在贝塔手下。她的饭碗已经成了贝塔的饭碗。主人惊奇地发现,近几天从未丢吃的,他还以为这是眯丽的功劳,因此决定好好慰劳她。主人每天往咪丽的饭碗里放好吃的,他哪儿知道,咪丽一点儿没吃着,全让贝塔享用了。 
        咪丽已经整整4天没吃东西了。这天中午,她悄悄爬上了餐桌…… 
        “好啊,你竟敢偷吃东西!我每天给你那么多饭还不够你吃!你个馋猫!” 
        主人看见咪丽居然敢爬到餐桌上偷吃他的饭,大发脾气,抄起(又鸟)毛掸子没命地打咪丽,吓得咪丽在屋里上蹿下跳。 
        贝塔在床底下开心极了。当咪丽躲到床底下时,他就开炮把她轰出去。 
        这天晚上,主人用绳子把咪丽捆在椅子腿上,惩罚她。 
        贝塔的坦克缓缓地停在咪丽身旁,当贝塔确信咪丽已经被捆得结结实实之后,他打开坦克舱盖,钻出来坐在炮塔上,二郎腿一跷,悠闲自得地看着咪丽。 
        咪丽看了一眼贝塔,闭上了眼腈。她饿极了,再加上全身被打得火烧一样的疼,浑身无力,骨头都快散架了。 
        贝塔本来想好好取笑她一番,可看到咪丽这副可怜的样子,贝塔想起了自己从前挨饿的日子,他开始同情咪丽了,贝塔后悔不该把咪丽弄到这个地步。 
        贝塔从坦克上跳下来,走到咪丽身旁。 
        “饿肚子最难受了,我知道。”贝塔一边说,一边开始咬捆在咪丽身上的绳子。 
        咪丽睁了一下眼睛,看看贝塔,又闭上了。 
        “我一会儿给你点儿吃的。”贝塔继续咬绳子。 
        尼龙绳很结实,贝塔的牙齿都咬疼了,还剩最后一根。贝塔稍微歇了一会儿,用劲把最后一根绳子咬断了。 
        咪丽猛一回身,一口咬住了贝塔。 
        贝塔万万没想到,咪丽会来这一手,他不顾身上火辣辣的疼痛,回头咬了咪丽鼻子一口。 
        咪丽疼得大叫一声,松开了嘴。她实在太饿了,无力追捕贝塔。 
        贝塔钻进坦克,把坦克开到床下,他听到主人起来了。 
        主人听到咪丽叫,打开灯一看,咪丽居然敢把绳子咬断了。他勃然大怒,拿起(又鸟)毛掸子又是一顿猛打,这回咪丽连跑的劲儿都没有了。 
        打完之后,主人又把咪丽捆在椅子腿上。 
        贝塔通过潜望镜看着这一切,开始他觉得挺出气,可后来又觉得咪丽挺可怜。但贝塔实在想不通刚才咪丽干吗恩将仇报咬他呢? 
        贝塔觉得要是自己能吃上饭,咪丽就吃不上饭。如果咪丽有饭时,那他贝塔就得挨饿。要是他俩能一起吃该有多好。可看样子咪丽不会这样干。 
        “干脆,我离开这个屋子,自己到外边去闯荡吧。”贝塔拿定了主意。他不愿意让咪丽总是饿肚子。 
        贝塔的坦克又缓缓地停在了咪丽身旁。这回,咪丽连眼睛都不敢睁了,她知道贝塔一定会狠狠地报复她。 
        咪丽觉得鼻子前面有香味儿,她睁开眼睛一看,贝塔把坦克里的食物搬出来放在咪丽面前。 
        “我要走了,请原谅,我实在不敢再把绳子咬断了。”贝塔说,“你吃吧,饿肚子最难受了,好了,后会有期。” 
        贝塔说完钻进坦克里。一想到再见不到咪丽了,贝塔心里还有点儿酸溜溜的感觉,奇怪。 
        贝塔又把坦克舱盖打开,最后看一眼咪丽。咪丽正大口大口地吃着贝塔给她的食物。贝塔头一次看见,咪丽的眼睛里有晶莹的泪水。 
        贝塔盖好舱盖,驾驶着坦克,从咪丽出入的小门驶出了屋于。 
        外边是满天星斗。   第5集 
        贝塔的炮弹打伤了小麻雀; 
        舒克的直升机营救小麻雀   
        贝塔开着坦克来到野外,天黑得伸手不见五指,什么也看不见,他打开了照明灯。 
        通过潜望镜,贝塔看见四周都是灌木丛,前方有一堆小石子。 
        “拿花生米当炮弹太可惜,”贝塔想,  “不如用小石子当炮弹。” 
        贝塔对屋外这个世界还很陌生,由丁他一生下来就在惊恐中生活,养成了谨小慎微的习惯。这次如果没有坦克给他壮胆,他是无论如何也不敢跑到外面来的。贝塔决定把炮弹储备得足足的,以防万一。 
        贝塔把坦克停在石子堆旁边,听听四周没有动静,他轻轻打开舱盖儿,钻出来,将许多小石子运进坦克。有这么多炮弹,贝塔心里踏实多了。 
        贝塔忙完后,吃了两颗花生米,躺在坦克里他的软床上,唾着了。 
        一阵吵闹声惊醒了贝塔,他趴在潜望镜上一看,天已亮了,一群麻雀落在他的坦克上,正叽叽喳喳地议论着。 
        “这是什么?昨天还没有呢!” 
        “可不是吗,怎么一动不动呀?” 
        “是个死东西吧?” 
        “讨厌!”贝塔决定吓唬他们一下。他悄悄发动了坦克,猛然向前一冲,吓得麻雀们都飞了起来。 
        贝塔得意极了。他操纵坦克掉回头来,通过潜望镜看着落在树枝上的麻雀们。 
        “这是乌龟吧?”一只小麻雀说。 
        贝塔觉得“乌龟”是骂人的话,咪丽就这样骂过他。他要教训这只小麻雀一下。 
        贝塔把炮口对准了小麻雀,装上石子炮弹,一按电钮,只听“啪!”的一声,小麻雀掉在地上一蹦一蹦的,贝塔的炮弹打中了他的翅膀。 
        贝塔清楚地看见小麻雀的翅膀在滴血,他原以为打小麻雀也像打咪丽一样,不会打伤。没想到小麻雀这么娇气,再加上炮弹由花生米换成了石子。 
        贝塔挺后悔,他把坦克开到小麻雀身旁.可又不敢走出坦克。 
        贝塔这一炮可把麻雀们吓坏了,他们眼巴巴地看着小麻雀在地上挣扎,眼睁睁地看着坦克朝麻雀开过来,干着急没办法。 
        “对了,快去叫舒克!”一只麻雀忽然想到了舒克。 
        舒克神通广大,在这一带已出了名。 
        舒克正在擦他的直升机,一只麻雀气喘吁叮地飞过来,差点儿撞在飞机上。 
        “舒克,快去,不好了……小麻雀被……被一个怪物……打断了……翅膀……” 
        “啊?!”小麻雀是舒克的好朋友,舒克曾经救过他,他也救过舒克。舒克一听说小麻雀遇到不幸,急得直跺脚。 
        “快上飞机!”舒克和那只麻雀钻进直升机。不到5秒钟,直升机便腾空而起,以最快的速度朝出事的地方飞去。 
        正当贝塔犹豫着是不是应该出去给小麻雀道歉时,忽然听见天上传来一阵发动机的声音。他往上一看,是一架直升机。贝塔在电视里见过这玩意儿,似乎也挺厉害。 
        舒克操纵飞机下降,看清了,那怪物是一辆坦克。舒克决定先把小麻雀救出去再收拾那坏蛋坦克。 
        直升机在坦克上空盘旋,贝塔弄不清它要干什么。只见飞机下边伸出来一根绳子,飞机上的麻雀喳喳地叫着,受伤的小麻雀抓住绳子头儿,被救上去了。 
        贝塔心里挺不是滋味儿,他有点儿恨那架直升机,说不清为什么。   第6集 
        舒克的直升机和贝塔的坦克之间展开一场大战   
        正当贝塔准备开着坦克离开这块是非之地时,坦克猛烈地晃动了一下,他的头重重地撞在炮膛上,起了一个大包。 
        只听一阵飞机轰鸣声由近而远。 
        当贝塔还没明白过来是怎么回事时,坦克又一次震动,贝塔的头也就又撞了一次炮膛,两个大包了。 
        一阵飞机轰鸣声由近而远。 
        贝塔清醒了,他一面捂着脑袋一面往外看,原来是那架米黄色的直升机故意使劲地往贝塔的坦克身上落。贝塔火了。他找出坦克帽戴在头上,这样就不怕撞了。他把坦克发动起来,停在原地不动,等着直升机再一次往下压他的坦克。 
        舒克的直升机第三次降下来压贝塔的坦克,就在直升机的轮子刚要撞着坦克时,贝塔操纵坦克躲开了,舒克的直升机控制不住,撞在地上,把地撞了一个坑。 
        贝塔操纵坦克来了个一百八十度的转弯,全速朝舒克的直升机撞过来。 
        舒克毕竟是有丰富经验的飞行员,就在坦克要撞上飞机的一刹那,直升机拉起来了,而贝塔的坦克刹不住车,撞在一棵树上,把树撞倒了。幸亏贝塔戴着坦克帽,要不然头上又该多一个大包了。 
        这次贝塔可真生气了,他瞄准悬在前方空中的直升机就是一炮,直升机被打穿了一个小窟窿。舒克害怕了,连忙把飞机拉得高高的。 
        “这家伙真坏,仗着自己有武器就欺负人。”舒克看看躺在机舱里受伤的小麻雀,心想,一定要治 治这个开坦克的坏蛋。 
        舒克开着直升机离开了贝塔的坦克,他到河边装石头去了。贝塔以为自己把直升机打跑了,很得意。 
        “嗵!”舒克从天上往下扔石头,就像飞机扔炸弹一样。石头砸在坦克上,几乎砸穿了车身。 
        贝塔开着坦克就跑,舒克驾驶着直升机在天上追,边追边扔石头,可是,不是扔早了就是扔晚了,再加上贝塔一会儿开快,一会儿开慢,老砸不着。 
        贝塔看见前方有一片小树林,他想出了一个主意,贝塔操纵坦克用最大速度朝小树林驶去,舒克在空中紧追。 
        贝塔的坦克钻进了小树林,舒克的直升机也在小树林中穿行。贝塔的坦克一会儿往左拐,一会儿往右拐。终于,舒克的直升机被挂在树上了。 
        这下贝塔可得意了,他往炮膛里装了一颗大炮弹,瞄准了直升机的驾驶舱,但是贝塔的手没有按电钮,他也不知为什么。 
        舒克清清楚楚地看见坦克的炮口对着自己,他一点儿办法也没有,只好闭上眼睛等着坦克开炮。 
        就在这时,几十只麻雀飞来落在挂着直升机的树枝上,他们一起使劲儿摇树枝,直升机掉下来了,就在接地的一刹那,飞机的螺旋桨起动了,直升机拔地而起。 
        “这家伙技术不错。”贝塔不得不承认。 
        直升机飞走了,大战宣告结束,谁也没赢,谁也没输。贝塔真没想到,外面的世界这么复杂,刚出来就打了一仗,总算还平安。贝塔有点儿累,他检查了一遍舱盖儿确实锁牢了,就躺在他的软床里睡觉了。   第7集 
        贝塔的坦克飞到了天上; 
        坦克和飞机在空中同老鹰展开空战   
        贝塔睡得迷迷糊糊的,忽然觉得身体摇晃起来,他睁开眼睛一看,自己还在坦克里。是做梦吗?贝塔爬起来,趴在潜望镜上往外一看,差点儿叫出声来——他的坦克飞起来了,旁边是蓝天,下边是大地。 
        贝塔揉揉眼睛,没错,他的坦克上天了!这是怎么回事? 
        贝塔发动坦克,没用,车轮只能空转。 
        贝塔忽然听见头顶上有飞机的声音,他把炮塔上的舱盖儿打开一条小缝儿,往上一看,那架直升机用绳子把他的坦克吊起来了。 
        原来,舒克和小麻雀们昨天晚上商量了一下,觉得这辆来历不明的坦克严重危害大家的安全,舒克想了这么个办法,用他的直升机把坦克吊到很远的地方驱逐出境。于是,趁贝塔睡觉的工夫,大家悄悄用绳子把贝塔的坦克捆了起来。天一亮,舒克 就驾驶着直升机把贝塔和他的坦克一起吊到了空中。 
        现在,舒克正吊着贝塔的坦克用最高速度朝西北方向飞去,他也不知道把坦克运到哪儿去,反正越远越好。 
        贝塔急了,他往炮膛里塞了一颗炮弹,可他的炮口不能往上抬九十度射击。贝塔按住炮塔旋转按钮不放,他的炮塔发疯一样地旋转起来,可绳子是捆在炮塔上的,一点儿用也没有。贝塔的头都转晕了,他一松开按钮.拧成麻花的绳子又往回转,整个坦克也跟着往回转,转得他都快吐了。 
        忽然,贝塔觉得上面往下滴水,他还以为下雨了,挺高兴,贝塔一天没喝水了。他立刻把嘴接在滴水的地方,很臊,不是雨,是尿。原来舒克怕绳子不结实,急中生智想出了这个办法,把绳子弄湿了不就不会断了吗?贝塔知道又上当了,可他干着急,干生气,一点儿办法也没有。 
        舒克觉得飞得够远的了,即使这辆坦克日夜兼程往回开,也得开上三天三夜。他准备找一个合适的地方着陆。 
        就在这时,舒克发现在他的直升机上方出现了一个黑点儿。那黑点儿越来越大,舒克觉得脖子后面有点儿发凉。他看清楚了,那是一只老鹰! 
        老鹰的眼睛最尖,他一眼就看清楚直升机里的舒克是他喜欢吃的食物,他收拢翅膀,飞快地俯冲下来。 
        贝塔虽然没看见老鹰,但他也本能地预感到有危险要降临,他的脖子后面也阴森森的。 
        就在老鹰扑过来的一瞬间,舒克拉起了直升机,老鹰扑了个空。 
        贝塔从潜望镜里看清楚了,是一只老鹰!老鹰转过身子,又扑过来。 
        贝塔的炮里正好有一发炮弹,他瞄准老鹰开炮,没打中。在空中射击非常困难,双方都在运动中,很难打中。 
        舒克原打算立即把坦克扔下去,吊着它非常不灵活,很难躲过老鹰的袭击。当舒克发现吊在下面的坦克冲老鹰开炮后,他马上把这个敌人当成了自己的同盟军。 
        老鹰又扑过来了。 
        贝塔装上一发炮弹,瞄准了目标。 
        近些,再近些!贝塔的手直哆嗦,他看清了老鹰那带勾的嘴和刀子一样锋利的爪子。就在老鹰的爪子刚要抓住吊坦克的绳子时,贝塔按下了射击按钮。 
        打中了!老鹰掉了下去.但马上又飞了起来,老鹰毕竟是老鹰,不像麻雀那样娇气。 
        老鹰没想到对手还有武器,他同直升机保持了一段距离,在想对策。 
        “这家伙还真有两下子!”舒克到现在还不知道坦克里是谁,不过他已经喜欢上他了。舒克忽然想起直升机里的电台,他戴好耳机,对着话筒喊起来。 
        贝塔的坦克里也有电台,贝塔不知道它的用处。现在电台里传出了声音,贝塔觉得挺好玩。 
        “喂!喂!喂!”舒克呼叫。 
        “干吗?干吗?” 
        “我是舒克!” 
        “什么舒克?” 
        “飞行员舒克。” 
        “啊?就是你把我吊到天上来的!” 
        “真对不起,你是谁?” 
        “不告诉你。” 
         贝塔不愿意让人家知道他的身分,他觉得这个世界上谁都可以欺负他。 
        “谢谢你开炮打跑了老鹰。” 
        “这算什/厶,我还没使劲儿打呢!” 
        “咱们交个朋友吧!” 
        “你先把我放到地上去。” 
        “不好,老鹰又来了!” 
        贝塔一看,狡猾的老鹰从下面往上飞扑过来。 
        贝塔的炮打不着他。 
        “降低高度!”贝塔命令。 
        “明白!”舒克操纵飞机急速下降。 
        贝塔的炮口瞄准了老鹰。 
        “这次你使劲儿按炮钮。”舒克说。 
        “少废话!” 
        义打中了!看来这次老鹰疼得够呛,挣扎着飞走了。 
        舒克和贝塔胜利了。   第8集 
        舒克的直升机坏了; 
        舒克和贝塔降落在一个陌生的地方; 
        他俩终于见面了   
        舒克现在对吊在下面的这个盟军佩服得五体投地。 
        “你真行!” 
        “少来这套。”贝塔想起舒克把他吊到天上来就有气,连招呼也不打! 
        “我现在就送你回去。”舒克想起自己把人家吊到天上来,心里挺过意不去。他开始操纵直升机返航。 
        “这还差不多。”贝塔往嘴里塞了一颗花生米,“你刚才往下撒尿了?” 
        “是,我怕吊坦克的绳子太干燥会断。” 
        “我以为是往下淋水呢,就伸嘴接着,上你的当了。” 
        “实在抱歉。”舒克说完觉得发动机的声音有些不正常,糟糕,飞机出故障了! 
        “注意,飞机出故障,马上要迫降!”舒克通知贝塔。 
        “什么破飞机,白给我都不要。”贝塔嘴上这么说,心里挺害怕。他知道,飞机要是掉下去,他就没命了。 
        舒克发现地面上有一座城堡,他想操纵飞机绕过这座城堡,追降在野地里。可飞机已经不听他的指挥了,一个劲儿往下掉。舒克没办法,只好在城堡里着陆。 
        如果舒克知道这是一座什么城堡的话,那他宁愿摔死也不敢在这里降落。这是一座猫城——克里斯王国。克里斯王国的所有公民都是猫。 
        舒克总算平安地把飞机降落在一块开阔地上,坦克先着陆,直升机落在一旁。 
        贝塔从潜望镜里看着停在旁边的直升机,他想看看这个同盟军是什么模样。 
        直升机的舱门打开了,贝塔不相信自己的眼睛。怎么?飞行员舒克也和他贝塔一样,是老鼠?! 
        舒克跳下飞机,把悬吊贝塔坦克的绳子解开收好。 
        舒克正准备修理飞机,他突然呆在那里,一动不动。舒克和贝塔几乎是同时看见远处有三只穿着军装的猫。 
        “快进来!”贝塔打开坦克舱盖儿冲舒克喊。 
        舒克想了一下,觉得坦克比飞机安全,因为飞机已经不能飞了。 
        舒克急忙钻进贝塔的坦克。贝塔把盖儿盖紧,锁牢。 
        “怎么,你也是…”舒克也没想到这个盟军竟是自己的同胞。 
        “我叫贝塔。”能见到自己的同胞,贝塔很高兴。 
        “我是舒克。”舒克和贝塔握手。 
        “你看。”贝塔看见那三只猫士兵朝坦克走过来。 
        “别怕。”舒克安慰贝塔,其实他的心跳得特快。   第9集 
        舒克不让贝塔开炮打猫宪兵; 
        贝塔驾驶坦克甩掉猫宪兵: 
        舒克贝塔被包围   
        三只穿军装的猫宪兵朝舒克和贝塔走过来,贝塔把坦克舱盖儿锁牢。他俩的心脏发出嗵嗵嗵的响声,吓得连大气也不敢出。 
        贝塔趴在潜望镜上,看见猫宪兵越走越近。 
        “你的飞机怎么维修的?够呛!”贝塔小声埋怨舒克。他太怕猫了。 
        “从来没出过故障,准是你的坦克太重了!”舒克把责任推到贝塔身上。 
        “我又没请你把我的坦克吊到天上!”贝塔生气了。 
        “算了算了。”舒克觉得现在不是吵嘴的时候,“快看看,他们要干什么?” 
        通过潜望镜,贝塔看见三只猫宪兵站在坦克前面,他们好奇地看着坦克,其中一只猫还摸摸坦克的履带。另一只猫宪兵朝舒克的直升飞机走过去。 
        贝塔往炮膛里装了一发炮弹,瞄准了一只猫。 
        “别打!”舒克小声说。 
        “干吗?”贝塔不明白。 
        “现在他们不知道坦克里边是什么东西,你一开炮,他们该报复咱们了。再说,咱们还不知道这是座什么城堡,看样子想逃出去不大容易,还是让他们弄不清咱们的底细安全些。” 
        贝塔觉得舒克说得挺有道理,他没开炮。 
        “那咱们也不能老呆在这儿呀!,贝塔实在害怕这三只猫。 
        “咱们去别处看看,先躲开猫再说。”舒克也怕猫。那次他去赴蚂蚁皇后的宴会时,差点儿被猫吃了的经历一直没忘,想起来身上就发抖。 
        这时,一只猫爬上了坦克。 
        贝塔按了起动按钮,坦克猛然向前开去,把那只猫甩到地上。 
        三只猫宪兵定了定神儿,跟在坦克后面追上来。 
        “快,再快点儿!”舒克催贝塔。 
        贝塔已将速度按钮按到底了,坦克呼啸着朝前驶去。 
        “你的飞机不要了?”贝塔边开边问。 
        “先把这些猫引开,一会儿回来修。” 
        “这家伙挺鬼!”贝塔想。他不得不承认舒克点子多。 
        舒克看见坦克里有花生米,拿起一颗。 
        “可以吗?”舒克一边往嘴里送一边问。 
        贝塔点点头。舒克大口大口吃起来。 
        绕过两座房子,坦克来到街上。 
        潜望镜里的情景使贝塔大吃一惊,他操纵坦克来了个急刹车。舒克的头重重地撞在舱壁上。 
        “你干什么?”舒克火了,  “刹车也不告诉一声!” 
        “你看!”贝塔离开潜望镜.让舒克看。 
        舒克趴在潜望镜上一看,心脏几乎停止了跳动:街上到处都是猫。 
        “倒车!”舒克忙说。 
        坦克掉过头,朝相反的方向开去。没开多远, 
        又是一个急刹车。 
        舒克和贝塔终于明白了,这是一座猫城。 
        这时,街上的猫都被这个新奇的玩艺儿吸引住了,潮水般朝坦克围过来。 
        舒克和贝塔无路可逃。   第10集 
        贝塔用炮塔把猫公民们吓跑; 
        猫宪兵把坦克翻了个底朝天; 
        坦克变成了潜水艇; 
        舒克和贝塔被活捉       
        “反正他们打不开坦克的舱盖。”贝塔自己给自己壮胆。其实他的腿直哆嗦。 
        “就是,别看他们是猫,根本治不住坦克。”舒克也一边发抖一边给自己鼓劲儿。 
        贝塔又检查了遍舱盖,确实锁牢了。 
        这时,几百只猫把坦克围得水泄不通。 
        他们的议论声传进了坦克里。 
        “这是什么东西?” 
        “不知道。” 
        “没见过。” 
        “从哪儿来的?” 
        “昕说是从天上掉下来的。” 
        “天上?” 
        ”能打开吗?” 
        “试试。” 
        于是就传来了猫爪子抓坦克舱盖的声音。 
        舒克和贝塔紧紧靠在一起,眼睛死盯着舱盖。 
        舒克想起了小花猫。贝塔想起了咪丽。他俩几乎是同时蹦起来。 
        “咱们不能等死!”舒克说。 
        “就是,拼拼看!”贝塔立即支持。 
        贝塔坐到驾驶座上。舒克坐在炮手的位置上。他们系好安全带。 
        “我转炮塔,吓他们一下。”贝塔让舒克作好准备。 
        “转吧!”舒克说。 
        “你不怕晕吧?”贝塔问。 
        “飞行员还有怕晕的?”舒克觉得贝塔太小看他了。 
        “那我就转了。”贝塔按下炮塔旋转按钮。 
        坦克上的炮塔飞快地旋转起来,炮管把好几只猫撞出去老远。 
        克里斯王国的公民们吓坏了,他们扭头就跑,边跑边发出尖叫声。他们弄不清这是什么怪物。 
        猫们发现怪物没追上来,才停住,他们站在老 远的地方胆怯地望着坦克。 
        “他们怕我的坦克!”贝塔兴奋了。 
        “他们怕咱们!”舒克也来劲儿了。 
        原来猫也有害怕的时候,舒克和贝塔决定治治这些老鼠的冤家。 
        “用坦克撞他们!”舒克提议。 
        “行!”贝塔发动了坦克,一推操纵杆,坦克的履带飞快地转起来。 
        “坦克怎么不动了?”贝塔从潜望镜往外一看,坦克纹丝不动,可履带却在转。 
        舒克凑过去一看,慌了。两只猫宪兵把坦克抬起来了,坦克的轮子在空转。 
        紧接着,舒克和贝塔觉得天旋地转,他俩头朝下了!要不是系着安全带,非得重重地撞一下头不可。 
        猫宪兵把坦克翻过来了。坦克轮子朝天,任贝 塔怎么加大速度,轮子只能空转。 
        贝塔和舒克傻眼了。 
        看到怪物被治住了,吓跑的猫公民们又慢慢围拢过来,但他们作好了随时跑的准备。 
        “这回安全了,舱盖想打也打不开了。”贝塔说,他头朝下吊着。 
        “我的头有点儿受不了了。”舒克也是头朝下,他的脸憋紫了。 
        “飞行员还怕头朝下?”贝塔撇撇嘴。 
        舒克不吭气了。他原想解开安全带,把身子正过来,可又怕贝塔笑话飞行员还不如坦克兵。舒克只好忍着。 
        其实贝塔也快不行了,但他下决心一定要坚持到舒克忍不住为止,煞煞他那飞行员的优越感。 
        “把这怪物扔到池塘里去吧,放在大街上太危险。”一只猫提议。 
        所有的猫都赞成这个办法。 
        舒克和贝塔的心本来就快到嗓子眼儿了——现在他俩的心差点儿从嘴里掉出来。 
        坦克被翻过来了,猫公民们抬着坦克朝池塘走去。 
        舒克和贝塔慌了。 
        “你这坦克漏水吗?”舒克问。 
        “又不是船,当然漏。”贝塔说。 
        “咱们要是鱼就好了。” 
        “我可不愿意当鱼。” 
        “怎么?” 
        “当鱼还得让猫吃!” 
        “真是的。那咱们要是乌龟就好了。” 
        “少废话,想点儿办法吧!” 
        贝塔把床上的棉花拿起来,见缝儿就塞。舒克也学着贝塔的样子堵缝儿。   
        通过潜望镜,贝塔看见他们的坦克已被抬到池塘边上。 
        “一、二、三!”猫们一起使劲儿,只听“扑通”一声,坦克被扔进池塘里。 
        还好,坦克里边没进水!舒克和贝塔只觉得气短,呼吸越来越急促。 
        “糟糕,坦克里快没空气了。”贝塔说。 
        “把炮管抬起来,说不定能伸出水面。”舒克灵机一动。 
        贝塔按电钮操纵炮管往上抬,炮管果真伸出了水面。 
        “快,把炮弹退出来!”舒克说。 
        贝塔把炮弹从炮膛里退出来,然后把嘴对在炮膛上,有空气了!坦克变成了潜水艇。 
        “你来!”贝塔让给舒克。 
        舒克只吸了一口,又让给贝塔。 
        “你多吸两口!”贝塔说。 
        “我是飞行员,体质好。”舒克说。 
        又来了,贝塔最讨厌舒克跟他摆飞行员架子。 
        “我的体质也不差!”贝塔赌气,不吸。 
        舒克呼吸越来越急促。 
        贝塔也快挺不住了。可谁也不去吸空气。不过,他俩的头不由自主地离炮膛越来越近。 
        贝塔忽然觉得脚有点儿凉,他低头一看,坦克漏水了。贝塔忙拿棉花去堵漏洞。舒克也帮着堵。 
        漏进来的水越来越多,已经没到舒克和贝塔的胸部了,水位还在上涨。 
        “等舱里的水满了,咱们也就完了。”舒克耸耸肩膀。 
        “好在你是飞行员,体质好,不怕。”贝塔冲舒克挤挤眼睛。 
        “当然。不过……”舒克喝了一口水。水已经到他下巴了。 
        “出去吧?”贝塔问。 
        “当然。等着淹死不如出去碰碰运气。”舒克脱下套在飞行服外面的救生衣,递给贝塔,“你穿吧,这是救生衣,能浮在水面上。” 
        “不要,我会游泳。”贝塔摇摇头。 
        “穿上吧!我……体质好。”舒克没敢再提飞行员。 
        贝塔穿上了救生衣。冲舒克笑笑。 
        这时,坦克舱里的水已经快满了。贝塔恋恋不舍地看看自己心爱的坦克,打开了舱盖。 
        舒克和贝塔钻出坦克,向上游去,把头露出水面。 
        池塘四周都是看热闹的猫。他们看见舒克和贝塔,叫起来,几只猫跑去拿来打鱼的大网,把舒克和贝塔罩住了。   第11集 
        克里斯王国的猫没见过老鼠; 
        猫公民帮助舒克和贝塔打捞坦克; 
        飞行表演   
        舒克和贝塔被克里斯王国的猫公民们用渔网从池塘里捞了上来。 
        舒克和贝塔明白,他俩的末日到了。 
        “再见了,飞行员舒克。”贝塔在渔网里递给舒克一个勉强的笑容。 
        “再见,坦克兵贝塔。”舒克耸耸鼻子。 
        “我真想让你再把我吊到天上。”贝塔说。 
        “我也想让你用炮再打我一次。”舒克说,“说实话,你的炮打得不赖。” 
        “说实话,你的体质还真不错,不愧是飞行员。”贝塔说。 
        渔网被拖上岸了。舒克和贝塔被猫们从渔网里拽出来。他们闭上眼睛,等着猫吃他们。, 
        “这么多猫,还不把我和贝塔撕成碎片啊!”舒克想。 
        一分钟过去了。两分钟过去了。十分钟过去了,怪事,这些猫干什么哪? 
        舒克和贝塔睁开眼睛,他俩身边全是猫。这些猫好奇地看着舒克和贝塔。 
        贝塔和舒克纳闷了,他们怎么还不吃,看什么?难道还有不认识老鼠的猫? 
        “请吃吧!”舒克实在受不了这种侮辱,他要维护一个飞行员的尊严,“还等什么?” 
        “吃什么?”一只猫问。 
        “吃我们呀!”贝塔说。真怪,他现在反而一点儿也不怕了。 
        “吃你们?你们能吃?”猫公民们惊讶极了。 
        猫不知道老鼠能吃! 
        “你们听说过老鼠吗?”舒克心里一动,他问猫公民们。 
        “老鼠?”猫公民们摇摇头。 
        原来克里斯王国的猫没见过老鼠,他们这儿从来没有过老鼠!他们不知道猫和老鼠是冤家。 
        “你们是谁?”一只猫宪兵问。 
        “我是飞行员舒克。他是坦克兵贝塔。” 
        “干吗到我们克里斯王国来?” 
        舒克把他和贝塔怎么在天上遇到老鹰,飞机怎么出了故障,以及他们怎么在克里斯王国迫降等等,统统讲了一遍。 
        “飞机真能飞到天上去?”猫们不信。 
        “池塘里那怪物是坦克?”猫公民们感兴趣了。 
        他们决定帮舒克和贝塔把坦克从池塘里捞上来。他们有的去捞坦克,有的给贝塔和舒克端水喝。 
        在舒克和贝塔心目中,猫是凶恶的化身。他们万万没想到,猫还会笑着说话,还会这么热情。 
        坦克捞上来了,贝塔和舒克在猫公民的帮助下,把坦克车身上的水擦干净。 
        虽然同猫在一起干括舒克和贝塔还不大适应,但他俩觉得这些猫仗义。 
        “真对不起!”一只猫说。 
        他大概是世界上第一只给老鼠道歉的猫。舒克和贝塔激动得不知说什么好。 
        “要是世界上的猫都像他这样就好了。”贝塔想。 
        “没关系。”舒克说。 
        坦克擦干净了,贝塔钻进去,试试车,发动机正常。 
        “给我们表演一下行吗?”一只猫提议。 
        “当然可以!”贝塔很乐意显示一下自己高超的驾驶技术。他招呼舒克进来。 
        舒克也钻进坦克,坐在贝塔旁边。 
        贝塔猛一按启动电钮,坦克风驰电掣般地围着猫公民们绕开了圈子。猫公民们的身体跟着坦克转,他们的头都快转晕了。贝塔的坦克开得快极了,几乎只能看见一道光圈围着猫公民们转。 
        “开一炮,让他们看看!”舒克说,他从炮弹箱里取出一发炮弹,递给贝塔。 
        贝塔把坦克停住.将炮弹塞进炮膛,瞄准一棵大树上的树叶,按下了射击按钮。 
        树叶被打掉了好几片。 
        猫公民们欢呼起来。 
        “我也能打落树叶。”舒克觉得树上那么多树叶,随便射击都能打掉几片。 
        “我是瞄准这几片树叶开炮的!”贝塔吹牛。 
        舒克撇嘴。 
        “能让我们看看飞机上天吗?”等坦克停卜来,一只猫趴在坦克上问舒克和贝塔。 
        “当然可以!”舒克来幼儿了,“我先去把飞机修好。” 
        克里斯王国的猫公民们抬起贝塔的坦克,浩浩荡荡地朝舒克的直升机走去。贝塔和舒克站在坦克的炮塔上,神气极了。 
        直升机的发动机出了故障,舒克对他的直升机挺熟悉,很快就修好了。 
        猫公民们把舒克的直升机和贝塔的坦克围得水泄不通。他们很喜欢这两位身怀绝技的飞行员和坦克兵。 
        舒克钻进直升机的驾驶舱。 
        “上来吧?”舒克邀请贝塔。 
        “你自己去吧,我在下边等你。”贝塔对舒克的直升机的安全性表示怀疑。 
        直升机的螺旋桨转起来了。紧接着,直升机腾空而起。 
        猫公民们先是愣了一下,接着一阵欢呼。 
        舒克驾驶着直升机在克里斯王国上空作着各种高难度的飞行动作。一会儿空中悬停,一会儿急转弯,一会儿垂直起落…… 
        猫公民们看花了眼,赞叹声连成一片。 
        贝塔把头伸出炮塔,他不得不承认舒克的飞行技术是第一流的。贝塔想,他的坦克如果和舒克的直升机联合起来,就谁也不怕了。 
        “贝塔!贝塔!我是舒克!你听见了吗?”贝塔的耳机里传来舒克的呼叫声。 
        “我是贝塔!我是贝塔!”贝塔兴奋极了,望着天上的直升机答应着。 
        “我现在要给他们作一次超低空飞行表演,请你在地而指挥!请你在地面指挥!明白吗?” 
        “明白!你说话小点儿声,把我耳朵都快震聋了!” 
        “开始吧?”舒克请示地面。 
        贝塔看见舒克的直升机盘旋了一圈儿,悬停在空中,做好了超低空飞行的准备。 
        “开始!”贝塔命令,“降低高度!” 
        舒克的直升机一边降低高度一边朝猫公民们飞来。 
        “再低些!再低些!”贝塔指挥着。 
        “明白!”舒克回答。 
        直升机几乎是擦着地面飞过来,吓得观看表演的猫公民们都趴在地上。 
        “拉起来点儿!”贝塔全神贯注地观看着直升机与地面的距离。他知道,只要稍一疏忽,就会机毁鼠亡。 
        直升机擦着猫公民们的头飞过去了,螺旋桨掀起的强大的风吹得猫们站不住脚。 
        克里斯王国的公民们大饱了眼福,他们对舒克和贝塔佩服得五体投地。   第12集 
        克里斯王国的国王要接见飞行员和坦克兵; 
        舒克和贝塔听说国王会放电; 
        舒克和贝塔见到克里斯国王   
        舒克驾驶直升机安全着陆。 
        猫公民们拥到飞机旁边,把舒克和贝塔抬起来,抛向空中,然后接住,又抛起来……当舒克和贝塔被抬到一片绿色草坪上的时候,迎接他们的是丰盛的宴会。猫公民们纷纷从自己家里把最好吃的食物拿来。舒克和贝塔已经饿坏了,他们准备饱餐一顿。 
        再说,猫宴请老鼠,这意义也不一般。 
        这时,走来一队猫宪兵。 
        “我们的国王听说你们来了,要接见你们。”宪兵对舒克和贝塔说。 
        “国王!”贝塔一愣,看着舒克。 
        舒克也觉得国王一定是见多识广的老猫,他肯定认识老鼠。 
        “我们吃完饭去,行吗?”舒克说,他想来个缓兵之计.等吃完饭,他和贝塔立即飞走。 
        “可以,请快点儿吃。”猫宪兵站在远处等着。 
        舒克悄悄地把他的计划告诉了贝塔,他动员贝塔把坦克丢掉,跟他一起坐直升机先跑,过几天再悄悄回来把坦克吊走。贝塔同意了。 
        “你们的国王很老吗?”舒克边吃边问猫公民。 
        “没见过。”坐在舒克身边的一只猫说。 
        “没见过?”贝塔不信。 
        所有的猫都摇头。 
        “国王厉害吗?”舒克问。 
        一提起国王,猫公民们的脸都吓白了。舒克和贝塔看出,克里斯王国的国王一定很凶。 
        “国王有法术,会放电。”一只猫小声告诉舒克。 
        “会放电?”舒克大吃一惊。 
        “咱们快走吧!”贝塔催舒克。 
        舒克觉得应该和猫公民们打个招呼,不然太不够意思了。 
        “多谢大家,我们走了。”舒克站起来和猫公民们告别。 
        一听说舒克和贝塔不见国王就走,猫公民们吓坏了。 
        “你们要足走了,我们可就没命了。”一只猫说。 
        “国王该发脾气了!”另一只猫一边打哆嗦一边说。 
        舒克和贝塔愣住了。 
        “还走吗?”舒克问贝塔。 
        贝塔耸耸肩膀,坐下了。 
        舒克也坐下了。 
        猫公民们感激地看着舒克和贝塔。 
        舒克和贝塔小声商量了一会儿,决定开着坦克和飞机去王宫,见机行事。 
        贝塔钻进坦克,在猫宪兵的指引下,向王宫驶去。舒克的直升机在空中跟着。 
        克里斯王国的王宫很漂亮,是大理石建筑。王宫前面有一座广场。 
        贝塔的坦克停在王宫门前的台阶下边。舒克的直升机在坦克旁着陆。 
        舒克和贝塔跟着猫宪兵走进王宫。他俩数着,一共经过了37道岗! 
        “跑不成了。”舒克小声说。 
        “这个国王一定是坏蛋!”贝塔说。 
        “怎么?” 
        “设这么多岗,怕别人看见他干坏事!” 
        “就是,岗越多,干得坏事越多,越心虚。”舒克同意。 
        国王宣召舒克和贝塔进殿。 
        舒克和贝塔硬着头皮走进去,他俩看见国王后大吃一惊:国王是一只老鼠!一只白老鼠!! 
        猫国的国王是老鼠!!! 
        舒克和贝塔高兴了。 
        总算来到了一座大殿门口,宪兵示意舒克和贝塔等一会儿。   第13集 
        克里斯王国国王的来历; 
        国王设宴招待舒克和贝塔; 
        舒克和贝塔拒绝吃猫肉 
        一年以前,一只名叫白路的供医学试验用的小白鼠,当医务人员在他的身上移植了老虎胆和人工心脏后,他利用医务人员的疏忽,逃出了医院。 
        可想而知,这只装着老虎胆和人工心脏的小白鼠来到外界后,围绕着他,一定会发生一系列极为有趣的事件。 
        下面是他出逃后的经历。 
        白路首先遇到一只大猫。 
        这么干净的白老鼠,大猫还是头一次见到。他咂咂嘴.朝白路逼过来。 
        “你不怕我电死你?”白路说,他原地不动,像没事儿一样。 
        “电?”大猫站住了。他怕电。 
        白路按了一下胸脯,人工心脏上的红灯一闪一闪地亮起来。 
        大猫被吓住了,这小子身上还真有电! 
        “怎么样,想尝尝电的滋味儿吗?”白路朝大猫走过去。 
        大猫连连倒退着。 
        “不,不,哪儿的事呀!”大猫扭头就跑。 
        “站住!再跑我就放电啦!”白路吓唬他。 
        大猫站住了。 
        “跟我走。”白路说完转身就走,连头也不回。 
        大猫断定白路身后有电眼,只好老老实实地在后边跟着。 
        从此,白路就把这一带的猫都镇住了。猫们每天向他进贡食物。他们都怕电,他们不知道电是什么东西,只听说厉害。 
        一天,白路听说几十里外有一个克里斯王国,王国里有几千只猫,于是,他决定去克里斯王国当国王,享享福。 
        白路轻而易举地就把克里斯王国的国王赶下了台。一是因为克里斯王国的猫公民从来没有见过老鼠;二是因为他们也怕电,怕得要命。 
        白路当上了克里斯王国的国王,只有少数几个大臣可以见到他。白路命令王国的公民们为他修建了豪华的宫殿,猫公民们都怕国王放电,只好老老实实地侍候国王,又怕又恨。 
        其实,只要有一只猫公民稍微试一下,就能知道国王身上只不过装着一节电池,根本没有杀伤力。可是谁也不敢试,而且越传越神,越神越怕,越怕越老实。 
        再说舒克和贝塔。当他们发现克里斯王国的国王是他们的同胞后,放心了,他们觉得不会有危险了。 
        白路国王万万没想到召见的飞行员和坦克兵是两只老鼠——他的同胞!白路国王心里一惊,他怕舒克和贝塔把他的底细泄露出去。再说,看到两只老鼠穿着飞行服和坦克装,他心里也有点儿发颤。 
        国王眼珠一转,想出计策,他要害死舒克和贝塔。 
        国王喝退了大臣们。大殿里只剩下他、舒克和贝塔。 
        “你怎么能当上猫国的国王?”舒克亲热地问。在猫国里碰见老鼠国王,他感到很开心,一点儿戒心也没有了。 
        “一下还说不清。”国王也亲热地说,“你们怎么当上飞行员和坦克兵的?” 
        舒克和贝塔把经过告诉国王。 
        “你们是咱们老鼠家族的英雄。”国王竖起大拇指,“我宴请你们。” 
        “我们刚刚吃过饭。”舒克说。 
        “那也得吃。”国王说。 
        盛情难却,舒克和贝塔只得从命。 
        国王命令侍从去准备宴席,他悄悄吩咐部下在舒克和贝塔的碗里下毒药。 
        舒克和贝塔很感激国王,他们觉得老鼠当国王一定比猫心地善良,老鼠受的苦太多了。 
        在国王的陪同下,舒克和贝塔来到宴会大厅。高大宽阔的宴会厅到处是鲜花,宽大的餐桌上摆满桌丰盛的饭菜。舒克和贝塔眼睛都看花了。 
        “请入席。”国王说。 
        舒克和贝塔挨着国王坐下。 
        “这是红烧猫肉。这是清炖猫肉。这是炒猫肝儿。”国王给客人介绍着。 
        正准备进餐的舒克和贝塔停住了,怎么?这一桌子都是猫肉?国王吃自己臣民的肉? 
        “吃牙!”国王急了,他怕舒克利贝塔不吃,饭菜里有毒药。 
        “你天天吃猫肉?”舒克问。 
        “是的。猫肉很好吃,别客气,快吃!”国王催促道。 
        舒克和贝塔想起了猫公民们害怕国王的情景。他们万万没想到,老鼠当了国王,比猫更残忍。 
        “你怎么能吃猫肉呢?”贝塔火了。生来怕猫恨猫的贝塔,居然替猫说话了。 
        “猫怎么能吃老鼠肉呢?”国王反问。 
        “这……”舒克和贝塔答不上来,反正他们觉得国王吃自己的臣民不对。 
        看见舒克和贝塔不吃,国王急了。   第14集 
        舒克和贝塔大战克里斯国王; 
        猫公民们要吃自己的国王; 
        舒克和贝塔带着白路离开王国   
        “你们吃不吃?”国王拉下脸。 
        舒克和贝塔摇摇头。 
        “我放电了?你们不怕电?”国王按了一下胸脯。人工心脏上的红灯一闪一闪地亮了。 
        舒克和贝塔互相看了一眼,笑了。原来这就是国王身上的电!舒克和贝塔的飞机坦克上都装着电池,他们懂得电,所以不怕电。 
        国王见舒克和贝塔不怕他身上的电,有点儿慌。他站起来,走近舒克和贝塔。 
        “我放电了?”国王拿出放电的架势。 
        “放吧,我身上正需要电呢!”舒克张开双臂。 
        国王傻眼了。其实他根本放不出电。 
        “来人呀!”国王大声喊叫起来。 
        几只猫宪兵跑进来。 
        “把他俩抓起来!他们是老鼠!猫应该吃老鼠!”围王一急,忘了自己也是老鼠了。 
        真投想到,老鼠当了国王,对自己的同胞比猫还凶!舒克和贝塔同时朝国王扑过去。 
        舒克一拳将国王打倒,贝塔用最快的速度打开国王的人工心脏,取出了心脏里的电池。国王躺在地上不动了。 
        猫宪兵早就恨国王了,国王吃了他们不少亲戚朋友。看见国王躺在地上不动了,猫宪兵们欢呼着跑出王宫,把喜讯告诉全体公民们。 
        转眼间,猫公民们潮水般地涌进王宫。他们要吃掉国王。 
        舒克和贝塔不干了。他们一听说猫吃老鼠就火冒三丈。 
        猫公民们才不管舒克和贝塔的劝阻呢,他们叫骂着冲上前来。 
        “快把电池装上!”舒克急中生智。 
        贝塔忙把电池装进国王的人工心脏里。 
        国王站起来。猫公民们吓得连连后退,纷纷跪在地上磕头。 
        舒克和贝塔笑得前仰后合。他们明白了,白老鼠之所以能在克里斯王国称王称霸为所欲为,责任不在白老鼠,而在猫公民。 
        舒克和贝塔动员猫公民们先离开王宫,由他俩治服国王。猫公民们退出去了。 
        国王老老实实地把他的来历告诉给舒克和贝塔。舒克和贝塔为难了:把白老鼠留下吧,他会继续欺负猫公民们;把他身上的电池取出来吧,猫公民们又要吃他。 
        看来,只有把白老鼠带走。带到一个人人懂电而又没有猫的地方去,他才会老老实实地生活。 
        “把他送到发电厂去。那儿人人懂电,听说也没有猫。”舒克提议。 
        贝塔同意了。 
        白路不敢反对。 
        舒克和贝塔走出王宫,向克里斯王国的猫公民们宣布,白路国王辞职了,由他们把国王带走。 
        猫公民们高兴得跳起了舞。看见自己下台后臣民这么高兴,白路心里挺不是滋味儿。 
        舒克和儿塔开始检查直升机和坦克。为了方便,舒克和贝塔在坦克上安装了一个铁环,在直升机下边安装了一个铁钩子。这样,直升机吊起坦克就能起飞。 
        一切准备工作完成了。猫公民们送给舒克和贝塔好多食物,几乎把直升机和坦克都塞满了。 
        白路和舒克登上了直升机。 
        贝塔钻进坦克。 
        直升机起飞了,它悬停在坦克上空,用钩子钩住了坦克。 
        “准备好了吗?”舒克通过无线电台问贝塔。 
        “准备好了!”贝塔回答。 
        舒克一拉操纵杆,直升机向天上飞去,坦克跟着拔地而起。 
        猫公民们向舒克和贝塔招手,白路在飞机里挺惭愧。 
        克里斯王国的城堡越来越小了。   第15集   
        在去发电厂的途中。自路准备劫持舒克的直升机; 
        舒克和白路在空中进行搏斗   
        舒克的直升机吊着贝塔的坦克,离开克里斯王国,飞到空中。 
        “贝塔!贝塔!你知道发电厂在哪儿吗?”舒克通过无线电询问贝塔。 
        贝塔正躺在坦克里的床上吃东西。 
        “不知道。我在电视上见过,发电厂有大烟囱,还有许多电线。你把飞机拉高一点儿,看看四周有没有。”贝塔一边吃一边回答。 
        舒克操纵直升机向高空飞去。 
        “你也帮着找找。”舒克对白路说。 
        白路暗中一直注意观察舒克是怎样驾驶飞机的,他准备劫持舒克的直升机。 
        自从登上直升机,白路就被这个现代化的空中飞行器迷住了,他觉得当个飞行员比当国王还要带劲儿!在王官里是贝塔和舒克一同对付他,而现在飞机上只有舒克自己,一比一,白路不怕舒克!你别忘了,白路身体里装的是老虎胆。 
        白路眼光不离舒克,他已经摸到一点儿驾驶飞机的门道了。 
        “你老看我干吗?还不快帮着找找发电厂。”舒克说。他一点儿也没发现白路的企图。 
        “我到后边看看。”白路走到机舱的后边,假装往下看。 
        下边是一片麦地,还有村庄和河流。 
        “找到发电厂了!”舒克兴奋得叫起来。 
        白路跑到舒克旁边往前一看,真的,一座雄伟的发电厂出现在机头前方。 
        “舒克,舒克,我也看见了!”耳机里传来贝塔的声音。 
        白路觉得要是再不动手,就来不及了。他趁舒克正聚精会神地操纵飞机,悄悄地来到舒克背后。 
        舒克正在寻找合适的着陆地点,他猛然觉得胳肢窝特别痒痒,他回头一看,白路在胳肢他。 
        “你……你要……干吗?”舒克忍住笑,夹紧胳膊,死死握住驾驶杆。 
        “我要你的飞机!”白路更加使劲地胳肢舒克。 
        “别……别闹,飞……机会掉……下去的!”舒克还以为白路同他逗着玩呢。 
        “谁和你闹,我真要你的飞机!”白路腾出另一只手来搔舒克的肚子。 
        舒克万万没想到白路会来这一手。他痒痒得受不住了。舒克松开驾驶杆,和白路搏斗起来。直升机失去了控制。 
        躺在床上的贝塔忽然觉得坦克忽上忽下,他弄不清舒克在搞什么名堂。 
        “舒克,你在干什么?我刚吃了东西,你这样折腾会弄得我消化不良的!”贝塔通过无线电台喊起来。 
        舒克正和白路在直升机上滚作一团。他听见贝塔的声音,急忙对着话筒喊: 
        “贝塔,白路要劫持飞机!” 
        白路在劫持飞机?贝塔傻眼了,他后悔没坐在直升机上。就靠舒克自己,很难打过白路。 
        贝塔急得团团转,直升机就在头顶上,他干着急,上不去。 
        贝塔趴在潜望镜上往外一看,吓出了一身冷汗,前方是一个巨大的烟囱,眼看直升机就要撞到烟囱上了! 
        “舒克!舒克!快拉杆!!!”贝塔大喊一声,接着闭上了眼睛,他知道来不及了。 
        舒克听到贝塔的叫声,知道一定有紧急情况,他顾不上往外边看——也看不成,白路压在他身上。舒克用脚使劲往后一勾驾驶杆,直升机笔直地向天空升去,螺旋桨几乎擦着了烟囱!好险! 
        闭着眼睛等待和烟囱帽撞的贝塔睁开了眼睛,他的坦克服被冷汗湿透了。贝塔决定去支援舒克,可怎么上去昵?   第16集   
        白路企图把贝塔甩下直升机: 
        直升机掉进烟囱; 
        舒克操纵飞机在阳台上着陆   
        贝塔打开坦克舱盖,从坦克里伸出头来,耳边的风很大,呼呼地刮。贝塔把帽子系好。 
        直升机在头顶上轰鸣着。贝塔抬头一看,铁钩子又细又长,要想顺着它爬到直升机上去不容易,弄不好就会掉下去,摔得粉身碎骨。 
        贝塔往下一看,头直发晕,两腿发软。 
        直升机忽上忽下,忽左忽右,随时都有摔下去的危险。 
        “反正也是死!”贝塔一咬牙,钻出坦克。他两手抓紧铁钩子,开始向上爬。 
        往常贝塔根本不把爬桌子、爬柜子放在眼里,可现在每爬一步,贝塔都要使出全身的力气。光是风就可以把他吹走。 
        正当贝塔快要抓往直升机的轮子时,白路发现了贝塔。 
        白路明白,只要贝塔爬上直升机,他劫持飞机的企图就会落空。那时,舒克和贝塔不把他从飞机上扔下去才怪。 
        自路突然松开舒克,扑向驾驶台,他猛烈地摇晃驾驶杆,他想把贝塔甩下去。 
        直升机开始剧烈地晃动,贝塔一下没抓住,松开了手和脚,被抛到空中。 
        幸亏贝塔早有提防,把自己的尾巴拴在铁钩子上当作安全带。 
        贝塔的身体在空中飞舞着,他的尾巴死死地系在铁钩子上。 
        舒克发现了贝塔的危险处境,他扑过去用劲把白路从驾驶台前推开。 
        直升机垂直下降。 
        贝塔忽然觉得眼前一黑,一股呛人的烟味儿刺得他直咳嗽。 
        直升机和坦克掉到大烟囱里了。 
        滚滚的浓烟刺得贝塔两眼流泪,连连咳嗽,直升机越往下,温度越高。 
        白路已经吓傻了,老虎胆在烟囱里也不管用了。 
        烟囱里黑古隆咚,什么也看不见。舒克拉起了驾驶杆,直升机向上升去。舒克知道,直升机随时都有和烟囱相撞的危险,可他一点儿办法也没有,什么也看不见。听天由命吧。 
        奇迹发生了,直升机居然飞出了烟囱。 
        贝塔深深地吸了口气,他用力向上一蹿,抓住了直升机的轮子。 
        贝塔把尾巴从铁钩子上解开,爬上了直升机。 
        白路想把飞机门插死,但他动作慢了一步,舒克已经把机门打开了。贝塔冲进机舱。 
        “我投降!我投降!”白路退到机舱的角落里。 
        舒克和贝塔把他身上的电池取出来,白路倒在地板上。 
        舒克和贝塔紧紧地抱在一起。贝塔全身都被熏黑了。 
        舒克和贝塔反劫机成功。 
        “你真行!”舒克一边操纵飞机一边夸贝塔。 
        “哪儿有你飞行员厉害呀,用脚丫子开飞机!”贝塔听舒克说了倒勾驾驶杆的精彩技艺,十分佩服。 
        “这家伙劲儿真大。”舒克回头看看白路。 
        “找个地方,把他放下吧!”贝塔提议。 
        “这儿正好是发电厂,这里的动物一定懂电。”舒克同意。他寻找着陆地点。 
        直升机开始下降。贝塔的坦克先着陆,直升机随后停在一旁。 
        “糟糕,我的飞机电池不足了。”舒克说。 
        “这是白路身上的电池。”贝塔把电池递给舒克。 
        “那他…”舒克看看躺在机舱里的白路。 
        贝塔也意识到白路如果离开电池,心脏就不会跳动了。 
        “还是给他装上电池吧。这样把他扔出去,一会儿就会被猫吃了的。”舒克说。 
        贝塔把电池给白路装上。白路站起来。 
        “你走吧,这儿就是发电厂。”舒克打开机舱门,对白路说。 
        白路没想到舒克和贝塔这么宽大他,他愣在原地不动。 
        “快去吧!”贝塔催促。 
        “真对不起!”白路冲舒克和贝塔鞠了一躬,跑出机舱。 
        “咱们怎么办?”贝塔问舒克。 
        “来时我看见不远的地方有一座城市,咱们去那儿找电池。”舒克说。 
        “能飞到吗?”贝塔担心电量不够。 
        “行。”舒克发动了飞机,“你就在飞机上吧。” 
        贝塔点点头。他不敢离开直升机了。 
        直升机吊着坦克升到空中,向城市飞去。天渐渐黑了。 
        “白路不会再吓唬人吧!”贝塔说。 
        “发电厂的动物都懂电,谁也不会怕他。”舒克一边驾驶飞机一边说。 
        “你看,前边有那么多灯!”贝塔叫起来。 
        “城市到了。”舒克开始下降高度。 
        直升机飞到了城市的上空。 
        “没电了!”舒克来回摆了摆驾驶杆,直升机不受控制了。 
        “怎么办?”贝塔慌了。 
        “快找个着陆的地方。”舒克注意观察地面。 
        “下边是个阳台。”贝塔告诉舒克。 
        只有在这座大楼的这个阳台上迫降了。 
        舒克的直升机和贝塔的坦克悄无声息地在阳台上着陆了。 
        当舒克和贝塔准备开机舱门时,他们不约而同地打了个寒颤:一只大花猫蹲在阳台上,虎视眈眈地盯着直升机。   第17集   
        小花猫变成了大花猫; 
        舒克和贝塔被皮皮鲁抓获   
        借着月光一看,舒克大吃一惊,这不是以前蜜蜂皇后为他举办宴会时,要处死他的那只小花猫吗!转眼都长这么大了。 
        “糟了,这是我的冤家!”舒克小声告诉贝塔。 
        “快起飞!”贝塔把电池没电的事忘了。 
        “没电。”舒克提醒贝塔。 
        贝塔一屁股坐在皮椅子上。 
        大花猫觉得从天上落下来的这架直升机挺面熟,可一时又想不起来在哪儿见过。舒克的直升机原来是米黄色的,刚才在大烟囱里被熏黑了。 
        大花猫终于想起来了,这是一只名叫舒克的小老鼠的直升机!这只小老鼠化装成飞行员,到处招摇撞骗。 
        “你再为大家办事,也是一只老鼠!”大花猫一边想一边做好了扑上去的准备。 
        “他要向咱们进攻丁。”贝塔眼尖,他发现了大花猫的企图。 
        话音还没落,大花猫已经扑上来,死死抓住直升机,大声叫起来。 
        屋里的灯亮了。接着,阳台门打开了,走出一个男孩子。 
        “这下完了,人最恨咱们老鼠。”贝塔耸耸肩膀。 
        “你不恨我吧?”舒克忽然问贝塔。 
        “恨你?干吗恨你?”贝塔不明白。 
        “是我把你吊到天上,才有今天的。” 
        “当然恨你。恨你干吗把飞机从烟囱里开出来!还不如掉下去呢!” 
        舒克笑了。贝塔也笑了。笑得挺惨。 
        男孩子低头一看,眼睛亮了,一架直升机!后边还有一辆坦克! 
        “哪儿来的?”男孩子往阳台下边看看。12层高的楼,大花猫不可能叼着直升机和坦克爬上来。 
        “自己飞来的!”男孩子激动了,他弯腰拿起直升机和坦克,冲进屋里。 
        他把直升机和坦克放在桌子上,大花猫蹿上了桌子,蹲在旁边,随时准备抓获舒克和贝塔。 
        男孩向飞机里面看,他张大了嘴,半天说不出话来。直升机里有一只穿着飞行服的小老鼠和一只穿着坦克装的小老鼠。是这两只小老鼠驾着直升机到他的阳台上来的。 
        男孩子乐了,他打开直升机的舱门。 
        大花猫一下扑上去,几乎把直升机撞到桌子下边。 
        “于什么!”男孩子火了,“不许你动他们!你要动他俩一根毫毛,我就对你不客气了。” 
        大花猫愣了,怎么,人不许猫抓老鼠! 
        “下去!”男孩子命令。 
        大花猫乖乖地从桌子上跳下去。 
        贝塔和舒克松了一口气,他们很感激这个男孩子。 
        “咱们交个朋友好吗?”男孩子说,“你们叫什么名字?” 
        名字!人问老鼠叫什么名字!舒克和贝塔差点儿流出眼泪来,从前,他俩只知道人管他们统统叫老鼠,没想到这个男孩子这么尊重他俩。 
        “我叫舒克,他叫贝塔。”舒克说。 
        “我叫皮皮鲁,咱们是朋友了。”皮皮鲁兴奋地说,“你们于吗到我家来?” 
        舒克把他和贝塔怎样到克里斯王国,又怎样把白路送到发电厂,以及同白路在空中搏斗,后来又怎样没有电池了等等,统统告诉给皮皮鲁。 
        皮皮鲁听着,眼睛一下不眨,而且越睁越大。 
        舒克得意极了。原来他还以为,人对他们老鼠的生活一点儿也不感兴趣。 
        孩子比大人好。这是舒克和贝塔的共同感觉。 
        “我去给你们找电池!”皮皮鲁说完拉开柜门,从柜子里拿出爸爸的电动刮胡刀,取出里边的电池。又从半导体收音机里拿出电池。 
        舒克和贝塔感激地看着皮皮鲁,他俩觉得要是再不从直升机里出来,就是不相信朋友了。 
        舒克和贝塔走出直升机。皮皮鲁笑了。   第18集   
        舒克和贝塔为皮皮鲁作飞行和坦克表演; 
        皮皮鲁款待舒克和贝塔; 
        舒克和贝塔拨表   
        舒克把电池装进直升机。 
        “你能给我表演一下吗?”皮皮鲁问舒克。 
        “当然可以!”舒克看了看房间,足够他折腾了。 
        “我为你表演开坦克。”贝塔也愿意为朋友效劳。 
        “太好了!”皮皮鲁叫起来。 
        贝塔钻进坦克。舒克登上直升机。 
        螺旋桨转起来了,紧接着,直升机升到空中。 
        舒克大显身手,一会儿绕着电灯飞,一会儿在衣柜上着陆。逗得皮皮鲁哈哈大笑。 
        贝塔正准备也露一手,忽然他觉得坦克晃动起来。原来,舒克把他的坦克吊到空中了。 
        “你干什么?就显摆你啦?”贝塔不高兴了,通过无线电向舒克抗议。 
        “就一会儿,就一会儿!”舒克吊着坦克在屋里只飞了一圈,就把坦克放回到桌子上。 
        “真棒!”皮皮鲁大加赞扬。 
        贝塔也给皮皮鲁表演了几个高难度动作。 
        皮皮鲁快活极了。 
        大花猫蹲在墙角气得要死。 
        表演结束后,皮皮鲁帮助舒克和贝塔把直升机和坦克身上的烟迹擦干净。还给贝塔洗丁澡。 
        最令大花猫不能容忍的是,皮皮鲁竟然把大花猫的饭碗给舒克和贝塔端去,请他俩吃饭。 
        “今晚你们就住在我家吧。”皮皮鲁说。 
        舒克和贝塔商量了一下,同意丁。他俩决定明天晚上飞走。舒克和贝塔有一个心愿,就是想为皮皮鲁干点儿什么。 
        为了安全起见,舒克和贝塔钻进坦克,甜甜地睡了一觉。大花猫无可奈何。 
        第二天早晨,舒克和贝塔发现皮皮鲁小大高兴。 
        “我们能为你干点儿什么吗?”舒克问皮皮鲁。 
        皮皮鲁耸耸肩膀。 
        “你不高兴了?”贝塔问。 
        “该去上学了,你们要是有能让我提前放学的本事就好了。”皮皮鲁背起书包,一步三回头。 
        “咱们帮皮皮鲁一次忙吧?”舒克对贝塔说。 
        “怎么帮呢?咱们又不能改变时间。”贝塔无可奈何地说。 
        “你看见那座大楼上的钟了吗?他们全城的人都以这个钟为标准时间。咱们开着直升机去把表针拨快半圈,皮皮鲁不就能早放学了吗?”舒克说。 
        “真有你的!”贝塔对舒克佩服到家了。 
        舒克和贝塔开始做准备工作。他俩找了一根绳子,一头挽成一个圈套,另一头系在直升机上。 
        快到11点半时,舒克和贝塔驾驶直升机起飞了。 
        “看,就是那座大楼。”舒克一边操纵飞机一边告诉贝塔。 
        “这表真大。”贝塔吐吐舌头。 
        直升机飞到大表跟前。这时,正好11点半,分针垂直向下。 
        “我操纵飞机靠近表,你把绳子套在分针上。”舒克说。 
        “行”。贝塔二话没说,打丌机舱门,他一手抓住扶手,一手把绳子甩出去套表针。 
        舒克和贝塔想得太简单了,在空中用绳子套表针,谈何容易。 
        几十次都失败了。分针又走了五分钟。 
        想到朋友在课堂上盼着下课的难受样子,贝塔决定冒一次险。 
        贝塔把绳子拴在自己腰上,跳出了直升机。贝塔抓住了分针,他死死地抱住。舒克拉起了驾驶杆,直升机向上升去,分针被直升机往上拉了将近半圈,12点了! 
        “当!当!当!”的报时钟声差点儿把贝塔耳朵震聋。 
        全城所有的学枝都提前25分钟放学了。全城所有的人都发现自己的表慢了近半小时。没有人怀疑钟楼的表不准。 
        钟表修理店门口排起了长长的人龙。   第19集   
        舒克和贝塔驾驶直升机参加航模比赛; 
        航模选手们决定击落舒克的直升机   
        当皮皮鲁知道是舒克和贝塔帮他提前放学时,很感谢这两位朋友。 
        “全城的人都去修表了。”皮皮鲁觉得有趣, 
        “咱们痛痛快快玩吧l” 
        可惜好景不长,1点钟提前25分钟到了。皮皮鲁下午要提前去上学了。 
        “我们再把表拨回来。”贝塔提议。 
        “千万别去,被大人们发现,非抓住你们不可。”皮皮鲁把阳台门关上。 
        “那你……”舒克觉得挺对不住朋友。 
        “没关系。早上早下嘛!”皮皮鲁倒想得开。 
        “你干吗不喜欢上学?”贝塔问。 
        “老师不喜欢我,总是看我不顺眼。”皮皮鲁委屈地说。 
        舒克和贝塔同情地看着皮皮鲁。没想到,人群里也有像他们老鼠一样被别人瞧不起的人。 
        “我申请参加航模小组,老师说我学习成绩不好,不批准。唉,明天就要举行全市航模比赛了。”皮皮鲁叹了口气,他非常喜欢航模。 
        “什么叫航模比赛?”舒克觉得航模似乎同飞机有关。 
        “就是飞机模型比赛。”皮皮鲁拉开门,准备去上学。 
        “我明天帮你去参加航模比赛,行吗?”舒克问。 
        皮皮鲁眼睛一亮,要是舒克开着直升机出现在比赛场上,保准把全场都镇了。 
        当天晚上,舒克、贝塔和皮皮鲁做准备工作。听说航模比赛还有空战项目,皮皮鲁特意把自己的两支弹弓枪安装在舒克的直升机上,让贝塔担任射手,并为他提供了充足的石头子弹。 
        第二天上午,全市航模比赛开始了。整座体育场人山人海。皮皮鲁和本校师生坐在观众席上,老师还差点儿不让皮皮鲁来呢! 
        当本校航模队人场时,师生们一阵欢呼。只有皮皮鲁无动于衷。原先,皮皮鲁也极力为本校队员喊“加油”,谁都希望自己的学校光彩,可每次老师都说他是“假招子”。 
        “你要真想给本校争光,考试得100分呀!”这是老师挖苦皮皮鲁的口头禅。后来,皮皮鲁索性无动于衷了。 
        航模比赛开始了。一架架小飞机呼啸着升到空中,开始表演各种飞行动作,它们不断赢得喝彩声。 
        “舒克准备!舒克准备!”皮皮鲁悄悄按书包里的坦克,利用上面的无线电台同舒克联系。 
        舒克和贝塔此时正在直升机里。直升机停在皮皮鲁家的阳台上待命。 
        “明白!”舒克回答。 
        “起飞!”皮皮鲁下令。 
        一架米黄色的直升机出现在体育场上空,它立刻引起了全场观众的注意。 
        裁判员愣了,参加比赛的飞机中没有直升机呀! 
        只见直升机忽阿空中悬停,忽而垂直降落,忽而盘旋,简直就像有人驾驶一样灵活。观众席上爆发出一阵阵富鸣般的掌声和欢呼声。 
        裁判员也不得不连连点头。 
        “擦着观众的头飞!”皮皮鲁发令。 
        “明白!”舒克一压驾驶杆,直升机擦着观众的头绕场一周。 
        观众们先是一惊,紧接着又爆发出一阵掌声。   第20集 
        舒克和贝塔同航模飞机展开了一场真正的空战; 
        皮皮鲁不让贝塔朝本校的飞机开火; 
        大花猫暗算舒克和贝塔   
        所有参加航模比赛的选手都被激怒了,他们立即联合起来,决定在下一个比赛项目中击落这架直升机。 
        当裁判员刚一发出“空战开始”的口令时,几十架航模飞机腾空而起,同时向舒克和贝塔的直升机扑去。 
        “舒克,快撤退!”皮皮鲁见这么多飞机围攻舒克的直升机,慌了。 
        “别撤!咱们得给皮皮鲁争口气!”贝塔不同意撤退。 
        “对,你快准备子弹!”舒克说。 
        一架红头飞机抢先朝直升机冲过来。 
        贝塔把子弹装进弹弓枪,拉满了橡皮筋,瞄准红头飞机。 
        “打!”舒克说。 
        贝塔一勾扳机,石头子弹射了出去。红头飞机被击中了。 
        全场欢呼。直升机上有真炮!能击落对方!孩子们激动了,这是真正的空战。 
        皮皮鲁把手都拍红了。 
        “注意后方!”皮皮鲁提醒舒克。 
        舒克早就注意到后方有一架蓝飞机想偷袭他。这时,前方正好有一架双翼飞机扑过来。 
        就在双翼飞机快要撞上直升机的一瞬间,舒克操纵直升机垂直升起来。 
        双翼飞机和蓝飞机相撞了。 
        连裁判员都击掌叫好。 
        贝塔又接连击落了两架飞机。 
        “你的枪法真准!”舒克夸奖贝塔。 
        “炮手打枪,小意思。”儿塔得意了。 
        这时,空中还有十几架敌机,它们不敢靠近升机,躲在远处盘旋。 
        “咱们进攻一下吧?”贝塔提议。 
        “行。”舒克掉转机头,朝一架白色的飞机冲过去。 
        贝塔瞄准了白飞机。 
        “别打!别打!那是我们学校的飞机!”耳机里传来皮皮鲁急切的声音。 
        “别打!”舒克赶快制止贝塔。 
        “怎么?” 
        “那是皮皮鲁学校的飞机。” 
        “他们学校不是不让他参加航模小组吗?” 
        “谁知道怎么回事!他不让打就别打呗,飞行员得服从地面指挥,懂吗?” 
        “还有哪架不能打,先说!”贝塔不高兴地说。 
        经过一个小时的空战,体育场上空只剩下直升机和皮皮鲁学校的飞机了。 
        全枝师生潮水般地涌向校航模队的运动员们,把他们抬起来,抛向空中。 
        其他观众和裁判员都为那架米黄色的直升机悄然离去感到迷惑不解。 
        舒克和贝塔按照皮皮鲁的命令返航了。他俩一点也不明白皮皮鲁为什么这么做,他们本想址皮皮鲁大大地神气一番。 
        “你看人家,都为本校得冠军高兴,就你无动于衷,一点儿荣誉感也没有!”老师又挖苦皮皮鲁了。 
        皮皮鲁顾不上理老师,他撒腿就往家跑,去感谢舒克和贝塔。 
        皮皮鲁做梦也不会想到,舒克和贝塔已经大祸临头了。 
        舒克和贝塔在阳台上刚一着陆,埋伏在阳台上的大花猫趁皮皮鲁不在家,扑上去抓住直升机。大花猫把直升机连同飞机里的舒克和贝塔塞进准备好的纸箱子里,再把纸箱子封死,推到床下的最里头。 
        当皮皮鲁跑进屋子时,大花猫正趴在桌子上假装睡觉。 
        皮皮鲁跑到阳台上一看,没有直升机。屋里也没有。 
        “看见舒克和贝塔了吗?”皮皮鲁拍拍大花猫。 
        大花猫打个哈欠,摇摇头。 
        皮皮鲁慌了。他站在阳台上往外面看,没有直升机的影子。 
        “又没电池了?被人抓走了?出飞行事故了?”皮皮鲁猜测看。 
        皮皮鲁想起书包里的坦克。他拿出坦克,打开舱盖儿,从里边拿出小话筒。 
        “舒克,舒克!你在哪里?”皮皮鲁呼叫。 
        “我是舒克!我是舒克!我在床底下的纸箱子里!我在床底下的纸箱子里!'. 
        “床底下?纸箱子里?”皮皮鲁莫名其妙。他爬到床下,拉出纸箱子,打开一看,直升机真在里边。 
        大花猫吓傻了,他浑身开始哆嗦起来,他相信皮皮鲁一定饶不了他。 
        皮皮鲁把直升机从纸箱子里拿出来。 
        “你们怎么藏在这儿?”皮皮鲁惊奇地问。 
        “跟你开个玩笑呗!”舒克看见大花猫浑身发抖,不忍心揭发他。 
        “对,开个玩笑。”贝塔点点头。 
        “你们真逗,把我急坏了。”皮皮鲁笑了。 
        大花猫松了一口气,表情挺不自然。 
        皮皮鲁用最丰盛的饭菜款待舒克和贝塔。吃完饭后,舒克和贝塔决定和皮皮鲁告别,他俩觉得待在大花猫身边凶多吉少。皮皮鲁找来几节新电池,送给舒克和贝塔,又赠送给他俩许多食物。 
        “我们以后来看你。”舒克说。 
        “我等着你们。”皮皮鲁舍不得让舒克和贝塔飞走,可他不敢长期留舒克和贝塔。要是让妈妈发现这架来历不明的飞机,她会把飞机交给学校老师的。 
        舒克登上直升机,贝塔钻进坦克。 
        夜色降临了。直升机吊着坦克起飞了。皮皮鲁站在阳台上冲舒克和贝塔招手。 
        一场恶战在野外等待舒克和贝塔。   第21集 
        舒克和贝塔在空中听到紧急呼救声; 
        贝塔大吃一惊; 
        贝塔的坦克和野猫赛跑   
        舒克和贝塔离开皮皮鲁家,朝城外飞去。 
        “贝塔,你在干什么?”舒克一边开飞机一边通过无线电台同坦克里的贝塔。 
        没有回答。 
        “贝塔!贝塔!”舒克以为贝塔出了什么事。 
        贝塔正在坦克里偷偷掉眼泪。他觉得皮皮鲁真可怜,没人理解他。不知怎么搞的,贝塔想起了自己从前在家里时的处境,想起了咪丽欺负他的情景。 
        “贝塔!贝塔!”舒克叫着。 
        “干吗?”贝塔反问舒克。 
        “我以为你被大花猫绑架了呢!”舒克说。 
        “净瞎操心。”贝塔说完打开坦克舱盖,把头伸出来,他觉得坦克里憋得慌。 
        天上挂满了密密麻麻的星星,贝塔一直弄不清这些星星是怎么被人安到天上去的。 
        “大概也是用飞机运上去的吧?”贝塔想。 
        “救命啊——”忽然从地面上传米一阵呼救声, 
        贝塔觉得这声音挺耳熟,他顾不上细想,忙叫舒克: 
        “舒克!舒克!地面有呼救声!地面有呼救声!请你降低高度。” 
        “明白。”舒克操纵直升机下降。 
        呼救声越来越大,借着月光,贝塔看见三只大野猫在咬一只猫。那猫拼命挣扎。 
        “地面上怎么回事?”舒克问。 
        “三只猫在欺负……”贝塔还没说完,舒克就急了:“准备参战!” 
        “三只猫在欺负一只猫!”贝塔把话说完。 
        “猫和猫打架?”舒克操纵直升机悬停在空中,他觉得似乎没必要去干涉猫之间的战斗。 
        贝塔也是这么想。 
        直升机现在离地面很近了,贝塔忽然呆住了:那只喊救命的猫是咪丽! 
        “舒克!舒克!帮帮咪丽吧!”贝塔请求。 
        “咪丽?什么咪丽?”舒克不明白。 
        “就是我原来跟你说过的那个咪丽呀!” 
        “就是曾经欺负你的那只猫?”舒克不信,哪儿有这么多巧事。 
        “就是她!没错。”贝塔肯定地说。 
        “去救她?”舒克觉得贝塔的心眼儿不错。 
        “救她!你把我的坦克放到地面上,你在空中掩护我。”贝塔说。 
        舒克同意了。他一推驾驶杆,直升机迅速下降。贝塔觉得坦克一阵震动,着陆了。 
        舒克用高超的飞行技术摘下钩在坦克上的铁钩子,驾驶飞机升到空中。 
        贝塔好长时问没开坦克打仗了,他的手早痒痒了。贝塔把一发炮弹塞进炮膛。通过潜望镜,贝塔看见三只大野猫正围着咪丽咬。 
        贝塔驾驶着坦克朝三只野猫冲过去。 
        一只大野猫的屁股正对着坦克,贝塔加大速度撞上去,大野猫连打了两个滚儿。 
        另外两只野猫愣了一下,马上朝贝塔的坦克扑上来。他们没把这个小玩艺儿放在眼里——野猫的身体比贝塔的坦克大一倍。 
        贝塔对准其中一只野猫的肚子开炮了,那只野猫挨了炮弹后稍稍停顿了一下,又冲上来。大野猫不怕贝塔的炮弹。 
        贝塔索性一按电钮,坦克迎着野猫开上去。履带压着了一只野猫的脚,疼得他大叫起来。 
        咪丽被这突如其来的变化惊呆了。她眼睛忽然一亮:是贝塔的坦克。 
        三只野猫凑到一起碰了下头,一起朝坦克扑过来。 
        贝塔操纵坦克掉头就跑。一来他想把野猫引开,让咪丽脱离险境;二来他怕这三只野猫把他的坦克翻个底朝天。 
        野猫奔跑的速度非常惊人。贝塔的坦克几乎飞了起来。野猫在后边紧紧跟着坦克,眼看就要追上了。 
        “刹车!”从空中传来舒克的声音。 
        一句话提醒了贝塔。贝塔突然来了个急刹车。三只野猫停不住,冲到前边去了。贝塔掉头往回开。 
        贝塔从潜望镜里看见咪丽还站在原地,一动不动,像傻子一样。 
        野猫又追上来了。   第22集 
        舒克驾驶直升机参战; 
        舒克把野猫吊离战场; 
        贝塔决定和咪丽一起回家   
        贝塔又装上一发炮弹,他掉转坦克,瞄准了为首的那只野猫的脑门。 
        贝塔按下了射击按钮,只见那只野猫大叫一声,蹦得老高。打中了! 
        可野猫毕竟不是麻雀,贝塔的石子炮弹打不伤他们。野猫们被激怒了,他们三个从不同的方向朝坦克扑过来。 
        “请求空中支援!请求空中支援!”贝塔对着话筒喊起来。 
        舒克早已做好了准备,他把两只弹弓枪都压上了子弹。 
        “贝塔,你掉头跑!”舒克说。 
        贝塔操纵坦克掉头就跑,野猫在后边追。 
        舒克驾驶直升机压在野猫的头顶上飞。直升机离野猫只有一尺的距离。 
        舒克一手握驾驶杆,另一只手搂住弹弓枪的扳机。 
        枪口几乎挨着一只野猫的后脑勺。舒克抠动了扳机。 
        那只野猫惨叫一声,在地上连着打了好几个滚儿。 
        另外两只野猫还是死咬住坦克不放。 
        静静的夜晚,在郊外发生着一场激烈的搏斗:一辆坦克在前边跑,两只野猫在后边追,一架直升机压在野猫头顶上飞。说起来也好笑,两只老鼠为一只猫打抱不平。 
        “舒克!用铁钩子钩野猫的耳朵。”贝塔想出一个办法。 
        “太棒了!”舒克忘了发挥铁钩的作用,经贝塔这么一提醒,他觉得铁钩子一定厉害。 
        野猫还在高速奔跑着。舒克的直升机与野猫保持着同等的速度,真是一场立体战争!直升机下边的铁钩子在一只野猫的耳边来回晃动着,要想钩住他的耳朵也不容易,舒克全神贯注地操纵飞机。 
        终于钩上了!直升机加足马力向天上飞去,可野猫太重了,飞机只能把他的两条腿吊离地面。这就够了,野猫被直升机拖着,疼得他大声喊“饶命”。 
        “把他拖远点儿!”贝塔说。 
        舒克驾驶直升机拖着大野猫朝远处飞去。 
        剩下的一只野猫不敢再追贝塔的坦克了,他跑到那只被击中后脑勺的野猫旁边,两只野猫商量了一会儿,溜走了。 
        贝塔驾驶坦克来到咪丽身边,咪丽感激地看着坦克。 
        “谢谢你救了我,贝塔。”咪丽肯定坦克里一定是贝塔。 
        贝塔不敢出来。他牢牢记着眯丽猛然回头咬他一口的教训。 
        “你怎么到野外来了?”贝塔在坦克里问。 
        “你走后不久,主人就把我从家里赶出来了。”眯丽委屈地说。 
        “为什么?”贝塔不明白。他不在了,咪丽应该生活得好呀! 
        “主人说,没有老鼠,养猫也就没用了。”咪丽伤心地说,“原来怪我不好,原谅我吧,贝塔!现在我懂了,没有你,丰人根本不会养我。” 
        贝塔简直不相信这是真的,猫是因为有老鼠才受到人的优待。 
        “我原来不该恨你,应该感谢你才对。”咪丽对着坦克说,“你出来吧,贝塔,我不会咬你了。” 
        贝塔半信半疑地从坦克里伸出头来,他做好了随时钻回去的准备。 
        “刚才他们干吗欺负你?”贝塔看到咪丽浑身是伤。 
        “我好不容易找到一点儿吃的,他们来抢,我不给,他们就咬我,我已经好几天没吃饭了。”咪丽一边说一边掉眼泪。 
        要不是亲眼看见,说什么贝塔也不会相信猫咬自己的同胞时比咬老鼠还狠。 
        贝塔钻回坦克,给咪丽拿出一根香肠。 
        “你吃吧!”贝塔把香肠递给咪丽。 
        咪丽想起从前自己不让贝塔吃饭,惭愧极了。 
        “吃吧!”贝塔又说了一遍。 
        咪丽大口大口吃起来。 
        “你以后怎么办?”贝塔问。 
        咪丽摇摇头。 
        “你在家里过惯了舒服日子,出来真够受罪的。”贝塔说。 
        咪丽哭了。 
        一个念头在贝塔脑子里产生了,他想帮助咪丽。 
        “我帮你再回到主人家怎么样?”贝塔问。 
        “再回到主人家?”咪丽摇摇头,不相信。 
        “我先回去,你在屋外等着。主人一看见我回来了不就又会收养你了吗?”贝塔说。 
        咪丽感动了。 
        咪丽和贝塔就这样决定了。   第23集 
        舒克决定去看妈妈; 
        贝塔和舒克约定一小时通过无线电台联络一次   
        “舒克!舒克!你在哪里?你在哪里?”贝塔呼叫舒克。 
        “我把大野猫扔到河里了,他洗了个澡!我马上回来!”贝塔的耳机里传来舒克兴奋的声音。 
        不一会儿,直升机出现在贝塔和咪丽的头顶上。 
        从空中看到贝塔和咪丽在一起,舒克吓了一跳。 
        “贝塔!注意安全!”舒克提醒贝塔。 
        “你着陆吧,没事。”贝塔说。 
        直升机在坦克旁边着陆了。舒克打开驾驶座旁边的玻璃窗,他不敢下来。 
        贝塔把咪丽介绍给舒克,并把他要帮咪丽回家的决定告诉舒克。 
        “和我们一起去吧!”贝塔说,“就一天。” 
        听说贝塔要回到原来住的地方去,舒克忽然想起了自己的妈妈。自从他开着直升机离开家后,舒克还从未见过妈妈。尽管自己的妈妈有着不光彩的名声,可她毕竟是妈妈。 
        “我想回家去看看妈妈。”舒克说。 
        “这样吧,你去看妈妈,我去帮咪丽回家,咱们随时用无线电台联系,争取明天下午会合,行吗?” 贝塔提议。 
        “好吧,一小时联络一次。”舒克说。 
        朋友要分手了,虽然只有一天,可心里还挺难受。他们互相告诉了地址,再次约定好每小时联络一次。 
        舒克钻进直升机,他冲贝塔和咪丽摆摆手。直升机起飞了。 
        “祝你一路平安!”贝塔说。 
        “祝你顺利!”耳机里传来舒克的祝愿。 
        “咱们走吧!”贝塔对咪丽说。 
        咪丽心里挺不好受,是她把贝塔从家里逼走的。而现在,却是贝塔送她回家。 
        贝塔心里挺得意,一想到没有他,主人就不养咪丽了,贝塔美滋滋的。 
        “贝塔!贝塔!你怎么样了?请回答。”耳机里传出舒克的询问声。 
        “我很好,请放心。你怎么样了?”贝塔问。 
        “我已经接近家了,正在寻找降落的地方。”舒克说。 
        “注意安全。多在空中观察一会儿。”贝塔嘱咐舒克。 
        “一小时后再联系。” 
        “好,一小时后再联系。” 
        从潜望镜里,贝塔已经看见他原来居住的那座房子了。 
        贝塔把坦克停在咪丽身旁,打开舱盖,探出半个身子。 
        “你在这儿等着。昕到主人在里边喊叫后,你就进去,保准主人对你好。”贝塔对眯丽说。 
        “你不会有危险吧?”眯丽有点儿替贝塔担心。 
        “没事儿!”贝塔钻回坦克,把舱盖锁牢。 
        坦克从咪丽出入的小门驶进了屋子。   第24集 
        贝塔大闹一场; 
        咪丽受到主人热情的欢迎; 
        贝塔教咪丽学老鼠叫; 
        舒克贝塔失去联系   
        屋里黑咕隆咚,主人睡觉了。 
        贝塔把坦克丹到床底下隐蔽好,他悄悄从坦克里钻出来。 
        这里的一切对贝塔来说太熟悉了。衣柜,写字台,电视机……几乎一点儿变化也没有。贝塔想起了他从前的生活。 
        贝塔走进他原来居住的洞里,他觉得这洞又黑又小,他感到奇怪,从前他住在这儿怎么一点也没觉得小。脚下一个东西差点儿绊了贝塔一个跟头,他低头一看,是他从前用来装香味的布口袋。 
        贝塔想起了自己昔日饿肚子时的难受劲儿,他可怜自己。 
        贝塔想起咪丽还在屋外等着,他准备行动了。 
        贝塔钻出洞,爬上食品柜。 
        食品柜上放着一盘油炸花生米,贝塔不客气地大吃起来,还故意把花生米撤了一地。 
        主人睡得挺香。 
        贝塔把一个铁缸子从柜子上推下去。 
        咣当! 
        主人被吵醒了。 
        贝塔趁机大叫起来。 
        主人打开电灯,看见了食品柜上的贝塔。 
        “老鼠!”主人一惊,掀开被子朝食品柜扑过来, 
        “抓住它!” 
        贝塔一溜烟儿钻到床底下。 
        “谁让你把咪丽轰跑了,看,老鼠又回来了吧!” 
        “这……” 
        “老鼠把花生米撤了一地!” 
        主人家里吵翻了天。 
        正在全家手足无措时,咪丽像天使般出现在主人面前。 
        “咪丽!”主人兴奋得大叫起来。 
        “咪丽回来了!” 
        “咪丽回来了!”主人全家一片欢呼。 
        “快去给咪丽拿吃的!” 
        “快去给咪丽洗澡!” 
        咪丽受到了最隆重的接待。 
        贝塔在床卜看着这一刹,心里有点儿那个。他也想象咪丽这样受到人们的欢迎,贝塔明白这是做梦。不过他很清楚,是他使咪丽受到这样隆重的欢迎的。然而他却扮演着一个不光彩的角色,用来换取咪丽在主人面前大放光彩的位置。 
        咪丽要感谢贝塔,她钻进床底下。 
        “咪丽去抓老鼠了!”主人喊道。 
        这话吓了贝塔一跳。他慌忙钻进坦克,锁好舱盖。准能保证咪丽不是来抓贝塔向主人献殷勤的呢? 
        “贝塔!贝塔!”咪丽站在坦克旁边叫。 
        “干吗?”贝塔问。 
        “谢谢你!”咪丽感激地说。 
        “……”贝塔投说话。 
        “你出来呀,我给你带来好吃的了。”咪丽说。 
        贝塔越想越不是滋味:咪丽为什么可以光明正大地受人宠爱,而他贝塔却要躲在这阴暗的床下。尤其使贝塔生气的是,咪丽还是打着抓他的旗号钻进床底下来的。 
        咪丽明白贝塔为什么伤心了,她哭着说:“贝塔你别伤心。我真想和你换换,让你当猫,我当老鼠。是你帮我回来的你却只能藏在床下挨骂。刚刚听主人骂你,我心里真难过。咱们走吧,贝塔,我宁愿去野外流浪。” 
        坦克舱盖打开了,贝塔钻出来。 
        “别哭了。净说傻话,这儿过得多舒服!我一点儿也不伤心,只要你不挨饿就行了。”贝塔一边说一边抹眼睛。 
        咪丽给贝塔食物。 
        “我明天就走了,你在这儿好好过吧。’贝塔告诉咪丽。 
        ‘你一走主人又该轰我了,”咪丽说。 
        贝塔觉得咪丽的话有道理。 
        “你就留在这儿吧,每天有吃有喝。”咪丽提议。 
        “那可不行,我得和舒克在一起。”贝塔不干。 
        咪丽挺惭愧,她觉得贝塔对朋友讲义气。 
        贝塔忽然想出了一个主意。 
        “咪丽,我想出一个办法,不过你别嫌脏。” 
        “什么办法?” 
        “我给你留点儿我的屎,就是耗子屎。你每天拿一点儿撒在主人的饭桌上,主人肯定就不会轰你走了。” 
        咪丽一想,这办法不错。 
        “可要是用光了呢?你每隔两天能给我送一次吗?”咪丽不放心。 
        “两天送一次?这我可做不到。”贝塔吐吐舌头。 
        “那老鼠屎用完了以后主人又该轰我了。”咪丽发愁。 
        贝塔眼睛一亮,对咪丽说:  “干脆我教你学老鼠叫吧!你学会了老鼠叫,每天晚上叫一会儿,主人听到这种声音就不会轰你了,天天还得给你好吃的。” 
        咪丽觉得这办法好。 
        “来,现在就教。”贝塔当老师。 
        “吱——”贝塔作示范。 
        “喵——”咪丽跟着发音。 
        “不是喵,是吱——” 
        “吃——” 
        “也不是吃,是吱——,你注意看我的口型。” 
        贝塔把牙露出来,嘴角向后撇。 
        咪丽模仿贝塔的口型。 
        “气——” 
        “不对,不对,是吱——” 
        “……”咪丽不敢发音了。 
        “别灰心,要想生活得好就得下功夫。你看主人学外语时不是也很费事吗?来,再试试,吱——” 
        “次——” 
        “好,快了!吱——” 
        “次——吱——” 
        “对!就这样!再来一遍。” 
        “吱——吱——吱——” 
        咪丽学会了老鼠叫。贝塔走后,主人还会继续宠爱她。 
        为了保险起见,贝塔决定让咪丽演习一次。 
        咪丽在床底下连续发出“吱——”的叫声,同时用爪子抓纸箱子。 
        主人被吵醒了,他打开灯。 
        “吱——吱——” 
        “喵——喵——” 
        “吱——吱——” 
        “喵——喵——” 
        咪丽一会儿学老鼠叫,一会儿发出猫叫的声音,同时在床下乱踢乱抓,好像床下正发生着一场猫鼠之问的恶战。 
        “你们快听,咪丽抓老鼠呢!”主人对家里人说。 
        床下越打越热闹。床上的主人高兴得止不住笑。衣柜上的贝塔手舞足蹈。 
        老鼠叫声没有了。猫叫声继续着。 
        “抓住了!抓住了!”主人兴奋地跳下床,趴在地上往床底下看。 
        咪丽一边抹着嘴一边从床底下钻出来。 
        主人喜爱地拍拍咪丽的头。 
        “这么快就把该死的老鼠吃了!”主人夸奖咪丽。 
        贝塔心里又有点儿不是滋味。但一会儿就过去了。 
        主人睡觉后,贝塔从衣柜上下来,钻进床底下。 
        “像真的一样!”贝塔认为咪丽可以毕业了。 
        “对小起,足你帮助了我,向主人每次夸奖我时都要骂你,真对不起。”咪丽心里很难过。 
        “投关系,我不怕骂。”贝塔安慰咪丽。 
        贝塔忽然一拍脑袋:“哎呀,该和舒克联系了!” 
        贝塔钻进坦克,戴上耳机。 
        “舒克!舒克!我是贝塔!请你回答!” 
        “……” 
        “舒克!舒克!我是贝塔!请你回答!” 
        “……” 
        “舒克!舒克!我是贝塔!……” 
        “……” 
        “糟糕,舒克一定出事了!”贝塔钻出坦克,焦急地对咪丽说,“我得马上走。再见了,咪丽。” 
        “我和你一起去!”咪丽连想都没想就说。和贝塔相处时间虽然不长,可咪丽的身上已经起了变化。 
        “你?不在这儿过舒服日子了?”贝塔问。 
        “说不定我能帮你们忙呢!”咪丽说。 
        贝塔觉得有一只猫跟着他的确安全些,就同意了。 
        咪丽和贝塔迅速离开屋子,用最快的速度朝舒克家奔去。   第25集 
        舒克的直升机在房项上着陆; 
        舒克在窗台上碰见蓝鹦鹉和绿鹦鹉; 
        绿鹦鹉和蓝鹦鹉反对舒克看妈妈; 
        舒克遇险   
        舒克和贝塔分手后,驾驶着直升机去看妈妈。 
        舒克飞到了自己熟悉的地方。他看见了和贝塔打仗的地方,看见了蜜蜂皇后宴请他的地方,还有小麻雀的家。 
        舒克很想见他们。自从他把贝塔的坦克吊走后,还一直没回来过,小麻雀他们一定急坏了。 
        舒克决定还是先去看妈妈。他已经看见了妈妈住的那座房子。直升机朝房子飞去。 
        为了安全起见,舒克把直升机停在房顶上。他把一根绳子拴在飞机上,另一头扔下来,绳子正好经过窗户。 
        舒克顺着绳子溜下来,落到窗台上。 
        窗户没插。舒克悄悄钻进屋里。 
        舒克借着月光一看,屋里变化挺大,床和桌子都挪了位置。舒克从窗台跳到桌子上。 
        “这不是舒克吗?”黑暗里传来一个声音。 
        舒克顺着声音传来的方同看去,是鸟笼里的蓝鹦鹉和绿鹦鹉。 
        “你们好!”舒克问候。 
        “你好!”蓝鹦鹉热情地说,“我们听说你现在变得可好了,净帮助别人。” 
        “听小麻雀说,你救过他的命。”绿鹦鹉说。 
        “应该做的。”舒克不好意思了。 
        “你来干什么?”蓝鹦鹉好奇地问。 
        “我来看妈妈。”舒克说。 
        “你妈妈还偷东西哪!”蓝鹦鹉提醒舒克。 
        “你不应该看她!有这样的妈妈真丢人!”绿鹦鹉说。 
        “可……她……是我的妈妈……”舒克说。他觉得妈妈就是妈妈,偷东西和不偷东西是另一回事。 
        绿鹦鹉和蓝鹦鹉开始撇嘴了,接着他俩又交头接耳地嘀咕起来。 
        “再见。”舒克说完从桌子上爬下来,朝自己家走去。 
        舒克的家没有变化,洞口还是老样子。 
        舒克趴在洞口昕听,里面没动静。他蹑手蹑脚地钻进去,生怕吓着妈妈。 
        一钻进洞里,舒克立刻想起了自己的童年,想起妈妈每天从外边带吃的回来喂他的情景。 
        “谁?”黑暗中传来颤抖的声音。 
        舒克定了定神,走过去一看,角落里躺着一只年迈的老鼠,正是他的妈妈。 
        “妈妈,我是舒克!”舒克简直不相信这是自己的妈妈,她老了,牙齿都快掉光了。 
        “舒克?舒克!”妈妈惊讶地欠起了身子,一把抓住舒克的胳膊,又躺下了。 
        “妈妈,你病了?”舒克问。 
        “我老了,不行了,好几天没吃上东西了。”妈妈有气无力地说,“听说你在外边混了个好名声,妈妈也就放心了,千万要保住这个好名声。妈妈知道,老鼠混个好名声不容易。” 
        望着饿得有气无力的妈妈,舒克忽然恨起自己来:为了自己出去混个好名声,把年迈的妈妈扔在家里不管。好名声到手了,可良心到哪儿去了?没有良心的好名声能算好名声吗? 
        “妈妈,我对不起你!是你把我养大的,我却……”舒克哭了。 
        “别这样说,你快走吧!妈妈能见上你一面,也就放心了。记住,保住好名声,保住好名声啊!”妈妈推开舒克的胳膊,闭上眼睛。 
        舒克恨死名声这个东西了。为了得到好名声,他抛弃了生他养他的妈妈,可谁也没有谴责过他,就因为他妈妈是老鼠!舒克真可怜自己的妈妈,她应该和猫的妈妈享有样的做母亲的权利呀! 
        舒克擦干眼泪,他决定留在妈妈身边,伺候妈妈。什么名声不名声,去他的吧!丧失良心的名声再好,舒克也不稀罕了。 
        舒克钻出洞,给妈妈找吃的。妈妈已经饿得昏过去了。 
        他来到食品柜旁边,食品柜锁着。舒克发现食品柜上放着一只碗。 
        他爬上食品柜,碗里是香肠。舒克拿了一根香肠,回到地上。他还没站稳,就觉得背后刮起一阵疾风,紧接着,舒克的肩膀被死死地抓住了。 
        舒克回头一看,是大花猫!皮皮鲁家的大花猫!   第26集 
        舒克又见到小麻雀他们; 
        朋友不理解舒克; 
        大花猫准备处决舒克; 
        咪丽和哥哥重逢   
        原来,自从舒克和贝塔离开皮皮鲁家后,大花猫越想越生气,他不但不感谢舒克和贝塔“包庇”了他一次,反而更恨舒克了。他悄悄离开皮皮鲁家,来到舒克家附近潜伏着,他下决心一定要抓住舒克。 
        果然,舒克开着直升机回来了。大花猫等舒克从窗户钻进屋子后,他也跟着钻了进去。 
        “你还有什么说的?装成飞行员,到处招摇撞骗,实际上是小偷!”大花猫冷笑了一下,死死抓住舒克不放。 
        舒克觉得肩膀像火烧一样疼,他请求大花猫:“让我把香肠给妈妈送去行吗?你别松开我,我把香肠塞进洞里就行,妈妈快饿死了。” 
        “我让你把香肠送给你的老鼠妈妈?老鼠也配当妈妈?笑话!这根香肠正是你的罪证!”大花猫不同意。 
        “咱们从前听到的舒克变好了的消息都是假的。”蓝鹦鹉对绿鹦鹉说。 
        ‘就是,他这么留恋他的妈妈,真不像话。”绿鹦鹉说。 
        一想到妈妈在家里饿得昏丁过去,听着刚才这些侮辱妈妈的话,舒克闭上了眼睛。 
        “走,去见见小麻雀他们,让大家认识认识你的真面目!”大花猫拎起舒克,拿着他的罪证香肠,从窗户跳了出去。 
        天,渐渐亮了。 
        大花猫押着舒克来到小树林里,这里的一草一木舒克都非常熟悉。 
        “大家快来看!我抓住了一个小偷!”大花猫扯着嗓子喊。 
        小麻雀飞来了。小蜜蜂飞来了。蚂蚁们跑来了。 
        “舒克!”朋友们惊喜地喊叫起来,自从舒克开直升机把贝塔的坦克吊走后,他们一直在找舒克。 
        “你干什么?”小麻雀生气地质问大花猫。 
        “他是小偷!”大花猫说完用力压丁压舒克的肩膀,舒克差点儿趴在地上。 
        “你胡说!”小麻雀火了。 
        “你放开他!”小蜜蜂飞到大花猫头上,准备蜇他。 
        “让他自己说,他是不是小偷?这香肠就是他偷的!”大花猫把香肠往大家面前一扔。 
        “舒克,这不是你偷的!”小麻雀说。 
        “是我偷的。”舒克说。 
        小麻雀们都愣了。 
        “不,不是你偷的!”小麻雀急了,他不相信舒克会偷东西。 
        “是我偷的。”舒克义重复了一遍。 
        舒克现在什么也不怕了。名声,面子,他统统不去想。舒克现在惟一惦记的是他的妈妈在挨饿。妈妈快饿死了,而食品柜上放着吃的,为什么不能去拿呢?管这叫偷也好,叫拿也好,反正舒克不能看着妈妈饿死。 
        “舒克,你干吗要去偷吃的呢?”小麻雀还是不信。 
        “我妈妈快饿死了。”舒克说。 
        “你妈妈!”小麻雀愣了一下,他头一次想到舒克的妈妈,一只老鼠。听见舒克管老鼠叫妈妈,小麻雀有点儿不习惯。 
        小蜜蜂不明白舒克为什么留恋一个不光彩的妈妈。 
        “我现在处决他!”大花猫看到大家认出了舒克的真面目,得意极了,他拎起舒克,朝草丛里走去。 
        小麻雀忍不住飞过去,但他又落在树枝上了。他心里很难过,舒克干吗要为那样一个妈妈而偷东西呢? 
        大花猫把舒克拖进草丛,正准备动手。忽然一颗石子炮弹打在他后脑勺上。 
        大花猫大叫一声。 
        一辆电动坦克朝大花猫撞过来。 
        大花猫定定神,认出是贝塔的坦克。他松开舒克,准备朝坦克扑过去。 
        “哥哥!”坦克后边传来一个熟悉的声音。 
        大花猫一看,是他分别已久的妹妹咪丽。 
        “咪丽!”大花猫顾不上坦克了,他跑到咪丽身边。 
        “咪丽,你从哪儿来?”大花猫激动得喘不过气来,自从小时候和妹妹分离后,他几乎天天在想妹妹。 
        “我来救舒克。”咪丽没想到抓舒克的是她的哥哥。 
        “救舒克?”大花猫吃了一惊。 
        咪丽点点头。 
        “舒克是老鼠!”大花猫把老鼠两个字说得特别重。 
        “我知道舒克是老鼠。”咪丽说。 
        “那你?”大花猫退后一步,仔细打量着自己的妹妹。 
        咪丽把舒克和贝塔怎么救她,主人因为没有老鼠就把她轰出来了以及贝塔怎么帮她回家等等都讲给哥哥听。 
        大花猫听着听着头慢慢地垂下来了。“没有老鼠,人不会养猫”,他觉得咪丽这话挺有道理。再说,舒克还救过妹妹的命! 
        大花猫走到舒克身边,给舒克拍拍身上的土,什么话也没说。 
        舒克扭头就跑。 
        “你去干吗?”贝塔急了。 
        “我去给妈妈送吃的。”舒克头也不回。 
        “我跟你去!”贝塔跳出坦克,跟着舒克跑。 
        小麻雀和小蜜蜂飞过来。 
        “怎么啦?”小麻雀问大花猫。 
        “舒克是我们的朋友,你们不该这样!”咪丽说。 
        “他……他妈妈……”小麻雀结结巴巴。 
        “舒克的妈妈也是妈妈,她也有生存的权利,就你的妈妈是妈妈!”咪丽不客气地训斥小麻雀,“人家还救过你的命呢!关键时刻不够朋友!” 
        “我……”看到猫都为舒克辩护,小麻雀惭愧了。就是,舒克心疼自己的妈妈,有什么不好呢?一个连妈妈都不爱的人,怎么会爱大家呢? 
        小蜜蜂脸也红了。 
        “咱们去给舒克道歉。”小麻雀提议。 
        舒克的妈妈醒过来了。舒克和贝塔把妈妈从家里抬出来,朋友们要慰问她。 
        看到舒克的妈妈骨瘦如柴的样子,大家心里都挺难受,都对舒克不满了,他怎么早没想起照看自己的妈妈呢! 
        “老鼠妈妈,您受苦了。”小麻雀说。 
        “老鼠妈妈,我们对不起您。”小蜜蜂说。 
        大花猫给舒克的妈妈找来许多吃的。 
        听到这么亲切的话,受到这样的尊敬,舒克的妈妈感到承受不了。她一辈子都是在歧视和侮辱中度过的。 
        “我也想体体面面地过日子,我也恨自己干吗是一只老鼠。我生舒克时也像你们的妈妈一样受罪,可为什么我的儿子只有抛弃了我才能混到好名声呢!”舒克的妈妈哭了。 
        从来不掉泪的大花猫也哭了。 
        经过商量,大家决定今后由咪丽把舒克的妈妈带回主人家抚养。这样一举两得。 
        舒克和贝塔决定成立舒克贝塔航空公司,为朋友们服务。   第27集 
        舒克贝塔航空公司成立; 
        首航运送客人遇故障   
        飞行员舒克和坦克兵贝塔决定成立舒克贝塔航空公司,他俩在一条小河旁选择了一块平地作为机场,朋友们都来帮助舒克和贝塔建造机场。蚂蚁挖地基,小麻雀运木料,咪丽盖塔台和候机室,小蜜蜂为餐厅准备食物。 
        经过一个月的努力,机场终于竣工了。漂亮的塔台矗立在停机坪和跑道之间,宽敞明亮的候机室四周绿草如茵。舒克的直升机停在停机坪上,机身在阳光的照射下闪闪发光。 
        舒克担任飞行员,贝塔担任地面指挥,航空公司还缺机械师和空中小姐,舒克和贝塔决定招聘机场工作人员。 
        招聘广告贴出后,有70多只小老鼠前来报名。 
        他们早就听说舒克和贝塔的人名,很是羡慕,也想摆脱“小偷”的坏名声。 
        舒克和贝塔坐在办公室里,他们见到这么多同胞,挺高兴,他们想帮助所有的老鼠同胞改邪归正,靠自己的劳动生活。 
        “咱们好像不需要这么多工作人员吧?”贝塔问舒克。 
        舒克感到为难,可他不忍心拒绝同胞。 
        “咱们算算。”舒克同贝塔商议,  “机务人员五个,空中小姐四个,扫跑道的八个,餐厅厨师两个……” 
        七算八算,70多名老鼠都收下了。经过智力测验,舒克和贝塔给他们分了工,有的当地勤机务人员、有的当空中小姐,有的当货运员,还有清洁工、广播员、警卫… 
        舒克为机务人员举办了训练班,教给他们怎么修理和维护直升机。五名机务人员中,舒克挑选了一个叫臭球的小老鼠担任机械师,全面负责直升机的维修。舒克觉得臭球脑子灵活,一点就通。 
        经过一个月的准备,这天,舒克贝塔航空公司正式营业。首航运送七只小松鼠去远方探亲。 
        机场上一派繁忙景象:餐厅往飞机上运送饮料,机务人员最后一次检查飞机,空中小姐招呼旅客登机,贝塔手持话筒站在塔台上,舒克钻进飞机驾驶舱。 
        朋友们都赶来祝贺舒克贝塔航空公司首航。咪丽和哥哥挥舞着彩旗,蚂蚁王和蜜蜂皇后送来精美的食品,小麻雀要为直升机护航。 
        “起飞!”贝塔发令。 
        直升机的螺旋桨慢慢旋转起来,它越转越快,逐渐形成一股巨大的魔力,把直升机拉上了天空。 
        “请各位系好安全带。”空中小姐关照旅客。 
        小松鼠们还是头一次坐飞机,他们感到新鲜和有趣,有一只老松鼠有点儿害怕,他问空中小姐会不会有危险。 
        “没问题,绝对安全!是我亲自检查的飞机。”臭球机械师走过来说。 
        松鼠点点头,放心了。 
        直升机朝目的地飞去。 
        舒克不时同贝塔保持联系。 
        “请报告飞行状态。”贝塔问。 
        “一切正常。”舒克回答。 
        就在这时,舒克忽然看见仪表盘上的发动机转速指示表的指针来回抖动,紧跟着,飞机开始急剧下降。 
        “不好,发动机停车!”舒克大喊一声。 
        “快采取措施迫降!”贝塔急了。 
        机舱里乱成一团,旅客们吓得面如土色,他们很清楚和飞机一同掉到地上是什么后果。空中小姐们也慌了,尽管她们已乘坐过几次飞机作适应性训练,可万万擞想到头一次飞航班就遇上了空难。空中机械师臭球后悔没带降落伞。 
        舒克打开了应急开关。想重新起动发动机,无效。飞机像秤砣一样往下掉。 
        “让飞机挂在树上!”舒克仗着自己经验多,大胆地操纵飞机朝一棵大树降下去。舒克明白,要是头一次飞行就出大事故,那以后谁也不敢乘坐舒克贝塔航空公司的飞机了。舒克要不惜一切代价保住公司的声誉。 
        飞机飞速下坠……   第28集 
        直升机脱险; 
        臭球机械师被停职; 
        首航成功   
        直升机迅速下坠,机舱里一片混乱。 
        飞行员舒克沉着地操纵飞机朝一棵大树降下去,飞机奇迹般地挂在了树枝上。 
        松鼠旅客们拥到机舱门口,拼命想往出挤,可谁也打不开门。 
        “别挤!别挤!”空中小姐喊道。她打开舱门,让松鼠们离开飞机。 
        松鼠们站到树枝上,都松了一口气。 
        “我以后再也不坐飞机了。”一只松鼠说。 
        “真可怕!”另一只松鼠说。 
        “谢天谢地,舒克的驾驶技术真不错。”又一只松鼠说。 
        “咱们怎么回家呀?”一只松鼠望着这陌生的地方,说。 
        这时,舒克在机舱里把臭球机械帅和空中小姐召集到一起,说:“咱们抓紧时间排除发动机故障,然后把旅客送到目的地。” 
        “飞机挂在树上,怎么排除故障?”臭球机械师说,“这飞机太老了,该换新的了。” 
        “别说挂在树上,我在烟筒里还开过飞机呢!”舒克瞪了臭球一眼。 
        臭球不吭气了,他知道舒克的厉害。 
        “你去照看旅客。”舒克对空中小姐说。“让他们放心,飞机一会儿就能修好。” 
        空中小姐离开机舱。 
        这时,话筒响了: 
        “舒克!舒克!我是贝塔,我是贝塔!请回答!” 
        舒克戴上耳机。 
        “我是舒克!我是舒克!请讲。” 
        “飞机现在何处?” 
        “发动机停车,我把飞机迫降在一棵大树上。乘员无伤亡。正准备排除故障。”舒克说。 
        “随时报告情况,祝你顺利。” 
        舒克摘下耳机,同臭球机械师一起检查发动机。 
        “起飞前你检查过发动机吗?”舒克一边开发动机的盖一边问臭球机械师。 
        “检查了好几遍。”臭球毫不含糊地说。 
        舒克从发动机里找出一把改锥。 
        “这工具是你丢在发动机里吧?”舒克的脸一沉。 
        臭球机械师傻眼了。的确,这改锥是他的,干活时马虎,留在发动机里了。 
        “胡闹!差点儿把大家的命都送了!”舒克火了,训斥臭球机械师。 
        “我……”臭球机械师无言以对。他总觉得自己聪明,是从几十位同胞中选拔出来当机械师的,没想到头一次出航就栽了。 
        “不让你当机械师了,回机场后,扫跑道去!”舒克解除了臭球的机械师职务。 
        臭球没意见,谁让自己粗心大意呢!他赶紧帮助舒克更换损坏的发动机零件。 
        经过一个小时的工作,发动机修好了。 
        直升机得从树枝上起飞,这很危险。舒克让臭球离开飞机。臭球不干。 
        “为什么不离开?”舒克问。 
        “要死一块死。”臭球说。 
        舒克心说,别看这小子粗心,还挺仗义,就同意了。他让臭球观察四周的情况。 
        空中小姐把旅客都领到地面上。 
        舒克按下了启动按钮。螺旋桨转起来了。树叶纷纷落下,树枝剧烈地摇晃着。 
        直升机挣脱了树枝的束缚,升到了空中。舒克驾驶飞机在空中转了一圈儿,平稳地落在地上。 
        臭球打开舱门,招呼旅客上飞机。可松鼠们谁也不敢上,都怕再出事。 
        “请你们相信我。”舒克对旅客说。 
        松鼠们望着舒克真诚的目光,他们感到必须信任舒克,必须信任这种目光。 
        乘客们登上了飞机。空中小姐关好舱门。 
        舒克戴上耳机,向贝塔请示:“飞机故障已排除,请求起飞。” 
        耳机里传来贝塔的声音: 
        “同意起飞!” 
        直升机稳健地升到空中,朝目的地飞去。 
        空中小姐给乘客端来了汽水。松鼠们一边喝饮料一边观看窗外的景色。 
        经过一个多小时的飞行,直升机终于到达了目的地。松鼠们的亲戚早就等在那里了。 
        舒克、臭球和空中小姐帮助旅客把行李搬下飞机。旅客依次紧紧握着舒克的手,感谢他临危镇静,保证了大家的安全。 
        舒克为航空公司赢得了荣誉和信任。 
        臭球惭愧地低下了头。 
        当直升机降落在舒克贝塔航空公司的机场时,受到机场全体工作人员的热烈欢迎,大家像迎接凯旋的勇上一样迎接舒克。 
        “我去扫跑道。”臭球小声对舒克和贝塔说。 
        “我看还让他当机械师吧!”舒克说。他相信,臭球不会再马虎了,他是聪明人,不会在同一件事上犯两次错误。 
        贝塔同意了。 
        臭球机械师乐了。   第29集 
        冰淇淋和牛奶; 
        舒克去奶牛场; 
        海盗袭击舒克和臭球   
        舒克贝塔航空公司白开航以来,十分繁忙。机场每天都是热闹非凡,旅客进进出出,飞机时起时落。 
        候机大厅里坐着许多等候登机的旅客。天气炎热,旅客们感到口渴。 
        一位刺猬旅客找到贝塔。 
        “我提个建议。”刺猬说。 
        “欢迎。”贝塔请刺猬坐在沙发上。 
        “候机大厅应该开设一个冷饮部。”刺猬说。 
        “这个建议很好。”贝塔同意了。 
        贝塔拨通了机场餐厅的电话。 
        “喂,是罗丘吗?”贝塔问。罗丘是餐厅主任。 
        “我是。”罗丘说。 
        “我是贝塔。你会制作冰淇淋吗?” 
        “冰淇淋?没做过。” 
        “候机大厅要开设一个冷饮部,这事交给你办,快点儿试试做冰淇淋或雪糕什么的。” 
        “是。” 
        餐厅主任罗丘放下电话,把手下的人召集到一起。 
        “谁会制作冰淇淋?”罗丘问。 
        “我吃过,真好吃。”一只小老鼠抹抹嘴。 
        “味很好,特甜。”端盘子的老鼠姑娘说。 
        “我不是问好吃不好吃,是问谁会做。咱们要开一个冷饮部。”罗丘说。 
        “听说冰淇淋要放牛奶。” 
        “还有(又鸟)蛋。” 
        “还得有冰箱才行。” 
        罗丘拿起电话听筒。 
        “是贝塔吗?做冰淇淋需要牛奶和(又鸟)蛋,可我们没有牛奶,也没有(又鸟)蛋。”罗丘说。 
        “我同舒克联系一下,让他去搞。”贝塔说。 
        贝塔放下电话听筒,问导航员:“舒克现在在哪里?” 
        “在飞往黑山寨的途中。” 
        “我和他通电话。” 
        导航员要通舒克。 
        “舒克,舒克,我是贝塔。” 
        “我是舒克,请讲。” 
        “咱们的候机大厅要增设一个冷饮部,需要牛奶和(又鸟)蛋,你能不能设法弄一些来?” 
        “行,我想想办法。” 
        “祝平安!” 
        “谢谢。” 
        舒克一边开飞机一边把臭球机械师叫到驾驶舱来。 
        “你知道哪儿有奶牛场吗?”舒克问。 
        “干吗?”臭球不明白。 
        “咱们的机场要设冷饮部.需要牛奶和(又鸟)蛋。”舒克调整了一下飞机的方向。 
        “做冰淇淋用?”臭球挺精通。 
        “对。”舒克点点头。 
        “我知道奶牛场在哪儿。”臭球朝地面望去。“翻过前边那座山,山脚下有座奶牛场。” 
        “咱们先把旅客送到目的地,再去奶牛场。”舒克说。 
        直升机穿过白云,穿过蓝天。 
        送完旅客,舒克驾驶直升机朝奶牛场飞去。 
        “就是那座山。”站在舒克身边的臭球机械师给舒克指路。 
        直升机飞临山旁,在奶牛场上空盘旋。 
        “你看,有多少奶牛!”臭球机械师指指下边,“那些铁桶里都是牛奶。” 
        “着陆。”舒克一推驾驶杆,直升机笔直地下降。 
        “注意观察地面!”舒克吩咐臭球。 
        臭球把脸贴在窗玻璃上,往下看。 
        “就在这座房子后边的草丛里着陆。”臭球对这一带还挺熟悉。 
        舒克操纵直升机平稳地降落在草丛里。 
        “我去弄牛奶。”臭球边说边离开驾驶舱。 
        “怎么弄?”舒克叫住了臭球。 
        “拿呀!”臭球机械师说。 
        “不行。那叫偷。”舒克皱了皱眉头。 
        “那你说怎么办?”臭球机械师一摊手。 
        “去跟奶牛要。”舒克说。 
        “老鼠跟奶牛要牛奶?笑话,人家才不会给呢!”臭球机械师觉得舒克太天真。 
        “咱们一起去。”舒克说完把空中小姐叫过来,“你看守飞机,把舱门从里边锁好,除了我们俩,谁来也别开门。” 
        空中小姐点点头。 
        臭球机械师从货舱里找了两个小桶,然后和舒克下了飞机。 
        他们沿着墙角往牛栏走。 
        “当心点儿,屋里有人。”臭球机械师提醒舒克。 
        舒克蹑手蹑脚地朝牛栏走去,臭球机械师同他保持着距离。 
        一头小奶牛先发现了舒克,她忙告诉妈妈。 
        “妈妈,老鼠又来了!”小奶牛叫道。 
        奶牛们顿时警惕起来,她们恨老鼠。老鼠经常来偷喝牛奶。 
        “你们好!”舒克站在牛栏外面说。 
        “还假装有礼貌呢!”一头奶牛撇撇嘴。 
        “黄鼠狼给(又鸟)拜年,没安好心。”另一头奶牛说。 
        “你们误会了,我是飞行员舒克,是舒克贝塔航空公司的飞行员,不是小偷。”舒克说。 
        “老鼠能当飞行员?”小奶牛不信。 
        “你们看,他还真穿着飞行服呢。”一头见过世面的奶牛说。 
        “说不定,是海盗他们耍的新花招儿。”另一头奶牛提醒大家。 
        “海盗?”舒克觉得好玩,这大山里哪来的海盗? 
        “海盗是一只老鼠的名字,他是这一带的老鼠头儿,很坏。”小奶牛说。 
        “别理他,他是装傻呢!”小奶牛的妈妈对女儿说。 
        “我跟海盗根本不认识。再说一遍,我是舒克贝塔航空公司的飞行员,我们机场要开设冷饮部,耍做冰淇淋,需要牛奶,一点儿就够。”舒克拍拍手中的小桶。 
        “什么叫冰淇淋?”小奶牛好奇地问。 
        “冰淇淋…就是……”舒克没吃过。 
        “冰淇淋就是白的……也有黄的,凉凉的,软软的,甜甜的那么一种食物,很好吃。”臭球机械师有幸吃过。 
        “机场和冰淇淋有什么关系?”一头奶牛问。 
        “就足,难道你们的飞机是靠冰淇淋作燃料飞行的吗?”见过世面的奶牛问。 
        “现在天气太热,旅客吃些冷饮,就凉快了。”舒克解释道。 
        “你们的旅客都是老鼠吗?”小奶牛问。 
        “你们把全世界的老鼠运来运去,这不是提供作案工具吗?”见过世面的奶牛还真掌握不少名词。 
        “我们的旅客有老鼠,可大多数是小动物,像松鼠啦,刺猬啦,蜗牛啦……再说,老鼠也不全是坏蛋。”舒克有些不耐烦了。 
        “妈妈,给他们一点儿牛奶吧,我看舒克不像坏蛋。”小奶牛的直觉起作用了。 
        女儿的话妈妈总是听的,奶牛们商量了一下,决定给舒克两小桶牛奶。 
        “真有你的!”臭球佩服舒克。 
        舒克和臭球机械师谢过奶牛们,拎着两桶牛奶朝飞机走去。 
        他俩拐过墙角,只听一声人喝:  “站住!” 
        舒克抬头一看,几十只老鼠把他和臭球围住了。 
        “干吗?”舒克预感到不妙了。 
        “干吗?这是我的地盘,谁让你们来的?收获还不小呀!”一只蓝眼睛的老鼠冷笑着说。 
        “你是谁?”舒克问。 
        “说话注意点!这是我们的大王,绰号海盗,威震天下。”一只老鼠说。 
        “我看你刚才对奶牛说话挺懂礼貌嘛,怎么,对自己的同胞倒不讲礼貌了?噢,对坏蛋是不能讲礼貌的,这样才能显出你好来,对吧?”海盗一边嚼着半根香肠一边说。 
        舒克感到这个对手不一般。 
        一只海盗的部下从臭球手中抢过牛奶桶,递给海盗。海盗一仰脖,喝了个痛快。 
        臭球气得直咬牙,无奈,寡不敌众。   第30集 
        海盗奔袭舒克贝塔航空公司; 
        舒克和贝塔大战海盗及其喽罗; 
        直升机把海盗吊到空中   
        海盗喝足了牛奶,抹抹嘴,问舒克:“刚才你和奶牛说,你是什么航空公司的飞行员?吹什么牛!不过你真有两下子,大模大样就骗来两桶奶,比我们高明!” 
        “报告大王,草丛里真有一架飞机!”一个小喽罗跑来禀报。 
        “噢?”海盗用异样的眼光看了看舒克,转身去草丛里看飞机。 
        舒克冲臭球使了个眼色,臭球撒腿往东跑,舒克往西跑。 
        “抓住他们!”小喽罗们喊起来。 
        海盗的部下太多了,舒克和臭球又被抓回来,这次是五花大绑。 
        海盗来到舒克面前。 
        “我要接管你们的飞机场,同意吗?不同意?那我就烧了你的飞机!如果同意,现在就运我们去。”海盗对舒克下了最后通牒。 
        舒克点点头,他不能眼看着海盗烧了他心爱的飞机。只要到了空中,就是舒克的天下,会有办法打败海盗的。 
        臭球机械师不解地看了舒克一眼,他明白,这一群强盗乘飞机降落在机场,毫无准备的贝塔和整个机场都会成为海盗们的俘虏。 
        舒克朝臭球使个眼色,示意他别胡来。 
        “给他俩松绑。”海盗下令。 
        舒克来到直升机跟前,叫空中小姐开门。 
        飞机舱门打开了,海盗和部下们一拥面上。舒克走进驾驶舱,臭球机械师检查发动机。空中小姐拒绝为海盗们服务,她躲进货舱。 
        海盗走进驾驶舱,他被仪表弄得眼花缭乱,由此倒生出几分对舒克的敬畏之意。 
        “起飞吧!”海盗下令。 
        “驾驶舱不能进外人,请去客舱。”舒克说。 
        “噢,我才不是傻瓜,你好想往哪儿开就往哪儿开,不行,我得看着你。”海盗说。 
        “你会看罗盘吗?你会看航行图吗?”舒克指指仪表盘上的罗盘表,又指指航行图,“不会看这个,上了天你连东西南北也分不清。” 
        海盗看看罗盘,义看看航行图,乖乖地回客舱了。 
        “发动机正常吗?”舒克问臭球机械师。 
        “一切正常,可以起飞。”臭球盖上发动机罩,钻进飞机。 
        舒克按下启动按钮,直升机升到空中。海盗和喽罗们惊叫起来,他们感到新奇,纷纷趴在窗口往外看。 
        “肃静!”海盗叫着,“听着,飞机一降落,你们马上冲下去,占领机场!” 
        海盗像个军事指挥官,给部下分工。 
        舒克回手关好驾驶舱的门,悄悄接通了电台。 
        “贝塔!贝塔!我是舒克!我是舒克!请回答!请回答!”舒克小声呼叫。 
        “我是贝塔。我是贝塔。请讲。”贝塔回话。 
        “飞机现在被一群老鼠强盗占领了,他们现在乘飞机去机场,要占领机场,请作好战斗准备。” 
        “他们有多少?”贝塔问。 
        “27只。”舒克早数好了。 
        “放心吧,我的坦克都呆烦了。”贝塔挂上耳机,拉响了警报。 
        机场各部门的负责鼠都来到贝塔的办公室。 
        “有一伙强盗乘飞机马上来咱们机场,大家赶紧作好准备。你带部下守住候机大楼;你带部下埋伏在停机坪四周;你带部下作好增援准备……”贝塔布置任务。 
        整座机场都忙碌起来,好多旅客也加入了保卫机场的行列。 
        贝塔来到车库,钻进他心爱的坦克。坦克里有充足的炮弹。贝塔把坦克开到停机坪旁的草丛里隐蔽起来。 
        直升机出现在机场上空。 
        整座机场鸦雀无声,只有飞机的发动机声。 
        海盗走进驾驶舱。 
        “你刚才同贝塔的通话我都听见了,大概你还不清楚我的部下的力量。来人!”海盗大喝一声。 
        一个小喽罗走进驾驶舱。 
        “把这根铁棍子窝成圆圈儿。”海盗发话。 
        小喽罗轻而易举地把一根铁棍子窝成了圆圈。 
        舒克愣住了。 
        “他们都会气功,你的同伙是打不过我们的,哈哈!”海盗得意极了。 
        舒克真想一推驾驶杆,来个机毁鼠亡。 
        就在这时,舒克看见草丛里的坦克。他在心里笑了。海盗的部下绝对打不过贝塔的坦克。 
        “做好准备!”海盗回到客舱,向部下发令。 
        小喽罗们个个摩拳擦掌.臭球机械师和空中小姐已被捆了起来塞进货舱。 
        直升机徐徐降落了。螺旋桨还在旋转,海盗就打开舱门,率领部下冲出飞机。 
        埋伏在停机坪四周的机场工作人员呼喊着朝强盗们包围过来。 
        海盗一挥手,小喽罗们四面迎战。 
        机场上作人员不是这伙强盗的对手,已有两名工作人员被摔倒在地上。 
        贝塔的坦克冲出草丛,朝强盗们撞去。 
        海盗弄不清坦克的威力,犹豫之间,已被坦克撞了个跟头。 
        只见他大喊一声,招呼过来几个部下,一起向坦克冲去。 
        贝塔腊准了其中一个小喽罗开炮。 
        炮弹射中了小喽罗的耳朵。耳朵被削去了一半,疼得他大叫不止。 
        毕竟是海盗,凶猛顽固。海盗命令一部分喽罗围攻坦克,另外一部分跟他去占领候机大楼。 
        这回贝塔傻眼了,他不能把坦克分成两辆。 
        在飞机上观战的舒克灵机一动,他跑进货舱放出臭球机械师和空中小姐。 
        “你们作好准备,货舱里有一箱子弹,咱们从空中打击他们。”舒克说完发动飞机。 
        直升机起飞了,擦着地面追赶企图去占领候机大楼的海盗们。 
        臭球机械师把子弹箱扛来了。直升机上有皮皮鲁安装的弹弓枪。 
        直升机追上海盗了。 
        “开火!”舒克命令。 
        臭球机械师接过空中小姐递来的子弹,装进弹弓枪,瞄准海盗的后脑勺.抠动了扳机。 
        打偏了,子弹擦着海盗的脑袋飞过去,打倒了他旁边的一个喽罗。 
        “瞄准海盗打!”舒克懂得擒贼先擒王。 
        臭球又装了一发子弹。 
        还是没打中。海盗真狡猾,拐着弯跑。 
        眼看海盗就要冲进候机大楼了。舒克急了,他要用飞机的起落架压海盗。 
        直升机擦着海盗的头飞。海盗一会儿往左躲,一会儿往右躲,飞机就是压不着他。 
        舒克吸了一口气,撞撞运气,这回就往左落。 
        飞机在海盗的头上飞。海盗知道飞机要从上往下压他,他突然往左一闪。上帝保佑,舒克也是往左一落,起落架牢牢地把海盗压在地面上。 
        “饶命!饶命!!”海盗吓坏了,只要舒克让飞机全部落地,海盗就一命呜呼了。 
        舒克见海盗的两只手死死抓住起落架,他突然一拉杆,直升机拔地而起,把海盗带上了天空。 
        海盗不敢松手。飞机飞得越高他越不敢松手,可又义爬不上去,就这样被吊在空中。 
        “哈哈,太棒啦!”臭球机械师乐了,他打开机舱门,跷着二郎腿逗海盗: 
        “累了吧?头儿!这叫健美锻炼,专练臂力肌肉。你不足会气功吗吗?” 
         海盗的威风全没了。 
        “舒克,来个急转弯,练练他的功。”臭球大声喊。 
        “好吧!”舒克操纵直升机来了个急转弯。 
        海盗的身体被风吹得和飞机平行了。 
        “再来一个俯冲!”臭球愈发得意,他要出出被绑的气。 
        “再来一个急降!”臭球还挺懂飞行姿态。   第31集 
        舒克贝塔航空公司战胜海盗; 
        贝塔想学开飞机; 
        海盗越狱逃跑   
        舒克驾驶直升机将海盗吊到空中,尽情地折腾他,眼看着海盗的臂力不支了,舒克把直升机悬停在空中。 
        “干吗不继续折腾他?”臭球问。 
        舒克不忍心把海盗从天上摔下去。 
        臭球钻进后舱,找出一根铁棍子。 
        “你要干什么?”舒克从驾驶舱探出头问。 
        “我把这强盗打下去。”臭球说完抡起棍子要往机舱外边打。 
        “住手!留着他有用。”舒克大喝一声。 
        臭球的棍子在空中停住了。空中小姐走过来夺走臭球手中的棍子。 
        舒克打开电台。 
        “贝塔,贝塔,我是舒克,请回答!”舒克呼叫。 
        “我是贝塔,我是贝塔,请讲!’贝塔在坦克里说话。 
        “你让海盗的部下马上投降,否则我就把他们的头儿从天上扔下去!”舒克说。 
        “明白。”贝塔关上电台,打开坦克舱盖儿,将头探出坦克。 
        机场上战斗仍在继续,海盗的喽罗们还挺顽固。 
        “海盗的部下们!”贝塔大声喊话,“你们往天上看!你们的头儿正吊在空中。如果你们不投降,我们就把他从天上扔下来!” 
        海盗的部下往上一看果然看见首领被吊在空中,他们只好纷纷投降。 
        贝塔吩咐将俘虏集中到一起,关进机场的库房。 
        “舒克,舒克,我是贝塔,战斗已经结束,请你着陆。”贝塔站在塔台上说。 
        “明白!请你布置人马准备活捉海盗!”舒克边说边操纵直升机下降。 
        停机坪上严阵以待。 
        海盗的两条腿还没落地,就被捆了起来。 
        直升机着陆了。舒克出现在机舱门口,大家像欢迎凯旋的英雄那样冲舒克欢呼鼓掌。 
        贝塔和舒克紧紧拥抱。 
        臭球把五花大绑的海盗关进仓库旁的一间小黑屋。 
        “怎么处置他们?”贝塔问舒克。 
        舒克一时答不上来。处死了于心不忍,都是同胞。放走?他们又要去干坏事。留下?不敢。 
        “你说呢?”舒克问。 
        贝塔耸耸肩膀,也想不出办法。 
        “先关几天,”舒克扭头叫来餐厅部主任:“派人给他们送点儿食物。” 
        餐厅部主任罗丘点点头,他忽然想起了什么,问舒克:“牛奶弄到了吗?我还等着做冰淇淋呢!” 
        舒克这才想起牛奶被海盗喝光了。 
        “我现在就去弄。”舒克转身朝飞机走去。 
        “歇会儿,你得吃点儿东西。”贝塔拉着舒克朝餐厅走去。 
        餐厅里有不少旅客在用餐,他们因海盗袭击机场而延误了起飞时间。 
        舒克看见这么多旅客滞留在机场,他隐隐约约感到光靠一架直升机运载旅客已力不从心。 
        贝塔给舒克端来一份丰美的饭菜。闻到香味儿.舒克才发现自己早就饿了。 
        “你去通知旅客,下一班次马上起飞。我送完这批旅客就去弄牛奶。”舒克边吃边对贝塔说。 
        “我看我该学开飞机了。”贝塔心疼舒克。再说,整个航空公司就一个飞行员,也显得少了点儿。 
        “过几天我教你。”舒克没意见。 
        半小时后,舒克的直升机满载着旅客起飞了。 
        贝塔坐在塔台里随时同舒克保持联系,不敢有一点儿疏忽。 
        两个小时后,直升机平安返航了。舒克将两桶牛奶递给餐厅主任罗丘。 
        “快去做冰淇淋吧。”贝塔对罗丘说。 
        “好,马上做,”罗丘拎着两桶牛奶走了。 
        “从明天开始,你教我开飞机。”贝塔坐在沙发上说。 
        “行。”舒克疲劳地躺在长沙发上。 
        臭球一阵风似地跑进舒克的办公室。 
        “海盗跑了!”臭球报告。 
        舒克和贝塔“腾”地从沙发上蹦起来。 
        “怎么跑的?”舒克不信,海盗被捆得很结实。 
        “绳子都断了,他把部下也都放跑了。”臭球后悔当初没有把海盗从天上扔下来。 
        舒克和贝塔感到了海盗的厉害。 
        “赶快搜索机场,加强直升机的警卫。”舒克下令。 
        “是!”臭球跑出去。 
        天已经黑了,探照灯在机场上扫来扫去。工作人员搜索机场的每一个角落。   第32集 
        舒克带罗丘去城里学做冰淇淋; 
        直升机落在冷饮店房顶上; 
        罗丘遇险   
        机场上没有海盗和他的喽罗们的踪影。 
        “没有潜伏在机场,确实跑了。他们大概也被坦克和飞机吓破胆了。”臭球分析。 
        “也许。”舒克点点头。 
        “咱们去看看冰淇淋。”臭球念念不忘。 
        舒克、贝塔和臭球来到餐厅,只见罗丘主任正冲着桌上的一个方盒子皱眉头。 
        “冰淇淋做好了?”贝塔凑过去看。 
        只见方盒子里冻着一块白颜色的冰块,硬得啃都啃不动。 
        “做不成。”餐厅主任泄气了。 
        “我带你去城里学学。”舒克拍拍罗丘的肩膀。 
        他觉得机场开设个冷饮部还是很必要的,何况正是为做冰淇淋才同海盗开了战,要是做不成冰淇淋,岂不太亏了。 
        “什么时候去?”罗丘来情绪了,他只要看一遍冰淇淋的制作过程,就能学会。 
        “趁着天黑,现在就去。”舒克也来劲了,他好久没进城了。 
        “我也去。”贝塔想去看看咪丽。 
        “你得在家值班。等你学会了开直升机,就可以自己进城了。”舒克说。 
        贝塔无奈,只得留在机场。 
        舒克、罗丘和臭球朝直升机走去。臭球打开发动机盖,检查发动机。 
        舒克和罗丘钻进机舱。 
        “怎么样?”舒克从驾驶舱伸出头来问臭球机械师。 
        “一切正常,可以起飞。”臭球盖好发动机盖,也钻进飞机。 
        “报告塔台,请求起飞。”舒克请示贝塔。 
        “可以起飞,注意安全,随时保持联系。”贝塔回答。 
        机场上灯火通明。 
        舒克好久没飞夜航了,他很兴奋。发动机开始运转,螺旋桨开始旋转,机身开始离地。整个机身转着圈地升到空巾,径直朝城市飞去。 
        罗丘和臭球把鼻子贴在舷窗上往下看。 
        “这是电影院。这是体育场。这是商店。”臭球对城市建筑挺精通。 
        “冷饮店!”罗丘喊道。 
        舒克往下一看,一座灯火闪烁的冷饮店出现在机身下方,店门口人来人往,热闹非凡。 
        “注意,飞机降落!”舒克告诉机上人员。 
        直升机缓缓地在冷饮店屋顶上着陆了。 
        “臭球,你看守飞机,我和罗丘去看看。”舒克说。 
        “嗯。”臭球不大情愿地点点头,他也想看冰淇淋是怎么做出来的。 
        舒克和罗丘沿着下水管道钻进冷饮店。他们来到冷饮店后边,这里是做冰淇淋的地方。几个穿白大褂的人在做冰淇淋。 
        舒克看见一张桌子上有一堆瓶瓶罐罐,他和罗丘躲列瓶瓶罐罐的后边,这里视野开阔,能看到整个房间。 
        罗丘的眼睛直勾勾地盯着那几个做冰淇淋的人。 
        一个人正往盆里打(又鸟)蛋。一个胖胖的人走过来。 
        “打这么多(又鸟)蛋!”胖子有些不满。 
        “经理,按规定做50公斤冰淇淋就得放这么多(又鸟)蛋。”打(又鸟)蛋的人说。 
        “以后少放一半儿(又鸟)蛋!”胖经理说。 
        “这……” 
        “人家吃不出来!”胖经理又对放牛奶的人说:“牛奶也要少放。多放色素,多放糖精。” 
        舒克和罗丘相互看看,无话可说。他们都知道糖精不是好东西,牛奶和(又鸟)蛋是好东西。舒克替门口那些掏钱买冰淇淋的人担心。 
        罗丘把制作冰淇淋的全部过程都记在心里。 
        “咱们走吧。”舒克一转身,碰翻丁桌上的一个小瓶子。瓶子滚到地上,碎了。 
        响声惊动了屋里的人,他们的视线“刷”地扫向桌子上。 
        “快跑!”罗丘和舒克撒腿就往外跑。 
        “老鼠!抓老鼠!!”人们喊起来。 
        一阵杂乱的脚步声尾随着舒克和罗丘。 
        舒克和罗丘跑进营业大厅,许多顾客在吃冷饮。人们一听说老鼠,纷纷站起来观察自己脚下。 
        “分头跑,你往左,我往右,到房顶集合!”舒克冲罗丘喊。 
        罗丘朝左边跑去。舒克往右边跑。 
        “堵住门口,别让它跑了!偷吃我的食物,真可恶!”胖经理怒不可遏。 
        顾客们齐心帮着店员抓老鼠。 
        舒克想起胖经理少往冰淇淋里放(又鸟)蛋的事,他觉得胖经理和老鼠差不多,可人却不抓他。 
        舒克毕竟是经验丰富,他成功地绕过无数只脚,逃出丁冷饮厅。 
        臭球正躺在飞机里睡觉呢,他被舒克剧烈的砸门声惊醒了。 
        “罗丘设有回来?”舒克劈头便问。 
        “罗丘?”臭球揉揉眼睛。 
        “糟糕!”舒克扭头就走。 
        “等等,出了什么事?”臭球抓住舒克问。 
        “罗丘大概被人抓住了!我去救他,你快同贝塔联系。”舒克说完便消失在夜色中。 
        果然,罗丘被人抓获,他被关在一个铁笼子中,全身打着哆嗦。   第33集 
        罗丘脱险; 
        舒克到皮皮鲁家作客; 
        舒克贝塔航空公司增添大型喷气客机   
        舒克眼见罗丘被人抓住,他束手无策。就在这时,舒克看见冷饮店门口出现了一个熟悉的身影。 
        舒克一愣,是皮皮鲁! 
        自从和皮皮鲁分手后,舒克经常想起他,他感激皮皮鲁对他和贝塔的友情。 
        舒克跑过去拽皮皮鲁的裤腿。 
        “我是舒克!”舒克扯着嗓子喊。 
        “舒克?”皮皮鲁惊讶。他蹲下去,借着灯光一看,果然是舒克。 
        舒克把自己来到城里以及罗丘怎么被抓住等等统统告诉了皮皮鲁。 
        “你藏在我兜里,我去救罗丘。”皮皮鲁把舒克装进口袋里,走进冷饮店。 
        “把这老鼠交给我吧,由我来处决他!我家有只猫。”皮皮鲁对大家说。 
        没有人反对。 
        皮皮鲁掀开铁笼子,用手抓起罗丘。 
        “这孩子,用手抓老鼠。” 
        “真不讲卫生!” 
        “……” 
        人们议论纷纷。 
        皮皮鲁旁若无人地走出冷饮店。 
        “谢谢你!皮皮鲁。”舒克感激地说。 
        “到我家去歇会儿吧!咱们得好好聊聊。”皮皮鲁邀请舒克。 
        “行!”舒克同意了,“告诉我怎么走,我开飞机去。” 
        皮皮鲁将方位和标志告诉舒克。 
        “你先走吧,我们随后就到。”舒克和罗丘说完顺着下水管爬上屋顶。 
        臭球正在驾驶舱里同贝塔通话。舒克接过话筒。 
        “贝塔,我是舒克,罗丘已脱险。” 
        “太好了,马上返航!,’ 
        “我先去趟皮皮鲁家。” 
        “皮皮鲁?你见到皮皮鲁了?” 
        “回去详谈。” 
        “注意安全。” 
        “明白。” 
        舒克摘下耳机,准备起飞。 
        臭球给罗丘包扎伤口。罗丘身上受了几处伤,是被人用扫帚砸的。 
        直升机准确地降落在皮皮鲁家的阳台上。皮皮鲁已经在阳台上恭候舒克了。 
        一顿丰盛的晚餐等待着舒克和伙伴们。舒克把臭球介绍给皮皮鲁。 
        舒克一边吃一边给皮皮鲁讲舒克贝塔航空公司的故事。 
        “一架直升机就能开航空公司?”皮皮鲁撇撇嘴。 
        “是少了点儿。旅客多,飞机少。”舒克承认。 
        “我送你们一架大型喷气式客机。”皮皮鲁说完从书柜里拿出一架极豪华的玩具大型客机,“这是我过生日时,舅舅送我的。” 
        “这……”舒克有点儿不好意思。 
        “送给你!放在我这儿也没用。”皮皮鲁豪爽地说。 
        “可我不会开呀!”舒克望着巨大的喷气机,为难地说。 
        “你当了这么长时间飞行员,大同小异,明天咱们到楼顶上的大平台试飞。”皮皮鲁说。 
        “谢谢你!”舒克顾不上吃饭了。 
        “咱们参观参观。”臭球提议。 
        “进去看吧!”皮皮鲁打开机舱门。 
        舒克、臭球和罗丘走进客舱,整座客舱富丽堂皇。绿色的地毯,舒适的高背椅,冷气设备,灯光设备、音响设备……靠近驾驶舱的是头等舱,头等舱里设备更齐全。 
        “还有二楼呢!”臭球指指上边。 
        “这是仿造波音747飞机。”皮皮鲁在外边告诉舒克他们。 
        “咱们把好消息告诉贝塔。”舒克走进驾驶室,打开电台。 
        “贝塔,贝塔,我是舒克,请回答。” 
        “我是贝塔,请讲!” 
        “皮皮鲁送给咱们一架大型喷气客机,我们现在在客机上同你讲话。” 
        “真的?太好啦!” 
        “我想给这架飞机定名为皮皮鲁号,行吗?” 
        “同意!” 
        “请你马上组织扩建机场跑道,我明天下午驾驶皮皮鲁号试航!” 
        “明白!”   第34集 
        皮皮鲁号安全抵达机场; 
        舒克和贝塔决定拍电影   
        贝塔和舒克通过话以后,立即组织扩建机场跑道的工作。航空公司全体人员出动,将跑道长度扩容了一倍。 
        第二天上午,贝塔通过无线电告诉舒克,机场跑道扩建完毕。 
        “我今天下午驾驶皮皮鲁号返回机场。”舒克说。 
        “直升机怎么办?”贝塔问。 
        舒克这才想起还有直升机。 
        “我把直升机开回去。”臭球在一边说。 
        “你?”舒克不放心。 
        “我看都看会了。”臭球的牛劲上来了。 
        舒克想想,也只好冒这个险了。 
        “我先训练你一下。”舒克说。 
        舒克和臭球钻进直升机,舒克给臭球作示范飞行,臭球脑子不笨,一会儿就能单独飞行了。 
        皮皮鲁在一旁看着,很开心。 
        下午,皮皮鲁将喷气式客机和直升机都拿到楼顶的太平台上。 
        舒克和罗丘钻进皮皮鲁号,臭球钻进直升机。 
        大型喷气式飞机开始在平台上滑行,舒克给飞机不断加大马力。 
        飞机的机头离开地面,紧跟着,整个机身都离地了。飞机上天了。 
        臭球也操纵直升机起飞。 
        皮皮鲁向他们挥手。 
        舒克的飞机升到了空中。飞机突然开始摇晃起来,舒克觉得喷气式飞机比直升机难驾驶,他努力体会驾驶窍门。 
        “舒克,舒克,我是贝塔,请回答。”耳机里传来贝塔的呼叫。 
        “我是舒克,我的飞机现在空中。” 
        “情况怎样?” 
        “有点儿摇摆,问题不大,放心吧。” 
        “祝你成功!” 
        “你再同臭球联系一下。”舒克还想着臭球。 
        飞机开始平稳飞行了。舒克知道,关键是着陆,弄不好就会机毁鼠亡。 
        机场出现在前方,舒克紧张地握着驾驶杆,眼睛盯着下边。 
        “皮皮鲁号请求着陆。”舒克请示塔台。 
        “同意着陆。”贝塔的声音也很紧张。 
        跑道旁边停着消防车和救护车。 
        巨大的皮皮鲁号离跑道越来越近。 
        “快拉起来!快!”贝塔对着话筒大叫。 
        舒克来不及问为什么,就在飞机与跑道尚未接触的一刹那,将飞机拉了起来。 
        “你忘了放起落架!”贝塔惊魂未定。 
        舒克出了一身冷汗。他把起落架放出机舱。 
        飞机绕场一圈,第二次对准了跑道。 
        成功了,皮皮鲁号平安着陆。大家涌向这架巨大的客机,一片欢呼。 
        臭球驾驶的直升机也安全着陆。 
        罗丘跑步去餐厅制作冰淇淋。 
        舒克带大家参观皮皮鲁号,大家都被皮皮鲁号的豪华和气势惊呆了。 
        舒克和贝塔决定成立皮皮鲁号机组,他们选出了七名精干的工作人员,分别担任空中小姐和机械师,臭球担任副驾驶,舒克担任机长。 
        舒克教贝塔学会了驾驶直升机。 
        这天清晨,皮皮鲁号首航运送客人。几百名旅客依次登上飞机,他们去南方旅游。 
        皮皮鲁号满载着旅客起飞了。它昂着头,插进云端。 
        舒克定好方位,打开自动驾驶仪。 
        “你在这儿值班,我去客舱看看。”舒克吩咐副驾驶臭球。 
        “放心吧。”臭球说。 
        舒克走进客舱,看见空中小姐正给旅客分发饮料和冰淇淋。有的旅客往舷窗外看,有的在打瞌睡。 
        “应该丰富旅客的旅途生活。”舒克想。他抬头看见了悬挂在客舱前方的电影银幕。 
        舒克回到驾驶舱。 
        “贝塔,贝塔,我是舒克,请回答。”舒克打开电台。 
        “我想在飞行中为旅客放电影解闷,可咱们没有电影片子,你准备一下,咱们自己拍电影。”舒克说。 
        “拍电影?行啊。”贝塔挺兴奋,“我去筹备,谁当导演呀?” 
        “你当吧。” 
        “编剧呢?” 
        “臭球当。” 
        “制片主任呢?” 
        “……” 
        美工呢?” 
        “……” 
        “摄影呢?” 
        “行啦行啦,我看你挺内行,就都包了吧!”舒克关上电台。 
        “到了。”臭球提醒机长。舒克这才意识到大飞机没有跑道是无法着陆的。他们忘了跑道的事。 
        皮皮鲁号在空中盘旋。   第35集 
        皮皮鲁号在公路上迫降; 
        险些同大卡车相撞; 
        艾丽担任故事片编剧   
        没有跑道,喷气式客机皮皮鲁号无法着陆。舒克埋怨自己粗心大意,现在后悔也晚了。 
        飞机在空中盘旋。 
        旅客们发现飞机老在原地打转,觉出不对头了,纷纷趴在窗口往外看。 
        “你看!”臭球让舒克往下看。 
        舒克看见地面上有一条宽大的公路,公路上行进着来往的车辆。 
        舒克眼睛一亮,对,在公路上迫降。 
        公路上车辆很多,得避开它们。 
        舒克来到客舱,对旅客们说: 
        “请大家原谅,由于我们的疏忽,忘记修跑道了。现在,我们要在一条公路上降落。希望大家坐在座位上不要动,系好安全带,保持飞机平稳。” 
        没有旅客起哄,也没人谴责舒克的粗心,既然人家已经承认了错误,何况现在是生死与共。 
        舒克回到驾驶舱,操纵飞机在公路上空盘旋,等候时机。 
        终于,公路上出现了一个空白带,没有车辆。 
        皮皮鲁号对准了公路。舒克一推驾驶杆,飞机朝公路逼近…… 
        机轮挨到了地面,迫降成功。飞机在公路上滑行。 
        “注意对面!”臭球尖叫一声。 
        舒克抬头一看,对面驶来一辆大卡车。 
        操纵飞机拐弯已经来不及了,惟一的出路是从卡车下边钻过去。 
        舒克双手紧握驾驶杆,两脚踩着转向舵。不断调整着飞机的方向。 
        卡车司机显然看见了皮皮鲁号,他来了个急刹车。皮皮鲁号从卡车下边钻了过去。 
        路旁是草丛。舒克操纵飞机钻进草丛隐蔽起来。卡车司机从车上跳下来,找那架玩具飞机。他揉揉眼睛,怀疑是自己眼花看错了。 
        皮皮鲁号里一片掌声,大家庆贺迫降成功。 
        舒克操纵飞机滑行到一座小山坡旁,他觉得在这儿修建飞机场挺合适。 
        旅客们离开飞机,他们站在飞机旁不走。 
        “怎么回事?”舒克问空中小姐。 
        空中小姐跑到飞机下边,同旅客们说着什么,然后爬进机舱告诉舒克: 
        “他们说,要帮咱们修完跑道再走。” 
        舒克原来还以为旅客要同他算帐,他感动得眼眶湿了。 
        说干就干。舒克和臭球跳下飞机,拿尺子测量土地。旅客们有的拔草,有的平地。 
        经过一天的紧张施工,跑道修好了。 
        几位旅客毛遂自荐当机场的工作人员,舒克同意了。 
        这天上午,皮皮鲁号缓缓滑上了新修的跑道。转眼间,喷气客机(禁止)云端。 
        当皮皮鲁号平安降落在舒克贝塔航空公司机场时,机场的工作人员都拥到飞机旁,迎接首航归来的勇土。 
        “咱们以后可得细心点儿,”舒克对贝塔说,“哪儿有不修跑道就搞空运的呀!这在世界航空史上也算奇迹了。” 
        贝塔耸耸肩膀。 
        “快去餐厅吃饭吧。”贝塔拉着舒克来到餐厅。 
        餐厅主任给舒克和贝塔端来丰盛的午餐。舒克大口大口吃起来。 
        “咱们商量商量拍电影的事。”舒克往嘴里塞了一块肉。 
        “我看场务组里有个负责扫跑道的叫艾丽的姑娘,平时喜欢诌两句,就让她当编剧吧。”贝塔对部下的特长挺了解。 
        “行,你让她快点儿把剧本写出来。”舒克抹抹嘴。 
        饭后贝塔打电话把艾丽叫来。 
        “咱们公司准备拍电影。”贝塔对艾丽说。 
        “拍电影?”艾丽觉得新鲜。 
        “决定由你当编剧。” 
        “我?”艾丽怀疑自己的耳朵。 
        “尽快把剧本写出来。舒克已派人去弄拍电影的器材了。”贝塔说。   第36集 
        审查小组通过了剧作家艾丽的电影剧本; 
        臭球当电影演员; 
        臭球借着拍电影大吃花生米   
        一个星期后,一部为舒克贝塔航空公司歌功颂德的剧本诞生了。 
        为此,公司专门成立丁一个审查小组,研究通过该公司的第一部故事片剧本。 
        贝塔担任组长。组员有舒克、臭球、罗丘,都是见过世面的人物。 
        这天,航空公司停飞一天,专门讨论剧本。 
        艾丽捧着剧本坐在会议室里,像等待审判一样。 
        “这是咱们公司的第一部电影,当然,主要是在飞机上放映。不过,如果拍好了,说不定也能去参加奥斯卡金像奖评选。”贝塔先说几句。 
        “什么叫奥斯卡?”罗丘问。 
        “奥斯卡 ……奥斯卡就是奥斯卡,世界上最权威的电影评奖,在外国。”贝塔解释。 
        ‘咱们老鼠拍的电影也能参加评选?”臭球表示怀疑。 
        “这……我想能吧。几十年评来评去都是评人拍的电影,突然来了一部老鼠拍的,大家准感兴趣。”贝塔说。 
        “让艾丽把剧本先念一遍。”舒克提议。 
        艾丽清清嗓子,开始念剧本。 
        “剧本的名字叫《会开直升机的老鼠》,是以舒克的经历为原型写的。”艾丽说。 
        “这电影名字长了点儿。”臭球发表意见。 
        “嗯,老鼠两个字出现在片名上有点儿那个。”罗丘谈自己的看法。 
        “改成《舒克和直升机》怎么样?”贝塔提议。 
        “我反对用真名。”舒克不同意。 
        艾丽感到为难,不知如何修改片名。 
        “先念剧本吧。”贝塔说。 
        艾丽把剧本念了一遍。 
        大家七嘴八舌地提修改意见。这个说主要人物性格不突出,那个说猫的形象太高大,另一个说某一个细节有丑化老鼠之嫌…… 
        艾丽记录下来的意见比原剧本的字数多出一倍。 
        “回去好好改改,尽快写好。”贝塔吩咐。 
        “嗯。”艾丽面有难色,但还是答应了。 
        “你会改好的。”舒克像个大学教授。 
        “语言再精炼些。”臭球像大作家。 
        “增加些悬念。”罗丘不甘落后。 
        大家都得到了极大的满足,都没想到自己这辈子还能审查电影剧本。说真的审查剧本比当编剧舒服多了,神气多了。 
        数日后,电影剧本审查小组再次开会研究艾丽写的电影剧本。 
        艾丽上次开完会后,回去把大家的意见看了一遍。认为按照这些意见无法修改剧本,许多意见都是自相矛盾的。她干脆一字不改。 
        “我按照你们的意见把电影剧本重新写了一遍,现在念给你们听听。”剧作家艾丽说。 
        大家冼耳恭听。 
        “改得好。”舒克听完后点头喝彩。 
        “不错,具备了获得奥斯卡奖的条件。”贝塔点头。 
        罗丘和臭球也是赞不绝口。 
        艾丽本来悬着的一颗心放下了。 
        为了奖赏剧作家,罗丘赠给她一碗试制成功的冰淇淋。 
        电影要正式开拍了。贝塔担任导演。让舒克演舒克最合适不过,可舒克要飞行,没时间。 
        “让臭球主演吧,他会开飞机。”舒克推荐。 
        就这样,臭球当上了电影演员。 
        这天下午,摄制组正式开机。 
        第一个镜头:舒克(臭球扮演)跟着妈妈出来找吃的,碰到一盘花生米。 
        妈妈由剧作家艾丽扮演。 
        “开机!”贝塔一声令下。 
        “妈妈”带着“舒克”从家里走出来。“舒克”比“妈妈”还神气,还对这个世界不屑一顾。 
        “停!”贝塔挥手。他走到臭球面前,“你是有生以来第一次见到这个世界,要惊讶,要东张西望,别老摆出电影天皇巨星的派头。” 
        臭球撇嘴。 
        “重拍!”导演对摄影师说。 
        摄影师名叫四黑.是从导航室抽调来的。 
        “舒克”跟着“妈妈”东张西望地从家里走出来,他们j来到一张桌子上。桌上放着一盘油炸花生米。 
        臭球一见到油炸花生米就忘了一切。他扑上去大吃起来,忘了说台词。 
        “台词!台词!”贝塔急了。 
        臭球还是不顾一切地吃。 
        “停!”贝塔冲上去拉拉臭球,“你怎么搞的?” 
        “我,我忘了台词。”臭球抹抹嘴。 
        “重拍!”贝塔下令。 
        臭球眼前一亮,刚才的花生米白吃了。对,就故意出错,直到把这盘花生米全吃完为止。 
        “台词错了,重拍!”贝塔说。 
        “又错了,重拍!”贝塔说。 
        “重拍!” 
        “重拍!” 
        盘中的花生米急剧减少。 
        当剩下最后五颗花生米时,拍摄成功。 
        “拍下一个镜头。”贝塔宣布。 
        臭球的肚子胀得像皮球,躺在地上一动不动。 
        经过一个月的紧张拍摄,故事片完成了。 
        舒克提出看看样片,摄影师四黑这才发现,拍摄过程中,摄影机里始终没装胶片。   第37集 
        舒克贝塔航空公司发展壮大; 
        皮皮鲁号遭受来历不明的歼击机袭击   
        当贝塔听说摄影机里忘了装胶片后,当即昏了过去。臭球倒挺高兴,他又可以大吃一番花生米了。 
        没办法,电影只得重拍一次。这回为了保险起见,摄影师四黑往摄影机里装了两副胶片。 
        故事片终于拍成。舒克看后挺满意,但觉得送奥斯卡奖评比还差一截,因为导演时不时也出现在画面上。 
        皮皮鲁号喷气机在飞行途中能为旅客放电影了,旅客们再不会感到旅途是无聊寂寞的了。 
        经过不断的翻修和扩建,舒克贝塔航空公司的主机场越来越宏伟,现代化设施星罗棋布,皮皮鲁号喷气机和直升机都投入空运。舒克又培训出六名飞行员。 
        在一个万里无云的晴天,皮皮鲁号客机送100多只青蛙去远方。 
        飞机平稳地飞行。客舱里,旅客们在看电影。 
        舒克和副驾驶臭球聚精会神地驾驶飞机。突然,飞机前方山现了三个小黑点。 
        “臭球,你看那是什么?”舒克说。 
        “是大鸟?”臭球拿不准。 
        舒克和臭球仔细看。 
        “是三架飞机!”臭球叫起来。 
        舒克定神一看,真是三架飞机,三架同皮皮鲁号比例一样的小飞机。 
        舒克没听说过附近还有小动物开飞机的。 
        “是歼击机!”臭球报告说。 
        三架涂得花花绿绿的歼击机朝皮皮鲁号逼近,它们摆开了三角队形,冲过来。 
        “快闪开!”臭球叫起来。 
        舒克忙操纵飞机降低高度。 
        三架歼击机同时开火了。皮皮鲁号的机身急剧晃动。 
        舒克和臭球慌了,他们弄不清这三架来历不明的歼击机干吗要袭击皮皮鲁号。向客机开火,是严重违反国际航空法的行为。 
        “你去告诉旅客,系好安全带。”舒克看见那三架飞机掉头义朝皮皮鲁号飞来,他对臭球说。 
        臭球跑进客舱,把紧急情况通报给人家。 
        舒克决定返航。他拿起话筒。 
        “贝塔,贝塔,我是舒克,请回答。” 
        “我是贝塔,请讲。” 
        “有三架来历不明的歼击机袭击我们,皮皮鲁号已经受伤,我现在驾机返航,请你做好准备。” 
        “歼击机?开炮打客机?”贝塔感到问题十分严重。 
        “歼击机又来了!”舒克握驾驶杆的手心出汗了。 
        “当心!”贝塔叮嘱舒克。他恨自己帮不上舒克的忙。 
        三架歼击机从皮皮鲁号上方压下来。 
        “嘿嘿!”舒克的耳机里传出一阵冷笑。 
        舒克感到这声音很熟悉。他知道这是从歼击机上传来的声音。 
        “你是谁?”舒克问,“干吗袭击我们?” 
        “我名叫海盗,上次你把我抓去,我越狱成功了。现在我是海盗飞行大队总队长,找你算账来了!” 
        海盗洋洋得意。 
        舒克和臭球傻眼了。 
        “请你不要朝客机开火,我们可以到地面上谈判。客机上有100多名旅客。”舒克说。 
        “我管你有多少旅客!僚机注意,目标,皮皮鲁号,开火!”海盗发狠了。 
          随着猛烈的炮火,皮皮鲁号的机身开始倾斜。 
        “报告机长,机舱左侧出现了一个窟窿,有一只青蛙受伤。”空中小姐闯进驾驶室。 
        “你们给他包扎伤口!告诉旅客不要紧张。”舒克一推驾驶杆,飞机快速下降高度。 
        “咱们到树林里去飞。歼击机速度快,会撞在树上的。”舒克告诉臭球。 
        “你注意驾驶,我观察敌机情况。”臭球说。 
        皮皮鲁号一头扎进一片树林。 
        海盗很狡猾,没有跟进来。 
        舒克松了口气。 
        “你去检查一下飞机损伤情况。”舒克吩咐臭球。 
        臭球来到客舱,只见左侧有几处弹孔,其中一个弹孔就挨着发动机,好险。 
        “舒克,舒克,我是贝塔!快回答!” 
        舒克的耳机里传出贝塔的紧急呼叫。 
        “我是舒克,快讲!” 
        “那三架歼击机来袭击咱们机场,还扔了两颗炸弹!”贝塔上气不接下气。   第38集 
        皮皮鲁号超低空飞行,撞在渔网上; 
        海盗袭击机场; 
        舒克引开海盗   
        “海盗去袭击咱们机场了!”舒克告诉臭球。 
        “咱们往哪儿飞?”臭球急得直揪自己的耳朵。 
        “先在这儿盘旋,我再同贝塔联系。”舒克按下电台上的通话按钮。“贝塔,贝塔,我是舒克,快讲话!” 
        没有回答。 
        “贝塔,贝塔,我是舒克,请回答!” 
        还是没有声音。 
        准是机场遭到了严重破坏。 
        “飞回去!”舒克决定冒险。 
        大型客机作超低空飞行是非常危险的。但只有超低空飞行能躲开海盗的歼击机。 
        “你驾驶一会儿,我去同旅客们说。”舒克对臭球说完来到客舱。 
        “各位旅客,请大家不要慌。”舒克镇静地说: 
        “我们的飞机遇到了空中强盗的拦截,现在这帮强盗又去袭击我们的机场。我们现在返航。为了保证大家的安全。要超低空飞行,请大家放心。” 
        “这些空中强盗真可恶!” 
        “皮皮鲁号上要是有炮就好了!” 
        旅客们议论纷纷。 
        舒克放心了。青蛙们没有胆怯。 
        舒克回到驾驶舱,对臭球说:“你注意观察,我驾驶。” 
        皮皮鲁号喷气客机开始降低高度,擦着地面飞行。舒克眼睛都不敢眨巴,生怕撞在什么东西上。 
        “注意,前方有树!”臭球站着说。他不敢坐下观察,怕发现障碍物太晚。 
        客机绕过大树。 
        前边有一张大渔网,晾在竹竿上,臭球和舒克都没看见渔网。皮皮鲁号撞在渔网上,把鱼网拉走了。 
        渔网罩在飞机上,把飞机包住了。飞机像是被装在网兜里。但没有影响飞行。 
        舒克出了一身冷汗。 
        前方是一片高高的草丛。 
        “海盗的飞机飞回来了!”臭球发现了上空的一架歼击机。 
        舒克一推驾驶杆,飞机钻进卓丛,在草丛里飞行。渔网挂上了许多草叶,飞机穿上了伪装服。 
        此时还有两架海盗的歼击机正在轮番攻击舒克贝塔航空公司的机场。海盗派了一架飞机去找皮皮鲁号。 
        海盗坐在座舱里,他得意极了。自从上次越狱逃跑后,海盗就发誓要成立一个飞行大队,他纠集了十几只老鼠家族中的亡命徒,弄来十几架歼击机,天天训练飞行。 
        海盗两次从天上掉下来,又奇迹般地死里逃生。终于,他练出了一手过硬的飞行本领,还培训出十几名歼击机飞行员。 
        现在,海盗报仇的机会来了。他忘不了自己被舒克吊在空中的耻辱。 
        “僚机跟上!”海盗又冲舒克贝塔的机场发起了一次进攻。 
        两架歼击机呼啸着俯冲下来。 
        海盗用瞄准具的光环套住了塔台,他使劲按下驾驶杆上的射击按钮。 
        一串炮弹射出去,塔台起火了。 
        贝塔的腿被打伤了。电台也被打坏了。 
        “快,快修电台!”贝塔倒在地上喊。 
        无线电员扑上去抢修电台。 
        急救车赶来抢救贝塔。 
        海盗的飞机又俯冲下来。 
        贝塔连一点儿招架的方法也想不出来,他气得直喘粗气。 
        一串炮弹打中了候机大楼。 
        “叫旅客都躲到地下室去。”贝塔命令。 
        “电台修好了。”无线电员报告。 
        “接舒克!”贝塔戴上耳机。 
        舒克听到了贝塔的呼叫。 
        “海盗的飞机还在袭击机场,机场损失惨重,跑道被炸了两个大坑,皮皮鲁号无法着陆。”贝塔说。 
        “我已接近机场,我想法把海盗引开,保护机场。”舒克听说机场损失严重,心疼极了。他决定用皮皮鲁号引开海盗。 
        舒克一拉驾驶杆,飞机冲上天空。 
        海盗派来的歼击机发现了披着伪装网的皮皮鲁号。他立即报告了海盗。 
        “盯住它,我们马上赶到。”海盗说完带领僚机离开了机场上空。   第39集 
        舒克调虎离山; 
        海盗的飞机被鹰击中; 
        海盗再次空袭舒克贝塔航空公司机场   
        那架歼击机死死咬住皮皮鲁号不放。 
        舒克减速,它也减速。舒克加速,它也加速。 
        客舱里不少旅客呕吐了。空中小姐忙着给他们收拾清洁袋。 
        “舒克舒克,我是贝塔。海盗已离开机场,你要当心。”贝塔通知舒克。 
        “明白。请赶快抢修跑道。”舒克说。 
        “明白。”贝塔带伤指挥机场工作人员抢修跑道。许多旅客也来帮忙。 
        皮皮鲁号摆脱不了歼击机的追逐。这时,海盗带着僚机来了。 
        三架歼击机压在皮皮鲁号上方。海盗还操纵飞机来回晃机翼,故意气舒克。 
        舒克忽然想起前边那座山头上有个鹰巢,他有办法了。 
        “咱们把飞机开到鹰巢旁边,把鹰引出来。”舒克对臭球说。 
        “这办法不错,也够危险的。”臭球点点头。 
        皮皮鲁号朝鹰巢飞去。 
        海盗觉得马上就把舒克的飞机打下来太便宜他了,他想折腾舒克,拿舒克开心。 
        舒克驾驶飞机擦着鹰巢飞过去,机翼尖碰掉了鹰巢的几根树枝。 
        鹰被激怒了,它“呼”地一下飞到空中,寻找挑衅者。 
        正好三架歼击机飞过来,它们自然成为鹰的攻击目标。 
        海盗正一边吹口哨一边开飞机,忽然觉得眼前一黑,原来是鹰用翅膀拍他的飞机。海盗的飞机失去了控制,螺旋着朝地面栽下去。 
        海盗的飞行技术堪称世界一流,他在飞机撞地的一刹那问,把飞机拉起来了。 
        鹰又冲过来。 
        “返航。快返航!”海盗招呼部下。 
        三架歼击机慌忙逃窜,躲避鹰的攻击。 
        皮皮鲁号降落在弹痕累累的机场上。 
        贝塔躺在担架上来接舒克。 
        看着被破坏的机场,看着受伤的朋友,舒克咬牙切齿: 
        “我非再把这个海盗吊到天上去不可!” 
        “也不知那鹰消灭海盗没有?”臭球说。 
        空中小姐跑过来问舒克: 
        “青蛙旅客们怎么办?” 
        “安排在机场宾馆住下,明天照常飞行。”舒克说。 
        空中小姐把旅客们带到宾馆去休息。 
        天黑了。舒克和臭球吃完饭后,带着机械师们修补飞机。 
        机场工作人员修整被炮弹打坏的房屋。 
        舒克、臭球和地勤人员把挂在飞机上的渔网“伪装服”摘下来,拿胶水补机身上的弹孔。 
        空中小姐清扫客舱里的垃圾。 
        飞机修好时已是深夜了。 
        “大家回去好好休息,明天早晨起飞。”舒克吩咐道。 
        第二天早晨,皮皮鲁号准备起航。舒克检查发动机。旅客们陆续登机。 
        突然,警报响了。 
        “怎么回事?”舒克从驾驶室里探出头来。 
        “海盗的战斗机又来了!”一名地勤人员指指天上 
        舒克一看,远处空中出现了几个黑点。那黑点越来越火。不是三架,而是几十架! 
        舒克急忙发动了飞机,把皮皮鲁号滑进停机坪的草丛里。 
        “臭球,快派人把伪装网盖在飞机身上。”舒克一边说一边跑下飞机。 
        空中小姐疏散旅客。 
        地勤人员飞速把伪装网盖在飞机身上。 
        海盗的飞机飞临机场上空了,黑压压一片排着整齐的队形。 
        其中的两架歼击机俯冲下来,一阵排炮,机场上硝烟骤起。 
        又是两架歼击机俯冲下来扫射。 
        海盗的飞行大队就是这样两架一组地轮番进攻机场。舒克没有招架和还手的可能。 
        机场上的不少建筑被摧毁了,一些工作人员和旅客受伤了。 
        海盗们打够了,就驾机在机场上空作飞行表演:翻筋斗,空中开花,拉烟…… 
        舒克气得咬牙切齿。 
        “舒克,贝塔叫你去一趟。”餐厅主任罗丘爬过来对草丛中的舒克说。 
        舒克来到伤员住的地下室,贝塔在这儿躲空袭。 
        “把坦克前边垫起来,炮就能往空中打了。”贝塔说。 
        舒克眼睛一亮,这倒是个办法。 
        “臭球,跟我来!”舒克招呼。 
        “抬着我去!”贝塔叫道,“你不会开坦克。” 
        “你的腿动不了,怎么开?我早看会了。”舒克跑出去。   第40集 
        舒克驾驶坦克险些撞墙; 
        坦克变成高射炮; 
        贝塔击中海盗   
        贝塔对医生说:“快抬着我去,他开不了坦克。别以为会开飞机就什么都会开。” 
        医生说:“你这伤可开不了坦克。” 
        贝塔急了:  “我坐在他旁边指挥他,快抬我去!” 
        医生忙叫护上抬着贝塔去追舒克。 
        轰炸后,扫射又开始了。 
        舒克和臭球好不容易来到停放坦克的库房旁,臭球打开库房门。 
        坦克已蒙上了一层尘土。 
        舒克打开坦克舱盖儿,钻进去。 
        臭球站在炮塔旁边,从舱口往里看舒克操纵。 
        舒克回忆着从前看贝塔开坦克的情景,他按了一个按钮。 
        坦克“呼”地一下子冲出车库,差点儿把臭球甩下来。舒克不知道怎么停,坦克朝候机大楼撞过去。 
        “按蓝色的按钮!”贝塔躺在担架上喊。 
        臭球转告给舒克。 
        坦克在距候机大楼1厘米的地方停住了。 
        贝塔坐着担架过来了。 
        “把我抬到坦克里边去。”贝塔对hushi说。 
        臭球帮助hushi把贝塔塞进坦克,坐在舒克身边。 
        “我指挥你。”贝塔冲舒克挤挤眼睛,“飞行员不一定什么都能开。” 
        舒克耸耸肩,无话可说。 
        “臭球,快去堆一个斜土坡,角度大点儿。”贝塔抬头对趴在舱盖上的臭球说。 
        “当心空袭!”舒克提醒臭球。 
        “这个按钮是倒车,这个按钮是启动,这个按钮是加速,这个是制动……”贝塔教舒克。 
        尽管舒克觉得坦克兵教飞行员有点那个,可这是战争时期,他也顾不上面子了。 
        “你试着开开。”贝塔说。 
        舒克觉得“试”字挺刺耳,他撇撇嘴。 
        坦克在原地来了个360度旋转,再来个180度转弯,向前开去。 
        贝塔在心里不得不佩服舒克,毕竟是飞行员出身,学起来就是快。 
        海盗的歼击机继续在机场上空横行霸道,跑道上布满了弹坑。 
        舒克通过潜望镜看见臭球他们已经把土坡筑好了。 
        坦克朝土坡开去。 
        “掌握好速度!”贝塔叮嘱舒克。 
        坦克驶上了土坡。车身几乎垂直。 
        臭球指挥机场上作人员拿石头顶在坦克的尾部。 
        海盗在天上发现了坦克。 
        “攻击那辆坦克!”海盗F令。 
        歼击机轮番向坦克俯冲。炮弹打在坦克四周。 
        贝塔也用瞄准镜瞄准了正在俯冲下来的海盗的飞机。 
        双方同时按下射击按钮。 
        坦克被击中了。海盗的飞机也被击中了。 
        飞机冒着烟往下掉。海盗不敢跳伞,他知道被抓住下场好不了。他一边招呼部下返航,一边强行操纵飞机向机场以外的地域滑翔。 
        坦克也起火了,舒克忙打开舱盖儿,可坦克是垂直的,贝塔腿上有伤,出不去。 
        消防车来了,扑灭了坦克的火。 
        救护车停在坦克旁边,hushi们把贝塔从坦克里抬出来。 
        舒克从坦克里钻出来,满脸是灰。 
        空中小姐跑过来说:“旅客们问什么时候通航?” 
        “机场暂时关闭。”舒克决定。他相信,海盗明天还会来捣乱。 
        “通知全体人员,到会议室开紧急会议。”舒克告诉臭球。 
        一听说开会商量对策,工作人员们争先恐后地赶到会议室。他们都希望早点儿想出办法来治治海盗。 
        “大家出出主意。”舒克说。 
        “咱们应该有高射炮。” 
        “咱们也应该有战斗机。” 
        “咱们……” 
        大家七嘴八舌。 
        “到哪儿去弄高射炮和战斗机呢?”舒克为难地说。 
        “去找皮皮鲁。”臭球提议。 
        舒克服睛一亮,对,去找皮皮鲁!   第41集 
        舒克和臭球开直升机到皮皮鲁家; 
        皮皮鲁送给舒克吸铁石; 
        海盗拦截直升机   
        趁着天黑,舒克和臭球驾驶直升机进城去找皮皮鲁。 
        直升机降落在皮皮鲁家的阳台上。 
        “你在飞机上等着,我进去看看。”舒克对臭球说。 
        “当心点儿。”臭球看看飞机外边。 
        舒克轻轻打开飞机舱门,蹑手蹑脚地溜下飞机。 
        阳台门关着。舒克顺着墙爬上窗台,他看见屋子里皮皮鲁正在看电视。 
        舒克使劲撞窗户,皮皮鲁听见声响回过头来,他看见了窗台上的舒克。 
        皮皮鲁“腾”地从椅子上蹦起来,打开阳台门。 
        “真想你呀!那架大飞机怎么样?运了几次旅客啦?”皮皮鲁提了一连串问题。 
        舒克走进屋里,叹了口气。 
        “怎么啦?”皮皮鲁看出舒克情绪不好。 
        “海盗不知从哪儿弄来一些战斗机,拦截我们的客机,他们还空袭机场。”舒克说。 
        “就是你上次跟我说过的那个海盗?”皮皮鲁问。 
        “嗯。”舒克点点头,“帮我们想想办法吧!” 
        皮皮鲁眼睛盯着电视,在想。 
        “帮我搞几架歼击机。”舒克说。 
        “现在我这儿没有,得等明天晚上。”皮皮鲁说,“可明天早晨海盗又会去袭击你们呀!” 
        “高射炮也行。”舒克说。 
        “高射炮现在也没有。”皮皮鲁摇头。 
        舒克绝望了。 
        “有办法了。”皮皮鲁一拍腿。 
        舒克兴奋。 
        “我这儿有几块吸铁石,你把它们带回去,安放在机场上,架起来。准能把海盗的飞机吸住。”皮皮鲁说。 
        “能行吗?”舒克不大相信。 
        皮皮鲁打开抽屉,拿出几块圆形的黑磁铁,放在地上,他又从图钉盒里掏出一把图钉,朝吸铁石扔过去。 
        图钉都被磁铁石吸过去了,牢牢地依附在上边。 
        舒克乐了。 
        “你用直升机把这几块吸铁石运回去,明天我去给你搞战斗机。”皮皮鲁说。 
        “这磁铁会把直升机吸住吧?”舒克担心。 
        “每次运一块,别装在飞机里。吊着,绳子放长点儿。”皮皮鲁说。 
        “现在就运。”舒克迫不及待。 
        “歇会儿,咱们聊聊天。”皮皮鲁不想让舒克现在就走。 
        “等我打败了海盗,来陪你聊三天。”舒克说。 
        皮皮鲁无奈,他从抽屉里找出一捆塑料绳。 
        舒克和皮皮鲁来到阳台上,皮皮鲁帮助舒克捆好吸铁石。 
        “你把飞机悬停在空中,我把绳子给你系在飞机上。”皮皮鲁说。 
        舒克钻进直升机。飞机升到空中,在离皮皮鲁鼻子不远的地方悬停住。 
        皮皮鲁把绳子系在机身下边的铁环上。 
        “起飞吧!”皮皮鲁招手,“一会儿再来运!” 
        舒克冲皮皮鲁招招手。直升机离开阳台,返航了。 
        吊着这么一块大磁铁,直升机飞得很吃力。加上绳子太长,只能慢慢飞。 
        “注意观察。”舒克叮嘱臭球。 
        臭球困得都快睁不开眼睛了,他使劲儿掐自己的耳朵。 
        “贝塔,贝塔,我是舒克!听见了吗?请回答!”舒克要了解一下机场的情况。贝塔虽然受伤了,但他的病床放在塔台上。 
        “我是贝塔。请讲。” 
        “我已接近机场,可以着陆吗?” 
        “可以。”贝塔说。 
        舒克操纵飞机朝机场飞去。 
        “注意!”臭球大叫一声。 
        舒克往前一看,一群星星在他眼前飘行。 
        “是什么?”舒克把飞机悬停住。 
        臭球揉揉眼睛。 
        “飞机!”臭球脱口而出。 
        海盗的飞机!他们在这里等着舒克。原来,海盗的飞机被击中后,他凭着高超的驾驶技艺,硬是把伤机开回了他的机场。逃跑中他没忘了留下一架飞机侦察舒克的情况。当他得到情报说舒克驾驶直升机进城后,就带领自己的飞行大队埋伏蹲守在空中,等候舒克。 
        “海盗还会飞夜航!”舒克咬咬牙。 
        “咱们快把磁铁扔了吧?”臭球边说边掏出小刀,准备割绳。 
        的确,吊着磁铁太不灵活。 
        “别割!”舒克制止臭球,  “说不定能吸住敌机!” 
        “吸住了咱们也吊不动呀!”臭球说。 
        “那就往下掉,反正是先摔它。”舒克说。 
        臭球觉得有道理。但是够冒险的。   第42集 
        吸铁石“击落”两架歼击机; 
        海盗飞行大队撤退; 
        舒克驾机平安着陆   
        海盗的飞机压过来了,它们对舒克的直升机形成了一个包围圈。 
        “臭球,你到后舱去拿两个伞包,万一不行咱们就跳伞”舒克说。 
        臭球站起身,摸黑来到后舱,取出两个降落伞包,拎到驾驶舱。 
        一架歼击机冲直升机开火了,炮弹擦着直升机飞过去。好险。 
        “下边飞过来一架!”臭球把脸贴在玻璃上往下看。 
        “你指挥!”舒克授权臭球指挥他。 
        “拉杆!”臭球根据下边的敌机与吸铁石的距离指挥舒克。 
        直升机向上升去。吸铁石与敌机平行了。 
        “去看看直升机下边吊的是什么东西?”海盗命令部下。 
        那架飞机朝吸铁石靠拢。突然,它感到自己失去了控制,身不由己地朝吸铁石贴过去。 
        吸铁石把歼击机吸住了。就在同时,由于重量猛增导致直升机急剧下降。 
        臭球慌了,忙背上伞包。 
        直升机迅速朝地面坠落,只听“轰”的一声,贴在吸铁石下边的歼击机撞地爆炸了。 
        就在爆炸的同时,舒克操纵直升机拉起了高度。 
        叉一架歼击机靠过来。舒克驾驶直升机主动迎上去。 
        “降低高度!”臭球干脆把头探出舷窗。 
        第二架歼击机又被吸铁石吸住了。 
        直升机急剧下降。“轰”的一声,吊在直升机下边的歼击机爆炸了。 
        直升机再次升到空中。 
        海盗傻眼了,他断定直升机装备了新式武器。一瞬间两架飞机报销了,海盗决定返航。 
        舒克和臭球看到敌机飞走了,高兴得哼起了进行曲。 
        “舒克,舒克,我是贝塔,请回答!”耳机里传出贝塔的呼叫。 
        “我是舒克,我是舒克,请讲!”舒克答话。 
        “出了什么事啦?怎么还没到机场?”贝塔不放心了。 
        “刚才我们遇到了海盗的拦截。我们打掉了他两架飞机!”舒克报捷。 
        “真的?拿什么打的?” 
        “拿磁铁!” 
        “磁铁?” 
        “就是吸铁石!” 
        “吸铁石?”贝塔还是不明白。 
        “回去你就知道了。我们现在返航。”舒克告诉贝塔。 
        lO分钟后,舒克的直升机出现在机场上空。 
        地勤人员指示出直升机着陆的地点。 
        “贝塔,你通知地勤,把绳子剪断。我再换个地方着陆。”舒克同贝塔联系。 
        贝塔立即让hushi通知地勤人员。 
        直升机吊着吸铁石成功地着陆了。   第43集 
        吸铁石构成防空网; 
        海盗向自己的飞机开炮; 
        海盗驾机同吸铁石展开空战   
        舒克和臭球驾驶直升机吊着吸铁石成功地在机场着陆后,贝塔躺在担架上赶到直升机旁边。 
        “要吸铁石干什么?”贝塔不明白。 
        “吸海盗的飞机!刚才我们就是靠吸铁石打掉海盗两架飞机的。”舒克得意极了。 
        贝塔恍然大悟。 
        “我们还得去皮皮鲁家运吸铁石,你指挥大家修几座圆柱形的高台子,把吸铁石固定上去。”舒克说完钻回直升机。 
        臭球和舒克驾驶直升机消失在夜色中。 
        贝塔立即将工程师叫来,吩咐他马上画图纸,设计吸铁石底座。 
        图纸画好了,施工开始。 
        天亮时,机场的四个角耸立起四座细长的建筑,每座建筑的顶部都安装着吸铁石。 
        舒克和臭球在餐厅用早餐,昨天夜里他俩整整空运了一夜吸铁石。 
        餐厅主任罗丘亲自给舒克和臭球端来了丰盛的饭莱。有面包、香肠、油炸花生米、银耳汤。 
        舒克和臭球大吃特吃。 
        “我从来没吃过这么香的饭。”臭球说。 
        “这得感谢海盗。”舒克抹抹嘴。 
        话音没落,空袭警报响了。 
        舒克兴奋得蹦起来,好像空投罐头的飞机来了似的。臭球也急不可待地往外跑。 
        天空中出现了海盗的机群,它们耀武扬威地在机场上空作着各种放肆的飞行动作。 
        “妈的,你们高兴不了几分钟了!”臭球跺跺脚。 
        海盗机群拉起了高度。两架歼击机编队向机场俯冲下来。 
        精彩的场面出现了。 
        飞在前面的长机突然偏离了航线,向右边栽下去,一头粘在吸铁石上。后边的僚机刚想去给长机保驾,忽然被一股看不见的神秘力量拉向左边,死死地钉在另一块吸铁石上。 
        在空中指挥袭击机场的海盗愣了,他在飞机里拼命呼叫: 
        “05——05——快拉起来!快拉起来!!” 
        “报告头儿,拉不起来!我被吸铁石吸住了!”05号机的飞行员回答。 
        “是吸铁石!”海盗明白了。 
        “他们来抓我了,快来救我!”两架被吸住的飞机的飞行员求救。 
        海盗往下一看,几十名机场工作人员朝两架被吸住的飞机跑去。 
        “不能让他们得到飞机,快摧毁它们!”海盗发狠了,命令部下自己打自己的飞机。 
        几架歼击机开始远距离地向粘在吸铁石上的飞机开火,可惜距离太远,打不中。 
        “近点儿!”海盗命令。 
        又一架飞机被吸上了。 
        海盗急红了眼,他亲自驾驶飞机俯冲。他不能眼看着自己的三架飞机落到舒克手里。 
        舒克认出了海盗的飞机,他知道海盗有高超的驾驶技术,忙招呼大家隐蔽。 
        果然,海盗驾驶的歼击机像喝醉了酒一样摇摇晃晃地俯冲下来,他利用这种摇晃来摆脱吸铁石的磁力。 
        海盗开炮了。 
        一架粘在吸铁石上的歼击机着火了。飞机里的飞行员慌忙逃出座舱,可飞机离地面很远,他只好跳下来,摔断了腿。 
        着火的飞机爆炸了。吸铁石也被炸碎了。 
        海盗又开始向另一架粘在吸铁石上的飞机进攻。 
        舒克原想缴获海盗的几架歼击机武装自己,现在眼看着愿望就要落空,急得他直揪自己的胡子。 
        候机大楼的玻璃窗反射的阳光晃了一下舒克的眼睛。舒克笑了。 
        “快去叫大家拿镜子晃海盗的飞机!”舒克大喊。 
        大家纷纷跑回宿舍,拿来自己的镜子。 
        当海盗再次俯冲时,他只觉得地面上有几十道强光直射他的眼睛,刺得他睁不开眼睛,看不见目标。 
        海盗失去了控制,身不由己地向下边坠落。 
        “糟糕,吸铁石!”海盗一惊,他拼尽全力扳驾驶杆,飞机和吸铁石之间展开了一场力的搏斗。 
        吸铁石想把飞机吸下来,飞机想摆脱吸铁石,双方僵持住了。 
        这是空战史上的奇迹。海盗的歼击机像直升机那样停留在空中。 
        海盗毕竟是海盗,靠他强大的臂力,把飞机一点儿一点儿向空中挪。 
        精彩的场面出现了。吸铁石离开了底座,升到空中,朝海盗的飞机扑去。 
        只听“通”的一声,吸铁石吸在海盗的飞机下边。 
        海盗的飞机急剧下降,在坠地的一刹那,又吃力地昂起机头。海盗艰难地拉起高度,率领着部下,返航了。   第44集 
        直升机吊歼击机; 
        舒克驾驶歼击机试飞; 
        两名飞行员俘虏瞠目结舌   
        海盗的机群败退了,被他们遗弃的两架歼击机孤零零地挂在吸铁石上。 
        “我和他们通话。”舒克跑到塔台上,打开无线电通讯设备。 
        “我是舒克,请回答!”舒克冲两架挂在吸铁石上的飞机里的飞行员喊话。 
        不回答。 
        “你们的头儿都冲你们开炮了,你们还这么死心塌地!”舒克说。 
        “你准备怎么处置我们?”飞行员答话了。 
        “放你们走,要你们的飞机。”舒克说。 
        “……”对方显然不信。 
        “我现在就驾驶直升机把你们的飞机吊到地面上,请你们配合。”舒克说完把话筒交给贝塔,“你指挥。” 
        舒克驾驶直升机升到空中,悬停在一架歼击机上空。臭球打开舱门,扔下绳索。 
        几名机场工作人员已爬上吸铁石塔,把绳索捆在歼击机上。 
        “上升!”贝塔在塔台指挥。 
        舒克对能不能把歼击机吊到空中心里没底,他使劲儿拉操纵杆,直升机纹丝不动。 
        “我来试试。”臭球从机舱走进驾驶舱。 
        舒克离开驾驶员的座位。 
        臭球把脸都憋红了,直升机还是不动。 
        舒克从驾驶舱探出头往下看,歼击机是头朝上粘在吸铁石上的。 
        “有办法了!”舒克拿起话筒。 
        “海盗的部下,我是舒克,请回答!” 
        “我是海盗的部下,请讲!”尽管海盗的部下不喜欢这个称呼,可也顾不上更正了。 
        “打开你的发动机!”舒克下令。 
        “干什么?”海盗的部下觉得这命令莫名其妙,粘在吸铁石上的飞机启动发动机干什么? 
        “让你开你就开!”舒克没时间向他解释。 
        歼击机的发动机启动了,喷气管向地面喷射出强大的气流。臭球明白了,原来舒克是让歼击机也一起往上使劲。 
        直升机把歼击机和磁铁一起吊到空中。地勤人员挥舞着红旗指挥直升机在草坪上着陆。 
        歼击机和磁铁先着陆了。地勤人员解开绳索。歼击机座舱里的飞行员被押离飞机。 
        舒克和臭球又把另一架歼击机吊到地面上。 
        两架歼击机完好无损地并排停在停机坪上,地勤们围着飞机看。 
        舒克来到歼击机旁,他钻进座舱。 
        歼击机的座舱设备与直升机和客机大同小异。舒克知道歼击机速度快,驾驶难度高,但他知道海盗一会儿还得来袭击机场,他想驾驶战斗机同海盗打空战。 
        “去把俘虏叫来。”舒克吩咐。 
        两名飞行员俘虏来了。 
        “说说这飞机的性能。”舒克说。 
        “……”俘虏不开口。 
        “说了就放你们走。”舒克说。 
        两名俘虏对看了一眼。 
        “说话算数。”臭球在一旁帮腔。 
        一名俘虏开口了,他把飞行速度、高度、航程、转弯半径等飞行数据告诉了舒克。 
        “等我平安着陆,就放你们走。”舒克关上座舱盖,他要试飞。 
        “我是贝塔!我是贝塔!请回答。”舒克的耳机里传出贝塔的声音。 
        “我是舒克,请求试飞。” 
        “太危险,你还不了解这飞机的性能。” 
        “我问过了。” 
        “不行!”贝塔坚决不同意。 
        “一会儿海盗还会来,咱们不能老是被动挨打呀!”舒克急了。 
        “干万当心!”贝塔只好同意。 
        “放心吧,我是老飞行员了。”舒克又摆出大飞行员主义。 
        歼击机徐徐滑上刚抢修好的跑道,稳稳地停在起飞线上。 
        “舒克请求起飞。”舒克向塔台报告。 
        “同意起飞。”贝塔回答。 
        舒克给发动机增大了转速,歼击机轰鸣着向前冲去.像离弦的箭。 
        机头昂起了,机轮离地了,机身腾空了。 
        机场上一片欢呼。 
        舒克觉得歼击机太灵活,他还不大习惯,他仔细地揣摩。 
        渐渐地,舒克掌握了驾驶要领,他准备翻个筋斗。当了这么长时问飞行员,舒克还没在天上驾机翻过筋斗。这两天看着海盗在天上翻来翻去,他早眼红了。 
        “舒克请求翻筋斗!”舒克请求贝塔。 
        “翻吧!”贝塔已认定舒克是飞行天才了。 
        一个漂亮的前滚翻。 
        连两个俘虏也佩服得五体投地,要知道这个动作他们整整练了一个月! 
        “舒克请求着陆!”舒克知道歼击机着陆难度非常大。 
        “同意着陆。”贝塔回答。 
        就在这时,舒克看见前方出现了海盗的机群。 
        海盗正准备袭击舒克的机场,忽然看见了自己部下的飞机,很是高兴。他之所以这么快回来,就是怕舒克学会开歼击机。   第45集 
        舒克同海盗飞行大队展开激烈的空战: 
        舒克假装被击落; 
        臭球驾机升空参战   
        舒克见海盗误会了,决定将计就计。 
        舒克在驾驶杆上找到了射击按钮,他做好了射击准备。 
        “归队!”海盗通过电台命令舒克。 
        “明白!”舒克假装答应。 
        舒克的歼击机朝海盗靠拢,他用瞄准具的光环套住了海盗的飞机,只要把海盗打掉,海盗大队就 垮了。 
        距离海盗越来越近。舒克已经看见海盗的头了。 
        海盗已经觉察到自己的这位部下不大对头,从飞行姿态上可以看出不是他训练出来的飞行员。 
        就在这时,舒克按下了射击按钮,一串炮弹拖着蓝色的尾烟向海盗的座机射去。 
        海盗猛一拉杆,飞机来了个筋斗.炮弹擦着飞机肚皮钻过去,击中了另一架飞机。 
        那架飞机螺旋式地朝下栽去。 
        舒克不放过机会,他对准机群猛打一顿,又有几架飞机被击中。 
        “快散开!”海盗在电台里拼命喊。 
        机群迅速散开了,它们对舒克形成了一个包围圈。 
        舒克现在是无路可逃。如果着陆,肯定被击落。舒克心里清楚,他打掉了海盗四架飞机,海盗不会饶了他的。 
        果然,海盗组织进攻了。 
        两架歼击机从左边飞过来.两架歼击机从右边飞来。舒克发现下方也有。 
        “舒克注意!后边也有!”贝塔在塔台里提醒舒克。 
        机场上的工作人员都为舒克捏了一把汗,这是一场力量悬殊的空战,况且舒克是头一次驾驶歼击机。 
        舒克背好伞包,作好了随时跳伞的准备。他严密注视着四面八方的敌机动态。 
        右边的那两架敌机进入了最佳射击距离和角度,舒克断定它们就要开火了。 
        舒克猛一推驾驶杆,飞机朝地面俯冲下去。在这同时,那两架敌机开炮了。 
        海盗和部下们还以为舒克被击中了,他们高兴得直抖机翼。 
        只见舒克的飞机在坠地之前拉了起来,机身几乎是垂直向上升去。 
        地面上的尘土被吹起老高。 
        舒克发现头顶上正有一架敌机,他果断地按下射击按钮。 
        敌机被击中了,冒着黑烟栽下去。 
        海盗勃然大怒。又上了舒克的当!他恨不得把舒克吃了,他要亲手打掉舒克。 
        海盗的飞机朝舒克逼过来。 
        舒克现在已经完全掌握了歼击机的驾驶要领,他自如地操纵着飞机,躲避着敌机的攻击。 
        地面上的臭球见舒克孤军奋战,决定助他一臂之力。 
        臭球跨进停机坪上另一架缴获的歼击机的座舱。 
        “贝塔,贝塔,臭球请求起飞!”臭球向塔台报告。 
        “你要干什么?”贝塔问。 
        “我去支援舒克。” 
        “可你不会开歼击机呀!” 
        “舒克也不会!” 
        “他比你……”,贝塔想说“他比你聪明”,后半截话咽回肚里了,太伤人。 
        臭球不管那么多了,他启动了发动机。 
        歼击机滑向跑道。 
        “背好伞包。”贝塔提醒臭球。 
        “谢谢。”臭球把屁股下边的伞包的带子系在肩上。 
        “臭球请求起飞。”臭球请示塔台。 
        “同意起飞,注意安全。”贝塔咬咬牙。 
        舒克在空中看见了臭球。 
        “握紧驾驶杆,慢慢往后拉!”舒克提示臭球。 
        海盗听见了舒克的话,他往下一看,糟糕,又有一架歼击机要来参战。 
        “打掉它!不让它升空!”海盗命令部下。 
        两架敌机呼啸着朝正在跑道上滑行的臭球扑去。 
        “快加速!”舒克冲着话筒喊。 
        臭球把发动机的转速增到最大,飞机像脱缰的野马一样在跑道上疾驰。天空中,两架敌机从上而下地压在臭球的飞机上空,阻止他起飞。 
        臭球的飞机的机轮离地了。 
        “打掉它!”海盗急了。 
        舒克趁海盗分心的时机,摆脱了他的追击,一推驾驶杆去增援臭球。臭球头顶上的两架敌机死死压住臭球。臭球急了,猛一拉杆,飞机突然上升,险些撞上那两架敌机。吓得两架敌机忙闪开。 
        “笨蛋!”海盗骂人了。 
        “好样的!”舒克表扬臭球勇敢。 
        臭球得意了,他想翻个筋斗,就像刚才舒克翻的那样。 
        海盗清点部下,还有14架飞机。 
        十几只老鼠,驾驶着歼击机在天上进行一场史无前例的空战。   第46集 
        臭球在空中现眼; 
        舒克保护臭球跳伞; 
        舒克的座舱盖被海盗击落   
        臭球想向大家证明一个道理:他的智商不比舒克低。他决定驾驶歼击机也在空中翻一个筋斗,给贝塔他们看看,也让海盗他们知道知道。 
        臭球一拉杆,飞机先是往上升,可没升多久,就往下掉,接着是头朝下进入螺旋状态。 
        “失速!”舒克大叫一声。飞行员都知道“失速”是飞行中最可怕的事情。 
        臭球的飞机急速下降。 
        “臭球,快跳伞!”舒克通过电台大喊。 
        臭球只好放弃了飞机,弹射跳伞。 
        歼击机撞地后发出巨响。 
        臭球的降落伞在空中张开了。 
        “打他!”海盗下令打悬在降落伞下的臭球。 
        几架飞机朝臭球逼过去。 
        舒克火了。海盗这足公然违反国际上有关空战的规定:跳伞后的飞行员失去了武装,不准向他射击。 
        舒克从左下方向袭击臭球的敌机开火了。一架飞机起了火,飞行员也跳伞了。 
        舒克喊话: 
        “海盗!如果你打我们的人,我就打你们的人!” 
        舒克威胁海盗。现在天上飘着两顶降落伞,一边一个。 
        “你打吧!”海盗恶狠狠地说,“反正我要打你的。” 
        舒克这回彻底知道了海盗的心有多狠,他决定正面同海盗展开空战,拼了。舒克在歼击机的飞行技术上一直怵海盗,现在他顾不了那么多了。 
        海盗见舒克的乜机向他扑来,高兴了。他就想让舒克跟他拼,他认定自己的飞行技术是世界一流水平。 
        “你们去打降落伞!”海盗命令部下。 
        奇怪的是海盗的部下都不去打臭球。他们知道自己也有跳伞的那天,都觉得头儿做得太绝情。 
        臭球平安着陆了。救护车开到他身边。 
        臭球满脸通红,羞愧万分。 
        天上,舒克正同海盗展开一场惊心动魄的空战。 
        海盗拿出江湖大侠的派头,命令部下在一边巡航,不得参战。他要同舒克单练。 
        舒克握驾驶杆的手心出汗了,他明白这是生死之战。即便是他被击中跳伞,海盗也会在空中击毙他。 
        “我让你先打我。”海盗嘲笑舒克。他的飞机故意飞在舒克的正前方。 
        舒克用瞄准具的光环套住海盗的飞机,他按下了射击按钮。 
        海盗轻轻一抬机翼,炮弹从机身下边飞过去。 
        “现在该我打你了!”海盗掉转机头。 
        舒克不得不飞到海盗前边让他打。 
        舒克认定海盗或向他的机身下边开炮或向他的机身上边开炮,往上躲还是往下躲呢?舒克犹豫着,这简直像足球比赛发点球时的守门员心理。 
        “往下。”舒克碰运气。 
        海盗开炮了。 
        与此同时,舒克操纵飞机下降高度。 
        一发炮弹掀掉了舒克的座舱盖,强大的气流吹得舒克睁不开眼睛。 
        海盗来劲儿了,他驾机继续向舒克进攻。 
        舒克一边把飞行帽上的风镜拉下来,一边躲避着海盗的攻击. 
        地面上的小老鼠们都急了。贝塔恨自己不能上天帮助舒克。餐厅主任罗丘挥舞着勺子冲着天上的海盗大骂。臭球在一旁揪自己的耳朵生自己的气。 
        舒克想,如果自己在机场上空被击落,海盗准还得袭击机场,不如干脆把海盗引开。 
        舒克驾驶飞机朝远处飞去,海盗紧追不舍。 
        “你快投降吧,舒克!”海盗向舒克喊话。 
        “有本事你打掉我!”舒克说。 
        海盗射出一串炮弹。 
        舒克回头看看,十几架敌机都跟在后边。 
        舒克的头被风吹得发晕,他只顾加速往前开,也不知开到哪儿了。 
        “你到城里去找死呀!”海盗骂舒克。 
        舒克低头一看,下边是城市,皮皮鲁居住的城市。 
        海盗怕城市。他曾经差点儿在城里送命。海盗决定在城外于掉舒克。 
        舒克的飞机由于失去了座舱盖,速度受到影响。海盗们追上来。 
        “所有飞机都瞄准敌机。”海盗下令。 
        十几架歼击机同时瞄准了舒克的飞机。 
        舒克左右摇晃飞机,给敌机的瞄准制造困难。 
        到城市上空了。 
        “开炮!”海盗不敢飞临城市上空,他下令。舒克的飞机被击中了好几处,冒烟了。舒克发现飞机操纵失灵,他跳伞了。 
        海盗断定舒克降落后一定会被人抓住处死或被猫吃掉。他招呼部下返航。 
        舒克操纵降落伞寻找合适的着陆点。   第47集 
        舒克降落在湖中; 
        舒克乘坐摩托车; 
        舒克当了空军司令   
        舒克看见下边是一个公园,公园里有一个湖。舒克觉得落在水里比落在地上安全。 
        他操纵降落伞朝湖中心降落。 
        在落水之前,舒克给救生艇充了气。 
        舒克的身体落在湖水里,他从降落伞下边钻出来,爬进救生艇。 
        四周静悄悄的。舒克趴在救生艇里向四面张望,没发现不安全因素。 
        救生艇靠岸了,舒克看看岸上没人,迅速爬上岸。 
        舒克爬上一棵大树,判断着皮皮鲁家的方位。过去他都是开飞机来,没从地面走过。 
        舒克看见了装有大钟的楼,他曾经和贝塔拨过那座钟的表针。舒克知道皮皮鲁家的方向了。 
        白天行走是危险的,但舒克一想到海盗还会去袭击机场,就豁出去了。他溜到树下,朝皮皮鲁家的方向跑去。 
        出了公园,来到大街上。舒克藏在一根水泥管子后边。街上人来人往。舒克寻找着机会。 
        一辆摩托车停在水泥管子旁边,骑车的小伙子下车买东西。 
        舒克无法断定这辆摩托车往哪边开,他只有碰运气了。舒克钻进摩托车侧盖里边,在机油箱上坐好。 
        摩托车启动了。舒克抓紧一根电线。 
        也不知开了多长时间,车停了。舒克听见两个人在谈话。 
        “修车吗?” 
        “嗯。” 
        “什么毛病?” 
        “烧机油。” 
        “把侧盖打开。” 
        有人用螺丝刀拧侧盖的螺钉。 
        舒克慌了,他无路可逃。眼看侧盖就要被打开了,舒克只得使劲儿钻进机油箱后边。这里的间隙很小,舒克的身体被压成了馅饼状。 
        侧盖打开了。 
        “这机油什么时候加的?” 
        “前天。” 
        “是够费的。放在这儿修吧。” 
        “什么时候取车?” 
        “后天。” 
        原来是修车店。 
        小伙子走了。舒克悄悄探出头,看见修理工去修另一辆摩托车。 
        舒克蹑手蹑脚地溜下摩托车,一步三回头地朝门口溜去。他光顾得回头看那修理工,不留意碰翻了一个玻璃瓶。 
        修理工一回头,看见了舒克。他大喝一声。 
        舒克“噌”的一下钻出门外,没命地跑。修理工根本就没追出来。 
        一场虚惊,舒克站住定定神。 
        当他看清自己周围的建筑物时,笑了。这是皮皮鲁家。 
        舒克藏进皮皮鲁家所在的单元门口的花坛里,等着皮皮鲁放学归来。 
        这两天真累。舒克睡着了。 
        舒克做了一个梦。他梦见自己当了空军司令,统率着几千架战斗机。当海盗的机群来犯时,他一声令下,几千架战斗机腾空而起,遮天蔽日。 
        止当他在空中指挥战斗时,耳机里传出了皮皮鲁的声音。 
        舒克睁开眼睛,透过草丛,他看见皮皮鲁正和同学们一边打闹一边朝单元门口走去。 
        舒克知道错过这个机会就完了,他冲出花坛,站在皮皮鲁面前。 
        “老鼠!”一位同学惊叫起来。 
        皮皮鲁定睛一看,是舒克!忙蹲下把舒克捡起来。 
        “你……你抓老鼠!”那位同学直皱眉头。 
        皮皮鲁冲他一笑,拿着舒克朝楼上跑去。 
        到了屋里,皮皮鲁把舒克放在桌子上。 
        “快说,吸铁石管用吗?”皮皮鲁一直挂念着舒克的机场。 
        “管用!吸住了海盗的好几架飞机。”舒克接着把今天的激烈空战讲给皮皮鲁听。 
        皮皮鲁听呆了。 
        “明天你就不用怕海盗了。”皮皮鲁说。 
        “为什么?”舒克预感到有好事。 
        “你看!”皮皮鲁打开他的抽屉。 
        抽屉里全是战斗机,足足有20架。 
        “这是歼击机。”皮皮鲁拿出几架放在桌上。 
        “这是强击机。” 
        “这是轰炸机。” 
        “这是侦察机。” 
        现在舒克毫不怀疑自己真要当空军司令了。 
        “这么多飞机,我怎么开回去?”舒克为难了。 
        “别急,先吃饭。一会儿我骑自行车送你回去,把飞机都运去。”皮皮鲁说。 
        舒克在皮皮鲁家美美地吃了一顿饭。 
        舒克贝塔航空公司和海盗飞行大队之问将展开一场真正的国际水平的空中大战。   第48集 
        皮皮鲁训练战斗机飞行大队; 
        舒克兼任战斗机飞行大队长; 
        臭球干着急   
        舒克在皮皮鲁家吃完饭后,皮皮鲁把书包里的课本倒出来,把战斗机都装进书包。 
        “委屈你在书包里呆会儿,我骑自行车把你和飞机送到你的机场去。”皮皮鲁对舒克说。 
        舒克告诉皮皮鲁机场的方位。 
        皮皮鲁趁爸爸妈妈还没下班,骑自行车直奔郊外。 
        舒克不时从书包里探出头给皮皮鲁指路。 
        在一座小山坡后边,皮皮鲁找到了舒克的机场。 
        “真壮观呀!”皮皮鲁看着脚下这座微型飞机场,直吐舌头。 
        “你看,那么多建筑都被海盗的飞机炸坏了。”舒克从书包里探出头来。 
        “没关系,你马上就能击败他!”皮皮鲁信心十足。 
        皮皮鲁打开书包.先把舒克放在地上,然后将战斗机一架一架放在机场的停机坪上。 
        舒克朝塔台跑去。 
        贝塔已经在塔台里看见了皮皮鲁,他的伤好些了,已经能下地行走。 
        “快扶我去见皮皮鲁。”贝塔对舒克说。 
        舒克扶着贝塔来到皮皮鲁身边。 
        “光荣负伤啊!”皮皮鲁蹲下对贝塔说。 
        贝塔耸肩。 
        “我来帮你们训练战斗机。”皮皮鲁说。 
        “太好了,我去挑选飞行员。”舒克跑回机场。 
        机场全体工作人员集合。 
        舒克讲话。 
        “现在,咱们要成立战斗机飞行大队,为运输机护航。想当战斗机飞行员的举手!” 
        “呼啦”,都举手了。 
        舒克挑选了十几名身强力壮的。 
        “怎么没我?”臭球对自己落选感到意外。 
        “你负责开运输机。”舒克拍拍臭球的肩膀。 
        “人选的飞行员留下,其他人各就各位,准备飞行的地面工作。”舒克一挥手,大家散开了。 
        场务组去扫跑道,油车、电瓶车穿梭不息,机务人员检查飞机,气象人员报告风向、风力,餐厅的大师傅们准备可口的饭菜……机场一派繁忙景象。 
        舒克给新飞行员简要地讲述了飞行原理和操作要领,然后分配飞机。 
        舒克兼任战斗机飞行大队队长,下设歼击机飞行中队,轰炸机飞行中队和强击机飞行中队。 
        训练开始了,几十架飞机整齐地排列在起飞线上。皮皮鲁站在机场旁边像一座巨塔,指挥飞行。 
        舒克钻进一架最新式的喷气式超音速可变翼歼击机里,他关上了座舱盖儿。 
        “舒克请求起飞!”舒克请示。 
        皮皮鲁对着话筒说:  “同意。” 
        歼击机滑上了跑道。舒克感到这飞机很好操纵。转眼间,歼击机插上蓝天。 
        “俯冲!” 
        “翻滚!” 
        “拉杆!” 
        皮皮鲁不时发出指令。舒克的歼击机作着各种高难度的飞行动作。 
        “这飞行技术,没治了!” 
        皮皮鲁连连点头。 
        “01号要求返航。”舒克请示。 
        “同意!” 
        舒克驾驶歼击机绕机场一周,通过四转弯降低高度,对准了跑道。 
        一个漂亮的着陆动作,机轮挨地时连烟都没冒。 
        第二架歼击机准备起飞了,这是新飞行员。飞机摇摆着冲上了跑道,像喝醉了酒似的在跑道上蹦跳着向前冲去。 
        飞机离开了地面,飞行员拼命往后拉驾驶杆,想让飞机快些升高,没想到拉杆过猛,造成了飞机失速。 
        飞机急剧向地面驶去。眼看就要机毁鼠亡。在这紧急关头,皮皮鲁伸手接住了飞机。 
        一场空难避免了。 
        舒克突然意识到了有皮皮鲁在的优点:皮皮鲁可以甩手扶着飞机飞行,不断纠正新飞行员的动作,还可以保证不摔飞机。 
        飞机训练开始了。飞机一架接一架地起飞,皮皮鲁一会儿纠正这架飞机错误的飞行动作,一会儿抢救那架出故障的飞机,忙得不可开交。 
        经过两个小时的训练,新飞行员们已经熟练地掌握了战斗机的驾驶技术。 
        臭球在地面心里痒痒得直搓手心。 
        紧接着又进行了编队、射击、轰炸投弹等科目训练。 
        天快黑时,皮皮鲁对舒克说: 
        “行啦,你的战斗机飞行大队可以参加空战了我该回家了,祝你们胜利!” 
        皮皮鲁骑车回城丁。   第49集 
        歼击机为客机护航: 
        海盗飞行大队与舒克战斗机大队展开空战; 
        舒克摆脱导弹   
        晚上,舒克召集全体机场工作人员开会,他宣布从明天起,舒克贝塔航空公司正常空运。 
        “由臭球负责空运,我负责护航和保卫机场,贝塔负责地面指挥。”舒克像个军事家。 
        “除了值班的,大家都早点儿休息。”贝塔关照大家。这几天,谁也没休息好。 
        第二天.当太阳刚露出笑脸,机场上就忙开了。 
        大客机停在候机大楼旁边,闻讯而来的旅客挤满了候机大厅,他们坐过飞机后,再坐什么都嫌慢。这几天关闭机场,把他们急坏了。 
        旅客们依次登上客机,空中小姐站在机舱门口为旅客提供服务。 
        战斗机也作好了起飞的准备。 
        贝塔在塔台上注视着机场和空中的情况。 
        臭球启动皮皮鲁号的发动机。 
        “皮皮鲁号请求起飞。”臭球请示塔台。 
        “请战斗机为皮皮鲁号护航。”贝塔说。 
        “战斗机明白!”舒克坐在歼击机座舱里说。 
        “同意起飞!”贝塔发令, 
        皮皮鲁号滑上了跑道。六架歼击机护卫在皮皮鲁号的两侧。 
        七架飞机同时起飞,从形紧密,声如雷鸣,壮观无比。 
        机群进人了航线,臭球高枕无忧地打开自动驾驶仪,听筒里传出舒克的声音: 
        “臭球注意!臭球注意!发现敌机!” 
        臭球往前一看,几架海盗的歼击机正扑过来。 
        “05、06号掩护皮皮鲁号,其他飞机跟我出击!”舒克一声令下。 
        四架歼击机迎着海盗的飞机飞过去。 
        海盗没见过这种飞机,有点儿纳闷:这是谁的飞机?正当他犹豫的时候,舒克开火了。一架敌机被击中。 
        海盗这才清醒过来,忙招呼部下反击。 
        一场真正的空战开始了。一群老鼠驾驶着世界上最先进的飞机在空中搏斗。 
        海盗一共是10架飞机,舒克除去为客机护航的两架,还有四架,敌我力量悬殊。 
        “贝塔,贝塔,我们在6号空域与海盗相遇,请求增援!”舒克打开电台。 
        “马上增援。”贝塔立即命令待命的歼击机统统起飞。 
        海盗没想到舒克降落在城里没有死,而且居然在一夜之间组建了战斗机群,他心里暗暗打了个哆嗦。必须置舒克于死地,趁着力量悬殊。海盗发令: 
        “冲散他们的队形,单个打!” 
        海盗的部下操纵飞机冲进舒克的机群,舒克的飞行员驾驶技术还不熟练,一下就乱了阵脚。舒克回头一看,他的僚机已经不见了。 
        “02,02,你在哪里?”舒克呼叫。 
        “我在你的上方。” 
        舒克抬头一看,僚机跑到了长机的上边。 
        “跟着我!”舒克加速朝前飞,僚机降低高度跟上了。 
        舒克寻找海盗的飞机。 
        “长机注意!右前方有敌机!”僚机报告。 
        舒克一看,一架敌机鬼头鬼脑地正在瞄准他的飞机。 
        舒克一推驾驶杆,飞机俯冲下去。接着一拉杆,飞机又向上冲去。现在,那架敌机的肚皮正处在舒克的射击圈内。 
        舒克按下了射击按钮,三门大炮同时开火,敌机在空中开了花。 
        就在同时,舒克的僚机被海盗击中了。原来这是海盗使用的调虎离山计。 
        “快跳伞!”舒克命令僚机。 
        僚机飞行员弹射跳伞了。 
        舒克寻找着自己的另外两架飞机。他看见它们被敌机包围了。舒克驾机冲过去给他们解围。 
        海盗感到胜利在握了,他要亲手打掉舒克。 
        “舒克,你敢跟我决斗吗?”海盗通过无线电台问舒克。 
        “来吧!”舒克不含糊。 
        “咱们头对头飞,看谁能把谁打下来!”海盗恶狠狠地说。 
        “来!”舒克驾机转了一个弯。 
        海盗的飞机也转了过来。现在他们的飞机头对头,处在一个高度上。 
        两架飞机对着头互相飞过去。这样射击想命中难度很高。舒克用光环套住了海盗的飞机。海盗也用光环套住了舒克的飞机,两人都开炮了,谁也没打中对方。 
        眼看两架飞机要撞上了,舒克一拉杆,海盗一推杆,错开了。 
        “再来!”海盗说。 
        舒克二活没说,转弯,掉头。 
        两架飞机又开始了头对头的飞行。 
        这回海盗想发坏了,他的机翼下挂着一对空对空导弹,他一直舍不得使用它们,要知道,用导弹打目标几乎是不用瞄准的,导弹能自己跟踪目标。海盗决定用导弹击落舒克。 
        舒克正用瞄准具的光环套海盗,忽然看见从海盗的机翼下钻出两条火龙。导弹!舒克大惊失色。说时迟,那时快,舒克一拉杆,飞机成90度角向上升去。导弹也向上升去,紧紧咬住舒克的飞机。 
        舒克急了,他打开加力系统,飞机进入了超音速。导弹在后边跟着,只差五米! 
        飞机甩不掉导弹,导弹也追不上飞机。舒克不停地翻滚,俯冲,上升,仍然甩不掉。 
        海盗兴奋得直吹口哨。 
        舒克忽然想起这种导弹是采用红外线制导的,专咬发动机排气管的火焰,如果关闭发动机,导弹就会迷航。但空中停车是十分危险的,可只有此路一条了。 
        舒克在一个俯冲过后,关闭了发动机。 
        导弹失去了目标,它们经过短暂的迷失后,重新咬住了海盗的两位部下。 
        几秒钟后,海盗的两架飞机被自己的导弹击碎了。 
        海盗气得把驾驶杆拧成了拐棍。 
        舒克成功地在空中重新启动了发动机。 
        这时,增援的飞机赶来了。双方的力量对比发生了有利于舒克的变化。 
        海盗的飞机一架接一架地冒烟。   第50集 
        海盗逃跑; 
        庆功宴会; 
        舒克轰炸海盗机场   
        舒克率领部下把海盗的飞机打了个落花流水。当空中还剩下一架敌机时,舒克才发现海盗已经逃跑了。 
        舒克感到惋惜,他把那架敌机打掉后,打开电台同臭球联系。 
        “臭球,臭球!”舒克呼叫。 
        “我是臭球,请讲。” 
        “你在哪里?” 
        “我已平安到达目的地,现正准备起飞返回机场。” 
        “我们去为你护航。” 
        当皮皮鲁号客机在十几架歼击机的护卫下出现在机场上空时,机场上一片欢腾。大家都知道空战打赢了。 
        餐厅主任罗丘早已准备好了庆功宴会,贝塔吩咐准备节目,召开联欢会。 
        空中的飞机一架接一架地降落在跑道上,机务人员迎上前去检查飞机。飞行员一跨出座舱就接到了姑娘们献上来的鲜花。 
        “海盗打掉了吗?”贝塔一见舒克就问。 
        “让他溜了!”舒克叹口气。 
        贝塔清楚只要海盗活着,他和舒克就不会安宁。 
        舒克一眼看见了停机坪上的轰炸机群。 
        “派一架侦察机去侦察海盗的机场位置,咱们去把他的机场炸了。”舒克说。 
        “好主意!”贝塔乐了。 
        侦察机起飞了。 
        舒克和飞行员们来到餐厅出席庆功宴会,宴会很丰盛。望着桌上的油炸花生米,舒克想起了第一次同妈妈离开洞出去寻找食物的情景。现在,舒克靠自己的劳动生活了,他为老鼠赢得了好名声。舒克想妈妈了,他决定过几天去看妈妈。 
        宴会开得十分热闹,大家要求舒克讲讲空战的经过。舒克来劲儿了,他绘声绘色地讲起了空战,一会儿拿杯子当飞机比划,一会儿拿嘴当机关炮开火,大家都听呆了。 
        “可惜是老鼠跟老鼠打,要是同别人打,多长威风!”不知谁说。 
        舒克的眼睛无光了,他耸耸肩膀。 
        “还记得白路吗?”舒克问贝塔。 
        贝塔点点头,说:“也不知他在发电厂生活得怎么样了?” 
        贝塔喝了口酒。 
        他想起了克里斯王国,想起了咪丽。 
        “我有个想法。”舒克小声对贝塔说。 
        贝塔把耳朵凑过去。 
        “等把海盗收拾了,把航空公司交给臭球经营,咱们再去开始新的生活。”舒克说。 
        “对,重新开始,从零开始,如果咱们再干成一件大事业,不就等于活了两辈子吗?”贝塔同意,而且有高见。 
        “就是,一般人干成了一件事,就拿它当自己的终身事业,挺傻。其实把旧事业扔了,再重新弄另一个新事业,才够味儿。”舒克说。 
        机场上传来轰鸣声,舒克朝窗外一看,侦察机着陆了。 
        舒克和贝塔来到停机坪,飞行员正从侦察机上下来。 
        “找到了,方位是……”侦察机飞行员报告说。 
        “去叫轰炸机中队准备起飞,多装炸弹!”舒克叫喊道。 
        机场上又忙碌起来。军械员们推着炸弹车往轰炸机的肚子里塞炸弹。 
        “你去吗?”贝塔问舒克。 
        “当然。我得出出气。”舒克对海盗极为不满。 
        “我也去。”贝塔的伤已经好了。 
        “行!”舒克答应了。反正轰炸机能坐好几个人。 
        舒克委派臭球担任地面指挥。 
        轰炸机中队整装待发。 
        舒克和贝塔钻进一架轰炸机。 
        “我当驾驶员,你当领航员。”舒克对贝塔说。 
        “行,反正好久没坐飞机了。”贝塔早已对机场有规律的生活厌倦了。他向往睡在坦克车里的生活。 
        “侦察机带路,准备起飞。”臭球在塔台发令。 
        侦察机起飞了。舒克的轰炸机紧跟着也起飞了。所有轰炸机都起飞了。 
        这是舒克贝塔航空公司头一次主动去揍海盗,大家都感到激动。 
        轰炸机群浩浩荡荡向海盗的机场飞去。 
        “等会儿把海盗收拾了,咱们就告别机场吧!”舒克边开飞机边对贝塔说。 
        “你可真是急性子,还吊着我的坦克去吗?”贝塔冲舒克笑笑。 
        “当然,如果你不反对的话。”舒克修正着航向。 
        贝塔回头看看,见轰炸机群的队形编得很整齐。 
        “请注意,目标已出现!”侦察机通报。 
        舒克往下一看,果然有一座机场,停机坪上停放着几架飞机。没错,正是海盗的飞机。 
        “准备轮番轰炸,跟着我来!”舒克说完驾驶轰炸机朝海盗的机场俯冲下去,他看准了目标,按下投弹按钮。 
        轰炸机肚皮下的弹舱盖打开了,炸弹成串地向地面坠落,转眼间机场变成了火海。 
        轰炸机一架接一架地投弹,海盗的机场被掀了个底朝天。有一架歼击机想强行起飞,在跑道中间被炸翻了。 
        海盗正在喝闷酒,这回他知道舒克的厉害了,他的机场被炸平了,飞机都炸坏了。海盗躲在防空工事里,他发誓要报仇。 
        “返航!”舒克一声令下,轰炸机群凯旋。   第51集 
        舒克和贝塔任命臭球为航空公司经理; 
        舒克和贝塔告别机场; 
        直升机又吊着坦克飞到空中   
        当天晚上,舒克贝塔航空公司就像过狂欢节一样热闹。 
        舒克和贝塔悄悄来到直升机和坦克旁边。 
        舒克拉开直升机的舱门,钻进去。机舱里弥漫着一股舒克熟悉的机器味儿。他把飞机检查了一遍。 
        贝塔擦掉坦克身上的灰尘,然后钻进坦克里。他看见了自己的软床,看见了石子炮弹。 
        “怎么样,正常吗?”舒克趴在舱口问。 
        “我试试。”贝塔发动了坦克,“完全正常。” 
        “咱们去同臭球和罗丘谈谈。”舒克说。 
        他们来到办公室,打电话把臭球和罗丘叫来。 
        “我们准备走了。”舒克开门见山。 
        “走?去哪儿?”臭球一愣。 
        罗丘也摸不着头脑。 
        “回去。”贝塔说。 
        “回哪儿?”臭球说。 
        “从哪儿来,回哪儿去。”舒克说。 
        “回地洞里去?!”罗丘不信。 
        “我们不愿意过这种躺在成功事业上的生活。我们喜欢冒险,喜欢干新的事业。”贝塔说。 
        “干什么?”臭球好奇地问。 
        “还不知道。”舒克一摊手。 
        “公司怎么办?”罗丘问。 
        “交给你们俩。”舒克说。 
        臭球和罗丘大眼瞪小眼。 
        “臭球当经理,罗丘当副经理。把空运搞好,大家需要飞机。你们把轰炸机改装成客机,歼击机留着,以防万一。”贝塔吩咐。 
        “这……”臭球一时还接受不了这个现实。 
        “别紧张,大胆干。”舒克给臭球打气。 
        “你们什么时候走?”罗丘问。 
        “马上就走。”贝塔说。 
        “这么急?” 
        “我先去看妈妈。”舒克说。 
        “以后有事还可以用无线电联系嘛!”贝塔提醒臭球和罗丘。 
        “我去给你妈妈准备些点心。”罗丘说。 
        听说舒克和贝塔要离开机场了,机场的全体空地勤人员都来到直升机和坦克旁边。 
        舒克戴上飞行帽,钻进直升机驾驶舱。 
        贝塔跨人坦克。 
        “再见了,朋友们,祝你们好运气!”舒克朝朋友们挥手。 
        “再见!”贝塔的眼睛湿了。 
        人群中哭泣声越来越大。 
        舒克赶紧起飞。直升机升到空中后,用钩子吊起了坦克。 
        直升机吊着坦克飞走了。舒克和贝塔不喜欢过有条不紊的生活,他们喜欢冒险,喜欢过不知道第二天会发生什么事的生活。   第52集 
        舒克见到妈妈; 
        贝塔见到咪丽; 
        舒克和贝塔去教堂见鼠王   
        直升机趁着夜色朝眯丽家飞去。 
        “喂,舒克!”贝塔在坦克里通过电台同舒克聊天。 
        “我又想起了同咪丽周旋的日子。”贝塔深有感触地说。 
        “吃不饱吧!”舒克笑了。 
        “我还拿个口袋贮存香味儿呢,饿了就打开闻闻。”贝塔鼻子开始发酸。 
        “别忆苦思甜了。”舒克注意观察地面。 
        地面上是个小操场,许多人坐在罩边,像是在开会。 
        “你看下边干什么呢?咱们看看吧?”舒克跟贝塔商量。 
        贝塔从炮塔里搽出头来。 
        “下去看看。”贝塔同意。 
        直升机悄悄地在操场旁的房顶上着陆了。 
        一个老太婆在讲话。 
        “这次灭鼠运动是全市统一行动,家家都要投放鼠药。”老太婆说。 
        “没有老鼠的家也放鼠药吗?”有人问。 
        “当然要放!”老人婆说。 
        “那毒谁呀?”又问。 
        “这是规定!”老太婆不再理他,继续讲话。 
        “听见没有?贝塔,要灭鼠呢!”舒克身上直出冷汗。 
        “你听你听,药里还掺和糖,引诱老鼠吃!”贝塔起了一身(又鸟)皮疙瘩。 
        老鼠再坏,也不该下毒药害死人家呀!舒克和贝塔不满了。再说你们是那么巨大的动物,跟小小的老鼠叫真,也太丢份子了。 
        “明天下午发鼠药,当心别让孩子吃了。”老太婆不知为什么掏出个红袖章戴上。 
        “怎么办?”贝塔问舒克。 
        “赶紧通知同胞们,叫他们别吃鼠药。”舒克想了个主意。 
        “太棒了!”贝塔一拍大腿。 
        “先去告诉妈妈。”舒克操纵飞机吊着坦克起飞了。 
        直升机降落在贝塔原先的主人家门口。舒克走下飞机,钻进坦克。 
        贝塔开着坦克驶进屋里。屋里的人都在看电视。贝塔驾驶坦克顺着墙根溜进床下。 
        咪丽正同舒克的妈妈在一个碗里吃饭。咪丽认出了坦克,她兴奋得扑了过去。 
        贝塔打开舱盖儿,钻出来。 
        “咪丽,真想你。”贝塔真诚地说。 
        “我也是。”咪丽也不假。 
        “妈妈——”舒克从坦克里钻出来,奔到妈妈身边。 
        见到了出名的儿子,妈妈流下了眼泪。 
        “谢谢你,咪丽,把我妈妈照看得这么好。”舒克觉得一只猫能照看老鼠,真太不容易了。 
        “我也得感谢你妈妈,没有她,主人就不要我了。”咪丽说。 
        “对了,”舒克告诉妈妈,“人间要灭鼠了,从明天下午开始,家家投放鼠药,您可要当心,千万别吃!” 
        妈妈点点头。 
        “我们要去通知整个老鼠家族。”舒克说。 
        “这么多老鼠,你们怎么通知得过来?”妈妈问。 
        舒克和贝塔为难了。 
        “听说这座城市的鼠王住在一座教堂里,你们去找找。只要告诉鼠王就行,他马上可以传达到每只老鼠。”咪丽提供了一个信息。 
        舒克把罗丘做的点心留给妈妈,他和贝塔去找鼠王。 
        直升机降落在一座教堂顶上,舒克和贝塔从飞机和坦克里出来。 
        “飞机和坦克就放在这儿,保险。咱们爬下去。”舒克说。 
        他俩顺着墙缝儿往下爬,来到教堂的院子里。 
        教堂里静极了,黑乎乎的。舒克和贝塔隐约感到这寂静后边隐藏着恐怖的冷笑。 
        果然,就在他俩转身的时候,从黑暗中冲出几十只鼠兵,把他们围住了。 
        “干什么的?”一只老鼠问。看样子是小头目。 
        “你们是干什么的?”舒克见世面见大了。 
        “还挺厉害!我们是鼠王的御林军!你们深夜来王宫干什么?”小头目说。 
        “找鼠王有事!”贝塔不耐烦了,“快去通报!” 
        “你们是干什么的?”小头目要对鼠王的安全负责。 
        “两只老鼠。”舒克没好气地说。 
        大概是小头目也觉出这两位同胞气质不一般,忙去通报。 
        片刻后,舒克和贝塔被带进了王宫。 
        王宫坐落在一个墙角的地洞内,里边灯火辉煌。 
        鼠王坐在王位上。 
        “什么事?”鼠王傲慢地问。 
        舒克把人类灭鼠的信息告诉鼠王。 
        “真有此事?”鼠王吃惊。 
        舒克点点头。 
        “快通知臣民,万万不得食用鼠药!”鼠王下旨。 
        “且慢!”宰相出来说话了。 
        “讲。”鼠王说。 
        “臣有一计,何不命令臣民们将鼠药放人人类的食物中,给他们点儿颜色看看。”宰相献计献策。 
        鼠王大喜,照此传旨。 
        “这可不行!”舒克急了。 
        “怎么?”鼠王问。 
        “这是要死人的呀!”舒克说。 
        “他们怎么能毒死我们呢?”鼠王问。 
        “这……无论如何不行!”舒克恨死那宰相了。 
        可鼠王的圣旨已经传下去了。   第53集 
        舒克和贝塔在夜幕的掩护下直飞皮皮鲁家; 
        皮皮鲁闻讯大吃一惊   
        舒克和贝塔没想到鼠王会向老鼠家族发布将鼠药放到人的食物里去的圣旨,他俩气疯了。 
        “你们立了大功,我赏你们食物。”鼠王对舒克和贝塔说。 
        “你留着自己吃吧。”舒克说完拉起贝塔就走。 
        他们来到直升机和坦克旁边。 
        “这帮家伙,心眼儿太坏。”舒克为自己的同胞素质太差感到羞愧。 
        “咱们得想个办法,明天就要投放鼠药了。”贝塔急得直搓手。 
        一家一家通知?不可能,第一家人就会把他俩抓起来。去告诉鼠同胞别这样干?也来不及了,明天下午就要投放鼠药了,全城有多少老鼠! 
        真要把全城的人都毒死,太可怕了。 
        几乎是同时,舒克和贝塔想到了皮皮鲁的安全。 
        “咱们先去通知皮皮鲁!”舒克和贝塔异口同声。 
        舒克钻进直升机,贝塔钻进坦克。 
        直升机吊着坦克起飞。 
        舒克辨别了一下皮皮鲁家的方向,驾机直飞皮皮鲁家。 
        贝塔打开坦克舱盖儿,把头探出坦克。 
        月光覆盖了整座城市,微风轻拂过建筑物,万家灯火富有生命力地闪烁着。城市呈现安祥,好像在梦中微笑。贝塔看不出潜藏着危机。 
        “舒克,你看这城市还挺美。”贝塔通过无线电台同舒克说话。 
        “明天晚上就不美了。”舒克往地面看了一眼,说。 
        “唉,干吗都想互相置对方于死地呢。”贝塔叹了口气。 
        “到皮皮鲁家了。”舒克通知贝塔,  “注意观察,我着陆了。” 
        直升机吊着坦克平稳地降落在皮皮鲁家的阳台上。 
        舒克钻出直升机,贝塔钻出坦克。 
        阳台门关着。 
        舒克爬上窗台。屋里漆黑,皮皮鲁躺在床上睡觉。 
        “皮皮鲁!皮皮鲁!”舒克叫。 
        可惜他声占太小,熟睡中的皮皮鲁无动于衷。 
        “把门上的纱窗撕开吧?”贝塔觉得不能等到天亮了。 
        “行,快去拿工具。”舒克说完从窗台上溜下来。 
        贝塔钻进坦克,拎出工具箱。 
        舒克从工具箱里拿出剪子,把纱窗剪开一个口。 
        贝塔先钻进去,舒克紧跟着也钻进去了。 
        他们来到皮皮鲁的枕头旁边。 
        舒克拽拽皮皮鲁的耳朵。 
        “别,别。”皮皮鲁翻了个身。 
        贝塔趴在皮皮鲁耳朵旁边,大声说:“皮皮鲁,快醒醒,考试啦!” 
        皮皮鲁“噌”地坐起来。 
        这一着真灵。 
        “皮皮鲁,别紧张,是我们,舒克贝塔。”舒克赶紧说,生怕考试吓坏了皮皮鲁。 
        皮皮鲁拉开床头灯,笑了。 
        “怎么进来的?”皮皮鲁问。 
        “从那儿。”贝塔指指纱窗门上的豁口。 
        “破门而入呀!”皮皮鲁说。 
        “有急事找你。”舒克说。 
        “海盗又来了?”皮皮鲁问。 
        “我们已经把海盗打败了。”贝塔得意地说。 
        “真的?太棒了!”皮皮鲁必奋了。 
        “我俩离开航空公司了。”舒克告诉皮皮鲁。 
        “为什么?”皮皮鲁以为出了事情。 
        “我们不愿意老在一个地方呆着,干成了的事情,就把它当成终身职业,没劲!”贝塔解释。 
        皮皮鲁赞同地点头。 
        “你知道要灭鼠吗?”舒克坐在皮皮鲁的枕头上说。 
        “鼠王已下令,让老鼠们把鼠药放到你们人类的食物里去。”贝塔说。 
        皮皮鲁愣了。 
        “我俩挺着急,又想不出办法通知所有的人,只好先来告诉你,你可千万注意!”舒克一口气说下来。 
        “谢谢你们。”皮皮鲁说。不过他关心着全城居民的生命安全,“这鼠王也太坏了。” 
        “嗯,也怪我们,不该去告诉他别吃鼠药。”贝塔说。 
        皮皮鲁这才意识到是人先要毒死老鼠的。 
        “怎么办?皮皮鲁,你快想个办法。”舒克急切地说。 
        皮皮鲁看了一下表,已经是凌晨两点钟了。今天下午全城就要发放鼠药。   第54集 
        舒克建议登报; 
        舒克、贝塔和皮皮鲁前往报社; 
        同编辑见面   
        “那你就赶快去通知邻居,告诉一家算一家!”舒克对皮皮鲁说。 
        “我去告诉人家,人家准不信。他们会问,你是从哪儿知道的?”皮皮鲁想到这个问题。 
        “这倒是。”贝塔点头。 
        舒克一眼看见了桌上的报纸。 
        “登报!”舒克脱口而出。 
        皮皮鲁眼腈一亮。 
        “这主意不错,我知道《晚报》杜在哪儿,《晚报》发行量大,几乎家家有,咱们去找《晚报》的编辑。”皮皮鲁说完用最快速度穿衣服。 
        “编辑会同意吗?他准不信。”贝塔提醒皮皮鲁。 
        “去试试,没别的路了。”皮皮鲁说。 
        “咱们一起去,我们到外面等你。”舒克说。 
        “行,我得悄悄出去,不能惊动爸爸妈妈。”皮皮鲁穿好衣服,关上灯。 
        舒克和贝塔钻到阳台上,直升机吊着坦克升到空中,然后降低高度,在单元门口附近等着皮皮鲁。 
        皮皮鲁顺利地出来了,他骑上自行车朝《晚报》社奔去。舒克的直升机跟在他身后飞。 
        “拐过前边那个路口就是。”皮皮鲁告诉舒克。 
        《晚报》社到了。 
        “你们从墙上飞进去。”皮皮鲁告诉舒克。 
        报社的大门关着。皮皮鲁敲门。 
        一位老大爷探头出来。 
        “什么事?这么晚!” 
        “我有急事找编辑。”皮皮鲁说。 
        “急事?先跟我说说。”老大爷要把关。 
        “人命关天的事。”皮皮鲁说。 
        “人命关天?报警呀!”老大爷说。 
        “关系到全城人的生命,得登报。”皮皮鲁说。 
        “你是在编童话吧!”老大爷眯着眼睛看皮皮鲁,“我可没时间听你瞎侃。” 
        “真的,您快让我进去吧!”皮皮鲁急了。 
        老大爷看看皮皮鲁,不像有意来调皮的孩子,再说,哪个孩子会半夜三更来报社捣乱呢? 
        “这样吧,我打电话让夜班编辑下来。”老大爷打开大门,让皮皮鲁进去了。 
        皮皮鲁坐在会客室里,他看见舒克和贝塔趴在窗台上往里看。 
        一位戴眼镜的中年男人走进会客室。 
        “你找编辑?”眼镜编辑对于这么小的读者深夜来访感到惊讶。 
        “您是编辑?”皮皮鲁问。 
        “嗯,有事对我说吧。”眼镜编辑坐在沙发里。 
        “是这样,”皮皮鲁措着词,  “今天下午全城要投放鼠药,您知道吗?” 
        眼镜编辑点点头,示意皮皮鲁继续往下说。 
        “老鼠世界有个鼠王,您知道吗?” 
        眼镜编辑笑了,他摇摇头。 
        从这一笑中,皮皮鲁感到这位编辑可以信任。 
        “鼠王知道了咱们要毒他们,他就下令让全体老鼠把鼠药放回到人类的食物里。”皮皮鲁一字一句地说。 
        眼镜编辑瞪大了眼睛,他在判断面前这个男孩子神经是否正常。 
        “你怎么知道?”编辑问。 
        “我有两只老鼠朋友,他们特意来告诉我。”皮皮鲁说。 
        “老鼠朋友?”眼镜编辑不信。他感到越发奇了。 
        “真的。”皮皮鲁脸真诚。 
        眼镜编辑不说话了,他注视着皮皮鲁。 
        皮皮鲁看出对面这个人心眼儿不坏,可以冒一下险,况且他也没本事抓住舒克和贝塔。 
        “我的老鼠朋友就在外边,我把他们叫进来,您看看?”皮皮鲁问。 
        “在外边?你能叫进来?”眼镜编辑显然不信。 
        “您必须答应一个条件。” 
        “你说吧。” 
        “为舒克和贝塔保密。” 
        “谁是舒克贝塔?” 
        “就是我的老鼠朋友呀!” 
        还有名字! 
        “行,我答应这个条件。”眼镜编辑有点儿信皮皮鲁了。 
        皮皮鲁走出会客室。 
        他在外面同舒克贝塔制定了应急措施后,领着他们走进会客室。 
        眼镜编辑见皮皮鲁真的领进来两只老鼠,而且老鼠身上还穿着飞行服和坦克兵服,他呆了。 
        “您信了吧?”皮皮鲁同。 
        “这……这是怎么回事?”编辑还反应不过来。 
        “鼠王真的下了道命令,让他的臣民把鼠药放回到人的食物里。”舒克说。 
        眼镜编辑差点儿从沙发上蹦起来,老鼠说话了。 
        他现在确信不疑关于老鼠要毒人类的事了。 
        “你们想怎么办?”他问。 
        “登报,告诉全城的人,千万提高警惕。”皮皮鲁说。 
        “登报!在报上说老鼠要毒人类?!谁信呀!”眼镜编辑嚷嚷道。太滑稽了。 
        “这是人命关天的事。”皮皮鲁提醒眼镜编辑。 
        “让人们每天检验食物里是否有毒?”眼镜编辑问。 
        皮皮鲁愣了。也是,谁家有天天检验食物是否含毒的设备? 
        “那就告诉大家别把鼠药投放出去。”皮皮鲁说。 
        “只有这么办了。可这不是干扰灭鼠吗?”眼镜编辑为难了。 
        “上百万人的性命重要。”皮皮鲁说。 
        “好,我同意!但还得同总编辑商量。”眼镜编辑站起来,“你在这儿等我。” 
        眼镜编辑去找总编辑了。 
        皮皮鲁让舒克和贝塔躲出去。   第55集 
        总编辑坚决反对刊登消息; 
        皮皮鲁说出校名也没用; 
        舒克和贝塔想出高招儿   
        眼镜编辑来到夜班总编辑办公室。他将老鼠要毒人类的紧急情况向总编辑汇报。 
        “什么什么?你是在做梦吧?”总编辑放下手中的红笔,惊诧地看着眼镜编辑。 
        “这是真的!”眼镜编辑认真地说。 
        “现在是上班时间,别开玩笑。”总编辑开始生气了。 
        “怎么是玩笑?!” 
        “你是说,全城的老鼠准备把人类毒它们的鼠药反过来毒人类?” 
        “千真万确。” 
        “简直是无稽之谈。” 
        “总编辑,你怎么能不相信!” 
        “我怎么能相信?!” 
        “那孩子就在传达室坐着。” 
        “好,你叫他上来。” 
         眼镜编辑用最快的速度跑到传达室的会客厅。 
        “总编辑不信,他叫你上去。”编辑对皮皮鲁说,“也难怪,谁信呀!” 
        “你没把舒克贝塔供出来吧?”皮皮鲁不放心。 
        “没有。”编辑恪守诺言。 
        “我去同总编辑说。”皮皮鲁站起来。 
        眼镜编辑领着皮皮鲁来到总编室。 
        总编辑注视了皮皮鲁半分钟,判断这孩子干吗深更半夜跑到报社来捣乱。 
        还没等总编辑开口,皮皮鲁先说了。 
        “这都是真的!” 
        “你是哪个学校的?”总编辑突然提出这样的问题。 
        皮皮鲁噎住了。 
        “哪个学校?”总编辑重复了一遍。 
        皮皮鲁清楚,这事要是通知了学校,挨处分是轻的。 
        “你看,他连学校都不敢说出来,这事能真吗?”总编辑问眼镜编辑。 
        “这……”眼镜编辑完全理解皮皮鲁为什么不愿意暴露自己的“单位。” 
        “好了,你回去吧,这次原谅你。以后如果再来报社调皮,我可不客气喽。”总编辑宽容地笑笑。 
        皮皮鲁说出了自己的校名。他觉得值得。 
        总编辑愣了一下,马上把校名记在台历上。 
        “可以登报了吗?”皮皮鲁问。 
        “当然不行。”总编辑毫不动摇。 
        “我负责。”皮皮鲁说。 
        “谁负我的责?这消息一登,我这总编辑就得被撤职。” 
        “这关系到全城人的性命!”皮皮鲁急切地说。 
        “够了!我没时间陪你胡闹了。领他出去吧!”总编辑示意眼镜编辑。 
        眼镜编辑还想作一次努力,当他看见总编辑的目光时,把话咽了回去。 
        “走吧。”眼镜编辑对皮皮鲁说。 
        皮皮鲁还想说什么,总编辑扬扬手。 
        皮皮鲁只好跟着眼镜编辑离开了总编室。 
        “没办法。”眼镜编辑站在报社门口,冲皮皮鲁耸耸肩膀。 
        “您多加注意,把食物保管好。”皮皮鲁很感谢这位编辑。 
        “我家那栋楼我包了。” 
        “再见。” 
        皮皮鲁离开报社。舒克和贝塔在外边等他。 
        “不行。”皮皮鲁对舒克和贝塔说。 
        “我们趴在窗户上听见了。”贝塔说。 
        天蒙蒙亮了。 
        “我们刚才在院子里转了一圈,看见印刷厂就在报社的院里。”贝塔说。 
        “这有什么用?”皮皮鲁找了块砖头坐下。 
        “印刷车问里放着《晚报》的版,咱们偷偷把消息放上去不就行了?”舒克说。 
        “总编辑要审稿的呀!”皮皮鲁说。 
        “咱们等他审完了,签上字,再去改!”舒克已经把报纸出版的程序摸清了。 
        “太棒了!”皮皮鲁一跃而起。 
        “我们打听了,报纸是上午11点钟付印。你把消息的内容写给我们,你就去上学,我们目标小,我们来改版。”贝塔振振有词。 
        “伟大!”皮皮鲁冒出这么一句,“你们向谁打听的?” 
        “向住在报社的老鼠呗。”舒克为自己的同胞遍布全城自豪了。 
        “他们不会向鼠王报告?” 
        “老鼠也不都坏。”贝塔说。 
        “当然!当然!”皮皮鲁深有感触。 
        “快写消息吧。”舒克催促。 
        皮皮鲁掏出纸和笔,垫在膝盖上写起来。 
        “嗯……先写个标题,就写紧急通知吧。”皮皮鲁边写边说。 
        “本报紧急消息,得知鼠王发布了一道圣旨,命令全城的老鼠把人类投放的鼠药再放回到人类的食物中去。为此,本报提醒全城市民千万别投放鼠药,并请互相转告。” 
        皮皮鲁写完后又看了一遍,然后交给舒克。 
        “就按这个排版。”皮皮鲁的口气像主编。   第56集 
        名叫头版的老鼠帮助舒克和贝塔; 
        印刷机印刷皮皮鲁写的消息   
        皮皮鲁回家了,舒克和贝塔将直升机和坦克隐蔽在报社院内的草丛中。 
        舒克把居住在报社的小老鼠叫到直升机旁边。 
        “真厉害,你还会开飞机。”那小老鼠惊叹道。 
        “这儿还有坦克呢!”贝塔对同胞的眼神不济感到遗憾。 
        小老鼠惊讶得张大了嘴巴合不拢。 
        “你叫什么名字?”舒克问。 
        “头版。”小老鼠说。 
        “头版?”舒克和贝塔异口同声,他们觉得这不像是老鼠的名字。 
        “我妈生我的时候,是垫着报纸的第一版生的我,所以给我取名叫头版。”头版解释说,“我妈说这名字不俗,没人叫。” 
        “我叫舒克,他叫贝塔,咱们是朋友了。”舒克对头版说,“谢谢你晚上帮助我们。” 
        “别客气,应该的。”头版爽快地说。 
        “咱们到飞机里坐会儿。”舒克邀请头版上飞机。 
        头版兴奋地钻进直升机。 
        舒克、贝塔和头版分别在皮椅上就座。 
        “你们到底要在这儿干什么?”头版好奇地问。 
        舒克和贝塔对视了一下,认为可以信任头版。 
        “你知道鼠王的圣旨吗?”贝塔问。 
        “是把鼠药放到人的食物里去的圣旨吗?”头版问。 
        “正是,我们想制止这样的惨事。”舒克看着头版说。 
        “你们要帮助人类?”头版站起来。 
        “对。”贝塔说。 
        “叛徒。”头版咬牙切齿地吐出两个字。 
        “人类要投鼠药的消息是我们告诉鼠王的。”舒克说。 
        “那你们干吗又反过来帮人类的忙?”头版不理解。 
        “我们希望大家都活在这个地球上,谁也别害谁。”贝塔说。 
        “人总是害我们!”头版提醒两位同胞。 
        “他们以后会明白的,真要是把老鼠消灭光了,世界将会变成什么样。可我们也不能把人都毒死呀!”舒克说,“有个叫皮皮鲁的男孩子,就对我们特别好。” 
        “你们想怎么办?”头版问。他已经动摇了。 
        “登报。提醒市民们注意,别投放鼠药了。”贝塔说。 
        “得请你帮忙。等总编辑签字后,你去把版上的铅字换了。”舒克把消息的稿子递给头版。 
        头版看了一遍稿子,同意了。 
        “这事千万别让你家的其他老鼠知道,要保密。”舒克告诫头版。 
        头版点点头。 
        “咱们制定一下计划……”舒克说。 
        三只老鼠躲在直升机里策划着修改《晚报》头版内容的步骤。 
        “好了,现在咱们睡觉,养精蓄锐。”舒克说。 
        “我给你们放哨。”头版要尽地主之谊。 
        舒克和贝塔躺在皮椅上睡着了。 
        “快到点了,醒醒!”头版叫舒克和贝塔。 
        舒克和贝塔一边揉眼睛一边坐起来。 
        “行动吧!”舒克打开飞机舱门。 
        三只老鼠离开直升机。 
        他们潜入印刷车间,躲在一台机器下边。 
        “看,那台子上放的就是。”头版指给朋友看。 
        一个人拿着一张纸走到版跟前。 
        “总编辑签了字,行动吧!”头版极有经验,他从生下来就看印刷出版,对这套把戏熟透了。 
        舒克从机器下边冲出来,故意从那人的鞋上溜过去。 
        “啊,老鼠!抓老鼠呀!”那人看见舒克,大喊起来。 
        整个车间沸腾了,人们抄起扫帚、拖把,蜂拥而来,又蜂拥而去。 
        “快,跟我来!”头版招呼贝塔来到捡字盒边,盒里盛着各种铅字。 
        头版飞快地从盒里抽出所需要的铅字,贝塔按顺序排好。那时的报社还采用铅字印刷。 
        等人们精疲力竭地回到岗位上时,版已被换了。 
        印刷机开始工作了。 
        震耳欲聋的轰鸣声仿佛要把房顶揭开。 
        头版偷了一张印好的报纸,给舒克和贝塔看。 
        “成功了!”舒克兴奋得在头版脸上亲了一下。 
        “我们走了,以后来找你玩!”贝塔把报纸收好。 
        “别忘了我!”头版恋恋不舍地同两位朋友告别。 
        舒克和贝塔分别钻进飞机和坦克。 
        直升机吊着坦克起飞了,朝皮皮鲁家飞去。   第57集 
        总编辑被撤职去当排字工; 
        皮皮鲁挨处分还挺高兴; 
        人鼠平安   
        下午,全城都在发放鼠药。 
        老鼠世界也作好了充分的准备,时刻准备将鼠药放进人类的食物里。 
        就在人们正要投放鼠药时,他们看到《晚报》头版上的《紧急通知》。 
        尽管人们对这条消息的真实性表示怀疑。可没人愿意拿自己的性命作试验。 
        鼠药都被小心翼翼地保管好。 
        市灭鼠委员会火了,他们打电话给《晚报》社。 
        “喂,是《晚报》总编室吗?” 
        “是的,什么事?” 
        “我是灭鼠委员会,我找总编辑。” 
        “我就是。” 
        “我们抗议!最强烈的抗议!!” 
        “抗议?为什么?” 
        “你们干扰灭鼠!” 
        “我们干扰灭鼠?” 
        “今天《晚报》的头版!” 
        “头版?你等等。” 
        总编辑从写字台右侧拿起当天的《晚报》,他的脸色渐渐白了。 
        “这,这……”总编辑说不出话来。 
        “我们要告你!” 
        灭鼠委员会的人挂了电话。 
        总编辑突然想起了那个深夜来访的男孩子。 
        他给眼镜编辑打电话。 
        “你看看头版!” 
        “我看见了。” 
        “那你为什么不告诉我?” 
        “我以为是您安排的稿子。” 
        “我安排的?!这到底是怎么回事?” 
        “我也不知道。你可以问问印刷车间,我根本就没去过。” 
        总编辑又打电话给印刷车间,答复是从未有人来过。 
        负责灭鼠的副市长来电话了,声称要追究法律责任。 
        总编辑慌了,他翻出台历上皮皮鲁的校名,又通知查号台查出了学校的电话号码。 
        电话通了。 
        “我找校长,教导主任也行。”总编辑说。 
        “我是教导丰任。” 
        “我是《晚报》总编辑。” 
        “哟,您好!”教导主任受宠若惊。 
        “昨天深夜,有贵校的一个男学生到我们报社来,声称老鼠要毒人类,你能帮助查一下是谁吗?” 
        “多大岁数?穿什么衣服?特征?” 
        总编辑把皮皮鲁的特征讲给教导主任听。 
        皮皮鲁在学校是著名人士,教导主任猜到是皮皮鲁。 
        “我现在去证实一下,马上答复您。”教导主任放卜电话。 
        教导主任蹬上自行车,直奔皮皮鲁家。 
        皮皮鲁全家正在吃饭。 
        敲门声。 
        皮皮鲁去开门。 
        “哟,教导主任!”皮皮鲁一愣。 
        “你昨天夜里去《晚报》社了吗?”教导主任劈头就问。 
        皮皮鲁知道是查灭鼠的事,他点点头。 
        “你惹了大乱子!”教导土任说。 
        “怎么回事?”皮皮鲁的爸爸皮威端着碗出来问。 
        教导主任把经过简要地讲给皮皮鲁的家长听。 
        “这,这可能吗?”皮威不相信。 
        “我去了,可总编辑根本不相信我的话,他说了不发呀!”皮皮鲁装傻。 
        “你从哪儿知道老鼠要毒人类?老鼠来告诉你了?”皮威生气了。不管怎么说,反正儿子是去报社了。 
        “我先去给报社总编辑打个电话。”教导主任转身下楼去了,那口气仿佛报社总编辑是他的老朋友。 
        市政府专门为此事件成立了调查组,总编辑即使浑身是嘴也说不清楚。调查组不信皮皮鲁这么小的男孩子能左右报社的版面。 
        “确实是他来说的。”总编辑有气无力。 
        “他说了,你就登报?”官员问。 
        “我不同意。” 
        “那报上怎么登了?” 
        “这……” 
        市政府决定撤销《晚报》总编辑的职务,下放到印刷厂当排字工。 
        学校也给皮皮鲁记过处分。 
        “值得,值得!”舒克说。 
        第二天的《晚报》头版上发表了检查,并动员市民们马上投放鼠药。 
        鼠药是投放了,可人们多了心眼儿,把自己家的食物都严加看管起来,以防老鼠发坏。 
        老鼠们无奈,无从下手。 
        老鼠不吃一粒鼠药,一只老鼠未毒死。 
        人将食物严加看管,一粒鼠药放不进去。 
        “这结局不错。”贝塔得意地说。 
        “咱们该庆祝庆祝。”舒克说。 
        “我可背了个处分。”皮皮鲁说。   第58集 
        海盗担任追捕舒克贝塔别动队队长; 
        舒克不知道陷阱正等待着他   
        自从鼠王颁发了那道圣旨后,他就梦想着全城的人都被毒死后的情景。 
        “那我就是全城的大王了。”鼠王得意洋洋地想。 
        “禀报鼠王,人类拿着鼠药不撒手。”一位大臣前来汇报。 
        “为什么?”鼠王不解。 
        “据说是一家什么报纸发表文章警告市民。”大臣说。 
        “有这样的事?快去弄一份来!”鼠王的梦碎了。 
        大臣立即派部下去找报纸。 
        半小时后,《晚报》到了鼠王手中。 
        鼠王不认识人类的文字,让一位大臣给他念。 
        “这是谁干的?!”鼠王听完大怒,他认定自己的部下中出了叛徒。 
        “禀报鼠王,外边有一只老鼠求见。”门卫进来说。 
        “让他进来。”鼠王没好气地说。 
        一只老鼠来到鼠王面前,他的胳膊上缠着绷带。 
        “你有什么事?”鼠王问。 
        “我知道是谁向人类告的密。”那老鼠慢悠悠地说。 
        “快说!”鼠王急不可待。 
        “是两只叫舒克和贝塔的老鼠干的。” 
        “舒克贝塔?”鼠王从记忆中搜寻着这两个名字。 
        “就是那两只来向您报告人类灭鼠消息的老鼠。” 
        “是他们!”鼠王想起来了。 
        “这两个家伙是我们老鼠家族的败类,他们利用直升机和坦克,干了一系列违背我们老鼠品质的坏事,还和猫勾结起来。” 
        “啊!”鼠王对此一无所知,“你这么了解他们?” 
        “我和他们打过空战!” 
        原来是海盗。 
        自从舒克和贝塔率领轰炸机炸平了海盗的机场后,海盗逃跑了。他一直追踪舒克和贝塔的足迹,寻机报仇。机会终于来了。 
        “通缉这两个叛徒!”鼠王下令。 
        “要抓活的,审判他们!”海盗建议, 
        “对,抓活的!”鼠王说。 
        “这两个家伙不好对付。”海盗提醒鼠王。 
        “嗯。成立一个特别行动队,由你担任队长。”鼠王封海盗为追捕舒克和贝塔别动队队长。 
        “愿为鼠王效劳!”海盗向鼠王叩首。 
        “行动吧!”鼠王说。 
        海盗当上了别动队长。追捕舒克和贝塔的天罗地网张开了。 
        这几天,舒克和贝塔一直在皮皮鲁家休息,皮皮鲁拿最好吃的食物款待他们。 
        “咱们该出去闯闯了吧?”贝塔不喜欢这种安宁的日子,没味儿。 
        “走。先去看看妈妈和咪丽。”舒克也不愿老呆着。 
        “等等,”皮皮鲁来到阳台上,递给舒克一把东西,又递给贝塔一把。 
        舒克和贝塔一看,是石头炮弹,他们收下了。 
        “起飞!”皮皮鲁一挥手。 
        直升机缓缓升到空中,坦克也离地了。 
        “你看,下边是报社。”舒克通过无线电台对贝塔说。 
        “咱们去看看头版吧?”贝塔提议。 
        “行。”舒克也很想见见头版。 
        直升机和坦克悄悄地降落在花丛中。 
        舒克跳下飞机。贝塔钻出坦克。 
        “你守在这儿,我去找。这样保险。”舒克说。 
        “你快点儿。”贝塔又爬进坦克。 
        舒克钻出花丛,朝头版家走去。 
        草丛中有一双眼睛盯着舒克,这是头版的哥哥,他认出了面前这只老鼠就是鼠王通缉的罪犯。 
        头版的哥哥飞速回家报信。他知道,抓住罪犯有重赏。 
        “快!快!”头版的哥哥气喘吁吁跑进洞。 
        “出什么事了?”头版的爸爸问。 
        “鼠王通缉的罪犯朝咱们家走来了!” 
        “真的?”头版的妈妈大喜。 
        “千真万确。” 
        “快布陷阱抓他!”爸爸觉得发财的时机到了。 
        头版慌了,他想去通知舒克和贝塔。 
        “你干什么去?”哥哥拦住了弟弟。 
        “我、我有点儿小事。”头版支支吾吾。 
        “不许出去!”哥哥厉声道。 
        一张大网悬在洞口上边,准备扣舒克。 
        “我出去引他来。”哥哥出去了。 
        头版的心提到了嗓子眼儿。   第59集 
        舒克洞口遇险; 
        贝塔炮击头版的爸爸; 
        头版入伙   
        舒克陕走到头版家的洞口时,迎面走来一只老鼠。 
        “请问头版住这儿吗?”舒克问。 
        头版的哥哥愣了一下,他不知道舒克是怎么认识他弟弟的。管他呢,先抓住舒克再说。“我是头版的哥哥,我带你去。” 
        “谢谢你啦!”舒克一点儿没怀疑。 
        “头版在家吗?”舒克边走边问。 
        “在家等你呢!” 
        “他怎么知道我来了?”舒克放慢了脚步,他感到奇怪。 
        “他,他……”头版的哥哥知道说漏了,忙往回找,“他刚才看见你了,先回去给你准备吃的,让我来接你。” 
        舒克点点头,又跟着头版的哥哥朝他家走去。 
        到洞口了。 
        “请。”头版的哥哥站在洞口旁请舒克进去。 
        舒克径直朝里走。 
        就在这时,只听洞里传出头版的喊声: 
        “舒克,快跑,他们要抓你!” 
        舒克心里已经有了准备,听到喊声,他转身就跑。 
        头版的哥哥扑上来,死死掐住舒克的脖子。 
        舒克使劲儿踢了他小腹一脚,头版的哥哥捂着肚子蹲下了。 
        这时,头版的爸爸、妈妈,姐姐、弟弟都从家里跑出来追捕舒克,舒克知道寡不敌众,忙朝花丛中跑去。 
        贝塔正懒洋洋地躺在坦克里哼歌,突然听见纷乱的脚步声。他把头探出坦克看,只见舒克身后有好几只老鼠在追他。 
        “怎么回事?”贝塔大声问。 
        “开炮打他们!”舒克钻进直升机。 
        贝塔钻进坦克,瞄准了跑在最前边的老鼠,那是头版的爸爸。 
        贝塔按下了射击按钮。 
        头版的爸爸大叫一声,倒下了。 
        老鼠们站住了,惊恐地望着面前的坦克和直升机。 
        贝塔接通了电台,问舒克: 
        “还打吗?” 
        “暂停。看看他们的动静。” 
        “怎么回事?” 
        “他们设好陷阱抓我。” 
        “头版家?” 
        “头版救了我!” 
        “他在哪儿?” 
        “对了,咱们应该找到他,问问是怎么回事。” 
        “我瞄准,你喊话,吓唬他们!” 
        舒克打开飞机舱门,冲老鼠们说: 
        “快去把头版叫来,不然我开炮了!” 
        老鼠们不知道什么是开炮,站着不动。 
        “开炮!”舒克对贝塔下令。 
        “嗵——” 
        头版的姐姐身边的花被打碎了。 
        她的脸吓白了。 
        “快去叫头版来!”舒克说。 
        头版的姐姐跑回去放头版出来,原来,他们已经把头版关禁闭了。 
        不一会儿,头版来了。 
        “快上飞机!”舒克对他说。 
        头版钻进直升机。 
        “我们可以走了吗?”头版的姐姐胆怯地朝直升机问话。 
        “呆会儿,先别动!”舒克说。 
        头版对舒克说: 
        “鼠王知道了是你和贝塔向人类告的密,现在正通缉抓你们!还专门成立了别动队。” 
        “原来是这样!”舒克明白了。 
        “我也回不了家了,他们会处决我的。”头版说。 
        “跟我们走吧。”舒克让头版先在客舱休息。 
        “贝塔,准备起飞。”舒克拿起话筒。 
        “明白。”贝塔回答。 
        直升机吊着坦克起飞。 
        头版的家人仰着头一边看天上一边发呆。他们这才知道自己根本不是舒克和贝塔的对手。 
        等飞机飞远后,头版的妈妈最先清醒过来,她大喊一声: 
        “快去报案!” 
        海盗接到报案后,立即率领别动队赶到现场。 
        “他们朝哪个方向飞的?”海盗问: 
        “东边。”头版的妈妈指给海盗看。 
        “你们逃不出我的手心!”海盗咬牙。 
        “但愿快点儿抓住。”头版的妈妈觉得直升机和坦克太可怕。 
        “这么说,你的儿子也入伙了!”海盗盯着头版的妈妈问。 
        “对。”头版的妈妈大义灭亲。 
        “连头版一起通缉。”海盗下命令。   第60集 
        舒克、贝塔和头版化装; 
        用花生米换情报; 
        电击鼠王卫兵   
        舒克和贝塔本来马上要离开这座城市去外边闯闯,现在他们知道了鼠王在通缉他们,他们决定不走了,跟鼠王的别动队较量较量。 
        舒克把直升机降落在一座楼顶上。 
        “咱们化一下装,下去玩玩。”舒克对伙伴们说。 
        “行。”贝塔把坦克服脱了。 
        舒克把飞行服也脱了,又往脸上涂了点土。 
        头版也打扮了一番。 
        “走!”舒克看看飞机和坦克挺安全后,说。 
        天渐渐黑了。 
        舒克、贝塔和头版从楼顶上爬下来,舒克看看四周,没有可疑迹象。 
        “走,去鼠王的王宫。”舒克说。 
        “咱们也认识认识别动队长。”贝塔说。 
        鼠王居住的那座教堂就在附近。 
        “咱们从地沟走。”舒克带头钻进地沟。 
        贝塔和头版跟在后边。 
        舒克知道,地沟是老鼠的天下。越是在老鼠聚集的地方,他们就越安全。 
        这段地沟是老鼠的集市,许多老鼠在这儿交换食物。有的用花生换香肠,有的拿面包换腊肉。 
        “这地沟准通王宫,咱们打听一下。”舒克小声对贝塔和头版说。 
        一只老鼠朝舒克他们走来。 
        “要咸菜吗?”那老鼠问。 
        “什么咸菜?”舒克装做感兴趣的样子。 
        “甜辣萝卜干。” 
        “有泡菜吗?”头版问,他爱吃泡菜。 
        “没有。”那老鼠摇摇头,“你们有什么?” 
        “花生米。”舒克说。 
        “换点儿吧?”那老鼠一听说花生米,咽了口唾沫。 
        “你拿什么换?”舒克问他。 
        “我只有……甜辣萝卜干。”那老鼠面带难色。 
        “你知道哪条路通王官吗?”舒克小声问。 
        “当然知道,干什么?”那老鼠看看舒克。 
        “你告诉我,我给你花生米。”舒克冲他挤挤眼睛。 
        “真的?”那老鼠不信有这么便宜的事。 
        “当然是真的。”贝塔拍拍他的肩膀。 
        “从这往前,过两个口,往右就是。不过进不去,有鼠兵把守。”那老鼠说。 
        舒克给了他一把花生米。 
        那老鼠一看是真的,兴奋了。一边往嘴里塞花生米一边问: 
        “你们还想打听什么路?” 
        “我们没那么多花生米。”贝塔耸耸肩膀。 
        舒克、贝塔和头版朝通向王宫的路口走去。果然,路口站着两名鼠王的卫队队员。 
        舒克假装不知道,径直往里走。 
        “站住!”卫兵喝道。 
        舒克吃惊地望着卫兵。 
        “这是王宫。”卫兵说。 
        “我就是去王宫。”舒克说。 
        这时,贝塔和头版分别靠近两名卫兵。 
        “去王宫干什么?”卫兵盘问。 
        “你看他们手里拿的什么?”舒克让卫兵看贝塔和头版。 
        卫兵的目光刚移过去,贝塔和头版打开了手中电棍的开关,将电棍触到卫兵的身体上。 
        两名卫兵顿时休克了。 
        其实,电棍里只有一节二号电池,这是舒克和贝塔创造的新式武器。 
        “快走!”舒克一挥手。 
        他们朝深处跑去。 
        走到无路可走时,贝塔指指上边,说: 
        “就在上边。” 
        三位勇士朝上爬去,他们看见了亮光。 
        “你们等一下,我先侦察侦察。”舒克对同伴说。 
        舒克从地沟的出口探出头去,有一名卫兵守在出口,正打瞌睡。 
        这是王宫的院子。 
        舒克冲贝塔招招手,贝塔过来了。舒克趴在他耳朵上说了几句。 
        舒克和贝塔同时冲出去,将卫兵拖进地沟里。头版冒充卫兵出去站岗。 
        “你……你们……”卫兵醒进来时,发现自己被劫持了。 
        “鼠王住在哪儿?”贝塔用电棍指着卫兵问。 
        “……”卫兵不说。 
        贝塔用电棍轻轻碰了碰他。卫兵马上说了。   第61集 
        舒克和贝塔到鼠王的床边; 
        贝塔拔鼠王的胡子; 
        海盗同老对手见面   
        贝塔掏出绳子,把卫兵捆上。舒克往卫兵嘴里塞布条,堵住他的发声通道。 
        舒克和贝塔将卫兵安置好后,钻出地沟。 
        “你在这儿替他站岗,我们去治治鼠王。”舒克对头版说。 
        “当心点儿。”头版说。 
        贝塔和舒克接近了鼠王的住处,洞口又有两名卫兵。 
        “讨厌,设这么多卫兵干什么!”舒克觉得官越大越怕死。 
        “你还记得克里斯国王吗?”舒克问。 
        “嗯,差不多。”贝塔说。 
        “准备电棍吧。”舒克掏出电棍。 
        贝塔也准备好。 
        他俩朝两名卫兵走去。 
        “站住。”卫兵喝道。 
        “给鼠王送东西。”舒克将打开了开关的电棍递给卫兵。 
        贝塔也如法炮制,将电棍递给另一名卫兵。 
        两位卫兵同时伸手接电棍,又同时凝固了。 
        舒克和贝塔推倒休克的卫兵,冲进鼠王的卧室。 
        鼠王正躺在床上吃香肠。 
        “你们好大胆,不禀报就闯进来!”鼠王大怒。 
        “禀报鼠王,我们是来报告舒克和贝塔的消息的。”舒克说。 
        “抓住了?!”鼠王一跃而起。 
        “嗯。”贝塔点头。 
        “太好了!”鼠王捋捋胡须,满面顿生春风。 
        “怎么处置?”舒克想知道他俩如果落到鼠王手中的下场。 
        “斩首!”鼠王毫不犹豫,“不过,这太便宜了他们。应该千刀万剐。还不过瘾。用开水烫。用开水烫?不解气,对,把他们的腿砍了,扔到大街上,让人去处置他们!谁让他们当叛徒呢!” 
        舒克和贝塔身上同时起了一层(又鸟)皮疙瘩。 
        “他们已经来了。”贝塔说。 
        “在哪儿?”鼠王看看洞里,没有。 
        舒克和贝塔把伪装拿掉,露出真容。 
        鼠王呆了,他刚想喊,贝塔冲上去按住他的脖子。 
        “我装鼠王,咱们见见那别动队长。”舒克开始化装成鼠王的模样,穿上鼠王的衣服。 
        “像吗?”舒克问贝塔。 
        “就是胡子不像。”贝塔左看右看。 
        “把他的胡子拽下来,粘在我嘴上。”舒克想了个好办法。 
        贝塔按住鼠王,把他的胡子一根一根揪下来。 
        气得鼠王直咬牙。 
        鼠王的胡子粘在了舒克的嘴上。 
        “真像!”贝塔说。 
        舒克把事先准备好的飞行服给鼠王穿上,贝塔把鼠王捆好,又将嘴堵上,再用一根黑布条把他的眼睛蒙上。谁也认不出这是鼠王。 
        “你冒充卫兵去叫别动队长,我在这儿等着。”舒克躺在床上。 
        鼠王坐在地上。 
        贝塔跑出去,先把两个昏迷的卫兵拖进来塞到床底下,然后去叫别动队长。 
        当贝塔找到别动队长时,他差点儿喊出来,是海盗。 
        海盗同贝塔见面不多,投认出来,他听说鼠王传他,忙跟贝塔去了。 
        贝塔的心已经提到嗓子眼儿,他生怕海盗认出舒克来。 
        走到鼠王的卧室洞口时,贝塔对海盗说: 
        “你等一下,我去禀报一声。” 
        贝塔先进去,告诉舒克别动队长的姓名。 
        “海盗!”舒克大吃一惊,真是冤家路窄。 
        “怎么办?”贝塔问。 
        “把蜡烛弄暗点儿。”舒克说,“让他进来.够刺激。” 
        贝塔来到洞口,对海盗说: 
        “鼠王有请。” 
        海盗不耐烦地往里走。他早想好了,只要一抓住舒克和贝塔,马上就搞政变,篡夺王位。   第62集 
        海盗鞭打鼠王; 
        鼠王咽不下这口气; 
        咪丽中计   
        海盗来到鼠王的床边。 
        “您有什么吩咐?”海盗问。 
        “我抓住舒克了。”舒克指指蹲在地上的鼠王。 
        “真的?”海盗乐了,尽管他觉得鼠王的发音不大对头,可他顾不上分析了。 
        海盗走到蹲在地上的鼠王旁边,他一眼认出舒克的飞行服。 
        “没错,就是他!”海盗抡圆了胳膊打了鼠王一记耳光。 
        “你这么恨他?”舒克坐在床上问。 
        “他是咱们老鼠家族的败类!”海盗说完又狠踢了鼠王一脚。 
        “把他交给你了!”舒克说。 
        “由我来惩治他!”海盗拎起鼠王朝洞外走去。 
        “快溜!”贝塔对舒克说。 
        舒克从床上跳下来,和贝塔离开了鼠王的卧室。 
        头版见舒克和贝塔回来了,问: 
        “怎么样?” 
        “完事了,快走!”舒克说完先钻进地沟。贝塔和头版跟着钻进去。 
        他们平安地回到直升机上。 
        舒克和贝塔给头版表演冒充鼠王的经过,逗得头版差点笑破了肚皮。 
        “海盗怎么跑到这儿来了?”舒克自言自语。 
        “这家伙不一般,咱们得留心点儿。”贝塔说。 
        “嗯。”舒克又想起同海盗空战的情景。 
        海盗将“舒克”押到别动队总部,他抄起一根皮鞭,劈头盖脑抡过去,“舒克”嘴被堵住了,喊不出声来,他拼命挣扎。 
        “让你知道知道我的厉害!”海盗使出全身力气抽“舒克”。 
      “舒克”在地上打滚儿。 
      “把他嘴里的布掏出来,我要问话。”海盗对部下说。 
        布刚从嘴里掏出来,鼠王立刻从地上爬起来,大喊:“混蛋!我是鼠王!” 
        “你还敢冒充鼠王!”海盗一鞭子抽过去,打得鼠王嘴角直淌血。 
        “我是鼠王,快给我松绑!”鼠王气急败坏地跺脚。 
        又是一鞭子。鼠王沉默了。 
        “把他的遮眼布摘了!”海盗命令。 
        遮眼布一摘,海盗傻眼了。 
        “快,快松绑!”海盗一边叫一边亲自去给鼠王松绑。 
        鼠王一脚把他踢出去老远。 
        “来人,把海盗抓起来!”鼠王要出这口气。 
        “大王息怒!”鼠王的一位大臣上前劝阻,“要想抓住逃犯,非他不可。再说他也是恨逃犯呀!” 
        鼠王摸着身上的伤处,他实在咽不下这口气。 
        “大王恕罪!”海盗跪在地上给鼠王磕头,“我三天之内非抓住舒克不可,不然拿我问罪!” 
        “好,给你三天时间!”鼠王一屁股坐在地上,他疼得站不住。 
        “我派出去的密探打听到舒克的妈妈就住在本城,咱们去把他妈妈抓来,舒克孝顺,还怕他不来上钩?”海盗用心毒辣。 
        “马上去抓!”鼠王下令。 
        海盗率领着别动队出发了,他们直奔咪丽的主人家。 
        咪丽和舒克的妈妈正在床底下进餐,咪丽忽然支起耳朵。 
        “有老鼠来了。’咪丽说。 
        “是舒克吗?” 
        “不像。”咪丽趴在地上往外看。 
        当海盗看见床下有猫时,他们立刻分散不敢靠近床。 
        “他们看见你,不敢过来。”舒克的妈妈对咪丽说。 
        “您去跟他们说。”咪丽往后退了几步。 
        舒克的妈妈走出去问:  “你们是舒克派来的吗?” 
        “正是。”海盗灵机一动,  “舒克让我们来接他妈妈。” 
        “我就是舒克的妈妈,舒克在哪儿?” 
        “舒克找了座新房子,让您看看。”海盗诓她。 
        “我去看看,一会儿回来。”舒克的妈妈对咪丽说。 
        咪丽同意了。   第63集 
        坦克直逼鼠王王宫; 
        贝塔用炮管抡鼠兵; 
        舒克遇险   
        当咪丽发觉上当时,已经是两天后的事了。 
        这天夜里,舒克和贝塔来到咪丽面前。 
        “我妈妈呢?”舒克问。 
        “不是被你接走了吗?”咪丽说。 
        “我接走了?”舒克看看贝塔。 
        贝塔两手一摊,耸耸肩膀。 
        “你没派两只老鼠来接你妈妈?”咪丽傻眼了。 
        舒克感到不妙了。“什么时候的事?”贝塔问。 
        “两天前。”咪丽把海盗的样子形容了一番。 
        “是海盗!”舒克一跺脚。 
        “什么海盗?”咪丽头一次听说。 
        贝塔把海盗的来历以及他们同海盗的仇怨告诉咪丽。 
        “中计了!中汁了!”咪丽气得直咬牙。 
        “我去救妈妈!”舒克转身准备出去。 
        “这肯定是海盗的圈套。”贝塔提醒说。 
        “我跟你去!”咪丽决定治治这些老鼠。 
        “他们是想用妈妈当诱饵抓我,咱们得提防点儿。”舒克挺清醒。 
        “这回不能从地沟走了,咱们开坦克去。”贝塔说。 
        舒克同意了。他将直升机藏在咪丽家的外边。 
        舒克、贝塔和头版钻进坦克。咪丽在一旁护卫,向鼠王的王官进发。 
        贝塔好长时间没开坦克了,他坐在驾驶员的座位上,聚精会神地操纵坦克行驶。 
        舒克坐在炮手的座位上。头版躺在贝塔的软床上。 
        “你准备好炮弹。”贝塔对舒克说。 
        舒克从弹箱里翻出几颗重型炮弹。他真后悔在王宫里没把海盗干掉。 
        夜,漆黑一片,街上没有行人。 
        坦克接近那座教堂了。 
        舒克从坦克里探出头来,对咪丽说: 
        “注意,前边就是!” 
        咪丽用最快的速度跑到教堂旁的阴影下躲起来。 
        舒克从坦克里钻出来,跟咪丽隐蔽在一起。 
        贝塔开始采用调虎离山计。他驾驶着坦克明目张胆地朝鼠王王官闯去。 
        头版坐在炮手的位置上,随时准备射击。 
        王宫被惊动了。 
        鼠王的卫队全副武装迎战。他们还不知道坦克的厉害,竟然手持刀棍朝坦克冲来。 
        “开炮!”贝塔发令。 
        头版瞄准冲在最前边的鼠兵开炮。 
        两名鼠兵应声倒在地上,疼得大叫。 
        剩下的卫兵毫无惧色,继续冲击。 
        “开炮!”贝塔喊。 
        头版不会装炮弹,刚才那发是舒克装的。 
        眼看鼠兵们接近坦克,如果他们爬上坦克,情况就不妙了。 
        装炮弹已经来不及了,贝塔决定转炮塔,用大炮的炮管抡鼠兵。 
        “坐好,抓紧!”贝塔对头版说, 
        贝塔从潜望镜里看见鼠兵靠近坦克,他猛地按下炮塔旋转按钮。炮塔飞速旋起来,炮管把几名鼠兵抡出去数米远。 
        其余的鼠兵吓坏了,朝王宫撤退。 
        这时,趁鼠兵们同贝塔交战之际,舒克和咪丽翻墙进了王宫。 
        王宫里很平静。舒克看见一名鼠兵在站岗。 
        舒克拿着电棍靠近他。 
        鼠兵发现了舒克,刚要喊,舒克用电棍碰了他一下,鼠兵不敢动了。 
        “舒克的妈妈关在哪儿?”舒克问。 
        鼠兵指指一个洞口。 
        舒克蹑手蹑脚朝那洞口走去,咪丽进不去,躲在一旁等他。 
        舒克钻进洞,一张大网从天而降,扣住了他。 
        ‘哈哈,你终于落到我的手里了!”海盗出现在舒克面前。 
        舒克明白中计了,他拼命挣扎,越动网子越紧。 
        咪丽在外边干着急,她进不去。所有的老鼠都躲进洞里,咪丽一只也抓不着。 
        “你和猫也勾结起来了。”海盗冷笑着对舒克说。   第64集 
        贝塔向皮皮鲁求援; 
        电子捕鼠器显示威力; 
        海盗又设圈套   
        咪丽救不出舒克,只好去向贝塔报信。 
        “啊?!舒克被抓啦?”贝塔差点儿从坦克炮塔上弹射出去。 
        “咱们中计了。海盗太狡猾。”咪丽真想一口把海盗吞了。 
        “坦克能冲进去吗?”贝塔问。 
        “进不去。直升机差不多。”咪丽望望教堂的高墙。 
        贝塔清楚,耽误一分钟,舒克就多一分危险。 
        “去向皮皮鲁求援!”咪丽提醒贝塔。 
        贝塔眼睛一亮:“对,皮皮鲁准有办法。” 
        “我在这儿守着,他们出来一只我抓一只,你们快去找皮皮鲁。”咪丽说。 
        贝塔和头版驾驶坦克朝皮皮鲁家开去。过去贝塔都是从空中去皮皮鲁家,他根本不认路。 
        坦克在城里转来转去,直到天快亮时才找到皮皮鲁家。 
        贝塔把坦克隐蔽在草丛里,对头版说: 
        “你守着坦克,我上去找皮皮鲁。” 
        头版点点头。 
        贝塔顺着下水管爬上皮皮鲁家的阳台,阳台门关着,纱窗上还留着那个洞口。贝塔从洞口钻进屋里。 
        皮皮鲁睡得正香,被贝塔叫醒了。 
        “什么,舒克被海盗抓住了?!”皮皮鲁一个鲤鱼打挺坐起来。 
        “你快想办法救他吧!”贝塔急切地说。 
        “这些坏老鼠,真该……”皮皮鲁想起贝塔也是老鼠,把“真该毒死它们”后半截话咽回去了。 
        “有办法吗?”贝塔等不及了。 
        “我想想。”皮皮鲁知道自己钻不进老鼠洞去救舒克。 
        贝塔注视着桌上闹钟的指针。 
        “我用电子捕鼠器治它们!”皮皮鲁说。 
        光“电子捕鼠器”这名字就让贝塔打了个哆嗦。 
        “你有吗?”贝塔问。 
        “学校有,一会儿我去拿。”皮皮鲁三下两下穿上衣服。 
        “我们呢?”贝塔问。 
        “我带你们一起去。”皮皮鲁背上书包,“把坦克装进我的书包。” 
        舒克现在被海盗结结实实地捆起来,同妈妈关在一起。 
        “我要亲手处决他!”鼠王要出这口气。 
        “不忙,等把贝塔抓住,一起处决才爽。”海盗慢悠悠地说。 
        “什么时候能抓住贝塔?”鼠王问。 
        “快了,贝塔准来救舒克,我已布置好了陷阱。”海盗胸有成竹。 
        舒克为贝塔捏了一把汗。 
        就在这时,老鼠们突然感觉到有一种无形的力量在呼唤他们。几只老鼠兵身不由己地出去了。 
        “怎么回事?”海盗感到奇怪。 
        “我出去看看。”鼠王也身不由己地想出去。 
        “等等,怎么都不回来了?”海盗觉出不妙了。 
        舒克认定这和贝塔有关。 
        “妈妈,准是贝塔来救咱们了。”舒克小声说。 
        “不好了!不好了!”一只鼠兵跑进来。 
        “怎么回事?”海盗急忙问。 
        “有一台仪器放在院里,咱们的鼠兵被那台仪器抓住好多只,可还是有不少鼠兵往那儿跑!”那鼠兵报告。 
        “电子捕鼠器!”海盗恍然大悟。 
        这时,一只被五花大绑的鼠兵跑进来。 
        “他放我回来,说要是不把舒克和他妈妈马上放了,就把咱们都抓起来。”五花大绑的鼠兵哭丧着脸说。 
        “他?他是谁?”海盗问。 
        “是个人,叫皮皮鲁。”鼠兵说。 
        “好啊,舒克和人也勾结起来了,我说你哪儿来那么多飞机!”海盗恶狠狠地看着舒克说。 
        舒克不理他。 
        “你去告诉他,晚上放舒克。”海盗对鼠兵说。 
        “不能放!”鼠王急了。 
        “这是计。”海盗小声对鼠王说着什么,鼠王笑   第65集 
        老鼠家族组成敢死队; 
        皮皮鲁被围困; 
        贝塔呼救   
        鼠兵将海盗的话转告皮皮鲁。 
        “好,我晚上来。如果到时候不放舒克,我就不客气了。”皮皮鲁看时间不早了,该去上学了,就答应了。 
        “我在这儿监视他们。”咪丽说。 
        “行,贝塔和头版坐在坦克里跟我上学去吧,这里危险。”皮皮鲁对贝塔说。 
        贝塔同意了。 
        皮皮鲁将电子捕鼠器收起来,把俘虏们关进事先准备好的笼子,交给咪丽看押。 
        贝塔和头版钻进坦克。皮皮鲁把坦克装进书包,去上学。 
        “咱们睡会儿。”贝塔在坦克里对头版说。 
        海盗得知皮皮鲁走了,立即召来两名部下,吩咐道:“门口有猫,你们从暗道出去跟着皮皮鲁,随时向我报告他的动向,” 
        两名鼠兵去跟踪皮皮鲁。 
        “大王,咱们得治治这个皮皮鲁。”海盗对鼠王说。 
        “嗯。你快想办法。”鼠王说。 
        “发动全城的老鼠去咬他,攻击他!”海盗说,“全城共有多少老鼠?” 
        掌管鼠数的大臣递上花名册。 
        “5000只就足够了。”海盗说。 
        “传我的命令,马上挑选5000只身强力壮的老鼠,准备同皮皮鲁较量。”鼠王说。 
        鼠王的圣旨立即被传达到全城。5000只老鼠精选出来了,他们磨尖了牙齿,随时待命准备同那个与老鼠为敌的皮皮鲁拼命。 
        “报告,皮皮鲁现在学校的大操场上。”侦察鼠向海盗报告。 
        “几个人?”海盗问。 
        “就他自己。” 
        “干什么呢?” 
        “好像在背书。” 
        “传令,敢死队出击!目标,皮皮鲁。”海盗全身披挂,亲自督战。他断定一个男孩子绝对不足5000只老鼠的对手。 
        下午放学后,皮皮鲁一个人在学校的大操场上背书,应付明天的考试。同时等天黑了去鼠王王宫接舒克。 
        贝塔和头版在坦克里睡觉。坦克在皮皮鲁的书包里。 
        5000只老鼠组成的敢死队从学校的下水道里冲出来了,他们朝皮皮鲁围过来。 
        皮皮鲁正在埋头背书,忽然感到远处的地上有什么东西在蠕动,他抬头一看,愣了。黑压压的一片老鼠正朝他冲过来。 
        皮皮鲁往后退了几步,一回头,身后全是老鼠。再往左右看,皮皮鲁明白是怎么回事了。密密麻麻的老鼠把皮皮鲁围在了中间。 
        大操场上空无一人。没有可以用来当武器的东西。皮皮鲁完全清楚这么多的老鼠能把他怎么样。 
        贝塔被同胞的脚步声惊动了。他从书包里探出头,傻了。 
        “皮皮鲁,快把坦克从书包里拿出来!”贝塔喊。 
        皮皮鲁明白贝塔的坦克起不了什么作用,老鼠太多了,但总比不抵抗强。皮皮鲁从书包里掏出坦克,放到地上。 
        “快装炮弹!”贝塔对头版说。 
        头版的手直哆嗦,炮弹装上了。 
        “打!”贝塔一挥手。 
        坦克开炮了。两名老鼠敢死队员倒下了,敢死队员踩着同胞的身体继续冲锋。 
        皮皮鲁手里只有书包,他用书包抡已经接近他的老鼠。 
        贝塔操纵坦克撞老鼠,坦克在老鼠群里横冲直撞。可老鼠太多了,已经有许多老鼠爬上了坦克,有的在使劲掀盖子。 
        皮皮鲁的身上已经爬上了几只老鼠。皮皮鲁把他们打下来,又有老鼠爬上去…… 
        情况万分紧急。 
        贝塔没想到海盗这么狠毒,他认定自己和舒克加上皮皮鲁都会葬送在海盗手里。 
        危急之中,贝塔看见了电台。他眼睛一亮。 
        贝塔戴上耳机,拿起话筒。 
        “罗丘!臭球!贝塔呼叫!贝塔呼叫!请回答!”贝塔太声喊着。 
        没有回声。 
        “臭球!臭球!我是贝塔!我是贝塔!”贝塔带着哭腔喊。   第66集 
        臭球率领战斗机群大败敢死队; 
        摄像记者激动得忘记放录像带; 
        舒克、贝塔和臭球相逢   
        “我是臭球!我是臭球!请讲!”贝塔的耳机里传出了臭球的声音。 
        “臭球,我是贝塔!我们遇到了险情,请立即派战斗机来支援!请立即派战斗机来支援!”贝塔喊。 
        “方位?” 
        贝塔将方位告诉臭球。 
        臭球拉响了机场的警报器。 
        民用机场转眼变成了军用机场。歼击机、强击机和轰炸机腾空而起,朝城市扑去。 
        此时皮皮鲁的身上已经爬上了几十只老鼠,他们开始咬皮皮鲁的衣服,咬皮皮鲁的手。 
        皮皮鲁一边抡书包一边把身上的老鼠往下赶,他大声呼喊。 
        学校的教导主任闻声赶来,这场面把他吓坏了。他想起皮皮鲁深夜去报社说老鼠要毒人的事,他信了。 
        教导主任不敢上前,他跑回去打电话。 
        贝塔的坦克已经被老鼠敢死队围死了,炮塔也转不动了。老鼠们正使劲翻坦克。 
        海盗神气活现地在一旁指挥,他终于知道了自己的力量,把人都治住了!他想好了,只要一抓住贝塔,就立即篡夺王位。 
        皮皮鲁已经精疲力竭,老鼠敢死队把对人类的仇恨都集中到皮皮鲁身上。 
        正当海盗认定稳操胜券的时候,天空中出现了战斗机群。 
        “贝塔,贝塔,我们到了,坚持住!”臭球在歼击机里喊话。 
        “快打!使劲儿打!”贝塔兴奋了。 
        臭球率领歼击机群开始轮番俯冲扫射,数十只老鼠敢死队员倒在地上。 
        强击机编队俯冲。 
        轰炸机时而俯冲投弹,时而水平投弹。 
        操场上爆炸声此起彼伏。 
        皮皮鲁趁机逃出操场,躲在树后观看老鼠之间的大战。 
        贝塔来劲儿了,他驾驶坦克横冲直撞。 
        “瞧,海盗!”头版从潜望镜里看见丁海盗。 
        “撞他!”贝塔驾驶坦克朝海盗轧过去。 
        敢死队员们被战斗机打得抱头鼠蹿。 
        臭球不断地下达攻击命令: 
        “一中队,截住往南逃的那股!” 
        “三中队,狠狠打西边那一小撮儿!” 
        “轰炸机5号,往东边投颗催泪弹!” 
        “强击机!压住北边那伙!” 
        学校里没回家的学生和老师都跑到操场上目睹这场老鼠之间的大战,他们看呆了。要不是亲眼所见,谁会相信! 
        不知是谁打电话把电视台的摄像记者叫来了。记者扛着摄像机猛拍,要不是因为他激动忘了放录像带,全世界都能欣赏这场大战。 
        贝塔的坦克死盯着海盗追。海盗拐着弯跑,他没想到形势变化这么快,海盗看见操场边有一个下水道口,他钻了进去。 
        贝塔没办法了。 
        “臭球,舒克被关在教堂下边的鼠王王宫里,你快派部下去救他!”贝塔通过电台喊叫。贝塔清楚,如果海盗先回王宫,舒克准倒霉。 
        臭球留下一些飞机继续攻击老鼠敢死队,他带领另一批飞机直飞教堂。 
        操场上的战斗已接近尾声,皮皮鲁跑进操场拿起贝塔的坦克,朝教堂跑去。 
        臭球的空降兵在教堂着陆了,他们在咪丽的指引下,杀进鼠王的王宫。 
        鼠王正在等待凯旋的海盗,忽闻宫外杀声骤起。 
        “不好了,人王,数百只鼠兵从天而降,杀进王宫来了!”卫队长慌忙跑来向鼠王禀报。 
        鼠王面如土色。 
        臭球的歼击机在教堂的围墙上着陆。 
        臭球率领部下生擒了鼠王。 
        “快说,舒克关在哪儿?”臭球问。 
        鼠王筛糠着交待。 
        舒克和妈妈得救了。 
        “谢谢你,臭球!”舒克和臭球拥抱。 
        贝塔和头版赶到了,大家十分激动。 
        “怎么处置他?”臭球指指鼠王。 
        “罢免他的王位,放他当普通老鼠。”舒克说。 
        鼠王感激不尽。 
        “抓住海盗了吗?”舒克穿上飞行服问。   第67集 
        舒克不同意搜捕海盗; 
        舒克和贝塔决定办报; 
        头版担任印刷厂厂长   
        “又让他跑了。”贝塔惋惜地说。 
        听说海盗又溜了,舒克直跺脚。 
        “咱们在全城搜捕他!”臭球提议。 
        舒克想了想,摇摇头,说:“算了,闹得大家惊慌不安。再说,抓住他又能把他怎么样呢?” 
        大家明白,舒克不会处死海盗。 
        “我觉得老鼠家族的品质应该改变改变。”舒克说,“单靠抓住一个海盗起不了作用。” 
        “这可不容易。”臭球耸耸肩膀。 
        舒克看见了头版,他眼睛一亮。 
        舒克把《晚报》发消息不让人类投放鼠药的故事讲给臭球听。 
        “报纸真厉害。”臭球说。 
        “咱们办张报纸,改变老鼠家族的品质。”舒克说。 
        “办报纸!” 
        “太好了!” 
        大家一致同意。 
        “办报不大容易吧?”贝塔问。 
        “其实不难,把三方的关系摆对就行。”头版说。 
        “哪三方?”舒克问。 
        “读者、编者和作者。”头版说,“读者是爷爷,编者是爸爸,作者是孙子。这样的关系,准能办好报。” 
        ‘如果倒过来呢?作者是爷爷,编者是爸爸,读者是孙子。”舒克问头版。 
        “世界上凡是没人看的报刊,准因为像你说的这样摆三方的关系。”头版一口咬定。 
        舒克连连点头。 
        “我们机场订一百份!”臭球当了老鼠报第一个订户。 
        “将来还得靠你的运输机给我们往各地送报。”贝塔拍拍臭球的肩膀。 
        “没问题,包在我身上。”臭球拍胸脯。 
        “说干就干,咱们先给报纸起个名字。”贝塔说。 
        “叫《老鼠日报》怎么样?”舒克说。 
        “一开始就出日报恐怕困难,”头版经验丰富,“其实叫《老鼠报》就行。” 
        “对,就叫《老鼠报》。”舒克同意了。 
        “舒克当总编辑。”贝塔提议。 
        没人反对。 
        “头版当印刷厂厂长。贝塔当发行部经理。”舒克任命。 
        “编辑部设在哪儿?”头版问。 
        “就设在直升机上。”舒克说。 
        “保准是全世界第一家空中报社。”臭球羡慕。 
        “咱们去告诉皮皮鲁。”舒克同大家来到外边。 
        皮皮鲁正在检阅他的飞机。 
        “办报?我也有希望发表作品啦!”皮皮鲁听说后乐得拍手。 
        “头版,你现在就去筹备印刷厂,有什么困难?”总编辑舒克问。 
        “我回家去动员我家里人一起干,他们都在行。”头版说。 
        “我送你去。”贝塔钻进坦克。 
        “我们先返航了,有事随时联系。”臭球说。 
        舒克拥抱臭球。 
        机群升空。 
        皮皮鲁也告辞。 
        “我得物色几名编辑和记者。”舒克躺在直升机里的软椅上想。 
        有人敲直升机的玻璃窗。 
        舒克坐起来往外看,是两只老鼠。 
        舒克打开舱门,问:“什么事?” 
        “我叫松果,她叫荷叶。我们的父母被海盗强迫去当敢死队员,都战死了,我们没有亲人了。”叫松果的老鼠说。 
        舒克叹了口气。 
        “你是舒克吧?”荷叶问。 
        “嗯。” 
        “你的本事大,你收留我们吧!”荷叶请求道。 
        舒克拿不定注意。   第68集 
        舒克招聘记者和编辑; 
        总编辑亲自撰写发刊词; 
        荷叶的文章通过了   
        “你们识字吗?”舒克问松果和荷叶。 
        他们点头。 
        “我们办了张《老鼠报》,你们给我当记者和编辑吧。”舒克说。 
        “记者?编辑?我们行吗?”松果怀疑自己的能力。 
        “听说记者和编辑都得大学毕业。”荷叶说。 
        “那是人类同自己过不去,大学是给那些没本事又想比别人活得好的人准备的。其实,有学历的笨蛋更麻烦。咱们不管那么多。你们试试。”舒克说完请松果和荷叶上飞机。 
        荷叶和松果登上直升机。 
        “这就是报社。来,咱们把机舱改装一下,像个编辑部的样子。”舒克找来工具箱。 
        他们一起动手,把皮椅移动位置,再安上办公的桌子,机舱变了样儿,像办公室。 
        “我来考考你们,看看你们适合干什么。”舒克坐在总编辑的座位上。 
        荷叶和松果坐在总编辑对面。 
        “你们谁知道记者是干什么的?”总编辑发问了。 
        “记者就是把事情记着。”松果说。 
        “记者就是……就是到处跑。”荷叶说。 
        舒克也不知道记者的确切含义。他又问:“编辑呢?” 
        “编辑就是把事情编起来。”荷叶先回答。 
        “大概是这样的。”松果对编辑一无所知。 
        “松果当记者。荷叶当编辑。”总编辑任命部下。 
        舒克找出几杆笔。分给记者和编辑。 
        这时,电台里传出贝塔的呼叫声。 
        舒克走进驾驶舱,戴上耳机。 
        “我是舒克。请讲。” 
        “我是贝塔。我们已到达头版家。头版的全家愿意筹办印刷厂。他们还向你道歉。现在我们正在工作。” 
        “很好,我也物色到了记者和编辑。什么时间可以印报?” 
        “再有两小时就可以,头版家的人很熟悉这项业务。” 
        “好,两小时后我给你送稿子去。”总编辑挂上耳机。 
        舒克叫来荷叶和松果。 
        “现在你立即出去采访。”总编辑对松果说。 
        “采访什么?”松果还摸不着头绪。 
        “到老鼠居民的家里去,问问他们对鼠王下台的看法什么的。”总编辑吩咐,“一小时后回来。” 
        松果离开飞机。 
        “我干什么?”荷叶编辑请求。 
        “你写一篇文章,说说鼠王下台的经过。”舒克边想边说,“还应该有一篇发刊词。” 
        “发刊词我可不会写,听说这样的文章都是总编亲自写。”荷叶说。 
        “发刊词我写。通讯你写。”总编辑开始埋头写作。 
        对于舒克来讲,并未觉得写文章比开飞机难。 
        很快,他将发刊词写好了,全文如下: 
        发刊词 
        我们想办《老鼠报》,我们就办了。想看的人就看,不愿看的人就别看。 
        后来贝塔说这是全世界最精彩的发刊词,没一句废话。大家也认定舒克是大手笔,有写作才能。 
        编辑的文章也写好了,交给总编辑审查。 
        舒克边阅边修改。荷叶站在一旁直冒汗。她很喜欢编辑这工作,她害怕失去。 
        “以后努力提高业务水平。”舒克说。 
        “通过了?”荷叶问。 
        “第一次写成这样就不错了。”总编辑点点头。 
        这时,记者松果回来了。 
        “怎么样?”舒克问。 
        松果擦擦头上的汗,说:“我还得整理一下。” 
        “快点儿,半小时后咱们起飞去印刷厂付印”总编辑检查飞机去了。   第69集 
        总编辑舒克将记者松果的文章改为《挑毛病专栏》; 
        印刷厂遭劫   
        松果趴在办公桌上写稿。 
        “我写的文章已经通过了。”荷叶沾沾自喜地告诉同事。 
        松果连头也顾不上抬。 
        舒克检查完飞机,走到松果身边。 
        “交稿。”记者将笔往桌上一扔,拿起稿件递给总编辑。 
        舒克边看边皱眉头。这篇采访写得驴唇不对马嘴,错别字连篇。 
        修改的时间没有了。舒克灵机一动,干脆来个“请读者挑毛病”专栏吧,让读者挑出这篇文章的毛病,不是还可以提高读者的阅读水平吗? 
        尽管松果有点儿伤自尊心,可也无可奈何。 
        总编辑将稿件收集好,驾驶直升机起飞。荷叶和松果头一次坐飞机,趴在窗户上往外瞧个没够。 
        直升机飞临《晚报》社上空,舒克看见草丛中贝塔为飞机着陆设置的标志,他操纵直升机缓缓降落。 
        贝塔等候在草丛中,坦克停在一边。 
        “印刷厂建好了。”贝塔说。 
        “真快!在哪儿?”总编辑对部下的工作效率感到满意。 
        “在《晚报》的印刷车间里,东西都是现成的。”贝塔像个经理。 
        “危险吗?” 
        “咱们的印刷厂藏在一堆铅字后边,一般发现不了。” 
        “现在就去开印。”总编辑拿上稿件,然后吩咐荷叶跟着去,留下松果看守飞机和坦克。 
        他们来到印刷车间,工人们还在干活。 
        “跟我来。”贝塔领着舒克和荷叶绕过人群,顺着墙根儿来到《老鼠报》的印刷厂。 
        头版欢迎总编辑的到来。 
        舒克见了头版的家人,并无成见。这使头版全家十分感动。 
        印刷设备隐蔽在暗处。舒克把稿件递给头版。 
        “还没划版?”头版是内行,“咱们不用打校样,直接排版。” 
        “你就负责排吧。”总编辑把排版的任务交给了印刷厂长。 
        头版忙起来。 
        这时,一位印刷工人到铅字堆旁取铅字,他无意中发现了一只老鼠。他悄悄地探头一看,啊,一群老鼠。 
        他就是由于刊登不让居民投放鼠药而被撤职的原《晚报》总编辑,他恨死老鼠了。 
        “抓老鼠呀!”原总编辑嚷嚷起来。 
        “不好,快分头跑!去草丛集合!”舒克大喊。 
        “印刷厂怎么办?”头版问。 
        “不要了!”舒克当机立断。 
        大家四散逃命。 
        车间里的工人们拿着扫帚拖把朝这边围过来。 
        “老鼠在哪儿?” 
        “藏哪儿了?” 
        “这是什么?” 
        工人们围着看《老鼠报》的印刷厂。 
        老鼠们溜了。 
        “它们还要办报?”原《晚报》总编辑有点相信老鼠会往人的碗里放毒药了。   第70集 
        印刷厂搬进飞机和坦克里; 
        总编辑签字付印创刊号   
        《老鼠报》社全体工作人员平安到达草丛里。 
        “贝塔,你认出来了吗?刚才那个工人就是《晚报》原来的总编。”舒克气喘吁吁地对贝塔说。 
        “没错,是他。”贝塔一边擦汗一边说。 
        “咱们的印刷厂完了。”头版伤心。 
        “没关系,重新建。”舒克充满信心。 
        “出报得推迟了。”贝塔有些惋惜。 
        “稿子在这儿。”荷叶把她抢出的稿子递给总编辑。 
        舒克这才想起稿子不在自己手里了。看到荷叶在危急关头还能注意到稿子,舒克对她刮目相看。他考虑在适当的时候提拔荷叶当编辑室主任。 
        “咱们不能在车间里建印刷厂了,太危险。”贝塔说。 
        “嗯。”舒克点点头,  “把印刷厂建在直升机和坦克里。” 
        “太棒了,在直升机里排字,在坦克里印刷。”头版兴奋。 
        “说于就干!”贝塔卷起袖子。 
        “我去侦察一下,看看他们的动静。”头版离开草丛,朝印刷车间跑去。 
        “当心点儿!”舒克叮嘱。 
        一会儿,头版回来了。 
        “他们又开始干活儿了,咱们的印刷厂被破坏了。”头版说。 
        “重建一座。”舒克说,“头版,你指挥。” 
        头版来劲儿了,他开始分工。 
        大家先将直升机和坦克收拾好。直升机舱里除了编辑部外,又隔出一间排字房。这样倒方便了,总编辑审完稿后,直接就进排字房排字。坦克舱里也收拾出安放印刷机的位置。 
        然后,头版带领几名老鼠去运印刷机,头版的哥哥带几名老鼠去运铅字。 
        经过几个小时的奋斗,印刷厂终于在直升机和坦克里竣工了。 
        “咱们离开这个地方,找个安全的地点印刷。”舒克招呼大家上飞机和坦克。 
        编辑和排字工上了直升机,贝塔和印刷工们进了坦克。 
        直升机吊着坦克起飞了。舒克感到驾驶杆很重,直升机还是头一次负荷这么大。 
        舒克寻找合适的着陆点。 
        “那座高楼上有个大平台。”贝塔通过无线电告诉舒克。 
        舒克看见了。他操纵直升机降落在高楼顶上。 
        贝塔打开坦克舱盖,探出头,看看四周没有危险,然后跳出坦克,招呼大家出来。 
        《老鼠报》社的工作人员们纷纷从直升机和坦克里钻出来。 
        “咱们抓紧出报,贝塔你负责警卫,其他人各就各位。”总编辑舒克说。 
        贝塔将稿件交给头版排字,头版的哥哥在排字间忙碌着。 
        版拼好了,头版印出小样让总编辑检查。 
        “如果没问题,请签字付印。”头版对舒克说。 
        舒克一本正经地检查了一遍,郑重地在校样上签了字。 
        版被运进坦克舱,印刷机转起来。 
        “印多少?”头版的爸爸从坦克里伸出头来问总编辑。 
        “5000份。”舒克说。 
        “坦克里装不下,怎么办?”头版的爸爸为难了。   第71集 
        《老鼠报》在空中印刷; 
        老鼠世界第一份报纸出版   
        贝塔想了个办法,他打开坦克底盘上的紧急出口。 
        “让报纸从紧急出口出来。”贝塔对印刷工说。 
        印刷机又开始转动了,报纸飞快地从坦克下边的出口飞出。不一会儿,坦克下边的报纸就把坦克顶了起来。 
        “停机!停机!”贝塔忙叫。 
        印刷机关闭了。 
        “怎么了?”舒克问贝塔。 
        “这么下去,我的坦克被越顶越高,会掉下来摔坏的。”贝塔爱惜他的坦克。 
        “又不能印了?”舒克惋惜地耸耸肩。 
        “用你的直升机把坦克吊起来再印。”贝塔想了个主意。 
        “不错。”舒克启动直升机。 
        直升机悬停在坦克上空,用钩于钩住坦克,将坦克吊到空中。 
        “开印。”贝塔坐在坦克里下令。 
        印刷机运转了。 
        由于坦克距离楼顶太远,印出的报纸被风刮走好多。 
        “低点儿!降低高度!”留在楼顶上的头版大喊。 
        直升机凋整高度,直至印出的报纸正好落在楼顶上。 
        舒克坐在驾驶舱里感到惬意。他望着地面上不断增多的报纸,他没想到自己还能当上总编辑。 
        耳机里传出贝塔的喊声,通知舒克报纸已全部印完,可以着陆了。 
        舒克和直升机着陆后,大家争先恐后地抢报纸看。 
        “真不错!”贝塔边看边夸。 
        “这是咱们老鼠家族的头一张报。”荷叶兴奋地说。 
        “咱们用这张报改变改变老鼠家族的品质。”舒克边说边将报头上“总编:舒克”几个字连看了100遍。 
        “下边就是发行的问题了。”头版提醒贝塔。 
        “没问题,我包了。”贝塔胸有成竹。 
        贝塔钻进坦克,要通了臭球的电台。 
        “我是贝塔!报纸已出版,快来拿!”贝塔说。 
        “什么报纸?”臭球已经忘光了。 
        “《老鼠报》呀!你不是汀了100份吗?”贝塔提醒臭球。 
        “噢,对!对!我马上亲自去取。方位?”臭球还来个亲自! 
        贝塔告诉他。 
        不一会儿,天空中出现了大型运输机。 
        “请导航。”臭球提要求。 
        舒克拿起话筒,指引臭球着陆。 
        “这家伙现在还真行,什么飞机都能开。”贝塔边看飞机着陆边说。 
        臭球从飞机上走下来。 
        “报纸出得真快。”臭球拿起一张报说。 
        “给你100份。”贝塔将捆好的报纸交给臭球。 
        “别急,”臭球顾不上接,还在埋头看报,“呆板了些,应该登些文艺作品,还有笑话什么的。” 
        舒克设想到臭球先提意见,虽然他心里不舒服,可还是觉得臭球说得有道理。 
        “第一期仓促了点儿,以后就好了。”舒克说。 
        “还没发现海盗?”臭球问。 
        “没有。”贝塔摇头。 
        “好,我走了。第二期出版后我再来拿。”臭球同机组人员一起将报纸运上飞机。   第72集 
        贝塔开坦克送报; 
        记者松果向总编辑报告《老鼠报》发行情况   
        “下边看你的啦!”舒克对贝塔说。 
        “天一黑我就出发。”贝塔信心十足。他吩咐头版:“把坦克里装满报纸。” 
        头版和印刷工们将一部分报纸装进坦克。 
        太阳慢慢落山了。 
        “咱们吃饭吧。”舒克从飞机里取出一些食物,分给大家。 
        大家都觉得这种生活很有意思。 
        吃完饭,天黑了。 
        “你得把我送到地面上去。”贝塔对舒克说。 
        舒克驾驶直升机将贝塔的坦克吊到地面上,贝塔从坦克里探出头来.说: 
        “你回去等好消息吧,我回来后用电台通知你。” 
        “当心,常通话。”舒克驾驶直升机返回楼顶。 
        贝塔把坦克舱盖儿关严,坐在驾驶座位上,操纵坦克向前驶去。 
        这是一片草坪,绿油油的草叶摇来晃去,贝塔透过草叶看见了一只小老鼠。 
        小老鼠听见坦克的声音,警惕地望着这个怪物。 
        贝塔把坦克停在小老鼠附近,他探出头来:“别怕,我也是老鼠。” 
        小老鼠感到惊奇。 
        “你家在哪儿?”贝塔问。 
        “干什么?”小老鼠起疑。 
        “别紧张。听说过舒克贝塔吗?”贝塔问。 
        “就是打败鼠王的那两个勇士?” 
        “对,就是那两个勇士!”贝塔很满意“勇士”这个头衔。 
        “当然知道。” 
        “我就是贝塔。” 
        小老鼠信了。 
        “我们办了一份报纸,送给你家几份。”贝塔边说边递给小老鼠一份报纸。 
        “用坦克送报纸?”小老鼠觉得有点儿那个。 
        “我没有别的交通工具。’贝塔也意识到全副武装送报纸挺吓人。 
        小老鼠翻看报纸。 
        “你家几只鼠?”贝塔问。 
        “17只。” 
        “给你五份,轮着看吧。”贝塔又递给小老鼠四份。 
        “不要东西换?自给?”小老鼠不信。 
        “当然是自给。这叫赠送。”贝塔准备走了。 
        “你一家一家送报?”小老鼠问。 
        “对。” 
        “那得多少时间!一个星期也送不完。” 
        贝塔一想也真够麻烦的。 
        “我让我们全家帮你送吧?”小老鼠热情地说。 
        “他们同意吗?”贝塔喜出望外。 
        “咱们去同他们谈谈。” 
        贝塔的坦克尾随着小老鼠来到他家门口。贝塔没想到非常顺利,小老鼠全家一致同意帮助贝塔发行报纸。 
        贝塔将一坦克报纸全部交给了他们。 
        贝塔驾驶坦克回到高楼下边。他拿起话筒呼叫舒克。 
        舒克开直升机将贝塔的坦克吊上楼顶。 
        “这么快?!”舒克不信。 
        “你能这么快出报,我就不能这么快发行?”贝塔得意。 
        第二天下午,记者松果向总编辑汇报,有老鼠在市场上高价出售《老鼠报》。 
        贝塔傻眼了,他断定是小老鼠全家干的。他感到给老鼠送报不光应该用坦克,还应该架着机关枪。   第73集 
        《老鼠报》社空中转移; 
        贝塔报告紧急情况   
        总编辑舒克问发行部经理贝塔:“怎么回事?” 
        “我委托了几位老鼠帮我发行,大概是他们把报纸拿去卖了。”贝塔挠后脑勺。 
        “走,找他们算账去!”舒克火了。 
        “我自己去就行了,你们编报吧。”贝塔说。 
        “我用直升机送你下去。”舒克说。 
        “咱们换个地方怎么样?呆在这么高的楼上,实在不方便。”贝塔提议。 
        “行,换个地方。”舒克对大家说:“各就各位,准备起飞。” 
        大家分别登上直升机和坦克,直升机吊着坦克离开了楼顶,升到空中。 
        “那边有一片草丛。”荷叶指给舒克看。 
        舒克驾驶直升机飞临那片草丛上空。 
        “观察一下。”舒克对荷叶和松果说。 
        荷叶和松果从不同的方向往下看,没发现危险因素。 
        直升机在草丛中着陆了。 
        大家纷纷从飞机和坦克里钻出来,呼吸新鲜空气。 
        “侦察一下。”舒克让大家分头看看地形。 
        这片草丛两面临水,两面通陆,十分安全。 
        “咱们就在这里安营扎寨。”舒克对这里挺满意,“贝塔,体去找那窝老鼠吧!” 
        贝塔开着坦克走了。 
        “咱们开始编第二期《老鼠报》。”总编辑对记者和编辑说。 
        “什么内容?”记者等候总编辑指示。 
        “你把卖报的事写篇通讯,狠狠挖苦一下这些利欲熏心的同胞。”总编辑说。 
        “我呢?写什么?”荷叶问。 
        “你写篇小说。”舒克想起臭球的建议。 
        “可我从来没写过小说呀!”荷叶为难地说。 
        “写一次就会了。”总编辑说。 
        “什么叫小说?”编辑向头儿请教。 
        “小说……小说就是……小说,就是不是诗,不是剧本也不是散文的一种东西。”总编辑给小说下定义了。 
        “……”编辑茫然。 
        “你就编一个故事,编得让大家爱看,就行了。”舒克开导部下。 
        舒克坐在总编辑的座位上,看着记者和编辑伏案书写,满意地点点头。 
        “头版,准备排字。”舒克透过机窗叫头版。 
        头版从草地上站起来,走进机舱。 
        舒克审查了荷叶写的小说,认为完全达到发表水平。荷叶激动得脸都红了。 
        “嗯.不错,你继续努力,完全可以得诺贝尔文学奖。”总编辑把稿子交给头版。 
        机舱外传来轰隆隆的卢音。 
        “怎么回事?”舒克警惕。 
        “水里在举行舰模比赛。”头版的哥哥来报告。 
        舒克跑到临水的那边一看,水面上行驶着几十艘航模舰艇,有军舰,有油轮,有汽艇,对面岸上不少人在操纵舰模。 
        “贝塔回来了,他说有紧急情况!”松果跑到舒克身边报告。 
        舒克忙跑回到直升机旁边,只见贝塔正气喘吁吁地擦汗。 
        “咱们……这些同胞……真没治了!”贝塔上气不接下气,  “咱们的报纸成了稀有物品,身价越来越高,现在已经成了老鼠世界的钱,开始流通使用了。” 
        “什么?报纸成了货币?”舒克对自己的同胞的品质太不了解了。   第74集 
        老鼠同胞来抢“钱”: 
        舒克的直升机无法起飞; 
        航空母舰情况危急   
        “刚才我去找那家老鼠,他们一见我就扑上来要报纸,瞧,衣服都被撕破了。我好不容易打散他们,跳上坦克就跑,他们还在后边追呢,一会儿准来。”贝塔说。 
        “报纸当钱用?”舒克还半信半疑。 
        “千真万确。那家老鼠卖了那批报纸后,又被别的老鼠以更高的价卖出。当然你知道,咱们老鼠世界没有钱,只有易货贸易。现在大家就拿报纸当钱了。他们要来抢钱呀!”贝塔说。 
        “头版。你爬到树上了望一下。”舒克吩咐。 
        头版刚爬到树上就大喊起来:“他们来了,黑压压一片!” 
        舒克爬上树一看,身上不由打了个哆嗦。他边往下跑边说:“快准备起飞,咱们不是他们的对手。” 
        舒克做梦也没想到给老鼠同胞办报办出这么个结果来。看来品质这东西是难以改变的。 
        大家急忙钻进坦克和飞机。 
        舒克打开开关,发动机不转!舒克再按应急开关,还是不转!糟糕,电池没电了。 
        “贝塔,快把坦克上的电池卸下来给我,我的电池没电了!”舒克拿起话筒。 
        “不是一个型号!”贝塔提醒舒克。 
        舒克傻眼了。 
        “怎么办?”贝塔问,“他们已经接近这边了。” 
        舒克透过草丛看见了水中的一艘军舰模型。舒克跳下飞机,对坦克里的贝塔说: 
        “我去想办法把那艘航空母舰弄过来,你用坦克把直升机拖到水边去,快!” 
        舒克朝水边跑去,头版跟在他身后。 
        这艘航空母舰模型很大,甲板宽阔甲坦,足足可以停放几百架舒克的直升机。它正平稳地行驶在水中。 
        舒克跳进水里,朝航空母舰游去。头版也跳入水中。 
        舒克够不着航空母舰的船舷,头版拼命在水中托起舒克。舒克爬上了甲板,他回身拉上头版。 
        “去驾驶室!”舒克朝驾驶室跑。 
        舒克钻进驾驶室,密密麻麻的仪表令他眼花缭乱。他试着转了转轮舵,军舰不听他摆布。 
        “头版,去把那根天线拿掉。”舒克对头版说,他断定是那根天线在接收岸上发出的遥控信号。 
        头版三下两下把天线拆了。 
        航空母舰听从舒克操纵了!舒克让航空母舰朝贝塔的坦克停着的方向靠拢。 
        贝塔的坦克拖着直升机停在岸边。来抢钱的老鼠们已经拥过来,他们看见了坦克和直升机。 
        “快开上来!”舒克冲贝塔大喊。 
        贝塔的坦克拖着直升机驶上了航空母舰的甲板。舒克操纵航空母舰驶离岸边,几只冲在前边的老鼠掉进水中。 
        航空母舰上一阵欢呼。 
        贝塔从坦克里钻出来,冲进驾驶室,拥抱舒克。 
        对岸上操纵舰模的人们被航空母舰的异常举动惊呆了,他们不明白这艘舰模怎么了。 
        一位选手借来望远镜。 
        “老鼠!还有坦克和直升机!”他刚举起望远镜就大叫起来。 
        “截住它!它要跑!”一个人大喊。 
        十几名选手操纵自己的军舰朝航空母舰追过来。 
        “我去试试舰尾的那门大炮。”贝塔跳出驾驶室。   第75集 
        一场惊心动魄的海战; 
        贝塔驾驶坦克朝水里开   
        十几艘军舰朝航空母舰逼过来。有驱逐舰,有巡洋舰,还有炮舰和鱼雷快艇等。 
        贝塔朝航空母舰的护尾炮跑去。 
        气氛空前紧张。 
        舒克一边操纵航空母舰加快速度,一边对着话筒大喊: 
        “头版,去帮助贝塔装炮弹!松果,了望海面!荷叶,组织其他人员进船舱隐蔽!” 
        贝塔跑到炮塔旁边,他看见大炮下边有一箱炮弹。 
        头版跑过来。 
        “我帮你装炮弹,你瞄准。”头版说。 
        贝塔寻找目标。 
        一艘黑色的驱逐舰跑在最前边,气焰十分嚣张。 
        贝塔决定拿驱逐舰开刀。 
        炮弹塞进了炮膛。 
        贝塔边瞄准边喊:“目标——驱逐舰.放!” 
        轰! 
        驱逐舰起火了。 
        “打中了!打中了!”负责了望海面的松果嚷嚷起来。他出于职业习惯还想记录下这场面,遗憾的是报纸不办了。 
        “贝塔加油——”躲在舰舱里的荷叶喊。 
        贝塔没激动,他认为这成绩对于神炮手太寻常了。 
        一艘炮艇飞速朝航空母舰冲过来,那神态活像敢死队员。 
        贝塔把准星钉在炮艇上。 
        炮艇狡猾极了,开始左右晃动,曲线行驶,贝塔无从下手。 
        炮艇开炮了。 
        炮弹落在航空母舰周围,溅起大水花。 
        岸上一阵欢呼。 
        航空母舰躯体庞大,无法躲避炮艇。舒克急出一头汗。 
        一发炮弹落在航空母舰甲板上爆炸了,松果被冲击波从了望台上震下来,摔伤了。 
        贝塔红眼了。他在炮艇向左晃的时候,瞄准右边按下射击按钮。 
        炮艇被打成了两截,沉了。 
        舒克一高兴,头撞在舱壁上。 
        荷叶跑上甲板把松果背进船舱。 
        贝塔来劲儿了,冲着后边的几艘军舰打起了连发。又有两艘军舰起火了。 
        “没炮弹了!”头版喊道。 
        贝塔傻了。一艘鱼雷快艇劈开水面冲过来,那架势分明是要和航空母舰同归于尽。 
        “必须挡住它!”舒克冲贝塔喊。 
        贝塔想起了他的坦克。 
        坦克停在甲板上。贝塔朝坦克跑去。 
        贝塔钻进坦克,驾驶坦克开到甲板后边。贝塔将一发炮弹塞进炮膛,通过潜望镜瞄准了鱼雷快艇。 
        鱼雷快艇正准备发射鱼雷,被贝塔一炮击中,爆炸了。 
        贝塔开心极了,他喜欢在航空母舰上用坦克炮打海战。他忘记了是在甲板上,开着坦克随心所欲地驰骋。 
        贝塔的坦克离船舷只有一米了,他还不知道。 
        “快停住!”头版急了。 
        贝塔没有听见,他想离那艘巡洋舰再近点儿。 
        头版不顾一切地跨上坦克,掀开坦克舱盖儿。 
        “停车!”头版闭着眼睛喊。 
        坦克的半截车身已经悬空了。头版认定自己将随坦克一起掉进水里。   第76集 
        子弹擦过贝塔的头; 
        航空母舰“投降”; 
        豪华的舰舱   
        贝塔毕竟是久经沙场的坦克兵,他控制住了坦克。坦克的半个身子悬空在甲板外边。 
        头版大喊一声:“贝塔万岁!” 
        贝塔缓慢地倒车。坦克回到了甲板上。 
        敌舰还是不放过航空母舰。 
        由于航空母舰速度慢,包围圈渐渐形成了。有三艘军舰已经出现在航空母舰前方。 
        舒克无法操纵航空母舰往前行驶了。 
        贝塔从坦克里探出头来,他看见航空母舰四周都是军舰。 
        贝塔迅速从坦克里跳出来,往驾驶室跑。 
        一排子弹朝贝塔射过来。贝塔赶忙趴在甲板上。子弹擦着贝塔的头飞过去。 
        贝塔匍匐前行进入驾驶室。 
        “怎么办?”贝塔问舒克。 
        “这大家伙太笨,要不然早跑没影了。”舒克在这紧急关头,首先想到的是自己的声誉。 
        “先说说怎么办吧!”贝塔望着窗外的军舰群说。 
        “叫臭球吧!”舒克没别的办法,尽管他觉得动不动就搬援兵挺丢人。 
        “有电台吗?”贝塔问。 
        舒克将驾驶室里的电台打开。 
        “臭球!臭球!我是贝塔,清回答!!”贝塔呼叫。 
        “……” 
        “臭球!臭球!我是贝塔,请回答!!”贝塔头上出汗了。 
        “我是臭球!我是臭球!请讲!” 
        “我们遇到了危险,请求空中支援!” 
        “机种?” 
        “歼击机10架。强击机20架。轰炸机20架。” 
        “有备降机场吗?” 
        “有!太有了!我们现在航空母舰上。” 
        “航空母舰上没飞机?哪儿找的破船!” 
        “哎呀别罗嗦了,快来吧!”贝塔将方位告诉臭球。 
        “坚持住!”臭球关上电台。 
        舒克和贝塔松了口气。 
        头版冲进来,说:“一艘巡洋舰向咱们打信号。” 
        贝塔往右边看。 
        “它说让咱们投降,不然就开炮了。”贝塔懂信号灯语。 
        “回答投降。”舒克用缓兵之计。 
        贝塔按动航空母舰上的信号灯开关。 
        “它说让咱们往回开。”贝塔将巡洋舰的灯语翻译给舒克。 
        “慢慢往回磨。”舒克操纵航空母舰掉头。 
        航空母舰在十几艘大小军舰的武装押送下,向岸边驶去。 
        “我去舱里转转,你再开慢点儿。”贝塔说完离开驾驶室。 
        航空母舰的船舱很大。豪华无比。有医院、餐厅、卧室、还有电影院。 
        贝塔眼睛都看花了。 
        荷叶看见贝塔来了,说:“松果在医院里躺着。” 
        贝塔问:“仓库在哪儿?” 
        荷叶指指前边。老鼠都有天生的找仓库的本领。 
        贝塔走进库房。他看见了弹药箱,还看见了舒克的直升机所需要的电池。 
        贝塔拿了几节电池。往甲板上跑。当他站在甲板上时,发现航空母舰已经快靠岸了。岸上的人们还挺激动。   第77集 
        舒克决定抛弃直升机和坦克; 
        臭球率领机群狂轰滥炸; 
        一架飞机险些误打自己人   
        贝塔跑进驾驶室。 
        “有电池了!”贝塔对舒克说。 
        “来不及装了。”舒克指指临近的岸。 
        “咱们不能让他们抓住呀!”贝塔急了。 
        “告诉大家拿救生圈准备跳水。”舒克说。 
        “直升机和坦克呢?’,贝塔问。 
        “以后再想办法弄回来吧!”舒克拍拍贝塔的肩膀。 
        贝塔舍不得,可也没办法。 
        “听!”舒克精神一振。 
        天空传来飞机的马达声。 
        “臭球!”贝塔一拳砸在驾驶台上。 
        舒克拿起话筒。 
        “臭球!臭球!我是舒克!” 
        “我是臭球!请指明目标!请指明空袭目标!!” 
        “除了航空母舰,随便打!打呀!!”舒克挥手。 
        双方势力立刻发生了显著变化。 
        臭球率领的机群对军舰们开始了灾难性的袭击。 
        歼击机俯冲扫射。轰炸机投弹。强击机低空点射。几乎是同时,军舰都起火了。 
        岸上的人们愣了。今天是怎么啦,先是舰模失控,接着又不知从哪儿冒出一群类似航模飞机似的战斗机,狂轰滥炸。 
        “打!打!!使劲儿打!!!”贝塔乐得直蹦。 
        一架飞机朝航空母舰俯冲。 
        “错了,自己人!!”贝塔握着话筒喊。 
        幸好飞机没开炮,避免了一起大水冲龙王庙的恶性事故。舒克操纵航空母舰掉头,从两艘被击沉的敌舰中间穿过去。 
        眼看着航空母舰走了,岸上的人们无计可施。 
        机群在空中为航空母舰护航。 
        舒克同臭球通话。 
        “我们返航了?”臭球问。 
        “别急,下来呆会儿,我还有事呢。”舒克说。 
        臭球从空中俯瞰,航空母舰已进人一个大湖泊,没什么危险了。 
        “07号机留在空中警戒,其他飞机跟我在航空母舰的跑道上着陆。” 
        臭球率领机群在航空母舰的跑道上鱼贯着陆。 
        “哪儿搞的这家伙,真漂亮!”臭球钻出座舱,看着航空母舰说。 
        “办报纸办出来的!”舒克幽了一默。 
        “酷!”臭球说。 
        “我想乘这艘航空母舰去大海逛逛,行吗?贝塔。”舒克问贝塔。 
        “太行了!”贝塔说。 
        “请你把我妈妈和咪丽接到你的机场去。”舒克对臭球说,“帮我照看一下。” 
        “没问题。”臭球说。 
        “再给我留六架飞机。两架歼击机,两架强击机,两架轰炸机。”舒克说。 
        “可以。”臭球当即选出了留下的六架飞机和六名飞行员。 
        贝塔领臭球参观航空母舰。臭球嫉妒得要死。 
        “你可以回去了。”舒克拍拍臭球,“有事再叫你!” 
        “你自己有了战斗机,别动不动就搬援兵了。”臭球说。 
        臭球的机群返航了。   第78集 
        无敌号航空母舰驶向大海; 
        荧光屏上奇怪的黑影; 
        一级战斗警报   
        “怎么样,去大海闯闯吧?”舒克问贝塔。 
        “你还想去哪儿?”贝塔显然非常同意。 
        “去太空。”舒克说。 
        “航海对咱们来说可是新鲜事儿,得准备准备吧?”贝塔说。 
        “咱们先给航空母舰起个名字,叫无敌号怎么样?”舒克说。 
        “够俗的。”贝塔撇撇嘴。 
        “你起一个。”舒克说。 
        “就叫无敌号吧,一时想不出别的。”贝塔作出无可奈何的样子。 
        “你当舰长,我当空军司令。”舒克提议。 
        “不错。”贝塔没想到自己这辈子还能当一回航空母舰舰长。 
        “咱们到会议室召集大伙分分工,准备启锚。”舒克整整飞行服。 
        航空母舰的会议室宽大明亮,豪华型沙发围成两圈。大家都来到会议室,连受伤的松果也来了。 
        “大家愿意开着这艘航空母舰去大海玩玩吗?”贝塔跷着二郎腿问。 
        “当然愿意!” 
        “太棒了!” 
        “……” 
        没人反对。 
        “我担任这艘无敌号航空母舰的舰长。舒克当空军司令。我任命头版当炮长,松果当雷达兵,荷叶当医生,头版的爸爸当大师傅,头版的妈妈当……”贝塔行使着舰长的人事大权,感到十分过瘾。 
        舒克给飞行员编了队。 
        “各就各位,准备启锚!”贝塔一声令下。 
        水兵们奔向自己的岗位。 
        舒克将电池装进直升机。飞行员们将战斗机固定在甲板上。贝塔将坦克开进隐蔽舱。餐厅里传出香味儿。 
        头版检查了舰上的几十门大炮。 
        贝塔站在驾驶室里,对着话筒下令: 
        “无敌号启锚,目标——大海!” 
        无敌号航空母舰徐徐开动了。 
        “大海在哪边?”掌握轮舵的贝塔想到方向问题,问舒克。 
        “哪儿宽阔往哪儿开。”舒克说。 
        “我开直升机升到空中看看。”舒克说完跑出驾驶室,钻进直升机。 
        直升机离开了航空母舰,升到空中。从空中看航空母舰,舒克觉得好玩。他没想到自己还能拥有一座水上机场。 
        “贝塔,贝塔,往南开。”舒克通过元线电台指挥航空母舰。 
        “我这个舰长还得听他指挥。”贝塔耸耸肩,操纵航空母舰往南开。 
        航空母舰终于开到大海里,这已经是一个星期以后的事了。舒克和贝塔见到了大海,他们一下子觉得自己渺小起来。他们后悔不应该办报而应该把老鼠同胞都带到航空母舰上来看大海。船员们都到甲板上看大海。 
        “哎呀,哎呀,哎呀!”头版连说三声哎呀,没别的词儿。 
        荷叶连一句话都说不出来。 
      “舰长,快到雷达室来!”松果通过扩音器喊贝塔。 
        贝塔跑进雷达室。 
        “你看,那是什么?”松果指着雷达扫描荧光屏上的黑点儿时贝塔说。 
        荧光屏上的黑点儿越来越大。显然是离航空母舰越来越近。 
        “一级战斗警报!”贝塔拉响了警笛。   第79集 
        舒克潜水; 
        黑色的物体是什么; 
        舒克遇险   
        “怎么回事?”舒克跑来问贝塔。 
        “海里有不明物体在靠近本舰。”贝塔指指荧光屏。 
        “在水下?”舒克挠挠后脑勺,他的空军技术无法发挥优势。 
        “咱们缺个潜水员。”舰长贝塔自知失职,分工时欠周全。 
        “我去。”舒克说,“快拿潜水服来!” 
        “也只好这样了,反正你是飞行员,身体素质好。”贝塔吩咐部下去仓库取潜水用具。 
        舒克穿戴好潜水服,准备人海。 
        “潜水头盔里有通话器,注意联络,听舰长指挥!”贝塔对舒克下令。 
        “可当了一回舰长。”舒克冲贝塔一笑,跳进大海。 
        “飞机随时准备起飞。”贝塔下令,  “大炮准备射击。” 
        飞行员钻进座舱,炮手坐上炮位。 
        舒克来到大海里,他擦着航空母舰往下潜水。许多巨大的鱼在舒克身边游来游去,不过对舒克都挺友好。 
        “舒克,舒克,我是贝塔,请回答。”舒克的耳机里传出贝塔的呼叫。 
        “我是舒克,请讲。”舒克说。 
        “报告你的方位。” 
        舒克看看潜水表上的罗盘,报告着自己的方位。 
        “你已经接近了那家伙,注意!请尽快报告观察结果。” 
        舒克一回头,看见水中有一个黑色的物体,正缓缓地游着。 
        “我看不清是什么,我靠近它。”舒克告诉贝塔。 
        “注意安全。”贝塔叮嘱道。 
        舒克悄悄地游到黑色的物体身边,他摸摸它,很硬。 
        舒克绕着它游了两圈儿。 
        “不是动物。也不是植物。”舒克随时向贝塔报告。 
        “是船?”贝塔问。 
        “可它全身都在水里!”舒克否定了。 
        “潜水艇!”贝塔大喊一声。 
        舒克被提醒了,没错,是一艘潜水艇。 
        “向它发信号,问它是谁?”舒克对贝塔说。 
        过了一会儿,贝塔告诉舒克,没回答。 
        “我进去看看。”舒克决定历险。 
        “祝你好运!”贝塔相信舒克的胆量和运气。 
        舒克很容易就找到潜水艇的潜水舱,他的身体 
        刚一钻进潜水舱,舱门就关上了。 
        “我中计了。”舒克通知贝塔。 
        只听一阵巨响,舒克的身体离开了潜水舱,进入潜水艇内。 
        舒克的胳膊被控制住了,眼睛被蒙上了黑布。他被带到一个舱里。 
        “解开。”一个熟悉的声音。 
        舒克眼前一亮,他看见海盗坐在对面的沙发上,穿着海魂衫。 
        “没想到吧?我一直跟踪你们,你的雷达兵该撤职。”海盗笑笑。 
        “够刺激!”舒克对这次历险的惊险度和出人意料度表示满意。 
        “这回刺激你回老家!皮皮鲁救不了你了吧?”海盗打了个榧子。 
        舒克看看四周,他身后站着四名水兵。 
        “你征兵还挺有办法。”舒克说。 
        “老鼠家族,找人干坏事还不是召之即来?”海盗盯着舒克笑。 
        舒克摇摇头,为同胞的素质遗憾。 
        “报纸办得有效果呀!”海盗不知从哪儿摸出一张舒克出版的报纸,扔给舒克。 
        “你想怎么办?”舒克问海盗。 
        “我一按电钮,鱼雷就能击沉你的航空母舰。”海盗一乐:“可我不这样干,太损了。我想当航空母舰舰长,统率这座水上机场,当大海的主人。” 
        舒克转眼珠。 
        “这回你跑不了,别瞎费脑子了。当然,为了不使我在发生意外后感到遗憾,我现在就把你的尾巴割下来。我正缺一根腰带。”海盗像大元帅看俘虏似地看着舒克。 
        “我要是帮助你得到航空母舰呢?”舒克认为尾巴比航空母舰重要。 
        “航空母舰和尾巴我都要。”海盗向部下招手,“拿刀来。”   第80集 
        尾巴捆住海盗的手; 
        舒克按了发射鱼雷的按钮   
        部下把刀子递给海盗。 
        “把尾巴伸过来。”海盗冲舒克飞了个吻。 
        舒克老老实实走到海盗跟前,把屁股转过来。, 
        海盗弯腰抓住舒克的尾巴,另一只手握刀准备割。 
        舒克飞快地用胳膊夹住海盗的脖子,另一只手夺过海盗的刀,然后用刀尖顶住海盗的太阳穴。 
        水兵们冲上来。 
        “别动,再往前走我就把刀捅进去了。”舒克警告水兵们。 
        水兵们等海盗的指示。 
        海盗挥手示意部下听舒克指挥。他真后悔自己麻痹大意,让舒克钻了空子。 
        “都出去。”舒克说。 
        水兵们退了出去。 
        舒克用海盗的尾巴捆住海盗的手。 
        “带我去驾驶舱。”舒克用刀顶着海盗的后脑勺。 
        海盗顺从地带着舒克去驾驶舱。走廊里的水兵们死盯着舒克手中的刀,谁也不敢轻举妄动。 
        “这就是驾驶舱。”海盗停在一座小圆门跟前。 
        舒克推开门,里边有两名水手在操纵潜水艇。 
        “出来!”舒克对驾驶舱里的水手说。 
        “你是谁?”水手没见过舒克。 
        “出来!”海盗喝了一声,他感觉到后脑勺上的刀尖有变化。 
        水兵老老实实出来了。 
        舒克押着海盗走进驾驶舱,把舱门关死。 
        荧光屏上显示出无敌号航空母舰。 
        “电台在哪儿?”舒克问海盗。 
        海盗十分不情愿地把电台的方位告诉舒克。 
        舒克打开电台,戴上耳机,他调整着频率。 
        “贝塔,贝塔,我是舒克.请回答!” 
        “我是贝塔。你怎么搞的,半天不吭声。”贝塔质问。 
        “你猜我在潜水艇里碰见谁了?” 
        “反正不会是海盗。” 
        “正是海盗。”舒克加重了语气。 
        “说着玩吧?” 
        “我让海盗跟你说话。”舒克把话筒递到海盗嘴边,“你说。” 
        海盗真想一口吃了舒克,这会儿还拿他开心。可他不得不说,刀尖又动了。 
        “我是海盗。”海盗对着话筒说。 
        贝塔听山是海盗,他吓出了一身汗。 
        “我缴获了他的潜水艇。”舒克拿回话筒说。 
        “我给你记功!”贝塔拿出舰长的口气。 
        舒克不稀罕贝塔的嘉奖。 
        “操纵潜水艇往后退的开关在哪儿?”舒克问海盗。 
        海盗把发射鱼雷的开关告诉舒克。 
        “贝塔.我现在操纵潜水艇往后退,离开航空母舰。”舒克说完往下按电钮。 
        一阵巨响。 
        “怎么回事?”舒克一惊,他隐约感到上了海盗的当。 
        “无敌号中弹了!”贝塔大叫。 
        “你按的是发射鱼雷的按钮。”海盗狂笑起来。 
        舒克一拳打在海盗头上,海盗倒下了。 
        “情况怎么样?”舒克问贝塔。 
        “有两个舱进水了!”贝塔说。 
        “快堵住!我把潜水艇开出海面,你让头版打沉它!”舒克说。 
        “那你呢?” 
        “我能跑出去。”舒克说。 
        舒克蹲下又打了海盗一拳,他确信海盗真昏过去后,才放心地站在操纵台前研究怎么开潜水艇。 
        舒克的耳机里不断传来贝塔指挥抢救航空母舰的声音。 
        舒克渐渐摸着了潜水艇的门路。潜水艇向上升去。 
        海盗醒了,他轻轻地站来,悄悄走到舒克背后。   第81集 
        舒克将潜水艇浮出水面; 
        头版击碎潜水艇; 
        贝塔当不成舰长了   
        舒克早已从操纵台上的反光镜里看见了海盗,他猛一回手,给了海盗一拳加一脚。继续操纵潜水艇。 
        海盗又昏过去了。 
        “贝塔,贝塔,潜水艇马上就要浮出水面。在航空母舰右侧,准各打。”舒克喊话。 
        “你离开潜水艇我就打。”贝塔说。 
        舒克确信潜水艇已经浮出水面后,叫醒了海盗。 
        “快叫你的部下跑吧.一分钟后潜水艇将爆炸!”舒克给海盗解开捆着手的尾巴。 
        海盗不解地看看舒克,他不明白舒克干吗不处决他。 
        “快点儿!”舒克大喝一声。 
        海盗打开舱门跑了。 
        “贝塔,一分钟后开炮!”舒克说完扔掉话筒,跑出驾驶舱。 
        贝塔指挥头版操纵大炮瞄准了水面上的潜水艇。 
        “先别打,还没看见舒克出来。”贝塔对头版说。 
        “打呀!”舒克从航空母舰的船舷上伸出头来喊,他已经回来了。 
        头版终于有了显示他具备射击天才的机会,他冲着浮在海面上的近在咫尺的潜水艇乱打一通。潜水艇被打得碎片横飞,粉身碎骨。 
        “行啦行啦!”贝塔制止头版。 
        “报告舰长,缺口堵不住了,舱里进了好多水!”松果跑来报告。 
        贝塔和舒克跑进舱里一看,水已经漏进舱里许多,荷叶正指挥水兵们堵缺口。 
        “看样子堵不住了。”舒克遗憾地说。 
        “这是你的杰作,自己打自己。”贝塔拍拍舒克的肩膀。 
        舒克苦笑。 
        “怎么办?”贝塔问舒克。 
        “不要航空母舰了,咱们都坐飞机走。”舒克说。 
        “我的舰长当不成了。”贝塔喜欢当舰长。挺过瘾。 
        舒克耸耸肩。 
        “飞机坐得下吗?”贝塔能上能下,不当舰长就不当。 
        “轰炸机的弹舱里可以坐几位。”舒克已经想好了。 
        航空母舰在迅速下沉。 
        “都到甲板上去,准备登飞机。”贝塔命令部下。 
        舒克召集飞行员下达任务。 
        “我驾直升机,吊着坦克。歼击机和强击机护航。轰炸机用弹舱装人,千万别按投弹按钮。”舒克对飞行员们说。 
        飞行员们奔向自己的飞机。 
        这时海面上传来一阵“救命”声。 
        舒克一看,是海盗和他的水兵。他们在海水里扑腾着.喊叫着。 
        “给他们抛救生圈。”舒克不忍心看自己的同胞遭受灭顶之灾。 
        “救上来怎么办?”贝塔问。 
        “管他呢。先救上来再说。”舒克一边说一边往海里扔救生圈。 
        潜水艇的水手们在抢救生圈。 
        “上来一个捆一个。”舒克说。 
       头版找来了绳子。 
       海里的潜水艇水手都陆续爬上了航空母舰,又都陆续被捆起来——那他们也愿意。 
        海盗一爬上航空母舰就被五花大绑。 
        “见到您真高兴。”贝塔冲海盗点点头。 
        海盗不说话。他看着下沉的航空母舰,心里高兴,反正大家同归于尽。   第82集 
        海盗被囚禁在孤岛上; 
        舒克率领机群返回陆地   
        “准备起飞!”舒克一声令下。 
        海盗愣了。他忘了这是航空母舰,有机场。 
        “我们怎么办?”海盗的部下哀求道。 
        “一起走。不过你们得老实点儿。”舒克说。 
        “海盗呢?”头版问。 
        “也走。不过不能让他回到陆地上去。”舒克看看海盗,  “咱们来时见过一座孤岛,把他扔到孤岛上。” 
        航空母舰的甲板快和海水平行了。 
        “快登机!”舒克大喊。 
        大家押着俘虏分头登上飞机。 
        “歼击机队,起飞!”舒克坐在直升机驾驶舱里指挥。 
        歼击机风驰电掣般插进云霄。 
        “强击机起飞!”舒克下令。 
        强击机闪电般甩开航空母舰。 
        海水涌上了甲板。 
        “轰炸机,快起飞!”舒克一边下令一边操纵直升机吊着坦克离开了甲板。 
        一架轰炸机趟着海水起飞了。 
        “报告司令,02号轰炸机出现故障!”另一架轰炸机的飞行员向空军司令舒克报告。 
        舒克从空中往下一看,海水已淹没了02号轰炸机的起落架。再耽搁一分钟。轰炸机将葬身海底。 
        “什么故障?”舒克问。 
        “发动机转速不够。”飞行员回答。 
        “强行起飞!”舒克命令。尽管他知道发动机转速不够是不能强行起飞的,可他没有别的办法。 
        轰炸机像船一样劈开海水疾驶,它离开甲板后很长一段时间还在海面上滑行。 
        “慢慢拉杆,别太猛!”舒克提醒轰炸机飞行员。 
        轰炸机终于吃力地离开了海面,艰难地跃上天空。 
        机群向着陆地飞行。 
        航空母舰沉没了。 
        贝塔从空中俯瞰海面,他真想让舒克把海盗从直升机上扔下去。贝塔没当够舰长。 
        舒克看见了那座孤岛。岛上有植物。海盗饿不死。 
        “各机组在空中盘旋,我把海盗放到孤岛上去。”舒克通过电台指挥机群。 
        直升机吊着坦克在孤岛上着陆了。 
        “你就留在这里吧!”舒克把海盗押离直升机。 
        “不!不!!”海盗抗议。 
        “你还想回陆地?”舒克问。 
        “……”海盗盯着舒克,不回答。 
        “你总想着霸占东西,呆在这里最合适,这座岛归你霸占了。”舒克给海盗松绑。 
        海盗一屁股坐在地上。他承认自己喜欢霸占,可如果就剩下他自己,他觉得霸占的东西再多也索然无味。霸占是占给别人看的。 
        直升机起飞了。 
        海盗绝望地大喊了一声。 
        “有机会我一定来看看他。”舒克想。 
        松果和荷叶问舒克:“咱们现在去哪儿?” 
        “到机场就把他们放了。”舒克想想说。 
        经过五个小时的飞行,机群飞临陆地上空。 
        舒克开始同臭球通话。 
        “我们请求在机场着陆。”舒克说。 
        “同意。”臭球回答。 
        “又要回家了。”贝塔想。 
        臭球指挥战斗机陆续在跑道上着陆。 
        舒克操纵直升机直接降落在停机坪上。贝塔从坦克里跳出来,他揉揉眼睛,机场变化真大。 
        候机大楼扩建了,塔台增高了,跑道加长加宽了。 
        臭球朝直升机跑过来,后边跟着罗丘和机场的好多工作人员。   第83集 
        罗丘在宴会厅举行盛大宴会欢迎舒克 
        和贝塔; 
        舒克和贝塔开始新的生活   
        “你们干得不错。”舒克看着机场对臭球和罗丘说。 
        “创业最难。”罗丘回赠舒克一顶高帽子。 
        “先去客厅休息吧?”臭球提议。 
        “我们带回来一些俘虏。先把他们放了。”舒克说。 
        “俘虏?”罗丘问。 
        “要不是海盗捣乱,我们现在还在海上航行呢!”贝塔不无遗憾地说。 
        “海盗追到海上去了?”臭球吃惊。 
        “我们把他留在一座孤岛上了。”舒克说。 
        头版跑来报告舒克,俘虏都释放了。有几名不愿意走,想留下找个工作干干。 
        “行。”臭球点头。 
        “咱们去客厅休息,”臭球扭头对舒克说,“你妈妈已经接来了,她现在客厅等你呢。” 
        “你怎么不早说!”舒克撒腿往客厅跑。 
        舒克见到了妈妈,贝塔见到咪丽。又是一番热闹场面。 
        罗丘在宴会厅为客人们操办了盛大的宴会,大家频频为机场的创建人舒克和贝塔干杯。 
        “下一步干什么?”贝塔喝了口酒小声问舒克。 
        “咱俩还是走吧?”舒克说。 
        “头版他们呢?” 
        “留在机场工作。” 
        “不错。” 
        “就这么定了。” 
        舒克举起酒杯。大家静下来。 
        “我和贝塔准备明天走了……”舒克刚说了一半,就被人家打断了。 
        “我呢?”头版问。 
        “我呢?”松果问。 
        “还有我!”荷叶嚷嚷。 
        “我……” 
        “我……” 
        “你们留在机场,让臭球给你们分配工作。”舒克说。 
        “我是编辑。”荷叶大声说,“机场不需要编辑。” 
        “机场马上办报。”臭球给舒克解围。 
        “贝塔需要炮手!”头版提醒贝塔别白培养他。 
        “机场有高射炮。”罗丘说。 
        “我们会经常来看你们的!”贝塔干了一杯。 
        大家知道,舒克和贝塔喜欢冒险,喜欢过有刺激性的生活。他们不愿意有负担。 
        “祝你们好运气!”荷叶举杯。 
        “谢谢,谢谢!”舒克和贝塔眼眶湿润了。 
        第二天一早,大家到停机坪上为舒克和贝塔送行。 
        罗丘给舒克和贝塔准备了充足的食物。臭球给贝塔的坦克储备了足够的弹药。 
        舒克跨进直升机驾驶舱。贝塔钻进坦克。他们心里挺难受。生活就是这样,一会儿分离,一会儿合聚,有喜有忧。乐在其中。 
        直升机的螺旋桨开始旋转了。 
        舒克和贝塔又开始了新的生活。   第84集 
        雷电袭击直升机; 
        舒克和贝塔跳伞; 
        火车顶上历险   
        舒克驾驶直升机吊着坦克升到空中,贝塔坐在舒克身边。 
        “还记得咱们头一次打交道吗?”舒克想起了第一次见贝塔时打仗的情景。 
        “当然。”贝塔感到亲切,“你从空中往下压我的坦克,够狠的。” 
        “你把我的飞机挂在树上,也不善。”舒克说。 
        “咱们现在去哪儿?”贝塔向地面了望。 
        “不知道。”舒克说。 
        “去看皮皮鲁吧?”贝塔提议。 
        “行。”舒克调整航向。 
        直升机朝皮皮鲁居住的城市飞去。 
        “好像要变天。”贝塔发现天上有乌云在活动。那乌云仿佛张开血盆大口,要吞噬蓝天和白云。 
        舒克把头伸出机舱,吐舌头。 
        “马上着陆。”舒克说。 
        乌云的速度比直升机的速度快,一道霹雳般的闪电把天空劈成两半,直升机一阵剧烈晃动后,起火了。 
        “雷击!”舒克大叫。 
        贝塔打开通向后舱的门,一阵浓烟扑进驾驶舱。 
        “飞机失控!”舒克惊呼。 
        “怎么办?”贝塔问。 
        “跳伞!”舒克离开座位。 
        “飞机?坦克?!''贝塔急了。 
        “命要紧!”舒克顶着火焰跑进后舱,拿出两个伞包,扔给贝塔一个。 
        舒克踢开舱门。 
        直升机急剧下降。离地面不远了。 
        “快背好伞包!跳!”舒克把贝塔推出机舱,自己也跟着跳出去。 
        两顶降落伞张开了。 
        直升机和坦克冒着烟坠毁了。 
        舒克调整着降落伞的方向,同贝塔平行。 
        “我的坦克!”贝塔心疼地大喊。 
        “注意地面!”舒克提醒贝塔,“坦克还会有的。” 
        贝塔往下看,地面是一条铁路。 
        豆大的雨点从天而降,降落伞被打湿了。舒克和贝塔下降的速度越来越快,马上就要着陆了。 
        就在这时,一列火车呼啸着开过来,把舒克和贝塔同大地隔开。 
        舒克落在火车顶上,降落伞挂在火车顶上的排气筒上。舒克趴在车顶上,巨大的气流从他的后背席卷而过。 
        雨越下越大。 
        舒克往四周看看,没有贝塔的影子。 
        “贝塔!贝塔!!”舒克喊叫。 
        没有回答。同暴风雨比起来,舒克的呼叫声太微弱了。 
        贝塔没落到火车上?被火车轧了?舒克猜测。失去直升机舒克不在乎,可失去贝塔使他神思恍惚。 
        舒克掏出小刀,割断伞绳,他决定在车顶上找贝塔。如果没有,就跳车沿路往回找。 
        在火车顶上行走很危险,何况还下着雨。舒克的身体紧贴着车顶,一点儿一点儿挪。 
        爬上两节车厢,舒克看见了惊心动魄的一幕:一顶降落伞缠绕在车厢顶部的金属丝上,降落伞的伞绳飞舞在车厢的外边,贝塔的身体在随风飘动,他够不着车厢! 
        贝塔已经绝望了。 
        舒克爬到降落伞旁,他用力往车厢这边收伞绳,力图使贝塔的身体够着火车。 
        一阵狂风袭来,舒克差点儿被吹下火车。他把自己的尾巴拴在伞绳上当安全带,继续努力。 
        贝塔发现了舒克,他开始配合舒克。正好一阵风刮来,贝塔借着风力,身体挨上了车厢,舒克伸手拉住了他。   第85集 
        在行李架上休息; 
        贝塔发现一辆小卧车; 
        舒克想开汽车   
        “抓紧!”舒克喊道。 
        贝塔趴在车顶上,死死抓住伞绳。 
        “你当伞兵太差。”舒克说。 
        “全世界也找不到比我技术高的伞兵了。”贝塔缓过劲儿来了。 
        “怎么样?”舒克抹脸上的雨水。 
        “够刺激。”贝塔全身精湿。 
        “咱们去车厢里暖和暖和。”舒克说。 
        “走。”贝塔从身上摸出伞刀割绳子。 
        舒克和贝塔钻进通风管道。管道里很黑,但暖和。 
        “那儿有亮光。”贝塔指指前边。 
        他们朝亮光走去。 
        舒克趴在亮光上往下看,是行李架。行李架下边是卧铺车厢。 
        “有人吗?”贝塔问。 
        “很多人。”舒克说。 
        “能下去吗?” 
        “小心点儿。” 
        舒克先下去了,他躲在一个旅行包后边。贝塔随后跳下去。 
        车厢里热气腾腾,对刚从雨中进来的舒克和贝塔来说是天堂。 
        “把衣服脱下来拧干。”贝塔说。 
        “完全必要。”舒克同意。 
        他们把飞行服和坦克服脱下来。 
        “没有飞机和坦克了,还穿这个干什么?”贝塔发牢骚。 
        “留个纪念嘛。”舒克说。 
        “注意。”贝塔警告道。 
        一个人伸手到旅行货架上找东西。舒克和贝塔忙躲在旅行包后边。 
        那人找完东西下去了。 
        “这火车往哪儿开?”贝塔嘀咕。 
        “天知道。”舒克伸了个懒腰。 
        贝塔往下看,车厢里有人在打扑克,有人看书,有人睡觉。 
        “也不知道他们都去干什么?”贝塔自言白语。 
        “忙呗。活着就是忙。”舒克说。 
        “你看那边!”贝塔的声音有些激动。 
        舒克顺着贝塔指的方向看,行李架上有一个华丽的玩具汽车盒子。 
        盒子上画的汽车神气极了。 
        “去看看。”舒克的手痒痒了。 
        他俩蹑手蹑脚地爬到汽车盒子旁边。盒子用塑料绳捆着。 
        “钻进去看。”舒克掀开盒盖的一角。 
        贝塔和舒克先后钻进去。 
        一辆黑色的超豪华小轿车呈现在舒克和贝塔眼前。 
        “真棒。”舒克脱口而出。 
        “是遥控的?”贝塔判断。 
        “进去看看。”舒克拉开车门,坐在驾驶员的位置上。 
        贝塔坐在舒克身边。 
        车里宽敞舒适,设备都是一流的。 
        舒克转转方向盘,挺过瘾。 
        “这比坦克好开多了。”贝塔冒出这么一句。 
        “跟飞机更没法比了。”舒克说。 
        这时,汽车忽然晃动起来。 
       “注意,他们要打开盒子!”舒克说。 
        “咱们这叫自投罗网。”贝塔低头看看座位下边,“这底下可以藏。” 
      盒子没有打开,却继续晃动着。 
      “汽车的主人下火车了。”舒克断定。   第86集 
        小个子和眼镜的交易; 
        舒克开汽车的技术不如贝塔; 
        舒克拒绝小个子的要求   
        舒克猜对了,汽车的主人到站下火车了。 
        “怎么办?”舒克问贝塔又像是在问自己。 
        “听天由命,反正现在跑不出去。”贝塔整整自己的坦克服。 
        汽车继续晃动,舒克练习想像力: 
        “现在上公共汽车了。现在下车了。现在快到家了。” 
        汽车终于停止了摇晃。 
        “你蒙得还挺准。”贝塔说完准备开车门下去。 
        “等会儿,听听动静。”舒克制止贝塔。 
        果然,盒盖被掀开了。一只大手伸进来将汽车从盒子里拿出去。 
        “就是这车,样子很棒,但性能不行。我们厂长说,请你多关照。”一个粗声音说。 
        舒克趴在窗玻璃上往外看,拿车的是一个三十多岁的男人,小个子。 
        “这回来参展的玩具很多,不好照顾呀。”另一个人说。 
        舒克顺着声音看去,是一个戴眼镜的中年男人。 
        “这是点儿小意思。”小个子塞给眼镜一个信封。 
        眼镜打开一看,是钱。笑了。 
        “我会尽力的。”眼镜改口道。 
        小个子把汽车放在地上,同眼镜聊天。 
        舒克和贝塔渐渐听明白了,这里要举行玩具博览订货会。许多厂家都将自己的产品送来参展,寻找销路。小个子的工厂生产的这种汽车外观华丽,但性能不好。他们收买了博览会的评委工作人员,弄虚作假。 
        “够邪的。”贝塔吹了声口哨。 
        “现在是机会,走吗?”舒克问。 
        “怎么着也得开车过一下瘾吧?急什么?大风大浪都闯过来了。”贝塔想开汽车。 
        “行,行。”舒克没意见,好在窗玻璃是茶色的,外边看不进来。 
        小个子送客去了。 
        舒克发动汽车。马达声很大。 
        “发动机质量够呛!”贝塔亮出行家的口气。 
        舒克很快就掌握了驾驶汽车的技术。 
        “人真蠢,听说考个驾驶执照要几个月时问,其实半个小时就能学会。自己折腾自己。”舒克边开边说。 
        “让我开一会儿。”贝塔说。 
        舒克把驾驶员的座位让给贝塔。 
        舒克不得不在心里承认,贝塔开汽车的技术比他高,毕竟是开坦克出身。 
        小个子送客回来,走到房间门口听见屋里汽车响,他悄悄把门推开一道缝儿往里看,愣了。 
        汽车自己在地上来回跑着,没有人操纵!而且动作灵活,令人眼花缭乱。 
        小个子突然推开门,闯进屋里从地上拿起汽车。 
        当舒克和贝塔反应过来时,车门已经被小个子打开了。 
        他惊讶极了,车里是两只穿着衣服的小老鼠! 
        贝塔冲小个子点点头,无可奈何地苦笑一下。毕竟是玩具工厂的推销员,小个子很快就接受了这个现实,这两只小老鼠会开汽车,而且开得很好,绝了!小个子心里萌生了一个伟大的念头。他后悔给眼镜塞钱塞早了。 
        “听着,我不抓你们。但明天你们帮我表演。懂吗?我拿着遥控器,你们开车,明白吗?”小个子对舒克和贝塔说。 
        舒克和贝塔明白了,小个子要他俩帮他作弊。小个子要在博览会上假装拿着遥控器遥控汽车。其实呢,汽车是由舒克和贝塔操纵。这样就能让人觉得汽车菲常灵活。 
        “做梦。”舒克忍不住冲口而出。 
        小个子没吃惊,他早就断定会开车的老鼠准会说话。他把车门一关,从外边用胶带封死了。   第87集 
        贝塔被扣为人质; 
        舒克在玩具博览订货会上表演开汽车   
        小个子还怕不保险,又把汽车塞进床头柜里,锁住。 
        舒克试了试,车门根本打不开。 
        “别管他,咱们先睡觉。”贝塔在后座上躺下。 
        “睡就睡。”舒克躺在前座上。 
        舒克和贝塔在睡梦中被一阵大喝声惊醒了。 
        “快起来,睡得真舒服呀!”小个子拉开车门喊。 
        舒克和贝塔坐起来。 
        小个子伸进两个手指,把贝塔从车里夹出去。舒克急了。 
        “别急。你今天好好表演,我不会亏待你们。如果捣乱,他就是人质,我把他交给猫。”小个子说完把贝塔塞进准备好的铁盒子里。 
        舒克惊讶小个子能想出这么恶毒的办法,他无计可施。 
        “现在就去参加订货会,我冲你打手势,你就开车,越灵活越好。”小个子给舒克下指令,“你的同伴就在我的提包里,你要不老实,他可要吃苦了。” 
        舒克想一口吃了小个子。 
        汽车又被装连纸盒子。舒克透过纸盒子上的玻璃纸窗,看见小个子走进一座富丽堂皇的展览馆。展览馆里有许多各种各样的玩具,许多围着玩具看,大概是准备订货。 
        小个子走到自己的展台前,从纸盒子里掏出汽车,放在展台上。 
        展台有两平方米,高出地面一米。 
        小个子拿出一盘磁带,插进录音机。 
        “这是本厂生产的最新式AQ——20型遥控汽车,性能良好,造型美观……”录音机的喇叭叫唤着。 
        围过来几个人。 
        “现在我给各位表演。”小个子假摸三刀地拿起遥控器。 
        人们的眼睛盯着汽车。 
        小个子冲舒克打了个手势。舒克不得不启动汽车,他怕贝塔受罪。 
        舒克驾驶汽车在展台上行驶着。他一会儿转弯,一会儿倒车。 
        “真灵活!” 
        “不错!” 
        观众赞叹着。 
        “我订五千辆!” 
        “我订两千辆!” 
        “我订一万!” 
        “……” 
        小个子应接不暇地签合同。 
        舒克试图打开车门逃跑,车门被小个子从外面封死了。 
        舒克坐在车里,看着那么多人轻而易举地上了小个子的当,而自己正是小个子的帮凶,他气坏了,越想越不甘心。 
        舒克想发动汽车从展台上捧下去,可又怕贝塔倒霉,只好老老实实坐在车里。 
        过了一会儿,小个子又让舒克表演。这回吸引了更多的人。人们惊叹这种遥控汽车的性能。 
        舒克整整开了一天车,小个子一点儿饭也不给他吃。 
        闭馆的时候,小个子从铁盒里拽出贝塔,把他塞进汽车。 
        “你表演得不错,你的朋友得谢谢你。”小个子替贝塔谢舒克,阴阳怪气。 
        舒克在心里骂了一句最难听的话。       第88集 
        舒克和贝塔撕毁合同书; 
        胖子和瘦子略施小计; 
        舒克和贝塔再次遇难   
        小个子把车门封死,然后将汽车塞进床头柜锁上。 
        汽车里一片黑暗。 
        舒克打开车灯。 
        “这家伙真坏,坑人。”舒克咬牙切齿。 
        “咱们得治治他。”贝塔掏出伞刀。 
        “把玻璃扎碎。”贝塔说。 
        玻璃碎了。 
        舒克爬出汽车。 
        “当心玻璃碴。”舒克提醒贝塔。 
        贝塔也从车里钻出来。 
        床头柜下边有一道缝儿,舒克试了试,太窄。 
        舒克和贝塔一起用刀扩张那条缝儿。 
        “行了。”贝塔收起刀,先钻出去。 
        舒克紧跟着钻出去。 
        房间里没人,小个子大概吃饭去了。 
        “把他订的合同都给撕了。”舒克说。 
        “太应该了。”贝塔说完爬上小个子的公文包。 
        公文包里是满满一包合同书,舒克和贝塔连撕带咬,合同书都粉身碎骨。 
        “还应该去告诉那些订货的人,别上当。”舒克提议。 
        “走!”贝塔同意。 
        舒克和贝塔从门缝儿底下钻到走廊上。这是一座旅馆,被参加玩具博览订货会的人包下了。 
        “咱们到每个房间去说。”贝塔提议。 
        “行。”舒克同意。 
        他们钻进第一个房间。 
        房间里两个人在煮方便面。 
        “今天那遥控汽车不错,我订了五千辆。”胖子说。 
        “我那儿子见了就不走了,我也订了不少。”瘦子说。 
        “那车质量不好,你们受骗了!”舒克大声说。 
        胖子和瘦子吓了一跳,看看门,关着。 
        胖子看看瘦子。瘦子看看胖子。 
        “你们上当了。”贝塔大喊一声。 
        胖子和瘦子往地上一看,两只老鼠。 
        闹鬼了! 
        胖子壮着胆问:“你们怎么知道?” 
        舒克把经过说了一遍。 
        胖子冲瘦子使个眼色,瘦子领会了胖子的意图,他朝门口走去。 
        舒克和贝塔的退路被堵死了。他俩还不知道等待自己的是什么。 
        “感谢你们来报信。”胖子蹲下来对舒克和贝塔说。 
        一条枕巾从天而降,扣住了舒克和贝塔。 
        瘦子得意极了,大叫:“抓住了!抓住了!” 
        “放哪儿?”胖子问。 
        “扣在玻璃杯里,一个杯子里一只。”瘦了说。 
        舒克和贝塔被分别扣在两只玻璃杯里,他们不明白这两个人于吗来这一手。 
        胖子和瘦子像看天外来客似地看舒克和贝塔。 
        “你说会说话的老鼠值多少钱?”胖子直接进人问题的实质。 
        “少不了。”瘦子说。 
        “咱们一人一只,怎么样?”胖子斜眼看瘦子。 
        瘦子点点头,没吭声。 
        舒克看着被关在对面杯子里的贝塔苦笑。       第89集 
        瘦子威胁舒克; 
        雷雷和舒克、贝塔交谈; 
        猫来了   
        “你们订的遥控汽车是假的!”贝塔在玻璃杯里冲胖子和瘦子大喊。 
        “现在什么不是假的?”瘦子反问贝塔。 
        “大惊小怪!”胖子笑笑,“假药,假酒,还少吗?你说玩具汽车是假的,又害不了人命,有什么关系?” 
        舒克和贝塔愣了,他们感到人和老鼠差不多,一个层次。 
        房间门开了,一个六岁左右的男孩子跑进来。 
        “雷雷,看爸爸抓了只什么?”瘦子叫儿子过来。 
        雷雷跑到桌子前一看,乐了。 
        “穿衣服的小老鼠!”雷雷兴奋了。 
        “还会说话呢!”胖子说。 
        “会说话?!”雷雷不信。 
        “说句话!”瘦子冲舒克说。 
        舒克不说。 
        “爸爸骗人!”雷雷给爸爸定性。 
        “你他妈说不说?”瘦子不能背这个黑锅,他一捋袖子,指着舒克骂道。 
        “你不说?我倒上汽油点了你!”胖子加强攻势。 
        舒克打了个哆嗦。 
        “别吓唬它们。”雷雷不满了。 
        “雷雷,你在这儿看着它们,我们出去一下。”瘦子对儿子说。 
        “去干吗?”雷雷问。 
        “小孩子不懂。”瘦子拍拍儿子的头,然后和胖子嘀咕了几句什么,两人出去了。 
        舒克听见了。瘦子说,既然汽车是假的,咱们订了不少,就应该敲小个子一下,让他给什么“回扣”。 
        雷雷趴在桌子旁,盯着舒克和贝塔看。 
        “你们真会说话吗?”雷雷问。 
        贝塔点点头。 
        雷雷惊讶了,瞪大了眼睛。 
        “你们怎么还穿着衣服?”雷雷问。 
        舒克觉得这个孩子的眼睛挺善良,决定和他谈谈。 
        “我是飞行员舒克,他是坦克兵贝塔。”舒克说,“我们是偶然来到这个展览会的……” 
        舒克把事情的经过讲给雷雷昕。 
        “我爸爸这么坏?”雷雷站起来。 
        舒克和贝塔看到了希望。 
        “我放你们走!”雷雷掀开了两只玻璃杯。 
        舒克和贝塔没想到瘦子生了这么个好儿子。真怪。 
        “谢谢你。”舒克说。 
        “再见!咱们还能见面吗?”看得出,雷雷舍不得和两只小老鼠分手。 
        “能见着!”贝塔说。 
        门外传来脚步声。 
        “快躲到门后去!”雷雷说。 
        舒克和贝塔藏到门后。 
        瘦子、胖子和小个子推门进来。 
        “老鼠呢?”瘦了一眼看见玻璃杯里的老鼠没了。 
        舒克和贝塔趁机溜出门外。他们听见“啪”的一记耳光。紧接着是雷雷的哭声。 
        脚步声。 
        “快进这个房间,他们追出来了!”贝塔拉了舒克一把,他们钻进临近的一个房间。 
        “跑不了,你快去餐厅把那只大猫抱来。我在这儿守着!”瘦子怒气冲冲地对胖子说。 
        小个子在一旁幸灾乐祸,当然他也希望抓住舒克和贝塔,出出气。 
        猫来了。   第90集 
        提包救了舒克和贝塔; 
        半夜里发生的事; 
        舒克和贝塔对世界失去信心   
        舒克和贝塔进的这间屋子黑着灯。他们钻到床底下。 
        走廊里一阵喧嚣。 
        猫嗅到了舒克和贝塔藏匿的房间门口。 
        “在这儿!”瘦子断定。 
        胖子敲门。 
        “找谁?”一个女人的声音从舒克和叭塔躲藏的床上发出。 
        “您的房间有老鼠!”瘦子说。 
        “老鼠?!”女人吓了一跳,从床上蹦起来,下地开灯。 
        “快躲进她的提包里!”舒克看见了靠在衣柜旁边的提包。 
        趁女人开门的机会,舒克和贝塔钻进提包。 
        猫冲进房间,在地上嗅着。它清楚老鼠就在屋里。 
        “对不起,除害嘛!”胖子冲女人笑笑。 
        “没关系,我最怕老鼠,也最讨厌老鼠。”女人不介意。 
        猫在床底下折腾了一阵,又跑出来,停在提包旁边。 
        猫围着提包绕了一圈。 
        “在提包里!”瘦子说。 
        女人的脸色变了,她怕别人开她的提包。 
        “不可能!这提包一直关着。”女人反对。 
        “打开看看?”胖子征求女人的意见。 
        “不行。”女人不干。 
        瘦子看看胖子,耸耸肩,无可奈何。 
        “还找吗?”女人问。 
        “算啦。”瘦子看了提包一眼,悻悻地说。 
        胖子抱起大猫。 
        他们走了。 
        舒克和贝塔在提包里松了一口气。 
        “也不知这包里装的是什么?”贝塔看着一个个纸包说。 
        “反正它们救了咱们。”舒克说。 
        “咱们在这里睡会儿,夜里再溜。”贝塔把身体挤进两个纸包中间。 
        “行。”舒克也选择了一个舒适的位置,睡了。 
        半夜,一阵晃动惊醒了舒克和贝塔。 
        提包被拉开了。一道刺眼的亮光射进提包里。 
        舒克和贝塔忙往里躲。 
        纸包一个一个被拿出去了。 
        跟看无处躲藏了,舒克和贝塔钻进了最后一个纸包。纸包里有一股令人窒息的气味。舒克想吐。贝塔捂着鼻子。 
        最后一个纸包被拿出提包,放在床上。 
        舒克探头一看,女人坐在床上,数着大把大把的钱。 
        提前睡觉,半夜点钱。舒克觉得这个世界被弄得充满了小家子气。 
        “你看她脸上。”贝塔小声对舒克说。 
        女人的脸上有哲学,有希望,有恐惧,有快感,有一切——当她数大把大把的钱的时候。 
        随着她的手指的移动,嘴唇的张合,舒克和贝塔对这个世界渐渐失去了信心。 
        “活着太难了。”舒克说。 
        “是。”贝塔同意。 
        他们想起了像海盗那样的同胞的霸占欲,想起了白路国王,想起了冷饮店的老板,还有小个子、胖子、瘦子…… 
        再加上这纸包里钞票的气味儿。 
        “咱们走吧。”贝塔说。 
        “去哪儿?”舒克无精打采。 
        “太空。” 
        “太空!怎么去?” 
        “我看见博览会上有宇宙飞船。”   第91集 
        贝塔开卡车去拉食物; 
        舒克检查宇宙飞船; 
        准备点火   
        舒克心里一震,乘宇宙飞船去太空?离开地球?! 
        的确,舒克和贝塔在这个星球生活得太难了。偷偷摸摸,躲躲藏藏,就因为有一个老鼠的外表。谁都可以正大光明名正言顺地欺负他们。舒克和贝塔渴望能堂堂正正地走在大街上,梦想能像其他动物一样同人类交往。 
        他们清楚这是不可能的。他们对在地球生活失去了信心。 
        “去太空!”舒克决定了,尽管他有些舍不得地球。地球是他的故乡,不能因为他是老鼠就剥夺了他有故乡的权利。 
        “怎么出去?”贝塔问。 
        “大摇大摆出去。现在她准不会大喊大叫。”舒克断定她数钱的时候不会叫。 
        正当女人准备数最后一包钱的时候,从纸包里钻出两只小老鼠。 
        她想喊,但自己用手捂住了自己的嘴。 
        两只老鼠当着她的面大摇大摆地出了屋子。 
        “咱们得谢谢钱。”贝塔在走廊里说。 
        “要是全世界的人每分钟都数钱,咱们就可以平安无事了。”舒克深有感触地说。 
        “快走,去找宇宙飞船。”贝塔说。 
        舒克和贝塔来到展览大厅,大厅里陈列着许多玩具。 
        一艘宇宙飞船醒目地耸立在大厅中央。 
        舒克和贝塔顺着扶梯爬上宇宙飞船,舒克打开舱门,钻进去。贝塔站在梯子上放哨。 
        宇宙飞船的座舱和直升机的大不一样。舒克凭自己的飞行经验判断着宇宙飞船的驾驶系统。 
        “怎么样,能开走吗?”贝塔把头伸进舱里问。 
        “没有问题。”舒克说。 
        “现在起飞?”贝塔迫不及待,他一刻也不想在地球上呆了。 
        “咱们到太空吃什么?”舒克想到食物问题。 
        “对,得弄足了食物。”贝塔拍拍脑袋。 
        “你去找食物,我在这儿熟悉一下操纵系统。”舒克吩咐。 
        贝塔顺着梯子下去,他找了一辆电动卡车,开着弄食物去了。 
        舒克检查了一遍宇宙飞船所有的舱,他对它的性能和设施很满意。 
        舒克把头伸出舱门,往上看。大厅的天花板是由玻璃组成的,必须打开玻璃窗,宇宙飞船才能发射出去。 
        贝塔的卡车拉着满满一车食物停在宇宙飞船旁。 
        “你真行。”舒克表扬贝塔。 
        “你在干什么?”贝塔抬头看舒克。 
        “不把上边打开,咱们出不去。”舒克指指天花板。 
        “玻璃挡不住宇宙飞船。”贝塔说。 
        “这倒是。”舒克一时想不出打开天花板的办法。 
        他们把食物运进宇宙飞船的储备舱。 
        贝塔也对宇宙飞船表示满意。 
        一切准备工作就绪。 
        “发射宇宙飞船不是每次都成功。”贝塔说。 
        “但愿咱们运气好。”舒克系安全带。 
        “就要离开地球了。”贝塔跟睛湿润了。 
        “……”舒克说不出话来,他的嗓子里像有一块东西堵着声道。 
        地球不容他们。 
         沉默了5分钟。 
        “点火啦?”舒克颤抖着声音说。 
        “点……点吧……”贝塔闭上眼睛。   第92集 
        舒克和贝塔飞向太空; 
        一顿神奇的饭; 
        不明飞行物出现   
        宇宙飞船点火了。 
        舒克和贝塔屏住呼吸,他们期望发射成功。 
        只听“轰”的一声巨响,宇宙飞船撞碎了展览大厅天花板上的玻璃,冲向夜空。 
        舒克和贝塔松了口气。 
        “和臭球他们告别。”舒克示意贝塔接通电台。 
        贝塔调整着频率。 
        “臭球,臭球,我是贝塔,我是贝塔,请回答!”贝塔呼叫。 
        “我是值班员,请等一下。”对方回答。 
        “请抓紧时间!”贝塔说。他清楚,宇宙飞船穿过大气层后,可能就通不上话了。 
        “我是臭球,我是臭球,请讲话!” 
        “我是贝塔。我和舒克现在驾驶宇宙飞船正离开地球,前往太空。” 
        “离开地球?”臭球吃了一惊。 
        “我们祝你们走运。请照顾好舒克的妈妈。请转告皮皮鲁。”贝塔说。 
        “为什么?为什么离开地球?还回来吗?”臭球的回话声越来越小。 
        “再见了,臭球!”贝塔哽咽了。 
        舒克咬着嘴唇。 
        宇宙飞船穿过大气层.进入了轨道,开始围绕着地球飞行。 
        “太空真美!”舒克从圆窗口往外看。 
        “宇宙太神秘了。”贝塔出神地说。 
        “现在不用驾驶了,宇宙飞船自己飞行了。咱们吃点儿东西吧。”舒克开始解安全带。 
        安全带一解开,舒克的身体就飘了起来。 
        “哎呀,失重!”舒克突然想起来了。 
        “真逗。”贝塔也飘起来。 
        “没有地球吸引力了。”舒克调整着身体飘向储备舱。 
        “等等我。”贝塔喊。 
        他们好不容易从储备舱里抓住了满天飞的食物,又好不容易飞回了驾驶舱。 
        “快,帮忙把我捆在椅子上。”舒克一边笑~边向贝塔求援。 
        “我捆上你,我怎么办?”贝塔也笑得喘不过气来。 
        舒克手中的一袋花生米撒了,花生米上下飞舞着。 
        贝塔和舒克张开嘴,追着花生米吃。 
        贝塔索性把手里的食物都扔了,舒克也扔了。他们同食物一起飘飞,抓住机会就吃。 
        舒克和贝塔玩得开心。太空里没人干涉他们,他们不用担惊受怕,可以尽情地大声喊叫,宣示生命的存在。 
        几天过去了,新鲜劲儿没有了。舒克和小塔开始感到寂寞。 
        “说点儿什么,”舒克说。 
        “话都说得差不多了。”贝塔说。 
        “这没有交往的滋味儿也挺难受。”舒克扒着圆窗往外看。 
        “被人歧视也是一种享受。”贝塔像是哲学家。 
        “快看,那是什么?”舒克叫起来。 
        贝塔往外一看,一个庞大的飞行物向他们的宇宙飞船飞过来。 
        “注意观察。”舒克打开操纵系统,准备应付突发事件。 
        不明飞行物继续向宇宙飞船靠近,显然它已经发现了舒克和贝塔的飞船。 
        “是什么东西?”舒克问贝塔。 
        贝塔揉揉眼睛。 
        “是宇宙飞船!人类的宇宙飞船!”贝塔大喊。 
        “摆脱它!”舒克说完开始操纵小宇宙飞船躲开大宇宙飞船。 
        晚了。   第93集 
        勇敢号字宙飞船发现“外星人”; 
        舒克和贝塔轰动地球   
        人类发往太空的勇敢号载人宇宙飞船在太空轨道运行的第49天发现了一架微型飞行器。 
        宇航员们争先恐后从荧光屏上看这神秘的飞行器。 
        “这么小!”一位宇航员叫道。 
        “请示地面!”机长命令负责通讯的宇航员。 
        “我是勇敢号,我是勇敢号。我们在太空发现不明飞行物。可能是外星生物,请指示。”宇航员向地球报告。 
        “设法接近它,争取将它带回地球。”地球指示勇敢号。 
        “明白。” 
        勇敢号做好了应付一切突然事变的准备后,开始小心翼翼地按近不明飞行物。 
        不明飞行物试图摆脱勇敢号的纠缠,但没有成功。 
        “贝塔,咱们的飞船进了它的肚子。”舒克无可奈何地说。 
        舒克和贝塔的宇宙飞船被勇敢号宇宙飞船“吸”进自己的舱内。 
        宇航员们看着这微型宇宙飞船,惊讶至极。 
        “我们和不明飞行物‘对接’成功!”勇敢号向地球报告。 
        “立即查明对方来历。”地球下达指令。 
        机长小心翼翼地抓住了微型宇宙飞船,他从圆窗往舱里看。 
        “外星人!”机长大叫。 
        舱内一片震惊。 
        宇航员们抢着先睹为快。 
        电波迅速将这一消息传给地球。地球在一瞬间目瞪口呆。 
        “立即转播实况!”地球清醒过来后迫不及待。要知道,她一直在寻找宇宙中的外星生命,地球太孤独了。 
        刹那问,整个地球都知道舒克和贝塔的光辉形象了。地球人类从电视荧光屏里一睹外星人的风采。 
        播音员说:“这两个外星人很像我们地球上的老鼠,当然他们和我们的老鼠有着质的区别,外形上也有很大不同。他们是高等动物,智慧生物,能驾驭现代科学技术。据悉,他们的服装的质地非常现代化,我们地球上还没有这种纺织品。据一位专家初步分析,可能是原子服装或超导服装。至于他们来自哪个星球,还有待于进一步考证。” 
        地球沸腾了。 
        播音员最后说:“告诉观众一个好消息,勇敢号宇宙飞船将把外星人带回地球。我们地球将以热情的姿态欢迎第一批光临地球的外星人!” 
        地球忙碌起来,打扫卫生,抢建宾馆;全世界的上千名字宙学家、历史学家、考古学家、心理学家、语言学家、教育学家、遗传学家……云集外星人将下榻的宾馆,准备研究外星人。同行是冤家,同行的科学家们剑拔弩张,要一决雌雄。 
        舒克和贝塔原以为等待他们的是歧视和侮辱,他们做梦也没想到自己到太空转了一圈,从根本上改变了自己的位置。 
        当勇敢号宇宙飞船带着舒克和贝塔返回地球时,这天成了地球的盛大节日。全球放假,几十亿双眼睛死盯着电视机。几亿根电视天线死咬着电视信号。 
        迎接他们的是鲜花和微笑,摄像机和照相机。   第94集 
        舒克和贝塔大出风头: 
        贝塔吐了一次舌头; 
        自有电视以来最昂贵的广告收费   
        舒克和贝塔一出机舱就被记者包围了。 
        上百名全副武装荷枪实弹的警察保护着舒克和贝塔的安全。 
        “怎么了?”贝塔小声问舒克。 
        舒克耸耸肩。 
        “他们交谈了!”电视播音员说,“据语言学家说,他们使用的是一种最简洁的语言,只有主语,没有谓语,没有形容词。这样可以节省大量时间。由此可见外星人是讲效率的。” 
        全球的观众十分羡慕外星人。 
        “把咱们当外星人了。天外来客!”舒克恍然大悟,对贝塔说。 
        “天哪!”贝塔吐了下舌头。 
        播音员对全球观众讲解:  “请大家注意!外星人吐舌头比我们有水平,比我们自然!时机也比我们掌握很好!请大家再看一次外星人吐舌头的慢镜头。瞧,分寸感多强!好,现在我们请著名的表情专家给我们分析一下外星人对吐舌头这一表情的运用。” 
        全球的人类跟着电视向舒克和贝塔学吐舌头。 
        舒克和贝塔乘坐超豪华小轿车在几百辆摩托车的护卫下驶向宾馆。 
        沿途的群众向车队抛掷鲜花和彩带。 
        舒克发现了几张熟悉的面孔,他仔细一看,是小个子,胖子和瘦子。他们起劲儿地扔鲜花和喊口号。 
        舒克和贝塔得意了,他们终于受到了人类的尊重和承认,他们决定享受这一现实。 
        舒克和贝塔分别向窗外招手致意。 
        电视播音员激动了: 
        “外星人终于向我们表态了!他们不沉默了!他们承认我们星球了!!!” 
        被承认的地球自豪了。地球的几千年文明发展被承认了!没白干。人类掉泪了。 
        在宾馆,当地最高首脑拜见了舒克和贝塔,他代表市民将本城的金钥匙赠送给外星人,并请舒克和贝塔担任该市的荣誉市民。 
        “我们非常荣幸。”贝塔张口说话了。 
        电视播音员几乎足喊起来: 
        “他们已经掌握了我们的语言!现在请著名语言学家白活博士分析!” 
        白活博士出现在荧光屏上。 
        “外星人只用了半个小时就掌握了我们的语言,可见他们的大脑结构比我们先进,比我们发达,比我们合理。他们的听觉和声带都是宇宙第一流的。我建议,应该让外星人帮我们改良人种,优生优育。”白活博士侃侃而谈。 
        立即有上千名妇女报名。 
        “能够来到地球,我们很高兴!”舒克也发话了,他不能让贝塔一人出风头,“地球比我们想像得要好一些。” 
        整个地球安静极了,洗耳恭听天外来客的教诲。 
        “请问你们的食物是什么?”一位负责外星人食宿的官员问。 
        “在我们的星球上,主要靠油炸花生米。当然,我们不叫花生米,叫巨豆。”贝塔回答。 
        所有商店出售花生米的牌子都更名为巨豆。 
        花生米立即在全球脱销。 
        “请问贵星球生命的历史?”一位记者问。 
        “比地球生命早四倍时间。”舒克说。 
        全球肃然起敬。 
        在电视实况转播采访外星人期间,播了一秒钟广告。该广告收费占全球生产总值的百分之四十。   第95集 
        有一个国家改名为舒克国; 
        舒克和贝塔觉得出名的滋味不好受; 
        皮皮鲁上主席台   
        从此,舒克和贝塔成了地球瞩目的中心。人类敬重他们,羡慕他们,崇拜他们。 
        研究舒克和贝塔的上万篇论文争先恐后地问世。根据舒克和贝塔为原型创作的电影、电视剧、小说、报告文学、连环画、戏剧充斥了影尉坛、文坛…… 
        舒克和贝塔住在豪华的饭店里,吃着山珍海味,享受着一流的服侍。 
        以舒克和贝塔的名字命名的街道、城市比比皆是,甚至有一个国家要改名为舒克国。 
        没有歧视,没有饥饿,不用为生存发愁,但舒克和贝塔觉得很累。 
        “有意思吗?”一天晚上,舒克问贝塔。 
        “够烦的。”贝塔说。 
        “我觉得恶心。咱们还是咱们,就因为当了外星人,待遇全变了。”舒克撇撇嘴,“我看人类也就那么回事,没一个人发现咱们和地球上的老鼠一模一样。” 
        “真可怜。”贝塔打心眼儿里同情人类。 
        “我看出名是最痛苦的事了。”舒克说。 
        “没错,出了名老得端着架子活。老想着怎么才能不辜负自己的名,活活能把人累死。这些天我自己都没了。” 
        “咱们走吧!” 
        “去哪儿?” 
        “不知道。”舒克为难了。的确,地球上是没法呆了,受尊重和受歧视都不好受,太空又太寂寞,唉,活着真难。 
        “咱们去找真正的外星人。”贝塔灵机一动。 
        “这主意不错。”舒克兴奋了。 
        “什么时候走?”贝塔问。 
        “咱们去看看皮皮鲁,还有臭球他们,然后就走。”舒克说。 
        “看皮皮鲁可以,看臭球会给他们带来危险。你想,咱们不能自由行动呀!走到哪儿都有人跟着,还有照相机和摄像机。”贝塔反对去见臭球他们。 
        舒克认为贝塔的话有道理。 
        “明天去见皮皮鲁。”舒克说。 
        “祝你做个好梦。”贝塔临睡前说。 
        “晚安。”舒克没说完就睡着了。 
        皮皮鲁早就从电视荧光屏上认出了舒克和贝塔,他要求见舒克和贝塔,但被有关方面拒绝了,理由是皮皮鲁的地位太低,现在只有国家元首级的人才能有幸亲眼见外星人。 
        这天,皮皮鲁正在上课,只见校长神情紧张而激动地把班主任从教室里叫出去了。 
        不一会儿,班主任叫皮皮鲁出去。 
        “又倒楣了。”皮皮鲁想。 
        “皮皮鲁,祝贺你!”校长伸出手来。 
        “……”皮皮鲁茫然。 
        “上边来通知,说外星人要见你!”校长满面春风。 
        皮皮鲁一蹦老高。 
        “中午1点接见,现在学校给你开欢送会。”校长不觉得恶心,也不知开哪门子欢送会,见外星人就在本市。 
        皮皮鲁终于坐上了学校会场的主席台。 
        欢送会后,皮皮鲁乘专车前往舒克和贝塔下榻的宾馆。 
        有关方而对外星人提出要见一个小学生感到吃惊,而且指名道姓。专家们更感到外星人具有遥感能力,能预知一切。 
        皮皮鲁和舒克、贝塔见面了,他们都很激动。在外星人的要求下,工作人员和记者都退下去丁。 
        舒克小声将事情的经过讲给皮皮鲁听。 
        皮皮鲁手舞足蹈。 
        “我们是来向你告别的。”贝塔说。 
        “告别?”皮皮鲁愣。   第96集 
        “外星人”离开地球; 
        一个布满大石头的星球; 
        食物告急   
        舒克把原因告诉皮皮鲁。 
        “那就走吧。”皮皮鲁像大人似地点点头。 
        “我们以后还会来看你。”贝塔说。 
        “我等着。”皮皮鲁相信。 
        会见结束。 
        贝塔按电铃叫工作人员进来。 
        皮皮鲁回学校参加学校为他召开的欢迎会去了。 
        舒克和贝塔当着记者宣布,他们将于明天返回自己的星球。 
        这一爆炸新闻立即传遍了全球。 
        地球忙起来了,它要用最隆重的仪式欢送外星人。各种欢送方案被送到“欢送外星人筹备委员会”,计算机辨别筛选最佳方案。 
        勇敢号宇宙飞船承担了将外星人送回太空的历史性重任。机组成员早已成为家喻户晓的传奇英雄。 
        舒克和贝塔就要离开地球了。他们回到地球时间不长,因研究他俩取得成果而晋升教授的就有五万人。因推销和他俩有关的商品而发财晋升为百万富翁的商人不下十万人。 
        人类决定将舒克和贝塔离开地球的这一天定为今后每年的“国际外星人日”,年年纪念。 
        勇敢号宇宙飞船就要起飞了,舒克和贝塔通过荧光屏看见了全世界欢送他俩的场面。 
        一想到整个地球在欢送两只货真价实的老鼠.舒克和贝塔就想笑。 
        勇敢号宇宙飞船点火了。 
        舒克和贝塔心里挺不是滋味儿。 
        地球越来越小。 
        勇敢号飞出了大气层,来到太空。 
        机长来到舒克和贝塔身边,恭敬地请示:“什么时间离机?” 
        “现在。”舒克说。 
        “祝你们顺利。”机长说。 
        “谢谢。”贝塔说。 
        舒克和贝塔的宇宙飞船离开了勇敢号,来到广袤无垠的太空。 
       “自由了!”贝塔大喊起来。 
        “咱们要摆脱地球吸引力,直飞外星。”舒克说。 
        “看你的了!”贝塔知道这需要很高的速度。 
        舒克全神贯注地驾驶宇宙飞船。 
        贝塔盯着仪表盘。 
        “再快一点儿!再快一点儿!还差一点儿!”贝塔给舒克加油。 
        宇宙飞船终于摆脱了地球的吸引力,像脱缰的野马,自由地向太空飞去。 
        舒克和贝塔心旷神怡。 
        “也不知哪颗星球上有生命。”舒克望着太空里众多的星球说。 
        “挨个找呗。”贝塔趴在圆窗上往外看。 
        “那儿有一颗。”舒克说。 
        “挺大,落上去看看。”贝塔一拍腿。 
        舒克驾驶宇宙飞船朝那颗星球逼近。 
        宇宙飞船在星球上着陆了。 
        “我出去看看。”贝塔离开飞船。 
        星球上全是石头,没有植物。更没有动物。 
        贝塔回到飞船上。 
        “没有。”贝塔耸耸肩膀。 
        “够荒凉的。”舒克往舱外望去。 
        “上帝也偏心,同样是星球。”贝塔想起地球上的景象。 
        “没有生命也不亏。”舒克说。 
        “那倒是。”贝塔赞同。 
        “再接着找。”舒克按动了起飞按钮。 
        宇宙飞船继续在太空寻找外星人。 
        舒克和贝塔又陆续在20多个星球上着陆,都没有生命。他们的食物不多了。   第97集 
        终于找到有生命的星球; 
        外星人见到外星人不吃惊; 
        外星人送给舒克和贝塔翻译机   
        “现在离地球很远了吧?”舒克问贝塔。 
        “太远了。”贝塔望望窗外。 
        “咱们的食物还够吃几天?” 
        “如果省着吃,还能吃三四天。” 
        “你不后悔吧?” 
        “不。你呢?” 
        “也不。” 
        “我去清点一下食物。” 
        贝塔来到储藏室,把剩下的一点儿食物清理了一下,还有两根香肠,一筒罐头,一包压缩饼干。 
        “贝塔,快来,又有一颗星球!”从驾驶舱传来舒克的喊声。 
        贝塔对新星球已没有兴趣了。他坚信根本没有外星人,宇宙里只有地球上有生命。 
        “快来!”舒克又叫。 
        ‘犬惊小怪。”贝塔边说边来到驾驶舱。 
        “快看!”舒克激动。 
        贝塔往窗外一看,一颗布满亮光的星球充满生机地出现在宁宙飞船的左侧。 
        “真漂亮。”贝塔情不白禁地喊。 
        “准备着陆。”舒克发令。 
        贝塔系好安全带,戴上耳机。 
        宇宙飞船靠近那颗美丽的星球。距离越来越近。 
        “有房子!”贝塔把安全带挣断了。 
        “扶好,我着陆了。”舒克看准了一片空地,驾驶宇宙飞船着陆。 
        宇宙飞船自信地着陆。 
        “他们该隆重欢迎我们。”贝塔预测。他想起地球人欢迎他们的情景。 
        “那可真够烦的。”舒克解开安全带。 
        字宙飞船四周是造型奇物的建筑,没有相同的房子。绿树环绕,碧水点缀,像仙镜。 
        舒克和贝塔从宇宙飞船里钻出来,他们伸伸胳膊踢踢腿,等待着外星人的出现。 
        “来了!”贝塔往左边指指。 
        两个外星人走过来,他们和地球人长得差不多,就是耳朵特别特别小。 
        他们看见舒克和贝塔没有表现出吃惊。他们像什么事也没发生那样,继续走路。 
        “请问,这是什么星球?”贝塔看外星人不理睬他们,追上去问。 
        外星人站住了,其中一个说: 
        “巴九红七蒲定川。” 
        “语言不通!”舒克对贝塔说。 
        贝塔冲外星人摇摇头,指指自己的耳朵。 
        外星人马上明白了舒克和贝塔听不懂他们的话。其中一个外星人掏出一个小仪器,调了调旋钮,他示意贝塔再说几句话。 
        贝塔又说了几句。 
        外星人凋旋钮。 
        “行了,可以对话了。”外星人说。声音是从小仪器里传出发来的。看来小仪器是一台翻译机。 
        “你们从哪儿来?”外星人同。 
        “地球。”贝塔回答。 
        “愿意在我们这里住住吗?”外星人问。 
        “愿意。”舒克说。 
        “这些房子都可以住,随便挑吧!”外星人说,“这个翻译机送给你们,再见。” 
        外星人走了。 
        “他们听说咱们从地球来一点儿也不吃惊。”贝塔感到有点遗憾。 
        “他们不因为咱们是外星人就抬高咱们的身价,和地球人不一样。”舒克说。 
        “走,咱们去那座房子看看。”贝塔指指飞船旁一座房子。 
      房子里没人,设备很好,房门上的一块牌子上写着:无人住。舒克把牌子翻过来,背面写着:有人住。 
        “咱们就住这儿吧?”舒克问贝塔。 
        “行。”贝塔同意。   第98集 
        双子星球上所有动物都有生存的权利; 
        舒克和贝塔望着窗外发呆; 
        返回太空   
        舒克和贝塔在小房子里定居下来,他们给小房子起名为宇宙公寓。 
        舒克和贝塔给他们的宇宙飞船搭了个棚子,以防风吹日晒。 
        宇宙公寓里有彩色电视机,电话,还有食物传送柜。你需要什么食物,只要按不同的按钮,食物就会从传送柜里送到你手中。 
        舒克洗了个痛快澡。贝塔顾不上洗澡,他被电视节目吸引了。 
        “这个星球叫双子星球,离地球远极了。”贝塔把他刚从电视上获得的信息告诉舒克。 
        “咱们吃完饭出去走走。”舒克穿着浴衣说。 
        “车库里有汽车。”贝塔说。 
        舒克和贝塔吃完饭,开车去外边转转。 
        双子星球给人以和缓的感觉。 
        双子星球人类的耳朵特别小,舒克和贝塔一打听,才知道双子星球上的人不在乎别人怎么说他,自己干自己的事,不理会别人对他的看法。时间长了,耳朵就退化了。 
        “地球上的人太重视别人对他的看法。”贝塔说。 
        “活得太累。”贝塔觉得身上最多余的器官就是耳朵。耳朵累人。 
        舒克想起自己为了争个好名声,抛弃了妈妈驾飞机出走的经历,他认定地球上的动物不如双子星球上的动物智商高。 
        双子星球上的所有动物都过着安适的生活,包括老鼠和苍蝇。双子星球认为所有来到这个星球上的动物都有平等的生存权利,苍蝇经过几代的消毒措施已经无病菌,老鼠有食物也不会去偷东西了。 
        舒克和贝塔喜欢双子星球,他们决定在这里永久定居,过舒心的生活。 
        他们结交了许多新朋友,有不少新经历。双子星球的食物是抗衰老的,该星球动物的平均寿命是八百岁。舒克和贝塔愉快地生活着。 
        30天过去了。 
        一天早晨,舒克起床后望着窗外发呆。 
        “你怎么了?”贝塔走到舒克身边。 
        “我想家了。”舒克说。 
        “我也是。”贝塔这几天心里感到怅然若失。 
        很怪,生活在这么好的地方,还要想地球。就因为地球是故乡。 
        “看来咱俩呆在哪儿也不行。”贝塔给自己和舒克下了个结论。 
        “说不定是不习惯,再住住看。”舒克确实不愿意再回去遭受歧视。 
        第二天,他们又站在窗前发呆。 
        他们这才知道还有比自尊更重要的东西。 
        居住在不属于自己的地方,生活得再好.也不会幸福。 
        “回吧?”舒克问贝塔, 
        “嗯。”贝塔点点头。 
        舒克和贝塔开始做返回地球的准备工作,他们拆除了宇宙飞船的棚子,往飞船里搬运食物。 
        发射飞船的时问定在下午两点整。 
        舒克将宇宙公寓门口的小牌子翻过来。 
        他们恋恋不舍地向双子星球告别。舒克和贝塔的朋友来为他们送行。 
        宇宙飞船点火了。 
        舒克和贝塔又回到了太空,他们驾驶飞船向地球驶去。   第99集 
        地球变了; 
        小老鼠的爷爷的爷爷的故事; 
        地球出了什么事   
        舒克和贝塔的宇宙飞船经过一段时间的航行,接近地球了。 
        “这回可别再被人类的宇宙飞船发现了。”贝塔提醒舒克。 
        “看咱们的运气了。”舒克说。 
        贝塔向太空了望。 
        宇宙飞船即将穿越大气层。 
        “注意!发现飞行物!”贝塔喊道。 
        舒克往外一看,一架飞行器正从右后侧接近宇宙飞船。 
        “摆脱它!”贝塔不想再忍受隆重欢迎。 
        舒克驾驶飞船变了个方向,那飞行器没有追过来。 
        贝塔松了口气。 
        “大概是卫星。”舒克说。 
        “谢天谢地。”贝塔说。 
        宇宙飞船穿过大气层。舒克和贝塔渐渐看见地球上的景物了。 
        “又回来了!”贝塔感到亲切。 
        舒克聚精会神地寻找着陆地点。 
        “地球好像变了!”舒克惊奇地发现。 
        贝塔往下看,的确,地球变了,树木变了,汽车的形状变了。 
        “才走了几十天,变化这么大?”贝塔不相信。 
        “下边好像是臭球的机场的位置,我着陆了。”舒克操纵宇宙飞船下降高度。 
        宇宙飞船平稳地下降。贝塔拿起话筒。 
        “臭球!臭球!我是贝塔,请回答!” 
        “……” 
        “臭球!臭球!我是贝塔,请回答!” 
        “……” 
        贝塔看看舒克,他预感到发生了什么事。 
        飞船降落在一片草丛里。 
        “这地方还挺熟悉,记得吗?”舒克下飞船后看看四周,对贝塔说。 
        “嗯,记得。”贝塔记得这里离他们的机场不远。 
        舒克和贝塔把宇宙飞船藏好,他俩朝机场走去。 
        哪里还有什么机场,只有空空的一片开阔地。 
        舒克和贝塔愣了。 
        “找错地方了?”舒克往好里想。 
        贝塔发现了地上的一个东西,他捡起来。 
        “皮皮鲁号运输机上的天线!”贝塔和舒克异口同声。 
        从前这里就是机场,可机场呢? 
        旁边的草丛里一阵颤动。 
        “谁?”舒克问。 
        “是我。”一只小老鼠钻出来。 
        “请问这里的机场呢?”舒克问。 
        “机场?”小老鼠纳闷。 
        “就是舒克贝塔航空公司。”贝塔迫不及待。 
        “航空公司?”小老鼠摸不着头脑。 
        舒克看看贝塔,贝塔看看舒克。 
        “你是刚来这里生活?路过?”舒克不信这一带的小动物不知道他的航空公司。 
        “我从一生下来就在这里。”小老鼠说。 
        “你从没见过或听过有许多小老鼠开飞机的事?”贝塔问。 
        “慢着,我想起来了,我爷爷给我讲过这事。他说原先这里有一座老鼠机场,不过那是古代的事情了。”小老鼠拍拍头。 
        “古代?”舒克和贝塔几乎蹦起来。 
        “我爷爷还是听我爷爷的爷爷说的呢!”小老鼠比划着。 
        “咱们才离开地球几十天呀!”舒克弄不清地球出了什么事儿。   第100集 
        三十年过去了; 
        臭球留下的信; 
        舒克和贝塔决定去看皮皮鲁   
        经过一阵惊愕,舒克和贝塔冷静下来。 
        “我从前好像听说过,天上一天,地上一年。咱们去的双子星球上的一天,说不定就是地球上的一年。”舒克说。 
        “准是这样。”贝塔断定。 
        在舒克和贝塔离开地球的这30多天中,地球度过了30年的漫长岁月。 
        “我带你们去见我爷爷,他知道得多一些,听说他还有传家宝呢。”小老鼠说。 
        “去看看。”舒克想多知道些关于机场的事。 
        小老鼠领着舒克和贝塔走进草丛,钻进一个地洞。 
        舒克和贝塔对地洞已经十分陌生了,贝塔还下意识地捂了一下鼻子,他受不了地洞里的潮气。 
        小老鼠的爷爷有气无力地躺着。舒克想起了自己的妈妈。 
        “我叫舒克,他叫贝塔。听说您知道关子机场的事,我们想听昕。”舒克说。 
        “舒克?!贝塔?!”老鼠爷爷激动地坐起来。 
        “你知道我们?”贝塔问。 
        “我的祖宗是臭球,他临去世前留下一封信,让后代转交给舒克和贝塔。这封信由我爷爷的爷爷的爷爷的爷爷交给我爷爷的爷爷的爷爷,又由我爷爷的……一直传到我手中。”老鼠爷爷说。 
        “我们就是舒克、贝塔,快把信给我们看看!”舒克急不可待。 
        “可你们……岁数比我还小呀!”老鼠爷爷不信。 
        “我们比您岁数大多了!”贝塔把去太空的事讲给他听。 
        老鼠爷爷从地底下挖出一个瓶子,从瓶子里拿出一封信。 
        舒克用颤抖的手打开这封30年前臭球写给他和贝塔的信。 
        亲爱的舒克和贝塔: 
        自从你们去太空后,机场一直很兴旺。 
        可就在一个月前,灾难从天而降。在一个夜晚,正当我们休息的时候,一支灭鼠队扫荡了机场。幸存者只有我。我对不起你们,把你们交给我的机场毁了,你们骂我吧! 
        还有一辆摩托车是完好的,留给你们,由我的儿子转交。祝你们好运。 
        臭球临终前 
        舒克把信递给贝塔,他看着墙,发呆。 
        贝塔看完信,一拳砸在地上。 
        他们的跟前出现了灭鼠队围歼机场上的老鼠的情最。飞机被毁坏,建筑被踏平。舒克想到妈妈。贝塔想到罗丘、荷叶、松果、头版…… 
        “摩托车在哪儿?”舒克问老鼠爷爷。 
        “是那箱东西吗?在外边的地下埋着。”老鼠爷爷说,“我带你们去。” 
        舒克和贝塔扶起臭球的后代,一同去找臭球留下的遗物。 
        埋藏在地下的箱子被挖出来了。 
        舒克和贝塔用工具打开箱子,一辆用塑料布包着的摩托车呈现在他们眼前。这是当年机场惟一的遗物。 
        贝塔掉泪了。 
        “咱们去看看皮皮鲁。”舒克拍拍贝塔的肩膀,强打精神地说。 
        “皮皮鲁有40多岁了吧?”贝塔擦干眼泪说。 
        “差不多。”舒克说。 
        舒克检查一遍摩托车.保存得很好。 
        “我记得你在机场开过摩托车,你发动一下试试。”舒克想骑摩托车去找皮皮鲁。 
        贝塔发动30年前的摩托车一次成功。 
        摩托车的排气管“突、突、突”地冒着热气,把积蓄了30年的寂寞一吐为快。 
        “你带着我,咱们去找皮皮鲁。”舒克看看天色,已近黄昏。 
        贝塔的性格好像突然之间变了,他沉默地跨上摩托车。舒克坐在后座上。 
        “再见,谢谢你们!”舒克深情地看看臭球的后代。 
        “再见!”臭球的后代终于完成了祖宗留下的遗愿。 
        贝塔驾驶着摩托车上路了。 
        夜色渐渐笼罩丁大地,月亮将它的光吝啬地投向原野。 
         舒克和贝塔辨认着道路。从前,他们是从空中去皮皮鲁家的。 
        “看,我原先居住的房子!”贝塔认出了他出生的地方,他和咪丽打仗的地方。贝塔想起他开着坦克出走的情景。 
        “停车去看看?”舒克建议。 
        “不。”贝塔继续开车,他也不说为什么。 
        摩托车直驶皮皮鲁家。 
        舒克和贝塔觉得自己长大了。一路上,他们想了许多,也什么都没想。 
        摩托车进入了城市,街道上行人稀少。 
        “看,钟楼!记得咱们为皮皮鲁拨表吗?”贝塔说。 
        “当然!皮皮鲁家在钟楼的东南方向。”舒克说。 
        摩托车停在皮皮鲁家的单元门口。舒克和贝塔攀着排水管爬上皮皮鲁家的阳台。 
        纱窗上的小窟窿还在! 
        “皮皮鲁还在这儿住!”舒克断定。 
        舒克和贝塔钻进屋里,他俩看见沙发上坐着一个中年男人,正在看电视。 
        “是皮皮鲁吗?”贝塔小声问舒克。 
        “不像。”舒克说。 
        “神态有刖有点儿像。”贝塔端详着。 
        “做好跑的准备,我冒一次险。”舒克说。 
        贝塔把退路看好。 
        舒克小心翼翼走到中年男子的脚下,他拽拽他的裤脚。 
        中年男子低头一看,愣了一下,好像是从大脑的记忆细胞里搜寻着什么。 
        见到老鼠而不吃惊,舒克断定他就是皮皮鲁! 
        “我是舒克!”舒克大叫。 
        “舒克?”中年男子一跃而起,好像变成了小孩子。 
        “你是皮皮鲁?”舒克问。 
        “正是!我是皮皮鲁!你还活着?都几十年啦!!!”皮皮鲁又激动又觉得不可思议。 
        贝塔从沙发后面跑过来。 
        ‘贝塔!!!”皮皮鲁大喊。 
        皮皮鲁把舒克和贝塔捧在手掌上,他看着自己童年时的朋友,眼睛湿润了。 
        舒克和贝塔把分别后的经历讲给皮皮鲁听。皮皮鲁也把自己这30年来走过的路告诉朋友。皮皮鲁现在是著名的物理学家,去年曾获得了诺贝尔奖的提名。 
        舒克和贝塔高兴极了,这是他们回地球后惟一高兴的事。 
        “祝贺你!”舒克和贝塔异口同声。 
        “谢谢。”皮皮鲁脸上有一丝愁云。 
        “你有不顺心的事?”贝塔问。 
        皮皮鲁点点头。 
        “在我的童年,整天就是上学,写作业,考试,根本没时间玩。幸亏认识了你们.要不然,我就没有童年了。”皮皮鲁说。 
        “可你换来了今天呀!你成名了呀!!”舒克提醒皮皮鲁。 
        “长大成就再辉煌,没有童年的人生也是不完整的人生。”皮皮鲁叹了口气。 
        舒克和贝塔不吭声了。 
        “你们看。”皮皮鲁指指书柜。 
        书柜里停放着一架绿色的直升机和一辆米黄色的坦克。 
        “这两样东西,我在书柜里放了30年。”皮皮鲁说,“现在送给你们。” 
        舒克和贝塔的心脏有力地跳动着,他们喜欢生命,尽管生命本身就是痛苦和欢乐的结合体,他们还是喜欢。 
        舒克和贝塔驾驶着直升机和坦克在屋里给皮皮鲁作飞行和行车表演。他们想补上皮皮鲁的童年,使他有完整的人生。 
        第二天,舒克和贝塔向皮皮鲁告别。他们要去海上的孤岛看看海盗。他们想,也许孤岛上的一天也等于陆地上的一年呢。反正他们非常想见从前同他们打过交道的人,不管是朋友还是敌人。 
        何况随着时间的推移,世界上没有敌人。 
        舒克驾驶直升机吊着贝塔的坦克,飞出了皮皮鲁家。 
        等待他们的,是新的一天。   第101集 
        舒克回忆《老鼠报》; 
        阴森的洞口; 
        贝塔不要手枪   
        舒克和贝塔同皮皮鲁告别后,去孤岛看海盗。 
        事隔三十年,舒克终于又驾驶直升机在天空飞行,他感到激动。 
        贝塔趴在窗口往外看,地球的变化很大。 
        “是皮皮鲁这一代人的功劳。”贝塔感慨地说。 
        “没错。”舒克同意,“我都找不着熟悉的地面标志了。” 
        舒克搜寻地面。 
        皮皮鲁送给他们的这架直升机是一架现代化的直升机,比舒克原来的那架直升机先进多了,在飞机上可以直接同皮皮鲁通电话。 
        “那儿好像是那家报社,还记得吗?”贝塔兴奋地指着飞机下边说,“咱们办过一张《老鼠报》,想起来了吗?” 
        “对。”舒克的眼睛发光了,“你还开着坦克送报纸呢!” 
        “也不知咱们老鼠家族现在生活得怎么样了,还受人类的歧视吗?”贝塔若有所思地说。 
        “恐怕还是老样子吧。”舒克清楚名声这东西不容易扭转。 
        “唉,咱们的机场如果还在就好了。”贝塔叹了口气,他有失去了家的感觉。 
        舒克的眼眶湿了,他的眼前浮现出机场被扫荡的场面。 
        直升机飞临海面上空,舒克和贝塔寻找那座孤岛。 
        “30年了,海盗准死了。”贝塔说。 
        “活的可能性不大。”舒克同意。 
        不管怎么说,他们想见海盗,不论是活的还是死的。他们就是想见从前和他们打过交道的人。特别是同胞。 
        “看,孤岛!”舒克先发现了海面上的孤岛。 
        贝塔证实了舒克的发现。 
        直升机在孤岛上空盘旋,寻找着陆点。 
        从飞机上俯瞰,孤岛仍是一座荒岛。 
        舒克选中一块稍平坦的草地着陆。 
        坦克先着陆,直升机降落在坦克旁边。 
        舒克解开安全带,贝塔打开舱门。 
        “先看看,这可是海盗呆过的地方。”舒克提醒贝塔。 
        贝塔站在舱门口向四周了望,没发现异常。 
        贝塔踏上了孤岛的土地。 
        舒克关好机舱门也离开了飞机。 
        孤岛从天上看不大,站在上面感觉却不小。 
        “咱们先往那边走。”贝塔指指直升机前边。 
        舒克点点头。 
        他们开始往前走。 
        孤岛是由礁石和草丛组成的,地形复杂,险象环生。 
        舒克和贝塔小心翼翼地前进。 
        一块大礁石横在他们面前。 
        “从旁边绕过去。”舒克说。 
        “你看这儿。”贝塔小声对舒克说。 
        舒克顺着贝塔手指的方向看,大礁石下边有个洞口。洞口前边的草被踩平了。 
        “有名堂。”舒克点点头。 
        “进去看看。”贝塔往洞口走。 
        “等等,我去把直升机开来,以防万一。”舒克说。 
        “我在这儿等着。”贝塔同意了。这洞口确实有点儿阴森。 
        两分钟后,舒克驾驶直升机吊着坦克在洞口旁边着陆了。 
        舒克从驾驶舱探出头,问贝塔:“要这个吗?” 
        舒克手里拿着一把手枪,是直升机里的武器装备。   第102集 
        晴天霹雳; 
        老鼠历史上的奇迹; 
        舒克和贝塔嫉妒海盗   
        “不要。”贝塔摇摇头。 
        舒克拿着手枪离开直升机。 
        探险开始了。 
        舒克打头,贝塔断后。他们走进洞口。 
        洞里很黑。舒克和贝塔停了一会儿,等眼睛适应了黑暗后,他们才看清这是一座城府很深的洞。 
        一进洞口是一座面积挺大的“厅”,厅的四周有三个小洞口。 
        “三室一厅啊。”贝塔幽默。 
        “我去那个洞里看看,你进中间的。剩下的那个咱们一起去。”舒克说。 
        贝塔同意。 
        “谁呀?”黑暗中传来一句问话。虽然声音平缓,但结结实实吓了舒克和贝塔一跳,犹如晴天霹雳。 
        舒克和贝塔不知该不该回答。 
        “有人吗?”第二次发问。声带明显地激动。 
        “你是谁?”舒克反问。 
        “来人啦,终于来人啦!!”那声音狂喜起来,就像见到了分别了一万年的血亲。 
        舒克和贝塔面面相觑。 
        一个身影从中间那个洞口出现了。 
        “海盗!!!”舒克和贝塔异口同声。 
        的确是海盗。 
        海盗还活着! 
        一只活了30多年的老鼠! 
        舒克和贝塔惊讶得瞠目结舌。 
        “舒克?贝塔?”海盗一愣,他无法想像面前这两个冤家过了30多年还这么年轻。 
        本来回到地球上积了一身寂寞感的舒克和贝塔,见到海盗后寂寞感一扫而光。 
        海盗更是欣喜若狂,他没想到在离开这个世界之前还能见到活物!这是他30年来日思夜想的事! 
        “你好!海盗!”舒克向30多年前的敌人伸出了手。 
        “你们好!”海盗颤抖着握住了舒克和贝塔的手,“谢谢你们还记着我。” 
        海盗变了一个人。 
        舒克和贝塔看得出海盗虽然老态龙钟,但身体硬朗,还有点儿道骨仙风。这不能不说是一个奇迹。 
        “这么多年就你自己在这个岛上?”舒克有些后悔当初对海盗狠了点了。 
        “今天是30年来我头一次见活物。”海盗老泪纵横。 
        舒克想向海盗道歉。 
        “你得感谢我们。如果你不是远离尘世,与世无争,早就不在这个世界上了。”贝塔抢在舒克前边说,他认定与世无争导致了海盗的长寿奇迹。 
        海盗承认了这个事实。 
        “那我得谢谢你。”海盗对舒克说。 
        “免了免了。”舒克不好意思。 
        “到我屋里坐吧。”海盗往卧室请昔日的两位敌人,那口气像是请亲爹。 
        时间能消除一切仇怨。 
        在时间面前,世问的一切仇恨都显得微不足道和软弱无力。 
        舒克和贝塔走进海盗的卧室,这里虽然简陋,但无污浊之气。 
        舒克和贝塔甚至有点儿嫉妒海盗的生活了。   第103集 
        在时间面前胜负不重要; 
        海盗三十年来第一次吃香肠; 
        奇怪的礁石   
        舒克和贝塔在海盗的卧室落座后,舒克问海盗: 
        “这么多年,就你自己在这孤岛上,够寂寞的吧?” 
        海盗叹了口气,说:  “第一年真难过,后来慢慢就习惯了。就像你们说的,如果不是远离尘世,我也活不了这么多年。” 
        “你就是这儿的国王。”贝塔想用幽默安慰海盗。 
        海盗苦笑了一下,说:  “现在回想起来,当初非要同你们争个高低,挺好笑。其实,活着大可不必那么事事认真。” 
        舒克和贝塔点点头,他们认为海盗的话有道理,他们想起仇恨老鼠的人类把他俩当外星人隆重接待的那次闹剧,他们认定世间的大部分事都是自欺欺人、自己和自己过不去。 
        “还记得你驾战斗机攻击我们的皮皮鲁号吗?”舒克问海盗。 
        “当然记得。”海盗的脸上焕发出光彩,“那场空战打得够劲儿。” 
        “说实话,你的飞行技术真不错。”舒克说。 
        “那还不是败在你手下了。”海盗现在已经不看重胜负了。 
        就是,在时间面前,胜负确实不值一提。 
        海盗三十年来头一次和别人说话,他显得兴奋无比,滔滔不绝,好像要在10分钟内把储存了30年的话一口气都说完。 
        他们谈舒克贝塔的机场,谈鼠王的通缉令,谈潜水艇。谁能想像他们在30年前是势不两立的仇敌呢?他们一边聊一边后悔当初不该那么敌对。 
        “你在岛上吃什么?”贝塔问海盗。 
        “吃素食。这可能也是长寿的原因。”海盗说,“夏天吃野果子。再把野果子制成酱留着冬天吃。” 
        “我的飞机上有香肠,我去拿。”舒克说完站起来。 
        香肠!!! 
        海盗的口水马上夺嘴而出。 
        舒克从直升机上给海盗拿来了香肠和其他精美的食物。 
        海盗狼吞虎咽。 
        舒克和贝塔羡慕海盗荣幸地当上了全世界吃饭吃得最香的冠军。 
        海盗抹抹嘴,吃饱了。 
        “这岛上没什么好玩的地方?”贝塔问海盗。 
        “对了,旁边那座礁石挺怪,我每次路过时都能听到里边好像有声音,可又找不到洞口。”海盗说。 
        “咱们看看去。”贝塔感兴趣地说。 
        舒克和贝塔在海盗的指引下来到那座礁石旁边。 
        这座礁石的气质不一般。舒克和贝塔一见到它,身上就紧张。 
        “你们把耳朵贴上去听听。”海盗说。 
        舒克和贝塔把耳朵贴在礁石上。 
        礁石里有一种类似于机械钟表齿轮转动的声音。 
        有着机械常识的舒克和贝塔脸上出现了惊讶的表情。 
        “定时炸弹?”贝塔第一判断。 
        “30年还没爆炸?”舒克反对这个判断。 
        “礁石里面能有什么机构?”贝塔围着礁石转了一圈,找不到裂缝什么的。   第104集 
        雷达测试礁石; 
        坦克炮击礁石; 
        神秘的洞口   
        “我去把直升机和坦克开来,飞机上可能有工具。”舒克强烈地想知道礁石里藏着什么。 
        “可以用飞机上的雷达测试。”贝塔灵机一动。 
        “你的脑子还行。”舒克冲贝塔挤眼睛。 
        海盗和贝塔留在礁石旁,舒克去开直升机。 
        贝塔仔细观察礁石的每一部分,他发现礁石上的图案很奇怪,是一种呈现出神秘的排列组合。 
        舒克驾驶着直升机吊着坦克在礁石旁着陆。 
        直升机的机头正对着礁石。 
        贝塔和海盗登上飞机。 
        海盗羡慕地看着豪华的机舱。他和舒克一样,觉得自己就是为了飞机而来到这个世界上的。 
        “我用雷达测一下。”舒克边说边打开雷达开关。 
        仪表盘上的荧光屏亮了。 
        “不足一般的礁石!”舒克只看了荧光屏一眼就大叫道。 
        的确,荧光屏上的显示令贝塔也瞠目结舌。 
        “里边肯定有机构!”舒克大喊。 
        海盗也激动了,他与这块礁石生活了三十年,却对它的内部一无所知。今天能揭开这个谜吗? 
        “找个薄弱的地方,咱们打开它。”舒克操纵雷达扫描礁石。 
        贝塔观察荧光屏。 
        “这儿好像薄一些。”贝塔指荧光屏上。 
        “嗯。”舒克比较了一下,同意贝塔的判断,“你去礁石上做个记号。” 
        贝塔离开直升机,在礁石上划记号。 
        舒克关闭雷达,和海盗离开直升机。 
        “用什么打开?”舒克问贝塔。 
        “用坦克炮击。”贝塔说。 
        “不会把礁石打爆炸了吗?”海盗有点儿担心,他毕竟在岛上住了三十年,岛上的一草一木都和他建立了感情。 
        “我悠着劲儿打。”贝塔有把握地说。 
        “你能控制炮弹的威力?”海盗觉得贝塔吹牛吹得没边了。 
        “我可以让坦克在不同的距离向礁石开炮。”贝塔确实聪明。 
        “开始行动吧!”舒克说完将直升机停在远离礁石的地方,他怕礁石爆炸毁了直升机。 
        贝塔跨入坦克。他驾驶坦克后退到距离礁石较远的地方。 
        炮弹入膛。 
        瞄准镜对准了礁石上划记号的地方。 
        “准备好了吗?”贝塔通过无线电台问舒克,“请离开礁石。” 
        “离开了,可以开炮。”舒克说。 
        贝塔按下了射击按钮。 
        嗵! 
        炮弹准确地击中目标。 
        礁石完好无损。 
        “再近点儿!”舒克指挥坦克。 
        坦克往前开。 
        重新瞄准。 
        “开炮!”舒克自封为指挥。 
        “不用你命令。”贝塔说。 
        嗵! 
        又是一炮。 
        礁石被炸出了一个洞口。 
        “成功了!”舒克喊。 
        洞里的景象令舒克、贝塔和海盗结结实实地吃了一惊。   第105集 
        令人震惊的发现; 
        一座控制中心; 
        海盗疯狂   
        神秘的礁石被贝塔的坦克炸开了一个洞口。烟雾散开后,舒克、贝塔和海盗争先恐后往洞里看。 
        礁石里的景象使他们大吃一惊。 
        密密麻麻的仪表和齿轮布满了礁石内部,不计其数的齿轮正在运转。 
        舒克、贝塔和海盗面面相觑。 
        “这也怪得有点儿出圈了。”贝塔挠挠后脑勺。 
        “真有意思。”舒克绞尽脑汁破这个谜。 
        海盗那沉睡了30年的欲望骤然间苏醒了,他凭直觉感到礁石里的机构非同异常。 
        “进去看看,我打头。”贝塔抢先进洞。 
        舒克拔出手枪跟在后边。 
        海盗老迈年高,腿脚不灵活,只好殿后。 
        礁石里不像其他山洞里那样潮湿和阴森,这里干燥凉爽,像有空调。 
        所有齿轮都在转动,发出机械磨擦的声音。 
        “有人吗?”贝塔大声问。 
        没有应答。 
        “你们看这儿。”舒克发现了什么, 
        贝塔和海盗顺着舒克的手指看。 
        一幅世界地图雕刻在墙壁上,地图上每个城市下边都有一行阿拉伯数字。 
        “地图!”贝塔说。 
        “这些数字是什么意思?”海盗问。 
        舒克和贝塔摇头。 
        “那儿有字。”贝塔发现头顶上的石头上刻着不少字。 
        “地球地震控制中心。”舒克念石头上最大的几个字。 
        地震控制中心! 
        舒克和贝塔全身冒冷气。他们站在能够控制整个地球地震的中心机构。 
        海盗的眼睛里射出了光。 
        舒克继续念石头上的字: 
        “本控制中心的任何部件都不能触摸,否则将引起地球总地震,从而导致地球毁灭。如果知道本中心的生物超过4个,也将诱发地球总地震。地图上各城市下边的阿拉伯数字标志着这个城市将发生的地震的时间和震级。” 
        “乖乖,哪儿都不能碰,一碰地球就完了。”贝塔看看自己的前后左右,生怕碰到什么。 
        “太好了,咱们三个能称霸世界了!!!”海盗大喊。 
        “你说什么?”舒克和贝塔异口同声。 
        “称霸世界!!!”海盗狂叫,  “咱们掌握地球的生死机构!咱们让人类于么人类就得于什么,不然咱们就炸了地球!” 
        贝塔和舒克一时说不出话来。 
        “咱们现在就乘飞机去陆地上向人类下通牒!”海盗说完跑出礁石洞。 
        “追上他!”舒克先反应过来。 
        贝塔一个箭步跨出洞口,舒克随后也跑出去。 
        岛上的场面令舒克和贝塔呆若木(又鸟)。 
        只见海盗围着直升机发疯似地转着圈奔跑,一边跑一边狂喊:“地球是我的啦!地球是我的啦!!” 
        舒克和贝塔手足无措。 
        海盗突然惨叫一声,倒在地上。 
        贝塔和舒克跑过去一看,只见海盗手捂着胸口,口中只有出的气,没有进的气。 
        “像是心肌梗塞。”舒克有点儿医学常识。 
        “我……要……地……球……”海盗咽气前说。   第106集 
        海盗墓地上的花圈; 
        令人吃惊的发现; 
        舒克和贝塔离开孤岛   
        海盗死了。 
        他在孤岛上与世无争生活了30年。 
        当他意外地发现了地震控制中心后只3分钟就死了。他想当全世界的头儿。 
        舒克和贝塔埋葬了海盗。他们的心情不好。 
        “是咱们的到来使他离开这个世界的。”贝塔内疚地说。 
        “如果没有发现地震中心,说不定他还能再活30年。”舒克也自责。 
        海盗的墓碑上刻着:海盗之墓。 
        坟上摆着舒克和贝塔送的花圈。 
        舒克和贝塔在海盗的墓前默哀了一会儿,他们为海盗惋惜。他们在地球上又失去了一个朋友。 
        一股莫名的孤独感侵扰着他们。 
        孤岛也变得凄凉了。 
        30年前舒克和贝塔决不会想到30年后他们会为海盗的死悲伤。 
        世界就是这么怪。 
        怪得令人不可思议。 
        舒克先想起了“地球地震控制中心”。 
        “我看咱们得把那个控制中心封死,如果让什么人发现了可不得了。”舒克为地球担心。 
        “如果把它炸了,地球永远不地震就好了。”贝塔异想天开。 
        “别说炸,就是碰一下地球就完了。”舒克提醒贝塔。 
        “咱们去把它封死。”贝塔同意。 
        他俩来到礁石旁。 
        “再进去看一次。”贝塔还想再看看这个恐怖的机构。 
        舒克和贝塔钻进“地震控制中心”。 
        “是谁建的这个中心?”贝塔自言自语,“外星人?” 
        舒克总觉得这个世界受一个超自然的力量主宰。 
        “舒克你看,皮皮鲁住的城市下个星期三有7.1级地震!”贝塔指着地图大声说。 
        舒克一看,不由得起了一身(又鸟)皮疙瘩。果然,地图上标明皮皮鲁居住的城市下个星期有大地震。 
        “咱们得去告诉皮皮鲁!”舒克说。 
        “把准确时间记牢。”贝塔说。 
        他们把地震的准确时间记死了。 
        皮皮鲁是最后一个能知道这个控制中心的生物。超过四个生物,就会诱发地球总地震。 
        舒克和贝塔离开地震控制中心,他俩用石头把礁石的洞口堵死。 
        贝塔又驾驶坦克运来一些大石块堆放在礁石旁边。舒克用灌木丛遮挡住堵严的洞口。 
        他俩围着礁石绕了三圈,看不出破绽。 
        “即使有人登上这个岛,也不会对这块礁石发生兴趣的。”贝塔放心地说。 
        舒克点点头。 
        “抓紧时间去皮皮鲁家吧。”贝塔说。 
        “走。”舒克拍拍手上的土。 
        直升机吊着坦克起飞了。 
        从飞机上俯瞰孤岛,孤岛越来越小,好像渐渐淹没在汪洋大海中。 
        舒克驾驶飞机朝皮皮鲁居住的城市飞去。 
        “也不知道地球上还有多少类似于藏在礁石里的地震控制中心这类机关。”舒克说。 
        “也许还有下雨控制中心,刮风控制中心,雷电控制中心……”贝塔遐想。 
        “要是有个道德控制中心就好了!”舒克一边给发动机加速一边说。   第107集 
        舒克和贝塔安抵皮皮鲁家; 
        皮皮鲁不在; 
        舒克在皮皮鲁的床上睡觉   
        “我得睡一会儿,太困了。”贝塔离开驾驶舱,走进客舱,躺在皮椅上打盹。 
        舒克聚精会神地驾机。 
        舒克驾驶直升机飞临皮皮鲁居住的城市上空。 
        贝塔还在客舱呼呼人睡。舒克按响了警报器。 
        贝塔从梦中惊醒,跑进驾驶舱。 
        “出什么事了?”贝塔惊魂未定。 
        “叫你起床。”舒克冲贝塔一笑。 
        贝塔给了舒克一拳。飞机差点儿倒栽葱。 
        “快到皮皮鲁家了。”贝塔打了个哈欠。 
        其实直升机上有电话,可以直接打到皮皮鲁家。可舒克和贝塔觉得这么大的事得当面谈,必须见皮皮鲁。 
        “你注意观察地面,我着陆了。”舒克全神贯注驾驶飞机。 
        直升机降落在皮皮鲁家的阳台上。屋里没动静。 
        贝塔爬上窗台往屋里看。 
        “屋里没人。”贝塔冲直升机里的舒克打手势。 
        舒克也来到窗台上。他俩从纱窗上的缺口进入皮皮鲁家。 
        “像是外出了。”贝塔根据桌子上的尘土判断。 
        “看看皮皮鲁的记事板。”舒克说。 
        写字台右侧竖着记事板。 
        记事板告诉舒克和贝塔,皮皮鲁去外地开学术会议了。大约两个星期后才回来。 
        “来不及了,咱们得去告诉他。”舒克对贝塔说。 
        贝塔把会议地点记下来。舒克在地图上找会议地点。 
        “在这儿。”贝塔指着地图说。 
        “还挺远。”舒克看看地图上的比例尺。 
        “你睡会儿,我多找点儿电池和食物,做长途跋涉的准备。”贝塔说。 
        舒克累极了,他伸了个懒腰,躺在皮皮鲁的床上睡着了。 
        贝塔把皮皮鲁的家翻了个遍,找到一盒新电池和一些食物,他把这些东西运进直升机的货舱。 
        贝塔在阳台上观看城市,他不能想像一星期后这里将被地震夷为废墟。他感到地震就像悬在人类头上的一把达摩克利斯宝剑,随时随地可以摧毁人类及其劳动成果。 
        贝塔望着一个个窗口,看着一个个家庭精心营造装修他们的居室,贴壁纸,铺地毯,再用各种现代化电器把它武装到牙齿。一次地震袭来,一切都将化为乌有。贝塔打了个哆嗦。 
        “得赶快去通知皮皮鲁。”贝塔呆不住了,他决定叫醒舒克。 
        “我睡了几个小时?”被贝塔推醒的舒克迷迷糊糊地问。 
        “两个小时。我看咱们还是早点儿走好。”贝塔说。 
        “走!”舒克同意。 
        “把坦克留在这儿,这样飞得快。”贝塔提议。 
        舒克和贝塔将吊坦克的钩子从坦克上摘下来。 
        舒克和贝塔坐进直升机的驾驶舱,舒克启动发动机。 
        螺旋桨旋转起来。直升机离开皮皮鲁家,升到空中。 
        夕阳的余辉洒在直升机的机身上。直升机朝即将下班的太阳飞去。   第108集 
        铺天盖地的鸽群; 
        罪恶的子弹; 
        直升机遇难   
        直升机在夜色中飞行了整整一个通宵。舒克和贝塔轮流驾驶飞机,轮流休息。 
        “方位没错吧?”贝塔一边驾驶一边问。 
        舒克看看罗盘指示仪,说:“非常正确。” 
        太阳睡够了,又千篇一律地从东边露出半张脸。 
        “舒克,快看后边是什么?”贝塔惊叫。 
        舒克把头伸出舷窗,好家伙,直升机后边是铺天盖地的鸽群,起码有几千只鸽子。 
        “快降低高度,躲开它们!”舒克怕鸽子撞直升机。 
        贝塔操纵直升机下降,避开鸽群。 
        鸽群遮天蔽日地飞到直升机的上方。 
        “哪儿来这么多鸽子?”舒克说, 
        “可能是信鸽竞翔大赛。”贝塔判断。 
        “没错。”舒克同意。 
        突然从地面传来一阵枪声。 
        紧接着从飞翔的鸽群中掉下去几十只鸽子。 
        又是一阵排枪鸣响。 
        直升机猛烈晃动。 
        驾驶舱里冒出烟雾。 
        “飞机失控!”贝塔大叫。 
        “我们中弹了!我来驾驶。”舒克迅速和贝塔交换位置。 
        直升机急速下降。 
        舒克使出了全身的解数拼命控制直升机不让它掉在地上。 
        “我去准备伞包。”贝塔说完跑进货舱,取出两个伞包,以防万一。 
        成百只信鸽中弹身亡。 
        血溅到直升机的玻璃窗上。 
        “这些人疯了?”贝塔咬牙切齿。 
        舒克满头大汗,直升机仍然往下掉。 
        驾驶舱起火了。贝塔拿灭火器灭火。 
        火被扑灭了。 
        直升机离地面还有五米的距离! 
        “贝塔注意,着陆了!”舒克提醒贝塔。这不是一般的着陆。说得确切点儿,是摔下来。 
        嗵!!!舒克和贝塔的头撞在舱壁上。 
        “舒克!糟了!!”贝塔捂着头指窗外。 
        一个身穿夹克拿着猎枪的男人跑过来,他发现了直升机。 
        “快来看,我打着什么了?”夹克喊。 
        一个穿旅游鞋的男人间声跑来。 
        “哟,直升机!是从天下打下来的?”旅游鞋半信半疑。 
        “没错,是我打下来的。”夹克说。 
        “可能是航模。”夹克说。 
        舒克对贝塔说:  “快躲进货舱,别让他们看见咱们。” 
        舒克和贝塔躲进货舱。 
        他们从窗口窥视外边的情况。 
        夹克把直升机装进一个大网兜,然后从地上捡鸽子。 
        几百只中弹的鸽子也被装进大网兜,有的已经永远离开这个世界。大部分受了伤。它们痛苦地注视着失去的天空。 
        “这两个小子真坏!信鸽的主人不定花了多少心血培养这些信鸽,好不容易盼到比赛了,却被他们打了下来。”舒克看到每只信鸽的脚上都有脚环,这些脚环标志着它们的高贵血统。 
        “这种人地震时都震死算了。”贝塔真不想让旅游鞋和夹克这种人逃脱地震的惩罚。 
        “咱们怎么去告诉皮皮鲁?”舒克惊叫。   第109集 
        舒克给信鸽协会打电话; 
        贝塔剪破网兜: 
        直升机无法起飞   
        夹克和旅游鞋把打死的和受伤的信鸽同直升机一起装进大网兜。 
        “歇会儿,抽支烟。”夹克掏出一包香烟,递给旅游鞋一支。 
        旅游鞋点燃打火机,给夹克和自己点烟。 
        他俩靠着一棵树干坐在地上。网兜在树背后。 
        “今天还行,收获不小。”夹克往天上喷了一口烟。 
        “值得,没白跑。”旅游鞋狠命嘬了一口烟,心满意足地往肚子里咽。 
        “能卖多少钱?”夹克掏出计算器。 
        “死的卖肉,活的到信鸽市场卖。”旅游鞋报数,活的多少多少只,死的多少多少只。 
        夹克熟练地按计算器上的按钮。 
        舒克和贝塔在机舱里能清楚地听见他们的对话。 
        “真黑。”贝塔咬牙切齿。 
        “咱们救这些受伤的信鸽吧?”舒克看着飞机外边那些滴血的信鸽说。 
        “怎么救?它们可飞不走呀!”贝塔想像不出怎么帮助这些寸步难行的信鸽逃脱手中有枪的恶棍的魔爪。 
        “打信鸽准犯法。”舒克灵机一动。 
        “役错。可谁知道他们在这儿打信鸽呢?”儿塔说。 
        “咱们飞机上有电话。”舒克捅贝塔的腋下。 
        贝塔兴奋了,一边笑一边回击舒克。 
        “好了好了,别让夹克他们听见。”舒克边躲贝塔边跑进驾驶舱。 
        舒克拿起电话听筒,发愁了:  “什么地方管信鸽?” 
        “信鸽协会。”贝塔从电视上看到的。 
        舒克拨114查号台。 
        “请问您查哪儿?”查号台小姐娇滴滴的声音。 
        “查信鸽协会。”舒克说。 
        小姐告诉舒克。 
        这大概是地球上的老鼠第一次和人类通电话。 
        舒克拨信鸽协会的电话。 
        “信鸽协会。”听筒里传出一个男人的声音。 
        舒克告诉他有人在打信鸽。 
        “在什么地方?”男人的声音显得很激动,听得出他很想抓打鸽子的人。 
        舒克不知道这里的地名。贝塔查地图。 
        舒克打开仪表查方位。 
        “是这儿。”贝塔指着地图上的地名说。 
        舒克把方位告诉信鸽协会。 
        “我们马上赶到。”电话挂了。 
        “快把网兜剪破,信鸽协会的人来了对咱们未必有利。”舒克说。 
        “我去。”贝塔从工具箱里取出一把剪子,然后打开舱门跳出去。 
        贝塔穿过鸽群,把网兜剪开一个口子。 
        舒克和贝塔费了九牛二虎之力将直升机拉出网兜。 
        飞机摔坏了,无法启动。舒克抢修,贝塔放哨。 
        “不好!他们要走了。”贝塔看见夹克他们站起来。 
        “必须在信鸽协会的人来之前拖住他们。”舒克急了。   第110集 
        直升机重上蓝天; 
        舒克在空中戏弄夹克; 
        手铐有事干了   
        夹克和旅游鞋准备回家了,他们站起来掸裤子上的土。 
        “怎么办?”贝塔急了。他知道只要夹克和旅游鞋走了,信鸽协会的人就找不着他们了。 
        “发动机修好了!”舒克顾不上擦汗就往机舱里跑,“贝塔,快上飞机!” 
        贝塔不知舒克想什么,只得跟着舒克钻进飞机。 
        “咱们开飞机拖住他们。”舒克发动飞机。 
        夹克和旅游鞋走过来了。他们没注意网兜旁边的直升机。 
        “快!他们来了!”贝塔往窗外看。 
        螺旋桨转了。 
        “你看!”旅游鞋发现了即将起飞的直升机,惊叫道。 
        夹克被眼前的景象惊呆了。 
        “快抓住它!这是一架很值钱的航模飞机!”旅游鞋喊。 
        夹克冲过去。 
        直升机升起来了。夹克扑了个空。 
        “别飞远了,就在他们头上转,拖时间。”贝塔提醒舒克。 
        舒克驾驶直升机在夹克头顶上大约一米远的地方盘旋。 
        夹克跳着够直升机。 
        “用这个!’旅游鞋递给夹克一根长树枝。旅游鞋个子没夹克高。 
        夹克用树枝打直升机。 
        舒克用娴熟的飞行技术逗夹克。 
        夹克满头大汗。 
        “你得给人家点儿希望。”贝塔说。 
        舒克故意让夹克的树枝碰到直升机几次。 
        时间在运行。 
        “妈的,用枪打下来算了!”旅游鞋不耐烦地说。 
        “打!”夹克扔掉树枝。 
        旅游鞋往猎枪里装子弹。 
        “注意,”贝塔发现了旅游鞋的企图,“他们要用枪打咱们。” 
        “能打着我的飞机的人还没生出来呢!”舒克口吐狂言。 
        “刚才你不是被打下来了吗?”贝塔说。 
        “那是暗箭。明着打绝对打不着。不信你看着。”舒克毫不含糊。 
        旅游鞋举枪瞄准直升机。 
        舒克从空中死盯着他右手的二拇指。 
        啪! 
        枪响了。 
        直升机灵活地躲过子弹。 
        “你还真行。”贝塔拍拍舒克的肩膀。 
        又是一枪。 
        直升机又躲开子弹。 
        “你真臭,浪费了两颗子弹,我来。”夹克装子弹。 
        啪! 
        没击中。 
        旅游鞋笑。 
        “我今天非把它打下来不可。”夹克冒火了。 
        连续射击。 
        连续打空。 
        “看,那边来警车了!”贝塔兴奋了。 
        几辆警车顺着枪声指引的方向开过来。 
        警察包围了夹克和旅游鞋。 
        正往猎枪里塞最后一颗子弹的夹克和旅游鞋被戴上了手铐。 
        “也不知偷猎信鸽该判几年刑?”贝塔还挺替夹克和旅游鞋操心,他觉得他俩可能刚到结婚的年龄,现在坐牢亏了点儿。 
        “少操点儿心,快去找皮皮鲁吧!”舒克按了贝塔鼻子一下。   第111集 
        两个哲学家的对话; 
        国际物理会议在春园饭店举行; 
        楼顶上的呼救声 
        夹克和旅游鞋被塞进囚车后,舒克驾驶飞机继续向目的地飞行。 
        “那些受伤的信鸽得救了。”贝塔松了一口气。 
        “也许几年才比赛一次,就这么失去了得冠军的机会。”舒克为受伤的信鸽惋惜。 
        “人类里怎么会有夹克和旅游鞋这种人?”贝塔不明白能造出卫星的人类怎么想不出方法净化自己的遗传基因。 
        “生命的本质就是弱肉强食。”舒克冒出一句有哲理的话。 
        “生命和残暴是孪生兄弟。”贝塔不甘落后,也跟上一句。 
        “去拿点儿吃的来吧。”舒克肚子饿了。 
        贝塔从货舱里取来食物。 
        “直升机飞行速度太慢。”舒克边吃边说。 
        “咱们要是有个飞碟就好了。”贝塔异想天开。 
        经过一天的不间断飞行,太阳从天上掉下来时,他们飞临皮皮鲁开会的城市上空。 
        “会在哪儿开?”舒克看着下面数不清的建筑物,不知怎么找皮皮鲁。 
        “打电话问。”贝塔拿起电话听筒。 
        先拨查号台,查旅游局的电话号码。 
        “请问国际物理会议在哪家饭店开?”贝塔给旅游局打电话。 
        “春园饭店。”旅游局告诉贝塔。 
        “行了,找春园饭店吧!”贝塔放下电话。 
        “等天黑了,饭店都有霓虹灯标志。现在咱们歇会儿。”舒克说。 
        贝塔觉得这个主意不错。直升机在一座高楼的楼顶上着陆。 
        “你睡会儿,我站岗。”贝塔说。 
        “先到外边呼吸新鲜空气。”舒克提议。 
        他俩离开飞机,站在楼顶上活动四肢。 
        “贝塔你听!”舒克小声说。 
        “听什么?”贝塔什么也没听到。 
        舒克趴下,将耳朵贴在地上听。 
        “救命——”从楼板下边传来微弱的呼救声。 
        “去看看?”舒克问贝塔。 
        “当然!”贝塔不古糊。 
        “可能就是下边这间屋子里传来的声音。”舒克判断。 
        “我去机舱拿根绳子。”贝塔钻进直升机。 
        舒克将绳子的一头拴在一根铁管子上,另一头扔到楼下。 
        “我先去看看,你在上边等我的信号。”贝塔抓紧绳子溜下去。 
        贝塔成功地站在了绳子经过的第一个窗台上。 
        窗户拉着窗帘。贝塔掏出小刀,在纱窗上开了一个口。 
        他悄悄钻进纱窗,轻轻地把窗帘拨开一条缝儿。 
        一个中年妇女被捆在椅子上,两个蒙面大盗正在翻箱倒柜找东西。 
        “说!钱放在什么地方?”一个蒙面大盗问妇人。 
        女人的嘴被堵着,她摇头。两个耳光。女人的鼻子出血了。 
        贝塔不忍心看这种场面。   第112集 
        舒克拨110电话; 
        贝塔遇险; 
        蒙面大盗和镜框; 
        女人脸上的哲理   
        贝塔按原路返回。他把房间里发生的情况告诉舒克。 
        “强入民宅抢劫,可恶!”舒克说。 
        “咱们去帮帮他们。”贝塔说。 
        舒克和贝塔制定援救方案。 
        “先打电话报警,然后想办法拖住大盗。”舒克想再用一回抓夹克的方法。 
        “我想办法找到大门钥匙,把大门从外边反锁上,让大盗出不去。”贝塔想了个拖住大盗的方法。 
        援救计划就这么通过了。 
        舒克跑进直升机拨110匪警电话。 
        “是警察局吗?”舒克问。 
        “是。” 
        “有两个闯入民宅的强盗正在作案。” 
        “作案地点?” 
        “……”舒克不知道此楼的地名及楼号。 
        “喂,作案地点?”对方又问。 
        “你等一下。”舒克驾驶直升机起飞,飞到楼的侧面。舒克知道楼号一般都写在楼的两侧。 
        舒克将楼名及楼号告诉警察。 
        直升机回到楼顶上。 
        “电话打通了。警察一会儿就到。”舒克离开飞机,对贝塔说。 
        “咱们下去吧。”贝塔说。 
        他们顺着绳子下到那家的窗台上。 
        贝塔拨开窗帘,让舒克看屋里的情景。 
        两个蒙面大盗还在翻箱倒柜。 
        “你到哪儿去找大门钥匙?”舒克问贝塔。 
        贝塔的目光已经把屋了扫荡了一遍,没有发现钥匙。 
        “我去同问那女人。”贝塔决定冒险。 
        “她会告诉你吗?”舒克表示怀疑。 
        “我悄悄爬到她肩膀上,在她耳朵边上告诉她咱们是来救她的。不会被大盗发现。”贝塔有把握。 
        “去吧,当心点儿。我在这儿接应你。”舒克拍拍贝塔的肩膀。 
        贝塔顺着暖气管下到房间的地板上。 
        舒克屏住呼吸,紧张地注视着大盗。 
        贝塔已经溜到女人的脚边上.他爬上了她的脚背,准备顺着她的裤腿往上爬。 
        “啊——”女人发出一声极恐怖的尖叫,恐怖得使她把嘴里塞的毛巾都喷出去了。 
        “你找死呀!”一个大盗压低了声音喝道。 
        “老鼠!”女人使劲挣扎被捆住的双脚,想把贝塔甩下去。 
        舒克愣了。 
        她见到蒙面大盗都没有这么恐惧和厌恶。 
        “求求你们,帮我弄死这老鼠,我告诉你们钱藏在哪儿!”女人请求大盗打死贝塔。她用两脚夹住了贝塔,贝塔无法脱身。 
        “你先说钱在哪儿?”大盗狡猾。 
        “在墙上的镜框里。”女人说。 
        大盗之一从墙上摘下镜框,然后往地上一磕,镜框碎了。 
        画后边露出一个信封,大盗撕开信封,里边有300张面额100元的钞票。 
        大盗将钱装进衣兜。 
        “撤!”另一个大盗说。 
        “求求你们打死这只老鼠!”女人哀求。 
        大盗冷笑一声,往女人脸上吐了口唾沫:“呸,我们又不是居委会的干部,谁管你老鼠不老鼠!” 
        蒙面大盗扬长面去。 
        舒克飞快地爬到女人身上,拔出匕首往女人胳膊上扎。 
        贝塔已经被女人的双脚夹得全身发紫了。 
        女人看见又来了一只老鼠,还向她进攻,疼得松开了贝塔。 
        舒克扶起贝塔往窗台上撤。贝塔站在窗台上看那女人的脸。他从那张脸上看出好多哲理。 
        警察赶到时,人盗早已无影无踪了。 
        女人一口咬定大盗是靠两只老鼠帮忙盗走了她的钱的。 
        警察打电话叫精神病医院的救护车。   第113集 
        从窗户外边找皮皮鲁; 
        电梯里的小姐; 
        保安人员当猫   
        舒克和贝塔回到楼顶上时,城市已经笼罩在夜色里,天上的星星和地上的灯火交相辉映。 
        “看!”贝塔大声说。 
        舒克顺着贝塔手指的方向看去.只见一座大厦上闪烁着四个大字:春园饭店。 
        “闹了半天,春园饭店就在咱们附近。”舒克说。 
        “走,去找皮皮鲁。”贝塔先钻进飞机。 
        舒克随后登机。 
        直升机呼啸着朝春园饭店飞去。 
        春园饭店金碧辉煌,每一个窗口都透着豪华和奢侈。 
        舒克驾机绕着饭店盘旋了两圈,寻找着陆地点。 
        “这么大的饭店,咱们怎么找皮皮鲁?”贝塔问舒克。 
        “干脆开飞机从窗口外边往里看,这样好找些。”舒克灵机一动。 
        “从最高一层开始。”贝塔提议。 
        直升机悬停在饭店最高一层最左边的窗口外边。 
        贝塔往窗口里看。 
        没有皮皮鲁。 
        “下一个窗口。”贝塔指挥舒克。 
        直升机又挪到第二个窗口。 
        还是没有。 
        直升机一个一个窗口挨着找,终于在第15层的第7个窗口看见了皮皮鲁。 
        皮皮鲁正坐在写字台前写东西。 
        “怎么进去?”贝塔发现饭店的窗户都是密封的。 
        “记住皮皮鲁的位置,咱们从楼顶的通风管道进去。”舒克说。 
        直升机降落在春园饭店的楼顶。 
        舒克和贝塔顺着通风管道进入饭店。 
        饭店的走廊灯光很暗,这对舒克和贝塔的行动十分有利。地上还铺着厚厚的地毯。 
        饭店一共37层,舒克和贝塔得下到15层。 
        贝塔看见了电梯。 
        “太危险了吧?”舒克觉得电梯里太亮,容易被人发现。 
        “咱们趁没人的时候爬到电梯顶上。”贝塔说。 
        止好电梯门开了,电梯里没人。 
        舒克和贝塔用最快的速度跑进电梯。这时他们才发现,电梯四壁光滑如镜,根本爬不上去。 
        电梯外边传来脚步声,跑出去已经来不及了。 
        “快!钻到地毯下边。”贝塔急中生智。 
        一位小姐走进电梯,她正好去15层。 
        电梯开始下降。 
        小姐无意中发现地毯的一角在抖动。 
        她好奇地掀起地毯。 
        “啊——”一声尖叫,“老鼠!” 
        电梯到了15层,电梯门打开了。 
        “快跑!”舒克大喊。 
        贝塔和舒克逃出电梯,往走廊里跑。 
        “抓老鼠!这饭店有老鼠!”小姐扯着嗓子喊,像是碰上了强盗。 
        保安人员闻声赶来。 
        “你们饭店有老鼠!”小姐声色俱厉。 
        “小姐,这不可能。”保安人员否认。 
        “我亲眼看见的!你们看,正往走廊那头跑呢!”小姐指给保安人员看。 
        保安人员揉揉眼睛,他们清清楚楚地看见两只鼠窜而逃的老鼠。 
        保安人员开始执行猫的职责,追杀老鼠。 
        舒克和贝塔危在旦夕。   第114集 
        保安人员大惑不解; 
        皮皮鲁保护老朋友; 
        海盗真死了   
        皮皮鲁正在房里修改论文,他是本届国际物理学会议的明星。 
        走廊里的喊叫声引起了他的注意,他本能地感到这喊叫声同自己有关系。 
        皮皮鲁放下笔,打开房间门。 
        “舒克!贝塔!!”皮皮鲁看见正好跑到他门口的舒克和贝塔。 
        “他们在追我们!”舒克告诉皮皮鲁。 
        “快进来!”皮皮鲁把老朋友放进屋里,然后关上门。 
        保安人员敲门。他们看见老鼠跑进这个房间。 
        “藏进去!”皮皮鲁打开他的手提箱。 
        舒克和贝塔钻进去。 
        皮皮鲁去开门。 
        “对不起,打搅您了。”保安人员对皮皮鲁说,“刚才进来的两只老鼠呢?” 
        “老鼠?饭店里怎么会有老鼠?”皮皮鲁装傻充愣。 
        “您没看见老鼠?”保安人员清清楚楚地看见皮皮鲁把老鼠让进了他的房间。 
        皮皮鲁摇摇头。 
        保安人员知道面前这位客人是世界上大名鼎鼎的物理学家,传说他今年获得诺贝尔奖的可能性极大。 
        “我们看见刚才您开门时,两只老鼠跑进了您的房间。”保安人员说。 
        “是吗?我没看见。要不你们进来找找?”皮皮鲁说。 
        “好的。请您原谅。饭店有老鼠会导致客人搬走,我们一定要抓住他们。”保安人员走进房间。 
        “你检查床下,我检查卫生间,把大门关上。”一位保安人员对同事说。 
        搜查开始。 
        没有老鼠的影子。 
        “怪事。”保安人员一边擦汗一边看皮皮鲁。 
        “我说没有吧!”皮皮鲁有几分得意,他感到开心。 
        “对不起,耽误您时间了。”保安人员悻悻地走出房间。 
        皮皮鲁锁上房间门。 
        “快出来吧!”皮皮鲁喊舒克和贝塔。 
        “憋死了。”舒克和贝塔从手提箱里钻出来。 
        “见到海盗了吗?”皮皮鲁问。 
        “见到了,而且有重大发现!”舒克说。 
        “海盗还活着?”皮皮觉鲁得不可思议。 
        “活着,还挺健康。”贝塔说。 
        “老鼠能活这么长时间?”皮皮鲁不信。 
        “他自己呆在一座孤岛上,与世隔绝,当然能长寿。他如果在陆地上,早死了。”舒克说。 
        皮皮鲁若有所思地点点头。 
        “怎么不把他带来?”皮皮鲁还挺想见海盗。 
        “死了。”贝塔耸耸肩。 
        “我说老鼠不可能活这么长吧!”皮皮鲁笑了,他认为舒克和贝塔是在逗他。 
        “海盗是在我们去了以后死的。”舒克告诉皮皮鲁。 
        皮皮鲁像是在听天方夜谭。 
        “别兜圈子了,快告诉他吧!”贝塔捅捅舒克。 
        舒克和贝塔你一言我一语将他们到孤岛上的经历讲给皮皮鲁听。 
       当听到地震控制中心时,皮皮鲁的脸色严峻起来。 
       “地球上只能有四个生物知道这个中心?超过四个,地球就会总地震?”皮皮鲁生怕自己没听清,又问了一遍。 
        舒克和贝塔点头。 
        “这么说,我是能够知道的最后一个人!”皮皮鲁扣除了舒克、贝塔和海盗后,得出这个结论。   第115集 
        舒克和贝塔驾驶直升机在小花园和皮皮鲁会合; 
        乘坐巨型客机; 
        炸药包的威胁   
        舒克将皮皮鲁居住的城市发生大地震的精确时问告诉皮皮鲁。皮皮鲁看了一眼日历,愣了。 
        地震将在星期四上午九点三十二分发生。现在是星期二晚上。 
        “我马上赶回去!”皮皮鲁看手表。 
        “回去?回去挨地震?”贝塔惊讶地看着皮皮鲁。 
        “我去告诉市长,让他在全市采取紧急措施。”皮皮鲁说。 
        “你不会泄露地震控制中心的秘密吧?”舒克提醒皮皮鲁。 
        “当然不会!我怎么会诱发全球总地震呢!”皮皮鲁开始收拾桌上的资料。 
        “你不开会了?”贝塔知道这是一次重要的国际学术会议,皮皮鲁在这次会议上唱主角。 
        “人命关天,不开了。”皮皮鲁往手提箱里塞东西。说实话,皮皮鲁讨厌开会。不管是什么会。 
        舒克和贝塔觉得市长会相信皮皮鲁的,今天的皮皮鲁是著名的科学家,已不是昔日那个喜欢恶作剧的孩子了。 
        “你们跟我一起去吗?”皮皮鲁不忍心让舒克和贝塔跟他一起奔赴地震现场。 
        “当然。”舒克知道不用征求贝塔的意见。 
        “我的坦克还存放在你家的阳台上呢。”贝塔说。 
        “你们的飞机太慢,跟我一起去坐真飞机吧。”皮皮鲁提议。 
        “太棒了。”舒克早就想坐坐人类的大飞机。 
        “我们去楼顶上把直升机开到饭店大门旁的小花园里,你在那儿等我们。”贝塔设计接头地点。 
        “我先把你们送到通风管道那儿去,省得别人找你们的麻烦。”皮皮鲁关好手提箱。 
        舒克和抛塔钻进皮皮鲁的衣兜。 
        皮皮鲁先到大会秘书组请假,说是有急事要连夜赶回去。 
        会议组织者百般挽留皮皮鲁,无奈皮皮鲁坚决要回去。 
        皮皮鲁乘电梯来到饭店的最高一层,舒克和贝塔钻进通风管道。 
        那两位保安碰巧看见了皮皮鲁将老鼠送入管道,他们瞠目结舌。 
        提前离会。保护老鼠。这使与会的科学家们感到皮皮鲁的行迹的确蹊跷。 
        皮皮鲁和舒克、贝塔在小花园会合后,皮皮鲁将直升机装进手提箱。 
        舒克和贝塔藏进他的衣兜。 
        皮皮鲁叫了辆出租车直奔机场。 
        “还有机票吗?”皮皮鲁问售票的小姐。 
        “正好有一张退票。”小姐认出了皮皮鲁,“马上就要起飞了,请您抓紧时间办理登机手续。” 
        皮皮鲁踏进机舱门时,飞机的发动机已经开始旋转了。 
        舒克偷偷往外看,这是一架能乘坐几百位旅客的巨型客机。 
        飞机滑上跑道。起飞。 
        “这小子技术差点儿。”舒克对驾驶员评头论足。 
        贝塔撇嘴。皮皮鲁笑。 
        一个小伙子经过皮皮鲁身旁朝驾驶室走过去。 
        “您需要帮助吗?”空中小姐问小伙子。 
        小伙子指着自己衣服里对空中小姐说:  “这里有一包炸药,如果你们不按照我的命令改变航线,我就炸了飞机!”   第116集 
        棉花炸药包; 
        飞行员受重伤; 
        飞机往下掉; 
        皮皮鲁不知道怎样和地面通话   
        劫持飞机!空中小姐的脸色变了。 
        “快打开驾驶室的门!”小伙子狂叫。 
        旅客们慌了。 
        “再不打开,我拉炸药包了!”小伙子威胁道。 
        空中小姐敲驾驶室的门。 
        飞行员从门镜里知道碰上了劫机,只好打开门。 
        小伙子冲进驾驶室,命令飞行员改变航线。 
        “真倒霉,偏偏碰上劫机!”皮皮鲁计算着时问,他怕不能在地震发生前赶回城市。 
        舒克和贝塔也替皮皮鲁着急。 
        飞机突然剧烈晃动起来。 
        “打起来了,快去帮飞行员!”皮皮鲁提醒旅客们。 
        几个膀大腰圆的旅客冲进驾驶室。 
        原来,一位飞行员突然抓住了劫机犯的双手,使他无法拉响炸药包,另一位飞机员撕开他的衣服,抢炸药包。 
        炸药包是假的,一包棉花。 
        劫机犯拼命反击,他用匕首连续刺伤了两位飞行员,旅客制服了劫机犯。 
        机舱内一片欢腾。 
        飞机突然急速下降。 
        两位飞行员受了重伤,飞机无人驾驶了! 
        每位旅客脸上的血色均荡然无存。 
        皮皮鲁估计再有3分钟飞机就会坠地。 
        “没人会开飞机?!”一位旅客绝望地喊叫。 
        “我会开!”舒克在兜里冲皮皮鲁喊。 
        皮皮鲁眼睛一亮,他问舒克:“这么大的飞机,你怎么开?” 
        “我指挥你开。”舒克说。 
        “我来试试,大家别慌,都系上安全带!”皮皮鲁朝驾驶室跑去。 
        旅客们把生存的希望全寄托在皮皮鲁身上。 
        皮皮鲁坐在驾驶员的位置上,面对数不清的仪表仪器,他感到束手无策。 
        舒克和贝塔从兜里跳出来,站在驾驶台上。 
        “拉杆!”舒克指挥皮皮鲁。 
        “什么叫拉杆?”皮皮鲁一窍不通。 
        贝塔跳上驾驶杆,指示皮皮鲁往哪个方向拉。 
        飞机的高度表指示飞机现在距地面只有300米了。 
        “快拉!”舒克喊。 
        皮皮鲁往后拉驾驶杆。 
        飞机艰难地抬起头,重返天空。 
        贝塔松了口气。 
        舒克继续向皮皮鲁下一系列的指令。飞机终于返回到九千米高空。 
        机舱里一片掌声。 
        “戴上耳机,同地面恢复联系。”舒克说。 
        皮皮鲁戴好耳机,打开通话开关, 
        耳机里传来呼叫声:“938!938!听到没有?请回答!请回答!!” 
        938是本机航班号。 
        “我在这儿!”皮皮鲁不懂通话术语。 
        贝塔跳到皮皮鲁嘴边,冲着送话器说:“我是938,我是938,我们遇到了劫机,现已制服劫机犯,飞行员受重伤,不能驾驶飞机……” 
        “你是谁?”地面问。 
        “我是贝……”贝塔自觉失言,忙改口,“我是一名旅客,我有一点儿驾机经验,不过还需要地面指挥。 
        地面显然大汗淋漓。 
        飞行不难,着陆才是关键。   第117集 
        航空史上的奇迹; 
        记者提出的问题像排炮   
        皮皮鲁操纵大型喷气客机飞临机场上空。 
        机场的每一根神经都快绷断了。消防车、救护车、清障车和警车像参加汽车大赛似地争先恐后往跑道两边云集。 
        机场上的每一位工作人员都知道938航班现在正由一名从未驾驶过飞机的旅客操纵着陆。 
        塔台的地面指挥死握着话筒,眼珠瞪出了眼眶。 
        飞机摇摇晃晃地飞过来。 
        “你绕场一周,降低高度,然后对准跑道,放下起落架。”地面指挥说。 
        “明白。”贝塔回答。 
        “你别紧张。”舒克看出皮皮鲁开始紧张。 
        对于皮皮鲁来说,从未驾驶过飞机的他要操纵这么巨大的飞机着陆,真是太难了。皮皮鲁满头大汗。 
        机舱里的每位旅客脸上都呈现出英勇就义前的表情。 
        “好,飞机已经对准跑道,慢慢往前推驾驶杆。”舒克指挥。 
        飞机下降高度。 
        “拉一点儿。再拉一点儿。往左偏。好。推杆。慢点儿推。对。”舒克指挥。 
        皮皮鲁全神贯注地操纵飞机着陆。 
        塔台上的人一个个都像水泥浇铸的塑像,死盯着对着跑道冲过来的飞机。 
        “他挺内行呀!”富有经验的地面指挥从皮皮鲁的飞行状态看出他有驾驶经验。 
        “没错,他不可能没开过飞机!”副指挥同意指挥的判断。 
        飞机已经接近跑道,成败在此一举。 
        “拉杆!”舒克指挥。 
        皮皮鲁的智商的确不同凡响,从未驾驶过飞机的他居然将巨型喷气客机平安地降落在机场上。上帝确有偏心,就像他让有的国家地底下到处是石油而有的国家一滴石油也没有一样,他让有的人的大脑充满智慧让有的人的大脑一片空白。 
        机舱里像过节。 
        旅客们冲进驾驶舱,拥抱皮皮鲁。 
        幸亏舒克和贝塔动作快,藏了起来。 
        “英雄万岁!”不知哪位旅客领头喊。 
        一片“英雄万岁”的喊声。 
        机舱门开了,地面指挥跑进飞机,他急于想看看那个没开过飞机却能把飞机降落在机场的航空天才。 
        “皮皮鲁!大物理学家!”地面指挥从电视上见过皮皮鲁,他惊呆了。 
        “我还有急事要见市长,请你马上给我准备一辆车。”皮皮鲁对地面指挥说。 
        副指挥去找车。 
        闻讯赶来的记者包围了皮皮鲁: 
        “请问您以前从未开过飞机吗?” 
        “您现在有什么感受?” 
        “物理学和飞机驾驶有共同的地方吗?” 
        “您后怕吗?”   第118集 
        窥视孔里的眼睛; 
        穿睡衣的市长; 
        皮皮鲁不能自圆其说   
        皮皮鲁使出浑身解数摆脱了记者,一头钻进机场为他安排的小轿车。 
        “直开市长家。”皮皮鲁看表,现在是凌晨四点整。 
        舒克和贝塔在皮皮鲁的兜里睡着了。 
        汽车停在市长的住宅门口。 
        皮皮鲁下车按门铃。 
        门上出现了一个方形的窥视孔。窥视孔里有一双眼睛。 
        “找谁?”冷冰冰两个字。 
        “找市长。”皮皮鲁回答。 
        “这么晚了,市长在睡觉。今天上午去办公室找他吧。”眼睛说, 
        “我有急事,非常重要的事。”皮皮鲁说。 
        “你是谁?”眼睛问。 
        “物理学家皮皮鲁。” 
        眼睛显然听说过这位本市赫赫有名的科学家,他打开了大门。 
        皮皮鲁被请进客厅。 
        “请稍等,我去叫市长。” 
        贝塔醒了,他推舒克。 
        “干吗?”舒克揉眼睛。 
        “到市长家了。”贝塔指指外边。 
        舒克往外看。 
        “市长家真漂亮呀!”舒克睡意全无。 
        “市长来了。”贝塔捅捅舒克。 
        一个穿睡衣的男人出现在楼梯上。 
        “有失远迎,有失远迎。”市长显然对皮皮鲁很尊敬。 
        “打扰您睡觉了,真对不起。无奈事情太重要了。”皮皮鲁站起来同市长握手。 
        “坐。”市长坐在皮皮鲁对面的沙发上。 
        “明天上午九点三十二分本市有强烈地震。”皮皮鲁说,“请您采取措施。” 
        “你说什么?”市长瞪大眼睛,“你再说一遍。” 
        皮皮鲁重复了一遍。 
        “你怎么知道的?”市长清楚皮皮鲁是物理学家,不是地质学家,更不是地震专家。 
        “我……”皮皮鲁不能说出孤岛上的地震控制中心。 
        “你开始研究地震了?”市长判断。 
        “没有……”皮皮鲁无法自圆其说。 
        “你有意外的发现?”市长想像力挺丰富,他猜想皮皮鲁在搞科研时意外发现了某种仪器有预报地震的功能。 
        “反正您必须采取措施,在明天早上九点三十二分时,让全市居民离开房屋。”皮皮鲁盯着市长的眼睛。 
        市长知道皮皮鲁不会半夜来和他开玩笑,但他也不能凭皮皮鲁一说就做这么大的决定。 
        “你知道,”市长粗略估算了一下,“全市所有人撤离房屋将给本市带来直接经济损失两亿元,这还不算本届市政府的信誉损失。所以你必须说出预报的科学根据,否则我不能采取任何措施。” 
        “你必须采取措施!”皮皮鲁急了。 
        “说说你是根据什么预报这次地震的?而且时间这么具体,连分钟都有。”市长茫然地注视着皮皮鲁,“既然你这么关心本市居民的生命安全,为什么不能告诉我呢?” 
        皮皮鲁说不出话来。 
        舒克和贝塔在衣兜里也是十着急。 
        “我以一个科学家的名誉向您担保我的预测。”皮皮鲁没别的招儿了。 
        “我也以一个市长的名誉向你担保,我不能在没有任何科学根据的情况下采取措施,我要向我的市民负责。”市长一字一句地说。 
        皮皮鲁绝望了。他不能说出地震控制中心,即使地球不会发生总地震,他也不能泄露这个秘密。他清楚,只要地震控制中心一暴露,世界上的所有国家都想得到这个孤岛,世界大战就会爆发。今天几桶石油都会导致操戈相见,何况地震控制中心! 
        市长和皮皮鲁对视了十分钟。无言。   第119集 
        地震局长和专家嘲笑皮皮鲁; 
        舒克提出首先要让孩子们躲避地震灾难   
        市长坚持让皮皮鲁说出他是根据什么预报明天上午九点三十二分有强烈地震的,如果他说不出来或不愿意说,那么市长就不能在全市采取紧急措施,皮皮鲁无可奈何。 
        “只好这样了。”皮皮鲁苦着脸站起来,“您最好在一些重要部门采取措施。” 
        市长站起来送客。 
        皮皮鲁走后,市长马上打电话把地震局长和专家叫到他家来。 
        地震局长和专家否认明天本市会发生大地震,他们说所有监测地震的精密仪器所显示的数据正常得不能再正常了。当他们听市长说物理学家皮皮鲁预测明天上午本市将发生大地震可又说不出科学根据时,不禁哈哈大笑,他们说历史上还没有物理学家预报地震的先例,还说隔行如隔山就差没说皮皮鲁狗拿耗子多管闲事了。 
        市长送走地震专家后无法入睡,他总觉得皮皮鲁这样有知名度的科学家不会开这样人的玩笑,尽管皮皮鲁说不出科学根据,但市长还是要在重要部门采取一些预防措施。 
        市长连夜打电话通知了几个重要部门,并叮嘱说这是绝密行动,不得向外界泄露。市长怕引起社会混乱。 
        皮皮鲁回到家里。舒克和贝塔从他衣兜里爬出来伸懒腰。 
        “你们说怎么办?”皮皮鲁满脸愁云。 
        “这个市长真够呛,连你都不相信。”贝塔摇头。 
        “也不能怪他,”皮皮鲁说公道话,“要是我当市长,某天夜里突然来了个人,告诉我说明天本市将发生大地震,而他又不说他是根据什么预测的,我也不敢贸然在全市采取措施,经济损失太大了,还有可能引发社会混乱。” 
        “可咱们又不能说是地震控制中心预报的,真窝囊。”舒克在桌子上跷着二郎腿说。他累了。 
        “咱们应该尽量多通知市民,减少伤害。”皮皮鲁边想边说。 
        “登报!”贝塔跳起来,“还记得从前咱们在《晚报》上刊登的那条不让市民投放鼠药的消息吗?” 
        “哪家报纸肯登这种没根据的预报?”皮皮鲁摇摇头。 
        “还像上次那样换版呀!”贝塔提醒皮皮鲁。 
        “现在都是激光排版,换不成啦!”皮皮鲁叹口气。 
        “先想办法救孩子。”舒克觉得孩子们来到这个世界的时间不长,一定不能让他们死于地震。 
        皮皮鲁点头。 
        “咱们明天上午分头给全市所有学校和幼儿园打电话,就说市长几点三十二分要来视察学校,让全校师生在操场上集合等候市长的视察,怎么样?”舒克提议。 
        舒克鬼点子多。 
        “这办法不错。”皮皮鲁同意。 
        “然后咱们再尽可能地用电话通知更多的单位。”舒克说。 
        “也只能这样了。”皮皮鲁无奈地说。作为一个科学家,他不能拯救全市居民的生命,皮皮鲁感到内疚。 
        贝塔跑到阳台上照料他的坦克。舒克擦飞机。皮皮鲁收拾他的珍贵资料,他不能让地震毁了他多年的心血。   第120集 
        无数个匿名电话; 
        学校和幼儿园为市长视察感到兴奋; 
        皮皮鲁在哪里   
        恐怖的星期四来到了。 
        天没亮,皮皮鲁、舒克和贝塔就醒了。 
        他们今天在九点三十二分以前的任务就是打电话,尽可能多的通知居民撤离房屋。 
        电话号码已经誊写在几张大纸上。 
        皮皮鲁站在阳台上眺望晨曦中的城市,他不能想像几小时后这里将变成一片废墟,人们呕心沥血营造的一个个温暖的家庭瞬间将化为乌有,人类在大自然面前将再一次显得无能为力。 
        舒克了解此刻皮皮鲁的心情,他站在皮皮鲁身边不吭声。 
        贝塔还在不停地理怨市长。 
        八点钟到了。 
        “开始打电话!”皮皮鲁神情严肃地说。 
        舒克和贝塔飞快地钻进直升机,舒克拨电话,贝塔给他念电话号码。 
        皮皮鲁也坐在写字台前打电话。 
        舒克拨通了第一个号码,这是一所小学。 
        “××小学。”听筒里说。 
        “我是教育局,请通知你们校长,今天上午九点三十分,市长到你们学校视察,请你们学校全体师生在操场集合。” 
        “市长要来?我马上去告诉校长!” 
        舒克接着拨第二个号码。 
        “我是××幼儿园。” 
        “我是教育局。请告诉你们园长,今天上午九点三十分市长去你们幼儿园检查工作,请你们集合全体孩子和职工在幼儿园门口迎接市长。” 
        舒克一口气打下去。 
        皮皮鲁也是马不停蹄。 
        墙上石英钟的表针似乎比平时走得快多了。 
        八点三十分。 
        八点五十分。 
        “你说话简练点儿。”贝塔嫌舒克说话太哕嗦。现在多打一个电话就能救几百条人命。 
        舒克瞪了贝塔一眼。 
        九点整。幼儿园和学校的电话全部打完了。 
        全市的幼儿园和学校听说市长要来这里视察,都兴奋得不能自已,校长或园长赶忙通知老师集合学生和小朋友到室外恭候市长的光临。 
        “再尽可能地多打,照着电话簿上的号码打。”皮皮鲁还想救更多的生命。 
        舒克和贝塔往医院打,往敬老院打,往大商场打…… 
        石英钟的指针显示现在的时间是九点二十二分。 
        “咱们该撤了!”舒克大声对正在打电话的皮皮鲁喊。 
        皮皮鲁看一眼表,又拨了一个号码。 
        打完最后一个电话,皮皮鲁站起来拿起他的资料箱。 
        皮皮鲁深情地看了一眼他的房间。他在这里生活了三十多年。 
        舒克驾驶直升机飞到阳台上吊起贝塔的坦克。 
        “你们悬停在空巾继续打电话,  直打到地震发生!我去通知这座楼上的住户。”皮皮鲁对舒克和贝塔说。 
        “你注意看时间!”舒克提醒皮皮鲁。 
        “我们在楼下等你!”贝塔把头探出机窗。 
        直升机离开阳台,升到空中。 
        舒克操纵飞机。贝塔继续打电话。 
        皮皮鲁拎着皮箱敲邻居的门。开门的是个妇女。 
        “快下楼,马上要地震!”皮皮鲁急促地说。 
        “地震?”妇女不信。 
        “没时间了,快下楼!”皮皮鲁用命令的口气说。 
        妇女知道皮皮鲁是科学家,她信了。 
        皮皮鲁用最快的速度挨个敲门通知邻居。 
        舒克的直升机悬停在楼旁的空中。 
        九点三十分。 
        “皮皮鲁怎么还没出来?”贝塔看见从楼里跑出许多人,可就是没有皮皮鲁。 
        “皮皮鲁准是在挨门挨户告诉人家!真糟糕,来不及了!!”舒克眼泪出来了。 
        九点三十一分。皮皮鲁还没有出来! 
        九点三十一分二十秒。 
        贝塔不顾一切地把头探出机窗,冲着楼里喊:“皮皮鲁,快出来!还有四十秒!” 
        九点三十一分五十秒!! 
        从楼里又跑出不少人。 
        九点三十一分五十五秒!!! 
        舒克的手颤抖了。 
        贝塔的心脏停止往全身供血。 
        皮皮鲁在本市是第一个知道即将发生地震的,可他现在还在楼里! 
        九点三十一分五十九秒。 
        大地震撼了。 
        山摇地裂。不知从哪传来的令人毛骨悚然的巨大声响袭击着人们那脆弱的神经,建筑物相继倒塌。几秒钟内,城市不复存在。   第121集 
        皮皮鲁的眼泪; 
        孩子们还在操场上等市长; 
        愤怒的市民包围市政府   
        大地震分秒不差地发生了。 
        舒克和贝塔在空中眼睁睁地看着一座座建筑相继倒塌,尘土和地光在空中搅拌在一起,葬身瓦砾的生命发出最后的呐喊。 
        “皮皮鲁!”舒克的眼泪夺眶而出。 
        皮皮鲁居住的大楼不是一下子倒塌的,而是采用缓慢的方式一层一层逐渐倒塌的。 
        “快到瓦砾堆里找皮皮鲁!”贝塔从电视上见过抢救地震后埋在废墟里的生命的场面。 
        舒克用袖子擦干眼泪,使用一个急速转弯的飞行姿态驾驶直升机朝残垣断壁俯冲下去。 
      “看!那是谁!!”贝塔喊起来。 
        舒克一看,废墟旁边站着皮皮鲁。只见他满脸是泪,左手拎着资料箱,右手抱着一个三岁左右的孩子。 
        舒克操纵直升机飞到皮皮鲁的身边。 
        “皮皮鲁,你怎么啦?”贝塔看见皮皮鲁两眼发直,呆果地看着废墟。 
        “我……没能……挽救……城市……”皮皮鲁喃喃地说。他的心碎了。 
        舒克和贝塔不说话了,他们理解皮皮鲁的心情。事先知道发生大地震的准确时间,却无法拯救所有生命,这对一个科学家的良心的折磨是残酷的。 
        “你已经做了你能做的一切。假如你说出孤岛,会给全球的人类带来灭顶之灾。”贝塔安慰皮皮鲁。 
        皮皮鲁长叹了一口气,看来他接受了贝塔的立论。 
        皮皮鲁是在地震发生的前一秒钟从一楼的窗户跳出米的,当时那家住户的女主人死活不信皮皮鲁的话,皮皮鲁一看表,离地震还有2秒,他强行从女主人怀里抢过孩子,破窗而出。 
        妈妈遇难,孩子得救了。 
        听了皮皮鲁的历险记,舒克和贝塔的后背直冒冷汗。 
        “你们快去市里的学校看看伤亡情况。”皮皮鲁还记挂着孩子们,他不知道是不是所有学校的校长都信市长要去视察的匿名电话。 
        “我陪你呆着,让舒克去。”贝塔不放心皮皮鲁。 
        舒克同意。他将直升机停在皮皮鲁身边的地面上,贝塔跳出飞机,把吊着坦克的钩子从坦克上脱开。 
        舒克操纵飞机起飞,用最快的速度绕全城侦察飞行一圈,舒克看见所有学校和幼儿园的孩子都站在操场上,他们望着倒塌的校舍发愣。 
        全市的在校学生和在园小朋友都得救了!当皮皮鲁得知这一信息后,笑了,虽然笑得挺苦,但毕竟是笑了。 
        根据当天官方公布的死伤人数,在这次大地震中,全市死亡人数为17万,伤72万,其他损失不可估量。 
        市长没有死,他记着九点三十二分这个时间,他秘密在电力、煤气、银行等重要部门采取了措施,虽然他不信皮皮鲁的话。 
        地震真的发生了。 
        当活着的市民们听说市长事先知道了要发生大地震,而且在某些部门采取了防范措施后,他们愤怒了。失去了亲人的市民们包围了市政府,他们要市长偿还血债。 
        “让市长出来!” 
        “你是双手沾满市民鲜血的刽子手!” 
        “你明知道九点三十二分要发生地震,为什么不告诉我们?” 
        “还我儿子!” 
        “还我丈夫!” 
        “还我母亲!” 
        数十万人包围了市长的办公室。   第122集 
        记者招待会扣人心弦; 
        市长把皮皮鲁推到全世界面前; 
        众矢之的   
        市长透过支离破碎的窗户注视着抗议的人潮,他怨恨皮皮鲁。 
        “啪!”市长拍桌子了,秘书吓了一跳。 
        “岂有此理!”市长怒不可遏,“身为科学家,测出了即将发生的大地震,却不告诉我是用什么方法测出的,这叫什么科学家?!造成这么大的伤亡!” 
        “是不像话!”秘书附和。 
        “马上开记者招待会,我要公布事实真相,以正视听。”市长冲秘书挥手。 
        一个小时后,本国和外国的众多记者云集到这座残城。 
        一个小时后,记者招待会在露天举行。找不到完整的会议室。 
        “我是××国××通讯社记者。请问市长先生,贵市市民传说巾长先生事先知道地震发生的精确时间,并且在小范围内采取了措施,有这回事吗?”一位外国记者首先提问。数百台录音机开录。 
        “我事先知道星期四上午九点三十二分本市将发生大地震。”市长镇静自若地回答。 
        记者招待会立刻炸了锅。记者们大眼瞪小跟。 
        “请问市长先生是用什么方法测出地震的?”有记者迫不及待地问。 
        要知道,预防地震是人类梦寐以求的愿望,是多少科学家为之终生奋斗的目标。该市长能如此精确地知道地震发生的时问,可见,人类在预见地震方面已经迈出了一大步。 
        “是物理学家皮皮鲁先生告诉我的。”市长卖了皮皮鲁。 
        皮皮鲁!记者们一愣。他们知道这位大名鼎鼎的物理学家,他今年再次被提名为诺贝尔物理学奖的候选人,且夺魁的可能性极大。 
        “请问市长先生,众所周知,皮皮鲁先生是物理学家,隔行如隔山,他怎么可能预报地震呢?”一位女记者提出疑问。 
        “请问市长阁下,既然皮皮鲁告诉了您贵市将发生地震,您为什么不在全市采取措施呢?”一位岁数挺大的记者问。 
        市长把皮皮鲁深夜找他的经过讲了一遍。然后说:“他不说出他是根据什么测出的地震,我能在全市采取措施吗?他如果是地震专家我还可以考虑他的活,可他偏偏是物理学家!” 
        记者们终于抓住了本世纪最大的爆炸性新闻,他们玩命地记录。 
        “请问您对这件事的判断?”一位记者发问。 
        市长顿了顿嗓子,一字一句地说:  “皮皮鲁先生成功地发明了一种仪器或一种方法,能够精确地预报地震,但他不愿意把这种仪器或方法公布于众,出于什么原因我不知道。但有一点我可以说,他违背了科学家的良心和道德准则!他可能没有触犯法律,但我认为在良心的法庭上,他应该对这数十万死难者负责,他应该接受审判!” 
        全场鼓掌。二十分钟后,全世界都知道了这条新闻。 
        皮皮鲁成了众矢之的。在人类文明高度发展的今天,人们不能容忍这种自私的科学家的存在。科学是人类的共同财富,任何人想把它据为己有都是不能允许的。 
        市政府在舆论的驱动下,决定向法院起诉皮皮鲁。   第123集 
        皮皮鲁上法庭; 
        诺贝尔奖评委会的决定; 
        皮皮鲁大脑中的“石油”无法开采   
        当皮皮鲁获悉市政府将向法院起诉他时,苦笑。他没想到事情发展到这一步。 
        “你救了那么多孩子,他们应该奖励你!”舒克为皮皮鲁打抱不平。 
        “应该向法院起诉那个市长!”贝塔忿忿不平。 
        “早知道这样,咱们真不该把孤岛上的地震控制中心告诉皮皮鲁。”舒克后悔了。 
        “还是告诉我好,不然那些孩子就没命了。”皮皮鲁不同意。 
        “法院如果判你有罪怎么办?”贝塔开始构思营救皮皮鲁越狱的方案了。 
        “我的律师是全市最好的律师。再说根据我掌握的法律常识,我还构不成犯罪。”皮皮鲁说。 
        城市展开重建工作,人们住在帐蓬里。皮皮鲁和舒克、贝塔也住在帐篷里。 
        法院开庭了。经过几天的激烈辩论,法院裁决皮皮鲁的行为没有构成犯罪。但法官同时说,皮皮鲁的行为违背了一个科学家应有的道德,这种不人道的行为应该受到舆论的谴责。 
        全世界数百位著名科学家联合发表了一份声明,他们一致谴责皮皮鲁,并宣布今后不再和皮皮鲁合作进行任何项目的科学研究。 
        诺贝尔物理学奖评委会宣布取消皮皮鲁的评选资格。 
        世界物理学学会决定开除皮皮鲁的会籍。报纸上攻击皮皮鲁的文章不胜枚举。有的文章还追根寻源,说皮皮鲁从小就品质不佳,上小学时就爱恶作剧,还说什么三岁看大七岁看老。 
        皮皮鲁成了人民公敌,他现在不能看报不能看电视不能听广播。只要他打开这些东西,就都是骂他的内容。 
        尽管如此,皮皮鲁还是坚决不说出孤岛,连舒克和贝塔都动摇过,他们不愿意看着皮皮鲁为了保护地球的安全而蒙受如此大的冤屈,他们鼓励皮皮鲁和盘托出地震控制中心,地球爱怎么着就怎么着,爆炸也好,打世界大战也好,都没关系。关键是维护皮皮鲁的名誉。再说舒克认为这些人也不值得保护。 
        皮皮鲁不同意这么做。他认为个人的名誉固然重要,但地球的安全和人类的延续更重要。 
        舒克和贝塔不得不钦佩皮皮鲁,人类居然有如此大的忍受冤屈的能力。贝塔认为人类的诸种忍耐力中,忍受委屈的耐力最强。 
        皮皮鲁忍受得了委屈,但他无法忍受不能进行科学研究的痛苫,人们剥夺了他进行科研的权利.没有实验室,没有科研项目,没有合作者,甚至连图书馆的借书证也被收回了。皮皮鲁天天望着天空发呆。 
        舒克和贝塔看着皮皮鲁,他俩难过极了。他们清楚,像皮皮鲁这样的充满智慧的大脑在人类中是不多的,就像有的国家石油蕴藏丰富而有的国家根本没有石油一样,是天生的,是上帝的安排。像皮皮鲁这样允满“石油”的大脑不能为人类服务,可惜。 
        皮皮鲁不能出门。不能见人。他每天的工作就是叹气。   第124集 
        令人兴奋的激光炮和皮皮鲁的体重; 
        无与伦比的现代化飞行器   
        舒克和贝塔不得不承认,人类与自然灾害抗争的能力很强。短短几个月,遭受地震淫威损害的城市已经恢复了正常。市民迁人了新居,皮皮鲁也得到了一套面积不大的住房。报纸还专门为此发表了一篇短评,题目是《对于不人道的科学家仍然要坚持人道主义》。 
        皮皮鲁已无法在这座城市生活,他不得不和舒克和贝塔把家迁移到另一座城市。 
        现在舒克和贝塔成了皮皮鲁惟一的朋友,如果没有他俩朝夕同皮皮鲁相处,皮皮鲁很可能无法忍受这种寂寞的生存环境。 
        大脑不能思维,皮皮鲁感到度日如年。他的手痒痒,他的脑细胞也痒痒。皮皮鲁每天像被关进笼子里的老虎那样在屋里焦躁不安地来回走动。 
        “咱们得想个办法。”舒克认定如此下去皮皮鲁会生大病。 
        “他每天吃的东西和咱们差不多。”贝塔知道皮皮鲁的体重在急剧下降。 
        “像皮皮鲁这样的大脑不能为世界工作真可惜。”舒克认为皮皮鲁的大脑能顶得上一万个一般人的大脑。 
        “人家不让他搞发明研究,咱们给他出题目怎么样?”贝塔眼睛一亮。 
        “好主意!”舒克使劲儿拍贝塔的肩膀。 
        舒克和贝塔开始绞尽脑汁给皮皮鲁出科研项目,他俩几乎把地球上的东西筛了一遍。 
        “让皮皮鲁给咱们的直升机设计一台激光炮怎么样?”舒克提议。 
        “这主意不错。难度大,又实用。”贝塔深知像皮皮鲁这种高档次的科学家喜欢研究难度大又有实用价值的项日。 
        这天晚饭后,舒克和贝塔同皮皮鲁谈话。 
        皮皮鲁无精打采。 
        “我们想请你帮个忙。”贝塔对皮皮鲁说。 
        “请我帮忙?”皮皮鲁感到贝塔口气不正常。 
        “噢,是这样。”舒克接过话头,“我的直升机上的武器系统比较薄弱,你能不能帮我们给直升机设计一门激光炮?” 
        皮皮鲁的瞳孔里射出了奇异的光束。 
        “行吗?”贝塔看出他们的计划成功了。 
        “当然行。说干就干。”皮皮鲁一跃而起,浑身顿时显示出有使不完的智慧。 
        舒克和贝塔美滋滋地看皮皮鲁翻资料做准备工作。 
        舒克和贝塔这才知道创造性劳动在人生中占有什么样的位置,吃喝玩乐是享福,但它们同创造性劳动比起来,实在是一种渺小的享福。人生真正的享福是进行前所未有的创造性劳动。 
        皮皮鲁像变了一个人,吃得多了,体重增加了,精神抖擞了,就因为他的大脑开动了。 
        皮皮鲁不愧是第一流的天才,他只花了三天就给直升机装上了一门微型激光炮。 
        舒克迫不及待地要试射。 
        直升机停在桌子上,舒克和贝塔钻进飞机。皮皮鲁坐在沙发上。 
        “请求试飞!”舒克请示地面。 
        “同意起飞!”皮皮鲁兴致勃勃。 
        直升机呼啸着离开桌面,升到空中。 
        “请求试炮。”舒克皓准了事先画在墙上的靶心。 
        “开炮!”皮皮鲁下令。 
        舒克按下了射击按钮。 
        一道纤细均匀的光柱射穿了厚厚的墙壁,在墙壁上留下了一个小洞,从洞里可以看到屋子外边。 
        “成功了!”兴奋中舒克差点儿把直升机撞到墙上。 
        “充满智慧细胞的人脑真是无价之宝!”贝塔感慨地说。 
        “你看皮皮鲁。”舒克的声音有些惊讶。 
        只见皮皮鲁坐在沙发上发愣。 
        舒克把飞机降落在沙发旁的茶几上。 
        “你怎么了?皮皮鲁。”贝塔从机窗上探头问。 
        皮皮鲁像从沉思中清醒过来。 
        “我要给你们设计一架最先进最现代化集人类科学成就于一身的飞行器。”皮皮鲁一口气说完。 
        舒克和贝塔愣了。他们激动得说不出话来。 
        对于皮皮鲁来说,终生奋斗目标就是进入想像的极限。 
        舒克和贝塔将结束驾驶直升机和坦克的历史,他们即将拥有一架名叫五角飞碟的极其现代化的超时空飞行器。这架飞行器的诞生将影响人类历史的进程,人类做梦也没想到,两只老鼠驾驶着五角飞碟将同他们发生怎样的关系。   第125集 
        划时代的航空器诞生; 
        五角飞碟试飞; 
        舒克历史性地按下起飞按钮   
        皮皮鲁把自己反锁在房间里,埋头进入五角飞碟的设计中。他好像变了一个人,全身焕发出令人兴奋不已的光环。 
        有的人的大脑生出来就是搞创造的,这种大脑最渴望的就是思维、运转,想别人想不到的事,想别人说不出的话。如果不让这样的大脑工作,就等于扼杀了它。自从地震事件以来,皮皮鲁的大脑就处于被扼杀状态。现在,他终于又可以发挥自己的智慧进行创造性劳动了。 
        舒克和贝塔再也不用为皮皮鲁的精神状态和身体健康发愁了,他们总算知道了有益的工作是最好的补药。 
        “我有预感,我们好像要告别直升机和坦克了。”舒克望着皮皮鲁房间的灯光对贝塔说。 
        “我还真有点儿舍不得。”贝塔看了一眼停在他身边的坦克。 
        “皮皮鲁已经干了几天了?”舒克问。 
        贝塔看了看日历,说:“哟,已经干了十天了。这十天他几乎没和咱们说话,他也不觉得闷。” 
        “这就叫进人状态。这才是享受呢,好像整个世界就一个人。自己就是世界。”舒克为皮皮鲁高兴。 
        “啊——”从皮皮鲁房问里突然传出一声极度兴奋的呐喊。 
        舒克和贝塔对视,他们觉得有好事的可能性比较大。 
        “舒克,贝塔,你们快来!!”皮皮鲁在自己的房间里喊。 
        舒克和贝塔认定大功告成了。他们握紧拳头在胸前抖动以示兴奋,然后竞赛似地冲进皮皮鲁的房间。 
        “你们看!”皮皮鲁指着桌子上对舒克和贝塔说。 
        “哇——”舒克和贝塔异口同声地说了一句从电视广告上学来的俗得不能再俗的感叹词。 
        一架黑色的有五个角的飞碟浑身透着威风凛凛的气质静卧在桌子上。任何人一眼就能看出,它是极度智慧和高科技的结晶。 
        五角飞碟的直径为50公分,厚度为15公分。 
        “人太厉害了!”舒克感叹道。 
        “喷喷……”贝塔看看五角飞碟,再看看皮皮鲁,什么也说不出来。 
        “你们进去看看。”皮皮鲁说。 
        舒克和贝塔走进五角飞碟,这里是一个虚幻般的世界,各种仪器设备全部由电脑控制。飞碟里除了工作舱外,还有全套现代化生活设施,包括寝室、卫生间、餐厅和娱乐室等。 
        舒克和贝塔把每个舱都参观了一遍。 
        “皮皮鲁,你太伟大了!”贝塔从飞碟里探出头来对皮皮鲁喊。 
        “你们还不知道它的功能和本事呢!”皮皮鲁得意地说。 
        “你快说!”舒克跳出飞碟。 
        “首先,咱们的飞碟用光速飞行,也就是说,每秒钟能绕地球飞行7.5圈。”皮皮鲁喝了一口咖啡。 
        “哇——”贝塔又俗了一次。 
        “它不光能在空中飞行,还能在水下飞行。” 
        舒克和贝塔的嘴巴“哇”不出声来了。 
        “它高度灵活,能在室内快速飞行。它还配有遥感系统,能监视数千公里外的景物。”皮皮鲁眉飞色舞。 
        “快说说它的武器系统。”舒克最关心这个。 
        “这种武器目前可必须保密,如果让人类掌握了,地球的末日就到了。所以,咱们的五角飞碟绝不能让别人得到。”皮皮鲁想起了发明原子弹的爱因斯坦,脸上掠过一丝阴影。 
        “它可以在5秒钟内同时击毁500架飞机。可以在三万米高空准确无误地击中地面上的一只蚂蚁。”皮皮鲁说。 
        舒克和贝塔的心跳几乎停止了,他们清楚这五角飞碟将赋予他们“超人”的头衔。 
        “最精彩的,还要算是它的通讯系统。”皮皮鲁指指桌上的一个小方盒子,“我在这房间里,可以指挥你们在全世界的任何地方飞行,还可以通过飞碟上的摄像装置发回的信号再从我的屏幕上观察到你们的行踪。你们也可以在飞碟上随时同我保持联系。如果我离开房间,就带上这个。”皮皮鲁从衣兜里掏出一个火柴盒大小的金属盒子,“走到哪儿都能同你们通话。” 
        “我想试飞。”舒克说。 
        “我也想。”贝塔手痒痒极了。 
        “我可是饿得不行了,先吃饭。”皮皮鲁反对。 
        舒克和贝塔表示遗憾。 
        美餐一顿后,皮皮鲁教舒克和贝塔驾驶五角飞碟的技术。 
        “驾驶五角飞碟比驾驶直升机和坦克容易多了,五角飞碟有自动驾驶仪、自动导航仪和自动跟踪仪,操纵也很简单,会按几个按钮就行了。”皮皮鲁说。 
        “五角飞碟飞行速度这么快,如果撞到楼上怎么办?”舒克问。 
        “五角飞碟是绝不会撞在墙上的,它有自动防撞装置。”皮皮鲁说。 
        “真够火的。”贝塔使用刚学到的地方俗语。“火”就是厉害的意思。 
        “现在咱们先进行室内试飞。”皮皮鲁宣布。 
        “是!”舒克和贝塔立正。 
        “登机!”皮皮鲁命令。 
        舒克和贝塔进人五角飞碟,机舱门自动密封。 
        五角飞碟的驾驶舱有两个操纵台,1号操纵台负责驾驶和通讯,2号负责武器系统和遥测、监视。 
        “谁当驾驶员?”舒克和贝塔同时问对方。 
        都想当。 
        “轮流当吧。”贝塔想出皆大欢喜的方案。 
        “我先当?”舒克想享受试飞的荣誉。 
        “你担任室内试飞,我担任室外试飞。”贝塔一笑。 
        舒克无奈,一会儿只好让贝塔负责室外试飞。 
        舒克坐在l号操纵台前,打开飞碟的总开关,驾驶舱内的各种仪表上的指示灯魔术般地竞相闪耀起来,多得像天上的星星。 
        “五角飞碟准备试飞。”皮皮鲁的声音叫荡在飞碟里。 
        “五角飞碟准备完毕。”舒克心潮澎湃。 
        “起飞。”皮皮鲁下令。 
        舒克庄严地按下起飞按钮。 
        贝塔羡慕得从头痒到脚。 
        五角飞碟的碟身开始原地旋转。渐渐地,飞碟离开了桌面,平稳地升到空中。 
        确实壮观。 
        五角飞碟时而像一只滑翔的鹞鹰,舒展着翅膀缓缓地掠过皮皮鲁头顶。时而像一只劈浪的海豚,曲线优美地从皮皮鲁两侧擦身而过。时而又像猛虎,呼啸着朝皮皮鲁扑过来,可就在撞上的刹那间,突然一个急转弯,化险为夷。 
        驾驶五角飞碟的确是一种享受,高科技的享受。 
        “感觉如何?”皮皮鲁问。 
        “一切良好!棒极了!”舒克激动地回答。 
        “准备室外试飞,同时测试武器系统。”皮皮鲁命令。 
        “请示试验项目?”贝塔要担任室外试飞员了。 
        “先在O.2秒内绕地球飞行一圈。”皮皮鲁下达试飞项目。 
        “明白!!!”贝塔不是回答,是回喊。   第126集 
        光速飞行的感觉; 
        运气坏到家的歹徒; 
        成功后皮皮鲁的脸上没有笑容 
        皮皮鲁打开窗户。窗外已是繁星满天。 
        贝塔将试飞项目指标输人电脑,然后打开自动驾驶仪。 
        舒克羡慕地看着坐在驾驶台前的贝塔。 
        “绕地球一圈试飞准备工作完毕。”贝塔向皮皮鲁通报。 
        “起飞!”皮皮鲁命令。 
        贝塔庄严地按下起飞按钮,他的食指上全是神圣。 
        五角飞碟瞬间就射出了皮皮鲁的房间。 
        舒克和贝塔还是头一次乘坐以光速飞行的飞行器,他们有一种自己和宇宙融合在一起的强烈感觉。 
        0.2秒后,五角飞碟绕地球一周后回到皮皮鲁的桌子上。 
        舒克和贝塔从飞碟里跳出来,激动地冲着皮皮鲁喊:“成功啦!成功啦!” 
        皮皮鲁满脸放光,他清楚地知道制造出用光速飞行的飞行器对于人类来说是一次革命。 
        “还不能算完全成功。咱们还得试试它的普通速度飞行和武器系统。”皮皮鲁说。 
        “这回该我驾驶了。”舒克冲贝塔挤挤眼睛。 
        “这次试验很重要。你们要对五角飞碟的各种飞行状态进行试验,还要试验它的遥感装置和武器。”皮皮鲁说。 
        “你放心吧,会成功的。”舒克边朝五角飞碟走去边说。 
        贝塔关舱门前还冲皮皮鲁敬了个礼。 
        “一切准备完毕,五角飞碟请求起飞。”舒克说。 
        “同意起飞!”皮皮鲁批准。 
        这回五角飞碟是缓慢地离开桌面的,它先在屋里飞了一圈,然后离开了房间,从窗户飞了出去。 
        “这才叫驾驶。”舒克得意地对贝塔说,“用光速飞行没劲,什么也看不见。” 
        “现在这速度和直升机没什么两样。”贝塔反唇相讥。 
        “悬停空中。”扩音器里传来皮皮鲁的指令。 
        “明白。”舒克回答。 
        舒克操纵飞碟悬停在空中不动。 
        “一切正常。”舒克向皮皮鲁汇报。 
        “从悬停状态突然进入光速飞行。”皮皮鲁指示。 
        “明白!”舒克操纵五角飞碟从悬停状态突然进入光速飞行。 
        “一切正常。”舒克告诉皮皮鲁。 
        “空中悬停!”皮皮鲁命令。 
        五角飞碟稳稳地停在空中。 
        “太棒了!”贝塔从座椅上跳起来。 
        “打开遥感器。”皮皮鲁下达试验项目。 
        五角飞碟的遥感荧屏亮了。舒克通过电脑指挥遥感器进行全球范围的扫描搜索。 
        地球各个角落的事轮番出现在荧光屏上。有大饭店,有学校,有体育场,有家庭,有医院,有总统府…… 
        皮皮鲁在家里的荧光屏上也能看到这一切。 
        一个镜头引起了皮皮鲁的注意。 
        “舒克,舒克,请重复刚才的画面。”皮皮鲁说。 
        舒克执行。 
        屏幕上显示出一个家庭。家庭中的每一个成员都愁眉苦脸,焦急万分。 
        “定向观察。”皮皮鲁指示。 
        画面停留在这个家庭中。舒克和贝塔死盯着屏幕。 
        原来,这家的孩子被歹徒绑架了,他们正一筹莫展。 
        “遥感搜查,找到那帮坏蛋!”皮皮鲁说。 
        舒克开始操纵遥感器在方圆一百里内扫描搜索。 
        “在那儿!”贝塔先发现了目标。 
        几个神色紧张的男人在一座房子里向窗外窥视,房间的角落里坐着一个六岁左右的女孩子。 
        “怎么还不送钱来?”一个大胡子边看表边说。 
        “过了九点钟就撕票。”一个戴眼镜的说。 
        舒克看了一眼表,现在是八点五十三分。 
        “就拿这帮坏蛋试试飞碟的武器系统吧!”皮皮鲁说。 
        “一举两得。”贝塔点点头。 
        “使用连续射击,打断他们所有人的双腿。”皮皮鲁挺恨这类绑架儿童的歹徒。 
        贝塔将皮皮鲁的指令编成程序输入武器系统的电脑。 
        “射击。”皮皮鲁命令。 
        贝塔按下了射击按钮。 
        “啊——”大胡子惨叫一声,倒在地上。另外几名歹徒相继断腿。 
        女孩子惊讶地看着眼前发生的不可思议的事。 
        “五角飞碟的武器太棒了!”舒克赞不绝口。 
        “你也不看看是谁操纵的。”贝塔得意非凡。 
        “注意那个大胡子!”皮皮鲁提醒五角飞碟上的两位勇士。 
        屏幕上的大胡子正朝女孩子爬过去。这是个亡命徒,他想杀人。 
        贝塔轻而易举就击穿了大胡子的两只手掌。 
        歹徒们都老实了,乖乖地躺在原地。他们没听见枪响,没看见人影,不知道警察是怎么发现他们又使用什么武器制服他们的。 
        “通知当地警察去抓他们。”皮皮鲁说。 
        舒克按了通讯系统电脑键盘上的几个键,电脑立即接通了当地警察局的电话。 
        “是警察局吗?”舒克问。 
        “是,什么事?”警察说。 
        “有个女孩子被绑架了,你们知道吗?” 
        “不知道。” 
        “那个女孩子被关在……”舒克扭头求贝塔,“快查查门牌号码。” 
        贝塔在电脑前一阵忙,关女孩子的那栋楼房的门牌号码显示在荧光屏上。 
        舒克把门牌号码告诉警察局。 
        “你是谁?”警察问。 
        “无可奉告。你们快去吧!说不定上司还会给你加薪呢!”舒克笑笑。 
        几分钟后,警察包围了那栋房子。 
        手持杀伤武器的警察小心谨慎地接近房子,舒克和贝塔觉得好笑。 
        歹徒们束手就擒。女孩子回到了家人的怀抱。 
        “返航!”皮皮鲁一声令下。 
        五角飞碟一个急转弯,转眼就落在了皮皮鲁的桌子上。 
        试飞宣告成功。试验结果证明,五角飞碟是目前人类中最先进最现代化最超前最神通广大的飞行器。 
        舒克和贝塔争先恐后地离开飞碟,跑到皮皮鲁身边,他们要向皮皮鲁祝贺。他们觉得,就凭这个飞碟,皮皮鲁得十次诺贝尔奖也不为过。 
        皮皮鲁脸上却没有笑容,只见他表情严肃地看着舒克和贝塔。 
        舒克和贝塔面面相觑,不知皮皮鲁是怎么回事。   第127集 
        皮皮鲁、舒克和贝塔庄严起誓; 
        爱喝酒的贝塔; 
        不敢收乘车费的出租汽车司机   
        “咱们三个得发个誓。”皮皮鲁严肃地对舒克和贝塔说。 
        “发誓?”舒克看看贝塔,不明白皮皮鲁是什么意思。 
        “这五角飞碟的问世非同小可。它能够促进世界和平也能毁灭人类,就看谁使用它了。”皮皮鲁一字一句地说。 
        舒克和贝塔点头,他俩明白皮皮鲁的意思了。 
        “发誓吧。”皮皮鲁伸出右手的食指。 
        舒克和贝塔把手放在皮皮鲁的食指上。 
        “我发誓我绝不利用五角飞碟干坏事。”皮皮鲁先发誓。 
        舒克和贝塔跟着发誓。 
        “不让五角飞碟落人坏人手中。”皮皮鲁又补充了一句。 
        舒克和贝塔跟着补充誓词。 
        “好了,咱们庆祝吧!”皮皮鲁的脸上出现了笑容。 
        “喝酒!”贝塔不知从什么时候起喜欢上了喝酒,他觉得酒后那种晕晕乎乎的感受令人向往。 
        皮皮鲁从玻璃柜中取出一瓶葡萄酒。 
        “为五角飞碟的试飞成功,干杯!”贝塔拿起酒杯,一饮而尽。 
        皮皮鲁也喝光了一杯。 
        舒克怵酒,他只沾了沾嘴唇。 
        “这可不行,你得喝光。”贝塔逗舒克。 
        “我不喜欢喝酒。”舒克坚持不喝。 
        “这酒一定要喝。这可是为五角飞碟干杯呀!”贝塔甚至觉得劝别人喝酒比自己喝酒还刺激。 
        舒克用眼光向皮皮鲁求援。 
        “少喝点儿吧!”皮皮鲁不偏不倚,公平对待两位朋友。 
        舒克皱着眉头喝了一口。 
        贝塔耸耸肩膀,表示无可奈何,他拿过舒克的酒杯,一仰脖喝干了杯中的酒。 
        皮皮鲁从冰箱中拿出准备好的食品,摆了满满一餐桌。 
        “他是有准备的呀!”贝塔冲舒克作了个鬼脸。 
        “皮皮鲁真有自信。他认定自己的杰作一定会成功,所以他事先连庆祝用的美酒佳肴都准备好了。”舒克说。 
        的确,自信与成功总是相伴而行。 
        庆祝宴会开始。 
        皮皮鲁坐在餐桌旁。舒克和贝塔坐在餐桌上。 
        近水楼台先得月。舒克和贝塔大口大口地进食,他们的肠胃对于接受这样的食物感到惬意。 
        皮皮鲁若有所思地看着餐桌上的两位朋友,他的眼角渐渐湿润起来。偌大世界,少说也有五十亿人类,竟然没有一个人理睬他。如果没有这两只小老鼠,他怎样去体验生命的温暖呢? 
        “你怎么了?”舒克先发现皮皮鲁的面部表情与庆功宴会的气氛不匹配。 
        “我真得感谢你俩。”皮皮鲁用餐巾纸轻轻擦去眼角的泪花。 
        “应该是我们感谢你。”舒克明白了皮皮鲁话里的含义。他想起了自己小时候和妈妈在洞里挨饿的情景,心头一阵发酸。 
        贝塔又喝干了一杯酒,对皮皮鲁说:“别理他们!我最讨厌那些道貌岸然的伪君于,和他们来往折寿。我看你就这么自己呆着最好。肮脏是人与人交往的润滑剂。不交往了,也就清白了,你说对吗?” 
        皮皮鲁笑了。他不得不承认,贝塔的话有几分道理。 
        贝塔忽然发现舒克的表情不对,他忙问:“舒克,你怎么了?” 
        “肚子疼……”舒克双手捂着肚子。 
       “怎么搞的?是吃得不舒服吗?”皮皮鲁问。 
        “不知道……突然疼的……哎哟,好疼……”舒克的脸上开始冒汗。 
        “什么地方疼?”皮皮鲁问。 
        “这儿……”舒克用手指。 
        皮皮鲁知道哪儿疼也没用,鬼知道老鼠肚子里的结构和人的一样不一样。 
        皮皮鲁印象中,舒克和贝塔还没生过病。 
        “哎哟,疼死了……”舒克开始在餐桌上打滚儿,接连碰翻了好几个碗和杯子。 
        贝塔手足无措。 
        “得马上送医院。”皮皮鲁决定。 
        “送医院?送什么医院?”贝塔问皮皮鲁。 
        是啊,尽管当今医疗技术如此发达,可地球上绝对没有一座为老鼠治病的医院。别说医院,就连为老鼠治病的医生也没有一个。 
        “医院光拿我们老鼠做实验,却没有一个医生去研究怎么为老鼠治病!”贝塔义愤填膺。 
        “真是!”皮皮鲁头一次从这个角度想问题,他感到人类那个了点儿。 
        舒克疼得忍无可忍,一声声喊叫撕裂着皮皮鲁和贝塔的心。 
        根据常识.皮皮鲁判断舒克不是胃穿孔就是阑尾炎,必须立即送医院手术治疗。 
        “我送他去医院!”皮皮鲁一拍桌子,他只能这样做。 
        一个人类的成员带一只老鼠去医院看病,其难度可想而知。贝塔对结局不乐观。 
        “你在五角飞碟里待命,我带上微型通讯器,咱们随时联系。”皮皮鲁对贝塔说。 
        贝塔点点头。 
        皮皮鲁用手帕将舒克的身体包起来放进自己的上衣口袋里。 
        贝塔目送着皮皮鲁离开家。他随后钻进五角飞碟,等候皮皮鲁的消息。 
        外边是繁星满天的夜色。 
        皮皮鲁拦了辆出租车。 
        “去哪儿?”出租车司机问。 
        “市第一医院。”皮皮鲁想让舒克享用最好的医疗条件。  “请快点儿,是送急病人去医院抢救。” 
        “病人在哪儿?”出租车司机回头看,只皮皮鲁一人。 
        “病人在……”皮皮鲁不知道该说什么好。 
        “先去接病人?”出租车司机问。 
        “不,不用了,直接去医院。”皮皮鲁无法自圆其说。 
        出租车司机从后视镜里结结实实看了皮皮鲁一阵,他怀疑皮皮鲁这个前言不搭后语的乘客是劫持出租车的惯犯。 
        “快开车呀!”皮皮鲁催促。 
        司机只好硬着头皮启动汽车。他一只眼睛看前边,一只眼睛看后视镜,一路提着心把皮皮鲁送到了医院急诊室门口。停车后,司机连钱都不敢收,等皮皮鲁下车后,忙关上车门一溜烟跑了。 
        急诊室的门口冷冷清清的。 
        皮皮鲁运了运气,朝急诊室走去。   第128集 
        医生拒绝给舒克治疗; 
        皮皮鲁的火气; 
        记者丈夫猎奇; 
        化验室的尖叫   
        急救室里坐着一位医生,一位hushi。 
        听到皮皮鲁的脚步声,医生朝门口看。 
        他觉得皮皮鲁面熟,可一时又想不起来在哪儿见过。 
        “是那个预报地震的物理学家!”hushi记忆力超众,她小声告诉医生。 
        “没错!”医生连连点头肯定搭档的判断。 
        “您看病?”医生问皮皮鲁。 
        “我……”皮皮鲁迟疑地看了hushi一眼,他觉得只能单独和医生谈。“我能单独和您谈谈吗?” 
        “单独和我谈?”医生看了看hushi,“为什么?谈什么?” 
        “我……”皮皮鲁无法回答医生的问题。 
        “如果不是急病,请您明天再来去门诊看病。”医生对皮皮鲁印象不佳,他的几位嫡亲死于邻城的那次地震。 
        “是急病!”皮皮鲁忙说。 
        “患者是您?”医生上下打量皮皮鲁,看不出他有什幺急病。 
        皮皮鲁点点头,接着又摇头。 
        “我还是想和您单独谈谈。”皮皮鲁再次请求。 
        医生断定皮皮鲁是得了什么见不得人的病,出于职业道德,他冲hushi努努嘴,示意她回避一下。 
        hushi白了皮皮鲁一眼,老大不情愿地出去了。其实喜欢猎奇的她并未走远,就躲在门口偷听。 
        “您说吧。现在就咱们两个人了。”医生看着皮皮鲁的眼睛说。 
        “我有个朋友得了急病,想请您帮助治治。”皮皮鲁说。 
        “让他来呀!不管是不是您的朋友,我们都责无旁贷地给治病。”医生觉得皮皮鲁不正常。要么他的朋友就是逃犯。 
        “他已经来了。不过请您不要吃惊,一定要设法救救他。”皮皮鲁说。 
        “来了?在哪儿?”医生越来越感到蹊跷。 
        皮皮鲁谨慎地从口袋里掏出舒克,将他放在左手掌上,托到医生眼前。 
        “你这是干什么?”医生不明白这位声名狼藉的前物理学家深更半夜拿着一只老鼠来医院捣什么乱。 
        “他刚才突然肚子疼,我估计是阑尾炎或胃穿孔,请您救救他。”皮皮鲁诚恳地说。 
        医生的两道眉毛迅速靠拢,他显然被气坏了,他认为皮皮鲁是在侮辱他。 
        “你给我出去!这里是医院,不是疯人院!”医生火冒三丈。 
        舒克显然已经支撑不住了,他进八昏迷状态。 
        皮皮鲁急了,他的理智失去了控制,他冲着医生大吼道:“你不能见死不救!你的职责是拯救生命!他虽然是老鼠可他也是生命!你们光拿人家做实验,却不绐人家治病,你们算什么医生!算什么医院!” 
        医生被皮皮鲁的吼声震住了,他呆呆地看着面前这位曾经声名显赫的物理学家,他的脑子里像被注入了万能胶,细胞们无法正常运转。 
        躲在门外偷听的hushi兴奋了,她认定这是大新闻,忙跑去给丈夫打电话——她丈夫是一家无聊小报的记者。 
        医生的大脑渐渐从痴迷状态回复正常。 
        “你让我给这只老鼠看病?”医生问皮皮鲁。 
        “对。”皮皮鲁点头,“我重谢你。” 
        “我只会给人看病,我不是兽医。再说,兽医也没有会给老鼠治病的。”医生提出了技术问题。 
        “你就拿他当人看。”皮皮鲁再次把舒克捧到医生眼前。 
        显然是“重谢”的许诺打动了医生的心,他眯起眼睛观察舒克。 
        “它什么地方疼?”医生问皮皮鲁。 
        皮皮鲁指给医生看。 
        “他告诉我的。”皮皮鲁说走了嘴。 
        “它告诉你的?!它会说话?”医生基本断定皮皮鲁是心理变态者。 
        皮皮鲁不吭声了。 
        “可能是阑尾炎,还要化验一下血才能证实。”医生开化验单。 
        皮皮鲁不知怎么带舒克去化验室抽血。 
        “姓名这一栏我怎么填?”医生抬头问皮皮鲁。 
        “舒克。舒服的舒,巧克力的克。”皮皮鲁告诉医生。 
        医生的笔仿佛在空中凝固住了,半天才落到化验单上。 
        “性别?” 
        “男。” 
        “年龄?” 
        “33岁。” 
        医生再次抬头看皮皮鲁。 
        “老鼠能活33年吗?”医生不信。 
        “他特殊。他去过外星球。”皮皮鲁解释道。 
        医生这次没看皮皮鲁,他已经适应了。 
        就在医生写化验单时,闪光灯亮了。 
        皮皮鲁忙收回托着舒克的手。他回头一看,一个手拿照像机的男子正在门口连续拍照。 
        “你干什么?”皮皮鲁愤怒了。 
        闪光灯继续闪,男子不理睬皮皮鲁。 
        皮皮鲁用另一只手挡住脸。 
        hushi叫来了当记者的丈夫。记者丈夫在电话里死活不信妻子提供的信息,他还嘲笑妻子开的这个愚人节玩笑太拙劣。直到妻子威胁说他如果不来就和他离婚,他才急忙赶到医院。 
        跟前的景象令他吃惊:前物理学家皮皮鲁手捧着一只老鼠站在医生旁边,而医生居然在给老鼠开化验单! 
        记者丈夫服务的小报就缺这样的新闻!他差点儿把照像机的快门按碎了。 
        看着已奄奄一息的舒克,皮皮鲁顾不上和记者丈夫纠缠,他往化验室跑。 
        记者丈夫迅速对医生进行采访。 
        “的确不可思议。”医生把详细经过添油加醋地向搭档的先生描述。 
        记者丈夫眉飞色舞地记录。 
        hushi在一旁帮助丈夫录音。 
        “我先去报社发稿,一会儿再回来。”记者丈夫拔腿就跑。 
        “我帮你盯着!”hushi是个十足的事儿妈。 
        化验室的化验员接过皮皮鲁递进窗口的化验单看了看,说:“伸手。” 
        从手指上取血。 
        皮皮鲁把捧着舒克的手伸进去。 
        “啊——”窗口里一声尖叫。 
        化验员小姐显然怕老鼠。 
        “流氓!”她声嘶力竭地喊。 
        皮皮鲁不知所措。 
        “你别喊,你听我说。这化验单就是医生给这只老鼠开的,他可能得了急性阑尾炎,请你帮忙给他验血。”皮皮鲁尽量将自己的声音调得柔和些。 
        “给老鼠抽血?”化验员还是不肯接受这个现实。 
        “我来帮你给他抽,你光化验就行了。”皮皮鲁估计一百年之内这位小姐也不会同意自己给老鼠抽血。   第129集 
        藏在抽屉里的录音机; 
        记者丈夫在主编的桌子上写稿; 
        主任家开一道门缝儿   
        化验员小姐看着皮皮鲁发愣。 
        “给我针,我给他抽血。”皮皮鲁伸手向化验员要钢针。 
        化验员从盘子里取出一根消过毒的钢针,她犹豫了一下,把钢针递给皮皮鲁。 
        皮皮鲁对舒克说:“舒克.你忍着点儿,我在你耳朵上扎一下。” 
        化验员像看童话剧。 
        hushi躲在一旁拼命记住每一个细节。 
        舒克的血取出来了。化验员小姐不得不坐在显微镜前给一只老鼠验血。 
        “有炎症。”化验员将化验单递给皮皮鲁。 
        “谢谢你。”皮皮鲁接过化验单往急诊室跑。 
        医生听见皮皮鲁的脚步声,他按下了搭档的记者丈夫留卜的微型录音机上的录音按钮。 
        皮皮鲁将化验单放在医生的桌子上。 
        医生一看化验单就说:“急性阑尾炎。” 
        “需要手术?”皮皮鲁问。 
        “对。如果耽误,就会穿孔。”医生看着皮皮鲁说, 
        “你可以给他做手术吗?”皮皮鲁问。 
        “对不起,我是内科医生。再说,就是外科医生,也不会给一只老鼠做手术。”医生摇摇头。 
        皮皮鲁觉得医生的话有道理,会给人做外科手术的医生未必能给老鼠做手术。他的头上开始出汗。 
        皮皮鲁从兜里掏出舒克,舒克已经奄奄一息了。 
        “舒克!舒克!你再坚持一会儿!”皮皮鲁大声呼唤。 
        医生将抽屉拉开一条缝儿,他看了一眼藏在抽屉里的正在工作的微型录音机,他想起搭档的记者丈夫叫嘱他的尽量多诱导皮皮鲁说话的吩咐。 
        “我建议你带这只老鼠去医科大学的动物解剖实验室,那儿的教授们经常解剖小白鼠,也许他们能给老鼠做阑尾炎手术。”医生故意反复强调老鼠。 
        皮皮鲁眼睛一亮,说完谢谢拔腿就跑。 
        来到急诊室外边,皮皮鲁才想起现在是深夜,医科大学的动物解剖室根本不会有人。 
        皮皮鲁从上衣口袋里掏出同五角飞碟联系用的微型通讯器。 
        “贝塔!贝塔!我是皮皮鲁,你听见了吗?”皮皮鲁对着通讯器呼叫。 
        “我是贝塔。我听见了。舒克怎么样?”贝塔的声音里全是焦急。 
        “你现在马上用电脑查询医科大学动物解剖实验室主任家的地址,快!”皮皮鲁说。 
        “明白!”贝塔回答。 
        再说那位hushi的记者丈夫从医院出来后直奔报社,他叫醒了正在睡觉的夜班主编。 
        夜班主编揉揉眼睛后看表,他不满意记者丈夫打断了他的好梦。 
        “有重要新闻。”记者丈夫对夜班主编说。 
        “抢劫?凶杀?(被禁止)?”夜班主编问。这些内容是支撑这张报纸的基础。 
        “比这些都刺激。”记者丈夫边说边从照像机里取出胶卷。 
        “快说,哪方面的?’夜班主编的兴趣被调起来了。 
        “您还记得皮皮鲁吗?”记者丈夫问。 
        “就是那个上法庭的物理学家吧?”夜班主编说。 
        “对,就是他。”记者丈夫故意慢慢说,他喜欢看上司眼中那种迫不及待的神色,“他现在在医院的急诊室里。” 
        “他病了?”夜班主编想小出急诊室能有什么爆炸新闻。 
        “他没病。他带别人去看病。”记者丈夫还是舍不得一下说出来。 
        “女的?”夜班主编的想像力纵横驰骋。 
        “一只老鼠。”记者丈夫终于抖开了包袱。 
        “老鼠?!你是说,皮皮鲁带一只老鼠去医院看病?”夜班主编精神头来了。 
        记者丈夫把目睹的场景绘声绘色地描述了一遍。 
        “你不是编小说吧?”夜班主编知道记者丈夫还是一位七流业余小说作者。 
        “百分之百的报告文学。”记者丈夫高举起手中的胶卷。 
        夜班主编打电话叫来值班的摄影部副主任。 
        “立即冲洗这卷照片,越快越好。”主编将胶卷递给摄影部副主任。 
        “你马上写文字稿,我去印刷车间给你留出版面。”夜班主编说。 
        记者丈夫坐在主编的大写字台上挥笔疾书,他把在大学中文系时往肚子里灌的那点儿墨水一古脑倒了出来,怎么危言耸听他就怎么写,怎么哗众取宠他就怎么编。 
        二十分钟后,文章写完了,标题是《夜半奇闻:物理学家给老鼠看病》。 
        夜班主编当即审稿。记者丈夫垂手侍立一旁,像被告等待法官宣判。 
        “嗯。”夜班主编一边看一边情不自禁地点头。 
        记者丈夫嘴角挂上了一丝笑意。他盘算着什么时候提加薪的要求火候最佳。 
        为了显示自己技高一筹,夜班主编用红笔在稿子上做了两个根本不需要的修改。 
        “画龙点睛。”记者丈夫捧功特火。 
        摄影部副主任送来冲洗出的照片。 
        夜班主编挑选见报的照片。 
        “就用这两张。”夜班主编从十几张照片中选出两张。 
        一张是皮皮鲁泪汪汪地看着手中的老鼠。另一张是皮皮鲁捧着老鼠和医生交谈。 
        “马上送照排车间。”夜班主编亲自往车间送稿。 
        记者丈大想起了他的那位在医院继续监视皮皮鲁的hushi配偶,他忙返回医院,搜集新的素材。 
        贝塔用电脑查出了医科大学解剖实验室主任家的地址,他告诉了皮皮鲁。 
        皮皮鲁拦了一辆出租车。 
        “去经纬路4号。”皮皮鲁告诉出租车司机。 
        经纬路不近。出租车足足跑了半小时才到。 
        这是一座二层的小楼。动物解剖实验室主任住在一层。 
        皮皮鲁看看表,现在是凌晨四点。 
        皮皮鲁按主任家的门铃。 
        10分钟后,门厅的灯才亮。 
        “您找谁?”门开了一道缝儿,露出一张不年轻的男人脸。 
        “我找主任。”皮皮鲁说。 
        “你认识他?”不年轻的脸从头至脚看了皮皮鲁一遍,问。 
        “不……不认识。可我有急事找他。”皮皮鲁说。 
        “什么事?”那人问。 
        “我见了他才能说。”皮皮鲁知道如果现在说了这辈子也甭想见那位主任了。 
        “我就是主任,你说吧。”那人亮出了身份。 
        “我可以进去吗?”皮皮鲁觉得站着说话的效果不如坐着说好。 
        “清您告诉我,您深更半夜来找我有什么事?”主任坚持不开门,只露一道缝儿。 
        皮皮鲁运了运气,他知道舒克的生命能否继续就看他的话能不能说服主任了。 
        “您是谁?”主任借着路灯的微弱光线,忽然感到皮皮鲁挺面熟。 
        皮皮鲁不知说出自己的身份对舒克是有利还是不利。 
        主任仔细端详了皮皮鲁一会儿,一拍脑袋,想起来了。 
        “你是皮皮鲁!”主任显然对于这位昔日声名显赫的人物理学家半夜来访感到惊讶。 
        “我……我想求您一件事。”皮皮鲁一边说一边观察主任的反应。 
        “什么事?”主任的口气缓和了一些,门缝儿的宽度也增加了。 
        “我想请您给我的一位朋友做手术,救他一命。”皮皮鲁终于说话了。 
        “做手术?我不会给人类做手术,我是搞动物解剖实验的。”主任愈发感到皮皮鲁是个怪人。 
        “我的这位朋友不是人类。”皮皮鲁说。 
        “您的朋友不是人类?”主任诧异。 
        “他是一只老鼠,得了急性阑尾炎,需要立即手术,否则性命难保。”皮皮鲁说。   第130集 
        茶几当手术台; 
        主任太太从梦中惊醒; 
        意想不到的事件   
        动物解剖实验室主任听皮皮鲁说他给一只老鼠动手术,他把门全部打开,让屋里的灯光照在皮皮鲁脸上。 
        “您再说一遍。”主任看着皮皮鲁。 
        皮皮鲁从口袋里拿出舒克,用双手捧到主任眼前。 
        “他得了急性阑尾炎,请您救救他。”皮皮鲁恳求道。 
        “它是您的朋友?”主任哭笑不得。 
        “是的,是我最要好的朋友,我们的友谊已经有几十年了。”皮皮鲁看着舒克说。 
        “几十年?!”主任清楚老鼠的寿命。 
        “请您救救他。”皮皮鲁继续恳求。 
        “您怎么知道它得了阑尾炎?”主任问。 
        “这儿有医院的诊断书和化验单。”皮皮鲁将诊断书和化验单递给主任。 
        “居然有给老鼠看病的医院?”主任边看边摇头。 
        “请您立即给它动手术。”皮皮鲁的忍耐到了极限,他的口气开始强硬起来。 
        “很遗憾,我不能给一只老鼠动手术。”主任将诊断书和化验单还给皮皮鲁。 
        “为什么?”皮皮鲁火了。 
        “因为它是老鼠。”主任瞪了皮皮鲁手中的舒克一眼。 
        “你是一个混蛋!”皮皮鲁再也控制不住自己了,“你这一辈子用多少老鼠做过实验?恐怕已经成千上万了吧?如果没有老鼠,多少医学成果不能用于临床?人类发明出一种新药,每次都是先让老鼠吃。发明出一种新的疫苗,也是先用老鼠做实验。有多少老鼠为了人类的寿命而失去了他们的性命?老鼠给人类带来的利益远远大于他们给人类造成的祸害,你算过这笔账吗?!你有这技术,你为什么不能救一只老鼠?你杀了他那么多同胞,你不怕死后他们的冤魂找你算账吗?你这不可理喻的笨蛋!!” 
        主任被皮皮鲁机关枪般的奚落骂傻了,他不得不承认皮皮鲁的话有道理。的确,经他的手,保守估计也夺去了五六万只老鼠的性命,正是这些老鼠用它们的生命为人类的医学做出了巨大的贡献,它们确实用自己的命救了很多人的命。人类要是真有良心,应该给这些老鼠立一座碑。可人类没有。忘恩负义与人类同在。 
        “我给它做手术。”主任说。他要使自己的心理获得平衡,尽管他的心理是刚刚被皮皮鲁的一番话弄倾斜的。 
        皮皮鲁的眼泪夺眶而出。 
        “快进屋,就在我家给它做手术。”主任终于让皮皮鲁进他家了。 
        “有设备吗?”皮皮鲁问。 
        “有手术刀,也有麻药,就在这上面做手术。”主任指指茶几,“我去消毒。” 
        皮皮鲁小心翼翼地将舒克放在茶几上。舒克已经昏迷了。 
        “再坚持一会儿,舒克!,,皮皮鲁攥紧拳头说。 
        主任将消毒过的手术器械放在舒克身边,又在舒克的身下铺了一块消过毒的毛巾。 
        有史以来,人类第一次为拯救老鼠的生命而做的手术开始了。 
        舒克的腹部被手术刀切开了。主任点点头,对皮皮鲁说:“是阑尾炎,马上就要穿孔了。” 
        皮皮鲁松了一口气。 
        手术进行到一半的时候,主任太太出现了,她显然是刚刚被吵醒。 
        “你在干什么?”主任太太看到丈夫深更半夜和一个陌生人在客厅里低头忙着什么,感到奇怪。 
        主任抬头看了太太一眼,没说话。 
        太太走到丈夫身边,她吓了一跳。 
        “你在茶几上解剖老鼠!”太太怒不可遏。 
        “不是解剖,是手术。”皮皮鲁更正。 
        “手术?什么手术?”太太不明白。 
        “这只老鼠得了急性阑尾炎,您先生正在给他做手术。”皮皮鲁解释道。 
        “给老鼠治病?你疯了?”主任太太冲丈夫大吼,“而且还在我的茶几上!” 
        太太冲上去夺丈夫手中的手术刀。 
        皮皮鲁慌了,他知道,中止手术会要舒克的命。他拦住主任太太。 
        “您不能这样,太太。”皮皮鲁不让主任太太接近主任。 
        “你是谁?半夜三更来我家于什么?”太太上下打量皮皮鲁。 
        “我是这只老鼠的朋友。”皮皮鲁站在太太和主任之间。 
        “……”太太歪着头看皮皮鲁。 
        “对不起,影响您睡觉了,我也是万不得已。舒克得了急性阑尾炎,如果不立即手术,就会……”皮皮鲁解释。 
        “舒克?”太太没想到老鼠还有名字。 
        “舒克就是这只小老鼠的名字。”皮皮鲁说。 
        “您有病吧?”太太觉得对于深更半夜闯到人家里要求给老鼠治病的人用不着讲礼貌。 
        “你说话注意点儿!”正在做手术的主任抬头看了太太一眼。 
        “我不许你给老鼠治病!老鼠是人类的敌人!”太太朝丈夫冲过来。 
        皮皮鲁忙拦住她。 
        “你干什么?这是我的家!你给我出去!”主任太太拼命想越过皮皮鲁冲到丈夫身边。 
        皮皮鲁回头看看,主任开始给舒克缝针了。 
        “好,你们等着!”太太没皮皮鲁的劲大,无法使用武力制止丈夫给老鼠做手术,她心生一计,转身走了。 
        皮皮鲁松了一口气。 
        手术已近尾声。皮皮鲁看得出,舒克已经脱离危险。 
        主任给舒克缝完了最后一针,他像是告诉皮皮鲁又像是自言自语:“好了,它得救了。” 
        “谢谢您!谢谢您!”皮皮鲁无法表达自己的感激之情。 
        “我也得谢谢您。”主任一边摘胶皮手套一边说。 
        “谢我?”皮皮鲁纳闷。 
        “人类是应该感谢老鼠,咱们拿它们做丁那么多医学试验,如果连个谢字都不说,也太……”主任若有所思地望着窗外。 
        皮皮鲁握住了主任的手。 
        一阵凉风掠过茶几。皮皮鲁一惊,他低头一看,一只大猫跃上茶几叼住了舒克。 
        主任太太养的猫。它是在太太的授意下出击的。   第131集 
        五角飞碟首次出击执行任务; 
        皮皮鲁的名片; 
        录音机和墨水瓶采访皮皮鲁   
        皮皮鲁见舒克遭到了大猫的袭击,忙扑过去抓猫。 
        “你放开他!”皮皮鲁大喊。 
        “(被禁止)!”主任大喊猫的名字。 
        大猫显然只听命于太太,它叼着舒克从大门下边它的专用出口钻到屋外去了。 
        皮皮鲁懵了,一时手足无措。 
        “这太不像话了,你给我出来!”主任叫太太。 
        在忙乱之中,皮皮鲁的手无意触到了衣兜里的微型通讯器。 
        皮皮鲁只有动用五角飞碟救舒克了。 
        皮皮鲁掏出通讯器,呼叫:“贝塔注意!贝塔注意!” 
        “我是贝塔。我是贝塔。请讲!” 
        “一只大猫叼走了舒克,请立即援救,要快!”皮皮鲁知道,一般的猫抓住老鼠都不马上吃,但也不会拖得太久。 
        “明白!”贝塔说。 
        主任惊讶地看着这一切。 
        “你同谁联系?”主任想像不出这位昔日的大物理学家手中掌握着什么高科技武器。 
        “这……”皮皮鲁不能说。 
        “请不要伤害(被禁止),它是我太太的命根子,就像那只老鼠是你的命根子一样。”主任请求。 
        一句话提醒了皮皮鲁,就是,(被禁止)和舒克一样,都不应该死,它们都是人类的朋友。 
        皮皮鲁对着通讯器说:  “贝塔注意,贝塔注意,不要伤害那只猫的性命,救出舒克就行了。” 
        “明白!”贝塔回答。 
        皮皮鲁看表,清晨五点三分。 
        贝塔驾驶五角飞碟升到空中后,打开电脑遥感。 
        只用了0.03秒,贝塔就找到了那只猫,它正在屋顶上捉弄舒克。 
        贝塔怒火心中烧,他操纵五角飞碟闪电般飞到猫的头顶上。 
        如果不是皮皮鲁提醒,贝塔准要了那猫的命。现在,贝塔只得使用超声波击昏了它。 
        五角飞碟落在屋顶上,贝塔将舒克背进飞碟。 
        五角飞碟腾空而起。 
        “皮皮鲁注意。皮皮鲁注意。我已经救出舒克,现正返航。”皮皮鲁身上的通讯器传出贝塔的声音。 
        从皮皮鲁给贝塔下命令到贝塔救出舒克,前后总共用了1.7秒钟。 
        主任呆坐在沙发上,两眼出神地看着皮皮鲁。 
        “谢谢您了,以后有什么事需要我帮忙,请找我。”皮皮鲁递给主任一张名片。 
        主任恭敬地用双手接过名片,然后小心翼翼地放进一个带锁的抽屉里。 
        “再见。”皮皮鲁向主任告辞。 
        主任送皮皮鲁出门,他想像不出那只刚做完手术的老鼠是被什么东西救走的。 
        大街上已经出现了曙光,皮皮鲁乘头班公共汽车回家。 
        五角飞碟停在写字台上。舒克躺在飞碟旁的一块毛巾上,贝塔正在擦舒克身上的血迹——大猫抓的。 
        “舒克怎么样?”皮皮鲁一边观察一边问贝塔。 
        “还没醒。”贝塔说,“要不是你阻止我,我非杀了那猫不可。” 
        “它的主人救了舒克,咱们不能恩将仇报。”皮皮鲁打了个哈欠。 
        “舒克醒了!”贝塔的声音里全是兴奋。 
        舒克的眼睛睁开了,他活了。 
        “别动,你刚动完手术。”皮皮鲁对舒克说。 
        “手术?”舒克抬头看自己身上。 
        “你的肚子被打开过了,你的肠子还被人切下去一截。”贝塔用手当刀在自己的肚子上做了一个切腹的动作。 
        “我得了什么病?”舒克问。 
        “剖腹产。”贝塔逗他。 
        “去——”舒克一笑,疼得他直咧嘴。 
        “开刀后最怕笑,别把线撑开了。”皮皮鲁说。 
        “是阑尾炎。你可把皮皮鲁折腾苦了,一晚上没合眼。在地球上能找到一个给老鼠动手术的医生可真不容易。”贝塔感慨万千。 
        舒克看着皮皮鲁。他没说感激的话。他觉得朋友之间不需要客套。他的目光就代表了一切。 
        “手术后,你遇到了一只大猫的袭击,是贝塔驾驶五角飞碟救了你。”皮皮鲁告诉舒克。 
        “真应该让猫吃了你。”贝塔冲舒克一笑。 
        舒克看了一眼墙上的石英钟,说:“你们快休息吧。” 
        皮皮鲁伸了个懒腰,对贝塔说:  “我睡会儿,你先照看一下舒克。” 
        贝塔说:“我当hushi。” 
        皮皮鲁的头还没挨着枕头,他就睡着了。 
        贝塔没有睡意,他把脑子里库存的最幽默的笑话调出来讲给舒克听。 
        舒克想笑又不敢笑,发誓等贝塔开刀时他一定把相声演员请来折腾贝塔。 
        皮皮鲁醒来时,已经是上午11点半了。 
        他洗漱后,到楼下拿报纸。 
        贝塔发现皮皮鲁进屋时表情不对,他边看报边皱眉头。 
        “怎么了?”贝塔问皮皮鲁。 
        “无聊!”皮皮鲁的表情像吃了苍蝇。 
        “报上又攻击你了?”贝塔问。 
        皮皮鲁把报纸放在桌子上。 
        贝塔看见报上刊登的皮皮鲁和舒克在医院的大幅照片。 
        敲门声。 
        皮皮鲁将桌子上的五角飞碟和舒克藏进卧室里,贝塔也躲了起来。 
        门外站着两个陌生的年轻人。 
        “请问您是皮皮鲁吧?”其中一个脑袋长得像录音机的问。 
        “是的,你们是?”皮皮鲁问。 
        “我们是广播电台的记者。您看今天的报纸了吗?我们想证实下那家报纸上的有关您的那条新闻的真实性。”另一个脸长得像墨水瓶的小伙子说。 
        皮皮鲁让他们进屋坐下。 
        “这是我个人的隐私,无可奉告。”皮皮鲁说。 
        录音机和墨水瓶对视了一下,皮皮鲁的回答等于是默认。 
        “那只老鼠得救了吗?”录音机突然问。 
        皮皮鲁看着他,不回答。 
        “谁给它做的手术?”墨水瓶问。 
        皮皮鲁不再沉默了,他盯着录音机和墨水瓶一字一句地说:  “当然是第一流的专家给他做的手术。请你们回去转告新闻界,别和一只老鼠过不去,有本事去找总统的碴儿。好了,你们可以走了。” 
        皮皮鲁站起来送客。 
        看得出录音机和墨水瓶挺满意这次采访,他俩连蹦带跳地下楼。 
        20分钟后,收音机里就播送了加评论的访问皮皮鲁的录音专题。该电台在转述皮皮鲁对新闻界的“指示”时,大肆添油加醋。这回皮皮鲁算是得罪了新闻界。 
        所有新闻媒介都把焦点集中在皮皮鲁身上,报纸,期刊,电台和电视台都向皮皮鲁宣战,说皮皮鲁是精神病,是变态狂。 
        贝塔实在无法忍受了,他从皮皮鲁口中了解到是那位记者丈夫最先发难的之后,决定使用五角飞碟捣毁那家报社。 
        贝塔知道皮皮鲁不会同意他这么干。他要单独行动,连舒克也不告诉。   第132集 
        贝塔的注意力集中在报社的一对狗男女身上; 
        记者丈夫被提升为编辑部主任; 
        假女士报警   
        贝塔将行动的时间定在午夜1点。 
        皮皮鲁和舒克都睡熟了。贝塔蹑手蹑脚地推开窗子。 
        贝塔钻进五角飞碟,他坐在驾驶台前打开总开关。五角飞碟内部立即灯火通明。 
        贝塔的手指放在起飞按钮上,他显然在犹豫。贝塔清楚,私自出动五角飞碟如果让皮皮鲁知道了,他肯定会生气。 
        “算了吧,这几天够皮皮鲁受的了。”贝塔的手缩回来了。他害怕再给皮皮鲁添新的麻烦。 
        贝塔解开安全带,从座椅上站起来。走到舱门口时,他又停住了。 
        贝塔想起一家电视台专门为讽刺皮皮鲁而安排的专题节目,那装腔作势的女播音员说皮皮鲁心理变态,还说皮皮鲁养老鼠是仇恨人类的表现…… 
        这全是那位记者丈夫引起的。贝塔一跺脚,又回到驾驶台前坐下,他实在咽不下这口气。如果不摧毁了那家报社,贝塔的心一辈子也甭想安宁。 
        五角飞碟起飞了。 
        贝塔通过电脑很快就找到了那家报社,他操纵五角飞碟悬停在报社上空。 
        贝塔打开遥感器,他通过荧光屏观察报社内部的情况。 
        报社里,各部门正忙着出当天的报纸。有看校样的有改稿的有排版的有划版的,说白了就是一句话:把用过几亿亿次的字再进行一次新的排列组合,然后把读者腰包里的钱变为报社的进账。 
        贝塔的遥感器停在一间挂着“编辑部主任办公室”牌子的房间里。屋子里坐着一男一女,两人的办公桌面对面。 
        男的长着一张俗不可耐的略显浮肿的脸,头发的大趋势显然是秃顶,年龄在50岁左右。坐在他对面的女子不到四十岁,贝塔发现她脸上的所有器官都是人工合成,双眼皮是刀割的,鼻子是经过不锈钢支架硬撑起来的,嘴唇里八成注入了化学充填剂,就连耳朵也经过一番整形,反正她的每一个表情每一个动作都透着假透着做作。 
        贝塔想吐,他刚要移动遥感器,那位全方位假女士的话引起了他的注意。 
        “搞,你当了主任有什么感受?”假女士用酸得不能再酸的腔调问。 
        “你问我的感受?”那位被称为搞的男士用猥亵的眼神看着假女士,“我说实话还是说官话?” 
        “瞧你!跟我当然是说心里话啦!”假女士撒娇时的表情能将一个集团军的官兵活活恶心死。 
        “假,我当主任后的第一感觉就是我成了这个编辑部的所有人的爸爸。要不,我怎么刚当上主任没5天就敢把你从外边调来还让你坐在我对面!我是他们的爸爸,你就是他们的妈妈。哈哈……”被称为搞的主任放肆地笑着。 
        “去你的!谁当他们的妈妈,折寿!我倒应该感谢皮皮鲁和那只老鼠。还不是因为你发现了这个新闻,才被提拔为主任的。”假女士说。 
        贝塔弄清楚了,这位搞主任就是那天深夜追踪皮皮鲁和舒克的记者丈夫。他由于报道了皮皮鲁给老鼠看病的新闻而被提升为编辑部主任,那位假女士是他上任后从外边调来给他当公关小姐的昔日的相好。 
        贝塔想起了刚才在荧光屏上依次出现的报社的那些工作人员,贝塔可怜他们,在这样的上司手下任职,真是三生不幸。人家在办公桌前谈情说爱,你们却在深夜埋头苦干。别的部门的职员只自一个顶头上司,你们却两个——一个爸爸加一个妈妈。 
        贝塔身上产生了一股使命感,他现在捣毁这家报社不光是为了给皮皮鲁出气,还为了拯救这家报社的职员们,把他们从那位无才无德的搞主任的淫威下拯救出来。 
        贝塔接通了五角飞碟武器系统的开关,他将自己设定的指令输入武器系统的电脑:1.该报社的建筑和办公用品在0.01秒内消失;2.对工作人员秋毫无犯;3.将搞主任身上那种对于男士来说除了心跳以外最重要的功能消除。 
        指令输人完毕。贝塔又复查了一遍,这可是人命关天的大事。当贝塔确信电脑准确无误地理解了他的指令后,他按下了射击按钮。 
        报社的建筑以及报社的一切办公用品诸如印刷机写字台电脑电传椅子纸篓……在0.01秒钟内荡然无存。 
        深夜,一群编辑记者印刷工人站在一块空地上茫然不知所措。他们不明白报社为什么突然间不翼而飞。搞主任还一时无法适应从温馨的氛围中一下子来到寒冷的夜空下的突变,他的惊慌的目光四处乱射。如果他知道自己今后只有用眼睛和假女士相处,他决不会像现在这样浪费目光。 
        “快!快去报案!”搞主任冲下属喊。 
        没人行动。谁也不认为这是罪犯所为。 
        “你去警察局报案!”搞主任对身边的假女士说。 
        假女士点点头,她一路跑着来到警察局。 
        “报案!”假女士冲值班的警察劈头就喊。 
        “有坏人非礼你?”警察问。大凡深更半夜闯进警察局的女士都是这类遭遇。 
        “有坏人把我们报社偷了!”假女士说。 
        “丢了什么?”警察打开记录本。 
        “所有的东西都丢了。”假女士盯着警察说。 
        警察抬头看假女士,说:“所有东西?连房子也丢了?” 
        “对,没错,连房子也丢了,一块砖都没留下。”假女士证实。 
        “房子也被偷走了?怎么偷的?”警察合上记录本,他断定面前这个女人是梦游症患者。 
        “你们去看看就知道了,这是真的。不信你看我的记者证。”假女士掏出记者证递给警察。这记者证是搞主任悄悄给假女士办的,假女士靠它四处招摇撞骗。 
       警察看完记者证,拿起电话听筒。 
        5分钟后,一辆警车拉着假女士驶往现场。 
        报社的工作人员仍然站在空地上。 
        “这儿的房子呢?”警察局副局长一下车就问。 
        “我不是说被人偷走了吗?”假女士的话音里有几分得意。 
        “这怎么可能?!”副局长在空地上走了一圈,地上没有任何这里曾经有过房子的痕迹。 
        报社的工作人员争先恐后地向警察述说报社被盗的经过。 
        不一会儿,几辆警车开来了。从车上跳下十几名警察,有的拍照,有的摄像,还有一名警察牵着一条大狗在地上嗅来嗅去。 
        贝塔坐在五角飞碟里笑得前仰后合。他看看表,吃了一惊,都快四点了。 
        ‘返航。”贝塔对自己说,“拜拜了,搞主任,还有假女士。” 
        五角飞碟神不知鬼不觉地回到皮皮鲁家。皮皮鲁和舒克睡得正香,贝塔轻轻地睡下。   第133集 
        牙刷在皮皮鲁嘴里落户; 
        业余天文爱好者的摄影作品; 
        假女士和搞主任成婚后大呼上当   
        清晨,皮皮鲁醒了。他刚坐起来,舒克也醒了。老鼠的听觉很敏锐。 
        贝塔没像往常那样和舒克一起醒来,仍在呼呼大睡。 
        “这家伙今天怎么了?”舒克看了看贝塔,对皮皮鲁说。 
        皮皮鲁耸耸肩,走进客厅打开电视机。皮皮鲁有收看电视早间新闻的习惯。 
        电视打开后,皮皮鲁一边听一边到卫生间洗漱。 
        “今天凌晨l点多钟时,本市发生了一件最奇怪的盗窃案。《××报》社的建筑以及一切办公设备在1秒钟之内突然不翼而飞……” 
        皮皮鲁嘴里叼着牙刷从卫生问跑出来看电视,舒克也从卧室里跑出来。他俩都清楚地记得这家最先报道皮皮鲁给老鼠看病的报社的名字。 
        电视上出现了该报社的建筑被“盗”走后的场面。 
        “下边请看本台记者现场对该报社编辑部主任搞先生的采访。”女播音员说。 
        皮皮鲁认出了荧光屏上的“搞先生”就是那天深夜在医院里给他和舒克拍照的记者丈夫。 
        “请您介绍一下事件的经过。”电视台记者将烧火棍般的麦克风捅到搞主任嘴边。 
        搞主任显然是第一次面对摄像机,他的表情与其说是沮丧不如说是兴奋。很明显,他到目前为止还不知道身上少了什么功能。 
        “是这样的!”搞主任顿顿嗓子,差点儿把烧火棍含在嘴里,“昨天晚上,噢,不对,是今天凌晨,我正在和同事上夜班,编今天的报纸。突然间,我们四周的一切,包括房子和所有办公设备,在几乎不到1秒钟内全都不翼而飞。” 
        “这太不可思议了。”电视台记者插话说,“有人员伤亡吗?” 
        “没有。”搞主任摇摇头。 
        “你们报警了?”记者问。 
        “报了。警方未查出线索。” 
        “您认为这是盗窃案?” 
        “……不像,可房子和设备确实没了。” 
        “会是外星人干的吗?”电视台记者突出奇语。 
        搞主任一愣,他没朝这方面想。他的脑子比较迟钝,远远跟不上他的嘴巴。他属子那种嘴巴档次最高,大脑档次最低的人。 
        皮皮鲁从嘴里拔出牙刷,他和舒克不约面同地扭头看正在熟睡中的贝塔,他们异口同声地喊: 
        “贝——塔——” 
        贝塔揉揉眼睛,坐起来。 
        “你昨天晚上出去了?”皮皮鲁问贝塔。 
        “出去?去哪儿?”贝装傻充愣。 
        “去报社呀!”舒克敲锣边。 
        “去报社?我深更半夜去报社干什么?”贝塔继续抵赖。 
        “那家报社不是你捣毁的?”舒克指指电视。 
        贝塔看了一眼电视,他明白了。 
        “没有,不是我干的,我发誓。”贝塔用比较低沉的不那么理直气壮的语调声明。 
        “这好办,五角飞碟有电脑记忆装置。”皮皮鲁冲贝塔一笑。 
        贝塔脸上全是尴尬。 
        “舒克,去五角飞碟里把记忆资料给我调出来。”皮皮鲁说。 
        “我招。”贝塔挡住舒克,  “全招”。 
        “说吧。”皮皮鲁拿毛巾擦嘴上的牙膏沫儿。 
        “我实在咽不下这口气,那小子太坏了,给你制造了这么大的麻烦,可他倒好,靠这个当上了编辑部主任,还调来一个假女人坐在他对面看着过瘾……” 
        “假女人?”皮皮鲁不明白。 
        “噢,对不起,我没说清楚。那女人浑身上下透着假。你们别忘了咱们五角飞碟上摄像装置有透视功能,我看到她为了显示自己的臀围,穿了一条棉裤叉。”贝塔冲舒克挤眼睛。 
        “别说这些。接着招。”皮皮鲁觉得在自己家里提这种女人晦气。 
        “我觉得我有责任利用五角飞碟为民除害,这家报社专靠揭人隐私发财,理应根除。我就把它连窝端了。”贝塔一边说一边观察皮皮鲁的表情。 
        “你忘了你发过誓?”皮皮鲁很严肃。 
        “没忘。我发誓不利用五角飞碟干坏事。我觉得我干的这件事是好事。对了,那个什么主任还对他的那位假女人说,他实际上是他的下属的爸爸,我把这些编辑从那混蛋主任的淫威下解救出来,当然是干了一件好事。”贝塔越说越气壮山河。 
        “舒克,你去五角飞碟里把记忆磁盘拿出来。我看贝塔不会只干了这些。”皮皮鲁说。 
        舒克从五角飞碟里取出记忆磁盘。 
        皮皮鲁将记忆磁盘(禁止)他的电脑。 
        “好啊,你还把搞主任的重要功能解除了,你可真够损的。”舒克边看电脑显示边对贝塔说。 
        “本来嘛,他这种人就不该有这种功能,我这是替天行道,净化人类遗传基因,使人类优生优育。”贝塔小声嘟囔。 
        皮皮鲁瞪了贝塔一眼。 
        电视台女播音员的声音再次引起皮皮鲁和舒克、贝塔的注意。 
        “刚才本台接到一位业余天文爱好者的电话。这位业余天文爱好者声称他在今天凌晨用自制的天文望远镜观测到一个UFO曾经光临过本市,这个飞碟光临本市的时间与那家报社建筑失踪的时问完全吻合。这位业余天文爱好者还给飞碟拍了照片,现在,本台正派专车去他家取照片,请各位观众稍等片刻。”女播音员滔滔不绝地说。 
        皮皮鲁和舒克从不同的方向瞪贝塔。 
        “真没想到,人类还有这种精神病,半夜不睡觉,抽疯看什么星星。”贝塔不敢直视皮皮鲁和舒克,只得自言自语。 
        “这回你可闯祸了,等着看你的照片上电视吧!”舒克揶揄贝塔。 
        电视台的女播音员情绪激动地对观众说:“照片到了。现在请各位观众欣赏外星飞碟光临本市的照片。” 
        荧光屏上出现丁那位业余天文爱好者的摄影杰作。五角飞碟清晰地悬停在空中。 
        全市肃静。静得恐怖。 
        “再向各位观众报告一个消息,刚才天文台证实了这位业余天文爱好者的发现,他们也观测到了这个飞碟,他们还为飞碟摄了像,本台在10分钟后将播放天文台摄制的录像。” 
        10分钟的广告像10年。 
        天文台摄制的录像片把五角飞碟栩栩如生地暴露在全市市民面前。 
        现在,人们明白了,是外星人飞碟掠走了那家报社的动产和不动产。它既然能把房子都搬走,还有什么搬不走的呢? 
        录像片还没放完,全市所有银行的门口都排起了取款的长龙。另一个排长龙的地方是结婚登记处和离婚登记处。谁都想在被外星人抢走之前了却自己的宿愿。 
        搞主任更是风驰电掣般地同hushi发妻离了婚。hushi发妻这才明白,想当成功男人背后的那个女人并不是好事。男人成功之日,就是身后的那个女人以泪洗面之时。搞主任庆幸自己没被外星人抓走,想想也真后怕。他又火速同假女士办了结婚登记。假女士同搞主任结婚后大呼上当受骗,她说还不如让外星人抢走痛快。 
        城市笼罩在片惊恐之中,人人惶惶不可终日。   第134集 
        五角飞碟光临电视台; 
        两名导演跳楼; 
        替补播音员临危不惧; 
        贝塔提醒舒克不能说再见只能说永别   
        城市继续笼罩在飞碟侵扰的恐怖气氛中。市民们人心慌慌,他们花钱时的表情就像花jiachao那样。人们无心上班无心上学无心干一切应该用心干的事。所有的人走路时平均每10秒钟抬头观察一次天空。 
        城市瘫痪了。垃圾成堆治安混乱通货膨胀商品匮乏停水停电停气。 
        贝塔自知理亏,这几天一声不吭。 
        “人类也太脆弱了,一个飞碟就吓成这样。”舒克站在窗台上望着窗外说。 
        皮皮鲁从沙发上站起来走到窗前。 
        “这样下去,用不了一个星期,这座城市就垮了。”皮皮鲁摇摇头,“人类也真是,航天飞机都上了天。别看人类平时趾高气扬,其实内心世界虚弱得很。” 
        “我小时候我妈妈对我说过一句话,我一直记着。”舒克若有所思。 
        “她怎么说?”贝塔小心翼翼地在非关键问题上插嘴。 
        “没事时胆子别太大,有事时胆子别太小。”舒克说。 
        “精彩。”皮皮鲁点点头。 
        “我看人类就是没事时胆子特别大,遇到事时胆子特别小。”贝塔趁机狠狠贬低一下人类。 
        “这是人类的弱点。”皮皮鲁眉头皱起来.他看到楼下的公路上有7辆汽车首尾相撞。这几天交通事故特多。 
        “我看谁也毁不了人类,能毁人类的就是人类自己。”贝塔说,“上次那么大的地震,没几天人们就又把房子盖起来了。” 
        “咱们得想个办法抢救这座城市。”皮皮鲁开始在房问里来回踱步。 
        “解铃还需系铃人,我看只有贝塔能完成这个使命。”舒克冲贝塔挤眼。 
        皮皮鲁眼睛一亮。 
        “你们俩驾驶五角飞碟出去飞一圈,然后悬停在电视台上空我给电视台打个电话……”皮皮鲁把行动计划告诉舒克和贝塔。 
        1O分钟后,五角飞碟起飞了。 
        皮皮鲁家与电视台之间的距离是5公里。 
        行人很快就发现了头顶上的飞碟。飞碟经过的地方的人们都双手抱头趴在地上一动不动。 
        “皮皮鲁,我们已经到达电视台上空。”舒克同皮皮鲁联系。 
        “我现在给电视台打电话。”皮皮鲁拨电视台的电话号码。 
        电话通了。 
        “是电视台吗?”皮皮鲁问。 
        “是。” 
        “我是一个观众,我发现外星飞碟现在就在你们电视台的上空。” 
        对方没有回答,过了几秒钟,皮皮鲁听见对方的话筒掉在了地上,那屋里一片混乱。 
        皮皮鲁挂上电话,打开电视机。 
        电视台大厦的所有窗户上都挤满了人头。当大家证实了那飞碟确实在电视台上空而不是哪位同事的恶作剧后,有两位导演一位制片人三位播音跳了楼——其中一位导演是从7层楼坠地的。他们认定那飞碟要像搬走那家报社一样搬走电视台大厦。他们目前还不想去外星球当俘虏,因为他们在地球上的事业挺发达。 
        一位有冒险精神的摄像师跑到楼顶的平台上扛着摄像机瞄准了五角飞碟。 
        另外几位勇敢的同事打开了现场直播发射机。 
        一位干了6年替补播音员一次也没上过荧屏的小姐极其英明地抓住了这个机遇,她主持了这次现场直播。 
        “各位观众,你们现在看到的画面是现场直播。那个抢走了报社建筑的飞碟现在就悬停在本电视台上空,它的意图我们还不知道。不过有一点儿可以告诉各位,它随时有可能抢走电视台大厦。我是第一次也可能是最后一次播音,如果我随同这座大厦一起被抢到外星球去,我不后悔,因为我在地球的最后一刻是同我的观众在一起度过的。” 
        替补播音员绝对有才气。 
        皮皮鲁知道五角飞碟成全了这位在事业上已经山穷水尽的播音员。他不得不佩服这位小姐的智商,她属于那种能抓住机会的智慧型人才。 
        “舒克,马上同那位播音员通话。”皮皮鲁向五角飞碟下达指令。 
        “现在,在这座大楼里只有四个人:我、楼顶上的摄像师、播音室摄像师和机房工程师,电视台的其他工作人员都已撤离大厦。我现在要向各位观众介绍两位摄像师和机房工程师。”替补播音员在没有脚本和导演的情况下显示出她非凡的主持人才干。 
        “你的电话,是外星人打来的,点名找你。”摄像师冲替补播音员喊。 
        “各位观众,飞碟上的外星人给我打来电话。现在,让咱们听听他们说些什么。”替补播音员拿起话筒。 
        整座城市屏住呼吸。 
        “我们是外星人。我们无意伤害地球人。我们搬走那家报社是因为它太低级,我们还要提醒报社的那位搞主任,请你好自为之,不要再干坏事。我们祝地球人幸福。现在,我们要走了,再……”舒克捏着鼻子装外星人。 
        “不能说再见,得说永别。你要说再见,他们晚上还睡不着觉,还得取钱花。”贝塔打断舒克。 
        “永别了,地球人.我们一去不复返了,再也不会光临地球了!”舒克粗声说。 
        “离开电视台,别太快,让摄像机拍你们缓缓离开的镜头。”皮皮鲁指挥五角飞碟。 
        电视屏幕上的五角飞碟优美地向高空飞去,越来越小,越来越小。 
        人们松了口气。 
        替补播音员说:“永别了,外星人!祝你们一路平安!” 
        鞭炮声四起。 
        去银行争先恐后存钱的人动手打了起来。 
        替补播音员名声大振,30分钟内她成为家喻户晓的明星播音员,许多观众打电话威胁电视台说,如果有一天该替补播音员不在屏幕上露面,他们就永远不看电视了。 
        城市恢复了正常。 
        没事了,人们的胆子就又大了。 
        皮皮鲁和舒克、贝塔正在家为自己的杰作得意时,有人敲门。   第135集 
        出版社的总编辑向解剖主任发坏; 
        黑心的责任编辑雪上加霜; 
        皮皮鲁批准行动计划   
        皮皮鲁透过门镜往外看,是那位给舒克做手术的动物实验室主任。 
        他来干什么? 
        “你们俩先到屋里躲一下,给舒克做手术的那位主任来了。”皮皮鲁对舒克和贝塔说。 
        “你的救命恩人来了,你应该报恩。”贝塔逗舒克。 
        “我能当面谢他吗?”舒克认真地问皮皮鲁。 
        “我先看看他来这儿的意图再决定。如果可以,我会叫你的。不过,五角飞碟可不能泄露。”皮皮鲁说完朝大门走去。 
        舒克和贝塔躲进里屋。 
        皮皮鲁开门。 
        “您好。对不起,我有点儿事想求您。”解剖主任显得拘谨和不知所措,和数日前的他判若二人。他手里拿着皮皮鲁送给他的名片。 
        “请进。”皮皮鲁热情地将舒克的救命恩人让进客厅。 
        解剖主任坐在沙发上环顾皮皮鲁的家。 
        皮皮鲁从冰箱里拿出一筒饮料递给客人。 
        “我……有点儿……事……”解剖主任吞吞吐吐。 
        “您先喝口水,只要我能办,一定尽力。”皮皮鲁说。 
        “您能帮我,只有您能帮我。”解剖主任的情绪突然进人激动状态。 
        “您先喝口水。”皮皮鲁说。 
        解剖主任喝了一口饮料,稳定一下情绪.说:“是这么回事,我从事动物解剖研究近40年了。在这加年里,我几乎放弃了一切娱乐,埋头动物解剖研究。近3年来,我将我的研究成果写成了一部50万字的学术著作,书名为《动物解剖学探秘》。” 
        “您真了不起。”皮皮鲁也是搞学术的.他知道其中的艰辛。 
        “您先别夸,书还没出版。” 
        “为什么?” 
        “书稿送给出版社后,出版社认为这是一部有重要学术价值的著作。” 
        “那就快出呀!” 
        “没这么简单。一天,该出版社的总编辑单独约我谈话,他提出将他的名字也印到书上。” 
        “他当这本书的责任编辑?” 
        “不是,他要作为书的作者和我的名字印在一起。” 
        “有这种事?!” 
        “他说他也很喜欢动物解剖学,还说他5岁时解剖过蚂蚁。他还暗不我说,如果不同意,这本书就出不来。还说这种书出一本赔一本,没人愿意出。” 
        “这不是强盗吗?!他图什么?” 
        “他说有了这本书,他就可以在下次评职称时评上编审了。” 
        “流氓。你同意了吗?” 
        “开始我不同意,后来我实在是走投无路,我太想看到自己的学术著作问世了,只能同意。” 
        “你错了。这好比一个强盗闯入你的家,你就把一半财产拱手送给了强盗。” 
        “我是错了。我以为我让了步书就可以顺利出版了。前几天又节外生枝,这本书的责任编辑突然向我提出要和我对半分这本书的稿费。” 
        “无赖!一群文化恶棍!知识痞子!” 
        “我开始不同意。他说他为这本书付出了巨大的劳动。” 
        “这是他的工作,他已经为此拿了工资!” 
        他说他家生活拮据,上有八十高龄的老母,下有嗷嗷待哺的婴儿……” 
        “干脆说自己是非洲灾民得了。” 
        “人心都是肉长的,我禁不住他的哭诉,就又同意了。可现在我越想越不是滋味。钱和名都是小事,重要的是我觉得我的尊严没有了。你想想看,一个没尊严的人即使他得了诺贝尔奖,即使他当了美国总统,如果在全球52亿人中排名次他不也只能排倒数第一名吗?” 
        “你说得好极了。你这段话我特爱听。” 
        “我不想在人类中当倒数第一,可我又无能为力把自己往前排,我想来想去,只有您能帮助我。” 
        “我?”皮皮鲁不解地看着解剖主任,“怎么帮?” 
        “把那总编辑的名字从书上拿下来。把那本书的稿费全部交给我。”解剖主任一字一句地说。 
        “我哪儿有这么大的本事?”皮皮鲁连连摇头,“我只能帮你出主意,去法院告他们。” 
        “那书可就出不来了,我希望书能按期出版。” 
        “这可太难办了。”皮皮鲁无可奈何。 
        “您能帮我!我知道您神通广大!我知道那家报社就是因为惹了您,您才连窝把它端了的!”解剖主任亮出了王牌。 
        皮皮鲁愣住了。 
        原来,给舒克做手术那天晚上,解剖主任就注意到了皮皮鲁使用了一种极其现代化的仪器把舒克从大猫嘴上救出来的。后来,他一直密切注意着皮皮鲁,注意到那家报社是因为死咬住皮皮鲁不放才倒霉的。 
        “我不明白您说的是什么。”皮皮鲁忙拿出挡箭牌抵挡。 
        “请您帮助我。您不会不主持正义的。这是那家出版社的名称和地址,上边还有那位总编辑和责任编辑的名字。那本书明天下午开印。”解剖主任站起来,他将纸条留给皮皮鲁。 
        “我信任您。”解剖主任告辞了。 
        皮皮鲁站着发呆,连送客都忘了。 
        贝塔和舒克从里屋出来。 
        “我看这个忙得帮,那个总编辑也太缺德了!”贝塔说。“咱们正经也是办过《老鼠报》的。舒克当总编辑时,多清白!” 
        “还有那个什么责任编辑,毫无职业道德,居然勒索作者,死后也不怕下地狱。”舒克忿忿然,“这种人怎么当上编辑的!想当初咱们的松果和荷叶,多有职业道德!” 
        “你以为编辑怎么样?你忘了那位搞主任了?我看干这行的档次高的不多。”贝塔给编辑职业下了定义。 
        “他们的问题就在于太把自己当人看可又不干人事。”皮皮鲁说话了。 
        “咱们帮帮解剖主任吧,他救过舒克的命呀!”贝塔特想驾驶五角飞碟行侠。 
        “我也想报答他一次。”舒克加入请求的行列。 
        “这事咱们好像帮不上忙。”皮皮鲁觉得解剖主任的要求难度比较大——又要出书又不让总编辑挂名不让责任编辑雁过拔毛。 
        “我有办法。”贝搭说。 
        “说。”皮皮鲁看看贝搭。 
        “在那本书开印之前,我和舒克去把版上的那位总编辑的名字去掉,等书印出来,就只剩解剖主任的名字了。”贝塔说。 
        “这办法咱们在30年前用过。皮皮鲁,你还记得吗?咱们在一家晚报上登过告诫读者提防老鼠把老鼠药放到人类的食物里的文章。”舒克提醒皮皮鲁。 
        “没错!”皮皮鲁兴奋了,他的表情回到了童年。 
        “批准了?”贝塔急不可待。 
        “批准了。”皮皮鲁同意了贝塔的方案。 
        “怎么治那位责任编辑?”舒克问。 
        贝塔如此如此这般这般。 
        “真够损的。”舒克说。 
        “就这么办吧!”皮皮鲁觉得对付责任编辑这种手黑的人就得用损招儿。 
        把总编辑的名字从版上去掉并不容易,总编辑在校样上签字后这本书才能开机印刷,而该总编辑签字前肯定要把自己的名字翻过来倒过去验明正身数百遍后才会签字同意付印。总编辑签字距离开机印刷还有多长时间现在还是未知数。   第136集 
        五角飞碟隐蔽在树叶中; 
        秃顶总编辑在上班时间看抽屉里边; 
        管道中一声大喝   
        第二天上午,舒克和贝塔走进五角飞碟,准备出发前往那家出版社。 
        “要注意隐蔽,不要被人发现。”皮皮鲁告诫舒克和贝塔。 
        “放心吧。”贝塔摩拳擦掌。他现在最爱干的事就是驾驶五角飞碟出击。 
        “五角飞碟已做好起飞准备。”舒克请示皮皮鲁。 
        “可以起飞。”皮皮鲁推开窗户。 
        五角飞碟离开桌面,在屋里环绕了一圈儿后,用极其潇洒的姿态从窗户飞了出去。 
        皮皮鲁脸上挂着明显自豪的笑容。 
        那家出版社的方位已输入五角飞碟的电脑。由于是白天飞行,为防止被人发现,舒克操纵五角飞碟使用超高速飞行。O.001秒后,五角飞碟已经抵达出版社上空。 
        “那儿有一棵大树,咱们可以藏在树叶里。”贝塔指着出版社旁边的一棵树说。 
        舒克认为贝塔的建议可行,他驾驶五角飞碟躲进那棵大树的茂密的树叶里。 
        “皮皮鲁,我们已到达目标,现在准备开始行动。”贝塔同皮皮鲁通话。 
        “小心谨慎。随时保持联络。祝你们成功。”皮皮鲁说了三句话。 
        “开始干吧,现在是你报答恩人的时候了。”贝塔拍拍舒克的肩膀。 
        舒克打开遥感器,荧屏上显示出整座出版社大楼里的全方位景象。 
        贝塔操纵旋钮一个一个房间观察。 
        荧光屏上轮番出现各种办公镜头,有的编辑在看稿,有的在用火柴棍掏耳朵,有的聊天,有的轮流用除了大拇指以外的所有手指头挖鼻孔,然后再使出浑身解数不达目的誓不罢休地将从鼻孔里挖出来的死也不愿意离开手指的东西甩离手指。 
        “停!”贝塔喊。 
        画面上是一间气派的办公室。大写字台。黑色真皮沙发。两部电话。墙上还有装饰画。书柜里全是书。 
        一个五十多岁的秃顶男人正坐在写字台前往拉开的抽屉里窥视,那表情分明告诉别人抽屉里有24K金子。 
        “看看他在看什么。”贝塔对舒克说。 
        舒克调整遥感器的角度。 
        抽屉里是一张电影女明星的半裸彩照。 
        舒克和贝塔不约而同地耸耸肩。 
        秃顶男人桌上的电话铃响了。他拿起话筒。 
        “我是总编辑,什么?”秃顶男人关上抽屉,换了一副工作表情,“好,现在送来吧。” 
        “就是他!”舒克说。 
        “上班时间看那种东西,这种人怎么当上总编辑的?!”贝塔觉得恶心。 
        他倒不是恶心秃顶总编辑看那照片,他恶心秃顶总编辑变换两种不同的表情时那么应用自如。 
        有人敲秃顶总编的门。 
        “请进。”秃顶总编整整领带。 
        一位风姿绰约的女士煞有介事地拿着一个文件夹迈着规范的步子走到秃顶总编的办公桌旁。 
        秃顶总编伸出手。 
        女士从夹子里抽出一摞纸,递给秃顶总编。 
        “这是您的大作,您再看看,如果没有问题,您签字后就开始印刷了。” 
        女士显然是秘书,她很有秘书的专业风度和气质——端庄里透着贱。 
        “lO分钟后你再来。”秃顶总编辑根本不看女秘书,整个儿一个不食人间烟火的正人君子。 
        可当女秘书转身往门外走时,秃顶总编辑的眼睛好像长在了女秘书的屁股上,直到她的身影消失为止。 
        “这总编辑也真够累的,大概他到现在也不知道那女秘书的正面是什么样。明明是爷爷,非要装孙子。”贝塔想起了搞主任。相比之下,还是搞主任活得“潇洒。” 
        秃顶总编认真地看女秘书送给他的那摞纸。 
        “他怎么只看第1页?”舒克看了看表,5分钟过去了,秃顶总编还在看最上面那一页。 
        “看看这页上边是什么。”贝塔怀疑秃顶总编看的那页上有透明度比较强的女明星的玉照。 
        舒克操纵遥感器观测那张纸。 
        纸上赫然印着《动物解剖学探秘》几个大字。书名下边是两位作者的署名。排在前边的是秃顶总编,排在后边的是解剖主任。 
        秃顶总编死抓住第l页不放原来是在享受成功。他校对这部书稿只是校他的名字。看一万遍还觉得不过瘾。 
        “也不知在别人的作品上署名是什么感觉?”舒克揣摩秃顶总编此时的心理状态。 
        “准和抢银行得了一百万元的感觉一样。”贝塔说。 
        女秘书又来了。 
        秃顶总编照例头也不抬,说:  “我已签字,马上付印。” 
        女秘书将书稿的校样装进公文夹,转身朝门口走去。 
        秃顶总编照例对女秘书的后背行定点注目礼。 
        “跟踪那校样。”贝塔站起来伸个懒腰。 
        舒克控制遥感器观察女秘书。女秘书把校样交给一个小伙子。小伙子坐着轿车将校样送到了两公里外的印刷厂。 
        印刷厂厂长看了秃顶总编的签字后,对生产科长说:“1小时后开机印刷《动物解剖学探秘》。” 
        舒克和贝塔将五角飞碟降落在印刷厂厂房的房顶上。 
        遥感器告诉他们,《动物解剖学探秘》的版现在印刷车间的文件柜里。这部书稿采用胶印,版是胶片,不是铅字。要将秃顶总编的名字从版上抹掉,就得把印有他名字的那一部分从胶片上割下来,给他开“天窗”。 
        这个行动的难度不小。如果使用激光,又怕伤了别的胶片,影响解剖主任的学术成果。因此只有人工将秃顶总编的大名从胶片上清除出去。 
        “我去,你接应我。”贝塔准备好一把小刀。 
        “我去。”舒克要去。 
        “你刚动完手术,行动不灵活,还是我去吧。”贝塔不同意。 
        “我得报答那解剖主任的救命之恩。”舒克从贝塔手里拿过小刀,别在腰间。舒克讲义气,知恩必报。 
        “我请示一下皮皮鲁吧。”贝塔拿起通话器同皮皮鲁通话。 
        皮皮鲁挺理解舒克的心情,他同意舒克亲自去把秃顶总编的名字从解剖主任的作品上弄下来。 
        “去报恩吧,我掩护你。”贝塔伸出两个手指头作了个V形手势。 
        舒克走到舱门旁。贝搭按电钮,舱门打开。舒克看看四周没有不安全因素后,离开五角飞碟。 
        贝塔关上舱门,他坐在操纵台前,打开遥感器,日不转睛地追踪观察舒克。 
        舒克找到一根管道,他准备通过管道进人印刷车间。 
        管道里黑乎乎的。舒克深一脚浅一脚地行进着。 
        “站住!别动!”一声大喊。 
        吓了舒克一跳。 
        连坐在操纵台前看荧光屏的贝塔也被吓了一跳。   第137集 
        头领认输; 
        扑克帮了舒克的忙; 
        意想不到的场面; 
        请皮皮鲁裁决   
        “你是谁?”舒克镇静下来,问。 
        “你是谁?”对方反问。 
        “我叫舒克。”舒克说。 
        “你来这儿干什么?”对方间。 
        “我路过这儿。”舒克说。 
        “路过?你去哪儿?” 
        “去……印刷车间……” 
        “这里都是我的地盘,不许别人进入。” 
        “你是谁?” 
        “和你一样。” 
        贝塔终于操纵遥感器找到目标了——藏在管道的一个凹处的十几只老鼠。 
        “舒克,是咱们的同胞,有十几只,你要当心!”贝塔通过无线电通讯器提醒舒克。 
        “你得留下买路钱。”同胞说。 
        “买路钱?我没钱。”舒克说。 
        “把你身上的衣服脱下来。”同胞看上了舒克的衣服。 
        “舒克,你没有多少时间了,版只要装上印刷机,咱们可就无能为力了,你的救命恩人只得在自己写的书上和那秃顶总编联合署名了。”贝塔提醒舒克 
        “你们最好让我过去。”舒克的口气开始强硬起来。 
        “他还真敢说。我来教训这小子一下。”一位同胞从黑影中走出来。 
        舒克刚动过手术,肯定打不过这位膀大腰圆的同胞。 
        “我帮你。你只要象征性地挥挥拳头就行了。”贝搭一边打开五角飞碟的武器系统一边通知舒克。 
        “你悠着点儿,别伤他们。”舒克通过别在领子上的送话器对贝塔说。 
        那位自告奋勇要教训舒克的同胞已经站在舒克的面前,其余的同胞起哄助威。 
        舒克出拳了——他离那位同胞的距离在一拳头够不着的地方。 
        那位同胞仰面朝天倒在地上,嘴里发出的音调都与自己的祖先有关。 
        “上!”发出命令的显然是头领。 
        十几只老鼠朝舒克扑过来。 
        舒克漂亮地转了一圈,同胞们争先恐后地倒下。 
        “还要买路钱吗?”舒克问黑影中的头领。 
        “不要了……”头领的声调里增加了颤音。 
        “这管道通印刷车间吗?”舒克问路。 
        “通……”头领庆幸对手没觊觎他的王位。 
        “以后活得仁义点儿,拜拜。”舒克礼貌地向头领告别后向管道的纵深处走去。 
        那头领望着这个穿衣服的同胞的背影发呆。 
        “他去印刷车间干什么?印刷车间里有一座暗室?暗室里是金库?”头领很有几分想像力。 
        舒克通过管道进入了印刷车间。 
        那个存放《动物解剖学探秘》胶版的木制文件柜坐落在车间的一个角落里。 
        几个工人聚在一起吃午饭。印刷机静静地躺在那里。 
        舒克沿着墙角溜到文件柜底下,他通过抽屉旁边的缝隙进入抽屉中。 
        舒克现在就站在厚厚的一叠《动物解剖学探秘》胶版上。最上边的一张就是印有解剖主任和秃顶总编大名的那张胶版。 
        舒克抽出小刀,准备将秃顶总编的名字从胶版上拉下来。 
        “舒克!注意!有人朝文件柜走来!”贝塔向舒克发警报。 
        舒克听到了脚步声,跑出抽屉已经来不及了,他只能往里躲。 
        “这个文件柜有四层抽屉,他未必开我这个抽屉。”舒克想。 
        贝塔将五角飞碟的武器对准了那人,随时准备保护舒克。 
        那人偏偏拉开了舒克呆的抽屉。 
        贝塔目不转睛地盯着荧光屏,手指放在射击按钮上。 
        抽屉拉开了四分之三,亮光刺得舒克睁不开眼睛。 
        “头儿,还没过午休时间,和我们打一回扑克吧!”围坐在一起的工人叫开抽屉的人。 
        “《动物解剖学探秘》马上就要开机了,我先把版装上去。”头儿说。 
        “呆会儿再装,先玩一把,离上班时间还有15分钟呢!”工人们坚持。 
        “也好,打一把就打一把。”头儿转身朝工人们围坐的地方走去。抽屉没关。 
        舒克和贝塔松了一口气。 
        “干吧,我替你放风。”贝塔在五角飞碟里对舒克说。 
        抽屉是打开的。舒克站到秃顶总编的名字上,用小刀开始往下割印有秃顶总编名字的胶片。 
        秃顶总编的名字被取下来了,舒克将那块胶片装进衣兜里。 
        “快回来。”贝塔对舒克说,他看见文件柜四周无人。 
        舒克迅速离开文件柜,沿原路返回五角飞碟。 
        贝塔帮舒克掸身上的尘土。 
        “那帮小子还挺凶,也不看看对手是谁。”贝塔神气地贬打劫舒克的那伙同胞。 
        “说实话,好长时间投见过同胞了,见了他们,我还有点亲切感呢。”舒克说。 
        “你自从动了手术后,特多愁善感。是不是麻药伤着你的哪根神经了?”贝塔睁大眼睛观察舒克的头部。 
        “去!”舒克用手掌挡贝塔的视线。 
        “舒克贝塔!舒克贝塔!我是皮皮鲁,进展如何?”五角飞碟舱里传出皮皮鲁的询问。 
        “任务已经完成。”贝塔拿起话筒说。 
        “返航。”皮皮鲁指示。 
        “明白。”贝塔在座椅上坐好。他开始启动飞碟的动力系统。 
        舒克心里感到不踏实,他又打开遥感系统。 
        荧光屏上的景象令他大吃一惊。 
        “贝塔,你看!”舒克大喊。 
        十几只老鼠手持刀枪棍棒正站在《动物解剖学探秘》的胶版上准备毁版。 
        版如果毁了,解剖主任的心血也就完了。舒克不但没报答恩人,反而恩将仇报。 
        “快制止他们!”舒克冲贝塔喊。 
        贝塔调整武器系统瞄准那群老鼠。 
        “别伤害他们,击倒就行了。”舒克提醒贝塔。 
        “你的事儿可真多。”贝塔撇撇嘴,按下射击按钮。 
        那群老鼠倒下了,可不知从哪儿又冲进抽屉几十只! 
        贝塔继续射击。老鼠不顾一切地前仆后继。 
        “马上要开印了,如果工人们发现抽屉里躺着几十只老鼠,准会重新检查胶版,只要检查,就会发现秃顶总编的名字毁了。”舒克说。 
        “那怎么办?”贝塔意识到局势的严重性。 
        “我要去一趟。我看还是刚才劫我的那帮家伙。”舒克往舱门走。 
        “我掩护你。见了同胞别光多愁善感,应该多长几个心眼儿。”贝塔说。 
        原来,头领派了一只老鼠跟着舒克,看他去车间里到底干什么。舒克离开抽屉后,跟踪舒克的老鼠判明抽屉里是一本名叫《动物解剖学探秘》的书的胶版后,回去向头领禀报。 
        “《动物解剖学探秘》?”头领一愣,“专门研究怎么杀动物的书?” 
        “正是。”跟踪舒克的老鼠证实。 
        “那只老鼠干了些什么?”头领问。 
        “他拿刀子划那本书的胶版。” 
        “破坏那本书的出版?” 
        “大概是。” 
        “好样的,咱们也去破坏,不能让这种研究怎么杀害动物的书出版!”头领义愤填膺。于是,就有了舒克和贝塔在荧光屏上看到的那一幕。 
        舒克赶到文件柜里时,头领正指挥部下赴汤蹈火般地往抽屉里冲。 
        “住手!”舒克大喝一声。 
        头领一看是舒克,忙说:“英雄!英雄!!这种书就是不能出版!” 
        “为什么?”舒克不明白。 
        “这是专门研究怎么屠杀咱们动物的书,当然不能让它出版!”头领义正词严。 
        舒克愣了。是呀,这的确是一本专门研究如何拿动物做试验的书,它的出版,必将导致许多动物被夺走生命。 
        “舒克,你在干什么?”贝塔问。 
        舒克将头领的话转述给贝塔。 
        “有道理,不能让这本书出版!”贝塔恍然大悟。 
        “可……”舒克想起了救命恩人,这是他几十年的心血。 
        “应该出版!”舒克必须报答救命之恩。 
        “不能出!”贝塔反对。 
        “应该出!”舒克坚持。 
        离印刷车间工人上班的时间只有5分钟了。 
        “让皮皮鲁裁决吧!”舒克想了个主意。   第138集 
        皮皮鲁语塞; 
        头领捶胸顿足; 
        解剖主任发誓给舒克当终身保健医生   
        贝塔通过无线通讯系统向皮皮鲁汇报突发事件。 
        “你说什么?再重复一遍。”皮皮鲁怀疑自己没听清。 
        “《动物解剖学探秘》是一本专门研究怎么拿动物做试验的书,说白了,是一本专门探讨怎么杀动物的书,我不同意让这本书出版!”贝塔情绪激动地说。 
        “这……”皮皮鲁显然没想到会出现这样的局面,“舒克也是这个意思?” 
        “舒克为了报答救命恩人,放弃了原则,他坚持让这本书出版。”贝塔说,“现在我们请你裁决。” 
        “还是让它出版吧,这是学术著作。”皮皮鲁做贝塔的工作。 
        “如果有一本专门研究怎么拿人做试验的书,你同意它出版吗?”贝塔问皮皮鲁。 
        “……当然……不……可是……”皮皮鲁语塞。 
        “动物也是生命,是生命就有生存的权利。人类光讲人权,这是种族主义的表现!应该讲生命权。每个生命都有生存权,都应该有生命权!人权是一种种族歧视的提法。这种提法不能再继续下去了!”贝塔口若悬河,像在宣读一项划时代的宣言。 
        皮皮鲁哑口无言。 
        “皮皮鲁,你快决定,印刷车间上班的时间快到了!”舒克加入争论,“我觉得咱们已经答应了解剖主任,要讲信用。” 
        “贝塔,你听我说。”皮皮鲁说,“我确实忽视了这本书的性质。但具体情况要具体分析。这本书的作者救过舒克的命,咱们已经答应了帮助他,如果咱们出尔反尔,这样的生命还有什么活头儿?我看这样吧,先让这本书出版,等到发行时,咱们再想办法阻止它发行到读者手中,你看怎么样?” 
        “……嗯……好吧,就这样。”贝塔同意了。 
        贝塔和舒克统一认识后,头领就不在话下了。几分钟后,头领和他的部下都被五角飞碟的武器系统“搬”离了文件柜。 
        印刷机开始转动。没有秃顶总编署名的《动物解剖学探秘》开始印刷。 
        印刷工人没发现秃顶总编的大名在开机印刷前被删除了。 
        印刷后的纸张经过流水线进入剪裁和装订工序,一本本装帧精美的《动物解剖学探秘》离开流水线被包装起来。 
        头领和部下们在管道里捶胸顿足,他们不明白那个穿衣服的同胞为什么要帮人类出版这本专门研究怎样杀动物的书。 
        舒克不忍心甩下同胞就走,在返航前,他第三次会见头领。 
        “你放心,我不会让这本书流传的。”舒克对头领说。 
        “那你干吗允许它印出来?”头领已经不敢惹舒克了。 
        “有些事挺复杂,一时说不清。但有一点请你记住,我也是动物。”舒克拍拍头领的肩膀。 
        头领一脸的迷惘。 
        五角飞碟顺利返航。皮皮鲁松了口气。 
        舒克和贝塔正在用餐。有人敲门。 
        皮皮鲁从门镜里往外窥视,是解剖主任。 
        舒克和贝塔躲进里屋继续用餐,解剖主任坐在客厅里同皮皮鲁说话。 
        “总编辑打电话通知我,明天上午开那本书的首发式和记者招待会。”解剖主任说。 
        皮皮鲁点点头,说:“我已将总编的名字从书上拿掉了。” 
        解剖主任大喜:  “太感谢你了!” 
        皮皮鲁摆手:“这是应该的,反对不劳而获人人有责。” 
        解剖主任犹豫了一下,又说:“那位责任编辑要从稿费中提成的事……” 
        皮皮鲁从台历上撕下一张纸,把自己的电话号码记在纸上递给解剖主任:“你就按他的要求把钱给他,给钱后打个电话告诉我就行了。” 
        解剖主任接过纸,揣进内衣的口袋里,他做了个几乎给皮皮鲁下跪的姿势,说:“您的恩情,我今生今世也报答不完。以后您的老鼠朋友再有什么病,您尽管找我,我全包了。” 
        “帮你把总编辑的名字从书上拿下来的正是你救的那只小老鼠。”皮皮鲁告诉解剖主任。 
        “善有善报。善有善报……”解剖主任感慨万分。 
        皮皮鲁送走解剖主任后,贝塔和舒克从里屋出柬。 
        “以后舒克得多少次阑尾炎都没事了,那主任说舒克的病他全包了。”贝塔逗舒克。 
        “阑尾炎只得一次,要得也该轮到你了。”舒克用手指捅贝塔的肚子。 
        “明天上午咱们注意观察首发式,仪式结束后,等新闻记者回去发了消息,你们就去库房把《动物解剖学探秘》都毁了。”皮皮鲁吩咐。 
        “对,决不能让更多的人掌握杀动物的窍门。”贝塔拍手称快。 
        “在毁那本书前,咱们应该和解剖主任打个招呼。”舒克念念不忘救命恩人的利益。 
        皮皮鲁点点头。   第139集 
        演员大作家甲乙丙丁给总编辑捧场; 
        青年记者向总编辑发难; 
        五角飞碟准备出击   
        第二天上午,皮皮鲁早早地打开了监视器的荧光屏,贝塔操纵五角飞碟的遥感器对准《动物解剖学探秘》的新书首发式会场。 
        会场布置得豪华典雅,鲜红醒目的会标横贯会场东西。会标下一条长桌面对听众席,长桌后边坐着一排一看就知道是属丁那种装孙了类的道貌岸然的人物。解剖主任也坐在里边。听众席上坐满了记者。有拿笔的,有拿照相机的,有拿录音机的,有拿摄像机的。 
        “中间那个就是总编辑。”贝塔指给皮皮鲁看。 
        皮皮鲁怀着幸灾乐祸的心情看着那满面春风的总编辑。总编辑显然对于自己的大名已经被人从书上拿掉毫无所知,只见他一会儿和来宾握手一会儿双手抱拳向熟人作揖,看得出,总编辑特喜欢出风头。 
        “祝贺你啊,真是大手笔呀!哈哈,一会儿别忘了签上名送我一本啊,哈哈……”一位连路都走不稳了的干巴老头儿向总编辑祝贺。 
        “他连那本书都没看过,怎么知道是大手笔?”贝塔愕然。 
        “哟!莫老也来啦!您可是文学泰斗啊!累坏了您的身子我可无法向您的读者交代呀!”总编辑忙做搀扶大作家状。 
        “他就是写《夜雨不再来》的那位作家?”舒克特喜欢那本书,他终于见到了该书的作者。 
        新书首发仪式开始举行,主持人向与会者介绍来宾。 
        总编辑还真有点儿面子,那么多闻名遐迩的大作家来给他捧场。随着长桌子后边的人一个挨一个地站起来让主持人介绍又一个挨一个地坐下,皮皮鲁听着那些如雷贯耳的大名一时不知所措,他还从来没见这么多大作家集中在一个屋顶下。舒克更是两眼发直。只有贝塔若无其事地一边嗑瓜子一边吹口哨。 
        大作家们统统致辞向新书的作者表示祝贺。大作家甲说《动物解剖学探秘》是里程碑式的学术著作;大作家乙说他和总编辑在30年前就是朋友;大作家丙说如果让他写这种书他一个字也写不出来;大作家丁说总编辑一生甘为他人作嫁衣裳如果总编辑也写书那么在座的作家可能都会没饭吃了。 
        “真假。原来这就叫新书首发式,不三不四的人说不痛不痒的话花不明不白的钱。”贝塔给新书首发式下定义。 
        舒克瞪了贝塔一眼,他不允许贝塔亵渎他崇拜的这些大作家。 
        “怎么啦?”贝塔不明白舒克干吗瞪他。 
        “你别乱骂人,这些人可是受人尊敬的作家。”舒克说。 
        “我怎么觉得他们一个个俗不可耐,你看他们坐在这种场合里的表情,就像吸了海洛因。你别瞎崇拜,不信咱们选一位大作家,一天24小时遥感监视他,保准到第6小时你这辈子也不想再见他丁。”贝塔反驳舒克。 
        舒克看皮皮鲁。 
        “依我看,作家不像贝塔说的那么不值钱,也不像舒克说的那么值钱。”皮皮鲁裁决。 
        “反正这个会场里的作家没一个正经作家,真正的大作家才不到这种场合来呢。你瞧他们脸上那种陶醉的表情。作为作家,这种表情只应该在写作的时候才会有。凡是在社交场合脸上出现这种表情的人都不是真正的作家。”贝塔剖析荧光屏上的作家。 
        皮皮鲁微微点头。 
        “快看,好戏开始了!”贝塔兴奋地指荧光屏。 
        总编辑和解剖主任分别将大作赠送给诸位大作家和记者们。 
        来宾纷纷要求作者在新书上签自己的大名。 
        大作家丁首先发现书上没有总编辑的名字。他把这个发现告诉了邻座的大作家甲。大作家甲又转告给大作家乙。 
        5分钟后,会场鸦雀无声,与会者从不同的角度注视总编辑。 
        精明的总编辑发现了会场上的这个变化。 
        “怎么啦?出什么事了?”总编辑问身边的大作家丙。 
        大作家丙把自己手中的一本《动物解剖学探秘》送到总编辑眼前,他用手指书上作者署名的那一块地盘。 
        总编辑的脸涨红了,是那种连傻瓜也能看出的做贼心虚式的脸红。 
        “太不像话了,这是严重的失误!”总编辑满场找秘书。 
        “我想向总编辑提个问题。”一个青年记者站起来。 
        会议主持人只能同意。 
        “您能说出这本书第二章写的是什么吗?”青年记者发难了。 
        “第二章……第二章……就是……”总编辑头上开始层出不穷地冒汗,“你们看我这记性,连自己写的书都记不住了。” 
        “我向解剖主任提个问题。”一位女记者站起来,“请问您这本书是哪年开始写的?” 
        “3年前。”解剖主任回答。 
        “您是哪年认识总编辑的?”女记者问。 
        “去年。”解剖主任看了总编辑一眼。 
        “这就怪了,您和总编辑是去年才认识的,你们怎么能在3年前就台作写书呢?”女记者不等解剖主任回答就坐下了。她不需要答案。 
        会场气氛凝固了。 
        大作家们开始交头接耳。 
        总编辑如坐针毡,他恨不得一口吞了解剖主任。 
        “不属于自己的东西最好别要。”贝塔叹了口气.好像十分同情总编辑目前的处境。 
        “看到别人有了成就,就想分享。”皮皮鲁连连摇头。 
        “可以分享别人的喜悦,但不能分享别人的成就。”舒克离开监视器,他不想再看这种场面了。 
        午饭后,电话铃响了。 
        解剖主任打来的。他除了向皮皮鲁道谢外,顺便告诉皮皮鲁,责任编辑从他的稿费中拿走4000元。 
        皮皮鲁记下了责任编辑的住址。 
        “你们驾驶五角飞碟去把责任编辑勒索解剖主任的那4000元钱拿回来,然后再……”皮皮鲁向舒克和贝塔下达任务。 
        “明白了。”舒克接过责任编辑的住址,和贝塔走进五角飞碟。 
        贝塔一走进五角飞碟就兴奋。就有安全感。就天不怕地不怕。 
        “我觉得,在五角飞碟外边,咱们是弱者。可一走进五角飞碟,咱们就成了强者。”贝塔若有所思地说。 
        舒克不吭气。 
        “你怎么了?”贝塔看出舒克有心事。 
        “你看那些作家,手无缚(又鸟)之力,可写出的作品却能征服千百万人。这才叫强者。”舒克一脸的憧憬。 
        “就刚才那几个破作家就把你羡慕成这样子?告诉你,你根本就没见过真正的作家。”贝塔教训舒克。 
        舒克看着贝塔没说话,他满脑子都是有关写作的思维。不知怎么搞的,今天舒克突然对当作家萌发了兴趣。 
        “准备好了吗?可以起飞了。”皮皮鲁问。 
        “准备好了。”舒克打开操纵台上的一系列开关。 
        “起飞。”皮皮鲁下令。 
        五角飞碟离开桌面,从窗户飞出屋子。 
        责任编辑的住址输入了五角飞碟的电脑。五角飞碟很快就到达责任编辑居住的那座楼房的上空。 
        “降落在楼顶上。楼顶无异常。”贝塔观察后告诉舒克。 
        舒克操纵五角飞碟稳稳地在楼顶上着陆。 
        贝塔打开遥感器。 
        一间布置奢侈的房间里坐着一个四十多岁的男人,那男人正在数钱。 
        “这大概就是责任编辑。”贝塔凭直觉断定。 
        舒克看了一下门牌号码,说:  “没错。” 
        责任编辑数完钱,装进自己的上衣口袋。 
        “喂,打电话了吗?”责任编辑冲厨房喊。 
        “我正做饭呢,你自己打!”厨房里传出一个女人的声音。显然是责任编辑太太。 
        责任编辑拿起茶几上的一则广告拨电话。   第140集 
        像电影一样的电视屏幕; 
        贝塔把钱变成废纸; 
        警察光临   
        “你猜他给哪儿打电话?”贝塔问舒克。 
        “猜不出。”舒克摇头。 
        “准是给警察局打。” 
        “警察局?为什么?” 
        “良心发现呗,向警察局自首。说自己勒索作家多少多少钱。”贝塔穷开心。 
        “即使转变,也不可能这么快。”舒克一本正经地否定贝塔的假设,“再说,像这种勒索作家稿费的罪过,法院轻判不了,他不会刚拿了钱马上就去自首,也没见他数完钱后看什么政治读物。” 
        “看样子电话拨通了,听听他说什么。”贝塔指荧光屏。 
        “喂,请问是飞花超级市场吗?我是客户。我在报纸上看到你们开展电视机送货上门服务,对对,我需要一台超大屏幕的彩色电视机。我的住址?请你记一下。什么时候要?现在就要。半小时后就可以送到?太好了。谢谢。”责任编辑挂上了电话。 
        “这小子用你救命恩人的钱更新换代他家的电视,真够黑的。你说怎么教训他?”贝塔把制定方案的大权拱手让给舒克。 
        “把他衣兜里的钱运到五角飞碟上来,再往他兜里塞一叠废钱。”舒克说。 
        贝塔笑得前仰后台:“我还以为你比我仁义呢,没想到你更损。” 
        “掌握好调换的时机,早了晚了都不行。”舒克告诫贝塔。 
        “放心吧,你救命恩人的钱一分也少不了。”贝塔调整操纵台上的各种开关旋钮,准备换钱。 
        责任编辑的太太端着饭菜从厨房走出来。 
        “电视一会儿就到!”责任编辑喜形于色。 
        “这是你用作者的稿费换的第五代电视了吧?”太太冲先生飞了个媚眼。 
        “以后还会有第六代第七代第一百八十代,你就跟着我享福吧!”责任编辑哼起了难听的小调。 
        “花人家的钱你就不怕老天报应?”太太逗先生。 
        “我才不信那个。再说了,没有我,他们的书出得来吗?”责任编辑把自己的脸往太太脸上凑。 
        “去去,你的嘴老有味儿,用多少牙膏也没用。”太太把饭菜放在餐桌上,夺路而逃。 
        责任编辑哈哈大笑。 
        “像个屠夫!这种人居然是编辑!”舒克紧锁眉头。 
        “他在编辑里大概还算上等货呢!不信我现在给你遥感扫描一遍世上所有的编辑,你比较一下?”贝塔说干就干,伸手调旋钮。 
        “算了算了,别误了换钱。”舒克阻止贝塔。 
        贝塔看了一眼屏幕:“哟,送电视的还真来了。” 
        责任编辑开门,两名超级市场的售货员抬着一个大纸箱子出现在门口。 
        “换钱!”舒克说。 
        贝塔身边出现了一叠钞票。责任编辑兜里的黑钱被五角飞碟运来了。 
        同时,一叠废纸神不知鬼不觉地进人责任编辑兜里。 
        “看戏吧。”贝塔伸了个懒腰,把两条腿跷到操纵台上。 
        商场售货员将电视机从包装箱里拿出来。 
        “乖乖,真大呀!”贝塔还没见过这么大的电视机。 
        售货员调试电视机。 
        “简直像电影!”贝塔妒嫉人类没边的创造力。 
        “如果没有法律限制,我看他们最终能把电视造得比地球还大。”舒克感慨。 
        “您满意吗?”售货员问责任编辑。 
        “满意,满意。”责任编辑满面红光。 
        “如果您方便的话,付款后我们就可以走了。”售货员彬彬有礼,训练有素。 
        “当然,现在就付款。”责任编辑从农兜里掏出纸币,连看都投看就递给售货员——他已经数了10遍。 
        售货员接过纸币愣了,一摞废纸! 
        “您这钱?”售货员问责任编辑。 
        “钱怎么了?”责任编辑把视线从电视机上移到售货员手中。 
        “这不是钱。”售货员把手中的废纸递到责任编辑眼前。 
        责任编辑把手伸进衣兜,什么也没有。他一把接过售货员手中的废纸,来回翻了一遍,连一分钱都没有。 
        “钱呢?我的钱呢?”责任编辑红了眼,他问太太。 
        “你好好想想,放在哪儿了?”太太还挺镇静。 
        “就放在这儿了!”责任编辑把兜儿整个翻了过来。 
        “这废纸是从哪儿来的?”太太脑子还能推理。 
        责任编辑盯着废纸发呆。 
        两位售货员对视了一分钟。 
        “对不起,我们只能先把电视机抬回去了。”一位售货员说。 
        “不行!这电视机是我的,你们不能抬走!”责任编辑的尊严受到了侵犯,他咆哮了。 
        “可您没有付款。”售货员提醒责任编辑。 
        “我付款了!钱被你偷换了,我要求对你搜身!你不能离开这屋子!太太,快打电话叫警察!”责任编辑乱了方寸。 
        太太心虚,不打电话。倒是一位售货员拨了110。 
        “有好戏看了。”贝塔手舞足蹈。 
        警察根据责任编辑的要求搜了两位售货员的全身,没有钱。 
        “您的钱是从哪儿来的?”警察问责任编辑。他见过不少jiachao,但还没见过拿废纸冒充钞票的。 
        “从银行取的。”责任编辑话一出口就后悔了。 
        “哪家银行?”警察又问。 
        “……就是……不是从银行取的,”责任编辑语无伦次,“是我自己放在家里的……” 
        “积累的工资?”警察问。 
        “对!对!是积累的工资。”责任编辑连连点头。 
        警察将责任编辑的太太拉到一旁。 
        “您先生这儿没毛病吧?”警察指指自己的头。 
        “他是编辑!大学毕业!”太太的感觉是被人侮辱了好几次。 
        “那您先生就得跟我们走一趟了。”警察说。 
        责任编辑的太太后侮了,还不如说先生脑子有毛病呢。 
        “你们凭什么抓我?我是堂堂的编辑!”责任编辑抗议。 
        “不是抓您,是请您去警察局,我们要了解一些情况。”警察笑容可掬地更正责任编辑的概念。 
        两名售货员将电视机装进包装箱。 
        “麻烦你们二位也到局里去一趟。”警察对售货员说。 
        售货员耸耸肩,表示无可奈何。   第141集 
        解剖主任绐皮皮鲁下跪; 
        蚂蚁的生命和人类的生命一样重要; 
        总编辑发呆 
        五角飞碟返航,平安回到皮皮鲁家中。 
        皮皮鲁看了录像片,对舒克和贝塔此次出击的结果表示满意。 
        “那编辑现在可能还在警察局呢!”贝塔一边喝饮料一边说。 
        舒克把从责任编辑兜里夺回的稿费交给皮皮鲁。 
        皮皮鲁给解剖主任打电话。 
        “您的稿费在我这儿,请您来一趟。”皮皮鲁对解剖主任说。 
        “您真是神人,我这就去。”解剖主任恨不得钻进电话线里直奔皮皮鲁家。 
        皮皮鲁挂上电话。 
        “咱们该毁那些《动物解剖学探秘》了吧?”贝塔问皮皮鲁。 
        “等那主任来了,我就告诉他这件事。”皮皮鲁看表。 
        当解剖主任从皮皮鲁手中接过那本来属于他的稿费时,他再也控制不住自己了。 
        “扑通!”解剖主任给皮皮鲁跪下了。 
        钱的力量。 
        皮皮鲁把解剖主任扶起来。 
        “还有一件事要告诉你。”皮皮鲁示意解剖主任坐下。 
        解剖主任把稿费放到贴身的内衣口袋里。 
        “我们得把您的著作全部销毁。”皮皮鲁看着解剖主任说。 
        “你说什么?”解剖主任以为自己没昕清楚。 
        皮皮鲁重复了一遍。 
        “你逗我玩儿?”解剖主任往好的方面想。 
        皮皮鲁摇头。 
        “为什么?”解剖主任呼吸急促起来。 
        “您写的这本书,是一本专门研究怎样拿动物做试验的书。说白了,是一本传授怎么杀害动物的书。我的老鼠朋友不能允许这本书流传。”皮皮鲁严肃地说。 
        “可……这是学术著作……”解剖主任为自己的书辩解。 
        “如果有人写了一本专门研究怎么拿人做试验的学术著作,您能同意吗?”皮皮鲁问解剖主任。 
        “动物和人……不能相提并论吧?”解剖主任说。 
        “都是生命。我以为,宇宙中的所有生命都一样重要。人的生命是生命,蚂蚁的生命同样是生命。从本质上讲,这两种生命的价值是一样的。人的生命不比蚂蚁的生命高贵,蚂蚁的生命也不比人的生命低贱。宇宙的最高级社会形式将是所有生命的平等相处。和平共处和互相尊重。”皮皮鲁用比较低沉的语气说,更显得他的话有力度。 
        “那你们为什么还要帮助我把总编辑的名字从书上拿掉?”解剖主任不理解。 
        “第一,报答你的救命之恩;第二,像总编辑那种人应该受到制裁。”皮皮鲁眼睛看着窗外说,好像空气中弥漫着正义。 
        “要报答救命恩人就不能毁了我的书。”解剖主任使用哭腔说。 
        “你已经出了名,明天的报纸就会有不少你的消息,稿费你也全部拿到了。我们允许你留下100本书,你可以把它们放在家里的书柜中。”皮皮鲁为解剖主任想得挺周全。 
        “计划不能改变了?”解剖主任知道皮皮鲁的厉害,他清楚不能违抗皮皮鲁。 
        “不能。我给您提个建议,您可以写一本有关爱护动物的学术著作,我一定支持您。”皮皮鲁说。 
        解剖主任伸出自己的两只手看。他的手解剖了成千上万的动物。用这双手写一本爱护动物的书?解剖主任苦笑。 
        “您解剖了那么多动物,有什么感受?您只救了一只动物,您又有什么感受?”皮皮鲁给解剖主任的大脑指示思维路数。 
        解剖主任微微点头。 
        “我回去就辞职不千了,我开一家动物诊所,专为动物治病。”解剖主任宣布。 
        “您能活两百岁。”皮皮鲁说。 
        解剖主任向皮皮鲁告辞。 
        解剖主任走后20分钟,五角飞碟出击。印刷厂库房里的全部《动物解剖学探秘》连同版一起被销毁。 
        当出版社总编辑得到所有的《动物解剖学探秘》都在库房里被销毁的信息时,他发呆了整整1个小时。一系列的怪事令他百思不得其解:自己的名字奠名其妙地从书上消失;责任编辑因使用废纸当钱被警察拘留;书未遇火灾未遇水灾未遇盗灾好端端的被毁…… 
        总编辑头一次隐约感到了上帝的存在。   第142集 
        贝塔半夜醒来发现舒克不在; 
        老鼠向人类作家挑战; 
        笔名叫舒皮贝   
        半夜,贝塔一觉醒来,看见身边没有舒克。 
        “这小子干什么去了?”贝塔坐起来。 
        客厅里有灯光。贝塔下床来到客厅门口扒着门缝儿往里看。舒克在灯下不知干什么。贝塔溜到舒克身后,只见舒克拿着一支铅笔在纸上写字。 
        “给谁写信呢?”贝塔问。 
        “什么写信,是写作。”舒克头也不回。 
        “写作?!”贝塔一愣,“什么写作?” 
        “就是写作品。”舒克不停笔。 
        “你想当作家?”贝塔笑弯了腰。 
        自从看了新书首发式,舒克对那些作家羡慕得不得了,他觉得作家是人类中最神奇最了不起的成员,他们靠一支笔竟然能编出一个个在地球上本没有的故事,他们把一个世界变成了几百个几千个乃至几万个世界。他们创造生命不像别人那样还得先爱再领结婚证再完婚再生孩子,他们的笔只要那么一动,一个个活生生的人就会诞生在地球上,人类就会多一个成员多一个朋友,而且是不需要占用能源占用住房占用就业机会占用异性的成员! 
        作家真是太伟大了。舒克走火人魔了,他也想当作家。 
        贝塔觉得好玩儿,他叫醒了皮皮鲁。 
        “你说什么?舒克要当作家?”皮皮鲁一边打哈欠一边往客厅走。 
        看到舒克坐在写字台上的台灯下边,皮皮鲁信了。 
        “我看看你写的作品。”皮皮鲁拿走舒克的手稿。 
        舒克的表情像等待法官判决。 
        “不错呀!”皮皮鲁睡意全无,“真正的大手笔!” 
        舒克激动了。贝塔像不认识似的看着舒克。 
        “舒克,你能当作家,一定能!”皮皮鲁肯定舒克的写作才能。 
        “舒克没有学历呀!作家得有学历吧?”贝塔给舒克浇冷水。 
        “其实,书读多了,人会变蠢。从这个意义上说,最高的学历就是无学历。”皮皮鲁给舒克打气。 
        “到底什么是作家?皮皮鲁,你给作家下个定义。”贝塔要弄清作家的含义。 
        “能控制住自己的精神病患者。”皮皮鲁给作家下了定义。 
        “舒克准行。”贝塔改变立场。 
        “你有什么计划吗?”皮皮鲁问舒克。 
        “我想写一部长篇小说,书名叫《人类,我是你的朋友》。”舒克向皮皮鲁和贝塔介绍了自己的构思。 
        “你口述,我打字,这样快。”皮皮鲁坐在打字机前。 
        “都是精神病患者,就快控制不住自己了,深更半夜写小说,世界上有那么多大作家,哪儿轮得上你们。”贝塔回房间睡觉去了。 
        舒克口述,皮皮鲁打字,《人类,我是你的朋友》的第一章很快就写出来了。 
        “舒克,真没想到,你还有写作天才!”皮皮鲁一边打字一边夸奖舒克。皮皮鲁认定写作是一种天赋,有的作家辛辛苦苦一辈子,由于没有写作天赋,只能当三流作家。写作不是靠学问,是靠感觉。感觉学不来,是天生的。舒克的感觉很准。 
        经过一个月的努力,《人类,我是你的朋友》完稿了,全书80万字。就连贝塔也不能不承认,这部小说太有吸引力了。 
        “舒克是杀手级作家。”贝塔说。 
        “杀手级作家?”皮皮鲁头一次听这个名词。 
        “杀手级作家能用自己的作品扼杀别的作家的作品。”贝塔解释自己创造的新名词。 
        “咱们给杀手作家舒克起个笔名吧。”皮皮鲁提议。 
        “干吗不用真名?”贝塔问。 
        “出版社如果知道书是老鼠写的,非有麻烦不可。”皮皮鲁提醒贝塔别忘了搞主任。 
        “就叫皮舒贝吧。”舒克给自己起了个笔名。 
        “书是你写的,还是叫舒皮贝吧!”皮皮鲁不同意把他的姓放在最前边。 
        “反正我总是最后。”贝塔耸耸肩。 
        小说稿全部完成,书名:《人类,我是你的朋友》,作者:舒皮贝。 
        “出版社怎么和咱们联系呢?不能留地址吧。”皮皮鲁说。 
        “当然不能让出版社知道作者的真实身分。你给编辑写一封信,信上说,此书的稿费全部捐献给全球动物保护委员会,版权也归动物保护委员会。”舒克献出了自己的作品的版权和版税。 
        “出版社如果不把稿费给动物保护委员会呢?”贝塔认定出版社都由恶狼控制。 
        “咱们有五角飞碟呀!”舒克指指桌上的五角飞碟。 
        “你这小子脑子是够用的。”贝塔不得不服。 
        上午9点整,皮皮鲁将书稿交给一家大出版社的传达室。 
        “请您将这部书稿转交给文学编辑室的编辑。”皮皮鲁把牛皮纸口袋递给传达室里的老头。 
        “你的通讯地址写清楚了吧?”老头连眼皮都不抬。 
        “写清楚了。”皮皮鲁说。 
        老头把书稿收进去。 
        皮皮鲁赶回家里。 
        “开五角飞碟的遥感器,看看那家出版社的编辑怎么处理舒克的大作。”皮皮鲁一进门就对贝塔说。   第143集 
        瘦编辑错失良机; 
        胖老板慧眼识英雄; 
        老鼠首次获诺贝尔文学奖   
        贝塔钻进五角飞碟,打开遥感器,调出了接收舒克大作的那家出版社。 
        皮皮鲁打开客厅中的电视监视器接收五角飞碟的信号,出版社出现在荧光屏上。 
        一位骨瘦如柴的男人走进传达室拿报纸。 
        老头把舒克的书稿递给他。 
        瘦编辑连看都不看就把牛皮纸口袋和报纸裹在一起拿走了。 
        回到编辑室,瘦编辑把牛皮纸口袋撕开,拿出舒克的书稿。 
        皮皮鲁、舒克和贝塔屏住呼吸,盯着瘦编辑的表情。 
        瘦编辑看了一眼作者署名。 
        “舒皮叹。”他嘟囔了一句,“无名小卒。” 
        瘦编辑把书稿塞回牛皮纸信封。 
        “这人运气太差,本来他可以出大名的。”皮皮鲁为瘦编辑惋惜。 
        一位大腹便便的老板模样的中年男子来到瘦编辑的办公桌旁。 
        “最近有什么好稿子吗?现在出版社之间的竞争太厉害了。”老板问。 
        “没什么好稿子,我正准备去向名家约稿。”瘦编辑毕恭毕敬地说。 
        老板顺手拿起舒克的书稿。 
        “谁写的?”老板从纸袋中抽出书稿。 
        “没名。是新作者。水平线以下。”瘦编辑信口胡诌,他连看都没看过舒克的稿子。 
        老板只看了三行字表情就变了,他忘记了自己是在下属的办公桌旁,他坐在瘦编辑的椅子上,埋头看舒克的大作。 
        瘦编辑傻眼了,诚惶诚恐地侍立在老板的身边。 
        “好!” 
        “太好了!!” 
        “哎呀,大手笔!!!” 
        “传世之作!!!!” 
        “小说终结者!” 
        老板赞不绝口。 
        不少编辑被老板的赞叹声吸引来了。 
        “谁写的。”同行们问瘦编辑。 
        瘦编辑记不住作者的名字。 
        “太——好——啦——”出版社老板一边看一边喊。 
        “比托尔斯泰还好?”一位编辑问老板。 
        “好100倍!”老板脱口而出。 
        “比莎士比亚呢?”又有人问。 
        “好l00倍!”老板眉飞色舞,眼不离稿。 
        “比巴尔扎克呢?” 
        “好l00倍!” 
        众编辑站成一排,流水作业看舒克的小说。 
        “把队伍排到排版室去,立即排版!”老板发令。 
        出版社从未有过的壮观景象: 
        老板一页一页看稿,编辑们依次排队一直站到排版室里,每个人看完一页稿子再传给下一个,一直传到排版室的打字小姐手中。 
        舒克心花怒放。皮皮鲁的拳头在空中无数次地挥动。贝塔连喝了10杯酒。 
        三天后,《人类,我是你的朋友》出版了。该书上市的第一天,就卖了15万本。创造了图书销售史上的最高记录。 
        评论家们争先恐后地评论舒克的大作,有位权威评论家说《人类,我是你的朋友》的问世宣告了古往今来所有作家的无能,还有一位评论家说只有上帝才能写出如此气势磅礴震动人心的作品。 
        舒克的书在短短一个月里就被翻译成了100种文字出版,累计印数超过了10亿册。 
        舒皮贝已经成了家喻户晓的名字,越是不露面.读者就越对舒皮贝感兴趣。出版《人类,我是你的朋友》的那家出版社发了大财,那老板还算仁义,他按舒克的要求将属于舒克的版税全部捐献给了动物保护委员会。 
        终于,诺贝尔文学奖评委会沉不住气了,他们决定将本年度的诺贝尔文学奖授予舒皮贝作家。原因是:舒皮贝使用史无前例的排列方法将人类的文字重新排列组合,其阵容震撼了全人类的每一位成员,使人类为自己的文化感到骄傲。 
        记者盼望舒皮贝能去领诺贝尔奖,以便一睹他的风采。诺贝尔奖评委会收到了作家舒皮贝的信,要求将奖金转交给动物保护委员会。 
        “你真给老鼠家族争气。”贝塔对舒克说。 
        “可惜你不能去领奖。”皮皮鲁遗撼地说,“还是不公平。如果读者知道了他们崇拜的作家是一只老鼠,不知会出现什么场面。” 
        “你说舒克为什么成功?”贝塔问皮皮鲁。 
        “两条。第一,他从旁观者的角度写人类。而人类的作家只是从人类的角度写人类。第二,他不和任何作家来往。”皮皮鲁总结道。 
        “作家只和作家交往无异于近亲结婚。”舒克出语不凡,  “生出的作品不是痴呆就是怪胎。” 
        “听说还有人专门把作家集中在一起,成立个协会什么的,这不是害作家吗?”贝塔偶尔也看报,知道点儿世界上的怪事。 
        舒克不懂文学。又没上过学。一个作家也不认识。他凭感觉写出的文学获得了诺贝尔文学奖。这是一件值得人类的每一位有作家头衔的人三思的事。   第144集 
        德高望重的大作家涛涛轰冒充舒皮贝; 
        舒克和贝塔相继呕吐; 
        记者证帮皮皮鲁进入会场   
        这天早晨,皮皮鲁从楼下的报箱里取回报纸,他坐在沙发上一边喝牛奶一边看报。舒克和贝塔仍在蒙头大睡。 
        头版头条的醒目大标题映入皮皮鲁的眼帘: 
        诺贝尔文学奖得主终于亮相 
        皮皮鲁吃了一惊,他放下手中的玻璃杯,双手拿起报纸,迅速阅读这条消息的正文。 
        这条新闻告诉读者,昨天下午,一位德高望重的老作家向新闻界透露,他就是《人类,我是你的朋友》的作者。舒皮贝是他的笔名。报纸还配发了那位老作家的照片,照片下的署名是涛涛轰。 
        皮皮鲁不敢相信这是真的。他从小就知道涛涛轰这位作家的大名,尽管他根本不知道涛涛轰写过哪些作品。只知道他名气贼大。皮皮鲁管这种作家叫七流作品一流名气。 
        “舒克!贝塔!”皮皮鲁喊。 
        “出了什么事?这么早就叫!”贝塔从五角飞碟里探出头。 
        舒克也从床底下钻出来。他俩睡觉一天换一个地方。 
        “有人冒充你!”皮皮鲁举起报纸给舒克和贝塔看。 
        “冒充我?人冒充老鼠?”舒克不信。 
        “说《人类,我是你的朋友》是他写的!”皮皮鲁指着报纸上的照片说。 
        “这不是涛涛轰吗?”舒克挺熟悉名作家的尊容。 
        “涛涛轰!”贝塔没听说过这名字。 
        “大名鼎鼎的作家!”舒克说。 
        “……写过……”舒克挠后脑勺,他怎么也想不起涛涛轰有哪些作品。 
        “涛涛轰对新闻界说,舒皮贝是他的笔名,《人类,我是你的朋友》是他写的。”皮皮鲁说。 
        “这不可能吧?”舒克死活不信德高望重的大作家会干这种低级的勾当。 
        皮皮鲁把报纸举到舒克眼前。舒克看完全文后,月光呆滞。 
        “你怎么了”贝塔用手在舒克眼前晃。 
        “怎么会呢?”舒克自言自语。 
        “报上还说,今天上午九点,电视台要为涛涛轰举行记者招待会。电视台现场直播记者招待会的实况。”皮皮鲁说。 
        舒克和贝塔看了一眼墙上的挂钟,现在是八点三十分。 
        “揭穿他!”贝塔大喊。 
        “我去电视台。当场戳穿他!”皮皮鲁穿外衣。 
        “我们呢?”贝塔问。 
        “你俩在家看电视。我带着通讯器,咱们随时保持联系。”皮皮鲁把微型通讯器装进衣兜里。 
        舒克和贝塔钻进五角飞碟等着收看涛涛轰的汜者招待会实况转播。 
        舒皮贝终于露面了,全世界的千百万读者得知这个消息后奔走相告,他们聚集在电视机前等候目睹这个崇拜已久的大作家的风采。 
        舒克自从看到这条消息后一言不发。贝塔了解舒克。他知道舒克到现在还不相信这是真的,他非得等到看了记者招待会才会相信。 
        九点整,记者招待会开始。 
        满面红光神采奕奕的涛涛轰出现在全世界面前。闪光灯组成的光环笼罩着他的全身。 
        “请问,您是《人类,我是你的朋友》的作者?”一位迷人的记者小姐首先提问。 
        “是的。”涛涛轰点头。 
        “您为什么要化名写这本书呢?”记者小姐继续问。 
        “我不想利用我的名气投稿,我想参与公平竞争。”涛涛轰回答。 
        全场鼓掌。 
        舒克吐了,贝塔给舒克捶后背。 
        “您的笔名为什么叫舒皮贝呢?这里边有什么含义吗?”另一位记者问。 
        “当然有含义。舒代表舒展,就是说写作要放得开。皮的意思是活生命要活本质,不要活皮毛。贝是说人类是宇宙中的宝贝。”涛涛轰信口雌黄。 
        贝塔也吐了。舒克忙给他捶后背。 
        “对于您的作品获得诺贝尔奖,您有什么感受?”有记者问。 
        “我当然高兴,特别是在诺贝尔奖评委会不知道这部作品是我写的前提下,授予我该项奖,更令我激动。这是对我的写作才能的肯定。”涛涛轰对答如流,风度翩翩。 
        “您认为对于一个作家来说,什么最重要?”一位记者问。 
        “良知。也就是人们常说的良心。”涛涛轰指着自己的心脏部位说。 
        “您最满意的作品是什么?” 
        “下一部。永远是下一部。” 
        “您为什么把诺贝尔奖金全捐给动物保护委员会?” 
        “这是我的初步决定。现在看来,这个决定还不成熟,可能还要修改。” 
        “他想拿这100万美元。”贝塔看出了涛涛轰的打算。 
        舒克缓慢地点头。 
        “皮皮鲁怎么还不露面?”贝塔等不及了。 
        “我呼叫他。”舒克打开通讯开关。 
        “皮皮鲁,我是舒克!请回答。” 
        “我是皮皮鲁。” 
        “你在哪儿?为什么还不揭穿涛涛轰。” 
        “我还在电视台门口,门卫不让我进,我没有记者证。”皮皮鲁说。 
        舒克对贝塔说:“快给皮皮鲁弄一个记者证。” 
        贝塔操纵五角飞碟的遥控装置将一个记者证“运”进皮皮鲁的兜里。 
        “皮皮鲁,你左边的衣兜儿里现在有一个记者证。”舒克告诉皮皮鲁。 
        皮皮鲁进人记者招待会会场。 
        舒克将涛涛轰刚才如何回答记者的问题都转告给皮皮鲁。 
        皮皮鲁走到涛涛轰面前,从他手中拿过麦克风。 
        “我说两句。”皮皮鲁对记者说。 
        记者们惊愕地看皮皮鲁,有人认出了他。 
        “他根本不是舒皮贝,他不是《人类,我是你的朋友》的作者。”皮皮鲁指着涛涛轰说。 
        会场哗然。 
        “请问,您是皮皮鲁吗?”还是那位记者小姐抢先问。 
        “我是皮皮鲁。” 
        “您的话有什么根据?”记者小姐问。 
        “我认识这本书的作者。”皮皮鲁说。 
        “你胡说!我是《人类,我是你的朋友的》的作者!”涛涛轰急了,他知道自己现在是面对全世界亮相,只要一露馅儿,这辈子就甭想在地球上混了。 
        “你有这本书的手稿吗?”皮皮鲁间。 
        “我……当然……”有……”涛涛轰的口气比较软。 
        “你能把手稿拿出来吗?”皮皮鲁问。 
        “我……我用电脑打字机写作,没有手稿!”涛涛轰出尔反尔。 
        “你用什么牌的电脑打字机?”皮皮鲁问。 
        涛涛轰张口结舌,他根本没有电脑打字机。 
        “我抗议!我请求将这个人驱逐出会场!”涛涛轰恼羞成怒。 
        “请问皮皮鲁先生,您是怎么知道涛涛轰先生不是该书的作者的呢?”记者小姐问。 
        “我认识这本书的作者!”皮皮鲁只能这么回答。   第145集 
        涛涛轰反击皮皮鲁; 
        作协主席的提议; 
        舒克再没拿笔; 
        相册勾起皮皮鲁的回忆   
        “作者是谁?” 
        “舒克。” 
        “舒克?”记者们没听到过这个名字。 
        “请您谈谈舒克的资讯好吗?”记者小姐要求。 
        “舒克是我的朋友,他从几个月前开始对写作产生了浓厚的兴趣,这本《人类,我是你的朋友》是他写成的第一部小说。”皮皮鲁自豪地向全世界介绍舒克。 
        “请问舒克的性别和年龄以及职业.”有记者问。 
        “舒克是一只男老鼠,年龄有三十多岁。”皮皮鲁回答。 
        “什么?老鼠?!”记者们吃惊。 
        “他是在侮辱全世界的所有作家!”涛涛轰喊叫道。 
        记者们开始交头接耳,他们回忆起皮皮鲁曾经深更半夜带一只老鼠到医院看病的事。 
        “他的神经大概有毛病了。” 
        “可能是在那次地震事件中受刺激了。” 
        “他本来已经快得诺贝尔奖了。” 
        记者们分析皮皮鲁。 
        “我要求将他驱逐出去!”涛涛轰再次向记者招待会主持人请求。 
        两名保安人员出现在皮皮鲁身边。 
        “帮皮皮鲁!”舒克说。 
        贝塔实施舒克的指示。 
        五角飞碟给皮皮鲁输送力气。 
        保安人员拽不动皮皮鲁。 
        “你们看的书确实是一只老鼠写的,你们不能不信。”皮皮鲁提高嗓门。 
        电视台接到数以千计的电话,要求将皮皮鲁轰出会场。,读者们认为皮皮鲁说他们喜爱的书是老鼠写的是对他们的嘲弄。 
        全球作家协会主席坐飞机赶到会场。 
        “我要向法院起诉你,”作协主席声色俱厉地对皮皮鲁说,“你说老鼠的作品获得了诺贝尔文学奖,这是对地球上所有作家的诬蔑,你要负法律责任!” 
        “那本书确实是老鼠写的。你必须承认这个事实。”皮皮鲁已无力招架来自四面八方的围攻。 
        “依我看,这个全球作家协会的作用就是给作家写作捣乱。”贝塔对舒克说。 
        舒克叹了口气。 
        几个穿白大褂的医生出现在会场上,他们拿着各种仪器走到皮皮鲁的身边。 
        “你们要干什么?”皮皮鲁问。 
        “我们是精神病医院的医生,应电视台的要求来给你做体检。”医生说。 
        皮皮鲁不反对。 
        各种仪器相继对皮皮鲁发功。 
        检查结果:皮皮鲁精神正常。 
        “我觉得,”那位记者小姐说,“既然皮皮鲁精神正常,那他就没有理由来这儿捣乱。我建议由作家协会和诺贝尔评奖委员会和读者代表共同鉴定这篇小说的作者是谁。” 
        记者小姐的提议被采纳了。 
        一周后,由作家协会、诺贝尔文学奖评委会和读者代表三方组成的《人类,我是你的朋友》作者鉴定委员会正式投票表决。 
        投票结果:涛涛轰为该书作者。 
        “早知道如此,你写它干什么?”贝塔对舒克说。 
        “真没想到,人类中有这样的作家!”舒克昔日对作家的崇拜心情已烟消云散。 
        “涛涛轰不能代表作家。”皮皮鲁为作家辩护。 
        “像涛涛轰这种冒牌作家在人类的作家中有多少?”贝塔问皮皮鲁。 
        “绝对超不过百分之九十。”皮皮鲁有把握地说。 
        “人类也真可以了,发明了新药,先让老鼠吃。生产了粮食,不让老鼠吃。一边恨老鼠,一边还要属老鼠。连电脑这么伟大的科学成果都离不开鼠标。老鼠写了小说,人类明明爱看,却不承认是老鼠写的。”贝塔越说越有气。 
        “不是全不承认,皮皮鲁不是收到几千封孩子的来信嘛!”舒克比较公正。 
        在鉴定《人类,我是你的朋友》期间,皮皮鲁收到几千名孩子写给他的信,孩子们支持皮皮鲁,他们认为那本书是舒克写的。可惜大人们不理会孩子的意见。 
        舒克的著作《人类,我是你的朋友》堂而皇之地署上了涛涛轰的名字出版。 
        皮皮鲁、舒克和贝塔无可奈何。他们懒得去全球的每一家印刷厂改版。 
        “还写吗?”贝塔问舒克。 
        “当然写。我写书又不是为了出名。”舒克的脸上全是特超脱的表情。 
        可后来皮皮鲁和贝塔一直没再见舒克写作。 
        一天下午,皮皮鲁从超级市场采购食品回来,看见舒克正在擦五角飞碟。 
        “贝塔呢?”皮皮鲁一边往冰箱里塞食品一边问舒克。 
        “在书柜里睡觉。”舒克指指书柜的某一层,“他说和书一块儿睡觉说不定醒了就变成作家了。” 
        “没有这个才能,就是天天在书海里洗澡也没用。”皮皮鲁说。 
        “这个本里是什么?”贝塔醒了,指着身边一个塑料本问皮皮鲁。 
        皮皮鲁走到书柜旁边,伸手从书柜上取出贝塔问的那个本子。 
        “是我的影集,里边都是我小时候的照片。”皮皮鲁擦掉影集封面上的尘土。 
        “我看看。”舒克感兴趣。 
        皮皮鲁将影集放在桌子上,翻开第一页。 
        30年前的皮皮鲁。舒克和贝塔看到皮皮鲁小时候的照片,他们感慨万千。那时的皮皮鲁还是一个孩子,脸上透着顽皮和稚气。 
        皮皮鲁长叹了一口气。 
        “你怎么了?”贝塔抬头看皮皮鲁。 
        “我的童年也够惨的,没怎么好好玩儿,天天就是写作业。”皮皮鲁为自己的童年惋惜。 
        “你还记得吗,为了让你早下学,我们还去钟楼上拨表呢!”贝塔兴奋地回忆. 
        “当然记得,你们还帮我参加航模表演和空战。”皮皮鲁用手做了个遥控航模飞机的动作。 
        “真快,一转眼30年就过去了。”贝塔说。 
        “依我看,人这一辈子,也就是童年最珍贵。千万别留下什么遗憾的事。”舒克说。 
        “童年最重要的事就是玩。”贝塔说。 
        “我不能让今天的孩子再重演我童年的悲剧。”皮皮鲁突然站起来。 
        “悲剧?”贝塔没听说过皮皮鲁小时候有悲剧。 
        “没时间玩,没地方玩,没朋友玩,这对于孩子来说就是悲剧。”皮皮鲁忿忿地说。 
        “你想怎么办?”舒克看出皮皮鲁有想法。 
        皮皮鲁曾说过,最大的享受,就是每天都有新想法。 
        “任何人都离不开衣食住行。孩子也不例外。咱们在孩子的衣食住行上动动脑子,开发具有娱乐功能的产品,比如服装,比如鞋帽,让孩子全身上下本身就是一座小型游乐园。”皮皮鲁说。 
        “好主意,你的愿望如果能实现,全世界的孩子都会对你呼万岁。”贝塔对皮皮鲁说。 
        “咱们成立一家开发公司,开发娱乐性的少儿用品,把孩子的童年变成天堂,让他们穿衣是娱乐,吃东西是娱乐,走路是娱乐,每一件日常用品同时又是玩具。”皮皮鲁越说越兴奋。   第146集 
        决定公司名称; 
        银行业务员刁难皮皮鲁; 
        舒克和贝塔出任总裁助理; 
        确定商标   
        自从皮皮鲁、舒克和贝塔决定成立一家以开发娱乐性的少儿用品为宗旨把孩子的全身上下变成一座小型游乐场的公司后,公司的名称整整困扰了他们三天。 
        “咱们起的名字有几百个了吧?”贝塔看着茶几上一堆涂满了字的纸说。 
        “给公司起名字真难。”舒克挠头。 
        “干脆,就叫舒克贝塔公司吧。”皮皮鲁出主意。 
        “同意。”贝塔说。 
        舒克没意见,他想起了舒克贝塔航空公司。 
        后来闻名全球的少儿用品跨国公司的名称正式诞生——舒克贝塔公司。 
        “办公司需要注册资金吧?”舒克问,他知道皮皮鲁没什么钱。 
        皮皮鲁打电话咨询,对方答复申请开办公司的注册资金最少为10万。 
        “10万元!”贝塔觉得皮皮鲁连一万元也拿不出。 
        “去银行贷款。”舒克出主意。 
        “我现在就去。”皮皮鲁一分钟也呆不住了,他恨不得明天就成立舒克贝塔公司。 
        皮皮鲁来到一家银行,他填写了申请贷款表。 
        银行业务员看了一眼表格上的皮皮鲁的名字,和同事交头接耳起来。 
        “我们不能贷款给您。’业务员把申请表退还给皮皮鲁。 
        “为什么?”皮皮鲁问。 
        “我们对您的信用有怀疑。”业务员说。 
        “为什么?”皮皮鲁感觉受了侮辱。 
        “您和老鼠是朋友?”业务员显然还记得记者丈夫的那次报道。 
        皮皮鲁不置可否地点点头。 
        “我们怎么会把钱借给和老鼠交朋友的人呢?”业务员冲皮皮鲁一笑。 
        皮皮鲁想往他脸上吐唾沫,但他忍住了。 
        “我会按期偿还贷款的。”皮皮鲁一字一句地说。 
        “谁为您担保?”业务员问。 
        “我的名誉为我担保。”皮皮鲁压着火回答。 
        “您的名誉?”业务员们都笑了,“您还以为您的名誉很好哪?您深更半夜到医院给老鼠看病,预报出地震却不说是用什么方法测出的……” 
        “你?!”皮皮鲁快控制不住自己了,“我要见你们银行的总裁。” 
        “楼上205房间。”业务员指着楼梯说。 
        皮皮鲁上楼敲205房间的玻璃门。 
        “请进。”女秘书娇滴滴地说。 
        “我要见总裁。”皮皮鲁说。 
        “您是?”女秘书调查来访者的身份。 
        “我叫皮皮鲁。”皮皮鲁说。 
        “请您稍等。”一楼的业务员显然已和总裁的秘书通过电话了。 
        一位满头银发西服革履的银行家出现在皮皮鲁面前。 
        “您找我有事?”银行家作手势请皮皮鲁坐。 
        皮皮鲁说明来意。 
        “舒克贝塔公司?”银行家表示感兴趣。 
        “本公司的宗旨是将每一个孩子的服装鞋帽全身上下变成一座小型游乐场,让孩子的童年生活得快活。”皮皮鲁向银行家介绍他的公司的目的。 
        银行家点点头,他小时候也是个没玩够的孩子,记忆中掺杂着无限的遗憾。 
        “我给你贷款。”银行家说。 
        绝对的伟人气质。 
        “我没有担保人。”皮皮鲁提醒银行家。 
        “我给你担保。”银行家说,“我也在你的公司入股行吗?” 
        皮皮鲁站起来,紧紧握住银行家的手。 
        “找陪你去楼下办手续。”银行家说。 
        那业务员见总裁亲自陪皮皮鲁来办贷款手续,脸都白了。 
        “给皮皮鲁先生办理30万元的贷款。”总裁又给皮皮鲁加了20万元。 
        业务员用最高的效率绐皮皮鲁办手续。他生怕总裁解雇他。 
        一个星期后,舒克贝塔公司正式宣告成立,皮皮鲁出任公司总裁。 
        舒克和贝塔任总裁特别助理——当然是秘密的,只有总裁一人知道。 
        公司还招聘了几十名开发、经营和管理人才。 
        “你们说,咱们第一件事该干什么?”皮皮鲁坐在宽敞的总裁办公室里问他的特别助理舒克和贝塔。 
        “先开发一种特别能打响的产品。”贝塔说。 
        “先设计一个商标。今后,咱们公司的产品都使用这个商标。”舒克不愧当过大作家,有知识产权意识。 
        皮皮鲁点点头,他认为舒克的建议很好。 
        “商标就是什么什么牌的意思吧?”贝塔问。 
        “差不多。”皮皮鲁说。 
        “叫苗苗牌怎么样?少儿用品嘛。”舒克提议。 
        “太酸了!千万别叫什么苗苗啦,什么花啦草啦,还有禁止大公(又鸟)猫眯小鸟之类的,都太俗太浅薄。”贝塔枪毙了舒克的构思。 
        “那你说叫什么牌?”舒克将了贝塔一军。 
        “我看就叫皮皮鲁牌。”贝塔脱口而出。 
        “皮皮鲁牌?”皮皮鲁重复了一句。 
        “你知名度高。用你的名字当少儿产品的商标,保准受孩子们欢迎。”贝塔在总裁的大办公桌上来回踱步。 
        “我觉得可以。”舒克投赞成票。 
        “就这么定了。”皮皮鲁同意了。 
        舒克贝塔公司的产品都将以“皮皮鲁牌”作为商标。 
        “我觉得还应该有个舒克贝塔牌。”皮皮鲁认为只是“皮皮鲁”一个商标太单调。 
        “我看不错。”贝塔投赞成票。 
        “再增加一个商标,就叫舒克贝塔牌。”皮皮鲁拍板。 
        后来,舒克贝塔公司一共拥有三个驰名全球的商标:皮皮鲁牌、鲁西西牌和舒克贝塔牌,这是后话。 
        这天,皮皮鲁坐在舒克贝塔公司宽大的总裁办公室里,他的目光透过窗户凝视着天上的一块云彩。 
        那云彩的形状触发了皮皮鲁的灵感。 
        皮皮鲁想起自己小时候经常将云朵等无生命的东西想像成有生命的动物。 
        皮皮鲁一边遐想一边构思公司的近期工作计划和远景规划。   第147集 
        皮皮鲁诱导开发部经理; 
        世界首创活玩具问世; 
        橡皮老鼠活了   
        皮皮鲁按电铃叫秘书。 
        秘书走进总裁办公室,问:“请问总裁有什么吩咐?” 
        “请开发部经理来一趟。”皮皮鲁说。 
        秘书转身去叫开发部经理。 
        皮皮鲁认为整个公司最重要的部门就是开发部。 
        “你计划先开发什么产品?”皮皮鲁问开发部经理。 
        “我准备先开发文具。现在的父母都望子成龙,他们肯在学习用具上为孩子投资。”开发部经理从文件夹中抽出计划递给皮皮鲁。 
        皮皮鲁边看边皱眉头。 
        “我看还是先开发玩具。玩具对孩子的智力有刺激作用。”皮皮鲁说。 
        “一般的家长还意识不到这一点,他们宁肯在孩子的学习上投资……”开发部经理坚持自己的观点。 
        “咱们这个少儿用品公司是为孩子服务的,要通过产品扭转家长的观念。玩具对启发孩子的智力有不可估量的作用。”皮皮鲁说。 
        开发部经理点头。 
        “你去拟一个开发玩具产品的计划。”皮皮鲁吩咐。 
        开发部经理走后,舒克和贝塔从总裁助理办公室——一个大抽屉里钻出来。 
        “我觉得你不能生产一般的玩具,必须开发一种新的玩具,过去从来没有过的。”舒克说。 
        “过去的玩具都是死的,咱们生产一种活玩具。”贝塔说。 
        “活玩具!”皮皮鲁一听到这个词就兴奋了。 
        “比如说,玩具包装盒里有一块橡皮泥,孩子可以随便把它捏成什么。盒里有一包液体,只要将这液体洒在用橡皮泥捏成的动物上边,橡皮泥就活了。”贝塔异想天开地说着。 
        “这只能写童话吧?”皮皮鲁耸耸肩膀。连他都觉得贝塔的这个设想太离奇了。 
        “别忘了你是科学家!你连五角飞碟都造出来了,没什么事能难住你的大脑,你准能研制出活玩具!”舒克给皮皮鲁打气。 
        “皮皮鲁,你一定能。关键是那液体,咱们就管它叫生命水吧。只要研究出生命水,就大功告成了。”贝塔兴奋地说。 
        活玩具对皮皮鲁的诱惑力太大了。皮皮鲁清楚,这是玩具史上的革命。 
        “好,我试试!”皮皮鲁决定了。 
        舒克和贝塔同时向皮皮鲁伸出手:“祝你成功!” 
        皮皮鲁郑重地同舒克和贝塔握手。他喜欢迎接挑战。他知道这次是向人类掌握的所有学科挑战。 
        皮皮鲁吩咐秘书不要让任何人打扰他。他将自己关在居所里,只有舒克和贝塔接触他。 
        皮皮鲁已经快一个星期没出房门了,他的屋子里摆满了大大小小的器皿。他不停地试验不停地记录,他的眼睛充满血丝。 
        舒克和贝塔这几天几乎没说话,他俩屏住呼吸注视着皮皮鲁房间的门。他们想起了皮皮鲁研制五角飞碟时的情景。 
        “能进行创造性思维的大脑实在是无价之宝。”舒克打破了沉寂。 
        “长脑子就应该想别人没想过的事,长嘴巴就应该说别人没说过的话,长腿就应该走别人没走过的路。这才叫活。”贝塔说。 
        “也是,如果长着一个大脑,老是顺着别人的思路想事,长着一张嘴,老是说别人说过的话,这样的生命就跟设活一样。”舒克若有所思。 
        “我看,绝人部分人的大脑不能进行创造性思维活动,挺可惜的。”贝塔摇摇头。 
        “如果有这样的学校就好了,专门培养学生的大脑想别人没想过的事。”舒克为人类惋惜。 
        “可惜没有。”贝塔说完走到皮皮鲁房间的门口,趴在门上听。 
        贝塔冲舒克招手。 
        “怎么啦?”舒克有好的预感。 
        贝塔说:“皮皮鲁哼歌呢!” 
        “要成功了!”舒克激动。 
        门开了,皮皮鲁左手拿着一瓶水,右手拿着一块橡皮泥。 
        舒克和贝塔的心怦怦狂跳起来,他俩死盯着皮皮鲁的嘴,等待那划时代的宣告。 
        皮皮鲁没说话,只是轻微地点了点头。 
        “成功啦!”贝塔去找酒。 
        “表演一下。”舒克想当第一个目睹生命水奇迹的生命。 
        皮皮鲁把水瓶放在茶几上,他的两只手飞快地将橡皮泥捏成一只和舒克一样大小的老鼠。 
        皮皮鲁将橡皮泥老鼠放在茶几上,他拿起水瓶往橡皮泥老鼠身上倒水。 
        5分钟后,橡皮泥老鼠活了,他惊讶地看着眼前这个陌生的世界。 
        “贝塔,快来看!”舒克喊。 
        拿着酒杯一边走一边喝的贝塔看见橡皮泥老鼠时愣住了。 
        “奇迹!”贝塔发自肺腑地说。 
        “它能活多久?”舒克问皮皮鲁。 
        “6天。6天后又恢复成橡皮泥了。再浇生命水还能再活6天。”皮皮鲁说。 
        “我给这玩具起个名字,就叫‘创造生命’怎么样?”贝塔喝干了一杯酒。 
        皮皮鲁和舒克同意。 
        舒克和贝塔点头。 
        两个星期后,舒克贝塔公司生产的名为“创造生命”的活玩具隆重推出。上市的第一天,50万盒被孩子们一抢而空。 
        地球上所有的玩具公司都被“创造生命”活玩具击昏了,他们拼命研究生命水的配方,却一无所获。 
        舒克贝塔公司又连续推出了一系列皮皮鲁牌的少儿用品,包括玩具、文具、服装、食品和一切与孩子们有关的产品。这些产品都有一个共同的特点,就是娱乐性。服装能当玩具,鞋子能当玩具,食品能当玩具,文具也能当玩具。 
        皮皮鲁成了商界的新闻人物,他每天的日程排得满满的。开发新产品,洽谈生意,出席新闻发布会…… 
        “我觉得你应该激流勇退了,新物色一名总裁。你每天连和我们说话的时间都没有了。”一天晚上,贝塔对皮皮鲁说。 
        “我已经物色好了总裁人选,明天就交班。”皮皮鲁喜欢开创事业,干成功了,就交给别人。 
        “这还差不多。”贝塔冲舒克一笑。 
        “最高层次的生命是在宁静中进行创造性劳动的生命。嘈杂的应酬性的生命是浪费生命。”舒克时不时还冒出几句作家的语言。 
        电话铃响了。 
        皮皮鲁拿起话筒:“喂?” 
        “你是皮皮鲁吗?” 
        “我是。您是哪位?” 
        “你听好,我们需要你立即研制出一种能使死人复活的生命水。” 
        “你是谁?” 
        “这个你别管。你必须在3天之内研制出来。” 
        “我不干。” 
        “那你就别想活了。你记住,3天后的这个时间,你等我们的电话,告诉你送生命水的地点。如果你报警,你就必死无疑了。” 
        对方把电话挂了。 
        皮皮鲁慢慢挂上电话。 
        “出什么事了?”贝塔看出皮皮鲁的表情发生了变化。 
        “有坏蛋威胁咱们。”皮皮鲁走到窗前,看外边。 
        “威胁咱们?”贝塔觉得好笑。 
        皮皮鲁把电话内容告诉舒克和贝塔,舒克和贝塔马上来了精神。   第148集 
        一个叫糕鱼氏的流氓给皮皮鲁打恐吓电话; 
        冰箱里的冻(又鸟)起死回生,成为有觉悟的裸(又鸟)       
        给皮皮鲁打恐吓电话的是一个名叫“黑旋风”的犯罪团伙,这个团伙的首领一个月前被警察捕获。经法院判决,定于3天后枪毙该首领。 
        黑旋风的其他成员绞尽脑汁想拯救他们的头领.他们想到了皮皮鲁。 
        “皮皮鲁既然能发明出把橡皮泥弄活的生命水,准能发明让死人复活的生命水。”一个外号叫糕鱼氏的歹徒最先想到皮皮鲁。他手里拿着一张介绍皮皮鲁发明活玩具的报纸。 
        “什么意思?”另一个外号叫忘拼命的同伙不明白糕鱼氏在这个时候提生命水干什么,他主张使用武力劫法场救头领。 
        “老大被枪毙后,咱们把尸首领回来,泼上生命水,他不就又活了吗?”糕鱼氏说。 
        众歹徒大喜。 
        于是,糕鱼氏给皮皮鲁打了那个电话。 
        “皮皮鲁干吗?”忘拼命问糕鱼氏。 
        “没明确答应,他总不会不怕死吧?”糕鱼氏狞笑道。他根本没把皮皮鲁这个科学家放在眼里。他脑子里的科学家都是手无缚(又鸟)之力的瘦弱生命。 
        “他如果到期研制不出生命水呢?”忘拼命问。 
        “那就绑架他!”糕鱼氏点烟。 
        “绑架他也救不了头儿的命呀!”忘拼命提醒同伙。 
        “咱们把头儿的尸体冻起来,再强迫皮皮鲁研制生命水。什么时候成功了,什么时候拿生命水往头儿的尸体上泼。”糕鱼氏早已想好了。 
        众歹徒点头。 
        自从皮皮鲁接到糕鱼氏的电话后,贝塔格外兴奋,连吃饭的时候都吹口哨。 
        “含蓄点儿,别太外露。”舒克对贝塔说。 
        贝塔用口哨回答舒克。 
        “你们说,这人要生命水干什么?”皮皮鲁问舒克和贝塔。 
        “把死人弄活呗。”贝塔脱口而出。 
        “救人?”皮皮鲁摇头。他觉得想救人的人不会使用威胁他人生命的方法达到救人的目的。 
        “这人不会等到3天后再找你,我看他出不了今天还会给你打电话。”舒克分析,“等他再打电话时,你和他多聊几句,拖延时间,我们用五角飞碟的仪器遥感他,弄清这小子到底想干什么。” 
        皮皮鲁点头。 
        贝塔拿麂皮擦拭五角飞碟,他恨不得天天驾驶五角飞碟外出。 
        电话铃响了。 
        舒克和贝塔对视了1秒钟,然后又同时扭头看皮皮鲁。皮皮鲁朝五角飞碟努努嘴。 
        舒克和贝塔飞快地钻进五角飞碟。 
        皮皮鲁拿起听筒。 
        “是皮皮鲁吗?”糕鱼氏的声音,“生命水研制了吗?” 
        “已经开始研制了,但困难比较大。”皮皮鲁拖延时间,“能不能放宽几天?” 
        “不行!必须在3天内研制出来!” 
        “我想知道一下,你准备让死了多长时间的人活过来?” 
        “刚死的。” 
        “现在已经死了?” 
        “还没有,3天后死!” 
        “病死?” 
        “你管那么多干什么?” 
        “我不了解死因,没法对症下药。” 
        “枪毙!” 
        “3天后枪击而死?你既然知道3天后他准受到枪击,干吗不制止?” 
        “废话,我制止得了吗?你别罗嗦了,3天内你必须研制出来,否则这生命水就只能留给你用了。” 
        电话挂断了。 
        皮皮鲁耸耸肩,走到五角飞碟旁边坐下,等待舒克和贝塔报结果。 
        贝塔极度兴奋地从五角飞碟里跑出来:“太有戏了,是一个犯罪团伙。他们的头儿被法院判了死刑,3天后枪毙,他们想用你研制的生命水使那头儿起死回生,继续领导他们为非作歹。” 
        “给你打电话的这小子外号叫糕鱼氏,现在是代理头目。”舒克补充,“品行极端恶劣,心狠手黑。” 
        “我看不用理他们。如果3天后他们送上门来,就教训他们一下。”皮皮鲁懒得和糕鱼氏这种歹徒打交道。 
        “那可不行!”贝塔急了,他不能眼看着到口的肥肉跑了,  “你怎么能放纵这些坏蛋继续作恶呢?应该和他们斗争呀!” 
        皮皮鲁从书柜里抽出一本书,一边翻一边问贝塔:“怎么斗?” 
        “你应该研制出能使死人复活的生命水,但这种生命水同时又具备另一种功能。”贝塔做神秘状。 
        “什么功能?”舒克问。 
        “能把坏人变成好人。”贝塔异想天开。 
        “童话。”舒克说。 
        “坏人坏在哪儿?不就是坏在脑子里吗?好人好在哪儿?不也是好在大脑吗?这有什么难的?把脑子的思维程序给改了不就行了吗。”贝塔滔滔不绝,“你想想,如果那帮坏小子把他们的头儿救活后,发现头儿的思想境界特别高,说不定他还会带着部下去警察局集体投案自首呢!” 
        “亏你想得出。”皮皮鲁对贝塔的建议感兴趣了。 
        “你发明这种东西还不跟玩似的!”贝塔掌握好时机给皮皮鲁戴高帽。 
        “就照你说的办。”皮皮鲁同意了。 
        贝塔冲皮皮鲁飞了个吻: 
        “不愧是人类最伟大的科学家。” 
        皮皮鲁从抽屉里取出一张纸,又从笔架上抽出一支笔,在纸上急促地写着什么。 
        “这纸上写的都是我做实验需要的化学药品,你们驾驶五角飞碟尽快把这些东西弄来。”皮皮鲁将纸交给贝塔 
        贝塔从接过纸到进八五角飞碟顶多用了3秒钟 
        当舒克跨进五角飞碟时,飞碟已经启动了。 
        “稳着点儿,别太得意忘形。”舒克对贝塔说。 
        第3天的上午,皮皮鲁从他的房问里走出来向舒克和贝塔宣布大功告成。 
        没经过试验怎么知道成功了?”舒克问皮皮鲁。 
        “数据证明成功了。当然,能试验一下更好,可拿什么试验呢?”皮皮鲁觉得找一具死尸不那么容易。 
        “不一定拿人试验,去菜市场买一只死(又鸟)试试也行。”贝塔出主意。 
        “这主意不错,冰箱里有冻(又鸟)。”皮皮鲁一拉冰箱门,拿出一只冻得梆梆硬的(又鸟)。 
        皮皮鲁手持装满生命水的喷罐朝冻(又鸟)身上喷射生命水。 
        舒克观察(又鸟)的眼睛。贝塔摸(又鸟)的皮肤。 
        皮皮鲁不停地看表。 
        7分钟后,那只(禁止)的冻(又鸟)恢复了生命,它站立起来,目光炯炯地看着皮皮鲁和舒克、贝塔。 
        “皮皮鲁,你太伟大了!”贝塔双手握拳振臂高呼。 
        “我觉得它和别的(又鸟)不大一样。”舒克绕着裸(又鸟)转了一圈儿。 
        “没错,它的眼神显得特别有觉悟,特革命,思想境界特高。”贝塔有同感。 
        皮皮鲁仔细观察那(又鸟),他显然兴奋了:  “咱们真的成功了!如果这生命水泼在(又鸟)身上都能把(又鸟)的思想觉悟提高了,那泼在人身上准没问题了!” 
        那(又鸟)在地上走了几步,每一步都透着有理想和大义凛然。   第149集 
        皮皮鲁连跑两个电话亭; 
        黄色小轿车; 
        流氓头领脱胎换骨,变成十大杰出青年   
        下午,皮皮鲁、舒克和贝塔坐在客厅里一边看电视一边等糕鱼氏的电话。现在是糕鱼氏给皮皮鲁3天期限的最后时刻。 
        “八成那头领又被改判死缓了吧?”贝塔比谁都着急,  “要不他们怎么没动静了呢?” 
        “越是没动静,就越是快了。”舒克说。 
        皮皮鲁的目光基本停留在电话机上,根本没正眼瞧过电视。 
        “铃——” 
        皮皮鲁、舒克和贝塔同时跃起。 
        皮皮鲁抓起听筒。 
        “皮皮鲁你听好,10分钟后,你到你家楼下路边的公用电话亭等我的电话,带上生命水。记住,如果你报告警方了,那今天就是你的末日。听清了吗?”糕鱼氏凶恶地说。 
        “听清了。”皮皮鲁极乖。 
        电话挂了。 
        “我按他说的办,你们进入五角飞碟待命。”皮皮鲁边说边将装有生命水的喷罐塞进皮包。 
        “我们监视你的一切,你放心去吧。”贝塔就差喊万岁了。 
        “别太冲动,听我的指令。”皮皮鲁打开窗户——开放五角飞碟的起飞通道。 
        舒克和贝塔进入五角飞碟。贝塔打开遥感器,开始观测皮皮鲁的一举一动。 
        皮皮鲁拎着皮包下了楼,他看看手表,正好是lO分钟。皮皮鲁走进公用电话亭。 
        电话铃准时响了。 
        皮皮鲁摘下话筒。 
        “5分钟后,你到×××大街左边的第2个公用电话亭等我的电话。”糕鱼氏说。 
        “你这是干什么?”皮皮鲁火了。 
        电话挂了。 
        皮皮鲁无奈,只得一溜小跑赶往糕鱼氏指定的电话亭。 
        这次糕鱼氏表扬皮皮鲁了: 
        “很好,你没有报告警察。现在,你离开电话亭,站在路边儿,当一辆黄色的小轿车停在你身边时,你将皮包扔进车里就行了。” 
        皮皮鲁照糕鱼氏吩咐的做,他离开电话亭,站在路边,注视着面前的车水马龙。 
        一辆黄颜色的轿车停在皮皮鲁身边,皮皮鲁通过摇下一半的玻璃窗看到后座上没人。 
        “看什么?还不快把包扔进来!”司机冲皮皮鲁吼。 
        皮皮鲁将皮包扔进车里。黄色轿车飞快地开走了。 
        皮皮鲁有几分怅然若失,他原以为会有不少惊心动魄的场面,没想到这么简单,就像什么事也没发生过似的。 
        舒克和贝塔在家迎接皮皮鲁。 
        贝塔满脸失望: 
        “我还以为会有几个手持冲锋枪戴墨镜的彪形大汉先绑架你再拿走生命水呢。” 
        “那都是电影里的场面。”皮皮鲁耸耸肩膀。 
        黑旋风的头领被押赴刑场执行枪决。枪决后,他的尸首被家属领走了。 
        糕鱼氏和忘拼命从头儿的家属手中要走了头儿的尸首。 
        这位首领是个五毒俱全的难得的犯罪人才,糕鱼氏和同伙之所以不遗余力地想让他起死回生,就因为他们清楚,如果离了他,黑旋风马上就会土崩瓦解。 
        糕鱼氏命令同伙将头儿的尸体放在床上,将他的衣服全部脱掉。 
        忘拼命用皮皮鲁给他们的喷罐向头儿的全身喷射生命水。 
        众歹徒眼巴巴地盼着头儿复活。 
        死尸一动不动。 
        “咱们被皮皮鲁涮了吧?”忘拼命问糕鱼氏。 
        “如果过半个小时头儿不活,咱们就去砸了皮皮鲁的家。”糕鱼氏咬牙切齿地说。 
        “头儿的眼皮动了!”一个歹徒大喊。 
        众歹徒俯身看。 
        看不出有动的迹象。 
        “瞎诈唬什么!”忘拼命瞪了那同伙一眼。 
        “又动了!”那人又喊。 
        这回歹徒们看清了,头儿的眼睛已经睁开了。 
        “这是在哪儿?”头儿坐了起来。 
        众歹徒狂呼,有一个歹徒还提议下次抢了银行把全部钱送给皮皮鲁当奖金。 
        “是我和糕鱼氏想尽办法使大哥起死回生的。”忘拼命向头儿邀功请赏。 
        “我是罪有应得,应该枪毙。”头儿说。 
        众歹徒愣了。 
        “大哥真幽默,越是危险时刻越爱开玩笑。”糕鱼氏只能这么理解。 
        “不,这是我的真心话,人活一世,要做有益于社会的事,不能把幸福建筑在别人的痛苦上。”被皮皮鲁的生命水改变了大脑思维程序的头儿真诚地说。 
        众歹徒大眼瞪小跟。 
        头儿开始用十大杰出青年的口气谆谆教导部下,告诉他们人生的意义在于给予,告诉他们生命的价值就是奉献,还引用了好多伟人的名言伟句,还说钱是万恶之源,还说男人和女人在一起只应该有友谊不应该有性爱…… 
        众歹徒像看天外来客似地看头儿,最后,所有人的目光都集中在糕鱼氏和忘拼命身上。 
        “大哥,您的神经是不是有点儿……”糕鱼氏试探头儿。 
        “我的神经非常正常,是你们不正常,你们现在跟我去警察局自首,争取宽大处理。”头儿一脸的严肃和神圣,还有几分使命感。 
        “什么?去自首?”众歹徒喊了起来。 
        “对,去自首!托尔斯泰说过……” 
        头儿还没说完,糕鱼氏冲同伙使了个眼色,众歹徒一拥而上,将头儿按在床上。 
        “放开我!真理必将战胜邪恶!曙光就在前面。你们的末日就要到了!……”头儿的声音越来越小。 
        掐头儿脖子的歹徒抬头请示糕鱼氏,是杀了头儿还是不杀头儿。 
        糕鱼氏点点头。 
        头儿咽气了,他这次是死在自己的部下手里。 
        “一定是皮皮鲁搞的鬼!”忘拼命说。 
        一句话提醒了糕鱼氏,他一拳砸在头儿身上: 
        “去找皮皮鲁算帐!” 
        皮皮鲁危在旦夕。   第150集 
        交通警察吊扣皮皮鲁的驾驶执照; 
        贝塔驾驶五角飞碟撞碎玻璃; 
        糕鱼氏觊觎五角飞碟   
        糕鱼氏吩咐同伙将头儿的尸体冷冻起来。 
        “还留着他干什么?整个一个劳模。”忘拼命厌恶地看了头儿的尸体一眼。他觉得,做了坏事生怕别人知道的人有救,而做了好事生怕别人不知道的人没救。刚才复活的头儿的表现就是一个地道的做了好事生怕别人不知道的形象。 
        “让皮皮鲁发明把头儿按原来的面目复活的生命水。”糕鱼氏咬牙切齿地说。 
        “他能听咱们的?”忘拼命感觉到皮皮鲁的倔强。 
        “绑架他!”糕鱼氏从牙缝中进出三个阴森森的字。 
        一群歹徒围坐在昏暗的灯光下,策划着绑架皮皮鲁的细节。 
        一天早晨,吃完早餐后,皮皮鲁对舒克和贝塔说: 
        “上午我去公司看看,你们准备干什么?” 
        “我陪你去。”舒克总觉得这两天皮皮鲁好像要出什么事,他的直觉挺厉害。 
        “我睡觉。”贝塔打了个哈欠。 
        皮皮鲁带着舒克来到楼下的停车场,他拉开自己的那辆白色轿车的门,坐在驾驶员的位置上。 
        舒克从皮皮鲁的上衣兜里探出头,注视着汽车的前方。 
        皮皮鲁驾驶汽车上路。 
        一辆黑色轿车尾随在皮皮鲁的身后。 
        此刻,糕鱼氏就在路旁的一座摩天大厦上,他手持望远镜和步话机,正在现场指挥绑架皮皮鲁的行动。 
        “很好。开始!”糕鱼氏的嘴角挂着一丝狞笑。 
        皮皮鲁的车前突然驶来一辆交通警察的巡逻车。巡逻车停在皮皮鲁的汽车的前方。 
        皮皮鲁急刹车。 
        两名警察从巡逻车里走出来。 
        “你违章了。”一名警察走到皮皮鲁的车旁,弯下腰对车里的皮皮鲁说。 
        “违章?我怎么违章了?”皮皮鲁清楚自己根本没违章。 
        “说你违章你就是违章了,驾驶执照!”那警察吼道。 
        皮皮鲁只得将驾驶执照递给警察。对于司机来说,交通警察的话就是圣旨。 
        “吊扣你的驾驶执照!”交通警察转身就走。 
        “你?!''皮皮鲁打开车门追警察。 
        尾随在皮皮鲁身后的那辆黑色轿车突然开到皮皮鲁身边,从车上跳下两个彪形大汉,一左一右,强行将皮皮鲁往黑色轿车里拽。 
        舒克明白了,那交通警察和彪形大汉是一伙的,他们设了圈套绑架皮皮鲁。 
        “贝塔!贝塔!皮皮鲁遇到了意外,快驾驶五角飞碟出击!快!”舒克使用皮皮鲁衣兜里的微型通讯器向贝塔呼救。 
        刚刚入睡的贝塔听到了五角飞碟里的警报声。他跑进五角飞碟,打开通讯器开关。 
        “舒克,我是贝塔,你再说一遍!”贝塔一边说一边启动五角飞碟。 
        “我们就在楼下的路上,现在皮皮鲁已被劫持进一辆黑色的轿车里。”舒克说。 
        “太棒了!”贝塔睡意全无。 
        “你说什么?”舒克怀疑自己的耳朵。 
        “我说太棒了!总算有事干了。”贝塔操纵五角飞碟在屋里转了一圈后,撞碎玻璃冲出屋子。 
        这时,皮皮鲁已被歹徒拽进黑色轿车。 
        “快开车!”坐阵楼上的糕鱼氏指挥。 
        “糕头儿,你看那是什么?”一名负责膝望的部下指着天上向糕鱼氏汇报。 
        糕鱼氏看见一个小飞碟从他眼前飞过。 
        “是从皮皮鲁家的窗户里飞出来的。”部下补充说。 
        “从皮皮鲁家飞出来的?!”糕鱼氏一愣。他忙举起望远镜观察飞碟。 
        “绝对的超现代化飞行器!”糕鱼氏的眼睛通过望远镜刚一接触五角飞碟他就大喊一声。 
        五角飞碟追上了黑色轿车。 
        贝塔按下射击按钮。 
        黑色轿车的四个轮胎全部放了气,汽车停下了。 
        皮皮鲁从车里出来了。 
        “弟兄们怎么了?”糕鱼氏看见皮皮鲁大摇大摆从车里出来,很吃惊。 
        “好像都倒了。”部下用望远镜观察。 
        “那飞碟使用了什么武器?无声无光。”糕鱼氏目击了五角飞碟袭击黑色轿车的全过程。 
        “现在怎么办?”部下问糕鱼氏,  “援救车上的弟兄们吗?” 
        糕鱼氏摇摇头,他的两眼望着窗外的天空出神。 
        “糕头儿,你怎么了?”部下有点儿害怕。 
        “我要那个飞碟!’糕鱼氏知道,如果自己拥有了那个飞碟,就谁也奈何不了他了。 
        一个从皮皮鲁家盗窃五角飞碟的计划开始在糕鱼氏的脑海里形成了。   第151集 
        贝塔摔碎了玻璃杯; 
        皮皮鲁发功击倒探长林   
        皮皮鲁和舒克回到家里时,贝塔正在桌子上擦五角飞碟。 
        “怎么样,没吓着吧?”贝塔问。 
        “还行,多亏你出击的速度快。”舒克表扬贝塔。 
        “有了这次的教训,他们大概再不敢惹皮皮鲁了。”贝塔判断说。 
        “是一伙亡命徒,而且好像还有点智力。”皮皮鲁一边喝水一边说。 
        敲门声。 
        皮皮鲁走过去扒在门镜上往外看,是两个陌生男子。 
        “你们躲进五角飞碟,我把飞碟放到沙发下边。如果是那帮坏蛋找上门来,你们就见机行事。到特惊险的时候再转危为安。”皮皮鲁对舒克和贝塔说。 
        舒克和贝塔钻进五角飞碟,皮皮鲁将飞碟藏在沙发下边。 
        皮皮鲁开门。 
        “请问你们找谁?”皮皮鲁间。 
        “听说您刚才遭到了歹徒的劫持,我们是警察局的,这位是探长林,我是他的助手。”身着便衣的警察说。 
        “你们的信息挺灵。”皮皮鲁请两位警察进屋。 
        “有过路人报案。”探长林在沙发上落座,他的身体下边就是五角飞碟。 
        皮皮鲁点点头。 
        “我们要了解一下情况,请问您的姓名?”探长林间皮皮鲁。助手记录。 
        “皮皮鲁。”皮皮鲁回答。 
        探长和警官同时惊讶,他们熟悉皮皮鲁的大名:著名物理学家;地震风波;《人类,我是你的朋友》版权风波;舒克贝塔公司风靡全球的产品…… 
        “您能谈谈今天早晨被劫持的经过吗?”探长林问。 
        皮皮鲁心不在焉地简述了一遍。 
        “您认识那些歹徒吗?”探长林叉问。 
        皮皮鲁摇头。 
        “您是怎么摆脱他们的呢?刚才我们看了现场,车上的几名歹徒都处于昏迷状态,您与他们搏斗过?”探长林连续发问。 
        皮皮鲁不知该怎么回答。 
        “能告诉我们吗?”探长再次请求。 
        “我会气功。”皮皮鲁只好胡诌。 
        “气功?”探长林看看助手,  “您是说,您是发功击昏那些歹徒的?” 
        “是的。”皮皮鲁点头。 
        “真让人难以置信。”探长显然怀疑皮皮鲁的话的真实性。 
        “我现在可以表演给你看。”皮皮鲁对探长说。 
        探长来精神了。过去他听说过不少有关气功的传闻,但他还从未亲眼见过,因此,他对气功一直抱怀疑态度。 
        “注意,皮皮鲁需要咱们帮忙了。”躲在五角飞碟里的舒克对贝塔说。 
        贝塔打开五角飞碟的总开关。 
        “你们看着这只玻璃杯。”皮皮鲁指着茶几上的玻璃杯对探长和警官说,  “我发功能把它从茶几的这边移到茶几的另一头。” 
        探长和警官对视了一下目光,他们极有兴趣地注视着那只杯子。 
        “听清了吗?帮皮皮鲁移茶几上的玻璃杯。”舒克提醒贝塔。 
        “知道了,保准天衣无缝。”贝塔打开操纵台上的遥控装置开关。 
        “请你们看好,我开始发功了。”皮皮鲁对警官说完,举起双臂,做发功状。 
        贝塔操纵五角飞碟的遥控装置移动茶几上的玻璃杯。 
        舒克通过荧光屏观察那只玻璃杯,他突然发现贝塔将那只玻璃杯的方向弄反了。 
        “反了反了,方向反了!”舒克扭头告诉贝塔。 
        已经晚了。 
        玻璃杯摔到了地上,碎了。 
        “怎么搞的?”皮皮鲁挺尴尬。 
        尽管玻璃杯没有照皮皮鲁说的那样从茶几的这一头移到那一头去而是掉到地上摔碎了,可两位警察还是大惊失色,因为确确实实没有人动那只杯子,是皮皮鲁发功将它移动的。 
        “厉害!”探长林对皮皮鲁的气功佩服之至。 
        警察弯腰捡地上的碎玻璃片,他瞥见了沙发下边的五角飞碟。 
        “这是什么?”警察问皮皮鲁。 
        探长林也低头看。 
        “玩具。儿童玩具。”皮皮鲁忙搪塞。 
        “您的气功能将人击昏?”探长林对儿童玩具显然没兴趣,他感兴趣的是皮皮鲁的气功。 
        皮皮鲁只能点头。他有点儿后悔用气功这个理由说明他是如何摆脱歹徒的了。 
        “您对我发功试试。”警察提议。 
        “这可不行,弄不好就致残了,绝对不行。”皮皮鲁生怕贝塔再像刚才那样失手,刚才的试验物是玻璃杯,摔了就摔了,这回是活人,人命关天。 
        “您对我发功试试,您肯定能掌握好分寸。”探长林要求皮皮鲁表演。 
        皮皮鲁仍然不同意。 
        “这也是我们调查这个案子的重要部分,如果您不让我们亲眼看看您的功力,我们怎么能相信是您战胜那些歹徒的呢?”探长林做最后的努力。 
        “如果您这么说,我只好表演给你们看了。”皮皮鲁在心中为这位探长祈祷,祈祷贝塔别再失手。 
        探长林站起来,面对皮皮鲁站好。 
        “这次我来操纵。”舒克对贝塔说。 
        贝塔离开操纵台,在舱内散步。 
        当皮皮鲁抬起手臂指向探长林时,舒克按下了射击按钮。 
        探长林感到有一股强大的力量将他推倒在沙发上,而皮皮鲁站在离他一米远的地方没动窝。 
        “我服了。”探长林心悦诚服地对皮皮鲁说,“很想同您交个朋友。” 
        皮皮鲁伸出手同探长握手,表示同意。 
        “关于我的气功,希望您不要对外界说。”送探长出门时,皮皮鲁叮嘱。 
        探长点头应允。 
        皮皮鲁关上门,将五角飞碟从沙发下拿出来放到桌子上。 
        贝塔打开舱门探出头: 
        “气功大师,感觉怎么样?” 
        “你怎么能让我当众出丑?”皮皮鲁说。 
        “他怕你功力太好了,被警察局拉去当教官。”舒克一边说一边从五角飞碟里出来,在桌子上做了两个俯卧撑。   第152集 
        五角飞碟遥感撞车逃逸肇事者; 
        小个子墨镜瞠目结舌; 
        糕鱼氏潜入皮皮鲁家   
        早晨,舒克起床看见皮皮鲁对着镜子系领带。 
        “你要出门?”舒克问皮皮鲁。 
        “上午去公司参加董事会议。”皮皮鲁系好领带后,又对照镜子修正领带上不尽人意的地方,然后满意地点点头。 
        “我跟你去。”贝塔从五角飞碟里探出头,“让舒克值班。” 
        自从和糕鱼氏那帮歹徒打上交道后,每逢皮皮鲁外出,总要留一位在五角飞碟里进行战斗值班。皮皮鲁、舒克和贝塔明白,没有五角飞碟,他们不是糕鱼氏这帮土匪的对手。 
        舒克只得同意,上一次是贝塔留守。 
        西服革履的皮皮鲁和贝塔锁上家门走了。 
        舒克坐在一本书上,靠着五角飞碟遐想。舒克喜欢没事时胡思乱想。他觉得胡思乱想是种享受,不会胡思乱想的生命不是高级生命。 
        “舒克,舒克,我是贝塔!听见请回答!”五角飞碟里传出贝塔的呼叫声。 
        舒克猛然一惊,从遐想中清醒过来,跑进五角飞碟。 
        “我是舒克,清讲!”舒克站在操纵台前同贝塔联络,他认定皮皮鲁又被劫持了。 
        “皮皮鲁的汽车停在楼下,不知被哪个小子的汽车给撞坏了,我们下楼正准备开车走,刚发现的。”贝塔说。 
        “撞得怎么样?”舒克松了一口气。 
        “左侧的两个车门都被撞瘪了。这小子也太缺德了,人家停在路边好好的,你撞了人家,也不留下来等着。”贝塔骂骂咧咧。 
        “要我做什么?”舒克问。 
        “用五角飞碟的遥感系统查查是谁撞的。皮皮鲁说不用查了,他还说什么吃亏是福,自己去修车就行了。我说不行,得让那小子知道不能这么做人。”贝塔忿忿然。 
        “你等两分钟,我查。”舒克按操纵台上的有关按钮。 
        荧光屏上显示出撞车时的情景:一辆黑色轿车在与对面驶来的大卡车会车时撞了停在路边的皮皮鲁的汽车,那司机停车后看看皮皮鲁的车里没人,又看看四周也没人注意他,一踩油门,溜了。 
        舒克将那辆车的车牌号及车型告诉贝塔。 
        “那辆车现在在什么地方?”贝塔问。 
        舒克查到了废车的方位。 
        “在莎雁商场门口的停车场上,车主大概是去商场里边了。”舒克说。 
        “我和皮皮鲁现在去找他。”贝塔说。 
        “他要是不承认呢?”舒克问。 
        “他的车身上准有撞伤,赖不掉的。”贝塔说,“随时联系,弄不好还需要你帮忙呢!” 
        舒克摇摇头,在皮椅上落座。其实舒克心里清楚,如果你恨一个人,治他的最好的方法就是放纵他的缺点。拿这个撞了人家的车后逃跑的司机来说吧,如果你让他这次得了逞,下次他撞了人也敢跑,那就真离进监狱不远了。如果你不饶他,非要让他知道不能这么做人,下次他准不敢再这样了。 
        皮皮鲁也不愿意去找那辆车,架不住贝塔坚决不干,皮皮鲁只得驾车前往莎雁商场。 
        那辆车确实停在商场门口的停车场上,车身右侧的撞伤与皮皮鲁的汽车左侧的撞伤严丝合缝,一看就知道是事故的孪生双方。 
        一个戴墨镜的小个子男士从商场里出来径直走到黑车旁边,他掏出钥匙开车门。 
        “快去!”贝塔命令皮皮鲁。 
        皮皮鲁走到小个子墨镜身边: 
        “请问,这车是您的?” 
        “是,怎么啦?”小个子墨镜回头看皮皮鲁。 
        “您还认识那辆车吗?”皮皮鲁指指停在黑车旁边的自己的车。 
        小个子墨镜侧头一看,脸色变了。 
        “撞了别人的车,可不应该跑呀!”皮皮鲁说。 
        “你……你怎么知道……是我撞的?”小个子墨镜不知所措。 
        “上帝告诉我的。”皮皮鲁笑眯眯地对小个子墨镜说,  “冥冥之中有一双公正的眼睛时刻在注视着我们这个世界,做了坏事是逃不掉的。本来撞车不算坏事,可你这么一溜,就算坏事了,如果你一直在车旁等着我,或者把你的地址姓名留在我的车上,我会因此而觉得世界美好。可是你这么一跑,我就觉得世界挺黑暗。上帝是不会让他的孩子对这个世界失去信心的。” 
        小个子摘下墨镜,眼睛里全是愧疚和自责的眼神。 
        “我错了,我赔偿您的车的损失。”他掏钱包。 
        “算了,你以后不这样做人就行了。”皮皮鲁同他握了握手,转身钻进了自己的汽车。 
        小个子果呆地看着皮皮鲁发动汽车,他的确和刚从市场出来时判若两人。 
        皮皮鲁和贝塔驱车去公司开董事会,他们准备开完会再去修车。 
        贝塔在车上和舒克通话。 
        “找到那小子了,他还有救。”贝塔告诉舒克。 
        “这家伙运气太差,其实撞了皮皮鲁的车如果留下来等着车主,或者在雨刷器上夹一张纸条,皮皮鲁会和他交一辈子的朋友,还一分钱也不会让他赔。”舒克为小个子惋惜。 
        “我们已经到公司了,你可以放松放松了。”贝塔对舒克说。 
        舒克走出五角飞碟,他想到书柜里找本书看。 
        舒克刚爬上书柜的第三层,他突然听到大门开启的声音。 
        皮皮鲁和贝塔刚到公司,这么快绝对回不来!舒克预感到不妙,他用最快速度从书柜里跑出来,想钻进停在桌子上的五角飞碟里,可是已经来不及了。 
        两个贼头贼脑的男人走进屋里,他们一眼就看见桌子上的五角飞碟。 
        “在这儿!”两人异口同声。 
        舒克一惊,这两个不速之客是冲着五角飞碟来的! 
        “糕头儿,你看着飞碟,我看看还有没有什么别的宝贝。”一个男人对另一个男人说, 
        “什么都不要了,就要这个。”被称做糕头儿的人抱着五角飞碟神采飞扬。 
        舒克不顾一切地冲出去。 
        “老鼠!”两个男人见到从书柜底下冲出一只老鼠,惊叫道。 
        “踩死它!”糕头儿恶狠狠地说。 
        “甭理它,我才不给皮皮鲁家除四害呢!”另一个说。 
        两个男人抱着五角飞碟走了。 
        舒克一筹莫展——离开五角飞碟,他不是那两个人的对手。 
        当皮皮鲁和贝塔得知五角飞碟被人偷走时,皮皮鲁一屁股坐在沙发上。 
        “是糕鱼氏干的。”皮皮鲁从舒克提供的线索中得出结论。 
        “糕鱼氏这个王八蛋!”贝塔怒不可遏。 
        “没有咱们俩,糕鱼氏要五角飞碟没用,谁给他驾驶呀?”舒克说。 
        皮皮鲁点点头,他认为舒克的话有道理。 
        “看看发展再说吧!”贝塔不乐观。   第153集 
        五只老鼠和一只笼子; 
        火柴盒决定了利的命运; 
        小母鼠走进五角飞碟   
        糕鱼氏得到五角飞碟后如获至宝,他认定自己很快将登上全球黑社会总大王的宝座。 
        糕鱼氏目击过五角飞碟营救皮皮鲁的场面,现在他只要一闭眼腈,脑海里就浮现出五角飞碟那天衣无缝斩钉截铁般的飞行姿态和所向无敌不留蛛丝马迹的绝技。 
        现在,终于把五角飞碟弄到手了。糕鱼氏点燃一只烟,边吸烟边眯着眼睛欣赏桌上的五角飞碟。 
        “这东西行吗?”忘拼命不信貌似玩具的五角飞碟能称霸世界。 
        “绝对没问题。”糕鱼氏自从有了五角飞碟后,说的每句话都透着刚愎自用。 
        “让它飞一圈试试?”忘拼命提议。 
        糕鱼氏将抽了一半的烟扔到地上,他弯下(禁止)子寻找五角飞碟的开关。 
        “怎么没有开关?”糕鱼氏拿着五角飞碟翻过来倒过去地看。 
      “是遥控的吧?”忘拼命说。 
      “遥控的应该有天线呀!”糕鱼氏小时候玩过遥控飞机。 
        “这儿好像有个门。”忘拼命指给糕鱼氏看。 
        糕鱼氏用小刀撬开五角飞碟的门。 
        “拿手电来。”糕鱼氏说。 
        部下递给糕鱼氏手电筒。 
        糕鱼氏举起手电往五角飞碟里照,忘拼命把头凑过来。 
        “是有人驾驶的飞碟!”糕鱼氏看清了五角飞碟内部的设施,看见了操纵台,还有座椅。 
        “就是刚生下来的小孩儿也进不去呀!”忘拼命边说边用手指量五角飞碟的舱门。 
        “太怪了……”糕鱼氏又抽烟。 
        “是不是皮皮鲁有能把人缩小的药?”忘拼命觉得能制出让死人复活的药水的人什么药都能发明。 
        糕鱼氏若有所思地点点头,可马上又摇头。 
        “咱们绑架皮皮鲁那天,这飞碟营救皮皮鲁时,皮皮鲁是在咱们的车上,肯定不在这飞碟里。据咱们对皮皮鲁的监视,皮皮鲁家就他一人,绝对没有第二个人。”糕鱼氏分析。 
        “那是谁驾驶飞碟呢?”忘拼命懒得动这个脑子了,他觉得有这时间够抢两家银行了。 
        说实话,糕鱼氏有一定的智力,他要是把自己的脑细胞往正路上引,说不定也拥有几项发明专利了。可他自小品质恶劣,所有脑细胞天天活跃在丑恶和无聊的脑海中。 
        此刻,糕鱼氏调动了全部的脑细胞思索飞碟驾驶员之谜。 
        他的肿泡眼里闪过一道光。 
        “老鼠!”糕鱼氏一拍大腿。 
        “老鼠?”忘拼命不明白。 
        “皮皮鲁曾经带老鼠去医院看过病,这事当时轰动了新闻界。刚才我们去皮皮鲁家偷飞碟时也看见了一只老鼠!”糕鱼氏分析说,“没错,皮皮鲁准是训练老鼠驾驶这个飞碟!” 
        “咱们再去皮皮鲁家把那老鼠抓来?”忘拼命眉飞色舞。他只要一听到有坑蒙拐骗偷(又鸟)摸狗的任务就兴奋。 
        “别犯傻了,皮皮鲁的老鼠能听咱们的?它保准开了飞碟直接就回皮皮鲁家去了。”糕鱼氏说。 
        “直接回去就好了,要是先干掉咱们再回去,咱们可就惨了。”一个同伙说。 
        沉默。几双贪婪的眼睛从不同的角度注视着五角飞碟。到手的一块肉,不能吃,比没肉还难受。 
        “他皮皮鲁能训练老鼠开飞碟,咱们也能!”糕鱼氏站起来。 
        “对,严格说,老鼠和咱们是同行,有共同语言。”忘拼命赞成,“不听咱们的听谁的?!” 
        糕鱼氏命令手下在1小时内弄到5只老鼠,他要择优录取。 
        一小时后,同伙拎来一只鸟笼子,笼子里囚禁着5只老鼠。它们在笼子里上蹿下跳,显然是想重获自由。 
        “一会儿等它们饿了就老实了,”糕鱼氏说,“只有吃饱了的人才要人权,饿肚子的人只要生存权。要让它们老老实实听话,只有一个办法,就是始终不让它们吃饱。只要一让它们吃饱,它们就想自由,想出来。” 
        果然,老鼠们折腾得渐渐没有体力了。 
        糕鱼氏将一块食物放进一个火柴盒里,扔进笼子。 
        饥肠辘辘的老鼠们嗅到了香味儿,他们围着火柴盒束手无策,只有一只老鼠打开了火柴盒,众老鼠一拥而上,瓜分食物。 
        “就要那只。”糕鱼氏指着打开火柴盒的老鼠对手下说。 
        那只老鼠被留在笼子里,其余的被放掉了。它的智慧使它被囚禁在笼子里。它的同胞由于智力低下而获得了自由。 
        想像的空间越辽阔,生存的空间越狭窄。想像的空间越狭窄,生存的空间越辽阔。上帝的幽默。 
        这是一只小母鼠,它和糕鱼氏对视了足足10分钟。糕鱼氏坚信自己的选择是正确的。他从小母鼠的目光中看到了智慧,还从那目光中看出只有异性相互对视时才会有的那种求助信号。 
        它确有灵气。 
        “从现在起,不要让它吃饱。饿着肚子,它能学会一切。”糕鱼氏深谙训练动物之道,他知道所有马戏团的训练师都是在动物前心贴后心的状态下把它们变成摇钱树的。 
        糕鱼氏决定亲自承担训练小母鼠的任务,他还制订了训练大纲。 
        第一步先要让小母鼠掌握人类的语言。 
        糕鱼氏给小母鼠起了个名字:利。 
        他要让它在一天内知道自己叫“利”。 
        他先饿它。然后每给它一次只够塞牙缝儿的食物时就叫它利。 
        一天下来,只要他一喊“利”,它就过来。 
        利的确聪明,通过两个星期的饥饿学习法,它已经掌握了人类的语言,成为继舒克和贝塔后,地球上第三只能和人类对话的老鼠。 
        我们现在可以称它为她了。 
        今天对糕鱼氏来说是历史性的日子,他要让利进入飞碟。 
        “利,你看这是什么?”糕鱼氏将五角飞碟拿到笼子旁边。 
        利摇头。她没见过飞碟。 
        “这叫飞碟,是一种有威力的飞行器,它是专门为老鼠设计制造的。你的同胞曾经驾驶过它,现在看你的了。”糕鱼氏激利。 
        利心气特高,不服输。 
        糕鱼氏打开笼门,利自被捕后头一次从笼子里走出来。她不但不跑,反而向糕鱼氏投去感激的一瞥。动物懂了人话,思维很容易不正常。 
        利走进五角飞碟。 
        糕鱼氏屏住呼吸。 
        还没超过1分钟,五角飞碟里灯火通明——利找到了照明开关。 
        糕鱼氏在地上打了个滚儿,他知道自己已经成功了。   第154集 
        五角飞碟冲撞糕鱼氏的住所; 
        利要吃燕窝鱼翅; 
        忘拼命命归西天; 
        舒克朝直升机走去   
        糕鱼氏将脑袋凑到五角飞碟的舱门口,往里看。 
        利已经坐在操纵台前的皮椅上,她嗅到了同胞的气息。 
        “你的判断很正确,这飞碟是由老鼠驾驶的。”利扭头对糕鱼氏说。 
        “你也能学会。”糕鱼氏给利鼓劲儿。 
        “我需要这方面的专业知识。”利望着操纵台上密密麻麻的仪表和指示灯说。 
        利坐在座椅上,一边观察五角飞碟的操纵系统,一边听糕鱼氏向她口授航空理论及机械常识。 
        糕鱼氏不间歇地念了半本书,尽管口干舌燥他也不敢停顿,他怕干扰利的思路。他念书时有一种数钞票的感觉。 
        “行了。”利打断糕鱼氏,“我想试试。” 
        “试什么?”糕鱼氏没反应过来。 
        “试飞。”利同糕鱼氏的位置好像已经发生了变化,现在利成了发号施令的一方。 
        “你已经掌握了?”糕鱼氏不想让利拿飞碟冒险。 
        “差不多了。”利说。 
        与其说利是通过理论知识掌握了五角飞碟,不如说她是通过操纵台上舒克和贝塔残留下来的气味儿掌握五角飞碟的。 
        “试吧。”糕鱼氏走到窗户旁,检查了一遍窗户是否关好。他担心利由于生疏把飞碟从窗户摔出去。 
        糕鱼氏站在房问的一个角落里注视着桌子上的五角飞碟。 
        半天,飞碟没动静。 
        糕鱼氏刚要走过去看看是怎么回事,五角飞碟突然离升了桌面,像喝醉了酒一样在屋里横冲直撞,它先撞碎了电视机,接着撞翻了冰箱,撞烂了桌子,撞毁了床…… 
        糕鱼氏双手抱头趴在地上,他知道,只要五角飞碟挨上他,他就有资格参加伤残人奥运会了——如果还活着的话。 
        终于,五角飞碟慢慢冷静下来了,它开始不碰不撞地在屋里飞行——利能够驾驭飞碟了! 
        糕鱼氏从地上爬起来,掸掸身上的土,嘴角闪出一丝奸笑。他知道,自己称霸世界的时候到了。 
        五角飞碟在糕鱼氏脚旁着陆了。利从飞碟里走出来。 
        糕鱼氏真想马上向这只小母鼠求婚,他想通过婚姻控制她,可惜双方体积悬殊太大。糕鱼氏为自己无能为力和一只老鼠结婚而痛心疾首。 
        “祝贺你!你是地球上最智慧的生命。”糕鱼氏恭维利,他要让她永远为他服务。 
        “我想吃好的。”利说。她有了牛的资本了。 
        糕鱼氏从倒了的冰箱里拿出香肠,双手递到利面前。 
        利已经很长时间没饱食过了,她开始狼吞虎咽。 
        “只要咱们好好合作,你天天能吃到香肠。”糕鱼氏说。他清楚,不愁吃的下属不好管。 
        “吃香肠?”利抬眼看糕鱼氏,“我要天天吃熊掌、燕窝、鱼翅……” 
        “一定能!一定能!只要咱们合作,这个地球就是咱俩的了。”糕鱼氏忙不迭地说。 
        利把一根香肠吃完了。 
        “拿饮料来。”利说。 
        糕鱼氏把进口饮料递给利。 
        “飞碟的武器系统你掌握了吗?”糕鱼氏问利。 
        利点点头。 
        “你现在就出击,把这几个人干掉。”糕鱼氏递给利一张名单。 
        利接过名单,连看都没看就走进五角飞碟。 
        糕鱼氏打开窗户。 
        五角飞碟杀气腾腾地离开房间,去执行第一次使命。 
        那名单上的第一个名字就是忘拼命,其余的也全是糕鱼氏的手下。 
        糕鱼氏知道,一无所有时,大家都是朋友。有了一万块钱,有一半朋友会变成敌人。当你有了整个地球时,你的所有朋友都会成为你的死敌。 
        忘拼命和同伙们分别在寓所里被自己抓来的老鼠击毙了。 
        忘拼命到死也不知道是糕鱼氏杀的他。 
        将未来的最大的敌人消灭了以后,糕鱼氏开始策划下一个行动。 
        他把目标定在本市最大的银行。 
        这天下午,利驾驶五角飞碟在空中将那家银行里的所有职员和顾客都击昏,糕鱼氏大摇大摆地进去装了一麻袋大面额钞票,然后开着车回家了。 
        皮皮鲁同舒克和贝塔一起吃饭,自从丢了五角飞碟后,他们之间话很少。 
        电视台突然终止正常节目,插播重要新闻。 
        播音员说,本市最大的银行今天下午遭劫,被劫款项高达1400万元。 
        电视台记者采访银行职员,银行的所有职员都说,他们不知道怎么就晕了,醒来后一看。钱柜已经空了。没任何人看见歹徒。 
        电视台记者采访警方,警方说,这是一起最离奇的抢劫案,没有任何线索,不知歹徒使用的是什么麻醉武器。 
        皮皮鲁和舒克、贝塔面面相觑,他们心里都清楚,这是五角飞碟干的。 
        “这怎么可能?”舒克最先打破沉默。 
        “怎么不可能?”贝塔说。 
        “他找谁开五角飞碟?”舒克问。 
        “找老鼠呀!”贝塔脱口而出。说完他自己也一惊。 
        “糕鱼氏这小子,准是他训练老鼠干的!”皮皮鲁一拳砸在茶几上.玻璃碎了,血滴在地板上。 
        他们三个都明白,五角飞碟掌握在糕鱼氏这种人手里,后果不堪设想。说轻了是天灾人祸,说重了是世界末日。 
        他们还明白,他们现在不是糕鱼氏的对手。 
        “现在插播重要新闻,又一家银行被抢!”电视播音员情绪激动地说,“作案手段同上一次完全相同,警方认为,是同一案犯所为。” 
        舒克和贝塔发现,皮皮鲁盯着电视屏幕的眼睛已经有5分钟没眨一下了。 
        “皮皮鲁,皮皮鲁!”贝塔使劲叫皮皮鲁,怕他憋出病来。 
        “这是我造成的,我去把五角飞碟弄回来。”舒克说。 
        “你怎么弄?”贝塔问。 
        “开直升机去。”舒克说。 
        “直升机可不是五角飞碟的对手。”贝塔说。 
        “会有办法的。”舒克站起来,朝书柜走去。 
        书柜里有他的直升机。   第155集 
        舒克和贝塔驾驶马车直升机寻找糕鱼氏; 
        银行保险柜里的巨款不翼而飞   
        看到舒克朝书柜走去,皮皮鲁木讷地从沙发上起来,走到书柜旁边,他拉开书柜的玻璃门。 
        皮皮鲁不愿让舒克去冒险,他清楚舒克和直升机根本不是五角飞碟的对手。可他现在是一筹莫展。时间就是一切,皮皮鲁知道,糕鱼氏抢腻了银行以后,会用五角飞碟干更损的事。皮皮鲁只能同意舒克去试试。 
        直升机被皮皮鲁拿到桌子上,他用一块软布擦拭机身,他的眼睛直愣愣地盯着直升机。 
        “我也去。”贝塔说。 
        “你们的任务就是侦察到糕鱼氏的住处,千万别和五角飞碟正面冲突,知道了糕鱼氏的住处,我再想办法。”皮皮鲁反复叮嘱。 
        舒克和贝塔点头。贝塔看得出皮皮鲁很窝火,丢了五角飞碟又不能向警方报案。如果警方知道世界上有那么一架所向无敌的飞行器,用不了10分钟就会开新闻发布会。各国由此会发疯似地争夺五角飞碟的,那局面准比五角飞碟落在糕鱼氏手里还糟糕。 
        舒克拉开直升机的舱门,他嗅到机舱里有一股熟悉的气味儿。他感到亲切。 
        舒克和贝塔坐在驾驶台前,系好安全带。 
        皮皮鲁给打开窗户。 
        “我现在没有能同你们联系的通讯设备,你们要见机行事,不准和五角飞碟正面冲突。”皮皮鲁再次嘱咐,他看出舒克有驾机和五角飞碟相撞的心。 
        “放心吧,我们拿什么和五角飞碟冲突呀?如果相撞,我们粉身碎骨,人家汗毛都不会掉一根。”贝塔从直升机里伸出头来说。 
        皮皮鲁给直升机装上了大功率电池。 
        舒克按下启动按钮。 
        直升机缓缓升到空中,在皮皮鲁身边停留了片刻,突然一个急转弯,从窗户冲出了屋子。 
        望着直升机渐小的身影,皮皮鲁叹了口气,义摇摇头。他后悔没在五角飞碟里安装一个遥控爆炸装置。 
        开惯了五角飞碟,再开直升机,舒克的感觉是开完飞机再驾马车。 
        “真慢。”舒克抱怨。 
        “咱们到哪儿找糕鱼氏家?”贝塔问舒克,  “这直升机设备太原始了,要是五角飞碟,打开电脑一扫描,马上就能查出来。” 
        “一个区域一个区域地找,要是能碰上五角飞碟就好了。”舒克一边往前看一边说。 
        “就是看见了,也追不上。”贝塔提醒舒克。 
        舒克咬牙切齿。 
        “要么干脆就别现代化,要么就一直现代化下去,千万别现代化了一半又退回去,这滋味儿可真不好受。”贝塔已经不习惯用肉眼搜寻目标了。 
        舒克和贝塔在空中漫无目的的飞行,他们判断不出糕鱼氏住在哪座楼房里,也没碰上五角飞碟。 
        天渐渐黑了。 
        “返航吧!”贝塔提议,“皮皮鲁准着急了。” 
        舒克驾驶直升机返航。 
        皮皮鲁坐在窗前,两只手托着下巴正发呆呢。 
        直升机从窗户飞进屋里。 
        “怎么样?”皮皮鲁问先从舱门出来的贝塔。 
        “找不者。”贝塔双手一摊。 
        “明天接着去。”舒克一离开飞机就大口大口喝凉水。 
        “真是大海捞针。”皮皮鲁望着窗外的满天繁星说。 
        电视新闻里说,下午,又有一家银行被抢。作案手段和前几次一模一样。 
        播音员还说,由于破不了这个案子,警察局长已绎辞职。 
        房间里一片沉默,谁也不说话。 
        皮皮鲁开始抽烟,一根接一根地抽。 
        “明天咱们选择一家尚未被抢的大银行蹲守,我想,会碰上五角飞碟的。”贝塔打破沉默。 
        “这不是守株待兔吗?”舒克觉得全市有上百家银行,无法确定糕鱼氏将抢哪家。 
        “只有碰运气了。”皮皮鲁想不出更好的办法。 
        第二天一早,舒克和贝塔就驾驶直升机降落在一家还没遭劫的大银行的楼顶上。 
        “咱俩分工吧,我负责天上,你负责地面。”贝塔对舒克说。 
        舒克点点头,他走出飞机,趴在楼边往银行门口看。 
        贝塔也离开飞机,抬头观察天空。 
        银行的门口站着一排防暴警察。看来,每家银行都加强了保安措施。 
        这一天过得真慢。 
        到了返航的时间,舒克和贝塔一无所获。 
        直升机一进屋,贝塔就看出皮皮鲁的情绪已经坏到了极点。 
        “准是糕鱼氏又干坏事了。”贝塔一边解安全带一边对舒克说。 
        舒克皱紧了眉头。 
        “又有银行被抢了?”贝塔一下飞机就问皮皮鲁。 
        皮皮鲁紧闭着嘴,不吭气。 
        舒克发现了桌上被撕碎的一张报纸。他把报纸拼起来。 
        报上的一则消息给了舒克最沉重的打击。消息说,今天有三家银行锁在保险柜里的巨款不翼而飞,而保险柜的门根本没打开过。 
        只有五角飞碟有这种本事。五角飞碟的操纵者正逐步发现和掌握五角飞碟的性能。 
        “再造一个五角飞碟吧!”舒克对皮皮鲁说。 
        “结果是两败俱伤。”皮皮鲁一字一句地说。 
        “我愿意和它同归于尽。”舒克认真地说。 
        皮皮鲁先摇头,又点头,然后又摇头。   第156集 
        利驾驶五角飞碟偷拍世界名人; 
        政治家对儿子说的悄悄话吓了糕鱼氏一跳; 
        匿名照片轰动世界   
        糕鱼氏的住所里的所有地方都堆着大把大把的钞票,用心花怒放来形容糕鱼氏现在的心态是太委屈他了。 
        糕鱼氏最得意的不是钱,而是自己的慧眼,他为自己能在几只老鼠中一跟看中利而得意非凡。 
        利的确智力出众,她每天都能发现五角飞碟的新功能。现在,糕鱼氏根本不用亲自出马,只要利驾驶飞碟悬停在银行的上空,银行保险柜里的钞票就会鱼贯而人糕鱼氏家易主。 
        糕鱼氏和利天天吃山珍海味,没过几天,他俩做梦都想吃韭菜馅的馅饼。 
        糕鱼氏这才知道,敢情天天吃山珍海昧和天天饿肚子一样是受罪。 
        望着一屋了钱,糕鱼氏反倒有一种失落感,他觉得没什么奔头了。 
        “有了钱,你应该当个上等人。”利说。 
        一句话提醒了糕鱼氏。过去,他的确非常羡慕那些腰缠万贯衣冠楚楚的大款,羡慕那些社会名流。别看糕鱼氏是个流氓,可他最爱说的一个词是“档次”。 
        “你帮我干一件事,”糕鱼氏来精神了,“你去世界各地把所有的名人和亿万富翁的生活拍下来,我学习学习,也当个上等人,否则要这么多钱干什么?” 
        “你开个名单。”利说。 
        糕鱼氏把他能记得起来的世界上的所有名流的名字都写在一张纸上,有电影明星、体育明星、国家元首、政府首脑、军队总可令、歌星、政党党魁、亿万富翁、诺贝尔奖得主…… 
        利将这些名字输入五角飞碟的电脑。 
        “我什么时候去?”利请示糕鱼氏。 
        “现在就去。”糕鱼氏急不可待要提高自己的档次。 
        “现在是晚上!”利提醒糕鱼氏。 
        “我不想等了!”糕鱼氏说。 
        “好,我去。”利朝五角飞碟走去。 
        “需要多长时间?”糕鱼氏问。 
        “12小时。”利说,“是全球旅行。” 
        糕鱼氏不知自己怎么熬这漫长的12个小时。 
        利操纵五角飞碟离开糕鱼氏家,去执行特殊的任务。 
        飞碟里的电脑按这些名人居住的地点给他们排列了顺序。利按下自动按钮,五角飞碟开始环球飞行,按设定好的程序拍诸位名流的录像。 
        利一边哼歌一边掏出镜子照,她对于现在的生活挺满意。她清楚自己实际上已经是这个世界的主人了,她现在完全可以一按电钮就毙了糕鱼氏,她很讨厌这个无赖。可她没有,利怕孤独,她需要伴儿。尤其是在她掌握了人类的语言后,她更渴望同人类用语言交流。利曾经梦想有个会说人话的老鼠同胞就好了,但她知道这是不可能的。 
        12小时后,利完成了任务,她驾驶五角龟碟回到糕鱼氏的住所。 
        利将录像带递给糕鱼氏。 
        糕鱼氏正襟危坐,开始一本正经地观看五角飞碟偷拍的世界名流生活片。 
        屏幕上首先出现的是一位大名鼎鼎的歌星,她的唱片曾在全世界创下了销售5亿张的记录。糕鱼氏很喜欢她的嗓子。 
        该歌星刚刚结束一场演出。她乘坐豪华轿车回她的别墅。一进家门,她什么也不干,坐在沙发上一边舔指头一边数着大把大把的钞票——刚才演出的出场费。 
        糕鱼氏原以为歌星演出完了回到家后准有不少浪漫的场面。 
        歌星数完钱后,擤了一把鼻涕。 
        “我唱一首歌能挣一万元!”歌星对着镜子大喊,那神态那举止俨然一个赌徒。 
        糕鱼氏愕然。 
        第二个出现的是一位著名的政府首脑。这位政府首脑在地球上的知名度仅次于太阳。 
        他正在一间豪华的小型放映厅里欣赏录像片,表情庄严,不苟言笑。 
        糕鱼氏立即肃然起敬,他下意识地模仿那位闻名遐迩的政治家的表情。糕鱼氏不知道对方看的是什么节目,他从政治家的表情判断,一定是一部世纪性的政论片。 
        镜头慢慢从政治家的脸上移到了屏幕卜。屏幕上是不堪入目的男盗女娼场面,那情节连糕鱼氏这种高档次的流氓看了都脸红。 
        糕鱼氏这回真服了。人家不愧是政治家,连看黄色录像时都那么庄严那么透着使命感。 
        第三个出现的是一位超级球星,这位球星一上场就能勾走地球上少说20亿人的魂,不管什么劣质运动服,只要让他穿上10秒钟,球迷们就心甘情愿花巨款买。 
        现在,这位球星正在自己给自己打针。 
        糕鱼氏愣,继而恍然大悟,球星是在吸毒。 
        接着出现在屏幕上的是位令影迷们神魂颠倒的女影星,她正在同另位著名女影星争夺一部巨片的女主角。她使尽浑身解数包括最不该使用的解数向导演进攻,导演来者不拒包括最该拒的也不拒。最后,导演在两位女巨星中择优录取了一位,天知道他的标准是什么。 
        一位政党党魁在家中对准备继承父业的儿子传授秘诀:“所有政治主张部是骗人的。你如果要从事政治,必须明白这一点。你在世人面前却要做出随时准备为自己的主义献身的样子……” 
        糕鱼氏起了一身(又鸟)皮疙瘩——一生中头一回。 
        还有亿万富翁还有哲学家还有电视节目主持人还有作家还有政客还有警察局长还有大主教……他们有的同黑手党联系密切有的行贿受贿有的同小姨子关系暧昧有的剽窃别人的作品有的竞选舞弊有的为了给对手下绊不惜重金雇佣带艾滋病毒的暗娼…… 
        “你怎么啦?”利问。 
        “……我看……他们和我也差不多呀……”糕鱼氏说。 
        “我看有的还不如你呢。他们如果有了五角飞碟,准在一天之内抢光地球上所有银行。”利说。 
        “太不公平了!一样丑恶,为什么他们受人尊敬,我们却被人唾弃?!”糕鱼氏怒不可遏。 
        “其实,你当个大国元首绰绰有余。”利说。 
        “咱们先把这些录像翻拍成照片,整理一本画册,名叫《名人丑态一瞬间》送给出版社出版。”糕鱼氏忽发奇想。 
        “先别急着出书。先一张一张寄给报社,等全部发表完了,再出书。”利建议。 
        糕鱼氏自叹弗如。 
        第二天,糕鱼氏挑了一张女明星和名导演的具有超级内容的照片寄给了一家大报。 
        报社编辑漫不经心地撕开信封,抽出照片。立刻,他的眼珠像患了甲亢病一样突了出来。他在摔了5个跟头后鼻青脸肿地跑进总编辑办公室。 
        “为什么不敲门?”总编辑大发其火。 
        编辑将照片递给总编辑。 
        “哪儿来的?”总编辑双目发光。 
        “匿名寄来的。”编辑说。 
        “发明天第一版一个整版!”总编辑歇斯底里,“不,今天的报立即停印,换版,上这张照片!” 
        停印意味着巨额损失,可总编辑认为值得。他清楚,这张照片只要一刊登出来,他的报纸就会在全球家喻户晓,因为这两位名人的名气太大了。总编辑担心那位匿名寄照片的人还会把这张照片寄给别的报纸。 
        两个小时后,头版刊有女影星与男导演巨幅照片的报纸飞进了千家万户,照片旁只有一行字:女星与名导自编自导自演自拍的又一言情巨片(少儿不宜)。 
        该报发行量直线上升。读者纷纷来信来电话要求该报的18个版全部刊登这张照片并连续登一个月。总编辑心花怒放他一个月不用花制版费了不用组稿了不用高薪聘用记者了,就凭这一张照片就能稳赚800万。 
        “第二个寄谁?”糕鱼氏同利商量。 
        利从照片堆里随便挑出一张。 
        声名显赫的国家元首正在偷看女佣洗澡。 
        糕鱼氏将这张照片寄给了一家杂志。 
        该杂志紧急撤换当期封面并不惜血本换用一万克的铜版纸当封面,印上了该国家元首的玉照。 
        该杂志的销量超过了那家报纸。 
        第三张照片问世了。 
        第四张照片抛出了。 
        第五张…… 
        第六张…… 
        糕鱼氏每隔两天将一张照片寄给一家报纸或一家期刊或一家电视台。他还打电话告诉编辑们,他将陆续推出世界上所有名人的这类照片。 
        名人们慌了,他们24小时把神经都绷得紧紧的,他们知道邢可怕的镜头随时会对准自己。他们什幺坏念头也不敢动,每分钟每秒钟脑子里只想怎么推动历史前进怎么解放全人类怎么为灾区捐款他们不敢大便小便不敢干一切与腰下膝上有关的事他们穿着礼服睡觉系着领带上床套着大衣洗澡。 
        终于,汇集所有这类照片的画册《名人丑态一瞬间》由一家幸运的出版社隆重出版了。第一版印数即达50亿册。人类全体成员人手一册。不让谁买谁跳楼。 
        这些照片的作者是谁,成为困扰人类的一个谜。 
        只有皮皮鲁、舒克和贝塔清楚。 
        舒克每天驾驶直升机去找五角飞碟。他想像不出糕鱼氏的下个节目是什么。照片风波使名人的声望一落千丈,而没有名人的世界太寂寞了,人们不再看报纸不再看电影不再听广播,人们这才明白,电影电视报纸电台是为名人设立的,离了名人,它们便没有了价值。就像苍蝇也是生态平衡的一个不可缺少的环节一样,缺了名人,人类无法生存。只有众星,无月可捧,众星只是散沙一盘。从这个意义上说,糕鱼氏毁了人类。 
        一道闪电从舒克的直升机前擦过,舒克一惊,五角飞碟!   第157集 
        舒克驾驶直升机追五角飞碟; 
        利藏在树叶里和舒克、贝塔通话; 
        直升机突然下坠   
        当舒克确信从他眼前擦过的是五角飞碟后,他把油门拧到极限位置,操纵直升机开足马力追五角飞碟。 
        正在后舱睡觉的贝塔被惊醒了。 
        “出了什么事?”贝塔来到驾驶舱。 
        “看见五角飞碟了!”舒克激动。 
        “在哪儿?”贝塔问。 
        “往那边飞了。”舒克指右前方。 
        “已经没影了,追不上的。”贝塔理智地说。 
        舒克不昕,他死攥着油门把手不松。追不上也要追。 
        利驾驶五角飞碟正在返航,她看见警告灯亮了,这说明有危脸。 
        利打开扫描仪。 
        一架微型直升机出现在荧光屏上。 
        利再仔细看,驾驶直升机的是两只老鼠——她的同胞。 
        利操纵五角飞碟隐蔽在一棵大树茂密的树叶里,等直升机。 
        舒克驾驶直升机朝五角飞碟消逝的方向追。 
        “追不上的。”贝塔叹了口气。 
        舒克不说话,两眼发直。 
        躲在树叶里的利吃了一惊:直升机上的同胞也会说人话! 
        接着,一股暖流冲进利的血液,她抑制不住想同两位同胞交谈的愿望。 
        直升机离大树已经很近了,利打开五角飞碟上的通讯设备,找到了直升机的通话频率。 
        “我说追不上,别费劲了。”贝塔劝舒克。 
        “我不信追不上。”舒克脸红脖子粗。 
        “你们追不上我。”利说。 
        舒克和贝塔一愣,从哪儿冒出的女同胞的声音!尽管利使用的是人的语言,可舒克和贝塔还是一下就分辨出这话出自于同胞之口。 
        “别那么大惊小怪的,我在注视着你们。”利一边着荧光屏上的舒克和贝塔一边说。 
        舒克和贝塔明白了,五角飞碟确实是被他俩的同胞驾驶着。 
        贝塔冲舒克使了一个眼色,意思是不能暴露意图。舒克点点头。 
        “你们想搞鬼?”利决定先给这两位操纵原始飞行器的同胞点儿颜色看。 
        直升机突然下坠。 
        “快拉起来!”贝塔急叫。 
        舒克拼命拉杆,无济于事。飞机失控。 
        眼看直升机就要坠地了。 
        “再见了,舒克。下辈子我还和你做朋友。”贝塔闭上眼睛等死。 
        那声巨响没有出现。直升机在触地的一刹那又奇迹般地回升到空中。 
        舒克和贝塔面面相觑。茫然。 
        “哈哈,哈哈,这是大姐和你们开的玩笑。”利开心极了“不过,你们听好,如果你们想打我和我的飞碟的坏主意,我不费吹灰之力就能让你们丧命,刚才就是证明。” 
        贝塔倒吸了一口凉气。 
        “咱们聊聊,可以吗?”利的话里有不容置疑的口气。 
        “当然……可以……”舒克恨不得一口吞了这位女同胞。她坐在原本属于舒克的飞碟里,仗着优势居高临下。 
        这种通话极不公平。利通过先进设备能看见舒克和贝塔而舒克贝塔看不见利。 
        “你们叫什么名字?”利问。 
        “我叫毛克,他叫宝塔。”舒克胡编。 
        “你们从哪儿弄的直升机?” 
        “偷的。” 
        “真有你们的。” 
        ‘你叫什么名字?” 
        “利。” 
        “美丽的丽?” 
        “不,利益的利。” 
        “你驾驶的是什么?” 
        “飞碟。” 
        “飞碟?什么叫飞碟?’ 
        “飞碟是最现代化的飞行器。” 
        “能让我们上去看看吗?” 
        “这可不行。” 
        “你从哪儿弄来的飞碟?” 
        “也算是偷的吧。” 
        “你怎么会开这么先进的飞行器呢?” 
        “自学的。” 
        “你真聪明。” 
        “你们也不笨。其实咱们老鼠比人聪明,我估计咱们老鼠什么都能开,说不定还有开坦克的呢。” 
        “太有可能了。”贝塔插话。 
        “你们开着直升机天天干什么?”利感兴趣地问。 
        “到处打家劫舍。”贝塔说。 
        “可惜设备陈旧了点儿。”利说。 
        “你的设备好,你天天干什么?”舒克问。 
        “抢银行,偷拍名人照片。” 
        “偷拍名人照片?”贝塔假装不明白。 
        “最近你们没看报?”利问。 
        “我们不爱看人类的报纸,假话太多。”贝塔说。 
        “最近的报纸上登的可都是真事,净上名人的隐私,都是我拍的。”利不无得意。 
        “你一会儿去哪儿?”舒克套利的话。 
        “回家。”利挺油。 
        “咱们不能见见面?”舒克总想哄着利让他登上五角飞碟。 
        “我已经看见你们了。”利说。 
        “我们也想着你。你准挺漂亮。”贝塔恭维利。他知道,女性最喜欢异性恭维,只要恭维得恰到好处,她们就会奉献到底。 
        “明天这个时间,咱们还在这儿聊天。”利不理会贝塔的恭维,但她明天还想和同胞聊天。 
        一道黑色的闪电划出大树,五角飞碟走了。 
        舒克看看贝塔。贝塔看看舒克。   第158集 
        糕鱼氏想往利身上倒汽油; 
        皮皮鲁让舒克看《名人情书100篇》; 
        舒克充当色情间谍 
        糕鱼氏突然有不详的预感,他看看表,利怎么还不回来? 
        一阵旋风刮过,五角飞碟出现在糕鱼氏面前。他松了一口气。 
        利从五角飞碟里出来。 
        “没出什么事吧?”糕鱼氏问。他现在对利是毕恭毕敬,他清楚利可以易如反掌地杀死他。 
        “碰到两位开直升机的同胞。”利说。 
        “同胞?老鼠?开直升机?”糕鱼氏挺吃惊,他已经把利当同类了,听她说老鼠是同胞,糕鱼氏觉得不适应。 
        “我还和他俩聊了聊。”利余兴未尽。 
        “它们是干什么的?”糕鱼氏问。 
        “小打小闹抢点东西什么的。”利说。 
        “你应该击落它们。”糕鱼氏说。 
        “为什么?” 
        “不为什么,我觉得是威胁。” 
        “它们是我的同胞。”利瞪了糕鱼氏一眼。 
        “同胞最害同胞。”糕鱼氏说。 
        “你想击落你自己去。”利说。 
        糕鱼氏没词了。他恨不得一脚踩死利。他恨自己这辈子干吗不投个老鼠胎。如果他现在是一只老鼠,他就可以开着五角飞碟直接称霸世界。 
        糕鱼氏隐约感到利不会永远听他摆布,他必须想一个办法。他想掌握全世界的命运。 
        “明天你把全世界所有核武器的按钮给我弄来。”糕鱼氏忽发奇想。 
        “干什么?”利不傻。 
        “没什么目的,就是想要。” 
        “……” 
        “行吗?” 
        “我考虑考虑。” 
        糕鱼氏想浇上汽油把利点了。 
        舒克和贝塔回家后,把同五角飞碟的遭遇告诉皮皮鲁。 
        “利智商不低。”皮皮鲁说。 
        “我们争取明天能进入五角飞碟。”贝塔说。 
        “没戏。”皮皮鲁皱眉头,  “利不会上当。” 
        “那就和它撞。”舒克咬牙。 
        “直升机就是粉身碎骨也伤不了五角飞碟一根汗毛。”皮皮鲁说。 
        “那怎么办?”贝塔躺在沙发上揉腿。坐了一天直升机,挺累。 
        “只有一个办法。”皮皮鲁说。 
        “快说。”舒克迫不及待。 
        “爱情。”皮皮鲁知道这招儿特损,可他想不出第二个办法。 
        “贝塔有当演员的才能,贝塔去吧。”舒克推荐贝塔。 
        “我不去。又不是我丢的五角飞碟。再说了,我又没见过她,谁知道她是不是两条尾巴的怪物。”贝塔坚决不干。 
        皮皮鲁看舒克。 
        “我去吧。”舒克无可奈何。 
        “你一定要让她喜欢你,只要她能同意让你进五角飞碟,她提的任何条件你都必须答应,包括你最不想答应的也要答应。”贝塔特兴台。 
        “明天就舒克自己去赴约吧,你一定要见机行事。”皮皮鲁说。 
        “为了全世界的安全,你要不惜一切代价。”贝塔手舞足蹈。 
        皮皮鲁从书柜里取出一本书,递给舒克。 
        “今天晚上好好看看,这是业务学习。”皮皮鲁说。 
        贝塔凑过去一看,书名是《名人情书100篇》。   第159集 
        舒克以满分成绩通过情书考试; 
        糕鱼氏咬碎了自己一颗牙; 
        探长林发现飞蝶; 
        舒克上情场   
        舒克几乎通宵未睡,看完了《名人情书100篇》。 
        “你光看可不行,必须达到倒背如流的水平。你和利交谈时,要出口成章。女性一般都喜欢有文采的异性。”贝塔今晚一点儿也不困,像喝了lO杯咖啡。 
        这难不倒舒克。舒克写过小说。 
        “你考我吧。”舒克把书递给贝塔。 
        “请背诵马克思写给燕妮的第三封情书。”贝塔随手翻开一页,给舒克出题。 
        舒克一字不差地背出了马克思写给燕妮的第三封情书。 
        “再背克林顿20世纪70年代在美国耶鲁大学法学院上学期间写给同班女生希拉里的第一封情书。”贝塔又出难题。 
        舒克准确地背诵了这位名叫克林顿的美国总统当年写给他的女同学的情书。 
        “100分。”贝塔台上书,对舒克过目成诵的才能表示心悦诚服,“不得不承认你是背诵情书的天才。” 
        “是阅读天才,不管看什么书。”舒克纠正贝塔。 
        “爱情光靠语言可不行,更重要的是靠身体语言。”贝塔好像是专家。 
        “身体语言?”舒克没明白。 
        “也就是行动。”贝塔冲舒克挤眼睛,“现在的女性,不喜欢文质彬彬的男性。她们喜欢粗暴无礼的男性。” 
        “她们喜欢声带语言文质彬彬,身体语言粗暴无礼的男性。”舒克再次纠正贝塔的偏差。 
        “你确实概括力强。”贝塔五体投地。 
        舒克闭上眼睛,开始详细策划白天将要进行的这场爱情战役。 
        “你最好写一份可行性报告,让皮皮鲁和我审查一下。”贝塔又出馊主意。 
        舒克瞪了贝塔一眼。 
        糕鱼氏也是一夜没合眼。他绞尽脑汁想办法让利今天给他弄来核武器的发射按钮。糕鱼氏发现自己已经越来越控制不了利了,他对于利的智商不敢掉以轻心。 
        糕鱼氏现在有的是钱,怎么花也花不完,可他并没感到幸福。他有了钱以后第一件想干的事就是当一个高档次的人。可当他知道了高档次的人并不高档以后,他深刻地意识到生命的无聊。 
        他目前在活着的时候只有一件可干的事了:称霸地球。 
        利睡在五角飞碟里,她不敢睡在外边,她得防着糕鱼氏。她这一生印象最深的事就是糕鱼氏让她去干掉忘拼命。利知道,判断一个人,不能看他对你怎样,得看他对别人怎么样, 
        利也是通宵未眠。不知怎么搞的,自从和那两位同胞通话后,她心里一直处于兴奋状态,同胞就足同胞,感觉确实不一样。 
        利觉得此夜过得特别长,她盼望白天再次和同胞相会。 
        天刚蒙蒙亮,糕鱼氏就敲五角飞碟。 
        利打开舱门。 
        “这么早,干什么?”利问。 
        “去拿核按钮吧。”糕鱼氏说。 
        “我今天不舒服。”利的口气像皇后。 
        “我给你拿点儿药?”糕鱼氏像仆人。 
        “不用了,我一会儿出去散散心。” 
        “去哪儿?”糕鱼氏警觉道。 
        “瞎转转。”利做随意状。 
        “你可要注意安全,我已经不能没有你了。”糕鱼氏差点儿说出“我爱你。” 
        利忍住没吐出来。 
        “能告诉我你去哪儿吧?”糕鱼氏不放心。 
        利摇头。 
        “飞碟是属于咱们俩的。”糕鱼氏谨慎地措词。 
        “那你开吧。”利伸手做了个请的动作。 
        糕鱼氏咬碎了一颗牙齿。 
        “你别急,我现在就去再给你弄点儿钱来。”利看糕鱼氏有几分可怜。 
        “我不需要钱。”糕鱼氏说。 
        利没理他。她钻进五角飞碟。 
        “你到底去什么地方,”糕鱼氏声嘶力竭地喊。 
        利不理糕鱼氏,她驾驶五角飞碟走了。 
        现在距离和同胞约会的时间还早点儿,利驾驶飞碟来到一座大银行上空,她想给糕鱼氏一点儿安慰。 
        利使用飞碟上的超级设备将银行里的巨款“运”回糕鱼氏的住所。 
        当银行职员发现各自桌上堆积如山的钞票突然不翼而飞时,个个惊呼报警。 
        自从银行连续被劫后,警方在各个银行的屋里屋外都安置了摄像机。 
        探长林赶到被劫的银行后,第一句话就是要录像带。他已经知道没有任何歹徒进入过银行,他早已断定最近世界上发生的这一系列怪事属于高科技作案。 
        “倒带。重放这一段。”探长林对助手说。 
        荧光屏上出现了一个黑点儿。 
        “放大这个黑点儿。”探长林说。 
        助手先操纵录像机使画面定格,然后递给探长放大镜。 
        探长林使用放大镜观察屏幕E的黑点儿。 
        一架小飞碟! 
        “好像在哪儿见过?”探长林拍脑袋。 
        他一时想不起来。但探长林已经能够断定,这架小飞碟是抢劫银行的罪犯的作案工具。 
        利把银行的钱给糕鱼氏运了一些后,驾驶五角飞碟赴约。 
        舒克的直升机已经等候在那棵大树旁了。 
        利挺感动。 
        “怎么就你自己来了?”利打开通讯设备的开关。 
        舒克吓一跳,他没看见五角飞碟。 
        “我的朋友有事去了。你在哪儿?”舒克四处张望。 
        “在你头顶上。”利笑。 
        舒克抬头看见了五角飞碟。 
        “你的声音真好听。”舒克开始背《名人情书100篇》中的片断。 
        “真的?”利心跳加快。 
        “不骗你,能和你聊上一会儿,这辈子就算没白活。”舒克肚子里冒酸水,他强忍着没吐出来。 
        “你的声音也好听。”利回敬舒克。每个生命都渴望被别的生命欣赏。 
        “你活得好吗?”舒克问。 
        “还行。你呢?”利反问。 
        “不好。” 
        “为什么?” 
        “特寂寞。特孤独。找不到知音。活一辈子,最重要的就是有知己。” 
        “其实……我也有同感。” 
        “真的?” 
        “真的。” 
        “你觉得人类怎么样?” 
        “穿衣服的老鼠。” 
        “你觉得老鼠怎么样?” 
        “不穿衣服的人类。” 
        “我好喜欢听你说话。”舒克又甩出一句俗得不能再俗的言情小说作家常用的情话。 
        “我也是。”利的话里已经出现了明显的羞涩成分。   第160集 
        地球上少了一只穿衣服的老鼠; 
        舒克给利松绑; 
        糕鱼氏看着电视告别人类; 
        舒克携家眷返航   
        舒克知道利已经上钩了,他急于登上五角飞碟。 
        “我想吻你……”舒克说完就在心里发了一个誓,如果利拒绝他,他就推驾驶杆,和直升机同归于尽。 
        “我也想吻你……”利含情脉脉。 
        “我能进飞碟吻你吗?”舒克迫不及待。 
        “我去直升机里接受你的吻。”利虽坠爱河,但警惕性并未解除。 
        舒克差点儿晕过去。 
        “你把直升机降落在北边那片草丛里。”利为舒克选择着陆地点。 
        舒克无奈,只得按照利的指示操纵直升机在草丛里着陆,他希望利驾驶五角飞碟也落在草丛里。如果足这样,舒克就将利击昏,然后夺回五角飞碟。 
        舒克注意观察四周,没有五角飞碟的踪影。 
        敲击直升机舱门的声音。 
        舒克扭头一看,一只异性同类在叩门,显然她就是利。 
        舒克不得不承认,利长得挺漂亮。 
        舒克打开舱门,利走进直升机。 
        两只穿衣服的老鼠见面了。 
        利的脸上泛着红晕,她站在舒克面前,闭上眼睛,做等待吻状。 
        “你的飞碟呢?”舒克问。 
        利睁开眼睛。 
        “藏起来了。你问这干什么?”利觉得舒克有点儿煞风景破坏了她的感觉。 
        “噢,没什么。”舒克心想利智商的确高,她把飞碟藏起来了。现在即使杀了她,也找不到飞碟。 
        利又闭上了眼睛。 
        舒克清楚,如果他再不吻利,恐怕这辈子也夺不回五角飞碟了。 
        舒克运了运气。当他的脸刚挨上利的脸时,利就情不自禁地死死地抱住了舒克。 
        舒克经受了一场kuangfengbaoyu般的吻。 
        “我爱你……我爱你……”利反复说这一句话。 
        舒克毕竟熟读过《名人情书100篇》,他把那些世界名人的话背了一遍。半个小时过去了愣是没一句重复的。 
        “你能当作家!你真有文化!”利已经对舒克的内涵有了深刻的了解,她喜欢有文采的男性。 
        舒克又甩出一句重磅名人情言。 
        “我想……”利欲言又止。 
        舒克这时才发现,不知从什么时候起,利已经回归自然,成为所有同胞那样的裸鼠。 
        舒克起了一身(又鸟)皮疙瘩。 
        “你这是干什么?”舒克不想再演戏了。 
        “我想嫁给你。”利斩钉截铁地说。 
        “你有陪嫁吗?”舒克问。 
        “飞碟!”利毫不犹豫地说。 
        舒克有点儿感动了,看得出,利是真爱上他了。 
        “行,咱们到飞碟里去结婚。”舒克提条件。 
        “在这儿结婚,再去飞碟里。”利提条件。 
        “先去飞碟,后结婚。” 
        “先结婚,后去飞碟,或者在这儿结婚,去飞碟里再结一次。” 
        舒克无奈,只得在直升机里和利完婚。 
        利说话算数,她带舒克找到了藏在隐蔽处的五角飞碟。 
        当舒克踏八五角飞碟时,他心花怒放。 
        “你怎么了?”利看出新郎神色不对。 
        “我——太——高——兴——了——”舒克转身抱住利,他从兜里掏出一根绳子,将利的双手捆在了一起。 
        “毛克,你喜欢这样玩?”利还以为舒克和她玩呢。 
        “我不叫毛克,我叫舒克!”舒克终于能说实话了,他感到无比的痛快,“这飞碟本来就是属于我的,我才是它的主人。” 
        “你说什么?你在和我开玩笑吧?”利感觉出不对了,她惶恐地盯着舒克。 
        “你看着我开飞碟就知道我说的是实话了。”舒克说完坐到驾驶台前,熟练地将五角飞碟升到空中。 
        新娘子呆若木(又鸟)地看着跟前的一切。 
        舒克把五角飞碟的来龙去脉告诉利。 
        “皮皮鲁真了不起。糕鱼氏太坏了。”利听完后叹了口气。 
        舒克对于利的话挺吃惊。 
        “你比糕鱼氏也好不到哪儿去?”利说。 
        “为什么?”舒克问。 
        “使用欺骗人家爱情的方法夺回五角飞碟,小人。不算男子汉。”利露出鄙视舒克的表情。 
        “你……”舒克无言以对。 
        “我也恨糕鱼氏。你要是算男子汉,就给我松了绑,让我驾驶五角飞碟去消灭糕鱼氏这个人渣。”利说完看着舒克。 
        舒克突然觉得自己挺渺小。今天的做法是挺那个,他注视了利几分钟。然后走过去给利松了绑。 
        舒克想,即使利操纵五角飞碟干掉他,他也不后悔。否则,他心里将一辈子不安宁。 
        利坐在驾驶台前,驾驶五角飞碟直飞糕鱼氏家。 
        “我不想进糕鱼氏的住处。”舒克说。 
        “那就在外边射击。”利说。 
        荧光屏上显示,糕鱼氏正在看电视。 
        利按下了射击按钮。 
        糕鱼氏命归西天,结束了他丑恶的生命。 
        舒克对利肃然起敬。 
        “你以后去哪儿?”舒克问。 
        “和你在一起。”利说。 
        “和我?”舒克一愣。 
        “我是你妻子。”利说。 
        “这……不能算……”舒克慌了。 
        “为什么?”利问。 
        “又没举行婚礼……” 
        “这当然算。婚礼算什么?!” 
        “我得请示一下。”舒克不敢贸然把利带到皮皮鲁家。 
        皮皮鲁正和贝塔在家坐立不安,放在抽屉里的多日不用的五角飞碟通讯器响了,里边传出舒克的呼叫。 
        “舒克成功了。”贝塔跺脚。 
        皮皮鲁打开抽屉,取出通讯器。 
        “我是皮皮鲁!请讲!” 
        “我已经夺回了五角飞碟。糕鱼氏已经下地狱 
        “马上返航!”皮皮鲁说。 
        “我想带利一起回去。” 
        “你说什么,带谁?” 
        “带利。” 
        “这怎么可以?” 
        “她现……已经……是我妻子了。” 
        “舒克,你说什么?” 
        “利很不错的,请你相信我的判断。” 
        “……好吧,我同意。” 
        皮皮鲁将通讯器放在桌子上。 
        “舒克和利真的好上了?”贝塔问皮皮鲁。 
      “大概是吧,已经妻子妻子的喊了。”皮皮鲁耸肩。 
        “没白看《名人情书100篇》呀!”贝塔酸不溜丢地说。   第161集 
        皮皮鲁为五角飞碟制订值班制度; 
        探长林有火眼金睛; 
        贝塔想在舒克的新婚之夜开遥感仪   
        皮皮鲁打开窗户,贝塔站在窗台上,和皮皮鲁一起望穿天空,他们听到了彼此的心跳声。 
        “回来了!”皮皮鲁先看见了五角飞碟。 
        “在哪儿?”贝塔兴奋。 
        皮皮鲁指给贝塔看。 
        “糕鱼氏终究是笨蛋,斗不过咱们。”贝塔拍手称快。 
        五角飞碟像一阵旋风刮进屋里,稳稳地落在桌 
        舱门打开了,舒克面带羞涩地走出飞碟。 
        “谢谢你,舒克!”皮皮鲁眼角溢出泪水,他是替全人类感谢舒克。 
        “这是利。”舒克回身从飞碟里拉出妻子,介绍给皮皮鲁和贝塔。 
        贝塔不知从哪儿弄来一束花,递给利。 
        “欢迎嫂夫人。”贝塔冲利鞠躬。 
        “前一段给你们添麻烦了。”利红着脸说。 
        “不能怪你,主要是糕鱼氏那个恶棍。”贝塔说。 
        “我已经送他离开人类了。”利说。 
        “为民除害是每个生命的义务。”贝塔说。 
        “今后咱们立个规矩,五角飞碟里必须每时每刻有值班的,舒克和贝塔轮流。”皮皮鲁制定防范措施,他允许犯错误,但不允许在同一件事上犯两次错误。 
        “我先值班,让舒克度蜜月。”贝塔特为舒克着想。 
        “我和利是不是应该举行一个婚礼?”舒克请示皮皮鲁。 
        “太俗,用不着。”皮皮鲁摇头。 
        “就是,两颗心相爱,什么都有了。举行一百次婚礼,不爱也是白搭。”贝塔添油加醋。 
        有人敲门。 
        皮皮鲁蹑手蹑脚走到门后透过门镜往外看,是探长林! 
        探长林反复看那盘录像带上的小飞碟。终于,他想起来了,在皮皮鲁家见过它。是皮皮鲁操纵飞碟抢劫银行?凭直觉,探长林不信。可他又确实在皮皮鲁家见过这个飞碟。探长林决定登门拜访皮皮鲁,这是他自连续发生抢劫银行大案以来找到的惟一一条线索。 
        皮皮鲁让舒克、贝塔和利躲进五角飞碟。他把飞碟藏进壁橱。 
        皮皮鲁给探长林开门。 
        “久违了。”探长林满面笑容。 
        “欢迎,欢迎。”皮皮鲁也春风满面。 
        探长林落座。 
        “探长怎么想起到我这儿来了。公务?”皮皮鲁试探。 
        “有点儿小事,麻烦您一下。”探长林从皮包里掏出一张照片,递给皮皮鲁,“您见过照片上的飞碟吗?” 
        皮皮鲁接过照片一看,五角飞碟清晰地印在上边。 
        “这是什么?”皮皮鲁摇头。 
        “您确实没见过?”探长林看出皮皮鲁在撒谎。 
        “没见过。”皮皮鲁一脸的诚实。 
        “最近发生的一系列抢劫银行的案子,都与它有关。”探长林盯着皮皮鲁说。 
        “案子破了?”皮皮鲁问。 
        “还没有。”探长林叹了口气。 
        “以后不会再发生这种事了。”皮皮鲁说。 
        “为什么?”探长林对皮皮鲁的话感到意外。 
        皮皮鲁这才发觉说漏了嘴,忙往回找:  “有您这么神通广大的探长,以后当然就不会再发生这种事了。” 
        探长林足足看了皮皮鲁3分钟,没说一句话。 
        不知怎么搞的,探长林对皮皮鲁就是有信任感。他相信皮皮鲁说的“以后不会再发生这种事了”的话,还相信皮皮鲁与抢银行无关,还相信皮皮鲁与小飞碟有关。 
        人从生到死,干的最重要的事之一就是判断与你接触的人是朋友还是敌人。人类中不计其数的成员认敌为友,或认友为敌。看不准人的人很难成功。而今天的人类越来越隐蔽,越来越让人难以看清真面目。因此,成功的人越来越少。 
        探长林看人很准。尤其能一眼识破道貌岸然的伪君子。去年曾有一张姓嫌疑犯,举止文雅,谈吐不俗,张口历史,闭口哲学。探长林只看了张一眼,就断定他是案犯。果不出探长林所料,张是个国际级诈骗犯。张犯供职的那家跨国公司的老板为此送给探长林一幅巨匾,上书“慧眼金睛”。那老板感激探长林为该公司摘除了一个隐患,本来,该公司正准备任命张犯为公司的第一副总经理呢。 
        探长林认为皮皮鲁不会操纵飞碟作案。至于为什么,他说不清,但他相信自己的感觉。 
        “我告辞了。”探长林起身。 
        “欢迎再来。”皮皮鲁松了口气。 
        走到门口时,探长林忽然回过身问皮皮鲁:“真的不会再有人采用高科技手段抢银行了?” 
        皮皮鲁注视了探长林一会儿,使劲儿点了点头。 
        探长林同皮皮鲁握手,很用力。 
        关上大门后,皮皮鲁从壁橱里拿出五角飞碟。 
        贝塔先从飞碟里出来。 
        “他怀疑咱们了?”贝塔听见了探长与皮皮鲁的对话。 
        皮皮鲁点头。 
        “那怎么办?”舒克从五角飞碟里探出头,问。 
        “他相信我。”皮皮鲁说。 
        “为什么?”贝塔纳闷。 
        “我也说不清。”皮皮鲁挠后脑勺。 
        “还是提防着点儿好。别一会儿警车开到楼下了。”贝塔提醒皮皮鲁。 
        “绝对不会。”皮皮鲁肯定。 
        “咱们给舒克布置一间新房吧。”贝塔比舒克还兴奋。 
        “我有一个非常漂亮的盒子,就用它给舒克当新房。”皮皮鲁说完从柜子里找出一个方盒子,的确很漂亮。 
        皮皮鲁很利索地将盒子里布置了一番,有床,有桌子,有沙发…… 
        “皮皮鲁和糕鱼氏确实不一样。”利依偎在舒克身边说。 
        “人和人的差别比人和动物的差别大多了。”舒克有感触地说,“不一样就是不一样。” 
        当天晚上,舒克和利住进新房。贝塔睡在五角飞碟里值班。皮皮鲁在自己的卧室就寝。临睡前,贝塔问皮皮鲁:“我晚上可以打开遥感器吗?” 
        “遥感观察什么?”皮皮鲁想起了糕鱼氏的恶作剧《名人丑态一瞬间》。 
        “我怕利害了舒克,她万一是特务呢?”贝塔胡诌。 
        “今天晚上不准开遥感器,你要老老实实睡觉,不许胡思乱想。”皮皮鲁瞪了贝塔一眼。 
        “舒克如果被敌人害了,跟我可没关系。”贝塔说。 
        “我负责。”皮皮鲁说。 
        第二天早上,贝塔问舒克:“结婚有意思吗?” 
        “特傻。”舒克只说了两个字。   第162集 
        皮皮鲁同解剖主任通电话; 
        五角飞碟成了救护飞碟; 
        舒克当了爸爸; 
        舒利到了上学的年龄   
        一个月后,利怀孕了。 
        舒克要当爸爸了。 
        当舒克将这一信息传达给皮皮鲁和贝塔时,皮皮鲁和贝塔为舒克高兴。 
        “你能给你的孩子幸福吗?”贝塔一本正经问舒克。 
        “我想…我能……”舒克犹豫了一下,说。 
        “你如果不敢保证能给自己的孩子幸福,那你干脆就别要这个孩子。何必让世界上多一个受罪的生命呢?”贝塔教育舒克。 
        皮皮鲁认为贝塔的话有道理。他觉得地球上不合格的爸爸太多,他们本不具备当爸爸的资格,他们当了爸爸后惟一的乐事就是摧残孩子就是和孩子过不去。 
        “人类中不合格的爸爸比动物中不合格的爸爸多。起码动物家族中的爸爸不会在精神上折磨孩子。”舒克看着皮皮鲁说。 
        皮皮鲁点头同意舒克的话,心里挺为人类悲哀的。 
        利自从同舒克、贝塔和皮皮鲁一起生活以来,好像进入了另一个世界。以前,她一直生活在偷窃、狡诈、贪婪和欺骗的世界里,现在她才知道,同一个地球上并存着两个截然不同的世界。利喜欢皮皮鲁这个世界,她感到皮皮鲁、舒克和贝塔身上有着明显的贵族气质,这个贵族的含义不是奢华,而是高贵的一族。 
        这是皮皮鲁生活得最祥和的一段时问,他和舒克、贝塔还有利每天聊天、看电视、读书,皮皮鲁还经常出去给利买一些营养品。有时,皮皮鲁去舒克贝塔公司出席董事会。 
        一天下午,利突然肚子疼。 
        舒克赶忙叫皮皮鲁。 
        “大概是要生了。”皮皮鲁判断。 
        “快让她躺好了。”贝塔忙前忙后。 
        利疼得浑身痉挛,她的眼睛里充斥着绝望的目光。 
        “这就是母亲。”贝塔头一次目击生命的诞生,他感慨地说。 
        1个小时后,利仍然痛不欲生。舒克急得直揪自己的胡子。 
        “难产。”皮皮鲁嘴里进出两个字。 
        舒克清楚“难产”的含义:难产和死亡是同义语。 
        皮皮鲁想起了曾经给舒克动过手术的解剖主任。他找出电话号码本,给解剖主任打电话。 
        “喂,请问是解剖主任家吗?”电话通了,皮皮鲁问。 
        “是。”解剖主任太太说。 
        “请问解剖主任在吗?” 
        “去外地开会了。” 
        皮皮鲁傻眼了。 
        “去什么地方?”皮皮鲁急中生智。 
        解剖主任的太太将丈夫开会的地点和住址告诉皮皮鲁。 
        皮皮鲁往解剖主任开会的城市打长途电话。 
        解剖主任的声音出现在听筒里。 
        “我是皮皮鲁。” 
        “……噢,您好!”解剖主任挺吃惊。 
        “有个急事需要您帮忙。” 
        “请尽管说,只要我能办。” 
        “我的一位老鼠朋友,对了,就是您上次为他动过手术的那位,他的太太现在难产,需要您救她。” 
        “……可我现在在外地……”。 
        “两分钟后她就到您住的饭店。” 
        “如果只能留存一个,保母亲还是保孩子?” 
        皮皮鲁问舒克。 
        “保母亲。”舒克毫不犹豫地说。 
        贝塔叹了口气,他这才知道,孩子在母亲体内反而不受保护,谁都可以处死他。只有离开妈妈的身体,才受法律保护。对于孩子来说,躲在妈妈的身体里似乎是最安全的,其实,是最不安全的。 
        “贝塔,你驾驶五角飞碟送舒克和利去这座城市,不要让解剖主任看到飞碟。随时同我保持联系。”皮皮鲁将解剖主任的地址交给贝塔。 
        “放心吧。”贝塔拍拍胸脯。 
        舒克和贝塔将利抬进五角飞碟.把她安置在寝室里。 
        贝塔坐在驾驶台前。 
        “五角飞碟请求起飞。”贝塔说。 
        “同意起飞。”皮皮鲁说。 
        贝塔按下起飞按钮。 
        五角飞碟闪电般地飞出皮皮鲁家,直奔解剖主任开会的城市。 
        舒克单膝跪在利的身边,照料她。 
        “再坚持一会儿,马上就到了。”舒克发现利的呼吸越来越微弱。 
        贝塔驾驶飞碟已经到达解剖主任下榻的饭店上空。 
        “我们已经到达目的地。这是一座很大的饭店。如果我们在楼顶着陆,抬着利去解剖主任的房间,起码需要半个小时,我怕来不及了,利的情况很危险。”贝塔向皮皮鲁请示。 
        “通过遥感器观察,如果房问里就解剖主任自己,你们就破窗而入吧。”皮皮鲁也豁出去了。 
        荧光屏上显示房间里只有解剖主任一人。 
        五角飞碟撞碎玻璃闯进解剖主任的房间。降落在写字台上。 
        解剖主任大吃一惊。 
        舱门打开,舒克和贝塔抬出利。 
        “请您快救她。”舒克对解剖主任说。 
        解剖主任回过神来,他在写字台上铺了一块毛巾,将利放到上边。 
        “她已经停止呼吸了,现在要做的事只有救孩子了。”解剖主任给利做完检查后,说。 
        舒克呆若木(又鸟)。失去亲人的痛苦像飓风一样袭击着他身上的每一个细胞。 
        贝塔紧紧攥着舒克的手,他让自己身体里的能量输送到舒克身上,帮助舒克战胜打击。 
        解剖主任通过手术保住了舒克的孩子。 
        “雌鼠。不,是女婴。”解剖主任对舒克说。 
        舒克有了一个女儿。 
        “谢谢您。”贝塔替舒克谢解剖主任。舒克神情恍惚。 
        贝塔和舒克将利的遗体搬回五角飞碟,舒克抱着女儿冲解剖主任鞠了一躬。 
        五角飞碟当着解剖主任的面飞走了。解剖主任坐在屋里望着窗外的天空5个小时没眨一下眼皮。 
        皮皮鲁、舒克和贝塔安葬了利,他们共同承担起哺育舒克的女儿的责任。 
        皮皮鲁为舒克的女儿起名为舒利。 
        舒利在舒克、贝塔和皮皮鲁的精心照料下,一天天长大了。 
        舒克要送女儿上学,可哪个学校会收一只老鼠入学呢? 
        “现在,只要给钱,哪个学校都会收!别说收老鼠,你要是给够了钱,他们会给苍蝇发大学本科文凭。”贝塔出主意。   第163集 
        徐老师升为副校长; 
        舒利被校方拒绝; 
        舒利上电大; 
        解不开眉头的女教师   
        舒克这几天一直为女儿舒利上学的事发愁。 
        “正儿八经地去上学不大可能。”皮皮鲁不信哪家学校会收一只老鼠入学。 
        “舒利必须上学。”舒克不能让女儿成为文盲。 
        “要不你去找所学校试试?”贝塔对皮皮鲁说。 
        “希望不大。”皮皮鲁说,“如果让新闻媒介知道了,又要铺天盖地拿我充版面了。” 
        “去找你的小学老师!”舒克眼睛一亮,想出个好主意。 
        “这倒可以试试。”皮皮鲁点点头。小学老师起码不会把这事张扬出去。 
        皮皮鲁站起来穿外套: 
        “用我们驾驶五角飞碟跟你去吗?”贝塔问。 
        “不用,你们在家等消息吧。”皮皮鲁拉开大门。 
        皮皮鲁驾驶汽车来到母校,看门的老头不让皮皮鲁的汽车进学校。皮皮鲁将汽车停在校门外。 
        “你找谁?”老头摘下眼镜问皮皮鲁。 
        “找徐老师。”皮皮鲁说。 
        “徐老师现在是徐副校长了。”老头告诉皮皮鲁,“上楼第二层。” 
        皮皮鲁敲挂有“副校长办公室”牌子的房门。 
        “请进。” 
        皮皮鲁推门,看见满头银发的徐老师坐在办公桌前看文件。 
        “徐老师,您好。”皮皮鲁问候启蒙老师。 
        “哟,是皮皮鲁!请坐。”徐老师当年教皮皮鲁时,对他时不时歧视一回。后来皮皮鲁出息了,徐老师每当在报纸上看到皮皮鲁的消息,心里很不是滋味儿,她从这件事上知道了学生的学习成绩好坏与他长大的成就无关。后来,徐老师再不歧视差生了。 
        “祝贺您当副校长了。”皮皮鲁说。 
        “惭愧。”徐老师总觉得当年对不住皮皮鲁。 
        “有件事想求您。”皮皮鲁说。 
        “只要我能办到。”徐老师说。 
        “我有个朋友,他的孩子到了上学年龄,我想让她上咱们这所小学,不知……” 
        “智力正常?” 
        “正常。” 
        “住处离这儿不远?” 
        “挺近。” 
        “可以。”徐副校长想拿这事还当年歧视皮皮鲁的债。 
        “只是……她的情况……有点儿特殊……”皮皮鲁吞吞吐吐。 
        “残疾?”徐副校长问。 
        皮皮鲁摇头。 
        “留级生?” 
        皮皮鲁摆手。 
        “品行不好?” 
        皮皮鲁否认。 
        “那她?”徐副校长想不出别的问题了。 
        “她不是人。”皮皮鲁终于说出了舒利的问题所在。 
        “你说什么?你在开玩笑?”徐副校长茫然地望着面前这个若干年前搞恶作剧的差生。 
        “她是一只老鼠。”皮皮鲁一本正经地说。 
        “老鼠?你的朋友的女儿是一只老鼠?”徐副校长盯着皮皮鲁的眼睛。 
        “我的朋友也是老鼠,叫舒克。她的女儿叫舒利。”皮皮鲁告诉徐副校长。 
        “你在和我开玩笑?”徐副校长又问了一遍。 
        “不是玩笑,是真事。”皮皮鲁严肃地回答。 
        “你是想让我接受一只老鼠做我的学校的学生?”徐副校长加重了老鼠和学生两个字的发音。 
        “舒利不是一般的老鼠,她会说人话。”皮皮鲁告诉徐副校长。 
        “会说什么话也不行。我怎么可能让一只老鼠入学?皮皮鲁,你都这么大了,毛病还没改,就喜欢恶作剧。你现在出名了,是不是专门来戏弄我?出出童年的气?”徐副校长眼圈红了。 
        “您别这样,我说的是真事。既然您不同意,就算了,我再想别的办法。”皮皮鲁忙站起来告辞。 
        徐副校长没有送皮皮鲁。 
        舒克和贝塔一看皮皮鲁的脸色就知道事情没办成。 
        “舒利上学的事难度挺大,你别着急。”舒克安慰皮皮鲁。 
        “学校为什么不收老鼠?”舒利问。 
        “那是人类的学校。”贝塔告诉舒利。 
        “人类不喜欢和别的生命在一起?”舒利问。 
        “这要看怎么说了。人类喜欢养狗养猫养鸟,可如果让他们和动物在一起上学,他们就不干了。”贝塔说。 
        “一块上班也不行。”舒克补充。 
        “有办法了!”皮皮鲁从沙发上一跃而起。 
        贝塔和舒克异口同声: 
        “快说!” 
        “咱们让舒利坐在五角飞碟里上学。”皮皮鲁说。 
        “在五角飞碟里上学?”舒克纳闷。 
        “咱们使用五角飞碟里的遥感装置遥感某所学校的教室,舒利一边看荧光屏一边上学就行了,不是有电视大学吗?一个道理。”皮皮鲁滔滔不绝。 
        “太棒了!”贝塔情不自禁地爬到皮皮鲁肩膀上亲了皮皮鲁一下。 
        “这么说,全世界的所有学校都属于舒利了,咱们可以挨个儿遥感,哪所学校好就上哪所学校。”舒克兴奋得手舞足蹈,看样子他不把女儿培养成博士后绝不罢休。 
        贝塔看了一眼墙上的挂钟:“现在正是上课的时间,咱们给舒利挑个好学校。” 
        “去吧。”皮皮鲁朝五角飞碟挥挥手。 
        舒克、贝塔和舒利跑进五角飞碟。 
        皮皮鲁拉开冰箱喝饮料。 
        舒克让女儿坐在操纵台前的皮椅上,他打开遥感器。 
        荧光屏上出现了一座明亮的教室。一个年轻的女教师正在讲课。 
        舒利还从来没见过学校,荧光屏上的场面显然吸引了她。她目不转睛。 
        突然,女教师大吼一声:“王川,站起来!” 
        舒利被吓了一跳。 
        一个瘦瘦的男孩子缓缓地从座位上站起来。 
        “你为什么皱眉头?”女教师质问。 
        “我……皱……眉头……了?”王川胆怯地支支吾吾。 
        “你到前边来!”女教师怒火中烧。 
        王川战战兢兢地走到讲台上。 
        “你对我讲的课不满?”女教师又问。 
        “没有。” 
        “你讨厌我上课?”女教师又问。 
        “不是。” 
        “那你干吗皱眉头?”女教师大喝。 
        “我没皱……” 
        “住嘴!你现在就站在这儿皱眉头,一直皱到下课!”女教师宣布。 
        王川只好站在讲台上,面对着全班同学皱起了眉头。 
        “太轻,皱得再重点儿!”女教师对王川眉心皱纹的起伏度不满意。 
        王川的两道眉毛拧到了一起。 
        开始有几个同学笑,女教师一瞪他们,没人敢笑了。 
        教室里鸦雀无声。 
        王川感到眉头发酸,他脸上的肌肉开始哆嗦,他支持不住了。 
        “继续皱!”女教师喊叫。 
        泪水顺着王川的鼻冀两侧流进了他嘴里,羞辱、愤懑、人的尊严被亵渎…… 
        舒利惊讶地看着荧光屏上的一切。 
        “爸爸,这就是学校?”舒利抬起充满泪水的眼睛看舒克。 
        “这不是学校!是监狱!!”贝塔大骂。 
        “怎么了?”皮皮鲁在五角飞碟外边问。 
        舒克告诉皮皮鲁。 
        “帮帮王川!”皮皮鲁跺脚。他觉得那位女教师把一班学生当牲口对待了。 
        “我来。”舒克从贝塔手中夺过操纵杆。 
        荧光屏上出现了精彩的一幕: 
        女教师突然也皱起了眉头,一看便知她是违心的。她想解放自己的眉头,可她做不到,五角飞碟的遥感设施牢牢地把她的两道眉毛拧存了一起。 
        “怎么回事?!”女教师喊。 
        全班同学包括王川都惊讶地看着班主任。 
        “难受死啦——”女教师痛苦不堪。 
        学生们一个个呆若木(又鸟)。   第164集 
        舒利拒绝上学; 
        皮皮鲁笔枪风靡全球; 
        护身符保佑皮皮鲁战胜原告; 
        联合国慷慨奖励皮皮鲁   
        “饶她吗?”舒克的心肠先软了。 
        “再让她皱一会儿。这叫以其人之道还治其人之身。”贝塔铁石心肠。 
        女教师恼羞成怒,对着学生大吼: 
        “你们看什么!都给我闭上眼睛!谁睁眼睛我就让谁抄一万遍语文书!!!” 
        “简直是虐待狂!”舒克忿然。 
        “哪个家长把孩子送到这种教师手中算是瞎了眼了。”贝塔叹了口气。 
        舒利替那女教师求情了:“饶了她吧。” 
        舒克松开操纵杆。 
        “再换个学校。”贝塔提议。 
        另一间教室出现在荧光屏上。 
        一位男教师正在揪一个学生的耳朵。 
        “你说,是不是你偷的?”男教师吼道。 
        “不是。”学生否认。 
        “我看就是你偷的!你爸就是三只手,这是遗传!”男教师说。 
        “……”学生咬破了嘴唇,血顺着嘴唇流到衬衣上。 
        “这小子在冤枉好人。”贝塔断定。 
        “老师想怎么对待学生就怎么对待?”舒克自言自语,  “孩子是有尊严的人,不是任人侮辱的牲口!” 
        荧光屏不住地更换画面,舒克和贝塔不停地摇头。 
        舒利站起来宣布:“我不上学了。” 
        舒克与贝塔面面相觑。 
        贝塔说了一句话:“舒克,你要是送你女儿去这样的学校上学,你就不配让舒利管你叫爸爸!少去这种学校上一天学,长大就多一分出息。” 
        舒克、贝塔和舒利愁眉苦脸地走出五角飞碟。 
        “选到好学校了吗?”皮皮鲁看出情况不妙。 
        “都在摧残孩子。”舒克小声说。 
        “孩子需要自卫,他们太无助了,简直是一群任人宰割的羊羔。”贝塔说。 
        皮皮鲁想起了自己的学生时代。 
        ‘咱们应该发明一种武器,让上学的孩子们自卫用。”贝塔提议。 
        ‘自卫什么?”皮皮鲁问。 
        “自卫尊严。自卫人格。自卫人权。”贝塔说。 
        皮皮鲁若有所思地点头。 
        “发明一种微型枪,枪里装有五角飞碟的那种遥控系统,一旦老师欺侮学生,学生能用这枝枪来自卫,保卫自己的尊严不受侵犯。”舒克异想天开。 
        “这种武器最好同学生最常用的物品结合起来。”皮皮鲁说。 
        “学生最常用什么?”贝塔问。 
        “笔。”舒克说。 
        “笔枪。”皮皮鲁脱口而出。 
        “咱们就研制笔枪。给每个学生装备上。”贝塔说。 
        皮皮鲁把自己关在房间里,像发明五角飞碟那样,几天几夜没出屋,埋头研制笔枪。 
        5天后,世界上第一枝专供学生自卫用的笔枪问世了。这种笔枪外形和钢笔一样,能写字,它配有5发子弹。枪里装有高科技的遥控系统,能抵御来自老师的攻击。同时,没有杀伤力。该笔枪还能高速书写作业,每分钟能写1万个字。 
        “就叫皮皮鲁笔枪!”贝塔给笔枪命名。 
        第二天,皮皮鲁将“皮皮鲁笔枪”交给舒克贝塔公司大批量生产,随后投放市场,武装学生。 
        皮皮鲁笔枪立即受到孩子们的空前欢迎。几乎在一夜之间,全球所有的孩子都装备上了这种“新式武器”。在一所小学里,一位老师罚一名学生抄写一个生字达300遍。该学生使用笔枪自卫。该教师不由自主地抄了300遍那个字。 
        人类有多少孩子就有多少枝皮皮鲁笔枪,还有的孩子同时携带好几枝皮皮鲁笔枪。 
        老师再不敢虐待学生了,他们终于知道了当老师最重要的是要尊重学生。处处和学生过不去的老师不是老师。 
        舒利开始在五角飞碟里通过荧光屏上学。有了皮皮鲁笔枪以后的教室温暖多了,舒利喜欢这种平等的师生关系。 
        “在这个世界上,枪很重要。”贝塔深有感触地说。 
        “人类恨枪,但又离不开枪。学校的师生关系靠枪来维持,这是人类的悲哀。”皮皮鲁说。 
        舒克没想到女儿舒利的入学问题居然改变了人类的学校。 
        一位老师向法院控告皮皮鲁发明皮皮鲁笔枪有罪。法院传皮皮鲁出庭。 
        “带上你的护身符,谁也不能伤害你。”舒克提醒皮皮鲁。 
        皮皮鲁从小就佩戴一个护身符,它保佑皮皮鲁逢凶化吉刀枪不入。 
        在法庭上,皮皮鲁击败了原告。法官判皮皮鲁无罪。 
        法庭外,几十万名手持皮皮鲁笔枪的孩子声援皮皮鲁。他们个个佩截皮皮鲁护身符。 
        皮皮鲁护身符迅速出现在人类每一个孩子的身上,它保佑孩子们不受侵犯不受伤害不受欺侮。 
        在联合国大会上,皮皮鲁笔枪被推选为奉世纪最伟大的发明。它的问世拯救了人类的下一代。 
        联合国为此奖给皮皮鲁一辆超豪华轿车。皮皮鲁再次成为全球新闻的焦点。   第165集 
        鲁西西回国定居; 
        出任舒克贝塔公司服装部经理; 
        张总向鲁西西射出暗箭   
        舒克和贝塔在擦拭五角飞碟。舒利在飞碟里跟着荧光屏听课。皮皮鲁躺在沙发上看书。 
        电话铃响了。 
        皮皮鲁拿起话筒。 
        “你好,我是皮皮鲁。”皮皮鲁听到对方报名后十分兴奋,“鲁西西!你什么时候回国?乘哪次航班?明天下午到?我去机场接你!” 
        舒克和贝塔相视一笑。他们知道皮皮鲁的妹妹鲁西西长期在国外从事服装设计,已经在国际服装界小有名气。鲁西西很少回国。由于阴差阳错,舒克和贝塔从来没见过鲁西西。 
        “祝贺你,皮皮鲁,要和妹妹见面了。”贝塔说。 
        “谢谢。鲁西西说,这次回国,她就不走了,要在国内定居。”皮皮鲁眉开眼笑。 
        “咱们给鲁西西收拾出一个房间吧?”舒克建议。 
        “这主意不错。”皮皮鲁点头。 
        大家一起动手,为鲁西西布置房间。 
        第二天下午,皮皮鲁驾驶他的豪华轿车到国际机场接鲁西西。 
        国际机场有秩序地忙碌着,一架架飞机穿梭似地在跑道上起降着。下飞机的旅客脸上都有白捡一千元钱的表情,等候上飞机的旅客一脸的忐忑不安。乘坐飞机旅行毕竟是通过远离大地的方式到达目的地,谁心里也没底。说白了,坐飞机是拿生命和蓝天赌博。 
        鲁西西乘坐的航班正点进港,那是一架垂直尾翼上涂满了花花绿绿颜色的巨型喷气客机,它像一只张开翅膀的老鹰向地面扑来。 
        当飞机的三组起落架全部落地后,皮皮鲁松了一口气。 
        皮皮鲁同鲁西西已经几年没见了,当鲁西西出现在出口处时,皮皮鲁眼睛不觉一亮。鲁西西光彩照人,她那时装模特一样的身材配上得体的服装,吸引了无数的目光。 
        皮皮鲁接过鲁西西手中的皮箱。 
        “皮皮鲁,你没什么变化。”鲁西西一边端详哥哥一边说。 
        “你可是越长越漂亮了。”皮皮鲁说。 
        “你还是一个人生活?”鲁西西边走边问。 
        “四个。”皮皮鲁说。 
        “四个人?”鲁西西吃惊。 
        “还有三位朋友。”皮皮鲁走到自己的汽车旁边,打开后备箱,将鲁西西的皮箱放进去。 
        “三位朋友和你住在一起?”鲁西西担心自己没地方住了。 
        “他们不占地方。”皮皮鲁拉开车门,请妹妹上车。 
        “怎么回事?”鲁西西觉出哥哥在和她开玩笑。 
        “是三只老鼠朋友。舒克、贝塔和舒利。”皮皮鲁发动汽车。 
        “老鼠朋友?真逗!很可爱吧?”鲁西西系好安全带。 
        “他们在家等你,你准会喜欢他们。”皮皮鲁驾驶汽车驶离国际机场。 
        汽车在高速公路上飞驰,鲁西西惊讶地望着车窗外。 
        “变化太大了。”鲁西西深有感触地说。 
        “城里变化更大。”皮皮鲁说。 
        当皮皮鲁把鲁西西介绍给舒克他们时,双方在1秒钟内就成了好朋友,鲁西西太喜欢这三只小老鼠了。 
        “外国好吗?”几乎去过地球上所有国家的贝塔问鲁西西。 
        “不像我小时候听说的那么不好,也不像现在国内的人说的那么好。”鲁西西回答。 
        “你回国准备干什么?”舒克问。 
        “原来没想好。之所以回来,就是觉得是什么地方的人就应该生活在什么地方。现在我知道自己该干什么了。”鲁西西说。 
        “干什么,刚决定的?”皮皮鲁问。 
        “从机场到家里这一路上,我看见咱们国家的孩子穿的服装太单调了。 
        我要给他们设计服装,打扮他们。”鲁西西已经做出了决定。 
        “太好了,你加盟我的公司吧?”皮皮鲁说, 
        “就是舒克贝塔公司。”贝塔说。 
        “行。”鲁西西爽快地答应了。 
        “你可以担任公司的服装部经理。”皮皮鲁给妹妹封官。 
        “可以。”鲁西西同意。 
        “鲁西西应该是副总经理兼服装部经理。”贝塔说。 
        “就当服装部经理。如果干好了,你们再提拔我。”鲁西西说。 
        “在国外呆过的人,回国经商准成功吧?”舒利问。 
        “那可不一定。外国的笨蛋一点儿不比中国少。”贝塔说。 
        “没错。”鲁西西证实贝塔的话。 
        “鲁西西能干好。”皮皮鲁了解妹妹。 
        鲁西西的确是服装设计大师,她设计的皮皮鲁牌和鲁西西牌服装大受欢迎,一上市就供不应求。孩子们以穿皮皮鲁牌和鲁西西牌服装为荣。鲁西西干脆将两个品牌合二为一,称为“皮皮鲁鲁西西牌”服装。 
        鲁西西还组建了“皮皮鲁少年模特队”,定期向社会发布新服装信息。 
        舒克贝塔公司在服装界掀起了一场强大的旋风.令所有同行膛目结舌。 
        就像所有名牌都被不法之徒觊觎一样,一个专门从事生产假冒产品的黑公司盯上了皮皮鲁鲁西西牌的服装。 
        这天,黑公司的张总经理正在为公司庞大的债务发愁,经理助理推门进来。 
        “张总,您看这件衣服。”助理递给总经理一件款式新颖的上衣。 
        “这衣服怎么了?”张总看不出名堂。 
        “这个牌子的服装最近在市场上火极了。” 
        “什么牌子?” 
        “皮皮鲁鲁西西牌,据说1个月就销售了70万件。” 
        “就吃它了!”张总脸上露出狞笑。 
        说干就干,黑公司把积压在仓库里的衣服全部印上了皮皮鲁鲁西西商标。 
        各大商场一听说黑公司有皮皮鲁鲁西西牌服装,就争先恐后地进货。 
        这天下午鲁西西正在总经理办公室审查新服装款式,电话铃响了。 
        鲁西西拿起话筒。 
        “请问是鲁经理吗?”对方问。是鲁西西的秘书。 
        “你好,我是鲁西西。” 
        “今天市场上开始出现大量假冒咱们商标的伪劣产品!” 
        “你说什么?!?”鲁西西脸色变了。 
        对方重复了一遍。 
        “你马上买几件给我送来。”鲁西两放下电话。 
        20分钟后,秘书把假冒伪劣产品放在鲁西西的大办公桌上。 
        “卑鄙。”鲁西西一边翻看一边说。 
        “有的顾客买了后大喊上当。”秘书说。 
        “派公司所有人员出动.把这些假货全买下来。”鲁西西毅然决定。 
        “这……这得需要很多资金。”秘书为难。 
        “快去!”鲁西西下令。她知道,让这些假货在市场上多呆一分钟,她的公司就多一份危险。 
        舒克贝塔公司的全体职员出动,到街上的所有商店去买假冒的皮皮鲁鲁西西牌服装。公司为此耗资60万元。 
        第二天,假冒产品又出现了。 
        “怎么办?”秘书问鲁西西。 
        鲁西西最恨假冒名牌的奸商。可她现在束手无策。派出去追查假冒厂家的人都空手而归,没找到半点儿蛛丝马迹。 
        鲁西西在绝望中想到了五角飞碟。 
        她给皮皮鲁拨电话。   第166集 
        张总请美貌小姐进餐; 
        jiachao困扰张总; 
        警察在张总住处查出不法之物; 
        舒利向舒克提出考试要求   
        皮皮鲁放下电话,对舒克和贝塔说: 
        “你们有事干了。” 
        贝塔兴奋:“杀富济贫?” 
        皮皮鲁:“差不多。有人假冒鲁西西的服装,你们去把那奸商找出来,教训他一下。” 
        贝塔:  “太棒了!那奸商也忒傻,假冒什么不好,偏偏假冒皮皮鲁牌,这不是往枪口上撞吗。” 
        舒克:“什么时候起飞?” 
        皮皮鲁:“现在。” 
        舒克对舒利说:“你在家等爸爸。” 
        舒利点头。舒克和贝塔钻进五角飞碟。 
        皮皮鲁掏出微型通讯器。 
        “五角飞碟请求起飞。”舒克坐在驾驶台前。 
        “同意起飞。”发皮鲁指示。 
        五角飞碟离开了皮皮鲁家。 
        贝塔坐在皮椅上抬起双脚表示高兴,他早就想驾驶五角飞碟四处“找碴儿”了。 
        “悠着点,高兴的时候还没到呢!用电脑把那小子查出来。”舒克说。 
        贝塔打开遥感电脑。 
        “姓张。黑公司总经理,今年40岁……”贝塔一边注视荧光屏一边读数据。 
        “这小子一共假冒了多少件服装?”舒克问。 
        “超过了100万套。”贝塔说。 
        “咱们怎么处置他。”舒克问。 
        “干掉算了,为民除害。”贝塔说。 
        “杀人是法院的事,咱们可不能轻易下这个手。”舒克挺懂法律。 
        “糕鱼氏不就是你杀的吗?”贝塔说。 
        “是利杀的。”舒克更正。 
        “咱们逗逗这个张总吧?”贝塔想玩恶作剧。 
        “你构思一下。”舒克不反对。 
        贝塔懒得动脑筋,他让电脑帮他构思捉弄张总的剧本。 
        贝塔从10个不同的构思中选定1个。 
        舒克同意了。 
        恶作剧开幕。 
        张总同一小姐到大饭店进餐。张总靠假冒皮皮鲁牌服装发财后,最爱干的事就是请小姐吃饭。 
        张总风度翩翩地在餐桌旁同小姐一边进餐一边侃侃而谈,其春风得意状溢于言表。 
        小姐吃着人家的饭,用一脸的崇拜回报,其洗耳恭听状十分虔诚。 
        席间,张总时不时谆谆纠正小姐的错误动作,诸如喝咖啡时一定要把勺子拿出杯子吃完基围虾要用茶水涮指头餐巾只能擦嘴不能擦汗询问厕所在哪儿不能说厕所只能说卫生间如果说出盥洗室就更高雅了好像同样是大小便只要说卫生间排出来的就不是粪便而是自助餐。 
        饭毕。张总用极潇洒的手势通知服务员结账。 
        服务员拿来账单。 
        张总付款。 
        5分钟后,服务员小姐和一位保安人员来到张总面前。 
        “对不起,先生,您付的是jiachao。”服务员小姐对张总说。 
        “你说什么?”张总怀疑自己的耳朵。 
        “您付的是jiachao。”这回轮到腰间挂着电棍的保安人员说话了。 
        “这不可能!”张总看了陪他吃饭的小姐一眼,他的自尊被伤害了。 
        小姐惊讶地注视着眼前发生的一切。 
        张总从兜里掏出一把百元大钞,递给保安人员,说:“这难道都是假钱吗?你们的视力有问题吧?” 
        保安人员不客气地接过张总手上的大把钞票,对着光照了一遍。 
        “全是jiachao。”保安人员神情严肃地宣布。 
        “你胡说!”张总恼羞成怒。 
        “你自己看看。”保安人员把钱还给张总。 
        张总有识别jiachao的常识。他一看,傻眼了。 
        “我……我用信用卡付款。”张总掏出信用卡,他要挽回面子。 
        “这信用卡也是假的。”服务员小姐查阅黑名单后说。 
        “请您跟我们走一趟。”保安人员说话时故意把手放在电棍的把手上。 
        “这不可能!”张总抗议。 
        “再见了,张先生。”陪同张总吃饭的小姐不愿被牵连,溜了。 
        “你等等……”张总喊小姐。 
        小姐头也没回。 
        张总被保安人员带进一间办公室。几分钟后,进来两名警察。 
        “这些jiachao是哪儿来的?”警察问。 
        “挣……来……的……”张总吞吞吐吐。 
        “你家住在哪儿?”警察又问。 
        张总不敢说,他家和公司全是假冒产品。 
        不说警察也有办法,张总头一次坐上了警车。 
        十几名警察搜查了张总的公司和家,他们虽然没有找到制造jiachao的机器,但查出了大量的假冒皮皮鲁和鲁西西牌伪劣产品。 
        检察院对张总提起公诉。 
        法院以侵权罪判处张总无期徒刑。 
        当皮皮鲁和舒克贝塔从电视新闻中看到这一新闻时,弹冠相庆。 
        鲁西西打来电话,向舒克和贝塔表示感谢。她还说,舒克贝塔公司的产品又大受孩子们欢迎了。她还在筹备成立一家打击假冒产品的公司,专门维护名牌产品的声誉。 
        舒利每天坐在五角飞碟里跟着荧光屏上学,她已经可以看书了。 
        “爸爸,我想拿文凭。”一天,舒利对舒克说。 
        “要文凭干什么?那玩意是自欺欺人的东西。”舒克教育女儿。 
        “我想考试。”舒利上了这么长时间学,还没考过试。她羡慕考场上的那些学生。   第167集 
        舒克动员贝塔加入同盟军; 
        皮皮鲁给鲁西西打电话; 
        支票换来舒利的考试资格   
        舒克将舒利想去学校参加考试的想法告诉贝塔。 
        “去学校参加考试?还是自己主动要求?舒利疯了!”贝塔吃惊。 
        “这个世界上最完善的人就是疯子。”舒克口吐哲言。 
        “考试是一种虐待行为,是先得到知识的人对后得到知识的人的一种虐待。一种幸灾乐祸。一种五十步笑百步。”贝塔一边擦拭五角飞碟一边说。 
        “她非要参加,我也没办法,这事得皮皮鲁帮忙。”舒克说。 
        “你忘了上次舒利要上学,皮皮鲁去联系学校碰钉子的事了。”贝塔提醒舒克。 
        “也是,哪所学校会同意让老鼠参加考试呢?”舒克恨自己的身上没有人类的血统。 
        “其实,只要你有钱,当初找个人类姑娘,现在起码人家可以同意舒利半日制上学呀!”贝塔揶揄舒克。 
        “可我没钱呀!只要有钱,别说老鼠,就是苍蝇也会有人类姑娘排队候嫁的,这点儿我比你清楚。”舒克坚信愿意和钱结婚的人类姑娘为数不少。 
        楼下传来汽车引擎声。 
        “皮皮鲁回来了。”贝塔听得出是皮皮鲁的汽车声。 
        “帮我说服皮皮鲁。”舒克动员贝塔。 
        “咱们又要把皮皮鲁往火坑里推了。”贝塔一脸的无可奈何。 
        皮皮鲁进门后先去卫生间洗手。 
        舒克和贝塔还有舒利正襟危坐。 
        “有事?”皮皮鲁一进客厅就发现气氛异常。 
        “是这样。”舒克看了舒利一眼,  “舒利想去学校参加考试……” 
        “去学校参加考试?为什么?”皮皮鲁问舒利。 
        “我学了这么长时间,想验证一下。”舒利回答。 
        “你觉得你学的怎么样?”皮皮鲁问。 
        “挺好。”舒利说。 
        “那就行了,干吗非要得到别人的肯定?”皮皮鲁自幼深受考试之苦。他没想到舒利会自己往虎口里钻。 
        “我觉得……不经过考试就不算学习过。”舒利走进了误区。 
        “不经过考试就能学到知识的人是世界上最幸福的人。”皮皮鲁一字一句地说。 
        “我想和他们比比,看谁学得好。”舒利不能自拔。 
        “那我们考你。”皮皮鲁说。 
        “对,我们给你出题。”贝塔眼睛一亮,他一生干过许多事,惟独没当过考官。 
        “爸爸给你判卷子。”舒克也来劲了。 
        “不,我要去学校考试。”舒利开始掉眼泪。 
        “别哭别哭,咱们想想办法。”皮皮鲁坐在沙发上安慰舒利。 
        “舒利,你想参加什么级别的考试?”贝塔问。 
        “高中。”舒利说。 
        皮皮鲁想到鲁西西。 
        “鲁西西有说服人的才能,让她想想办法。”皮皮鲁给鲁西西拨电话。 
        鲁西西表示愿意和皮皮鲁一起去学校联系舒利的考试问题。 
        “你们在家等消息吧。”皮皮鲁拿着汽车钥匙离开家。 
        “凶多吉少。”贝塔望着皮皮鲁的背影说。 
        皮皮鲁驾车赶到舒克贝塔公司,接上鲁西西一同去某中学。 
        “有戏吗?”皮皮鲁一边开车一边问鲁西西。 
        “难度够大的。”鲁西西注视着前方说。 
        “就去这所中学吧,挺有名气。”皮皮鲁指指路旁的一所中学。 
        “行。”鲁西西点点头。 
        汽车驶进路旁的中学。 
        校长惊讶地注视着走进他的办公室的两位不速之客。 
        鲁西西递上自己的名片。 
        校长显然知道鲁西西的大名,他忙招呼手下给客人沏茶倒水。 
        “快期末考试了吧?”鲁西西问校长。 
        “最近就忙这个。”校长说。 
        “校外学生可以申请参加贵校的考试吗?”鲁西西问。 
        “没有先例。”校长不明白鲁西西问这个干什么。 
        “我能介绍一位学生参加贵校的期末考试吗?”鲁西西同。 
        “他是哪所学校的?”校长问。 
        “自学,没上过学。”鲁西西说。 
        “为什么没上学?” 
        “种族歧视。” 
        “种族歧视?” 
        “是的。” 
        “什么族?” 
        “……”鲁西西看了身边的皮皮鲁一眼,她在找不会吓着校长的词。 
        “人类的朋友。”皮皮鲁插话。 
        校长茫然。 
        “这么说吧,世界上什么特殊情况都有。可人们总爱用老观念来观察和评判周围的事物。人和人交朋友没人奇怪,可如果人和老鼠交朋友就会招来非议。”鲁西西诱导校长朝容忍老鼠和人类交朋友的方向发展。 
        “世界上的伟人千姿百态,但他们都有一个共同点——怪。”皮皮鲁协助鲁西西把校长往伟人堆里推。 
        “有话请直说。”校长不是学生,不需要循循善诱。 
        “您是明白人。”鲁西西先送校长一顶高帽子,“您能同意一只有知识的老鼠参加贵校的考试吗?” 
        “不同意。”校长说。 
        “如果我们出钱呢?”鲁西西使出了杀手锏。 
        “出钱?”校长一愣。 
        “如果您同意让舒利——就是我们的老鼠朋友——参加考试,我的公司将赞助贵校一笔可观的经费。”鲁西西从包里拿出支票本。 
        校长迟疑了。 
        “你出多少?”校长问 
        “您想要多少。”鲁西西反问。 
        校长伸出两个手指头。 
        “两万。”鲁西西问。 
        校长脸一红,忙点头。其实他的原意是两千。 
        “可以,”鲁西西在支票上填阿拉伯数字,再签上自己的大名,然后撕下支票递给校长。 
        校长接过支票,眼睛里放出欣喜的光。他的学校的教育经费太少了,他甚至想,如果鲁西西赞助他100万,让他把学校改名为老鼠中学他都干。 
        “让舒利后天上午8点钟到校,在第5教室参加考试。”校长对皮皮鲁和鲁西西说。 
        “谢谢您。”皮皮鲁认定该校长当总理都绰绰有余。 
        ”后天见。”鲁西西要亲自陪舒利来参加考试。   第168集 
        老鼠花钱能当总统; 
        教师扬言绝食; 
        动物学家在床上看到了出头之日; 
        舒克送女赴“刑场”   
        皮皮鲁一进家门,舒克就知道女儿可以去学校参加考试了。皮皮鲁是哼着歌进来的。 
        “舒利,舒利。”皮皮鲁叫舒利。 
        “可以啦?”舒利从五角飞碟里钻出来,她也从皮皮鲁的声音里判断出有戏。 
        “学校同意你去参加考试了。”皮皮鲁不无得意地说。 
        “太阳从北边出来了。”贝塔一边说一边试图开酒柜的玻璃门,“咱们应该喝点儿酒庆祝庆祝,毕竟是人类第一次同意老鼠参加他们的考试!” 
        贝塔越来越喜欢喝酒了,他觉得酒后那种朦胧的感觉特享受,特愉悦。 
        “多亏鲁西西,是她出了钱校方才同意的。”皮皮鲁指指身后的鲁西西,说。 
        “谢谢你,鲁西西。”舒利说。 
        “你有把握考好吗?”鲁西西挺希望舒利给她增光。 
        “我对分数不感兴趣,进考场是为体验那种感受。”舒利说。 
        “舒克,你说如果咱们出够了钱,说不定会有国家让咱们老鼠当总统呢!”贝塔喝干了杯中的酒,说。 
        “那肯定。这世界上,钱除了买不到幸福以外,什么都能买到。”舒克说。 
        “后天我和鲁西西护送舒利去考试,你们在家等舒利的好消息。”皮皮鲁对舒克和贝塔说。 
        “遵命。”贝塔又喝了一杯酒。 
        鲁西西和皮皮鲁离开中学后,校长坐在办公室里看着桌上的那张支票发愣。 
        “校长,该开会了,各年级老师都到齐了。”教导主任探头进来叫校长。 
        全校教师会议。 
        校长向教师们布置期末考试注意事项。 
        “对了,还有一件事要和大家打个招呼。”校长说,“本次期末考试。将有一位特殊身份的考生参加本校的考试。” 
        教师们以为是国家元首的私生子。 
        “是一只老鼠。”校长宣布。 
        “老鼠?” 
        “校长说什么?” 
        “是老鼠,没错!” 
        教师们交头接耳。 
        “我再说一遍。后天,有一只老鼠参加咱们学校的期末考试。希望大家到时不要吃惊,也不要影响学生考试。”校长说。 
        “是拍科幻电影?”教导主任询问。 
        校长摇摇头。 
        “那是?”教导主任不明白。 
        “她要求参加考试。”校长说。 
        会场突然鸦雀无声,教师们初步断定校长中了什么魔法。 
        “这不行,如果学生们知道一只老鼠和他们一起考试,那还不翻了天?谁还有心考试?”教导主任旗帜鲜明地反对。 
        “对,不能让老鼠参加考试,这简直是亵渎知识”。又一位老师揭竿而起。 
        “不让老鼠考试!” 
        “我也不同意!” 
        “如果让它参加考试,我要求辞职!” 
        “我罢教!” 
        “我绝食!” 
        教师们对校长群起向攻之。 
        校长一言不发,任凭下属们发泄。 
        “说完了没有?”校长等教师们嚷嚷完了,问。 
        “我还要说……” 
        “我说……” 
        “太不像话了……” 
        教师们意犹未尽。 
        “先听我说,人家可是出了钱的。考一次试,给咱们两万元!”校长大声说。 
        没人吭声了。 
        每个教师都在心里用两万除以全校教职上的总数。数学教师比语文教师先得到答案。 
        “其实,让老鼠参加考试也无所谓。说不定还能提高学生对考试的兴趣呢!”一位数学教师率先扬帆转舵。 
        “就是,让它参加我们班的考试吧!”语文教师虽然在心算上较之数学教师慢了半拍,但她从社会学角度算计出老鼠在哪个班考试,哪个班的班主任的劳务费就应该比别的班的班主任高半拍。 
        “欢迎它到我的班来。” 
        “我强烈要求老鼠到我们班来!” 
        “来我的班!” 
        “不来我罢教!” 
        “不来我绝食!” 
        教师们以工作以身体健康以死威胁校长。 
        “我已经决定了,让老鼠去第5教室考试。”校长高声宣布。 
        所有目光集中在第5教室班主任身上,那目光交织成一张网。网上有刺。 
        一位生物教师回家后在床上对她先生说了这件新闻。她先生是位小有名气的动物学家。 
        “你是开玩笑?”动物学家不喜欢太太在床上幽默。他研究了半辈子动物,他觉得人和动物的区别就在于人在床上特脆弱特不堪一击。 
        “真事。”太太说。 
        “老鼠参加考试?”动物学家坐起来问。 
        “是的。”太太把经过详细叙述一遍。 
        动物学家突然容光焕发,他意识到自己从小有名气转变到闻名遐迩的契机来到了。 
        “后天我去你们学校。”动物学家离开床,在房间里兴奋得来回踱步。 
        “干什么?”太太问。 
        “研究那只老鼠,如果它真能考试的话。”动物学家笑了。是一种动物的笑。 
        后天到了。 
        清晨,皮皮鲁和鲁西西为舒利准备考试用具。 
        舒克在一旁看着。 
        “什么心情?”贝塔一边打哈欠一边问舒克。 
        “有送孩子上刑场的感觉。”舒克断定天下所有真正爱孩子的父母在送亲骨肉去考试时都有大义灭亲送子赴刑场的感觉。 
        “发明考试的人现在准在地狱里熬着呢。”贝塔说。 
        “咱们出发吧?”鲁西西说。 
        时钟指到七点三十分。 
        “祝你好运!舒利。”贝塔说。 
        “不管你考多少分,在爸爸眼里,你都是全世界最可爱的孩子。”舒克说了一句世界上所有爸爸送孩子考试前说的最伟大的临别赠言。 
        舒利在皮皮鲁和鲁西西的陪同下,前往人类的中学参加期末考试。 
        老鼠参加人类的考试,毕竟在地球的历史上还是头一次。 
        头一次发生的事,注定会有一番磨难。   第169集 
        全班考生交白卷还得100分; 
        动物学家闯入考场; 
        校长担心支票作废; 
        舒利生死未卜   
        校长在教学楼门口等皮皮鲁和鲁西西。 
        汽车停稳后,鲁西西下车走到校长面前。 
        “舒利来了?”校长问。 
        鲁西西点点头。她对校长管舒利叫名字不叫老鼠感觉非常好。 
        “咱们去考场吧。”校长引路。 
        皮皮鲁和鲁西西跟着校长走进第5教室。教室里已经坐满了考生。考生们显然已事先被告知有特殊考生和他们同堂应考,他们的目光齐刷刷地射向皮皮鲁和鲁西西。 
        “舒利就在这张桌子上考试。”校长指着最后一排的一个空位子说。 
        鲁西西从包里掏出舒利,放在课桌上。 
        舒利环顾四周,感到新奇。 
        学生们尽管已被校方打了预防针,可他们还是兴奋无比。 
        “它真会写字?” 
        “这只老鼠认识咱们人类的文字?” 
        “是神鼠?” 
        “我敢说,这是老师想出的新招儿,考验咱们精力集中不集中。” 
        学生们议论纷纷。 
        “你们可以离开教室了,昕到打下课铃,来接舒利就行了。”校长客气地对鲁西西和皮皮鲁说。 
        皮皮鲁对舒利说:  “我们去车上等你。” 
        舒利点点头。 
        皮皮鲁和鲁西西离开教室,回到汽车里。 
        校长答应皮皮鲁和鲁西西绝对保证舒利的安全。 
        考试开始。 
        老师发考卷。 
        往常考试轮到发考卷时,考生们都埋头看考卷。今天几乎全体考生不约而同地看舒利。 
        他们不信舒利能看懂考卷。 
        舒利开始答题了。她用鲁西西送给她的考试专用笔飞快地写着。 
        邻近座位的学生最先看到舒利的答案完全正确。 
        “她答出来了!答对啦!!”那男孩儿不顾一切地喊道。 
        全班同学呼啦一下围了过去。 
        “回到座位上去!快回去!!太不像话了,这是考试!!!”监考老师气急败坏。 
        学生们一步三回头地回到自己的座位上。 
        动物学家带着两个助手出现在教室门口。 
        “你们找推?”监考老师问。 
        “我们是动物研究所的,听说你们这儿有一只不同寻常的老鼠?”助手之一问。 
        监考老师说:“现在正在考试,请你们先出去。” 
        “在那儿,我看见了。”助手之二指给头儿看。 
        动物学家一个箭步跨到舒利桌旁,他的小眼睛放大了两倍。 
        舒利流利地答题,她全神贯注,没注意到身边发生的事。 
        助手之一掏出微型摄像机拍摄舒利答题的镜头。 
        “奇迹。”动物学家明白历史从现在起就要改写了——由他执笔。 
        “带走。”动物学家小声对助手之二说。 
        助手之二从皮包里掏出一个小铁笼子。 
        “快跑!”坐在舒利旁边的男孩儿大喊。 
        舒利听到喊声抬头一看,一只大手从天而降抓住了她。 
        “干什么?”舒利抗议。 
        助手之二迅速将舒利塞进笼子里,他知道这只老鼠是无价之宝。 
        “你们不能抓走它!”坐在舒利旁边的男生站起来说。 
        “对,你们不能抓它!”学生们纷纷站起来。 
        “你们这些小笨蛋!让老鼠和你们一起考试是对你们的侮辱!”助手之一对学生们说。 
        “你才是笨蛋!”学生们听到助手之一骂人,集体回骂他。 
        “这班学生算是完了。”动物学家摇头。 
        “你才完了呢!”学生们集体回敬他。 
        监考老师喊来校长。 
        “你们干什么?”校长看着笼子里的舒利,质问动物学家。 
        “你干什么?”动物学家反问。 
        “什么我干什么?”校长发怒。 
        “您身为校长,却让一只老鼠和学生同堂考试,这事如果张扬出去,您这位校长的宝座恐怕坐不成了吧?”动物学家振振有词。 
        校长无言以对。 
        “走。”动物学家一挥手。 
        “你们不能带走它!”校长说。 
        “我们从教室里抓走一只老鼠,你告到哪儿也没用。”助手之二推开校长,昂首走出教室。 
        动物学家和另一个助手也趾高气扬地离开教室一 
        校长发呆。 
        “校长,您应该去通知带老鼠来的那两个人。”监考教师提醒校长。 
        校长顿悟,他三步并作两步跑到皮皮鲁的汽车旁。 
        “出事了!”校长一脸的紧张,他怕鲁西西要回支票。 
        皮皮鲁摇下车窗。 
        “来了三个人,把舒利抓走了!”校长说。 
        皮皮鲁和鲁西西对视了一秒钟。 
        “谁。什么时候?”皮皮鲁问。 
        “刚才,他们刚出去。”校长指校门。 
        “就是刚过去的那三个人?”皮皮鲁对他们有点儿印象。 
        “我去教室看看,你去追他们。”鲁西西下车往教室跑。 
        皮皮鲁发动汽车去追劫持舒利的凶手。 
        鲁西西在舒利的课桌上拿到了她的考卷。考卷基本上答完了。百分之百的正确。 
        下课铃响了,第5教室的考生除了舒利全是白卷。 
        校长说,全算100分。 
        皮皮鲁驾车沿着学校附近的道路跑了几遍,没看到动物学家的影子。 
        鲁两西上车后对皮皮鲁说:“学生告诉我,那人自称是动物研究所的。对了,舒利考试能得100分。” 
        “去动物研究所!”皮皮鲁咬牙切齿。 
        汽车风驰电掣。 
        动物研究所是一座三层楼房。皮皮鲁找遍了所有房间,没有舒利和那三个人。 
        传达室的工作人员告诉鲁西西,那三个人早晨出去的,现在还没回来。 
        舒利在哪儿? 
        动物学家没有将舒利带回研究所,他不愿意和同事们分享成果,他崇拜独吞。 
        助手之一是单身汉,住一套一居室的单元房。动物学家和两位助手将舒利藏匿于该单元房中。 
        动物学家明白解开这只老鼠的高智商之谜不用研究,只需审讯即可——因为它会说话。 
        助手之二将笼子放在桌子上。 
        三人面对笼子坐定,像是三位法官。 
        审讯开始。 
        动物学家想好了,如果它不开口,就用刑。动物学家从小就对动物感兴趣,他曾经有过给一只老鼠浇上汽油点着了的劣迹。   第170集 
        动物学家当法官; 
        筷子伸进笼子里; 
        五角飞碟撞碎皮皮鲁的窗户; 
        电线向舒利逼近   
        人都想把自己的生命活得比别人精彩。活得好离不开机会。一般人是等待机会,聪明一些的人是把握机会,再高明一点儿的人是创造机会。 
        动物学家属于把握机会的人。 
        他自幼就对人类以外的动物感兴趣,他研究它们的饮食,研究它们的婚恋,研究它们的传宗接代,研究它们的一切。 
        遗憾的是,动物学家的努力并未使他的生命比别人的生命辉煌,他已经4l岁了,还没上过一次报纸。电视新闻倒是露过一次脸,可那是一则交通事故的现场报道,当时动物学家正好路经事故现场,摄像机将他和其他人一起作为围观者摄人了镜头。每当他看到报纸或电视上的人物专访时,一股无名的妒嫉之火就焚烧着他的身心。 
        动物学家绝不会放过舒利,他必须把握这个机会。 
        舒利在笼子里惊恐地注视着笼外的三个男人,她过去只在书本上知道“劫持”这个词。 
        动物学家示意助手打开录音机。 
        助手之二按下了录音机上的红色按键。 
        “你必须认真回答我提的每一个问题,希望咱们合作愉快。”动物学家开始发问,口气十分居高临下。 
        “你是谁?为什么劫持我?”舒利并不怵这三个男人,她从小是在五角飞碟里长大的,不知道什么叫害怕。舒利反问动物学家。 
        “我是这个世界上专门研究你们的专家。”动物学家不放过任何一个暴露自己身份的机会。 
        “他有高级职称。”助手之一补充。 
        “为什么绑架我。”舒利问。 
        “绑架?”动物学家看了两位助手一眼,“你管人抓动物叫绑架?那咱们这个世界上每天发生的绑架事件可太多了。” 
        “绑架是违法行为。”舒利提醒他们。 
        “那要看被绑架的对象是谁了。”助手之一说,“绑人犯法,绑动物可就没人管了。” 
        “你现在回答我的问题,你从什么时候开始会说人话的?”动物学家开始审讯舒利。 
        舒利拒绝回答。 
        “说话!”助手之二高声喝道。 
        动物学家示意助手不要大声呵斥。 
        “她很聪明,她会说话的。”动物学家意味深长地看着舒利,说。 
        舒利从动物学家的目光中看到了狼的眼神,她从电视节目里的《动物世界》专栏中见过狼。 
        “回答专家的问题。”助手之一督促舒利。 
        “生下来就会说。”舒利不想吃眼前亏。她知道皮皮鲁很快就会设法营救她的。 
        “你的爸爸妈妈会说人话吗?”动物学家眯起眼睛注视舒利。 
        “我认识的老鼠都会说人话。”舒利说。 
        动物学家显然兴奋了,他希望抓到100只会说人话的老鼠。 
        “你住在哪儿?”动物学家想抄家。 
        “地球上。’舒利回答。 
        “地球上的什么地方。”动物学家耐着性子问。 
        “东半球。”舒利继续逗动物学家。 
        “可以告诉它应该怎么回答问题了。”动物学家对助手们说。 
        助手之一将一根筷子插进笼子里,他用筷子使劲儿捅舒利。 
        舒利试图躲避筷子的攻击,但笼子的空间太小了,她的身上被狠狠地戳了几下,疼得她尖叫起来。 
        舒利终于明白了一个道理,地球上最没有人性的就是人。 
        皮皮鲁和鲁西西在动物研究所没找到舒利,皮皮鲁对鲁西西说:“赶快回家,动用五角飞碟。” 
        皮皮鲁一路连闯十几个红灯,交通警察在后边跺脚骂街。 
        皮皮鲁风风火火打开家门,舒克和贝塔一看便知不妙。 
        “舒利呢?”舒克劈头便问。 
        “被一个混蛋绑架了。”皮皮鲁直奔五角飞碟停放处。 
        “怎么回事?”贝塔问鲁西西。 
        鲁西西将经过叙述一遍。 
        “又一个往枪口上撞的。”贝塔为那位动物学家惋惜。他有眼无珠,绑架了五角飞碟驾驶员的女儿。 
        “快查明舒利的方位!”皮皮鲁近似于大吼。 
        舒克和贝塔钻进五角飞碟。 
        舒克操纵电脑查找女儿。 
        贝塔两眼死盯着屏幕。 
        “在这儿!”贝塔发现了被关在笼子里的舒利。 
        一个青年人正用筷子在戳舒利。 
        “我杀了他。”舒克按五角飞碟的起飞按钮。 
        五角飞碟突然腾空而起,径直撞碎了玻璃射出窗外。 
        “悠着点儿!连个招呼也不打!”皮皮鲁其实挺希望五角飞碟教训那厮一下。 
        “这回你该说了吧?”动物学家问舒利。 
        舒利浑身火辣辣地疼。 
        “你如果还不说,我就给你通电。”助手之一笑容可掬地对舒利说。 
        舒利注视着人类的这三位成员,她有世界末日的感觉。 
        “我现在开始倒数计时,lO,9,8……”动物学家过足了当主宰者的瘾。 
        人类最大的瘾就是主宰瘾。最上等的主宰瘾,是主宰同胞,第二等的是主宰金钱,什么都不行的就去主宰学问。这三样都主宰不了的,只好去主宰动物了。 
        电线已伸进了笼子里,两根线头从不同的方向逼近舒利。 
        “我说。”舒利不吃眼前亏,她要拖延时间,等待五角飞碟。 
        “你不傻。”动物学家对于舒利这么快就投降感到有点儿遗憾。他真希望这种场面一直维持到他的生命终结。他觉得这太享受了。 
        “你刚才问什么问题?”舒利装傻。 
        “你的住处。” 
        “我住在北合雁大街……”舒利做沉思状,  “我想想门牌号码……” 
        “最好别记错。”助手之一怪声怪气地提醒舒利。 
        “好像是一百多少号……”舒利说。 
        “我看该给它通通电了。”动物学家看出舒利在涮他。 
        舒利想躲电线。 
        “电它!”动物学家下令。 
        “哐啷——” 
        窗玻璃碎了。五角飞碟从外边冲进屋里。舒利乐了。 
        动物学家和助手受惊了。他们恐慌地看着空中的这个不明飞行物。 
        舒克会放过他们吗?   第171集 
        拖把未发射即流产; 
        台灯成为导弹击中专家的关键部位; 
        贝塔当法官; 
        动物学家剖析灵魂深处   
        动物学家和两位助手被破窗而人的五角飞碟惊呆了。他们一时手足无措。他们都是有科学头脑的人类成员,他们清楚这种造型的飞行器(不管体积如何)意味着什么。 
        片刻,动物学家先回过神儿来。他暗示助手们对这架飞行器发动攻击。 
        助手之一突然从门后抄起一个拖把,他试图用木棍击打五角飞碟。 
        他的速度非常快,用闪电形容不过分。 
        然而,当他手中的木棍就要击中五角飞碟时,他扔掉了拖把,并且伴随着一声痛苦的喊叫。 
        他的手像被电击了一样,麻木不仁。在一瞬间,他感到自己已经失去了双手。 
        另一位助手抄起桌上的台灯,他想用台灯砸五角飞碟。 
        他倒没有遇到反击,台灯顺利地从他的手中飞出。 
        只是台灯没有按他的意愿飞向五角飞碟,而是像导弹那样拐了一个潇洒的弯,击中了动物学家的下(禁止)。 
        动物学家捂着小腹蹲下了。 
        贝塔的杰作。 
        五角飞碟降落在笼子旁。 
        动物学家和两位助手忐忑不安地注视着飞碟。 
        五角飞碟的舱门开启了,舒克走出飞碟,跑到铁笼子的小门旁。 
        一只老鼠从飞碟里出来,来救另一只老鼠。 
        动物学家的脑海里闪过一个可怕的念头:地球上有一部分老鼠已经进化,掌握了先进的科学技术和人类的语言。他明白这个发现的价值,他不放过这个成名的机会。 
        动物学家不顾下(禁止)的不适,他像猛虎一样扑向五角飞碟和舒克。 
        他不知道飞碟里还有值班的。 
        贝塔用五角飞碟的武器将动物学家击倒在地。 
        两位助手扶起老师,他们再不敢轻举妄动了。 
        舒克打开铁笼子的门。 
        舒利从笼子里跑出来和爸爸拥抱。 
        “我知道你们会来。”舒利对爸爸说。 
        “他们没伤害你吧?”舒克问。 
        “这三个人够损的,一直在审我,还用刑。”舒利挺委屈。 
        贝塔从飞碟里探出头:“舒克,你和舒利进飞碟,我来审审他们!” 
        舒克同意了。他和舒利钻进五角飞碟里。贝塔根本不把这三个大男人放在眼里。 
        动物学家和助手们像看科幻片一样。 
        “你们三个听好,并排坐在地上。接受我的审讯。”贝塔说。 
        三个人不动。 
        舒克在五角飞碟里强迫他们并排坐在地上。 
        “把录音机关了。”贝塔对五角飞碟说。 
        桌上的录音机先是冒烟,继而爆炸。 
        动物学家感到生命危在旦夕。 
        “你叫什么名字?”贝塔指着动物学家问。 
        动物学家感到奇耻大辱。堂堂人类怎么能接受动物而且是老鼠的审问呢?特别是他居高临下和动物打了一辈子交道,最终栽在了动物手里。 
        “不回答?”贝塔问。 
        动物学家不吭声,他想维护人类的尊严,想当英雄。 
        “给他通电。”贝塔说。 
        不大不小的电流进人了动物学家的躯体。动物学家挂白旗: 
        “我说!我说!” 
        “你叫什么名字?”贝塔问。 
        一个特俗的名字。 
        “年龄?” 
        一个高不成低不就的岁数。 
        “住址?” 
        一个脏乱不堪的地方。 
        “邮政编码?” 
        一个不吉利的数字。 
        “职业?”贝塔问。 
        “动物学家。” 
        “专门研究动物的?” 
        “是的。” 
        “人和动物的最大区别是什么?”贝塔问。 
        “……”动物学家犹豫。 
        “快回答!”贝塔又做电他的手势。 
        “人会使用工具,动物不会。” 
        “胡说!我不就会开飞碟吗?”贝塔反驳,“告诉你,人和动物的最大区别就是人有幽默,动物没有。不过我看你一点儿幽默感也没有,所以你不能算是真正的人类。” 
        动物学家哑口无言。 
        “你喜欢动物吗?”贝塔又问。 
        “喜欢……”动物学家说。 
        “为什么喜欢?” 
        “这是我的职业……” 
        “你是研究动物的,你设身处地为动物想过吗7” 
        “……” 
        “你想下辈子当一回动物吗?” 
        “……” 
        “如果你下辈子必须当动物,你又有选择权,你当什么动物?” 
        “……” 
        “这个问题必须回答!”贝塔提高了嗓门。 
        “当……熊猫……”动物学家说,“褐马(又鸟)也行。” 
        “你倒不傻,尽捡珍贵的当。想过当老鼠吗?” 
        “没……有……” 
        “那你为什么抓舒利?” 
        “舒利?” 
        “就是你刚才抓的那只老鼠?” 
        “想……研究……它……” 
        “你过辈子最想干什么?”贝塔问,“你必须说实话,我们的飞碟上有测谎器,你一说假话,红灯就亮。我是人类学家,动物家族里专门研究人类的。” 
        “我最……最……” 
        “别吞吞吐吐,利索点儿!” 
        “最想干成事业。” 
        飞碟上红灯亮了。 
        “说谎!再回答。” 
        “最想……出名……。” 
        红灯又亮了。 
        “你这次必须说真心话,否则你将被电击10分钟。” 
        ‘我最……想……娶……10个……老婆。”动物学家面对电流,不得不说实话。 
        五角飞碟上的绿灯亮了。 
        两位助手吃惊地扭头看老师。 
        “你对你的助手怎么看?”贝塔又问。 
        “我想培养他们成为……” 
        红灯大亮。 
        动物学家被电击中,痛苦不堪。 
        “我心里想让他们永远不如我。永远给我当催巴儿,什么绝招儿也不教他们。” 
        绿灯亮。   第172集 
        动物学家每周工作七天; 
        助手之二被电老虎追赶; 
        麻雀和鹦鹉; 
        伟人和小人   
        两位助手比看见五角飞碟还吃惊地看着昔日他们顶礼膜拜的导师。 
        “如果让你再结一次婚,你最想和谁成亲?”贝塔又回到了高精尖的问题上。 
        动物学家满脸通红。 
        他说了一个女人的名字。红灯亮。电击。 
        他又说了一个名字。红灯又亮。又电击。 
        动物学家不敢说了。 
        电击。 
        他实在忍受不了电击,终于说出一个令两位助手瞠目结舌的名字。 
        绿灯亮。 
        “这人是谁?”贝塔问助手之一。 
        “我……的……未婚妻……”助手之一羞愤交加。 
        动物学家无地自容。其实,往日他花在研究动物身上的精力还不及他花在研究助手未婚妻身上的精力的十分之一。他一天起码有10个小时在琢磨那姑娘,还是周七工作日。 
        “你可够损的,你和她有关系了?”贝塔往深里问。 
        “没有!没有!”动物学家不停地摆手。 
        绿灯亮。 
        他说的是实话。 
        助手之一稍稍松了口气,本来他已担心未婚妻不是原装的了。 
        “你把同她结婚纳入你的生活计划了?”贝塔觉得特过瘾。 
        ‘没有没有……”动物学家矢口否认。 
        绿灯又亮。 
        “那你想干什么?”贝塔大喝一声。 
        “……我……” 
        “快招!”贝塔催促。 
        “我……我……就是在脑子里瞎想她……”动物学家认定自己今生今世再也不能见助手之一了。 
        “我明白了,在你的想像世界里,你早已和她结婚了,是吗?”贝塔厌恶地看着动物学家,他觉得他很脏。 
        动物学家承认。 
        “依我看,动物和人类的最大区别就在于,动物看上别人的配偶,就去竞争。而人类看上了别人的配偶,就在脑子里瞎想。”贝塔又有新发现。 
        动物学家和助手们洗耳恭听贝塔训示,他们清楚自己是地球上惟一被老鼠拘禁的人类成员。 
        贝塔不想再审动物学家了,他觉得恶心,他知道如果再这样问下去,保准动物学家在脑子里不光只和助手之一的未婚妻结过婚。 
        “现在你回答我的问题。”贝塔对助手之二说,他对助手之一的未婚妻被老师惦念表示同情,给他一个反思的时间。 
        “都干过什么坏事?”贝塔劈头便问。 
        “……”助手之二不知所措。 
        “捡最坏的说。” 
        “写论文时,抄袭别人的……” 
        红灯亮。电击。 
        “不是最坏的,再说!”贝塔知道为什么有好多人爱当法官了。法官的含义就是把自己的坏藏起来,把别人的坏暴露出来。 
        “有一次,坐公共汽车的人特别多,一位……小姐……挨着我……” 
        “冬天还是夏天?” 
        “夏天……我……” 
        红灯亮。电击。 
        “还不是最坏的,再交代!”贝塔回头看看五角飞碟,舒克伸出头翘大拇指。 
        皮皮鲁已经同舒克联系上了,他和鲁西西在家收听现场直播。 
        助手之二慌了,他真的记不清自己干过的最坏的事了。 
        他像一个被猛虎追杀的人,力图躲避电击的袭击: 
        “我捡到过一个钱包,把钱花了……” 
        电击。 
        “我不爱我的女朋友,可为了得到她的父母的权势,我假装爱她……还……” 
        电击。 
        “我诬陷过一个人……” 
        电击。 
        “我希望我的老师……早……死……” 
        电击。 
        动物学家希望电击死这个助手。 
        当贝塔认定如果再电击下去,助手之二必死无疑时,他放了他。 
        “轮到你了。”贝塔对助手之一说。 
        助手之一麻木地看着贝塔。 
        “你研究了多少年动物?” 
        “7年。” 
        “最大的感受是什么?” 
        “动物研究动物。” 
        “你最喜欢什么动物?” 
        “麻雀。” 
        “为什么?” 
        “不在笼子里生存,不苟且偷安。” 
        “你最讨厌什么动物?” 
        “鹦鹉。” 
        “为什么?” 
        “只会重复别人的话。” 
        “你过去怎么看你的这位老师?” 
        “伟人。” 
        “现在怎么看?” 
        “小人。” 
        “这两者之间有什么关系?” 
        “伟人往往是小人。” 
        “你为什么选择这个职业?” 
        “不想和人打交道。” 
        “如果下辈子当什么让你选择,你当人还是当动物?” 
        “不想还有下辈子。” 
        助手之一一副看破红尘的表情。 
        “你对今天的遭遇有什么感觉?” 
        “天外有天。做事一定要留有余地。” 
        贝塔看出助手之一醒悟了不少。自己的未婚妻被自己的老师惦记,再傻的人也会醒悟。 
        “该返航了。”皮皮鲁指示舒克。 
        舒克让舒利告诉贝塔。 
        “你对他还有什么说的吗?”贝塔问受害者。 
        “以后做事别太恶,尤其是不要使用武力对待别的生命。”舒利告诫三位研究动物的人。 
        三个人使劲几点头。那诚恳劲儿使人相信往后他们对苍蝇都会彬彬有礼。 
        “今天的事不能说出去。”贝塔临走甩下一句话。 
        三个人抢着点头效忠。 
        贝塔和舒利钻进五角飞碟。 
        动物学家和弟子用全新的眼光看他们眼前的动物。 
        五角飞碟升到空中,故意在三个人头上绕了一圈儿,然后撞碎了另一块玻璃,飞到窗外。 
        整个儿一个地球的主宰。   第173集 
        动物学家和助手决定继续学先进; 
        鲁西西看着舒利想起了罐头小人; 
        贝塔为罐头小人不穿衣服操心   
        五角飞碟离开后,动物学家和两位助手望着支离破碎的窗户发愣,他们就这样望着窗户一言不发地在地上坐了整整两个小时。 
        他们在梳理自己的思想,他们必须重新认识这个世界。既然已经有老鼠掌握了超现代化的工具,会不会还有别的动物也掌握了如此先进的科学技术呢? 
        人类还被蒙在鼓里,还在为自己是地球的主宰而沾沾自喜洋洋得意。动物学家悲怆地想。 
        两个小时的心潮澎湃终于失落了,他们现在要面对的是必须重新审视三个人之间的人际关系。那架可恶的由老鼠操纵的飞碟破坏了他们的师生和同事关系。如果所有朋友都必须面对面地说真话,这个星球上的字典里将不再收留“朋友”这个词汇。 
        沉默。 
        尴尬。 
        无地自容。 
        助手之一长叹一口气,先站起来。 
        “我不想再见到你们了。”他说。 
        “我也是。”助手之二虽然庆幸自己的未婚妻没被老师朝思暮想,但他后怕。他不愿意再给这种人当学生了。 
        面对高徒毁约,动物学家无可奈何。 
        “希望您不要将今天的事说出去。会给咱们带来灾难的。”助手之一警告老师。他了解老师,他担心老师已经有了一把年纪,打这张牌在自己的生命尽头出名。 
        动物学家使劲儿点头,像孙子对爷爷承诺那样。他觉得自己欠助手之一,尽管他连一个指头都没有碰过助手之一的未婚妻。真要碰了也算是个男人。他这样做更脏,人所不齿。 
        三位专门研究动物的人不约而同地决定继续研究动物。在人类比动物虚伪时研究动物,是学先进。 
        五角飞碟回到皮皮鲁家,鲁西西将舒利放在手掌上,一阵慰问。 
        “你很勇敢。”鲁西西夸舒利。 
        “主要是有五角飞碟撑腰。”舒利说实话。 
        “舒利考试全校第一。”贝塔从五角飞碟里探出头来。 
        “你怎么知道?”皮皮鲁问。 
        “我用五角飞碟遥测的。”贝塔说。 
        “人类这么聪明,却想不出好的测试人的聪明程度的方法。考试这种方法可真够蠢的。”舒利说。 
        鲁西西看着手掌上的舒利出神儿。 
        “你怎么了?”皮皮鲁问。 
        “我想起了罐头小人。”鲁西西说。 
        “罐头小人?”皮皮鲁兴奋。 
        “什么罐头小人?”舒利坐在鲁西西的手掌上问。 
        “我和皮皮鲁小时候碰到的一件事,有意思极了。”鲁西西继续出神儿地看着舒利。 
        (关于罐头小人,请参见学苑出版社出版的《鲁西西传》,各地书店有售) 
        “罐头小人?是像罐头那么大的人?”舒克问。 
        “比罐头小多了,只有火柴棍那么高。”鲁西西说。 
        “比我们老鼠还小!”贝塔从五角飞碟里钻出来.他显然对罐头小人感兴趣。 
        “怎么从来没听你说过?”舒克问皮皮鲁。 
        “我们小时候,有趣的经历多了,怎么可能一个一个告诉你呢?”皮皮鲁在怀念自己的童年。 
        “以后没事时,讲给我们昕吧。”舒利要求。 
        鲁西西点头。 
        “先说说罐头小人是怎么回事?”叭塔急不可待,他一听说地球上还有比他小的人,就来劲了。 
        鲁西西坐在沙发上,皮皮鲁打开冰箱给她开了一筒饮料。 
        舒克坐在皮皮鲁膝盖上。 
        贝塔靠在五角飞碟上。 
        “大概是在我12岁那年……”鲁西西说。 
        “不,好像是11岁。”皮皮鲁插话。 
        “反正就是十一二岁吧。”鲁西西说,  “有一个星期日,家里来了客人,妈妈让我帮她打开一筒肉罐头。我把罐头拿回自己的房间,用罐头刀开启它。” 
        “其实我最爱开罐头,可妈妈不让我开,怕我开完了就先吃掉一半。”皮皮鲁如今已是四十多岁的人了,说这话时,还像个孩子。 
        “我漫不经心地打开了这筒肉罐头,当我掀开罐头盖往里看时……”鲁西西制造了一个停顿。 
        舒克、贝塔和舒利急了:“怎么了?快说呀!” 
        “罐头里边没有肉,是5个luoti(被禁止)小人,火柴棍那么大。”鲁西西脸上的表情和30年前刚发现罐头小人时一样。 
        “全是男的?”贝塔一听是luoti(被禁止),马上想到性别领域。 
        “三男二女。”鲁西西说。 
        贝塔吹了声口哨。 
        “罐头里怎么会有小人呢?里边的肉呢?”舒利问。 
        “不知道。这个谜应该让皮皮鲁研究一下。”鲁西西说,“正在我惊讶的时候,妈妈在厨房喊我,让我把罐头拿过去。” 
        “你把这个发现告诉妈妈了吗?”舒利问。 
        “我正想这么做。可当我拿着罐头走到房间门口时,我站住了。”鲁西西说。 
        “怎么了?”贝塔嫌鲁西西讲述的速度太保守。 
        “我突然想起,爸爸妈妈特反对皮皮鲁养小动物,他们扔过哥哥养的不少小动物。”鲁西西看看皮皮鲁。 
        “没错,他们觉得养宠物影响学习,思维特怪。”皮皮鲁证实。 
        “那你可不能把罐头小人交给妈妈。”舒利担心了。 
        “所以我把罐头小人藏起来了。只把空罐头盒给了妈妈。”鲁西西说。 
        舒克、贝塔和舒利松了一口气。 
        “鲁西西要倒霉了。”皮皮鲁到今天还对妹妹倒霉幸灾乐祸。 
        “怎么会呢?”舒利不明白。 
        “妈妈让她开罐头,她交给妈妈一个打开的空罐头盒,妈妈能干吗?”皮皮鲁说。 
        “妈妈一看罐头盒里是空的,就问我里边的肉哪里去了,我说打开就没有。妈妈说她忙着呢,让我别开玩笑了。我说我没开玩笑,这罐头里边真的没有肉。妈妈不干了,叫来了爸爸,后边发生的事就可想而知了。”鲁西西说。 
        “他们搜查你的房间了?”贝塔清楚人类的父母有个习惯,爱搜查儿女的物品。 
        “搜了,可是没找到。他们以为我把肉藏在哪儿了。”鲁西西得意地说。 
        “后来鲁西西可背了不少黑锅,受了冤枉。”皮皮鲁说。 
        “想像得出。”舒克说。 
        “罐头小人会说话吗?”贝塔问。 
        “何止会说话,智商还相当高。他们还各有职业呢!”鲁西西说。 
        “都于什么?”舒利问。 
        “一个是军人,叫少校。一个是博士。还有一个叫约翰,外语说得特棒。这三个是男士。两位小姐一个是歌唱家,一个是艺术家。”皮皮鲁说。 
        “他们一直不穿衣服?”贝塔爱浮想联翩。 
        “我给他们做了衣服。”鲁西西说。 
        “原来鲁西西从小就显示出非凡的服装设计天才,要不今天怎么能设计出受欢迎的皮皮鲁牌服装呢?”舒克恍然大悟。   第174集 
        舒克不敢写《丑陋的老鼠》这本书; 
        贝塔每天要赞扬别人三次以上; 
        振奋人心的决定   
        “开始,我对皮皮鲁也瞒着罐头小人。”鲁西西说,“所以我是孤军奋战。” 
        “我凭直觉感到鲁西西在罐头这件事上挺蹊跷,当时我偷偷在外边养了一条狗,叫福尔摩斯,我还让福尔摩斯帮我侦破这个谜。”皮皮鲁喝了口饮料,脸上都显得甜蜜。 
        “你俩应该结成同盟军。”贝塔说。 
        “后来结盟了。”鲁西西将舒利换到另一只手掌上。“罐头小人给我们的生活增添了好多乐趣。博士发明了一种便捷的学习方法,我和皮皮鲁采用这种方法上学,不费吹灰之力就能考100分。” 
        “约翰帮鲁西西说外语,震了一个外国教育考察团,校长当时都傻了。”皮皮鲁说。 
        “艺术家雕刻的微雕作品轰动了世界。”鲁西西说。 
        “什么叫微雕?”舒利问。 
        “就是在很小的东西上雕刻。’鲁西西解释.“那次,艺术家在一颗大米粒上雕出了万里长城全景图。” 
        舒利吐舌头。 
        “罐头小人后来去哪儿了?”舒利想知道罐头小人的结局。 
        “有一天,一群老鼠啃我爸爸的书,被少校发现了,少校和它们交战,负了重伤。”鲁西西说。 
        “我们的同胞太应该提高素质了。”舒克为自己的同胞在30年前伤了少校感到遗憾。 
        “你应该写一本书。书名就叫《丑陋的老鼠》,印发给老鼠家族的每一位成员。”贝塔对舒克说。 
        “老鼠尽管丑陋,可没有一只老鼠认为自己丑陋。这正是老鼠最丑陋之处。如果我写这样一本书,同胞们非吃了我不可。”舒克不敢写。 
        “越是丑陋的东西,越怕别人说它丑陋。越是完美的东西,越爱说自己不完美。”贝塔总结道。 
        “真正高档次的生命,每天都会赞扬别人3次以上,自嘲两次以上。凡是乐于赞扬别人和敢于自嘲的人都是伟人。”皮皮鲁说。 
        “很对,老说别人的缺点和自己的优点的人,都是骨子里自卑的人。”鲁西西投赞成票。 
        “我以后每天赞扬别人3次以上。”贝塔一本正经地宣布。 
        “你最好先从每天自嘲3次以上开始。”舒克对贝塔说。 
        “少校的伤治好了吗?”舒利问。 
        “治好了,许多名医给他动的手术。”鲁西西说。 
        “名医?人类的医生?’’舒克不大相信人类的医生会给这么小的人动手术。 
        “少校他们暴露了?让别人知道了?”贝塔为罐头小人的命运担忧。 
        “后来爸爸妈妈和学校的老师都知道了,他们理解了我们,对罐头小人也不错。”鲁西西说。 
        “真不容易。”贝塔长出了一口气。 
        “再后来呢?”贝塔刨根问底。 
        “有一天下午.我自己在家,有人敲门。”鲁西西说,“我开门一看,是一位非常漂亮的小姐。她说她是记者,她听了罐头小人的事,来采访我。她还说她和一位有童话大王之称的作家是特好的朋友,她想让他把这件事写成童话。” 
        “你答应了?”舒利问。 
        鲁西西点点头。 
        “后来,电视台的导演也来了,要把罐头小人的事拍成动画片,片名就叫《鲁西西奇遇记》。经他们这么一煽乎,罐头小人就名扬四海了。”鲁西西说。 
        “麻烦也就来了。”皮皮鲁皱起了眉头,“先是登门要求见罐头小人的,每天有上万人在我们家门口排队。接着有关部门提出要我们交出罐头小人,说是要研究他们,弄得我们家不得安宁,同时也为罐头小人的命运担心。” 
        “可不能交,一交他们就完了,和蹲监狱差不多。”贝塔说。 
        “那叫软禁。”舒克说。 
        “蹲监狱叫硬禁吗?”舒利联想力极强。 
        “差不多。”舒克说。 
        “那你们怎么办?”贝塔面部表情开始严肃。 
        “一天晚上,我们在家和五位罐头小人开了一个会,商讨对策。”皮皮鲁说。 
        鲁西西站起来,在屋里踱着步:“罐头小人认为他们给我们家带来了麻烦,他们挺过意不去。他们想走。” 
        “他们特别怕失去自由,他们认为不能主宰自己命运的人是可悲的。”皮皮鲁说。 
        “他们想去哪儿?”舒克问。他觉得罐头小人在这个星球上生存挺难。 
        “他们不想去一个地方,各有各的目标。”鲁西西说。 
        “分开更危险,”贝塔说。 
        约翰想去国外,大概会点外语的人都巴不得天天和操持这种外语的人对话。少校要去军事学院深造。”皮皮鲁说。 
        “这不和舒利想上人类的学校性质差不多吗?哪家学院会收他?”舒克说。 
        “歌唱家要求去贝多芬的故乡。艺术家想去大自然寻找灵感。”鲁西西说。 
        “博士呢?”舒利问。 
        “博士认为人类的学校采用的教育孩子的方法太愚蠢,他想一个学校一个学校去推广他的高明而又简捷的学习方法,让每个孩子在童年既玩得开心,又能学到知识。” 
        “真了不起。”舒利说,“你们同意了?” 
        “我们应该尊重他们的选择,这还是个人权问题。”皮皮鲁说。 
        “爸爸专门为这件事出了趟国,把约翰送到了美国,把歌唱家送到了贝多芬的故乡——德国。”鲁西西说。 
        “妈妈把少校送到了一家军事学院。”皮皮鲁说。 
        “军事学院收留少校了?”舒利不信。 
        “哪儿会收呀!妈妈把少校放在校园里了。”皮皮鲁说。 
        “我和皮皮鲁把艺术家送到黄山,她兴奋极了。博士自己走了。”鲁西西怅然若失地说。 
        “他们5个人现在还活着吗?”舒利问。 
        皮皮鲁和鲁西西摇头:“不知道。” 
        “他们的经历一定很曲折,也很有意思。”贝塔说。 
        沉默。 
        “我有个建议。”贝塔说。众人看贝塔。 
        “咱们去找他们!肯定还有活着的!”贝塔说出了石破天惊的话。 
        “找他们?!”皮皮鲁和鲁西西脱口而出。 
        “对,找他们!!”贝塔再次宣布自己的伟大建议。 
        贝塔的建议太有诱惑力了。 
        皮皮鲁和鲁西西对视。两个人的右手手掌对拍。他们同意了。 
        舒克、贝塔和舒利欢呼。   第175集 
        五角飞碟遥感罐头小人; 
        皮皮鲁决定先找歌唱家; 
        贝塔掷硬币; 
        外国的母蚊子生孩子时怕公蚊子骚扰吗   
        在地球上寻找五个没有地址的人类正常成员都很困难,何况是五个罐头小人,而且还是30年前的线索! 
        皮皮鲁认为,首先要判断五个罐头小人是否还活着。 
        舒克提出用五角飞碟遥感。 
        “30年前的残留信息,不知五角飞碟能否遥感到?”皮皮鲁不敢肯定。 
        “再说罐头小人那么小,五角飞碟的遥感系统未必能感应。”鲁西西也担心。 
        “我去试试。”贝塔钻进五角飞碟。 
        舒利随后跟进去,她想最先得到罐头小人的信息。 
        贝塔打开五角飞碟的遥感系统。 
        输入和罐头小人有关的资料。 
        屏幕上开始出现各种信息,经过一番去粗取精,贝塔捕捉到了几个有价值的数据。 
        舒克问:“罐头小人还活着吗?” 
        贝塔说:“活着。但是他们太小了,而且时间太久,信息很微弱,只能证明还活着,其他的就看不出来了。” 
        舒利跑出五角飞碟。 
        “怎么样?”鲁西西迫不及待地问。 
        “罐头小人还活着!”舒利兴奋地说。 
        “他们还活着!”皮皮鲁大喊一声。 
        “他们现在干什么呢?”鲁西西问贝塔。 
        “信号很弱,只能表明还活在地球上。”贝塔说。 
        “这就够了,咱们去找他们!”皮皮鲁的手臂在空中使劲儿一挥。 
        “五个人在五个不同的地方,先找谁?”鲁西西说。 
        “先找歌唱家。”皮皮鲁说。 
        “歌唱家最远,在国外。”鲁西西说。 
        “所以先找她。”皮皮鲁推开窗户,望着窗外说。 
        “制定计划吧。”贝塔提议。 
        “我和舒克出国去找歌唱家,鲁西西留守主持舒克贝塔公司的工作,贝塔和舒利协助鲁西西,五角飞碟给你们留下。”皮皮鲁宣布方案。 
        贝塔一听就急了。 
        “就你们两个满世界找罐头小人?也太势单力薄了。罐头小人那么小,必须具备和他们体积差不多的雄厚力量,光靠舒克可达不到。”贝塔说。 
        “你说怎么办?”舒克问贝塔。 
        “咱们三个去。”贝塔指指皮皮鲁和舒克,“再带上五角飞碟。” 
        “两位女士留守?”皮皮鲁问。 
        “随时保持联系,一旦家里出了意外,我驾着五角飞碟立马就回来了。”贝塔说。 
        “我觉得贝塔的话有道理,现在世界上复杂得很,国外还有黑手党光头党什么的,你们应该带上五角飞碟。”鲁西西赞成贝塔的提议。 
        皮皮鲁犹豫不决。 
        “掷硬币。”儿塔找出一枚硬币。 
        掷硬币的结果是皮皮鲁、舒克和贝塔带五角飞碟去找歌唱家,鲁西西和舒利留守。 
        “上帝最英明。”贝塔得意地说。 
        “就这么决定了。”皮皮鲁表态。 
        “什么时间出发?”鲁西西问。 
        “后天。”皮皮鲁说。 
        “预计要花多长时间才能把五个罐头小人都找到?”鲁西西问。 
        “最快也得五个月。”皮皮鲁想了想,说。 
        “我有个建议。”贝塔说。 
        “你的建议太多。”舒克说。 
        “我觉得皮皮鲁岁数也不小了,忙完这事,也该操办婚事了吧?”贝塔早就想说这事。他还觉得鲁西西也该将此事提上议事日程了,可他不好意思说鲁西西。 
        “爱因斯坦给婚姻下过一个定义:试图把一个偶发事件变成持久关系的徒劳之举。”皮皮鲁说。 
        “不愧是相对论的发现者,看问题一针见血。”鲁西西说。 
        “从明天开始,做出发的准备工作。”皮皮鲁宣布。 
        晚上,舒克,舒利和贝塔兴奋得睡不着觉,他们躺在五角飞碟里聊天。 
        舒利打了个哈欠:“我真想和你们一起去,我还没有出过国呢!,’ 
        贝塔:“咱们出国还不容易?又不要护照不要签证什么的。其实还是当动物好,比如说一只苍蝇想出国,只要往飞机里一藏不就出去了吗?人类也忒惨了点儿。出趟国就像换心换肝一样麻烦。” 
        舒克:“皮皮鲁从小就最怕蚊子咬,听说外国蚊子特厉害,身上都是花颜色的,像他们的战斗机穿迷彩服一样恐怖。咱们得给皮皮鲁准备点儿防蚊子的东西。” 
        舒利:“昨天我看电视,说是有一家公司生产了一种叫做振动防蚊盒的新产品,像BP机那么大,挂在腰间,蚊子就不会咬了。” 
        贝塔:  “不大可能,蒙事的。” 
        舒克:“什么原理?” 
        舒利:  “说是咬人的蚊子都是雌蚊子,雄蚊子不咬人。而雌蚊子也是在生小蚊子时才咬人吸血。因此,雌蚊子在繁殖期间,有尽量避开雄蚊子性骚扰的习性。利用这一原理,振动防蚊盒模仿雄蚊子飞行时的振动频率,雌蚊子闻声便会躲避。” 
        贝塔:“有意思。咱们去给皮皮鲁弄一个振动防蚊盒吧,让他挂在腰间,出国省得挨外国母蚊子咬。” 
        舒克:“外国的母蚊子和咱们国家的母蚊子不会习性止好相反吧?” 
        贝塔:“反正外国够开放的,蚊子没准也一样,才不管是不是生孩子期间呢。” 
        舒利:“如果习性真是相反,皮皮鲁挂上振动防蚊盒得招来多少母蚊子呀?” 
        舒克、贝塔和舒利经过七嘴八舌的探讨,一致认为既然是蚊子,就有共性。 
        “咱们去给皮皮鲁弄一个防蚊盒”。贝塔提议。 
        “什么时候?”舒克问。 
        “就现在。”贝塔说。 
        “怎么去?”舒克问。 
        “开五角飞碟呀!”贝塔说。 
        “不行。你别给皮皮鲁惹祸了。”舒克坚决否定。 
        “那咱们自己去。”贝塔说。   第176集 
        纸箱子里传出的声音; 
        贝塔和舒利探险; 
        图钉受伤; 
        舒克贝塔和舒利赛跑   
        舒克、贝塔和舒利决定利用夜色外出给皮皮鲁弄振动防蚊盒,此时是深夜0点25分,皮皮鲁正在熟睡中。 
        贝塔在五角飞碟里遥感出研制防蚊盒的公司的方位。 
        舒克、贝塔和舒利蹑手蹑脚溜出皮皮鲁家。 
        “好久没有走路下楼了。”贝塔一边下楼一边伸胳膊踹腿。 
        “老不运动,出门就是五角飞碟,骨头都退化了。”舒克说。 
        舒利更是兴奋,这是她第一次步行接触这个世界。 
        在最后一层楼梯的拐弯处,堆放着几个纸箱子。 
        当舒克他们经过纸箱子时,贝塔昕到纸箱子里有声响。 
        “嘘——”,贝塔站住了,“你们昕!” 
        舒克和舒利屏住呼吸听,纸箱子里果然有响动。 
        贝塔走近纸箱子,将耳朵贴在箱壁上听。 
        舒利也凑上去昕。 
        “好像是我们的同胞。”贝塔小声对舒克说。 
        “他们干什么呢?”舒克问。 
        “好像在搬运东西。”贝塔根据声音判断。 
        “我想见同胞。”舒利说。 
        “咱们还是去办正事吧。”舒克催促道。 
        “我进去看看他们于什么呢?”贝塔止不住好奇。 
        “我和你去。”舒利和贝塔叔叔站在一边。 
        舒克只得同意。 
        贝塔绕着纸箱子转了一圈,发现一个洞。 
        “就从这儿进去。”贝塔回头对舒利说,  “你在这儿等着。” 
        “不,我跟你进去。”舒利不干。 
        贝塔回头看舒克。 
        “让她跟你去吧。”舒克鼓励女儿冒险。 
        贝塔冲舒克翘了翘大拇指,转身钻进纸箱子里。舒利随后钻进去。舒克在外边等着。 
        纸箱子里乱七八糟地堆放着废报纸和说不出名字的破烂,贝塔好久没到过这么脏乱的地方了,舒利更是破天荒头一回。 
        “地球上还有这么脏的地方!”舒利皱着眉头说。 
        “这才是咱们老鼠本该呆的地方。你如果不是舒克的女儿,也会生活在这种地方。嗨,干什么也不如投个好胎。”贝塔小声说。 
        “声音好像没有了。”舒利提醒贝塔。 
        贝塔听了听,剐才的声音果然没有了。 
        “当心,说明它已经发现咱们了。”贝塔告诫舒利。 
        舒利觉得很刺激。她跟在贝塔后边往前走。 
        绕过一摞旧书,贝塔眼前出现了一个小空间,这里有明显的收拾过的痕迹,一个小酒盅倒扣着当茶几,几缕棉花铺成了一张简陋的床……一切都表明,这里有生命居住。 
        贝塔从气昧儿上判断,这个小空间的主人是他的同胞。 
        “出来吧,我们也是老鼠。”贝塔大声说。 
        没有回声。 
        贝塔朝一块木头走去。 
        木头后边藏着一只老鼠。 
        “你好!”贝塔见到同胞,感到亲切。 
        “你…是……谁……”同胞显然有些怵贝塔,他还没见过穿衣服的同胞。 
        “是朋友,别怕,咱们都住一座楼。”贝塔说。 
        “你们也住在这座楼里?”同胞的语气里少了几分恐惧。 
        “这里是你的家?”舒利问。 
        “嗯……”同胞迟疑了一下,承认道。 
        “就你自己住这儿?”舒利问。 
        “我在这儿养伤。”同胞一瘸一拐地从木头后边走出来。 
        一只挺精神的老鼠小伙子。 
        “你怎么受的伤?”舒利觉得他挺潇洒。 
        “被一只大老鼠咬的,他很凶,这一带的老鼠都怕他。”小伙子说。 
        “你吃什么?”贝塔问。 
        “已经几天没东西吃了.刚才正在啃这块木头。”小伙子说。 
        舒利不知道世界上还有处于如此艰难环境中的生命。 
        “你叫什么名字?”舒利说。 
        “图钉。” 
        “图钉?这名字挺绝。”舒利说。 
        “我小时候身子瘦,头大,我妈就给我起了这么个名字。”图钉说。 
        “我去给你弄点吃的。”舒利不忍心看受伤的同胞挨饿,何况还是个不错的异性同胞。 
        “咱们得先去给皮皮鲁弄防蚊盒,回来再给图钉送食物吧!”贝塔提醒舒利。 
        舒利点点头。 
        “在你受伤期间,你的食物我包了,每天给你送来。”舒利对图钉说。 
        “真的?”图钉有生以来头一次从同胞处得到温暖。 
        贝塔和舒利离开图钉的住处。 
        贝塔问舒利:“我看你是爱上他了。” 
        舒利说:“有点儿。” 
        贝塔钻出纸箱子后对舒克说:“恭喜你,快当老丈人了。” 
        舒克莫名其妙。 
        舒利把经过讲给舒克听。 
        舒克觉得女儿挺有同情心,也特希望女儿有异性朋友。 
        “咱们快去给皮皮鲁弄防蚊盒吧?”贝塔说。 
        舒克、舒利和贝塔离开楼房,来到大街上。 
        街上几乎没有人,只是偶尔有几辆出租车慢悠悠地转悠,等候客人。 
        “生产防蚊盒的公司在哪儿?”贝塔问舒利。 
        “电视上说,在和平南路。”舒利回忆。 
        “那离这儿不远。”舒克说。 
        “好久没运动了,咱们赛跑吧!”贝塔提议。 
        “和平南路在那边。”舒克指出终点的方向。 
        舒克、贝塔和舒利都做跑步的准备。 
        “预备——跑!”舒利说。 
        舒克、贝塔和舒利拼命往前跑。 
        舒克和贝塔毕竟岁数大了,他们很快就被舒利甩在了后边。   第177集 
        舒利被困; 
        白猫不想当太监; 
        贝塔和白猫的距离; 
        生死搏斗; 
        舒利给图钉送饭   
        当舒克和贝塔气喘吁吁地跑到那家公司门前时,舒利没有在那儿等他们。 
        “我们输了,舒利,你出来吧。”舒克以为女儿藏起来逗他们玩。 
        贝塔一边擦汗一边漫不经心地环顾四周。 
        “舒利,舒利。”舒克连叫了两声。 
        舒利没出现。 
        突然,一种不祥的预感让贝塔打了个冷战。 
        “好像不大妙!”贝塔说。 
        “你看见什么了?”舒克忙问。 
        “什么也没看见,感觉不妙。”贝塔又打了一个冷战。 
        “你们别过来,这儿有猫!”不远处传来舒利的大喊。 
        舒克和贝塔顺着喊声看去,眼前的场面令他俩大吃一惊。 
        一只全身雪白的大猫虎视眈眈地蹲在公司门前的一个角落,舒利躲在一个死角里,她出不去,大白猫由于体积大也进不去。 
        舒克和贝塔好久不出来了,他们忽视了夜间正是猫们逛大街找对象的时间。 
        必须立即解救舒利。 
        “我把他引开,你去救出舒利”。贝塔说。 
        “当心!”舒克叮嘱贝塔。 
        贝塔绕到大白猫的另一侧,他故意发出声响,吸引大白猫注意他。 
        白猫是一只家猫,近一个星期以来,他强烈感到身体里有一股能量跃跃欲出,他坐卧不宁,急于施放这股能量。一到晚上,他就急不可待地溜到大街上,想向某只母猫求爱。在今天吃晚饭时,主人全家的一段对话引起了他的注意。男主人说,这猫闹得太厉害,满屋臊昧儿。女主人说,明天我带他去兽医院,给他做手术算了。小主人问,做什么手术?男主人说,让他当太监。小主人又问,什么叫太监?女主人说,当不了爸爸的男人就叫太监。小主人说,那我现在就是太监啦?女主人说,你还不到年龄。到了当爸爸的年龄当不了爸爸的男人就是太监,小主人又问,什么地方有太监?男主人说,皇后身边才有太监。小主人说,给白猫做了手术,妈妈就成皇后了?白猫渐渐听明白了,主人是要对他施行一种斩草除根式的手术,他们要让他成为没有根的树,没有热量的太阳。他决定出走,他想保住自己的根本,可他无法想像自己怎样在野外生存。最后,他选择了一个折衷的方案,今晚一定要当一回真正的公猫。就像小主人虽然是男孩儿但却从未行使过男人的权力一样,他要在今晚完善自己,明天听任主人摆布,他离开这个家无法生存,他愿为此付出一切代价。主人人睡后,他就来到大街上寻觅能使他成为真正的公猫的目标,好在猫的世界尚无法律,没那么多麻烦,碰上谁是谁。白猫看见了舒利。他从没见过老鼠,但祖先留给他的遗传基因促使他凶猛地扑向舒利,完全是一种本能。舒利也从未见过真猫,她是从电视和爸爸口中知道猫的。她躲进了墙角的一个死角。 
        白猫不放过舒利,他蹲守在死角的出口。 
        贝塔的喊叫引起了白猫的注意。 
        他看到又有一只老鼠在招呼他。他清楚,只要他去抓那只老鼠,躲在墙角的这只就跑了。 
        他不上当。 
        任凭贝塔狂呼乱叫,白猫就是不动窝。 
        贝塔只好一步步向白猫靠近。 
        “不能再靠近了!”舒克警告贝塔。 
        白猫发现自己身后还有一只老鼠! 
        贝塔距离白猫只有两尺近了。 
        这个距离还在缩短。 
        白猫突然调转身体,他扑住了贝塔。 
        贝塔没来得及跑,他得感谢白猫的主人,他们把白猫的指甲都给剪掉了。 
        舒克顾不上舒利了,他冲到白猫身后,咬他的后腿。 
        白猫一抬腿,将舒克踢翻在地。 
        舒利跑出来支援爸爸,她也被白猫踢倒了。 
        三只老鼠和一只猫搏斗。 
        白猫占了明显的上风。 
        “舒克,快去开五角飞碟!”贝塔急中生智。 
        一句话提醒了舒克,他转身就跑。 
        白猫好像知道五角飞碟的厉害,他突然放了贝塔。 
        贝塔边朝舒克跑边喊舒克。 
        舒克、舒利和贝塔站在一个安全地段回头看白猫。 
        白猫看见了房顶上的一只母猫,他顾不上贝塔了,他想要那只漂亮的母猫。 
        舒克他们目睹了白猫征服母猫的全过程,猫的呐喊划破了夜空,显示着创造生命的骄傲。 
        “在那一瞬间,他们就是上帝。”舒克意味深长地说。 
        “是她救了咱们。”舒利感谢那只母猫。 
        “我觉得那白猫的喊叫像绝唱。”贝塔一副不屑一听的表情。 
        舒利突然撒腿就往回跑。 
        “舒利,你干吗?”舒克急了。 
        舒利不理,继续跑,贼快。 
        舒克和贝塔对视了一眼,贝塔说:“快追!” 
        舒克和贝塔追舒利。 
        等到舒克和贝塔跑到皮皮鲁家门口时,看见舒利拿着一包食物正从门底下钻出来。 
        “你干什么?”舒克不高兴了。 
        “我给图钉送吃的。”舒利径直下楼。 
        “你怎么连个招呼也不打,我们还以为你疯了。”舒克喘着气训斥女儿。 
        “也快当上帝了。”贝塔坐在门旁擦汗。 
        舒利冲贝塔一笑,下楼了。 
        “我这个侄女也是个人物。”贝塔说。 
        “得,振动防蚊盒也忘了弄了。”舒克拍脑袋。 
        门开了,皮皮鲁探出头。 
        “你们在这儿干什么?”他问。 
        舒克和贝塔进屋把经过告诉皮皮鲁。 
        “太危险了。”皮皮鲁挺后怕。 
        “白天你自己去买个振动防蚊盒吧?”贝塔打了个哈欠。 
        “舒利呢?”皮皮鲁问, 
        “扶贫去了。”贝塔说。 
        “扶贫?”皮皮鲁听不懂。 
        舒克把图钉的事告诉皮皮鲁。   第178集 
        贝塔担任首次值班任务; 
        安全检查对五角飞碟发生兴趣; 
        迷人的小姐纠缠皮皮鲁; 
        太太要给皮皮鲁名片   
        第二天,皮皮鲁上街买了一个振动防蚊盒,他还去航空公司售票处买了次日飞往德国柏林的机票。 
        当舒克和贝塔知道第二天就要启程去找罐头小人时,非常兴奋。 
        鲁西西帮助皮皮鲁收拾行装。 
        皮皮鲁专门为五角飞碟准备了一个小皮箱,体积比五角飞碟稍大一些。 
        “你们是呆在五角飞碟里还是呆在我身上?”皮皮鲁问舒克和贝塔。 
        “最好轮流值班,一个呆在你身上,一个呆在五角飞碟里,省得有情况时来不及。”贝塔提议。 
        “遇到紧急情况,你要立即打开箱子。”舒克告诫皮皮鲁。 
        “一般情况不打开也行,五角飞碟呆在箱子里也能显示威力。”贝塔说。 
        “这倒是。”皮皮鲁点头。 
        “舒利,你和鲁西西在家要当心,去给图钉送饭时更要提高警惕。”舒克叮嘱女儿。 
        “放心吧!”舒利说。 
        皮皮鲁向鲁西西交待公司的业务,鲁西西认真地往笔记本上记。 
        当晚,鲁西西掌勺,做了一顿丰盛的晚宴。为皮皮鲁、舒克和贝塔饯行。 
        皮皮鲁和鲁西西坐在餐桌旁。舒克、贝塔和舒利坐在餐桌上。 
        “祝你们一路平安!”鲁西西祝酒。 
        贝塔喝干了一杯酒,又给自己斟上一杯。 
        皮皮鲁也喝完了杯中的酒,他挺激动,很长时间没有出国了,皮皮鲁渴望旅行,渴望新奇的生活。 
        舒克也在家里呆烦了,早就想到外面的世界逛逛。 
        晚宴持续到ll点。 
        喝得醉醺醺的贝塔钻进五角飞碟倒头便睡。 
        第二天上午9点,皮皮鲁、舒克和贝塔准备出发。 
        舒克和贝塔都想呆在外边看风景。 
        “今天你们谁在五角飞碟里值班?”皮皮鲁问舒克和贝塔。 
        掷硬币决定。 
        贝塔运气不佳。 
        “我值吧。”贝塔边说边朝五角飞碟走去,“我给你们当保镖。” 
        舒克呆在皮皮鲁的衣兜里,鲁西西专门给那个特制的衣兜安装了能望到外边的小孔。 
        鲁西西驾车送他们去国际机场。 
        舒利同大家道别。 
        国际机场车来人往,繁忙得让人头晕,也不知怎么有那么多人在家呆不住。 
        皮皮鲁有两件行李,一件是装五角飞碟的手提箱,一件是装衣物的大箱子。 
        警察不让鲁西西在候机大厅门口停车。 
        “你回去吧,咱们随时电话联系。”皮皮鲁一边从汽车的后备箱里往外拿箱子一边对鲁西西说。 
        “到了柏林给我来个电话。”鲁西西朝皮皮鲁摆摆手,开车走了。 
        皮皮鲁拎着箱子走进候机大厅。 
        电子显示屏幕上不时变换着不同排列组合的那10个人类百用不厌的阿拉伯数字。喇叭里传出进港出港班机的信息。 
        通过登机安全检查时,警察注意到皮皮鲁手中的手提箱。 
        “请您打开它。”警察对皮皮鲁说。 
        皮皮鲁迟疑了一下,打开手提箱。 
        “这是什么。”警察指着五角飞碟问。 
        “儿童玩具。”皮皮鲁说。 
        警察拿起五角飞碟,仔细察看。 
        “这么好的箱子就为了装这么个玩具?”警察像是在问皮皮鲁,又像是自言自语。 
        “这个玩具很贵。”皮皮鲁解释。 
        警察还不放心,他叫来了上司帮他鉴定。他不想负责,他清楚,如果经他的手检查过的旅客劫了飞机,他就吃不成这碗饭了。所有机场安检人员都希望男女乘客分别乘坐飞机以便实行luoti(被禁止)乘机规定以确保飞机不被劫持或者在起飞前给乘客吃安眠药这样即使是再凶残狡猾的劫机犯也乖乖地睡到目的地,等他醒了后,随便在地面上怎么劫机都无所谓。 
        上司翻来覆去地看五角飞碟。 
        “这玩具怎么玩?”上司问皮皮鲁。 
        “电动玩具。”皮皮鲁说。 
        “电池呢?” 
        “太阳能电池。” 
        “遥控的?” 
       “声控。” 
        “能表演一下吗?”上司问。 
        “可以。”皮皮鲁从警察手中拿过五角飞碟,放在地上。 
        皮皮鲁拍了一下手。 
        贝塔操纵五角飞碟在地面上滑行。 
        皮皮鲁又拍了一下。 
        五角飞碟停住了。 
        “真好玩。”上司从地上拿起五角飞碟,还给皮皮鲁。 
        “您的腰带上戴着什么?”警察又有了新发现。 
        皮皮鲁掀开衣服。 
        “振动防蚊盒。”皮皮鲁从皮带上摘下振动防蚊盒递给警察。 
        警察检查后还给皮皮鲁。 
        “谢谢您的配合。”警察示意皮皮鲁已通过登机安全检查。 
        皮皮鲁拎着手提箱走进候机室,他找了一个空位子坐下。 
        候机室几乎坐满了等候乘飞机旅行的人,人们脸上的表情都挺复杂,聊天的人一脸的临终关怀。沉默的人像在构思遗嘱。 
        皮皮鲁的身旁还空着一个座位。坐在远处的一位小姐起身来到皮皮鲁身边。 
        “我可以坐在这几吗?”那位小姐问皮皮鲁。 
        皮皮鲁抬起头,他愣了一下,眼前这位小姐长得十分出众,面容姣好,亭亭玉立。 
        “请坐。”皮皮鲁说。 
        小姐在皮皮鲁身边落座。 
        “您去柏林?”小姐和皮皮鲁搭话。其实这个候机室里的人都将乘同一架飞机。 
        “是的。您呢?”皮皮鲁也不自觉地说废话。 
        “我也是。”小姐冲皮皮鲁嫣然一笑,迷人极了。 
        贝塔在五角飞碟里通过荧光屏将这个场面看得一清二楚,他看出这位小姐很喜欢皮皮鲁。 
        舒克也认定皮皮鲁走了桃花运。 
        “您的气质真棒。”小姐说。 
        皮皮鲁脸红了。 
        “比本届奥斯卡金像奖最佳男主角强多了。”小姐又说。 
        皮皮鲁感到不自在了。 
        贝塔差点儿吐了。 
        和皮皮鲁背靠背坐着的也是一位小姐。那位小姐回头对皮皮鲁说: 
        “先生,咱们能聊聊吗?” 
        皮皮鲁回头,又是一位妙龄女郎。 
        “我……”皮皮鲁想起了《聊斋》,他有点儿茫然。 
        “前往柏林的旅客请登机。”扩音器里传出女声 
         皮皮鲁赶忙站起来。 
        “我帮您拿箱子。”皮皮鲁身边的小姐说。 
        “不,不用,谢谢,我自己拿。”皮皮鲁还没见过小姐抢着为男士拿箱子的,何况还素不相识。 
        两位小姐簇拥着皮皮鲁往机舱门口走去,像两个跟屁虫。 
        飞机上的空中小姐一见到皮皮鲁就兴奋,几乎整个飞机上的所有空中小姐都抢着帮皮皮鲁找座位。 
        几乎旅客中的所有女性都要求将自己的座位调换到皮皮鲁身边,机舱里乱作一团。 
        男旅客们惊讶地注视着皮皮鲁。一位中年男子为此和自己的太太打了起来,因为那太太非要把自己的名片给皮皮鲁。   第179集 
        空中小姐失态; 
        皮皮鲁的脸上全是口红; 
        五名警察面对百名疯狂的女性束手无策; 
        总经理关心儿媳   
        皮皮鲁被飞机上的所有女性包围起来,有两位小姐甚至开始吻皮皮鲁。 
        机长急忙向塔台报告: 
        “1131航班向塔台报告,本机不能按时起飞。” 
        塔台: 
        “发生了什么故障?” 
        机长: 
        “机卜的所有女性乘客包括空中小姐都在纠缠一位男旅客,机舱内秩序大乱。” 
        塔台: 
        “你说什么?你喝酒了吧?” 
        机长: 
        “你们下来看看,真见了鬼了!” 
        塔台上的值班主任给机场保安部门打电话,5名警察赶到皮皮鲁乘坐的飞机上。 
        眼前的景象把警察吓坏了,飞机上的所有女性都在争夺皮皮鲁。皮皮鲁虽然是男性,但终因势单力薄寡不敌众而惨遭女性们的攻击,他的脸上布满了口红的痕迹。 
        男旅客们站在一边。全是惊讶和嫉妒。 
        “让开,让开!”警察们费力地往皮皮鲁身边挤。 
        皮皮鲁见到来了警察,像看见了救星,忙喊:“快救救我!” 
        “她们不认识您?”一警察发现皮皮鲁并非影星歌星,甚感纳闷,他问皮皮鲁。 
        “从来没见过,她们一上飞机就非礼我。”皮皮鲁大喊。 
        “都住手。”警察冲女士小姐太太夫人们厉声喝道。 
        女性们愣了一下,但马上又争先恐后地向皮皮鲁进攻。 
        5名警察将皮皮鲁团团围住。 
        一名空中小姐冲上来不顾一切地向皮皮鲁表白: 
        “你是我的白马王子,我爱你!没有你我活不成,我会从一万米高空跳卜去的!” 
        “你是机组人员,你要自重!”警察痛斥美丽绝伦的空中小姐。 
        “没有你我活不成……”空中小姐我行我素,依然爱得死去活来。根本不理睬警察的恫吓。 
        藏在皮皮鲁衣兜里的舒克差点儿被小姐们挤死,他极力躲避着小姐们身上的各个部位。包括最不应该躲避的部位他也必须忍痛割爱坚决躲避。 
        五角飞碟里的贝塔开始觉得这场面特好玩,渐渐地,贝塔感到不妙了。他意识到被许多女性在不同的时间不同的空间疯狂地爱着是一种享受,但被许多女性在同一时间同一空间疯狂地爱着绝对是灾难。 
        “怎么搞的?”贝塔自言自语地打开了五角飞碟里的遥感仪器。 
        真相大白了。 
        原来是皮皮鲁身上的防蚊振动盒在作怪。防蚊振动盒是通过模仿雄蚊子翅膀的振动频率来驱逐怀孕的雌蚊子的,不知雄蚊子和人类的男性成员有哪些共同之处,反正这防蚊振动盒将皮皮鲁身边的女性都吸引过来了。 
        看来地球上的所有生命都有共性,公的就是公的,母的就是母的。贝塔想。 
        必须想办法通知皮皮鲁关闭防蚊振动盒上的开关。 
        可现在皮皮鲁没有将通讯器插在耳朵上。 
        贝塔急得团团转,他出不去,装五角飞碟的箱子关得严严实实。 
        5名警察对付不了上百名发了疯的女性。 
        一位男旅客站到了座椅上: 
        “我提议让这位受欢迎受爱戴的男士下飞机,哪位小姐愿意跟他下去都可以,不能耽误我们的时间!” 
      “对,让他下去!” 
      “我同意!” 
      “下去!” 
      “下去!! 
        “他下飞机以后爱怎么着都行。” 
        早就被刺伤自尊的男旅客们像火山一样爆发了,他们用最损的话刺激皮皮鲁: 
      “那人准是流氓!” 
      “现在这女人真叫堕落了,男的不坏她们绝对不爱。” 
        “老实巴交的正人君子倒没人要了!” 
        “你看那监狱里放出来的,找的太太一个赛一个漂亮!” 
        “……” 
        “……” 
        警察小头目和机长商量了一下,觉得带皮皮鲁离机是上策。 
        “只好请您下飞机了,没办法。”机长耸耸肩,对皮皮鲁说。 
        “这和我没关系!我不离机!”皮皮鲁抗议。 
        “您不离开,她们就不会安宁。”警察指指躁动不安的女士们。 
        “你们干什么?我什么时候得罪你们了?”皮皮鲁突然冲女士们大吼。 
        “我—们-爱-你-”女性们异口同声。 
        “我从来没见过你们!”皮皮鲁嚷嚷。 
        “一-见-钟-情”女性们像士兵接受长官检阅那般整齐地回答。 
        “我讨厌你们!”皮皮鲁怒不可遏。 
        “我-们-喜-欢-你-”女性们决不退缩。 
        飞机已经推迟起飞半小时了。 
        航空公司总经理办公桌上的电话响了。 
        “你说什么?”总经理怀疑自己昕错了。 
        对方又重复了一遍。 
        “有这种事?”总经理对这1131航班印象很深,他的儿子的女朋友是这个航班的空中小姐。 
        “所有女士包括空中小姐都纠缠一位男旅客?”总经理皱起眉头,“xx空中小姐也在内吗?” 
        总经理点名问准儿媳的表现。 
        “她最凶,死抱着那男人的腰不放。还……” 
        “还什么?” 
        “还……” 
        “快说!” 
        “还当众亲那流氓?” 
        “什么,那人是流氓?” 
        “我看是。” 
        “他有什么流氓行为?” 
        “能吸引这么多女性,不是流氓是什么?” 
        总经理同意下属的这个观点。 
        “把他抓起来。”总经理指示。 
        “咱们恐怕没这个权力。”下属提醒上司别犯非法拘禁罪。 
        “让警方抓。” 
        “警方说只能将他带离飞机,还说其实犯法的是那些女性。” 
        “她们犯什么法?” 
        “性骚扰。” 
        “女人对男人进行性骚扰?”总经理在心里希望自己成为这种罪犯的袭击目标。 
        机舱里仍旧一片混乱。 
        皮皮鲁被5名警察簇拥着走向机舱出口。   第180集 
        男士的裤子造反; 
        皮皮鲁的一只脚伸出飞机; 
        空中小姐拒绝为皮皮鲁服务; 
        贝塔遇险   
        贝塔急了,他决定破釜沉舟,用五角飞碟上的武器击毁皮皮鲁身上的防蚊振动盒。 
        难度极高,防蚊振动盒佩挂在皮皮鲁的小腹附近,贝塔如果稍有闪失,将给皮皮鲁造成终生遗憾。 
        贝塔为慎重起见,要先找个靶子试试。 
        他选中了那个骂皮皮鲁最凶的男人,那人的皮带扣挺漂亮,是一个挺俗的名牌。 
        贝塔瞄准了那男人的皮带扣,按下了射击按钮。这一炮如果稍微打低点儿,那男人这辈子就前有古人后无来者了。 
        那男人看警察往出带皮皮鲁,兴奋得手舞足蹈,还边舞边喊:“快带走那流氓!” 
        他的裤子就掉了下来。 
        最先发现的一位小姐发出尖叫。 
        其余的异性顺声望去,尖叫声此起彼伏。 
        那男人忙提裤子,这才发现皮带已无法重新负担起保卫隐私部位的重任,他只好坐在椅子上,靠身体和大腿的90度角保持裤子应有的位置。 
        成功了,五角飞碟没有伤害到那男人的肌肤。贝塔赶忙掉转枪口。 
        这时,皮皮鲁已被警察带至舱门口。一群女性尾随其后。 
        “我抗议!”皮皮鲁不想离开飞机。 
        “请您谅解,否则飞机无法起飞。”警察劝皮皮鲁。 
        “这和我没关系!我是受害者,你们应该把她们带离飞机!”皮皮鲁说。 
        “我们和你在一起!”女性们异口同声向皮皮鲁表忠心。 
        防蚊振动盒发出的雄蚊波威力真大。 
        皮皮鲁叹了口气,摇摇头。 
        “你们爱我什么?你们了解我吗?”皮皮鲁双手拉住舱门框,做最后的努力。 
        “你是男人,我爱你是男人。我了解你是个男人,这就够了!现在世界上的男人有几个是男人?”航空公司总经理的未来儿媳回答。 
        “对,全世界的男人加在一起也没你有男人味儿!”女性们喊。 
        机舱里的男人们恨不得采用最不应该采用的方法立刻在女士们面前证明自己是男人。 
        “走吧。”警察往出推皮皮鲁。 
        贝塔按下了射击按钮。 
        皮皮鲁腰带上的防蚊振动盒粉碎了。 
        当皮皮鲁的一只脚跨出飞机时,他身后的追随者队伍发生了叛乱。 
        几乎所有的女性都像猛醒了似地往机舱里走,她们一边找自己的座位一边无地自容。 
        空中小姐更是无颜面对机组同事,有人失手打碎了暖瓶。 
        皮皮鲁伸出去的脚又收回到飞机里边。 
        “我不用下飞机了吧?”皮皮鲁问警察。 
        警察用步话机请示上司。 
        上司说让皮皮鲁回去试试。 
        皮皮鲁战战兢兢走到自己的座位旁,一路未受到性骚扰。 
        女性们表示对皮皮鲁不屑一顾。 
        皮皮鲁心中油然而升一种失落感。他这才知道,受到性骚扰的女性内心是喜悦的。真正不幸的人是在一生中从未受过性骚扰的女性。 
        警察向上司报告说机舱内恢复正常。 
        上司说皮皮鲁可以随机起飞。 
        机长和驾驶员们各就各位。空中小姐吩咐旅客系安全带。 
        皮皮鲁身旁的女性要求和男性换座位。两位男士一左一右地坐在皮皮鲁两侧。 
        皮皮鲁真后悔刚才自己太绅士,在遭到小姐们的亲吻进攻时还千方百计躲避,早知如此,当初真应该配合她们。 
        机长打开驾驶舱里所有应该打开的开关,同时向塔台报告: 
        “1131航班准备起飞。” 
        “1131可以起飞。”塔台批准。 
        1131的四台发动机开始旋转,它们产生的推力由小到大,大到可以把飞机推动时,飞机就滑上了跑道。 
        当飞机离开地面后,贝塔在五角飞碟里松了一口气。他特怕去不了柏林。 
        舒克在皮皮鲁的衣兜里揉着被挤疼的地方,他还想不通刚才的性骚扰事件是为什么。 
        皮皮鲁闭着眼睛回忆。 
        所有空中小姐都拒绝为皮皮鲁服务。 
        皮皮鲁原谅了她们。 
        经过10多个小时的飞行,飞机在柏林机场着陆。 
        皮皮鲁最后一个离开飞机,他不想和那些小姐太太一起走,他怕她们再失控。 
        机场海关检查皮皮鲁的护照和行李物品。 
        警察检查皮皮鲁的大箱子,又示意皮皮鲁打开手提箱。 
         皮皮鲁只得打开。 
        “这是什么?”外国警察也对五角飞碟感兴趣。 
        “玩具。”皮皮鲁说。 
        外国警察拿起五角飞碟,从各个角度观察它。 
        贝塔没系安全带,他在飞碟里摔了个跟头。 
        “妈的,这警察对外宾也太不礼貌了。”贝塔抓紧座椅的靠背,骂道。 
        警察记起上司最近提醒他们,说是有个国际恐怖组织近几天要带一批微型炸弹入境,上司吩咐警察们好生保卫祖国的大门。 
        “请您稍等。”警察对皮皮鲁说。 
        “为什么?”皮皮鲁不满。 
        警察耸耸肩膀,没答复皮皮鲁。他拿着五角飞碟走进旁边的一间屋子。 
        贝塔知道该进入戒备状态了,他坐在操纵台前,系好安全带。 
        警察将五角飞碟交给鉴定专家。 
        “鉴定一下,这是不是玩具。”警察对同事说。 
        专家接过五角飞碟,翻过来倒过去地看。 
        贝塔幸亏系好了安全带,好几次头朝下拿大顶。 
        专家拿仪器测五角飞碟。 
        “肯定不是炸弹。”专家说,“但也不是玩具。” 
        “那是什么?”警察特希望自己能发现异物。 
        专家摇头: 
        “这东西挺怪,表面连个螺丝也没有,怎么打开呢?玩具不会这么精致。” 
        “能让它入境吗?”警察问。 
        “让它的主人先走,把它留下来再检查一下,让它的主人明天来取。”专家说。 
        警察出来告诉皮皮鲁。 
        “这不行。”皮皮鲁坚决反对。 
        “很抱歉,我们不得不这样。”警察一脸的歉意。 
        皮皮鲁没办法,只好推着装行李的小车走出机场大楼。 
        贝塔和五角飞碟被扣留在海关。 
        “怎么办?”舒克在口袋里问皮皮鲁。 
        “我同贝塔联系一下。”皮皮鲁接通同五角飞碟保持联系用的微型通讯器。 
        一位服务生走过来问皮皮鲁要不要帮忙。 
        “谢谢,不用了。”皮皮鲁想找一个不引人注目的地方。 
        他将行李车推到一排椅子旁边。 
        皮皮鲁坐下来和贝塔通话。   第181集 
        德国牧羊犬怀疑五角飞碟; 
        四星级饭店的1610房间; 
        贝塔失踪; 
        皮皮鲁一口气喝冰镇饮料; 
        少女藏毒品   
        “贝塔,贝塔,我是皮皮鲁,你听见了吗?请回答!”皮皮鲁小声呼叫贝塔。 
        “我是贝塔,我怎么离开这儿?”贝塔请示皮皮鲁。 
        “他们还在琢磨五角飞碟?”皮皮鲁问。 
        “他们现在牵来了一条警犬,德国牧羊犬,他们让它嗅五角飞碟。”贝塔说。 
        “它有反应吗?” 
        “使劲儿叫,大概是闻到我的味儿了。” 
        其实五角飞碟可以轻而易举地摆脱海关的困扰,但皮皮鲁担心由此引起警方的注意,他不想惹麻烦。 
        “贝塔,你先呆在那儿,咱们随时保持联系。我们去找个饭店住下。我看白天五角飞碟不能在市区飞行,等到天黑了再说。”皮皮鲁说。 
        “你们先去吧,五角飞碟有我在,出不了事。”贝塔挺牛。 
        皮皮鲁推着行李来到机场候机大楼出口,一辆出租车开到他身边。 
        服务生帮皮皮鲁把行李装进汽车的后备箱。 
        “去哪儿?”司机回头问后座上的皮皮鲁。 
        “找一座饭店。” 
        “几星级的?” 
        “四星级。” 
        出租车驶入闹市。高楼大厦栉次鳞比,不同种族不同肤色的人川流不息,一个比一个显得忙.忙着得到想得到的东西。得不到就失落就急,得到了就先得意后空虚。得到的越多失去的越多,得到的越少失去的并不少。人类在地球上的不同空间不同时间前赴后继忙忙碌碌不遗余力地干着同一件事:从打生下来就都想比别人活得好,到死却都觉得活得不如别人好。没有贪婪和欲望,人类就永远不会前进。有了贪婪和欲望,人类就永远不会幸福。说到底.人类的每个成员都是被折磨死的。天折磨。地折磨。钱折磨。同类折磨。自己折磨自己。 
        皮皮鲁望着车窗外飞驰逝去的车水马龙,生出无限感慨,他已经好久没出国了,自从地震事件后,他几乎过着与世隔绝的生活。他现在懒得和人打交道。和人交往越多,他就觉得同胞越少。 
        前些年皮皮鲁参加世界物理学术会议时来过这座城市,几年过去,这城市又充实了许多也空虚了许多,皮皮鲁在惊叹人类的潜力的同时也为人类悲哀。 
        出租车停在一座富丽堂皇的大饭店门前,一位穿着元帅服的服务员给皮皮鲁开车门,他还把手掌插进车顶下边,怕皮皮鲁碰头,好像越有身份越有钱的人越蠢。 
        “好家伙,真漂亮!”藏在兜里的舒克通过他的专用窥视孔看到了饭店大厅的场面,他没想到人类这几年能把房子盖成这般豪华。 
        服务生用行李车将皮皮鲁的行李推进大厅,一位美艳无比的小姐向皮皮鲁问好。 
        皮皮鲁告诉她,他是来住宿的。 
        小姐将皮皮鲁带到住宿登记处。 
        为皮皮鲁办理住宿登记的两位小姐更迷人,舒克感到这饭店大厅里真是美女如云,他为皮皮鲁感到遗憾,他希望在飞机起飞前那场面应该在现在重演。舒克不知道是防蚊振动盒起的作用。 
        “您住160号房间。”小姐笑容可掬地告诉皮皮鲁。 
        皮皮鲁乘电梯到16层。从电梯里可以看到饭店外边.尽管城市的景色挺美,可皮皮鲁无心欣赏,他惦念着贝塔和五角飞碟。 
        1610房间是标准客房,两张席梦思床,卫生间、电视、电话、冰箱一应俱全。 
        当服务员关上房门离开后,舒克迫不及待地从兜里钻出来。 
        “闷坏了。”舒克在床上打滚,舒展筋骨。 
        皮皮鲁立即同贝塔通话。 
        “贝塔,贝塔,我是皮皮鲁,请回话。” 
        没有回答。 
        “贝塔!贝塔!你听见没有?请回话!”皮皮鲁声音有点儿变了。 
        舒克也停止打滚。 
        没有答复。 
        “贝塔!贝塔!……” 
        “贝塔!贝塔!……” 
        “贝塔!贝塔!……” 
        皮皮鲁感到口干舌燥,他打开冰箱,拿出一罐冰镇饮料,一饮而尽。 
        贝塔失踪了。 
        皮皮鲁和舒克走后,几位海关人员继续研究五角飞碟。 
        警犬冲着五角飞碟狂吠,引起了专家们的重视。 
        “有问题。”专家说。 
        “应该想办法打开它。”警察提议。 
        “这儿有个小门。”专家发现了舱门。 
        “不会是定时炸弹吧?”一位警官想像力丰富。 
        在场的所有人脸色都变了。 
        专家把耳朵轻轻贴上去。 
        “不像。”他说。 
        “里边好像有东西。”一位高鼻子警官也把耳朵贴在五角飞碟上。 
        “什么东西?”专家没听到。 
        “好像是活的东西?”高鼻子警官双手拿起五角飞碟,使劲儿摇晃。 
        贝塔的力气无法抵抗这地震式的大动荡,他的身体无数次地与舱壁相碰击,尽管舱壁是弹性材料制作的.可贝塔还是被撞昏了。 
        “你们听,里边确实有东西。”高鼻子警官摇给同事们听。 
        “一定要打开它!”专家开始找工具。 
        贝塔昏迷了,他听不到皮皮鲁的呼叫。 
        撬开五角飞碟的准备工作完成了,各种工具包括电钻和焊枪都摆在五角飞碟旁边。 
        突然,房间里的警报器响了,警察们蜂拥着跑出去。 
        这是发现有人携带违禁品入境的信号。 
        海关警察发现一个少女将可卡因藏在她身上最不容易被人发现的地方。 
        那少女间接救了贝塔,警官们都帮着同事处理这一案件去了,五角飞碟暂时没人管了。 
        贝塔仍然昏迷。 
        “咱们必须马上返回机场!”皮皮鲁对舒克说。 
        “走!”舒克二话没说,钻进皮皮鲁上衣口袋。 
        皮皮鲁在饭店门口上了一辆出租车,直奔机场。 
        在机场大厅的人口处,皮皮鲁小声对舒克说:“现在只有你能救贝塔了,你还认识那个房间吗?” 
        “认识。”舒克说。 
        “我尽量靠近那个房门,但也不能让他们看见我,否则该起疑心了。你自己想办法进到那个房间里。当然这很困难。”皮皮鲁看了看整洁如镜的大厅地面,他无法想像当这地面上出现一只老鼠时人们的反应。 
        “你放心吧,我是久经沙场了。”舒克说。 
        皮皮鲁装作漫不经心地朝海关检查口走去。   第182集 
        舒克在沙发下窥视; 
        五角飞碟的舱门被专家撬开; 
        警察就地卧倒; 
        金发女郎和皮皮鲁脸挨脸   
        皮皮鲁在距离扣押五角飞碟的房间10米远的地方站住了。舒克从兜里探出头来。 
        周围人不多。 
        “我出发了?”舒克问皮皮鲁。 
        “祝你好运!进入五角飞碟后,马上同我联系。”皮皮鲁边观察四周边说。 
        舒克钻出口袋,顺着皮皮鲁的身体溜到地面上。 
        皮皮鲁看到两名巡逻警察朝这边走过来,他忙示意舒克立即行动。皮皮鲁知道,外国警察分工没那么细,他们什么都管。 
        舒克飞快地跑到墙根儿,他先藏在一个卫生箱后边。 
        巡逻警察看到皮皮鲁两手空空站在那里,问他:“请问先生,需要帮助吗?” 
        “谢谢,不需要。”皮皮鲁自知没理由干站在这儿,他朝一排座椅走去。 
        皮皮鲁从报架上抽出一张报纸,他坐在椅子上佯装看报,把耳机塞进耳孔里。 
        舒克等警察走过去后,顺着墙根儿溜到扣押五角飞碟的房间门口。 
        房门没关严,有一道缝儿。 
        舒克透过门缝儿看到几个人在撬五角飞碟的舱门,他知道贝塔肯定出事了,否则他不会这么老实地任凭这些人折腾五角飞碟。 
        屋子的墙角有个沙发。舒克决定将这个沙发当作自己的掩蔽体。 
        舒克运了运气,他用最快的速度进人房间,然后钻进沙发下边。 
        没人发现舒克。 
        “这绝对不是玩具。”一位专家下结论了。 
        “也不像炸弹。”专家的同事说。 
        舒克从沙发下边看见那人将五角飞碟的门撬开了,他把五角飞碟放到桌子上。 
        “我去拿手电。”一位警官说完去找手电。 
        舒克不能再犹豫了,他现在只有冒险一条路了。 
        舒克突然从沙发下钻出来,他扒着桌腿爬上桌子。 
        “老鼠!”一名警官大叫。 
        “哪儿来的老鼠?!”专家看见一只老鼠爬上他的桌子,大吃一惊。 
        “抓住它!” 
        “别让它跑了!” 
        海关警察们拿出把关的责任感,展开了拘捕舒克的行动。 
        舒克径直跑进五角飞碟,然后把舱门关严。幸亏门的锁定装置没有损坏。 
        “老鼠钻进那个东西里了!”专家喊。 
        “别让它再跑出来!”一警官以为舒克是无意钻进死胡同的。 
        专家伸手死死堵住五角飞碟的舱门。 
        舒克松了一口气,一进五角飞碟,他心里就踏实了。 
        舒克看见了躺在地上的贝塔。 
        “贝塔,贝塔,你怎么了?”舒克摇贝塔,又摸他的鼻子,还有气。 
        贝塔仍然昏迷不醒。 
        舒克坐在驾驶台前,戴好耳机,系上安全带。 
        “皮皮鲁,我是舒克,请回答!” 
        “舒克,舒克!我是皮皮鲁,请讲!”皮皮鲁知道舒克已进入五角飞碟,否则舒克无法和他通话。 
        “我已进人五角飞碟,贝塔昏迷了。”舒克说。 
        “受伤了?”皮皮鲁用报纸遮住自己的脸,挡住别人的视线,和舒克通话。 
        “没有外伤。”舒克回头看了一眼地上的贝塔。 
        “现在他们都还在屋里?” 
        “在。我怎么办?” 
        “飞回咱们住的饭店,在楼顶上等我。” 
        皮皮鲁下命令。他担心夜长梦多,鬼知道这些外国海关警察一会儿又想出什么招儿来对付五角飞碟。 
        一屋子海关人员研究怎么把老鼠从飞碟里弄出来。 
        舒克接通了起飞开关。 
        五角飞碟出声了。 
        “炸弹!”专家大喊一声。 
        所有警察都趴在地上,反应极快。 
        五角飞碟升到空中,撞碎窗玻璃飞出房间。 
        警察们抱头俯卧等候爆炸。 
        除了玻璃碎了以外,没出现任何异常。 
        专家站起来,他最先发现桌上的五角飞碟不见了。 
        “我看见它自己从这儿出去了。”一位警官指着窗玻璃上的窟窿说。 
        “自己飞出去了?!”专家吃惊。 
        “是一种新式武器?”一位警官猜测。 
        “携带它入境的人呢?”专家问扣留五角飞碟的警官。 
        “他的护照复印件在这儿。”警官从文件夹里拿出皮皮鲁的护照复印件。 
        “皮皮鲁?”专家一边看皮皮鲁的护照复印件一边自言自语,“这名字很耳熟呀!” 
        专家打开电脑,他将皮皮鲁的名字输人计算机,又噼里啪啦按了几个键。 
        屏幕上显示出皮皮鲁的资料。 
        “原来是他!著名物理学家!”海关的专家恍然大悟,“我说怎么感到熟悉呢!” 
        “他携带入境的那个小怪物是干什么的?”一位女警官问专家。 
        专家摇头。 
        他拨通国家安全部门的电话,将皮皮鲁入境的情况通知他们。 
        “立即查找他的住处。查明他来德国的动机,不要惊动他。查清他带的这个怪物是什么。另外,我们国家很需要他这样的物理学家,一定要尽力留住他。”国家权威部门向警察局长下达指令。 
        警察局长只用了5分钟就查清了皮皮鲁下榻的饭店和房间。饭店说皮皮鲁不在房间里, 
        警察局长利用皮皮鲁不在的时候让部下给16lO房间安装了窃听器和微型摄像机。 
        皮皮鲁知道舒克已驾驶五角飞碟离开机场,他放下报纸,朝大厅出口处走去。 
        走在皮皮鲁前边的是一位金发女郎,她通过旋转门的时候在皮皮鲁的前一个格。 
        金发女郎走出旋转门时,一辆黑色的奔驰车疾驶而来,一个急刹车准确无误地停在她身旁,从车上跳出两个彪形大汉一左一右架住金发女郎往奔驰车里塞。 
        “救命——”金发女郎一边挣扎一边呼救。 
        此刻皮皮鲁刚刚走出旋转门,他的目光与金发女郎相遇了,那小姐的目光震撼了皮皮鲁。皮皮鲁一个箭步冲上去拽其中的一个大汉。 
        大汉抡起拳头击皮皮鲁的头部,皮皮鲁一闪,只躲过了百分之五十。彪形大汉将金发女郎塞进汽车后,捎带手把皮皮鲁也塞进汽车里。 
        等警察闻讯赶到时,奔驰车已一糟烟开走了。 
        皮皮鲁被一个大汉按在后座上,动弹不得。他感到脸上火辣辣的疼。皮皮鲁还从未让人这么打过。 
        皮皮鲁闻到一般香水味儿,他一抬眼皮,看见那小姐也被按在后座上,他们脸挨脸。   第183集 
        燕妮的姐夫站在别墅门口; 
        通讯器不能冒充收音机; 
        大卫给皮皮鲁出难题   
        皮皮鲁想起了看过的恐怖电影,他没想到在自己的生活中能遇到这样的事。 
        奔驰车的座椅是真皮的,皮皮鲁的脸挨着皮椅,闻到一股皮革的特有气味儿。都是皮,有的活着,有的死了。不管是活的还是死的,都满世界转悠。人最重视自己的脸皮,可人皮最没用,一死全完。越没用的东西越重视。瞧这羊皮,装在奔驰车上真提份儿。皮皮鲁想。 
        “谢谢你。”金发女郎的嘴正对着皮皮鲁的鼻子,她说。 
        一股香气朝皮皮鲁扑面而来。 
        不知怎么搞的,皮皮鲁感到心旌飘荡。 
        “你认识他们?”皮皮鲁问。 
        “住嘴,不准讲话!”压着皮皮鲁的大汉用劲儿拧皮皮鲁的胳膊,警告他。 
        皮皮鲁和金发女郎不敢说话了。汽车显然是上了高速公路,开得飞快。 
        皮皮鲁觉得挺享受,他活了四十多年,看了无数英雄救美女的小说和电影电视,现在终于让他碰上一回。皮皮鲁认为,一个男人如果在一生中没有救过一个女人,那他就不算男人。 
        汽车停了。 
        两个大汉先下车,然后他们对着皮皮鲁和金发女郎喊:“下车!” 
        皮皮鲁钻出汽车,车旁是一座造型优雅的别墅式建筑。 
        金发女郎从车的另一侧下车,当她的目光与皮皮鲁相遇时,两人的心灵都为之一震。 
        皮皮鲁还从没见过这么漂亮的女人,上帝显然把所有女性的优点都集中在她身上了,无论面貌、身材、肤色和气质无一不显示出她是一个十足的上等人。即使她站在路旁乞讨,也是贵族。 
        金发女郎也被皮皮鲁的气质吸引了,她还没见过如此透着成熟美和睿智的男人,她这时才真正理解了“男人的生命从四十岁开始”这句名言。皮皮鲁算不上英俊,黑发中已出现了白发。可她却喜欢这种黑白相间的发色。大脑是头发的土壤,头发是种植在大脑上的植物。单调的大脑只能产生单一颜色的头发,只有丰富的大脑才能滋养不同颜色的头发。所以她历来以为,判断一个人的大脑思维能力是否丰富活跃,只要看一眼他的头发就能知道。 
        她认为皮皮鲁的黑白相间的头发为他增添了美和力度。 
        “燕妮,你终于来了!”一位站在别墅门口的青年男子招呼金发女郎。 
        “是你?!大卫!你为什么绑架我?”燕妮脸上全是吃惊的表情。 
        “这是谁?”被燕妮称作大卫的男子指着皮皮鲁问手下。 
        “他冲过来帮燕妮,我们把他也一块弄来了。”劫持皮皮鲁的大汉对大卫说。 
        “燕妮,没想到你还有中国朋友。”大卫冷笑了一声。 
        “我们不认识。”皮皮鲁对大卫说。 
        “你要干什么?’燕妮质问大卫。 
        “咱们还是进屋说吧。”大卫冲彪形大汉们努努嘴。 
        彪形大汉将燕妮和皮皮鲁推进别墅。 
        “他是你什么人?”皮皮鲁趁进屋的机会问和他并排走的燕妮。 
        “姐夫。我姐姐上个月死了。”燕妮说。 
        “你家很有钱?”皮皮鲁凭直觉问。有钱人家才爱闹这种事。 
        “我爸爸是亿万富翁。”燕妮说。 
        “还活着?” 
        “不在了。” 
        皮皮鲁有点儿明白了。 
        “请坐。”大卫随后进屋,示意燕妮坐下。 
        皮皮鲁刚要坐,被大卫制止了。 
        “对不起,得检查一下您的身上。”大卫对皮皮鲁说。 
        一大汉过来搜皮皮鲁全身,看是否携带了武器。 
        “这是什么?”大汉从皮皮鲁上衣口袋里掏出五角飞碟微型通话器。 
        “收音机。”皮皮鲁说。 
        “我看看。”大卫要过去通话器。 
        皮皮鲁有点儿慌,原来他还指望通话器帮他救美女呢。 
        “你是警察?”大卫抬眼看了皮皮鲁一眼,  “这可不是收音机,是通讯装置!” 
        “大卫是学无线电的,还是硕士呢!”燕妮告诉皮皮鲁。 
        “把他捆起来!”大卫一挥手。 
        两名大汉冲上来将皮皮鲁五花大绑。 
        “你疯了?”燕妮斥责大卫。 
        “现在这个世界上,哪个人不是疯子?”大卫反问燕妮。 
        “你到底想干什么?”燕妮问。 
        “送你进天堂。”大卫把皮皮鲁的通讯器往地上一扔,通讯器掉到地毯上。 
        “你想杀我?”燕妮吃惊。 
        “说杀也行。”大卫从酒柜里拿出一瓶路易十三,拧开盖,给自己斟了点儿酒。 
        “为什么?”燕妮问。 
        “要你们家的全部财产。”大卫从牙缝儿里一个字一个字往出挤。 
        “你已经得到不少了。你杀了我,警方不会放过你的!”燕妮警告大卫。 
        “我会做得天衣无缝,会给你安排一起车祸,我还会痛哭流涕地出现在你的葬礼上,然后悲痛欲绝地在你的律师主持下继承你的遗产……”大卫喝光了杯中的路易十三。 
        “恶棍!”燕妮骂道。 
        皮皮鲁摇摇头,他觉得钱的确可以给人带来享受,但它带给人更多的是麻烦。钱能杀人。 
        “老板,怎么处理这小子?”大汉之一指着皮皮鲁问大卫。 
        “勒死,然后扔到野外去。”大卫轻松地说。 
        皮皮鲁傻眼了,他无法通知五角飞碟。 
        两个大汉往外推皮皮鲁。 
        “我根本不认识他,你杀他干什么?”燕妮愤怒了。 
        “他运气太差了。现在我能放他走吗?就让他给你陪葬吧。”大卫冷笑道。 
        “你最好放了我。”皮皮鲁做最后的努力,他要拖延时间。只要舒克打开遥感仪,就会发现皮皮鲁的处境。 
        “为什么?”大卫睁大双眼。 
        “我可不是一般人,我是超人。”皮皮鲁说。说完自己也觉得可笑。 
        “超人?”大卫双手抱在胸前,“我很欣赏你在死前还有幽默感。” 
        “你这样不择手段地弄钱,不怕下地狱?”皮皮鲁盯着大卫的眼睛。 
        “人世间就是地狱。您没觉得您就生活在地狱中?钱就是地狱的标志,凡有钱的地方都是地狱。”大卫一点儿不怵皮皮鲁的目光,他和皮皮鲁对视。 
        “我把我的钱都给你,你放了他。”燕妮对大卫说。 
        “我看你是爱上他了吧?”大卫阴阳怪气地嘲讽燕妮。他又扭头对皮皮鲁说,“这样吧,我给你个机会证明你是超人,如果你真是,我就不杀你了。话又说回来,如果你真是,我想杀也杀不了。” 
        皮皮鲁看着大卫。 
      “在5分钟内挣脱捆着你的绳子。”大卫宣布。   第184集 
        贝塔苏醒后给舒克猜谜语; 
        大卫给皮皮鲁倒数计时; 
        燕妮害怕和大卫演节目; 
        会拐弯的子弹   
        舒克驾驶五角飞碟撞碎机场海关的窗玻璃后,飞临皮皮鲁下榻的饭店上空。 
        舒克寻找安全的着陆地点。 
        现在的人恨不得用放大镜照着每一寸土地盖房子,恨不得把地球膨化了再住。饭店楼顶都被利用了,有露天酒吧,有网球场,还有楼顶花园。舒克几乎找不到落脚的地方。 
        五角飞碟悬停在饭店上空寻找着陆地点,舒克终于物色到一个比较安全的地方——楼顶霓虹灯广告架的顶端。 
        舒克操纵五角飞碟稳稳地降落在霓虹灯广告架上。舒克离开座椅,给昏迷的贝塔做人工呼吸。 
        贝塔醒了,他睁开眼睛的第一句话就是: 
        “外国警察劲儿真大,往死里摇。外国的狗真是多管闲事,碰上老鼠也叫。” 
        “这回可够惊险的。”舒克见贝塔醒了,松了口气。 
        “你救的我?”贝塔一边检查头上有没有外伤一边问舒克,舒克点点头。 
        “我给你猜个谜语吧?”贝塔活过来了还不甘心,总想拿外国警察出出气。 
        “你的脑子没撞出什么毛病吧?”舒克觉得能在这种时候创作谜语的大脑不正常。 
        “烈日下的警察。猜一种外国食品。”贝塔说完自己笑得前仰后合,一把鼻涕一把眼泪。 
        “烈日下的警察?”舒克真猜,“一种外国食品?” 
        贝塔还在为自己的杰作兴奋不已。 
        “猜不出来。”舒克投降。 
        “热狗。”贝塔把谜底告诉舒克。 
        舒克笑得死去活来,他这才知道刚才贝塔被外国警察折腾得有多惨。 
        “皮皮鲁呢?”贝塔问。 
        “马上从机场回来,我看看他到房间了没有。”舒克边笑边走到操纵台前打开遥感仪。 
        “这是谁?他干什么呢?”贝塔盯着荧光屏嚷起来。 
        皮皮鲁的房间里有两个男人撅着屁股在床底下安装什么。 
        “是警察,在装窃听器。”舒克说。 
        “咱们一入境就被警方盯上了,也忒笨了点儿。”贝塔摇头。 
        “以后五角飞碟不能让皮皮鲁带着,还是自己飞安全。”舒克说。 
        “皮皮鲁该回来了吧?”贝塔说。 
        舒克赶忙遥感皮皮鲁。 
        荧光屏上的景象令舒克和贝塔大吃一惊。皮皮鲁被五花大绑在一幢别墅里。 
        “他到那儿干什么去了?发现了歌唱家的线索?”贝塔分析。 
        “他干吗不告诉咱们一声?”舒克说。 
        “他可能没什么危险,不需要五角飞碟。”贝塔注视着画面说。 
        “你看,通讯器在那些人手里!”舒克指着屏幕说。 
        “皮皮鲁还真遇上麻烦了。”贝塔说。 
        “糟糕,他们要杀皮皮鲁!”舒克看见两名大汉开始用绳子勒皮皮鲁的脖子。 
        贝塔急忙打开通讯装置。 
        可通讯器的那一端不在皮皮鲁手中。 
        大卫给皮皮鲁5分钟时间让皮皮鲁证实自己是超人。 
        皮皮鲁一筹莫展。 
        “1分钟。”大卫宣布。 
        燕妮焦急地看着皮皮鲁,她希望皮皮鲁能挣脱绳子。 
        “两分钟。”大卫一边喝路易十三一边计数。 
        皮皮鲁的性命进入倒计时。其实每个人从出生起生命就开始进入倒计时状态,只不过上帝的语言没人听得懂。 
        燕妮内疚地看着皮皮鲁,她恨自己不该降生在一个有钱的家庭,在这个世界上,你有了钱,你就欠了所有的人。你的钱越多,你欠别人的越多——尽管你没向任何人借过一分钱。 
        “1分钟。”大卫得意极了。“我就要结束一个超人的生命了,我是超级超人。” 
        大汉们用狂笑给大卫捧场。 
        皮皮鲁不怕死。但当着这么漂亮的姑娘被人勒死总让人有点儿遗憾。 
        “你没时间了。”大卫说完冲手下挥手。 
        大汉用绳子勒住皮皮鲁的脖子。 
        燕妮冲上来使劲儿拽大汉的胳膊。 
        “把她也捆起来?”手下问大卫。 
        “先别捆,我还给她安排了一个精彩的节目,由我和她表演。”大卫看着燕妮说。 
        燕妮意识到大卫所说的节目是什么了。 
        “把我和他一块儿勒死吧!”燕妮请求。 
        “还愣着干什么?勒死他!”大卫不耐烦了。 
        “超人来了!”通讯器里传出贝塔的喊叫, 
        皮皮鲁乐了。 
        “谁在说话?”大卫往四周看。 
        打手们搜索一番,没人。 
        皮皮鲁对大卫说:“我现在开始给你表演超人的本领。” 
        大卫说:“别拖延时间了,你为自己祈祷吧。” 
        皮皮鲁一用劲儿,束缚他全身的绳子断成数截。 
        大卫傻眼了。 
        “你……”大卫接连后退了几步。 
        “正宗超人吧?”皮皮鲁冲大卫一笑。 
        燕妮不顾一切地冲上去吻了皮皮鲁一下。 
        皮皮鲁心里那个痛快啊,淋漓尽致。 
        贝塔虽然知道通讯器不在皮皮鲁手中,但他急中生智,故意大喊一声让皮皮鲁听见。 
        在贝塔喊叫的同时,舒克驾驶五角飞碟起飞了。 
        舒克驾驶五角飞碟朝那幢别墅扑去,贝塔开始使用五角飞碟的强大武器系统帮助皮皮鲁。 
        “干脆把那几个坏小子都干掉吧!”贝塔说。 
        “还是让皮皮鲁出面解决他们吧,他当着那么美丽的姑娘的面被人家捆着,丢面子了。”舒克说。 
        五角飞碟在别墅的房顶上着陆。 
        大卫拔出手枪,他双手平举着手枪,把准星、自己的眼睛和皮皮鲁的胸口三点连成了一线。 
        “放下。放下。”皮皮鲁面不改色就像大人哄小孩儿那样和蔼可亲地让大卫放下手中的枪。 
        打手们也都拔出了枪。 
        燕妮紧紧地和皮皮鲁靠在一起,她的身体在微微颤抖。 
        “别怕,我说一二三,他们的枪就都会掉在地上,”皮皮鲁安慰燕妮。 
        大卫没等皮皮鲁数数,他先勾动了扳机。 
        “砰——” 
        枪声在房间里显得沉闷,有力。 
        大卫倒在血泊里。 
        子弹回射到他自己身上。贝塔的作品。 
        打手们举枪朝皮皮鲁齐射。 
        打手们一个不剩地死于乱弹之中。   第185集 
        贝塔不同意关监视器; 
        皮皮鲁创世纪; 
        舒克贝塔关于富国男人和穷国男人的对话   
        “你是超人!”燕妮环顾一地的歹徒,搂着皮皮鲁的脖子,眼睛闪着泪珠。 
        皮皮鲁自豪极了,他终于战胜了一屋子坏蛋,并且当着美丽无比的姑娘的面! 
        皮皮鲁双手搂住燕妮的腰,低头吻她。 
        这是皮皮鲁有生以来第一次和异性接吻,面且起点挺高,初吻即是跨国性的。皮皮鲁体验到了一种境界,钢铁熔化的境界。他这才知道,造物主赋予人一张嘴,原来还有比吃饭更重要更美好的职能。 
        皮皮鲁和燕妮紧紧抱在一起,他们现在无需使用语言表达自己对对方的爱慕,他们各自身上的发音装置已经焊接在一起,心和心直接沟通了。 
        “把监视器关上吧?”舒克在五角飞碟里对贝塔说。 
        “我说不能关,万一又有哪个歹徒从地上爬起来冲皮皮鲁放冷枪呢?最好还是开着。”贝塔看着荧光屏,他为皮皮鲁高兴。四十多岁的男人,还没尝过和异性相处的滋昧儿,不能说不是一个特大的遗憾。 
        “这经历够传奇的,电视剧也不过如此。”舒克说。 
        “皮皮鲁也给中国男人争了口气。”贝塔目不转睛地盯着屏幕说。 
        “这话怎么讲?”舒克不明白。 
        “这世界上各个地方的男女比例都是上帝给搭配好了的,比如在一个国家里,有多少男人,就有多少女人,差不到哪去。可是有的国家先富了,那些富了的国家的男人不光占着自己国家的女人,还觊觎上了穷国的女人。于是,穷国的男人就惨了,本来属于自己的女人被富国的男人抢走了。”贝塔发表高见。 
        “这些穷国的女人也是,就为了那几个臭钱就卖身投靠。”舒克历来瞧不上这种女性。 
        “臭钱?没钱行吗?我觉得可以理解。谁不想过好日子?”贝塔反驳。 
        “什么叫好日子?对已有的一切满足,就是幸福。再说了,活得好不好,全在心情。不在物质。由物质带来的快乐是假快乐。这种假快乐是随环境变化的,是暂时而不经久的。”舒克说。 
        “什么叫真快乐?”贝塔问。 
        “真快乐是属于内心精神方面的,是不随环境变迁的,是永恒长久的。”舒克说。 
        “不管怎么说,那些富国男人从穷国男人那里抢走了不少本来属于穷国男人的穷国女人。穷国男人也真够可怜的,本来就穷得只剩下女人了,还被人家盯上了。你看看,那些富国驻穷国大使馆门前排队等着签证的穷国女子一个比一个漂亮。剩下的模样可就不敢恭维了,还没有咱们老鼠家族中漂亮的女性多呢。”贝塔特为穷国男人打抱不平。 
        “你说的也是,这富国男人也忒损了,好事全让它们占了,开好汽车,吃好饭,住好房子,拣好的地方旅游,最后连人家穷国的漂亮女人也给占了。”舒克开始愤愤不平了。 
        “其实人类中的男性不如咱们老鼠中的男性享福,钱能左右人的一切,咱们老鼠却不受钱的摆布,世间万种动物,男性和男性之间无时无刻不在竞争女性,人类的男性之间的这种竞争很大程度取决子钱。而咱们动物的男性之间都是公平竞争。咱们的女性绝不会因为哪个男性有钱而跟他走。”贝塔头一次为自己的种族感到自豪。 
        舒克意味深长地注视着荧光屏上紧紧拥抱的皮皮鲁和燕妮。 
        “你是来这儿旅游的?”燕妮问皮皮鲁。 
        “来寻找一个朋友。”皮皮鲁说。 
        “找到了吗?”燕妮含情脉脉地看着皮皮鲁。她已决定今生今世再也不离开面前这个中国男人,尽管她目前还对他的一切知之甚少。 
        “我今天刚到,还没开始找,就碰到了你。”皮皮鲁仿佛觉得上辈子就认识燕妮,“你信人的轮回转世吗?” 
        “轮回转世?”燕妮投听懂。 
        “就是有上辈子和下辈子。”皮皮鲁解释。 
        燕妮点点头。她信。 
        “我觉得咱们俩上辈子认识。”皮皮鲁说。他还在心里萌生了一个想法,他想给五角飞碟增加一个装置,一个能测出人的上辈子和下辈子的装置。 
        “我有同感。”燕妮又依偎在皮皮鲁怀里。 
        贝塔对舒克说:“咱们呼一下皮皮鲁吧,如果咱们不提醒他,他们这样能一直呆到下辈子。” 
        舒克想起了饭店客房里的窃听器,他同意贝塔的提议,认为有必要提醒皮皮鲁注意时间,反正(禁止)已经过去,既然胜券稳操,好日子留待将来慢慢享受吧。   第186集 
        皮皮鲁知道有人在他的房间里安放了窃听器; 
        燕妮了解皮皮鲁的童年; 
        舒克和贝塔感受到生命分量   
        “皮皮鲁,我是舒克,请回答。”舒克呼叫皮皮鲁。 
        吓了皮皮鲁和燕妮一跳。 
        “谁叫你?”燕妮诧异。 
        皮皮鲁从兜里掏出通讯器。 
        燕妮认出是大卫拿过的那个东西。 
        “你真是警察?”燕妮问。 
        皮皮鲁摇头,他在考虑如何向燕妮介绍舒克和贝塔,还有能否将五角飞碟告诉燕妮。 
        “皮皮鲁,我是舒克,请回答。”通讯器继续呼叫。 
        燕妮看出皮皮鲁在为难。 
        “你答应呀!我爱你,也爱和你有关的一切。你的朋友就是我的朋友。”燕妮对皮皮鲁说。 
        皮皮鲁被感动了。他有点儿内疚,他觉得自己应该完全相信燕妮。他们现在已经是一个人了。一个人有两颗心是最可怕的事。 
        “我是皮皮鲁,请讲。”皮皮鲁拿起通讯器。 
        “咱们在饭店的房间里被人安放了窃听器。”舒克说。 
        “为什么?什么人安的?”皮皮鲁觉得外国事还挺多,一档子接着一档子。 
        “还不清楚。当时突然发现了你的危险处境,我和贝塔就来了。”舒克说。 
        “贝塔没事吧?”皮皮鲁想起在海关贝塔昏迷不醒的事。 
        “好了。”贝塔插话,“祝贺你,皮皮鲁。” 
        “谢谢。你们都看见了?”皮皮鲁脸微微有点儿红。 
        “能看的时候都看见了,不能看的时候都没看见。”贝塔说。 
        “你们再稍等一会儿,呆会儿我把你们介绍给燕妮。”皮皮鲁要先把自己的经历讲给燕妮听。他还是担心燕妮理解不了人和老鼠交朋友的事。 
        燕妮极其温柔极其善良极其平静地看着皮皮鲁。 
        皮皮鲁从自己的童年讲起,讲那枯燥无味的学校生活,讲那没完没了的家庭作业,讲那名目繁多的考试。 
        “你们的老师怎么会给你们留那么多作业?真的要写到晚上10点钟吗?”燕妮皱眉头。 
        皮皮鲁叹了口气:“我没有童年。从小就是大人。我们的老师和父母对孩子的最高评价就是‘懂事’。他们要求孩子说大人话,办大人事,像大人一样成熟。其实,在孩子说大人话办大人事的国家里,大人准说孩子话办孩子事。” 
        燕妮不由得拉紧了皮皮鲁的手,好像怕皮皮鲁失踪似的。她说:“真像童话。” 
        “就是童话。我们的老师和父母还把听话作为培养孩子的最高准则。孩子必须无条件地听大人的话,听前人的话。”皮皮鲁有点儿激动。 
        “其实,人类历史上的所有发明创造和伟大学说都是不听话的结果。”燕妮的声音温柔和缓,但她这段话的每一个字都掷地有声。 
        “所以我的国家穷。我们从小就被训练成听话的人,听书本的话,听权威的话,不能发表自己的见解。我们甚至认为这个世界从来就是这样,以后也不会变化。”皮皮鲁说。 
        “看一个民族有没有前途,看它的孩子就知道了。”燕妮轻声说。 
        皮皮鲁喜欢说话轻柔的女孩儿。 
        “有一天,我在家写作业。忽然听见阳台上猫叫,我趴在窗户上一看,阳台上有一架直升机和坦克。” 
        “直升机落在阳台上?还有坦克?” 
        “是玩具直升机和坦克。” 
        “怎么回事?” 
        “是它们自己开来的。我把直升机和坦克拿进屋里,放在桌子上,你猜是谁驾驶它们?” 
        “有人驾驶它们?” 
        “对。猜猜。” 
        “……猜不出来。快告诉我。” 
        “两只小老鼠!” 
        “这怎么可能?” 
        “真的。他们还有名字,一个叫舒克,一个叫贝塔。舒克是飞行员,开直升机。贝塔是坦克手。” 
        “坦克也能飞?” 
        “是直升机吊着坦克在我的阳台上迫降的。直升机没电了。后来我们就成了好朋友。舒克和贝塔为我的童年增添了很多欢乐。他们帮我参加过航模比赛,我们还拨过城里最大的钟表呢……” 
        燕妮昕入了迷。 
        舒克和贝塔在五角飞碟里也沉浸在回忆之中。活了三四十年以上的生命,只有回忆,才能感受生命的分量。 
        皮皮鲁一直讲到自己成为物理学家。 
        “真为你高兴。”燕妮这才知道自己爱的人是有学问的。 
        “这是因为我从小就不循规蹈矩,我怀疑前人的所有学说。我看书的诀窍就是弄懂书上的每一句话然后再努力证明它们是错误的。”皮皮鲁说。 
        “循规蹈矩的人不会有创造性。”燕妮同意皮皮鲁的观点。 
        “在我成名后的一天,舒克和贝塔回来了。”皮皮鲁说。 
        “30年了,他们还活着?”燕妮吃惊。 
        “他们去了外星球。大概是天上一日,地上一年吧。”皮皮鲁解释。 
        燕妮知道刚才是谁和皮皮鲁通话了。   第187集 
        燕妮和舒克贝塔交朋友; 
        大卫的奔驰车变成了船; 
        鲁西西和皮皮鲁通国际长途电话   
        “快介绍我认识舒克和贝塔吧!”燕妮对皮皮鲁说。 
        “等等,我还要告诉你五角飞碟和歌唱家。”皮皮鲁说。 
        “五角飞碟?歌唱家?”燕妮觉得皮皮鲁是一个宝库。 
        皮皮鲁将五角飞碟和歌唱家及罐头小人的事讲给心上人听。 
        至此燕妮才知道皮皮鲁是怎样击败大卫的。 
        罐头小人更令燕妮着迷。 
        “歌唱家真在德国?”燕妮兴奋。 
        “在。”皮皮鲁肯定。 
        ‘咱们一起找她。我太熟悉我的国家了。”燕妮说。 
        “现在我介绍你认识舒克和贝塔。”皮皮鲁说。 
        五角飞碟像一道闪电,在皮皮鲁和燕妮面前的茶几上着陆。 
        舱门打开后,舒克和贝塔走出五角飞碟,向燕妮问好。 
        “认识你们真高兴。”燕妮一见舒克和贝塔就爱上了他们。 
        “咱们是一家人了,不是吗?”贝塔对燕妮说。 
        “是的。”燕妮略有点羞涩。 
        “祝贺你们。”舒克对皮皮鲁和燕妮说。 
        “谢谢。咱们应该离开这儿了吧?”皮皮鲁说。 
        “他们怎么办?”燕妮指着客厅地板上的几具尸体问。 
        “只好让他们留在这儿等着警察来处理了,咱们走吧。”皮皮鲁说。 
        “去哪儿?”舒克问。 
        “去饭店。”皮皮鲁说。 
        “我先看看饭店的情况。”贝塔钻进五角飞碟。 
        不一会儿,贝塔钻出来: 
        “不行,有几个人正翻你的箱子呢。” 
        “查查他们是干什么的。”皮皮鲁说。 
        贝塔又回到五角飞碟里遥感这些人的身份。 
        “都是便衣警察。”贝塔把结果告诉皮皮鲁。 
        “你不能回去了,去我的住处吧?”燕妮建议。 
        “过不了多长时间,警察就会发现这儿具尸体的,由于你和大卫的亲属关系,警方肯定会找你的。”皮皮鲁说。 
        “我在郊区有一幢别墅,没人知道。咱们去那儿。”燕妮说。 
        皮皮鲁想了想,同意了。 
        “那辆奔驰还停在门口,咱们开它去。”皮皮鲁往门外看了一眼,天已经黑了。 
        “我们呢?”贝塔问。 
        “你们在空中跟着我们。”皮皮鲁不敢再让五角飞碟和他一起行动了。只要五角飞碟处于自由状态,皮皮鲁就天不怕地不怕,他自信世界上还没有能对付五角飞碟的武器。 
        燕妮最后看了地上的大卫一眼,大卫是被钱杀死的。 
        皮皮鲁给燕妮开车门。 
        舒克环视看大卫的客厅,对贝塔说: 
        “你说他绑谁不好,非要把皮皮鲁弄来,这人运气太差。” 
        贝塔从茶几上拿起几块巧克力搬进五角飞碟,说: 
        “像他干这种伤天害理的事,即使这次得逞了,早晚也得倒霉。” 
        “皮皮鲁已经发动汽车了,咱们走吧。”舒克催贝塔,“有几块够吃就得了。” 
        五角飞碟起飞了,它升到夜空中为皮皮鲁保驾护航。 
        皮皮鲁驾驶奔驰驶上高速公路,奔驰汇入车的河流。 
        “皮皮鲁也真够聪明的,从没学过开车,一坐上去就会。”贝塔回想起前些年皮皮鲁第一次开车的情景。 
        “利没学过开五角飞碟,也是上去就会了。”舒克想起了自己的妻子。 
        “不学就会是天才。学了就会是人才。学了不会是蠢材。”贝塔总结。 
        皮皮鲁开车和心上人一起奔驰在高速公路上,心旷神怡。他觉得光有好车不是享受,必须好车配上死心塌地爱你和被你爱的人才是享受。开罗伊斯劳伊斯(一种昂贵的汽车)的人也许没有骑自行车的人幸福。上帝特爱和人类开这种玩笑。 
        皮皮鲁恍忽觉得自己是一个舰长,正驾驶着小船,和自己真心相爱的姑娘航行在大海里。他往常开汽车从没这种感觉。 
        皮皮鲁扭头看坐在他身旁的燕妮。 
        燕妮嫣然一笑: 
        “我有坐船的感觉。” 
        皮皮鲁握住燕妮的手。 
        “刚才你告诉我,你还有个妹妹?”燕妮问。 
        “她叫鲁西西,是我的孪生妹妹,双胞胎。她是学服装设计的,一直在国外,近期才回国,现在帮我坐镇舒克贝塔公司的业务。对了,我应该和她联系一下。”皮皮鲁想起自己曾经答应过鲁西西,到了德国马上同她联系。 
        “这电话能打国际长途。”燕妮拿起车上的电话,“你告诉我号码,我拨号。” 
        皮皮鲁把电话号码告诉燕妮。 
        “通了。”燕妮将话筒递给皮皮鲁。 
        皮皮鲁一手握方向盘,一手拿话筒。 
        “鲁西西吗?我是皮皮鲁。” 
        “你怎么现在才打电话?我都急死了。” 
        ‘一下飞机就遇到麻烦了,还挺惊险。” 
        “转危为安了?” 
        “当然。” 
        “有歌唱家的线索了吗?” 
        “还没开始找。” 
        “皮皮鲁给你找了个嫂子。”贝塔从五角飞碟上插话。   第188集 
        贝塔想和皮皮鲁住在一个房间; 
        皮皮鲁用喜酒给贝塔压惊; 
        皮皮鲁带燕妮坐火箭   
        “嫂子?”鲁西西没听明白。 
        “外国嫂子,特贤慧,特漂亮。”贝塔向鲁西西描述燕妮的品质和容貌。 
        “她叫燕妮。”皮皮鲁说。 
        “一见钟情?”鲁西西问。 
        “是的。相见恨晚。”皮皮鲁说。 
        “舒利怎么样?”舒克问鲁西西。 
        “整天去照顾图钉,也是相见恨晚。”鲁西西拿舒利逗皮皮鲁。 
        “我们明天开始找歌唱家。”皮皮鲁说。 
        “祝你们好运。”鲁西西挂上了电话。 
        “上前边那座立交桥,向右转弯。”燕妮给皮皮鲁指路,“再左转。” 
        汽车下,高速公路,驶上一条乡间小路路的两旁是茂密的灌木丛。 
        “我的别墅就在前边,对,就是那座小楼,看见了?”燕妮说。 
        “你平常不住这儿。”皮皮鲁问。 
        “这房子的产权原来是我妈妈的,妈妈去世前,把它给了我。我平时不来,有个佣人看守这幢房子。”燕妮说。 
        汽车停在院外,燕妮下车按铃。 
        一位男佣出来,见是燕妮,忙开门。 
        皮皮鲁将汽车开进院里。 
        五角飞碟在车顶上着陆。 
        “这车能停在院里吗?”皮皮鲁担心警方发现大卫的车而找他们的麻烦。 
        燕妮让男佣将汽车开进地下车库。 
        燕妮领皮皮鲁走进别墅。皮皮鲁抱着五角飞碟。舒克和见塔从飞碟里探头往外看。 
        这是一座造型别致的三层小楼,每一层都错落有致。第一层是客厅和餐厅、厨房。第二层是卧室和书房,第三层是卧室和健身室。 
        “咱们住二层,舒克和贝塔住三层,行吗?”燕妮问皮皮鲁。 
        皮皮鲁征求舒克和贝塔的意见。 
        “我们从来部是和皮皮鲁在一个屋子的。”贝塔故意给燕妮出难题。 
        “别理他,就照你说的住。”舒克对燕妮说。 
        “我的箱子还在饭店里,怎么办?”皮皮鲁说。 
        “箱子就别要了,又没什么宝贝。”贝塔肚子开始叫唤了,他饿了。 
        “护照在箱子里。”皮皮鲁说,“没护照出不了境,也不能坐飞机。” 
        “最好再别坐飞机了。我觉得飞机是穿梭在天上的十字架。”贝塔说。 
        舒克想起女乘客纠缠皮皮鲁的情景,他笑出声来。 
        “你笑什么?”燕妮好奇。 
        舒克把那场面讲给燕妮听。 
        “真的?”燕妮不信。 
        皮皮鲁点头。 
        “我告诉你们是为什么吧!”贝塔把驱蚊器的事讲给大家听。 
        “你今天没戴驱蚊器吧?”燕妮逗皮皮鲁。 
        “如果戴了,你在机场候机大厅就会求我和你结婚的。”皮皮鲁幽默地说。 
        “我饿了。”贝塔宣布。 
        “我也饿了。”皮皮鲁和舒克异口同声,他们到德国还没吃过饭。 
        燕妮吩咐男佣去弄饭。 
        皮皮鲁打开电视机。 
        屏幕上一位小姐正在播报新闻,她一会儿说非洲干旱饿死了多少多少人,一会儿说某国有一架飞机被歹徒劫持到本不该去的国家。 
        “人吃(又鸟)蛋不叫新闻,(又鸟)吃人蛋才叫新闻。”贝塔边看电视边给新闻下定义。 
        晚饭准备好了。 
        燕妮让男佣回自己的房间休息,她怕男佣看见舒克和贝塔在餐桌上和人共进晚餐吃惊。 
        “能喝点儿酒吗?”贝塔边吃边请示皮皮鲁,他刚才就想拿大卫的路易十三,拿不动。 
        “喝吧,今天在机场你也算历险了。”皮皮鲁同意用酒给贝塔压惊。 
        燕妮给贝塔拿来酒。 
        “这就算是你们的喜酒吧。”贝塔举起酒杯对皮皮鲁和燕妮说。 
        皮皮鲁点点头,给自己、燕妮和舒克都斟了一杯。 
        “祝你们美满幸福!”贝塔说完一饮而尽。 
        “谢谢。”燕妮觉得和这些新认识的朋友在一起特快活,她有做梦的感觉。 
        饭后,皮皮鲁提议早点儿休息,明天开始寻找歌唱家。 
        舒克和贝塔也感到累了,他们到三层的卧室休息。舒克睡在五角飞碟里。贝塔睡在席梦思床上。 
        皮皮鲁和燕妮住二层。皮皮鲁休浴后走进灯光朦胧的卧室。 
        穿着睡衣的燕妮娇艳动人,她正在挑选唱片,她希望在轻柔的音乐声中和皮皮鲁在一起。 
        唱片放进了唱机。唱针在唱片上开始谱写音符。一位女歌手的美妙歌喉展现在卧室里。 
        皮皮鲁和燕妮陶醉在音乐中…… 
        世界进入混沌状态。 
        只剩下女歌手的歌声在耳边萦绕。 
        忽然,皮皮鲁感觉这歌声挺耳熟,可他又想不起来在哪儿昕过。 
        “这歌是谁唱的?”皮皮鲁问身边的燕妮。 
        “胡安娜。超级歌星。”燕妮说。   第189集 
        曲线先进的子弹轨道困扰总统助理; 
        皮皮鲁和爱因斯坦; 
        安东尼的吻和手机   
        深夜,国家最高情报机构和警察局召开紧急联席会议。总统助理出席。 
        “中国籍著名物理学家皮皮鲁今天入境,他携带了一个玩具大小的飞行器。海关扣留该飞行器后,飞行器逃脱。皮皮鲁下榻皇都饭店1610房间。”一位佩戴少将衔的警官首先发言。 
        “据我们了解,皮皮鲁是极有潜力的物理学家,其智慧绝不亚于爱因斯坦。他的才智如果正常发挥,完全能发明出改变世界的东西。我们应该尽最大努力留住他,让他的才能为我国服务。退一步说,即使留不住,也不能放他走。哪个国家得到他,哪个国家就会领先一步。”情报机构的首脑说。 
        “若千年前,他曾预报过一次大地震。他一定是掌握了某种极为先进的预报地震的方法。”一位特邀参加会议的科学家说。 
        “今天傍晚,在一幢别墅里发生了一起很奇怪的人命案。5名男士用自己的枪打死了自己。经过现场勘查,证实他们不是自杀而是他杀。可是凶手是如何让死者自己朝自己开枪的呢?经过计算子弹的飞行距离,大约是在两米左右,谁能自己拿着枪在距离自己两米远的地方开枪向自己射击?”少将警官说。 
        “那是怎么回事?”总统助理问。 
        “只有一种解释,子弹射出后,又拐了180度的弯儿,射到了自己的身体里。”少将警官说。 
        “这不可能。”有人反驳。 
        “我们在现场发现了两个人的脚印,一男一女。其中男人的脚印与皮皮鲁的身材相符。”少将打开文件夹,拿出脚印照片。 
        “皮皮鲁!他和那些人有什么关系?为什么要杀他们?”总统助理皱眉头。 
        “那几个人的身份已经查明了,其中一个叫大卫,是一家集团公司的总裁。另外的几个人是雇员。”少将警官说。 
        “皮皮鲁和大卫从前认识?”有人问。 
        “不清楚。”少将警官说。 
        “就是说,皮皮鲁可能是使用带来的那架小型飞行器击毙大卫的?”总统助理发挥自己的想像力。 
        “可能是大卫先开的枪,那架小型飞行器调转了子弹的运行方向。”少将警官说出了令人毛骨悚然的推断。 
        “立即以谋杀罪逮捕皮皮鲁,不要伤害他,让他为我们服务。”总统助理说,“同时封锁一切出境口,不让皮皮鲁离境。” 
        “我们决定成立特别行动小组,任命最出色的侦探为组长。”少将警官说。 
        “有人选了吗?”总统助理问。 
        “安东尼。”少将警官说。 
        在座的人都点头。安东尼是该国最著名的侦探,破案率百分之百。 
        “有进展随时向我报告。”总统助理站起来。 
        会议结束。 
        安东尼41岁。日尔曼人。高大魁梧。络腮胡子。能双手同时持枪射击。弹无虚发。会驾驶各种车辆。有飞行执照。大脑敏捷。脑沟回错综复杂。有神探之称。 
        此刻,安东尼正驾驶汽车行使在闹市区。他刚结了一桩诈骗案,准备去情人的公寓松弛一下。安东尼和两种人在一起最能体现自己的价值,凶犯和女人。他有征服这两种人的超级本领。 
        汽车停在一幢公寓楼旁。安东尼乘电梯上楼。他按那个按过无数次的门铃。 
        一个迷人的姑娘开门。她看见是安东尼,眼睛里透出欣喜的光芒。 
        安东尼刚要吻她,兜里的手机响了。 
        “讨厌。”安东尼一边吻姑娘一边掏电话。 
        “喂,什么?局长找我?晚会儿不行吗?就晚10分钟,不行?这个断子绝孙的老狐狸。”安东尼把手机塞进兜里。 
        “马上就走?”姑娘显然不愿意。 
        “呆会儿再来,我这人你知道,该今天办的事绝不拖到明天。”安东尼冲姑娘作个鬼脸,走了。 
        姑娘特痛苦。安东尼还不如不来。本来她看电视看得好好的。 
        安东尼开车响着警笛赶回警察局。 
        局长将拘捕皮皮鲁的任务交给他。 
        “总统已经关注此事。你务必在3天之内找到皮皮鲁。”局长说。 
        安东尼点点头。 
        “需要几个人?”局长问。 
        “我自己都多。”安东尼站起来准备告辞。 
        “轻视不得。他可能有特殊武器。”局长提醒安东尼不要麻痹。 
        安东尼冲局长一笑,转身走了。 
        他赶到皮皮鲁下榻的皇都饭店,埋伏在那里的便衣一见头号侦探来了,就知道自己监视的目标上档次了。 
        安东尼把皮皮鲁的箱子检查了一遍。他把皮皮鲁的护照装进自己兜里。皮皮鲁的照片已经印在他脑海里了。 
        箱子里的一套微型餐具引起了安东尼的注意。 
        这是一套类似子小女孩过娃娃家的餐具。安东尼掏出手机。 
        “皮皮鲁同行一共几个人?”安东尼问。 
        “就他自己。”警察局值班员回答。 
        “没有小孩儿?” 
        “绝对没有。” 
        安东尼将小餐具放在手掌上反复看。 
        没有孩子同行,带这么小的餐具干什么?癖好?如今这世界上,收藏什么的都有。安东尼想。他还是把小餐具装进自己兜里。 
        那是舒克和贝塔的餐具。 
        安东尼又驱车赶到大卫的别墅,他凭直觉认定皮皮鲁来过这儿。 
        安东尼开始撒网。   第190集 
        胡安娜的歌声; 
        安东尼这个不速之客; 
        贝塔在梦中结婚被皮皮鲁破坏; 
        瓮中之鳖的感觉   
        皮皮鲁觉得胡安娜的声音很耳熟。 
        “你听过她的歌?”燕妮问。 
        “没有。”皮皮鲁活这么大,头一次听说有个歌星叫胡安娜,还挺有名。 
        “也许唱歌的嗓子都差不多。”燕妮说。 
        皮皮鲁让燕妮将胡安娜的唱片再放一遍。 
        皮皮鲁边听边极力在自己的记忆中搜寻,一个个熟悉的面孔从皮皮鲁脑际里闪过。 
        “是歌唱家!”皮皮鲁大喊一声。 
        “什么歌唱家?”燕妮不明白。 
        “胡安娜的声音和罐头小人歌唱家的声音很像。”皮皮鲁说,“像极了。” 
        “是巧合吧?”燕妮说。 
        “最近有胡安娜的演出吗?”皮皮鲁问。 
        燕妮打电话。 
        “明天晚上有。”燕妮放下话筒说。 
        “咱们去看看。”皮皮鲁说。 
        燕妮点点头。 
        第二天清晨,梦乡中的皮皮鲁和燕妮被门铃声惊醒了。 
        “谁这么早来?”燕妮看了一眼墙上的挂钟,才六点半钟。 
        皮皮鲁披上衣服走到窗户旁,撩开窗帘的一角往楼下看。 
        门外停着一辆汽车,一位高大的男子站在车旁按门铃。 
        “你认识这个人吗?”皮皮鲁问燕妮。 
        燕妮走到窗前往下看。 
        “不认识。”她摇头。 
        “你让男仆去给他开门,我让五角飞碟看看他是干什么的。”皮皮鲁对燕妮说。 
        男子走进客厅。燕妮带着疑惑的表情同他见面。 
        “对不起,燕妮小姐,打扰您了。我叫安东尼。”安东尼将标明警察身份的证件出示给燕妮。 
        “找我有什么事?”燕妮没让安东尼坐。 
        “您的姐夫叫大卫吧?”安东尼注视着燕妮的眼睛问。他在心里为面前这位美貌女子的姿色叫绝。 
        “是的。”燕妮回答。 
        “大卫昨天晚上被杀了。”安东尼说。 
        “谁杀的?”燕妮懒得装伤心和惊讶。 
        “还不知道。我想向您了解一下有关大卫的情况。”安东尼说。 
        “我们不大来往。我不喜欢这个人。”燕妮说。 
        安东尼耸耸肩。 
        “您是怎么知道我住在这儿的?”燕妮奇怪这个侦探居然能找到她的隐蔽住处。 
        “职业病。”安东尼说。 
        “你有什么问题请讲,我一会儿还有事。”燕妮说。 
        安东尼开始提问。他有意无意地把一个问题分成几问。他想多和这位姑娘呆一会儿。 
        皮皮鲁来到三楼的卧室。舒克和贝塔还在酣睡。 
        “快起来,帮个忙。”皮皮鲁的嘴对着贝塔的耳朵说。 
        贝塔揉揉眼睛,坐在席梦思床上向皮皮鲁抗议: 
        “你就会欺负惟一的单身汉,我正做好梦呢!” 
        “下边来了个陌生人,你进五角飞碟测测他,看他是干什么的。”皮皮鲁一把将贝塔从床上拿起来,放到桌上的五角飞碟旁边。 
        “绑架!非法劫持!”贝塔一边打哈欠一边打开五角飞碟的舱门。 
        舒克也在熟睡中。 
        贝塔推醒他。 
        “开机!”贝塔冲舒克大喊一声。 
        舒克以为出现了紧急情况,他一跃而起,坐在座椅上系安全带。 
        贝塔哈哈大笑。 
        舒克回身给了贝塔一拳。 
        “我也是受害者。”贝塔一边笑一边打开遥感器。“正做好梦呢,被皮皮鲁叫醒了。” 
        “什么好梦?”舒克问。 
        “结婚呗。”贝塔注视着荧光屏说。 
        “和谁结婚?”舒克也盯着屏幕。 
        “陌生姑娘。”贝塔的注意力开始被屏幕上吸引了。 
        “可以理解。我结婚前尽做结婚的梦。没结婚的人不做这种梦才不正常呢。”舒克说。 
        “楼下这小子是警察,还特有名。哟,他是冲着咱们来的!瞧,他们还专门为留住皮皮鲁成立了一个什么行动小组。”贝塔指着屏幕说。 
        “他们认为皮皮鲁是难得的人才,想留住皮皮鲁。”舒克看着昨晚该国召开的紧急会议画面说。 
        “发达国家也够损的,他们之所以富,就因为人才多,越富越网罗人才。穷国本来人才就少,还都被富围挖走了。”贝塔挺为穷国打抱不平。 
        “我去告诉皮皮鲁,你继续监视。”舒克离开五角飞碟。 
        皮皮鲁听舒克述说遥感结果。 
        “所有出境的地方都被封锁了,机场和车站都印发了你的照片,他们以谋杀嫌疑为借口留住你,让你为他们服务。”舒克说。 
        有本事的人往往是不幸的。 
        皮皮鲁在屋里来回踱步。 
        “咱们可以用五角飞碟的威力帮助你出境。”舒克说。 
        “那样我就成了新闻的焦点了,回国后也安生不了。”皮皮鲁现在最怕出名,他觉得过宁静的生活是人生的最高享受。大红大紫的名人上辈子都做过孽。 
        皮皮鲁有瓮中之鳖的感觉。 
        燕妮打发走安东尼后,上楼找皮皮鲁。 
        “是个警察,来调查大卫被杀的。”燕妮对皮皮鲁说。 
        “还是个特有名的警察。他们已经怀疑皮皮鲁同大卫的死有关系了。”舒克说。 
        燕妮惊讶地看皮皮鲁。 
        皮皮鲁点头证实舒克的话。 
        “你们怎么知道的?”燕妮问。 
        “五角飞碟。”皮皮鲁伸出五个指头。 
        “太厉害了。五角飞碟如果落到坏人手里,这世界就完了。”燕妮眼里闪过一丝忧郁。 
        “所以我和贝塔经常提高觉悟,改造世界观。”舒克说。 
        “我现在很难出境了。贵国的所有出口都对我封锁了。”皮皮鲁告诉燕妮。 
        “就为大卫?”燕妮问。 
        “不。为大卫是借口,主要是想用我的大脑为贵国服务。他们还对五角飞碟感兴趣。”皮皮鲁说。 
        “没错,发达国家对人才最敏感。他们恨不得给全世界的人才发他们的护照。其实你留在这儿也不错,你的才能会得到最大的发挥。”燕妮说。 
        “我一生最大的愿望,就是只拥有中国护照。”皮皮鲁说,“驴就是驴,马就是马,我不当骡子。骡子上无古人,后无来者,没着没落。”   第191集 
        皮皮鲁搂着一座金山; 
        安东尼对警察表示爱情; 
        无声手枪击瘪轮胎; 
        燕妮没想到是安东尼   
        燕妮被皮皮鲁这一段掷地有声的语言深深震撼了,她钦佩皮皮鲁,她觉得他是一个男子汉。燕妮喜欢热爱自己的国家自己的民族的人。因为国家穷就抛弃国家投奔富国的人不是男子汉。女人这样做可以理解,男人这样做就是懦夫。男人应该通过自己的努力把自己的国家变富。没有哪个国家生下来就是富国。 
        “你说得对。”燕妮情不自禁地吻皮皮鲁。 
        舒克假装闭眼睛。 
        “大可不必。”皮皮鲁对舒克说。 
        燕妮冲舒克笑笑。 
        “他是作家,需要观察生活。”皮皮鲁说。 
        “舒克是作家?”燕妮以为皮皮鲁在开玩笑。 
        “《人类,我是你的朋友》就是他写的。”皮皮鲁说一 
        “你说的是获得诺贝尔文学奖的《人类,我是你的朋友》?作者不是叫涛涛轰吗?”燕妮拜读过那本轰动一时的小说。 
        “舒克是真正的作者,涛涛轰沽名钓誉。我去揭穿他,可没人信我的话。”皮皮鲁叹了口气。 
        “人也真是的,已经堕落到和老鼠争名的地步了。”燕妮内疚地看着舒克。 
        “皮皮鲁是会回国的,你呢?”舒克问燕妮。他看出皮皮鲁是真爱上燕妮了。舒克担心燕妮不走。 
        “皮皮鲁去哪儿,我就去哪儿。”燕妮一字一句地说。 
        她认定皮皮鲁是无价之宝,和他这样的男人在一起,不管生活在哪儿,都等于同时拥有美国、日本、德国、加拿大…… 
        皮皮鲁将燕妮紧紧搂在怀里。 
        舒克的眼眶里涌出了泪水。他知道皮皮鲁搂着的是一座金山。金山搂金山。 
        “中国的路上都是马车吗?”燕妮仰脸问皮皮鲁。 
        舒克和皮皮鲁笑出了眼泪。 
        燕妮知道自己出洋相了。 
        “皮皮鲁在中国开你们国家生产的豪华轿车。”舒克一边擦眼泪一边告诉燕妮。 
        “你们国家的人对中国的了解太少了。中国人对你们的了解很详细。”皮皮鲁说。 
        “你们了解埃塞俄比亚吗?了解布隆迪吗?”燕妮问。 
        皮皮鲁摇摇头。 
        “谁都愿意了解比自己生活得好的地方的人怎么生活,谁也不会对比自己生活得差的地方感兴趣。这大概就是我们不了解中国的原因。”燕妮分析。 
        皮皮鲁若有所思地点头,他觉得燕妮的话有道理。想让人家对你感兴趣,最好的办法就是让自己富起来。 
        吃早餐时,大家商量下一步的行动计划。 
        “先找歌唱家,然后想办法突围。”贝塔边喝牛奶边说。 
        “是出境。”舒克纠正,“不是突围。” 
        “我觉得胡安娜的声音和歌唱家很像。”皮皮鲁吃水果,早晨吃水果是吃金子。 
        “胡安娜是谁?”贝塔问。 
        “我国最著名的歌星。”燕妮说。 
        “我总觉得胡安娜和歌唱家有什么关系,咱们晚上去看她的演出。”皮皮鲁说。 
        “你现在出去可够危险的。”舒克提醒皮皮鲁。 
        “皮皮鲁可以化装。”贝塔出主意。 
        “我一会儿去剧场买票。”燕妮说。 
        安东尼离开燕妮的别墅后,心中一阵怅然若失。他居然在1小时之内1秒钟也没想皮皮鲁的事,满脑子全是燕妮。安东尼接触过无数个女孩子,可从没有任何女孩子能像燕妮这样令他心神不定。 
        安东尼驾车闯了一个红灯,自己全然不知。 
        一位执勤的警察把摩托车横在安东尼的车前。 
        “您违章了。”警察趴在车窗前对安东尼说。 
        “什么时候?”安东尼眼前全是燕妮。 
        “刚才。闯红灯。”警察说。 
        “我爱你。”安东尼把警察当成了燕妮。 
        吓得警察往后退了一步。 
        安东尼用看姑娘的眼神看那警察。 
        警察定定神,说: 
        “您违章了。” 
        安东尼从幻觉中清醒过来,他将自己的证件递给警察。 
        警察一看证件,“啪”地给安东尼敬了个礼。 
        安东尼愣了一下,只见他凋转车头,往回开。 
        警察直发呆,拿不准是不是自己得罪了安东尼。 
        安东尼驱车回到燕妮家附近的一片灌木丛旁,他还想见到燕妮。 
        他过去经常这样蹲守。但等一位姑娘这是头一回。安东尼把皮皮鲁忘得一干二净。 
        别墅的院门打开了,燕妮开着一辆蓝色的汽车驶出别墅。 
        安东尼驾车跟上燕妮。 
        两辆汽车一前一后驶在高速公路上。安东尼掏出带消音器的手枪,瞄准燕妮的汽车的后轮开了一枪。 
        燕妮的车胎瘪了。 
        燕妮停车检查。 
        安东尼开车过来间燕妮: 
        “需要帮忙吗?” 
        燕妮没想到是安东尼。   第192集 
        燕妮给安东尼递工具; 
        倒霉的备胎难逃厄运; 
        安东尼飞起一脚; 
        色狼跃出3米   
        安东尼将自己的汽车停在燕妮的汽车前边,他走到燕妮身边,问: 
        “轮胎被扎了?” 
        燕妮点点头。 
        “有备胎吗?”安东尼问。 
        燕妮点点头。 
        “我帮你换。”安东尼不等燕妮回答,就打开后备箱拿备胎。 
        燕妮有点儿茫然地看着紧忙活的安东尼,她不知道他是碰巧路过这儿还是有意跟踪。 
        “把套筒扳手递给我。”安东尼有意无意地使用和特熟悉的人说话才使用的口气。 
        燕妮迟疑了一下,把套筒扳手递给安东尼。安东尼熟练地更换轮胎。 
        安东尼将千斤顶放人车身下,然后旋转摇把将车身顶起。瘪轮胎离开了地面。 
        新轮胎换好后,安东尼让燕妮去他的车上拿测量轮胎气压的测量计。 
        燕妮还没走到安东尼的汽车旁边,安东尼掏枪又给了新换上的备胎一枪。 
        燕妮回到自己的车旁时,安东尼告诉她,“备胎也是坏的。” 
        “这怎么可能?”燕妮不信。 
        安东尼撤出千斤顶,新换上去的轮胎刚一挨地就疲软了,汽车又歪了。 
        燕妮看表。 
        “怎么,有急事?”安东尼问。 
        “买票。”燕妮料定安东尼还会继续问,她索性先说了,“胡安娜的音乐会。” 
        “搭我的车去吧,我正好路过那座剧院。”安东尼一边擦手一边说。 
        “谢谢,我坐出租车去。”燕妮谢绝。 
        安东尼耸耸肩,靠在自己的车上看燕妮“打的”。 
        燕妮运气不佳,10分钟过去了,没截到一辆出租车。 
        安东尼将手伸进车里,按了一下喇叭。 
        燕妮闻声回头。 
        安东尼指指自己的汽车,冲燕妮招手。 
        燕妮想了想,钻进安东尼的汽车。她一边系安全带一边对安东尼说:“谢谢。” 
        安东尼操纵汽车采用强爆发力起步。此时的车速和他的心情一样,狂喜不已。 
        安东尼视力的余光可以看见坐在身旁的燕妮,他的心脏一反循规蹈矩的频率,突然加快了跳动的次数。安东尼身边的这个座位坐过无数个姑娘,可他的心跳却始终如一,从没像今天这样失控。 
        安东尼现在最大的愿望就是堵车,最好堵上三天三夜。不算今天。 
        上帝不太关照安东尼,路面情况特好。安东尼真想掏枪给自己的车轮子一枪。 
        “你很漂亮。”安东尼赞美燕妮。 
        “谢谢。”燕妮说。 
        “我会很快查出杀害你姐夫的凶手。” 
        “是吗?” 
        “你非常讨厌大卫?” 
        “是的。” 
        “你认为谁会害他?” 
        “他自己。” 
        “……” 
        “有线索吗?” 
        “有。” 
        “我可以知道吗?” 
        “当然。不过,目前还只是推测……”安东尼等于什么也没说。 
        “你开车很文明,还会让着别的车。我原来以为,警察开车都是横冲直撞。”燕妮说。 
        “他们在事业上让着我了,我理应在行车上让着他们。”安东尼还有幽默感。终于说了一句让燕妮对他刮目相看的话。 
        汽车在剧场门口停住了。 
        “我到了,你忙去吧,谢谢。”燕妮打开车门下车。“对了,我姐姐上个月死的,你最好查查是不是大卫害的。” 
        安东尼极有失落感地注视着燕妮走向售票窗口的背影。燕妮身上的曲线是无与伦比的,每一个起伏每一个轮回每一个折扣都体现了天衣无缝尽善尽美的原则。 
        局长给安东尼定的指标是3天内找到皮皮鲁。 
        现在安东尼给自己定的指标是,3天内征服燕妮。安东尼靠自己的职业征服过许多姑娘。这个世界很怪,越是漂亮的女孩子越怕警察。大概因为漂亮往往和丑恶连在一起。 
        燕妮的姐姐卜个月刚去世,姐夫昨天又死了,她还有心看歌星演出?安东尼左手的中指有节奏地敲着方向盘,眼睛一刻不离地盯着在窗口买票的燕妮,大脑推理。 
        两个小伙子好像在纠缠燕妮,其中一个伸手要往燕妮身卜摸。燕妮碰上了色狼。 
        安东尼跳下汽车,跑到燕妮身旁.拍了拍两个小伙子其中的一个。 
        “干吗?关你什么事?”那小子横眉怒视安东尼。 
        另一个出手给安东尼一拳。 
        安东尼一侧身,轻松地将那厮重重摔在地上。安东尼又飞起一脚,把准备扑上来的那小子踢出去足有3米远。 
        周围的人发出惊叹声。 
        “买好票了吗?”安东尼注意到燕妮手中的两张票。他对于这个数量比较吃醋。 
        燕妮点点头,说:“谢谢您的帮助。” 
        “我可以请你喝点儿什么吗?”安东尼指指路边的一座咖啡厅。 
        “谢谢,不了。我还有事。”燕妮觉得安东尼这个警探并不令人讨厌。 
        “我送你回家。”安东尼说完不等燕妮同意,就自己往汽车那儿走。 
        燕妮只得上了安东尼的车。她不得不承认,安东尼也算得上是男子汉。   第193集 
        贝塔的未婚男性直觉特准; 
        燕妮的汽车无人驾驶启动; 
        安东尼掏枪却没有目标   
        燕妮回家后,将买票的经历讲给皮皮鲁、舒克和贝塔听。 
        “那侦探送你回来的?”贝塔表示惊讶。 
        燕妮点头。 
        皮皮鲁走到窗前,看到安东尼的车刚走。 
        “他爱上你了吧?”贝塔说。没结婚的男性都有这种直觉。 
        “他大概是想从我这儿得到线索。”燕妮说。 
        “你的汽车还在半路上停着?”舒克问燕妮。 
        “停在高速公路的停车带上。”燕妮几乎已经把抛锚的汽车忘了。 
        “用五角飞碟把车遥控开回来。”皮皮鲁说。和特熟悉的人说话才使用的口气。 
        燕妮迟疑了一下,把套筒扳手递给安东尼。安东尼熟练地更换轮胎。 
        安东尼将千斤顶放人车身下,然后旋转摇把将车身顶起。瘪轮胎离开了地面。 
        新轮胎换好后,安东尼让燕妮去他的车上拿测量轮胎气压的测量计。 
        燕妮还没走到安东尼的汽车旁边,安东尼掏枪又给了新换上的备胎一枪。 
        燕妮回到自己的车旁时,安东尼告诉她,“备胎也是坏的。” 
        “这怎么可能?”燕妮不信。 
        安东尼撤出千斤顶,新换上去的轮胎刚一挨地就疲软了,汽车又歪了。 
        燕妮看表。 
        “怎么,有急事?”安东尼问。 
        “买票。”燕妮料定安东尼还会继续问,她索性先说了,“胡安娜的音乐会。” 
        “搭我的车去吧,我正好路过那座剧院。”安东尼一边擦手一边说。 
        “谢谢,我坐出租车去。”燕妮谢绝。 
        安东尼耸耸肩,靠在自己的车上看燕妮“打的”。 
        燕妮运气不佳,10分钟过去了,没截到一辆出租车。 
        安东尼看见了前方燕妮那辆停在路边的车,他感到为燕妮换轮胎的事好像发生在上个世纪。 
        燕妮的汽车启动了,它离开停车带,驶入行车道。 
        “有人偷了燕妮的车!”这是安东尼的第一判断。 
        他加大油门追燕妮的车。 
        安东尼驱车驶人超车道,他的汽车经过努力与燕妮的汽车平行了。 
        安东尼掏出手枪,准备命令窃车贼停车。紧接着他的眼珠差点儿成为子弹射出去——燕妮的汽车内空无一人! 
        安东尼晃晃头,再看。绝对没人。汽车自己行驶在高速公路上。 
        安东尼是无神论者,现在他相信上帝的存在了。 
        “这怎么可能?”安东尼想尝试拦住燕妮的汽车。 
        他加速超过燕妮的汽车,然后往右打方向盘,别燕妮的汽车。 
        燕妮的汽车将安东尼的车顶到旁,继续奔驰。安东尼的车差点儿撞到隔离栏上。 
        “这是那侦探吧?”贝塔边遥控燕妮的车边问舒克。 
        “是他,真是冤家路窄,怎么老碰上他?”舒克盯着荧光屏,说。 
        “我看这小子是看上燕妮了,想戗行,没门!他们外国男人从咱们中国男人身边戗走了多少女人,这回皮皮鲁好不容易为中国男人露了回脸,可不能前功尽弁。”贝塔想通过遥控燕妮的车煞煞安东尼的威风。 
        “我看你是杞人忧天。燕妮才看不上安东尼呢。”舒克以过来人的身份分析。 
        “麻痹不得。那侦探人高马大,又会讨好女人,现在这世界上的女人,别看一个个吃得好,挺丰满,其实都像一张纸那么浅薄,只要男人奉承得恰到好处,让她们干什么她们就干什么,谁不爱听好话呀。”贝塔说。 
        “这你就是外行了。燕妮如果没见过皮皮鲁,她没准会喜欢安东尼,可她认识了皮皮鲁,就不会对安东尼感兴趣了。”舒克转身从五角飞碟驾驶台旁的冰箱里拿出饮料。 
        “为什么?”贝塔问。 
        “往本质上说,皮皮鲁这种男人最害女人。”舒克喝了一口饮料,“你别光听说话,看着点儿屏幕,别让安东尼把你的车挤到路下边去。” 
        “皮皮鲁害女人?”贝塔一边遥控燕妮的汽车和安东尼斗法一边问舒克。 
        “不管什么女人,只要一接触皮皮鲁这种高智商高才能的男人,她们就不会再对其他男人感兴趣了.谁用了头等品还会再用二等品三等品?而皮皮鲁这样的男人在人类中是凤毛麟角,上千万人里也许都摊不上一个。这样的男人一生撑死真爱5个女人。算起来,全人类也就有百十来个女人能享受上皮皮鲁这种男人。又见过接触过又不能拥有还由此对其他档次的男人不感兴趣,这不是害人是什么?”舒克滔滔不绝。 
        “幸亏皮皮鲁接触女性少,否则真不知得害多少女人今生今世让自己先生过不好。”贝塔连连点头。 
        “皮皮鲁的脑子多聪明,你看看他发明创造的这五角飞碟。男人爱女人,爱的是身体。女人爱男人,爱的却是大脑。男人的大脑能让地球倒着转。多厉害。”舒克说。 
        “女人的身体也能让地球倒着转。”贝塔说。 
        “这倒是。”舒克不得不点头。   第194集 
        安东尼跌份儿; 
        警车筑起铜墙铁壁; 
        贝塔操纵燕妮的汽车大出风头; 
        地球是一辆公共汽车   
        安东尼的倔劲儿上来了,他非要制服这辆无人驾驶的汽车不可。 
        燕妮的汽车上了一座立交桥,它调头往回开。安东尼纳闷了:这车是受燕妮控制的?还是自己认家?它有这么大本事,那刚才我用枪射击它的轮胎时,它怎么就那么乖地服输了?” 
        安东尼紧迫燕妮的车不放,他边开车边拿起车载电话。 
        安东尼告诉警察局值班中心,他在某号高速公路上发现一辆可疑的汽车,需要10辆警车增援。 
        10辆警车一字排开,封锁了安东尼和燕妮的汽车行驶前方的公路,警察们平端着冲锋枪站在警车后边。 
        “看你还往哪儿开?”安东尼减慢车速,等着着好戏。 
        “怎么办?”贝塔问舒克。 
        “操纵燕妮的汽车从警车头顶上飞过去。就像警匪片里演的那样。”舒克说。 
        “太棒了。”贝塔神采飞扬。 
        面对前方的警车防线,燕妮的车没有减速,跟在后边的安东尼傻了。 
        只见燕妮的车凌空而起,飞跃过警车,着陆后继续行驶。 
        安东尼光顾着看燕妮的车,自己的车和警车撞上了。 
        安东尼的脸被破碎的挡风玻璃划破了,血顺着鼻翼流进嘴角。 
        警察们跑过来将安东尼从汽车里拉出来。 
        “您受伤了。”警察们都认识大名鼎鼎的神探。 
        “没事,这倒成全了我。”安东尼历来认为,脸上没伤疤的男人不叫男人。 
        “追吗?”一警察请示安东尼。 
        “跟着,看它去哪儿。”安东尼接过警察递过来的毛巾擦脸上的血。 
        “车上的歹徒有枪吗?”准备跟踪的警察问安东尼。 
        “那车上没人。”安东尼说。 
        “没人?”警察断定神探准是脑震荡了。 
        “快追吧。”安东尼挥挥手。 
        警察们驱车追赶燕妮的汽车。公路上出现了惊心动魄的场面。其他车辆都靠边给燕妮的车和警车让路。某嗅觉特别灵敏的电视台已出动直升机在空中摄像。 
        8辆警车硬是堵不住燕妮的车。只见那车一会儿腾空,一会儿落地,左右逢源般地如人无人之境。警察们清清楚楚地看到车内无人。 
        “你这样和他玩下去,一会儿就把警察都吸引到这儿来了。” 
        舒克提醒贝塔。 
        “咱们直接让它飞回来吧。”贝塔遥控燕妮的车升空,用闪电般的速度回到别墅。 
        追赶燕妮的车的警察们大眼瞪小眼,目标丢了,是飞上天后丢的,没再落下来。 
        安东尼听警察的汇报。 
        “往哪个方向飞的?”安东尼问。 
        警察指指东边。 
        安东尼心里有数了,燕妮的车是自己回家了。 
        在这个时代,每天都有新的发明问世,并以最快的速度为人类服务。安东尼对于无人驾驶的汽车和能飞上天的汽车并不特别吃惊,他吃惊的是自己对此一无所知。 
        安东尼驱车赶到该市科技情报中心。 
        中心主任听明安东尼的来意后,安排一位资深的汽车专家为神探咨询。 
        汽车专家的办公室里全是汽车模型和照片,从世界上第一辆汽车一直到想像中的未来汽车。 
        地球实际上是一辆在宇宙中行驶的公共汽车,它经历的四个车站分别是春、夏、秋、冬。它的乘客上车下车穿梭不停。没有哪位乘客能永久乘车赖着不走。都是乘客,迟早都会下车。有钱没钱有名没名全一样。一走进汽车专家的办公室,安东尼就产生了如上的感受。 
        “请坐。”汽车专家将发愣的神探招呼回现实中。 
        安东尼落座。 
        汽车专家注视着面前这位脸上流血的大名鼎鼎的侦探。 
        “请教您一个问题。”安东尼说。 
        “请讲。”汽车专家不客气。 
        “现在有无人驾驶的汽车吗?” 
        “有。不过只局限在矿山开采等不适合人操纵的危险行业。” 
        “在城市的道路上绝对没有无人驾驶的汽车行驶?” 
        “当然。这您应该比我清楚,您是警察。” 
        “有能够乘坐人的遥控汽车吗?” 
        “没有。” 
        “有能飞上天的汽车吗?” 
        “……”汽车专家用惊讶的目光回答侦探。 
        “今天我在路上碰见一辆无人驾驶的汽车,它还能腾空而起。您能解释这件事吗?” 
        “您说什么?”汽车专家又问了一遍。 
        安东尼简要叙述了事情的经过。 
        “这不可能。”汽车专家挠了挠没剩下几根头发的头。 
        “我亲眼所见。”安东尼平静地说。 
        “那车什么牌子?”汽车专家问。 
        安东尼说出了燕妮的汽车的厂牌。是美国造。 
        “绝对不可能。那个厂家里有我们的情报人员,我们在他们每推出一辆新车前的半年就拿到了该汽车的各种信息。”汽车专家特自信,也特自豪。 
        现在的国际间谍绝大部分工作在经济领域。   第195集 
        汽车专家的脚麻了也不敢动; 
        皮皮鲁的小学考试成绩出现在电脑上; 
        安东尼的脑细胞大多是雄性脑细胞   
        “就是说,据您现在掌握的信息,世界上还绝对没有能飞的无人驾驶汽车?”安东尼请汽车专家就此事下结论。 
        汽车专家特权威特肯定地点头。 
        安东尼的大脑开始高速运转,所有脑细胞接到思维的指令后都各尽其职地工作。 
      忽然,安东尼眼睛一亮。 
      他想起了局长给他下达任务时说的皮皮鲁可能拥有一个超现代化的先进武器的话,那武器的外形像玩具飞碟。 
        安东尼的脑细胞又将皮皮鲁皮箱里的小餐具向安东尼提示。 
        大卫死亡的现场有皮皮鲁的脚印;大卫是燕妮的姐夫;燕妮的一辆根本不可能飞行的汽车飞上了天;皮皮鲁的玩具飞碟;皮皮鲁箱子里的微型餐具…… 
        这些表面上互不相连的现象有内在联系吗?安东尼指示自己大脑里的板凳细胞替补细胞后备细胞民兵细胞预备役细胞统统上阵思考推理。 
        汽车专家一动不动地注视着安东尼,他知道神探在思考,他清楚爱思维的人在思维时都不喜欢被打断。爱思维的人和不爱思维的人的区别就在于:爱思维的人想别人没想过的事,不爱思维的人想别人想过的事。 
        安东尼就这么呆坐了整整半个小时。 
        汽车专家的脚都麻了,也不敢动。 
        突然,安东尼猛地站起来,吓了汽车专家一跳。专家为安东尼高兴,他知道神探的脑细胞们的集体劳动出了成果。 
        “谢谢您。”安东尼和汽车专家握手告别。 
        安东尼驾车直奔警察局检测中心。 
        他将从皮皮鲁的箱子里发现的微型餐具递给检验员。 
        “这是什么?”检验员看着塑料袋里的小餐具问。 
        “你化验一下,这餐具是谁用的?”安东尼吩咐道。 
        15分钟后,检验结果出来了。 
        “餐具上有老鼠的唾液,不知是老鼠用的还是刚被老鼠碰过。”检验员说。 
        老鼠?!安东尼又命令细胞们赤膊上阵,非想个水落石出不可。 
        皮皮鲁的箱子是封闭的,不可能钻进去老鼠。就是说,老鼠接触这餐具,皮皮鲁应该是知道的。 
        皮皮鲁,人类著名物理学家。老鼠,地球上最为人类唾弃的动物之一。这两者之间能有什么联系呢? 
        安东尼乘电梯来到警察局档案中心,他坐在电脑前,开始查阅皮皮鲁的所有资料。从皮皮鲁在妈妈肚子里受孕开始,一直查到来德国前。 
        档案中心储存了世界上所有名人的资料,从最著名的到只在报刊上露过一次名的最不著名的。 
        安东尼聚精会神地注视着屏幕上出现的皮皮鲁的所有信息。 
        出生时就与众不同,胆特别大;与他同时出生的还有一位孪生妹妹,叫鲁西西,现为著名服装设计师,曾到德国参加过汉堡时装节并获一等奖,目前在中国主持舒克贝塔公司的工作;皮皮鲁小时候就比较淘气,考试成绩较差,不适应循规蹈矩灌输式教育方法;曾有一系列奇遇被登载在报刊上,在孩子们中知名度颇高;成人后出人意料地成为著名物理学家,如未遇到地震事件,稳操诺贝尔物理学奖胜券;地震事件后过隐居生活,后成立舒克贝塔公司东山再起,用自己的名字注册的商标成为世界驰名商标…… 
        突然,安东尼的视网膜细胞死死地咬住了屏幕上新出现的一行字: 
        曾因在深夜带一只老鼠到医院看病而被传媒广泛报道,又因声明诺贝尔文学奖获奖作品《人类,我是你的朋友》是老鼠所著而再次被媒介曝光。 
        老鼠!皮皮鲁果然和老鼠有关系!安东尼一拳砸在桌子上,震得电脑直摇晃,他很兴奋。 
        “您悠着点儿。”站在神探旁边的警察小姐制止安东尼再敲打与精密电脑一脉相承的桌子。 
        安东尼站起来冲女警察一笑,亲了亲她的而颊。安东尼每每取得成就后,最想干的事就是吻小姐,他一贯认为没有成就的男人无权吻小姐,只能给小姐烧饭。 
        警察小姐特受宠若惊。 
        安东尼三步并作两步跑回自己的办公室,他在一张白纸上涂涂画画,他办案时有这种习惯。 
        安东尼断定皮皮鲁的玩具飞碟是一架由他发明的目前还不为世人所知的超级现代飞行器,这架飞行器是由老鼠操纵的。现在,皮皮鲁已将这架飞行器带到德国来了。有三个问题目前还没有答案:一、既然是超级飞行器,为什么还要通过海关检查才入境,直接飞进来不就行了吗?二、皮皮鲁带着超级飞行器来德国干什么?三、皮皮鲁和燕妮到底有没有关系? 
        安东尼基本上可以断定,燕妮的汽车刚才是由那架超级飞行器遥控的,否则无法解释。 
        想到燕妮的汽车,安东尼不由又想到了燕妮。脑海里一出现燕妮的容貌,安东尼就无法再工作了,他的脑细胞大都是雄性的,一想女人就特别兴奋,工作起来绝不会偷懒,一个比一个卖力气。   第196集 
        皮皮鲁有了络腮胡子; 
        舒克和贝塔关于眼镜的讨论; 
        五角飞碟担任空中保镖; 
        少尉和上尉谈论娜莎 
        安东尼想起了燕妮晚上要去看胡安娜演出,而且买的是两张票。安东尼要去剧场看看是哪个小子这么有福气,和如此美丽的姑娘去昕音乐会还要女方去买票。 
        现在对于安东尼来讲,征服燕妮比找皮皮鲁重要。 
        晚饭后,燕妮开始给皮皮鲁化装,他们准备去听胡安娜的音乐会。 
        燕妮先拿来一把假络腮胡子,给皮皮鲁戴上。皮皮鲁对着镜子一照,乐了。 
        “整个一个恩格斯。”皮皮鲁说。 
        “恩格斯?”燕妮没听明白。 
        “你的同胞,思想家,《共产党宣言》的著作权人之一。”皮皮鲁说。 
        “还真挺像。”燕妮弄清了皮皮鲁说的是谁后,进行比较。 
        “应该再给皮皮鲁戴一付眼镜。”舒克提议。 
        燕妮找出一付金丝眼镜,给皮皮鲁戴上。 
        “绝了,显得特有学问。”贝塔在一旁评头论足,“你说这人类是聪明,视力不行,发明出个眼镜戴上,戴着戴着就成了装饰品,连视力贼好的人都戴。我们老鼠视力最差,怎么就没谁发明出眼镜呢?就算自己发明不出来吧,人家有了,拿过来用就是了呗,可老鼠不,自己弄不出来,还死也不用别人弄出来的。所以一辈子翻不了身,世世代代当老鼠。” 
        “你说的也对也不对。”舒克说。 
        “哪儿不对?”贝塔问。 
        “这世界上的事,其实结局并不重要,重要的是过程。”舒克毕竟是当过作家的,话出来就是透着有学问。 
        “这话怎么讲?”贝塔还没听明白。 
        “就拿人说吧,生下来的结局肯定是死,既然知道结局是死,还活什么劲儿呢?不,他们要活,还特想活,活什么?活从生到死这之间的过程。如果一个人生下来马上就死了,这叫死婴,他算不算人呢?当然算!可没人把他当一个完整的人看待。为什么?他没有过程,他的开始和结局之间的距离是零。人家发明眼镜也有一个过程,从发明玻璃起,一直到现在的五花八门的隐形眼镜,经历了一个漫长的过程,他们的乐趣就在这个过程当中。只有这样一步一步循序渐进地走过来,才能适应才能真正享受,这就是过程的重要。你没有这个过程,把人家发明的东西拿过来,等于生下来立马就死了,这样做的结果只能是变态只能是扭曲只能是失落只能是空虚只能是无所适从只能是没着没落只能是尔虞我诈……” 
        “行了,行了,”贝塔打断舒克,“不过你说的确实有道理。这样说来,幸亏老鼠没戴眼镜了。” 
        “舒克,你再说下去。”燕妮说。她和皮皮鲁都听人了迷。 
        “在这个世界上,过程是最重要的。所有幸福所有享受所有意义都是通过过程获得的。最可怕的事就是没有过程。让一个人活完了5岁直接活30岁,这人准疯。”舒克说。 
        “这么说,学生上学跳级也不好了?”贝塔问。 
        “让自己的孩子跳级的父母都是扼杀孩子生命过程的凶手。”舒克下定义。 
        “人类发展也是,一定要有过程。昨天晚上还在拿黑白照片当宝贝,今天清晨就用上了激光视盘。今天上午还扯着嗓子打传呼电话,中午就用上了手机。一点儿过程没有,等于让一个人从幼儿园跳到研究生班,不疯不傻才怪呢。”皮皮鲁顾不上化装了。加入讨论。 
        “没有过程的人,是死婴。没有过程的民族,是短命的民族。”燕妮一边给皮皮鲁化装一边说,“开始有过程,后来没过程的人是夭折的人。开始有过程,后来没过程的民族是半途而废的民族。” 
        皮皮鲁化完装了。同原来的皮皮鲁判若二人。 
        “咱们该走了。”燕妮看表。 
        “我和燕妮开车去剧场看胡安娜唱歌,你俩开五角飞碟跟着,咱们随时联系,你们要注意隐蔽。”皮皮鲁对舒克和贝塔说。 
        “放心吧,我们是你们的空中保镖。”贝塔比谁都兴奋。 
        “出发。”皮皮鲁声音不大,却有分量。 
        舒克和贝塔钻进五角飞碟。 
        “你负责观察,我驾驶,怎么样?”贝塔站在操纵台前,问舒克。 
        “去的时候你驾驶,回来我驾驶。”舒克讨价还价。 
        贝塔只得同意。他在驾驶员的座位上坐好,系安全带。 
        舒克打开电脑屏幕。 
        “皮皮鲁,我是舒克,请回答。”舒克同皮皮鲁联系。 
        “我是皮皮鲁,我们现在在车库里,已经发动了汽车,准备出发。”皮皮鲁回话。 
        皮皮鲁和燕妮坐在一辆超豪华无级变速伊亚牌SEL轿车里,燕妮的车库里有七八辆各种牌号的轿车。燕妮的亿万富翁爸爸生前有收藏轿车的癖好。 
        “我们已经离开别墅,你们出发吧!”皮皮鲁通过通讯系统告诉舒克和贝塔。 
        “起飞!”舒克对贝塔说。 
        贝塔按下起飞按钮。 
        五角飞碟从窗户飞出别墅,升到空中。 
        皮皮鲁和燕妮的汽车已驶上公路。五角飞碟在空中密切注视着公路上的情况。 
        德国防空雷达值班室。 
        一名少尉和一名上尉一边聊天一边注视着雷达扫描荧光屏。 
        “和娜莎吹了?”上尉问少尉。 
        “那姑娘太贪财。她还觉得女人和男人在一起是女人吃亏,像个东方人。让人受不了。”少尉说。 
        “我昨天从报纸上看到一篇报道,说中国女人现在都不这么认为了,你那女朋友也太落伍了。”上尉说。 
        “这是什么?”少尉指着荧光屏上的一个黑点儿,说。 
        五角飞碟。   第197集 
        108号空域的不明飞行物; 
        飞行员用黑话在空中骂人; 
        胡安娜满场飞吻; 
        望远镜里的皮皮鲁   
        上尉翻开飞行图册。 
        “这里没有民航航线。也不是战斗机训练空域。”上尉说。 
        “我问问空管局。”少尉拨电话号码。 
        “是空管局吗?”少尉问。 
        “是。” 
        “请问108号空域现在有飞机飞行吗?” 
        “你等等。我看一下……没有。” 
        “谢谢。” 
        “没有?”上尉开始紧张,“立即报告上司。” 
        有关部门接到了发现不明飞行物的报告。 
        “是军用飞机?”上司问。 
        “还不清楚,体积很小,但信号极强。”上尉汇报。 
        “继续观察,并随时报告它的方位。我派战斗机拦截。”上司说。 
        某空军基地的战斗警报响了。 
        值班的飞行员们从休息室里蜂拥而出,跑向自己的飞机。 
        一排歼击机停放在跑道尽头的起飞线上,座舱盖支撑着,随时恭候飞行员的进入。机身呈流线型俯卧在地面上,机头冲着远方的天空,仿佛天空欠它们许多钱。 
        飞行员们跨人座舱,机械师为飞机起飞做最后的准备。他们的工作不能有一丝疏忽,飞行员的生命在他们的手中。 
        八架歼击机排队滑向跑道,稍停片刻后,它们橐着膀子扯着嗓子跃上天空,尾部甩出淡淡的黑烟。 
        01号飞机是这组机群的头儿。01号带着020304050708恶虎扑食般地去拦截不明飞行物。他们不允许别的带翅膀的东西在属子他们的天空飞。 
        “没发现目标。”01到了不明飞行物出现的方位后找不到目标,他向地面报告。 
        “仔细搜索,它还在雷达荧光屏上。”地面指示。 
        8位飞行员睁着他们的那2.0的眼睛寻找不明飞行物,看不见。 
        舒克发现了头顶上的歼击机群。 
        “这些飞机是来找咱们的。”舒克告诉贝塔。 
        贝塔看了一眼屏幕,这一群歼击机使他想起了海盗。三十年前,海盗领导的歼击机群空袭过舒克贝塔航空公司的机场。 
        “击落他们?”贝塔问舒克。 
        “别开玩笑。问问皮皮鲁。”舒克同皮皮鲁联系。 
        “雷达发现五角飞碟了?战斗机都起飞了?”皮皮鲁得到信息后吃了一惊。 
      “怎么办?”舒克请示。 
      “你们先去剧场的房顶上着陆,等着我们。”皮皮鲁说。 
        五角飞碟先走了。 
        飞行员们死活找不到不明飞行物,有几个飞行员已经开始在空中用黑话骂雷达值班员是疯子。 
        耗光了油,8架飞机无精打采地返航了。 
        剧场里灯火辉煌,人头攒动。 
        燕妮挽着皮皮鲁的胳膊走进剧场,立即引来无数男性的目光。 
        礼仪小姐送给燕妮一张节目单,燕妮用优雅的姿势将节目单递给皮皮鲁。 
        “票价多少?”皮皮鲁问燕妮。 
        “一张300马克。”燕妮说。 
        “这么贵?”皮皮鲁咋舌。 
        “胡安娜是我国收入最高的歌星之一,她的唱片销量经常名列前茅。”燕妮说。 
        皮皮鲁和燕妮找到自己的座位。 
        皮皮鲁将通讯器的耳机插进耳孔。 
        “皮皮鲁,皮皮鲁,请回答。”耳机里传出舒克的呼叫。 
        “我是,请讲。”皮皮鲁极力压低声音,并做出和身边的燕妮交谈的姿势。 
        “我们已在剧场的房顶上着陆,那些歼击机没发现我们,回去了。” 
        “注意观察剧场里的情况,随时保持联系。”皮皮鲁说。 
        “明白。”舒克关闭通讯器。 
        预示演出即将开始的钟声响了。 
        剧场里的灯光渐渐昏暗下来。一束光投在舞台紫红色大幕的局部。 
        一位英俊男子从幕后钻出来,特准确地站在光柱里。 
        全场肃静。 
        “谢谢各位光临胡安娜的音乐会。胡安娜小姐今天晚上将奉献给你崇高的享受。现在,她来了!”男子抬起右臂,同时迅速向左边撤去。 
        大幕懒洋洋地拉开,全场灯光大亮,珠光宝气的胡安娜极潇洒地站在舞台中央。 
        山崩地裂般的掌声喊叫声口哨声。 
        皮皮鲁也被感染了,他情不自禁地也使劲儿鼓掌。人被别人崇拜到这份儿上,死也瞑目了。世界上有几十亿人,能被人崇拜的还到不了一万人。 
        胡安娜频频向观众飞吻。每一次飞吻都换来更猛烈的欢呼浪潮。 
        整个剧场里只有一个人没有鼓掌,举着体积小但倍数并不小的望远镜聚精会神地看。他没看胡安娜,看的是皮皮鲁。 
        安东尼用望远镜观察皮皮鲁,这个同燕妮一起来听音乐会的大胡子男士深深刺伤了安东尼,他心里的滋味儿很难形容,像喝了一杯红药水。 
        安东尼毕竟是神探,有着去伪存真的火眼真睛,他发现大胡子男士的胡子是假的。 
        “化装?”安东尼认为化装听音乐会的人十个有九个是嫌疑犯。 
        安东尼觉得大胡子男士有点儿面熟,好像在哪儿见过。   第198集 
        皮皮鲁攥疼了燕妮的手; 
        五角飞碟透视著名女歌星的胸部; 
        山峰之间藏着秘密; 
        舒克说贝塔盼着加刑   
        胡安娜的第一支歌就把皮皮鲁征服了。皮皮鲁听得如醉如痴,右手紧紧攥住燕妮的手。 
        燕妮也很激动,皮皮鲁如此喜欢一位德国歌星,燕妮高兴。 
        第一曲毕,全场起立欢呼。 
        皮皮鲁也站起来,一边鼓掌一边吻燕妮。胡安娜的嗓子的确好,好就好在特殊,特特殊。 
        特殊。与众不同。从没见过。这三个词是对人对事的最高评价,比伟大伟大多了。千人一面。千篇一律。和别人一样。这三个词是对人对事的最高贬斥。比笨蛋笨蛋多了。 
        胡安娜的歌声说穿了不属于人类,人类的声带发不出这样的声音。大家都发不出来,你发出来了,你就是金嗓子。 
        当胡安娜唱第二支歌时,皮皮鲁心头突然一震,他不由自主地攥疼了燕妮的手。 
        “你怎么了?”燕妮问。 
        “这声音是歌唱家的!没错,肯定是她。”皮皮鲁越听越像。 
        “歌唱家?噢,是罐头小人?”燕妮反应过来,,“罐头小人歌唱家长成大人了。” 
        皮皮鲁摇摇头,他觉得没有这种可能,他仔细观察舞台上的胡安娜。 
        胡安娜风情万种地在舞台上恣意一边唱一边摆臂,她手中的麦克风将她的歌声输送到每一位观众的耳膜里。 
        “你发现了吗?胡安娜的话筒不是对着嘴,而是对着胸。”皮皮鲁有了新发现。 
        燕妮注意胡安娜手中的麦克风。 
        的确,胡安娜的话筒总是对着自己的胸部。只有在不唱歌的时候,她才将话筒对着自己的嘴做做样子。 
          想像力丰富的皮皮鲁已经推理出答案了,他现在需要五角飞碟帮助证实。 
        “是有点儿怪。”燕妮证实皮皮鲁的发现。“她是用胸音唱歌?” 
        “咱们马上就知道了。”皮皮鲁准备和舒克通话,他将喉头送话器卡在自己脖子上。 
        “舒克,我是皮皮鲁,请回答。”皮皮鲁尽量小声呼叫舒克。 
        “我是舒克?请讲。”舒克回答。 
        “遥感胡安娜的胸部。”皮皮鲁下达任务。 
        “你说什么?” 
        “遥感胡安娜的胸部。”皮皮鲁重复。 
        “这……这可是违背五角飞碟的宗旨呀,不是说不能用五角飞碟干不道德的事吗?”舒克刚才看到了皮皮鲁对胡安娜歌声的狂热,他断定皮皮鲁又爱上了胡安娜。 
        “对,不能遥感她的胸部,这也太对不起燕妮了,何况燕妮就坐在他身边!”贝塔发表意见,他支持舒克。 
        “你们说什么呀!”皮皮鲁哭笑不得,“她的乳罩里有问题!” 
        “乳腺癌?”贝塔猜测,他以为皮皮鲁要救胡安娜的命。再当一回英雄。 
        “什么乳腺癌,快遥感,别耽误时间了。”皮皮鲁的口气没有开玩笑的成分。 
        舒克和贝塔对视。 
        “只好遵命吧,你不想看?”贝塔问舒克。 
        “如果是工作,看看也没什么损失。”舒克说。 
        “我也无所谓。其实人类挺怪.把一样的地方露出来,把不一样的地方藏起来。要是我,就把一样的地方藏起来,把不一样的地方露出来,这才叫扬长避短。还是咱们动物省心,一样不一样全露出来,所以动物里没有流氓。”贝塔说。 
        舒克打开遥感仪,遥感目标为胡安娜的胸部。人类身上不一样的地方。 
        荧光屏上出现了胡安娜的胸部,穿着衣服。 
        “衣服上什么也没有。”舒克向皮皮鲁汇报。 
        “透视。”皮皮鲁让舒克往纵深发展。 
        荧光屏上出现了漂亮的乳罩。 
        贝塔目不斜视地盯着屏幕。 
        “乳罩上没发现异常。不过是名牌产品。”舒克向皮皮鲁汇报。 
        “再透视。”皮皮鲁说。 
        “就快给她做心电图了。”贝塔说。 
        舒克熟练地按了几个键。 
        荧光屏上出现了雪白的(禁止),贝塔吹了一声口哨。 
        “别老盯着山上看,看看山下有什么!”舒克推了贝塔一下,话音很激动。 
        两座(禁止)中间的低谷里有一个人,一个火柴棍大小的人,女性,正在唱歌。她站在乳罩上,乳罩显然是特制的,下部很紧,不会让小人掉下去。 
        “歌唱家!”贝塔大喊。 
        “皮皮鲁真行,猜中了。”舒克兴奋。 
        “快告诉皮皮鲁!”贝塔说。 
        “皮皮鲁,皮皮鲁,我是舒克,听见了吗?请回答。”舒克几乎是喊。 
        “快说。”皮皮鲁昕声音就知道有成果了。 
        “歌唱家在胡安娜的乳罩里。歌是歌唱家唱的,不是胡安娜唱的。胡安娜只是对口型,做做样子。”舒克说。 
        “歌唱家在乳罩里空问很小吧?”皮皮鲁为罐头小人歌唱家担忧。 
        “还行,胡安娜的乳罩是大号的,中间有一定的空间,不过也够闷的,胡安娜的乳罩是特制的,我们测了,用刀子都割不破,实际上是囚禁歌唱家的牢房。”舒克说。 
        “如果我被囚禁在那儿,我就咬她。”贝塔狠狠地说。 
        “咱们想办法把歌唱家救出来。”皮皮鲁说。 
        “现在可不容易,这么多人盯着胡安娜。”舒克认为有相当的难度。 
        很怪,刚才皮皮鲁听胡安娜的歌特享受,现在是同样的歌,皮皮鲁听了想哭。       第199集 
        皮皮鲁打消冲上舞台撕歌星衣服的念头; 
        胡安娜有3名保镖; 
        皮皮鲁的汽车被跟踪   
        “怎么样?”燕妮问皮皮鲁五角飞碟遥感胡安娜的结果。她感觉皮皮鲁在生气。 
        “她的歌不是她自己唱的,是歌唱家唱的。”皮皮鲁几乎趴在燕妮耳朵上说。 
        “这怎么可能?”燕妮难以相信。 
        “她把歌唱家藏在乳罩里,是歌唱家在唱歌,她对口型,你仔细看。”皮皮鲁说。 
        燕妮盯着胡安娜的嘴,她不得不承认皮皮鲁的话是正确的,胡安娜的口型与歌声不同步,但不仔细看绝对看不出来。 
        “太卑鄙了,她怎么会这样?!”燕妮满脸通红。她为自己有这样的同胞感到羞耻。 
        “我让舒克立即去救歌唱家。舒克说有难度,现在全场观众的注意力都在胡安娜身上。”皮皮鲁焦急地说,他恨不得冲上去揭穿胡安娜。 
        “你可不能上去。”燕妮看出皮皮鲁在极力克制自己,“你如果上去撕她的乳罩,这些崇拜者能撕了你。” 
        皮皮鲁愤愤然地坐在那里咬牙切齿。 
        “五角飞碟能空运人吗?”燕妮问。 
        “能空运东西,运人有一定的危险性,不能冒这个险。”皮皮鲁说。 
        “皮皮鲁,皮皮鲁,我是舒克,请回答。”舒克呼叫皮皮鲁。 
        “我是皮皮鲁,请讲。”皮皮鲁回答。 
        “我们已经找到了胡安娜的汽车,等一会儿音乐会结束后,在她回家的路上,咱们救歌唱家。”舒克提出解救罐头小人歌唱家的方案。 
        “只有这样了。”皮皮鲁憋着气说。 
        胡安娜在舞台上每扭一次屁股,皮皮鲁就想杀一次人。 
        安东尼的望远镜一直对着燕妮和皮皮鲁。他终于认出坐在燕妮身旁的带着假胡子的男人是谁了。皮皮鲁。 
        皮皮鲁果然和燕妮有关系,而且看得出不是一般的关系。那么大卫的死不仅和皮皮鲁有关,和燕妮也有关系了。安东尼的脑子头一次不够用了。 
        皮皮鲁这个中国物理学家怎么和燕妮认识的?他俩为什么和大卫过不去?在这种关头,他们怎么还有心来看音乐会? 
        安东尼心里还有酸溜溜的感觉,他没想到燕妮会跟一个中国男人。安东尼下决心一定要从皮皮鲁手里把燕妮夺回来。 
        安东尼从望远镜里察觉皮皮鲁和燕妮遇到了麻烦,他俩时而窃窃私语,时而愤愤不平,通过观察口型,安东尼看出这异常与歌星胡安娜有关。 
        他们和胡安娜又有什么关系?皮皮鲁耳朵里塞的那条线是什么?助听器? 
        安东尼想起了微型飞行器,想起,老鼠的餐具,想起了大卫射出的子弹被人拐了弯儿。 
        他拿出手机,同警察局档案中心联系,安东尼要燕妮和胡安娜的资料,他要查查她们和皮皮鲁到底有什么关系。 
        档案中心在3分钟内就给了安东尼答复。答复令神探失望:燕妮和胡安娜均未去过中国,电脑记录显示,她们也不可能和皮皮鲁早就认识。 
        安东尼的两道眉毛像有雌雄一样死死绞在一起,连上个月他破获飞雄公寓浴缸里的粉碎女尸案时两道眉毛也没这么亲热过。 
        他决定和皮皮鲁较量。战场和情场都打。 
        音乐会已接近尾声,胡安娜的情绪也达到(禁止),她开始和紧挨舞台的观众握手。 
        皮皮鲁和舒克通话。 
        “散场后,你们立即盯上她,并向我随时通报她的汽车的位置。”皮皮鲁说。 
        “贝塔驾驶五角飞碟在她的汽车上着陆,我想办法钻进她的汽车里。”舒克说。 
        “不行,还是你驾驶,我钻,说好了的,回去你开五角飞碟。”贝塔插话。 
        “这样不行。最好你们击穿她的汽车轮子,我驾车靠上去,抢走歌唱家。”皮皮鲁说。 
        “你去撕她的上衣?”舒克怕皮皮鲁被警方以性骚扰罪逮捕。 
        “她大约有3名保镖。”贝塔看着荧光屏说。 
        “用五角飞碟的麻醉武器击昏他们。”皮皮鲁说。 
        最后商定,五角飞碟在胡安娜回家的途中使她的汽车抛锚,尔后击昏她和她的保镖。皮皮鲁去解放歌唱家。 
        演出结束了,胡安娜一次又一次地谢幕,追星族们还是不依不饶,死活不走。 
        皮皮鲁和燕妮先走了,他们离开剧场,钻进自己的汽车。观众开始退场。 
        “她在化装室卸装。”舒克随时向皮皮鲁通报胡安娜的情况。 
        “她现在进了卫生间。” 
        “……” 
        “出来了,在穿大衣,准备走了。” 
        皮皮鲁发动了汽车。 
        “你们最好别紧跟着,隔几辆车。”舒克提醒皮皮鲁。 
        胡安娜在歌迷们的夹道欢呼声中,钻进自己的汽车。汽车开动了。 
        皮皮鲁驾车跟了上去。他的车和胡安娜的车中间隔着两辆车。 
        汽车越开越快。皮皮鲁的脸涨得通红,他一秒钟也等不了了,他想像得出歌唱家在胡安娜的那个地方准是度日如年。 
        “准备行动。”皮皮鲁下命令了。 
        “明白。”舒克回答。 
        就在这时,皮皮鲁身后的一辆汽车突然加速,超过皮皮鲁的汽车后,往路边别皮皮鲁的汽车。 
        “注意!扶好!!”皮皮鲁大声告诫燕妮。 
        皮皮鲁紧急刹车。那辆车停在了皮皮鲁的汽车前边。 
        车门打开了,黑暗中走下一个高大的人。他走过来打开皮皮鲁的车门。 
        安东尼。   第200集 
        安东尼赢了第一个回台; 
        皮皮鲁说出安东尼在晚上11点27分的行为; 
        安东尼测出皮皮鲁的大脑数量   
        “皮皮鲁先生,我们终于见面了。”安东尼对皮皮鲁说,“您好。” 
        “我不明白您的话。”皮皮鲁装傻。 
        安东尼伸手拽下了皮皮鲁的假胡子。 
        “对不起,请原谅我的粗鲁。”安东尼将假胡子还给皮皮鲁。 
        “您要干什么?”燕妮对安东尼怒目而视。 
        “您的这位朋友涉嫌一起谋杀案。”安东尼对燕妮说。 
      “谋杀案?”燕妮问。 
      “就是您的姐夫大卫的命案。”安东尼用和蔼的口吻和燕妮说活。 
        正准备攻击胡安娜的汽车的舒克发现皮皮鲁出了意外,他中止了行动。 
        “先回去帮皮皮鲁吧,安东尼这小子又来找麻烦了。”舒克说。 
        “那歌唱家怎么办?”贝塔问。 
        “咱们测出胡安娜的家,她跑不了。”舒克说。 
        五角飞碟调转方向,悬停在皮皮鲁的汽车上空。贝塔和皮皮鲁联系。 
        “皮皮鲁,我是贝塔,你不用回答,我看见那个侦探在纠缠你。我们现在就在你头顶上,你可以和他较量,想怎么着就怎么着。”贝塔的口气牛极了。 
        皮皮鲁什么都不怕了,他决定教训一下面前这个蛮横无礼的家伙。被一个索不相识的人摘下假胡子,算得上是一种羞辱。 
        “大卫是个什么人,你清楚吗?”燕妮问安东尼。 
        “他是个恶棍,觊觎你家的全部财产,死有余辜。”安东尼说的话石破惊天。他不愧是神探.已经查清了。 
        皮皮鲁和燕妮对视。 
        “我可以和皮皮鲁先生单独谈谈吗?”安东尼盯着皮皮鲁问,盯完了皮皮鲁又盯燕妮。 
        “你想抓他?”燕妮问。 
        安东尼摇摇头: 
        “像两个男人谈话那样聊聊。” 
        “可以。”皮皮鲁解开安全带,对身边的燕妮说,“你就坐在这儿等着,没事。” 
        燕妮点点头。 
        皮皮鲁下了汽车,和安东尼走到车头前边1O米远的地方,面对面地站着。 
        舒克和贝塔眼睛都不敢眨,死盯着荧光屏。 
        燕妮屏住呼吸,看着车头前站在黑暗中的皮皮鲁和安东尼。她听不见他们的对话。 
        安东尼和皮皮鲁对视了1分钟,谁的目光都不从对方的脸上移开。 
        安东尼先开口了,他的话令皮皮鲁极为吃惊。安东尼不是一般的侦探,他懂心理学。 
        “皮皮鲁先生,2月1O日是您的生日,还有您孪生妹妹鲁西西,也是2月1O日生日,今天是2月8日。也就是说,后天就是您的生日,请接受我的祝贺。”安东尼说。 
        连皮皮鲁自己都忘了后天应该过生日。 
        “谢谢。”皮皮鲁不得不承认第一个回合自己输了。 
        五角飞碟里的贝塔骂了一句:“这个坏小子,少来这套。”他只用了0.07秒就查出了安东尼的生日,并迅速通报给戴着耳塞机的皮皮鲁。 
        “您的生日和我的挨得很近,是2月16日,我也祝贺您,”皮皮鲁平静地说。 
        安东尼瞠目结舌。 
        他不服输,他查阅过存储在警察局档案中心的电脑里的皮皮鲁的全部资料,他就不信皮皮鲁能查出他的资料,安东尼光足化名就有17个。 
        “您的公司生产的皮皮鲁牙膏的膏体采用的是国际上最高档的原料,叫磷酸氢钙,对吗?”安东尼说完后特得意地看着皮皮鲁。 
        “您昨天晚上看《花花公子》画报时先是非礼画报上的影星进而发展到施暴,11点27分开始,11点35分结束。”皮皮鲁将贝塔传给他的信息转述给安东尼。“这期间,您一共翻了4页,被您骚扰时间最长的那页,是黑人。” 
        安东尼呆若木(又鸟)。他要是再不甘拜下风他就是王八蛋。他自己这么想。 
        “您现在想,您要是不认输您就是这个。”皮皮鲁伸出右手,漂亮地模仿着甲鱼。 
        贝塔在遥感安东尼的思维。 
        安东尼什么也不敢想了。 
        “皮皮鲁,请注意,安东尼的手枪被我运到你的兜里了,你还给他。”贝塔玩上瘾了。 
        皮皮鲁把手插进兜里。 
        安东尼反应极快,他以为皮皮鲁在掏枪,他几乎是与皮皮鲁同时把手伸进西服里边掏枪。枪套是空的。 
        “这是您的枪,还给您。”皮皮鲁将手枪递给安东尼。 
        安东尼愣是迟疑了3分钟才接枪。 
        他现在确信无疑皮皮鲁的那架小型飞行器是超现代化器物了,他也彻底明白了上司为什么要不惜一切代价留住皮皮鲁。 
        表面上看,每个人都是一个脑袋里装着一个大脑,实际上,有的人的脑袋里装着100个大脑,有的人的脑袋里一个大脑也没有,只有一个空壳。同样一个头,人比人,气死人。比什么?比大脑的数量。本来,大脑和脑袋的数量上帝是按1比1的比例分配给人类的,可后来上帝发现有的人不爱使用大脑,上帝最珍惜资源,他就将闲置的大脑调拨给爱用脑子的人。于是,聪明人由于大脑越来越多而越来越聪明,笨人由于没有了大脑而一笨无余。天才是集纳了众多人的大脑才进化为天才的,蠢才是将自己的大脑进贡给了天才才退化为蠢才的。天才不必骄傲,上帝由于你爱用脑子,将众人的大脑都调拨给你,你才成为天才的。蠢材也不必自卑,天才的智慧中,有你一份,反正你的大脑闲着也是闲着,不如让天才废物利用,推动历史,推动人类。 
        安东尼清楚,皮皮鲁的脑袋里最少有300个大脑。发达国家的宗旨都一样:敞开大门欢迎有两个以上大脑的人类成员。将只有空壳脑袋的人拒之于国门之外,离得越远越好。 
        安东尼拼死也要改变皮皮鲁的国籍。   第201集  
        安东尼的冠状动脉会说话;  
        天才的标志是什么;  
        现在的苍蝇和蚊子上辈子都是人      
        “皮皮鲁先生,您是超天才。”安东尼不是恭维,他的声音是直接从冠状动脉发出的。  
        “谢谢。”皮皮鲁也不客气,他了解自己的价值。他有充分的自信。  
        成功有三个必须的条件,缺一不可:才能。自信。机会。  
        “您如果留在我的国家,能够获得更多的发展机会。”安东尼开始使用朋友之问的语气和皮皮鲁说话。  
        “您说什么?”皮皮鲁以为自己听错了。  
        安东尼重复一遍。  
        “留在您的国家?”皮皮鲁听清了仍以为没听清。        
        安东尼肯定地点头。  
        “为什么?”皮皮鲁问。  
        “您是天才,我的国家需要天才。”安东尼的眼睛亮得发光,虽然是在黑暗中。  
        “您怎么知道我是天才?”皮皮鲁是明知故问。  
        “实话跟您说吧,”安东尼觉得和皮皮鲁这种超级天才谈话还是坦诚点儿好,“我的国家交给我的任务就是不惜一切代价留住您。”  
        “我如果不同意呢?”皮皮鲁问。  
        “您不会不同意。”安东尼说。  
        “为什么?”皮皮鲁对于安东尼说这话时的肯定语气表示惊讶。  
        “天才都希望获得更多的发展,我们这儿能给您提供这种机会。”安东尼说。  
        皮皮鲁摇头:  
        “真正的天才的标志只有一个,就是改变。改变人类原有的物质。改变人类原有的思想,改变人类原有的行为。真正的天才最希望自己生存在一个需要改变的地方,这样他才能施展雄才伟略。从穷国往富国跑的人当中,绝对不会有天才。因为对于天才来说,穷国是千载难逢的施展才能的环境。”  
        安东尼在心里点头。  
        “不管您同意否,您都走不了了。”安东尼告诉皮皮鲁,“我们已经封锁了边境。如果您执意要走,        我们就以涉嫌谋杀拘捕您。”  
        “如果我想走。你们拦不住我。”皮皮鲁平静地告诉安东尼。  
        “这我知道。不过,我想您即使回了国,我们也可以通过外交途径以谋杀罪要求贵国政府引渡您。据我们了解,自地震事件后,您的处境并不十分好,您被引渡回来的可能性几乎是百分之百。”安东尼说。  
        皮皮鲁这才意识到问题的严重,安东尼说得对,即使他回了国,也会被送回来的。  
        “留下吧,皮皮鲁先生,我们为您提供第一流的研究设施,诺贝尔物理学奖在等着您。”安东尼全面进攻。  
        皮皮鲁用非常慢但非常有力度的频率摇头,然后,他嘴里发出音量不大但却掷地有声的话:  
        “我从生下来开始,思想每天都在变化,细胞也在变化,生活也在变化,没有变化就没有人类。但有一样东西我不让它变化,就是我的国籍。降生在哪儿,是上帝的安排。改变国籍是违背上帝的旨意的。这地球上没有天生就富的国家,弃穷国籍投富国籍的人都是懦夫,尤其是那些将生身父母扔在穷国自己投奔富国的人更是不齿于人类。安东尼先生,您拿正眼瞧过那些穷国移民吗?您会让他们去您家作客吗?”        
        安东尼不得不承认皮皮鲁是一个真正的男子汉。国家穷,仍然不离开,这才是人。上帝创造人类,就是让人类把属于自己的领域变富的,上帝允许有的地域先富起来。如果还没富的地方的人都跑到已经富的地方去,上帝造人干什么?还不如多造一些动物,向人类供给蛋白质。上帝之所以造人,就因为人能改变一切。不能改变自己生活环境的人,下辈子上帝准不让他再投人胎了。现在我们看到的苍蝇、蚊子和牛猪马什么的,上辈子可能都是抛弃穷国投奔富国的人,因为他们没有将穷国变成富国,他们失去了做人的意义,将珍贵的人生变成了动物的一生,上帝不再给他们第二次机会了。安东尼这样想。  
        “还有事吗?”皮皮鲁问安东尼。  
        安东尼从遐想中醒过来,他对皮皮鲁说:“说实话,我很钦佩您。您好自为之吧,请多保重。在这个世界上,有本事和不幸往往是同义语。”  
        “谢谢。”皮皮鲁清楚自己现在的处境。他想离开这个国家,非常难。更难的是,当他回国后,仍会被送回来。  
        皮皮鲁回到自己的汽车里。  
        安东尼注视着皮皮鲁,他其实挺希望皮皮鲁回国,他想得到燕妮。他不信燕妮会跟皮皮鲁去中国。  
      皮皮鲁发动汽车,他打开车灯。        
        安东尼仍然站在原地,他那高大的身影暴露在灯光下。皮皮鲁变换了一下远近灯光,以此向安东尼传递一个信息,皮皮鲁不讨厌你安东尼。的确,第一次和安东尼面对面交锋,皮皮鲁觉得他也算是个男人。  
        “皮皮鲁,我是舒克,咱们现在去哪儿?”舒克请示皮皮鲁。  
      “先回家。”皮皮鲁说。  
      “那歌唱家呢?”舒克问。  
      “你们查清胡安娜的住处了?”皮皮鲁问。  
      “查清了。”舒克回答。  
      “咱们先回家,现在去胡安娜家,我担心安东尼会跟踪。”皮皮鲁说。  
        “明白。”舒克说。  
        皮皮鲁驾车驶向乡间别墅,皮皮鲁从反光镜里看到安东尼果然在后边跟着。  
        “帮我甩了他。”皮皮鲁对舒克说。  
        舒克用五角飞碟帮助皮皮鲁的汽车加速。  
        时速表已显示皮皮鲁的汽车每小时300公里。  
        安东尼在后边叹了口气,他不得不承认大脑的差别。他安东尼就发明不出这样的器物。          第202集  
        总统助理同意清晨拘捕皮皮鲁;  
        贝塔吃夜宵时喝酒;  
        窗台上的遐想    
        安东尼回警察局向局长汇报。  
        局长立即向上司汇报。  
        总统助理立即召集紧急会议。安东尼参加r这个高级别的会议。  
        “先请安东尼介绍情况。”主持会议的一位将军说。  
        “据我调查,皮皮鲁和大卫的死没有直接关系。大卫是一个亿万富翁的女婿,为了得到岳父的财产,他先害死了妻子,尔后又准备杀害亿万富翁的另一个女儿,也就是他的小姨子燕妮,以攫取全部财产。不知什么原因,皮皮鲁一下飞机就被卷入此事件中,皮皮鲁出现在大卫的家中。”安东尼说。        
        “皮皮鲁从前认识这位叫燕妮的小姐吗?”总统助理一边擦眼镜一边问。  
        “据我现在掌握的资料,他们从前不认识。”安东尼说,“当大卫准备杀燕妮和皮皮鲁时,皮皮鲁使用极其先进的武器——可能就是那架在海关被扣的小型飞行器——调转了大卫射出的子弹的方向,使人卫和手下当场毙命。”  
        众人点头。  
        “现在的问题是,皮发鲁来咱们国家干什么?他的目的是什么?据我今天下午和傍晚观察,皮皮鲁对歌星胡安娜表现出了特别的兴趣。”安东尼将他偷拍的皮皮鲁观看胡安娜演出的录像带放给与会人士看。  
        “皮皮鲁的胡子这么长?”总统助理问。  
        “是假胡子,他化装了。”安东尼解释。  
        “坐在皮皮鲁身边的小姐是谁?”警察局长注意到了燕妮的美貌。  
        “她就是燕妮,亿万富翁的女儿,现在和皮皮鲁的关系非同一般。”安东尼的话里有醋的成分。  
        “皮皮鲁干吗一边看着胡安娜一边咬牙切齿?”总统助理问。  
        “不清楚。”安东尼说。  
        “他和胡安娜认识?”  
        “据我了解,从没见过面。”安东尼说。        
        “的确很怪。”总统助理摇头。  
        “我们的雷达也发现了不明飞行物。飞机起飞后找不到目标。”一与会者说。  
        “正是皮皮鲁外出的时间。”安东尼说。  
        “你同皮皮鲁已经正面接触了?”总统助理问安东尼。  
        安东尼点点头:  
        “我动员他留下,而他坚决不干。”  
        警察局长说:  
        “给他我们的国籍他都不干,傻子。”  
        总统助理:  
        “不,这可不是傻子。改变国籍的人才是傻子。”  
        “我估计,他执意要走,咱们拦不住。”安东尼说。  
        “除非他动武。他如果使用武力离境,咱们就要求他的国家把他送回来。他总不能把自己变小塞进那架小飞行器吧!”一专家说。  
        “我主张明天清晨拘捕皮皮鲁,趁他睡觉的时候,连同他的小飞行器一起。”总统助理说,“咱们一定要留下他。皮皮鲁绝对是一个难得的人才。真不知他的国家的学校是怎么培养出这样的人才的。”  
        “皮皮鲁在学校不是听话的学生,考试经常不及格。”安东尼说。  
        “天才上学时循规蹈矩的少。”总统助理若有所        思地说。  
        “明天清晨,拘捕皮皮鲁。同时从现在起,派警员严密监视胡安娜的住所。”警察局长报方案。  
        总统助理批准。  
        “注意,绝对不准伤害皮皮鲁一根毫毛,他的每一根毫毛所凝聚的智慧比一个普通人的整个大脑都多。”总统助理强调。  
        皮皮鲁一行回到乡间别墅,夜已深了。  
        他们围在餐桌旁吃夜宵。  
        “胡安娜怎么会这样?”燕妮吃东西一点儿也不香。  
        “歌唱家被憋在那儿可够受罪的。”贝塔一边喝酒一边对歌唱家的处境表示同情。  
        “咱们什么时候去救她?”舒克问皮皮鲁。  
        “明天晚上。今天好好休息,你们两个不要上楼睡觉,就和我们住一间卧室,一个睡五角飞碟里,另一个值班。”皮皮鲁知道自己已成为这个国家密切注意的人物,人家随时可能对他采取行动。科技发达的国家对高科技最敏感。越是发达的国家越怕别的国家发达,越怕自己不再发达。  
        “我先值班,舒克在五角飞碟里睡觉。”贝塔有点儿醉了,他要求守夜。  
        “救出歌唱家可能不会很难,真正难的是咱们怎么出境。”皮皮鲁紧皱眉头。        
        “车到山前必有路。”贝塔一仰脖,又喝了一杯酒。  
        吃完夜宵,皮皮鲁和燕妮抱着五角飞碟走进卧室,舒克和贝塔在五角飞碟里。  
        皮皮鲁将五角飞碟放在酒柜上边。贝塔从五角飞碟里钻出来。  
        “你们睡吧,我放哨。”贝塔对皮皮鲁和燕妮说完后将五角飞碟的舱门从外边关好。  
        舒克在五角飞碟里睡觉,属于战斗值班性质。  
        皮皮鲁和燕妮熄灯就寝。  
        贝塔坐在窗台上看外边的夜色。望着布满星星的天空,贝塔想起了30年前他和舒克驾驶宇宙飞船去双子星球的情景。宇宙里有生命的星球比比皆是,每分钟都有新的星球诞生,每分钟都有旧的星球死亡。每分钟都有新的人类诞生,每分钟都有旧的人类死亡。  
        人类是地球上最爱折腾的生命,但不管怎么折腾,到头来结局和其他动物都一样——死亡。人类成员之间也是这样,从出生起,目标全都一致:死亡,不管你在这个过程中是百万富翁还是乞丐。贝塔坐在窗台上看夜空,觉得宇宙特有意思。          第203集  
        吃过爱因斯坦剩下的面包的老鼠;  
        老鼠科学院的成果;  
        贝塔碰上了鼠小姐;  
        皮皮鲁的梦乡    
        爱因斯坦在德国居住时,他的寓所里有一只老鼠。这只老鼠最爱干的事,就是每天躲在书柜上看爱因斯坦思索。  
        爱因斯坦思想时,双目炯炯有神,仿佛将整个宇宙都装进自己的大脑里。他的脑细胞无时无刻不在与宇宙较量,好像不把宇宙的奥秘剥得体无完肤决不罢休。  
        爱因斯坦发现相对论时,这只老鼠一直注视着全过程。爱因斯坦的大脑发射的脑电波非常强大,那老鼠受到了爱因斯坦脑电波的辐射,沉睡的脑细胞日益活跃,竟然也开始思索宇宙间的事。        
        一天,老鼠在餐厅里吃完爱因斯坦吃剩的一块儿面包后,悄悄来到爱因斯坦的书房。  
        老鼠爬上书柜,居高临下地观察正在看书的爱因斯坦。老鼠忽然有了一个奇特的想法,从上边往下看爱因斯坦,它觉得爱因斯坦挺渺小。老鼠终于明白人类为什么看不上老鼠了,就因为老鼠小。原来人类和老鼠从一诞生起,就不处于公平竞争的同一起跑线上。所以人类现在这么蛮横,所以老鼠现在这么受气。  
        老鼠一边看爱因斯坦一边产生了一个伟大的构思,它要发明一种把人类变小的东西,变得和老鼠一样大小,然后公平竞争。  
        这只老鼠立即开始了行动,它偷阅了爱因斯坦的许多书籍,更重要的是,它看了爱因斯坦不少笔记,它从爱因斯坦的思维中吸取智慧。  
        后来,爱因斯坦受希特勒迫害,到美国躲灾去了。老鼠没能跟去,留了下来。它继续致力于将自己的伟大设想变成现实。  
        在它去世前,没能完成这一宿愿。老鼠将遗愿传给了儿子。儿子终生努力后,又传给了孙子。孙子玩命地继承爷爷的遗志,把命玩完后又传给了自己的爸爸的孙子。  
        这一家老鼠世世代代孜孜不倦地在研究能使人变小的物质,它们的脑细胞里有爱因斯坦脑电波影        响的遗传。爱因斯坦的后代没什么出息,可爱因斯坦住所的老鼠的后代却日见出息。  
        经过了数百代的不懈努力和前赴后继,加上它们生活的国家有幸发展为高科技国家,为它们的科研提供了应有尽有的设备,它们终于逼近成功了。  
        爱因斯坦家的老鼠的后裔,现在就定居在燕妮的乡间别墅里。这已经是一个有几百只老鼠的大家族了,它们拥有各种科学实验设备,是一个十足的科学院。它们禀承祖先的遗志,为完成将人类变得和老鼠一样小这一伟大愿望而奋斗终身。它们知道自己快成功了,这是经过了几百dai kao鼠努力的结果。它们准备在成功后立即着手研制核武器,它们有这方而的遗传细胞,它们的那位老祖宗和爱因斯坦同居一室时,爱因斯坦老想核武器。  
        老鼠们发明出了一种绿豆大小的名为“微缩粒”的药粒,按照设想,只要将微缩粒放在人的鼻孔里,两分钟后,人就会缩成老鼠那么小。  
        老鼠们将试验的时间定在今天夜里,也就是皮皮鲁看胡安娜演出的这一天。  
        老鼠们原计划拿燕妮的男佣当试验品,它们没想到在临近试验日时别墅里又来了两位人类,它们选定皮皮鲁为试验对象。  
        “这人大脑不一般,不亚于爱因斯坦。如果他能变小,人类的其他成员就都不会有抗药性了。”老鼠        科学院院长说。  
        众老鼠科研人员同意。  
        皮皮鲁和燕妮熄灯后,老鼠们开始行动。  
        “窗台上有一只咱们的同胞,像是给那男人放哨呢。”一负责安全保卫的老鼠说。  
        “老鼠怎么会和人类为伍呢?”科学院院长问。  
        “不知道,是外国老鼠。”  
        “你去缠住他。”科学院院长对一只漂亮的鼠小姐说。  
        “保证完成任务。”鼠小姐说。她奶奶的奶奶的奶奶的奶奶---一是爱因斯坦家的女鼠。爱因斯坦喜欢女人,鼠小姐的祖奶奶从爱因斯坦身边的女人身上学到了不少先进经验。  
        贝塔正坐在窗台上看夜景,有人拽他的衣服。  
        贝塔扭头.身边是一只迷人的鼠小姐。  
        “你好!”鼠小姐笑眯眯地说。  
        “你……好……”贝塔有点儿手足无措。  
        “你从哪儿来?在这儿干什么?”鼠小姐问。  
        “我……”贝塔还有警惕性,他没告诉鼠小姐,“你是谁?”  
        “我就住在这栋房子里。”鼠小姐说。  
        “噢.是房东。”贝塔说。  
        “你真帅。”鼠小姐开始奉承贝塔。  
        “你也很漂亮。”贝塔还有点儿醉意。        
        女鼠往贝塔身上靠了靠。  
        贝塔的心脏立刻加速。  
        “你……”贝塔开始浯无伦次。  
        “我喜欢你。”鼠小姐娇滴滴地说。  
        贝塔晕了。  
        紧接着就发生了顺理成章的在地球上重演过无数次的故事。  
        透过窗帘,老鼠科学院的院士们已经看见窗帘后边的进展。院长下令负责投放微缩粒的老鼠出击。  
        携带微缩粒的老鼠迅速爬上皮皮鲁的床。它绕过燕妮,来到皮皮鲁身边。  
        皮皮鲁正在梦乡中。他梦见自己的童年,梦见他和鲁西西在一个红沙发里发现了一座音乐城,还梦见熊猫鲍尔和幻影号汽车。  
        老鼠站在被子上,将一颗微缩粒放在了皮皮鲁的鼻孔前边。  
        老鼠科学院院长开始看表计时。  
        贝塔正和鼠小姐缠绵。  
        舒克正在五角飞碟里酣睡。  
        燕妮呼吸均匀,是一个真正的睡美人。  
        灾难都是从平静中孕育出来的。          第204集  
        直冲云霄的电梯;  
        皮皮鲁失踪;  
        贝塔在五角飞碟的荧光屏上出演三级片    
        燕妮和皮皮鲁走进一座豪华的摩天大楼,他俩走进电梯,皮皮鲁按关门按钮。  
        电梯门关上后,电梯突然加速向上升去,速度之快,令皮皮鲁和燕妮惊慌失措。  
        “怎么回事?”皮皮鲁试图使电梯停止运行,他几乎按了操纵盘上的所有按钮,都是徒劳。  
        电梯不依不饶地一个劲儿向上升,燕妮保守估计,电梯早已脱离大厦了。  
        “快和五角飞碟联系!”燕妮急中生智,她提醒身边的皮皮鲁。皮皮鲁从兜里掏出微型通讯器,呼叫舒克和贝塔:  
        “舒克!舒克!我是皮皮鲁!请回答!”        
        没有回音。  
        “贝塔!贝塔!我是皮皮鲁!请回答!”  
        没有答复。  
        燕妮从电梯门缝儿往外看,外边已是蓝天白云了。  
        燕妮回头告诉皮皮鲁。皮皮鲁不见了!  
        电梯里只剩燕妮一人,她绝望地大叫:  
        “皮皮鲁!皮皮鲁!舒克!贝塔!”  
        正在窗帘后边与鼠小姐风风雨雨的贝塔听见燕妮的呼叫,忙撇下鼠小姐,朝燕妮的床上看去。  
        燕妮被噩梦吓醒,她坐起来,满头大汗。  
        “做噩梦了?没事吧?”贝塔撩起窗帘一角问燕妮。  
        “没……没……事……”燕妮醒来方知是梦,惊魂未定地擦汗。  
        擦完汗,她才发现身旁没有皮皮鲁。  
        “贝塔,皮皮鲁去卫生问了?”燕妮问。  
        贝塔同燕妮对完话,回身发现鼠小姐不见了。正在这时,他听到燕妮问皮皮鲁。  
        “没有啊,皮皮鲁不在床上?”贝塔有不祥的预感。  
        燕妮打开灯,皮皮鲁的被子是空的!  
        燕妮楼上楼下找了一遍,没有皮皮鲁。  
        贝塔叫醒五角飞碟里的舒克。        
        “什么?皮皮鲁失踪了?”舒克不信,以为贝塔和燕妮联袂逗他。  
        “真的!”贝塔急了。  
        “你不是在值班吗?你干什么呢?”舒克问贝塔。  
        “我……”贝塔脸红了,“反正我没睡觉。”  
        “我遥感一下,看看你刚才干什么了。”舒克转身要往五角飞碟里走。  
        “别,别,”贝塔慌了,“我就是……稍微……松弛了一下。”  
        “啊——”燕妮的喊声极其恐怖。  
        舒克和贝塔朝燕妮看去,燕妮指着床上继续惊讶。床上什么也没有。  
        “怎么了?”舒克问燕妮。  
        “你看,皮皮鲁!“燕妮指着空无一人的床说。  
        舒克从桌子上跑到床上,他愣了,床单上躺着熟睡中的皮皮鲁,和舒克的体积一样大小的皮皮鲁!  
        “这……”舒克目瞪口呆。  
        贝塔也过来,他看看皮皮鲁,又看看舒克,再看看燕妮,然后在自己腿上狠命掐了一把.疼得他大叫。  
        不是在梦中。  
        “怎么会这样?”贝塔看着和自己一般大小的皮皮鲁,脑子里一片空白。  
        燕妮哭了。        
        “他还活着?”贝塔问舒克。  
        舒克将手掌放在皮皮鲁鼻子前边。  
        “活着。睡得真死。”舒克说。  
        “叫醒他?”贝塔征求大家意见。  
        舒克推皮皮鲁。  
        皮皮鲁醒了,他睁开眼睛,眼前是硕大无比的舒克和贝塔。  
        “你们开什么玩笑?吓死我了。”皮皮鲁边说边坐起来。  
        “你看看四周!不是我们变大了,是你变小了。”贝塔提醒皮皮鲁。  
        皮皮鲁往四周一看,他傻眼了。  
        摩天大厦般的衣柜,足球场一样的床,体育馆似的电话机,还有巨人燕妮。  
        “这……这……”皮皮鲁不知所措,“是做梦吗?”  
        “绝对不是。”贝塔一字一句地说。  
        皮皮鲁看见放在床头柜上的他的手表像一辆汽车。他这才知道世界上的东西没有大也没有小,没有多也没有少,没有富翁也没有穷鬼,就看你和谁比了。  
        “皮皮鲁!”燕妮跪在床边,泪眼看着变小的皮皮鲁。  
        “这……是怎么回事?”皮皮鲁看着心爱的燕妮变成了巨人,他终于明白什么叫般配了。  
        “我醒来发现你不在床上,我们找遍全楼也没有你,后来我看见你变小了躺在床上。”燕妮泣不成声。  
        皮皮鲁思索。  
        屋里静极了。  
        “用五角飞碟测试!”皮皮鲁智力依然超群,他想出了水落石出的办法。  
        舒克和贝塔百米赛跑似地往五角飞碟里钻。贝塔抢先坐在操纵台前,他打开遥感仪。  
        荧光屏上开始依次显示从昨晚9时起这栋别墅里发生的所有事情。  
        老鼠科学院。爱因斯坦。微缩粒。鼠小姐和贝塔。老鼠将微缩粒放在皮皮鲁鼻前……  
        贝塔无地自容。  
        舒克瞠目结舌。  
        “你中了人家的美人计了。”舒克瞪了贝塔一眼。  
        “我是酗酒驾车,过失罪,过失罪。”贝塔寻找能为自己辩解的词汇,他心里清楚自己栽了。          第205集  
        贝塔给亲人下定义;  
        皮皮鲁签发逮捕证;  
        贝塔破门而入;  
        鼠小姐再施美人计    
        “你出去告诉皮皮鲁和燕妮!”舒克关上电脑遥感仪,推贝塔。  
        “我……还是……你去吧。”贝塔往后躲。  
        “不行,一定要你自己去说。”舒克要让贝塔记住这次教训。  
        “舒克……你说咱俩的关系……是朋友还是亲人?”贝塔一脸的苦大仇深。  
        “是亲人。”舒克将自己和贝塔的关系定了性。“这又怎么样?”  
        “什么叫亲人?亲人就是能互相容忍缺点。什么叫外人?外人就是不能容忍对方的缺点。有的夫妻或父母在家为配偶或子女的一点儿小毛病打得死去活来,到了单位对同事的缺点却宽容得一塌糊涂,这是人吗……”贝塔拖延时间。  
        “得了得了,我去向皮皮鲁说。”舒克站起来。  
        “这就叫亲人。”贝塔说完义摇摇头,“不过,我确实犯了不可饶恕的错误,那个德国女鼠也忒坏了点儿。”  
        舒克瞪了贝塔一眼,朝五角飞碟门口走去。  
        贝塔躲在五角飞碟舱门里边听舒克说话。  
        “怎么回事?”燕妮看见舒克从五角飞碟里出来,迫不及待地问。  
        舒克来到皮皮鲁身边,现在他和皮皮鲁的体积一样大。  
        “你说吧。”皮皮鲁从舒克的脸上看出问题有一定的严重性。  
        “爱因斯坦家有一只老鼠……”舒克从头讲起。  
        “爱因斯坦?”皮皮鲁不明白自己变小和爱因斯坦有什么关系。  
        舒克从爱因斯坦家的老鼠一直讲到燕妮家的老鼠科学院以及贝塔中了美人计和皮皮鲁被微缩粒变小等等。  
        皮皮鲁呆若木(又鸟)。  
        燕妮满脸是泪。  
        “皮皮鲁还能再变大吗?”燕妮哽咽着问舒克。        
        舒克摇摇头:“不知道。”  
        贝塔从五角飞碟里一步一步蹭到皮皮鲁身边,说:“皮皮鲁,我对不起你,我真该死。”  
        皮皮鲁看看贝塔,他叹了口气,拍拍贝塔的肩膀:  
        “没关系,这不能怪你。现在咱们一样大了,不是更方便当朋友了吗?”  
        贝塔鼻子一酸,眼圈儿红了。  
        “那我呢?”燕妮颤抖着声音问。  
        皮皮鲁抬头着着面前摩天大厦般的燕妮,他的心像被飞镖刺中了,现在他如果想吻燕妮,只能够着她的脚脖子。  
        贝塔意识到,皮皮鲁被微缩粒变小的最大损失,就是打破了他和燕妮之间的平衡。这个平衡不是感情上的,而是身体上的。  
        “我去把他们抓来!”贝塔突然怒不可遏地往五角飞碟里跑。  
        皮皮鲁没有拦贝塔,他坐在篮球场那么大的枕头上,对舒克说:  
        “你去帮他。”  
        舒克尾随贝塔进人五角飞碟。  
        贝塔遥感到老鼠科学院的准确方位,出口在书房的一排书柜后边。  
        “我去抓它们,你用五角飞碟帮我。”贝塔对身边的舒克说。  
        舒克点点头。  
        老鼠科学家们正在庆祝微缩粒试验成功。它们欣喜若狂,弹冠相庆。  
        “终于可以告慰祖宗了!”院长将杯中酒转着圈儿撒在地上,给地下的祖宗们喝。  
        “人类全变小后,咱们和他们较量一下,公平竞争,人类不一定是咱们老鼠的对手。”一鼠科学家说。  
        “咱们的数量比他们多几十倍。”勾引贝塔的有献身精神的鼠小姐说。  
        “微缩粒使用不方便,咱们得抓紧将它改造为气体,或者像艾滋病一样通过那个传染。”院长若有所思地说。  
        “第二个方法好,人类离不开那个。就叫微缩艾滋菌吧,传上就变小。”一鼠教授说。  
        一只负责放哨的老鼠跑过来。  
        “那只给人类站岗的老鼠朝咱们的洞口走来了。”  
        “就他自己?”院长问。  
        哨兵点头。  
        “把大门关死,不让他进来。”院长说。  
        贝塔借助五角飞碟的威力,轻而易举地就破门而入了,他又接连打倒了十几只阻挡他的同胞。  
        现在,贝塔站在院长面前。        
        “祝贺你呀,爱因斯坦家的老鼠的杰出后代。”贝塔双臂抱在胸前,冷笑道。  
        院长知道面前这位同胞不一般了,他什么都知道。  
        “你想怎么样?”院长问贝塔。  
        “你们把我的朋友变小了,我限你们在十分钟内再把他变回去!”贝塔厉声喝道。  
        院长摇摇头:“这不可能,我们没研制过变回去的药。”  
        “那您就别怪我不客气了。你们全部都跟我去见我的朋友,听任他发落。”贝塔说。  
        “我们要是不去呢?”院长问。  
        “不去?”贝塔看着院长屁股下边的椅子说,“就像这把椅子。”  
        院长身下的椅子着火了。舒克的杰作。  
        老鼠们大惊失色。  
        “你为什么帮助人类呢?咱们才是一家人呀!”鼠小姐再次向贝塔展开攻势。  
        “真正害你的准是你的同胞。”贝塔瞪了鼠小姐一眼。  
        鼠小姐面红耳赤。          第206集  
        皮皮鲁决定将老鼠科学院流放到北极;  
        舒克为同胞辩护;  
        燕妮的惊人之举    
        “走吧?”贝塔对异国同胞们说。  
        “全去?”院长想保存实力,哪怕只留一位同胞也行,就能保存科学成果。  
        “对。”贝塔点头。  
        老鼠科学家们排队让贝塔点名。一位教授在院长的暗示下躲了起来。  
        “出来吧。”贝塔在舒克的提示下冲着教授躲藏的地方喊话。  
        院长认定贝塔的祖先是耶稣家的老鼠。  
        贝塔押着老鼠科学院的所有老鼠来到皮皮鲁面前。  
        燕妮没想到自己的别墅里有这么多老鼠,她皱眉头:  
        “谁是头儿?”皮皮鲁开门见山。  
        院长站出来。  
        “你们想把人类全都变成我这样儿?”皮皮鲁盯着院长的眼睛问。  
        “是的。”  
        “你的试验成功了?”  
        “是的。”  
        “你高兴得太早了,因为你的运气太差了。你碰上了我,你的阴谋到此为止了。”皮皮鲁说。  
       “……”院长沉默了,它知道皮皮鲁准比贝塔还厉害。  
        皮皮鲁要拯救人类,他不能让这帮老鼠的阴谋得逞。  
        “贝塔,用五角飞碟摧毁老鼠科学院和所有设备以及微缩粒。”皮皮鲁向贝塔下指示。  
        “明白。’贝塔觉得特出气,“他们呢?”  
        “用五角飞碟把它们流放到北极去。”皮皮鲁丝毫没觉得对不住爱因斯坦。  
        贝塔跑进五角飞碟将皮皮鲁的决定告诉舒克。  
        舒克反对。  
        “为什么?”贝塔不明白舒克干吗反对。  
        “我去同皮皮鲁说。”舒克跑出五角飞碟。  
        老鼠科学院的院士们看到又出现了一位同胞, 感到惊讶。  
        “皮皮鲁,你这么做不公平。”舒克说。  
        “不公平?”皮皮鲁没想到舒克会反对他决定。  
        “地球上的一切生物都是处于竞争之中,上帝的旨意是让所有生物公平竞争,可现在的确不公平,人那么大,老鼠那么小。老鼠家族好不容易出了这么个科学院,你又要利用优势摧毁它。你这叫不正当竞争。你们人类之间有反不正当竞争法,地球上的所有生物之间也应该有反不正当竞争法……”舒克越说越激动。  
        皮皮鲁一时无言以对。  
        院长看着皮皮鲁说:  
        “其实任何事都有利有弊。你别以为把你们人类变小是坏事。我计算过,照人类现在的速度繁殖下去,再有83年,地球上的食物就不够吃了,还有能源.还有住房,还有就业,统统会出现危机。人变小以后,这一切同题都迎刃而解了,全市人有这样20栋别墅就够住了。不变小,还能再折腾83年,变小了,能再在地球上生存480年。你不但不应该摧毁我们,还应该给我们树碑。”  
        皮皮鲁不得不承认老鼠院长和舒克的话都有道理。的确,世界上发生的每一件事都是既有利又有弊,人大可不必生气和高兴。其实,每一次高兴都包含着生气,每一次生气也包含着高兴。乐中有悲。悲中有乐。乐极生悲。悲极生乐。  
        “你同意了?”舒克看出皮皮鲁对自己的决定动摇了。  
        “你的话有道理,随它们去吧。”皮皮鲁冲爱因斯坦家的老鼠的后代们挥挥手。  
        老鼠们欣喜若狂,它们集体冲皮皮鲁鞠了一躬,转身就走。  
        “等等!”一直沉默的燕妮说。  
        老鼠们站住了。  
        “你们还有微缩粒吗?”燕妮问老鼠们。  
        “有……”院长不知燕妮是什么意思。  
        “给我一粒。”燕妮说。  
        “干什么?”皮皮鲁问。  
        “我要变小,和你一样。”燕妮平静地说。  
        屋里的所有生命都受到强烈的震撼。  
        “变小了可就变不回来了。”院长对燕妮说。  
        “只要不能和皮皮鲁平等相处,我的身体再大,也等于没有。”燕妮说。  
        “燕妮!”皮皮鲁变小后没掉眼泪,现在他哭了,“你不能变小。”  
        “我必须变小。”燕妮坚定地说。  
        贝塔靠在五角飞碟舱门口热泪盈眶。燕妮这才叫献身,鼠小姐那不叫献身,叫一举两得。  
        “我去给你拿。”院长说。        
        皮皮鲁没阻拦它。他相信自己能发明出使自己复原的药。  
        燕妮躺在床上,院长将微缩粒放在她的鼻子前边。  
        床周围全是老鼠。  
        屋里死一般寂静。  
        每个生命的眼眶里都充满了泪水。  
        燕妮变小了。  
        她和皮皮鲁紧紧抱在一起。  
        老鼠们鼓掌。  
        “我会很快发明出使我们复原的药的。”皮皮鲁向老鼠们宣布。  
        “祝你早日成功。”院长说。  
        “这才叫公平竞争。”舒克说。  
        “我还要发明一种对微缩粒免疫的药,人吃了后,身体里就对你们的微缩粒产生抗体,不会变小。”皮皮鲁对老鼠院长说。  
        “我相信你能发明出来,可我断定没人会吃你这种药。你怎么动员你的同胞吃?你对他们说,老鼠要把人类变小,吃了我的药,老鼠就拿你没办法了。你的同胞有几个会信你的话?人类是伟大,但人类也有缺陷,最大的缺陷就是只信见过的事。”老鼠院长说。  
        皮皮鲁不得不承认老鼠院长的话有道理。          第207集  
        皮皮鲁首次驾驶五角飞碟;  
        贝塔教官给皮皮鲁发驾驶执照;  
        燕妮喜欢五角飞碟里的寝室    
        “我们会加快速度的,今天我们就开始同各国的老鼠联系,在全球推广微缩粒。”老鼠院长说完向皮皮鲁告别。还送了皮皮鲁1瓶微缩粒。  
        皮皮鲁虽然同意放了这帮爱因斯坦家的老鼠的后裔,但当他目睹着它们安然无恙地离开这间屋子时,他还是恨不得把它们发配到北极的冰天雪地里去。  
        “人类够危险的。”贝塔说。  
        “那家伙说得对,即使你发明出免疫药,谁会信你的话而服药呢?”舒克对皮皮鲁说。  
        “皮皮鲁,我有个高招儿。”贝塔说,  “咱们公司不是生产皮皮鲁牙膏吗?你把微缩粒免疫药掺在        皮皮鲁牙膏的膏体里,孩子们就不会变小了。”  
        “那他们的爸爸妈妈呢?”燕妮问。  
        “先救孩子吧。上次皮皮鲁预报地震,人家不信,皮皮鲁也是只好先救孩子。”贝塔说。  
        “到时候孩子因为用了皮皮鲁牙膏而没变小,他们的爸爸妈妈却变小了,谁是谁的爸爸妈妈呀?”舒克说。  
        “我看出现这样的局而也不错,这是让今天的父母与子女平等相处的最好办法——微型父母养育巨人子女,反正想打是打不动了。”贝塔说。  
        “咱们回国后我马上就研制免疫药,掺到皮皮鲁牙膏里。”皮皮鲁决定了。  
        “先研制复原药还是先研制免疫药?”舒克觉得应该先研制复原药。  
        “先研制免疫药。”皮皮鲁说。  
        燕妮再次庆幸自己的眼力,她吻了皮皮鲁一下。  
        “咱们的动作必须快,我们这些同胞不是等闲之辈。”贝塔说。  
        “现在你们就驾驶五角飞碟去救歌唱家。”皮皮鲁对舒克和贝塔说,“咱们争取今天就回国。”  
        “干吗我们去?你和燕妮可以一起去呀!”舒克提醒皮皮鲁注意自己的体积。  
        皮皮鲁一拍头,他刚刚意识到自己可以进入五角飞碟了!        
        “咱们进五角飞碟看看。”皮皮鲁拉着燕妮跑进五角飞碟。  
        五角飞碟虽然是皮皮鲁制造的,但他从未置身其中。皮皮鲁尽管熟知五角飞碟的内部结构,可当他头一次身临其境时,还是被五角飞碟的现代化设备震惊了。  
        “太棒了。”燕妮惊叹道。  
        “我给你们当导游。”贝塔说,“这是驾驶台。这是遥感仪。这是武器系统。你们再看这儿,这个门里是卫生间。卫生间旁边是寝室。还有餐厅、娱乐厅,可惜餐厅和娱乐厅都没使用过。”  
        皮皮鲁和燕妮眼花缭乱。  
        “我想开五角飞碟。”皮皮鲁说。  
        “太容易了,我教你。”贝塔让皮皮鲁坐在驾驶员的位置上,他讲解。  
        舒克在一旁补充。  
        “你可以试飞了。”10分钟后,贝塔对皮皮鲁说。  
        五角飞碟的舱门关上了,进入待飞状态。  
        舱内的各种指示灯竞相闪烁。  
        皮皮鲁心花怒放。他再次体会到塞翁失马这句中国成语的伟大真理性。  
        五角飞碟起飞了,皮皮鲁驾驶它在屋子里盘旋了几圈儿。  
        “可以发给你驾驶执照了。”贝塔用教官的口气对皮皮鲁说。  
        “咱们应该储存一些食品,生活设施也应该改装一下。”燕妮提议。  
        皮皮鲁意识到自己现在已经成为真正意义上的巨人超人,他可以生活在五角飞碟里,想去哪儿就去哪儿,想干什么就干什么。小里包含着大,大里包含着小。最小的也许就是最大的。最大的说不定就是最小的。  
        “您想得有道理。”舒克从遥感仪的屏幕上看到了皮皮鲁的思维,  “打个比方说吧,在这个世界上,最有钱的人实际上是最没有钱的人,最没钱的人其实是最有钱的人。”  
        “这话怎么讲?”燕妮问。  
        “你有了钱,你就欠世界上所有的人。你没钱,世界上所有的人都欠你。你看,钱越多,欠人家的就越多。世界上最富的人就是最大的债务人,最穷的人就是最大的债权人。你爸爸准有体会。他要不是亿万富翁,你姐夫才不会杀你和你姐姐呢。”舒克对燕妮说。  
        燕妮使劲儿点头。  
        改装五角飞碟生活设施的工程开始了。  
        舒克和贝塔根据皮皮鲁的要求去找材料。燕妮去厨房筹集食物。现在,一个(又鸟)蛋够他们四人吃两个月。        
        皮皮鲁为自己和燕妮在五角飞碟里装修了豪华的寝室,寝室里还配有卫生间。舒克和贝塔同住一室。餐厅也扩建了,可同时容纳6个人用餐。储藏室里储备了丰富的食品,他们即使半年不出五角飞碟也能活得贼好。  
        “越小越有安全感。”燕妮躺在五角飞碟里的席梦思床上对皮皮鲁说。  
        皮皮鲁低头吻燕妮。  
        别墅的门铃响了。  
        “怎么回事?”皮皮鲁走出寝室问舒克。  
        正在餐厅收拾房间的舒克跑到驾驶台前打开遥感仪。  
        “警察包围了这栋房子,是来抓你的。”舒克看着屏幕说。  
        “咱们到房顶上去。”皮皮鲁说。  
        舒克驾驶五角飞碟飞出房间,在别墅的房顶上着陆。  
        清晨,安东尼率领近百名警察包围了燕妮的别墅,他是来抓皮皮鲁的。  
        男仆开门后,安东尼连话都没和他说,带着荷枪实弹的警察直奔楼上。  
        所有房间都空无一人。          第208集  
        牧羊犬上燕妮的席梦思床;  
        窗户上的洞;  
        胡安娜的城堡;  
        神探凌晨登门拜访歌星    
        “皮皮鲁走了?”安东尼问负责监视这栋别墅的警官。  
        “绝对没有。我们采用全封闭监视,连一只鸟都不可能飞走。”通宵监视皮皮鲁的警官说。  
        “叫警犬。”安东尼冲警官挥手。  
        一条像小牛一样强壮的德国牧羊犬被牵进别墅,安东尼掏出皮皮鲁的护照给它嗅,让它识别皮皮鲁的气味儿。  
        “找!”牵警犬的警察命令警犬找皮皮鲁。  
        牧羊犬径直上了燕妮的床。  
        “皮皮鲁刚才在这床上睡过觉。”警察根据牧羊犬的姿态判断后告诉安东尼。 
        “我想知道皮皮鲁现在在哪儿?”安东尼两眼盯着警犬。 
        警犬的鼻子紧挨着床,它下到地上,一路嗅到窗台下,然后抬起头,冲着窗外猛叫。 
        “从这儿跑了。”警察抬着窗户对安东尼说。 
        安东尼这才注意到窗户上有一个小窟窿。他想起了玩具飞碟。 
        “你们继续在这儿找,查查有没有暗室。”安东尼一边向警察们布置一边往别墅外边跑。 
        他想起了歌星胡安娜。 
        安东尼发动了汽车,他一边开车一边同警察局联系: 
        “马上加强胡安娜住宅的警力,再给我找10只大猫,会抓老鼠的那种。” 
        安东尼断定皮皮鲁派飞碟去胡安娜家了,那飞碟八成是由老鼠驾驶的。 
        别墅里没有燕妮,导致安东尼的心态极其失落,他开车时不停地叹气。 
        燕妮和皮皮鲁去哪儿了?安东尼坚信警察对燕妮的别墅的封锁是固若金汤的。安东尼不得不承认,这个回合,他败在了皮皮鲁手下。 
        他要在胡安娜家转败为胜。 
        胡安娜在该国歌星中是首富,她的住所是一座名副其实的城堡,造价据说高达六千万美元。胡安娜完全是靠罐头小人歌唱家出名发财的,在深更半夜时,她经常做噩梦,就像所有发不义之财的人那样。可一醒来。她就不怕了,继续肆无忌惮地靠欺骗暴敛。 
        胡安娜曾经给自己定了赚足一百万后即收手的目标。可当她有了第一个一百万后,义想第二个一百万。有了一千万后,又想一亿。 
        在这个世界上,没钱不是真穷,贪婪才是真穷。有钱不是真富,知足才是真富。胡安娜是真穷的那种人。 这天演出后,胡安娜回到家中,她先解开上衣,从乳罩里拿出歌唱家。 
        “你今天唱得不错,赏你一杯牛奶。”胡安娜将歌唱家塞进一座特制的铁房子里。铁房子像普通电视机那么大,里边的墙都由弹性材料制成,以防歌唱家撞墙自杀。 
        这座房子实际上是囚禁歌唱家的牢房。 
        敲门。 
        “进来。”胡安娜说。 
        胡安娜的贴身保镖走进屋子。 
        “咱们的房子四周有警察。”贴身保镖向胡安娜报告。 
        “警察?”胡安娜一惊。 
        贴身保镖点头。 
        “为什么?”胡安娜立刻想到歌唱家。 
        “不清楚。”贴身保镖说。 
        “继续观察,有新情况随时告诉我。”胡安娜示意保镖出去。 
        歇唱家的秘密,只有胡安娜自己知道,她不相信任何人。 
        警方还从未找过胡安娜的麻烦。胡安娜也拿不准冒用歌唱家的嗓子算不算诈骗行为。她决定将歌唱家藏到一个万无一失的地方。想来想去,还是自己的乳罩里保险。 
        胡安娜让歌唱家胡乱喝了两口牛奶,又将她塞进自己的乳罩里。 
        这一夜,胡安娜没睡着。 
        凌晨,贴身保镖又敲胡安娜卧室的门。 
        “有位侦探要见您。”贴身保镖说。 
        “……”胡安娜坐起来,下意识地护住自己的胸,她怕失去歌唱家。 
        “您见吗?”保镖问。 
        “他有什么事?”胡安娜声音有些发颤。 
        “说是关于您的安全。”保镖说。 
        “我的安全?”胡安娜说。 
        “外边警察越来越多了。”保镖说,“您最好见他一下。” 
        胡安娜点头。 
        在胡安娜家豪华的会客厅里,安东尼见到了著名歌星胡安娜。 
        “对不起,打搅您了。我叫安东尼,警察局的。”安东尼先自我介绍。 
        胡安娜点点头,她知道大名鼎鼎的神探安东尼。她的表情有点儿失控,不听她的大脑指挥,挺不自然。 
        “找我有事?”胡安娜小心翼翼地问。 
        “您认识这两个人吗?”安东尼递给歌星两张照片。 
        一张是皮皮鲁。一张是燕妮。 
        胡安娜仔细看过照片,摇头: 
        “从来没见过。” 
        “再仔细看看。”安东尼说。 
        胡安娜看后仍然摇头。 
        “这位男士是中国人,叫皮皮鲁。”安东尼指着皮皮鲁的照片说。 
        中国人?!胡安娜心里一惊。她清楚地知道歌唱家是从中国来的。 
        就在这时,胡安娜感觉到藏在她乳罩里的罐头小人歌唱家动了一下。可见罐头小人也听见了皮皮鲁的名字。 
        “这位小姐叫燕妮。”安东尼又指着燕妮的照片对歌星说。 
        胡安娜茫然地傻摇头。安东尼已经看出,面前的这位歌星和皮皮鲁有某种关系。   第209集 
        10个抱猫的警察进入胡安娜的住宅; 
        天罗地网等待皮皮鲁; 
        罐头小人歌唱家惨遭酷刑   
        “这两个人涉嫌一起谋杀案,警察局已下达对他们的拘捕令。很遗憾,他们失踪了。10分钟前,我们已向全国发了通缉令。”安东尼收起皮皮鲁和燕妮的照片,对胡安娜说。 
        “这和我有什么关系?”胡安娜谨慎地问。 
        “我发现他们……尤其是那个中国人皮皮鲁,对你极为关注。他来咱们国家的目的还不清楚,好像同你有关系。”安东尼说话时始终盯着胡安娜的眼睛。 
        胡安娜的瞳孔里闪过的一丝恐慌,被安东尼犀利的目光捕获了。 
        安东尼对于在胡安娜家抓住皮皮鲁有了信心。 
        “您家里有什么和中国有关系的东西吗?比如说古代字画什么的?”安东尼问歌星。 
        胡安娜拼命摇头。她心里已经清楚这个叫皮皮鲁的中国男人找她要什么了。 
        “我要对您采取保护措施。”安东尼说,“希望得到您的配合。” 
        胡安娜的声带似乎不会发音了,她使劲儿点头。 
        安东尼掏出手机。 
        “猫到了吗?”安东尼对着手机的话筒问。 
        胡安娜以为神探使用暗语说话。 
        “到了?送进来。”安东尼说。 
        10名警察每人抱着一只大猫,走进胡安娜的客厅。 
        胡安娜惊诧地看着这些抱着猫的警察。 
        “请您告诉我,您的哪些房间是需要保护的?”安东尼问胡安娜。 
        “让猫保护?”胡安娜问安东尼。 
        “我们的对手神通广大,他们大概能让任何动物为他们服务。”安东尼说。 
        胡安娜想让10只猫都留在自己身边,还恨不得将一只猫塞进自己的乳罩里。 
        “留两只猫在我身边,其他的放在书房、起居室、健身房……”胡安娜不敢把10只猫都留在自己身边,怕引起神探的注意。 
        安东尼凭胡安娜留两只猫在自己身边这个细节判定秘密就在歌星身上。 
        安东尼开始为皮皮鲁部署天罗地网。他命令所有警察隐蔽后将胡安娜的住宅包围得水泄不通。他还调来特种兵使用最尖端的武器严阵以待。雷达不停地搜索胡安娜住宅的上空。 
        “请您去卧室佯装睡觉。我就在您的隔壁。”安东尼对胡安娜说。 
        胡安娜脸色苍白,她一边发傻一边点头。 
        安东尼指挥警察们将各种通讯仪器安置在胡安娜卧室隔壁的房间里,安东尼将在这里指挥围捕皮皮鲁的行动。他身边的通讯系统装置既能同每一个执行任务的警察通话,又能直接同总统助理通话,还有和雷达部队、边防部队甚至宇航中心的热线电话。此外,一台和警察局信息中心接驳的微型电脑也安放在侦探身边的桌子上。 
        一切准备就绪。 
        “各就各位。”安东尼发出命令。 
        胡安娜的住宅变成了一座陷阱。 
        罐头小人歌唱家到德国已经三十多年了,其间的经历可谓历尽沧桑,用一部上百万字的小说也难以概括。自从被胡安娜奴役后,歌唱家无时无刻不想逃脱胡安娜的魔掌,但每次都以失败和更加严格的防范而告终。 
        歌唱家有一副人类不可能有的独特的嗓子,正是这独特的嗓子,给她带来了灾难。 
        作为社会,没有独特,就没有平安。作为个人,有了独特,就有了危险。 
        歌唱家过着囚徒般的生活,她的全部生活空间就是两个地方——铁房子和胡安娜的乳罩里。她想念她的四位同胞,她想念皮皮鲁和鲁西西。 
        今晚演出后,歌唱家被胡安娜从乳罩里取出来刚放进铁房子里,又被她塞回乳罩里。 
        这是从来没有过的反常举动。 
        歌唱家感觉到要出事,她不知道是凶还是吉。 
        当那位被称作侦探的男子说出“皮皮鲁”三个字时,歌唱家身上的血液立刻沸腾了,她终于盼到了这一天。尽管她无法相信这是事实,但那侦探的声带确确实实发出了“皮皮鲁”这个声音。 
        紧接着,她听到了侦探部署抓皮皮鲁的方案,歌唱家为皮皮鲁捏了一把汗,她想告诉皮皮鲁,可她无能为力——胡安娜的乳罩吲若金汤。 
        歌唱家心急如焚。 
      “你动什么?”躺在床上装睡的胡安娜斥责乳罩里的歌唱家,“别得意。你认识那个什么皮皮鲁?” 
        “对,他是我的朋友。”歌唱家坐在胡安娜的胸上说。 
        “中国人?”胡安娜的声调里有不屑一顾的成分。 
        “对,中国人。”歌唱家特自豪。 
        “他来救你?”胡安娜问。 
        “对,来救我。”歌唱家故意气胡安娜。 
        “他马上就要进监狱了,警察正等着他呢!”胡安娜咬牙切齿。 
        “我看是你马上要进监狱了。”歌唱家不知从哪儿来的勇气反驳胡安娜。往常,她如果这样和胡安娜说话,起码三天吃不上饭。 
        胡安娜气得用两只手从两侧往中间挤压自己的胸部,压迫歌唱家。 
        歌唱家被挤得喘不过气来,脸色变紫了。 
        美好的东西如果配上丑陋的灵魂,就会变成丑陋的东西。丑陋的东西如果配上美好的灵魂,就会变成美好的东西。美好不美好,关键看灵魂。 
        胡安娜身上的每一个部位都是无可挑剔的,但它们却是丑陋的。 
        歌唱家被胡安娜挤压得死去活来。   第210集 
        皮皮鲁驾驶五角飞碟绕地球一圈; 
        在美国上空吃夜宵; 
        总统的枪里没有子弹全是酒; 
        贝塔孤胆闯虎穴   
        五角飞碟停在燕妮的别墅的房顶上。 
        贝塔和皮皮鲁通过荧光屏观察别墅里的情景。 
        舒克收拾他和贝塔的房间。燕妮整理她和皮皮鲁的房间。 
        “还弄来一条狼狗。”贝塔从屏幕上看到警察牵着狼狗闻皮皮鲁的床。 
        皮皮鲁这回算是体验到什么叫安全了。在五角飞碟里,踏实得让人心慌。什么都不怕。 
        “他们走了。”贝塔关上遥感仪。 
        “我想驾驶五角飞碟绕地球转一圈。”皮皮鲁手特痒痒,老想开五角飞碟。 
        “这太容易了,眨眼之间的事。”贝塔告诉皮皮鲁操纵方法和注意事项。 
        燕妮从卧室走出来,她看见皮皮鲁系安全带,问: 
        “干什么?” 
        “绕地球飞一圈。”皮皮鲁说。 
        燕妮吐舌头。 
        “我饿了,咱们吃点儿夜宵吧?”贝塔提建议。 
        “我去做饭。”燕妮说。 
        “我先练练飞行。咱们吃完夜宵就去救歌唱家。”皮皮鲁说。 
        五角飞碟绕地球飞了一圈,燕妮还没走进厨房的门。皮皮鲁干脆驾着五角飞碟满世界乱飞,一会儿欧洲,一会儿亚洲,一会儿非洲,一会儿南极。 
        “吃夜宵吧!”燕妮在餐厅招呼。 
        “现在是哪儿?”皮皮鲁兴致勃勃地问身边的贝塔。 
        贝塔按电脑的键盘。 
        “华盛顿上空4万公尺。”贝塔说。 
        “就在华盛顿头上吃夜宵。”皮皮鲁操纵五角飞碟悬停在美国首都华盛顿上空4万米处。 
        皮皮鲁踌躇满志地走进餐厅,餐桌上已是琳琅满目的丰盛食物。 
        燕妮告诉每位应该坐的位置。 
        “我觉得咱们不应该挨着坐。”皮皮鲁反对和燕妮并排就座。 
        “为什么?”舒克不解。 
        “面对面坐才能看见。”皮皮鲁说。 
        燕妮和贝塔调整了座位。 
        “皮皮鲁一点儿也不人道,他们俩面对面赏心悦目,让我和舒克面对面看什么?”贝塔一边大吃一边说。 
        “真应该把爱因斯坦家那只老鼠后代小姐带上五角飞碟。”皮皮鲁说。 
        “侵犯隐私权。”贝塔说。 
        “咱们现在在哪儿?”燕妮笑着问皮皮鲁。 
        “华盛顿。”贝塔说。 
        “哇——”燕妮差点儿噎着,  “咱们以后去哪儿都不用办签证了?” 
        “对,地球上所有国家都对咱们免签。”贝塔吃香肠。 
        “还省了差旅费。”舒克喝牛奶, 
        “皮皮鲁太伟大了。”燕妮冲坐在对面的五角飞碟发明者皮皮鲁飞了个吻。 
        皮皮鲁得意极了。他最在乎来自亲人的夸奖,最希望亲人享受他的成就。 
        “咱们看看美国总统在干什么?”皮皮鲁俨然是上帝的口气。 
        干这事贝塔最积极,他打开镶嵌在餐厅墙璧上的荧光屏。 
        五角飞碟的遥感仪瞄准了白宫。 
        美国总统正在睡觉。 
        两位人类成员和两只老鼠在4万米高空一边吃夜宵一边欣赏美国总统睡觉。 
        “总统睡觉的样子和普通老百姓也没什么区别呀!”舒克说。 
        “看看美国总统做什么梦。”贝塔往嘴里塞了一颗花生米,然后走到电脑旁边调整遥感仪。 
        总统在梦中回忆自己的童年。总统的爸爸喝醉酒后疯狂地摔打家中的东西,总统的妈妈上前劝阻,被总统的爸爸一脚踢翻在地。 
        总统胆怯地看着眼前的景象,他想帮妈妈,但力不从心。总统突然看见地上有一支枪,他端起枪,瞄准爸爸大喝一声。 
        总统的爸爸看见儿子拿抢瞄准他,满脸通红地说:“好小子,敢用枪打你爸爸。开枪吧!” 
        总统的手开始哆嗦,但他终于还是勾动丁扳机。枪口里射出的不是子弹,而是液体。 
        总统的爸爸闻了闻,是酒。他张开嘴对着枪口接酒喝…… 
        “这算噩梦吧?”皮皮鲁一边吃一边说。 
        “按说当总统不该做噩梦呀!”燕妮说。 
        “官越大越爱做噩梦。”贝塔说。 
        “这叫居高思危。”舒克说。 
        “哟,咱们该去救歌唱家了。”皮皮鲁用餐巾擦嘴。 
        舒克和贝塔同皮皮鲁来到驾驶舱。燕妮收拾餐桌。 
        皮皮鲁驾驶五角飞碟返回德国。 
        贝塔通过电脑迅速查出了胡安娜家的方位。皮皮鲁操纵五角飞碟在胡安娜家的房顶上着陆。 
        “查查罐头小人歌唱家在哪儿?”皮皮鲁对贝塔说。 
        贝塔一边吹口哨一边操纵遥感仪找罐头小人歌唱家。 
        “哟,这胡安娜也太那个了,连睡觉都不放过咱们歌唱家,还捂在她的乳罩里。”贝塔说。 
        皮皮鲁皱眉头。他想像得出,歌唱家整天果在那个地方并不享受。 
        “我去救歌唱家。”贝塔说。 
        皮皮鲁点点头。 
        “我和你一起去。”舒克说。 
        “我自己去就行了,皮皮鲁操纵五角飞碟还不熟练。”贝塔说。 
        皮皮鲁同意了。   第211集 
        真皮沙发掩护贝塔; 
        巨爪从天而降; 
        贝塔落网; 
        电脑显示结果令安东尼吃惊: 
        还有更令人吃惊的事   
        “趁胡安娜在睡觉,我去了。”贝塔对皮皮鲁说。 
        “罐头小人歌唱家会跟你走吗?”燕妮对于歌唱家是否会跟一只老鼠走表示怀疑。 
        正准备开舱门的贝塔站住了,他觉得燕妮的分析有道理。 
        “你提我的名字就行了。”皮皮鲁说。 
        “去吧,我们给你当保镖。”舒克拍拍贝塔的肩膀,“遥感仪始终跟着你。” 
        “当然。没有五角飞碟保护,我才不去歌星的乳罩里冒险呢。”贝塔走了。 
        胡安娜的住宅造型比较奇特,贝塔转弯抹角下到地面上,他顺着墙根儿走到一扇门旁。 
        门虚掩着,贝塔探头往屋里看,他还没见过如此豪华和气派的私人住宅。 
        “妈的,都是歌唱家给她赚的钱。”贝塔在心里骂了一句,他溜进屋里。 
        这是客厅,静得出奇,只有一座一人高的座钟发出那种一成不变的告诉人生命越来越短的声响。 
        贝塔钻到皮沙发下边,他要判断这屋里的六个门哪个通胡安娜的卧室。 
        贝塔选中了离他最近的一个门,他从沙发下边钻出来,正要朝那扇门跑去时,被猫抓住了。 
        这是一只很凶的大猫,它抓住贝塔后使劲儿叫。贝塔绝对挣脱不出猫的利爪,他索性放弃了反抗。 
        隐蔽在隔壁的安东尼闻声赶来,他从猫的利爪中将贝塔抢到自己手中。 
        看着面前这个穿着衣服的老鼠,尽管早有预料,安东尼还是惊讶极了,他一边吩咐警察们继续隐蔽,一边回到自己的“指挥中心”,详细观察手中的贝塔。 
        安东尼忽然觉得贝塔有点儿面熟,好像在哪儿见过。他将贝塔扣在一个玻璃杯里,用数码相机给贝塔照了一张像。 
        安东尼将照片输人电脑,荧光屏上的显示使安东尼目瞪口呆——这只老鼠是30年前光顾过地球的外星人! 
        安东尼想起来了,在他的童年时代,地球曾经接待过两位具有老鼠形态的外星人。当时他安东尼还不满10岁,是从电视屏幕上一睹外星人风采的。 
        没错,就是他,两个外星人中的一个! 
        越来越复杂了。中国人。外星人。都到胡安娜家来干什么? 
        安东尼陷人沉思。 
        手机铃声响了,差点儿吓死安东尼。 
        “我是安东尼,你是谁?”安东尼从兜里掏出手机,没好气地问。 
        “我是皮皮鲁。” 
        “皮皮鲁!你在哪儿?”安东尼下意识地看看房门。门关得严严实实。 
        “安东尼,请你听好,快把你面前的那个玻璃杯从我的朋友身上拿开。”皮皮鲁使用的完全是爸爸命令儿子的口气。 
        “我如果不拿呢?”安东尼的自尊心迫使他必须这么说。 
        “我替你拿。”皮皮鲁话音还没落,扣贝塔的玻璃杯自己飞到空中,然后掉到地上摔碎了。 
        安东尼现在相信皮皮鲁也能把他升到空中再不管了让他按照万有引力定律自由落地。 
       “有外星人帮你。”安东尼不服气。 
        “什么外星人?”皮皮鲁没听明白。 
        “这只老鼠就是30年前光临地球的外星人。我查了电脑。”安东尼说。 
        安东尼如果不说,皮皮鲁都忘了30年前地球人错把舒克和贝塔当外星人的事了。 
        “你们到胡安娜家来干什么?”安东尼的口气缓和多了。谁也不会使用强硬口气和比自己强大的人说话。 
        “你想知道吗?”皮皮鲁问。 
        “当然。”安东尼说。 
        皮皮鲁自从变小进人五角飞碟后,他就什么也不怕了。当他发现贝塔中了埋伏后,就准备驾驶五角飞碟直接进入胡安娜的床上接走歌唱家,是游戏心理使他先同安东尼逗一逗。 
        “我的一个朋友被胡安娜劫持了。”皮皮鲁说。 
        “也是一只老鼠?”安东尼问。 
        “是人类。” 
        “人类!”安东尼吃了一惊。 
        “对,而且是一位杰出的歌唱家。胡安娜根本不会唱歌,是我的朋友的嗓子使胡安娜出了名。” 
        安东尼眼睛里一水的迷惘。 
        “您的朋友是中国人?”安东尼问。 
        “对。” 
        “胡安娜从未去过中国。” 
        “我的朋友到贵国来,被她绑架的。” 
        “您的这位被绑架的朋友现在在哪儿?” 
        “就在这座住宅里。” 
        安东尼猛地站起来,又坐下了。 
        “您知道您的朋友被关在哪间屋子里?”安东尼见胡安娜的第一眼,就对她没好感。 
        “在胡安娜的乳罩里。”皮皮鲁说。 
        “你说什么?”安东尼怀疑自己的耳膜怠工了。 
        “我说我的朋友现在就在胡安娜的乳罩里。”皮皮鲁一字一句地说。 
        “你的朋友是人?” 
        “人。” 
        “怎么可能在乳罩里!” 
        “很小的人,和老鼠差不多。” 
        “……”安东尼看见桌子上的那只穿衣服的老鼠在冲他笑。 
        “和老鼠一样小的人。”安东尼自言自语道。 
        “你不知道的事还多着呢。”贝塔说话了。 
        安东尼活这么大,头一次感受到学校老师教给他的知识有一多半不是真理。   第212集 
        安东尼拔枪命令歌星敞开胸怀; 
        贝塔趁火打劫小酌; 
        胡安娜从亿万富姐到打工妹   
        皮皮鲁索性将罐头小人的来龙去脉以及他对胡安娜奴役歌唱家的推断简要地讲给安东尼听。 
        “真要是这样,胡安娜就太卑鄙了。”安东尼说。 
        皮皮鲁感觉安东尼身上还有正义感。 
        “你现在去帮我救出歌唱家吧,既解救了被绑架的人,又证实了我的话。”皮皮鲁说。 
        安东尼站起来,他拉开通向胡安娜卧室的门。 
        胡安娜正躺在床上佯睡。 
        安东尼打开电灯。房问顿时亮如白昼。 
        胡安娜坐起来,诧异地看着走到床边的侦探安东尼。 
        “解开您的上衣,胡安娜小姐。”安东尼站在距离歌星一尺的地方发出命令。 
        “您?”胡安娜双臂抱紧,护住前胸。 
        “请立即解开上衣!”安东尼的男低音威严有力。 
        “您要干什么?”胡安娜乱了方寸。 
        安东尼掏出手枪。他自己又觉得这个动作挺可笑。 
        手枪还是起了作用。 
        胡安娜解开了上衣。 
        “拿掉乳罩。”安东尼过五关斩六将。 
        “我抗议!”胡安娜明白侦探是冲着罐头小人歌唱家来的了,她绝望了。 
        “不用我帮忙吧?”安东尼催促大明星。 
        胡安娜闭上眼睛,缓缓地摘下了乳罩。 
        一个小人暴露无遗。 
        安东尼小心翼翼地将歌唱家捧到自己手心上,再拿到自己跟前仔细看。 
        歌唱家看着侦探有点儿紧张。 
        “你会唱歌?”安东尼问。 
        歌唱家点头。 
        “她的歌其实是你唱的?”安东尼指指胡安娜。 
        “是的。”歌唱家说。 
        “你认识皮皮鲁?” 
        “皮皮鲁是我的朋友。” 
        “她强迫你唱了多少年?” 
        “7年。” 
        人侦探的眼角居然湿润了。 
        “你犯有非法绑架罪、剽窃罪、诈骗罪……”安东尼对呆坐在床上的胡安娜说,“把衣服穿上!” 
        胡安娜不动,继续袒胸直面人生。 
        安东尼忽然想起了贝塔,他捧着歌唱家回到他的“指挥中心。” 
        贝塔正喝胡安娜酒柜里的好酒呢。 
        手机又响了。 
        “谢谢您。”皮皮鲁说。 
        “你说的是实话。”安东尼说。 
        “我现在就去接歌唱家和贝塔。”皮皮鲁说。 
        “这房子四周埋伏了很多警察,你一露面就会被捕。”安东尼提醒皮皮鲁。 
        “麻烦你打开窗户。”皮皮鲁说。 
        安东尼没想到皮皮鲁这个物理学家还会飞檐走壁。 
        窗户扣开丁。 
        一道闪电划过安东尼眼前,五角飞碟在桌子上着陆。 
        安东尼终于亲眼见到了这架微型飞行器。 
        飞碟的舱门打开了,皮皮鲁走出来。 
        “您好,安东尼。”皮皮鲁对安东尼说,“把歌唱家和贝塔还给我吧。” 
        安东尼张大了嘴,说不出话来。他终于知道了,这个世界是非常奇妙的,最大的奇妙就是生活在其中的人并未意识到它的奇妙。 
        “能把燕妮让给我吗?”安东尼总算能说话了。他向皮皮鲁提了个条件。 
        皮皮鲁冲五角飞碟里招招手,燕妮出来了。 
        “您如果曾经喜欢过她,现在还喜欢吗?”皮皮鲁问安东尼。 
        安东尼看着还没有手掌大的燕妮,不知说什么好。 
        “得,您还不是真爱。要是真爱。她就是变成蚂蚁,您也会跟着往洞里钻。”皮皮鲁说完冲站在安东尼手掌上的歌唱家招手。 
        安东尼将罐头小人放到皮皮鲁身边。 
        “皮皮鲁!”看到和自己一样大小的皮皮鲁,歌唱家不顾一切地冲上去和皮皮鲁拥抱。 
        皮皮鲁热泪盈眶。 
        安东尼读过许多书。今天的经历等于宣告这些书全是废纸。 
        迷信不可怕。可怕的是不信。 
        “你们想怎么处置胡安娜?”安东尼问。 
        大家都看歌唱家。 
        “咱们走吧,不用理她。”歌唱家知道,对一个人最大的惩罚,就是宽容他。 
        “把胡安娜账号上的存款都转给德国的穷人吧?”贝塔说。 
        “这主意不错,她的钱都是歌唱家给她挣的。”皮皮鲁命令飞碟里的舒克完成这一转账程序。 
        “舒克?”安东尼听到舒克的名字挺耳熟 
        “另一个30年前光顾地球的外星人。”贝塔一边喝酒一边对神探说。 
        “看过《人类,我是你的朋友》吗?”皮皮鲁问侦探。 
        “看过,特爱看。”安东尼说。 
        “那书就是舒克写的。”皮皮鲁挺自豪。 
        “那本书是涛涛轰写的。”安东尼看书最爱记作者的名字。 
        皮皮鲁又是一番解释,澄清事实。 
        安东尼再次为人类脸红。 
        舒克在飞碟里将胡安娜变得一贫如洗,她的存款都被转到了慈善机构的户头上。   第213集 
        胡安娜以泪洗面; 
        警察局长命令严密封锁海关; 
        鲁西西咬自己的舌头   
        “我们该走了,射谢你的帮助。”皮皮鲁对安东尼说。 
        安东尼心里明白,即使他不帮皮皮鲁,皮皮鲁也会易如反掌地从胡安娜的乳罩里拯救出歌唱家。 
        安东尼没说话,他冲皮皮鲁点了一下头,还意味深长地看了看燕妮。 
        燕妮、歌唱家和贝塔相继走进五角飞碟,皮皮鲁最后一个进去,他在舱门关闭前回过身子,冲安东尼挥挥手。说实话,皮皮鲁挺喜欢这位德国大侦探。世界上最不幸的事就是两个正派人成为对手。 
        真理和真理打架的结局就是少一个真理。 
        五角飞碟在安东尼的注视下起飞了,它穿过窗户,融进已是霞光满天的晨曦。 
        安东尼望着窗外愣了足足半个小时,像一尊雕像。几乎人类的所有成员终身都在做着改变自己命运的努力,其实命运是无法改变和不可抗拒的。好运气是靠前世不干损人利己的事铸造出来的。安东尼想了很多,又什么都没想。 
        隔壁胡安娜的哭声将安东尼拉回到现实中,他关上窗户,走进胡安娜的卧室。 
        “你哭什么?”安东尼觉得好笑。 
        泪眼汪汪的胡安娜看了大侦探一眼,索性嚎啕大哭起来。 
        “别哭了!”安东尼大吼一声。 
        胡安娜的眼泪被吓回去了。 
        “你的运气够不错了,我要是皮皮鲁,非把你送进监狱不可。”安东尼说。 
        “你不抓我?”胡安娜从侦探的话中看到一丝希望。 
        “你离开这座房子吧。这房子从现在起归政府了。”安东尼说。 
        胡安娜胡乱穿上衣服,走了。她不在乎这栋房子,她的资产能盖几十栋这样的住宅。如果她知道自己的存款已经是零,她绝不会走得这么潇洒,什么都不拿。 
        安东尼赶回警察局向局长汇报。 
       “没抓住?”局长沮丧, 
        “皮皮鲁确实神通广大。”安东尼头一次服输。 
        “他现在大概在哪儿?”局长问。 
        “可能回国了。”安东尼望着窗外的天空说。 
        “应该立即严密封锁所有出境的关口。”局长抓起电话听筒给总统助理打电话。 
        安东尼没有制止局长。 
        五角飞碟里。 
        皮皮鲁将燕妮、舒克和贝塔一一介绍给歌唱家。 
        “贝塔为了救你差点儿让猫吃了。”舒克对歌唱家说。 
        “谢谢您,贝塔。”歌唱家终于又享受到生命的温暖。 
        “都是自己人,不用客气。”贝塔说。 
        “是那侦探布置的猫。”歌唱家说。 
        “安东尼智商不低。也怪我们光顾看人家美国总统做梦了,在胡安娜家着陆前也没遥感侦察一下。”皮皮鲁说。 
        “你怎么会变得这么小?”歌唱家问皮皮鲁。 
        皮皮鲁将经过告诉她。 
        “老鼠科学院的设想其实也还挺有道理。”歌唱家意味深长地说。 
        “你可别以偏盖全。”皮皮鲁笑了。 
        “咱们回家吧?”舒克想女儿了。 
        “返航。”皮皮鲁说。 
        五角飞碟回国。 
        “给鲁西西打个电话。”皮皮鲁一边驾驶五角飞碟一边对贝塔说。 
        坐在皮皮鲁身边的贝塔接通了鲁西西的电话。 
        “鲁西西吗?我是皮皮鲁。”皮皮鲁说。 
        “这么长时间不来电话!”鲁西西不满。 
        “我们找到歌唱家了。” 
        “真的'什么时候回来?” 
        “现在就回去,请你打开窗户。” 
        “打开窗户?” 
        “对” 
        “开窗户干什么?” 
        “五角飞碟从窗户进家呀!” 
        “五角飞碟先回来?你怎么回来?” 
        “你现在就把窗户打开吧。” 
        鲁西西放下电话,打开窗户。 
        “舒利,快来,你爸爸就要回来了!”鲁西西把好消息告诉舒利。 
        舒利没回答。 
        鲁西西找遍了房间,没有舒利。 
        五角飞碟回到家里,平稳地在鲁西西面前着陆。 
        舱门打开后,贝塔先出来。 
        “鲁西西,你好。”贝塔风尘仆仆地向鲁西西问候。 
        “贝塔,辛苦了。”鲁西西往贝塔身后看。 
        歌唱家出现在舱门口。 
        “歌唱家!”鲁西西眼泪夺眶而出。她俯下(禁止)子,将歌唱家放到手掌上。 
        歌唱家激动得说不出话来.三十多年没见了,她本来对于自己还能在今生今世见到鲁西西已不抱幻想了。 
        “别哭了,还有重要人物等待你的接见呢。”贝塔对鲁西西说。 
        鲁西西擦去眼泪。 
        “鲁西西!”五角飞碟里传出皮皮鲁的声音。 
        鲁西西不知所措。 
        皮皮鲁从五角飞碟里走出来。 
        “皮皮鲁!这……这……是…怎么回事……”鲁西西不由自主往后连退了几步,身体撞在书柜上。 
        “别怕,等会儿告诉你。先见见你嫂子。”贝塔从五角飞碟里引出燕妮。 
        鲁西西在没吃任何东西的情况下咬了自己的舌头。   第214集 
        舒利和图钉失踪; 
        豪华歌厅里的噪音; 
        老鼠点歌   
        “这是爱因斯坦的杰作。”贝塔指着皮皮鲁对鲁西西说。 
        “爱因斯坦?”鲁西西皱眉头。 
        皮皮鲁将他和燕妮变小的经过讲给鲁西西昕。 
        “老鼠科学院的这个计划太可怕了。”鲁西西为人类担心,“咱们应该赶快研制免疫药。” 
        “明天就开始研制。”皮皮鲁信心十足。 
        “有个好脑子真不错,说话就是牛气。”贝塔说。 
        “过去是向土地要财富的时代,现在是向人脑要财富的时代。”鲁西西说。 
        在知道了燕妮是为了与皮皮鲁风雨同舟而自己申请变小的后,鲁西西对这位外籍嫂子肃然起敬。 
        “舒利呢?”舒克问鲁西西。 
        “昨天晚上我在公司值班,没回来。,今天早晨回家没见到舒利。町能她去楼道里给图钉送饭了。”鲁西西说。 
        “我去看看。”舒克说. 
        “大白天的,你去太危险,我去找他。”鲁西西说完站起来去楼道找舒利。 
        “感觉怎么样?”皮皮鲁问燕妮。 
        “很好。”燕妮依偎在皮皮鲁身边说。 
        “比你家穷吧7”贝塔问燕妮。 
        “幸福和富有不一定划等号。”燕妮说。 
        “特对。”歌唱家赞成。 
        “皮皮鲁的公司大着呢。在向土地要财富的时代,知识分子富有不正常。在向人脑要财富的时代,知识分子不富有不正常。”贝塔告诉燕妮。 
        “我知道,就他这个脑袋,保守估计也值一千亿美元。整个一个世界首富。”燕妮转身吻皮皮鲁。 
        “皮皮鲁的无形资产绝对在干亿美元之上。”歌唱家说。 
        “你这三十多年的经历一定很有意思。”舒克对歌唱家说。 
        “一会儿讲给你们听。”歌唱家说。 
        “讲给舒克昕要收费,他是作家,会把你的经历当作素材写成小说,再卖给你看。”贝塔提醒歌唱家。 
        “真的?”歌唱家看舒克。 
        “退休作家.早就不写了。”舒克连连摆手。 
        “今天早上在胡安娜家时,听那侦探说,好像他都看过你写的书。”歌唱家想起来了。 
        “作家特累。别人用手创造世界。作家用脑子创造世界。”舒克说。 
        “出事了。”鲁西西急匆匆冲进屋里。 
        “舒利怎么了?”舒克马上想到女儿。 
        “图钉住的那个纸箱子里没有舒利,图钉也不在。整个楼道我都找遍了,没有舒利和图钉。”鲁西西团团转。 
        “用五角飞碟遥感。”贝塔撒腿往五角飞碟里跑。 
        舒克追上去。 
        贝塔打开遥感仪,按了几个键,遥感舒利。 
        舒克盯着屏幕。 
        皮皮鲁也跑进五角飞碟,他急于想知道舒利的下落。 
        头一天晚上,舒利到图钉住的纸箱子里看他,还给他带了食物。 
        图钉的腿伤已经好了。 
        “你爸爸回来了吗?”图钉问。 
        “没有。”舒利坐在图钉的草铺上,感觉还挺舒服。 
        “咱们出去玩玩吧?”图钉建议。 
        “行。去哪儿?”舒利问。 
        “去歌厅,我喜欢听歌。”图钉特别喜欢音乐。 
        舒利和图钉离开纸箱子上街去找歌厅,夜色掩护他们不被人发现。 
        “当心猫。晚上猫爱出来发泄。”图钉提醒舒利。 
        “那儿是歌厅吧?”舒利看见不远处有一座灯火辉煌的建筑。 
        “过去看看。”图钉一边左右环顾一边往前走。 
        一座富丽堂皇的歌厅,建筑的每一个拐弯处都有彩灯在闪烁。美丽的小姐穿着让人想人非非的服装站在歌厅门口向每一个过往的行人送去令人想人非非的微笑。 
        “咱们找机会进去。”图钉说。 
        “先藏到那辆小轿车下边。”舒利觉得从小轿车下边跑进歌厅比较容易。 
        还算顺利,经过1个小时的努力,舒利和图钉进到了歌厅里边。 
        歌厅里灯光昏暗,男男女女们一边听歌一边喝酒喝饮料。音响的音量大得惊人。 
        “太吵了。”舒利捂着耳朵。 
        “这才刺激。歌厅是本世纪最伟大的发明。”图钉很喜欢这里的气氛。 
        “人晚上到这种地方来纯粹是发泄,和猫晚上出来发泄没什么区别。”舒利看着那些脸上不断电闪雷吗的人说,“如果人类社会也像猫的社会那样可以随心所欲,保准没人来歌厅了。” 
        一位风度翩翩的女歌手拿着话筒登上一座红色的台子,她一边自我介绍一边向客人鞠躬。 
        “欢迎朋友们点歌。”她说。 
        立刻有人给她递纸条。 
        “我也想点歌。”图钉说。 
        “我给你写。”舒利上过学,会写人类的字。 
        图钉从地上捡了一张纸,舒利找来一枝笔。 
        “你点什么歌?”舒利问图钉。 
        图钉说了歌名。 
        “怎么递给她?”图钉为难了。 
        “有人给她送花。咱们把纸塞进花里就行了。”舒利出主意。   第215集 
        从天而降的手帕; 
        音乐使图钉发狂; 
        舒利急中生智; 
        扩音器变成摇钱树   
        距离舒利和图钉呆的地方不远的一张桌子旁坐着一位先生和一位小姐,那小姐身旁的一把空椅子上放着一束鲜花。 
        “我把纸条塞进那束鲜花里。”舒利指给图钉看。 
        “挺危险。”图钉看到那张桌子和他们之间还隔着一张桌子。 
        就是说,舒利要经过一张坐满了人的桌子才能到达有鲜花的地方。 
        “没关系。歌厅里光线挺暗。再说这些人的目光都盯着歌台上.没人注意脚底下。我去。”舒利愿为心爱的朋友赴汤蹈火。 
        “行吗?”图钉犹豫,但他太想听到那女歌星为他唱一支歌了。 
        舒利拿着点歌纸,深情地看了图钉一眼,出发了。女性一旦痴情,月亮都能被她变烫。男性一旦痴情,太阳都能被他变凉。 
        图钉目不转腈地注视着舒利的身影。 
        舒利明白,离那些人越远,越容易被他们发现。她索性擦着他们的脚行走。 
        图钉以为舒利疯了。 
        舒利成功地经过了一位男士的脚,她在通过一位小姐时,碰巧小姐的手帕从桌子上掉下来,铺天盖地扣在舒利身上。 
        “我给你捡。”小姐身边的先生说。 
        那男士弯下腰,捡手帕。 
        图钉闭上了眼睛,他知道完了。 
        男士低头捡手帕时,目光没有射到手帕上,而是投射到小姐的腿上。 
        舒利趁机溜了。有惊无险。 
        现在,舒利已经到达放鲜花的椅子下边,她探头观察小姐和男士的状态。 
        他们听歌听得如醉如痴,根本不注意身旁的鲜花。舒利顺着椅子腿爬上椅子,她将点歌单塞进花束中,留一截在外边提醒歌星。 
        返程还比较顺利。没遇到意外。 
        “下边就看你的运气了。”舒利为图钉做了事,心里特享受。 
        图钉刚想说什么,他看见那小姐拿着鲜花走到台前给了歌手。 
        “谢谢。”歌手冲观众挥舞鲜花,一张纸从花束中飘落到台上。 
        歌手捡起纸,看纸上的文字。 
        “舒利小姐为图钉先生点一首歌,好,我现在就为图钉先生演唱,祝你们今宵愉快。”歌手亮开喉咙,为图钉引吭高歌。 
        图钉的眼里闪烁着泪花,人类为老鼠唱歌,这意义实在不一般。 
        图钉身上的确有音乐细胞,他随着旋律开始扭动身体,渐渐进人痴迷状态。 
        舒利喜欢图钉这种艺术家的气质。 
        一曲终了,歌手向观众鞠躬。 
        图钉突然不顾一切地向歌手跑过去。 
        舒利人吃一惊。她不顾一切地追上去。 
        唱歇的小姐正准备继续演唱,她看见了落地式扩音器铁棍上有一只老鼠。 
        “呀——”歌手发出刺耳的尖叫。 
        所有顾客都站起来往台子上看。 
        “老鼠!老鼠!”歌手指着扩音器。 
        闻声跑过来两名保安人员,他们每人抄起一把扫帚,准备围歼老鼠。 
        “歌厅有老鼠.真不像话!” 
        “下次不来了!” 
        “退款!” 
        顾客一个赛一个愤怒。 
        歌厅老板慌了,他指挥保安消灭老鼠,以挽回恶劣影响。 
        一名保安人员抡起扫帚要打扩音器上的图钉。 
        “别打,他是会唱歌的老鼠!”一个声音大喊。 
        保安人员的扫帚在空中停住了。 
        大家环顾四周,找说话的人。 
        “我在这儿,我也是老鼠。在另一个扩音器上。” 
        台上有两个落地式扩音器,大家看另一个扩音器,话筒上果然站着一只老鼠。 
        两只老鼠! 
        “你们想听他唱歌吗?”舒利问。 
        众人兴奋了,他们听人唱歌早就昕腻了,总是那么几个俗得不能再俗的动作.总是那么几句酸得不能再酸的歌词。能听到老鼠唱歌,够刺激。 
        “别打它们,让它们唱歌!', 
        “老鼠会说话,神了!” 
        “快唱!” 
        “……” 
        歌厅老板示意保安人员退下, 
        图钉站在话筒上唱歌,声音很怪,越怪人们越爱听,口哨声欢呼亩震耳欲聋。 
        图钉兴奋了,他在话筒上一边跳迪斯科一边唱歌,乐队居然跟着他的旋律为他伴奏。 
        舒利看着狂热的人群,她为图钉高兴,她觉得图钉不如刚才那个歌手唱得好,可大家偏偏爱听他唱,就因为他是老鼠。中国人说中国话没人觉得了不起,中国人说外国话就会被人认为了不起。干人事的人不容易出名。不干人事的人特容易出名。 
        歌厅老板原以为砸了饭碗,这突如其来的变化令他难以置信。他瞪大眼睛看着两个扩音器上的两只老鼠,扩音器变成了两颗摇钱树。 
        “一定设法留住它们!”歌厅老板对身边的助理说。“从明天起咱们歌厅就改名为老鼠歌厅。” 
        “保准财源滚滚!”助理坚信拥有老鼠歌手的歌厅准击败所有歌厅。 
        “去准备网子。”老板低声对保安人员说。 
        “我建议不动武。”助理小声说,“既然它们会说话,完全可以谈条件。” 
        老板点头。   第216集 
        图钉成为第一个和人类签约的老鼠歌手; 
        燕妮钻进五角飞碟; 
        唱片公司经理拿到唱片独家发行权   
        贝塔指着荧光屏大叫: 
        “你们看,舒利和图钉在干什么?!” 
        皮皮鲁和舒克看见舒利和图钉分别站在两个扩音器上唱歌,一屋子的人边昕边喝彩。 
        “她……她……”舒克说不出话来。 
        “是个歌厅。现在是上午,哪儿有上午开业的歌厅?”皮皮鲁纳闷。 
        “舒利和图钉在为他们表演?”贝塔使劲儿挠自己的后脑勺。 
        “中了谁的魔法吧?”舒克不相信自己的女儿在神智清醒的时候会干这种事。 
        歌唱家走进飞碟。 
        “你有同行了。”贝塔指着屏幕对歌唱家说。 
        “是你女儿?”歌唱家问舒克。 
        “这个是舒克的女儿,那个是舒克的准女婿。”贝塔抢先为歌唱家介绍。 
        “他们也喜欢唱歌?”歌唱家挺高兴。 
        “他们唱得怎么样?”皮皮鲁问。 
        “女儿一般,女婿嗓子不错。”歌唱家如实说。 
        “遥感一下,看看到底是怎么回事?”舒克操纵遥感查事情的经过。 
        真相大白了。 
        后来那歌厅老板果然没有使用武力挽留图钉和舒利,他在结束营业后,同图钉和舒利谈判。 
        歌厅老板称赞图钉是音乐天才,他说他要包装图钉,要和图钉签约,要在1个月内让图钉在歌坛上大红大紫,成为大腕,名利双收。 
        图钉欣喜若狂。舒利虽然心里不安,但她看到图钉这么高兴,只好同意了。 
        歌厅老板拿出合同书,图钉在上边按了手印,这大概是世界上第一份老鼠和人类的契约。 
        当天夜里,歌厅老板就派人满大街贴海报,说是他的歌厅有一只会唱歌的老鼠,欢迎歌迷欣赏。 
        第二天清晨,歌厅门口就排起了买票的长队。 
        “这小子,拿图钉赚钱!”贝塔指着屏幕上的歌厅老板咬牙切齿。 
        “和胡安娜差不多。”歌唱家眼里出现了泪水。 
        “把他们接回来吧?”舒克问皮皮鲁。 
        “现在就去!”皮皮鲁说。 
        “驾驶五角飞碟?”贝塔问。 
        “当然。”皮皮鲁说。 
        “现在?大白天?”贝塔又问。 
        “对。”皮皮鲁点头。 
        “皮皮鲁万岁!”贝塔手舞足蹈。 
        “你和她们在家等着,我们马上就回来。”皮皮鲁对歌唱家说。 
        歌唱家走出五角飞碟,告诉鲁西西和燕妮。 
        “你说什么,舒利去歌厅当歌女了?”鲁西西不信。 
        “我在五角飞碟亲眼看见的。”歌唱家说。 
        “他们现在去救舒利?”燕妮问。 
        “对。” 
        “我也去。”燕妮要和皮皮鲁在一起。 
        燕妮跑进五角飞碟。 
        鲁西西打开窗户。 
        “起飞。”皮皮鲁下令。 
        舒克操纵,贝塔观察屏幕。皮皮鲁和燕妮站在他们身后。 
        歌厅里气氛狂热,听腻了人唱歌的歌迷们听老鼠唱歌特兴奋。一位唱片公司的经理缠着歌厅老板要给图钉出专辑。 
        “你出个价,”歌厅老板对唱片公司经理说。 
        “20万。”唱片公司经理说。 
        “开什么玩笑?太少。” 
        “这可是大歌星的价。” 
        “会唱歌的人满世界都是,会唱歌的老鼠全世界就这么一只。” 
        “你出个价。” 
        “200万。 
        “就这么定了。” 
        “300万。” 
        “你?!” 
        “不干算了。” 
        “行,就300万。独家出版,独家发行。” 
        唱片公司经理和歌厅老板签合同。 
        “这小子拿图钉挣了300万。”贝塔回头对皮皮鲁说。 
        “还真有人出大价钱给图钉出唱片。”皮皮鲁心里为舒克打抱不平,同是老鼠,舒克写了书就没人承认。 
        “舒克有个富翁姑爷了,以后让他多给我买点儿好酒。”贝塔说。 
        舒克问皮皮鲁: 
       “咱们直接闯进去?” 
        “对,停在麦克风旁边.接他们走。谁过来干涉就击昏他。”皮皮鲁自从变小后,口气贼大。 
        五角飞碟撞碎玻璃飞进歌厅,它悬停在舒利身边,舒利站在扩音器上。图钉在另一个扩音器上。歌迷们以为五角飞碟是歌厅的新设备。 
        舱门打开,舒克探头招呼舒利。 
        看见爸爸和五角飞碟来了,舒利很是惊喜,她钻进五角飞碟。 
        五角飞碟移到图钉身边。 
        “图钉,快进来!”舒利叫图钉。 
        “我不走,我是签约的歌手。”图钉不走。 
        “那老板拿你卖了几百万!”贝塔对图钉说。 
        “他爱挣多少我不管,只要我活得有意思就行。不要怕别人挣钱,怕别人挣钱的人绝对挣不着大钱。”图钉坚决不跟五角飞碟走。 
        歌迷们发现不对头了,他们围上来。   第217集 
        歌厅老板向歌迷的恶作剧抗议; 
        五角飞碟绑架图钉; 
        探长林神秘出现; 
        歌唱家开始讲述自己的经历   
        “快让图钉上飞碟!”皮皮鲁冲舒利喊。 
        舒利在五角飞碟里听到皮皮鲁的声音一愣,她回头一看,吓得连退几步,要不是舒克拽住她,她就从舱门口掉下去了。 
        缩小的皮皮鲁吓坏了舒利。 
        “皮皮鲁……你……这……是……”舒利结巴。 
        “呆会儿再告诉你是怎么回事,快让图钉上飞碟!”舒克看见已经有歌迷登上了演出台。 
        “图钉,快上来!”舒利伸手拽图钉。 
        “干什么?”图钉挺生气,自己唱得好好的,让这个怪物飞行器给搅了。 
        “这是我爸爸开的飞碟,来接咱们的。”舒利对图钉说。 
        “我和人家签了约,我不走。”图钉坚持不离开歌厅。 
        歌厅老板闻声赶来,他手里还攥着同唱片公司经理刚签的那份墨迹未干的合同书。 
        “谁在捣乱?”歌厅老板看见一架小型飞行器悬停在歌台上,以为是哪位歌迷的恶作剧。 
        “使用武力让图钉上飞碟!”皮皮鲁下命令了,他不能让歌唱家的悲剧再次重演。 
        贝塔驾驶五角飞碟绕着图钉飞,使歌厅老板和歌迷们无法接近图钉,舒克使用五角飞碟的遥控装置将图钉强行“运”进五角飞碟。 
        “返航!”皮皮鲁冲贝塔挥手。 
        贝塔驾驶五角飞碟故意在歌厅里飞了几圈,他看见歌厅老板和歌迷们一个个张大了嘴巴睁圆了眼睛望着空中发呆。贝塔又擦着他们的头超低空飞了一回,然后撞碎玻璃走了。 
        “我不走!”图钉抗议。 
        “你的嗓子不错。”燕妮走到图钉面前说。 
        “你?”图钉看见和自己差不多大小的人类,实实在在吃了一惊。 
        “你是谁?”舒利问燕妮。 
        舒克把燕妮介绍给舒利和图钉,又将皮皮鲁变小的经过简要地告诉舒利。 
        “歌唱家找到了?”舒利问。 
        “找到了。现在就在咱们家。”舒克说。 
        “到家了。”贝塔一边解安全带一边站起来。 
        五角飞碟的舱门打开了,舒克领着舒利先走出去。 
        “舒利!”鲁西西看见舒利回来了,很是高兴。 
        “让你担心了。”舒利不好意思。 
        “让我们见见图钉。”鲁西西说。 
        图钉沉着脸走出飞碟,他对于五角飞碟使用武力“劫持”他感到不快。 
        舒利将图钉介绍给大家。 
        皮皮鲁将燕妮和歌唱家介绍给舒利和图钉。舒克、贝塔和鲁西西也和图钉认识了。 
        “咱们的大家庭越来越兴旺。”贝塔说完看了歌唱家一眼。不知怎么搞的,刚才去救舒利时,贝塔心里强烈感受到缺了什么,现在他才知道,是因为歌唱家没在飞碟上。 
        贝塔挺吃惊,他还从来没有因为谁不在身边而想过谁。 
        “贝塔说得对,咱们的大家庭越来越兴旺,昨天家里还是冷冷清清的,今天一下子来了这么多朋友。”鲁西西看看桌子上的一群朋友说。 
        这屋子里,就鲁西西一个庞然大物,其他生命都是微型的。 
        有人敲门。 
        “都进五角飞碟。”皮皮鲁说。 
        图钉不想进去。 
        “是歌厅老板找我来了吧?”图钉猜测道。 
        “他怎么可能知道你在这儿,快进去吧。”舒利将图钉往五角飞碟里推。 
        “你的嗓子挺好。”歌唱家边往五角飞碟里走边对图钉说。 
        “你听过我唱歌?”图钉碰到了知音,情绪开始好转。 
        皮皮鲁最后一个进人五角飞碟。 
        鲁西西将五角飞碟藏到床底下。 
        敲门声继续。 
        鲁西西从门镜往外看,两个陌生男子。 
        鲁西西将门打开一道缝儿,问: 
        “请问您找谁?” 
        “我是探长林,这是我的助手,”探长林指指自己身边的小伙子,  “请问皮皮鲁在家吗?” 
        “探长?”鲁西西上下打量来人。 
        “我和皮皮鲁是朋友,他认识我。请问您是?”探长林问。 
        “我是皮皮鲁的妹妹,叫鲁西西。”鲁西西说,“皮皮鲁出国了。” 
        “还没回来?”探长林问。 
        “没有。”鲁西西摇头,“有什么事吗?” 
        “是这样,德国当局要求我们引渡皮皮鲁,说是皮皮鲁在德国涉嫌一起凶杀案……” 
        “你们是来抓皮皮鲁的?”鲁西西问。 
        “噢,您别误会。我们分析后,认为他们的话不可信,想帮助皮皮鲁。”探长林说。 
        “他还没回来。”鲁西西说。 
        “我们去海关查过了人境登记,确实没有皮皮鲁的名字。如果他回来了,请转告他迅速同我联系。这是我的名片,上边有电话号码。”探长林递给鲁西西名片。 
        鲁西西点点头。 
        探长林和助手走了。 
        鲁西西从床底下拿出五角飞碟。 
        “安东尼还真向中国政府要求把皮皮鲁送回去了。”贝塔一出五角飞碟就说。 
        “他也得走走形式。”燕妮说。 
        “我想回歌厅。”图钉说。 
        “歌唱家,把你这三十多年的经历讲给大家听听。图钉,如果你听完后还想去歌厅,我们不拦你。”皮皮鲁说。 
        图钉同意了。 
        以下是歌唱家讲述的自己的真实经历。   第218集 
        歌剧院的草坪; 
        舞台上的人生感受; 
        在纸床上倒时差; 
        见到贝多芬说不出话   
        三十多年前,皮皮鲁的爸爸带着我和约翰登上了飞往国外的飞机。飞机先到柏林,我们出了机场后,皮皮鲁的爸爸问我准备在哪儿落脚。 
        我说找座剧院吧。 
        皮皮鲁的爸爸叫了辆出租车,将我送到柏林一家有名的大剧院。 
        我和约翰一路上就藏在皮皮鲁爸爸的上衣兜里。分手时,我挺难过。 
        “多保重,后会有期。”约翰对我说。 
        “咱们还能见面吗?”我问约翰。我知道一会儿皮皮鲁的爸爸就要把约翰送到美国去了。 
        “能见。”约翰回答得特肯定。 
        “人世间挺复杂,善良和凶恶像孪生兄弟一样形影不离,你好自为之。”皮皮鲁的爸爸小心翼翼地将我放在剧院旁的草坪上。 
        “再见。”我冲皮皮鲁的爸爸和约翰招手,我们的眼睛里都有泪珠。 
        他们走了。我一直到看不见他们的身影后,才定下神来打量周围的环境。 
        自从我降生到这个世界上,还从未体验过孤独的滋味儿,在皮皮鲁的爸爸的身影消失在我的视野里的一瞬间,我知道什么叫孤独了。 
        人在这个世界上,最重要的就是同别人交往,交往的秘密是希望得到别人的欣赏。孤独的本质是没人欣赏你了。没人欣赏的人就像缺水的花草,结局必然是枯萎。人生实际上是一个炫耀的过程,炫耀自己的才能,炫耀自己的相貌,炫耀自己的财产,炫耀自己的亲属……炫耀必须有接收者,否则那叫孤芳自赏。严格地说,炫耀的接收者是认识你或知道你的人。没有炫耀接收者的人,就是孤独。这三十年来,我接触了一些名人,我的感受是,不要同名人打交道,否则你永远是他们的炫耀接收者,是滋养他们自尊的营养液。他们愈发挺拔伟岸,你愈发自惭形秽。名人如果离开炫耀接收者,他们就不是名人。普通人如果离开名人,他们也就不是普通人了。在我和胡安娜相处的几年中,我的这种感受特别强烈。 
        当然这都是后话,当时皮皮鲁的爸爸和约翰离开我以后,我在草坪上的感受就是孤独,那种滋味儿真不好受,我索性哭了一会儿。 
        哭完了,心里踏实了点儿,这毕竟是我自己要求孤身一人到贝多芬的故乡来的。我们几个罐头小人想自己到人世间闯荡,我们不想老是过受别人保护和关照的生活。生命的乐趣就是奋斗。没有奋斗的生命不叫生命。到一个举目无亲的地方,经过一番拳打脚踢,开创令人触目惊心的事业,这才叫生命。 
        我喜欢音乐,音乐是我同这个世界交谈的语言。我崇拜贝多芬,我觉得,能产生贝多芬的土地上一定有与众不同的因素。当我终于站到了魂牵梦萦的地方,排遣了瞬间的孤独感后,心情进人了喜悦状态。 
        我开始观察四周。 
        我置身于一块绿色的草坪上,草坪旁边是高大宏伟的歌剧院,歌剧院的墙上布满了浮雕,每一块浮雕都是一段历史,一个故事,一首歌。 
        草坪紧挨着剧院的一扇小门,我决定从这扇小门进入歌剧院。当时是中午,四周没什么人,我很顺利地进入了歌剧院。 
        剧院里空空荡荡,地面亮得能照见人影,还特滑,我连着摔了两个跟头。 
        我进的这扇门是通后台的,趁着中午没人,我将后台转了一遍。化妆室里全是镜子,还有灯光控制室。后来我跟着胡安娜无数次进过后台,但第一次那种神奇的感觉再也没有了。 
        当我站在空阔的舞台上时,说心潮澎湃一点儿也不夸张。鲁西西给我起名叫歌唱家,歌唱家和舞台有天然的联系,没上过舞台,就不能叫歌唱家。 
        当时我真想唱歌,可又怕惊动别人,只好在心里唱,舞台这东西是很怪,不管什么人,往上这么一站,再往台下一看,整个一个被重视的感觉。 
        其实,每个人的一生都是在舞台上表演,关键看你的演出有没有观众。一般来说,观众越多,你的成就越大。可也不一定,依我看,最重要的观众是你的亲人,特别是先生或太太。有的人观众特多,可里边偏偏没自己的亲人,这样的人成就再大,也等于没成就。像胡安娜,演出时那么多歌迷向她欢呼,下台后没有亲人同她来往,其实特可怜。依我看,在人生舞台上,亲人观众最重要。 
        我在后台找了一个不易被人发现的角落,睡觉倒时差。不知什么人扔在地上一块没使用过的纸巾,我将它叠成一张床,睡着还挺舒服。 
        刚睡着就有人推我,我睁开眼睛一看,是约翰。 
        “约翰?你没去美国?”我惊讶。 
        “我们刚离开你,就碰见一个人,你猜是谁?”约翰神秘地对我说。 
        “谁?你们在德国又不会有熟人。”我说。 
        “贝多芬!”约翰兴奋地说。 
        “贝多芬?!”我不信。 
        “皮皮鲁的爸爸对他说,有个罐头小人歌唱家特崇拜你,想拜你为师,贝多芬说那就叫她来吧。这不,皮皮鲁的爸爸让我叫你来了。”约翰神采飞扬。 
        “真的!”我一跃而起,跟着约翰去见贝多芬。 
        约翰没骗我,贝多芬真的和皮皮鲁的爸爸在一起,他们坐在一辆特豪华的汽车里。 
        贝多芬穿得一点也不讲究,如果不知道他是大音乐家的人,看见他准以为是乡下来的农民。 
        “我教你作曲。”贝多芬见我面的第一句话。 
        “……”我激动得一句话也说不出来。 
        “陕谢谢大师呀!”约翰推我。 
        我还是说不出话,后来我才知道,所有人头一次见到崇拜已久的名人时都犯这毛病。   第219集 
        女主角失去白马王子; 
        留在车门外的腿; 
        飞来的横祸; 
        不幸中的万幸   
        就在这时,突然铃声大作。我被惊醒了,这纸床还挺舒服,睡上去居然做了美梦。 
        我的周围都是纷乱的脚步声。 
        “到时问了,准备上场。” 
        “叫帕蒂快点!” 
        “来啦……” 
        “谁看见我的帽子了?” 
        我根据这些说话声判断,演出就要开始了,我终于能在贝多芬的故乡听到音乐了。 
        我顺着墙角溜到舞台的右侧,藏在一幅幕布下边,我的前边就是舞台。 
        这是我第一次看歌剧,我的心被那瑰丽逼真的场景和音域宽广的嗓子震撼了,我发现歌剧实质上是人类通过呐喊对生命的理解,那些悲剧那些喜剧不通过呐喊不足以宣泄。我还发现不管是在艺术中还是在生活里,喜剧是短暂的,悲剧是永恒的。美丽是短暂的,丑陋是永恒的。欢乐是短暂的,痛苦是永恒的。所以人类要唱,说得更确切些,是喊。 
        女主角的歌声太棒了,我呆呆地看着她,还有她眼中的泪水。当她心爱的白马王子离她而去时,我哭了。 
        我知道这是戏,假的。但在这个世界上,假的比真的更能打动人。 
        我决定跟女主角走,向她学声乐。 
        这难度很大,我开始制定计划。我准备在演出结束时想办法爬到她身上,这样就可以她到哪儿我到哪儿了。 
        演出结束了,她一再谢幕,观众狂热地向她抛掷鲜花和飞吻。 
        我希望她退场时走我这边。 
        糟糕,她从另一侧退场。我只好绕过后台找她。人很多,我随时都有被发现的可能。 
        我尽量躲开人们的视线,好在他们好像都很忙,几乎没人往地上看。 
        当我赶到化妆室时,她已经在几名彪形大汉的簇拥下朝出口走去。 
        我抓住了一个人的裤角,让他带着我走,当我随他走出歌剧院时,我看见女主角钻进一辆很长的轿车。 
        我不顾一切地朝长轿车跑去,那轿车没有等我跑到就开了,我傻眼了。我知道,错过这个机会,再找到她就不容易了。 
        正好我身边有一辆打开门的轿车。我觉得坐上它就能追上她,我跑到那扇开着的车门旁,那个坐在驾驶员座位上的人的一条腿还在车门外。 
        我抓住他的裤腿,他将腿收进车里。就在他的小腿和车座相摩擦的时候,我被蹭掉了。我的一条腿留在车外时,车门关上了。 
        剧痛使我大喊起来,我当时的感觉就是腿没了。 
        那人听到喊声吃了一惊,我想他的第一个反应就是他的车轧了人——尽管他的车还没发动——完全是条件反射,他迅速打开门往车下边看。 
        我的腿虽然解除了挤压,但是疼痛难忍。当我看到他的手伸向车门准备再度关门时,我急了,因为我的腿已经动弹不了,只有呆在原地等候第二次打击。 
        “别关门!”我大喊。 
        他的手停止了行动。 
        在他确信车外无人冲他喊叫后,他打开车内的照明灯开始往脚下看。 
        我想躲,可我的一条腿罢工了,另一条腿难以肩负双倍的负荷。 
        他看见我了,目光里全是惊奇。 
        “谁的玩具丢在这儿了?”他自言自语地用一只手将我从他的脚边捡起来。 
        他的手很宽大,也很温暖。 
        “我不是玩具,是人。”我对他说,“你关车门时夹伤了我的腿。” 
        “会说话的玩具!”他还是认定我是玩具。 
        “请你仔细看看,我是人!”我再次向他声明,同时还挥舞了几下手臂,以此表示我是血肉之躯,不是机械组装的。 
        他一只手托着我,另一只手掀我的衣服。 
        “你干吗?”我抗议。 
        “没有电池。”他自言自语。 
        原来他在我身上找电池。 
        “再告诉你一遍,我是人,不是玩具。”我大声说,“还有,你把我的腿夹伤了,现在我很疼。” 
        他捏了捏我的胳膊,在他确信我的身体不是塑料而是血肉之躯后,他很是吃惊。 
        “请你解释。”他干脆利落地甩出几个字。 
        我对这人开始有了好感。我一直觉得男人话不能多,话多不是男人。声带属于女人,行动属于男人。 
        我将简历告诉他。不知怎么搞的,我觉得他可以信任。 
        “希望你能帮助我。我们认识了,就是朋友,对吗?”我说。 
        “孩子把玩具当朋友,成人把朋友当玩具。”他说。看得出,他被朋友坑过。 
        “所有孩子都把玩具当朋友,但不是所有成人都把朋友当玩具。你就不会。’’我说。 
        他点点头。 
        “先看看你的伤。”他试着动我的腿。 
        “好疼!”我叫。 
        “可能断了。”他轻轻叹了口气。 
        “断了?!”我感到沮丧。到贝多芬故乡的第一天,腿就断了。我知道骨折意味着什么,少校的腿骨折后休养了四五个月才痊愈。 
        “马上去医院。”他小心翼翼地将我放在旁边的座位上,发动汽车。 
        汽车飞驰电掣。 
        “你们的医院会给我这么小的人看病吗?”我问他。 
        他没说话,但汽车却明显减速了。   第220集 
        汽车设计师乔治; 
        艾米以为乔治和她开玩笑; 
        乔治的寝室里全是汽车; 
        大脑的荧光屏   
        “我有个朋友在医院,去找她。”他经过一番思索,重新给汽车加油。 
        汽车明显改变了原来的预定路线,掉头朝另一个方向驶去。他已经意识到,带我去医院看病有相当的难度,找熟人好一些。 
        我忍着腿疼躺在座椅上,我看不到车窗外,但我能感受到这汽车在行驶中非常平稳,像贴着地面飞的鹰。 
        “你的汽车真好。”我渴望聊天,想以此分散我对腿疼的注意力。 
        “谢谢。我设计的。”他说。 
        “你设计的?”我挺吃惊。说实话,自从我来到人间,除了歌唱家,最钦佩的就是汽车。我觉得,汽车是人类智慧的结晶。人类渴望空间,而住宅总是有限的。汽车的诞生满足了人类对生存空间的需求。有了汽车,这座城市就都是你家了。汽车是住宅的延伸。 
        “我叫乔治,在一家汽车制造公司从事汽车设计。”他终于向我介绍他自己了。 
        我对他肃然起敬。 
        乔冶专注地驾驶汽车,从侧面看,他大概三十六七岁,一个标致的胖男子。 
        “开自己设计的汽车,感觉特棒吧?”我问他。 
        以汽车为话题,他的话明显多了些。但仍然简练。 
        “作家看自己写的名著。父亲代儿子上台领奖。”他说。话里透着得意。 
        汽车减速,上坡,转弯,停车。 
        “到了。”他解开安全带,  “你先等会儿,我去找艾米。” 
        艾米是他的一个朋友,在这所医院当大夫。 
        乔治一走,我的腿疼得就厉害了。由此可见,有痛苦时,万万不可一人独处,孤独是痛苦的膨化剂,交往是痛苦的镇静剂。 
        车门打开了,乔治和一位穿白大褂的小姐出现在我的视野里。 
        “乔治,我可没时间陪你消遣,你看见我的房间里有病人。”小姐显然没看见我,她以为乔治是和她开玩笑。 
        “艾米,你仔细看。”乔治的手指为艾米的目光导向。 
        艾米看见了我。她显然挺兴奋。 
        “真有这么小的人?”艾米的脸几乎挨到我的脸,她的眼睛里透着友善和欣喜。 
        “她的腿受伤了,是我造成的,你快给她看看。”乔治催促艾米。 
        艾米小心翼翼地将我放在她的手心上,我忍着疼,跟着他们来到医院里。 
        这是我第一次进医院。医院是生命的始发站和终点站,也是生命的维修站。一进医院,我就有一种强烈的感受,我感受到任何人都无法拒绝死亡,可几乎所有人都力图拒绝死亡。其实,在茫茫宇宙中,没有诞生的生命是最幸福的。睡眠不是生命休息的形式,生命真正得到休息的惟一形式就是死亡。医院是一个不让人休息的场所,它所做的事实质上是不择手段地延续人的痛苦。太平间里躺着的那些人才是医院人道主义精神的真正体现。 
        艾米将我和乔治带到她值班住的寝室,她不想引起别人对我的注意,她知道,人类的本性就是对不正常的事特感兴趣。而一旦大家对你感兴趣了,你就不能再为自己活着了。 
        艾米开始为我检查。 
        “初步诊断是骨折。也就是腿断了。”艾米说,“还得拍张片子确定一下。” 
        乔治用歉疚的目光看着我。 
        “段关系,我的一个朋友的腿也出过这样的事,很快就好了。”我对乔治说,“再说,责任也不在你,怪我动作太慢。” 
        “以后再设计车门时,一定要安个报警系统。”乔治说。 
        两个小时后,艾米为我的伤腿打好了石膏。我在两个月内无法下地走动。 
        当一件东西属于你时,你体会不到重要。当你失去它时,你才意识到它的重要。能造出航天飞机的人类无法走出这个误区。 
        我的腿完好无损时,我从未渴望过走路。当我的腿不能动时,我却极其渴望行走。能干时不想干,不能干了才想干。 
        “乔治,你带她回家休息,我定期去给她检查。她不能住院。”艾米对乔治说。 
        “谢谢你。”乔治接过艾米递给他的一个纸盒子,将我轻轻放在盒子里。 
        我跟着乔治回到他的寓所。 
        乔治显然是过着单身生活,他的房间的墙上贴了许多不同时代的汽车的图片,像是汽车博览会。 
        “如果你不反对,就在我家养伤,等伤好了,再去拜师学歌。”乔治对我说。 
        “谢谢,给你添麻烦了。”我庆幸自己碰上了善良的人。 
        我将我的感受告诉乔治。 
        他略加思索后,说: 
        “不错。不过,相识后,有的成为朋友,有的成为敌人。我就得罪过不少人。” 
        “像你这种人,如果得罪了谁,那是他的运气不好。”我说。 
        能设计出如此漂亮汽车的人,有如此非凡的创造力的人,如果他得罪了谁或谁得罪了他,不只能说明那人运气太差吗? 
        我看到乔治的瞳孔里显示出“幸遇知己”四个字。眼睛是大脑的荧光屏。知己能从朋友的荧光屏上读到对方的思维。   第221集 
        大两岁的太太原是大丹; 
        董事长喊来警察; 
        50亿部电视剧同时上演; 
        别让脑细胞系安全带   
        我开始了在乔治家养伤的日子。白天,乔治去汽车公司上班,晚上他下班回家后,我们就聊天。 
        “你为什么不结婚?’一天晚上,我问乔治。 
        “有过一次婚姻,结束了。”乔治说。 
        “为什么?” 
        “她太自卑。” 
        “自卑是离婚的理由?” 
        “人生最大的不幸,莫过子和一个极度自卑的人结婚了。” 
        开始我不清楚乔治这句话的含义,待他讲完了他的短暂的婚姻史,我才明白了。 
        乔治在10年前娶了一位大他两岁的太太,一般说来,女方年龄大于男方的婚姻注定是悲剧。这其中的心理因素大于生理因素。乔治的太太对乔治的行动控制极为严密,平均每隔45分钟要给乔治往公司打一个监督电话,如乔治不在公司,太太便会坐卧不宁,眼前顿时呈现云云雨雨景象。如乔治在公司,太太又会萦生乔治与公司小姐眉目传情之幻觉。弄得乔治如芒在背如坐针毡如履薄冰如丧家之犬惶惶不可终日,整个一个娶了一位女侦探女警官女黑贝女大丹的感觉。 
        “我不能和任何异性说话。上至80岁老妪,下至未过满月女婴。”乔治叹了口气,“只要被她看见,就会和我大动干戈,甚至以舞刀弄枪相威胁。” 
        “她大概太爱你了。”我说。 
        “不,这不是爱,是虐待。说透了,属于性虐待范畴。心理学家告诉我,这是极度自卑的表现。这种女性内心深处认为自己在全球25亿女性中排名次列倒数第1名,她认定任何一个女性都可以轻而易举地将她的先生从她身边夺走,她每分每秒都感到岌岌可危朝不保夕。真正有自信的女性是不怕自己的先生和异性接触的,曾为沧海难为水。自信的女性认定自己是沧海,她坚信拥有过沧海的先生对水绝对不屑一顾绝对坐怀不乱。”乔治头一次说这么多话。 
        我认为乔治的话有道理。自卑是“醋”的酿造原料。 
        “后来你们就分手了?”我感受到人生的游戏性。 
        “后来她无端怀疑我和设计室的一位小姐关系暖昧,有了这种猜测后,她再也不能自拔,在她的想像世界中,我和那位小姐连孙子都抱上了。她终于发展到给那位无辜的小姐家中打骚扰电话,还找公司的董事长,要求董事长辞退那位小姐。”乔治面无表情地说。 
        “太卑鄙了,你们董事长不会理她吧了”我为那小姐的命运担心。 
        “当然不会,否则他也当不了董事长。董事长叫来警察将她驱逐出办公室。那小姐到法院控告她犯了诽谤罪。法院裁定她有罪,判她拘役11个月。”乔治注视着墙上的一辆汽车说。 
        “人生两大支柱:事业和家庭。人要活得好,起码得有一个支柱。一个没有,生命就塌了。你的前妻在事业上给你添乱,在家庭里给你捣乱,这样的妻子,绝对要休,否则贻害终身。”我为乔治高兴。 
        “每个人的一生都是一个充满戏剧性的故事。我如果是导演,我根本不需要什么作家当编剧在那里冥思苦想,我就在大街上随便拉住一个人,塞给他l000马克,请他讲他自己的经历,然后把这人的经历如实地拍成电影电视剧。”乔治双臂抱在脑后,望着空中说。 
        我想起了皮皮鲁,想起了鲁西西,想起我见过接触过的每一个人。其实,每个人的一生就是一部电视连续剧,一年为一集。在这个地球上,每天都同时上演着50亿部不同的连续剧。这人如果崇尚爱情,就是一部言情片。这人如果是坏蛋,就是一部警匪片。这人如果酷爱音乐,就是一部音乐片。这人如果早死了,就是一部古装片。 
        艾米每个星期都来为我检查,我和她成了好朋友。我很喜欢她,她身上有一种令人赏心悦目的素质。她是那种不会为了钱而献出一切的女性。女性达到这种境界,才会真正令男性肃然起敬。我懂。旁观者清。 
        我发现,我的腿不能动了以后,我的脑细胞却爱动了。由此可见,身体处于运动之中时,脑细胞大概都被系上了安全带,动弹不得。只有在身体处于静止状态时,脑细胞才如脱缰的骏马,纵横驰骋,伟大的想法都产生于人的独处静止状态。 
        艾米送给我几张唱片,她说她喜欢听歌剧,但不喜欢男性唱歌,每当她看见男人在舞台上扯着嗓子唱歌就想吐,还说男人选择唱歌为职业是一种性变态。 
        我不同意艾米的观点,但我尊重她的看法。每个人都有对任何事保持自己的看法的自由,既要维护自己的观点,又要尊重别人的不同观点。在这个世界上,任何事物都不会由于人类拥有不同的观点而改变。爱怎么想就怎么想。山还是山。水还是水。在乔治和艾米的悉心照料下,我的腿终于痊愈了。当我独立行走迈出第一步时,艾米问我的感觉。 
        我不知怎么冒出这么一句: 
        “久别胜新婚。” 
        乔治和艾米愣了一瞬间,然后笑出了眼泪。 
        我觉得他俩如果组成家庭准幸福,我挺想给乔治和艾米当一回媒人。 
        乔冶和艾米异口同声谢绝了我的好意,他们说,正因为他们非常珍惜和对方的关系,所以绝对不能结婚。保持关系的最好方法就是保持距离,没有距离就意味着断绝关系。 
        亲密无间的结局必然是分道扬镳。 
        艾米晚上没有走。但她不和乔治结婚。永远拥有对方的最好办法就是让对方永远不属于你。   第222集 
        乔治的汽车制造公司; 
        蔬菜型汽车是未来新款; 
        大脑是银行; 
        电线杆子电话和急救车   
        腿伤好了,我和乔治告别的日子也到了。 
        “你答应过我,带我去参观你们的汽车制造公司。”我装做若无其事地说。其实我和乔治明白,分手对我们来说,已经是一种痛苦。 
        “明天就去。”乔治表情略显呆滞。 
        这天晚上,我睡不着。我在这栋房子里生活了近3个月,初来乍到时的陌生已被熟悉取代,我跟这房子里的每一物件都有了感情。 
        第二天上午,乔治带我到汽车公司参观。这是德国三大汽车公司之一,它生产的汽车驰名世界,我一直想知道汽车是怎么造出来的。 
        我藏在乔治的衣兜里。乔治是这家公司的汽车设计师,他走到哪儿,职员们都向他投来感激的目光。是他设计的汽车给他们提供了谋生的机会。 
        人的确是一种不可思议的动物。不同的物质通过装配流水线经过不同的排列组合,居然变成了一辆人见人爱的汽车。这些零部件配合得是那么默契那么天衣无缝那么无懈可击那么珠联璧合。乔治供职的这家汽车制造公司每1秒钟生产两部轿车。一切都是如此有条不紊。 
        “人类现代文明是伴随着汽车产生的。没有汽车,就没有人类的现代文明。”乔治对我说。 
        “汽车公司之间竞争很厉害吧?”我问。 
        “硝烟弥漫。”乔治使用这四个字来形容汽车制造商之间的竞争。 
        “靠什么竞争?” 
        “新产品。人类的本质是喜新厌旧。人类社会是在喜新厌旧中前进的。我带你去我的设计室看看,你就明白了。” 
        乔治的设计室很气派,使人不禁想起宇航中心。各种构思中的新车模型静静地蛰伏在屋子里,虽然还是模型,却已体现出锋芒毕露的气质。 
        “这是前年的款式,如今大街上已经很少见了,我预计,8年以后,它将东山再起。” 
        乔治指着一辆样式陈旧的汽车模型说。 
        “字宙中就这么多形状,人类不可能永远弄出新形状。所谓新款,其实就是把别的领域的东西的款式拿过来。我看近期汽车的发展趋势大概是拿军火产品的款式来设计汽车,向坦克靠拢,向子弹靠拢。”我回忆最近在大街上见到的汽车说。 
        “我们公司明年的新款汽车,全都向蔬菜靠拢。这是高级商业机密。”乔治小声告诉我。 
        “蔬菜?”我觉得汽车造成蔬菜模样未必受欢迎。 
        “土豆。胡萝卜。圆白菜。西红柿……”乔治如数家珍,一口气说下去。 
        我服了人类的大脑。上帝的杰作。 
        我陪着乔治上了一天班。他坐在桌前搞设计。我藏在桌子上的一个文件盒里看他。 
        看着看着,我明白了,原来这世界上的东西不管是什么,除了自然界的,都是人脑变的。汽车是人脑变的,高楼大厦是人脑变出来的,飞机火车电话电视激光唱机卡拉OK也是人脑变出来的。 
        对于人来说,脑子最重要。脑子是你的银行,聚宝盆,摇钱树,可有人却偏偏忽视大脑,重视别的器官。 
        我和乔治早晨就约定了,他下班后,就送我去歌剧院,他还要帮我踏上女歌唱家的汽车。 
        乔治下班后,开车带我去歌剧院。 
        我坐在方向盘前的平台卜,看着车窗外的景色,想起和乔治第一次见面的情景。 
        “乔治,还记得你对我说的第一句话吗?”我背对乔治问他。 
        乔治没回答。 
        车速突然减慢。 
        我回头,看见乔治脸色铁青,额头上冒出豆大的汗珠。 
        “乔治,你怎么了?”我大喊。 
        乔治看了我一眼,好像想说话,却说不出来,只见他咬牙将汽车停在路边,头耷在方向盘上。 
        “乔治!乔治!”我慌了。一时束手无策,我没听说过他有什么病。 
        乔治趴在方向盘上,汗珠层出不穷。 
        我判断乔治是得了什么急病,必须马上叫救护车,我看见了车载电话。 
        我无数次见过别人打电话,可我自己却从投打过。电话对于我来说,简直是庞然大物,我根本拿不动话筒。 
        我要做的第一件事是把话筒从机座上移开,这相当于一个成年人移动一根电线杆子。要在平时,我根本不可能搬得动电话听筒,可当时看着乔治那样子,我奇迹般地将电话听筒移开了。 
        下面要做的事就是拨号码。我知道急救中心的电话号码。我必须将整个身体站到按键上去才能把按键压下去。 
        好在急救中心的电话号码只有3位数,否则我的体力根本无法完成拨号任务——我的腿伤刚好。 
        号码拨完了,告诉医生需要急救的病人的方位更是一件难事,我要在送话器一端说话,然后跑到听筒那边听话。 
        所有这些事都忙完了后,我已是大汗淋漓。乔治还处于昏迷状态。 
        我听到了急救车的呼啸声,我赶忙钻进乔治的上衣兜藏起来。 
        车门打开了,医生将乔治抬离汽车,放在担架上。一痛体检。 
        “心肌梗塞。”医生诊断。 
        乔治被抬上救护车。救护车飞驰而去。 
        “谁给咱们打的电话?”医生忽然想起了这个问题,他问同事。 
        “是挺怪。”同事摇头。 
        “找找他的证件,通知家属。”医生说。 
        hushi的手向我藏的衣兜伸过来。   第223集 
        乔治被仪器包围; 
        死囚犯救了乔治的命; 
        警察拘捕乔治; 
        并非天方夜谭的答案   
        我意识到危险来临,我拼命往兜的深处躲。可再躲也离不开衣兜。 
        那hushi大概属于智商高的,她没把手伸进衣兜,而是从外边按了按,以此判断这个衣兜里有没有证件。 
        我松了口气。其实,在这个世界上,聪明人恰恰是最愚蠢的人。 
        急救车驶进医院,乔治被抬进急救室。 
        各种仪器立即将乔治包围起来。人类不光会造汽车,他们还制造了许多延长自己生命的器物。 
        经过几个小时的抢救,乔治苏醒了。 
        医生和hushi的脸上露出了几乎看不见的笑容。他们站在床边看乔治的表情,就像乔治看自己设计的新车模型。 
        “我这是在哪儿?”乔治的声音很微弱。 
        “在医院。您在驾车的途中心脏出了毛病。”医生说,“我们接到电话后,将您接来了。” 
        “电话?”乔治问。 
        “对了,现在我们还不知道是谁打的电话,应该说,是他救了您的命。”医生说。 
        乔治的手臂抬起来,想摸兜,他已经猜到是我给急救中心打的电话。 
        乔治被安置住一间单人病室里。病室里有卫生间、电视和必要的急救设备。 
        乔治属于特级护理,hushi就坐在床边观察仪器,我无法和乔治交谈。 
        趁hushi出去拿药的时候,我从衣兜里钻出来,藏在乔治的枕头下边。 
        我告诉乔治事情的经过。乔治一动不动,我知道他听见了,因为我看见他的耳朵上有一颗泪珠。 
        第二天,医生要通知家属,乔治将艾米的电话号码给了他们。 
        艾米很快就来了,她很是吃惊。医生没让她多坐,就把她叫出去了。 
        我在病房里吃不上饭,肚子饿得难受。 
        艾米回来后,她说带我出去吃东西。 
        艾米愁容满面地告诉我,医生说,乔治的心脏出了大问题,基本上不能使用了。 
        “那怎么办?”我急了。 
        “惟一的办法是移植别人的心脏。”艾米说。 
        “器官移植。”我知道这个词。 
        艾米点点头。 
        “谁会把自己的心脏给别人?’.我问。 
        “只有等死刑犯人。”艾米说。 
        我沉默了。 
        人每天早晨一睁开眼,应该为自己今天没有生病高兴得手舞足蹈。健康是享受人生的基本保证。 
        乔治的心脏虽然已经复工,但它是在仪器的督促下老大不情愿地工作的。 
        我日夜守在乔治的身边,医生和hushi居然没有发现我。他们不在病房的时候,我就给乔治唱歌。乔治说,我将来肯定是大歌星,嗓子特殊。 
        我问乔治在死亡边缘的感受。 
        “好像刚出生就死了。”乔治望着床头柜上的鲜花说。花是艾米送的。 
        医生满面春风地走进来。我忙躲进枕头下边。医生告诉乔治,有了一个死刑犯,身强力壮,他本人已同意执刑后将身体里所有的器官贡献给需要的病人。现在已有6个不同的病人移植该犯身上不同的器官。乔治移植他的心脏。 
        乔治问医生那死囚犯的什么罪。医生说是抢劫银行故意杀人。乔治叹了口气。 
        手术成功了。乔治换了心脏。 
        乔治出院那天,还真来了不少记者,摄像机照像机不停地忙活。那时候,心脏移植手术还不多见,成功了自然鼓舞人心。 
        艾米驾车接乔治出院,我决定陪乔治住一个月,然后就去拜师学歌。 
        在乔治出院后的第二个星期,意想不到的事发生了。 
        这天中午,乔治说出去走走,我没在意,就让他去了。当时我正在尝试写一首歌。 
        两个小时后,艾米气喘吁吁地来了。 
        “出事了!”艾米说。 
        我的第一反应就是乔治新换的心脏出毛病了。 
        “乔治怎么了?”我急忙问。 
        “乔治被警察抓走了。”艾米茫然地站在屋于中央,不知所措。 
        “被警察抓走了?为什么?”我问。 
        “乔治去抢银行……”艾米极不情愿地说出这几个字。她的眼圈是红的。 
        “胡说!”我根本不信,“警察吃错药了吧?据说世界上的警察抓的好人比坏人多。” 
        “乔治是去抢银行了。”艾米低着头说,  “刚和我去了警察局,还看了银行电视监视器录的抢劫经过录像片。” 
        “你就信了?”我使用谴责的口气。 
        “我也觉得不正常,可乔治抢银行确实是事实,还有很多目击者。”艾米说。 
        “乔治被关起来了?”我问。 
        “押在警察局,等待审判。”艾米说。 
        “抢银行?抢银行?……”我忽然感觉最近怎么经常听到这句话。 
        我想起来了,将心脏移植给乔治的那个死囚就是因为抢银行被判死罪的。 
        莫非是那罪犯的心脏移植到乔治的身上后仍然恶习难改?我将自己的想法告诉给艾米。 
        “法官不会相信这个诊断。太离奇了。”艾米说。 
        “你去查一下,今天还有没有别的人抢劫银行,那死囚犯身上的器官不是移植给了6个人吗?”我提议。 
        艾米抓起电话就打。结果不出所料.另外5个人今天也都因抢银行而被捕。 
        乔治有救了。   第224集 
        苹果树上结了蟑螂; 
        刮胡刀使世界上没有了男人; 
        伟人身上也有犯罪细胞; 
        超级市场风云突变   
        艾米为乔治请了一流的律师,并将移植罪犯器官的事告诉他。律师调查后确信乔治无罪。 
        我和艾米去拘留所看望乔治。乔治现在成了新闻人物。受人尊敬的汽车设计师沦为抢劫犯,这巨大的反差使记者们激动得睡不着觉。记者所从事的职业从本质上说就是报道反差,他们恨不得苹果树上结出蟑螂。如果没有反差,他们就制造反差。 
        满脸胡子的乔治隔着铁杆坐在我们对面,他的身后站着一名铁塔似的警察。 
        我只能躲在艾米的围巾里偷看乔治,我发现,在拘留所或监狱里呆过的人都特有深度,他们的表情就是一部哲学著作。 
        “歌唱家好吗?”乔治的第一句话。 
        “好。”艾米有意低头看自己的围巾,向乔治示意我就藏在她的围巾里边。 
        乔治看见了我,他的嘴角露出一丝笑意。我还是头一次见乔治满脸胡子的样子。我觉得男人不留胡子实在是一个遗憾。刮胡刀把男人变成了女人。 
        艾米告诉乔治,她为他请了最好的律师。艾米没有太多的钱,她的衣着永远是朴实无华的。穷,不丢人,丢人的是掩盖穷。 
        “你抢银行和移植心脏有关。”艾米告诉乔治,她的声调是不容置疑的。 
        “移植心脏?”乔治下意识地抬起左手放在自己的左胸口上边。 
        “你移植的是一个抢劫银行的死凶犯的心脏。”艾米提醒乔治。 
        乔治缓慢地摇头,他显然难以相信这是事实。 
        法院开庭那天,我和艾米去了。 
        检查官以抢劫罪起诉乔治。 
        律师为乔治辩护,当他声称乔治犯罪是由子他移植了罪犯的心脏时,全场哗然。 
        律师出具另外5名移植了该罪犯器官的人也于近期犯罪的证据。 
        遗传学家、犯罪心理学家和医生闻讯赶来,这对于他们来说是一个新课题。法官宣布休庭,还说下次开庭6名移植了罪犯器官后犯抢劫罪的人同堂审理。 
        再次开庭时,法庭内外人山人海。电视台现场直播。记者们绝不放过这次强烈的反差。 
        法官最终宣布乔治和其他5名被告无罪,因为他们是在移植了死囚的器官后无意识犯罪的。 
        乔治和其他5人在判决后一致要求去医院摘除移植的器官,他们说宁可死也不能干继承死囚遗志化悲痛为力量继续犯罪的事。 
        一位颇有声望的医生站起来对乔治他们说,他近日突击研究了这个现象,他发现每个人身体里都有两种细胞,一种犯罪细胞,一种守法细胞。再伟大的人物身上也有犯罪细胞,最凶恶的罪犯身上也有守法细胞,关键是看两种细胞的比例。那个死囚的器官移植到这6位不幸的良民身上后,增加了他们身体里的犯罪细胞的阵容,从而导致他们失足。他说,他正在发明一种药,注射后能减少人体内的犯罪细胞数量,使人的一言一行都遵纪守法都按基督教佛教道教的旨意行事绝不再干(又鸟)鸣狗盗杀人越货偷漏税款损人利己的勾当。 
        全场掌声雷鸣泪花电闪。 
        我在艾米的衣兜里也拼命鼓掌。同时也为假如世界上没有了罪犯必然导致收视率最高的警匪片题材素材减少从而使影坛视坛一片箫条广告收入大幅下降电视台电影制片厂难以为继电视机生产厂家销量下降库存上升只好用解雇职工来降低成本而担忧。 
        乔治被法官宣布当庭释放。回到家后,我们竟然相视无言百感交集。 
        “人生最重要的,是自由。”这是乔治从拘留所回到家里后说的第一句话。 
        艾米点头。 
        “我要设计一种最能体现自由精神的新款汽车,它将成为后年的汽车流行款式。”乔治一字一句地说。 
        我终于意识到一个真理:天才的所有经历对社会来说都是黄金,不管是顺境还是逆境,不管是成功还是失败。 
        “我去超级市场采购些食品,咱们应该聚餐庆祝乔治获得自由。”艾米说。 
        “我跟你去,我知道乔治喜欢吃什么。”我说。 
        其实,艾米对乔治的了解比我多,我是想选购一件小礼物,在分别时送给乔治。我该去学歌了。 
        超级市场人挺多,我坐在艾米放钱包的那个衣兜里。艾米挤过一条较窄的通道时,一只手伸进了艾米的衣兜——我呆的那个衣兜。 
        当我明白那只手是来偷钱包时,钱包已经在向外边移动了,我紧紧抓住钱包,力图与小偷抗争。 
        抵抗以失败告终,连我自己也被牵出了衣兜,我大声喊叫,无奈环境太嘈杂,我的声音被淹没了。 
        艾米毫无察觉。 
        那只手将钱包和我一起塞进他的口袋。然后快步朝出口走去。 
        我试图钻出口袋,然而他的口袋有尼龙拉链,是密封的。看来小偷最懂得防贼术。 
        那人走进一坐小花园,假装坐在石椅上晒太阳。他掏出钱包数钱。 
        钱包里是艾米一个月的工资。 
        我希望现在天上下刀子雨。 
        他数完钱,似乎对于钱的数目还不满意,又将手伸进口袋。他摸到了我。 
        他的手迟疑了一下,然后抓住我,将我拿出口袋。我不顾一切地咬了他的手一口。 
        他大叫一声,松了手。 
        我摔到草地上,爬起来就跑。 
        当他看清我的模样后,追过来。我往草多的地方跑,迷惑他的视线。 
        我只顾逃跑,没看见前方有一个下水道的铁网盖儿,我一脚踩下去,从铁盖儿的缝儿之间掉了下去。   第225集 
        饭盒船成为我的生命之舟; 
        乡村学校里的歌声; 
        善良的音乐教师; 
        冒名顶替参加大赛   
        从某种意义上说,这下水道救了我。我在坠落的过程中抬头向上看,那窃贼还守在铁盖儿旁边。 
        我掉进一条宽阔的大河,这是这座城市的地下污水河道。刺鼻的臭气令人窒息。 
        我不会游泳,灭顶之灾似乎已是定局,就在这时,我身边漂过来一个用过的一次性饭盒。 
        我抓住它的边缘,然后奋力爬进饭盒。我没被淹死,这个饭盒对我来说,是生命之舟。 
        污水河流动的速度挺快,我无法操纵我的船靠岸。河面越来越宽,水流也越来越急。 
        我索性躺在饭盒里听天由命。我想起了艾米,想起了乔治,他们现在一定为我担心。人生真是变幻莫测。 
        我的肚子饿了。饭盒里有一些残羹剩饭,实在难以下咽。可当我饿得受不了时,它们变成了山珍海味。 
        我一边流泪一边吃,我觉得委屈。人真是,多大的福也享不够,多大的苦也能受。我把肚子填饱后,只觉得嘴里发酸,当时如果有皮皮鲁牙膏,我非一次用光一箱子刷牙不可。 
        后来我睡着了。当我睁开眼睛时,我看见了光,我知道,我的船已经离开了地下污水河。 
        我坐起来往四周看,我的船漂流在一条河上。河的两旁有水草,远处有房子。 
        我必须想办法靠岸。我开始尝试利用身体的重量左右船的运行方向。经过一段时问的摸索,我渐渐掌握了。 
        在失败了十七次之后.我成功地登陆了。望着漂泊而去的饭盒,我百感交集。 
        终子又踏在了坚实的土地上,我有了安全感。据说生命起源于大海,可人如果离开大地就没着没落。大海是人类的母亲,大地是人类的父亲。 
        离我不远的地方有一幢小楼。我朝小楼走去。我不知道是什么命运在等着我。 
        当我走近楼房时,我昕到了歌声,是孩子的歌声。我明白了,这是一所乡间学校。 
        我听到歌声就像注射了兴奋剂,这是一首很美的歌曲。我在教室外边跟着唱,我忘记丁身边的一切。 
        当下课的铃声将我从梦境般的陶醉中惊醒时,最先跑出教室的两个男孩子发现了我。 
        “快来看!这是什么?”一个长着一双蓝眼睛的男孩子呼喊同学。 
        “小人!这么小的人!”另一个男孩子惊叫道,他朝我跑过来。 
        转眼间,我被孩子们包围了,他们都蹲下(禁止)子看我,但没一个人动我。 
        “去叫老师!”蓝眼睛说。 
        “胡安娜,去叫你妈妈。”另一个长着黄头发的男孩子对一个女孩子说。 
        被称做胡安娜的女孩子跑回教室。 
        音乐教师是胡安娜的妈妈,她以为女儿和同学们在和她开玩笑。 
        当她看见我时,很吃惊。 
        “真有拇指姑娘?”音乐教师像孩子们一样,跪在地上看我。 
        我不能放过这个机会,我刚才听了她唱歌,纯朴而清新,我想拜她为师。 
        我冲她招招手,示意她听我说话, 
        她小心翼翼地将我放在她的手心上,把我拿到她眼前。 
        “你能帮助我吗?”我同她。 
        “你会说话?”她瞪圆了眼睛。 
        “我想向你学唱歌。”我说。 
        “你是谁?从哪儿来?’她问。 
        “从中国来。”我把我的经历告诉她。 
        “我答应你。”她说。 
        “谢谢。”我向她鞠了一躬。 
        同学们鼓掌。 
        放学后,我跟着音乐教师和胡安娜来到她们的家。胡安娜的爸爸和妈妈在去年离婚了,音乐教师和女儿一起生活。 
        我喜欢农村的恬静。人类是一种爱凑热闹的动物,城市的诞生满足了人类的这一本性。城市越来越多,村庄越来越少。 
        胡安娜当时只有7岁。 
        音乐教师和女儿给我洗澡,换衣服。她们还为我制作了一张舒服的小床。晚餐后,我美美地睡了一觉。 
        第二天,音乐教师开始教我识谱,胡安娜和我一起学,她的理想也是当歌星。 
        那段日子真是愉快,无忧无虑。我们在一起谈音乐,谈贝多芬,谈中国,谈城市,谈农村,谈一切能谈的事情。 
        很快,我就识谱了。音乐教师开始教我唱歌,她说我的嗓了特殊,还说特殊是成功的前提。 
        我的歌声越来越成熟,每天晚上都成了我的独唱音乐会,我一天不唱,音乐教师和胡安娜就无法入睡。 
        两年后的一天下午,胡安娜拿着一张报纸来找我。 
        “报上说,电台举行儿童歌星大赛,每位参赛者寄一盘自己的录音带去。”胡安娜对我说。 
        “那你快录一盘寄去呀!”我说。 
        “我……唱得一般……”胡安娜吞吞吐吐。 
        “那怎么办?”我清楚胡安娜的嗓子,音乐教师说女儿搞声乐发展不大。 
        “你帮我录一盘。”胡安娜说。 
        “……我……录……”我不明白她的意思。 
        “录好了寄去,就说是我唱的。”胡安娜说。 
        “……那……行吗?”我觉得这样做不合适。 
        “帮我一回吧!我们可帮了你。”胡安娜央求我。 
        我也很想报答音乐教师和胡安娜,尽管我觉得这么做不道德,可我还是为胡安娜录了一盘磁带。 
        灾难的种子就这么种下了。 
        胡安娜获得了那次童星歌唱比赛的冠军。   第226集 
        脾气和蔼的音乐教师勃然大怒; 
        梦想为崇拜者签名的胡安娜; 
        坏蛋股份有限公司名扬四海; 
        歌唱家开始铁笼生涯   
        当音乐教师看到胡安娜给她的获奖证书时,很是吃惊。她不相信女儿的水平能获奖,而且是头奖。 
        当她弄清了是我帮胡安娜录的音后,她的脸色很难看,我头一次见她发这么大火。 
        “你怎么能干出这么申鄙的事?!”音乐教师训斥女儿,“不属于自己的东西,绝对不能享用。” 
        胡安娜低着头,不说话。 
        “你这是帮助她吗?”音乐教师转而训我,  “这是害她。凡是享用不属于自己的东西的人,上帝总会惩罚他的。” 
        “我……”我无言以对。 
        “你这样做,就是恩将仇报了。”音乐教师说出了令我汗颜和无地自容的话。 
        “我错了。”我看了胡安娜一眼,说。 
        胡安娜在一旁抹眼泪。 
        受人滴水之恩,当以涌泉相报是中国的一句老话。但报恩心切的人,往往干出恩将仇报的事。 
        从那以后,胡安娜好像变了一个人。她不再练唱歌,但她的歌星梦一刻也没有停止做,她的床头贴满了大歌星的照片。她羡慕他们的名气,羡慕他们的地位,羡慕他们的收入。 
        “被人崇拜不知是什么滋味儿。”一天晚上睡觉前,胡安娜对我说。我的床就放在她的枕头旁边。 
        “当然不错,不过一定也很累,到哪儿都得端着样子。”我说。 
        “人活一世,一定要出人头地,要让别人都知道你,要有很多的钱。”她的眼睛望着天花板说。 
        听了这话,不知怎么搞的,我的心中掠过一丝不安。有才能的人想出人头地是成功的动力。没才能的人想出人头地是灾难的开端。 
        “给崇拜者签名的感觉一定特棒。”胡安娜像吃了迷幻药。 
        “名人的签名对于崇拜者来说,确实有价值,但对于名人自己来说,毫无意义。被迫做对自己毫无意义的事,不是一种享受吧?”我没当过名人,但我能想像出那种心境。 
        “我喜欢。如果我是名人,我就站在大雨中给崇拜者签名,站在寒风中给他们签名,站在雪地里给他们签名,走到哪儿签到哪儿……”胡安娜一脸的憧憬。 
        这天晚上.我做了一个噩梦。我梦见我身边的胡安娜先是变成了一只大丹,人丹是德国名狗,和黑贝齐名。胡安娜变的大丹向我扑来,我极力躲闪。忽然,她又变成一条狼,我没命地跑,她在后边追。我看到前边有一个人,我就喊救命,那人一回头,是乔治。可乔治并不救我,他指着自己的心脏说,他的心脏是坏人的心,不应该救人。他还说,他专为坏人设计了一种坏蛋汽车,没想到坏蛋汽车大受欢迎,一上市就被抢购一空。他受此启发,干脆成立了一家坏蛋股份有限公司,公司的职员从董事长到看门打更的全是清一色的坏蛋,有男坏蛋女坏蛋老坏蛋小坏蛋。公司相继推出的坏蛋服装坏蛋食品坏蛋住宅坏蛋眼镜坏蛋帽子坏蛋休闲鞋成为最抢手的商品简直卖疯了根本不需要库房从流水线下来直接就进了消费者的购物袋,商场老板对别的厂家的方针是你给我产品时是我的儿子等到东西卖完了跟我要货款时就成了我孙子。可对坏蛋公司却来了个一百八十度,坏蛋公司的推销员到哪家商场都是爷爷,坏蛋公司的推销员严格说不叫推销员叫通用爸爸。乔治还说他到现在也没弄明白人们为什么讨厌坏蛋却喜欢坏蛋产品他还说有一家公司积压了几十万件文化衫老板天天为此愁眉苦脸吃喝嫖赌时都心不在焉老想着库里的那些堆积如山的包装箱,那老板受坏蛋公司的启发,把积压的文化衫都印上“我是坏蛋”四个大字,没想到销售一空,据调查,买坏蛋文化衫的消费者中有前科的一个没有,全是守法良民。乔治喋喋不休地说着,胡安娜在后边步步紧追,我一头栽进下水道…… 
        我醒来后发现自己全身大汗淋漓,我恐惧地看看在一旁熟睡的胡安娜,她的嘴角挂着笑容,她在做好梦。 
        在胡安娜19岁那年,音乐教师病故了。 
        她终于可以休息了。我呆在胡安娜的衣兜里参加音乐教师的葬礼时,这样想。 
        所有的人都会死。早死的人比晚死的人幸福。死不是死者的不幸,而是生者的不幸。 
        在音乐教师去世一个月后的一天,胡安娜对我说: 
        “我要去参加全国电视业余歌手大奖赛。” 
        “……”我一时不知说什么好。因为她的嗓于不行。 
        “你出歌声。我出形象。”她看看我,说。 
        我明白了。我想起了恩将仇报这句话。我不能再做这样的事,否则我对不起九泉之下的音乐教师。 
        在音乐教师这么多年的培养下,我的歌声已到了炉火纯青的境界,我还学会了作词作曲。 
        “你必须照我说的做,现在这个家我说了算。”胡安娜斩钉截铁地说。 
        我想起了许多年前的那个噩梦。恩师的后代的事。 
        我的话音剐落,一个铁笼子扣在了我的身上。我惊讶地看胡安娜,她将我囚禁了。 
        我和胡安娜在一起生活了12年,关系不亚子血亲,为了自己的利益,她居然能…… 
        十二生肖中缺一种属相。在人类的成员中,有不少人属狼。据说医学界正在尝试移植动物的器官给人。我断言,成功率最高的,保准是移植狼的器官。   第227集 
        胡安娜被切割成小块的身体; 
        歌唱家的故事讲完了; 
        图钉语惊四座; 
        舒利毅然出走   
        从此,我就失去了人身自由。 
        胡安娜不给我食物不给我水喝我都能忍受,但她不让我唱歌使我痛不欲生,我在铁笼子里一唱歌她就用棍子戳我。 
        胡安娜报名参加电视大奖赛。 
        她将我装在一个特小的全封闭铁笼子里去一座城市参加比赛。行前她警告我,如果我不配合,她回家就砸了她妈妈的墓。 
        我不能让我的恩师在天之灵得不到安息,我只得屈从。我从此再不相信什么遗传学。否则音乐教师绝对不可能生出这么一个混蛋来。人的生育,实际上是投胎。母亲生孩子不是付出而是接收。不必太认真。越认真,越失望。人类社会本质上是一座大福利院,全是领养关系。 
        胡安娜精心策划了登台演唱的每一个细节,她将我藏在她的乳罩里,让我对着麦克风唱歌,她对口型。 
        胡安娜的身材是第一流的。对我来说,却是丑陋的。同样的东西,在不同的背景下,可以生出不同的效果。 
        胡安娜登台了。我被她的胸部束缚住,说得确切些,是囚禁。 
        我的歌声把评委们惊呆了,他们统统给胡安娜打了满分。电视台主持人甚至做出了不让下边的参赛歌手再唱的决定。比赛才进行到一半,冠军就被宣布产生了,而且所有人都口服心服。 
        一夜之间,胡安娜大红大紫。 
        当天晚上,她什么也没干,苦练了一夜的签名。我在笼子里透过铁网看她,她的身体被铁网切割成无数小块儿。 
        胡安娜梦寐以求的生活到来了。她成为万众瞩目的新星,她的第一张唱片的销量突破了历史最高记录。她的钱包迅速膨胀,以至于到了止都止不住的地步。她身边聚集的为她服务的人越来越多。不管她走到哪儿,都成为人们瞩目的中心,她有七八部豪华汽车,有高级住宅。最重要的,是她拥有数以万计的追星族。而这一切,都是我的歌声给她带来的。 
        我曾多次试图逃跑,但都失败了。 
        在一次演出过程中,我拿出事先准备好的一根铁丝,突然猛刺她的乳罩,我想划破乳罩后越狱。胡安娜察觉到了,她中止了对口型,用双手从两侧往中间挤压我,歌迷们以为她在做取悦他们的挑逗动作,他们没命地吹口哨喊叫欢呼,我却在里边差点儿骨折。 
        我也尝试过自杀,没有一次成功。 
        如果让我用一句话来概括我对人类的看法,那就是:我爱人类。我恨人类。 
        如果让我用一句话来描述人类,那就是:乐亦苦,苦亦乐,皆在追逐中。 
        歌唱家讲完了她离开皮皮鲁家后的经历。 
        在场的所有朋友都听傻了。鲁西西、燕妮和舒利擦眼泪。皮皮鲁两眼冒火,舒克咬牙切齿。 
        图钉和贝塔的表情最奇怪。 
        图钉看着歌唱家发愣。贝塔的眼睛好像不敢看歌唱家,拐个弯看窗外。 
        歌唱家被图钉看毛了,她问图钉: 
        “你怎么了?” 
        “我爱你。”图钉甩出了重磅炸弹。 
        “你说什么?”歌唱家以为听错了。 
        “我——爱——你——”图钉大声重复了一遍。他被歌唱家的经历感动了。 
        “你?!”舒利惶惶地看着图钉。 
        “舒利,请你原谅我,现在我才知道,咱们那不叫爱情。从歌唱家说第一句起,我就觉得她身上有一块磁铁那样的东西吸引着我,可你身上没有这块磁铁。”图钉坦然地说。 
        舒利呆若木(又鸟),任凭泪水流淌。 
        舒克走过去安慰女儿,他告诉舒利,爱情这东西绝对不能强求,强求爱情就是播种灾情。 
        歌唱家不知所措地看着图钉和舒利。不知为什么,她还下意识地看了一眼贝塔。 
        贝塔眼睛仍然望着窗外,就在图钉向歌唱家求爱的一刹那,贝塔明白自己已经爱上了歌唱家,因为他的心刚才强烈震动了一下。 
        “你能接受我的爱吗?”图钉的艺术家气质上来丁,当众向歌唱家求爱毫不难为情。 
        歌唱家摇摇头。 
        “为什么?”图钉问,“因为我是老鼠?” 
        “老鼠不比人类差。”歌唱家说,“我还不了解你。何况你现在还是舒利的恋人。” 
        “我为有胡安娜这样的同胞感到羞愧。”燕妮插话。她同时想为歌唱家解围。 
        “你还应该为有乔治和艾米那样的同胞感到自豪。”歌唱家说。 
        “没有好人的民族是不存在的。没有坏人的民族也是不存在的。”皮皮鲁说。 
        “我们家原来有一个红沙发,红沙发里有一座音乐城,他们演奏的音乐好听极了。皮皮鲁,你还记得吗?”鲁西西问皮皮鲁。 
        “当然记得。还是你用妈妈的听诊器先发现的呢。”皮皮鲁想起了红沙发音乐城。 
        “是我离开你家以后的事吗?”歌唱家没听说过红沙发音乐城。 
        皮皮鲁和鲁西西点头。 
        “你们俩的经历太丰富了,编成电视剧,1000集也打不住。”燕妮说。 
        “还能找到红沙发音乐城吗?”歌唱家问。 
        “可以试试看。不过,咱们应该先找另外几个罐头小人。”皮皮鲁说。 
        “你还是先发明防微药吧,如果爱因斯坦家的老鼠的计划得了逞,人类可就惨了。”鲁西西提醒皮皮鲁。 
        皮皮鲁一拍脑袋,差点儿把大事忘了。 
        第二天,舒利失踪了。 
        她给大家留下了一封信。说得确切些,是出走。   第228集  
        图钉当研究生;  
        舒利决定独立生活;  
        舒克感谢歌唱家帮助女儿;  
        贝塔借酒浇愁   
        最先发现舒利出走的,是歌唱家。  
        头天晚上,大家是这样就寝的:皮皮鲁、燕妮和贝塔住在五角飞碟里。舒克和舒利住在鲁西西做的一座小房子里。歌唱家睡在鲁西西的枕头旁边。图钉随意睡在沙发上。  
        天刚亮,图钉就来找歌唱家聊天。  
        “我还想听您讲贝多芬故乡的故事。”图钉几乎一夜没睡,爱歌唱家爱得死去活来。  
        “大概就是那么多了。”歌唱家往舒利住的小房子那边看了一眼。  
        “你认为音乐是什么?”图钉拿出开学术讨论会的架式,俨然研究生请教导师。  
        “音乐是人类的一种通用语言。”歌唱家说,她有点儿无可奈何地看着图钉。  
        “我觉得音乐是动听的噪音。生命有时需要喧嚣。”图钉发表自己对音乐的见解。  
        歌唱家开始对图钉刮目相看了。她毕竟喜欢音乐,愿意和有共同语言的生命聊天。  
        舒利在小房子里通过窗口看着歌唱家和图钉交谈的场面,她表情呆滞。  
        “舒利,相爱是缘分。爱情不能强求。”舒克开导女儿,他清楚女儿的情感遇到了障碍。  
        舒利一言不发。  
        舒克轻轻叹了口气。应该说,作家最了解人的感情。实质上,舒克明白,女儿是失败者,恋人看上了别的异性。  
        舒利走出小房子,她来到歌唱家和图钉面前。  
        “我可以和你谈谈吗?”舒利问图钉。  
        谈兴正浓的图钉戛然而止,他勉强点点头,不情愿地跟着舒利走到一边。  
        舒利直视图钉:“你真的爱上歌唱家了?”  
        图钉眼睛看着地面:“是的。”  
        舒利:“……”  
        图钉:“真抱歉……”  
        舒利:“不用道歉,这是你的自由。”        
        图钉:“谢谢你的理解。”  
        舒利:“她爱你吗?”  
        图钉:“我想她会的。”  
        舒利:“祝福你们。”  
        图钉:“……”  
        舒利回到小房子里。  
        歌唱家一直在一旁注视着他们,她目送舒利走进小房子。当图钉再次要求和她探讨音乐时,她谢绝了。  
        快吃早饭时,歌唱家去叫舒利。  
        小房子里没有舒利,桌子上有一封信。  
        歌唱家看完信的内容后,大惊失色。  
        她跑出小房子,把舒利留下的信交给舒克。舒克看信:    
        舒克和所有的朋友们;  
        我走了。我想独立生活。从我出生起。就几乎没离开过爸爸。从歌唱家身上,我发现了一种魅力,这种魅力是独立生活的人所特有的。我也想自已去世界上闯荡,希望你们不要找我——如果你们真对我好的话。更不要使用五角飞碟找我。这是我唯一的要求。  
        舒利    
        大家不知所措。        
        “我下楼找她,也许还没走远。”鲁西西说。  
        “我去五角飞碟里遥感。”皮皮鲁说。  
        “都不用了。舒利说得对,她需要独立生活。独立生活的人身上有一种独特的魅力。我想起了我驾驶直升机离开妈妈的情景。”舒克说。  
        大家都不吭气了。  
        “都怪我。”歌唱家谴责自己。  
        “是你帮助了舒利。”舒克这句话极深刻。  
        皮皮鲁点头。  
        “依靠父母一生的人,等于没有生命。”鲁西西说。  
        “这就是名人之后很难成为名人的道理。”燕妮说,“表面上看,名人的后代很幸福。其实,他们的奋斗比一般人要难。”  
        “真的不找舒利了?”皮皮鲁问舒克。  
        “不找了。让她自己闯荡吧。”舒克说完转过身去。皮皮鲁猜测舒克哭了。  
        贝塔靠着五角飞碟始终一言不发。他知道自己已经爱上歌唱家了。从他参与营救歌唱家开始,他就喜欢上了她。贝塔自从被爱因斯坦家的老鼠小姐迷惑后,发誓今生今世不沾女鼠。贝塔听了歌唱家的曲折经历后,对她更是好感倍增。  
        爱情如果建立在崇拜上,其结果就是你在这个世界上失去一个可以崇拜的偶像。爱情如果建立在同情上,其结果就是你在这个世界上失去一个值得同情的对象。  
        贝塔对歌唱家的爱情不是崇拜也不是同情,而是一种共鸣。贝塔在歌唱家身上看到了自己。  
        图钉的出现使贝塔确信自己爱上了歌唱家,当图钉向歌唱家表白爱情时,贝塔感到自己全身的血液突然倒流,他有世界末日的感觉。  
        当他看到歌唱家和图钉津津有味地探讨音乐时,他自卑。其实贝塔不明白,同行的结合等于近亲结婚。再说透彻点儿,是一种事业同性恋。  
        失恋的感觉袭扰着贝塔,他万念俱灰。  
        皮皮鲁开始埋头研制防微药,燕妮给他当助手。鲁西西出去为皮皮鲁采购原料。  
        图钉仍缠着歌唱家讨论音乐。贝塔觉得世界上的事没什么值得讨论的,该干什么干什么就是了。历史不是讨论出来的。喜欢讨论的人绝对干不成大事。  
        贝塔自己躲在五角飞碟后边喝酒。他喝了很多酒,他希望通过大脑麻木来回避现实。不想了,就什么事也没有了。          第229集  
        贝塔看见了两个地球;  
        茶水泼了歌唱家一身;  
        五角飞碟能让地球上的人都有好运气;  
        歌唱家做最后的努力    
        随着一杯杯酒的下肚,贝塔的意识越来越朦胧,他的脑细胞像在开舞会。  
        贝塔眯着眼睛着屋里的一切,他发现每一样东西都变成了两个。他又瞪大眼睛看世界,他看见地球也变成了两个,一个地球上住着的都是幸运的人,另一个地球上住着的都是不幸的人。  
        “原来有两个地球,怪不得有人运气好有人运气差。”贝塔恍然大悟。  
        贝塔将手中的最后一杯酒喝光后,站了起来:“我去将那坏运地球上的人都迁移到好运地球上,然后炸毁坏运地球,让所有人都鸿运高照。”        
        贝塔为自己的伟大想法和善举兴奋不已,他摇摇晃晃地从五角飞碟后面走出来。  
        歌唱家和图钉正坐在五角飞碟舱门口谈论音乐。歌唱家看见了东倒西歪的贝塔。  
        “贝塔,你怎么了?”歌唱家站起来走上前去扶贝塔,她闻到了酒味儿。  
        “他喝醉了。”图钉说。  
        “你才喝醉了,只有喝醉了的人才没完没了地探讨问题。我很清醒。”贝塔想挣脱歌唱家扶他的手,其实他喜欢她的手接触他的身体。  
        歌唱家渐渐明白了贝塔是借酒浇愁,她看到贝塔酗酒后心里很难过,还有一种说不清的感觉。其实,当歌唱家在胡安娜家第一次见到贝塔时,她就对贝塔有好感。  
        “放开我!”贝塔挣脱了歌唱家的手,其实他想让这只手长在他身上。  
        “我去给你拿点儿解酒的东西。”歌唱家说完转身对图钉说:“你照看他一下。”  
        图钉走到贝塔身边:“您坐一会儿。”  
        “你是谁?你是小歌唱家?会唱几嗓子就了不起了?上帝给你声带是让你说话的,你却用它唱……”贝塔的舌头运转不灵活。  
        歌唱家拿着一杯茶水过来了。  
        “贝塔,喝点儿茶。”歌唱家将茶杯递到贝塔手中,她劝贝塔饮茶。  
        “我只喝酒,不喝茶。”贝塔抬手打翻了歌唱家递过来的茶杯,茶水泼了歌唱家一身。图钉瞪了贝塔一眼,他为歌唱家擦身上的水。  
        贝塔看着眼前的一切,他笑了,笑得挺吓人。歌唱家不让图钉给她擦身上的水。  
        “你们好好讨论吧,我要去帮别人搬家了。”贝塔说完朝五角飞碟舱门走去。  
        “搬家?搬什么家?帮谁搬家?”歌唱家问贝塔,她预感到要出事。  
        “你们说有几个地球?”贝塔斜着眼睛问歌唱家和图钉,像有重大机密要泄露。  
        “一个。”歌唱家示意图钉去叫皮皮鲁。图钉此时没了悟性,愣是站在原地不动。  
        “不对,有两个地球。一个好运气地球,一个坏运气地球。住在好运气地球上的人都有好运气,住在坏运气地球上的人都是坏运气。明白了吗?所以这个世界上有人运气好,比如说我。有人运气不好,比如说他。”贝塔指图钉,“我现在,要去把坏运气地球上的人都搬迁到好运气地球上去,从此天下再没有运气坏的人了。”  
        “你想怎么搬?”歌唱家心中猛然一惊,她意识到一个可怕的现实。  
        “当然是用五角飞碟。”贝塔踉踉跄跄朝五角飞碟走去。  
        “图钉,快去叫皮皮鲁!”歌唱家急了,她清楚,如果酗酒的贝塔驾驶五角飞碟飞出这间屋子,地球就真要变成坏运气星球了。  
        图钉跑到另一间屋子喊正在研制防微药的皮皮鲁。燕妮和皮皮鲁在一起。  
        “你说什么?”皮皮鲁还没完全从研制中摆脱出来,他以为听错了。  
        “图钉说,贝塔喝醉了酒,要驾驶五角飞碟出去。”燕妮替图钉重复了一遍。  
        皮皮鲁怔了片刻,他突然像火箭发射那样从座位上弹射起来,往五角飞碟停放的那间屋子跑。燕妮也意识到事情的严重性,她跟着皮皮鲁往外跑。  
        当皮皮鲁跑到五角飞碟旁边时,贝塔已经在关舱门了。满屋子都是酒气。  
        “贝塔!你要干什么!”皮皮鲁大喊。  
        五角飞碟的舱门无情地关闭了。  
        “你不能!”皮皮鲁急了,他跑到五角飞碟旁边,用力拍打舱门。  
        “舒克在里边吗?”燕妮抱着一丝希望问呆若木(又鸟)的歌唱家。  
        歌唱家摇摇头。  
        舒克在他的小房子里。他闻声从小房子里走出来,当他弄清楚发生了什么事时,傻了。舒克跑到皮皮鲁身边,提醒皮皮鲁用通讯器和贝塔通话。  
        当皮皮鲁找到通讯器时,五角飞碟已经摇摇晃晃地离开了地面。  
        “贝塔!贝塔!!请回答,我是皮皮鲁!”皮皮鲁趴在通讯器上试图同贝塔通话,现在,通讯器对皮皮鲁来说已经是庞然大物了。  
        “我…我…贝…塔…,请…讲…讲…讲…吧……”贝塔的酒劲儿全上来了。  
        “你关闭五角飞碟,不要胡闹。”皮皮鲁心急火燎地对贝塔说。  
        “谁…胡…闹…了…,我…这…是…去…帮…助…运…气…不…好…的…人…这…可…不…是…胡…闹……”贝塔驾驶五角飞碟在屋里盘旋。  
        “贝塔,你忘了你发过誓,不用五角飞碟干坏事。”皮皮鲁提醒贝塔。  
        “我这…算…干…坏事…吗…那…世界…上就…没有…好事…了…,地球…上…有…那…么多…不幸…的…生命…我…要…去…救他…们……”  
        “他怎么了?遇到什么事了?”皮皮鲁问歌唱家和舒克。  
        燕妮趴在皮皮鲁耳朵上说了几句话。  
        皮皮鲁看看歌唱家,歌唱家脸红了。  
        皮皮鲁明白了。          第230集  
        撒酒疯的五角飞碟;  
        手巾堵住燕妮的嘴;  
        四平八稳的家庭最有危机;  
        歌唱家坦陈心中的白马王子    
        “你劝劝贝塔,一定要制止他!”皮皮鲁对歌唱家说,“用一切办法。”  
        歌唱家来到通讯器旁边。  
        “贝塔,我是歌唱家,我知道你为什么烦恼,希望你能听我一句话。”歌唱家对着通讯器说,她的眼睛里已经开始出现泪花。  
        “你…说…吧……”贝塔回答。  
        皮皮鲁和朋友们都觉得有戏。  
        “只有一个地球。这地球上确有不少不公平的事。作为生命,必须能承受不公平。比如说我,被胡安娜奴役了那么多年,我也承受下来了。”歌唱家做贝塔的工作。  
        “我…痛…恨…不公…平…,我…现…在…就…去…铲…除…世界…上的所有…不…公…平…,我有五…角…飞…碟,我…能…做…到…”贝塔大喊。  
        五角飞碟在屋里加速转了几圈。  
        朋友们的心都揪到了嗓子眼。  
        皮皮鲁对着通讯器做最后的努力。  
        “贝塔,你听着,世界上不公平的事太多了,况且有些不公平的事并没有触犯法律,连法律都奈何不得,你更不能去管!”  
        “合…法…的…不…定…是…好…事…,违…法…的…不一定是…坏事…,皮…皮…鲁…你…也…挺…虚伪…,你…有…五角飞…碟…这…么先…进…的…武…器…,你…为…什…么…不…能…支持…正义…,你…们…的…老…祖…宗…不…是…有…个…叫…包…青…天…的…吗…你…完全…可…以…当…一个…皮…青…天…你…不…当…我当…,一会儿…全…世…界…都…知道…有…个…贝…青…天…了…哈哈……”  
        皮皮鲁无言以对。  
        贝塔驾驶着五角飞碟在屋子里做着各种飞行动作,一会儿绕圈,一会儿俯冲,一会儿悬停,飞行姿态毫无规律,像一个醉汉撒酒疯。        
        “没任何办法吗?”燕妮问皮皮鲁。  
        皮皮鲁摇头。  
        “但愿他别出去,出去就完了。”  
        “贝塔不会驾驶五角飞碟去闯祸吧?”燕妮清楚五角飞碟的所向无敌。  
        “他不是要铲除地球上的不公平现象吗?他要是出去了,那些制造不公平的人可就要倒霉了。”皮皮鲁忐忑不安地一边注视着五角飞碟一边说。  
        “贝塔,我和你一起去!”舒克施计。  
        “谢…谢…舒…克…还是…我…自己…去吧…我…会…给…你…搜…集…素…材…让…你…再写…一…部…长…篇…都…用…不…完……”  
        五角飞碟突然停在空中一动不动。  
        所有人的心脏都停止了跳动。目光从不同的角度网罗住五角飞碟。  
        “啪!”  
        五角飞碟撞碎玻璃,飞出了房间。  
        皮皮鲁呆呆地注视着窗外,嘴巴好长时问没闭上。燕妮用手巾堵住了自己的嘴。歌唱家用劲揪自己的头发。舒克一边叹气一边摇头。图钉不是很清楚五角飞碟的威力,只有他显得轻松。  
        刚从外边回来的鲁西西被这场面弄懵了,当她知道发生了什么性质的事后,一屁股坐在地板上。  
        半小时过去后,皮皮鲁最先清醒过来。        
        “他现在已经可以绕地球几百圈了。”皮皮鲁神情沮丧地说。  
        “贝塔喝这么多酒干什么?”鲁西西问。  
        “我看和歌唱家有关。”舒克点破了。  
        “和歌唱家有关?”鲁西西不明白。  
        “贝塔大概是爱上歌唱家了。”舒克看着歌唱家说。  
        大家看歌唱家。  
        歌唱家脸红了。  
        “真不好意思,我来了还不到一天,就给你们带来这么多麻烦。先是舒利走了,现在贝塔又走了,还开走了五角飞碟。”歌唱家检讨自己。  
        没人说话。  
        “我想离开这儿,行吗?”歌唱家问皮皮鲁。  
        “生活完全四平八稳,毫无曲折和戏剧性,也没有意思。许多家庭解体的原因其实就是平淡无奇。你别走。”皮皮鲁说。  
        “歌唱家,你爱图钉吗?”鲁西西认为到了让歌唱家表态的时候了。她担心过两天舒克说不定也会叛逃。  
        众目睽睽。  
        图钉更是目不转睛。  
        歌唱家看看大家,然后直视着图钉坚定地摇了摇头。        
        “为什么?!”图钉绝望地喊。  
        “不知道。反正不爱。”歌唱家说。  
        “你爱!你是内疚,觉得对不起贝塔和舒利。所以你当逃兵了!你说的不是心里话!”图钉冲上去摇歌唱家的肩膀,声音里有哭腔。  
        “图钉,我真的不爱你。你有唱歌的才能,但不一定凡是有歌唱才能的异性我都得去爱。刚才贝塔一走,我就知道我爱谁了。”歌唱家激动地说。  
        “你爱谁?”图钉眼睛红了。  
        “贝——塔——”歌唱家一字一句地说,咬字非常清晰,不容置疑。  
        “他?他身上一个音乐细胞也没有!”图钉提醒歌唱家。  
        “我身上全是音乐细胞,我不再需要别人的音乐细胞。”歌唱家说。  
        “祝福你和贝塔。”皮皮鲁对歌唱家说。  
        “祝福你们。”大家争先恐后祝福歌唱家。燕妮还吻了歌唱家。  
        当天晚上,图钉走了,他留下一张纸条。说要去人间的歌厅发展。  
        大家集体叹了一口气。  
        舒利和图钉先后离家出走后,都分别有极为精彩的经历。这是后话,暂且不表。          第231集  
        电脑荧光屏上的字无穷无尽;  
        贝塔的沉默令人毛骨悚然;  
        鲁西西别墅竣工;  
        喜欢游泳的燕妮    
        贝塔驾驶五角飞碟离开皮皮鲁家干的第一件事,就是用五角飞碟上的电脑查阅地球上所有不公平的事。  
        贝塔还是第一次喝这么多的酒,他的感觉是换了一个大脑,他很亢奋。  
        电脑荧光屏上开始按顺序显示世界上所有不公平的事。一排排小字由下向上移动,贝塔醉眼朦胧地死盯着。一个小时过去了。  
        荧光屏无穷无尽地倾吐着不公平,像是永无止境。贝塔惊讶,能造出电脑的人类,自身的人脑怎么能忍受如此之多的不公平?  
        一行行的字还是层出不穷,贝塔用包公的目光审视着它们。  
        “够了!从头开始,一件一件去制止!”贝塔意识到如果真等电脑结束显示世界上不公平的事,恐怕10年时间也不够用。他重新按键,让电脑从头定格显示。  
        贝塔决定从电脑显示的第一件不公平的事开始,依顺序去铲除邪恶。  
        他盯着屏幕,全身发烫,身上的每一个细胞每一条神经都透着使命感。  
        在这个世界上,醉酒的人有时反而最清楚,清醒的人有时反而最糊涂。  
        贝塔酗酒驾驶五角飞碟出走后,皮皮鲁让鲁西西打开电视机。  
        “你现在还有心看电视?”鲁西西感到奇怪。  
        “不是看电视,是看贝塔闯什么祸。电视台马上就有新闻可播了。”皮皮鲁说。  
        电视机开启后,1个小时之内没有贝塔的信息。  
        贝塔的沉默更使大家感到恐怖。  
        “他的酒大概醒了,也许一会儿就会回家。”歌唱家往光明方向构思。  
        “他喝了整整一瓶酒,一天之内醒不过来。”舒克保持低调。  
        “电视机一直开着,大家轮流监视。”皮皮鲁说。他清楚,离了五角飞碟,他们这些微型人就成了正宗的弱者。必须抓紧研制防微药和复原药。  
        鲁西西决定为皮皮鲁、燕妮、歌唱家和舒克盖一幢豪华别墅楼。她开始画图纸。  
        “你在于什么?”歌唱家问。  
        “给你们盖一座房子。没有了五角飞碟,你们得有个住的地方。”鲁西西说。  
        “楼房?”歌唱家看着图纸问。  
        “豪华别墅。”鲁西西说。  
        “设计可以请燕妮当参谋.她从小就住这种房子。”歌唱家建议。  
        一句话启发了鲁西西。  
        燕妮正在给皮皮鲁当助手。皮皮鲁马不停蹄地研制防微药,他估计爱因斯坦家的老鼠的后代已经快行动了。人类危险。  
        燕妮听说鲁西西给他们造别墅,乐了。她知道由于她和皮皮鲁变小了,给他们盖房子对子鲁西西这样的巨人来说,易如反掌。  
        “我估计有几个小时别墅就能竣工。”鲁西西口气特大,俨然超级建筑师。  
        燕妮给鲁西西出主意,卫生间设在哪儿,厨房设在哪儿,客厅设在哪儿。  
        “应该有座游泳池,最好是室内的。”燕妮说,“再盖一座健身房,还要给皮皮鲁设计一间试验室。”  
        “给我盖一间吊嗓子的隔音房间。”歌唱家说。        
        鲁西西完善着图纸。  
        “舒克住一层。歌唱家住二层。皮皮鲁和燕妮住三层。每层都有卫生间。游泳池也在一层。厨房和餐厅在二层。会客室在一层。健身房在三层……”鲁西西边画图纸边说。  
        歌唱家和燕妮表示满意。  
        “舒克在值班监视电视,我去征求他的意见。”歌唱家说完去问舒克。  
        舒克只说了一句话:  
        “在我的起居室旁边,给贝塔留一间,再给舒利留一间。”  
        鲁西西开始建造豪华别墅,说白了,就像塔积术一样。她使用的建筑材料都是家里有的东西,十分方便。  
        “要是有个巨人帮助人类盖房子就好了。”燕妮一边看一边说。  
        豪华别墅建造成功,非常漂亮。  
        燕妮去叫皮皮鲁。  
        沉浸在发明的兴奋之中的皮皮鲁见到豪华别墅吓了一跳,他还没见过如此精彩的建筑。  
        “哪儿来的?”皮皮鲁问。  
        “鲁西西给咱们盖的。”燕妮说。  
        “太棒了!”皮皮鲁迫不及待地往别墅里走。燕妮跟在后边给他当导游。        
        “这层是舒克的。这个房间是给贝塔留的,这间是舒利的。二楼歌唱家住,吃饭也在二层。对了,刚才还忘了带你看游泳池了。第三层是咱们的,这是卧室,这是书房,这是你的试验室……”燕妮向皮皮鲁介绍。  
        “不可想像,地球上亿万富翁的住宅也到不了这个水平。”皮皮鲁悟出了一个道理,小就是大,大就是小。越小越有回旋余地。  
        皮皮鲁将试验用品都搬进别墅的试验室里。歌唱家和燕妮也一通忙活。舒克比较深刻,没表示出特别的兴奋。  
        “我都想变小了。”鲁西西嫉妒地说。  
        “越小,空间就越大。大其实是一种不幸。”歌唱家深有体会地说。  
        “你就是因为小,才被胡安娜奴役了这么多年。”舒克说。  
        “胡安娜就是因为大,才落得这么悲惨的下场。”歌唱家说。  
        “咱们给这座别墅起个名字吧?”燕妮提议。她们家的房子都有名字。  
        “叫鲁西西别墅吧。”皮皮鲁说。  
        一致通过。  
        皮皮鲁开始在鲁西西别墅里发明防微药。燕妮和歌唱家在别墅的游泳池里游泳。          第232集  
        鲁西西看着皮皮鲁出神儿;  
        班干部长大后的名气;  
        人类变小后不是老鼠的对手;  
        鲁西西承担重任    
        燕妮游泳的姿势很漂亮,像鱼。  
        歌唱家不会游泳,但她要学。异国下水道的那段历险她至今记忆犹新,她学游泳是为了多一条生存之路。  
        燕妮教歌唱家学游泳。  
        鲁西西透过别墅的窗户看燕妮和歌唱家在游泳池里玩,羡慕极了。  
        “你们不是从德国带回一瓶微缩粒吗?我也要用一粒,变小太好了。”鲁西西要求变小。  
        表面看,越大越能主宰一切。其实,越小越能拥有一切。失去最多的,是大。        
        “那药在皮皮鲁的试验室里,你去问他要吧。不过,你可得想好。”燕妮从水里钻出来,她穿着游泳衣站在游泳池边,身上的线条美妙绝伦。  
        鲁西西将目光移到皮皮鲁的试验室的窗口里。皮皮鲁正在埋头研制微缩粒免疫药.从他的表情看,已是胜利在望了。他脸上有笑容。  
        “皮皮鲁让你服微缩粒吗?”燕妮从别墅里探出头问鲁西西。  
        “嘘——”鲁西西右手的食指垂直挡在自己嘴唇上,“小声点儿,他马上就要成功了.现在不能打扰他,顶多再有1个小时。”  
        燕妮点点头,返回游泳池继续教歌唱家游泳。歌唱家唱歌是天才,学游泳却是弱智,死活学不会,而且毫无悟性,怎么点都不通。  
        鲁西西静静地注视着鲁西西别墅三层窗口里的皮皮鲁,她知道现在的皮皮鲁是最幸福的:奋斗已接近成功但还没有成功。真正的天才成功后的喜悦感都不会保持多长时间,天才的喜悦感产生于奋斗的过程中,一旦成功,伴随他们的将是空虚和孤独。他们只得再在新的领域奋斗,寻找奋斗过程中的快感。只有低智商的人才会在偶尔的成功后喜悦终身。  
        鲁西西看着皮皮鲁出神儿,她想起皮皮鲁小时候的许多事情。有一次,皮皮鲁在家里的收音机里发现了一个神秘的频率,那个频率能预知第二天将        要发生的事情。结果,皮皮鲁被老师好一顿冤枉。还有一次,皮皮鲁发现打电话时先拨两个0就能打到别人心里,知道他想什么。皮皮鲁把电话打到班上的优秀生心里,结果发现优秀生张全全居然在考试前希望别的同学考试不及格。当他把张全全的心里话告诉老师时,老师勃然大怒,说皮皮鲁嫉妒优秀生。就是这样一个在学生时代受老师歧视的皮皮鲁,长大后竟然一举成名出人头地。鲁西西回想起皮皮鲁班上那些班干部和门门考100分的优秀生,他们在成年后并没有什么惊人之举。这是对教育的讽刺。拿皮皮鲁的经历就可以断言今天的学校教育是失败的。  
        鲁西西还想起了熊猫鲍尔,想起了豆芽兵,想起了自己在儿童时代曾经拥有过一枝能和动物通话的圆珠笔,还想起皮皮鲁曾经乘坐二踢脚爆竹上天。最让鲁西西激动的是,她想起了小时候居住的那栋楼房里的309暗室,银门里的觅工和致聪盔,名人大脑试验室。  
        世界上的事情真是千奇百怪,每天都有令人吃惊的事发生。每个人都想在一生中干出一两件让同类吃惊的事。人类社会就是在不断的吃惊中前进的。  
        鲁西西看见皮皮鲁站起来了,她知道微缩粒免疫药大概已经诞生了,又是一件让人类吃惊的事。成千上万的人中才会有一个人干出让人类吃惊的事,        而皮皮鲁一个人就干出了许多件让人类吃惊的事。  
        皮皮鲁走出试验室,他快步沿着楼梯来到别墅一层的游泳池旁。  
        身着三点式泳装的燕妮浮出水面,来到皮皮鲁眼前。皮皮鲁的大脑一时无法适应微缩粒免疫药与燕妮的胴体的反差,他傻呆呆地看着燕妮。  
        “成功了?”燕妮被皮皮鲁的目光盯毛了,她一边披浴衣一边问皮皮鲁。  
        皮皮鲁点点头:  
        “还需要做试验。’  
        “怎么试验?”燕妮问。  
        “先让一个人服免疫药,然后再给他用微缩粒。如果他不变小,就说明免疫药成功了。”皮皮鲁觉得燕妮穿上浴衣更美。一览无余不是美。真正的美必须给人留出想像的余地。想像是美的基础。  
        “用谁做试验?”燕妮问。  
        皮皮鲁愣了。他还没想到这个问题。  
        “我!”鲁西西的声音从别墅外边传进游泳大厅里,吓了皮皮鲁一跳。  
        皮皮鲁走出别墅。  
        “我来试验免疫药。”鲁西西说。  
        “这可有一定的危险。万一免疫药不起作用,你可就也和我们一样了。”皮皮鲁提醒妹妹。  
        “你现在找不到别人试验免疫药,再说,我刚才已经想服微缩药了,现在正好给你试验,如果免疫药成功了,我就变不小,如果失败了,变小也挺好。”鲁西西说。  
        “你看我们变小挺快活,那是因为有五角飞碟给我们撑腰。如果没有五角飞碟,这么小非得受人欺负。现在如果这屋子闯进来10只老鼠,咱们就完蛋了。”皮皮鲁说。  
        “所以必须赶紧试验,不然爱因斯坦家的老鼠一旦实施他们的计划,人类就完了。人类的体积如果和老鼠一般大,肯定斗不过老鼠,因为人类最大的本性是自己和自己打。”鲁西西说。  
        皮皮鲁只好同意。  
        试验在客厅进行。舒克、燕妮和歌唱家都在场,他们对鲁西西的牺牲精神表示钦佩。  
        电视上还没贝塔制造的新闻。  
        鲁西西先服用了微缩粒免疫药。  
        片刻后,她躺在沙发上,皮皮鲁将爱因斯坦家的老鼠的后代留给他的微缩粒放在了鲁西西的鼻孔前边。  
        屋里静得只听见钟表的秒针的脚步声。          第233集  
        吻天才就是吻世界;  
        中学生专用皮皮鲁牙膏;  
        相声电视新闻;  
        安东尼在荧屏上与燕妮重逢    
        1个小时过去了,鲁西西没有变小。  
        两个小时过去了,鲁西西还是原大。  
        “成功了。”皮皮鲁轻轻地说。  
        鲁西西一跃而起,她为人类高兴。  
        燕妮紧紧抱住皮皮鲁,热烈地吻他。她和皮皮鲁认识以来,头一次强烈感受到吻天才的享受。一个女人,在一生中如果没有吻过一个天才,实在是遗憾。天才的本质是创造。没有创造就没有世界。吻天才就是吻世界。天才每天都在创造世界上原本没有的东西。作为女人,拥有了天才就是拥有了世界。        
        皮皮鲁经常给人以意想不到,给人以惊喜,给人以新奇感。对于女性来说,意想不到、惊喜和新奇比首饰和服饰重要一万倍。所有女性都渴望意想不到、惊喜和新奇。得不到时,就转而追求服饰首饰。穿金戴银的女子其实表现的是一种无奈,一种对自己无缘拥有天才的补偿。珠宝商和服装设计师的竞争对手实质上是各路天才,是天才的稀有给珠宝商和服装设计师带来了庞大的市场。  
        “行了行了,我们都嫉妒了。”歌唱家对燕妮说。  
        “别说爱因斯坦家的老鼠,就是爱因斯坦本人的大脑也敌不过皮皮鲁的。”燕妮自豪地说,她没想到皮皮鲁能在这么短的时间里就发明了抵抗爱因斯坦家的老鼠奋斗了几十年的成果。  
        “应该给皮皮鲁的大脑上保险。”舒克说,  “起码值一千亿元。”  
        “必须马上将免疫药转人实用。人类数量太大,咱们先管孩子吧,孩子来到这个世界上时间短,不能让他们倒霉。”皮皮鲁说。  
        “有什么办法能让孩子服用微缩粒免疫药呢?绝大多数父母都不会信这种事。”鲁西西说。  
        “在德国时,贝塔建议将免疫药掺在皮皮鲁牙膏里,我看这个主意不错。现在的孩子都使用皮皮鲁牙膏,只要用了一盒,就会对微缩粒产生免疫力,老鼠的阴谋就破产了。”舒克说。        
        “这个主意不错。可是皮皮鲁牙膏一般都是小学生以下的孩子使用,中学生怎么办?”鲁西西主持舒克贝塔公司的日常工作,对市场了如指掌。  
        “咱们再专门为中学生生产一种青春型皮皮鲁牙膏,也就是中学生专用牙膏,将免疫药掺进膏体,确保中学生的安全。”燕妮提议。  
        “其实中学生早就该有自己的牙膏了,他们这个牙龄,用少儿牙膏太大,用成人牙膏又太小。”皮皮鲁说。  
        “就这么决定了,将微缩粒免疫药掺进皮皮鲁牙膏里,同时专门为中学生推出青春型皮皮鲁牙膏。”鲁西西拍板。  
        “刻不容缓,立即行动。”皮皮鲁凭直觉已经感到爱因斯坦家的老鼠在和他争分夺秒。  
        鲁西西马上就给公司的开发部经理挂电话,向他布置新业务。  
        皮皮鲁将免疫药的配方交给鲁西西,鲁西西迅速将配方传真给公司,并指示公司马上大批量投产。  
        “快看电视新闻!”舒克从屏幕上发现了异常。  
        所有人的目光都迅速投向电视屏幕,心也随之提升到喉咙口。  
        一位全身冒傻气的男播音员正在播新闻。电视台好像成心和观众过不去,绞尽脑汁把全世界最傻的人弄来当播音员,逼着观众把电视新闻当相声小品看。  
        “昨天,柏林出现了一件奇事。警察局的5名警察的身体突然变小,大约只有两寸高,其中包括该国最著名的神探安东尼。”男播音员的口气里有幸灾乐祸的成分。  
        屏幕上出现了变小后的5名警察,里边真有安东尼。他们身边是庞然大物似的一群人。  
        “我们认识的人!”皮皮鲁指着安东尼对鲁西西说,“爱因斯坦家的老鼠已经行动了!”  
        “我马上去公司,亲自督战,立即投产掺免疫药的少儿型皮皮鲁牙膏和中学生专用的青春型皮皮鲁牙膏。”鲁西西匆忙穿外套。  
        “越快越好!”皮皮鲁急了。  
        鲁西西跑着离开家。  
        燕妮看着电视荧光屏上的安东尼,笑出了眼泪。  
        “他准以为是我害了他。”皮皮鲁说,“安东尼见过咱们变小,当时他特吃惊。”  
        “其实他变小了更有利于破案。”舒克落井下石。  
        “全世界的人都变小了,对我有利。”歌唱家说,“我就可以光明正大当歌星了。”  
        屏幕上一位女记者将话筒伸到安东尼身边,她提问:  
        “请问作为大侦探,您对自己身体突然缩小有什么看法?”        
        安东尼面对巨大的话筒,神情还算镇静。  
        “我挺高兴,因为我喜欢的一位姑娘就这么小,我终于可以般配地向她求爱了。”安东尼居然还能幽默。  
        皮皮鲁拍拍燕妮的肩膀。  
        燕妮耸耸肩。  
        “您以为这是怎么回事?”女记者继续提问。  
        “有人想成全我和我爱的人喜结良缘。”安东尼继续幽默,他是冲着皮皮鲁说的。  
        “我该研制复原药了。”皮皮鲁往鲁西西别墅里走。  
        燕妮看到自己心爱的人又去创造新的业绩,她感到无比自豪和享受。她现在要做的准备,就是在他成功后吻他。  
        掺有免疫药的少儿型皮皮鲁牙膏和中学生专用青春型皮皮鲁牙膏迅速投放市场。小学生和中学生获得了对微缩粒的免疫力。有一些运气好的成年人偶然使用了皮皮鲁牙膏,他们将成为地球上很少的没有变成小人的成年人。          第234集  
        胶水一看数学书就头疼;  
        贝塔用电脑构思恶作剧;  
        想潇洒走一回的校长太太;  
        数学老师的罗圈腿    
        胶水上小学六年级。用一个词就可以形容他上学的心情:度日如年。  
        他的数学很差,他从生下来就对数学没有兴趣。上学后,他的数学考试分数总是徘徊在65—80分之间,他一看数学书就头疼。  
        胶水的语文却十分出色,他写的作文多次被语文老师当作范文在班上朗读。胶水喜欢看书,他从二年级就开始阅读大部头的文学名著,说来也怪,有些字他根本没学过,可一看就会。胶水对文字有天生的理解力和悟性,对数学却反应迟钝。  
        从胶水上小学一年级起,爸爸妈妈和老师就发现了胶水的这个短处,他们联合起来帮助胶水克服短处。六年都快过去了,胶水每天都在同自己的这个弱点搏斗。  
        不管胶水怎么用功,数学就是不行。渐渐地,他有了自卑心理,他一见数学书就害怕。他的语文再好也没用,老师要求全面发展。  
        明天又是数学考试,爸爸说,如果胶水这次的考分上不了90分,就再也不让儿子看课外书了。妈妈说,都是看课外书看的,影响了精力。  
        胶水之所以还能活下来,就因为有书陪伴。每当看书时,他就忘记了身边的一切。他觉得书能提高人生的质量。不看书的人,只有一个大脑,孤军奋战人生。看书的人,借用别人的大脑智慧丰富自己的人生。  
        胶水清楚自己明天的数学考试绝对到不了90分。他相信爸爸的诺言,他将不能看课外书了,他感到绝望。他手中的数学书,在他眼中模糊了。  
        胶水对自己的智力确实产生了怀疑,他清楚地记得老师曾当着全班同学说他:脑子行什么都行,脑子不行什么都不行。胶水同学语文好数学不好,说明他脑子不是真好,如果真好,数学也不会差。  
        胶水从一年级起就同自己身上的短处较量,他的主要精力被爸爸妈妈和老师强迫来改正所谓的缺点。        
        胶水将自己最喜欢的几本书装进书包,他计划明天数学考试后不回家了,他也不知道去哪儿,只要不离开书,去哪儿都行。  
        胶水是贝塔从五角飞碟电脑屏幕上看到的世界上第一件不公平的事。  
        贝塔驾驶五角飞碟去帮助胶水走出困境。贝塔觉得人类挺可笑,特聪明也特蠢,聪明到把航天飞机都弄到太空去了,蠢到在培养后代时让后代把全部精力放在自己的短处上。贝塔一边驾驶五角飞碟一边笑,他笑人类不懂自己。其实每个人都有长处和短处,没有任何一个人干什么都行。正确的教育应该是发现人的长处,培养他的长处,告诉他什么地方行。错误的教育是发现人的短处,揭露他的短处,告诉他什么地方不行。不少家长和老师让孩子将全部精力放在改正他的短处上,结果,短处没有变成长处,长处也没有了。  
        胶水就是这样,长处是语文,短处是数学。老师和父母发现了他的短处,要求他把这个短处变成长处。而才能是先天的,知识是后天的,知识并不是才能,才能是学不来的。将全部精力放在学不来的事情上,这不是毁人是什么?结果,胶水的短处长不了,长处也失去了发展的机会和条件。  
        贝塔越想越好笑,他要让胶水的爸爸妈妈和老师设身处地尝尝胶水过的是什么日子。        
        贝塔用五角飞碟的电脑编好了运作程序,他拿胶水的老师先开刀。  
        胶水的校长已经换上睡衣准备就寝,他躺在床上突然发呆。  
        “你怎么了?想什么?”太太温柔地问,她今晚很想潇洒走一回。  
        “我想给学校的老师调换工作。”校长说完自己也吃了一惊,他不明白自己的声带怎么好像不受大脑控制。  
        贝塔的恶作剧。  
        “换什么工作?”太太最讨厌先生在床上想工作。她认为好丈夫的惟一标准就是不在床上想工作。  
        “让数学老师去教体育,让语文老师去教外语,让外语老师去教数学……”校长说自己的计划。  
        “你疯了?为什么?”太太惊讶。  
        “作为校长,我应该对他们负责,让他们全面发展,比如说数学老师吧,体育是弱项,不能因为体育是他的短处就不让他教体育,相反,更应该让他把精力放在克服自己的短处上。”校长眉飞色舞。  
        “你吃错药了吧?”太太感觉不对。  
        “我现在就通知他们。”校长从床头柜上找出电话本.首先翻到了胶水的数学老师的电话号码。  
        “喂,是数学老师吗?我是校长。”  
        “校长好!这么晚了,有什么事?”        
        “从明天起,你不再教数学了。”  
        “不教数学了?那我干什么?”数学老师欣喜若狂,以为即将被提拔到校领导层。  
        “改教体育。”  
        “您说什么?”  
        “改教体育。”  
        “……教体育?我?”  
        “对。!”  
        “为什么?在体育方面我什么都不懂呀!”  
        “正因为体育是你的短处,学校才决定让你去教体育,全面培养你。”  
        “我是平足,还是罗圈腿,另外,您也不是不知道,我身高只有1.59米,当体育老师怎么现身说法以身作则身教重于言传?”  
        “你可以以自己的身材对学生进行反面教育嘛,就说你从小不注意体育锻炼,所以才把身体弄得这般紧凑。”  
        “您?”  
        “服从校领导的决定。”校长挂断电话,继续打第二个。          第235集  
        体育老师的脸红到肚脐;  
        蛙拳仰拳宣告诞生;  
        对弈变成了武林比武;  
        患者往医生脸上吐口水    
        第二天第一节课是数学考试,出现在教室里的却是外语老师。  
        同学们感到惊奇。胶水却挺高兴,他怕数学老师的挖苦。  
        外语老师宣却:  
        “学校决定,从今天起,我当你们班的数学老师,你们原来的数学老师改教体育了。”  
        同学们叽叽喳喳地议论起来。他们了解数学老师的身体状况,他们也听说过外语老师的数学水平。  
        胶水强烈感受到让数学老师去教体育课和自己拼命攻数学性质差不多。        
        数学考试结束了,胶水一边交考卷一边问新的数学老师:  
        “老帅,我这次考得怎么样?”  
        新数学老师的目光假模假样扫了一遍考卷,点点头,说:  
        “我看很好,不错。”  
        其实他根本看不懂。  
        胶水心花怒放,他终于碰上了好老师。在教育方法有问题的学校里,不管学生的老师是好老师。  
        第三节课是体育课,同学们与从前的数学老师喜相逢。  
        身着紧身运动服的前数学老师像一棵没发育好的葱,栽在全班同学的面前。  
        “立正!”前数学老师发口令。  
        同学们忍不住都笑了。前数学老师的两条腿呈O型,无法并拢。  
        “笑什么?严肃点儿!”前数学老师清楚同学们笑什么,他的脸一直红到肚脐。  
        前数学老师此时恨透了校长,他不明白校长为什么要将他的缺点暴露给学生们,让他出丑。他的数学从小就是他的强项,而体育从来不及格。  
        一个人的优点和缺点自己和别人都清楚,很多人一生在同自己的缺点搏斗。其实,缺点不用改,重要的是发挥优点。有的人改了一辈子缺点,到头来优点一点儿没发挥,成了一个名副其实的只有缺点的人。  
        每个人都有优点和缺点。只看到自己优点的人,实质上就成了只有优点的人。只看到自己缺点的人,实质上就成了只有缺点的人。  
        当前数学老师给同学们做跳高示范时,他悟出了以上真理。他连跳了7次都没跳过0.3米的高度。如果他从此苦练跳高,任凭他挥洒多少血汗,也只能培养自己那与日俱增的自卑感。数年后当他重返数学讲台时,将发现自己连阿拉伯数字也认不全了——没有自信的人都犯这毛病。  
        胶水的爸爸是某体校校长。贝塔在他上班后也和他玩了一回。  
        体校校长一到学校就召集全校老师开会,连他自己也不知道为什么,他有强烈的鬼使神差的感觉。  
        体校有田径班、篮球班、足球班、武术班、棋类班、游泳班、跳水班……  
        “今天开会,是要宣布一项新决定。”校长说,“让田径班的学生改学跳水,让武术班的学生改学棋类,让游泳班的学生改学武术,让跳水班的学生改学足球……”  
        教练们大眼瞪小眼。  
        “校长,咱们分班可是根据学生的特长分的,要是让田径班的去改练跳水,非拍死几个不可。”田径班教练反对。  
        “我那武术班的学生有的连国际象棋和中国象棋都分不清,让他们改学下棋,不是误人子弟嘛?”武术班教练说。  
        “正因为他们不懂下棋,才让他们去学,以此培养他们的奋斗精神,磨练他们的坚强意志。”校长的口气十分坚定。“就这么定了,现在你们去执行吧。”  
        教练们面面相觑。武术班教练在心里已经将校长来了十几遍大背跨了。  
        跳水班的学员在足球场上老做空中转体动作,气得足球教练昏了头骂自己的祖宗。  
        田径班的学员爬上10米跳台后全身哆嗦,一个个被教练强行从跳台上推下,当场因肺部呛水被急救车送走两名。  
        武术班的学员下棋愁眉苦脸,后来双方干脆直接动手厮杀,还把劝架的棋类班教练打得鼻青脸肿。  
        最逗的要数游泳班学员改练武术,蛙拳仰拳自由拳蝶拳从此诞生武坛,乐得贝塔守着荧光屏笑得死去活来。  
        培养人才的最好方法是扬长避短。扼杀人才的最好方法是取长补短。  
        不明白这个道理,就没资格为人父为人母为人师。        
        贝塔又将五角飞碟的“袭击”目标定在了胶水的妈妈身上。胶水的妈妈是一家小医院的副院长,这天碰巧院长出去开会了,胶水的妈妈上着上着班突然召集各科室主任开会。  
        会议的结果就是肛肠科医生和口腔科医生对调,皮肤科医生和骨科医生对调,肿瘤科医生和妇产科医生对调……  
        一位口腔疾病患者坐在医生的桌子旁边,医生示意他张开嘴。  
        患者张开了嘴。医生举着手电翻来覆去在患者嘴里找异物,一边找一边摇头。  
        “没有痔疮呀!”医生刚从肛肠科调来,还喜欢使用那边的术语。  
        “你说什么?”患者火了。  
        “难道是内痔?”医生执迷不悟。  
        患者将一口唾沫儿吐在医生脸上。  
        贝塔在五角飞碟里笑得满地打滚儿。  
        新上任的妇产科医生用摘除肿瘤的动作给孕妇接生,这可是人命关天的大事。贝塔尽管酗酒,还是意识到不能开这种玩笑,他赶紧通过五角飞碟让胶水妈妈收回对调的指令,否则胶水的妈妈非得因为医疗事故蹲监狱不可。这世界上的法律的确很童话:扼杀人的性命犯法,扼杀人的前程不犯法。从某种意义上说,前程和性命同等重要。          第236集  
        胶水的爸爸撕数学书;  
        银行是罪犯的保险柜;  
        浴缸里的伊丽莎白接电话;  
        没有想像力的总编犯了大错误    
        当天晚上,胶水的爸爸妈妈回到家里后不但不问胶水的数学考试,反而抢着撕儿子的数学课本。胶水在一旁直发傻。  
        “你们干什么?”胶水以为爸爸妈妈准备撕他所有的书,包括课外书。  
        “你从今天起可以放弃数学了!”爸爸甩出了一句振聋发聩的话。  
        “对,数学是你的短处,我们不应该把眼睛盯在你的短处上。从此我们的目光只看你的长处。”妈妈整个一个豁然悔悟茅塞顿开。  
        胶水有重获新生的感觉。        
        贝塔得意地摇头晃脑。他看电脑屏幕上的第二件不公平的事。他的酒意丝毫没减。  
        五角飞碟将世界上所有财产来历不明的人的名单显示出来,足足有几千万人。这些人靠非法手段攫取公众和私人财产,形成了遍布世界的不公平。  
        “这么多人!”贝塔喷着酒气,“如果一个个收拾太慢了,不如连窝端吧。”  
        贝塔构思着方法。  
        还都是地球上有权势的人。贝塔头一次知道,银行里的存款有五分之一是赃款。  
        从这个意义上说,银行其实是坏蛋的保险柜。  
        贝塔决定先通过世界上最大的通讯社发全球通电,宣布冻结没收银行里的所有不义之财,然后将这些钱建造公众设施。  
        伊丽莎白是地球上最大最权威的通讯社的记者部主任。她忙了一天后,正泡在浴缸里沐浴。她认为最享受的事就是疲劳不堪后洗澡。  
        电话铃响了。  
        伊丽莎白沾满肥皂沫的手臂伸出浴缸,从墙上摘下电话听筒。  
        贝塔打来的电话。  
        “你好,请问是伊丽莎白小姐吗?”贝塔越醉越有礼貌,就像表面看越坏实际上并不坏的人一样。  
        “我是,您是?”伊丽莎白不熟悉对方的声音,只要同她通过一次电话,她就忘不了。  
        “对不起,我还不能告诉您我的名字。不过这并不影响我们交谈。我想送给您一条大新闻,能轰动世界的新闻。”  
        “请讲。”伊丽莎白见过大世面,曾经单枪匹马去战乱国家采访,并且有被俘被绑架被劫持的经历,什么都见过。  
        “这世界上有很多人的财产是不义之财,您信吗?”贝塔问。  
        “当然。”伊丽莎白喜欢这个话题,她痛恨那些发不义之财的人,比如那些受贿的歹徒。  
        “我知道地球上所有财产来历不明的人的准确人数和他们的国籍姓名住址。”贝塔一口气说下去,他挺激动。  
        “……”伊丽莎白显然不信。  
        “您感兴趣吗?”贝塔问。  
        “当然……不过,您是怎么知道的?”  
        “人类干过许多可笑的事,对吗?”  
        “是的。”  
        “其中最可笑的,就是只信能解释通的事,对于解释不通的事,绝对不信。”  
        “没错。”  
        “我有一种极其先进的仪器,能遥感世界上一切事。”        
        “您不信?”  
        “难以置信,请原谅。”  
        “这样吧,我现在就遥感您在干什么,行吗?”贝塔想出了让伊丽莎白相信的方法。  
        “可以,如果您能猜出我现在正在干什么,我就对您的话百分之百地相信。”伊丽莎白泡在浴缸里说。  
        贝塔的荧光屏上出现了伊丽莎白沐浴的镜头。贝塔耸耸肩,他没想到伊丽莎白是位貌美的白种人小姐。  
        “您正在洗澡。”贝塔说。  
        “!!!”伊丽莎白愣了。  
        “坐在浴缸里,您现在用双手捂住了自己的胸部。您顾不上冲掉香皂沫,在找浴衣。”贝塔几乎采用与伊丽莎白的动作同步的语言叙述。  
        伊丽莎白终于醒悟了,原来世界上并没有童话。所有被冠之以童话的东西其实都是现实。  
        “您的账号上共存有78615元,对吗?您的存款密码是8809…”  
        “我信了。您是谁?”伊丽莎白想弄清打匿名电话者的身份。  
        贝塔尽管醉了,还没忘记为自己和五角飞碟保密。  
        “我决定在明天上午八点,将地球上所有来历不明的财产统统没收。您可以发个通电,公布我的这一措施。”  
        “太好了!您会非常公正吧?比如说,不会制造冤假错案吧?不会把人家辛辛苦苦挣来的血汗钱移走吧?”伊丽莎白高兴之余又有些担心。  
        “绝对不会。比上帝还公正。”贝塔说。  
        “我现在就向全世界发通电。”伊丽莎白中止沐浴,穿着浴衣跑进书房打字写稿。  
        贝塔在五角飞碟里跳迪斯科。  
        伊丽莎白写完新闻稿,开车赶到通讯社。  
        值班总编见伊丽莎白深更半夜赶到通讯社,知道有了重大新闻,眉开眼笑。  
        伊丽莎白将新闻稿递给值班总编。  
        “这是愚人节的玩笑吧?”值班总编看完新闻稿后抬起迷惘的眼睛看伊丽莎白。  
        “绝对真实,请快签字发通电。”伊丽莎白催促值班总编辑。  
        值班总编又看了一遍,摇头:  
        “不行,这会砸了咱们的牌子。”  
        “您如果不发,才是砸了咱们的牌子。”伊丽莎自提高了音量。  
        “这话怎么讲?”值班总编不明白。        
        “如果您坚持不发,我只好将它送给别的通讯社了。”伊丽莎白说。  
        “太好了,请吧!”值班总编巴不得让竞争对方出洋相,“你要能让别的通讯社发这条新闻,我给你发奖金!”          第237集  
        伊丽莎自在稿件上签字;  
        鲁西西叫醒皮皮鲁;  
        歌唱家喊贝塔万岁;  
        舒克骂贝塔贝塔不生气    
        伊丽莎白拿起新闻稿头也不回跑出通讯社,她开车来到了另一家通讯社——她供职的通讯社的竞争对手。  
        对手通讯社的主编对于敌方主将深夜来访感到震惊,他就缺伊丽莎白这样的名记。  
        “请坐,请坐!”对手通讯社主编殷勤地招呼伊丽莎白落座。  
        “这份新闻稿您要不要?”伊丽莎白开门见山地将新闻稿递给主编。  
        主编看完稿件,他抬起头看看坐在对面的伊丽莎白,当他确信伊丽莎白不是梦游后,又将新闻稿看了三遍。  
        “贵社不用这稿?”主编推断。  
        伊丽莎白点头。  
        “使用这稿件时可以属上您的名字吗?”主编问伊丽莎白。他觉得只要能属伊丽莎白的名字,全是假的也没关系。  
        伊丽莎白再次点头。  
        “请您在这儿签字。签字后,我们马上发稿。”主编让伊丽莎白在稿末签名。  
        伊丽莎白签字。  
        “希望这不是我们的最后一次合作。”主编一边叫秘书拿走稿件一边对伊丽莎白说。  
        “这条新闻一发,最迟到明天中午,贵社就是全球最有名的通讯社了。”伊丽莎白极为自信地说。  
        主编半信半疑地点点头。  
        第二天清晨,世界各通讯社都转发了伊丽莎白的新闻,他们对这条新闻的真实性几乎不屑一顾,但是他们对伊丽莎白的名字感兴趣——既然名记敢和公众开这样的玩笑,大家都愿意奉陪并借机提高自己的知名度。  
        鲁西西从广播电台的早间新闻中听到了这条新闻,她来到别墅旁叫皮皮鲁。  
        皮皮鲁研制复原药到深夜,现在睡眠正值(禁止)。燕妮推醒他。        
        “鲁西西叫你。”燕妮对皮皮鲁说。  
        皮皮鲁推开窗户。  
        “刚才电台播了一条新闻,说是请全世界所有有不义之财的人注意,今天上午八点,全球所有银行存款中的不义之财将统统自行消失。还说经过精确统计,人类中共有多少多少人贪污过,多少多少人受贿过。还说越是有不义之财的人表面看越清廉,还说往这机构那机构捐款越多的人心里越有鬼。还说由此可见,越革命的人骨子里越反gemin。”鲁西西向皮皮鲁复述新闻的内容。  
        “贝塔干的!他终于行动了!”皮皮鲁对鲁西西说,“我马上下去。”  
        皮皮鲁脱下睡衣,燕妮递给他衣服。  
        皮皮鲁和燕妮下楼,他们经过二楼时,歌唱家已经等在那儿,她听到了鲁西西刚才的话。  
        舒克也站在一楼的出口处。  
        当他们走出鲁西西别墅时,鲁西西已将房间里的电视机打开了。电视台中断了正常节目,不停地插播这一新闻。  
        “肯定是贝塔干的。”歌唱家问皮皮鲁。  
        “除了五角飞碟,谁还能干这种事?”皮皮鲁不知是自豪还是自责。  
        “五角飞碟不会出差错吧?全世界有那么多银行,银行里有那么多存款,别把善良人辛辛苦苦挣的钱给弄没了。”燕妮有些担心。  
        “五角飞碟的误差是零。你放心,不义之财一分也跑不了。正当收入一分也没不了。”舒克说。  
        “如果我现在还被胡安娜奴役,今天胡安娜的钱就全没了?”歌唱家问。  
        “对。”舒克说。  
        “贝塔万岁。”歌唱家认定世界上还有无数个胡安娜,他们今天都会痛不欲生。  
        “如果他回来,再不许他喝酒了。”皮皮鲁盯着电视屏幕说。  
        “那些有不义之财的人的赃款不该没收吗?”歌唱家问皮皮鲁。  
        皮皮鲁无言以对。  
        “我觉得,五角飞碟没有起到应有的作用,你们早就应该用它在全球主持正义。”歌唱家为贝塔鸣不平了。  
        “人类有法律。”皮皮鲁说。  
        “法律应该改名叫乏力。”歌唱家深有体会地说,“如果你们到法院起诉胡安娜,说胡安娜奴役我,法院会宣判她有罪吗?”  
        大家不吭声了。谁都明白,法院不会拿歌唱家当人类的一员对待。  
        “试着和贝塔联系一下,看看他的酒醒了没有。”舒克提议。   
        鲁西西拿来五角飞碟通讯器,舒克开始不停地呼叫贝塔。  
        “贝塔,贝塔,我是舒克。请回答。”  
        “贝塔,贝塔,请回答,我是舒克。”  
        大家眼巴巴地看着通讯器。  
        “我是贝塔,什么事?”贝塔回话了。  
        “你在哪儿?”歌唱家问。  
        “在天上。”贝塔说。  
        “回来吧,我有话对你说。”歌唱家说。  
        “我还没办完事呢,办完事就回去。”贝塔不上勾,“问图钉好。”  
        “图钉已经走了。”歌唱家向贝塔报喜。  
        “有戏。”舒克小声说。  
        “我起码要主持三次公道,现在才是第二次,还有一次,主持完就回去。”贝塔显然被图钉离开的消息鼓舞了。  
        “这已经是第二次了,头一次他干了什么?媒介怎么没报道?”皮皮鲁小声说。  
        “别嘀嘀咕咕了,我听见了。等我回去告诉你们吧。”贝塔声说。  
        “这个混蛋。”舒克骂贝塔。  
        “我爱你,舒克。”贝塔一点儿不生舒克的气。        
        “贝塔,制裁时别出误差。”歌唱家提醒贝塔。  
        “放心吧,不冤枉一个好人。不放过一个坏人。”贝塔特贫。          第238集  
        牛滑的八十万赃款付之东流;  
        贝塔乘胜进军;  
        财务小姐被人行注目礼;  
        有个人胃里着了火    
        他叫牛滑,表面看道貌岸然,彬彬有礼,还是议员,其实一肚子坏水。  
        牛滑从小就偷(又鸟)摸狗,心眼儿绝对没长正。少年时代曾往人家的汽车上涂写脏字而被拘留并遭事主痛打。成人后悟出一个道理:想干坏事就不能有坏样儿,必须披一张羊皮干狼事,那才叫高手。于是牛滑重塑外表,经过数年努力,这厮居然当上了议员。  
        有了议员的身份,牛滑再干坏事就上档次了,他的收人百分之百是不义之财。一边参政议政,一边为非做歹。  
        这天早晨,牛滑照例听新闻。当他听到伊丽莎白那条新闻时,愣了一下。  
        他不信,可又心虚,一过八点,牛滑就拿着存款单到银行去了。他在银行存了八十多万赃款,用的是假名。  
        有许多人在银行门口徘徊,看得出,心情都和牛滑一样忐忑不安。  
        牛滑戴着墨镜将存款单交给营业员小姐。他怕别人认出他来。  
        “对不起,您的存款已经没有了。”小姐查底后和蔼地告诉牛滑。  
        “为什么?”牛滑急了。  
        “不知道。”小姐冲牛滑一笑,“您早晨看新闻了吗?”  
        牛滑不知为什么摇摇头。  
        小姐从柜台里递给牛滑一张报纸。  
        “这是今天的报纸,您看看这条新闻。”小姐意味深长地对牛滑说。  
        报纸上刊登了伊丽莎白的新闻。  
        “无稽之谈!”牛滑声色俱厉,“堂堂银行,对储户要讲信誉,怎么能轻信谣言呢!”  
        “我们接到政府通知,遇到您这样的情况,一律记下您的身份证号码和住址,先查财产来源,然后再决定怎么办。”小姐递给牛滑一张登记表。  
        牛滑软了,他不敢填写这张表,因为他无法说出这八十多万元的来源。  
        牛滑攫取的不义之财就这么付之东流了,他想生吞了下这事的人。  
        “您快走吧。”小姐小声对牛滑说,“听说,一会儿警察就要进驻我们银行,凡碰上您这样的,一律逮捕审查。”  
        牛滑昕罢掉头就跑。  
        全世界的牛滑都被贝塔治了。各国政府弹冠相庆,银行里的巨额存款早就弄得他们夜不能寐,并称之为关在笼子里的老虎。现在好了,老虎没有了,剩下的都是羊羔了。而且这些储户竟然没有一个人敢出来抗议。  
        伊丽莎白在全球大出风头,她原来供职的通讯社总编从六十层楼上跳了下来,不这么做不足以表现他的痛心疾首。  
        发伊丽莎白新闻的那家通讯社一跃而为全球通讯社之大哥大,那老板奖励伊丽莎白50万美元,还高薪聘她为该通讯社社长。  
        人们最关心的,是伊丽莎白从哪儿得到的这条新闻。还有,是谁通过什么方法查清了地球上所有不义之财,而且如此准确如此明察秋毫。据说某国一位妇人在27年前曾捡到一元钱,后连同其他正当收人一并存人银行。今天上午,她的这一元钱以及27年来的利息一起化为乌有,吓得那妇人连连给上帝磕头并发誓今后再不敢犯坏了。  
        贝塔又给伊丽莎白打了电话,通知她再发一条新闻,就说一个小时之后地球上所有在银行之外的赃款将自行起火燃烧。  
        伊丽莎白兴奋得差点儿脑溢血。  
        各国新闻机构百米赛跑似的抢发这条新闻,人们有的高兴得发疯有的气得发疯。  
        一个小时后,地球上许多房屋突然起火,原因都是钞票起火所导致。还有人走在大街上在会议室里在社交场合在总统府里装在兜里的钱突然起火弄得当事人不知所措无地自容。  
        有一位总统在出访他国享受仪仗队的诌媚时身上的钞票冒了烟,吓得安全保卫人员全都拔出手枪,以为有敌对势力恐怖组织往总统身上安放了定时炸弹。  
        还有一位公司的财务小姐想买一件时装但又没有那么多钱她就挪用了公司的款项去买时装当时她正在试农室试新装没想到挂在勾子上的旧衣服里的公司款就着了起来进而烧毁了她的原装,结果该财务小姐只好在众目睽暌之下跑回公司,一路上不知被多少坏男人死盯着行注目礼夹道迎送。就在该小姐快跑到公司门口胜利在望时,她的内裤突然起火。原来该小姐日前哄骗经理说应该和税务局搞好关系最好送税务小姐一件毛衣经理同意后该小姐在给税务小姐买毛衣的同时给自己也买了一件还觉得不过瘾又给自己买了两条裤衩。贝塔在五角飞碟里越玩越来劲儿干脆把赃款买的不义之物也列入焚烧程序。  
        电视台记者的镜头全部对准了火灾,皮皮鲁、燕妮、歌唱家和舒克坐在电视机前目不暇接。歌唱家拍手称快,还说今生今世非贝塔不嫁。舒克手特痒痒,后悔自己没参加这一壮举。  
        “贝塔成纵火犯了。”皮皮鲁说。  
        “人类应该给这样的纵火犯发勋章。”歌唱家已经对贝塔崇拜得五体投地。  
        “你们看看窗户外边,快成火海了。”鲁西西刚从外边回家,说。  
        朋友们都被鲁西西拿到窗台上。  
        大街上的不少人都在忙着灭自己身上的火,有的人衣服着了,有的人提包着了,有的人鞋子着了,有的汽车着了。还有几个人胃里着了火,因为他们刚用公款吃完饭。他们在抢一个灭火瓶,轮流嘬里边的灭火剂。  
        “我服了贝塔了。”鲁西西说。  
        “也就他干的出来。”舒克特嫉妒。  
        “他在替天行道。”歌唱家说。  
        “五角飞碟从前是不是有点儿大材小用?”燕妮看着皮皮鲁说。  
        皮皮鲁不吭气。          第239集  
        贝塔在欧洲上空;  
        成田的儿子没有皮皮鲁书包;  
        电视评论员输了一万元;  
        汽车是流动的城堡    
        贝塔的酒劲儿慢慢醒了,他惊讶地看着四周,他感到头挺疼。  
        “我这是在哪儿?”贝塔自言自语,他看了一眼方位仪表盘。  
        “欧洲上空两万米!”贝塔极力回忆,“我怎么一个人跑到欧洲来了?”  
        贝塔隐隐约约回忆起自己的所作所为,他吓出了一身汗,忙打开五角飞碟的记忆系统,像看录像那样看自己干过的事。  
        贝塔的心稍微踏实了,他为自己酗酒后干过的一系列事感到自豪。        
        贝塔走进卫生间,用冷水冲头。  
        他看到了镜子中的自己,贝塔想想觉得好笑,一只老鼠居然能把世界折腾得天翻地覆,全仗着五角飞碟。可见科学技术越发达,人类离末日就越近。  
        贝塔回到驾驶舱,他准备返航了。他知道回家后等待他的必将是训斥,弄不好这辈子再也别想沾五角飞碟了。  
        贝塔在好奇心的驱使下,又打开遥感仪,想看看排在不公平的第三位的是什么事,他为它的运气不好惋惜——轮到解决它的问题时,贝塔的酒醒了。  
        不看则已,一看贝塔又想路见不平,拨刀相助了。贝塔不能眼看一个善良正直的人终身受冤屈。贝塔要帮助他,让他得到和自己的品质成正比的待遇。  
        他叫成田,45年前降生在一个不穷不富的国家,活到现在没有干过一件损人利己的事,活得却一天不如一天。亲友和过去的同学都嘲笑他。成田认为正直和良心是一个人最重要的财富。正直的百万富翁不少,不正直的穷人也不少,可成田却是一个正直的穷人,精神上是亿万富翁,物质上一贫如洗。  
        成田上小学的儿子在同学面前为自己摊上了一个无名鼠辈的爸爸而自惭形秽,他不明白为什么是同龄人却有的人的爸爸是名人是富翁有的人的爸爸是普通人是穷人。每当他问爸爸或向爸爸要零花钱时,成田心里都特别不好受,他清楚,为人父就应该有个父亲样,起码应该让自己的孩子有生活在天堂里的感觉,无论是精神上还是物质上。儿子想买一个皮皮鲁书包说了两年他都无法满足儿子的这个要求。成田上个星期捡到了一个装有一万元现金的提包然后分文不少地交还失主。成田清楚,在家庭关系中,最可怕的事就是贫困家庭溺爱孩子。贫困家庭溺爱孩子是种祸根。  
        贝塔决定利用五角飞碟帮助成田又出名又挣钱,他要让成田感受到上帝的存在。  
        人类最想知道还没发生的事的结局。尽管顺应这种“市场’需求诞生了不少预言家,但真正能说准的还几乎一个也没有,不是说歪了就是大相径庭,还没有一个真正能让世人口服心服的预言家。  
        贝塔要让成田出这个风头挣这笔钱填补这个空白当这个领域的鼻祖。  
        本届世界杯足球赛决赛明天在成田居住的城市举行,贝塔选中这场举世瞩目的球赛让成田成名。  
        成田喜欢看足球,他正坐在电视机前看体育节目评论员预测明天的决赛谁胜谁负。成田眼前突然出现了明天决赛的结局,而且还有准确的分数:A国队以三比零胜B国队。现在球迷普遍看好B国队,包括电视台的体育节目评论员。  
        成田产生了想去告诉电视台体育节目评论员明天决赛结局的念头,开始他觉得荒唐,但这种念头越来越强烈,最终征服了他的躯体,成田动身去了电视台。  
        著名体育评论员接待了成田。  
        “明天的决赛是A国赢,比分是三比零。您如果这样预测,您将声名大震。”成田对评论员说。  
        “您怎么知道?”评论员客气但矜持地看着成田的眼睛问他。  
        “预感。”成口说。  
        “几乎可以肯定,是B国得冠军。”体育评论员胸有成竹地说。  
        “我将比赛结果写在纸上,给您一份,等决赛结束后,您就相信我了。”成田递给评论员他写好的预测。  
        “我可以和您打赌,B国赢。”评论员当着一屋子同事对成田说。  
        同事们都用嘲笑的目光看成田。  
        在电视台同事们的“公证”下,成田和评论员打了一万元的赌。  
        第二天,世界杯足球决赛结果:A国队以三比零胜B国队捧走冠军杯。三个球都是在最后三分钟内踢进去的。  
        成田准确预测球赛结果的新闻盖过了A国队大爆冷门的新闻,众多记者争先恐后采访成田。 
        “您还能预言其他事吗?”一记者问。  
        “明天本市共发生交通事故23起,死亡7人,伤29人。其中,市区13起,郊区10起。”成田如数家珍。  
        记者们马上将成田的预言公之于世。  
        第二天,交通事故的数字与成田的预言丝毫不差。成田顿时声名大噪,各新闻媒介均在成田的名字前冠以“大预言家”的头衔。  
        每个人都想知道自己以后的命运会不会锦上添花会不会时来运转会不会甜尽苦来会不会突遭横祸。每个国家都想知道今后有没有发展有没有战乱有没有瘟疫。人们都想在成田嘴里找到答案他们花钱竞争比赛谁有本事让成田先给他预测一家汽车制造公司的老板不惜花血本送给成田豪华轿车当见面礼还说名车必须名人开才般配还说没有名的人开名车是浪费是糟蹋是自己打自己嘴巴说一千道一万老板的目的是让成田预测他的公司能否击败竞争对手。成田一边试车一边说汽车是流动的城堡现代人绝对离不了谁也不愿意死住在一个地方可是老搬家又太麻烦汽车的诞生解决了这个难题成田告诫那老板不要靠降低售价竞争世界上根本就没有价廉物美只有货真价实。汽车老板茅塞顿开转身又送给成田一辆流动的城堡。          第240集  
        歌唱家在纸上涂写贝塔的名字;  
        贝塔自称替天行道;  
        皮皮鲁当证婚人;  
        划时代的婚礼    
        成田一举成名后,贝塔的酒完全醒了,当他清楚自己酗酒后擅自驾驶五角飞碟外出为所欲为后,吓出一身冷汗。  
        贝塔打开遥感仪,观察皮皮鲁他们在干什么。荧光屏上出现了鲁西西别墅。  
        “这么漂亮的房子!”贝塔脱口而出。  
        皮皮鲁在试验室里发明复原药。舒克在看电视。燕妮躺在床上看书。  
        “歌唱家呢?”贝塔操纵遥感仪找歌唱家,他的心脏也随之增加了跳动的频率。  
        歌唱家在自己的房间趴在桌子上写着什么,贝塔让遥感仪显示歌唱家面前那张纸的特写镜头。  
        一张纸上涂满了字,全是“贝塔”两个字。贝塔受宠若惊,他听说过,情人在热恋中就特爱在纸上无意识地反复写恋人的名字。  
        贝塔迫不及待地要返航,他想先通报一声,免得吓朋友们一跳。  
        “皮皮鲁,我是贝塔,请回答。”  
        “舒克,我是贝塔,听见了没有,请回答。”  
        舒克听见了通讯器里贝塔的呼叫声。  
        “我是舒克,我是舒克,贝塔你在哪儿?”舒克一直在等贝塔的消息。  
        “我在拉丁美洲上空,我想回家,可以吗?”贝塔的声音让人觉得挺可怜。  
        “当然可以,快回来吧。”舒克生怕吓着贝塔,导致他不敢回来。  
        “皮皮鲁很生气吧?”贝塔想预先知道对自己的“量刑”标准。  
        “有点儿,但也不太生气,你毕竟没干太出圈儿的事。”舒克往皮皮鲁的窗口看了一眼。  
        “就是,我干的都是替天行道的事。”贝塔给自己脸上贴金。  
        “得了得了,你也够损的。快回来吧,我们给你打开窗户。”舒克说。  
        当贝塔驾驶五角飞碟降落在鲁西西别墅旁边时,朋友们都等在那里。  
        飞碟停稳后,贝塔半天没敢打开舱门。  
        舒克上前敲五角飞碟舱门。  
        贝塔打开舱门,磨磨蹭蹭地从飞碟里走出来。舒克立刻从他身边钻进飞碟,掌握了五角飞碟的控制钮。  
        “出来吧。英雄!”皮皮鲁对贝塔说。  
        贝塔眼皮也不敢抬。  
        “怎么办?”贝塔问。  
        “你说怎么办?”皮皮鲁问。  
        “宽容是一种美德。”贝塔背名言。  
        “今后不许再喝酒,一滴也不能喝。”皮皮鲁说出了处罚方法。  
        贝塔点头同意。这比他预想的处罚要轻得多,他最怕再不让他开五角飞碟。  
        歌唱家冲上来拥抱贝塔。贝塔不知所措,满脸通红。  
        “我爱你,你是真正的男子汉。”歌唱家大声向贝塔表达爱情。  
        “失恋就酗酒,不算男子汉。”舒克从五角飞碟里探出头,说。  
        贝塔瞪了舒克一眼,紧紧拥抱歌唱家。  
        歌唱家由于受胡安娜奴役多年,对人类已经失去了信心。贝塔被爱因斯坦家的鼠小姐骗一回,对老鼠也失去了信心。歌唱家和贝塔的结合有了互补性。  
        “早知道酒后驾飞碟能找到老婆,我也去了。”舒克逗贝塔。  
        “咱们为歌唱家和贝塔举行一个婚礼吧?”鲁西西提议。她觉得贝塔和歌唱家结婚意义不一般。  
        “我赞成”,燕妮说,  “我去帮歌唱家收拾新房。”  
        “我给你们做结婚礼服。”鲁西西说。  
        贝塔看着歌唱家,他感到自己很幸福,他毕竟是老鼠家族中第一个娶人类姑娘的。  
        鲁西西开始给歌唱家和贝塔设计和制作结婚服,她想起了30年前第一次发现罐头小人时给他们做衣服的情景,时间过得真快。人生短暂得令人吃惊,不到四十岁的人绝对体会不到。  
        燕妮将歌唱家和贝塔的新房收拾得富丽堂皇,贝塔和歌唱家住鲁西西别墅的二层。  
        婚礼即将开始。  
        皮皮鲁当证婚人。  
        当身着婚纱的歌唱家在燕妮的陪伴下走出鲁西西别墅时,她的美貌博得了朋友们的掌声。  
        贝塔都看傻了。  
        歌唱家看到身穿西服的贝塔,也觉得他英俊潇洒。歌唱家认定找伴侣主要是找心,其他一切都可以不考虑。遗憾的是人类的许多成员在完成这一人生必然程序时目光总是盯着心以外的东西。  
        “你为什么要娶歌唱家?”皮皮鲁用男中音的嗓子问贝塔。  
        “因为她是人。”贝塔的回答令人回昧。  
        “你为什么要嫁给贝塔?”皮皮鲁用同样的问题问新娘子。  
        “因为他不是人。”歌唱家的回答耐人寻味。  
        “你会永远爱她吗?”皮皮鲁问贝塔。  
        “说不好。但现在爱。”贝塔说出有史以来在结婚典礼上新郎说的最伟大的一句话。  
        “你呢?”皮皮鲁又问歌唱家。  
        “希望能永远爱他。”歌唱家也不玩虚的。  
        “人只和人通婚,本质上也是近亲结婚。”舒克说,“贝塔和歌唱家生的孩子准聪明,说不定还是个国家元首胚子呢!”  
        舒克只是随便开句玩笑,没想到后来的事还真验证了他的这句玩笑话。贝塔和歌唱家生的孩子果然与众不同,世上已有狼人猫人,但有鼠人还是头一次。鼠人误食皮皮鲁发明的复原药后身体变得和人一般大,他叱咤人类政坛,演出一系列精彩故事,此为后话,暂且不表。          第241集  
        鲁西西要求参加喜庆婚宴;  
        皮皮鲁说出有史以来最伟大的结婚祝辞;  
        新闻和垃圾同时播放时千万别生育   
        喜庆宴会将在鲁西西别墅的餐厅里举行,贝塔挽着新娘在朋友们的簇拥下正要往别墅里走,鲁西西说:  
        “我也要参加贝塔和歌唱家的结婚宴会。”  
        大家闻声站住了。  
        “你这么大,怎么走进去呢?”燕妮抬头问庞然大物鲁西西。  
        “给我一颗微缩粒。”鲁西西说。  
        “复原药还没研制成功,你现在服用微缩粒,万一复原药成功不了,你可就永远变不回去了。”舒克提醒鲁西西。        
        “我看变小也挺好。”鲁西西说。  
        “那谁管公司的事?”歌唱家说,“公司的总经理不能只有火柴棍大吧?”  
        鲁西西不说话了,片刻后,她问皮皮鲁:  
        “你没有把握研制出复原药吗?”  
        自从电视上披露了安东尼变小的信息后,皮皮鲁就日以继夜地研制复原药,可以说,他现在已是稳操胜券了。  
        “有把握,只是还需要几天时间。”皮皮鲁说,“给你一颗微缩粒。”  
        鲁西西乐了。  
        “去把家门锁好,把食物往别墅里多放点儿。”皮皮鲁说。  
        “我把五角飞碟挪到鲁西西别墅旁边,以防万一。”舒克说。  
        “我给你拿过来就是了。”鲁西西伸手将五角飞碟拿到别墅旁边。  
        “还是大好。”舒克羡慕大。  
        “都是因为你小了一辈子。”鲁西西羡慕小。  
        “说穿了,都是互相羡慕,没什么想什么。”贝塔总结。  
        鲁西西检查家门锁好了,又将丰富的食物搬进别墅许多,还储备了水。  
        “行啦,我可以变小了。”鲁西西说。        
        皮皮鲁回到鲁西西别墅的试验室里取出一颗微缩粒,放在鲁西西鼻子前边。  
        鲁西西变成了小人。  
        “太棒了!”当鲁西西看到周围的一切突然变得无比巨大时,兴奋不已。  
        朋友们走进鲁西西别墅,为贝塔和歌唱家举行新婚喜庆宴会。  
        鲁西西和燕妮下厨房操办菜肴,舒克和皮皮鲁忙着布置餐厅。  
        贝塔和歌唱家走进新房,他们被一种从来没有过的感觉笼罩着,这里将是他们把两个生命对接在一起的熔炉。  
        喜庆宴会开始。  
        皮皮鲁首先为贝塔和歌唱家祝酒:  
        “贝塔和歌唱家都是我和鲁西西的老朋友,在我们小时候,咱们的友谊为我们的童年增添了无穷的乐趣。今天,贝塔和歌唱家喜结良缘,我非常高兴。”  
        “别说太俗的祝辞。”鲁西西打断皮皮鲁,她怕他说“白头到老”之类的假话。  
        “祝贝塔和歌唱家随心所欲,顺其自然。干杯!”皮皮鲁喝光了杯中的酒。  
        “这是世界上最伟大的结婚祝酒辞。”舒克为皮皮鲁的祝辞喝彩。        
        “祝贝塔和歌唱家随心所欲顺其自然!”大家一同重复皮皮鲁的祝辞并干杯。  
        歌唱家眼睛里含着泪花,她为自己拥有这么多心地善良善解人意一片真诚的朋友而感动。  
        贝塔喝光了杯中的水——他戒酒了,今天是以水代酒。  
        “歌唱家为我们唱一支歌吧!”鲁西西提议。歌唱家回来后,还没唱过歌。  
        歌唱家为胡安娜唱了7年歌,她的嗓子已经不属于她自己了,今天,她要用自己的嗓子为朋友们唱歌。  
        当你意识到你的东西不再属子你时,你的生命就没有了质量。当你想占有不属子你的东西时,你的生命已不是生命。  
        歌唱家不想占有别人的东西,也不想让别人占有自己的东西,她一直在努力达到这个境界。今天,她终子如愿以偿。她开始为朋友们演唱,她的歌声不是从声带发出的,是从心脏直接产生的。  
        朋友们被歌唱家的歌声震撼了,这才叫歌,每一个音符每一句歌词都是对生命意义的诠释。  
        地球上的生命还在竞争,此时此刻,鲁西西别墅成了世外桃源,没有金钱没有世俗没有困扰生命的一切难题,只有歌声和真情。  
        皮皮鲁、鲁西西和燕妮活到今天,才体会到什么叫幸福和惬意。  
        “贝塔,祝你早日当爸爸。”鲁西西逗贝塔,敬他一杯水酒。  
        “我肯定能生一个伟人。”贝塔特有把握地说,好像他天生就是给伟人当爸爸的料。  
        “这话怎么讲?”皮皮鲁不明白贝塔依据什么放百分之百的缔造一代伟人的大话。  
        “我研究过了。你们看,咱们坐在这儿,表面看都好好的,其实每个人都是万箭穿身。”贝塔开始阐述他的生伟人的理论。  
        “万箭穿身?”燕妮转动身体,没有感觉。  
        “空气中布满了各种电波,有电视台的,有电台的,有BP机的,有手机的。这些电波每时每刻都穿过我们的(禁止)。”贝塔一边说一边用手做利箭穿心状。  
        “这倒是。”皮皮鲁点头。  
        “生什么样的孩子,就看父母有这个孩子的那次(**)发生在什么时间,如果届时电视台或厂播电台正在播相声或小品节目,这个孩子生下来准特幽默。如果正在播放数学讲座,这孩子将来是数学家的可能性特大。”贝塔说。  
        “如果正在播新闻呢?”燕妮表示出对贝塔的理论的极大兴趣。  
        “百分之百是政治家。不信你去查当今官场上的人物,他爸他妈有他时的电视台准是在播新闻节目。”贝塔说,“你们是不是发现有不少政治家特虚伪?嘴上说一套心里想一套?”  
        “没错。”燕妮深有体会地说。  
        “这说明,他爸他妈有他时,广播电台在播新闻,而电视台却在播垃圾污染环境的画面。”贝塔说。  
        “你准备在电视台电台播什么节目时要孩子?”皮皮鲁问贝塔。  
        “这是我的专利,保密。”贝塔看了歌唱家一眼。  
        宴会结束后,皮皮鲁和燕妮抢着看《广播电视报》。          第242集  
        燕妮的男佣成为老鼠的男佣;  
        老鼠科学院瓜分世界;  
        鼠小姐乘飞机去中国;  
        胖老鼠和鼠小姐套瓷    
        自从皮皮鲁释放了爱因斯坦家的老鼠以后,他们日以继夜地加速实施将人类变小的计划,他们是在同皮皮鲁的大脑竞赛。  
        “咱们有优势,”老鼠科学院院长对下属说,“咱们叫以动员全世界的老鼠同胞帮忙,而皮皮鲁却不可能说服人类吃预防药,别人会拿他当疯子。”  
        “咱们必须多设计几种微缩粒传播方式,比如往人类的饮用水里投放,像艾滋病那样传染,利用蚊子传播……”一鼠博士献策。  
        科学院院长点头,他给部下分工。  
        燕妮家的别墅已无人居住,老鼠们决定占用整栋房子作为大批量制作微缩粒的加工厂。  
        “将那男佣变小,让他给咱们当男佣,咱们太需要劳动力了。”科学院院长说。  
        一天夜里,燕妮家忠心耿耿的男佣人睡后,老鼠们将一颗微缩粒放在了他的鼻孔前边。  
        当男佣醒来时,他发现自己身边全是巨大的老鼠,他再往远看,家具都变成了巨无霸。  
        “从现在起,你就是我们的佣人了,你要尽心尽力,听从我们的调遣。”一只老鼠对男佣说。  
        “这……这是怎么回事?”男佣站起来,他的体积和老鼠一般大小。  
        “这是科学,我们把你变小了,我们还要把全人类都变小,然后公平竞争。你很荣幸,是人类中第二个变小的,你们管这个名次叫亚军。”另一只老鼠得意地说。  
        男佣想反抗,被老鼠们制服了,他终子成为老鼠的奴隶,为老鼠干活。  
        整座别墅成了老鼠科学院的微缩药加工厂,他们大张旗鼓地在房子里生产微缩药。  
        “这样还是太慢,咱们应该把配方交给各国的老鼠同胞,各国的老鼠承包自己的国家。”一位足智多谋的爱因斯坦后代向院长建议。  
        “你是说,在世界各地建分厂?”院长觉得这个建议很好。        
        “咱们分工,一人去一个国家。”那老鼠说。  
        “就这么定了。”老鼠科学院院长点头,他恨不得明天就把全人类变小。  
        “我去美国。”一只老鼠自告奋勇。  
        “我负责法国。”又一只老鼠请战。  
        “瑞士我包了。”  
        “我管澳大利亚。”  
        “日本给我吧。”  
        “我要南非。”  
        “我去中国。”鼠小姐说。不知怎么搞的,和贝塔有过那么一次以后,鼠小姐的脑海里总是出现贝塔的身影。这世界很怪,有时终身厮守,却毫无印象。有时偶尔为之,却刻骨铭心。  
        老鼠们趴在地图上瓜分了世界,男佣在一旁违心地为他们做记录。  
        当天晚上,老鼠们带着样品配方分头出发了,只留下院长和分管德国的老鼠看守大本营。  
        鼠小姐比较顺利地米到了国际机场,她在寻找去中国的飞机。  
        鼠小姐去中国的真正目的除了扩散微缩药以外,还要找贝塔,她的肚子里,已经有了贝塔的孩子。鼠小姐没将这事告诉科学院的同胞。  
        去中国的飞机被鼠小姐找到了,但要登上这架飞机还有相当的难度,鼠小姐抓住机遇,混在行李里边进入了飞机的货舱。  
        孕妇坐在飞机的货舱里旅行,鼠小姐不免为老鼠家族的地位感到悲哀,但她相信这种悲剧不会再重演了。只要人类的体积和老鼠一样大,驾驶飞机的就将不再是人类而是老鼠了。鼠小姐对这一点坚信不疑。  
        飞机平安抵达目的地,鼠小姐决定先办公事,然后再去给孩子找爸爸。  
        老鼠找老鼠是很容易的,尽管国籍不同,但它们身上有相同的气味。鼠小姐首先要同机场的老鼠取得联系。  
        两只闲逛的老鼠看见了来自异国的外籍鼠小姐。  
        “你好,你很漂亮。”一只胖老鼠上前对鼠小姐说。中国的老鼠生活很开放,在对待异性上像西方人。  
        “谢谢。我想见你们鼠王。”鼠小姐说。  
        “见鼠王?”瘦一些的老鼠上下打量鼠小姐,“你想见全城的鼠王还是本机场的鼠王?”  
        “最好是全城的鼠王,如果麻烦,见你们机场的也行。”鼠小姐说。  
        “先见机场的吧,全城的鼠王连我们也见不到,级别在那儿摆着。”胖老鼠的眼睛总是盯着鼠小姐,他是那种会用眼睛进行性骚忧的老鼠。  
        “我到现在还没见过部级老鼠。”瘦老鼠对鼠小姐说,“跟我们走吧。”  
        鼠小姐跟着两位异国同胞进人一个阴冷的下水道。  
        “打听一只老鼠,叫贝塔,听说过吗?”鼠小姐边走边问。  
        “贝塔?这名字好像有点儿耳熟,一时想不起来。”胖老鼠拍拍头。”  
        “贝塔?”瘦老鼠站住了,“听我爷爷说,咱们老鼠历史上好像有个什么舒克贝塔航空公司。”  
        “我想起来了,我姑姑在我小时候给我讲过一个故事,贝塔是开坦克的。”胖老鼠说。  
        “不对,贝塔开飞碟。”鼠小姐认为他们说的不是一个人,“再说,贝塔一点儿也不老,特有活力。”  
        “也许是重名。”瘦老鼠点点头。  
        “鼠王的住处到了。”胖老鼠指着前边一个洞口对鼠小姐说。  
        “你在这儿等一下,我先进去禀报。”瘦老鼠对鼠小姐说。  
        胖老鼠趁瘦老鼠去禀报的时候,猛和鼠小姐套瓷,无奈有身孕的鼠小姐一般不爱理异性。          第243集  
        吃航空盒饭的鼠王;  
        孔子是爱因斯坦的爷爷;  
        罗丘的后裔;  
        机场鼠王让贝塔改名    
        机场鼠王的王宫很有航空特色,鼠王的宝座是用飞机轮胎加工的,鼠王面前的桌子上摆着飞机上使用的盒饭。鼠王身旁还有一架精致的飞机模型。  
        鼠小姐将来意告诉机场鼠王。  
        “这怎么可能?”鼠王盯着外国米的鼠小姐,“你是说能把人类变成和咱们一样小?”  
        “千真万确。”鼠小姐点头。  
        “您听说过爱因斯坦吗?”  
        “爱因斯坦?一种药?”鼠王说。  
        “爱因斯坦是人类的一位大科学家。”鼠小姐给机场鼠王扫盲。        
        “我们鼠王是航空方面的专家,什么飞机都知遭。”鼠王侍从维护鼠王的尊严。  
        “爱因斯坦是我们德国人,很有名。”鼠小姐略带几分自豪感地说。  
        “我们鼠王给名人下过一个定义。”鼠王侍从讨厌这位来自发达国家的同胞语言中的优越感。  
        “名人是给普通人的生活提味儿的佐料。”鼠王说完特得意,“名人实质上是普通人的奴隶。”  
        “原子弹就是爱因斯坦发明的。”鼠小姐不理会鼠王的人生哲理,继续说。  
        “原子弹!”鼠王知道原子弹,“这么说,爱因斯坦是我们的朋友了,他发明了能够毁灭我们的宿敌——人类的武器。”  
        “我的祖先是爱因斯坦家的老鼠。我们家族世世代代从事……”鼠小姐还没说完,被侍从打断了。  
        “我们大王的祖先是孔子家的老鼠。孔子,知道吗?”鼠王侍从不喜欢鼠小姐拿名人压同胞。  
        “孔子?不知道。”鼠小姐摇头。  
        “什么?你连孔子都不知道?我告诉你,孔子起码可以给爱因斯坦当爷爷。”鼠王侍从说。  
        鼠小姐耸耸肩。  
        “你接着说。”鼠王对鼠小姐的话开始感兴趣了。  
        “我的家族世世代代研究将人类变小的方法,我们成功了。”鼠小姐说。        
        “你们真的掌握了将人类变小的方法?”鼠王的眼睛里射出了一道兴奋的光。  
        “这是微缩药的配方。我的家族现在向全球的同胞推广微缩药,让全球的人类同时变小。”鼠小姐掏出配方递给鼠王。  
        “咱们老鼠有出头之日了。”鼠王感慨万千。  
        “如果人变得和咱们老鼠一样大,他们斗不过咱们。”鼠王侍从说。  
        ‘绝对斗不过。”鼠小姐说。  
        “你是让我生产这种药?”鼠王问鼠小姐。  
        “通过您,把这个配方交给贵市的鼠王,再由市鼠王转交给贵国的鼠王国家元首,再由鼠王国家元首下旨命令全国的鼠同胞一起生产微缩药并向人类投放。”鼠小姐将实施步骤告诉机场鼠王。  
        “我直接禀报敝国鼠王元首就行了。”机场鼠王说,  “此事行动一定要快,不然的话,外国的人类先变小了,敝国的人类该警觉了。”  
        “贵国没有等级制度?”鼠小姐对于一个基层鼠王能直接向国家元首鼠王禀报感到惊讶。  
        “国家鼠王是我们鼠王的姑夫。”侍从充满自豪地告诉鼠小姐。那口气就像国家鼠王是他姑夫似的。  
        “那太好了,您现在就去吧。”鼠小姐催促。  
        “你现在就回国吗?”机场鼠王问鼠小姐。  
        “我想在贵国找一位叫贝塔的同胞。”鼠小姐一提贝塔,她下意识地摸摸肚子。  
        “贝塔!”机场鼠手双手抱拳,做了一个尊重的手势。  
        “您认识贝塔?”鼠小姐看出机场鼠王和贝塔的关系不一般。  
        “舒克贝塔航空公司的创始人之一。我的祖先的挚友。”机场鼠王说。  
        “我们鼠王的祖先是大名鼎鼎的罗丘,曾任舒克贝塔航空公司副总经理。”侍从告诉鼠小姐。  
        “我说的这个贝塔和您年龄差不多。”鼠小姐提醒机场鼠王。  
        “可能是重名。”机场鼠王说,  “贝塔不可能活到现在。这样吧,我让手下通过派出所的老鼠帮你查查。”  
        “不好查吧?这么大的国家。”鼠小姐觉得在这么大的国家找一只老鼠比大海捞针还难。  
        “不难,派出所的老鼠向人类学了一手户籍管理的本领,好查。”鼠王说。  
        鼠王吩咐侍从立即随他去见国家鼠王元首,同时委派部下陪同鼠小姐去打听贝塔的下落。  
        “找到你的朋友,让他改个名,不要叫贝塔,这名字不是他可以随便叫的。”机场鼠王和鼠小姐话别时说。  
        “这有什么关系?您的祖先的朋友叫过的名字,别人就永远不能叫了?”来自德国的鼠小姐没听说过这种道理,“在我们那儿,国家头儿叫什么名字,新出生的孩子就有不少也叫什么。我舅舅在美国,已经有了老鼠绿卡,那儿的好多这两年出生的老鼠都起名叫克林顿。”  
        “你们那儿起同样的名字是尊重。我们这儿起同样的名字是冒犯。你既然到了我们这儿,还是入乡随俗吧。”机场鼠王说。  
        “我肚子里有贝塔的孩子。孩子生出来,我还让他叫贝塔。”鼠小姐宣布。  
        “这个贝塔去了贵国?”机场鼠王一听说外国鼠小姐肚子里有敝国老鼠播的种,很是吃惊。  
        “对呀,贝塔到过我们家。”鼠小姐说。  
        “他怎么去的?”机场鼠王还承担着该国鼠家族的海关出入境任务。          第244集  
        飞碟引起机场鼠王的兴趣;  
        侍从和鼠小姐边走边聊;  
        电话局的人工记次电脑;  
        假酒差点儿喝瞎了市鼠王的眼睛    
        “贝塔是坐飞机去的德国吗?”机场鼠王对本国鼠偷渡行为深恶痛绝。  
        “他自己有飞行器,是飞碟。”鼠小姐用手比划飞碟的样子。  
        “飞碟?”机场鼠王吃惊,“你是说,我们国家有一只老鼠拥有飞碟?”  
        “没错。”鼠小姐证实。  
        “那飞碟多大?”机场鼠王问。  
        “就这么大。”鼠小姐在地上画了一个圆圈儿,“看上去非常先进。”  
        “贝塔自己制造的?”机场鼠王问。        
        “不清楚。不过贝塔的朋友中有人类,也许是那个人制造的。”鼠小姐说。  
        “老鼠和人类交朋友?”机场鼠王吓了一跳。  
        “我们已将那个人变小了。”鼠小姐说。  
        “我这就去见国家鼠王元首。我派我的侍从陪你去找贝塔。”机场鼠王现在对贝塔的兴趣比对微缩人类还大,如果真像鼠小姐说的那样,老鼠的同胞中已经有一位成员掌握了先进的飞碟,那老鼠的出头之日就快到了。  
        机场鼠王去见国家元首鼠王。侍从陪鼠小姐去找贝塔。机场鼠王暗地里叮嘱侍从一定要想方设法找到贝塔。  
        “趁天黑,咱们现在就走吧?”侍从看出鼠王对贝塔十分感兴趣,他要抓住这个立功的机会。  
        “行。”鼠小姐没意见。  
        侍从和鼠小姐出发了。  
        “咱们去哪儿?”鼠小姐边走边聊。  
        “派出所。先让咱们住在派出所的同胞查查这个贝塔在哪儿。”侍从说。  
        “但愿很快能找到贝塔。”鼠小姐说。  
        “你拿来的那药真的能把人类变小?”侍从还是有点儿不信。  
        “真的。我们已经把好多人变小了。其中还有我们国家最有名的侦探。”鼠小姐说,“他叫什么来着?对了叫安东尼。”  
        “干吗先把侦探变小?”侍从不解,“应该先把鼠药厂的厂长变小。”  
        “侦探是专门抓人类中的坏蛋的,人类中坏蛋越多,对咱们老鼠越有利,如果那侦探把人类中的坏蛋都抓完了,对咱们不好。”鼠小姐告诉侍从。  
        “这倒是。”侍从点点头,“不过,我觉得人类中的坏蛋抓不完。依我看,人类差不多都是坏蛋,特虚伪。他们老说老鼠坏,其实,人类比老鼠坏多了。他们干的那些事杀了咱们也想不出来。你听说了吗?有个世界滑冰冠军,为了不让竞争对手超过她,于脆雇人打断了对手的腿,这就是人类。”  
        “这算什么。我们住的那栋别墅每个月要交电话费,那房子根本没人住,也就是说,没有使用电话,你猜怎么着?上个月电话费高达四千五百马克。看守别墅的男佣急了,问电话局是怎么回事,电话局说你先交费再说,你不交费就停你的电话。我们觉得挺逗,怎么没人打电话电话费反而特高呢?正好我们院长的小姨子就住电话局,一打听才知道,原来电话局每个月的电话费是有指标的,完不成指标,就扣员工的奖金。那个月一结算,得,电话费没达到指标,这难不倒电话局的员工,他们就手工操纵计次电脑,胡乱往人家的电话费单上增加阿拉伯数字。院长的小姨子还说,每个月电话费都这么干,一般是小不溜的,每个用户增加个三五元的,显不出来,谁还打一次电话往墙上写正字计数呀!可能上个月某位员工一不留神给我们住的那家的电话费后边多打了两个零。你说人类损不损?”  
        “最没道德的,就是人类。上次我们鼠王从飞机货舱里弄到了一瓶酒,往国外运的酒,你想能差吗?”  
        “那不一定吧?我们那儿的人好酒留下自己喝,差酒才往外国运给外国人喝呢!”  
        “我们这儿的人不,好酒先紧着外国人,差酒自己喝。”  
        “品德高尚。”  
        “我们鼠王舍不得喝这瓶酒,就把它进贡给了市鼠王。”  
        “你们鼠王不爱喝酒?”  
        “特爱喝。这年头,当官的有几个不爱喝酒的?”  
        “那他干吗不留着自己喝?”  
        “讨好呗。其实我明白,给头送东西,是为了从头儿那儿拿回更多的东西。表面上看下属是巴结上司,其实下属是银行,上司是借贷的债务人。送礼好比贷款,贷你一万元,等你还的时候,连本带利,就是两万元了。”侍从说。  
        “市鼠王喝了那酒特高兴吧?”鼠小姐想起了贝塔酒后的风采。        
        “别提了,那是一瓶假酒!差点儿把市鼠王眼睛给喝瞎了。你说这人类坏不坏,这不是残害他们自己的同胞吗?他们造假酒的时候肯定没想到毒老鼠。人类造的假东西多了,你没听说好多女士花大钱买化妆品往自己脸上抹吧?好家伙,抹完了不但没有美丽反而长了一脸的癌细胞。”侍从滔滔不绝。  
        “市鼠王喝了假酒,该怪罪你们鼠王了吧?”鼠小姐不禁为机场鼠王担心。  
        “要是一般的处级鼠王,市鼠王早就宰了他了。可你别忘了,我们鼠王和国家鼠王沾亲呀!”  
        “我明白了。我们住的那栋别墅旁是高速公路,那路上隔三差五就出车祸。有一回,三辆车追了尾,头尾相撞。警察来了,先把后边那辆车的司机揪出来了,一问,那人是区议员!警察二话不说,就盘揪中间那辆车的司机,没想到那人是市议员。警察只好找最前边那辆车的碴儿,你说追尾有怨前边车的吗?也赶上碰巧了,前边那车坐着的是国会议员!得,警察只好把区议员带走了。”  
        “你说人类花那么多钱,投人那么多精力研制屠杀人类自己的武器,他们怎么这么傻?每年花在军费上的开支海了去了,如果都用来办学校,人类哪儿会像现在这样子。”  
        “赶快把他们都变小,地球上的各种生命公平竞争。人类再这么发展下去,地球就让他们给毁了。”鼠小姐憧憬人类变小后的地球未来。  
        “前边就是派出所了。”侍从告诉鼠小姐。          第245集  
        老鼠户口登记簿上有三个贝塔;  
        高老鼠使用武力驱逐贝塔;  
        一轮红日喷薄欲出;  
        地球上又多了一个妈妈    
        居住在派出所的老鼠受环境的熏陶,一个个看鼠小姐时就像审视嫌疑犯。  
        侍从和派出所的老鼠很熟,他向同胞们说明来意并将鼠小姐介绍给他们。  
        “贝塔?好像听说过这个名字,我给你查查。”一只高大的老鼠翻登记簿。  
        “他们根据人类派出所的户口登记簿,相应建立了老鼠登记簿,就是住在哪个人类家庭的老鼠,就和哪个人类家庭一个登记簿。”侍从告诉鼠小姐。  
        “我只能给你查我们管片儿的。”高老鼠一边翻登记簿一边说。        
        鼠小姐感激地看着高老鼠。  
        “我们管片儿有三个叫贝塔的。”高老鼠抬头将查询结果告诉鼠小姐。  
        “三个!必有一个是你要找的。”侍从兴奋地转身对鼠小姐说。  
        鼠小姐激动得一时说不出话来。  
        “是您自己拿着地址去找呢,还是我把他们都叫来由您认?”高老鼠问鼠小姐。  
        “还是麻烦您给叫来吧,这样节省时间。”侍从对高老鼠说。  
        “你们在这儿等着,我们分头去叫。”高老鼠对鼠小姐和侍从说。  
        鼠小姐点点头,坐下来.她感到有点儿累,还觉得肚子里在动。  
        “到你们这儿就跟到家似的,你们对国外来的老鼠真不错。”鼠小姐深有体会地说。  
        “受这儿的人类影响,他们对外国人也特好,特宠着,什么都优先老外。”侍从说。  
        “那为什么?”鼠小姐觉得自己国家的老鼠就不这样。  
        “还不是认为外国生活好呗,这是一种羡慕。羡慕一般都和低三下四同时出现。”侍从说,“同样是生命,有的生下来就在福窝里,有的生下来就吃糠咽菜,全是运气。”        
        “这倒是,同样是墨水,有的被大作家用来写了传世之作,有的被死囚在判决书上签了名,唉,命运这东西,谁也无法改变。”鼠小姐摸着肚子说。  
        怀孕的女性都特爱浮想联翩,都特有想像力,因为她的身上多了一个大脑。  
        “来了一个贝塔,你看看是不是。”高老鼠领来一只老鼠,他对鼠小姐说。  
        鼠小姐站起来往高老鼠身后看,只见那老鼠长得贼眉鼠眼,不是贝塔。  
        “你终于来了!”贼眉鼠眼老鼠热情地和鼠小姐打招呼,听口气他俩熟极了。  
        鼠小姐一愣,她从没见过他。  
        原来,这只叫贝塔的老鼠听说有只德国来的鼠小姐觅夫,乐坏了,他当即承认自己是鼠小姐的先生,跟着派出所的老鼠来认妻。  
        “你眼睛怎么了?看不见东西了?”贼眉鼠眼老鼠装模作样地用手在鼠小姐眼前晃。  
        “我不认识你。”鼠小姐说。  
        “不是他?”高老鼠问鼠小姐。  
        “不是。”鼠小姐厌恶地摇头,她有受侮辱的感觉。  
        另外两个叫贝塔的老鼠也来了,都不是鼠小姐要找的贝塔,可他们却异口同声说自己是鼠小姐的丈夫。  
        这三只和贝塔重名的老鼠处境都很艰难,穷困潦倒,可以说在生命的旅程中走投无路,他们做梦也没想到上天赐给他们这样一个机会,他们不能轻易放过。不管高老鼠怎么劝说吓唬,他们就是不走,还互相争风吃醋。  
        鼠小姐还从没跟这么穷的同胞打过交道,她被他们吓坏了,穷的生命觉得全世界都欠他,做梦都想得到没有的东西,得到了生怕失去,会不择手段地保卫从前没有现在有了的东西。  
        “三代富有才能出一个上等人。”鼠小姐想起老鼠科学院院长经常挂在嘴边的这句托尔斯泰名言。  
        “都不是吧?”高老鼠问鼠小姐。  
        “都不是。”鼠小姐说这话时直恶心。  
        “都滚!”高老鼠大吼一声。  
        三只和贝塔重名的老鼠赖着不走,他们没有尊严地胡搅蛮缠。  
        “真正的男性绝不会赖着不走。此处不留爷自有留爷处。赖着不走的都是假男性。”侍从说。  
        高老鼠叫来几位同事,用武力将三位穷得找不到太太的同胞押了出去。  
        “我看他们一辈子也找不到太太。不是因为穷,而是因为他们实际上不是男性。女性最讨厌假男人。”鼠小姐对侍从说。  
        “没错,穷男性一般还都爱嚼舌头,越嚼越穷,越穷越没骨气,抓住机会就赖着不走。”高老鼠驱逐了几位假男鼠后回来说。  
        “你怎么办?”侍从问鼠小姐,“是跟我回去还是自己去找贝塔?”  
        “我自己去找。”鼠小姐说完突然弯下腰,脸上还有大汗珠出现。  
        “你怎么了?”侍从问。  
        “肚子疼。”鼠小姐倒在地上。  
        “看样子要生孩子。”高老鼠当过爸爸,曾目睹过妻子生产。  
        老鼠们一阵忙乱,有的找棉花,有的找热水,忙乱中还有喜悦。在人类的女性抢着去外国生孩子的时候,外国鼠小姐能到这里来生孩子,实在是一件令人高兴的事。  
        鼠小姐躺在棉花上痛苦地呐喊,她在呼唤一个新的生命的降临,从这个意义上说,老鼠妈妈和人类妈妈一样伟大,还有苍蝇妈妈兔子妈妈狗妈妈细菌妈妈,在她们缔造下一代使本种族得以延续的时候,不由得不让人联想到初升的鲜红的太阳。不管她们生的是什么,对这个世界都有用。没用的早就被历史淘汰了。  
        来自德国的鼠小姐在中国的一家派出所的老鼠的关照下,给贝塔生了一个混血儿子。  
        鼠小姐给他起名叫小贝塔,她的国家有人曾叫过小施特劳斯。          第246集  
        国家元首鼠王想长寿;  
        国家元首想给贝塔颁奖;  
        鼠小姐险当人质;  
        水库里的鱼爷爷变成鱼孙子    
        机场鼠王将微缩药的配方和样品交给国家元首鼠王。元首鼠王不信。  
        “有这等事?”国家元首鼠王一边看配方一边向身边的大臣们征询。  
        “微臣倒是在人类的报纸上见到过德国有人变小的新闻。”负责搜集人类信息的大臣说。  
        “德国来的鼠小姐说他们是什么爱因斯坦的后代,脑子特聪明,所以能发明出这种药。”机场鼠王说。  
        “爱因斯坦的后代是老鼠?”国家元首鼠王知道爱因斯坦,但他不信爱因斯坦会生老鼠。        
        “是住在爱因斯坦家与爱因斯坦朝夕相处的老鼠的后代。”机场鼠王更正语误。  
        “拿样品找人试验一下,如果是真的,马上大批量投产,把咱们这儿的人类全变小。”国家元首鼠王做梦都想当真正的国家元首.他清楚自己这个“国家元首”和小孩儿玩过娃娃家没什么两样,假的。  
        有名称没权力是悲剧。有权力没名称是喜剧。前者折寿。后者长寿。  
        “用活人做试验?”奉命拿微缩粒找人做试验的大臣问。  
        “对。他们人类不是每生产一种新药都拿咱们老鼠试验吗?咱们也回敬一次。”国家元首鼠王哈哈大笑。  
        “那异国来的鼠小姐现在何处?”一大臣向机场鼠王发问道。  
        “她去找她的先生了。’机场鼠王说。  
        “先生?她的先生在哪儿?”国家元首鼠王感兴趣但又觉得不解。  
        “她的先生是咱们国家的老鼠,她还怀有身孕。”机场鼠王如实禀报。  
        “她来过咱们这儿?”鼠王对于德国女鼠能和本国男鼠结为伉俪成秦晋之好感到惊讶。  
        “不,是咱们这儿老鼠去的德国。”机场鼠王小心翼翼地说。他担心国家元首鼠王怪罪他没把好出境关。  
        “咱们的老鼠去了德国?还把德国鼠小姐的肚子弄大了?”国家元首鼠王恨不得马上提拔这位同胞,给他颁奖。  
        “他叫贝塔,我让手下陪那鼠小姐去找他了。”机场鼠王继续禀报。  
        “贝塔!”国家元首鼠王吃惊,“和咱们的历史人物同名?”  
        大臣们也都知道老鼠家族历史上有过舒克和贝塔两位赫赫有名的人物。  
        “据鼠小姐说,这位叫贝塔的同胞不是乘坐民航飞机去的德国,她说他驾驶一个飞碟,很是先进。”机场鼠王一点儿一点儿说。  
        “驾驶飞碟?朕怎么没听说过臣民中有拥有飞碟的?”国家元首俯视大臣们。  
        “臣的确没听说过。”大臣们异口同声。  
        “朕要找这个贝塔。”国家元首鼠王说,他对子飞碟特别感兴趣。第一想拥有。第二怕有飞碟的那厮利用飞碟来夺王位,  
        “据那鼠小姐称,贝塔和人类在一起。”机场鼠王又甩出石破天惊的话。  
        “和人类在一起?怎么个在一起法’”国家元首鼠王瞪着眼睛问。  
        “和一个男人一起去的德国,鼠小姐他们把那男人变小了。”机场鼠王说。  
        这时,去拿人做试验的大臣回来了。  
        “启禀元首鼠王,这药果然灵验,被选中做试验的那人一下就变小了!”大臣说。  
        国家元首鼠王闻声喜形于色,立即降旨大批量投产后立即投放改变人类的体积。  
        “这个贝塔和人类在一起,必是鼠家族的叛徒,一定要抓到他,还要没收他的飞碟。”鼠王降旨。  
        “微臣建议拿那鼠小姐当人质和诱铒,那贝塔即为鼠父,必怜其子。”一大臣献阴招儿。  
        “臣以为未必,”另一大臣反驳,如今之时代,连人类中的男性能对妻儿负责的都与日递减,君不见日前那医院鼠王所言人类女性去做人工流产者身边有男性陪伴的愈来愈少,何况咱们老鼠家族。依臣愚见,那贝塔未必拿这异国鼠小姐腹中之子当回事,大凡无意弄大了人家肚子的男性,都巴不得有人将那准妈妈绑架了去。”  
        国家元首鼠王微微点头。  
        “那就派手下跟踪她,一定要抓住贝塔。”国家元首鼠王说,“武卿!”  
        “臣在。”武官出列。  
        “朕命你拿管此事,务必早日缉拿贝塔,还有飞碟。”  
        “臣领旨。”武官说。        
        缉拿贝塔的工作迅速在全国展开。  
        凡是叫贝塔的老鼠都被抓起来审查。  
        居住在人类各制药厂的老鼠们遵照国家元首鼠王的命令日以继夜地生产微缩药,他们边劳动边手舞足蹈边弹冠相庆,他们做梦也没想到能把人类变得和老鼠一样小,他们终于熬到了出头之日,终于可以和人类公平地较量一回了。  
        有的老鼠还提议给猫也服用微缩药,还说如果猫也和老鼠一般大绝对不是老鼠的对手。  
        当国家元首鼠王被告知微缩药已经生产出足够将该国的人类变小两次的数量后,元首鼠王举行了一个隆重的“投放微缩药典礼。”  
        鼠小姐带来的微缩药配方主要是通过饮用水使人类变小。  
        国家元首鼠王决定全国的老鼠同时往人类的水源中投放微缩药。  
        在“投放微缩药典礼”上,国家元首鼠王亲自将一包微缩药粉撤入该国最著名的饮用水水库,参加典礼的文武大臣们高呼万岁。  
        随后成千上万的老鼠往水库里投药,尽管是在夜里,尽管老鼠的视力不佳,可还是有很多老鼠目睹了水库里的大鱼由爷爷变孙子的全过程。          第247集  
        马鸣生下来就欠父母的债;  
        爸爸不同意报警;  
        两根方便面就能吃饱;  
        人生只有两件事    
        马鸣上小学,他不明白为什么自己的爸爸妈妈总是和他过不去,他们和他说话时,总是采用训斥的口气,好像儿子从生下来就欠他们的债。  
        马鸣觉得爸爸妈妈怎么看他怎么不顺眼,学习不行,吃饭的样子不行,走路的样子也不行,马鸣每天都生活在巨大的压力之中。  
        人生其实只有两件事,被人关照和关照别人。从出生到成年是被人关照的时代。成年到壮年是关照别人的时代。老年又回到被人关照的时代。一个人从出生到成年,他要做的事,就是被人关照。而马鸣没有这种感觉,他觉得自己每分每秒都在关照   爸爸妈妈,为他们学习,为他们努力。  
        马鸣每天晚上写作业都要写到11点钟,而早晨6点都没过,爸爸妈妈就会叫他起床。马鸣对幸福的理解就是有足够的睡眠。  
        这天早晨,马鸣睡得很香,没人叫他,他是被照在脸上的阳光叫醒的。  
        马鸣翻身看表,吓了一跳,表针显示现在是8点钟!马鸣吓坏了,他用最快的速度穿衣服。  
        爸爸妈妈睡过了?马鸣推开爸爸妈妈卧室的门,床上是空的,但被子没有叠。  
        “哪儿去了?”马呜自言自语,他刚要转身去厨房,好像听到一个微弱的声音在叫他。  
        马鸣屏住呼吸昕,确实有个很小的声音在叫马鸣。像是妈妈。  
        马鸣顺着声音找,声音是从床上发出的。  
        当马鸣看到床沿边儿上只有老鼠那么大的爸爸妈妈时,他吓坏了。  
        “你你…们…是…谁……”马鸣连连后退。语无伦次说不成话。  
        “我是你爸爸!”那小爸爸说。  
        “我是妈妈呀!”那微型妈妈也极力证实自己和马鸣的血缘关系。  
        “不是!”马鸣站在一个安全距离外否认他们的监护人资格。        
        “马鸣,我们真是爸爸妈妈。不知怎么搞的,今天早晨一睁眼,我们就变成了这个样子。我们想去叫你上学,可我们下不去床。”爸爸对儿子说。  
        从声音上判断,除了音量小,这的确是爸爸的声音。马鸣还是头一次听爸爸用这种音量和他说话。他有点儿不适应。他习惯于60分贝以上。  
        马鸣一步步移到床边,他蹲下来看那两个小人,确实是爸爸和妈妈。  
        “这是怎么回事?’马鸣问。  
        “不知道,一醒来就这样,我们还以为你也变小了。”妈妈哭着说。  
        “要去报警吗?”马鸣问。  
        “报警?”爸爸想了想,“别报警,这不会是人为的吧?”  
        “最好别让人知道,人家该当新闻了。”妈妈特在意别人怎么看她。  
        “你们不上班了?”马鸣看看表。  
        “这样子怎么去上班?”爸爸看着面前巨大的儿子说。  
        “你给我们单位打个电话,就说我们病了,今天不去了。”妈妈特怕老板和同事知道她现在的状况。  
        马鸣打完电话,回到床边,问:  
        “你们吃饭吗?”  
        “挺饿的。”爸爸说。        
        “我也饿了。”妈妈说。  
        “我给你们弄点儿吃的。”马鸣去厨房里给爸爸妈妈冲方便面。马鸣的爸爸妈妈是被微缩药弄小的第一批人类成员。马鸣使用皮皮鲁牙膏刷牙,有了免疫力,免遭劫难。  
        马鸣将爸爸妈妈从床上拿到餐桌上,用爸爸的酒杯当碗,给他俩每人盛了几根面条。  
        妈妈一边吃一边哭,爸爸也是唉声叹气。他们最难过的,是不知道今后怎么管教儿子。  
        马鸣开始时的惊恐已经烟消云散,他渐渐觉得这样的父子母子关系挺开心。  
        “我……还是爸爸吗?”爸爸吃了两根面条就饱了,他坐在餐桌上问儿子。他担心儿子利用体积上的优势反过来给他当监护人。  
        “当然是,不管我比你大多少,都管你叫爸爸。”马鸣不是那种利用体积上的优势称王称霸的人。  
        “谢谢!谢谢!”爸爸见保住了父位,松了一口气,头一次对儿子说谢谢。  
        “我呢?”妈妈忐忑不安地看着巨人儿子,她骂他骂得最多。  
        “当然永远是我妈妈,就是你变成蚂蚁那么小,也是我妈妈。”马鸣向娘表忠心。  
        “你该上学去了。”爸爸开始行使父亲的权力了。  
        “对,赶快去上学,好好听讲,不要走神儿。”妈妈的声音虽然小,但仍充满威严。  
        “我讨厌上学!”马鸣突然爆发了。他原来不敢说。现在敢说了,不管怎么说,他毕竟在体积上处于绝对优势地位。在这个世界上,没有实力就没有发言权。  
        “这叫上学吗?这叫监狱!这叫折磨人!这叫摧残!在家你们是皇帝,在学校老师是皇帝。报上还有人管我们叫小皇帝,我们算什么皇帝?历史上有这么惨的皇帝吗?有想干什么不能干什么不想干什么偏得干什么的皇帝吗?你们知道小是什么滋味吗?年龄小辈份小身材小,你小就得听比你大的人的话。现在你们尝到了小的滋味儿了吧?其实越小越应该受尊重,越小越应该受关照,我比你们小的时候,你们给我做饭,现在你们比我小了,我就给你们做饭。你们从前弄反了,谁小欺压谁,谁小拿谁出气……”马鸣爆发了。  
        爸爸和妈妈面面相觑,现在他们已经失去了反驳的实力,如果儿子不管他们,他们将饿死在这餐桌上——他们不可能从这摩天大厦上爬下去。  
        爱因斯坦家的老鼠帮了马鸣的忙,马鸣在家里和爸爸妈妈平等了。          第248集  
        B女士状告D君;  
        天花板不能当地板;  
        八国联军使B女士产生幻觉;  
        D君自称是正当防卫    
        去法院打官司,无非两种情况,好人告坏人和坏人告好人。D君成为被告就属于后一种。  
        在一个阳光明媚的上午,D君接到了邮递员送来的一件挂号信函。信封上落款赫然属名法院。D君不记得自己在法院有朋友,他打开信封,是一张传票和起诉书副本。  
        D君成为了被告,告他的是B女士,理由是D君和B女士住楼上楼下,但D君和B女士在楼道相遇时D君从不看B女士,更谈不上打招呼了,久而久之,使B女士感到十分自卑,精神恍惚,受到了莫大的刺激,致使健康状态每况愈下,食欲锐减,彻夜失眠,体内一些物质该来的时候不来不该来的时候猛来且忽明忽暗忽多忽少。B女士于是以D君干扰其正常生活摧残其健康实为伤害罪将一纸状书递到法院。此外还附加抢占房屋罪。  
        于是,D君一生中首次被推上被告席。  
        当D君拿到法院的传票时,他的感觉不是看起诉书,而是看一篇童话,他认定即使最伟大的天皇级童话大师都不敢这么写童话。D君的确住在B女士的头顶上,但他一直没注意过B女士的存在,直到看见起诉书上的签名,D君才晓得地球上还有一个叫B女士的人类成员。  
        D君不喜交往,除直系血亲和工作关系外,鲜与任何人来往。20岁至30岁期间,他很是忙过一段浪漫,恨不得天天沾花,日日惹草。而立之年一过,即感觉倒了胃口,从此拨乱反正守身如玉坐怀不乱。D君手持起诉书极力回忆楼下B女士之相貌。然一无所获。后经家人指点迷津,从窗口望出,终俟得那B女士身影,不看则已,看后D君一周未食肉腥,误以为自己患了甲肝。即使通过时间隧道倒转时光lO年,以园丁为职业的D君也决不会将B女士纳人自己的培植摘采计划。  
        D君在法院指定的日期前往答辩。他说人类中有那么多异性他不可能人人都理都打招呼都多看一两眼总会有照顾不到的地方况且自己现在也不想照顾也没这个义务没这份精力,D君认为自己够不上伤害罪。  
        法官说我没让你去理会全世界的异性但你住在原告头上表面上看美其名曰楼房事实上是在人家头上拉屎撒尿已经不公平了你见到人家时应该有歉意应该有表示应该多看人家几眼特别是B女士这种没有容貌没有事业没有美满家庭的三无女士你更应该关照,你可以不理三有女士不理二有女士不理一有女士但你绝对不能不理三无女士如果是人家住在你头上拉屎撒尿还情有可原可偏偏是你在人家头上拉屎撒尿你更责无旁贷更应该理人家了。法官还说B女士不光告你伤害罪还告你侵犯房屋使用权你的房子盖在B女士的房子上边你竟然用人家的天花板当你家的地板这不是强盗行径是什么不是抢占民房是什么?  
        D君哑口无言。  
        法官向D君传达B女士的诉讼要求一是要求被告D君立即停止侵害立即和原告笑脸相迎最好成为密友天天形影不离二是被告不得继续使用原告的天花板当地板必须立即停止侵犯原告房屋产权的行为在尚未搬离或新造地板之前被告不得在原告进餐的时间大小便。  
        D君申辩说让我不以B女士的天花板为地板我去哪儿住干脆和她共踩一块地板共顶一块天花板得了我看她就是这个意图可那也太癞蛤蟆想吃天鹅肉了我宁可出家当科级和尚也不会和B女士同房法院真要是这么判决我就天天去义务献血一直献到弹尽粮绝慷慨就义为止。  
        法院经过长达十儿个月的调解无效,终于决定今天开庭宣判。  
        D君坐在被告席上。B女士坐在原告席上。  
        最让B女士忍无可忍的是,D君居然敢在法庭上仍然不看她一眼,仍然犯伤害罪。  
        由于这是历史上第一宗不理伤害案和天花板侵权案,吸引了大批的记者和民众,致使法庭内外被挤得水泄不通,那些住在一层(不含一层,有半地下住户的除外)以上的居民无不关注这一案件的判决,他们知道只要判D君有罪他们也就完了也就兔死狐悲了也就要步D君的后尘轮流坐庄享受被告席了。  
        B女士一生最大的心愿就是出名就是被人关注,当她看到自己的理想无法实现后,就采用了告状这一最能引人注目的方法来实现自己的宿愿。现在,她成功了,她终于面对摄像机面对录音机面对照相机了,面且这不是一般的摄像机不是一般的录音机不是一般的照相机,是电视台的摄像机是广播电台的录音机是报社的照相机,终于可以有成千上万的人看她了。  
        “我宣布,现在开庭。”法官洪亮的声音响彻法庭,喧闹的法庭顿时安静。  
        “你是原告B女士吗?”法官要验明正身。  
        “我是B。”B女士享受到了成功,她终于被人关注了,她不再是无名鼠辈了。  
        “你是被告D君吗?”法官问被告。  
        “是。”D君半辈子出人头地,近几年的最大愿望是当无名鼠辈。  
        “现在请原告陈述理由。”法官让B女士先说。  
        B女士口若悬河滔滔不绝声泪俱下地控诉被告的一系犯罪事实,说她受到伤害的程度时还出示了医院的体检证明还说让人踩在脚下的日子真是度日如年忍辱负重和被八国联军蹂躏强暴没什么两样好像她上辈子曾亲身经历过那场浩劫一样。B女士要求法庭为她主持正义。  
        D君开始答辩他说他确实没有看过B女士一眼但他认为这够不上伤害罪,相反,D君认为如果他看B女士,B女士就构成了对他的伤害罪,因为他看她一眼的后果肯定是立马变成太监,他是为了保护自己的人身安全才不看她的,在法律上,这叫正当防卫。D君还说如果住在B女士楼上就叫侵占房屋那纯属强词夺理这世界是立体的三维的总得有人在上边有人在下边如果都想在上边那成什么了当然如果B女士特想在上边我也可以满足她不就是换换吗?          第249集  
        法律允许的吵架;  
        摄像机取景框的重大发现;  
        D君将B女士攥在手心里;  
        某老师藏在扫帚下边    
       法官宣布法庭辩论开始。  
        B女士和D君均未请律师,双方开始短兵相接唇枪舌剑互相攻讦。  
        只见那B女士神气活现精神亢奋将被告D君说得体无完肤整个一个希特勒东条英机B女士将自己美化成雅典娜女神观音菩萨嫦娥小白菜…… 
        D君只得用反唇相讥来驳斥B女士使用一切可以使用的词汇来向法官和听众证明B女士是一个患有癔想症的疯子这种女人恨不得全世界的男人都拜倒在她的石榴裙下遗憾的是事与愿违恰恰没有任何一个上档次的异性理睬她于是她就愤怒就自卑就疯狂就吃不好睡不好这种女人每天起码要撕100张以上的女明星画报彩照来发泄自己的心头之恨发展到后来连男明星的照片也撕也恨铁不成钢再到后来发现自己芳龄已逾四十还没混出个模样来再没有出头之日了干脆铤而走险涮一回法律玩一把法官这就叫孤注一掷破釜沉舟不成功便成仁……  
        法庭辩论说穿了就是法律允许的不带脏字的谩骂吵架。  
        法官和旁昕席上的观众洗耳恭听,人类成员都有喜欢看吵架的本性,但在大街上看吵架太没档次太失身份于是有地位有气质有身份的都到法庭来看吵架,既享受到那种坐山观虎斗的乐趣又美其名日懂法守法有法律意识何乐而不为。  
        正当原告被告双方辩论进人白热化阶段时,有一摄像记者突然从取景框里发现B女士的身体在缩小,他开始时怀疑自己的眼睛出了毛病,后来索性用摄像机的取景框套住B女士的身体,这一套不要紧,B女士的身体果然在一寸一寸地缩小叫疲软也可以。  
        该记者立即到法官身边,将自己的重大发现向法官大人通报。  
        法官大人眯着眼睛测量B女士的身材,他很快就证实了那位记者所说之话的真理性。  
        很快整个法庭的人除了B女士以外都注意到这个现实D君也停止辩论,呆呆在看着B女士逐步萎缩的(禁止)。 
        B女士终于得到了D君的目光,她以为自己赢了,她看到整个法庭的人都那么聚精会神地向她行注目礼,她还看到电视台的摄像师几乎将摄像机的镜头凑到她的鼻子上摄像,她心花怒放荡气回肠踌躇满志她终于尝到了成功的滋味尝到了被不认识的人注意的滋味她满意他们看她的眼神奥斯卡最佳女主角领奖时接受的眼光也不过如此。  
        当B女士发现自己身体的变化时,她慌了,她死死拽住身后的椅子靠背,想阻止自己的身体变小,但无济于事,她的身体当众一点儿一点儿地萎缩着。  
        “不!不!!”B女士绝望地喊叫。  
        法官从未在开庭时遇到过这样的情况,他一时束手无策。法官虽然精通法律,但他却笃信冥冥之中有超自然的力量主宰正义,他认定是老天在惩罚B女士,否则怎么光她一个人缩小身体而且是在法庭上。  
        “救救我!我撤诉!法官大人,快驳回我,快宣判我败诉!”B女士作贼心虚,她也以为是老天爷不干了,不再让她无理取闹胡作非为了。  
        一句话提醒了法官大人,他站起来宣读判决书。他宣告B女士败诉并承担一切诉讼费用还要向D君赔偿一万元。        
        宣布B女士败诉也不能阻止B女士的身体变小,她是喝了国家元首鼠王投放的自来水里的微缩药变小的。D君与B女士同饮一管水为何安然无恙只因D君使用皮皮鲁牙膏,那D君自幼坚持使用儿童牙膏至不惑之年,他认为只有儿童牙膏既能保护牙齿又不会伤害牙齿,成人牙膏里掺的乱七八糟的东西太多,就好像好多小姐使用儿童护肤霜后青春焕发的道理一样,返老还童的最好方法就是使用儿童牙膏涂儿童护肤霜阅读童话。  
        B女士在众目睽睽之下缩小到老鼠那么大,全过程历时7分53秒,摄像机真实地记录了下来。  
        D君尽管胜诉可一点儿也不喜悦,他看着原告席上老鼠那么大的B女士,心中产生了内疚,他认定是由于自己不理不看不睬B女士而导致今天的悲剧的发生,他发誓今后走在大街上走在楼道里最重要的任务就是对所有异性行注目礼看她们和她们打招呼。  
        法官要求D君将B女士护送回家。D君责无旁贷地应允。他突然觉得欠B女士情。  
        D君将B女士攥在手心里拿回了她家,B女士终于和D君有了身体上的接触,她死也瞑目。  
        电视台当晚就播放了某法院开庭时原告B女士身体当庭变小的录像,市民们无不惊讶。  
        马鸣和爸爸妈妈一起观看了这一新闻,当他们意识到变小并不是自己家的专利时,他们感到有救了。  
        马鸣马上给电视台打了电话。  
        记者们蜂拥而至马鸣家。拍照摄像采访忙得一塌糊涂。  
        电视台的值班电话变成了热线电话,报案说自己的家人或自己变小的电话络绎不绝,原先都不敢说,一看有了伴儿就都说了。  
        有一位老师在课堂上正训斥学生,训着训着他就没有了。同学们到处找最后在扫帚下边找到了变小的老师。还有一位国家元首来访正当他站在台子上听国歌时突然失踪了吓得保镖们拔枪朝天上乱打,后来,那元首在地毯下边继续对国歌立正敬礼。还有一个蒙面大盗抢银行,当他将钱装进大口袋时他突然没有了。银行职员半天仍然双手抱在头上不敢轻举妄动以为大盗会隐身术,后来才发现大盗变成了小盗委身于钱袋之中。还有一对新婚夫妇在新婚之夜新娘在床上突然变小弄得新郎猝不及防……         第250集  
        轻伤不下火线的科学家;  
        皮皮鲁没吃早饭:  
        人的大脑是宇宙;  
        电视台主持人险遭厄运    
        皮皮鲁和朋友们从电视屏幕上得知爱因斯坦家的老鼠已经开始在全球推广微缩人类的计划,数以千百万计的人类成员正在被微缩,而且这个数字每天以数十万的速度递增。  
        人类科学家们绞尽脑汁研究这一异化现象,有几位有科学院院士头衔的名科学家在被微缩后仍然轻伤不下火线地攻关。这些科学家都被一个奇怪的现象困扰了——人类中的孩子很少变小。  
        有位科学家在电视台采访他时大放厥词,说什么这种现象是某大国研制的新式武器,记者问他到底是哪个国家因为世界上所有国家的人都有变小的而且越是大国变小的人数越多没听说过发明了新式武器先对自己国家的人民使用的。该科学家顿时理屈词穷。  
        鲁西西给电视台打了匿名电话,她告诉电视台的人使用皮皮鲁牙膏可以防止人体微缩还说这就是为什么大多数孩子投有变小的原因因为大多数孩子都使用皮皮鲁牙膏刷牙。  
        成千上万的家庭出现了新的家庭关系,爸爸妈妈变小而孩子依然魁梧。爸爸妈妈失去了训斥孩子的实力,他们被孩子攥在手里拿来拿去照顾孩子喂他们饭给他们盖小房子。爸爸妈妈们终于知道了处于劣势时的感觉他们后悔当初利用身体和年龄上的优势不平等地对待亲骨肉。  
        皮皮鲁没日没夜地在鲁西西别墅里发明复原药。朋友们每次进餐时谈论最多的话题就是微缩人类。  
        鲁西西变小后也住进别墅。大家生活在豪华舒适的别墅里十分开心,游泳、下棋、做饭、聊天。最美的是贝塔,整天和歌唱家厮守在一起。  
        这天吃早饭时,皮皮鲁没来。  
        “皮皮鲁呢?怎么投来?”歌唱家看见皮皮鲁的座位空着,问燕妮。  
        “他昨晚一夜没睡,现在还在实验室里,可能是快成功了。”燕妮说。  
        “皮皮鲁也真够辛苦的。”歌唱家给自己斟了一杯咖啡,说。  
        “表面上看,名人们挺快活,有名有利。其实依我看,名人才是伺候全人类的佣人呢。”舒克当过作家。特有体会。  
        “皮皮鲁每天早晨5点钟准时起床,他说真正伟大的思想家都是和太阳一起初升的。”燕妮一边往面包上涂黄油一边说。  
        楼上传来皮皮鲁的歌声。  
        “成功了!复原药成功了!”燕妮扔下面包,撞翻了椅子,朝楼上冲去。  
        大家争先恐后地尾随燕妮往楼上跑。  
        皮皮鲁在实验室里一边唱歌一边收拾桌子,他终于将复原药研制出来了。  
        燕妮拥抱皮皮鲁并给他热烈的吻,皮皮鲁十分享受。事业上一事无成的男人在和异性接吻时绝对体会不到真享受。女性的吻说穿了是一种崇拜现象。男人的标志就是事业,没有事业就不是男人。不是男人就是女人。女人吻女人的感觉可想而知。  
        “拿我试验。”鲁西西一进实验室就说。  
        “应该出去试验。”歌唱家说,“不然的话会把咱们的别墅撑破了。”  
        大家来到鲁西西别墅外边。  
        鲁西西服用了皮皮鲁研制的复原药。  
        大家期待着那个伟大时刻的到来,期待着宣告爱因斯坦家的老鼠改变人类的计划的破产。  
        鲁西西的身体开始变大。  
        “太伟大了。”舒克佩服皮皮鲁的大脑。  
        人类的大脑就是宇宙,每一颗脑细胞就是一颗恒星,它们的能量不可估量。  
        鲁西西的身体恢复了原样,她觉得周围的一切都变小了。  
        “皮皮鲁,你也恢复原样吧?”鲁西西对皮皮鲁说。  
        “警方还在找皮皮鲁,他如果变大了,走出去,大家还不抓他呀!”贝塔提醒大家。  
        “这倒是。”鲁西西点头,  “等风声过了再变回来吧。”  
        “再说燕妮也不能变大,变大了还要办签证吧?外国人住在这儿还要好多手续吧?何况根本查不出你是从哪儿入的境,整个一个偷渡客。”舒克也说。  
        “大了真不方便。”燕妮说。“太受限制。”  
        “反正咱们变大变小的药都有,想怎么着就怎么着。”鲁西西说。  
        “应该马上投产复原药,给那些变小的人服用。”皮皮鲁说。  
        “我拿配方去公司生产。”鲁西西说。  
        “怎么让人家吃呢?”贝塔问。  
        “生产一种皮皮鲁口服液,把复原药掺进去,喝了就能复原。没变小的孩子喝了能长大个。”燕妮出主意。  
        “很好。就这么办。”皮皮鲁拍板了。  
        鲁西西火速赶到舒克贝塔公司,召集开发部人员开会,下达生产任务。公司一切工作都给皮皮鲁牙膏和皮皮鲁口服液让路。  
        鲁西西给电视台打了电话,通知他们皮皮鲁口服液能将微缩的人类复原。电视台不信,一位主持人偷偷将一瓶皮皮鲁口服液绐变小的台长喝了,那台长立刻恢复了原样。全台工作人员差点儿把那主持人掐死,因为台长变小后他们有解放的感觉。随着台长的复原,大家叉回到水深火热之中。  
        皮皮鲁口服液挫败了爱因斯坦家的老鼠的计划,人类避免了一次危机。谁都清楚,如果体积一样大,人类未必是老鼠的对手。老鼠不残害同类。          第251集  
        老鼠歌星被装进口袋;  
        严刑之下图钉供出贝塔;  
        鼠小姐和儿子被利用;  
        危机笼罩皮皮鲁家    
        自从国家元首鼠王下令向人类投放微缩药后,部下每天都向鼠王禀报战绩。几位部下为了让鼠王一睹变小后的人类风采,还绑架了一位名叫B女士的人类成员。  
        当B女士被告知她面前坐着的是全国的鼠王时,她激动得差点儿晕了。B女士从未见过大官。  
        鼠王把全球老鼠家族将人类缩小的汁划告诉B女士,B女士十分兴奋,她表示愿意为鼠家族效劳,投奔鼠家族。  
        国家元首鼠王终于了解人类了,人类和动物的最大区别其实是人类中有叫汉奸或其他什么奸的东西,而动物中没有。  
        鼠王现在梦寐以求的,是贝塔的飞碟。  
        派出去寻找贝塔的部下纷纷回报,所有叫贝塔的老鼠都仔细查过了,都不是鼠小姐要找的那个贝塔。  
        跟踪鼠小姐的老鼠向鼠王禀报,说那外国鼠小姐整日带着儿子满世界找贝塔,精神十分可佳,但至今尚未发现贝塔的踪迹。  
        这天,一名部F向国家元首鼠王禀报:  
        “启禀鼠王,微臣的手下在人类的一家歌厅里发现一只为人类唱歌的老鼠。”  
        “为人类唱歌的老鼠?”国家元首鼠王吃了一惊,“人类听老鼠唱歌?”  
        “那家歌厅自从有了咱们这位同胞唱歌后,生意特火,干脆改名叫老鼠歌厅。”大臣说,“臣记得那来自异国的鼠小姐说过贝塔和人类是朋友,臣推断,那歌厅的老鼠也和人类搅在一起,会不会认识贝塔?”  
        “言之有理,”国家元首鼠王点头,“快去将那在歌厅为人类唱歌的叛徒抓来。”  
        图钉离开皮皮鲁家后,又回到那家歌厅。歌厅老板见与他签约的老鼠歌手毁约后又回来了,甚是高兴。这回歌厅老板学精了,他弄来一只猫,策划那猫袭击图钉,在图钉生死攸关千钧一发之际,老板奇迹般地出现打死了那猫,成为图钉的救命恩人再生父母。图钉的腿瘸了,跑不了了,可图钉还要感激歌厅老板的救命之恩。  
        图钉每天瘸着腿为人类唱歌,他喜欢唱歌,他觉得站在台上看着台下的人群是一种最大的享受。  
        这天夜里图钉唱完歌后睡得正香,梦中他觉得自己被装进了口袋。  
        图钉挣扎,无济于事。  
        当他被从口袋里放出来时,视野中出现的场景已经不是歌厅了。  
        全是同胞。  
        “你叫什么名字?”国家元首鼠王问。  
        “你们干什么?”图钉抗议。  
        “放肆,这是全国的鼠王,还不快回话!”大臣们冲图钉厉声喝道。  
        “叫图钉。”图钉一听是全国鼠王,不敢厉害了。  
        “图钉?”国家元首鼠王说,“你会唱歌?”  
        “是的。”  
        “为什么伺候人类?”  
        “他们懂歌。”  
        “你的意思是咱们老鼠不懂歌?”  
        “……”  
        “你知道犯了什么罪?”  
        “……”        
        “里通外族!为敌人服务!”  
        “……”  
        “朕给你一个戴罪立功的机会,”国家元首鼠王说,“你认识一只叫贝塔的老鼠吗?”  
        “贝塔?”图钉一愣,他不明白国家元首鼠王怎么会提起贝塔。  
        从图钉的神色中,国家元首鼠王看出图钉认识贝塔。鼠王还从图钉脸上看到了飞碟。  
        “告诉朕,那贝塔住在何处?”国家元首鼠王问图钉。  
        “我不认识。”图钉摇头,他预感到国家元首鼠王找贝塔没好事。  
        “你认识。”国家元首鼠王从牙缝儿里挤出这三个字。“你最好说。”  
        图钉眼睛看地板。  
        “来人,让他说。”国家元首鼠王吼道。  
        酷刑开始关照图钉。  
        国家元首鼠王从人类那儿学了不少上刑的方式方法。图钉被打得皮开肉绽。  
        “我说……”图钉实在无法忍受这种痛苦,他供出了皮皮鲁家的地址。  
        “贝塔是不是有一个飞碟?”打手问图钉。  
        “是……”图钉声音微弱。  
        国家元首鼠王在得知贝塔的住址的同时,还知道了人类已经有一种叫皮皮鲁口服液的东西,这种口服液能使微缩的人恢复原大。  
        “这么快?!”国家元首鼠上不得不对人类的智慧刮目相看。  
        “微臣认为,那贝塔的朋友也叫皮皮鲁,这皮皮鲁口服液会不会和贝塔的朋友有关系?”一位大臣站出来说。  
        “很可能是一个人。”国家元首鼠王点头,同意那大臣的想法。  
        “应该把他们一网打尽。制止皮皮鲁口服液流行。夺取飞碟。铲除叛徒。”大臣们一致向鼠王要求。  
        国家兀首鼠王开始和大臣们制定计划。  
        一大臣献计说应该将鼠小姐和儿子作人质,万一贝塔不束手就擒,就以鼠小姐和儿子威胁贝塔。  
        另一大臣担心贝塔和鼠小姐的感情是否深到这个程度,就怕适得其反,那贝塔巴不得让绑匪撕票呢。  
        鼠王决定,先让一奸细陪鼠小姐带着儿子去皮皮鲁家找贝塔,等打探清楚皮皮鲁家的虚实后,鼠王再出击。  
        危机笼罩着皮皮鲁家。          第252集  
        奸细进入皮皮鲁家;  
        舒克看书时发现门口有动静;  
        贝塔与鼠小姐重逢;  
        超级拥抱三明治    
        国家元首鼠王选了一只精干智商高的老鼠担任奸细,和德国鼠小姐一起去皮皮鲁家找贝塔,奸细的任务是查清皮皮鲁家的实力,为鼠王元首夺取飞碟打前站。  
        鼠小姐带着儿子四处找贝塔,她去了很多地方,见过不少叫贝塔的老鼠,可没有一个是她要找的贝塔。  
        “我知道贝塔住在哪儿。”奸细老鼠找到鼠小姐母子,对她说。  
        “叫贝塔的老鼠挺多。”鼠小姐对于这位素不相识的异国同胞提供的信息不抱希望。        
        “这个贝塔有个人类朋友,叫皮皮鲁。”奸细注意鼠小姐的面部表情。  
        “皮皮鲁!”鼠小姐隐隐约约感到在燕妮家的别墅被他们变小的那个人就叫皮皮鲁。  
        “你快带我去!”鼠小姐迫不及待。  
        “走。他们住的地方离这儿不远。不过,皮皮鲁会不会抓住咱们?”奸细问:  
        “不会。皮皮鲁在德国把我们都放了,这人好像爱和老鼠交朋友。”鼠小姐催着奸细快带路。  
        皮皮鲁和燕妮在游泳。歌唱家和贝塔坐在游泳池旁聊天。舒克在自己的房间看书。  
        舒克突然听见大门口有响动。  
        他放下书,从别墅的窗口往大门望去。  
        一只贼头贼脑的老鼠正从门下边往屋里钻。  
        舒克跑去告诉皮皮鲁和贝塔。  
        “来找食物的吧?”贝塔判断。  
        “我去问问他。”舒克说。  
        当奸细正在门厅左顾右盼时,舒克出现在他面前。面对这位身着飞行服的同胞,奸细着实吃了一惊。他以为舒克就是贝塔。  
        “你是贝塔?”奸细先问。  
        “你认识贝塔?”舒克惊讶。  
        “我找贝塔。”  
        “你是谁?”        
        “贝塔的朋友找他,我是带路的。”  
        ‘你怎么知道贝塔住这儿?”  
        “我们找了好多地方,一家一家挨着找呗。”  
        “贝塔的朋友在哪儿?”  
        “在门外等着。”  
        “也是老鼠?”  
        “对。你不是贝塔?”  
        “不是,你出去叫贝塔的朋友进来。”舒克说。  
        鼠小姐和儿子进来了。  
        舒克认出鼠小姐了,他没想到德国的老鼠能跑到这儿来找贝塔。  
        鼠小姐也认出了舒克,她知道自己终于找到夫君了。  
        “叫叔叔。”鼠小姐让儿子叫舒克。她入乡随俗,让儿子见着谁叫谁。  
        贝塔的儿子叫舒克。  
        “你找贝塔干什么?”舒克问鼠小姐。  
        “贝塔在吗?”鼠小姐不回答舒克的问题。  
        “你必须先回答我的问题。”舒克说。  
        “我带儿子来见爸爸。”鼠小姐说。  
        舒克愣了。他仔细看鼠小姐的儿子,的确挺像贝塔。  
        舒克不能不让鼠小姐见贝塔。  
        “他是谁?”舒克指着奸细问鼠小姐。        
        “他帮我找贝塔。”鼠小姐说。  
        “你们跟我来吧。”舒克说。  
        鼠小姐领着儿子还有奸细跟着舒克来到鲁西西别墅外边。奸细的眼睛四处乱转。  
        “你们在这儿等会儿。”舒克走进鲁西西别墅。  
        “贝塔,你出来一下。”舒克在游泳池大厅门口叫贝塔。  
        贝塔和歌唱家一起走过来。  
        舒克的原意是叫贝塔自己过来。  
        “外边有人找你。”舒克对贝塔说。  
        “有人找我?”贝塔看看歌唱家,“咱们不都在这儿吗?是鲁西西找我?”  
        舒克摇摇头。  
        “你自己去看吧。”舒克对贝塔说,“歌唱家,咱俩聊聊。”  
        舒克想阻止歌唱家出去,他怕她受刺激。这种场面毕竟不会令人愉快。  
        歌唱家跟着贝塔走出别墅,她好奇。  
        舒克耸耸肩,去游泳池边向池中的皮皮鲁和燕妮通报。皮皮鲁为贝塔捏了一把汗。  
        燕妮赶紧去换衣服,她边换边准备安慰歌唱家的话。  
        贝塔和歌唱家走出别墅,当贝塔看到爱因斯坦家的鼠小姐时,愣了。        
        “贝塔!”鼠小姐热泪盈眶,她终于找到孩子的爸爸了,她不顾一切地冲上去紧紧抱住贝塔。  
        歌唱家惊讶地看着眼前发生的事。  
        “别,别……”贝塔一边往外推鼠小姐一边回头看歌唱家。  
        鼠小姐突然想起了什么,她转身拉过儿子。  
        “快,叫爸爸,这就是你爸爸呀!”鼠小姐将儿子推到贝塔面前。  
        “爸爸!”儿子叫贝塔。  
        “这?!……”贝塔不知所措。  
        “这是你的儿子!你和我的儿子!”鼠小姐又冲上去抱贝塔,这次把儿子夹在中间,就像三明治。  
        皮皮鲁、燕妮和舒克赶来了。  
        燕妮站在歌唱家旁边,随时准备应付歌唱家的轻生行为。  
        贝塔向皮皮鲁投来求救的目光。  
        皮皮鲁将此事的来龙去脉简要地讲给歌唱家听,燕妮在一旁补充。  
        歌唱家走到贝塔身边。贝塔准备挨耳光,一般电影里都是这样安排的。  
        “祝贺你,贝塔!我爱你!”歌唱家说。  
        大家都愣了,不明白歌唱家是什么意思。          第253集  
        贝塔和歌唱家都算伟人;  
        鼠小姐狼吞虎咽;  
        奸细从卫生间看到飞碟;  
        皮皮鲁和燕妮登上五角飞碟    
        贝塔怔怔地看着爱妻。  
        大家都看出歌唱家的话是真心话。  
        皮皮鲁和舒克知道贝塔和金子结婚了。  
        “谢谢你。”贝塔感动地对歌唱家说。  
        “她是谁?”鼠小姐问贝塔。  
        “我太太。”贝塔说。  
        “你和人类通婚?”鼠小姐呆了。  
        奸细也没想到这个叫贝塔的同胞居然能娶到人类太太。  
        “那我怎么办?”鼠小姐问贝塔。  
        “什么你怎么办?”贝塔不明白。        
        “我生了你的儿子!”鼠小姐眼泪汪汪。  
        “这孩子不是爱的结晶,你当时也不爱我,而是为了工作,你同我的那次不叫结合,叫工作,你说对吗?”贝塔对鼠小姐说。  
        “可是……毕竟我们有了共同的孩子……”鼠小姐说的是心里话,当时她并不爱贝塔,可当她知道肚子里有了贝塔的孩子后,感觉就不一样了。  
        “把孩子留下,我们抚养他。”歌唱家对鼠小姐说。  
        “我舍不得。”鼠小姐反对。  
        “那你就将小贝塔带回去抚养,不管到什么时候,贝塔都是他爸爸。”歌唱家对鼠小姐说。  
        鼠小姐点点头。  
        贝塔看着面前的儿子,心里有一种亲切感,血缘的魔力。同时,他也内疚。  
        “不知有小贝塔时电视台正在播什么节目?”舒克小声逗贝塔。  
        “谢天谢地别播凶杀警匪片。”贝塔说。  
        奸细留意四周,他没看见飞碟。  
        “她们母子怎么回德国?”奸细问贝塔,“鼠小姐来的时候就她自己,回去多了个儿子,行走很不方便。”  
        贝塔看皮皮鲁。  
        皮皮鲁觉得这德国鼠小姐也挺通情达理,就说:        
        “我们可以送她回去。”  
        “怎么送?”奸细看到皮皮鲁上钩了,接着问。他希望见到飞碟。  
        “我们自有办法。”舒克告诉奸细。  
        “你到中国来就为了找贝塔?”皮皮鲁问鼠小姐。  
        “我还有传播微缩人类的药物的任务。”鼠小姐不隐瞒实情。  
        “这儿的变小的人类都是你的杰作?”贝塔问鼠小姐。  
        鼠小姐点头。  
        “你的目的达不到了,我们已经研制出了皮皮鲁口服液。”燕妮告诉鼠小姐。  
        奸细不漏掉任何一句话。  
        “进屋吃点儿东西,休息一下。”歌唱家对鼠小姐说。  
        鼠小姐跟着大家走进鲁西西别墅。贝塔领着儿子。奸细跟在后边。  
        大家在餐桌旁坐好,歌唱家去厨房给鼠小姐和贝塔的儿子准备饭。  
        “我家的房子还那样?”燕妮问来自故土的老乡。  
        “我们把你家的男佣变小了,他现在为我们服务。”鼠小姐告诉燕妮。  
        “你们怎么能这样?”燕妮急了。  
        鼠小姐低下头,她第一次对于把人类变小有点儿负罪感。因为这些人对她很好。  
        “麻烦你带一支皮皮鲁口服液给他,行吗?”燕妮请求鼠小姐。  
        “院长发现了,会处决我的。”  
        鼠小姐犹豫不决,这件事对她来说太大了。  
        “你有办法不让院长发现,”皮皮鲁对鼠小姐说。  
        鼠小姐点头同意了。  
        “厕所在哪儿?”奸细佯装上厕所。  
        “我带你去。”舒克对这位同胞没什么好感,他的眼睛太贼。  
        奸细进了卫生间,把水龙头拧开假装小便,他从窗口往外看。  
        五角飞碟停在鲁西西别墅后边。  
        奸细看到了飞碟。他记下了飞碟的位置。  
        饭后,奸细起身告辞:  
        “我得回家了。”  
        “谢谢你。”鼠小姐对奸细十分感激。  
        “我送你出去。”舒克对奸细说。  
        “再见。”好细向大家告别。  
        “谢谢你。”大家说。  
        舒克将奸细送出皮皮鲁家。  
        鼠小姐最近没吃过一顿正经饭,她和儿子狼吞虎咽。小贝塔吃饭的样子很像贝塔。  
        “吃完饭你们休息一会儿,晚上我开飞碟送你们回国。”皮皮鲁说。  
        “我也去。”燕妮说。  
        “给安东尼带点儿皮皮鲁口服液吧。”歌唱家说,她觉得安东尼这人不坏。  
        燕妮看皮皮鲁。  
        “可以。”皮皮鲁同意。  
        鼠小姐和儿子沐浴后在贝塔和歌唱家的床上睡得贼香,她出国后还没睡过一个好觉。  
        鼠小姐做了一个梦,她梦见地球上的生命都和平相处,谁也不拆谁的台。  
        皮皮鲁和舒克检查五角飞碟。  
        “我还去吗?”舒克问皮皮鲁。  
        “我单独驾驶一次,行吗?”皮皮鲁问舒克。  
        “没问题。”舒克说。  
        夜幕降临了。  
        贝塔叫醒鼠小姐和儿子。  
        皮皮鲁和燕妮送鼠小姐母子回国。  
        五角飞碟的舱门打开了,皮皮鲁和燕妮在飞碟里等候鼠小姐母子登机。  
        鼠小姐要求吻贝塔。贝塔看歌唱家,歌唱家不反对。贝塔被鼠小姐吻之后又吻儿子。          第254集  
        贝塔不和小贝塔告别;  
        五角飞碟在燕妮家着陆;  
        皮皮鲁被侦探猛揍;  
        燕妮痛斥安东尼    
        鼠小姐带着小贝塔登上五角飞碟,贝塔看着儿子眼角有点儿发湿。  
        “你应该和你的德国儿子说几句临别赠言。”舒克提醒贝塔。  
        贝塔什么也没说。  
        五角飞碟的舱门关闭了。  
        皮皮鲁驾驶五角飞碟起飞了。  
        燕妮陪鼠小姐和小贝塔在餐厅里说话。  
        鼠小姐和小贝塔都被五角飞碟的现代化震惊了,鼠小姐清楚自己的家族不是皮皮鲁的对手了。  
        “我爸爸会开飞碟吗?”小贝塔问燕妮。        
        燕妮点点头。  
        “我长大也要开这个飞碟。”小贝塔说。  
        燕妮不知道怎么叫答。  
        皮皮鲁是第一次单独驾驶五角飞碟,他有操纵地球飞行的感觉。  
        五角飞碟平稳地在燕妮家的别墅房顶上着陆。  
        皮皮鲁从驾驶舱来到餐厅。  
        “到了。”皮皮鲁对鼠小姐说。  
        “这么快!”鼠小姐不信。  
        五角飞碟的舱门打开了,鼠小姐往外看,她认出这房子的确是她的家。  
        鼠小姐领着儿子准备离开飞碟。  
        “别忘了给男佣皮皮鲁口服液。”燕妮叮嘱鼠小姐。  
        鼠小姐点点头。  
        皮皮鲁拍拍小贝塔的肩膀。不知道是什么意思。  
        “再见。”皮皮鲁和燕妮同鼠小姐母子告别。  
        鼠小姐带着儿子消失在夜色中。  
        燕妮看着自己家的房子,百感交集。  
        “我知道你们还会来,我在这儿等了你们整整一个月了。”黑暗中传来一个声音。  
        皮皮鲁和燕妮吓了一跳。  
        一个小人强行钻进了五角飞碟。  
        皮皮鲁和燕妮一看,愣了。        
        “安东尼!”燕妮吃惊。  
        “对,正是我。”安东尼冷笑道,“没想到吧?”  
        “你要干什么?”皮皮鲁退到驾驶台前,他伸手按电钮关上了五角飞碟的舱门,他不知道外边还有没有安东尼的同伙。  
        “皮皮鲁,你仗着自己脑子聪明,就这么祸害人类吗?你想把人类都变小?我原来以为你算个男子汉,没想到你是个恩将仇报的家伙,我帮你救出了歌唱家,你却把我变小了!我今天要为人类除害!”安东尼义愤填膺。  
        安东尼认定是皮皮鲁将他变小的,他在胡安娜家亲眼看见过变小的皮皮鲁和燕妮。后来他明白了,歌唱家也一定是被皮皮鲁变小的人类成员。安东尼推论皮皮鲁发明了一种能把人变小的方法,并且他有将全人类变小的阴谋。安东尼被变小后,他注意到世界各地有越来越多的人被皮皮鲁变小了,他要挫败皮皮鲁的阴谋。他知道自己可能不是皮皮鲁的对手,但他要尽自己的力量。。他找到皮皮鲁的惟一办法就是在燕妮家的屋顶上蹲守。  
        安东尼朝皮皮鲁扑过去。  
        皮皮鲁按下了起飞按钮和自动驾驶仪开关。五角飞碟升到空中。  
        安东尼和皮皮鲁扭打起来。  
        燕妮呆了。        
        安东尼会拳术。皮皮鲁脑细胞发达,四肢却并不发达,动起武来,皮皮鲁绝对不是安东尼的对手。  
        皮皮鲁手上重重挨了安东尼一拳。  
        皮皮鲁头上又挨了安东尼一拳。他嘴角流出一缕细细的殷红的血。  
        又是一拳。  
        皮皮鲁捂着肚子蹲下了。  
        燕妮冲上去挡在皮皮鲁和安东尼之问:  
        “你为什么打他?你除了打人还会干什么?你还敢打他的头!他的头是全人类的财富!”燕妮痛斥安东尼。  
        “什么全人类的财富!明明是人类的克星!我这是为人类除害!”安东尼见到挡在中间的燕妮,停止了对皮皮鲁的攻击。  
        “你说的这些话到底是什么意思?”燕妮不明白安东尼说的是什么意思。  
        “别装傻了,谁把你变小的?谁把我变小的?谁把那么多人变小的?还不都足皮皮鲁!”安东尼痛斥道。  
        燕妮明白了,安东尼是把微缩人类的罪恶加在皮皮鲁头上了。  
        “把人类变小不是皮皮鲁干的!皮皮鲁在拯救人类!”燕妮大声喊。  
        “胡说!”安东尼咆哮。        
        “是老鼠干的!爱因斯坦家的老鼠!”燕妮嚷嚷。  
        “谎话!”安东尼用更大的声音回击燕妮,“什么?爱因斯坦家的老鼠?你在哄小孩儿吧?把爱因斯坦都搬出来了。”  
        燕妮把爱因斯坦家的老鼠的计划告诉安东尼,安东尼像听天方夜谭。  
        皮皮鲁站起来,用手指擦嘴角的血液,他对安东尼说:“我用遥感仪让你看看。”  
        皮皮鲁打开电脑屏幕,遥感爱因斯坦家的老鼠科学院。  
        屏幕上出现了如下场面:  
        鼠小姐带着小贝塔回到老鼠科学院。  
        院长看到小贝塔,问:  
        “这是谁?”  
        “我的孩子。”鼠小姐说。  
        “孩子!他爸爸是谁?”院长问。  
        “就是上次你们派我去迷惑的那只中国老鼠。”鼠小姐告诉同胞。  
        因公受孕,同胞们无话可说。  
        “微缩人类的任务完成了吗?”院长问鼠小姐。  
        “完成了。”鼠小姐说。          第255集  
        安东尼在房顶上喝皮皮鲁口服液;  
        皮皮鲁和燕妮在南非进餐;  
        鲁西西别墅里空无一人;  
        元首鼠王让奸细领路    
        安东尼傻眼了,他亲耳听到从一只老鼠嘴里说出了“微缩人类”四个字。  
        皮皮鲁抬手给了安东尼一拳。  
        安东尼没还手。  
        “再打三拳。”燕妮说。  
        皮皮鲁没再打。  
        “我们还专门从中国给你带来一瓶皮皮鲁口服液,你喝了马上就能变回去。你这才叫恩将仇报。”燕妮说安东尼。  
        安东尼无地自容。  
        “把皮皮鲁口服液给他,让他出去吧。”皮皮鲁身上火辣辣地疼,他对燕妮说。  
        燕妮将皮皮鲁口服液交给安东尼。  
        “谢谢。”安东尼悻悻地朝舱门走去。  
        皮皮鲁这才想起,五角飞碟还在空中。他操纵五角飞碟返回燕妮家的别墅。  
        安东尼走了。  
        皮皮鲁驾驶五角飞碟起飞,悬停在空中,他打开遥感仪。  
        皮皮鲁想看看安东尼喝皮皮鲁口服液的情况。  
        电脑荧光屏上出现了小人安东尼站在燕妮家房顶上的画面。安东尼借着月色在看手中的皮皮鲁口服液。  
        “他还不放心,怕喝了变得更小。”皮皮鲁对身边的燕妮说。  
        “变小对他肯定是件特痛苦的事。”燕妮无法想像一个这么小的人如何破案。  
        安东尼终于喝了手中的皮皮鲁口服液。  
        他闭上眼腈,站在房顶上,等待变化。  
        安东尼复原了,他睁开眼睛,兴奋地在房顶上跺脚。  
        “别踩塌了你的房子。”皮皮鲁对燕妮说。  
        “你刚才应该再打他几拳。”燕妮对安东尼打皮皮鲁耿耿于怀。  
        “我是靠智力当冠军,不是靠体力当冠军。”皮皮鲁讨厌打。  
        “你看他在干什么?”燕妮指着屏幕对皮皮鲁说。  
        安东尼从房顶上下来,他走到燕妮别墅的门口,找了根铁丝撬门上的锁。  
        门很快打开了,安东尼进屋后弯腰一个房间一个房间在墙角找什么。  
        “他干什么?”皮皮鲁盯着荧光屏。  
        “他在找老鼠科学院。”燕妮判断。  
        皮皮鲁同意燕妮的推测。  
        “他要摧毁老鼠科学院。”燕妮从安东尼的面部表情得出的结论。  
        “咱们就别干涉人家的内政了。”皮皮鲁说。  
        “那小贝塔的安全呢?”燕妮想起贝塔的儿子。  
        “我看安东尼未必是老鼠科学院的对手,他的拳术派不上用场。”皮皮鲁说。  
        燕妮点点头。  
        皮皮鲁驾驶五角飞碟离开燕妮家上空。  
        “我有点儿饿,咱们在哪儿吃饭?”皮皮鲁问燕妮。  
        “到南非吃吧。”燕妮说,“我去给你做饭。”  
        皮皮鲁驾驶五角飞碟来到南非上空,他和燕妮在五角飞碟的餐厅里用餐。  
        “我喜欢旅行。”燕妮边吃边说。  
        “我不喜欢旅行。”皮皮鲁说,“不过我可以陪你在几天之内转遍世界上每一个国家。”  
        “乘坐五角飞碟转不能算旅行。”燕妮说。  
        皮皮鲁和燕妮在南非上空一边吃一边聊。  
        用餐后,皮皮鲁驾驶五角飞碟在南非首都上空超低空飞行了一圈儿,让燕妮饱览一番南非首都夜景。  
        “回家吧?”皮皮鲁问燕妮。  
        “回家。”燕妮同意。  
        五角飞碟返航。  
        家里的景象令皮皮鲁和燕妮大吃一惊,鲁西西别墅前像刚打过世界大战,一片零乱。  
        五角飞碟在鲁西西别墅前着陆。  
        皮皮鲁对燕妮说:  
        “你呆在飞碟上别下来,这是舱门开关,我去看看。”  
        皮皮鲁离开五角飞碟跑进鲁西西别墅。  
        别墅里到处是打斗过的痕迹。皮皮鲁在一层里没有看见舒克。  
        “舒克!”皮皮鲁叫。  
        没人答应。  
        皮皮鲁往二层跑,贝塔和歌唱家也不在。三层也没有。房间被翻得乱七八糟。  
        皮皮鲁跑回五角飞碟,往公司给鲁西西打电话。  
        “什么?都不见了?”鲁西西接到皮皮鲁的电话吃了一惊。  
        “你没回来过?”皮皮鲁问。  
        “我一直在公司盯着生产皮皮鲁牙膏和皮皮鲁口服液,需要量太大了,我没回家。”鲁西西说。  
        “我用五角飞碟遥感。”皮皮鲁说。  
        真相大白了。  
        国家元首鼠王昕了从皮皮鲁家回来的奸细汇报后,决定趁夜晚立即出兵抓获贝塔及其同伙,缴获飞碟。  
        “你看清楚了,飞碟确实在?”国家元首鼠王问奸细。  
        “千真万确,停在他们住的房子后边。”奸细描述五角飞碟的外观。  
        “你带路,马上给朕把他们连同飞碟都抓来。”国家元首鼠土发令。  
        500只身强力壮的老鼠出发了。  
        奸细带着同胞们来到皮皮鲁家的楼下。奸细留下50只老鼠把住楼梯门口,其余的和他一起上楼。  
        贝塔和歌唱家正在卧室里谈论鼠小姐和小贝塔。舒克躺在床上看书。  
        奸细领着四百多只老鼠从门缝儿下钻进皮皮鲁家。          第256集  
        台灯砸中鼠兵的头;  
        贝塔踢中奸细的要害部位;  
        鼠王处决大臣;  
        贝塔反对鼠王和人类通婚    
        当舒克看见奸细时,鲁西西别墅已经被几百只老鼠围得水泄不通。  
        奸细出现在舒克的卧室里,舒克意识到危机降临,他迅速从床上跳下来,顺手抄起台灯准备自卫。  
        “飞碟呢?”奸细问。他一进皮皮鲁家就先奔飞碟去,发现飞碟不在了。  
        “你想干什么?”舒克以为就奸细自己,他不怵。  
        “我奉全国鼠王的命令,来抓你们和飞碟!”奸细神气活现地说。  
        舒克抡起台灯朝奸细砸过去,奸细一闪,台灯砸在他身后的一只老鼠头上。        
        “抓住他!”奸细一挥手。  
        冲进卧室里几十只老鼠。舒克奋力反抗,终因寡不敌众,被五花大绑起来。  
        “贝塔,快跑!”舒克大喊。  
        “好像是舒克叫你。”歌唱家对贝塔说。  
        贝塔刚坐起来,卧室的门就被踢开了,几十只老鼠冲进来,将贝塔按在床上捆起来。  
        “混蛋,你们吃了豹子胆!”贝塔大骂。  
        一只老鼠给了贝塔一拳:  
        “你才吃了豹子胆,我们是国家元首鼠王派来抓你们的,你还敢骂!”  
        老鼠们翻遍r别墅投找到皮皮鲁。  
        “准是皮皮鲁把飞碟开走了。”奸细推断。  
        贝塔和歌唱家被押下楼,见到了舒克。  
        “和鼠小姐一起来的原来是奸细,咱们中计了。”舒克对贝塔说。  
        “我看鼠小姐可能不知道奸细的身份。”歌唱家为鼠小姐的清白辩护。  
        “他们好像是冲着五角飞碟来的。”舒克小声对贝塔和歌唱家说。  
        “幸亏皮皮鲁开走了,真玄。”贝塔后怕。  
        “最好皮皮鲁现在回来。”歌唱家祈祷。  
        奸细杀气腾腾地走过来。  
        “快说,皮皮鲁是不是把飞碟开走了?”奸细问舒克和贝塔。  
        “当然,要不怎么叫神机妙算呢?”贝塔冲奸细一笑,“我还得感谢你把儿子给我送回来了。”  
        啪。奸细打了贝塔一个耳光。  
        贝塔抬腿冲奸细的下(禁止)狠狠踢了一脚。奸细捂着肚子蹲下了。  
        上来几只老鼠将贝塔按在地上猛打。  
        歌唱家冲上去用身体护住贝塔,几只老鼠乘机在歌唱家身上占便宜。舒克猛踢那几只老鼠的下(禁止)。冲上来更多的老鼠打舒克。  
        “别打死了!”奸细强忍着疼站起来,他的感觉是失去了身上最宝贵的东西,“押回去!”  
        舒克、贝塔和歌唱家被老鼠们押进鼠王的王宫。  
        国家元首鼠王看见歌唱家眼睛一亮。  
        “谁是贝塔?”国家元首鼠王问。  
        “我是。”贝塔上前一步,“你是谁?”  
        “放肆!”一位大臣喝道。  
        贝塔扭头瞪了那人臣一眼。  
        “你是谁?”鼠王问舒克。  
        舒克不理他。  
        “他叫舒克。”奸细禀报。  
        “她呢?”鼠王指指歌唱家。  
        “她是我太太。”贝塔说。  
        鼠王盯着歌唱家看,他想让歌唱家给他当王后,鼠王还是头一次看见这么有魅力的变小了的人类女性。  
        “飞碟呢?”国家元首鼠王问奸细。  
        “大概是皮皮鲁把飞碟开走了,一会儿我们再去他家。”奸细说。  
        “你们是鼠家族的叛徒,你们居然和人类勾结在一起!”鼠王对舒克和贝塔说。  
        “和人类交朋友有什么不好?”贝塔反问国家元首鼠王。  
        “人类是我们老鼠的敌人!”国家元首鼠王咆哮道。  
        “世世代代这么仇恨下去,活着真没意思。”舒克说。  
        “宇宙就是由仇恨组成的。”国家元首鼠王冷笑着说,“所有生命都互相来往,宇宙的末日就到了。”  
        大臣们为鼠王的精彩语言鼓掌。  
        国家元首鼠王示意一位大臣到他身边来。鼠王和那大臣耳语着什么。  
        大臣听着听着脸色变了:  
        “启察鼠王,万万使不得。”  
        “为什么?”鼠王不高兴了。  
        “这会改变咱们种族的血统呀!”大臣在极力劝阻国家元首鼠王。        
        原来国家元首鼠王准备娶歌唱家做王后,他在同大臣商议此事。  
        鼠王没想到大臣反对这门亲事。  
        “朕要娶这位人类小姐为后,你们意下如何?”国家元首鼠王指着歌唱家问大臣们。  
        大臣们都对于鼠王要和人类通婚感到惊讶。  
        有几位大臣站出来反对。  
        “来人,把他们拉出去处决。”国家元首鼠王招呼卫兵。  
        卫兵们蜂拥进来把几位反对鼠王和人类通婚的大臣拉出去处决了。  
        “臣认为鼠王完全可以娶这位人类小姐,这样还可以改良我们的后代。”一位大臣投赞成票。  
        “往深刻了说,这也是咱们老鼠对人类的一种报复,出口气。”另一位大臣从理论角度赞成。  
        活着的大臣们争先恐后向鼠王表态。  
        国家元首鼠王笑了。他认为当大王最重要的是要有武力。武力是统一看法的最佳方法。  
      “我反对!”贝塔急了。  
      “你是做梦!”歌唱家对鼠王说。  
      “你刚才不是还在说老鼠应该和人类交朋友吗?”国家元首鼠王揶揄贝塔。          第257集  
        舒克和贝塔被押入大牢;  
        B女士游说歌唱家;  
        贝塔捶胸顿足;  
        舒克制定越狱方案    
        “你敢!”贝塔威胁国家元首鼠王,“你敢动她一下,五角飞碟会要你的命!”  
        “五角飞碟?”国家元首鼠王知道了飞碟的名称,“朕才不怕什么五角飞碟,就是六角飞碟也拦不住朕娶她。”  
        “无耻!”歌唱家显然开始着急。  
        一大臣凑到国家元首鼠王耳边献计:  
        “臣以为,强扭的瓜不甜,特别是这种事。不如让那B女士劝降,动员她高高兴兴当王后。”  
        国家元首鼠王点头称是。  
        “先把这两个叛徒押人大牢,待朕结婚后再审他们。将王后好生照顾,一小时后朕与她成亲。”国家元首鼠王下旨。  
        卫兵上来押舒克和贝塔。  
        “你敢!混蛋!”贝塔大骂国家元首鼠王,可他无能为力阻止鼠王。  
        舒克盼望五角飞碟出现。  
        卫兵们把舒克和贝塔押人大牢。歌唱家要和贝塔一起进大牢,鼠王当然不依,他让卫兵将歌唱家软禁在鼠王的卧室。  
        歌唱家在鼠王的卧室里咬牙切齿,她在人间受尽了侮辱,现在又轮到老鼠折磨她了。歌唱家觉得生命的过程就是受难,没完没了的苦难组成了生命。  
        门开了,歌唱家本能地站起来准备自卫,她以为是国家元首鼠王来骚扰她了。再大的官,在这种时刻也和流氓一样。  
        进来的是一位人类女士。  
        歌唱家松了口气。  
        “你是?”歌唱家没想到在国家元首鼠王的卧室里能碰到同胞,而且是同性。  
        “我叫B女士,你呢?”B女士受鼠王委托前来对歌唱家策反。其实她心里很不是滋味儿。  
        自从B女士来到鼠王身边后,一直想当王后,过去她看了不少有关这方面的传记,什么武则天啦慈禧啦江青啦撒切尔夫人啦她都能将她们的经历倒背如流,可惜从前没机会接近权势级人物。现在,天赐良机她终于和一位大王级的生命共呼吸10平方米之内的同一氧气,她不能放过这个当王后的机会。  
        B女士想尽一切方法讨好国家元首鼠王包括最不应该使用的手段最丧失人格的计谋她都使用了,无奈国家元首鼠王毕竟是见过异性世面的人物,尽管鼠王还没涉足过人类,但鼠王和B女士接触时间不长就感到B女士还不如女老鼠可爱。所以尽管B女士使尽浑身解数鼠王就是不干。  
        大臣刚才给B女士派任务时许愿,只要B女士能说服歌唱家,鼠王就能满足B女士的一切愿望。  
        “我叫歌唱家。”歌唱家说,“你是怎么在这儿的?”  
        “我也是被抓来的。”女士最大的才能就是撒谎和颠倒黑白。  
        “鼠王也让你当王后了?”歌唱家问。  
        “是的,可我不愿意。”B女士终于过了一次嘴瘾。  
        “这个鼠王太坏了。”歌唱家同情地看着B女士。  
        “不过,通过在这儿呆了一段时间,我倒觉得老鼠不坏。”B女士循序渐进。  
        “当然有好老鼠,我先生就是老鼠。”歌唱家说。  
        “你先生是老鼠?!”B女士还没听说过人和老鼠结婚的。        
        歌唱家点头:  
        “也被鼠王抓来了。”  
        “既然都是老鼠,你于吗不答应鼠王?”B女士认定歌唱家是傻瓜。  
        “老鼠和老鼠可不一样,就好比人和人也不一样。你说对吗?同样是人,你说差别有多大?有的人是金子,有的人是粪便。”歌唱家说。  
        “那倒是。鼠王在老鼠里算是好老鼠,要不他怎么能当上鼠王呢?还不是受到了老鼠们的拥戴?”B女士说。  
        “这可不一定。我看当鼠王的没几只好老鼠。即使原来是好老鼠,当上鼠王也就变坏了。”歌唱家说。  
        “咱们这位鼠王不错,我觉得你不应该放弃这个机会。”B女士有点儿不耐烦了,她真恨自己不争气,没在鼠王那儿受宠。  
        “那你怎么不当王后?”歌唱家开始怀疑B女士的真正来意了。  
        “我没这个福气!”B女士歇斯底里了,她再也不能忍受这种动员别人去夺自己所爱的特殊任务了。  
        “鼠王不要你?”歌唱家更同情B女士了,连老鼠都不要她。  
        “你告诉我,你是怎么吸引男人的?你必须说!你准有诀窍,要不男性怎么一看你就喜欢!在人间时,我为此和男性打过官司,我看有的女人长得还不如我,怎么男性就会喜欢她们?你说!你说!不说我让卫兵杀了你!”B女士索性嚎啕大哭。  
        歌唱家呆呆地看着泪如雨下的B女士。  
        想干的人家不干。不想干的人家想干。要什么没什么,不要什么来什么。打麻将抓牌是这样,人生也是这样。  
        B女士做不通歌唱家的工作。国家元首鼠王闻讯决定强行和歌唱家结婚。  
        大臣们去操办鼠王的婚事。  
        看守舒克和贝塔的狱卒将这一消息告诉给贝塔,贝塔捶胸顿足,发誓要杀了鼠王。  
        舒克安慰贝塔。  
        “咱们应该想办法越狱,去救你老婆。”舒克小声对贝塔说。  
        贝塔化悲痛为力量,点头。  
        舒克和贝塔悄悄制定计划。  
        舒克突然大叫肚子疼。  
        狱卒打开牢门问怎么了。  
        贝塔猛击狱卒头部,狱卒倒下了。          第258集  
        婚礼进行曲奏响;  
        B女士泪往肚子里流;  
        床下的舒克和贝塔;  
        冰天雪地的北极    
        舒克将狱卒拖到门后藏起来。  
        贝塔解下狱卒身上的匕首。  
        “走。”舒克将牢门锁上后对贝塔说。  
        舒克和贝塔蹑手蹑脚擦着墙往外走。监狱门口有两只狱卒把守。  
        “我对付左边那只,右边那只交给你了。”贝塔趴在舒克耳边说。  
        舒克点点头。  
        贝塔和舒克同时朝狱卒扑过去。贝塔手中有匕首,很快就制服了对手。  
        舒克赤手空拳和手持器械的狱卒较量,胳膊被狱卒的刀子划破了。  
        贝塔过来增援舒克,从后边将那狱卒击昏了。舒克缴获了狱卒的匕首。  
        监狱外边传来乐曲声。  
        贝塔一愣。  
        “在举行婚礼。快!”舒克提醒贝塔。  
        贝塔拔腿就跑,舒克紧跟。  
        国家元首鼠王在大臣们的簇拥下举行迎娶人类女性为王后的婚礼,元首鼠王春风得意,大臣们七嘴八舌恭维鼠王。  
        “鼠王婚事一办,就是人类的半个大王啦!”一个极会谄媚的大臣说。  
        “人类都变小了以后,咱们老鼠统统和人类通婚,改变人类的血统,哈哈……”另一个大臣一脸的坏笑。  
        歌唱家被四名鼠兵押着参加婚礼,她的嘴里被塞上了棉花,胳膊捆在身后。  
        B女士站在歌唱家身边。她嫉妒歌唱家,她的眼泪往肚子里流。  
        国家元首鼠王对婚礼并不感兴趣,他急于想进洞房。  
        大臣们明白国家元首鼠王的心思,婚礼迅速转人洞房阶段。  
        歌唱家被押进新房,鼠兵们警告歌唱家必须配合国家元首鼠王新郎.否则将受到严惩。  
        鼠兵们退了出去。  
        歌唱家依然被捆着,她现在才真懵了。  
        “别怕,我们在这儿。”床底下传出贝塔的声音。  
        歌唱家大喜,她蹲下一看,舒克和贝塔都藏在床下。  
        国家元首鼠王推门进来了。  
        “王后,今天是咱们大喜的日子,你为何不说话?”国家元首鼠王还挺酸。  
        歌唱家示意嘴里有东西堵着。  
        国家元首鼠王将歌唱家嘴里的棉花取出来。  
        “我的手还被捆着。”歌唱家转过身体。  
        国家元首鼠王给歌唱家松绑。他还没发现,舒克和贝塔已经站在了他的身后。  
        国家元首鼠王不想等了,他伸出双臂要拥抱歌唱家。  
        贝塔从后边猛踢国家元首鼠王的双腿,国家元首鼠王给歌唱家跪下了。  
        舒克用被子蒙住国家元首鼠王的头,贝塔玩命拳击国家元首鼠王的下半身。  
        国家元首鼠王这辈子在本质上不容易娶王后了。  
        舒克将被子拿开,只见国家元首鼠王呲牙咧嘴,疼得满头是汗。  
        贝塔用胳膊勒住国家元首鼠王的脖子,另一只手握着匕首顶在鼠王喉咙上。  
        “你保护歌唱家,我拿鼠王当人质,咱们走!”贝塔对舒克说。  
        开门,_贝塔勒着国家元首鼠王走出新房。  
        文武大臣和卫兵们傻眼了。  
        “谁上来,我就杀了他!”贝塔大喊。  
        “别…别…轻举…妄动……”国家元首鼠王不让部下动武。  
        卫兵们闪开一条路,让贝塔他们走。  
        “一个觊觎王位的大臣想趁这个机会干掉鼠王,他命令卫兵袭击贝塔。  
        就在舒克和贝塔即将走出王宫时,鼠兵们从后边冲上来同舒克和贝塔开打。  
        贝塔没有杀鼠王,他放开鼠王和鼠兵们搏斗。上百只鼠兵将舒克、贝塔和歌唱家团团围住。  
        “这回真完了。”舒克渐渐力不从心了。  
        “反正那鼠王甭想真娶歌唱家了。”贝塔以此宽心,他也招架不住了。  
        就在鼠兵们要一拥而上生擒舒克和贝塔时,跑在最前边的鼠兵突然一个接一个地摔倒了。  
        “皮皮鲁来了!”贝塔大喊。  
        “五角飞碟!”舒克一蹦老高。  
        歌唱家冲上去和贝塔拥抱。  
        大臣们不知道五角飞碟的厉害,还指挥鼠兵们往上冲。冲过来的鼠兵都栽倒在地上。  
        大臣们傻眼了。  
        舒克和贝塔知道,五角飞碟就在王宫外边。贝塔决定把鼠王流放到北极去。  
        舒克没意见。  
        “有位B女士很爱鼠王,我去问问她是不是陪鼠王去北极。”歌唱家一片好心。  
        B女士坚决不去,她知道鼠王已经失去了王位。B女士要与新鼠王结合。  
        歌唱家知道B女士为什么不讨男性喜欢了。  
        贝塔和舒克押着国家元首鼠王走出王宫,五角飞碟在他们面前着陆。  
        歌唱家和燕妮拥抱。  
        “带他来干什么?”皮皮鲁指着国家元首鼠王问贝塔。  
        “这家伙是全国的鼠王,他想和歌唱家结婚,挺坏,贝塔说把他流放到北极去。”舒克说。  
        “我都知道了,开飞碟送他去。”皮皮鲁说。  
        国家元首鼠王总算见到了飞碟。  
        贝塔驾驶五角飞碟将国家元首鼠土送到了北极。望着无边无垠的冰天雪地,国家元首鼠王全身颤抖,他乞求贝塔不要丢下他。  
        “我刚才在你的王宫就警告你不要打我老婆的主意,你不听,这是你咎由自取。”贝塔铁石心肠。  
        “他自己在这儿是挺可怜,把B女士接上来吧。”舒克建议。          第259集  
        B女士与前鼠王在北极成亲;  
        燕妮对自己的上辈子感兴趣;  
        鲁西西在家恭候朋友;  
        贝塔吃不下饭    
        “可以。”皮皮鲁也觉得前国家元首鼠王自己呆在北极的冰天雪地里太孤独。  
        五角飞碟只用了几秒钟就回到国家元首鼠王的王宫,贝塔和舒克进王宫找B女士。  
        王宫里毫无失去鼠王的悲哀,倒是一片张灯结彩,老鼠们在庆祝新鼠王登基。  
        “这么快?前鼠王离开也就不到1分钟!”舒克对于同胞办事的效率感到惊讶。  
        “所有生命在办这件事时都是高效率的。”贝塔说,“前鼠王的亲友还得倒霉。”  
        舒克看见B女士正向新鼠王献花。        
        大臣们看见了舒克和贝塔,他们一个个面如土色。新鼠王是原来的武臣,看见贝塔也打了个哆嗦。  
        贝塔走到B女士身边,说:  
        “请跟我们走一趟。”  
        “干什么?”B女士一点儿不紧张,只要有异性和她说话,她就满足。  
        “去旅游。”贝塔说。  
        “去哪儿?”B女士问。  
        “北极。”贝塔说。  
        B女士挺欣赏贝塔的幽默,她根本不相信贝塔能带她去北极。  
        B女士一进五角飞碟就相信能去北极了,飞碟里的现代化设施令她眼界大开。  
        当B女士站在北极的冰雪之中得知贝塔他们要将她和前国家元首鼠王留在这里时,她急了。  
        歌唱家想起了B女士劝她和鼠王结婚的场面,她觉得这是报应。  
        五角飞碟起飞了。  
        B女士和前国家元首鼠王留在了北极。  
        前国家元首鼠王叹了口气,说:  
        “做事一定不要做得太绝。”  
        他后悔抢贝塔的太太与自己成亲。  
        B女士想起了D君,还想起了去法院打的那场官司。天一冷,脑子倒好使了。        
        五角飞碟上,皮皮鲁和燕妮给舒克他们讲去德国送小贝塔的经历。  
        “安东尼跟咱们有点儿缘分。”舒克说。  
        “也许上辈子是咱们的朋友。”贝塔说。  
        “啊,对丁,皮皮鲁,你上次去德国时,说能给五角飞碟增加一个测上世下世的功能。”燕妮对自己的上世和来世极感兴趣。  
        “你们信生命有上世下世吗?”皮皮鲁问大家。  
        “我信。”舒克说。  
        “我也觉得有。”贝塔说。  
        “大概有吧。”歌唱家半信半疑。  
        “肯定有。”燕妮最坚定。  
        “我给五角飞碟的遥感仪加一个元件就行了。如果真有上世下世,就能测出来。如果没有,就测不出来。”皮皮鲁说。  
        舒克驾驶五角飞碟返回皮皮鲁家。鲁西西在家里等他们,被老鼠弄乱的房间已经收拾干净。  
        “经历又够写一本小说了吧?”鲁西西问大家。  
        “歌唱家差点儿当上王后。”皮皮鲁告诉鲁西西。  
        “我还把全国的鼠王流放到北极去了。”贝塔说。  
        “贝塔太损了,对付情敌也忒狠了点儿。”舒克说。  
        “我够仁慈的了,你说呢?”贝塔问歌唱家。  
        “还行。”歌唱家赞扬先生。        
        大家分头去自己的房间洗澡。鲁西西使用微缩粒把自己变小后给朋友们做饭。  
        太阳又一次不厌其烦地出现在东方地平线上,又一次千篇一律地照亮东半球。  
        鲁西西为朋友们准备了丰盛的早餐。皮皮鲁的座位空着。  
        “皮皮鲁呢?”舒克问燕妮。  
        “在五角飞碟里改装遥感仪。”燕妮往面包上涂黄油,“他说几分钟就好。”  
        “也不知道我上辈子是什么。”贝塔边喝牛奶边说。  
        “我只关心我的下辈子。”舒克说。  
        “生命真会轮回转世?”鲁西西表示怀疑。  
        “一会儿就知道了。”歌唱家说。  
        五角飞碟停在鲁西西别墅门口,皮皮鲁在飞碟里忙碌着。大家边吃边通过窗口看五角飞碟。  
        皮皮鲁从五角飞碟里出来了。  
        贝塔从座位上一跃而起,他要去测自己的上世。燕妮也惟恐落后,饭都不吃了,跟在贝塔后边往外跑。  
        皮皮鲁和贝塔撞个满怀。  
        “先给我测。”贝塔对皮皮鲁说。  
        “我得先吃饭,都快饿晕了。”皮皮鲁往回推贝塔。        
        “到底有没有上下世?”贝塔迫不及待地问。  
        “我还没试验,等吃完饭就拿你试。”皮皮鲁拉着燕妮的手往餐厅走。  
        贝塔一口饭也吃不下了。  
        皮皮鲁吃了(又鸟)蛋,喝了牛奶,又吃了一块牛肉和面包,每次进行脑力劳动后,皮皮鲁都特能吃。  
        鲁西西看着哥哥吃饭,她想起了小时候有一次皮皮鲁躺在电褥子上给自己的身体充电,充电后的皮皮鲁双手一拍就能发出击伤人的闪电,一个外国奸商绑架了皮皮鲁,皮皮鲁和一个叫罗莎的女孩子共同挫败了外国奸商。  
        “你想什么呢?”燕妮看见鲁西西两眼出神儿,她问鲁西西。  
        “我想起皮皮鲁小时候用电褥子给自己的身体充电,”鲁西西说,“就像上个世纪的事。”  
        “皮皮鲁从小就会标新立异?”燕妮对皮皮鲁小时候的事特感兴趣。一个人爱另一个人时,总爱把被爱者的年龄想得很小。  
        “快吃,我等不及了。”贝塔催促皮皮鲁。皮皮鲁吃饭比较慢。          第260集  
        贝塔身边打过21响礼炮;  
        总统的耳朵和贝塔的耳朵一样;  
        汽车商拐走了妈妈;  
        床头的八字座右铭    
        皮皮鲁吃完饭大家簇拥着他走进五角飞碟。贝塔最积极。  
        “先测我上辈子是什么。”贝塔对皮皮鲁说。  
        “还不知道有没有轮回转世呢!”皮皮鲁开始操纵改装后的遥感仪遥感贝塔的上辈子。如果没有轮回转世,荧光屏上将是一片空白。  
        大家聚精会神地盯着屏幕。  
        屏幕上出现了一些类似于电波干扰那样的波纹.皮皮鲁调整遥感仪的旋钮,同时将贝塔的有关数据输入电脑。  
        “不可能有上辈子和下辈子。”鲁西西盯着荧光屏断言。  
        荧光屏上出现了画面!  
        大家先是一愣,接着便是欢呼。鲁西西瞪大眼睛看,她突然感到人类对生命的认识是极其浅薄的。  
        贝塔上辈子是某国总统,他正陪外国来访的元首检阅本国三军仪仗队。贝塔神气极了,他身后是全副武装身上挂着这穗那带的军人保镖,他还把自己国家的部长们介绍给外国元首,然后又煞有介事地去和随从外国元首来访的高级官员们一一握手。  
        假炮21响。洪亮而不震耳欲聋。有气氛而不具杀伤力。  
        “贝塔上辈子是总统!”歌唱家情不自禁地大喊,她庆幸自己的眼力。  
        “还是个有名的总统呢!”鲁西西上学时历史课总是优秀,她知道这位总统的大名。  
        “真没想到贝塔上辈子是人,还当过总统。”皮皮鲁不得不回头对贝塔刮目相看。  
        贝塔绝对没想到自己上辈子是人类的成员,而且是一国之主,他死盯着屏幕说不出话来。  
        “还真有点儿像,你们看那总统的耳朵,和贝塔的一样。”燕妮从遗传学角度发现了贝塔上辈子和现在的相似之处。  
        大家仔细一看,的确,那总统的耳朵和贝塔现在的耳朵一模一样。        
        “祝贺你,贝塔,上辈子原来是总统。”舒克伸出手和贝塔握手,向他表示祝贺。  
        歌唱家拥抱贝塔。  
        “按说总统下辈子不至于变成老鼠呀!”鲁西西说完自觉失言,忙补充道,“我不是说身分低微,我是说……”  
        “没关系,你说得对,老鼠名声就是不好。我也觉得总统不会下辈子当老鼠。”舒克同意鲁西西的疑问。  
        “咱们从头到尾看贝塔的上辈子吧?”皮皮鲁征求大家的意见。  
        一致拥护。  
        皮皮鲁操纵遥感仪从贝塔上辈子出生那天开始看,大家就像看一部系列电视剧。  
        “等等,先暂停,我去搬椅子,再拿点儿零食。”燕妮说。她喜欢舒舒服服欣赏电视剧。  
        大家一通忙活,各自都找到了最舒适的姿势。五角飞碟里像一个小剧场。  
        贝塔上辈子生平传记片开播。  
        贝塔上辈子的爸爸是某国一家公司的小职员,妈妈是自选商场的收款员。贝塔的爸爸在一次推销产品时,认识了贝塔的妈妈。  
        三来二去,两个人就结婚了。  
        贝塔出生时,他爸爸等候在产房外边的长椅子上,其面部表情很是特别,有焦虑,有欣喜,有企盼,有担心。整个一个人类表情博览会。  
        当hushi出来告诉贝塔的爸爸他太太生了一个四肢俱全的男孩子时,贝塔的爸爸忘乎所以地从椅子上一跃而起,不顾一切地往产房里冲,被hushi拦住了。  
        贝塔的爸爸迫不及待地想见儿子,他知道那是他的复制品,他生命的延续。他和他太太结婚实际上是组成了一台复印机,拷贝出另一个生命。  
        贝塔的爸爸在第三天见到了儿子,他想像力有限,没想到儿子将来会做这个国家的总统,他顶多想到儿子当一个公司的总经理。  
        贝塔的童年挺不幸,在他3岁时,他妈妈爱上了一个汽车商。贝塔目睹了爸爸和妈妈离婚的全过程。  
        随后,贝塔和爸爸一起生活。男孩儿身边可以没有母亲,但绝对不能没有父亲,就像女孩儿身边可以没有父亲,但绝对不能没有母亲一样。  
        贝塔从上小学开始就只对政治感兴趣,他上二年级时就能背出地球上几十个重要国家的头儿的名字,他看的第一本书是领袖传记。  
        一次,老师让全班同学挨个说长大想子什么。轮到贝塔时,他说他要当总统。  
        同学们都笑了。  
        “你为什么要当总统?”老师问贝塔。  
        “总统能出人头地。”贝塔回答说。他生活在贫民窟里,十分渴望改变处境。  
        “企业家也能出人头地呀!”老师说。  
        “有人先有钱,后有名。有人先有名,后有钱。前者靠钱买名,后者靠名挣钱。我要靠名挣钱。”贝塔认真地说。  
        贝塔小小年纪竟然能说出这样的话来,老师不得不对他刮目相看,同时也为贝塔的未来捏一把汗。  
        官场险象环生,心不狠手不辣说话算数的人很难有大的发展。对于这一点,贝塔的老师了如指掌。老师的父亲曾经是国会议员,在一次竞选中挨了竞争对手的黑枪,命送黄泉。  
        老师劝贝塔放弃从政的念头,还是学一门知识用以谋生来得稳妥和潇洒。然而贝塔坚定不移当总统的信念,他对老师说,人活着就要让很多人知道你,知道你的人越多,你的生命就越辉煌。  
        老师说,成名不等于成功。  
        贝塔说,成功必须成名。  
        从此,贝塔朝自己的既定目标走下去。他上中学时,在自己的床头写了八个字,作为自己的座右铭。那八个字是:道貌岸然,五毒俱全。  
        贝塔研读了大量的政治家的生平传记,总结出了这个八字座右铭。  
        贝塔的爸爸第一次看见儿子床头这八个字时,呆了半晌,比贝塔的妈妈第一次向他提出离婚时呆的时间还长。          第261集  
        总统府的枪声;  
        挂名总司令贝塔;  
        鲁西西相信报应说;  
        著名女歌星上辈子是老鼠    
        在经历了不计其数的失败和挫折后,在历尽了成千上万的谎言和阴谋后,苍天不负有心人,贝塔终于当上了该国总统。  
        贝塔宣誓就任总统时,心花怒放。他身边是国旗,身后是国徽,场面庄严得不能再庄严了。只有他自己心里明白自己是什么东西。骗得所有人都信你时,你就是总统了。  
        贝塔终日陶醉于出国访问和接待外国首脑,他喜欢站在红地毯上看那些当兵的像机器人一样走来走去,还喜欢和别的国家元首会谈时说几句事先准备好的幽默话。贝塔根本不关心老百姓的生活,也不为老百姓办事。不管他在那儿,身边都跟着摄像师,每天晚上的电视节目必须有他才行。  
        贝塔醉心于和别的国家元首交朋友,他给自己定的目标足在任时要访遍地球上所有国家。其实,贝塔不明白一个道理,世界上最虚伪的友谊,就是国家元首之间的友谊。  
        一天夜里,贝塔正在总统府里高枕无忧地睡觉,枪声惊醒了他。  
        “怎么回事?”贝塔一骨碌爬起来,按铃叫警卫。  
        “政变!他们打进来了。”警卫拎着微型冲锋枪跑进总统的寝室。  
        “挫败他们。”贝塔大声喊。  
        “恐怕已经来不及了。”侍卫长跑进来对总统说,“他们有很多人,都是军人。”  
        军队政变。贝塔被弄了个措手不及。  
        想当总统的不止贝塔一个人。  
        “我是军队总司令!”贝塔气急败坏地嚷嚷,“军队要听我指挥。”  
        “你是挂名的总司令。”一位持枪的中校军官冲进总统寝室,对贝塔说。  
        贝塔被废黜了。  
        当天,新总统将贝塔从总统府里扫地出门。还不错,贝塔的命还在。  
        拔了毛的凤凰不如(又鸟)。当了总统再当平民,贝塔一落千丈,心灵日日受煎熬。  
        除非死在总统宝座上,否则千万别当总统。这是贝塔写回忆录时题在扉页上的一句话。  
        贝塔去世时很是凄凉,原先围绕在他身边的人都是政客,政客的特点就是你有权时他们就来,你没权时他们就走。贝塔下台后,政客们就都离他而去了。贝塔一生没交一个政客之外的朋友。  
        贝塔葬在墓地里一块最不起眼的地方。墓穴旁边的树每年都长出新树叶。  
        贝塔上辈子的传记片结束了。  
        大家都知道贝塔这辈子为什么是老鼠了。  
        “其实,成功并不一定要出名。老老实实活一生,不坑蒙拐骗,自食其力,不花不是自己挣的钱,就是成功。”贝塔深有体会地说。他为自己的上辈子悲哀和羞愧。  
        “花不属于自己的钱的人,都不算成功的人生。”皮皮鲁若有所思地说。  
        “人生比较重要的,是晚年的幸福,想让自己晚年不幸福的最好方法,就是当大官。”歌唱家根据贝塔上辈子的经历总结出这样一段话。  
        “贝塔当总统时,肯定没想到下辈子当老鼠。”舒克感慨万千。  
        “说不定你上辈子也是总统呢,也许,咱俩还在两国和平协议上签过字。”贝塔估计在今天的老鼠同胞巾上辈子当过大官的可能是多数。  
        “这辈子做坏事,下辈子肯定没好报?”鲁西西问。  
        “绝对。”贝塔现身说法。  
        “什么叫坏事?”鲁西西又问。  
        “损人利己。坑蒙拐骗。陷害忠良。占有本不属于自己的财产。打小报告。虐待父母儿女……”皮皮鲁历数坏事的项目。  
        “人活着一定要积德,要与人为善,帮助别人,有同情心,有正义感,要自食其力,不觊觎他人的财产……”燕妮说着说着就想起了多行不义而自毙的姐夫大卫。  
        “贝塔这辈子应该说是善良正义了吧?”鲁西西问。  
        大家都点头。  
        贝塔很感动,就像拿到了鉴定书。  
        “那贝塔下辈子一定应该辉煌了吧?”鲁西西还是对轮回转世善有善报恶有恶报种瓜得瓜种豆得豆半信半疑。  
        “那当然。”皮皮鲁说。  
        “我想看看贝塔下辈子怎么活。”鲁西西说。她想,按贝塔此生的表现,下辈子一定很美满。如果真是这样,她就死心塌地相信轮回转世和善有善报恶有恶报是科学。        
        “能看吗?”大家问贝塔。  
        “当然可以。”贝塔也急于想知道自己下辈子有什么样的经历。  
        皮皮鲁操纵遥感仪寻找贝塔的来世。  
        荧光屏上出现了一个美艳绝伦的女歌星,她风姿绰约的正在举行赈灾义演,她的歌声圆润甜美。一举手一投足都让人叫绝。  
        贝塔的下辈于是闻名全人类的天皇级女歌星和女影星,崇拜她的歌迷影迷遍地都是,上至九十高龄的老人,下至呀呀学语的婴儿,连各国总统都以得到一张有她亲笔签名的照片为荣耀。  
        谁能想到,这样一位功成名就闻名遐迩的女歌星女影星上辈子竟然是一只老鼠。  
        鲁西西信了,彻头彻尾地信了。她为人类中那些不善良不地道专干损事的人的下辈子担心,她希望他们赶紧改邪归正。  
        老鼠只要积德,下辈子也能辉煌。  
        “难怪贝塔和歌唱家结婚,原来是同行。”燕妮说。  
        “你们看贝塔下辈子真够辉煌的,挣了钱都捐献给社会。”舒克说。  
        “其实大款捐钱多是一种赎罪心理。不过贝塔下辈子捐钱是真心的,看得出来。”歌唱家说。  
        “给歌唱家喝皮皮鲁口服液,让她变大了去人类风光一回。”燕妮突发奇想。          第262集  
        五角飞碟测出皮皮鲁上世;  
        五角星球上的皮皮鲁;  
        王宫花园一见钟情;  
        歌唱家谢绝测上辈子    
        大家认为燕妮的建议十分精彩。  
        当歌唱家意识到自己服用皮皮鲁口服液后就能像正常人一样登台唱歌后,她兴奋得两眼放光。她渴望登台,渴望面对观众唱歌。  
        “电视台正在举办青年歌手大奖赛,明天让歌唱家去报名。”鲁西西说。  
        大家一致赞同。  
        “咱们看看皮皮鲁上辈子干什么吧?”贝塔提议。  
        “行。”皮皮鲁遥感自己的上世。  
        荧光屏上出现了茫茫宇宙,无数个星球,无数个星系,还有黑洞。        
        大家都屏住呼吸。皮皮鲁尤其紧张,在即将知道自己身世的前夕,任何人都会忐忑不安。  
        皮皮鲁上辈子是外星人!  
        “原来整个宇宙中的生命都可以互相轮回转世。我原来还以为只是地球上的生命自己转来转去呢。”燕妮说。  
        皮皮鲁上世所在星球看样子离地球挺远,那个星球上的生命和人类比较接近,科学技术比现在的地球还发达。  
        “比咱们去过的双子星球还好。”贝塔对舒克说。  
        舒克点头同意。他和贝塔在双子星球上住过30天,感觉很不错。  
        皮皮鲁上辈子居住的那个星球叫五角星球,星球上的惟一交通工具就是飞碟。孩子三岁就会驾驶飞碟。  
        “难怪皮皮鲁能做出五角飞碟,连名字都和五角星球一样,原来是遗传。”燕妮盯着屏幕说。  
        大家都对生命的延续性感到惊讶。  
        皮皮鲁在五角星球上是知名人物,他为同胞发明了许多东西。在五角星球上,人人以能发明为荣。  
        皮皮鲁也没想到自己曾经是外星人,他激动地看自己上辈子的经历。  
        “看看我的上辈子。”燕妮迫不及待地要求。  
        皮皮鲁操纵遥感仪遥感燕妮的上世。        
        燕妮的上辈于是地球上某国国王的一位公主。这是一位善良美丽的公主。一天,公主在王宫的花园中赏月,一个飞碟降落在公主身边。  
        公主看着从来没见过的器物,有点儿害怕。  
        一位外星人从飞碟里走出来。  
        “是皮皮鲁!”鲁西西大喊。  
        从飞碟里出来的外星人真是皮皮鲁。  
        皮皮鲁对公主说:  
        “您好,这是什么星球?”  
        “您好,这儿是地球。您不是地球上的人?”燕妮公主好奇地问。  
        “我来自五角星球。”外星人皮皮鲁说,他盯着燕妮公主看。他喜欢她。  
        “五角星球?”燕妮公主显然是头一次听说。“你们的星球有月亮吗?”  
        “有三个月亮。”外星人皮皮鲁说。  
        “比我们这儿多两个。”燕妮公主喜欢月亮。“能带我去你们那儿看月亮吗?”  
        燕妮公主喜欢外星人皮皮鲁,从看他第一眼起就喜欢。  
        两性之间只有一见钟情。爱情是火花,只有碰撞才能产生。爱情“培养”不出来。  
        “当然可以。”外星人皮皮鲁很希望能娶这个地球公主为妻。        
        燕妮公主准备登上飞碟。  
        仆人们都跪下了,他们阻止公主随外星人走。一位仆人去向国王通报。  
        国王派相当于一个连的卫兵在一个相当于副团级的武士带领下来抓“绑架”公主的外星人。  
        外星人皮皮鲁正要和燕妮公主踏上飞碟,花园里一片杀声。  
        “怎么了?”外星人皮皮鲁问燕妮公主。  
        “父王不让我走。”燕妮公主眼泪汪汪。  
        “你想走吗?”  
        “想。”  
        “那他们阻止不住咱们。”  
        “你打得过他们?”  
        “小菜一碟。”  
        外星人皮皮鲁用飞碟轻而易举就将相当于副团级的武士和相当于一个连的卫兵击倒在地上。燕妮公主在飞碟里拍手叫好,她还吻了外星人皮皮鲁。  
        “真是超级青梅竹马。”舒克边看边说。  
        燕妮公主乘坐飞碟来到五角星球,该星球的所有人都对皮皮鲁弄来一个外星人当妻子表示反对,经过投票表决,一致通过将外星球燕妮公主送回地球的决议,不管皮皮鲁怎么抗议怎么央求也没用。  
        表决是给谬误发的通行证,因为真理往往掌握在少数人手中。        
        皮皮鲁只得将燕妮公主送回地球。  
        分别的时候两人悲痛欲绝。  
        “下辈子我一定嫁给你。”燕妮公主泣不成声。  
        “下辈子我一定投胎到地球上。”皮皮鲁指着天空发毒誓,“下辈子如果不能投胎到地球上,我宁可……”  
        燕妮公主和外星人皮皮鲁生离死别,场面令人潸然泪下,赛过古今中外所有悲剧。  
        “这就叫缘份。”贝塔看着身边的皮皮鲁和燕妮,说。  
        “保守估计,歌唱家上辈子准是贝塔总统的秘书。”舒克说。  
        “我不想测。”歌唱家挺怕知道自己的上世,更怕知道来世,她觉得还是糊涂活着幸福。          第263集  
        涛涛轰上辈子的职业;  
        将成为细菌的亿万富翁;  
        皮皮鲁为歌唱家起名字;  
        歌唱家喝皮皮鲁口服液    
        “测测皮皮鲁下辈子干什么。”贝塔提议。  
        “我不想提前知道,万一下辈子是棵树什么的这辈子也活不好了。”皮皮鲁说。  
        “咱们测一下别人吧?”舒克出了个主意。  
        “别人?”鲁西西不知道舒克说的别人指谁。  
        “世界上那些有名的人,比如说,当今各国元首,歌星体育明星影星作家亿万富翁什么的。”舒克说。  
        “这主意不错。”贝塔_兴致勃勃。  
        “你开个名单,咱们挨个遥感。”皮皮鲁对舒克说。        
        舒克去五角飞碟他的卧室里拿来纸和笔,埋头开名单。  
        皮皮鲁接过名单,第一名是著名作家涛涛轰,第二名是一位国家元首,接着是影星歌星政星富星  
        皮皮鲁操纵五角飞碟遥感仪测作家涛涛轰的上世。舒克还记着涛涛轰沽名钓誉剽窃他的传世之作《人类,我是你的朋友》的事。  
        大作家涛涛轰的上辈子令大家吃惊。涛涛轰上世是烟花女,说通俗点儿,就是(禁止)。  
        荧光屏上出现了作家涛涛轰上辈子工作的场面。燕妮、鲁西西和歌唱家闭上眼睛不看。  
        “关上吧。”鲁西西说。  
        “看看涛涛轰下辈子是什么。”舒克在心里原谅涛涛轰了,从他上辈子所从事的职业看,这辈子干成这样就不错了。  
        荧光屏上出现了一只填鸭,饲养员正在往它嘴里塞填食物,填鸭以每小时净增1两的速度增加着体重。  
        作家涛涛轰下辈子是填鸭。  
        “上帝挺公正。”舒克用满意的口气说。  
        涛涛轰在长到5公斤时,被送到烤鸭店的熔炉里挂起来接受熏陶。  
        皮皮鲁按舒克开的名单顺序挨个遥感他们,一位国家总理下辈子是鸟一位体育明星下辈子是医生一位著名歌唱家下辈子是哑巴还有一位亿万富翁下辈子是细菌。  
        “一定不能干损人利己的事。一定要尽量帮助别人。做事一定要留有余地。”鲁西西边看遥感仪边说。  
        “恶有恶报,善有善报,这绝对是真理。多行不义必自毙。可惜好多人不信。”歌唱家又想起了胡安娜。  
        “咱们没时间。如果咱们把世界上每个人的上世和下世经历录到录像带上,送给每个人,准对他活好这辈子起作用。”贝塔总是有歪点子。  
        “那准有不少人自杀,当他们看到下辈子变猪变狗,准连这辈子也活不下去了。”舒克说。  
        朋友们七嘴八舌讨论起人生,最后一致同意鲁西西的观点:善良就是成功的人生,即使默默无闻。  
        到了吃午饭的时间,燕妮为大家做了一顿西餐,朋友们在鲁西西别墅里进餐。  
        吃西餐,鲁西西想起了罐头小人约翰。  
        “咱们什么时候去找约翰?”鲁西西一边喝奶油蕃茄汤一边问。  
        “等歌唱家参加完青年歌手电视大奖赛,咱们就去找约翰。”皮皮鲁说,他很想让歌唱家堂堂正正地在歌坛上施展才华。        
        “我带歌唱家去报名。”鲁西西说,“咱们得给歌唱家起个艺名,没名字无法参加比赛。”  
        “皮皮鲁给她起个名字吧。”贝塔要求皮皮鲁给歌唱家起艺名。  
        皮皮鲁思索。起名字最难,有的父母搜肠刮肚给孩子起名字,平均每个孩子会轮番享受上百个名字最后才在一个名字上定居。其实,名字越简单越显得父母有知识。  
        “叫贝一怎么样?”皮皮鲁说。  
        “贝一?”舒克琢磨。  
        “国外的太太嫁给先生后就随先生的姓,歌唱家嫁给贝塔,理应姓贝。一嘛,就是第一的意思,也有惟一的含义。”皮皮鲁解释。  
        “就叫贝一,我赞成。”歌唱家喜欢这个艺名,她终于有了自己的名字。  
        “喝皮皮鲁口服液吧!”鲁西西拿给歌唱家贝一一瓶皮皮鲁口服液。  
        歌唱家在大家的陪同下来到鲁西西别墅外边,她心情很激动。歌唱家一直羡慕有着高大身躯的人类,她因为小而承受了太多的苦难。  
        “你变大了。我怎么办?”贝塔小声问歌唱家太太,他怕婚姻因此破裂。  
        “贝一参加完比赛回家就可以再服微缩药变小,保你天天能和太太团聚。”舒克听见了贝塔的话,他安慰贝塔。  
        歌唱家贝一喝皮皮鲁口服液。  
        她变成了一个丰满迷人的漂亮女人。  
        “就凭贝一的风采,拿歌手大奖赛冠军不成问题。”贝塔很自豪。  
        鲁西西也喝了皮皮鲁口服液。  
        “我给你办报名手续。报上说,要准备照片两张,还要录一盘自己唱的歌。对了,还要身份证。皮皮鲁,贝一没有身份证怎么办?”鲁西西问桌子上的皮皮鲁。  
        “咱们总不能伪造身份证吧?”皮皮鲁无能为力,“你先带贝一去照像,然后录歌,我们想想办法。”  
        “我还要为你设计演出服,以后你穿什么,社会上就流行什么,你的身材当模特都绰绰有余。”鲁西西上下打量贝一。  
        歌唱家贝一已经热泪盈眶了。小了几十年,突然变大了,她自己明白这变化的意义。不管你有多大的本事,只要你小,就没办法为社会服务。歌唱家贝一很想用自己的嗓子为地球做一些事,她现在有资格了。  
        歌唱家贝一能参加电视青年歌手大奖赛吗?        第264集  
        便衣警车尾随鲁西西;  
        探长林在照像馆门口  
        下午是青年歌手电视大奖赛报名截止时间,鲁西西带歌唱家贝一去街上的照像馆照像,然后再去舒克贝塔公司录音。  
        “带上五角飞碟通讯器,万一遇到麻烦就告诉我们。”皮皮鲁对鲁西西说。  
        “我在五角飞碟里值班。”贝塔说。  
        鲁西西和歌唱家离开家。  
        停在皮皮鲁家楼下的一辆小轿车里的男人看见鲁西西和歌唱家出来后,拿起车载电话。  
        “报告探长林,鲁西西和一个陌生女子从家里出来。”那人说。  
        “陌生女子?什么时候去鲁西西家的?”探长林在电话里问。        
        自从皮皮鲁失踪并被德国警方要求引渡后,探长林派人24小时蹲守在皮皮鲁家的单元门口。  
        “我从没见过这位女子。”  
        “没见过?她从天上飞进皮皮鲁家的?”探长林不满意下属的这个回答。  
        “确实没见过。”  
        “她们干什么去了?  
        “现在她们坐进了皮皮鲁的汽车,鲁西西开车,那女子坐在她身边。”  
        “跟着他们。”探长林下指令。  
        “明白。”  
        便衣警车跟在鲁西西的汽车后边。  
        鲁西西和歌唱家毫无察觉。  
        歌唱家看着外边的一切都感到新鲜有趣,她觉得世界变小了。过去.她每次出来都把主要精力放在隐蔽自己上,没注意过四周的景色。  
        “咱们先去照快像,然后去公司录音。”鲁西西边开车边说,“下午5点钟之前一定要去电视台报名。”  
        “公司在哪儿?离这儿远吗?”歌唱家问。  
        “公司在北合雁大街161号,东冠门十字路口北边。对了,跟你说地址也没用,你不熟悉这座城市。”鲁西西笑了。  
        鲁西西将车停在一家照像馆旁边。她和歌唱家走进照像馆里。  
        便衣警车忙向探长林汇报。  
        “进了照像馆?”探长林感到新鲜,“去看看她们干什么。”  
        探长林自从第一次和皮皮鲁打交道,就为皮皮鲁的魅力所折服,他对这位前物理学家有相当的好感。自从皮皮鲁失踪后,他一直为皮皮鲁的安全担忧,他担心皮皮鲁被国际黑社会谋害。所以他派人日夜监视皮皮鲁的住所,只要皮皮鲁一回家,马上对他实施保护。  
        可是皮皮鲁一直没露面。  
        他去哪儿了呢?  
        部下向探长林报告的信息非常重要,皮皮鲁的家里出现了一个陌生女子,她是谁?从哪儿来?到皮皮鲁家干什么?  
        探长林立即驱车赶到那家照像馆门口,部下正从照像馆里出来。  
        探长林示意部下上他的车。  
        “她们去照像馆干什么?”探长林问部下。  
        “鲁西西陪那陌生女子照像,照的是快像。”部下向探长林汇报。  
        “照快像?”探长林思索。  
        “她们出来了。”部下指指照像馆门口。  
        鲁西西和歌唱家一边看照片一边走出照像馆,上了汽车。  
        探长林从没见过歌唱家,他凭直觉认为这位陌生女子同皮皮鲁失踪有关系。  
        “我跟着他们,你回到皮皮鲁家继续蹲守。”探长林对部下说。  
        鲁西西和歌唱家上了汽车,系安全带。  
        “咱们现在去舒克贝塔公司,东冠门离这儿不远,如果不堵车,有5分钟就能到。”鲁西西发动汽车一  
        “这路真宽,德国可没有这么宽的路。”歌唱家望着窗外说。  
        “这叫长安街,是中国最有名的路。”鲁西西驾驶贝壳色的汽车自东向西行驶在长安街上。  
        汽车停在公司门口。  
        “到了。”鲁西西解安全带。  
        歌唱家看到了路旁的舒克贝塔公司。  
        “真漂亮。”歌唱家目不暇接。  
        鲁西西陪歌唱家参观舒克贝塔公司。  
        探长林的车停在鲁西西的车旁边,他戴上墨镜,从车上往舒克贝塔公司里看。  
        “我带你去我的办公室。”鲁西西对歌唱家说。  
        歌唱家跟着鲁西西上二楼,走进鲁西西的办公室。          第265集  
        公司职员被歌声迷住;  
        卡车司机出口伤人;  
        贝塔略施小计;  
        身份证不让贝一报名    
        鲁西西的办公桌后边竖着四幅抽象派画,画面上有说不清的景物,给办公室平添了童话色彩。  
        鲁西西让秘书准备录音机,为歌唱家录制一盘报名用的录音带。  
        歌唱家的歌声使得公司职员们放下手中的工作,他们从没听过这么动听的歌声。  
        鲁西西也被歌唱家的歌声迷住了,她想起自己第一次发现罐头小人的情景。鲁西西感受到人生的戏剧性。  
        录音完毕,鲁西西和歌唱家离开公司去电视台报名参加青年歌手大奖赛。        
        探长林驾车跟在鲁西西的汽车后边。  
        一辆卡车和鲁西西的汽车并排行驶,鲁西西超过那辆卡车。卡车司机是个没有自尊的人,自卑感极重,他非要再将鲁西西的车超过去。  
        卡车司机红了眼,他的车上还坐着几位男士,那些人给司机打气。  
        道路前方出现了堵塞,鲁西西减速。  
        卡车乘机追了上来,用车头别住鲁西西的汽车。鲁西西惊讶地往右边看,她没有意识到刚才这辆卡车在和她较劲。  
        卡车上的人一边破口大骂一边打开车门跳下车,那司机跑到鲁西西这边,强行拉车门。  
        探长林真想下车将那混蛋司机揍一顿,但他忍住了,他不想让鲁西西认出他。  
        鲁西西将五角飞碟通讯器递给歌唱家:  
        “必要时,让贝塔帮忙。”  
        鲁西西下车。  
        “干什么?”鲁西西问那满脸通红的司机。  
        “×××!”  
        那司机张嘴就骂。  
        “你怎么骂人?!”鲁西西生气了。  
        “我骂你啦!你他妈开那么快干什么?×××!”卡车司机气急败坏,被别人的车超了过去,就如丧考妣。        
        卡车上的那几个男人跟着起哄,说一些不堪入耳的话。后边的汽车受阻,停了长长的一排。  
        探长林戴上墨镜,准备下车教训那几个混蛋。  
        歌唱家呼叫贝塔。  
        “我是贝塔,怎么了?”贝塔忠于职守,呆在五角飞碟里。  
        “我们在路上碰到一个流氓司机,现在他正骂鲁西西。”歌唱家说。  
        “太好了!”贝塔立刻兴奋。  
        “你说什么?”歌唱家以为自己昕错了。  
        “我说太好了。”贝塔打开五角飞碟遥感仪,他看清了现场。  
        鲁西西被那混蛋司机激怒了,她回头看车里的歌唱家,歌唱家冲她点点头,鲁西西知道五角飞碟能够帮她了。  
        “我警告你,你如果再骂,你就要倒霉了。”鲁西西大声对卡车司机说。  
        “×××!”卡车司机一字一句地又骂了一遍。骂完后他挑衅地看着鲁西西。  
        “我让你一个月说不出话。”鲁西西宣布。  
        那司机刚要嘲笑鲁西西,他张开嘴,却真的说不出话了。只见拼命摇晃脑袋,还使劲儿咳嗓子。  
        卡车司机急了,他不顾一切地朝鲁西西扑过来。  
        鲁西西用手一指他,他摔倒在地上。        
        卡车上的几个男人跳下车来帮同事。  
        鲁西西挨个将他们打翻在地。  
        探长林的眼睛睁得比墨镜还大,他没想到皮皮鲁的妹妹也有绝招儿,探长林想起了他曾经在皮皮鲁家见过皮皮鲁用意念移动杯子。  
        鲁西西对地上的几个男人说:  
        “起来!记住,天外有天,做事一定要留有余地。”  
        那几个男人慌忙爬起来回到卡车上。那司机一句话也说不出来。他用手和鲁西西比划,意思是真的只有一个月说不出话吗?  
        “就一个月。”鲁西西告诉他,“以后别再骂女士。”  
        鲁西西启动汽车,爆发力极强的汽车从静止状态突然提速到每小时100公里只需8秒钟。  
        探长林险些被甩得无影无踪。  
        电视台门口的报名处排着队。  
        鲁西西和歌唱家下车排队。  
        探长林将车停在队伍的右侧.他皱着眉头分析鲁西西和陌生女子到电视台来干什么。  
        “报名参加唱歌比赛?”探长林思索。  
        突然,探长林想起了德国警方曾说皮皮鲁和德国著名歌星胡安娜有牵连。  
        “唱歌!皮皮鲁怎么总对唱歌有兴趣?”探长林盯着排队报名的鲁西西和陌生女子想。  
        轮到鲁西西和歌唱家了。  
        “姓名?”报名处的工作人员问。  
        “贝一。”歌唱家回答。  
        “报名磁带。”工作人员伸手。  
        鲁西西从包里拿出录音带交给工作人员。  
        “身份证?”  
        “没有。”鲁西西说。  
        “怎么会没有身份证?”工作人员探头看歌唱家。  
        “我有介绍信。”鲁西西将舒克贝塔公司介绍歌唱家参赛的介绍信递给工作人员。  
        “没用。只要身份证。”工作人员不看介绍信。  
        “她有唱歌天才,绝对会受欢迎,您就让她报名吧。”鲁西西说好话。  
        “没有身份证,不行。下一个。”报名处的工作人员不同意。  
        鲁西西和歌唱家站在队伍旁边看别人报名。  
        “怎么办?”歌唱家眼里有泪水。  
        “不参加了,什么破电视大奖赛,当了冠军,唱不好也没用。”鲁西西忿忿地说。          第266集  
        探长林查出还有一个叫贝一的;  
        爱因斯坦9岁才会说话;  
        牛顿上学时被编入差班;  
        鲁西西第一次见搜查证    
        鲁西西和歌唱家驾车打道回府。  
        探长林飞快地下车凑到报名处工作人员面前。  
        “刚才那两位女士是报名吗?”探长林问那工作人员。  
        “你问这干什么?”工作人员白了探长林一眼。  
        探长林掏出证件给她看。  
        “是报名。”工作人员说。  
        “看样子她们好像不高兴?”探长林问。  
        “没让她们报名。”  
        “为什么?”  
        “她没有身份证。”        
        “她们两个都报名?”  
        “不,只有一个报名。”  
        “个高的还是个矮的?”  
        “个子矮一些的。”  
        “她叫什么名字?”  
        “贝一。”  
        “她说她没有身份证?”  
        “是的。”  
        “谢谢。”  
        探长林回到汽车上,拿起车载电话,让局里给查叫贝一的人的资料。  
        “本市设有叫贝一的。”答复说。  
        “在全国查。”探长林边开车边说。  
        “全国只有一个,是男的,而且已经七十多岁了。”  
        探长林接通蹲守皮皮鲁家的警官。  
        “鲁西西回家了吗?”探长林问。  
        “还没有。不,你等一下,现在回来了,车里一共两个人,还是刚才那位女子。她们下了车,走进单元门。”蹲守警官报告。  
        “我马上赶到。”探长林说完加大油门。  
        鲁西西和歌唱家回到家中,皮皮鲁和朋友们都从鲁西西别墅里出来,贝塔也从五角飞碟里出来,大家问结果。        
        “歌唱家没有身份证,人家不给报名。”鲁西西略显沮丧地说。  
        “我看参加那个什么歌手电视大奖赛没什么意思,只要唱得好,有听众,就行了,要那个虚衔没劲。”贝塔说。  
        “就是,我从前在一张报纸上看到过一篇文章,说是这种大赛猫腻特多。”舒克说。  
        “不过,得了大赛的冠军,也算是一种文凭吧?介绍贝一时,就可以说,她获得过什么什么奖。”燕妮还是为歌唱家不能参加比赛感到遗憾。  
        “这是虚的,要不要没关系。”皮皮鲁反对,“著名科学家达尔文小时候学习很糟糕,上爱丁堡大学读书时考试经常不及格,他爸爸对达尔文说,将来你不仅会给自己丢脸,还会给家里丢脸。可是后来达尔文成了大科学家。”  
        “这倒是,”燕妮觉得还是皮皮鲁的道理对.“就拿我的同胞爱因斯坦来说吧,他在9岁前连话都说不好,爱因斯坦的父母认定爱因斯坦有智力障碍,上中学时,爱因斯坦各门功课都很糟,老师要他退学,还对他说,你将无所作为。”  
        “爱迪生小时候更笨,以至于被学校开除了,校长曾对爱迪生下断言:终身一事无成。”舒克看了不少书,“还有牛顿,少年时代懒散,在学校被编入差班,还打架。”        
        “大画家毕加索上学时,除了画画,其他学科都不愿意学。他10岁离开小学时,读书和写字几乎都不会,后来毕加索虽然考上了美术学校,可他仍然无法适应学校生活,中途退学。”鲁西西说。  
        “我看,真正有成就的人,小时候能规规矩矩上学的少。那些所谓的神童,长大了有几个有出息的?小叫候成功,长大了就不会成功,长大了成功,小时候就不成功。人可能一生不成功.不可能一生都成功。”贝塔总结。  
        歌唱家感激地看着大家,她知道朋友们都在为她操心,其实她也不在乎拿什么奖杯,只不过有些失望罢了。  
        “你自己作一首歌,咱们将它推上社会,我就不信好歌没人唱,保准流行得特快。”贝塔说。  
        歌唱家点头。  
        敲门声。  
        鲁西西蹑手蹑脚走到门后通过门镜往外看。她一愣。  
        “是探长林。”鲁西西回来告诉大家。  
        “快给歌唱家用微缩药。”贝塔说。  
        歌唱家变小了。  
        “进五角飞碟。”皮皮鲁说。  
        大家钻进五角飞碟。  
        鲁西西将五角飞碟藏进阳台上的一个纸箱子里。        
        鲁西西开门。  
        “您找谁?”鲁西西问探长林。  
        “皮皮鲁还没回来?”探长林问。  
        “没有。”  
        “您家里就您自己?”  
        “对。”  
        “刚才我看见一位女士和您一起进来的。”探长林说。  
        “没有。”鲁西西否认。  
        “我能进屋看看吗?”  
        “你没有权利。”  
        探长林掏出搜查证。  
        鲁西西摇摇头,让他进屋搜查。  
        所有房间都看过了,探长林没有看见那位叫贝一的陌生女子的踪影。  
        “搜查完了?”鲁西西问探长林。  
        “她喜欢唱歌?”探长林突然发问。  
        “她是谁?”鲁西西知道刚才被跟踪了。  
        “她叫贝一。”探长林盯着鲁西西的眼睛说。  
        “莫名其妙。”鲁西西说。  
        “我想保护皮皮鲁。”探长林说。  
        “谢谢。”鲁西西对这位探长并无反感。          第267集  
        皮皮鲁是国宝;  
        全天候监视探长林;  
        找红沙发音乐城被列入计划;  
        破案是脑力劳动    
        “皮皮鲁是国宝,他应该受到保护。”探长林认真地对鲁西西说。  
        “谁说皮皮鲁是国宝?”鲁西西问。  
        “我这样认为。”探长林说。  
        鲁西西挺感动。  
        “皮皮鲁没来过电话?”探长林问。  
        鲁西西摇头。  
        “据我分析,人家是想留皮皮鲁在国外,而皮皮鲁不干。”探长林一边说一边观察鲁西西的表情。  
        鲁西西不说话。  
        “发达国家之所以发达,就因为它们善于网罗人才,像皮皮鲁这样的人,所有发达国家都排着队望眼欲穿给他发绿卡。”探长林接着说,“我虽然和皮皮鲁接触不多,但我强烈感受到他是一个有高尚人格的人。”  
        鲁西西轻微地点头。  
        “贵公司最近推出的皮皮鲁口服液,我认为就是皮皮鲁先生的杰作。”说到这儿,探长林的眼睛里突然一闪。  
        皮皮鲁既然能发明将人类复原的口服液,那他一定也有能把人缩小的药,皮皮鲁是不是变小后藏起来了?刚才那女子是否也是变小了藏起来的?探长林茅塞顿开,他断定皮皮鲁和那陌生女子就在这屋里。  
        “我觉得皮皮鲁就在这屋里,他变小了。”探长林用稳操胜券的口气说。  
        鲁西西摇头。  
        “这是我的电话,我希望皮皮鲁先生能给我打电话,再见。”探长林掏出名片递给鲁西西后走了。  
        鲁西西关好门,将五角飞碟从箱子里拿出来,贝塔第一个从五角飞碟里出来。  
        “这探长的脑子够好用的。”贝塔对鲁西西说。  
        “我看这人可以信任。”舒克谈自己的看法。  
        “原来他一直在监视咱们。”皮皮鲁说。  
        “咱们也遥感监视他三天,全天候24小时监视,如果他的人品真的不错,就可以联系一下。”皮皮鲁说。  
        “我来监视。”贝塔说。  
        “三班倒。一个班8小时,贝塔第一,我第二,舒克上二班。”皮皮鲁说。  
        “我们也可以值班。”燕妮指着歌唱家和自己说。  
        “女士全天候监视男士,不大方便吧?”贝塔反对。  
        燕妮觉得贝塔的话有道理,放弃了。  
        “歌唱家,你写几首歌,我联系一个录音棚,咱们录一盘磁带。”鲁西西提议。  
        “还得找个乐队伴奏。”舒克说。  
        “找红沙发音乐城不就得了吗?”贝塔一直念念不忘红沙发音乐城(参见学苑出版社出版的《鲁西西传》,各地书店有售)。  
        “红沙发音乐城如果和歌唱家联袂,产生的就是世纪性的音乐了。”鲁西西说。  
        “等把探长林这件事办完,咱们就去找红沙发音乐城。”皮皮鲁说,“要不然,这探长老跟踪咱们。也怪别扭的。再说了,也耽误他的时间。”  
        “我现在就去值班,看看那探长整天干什么,别老是把别人当坏人抓。”贝塔钻进五角飞碟。  
        “咱们游泳去吧?”燕妮对歌唱家说。  
        歌唱家同意了。        
        皮皮鲁跟着贝塔钻进五角飞碟。  
        贝塔正在调整遥感仪,寻找探长的方位。  
        “忍不住想看看探长林?”贝塔一边操作电脑一边问皮皮鲁。  
        皮皮鲁点点头。他对这位探长有好感。  
        荧光屏上出现了探长林,他正驾驶汽车行驶在一条大街上,他的车载电话铃响了。  
        “我是3号。”探长林拿起话筒。  
        “3号注意,××小区发生一起凶杀案,请你马上赶到现场。”  
        “3号明白。”探长林放下话筒,从方向盘底下取出警灯,手伸出窗外将警灯放在车顶上。  
        探长林的汽车呼啸着朝××小区疾驶而去。  
        “还真像警匪片。”贝塔说。  
        “当个侦探也不容易,哪儿死人往哪儿跑。”皮皮鲁说。  
        “他们准最理解生命的意义。”贝塔说。  
        “对,不真正了解死,就不可能理解生的含义。”皮皮鲁同意贝塔的话。  
        探长林驱车赶到了案发现场。  
        几辆警车停在一座楼旁,四周全是围观的人。探长林拨开人群,走到警车旁的一位警察身边。  
        警察见侦探来了,忙引导他进人楼房。  
        这是一起谋财害命案,被害人是七十多岁的老太太,她是被勒死的。家中整齐,没有被翻过的迹象。  
        据被害人的儿子讲,衣柜中的五千元现金被盗。  
        探长林仔细地察看现场,他一边观察一边思索,每分每秒都在同案犯较量。  
        “破案实际上是脑力劳动。”皮皮鲁边看边说,他觉出那探长的脑细胞在工作。  
        “隔壁住的什么人?”探长林问死者的儿子。  
        “一个小伙子,和我们关系非常好,没工作,自己跑点小买卖。”死者之子答道。  
        “他会开车吗?”探长林问。  
        “会开,他自己有一辆灰色的小面包车。”  
        探长林走到隔壁敲门。  
        小伙子将门打开了。  
        “咱们聊聊。”探长林进屋后随手将门关上了。  
        小伙子看着探长林,不说话。  
        “那老太太是你杀的。”探长林平静地说。  
        小伙子扑通一下给探长林跪下了。          第268集  
        凶手给探长下跪;  
        皮皮鲁穿着浴衣进五角飞碟;  
        宾馆浴缸里的女尸;  
        黎经理的公司像殡仪馆    
        探长林的确是神探。他从死者被勒死的脖痕上,看出了凶手右手有开车挂挡的习惯性动作。再加上凶手极为熟悉死者家中的情况,没有翻找财物,而是直接拿钱,因此,此案必为熟人所为。  
        “我这儿一共有一万元,全给您了,求您放我一条生路。”小伙子见只有探长一人,觉得事情还有转机,他做最后的努力。  
        “谋杀加行贿。”探长林接过一万元现金后,将凶手铐上了。  
        探长林拉开身后的门,叫警察带走凶手,同时将一万元钱交给部下。        
        从探长林抵达案发现场到破案,只用了15分钟。  
        “这人还行。”贝塔给探长林一个高评价。  
        皮皮鲁点头,刚才那凶手向探长林行贿被拒绝的场面给皮皮鲁留下了深刻的印象。受贿是人类社会的毒瘤,是商品经济对人性的摧残。受过贿的人,即使这辈子侥幸逃脱,下辈子也决然不会有好结果。这是五角飞碟告诉皮皮鲁的。  
        皮皮鲁在心里已经同意和探长联系了。拒绝受贿的人,综合品质不会差。  
        “你再继续监视他。”皮皮鲁对贝塔说。  
        贝塔像看警匪片。  
        皮皮鲁离开五角飞碟,走进鲁西西别墅,正好碰见舒克往外边走。  
        “探长林怎么样?”舒克也对探长感兴趣,他等不到他值班时再看,正准备去五角飞碟提前接班。  
        “人品不错。”皮皮鲁将探长林刚才的破案经过和拒贿讲给舒克听。  
        “受贿的人最可恶。难得有探长林这样的人。”舒克说,“我去看看探长林现在干什么。”  
        “去吧,我游一会儿泳,然后就给探长林打电话。”皮皮鲁说。  
        舒克钻进五角飞碟和贝塔一起监视探长林的行踪。皮皮鲁换上游泳裤,和燕妮、歌唱家一起游泳。  
        歌唱家已经学会了游泳,她现在除了唱歌,最喜欢干的事就是游泳。  
        “你再掉进下水道里,没有饭盒也能活了。”皮皮鲁游到歌唱家身边说。  
        “上帝保佑我这辈子再也别掉进脏水里了。”歌唱家在胸前画了个十字。  
        燕妮游过来说:“在游泳池里游泳没太大意思,有机会咱们去大海里游泳。”  
        皮皮鲁正要搭话,舒克跑到游泳池边对皮皮鲁说:  
        “探长林遇到难题了,你来看看。”  
        皮皮鲁连游泳裤都没换,跑进五角飞碟。燕妮让舒克带浴衣给皮皮鲁穿上。  
        “一位小姐在宾馆被杀,探长林……”贝塔看见皮皮鲁来了,想将经过告诉他。  
        “给我放一下录像。”皮皮鲁想看实况。  
        贝塔将五角飞碟遥感电脑自动录制的录像带放给皮皮鲁观看。  
        这是一家四星级宾馆。客房服务员正在挨个打扫房间,楼道里停着装有清洁工具和各种以新换旧的客房必备卫生用品的小车。  
        一位女服务员用钥匙打开1428房间,她先打扫卧室的卫生。当她准备打扫卫生间推开卫生问的门时,她发出了一声动物般的喊叫,她看见浴缸里泡着一具女尸。        
        闻讯赶来的宾馆保安人员拨通了报警电话。刚刚离开××小区的探长林立即掉转车头直奔发案的宾馆。  
        经查实,死者是某时装公司的黎经理,该经理是喝了(被禁止)后,被人按人裕缸中窒息而死的。  
        “谋杀案。”探长林对助手说。  
        “今天谁来过这个房间?”探长林的助手问服务员。  
        “没注意。”服务员剐忆不起来。  
        “这位黎经理在这儿住了多长时间?”探长林问服务员。  
        “一个星期。”服务员说。  
        “去死者的公司。”探长林对助手说。  
        黎经理的时装公司看上去很不景气,职员都无精打采,虽然公司外边挂着时装公司的招牌,可如果拆掉牌子,别人准以为是殡仪馆,气氛十分悲痛。  
        时装公司的职员们显然还不知道经理被谋杀的消息他们对于探长的造访表示惊讶。  
        探长林先和副经理谈话。  
        “贵公司里有没有和黎经理矛盾很大的人?”探长林问副经理。  
        “本公司的所有人几乎都和黎经理有矛盾,黎经理是一个心胸狭窄的人,小肚(又鸟)肠。”副经理显然也同经理不和。        
        探长林不能怀疑所有的职工。  
        “公司经营状况?”探长林又问。  
        “不好。您刚才大概注意到我们的商店了,最少的时候一个月的营业额只有一元七角。”副经理说。  
        “那你们靠什么维持?”  
        “靠银行贷款。”  
        “银行怎么会贷款给你们这样的公司呢?”  
        “黎经理别的本事不大,但贷款却很有能力,每次都能贷个几十万元。”  
        探长林皱起眉头。  
        他感觉到贷款里有问题。  
        这时,助手叫探长林接电话。  
        “谁把电话打到这儿来找我?”探长林站起来往隔壁房间走。  
        “局里。”助手说。  
        局里告诉探长林,这件案子可能比较简单,黎经理因经营不善无法还债而自杀。局里还说让探长林再去接一个新案子。  
        探长林挂上电话后,眼睛望着窗外。  
        “咱们走吗?”助手问。  
        “这个案子我一定要查个水落石出,谁干扰也不行。”探长林一拳砸在人家的桌子上。          第269集  
        旗袍小姐给客人点烟;  
        早晨都排满的饭局;  
        部长秘书在饮料里下药;  
        黎经理命归浴缸    
        “探长林遇到麻烦了,有人不让他再管这个案子了。”贝塔关上录像机。  
        “怎么回事?”皮皮鲁不明白,但他感到这个黎经理溺死浴缸案不一般。  
        “咱们遥感一下。看看究竟是怎么回事。”舒克提议。  
        皮皮鲁同意。  
        贝塔调整遥感仪,荧光屏上迅速变化着不同的画面。  
        “就是这个。”舒克在画面上看到了时装公司的黎经理。        
        黎经理正和几个男士在进餐。进餐的环境十分豪华,服务员小姐身着紧身旗袍,旗袍侧面的裂缝儿几乎一直裂到腋下。她们围站在餐桌四周,提供包括点烟在内的各种服务。就差往客人嘴里喂饭了。  
        进餐的人一个个吃得油光满面,说话时唾沫星子满桌飞。黎经理招呼男士们吃饱喝足,显然是她请客。  
        “她的时装公司如此不景气,她怎么还能这么花钱?”皮皮鲁自言自语。保守估计,这顿饭也得花两千元。  
        “肯定是要求这些人办事,她才不会白花钱。”贝塔说。  
        “女性在生意场上奋斗是悲剧。像黎经理这样,天天同各种男人周旋,万幸的是她长得不漂亮,否则还不知要搭上多少资本。”舒克说。  
        “我倒觉得女人好做生意。”贝塔说。  
        “那这黎经理怎么做不成?”舒克问。  
        “智商低。模样差。人品坏。”贝塔给黎经理经商不利总结了三条。  
        “贷款的事,还靠各位帮忙。”那黎经理端着酒杯站起来,“我敬各位一杯。”  
        “我们不能一起喝,要一个一个和你喝。”一个戴眼镜的瘦小男人醉眼朦胱的看着黎经理,实际上,他的目光投射在黎经理身旁的服务小姐身上的旗袍的断裂层上。  
        “我不会喝酒,”黎经理装孙子,“今天是破例。”  
        其实黎经理能喝酒。  
        黎经理又轮流和餐桌上的每一位男士干杯,那些男人被黎经理称为“科长”“处长”。  
        他们都是银行负责贷款的官员。  
        “我们可不缺吃,对吗?”瘦小男人一边用牙签肆无忌惮地剔牙一边对同事说。  
        “当然,哈哈,我们如果想吃,连早晨都排满了饭局。”另一位皮肤白得让人恶心的男士说。  
        “我知道,各位今天能来,实在是赏我脸,我给各位准备了一点小礼物。”黎经理从包里拿出几个大信封,递给每人一个。  
        “这是什么?”瘦小男人问。  
        “优惠券。购买本公司产品的优惠券。”黎经理诡秘地笑了笑。  
        瘦小男人将信封口的订书钉打开,用手一捏信封,信封口张开一道缝儿,他往里一看,笑了。  
        信封里是满满一沓钱。  
        “行贿?”皮皮鲁一愣。  
        “遥感看看有多少钱?”舒克对贝塔说。  
        贝塔遥感。  
        “每人一万元。”贝塔说。        
        “一万元!”皮皮鲁吓了一跳。  
        “他们真敢要!”舒克不大相信。  
        几位男士将钱收进自己的腰包。  
        皮皮鲁瞠目结舌。  
        “贷款拿到后,我还会谢各位的。”黎经理又举起了酒杯一饮而尽。  
        第二天,银行给时装公司贷了80万元,黎经理又给几位男士每人送了两万元。  
        接着,黎经理开始用这笔钱继续向各层次各方面与她的公司业务有关的官员行贿,官最大的一位是部长,还有两位副部长。  
        皮皮鲁和舒克、贝塔都看傻了。  
        “他们真敢要。”舒克说。  
        “相比之下,探长林真不错。”贝塔说。  
        “再往下看,看看是谁杀的黎经理。”皮皮鲁说。  
        那位受贿的部长的秘书接到一个电话,打电话的人自称是黎经理手下的人,她说她知道黎经理向部长行贿的事,她还说她随时可能举报此事。部长的秘书最后明白了,这位小姐十分恨黎经理,想以此事报复黎经理。  
        为了保护部长,也是为了保护自己,秘书制订了谋杀黎经理灭口的计划。  
        在一天晚上,正当黎经理在宾馆里洗澡之时,部长秘书来访。黎经理穿上浴衣和秘书交谈,喝了秘书下了药的饮料,昏睡后被秘书背到浴缸里淹死。  
        “我要和探长林通电话。”皮皮鲁说。  
        贝塔给皮皮鲁拨探长林的电话。  
        此刻探长林正驱车行驶,他决定暗中调查浴缸女尸案。  
        车载电话铃响了。  
        探长林拿起话筒。  
        “请问是探长林吗?”  
        “我是。”  
        “我是皮皮鲁。”  
        “……”探长林竟然没有反应过来。  
        “我是皮皮鲁。”  
        “你好.你在哪儿?”  
        “我有重要的事情告诉你。”  
        “请讲。”  
        “我们最好约个地方。”  
        “你说吧。”  
        “在舒克贝塔公司南边的那家大饭店的咖啡厅见面。一小时以后。”  
        “可以。”探长林同意。  
        皮皮鲁吩咐贝塔将刚才看过的黎经理贷款及被害的全过程给探长林复制一盘录像带。  
        “你现在一出门,就会被蹲守在门口的警察抓获。”舒克提醒皮皮鲁。          第270集  
        五角飞碟贴在汽车底盘下;  
        地下停车场里的皮皮鲁;  
        探长林看录像;  
        部长危在旦夕    
        “我先驾五角飞碟去,等到了那家饭店,再喝皮皮鲁口服液变大。”皮皮鲁有对策。  
        “我跟你去。”贝塔说。  
        “我也去。”舒克说,“最好把歌唱家和燕妮都带上,在五角飞碟里保险。”  
        皮皮鲁同意。  
        “贝塔,你准备起飞,我去叫她们。”皮皮鲁说完去鲁西西别墅叫燕妮和歌唱家。  
        正在游泳的燕妮和歌唱家听说要乘五角飞碟出去,十分兴奋。她们淋浴更衣后,和皮皮鲁一起走进五角飞碟。        
        “鲁西西在她的卧室里看书,我去告诉她一下。”舒克说。  
        鲁西西过来打开窗户,她叮嘱皮皮鲁和探长林见面时要小心。  
        “没关系,有五角飞碟当保镖。”贝塔从五角飞碟里探出头来,说。  
        五角飞碟舱门关闭,进入准备起飞状态。  
        舒克和贝塔坐在驾驶台前。燕妮和歌唱家在卧室里聊天。皮皮鲁坐在驾驶舱的皮沙发上设想和探长林见面时可能发生的意外情况。  
        五角飞碟抵达舒克贝塔公司南边的那家大饭店上空,贝塔不知在哪儿着陆。  
        “应该在饭店里边着陆,否则皮皮鲁变大后怎么进饭店?”舒克说。  
        “贴在一辆汽车的底盘上,和车一起进地下停车场。”皮皮鲁知道地下停车场有电梯可直达饭店大厅。  
        “就贴那辆车!”看见一辆车身很长的超豪华汽车正朝地下停车场人口驶去。  
        贝塔操纵五角飞碟神不知鬼不觉地飞进那辆汽车下边。五角飞碟上部紧贴着汽车的底盘,与汽车同步进入饭店的地下停车场,没人发觉。  
        汽车司机下车走了以后,皮皮鲁准备喝皮皮鲁口服液。        
        “你在五角飞碟里喝,然后马上走出去变大。”燕妮对皮皮鲁说。  
        “别吓着看车的老头。”贝塔说。  
        皮皮鲁喝了一瓶皮皮鲁口服液,拿上录像带和微型电视放像机,走出五角飞碟,来到汽车外边。  
        正好四周没人。  
        皮皮鲁变大了。这是他被爱因斯坦家的老鼠变小后第一次复原,他看着周围的一切都感到新鲜。  
        人生必须变化。一成不变的人生没有乐趣。  
        皮皮鲁乘电梯来到饭店大厅,这是一家五星级饭店,装饰豪华典雅。  
        咖啡厅坐落在一个大天井里,旁边是观景电梯和喷泉,还有一位小姐为客人弹奏钢琴助兴。  
        探长林已经坐在一张桌子旁边了。皮皮鲁通过五角飞碟通讯器问贝塔:  
        “贝塔,看看有没有埋伏?”  
        “没有,就他一人,你放心去吧。如果他敢和咱们捣乱,他就是这个世界上最不幸的人了。”贝塔说。  
        皮皮鲁缓步朝探长林走去。  
        探长林接到皮皮鲁的电话后很是兴奋,他暂时忘记了局里不让他插手黎经理溺死案的烦恼。他自从和皮皮鲁认识后,就对皮皮鲁有好感,他认定皮皮鲁是国宝。        
        探长林看见了皮皮鲁,他站起来招呼皮皮鲁落座。握手。  
        小姐过来问皮皮鲁喝什么。  
        “矿泉水。”皮皮鲁说。  
        “先说哪件事?”探长林问皮皮鲁说。  
        “先说你手中的这件案子。”皮皮鲁说。  
        “我手中的案子?什么案子?”探长林看着皮皮鲁问。  
        “黎经理浴缸溺死案。”皮皮鲁说。  
        “你怎么知道?”探长林警觉。  
        “你知道为什么你的头儿不让你再管这件案子了?”皮皮鲁说。  
        “为什么?”探长林和别人交谈头一次处于下风。  
        “这是一起谋杀案,凶手是一位部长的秘书。”皮皮鲁说。  
        “部长?!”探长林吃了一惊。  
        皮皮鲁说出了那位部长的名字,这是一个经常出现在电视报纸上的名字。  
        “你有什么证据?”探长林问。  
        皮皮鲁拿出微型电视放像机。  
        “这是什么?”探长林没见过。  
        皮皮鲁打开机盖,露出小屏幕。皮皮鲁按了几个按钮,屏幕上开始出现黎经理案的全过程。  
        探长林眼睛一下都没眨,看完了全部录像。        
        “这是怎么回事?”探长林问。他不明白皮皮鲁怎么能偷拍到这种镜头。其中还有部长秘书和部长老婆的一段不堪入目的镜头。  
        “这录像是怎么拍的,目前还不能告诉你。如果你需要,我可以给你提供所有案件的录像。当然是犯罪全过程的。”皮皮鲁说。  
        探长林早就感觉到皮皮鲁手中有高科技器物,这从糕鱼氏的案件就能看出。探长林知道既然皮皮鲁不愿意,自己也就别问了。  
        皮皮鲁从微型电视放像机里取出录像带,交给探长林。  
        “送给你了,有这个证据.你就可以破这个案子了。对了,你准备怎么对待那位部长?”皮皮鲁问。  
        “送他见阎王。”探长林斩钉截铁地说。  
        皮皮鲁佩服探长林。  
        “说说我的事吧?”皮皮鲁说。  
        “到底是怎么回事?”探长林问。  
        “我在德国有个朋友……”皮皮鲁简要地叙述他没有说歌唱家的体积。  
        “她叫贝一?”探长林问。  
        “……就算吧。”皮皮鲁说。          第271集  
        皮皮鲁忘了带微缩药;  
        部长戴上了手铐;  
        天眼电视台大受欢迎;  
        局长钓完鱼就进监狱    
        皮皮鲁把能告诉探长林的事都告诉他了,对五角飞碟只字不提。  
        “我马上解除对你的监视。希望我们今后常联系,遇到难题,还要请你帮助。”探长林说。  
        “为民除害,责无旁贷。”皮皮鲁站起来和探长林握手道别。  
        皮皮鲁回到地下停车场,他找到那辆汽车,准备用微缩药变小后乘坐五角飞碟回家。  
        贝塔从五角飞碟里探出头来,他告诉皮皮鲁忘了带微缩药。  
        “我坐出租车回家,你们先回去吧。”皮皮鲁说。        
        “我们给你当保镖。”贝塔说。  
        皮皮鲁来到饭店大厅门口,招手叫了一辆出租车。皮皮鲁已经很长时间没在大街上行走了,他感到新鲜。  
        出租车将皮皮鲁送到住处,埋伏在楼下的探长林的部下见到了皮皮鲁大为兴奋,他拿起车载电话向探长林报告。  
        探长林刚刚走进局长的办公室。  
        “我知道了,你可以撤离了。”探长林说。  
        “撤离?”部下不明白发现了盼望已久的目标,为什么要撤离。  
        “撤离。”探长林重复了一遍。  
        “是。”部下说。  
        坐在办公桌后边的局长看着探长林,他好像知道探长林的来意。  
        “黎经理浴缸溺死案,我已经破了。”探长林平静地告诉顶头上司。  
        局长原以为探长林是来要求继续侦破“黎溺案”,没想到案子已经被他破了。  
        局长原本被上边的某大人物关照不要继续侦破此案,局长心中虽然不悦,但也无奈,只得下令让探长林离开此案,其实心中委实不快。现在听说案子已破了,脸上明显出现笑容。  
        “怎么回事?”局长问。        
        探长林将谋杀经过讲述一番。  
        一听涉及部长要员,局长忙问:  
        “证据呢?”  
        探长林递上录像带。  
        “都在这上边。”探长林说。  
        局长叫来技术人员:  
        “立即放给我看。”  
        技术人员迅速将微型录像带转录到普通录像带上,然后将普通录像带(禁止)局长办公室的录像机。  
        荧光屏上像演电视剧一样展示了此案犯罪的全过程。局长看罢怒火中烧。  
        “你是怎么拍的?”局长问探长林。  
        ‘我早就发现他们贪赃枉法,一直派人监视。”探长林不得不撒谎。  
        局长对探长林大加赞赏。  
        “通知检察长,请他立即来我这儿。”局长对秘书说。  
        探长林产生了一个大想法,他想请皮皮鲁利用先进手段把所有正在利用职权舞弊的各级官儿都查出来。  
        检察长看了录像后,立即签署了逮捕部长和银行职员的逮捕令。  
        当部长被戴上手铐时,他突然意识到他的职位离鲜花和手铐都很近。        
        皮皮鲁从电视上看到了法院对此案人犯的判决,舒克破天荒喝了一杯酒。  
        电话铃响了,鲁西两拿起话筒。  
        “请找皮皮鲁先生,我是探长林。”  
        “请稍等。”鲁西西让皮皮鲁接电话。  
        “我想请你帮个忙。”探长林说。  
        “请讲。”皮皮鲁说。  
        “能不能把所有利用职权犯法的官都查出来?”探长林问。  
        皮皮鲁想起了贝塔酗酒后干的那件事。  
        “我可以全天候监视所有科级以上官,每人都录像,然后将这些录像带送到电视台,请电视台设个专栏,天天播放,让老百姓监督。”皮皮鲁说。  
        “太棒了!”探长林大喊。  
        皮皮鲁和探长林达成协议,由皮皮鲁负责监视和录像,由探长林拿着第一批录像带去电视台交涉设立栏目和播出事宜。  
        “栏目就叫天跟。”舒克说。  
        贝塔特兴奋,他自告奋勇要搬到五角飞碟去住,专门负责监视全国所有的科级以上官的24小时活动,谁贪赃枉法就曝谁的光。  
        探长林拿着第一批录像带到了电视台。  
        台长听秘书说神探见他,以为自己犯了什么事,忐忑不安地和探长林见了面。        
        探长林将一摞录像带放在台长的桌子上。他说明来意。  
        台长悬着的一颗心放下了,他只能同意。  
        电视台从即日起开设新栏目《天眼》,每天播出1小时。  
        《天眼》播出后,大受观众欢迎,以至于观众强烈要求电视台全天24小时只播出这个节日,后来电视台不得不专辟一频道全天播出,该频道被称为天眼电视台。  
        全国科以上的官都24小时处于五角飞碟监视之中,他们只能为老百姓做事,不能坑害老百姓。自从天眼电视台开播以后,全国有不计其数的官辞职,他们无法忍受不干坏事的当官生涯。  
        有一位处长利用职权到一家餐厅吃饭,吃完饭不但不给钱还接受了人家送的一条香烟,该处长当晚即被天眼电视台曝光,可怜那处长第二天即被送上法庭。法庭查实该处长此餐共消费384元,定了受贿罪,被判处有期徒刑13年。  
        还有一位局长乘坐公车去钓鱼,也被判了7年刑,他不服上诉。高级法院驳回了他的上诉,理由是,身为公务员,不为老百姓谋福利,反而使用纳税人的汽车和汽油为自己娱乐提供交通工具,实属罪大恶极民愤极大,维持原判。          第272集  
        嫉妒成性的副科长;  
        红沙发音乐城在农村;  
        鲁西西感慨万千;  
        皮皮鲁和燕妮留守大本营    
        自从有了复原药,鲁西西可以随心所欲地变大变小。去公司主持业务,她就变大。回到家里,她喜欢和朋友们住在鲁西西别墅里,一进家门,她就变小。  
        这天吃完晚饭,大家趴在别墅的窗户上看电视。电视机放在别墅的旁边,像一座城堡。  
        电视节目是歌手大奖赛。打扮和气质都俗得不能再俗的选手们一个个粉墨登场。  
        “歌太难听。”燕妮说,“没有才气的作曲家是在给地球制造噪音。”  
        “嗓子也不行,没有特点。”舒克评价。        
        “咱们应该赶紧推出歌唱家的录音带。”鲁西西说,“让听众享受真正的歌。”  
        “去找红沙发音乐城。红沙发音乐城和歌唱家联手,恐怕世界上所有歌星和作曲家都得失业了。”皮皮鲁说。  
        “我去找贝塔,咱们策划一下。”舒克说。  
        贝塔还坚持不懈地坚守在五角飞碟里给所有科级以上官录像,而且兴趣与日俱增。  
        舒克走进五角飞碟,看到贝塔的两条腿翘到驾驶台上,眼睛盯着荧光屏,嘴里嚼着口香糖,只是脸上的表情不太好看。  
        “你怎么了?”。  
        “这人也太损了。”贝塔措着荧光屏说。  
        “他是谁?”舒克看到屏幕上有一个獐头鼠目的人坐在办公桌前。  
        “一个小城市邮局报刊科的副科长,他每个月都把发行量大的刊物扣住不发,起码压一个月。”贝塔说。  
        “为什么?”舒克不知道这样做能给那副科长带来什么好处。  
        “这是一个嫉妒心非常强的人,他嫉妒世界上所有比他强的人。他原先喜欢写作,但投稿从未被采纳。凡是他投过稿的刊物,他都利用职权压住不发。”贝塔说。        
        “他这么活也太累了。”舒克摇摇头,“心理太阴暗。人应该善待自己。善待自己的最好办法是善待别人。”  
        “不善待别人就是虐待自己。”贝塔说,“你看他这样活并不享受,心中充满怨恨。时间长了,准得绝症。”  
        舒克将寻找红沙发音乐城的事告诉贝塔,让他去鲁西西别墅共商计划。  
        “这位副科长自以为聪明,他不知道冥冥之中有架摄像机正对着他。这算渎职罪吧?”贝塔问舒克。  
        “起码判5年刑。”舒克说。  
        贝塔站起来和舒克离开五角飞碟,走进鲁西西别墅。  
        “红沙发音乐城所在的那个红沙发恐怕已经不在了,这是30多年前的事了。”皮皮鲁说。  
        “红沙发音乐城会移居到别的地方吧?”鲁西西猜想。  
        “用五角飞碟测一下。”贝塔说。  
        “走。”皮皮鲁站起来。  
        大家来到五角飞碟里,皮皮鲁操纵遥感器寻找红沙发音乐城的方位。  
        鲁西西格外激动,她似乎回到了童年。  
        歌唱家全神贯注地注视着屏幕,她渴望同红沙发音乐城合作。        
        荧光屏上轮番出现各种图形和线条,皮皮鲁转动旋钮寻找目标。  
        屏幕上出现了一个亮点。  
        “有信号!红沙发音乐城还在!”皮皮鲁改用微调寻找红沙发音乐城。  
        遥感电脑测出红沙发音乐城在一个山区农村里,那地方比较穷。  
        “大概是被收购旧家具的给卖到农村去了。”皮皮鲁说。  
        “咱们出发吧!”贝塔急不可待地说,他几乎没去过农村。  
        “不必都去吧?家里也要留人。”皮皮鲁说。  
        “我要去找红沙发音乐城。”鲁西西说。  
        “那我只好留守。”皮皮鲁耸耸肩。反正现在探长林也不找他的麻烦了。  
        “我和你留在家里。”尽管燕妮很想去农村看看,可她更想和皮皮鲁在一起。  
        贝塔、歌唱家、舒克和鲁西西做出发前的准备工作。他们要驾驶五角飞碟去农村找红沙发音乐城。  
        贝塔将五角飞碟外表擦得锃亮。皮皮鲁和舒克检查电路系统。  
        “祝你们顺利。”皮皮鲁对鲁西西和朋友们说。  
        鲁西西最后一个登上五角飞碟,她转身向皮皮鲁和燕妮挥手:“你们等好消息吧。”        
        五角飞碟的舱门关上了。  
        皮皮鲁拿起通讯器。  
        “我们起飞了?”舒克请示皮皮鲁。  
        “起飞。”皮皮鲁说。  
        五角飞碟升到空中,稍停片刻后,飞出了窗户。燕妮依偎在皮皮鲁肩头注视着远去的五角飞碟。  
        鲁西西还是头一次乘坐五角飞碟,她对皮皮鲁的这个杰作佩服之至,她想到皮皮鲁上小学时考试经常不及格,而皮皮鲁长大后竟然能造出如此超现代化的飞行器。可见,少年时代的学习成绩同长大后有否出息没有任何内在关系。  
        “你在想什么?”歌唱家看见鲁西西站在舒克身后发愣,问。  
        鲁西西说了自己的感受。  
        “从某种意义上说,学校是把听得懂的话往听不懂了说,把简单的事往复杂了说。”歌唱家说。  
        “几十个不同性格不同智商不同脑沟回的孩子坐在同一间屋子里听同一个老师讲课,这种教授方法太童话了。依我看,认了字的孩子就完全可以自学了。自学是最好的学习方法。独立性,想像力,凭兴趣选择。有百利而无一弊。”舒克说。          第273集  
        五角飞碟在石头旁着陆;  
        寸草不生的山:  
        草房前边的缸;  
        总指挥不信鲁西西能进来    
        “不会自学的人是真正的文盲。”贝塔边开五角飞碟边说。  
        “真正的天才无法忍受课本的束缚。”鲁西西在五角飞碟里说这句话显得底气特足。  
        “接近目标了,你们看,信号越来越强。”舒克指着荧光屏对鲁西西和贝塔说。  
        歌唱家闻声从餐厅跑过来看。  
        “准备着陆。”贝塔打开监视器,寻找安全的着陆地点。  
        五角飞碟飞临山区上空。这里的山几乎寸草不生,一片贫瘠荒凉的景象。几间矮小的茅屋坐落在山角下,给寂寞的山增添了一点儿生机。  
        五角飞碟降落在一块大石头侧面。  
        “舒克,你在五角飞碟里值班,我们去找红沙发音乐城。”贝塔一边说一边解安全带。  
        “最好你和歌唱家值班,我和鲁西西去。”舒克不愿意值班,他想去找红沙发音乐城。  
        “去找歌唱家时,就是我在五角飞碟里,你忘了?在德国机场,我还和五角飞碟一起被扣在海关了。”贝塔理由充足。  
        舒克瞪了贝塔一眼。  
        贝塔、歌唱家和鲁西西走出五角飞碟,舒克用遥感器观察他们的行踪。  
        眼前的景象令鲁西西他们吃惊,原来地球上还有这么穷的地方。这里没有公路没有电线没有汽车没有高楼大厦,只有石头只有土只有新鲜的空气。  
        “红沙发音乐城会在这儿?”歌唱家表示怀疑。  
        “还有这么穷的地方。都是人,差别太大了。”贝塔感慨万千。  
        “这儿的人未必比城里人痛苦。空气起码就比城里新鲜。”鲁西西说。  
        “咱们先到那座房子里看看。”贝塔指着离他们最近的一座草房子说。  
        他们小心翼翼地接近那座房子,当他们走到房子跟前时,他们惊呆了,这哪是房子,只能说是一个挡风的棚子。  
        房子前边有一口缸,其他就一无所有了。  
        “你们在这儿等着,我进屋里去看看。”贝塔担心有危险,他说。  
        “一起去。”鲁西西想起了小时候和皮皮鲁到309暗室探险的经历,她愿意冒险。  
        歌唱家也表示坚决同去。  
        贝塔只好同意,他说:  
        “听我的指挥,别出声。”  
        “我连骷髅城都去过。”鲁西西说。  
        贝塔撇撇嘴。  
        草棚子里很黑,没有窗户。贝塔的视力适应黑暗,他看见屋里有一个十几岁的男孩子,男孩子衣衫褴褛,躺在一张床上。鲁西西和歌唱家的眼睛进入由亮转暗的地方,一时什么都看不清。  
        当鲁西西的眼睛适应了暗光以后,她认出了那男孩子身下躺的就是红沙发,是红沙发音乐城居住的红沙发。  
        尽管已经是30多年前的事了,可鲁西西仍能一眼辨别出红沙发,它给她留下的印象太深刻了。  
        鲁西西激动地把自己的发现告诉朋友们。  
        “红沙发怎么到这么穷的地方来的?”歌唱家自言自语地说。  
        “可能是被收购旧家具的人卖到这来了。”贝塔分析说。  
        “这男孩儿大白天的干吗躺着?”歌唱家觉得有点儿不对头。  
        “咱们先想办法进到红沙发里边去。”鲁西西说。  
        贝塔打头,歌唱家在中间,鲁西西在最后,他们开始接近红沙发。  
        贝塔在红沙发底部找到了一道裂缝儿,他用匕首将裂缝儿拓宽。  
        “我先进去,如果没危险,就出来叫你们。”贝塔小声对歌唱家和鲁西西说。  
        鲁西西和歌唱家点点头。  
        贝塔钻进红沙发。  
        果然是红沙发音乐城!贝塔的出现,给红沙发音乐城的居民造成了惊恐,红沙发音乐城总指挥出现在贝塔面前,他显然是在紧张之中。  
        “我们这儿没有食物,对不起,老鼠先生。”总指挥对贝塔说。  
        “这儿是红沙发音乐城吗?”贝塔问。  
        总指挥犹豫了一下,然后点点头。  
        “鲁西西来找你们了。”贝塔说。  
        “鲁西西?”总指挥一愣。  
        “皮皮鲁的妹妹。想不起来了?”贝塔问。  
        “她在哪儿?”总指挥激动地问。  
        “我去叫。”贝塔转身要走。        
        “慢着,你是说,你叫鲁西西进来?”总指挥断定面前这只老鼠是骗子,鲁西西怎么可能进到红沙发里边来”  
        “鲁西西现在想变大就变大,想变小就变小。”贝塔说完跑出去。  
        当鲁西西和歌唱家出现在总指挥面前时,总指挥已经很难认出鲁西西,况且30年前他们并没有见过面,只是互相熟悉声音。从鲁西西的声音中,总指挥找出了30年前的痕迹。  
        鲁西西把歌唱家介绍给总指挥。总指挥把音乐城的大腕作曲家一一介绍给鲁西西和她的朋友们。  
        “跟我们走吧。你们和歌唱家合作,保准能创作出一流的音乐作品。”鲁西西说。  
        “我们不能走。”总指挥说。  
        “为什么?”贝塔问,“这地方这么穷,有什么可留恋的?”  
        总指挥告诉鲁西西和朋友们,躺在红沙发上的这个男孩子从小就不会走路。这地方穷极了,一年几乎不下雨,水比油还贵。  
        每家门前放一个缸,接雨水。水里长了鱼虫,还是宝贝,还要喝。庄稼根本无法生存。  
        鲁西西、歌唱家和贝塔从未意识到水的珍贵。          第274集  
        有一个叫牛的穷孩子;  
        皮皮鲁批准五角飞碟打井;  
        黑色的液体喷了五角飞碟一身;  
        歌唱家提起海湾战争    
        红沙发音乐城的总指挥告诉鲁西西,躺在红沙发上的男孩子叫牛,牛的爸爸是农民,当他发现自己的儿子不会走路后,并没有抛弃儿子,而是更尽心地抚养他。  
        牛的爸爸很穷,一年四季几乎只有两身衣服,他靠卖血买回了旧的红沙发,给儿子睡。红沙发音乐城为了让牛得到一点儿享受,就给他演奏音乐。现在,听音乐已经成了牛的惟一乐趣。  
        “我们准备用音乐治好他的病,我们专门针对他的病谱了曲,一共20个乐章,也可以说是20个疗程,现在已经进行到第7个疗程了,所以我们不能走。”总指挥说。  
        鲁西西感动了,她支持红沙发音乐城继续给牛治病。  
        “不过,这也不影响我们为音乐家伴奏,如果你们有录音机,录下来就行了。”总指挥说。  
        “能录。”贝塔说。五角飞碟上的舒克可以遥控录音。  
        “他们为什么不打井?”鲁西西问。  
        “打,他们世世代代打井,有了点儿钱就打井,但从没打出过水。”总指挥说。  
        “我们试着帮他们打口井。”贝塔说。他认为五角飞碟有这个能力。  
        “你们?这么小能打井?”总指挥显然不信。  
        “我去五角飞碟和舒克商量一下。”贝塔小声对鲁西西和歌唱家说。  
        鲁西西和歌唱家赞成贝塔打井的主意。  
        贝塔回到五角飞碟上。  
        “我都知道了。”舒克一直在遥感贝塔他们刚才的经历。  
        “你说行吗?”贝塔问。  
        “我觉得五角飞碟打口井没什么问题。他们原来打井,打不了很深。咱们打,应该能打到水。”舒克说。  
        “和皮皮鲁商量商量。”贝塔说。        
        舒克呼叫皮皮鲁。  
        “什么事?”皮皮鲁在鲁西西别墅里回答。  
        “我们找到了红沙发音乐城。”贝塔说。  
        “太好了,什么时候回来?”  
        “这地方很穷。严重缺水。红沙发音乐城不想走。”  
        “为什么?”  
        “有一位叫牛的男孩子从生下来就不会走路,现在他就躺在红沙发上,音乐城每天给他演奏音乐,为他提供娱乐。”  
        “……”  
        “红沙发音乐城现在通过音乐给牛治病,如果他们走了,牛就只有一辈子躺着了。”  
        “那不能走。”  
        “对。我们想用五角飞碟给这儿打口水井,行吗?”贝塔请示皮皮鲁。  
        “当然可以。你们打吧,一定打出水来。”皮皮鲁同意。  
        “红沙发音乐城的总指挥说,他们可以给歌唱家伴奏,我们给录下来。”贝塔说。  
        “太好了。”皮皮鲁说。  
        贝塔准备去告诉鲁西西她们。  
        “这地方穷,大慨是因为没水,我觉得住在这儿的人应该搬家,也就是移民。”舒克说,“也许越穷越不愿意离开。”  
        “地球上所有的地方都是宝地,就像没有优点的人不存在一样。”贝塔这样认为。  
        所有人都是天才,只不过绝大多数人找不到自己的最佳才能或者没有机会。所有地方都是富饶的土地,只不过有很多地方没有发现自己的富饶所在。  
        贝塔回到红沙发里。  
        “皮皮鲁同意用五角飞碟在这儿打一口井。”贝塔对鲁西西和歌唱家说。  
        “用什么打井?”总指挥问。  
        “极为先进的东西。”贝塔说。  
        “我能告诉牛吗?”总指挥问。  
        “可以。”鲁西西愿意让牛高兴。  
        总指挥和牛联络有规定的信号。  
        “牛,告诉你一个好消息。”总指挥说。  
        “又有新曲子了?”牛问。  
        “不是,比这还好的消息。我的几个朋友要帮你打一口水井。”  
        “你的朋友?他们在哪儿?”  
        “就在我身边。”  
        “和你一样大的朋友?”  
        “对。”  
        “他们要帮我打井?谢谢。”牛显然不相信这么小的人能打井,但他感谢总指挥的好意。总指挥是为了让他高兴。  
        “是真的。你选个地方吧。”  
        “就打在我的门口,我能看见的地方。”牛说。他愿意想像有水的场面。  
        贝塔回到五角飞碟上。  
        舒克驾驶五角飞碟升到空中,五角飞碟悬停在牛家的门口上方。  
        五角飞碟准备垂直往下打井,用激光。  
        鲁西西和歌唱家还有总指挥和牛见了面,当牛看到五角飞碟时,他信了。  
        贝塔按下了激光发射按钮。  
        一道纤细的光穿透了地面。  
        土地变得发烫,连红沙发都热了。  
        地上出现了碗口大的洞。  
        牛瞪大了眼睛,他从生下来就没离开过这间屋子。现在,他突然开阔了眼界。  
        有时候,见得早不如见得晚印象深刻。  
        早期智力开发其实是扼杀人才。毁就毁在一个“早”上,事物都有自己的规律,早不得,也晚不得。恰到好处最好。  
        突然,从洞里喷射出了液体,足足有十几米高,喷了五角飞碟一身。  
        大家刚要欢呼,都愣了,那液体是黑色的,不是水。        
        舒克操纵满身是黑色液体的五角飞碟躲开喷射的井口。  
        “这是什么?”贝塔吃惊地问。          第275集  
        鲁西西从黑色液体中看到了汽车;  
        舒克和贝塔纵论人类前途;  
        爸爸妈妈把孩子培养成动物;  
        贝塔撰写歌词   
        舒克用五角飞碟测那黑色的液体。  
        “石油!是石油!”舒克大叫。  
        石油!  
        鲁西西和歌唱家知道石油的价值。  
        “这地方马上就会富甲天下!”鲁西西兴奋地对牛和总指挥说。  
        “真的?!”牛半信半疑。  
        “人为了得到石油会动刀动枪,你们没听说过海湾战争?”歌唱家说。  
        牛的脸上掠过一丝阴影,但很快就消失了,他愿意让自己的家乡富起来,不管付出什么代价。        
        “应该尽快通知外界。”鲁西西说。  
        石油继续喷射着。鲁西西和歌唱家从黑色的原油中看见了汽车、公路、经济……  
        “又有几万人能就业了。”舒克从空中俯视石油喷薄的场面,有感触地说。他知道,人类中的不少人没有工作。  
        “人人有工作的国家,准是最贫穷的国家。”贝塔说。  
        “这话怎么讲?”舒克不明白。  
        “失业的人越多,证明这个国家越发达。就像通货嘭胀等于零的国家,经济增长肯定也等于零。”贝塔尽是歪理。  
        “为什么?”舒克问。  
        “你想啊,计算机越来越普及,原来需要十个人干的工作,现在只用一个人就行了,那九个人不就没工作了吗?科技越发达,失业的人越多。”贝塔振振有词。  
        舒克不得不点头。  
        “依我看,人类社会的发展趋势,随着科学技术的发达,必将有越来越多的人没有工作,只有少数高智商高才能的人工作。”贝塔继续发挥。  
        “那些没工作的人怎么活?”舒克问。  
        “照样活得好,社会福利保障他们的生存。那些有工作的人创造的价值交税,国家靠有工作的人的税养活没有工作的人。”贝塔一边观察喷石油的井一边说。  
        “我觉得人如果不工作不可能活得好,天天山珍海味吃着没事干,这不和动物园里的动物一样了吗?”舒克为人类担忧。  
        “你说得特对,这世界发展下去,就是少数有工作的人把大多数没工作的人当动物养。所以根本不用担心地球上动物灭绝,动物只会越来越多,这动物就是没工作的人类成员。”贝塔大放厥词。  
        舒克居然会为贝塔的话陷人沉思。人类之所以是人类,就是要工作、要创造。而科学技术越发展,劳动力之需要量就越少,没工作的人就越多。没事干的人不能算是人。  
        “这就是说,人类的表面数量虽然在增长,但实际数量却在下降。动物的表面数量虽然在下降,但实际数量却在增长。”舒克大彻大悟。  
        “科学技术是个魔鬼,它既帮助了人类,也毁了人类。”贝塔说。  
        “这都是人的大脑思维的结果。所有科学技术都是人的脑子想出来的。人的大脑能创造世界,也能毁灭世界。”舒克说。  
        “今后人类必将两极分化,脑子聪明,就当人。脑子不聪明,就当动物。”贝塔继续胡说八道,“别看现在有的父母为自己的孩子考试得100分而沾沾自喜,没用。考试成绩好并不能证明孩子脑子聪明,考试充其量考的是记忆力,而脑子聪明的标志是有丰富的想像力,考试能考出想像力来吗?”  
        “挺可怕,这些家长在玩命抓孩子的分数时,把孩子培养成了动物。只有记忆力的人长大后是找不到工作的,因为电脑比人脑记忆力强多了。”舒克说。  
        “人类越发展,越不需要记忆力,只需要想像力创造力。父母要想让自己的孩子成为人而不是动物,就要从小培养孩子的想像力和创造力,有创造力的人的一个突出特点就是不循规蹈矩,而考试却是培养人的循规蹈矩,培养人的不敢越雷池一步。”贝塔说。  
        “有想像力和创造力的电脑不会诞生?”舒克自问又问贝塔。  
        “一定会。因为人类太聪明了。”贝塔断言人类终将造出会想像和创造的计算机。  
        “那人类还子什么?不是都该没工作了吗?”舒克说。  
        “那就是人类的末日了,科学技术发达到最后,人类将集体失业,集体变成只会吃喝的动物。”贝塔叹了口气。  
        舒克也跟着叹了一口气。  
        两只老鼠在五角飞碟里,由于找出了石油,在空中以旁观者的身分讨论着人类的现在和未来。  
        “舒克!贝塔!怎么样?打出水了吗?”话筒里传出皮皮鲁的呼叫。  
        “皮皮鲁,你猜我们打出了什么?”贝塔说。  
        “果汁。”皮皮鲁幽默。  
        “石油!”舒克说,“我们打出了石油!”  
        “太棒了!那儿马上就会富了。”皮皮鲁为那个不曾发达的贫穷地方高兴。他愿意天下所有地方都富裕。  
        石油是上帝藏在地下的钱。  
        “你们应该马上通知外界。”皮皮鲁说。  
        “我们想先给歌唱家和音乐城录音,录完音再通知,要不然这儿马上就人山人海了。”舒克说。  
        “不会那么快吧?”皮皮鲁说。  
        “只要他们信了,马上就会开直升机来。”贝塔断定。  
        “好吧。”皮皮鲁同意了。  
        五角飞碟降落在牛家的房顶上,贝塔钻出五角飞碟时蹭了一身石油。  
        红沙发音乐城的总指挥让几位作曲家为歌唱家谱写歌曲,贝塔自告奋勇写歌词。          第276集  
        上帝的歌声;  
        贝塔和鲁西西关于书的对话;  
        卡拉OK被贝塔推上断头台;  
        男人为什么跳舞    
        歌唱家看了由贝塔作词、红沙发音乐城作曲家作曲的歌,立即大放异彩,她从未见过如此精妙绝伦的歌。  
        贝塔的歌词入木三分,每个标点符号都是人生哲理。红沙发音乐城谱的曲更是巧夺天工史无前例。  
        “咱们排练一次。”总指挥对歌唱家说。  
        歌唱家又看了一遍曲子和歌词。  
        乐队做准备。  
        总指挥拿起了指挥棒,他看歌唱家。  
        歌唱家冲他点点头。  
        总指挥举起了指挥棒。乐队每一位成员的乐器都进入准备演奏状态。  
        贝塔和鲁西西屏住呼吸在一旁观看。舒克在五角飞碟里遥感现场。  
        音乐先行,引出歌唱家的歌声。  
        “这不是歌,是上帝的声音。”鲁西西发自肺腑地说,她从没听过这么美妙的歌。  
        贝塔听傻了。舒克在五角飞碟里听呆了。  
        当总指挥的指挥棒在空中为这支歌划句号的时候,鲁西西和贝塔连鼓掌都不会了。他们呆呆地站在那里,两眼发直。  
        “怎么了?不好?”歌唱家问。  
        “什么?”贝塔回过神来,“太棒了!”  
        “此曲只应天上有!”鲁西西说。  
        “曲子好,嗓子好!”贝塔说。  
        “歌词特别好。”歌唱家说。贝塔不懂写作。他写的歌词全是大白话,但没有一句废话,全是对生命的真实感受。  
        “贝塔,我看你很少看书,有时几个月才看一页,你怎么能写出这么好的歌词?”鲁西西认为只有看书多的人才能写出好作品。  
        “看书不在多少,关键看你把书当什么。”贝塔说,“你发现没有,皮皮鲁和我一样,看书并不多。我和他探讨过这个问题。”  
        鲁西西对这场对话感兴趣了,她等着贝塔继续说下去。  
        歌唱家和红沙发音乐城的音乐家们也被鲁西西和贝塔的谈话吸引了,大家心中历来都有一个谜,为什么同样的书,有人看了大受启迪,有人看了却一无所获。  
        见大家都在注意听,贝塔的口才进人最佳竞技状态。  
        “经常听人把书说成是精神食粮,如果看书时将书当成粮食,那就糟了。”贝塔做了个万分惋惜的动作。  
        “为什么?”鲁西西也一直把书比作精神食粮。  
        “如果把书当成粮食,大脑就成了肠胃,看完一本书,大脑把书变成了什么?”贝塔问鲁西西。  
        “什么?”鲁西西反问。  
        “粪便。”贝塔说,“这种人看书,就是拼命吃,拼命拉。”  
        “你把书当成什么?”鲁西西问。  
        “我把书当火柴。”贝塔神采飞扬,“我的大脑是一座燃料库.每一页书都能点燃我的思维火花,燃起冲天大火。我和皮皮鲁聊过这种感受,我们每看一页书,有时甚至只有几行字,大脑燃料库就被点燃了,止都止不住。这时,你说我们还能继续看吗?再看脑子还不爆炸了?所以,像我和皮皮鲁这种人,表面看,看书很少,实际上,我们看的一页书,比那些把书当粮食吃的人看的100本书都管用。”  
        鲁西西想起往常经常看到皮皮鲁捧着书半天也不翻一页,那大概就是贝塔说的,那页书作为火柴点燃了他的思维火花。  
        “书是火种,不是粮食。”贝塔总结性地说。  
        鲁西西顿悟。歌唱家鼓掌。  
        “对于我们这些一脑子燃料的人来说,不光书是火种,生活中的每件事都足火种。靠火种才能点燃思维火花的人,还属于初级阶段,像我和皮皮鲁,大脑里已经有了电打火,开关就在眼皮上,只要眼睛一睁开,电打火就点燃了大脑燃料库,思想的大火就会熊熊燃烧……”  
        “得了得了,别吹了。”歌唱家表面上制止贝塔,实则是夸奖。歌唱家认为,不管贝塔大脑里是否真的有电打火自燃装置,能说出刚才那番话的,就绝对是天才。  
        鲁西西对贝塔的话服气,因为他写的歌词摆在那儿。当过10次研究生的人也写不出那样的歌词。  
        “确实有很多学校把学生的大脑培养成了肠胃,它们真应该把学生的大脑培养成燃料库。”鲁西西感慨。  
        “咱们录音吧?”贝塔俨然以导师的口气说话,神气十足。  
        舒克在五角飞碟里做好了准备,五角飞碟遥感录音。  
        总指挥举起了指挥棒。  
        只应天上有的歌声再次出现。  
        舒克录音。  
        鲁西西和贝塔再次陶醉得死去活来。  
        录音完毕,舒克听了一遍。  
        “怎么样?我是说录音效果?”贝塔通过通讯器问舒克。  
        “很好,如果制成卡拉0K伴奏带,咱们准发大财。”舒克说。  
        “我最讨厌卡拉0K。”贝塔说。  
        “为什么?”鲁西西问。  
        “卡拉OK本质上是给那些在事业上一事无成的人一个成功的幻觉,使那些从未成为被注意的中心的人在有限的空间和有限的时间里成为被注意的中心,过一过成功瘾,不信你用五角飞碟遥感,真正成功的人没有喜欢卡拉OK的。有个资料显示,在喜欢跳舞的男人中,有百分之八十五的人需要去看某科医生。正因为他们不行,所以才要换一种方式证明自己是男人。这和唱卡拉0K异曲同工。”叭塔滔滔不绝。看来他的大脑这次着火着得挺大。   第277集  
        摄像机插进油井;  
        长得像海棠花的妇女;  
        皮皮鲁和燕妮打赌;  
        五角飞碟归途漫漫   
        “咱们该通知政府了吧?”舒克在五角飞碟里问贝塔和鲁西西。  
        鲁西西愿意让牛的家乡马上富起来,她对总指挥说:  
        “我们现在就通知政府这儿发现了油田,牛家立刻就会富的。”  
        “太感谢你们了。”总指挥说。  
        鲁西西、歌唱家和贝塔同红沙发音乐城的音乐家们告别,他们走出红沙发音乐城后又同牛告别。  
        “你的歌声真好听。”牛对歌唱家说。  
        “希望你早日站起来。”歌唱家说。        
        “谢谢你们。”牛含着热泪说。  
        “说不定你以后要骂我们。”贝塔说。  
        “为什么”牛瞪大丁眼睛。  
        “有钱了就要受罪了,真正的高档次的痛苦都是有钱以后才产生的。”贝塔说。“不受罪不可能有钱,有钱后必然要受罪。”  
        “行了行了,你别瞎说了。”歌唱家制止贝塔。  
        牛似懂非懂地琢磨贝塔的话。  
        贝塔、歌唱家和鲁西西登上五角飞碟。舒克接通了政府有关部门的电话。  
        “我们在××省××县发现了大油田。”舒克说。  
        “你们?你们是谁?”  
        “勘探队。”舒克蒙他。  
        “哪支勘探队?”  
        “你们快派人来吧!”  
        “别瞎逗了,你要是恶作剧呢?”那人把电话挂  
        “给电视台打电话。”贝塔的大脑再次燃烧。  
        舒克给该省的电视台打电话。  
        电视台正愁没新闻,二话没说,来了。  
        五角飞碟藏在牛家的房顶上。  
        扛着摄像机的记者一看喷薄而出的石油就兴奋了,差点儿把摄像机插进油井里。  
        石油部的大队人马闻讯赶来了,天上是直升机,地上是成群的汽车,修路的来了,盖房子的来了,派出所成立了,职工子弟学校开建了,计划生育委员会成立了……  
        “我看那女人有点儿面熟。”鲁西西指着荧光屏上一位五十多岁的妇女说。  
        “我没见过。”贝塔摇头。  
        “想起来了,她叫海棠花!”鲁西西一拍脑袋。(海棠花是《龙珠风波》里的人物,请参阅《鲁西西传》)  
        “居然碰到熟人了。”贝塔觉得好玩。  
        “这名字有点像上个世纪的。”舒克说。  
        “是绰号。”鲁西西突然想起了什么,“海棠花好像被鹰钩鼻子杀害了呀!”  
        “什么鹰钩鼻子?”贝塔问。  
        “也可能是长得像,要不要遥感证实一下?”舒克问鲁西西。  
        “算了。”鲁西西摇摇头,“咱们返航吧,皮皮鲁该等急了。”  
        舒克操纵五角飞碟起飞。返航。  
        皮皮鲁和燕妮在家里一边聊天一边等五角飞碟的消息。  
        “有红沙发音乐城谱曲演奏,再加上歌唱家的嗓子,怕是从明天起这星球上唱歌的就都没饭吃了。”皮皮鲁说。        
        “红沙发音乐城真的那么厉害?”燕妮出生于音乐之乡,见过超级音乐。  
        “你一听就知道了。如果把红沙发音乐城比喻成博士,贝多芬顶多是幼儿园中班第一学期水平。”皮皮鲁一笑。  
        燕妮撇嘴,但她心里信皮皮鲁的话。  
        通讯器响了。  
        “皮皮鲁,皮皮鲁,我是舒克,我是舒克,我们现在返航!”舒克的声音。  
        “我是皮皮鲁,请返航。”皮皮鲁喜上眉梢。  
        “我去给他们做饭。”燕妮起身去厨房。  
        “我敢说,你还没走到厨房,他们就回来了。”皮皮鲁清楚五角飞碟的速度。  
        “打赌。”燕妮说。  
        “赌什么?”皮皮鲁问。  
        “如果我赢了……”燕妮一时想不出。  
        “如果你走进厨房时五角飞碟还没回来,这顿饭我做。”皮皮鲁说。  
        “不行!”燕妮坚决摇头。  
        “为什么?”皮皮鲁不解。  
        燕妮历来认为让先生下厨房是妻子的耻辱,她觉得真正的女人(不管有多大成就)都不会让自己的先生做饭。如果要做,就去当闻名遐迩的特级厨师。燕妮的妈妈对女儿说的一句话至今被燕妮奉若神明:在家里下厨房的男人有出息的不多。  
        即使是打赌,燕妮也不能拿让皮皮鲁下厨房打赌。  
        “快说赌什么,不然五角飞碟就回来了。”皮皮鲁往窗外看了一眼,  “这样吧,你赢了你亲我一下。我赢了我亲你一下。”  
        “行。”燕妮同意。  
        皮皮鲁目送燕妮下楼朝厨房走去。  
        “我已经进厨房了,五角飞碟回来了吗?”燕妮在楼下大声喊。  
        “我输了。只好让你非礼了。”皮皮鲁下楼走进厨房,“怎么回事?该回来了。”  
        燕妮享受胜利果实后,建议皮皮鲁和五角飞碟联系一下。  
        皮皮鲁拿起通讯器。  
        “舒克,舒克,我是皮皮鲁,你们怎么还没到家?”皮皮鲁呼叫。  
        “我们就在离家不远的空中,不知怎么搞的,五角飞碟动不了了,出故障了!”舒克说。  
        五角飞碟从未出过故障。  
        皮皮鲁跑出鲁西西别墅,爬上窗台往外看,五角飞碟真的悬在距窗户200米的地方,高度与皮皮鲁的窗户平行。  
        现在是大白天,如果被行人发现,后果不堪设想。          第278集  
        五角飞碟应急系统不起作用;  
        家庭妇女目光最敏锐;  
        探长林为皮皮鲁两肋插刀;  
        鲁西西发表老板宣言    
        五角飞碟破天荒出故障了,它悬停在距离皮皮鲁家200米的地方,动弹不得。  
        “舒克,怎么搞的?”皮皮鲁站在窗台上看着悬停在空中的五角飞碟,问。  
        “操纵失灵!”舒克的声音里充满了焦虑。  
        “起动应急系统!”皮皮鲁指挥。  
        舒克按下了红色的应急起动按钮,五角飞碟无动于衷。  
        “应急系统失灵!”舒克向皮皮鲁报告。  
        “这回彻底完了。”贝塔回头冲鲁西西和歌唱家说。        
        “没办法吗?”歌唱家不信皮皮鲁会被五角飞碟的故障难倒。  
        “当然会有办法。”鲁西西故意用轻松的口气说。她知道,越是在紧急关头,越不能制造紧张空气。  
        “咱们现在高度很低,如果被行人发现,够麻烦的。”贝塔边说边打开遥感仪,观察地面。  
        荧光屏上出现了五角飞碟下边的场景。  
        人们匆匆忙忙地鱼贯而过,有的人夹着公文包,有的人拿着手机,还有的人戴着随身听,都在忙着去干自己的事,一个比一个显得忙,但绝大部分人是白忙,上帝就爱和人类开这种玩笑。  
        “他们都好像心事重重。”歌唱家看着荧光屏上的人流,说。  
        “想事儿呢。”鲁西西说。  
        “人类一思索,上帝就发笑。”贝塔说。  
        歌唱家惊讶地看贝塔,她觉得刚才贝塔这句话很精彩,很耐人寻昧。  
        “不是我的话,是欧洲谚语。”贝塔赶忙澄清。  
        “有人发现咱们了!”舒克终于宣布。  
        一位四十多岁的家庭妇女模样的女士看见了悬在空中的五角飞碟,她瞪大了眼睛边走边看,然后站了下来。  
        行人看见她往天上看,都站下来往上看。  
        “是飞碟!”一个男人惊叫。        
        “这么小的飞碟?”一个老者眯着眼睛观察五角飞碟。  
        围观的人越来越多,汽车都停下来往上看,半条街挤满了人。  
        皮皮鲁头上出汗了,他看着窗外的场面一筹莫展。他知道,这回五角飞碟是遇上真正的危机了,他最担心的是里边的舒克、贝塔、歌唱家和鲁西西。  
        “我有个办法。”燕妮急中生智。  
        “快说。”皮皮鲁清楚女人遇到危急关头往往比男人冷静。  
        “给探长林打电话,向他求救。”燕妮说。  
        “探长林!”皮皮鲁脱日而出。他想了想,认为可行。  
        皮皮鲁要通了探长林的电话。  
        “我是皮皮鲁,我遇到了难题,想请您帮忙。”皮皮鲁开门见山。  
        “请讲。”探长林愿意帮助皮皮鲁,愿意和皮皮鲁交朋友。他认定,像皮皮鲁这种人,谁和他交朋友,谁就有好运。  
        “我有一架小型飞行器,干脆告诉你吧,叫五角飞碟,它现在出了故障,就悬在我的窗户外边,有许多人围观,情况非常危急,请你想办法驱散人群。”皮皮鲁说。  
        探长林早就料到皮皮鲁手中有超现代化的器物,现在,他终于可以一睹它的风采了。  
        “我马上就去。”探长林挂上电话。  
        皮皮鲁松了一口气。  
        现场足有三千多人抬头看五角飞碟,一个小伙子提议想办法将五角飞碟弄下来。  
        “我看它是出故障了,咱们应该把它弄下来。”那小伙子大声说,“肯定是个有价值的东西。”  
        不少人响应。  
        “这小于活得不耐烦了。”贝塔伸手去摸射击按钮,他想试试五角飞碟的武器系统是否也出故障了。  
        “算了,别理这种人。”舒克制止贝塔。  
        “我的公司里也有这样的职员。”鲁西西冒出一句。  
        “什么样?”歌唱家问。  
        “比方说有甲乙二人。甲月薪500元,乙月薪1000元。有三种甲,第一种无所谓,第二种立志要超过乙的月薪,第三种要把乙的月薪拉回到500元。”  
        “这第三种甲最讨厌。”舒克说。  
        “和要给咱们使坏的那小子差不多。”贝塔还是想教训那小子。  
        “对于这种人,最好离他远点儿,特可怕。”鲁西西好像心有余悸,“嫉妒成性,自己又不努力。”  
        “碰上这样的人,你怎么办?”歌唱家感兴趣地问鲁西西。        
        “辞退。”鲁西西很干脆。  
        “说实话,到你的公司工作,就等于人生遇到了大机会,遇到了机会又把握不住而丧失,实在是智商太低的表现。”贝塔为那此被辞退的员工感到惋惜。  
        “其实我现在对赢利并没有太大的兴趣,钱这东西,到什么时候是够?1亿?20亿?100亿?钱太多了,反而有一无所有的感觉。我现在经营公司,就是想替天行道。”鲁西西说。  
        “替天行道?听起来像农民起义军。我还从没听说过办公司打出替天行道口号的。”贝塔说。  
        “我觉得.凡是能在我的公司工作的人,都和我有缘分。我就是要褒奖那些能吃苦、敬业的人.贬斥那些偷奸耍滑的人。我要让那些勤劳敬业的人得到丰厚的报酬,让他们感受到上帝的存在。同时让那些懒惰成性的人拿低于贫困线的工资,让他们终日挣扎在嫉妒的泥潭里,除非他们改变。”鲁西西像在发表宣言。          第279集  
        智力抢劫和暴力抢劫;  
        探长林声明是拍电影;  
        石油封闭感应器;  
        舒克走出五角飞碟    
        “这样的老板宣言我还是头一次昕到。”舒克说,“的确可以称得上是替天行道。正直的人勤劳的人敬业的人就是应该比懒惰的人偷奸耍滑的人收入高活得好。”  
        “我觉得商人有三个阶段。第一阶段是奸商,第二阶段是儒商,第三阶段是佛商。我看鲁西西已经算是佛商了。”贝塔又发奇论。  
        “鲁西西是佛商。”舒克认为可以给鲁西西发这张文凭。  
        “世界上的钱就这么多,你多挣一元,别人就少挣一元。世界上每增加一个富人,同时就增加一个        穷人。”贝塔又发怪论,“我觉得,爱人类爱同胞爱世界的最好方法就是少挣钱。一个人对人类最大的贡献就是少挣钱,把钱让出来让别人挣。”  
        “乱讲。”歌唱家瞪了贝塔一眼。  
        “你别说,我觉得贝塔的话有道理。这个世界上,每增加一个百万富翁,肯定会增加100个穷人,他把该别人挣的钱挣走了嘛。贝塔说得对,世界上就这么多钱,你多挣一元,就有人少挣一元。”舒克说。  
        “所以富人挣多了钱总是会捐。”歌唱家说。  
        “人也真怪,拼命和同类争夺钱,争到了,再捐给同类,宇宙第一怪物。”贝塔说,“抢了你的东西,再还给你。于是就成了英雄,成了事业上的成功者。”  
        “从别人那儿抢的钱越多,你的人生就越成功,你的生命就越有价值。人类每年还评选首富呢!”舒克说。  
        “抢这个词不太准确吧?”鲁西西不大同意把挣钱说成抢钱。  
        “本质上一样,只不过这种抢不是用暴力,而是用智力。”贝塔一句话,把人类中的成功者都编人了抢劫犯的行列。  
        “快看下边!”歌唱家指着荧光屏说。  
        几辆警车响着警笛驶到五角飞碟下边。        
        “连警察都来了。”贝塔吹了声口哨。  
        “我请示一下皮皮鲁,看看能不能使用武器自卫。”舒克同皮皮鲁联系。  
        皮皮鲁告诉舒克,警车里坐着探长林,自己人,来替五角飞碟解围的。  
        “皮皮鲁疯了,请警察来救咱们!”贝塔吃了一惊。  
        “我看那探长林人不错。”鲁西西注视着荧光屏说。  
        探长林从警车里钻出来,他命令手下驱散人群。疏导车辆。  
        “不要看了,这是在拍科幻电影!”探长林冲人群喊话。  
        听说是拍电影。人群立即减少了三分之二。还有三分之一对拍电影有兴趣,怎么轰也不走。  
        “皮皮鲁,你想办法排除故障吧,这些人死活不走。不过,没有人会伤害你的五角飞碟。”探长林用于提电话同皮皮售联系。  
        “谢谢你。我想办法排除故障。”皮皮鲁说。  
        “需要我做什么?”燕妮问。  
        “去拿皮皮鲁口服液,咱俩变大了才方便排除故障。”皮皮鲁说。  
        燕妮和皮皮鲁喝了皮皮鲁口服液后,变大了。  
        皮皮鲁从壁柜中找出一架望远镜。他站在窗前双手举起望远镜仔细观察五角飞碟。  
        五角飞碟浑身涂满了黑乎乎的液体。  
        “舒克,五角飞碟外壳上沾了什么东西?”皮皮鲁问舒克。  
        “石油!”舒克说。  
        “石油把五角飞碟外壳上的感应器封住了。”皮皮售找到了故障所在。  
        舒克和贝塔恍然大悟。  
        “怎么办?”舒克问皮皮鲁。  
        “必须将感应器上的石油清除掉。”皮皮鲁说。  
        “怎么清除?”舒克问。  
        “出去清除。”皮皮鲁清楚这项工作难度太大了。  
        “我去。”贝塔站起来。  
        “还是我去吧,你拖家带口的。”舒克把贝塔按在座椅上。  
        “抽签。”贝塔提议。  
        抽签结果,舒克赢了。  
        舒克做在空中走出五角飞碟的准备,皮皮鲁叮嘱舒克小心。  
        “探长林,我的朋友开始排除五角飞碟的故障,请你注意保护。”皮皮鲁通知探长林。  
        “明白。”探长林回答。  
        五角飞碟的舱门打开了,舒克伸出一只手抓住五角飞碟的一个角。        
        “当心!”鲁西西对舒克说。  
        舒克的身体全部钻出五角飞碟。歌唱家紧张得闭上眼睛不敢看。  
        贝塔为舒克捏了一把汗。  
        舒克艰难地到达了感应器的位置。他一只手抓住天线,另一只手掏出布清除感应器上的石油。  
        皮皮鲁握望远镜的手出汗了。  
        舒克将石油清除干净了。皮皮鲁松了一口气。燕妮给皮皮鲁擦脑门上的汗。  
        舒克开始返回,就在这时,突然刮起一阵疾风,舒克没有准备,他双手一松,身体向下坠落。  
        “舒克!”皮皮鲁大喊。  
        “啊——”歌唱家尖叫。  
        舒克的身体以极快的速度下坠。  
        探长林反应极快,他迅速跑到舒克坠落的下方。  
        探长林接住了舒克。  
        皮皮鲁不敢相信自己的眼睛。          第280集  
        五角飞碟俯冲地面被皮皮鲁制止;  
        探长林拿着一个烫(又鸟)蛋;  
        红沙发音乐城比贝多芬棒;  
        歌唱家选择广播电台    
        贝塔不知道舒克是死是活,他启动五角飞碟,成功了。  
        “快去救舒克!”鲁西西说。  
        五角飞碟闪电般地朝地面俯冲。  
        “贝塔,别胡来,舒克被探长林接住了。你们先别直接回家,这么多人看着。”皮皮鲁制止贝塔采用疯狂动作救舒克。  
        “我们先去高空呆一会儿,等你的指令返航。”贝塔听说舒克没死,放心了。  
        鲁西西和歌唱家互相拥抱。  
        不少人想知道探长林接住了什么,警察将好奇的人群驱赶开。  
        探长林没想到皮皮鲁的朋友是一只老鼠,他感到手中的老鼠突然变成了烫人的(又鸟)蛋,扔也不是,拿也不是。  
        “谢谢你救了我。”舒克说。  
        探长林吓了一跳。老鼠会说话。  
        “送我回皮皮鲁家好吗?”舒克请求。  
        探长林点点头,他拿起移动电话。  
        皮皮鲁表示欢迎探长林。  
        当探长林将舒克交到皮皮鲁手中时,皮皮鲁很激动,就像见到一个久别的亲人。  
        燕妮藏在另一间屋里没露面。  
        “五角飞碟的故障排除了?”探长林问皮皮鲁。  
        “排除了,多亏你。”皮皮鲁再次向探长林致谢。  
        “应该的。”探长林说。  
        皮皮鲁觉得尽管探长林不问,可如果这个时候还不将五角飞碟的秘密告诉他,就太不够朋友了。  
        皮皮鲁将五角飞碟的来历简要地讲给探长林听,探长林的表情极为惊讶。  
        “怎么说,糕鱼氏曾经抢走过五角飞碟?”探长林想到了数年前的案子。  
        皮皮鲁点头。  
        “后来你又夺回五角飞碟并用它消灭了糕鱼氏?”探长林恍然大悟。        
        皮皮鲁继续点头。  
        “最近你给我提供的录像带也是五角飞碟拍摄的?”探长林想起了浴缸溺尸案。  
        “是的,今后你碰上疑难案,给我打个电话就行了。”皮皮鲁说。  
        “你为什么不多制造几个五角飞碟为人类服务?’探长林不明白这么好的东西干吗藏着。  
        “照人类的本性,有了这个东西,世界还能太平吗?”皮皮鲁问探长林。  
        “这倒是。”探长林细一想,觉得后果的确可怕。比如糕鱼氏。  
        皮皮鲁从冰箱中取饮料给探长林喝。  
        “你把老鼠训练得会说话了?”探长林边喝饮料边问。  
        “舒克是我小时候的朋友。”皮皮鲁指指桌子上的舒克,说。  
        “你小时候的朋友?”探长林不信老鼠能活这么久。  
        “他去过外星球,寿命自然长。天上一天,人间一年嘛。”皮皮鲁解释。  
        探长林觉得这个世界很有意思。  
        “除了破案,你平时喜欢干什么?”皮皮鲁换了一个轻松的话题。  
        “看足球。”探长林说。        
        “体育比赛将人类之间的竞争表面化,白热化,而且胜负马上就见分晓。人们表而上看的是体育比赛,实质上看的是人生竞争。”皮皮鲁说。  
        “没错。你当了冠军,别人就当不上冠军,客观上你就压制了别人。”探长林说。  
        “好在谁也不可能永远当冠军。冠军永远属于那些没当过冠军的人。”皮皮鲁说。  
        探长林点头。他觉得交皮皮鲁这个朋友值得,和他交谈就是一种享受,皮皮鲁总能说出你从来没听过的道理。  
        “我走了,常联系。”探长林站起来。  
        舒克再次向探长林致谢。  
        皮皮鲁将朋友送到楼下。  
        贝塔、歌唱家和鲁西西在高空中密切注视着皮皮鲁和探长林。见探长林走了,贝塔问皮皮鲁:  
        “我们可以返航了吧?”  
        皮皮鲁从窗户往楼下看,已经没人围观了。  
        “返航。”皮皮鲁说。  
        贝塔操纵五角飞碟回到皮皮鲁家。  
        皮皮鲁和燕妮挨个亲从五角飞碟上下来的勇士们。  
        “你们最好还是变小,悬殊太大,每亲一下都给人泰山压顶的感觉。”贝塔说。  
        皮皮鲁和燕妮服药后变小。        
        大家走进鲁西西别墅,有说不完的话。  
        皮皮鲁和燕妮听了红沙发音乐城为歌唱家谱的歌曲,半天说不出一句话。  
        “比贝多芬伟大多了。”燕妮不得不服。  
        “这歌一唱,市面上那些作词作曲和唱歌的就都完了。”皮皮鲁为他们惋惜。  
        “不管哪个行业,冠军只有一个。有点残酷。”贝塔说,“每当我看电视歌手大赛时,我都觉得他们手里握着的不是话筒,而是手榴弹。”  
        “吃饭”燕妮招呼大家。  
        一顿丰盛的晚宴。  
        席间,大家边吃边商议次日如何向社会推出歌唱家和她的歌。  
        “必须通过电视台。”舒克说。  
        “组织一场音乐会也行。”燕妮说。  
        “找出版商出音带。”鲁西西说。  
        “去广播电台。”贝塔提议。  
        “你想怎么办?”皮皮鲁问歌唱家。  
        “去广播电台歌曲排行榜节目当一回嘉宾主持。”歌唱家说,她经常听电台的节目。  
        “我在广播电台有个朋友,明天我去联系。”鲁西西说。  
        广播电台能让名不见经传的歌唱家进播音室吗?          第281集  
        鲁西西和魏楠通电话;  
        主持人孪子会见歌唱家;  
        发票被鲁西西拒绝;  
        有歌声的地方准有狗    
        第二天,鲁西西给在广播电台工作的朋友打电话,那朋友是她的小学同班同学,叫魏楠。  
        “魏楠吗?我是鲁西西。”鲁西西说。  
        “你好!你现在可是大企业家了。”魏楠想起上小学时鲁西西就爱给玩具娃娃做服装。  
        “见笑。”鲁西西说。  
        “昨天我从一张报上看到一位大公司老板的自白,他说他办公司的目的就是让所有离开他的公司的人终身后悔。企业家都有这种心态吗?”  
        “好像有点儿。对了,我想请你帮个忙。”  
        “请讲。”        
        鲁西西说,有个叫贝一的业余歌手,想上电台的歌曲排行榜,最好能到直播室当一回嘉宾主持。  
        “她参加过大奖赛吗?”魏楠问。  
        “没有。”  
        “凭空就上电台排行榜?”  
        “什么叫凭空?”  
        “就是一点儿名气没有。”  
        “她的嗓子和歌绝对是第一流的。”  
        “我安排你和歌曲排行榜的主持人见一面,你带上贝一。”魏楠说。  
        “谢谢。”鲁西西认定歌曲排行榜节目主持人只要一见歌唱家,准因为发现了千里马而兴奋得晕过去。  
        喝皮皮鲁口服液后变大的歌唱家随鲁西西来到广播电台的会客室。  
        “别忘了你的名字是贝一。”鲁西西叮嘱歌唱家。  
        歌唱家坐在会客室的皮沙发上,她观察这间屋子里的人,她发现等候会见的人都有一种虔诚的表情。歌唱家不禁对在广播电台工作的人肃然起敬。  
        魏楠和一位中年妇女从另一个门走进会客室,鲁西西迎上去。  
        “这是著名主持人孪子小姐。这位是我的同学鲁西西。”魏楠为双方引见。  
        鲁西西和孪子握手。交换名片。        
        鲁西西将歌唱家介绍给孪子:  
        “她叫贝一,我的朋友。”  
        孪子矜持地伸出两根指头和贝一握手。  
        “你们谈吧。我还有个节目要制作。”魏楠向鲁西西告辞。  
        “谢谢你。”鲁西西对魏楠说。  
        “这地方太乱,隔壁的饭店有间咖啡厅,很有情调,咱们去那儿谈吧。”孪子提议。  
        鲁西西和歌唱家随孪子走进那家饭店的咖啡厅,小姐为她们引座。  
        鲁西西将歌唱家的情况介绍给孪子。  
        “没上过音乐学院?”孪子呷了一口饮料,问歌唱家。  
        歌唱家摇头。  
        “没参加过任何声乐比赛?”李子又问。  
        歌唱家又摇头。  
        “就是自己喜欢唱歌?”  
        “是。”  
        “想上我主持的音乐排行榜?”  
        “是的。”鲁西西替歌唱家说,“她的嗓子很有特点,她的歌肯定会引起轰动。”  
        孪子脸上露出一丝讥笑,她用极轻微但显而易见的动作摇了摇头。  
        “这是她录的带子,您可以听一下。”鲁西西从包里拿出小型录放机。  
        孪子不屑一顾地摆摆手。  
        “能上我的节目的歌手,都有一定的名气。不过,我可以让她上。”孪子对鲁西西说。  
        “太感谢了。”鲁西西说。  
        “你知道,我们工作也很辛苦,外人看我们挺潇洒,每天往直播间一坐,千万人听你神侃。其实,日子长了,直播间和囚禁室没什么两样。同样的话被你说过一千遍后,那还叫话吗?说是狗叫还差不多。”孪子眼中露出忧郁。  
        鲁西西不明白这位闻名遐迩的主持人说这些千什么。  
        “我们的工资特低,劳动强度特大,连上下班的路上都得编词儿。”孪子叹了口气。  
        鲁西西和歌唱家同情地看着主持人孪子。  
        “你是企业家、服装设计大师,希望你能帮助我和我的节目。”孪子露出了狰容。  
        “帮助?怎么帮助?”鲁西西不解。  
        孪子从衣兜里掏出一叠发票。  
        “这是我们组最近的一些开销,报不了帐,希望你能帮助解决。”孪子厚颜无耻地说。  
        “多少钱?”鲁西西问。  
        “一万三千元。”  
        “如果我给你报销了,贝一的歌就能上你的节目?”鲁西西有吞苍蝇的感觉。  
        “肯定。”孪子拍胸脯。  
        “如果我不给你报销这些发票,贝一就不能上你的节目?”鲁西西又问。  
        孪子没点头没摇头也没说话,就那么直盯盯地看着鲁西西。  
        歌唱家觉得孪子很脏,很卑鄙,她还觉得等在电台会客室的那些人脸上的虔诚表情十分好笑也十分可怜,就像一个人在崇拜一只臭虫一样滑稽。  
        “我不想上歌曲排行榜了。”歌唱家站起来。  
        孪子惊讶地看着歌唱家,她又看鲁西西。  
        “对不起,我不能给你报销这些发票。”鲁西西说完招手叫小姐结饮料的帐。  
        孪子脸上红一阵白一阵。  
        鲁西西和歌唱家向孪子告辞。  
        “任何一家电台都不会接受你,我和他们都是哥们儿。”孪子一字一句地说。  
        “上帝怎么会容忍你这种人主持音乐节目呢?”鲁西西摇摇头。  
        歌唱家和鲁西西走出那家大饭店,她们没有叫出租车,她们想走一走,排遣心中的烦闷。  
        “哪里有歌声,哪里就有狗。”歌唱家想起了自己的经历,说。          第282集  
        贝塔的排队学说;  
        窝头哲学属于舒克;  
        巡警听歌忘了执勤;  
        一炮走红闻名遐迩    
        鲁西西和歌唱家回到家里,大家从她俩的脸上就看到了一切。  
        “失败了?”皮皮鲁问。  
        鲁西西把经过告诉大家。  
        “太卑鄙了。”燕妮忿忿然。  
        “别往心里去。”贝塔安慰妻子,“其实人生就是排队。所有人从生下来起,就都排一个队,排队的目的就是死。有的人排得不耐烦了,就唱支歌,结果成了歌唱家。有的人排烦了,就画一幅画,结果成了画家。有的人什么也没干,一样是排队。”  
        “贝塔歪理太多。”燕妮说。        
        “贝塔的话有道理,咱们不必为这些小事烦恼。那位孪子小姐也是在排队,她排烦了,想花别人的钱,就让她花去吧。”舒克说。  
        “人生其实没有任何意义,就和蚂蚁的一生一样,只不过是一种自然现象,一个过程。所有的人生都是真正的人生,伟人和罪犯都是真正的人生,你们能说哪个人的人生是假的吗?”贝塔胡说八道。  
        “反对。”皮皮鲁不同意贝塔的论点,当然他清楚,贝塔是为了安慰妻子。  
        “我觉得我现在很幸福。想想过去被胡安娜奴役的时候,我现在真的很幸福。”歌唱家为朋友们宽心。  
        “你们说,为什么同样是唱歌,过去她在胡安娜那儿唱的时候是痛苦,而现在唱却是幸福?”贝塔问大家。  
        “这道理和吃窝头一样。”舒克说,“过去人们吃窝头时感到委屈,现在吃窝头感到幸福,为什么?就因为过去是必须吃,而现在是可吃可不吃。必须做的事,人就会委屈。能够自己选择时,人就会感到幸福。”  
        “照你这么说,一个家庭要想稳固,一方应该给另一方选择的余地。如果一方说,你必须和我过,另一方准特痛苦。就和以前吃窝头一样。”皮皮鲁说。        
        “没错,这就是我的窝头哲学。”舒克说。  
        歌唱家笑了,她觉得舒克的窝头哲学很有道理。人就是这样,同样一件事,必须做的时候,准不想做。能选择的时候,倒可能想做了。  
        “我就不信所有广播电台的编辑和主持人都和孪子一样卑鄙,咱们挨个给他们打电话,什么也不说,就让歌唱家对着话筒唱歌!”鲁西西忽发奇想。  
        “这主意不错。”燕妮投赞成票。  
        鲁西西通过查号台将本市所有广播电台的电话号码记在一张纸上,然后挨个给他们打电话。  
        第一个电台的编辑在电话里大叫,说是碰上了精神病患者。  
        第二个电台的主持人一听到听筒里有歌声就把电话挂了。  
        第三个是一位小姐,她在电话里人叫一声,她连声称赞歌唱家的歌声精彩。  
        “有戏!”贝塔说。  
        “请问你是谁?”小姐问歌唱家。  
        “我叫贝一。”歌唱家对着话筒说。  
        “我现在能见你吗?”小姐迫不及待。  
        歌唱家捂住话筒问鲁西西。  
        “当然可以,咱们马上去她的电台。”鲁西西说。  
        “你有发票吗?”歌唱家一朝被蛇咬,十年怕井绳。        
        “什么发票?”对方莫名其妙。  
        鲁西西从歌唱家手里抢过电话,对着话筒说:“一会儿见!”  
        大家笑歌唱家。  
        “我还以为天下乌鸦一般黑呢。”歌唱家为自己辩解。  
        半个小时后,鲁西西和歌唱家赶到那家电台门口。那位小姐是该电台每周一歌节目的主持人。  
        “刚才在电话里唱歌的是谁?”小姐问鲁西西和歌唱家。  
        “是她。”鲁西西指着歌唱家,说。  
        “你现在能再唱一遍吗?”那小姐急不可待。  
        歌唱家在电台门口一展歌喉。  
        过路的行人全昕傻了,他们团团围住歌唱家。巡警都忘了执勤。  
        “太棒了,你现在就跟我去直播室。”小姐拉着歌唱家就往电台里边跑。  
        鲁西西到传达室给皮皮鲁打电话,让他们在家收听。  
        小姐将歌唱家领进直播间,歌唱家看着一屋子的仪器仪表,感到眼花缭乱。  
        “你准备一下,我让你唱时,你就对着话筒唱。”小姐让歌唱家坐在她对面。  
        “今天,我向听众朋友介绍一位歌手,她叫贝一,您可能是第一次听到她的名字,但我相信,您一会儿听了她的歌后,就再不会忘记这个名字了。由于时间仓促,她还没有乐队伴奏,虽然是清唱,但您仍会激动不已。”小姐说完示意歌唱家开始唱。  
        歌唱家十分珍惜这次机会,这是她有生以来第一次演出,第一次给许多人唱歌。她对着话筒唱歌,她想起了自己的坎坷经历,想起了皮皮鲁和鲁西西,想起了舒克和燕妮,想起了贝塔……  
        千百万听众从收音机里听到了歌唱家的歌声,他们放下手中的一切事情,他们如醉如痴。  
        头脑敏捷的音像出版商们争先恐后地赶到那家电台门口等候贝一,他们要和她签约,他们要包装她,他们要拿她赚大把大把的钱。  
        歌唱家的歌刚唱完,电台编辑室的几部电话就此起彼伏地铃声大作,都是要求重播的电话,有一个听众甚至扬言如果电台不在10分钟内重播贝一的歌,他就从24层楼上跳下去。  
        电台台长当即决定提拔那小姐为主任。          第283集  
        自打耳光的电台主持人;  
        过关斩将式的验车;  
        崔姓盗版犯大难临头;  
        盗版人肩负重任出征    
        歌唱家的歌在广播电台歌曲排行榜上连续17个星期排列第一名,创造了有目共睹的乐坛奇迹。贝一成为家喻户晓的名字,就像所有在一夜之间突然成名的歌手一样,人们对她充满了神秘探求心理和羡慕。  
        大街小巷到处可以听到歌唱家的歌,人人以能哼唱贝塔作词红沙发音乐城作曲贝一演唱的那首歌为荣。电视台电台争先恐后地邀请歌唱家去当嘉宾,原先向歌唱家伸手要钱的那家电台的主持人孪子痛哭流涕地当着歌唱家的面连打自己十几个嘴巴忏悔,歌唱家还是没去。        
        贝塔每天忙着在鲁西西别墅里写歌词。他写的歌词都是大白话,比如有一首名为《父与子》的歌词,只有几句话:  
        我和我年幼的儿子的关系很简单——  
        现在他帮我花钱,  
        将来我帮他花钱。  
        这首歌词经过歌唱家谱曲后一唱即大红大紫。  
        还有一首《男人的身高》也一鸣惊人,歌词只有一句:  
        男人的身高不看尺寸,看钱包。  
        歌唱家说全仗着贝塔的词好。就连当过几天作家的舒克也不得不承认贝塔的歌词写得精彩。舒克总结其原因是贝塔根本就不懂文学。  
        这天吃早饭时,皮皮鲁从报纸上看到交通管理局关于更换汽车牌照的公告。早饭后,皮皮鲁准备去更换他的汽车牌照。  
        “我跟你去。”燕妮说。  
        “变大了去?”皮皮鲁问。  
        “嗯。”燕妮很想变大了出去逛逛。  
        皮皮鲁想了想,同意了。  
        燕妮种皮皮鲁服用皮皮鲁口服液后,变大了。  
        他们下楼到停车场,皮皮鲁为燕妮开车门。燕妮坐在皮皮鲁身边。  
        皮皮鲁驾驶汽车驶向汽车检测场,燕妮贪婪地看着车窗外的景色。  
        汽车检测场门口堆满了等待验车的汽车,皮皮鲁摇摇头,将车排在最后。皮皮鲁最怕汽车换牌照,每次换牌照他都有被扒一层皮的感觉,用过五关斩六将来形容其麻烦程度一点儿不过分。  
        ‘要磁带吗?”一个小伙子从车窗外问皮皮鲁。  
        “磁带?”皮皮鲁不明白。  
        “最流行的,整盘全是贝一的歌,比商店便宜一半儿。”  
        那小伙子从包里掏出一盘磁带出示给皮皮鲁看。  
        皮皮鲁对歌唱家出版的磁带再熟悉不过了,他一眼就看出那小伙子手中的磁带是盗版带。  
        皮皮鲁接过磁带。  
        “哪儿来的?”皮皮鲁问。  
        “什么哪儿来的?你买不买?”小伙子不软。  
        “这是盗版。”皮皮鲁说。  
        “废话,不是盗版能这么便宜?”小伙子的脸上开始难看了。  
        皮皮鲁递给小伙子钱,他买了。  
        “买它干什么?”燕妮问皮皮鲁。  
        “咱们回去用五角飞碟查出盗版商,治治他们。”皮皮鲁说。  
        “是该治治他们。这种干盗版勾当的人实在可恶,完全是一种抢劫行为。”燕妮说。        
        皮皮鲁在百般折腾就快自杀前,终于为他的汽车更换了牌照。他的新牌照是黑颜色的,号码是1222。  
        “这号码暗示咱们会有儿子。”皮皮鲁对新牌照的号码很满意。  
        “你能肯定咱们生的孩子是儿子?”燕妮问皮皮鲁。  
        “牌照号码是这么说的,应该没错吧。”皮皮鲁说。  
        燕妮决定回家后用五角飞碟的遥感器预测一下她和皮皮鲁未来的孩子。  
        皮皮鲁驾车和燕妮在环城高速路上兜风,燕妮十分开心。  
        回家后,皮皮鲁将盗版录音带给歌唱家和贝塔看。  
        “效率真高呀!”贝塔对盗版的速度表示惊讶。  
        “查查谁干的。”舒克说。  
        贝塔走进五角飞碟,打开遥感仪。  
        舒克和歌唱家跟进来,站在贝塔身边看。  
        盗版犯被显示在荧光屏上,是一个姓崔的中年男人,此刻他正挽着一位小姐步入一座豪华的酒店。  
        “查查他盗印歌唱家的音带赚了多少钱。”舒克提议。  
        电脑显示的数字是130万元。        
        “让他尝尝盗版的滋味儿。”舒克说。  
        “我有高招儿。”贝塔在删除了盗版崔犯的所有积蓄后,说。  
        歌唱家和舒克洗耳恭听。  
        “咱们用五角飞碟制一个和崔犯一模一样的人,然后让这个复印人也就是盗版人去搅和崔犯的生活?”贝塔异想天开。  
        ‘五角飞碟有这个功能吗?”歌唱家问。  
        “去问问皮皮鲁。”舒克说。  
        皮皮鲁听了贝塔的设想,感到十分开心。  
        “五角飞碟完全可以利用光学原理制作出一个假崔犯,咱们试试。”皮皮鲁兴奋地说。  
        5个小时后,五角飞碟制作了一个崔犯盗版人,和荧光屏上的崔犯一模一样。  
        盗版人完全听从五角飞碟的指令。严格地说,盗版人是新一代的智能型机器人。  
        盗版人肩负着五角飞碟的指令出发了,它前往崔犯的住宅,它去搅乱崔犯的生活,使他不得安宁,享受盗版的痛苦。  
        崔犯做梦也没想到,这回盗版摸了老虎屁股。          第284集  
        拒载的出租车被钉在路上;  
        崔太太真假不分;  
        麻将桌上输掉70万;  
        穷作品的危机    
        崔犯盗版人离开皮皮鲁家,他来到大街上寻找出租车去崔犯家。  
        一辆“面的”驶来,盗版人伸手拦车。  
        “面的”停在盗版人身旁。盗版人上车后对司机说目的地。  
        “不去。”司机连头都不回,冷冰冰甩过来两个字。  
        “为什么?”盗版人知道拒载是出租车的大忌,是违法行为。  
        “我不想去。”面的司机大概觉得这趟活儿赚不了什么钱,就坚决拒载。        
        盗版人摇摇头,下车了。他觉得自己的尊严受到了侵犯,人格受到了侮辱。  
        盗版人使用超人手法惩治面的。  
        面的开出去不到两米,突然停在路中央,不动了。后边马上堵了一长串车。  
        警察走到面的旁,问那司机停在路中央干什么。  
        “我拒载。”司机说。  
        “什么?”警察以为自己听错了。  
        “我——拒——载——”面的司机扯着嗓子喊,脖子上的青筋呈欲爆状。  
        警察没收了面的司机的驾驶证。  
        开来三辆清障车都拖不动那拒载的面的。  
        崔犯的太太在家看电视,敲门声。  
        太太从门镜往外看,见是丈夫,高兴得立刻开门,崔太太就怕一个人在家呆着,寂寞难挨,她最希望下午3点左右丈夫在身边。她的生物钟她自己清楚。  
        盗版人走进崔犯家,崔太太显然认定盗版人是她先生,一阵kuangfengbaoyu式的拥吻。  
        盗版人差点儿被憋死,他极不情愿地应付崔太太,做本不该他做的事。  
        “黑子他们刚才来电话让你去打麻将。”崔太太容光焕发地刘先生说。  
        盗版人赶到黑子家,打麻将赌博。  
        盗版人故意输,只3个小时,就帮崔犯输了70万元。  
        “欠着。”盗版人给麻友们写欠条,  “我一小时后送钱来。”  
        一个小时后,黑子们没等到送钱的崔犯。急了。他们扣崔犯的手提电话。  
        “我是,什么事?”崔犯问。  
        “你小子取钱这么长时间,涮哥们儿呀?”黑子问。  
        “取什么钱?”崔犯莫名其妙。  
        “你欠我的钱呀!..  
        “我欠你的钱?”  
        “你欠我70万!”  
        “放你的狗屁!”  
        “你敢赖账?”黑子急了。  
        20分钟后,崔犯被黑子和另外两个哥们儿打了个半死。  
        在医院的急救室里,满脸是血的崔犯又接到两个催赌债的电话,崔犯傻眼了。  
        盗版人在崔犯的朋友圈子里频频出人,以崔犯的形象和名义瞎输钱瞎许愿瞎坑蒙拐骗。终于,崔犯意识到有一个人冒用他的名义在败坏他,崔犯决定抓获那个小子。  
        一天晚上,当盗版人和崔犯的情人约会时,被崔犯当场抓住。  
        崔犯看到盗版人的正面时,怔了。        
        活脱脱一个自己,连右眼角的痣都一模一样。  
        崔犯的情人看见自己面前出现两个崔犯时,立马毫不犹豫地加人精神分裂症的大军里,一会儿哭一会儿笑,还严正声明自己是托塔李天王的干女儿。  
        “你是谁?”崔犯问盗版人。  
        “你是谁?”盗版人反问。  
        “我是崔犯!”崔犯大喝。  
        “我是崔犯!”盗版人大喝。  
        “你是盗版!”崔犯得了职业病,下意识甩出一句专业术语。  
        “你才是盗版!”盗版人反驳。  
        崔犯的父母和老婆闻讯赶来,面对两个儿子和丈夫,他们束手无策,纵是火眼金睛也难辨真伪。  
        贝塔他们坐在五角飞碟里的荧光屏旁笑得死去活来。舒克说应该给地球上所有盗版、抄袭他人作品的不法分子每人造一个盗版人陪他们度过一生。  
        贝塔连说好主意并立即付诸实施。  
        一位相貌和品质完全同步的小姐在给一家杂志画插图时抄袭别人的插图,欺骗编辑愚弄读者大捞稿费,盗版人光临她的生活搅得她六神无主家无宁日只有她的男朋友窃窃私喜。  
        一位喜欢写作的人苦于写不出传世之作,干脆嫁接前人的作品,东拼西凑,欺世盗名。贝塔派去的盗版人给他生了个小盗版人他还以为是自己的亲骨肉天天抱着不撒手。  
        “作家是不是都希望自己的作品流芳百世?”皮皮鲁问当过作家的舒克。  
        “人概是。”舒克说。  
        “什么样的作品才能世代流传?”燕妮问。  
        “应该说什么样的作家的作品才能世代流传。”舒克说。  
        “这话怎么讲?”歌唱家问。  
        ‘比同时代的大多数人富的作家才可能写出传世之作。经济基础决定一切。读者容易接受与他们经济水平一样的作家的作品。下个时代的人肯定比这个时代的人富有,所以,在这个时代比大多数人富的作家,才可能和下个时代的大多数人一样富,他在上一个时代写的作品才会被下一个时代的读者接受。”舒克说。  
        “这种说法我还是头一次听到。”皮皮鲁挺感兴趣,“你的意思是说,只有比同时代人富有的作家才可能写出下个时代的人爱看的作品?”  
        “对。”舒克肯定。  
        有人点头有人摇头。  
        “不管怎么说,托尔斯泰和曹雪芹摆在那儿。”舒克为自己找论据。  
        穷生仇恨富生爱。仇恨不会流传。舒克一直这样认为。          第285集  
        五角飞碟里的灯光;  
        皮皮鲁的儿子叫皮亚洁;  
        拒绝上高中的儿子;  
        课本上好多东西没用    
        夜已经深了。  
        鲁西西别墅里一片寂静,大家都休息了。  
        燕妮在床上辗转反侧,无论如何睡不着,她脑子里全是皮皮鲁的汽车新牌照号码1222,她迫切想知道自己未来的孩子的有关资讯。  
        皮皮鲁睡得很香,发出轻微的鼾声。  
        燕妮坐起来,轻轻下了床。  
        她拉开卧室的门,蹑手蹑脚走下楼。  
        五角飞碟停在鲁西西别墅旁边,今晚是舒克在里边值班。  
        燕妮敲五角飞碟的舱门。        
        舒克一边揉眼睛一边打开舱门,见是燕妮,他以为发生了意外。  
        “出什么事了?”舒克问。  
        “我想请你帮个忙。”燕妮说,“帮我用五角飞碟的遥感仪预测一下我的孩子的一生。”  
        “你的孩子?”舒克以为燕妮怀孕了。  
        “现在还没有。五角飞碟不是能百分之百地准确预测吗?”燕妮说。  
        “我明白了,是皮皮鲁新换的汽车牌照的号码把你搅得心神不定吧?”舒克笑了。  
        “我想知道那号码准不准。”燕妮耸耸肩膀。  
        “来,咱们看看小皮皮鲁的一生。”舒克爽快地答应了。  
        燕妮和舒克在驾驶台前坐好,舒克开始调整遥感仪。燕妮有几分紧张地注视着屏幕。  
        遥感仪能像电视剧那样简要地显示皮皮鲁和燕妮的孩子的一生。  
        “人的命运是不可改变的。是你的,你不用争。不是你的,你也争不到。”舒克给燕妮打预防针,他不知道小皮皮鲁的命运如何。  
        屏幕上出现了一家医院,皮皮鲁在妇产科走廊的长椅上焦急地看手表。  
        燕妮明白自己正在产房生孩子。  
        一个漂亮的男孩儿。        
        燕妮这回死心塌地的相信汽车牌照号码对车主命运的预示性了。  
        皮皮鲁走进病房看望燕妮,他是一个伟大的丈夫,他根本没问是男孩儿还是女孩儿,他只关心妻子的身体。  
        燕妮十分感动。一屋子产妇嫉妒得死去活来。她们的先生进屋后第一句话的主题都是关于孩子的性别。  
        几天后,皮皮鲁将燕妮和儿子接回家。  
        皮皮鲁给儿子取名为皮亚洁。  
        皮亚洁很少哭,一双大眼睛平稳地注视着这个世界,好像这个世界都属于他,根本不用着急。他最爱干的事,是躺着时将自己的双腿翘起来与身体呈九十度角,注视自己的两只脚。  
        皮皮鲁倾其所有给皮亚洁买玩具,他觉得玩具对孩子来说是最好的教育工具。把地球上所有的东西缩小了让孩子玩,等于让孩子演习当上帝。  
        皮皮鲁在教育孩子上有自己独特的见解和方法,他认为必须培养孩子的爱国主义精神。而培养孩子爱国主义精神的最好方法就是尽其所能给他买玩具,让他切实产生生活在这个国家的优越感。对于孩子来说,想要什么没什么就是天天升国旗也有不了爱国主义细胞。  
        皮亚洁上学了。除了第一天想去学校,剩下的日子都是不想去,他不适应学校式教育。  
        “对于文学艺术家来说,最悲哀的,莫过于起点就是终点了。对人生也一样。”舒克插话。  
        燕妮一时没明白舒克的话是什么意思,她扭头看舒克。  
        “打个比方,有位作曲家谱了一首歌,一鸣惊人,而以后他一生谱的所有曲子都超不过这首歌的水平,这就叫起点就是终点,是一种悲哀。换句话说,第一部作品就是代表作,惨了点儿。”舒克说。  
        “这和皮亚洁上学有什么关系。”燕妮还是不大懂舒克的话。  
        “只有第一天上学时愿意去学校,这不是起点就是终点吗?”舒克说。  
        燕妮明白了。但她认为这不是孩子的悲哀,而是学校的悲哀。  
        皮亚洁不喜欢上学,他无法适应填鸭式死记硬背式教育方法,但他喜欢知识,喜欢玩具,他通过玩玩具认识这个世界。他不明白怎么有那么多父母不让孩子玩玩具而热衷于让孩子写作业,写作业使孩子怨恨这个世界,玩玩具使孩子爱这个世界。  
        皮皮鲁敞开了给皮亚洁买玩具,皮亚洁的房间四壁竖着顶天立地的玩具柜,里边摆满了各种各样的玩具。人、汽车、怪兽、船、飞机、武器、机器人、动物……整个儿一个地球的缩影。        
        “你的儿子太幸福了。”舒克对燕妮说。  
        “不会宠坏了他吗?”燕妮有点儿不安。  
        “宠是宠不坏的,要坏也是本质就坏,你不宠,他也照样坏。宠只会越宠越好。宠是爱的一种表现。”舒克打消燕妮的顾虑,“不信你往下看。”  
        好在皮亚洁的一生马上就能看到。  
        皮亚洁初中毕业后坚决不继续上学了,他说他无法忍受这种教育方法。  
        “不行!”燕妮作为母亲绝对不同意儿子只有初中学历。  
        “我看可以不上了。”皮皮鲁作为父亲支持儿子的选择。  
        “皮亚洁才16岁,不上学干什么?”燕妮问皮皮鲁。  
        “你想干什么?”皮皮鲁问儿子。  
        “我先在家看两年书,然后发明一种改变现行教育方法的东西。”皮亚洁早有打算。  
        “看书?与其在家看书,为什么不去学校看书?”燕妮问儿子。  
        “学校的课本上大多数东西根本没用,浪费脑细胞。”皮亚洁说。  
        “你准备看什么书?”皮皮鲁感兴趣地问儿子。          第286集  
        看书特慢的儿子;  
        学校退出历史舞台;  
        诺贝尔玩具奖设立;  
        最佳人生公式问世    
        “看经济管理方面的书,看财会方面的书,看成功企业家的传记。”皮亚洁说。  
        燕妮知道儿子要往哪个方向发展了。她也承认现在学校灌输给学生的东西有许多是根本没用的。但她也理解学校为什么这样做——打发学生的时问。如果全学有用的东西,一个人只上四年学就足够了,可一个11岁的孩子到哪儿去上班呢?雇用童工在地球的任何一个角落都是违法的。于是,学校只好用许多根本没用的所谓知识填充学生的时间,硬将四年扩展为16年(如果上大学的话)。人类的童话。  
        燕妮同意了儿子的选择。        
        皮亚洁开始了在家读书自学的生活。他专挑那些对自己有用的书读。  
        有用的书多看,没用的书一个字也不看。如果一个人在少年时代能拥有这种权利,是他一生的福气,他必将成就伟大的事业。可惜百分之九十九的人没有这种权利,所以伟人少。  
        皮亚洁的性格几乎和皮皮鲁一样。燕妮坐在五角飞碟里观看儿子的生平时,这样想。  
        “你说他能有出息吗?”燕妮问舒克。  
        “还不知道。”舒克盯着屏幕上的皮亚洁说,“反正已经看出与众不同了。有本事的人都与众不同。”  
        “脾气还挺大。”燕妮对儿子的个性保留意见。  
        “有本事的人都有脾气。不过真正有本事的人不在家里发脾气。”舒克安慰燕妮。  
        皮亚洁看书有个特点,边看边想,一本书能看几个月。合上书就看屋里的玩具。皮亚洁的房间完全可以称之为玩具博物馆。  
        “这么大了还喜欢玩具。”燕妮叹了口气。  
        “到了皮亚洁这个年龄,还喜欢玩具不一定没出息。不喜欢玩具不一定有出息。”舒克说。  
        皮皮鲁仍然根据儿子的需求给他买玩具,燕妮开始反对。皮皮鲁说,表面上看是爸爸给十几岁的儿子买玩具,实际上是爸爸在投资。燕妮不明白。        
        “我有一种预感。”皮皮鲁说。  
        “什么预感?”妻子问。  
        “咱们的儿子会给这个星球带来一场玩具革命。不,准确点儿说,可能是教育革命。”  
        “玩具和教育有什么关系?”燕妮对丈夫的预感表示怀疑。  
        “我也说不清,反正我有这种预感。我现在花钱给他买玩具,本质上是我对人类的贡献。”皮皮鲁振振有词。  
        “有其父必有其子。”燕妮瞪了皮皮鲁一眼。  
        “但愿你的话是真理。”皮皮鲁最喜欢有其父必有其子这句话,因为他底气足。许多爸爸最怕这句话在自己身上验证。  
        皮亚洁果然用行动为“虎父无犬子”这句话提供了新的依据。他在17岁时发明了一种全新概念的玩具,这种玩具集娱乐、教育为一体,采用变化、声光、电子、拆装、手工等一系列手段,构成了玩具和教育领域的真正革命。该玩具适合任何年龄达到5岁的孩子玩,从头到尾玩一遍需要6年时间,在玩的过程中,孩子会不知不觉地学会足以使他的人生幸福的知识和技能。6年之内学到的知识,相当于上16年学,而且有过之而无不及。  
        皮亚洁为自己的发明申请了专利,他还用父亲的名字命名这一玩具——皮皮鲁玩具。        
        皮皮鲁玩具投放市场的第二天,世界上所有学校——从小学到大学,集体宣布闭校解散,学校这一教育形式正式退出人类历史舞台,成为字典中的古董词汇。  
        如今孩子们只要买一套皮皮鲁玩具,在家玩上6年,就掌握了学习的最科学方法和前人留下的有用的知识。  
        “毕业”后的孩子去求职时,招聘单位总是问一句相同的话:  
        “玩过皮皮鲁玩具吗?”  
        如果玩过,就具备了应聘的条件。  
        皮皮鲁玩具问世一年后,皮亚洁宣布放弃专利,将他的伟大发明无偿奉献给人类使用。  
        “了不起!”舒克大加赞叹。  
        燕妮激动得一句话也说不出来。她知道自己将给一个什么样的孩子当妈妈了。  
        皮亚洁不断修改和完善皮皮鲁玩具。他又开发出皮皮鲁玩具第二代,成为11岁以后的孩子玩的玩具。第二代皮皮鲁玩具玩两年,分为作家型皮皮鲁玩具、画家型皮皮鲁玩具、物理型皮皮鲁玩具、作曲型皮皮鲁玩具、清洁工型皮皮鲁玩具、工人型皮皮鲁玩具、侦探型皮皮鲁玩具……  
        第二代皮皮鲁玩具成为定向培养各类人才的最佳助手。        
        皮亚洁闻名全球,诺贝尔奖评委会专门为他新设立了诺贝尔玩具奖。不管皮皮鲁走到哪儿,别人都这样介绍:  
        “这位老先生是皮亚洁的爸爸。”  
        “真是要发发儿。”燕妮说。  
        “皮亚洁如果这么一直干下去也挺累。”舒克有点儿替皮皮鲁和燕妮的儿子担心。他认为最佳人生公式是:奋斗——享受成功。  
        人生其实是走一段路程,四边的风景很是好看。如果光是急着奔走,就不能欣赏美好景色。但如果一生下来就坐着看风景,又没有欣赏的资格。最好的方法是先奔走(奋斗),有了成就,就坐下来欣赏,享受成功。  
        “戴着冠军桂冠坐在主席台上观看别人比赛,是人生的最大享受。”舒克说,“没当过冠军的人看别人夺冠军,是人生的痛苦和烦恼。当了冠军还不依不饶继续和别人比赛,是人生的最大不幸。”  
        舒克的这段话燕妮一听就懂了,她希望皮亚洁戴着冠军桂冠坐下来看别人争夺冠军的比赛。   第287集  
        荧光屏上的亮点;  
        慧星飞行的轨道;  
        地球上有过几次人类;  
        全都顾不上吃早饭了    
        燕妮回到卧室时,皮皮鲁还在酣睡。  
        燕妮恨不得现在就有孩子。  
        皮皮鲁终于醒了,他听到燕妮自言自语了一句:“皮亚洁还真行。”  
        “皮亚洁?”皮皮鲁不知道皮亚洁是何许人物。  
        “你的儿子。”燕妮说。  
        “我的儿子?在哪儿?”皮皮鲁糊涂了。  
        燕妮将她用五角飞碟测出皮亚洁一生的事讲给皮皮鲁听。  
        “真的?太棒了,我也去看看。”皮皮鲁一骨碌从床上爬起来,穿着睡衣往楼下跑。        
        在楼梯拐弯处,皮皮鲁和贝塔撞了个满怀。  
        “出什么事了?”贝塔问皮皮鲁。  
        “去五角飞碟看皮亚洁。”皮皮鲁兴奋地说。  
        “皮亚洁?”贝塔一愣。  
        皮皮鲁简要地将儿子情况向贝塔做了介绍。  
        “我和你一起去。也测测我和歌唱家的孩子。”贝塔跟着皮皮鲁走进五角飞碟。  
        舒克回自己的房间休息去了。  
        皮皮鲁和贝塔坐在驾驶台前,开始摆弄密密麻麻的按钮和键盘。  
        荧光屏上开始出现形形色色的图案,皮皮鲁调整屏幕上的图像。  
        “这是什么?”贝塔指着屏幕上一个亮点儿问皮皮鲁。  
        “像是一颗星星。”皮皮鲁分析。  
        “好像挺凶。”贝塔冒出这么一句。  
        皮皮鲁仔细观察那亮点儿,他觉得贝塔的话有道理。  
        “查查它,看是怎么回事。”皮皮鲁将遥感仪的聚焦点对准那亮点儿。  
        “彗星。一颗人类还未发现的彗星。”贝塔先看到了结论。  
        皮皮鲁一边点头一边继续观测该彗星。  
        结果使皮皮鲁和贝塔大吃一惊。        
        这颗不小的彗星虽然距离地球挺远,但它的运行方向却将与地球的轨道重合,也就是说,它最终将与地球相撞,准确的时间是2160年8月16日格林威治时间13点27分59秒。  
        “看看它与地球撞击的结果。”贝塔提议。  
        荧光屏上的景象令皮皮鲁和贝塔目瞪口呆,该彗星与地球的撞击使人类遭受了灭顶之灾,无一人生还。人类世代苦心经营的所有文明随之消亡,地球重又回复到满目苍凉的状态。  
        “这……这……怎么可能?”皮皮鲁望着屏幕发呆。  
        贝塔又对那颗彗星遥测了一遍,与第一次遥感的结果完全一样。  
        “都白干了?”皮皮鲁沮丧地说。他为人类为改变地球而付出的劳动感到惋惜。  
        “宇宙这么大,怎么会撞上呢?”贝塔自言自语。  
        “宇宙里每天都有行星相撞,又没有交通警察指挥疏导,不撞才不正常呢。”皮皮鲁说。  
        “应该捡没人的星球撞呀!”贝塔说。  
        “它才不管你有没有人住呢。”皮皮鲁清楚彗星没有头脑和眼睛,碰上谁是谁。  
        “咱们用五角飞碟测测人类的历史吧?”贝塔话音里有哀乐的抑扬顿挫。  
        皮皮鲁沉重地点点头,他的头脑里有关人类史的知识都足从书上得到的,对于真正的人类史,他还一无所知。  
        测试结果再一次让皮皮鲁和贝塔瞠目结舌,地球上一共有过3次人类发展史。也就是说,地球上出现过3次互不相下毫无联系的人类!  
        第一次的人类在地球上生存了52万年,是原装地球人,他们建立的文明已超过今天的人类文明,一颗彗星与地球相撞,结束了这一切。如果那彗星晚与地球相撞10年,人类就能发明出击毁该彗星的武器。  
        23万年以后,火星上的人类预测到一颗彗星将与火星相撞,他们依仗自己的先进科学技术,将火星人全部迁徙到地球定居,从而避免了死亡。火星人在地球上生括了18万年,当又一颗彗星瞄准地球飞过来时,人类自以为已经拥有了足以击毁该彗星的武器。但是他们低估了那颗彗星的能量。当他们的武器发射后,没能阻止住迎面撞来的彗星。于是,人类第二次被摧毁。  
        “又白忙了一次。”贝塔叹了口气。  
        至于今天生活在地球上的人类,则完全是来自四颗不同外星球的移民的后裔,这四颗外星球在同一时间分别向地球选送了黑种人、白种人、黄种人和棕种人,每一个人种各留下两男两女。经过十几万年的繁衍,这八个人将地球变成了今天这副模样。        
        然而,公元2160年8月16日,彗星将第三次向地球人类伸出罪恶之手。  
        面对地球人类的3次沧桑历史,皮皮鲁的大脑忽而一片空白,忽而满脑经纶。他终于意识到,对于宇宙生命来说,死亡是永恒的,生存是短暂的。  
        歌唱家跑进五角飞碟叫皮皮鲁和贝塔吃早饭,当她看见皮皮鲁和贝塔如两尊雕像般坐在驾驶台前时,明白一定是发生了大事情。  
        闻讯赶来的鲁西西、舒克和燕妮同歌唱家一起观看了那颗彗星在2160年8月16目的杰作以及地球人类的3次发展史。  
        沉默。  
        “这一切都将不复存在。”舒克环顾四周首先打破寂静。  
        “我觉得这不能算是白干。”燕妮说,“人类在地球上生活了这么多代人,创造了这么多文明,使这么多人享受了。至于最后一代人遭到厄运,确实不幸,但不管什么事,都会有最后。”  
        “永远存在的东西是没有的。”皮皮鲁说。          第288集  
        人类的反应出乎意料;  
        去美国找约翰组团;  
        舒克贝塔深夜游故居;  
        有人给舒克打电话    
        “现在距离2160年还有一百多年,我觉得人类有能力研制出击毁那颗彗星的武器。”鲁西西说。  
        “对,事不过三。这次一定能。”歌唱家支持鲁西西的观点。  
        “不容易。依我看,第2次人类灭亡时的科技水平比现在高多了。”贝塔泼冷水。  
        “不管怎么说,应该把新发现的这颗彗星通知人类。”舒克说。  
        皮皮鲁当即给国际天文组织打了电话,告诉他们他和贝塔发现了一颗新的彗星,以及那颗彗星的方位和它将与地球相撞的时间。        
        经国际天文组织若干资深天文学家的观察,证实了皮皮鲁的话。  
        那颗彗星被命名为皮皮鲁——贝塔8号彗星。  
        皮皮鲁——贝塔8号彗星与地球相撞的日期经过电脑精确运算,与皮皮鲁所说完全吻合,分秒不差。  
        皮皮鲁再次成为新闻人物。  
        媒介迅速将2160年彗地相撞这一坏消息传播给人类的每一位成员。  
        几乎所有人都是先大吃一惊继而掐指一算与己无关后再长舒一口气。  
        皮皮鲁和鲁西西担心的那种恐惧场面并未出现。  
        “人其实是一种短期行为动物。”贝塔一边看电视一边说。  
        “咱们用五角飞碟测测人类能不能战胜皮皮鲁——贝塔8号彗星吧,”歌唱家提议。  
        “反对。”鲁西西、燕妮、皮皮鲁、舒克和贝塔异口同声。  
        他们不愿意先知道这件事的结局。他们担心自己的神经无法承受。  
        “那咱们去美国找约翰吧。”歌唱家放弃了第一提议。  
        “赞成。”大家再次异口同声。  
        “为什么先找约翰?不是还有少校、艺术家和博士吗?”燕妮问。        
        “少校、艺术家和博士都在国内,只有约翰在国外,先找远的嘛。”皮皮鲁解释。  
        “这次谁去?”舒克问。  
        “反正我去。”鲁西西说,“约翰和我关系最密切,我的英语就是他教的。”  
        经过激烈的讨价还价,最后决定,鲁西西、舒克和燕妮乘坐五角飞碟去美国找约翰,皮皮鲁、贝塔和歌唱家留守。  
        鲁西西、燕妮和舒克去美国寻找约翰的经历极其精彩和惊险,可以称得上是五角飞碟折腾美国。这是后话。  
        “舒克,不知为什么,我特想去我出道前的那座房子看看,你呢?”贝塔一脸的怀旧表情。  
        “我也是。”舒克想起了自己童年居住的洞,想起了妈妈第一次带他出洞寻找食物的情景,还想起了写字台上那盘香喷喷的花生米。  
        他有已经活了一个世纪的感觉。  
        “我们能回故居看看去吗?”贝塔问皮皮鲁。  
        “当然可以,开五角飞碟去吧。”皮皮鲁理解舒克和贝塔这种情怀。自从刚才看了人类的3次历史,他也下意识地想起了自己的童年。  
        “我们不想开五角飞碟,想走着去。”舒克说。  
        “走着去?”皮皮鲁觉得太危险。  
        “对。”舒克和贝塔说。        
        皮皮鲁想了想。  
        “应该让他们去。”鲁西西认为必须尊重朋友的选择。  
        皮皮鲁点点头。  
        “注意安全。等你们一回来,舒克就和鲁西西、燕妮出发去美国。”皮皮鲁叮嘱舒克和贝塔。  
        当天晚上,舒克和贝塔徒步离开皮度鲁家,他们借着夜色的掩护,来到大街上。  
        “我的故居离你的故居不远,都在北边儿。”贝塔指着北边说。  
        “我还用直升机把你的坦克吊到天上过,记得吗?”舒克边走边说。  
        “吊到克里斯王国了,差点儿让猫吃了。”舒克回忆过去的事觉得特享受,不管是幸运的经历还是不幸的经历。越是发达的人,越把回忆自己不幸的经历当作一种乐趣。  
        “咪丽的后代不知活得怎么样。”舒克想起了贝塔那位化敌为友的朋友。  
        “她的后代的数量到今天少说也得有一个团了。”贝塔想起当年自己驾驶坦克炮击咪丽的场面,脸上的表情特厚重。  
        舒克和贝塔边走边聊,他们还搭乘了一段路的自行车——躲在自行车挡泥板侧面。  
        舒克的故居先到了。房子正在拆迁,路旁的建筑广告牌上声明此地一年后将矗立起一座五十多层的大厦。  
        舒克的故居没闲着,一家老鼠住在里边。  
        舒克站在自己的出生地,百感交集。  
        “你们想干什么?”老鼠丈夫看出这两位不速之客不是等闲之辈,不敢动武。  
        “他30多年前在这儿生的,寻根来了。”贝塔指着舒克告诉同胞们。  
        “30多年前?!”老鼠太太摇头。她清楚,老鼠活不了这么多年。  
        “活吧。”舒克摸摸老鼠儿子的头,说了一句最简单又最深刻的话。  
        贝塔的故居已不复存在,那栋房了被一座大医院取代。贝塔和舒克看到医院的实验室里有一笼子供人做试验用的小白鼠,它们个个身价数百美元。它们是人类绞尽脑汁培养出的有先天性心脏病的老鼠,专供医生试验治疗心脏病的新药。  
        舒克和贝塔注视着笼子里的同胞,他们并没有可怜它们。也没有恨人类。因为他们想起了2160年8月16日。  
        凌晨5点,舒克和贝塔平安返回皮皮鲁家中。  
        就在舒克、鲁西西和燕妮准备启程去美国找约翰时,电话铃响了。  
        “舒克,你的电话。”皮皮鲁叫舒克。        
        “谁给我打电话?”舒克从五角飞碟里探出头问。  
        “舒利。”皮皮鲁说。          第289集  
        副市长的猫是舒克的驸马爷;  
        怕别人骂的人不强大;  
        歌唱家是业余华佗    
        “舒利的电话?!”舒克一个跟头从五角飞碟里翻了出来。  
        “快去接!”贝塔催舒克,“掌握分寸,别把孩子吓跑。”  
        天才有两个特征。一是善于把握机会,二是能够掌握分寸。做到第一点容易,做到第二点难。  
        大家都屏住呼吸看舒克接电话。  
        其实,五角飞碟能轻而易举找到舒利。但舒克不这么做。舒克知道好爸爸的标志之一是最大限度给孩子自由。  
        舒利离家出走后主动打电话来,朋友们为舒克高兴。        
        “舒利吗?我是舒克。”舒克的声音比较特别。  
        皮皮鲁在征得舒克的同意后,打开电话的扬声器。  
        “舒克,我有事求你。”舒利的声音急切。  
        “说吧。”舒克担心女儿遇到了麻烦。  
        “我的一个朋友病了,病得很重,快不行了,求你救救他。”舒利哭了。  
        “他是谁,得了什么病?你们在哪儿?”舒克问。  
        “他叫辰羽,是一只小猫。我们本来准备下个月结婚。不知怎么搞的,一个星期前辰羽突然不舒服,头晕,浑身投劲儿。今天更厉害了,现在已经昏迷了。他是副市长家的猫。我们的住址是……”舒利哽咽着说。  
        大家记舒利的地址。  
        “副市长养的猫和老鼠通婚。”贝塔意味深长了一回。  
        “那也没你的跨度大。”鲁西西对贝塔说完又对歌唱家说:“请原谅。”  
        歌唱家一笑。  
        “强大的国家不怕别人骂。强大的人不怕别人说。”贝塔为自己和妻子注解。  
        “快去帮助舒利吧。”皮皮鲁说。  
        “去请医生?”舒克问皮皮鲁。  
        舒克把皮皮鲁问住了。医生主动登副市长的家门要求给副市长的猫看病的可行件显然不大。  
        “用五角飞碟把辰羽送到医院去。”燕妮建议。  
        “他的体积太大,五角飞碟装不下。”贝塔说。  
        大家犯难了。他们印象中还没什么事能难住他们。他们没有这方面的经验。  
        经验往往是没有收获时的收获。  
        “我在德国时对病有过研究,我可以试试给辰羽看病。”歌唱家说。  
        朋友们从未听说过歌唱家会看病。  
        “胡安娜不可能带我去看病。我又不可能不得病。我只好自己给自己看病。”歌唱家说。  
        “你从哪儿弄药呢?”燕妮问歌唱家。  
        “我治病不用药。”歌唱家说。  
        “治病不用药?”鲁西西对歌唱家的医术表示怀疑。  
        “病有两种。一种有形,一种无形。有形的病是身病。无形的病是心病。身病源于心病,没有心病不会有身病。”歌唱家向朋友们展示自己的医术。  
        “什么是心病?”贝塔问。  
        “贪。痴。爱。恨……等等一切烦恼都是心病。”歌唱家回答。  
        “什么是身病?”贝塔又问。  
        “发烧感冒。五脏六腑发炎。癌症……等等一切(禁止)的疾病都是身病。”歌唱家说。        
        “所有身病都是心病引起的?”燕妮问。  
        “千真万确。治身病首先应该治心病。只治身病不治心病,身病会不断骚扰你。”歌唱家说。  
        皮皮鲁点头,他觉得歌唱家的话有道理。  
        “就让歌唱家去试试。”皮皮鲁说。  
        “歌唱家见舒利没事吧?”鲁西西想起了图钉。她怕舒利触景生情受刺激。  
        “舒利不会。”舒克了解女儿。  
        “咱们出发。”皮皮鲁说。  
        “都谁去?”贝塔问。他最怕不让他去。  
        “都去。”皮皮鲁说,“人多智慧多。”  
        “救完辰羽再去美国找约翰。”鲁西西说。  
        大家鱼贯而入五角飞碟。贝塔跑在最前面。          第290集  
        贝塔破窗而入副市长家;  
        蝙蝠给舒利指路;  
        舒利口福不浅:  
        病从天降    
        贝塔和皮皮鲁驾驶五角飞碟。舒克站在驾驶台旁激动。鲁西西、燕妮和歌唱家在餐厅聊天。  
        “天快亮了。”贝塔提醒皮皮鲁。  
        皮皮鲁看表,5点50分。  
        五角飞碟飞临副市长宅邸上空。皮皮鲁打开遥感仪寻找舒利。  
        屏幕上显示舒利在副市长的床下,她的身边是一只奄奄一息的猫。辰羽。  
        “咱们在哪儿着陆?”皮皮鲁征求舒克的意见。  
        舒克使用遥感仪遥感副市长宅邸的全貌。  
        “可以直接在副市长的床下着陆吗?”舒克急于见女儿。  
        “撞碎窗玻璃?”贝塔兴奋。  
        “会惊醒副市长的。”皮皮鲁说,“咱们先在窗台上着陆。用五角飞碟的激光武器划开玻璃,贝塔进去打开窗户。五角飞碟再在副市长的床下着陆。”  
        贝塔操纵五角飞碟在窗台上平稳地着陆。舒克使用激光武器将窗玻璃划开。  
        贝塔钻出五角飞碟,轻轻从窗户上取下那块被分裂的玻璃。  
        舒克和皮皮鲁通过屏幕监视副市长和太太的睡姿有无更改。  
        贝塔进人副市长卧室,他打开窗户。  
        贝塔返回五角飞碟。皮皮鲁驾驶五角飞碟闪电般在床下的舒利身边着陆。  
        “遥感搜索床下有无不安全因素。”皮皮鲁对贝塔说。  
        贝塔认真用遥感仪搜索。  
        “床下有几个破纸箱。其中一个纸箱里有15万元现钞。和旧报纸混在一起。没发现其他异常。”贝塔报告。  
        “我、歌唱家和舒克离开飞碟。其他人留守。贝塔不要离开驾驶台,不要关闭遥感仪。”皮皮鲁对大家说。  
        舒克迫不及待地打开舱门,舒利等候在舱门旁边。  
        “对不起,舒克。”舒利见到爸爸没有哭。  
        “……”舒克没说话,只顾百感交集。  
        皮皮鲁和歌唱家问候舒利。  
        “原谅我。”舒利对歌唱家说。  
        “你没做错什么。”歌唱家说。  
        “先看看辰羽。”皮皮鲁说。  
        舒利把他们带到辰羽身边。辰羽昏迷不醒。歌唱家仔细观察他。  
        “能简单说说经过吗?”歌唱家要求舒利。  
        “歌唱家会治病。”舒克对有些迟疑的舒利说。  
        舒利将自己离家出走后的经历简要告诉大家。贝塔、鲁西西和燕妮在五角飞碟里同步倾听。  
        舒利离开皮皮鲁家后,在城里游荡,适逢全城开展灭鼠活动,舒利几次险些丧命。一个偶然的机会,舒利认识了一只蝙蝠。那蝙蝠告诉舒利,应该去大官家落脚,官越大家里越不灭鼠。  
        舒利在蝙蝠的指点下,潜人这位副市长家定居。  
        果然,副市长家风平浪静,街道灭鼠队不敢到副市长家下毒撒药。  
        舒利只放松了一天。第二天,她发现副市长家有一只猫!  
        舒利知道猫和老鼠的关系,她清楚地记得在爸爸去德国找歌唱家之前自己曾经被一只白猫逼困在墙角的恐怖场面。  
        舒利绝望了。她已经几天没吃东西,身上没一点儿劲儿,无力离开副市长家。  
        副市长的猫不大,名叫辰羽,走起路来风度翩翩,像个绅士。辰羽是副市长全家的宠物。副市长家有吃不完的食物,副市长太太经常抱怨辰羽吃得太少,没有隔壁某部长家的猫吃得多。  
        一天下午,副市长家静无一人。濒临饿死的舒利孤注一掷地爬向辰羽的饭碗。  
        舒利狼吞虎咽一气呵成吃完了那碗肉饭,当她转身时,发现辰羽就站在她身后。  
        “谢谢你!”辰羽一脸的由衷。  
        准备夺路而逃的舒利听到这句话壮着胆子呆在原地没动。  
        “你是谁?”辰羽问。  
        “我叫舒利。”舒利随时会跑。  
        “主人的新宠物?”辰羽问。  
        舒利这才知道他是一只不认识老鼠的猫。世界上不认识老鼠的猫越来越多。舒利弄不清这是进步还是退步。舒利想起曾听贝塔说过一个叫克里斯王国的猫国举国都不认识老鼠。  
        “不速之客。”舒利介绍自己。  
        “别走了,帮帮我。”辰羽请求舒利。  
        “帮什么?”舒利问。她觉得挺刺激。        
        “帮我吃东西。我刚才看你胃口很好。我的主人老嫌我吃的少。”辰列很真诚。  
        “我决定拔刀相助。”舒利答应了。  
        从此,舒利在副市长家落户。她每天帮辰羽吃饭。辰羽由此大受副市长全家赞扬。  
        相处的时间长了,爱情就自然而然产生了,而且爱得死去活来。  
        就在舒利和辰羽准备结婚之前,辰羽突然病了。一病不起。          第291集  
        歌唱家是猫的婶婶;  
        贝塔成为人到病除的神医;  
        电视使辰羽一蹶不振;  
        太太怀疑床下    
        大家听完了舒利的故事。  
        “我可以和他谈谈吗?”歌唱家问舒利。  
        舒利有点儿一朝被蛇咬十年怕井绳。她想起了图钉。  
        “歌唱家已经和贝塔结婚了。她现在是你的婶。”舒克告诉女儿。  
        “辰羽现在大约一个小时醒一次。快醒了。”舒利同意歌唱家和辰羽交谈了。  
        床上不知是谁翻身。皮皮鲁提醒大家说话小声。  
        歌唱家来到辰羽身边,观察他。  
        辰羽的嘴唇微微动了一下。        
        “快醒了。”舒利有经验地说。  
        “去叫贝塔。”歌唱家对舒克说。她已估计出辰羽的心病是什么。  
        “叫贝塔干什么?”皮皮鲁问。  
        歌唱家和皮皮鲁耳语。  
        皮皮鲁点头表示同意歌唱家的判断。  
        “舒克,你去五角飞碟值班。让贝塔来。”皮皮鲁说。  
        “为什么?”舒克问。  
        “贝塔能治辰羽的病。”皮皮鲁冲舒克一笑。  
        舒克没再多问,去五角飞碟换贝塔。  
        “让我给你女婿治病?”贝塔清楚自己的医疗水平。  
        “是你太太推荐的。”舒克说。  
        “原来贝塔对我们还有保留。”鲁西西逗贝塔。  
        “我也是头一次知道自己身怀绝技。”贝塔边说边往五角飞碟外边跑。  
        辰羽醒了。  
        舒利将歌唱家、皮皮鲁和贝塔介绍给辰羽。  
        辰羽第一次见这么小的人类,很吃惊。  
        “他是我丈夫。”歌唱家特别把贝塔和她的关系亮给辰羽。  
        “你嫁给老鼠?”辰羽跟睛放光。他已知道歌唱家是名人,大名鼎鼎的歌星贝一。        
        舒利一愣。她原以为辰羽不懂老鼠。  
        “种族不重要,重要的是品质。嫁给一个坏人绝对不如嫁给一只好老鼠。”歌唱家一字一句地说。  
        辰羽一跃而起,病态全无。  
        舒利大吃一惊。  
        “你的任务完成了,去换回舒克吧!”歌唱家对贝塔说。  
        当舒克、鲁西西和燕妮听贝塔叙述他往辰羽身边一站辰羽的病立刻痊愈时,没人相信。  
        “你去看看就坚信不疑了。”贝塔往五角飞碟外边推舒克。  
        当辰羽用洪亮的嗓音管舒克叫岳父时,舒克惊诧不已。  
        “谢谢你。”舒利向歌唱家道谢。  
        “是辰羽自己治好了自己的病。”歌唱家说。  
        “到底是怎么回事?”舒克问歌唱家。  
        “让辰羽告诉你们吧。”歌唱家说。  
        辰羽看了看舒利,他在请求舒利原谅后,把自己得重病的经过讲给大家听。  
        辰羽爱上舒利时,不知道舒利的家族名声不好。不久前的一天下午,辰羽从电视上得知老鼠是恶劣的种族,还知道了自己是老鼠的天敌。当他在电视屏幕上看清老鼠的形象时,傻了。  
        就像新郎在结婚前夕得知新娘是江湖大盗一样,辰羽有受骗的感觉。  
        辰羽想质问舒利,但他又不敢设想失去舒利后他怎么活,他太爱舒利了。和舒利如期结婚?辰羽不但觉得对不起自己,他还觉得对不起副市长。堂堂副市长的猫和老鼠成亲?  
        正如歌唱家所言,心病导致身病。辰羽为此一病不起。而舒利却一无所知。  
        歌唱家和贝塔的婚姻使辰羽走出了心病的误区。心病除,身病不药自愈。  
        舒利和辰羽相拥而泣。是喜泪。  
        舒克感谢歌唱家。  
        “注意,副市长醒了!”贝塔使用无线通话器通知皮皮鲁。  
        皮皮鲁提醒大家注意。  
        贝塔在五角飞碟里用遥感仪监测床上。  
        副市长的眼睛睁开了,他看看身边的太太,目光无意识地在屋里扫动。  
        副市长的日光落在窗户上。他看见了玻璃上的洞。看见了打开的窗户。  
        “有贼!”副市长大喊。  
        太太闻声从梦中急醒,她看卧室的门,插销忠于职守地守护在门上,说明贼没从门出去。  
        “在床下。”太太小声对副市长说。          第292集  
        副市长不同意夫人什么都不穿去抓小偷;  
        贝塔让电警棍和副市长接轨;  
        舒利不离开辰羽;  
        五角飞碟去美国    
        一听夫人说贼藏在床下,副市长脸白了。  
        “……怎么办?”副市长小声问夫人,他的声音明显颤抖。他在床下建立的小金库瞒着夫人。  
        夫人从枕头下抽出一根小型电警棍。这是警察局长送给她自卫用的。  
        “你去抓他?”副市长诧异夫人的胆量。  
        “我拿电警棍吓唬他,你冲出去喊人。”夫人制定战术。  
        “你应该穿上衣服吧?”副市长向睡觉喜欢什么都不穿的夫人建议。他怕坏人想入非非。        
        夫人采纳了副市长的意见。  
        贝塔在五角飞碟里将床上的一举一动尽收眼底。  
        “副市长和老婆把咱们当贼了。”贝塔对身边的鲁西西说。  
        “她想到床底下抓咱们?”鲁西西看着电脑屏幕上正在穿衣服的副市长太太问贝塔。  
        “她计划用电警棍吓唬咱们。”贝塔笑着说。  
        “快通知皮皮鲁吧?”燕妮担心在五角飞碟外边的皮皮鲁他们的安全。一个五十多岁的女人拿着电警棍,让人感到不寒而栗。  
        “她要是知道她的床下是五角飞碟。杀了她她也不敢摸电警棍。别担心,我逗逗她。”贝塔玩兴十足。  
        “别太惹事,悠着点儿。”鲁西西叮嘱贝塔。她想尽快去美国找约翰。  
        “放心,绝对恰到好处,点到为止。”贝塔现在政策性极强。  
        当副市长和太太准备同时行动时,太太突然鬼使神差地使用电警棍猛击夫君。  
        从未被电击过的副市长大叫。  
        太太惊讶自己的行为,可又无法将电警棍从丈夫身上移开。  
        副市长怒不可遏,一拳击向夫人头部。昏迷的夫人的手仍然死攥着电警棍电副市长。        
        床下的皮皮鲁知道床上出事了,他用通话器问贝塔怎么了。  
        贝塔告诉皮皮鲁副市长夫妇醒了,不知他们为什么在床上发生了战事。  
        “咱们该走了。”皮皮鲁对舒克说。  
        “你还呆在这儿?”舒克问舒利。  
        舒利看看辰羽,点头。  
        “我会好好待舒利的。”辰羽向岳父保证。看得出,辰羽为自己成为地球上第一只和老鼠结婚的猫感到自豪。  
        “经常给我打电话。”舒克对女儿说。  
        舒利点头。  
        “我们走了。”皮皮鲁向舒利和辰羽道别。  
        “谢谢你。”舒利特别向歌唱家道谢。  
        “希望你幸福。”歌唱家真诚地祝福舒利。  
        皮皮鲁、舒克和歌唱家回到五角飞碟上。舒利也到五角飞碟里和鲁西西、燕妮见面。  
        “祝你们到美国找约翰顺利。”舒利对鲁西西和燕妮说。  
        “好好活!”鲁西西对舒利说。  
        舒利离开五角飞碟。舒克眼角湿润。  
        “生养孩子的结果往往是操心和伤心。”贝塔说。  
        五角飞碟舱门关闭。  
        “他们在干什么?”皮皮鲁通过五角飞碟上的电脑遥感屏幕看见副市长和太太在床上肉搏。  
        “贝塔的杰作。”燕妮说。  
        “人家是副市长!”皮皮鲁瞪了贝塔一眼。  
        “副市长就都是好人?”贝塔反问皮皮鲁。  
        皮皮鲁没吭气。  
        “起飞吧!”舒克说。  
        贝塔在操纵五角飞碟起飞的同时,解放了副市长和太太。  
        五角飞碟回到皮皮鲁的住所。  
        “咱们该出发去美国找约翰了。”鲁西西对舒克和燕妮说。  
        贝塔恋恋不舍地离开五角飞碟。  
        鲁西西、燕妮和舒克同太家告别后登上五角飞碟。  
        “保持联系。”皮皮鲁说。  
        燕妮在舱门口和皮皮鲁吻别。  
        舒克驾驶五角飞碟起飞前往美国。  
        约翰还活着吗?          第293集  
        金发女士成为约翰的免费交通工具;  
        约翰在教堂里发现自己有特异功能;  
        约翰决定选择马里欧作朋友;  
        马里欧担心约翰是定时炸弹    
        三十多年前,皮皮鲁的爸爸将罐头小人歌唱家和约翰送出国。  
        皮皮鲁的爸爸把歌唱家送到柏林后,又送约翰到了美国纽约。  
        纽约的繁华景象使约翰大开眼界。他更加确信自己的选择没错。  
        “人生地不熟,行吗?”皮皮鲁的爸爸有点儿不放心,他问约翰。  
        “我觉得我天生就是这儿的人。您放心吧,您回国让皮皮鲁和鲁西西也放心!”约翰对皮皮鲁的爸爸说。        
        “你靠什么生活?”皮皮鲁的爸爸凭直觉感到纽约比柏林难立身,这是他为什么不担心歌唱家而担心约翰的原因。  
        “我有办法。我喜欢这个国家,我会在这儿生活得很好的。”约翰看着济济一堂的高楼大厦和川流不息的车水马龙说。  
        “祝你好运。”皮皮鲁的爸爸和约翰分手了。  
        “祝您一路平安!”约翰和皮皮鲁的爸爸告别。  
        皮皮鲁的爸爸走后,约翰开始审视自己周围的环境。他清楚,他得在纽约交一位朋友,否则他无法在这里立足。  
        约翰的智商不低,他知道品质是他选择朋友的第一顺序条件。他如果结交品质恶劣的朋友等于自杀。  
        约翰看见不远的地方有一座教堂,他认为信教的人都很善良。约翰从收音机里听过《圣经》的片断,他晓得行善和帮助别人是基督教的基本教义之一。  
        约翰决定去教堂选择朋友。  
        现在是白天,约翰明目张胆地在纽约的大街上行走显然不行。  
        约翰藏在路边的草丛里寻找机会。  
        一位金发女士朝这边走过来。约翰做好准备。当她的腿经过约翰身边时,约翰抓住了她的一条裤腿并迅速转移到裤腿的内侧。  
        约翰希望女士往教堂的方向走。  
        约翰的运气很好,金发女士果然按约翰的意志行动。她的身上有一股约翰喜欢的香水味儿。  
        约翰的感觉是荡秋千。没人发现约翰。金发女士不知道有人拿她当免费交通工具。  
        在金发女士经过教堂时,约翰跳车。  
        教堂里的场面令约翰肃然起敬,众多教徒在虔诚地做祷告。  
        约翰开始在教堂里物色自己的未来朋友。他悄悄从教徒们脚下走过,仰视他们的面容。  
        一个50岁左右的白人男子有幸被约翰录取了,约翰认为他外表慈祥,表情善良,举止文雅。  
        约翰比较容易地钻进了他放在身边的小手提包。  
        手提包里有钥匙和汽车驾驶执照。不可思议的事发生了,在伸手不见五指的提包里,约翰清楚地看见了驾驶执照里的名字:马里欧。  
        约翰开始以为是幻觉,当他坐在漆黑的提包里向四面看时,他看见了外边的景物。约翰的眼睛有透视功能!也就是通常所说的人体特异功能。他从前没发现。也许通过异地获得了这种功能。  
        约翰更加相信这个国家能给他带来好运。  
        祷告结束后,马里欧拿起手提包驾车回家。  
        约翰在提包里看马里欧开车,他还透过马里欧的衣服看见了他戴在脖子上的十字架项链。  
        约翰试图看车外的景象,没有成功。他的透视功能的有效范围保持在3米以内。  
        马里欧的家在纽约市内一座公寓的4层,3居室。  
        “回来啦?准备吃饭吧!”马里欧的妻子南希和刚进家门的丈夫打招呼。  
        马里欧冲妻子点点头,他将提包放进自己的房间,然后到卫生间洗手。  
        约翰在提包里观察马里欧的家,他如果发现什么不对劲的地方,不露面还来得及。  
        马里欧夫妻显然是良民。约翰在他的家里没发现枪支弹药和毒品。  
        厨房的香味对约翰的刺激程度告诉约翰他饿了。  
        马里欧从卫生间出来,他走进自己的房间换衣服。  
        约翰认为亮相的时机到了。  
        “马里欧先生,能让我出来吗?”约翰在提包里大声说。  
        马里欧一惊。在丈夫刚回家的时候家里出现了另一个男人的声音一般来说不是令人愉快的事。  
        “你是谁?你在哪儿?”马里欧做好迎战的准备。  
        “我在你的手提包里。请你放我出来。”约翰说。  
        “在手提包里?”马里欧小心翼翼地接近放在桌子上的手提包。  
        “你是什么东西?”马里欧不敢贸然打开手提包他怕炸弹。他的国家的炸弹是无坚不摧无孔不入。          第294集  
        约翰为自己的眼力自豪;  
        约翰保证不当第三者;  
        麦克唐纳的饭碗危险;  
        约翰要和神祗心照不宣    
        “我是约翰。”约翰在包里报名。  
        “约翰?什么约翰?”马里欧想像不出能躲在手提包里说话的人是什么样。  
        “你打开就知道了。”约翰说。  
        马里欧把妻子从厨房叫出来。  
        “什么事这么神秘?”南希以为丈夫在楼下捡了10万美元。  
        “我的手提包里藏着一个男人!”马里欧小声告诉太太。  
        “是吗?你把麦克唐纳带回来啦?”南希以为先生在逗她玩儿。麦克唐纳是世界著名侏儒影星,片酬超过史泰龙。  
        “南希你好,让我出来好吗?”约翰已经知道了马里欧太太的名字,他在提包里和南希打招呼。  
        南希手里的玻璃杯以身殉职。  
        马里欧示意妻子不要慌,他冲到桌子前突然打开提包。  
        他看见了约翰。  
        “你好,马里欧先生,我是约翰。”约翰站在包里对马里欧说。  
        “…你是…什么…东西……”马里欧问。  
        “和你一样,是人。”约翰说。  
        “这么小的人?”马里欧不信,他认为约翰是智能玩具。  
        南希凑过来看,她见到约翰十分惊讶又十分喜欢。  
        “这么小的人?真有意思!”南希情不白禁。  
        “能让我出来说话吗?”约翰不习惯在包里和别人交谈。  
        南希用两根手指轻轻将约翰从包里夹出来放在桌子上,约翰感到她的手指很柔软。  
        “说说你是怎么回事?”马里欧问约翰。  
        “我从中国来。我认为我的根在美国。我想在这里生活。我需要朋友。我希望你们能和我交朋友。”约翰言简意赅。        
        “你从中国来?怎么来的,坐国际航班?”南希兴奋。  
        “是的。有朋友送我来。”约翰看出马里欧夫妻对他很友好,他松了一口气。  
        “你的意思是你想同我们一起生活?”马里欧问约翰。  
        “对。可以吗?”约翰说。  
        马里欧看妻子。南希点头首肯。  
        “咱们现在是一家人了!”马里欧宣布。  
        “谢谢!也谢谢我的眼力。”约翰乐了。  
        “不过有个条件,用中国话说,你可不能当第三者。”马里欧开玩笑。  
        “我就是想当,也不具备条件呀!”约翰喜欢幽默。  
        “你们别只拿我开心呀!”南希抗议。  
        “我饿了,可以吃饭吗?”约翰用一家人的口气说。  
        “现在开饭!”南希拿起约翰就往餐厅走。  
        “我有吃醋的感觉!”马里欧跟在妻子身后。  
        开吃前,马里欧和南希做祷告。  
        在教堂时,约翰已决定自己也要信教,他需要一种冥冥之中的强大力量督促他做一个善良的人。但约翰不喜欢有组织的宗教和形式上的宗教,他觉得这些都是假招子。约翰想在行动上信教,想让自己选择的宗教住在自己心里,他要和神祗心照不宣。  
        约翰认为太阳的伟大之处在于它把光明撒向人间的同时不让任何人接近它。  
        饭后,约翰和马里欧夫妻聊天,给他们讲自己的传奇经历,讲皮皮鲁和鲁西西,讲另外4个伙伴。  
        见多识广的美国公民马里欧和南希都听傻了。  
        南希和马里欧给约翰讲他们的生活,讲在美国挣钱特别容易特别不容易,竞争极其激烈,失业的人很多。  
        “你不用发愁,你只要往好莱坞大导演面前一站,麦克唐纳就该失业了。”南希逗约翰。  
        马里欧眼睛一亮。          第295集  
        魔术般的纽约;  
        好莱坞的导演助手不傻;  
        汤约拿与约翰会晤;  
        施瓦辛格即将同约翰联袂主演《两极世界》:  
        豪华别墅对于约翰来说唾手可得    
        晚上就寝前,马里欧和南希为约翰安排住处。  
        南希用一个漂亮的纸盒子给约翰做了一间卧室。  
        “咱们3人各住一间屋子。”南希说。  
        这天晚上,约翰睡得很香。  
        第二天的日程这样安排,上午马里欧夫妇驾车带约翰游览纽约市容。下午由马里欧陪同约翰乘飞机去位于洛杉矶的好莱坞见导演,约翰想当拿巨额片酬的影星。想通过自己的劳动让自己和马里欧夫妇过好日子。        
        早餐后,马里欧和南希为约翰当导游逛纽约。南希驾车,马里欧坐在南希旁边用手护住站在驾驶台上的约翰。  
        约翰目不暇接,他惊讶人类的创造力,惊讶人类居然能把城市造成这个样子。  
        南希和马里欧开心地给约翰解说每一处地方,告诉他某某大厦的主人3年前还是一个穷光蛋,某某中心的亿万富翁董事长一个星期前破产跳楼了。约翰决心通过自己的努力在美国站住脚,当一个受人崇拜的英雄。  
        下午,马里欧带约翰乘飞机去著名影城好莱坞。  
        “好莱坞的导演成百上千,咱们找哪个?”到好莱坞后,马里欧问约翰。  
        “当然找最有名的。”约翰口气很大。  
        “那就是汤约拿了,今年刚得了奥斯卡奖。”马里欧说。  
        “就找他吧!”约翰俨然美国总统的气派。  
        大导演根本不会见贸然来访的无名鼠辈。  
        “您找汤导什么事?”大导演的助手问马里欧。  
        “向他推荐未来的影帝。”马里欧说。  
        “每天起码有几百位‘影帝’自荐,汤导不可能都见。您是‘影帝’?”助手脸上浮现出一丝显而易见的讥笑。  
        “汤约拿如果不见会后悔终身的。”马里欧一字一句地说。  
        “您能不能先让我初审一下?”助手说。  
        马里欧从衣袋里掏出约翰放在手掌上给助手看。  
        “您好!我是约翰,很高兴认识您。”约翰说。  
        助手二话没说,转身跑步去叫导演。  
        约翰的起点的确高,还没出道就见到了世界排名第一的导演。导演之间不是爸爸和儿子的关系,不存在“代”,只有名次。“代”是鱼目混珠泥沙俱下滥竽充数的叫法。  
        “这么小的人?”汤约拿见了约翰大吃一惊。  
        “见到您很荣幸。”约翰说。  
        “……”见过大世面的汤约拿居然说不出话来。  
        “我能演电影吗?”约翰同汤约拿。  
        “当然……”汤约拿已经在构思一部由施瓦辛格和约翰主演的有强烈反差效果的巨片,他甚至连片名都起好了,就叫《两极世界》。  
        “我将我的电话留给您。欢迎同我联系。”马里欧同汤约拿告别。  
        “等等,咱们再谈谈。”汤约拿不想让约翰走,他担心约翰落在别的导演手里。  
        “您给我打电话吧!”马里欧知道约翰的价值了,他很牛。  
        在回家的飞机上,马里欧告诉约翰他们应该拿汤约拿一把。        
        约翰和马里欧回家后,南希为约翰高兴。  
        “等我拿了第一次片酬,咱们买一座豪华别墅。”约鞫兴奋得连饭都不想吃。  
        “你的一次片酬大概够买几十座别墅的。”马里欧说。  
        “那么多?”南希惊讶。  
        “你如果见到汤约拿看约翰的眼光,你就相信了。”马里欧对南希说。  
        约翰无比自豪。他觉得美国是一个伟大的国家,只要有才能,就可以在这里出人头地。  
        入夜,约翰躺在床上睡不着,他想像自己成为影帝后的情景。想像皮皮鲁全家看他主演的电影的情景。约翰甚至准备邀请在柏林的歌唱家为他主演的电影唱主题歌。  
        约翰好像隐隐约约昕见马里欧在和南希在隔壁争论什么。约翰知道夫妻在夜间发生争执是不需要旁人劝架的。  
        约翰正在朦朦胧胧即将入睡之际,南希慌慌张张来到约翰的寝室旁边。  
        “约翰!你醒醒!”南希小声说。  
        “我还没睡着呢。”约翰睁开眼睛。  
        “你快跑!”南希说。  
        “跑?往哪儿跑?为什么要跑?”约翰莫名其妙。          第296集  
        美国的玻璃瓶;  
        南希收拾东西;  
        约翰不相信这样的保镖;  
        不能随便在合同上签字    
        “马里欧要软禁你!”南希说。  
        “软禁我?为什么?”约翰知道软禁的性质和拘留没什么两样,他认为这又是南希在逗他玩儿。  
        “刚才汤约拿来了电话.他要出9000万美元买走你。马里欧不干,马里欧要当你的经纪人,赚更多的钱。”南希说。  
        “这不挺好吗?”约翰说。  
        “马里欧怕你离开我们,他想把你关起来,彻底控制你,拿你当赚钱工具!”南希说。  
        “马里欧不会,你是逗我玩呢!”约翰根本不信。  
        “我不骗你!快跑!”南希急了。        
        “我没说过不同意挣大钱呀?”约翰纳闷。  
        “他怕你成名后就不要我们了!在美国,只要你成名了,会有很多人向你靠拢,靠你吃饭。马里欧怕别人从他手里抢走你。他要垄断你!也不能全怪马里欧,你的诱惑力太大了!”南希说。  
        约翰有点儿信了。  
        “你为什么要救我?”约翰问。  
        “你以为美国人都只认钱?我觉得自由最重要。即使你挣再多的钱,没自由有什么意思?”南希说。  
        约翰感动。  
        “你干什么?”马里欧闯进约翰的房间,质问南希。  
        “约翰快跑!”南希使劲儿抱住马里欧。  
        约翰现在彻底信了。  
        然而他已经跑不了了。马里欧轻而易举地将妻子打翻在地,他抓住了约翰。  
        “你放开我!你要干什么?”约翰在马里欧的手心里挣扎。  
        马里欧根本不答理约翰,他将约翰塞进显然是事先准备好的一个小玻璃瓶里,玻璃瓶的塑料盖儿上扎了几个洞用来给约翰供氧。  
        马里欧将瓶盖儿拧紧。  
        “你不能这样对待约翰!”南希冲上来抢玻璃瓶。  
        “南希,你不要这样,咱们应该共同拿他挣钱!你不是早就想住别墅吗?”马里欧先击败了妻子的进攻,然后做她的思想工作。  
        “约翰是人!你不能剥夺他的自由!”南希坐在地上大喊。  
        “他是人?他来咱们美国有签证吗?他有护照吗,就算他是人,也是非法人境者!”马里欧法制观念不弱。  
        “真没想到你是这种人!约翰并没有说不给咱们挣钱呀!”南希哭了。  
        “你太天真了!他真正认识到自己的价值后,就该变了。你忘了你舅舅是怎么和你妈妈分遗产的了?再说,你妈妈从来就看不起我,嫌我没钱,穷,我只不过想让她拿正眼瞧我!”  
        约翰在玻璃瓶里看着眼前发生的事,他感到脑细胞不够用。  
        “咱们分手吧!”南希从地上爬起来。  
        “离婚?”马里欧惊诧。  
        南希点头,她开始往一只旅行箱里装自己的衣服。  
        南希拎着旅行箱走到马里欧手上的玻璃瓶旁。  
        “对不起,约翰。美国让你失望了。记住,美国人不都这样。”南希隔着玻璃向约翰道别。她确信自己无法从马里欧手里拯救约翰。  
        约翰冲南希点点头,没话。        
        南希走了。在深夜走了。  
        又一个美国家庭破裂了。  
        马里欧对约翰说:  
        “南希误解了我的好意。在美国,发大财的人都会被人觊觎,都会有危险,都需要雇保镖。我要给你当保镖兼经纪人。”  
        约翰苦笑。他不是傻子。让妻子半夜离家的男人能给别人当好保镖?  
        “咱们得签个协议。”马里欧从小受到良好的契约教育。  
        约翰不说话。  
        “你怎么不说话?”马里欧发现约翰半天没说话了。  
        约翰看着马里欧。  
        “我现在起草一份协议,内容是你授权我担任你的惟一经纪人,期限60年。”马里欧一边说一边写。  
        约翰看着马里欧。  
        协议写完了。约翰拒绝签字。  
        “为什么?”马里欧问约翰。  
        约翰不吭气。  
        “你必须签字!”马里欧说,“你不签字就永远别离开这个玻璃瓶。我认为你到我们美国来不是为了住玻璃瓶的吧?”  
        约翰坚决不签字。他知道在法制国家不能随便签字。  
        约翰还决定绝食。  
        约翰会死吗?          第297集  
        约翰要当虽死犹荣的英雄;  
        山珍海昧的诱惑;  
        蒙面大盗在卫生间门口恭候马里欧;  
        约翰惨遭劫持    
        约翰铁了心以死抗争束缚。他觉得与其过没有人权没有自由的生活不如死。  
        这已经是约翰绝食的第5天了。他蜷缩在玻璃瓶里,奄奄一息。  
        马里欧傻眼了,他万万没想到小得微不足道的约翰有视死如归的胆量和毅力。  
        约翰的神智是清醒的,他从马里欧的表情上已经看出自己是这场较量的胜者。不管自己是生是死,赢家都是他约翰。人间输赢不以生死论。有的人活到90岁,但从他上小学3年级时那次向老师告密出卖同学起就已经输了自己的一生。有的人只活了20岁,但他反抗过父母包办婚姻,他是生命的冠军。  
        马里欧试图强行给约翰喂食,但约翰太小使他无从下手。  
        马里欧软硬兼施,约翰死活一句话不说一口饭不吃。  
        马里欧故意将囚禁约翰的玻璃瓶放在餐桌上让约翰看他吃饭,诱惑约翰投降。  
        约翰在玻璃瓶里看着马里欧面对美酒佳肴做饭前祷告,然后享受一桌山珍海味。约翰同玻璃瓶外边传来的香味儿搏斗他不停地咽吐沫。  
        “我发现你比较傻。饿死的人最惨了。其实你马上就可以不饿。”马里欧边吃边说。  
        约翰闭上眼睛。失去视觉支持的味觉容易战胜。  
        绝食的滋味实在不好受。选择中止进食的方式抗争,对手百分之九十五是魔鬼。  
        在绝食的第6天,约翰知道自己必死无疑了。他的躯下由于没有能量的支援已经疲软无力。他后悔不该来美国。他特别想念皮皮鲁一家。他不甘心死在美国。  
        第7天上午,在约翰准备和这个世界告别的时候,他的透视功能使他看见一个头上套着女士长筒袜手里拿着手枪的男性黑人在撬马里欧的家门。  
        马里欧此时在卫生间大便,毫无察觉。  
        约翰看出那蒙面黑人是入室抢劫的盗贼。        
        约翰在死前确信上帝是存在的。  
        蒙面黑人撬开了门,他进屋后迅速关上门,然后举枪站在原地判断屋内是否有人。  
        他听见了卫生间的抽水马桶的冲水声。  
        蒙面黑人确信其他房间没人后敏捷地侧身等在卫生间门边。当马里欧一边提裤子一边走出卫生间时,一枝冰凉的枪管与他的右脸颊接轨。  
        “别动,转过身去!脸冲墙站好!手放在背后!”蒙面黑人用极恐怖的嗓音说。  
        从小看暴力电影长大的马里欧顺从地按要求做了。  
        蒙面黑人从腰间掏出绳子将马里欧的双手在背后捆了。  
        盗贼让马里欧转过身面对他。  
        “钱在哪儿?”蒙面黑人问。  
        马里欧犹豫。  
        枪口抬起来指向他的头部。  
        “在卧室的床头柜的最下边一层。”马里欧出卖了家庭银行的方位。  
        “你带我去!”蒙面黑人用枪指着马里欧说。  
        马里欧无可奈何地走进自己的卧室,蒙面黑人跟在后边。  
        “在这里。”马里欧站在床头柜旁边。  
        蒙面黑人拿到了850美元。        
        “谢谢。”蒙面黑人挺知足。  
        当蒙面黑人走进约翰呆的房间时,连约翰也不知为什么自己人喊了一声“救命”。  
        蒙面黑人猛地用手枪指向发出声音的方向。他没有发现人。  
        “这屋里还有人?”蒙面黑人质问马里欧。  
        “没有。”马里欧说。他不想失去约翰。  
        “胡说!我明明听见有人喊救命!”蒙面黑人双手甲端着枪慢慢搜索房间。  
        他看见了玻璃瓶里的约翰。约翰看见他的罩着长筒袜的脸,很吓人,再加上那枝乌黑的枪口,约翰后悔了。他有出狼窝进虎口的感觉。毕竟马里欧是良民身份。  
        蒙面黑人将玻璃瓶拿到眼前看,然后把玻璃瓶装进自己的衣袋。  
        “那是孩子的玩具!你没用!”马里欧央求盗贼不要带走约翰。  
        “你当我是傻子?你家根本就没有孩子住!”蒙面黑人的观察力不善。  
        约翰看见蒙面黑人用随身带的宽胶带粘上马里欧的嘴,再把他捆在暖气管子上,尔后将电话线拽断。  
        “拜拜。”蒙面黑人和马里欧告别后转身摘下长筒袜收工,撤离。   第298集  
        约翰吃蒙面大盗的饭;  
        出人头地是约翰的墓地;  
        罗勃特有了不用拿手枪交谈的朋友;  
        约翰睡了18个小时    
        约翰重见光明时,已在蒙面黑人的住所了。  
        那黑人拧开玻璃瓶的盖子,将约翰从瓶子里倒在一张肮脏的桌子上。约翰打了一个滚儿。  
        “你会说话?”黑人问约翰。  
        约翰是生平第一次面对坏人,他很想胆怯,遗憾的是他已然没有胆怯的劲儿了。  
        “我叫约翰。我已经7天没吃饭了,请给我一点儿吃的。”约翰用对好人说话的口气向大盗求援。  
        “我叫罗勃特。我头一次见你这么小的人。怎么,那小子不给你饭吃?你能吃多少东西?一块面包我看够你吃一个月。哈哈。他们这些有钱人很怪,        钱越多越小气。”罗勃特给约翰拿来吃的。  
        约翰知道多日不吃东西后不能狼吞虎咽,为了身体的安全,他尽量控制进食的速度和规模。  
        “吃饱了?你他妈怎么回事?生下来就这么小?还是被哪个良心让狗吃了的科学家拿你做试验用了?”罗勃特见约翰吃完,问。  
        约翰刚才一边吃一边打量罗勃特和他的家。罗勃特的住所用脏乱差3个字形容最恰如其分。后来约翰得知,这是纽约曼哈顿哈林区。罗勃特本人的仪容也同马里欧有天壤之别,根本不像一面国旗下的公民。  
        约翰在寻找逃跑的机会。  
        “我是原装的小。”约翰回答。  
        “那人干吗不给你饭吃?”罗勃特问。  
        约翰不想骗他,罗勃特毕竟救了他。约翰将自己绝食的原因告诉了罗勃特。约翰想,如果罗勃特效法马里欧,他就继续绝食。  
        “原来我以为我是美国最坏的人,看来现在我可以升级到倒数第二坏了。”罗勃特说。  
        约翰欣赏他的幽默。  
        “你的体力恢复了?”罗勃特问约翰。  
        约翰点点头,他估计罗勃特该带他去见导演了。  
        “拜拜。”罗勃特说。  
        “……”约翰不明白。        
        “你可以走了。”罗勃特说。  
        “走?去哪儿?”约翰疑惑,他不知道罗勃特想干什么。  
        “怎么,还舍不得我这个破家?你自由了,想去哪儿就去哪儿吧!”罗勃特说。  
        “真的?”约翰不相信。  
        罗勃特点点头:“别去警察局告发我。我看你不是忘恩负义的人。”  
        “我不是。”约翰说。  
        “祝你好运。”罗勃特冲约翰挥挥手。  
        “你不想拿我挣钱?”约翰极为纳闷。  
        “老弟,你把我看扁了。我就是再坏,也还没堕落到欺负你这么小的人的地步呀!我干活的时候先看那家有没有孩子的房间,如果有孩子,我就放了那家。和未成年人过不去的人才是地道的坏蛋。你虽然不是孩子,但我可以特批你在我这儿享受孩子的待遇。哈哈……”罗勃特一边喝酒一边说一边狞笑。他把入室抢劫叫干活。  
        约翰想起了罗勃特揭穿马里欧说约翰是孩子玩具时的快节奏。  
        沉默。  
        “我如果不走呢?”约翰突然说。  
        “不走?”罗勃特一愣,“我可不需要朋友.我这一辈子是被朋友坑大的。在这个世界上,我和别人相处时,只有手里拿着枪心里才踏实。”  
        “我不走了。我觉得对我来说和你在一起最安全。”约翰说。  
        “我是坏人!”罗勃特显然对于约翰要和他在一起感到吃惊。  
        “没错。但你不坏。”约翰说。  
        罗勃特放下手里的酒瓶,注视了约翰足足5分钟没说话,看得约翰直发毛。  
        “你应该去好莱坞混。”罗勃特说,“你能当人影帝。”  
        “我还没演电影就差点儿把命送了,我如果当了影帝人类还不为了争夺我将我五马分尸?我想明白了,我绝对不能靠展露头角挣钱谋生,出人头地对我来说意味着出殡葬地,我只能采取隐蔽方式生存。”约翰说。  
        “我破格收留你了!”罗勃特说,“反正你这么小,想害我也害不了。”  
        “谢谢。”约翰心里踏实了。  
        “我出去采购点儿吃的。”罗勃特说,“你睡会儿。”  
        约翰一觉睡了18个小时。          第299集  
        约翰是纽约警局的卧底探员;  
        有个坏爸爸不如没爸爸;  
        罗勃特的破车行驶在高速公路上;  
        约翰在大西洋赌城浪费形容词    
        “我养了个老爷。”罗勃特对睡醒的约翰说。  
        约翰看表。  
        “有脱胎换骨的感觉。”约翰伸懒腰。  
        “在坏蛋家里脱胎换骨?”罗勃特冲约翰翘大拇指。  
        “你为什么靠干这个过日子?”约翰问。  
        “瞧,来了吧!我现在怀疑你是纽约警察局派来卧底的。”罗勃特说完哈哈大笑。  
        “能回答我的问题吗?”约翰继续问。  
        “找不到工作。”  
        “怎么会?这么繁华的城市!”        
        “我没有大学学历。现在找个像样的工作必须有学历。”  
        “找力气活苦活干呀!”  
        “力气活苦活都被外国人特别是你们中国人囊括了。他们出价底,我竞争不过他们。算了,不说这些没意思的话了,咱们吃饭喝酒!明天我带你去大西洋赌城开开眼,你也试试手气。”  
        约翰陪罗勃特喝酒。  
        约翰从罗勃特口中获得了许多美国故事,罗勃特一边喝酒一边讲给他听。罗勃特从小没有父亲,到现在他也不知道自己的爸爸是谁。他为此感到沮丧。  
        “没关系,我也没有爸爸。依我看,世界上的爸爸好坏各占一半儿。与其摊上个坏爸爸还不如没爸爸。”约翰安慰罗勃特。  
        “这倒是。”罗勃特活这么大几乎是头一次被人安慰,他甚至有点儿不知所措。  
        “有的爸爸酗酒后打孩子,有的爸爸因为孩子学习成绩不好骂孩子,还有的爸爸由于孩子是女性而杀了孩子。”约翰看着罗勃特说。  
        罗勃特感激约翰对他说这些话,他从约翰身上得到了过去他从别人身上得不到的东西。世界上的东西不在于大小多少,关键是与其相处时有没有温暖感。当一个能给人以温暖的人有3件法宝:鼓励。肯定。安慰。  
        第二天,罗勃特驾驶他的破车带约翰去著名的大西洋赌城。罗勃特一有了钱就去大西洋赌城赌博。  
        汽车奔驰在高速公路上。罗勃特一边吹口哨一边开车,约翰坐在方向盘前的驾驶台上饱览美国风光。罗勃特用细胶条当安全带将约翰固定在驾驶台上。  
        “对了,我不能把你放在手掌上参观赌城呀!”罗勃特想到了约翰的安全问题。  
        “我藏在你的头发里,坐在你的耳朵上。你还可以用胶条把我固定在你的耳朵上。”约翰有经验,他曾经使用这个方法帮助鲁西西说外语震外国教育考察团。  
        “我发现你老弟的脑子够用。以后咱俩搭档干活吧!”罗勃特说。  
        “你想拉我下水,我不干。我觉得你完全可以换一种方法挣钱。”约翰从昨天晚上就开始想给罗勃特调工作。  
        “我能干什么?”罗勃特笑。  
        “我还没想好,反正我要让你改邪归正。你救了我,我也要救你,这叫回报。”  
        “你可真会往自己脸上贴金,这不叫回报,叫恩将仇报。要不你就是警察局新研制出的微型警探,像机械战警那样的机器人。哈哈……”        
        约翰对他同罗勃特的这段对话有一种宗教体验的感受。  
        约翰和罗勃特开心地行进在去大西洋赌城的途中。  
        两个小时后,罗勃特的破车安抵大西洋赌城。  
        下车前,罗勃特将约翰固定在自己的右耳朵上,再用长发盖住。  
        罗勃特从身上掏出手枪藏进车座下。  
        “干吗缴自己的枪?真的回头是岸了?”约翰问。  
        “去赌场有两个规矩,一是不能带枪,二是不能带照相机。”罗勃特说。  
        “为什么?”  
        “赌场老板怕你输急了将赢家就地正法。”  
        “干吗不让带照相机?”  
        “你把人家拍下来,人家日后竞选总统时,你拿这张照片破坏选民对他的信心,谁还敢来赌场?”  
        “你反正无此后顾之忧。”  
        “那可不一定。你能百分之百肯定我将来不会出任美国总统?美国就这点儿好,人们最愿意接受意想不到的事。”罗勃特边说边走向一座大赌场。  
        约翰看见了豪华得无与伦比的大西洋赌城。他惊讶人类居然斥巨资将赌场建造得如此富丽堂皇,可见绝大多数人生的本质就是赌博。  
        “漂亮吧?”罗勃特问约翰。        
      “真让人难以置信。”约翰说。  
        “真正让人难以置信的在里面。我劝你现在别太浪费形容词。”罗勃特忠告朋友。          第300集  
        罗勃特出师不利;  
        约翰扭转乾坤;  
        老板忍痛投赌博院士的赞成票;  
        无声手枪的威胁    
        赌场里边真真让约翰大开了眼界。最让约翰吃惊的是赌场的气氛。赌场是地球上惟一氧气充足却又让人感到心慌气短的地方。  
        各种赌博方法应有尽有任君选择。  
        “咱们试试运气。我觉得你能给我带来好运气。”罗勃特说着走向一张扇形的赌台,“咱们玩二十一点。”  
        约翰发现罗勃特一进赌场精神立刻进入兴奋状态,额头发亮。  
        那张赌台旁已有几个人在赌。一位身穿白衬衣袖口绣着花边的赌场工作人员通过发扑克牌主持赌局。  
        约翰看见赌台旁的赌客有人欢喜有人愁。  
        罗勃特购完筹码后在一个空位子上入座。他开始下注。  
        约翰是生平第一次进赌场,但他很快就看懂了二十一点的赌博方法。关键是发牌的那位赌场工作人员手边的木匣子里的未发出的扑克牌和已经发出的面朝下的扑克牌。你无法预先知道这些牌的身份,只能靠运气。  
        罗勃特一上来就输了一轮。  
        “倒霉!”罗勃特嘟囔。  
        就在这时,约翰的透视功能使他看见了赌台上的所有扑克牌的内容,不管是在木匣子里的还是在赌台上的。  
        “在你右手边上的那个圈儿里下注!”约翰趴在罗勃特耳朵上小声说。  
        “为什么?”罗勃特问。  
        赌友们都看罗勃特,不知他和谁说话。  
        罗勃特自知失言,不敢再和约翰交谈了。  
        “我让你在哪儿下注你就在哪儿下,包你立马成为大款。”约翰口气极大,俨然一赌侠。  
        罗勃特拿起一个筹码放在约翰看好的领域。  
        “太少!都押上!后边的牌是连续3张圈儿!”约翰说。        
        罗勃特根本不信,坚持只下一个筹码。  
        约翰果然正确!  
        罗勃特惊诧不已。  
        “从现在开始,你完全听我的指挥。”约翰命令罗勃特。  
        罗勃特用点头的方式表示在赌场对约翰称臣。  
        奇迹出现了。  
        罗勃特连连得手,他面前赢得的筹码堆成了小山,保守估计也有5万美元。  
        罗勃特乐疯了,他容光焕发神采奕奕,起码年轻了10岁。他恍若在梦中,左右逢源叱咤赌城,美钞雪片般飞来,挡都挡不住。  
        在罗勃特四周有数百人围观,他们惊异罗勃特的赌技和运气。  
        一位赌场保安人员打电话要求监视室密切注视15号台——罗勃特参赌的赌台,检查罗勃特是否作弊。  
        立刻,几乎整个赌场的所有隐蔽摄像机都将镜头对准了罗勃特。  
        专家们全方位观察罗勃特。  
        “他没有看第一张牌。”一位教授级赌博专家看着屏幕说。  
        “他是一个人,没有人和他联手作弊。”一位院士级赌博大师面前有7台不同角度的屏幕,他一边观察罗勃特一边下结论。  
        “他不可能老赢!不是作弊是什么?”赌场老板气急败坏,他痛恨手下找不出罗勃特的纰漏。  
        “想个办法中止他。”副总经理建议。  
        “围观的人太多,这样做会得罪顾客,有损咱们的声誉。”赌博教授反对。  
        老板认为赌博教授的话对。来赌场的人大部分输,当他们看到某个人大赢赌场时,会感到解气。赌场如果在这个时候没有任何理由地干扰赢家继续赌,就会触发众怒,于今后的经营不利。  
        “让他赌吧!”老板恶狠狠地说。  
        几乎整个赌场的顾客都闻讯聚集到罗勃特身边观赌,他们对于罗勃特的运气瞠目结舌。  
        罗勃特每赢一次,大家就慷慨地送给他海啸般的掌声。  
        “我觉得你可以见好就收了,你得让赌场活命呀!”约翰提醒罗勃特,“你大概已经挣了20万美元了!”  
        罗勃特想再赌最后一次。  
        “这次你自己赌。我该休息一会儿了。”约翰说。  
        罗勃特立刻不赌了。  
        赌场老板咬牙切齿地松了一口气。  
        当罗勃特拎着装有20万美元现金的手提箱离开赌场时,受到了赌客们的夹道欢送。        
        罗勃特进人自己的破车后,一个彪形大汉趴在车窗上用一枝装着消音器的大口径手枪对着罗勃特的头说:  
      “给你一个忠告,不要再来这家赌场了!”   第301集  
        跨入小康的罗勃特;  
        手持红牌的赌场老板;  
        价值5万美元的一个面包;  
        用手枪为赠款保驾护航    
        罗勃特没有理会那厮,他正处于极度兴奋状态,能宽容一切。  
        汽车刚上高速公路,罗勃特就迫不及待地将约翰从耳朵上拿下来放在驾驶台上。  
        “老弟,你是天才!你哪儿来的这本事?”罗勃特神采飞扬地对约翰说。  
        “我的眼睛能透视。”约翰说。  
        “不可思议。”  
        “不可思议的事还在后边呢!请节省形容词。”  
        “哈哈……”  
        “罗勃特,我又该当警察了。”        
        “你说。”  
        “你以后不用再操持旧业了,咱们就像今天这样搭档在赌场赚钱谋生。”约翰说。  
        “同意换工种。下一步准备竞选参议员。”罗勃特给自己办调动手续。  
        “你是一个好坏蛋。”约翰为罗勃特回头是岸高兴。  
        “你是一个坏好人。”  
        “马里欧才是坏好人。”  
        “这个世界上的坏好人越来越多。”罗勃特无缘无故按喇叭。  
        “小心警察罚你!”  
        “咱们现在怕罚?”  
        “那也不能这么花钱呀!对了,你准备怎么花这些钱?”  
        “先买辆好车。再买套公寓。再买……”  
        “最好再买个老婆。我觉得你应该结婚。”  
        “别害我了。我绝对不结婚。这么说吧,结婚奉质上是赌博。领结婚证是一赌,拿自己下注。生孩子是二赌,拿亲骨肉下注。我就是我妈第二次赌博的赌注。她老人家赌输了,我这个赌注倒霉。我觉得作为爹妈如果不能为自己的孩子提供好的前途,就他妈根本没资格生孩子!”,罗勃特又莫名其妙按喇叭。 
        约翰不说话了。他认为罗勃特的话有道理。结婚是人生的一次赌博。赌配偶的品质。赌配偶的才能。赌配偶的遗传基因。赌配偶的运气。赌配偶的健康。  
        在一个星期内,罗勃特和约翰买了新车和公寓。  
        钱花光后,他们又去大西洋赌城挣钱。再花再挣。直到大西洋的所有赌场都将他们拒之门外。  
        “咱们去内华达州的拉斯维加斯赌城!”罗勃特拿出东方不亮西方亮的劲头。  
        很快,拉斯维加斯赌城所有赌场的老板也相继对罗勃特亮出了红牌。  
        罗勃特很快成为任何赌场都不欢迎的人。  
        罗勃特和约翰前前后后在赌场赚了数百万元。罗勃特有钱后非常慷慨,凡是对他有过一丁点儿好处的人他都加倍回报,大把送钞票给人家。  
        罗勃特生长在哈林区,有一家小店的店主曾经在罗勃特5岁时给过罗勃特一个面包,那次罗勃特已经3天没吃饭了。  
        罗勃特驾车和约翰找到那家小店,那店主居然还在,已是步履蹒跚风烛残年的老者。小店的规模依然如故,看得出是惨淡经营。  
        “您还认识我吗?”罗勃特问年迈的店主。  
        店主眯起眼睛看罗勃特,他摇头。  
        “您在我5岁的时候白给过我一个救了我的命的面包。这足我还您的钱。”罗勃特从手提箱里掏出一个特大号信封塞进老人的手中。  
        信封里是5万美元。  
        老人显然已经将给罗勃特面包的事忘得一干二净。他打开信封,吓了一跳。  
        “这…我…不要……”老店主拒绝。  
        “这是那一个面包的利息,连本带利。”罗勃特转身走了。  
        老店主腿脚不便,追不上罗勃特。约翰看见他老泪纵横。  
        世界上终于有了价值5万美元的面包,约翰有荡气回肠的感觉。约翰的眼睛湿润了,他清楚,真正值钱的不是那个面包,是老店主的心,还有罗勃特的意。  
        一次,罗勃特驾车和约翰外出旅游。途中汽车的电瓶没电了。罗勃特到路边一家专营汽车电瓶的小店买新电瓶,罗勃特付款时由于粗心多付了50元。当罗勃特走出商店时女收款员叫罗勃特,还给他50元。罗勃特大为感动,当即拿出1万美元送给那中年女士作为酬谢。女士坚决不要。罗勃特掏出手枪逼迫她收,女士万般无奈只好收了。  
        上路后,约翰问罗勃特:  
        “你也忒大方了吧?”  
        “对于心眼儿好的人,就是要重奖。我穷的时候想,如果哪天我有了钱,就干这件事!现在我发了,能如愿以偿了。现在我才知道,真正享受的事是他妈给予!送别人钱的时候心里真痛快!”  
        “看着前边说话!小心追尾。您前边是一辆劳斯来斯!”约翰提醒罗勃特。  
        罗勃特吹了声口哨。  
        约翰发现,罗勃特的火爆脾气近来很少发作。  
        本事越大,脾气越小。约翰想。          第302集  
        罗勃特不想让房地产商跳楼;  
        约翰背叛自己的肤色;  
        纽约的所有保险柜都有约翰和罗勃特的股份    
        约翰就这么和罗勃特过着无忧无虑的开心日子,直到有一天公寓的房地产商找罗勃特。  
        ‘什么事?”罗勃特问登门求见的房地产商。  
        “…是这样…请原谅……”房地产商说话颇费踌躇。  
        “请痛快说。”罗勃特鼓励客人。  
        “如果您能将这座…公寓…出售给我……,我高价收购……”房地产商吞吞吐吐依然。  
        “为什么?”罗勃特问。  
        “…真对不起……”房地产商不自然。  
        “我明白了,自从我买了这套公寓,这整栋楼的房价由于我这个黑人的入住而价格大跌了吧?你要赔本了?”罗勃特打开酒瓶盖。  
        “是这样。人家一听说这楼里住了户黑人,就不买了。原有的住户也要搬走。”看见罗勃特明白,房地产商索性挑明了说。  
        “我如果不走呢?”罗勃特喝了一大口酒。  
        “那我就得倾家荡产了!我是小本生意,我把所有资金都投在这栋楼上了!我看出您特有钱,您不会见死不救的……”房地产商一脸的妻离子散。  
        罗勃特从抽屉里拿出房契,当着房地产商将房契撕得粉碎。  
        房地产商脸煞白。  
        “这房子白送你了!”岁勃特喝酒。  
        “我付钱……”  
        “滚!”罗勃特怒吼。  
        房地产商踉跄着夺门而出。  
        罗勃特将酒瓶摔得粉碎。  
        “为什么轰咱们?”约翰问。  
        “这是富人社区!”  
        “咱们现在是富人呀!”  
        “我是黑人!富白人不愿意和黑人住在一个社区!黑人住多了,整个社区的房地产准掉价!”  
        “什么事儿!王八蛋!”约翰发火了,他头一次骂人。        
        “明天咱们同哈林区住!”罗勃特说,他反而平静了。  
        “他们干吗歧视咱们黑人?”约翰愤愤不平。  
        “你是白人。”罗勃特纠正约翰的色盲。  
        “我宣布我现在加人黑人籍。”约翰光荣起义。  
        “咱们现在就走,我一分钟也不愿意在这儿呆了。”罗勃特忽然说。  
        “我也是,走!”约翰同意。  
        约翰和罗勃特又回到了哈林区住破旧的房子。  
        由于不能再去赌场挣钱,加上罗勃特喜欢用钱替天行道,罗勃特和约翰的财政渐渐出现了赤字。  
        “告诉你一个好消息,咱们总共还有10美元。”一天,罗勃特向约翰报喜。  
        约翰不说话。  
        “想挣钱的招儿哪?”罗勃特问约翰。  
        “我是逼上梁山。”约翰咬牙切齿。  
        “逼上什么山?”罗勃特不懂中国成语。  
        “咱们只有去向富白人借钱过日子了。”约翰懒得给罗勃特扫盲,他所答非所问。  
        “向富白人借钱?”  
        “当然是不打借条的借。”  
        “偷?你这不是拉我上贼船吗?”  
        “看来我必须给你解释逼上梁山的含义了。”  
        罗勃特后来说“逼上梁山”是人类最伟大的词汇。  
        好端端的约翰在美国彻底学坏了。他开始和罗勃特联手入室行窃。他们专偷富白人。方法是趁主人不在家时开人家的保险柜“借”款。  
        约翰的透视功能使他能清晰地看到保险柜号码锁的内部结构进而得知打开保险柜的密码,他将密码告诉罗勃特,罗勃特就像开自己家的保险柜那样易如反掌地开人家的保险柜。  
        连续的保险柜失窃案令纽约警察局大为恼火和头疼。约翰和罗勃特连联合国总部也光顾过了。  
        分工侦破系列保险柜失窃案的警官认定诸案均系一人所为,他们觉得不可思议的是这贼怎么什么保险柜都会开,而且开的速度比主人还快!  
        警察局悬赏50万元给提供保险柜窃贼线索的人。  
        约翰经常在夜晚想起皮皮鲁全家,他有时也内疚,感到对不起皮皮鲁和鲁西西。每当这种时候,他就掌“逼上粱山”宽慰自己。  
        皮皮鲁和鲁西西无论如何想不到约翰在美国纽约成了令警察闻风丧胆的保险柜大盗。          第303集  
        罗勃特骂美式足球运动员;  
        约翰向公蟹队派汉奸;  
        保险柜变成了保证危险柜;  
        罗勃特与警察局口头签约    
        约翰和罗勃特就这么令纽约警察局束手无策了5年。  
        他俩肆无忌惮地想开谁的保险柜就开谁的保险柜。他们从不拿光保险柜里的钱财,他们从不积累财富,总是花光了上一次的“收入”再开下一个保险柜。他俩仍然住在哈林区的老房子里,悠哉悠哉。  
        发生在帕克大道和麦迪逊大道之间的纽约东62街的一次突发事件改变了约翰和罗勃特的生活。  
        那是一个秋天的下午,黄色的阳光在屋里陪约翰和罗勃特悠闲地看一场美式足球比赛的电视转播。  
        “那小子太臭!教练怎么会用他!”罗勃特指着电视屏幕上的一名运动员说。  
        “他是我派去的。”约翰调侃。  
        约翰和罗勃特打了赌。约翰押公鳖队,罗勃特押疯蝇队,赌注是买一个面包的钱。  
        正当比赛进行到白热化阶段时,电视台突然中断了转播。一位女播音员出现在屏幕上。  
        “出大事了!底线是总统遇刺。”罗勃特极富经验地对约翰说,“不信咱们打赌。”  
        约翰还没来得及下注,女播音员的声带就宣告罗勃特败北。  
        女播音员表情焦灼地告诉市民,电视台之所以中断橄榄球比赛的转播,原因是现在本市发生了一件人命关天的事:位于纽约东62街的一家公司今天上午购买了一台一人高的大型保险柜,该保险柜暂时放在公司的大厅里,准备下午移人财务室。保险柜没有关门。公司某职员的两个学龄前孩子刚才随妈妈来公司办事,趁妈妈不在时,其中一个6岁的女孩儿钻进保险柜里关上柜门和4岁的弟弟玩捉迷藏。弟弟发现姐姐在保险柜里,他无意识地拧了保险柜的号码锁,保险柜打不开了!该公司现在同本公司惟一知道保险柜号码的职员联系不上,而出售该保险柜的公司的规矩是,保险柜一旦售出,该保险柜在该公司存档的密码立即销毁,以解除购买保险柜顾客的后顾之忧。据专家估计,经过计算保险柜里的氧气容量。那女孩儿只能在保险柜中存活45分钟。现在已经过去19分钟了。电视台代表孩子的母亲向社会求救。  
        电视屏幕上出现了抢救被困女孩儿的场面。几位保险柜专家在保险柜的号码锁前一筹莫展。电视台的现场记者对电视观众说这是一种高科技保险柜新产品,就连它的设计者在不知道密码的情况下也无法让它芝麻开门。  
        孩子的妈妈在屏幕上泣不成声。  
        有人提出使用电焊枪,专家说电焊枪产生的高温能烤死孩子。  
        约翰看罗勃特。罗勃特看约翰。  
        “咱们应该救她!”两人异口同声。  
        罗勃特站起来准备走。  
        “警察一定在那儿等咱们。”约翰提醒罗勃特。  
        罗勃特又坐下了,他相信约翰的这个判断。为抓不到保险柜窃贼而苦恼的纽约警察局不会放过任何与保险柜有关的事件。何况即使警察不去现场,电视台的摄像机镜头也会始终对着开保险柜救孩子的罗勃特。警察会对轻而易举就打开保险柜的罗勃特无动于衷?即使警察无动于衷,那些丢了保险柜里的钱财的失主也会无动于衷?  
        “咱们不去了?”罗勃特问约翰。  
        “凶多吉少。”约翰说。        
        电视屏幕上出现了神通广大的记者不知从哪儿弄来的那遇难女孩儿的天真可爱的照片。那女孩儿在保险柜外边时没人觉得她可爱,进了保险柜就立即可爱了,可爱得牵动亿万人民的心。  
        “咱们去!不救她我一辈子不得安心!”罗勃特看表。  
        “先和警察谈好条件!”约翰建议。  
        罗勃特同意。他给警察局打电话。  
        “你好!这里是纽约警察局。”  
        “我是你们抓不着的那个随便开别人保险柜的市民,我现在想去救那个被关在保险柜里的女孩儿。我的条件是你们不能抓我。”罗勃特开门见山。  
        “…请你等一下……”那警察显然做不了主。  
        “快点儿,保险柜里的氧气没了!”  
        “如果你能救出那孩子,我们不抓你。”警察局同意了罗勃特的条件。  
        罗勃特挂上电话,将约翰藏进头发里。  
        “警察说话算数吗?”约翰不放心。  
        “应该没问题吧!”罗勃特跑步出门。  
        “但愿。”约翰抓紧罗勃特的耳朵。  
        罗勃特驾车风驰电掣般赶往东62街。  
        那家公司的门口人山人海,电视台连直升机都出动了。不断增加的媒体使出浑身解数独辟蹊径通过变换报道角度与同行竞争。          第304集  
        够活1分钟的氧气;  
        见风使舵的记者;  
        罗勒特的表现极有修养;  
        约翰给自己拔牙;  
        鲁西西和约翰辩论    
        当人山人海得知罗勃特有办法打开保险柜时,立即给他让开一条路并用掌声鼓励。  
        罗勃特飞步跑到保险柜旁。专家告诉他,保险柜里的氧气只够女孩儿活4分钟了。  
        约翰用1分钟看清了保险柜的密码,他告诉罗勃特。  
        罗勃特用两分钟打开了保险柜。  
        女孩儿已经奄奄一息。医护人员立即对女孩儿实施现场抢救。女孩儿得救了。  
        记者激动地对着摄像机镜头极力渲染现场的喜庆气氛。  
        罗勃特成了英雄。  
        就在罗勃特被记者团团围住不得脱身时,几名警察给他戴上了手铐。  
        记者们一个个瞪大了眼睛。  
        警察告诉记者,罗勃特就是横行纽约数年的保险柜大盗,现在正式落网。  
        这戏剧性的变化令记者们狂喜不已。  
        约翰被这突如其来的变化惊呆了,他觉得真正的坏人不是罗勃特。  
        戴着手铐的罗勃特很平静,他除了对给他戴手铐的警官用很小的声音说了句“小人”外,没有说其他的骂人话。  
        有记者问罗勃特现在的感受。  
        “值得。”罗勃特吐出两个字。  
        记者又采访获救女孩儿的母亲。  
        “真没想到他是个坏人!”那母亲愤慨中透出一分遗憾,遗憾自己的孩子是被坏人救出的,遗憾孩子没能获得一个英雄的救命恩人并终生以此为荣。  
        罗勃特被警察押上了警车。  
        “你应该现在离开我,到里边会给我剃光头的,你就无处藏身了。”罗勃特小声提醒约翰。  
        “我不走。我有办法。”约翰义愤填膺。  
        数月后,法院以盗窃罪判罗勃特35年徒刑。当法官宣判时,藏在罗勃特身下的约翰咬碎了自己的一颗牙,他义无反顾地把碎牙咽进胃里。  
        罗勃特开始在监狱服刑,约翰陪他。一陪就是30年。  
        如果没有约翰,罗勃特早就自杀了,他对自己在这个国家的生存前途失去了信心。这是罗勃特告诉约翰的。  
        在一个月黑风高的晚上,体积和约翰一样大小的鲁西西出现在约翰面前。约翰认定自己是在梦中。  
        鲁西西、燕妮和舒克驾驶五角飞碟在美国疯找约翰。在经历了无数次无效劳动后,他们找到了这座监狱。  
        “约翰,你不认识我了?我是鲁西西!”鲁西西想尽快驱散约翰脸上的疑惑。  
        “我知道你是鲁西西。”约翰说。  
        “那你为什么不激动?”鲁西西纳闷。  
        “在梦里有必要激动吗?”约翰经常在梦中碰见不可思议的事,他的人生哲学是以平静的心对待梦中的事。  
        “这可不是梦!”鲁西西纠正约翰的误区。  
        约翰执迷不悟。  
        经过一个小时的辩论,鲁西西才使约翰确信他是处于清醒状态。  
        “你怎么也成罐头小人了?”约翰问鲁西西。        
        鲁西西将原因告诉约翰。  
        “世界真奇妙。”约翰感叹。  
        “你怎么在监狱里?”鲁西西问约翰。  
        约翰将原因告诉鲁西西。  
        鲁西西叹了口气。她钦佩约翰的义气。约翰在监狱里毕竟陪了罗勃特30年。  
        “还有5年罗勃特就出狱了。”约翰说。  
        “原来地球上的监狱不光是关坏人的地方。”鲁西西说。  
        “监狱是人生最好的高等学府。”约翰向鲁西西传经送宝。  
        鲁西西默默不语地环顾简陋的狱室。  
        “对了,你是怎么来美国的?”约翰问鲁西西。  
        “我是专门来美国找你的。”鲁西西将五角飞碟、皮皮鲁的现状、舒克和贝塔、歌唱家和燕妮等等都告诉给约翰。  
        约翰听傻了。  
        “皮皮鲁真的发明了五角飞碟?”约翰对五角飞碟表现出极大的兴趣。  
        “当然。五角飞碟就在床下。”鲁西西说。  
        “我在皮皮鲁上小学时就看出他不是等闲之辈!”约翰进监狱30年来头一次兴奋,“我可以看看五角飞碟吗?”  
        “太可以了,我们还要用五角飞碟接你回国呢!”鲁西西说。  
        “约翰,你在和谁说话?”睡在床上的罗勃特醒  
        “和朋友。从中国来的朋友。就是我和你说起过的鲁西西。”约翰将鲁西西介绍给罗勃特。  
        “你没说过鲁西西也是罐头小人呀?”罗勃特揉眼睛。  
        “皮皮鲁现在是业余科学家,有能让人变小再能恢复原样的药。”约翰说,“对了,鲁西西,把你的药给罗勃特吃点儿,让他变小后越狱。”  
        “你要吗?”鲁西西愿意帮助罗勃特。          第305集  
        罗勃特谢绝越狱;  
        约翰和皮皮鲁通电话;  
        鲁西西在纽约吃忆苦饭;  
        约翰问心有愧    
        “不要。”罗勃特摇头。  
        “为什么?”鲁西西和约翰异口同声惊讶。  
        “30年我都坚持下来了,要越狱早越了。还剩5年,一咬牙就过去了。现在出去还得东躲西藏,弄不好再抓进来,这辈子就交给监狱了。”罗勃特说。  
        鲁西西点头。约翰摇头。  
        “我去参观五角飞碟。”约翰对罗勃特说。  
        鲁西西和约翰进入五角飞碟。鲁西西将舒克和燕妮介绍给约翰。  
        “终于找到你了!”舒克对约翰说。  
        “我现在和皮皮鲁联系。”燕妮说。        
        约翰被五角飞碟内部的现代化设施震惊了。舒克带约翰参观五角飞碟,告诉他五角飞碟的种种令人不可思议的超级功能。  
        约翰极为振奋。  
        “能带我飞一圈儿吗?”约翰问鲁西西。  
        “当然。咱们干脆现在回国吧?”鲁西西征求约翰的意见。  
        “皮皮鲁要和你通话。”燕妮对约翰说。  
        约翰激动地拿起话筒。  
        “皮皮鲁,我是约翰!咱们30多年没见了!谢谢你还想着我!”约翰热泪盈眶。  
        “快回来,我们都在等你!歌唱家也在我家。我们给你准备好吃的,一会儿咱们聚餐。”  
        “我的朋友还在监狱里,我想再为他办点儿事……”约翰说。  
        “那你们就尽快吧!”皮皮鲁说。  
        约翰挂上电话。  
        “你现在不能走?”鲁西西问约翰。  
        “我想求你们一件事。”约翰说。  
        “怎么能说‘求’?你的事就是我们的事。说吧。”鲁西西说。  
        “30年前,是警察说话不算数坑了罗勃特和我的。我想请你们用五角飞碟治治纽约警察局,帮我和罗勃特出口气!”约翰说。        
        “…这……”鲁西西不敢贸然答应,“恐怕皮皮鲁不会同意……他特怕五角飞碟惹事……”  
        “我给皮皮鲁打电话问问。”燕妮说。  
        “不用了。”约翰阻止燕妮,“别给皮皮鲁找麻烦了。”  
        “真对不起。”鲁西西想起刚才自己夸海口帮约翰办事的话,有点儿不好意思。  
        “没关系。和纽约警察局作对不是小事,我能理解。”约翰善解人意。  
        “我们带你上天转转。”舒克说。  
        约翰看舒克驾驶五角飞碟起飞。  
        “咱们现在洛杉矶上空。”舒克告诉约翰。  
        “这么快?!”约翰吃惊。他和马里欧从纽约乘飞机去过洛杉矶找好莱坞导演,他知道纽约和洛杉矶之间的距离。  
        “咱们现在回中国用不了1秒钟。”燕妮说。  
        约翰觉得美国最大的失误就是皮皮鲁不是美国人。  
        舒克将五角飞碟的种种功能——介绍给约翰。  
        遥感装置、武器系统、光速飞行……  
        约翰心潮澎湃热血沸腾浮想联翩。  
        “你的脸怎么这么红?”鲁西西问约翰。  
        “激动。”约翰说。  
        “咱们现在回监狱?”舒克问约翰。        
        “我想回罗勃特家拿点儿东西。”约翰说。  
        “告诉我地址。”舒克说。  
        舒克将罗勃特的住址输入电脑。  
        五角飞碟自动在罗勃特家着陆。罗勃特的家布满尘土和蜘蛛网。约翰离开五角飞碟到一个抽屉里找东西。  
        “不知被罗勃特从保险柜里救出来的女孩儿现在活得怎么样了?”鲁西西注视着罗勃特的家说。  
        “她现在也有30多岁了。”燕妮说。  
        “说不定还不如当年死在保险柜里好,也许正在受到丈夫的虐待。”舒克往深刻了说。  
        “净瞎说。”鲁西西笑着反驳舒克。  
        “这是好的,说不定还在红灯区做献身工作呢!”舒克说。  
        “越说越不像话了。”鲁西西用双手冲舒克作暂停手势。  
        约翰回来了。  
        五角飞碟起飞回到监狱,在罗勃特的床下着陆。  
        “你和我们一起在五角飞碟上吃饭吧?”燕妮问约翰。  
        “行。”约翰答应了。  
        平时约翰和罗勃特共享监狱的囚饭。  
        燕妮和鲁西西到厨房做饭。  
        舒克和约翰聊五角飞碟。约翰是不耻下问。舒克是有问必答。  
        约翰对五角飞碟了解透彻后,到厨房要求给朋友们做一个美国菜。他的要求得到了鲁西西和燕妮的鼓励。  
        聚餐开始。尽管约翰的菜比较不好吃,可朋友们还是有意大吃特吃。鲁西西在心里管这叫吃忆苦饭。  
        约翰一边看朋友们吃他做的菜一边暗自请求上帝宽恕他。          第306集  
        罗勃特批评约翰;  
        约翰鬼迷心窍被复仇的火焰熏烤得不能自拔;  
        约翰考取空中驾驶执照;  
        倒霉的马丁辛名列黑名单之首    
        饭吃到一半时,燕妮说:  
        “我特困,想睡觉。”  
        鲁西西也开始打哈欠。  
        舒克发现不对时,已经晚了,他只觉得头重脚轻。  
        “你有问题……”舒克对约翰说,话没说完,他就睡着了。  
        鲁西西和燕妮在舒克之前入睡。  
        “对不起,实在对不起。你们大老远来美国找我,我这是恩将仇报。上帝饶恕我。”约翰打了自己两个清脆的耳光。  
        原来,约翰打从第一眼看见五角飞碟起就决定用它报复纽约警察局。约翰这口气憋了30年。在遭到朋友的拒绝后,他心生一计,回罗勃特家取了安眠药,放进他的美国菜中……  
        约翰将沉睡的鲁西西、燕妮和舒克逐一背出五角飞碟,交给罗勃特。  
        “你的朋友都怎么了?”罗勃特吃惊。  
        “我把他们交给你了,你要像对我一样确保他们的安全。”约翰对罗勃特委以重任。  
        “这到底是怎么回事?”罗勃特追问。  
        “我现在驾驶五角飞碟去给你…错了,是给咱们报仇。”约翰豪情满怀。  
        “你麻翻了从中国来找你的朋友然后劫持了人家的飞行器?”罗勃特30年来头一次对约翰的品质发生了怀疑。  
        “话一从你嘴里出来怎么让人觉得那么别扭?我成了劫机犯啦?而且还是劫朋友的飞行器!我连正宗的劫机犯都不如,正宗的劫机犯的职业道德是只劫敌人的飞行器。事实不是这样!我是借,借朋友的飞碟去完成我的夙愿,去教警察怎么当警察,怎么做人!”约翰义正词严驳斥罗勃特,“我不跟你罗嗦了,他们马上就该醒了,我下的药特少,从这点就不难看出我们是莫逆之交。如果是敌人能下这么点儿药?你照顾好他们!”  
        约翰不等罗勃特再说话就钻进五角飞碟。  
        罗勃特张口结舌。  
        罗勃特和约翰整天生活在一起,他早就发现约翰比他还恨警察局,耿耿于怀了30个春夏秋冬。  
        罗勃特清楚这场祸小不了。  
        约翰已经有意识地观察了舒克驾驶五角飞碟的全过程,不明白的地方也一一向舒克咨询过了。现在,约翰正襟危坐在五角飞碟的驾驶台前揣摩驾驶要领。  
        很快,五角飞碟起飞了。  
        罗勃特叹了口气,开始悉心照料熟睡中的约翰的朋友。  
        “什么用药量小,少说每人吃了5片!”罗勃特骂道。  
        现在,约翰驾驶五角飞碟悬停在纽约上空。  
        “我应该路考一下,自检。”约翰自言自语。  
        约翰决定首先环绕地球飞10圈。  
        在几秒钟内,约翰就完成了环绕地球飞行10圈的壮举。  
        “再桩考一次。”约翰对自己严格要求。  
        约翰驾驶五角飞碟沿着华盛顿——东京——汉城——北京——太原——莫斯科——巴黎——哥本哈根——渥太华——纽约的路线曲线飞行,检验自己的驾驶技术。  
        飞行结束时,约翰给自己打了91分。  
        “越是没人的时候,越要谦虚。”约翰自言自语。  
        约翰开始试用五角飞碟的遥感功能。他看见了在监狱里的罗勃特。看见了在北京的皮皮鲁。看见了正在日本访问的美国总统。看见了一位正在大便的诺贝尔医学奖获得者。  
        “这是真正的奇迹!”约翰心花怒放,他有自己变成了超人的感觉。  
        约翰想起皮皮鲁小时候上学的时候不受老师重视的情景。  
        “考试分数根本不能代表学生的未来。”约翰想。  
        现在,约翰要对自己进行最后也是最重要的考核了:五角飞碟的武器系统。  
        约翰选择了正在纽约一条高速公路上巡逻的一辆警车作为试验目标。  
        约翰瞄准了警车的一个后轮子按下射击按钮。  
        警车的那个轮子爆了。  
        约翰连续打击另外3个轮子。  
        警车趴在路上。驾车的警察跺脚骂街。  
        “太棒了!”约翰在五角飞碟里手舞足蹈得意忘形。  
        约翰开始制定报复计划,他首先要收拾的是30年前那个对罗勃特食言的警察局决策者,以及整个警察局。  
        还有起诉罗勃特的检察官。判罗勃特有罪的陪审团全体成员。监狱里那个总是同罗勃特过不去的狱卒。犯人中的几个老找罗勃特茬的狱霸。  
        约翰打开五角飞碟的遥感仪,他寻找30年前纽约警察局的档案。  
        约翰很快找到了那个决定在罗勃特救出关在保险柜里的女孩儿后拘捕他的纽约警察局副局长。他叫马丁辛,现已退体。  
        约翰看着屏幕上的马丁辛的照片,得意地笑了。  
        约翰要让马丁辛好好尝尝食言的滋味。          第307集  
        罗勃特当替罪羔羊;  
        燕妮想到了马里欧的安危;  
        鲁西西和燕妮在大学校园里放风;  
        史蒂芬斯命令8176打开口袋    
        舒克第一个醒过来。  
        “我这是在哪儿?”舒克看见了睡在他身边的鲁西西和燕妮。  
        “对不起……”罗勃特一生中头一次感到不好开口。  
        “我的五角飞碟呢?”舒克大惊失色。  
        “约翰……开走了……”罗勃特难以启齿。  
        “他在我们的饭菜里下了(被禁止)药?”舒克恍然大悟。  
        “哎,真没想到约翰这么记仇。”罗勃特摇头。  
        “您是幕后策划吧?”舒克盯着罗勃特的眼睛。        
        “不是。”罗勃特的目光迎接舒克的逼视。  
        舒克信了。他毕竟当过大牌作家,拥有敏锐的观察力。  
        “约翰驾驶五角飞碟干什么去了?”舒克问罗勃特。  
        “教训警察局。”罗勃特声音很低。  
        “他疯了?”舒克急了,“他怎么可以劫持朋友的飞碟去干违法的事?”  
        罗勃特不吭气。  
        “鲁西西,你醒醒!”舒克推鲁西西。  
        鲁西西极不情愿地睁开眼睛。  
        “出事了!'_舒克说,“约翰把五角飞碟开跑了!”  
        “你说什么?”鲁西西还没完全摆脱梦境。  
        “约翰在他给咱们做的美国菜里下了安眠药,等咱们都睡着了以后,他把五角飞碟开走了!”舒克说。  
        “约翰会开五角飞碟?”鲁西西巡视四周,发现自己确实不在五角飞碟里。  
        “我说他怎么对五角飞碟那么感兴趣呢!原来是让我给他办短期培训班。”舒克捶胸顿足大呼上当。  
        燕妮也醒了。  
        “约翰开五角飞碟干什么?”燕妮问。  
        “替罗勃特出气,报复纽约警察局。”舒克说。        
        “也是替他自己出气。”罗勃特说。  
        “咱们怎么办?”鲁西西问舒克。  
        “和五角飞碟联络用的微型通讯器在你身上吗?”舒克问鲁西西。  
        “在厨房做饭时我摘下来放在微波炉上了。”鲁西西后悔不迭。  
        “咱们现在同外界彻底失去联系了。不管是五角飞碟还是皮皮鲁。”舒克说。  
        “咱们的处境好像不妙吧?”燕妮意识到大家的安全已经没有了保障。  
        “约翰委托我保护你们。请你们放心,我会对你们的安全负责的。”罗勃特一脸的将功赎罪。  
        “约翰报复完警察局就会回来?”燕妮产生了怀疑,“他会不会再去报复赌场?还有马里欧。”  
        鲁西西的脸色变了。  
        罗勃特也明显开始局促不安。  
        监狱里突然铃声大作。  
        “到犯人放风的时间了,我带你们出去。你们留在这儿我不放心。”罗勃特说。  
        “我在这儿等约翰,万一他在放风的时候回来呢,你们俩跟罗勃特出去,顺便注意观察天上有没有五角飞碟。”舒克对鲁西西和燕妮说。  
        鲁西西和燕妮钻进罗勃特的囚服和罗勃特放风去了。        
        监狱的院子里全是犯人,他们有的打篮球有的做操有的靠着墙晒太阳,每人脸上都是一部哲学著作。荷枪实弹的狱警在高高的晾望台上嚼口香糖。  
        “约翰说得对,监狱是真正的大学。”鲁西西没想到自己这辈子能在监狱参加放风活动。  
        “所以很多大学生还没毕业就直接跳级到监狱来上学了。”燕妮说,“暗杀以色列总理拉宾的凶手就是以色列名牌大学法律系的在校大学生。”  
        “大学法律系学生暗杀像拉宾这样的代表进步的政府首脑,这真是对大学教育的绝妙讽刺。还是让监狱这所大学继续培养他吧。获得了诺贝尔和平奖的人被子弹打死,这个世界出问题了。”鲁西西说。  
        “咱们还是找五角飞碟吧。”燕妮透过罗勃特囚衣口袋的一处裂缝往天上看。  
        一个狱警朝罗勃特走过来。  
        “你们注意隐蔽,史蒂芬斯过来了。”罗勃特告诫鲁西西和燕妮。  
        “史蒂芬斯是谁?”鲁西西问。  
        “一个总是和我过不去的狱警,很坏。”罗勃特说。  
        鲁西西和燕妮藏好。  
        “8176,你过来!”史蒂芬斯走到距罗勃特3米的地方站住叫罗勃特。  
        8176是罗勃特的囚号。犯人在监狱里是没有名字的。  
        罗勃特走到史蒂芬斯身边。  
        “你的上衣口袋里装的是什么?”史蒂芬斯的鹰眼发现8176的囚衣口袋里鼓鼓囊囊。  
        鲁西西和燕妮的身躯。  
        “没什么……”罗勃特身上打了个激灵。  
        “拿出来!”史蒂芬斯命令道。          第308集  
        狱警的鼻子一分为二;  
        舒克下落不朋;  
        詹姆斯是罗勃特的可靠朋友;  
        鲁西西和燕妮在监狱长的办公桌上冒充玩具    
        罗勃特站着不动。  
        “8176 1我再说一遍,把兜里的东西拿出来!”史蒂芬斯边说边从武装带上解电警棍。  
        鲁西西和燕妮吓得缩成一团。  
        罗勃特站着不动。  
        史蒂芬斯勃然大怒,他打开了电警棍的开关。电警棍的周身立即蹿腾起火龙,还发出令人毛骨悚然的与自然界的雷电声截然不同的人工雷电声。  
        “我拿。”罗勃特佯装屈服,他担心藏在他身上的鲁西西和燕妮承受不了电警棍的电击。        
        就在史蒂芬斯等待罗勃特交出兜里的东西时,罗勃特突然劈头给了史蒂芬斯一拳。史蒂芬斯仰面朝天倒下。  
        罗勃特拔腿朝牢房跑去,他想让鲁西西和燕妮与舒克会合后迅速从他身上转移。  
        监狱里警笛声大作,狱警们从四面八方堵截罗勃特。  
        罗勃特又击倒了两名狱警,他跑进自己的牢房。床上床下都没有舒克。整个牢房里没有舒克。  
        “舒克不见了!”罗勃特气喘吁吁地告诉鲁西西和燕妮。  
        祸不单行。鲁西西和燕妮已经被这一系列意想不到的突发事件吓呆了。  
        追捕罗勃特的脚步声越来越清晰。  
        “你们不能再藏在我身上了,我马上就会被他们抓住。我把你们转移到我的一个朋友身上,他叫詹姆斯,就在隔壁,人很可靠。”罗勃特边说边跑进隔壁的牢房。  
        一个犯人躺在床上,好像在生病。  
        “詹姆斯,帮个忙!”罗勃特从口袋里掏出鲁西西和燕妮递给詹姆斯,“这是我的朋友,请你替我照看,狱警在抓我!”  
        罗勃特将鲁西西和燕妮塞到詹姆斯的手中后,他急忙回到自己的牢房坐以待毙,以免狱警怀疑詹姆斯。  
        狱警们抓走了罗勃特,一路痛打。  
        听着罗勃特的惨叫,鲁西西和燕妮浑身发抖。  
        罗勃特在放风的时候将狱警史蒂芬斯的鼻梁骨打断了,而且他拒不交代自己的衣兜里藏了什么。这在监狱是大事。罗勃特受到的惩罚是在水牢里度过3个星期并加刑10年。  
        狱方宣布,不管是谁,只要他举报罗勃特藏了什么,可获得减刑5年的奖励。如果谁能交出罗勃特藏匿的东西,狱方立即释放他出狱。  
        在经过一番踌躇后,詹姆斯将鲁西西和燕妮交给了狱方。  
        詹婀斯获释出狱。  
        几乎所有狱警都来围观罗勃特的藏匿物:两个微型女性。  
        “这是什么?”  
        “玩具?”  
        “我看像真人。”  
        “她们能帮8176越狱?”  
        “8176从哪儿弄来她们的?”  
        狱警们围着坐在监狱长办公桌上的鲁西西和燕妮开学术讨论会。  
        “都执勤去!”监狱长轰狱警们。  
        狱警们离开监狱长的办公室。现在只剩下监狱长一个人。  
        鲁西西平生第一次尝到了被人出卖的滋味儿。她一辈子也忘不了詹姆斯在把她们交给狱警时的嘴脸。  
        鲁西西暗示燕妮千万不要说话,鲁西西认为她们现在最好的转危为安的方法是冒充玩具。  
        监狱长开始拿放大镜观察鲁西西和燕妮。  
        鲁西西和燕妮尽量保持身体不动,但她们不能不眨眼皮。  
        “活的。”监狱长自言自语。  
        在一个表现一贯不错的犯人身上发现了两个微型女人,而那个在监狱里已经度过了漫长的30年还有5年就要出狱的犯人居然会为这两个小人打断了狱警的鼻梁骨!他十分清楚等待自己的惩罚是什么。这两个小人对他如此重要?  
        监狱长继续反复研究小人。  
        “还有一个是中国人!”监狱长惊异。  
        鲁西西也开始观察监狱长。监狱长50多岁,脸上已经有明显的皱纹。他的眼睛呈蓝色。  
        鲁西西又将挂在墙上的奖状和证书逐一认真看过,监狱长的履历一览无余。  
        监狱长名叫彼得富氏,20岁开始在洛杉矶警察局当警察。30岁调人纽约警察局。47岁因腿部被歹徒枪击致残而离开警界到本监狱出任监狱长继续同坏人打交道。  
        不知为什么,鲁西西凭直觉对彼得富氏没有恶感。她想冒一次险。          第309集  
        舒克被绑架到社区参议员办公室;  
        参议员听说舒克的国籍后大喜;  
        万元筵席诱惑参议员到中国定居;  
        舒克不愿意从事人蛇工作    
        现在处境最危险的,是舒克。  
        罗勃特带鲁西西和燕妮离开牢房去放风后,舒克抱着一线希望在床下等约翰驾五角飞碟回来。  
        舒克听见身后有响动,他还没来得及回头,两只胳膊已经被扭到了背后。舒克挣扎无效,他的两手被绳子捆在了一起。  
        “跟我们走!”一声大喝。  
        舒克回头一看,绑架他的是四只骨瘦如柴的异国同胞一一美国老鼠。  
        “你们要干什么?”舒克问美国同胞。  
        “听你的口气,好像你是这儿的主人。你到了我们的地盘上,说话要有礼貌。”一只美国老鼠教育舒克。  
        “我们早就注意你了。你进人我们的社区连个手续也不办,这叫非法入境。我们只得对你依法办事。现在我们带你去见参议员。”另一只美国老鼠对舒克说。  
        “参议员?”舒克觉得应该是见鼠王,“是鼠王吧?”  
        “鼠王是什么年代的职务了?我们这儿每个社区有一位民选参议员,是我们的头儿。我们老鼠家族的最高领导人叫总统。”又一只老鼠开导舒克。  
        “社区?”舒克嘀咕。  
        “在你们那儿大概叫洞。”一只老鼠告诉舒克。  
        舒克在异国同胞的押送下见到了监狱老鼠社区的参议员。舒克发现美国老鼠都很瘦弱,好像营养不良。  
        “你从哪儿来?”参议员问舒克。  
        “中国。”舒克不想隐瞒国籍。  
        “中国?”参议员和幕僚大惊。  
        “怎么了?”舒克纳闷。  
        “你真的是中国老鼠?”参议员笑容可掬地问舒克。  
        舒克肯定地点头。  
        “快给他松绑!你们怎么能这样对待从中国来的客人?”参议员喝令手下。  
        异国同胞抢着给舒克松绑。  
        舒克感激地看着对中国老鼠有深厚感情的参议员级美国老鼠同胞。  
        “请坐,请坐。可乐伺候!”参议员一边请舒克落座一边招呼手下给舒克上饮料。  
        舒克一时难以适应大反差,有点儿不知所措。  
        “看在咱们是同胞的份上,我想请您帮帮我。”参议员奴颜婢膝地央求舒克。  
        “我帮你?帮什么?”舒克问。  
        “帮我去中国定居。如果您能把我们社区的老鼠都弄到中国去最好,要是一下子都弄去有困难,就先把我弄去。等我在那边站稳了,再把他们接过去。”参议员向舒克和盘托出自己的计划。  
        “你放弃美国籍和参议员身份去中国定居?”舒克着实感动了一番。  
        “美国籍怎么能同中国籍比?”参议员以为舒克在讽刺他。  
        “……”舒克语塞。  
        “您好像不太了解当前世界老鼠的择地倾向。”参议员看出舒克是一只孤陋寡闻的中国老鼠,“告诉你,地球上的择地倾向目前是这样的:人类以拥有美国绿卡为荣,老鼠以拥有中国绿卡为荣。”  
        “为什么全世界的老鼠都想拥有中国绿卡?”舒克知道不少浅薄的人类成员以拥有美国绿卡为荣,他还是头一次听说全世界的老鼠憧憬中国绿卡。  
        “全世界的老鼠都知道在中国吃饭有保障,据说中国每年丢弃浪费的粮食高达数千亿斤,这能养活多少老鼠?还听说中国人在饭馆吃饭以剩的饭菜多为荣,都吃光了被认为是丢脸的事。中国人伟大!不像美国人下饭馆,最后恨不得把盘子都舔了,小气!一点儿也不给咱们老鼠留!依我看,美国人是最没有同情心的民族。我还听说中国人在饭馆吃饭特别丰盛,一顿饭花上万元的都有,吃完了能剩半桌子。我还听说连中国最普通的大学食堂的下水道里全是被学生倒掉的白花花的米饭和囫囵个的馒头……”参议员两眼发光馋涎欲滴。  
        “我不能帮助你偷渡去中国。”舒克打断参议员的话。  
        “为什么?”参议员问。  
        “不为什么,就是不帮。”舒克懒得解释,也想不出解释的理由。  
        “那你就死定了。”参议员威胁舒克。          第310集  
        舒克对于美国老鼠吃不饱肚子幸灾乐祸;  
        社区举行全民公决;  
        舒克力图通过高兴把自己的肉变得难吃;  
        克莉斯汀领先爸爸一步大义灭亲    
        “你为什么不自己去?”舒克问参议员。  
        “你们中国人到美国来,我们美国领事馆不也需要一个美国人为那个想获得签证的中国人当担保人吗?我自己去中国?住哪儿?我在这儿是参议员,到了中国谁认识我?”参议员驳斥舒克。  
        舒克不知道应该为自己是中国老鼠自豪还是悲哀。  
        “我希望咱们能够合作。我再给你一次机会。请立即答复我。”参议员给舒克下最后通牒。        
        “我不会给你获得中国绿卡当担保人。你就当一辈子美国老鼠吧,饿不死你。我就不信美国人吃完饭都舔盘子。”舒克嘲笑参议员。  
        “来人!把他给我捆起来!你别忘了这是监狱的老鼠社区,我们什么刑罚不会?你在巾国吃饱了喝足了跑到美国来嘲笑我们这些生活在水深火热之中吃了上顿没下顿的老鼠,你难道不知道越是发达国家的老鼠越食不果腹越是发展中国家的老鼠越丰衣足食?你是饱汉子不知饿汉子饥,今天我要拿你替我们美国老鼠出口气,凭什么你们整日山珍海味我们连舔盘子的权利都不能享有?”参议员大发雷霆。  
        一帮老鼠一拥而上,将舒克重新绑了。  
        尽管美国老鼠体质虚弱,但由于数量占绝对优势,舒克不能选择反抗的方法对付他们。  
        “用刑!”参议员凶相毕露。  
        几只弱不禁风的老鼠往舒克身上接电线,他们肚子里发出的表示饥饿的肠鸣声震耳欲聋。  
        “饿死你们!”舒克大骂。  
        “通电!”参议员挥手。  
        舒克惨叫。  
        美国老鼠们的狞笑声和肠鸣声抑扬顿挫此起彼伏。  
        舒克被电刑击昏了。  
        “泼水!”参议员咆哮。        
        监狱的水特别凉。舒克醒了。  
        “告诉你一个好消息,我们准备拿你会餐。听明白了吗?就是吃你。我想你不会拒绝整整一个星期没吃东西的同胞的这个最低要求吧?我们这个社区共有老鼠2851只,说实话,每人连一口都轮不上。”参议员对舒克说他的新构思。  
        “发刻!”舒克用英语骂参议员。  
        “听说死前越生气肉越好吃。谢谢你的好意!当然,你还可以多活一会儿,谁让我是民选参议员呢,我自己说了吃你不算,要经过整个社区的全民公决。你好好在这儿生气,我去动员我的选民就吃不吃你投票表决。拜拜。”参议员走了。  
        为了使自己的肉难吃,舒克极力强迫自己高兴。他回忆和利的新婚之夜。越回忆越不想死,越回忆越难过。  
        在参议员的主持下,社区的所有有投票权的老鼠就是否吃舒克的问题进行了投票表决。  
        投票结果,百分之九十五的老鼠赞成吃掉舒克。  
        “我只能顺应民意了。”参议员通知舒克全民公决的结果。  
        舒克知道自己的死期临近了。他没想到自己会死在美国,而且是作为食物被异国同胞吃了。  
        “你有什么遗言吗?”参议员当初是靠一篇人道主义演说当选的。        
        “希望今后美国人民吃饭时改变不剩饭菜的坏习惯。”舒克喃喃自语,“给自己国家的老鼠留一条生路。”  
        “但愿你的遗言变为现实。”参议员在胸前划十字。  
        舒克觉得参议员在亵渎上帝。  
        “你们给他洗个澡,再给他找一身正式场合穿的衣服。这是正式宴会。”参议员命令幕僚们。  
        美国是一个穿衣服严格分场合的自由国家,绝对不允许在正式场合穿非正式服装和在非正式场合穿正式服装。  
        舒克被捆着洗了澡。  
        他有奇耻大辱的感觉。  
        现在,舒克被关在一间小屋子里等美国老鼠吃他。  
        舒克想起了克里斯王国,想起了玩具博览会,想起了妈妈。还想起了中国餐馆里的那些关在笼子里恭候屠宰的活动物。  
        “皮皮鲁,鲁西西,贝塔,舒利,燕妮,歌唱家,再见了……”舒克在心里和朋友们告别。  
        有人开小门的锁。  
        舒克闭上眼睛开始创作自己就义时喊的口号的内容。  
        “我来救你!”一只女鼠的声音。        
        舒克睁开眼睛。  
        一只极瘦小的年轻女鼠。  
        “我叫克莉斯汀,是参议员的女儿。我不同意吃你。我特别讨厌我爸爸,他是个虚伪的政治家。”克莉斯汀一边说一边解捆舒克的绳子。  
        舒克激动万分。  
        就在这时,参议员率领手下闯了进来。  
        “背叛我的人没有好下场。不管她是谁。”参议员对女儿说。  
        众鼠将克莉斯汀和舒克一同绑了。  
        “举行新一轮投票,表决是否将他们两个一起吃!”参议员看着女儿说。  
        民意测验显示,参议员的大义灭亲导致社区选民对他的信任度急剧上升。  
        舒克每次历险都是死里逃生。这次他在劫难逃。   第311集  
        鲁西西破釜沉舟;  
        在监狱长的桌子上给皮皮鲁打越洋电话;  
        约翰不理皮皮鲁;  
        罗勒特头一次进监狱长办公室    
        监狱长彼得富氏在自己的办公室里一动不动地观察了半个小时8176号犯人藏匿的两个微型人,他不得不承认自己的想像力尚欠火候,他准备向纽约警察局疑案处求援。  
        彼得富氏拿起电话听筒,他低头在办公桌玻璃板下边的电话号码表中找纽约警察局的号码。  
        鲁西西察觉到彼得富氏的意图,她决定孤注一掷。  
        “监狱长彼得富氏先生,您好。”鲁西西突然说话。        
        彼得富氏手中的电话听筒掉在办公桌上。  
        “……你会说话?”彼得富氏的想像力大幅度上升。  
        “我们不是坏人,我叫鲁西西,是中国人。她叫燕妮,是德国人。我们希望得到您的帮助。”鲁西西诚恳地说。  
        “你们和8176什么关系?”彼得富氏问。  
        “罗勃特是我们的朋友的朋友,现在他已经是我们的朋友了。”鲁西西说。  
        “你们的身体为什么这么小?你们是怎么进入我的监狱的?你们来干什么?8176为什么拼死保护你们?”彼得富氏一口气提了4个问题。  
        鲁西西越来越觉得彼得富氏身上有同情心,她索性将一切告诉了彼得富氏。从罐头小人到舒克贝塔,从皮皮鲁到五角飞碟,从罗勃特因救锁在保险柜里的女孩儿而人狱到约翰劫持五角飞碟……  
        彼得富氏一脸的目瞪口呆。  
        “你怎么证实你的话是真的?”监狱长难以置信。  
        “我能使用您的电话给皮皮鲁打国际长途电话吗?这不就能证实我的话了吗?”  
        彼得富氏点头。  
        “告诉我皮皮鲁的电话号码。”监狱长从办公桌上拿起电话听筒。  
        鲁西西说皮皮鲁的电话号码。  
        彼得富氏给皮皮鲁拔电话。        
        “电话通了。”彼得富氏对鲁西西说。  
        “电话听筒太大,我听你说。”鲁西西对燕妮说。鲁西西和燕妮的身躯无法单独胜任同皮皮鲁通话的任务。  
        燕妮将嘴对准送话器。鲁西西将耳朵挨近听筒。鲁西西听见对方的话后再通过燕妮转达她要输出的话。  
        “是皮皮鲁吗?我是鲁西西和燕妮。”  
        “你是燕妮!鲁西西呢?”皮皮鲁昕出燕妮的声音。  
        “我现在代表鲁西西。”  
        “出事了?”  
        “对,我们现在的处境非常危险。”  
        “快说!”  
        “约翰偷偷给我们服用了安眠药,待我们昏睡后,他开走了五角飞碟。”  
        “……”  
        “舒克也失踪了!我现在和鲁西西在一起。”  
        “你们在哪儿给我打电话?”  
        “在约翰住的监狱的监狱长办公室。”  
        “监狱长办公室?没人?”  
        “监狱长彼得富氏是个可以信任的人,是他帮助我们给你打电话的。”  
        “谢天谢地。哪儿都是好人多。约翰开五角飞碟干什么去了?”  
        “报复纽约警察局。”  
        “什么?!他疯了?”  
        “我看差不多。也许太苦大仇深了。”  
        “那也不能劫持五角飞碟呀!”  
        “现在咱们怎么办?微型通讯器不在我们身上,我们无法同他联系。”  
        “我同他联系。就怕这小子不回话。把你们的电话号码告诉我,我和约翰联系后马上告诉你们结果。”  
        监狱长彼得富氏直接将电话号码告诉皮皮鲁。  
        皮皮鲁挂了电话。  
        监狱长彼得富氏相信了鲁西西刚才对他说的所有话。  
        “30年前,警察局抓8176是错误的。我愿意帮助你们。”彼得富氏对鲁西西和燕妮说。  
        “能给罗勃特减刑吗?”鲁西西提要求。  
        “我尽力给他办假释。”彼得富氏说。  
        “能现在把罗勃特从水牢里放出来吗?”燕妮得寸进尺。  
        彼得富氏点头。他打电话命令狱警把8176从水牢里转移到普通牢房。  
        鲁西西发现监狱长彼得富氏话很少,但办事果断。        
        在这个世界上,真正的污染是没用的话。这种污染目前已经到了严重危害人类身心健康的程度。  
        皮皮鲁来电话了。约翰不理他。  
        皮皮鲁估计约翰驾驶五角飞碟惹的祸小不了。他要求鲁西西和燕妮就呆在监狱长的办公室里。他估计约翰尽兴后会通过遥感仪找到鲁西西和燕妮的。  
        “看来你们只有在我这里等约翰了。”彼得富氏说。  
        在鲁西西的要求下,彼得富氏让罗勃特来他的办公室同鲁西西和燕妮见了面。  
        狱警们对于监狱长在自己的办公室接见8176和监狱长命令他们满监狱找一只老鼠并且不许伤害那老鼠大惑不解。  
        舒克无影无踪。  
        罗勃特为舒克捏了一把汗。燕妮和鲁西西更着急。          第312集  
        刑警队长罢工;  
        自己给自己整理遗容的人;  
        教你观察人的诀窍;  
        出版商遇到使用密写药水写作的人    
        在约翰的报复黑名单上名列榜首的是马丁辛,纽约警察局前副局长。现在退休在家安度晚年。  
        约翰坐在五角飞碟驾驶台前一边嚼口香糖一边通过遥感屏幕详细观看发生在30年前纽约警察局抓获罗勃特的经过:  
        值班的警察接到罗勃特的电话后立即向上司禀报。  
        局长当即拍板如果罗勃特真能救出那个关在保险柜里的女孩儿,就不抓他。  
        负责分管保险柜失窃案的副局长马丁辛离开局长办公室后立即命令刑警队长去东62街抓罗勃特。        
        “咱们不是答应不抓他吗?”刑警队长问。  
        “警察局和盗窃犯讲信用?送上门能不抓吗?快去,多带点儿人。”马丁辛催促。  
        “我不去。”刑警队长说。  
        “为什么?”马丁辛吃惊,他知道刑警队长为了抓罗勃特几年没休息过一天。  
        “不道德。”刑警队艮说。  
        “他从别人的保险柜里拿钱道德?你怎么同罪犯讲起道德来了?”马丁辛恼怒地看着刑警队长,“你去不去?”  
        “不去。这样抓他,我一辈子不得安宁。”刑警队长说。  
        “滚!不许告诉局长。”副局长脸色铁青。  
        马丁辛另找警察去抓了罗勃特。  
        当警察局长得知副局长下令在罗勃特救女孩儿的现场逮捕了他时,勃然大怒。  
        “混蛋!你怎么干这种事?”局长骂副局长。  
        “不抓他的人才是混蛋!”副局长不示弱。  
        这场争执一直打到警察总署。总署裁决马丁辛正确。  
        约翰在五角飞碟里咬牙切齿。他使用遥感仪遥感马丁辛的一生,约翰想看看为什么有的人品质好有的人品质不好,品质这东西到底是先天形成的还是后天的形成的。        
        遥感结果表明,马丁辛的品质从他出生就有问题。为了达到自己的目的,他可以不择手段。马丁辛的弟弟是残疾人,弟弟的妻子也是残疾人。更为不幸的是,由于遗传原因,弟弟的女儿也是残疾人。马丁辛和太太不愿意让弟弟一家与他们同住父亲留给他们的房子,使用极不人道的残忍手段将弟弟一家扫地出门。马丁辛的太太曾经在寒冷的冬天故意打开侄女卧室的窗户,冻得卧床的侄女瑟瑟发抖。最终,弟弟一家被迫迁走。  
        “一点儿同情心都没有。”约翰真想利用五角飞碟击毙地球上所有没有同情心的人。  
        “品质好坏是天生的。经常看到有的人原来品质不错,后来环境变了,他的品质也变了。其实这人的品质原来就不好,他是成功地掩盖丁自己的品质。人身上,品质最重要。其次是想像力。再其次是才能。”约翰自言自语。  
        遥感屏幕显示马丁辛现在住在纽约郊区的一栋别墅里。约翰驾驶五角飞碟光临马丁辛家上空。  
        马丁辛正在家撰写回忆录,有出版商出大价钱买他的回忆录的版权。  
        约翰用五角飞碟的电脑查询后得知,在人类写回忆录的人当中,品质不好的占绝大多数。回忆录的本质是自己给自己整理遗容。美容。  
        说自己潇洒的人在生活中准不潇洒。说自己淡泊名利的人准看重名利。一般来说,人是身上缺什么吆喝什么。观察人的诀窍是反着听他的话。  
        马丁辛写完了回忆录的最后一行,他如释重负。这部回忆录60万字,耗费了他3年的心血。  
        出版商如期来马丁辛家取书稿。  
        马丁辛将包好的书稿递给出版商。  
        “书名起好了吗?”出版商问。出版商最关心书名。  
        “就叫《警察局长的自白》怎么样?”马丁辛问书商。  
        “不叫座。叫《纽约警察局内幕》怎么样?”书商有经验。  
        马丁辛点头。  
        出版商打开书稿翻阅,他抬头用异样的眼光看马丁辛。  
        “怎么了?”马丁辛问。  
        “这是您写完的书稿?”书商问。  
        “是啊。”马丁辛点头。  
        “您是使用密写药水写的?”书商将书稿递到作者手中。  
        一摞白纸。一个字没有。  
        约翰的丰功伟绩。约翰在五角飞碟里手之舞之足之蹈之。  
        “这叫一字不著,尽得风流。”约翰哈哈大笑。          第313集  
        震惊世界的微型局部大地震;  
        甲鱼电视台摇身一变;  
        马丁辛曾经是显赫一时的神枪手;  
        歌唱家夫妇为约翰的品质辩护    
        “这不可能!”马丁辛愕然。  
        “请您将真正的书稿给我。”出版商不想陪马丁辛玩游戏。  
        “这就是真正的书稿……”马丁辛茫然不知所措。  
        “请您退还定金。”出版商的表情晴转阴。  
        “为什么?”马丁辛不愿意退定金。到了手的钱再失去的滋味不好受。  
        “您收了我的定金,到交稿日期给我白纸,还问我为什么收回定金?如果您坚持不退款,我打电话叫律师了。”出版商冷冰冰地说,他在心里发誓这辈        子再不找警察写书了。  
        马丁辛只得如数退还定金。  
        约翰从头舒服到脚。  
        “皮皮鲁伟大,居然发明了这等器物!”约翰对五角飞碟爱不释手。  
        约翰决定对马丁辛实施不危及生命的毁灭性打击。他在电脑上设计程序。  
        马丁辛送走出版商后,心烦意乱地在书房里困兽般徘徊。    
        突然,别墅天摇地动。  
        “地震!”马丁辛大叫。  
        家里就马丁辛一人,他想往外跑,却无法成功。不管他往哪个方向跑,哪个方向就加大震度摔倒他。马丁辛离不开别墅,别墅也倒不了,就是使劲儿摇晃。  
        最令马丁辛震惊的是,他透过窗户看见四周邻居的房屋固若金汤,甚至比平时更稳妥。  
        约翰接通了纽约一家名不见经传苦于提不高收视率的小电视台的电话。  
        “是甲鱼电视台吗?给你们提供一个能让你们名扬四海的机会。纽约郊区××社区的一栋别墅现在发生地球上罕见的局部大地震,什么叫局部地震?就是发生在方圆50平方米内的里氏7级以上地震。愚人节?我知道你们为什么收视率低了。我现在就通知别的电视台。什么?你们去?那就快去!”约翰推波助澜。  
        甲鱼电视台的摄像记者赶到马丁辛的别墅时知道自己的电视台扬眉吐气的日子到了。他们一边开始拍摄一边通知台里火速增兵,最好租直升机。  
        整个纽约都从电视屏幕上看见了发生在马丁辛别墅的大地震。当记者在现场告诉观众这场局部地震已经持续了35分钟并且没有停止的迹象时,纽约市民大为震惊,他们纷纷打电话给地震部门询问地震会不会扩大范围。  
        无数人离开建筑物躲到空旷的地方。马丁辛的邻居都揣着信用卡和存折预防性地跑到屋外。  
        地震专家面对这场地震张口结舌。  
        纽约救援人员试图从久震不衰的别墅里救出马丁辛,无数次努力以无数次失败告终。  
        地震已经持续了一个小时。全世界都看见了实况转播。甲鱼电视台终于一举脱贫,跻身美国三大电视台的行列。  
        “约翰干的。”皮皮鲁坐在电视剧前对贝塔和歌唱家说。  
        “构思还行。”贝塔评论,“不过比我上次还差点儿。”  
        皮皮鲁瞪了贝塔一眼。  
        鲁西西和燕妮在监狱长彼得富氏的办公室里收看了马丁辛别墅地震的实况转播。  
        “约翰不会震死他吧?”燕妮说。  
        “约翰绝不会让他死。约翰就是折腾他。”鲁西西极为肯定地说,她了解约翰的品质。  
        “震完了他就该回来了吧?”燕妮说。  
        “但愿如此。”鲁西西说。  
        马丁辛在自己的别墅里足足享受了两个小时的局部地震。他已经明白了这不是自然界的地震,这地震和刚才的空白书稿是同一件事,有人在和他过不去。马丁辛毕竟是警察出身。  
        “有什么话你他妈挑明了说!有种你出来!”五脏六腑都换了位的马丁辛冲着空气破口大骂。  
        地震戛然而止。  
        马丁辛迟疑片刻后迅速从桌子抽屉里拿出一把手枪,他将子弹推上膛,双手平举手枪指着正前方。  
        “你出来!”马丁辛声嘶力竭。  
        一个人影出现在门口。马丁辛射击。那人应声倒地。马丁辛的枪法当年在美国警界是赫赫有名的。  
        又出现了一个人影。马丁辛继续射击。两条人命。  
        被马丁辛打死的是甲鱼电视台的两位记者。  
        闻讯赶来的警察铐走了马丁辛,罪名是故意杀人。  
        “约翰过了!”皮皮鲁通过电视转播见到出了人命,急了。  
        “我觉得这不是约翰设计的情节,是那老头神经错乱了。”贝塔分析。  
        “没错。约翰不会这么干。”歌唱家为约翰辩护。  
        皮皮鲁玩命呼叫约翰。约翰仍然不回话。  
        “但愿约翰现在鸣金收兵。”歌唱家祈祷。  
        “他在美国住了30多年,仇人少不了。我估计好戏还在后头。”贝塔冷笑。          第314集  
        世界上惟一装备玩具手枪的警察局;  
        女元首身上的定时炸弹;  
        联合国大会变成了奥运会;  
        皮皮鲁忧心忡忡    
        约翰本想使用电话同马丁辛交谈,没想到马丁辛竟然连杀两人,约翰无奈地耸耸肩膀。  
        “最少也是终身监禁。在监狱过晚年吧。狱警又多了一个欺凌的靶子。”约翰叹了口气。  
        其实约翰心里明白,马丁辛进监狱后,对他真正构成威胁的,不是狱警,而是犯人。犯人对于同监的警察身份的犯人绝对不一视同仁绝对绞尽脑汁报复。  
        “现在该收拾警察局了。”约翰一脸的坏笑。  
        约翰只按了五角飞碟里的7个键,就将纽约所有警察的枪变成了玩具滋水枪。        
        一名警察得到报案说有一名男子正在洗劫一家小店。那警察赶到现场向歹徒鸣枪示警。警察的朝天的枪口射出的不是震耳欲聋的子弹而是无声的水,向上的水被地心吸引力强迫走回头路后落在警察头上。  
        围观的路人吃惊。  
        “是拍电影!”智商高的人先顿悟。  
        “他是史泰龙!”有人指着那警察喊。  
        那警察形象果然酷似史泰龙。  
        追星族立即找出一切可以用来签名的物质涌上去请史泰龙签名。  
        歹徒趁机溜了。  
        警察局长连续接到下属关于枪支异变的报告。  
        “他们的脑子出毛病了吧?”局长根本不信。  
        “我刚才试了试我的枪,真的变成了滋水枪。”一位探长向局长证实。  
        “你表演给我看看。”局长说。  
        探长掏枪朝局长办公室的天花板射击。  
        吊灯在巨大的枪声伴奏下被击得粉碎。  
        局长勃然大怒。探长诚惶诚恐。  
        “刚才确实是滋水枪!”探长一边往门口退一边为自己申辩。  
        “谁再说枪变成了滋水枪我就关谁禁闭!”局长大喊。        
        约翰在五角飞碟里笑得满地打滚。  
        联合国正在纽约召开一次意义非常的会议,许多国家元首参加会议。警察局正全力以赴保证会议和各国元首的安全。  
        约翰给警察局打电话。  
        “告诉你们一个阴谋,有坏人在××国女元首的文胸里安放了微型定时炸弹。”约翰的恶作剧开始上档次了。  
        “你是谁?”警察局的值班员问。  
        “你抓紧时间向上司汇报吧!一会儿联合国大会就开始了,炸了这么多头儿你负得了责吗?”约翰挂了电话。  
        警察局长尽管对这个举报嗤之以鼻,但他还是向他的上司汇报了。  
        中央情报局的特工从会场里叫出××国女元首。  
        “什么事?”秘书问。  
        特工小声告诉秘书并要求秘书配合检查。  
        女元首知悉后大怒,坚决拒绝检查。  
        特工坚持要检查。  
        女元首愤怒中自己的手无意触碰到自己的胸部,她的脸色急剧变化。  
        “怎么了?”特工看到了元首脸上的变化。  
        “是有……炸弹……”元首慌了,她摸到了炸弹。  
        特工不顾国际交往的起码礼仪在公众场合当众撕开了异性元首的上衣。一颗微型定时炸弹固定在元首的文胸上。炸弹上的红色指示灯不停地闪烁。  
        警犬般敏感的记者毫无惧色地包围了元首,摄像机照相机录音机疯狂地工作。  
        特工不顾一切地扯下元首的文胸,朝会场外边跑去。人们闪开一条路。  
        炸弹没有爆炸。中央情报局和纽约警察局没敢告诉记者炸弹为什么没爆炸。假炸弹。  
        就在联合国大会开到一半时,突然从会场天花板上降下一个用绳子拴着的冒烟的大炸弹,悬在半空中。  
        惊叫声骤起。  
        与会者争先恐后夺路而逃。各国元首在同一起跑线上开始了一次公平竞争。发达国家没有任何优势。电视台向全世界现场直播。  
        某发展中国家举国欢呼他们的总统将七国首脑远远甩在后边。  
        当手持防弹盾牌的排弹特警在满是鞋子的会场发现悬在空中的炸弹是道具时,警察总署署长命令手下:  
        “把纽约警察局长给我抓来!”  
        参加联合国大会的各国元首纷纷不辞而别打道回府。  
        美国总统限纽约警察局长3天找出拿美国脸面开涮的恶作剧歹徒。警察局长当即辞职,他甘拜下风清楚自己不是这个歹徒的对手。  
        约翰没有因为局长的辞职而停止折腾警察局,他继续调动自己的想像力极尽戏弄警察局。  
        继警察的枪出问题后,警车也步枪的后尘同时罢工。纽约大街上出现了罕见的民用车拖警车的景象。  
        纽约的歹徒们弹冠相庆纷纷走上街头公开作案。  
        银行被抢商店被盗富人被绑妇女遇害人人自危。总统不得不出动国民自卫队甚至军队恢复纽约的秩序。  
        无数国家的元首致电联合国要求联合国总部迁址。  
        皮皮鲁坐在电视机前发呆。  
        观察家发现皮皮鲁的手在哆嗦。          第315集  
        检察官起诉自己的没有犯罪的孙子;  
        陪审团成员主动交代受贿记录;  
        老罗勃特享受专线电视;  
        男人和流氓的区别    
        纽约警察局瘫痪了,不管他们想干什么都以失败告终。全世界的罪犯蜂拥到纽约共享欢乐。纽约成了罪犯联合国总部。  
        纽约警察局局长几乎以一天两任的频率更换。  
        约翰心满意足地结束了报复纽约警察局的工作。他开始收拾30年前判罗勃特有罪的法官、陪审团和起诉罗勃特的检察官。  
        鲁西西、燕妮和罗勃特在监狱里从电视上目睹丁纽约的天翻地覆。大家一边看一边叹气。当看到众多罪犯肆无忌惮地在纽约为所欲为时,连罗勃特都说约翰过了。        
        “如果约翰能把五角飞碟还给咱们,皮皮鲁肯定会销毁五角飞碟。”鲁西西对燕妮说。  
        “我信。”燕妮知道皮皮鲁现在肯定极为沮丧。是皮皮鲁的发明给纽约带来了这场损失起码达到1000亿美元的灾难。  
        “约翰什么时候会回来?”鲁西西盼望。  
        “史蒂芬斯倒霉的时候,约翰就回来了。”罗勃特说。史蒂芬斯是约翰最恨的狱警。罗勃特估计约翰最后教训史蒂芬斯。  
        “您应该提前把史蒂芬斯藏起来。”燕妮对监狱长说。  
        “史蒂芬斯应该被教训一下。”监狱长彼得富氏说。  
        “往哪儿藏五角飞碟都能找出来。”鲁西西提醒燕妮。  
        正在过着夕阳红日子的检察官、法官和陪审团成员逐个祸从天降。有的突然失聪。有的永远失去判断力,连老伴儿都不认识了。有的大小便搞自由化终身失禁。一个检察官硬是坚持起诉自己的品学兼优的孙子。一位陪审团成员突然一天24小时胡言乱言,骂自己是白痴,还坦白自己担任陪审团成员期间受过多少多少贿。家人哭成一团。  
        约翰的报复黑名单上有马里欧的名字。但约翰先公后私,他先把罗勃特的仇人都收拾完了再出自己的气。  
        “罗勃特的对头还有谁呢?”约翰冥思苦想。  
        罗勃特判断得对。约翰把史蒂芬斯放在最后了,顺路。  
        约翰想起了罗勃特那没见过的生身之父。  
        约翰认为,罗勃特之所以今天在监狱里而不是在国会里,他的爸爸负有不可推卸的第一顺序责任。  
        约翰从一张报纸上看到一位专家的论文,那专家说,他经过32年研究证实,见不到父亲的孩子智商低、犯罪率高。那专家还说,男人职责最重要的工作就是多和自己的孩子在一起。经常能同父亲平等相处的孩子在社交方面表现得更有自信,对新环境适应能力强,更能应付变化。  
        约翰恨那些只管生孩子不管抚养孩子的父亲,他觉得这样的男人不是男人是畜牲,是罪犯生产机。对于孩子来说,没出生之前妈妈最重要,出生之后爸爸最重要。在孩子眼里,爸爸是正义、力量、勇敢、智慧的化身。作为爸爸,只要和孩子在一起.哪怕一句话不说,都能感觉到孩子的自信猛劲儿往上蹿。(打骂孩子的混蛋爸爸除外)。  
        作为父母,对孩子最重要的教育就是在孩子18岁前一定要父母同时和孩子生活在一起。这是对孩子最好最起码的道德教育。父母为了自己的利益,抛弃孩子,不对孩子负责,你还能指望你的孩子对社会负责?  
        约翰越想越有气,他打开遥感仪找罗勃特的爸爸。  
        罗勃特的爸爸居然还活着,他已经是风烛残年,孤苦伶仃地独自一人在一座简陋公寓里苟延残喘。  
        约翰对如此老人实在下不去手,他只能在五角飞碟的电脑中设计了一个温和的程序使罗勃特的爸爸在临终前心里稍微不好受点儿。  
        老罗勃特正躺在床上看电视,电视节目突然中断了,屏幕上是一封信。  
        老罗勃特开始以为这是一部新电视剧的片头,当他看见信的抬头是自己的名字时,一愣。电视话外音开始:  
        “50多年前,是你把我带到这个世界上来的。在我还没有出生的时候,你就离开了我和我妈妈。你知道孩子没有爸爸的滋味吗?尤其是男孩子。我从知道事起就生活在没着没落中。没有爸爸的孩子等于没有根基的房子。和同学在一起的时候,每当他们谈论自己的爸爸,我就抬不起头,我有自己是半个人的感觉。  
        你有勇气把我送进妈妈的子宫,却没有勇气当我的爸爸,你算男人吗?只会往女人子宫里送孩子的不是男人,是人难。真正的男人好汉做事好汉当,他会陪伴孩子直到自己离开这个世界。孩子长大后虽然可能不和父亲住在一起,但父亲的心会陪伴孩子左右,寸步不离。  
        我现在住在监狱里,我已经在监狱住了30年。你想过自己的儿子蹲监狱吗?我觉得合格的爸爸都应该经常想像自己的孩子蹲监狱的场面,然后尽力将孩子往监狱以外的地方培养。预防孩子蹲监狱的最好办法就是常和孩子在一起,给他爱。  
        我不知道你有多少孩子。即使你有20个孩子,你也不是一个男人而是一个流氓。男人和流氓的区别在于男人有了孩子管到底,流氓有了孩子不管或不管到底。  
        我不恨你。我不能恨一个什么都不是的人。其实你很可怜,连太监都不是。”  
        老罗勃特脸上的皱纹进人地震活跃期。心脏也是。          第316集  
        上帝同马里欧在教堂谈心;  
        马里欧到地狱进修;  
        警察局长的墓奋起直追林肯纪念堂;  
        衣兜变成聚宝盆    
        现在,约翰可以心安理得地找马里欧了。一想到马里欧,约翰就条件反射觉得肚子饿。  
        约翰将五角飞碟悬停在纽约联合国大厦上空,他走进厨房吃饭。  
        五角飞碟的通信系统不断传出皮皮鲁的呼叫。约翰无法回答,他不知道怎么对皮皮鲁解释他的行为,索性不予理睬,等完了事再负荆请罪往死里道歉检讨写检查。  
        吃饱了喝足了,约翰打开遥感仪找马里欧。  
        约翰从屏幕上看到马里欧正在教堂虔诚地作祷告。        
        约翰决定乔装上帝同马里欧进行一次谈话,如果马里欧能忏悔他对约翰的不义之举,约翰就宽恕他。  
        马里欧是虔诚的基督徒,他已经70多岁了,他一生最大的精神享受就是坐在教堂里面对上帝作祷告。  
        今天他像往常一样到上帝的家中通过心灵和上帝交谈。  
        教堂里有很多人,他们默默地坐在长条木椅上感受上帝。  
        “你是马里欧?”马里欧清晰地听见有人问他。  
        马里欧抬头左顾右盼。  
        “不要找,我是上帝。我已经观察你几十年了,知道你是一个虔诚的教徒。今天我要单独接受你的忏悔。你回答我的问题不必出声,在心里说就行。”  
        马里欧脸上闪过一丝恐惧。当信仰了一辈子上帝的马里欧真的听见了主的声音时,恐惧大于激动。  
        “你这辈子干的最令自己不能原谅自己的事是什么?”上帝问。  
        “没有。我一生问心无愧。”马里欧回答。  
        “上帝无所不在无所不知。你再好好想想。”上帝比较有耐心。  
        “…婚姻破裂……”马里欧说。  
        约翰想起了马里欧的妻子南希。        
        “婚姻为什么破裂7”上帝追问。  
        “…这是我的隐私……”马里欧错将上帝当法官了。  
        “教徒在上帝面前没有隐私。”上帝教导马里欧。  
        “她对我不忠。”马里欧对上帝撒谎。  
        “假话。”上帝说。  
        “…我们夫妻生活不和谐……”马里欧头上开始出汗。  
        “继续欺骗上帝。对上帝说假话不能超过3遍。”上帝定指标。  
        “……”马里欧沉默。  
        “信仰上帝不是为了在上帝面前装扮自己而是为了在上帝面前暴露自己。”上帝教诲马里欧。  
        “……”马里欧犹豫。  
        “不管信什么教,最重要的是行动。只将信教保持在形式上而没有行动的教徒不是信教是亵教:”上帝开导马里欧。  
        “是她离我而去的……”马里欧终于开始向真话靠拢了。  
        “为什么?”上帝为马里欧庆幸。  
        “…我作为男人…有生理上的缺憾……”马里欧坚持欺骗上帝的立场。  
        约翰在五角飞碟里摇头。他只能从严了。  
        “你的妻子是因为你对一个叫约翰的微型人起了不轨之心而离开你的。”上帝显示自己的洞察一切。  
        “……”  
        “你怎么能如此对待一个那么小的人呢?你有同情心吗?钱真的比良心重要?”上帝叹气。  
        “……”  
        “为了使你成为一个真正的基督徒,我现在提前送你去地狱进行短期培训,以完善你的人生。”上帝宣布。  
        “我不去!”马里欧在教堂里突然大喊。  
        教徒们惊异。目光齐刷刷射向马里欧。  
        大家清清楚楚看见马里欧的身体缓缓下陷,直至灭顶。  
        教堂里消失了一个人,大家惊叫着逃离。  
        约翰命令五角飞碟将马里欧送进地狱进行为期一周的短期道德培训。  
        马里欧在地狱里饱受煎熬,(禁止)的和心灵的同步进行。  
        “不知道如果上帝真能对教徒实行这种地狱式短期培训,还有多少人会信教。”约翰在五角飞碟里翘着二郎腿想。  
        就在约翰准备收兵回监狱时,他又产生了一个新想法。  
        “我不能光是惩罚坏人,还应该奖励好人。”约翰自言自语。        
        在被嘉奖的光荣榜上名列前茅的是南希,马里欧的妻子。  
        约翰决定采用物质奖励的方法。他从一个受贿的官员的账号上将贿金往南希的账号上调动了100万美元。  
        “好人之所以是好人,不会要不属于自己的钱财。”资金调动完毕后,约翰想到了这个问题。  
        约翰懒得再将那100万元物归原主了,他指挥五角飞碟作弊让南希中了(被禁止)彩头奖,3000万美元。  
        约翰又找30年前的纽约警察局长。可惜那人已经作古。约翰硬是往前警察局长的孙子家塞了200万美元,钱是从一个黑手党成员的账号上拆借的。约翰还将前警察局长的墓大肆整修一番,其规格仅次于林肯纪念堂。  
        30年前纽约警察局那位拒绝逮捕罗勃特的刑警队长也难逃好运。当他坐在街心公园晒太阳时,突然觉得怀里多了一个包。他解开衣服一看,是一个纸袋。他再打开纸袋一看,里边是数万美元现钞。  
        他当即将钱交给了巡警。  
        约翰无奈,只得让五角飞碟将前刑警队长的衣兜变成聚宝盆,里边的钱取之不尽用之不竭老拿老有。钱的来源反正都是赃款。          第317集  
        狱警史蒂芬斯光临水牢;  
        舒克就义前坐转椅;  
        参议员号召选民同UFO斗争;  
        皮皮鲁不再信任约翰    
        约翰怀着愉悦的心情通过五角飞碟先进的电脑记忆系统欣赏了一遍自己的杰作,他认为可以打99分。  
        “现在轮到教育史蒂芬斯了。然后我就该向鲁西西他们投案自首了。”约翰真舍不得离开五角飞碟。  
        史蒂芬斯是一个处处与犯人作对残酷虐待犯人的狱警。罗勃特刚进监狱那天,史蒂芬斯想给罗勃特一个下马威。他故意从铁栏杆外边往罗勃特的牢房里吐了口痰,他让罗勃特用手将痰迹擦干净。罗勃特拒绝。  
        为此史蒂芬斯把罗勃特关进水牢长达一个月。据其他犯人说,犯人进监狱头一天就下水牢的几乎没有。  
        约翰从屏幕上发现史蒂芬斯的脸上缠着纱布,约翰不知道罗勃特打折了史蒂芬斯的鼻梁骨。  
        约翰先让五角飞碟把史蒂芬斯运进水牢,然后再将他的四肢肌肉弄萎缩,终生不能打别人只能挨别人打如果有人打的话。  
        当狱警向监狱长报告史蒂芬斯不知被什么人弄进水牢而且不管怎么拽都拽不出来时,监狱长告诉鲁西西和燕妮,约翰大概返航了。  
        此时,在监狱下边的老鼠社区里,舒克和克莉斯汀赤裸全身被捆在一起。他们的身边是一台微波炉。  
        几乎整个社区的老鼠都来参加会餐。  
        “对不起,我连累了你。”舒克对克莉斯汀说。  
        “对于我,活着不如死了。”克莉斯汀死而无憾。  
        参议员出现在选民面前。掌声如潮。  
        “把他们放进去!”参议员下令。  
        几只老鼠将舒克和克莉斯汀推进微波炉,关上门。  
        一只老鼠给微波炉的烹调时间定时。  
        参议员的手掌作了个刀劈的动作。一只老鼠按下了微波炉的启动开关。        
        舒克和克莉斯汀的身体开始随着身体下边的玻璃盘子旋转。他俩透过玻璃门看见外边的老鼠同胞死盯着他俩馋涎欲滴。  
        舒克知道这次逃不过死了。他的体温在不断上升,他的思维渐渐进入混沌状态,他已经闻到了烤肉的香味儿。  
        约翰用史蒂芬斯给自己这次行动画了句号后,他恋恋不舍地驾驶五角飞碟在罗勃特的牢房着陆。  
        “你凯旋啦?”罗勃特绷着脸对从五角飞碟里钻出来的约翰说。  
        “怎么这副表情?像落选总统似的。鲁西西他们呢?”约翰问。  
        “在监狱长办公室。”罗勃特说。  
        “什么?落到他们手里了?”约翰扭头往五角飞碟里跑。  
        “鲁西西和燕妮没事,你还是先找找舒克吧!”罗勃特说。  
        “舒克没和鲁西西在一起?”约翰站住了。  
        罗勃特扼要告诉约翰事情的经过。  
        约翰跑进五角飞碟,他匆忙打开遥感仪找舒克。  
        屏幕上的景象差点儿吓死约翰:  
        舒克和另外一只不知名的女鼠瘫倒在微波炉里,他们的身体在旋转,微波炉的工作指示灯亮着。无数老鼠聚集在微波炉四周。  
        约翰在慌乱中脑子还算清楚,他首先切断了微波炉的电源。  
        当五角飞碟出现在微渡炉旁时,参议员号召选民们同不明飞行物争夺微波炉里的食品。  
        约翰使用五角飞碟的武器击昏了所有在场的老鼠。  
        约翰跳出五角飞碟,打开微波炉,把已经处于半熟状态的舒克和不知名女鼠相继背进五角飞碟。  
        舒克还有一口气。女鼠比舒克稍微强一点儿。  
        约翰驾驶五角飞碟在监狱长的办公桌上着陆。  
        鲁西西和燕妮获悉情况后立即进入五角飞碟对舒克和女鼠进行急救。彼得富氏为她们提供氧气和药品。  
        舒克伤得很重。女鼠先醒了,她将经过告诉大家。  
        约翰内疚。  
        鲁西西在五角飞碟里立即同皮皮鲁取得联系。  
        “舒克必须马上回国治疗,只有解剖主任能救他!”皮皮鲁说。  
        “让约翰驾驶五角飞碟送我们回去?”鲁西西问。  
        “绝不能再让约翰开五角飞碟了!”皮皮鲁是一朝被蛇咬十年怕井绳。  
        “那我们怎么回国?”鲁西西恨自己没学驾驶五角飞碟。  
        “我去!你们在纽约等我。”皮皮鲁决定亲自来美国开回五角飞碟。  
        “你以正常人的身体来美国?能出境吗?”鲁西西没忘德国的事。  
        “我会有办法。记住,千万别再让约翰呆在五角飞碟里。”皮皮鲁叮嘱鲁西西。“知道了。”鲁西西说。          第318集  
        皮皮鲁给解剖主任戴高帽;  
        探长林不给皮皮鲁返程机票;  
        约翰戴手铐乘坐五角飞碟;  
        监狱长和犯人游览长城    
        “咱们去美国。”皮皮鲁对贝塔和歌唱家说。  
        “真的?”贝塔兴奋。  
        “舒克受伤了,让约翰开五角飞碟回来我不放心,没准他又想起什么仇人。”皮皮鲁说。  
        “就是,在美国呆了那么多年,早变了。”贝塔落井下石。  
        “舒克怎么了?”歌唱家问皮皮鲁。  
        皮皮鲁将舒克的现状告诉贝塔和歌唱家。  
        “咱们怎么去美国?”贝塔认为离了直角飞碟去美国不是一件容易事。  
        “我找探长林帮忙。”皮皮鲁翻电话号码本。        
        探长林于20分钟内赶到皮皮鲁的住所。  
        “我有件事求你。”皮皮鲁开门见山。  
        “请讲。”探长林说。  
        “我的那架小飞碟最近在美国纽约闯了点儿祸……”皮皮鲁说。  
        “可不是‘点儿祸’,是大祸。”探长林纠正皮皮鲁的措辞。  
        “你知道纽约的事是我的飞碟干的?”皮皮鲁惊讶。  
        “还能是什么东西干的?只有你的飞碟有这个本事。”探长林翘大拇指。  
        “驾驶员受伤了,我必须马上去美国将五角飞碟弄回来。如果再出意外,搞不好美国该解体了。”皮皮鲁说,“希望你能帮助我出境。你知道自从上次我去德国后,我还没公开露过面。”  
        “这可不容易。”探长林说,“我办这件事是要坐牢的。”  
        “不容易才找你。”皮皮鲁说。  
        “你在家等着,一个小时之内答复你。就算我帮美国一次忙,美国欠我一次。”探长林走了。  
        皮皮鲁给解剖主任打电话,要求他做好抢救舒克的准备。  
        “为什么不现在抢救?”解剖主任纳闷。  
        “舒克现在美国。”皮皮鲁说。        
        “为什么不在当地找医生抢救?又不是在非洲沙漠。”解剖主任说。  
        “我们只信任你,不信任美国医生。”皮皮鲁说。  
        解剖主任从头舒服到脚。  
        50分钟后,探长林回来了。  
        他从西服内兜掏出一个信封。  
        “这是你的护照,签证也办好了。这是单程机票,相信你不需要返程机票。飞机1小时后起飞。”探长林从信封里拿出护照和机票递给皮皮鲁。  
        “谢谢!”皮皮鲁感激地说。  
        “我送你去机场,否则你赶不上飞机。我的车有警笛。”探长林好事做到底。  
        皮皮鲁略显踌躇,他不知道怎样当着探长林将贝塔和歌唱家装进自己的衣兜。  
        “我闭上眼睛,你爱干什么干什么。”探长林拥有一流的观察力。  
        “我没什么事需要瞒着你,你尽可以大睁双目。”皮皮鲁不好意思,他觉得如果现在还不对探长林开诚布公就太不够意思了。  
        贝塔和歌唱家和探长林打招呼后藏进皮皮鲁的衣兜。  
        探长林驾车送皮皮鲁去机场。一路呼啸。  
        到机场后,探长林送皮皮鲁出海关。  
        “一路平安。”探长林冲进入安全通道的皮皮鲁挥手。  
        “我回来后给你打电话。”皮皮鲁对探长林说。  
        经过十几个小时的不间断飞行,皮皮鲁、贝塔和歌唱家抵达美国纽约。  
        刚刚经过浩劫的纽约千疮百孔,皮皮鲁在去监狱的路上看着惨不忍睹的车窗外的街景,自责不已。他下决心销毁五角飞碟。他相信一旦五角飞碟落入坏人手中,这世界就完了。约翰不是坏人,尚且用五角飞碟把纽约折腾成这样。  
        皮皮鲁在监狱长彼得富氏的办公室见到了他该见的所有人。  
        彼得富氏和罗勃特向皮皮鲁致意,他们对于能发明五角飞碟的人佩服得五体投地。  
        约翰请求皮皮鲁处置他,皮皮鲁苦笑着拍了拍约翰的肩膀。  
        “我们必须立即回国,抢救舒克和克莉斯汀。”皮皮鲁对彼得富氏和罗勃特说。  
        “我有个请求,不知道能不能说。”彼得富氏问皮皮鲁。  
        “当然可以说,这次多亏了监狱长救了我和燕妮的命。”鲁西西替皮皮鲁回答。          第319集  
        监狱长带犯人出国旅游;  
        解剖主任磨炼承受力;  
        舒克接受治疗;  
        鲁西西别墅里的婚礼    
        皮皮鲁点头。  
        “我这辈子最大的愿望是看看中国的长城,可是一直没有机会。我知道五角飞碟能帮我完成这个夙愿,我能同你们一起去看看长城吗?”彼得富氏红着脸说。  
        皮皮鲁犹豫,他不知道能不能再让外人进入五角飞碟。  
        “为了安全起见,你们可以把我铐起来。”彼得富氏看出皮皮鲁的犹豫原因。  
        “让彼得富氏去吧!来回也就几十分钟。”鲁西西替监狱长求情。 
        燕妮也加入求情的行列。  
        “欢迎你去中国,不用戴手铐。要戴你早给鲁西西和燕妮戴了。”皮皮鲁同意了。  
        “我也想沾光陪监狱长去看看长城。”罗勃特说。  
        “我也去。”约翰小声说。  
        “你进入五角飞碟得戴手铐。”皮皮鲁对约翰提条件。  
        “同意。在监狱里陪读了30年的人还怕手铐?”约翰没意见。  
        “是陪住吧?”歌唱家纠正约翰。  
        “是陪读。监狱是地球上最伟大的大学。”约翰给歌唱家扫盲。  
        “你们得服药变小了才能进入五角飞碟。”皮皮鲁告诉彼得富氏和罗勃特。  
        “就是爱因斯坦家的老鼠发明的那种药。”鲁西西对彼得富氏说。她已经给监狱长讲过这个故事了。  
        “我们知道。”彼得富氏对皮皮鲁说。  
        “你需要安排一下工作吧?”皮皮鲁觉得监狱长和一个犯人同时失踪应该向同事打个招呼。  
        彼得富氏用电话通知值班狱警不要让任何人打搅他和8176的谈话。  
        微缩粒将皮皮鲁、彼得富氏和罗勃特变小了。  
        大家进人五角飞碟。  
        贝塔担任驾驶员。        
        五角飞碟径直在解剖主任家着陆。鲁西西和歌唱家留在解剖主任家陪同舒克和克莉斯汀治病。  
        解剖主任看见微型皮皮鲁和其他人后呆了5分钟。  
        “下个世纪全是让人这么吃惊的发明,您先适应一下也好。”皮皮鲁对解剖主任说。  
        五角飞碟送彼得富氏和罗勃特去长城。  
        彼得富氏和罗勃特是第一个乘坐五角飞碟游览长城的外国人,也是第一个监狱长和在押犯人携手游览长城的外国人。  
        五角飞碟将彼得富氏和罗勃特送回美国监狱,约翰表示留在监狱里继续陪罗勃特。  
        “我会很快给罗勃特办假释出狱。”监狱长对约翰说。  
        “那我也留在美国。我已经是美国人了。”约翰说。  
        “祝你们好运。”皮皮鲁给罗勃特和彼得富氏服用恢复原大的皮皮鲁口服液后向他们告别。  
        约翰、罗勃特和彼得富氏目送五角飞碟起飞。  
        五角飞碟在解剖主任家着陆时,舒克已经苏醒了。  
        “舒克脱离危险了,请放心。”解剖主任告诉微型皮皮鲁。  
        “谢谢你。”皮皮鲁感谢解剖主任。  
        “这些药你拿上,每天按时给他俩吃。有一个星期就能恢复。”解剖主任对皮皮鲁说。  
        皮皮鲁和鲁西西将舒克和克莉斯汀抬上五角飞碟。  
        解剖主任不惊讶了,他学会了以平常心看离奇的事,否则他进入21世纪后眼珠会瞪出来。  
        五角飞碟在皮皮鲁家着陆。  
        大家将舒克和克莉斯汀安置在鲁西西别墅里。死里逃生的舒克显得极有深度。  
        克莉斯汀喜欢这些新朋友,喜欢中国。  
        皮皮鲁给探长林打电话报平安。  
        皮皮鲁准备在舒克痊愈后销毁五角飞碟。  
        一个星期过去了。舒克和克莉斯汀的身体恢复了正常。  
        一天早晨,舒克告诉贝塔他准备和克莉斯汀结婚。  
        “涉外婚姻呀!”贝塔羡慕,“咱哥们儿居然娶了美国老鼠!”  
        皮皮鲁、鲁西西、燕妮和歌唱家都为舒克高兴。大家认为舒克和克莉斯汀的爱情是经过患难检验的。  
        鲁西西、燕妮和歌唱家在鲁西西别墅里为舒克和克莉斯汀布置了新房。  
        “应该通知舒利和辰羽参加婚礼,舒利是兴高采烈参加爸爸娶继母的婚礼的那种现代青年。”贝塔提议。  
        舒克和克莉斯汀在鲁西西别墅里举行了婚礼。歌唱家用歌声代表大家为他们祝福。          第320集  
        皮皮鲁宣布重大决定;  
        贝塔孤立无援;  
        贝戈展露天才;  
        上学实际上是上环境    
        这天,皮皮鲁向大家宣布他将销毁五角飞碟。  
        除了鲁西西,所有人都吃惊。  
        “为什么?”贝塔抗议。  
        “为了地球的平安。”皮皮鲁说。  
        大家想到了约翰,还有贝塔那次酗酒,以及糕鱼氏。  
        “我同意。”鲁西西投赞成票。  
        燕妮和歌唱家也举手。  
        贝塔看舒克。他希望舒克和他结盟反对。  
        “同意。”舒克脸上的表情比较痛苦。  
        “你是违心的。”贝塔冲舒克喊。        
        舒克摇头。  
        贝塔知道五角飞碟今天是必死无疑了。  
        “没有五角飞碟,咱们怎么找另外3个罐头小人?”贝塔做垂死挣扎。  
        “没有五角飞碟一样找罐头小人。咱们还有直升机和坦克。”皮皮鲁说。  
        “难度大多了!”贝塔强调。  
        没人声援贝塔。  
        “我有一个要求。”贝塔拖延销毁五角飞碟的时间。  
        “说吧。”皮皮鲁看着贝塔。  
        “我和歌唱家最近正为要不要孩子拿不定主意。我想通过五角飞碟的测上世下世功能测测我们的孩子的一生,就像燕妮测皮亚洁那样。如果我们的孩子一生幸福,我们就生他(她)。如果不幸福,我们就免生了。测完后,你再销毁五角飞碟,行吗?”贝塔说。  
        皮皮鲁看歌唱家。歌唱家点头。皮皮鲁同意了。  
        “现在就看?”鲁西西问贝塔。  
        “当然。呆会儿看皮皮鲁也不干呀。”贝塔说。  
        大家都很关注贝塔和歌唱家的孩子的命运,都表示要同歌唱家夫妇一起看。  
        大家进入五角飞碟找座位坐好,由子大家知道这是最后一次和五角飞碟相处,心情都十分特别。        
        皮皮鲁调试仪器。  
        屏幕上开始出现贝塔和歌唱家的孩子的一生。  
        贝塔和歌唱家的孩子是男孩儿,名字叫贝戈。他是地球上第一个鼠人,老鼠和人类通婚生育的后代。  
        贝戈具有人类的身躯,脸酷似老鼠,有尾巴。  
        大概由于贝戈是异族通婚的结果,他出生只几个月就无数次显示出天才征兆。地球上最聪明的动物除了人类就是老鼠,贝戈集中了人类和老鼠的优点。  
        在贝戈3个月时,一天皮皮鲁看《唐诗300首》,贝戈在一边玩。当皮皮鲁看完书将书放到沙发上后,贝戈开始自言自语朗诵唐诗。  
        开始时皮皮鲁没在意,当贝戈背到第5首诗时,皮皮鲁愣了。一字不差。紧接着贝戈一口气背完了整部《唐诗300首》。  
        皮皮鲁立即将这个信息传播给贝塔和歌唱家。  
        “你们的儿子是天才。”皮皮鲁下结论。  
        “我们应该怎么办?”贝塔和妻子异口同声。  
        “第一不要揠苗助长。第二为孩子创造机会。”皮皮鲁说。  
        第一条贝塔和歌唱家非常明白。  
        “我们怎么为他创造机会?”贝塔夫妇又一次不约而同张嘴。  
        “给贝戈喝皮皮鲁口服液,让他以正常人的身躯进入人类社会展示才能。”皮皮鲁说。  
        “贝戈的头基本上是我的头,人类社会能接受他吗?”贝塔担心。  
        “如果贝戈有突出的才能,我觉得问题不大。”皮皮鲁说。  
        “贝戈从什么时候开始接触人类?”歌唱家问。  
        “从上小学开始。”皮皮鲁说。  
        “贝戈这么聪明,还用上学吗?”贝塔问。  
        “上学不是为了学知识,上学也不能把孩子变聪明。上学实际上是上环境。这个环境不是建筑环境,而是人文环境,也就是由老师和同学共同营造的友爱环境。”皮皮鲁说。  
        “如果孩子在学校里得不到这样的环境呢?比如老师非常粗暴,学生见了老师吓得要死。”贝塔回忆起皮皮鲁小时候的老师。  
        “那是这个孩子一生的不幸。”皮皮鲁神情黯然。  
        “怎么能保证贝戈碰上一个好老师呢?我说的好老师是指对学生拥有真正的爱心那种老师,不是把分数看得高子一切的老师。”贝塔问皮皮鲁。  
        “只能碰运气。”皮皮鲁叹气。          第321集  
        贝戈在数学课堂上一鸣惊人;  
        家长反对自己的孩子和老鼠同在一个班;  
        校长拒绝给贝戈做体检;  
        曹雪芹差点儿写不出《红楼梦》    
        贝塔和歌唱家同意给贝戈服用能使他的身体变得和正常孩子一样大的皮皮鲁口服液。  
        在贝戈到了上学的年龄时,皮皮鲁将他送进附近的一所小学。  
        老师对于这个长得像老鼠的学生没有歧视,倒是有几个同学嘲笑贝戈的形象。  
        一天,在课堂上,一位同学公然嘲讽贝戈长得像老鼠。一些同学笑。老师批评那同学不能以貌取人,还说电视台有个栏目的女主持人长得就像老鼠,可观众照样爱看她主持的节目。        
        有一次上数学课,数学老师看见贝戈不听讲而是在埋头干什么。老师叫贝戈站起来。  
        “贝戈,你在干什么?”  
        “我在解哥德巴赫猜想。”贝戈说。  
        同学们大笑。  
        老师很生气,让贝戈把纸交给她。  
        “已经完成了。”贝戈将他利用半节课时间证明了的哥德巴赫猜想数学题交给老师。  
        同学们又笑。  
        数学老师在师范学院时是高材生,她对哥德巴赫待想这种困扰人类多年的数学难题有一知半解的了解。她发现贝戈给她的纸有重大价值。  
        发出笑声的同学从老师的脸上得出了不应该再继续笑的信号,他们改用惊讶的目光看贝戈。  
        “这是你刚刚解的?”老师问贝戈。  
        “是的。”贝戈回答。  
        老师示意贝戈坐下。  
        下课后,老师给科学院数学研究所打电话。  
        “什么?你是说你们学校有个一年级学生攻克了哥德巴赫猜想?开什么玩笑?”对方挂了电话。  
        数学老师重新拨通电话。  
        “你还有完没完?”对方火了。  
        “我把贝戈解的哥德巴赫猜想传真给你们看看。”数学老师不屈不挠。        
        数学研究所接到传真一看就傻了。  
        哥德巴赫猜想被贝戈这个小学一年级学生解决了。  
        一夜之间,贝戈的名字家喻户晓。  
        贝戈的同班同学的父母的自尊受到了不同程度的伤害,他们嫉妒贝戈的家长和贝戈本人。  
        一位同学的妈妈在见了贝戈后找校长要求给自己的孩子换班。  
        “为什么?”校长问。  
        “我不能让我的孩子和老鼠在一个班上课。”那母亲说。  
        “什么,什么老鼠?”校长听不懂。  
        “贝戈是一只老鼠。”那母亲说。  
        “您怎么这么说话?”校长正色道。  
        “我要求给贝戈做体检!”那母亲说。  
        “您没有这个权力。”校长拒绝这位学生家长的无理要求。  
        第二天,几乎全班学生的家长联名上书校方,要么给贝戈做体检,要么给他们的孩子办集体转班。  
        校长答复家长,他同意第二条。  
        于是,班里只剩贝戈一位学生了。  
        “我给你一个人配备最优秀的教师。是他们转班,不是你转班。”校长对贝戈说。  
        贝戈忍住眼泪点头。        
        从此,贝戈的班上只有贝戈一个学生。  
        贝塔在家里气愤,舒克安慰他。  
        “你不要把孩子上学看得太重。”舒克说,“咱们假设曹雪芹从小就上一所专门培养作家的学校,他后来能写出《红楼梦》?还有巴尔扎克,还有托尔斯泰。这些大作家如果从小上专门培养作家的学校,我断定他们长大后能成为作家的少。”  
        “你原来好像说过作家的才能是天生的,既然是天生的,应该上什么学校都不会影响作家的才能发挥呀!”贝塔反驳。  
        “所以说不能太看重学校教育。依我看,培养什么其实是摧毁什么。作家的才能是天生的,但如果作家从小上专门培养作家的学校,保准他的才能被学校摧毁。吃腻了是人们常说的一句真理。学腻了是人们尚且没有认识到的另一个真理。作为父母,如果真想让孩子往某个方向发展,他们应该干的最重要的事就是千万别让孩子学腻了。”舒克想尽办法使贝塔不难过。  
        “谢谢。”贝塔好受多了。          第322集  
        别人不是平白无故地找你的碴儿;  
        贝戈的尾巴吓跑了恋人;  
        总统候选人因父亲失踪退出竞选;  
        贝塔骑自行车    
        贝戈用3年时间连续跳级孤独地上完了小学,他在中学又遇到了同样的问题:由于贝戈智商太高,其他学生的家长以贝戈像老鼠为由不让自己的孩子和贝戈在一个班。  
        贝塔和歌唱家为此极为沮丧。  
        “人活在世界上,往往会有人找你的碴儿。本事越大,找你的碴儿的人就越多。没人找白痴的碴儿。如果贝戈没有你的血统,人们仍然会因为他太智慧而找他别的碴儿。”皮皮鲁宽慰贝塔夫妇。  
        “还不如生个傻子。”贝塔说气话。  
        “老鼠和人结婚能生傻子?我现在怀疑伟人都是人类和异类通婚的产物。贝戈太聪明了,人和人结婚根本生不出这么聪明的后代。人和人通婚本质上属于近亲繁殖。”皮皮鲁说。  
        贝戈到了青春期,他爱上了一个姑娘。爱得诚恳、痴情。  
        那姑娘爱才,她没有嫌弃贝戈的鼠相。  
        但当贝戈和姑娘的爱情发展到姑娘能够发现贝戈有一条尾巴的阶段时,姑娘挥泪离开了贝戈。  
        姑娘能忍受恋人身上旁人看得见的缺陷,不能忍受旁人看不见的缺陷。结婚实质上要的是配偶身上旁人看不见的东西,包括生理上的和心理上的。  
        贝戈从此将爱情的闸门牢牢锁死,他把所有精力用在事业上。  
        贝戈在许多领域创建丰功伟业后,最终走上了从政之路。他成为总统候选人。  
        竞争对手在即将败北时对贝戈使用了杀手锏。  
        在一次被选民倍加关注的电视辩论时,另一位总统候选人突然提出了贝戈的血统问题。  
        “恕我怀疑贝戈先生的血缘,我认为贝戈先生身上只有一半是人类的血缘,另一半我不知道是什么种族的。据说没有任何人见过贝戈先生的父亲。我希望贝戈先生能把自己的爸爸带到电视台来让选民过目。”那候选人说。  
        贝戈不能将贝塔带到电视台来。        
        贝戈被迫退出总统竞选。  
        在五角飞碟里观看儿子一生的歌唱家已是泪流满面,贝塔也神情凝重。  
        大家都不说话。皮皮鲁关机。  
        “咱们不要孩子了。”歌唱家对贝塔说。  
        贝塔点头。  
        “循规蹈矩是阻碍人类发展的最大障碍。”鲁西西说。  
        销毁五角飞碟的时候到了。  
        大家恋恋不舍地看五角飞碟里设施。  
        “我觉得你把五角飞碟的功能去掉就行了.留下外壳,也算是个纪念。”鲁西西对皮皮鲁说。  
        皮皮鲁想了想,同意了。  
        贝塔心里好受点儿。  
        皮皮鲁单独在五角飞碟里工作了30分钟,五角飞碟变成了一个名副其实的玩具飞碟。  
        大家站在四周看着,他们的脑海里是纽约惨不忍睹的场面。  
        贝塔不停地小声用英语骂约翰。  
        “皮皮鲁是对的。”燕妮说。  
        “五角飞碟一旦落到坏人手里,后果不堪设想。”鲁西西说。  
        皮皮鲁从五角飞碟里出来。        
        “我和歌唱家能不能住在五角飞碟里?”贝塔问皮皮鲁。  
        “可以。”皮皮鲁答应贝塔的要求。  
        “贝塔不会偷偷将五角飞碟的功能恢复了吧?”舒克半开玩笑  
      “绝对不会。”皮皮鲁说。  
      “小看人。”贝塔说。  
      “如果是贝戈准能恢复。”皮皮鲁说。  
      “那我们就生他。”贝塔搂着歌唱家的肩膀开玩笑。  
        中午,大家在鲁西西别墅共进午餐。  
        “舒克,听皮皮鲁说,你的直升机开得很帅,吃完饭给我们表演一下怎么样?”克莉斯汀在饭桌上说。  
        “我开得也不错。”贝塔插话。  
        “你不是开坦克吗?”克莉斯汀说。  
        “开坦克是正业,开直升机是副业。”贝塔说,  
        “讲讲你和舒克是怎么认识的。”克莉斯汀说。  
        “为了一只麻雀。舒克打抱不平用直升机压我的坦克。后来又把我的坦克吊到天上。我和舒克是不打不相识。”贝塔眉飞色舞。  
        “你们是怎么认识皮皮鲁的?”克莉斯汀觉得老鼠认识老鼠不难,老鼠认识人的难度比较大。  
        “一次,我们的直升机没电了,不得已在皮皮鲁家的阳台上迫降,就这么结识了皮皮鲁。”舒克说。  
        饭后,舒克和贝塔驾驶直升机和坦克为大家表演。  
        鲁西西问贝塔的感觉。  
        “开完了航天飞机再骑自行车。”贝塔一脸的不堪忍受。          第323集  
        皮皮鲁拉探长林下水;  
        皮皮鲁开车不认识路;  
        贝塔打扑克时走神儿;  
        舒克和贝塔为鼠标自豪   
        吃晚饭时,贝塔问皮皮鲁:  
        “咱们以后就这么老闷在家里?不出去了?”  
        “这么发达的社会,不用交通工具也能周游世界。”皮皮鲁一边往自己嘴里塞肉一边说。  
        除了克莉斯汀大家都停止进食,都对皮皮鲁的话感到不解。克莉斯汀到中国后大吃特吃,她终于摆脱了食不果腹的贫穷生活。  
        “现在,因特网已经将全世界的电脑联网,任何人只要坐在家里,就可以通过电脑的因特网周游世界。我明天就去舒克贝塔公司拿一台电脑回来。”皮皮鲁说。        
        “我能跟你去吗?”燕妮问。她想以正常人的身躯逛北京。  
        “我建议你托探长林给燕妮办中国绿卡,这么让咱们中国人扬眉吐气的婚事有关部门不会不开绿灯。”鲁西西对皮皮鲁说。  
        “我觉得皮皮鲁、燕妮和鲁西西平时还是应该用正常人的身躯生活,遇到特殊情况再变小。”舒克提议。  
        “鲁西西也该关注一下舒克贝塔公司的经营状况了,用我们的名字命名的公司如果搞不好不是砸我和舒克的牌子吗?”贝塔说。  
        “舒克贝塔公司运转得很正常,我任命的一位常务副总经理非常能干和敬业。”鲁西西说。  
        “舒克贝塔公司应该办一本少儿刊物。”歌唱家提议。  
        “这个建议可以考虑。”鲁西西说。  
        “舒克当过《老鼠报》主编。”贝塔推荐舒克的办报刊才能。  
        “办刊物可以缓一步,咱们先搬一台电脑回来。”皮皮鲁说。  
        “那电脑还能比五角飞碟上的电脑先进?”贝塔问。  
        “五角飞碟里的电脑没有联网,是孤立的。”皮皮鲁说。  
        晚饭后,皮皮鲁、燕妮和鲁西西喝皮皮鲁口服液变大了。        
        皮皮鲁给探长林打电话,请他帮助燕妮办中国绿卡。  
        “燕妮的国籍?”探长林问。  
        “德国。”皮皮鲁回答。  
        “同你的关系?”  
        “夫妻。”  
        “什么时候入境的?”  
        “无可奉告。”  
        “我已经快成偷渡专家了。”  
        “不好意思拉你下水。”  
        第二天,探长林给燕妮办了咖啡色的中国绿卡。  
        燕妮欣喜若狂,她终于可以大摇大摆逛中国的大街了。  
        皮皮鲁好长时间没开汽车了,他用了一个小时才把汽车擦干净。  
        “我们去舒克贝塔公司拿电脑,马上回来。”皮皮鲁对舒克、贝塔、歌唱家和克莉斯汀说。  
        “快去快回。”贝塔说。  
        皮皮鲁、燕妮和鲁西西驾车前往舒克贝塔公司。  
        “都不认识路了。”皮皮鲁边开车边说。  
        “几天不见就冒出一座楼。”鲁西西有同感。  
        燕妮的眼睛不够用。  
        到公司后,鲁西西去总经理办公室听取常务副总经理的工作汇报,皮皮鲁和燕妮满公司挑好电脑。        
        “这台不错。”皮皮鲁选中了一台电脑。  
        公司职员将该电脑里储存的资料复制。  
        鲁西西留在公司处理业务,皮皮鲁和燕妮驾车拉着电脑回家了。  
        舒克夫妇和贝塔夫妇在鲁西西别墅里打扑克,他们玩得很高兴。贝塔对现在的生活颇满意,他不知为什么想到了咪丽。  
        “出牌呀!”歌唱家催丈夫。  
        贝塔忙出牌。  
        “想什么呢?”舒克问贝塔。  
        “还记得咪丽吗?”贝塔问舒克。  
        “当然记得。她照看过我妈妈。”舒克说。  
        “谁是咪丽?”克莉斯汀问。  
        贝塔一边打牌一边给克莉斯汀和歌唱家讲咪丽的故事。  
        皮皮鲁和燕妮抱着电脑回家了。  
        “不玩了,帮皮皮鲁安装电脑去。”贝塔说。  
        舒克和贝塔在电脑土机箱后边连接插头。  
        “这是鼠标的插头。”皮皮鲁递给贝塔一个插头。  
        “咱们老鼠的确伟大,连电脑上的重要部件都用咱们的名字命名。”贝塔边连接鼠标的插头边对舒克说。  
        “老鼠是人类的朋友。”舒克说,“凡是地球上还活着的,都是人类的朋友。”  
        “你这话是真理。”贝塔表扬舒克。          第324集  
        舒克和贝塔惊呼电脑软件界面太丑陋;  
        贝塔决定开发硬软件;  
        皮皮鲁家鸦雀无声;  
        舒克反复声明自己是老鼠别人死活不信    
        电脑安装完毕,皮皮鲁开机。  
        当显示屏上出现图像时,贝塔和舒克不约而同大呼:  
        “太落后了!界面太难看了!”  
        的确不能和五角飞碟里的电脑同日而语。  
        舒克和贝塔没见过当前世界上流行的个人电脑,他俩只见过五角飞碟里皮皮鲁制作的电脑。  
        “这台电脑是目前世界上最先进的个人电脑。”燕妮说。        
        “不包括五角飞碟里的电脑吧?”贝塔叫真。  
        “当然。”燕妮承认自己的结论不严谨。  
        皮皮鲁按键盘打开一个著名的软件。  
        “太原始了!这不能叫软件。”贝塔摇头。  
        “叫软软件,太软。”舒克嗤之以鼻。  
        “你开发一个硬软件。”皮皮鲁逗贝塔。  
        “你给我制造一台适合我使用的微型电脑,我弄个硬软件出来回报人类,也算对得起人类用我们老鼠的名字命名电脑部件。”贝塔认真了。  
        “有笔记本电脑,也叫便携机,你用可能正合适。”燕妮对贝塔说。  
        “还大。我给贝塔和舒克专门制作两台微型电脑,就像普通钱包那么大。”皮皮鲁说,“看看贝塔能设计出何等伟大的硬软件。”  
        “人类的电脑史要改写了。”贝塔放话。  
        说干就干,皮皮鲁投入为舒克和贝塔制作微型电脑的工作。  
        两个星期后,两台钱包大的微型电脑问世。  
        麻雀虽小,五脏俱全。微型电脑的功能一点儿不比台式电脑少。  
        电脑好像有一种魔力,任何生命都无法抵御它。舒克和贝塔一上机就被电脑俘虏了。过去他们在五角飞碟里使用电脑都是为了别的目的,他们没有对电脑本身发生兴趣。现在不一样了,可以说他们与电脑融为一体了。  
        贝塔使用鼠标得心应手,他使用电脑的感觉是老鼠的智商加上电脑真可谓如虎添翼。  
        克莉斯汀和歌唱家也迅速对电脑发生了极大的兴趣,在她们的要求下,皮皮鲁又制作了两台微型电脑。  
        燕妮和鲁西西也配置了电脑。  
        皮皮鲁将这7台电脑统统加入了国际电脑因特网,他们可以坐在电脑前通过全球信息高速公路(INTERNET)同全世界交谈,可以收看世界各地各行各业的各种信息,游览风光名胜,参加各种电脑沙龙。  
        皮皮鲁的家里从来没这么安静过,能听见的声音几乎只有键盘的敲击声和鼠标的按击声。  
        贝塔在鲁西西别墅里终日与电脑为伍,其程度是废寝忘食如醉如痴。别人也毫不逊色。歌唱家守着电脑已经一个星期没唱歌了,她和贝塔还创造了一个星期没说一句话的记录。舒克夫妇也是每天死守电脑,舒克通过电脑互联网络同世界各地的电脑爱好者交谈。过去,除了皮皮鲁这几位朋友,舒克想同人类其他成员交谈的可能性几乎没有。现在,舒克可以光明正大地同人类成员畅所欲言。电脑是人类和动物真正成为朋友的纽带,可惜现在的地球上还只有舒克和贝塔少数几个动物使用电脑。        
        皮皮鲁和燕妮更是各守一台电脑忘乎所以。皮皮鲁开始改进电脑。  
        大家将经营舒克贝塔公司的任务全推给了鲁西西。  
        一天,舒克在电脑屏幕上收到了一条信息,一个人希望通过电脑网络交一个对环境保护感兴趣的朋友。  
        舒克呼应,表示愿意同他(她?)交朋友。  
        “你在哪个国家?”对方通过在电脑屏幕上显示文字问舒克。  
        “中国。你呢?”舒克敲键盘。  
        “俄罗斯。我叫约瑟夫。你叫什么名字?”  
        “我叫舒克。是一只老鼠。”  
        “你真幽默。我是苍蝇。”  
        “我真的是老鼠。”  
        “我真的是苍蝇。”  
        “你对环境保护感兴趣?”  
        “我是国际绿色组织成员,保护人类生存环境是我的终身奋斗目标。”  
        “干吗只是保护人类生存环境?应该包括动物。”  
        “你说得对,保护动物就是保护人类。”  
        “干吗总把保护人类当终极目的?地球是属于所有生命的。”  
        “……”        
        “你不同意?”  
        “同意!你的观点很新颖。动物是人类的朋友。”  
        “干吗不说人类是动物的朋友?”  
        “……”  
        “你不同意?”  
        “你的新观念真多。”  
        “我是老鼠呀!”  
        “你很逗。”  
        舒克就这么每天通过电脑互联网络和在世界各地的许多人交上了朋友,聊一切喜欢聊的话题。          第325集  
        划时代的BEITA诞生;  
        探员有的休克有的跳槽;  
        需要学习才能使用的电脑不是电脑是烦恼;  
        鲁西西代表朋友荣登世界首富宝座    
        一天清晨,大家还在梦中,贝塔大喊:  
        “成功了!我成功了!”  
        “什么成功了?”舒克睡眼惺松地问身边的克莉斯汀。  
        “大概是贝塔开发的硬软件成功了。贝塔的房间通宵亮着灯。”克莉斯汀说。  
        舒克连睡衣都顾不上换,往贝塔的房间跑。  
        自从皮皮鲁给贝塔配备了微型电脑后,贝塔一直在埋头开发划时代的电脑应用软件。  
        舒克跑进贝塔的房间时,看见贝塔正在自己的电脑前跳迪斯科。  
        舒克只看了一眼电脑屏幕,就知道地球上所有同电脑软件有关的公司将在一个星期内宣告破产。  
        贝塔的电脑屏幕上显示的软件界面酷到极点了,只能用耀武扬威和气吞山河来形容。  
        皮皮鲁和燕妮在鲁西西别墅外边往贝塔的房间看。  
        “回到你的电脑那儿去,我把新软件输送到你的电脑里去。”贝塔对窗外的皮皮鲁说。  
        皮皮鲁和燕妮回到自己的电脑桌前打开自己的电脑。  
        “贝塔万岁!”皮皮鲁看到电脑屏幕上的界面情不自禁。  
        贝塔反而谦虚地评价皮皮鲁刚才喊的口号:  
        “有点儿过。”  
        “不是有点儿过,是太过了。”舒克逗贝塔。  
        皮皮鲁使用贝塔开发的软件。  
        “全世界都在里面了!集人类和老鼠的智慧之大成!太伟大了!”皮皮鲁兴奋不已。  
        “人绝对开发不出来。”燕妮说。  
        “没错。老鼠的脑沟回和人的不一样。”皮皮鲁同意太太的结论。  
        大家聚在一起探讨贝塔的软件。  
        “应该投入市场。”鲁西西说。  
        “给这个软件起个名字。”舒克说。        
        “应该以贝塔的名字命名。”克莉斯汀说。  
        “就叫BEITA吧!”燕妮提议。  
        大家鼓掌同意。贝塔热血沸腾。  
        电脑史上最伟大的软件BEITA诞生。  
        “我们美国的软件没戏了。”克莉斯汀说。  
        “我组织一个新闻发布会,就由舒克贝塔公司生产BEITA软件。”鲁西西说。  
        在鲁西西主持的BEITA新闻发布会上,记者们被BEITA的风采迷住了。  
        “请问,贵公司不是电脑软件开发公司,为什么能开发出如此精妙绝伦的软件?”有记者问。  
        “有意栽花花不开,无心插柳柳成行。”鲁西西回答。  
        “请问,BEITA比目前流行的最好的软件先进多少?”又有记者问。  
        “最低估计先进85倍。”鲁西西说。  
        记者们对鲁西西的谦虚报以掌声,因为白痴也能看出BEITA比目前最好的软件先进150倍以上。  
        根本没把舒克贝塔公司开发的新软件放在眼里的世界几家头号软件公司在得知了BEITA的准确信息后火速派探员赶往新闻发布会现场,有的探员一看BEITA就休克了。没休克的探员纷纷找鲁西西求职。  
        BEITA软件在一个月内席卷全球。所有著名的电脑公司都同舒克贝塔公司签约在他们出售的电脑里预装BEITA软件,没有BEITA的电脑根本卖不出去。  
        BEITA软件的最大特点是任何人包括文盲都能在5分钟内掌握和使用它。  
        电脑是伺候人的。如果使用电脑还需要学,它就不配叫电脑。  
        贝塔解决了困扰人类多年的这个问题。  
        皮皮鲁和舒克也不甘落后。  
        皮皮鲁制造出了尖端电脑,其运转速度接近了音速。  
        舒克发明了傻瓜电脑,傻瓜电脑的结构简单得不能再简单了。任何幼儿园的小朋友都能在10分钟内拆装两遍以上傻瓜电脑。  
        预装BEITA的傻瓜尖端电脑的问世使电脑真正成为人类所有成员不可或缺的朋友。  
        无数吃电脑饭的专业人员失业了。傻瓜尖端电脑根本就没有修理这么一说,谁都能修。  
        舒克贝塔公司的资产迅速蹿升,一跃而为世界第一大企业。  
        鲁西西代表舒克、贝塔和皮皮鲁成为报刊上评选的世界首富。  
        舒克、贝塔和皮皮鲁在幕后不露面。  
        “电脑比五角飞碟有意思吧?”皮皮鲁问贝塔。  
        “电脑实际上是属于老鼠的。”贝塔说。  
        尽管皮皮鲁不同意,可他没反驳。          第326集  
        前总裁怀疑贝塔不是人;  
        当铺收到结婚戒指;  
        经济间谍是时代的产物;  
        亿万富翁和穷光蛋只差一个台阶    
        世界上所有破产的电脑和软件大公司的老板们在度过了跳楼的危险期后,云集在欧洲的一座小城共商对策。这些昔日的仇人今天摒弃前嫌团结一致共同对付舒克贝塔公司。  
        “太不像话了,把软件垄断了不说,又把硬件也垄断了!”一家前大电脑公司的前董事长痛斥舒克贝塔公司。  
        “他们的BEITA不是人开发的。”一家前大软件公司的总裁说。  
        “他们不是人,是强盗!”大家赞同总裁的话。  
        “我说的不是人不是骂他。我是说开发BEITA的人不是人类成员。”前总裁说。  
        “不是人是什么?”大家惊异。  
        “不知道。反正这样的软件依靠人类的智慧绝对开发不出来。这只能是人类和别的种族的生物共同开发的产物。”前总裁用极其肯定的口气说。  
        大家像听天方夜谭。  
        “还有傻瓜电脑,也不可能是人类成员的杰作。人类的脑子根本想不出这样的结构。人的两只眼睛的焦距做不出这样的东西。我精确计算过了,造傻瓜电脑的‘人’的两只眼睛之间的焦距比人远多了。”前总裁的话使大家面面相觑。  
        “只有尖端电脑像是出自人手。这种思路是人的思路。”前总裁说。  
        “咱们还能扭转败局吗?”前某王牌电脑的老板问。  
        “咱们必须设法搞清楚舒克贝塔公司到底是通过什么动物开发的BEITA软件和傻瓜电脑,而后,咱们也利用这种动物开发出更新的产品,击败舒克贝塔公司。”前总裁的确聪明。  
        “用什么方法刺探?”有人问。  
        “聘请中央情报局的特工去完成这件事。必要时绑架开发傻瓜电脑和BEITA软件的动物。”前总裁说。  
        “他们要价很高吧?”囊中羞涩的前老板们现在必须一分钱掰成两半花。  
        “对。但咱们一定要凑足这笔钱。否则咱们就永无出头之日了。”前总裁说。  
        “大概需要多少钱?”有人问。  
        “找到了目标付5万美元。找不到付5千美元。签合同时付300元定金。”前总裁说。  
        大家翻衣兜,共集资70美元。  
        为了事业,当即有3人当了自己的结婚戒指。  
        好不容易凑齐了300美元,大家委托前总裁前往中央情报局签合同。  
        由于冷战结束而资金不足的中央情报局巴不得通过扮演经济间谍的角色搞点儿创收。  
        “我们手又还有十几个类似的业务,等办完了这些,马上就办你们的。”接待前总裁的中央情报局官员说。  
        “希望能尽快。”前总裁说。  
        “去中国刺探经济情报有时很容易,有时很难。特别是你们要求弄活人回来,难度更大,”官员说。  
        “我估计这个‘活人’的体积大不了,说不定也就手提包那么小。”前总裁说。  
        “我们在全世界有情报网。我会在我们的特工出发前先让驻守中国的情报人员打探舒克贝塔公司的情况。”中央情报局官员说。  
        前总裁恨不得现在就能知道是什么动物使得他的软件被用户扔进垃圾堆的。  
        “您回家听信儿吧。”官员送客。  
        前总裁落魄后住在贫民区度日如年。  
        亿万富翁变成穷光蛋的滋昧不好受。  
        其实亿万富翁和穷光蛋之间的距离是地球上最近的距离,比百万富翁和十万富翁之问的距离近多了。          第327集  
        歌唱家吃电脑的醋;  
        音乐软件罢免作曲家;  
        解剖主任的电话;  
        电脑是职业杀手    
        贝塔在BEITA大获成功后,再也离不开电脑了,他每天除了吃饭睡觉就是和电脑在一起。  
        贝塔在电脑上通过全球信息高速公路几乎走遍了世界的每个角落。  
        “比乘坐五角飞碟旅行还方便吧?”皮皮鲁问贝塔。  
        贝塔点头。  
        “你干脆和电脑结婚吧!”歌唱家嗔怪贝塔。  
        “我给你编一个音乐软件,能帮你作曲。”贝塔对夫人说。  
        “真的?”歌唱家知道如果贝塔开发出音乐软件不光她受益,还将改变世界。  
        贝塔只用了一个星期就编制了可以作曲的电脑软件。  
        歌唱家在电脑上使用该软件谱的曲超过了红沙发音乐城。  
        “贝塔砸了所有作曲家的饭碗。”鲁西西说。  
        “真没想到贝塔有这种才能。”歌唱家感慨。  
        “电脑加鼠脑,无往而不胜。”舒克说。  
        “人类用老鼠的名字命名电脑的鼠标不是偶然的。”燕妮说。她想起了住在自己在德国的乡间别墅里的老鼠的智商。  
        “人类和老鼠较量了几千年也没战胜老鼠,说明老鼠的确聪明。智商起码不低于人类。”皮皮鲁说。  
        “那些电脑公司怎么就想不起来聘用老鼠为他们开发软件硬件呢?”燕妮说。  
        “一般的企业家没有这种眼光,这需要强大的想像力支持。”舒克说。  
        电话铃响了。  
        皮皮鲁接电话。  
        “我是皮皮鲁。您好。”皮皮鲁看了一眼舒克。  
        “解剖主任。”贝塔判断。  
        “差不多。”舒克同意贝塔的推理。  
        “您太客气,请讲。”皮皮鲁说。  
        “他有事求咱们。”贝塔又揣测。        
        “这件事…我想想再答复您…不过估计难度比较大。”皮皮鲁挂了电话。  
        “舒克的救命恩人又出书了?”贝塔笑。  
        “比出书麻烦多了。”皮皮鲁说。  
        “他遇到什么困难了?”舒克问。  
        克莉斯汀也非常关心。解剖主任也是她的救命恩人。  
        皮皮鲁向朋友们转述解剖主任遇到的难题:  
        解剖主任的弟弟乃一知名电影导演,艺名雄起。雄导最近执导一部名为《初升的夕阳》的电影,该片由一位声名显赫的女影星主演。在影片开机前,《初升的夕阳》剧组与女影星签订了合同,剧组按照合同向女影星支付了250万元拍摄费。  
        当影片拍摄到一半时,女影星突然提出要求增加一倍拍摄费。剧组不同意。于是女影星拂袖而去。《初升的夕阳》剧组搁浅。戏已拍了一半,换主角是不可能了。可不换又有什么办法?一急之下,雄导血压骤增,住进了医院。  
        “这女影星人品欠点儿火候。”贝塔说。  
        “看来雄导犯了个错误,他以为演员的演技和品质是划等号的。”舒克说。  
        “解剖主任向咱们求援,他觉得只有咱们有办法让那女影星回来接着拍电影。”皮皮鲁说。  
        “除非咱们出钱。”歌唱家说。        
        “不能惯她这毛病。虽说咱们是世界首富,钱也不能这么花。”燕妮说。  
        “不给她加钱想让她再接着拍可不容易。”克莉斯汀说。  
        “拿别人一把是人类的恶劣品质之一。”贝塔说。  
        “人活着千万不能被别人拿住。”歌唱家深有体会地说。  
        “咱们还是商量怎么帮解剖主任吧。”皮皮鲁说,“他在等答复。”  
        “我觉得电脑能帮解剖主任的忙。”鲁西西说。  
        大家的眼睛都亮了。  
        “舒克和贝塔设计一种电影软件,替代那位女影星。”鲁西西指路。  
        贝塔很是兴奋,他知道电影史上的真正革命到来了。演员这一职业将进博物馆。  
        “电脑才是货真价实的职业杀手,人类社会的很多职业将被它扼杀。”舒克说。  
        “只要有那女影星的全身照片,我就能开发出替代她的电脑影星。”贝塔信心十足。  
        皮皮鲁给解剖主任打电话。          第328集  
        雄导在病床上和哥哥通电话;  
        女影星的右眼皮跳动;  
        导演乐软件旭日东升;  
        女影星拍摄沐浴镜头    
        “我们设法帮助你弟弟。”皮皮鲁通过电话告诉解剖主任。  
        “太感谢了!”解剖主任激动。  
        “我们需要那位女影星的全身照片,最好有各种角度的照片。”皮皮鲁说。  
        “我马上和我弟弟联系。照片送到哪儿?”  
        “送到舒克贝塔公司。”皮皮鲁说。  
        解剖主任挂断皮皮鲁的电话后立即给弟弟打电话。  
        副导演将移动电话递给躺在病塌上的雄导。  
        “您哥哥的电话。”副导说。        
        雄导有气无力地和哥哥通电话。  
        “要她的照片干什么?”雄导对哥哥的要求不解。  
        “我的朋友能帮你度过难关。”解剖主任说。  
        “赞助?我不要!即使不拍了,我也不多给她一分钱!”  
        “你快派人把她的剧照送到舒克贝塔公司去,必须有全身照,反正照片越多越好。”  
        “舒克贝塔公司?就是那家全球最大的电脑公司?你怎么认识他们的?”雄导觉得有戏了。舒克贝塔公司的实力太雄厚了。人人皆知。  
        雄导派人将女影星的数十张照片火速送到舒克贝塔公司。  
        鲁西西将照片拿回家中。  
        贝塔和舒克使用扫描仪把女影星的光辉形象扫进微型尖端傻瓜电脑,舒克和贝塔开始在电脑上根据女影星的形像制作女影星的电脑模拟人。  
        舒克和贝塔边探讨边操作,鼠标在他们手里游刃有余左右逢源。  
        两只老鼠通过电脑随心所欲地改变世界。  
        大家趴在窗外看舒克和贝塔巧夺天工地开发影星软件,都觉得很享受。  
        “那女影星的右眼皮现在肯定在跳。”歌唱家说。  
        “什么意思?”克莉斯汀不懂。  
        “中国人认为左眼皮跳有好事,右眼皮跳有灾。”燕妮给克莉斯汀上风俗课。  
        3天后,影星软件问世。该软件可以模拟替代任何影星拍电影电视剧。还可以创造出新的影星。使用影星软件拍摄的电影电视剧同真人拍摄的相比一点儿不逊色,而且更逼真更生动更栩栩如生。  
        一部原本需要1亿美元投资的电影,使用电脑影星软件拍摄只需10万美元还绰绰有余。影片质量后者较之前者强许多。  
        “咱们给这个软件起个名字。”皮皮鲁说。  
        “叫‘导演乐’怎么样?”舒克提议。  
        “不错。先让导演乐几天,到咱们开发出导演软件为止。”贝塔说。  
        皮皮鲁立即打电话给解剖主任,通知他马上让雄导为《初升的夕阳》剧组添置电脑。  
        “我和燕妮携带‘导演乐’去剧组协助拍摄。”皮皮鲁说。  
        “等帮完了解剖主任,咱们使用‘导演乐’软件拍一部电影,舒克当编剧,皮皮鲁当导演。”贝塔提议。  
        “太棒了!歌唱家唱主题歌。”鲁西西兴奋。  
        “咱们在影视界也潇洒一回。”克莉斯汀说。  
        “我和贝塔在舒克贝塔航空公司拍过电影,有基础。”舒克说。  
        皮皮鲁和燕妮携带导演乐软件前往《初升的夕阳》剧组。  
        解剖主任带着半信半疑的弟弟雄导在剧组驻地门口迎候皮皮鲁。  
        客套话一完,雄导就迫不及待地问皮皮鲁:  
        “电脑能替代那女影星?”  
        “她在这里。”皮皮鲁从衣兜里拿出一张软盘给雄导看。  
        雄导一脸的迷茫。  
        “电脑在哪儿?”皮皮鲁问雄导。  
        “请跟我来。”雄导带客人走进放电脑的房间。  
        皮皮鲁将导演乐输入电脑。  
        女影星出现在电脑屏幕上。  
        雄导问:  
        “她能和真人一同拍电影?”  
        “真人先和空气一同拍,在后期制作时再把她补上去。”皮皮鲁说。  
        “电脑上的她能像真人那样听我指挥?”雄导问。  
        “比真人更听你指挥。你让她干什么她都无条件地干,而且分文不取。”皮皮鲁说,“咱们现在试试。”  
        雄导兴致大发,他指挥电脑上的女影星表演,皮皮鲁通过键盘向电脑中的女影星转达雄导的指令。  
        女影星不折不扣地执行雄导的意志,包括当众表演沐浴的场面而面无惧色旁若无人不知羞耻。          第329集  
        女影星状告《初升的夕阳》剧组;  
        舒克使用电脑写电影剧本;  
        男主角从100层楼往下跳;  
        皮皮鲁拒绝电影大奖    
        “恢复拍摄!”雄导声嘶力竭地大吼。  
        因女影星罢演抛锚长达两个月的《初升的夕阳》剧组东山再起,在没有那位女影星在场的情况下,继续拍摄以她为主角的电影。  
        皮皮鲁和燕妮头一次和电影拍摄剧组接触,他俩知道了许多内幕。每当他俩回家在进餐时向朋友们转述时,大家都相继呕吐。  
        那位女影星无论如何没想到离了她《初升的夕阳》照样封镜。  
        当女影星在电影院里看到完整的《初升的夕阳》后,她到法院状告《初升的夕阳》剧组侵犯她的肖像权。数百名影星签名声援女影星,他们有兔死狐悲的感觉。  
        法院开庭审理这一引起世人瞩目的案子.双方律师剑拔弩张。  
        贝塔和舒克同朋友们观看了电视转播法庭辩论实况,贝塔看得眉飞色舞。  
        当诉方律师宣读数百名著名演员的声援信时,贝塔决定立即推出导演乐软件。本来他准备将导演乐软件投放市场的时间定在后年,让影星们吃一口最后的晚餐。现在他改变主意了。  
        法院判决女影星败诉,理由是女影星毁约在先。  
        女影星当即在法庭上撒泼,被法警驱逐。  
        皮皮鲁决定使用导演乐软件拍一部全部由电脑演员出演的故事片,写剧本的任务落在舒克头上。  
        舒克使用电脑写作,他有文科和理工科结婚的感受。  
        “看舒克写作真是享受。”克莉斯汀在吃早餐时对歌唱家说。  
        “陪伴配偶进行创造性劳动的确享受。”歌唱家有同感。  
        舒克写的电影的名字是《固体海洋》。  
        “光是这片名就叫座。”鲁西西说。  
        由皮皮鲁担任导演的《固体海洋》正式开拍。片中演员全部由电脑合成,效果和真人一样。天衣无缝。  
        大家在一旁补充情节。  
        贝塔操纵电脑演员按照皮导的旨意行事。  
        “男士角从80层楼往下跳显得分量不够。”鲁西西说。  
        “让他从100层楼跳下来?”贝塔请示皮导。  
        “你再让他从80层楼跳一遍,我仔细看看效果。”皮导想了想,说。  
        男主角又从80层高的楼上往下跳了一回。  
        “是差点儿,让他从100层往下跳!”皮导说。  
        男主角改从100层楼跳楼。  
        “女主角站在火车顶上同反角搏斗不如站在飞机顶上同反角拼打。”歌唱家说。  
        在征得皮导的同意后,贝塔让女主角同反角在飞行中的飞机机身上搏斗。  
        “我怎么觉得男B角长得不太顺眼?”克莉斯汀从女性角度有感而发。  
        “好说,我让男B角的形象往舒克身上靠靠。”贝塔说。  
        克莉斯汀瞪了贝塔一眼。  
        只用了3个星期,《固体海洋》拍摄完毕。  
        歌唱家为《固体海洋》唱的主题歌估计能成为联合国国歌。  
        “皮皮鲁应该带咱们的电影去参加奥斯卡奖评选。”克莉斯汀提议。她在美国肚子饿得特别厉害时就看电视上的奥斯卡颁奖仪式止饿。  
        “寄给他们就行。”皮皮鲁说。  
        第二天,鲁西西将录有《固体海洋》的光盘寄给奥斯卡评委会。  
        评委会秘书长收到光盘后漫不经心地将光盘扔在办公桌上。  
        中午休息时,一个评委随意将《固体海洋》(禁止)影碟机。  
        “这是什么片子?”评委看了5分钟就急问秘书长。  
        “不知道。好像是随便寄来的。”秘书长回答。  
        “了不得的电影!”评委说。  
        秘书长忙看。果然不俗。  
        “都是陌生演员。导演也是新的。”评委说。  
        “我组织专场放映?”秘书长问。  
        “应该。”评委点头。  
        《固体海洋》专场放映开始。  
        放映结束时,全场起立鼓掌长达123分钟。  
        评委会主席知道自己如果再不当即宣布《固体海洋》囊括本届奥斯卡金像奖他就只有辞职一条路了。  
        评委会立即通知《固体海洋》的导演和演员来领奖。  
        皮皮鲁谢绝该奖。  
        奥斯卡评委会怅然若失。          第330集  
        来自布鲁塞尔的令人激动的信息;  
        少校现在是北约实际上的最高领导人;  
        小军人给少校敬礼    
        一天上午,舒克通过电脑的因特网在世界各地游历,他进入了位于布鲁塞尔的北大西洋公约总部的电脑系统。舒克被屏幕上出现的一个信息吸引了,对方试图阻止舒克进入该系统,在经过一番交谈后,舒克突然离开电脑叫皮皮鲁。  
        “皮皮鲁,你快打开电脑!”舒克趴在鲁西西别墅的窗户上喊皮皮鲁。  
        “什么事,”皮皮鲁问。  
        “我在因特网上发现了重要信息!”舒克说。  
        “什么信息?”贝塔插话。  
        “我在北大西洋公约总部遇到一个人,可能是少校。”舒克说。        
        “少校?”皮皮鲁一时没反应过来。  
        “罐头小人少校!”舒克提醒皮皮鲁。  
        皮皮鲁迅速打开自己的电脑。  
        “你是少校吗?”皮皮鲁敲击键盘问对方。  
        “少校?”对方不明白。  
        “我是皮皮鲁。”皮皮鲁继续在屏幕上打字。  
        “皮皮鲁?”对方显然吃惊。  
        “你是少校!”皮皮鲁断定。  
        “皮皮鲁!鲁西西!我是少校!”  
        “你怎么会在国外?30年前我妈妈不是把你送到国内的一家军事院校了吗?”  
        “说来话长。我现在为北约欧洲盟军最高司令工作,可以说是他的助理……”少校将自己离开皮皮鲁家以后的经历告诉给皮皮鲁。  
        少校现在实际上是北约的最高军事领导人。  
        皮皮鲁、鲁西西和歌唱家没想到少校在世界上的军事地位如此之高。  
        以下是30年前少校离开皮皮鲁家至今的简历:  
        30年前,鲁西西的妈妈将少校送到一家军事院校。妈妈在军事院校门口徘徊,她不知道用什么方法让这家军事院校接收少校入学。  
        “您悄悄把我放在学校门口就行,我想办法。”少校对鲁西西的妈妈说。他知道鲁西西的妈妈进不去,学校门口有站岗的士兵。        
        “行吗?”鲁西西的妈妈不放心。  
        “您忘了我是军人,不是孩子。”少校说。  
        鲁西西的妈妈在经过一番犹豫后,将少校放在了那家军事院校门外。  
        “保重,有空回家来看看。”鲁西西的妈妈说。  
        “谢谢您。”少校向鲁西西的妈妈敬礼。  
        少校目送鲁西西的妈妈离去。  
        这时,有一辆军用吉普车在进人军事院校时停在门口接受卫兵的验证,少校趁机爬上吉普车的后轮内侧。  
        吉普车开进学校,少校的身体随着车轮的旋转而旋转,他紧紧攥住车轮上的一块铁片。  
        吉普车停在一座楼前。从车上下来几个军人走进楼里。  
        少校迅速离开吉普车,他不敢贸然进人这栋楼房,他要先找一个能隐蔽身体的地方观察环境。  
        楼旁边是一个小花园。少校看看四周没人,他飞速跑进小花园的草丛里。  
        一顶军帽从天而降,扣在少校身上。  
        紧接看,一只手将军帽连同少校一起抓起来。  
        少校看见了一个满脸稚气的小军人。  
        “这是什么?机器娃娃?”小军人看看手里的少校惊讶。  
        “我是少校!”少校不能容忍军事院校的学员管他叫娃娃。  
        “少校?”小军人仔细一看,少校果然身着军服,“那我得给你敬礼了。”  
        小军人两个后脚跟使劲儿一磕,立正给少校敬礼。  
        少校还礼。  
        “你是什么?”小军人问少校。  
        “军人。”少校回答。  
        “哪国军队?”小军人问。  
        “……”少校语塞,“多国部队。”  
        “逃兵?”小军人逗少校。  
        “我是来军事院校脱产进修的。”少校说,他觉得可以信任这个小军人。  
        “这到底是怎么回事?你不会是敌对国家派来刺探情报的军事间谍吧?”小军人问少校。  
        少校将自己的身世告诉小军人。  
        小军人只有17岁,从小看童话长大的,他立即相信了少校的话。  
        “我刚入学,你就和我一起上学吧,我不会让别人发现你。”小军人说。  
        少校答应了。  
        “你刚才怎么发现我的?”少校问。  
        “我在捉蟋蟀。我喜欢斗蛐蛐。”          第331集  
        少校帮小军人考试;  
        少校乘坐特殊交通工具赴比利时;  
        挂勋章的将军只会出馊主意;  
        神秘的便笺出现在最高司令的办公桌上  
        从此。小军人每天上课都将少校藏在军装上衣兜里,少校和他一起听课。  
        少校是一个天生的军事天才,教师讲授的军事理论如同火柴点燃了他身上的军事才华。  
        小军人上课爱走神,少校经常给他补课。有时少校还要帮助小军人应付考试。  
        3年后,小军人从军事院校毕业了。  
        “你应该去有仗打的地方。你的军事能力组织一次世界大战绰绰有余。”小军人对少校说。  
        “我可不希望打世界大战。”少校说。   
        “你这样的人如果一辈子碰不上战争就太可惜了。”小军人说。  
        “我想去布鲁塞尔。”少校说。  
        “去那儿干什么?”小军人问。  
        “北约的总部在布鲁塞尔。北约经常参与战争。我在那儿会获得施展才能的机会。”少校显然经过深思熟虑。  
        “这主意不错。你怎么去?去了人家会用你吗?”小军人对少校的想法的可行性提出质疑。  
        “你把我装进一个包裹寄到布鲁塞尔北约总部去。”少校说。  
        “我在北约没有熟人,我寄给谁?”  
        “瞎写一个名字。”  
        “北约没有这个人,人家会把包裹退回来的。”  
        “我没等他们退就自己出来了。我准备好军刀。”  
        “你在包裹里大概要呆10天左右,行吗?”  
        “你忘了咱们在学校受过生存训练?”  
        “你绝对是干太事的人。”小军人知道才能和吃苦精神是干大事的必备素质。  
        一切准备就绪,小军人将少校藏在邮包里寄往比利时的布鲁塞尔北约总部。  
        少校在包裹里经受了一次耐力训练。  
        15天后,少校乘坐的邮包抵达北约总部。  
        “查理士当斯?咱们这儿好像没这个人。”北约总部的女秘书拿着装有少校的邮包自言自语。  
        女秘书将邮包放在一边。  
        少校拨出军刀划破邮包,他藏在女秘书办公桌上的电脑后边。  
        女秘书确定本部门没有查理士当斯这个人后,她拿起邮包。她看见了邮包上的裂痕。  
        “怪事,刚才还是好好的!”她往邮包里看。  
        邮包里是吃剩的食物和用塑料袋包装的粪便。  
        “恶作剧。”女秘书给邮包下定义。  
        少校在电脑后边观察了一会儿,他知道了欧洲盟军最高司令是北约的军事首脑,他决定争取在欧洲盟军最高司令身边施展才华的机会。  
        少校利用夜间潜入欧洲盟军最高司令的办公室。  
        第二天,最高司令来办公室上班。少校躲在窗台上的窗帘后边观察最高司令。  
        少校发现最高司令愁眉不展。  
        “通知开会。”最高司令使用电话对秘书下达指示。  
        几个胸前满是勋章的高级将领走进最高司令办公室开会。  
        少校渐渐听明白了,一场局部战争将北约拖了进去,北约的部队在那个国家遇到了难题,进退不得。  
        将领们在会上出谋划策。        
        少校认为全是馊主意。  
        最高司令边听边摇头。  
        会议不欢而散。  
        趁最高司令出去吃午饭的机会,少校给最高司令写了一张便笺。便笺上是解决难题的军事方案。  
        最高司令回到自己的办公室后发现了桌子上的便笺。  
        他的眉头在看便笺的过程中缓解了。  
        “大手笔!军事天才!”最高司令情不自禁拍桌子。  
        最高司令按铃叫秘书。  
        “这张便笺是谁写的?”最高司令问秘书。  
        秘书仔细验查便笺后说不知道。  
        “你坐在我的办公室门口,我的办公桌上多了一张便笺你说不知道?”最高司令盯着秘书的眼睛。  
        “您去吃饭的期间绝对没有任何人进过您的办公室。”秘书立正回答最高司令的质询。  
        最高司令了解自己的秘书,他看出秘书说的是实话。  
        最高司令顾不上查便笺的来源,他按照便笺的提议给部队下达命令。  
        没过午夜12点,北约就打赢了。          第332集  
        最高司令喜当傀儡司令;  
        贝塔反对北约军事干预P国侵略E国;  
        数字炸弹即将诞生;  
        战争可以不流血    
        当少校出现在最高司令的办公桌上作自我介绍时,最高司令连续服用镇静药。  
        “昨天那张便笺是我写给您的。谢谢您给了我证明自己的才能的机会。”少校对最高司令说。  
        “你是什么?”最高司令问。  
        “军人。少校。……”少校向最高司令汇报简历。  
        “竞有这等事……”最高司令惊愕。  
        “这世界上还有很多人类不了解的事。”少较说。  
        最高司令缓缓点头。        
        少校和最高司令聊克劳塞维茨,聊孙子,聊拿破仑。  
        少校的见解令最高司令耳目一新。  
        “我任命你为我的特别助理。”最高司令决定。  
        从此,少校就在最高司令身边为他出谋划策。而最高司令在实际上成为傀儡人物,一切都听少校的。  
        这些年来凡是北约出面的军事行动都是少校的杰作。  
        皮皮鲁和鲁西酉为少校自豪。  
        “真没想到能在电脑网络上邂逅你们。谈谈你们的现状。”少校在电脑上说。  
        皮皮鲁告诉少校他和鲁西西已经找到了约翰和歌唱家,目前还没有博士和艺术家的消息。皮皮鲁还将舒克、贝塔、克莉斯汀和燕妮介绍给少校。  
        “我使用的就是大名鼎鼎的BEITA软件。”少校说,“我无论如何想不到BEITA是老鼠开发的。”  
        “电脑是属于老鼠的。”贝塔说。  
        “这我可不能同意。”北约最高指挥官说。  
        “贝塔的话有道理。”皮皮鲁站在贝塔一边。  
        “皮皮鲁,你结婚了。鲁西西呢?”少校问。  
        “我不想结婚。但有男朋友。”鲁西西告诉少校。  
        “我们也是头一次听说。”皮皮鲁说。        
        “鲁西西应该把男朋友带回来让我们看看。”贝塔说。  
        “好像有情况了,我正在值班。一会儿再聊。”少校说。  
        朋友们抓紧机会审查鲁西西的男友的档案。  
        “为什么不结婚?”贝塔问鲁西西。  
        “我认为结婚证是世界上最不合理的契约。”鲁西西说,“世界上不管任何契约最重要的条款是期限,也就是有效期。没有期限的契约是无效契约。但是结婚契约就没有期限。什么时候结婚契约有期限了,我就结婚。比如结婚证的有效期是5年,没到期谁也不能走。到期双方都愿意继续过就续约,不愿意继续过就分手。如果有了孩子,婚约有效期自孩子出生起自动延长18年。在孩子18岁前解除婚约的夫妻将触犯刑律被处以极刑。”  
        “我估计人类走这一步是早晚的事。”舒克说。  
        少校的信息又出现在电脑屏幕上。  
        “一个小时前,P国侵略了E国,北约决定出兵干涉。我要去部署兵力了。”少校告诉朋友们。  
        “又有地方因为战争要死人了。”燕妮说。  
        “只能用枪炮制止P国侵略E国吗?”贝塔问少校。  
        “当然。克劳塞维茨说……”少校引经据典。  
        “听我说,少校。”贝塔打断少校的话,“尝试一下使用计算机逻辑炸弹阻止P国的侵略怎么样?”  
        “计算机逻辑炸弹?”少校头一次听说。  
        别说少校,就连和贝塔生活在一起的皮皮鲁们也是第一次听说。  
        “也可以叫电脑信息炸弹,或者叫数字炸弹。”可见贝塔是灵机一动,连名字还没统一。  
        “应该使用不流血的方法制止流血。用流血的方法制止流血太残酷。你在总部里制定战争计划,有多少士兵会死?”贝塔对少校说。  
        “我是制定制止侵略战争的计划。”少校更正贝塔。  
        “你听我说,我用鼠标代替枪帮你打一场仗,我越想越有把握,不需一兵一卒一枪一弹,保准让P国总统兵败如山倒。”贝塔极为兴奋。  
        “说说你的打算。要快。”少校对贝塔的话感兴趣了。  
        “现在任何国家的正常运行都离不开电脑,包括P国。而且这些电脑统统进入了因特网。电脑涉及的领域有通讯、银行、股票市场、军队、电力、水利……”贝塔说。  
        “这些我知道。请直接进入高级班。”少校为被侵略的E国人民着急。  
        “我用我的电脑信息炸弹通过因特网‘炸’毁P国的计算机系统。比如,我首先使用电脑病毒侵入P国的电话交换台,造成P国整个电话系统的普遍故障。电话通讯是一个国家的耳朵,没有耳朵的聋子能打仗吗?然后,我用激活信息炸弹摧毁P国控制铁路线的电子道岔,造成铁路运输堵塞。战争离开铁路运输就完了。我还要给P国的股票市场来点儿恶作剧……”贝塔口若悬河。  
        皮皮鲁们都听傻了。          第333集  
        贝塔出任北约PE行动总司令;  
        正义之神病毒诞生;  
        野牛师团的火车撞煤车;  
        母虾师团的飞机迷航    
        “我同意用鼠标进行这场战争。”少校说。“我现在任命你为北约PE行动总司令!”  
        “谁稀罕你那个什么北约南约的破司令。你任命我为银河系总司令吧!”贝塔逗少校。  
        “名不正言不顺。就这么定了。你现在是北约PE行动总司令。开始行动吧,用你的鼠标把P国侵略军赶出E国!”少校下命令。  
        贝塔撇嘴。  
        “你有把握吗?人家E国人民可是生活在火热水深之中呀!”歌唱家怕贝塔误事,她觉得用鼠标打仗是异想天开。        
        “咱们先制造一种威力无比的电脑病毒,用它教训P国总统。”贝塔说完跑到自己的电脑前,埋头制造病毒。  
        同样是制造电脑病毒,有时候是犯罪行为,有时候是正义之举。  
        贝塔是名副其实的电脑天才,20分钟后,名为“正义之神”的电脑病毒问世了。  
        “这是我献给战争的礼物。”贝塔得意非凡。  
        “向P国总统宣战吧!”皮皮鲁催促贝塔。  
        贝塔坐在电脑前,手持鼠标,表情神圣。  
        舒克、克莉斯汀、皮皮鲁、燕妮、鲁西西和歌唱家站在贝塔身后,像送壮士出征。  
        “咱们现在进入P国的计算机系统。”贝塔说。  
        只见贝塔熟练地敲打键盘,电脑屏幕上一行行字自下而上跳动。  
        “进人P国了!”大家异口同声。都是专家。  
        “先打击他的交通系统,让他的侵略军没有后援。”舒克建议。  
        贝塔将载有正义之神病毒的软盘(禁止)电脑。  
        “开火!”皮皮鲁说。  
        贝塔使用鼠标开火。  
        正义之神入侵控制P国交通的计算机,迅速吃掉电脑里的重要数据。        
        P国总统在总统府指挥本国军队侵略E国。  
        “报告总统,已进入E国的部队遇到了E国的顽强抵抗,急需增援!”一位将军向总统报告。  
        “把最精锐的野牛师团派上去!”总统下令。  
        “遵命!”将军跑出房间。  
        P国总统在房间里来回走,心情显然比被侵略国总统紧张。  
        “报告总统,运载野牛师团的火车同一辆运煤的火车相撞了,野牛师团损失惨重!”将军向总统汇报。  
        “混蛋!”总统勃然大怒,“怎么会出这种事?把铁道部长给我就地正法!”  
        “前方在等增援……”将军提醒总统。  
        “派母虾师团乘飞机去增援!”总统派出了他的御林军。  
        载有母虾师团的巨型军用运输机偏离了航线.直飞北约某空军基地,束手就擒。  
        P国总统得到这个信息后一屁股坐在地上。  
        “这是怎么回事?”总统问幕僚们。  
        “全国的交通都乱了,已经有300对火车相撞,239架飞机迷航,其中128架坠毁。更为严重的是,咱们的空军基地的导航计算机系统统统染上了一种不知名的病毒。准备起飞的战斗机不能起飞。已经起飞的战斗机都降落在纽约联合国总部了……”一位幕僚汇报。  
        “公路交通也乱套了,所有红绿灯都失控,要么同时是绿灯,要么同时是红灯,无数汽车相撞。我国的公路交通实际上已经瘫痪,您看看窗外就知道了。”另一位幕僚说。  
        总统站起来看窗外。  
        汽车在路上排起了长龙,司机们叫骂着。  
        “报告总统,进入E国的部队告急!”将军跑到总统身边。  
        “我要同已经攻入E国的部队的指挥官通话。”总统说。  
        “现在该破坏P国的通讯系统了。”皮皮鲁说。  
        贝塔轻轻一点鼠标,P国的通讯系统就被正义之神病毒搅乱了。  
        “报告总统,前方指挥官的电话接通了。”幕僚告诉总统。  
        总统拿起话筒:  
      “我是P国总统,你一定要在今天攻人E国首都!我会想办法给你增援。你一定不能迟于今天拿下E国首都。否则联合国秘书长那小子该…什么?你就是联合国秘书长?”总统脸色变了。        
        “谁接的电话?”总统扔掉话筒后问幕僚。  
        “我…可我接通的确实是前方指挥官呀!”幕僚急了。  
        “送军事法庭!”总统勃然大怒。          第334集  
        银行的计算机将储户的存款重新计算;  
        瑞士银行一样不保险;  
        《皮皮鲁王国》寻找艺术家和博士;  
        探长林的电话改变皮皮鲁的脸色    
        P国的通讯系统被正义之神彻底摧毁了,没有一个人在拨了正确的电话号码后能找到想找的人。  
        P国总统和前方的部队失去了联系,和总统府以外的任何人失去了联系。  
        “用计算机逻辑炸弹炸P国的银行计算机系统。”贝塔给自己下指令。  
        “损了点儿吧?”燕妮说,“老百姓存点儿钱不容易。”  
        “真正受损失的是有钱人。”贝塔坚持。  
        随着贝塔手中的鼠标的移动,P国银行的所有账号张冠李戴重新排列组合。  
        贝塔索性无毒不丈夫,又把P国的股票市场搅了个天翻地覆。  
        水,电、煤气也在劫难逃。  
        P国人民怨声载道纷纷揭竿而起。  
        当P国总统弃国而逃到瑞士提取他的秘密存款时,贝塔已经领先一步用鼠标把他的巨额不义之财转到了联合国难民署的账号上。  
        北约PE行动圆满结束。没有动用一兵一卒。  
        “未来的战争是在电脑上进行的。”舒克预言。  
        “用鼠标和键盘打仗。”皮皮鲁补充。  
        少校对贝塔表示感谢。  
        “我准备说服北约总司令解散北约。以后只要有一台电脑就能制止侵略战争。”少校说。  
        “除非侵略国没有电脑。”舒克说。  
        “设有电脑的国家侵略不丁别人。”少校说。  
        “这倒是。”舒克说。  
        “未来的战场是数字化的。”皮皮鲁说。  
        “真难以想像,咱们在家就能制止战争。”克莉斯汀说。  
        “随着计算机的普及,会有越来越多的事不需要离开家就能办。人们的见面机会将越来越少。”鲁西西说。  
        “电脑能减少汽车的数量。”贝塔说。        
        午餐后,鲁西西说:  
        “5个罐头小人咱们已经找到3个了,还差艺术家和博士。”  
        “在报纸和电视上登寻人启事怎么样?”舒克提议。  
        “这办法不错,我估计艺术家和博士都在国内。”鲁西西赞成。  
        “与其在人家的媒体上登寻人启事,不如自己办个刊物。”燕妮说。  
        “办个《皮皮鲁王国》月刊,艺术家和博士一看刊名就会来找咱们。”歌唱家说。  
        “可行。”鲁西西点头。  
        “我推荐舒克当主编。”贝塔说。  
        “老鼠当刊物的主编?”克莉斯汀担心舒克的种族。  
        “现在的绝大部分主编不如老鼠。”皮皮鲁说。  
        大家一致决定筹办《皮皮鲁王国》月刊,目的之一是寻找艺术家和博士。舒克出任《皮皮鲁王国》主编。鲁西西任杂志社社长。  
        舒克的确是办刊物的天才,《皮皮鲁王国》一问世就受到了少年读者和成年读者的欢迎,创刊号印数达80万份。  
        《皮皮鲁王国》创刊号上刊登了鲁西西给艺术家和博士的信。        
        朋友们天天守在电话机前等艺术家和博士的电话。  
        没有艺术家和博士的信息。  
        “毕竟过去30年了。”歌唱家安慰大家。  
        “我觉得艺术家和博士还活着,可能处境不好。”鲁西西凭直觉。  
        “真想再开五角飞碟去找艺术家和博士。”贝塔发感慨。  
        “开电脑就行了。”皮皮鲁对贝塔说。  
        贝塔伸懒腰。  
        电话铃响了。  
        “八成是博士。”贝塔猜测。  
        皮皮鲁拿起电话昕筒。  
        “我是皮皮鲁。探长林?你怎么了?”皮皮鲁的脸色往不好的方向转变。  
        大家屏住呼吸。  
        “这怎么可能?你还年轻!”皮皮鲁说,“我马上去医院看你。”  
        皮皮鲁放下电话听筒。  
        “探长林怎么了?”燕妮问。  
        “严重脑缺血,刚抢救过来。他说医生认为他时间不多了。”皮皮鲁告诉大家。  
        “探长林也就50岁左右吧?”贝塔说。  
        “我和你去医院看他。”燕妮对皮皮鲁说。        
        燕妮对于探长林使她获得了中国绿卡念念不忘。  
        “我也去。”舒克说。  
        一次舒克从五角飞碟上掉下来,是探长林接住他的。          第335集  
        急救室里的问候;  
        医生给皮皮鲁上课;  
        皮皮鲁驾车闯红灯;  
        抢救探长林的计划    
        皮皮鲁驾车和燕妮、舒克赶到医院。  
        探长林躺在急救室里,胳膊上连着输液的管子。液体一滴一滴地融人探长林的体内,担负着拯救探长林生命的重任。  
        舒克藏在皮皮鲁的衣兜里注视医院的环境,他对医院这座人体修理厂发生了兴趣。他对穿白大褂戴口罩的人有好感。  
        探长林醒着躺在病床上。他看见皮皮鲁和燕妮,冲他们笑了笑。  
        “这么快就来了。”探长林说。  
        “你的身体很好,怎么会这样?”皮皮鲁问。        
        “身体好的人一旦得病就是大病。那天我去看一个作案现场,突然觉得四周的物体摇晃,眼球震颤,然后就昏迷了。昨天才醒。我想假如我要死了最想见谁。想来想去只有你皮皮鲁。我就给你打了电话。”探长林说。  
        “你不会死。”皮皮鲁说。  
        探长林苦笑。  
        “舒克也来看你。”皮皮鲁指指自己的衣兜。  
        舒克从衣兜里探出头来向探长林致意。  
        “谢谢。”探长林对舒克说。  
        皮皮鲁、燕妮和舒克在探长林身边呆了20分钟,hushi下了逐客令。  
        皮皮鲁离开急救室后找到探长林的主治医生。  
        “探长林有危险吗?”皮皮鲁问医生。  
        “有危险。他的情况比较特殊。”医生回答。  
        “怎么回事?”皮皮鲁问。  
        “人有两根椎动脉血管,它们担负着向人的头部输送血液的任务。如果这两根血管的内壁上形成了硬化斑块,血液的流通就会不通畅。更有甚者,这些斑块如果在血液的冲击下脱落,会随着血液进人脑内,阻塞脑血管。探长林的两根椎动脉血管内壁的硬化斑块比较严重,脱落的斑块已经进入他的脑动脉,而且很严重。他随时有生命危险。”医生说。  
        皮皮鲁呆若木(又鸟)。        
        “还有问题吗?”医生问皮皮鲁。  
        皮皮鲁还发呆。  
        燕妮轻轻推皮皮鲁。  
        皮皮鲁回过神来,忙对医生说:  
        “谢谢。”  
        在回家的路上,皮皮鲁由于走神驾车连闯了两个红灯。  
        朋友们都很关心探长林的病情。皮皮鲁回家后神情黯然地向大家转述医生的话。  
        大家沉默。  
        “你们听说过纳米技术吗?”贝塔问大家。  
        皮皮鲁精神为之一振。  
        其他人表示不知道纳米技术为何物。  
        “纳米技术是一种在分子甚至原子规模上生产材料和机器的技术。”贝塔说。  
        皮皮鲁点头,他听说过纳米技术。  
        “在分子和原子上做出机器来?这么小,怎么可能?”燕妮惊讶。  
        “还没有成功。”贝塔说,“据科学家预测,使用纳米技术制造出的细胞大小的机器人可以穿过人体进人人的血管清除血管内壁的脂肪沉积物。”  
        “咱们赶快攻克纳米技术,让细胞机器人清除探长林的血管里的硬化斑块。”克莉斯汀说。  
        “这可不像生产电脑病毒那么容易,最少需要3个月,还得我和皮皮鲁联手。”贝塔说,“3个月,恐怕探长林等不了。”  
        大家再次沉默。  
        “我有个主意,不知道行不行。”燕妮说。  
        “快说。”皮皮鲁说。  
        “咱们不是有微缩粒吗?如果加倍服用微缩粒,没准能把咱们的身体变成细胞那么小。咱们再进入探长林的血管里,清除障碍……”  
        大家被燕妮的大胆想像震住了。  
        “伟大的想法!”舒克说。  
        “怎么进入探长林的身体?”歌唱家问。  
        “通过输液进入血管,再顺着血流进入头部。”贝塔说。  
        “完成任务后怎么出来?”鲁西西问。  
        “事先约好时间地点,用注射器抽出来。”皮皮鲁说。  
        “这太难了。”鲁西西认为怎么出来是最大的难题,危险性极大。  
        “真的变成细胞那么小,喝皮皮鲁口服液还能再变大吗?”克莉斯汀问。  
        这又是一个未知数。  
        “我决定试试这个方案。”皮皮鲁说。          第336集  
        燕妮出任舒克和贝塔的游泳教练;  
        皮皮鲁绘制人体结构图;  
        歌唱家和克莉斯汀变成雕塑;  
        鲁西西调虎离山    
        “谁去?”歌唱家问。  
        “我。”皮皮鲁、舒克和贝塔异口同声。  
        “万一你们回不来了,我们这儿可就没男人了。”克莉斯汀说。  
        “留一个男人你们这儿也没男人了。”贝塔说。  
        女士们在心里认为贝塔的话是真理。  
        大家都明白,这是一次真正的历险,舒克、贝塔和皮皮鲁生还的可能性极小。但所有人都支持这次行动,大家懂得为别人献身是生命的顶级辉煌。  
        “我来分工。”皮皮鲁说,“燕妮立即帮助舒克和贝塔提高游泳水平。在我们进人探长林身体里以        后,鲁西西负责结识那所医院的某位hushi,委托她届时接我们出来。在我们变小后,燕妮和鲁西西去医院将我们放人探长林的输液瓶。在我们进入探长林的身体后,你们再告诉他。”  
        “我们干什么?”歌唱家和克莉斯汀问。  
        “你们俩现在使用电脑精确计算我们出来的时间和地点。”皮皮鲁说。  
        歌唱家和克莉斯汀跑向自己的电脑。  
        舒克和贝塔在燕妮的指导下苦练游泳。  
        皮皮鲁通过电脑查询人体内部结构并绘制地形图。  
        一切准备就绪。  
        “你们必须在进入探长林的血管后的第9个小时返回到他的右胳膊的这根血管的这个位置,注射器在这里将你们接出来。”歌唱家和克莉斯汀给3位勇士看图。  
        “明白了。”舒克将图刻在心里。  
        “开始服微缩粒吧。”皮皮鲁说。  
        分别的时刻到了。大家心里都明白这可能是永别。  
        没有任何人说任何一句话。  
        所有人都会分别。只有这种分别称得上是伟大的分别。  
        在加倍服用微缩粒后,舒克、贝塔和皮皮鲁消失了。他们的身体缩小到和细胞一样小。  
        鲁西西和燕妮携带装有舒克、贝塔和皮皮鲁的小瓶子前往医院。瓶子里有生理盐水。皮皮鲁们泡在生理盐水里。  
        歌唱家和克莉斯汀坐在鲁西西别墅里像两尊塑像,一动不动。  
        当燕妮和鲁西西赶到医院时,探长林睡着了。  
        一位hushi在急救室值班。  
        “我有个问题想请教您。”鲁西西对hushi实施调虎离山。  
        “病人在睡觉,咱们到走廊去说。”hushi中计了。  
        燕妮目送hushi和鲁西西离开急救室,她迅速掏出小瓶,将皮皮鲁、舒克和贝塔连同生理盐水倒入探长林的输液瓶。  
        “祝你们好运。”燕妮的眼眶里充溢着泪水。  
        鲁西西透过门上的玻璃看见燕妮已经将3位勇士送进输液瓶。  
        “您要问什么?”hushi问鲁西西。  
        “他的病有危险吗?”鲁西西没话找话。  
        “有。”hushi回答。  
        “您值班到几点?”鲁西西问。她要物色能将皮皮鲁他们接出来的人。  
        “我还有3个小时下班。”hushi说,“需要我做什么吗?”        
        “谢谢,不需要。”鲁西西说。  
        探长林醒了,他看见燕妮在床边,有点儿吃惊。鲁西西和hushi走进来。探长林和鲁西西点头。  
        “我来看您。”鲁西西说。  
        “谢谢。”探长林看出鲁西西和燕妮有心事。  
        “我想和她们单独聊聊。”探长林对hushi说。  
        “时间别太长,给你们lO分钟。”hushi同意。  
        等hushi离开急救室后,燕妮对探长林说:  
        “皮皮鲁他们帮助你脱离危险。”  
        “谢谢。怎么帮助?”探长林问。  
        “进入你的血管里清除堵塞血管的硬化斑块。”燕妮说。  
        “谁进人我的血管?”探长林问。  
        “皮皮鲁、舒克和贝塔。”鲁西西说。  
        “这怎么可能?”探长林笑了。  
        燕妮将原理告诉探长林。  
        “这是真的?”探长林不信。  
        “皮皮鲁他们已经在输液瓶里了。”燕妮指指输液瓶说。  
        “这太危险了!不行!”探长林拔输液针头。  
        “别拔,他们现在肯定已经进去了。他们需要输液的支持.这样可以加快他们在你的血管里的移动速度。”鲁西西制止探长林拔输液针头。  
        “我们需要您的配合。”燕妮对探长林说。        
        鲁西西发现探长林这个硬汉子的眼睛湿润了。  
        “请您向我们推荐一位能将皮皮鲁他们接出来的hushi。您了解这儿的hushi。”鲁西西说。  
        探长林点头。          第337集  
        舒克和贝塔在针头里等皮皮鲁;  
        汽锤的声音震撼人心;  
        探长林的大脑眉开眼笑;  
        舒克和贝塔失踪    
        身穿潜水服的舒克、贝塔和皮皮鲁进人输液瓶后,迅速潜入瓶底。  
        “快!”皮皮鲁冲贝塔和舒克挥手,他率先游进输液软管。  
        贝塔和舒克紧跟在皮皮鲁身后。  
        往常不起眼的输液软管现在变成了浩瀚的长江。舒克和贝塔的游泳技术不如皮皮鲁,他俩在湍急的水流中尽量保持身体的平衡。  
        在经过输液管的调节器时,舒克和贝塔同皮皮鲁失散了。  
        “皮皮鲁在后边。”贝塔告诉舒克。        
        “咱们只能在针头那儿等他。”舒克说,“别的地方停不住。”  
        舒克和贝塔的身体随着瀑布般的水流急速向下坠落。  
        “抓住针头内壁的凸起部!”贝塔提醒舒克,“进去就不容易找到皮皮鲁了。”  
        贝塔和舒克死命抓住针头内壁的凸起部位,不让波涛汹涌的大河冲走他们。他们必须等皮皮鲁。  
        “我快抓不住了!”贝塔告急。  
        水流太急了。舒克和贝塔的身体在水中像大风中的旗帜。  
        “坚持住!”舒克给贝塔打气,“在这儿建个水利发电站经济效益准好。”  
        “皮皮鲁来了。”贝塔说。  
        皮皮鲁顺流而下。  
        “没赶上你们那滴。”皮皮鲁说,“走吧。”  
        输液调节器是一滴一滴往下放行液体。  
        贝塔和舒克同时松手,3位勇士手拉手通过针头。  
        红河淹没了他们。  
        皮皮鲁、舒克和贝塔进入探长林的身体。  
        血河汹涌澎湃,舒克、贝塔和皮皮鲁拉紧手,随波逐流。  
        “椎动脉血管在哪个方向?”贝塔问皮皮鲁。        
        皮皮鲁用一只手看挂在脖子上的封塑地图。  
        “方向正确。”皮皮鲁说。  
        “注意!前边有急转弯!”贝塔大喊。  
        他们有惊无险地通过了急转弯。  
        “真难以想像,表面看挺平静的人,身体里却是河流纵横奔腾不息。”舒克感叹。  
        “每个生命都是宇宙的缩影。”皮皮鲁说。  
        “听!”贝塔说。  
        他们隐约听见类似汽锤的低沉的撞击声。那声音越来越大。  
        “是心脏的跳动。”皮皮鲁说。  
        “世界上最伟大的声音。”舒克说。  
        “血流的速度好像慢了。”贝塔说。  
        “快到探长林的头部了。”皮皮鲁判断。  
        “血管壁越来越厚。”舒克观察四周。  
        “是硬化斑块导致的。”皮皮鲁说。  
        血管突然变窄了,血也由奔腾的河流变成了静止的湖泊。  
        皮皮鲁看图:  
        “咱们大概已经到探长林的椎动脉了。”  
        “皮皮鲁当心头上!”贝塔猛推皮皮鲁。  
        一块巨大的硬化斑块从天面降,差点儿砸着皮皮鲁。  
        “人的血管实在应该定期清理。”舒克说。        
        刚刚掉下来的那块硬化斑块像浮冰那样在血管里飘浮着,随时有堵塞血液流通的可能。  
        “咱们先把它消灭了,它是隐患。”皮皮鲁说。  
        舒克和贝塔爬上硬块,掏出工具歼灭它。  
        皮皮鲁在下边配合作业。  
        硬块被粉碎了。  
        “咱们现在去探长林的头部。”皮皮鲁说。  
        头部的血管明显变细。  
        “看前边!”贝塔喊。  
        一块硬化斑块几乎堵死了一条血管。  
        皮皮鲁看图。  
        “这是一条极为重要的血管。导致探长林这次生命危机的就是它!”皮皮鲁指着硬块说。  
        舒克、贝塔和皮皮鲁从不同的角度粉碎硬块。  
        这个硬块十分顽强,它在和贝塔他们较劲儿。  
        经过4个小时的鏖战,硬块终于被粉碎了。血管畅通了,探长林的大脑又有了充足的血液的支援。  
        “进来一次不容易,咱们索性把探长林的重要血管都清理一遍。”舒克提议。  
        “行,不过要抓紧时间。现在离约好的返回时间还有3个小时。”皮皮鲁说。  
        “咱们应该分头行动,这样节省时间。”贝塔提议。  
        “两个半小时后,在注射器接咱们出去的地方集合。”皮皮鲁说。  
        大家分头对探长林的血管进行保健性清理。  
        在约定的时间,皮皮鲁赶到了约定的地点。没有舒克和贝塔。  
        皮皮鲁傻眼了。          第338集  
        探长林的血和皮皮鲁口服液融合;  
        医生在急救室威胁hushi;  
        探长林可以参加奥运会;  
        医生的观念即将弃暗投明    
        探长林向鲁西西推荐了一名hushi。  
        鲁西西叫那hushi到急救室来。  
        “什么事?”hushi来到探长林的房间。  
        “我们有一件重要的事想请你帮忙。”鲁西西对她说。  
        hushi看了躺在病床上的探长林一眼。  
        “您是病人的亲属?”hushi问鲁西西。  
        “不是亲属,是朋友。”鲁西西说。  
        “请讲。”hushi说。  
        “我们有3个朋友进到探长林的身体里帮他清除血管里的硬化斑块,我们想请你接他们出来。”鲁西西说。  
        “……”hushi仔细看鲁西西的瞳孔。  
        “行吗?”鲁西西同。  
        “是真的。你用注射器把他们抽出来就行。”探长林对hushi说。  
        “你们……”hushi尽最大努力把精神病3个字吞回去了。  
        “我好了!”探长林突然兴奋,“他们治好了我的病!”  
        探长林自己拔掉输液针头,在床上翻了个跟头。  
        hushi信了。今天早上医生还说探长林活不过本星期。  
        “怎么接你们的朋友?”hushi阀。  
        鲁西西告诉她方法。  
        “抽出来的血怎么处理?他们能自己再变回原样?”hushi又问。  
        “我们带着皮皮鲁口服液,将血液和皮皮鲁口服液融在一起就行。”燕妮说。  
        hushi开始准备接皮皮鲁、舒克和贝塔的针管。  
        “请保密。”探长林要求hushi。  
        “放心吧,我就是把这件事贴在医院太门口也投人相信。”hushi说。  
        “这倒是。”探长林说。  
        “您最好还是躺在床上,有利于皮皮鲁他们在您的体内行走。”鲁西西对探长林说。  
        探长林赶忙躺在床上为朋友创造条件。  
        约定的时间到了。  
        hushi将针头刺进探长林的右胳膊。  
        一管殷红的血离开探长林的身体。  
        “快,给我。”鲁西西从hushi手中接过针管。  
        鲁西西将针管里的血融人皮皮鲁口服液中。  
        大家眼巴巴地注视着脸盆里的皮皮鲁口服液。  
        没有动静。  
        10分钟过去了。  
        20分钟过去了。  
        “他们不在这管血里。”鲁西西用哭腔宣布。  
        燕妮的眼泪夺眶而出。  
        “再抽!快!”探长林命令hushi。  
        hushi跑步去hushi室拿新的针管。  
        第2管血融人皮皮鲁口服液。  
        还是没有皮皮鲁们的踪影。  
        “再抽!”探长林焦急万分。  
        “不能再抽血了!”hushi说。  
        “给我针管,我自己抽!”探长林大吼。  
        “再抽最后一管。”hushi屈服。  
        一位hushi向医生报告说她的同事在猛抽危重病人的血。  
        医生半信半疑地跑到急救室一看,勃然大怒。        
        “住手!”医生大喝一声。  
        正在抽第3管血的hushi吓得直哆嗦。  
        “是我让她抽的。”探长林对医生说。  
        “快把血给我!”鲁西西从hushi手中拿过针管。  
        燕妮将针管里的血融人脸盆中的皮皮鲁口服液中。  
        “你们在干什么?”医生质问鲁西西和燕妮。  
        大家顾不上理医生,全神贯注看脸盆。  
        “这是急救室,你们都出去!”医生火了。  
        没人理他。  
        “还没有……”鲁西西哽咽。  
        “你们找什么?”医生愤怒。  
        “再抽!”探长林吼叫。  
        hushi看医生。  
        “你敢再抽他的血?”医生威胁hushi。  
        “有人在我的血管里,你不让抽血你就犯了故意杀人罪!”探长林威胁医生。  
        “有人在你的血管里?”医生估计探长林精神失常了。  
        鲁西西不得不将事情的原委告诉医生。  
        “拿我当小孩儿?”医生嘲笑鲁西西。  
        “你怎么解释我的病已经好了?”探长林问医生。  
        医生这才注意到探长林的健康状况。已经被他判死刑的病人生龙括虎地站在他面前。          第339集  
        探长林碰上了一个伟大的院长;  
        100万元的支票:  
        皮皮鲁在太平洋;  
        苍蝇能治艾滋病    
        医生为探长林作了体检,结论是可以参加奥运会竞赛。  
        “你们说的是真的?”医生的观念动摇了。  
        “千真万确。”房间里的人异口同声,包括hushi。  
        医生瞪了hushi一眼。  
        “快给我抽血!”探长林催促。  
        “如果再没有呢?”医生问探长林。  
        “一直到抽光为止!”探长林斩钉截铁。  
        医生大为感动。  
        “可以给他全身换血。”hushi对医生说。  
        “对,全身换血!”探长林赞成。        
        “全身换血的费用相当高。”医生看探长林。  
        “我们出钱。”鲁西西说。她将自己的名片递给医生。  
        舒克贝塔公司是地球上最富的企业。  
        “我马上去向院长汇报。”医生跑出去。  
        医生是院长的高徒,他只用了5分钟就让院长相信了这件事。  
        “立即在手术室给探长林实施全身换血,抢救皮皮鲁们!”院长下令。  
        鲁西西得知院长的决定后当即打电话让舒克贝塔公司财物经理送来一张100万元的转账支票。  
        探长林被推进手术室。  
        鲁西西和燕妮被特许穿上白大褂进人手术室接皮皮鲁们。  
        院长亲自督战,开始为探长林大换血。  
        皮皮鲁、舒克和贝塔在探长林体内迷宫似的血管里历尽千难万险,谁也找不着谁。  
        他们知道自己的生命到了结束的时候,他们在不同的地方共同死而无怨。  
        贝塔在探长林的左腿里迷了路。  
        皮皮鲁在探长林的肝部徘徊。  
        舒克错把探长林的左胳膊当成了右胳膊。  
        “真没想到死在人的血管里。”精疲力尽的贝塔飘浮在血流上自言自语。  
        他想歌唱家,想朋友们,想电脑,还想五角飞碟。  
        “马上就什么都不存在了。”贝塔被红色包围着。  
        皮皮鲁已经没力气游泳了,他觉得自己处于无边无际的太平洋中,孤立无援。  
        他回想自己的生命历程中那些印象最深刻的片断。  
        “真正有成就的人不会感到忙。”皮皮鲁在临终前不知为什么冒出这么一句话。  
        舒克体质较弱,他已经处于半昏迷状态。朦胧中他发誓下辈子当医生,开一家医院,专门清扫人体里的血管的医院,就叫纳米医院。  
        “舒利……”舒克喃喃自语。  
        第1个获救的是贝塔。  
        当院长见到血液和皮皮鲁口服液融合后变出一只老鼠时,他呆呆地问鲁西西:  
        “你们派老鼠进入人体内为人治病?”  
        “是的。”鲁西西点头。  
        “以毒攻毒?”医生问鲁西西。  
        “真没想到用这种话表扬英雄。”贝塔失望。  
        “会说话的老鼠!”院长和医生的眼镜同时捧碎了。        
        第2个获救的是舒克。  
        医生和院长立即对舒克实施抢救。  
        舒克的生命脱离危险。  
        皮皮鲁最后离开探长林。  
        探长林得知皮皮鲁们统统获救后从手术台上一跃而起:  
        “我开车送你们回家。”  
        皮皮鲁决定送给院长和在场的医生hushi每人一台傻瓜尖端电脑。  
        院长和医生hushi目送欢天喜地离去的皮皮鲁们,一个个竞赛似的反思自己的观念太保守。  
        “我准备从苍蝇身上提取防治艾滋病的药物。”院长向部下宣布。  
        “我相信用《孙子兵法》同病魔打仗准比《本草纲目》强。”医生恍然大悟。  
        “我今天才明白,原来卖弄学识的人最无知。”给探长林抽血的hushi茅塞顿开。  
        后来,这家医院产生了10名诺贝尔医学奖得主,而日是10连冠。          第340集  
        舒克欲组建血管环卫局;  
        贝塔担任攻美小组负责人;  
        克莉斯汀的爷爷越狱成功;  
        燕妮替人类赎罪    
        克莉斯汀和歌唱家听了舒克、贝塔和皮皮鲁的经历后分别和夫君紧紧拥抱。  
        “我想办医院。”舒克向朋友们宣布。  
        “办医院?”克莉斯汀发现丈夫回来有变化。  
        “这次历险使舒克和医院有了感情。”皮皮鲁说。  
        “我进人探长林的身体才发现人的血管非常需要经常清扫,我想办一家使用纳米技术清扫人的血管的医院。就叫纳米医院。”舒克显然已经深思熟虑。  
        “咱们能攻克纳米技术吗?”歌唱家问。  
        “当然能!你别小看自己,如果评职称,咱们的底线都是科学院院长。”贝塔说。        
        “咱们齐心协力拿下纳米技术,造出细胞机器人。”皮皮鲁赞成舒克的想法,他知道人的血管急需清理。  
        电话铃响了。  
        鲁西西接电话。  
        “是探长林打来的,他提醒咱们注意,说发现外国一些电脑公司雇佣了中央情报局刺探舒克贝塔公司的经济情报。他已经对咱们公司采取了保护措施。”鲁西西放下话筒后向大家通报。  
        “在人类社会,出人头地和众矢之的永远是同义词。”贝塔说。  
        “不管他,咱们办咱们的医院。”舒克一门心思行医。  
        从第二天起,朋友们开始攻纳米技术。贝塔担任总负责,他给每位都分了工。大家各自在自己的电脑前忙碌着。  
        鲁西西不断为大家购置所需的设备。  
        两个月后,舒克贝塔们使用纳米技术制造出了细胞大小的机器人。  
        这种机器人进人人体后不需要再出来,它们能在人的血管里不体息地连续工作10年,能量耗尽后随着人体的新陈代谢排出人体。  
        “现在只剩做试验了。没有经过试验的新技术不能用于临床。”鲁西西说。        
        “在我身上试验。”克莉斯汀说。  
        “为什么?”歌唱家问。  
        “我的祖先是人类专门用来做试验的老鼠。我爷爷诞生在美国缅因州阿卡迪亚国家公园的杰克逊老鼠研究培育实验室,那个实验室是专门培育繁殖供人类做试验用的老鼠的。”克莉斯汀说。  
        “你说什么?美国有这种地方?”贝塔怒目圆睁。  
        “当然。那座实验室大极了,有36座大楼,光工作人员就有600多人,他们喂养着65万只老鼠,每年能赚数千万美元。”克莉斯汀说。  
        “你是说,你爷爷是人类培育出来的?”贝塔问。  
        “对。那儿的老鼠分两部分。一部分是正常老鼠。另一部分是转基因老鼠。”克莉斯汀说。  
        “什么叫转基因老鼠?”燕妮问。  
        “转基因老鼠就是由工作人员给健康的老鼠接种一种基因或一种病毒或一种细菌,人为使它们得一种病,比如感冒、糖尿病、癌症或心脏病等,然后拿它们做试验。”克莉斯汀说。  
        大家发现舒克和贝塔的脸色极其难看。  
        “人干吗总是拿老鼠做试验?”歌唱家说。  
        “听我爷爷说,这是因为从基因角度讲,老鼠和人类有90%的相似之处。”克莉斯汀说。  
        “在地球上,谁像人类谁倒霉!”贝塔咬牙切齿。  
        舒克想起他去美国找约翰临行前和贝塔在贝塔的故居处建起的一座医院里看见的有先天性心脏病的老鼠,那老鼠大概就是缅因州的产物。  
        “你爷爷怎么没被人类做试验?”贝塔问克莉斯汀。  
        “我爷爷逃出了杰克逊实验室。”克莉斯汀说。  
        “你们美国老鼠对于杰克逊实验室袖手旁观?”舒克问夫人。  
        “怎么可能?听爷爷说,美国的老鼠在1947年和1989年两次纵火焚烧了杰克逊实验室。”克莉斯汀说。  
        “那怎么现在它还存在?”燕妮问。  
        “烧毁了很快就又恢复了,因为太赚钱了。杰克逊实验室每天能卖出去两万五千只老鼠,每只70美元左右。杰克逊实验室有患1700种不同的病的老鼠。工作人员一般是将各种老鼠胚胎放在零下196摄氏度的冷冻室内冷冻。有了买主后,他们再从冷冻室取出胚胎植人母鼠子宫内繁殖。”克莉斯汀说。  
        鲁西西、燕妮和歌唱家不由自主地打了个哆嗦。  
        皮皮鲁叹了口气。  
        “你们该同意在我身上做纳米机器人试验了吧?”克莉斯汀说。她认为这是自己的优势。  
        “不行!”所有人异口同声。  
        “在我身上做试验。”燕妮说。  
        “在我身上做。”歌唱家也要替人类赎罪。        
        皮皮鲁和鲁西西也加入竞争。  
        最后只得通过抽签决定。  
        燕妮有幸中签。  
        一群细胞机器人通过注射器进入燕妮的血管。  
        燕妮感觉良好。  
        经过仪器鉴定,细胞机器人将燕妮的血管打扫得干干净净,就像婴儿的血管一样干净。          第341集  
        鲁西西的伟大提议;  
        胖是没有艾滋病的标志;  
        克莉斯汀喜欢单眼皮;  
        探长林给老鼠办身份证    
        随着纳米技术的成功,成立纳米医院提上了日程。  
        鲁西西负责选择院址。  
        大家一致推举舒克出任纳米医院院长。  
        “老鼠当院长工作不方便吧?”克莉斯汀说。  
        “病人知道纳米医院的院长是老鼠,谁敢来?”贝塔说。  
        “卫生局大概也不会给咱们发执照。”舒克说。  
        鲁西西眼镜一亮:  
        “那天克莉斯汀说,人的基因和老鼠的基因有90%是相同的,咱们干脆修改舒克身上那与人类不同的10%的基因,把舒克变成人,堂堂正正当院长。”  
        没人说话,只听见大家的心脏猛跳。  
        “伟大的想法!”皮皮鲁说。  
        “鲁西西不得了。”贝塔夸奖鲁西西。  
        “当然还要看舒克愿意不愿意改变种族。”鲁西西说。  
        “那还能不愿意!”贝塔说。  
        舒克和贝塔这么多年和皮皮鲁在一起时虽然没有任何自卑心理,但他们从内心深处羡慕人类。  
        能变成人在地球上生活.对于舒克和贝塔来说是梦寐以求的事情。  
        和攻克纳米技术相比,修改老鼠的基因对于皮皮鲁们是易如反掌的事。  
        “把贝塔和克莉斯汀的基因也一起修改了。”舒克说。  
        “我们没意见。”贝塔代表克莉斯汀表态。  
        克莉斯汀激动不已。  
        “凭空冒出3个人,到哪儿去办身份证?”歌唱家问。  
        “当然是探长林!”所有人一起说。  
        大家依靠电脑日以继夜地修改舒克、贝塔和克莉斯汀身上的与人类不同的那10%的基因。  
        皮皮鲁在电脑屏幕上绘制贝塔、舒克和克莉斯汀日后的相貌。  
        “我的身体一定要胖。”贝塔提要求。  
        “瘦好。要不干吗有那么多减肥药。”歌唱家说。  
        “胖是没有艾滋病的标志。”贝塔说,“我预言减肥药没有市场。”  
        “我要单眼皮。”克莉斯汀说,“我觉得单眼皮美。我还要当中国人,鼻子别那么高。”  
        “别把我的皮肤弄得这么自,一个大男人。我要黑皮肤,再糙点儿。行了。”舒克看着屏幕说,  
        皮皮鲁根据朋友们的要求在电脑屏幕上修改他们的面目。  
        在一个霞光满天的傍晚,舒克、克莉斯汀和贝塔变成了人。  
        服用皮皮鲁口服液后,他们和普通人的身材一样了。  
        歌唱家也喝了皮皮鲁口服液。  
        7个人欢聚一堂。  
        “科学技术是改变世界的力量。”歌唱家感慨。  
        皮皮鲁给探长林打电话。  
        “有事求你。”皮皮鲁说。  
        “说吧。”  
        “给4个人办身份证。”  
        “刚满18岁?”  
        “都40多岁了。”        
        “40多岁还没身份证?”  
        “对。”  
        “为什么?偷渡者?”  
        “不是。刚当人。”  
        “40多岁刚当人?”  
        “对。”  
        “……”  
        “不相信我?”  
        “你准备照片吧,我个小时后到你家拿照片。”  
        舒克、贝塔、歌唱家和克莉斯汀去照像馆照快像。  
        探长林来到皮皮鲁家,4个陌生人出现在他面前。  
        其中只有歌唱家让探长林觉得面熟。  
        “他们从哪儿来的?”探长林小声问皮皮鲁。  
        “其中两个进过你的血管。”皮皮鲁说。  
        探长林目瞪口呆。  
        “你还想干什么?”探长林对皮皮鲁佩服得五体投地。  
        “办医院。”皮皮鲁回答。  
        “开医院?”探长林吃惊。  
        “专门给人清扫血管的医院。”皮皮鲁说。  
        “医院的治安我包了。”探长林说。  
        “大材小用。”鲁西西插话。  
        第二天舒克等4人就获得了身份证。          第342集  
        舒克给供电局注射葡萄糖;  
        纳米医院是一块肥肉;  
        报纸使舒克院长不安;  
        把自己的钱包送给抢劫犯    
        纳米医院开诊当天就人满为患。  
        探长林安排了10名警察维持秩序还是有人为挂不上号动手打了架。  
        舒克任院长。贝塔任副院长。克莉斯汀任门诊部主任。歌唱家任院务处主任。  
        第一天开诊下班后,舒克在院长办公室召集纳米医院常委会。  
        与会者都是嫡系。  
        “我有当年办航空公司的感觉。”贝塔坐在沙发上说。  
        “太棒了,今天咱们救了多少人!”克莉斯汀兴奋。  
        “当人的感觉真不错。”舒克回味无穷。  
        “能治好别人的病是一种享受。”歌唱家脸上挂着灿烂的笑容。  
        “贝塔负责生产细胞机器人。要注意安全,探长林不是说外国情报机构注意咱们了吗?这项技术目前还要保密。”舒克对贝塔说。  
        “放心吧,院长。咱们合作又不是一天两天了。”贝塔洋洋得意,他还在回味以人的身份第一次上街的情景。  
        舒克们变人后,鲁西西为每个人购置了豪华轿车。那天,他们分乘4辆轿车上街兜风,好不痛快。  
        贝塔由于没有驾驶执照被交通警察扣了。  
        “我连飞碟都会开。”贝塔向交通警察表明自己的驾驶技术。  
        “我开过火箭。”交通警察嘲笑贝塔。  
        皮皮鲁使用手机向探长林求援。  
        探长林坐着警车给每个人送来一本除了不能开飞机什么都能开包括链式拖拉机的驾驶执照。  
        “贝塔,开会哪!”舒克看出贝塔走神了。  
        贝塔从交通警察身边回到纳米医院的院长办公室。  
        舒克院长继续极享受地主持会议。  
        纳米医院门庭若市,来这里要求注射细胞机器        
        人的健康人比病人还多。舒克给纳米医院定的宗旨是“先病人后健康人。先老年人后年轻人。先女士后男士。”  
        新闻媒介以近乎狂热的程度报道纳米医院的奇迹。舒克频频接受记者采访。  
        几乎人类的所有成员都想给自己注射细胞机器人。贝塔的生产能力有限。细胞机器人供不应求。  
        舒克严格执行纳米医院的宗旨。  
        一天上午,纳米医院停电。  
        “医院不是不停电吗?”舒克着急。  
        “大概是供电局治咱们。”歌唱家说。  
        “为什么?”舒克问。  
        “供电局的人昨天来医院要求注射细胞机器人,都是五大三粗的小伙子,被我拒绝了。”歌唱家说。  
        “你做得对。”舒克表扬院务处主任。  
        在连续停电3天后,舒克不得不带着细胞机器人到供电局送货上门。  
        电霸们接受注射后给纳米医院合上了闸。  
        舒克积了一口恶气。  
        “院长别生气,你带到供电局的那些瓶细胞机器人都是假的,葡萄糖。”贝塔小声告诉舒克。  
        舒克院长苦笑。  
        供水局来了。  
        还有市容局,卫生局,税务局,环卫局,街道局……  
        舒克疲于应付,敢怒不敢言。  
        “咱们是一块肥肉,被一群狼围着。”舒克在第二次院常委会上欲哭无泪。  
        “咱们医院的垃圾都堆成山了。”院务处主任诉苦。  
        “快给环卫局送葡萄糖去。”贝塔说。  
        “给他们点儿细胞机器人倒没什么,就是感觉上受不了,像当亡国奴,像把钱包送给抢劫犯。”舒克院长眉头紧皱。  
        “我看当人不如当老鼠。”贝塔说。  
        电话铃响了。  
        克莉斯汀接电话。  
        “专利事务所的电话,问咱们的细胞机器人为什么不申请专利?他说他们愿意给咱们当代理。收费优惠。”克莉斯汀转述。  
        “院长看看这份报纸。”歌唱家递给舒克一张报纸。  
        报上说,自从纳米医院开诊后,所有治疗脑血管和心血管疾病的医院都门可罗雀,它们最近连工资都发不出了。  
        “我觉得这些医院的院长饶不了咱们,你最好先做准备。”贝塔副院长提醒院长。  
        被贝塔说中了。          第343集  
        秃顶官员向舒克索要医学院文凭;  
        纳米医院门口的大红印章;  
        电脑人推迟问世;  
        贝塔的新构思    
        皮皮鲁在得知纳米医院遭遇的种种刁难后安慰舒克院长。  
        “人类一贯这样,你应该学会适应。”皮皮鲁对舒克说。  
        “以人的形式活在地球上其实是不幸。”舒克悲观。  
        “我估计更大的风浪还在后边。你要坚强。”皮皮鲁给舒克鼓劲。  
        舒克不置可否地点头。  
        一天,歌唱家焦急地跑进院长办公室。  
        “卫生局来了一帮人,点名说找你。我看来者不善。”歌唱家通知院长。  
        “叫他们进来。”舒克平静地说。  
        卫生局的官员同舒克见面。  
        “有17家医院的院长联名控告你和纳米医院。”一个秃顶官员对舒克说。  
        “控告我什么?”舒克问。  
        “控告你和纳米医院非法行医。”秃顶官员说。  
        “我们医院的行医执照是贵局签发的。”舒克指指自己身后墙上的执照。  
        “人家指控你骗取行医执照,说在你的医院里没有任何一个医务人员有医学院的毕业文凭。你知道,拥有医学院的文凭是行医的基本条件。”秃顶官员说。  
        舒克哑口无言。  
        “如果你能向我们出示哪怕一张医学院毕业文凭,我们就算你是合法行医。”秃顶官员说。  
        舒克没有。  
        “那就对不起了,我们只得依法办事,查封你的医院。”秃顶官员示意手下摘墙上的行医执照。  
        “我们确实治好了很多垂危病人。”舒克据理力争。  
        “人家还告你贩卖假药,欺世盗名,牟取暴利。至于有的病人碰巧病好了,人家说是心理作用。”秃顶官员将一张纸递给舒克。        
        舒克看见纸上是密密麻麻的字,标题是《揭开纳米医院的面纱》。17家医院的院长在文末签名。  
        舒克将纸撕得粉碎。  
        “藐视政府官员,罚款1万。”秃顶官员宣布,“纳米医院从现在起被查封,吊销行医执照。”  
        纳米医院的大门被贴上了盖有鲜红大印的×形封条。  
        媒介又一边倒地群起诋毁纳米医院,各种访谈节目统统将枪口的准星瞄准纳米医院,说纳米医院是本世纪最大的医疗骗局。  
        不管舒克走到哪儿,都有人指他的脊梁骨。  
        “真没想到办纳米医院的结局是这样!”克莉斯汀气愤填膺。  
        “始料未及。”鲁西西一边摇头一边说。  
        “人类完了。”燕妮消极。  
        “我本来还有一个大构思,现在本人宣布放弃,不帮人类了。”贝塔说。  
        “什么构思?”鲁西西问。  
        “贝塔想淘汰电脑。”歌唱家知道夫君的计划。  
        “淘汰电脑?”皮皮鲁感兴趣,  “说说。”  
        “在探长林的椎动脉血管里,我产生了一个想法。”贝塔说,“我认为完全可以将微型电脑芯片植入人的颈后与人的神经系统连接,该芯片可将人的思维直接转换为电脑语言,并通过芯片上的接收和发送装置进行人脑与电脑的信息交流。每个植入了电脑芯片的人都将心想事成,成为电脑人。人脑将直接进人全球信息高速公路。”  
        “你快发明这种芯片!”鲁西西激动。  
        “坚决不干了。”贝塔心灰意懒地说。  
        “贝塔的这项发明的成功之日,就是他的生命的结束之时。”舒克预言。  
        “医院将抢了电脑经销商的饭碗。”燕妮说。  
        “失业的上千万电脑营销人员不会放过贝塔的。”歌唱家说。  
        “我不想死。”贝塔说。  
        “我同意贝塔放弃这个构思。”皮皮鲁一脸的遗憾。  
        “我还有一个构思。”贝塔说。  
        大家看贝塔。  
        “我想把地球上的老鼠都变成人。”贝塔说。          第344集  
        贝塔和皮皮鲁冲突;  
        发生在皮皮鲁家里的全民公决;  
        贝塔放弃计划;  
        飞机上的笑话    
        “反对!”皮皮鲁毫不犹豫地表态。  
        “为什么?”贝塔问。  
        “地球上的人本来就很多了,再把比人类数量还多的老鼠变成人,地球承受不了。”皮皮鲁说。  
        “人类太欺负老鼠,像杰克逊实验室那样摧残老鼠的机构肯定不少。应该公平竞争。”贝塔说。  
        “你这个计划和爱因斯坦家的老鼠的计划如出一辙。”皮皮鲁情绪有些激动。  
        “我的计划和它们的计划在本质上不一样。它们是通过改变人类来使人类和老鼠平等,我是通过改变老鼠使人类和老鼠平等。”贝塔也激动。        
        “不管怎么说,我坚决反对!'’皮皮鲁提高声音。  
        “我坚决要把地球上的老鼠都变成人!”贝塔不示弱。  
        皮皮鲁和贝塔头一次争吵。  
        燕妮和歌唱家分别劝皮皮鲁和贝塔。  
        “表决吧!”舒克提议。  
        没人反对舒克的提议,但大家心里都不舒服,毕竟通过表决来统一认识在他们之间是头一次。  
        表决结果,4比3,皮皮鲁败北。  
        贝塔、歌唱家、燕妮和鲁西西对贝塔的方案投了赞成票。  
        皮皮鲁、舒克和克莉斯汀对贝塔的计划投了反对票。  
        贝塔尽管赢了,他还是猛瞪舒克。  
        皮皮鲁为地球担忧。  
        “你为什么反对把老鼠变成人?”贝塔质问舒克。  
        “50亿人已经把地球折磨成这副模样,你忘了咱们的医院是怎么被查封的了?你再把人类的数量翻几番,地球还活不活了?你能保证老鼠的素质比人类的强?不管怎么说,人类还没把我放在微波炉里。”舒克说。  
        贝塔沉思。  
        大家沉默。  
        “不变了!”贝塔一拍桌子。        
        “谢谢你,贝塔!”皮皮鲁拥抱贝塔,“人类是有很多可恶的地方!”  
        “不过你得答应我一个条件。”贝塔对皮皮鲁说。  
        “说吧。”皮皮鲁说。  
        “你给我办护照和签证,我要去美国缅因州捣毁那家老鼠实验室!”贝塔说。  
        “我也去!”舒克说。  
        “还有我。”克莉斯汀说。  
        “我也去!”歌唱家说。  
        “我答应。”皮皮鲁同意。  
        探长林责无旁贷地为舒克、贝塔、克莉斯汀和歌唱家办了出国护照,并在美国领事馆为他们获得了两个星期的旅游签证。  
        皮皮鲁、鲁西西和燕妮在一家大饭店为舒克、贝塔、克莉斯汀和歌唱家饯行。  
        “到了美国要当心。”皮皮鲁举杯叮嘱贝塔,他知道到别人的国家去捣毁一个有影响的实验室是凶多吉少的事。  
        “放心。”贝塔今天被特许喝酒。  
        “去看约翰吗?”鲁西西问舒克。  
        “看情况吧,估计时间比较紧,签证只有两个星期。”舒克边吃边说。  
        不知为什么,这顿饭的气氛有点儿悲凉。  
        第一天早晨,皮皮鲁驾车送舒克、贝塔、克莉斯汀和歌唱家去国际机场。  
        皮皮鲁、鲁西西和燕妮将舒克、贝塔、歌唱家和克莉斯汀送到不允许再送的地方。  
        告别。  
        通过安全检查后,贝塔一行在候机室等候登机。  
        “送给你一本书。”坐在舒克身边的一位长者递给舒克一本书。  
        舒克刚想问为什么,他看见了书名,没问。  
        “谢谢。”舒克向长者致谢。  
        这是一本佛教的书。  
        扩音器里的小姐告诉旅客可以登机了。  
        克莉斯汀是头一次坐飞机,感觉很新鲜,她坚决要求坐在紧挨窗户的座位上。  
        贝塔和舒克想起了皮皮鲁去德国找歌唱家那次乘飞机的经历,忍俊不禁。  
        “你们笑什么?”克莉斯汀问。  
        舒克给她讲皮皮鲁坐飞机的艳遇。  
        克莉斯汀想给舒克买振动防蚊盒。          第345集  
        空中小姐在万米高空拿旅客当填鸭;  
        总统套间的豪华浴池;  
        餐桌上谋划拯救美国老鼠的方案    
        舒克在十几个小时的飞行途中一直津津有味地看那本书。  
        “坐过五角飞碟再坐这玩艺儿,整个一个历史回顾展。”贝塔叹气。  
        漂亮的空中小姐在旅途中干的惟一的事就是往旅客肚子里塞各种东西。  
        飞机在美国缅因州着陆。  
        濒临大海的缅因州绿荫环抱,风景宜人。  
        舒克一行下飞机后乘坐出租车下榻毗邻阿卡迪亚国家公园的一座五星级饭店。  
        他们租了两个总统套间。舒克和克莉斯汀住一套,贝塔和歌唱家住另一套。        
        “人真会享受。”克莉斯汀环顾金碧辉煌的总统套间说。  
        “这房间也有咱们老鼠的功劳。”舒克一边脱外套一边说。  
        “怎么讲?”克莉斯汀问。  
        “建造这座饭店的人准有不少吃过先拿老鼠做试验的药。”舒克说。  
        “人类取得的成果其实都是地球上所有生命和人类共同取得的。到享受时,就轮不上别的生命了。”克莉斯汀恍然大悟。  
        电话铃响了。  
        贝塔从隔壁打来的。  
        “咱们先洗个澡,然后下去吃饭。”舒克挂上电话。  
        总统套问的冲浪按摩浴池令舒克和克莉斯汀心旷神怡。  
        “真希望地球上的所有老鼠都能享受总统套间。”贝塔在隔壁的浴池里对妻子发感慨。  
        “人类还不能都享受总统套间呢。”歌唱家提醒贝塔。  
        贝塔不吭气了。  
        沐浴后,4个人下楼到餐厅用餐。  
        在餐桌旁,他们开始策划行动细节。  
        “明天咱们扮装成买主去杰克逊实验室实地勘察。”舒克说。  
        “他们让买主参观吗?”贝塔问克莉斯汀。  
        “那得问我爷爷。”克莉斯汀幽一默。她爷爷早死了。  
        “如果不让参观,咱们回来再想办法。”舒克说。  
        “有五角飞碟就好了。”贝塔遗憾。  
        当天晚上,他们尽情享受五星级饭店的总统套间。          第346集  
        惨不忍睹的精美画册;  
        克莉斯汀看见了几万个爷爷;  
        杰克逊实验室的眼泪;  
        女卫生间里没有人    
        杰克逊老鼠实验室占地足有38公顷,规模庞大。  
        “这么大!”贝塔惊讶。  
        “全世界供试验用的老鼠有一大半儿是这里培育的。”克莉斯汀说。  
        “作孽。”歌唱家说。  
        “咱们进去。”舒克说。  
        一位工作人员接待他们。  
        “我们准备购买一些老鼠。”舒克说。  
        “做试验用?”工作人员问。  
        “是的。”贝塔说。  
        “这是目录,本实验室得什么病的老鼠都有。” 工作人员递给舒克一本印刷得极为精美的画册。  
        舒克翻阅。  
        画册里是患各种疾病的老鼠。惨不忍睹。  
        “我们可以参观贵实验室吗?”舒克问。  
        “有的地方可以。”工作人员说。  
        舒克一行在该工作人员的带领下参观杰克逊实验室允许参观的部位。  
        上千个笼子里的老鼠同胞注视着舒克一行。  
        贝塔恨不得立即释放它们。  
        克莉斯汀想起了爷爷,她止不住眼泪。  
        “小姐怎么了?”工作人员惊讶地看泪流满面的克莉斯汀。  
        “她可能不舒服。”舒克给妻子打掩护。  
        克莉斯汀看见笼子里的老鼠都像她爷爷。  
        她哭出了声。  
        工作人员诧异。  
        “我带她去卫生间。”歌唱家说。  
        “卫生间在那边。”工作人员指方向。  
        舒克贝塔和工作人员等她们。  
        10分钟过去了。  
        舒克看表。  
        20分钟过去了。  
        贝塔预感不妙。  
        “我去看看。”舒克说。        
        “我们在这儿等你。”贝塔看了那工作人员一眼。  
        舒克到卫生问门口叫克莉斯汀。  
        没人答应。  
        舒克又叫歌唱家。  
        还是没人答应。          第347集  
        中情局小头目喜出望外;  
        浴室里的毛发被化验;  
        特工拍照实验室里的老鼠;  
        小头目不让开玩笑    
        中央情报局受前电脑公司老板的雇佣派员到中国刺探舒克贝塔公司的经济情报。由于探长林和同事们的防范,情报局的特工没能得手。  
        正当他们回国后沮丧不已时,情报人员通知他们纳米医院前院长一行4人抵达美国。  
        情报局知道纳米医院和舒克贝塔公司的关系极为密切。他们喜出望外。  
        “通过电脑查这4个人的资料!”负责刺探舒克贝塔公司的中情局小头目命令手下。  
        “地球上根本没这4个人。”手下查完电脑向上司汇报。        
        “机器人?”小头目惊讶。  
        “不像。”同事看着电脑上的贝塔一行下飞机的照片分析。  
        “密切跟踪。”小头目部署,“随时向我汇报。”  
        “目标在××饭店落脚。”  
        “住的是总统套间。”  
        “晚上在饭店餐厅用餐。”  
        “……”  
        “……”  
        小头目不断得到部下传递的信息。  
        “你说什么,他们对杰克逊实验室表现了特别的兴趣?他们大老远来美国就对老鼠感兴趣?”小头目和监视舒克一行的特工通话。  
        “我们趁他们离开房间时化验了他们遗留在浴室的毛发,其中3个人的毛发含有老鼠的成分。”特工汇报。  
        “人的毛发里有老鼠的成分?”小头目再吃惊。  
        “我们怎么办?”特工请示。  
        “继续监视,我马上赶到。”小头目亲自出马。  
        小头目乘坐直升机赶到缅因州时,舒克一行已经进人了杰克逊老鼠实验室。  
        “他们果然对老鼠感兴趣!”小头目百思不解。  
        “我们还发现笼子里的老鼠看他们时的眼神不正常。”负责监视舒克们的特工向小头目报告。        
        “怎么不正常?”小头目感到越发奇了。  
        “它们普遍有同胞相聚的眼神。”特工说。  
        “胡说!”小头目不允许部下在工作时间开玩笑。  
        “这是真的。”特工给上司看刚拍的照片。  
        照片上的老鼠的眼睛里全是显而易见的亲情。  
        小头目目瞪口呆。  
        “舒克贝塔公司是用老鼠发明的傻瓜尖端电脑?”一名特工忽发奇想。  
        “胡说八道!”中情局小头目枪毙了部下的正确判断。          第348集  
        克莉斯汀擅自解放老鼠;  
        歌唱家众人拾柴火焰高;  
        贝塔用鼠语骂美国警察;  
        警察傻笑   
        克莉斯汀进入卫生问后再也控制不住情绪了,她放声大哭。  
        歌唱家不劝克莉斯汀,她陪哭。她知道克莉斯汀心里的难受程度,她知道当一个生命到了自己的祖先被别的生命使用非正常手法诞生的地方是什么心情。  
        “我要去救它们!”克莉斯汀的情绪失控了,“不能再等了!”  
        歌唱家手足无措。  
        克莉斯汀跑出卫生间,她狂奔选一间大实验室,该实验室里有上百个笼子,笼子里关着成千上万只老鼠。        
        克莉斯汀发疯般地连续打开笼子的小门。  
        笼子里的老鼠们争先恐后地越狱,它们还发出惊喜的尖叫声。  
        歌唱家也加人解放老鼠的行动。  
        转眼间,实验室的地上全是老鼠,汹涌澎湃。  
        一个房间解放完了,克莉斯汀和歌唱家又闯进另一个房问接着解放。  
        直到中隋局的特工奉命逮捕了克莉斯汀和歌唱家。  
        “以破坏罪拘捕这两个女士。那两个男的暂时别动。”小头目下令。  
        当舒克和贝塔得知克莉斯汀和歌唱家出事时,她们已经被警察带走了。  
        “为什么抓她们?”贝塔大怒。  
        “她们放走了我们实验室的几万只老鼠!”实验室副主任大吼。  
        舒克和贝塔相对无言。  
        现场的警察让舒克和贝塔回住处听信儿。  
        “我现在就要见她们!”贝塔根本不把美国警察放在眼里,他还记得约翰折腾纽约警察的场面。  
        “这是美国!”警察提醒贝塔。  
        “咱们先回饭店。”舒克劝贝塔。  
        舒克用老鼠话骂警察。  
        警察听了傻笑。          第349集  
        法医发现人身上流老鼠血;  
        彼得富氏飞缅因州;  
        大汉给律师一个黑皮包;  
        美国法庭上的老鼠脏话    
        中情局小头目吩咐法医为克莉斯汀和歌唱家做体检。  
        体检报告交到小头目手中,小头目揉揉眼睛。  
        “其中一个人的血是老鼠血?!”小头目尽管已有预感,还是又大惊小怪了一回。  
        人的身体里流的是老鼠血!小头目断定这是人类最新科技成果。小头目决定以破坏罪通过判刑将克莉斯汀和歌唱家留在美国让美国的科学家研究她们,小头目则从中收费。  
        小头目还决定对舒克和贝塔实施放虎归山,派人跟踪他们回国,使用最先进的间谍工具截取最有价值的经济情报。  
        舒克和贝塔在饭店的总统套间里心急如焚。  
        “向皮皮鲁求援?”舒克问贝塔。  
        贝塔摇头:  
        “别再麻烦他了。是咱们要来的。”  
        “克莉斯汀的情绪准是失控了。”舒克叹气。  
        “寻到了自己的根,非激动不可,这么残酷的地方。”贝塔表示理解。  
        “歌唱家大概觉得作为人类对不住老鼠,就替全人类将功赎罪了。”舒克分析。  
        贝塔同意舒克的判断。  
        总统套间成了牢房,舒克和贝塔在里边度日如年。  
        第3天上午,法院通知舒克和贝塔后天开庭审理克莉斯汀和歌唱家案。  
        “咱们必须为她们找最好的辩护律师。”舒克说。  
        “找约翰和罗勃特帮忙吧,他们熟悉美国。”贝塔说。  
        “罗勃特应该出狱了。”舒克说。  
        “咱们可以先给彼得富氏打电话问问。”贝塔说。  
        舒克通过查号台查到了监狱长彼得富氏的电话号码。  
        舒克给彼得富氏打电话。  
        “你好,我是彼得富氏。”        
        “你好,我是舒克。”  
        “舒克!你在哪儿?中国?”  
        “我在美国。”  
        “美国?你开五角飞碟来美国了?”  
        “五角飞碟已经被皮皮鲁销毁了。我和贝塔坐国际航班来的。”  
        “你们在哪儿?”  
        “缅困州。”  
        “到缅因州干什么?”  
        “我们现在需要你的帮助。对了,罗勃特和约翰出狱了吗?”  
        “我们从中国回来后,我就给罗勃特办了假释。什么事需要我帮助?”  
        “我们需要最好的律师。”  
        “律师?有官司?”  
        舒克尽量简要地将经过告诉彼得富氏。  
        “克莉斯汀会被警察拘捕?”彼得富氏不信美国警察逮捕老鼠。  
        “你现在看看我和贝塔就信了。”舒克说。  
        “我明天下午带律师去缅困州。”彼得富氏说。  
        当天晚上,舒克和贝塔彻夜未眠。  
        舒克通宵看在机场得到的那本书。贝塔站在落地窗前骂黑暗。  
        彼得富氏根本不相信给他开门的两个小伙子是舒克和贝塔。  
        舒克好不容易说服了他。  
        “律师在大厅里,我没敢让他和我一起上来,他不知道是为老鼠当律师。他是最好的律师。”彼得富氏说。  
        律师和舒克、贝塔见面后去拘留所会见克莉斯汀和歌唱家。  
        律师在离开拘留所返回饭店的途中,一辆汽车停在他身边。从车上下来一个拿黑皮包的大汉。  
        “我是中央情报局的。”大汉给律师看证件,“你不可以真的为那两个中国女人辩护。这是给你的报酬。”  
        大汉将黑皮包塞给律师。  
        汽车扬长而去。  
        律师看黑皮包里边。  
        满满一包百元钞。  
        在法庭上,舒克和贝塔见到了坐在被告席上的歌唱家和克莉斯汀。  
        舒克冲克莉斯汀和歌唱家打了个v手势。  
        检察官宣读起诉书。  
        法庭辩论开始。  
        陪审团退庭合议。  
        法官宣读判决书:        
        以破坏罪判克莉斯汀和歌唱家3年徒刑。  
        舒克和贝塔异口同声用老鼠话在法庭上破口大骂。          第350集  
        贝塔在高空看佛书;  
        皮皮鲁擦拭直升机;  
        6个月的艰苦飞行;  
        一贫如洗的世界首富    
        克莉斯汀和歌唱家被法警押送监狱服刑。  
        “真对不起,我找的律师不行。”彼得富氏向舒克和贝塔道歉。  
        “你已经尽了力。这不是一般的案子,有背景。”舒克不是傻子,他已经感觉到事情的复杂。  
        “我认识美国所有监狱的监狱长,我会托朋友关照她们的。”彼得富氏说。  
        “谢谢。”贝塔说。  
        缅因州当局的官员通过电话提醒舒克和贝塔的签证快到期了。  
        “美国对咱们下逐客令了。”舒克对贝塔说。        
        “咱们走吗?”贝塔不甘心,他想营救克莉斯汀和歌唱家。  
        “咱们这副人模样,不走往哪儿躲?”舒克说。  
        “当人没劲!”贝塔说。  
        舒克和贝塔在离开美国前到克莉斯汀和歌唱家服刑的监狱探望她们。  
        令他们吃惊的是该监狱说没这两个犯人。  
        舒克和贝塔又找到法院,法院咬定克莉斯汀和歌唱家就在那座监狱服刑。  
        彼得富氏给他的所有监狱长朋友打了查询电话。没有克莉斯汀和歌唱家的信息。  
        “有问题。我劝你们快离开这里,我继续找她们,有消息后打电话告诉你们。”彼得富氏对舒克和贝塔说。  
        舒克和贝塔乘飞机离开美国。他们感觉到他们的每一个举动都处于监视之中。  
        在飞机上,舒克向贝塔推荐那本书。贝塔从美国看到中国。  
        “克莉斯汀和歌唱家呢?”看见贝塔和舒克走进家门,鲁西西问。  
        皮皮鲁和燕妮发现舒克和贝塔的神情不对。  
        “出事了?”燕妮问舒克。  
        舒克点头。  
        “她们怎么了?”皮皮鲁问。        
        舒克叙述。  
        皮皮鲁摔碎了手中的茶杯。  
        “恢复五角飞碟!马上出发!”皮皮鲁发怒。  
        “快!”燕妮催促丈夫。  
        贝塔看舒克,舒克看贝塔。  
        “不用了。”舒克说。  
        “为什么?”鲁西西问。  
        “万事随缘吧。”贝塔说。  
        皮皮鲁、燕妮和鲁西西像不认识似的看了贝塔看舒克。  
        “不用了,皮皮鲁。”舒克说,“我和贝塔还变回老鼠。我们不想当人了。”  
        “是什么就是什么,人为改变自己只会走向灾难。”贝塔说。  
        皮皮鲁、鲁西西和燕妮面面相觑。  
        当天晚上,舒克和贝塔修改自己的基因,恢复了鼠身。  
        “我们想住在直升机里。”贝塔对皮皮鲁说。  
        皮皮鲁从书柜里拿出直升机,擦拭干净。  
        深夜,舒克和贝塔躺在直升机里睡不着,他们不约而同地回忆自己的生命历程。  
        “其实,心情平静最重要。”舒克打破平静。  
        “对。心情平静最难做到。”贝塔说。  
        “我实在累了。”舒克说,  “我觉得这么活挺傻。”  
        “咱们走。”贝塔说。  
        “去哪儿?”  
        “五台山。”  
        “出家?”  
        “不是出家,是在五台山的庙宇里落户,感受一下新的生活,练练宠辱不惊的功夫。”  
        贝塔没白看那本书。  
        “怎么去?”舒克问。  
        “当然是开直升机。”贝塔说。  
        “很远。”舒克说。  
        “咱们开不到?”贝塔说。  
        “当然能到。”舒克说。  
        “你准备东西,我给皮皮鲁、鲁西西和燕妮写封信。咱们就不辞而别吧。”贝塔说。  
        “咱们什么也不带,我去电脑上打一幅字带上就行了。”舒克说完悄悄离开直升机。  
        贝塔给皮皮鲁写信。他请皮皮鲁、鲁西西和燕妮原谅他们的不辞而别。拜托皮皮鲁定期去美国看望克莉斯汀和歌唱家。拜托探长林设法同美国交涉要回克莉斯汀和歌唱家。  
        清晨4点,直升机悄悄飞离皮皮鲁家。  
        舒克和贝塔经过6个月的艰苦飞行,抵达五台山。        
        他们定居在一座庙宇的老鼠洞中,心情宁静地活着。  
        洞里几乎一贫如洗,只有那幅舒克带来的用电脑书写的条幅。条幅上有10个字:  
        本来无一物,何处惹尘埃。  
        舒克和贝塔觉得自己是地球上最富有的生命。 

\backmatter
      
\end{document}
