% 舒克和贝塔全传
% 舒克和贝塔全传.tex

\documentclass[a4paper,12pt,UTF8,twoside]{ctexbook}

% 设置纸张信息。
\RequirePackage[a4paper]{geometry}
\geometry{
	%textwidth=138mm,
	%textheight=215mm,
	%left=27mm,
	%right=27mm,
	%top=25.4mm, 
	%bottom=25.4mm,
	%headheight=2.17cm,
	%headsep=4mm,
	%footskip=12mm,
	%heightrounded,
	inner=1in,
	outer=1.25in
}

% 设置字体,并解决显示难检字问题。
\xeCJKsetup{AutoFallBack=true}
\setCJKmainfont{SimSun}[BoldFont=SimHei, ItalicFont=KaiTi, FallBack=SimSun-ExtB]

% 目录 chapter 级别加点(.)。
\usepackage{titletoc}
\titlecontents{chapter}[0pt]{\vspace{3mm}\bf\addvspace{2pt}\filright}{\contentspush{\thecontentslabel\hspace{0.8em}}}{}{\titlerule*[8pt]{.}\contentspage}

% 设置 part 和 chapter 标题格式。
\ctexset{
	chapter/name={},
	chapter/number={}
}

% 设置古文原文格式。
\newenvironment{yuanwen}{\bfseries\zihao{4}}

% 设置署名格式。
\newenvironment{shuming}{\hfill\bfseries\zihao{4}}

\title{\heiti\zihao{0} 舒克和贝塔全传}
\author{郑渊洁}
\date{}

\begin{document}
	
\maketitle
\tableofcontents
	
\frontmatter
\chapter{前言}

舒克和贝塔是童话大王郑渊洁笔下最著名的童话形象,舒克和贝塔伴随了几代人的成长。郑渊洁花费十几年时间写作长达一百多万字的《舒克和贝塔全传》。大多数读者只看过《舒克和贝塔》前100集,而《舒克和贝塔》总共366集。县念迭起,扣人心弦,故事精彩,想象奇特是这部巨著的特色。《舒克和贝塔全传》是成年人回到童年、未成年人享受童年的灵丹妙药,是馈赠孩子和朋友的礼品精神大餐。

\mainmatter
	
\chapter{第1集}

舒克生在一个名声不好的家庭里;舒克吃了有生以来最香的一顿饭;舒克驾驶直升机离开了家

“舒克,你都大了,可以自己出去找东西吃了。”一天,妈妈对小老鼠舒克说。
   
“真的吗?”舒克高兴了。
  
舒克是一只生活在中国的小老鼠,他从生下来以后就一直憋在洞里,从来没有出去玩过。

“今大晚上,我带你出去,先认认路,以后你就可以自己去了。”妈妈一边说,一边磨牙。

舒克也学着妈妈的样子磨牙。他爱吃好东西。每次妈妈给他带回来好吃的.他都吃个没够。

夜里,舒克跟在妈妈身后出了洞。

“好大的屋子!”舒克惊叫道。

“小声点儿!”妈妈告诫舒克。

“为什么不能大声说话?”舒克问。

“对于咱们老鼠来说,在外边小声说话安全。”妈妈说。

妈妈告诉舒克,那是衣柜,那是写字台,那是电脑,那是床。舒克把眼睛都看累了,他觉得这个世界很有意思。

“这个柜子对咱们最有用,里面全是好吃的,叫冰箱。”妈妈把舒克带到一个柜子跟前。“可它的门总是关着,得找机会。现在,咱们到餐桌上去,那里有一盘花牛米。”

一听有花生米,舒克的口水快流出来了,他跟着妈妈爬上了餐桌,果然,桌上有一盘香喷喷的花生米。

舒克和妈妈大吃起来。

“小偷!这么小就学偷东西!”黑暗里传来一个声音,吓了舒克一跳。

“偷吃人家的东西,真不要脸!”又是一声。

舒克借着月光一看,窗台上有一个鸟笼子,笼子里有两只鹦鹉,一蓝一绿,刚才的话,就是他俩说的。

听人家管他叫“小偷”,舒克脸红了。他看看妈妈,妈妈就像没听见一样,继续吃着。

“你吃饱了?”妈妈看见舒克不吃了,问。

“妈妈,咱们这叫偷吗?”舒克小声问。

“傻孩子,什么偷不偷的,咱们老鼠世世代代就是这样活下来的。别理他们,贩卖正直的人最不正直。快吃吧。”

舒克又吃了两颗花生米,他觉得,今天的花生米不如以往的香。

第二天夜里,舒克自己出来找吃的了。他又来到写字台上,可那盘花生米不见了。舒克正准备下去,蓝鹦鹉喊起来:“小偷又来了!”

“真是的,有什么样的妈妈就有什么样的儿子。”绿鹦鹉也跟着说。

“胡说!我妈妈说,我们不是小偷!”舒克要争这口气,他大声对鹦鹉们说。

“这些吃的东西是你劳动得来的么?”蓝鹦鹉问舒克。

“这……”舒克说不出话来了。

“不是你劳动换来的,就是偷!”绿鹦鹉耸耸鼻子。

“哼,你妈妈不但偷,还净搞破坏,衣柜里的衣服就是被她咬坏的!”蓝鹦鹉说。

舒克愣住了。

“你出去打听打听,谁不知道你们老鼠是坏蛋!你敢大白天出去吗?人家都说,老鼠过街,人人喊打!”绿鹦鹉说。

舒克没想到自己家的名声这么坏,他委屈极了,自己干吗生下来就是只老鼠呢!舒克哭了。

舒克不愿意当小偷,他决定离开家,到外面去闯闯,通过劳动换取食物。

舒克看中了床头柜上那架米黄色的电动直升机,它有一副红色的塑料螺旋桨。舒克曾经从洞口里看见直升机在屋里飞过,很酷。

这天清晨,窗户大开着,直升机静静地停在床头柜上。舒克悄悄地钻进了飞机,这架直升机的机舱挺大,除了驾驶员坐的地方以外,后面还有两排皮椅子。

舒克想起了“老鼠过街人人喊打”的话,他决定化装一下,让人家看不出他是老鼠。

舒克忍着疼,把胡子都拔下来。他穿上飞行服.将尾巴缠在腰里。舒克看见床头柜上有一筒牙膏,他跑过去打开盖,挤出许多牙膏涂在脸上。

一切都准备好了,舒克坐进驾驶舱,戴上飞行帽。

“现在我已经不是老鼠了,是飞行员舒克。”舒克兴奋地想。他打开了启动器,红色的螺旋桨转了起来,它越转越快,不一会儿,直升机就离开了床头柜。

舒克驾驶着直升机在屋里盘旋了一圈,他还故意擦着鸟笼飞过去,当他看见鹦鹉们认不出他时,得意极了。

小老鼠舒克,不,飞行员舒克驾驶着直升机,从开着的窗户飞出了屋子。

外面是碧绿的田野,起伏的丘陵,还有宽阔的河流和盛开的花丛……舒克驾驶直升机尽情地在天上飞,他很兴奋。

舒克觉得肚子有点儿饿,他决定去找点儿吃的。舒克操纵直升机下降高度,他把头探出飞机,注意观察地面。

“救命!救命呀!”

舒克忽然听到地面上传来呼救声。

舒克一看,是一只蚂蚁掉进水洼里,他正在拼命挣扎。

舒克急忙将直升机开到了水洼上空,然后操纵飞机垂直下降。

“我来救你!”舒克把头探出飞机,大声喊。他将飞机悬停在空中,离水面只有两寸远。可飞机上没有绳子,蚂蚁怎么上来呢?

眼看小蚂蚁不行了,舒克忽然想起了自己的尾巴。他急忙解开裤子,把尾巴从腰上解下来,打开飞机舱门,将尾巴伸向水面。

“你抓住绳子爬上来,快!”舒克大声喊。

小蚂蚁抓住舒克的尾巴,爬上了直升机。

舒克关上舱门,操纵直升机拉起了高度。

“谢谢你,谢谢你!”小蚂蚁一边擦身上的水,一边感激地说。

活这么大,舒克头一次听到别人谢他。

“你叫什么名字?”小蚂蚁问。

“我叫飞行员舒克。”舒克说。

“这架直升机真漂亮。”小蚂蚁打量着机舱说。他忽然看见了舒克的尾巴,“哟,你的绳子真像老鼠的尾巴。”

“啊,是吗?”舒克一惊,这才想起忘了将尾巴藏起来,他一边把尾巴往裤子里塞,一边说:“直升机上的绳子都是这样的,有弹性。”

小蚂蚁仔细地打量舒克,笑了。

舒克担心小蚂蚁认出他是老鼠来.看样子没有,要不,小蚂蚁肯定不会再对他笑了。

“你家在哪儿?我送你回家。”舒克说。

小蚂蚁把头贴在玻璃上,给舒克指路:“就在那棵大树后面。对,再往前飞,绕过那个土坡。看见了吗?就是那个洞口。”

这是舒克第一次操纵直升机着陆,他聚精会神。

直升机平稳地降落在蚂蚁洞旁边。

舒克给小蚂蚁打开舱门,小蚂蚁跳了下去。

舒克赶紧把尾巴缠在腰里。不一会儿,小蚂蚁领着一大群蚂蚁走到飞机旁边。

“舒克,这是我们蚁王,她来谢你了。”小蚂蚁对舒克说。

一听是蚁王,舒克赶紧从飞机上下来。

“谢谢你救了我的孩子,我能为你做点儿什么吗?”蚁王和蔼地问舒克。

“不用谢,”舒克心里美滋滋的,“我,我有点儿…饿。”

“快去拿最高档的食物。”蚁王命令。

很快,几百只蚂蚁抬着许多米饭粒、面包渣放到舒克面前。舒克大吃起来,真怪,他觉得,这些东西比花生米还香。

“你们以后有什么事,就来找我帮忙,我经常在这一带飞。”舒克吃饱以后对蚁王说。

“我们也欢迎你经常来!”蚁王笑眯眯地回答。

“你可经常来呀!”小蚂蚁眼圈红了。

舒克心里也挺酸,可他不敢哭,要是眼泪把牙膏冲掉了,人家认出他是老鼠来,谁还理他!

舒克钻进直升机,冲大家招招手,起飞了。

\chapter{第2集}

飞机多转了一个弯;舒克为自己的名声苦恼;舒克永远是大家的朋友

舒克驾驶直升机来到一片花丛上空,他看见许多蜜蜂在采蜜。

“今天的蜜真多,都运不回去了,怎么办呢?”一只蜜蜂对同伴说。

“就是,怎么办呢?”大家都很着急。

舒克把头探出窗外:“我帮你们运吧?” 

蜜蜂们吓了一跳,抬头一看,是一架米黄色的大直升机悬在空中。

“你是谁?”

“我是飞行员舒克。”

蜜蜂们一看有飞机帮他们空运蜜,高兴了。

直升机在花丛中着陆,蜜蜂们把蜂蜜运进机舱。

“你自己送去吧,我们还得采蜜。我们家就在小河对岸那棵最高的树上。”一只金翅膀蜜蜂对舒克说。

一只白翅膀蜜蜂不放心,小声说:  “咱们又不认识他,要是他……”

“你别把人都想得那么坏,我看不会。”

舒克看见金翅膀蜜蜂这么相信他,很感动,说:“你们放心,我一定送到。”

直升机起飞了。机舱里充满了蜂蜜的香昧儿。小时候妈妈给舒克吃过蜂蜜,很香。舒克回头看了几眼蜂蜜,咽了一下口水,心想人家这么相信自己,自己可不能偷吃。

舒克看见了小河,他驾驶飞机转弯向小河对面飞去。

飞机转弯的时候,盆里的蜜洒出来一点儿。

舒克用手指蘸着一尝,味道不错。原来,这是没有加工过的花粉蜜。舒克想,这不算偷吃,是它自己洒出来的。这么想着,他又操纵飞机在小河上面做了一个更急的转弯,这回洒出来的蜜更多了。

“这倒不错,既没偷,又能解馋。”舒克满意地想。

舒克把蜂蜜安全地送到了蜜蜂的家。他来来回回帮助蜜蜂空运了十几次,蜜蜂们都很感谢他,收工时,蜜蜂给舒克搬来了一大盆蜂蜜。

“我说他是好人吧!”金翅膀对白翅膀说。

舒克想起自己在飞机上吃人家的蜜,有点儿后悔。

“我不要蜂蜜了。”舒克说。

“那不行,一定得留下。”蜜蜂们不容分说,将蜂蜜搬进了机舱。

“你以后想吃蜂蜜就来,咱们是朋友了,我们对朋友一点儿不吝惜。可上次有只老鼠来偷蜜,我们就狠狠地教训了他一顿。”金翅膀说。

舒克真怕蜜蜂看出他是老鼠,他向蜜蜂们告别后,急忙起飞了。

舒克开着直升机在天上转悠,他知道,只要人家认不出他是老鼠,都会对他友好。可一旦人家知道他是老鼠,一定不会理他了。想到这儿,舒克把飞行服整了整,再摸摸腰里的尾巴缠得牢不牢,又将飞行帽戴好。

“砰!”地面传来一声枪响。

舒克往下一看,一个小男孩拿汽枪将一只麻雀从树上打下来。麻雀的翅膀被打伤了,在地上一蹦一蹦地跳着,小男孩从远处追过去。

舒克驾驶直升机来了一个俯冲,落在小麻雀身旁。他打开舱门,喊:“快!快上来!”

小麻雀来不及细想,上了直升机。

好险!小男孩刚跑到跟前,米黄色的直升机腾空而起,小男孩愣在那里。

“你很勇敢!”小麻雀望着舒克说。

“伤得重吗?”

“翅膀伤了,特疼。”

“他干吗打你?”

“我也不知道,他总拿枪打我们。妈妈就是让他们打死的。”

“人比老鼠还坏吧?”舒克问。

“老鼠?老鼠最坏。”

“可老鼠没用枪打死别人呀!”舒克提醒小麻雀。

“老鼠名声不好。”

名声,就是这个名声!害得舒克整天穿着飞行服,戴着飞行帽,还把尾巴缠在腰里,热死了,也不敢脱。舒克恨死“名声”这个东西了。

“你怎么了?”小麻雀看到舒克不吭气了,“对了,我还忘了问你是谁呢?”

“飞行员舒克。”舒克不大情愿地回答。他不明白,自己救了他为什么不能理直气壮地说真名字——小老鼠舒克!

“你真好,谢谢你,飞行员舒克!”

这次听到人家谢他,舒克心里不是滋味儿。他想听“谢谢你,小老鼠舒克”。

不过,舒克一会儿就把不愉快的事忘了,他请小麻雀吃蜜,小麻雀说不喜欢吃蜜,喜欢吃虫子。舒克答应帮他抓。

天快黑了,舒克将直升机停在一座楼房的房顶上。他让小麻雀在机舱中休息,自己跑出去给他抓虫子。

舒克从来没抓过虫子,他费了九牛二虎之力,总算抓到了几条。

小麻雀看到虫子,高兴地吃了起来。舒克笑了。

第二天,舒克把小麻雀送回家。

舒克经常为大家办事。渐渐地,谁都知道有位飞行员舒克开着米黄色的直升机,爱帮助别人。

这天,经小麻雀提议,大家宴请飞行员舒克。主办宴会的是蚂蚁国王、蜜蜂皇后,还有许许多多受过舒克帮助的朋友都来了。

宴会很丰盛,有很多好吃的食物。大家坐好了在等舒克。

舒克开着直升机去参加宴会。这些日子,他通过自己的劳动,交了许多朋友。舒克看见下面有一片花丛,他操纵飞机在花丛中着陆,舒克要给朋友们带点鲜花。

舒克摘了一朵红花。

“这朵送给小麻雀,”他想。

舒克又摘了一朵黄花。

“这朵送给金翅膀小蜜蜂。”

忽然,舒克身后刮来一阵急风,他感到一阵颤栗,浑身发软,他还没明白是怎么回事,就觉得自己的肩膀已经被牢牢地抓住了。

“我当什么飞行员舒克,原来是只老鼠!”舒克身后传来一阵大笑。

舒克回头一看,天哪,是只小花猫!小时候,舒克就听妈妈说过,猫是老鼠最大的敌人,祖祖辈辈是冤家。

“你以为化装了,就能逃过我的眼睛?走,我要让小麻雀他们看看你的真面目,然后再处决你。”

一听说要带他去见小麻雀他们,舒克急了,他哀求道:“我求求你,现在就把我处死吧,千万别让他们知道我是老鼠。”

舒克宁可死了,也要保个好名声。

“想得倒美!走!”小花猫不理会舒克的苦苦哀求,用手轻轻一提,就把舒克拎走了。

“这下完了。”舒克闭上眼睛,想像着一会儿大家骂他的场面。

“你们的飞行员是只老鼠,看看吧!”小花猫把舒克往地上一扔,大声宣布。

舒克站起来,他不敢睁眼睛。

大家都愣了。

“我现在就去处决他!”小花猫像审判长一样宣布,他说完又抓起舒克。

“住手!”小麻雀飞到小花猫跟前,“你干吗要处决他?”

“因为他是老鼠!”

“可他没干过坏事呀!”

“老鼠都是坏蛋!”

“不对,舒克就不是坏蛋!”

“对,舒克不是坏蛋!”大家一起嚷道。

“他是一只老鼠呀!”小花猫急了。

“老鼠不老鼠我们不管,他是我们的朋友舒克!我们的朋友舒克!”小蚂蚁大声说。

“对,他是我们的朋友舒克,不许你伤害他!”金翅膀蜜蜂飞起来,只要小花猫敢动舒克一根毫毛,他就要蜇他。

舒克再也忍不住了,眼泪刷地流了下来,他不怕把脸上的牙膏冲掉了。

小麻雀过去给他擦干眼泪。

“舒克,来,宴会开始。”小麻雀宣布。

舒克笑了,他把飞行帽摘掉,坐在餐桌正中央。

“他们疯了!和老鼠一起会餐!”小花猫讨了个没趣,怏快地走了,他实在想不通。

从那以后,舒克再也不怕别人知道他是老鼠了,他每天驾驶着米黄色的直升机,为朋友们做事。

\chapter{第3集} 

贝塔用布口袋装香味儿;咪丽不让贝塔吃饭;贝塔学会驾驶坦克

贝塔也是一只小老鼠,从他降生的那天开始,就有一个可怕的影子始终跟踪着他,那影子是咪丽。

咪丽是一只猫。

咪丽害得贝塔两天没吃东西了,这天晚上,一股香味儿从洞外飘进来,贝塔忙拿出他的小布口袋,将香味儿装进去。这是贝塔想出的办法,每当香味儿飘进来时,就用口袋把它装起来,留着以后饿了时闻。

可今天贝塔实在太饿了,越闻香味儿就越想吃东西,他决定出去冒一次险。

贝塔先把头探出洞外,屋里静悄悄的。

“咪丽大概出去玩了吧?”他小心翼翼地出了洞。

冰柜旁边有只碗,那里边总足有好吃的,什么鱼呀、肉呀…贝塔就是饿死也不敢过去吃,那是咪丽的饭碗,主人每大往这个碗里放好吃的。

贝塔想在地上找点儿剩饭。就在这时,他忽然听见有响动。贝塔探头一看,好家伙,眯丽正盯着他呢!

他赶忙窜回洞去。吓得直喘粗气。

“小偷!你敢出来吗?”咪丽在洞口吓唬贝塔。

贝塔连答话都不敢。就这样,贝塔被咪丽一连堵了3天!他已经饿得全身无力,手脚发软了。

咪丽呢,每天放意当着贝塔大吃大喝。主人这几天似乎特别优待她。

看着咪丽大吃一气,贝塔咽口水。

“干吗她每天可以大模大样吃这么多东西?而我吃一点儿就是偷。要是主人每天也给我一点儿东西吃,哪怕比咪丽少得多,我就不会偷了,主人真是个怪东西。”贝塔想。

贝塔不想饿死,他得想办法活下去。

贝塔惟一的乐趣,就是每天晚上看电视。屋里的电视机正好对着贝塔的洞口,他不用出去就可以看电视。

这天,一部电视片吸引了贝塔,屏幕上的一群坦克在进攻,把敌人打得落花流水。坦克什么都不怕,连高大的墙都被它撞塌了。

“坦克这么厉害!”贝塔想起床底下有一辆绿色的电动坦克,他的眼睛闪出了奇异的光。

趁眯丽出去喝水的空儿,贝塔钻出洞,跑到床底下找到了那辆绿色的电动坦克。

贝塔学着电视上坦克驾驶员的样子,打开坦克上的舱盖,钻进坦克里边。

坦克里很宽敞,装几个贝塔都不成问题。贝塔关紧炮塔上的舱盖,从里面把插销插上,又使劲儿推了推,直到他确信咪丽从外面肯定打不开时,才松了一口气。

贝塔仔细打量坦克内部,他对这里的一切都很陌生。贝塔坐在驾驶员的座位上,他发现前面有一个小镜子,贝塔把脑袋凑过去,居然能看到外面!

贝塔想起来了,电视上说过,这叫潜望镜。

潜望镜下面有一排漂亮的电钮。贝塔试着按了一下红色的电钮,坦克启动了,飞快地向前冲去,贝塔又按了一下黄色电钮,坦克向后退去。

贝塔开心极了,他把所有电钮都按了一遍。有的能操纵炮塔转圈,有的能加大前进速度。有的能让坦克拐弯。不一会儿,贝塔就能熟练地操纵坦克了,现在,贝塔不怕咪丽了,他甚至盼着咪丽快点回来;这种心情贝塔还是头一次有。

贝塔把坦克隐蔽在床底下,焦急地盼望眯丽的出现。

\chapter{第4集} 

贝塔驾驶坦克大败咪丽;贝塔击退咪丽援兵;咪丽奄奄一息;贝塔出走

贝塔从潜望镜里看见咪丽回来了,他按了一下电钮,操纵坦克向咪丽冲过去。

眯丽看见一辆坦克从床底下开出来,她还没明白是怎么回事时,坦克已经撞到她身上,把她撞了一个跟头。她刚站稳,坦克又冲过来了,又是一个大跟头。咪丽急忙跳上桌子,气喘吁吁地看着这辆凶猛的坦克。

不一会儿,坦克上的盖子打开了,露出贝塔的头。

“喂,怎么样?害怕了吧?”贝塔嘲笑地说。

咪丽一看是贝塔,猛地从桌上扑下来。

贝塔连忙钻进坦克,等咪丽刚落地,坦克又把她撞了个跟头。这次撞在头上,咪丽两眼直冒金星。

咪丽傻眼了,忙逃回到桌子上。

这回,贝塔不理咪丽了,他开着坦克来到咪丽的饭碗旁边,把咪丽的食物都搬进了坦克。

贝塔在坦克里荡气回肠地狂吃。通过潜望镜,贝塔看见咪丽急得直跺脚,他得意地笑了。

这天咪丽没吃上东西。

贝塔决定以后就住在坦克里。他找来一些棉花,在坦克里铺了一张舒舒服服的软床。又找了一个纸盒子当贮藏食物的仓库。

白天,贝塔把坦克开到床底下隐蔽起来。晚上,他开着坦克出来吃主人给咪丽准备的食物。

一到夜里,整个屋子就成了贝塔的天下。他驾驶着坦克横冲直撞,追得咪丽满屋子乱蹿。

咪丽决定去搬援兵。

“臭贝塔,你等着,一会儿非把你的乌龟壳翻个底朝天不可,哼!”咪丽边说边跑出屋子。

“她要真叫来十几只猫,把坦克翻过来就糟了。”贝塔着急了。

他忽然看见了坦克上的大炮,对,用大炮打他们!可没有炮弹呀,贝塔眼珠一转,想出个主意。

床底下的篮子里有不少花生米,贝塔拿了个口袋,装了满满一口袋,搬进坦克里。他把一颗花生米塞进炮膛,一按电钮,“啪!”打出去一颗。

贝塔很快发现炮上也有一个小镜子,那是瞄准镜。他又装进一颗炮弹,瞄准挂在墙上的气球,一按电钮,“啪!”气球炸了。

现在贝塔什么都不怕了,他把炮口对准门口,装好炮弹,等着咪丽。果然,咪丽叫来了5只猫!

“他在哪儿?”一只黄猫刚进屋就说。他不相信一只老鼠能把猫治住。

话音未落,就昕“啪”的一声,黄猫的门牙被打掉了,疼得他“嗷嗷”直叫。

另一只灰猫朝着坦克冲过来。

贝塔瞄准他的鼻子又是一炮,炮弹打进灰猫的鼻孔里出不来了,疼得他掉头就跑。

另外几只猫都傻了眼,他们看见一辆绿色的坦克从床底下冲出来,一边开炮一边横冲直撞,猫们争先恐后逃出了屋子。

从此以后,不管白天晚上,整个屋子都成了贝塔的天下,就是咪丽跑到衣柜上,贝塔的炮弹也能打着她。

咪丽想了许多办法,可每次她都败在贝塔手下。她的饭碗已经成了贝塔的饭碗。主人惊奇地发现,近几天从未丢吃的,他还以为这是眯丽的功劳,因此决定好好慰劳她。主人每天往咪丽的饭碗里放好吃的,他哪儿知道,咪丽一点儿没吃着,全让贝塔享用了。

咪丽已经整整4天没吃东西了。这天中午,她悄悄爬上了餐桌……

“好啊,你竟敢偷吃东西!我每天给你那么多饭还不够你吃!你个馋猫!”

主人看见咪丽居然敢爬到餐桌上偷吃他的饭,大发脾气,抄起鸡毛掸子没命地打咪丽,吓得咪丽在屋里上蹿下跳。

贝塔在床底下开心极了。当咪丽躲到床底下时,他就开炮把她轰出去。

这天晚上,主人用绳子把咪丽捆在椅子腿上,惩罚她。

贝塔的坦克缓缓地停在咪丽身旁,当贝塔确信咪丽已经被捆得结结实实之后,他打开坦克舱盖,钻出来坐在炮塔上,二郎腿一跷,悠闲自得地看着咪丽。

咪丽看了一眼贝塔,闭上了眼腈。她饿极了,再加上全身被打得火烧一样的疼,浑身无力,骨头都快散架了。

贝塔本来想好好取笑她一番,可看到咪丽这副可怜的样子,贝塔想起了自己从前挨饿的日子,他开始同情咪丽了,贝塔后悔不该把咪丽弄到这个地步。

贝塔从坦克上跳下来,走到咪丽身旁。

“饿肚子最难受了,我知道。”贝塔一边说,一边开始咬捆在咪丽身上的绳子。

咪丽睁了一下眼睛,看看贝塔,又闭上了。

“我一会儿给你点儿吃的。”贝塔继续咬绳子。

尼龙绳很结实,贝塔的牙齿都咬疼了,还剩最后一根。贝塔稍微歇了一会儿,用劲把最后一根绳子咬断了。

咪丽猛一回身,一口咬住了贝塔。

贝塔万万没想到,咪丽会来这一手,他不顾身上火辣辣的疼痛,回头咬了咪丽鼻子一口。

咪丽疼得大叫一声,松开了嘴。她实在太饿了,无力追捕贝塔。

贝塔钻进坦克,把坦克开到床下,他听到主人起来了。

主人听到咪丽叫,打开灯一看,咪丽居然敢把绳子咬断了。他勃然大怒,拿起鸡毛掸子又是一顿猛打,这回咪丽连跑的劲儿都没有了。

打完之后,主人又把咪丽捆在椅子腿上。

贝塔通过潜望镜看着这一切,开始他觉得挺出气,可后来又觉得咪丽挺可怜。但贝塔实在想不通刚才咪丽干嘛恩将仇报咬他呢?

贝塔觉得要是自己能吃上饭,咪丽就吃不上饭。如果咪丽有饭时,那他贝塔就得挨饿。要是他俩能一起吃该有多好。可看样子咪丽不会这样干。

“干脆,我离开这个屋子,自己到外边去闯荡吧。”贝塔拿定了主意。他不愿意让咪丽总是饿肚子。

贝塔的坦克又缓缓地停在了咪丽身旁。这回,咪丽连眼睛都不敢睁了,她知道贝塔一定会狠狠地报复她。

咪丽觉得鼻子前面有香味儿,她睁开眼睛一看,贝塔把坦克里的食物搬出来放在咪丽面前。

“我要走了,请原谅,我实在不敢再把绳子咬断了。”贝塔说,“你吃吧,饿肚子最难受了,好了,后会有期。”

贝塔说完钻进坦克里。一想到再见不到咪丽了,贝塔心里还有点儿酸溜溜的感觉,奇怪。

贝塔又把坦克舱盖打开,最后看一眼咪丽。咪丽正大口大口地吃着贝塔给她的食物。贝塔头一次看见,咪丽的眼睛里有晶莹的泪水。

贝塔盖好舱盖,驾驶着坦克,从咪丽出入的小门驶出了屋子。外边是满天星斗。

\chapter{第5集}

贝塔的炮弹打伤了小麻雀;舒克的直升机营救小麻雀

贝塔开着坦克来到野外,天黑得伸手不见五指,什么也看不见,他打开了照明灯。

通过潜望镜,贝塔看见四周都是灌木丛,前方有一堆小石子。

“拿花生米当炮弹太可惜,”贝塔想,“不如用小石子当炮弹。”

贝塔对屋外这个世界还很陌生,由于他一生下来就在惊恐中生活,养成了谨小慎微的习惯。这次如果没有坦克给他壮胆,他是无论如何也不敢跑到外面来的。贝塔决定把炮弹储备得足足的,以防万一。

贝塔把坦克停在石子堆旁边,听听四周没有动静,他轻轻打开舱盖儿,钻出来,将许多小石子运进坦克。有这么多炮弹,贝塔心里踏实多了。

贝塔忙完后,吃了两颗花生米,躺在坦克里他的软床上,唾着了。

一阵吵闹声惊醒了贝塔,他趴在潜望镜上一看,天已亮了,一群麻雀落在他的坦克上,正叽叽喳喳地议论着。

“这是什么?昨天还没有呢!”

“可不是吗,怎么一动不动呀?”

“是个死东西吧?”

“讨厌!”贝塔决定吓唬他们一下。他悄悄发动了坦克,猛然向前一冲,吓得麻雀们都飞了起来。

贝塔得意极了。他操纵坦克掉回头来,通过潜望镜看着落在树枝上的麻雀们。

“这是乌龟吧?”一只小麻雀说。

贝塔觉得“乌龟”是骂人的话,咪丽就这样骂过他。他要教训这只小麻雀一下。

贝塔把炮口对准了小麻雀,装上石子炮弹,一按电钮,只听“啪!”的一声,小麻雀掉在地上一蹦一蹦的,贝塔的炮弹打中了他的翅膀。

贝塔清楚地看见小麻雀的翅膀在滴血,他原以为打小麻雀也像打咪丽一样,不会打伤。没想到小麻雀这么娇气,再加上炮弹由花生米换成了石子。

贝塔挺后悔,他把坦克开到小麻雀身旁.可又不敢走出坦克。

贝塔这一炮可把麻雀们吓坏了,他们眼巴巴地看着小麻雀在地上挣扎,眼睁睁地看着坦克朝麻雀开过来,干着急没办法。

“对了,快去叫舒克!”一只麻雀忽然想到了舒克。

舒克神通广大,在这一带已出了名。

舒克正在擦他的直升机,一只麻雀气喘吁叮地飞过来,差点儿撞在飞机上。

“舒克,快去,不好了……小麻雀被……被一个怪物……打断了……翅膀……”

“啊?!”小麻雀是舒克的好朋友,舒克曾经救过他,他也救过舒克。舒克一听说小麻雀遇到不幸,急得直跺脚。

“快上飞机!”舒克和那只麻雀钻进直升机。不到5秒钟,直升机便腾空而起,以最快的速度朝出事的地方飞去。

正当贝塔犹豫着是不是应该出去给小麻雀道歉时,忽然听见天上传来一阵发动机的声音。他往上一看,是一架直升机。贝塔在电视里见过这玩意儿,似乎也挺厉害。

舒克操纵飞机下降,看清了,那怪物是一辆坦克。舒克决定先把小麻雀救出去再收拾那坏蛋坦克。

直升机在坦克上空盘旋,贝塔弄不清它要干什么。只见飞机下边伸出来一根绳子,飞机上的麻雀喳喳地叫着,受伤的小麻雀抓住绳子头儿,被救上去了。

贝塔心里挺不是滋味儿,他有点儿恨那架直升机,说不清为什么。

\chapter{第6集} 

舒克的直升机和贝塔的坦克之间展开一场大战

正当贝塔准备开着坦克离开这块是非之地时,坦克猛烈地晃动了一下,他的头重重地撞在炮膛上,起了一个大包。

只听一阵飞机轰鸣声由近而远。

当贝塔还没明白过来是怎么回事时,坦克又一次震动,贝塔的头也就又撞了一次炮膛,两个大包了。

一阵飞机轰鸣声由近而远。

贝塔清醒了,他一面捂着脑袋一面往外看,原来是那架米黄色的直升机故意使劲地往贝塔的坦克身上落。贝塔火了。他找出坦克帽戴在头上,这样就不怕撞了。他把坦克发动起来,停在原地不动,等着直升机再一次往下压他的坦克。

舒克的直升机第三次降下来压贝塔的坦克,就在直升机的轮子刚要撞着坦克时,贝塔操纵坦克躲开了,舒克的直升机控制不住,撞在地上,把地撞了一个坑。

贝塔操纵坦克来了个一百八十度的转弯,全速朝舒克的直升机撞过来。

舒克毕竟是有丰富经验的飞行员,就在坦克要撞上飞机的一刹那,直升机拉起来了,而贝塔的坦克刹不住车,撞在一棵树上,把树撞倒了。幸亏贝塔戴着坦克帽,要不然头上又该多一个大包了。

这次贝塔可真生气了,他瞄准悬在前方空中的直升机就是一炮,直升机被打穿了一个小窟窿。舒克害怕了,连忙把飞机拉得高高的。

“这家伙真坏,仗着自己有武器就欺负人。”舒克看看躺在机舱里受伤的小麻雀,心想,一定要治治这个开坦克的坏蛋。

舒克开着直升机离开了贝塔的坦克,他到河边装石头去了。贝塔以为自己把直升机打跑了,很得意。

“嗵!”舒克从天上往下扔石头,就像飞机扔炸弹一样。石头砸在坦克上,几乎砸穿了车身。

贝塔开着坦克就跑,舒克驾驶着直升机在天上追,边追边扔石头,可是,不是扔早了就是扔晚了,再加上贝塔一会儿开快,一会儿开慢,老砸不着。

贝塔看见前方有一片小树林,他想出了一个主意,贝塔操纵坦克用最大速度朝小树林驶去,舒克在空中紧追。

贝塔的坦克钻进了小树林,舒克的直升机也在小树林中穿行。贝塔的坦克一会儿往左拐,一会儿往右拐。终于,舒克的直升机被挂在树上了。

这下贝塔可得意了,他往炮膛里装了一颗大炮弹,瞄准了直升机的驾驶舱,但是贝塔的手没有按电钮,他也不知为什么。

舒克清清楚楚地看见坦克的炮口对着自己,他一点儿办法也没有,只好闭上眼睛等着坦克开炮。

就在这时,几十只麻雀飞来落在挂着直升机的树枝上,他们一起使劲儿摇树枝,直升机掉下来了,就在接地的一刹那,飞机的螺旋桨起动了,直升机拔地而起。

“这家伙技术不错。”贝塔不得不承认。

直升机飞走了,大战宣告结束,谁也没赢,谁也没输。贝塔真没想到,外面的世界这么复杂,刚出来就打了一仗,总算还平安。贝塔有点儿累,他检查了一遍舱盖儿确实锁牢了,就躺在他的软床里睡觉了。

\chapter{第7集}

贝塔的坦克飞到了天上;坦克和飞机在空中同老鹰展开空战

贝塔睡得迷迷糊糊的,忽然觉得身体摇晃起来,他睁开眼睛一看,自己还在坦克里。是做梦吗?贝塔爬起来,趴在潜望镜上往外一看,差点儿叫出声来——他的坦克飞起来了,旁边是蓝天,下边是大地。

贝塔揉揉眼睛,没错,他的坦克上天了!这是怎么回事?

贝塔发动坦克,没用,车轮只能空转。

贝塔忽然听见头顶上有飞机的声音,他把炮塔上的舱盖儿打开一条小缝儿,往上一看,那架直升机用绳子把他的坦克吊起来了。

原来,舒克和小麻雀们昨天晚上商量了一下,觉得这辆来历不明的坦克严重危害大家的安全,舒克想了这么个办法,用他的直升机把坦克吊到很远的地方驱逐出境。于是,趁贝塔睡觉的工夫,大家悄悄用绳子把贝塔的坦克捆了起来。天一亮,舒克就驾驶着直升机把贝塔和他的坦克一起吊到了空中。

现在,舒克正吊着贝塔的坦克用最高速度朝西北方向飞去,他也不知道把坦克运到哪儿去,反正越远越好。

贝塔急了,他往炮膛里塞了一颗炮弹,可他的炮口不能往上抬九十度射击。贝塔按住炮塔旋转按钮不放,他的炮塔发疯一样地旋转起来,可绳子是捆在炮塔上的,一点儿用也没有。贝塔的头都转晕了,他一松开按钮.拧成麻花的绳子又往回转,整个坦克也跟着往回转,转得他都快吐了。

忽然,贝塔觉得上面往下滴水,他还以为下雨了,挺高兴,贝塔一天没喝水了。他立刻把嘴接在滴水的地方,很臊,不是雨,是尿。原来舒克怕绳子不结实,急中生智想出了这个办法,把绳子弄湿了不就不会断了吗?贝塔知道又上当了,可他干着急,干生气,一点儿办法也没有。

舒克觉得飞得够远的了,即使这辆坦克日夜兼程往回开,也得开上三天三夜。他准备找一个合适的地方着陆。

就在这时,舒克发现在他的直升机上方出现了一个黑点儿。那黑点儿越来越大,舒克觉得脖子后面有点儿发凉。他看清楚了,那是一只老鹰!

老鹰的眼睛最尖,他一眼就看清楚直升机里的舒克是他喜欢吃的食物,他收拢翅膀,飞快地俯冲下来。

贝塔虽然没看见老鹰,但他也本能地预感到有危险要降临,他的脖子后面也阴森森的。

就在老鹰扑过来的一瞬间,舒克拉起了直升机,老鹰扑了个空。

贝塔从潜望镜里看清楚了,是一只老鹰!老鹰转过身子,又扑过来。

贝塔的炮里正好有一发炮弹,他瞄准老鹰开炮,没打中。在空中射击非常困难,双方都在运动中,很难打中。

舒克原打算立即把坦克扔下去,吊着它非常不灵活,很难躲过老鹰的袭击。当舒克发现吊在下面的坦克冲老鹰开炮后,他马上把这个敌人当成了自己的同盟军。

老鹰又扑过来了。

贝塔装上一发炮弹,瞄准了目标。

近些,再近些!贝塔的手直哆嗦,他看清了老鹰那带勾的嘴和刀子一样锋利的爪子。就在老鹰的爪子刚要抓住吊坦克的绳子时,贝塔按下了射击按钮。

打中了!老鹰掉了下去.但马上又飞了起来,老鹰毕竟是老鹰,不像麻雀那样娇气。

老鹰没想到对手还有武器,他同直升机保持了一段距离,在想对策。

“这家伙还真有两下子!”舒克到现在还不知道坦克里是谁,不过他已经喜欢上他了。舒克忽然想起直升机里的电台,他戴好耳机,对着话筒喊起来。

贝塔的坦克里也有电台,贝塔不知道它的用处。现在电台里传出了声音,贝塔觉得挺好玩。

“喂!喂!喂!”舒克呼叫。

“干吗?干吗?”

“我是舒克!”

“什么舒克?”

“飞行员舒克。”

“啊?就是你把我吊到天上来的!”

“真对不起,你是谁?”

“不告诉你。”

贝塔不愿意让人家知道他的身分,他觉得这个世界上谁都可以欺负他。

“谢谢你开炮打跑了老鹰。”

“这算什么,我还没使劲儿打呢!”

“咱们交个朋友吧!”

“你先把我放到地上去。”

“不好,老鹰又来了!”

贝塔一看,狡猾的老鹰从下面往上飞扑过来。

贝塔的炮打不着他。

“降低高度!”贝塔命令。

“明白!”舒克操纵飞机急速下降。

贝塔的炮口瞄准了老鹰。

“这次你使劲儿按炮钮。”舒克说。

“少废话!”

又打中了!看来这次老鹰疼得够呛,挣扎着飞走了。

舒克和贝塔胜利了。

\chapter{第8集}

舒克的直升机坏了;舒克和贝塔降落在一个陌生的地方;他俩终于见面了

舒克现在对吊在下面的这个盟军佩服得五体投地。

“你真行!”

“少来这套。”贝塔想起舒克把他吊到天上来就有气,连招呼也不打!

“我现在就送你回去。”舒克想起自己把人家吊到天上来,心里挺过意不去。他开始操纵直升机返航。

“这还差不多。”贝塔往嘴里塞了一颗花生米,“你刚才往下撒尿了?”

“是,我怕吊坦克的绳子太干燥会断。”

“我以为是往下淋水呢,就伸嘴接着,上你的当了。”

“实在抱歉。”舒克说完觉得发动机的声音有些不正常,糟糕,飞机出故障了!

“注意,飞机出故障,马上要迫降!”舒克通知贝塔。

“什么破飞机,白给我都不要。”贝塔嘴上这么说,心里挺害怕。他知道,飞机要是掉下去,他就没命了。

舒克发现地面上有一座城堡,他想操纵飞机绕过这座城堡,追降在野地里。可飞机已经不听他的指挥了,一个劲儿往下掉。舒克没办法,只好在城堡里着陆。

如果舒克知道这是一座什么城堡的话,那他宁愿摔死也不敢在这里降落。这是一座猫城——克里斯王国。克里斯王国的所有公民都是猫。

舒克总算平安地把飞机降落在一块开阔地上,坦克先着陆,直升机落在一旁。

贝塔从潜望镜里看着停在旁边的直升机,他想看看这个同盟军是什么模样。

直升机的舱门打开了,贝塔不相信自己的眼睛。怎么?飞行员舒克也和他贝塔一样,是老鼠?!

舒克跳下飞机,把悬吊贝塔坦克的绳子解开收好。

舒克正准备修理飞机,他突然呆在那里,一动不动。舒克和贝塔几乎是同时看见远处有三只穿着军装的猫。

“快进来!”贝塔打开坦克舱盖儿冲舒克喊。

舒克想了一下,觉得坦克比飞机安全,因为飞机已经不能飞了。

舒克急忙钻进贝塔的坦克。贝塔把盖儿盖紧,锁牢。

“怎么,你也是…”舒克也没想到这个盟军竟是自己的同胞。

“我叫贝塔。”能见到自己的同胞,贝塔很高兴。

“我是舒克。”舒克和贝塔握手。

“你看。”贝塔看见那三只猫士兵朝坦克走过来。

“别怕。”舒克安慰贝塔,其实他的心跳得特快。

\chapter{第9集}

舒克不让贝塔开炮打猫宪兵;贝塔驾驶坦克甩掉猫宪兵:舒克贝塔被包围

三只穿军装的猫宪兵朝舒克和贝塔走过来,贝塔把坦克舱盖儿锁牢。他俩的心脏发出嗵嗵嗵的响声,吓得连大气也不敢出。

贝塔趴在潜望镜上,看见猫宪兵越走越近。

“你的飞机怎么维修的?够呛!”贝塔小声埋怨舒克。他太怕猫了。

“从来没出过故障,准是你的坦克太重了!”舒克把责任推到贝塔身上。

“我又没请你把我的坦克吊到天上!”贝塔生气了。

“算了算了。”舒克觉得现在不是吵嘴的时候,“快看看,他们要干什么?”

通过潜望镜,贝塔看见三只猫宪兵站在坦克前面,他们好奇地看着坦克,其中一只猫还摸摸坦克的履带。另一只猫宪兵朝舒克的直升飞机走过去。

贝塔往炮膛里装了一发炮弹,瞄准了一只猫。

“别打!”舒克小声说。

“干吗?”贝塔不明白。

“现在他们不知道坦克里边是什么东西,你一开炮,他们该报复咱们了。再说,咱们还不知道这是座什么城堡,看样子想逃出去不大容易,还是让他们弄不清咱们的底细安全些。”

贝塔觉得舒克说得挺有道理,他没开炮。

“那咱们也不能老呆在这儿呀!,贝塔实在害怕这三只猫。

“咱们去别处看看,先躲开猫再说。”舒克也怕猫。那次他去赴蚂蚁皇后的宴会时,差点儿被猫吃了的经历一直没忘,想起来身上就发抖。

这时,一只猫爬上了坦克。

贝塔按了起动按钮,坦克猛然向前开去,把那只猫甩到地上。

三只猫宪兵定了定神儿,跟在坦克后面追上来。

“快,再快点儿!”舒克催贝塔。

贝塔已将速度按钮按到底了,坦克呼啸着朝前驶去。

“你的飞机不要了?”贝塔边开边问。

“先把这些猫引开,一会儿回来修。”

“这家伙挺鬼!”贝塔想。他不得不承认舒克点子多。

舒克看见坦克里有花生米,拿起一颗。

“可以吗?”舒克一边往嘴里送一边问。

贝塔点点头。舒克大口大口吃起来。

绕过两座房子,坦克来到街上。

潜望镜里的情景使贝塔大吃一惊,他操纵坦克来了个急刹车。舒克的头重重地撞在舱壁上。

“你干什么?”舒克火了,“刹车也不告诉一声!”

“你看!”贝塔离开潜望镜.让舒克看。

舒克趴在潜望镜上一看,心脏几乎停止了跳动:街上到处都是猫。

“倒车!”舒克忙说。

坦克掉过头,朝相反的方向开去。没开多远,又是一个急刹车。

舒克和贝塔终于明白了,这是一座猫城。

这时,街上的猫都被这个新奇的玩艺儿吸引住了,潮水般朝坦克围过来。

舒克和贝塔无路可逃。

\chapter{第10集}

贝塔用炮塔把猫公民们吓跑;猫宪兵把坦克翻了个底朝天;坦克变成了潜水艇;舒克和贝塔被活捉

“反正他们打不开坦克的舱盖。”贝塔自己给自己壮胆。其实他的腿直哆嗦。

“就是,别看他们是猫,根本治不住坦克。”舒克也一边发抖一边给自己鼓劲儿。

贝塔又检查了遍舱盖,确实锁牢了。

这时,几百只猫把坦克围得水泄不通。他们的议论声传进了坦克里。

“这是什么东西?”

“不知道。”

“没见过。”

“从哪儿来的?”

“昕说是从天上掉下来的。”

“天上?”

“能打开吗?”

“试试。”

于是就传来了猫爪子抓坦克舱盖的声音。

舒克和贝塔紧紧靠在一起,眼睛死盯着舱盖。

舒克想起了小花猫。贝塔想起了咪丽。他俩几乎是同时蹦起来。

“咱们不能等死!”舒克说。

“就是,拼拼看!”贝塔立即支持。

贝塔坐到驾驶座上。舒克坐在炮手的位置上。他们系好安全带。

“我转炮塔,吓他们一下。”贝塔让舒克作好准备。

“转吧!”舒克说。

“你不怕晕吧?”贝塔问。

“飞行员还有怕晕的?”舒克觉得贝塔太小看他了。

“那我就转了。”贝塔按下炮塔旋转按钮。

坦克上的炮塔飞快地旋转起来,炮管把好几只猫撞出去老远。

克里斯王国的公民们吓坏了,他们扭头就跑,边跑边发出尖叫声。他们弄不清这是什么怪物。

猫们发现怪物没追上来,才停住,他们站在老 远的地方胆怯地望着坦克。

“他们怕我的坦克!”贝塔兴奋了。

“他们怕咱们!”舒克也来劲儿了。

原来猫也有害怕的时候,舒克和贝塔决定治治这些老鼠的冤家。

“用坦克撞他们!”舒克提议。

“行!”贝塔发动了坦克,一推操纵杆,坦克的履带飞快地转起来。

“坦克怎么不动了?”贝塔从潜望镜往外一看,坦克纹丝不动,可履带却在转。

舒克凑过去一看,慌了。两只猫宪兵把坦克抬起来了,坦克的轮子在空转。

紧接着,舒克和贝塔觉得天旋地转,他俩头朝下了!要不是系着安全带,非得重重地撞一下头不可。

猫宪兵把坦克翻过来了。坦克轮子朝天,任贝塔怎么加大速度,轮子只能空转。

贝塔和舒克傻眼了。

看到怪物被治住了,吓跑的猫公民们又慢慢围拢过来,但他们作好了随时跑的准备。

“这回安全了,舱盖想打也打不开了。”贝塔说,他头朝下吊着。

“我的头有点儿受不了了。”舒克也是头朝下,他的脸憋紫了。

“飞行员还怕头朝下?”贝塔撇撇嘴。

舒克不吭气了。他原想解开安全带,把身子正过来,可又怕贝塔笑话飞行员还不如坦克兵。舒克只好忍着。

其实贝塔也快不行了,但他下决心一定要坚持到舒克忍不住为止,煞煞他那飞行员的优越感。

“把这怪物扔到池塘里去吧,放在大街上太危险。”一只猫提议。

所有的猫都赞成这个办法。

舒克和贝塔的心本来就快到嗓子眼儿了——现在他俩的心差点儿从嘴里掉出来。

坦克被翻过来了,猫公民们抬着坦克朝池塘走去。

舒克和贝塔慌了。

“你这坦克漏水吗?”舒克问。

“又不是船,当然漏。”贝塔说。

“咱们要是鱼就好了。”

“我可不愿意当鱼。”

“怎么?”

“当鱼还得让猫吃!”

“真是的。那咱们要是乌龟就好了。”

“少废话,想点儿办法吧!”

贝塔把床上的棉花拿起来,见缝儿就塞。舒克也学着贝塔的样子堵缝儿。

通过潜望镜,贝塔看见他们的坦克已被抬到池塘边上。

“一、二、三!”猫们一起使劲儿,只听“扑通”一声,坦克被扔进池塘里。

还好,坦克里边没进水!舒克和贝塔只觉得气短,呼吸越来越急促。

“糟糕,坦克里快没空气了。”贝塔说。

“把炮管抬起来,说不定能伸出水面。”舒克灵机一动。

贝塔按电钮操纵炮管往上抬,炮管果真伸出了水面。

“快,把炮弹退出来!”舒克说。

贝塔把炮弹从炮膛里退出来,然后把嘴对在炮膛上,有空气了!坦克变成了潜水艇。

“你来!”贝塔让给舒克。

舒克只吸了一口,又让给贝塔。

“你多吸两口!”贝塔说。

“我是飞行员,体质好。”舒克说。

又来了,贝塔最讨厌舒克跟他摆飞行员架子。

“我的体质也不差!”贝塔赌气,不吸。

舒克呼吸越来越急促。

贝塔也快挺不住了。可谁也不去吸空气。不过,他俩的头不由自主地离炮膛越来越近。

贝塔忽然觉得脚有点儿凉,他低头一看,坦克漏水了。贝塔忙拿棉花去堵漏洞。舒克也帮着堵。

漏进来的水越来越多,已经没到舒克和贝塔的胸部了,水位还在上涨。

“等舱里的水满了,咱们也就完了。”舒克耸耸肩膀。

“好在你是飞行员,体质好,不怕。”贝塔冲舒克挤挤眼睛。

“当然。不过……”舒克喝了一口水。水已经到他下巴了。

“出去吧?”贝塔问。

“当然。等着淹死不如出去碰碰运气。”舒克脱下套在飞行服外面的救生衣,递给贝塔,“你穿吧,这是救生衣,能浮在水面上。”

“不要,我会游泳。”贝塔摇摇头。

“穿上吧!我……体质好。”舒克没敢再提飞行员。

贝塔穿上了救生衣。冲舒克笑笑。

这时,坦克舱里的水已经快满了。贝塔恋恋不舍地看看自己心爱的坦克,打开了舱盖。

舒克和贝塔钻出坦克,向上游去,把头露出水面。

池塘四周都是看热闹的猫。他们看见舒克和贝塔,叫起来,几只猫跑去拿来打鱼的大网,把舒克和贝塔罩住了。

\chapter{第11集}

克里斯王国的猫没见过老鼠;猫公民帮助舒克和贝塔打捞坦克;飞行表演

舒克和贝塔被克里斯王国的猫公民们用渔网从池塘里捞了上来。

舒克和贝塔明白,他俩的末日到了。

“再见了,飞行员舒克。”贝塔在渔网里递给舒克一个勉强的笑容。

“再见,坦克兵贝塔。”舒克耸耸鼻子。

“我真想让你再把我吊到天上。”贝塔说。

“我也想让你用炮再打我一次。”舒克说,“说实话,你的炮打得不赖。”

“说实话,你的体质还真不错,不愧是飞行员。”贝塔说。

渔网被拖上岸了。舒克和贝塔被猫们从渔网里拽出来。他们闭上眼睛,等着猫吃他们。

“这么多猫,还不把我和贝塔撕成碎片啊!”舒克想。

一分钟过去了。两分钟过去了。十分钟过去了,怪事,这些猫干什么哪?

舒克和贝塔睁开眼睛,他俩身边全是猫。这些猫好奇地看着舒克和贝塔。

贝塔和舒克纳闷了,他们怎么还不吃,看什么?难道还有不认识老鼠的猫?

“请吃吧!”舒克实在受不了这种侮辱,他要维护一个飞行员的尊严,“还等什么?”

“吃什么?”一只猫问。

“吃我们呀!”贝塔说。真怪,他现在反而一点儿也不怕了。

“吃你们?你们能吃?”猫公民们惊讶极了。

猫不知道老鼠能吃!

“你们听说过老鼠吗?”舒克心里一动,他问猫公民们。

“老鼠?”猫公民们摇摇头。

原来克里斯王国的猫没见过老鼠,他们这儿从来没有过老鼠!他们不知道猫和老鼠是冤家。

“你们是谁?”一只猫宪兵问。

“我是飞行员舒克。他是坦克兵贝塔。”

“干吗到我们克里斯王国来?”

舒克把他和贝塔怎么在天上遇到老鹰,飞机怎么出了故障,以及他们怎么在克里斯王国迫降等等,统统讲了一遍。

“飞机真能飞到天上去?”猫们不信。

“池塘里那怪物是坦克?”猫公民们感兴趣了。

他们决定帮舒克和贝塔把坦克从池塘里捞上来。他们有的去捞坦克,有的给贝塔和舒克端水喝。

在舒克和贝塔心目中,猫是凶恶的化身。他们万万没想到,猫还会笑着说话,还会这么热情。

坦克捞上来了,贝塔和舒克在猫公民的帮助下,把坦克车身上的水擦干净。

虽然同猫在一起干括舒克和贝塔还不大适应,但他俩觉得这些猫仗义。

“真对不起!”一只猫说。

他大概是世界上第一只给老鼠道歉的猫。舒克和贝塔激动得不知说什么好。

“要是世界上的猫都像他这样就好了。”贝塔想。

“没关系。”舒克说。

坦克擦干净了,贝塔钻进去,试试车,发动机正常。

“给我们表演一下行吗?”一只猫提议。

“当然可以!”贝塔很乐意显示一下自己高超的驾驶技术。他招呼舒克进来。

舒克也钻进坦克,坐在贝塔旁边。

贝塔猛一按启动电钮,坦克风驰电掣般地围着猫公民们绕开了圈子。猫公民们的身体跟着坦克转,他们的头都快转晕了。贝塔的坦克开得快极了,几乎只能看见一道光圈围着猫公民们转。

“开一炮,让他们看看!”舒克说,他从炮弹箱里取出一发炮弹,递给贝塔。

贝塔把坦克停住.将炮弹塞进炮膛,瞄准一棵大树上的树叶,按下了射击按钮。

树叶被打掉了好几片。

猫公民们欢呼起来。

“我也能打落树叶。”舒克觉得树上那么多树叶,随便射击都能打掉几片。

“我是瞄准这几片树叶开炮的!”贝塔吹牛。

舒克撇嘴。

“能让我们看看飞机上天吗?”等坦克停卜来,一只猫趴在坦克上问舒克和贝塔。

“当然可以!”舒克来幼儿了,“我先去把飞机修好。”

克里斯王国的猫公民们抬起贝塔的坦克,浩浩荡荡地朝舒克的直升机走去。贝塔和舒克站在坦克的炮塔上,神气极了。

直升机的发动机出了故障,舒克对他的直升机挺熟悉,很快就修好了。

猫公民们把舒克的直升机和贝塔的坦克围得水泄不通。他们很喜欢这两位身怀绝技的飞行员和坦克兵。

舒克钻进直升机的驾驶舱。

“上来吧?”舒克邀请贝塔。

“你自己去吧,我在下边等你。”贝塔对舒克的直升机的安全性表示怀疑。

直升机的螺旋桨转起来了。紧接着,直升机腾空而起。

猫公民们先是愣了一下,接着一阵欢呼。

舒克驾驶着直升机在克里斯王国上空作着各种高难度的飞行动作。一会儿空中悬停,一会儿急转弯,一会儿垂直起落……

猫公民们看花了眼,赞叹声连成一片。

贝塔把头伸出炮塔,他不得不承认舒克的飞行技术是第一流的。贝塔想,他的坦克如果和舒克的直升机联合起来,就谁也不怕了。

“贝塔!贝塔!我是舒克!你听见了吗?”贝塔的耳机里传来舒克的呼叫声。

“我是贝塔!我是贝塔!”贝塔兴奋极了,望着天上的直升机答应着。

“我现在要给他们作一次超低空飞行表演,请你在地而指挥!请你在地面指挥!明白吗?”

“明白!你说话小点儿声,把我耳朵都快震聋了!”

“开始吧?”舒克请示地面。

贝塔看见舒克的直升机盘旋了一圈儿,悬停在空中,做好了超低空飞行的准备。

“开始!”贝塔命令,“降低高度!”

舒克的直升机一边降低高度一边朝猫公民们飞来。

“再低些!再低些!”贝塔指挥着。

“明白!”舒克回答。

直升机几乎是擦着地面飞过来,吓得观看表演的猫公民们都趴在地上。

“拉起来点儿!”贝塔全神贯注地观看着直升机与地面的距离。他知道,只要稍一疏忽,就会机毁鼠亡。

直升机擦着猫公民们的头飞过去了,螺旋桨掀起的强大的风吹得猫们站不住脚。

克里斯王国的公民们大饱了眼福,他们对舒克和贝塔佩服得五体投地。   

\chapter{第12集}

克里斯王国的国王要接见飞行员和坦克兵;舒克和贝塔听说国王会放电;舒克和贝塔见到克里斯国王

舒克驾驶直升机安全着陆。

猫公民们拥到飞机旁边,把舒克和贝塔抬起来,抛向空中,然后接住,又抛起来……当舒克和贝塔被抬到一片绿色草坪上的时候,迎接他们的是丰盛的宴会。猫公民们纷纷从自己家里把最好吃的食物拿来。舒克和贝塔已经饿坏了,他们准备饱餐一顿。

再说,猫宴请老鼠,这意义也不一般。

这时,走来一队猫宪兵。

“我们的国王听说你们来了,要接见你们。”宪兵对舒克和贝塔说。

“国王!”贝塔一愣,看着舒克。

舒克也觉得国王一定是见多识广的老猫,他肯定认识老鼠。

“我们吃完饭去,行吗?”舒克说,他想来个缓兵之计.等吃完饭,他和贝塔立即飞走。

“可以,请快点儿吃。”猫宪兵站在远处等着。

舒克悄悄地把他的计划告诉了贝塔,他动员贝塔把坦克丢掉,跟他一起坐直升机先跑,过几天再悄悄回来把坦克吊走。贝塔同意了。

“你们的国王很老吗?”舒克边吃边问猫公民。

“没见过。”坐在舒克身边的一只猫说。

“没见过?”贝塔不信。

所有的猫都摇头。

“国王厉害吗?”舒克问。

一提起国王,猫公民们的脸都吓白了。舒克和贝塔看出,克里斯王国的国王一定很凶。

“国王有法术,会放电。”一只猫小声告诉舒克。

“会放电?”舒克大吃一惊。

“咱们快走吧!”贝塔催舒克。

舒克觉得应该和猫公民们打个招呼,不然太不够意思了。

“多谢大家,我们走了。”舒克站起来和猫公民们告别。

一听说舒克和贝塔不见国王就走,猫公民们吓坏了。

“你们要足走了,我们可就没命了。”一只猫说。

“国王该发脾气了!”另一只猫一边打哆嗦一边说。

舒克和贝塔愣住了。

“还走吗?”舒克问贝塔。贝塔耸耸肩膀,坐下了。

舒克也坐下了。猫公民们感激地看着舒克和贝塔。

舒克和贝塔小声商量了一会儿,决定开着坦克和飞机去王宫,见机行事。

贝塔钻进坦克,在猫宪兵的指引下,向王宫驶去。舒克的直升机在空中跟着。

克里斯王国的王宫很漂亮,是大理石建筑。王宫前面有一座广场。

贝塔的坦克停在王宫门前的台阶下边。舒克的直升机在坦克旁着陆。

舒克和贝塔跟着猫宪兵走进王宫。他俩数着,一共经过了37道岗!

“跑不成了。”舒克小声说。

“这个国王一定是坏蛋!”贝塔说。

“怎么?”

“设这么多岗,怕别人看见他干坏事!”

“就是,岗越多,干得坏事越多,越心虚。”舒克同意。

总算来到了一座大殿门口,宪兵示意舒克和贝塔等一会儿。

国王宣召舒克和贝塔进殿。

舒克和贝塔硬着头皮走进去,他俩看见国王后大吃一惊:国王是一只老鼠!一只白老鼠!!

猫国的国王是老鼠!!!

舒克和贝塔高兴了。 

\chapter{第13集}

克里斯王国国王的来历;国王设宴招待舒克和贝塔;舒克和贝塔拒绝吃猫肉

一年以前,一只名叫白路的供医学试验用的小白鼠,当医务人员在他的身上移植了老虎胆和人工心脏后,他利用医务人员的疏忽,逃出了医院。

可想而知,这只装着老虎胆和人工心脏的小白鼠来到外界后,围绕着他,一定会发生一系列极为有趣的事件。

下面是他出逃后的经历。

白路首先遇到一只大猫。

这么干净的白老鼠,大猫还是头一次见到。他咂咂嘴.朝白路逼过来。

“你不怕我电死你?”白路说,他原地不动,像没事儿一样。

“电?”大猫站住了。他怕电。

白路按了一下胸脯,人工心脏上的红灯一闪一闪地亮起来。

大猫被吓住了,这小子身上还真有电!

“怎么样,想尝尝电的滋味儿吗?”白路朝大猫走过去。

大猫连连倒退着。

“不,不,哪儿的事呀!”大猫扭头就跑。

“站住!再跑我就放电啦!”白路吓唬他。

大猫站住了。

“跟我走。”白路说完转身就走,连头也不回。

大猫断定白路身后有电眼,只好老老实实地在后边跟着。

从此,白路就把这一带的猫都镇住了。猫们每天向他进贡食物。他们都怕电,他们不知道电是什么东西,只听说厉害。

一天,白路听说几十里外有一个克里斯王国,王国里有几千只猫,于是,他决定去克里斯王国当国王,享享福。

白路轻而易举地就把克里斯王国的国王赶下了台。一是因为克里斯王国的猫公民从来没有见过老鼠;二是因为他们也怕电,怕得要命。

白路当上了克里斯王国的国王,只有少数几个大臣可以见到他。白路命令王国的公民们为他修建了豪华的宫殿,猫公民们都怕国王放电,只好老老实实地侍候国王,又怕又恨。

其实,只要有一只猫公民稍微试一下,就能知道国王身上只不过装着一节电池,根本没有杀伤力。可是谁也不敢试,而且越传越神,越神越怕,越怕越老实。

再说舒克和贝塔。当他们发现克里斯王国的国王是他们的同胞后,放心了,他们觉得不会有危险了。

白路国王万万没想到召见的飞行员和坦克兵是两只老鼠——他的同胞!白路国王心里一惊,他怕舒克和贝塔把他的底细泄露出去。再说,看到两只老鼠穿着飞行服和坦克装,他心里也有点儿发颤。

国王眼珠一转,想出计策,他要害死舒克和贝塔。

国王喝退了大臣们。大殿里只剩下他、舒克和贝塔。

“你怎么能当上猫国的国王?”舒克亲热地问。在猫国里碰见老鼠国王,他感到很开心,一点儿戒心也没有了。

“一下还说不清。”国王也亲热地说,“你们怎么当上飞行员和坦克兵的?”

舒克和贝塔把经过告诉国王。

“你们是咱们老鼠家族的英雄。”国王竖起大拇指,“我宴请你们。”

“我们刚刚吃过饭。”舒克说。

“那也得吃。”国王说。

盛情难却,舒克和贝塔只得从命。

国王命令侍从去准备宴席,他悄悄吩咐部下在舒克和贝塔的碗里下毒药。

舒克和贝塔很感激国王,他们觉得老鼠当国王一定比猫心地善良,老鼠受的苦太多了。

在国王的陪同下,舒克和贝塔来到宴会大厅。高大宽阔的宴会厅到处是鲜花,宽大的餐桌上摆满桌丰盛的饭菜。舒克和贝塔眼睛都看花了。

“请入席。”国王说。

舒克和贝塔挨着国王坐下。

“这是红烧猫肉。这是清炖猫肉。这是炒猫肝儿。”国王给客人介绍着。

正准备进餐的舒克和贝塔停住了,怎么?这一桌子都是猫肉?国王吃自己臣民的肉?

“吃呀!”国王急了,他怕舒克利贝塔不吃,饭菜里有毒药。

“你天天吃猫肉?”舒克问。

“是的。猫肉很好吃,别客气,快吃!”国王催促道。

舒克和贝塔想起了猫公民们害怕国王的情景。他们万万没想到,老鼠当了国王,比猫更残忍。

“你怎么能吃猫肉呢?”贝塔火了。生来怕猫恨猫的贝塔,居然替猫说话了。

“猫怎么能吃老鼠肉呢?”国王反问。

“这……”舒克和贝塔答不上来,反正他们觉得国王吃自己的臣民不对。

看见舒克和贝塔不吃,国王急了。
        
\chapter{第14集}

舒克和贝塔大战克里斯国王;猫公民们要吃自己的国王;舒克和贝塔带着白路离开王国

“你们吃不吃?”国王拉下脸。

舒克和贝塔摇摇头。

“我放电了?你们不怕电?”国王按了一下胸脯。人工心脏上的红灯一闪一闪地亮了。

舒克和贝塔互相看了一眼,笑了。原来这就是国王身上的电!舒克和贝塔的飞机坦克上都装着电池,他们懂得电,所以不怕电。

国王见舒克和贝塔不怕他身上的电,有点儿慌。他站起来,走近舒克和贝塔。

“我放电了?”国王拿出放电的架势。

“放吧,我身上正需要电呢!”舒克张开双臂。

国王傻眼了。其实他根本放不出电。

“来人呀!”国王大声喊叫起来。

几只猫宪兵跑进来。

“把他俩抓起来!他们是老鼠!猫应该吃老鼠!”围王一急,忘了自己也是老鼠了。

真投想到,老鼠当了国王,对自己的同胞比猫还凶!舒克和贝塔同时朝国王扑过去。

舒克一拳将国王打倒,贝塔用最快的速度打开国王的人工心脏,取出了心脏里的电池。国王躺在地上不动了。

猫宪兵早就恨国王了,国王吃了他们不少亲戚朋友。看见国王躺在地上不动了,猫宪兵们欢呼着跑出王宫,把喜讯告诉全体公民们。

转眼间,猫公民们潮水般地涌进王宫。他们要吃掉国王。

舒克和贝塔不干了。他们一听说猫吃老鼠就火冒三丈。

猫公民们才不管舒克和贝塔的劝阻呢,他们叫骂着冲上前来。

“快把电池装上!”舒克急中生智。

贝塔忙把电池装进国王的人工心脏里。

国王站起来。猫公民们吓得连连后退,纷纷跪在地上磕头。

舒克和贝塔笑得前仰后合。他们明白了,白老鼠之所以能在克里斯王国称王称霸为所欲为,责任不在白老鼠,而在猫公民。

舒克和贝塔动员猫公民们先离开王宫,由他俩治服国王。猫公民们退出去了。

国王老老实实地把他的来历告诉给舒克和贝塔。舒克和贝塔为难了:把白老鼠留下吧,他会继续欺负猫公民们;把他身上的电池取出来吧,猫公民们又要吃他。

看来,只有把白老鼠带走。带到一个人人懂电而又没有猫的地方去,他才会老老实实地生活。

“把他送到发电厂去。那儿人人懂电,听说也没有猫。”舒克提议。

贝塔同意了。

白路不敢反对。

舒克和贝塔走出王宫,向克里斯王国的猫公民们宣布,白路国王辞职了,由他们把国王带走。

猫公民们高兴得跳起了舞。看见自己下台后臣民这么高兴,白路心里挺不是滋味儿。

舒克和儿塔开始检查直升机和坦克。为了方便,舒克和贝塔在坦克上安装了一个铁环,在直升机下边安装了一个铁钩子。这样,直升机吊起坦克就能起飞。

一切准备工作完成了。猫公民们送给舒克和贝塔好多食物,几乎把直升机和坦克都塞满了。

白路和舒克登上了直升机。

贝塔钻进坦克。

直升机起飞了,它悬停在坦克上空,用钩子钩住了坦克。

“准备好了吗?”舒克通过无线电台问贝塔。

“准备好了!”贝塔回答。

舒克一拉操纵杆,直升机向天上飞去,坦克跟着拔地而起。

猫公民们向舒克和贝塔招手,白路在飞机里挺惭愧。

克里斯王国的城堡越来越小了。

\chapter{第15集}   
        在去发电厂的途中。自路准备劫持舒克的直升机; 
        舒克和白路在空中进行搏斗   
        舒克的直升机吊着贝塔的坦克,离开克里斯王国,飞到空中。 
        “贝塔!贝塔!你知道发电厂在哪儿吗?”舒克通过无线电询问贝塔。 
        贝塔正躺在坦克里的床上吃东西。 
        “不知道。我在电视上见过,发电厂有大烟囱,还有许多电线。你把飞机拉高一点儿,看看四周有没有。”贝塔一边吃一边回答。 
        舒克操纵直升机向高空飞去。 
        “你也帮着找找。”舒克对白路说。 
        白路暗中一直注意观察舒克是怎样驾驶飞机的,他准备劫持舒克的直升机。 
        自从登上直升机,白路就被这个现代化的空中飞行器迷住了,他觉得当个飞行员比当国王还要带劲儿!在王官里是贝塔和舒克一同对付他,而现在飞机上只有舒克自己,一比一,白路不怕舒克!你别忘了,白路身体里装的是老虎胆。 
        白路眼光不离舒克,他已经摸到一点儿驾驶飞机的门道了。 
        “你老看我干吗?还不快帮着找找发电厂。”舒克说。他一点儿也没发现白路的企图。 
        “我到后边看看。”白路走到机舱的后边,假装往下看。 
        下边是一片麦地,还有村庄和河流。 
        “找到发电厂了!”舒克兴奋得叫起来。 
        白路跑到舒克旁边往前一看,真的,一座雄伟的发电厂出现在机头前方。 
        “舒克,舒克,我也看见了!”耳机里传来贝塔的声音。 
        白路觉得要是再不动手,就来不及了。他趁舒克正聚精会神地操纵飞机,悄悄地来到舒克背后。 
        舒克正在寻找合适的着陆地点,他猛然觉得胳肢窝特别痒痒,他回头一看,白路在胳肢他。 
        “你……你要……干吗?”舒克忍住笑,夹紧胳膊,死死握住驾驶杆。 
        “我要你的飞机!”白路更加使劲地胳肢舒克。 
        “别……别闹,飞……机会掉……下去的!”舒克还以为白路同他逗着玩呢。 
        “谁和你闹,我真要你的飞机!”白路腾出另一只手来搔舒克的肚子。 
        舒克万万没想到白路会来这一手。他痒痒得受不住了。舒克松开驾驶杆,和白路搏斗起来。直升机失去了控制。 
        躺在床上的贝塔忽然觉得坦克忽上忽下,他弄不清舒克在搞什么名堂。 
        “舒克,你在干什么?我刚吃了东西,你这样折腾会弄得我消化不良的!”贝塔通过无线电台喊起来。 
        舒克正和白路在直升机上滚作一团。他听见贝塔的声音,急忙对着话筒喊: 
        “贝塔,白路要劫持飞机!” 
        白路在劫持飞机?贝塔傻眼了,他后悔没坐在直升机上。就靠舒克自己,很难打过白路。 
        贝塔急得团团转,直升机就在头顶上,他干着急,上不去。 
        贝塔趴在潜望镜上往外一看,吓出了一身冷汗,前方是一个巨大的烟囱,眼看直升机就要撞到烟囱上了! 
        “舒克!舒克!快拉杆!!!”贝塔大喊一声,接着闭上了眼睛,他知道来不及了。 
        舒克听到贝塔的叫声,知道一定有紧急情况,他顾不上往外边看——也看不成,白路压在他身上。舒克用脚使劲往后一勾驾驶杆,直升机笔直地向天空升去,螺旋桨几乎擦着了烟囱!好险! 
        闭着眼睛等待和烟囱帽撞的贝塔睁开了眼睛,他的坦克服被冷汗湿透了。贝塔决定去支援舒克,可怎么上去昵?   \chapter{第16集}   
        白路企图把贝塔甩下直升机: 
        直升机掉进烟囱; 
        舒克操纵飞机在阳台上着陆   
        贝塔打开坦克舱盖,从坦克里伸出头来,耳边的风很大,呼呼地刮。贝塔把帽子系好。 
        直升机在头顶上轰鸣着。贝塔抬头一看,铁钩子又细又长,要想顺着它爬到直升机上去不容易,弄不好就会掉下去,摔得粉身碎骨。 
        贝塔往下一看,头直发晕,两腿发软。 
        直升机忽上忽下,忽左忽右,随时都有摔下去的危险。 
        “反正也是死!”贝塔一咬牙,钻出坦克。他两手抓紧铁钩子,开始向上爬。 
        往常贝塔根本不把爬桌子、爬柜子放在眼里,可现在每爬一步,贝塔都要使出全身的力气。光是风就可以把他吹走。 
        正当贝塔快要抓往直升机的轮子时,白路发现了贝塔。 
        白路明白,只要贝塔爬上直升机,他劫持飞机的企图就会落空。那时,舒克和贝塔不把他从飞机上扔下去才怪。 
        自路突然松开舒克,扑向驾驶台,他猛烈地摇晃驾驶杆,他想把贝塔甩下去。 
        直升机开始剧烈地晃动,贝塔一下没抓住,松开了手和脚,被抛到空中。 
        幸亏贝塔早有提防,把自己的尾巴拴在铁钩子上当作安全带。 
        贝塔的身体在空中飞舞着,他的尾巴死死地系在铁钩子上。 
        舒克发现了贝塔的危险处境,他扑过去用劲把白路从驾驶台前推开。 
        直升机垂直下降。 
        贝塔忽然觉得眼前一黑,一股呛人的烟味儿刺得他直咳嗽。 
        直升机和坦克掉到大烟囱里了。 
        滚滚的浓烟刺得贝塔两眼流泪,连连咳嗽,直升机越往下,温度越高。 
        白路已经吓傻了,老虎胆在烟囱里也不管用了。 
        烟囱里黑古隆咚,什么也看不见。舒克拉起了驾驶杆,直升机向上升去。舒克知道,直升机随时都有和烟囱相撞的危险,可他一点儿办法也没有,什么也看不见。听天由命吧。 
        奇迹发生了,直升机居然飞出了烟囱。 
        贝塔深深地吸了口气,他用力向上一蹿,抓住了直升机的轮子。 
        贝塔把尾巴从铁钩子上解开,爬上了直升机。 
        白路想把飞机门插死,但他动作慢了一步,舒克已经把机门打开了。贝塔冲进机舱。 
        “我投降!我投降!”白路退到机舱的角落里。 
        舒克和贝塔把他身上的电池取出来,白路倒在地板上。 
        舒克和贝塔紧紧地抱在一起。贝塔全身都被熏黑了。 
        舒克和贝塔反劫机成功。 
        “你真行!”舒克一边操纵飞机一边夸贝塔。 
        “哪儿有你飞行员厉害呀,用脚丫子开飞机!”贝塔听舒克说了倒勾驾驶杆的精彩技艺,十分佩服。 
        “这家伙劲儿真大。”舒克回头看看白路。 
        “找个地方,把他放下吧!”贝塔提议。 
        “这儿正好是发电厂,这里的动物一定懂电。”舒克同意。他寻找着陆地点。 
        直升机开始下降。贝塔的坦克先着陆,直升机随后停在一旁。 
        “糟糕,我的飞机电池不足了。”舒克说。 
        “这是白路身上的电池。”贝塔把电池递给舒克。 
        “那他…”舒克看看躺在机舱里的白路。 
        贝塔也意识到白路如果离开电池,心脏就不会跳动了。 
        “还是给他装上电池吧。这样把他扔出去,一会儿就会被猫吃了的。”舒克说。 
        贝塔把电池给白路装上。白路站起来。 
        “你走吧,这儿就是发电厂。”舒克打开机舱门,对白路说。 
        白路没想到舒克和贝塔这么宽大他,他愣在原地不动。 
        “快去吧!”贝塔催促。 
        “真对不起!”白路冲舒克和贝塔鞠了一躬,跑出机舱。 
        “咱们怎么办?”贝塔问舒克。 
        “来时我看见不远的地方有一座城市,咱们去那儿找电池。”舒克说。 
        “能飞到吗?”贝塔担心电量不够。 
        “行。”舒克发动了飞机,“你就在飞机上吧。” 
        贝塔点点头。他不敢离开直升机了。 
        直升机吊着坦克升到空中,向城市飞去。天渐渐黑了。 
        “白路不会再吓唬人吧!”贝塔说。 
        “发电厂的动物都懂电,谁也不会怕他。”舒克一边驾驶飞机一边说。 
        “你看,前边有那么多灯!”贝塔叫起来。 
        “城市到了。”舒克开始下降高度。 
        直升机飞到了城市的上空。 
        “没电了!”舒克来回摆了摆驾驶杆,直升机不受控制了。 
        “怎么办?”贝塔慌了。 
        “快找个着陆的地方。”舒克注意观察地面。 
        “下边是个阳台。”贝塔告诉舒克。 
        只有在这座大楼的这个阳台上迫降了。 
        舒克的直升机和贝塔的坦克悄无声息地在阳台上着陆了。 
        当舒克和贝塔准备开机舱门时,他们不约而同地打了个寒颤:一只大花猫蹲在阳台上,虎视眈眈地盯着直升机。   \chapter{第17集}   
        小花猫变成了大花猫; 
        舒克和贝塔被皮皮鲁抓获   
        借着月光一看,舒克大吃一惊,这不是以前蜜蜂皇后为他举办宴会时,要处死他的那只小花猫吗!转眼都长这么大了。 
        “糟了,这是我的冤家!”舒克小声告诉贝塔。 
        “快起飞!”贝塔把电池没电的事忘了。 
        “没电。”舒克提醒贝塔。 
        贝塔一屁股坐在皮椅子上。 
        大花猫觉得从天上落下来的这架直升机挺面熟,可一时又想不起来在哪儿见过。舒克的直升机原来是米黄色的,刚才在大烟囱里被熏黑了。 
        大花猫终于想起来了,这是一只名叫舒克的小老鼠的直升机!这只小老鼠化装成飞行员,到处招摇撞骗。 
        “你再为大家办事,也是一只老鼠!”大花猫一边想一边做好了扑上去的准备。 
        “他要向咱们进攻丁。”贝塔眼尖,他发现了大花猫的企图。 
        话音还没落,大花猫已经扑上来,死死抓住直升机,大声叫起来。 
        屋里的灯亮了。接着,阳台门打开了,走出一个男孩子。 
        “这下完了,人最恨咱们老鼠。”贝塔耸耸肩膀。 
        “你不恨我吧?”舒克忽然问贝塔。 
        “恨你?干吗恨你?”贝塔不明白。 
        “是我把你吊到天上,才有今天的。” 
        “当然恨你。恨你干吗把飞机从烟囱里开出来!还不如掉下去呢!” 
        舒克笑了。贝塔也笑了。笑得挺惨。 
        男孩子低头一看,眼睛亮了,一架直升机!后边还有一辆坦克! 
        “哪儿来的?”男孩子往阳台下边看看。12层高的楼,大花猫不可能叼着直升机和坦克爬上来。 
        “自己飞来的!”男孩子激动了,他弯腰拿起直升机和坦克,冲进屋里。 
        他把直升机和坦克放在桌子上,大花猫蹿上了桌子,蹲在旁边,随时准备抓获舒克和贝塔。 
        男孩向飞机里面看,他张大了嘴,半天说不出话来。直升机里有一只穿着飞行服的小老鼠和一只穿着坦克装的小老鼠。是这两只小老鼠驾着直升机到他的阳台上来的。 
        男孩子乐了,他打开直升机的舱门。 
        大花猫一下扑上去,几乎把直升机撞到桌子下边。 
        “于什么!”男孩子火了,“不许你动他们!你要动他俩一根毫毛,我就对你不客气了。” 
        大花猫愣了,怎么,人不许猫抓老鼠! 
        “下去!”男孩子命令。 
        大花猫乖乖地从桌子上跳下去。 
        贝塔和舒克松了一口气,他们很感激这个男孩子。 
        “咱们交个朋友好吗?”男孩子说,“你们叫什么名字?” 
        名字!人问老鼠叫什么名字!舒克和贝塔差点儿流出眼泪来,从前,他俩只知道人管他们统统叫老鼠,没想到这个男孩子这么尊重他俩。 
        “我叫舒克,他叫贝塔。”舒克说。 
        “我叫皮皮鲁,咱们是朋友了。”皮皮鲁兴奋地说,“你们于吗到我家来?” 
        舒克把他和贝塔怎样到克里斯王国,又怎样把白路送到发电厂,以及同白路在空中搏斗,后来又怎样没有电池了等等,统统告诉给皮皮鲁。 
        皮皮鲁听着,眼睛一下不眨,而且越睁越大。 
        舒克得意极了。原来他还以为,人对他们老鼠的生活一点儿也不感兴趣。 
        孩子比大人好。这是舒克和贝塔的共同感觉。 
        “我去给你们找电池!”皮皮鲁说完拉开柜门,从柜子里拿出爸爸的电动刮胡刀,取出里边的电池。又从半导体收音机里拿出电池。 
        舒克和贝塔感激地看着皮皮鲁,他俩觉得要是再不从直升机里出来,就是不相信朋友了。 
        舒克和贝塔走出直升机。皮皮鲁笑了。   \chapter{第18集}   
        舒克和贝塔为皮皮鲁作飞行和坦克表演; 
        皮皮鲁款待舒克和贝塔; 
        舒克和贝塔拨表   
        舒克把电池装进直升机。 
        “你能给我表演一下吗?”皮皮鲁问舒克。 
        “当然可以!”舒克看了看房间,足够他折腾了。 
        “我为你表演开坦克。”贝塔也愿意为朋友效劳。 
        “太好了!”皮皮鲁叫起来。 
        贝塔钻进坦克。舒克登上直升机。 
        螺旋桨转起来了,紧接着,直升机升到空中。 
        舒克大显身手,一会儿绕着电灯飞,一会儿在衣柜上着陆。逗得皮皮鲁哈哈大笑。 
        贝塔正准备也露一手,忽然他觉得坦克晃动起来。原来,舒克把他的坦克吊到空中了。 
        “你干什么?就显摆你啦?”贝塔不高兴了,通过无线电向舒克抗议。 
        “就一会儿,就一会儿!”舒克吊着坦克在屋里只飞了一圈,就把坦克放回到桌子上。 
        “真棒!”皮皮鲁大加赞扬。 
        贝塔也给皮皮鲁表演了几个高难度动作。 
        皮皮鲁快活极了。 
        大花猫蹲在墙角气得要死。 
        表演结束后,皮皮鲁帮助舒克和贝塔把直升机和坦克身上的烟迹擦干净。还给贝塔洗丁澡。 
        最令大花猫不能容忍的是,皮皮鲁竟然把大花猫的饭碗给舒克和贝塔端去,请他俩吃饭。 
        “今晚你们就住在我家吧。”皮皮鲁说。 
        舒克和贝塔商量了一下,同意丁。他俩决定明天晚上飞走。舒克和贝塔有一个心愿,就是想为皮皮鲁干点儿什么。 
        为了安全起见,舒克和贝塔钻进坦克,甜甜地睡了一觉。大花猫无可奈何。 
        第二天早晨,舒克和贝塔发现皮皮鲁小大高兴。 
        “我们能为你干点儿什么吗?”舒克问皮皮鲁。 
        皮皮鲁耸耸肩膀。 
        “你不高兴了?”贝塔问。 
        “该去上学了,你们要是有能让我提前放学的本事就好了。”皮皮鲁背起书包,一步三回头。 
        “咱们帮皮皮鲁一次忙吧?”舒克对贝塔说。 
        “怎么帮呢?咱们又不能改变时间。”贝塔无可奈何地说。 
        “你看见那座大楼上的钟了吗?他们全城的人都以这个钟为标准时间。咱们开着直升机去把表针拨快半圈,皮皮鲁不就能早放学了吗?”舒克说。 
        “真有你的!”贝塔对舒克佩服到家了。 
        舒克和贝塔开始做准备工作。他俩找了一根绳子,一头挽成一个圈套,另一头系在直升机上。 
        快到11点半时,舒克和贝塔驾驶直升机起飞了。 
        “看,就是那座大楼。”舒克一边操纵飞机一边告诉贝塔。 
        “这表真大。”贝塔吐吐舌头。 
        直升机飞到大表跟前。这时,正好11点半,分针垂直向下。 
        “我操纵飞机靠近表,你把绳子套在分针上。”舒克说。 
        “行”。贝塔二话没说,打丌机舱门,他一手抓住扶手,一手把绳子甩出去套表针。 
        舒克和贝塔想得太简单了,在空中用绳子套表针,谈何容易。 
        几十次都失败了。分针又走了五分钟。 
        想到朋友在课堂上盼着下课的难受样子,贝塔决定冒一次险。 
        贝塔把绳子拴在自己腰上,跳出了直升机。贝塔抓住了分针,他死死地抱住。舒克拉起了驾驶杆,直升机向上升去,分针被直升机往上拉了将近半圈,12点了! 
        “当!当!当!”的报时钟声差点儿把贝塔耳朵震聋。 
        全城所有的学枝都提前25分钟放学了。全城所有的人都发现自己的表慢了近半小时。没有人怀疑钟楼的表不准。 
        钟表修理店门口排起了长长的人龙。   \chapter{第19集}   
        舒克和贝塔驾驶直升机参加航模比赛; 
        航模选手们决定击落舒克的直升机   
        当皮皮鲁知道是舒克和贝塔帮他提前放学时,很感谢这两位朋友。 
        “全城的人都去修表了。”皮皮鲁觉得有趣, 
        “咱们痛痛快快玩吧l” 
        可惜好景不长,1点钟提前25分钟到了。皮皮鲁下午要提前去上学了。 
        “我们再把表拨回来。”贝塔提议。 
        “千万别去,被大人们发现,非抓住你们不可。”皮皮鲁把阳台门关上。 
        “那你……”舒克觉得挺对不住朋友。 
        “没关系。早上早下嘛!”皮皮鲁倒想得开。 
        “你干吗不喜欢上学?”贝塔问。 
        “老师不喜欢我,总是看我不顺眼。”皮皮鲁委屈地说。 
        舒克和贝塔同情地看着皮皮鲁。没想到,人群里也有像他们老鼠一样被别人瞧不起的人。 
        “我申请参加航模小组,老师说我学习成绩不好,不批准。唉,明天就要举行全市航模比赛了。”皮皮鲁叹了口气,他非常喜欢航模。 
        “什么叫航模比赛?”舒克觉得航模似乎同飞机有关。 
        “就是飞机模型比赛。”皮皮鲁拉开门,准备去上学。 
        “我明天帮你去参加航模比赛,行吗?”舒克问。 
        皮皮鲁眼睛一亮,要是舒克开着直升机出现在比赛场上,保准把全场都镇了。 
        当天晚上,舒克、贝塔和皮皮鲁做准备工作。听说航模比赛还有空战项目,皮皮鲁特意把自己的两支弹弓枪安装在舒克的直升机上,让贝塔担任射手,并为他提供了充足的石头子弹。 
        第二天上午,全市航模比赛开始了。整座体育场人山人海。皮皮鲁和本校师生坐在观众席上,老师还差点儿不让皮皮鲁来呢! 
        当本校航模队人场时,师生们一阵欢呼。只有皮皮鲁无动于衷。原先,皮皮鲁也极力为本校队员喊“加油”,谁都希望自己的学校光彩,可每次老师都说他是“假招子”。 
        “你要真想给本校争光,考试得100分呀!”这是老师挖苦皮皮鲁的口头禅。后来,皮皮鲁索性无动于衷了。 
        航模比赛开始了。一架架小飞机呼啸着升到空中,开始表演各种飞行动作,它们不断赢得喝彩声。 
        “舒克准备!舒克准备!”皮皮鲁悄悄按书包里的坦克,利用上面的无线电台同舒克联系。 
        舒克和贝塔此时正在直升机里。直升机停在皮皮鲁家的阳台上待命。 
        “明白!”舒克回答。 
        “起飞!”皮皮鲁下令。 
        一架米黄色的直升机出现在体育场上空,它立刻引起了全场观众的注意。 
        裁判员愣了,参加比赛的飞机中没有直升机呀! 
        只见直升机忽阿空中悬停,忽而垂直降落,忽而盘旋,简直就像有人驾驶一样灵活。观众席上爆发出一阵阵富鸣般的掌声和欢呼声。 
        裁判员也不得不连连点头。 
        “擦着观众的头飞!”皮皮鲁发令。 
        “明白!”舒克一压驾驶杆,直升机擦着观众的头绕场一周。 
        观众们先是一惊,紧接着又爆发出一阵掌声。   \chapter{第20集} 
        舒克和贝塔同航模飞机展开了一场真正的空战; 
        皮皮鲁不让贝塔朝本校的飞机开火; 
        大花猫暗算舒克和贝塔   
        所有参加航模比赛的选手都被激怒了,他们立即联合起来,决定在下一个比赛项目中击落这架直升机。 
        当裁判员刚一发出“空战开始”的口令时,几十架航模飞机腾空而起,同时向舒克和贝塔的直升机扑去。 
        “舒克,快撤退!”皮皮鲁见这么多飞机围攻舒克的直升机,慌了。 
        “别撤!咱们得给皮皮鲁争口气!”贝塔不同意撤退。 
        “对,你快准备子弹!”舒克说。 
        一架红头飞机抢先朝直升机冲过来。 
        贝塔把子弹装进弹弓枪,拉满了橡皮筋,瞄准红头飞机。 
        “打!”舒克说。 
        贝塔一勾扳机,石头子弹射了出去。红头飞机被击中了。 
        全场欢呼。直升机上有真炮!能击落对方!孩子们激动了,这是真正的空战。 
        皮皮鲁把手都拍红了。 
        “注意后方!”皮皮鲁提醒舒克。 
        舒克早就注意到后方有一架蓝飞机想偷袭他。这时,前方正好有一架双翼飞机扑过来。 
        就在双翼飞机快要撞上直升机的一瞬间,舒克操纵直升机垂直升起来。 
        双翼飞机和蓝飞机相撞了。 
        连裁判员都击掌叫好。 
        贝塔又接连击落了两架飞机。 
        “你的枪法真准!”舒克夸奖贝塔。 
        “炮手打枪,小意思。”儿塔得意了。 
        这时,空中还有十几架敌机,它们不敢靠近升机,躲在远处盘旋。 
        “咱们进攻一下吧?”贝塔提议。 
        “行。”舒克掉转机头,朝一架白色的飞机冲过去。 
        贝塔瞄准了白飞机。 
        “别打!别打!那是我们学校的飞机!”耳机里传来皮皮鲁急切的声音。 
        “别打!”舒克赶快制止贝塔。 
        “怎么?” 
        “那是皮皮鲁学校的飞机。” 
        “他们学校不是不让他参加航模小组吗?” 
        “谁知道怎么回事!他不让打就别打呗,飞行员得服从地面指挥,懂吗?” 
        “还有哪架不能打,先说!”贝塔不高兴地说。 
        经过一个小时的空战,体育场上空只剩下直升机和皮皮鲁学校的飞机了。 
        全枝师生潮水般地涌向校航模队的运动员们,把他们抬起来,抛向空中。 
        其他观众和裁判员都为那架米黄色的直升机悄然离去感到迷惑不解。 
        舒克和贝塔按照皮皮鲁的命令返航了。他俩一点也不明白皮皮鲁为什么这么做,他们本想址皮皮鲁大大地神气一番。 
        “你看人家,都为本校得冠军高兴,就你无动于衷,一点儿荣誉感也没有!”老师又挖苦皮皮鲁了。 
        皮皮鲁顾不上理老师,他撒腿就往家跑,去感谢舒克和贝塔。 
        皮皮鲁做梦也不会想到,舒克和贝塔已经大祸临头了。 
        舒克和贝塔在阳台上刚一着陆,埋伏在阳台上的大花猫趁皮皮鲁不在家,扑上去抓住直升机。大花猫把直升机连同飞机里的舒克和贝塔塞进准备好的纸箱子里,再把纸箱子封死,推到床下的最里头。 
        当皮皮鲁跑进屋子时,大花猫正趴在桌子上假装睡觉。 
        皮皮鲁跑到阳台上一看,没有直升机。屋里也没有。 
        “看见舒克和贝塔了吗?”皮皮鲁拍拍大花猫。 
        大花猫打个哈欠,摇摇头。 
        皮皮鲁慌了。他站在阳台上往外面看,没有直升机的影子。 
        “又没电池了?被人抓走了?出飞行事故了?”皮皮鲁猜测看。 
        皮皮鲁想起书包里的坦克。他拿出坦克,打开舱盖儿,从里边拿出小话筒。 
        “舒克,舒克!你在哪里?”皮皮鲁呼叫。 
        “我是舒克!我是舒克!我在床底下的纸箱子里!我在床底下的纸箱子里!'. 
        “床底下?纸箱子里?”皮皮鲁莫名其妙。他爬到床下,拉出纸箱子,打开一看,直升机真在里边。 
        大花猫吓傻了,他浑身开始哆嗦起来,他相信皮皮鲁一定饶不了他。 
        皮皮鲁把直升机从纸箱子里拿出来。 
        “你们怎么藏在这儿?”皮皮鲁惊奇地问。 
        “跟你开个玩笑呗!”舒克看见大花猫浑身发抖,不忍心揭发他。 
        “对,开个玩笑。”贝塔点点头。 
        “你们真逗,把我急坏了。”皮皮鲁笑了。 
        大花猫松了一口气,表情挺不自然。 
        皮皮鲁用最丰盛的饭菜款待舒克和贝塔。吃完饭后,舒克和贝塔决定和皮皮鲁告别,他俩觉得待在大花猫身边凶多吉少。皮皮鲁找来几节新电池,送给舒克和贝塔,又赠送给他俩许多食物。 
        “我们以后来看你。”舒克说。 
        “我等着你们。”皮皮鲁舍不得让舒克和贝塔飞走,可他不敢长期留舒克和贝塔。要是让妈妈发现这架来历不明的飞机,她会把飞机交给学校老师的。 
        舒克登上直升机,贝塔钻进坦克。 
        夜色降临了。直升机吊着坦克起飞了。皮皮鲁站在阳台上冲舒克和贝塔招手。 
        一场恶战在野外等待舒克和贝塔。   \chapter{第21集} 
        舒克和贝塔在空中听到紧急呼救声; 
        贝塔大吃一惊; 
        贝塔的坦克和野猫赛跑   
        舒克和贝塔离开皮皮鲁家,朝城外飞去。 
        “贝塔,你在干什么?”舒克一边开飞机一边通过无线电台同坦克里的贝塔。 
        没有回答。 
        “贝塔!贝塔!”舒克以为贝塔出了什么事。 
        贝塔正在坦克里偷偷掉眼泪。他觉得皮皮鲁真可怜,没人理解他。不知怎么搞的,贝塔想起了自己从前在家里时的处境,想起了咪丽欺负他的情景。 
        “贝塔!贝塔!”舒克叫着。 
        “干吗?”贝塔反问舒克。 
        “我以为你被大花猫绑架了呢!”舒克说。 
        “净瞎操心。”贝塔说完打开坦克舱盖,把头伸出来,他觉得坦克里憋得慌。 
        天上挂满了密密麻麻的星星,贝塔一直弄不清这些星星是怎么被人安到天上去的。 
        “大概也是用飞机运上去的吧?”贝塔想。 
        “救命啊——”忽然从地面上传米一阵呼救声, 
        贝塔觉得这声音挺耳熟,他顾不上细想,忙叫舒克: 
        “舒克!舒克!地面有呼救声!地面有呼救声!请你降低高度。” 
        “明白。”舒克操纵直升机下降。 
        呼救声越来越大,借着月光,贝塔看见三只大野猫在咬一只猫。那猫拼命挣扎。 
        “地面上怎么回事?”舒克问。 
        “三只猫在欺负……”贝塔还没说完,舒克就急了:“准备参战!” 
        “三只猫在欺负一只猫!”贝塔把话说完。 
        “猫和猫打架?”舒克操纵直升机悬停在空中,他觉得似乎没必要去干涉猫之间的战斗。 
        贝塔也是这么想。 
        直升机现在离地面很近了,贝塔忽然呆住了:那只喊救命的猫是咪丽! 
        “舒克!舒克!帮帮咪丽吧!”贝塔请求。 
        “咪丽?什么咪丽?”舒克不明白。 
        “就是我原来跟你说过的那个咪丽呀!” 
        “就是曾经欺负你的那只猫?”舒克不信,哪儿有这么多巧事。 
        “就是她!没错。”贝塔肯定地说。 
        “去救她?”舒克觉得贝塔的心眼儿不错。 
        “救她!你把我的坦克放到地面上,你在空中掩护我。”贝塔说。 
        舒克同意了。他一推驾驶杆,直升机迅速下降。贝塔觉得坦克一阵震动,着陆了。 
        舒克用高超的飞行技术摘下钩在坦克上的铁钩子,驾驶飞机升到空中。 
        贝塔好长时问没开坦克打仗了,他的手早痒痒了。贝塔把一发炮弹塞进炮膛。通过潜望镜,贝塔看见三只大野猫正围着咪丽咬。 
        贝塔驾驶着坦克朝三只野猫冲过去。 
        一只大野猫的屁股正对着坦克,贝塔加大速度撞上去,大野猫连打了两个滚儿。 
        另外两只野猫愣了一下,马上朝贝塔的坦克扑上来。他们没把这个小玩艺儿放在眼里——野猫的身体比贝塔的坦克大一倍。 
        贝塔对准其中一只野猫的肚子开炮了,那只野猫挨了炮弹后稍稍停顿了一下,又冲上来。大野猫不怕贝塔的炮弹。 
        贝塔索性一按电钮,坦克迎着野猫开上去。履带压着了一只野猫的脚,疼得他大叫起来。 
        咪丽被这突如其来的变化惊呆了。她眼睛忽然一亮:是贝塔的坦克。 
        三只野猫凑到一起碰了下头,一起朝坦克扑过来。 
        贝塔操纵坦克掉头就跑。一来他想把野猫引开,让咪丽脱离险境;二来他怕这三只野猫把他的坦克翻个底朝天。 
        野猫奔跑的速度非常惊人。贝塔的坦克几乎飞了起来。野猫在后边紧紧跟着坦克,眼看就要追上了。 
        “刹车!”从空中传来舒克的声音。 
        一句话提醒了贝塔。贝塔突然来了个急刹车。三只野猫停不住,冲到前边去了。贝塔掉头往回开。 
        贝塔从潜望镜里看见咪丽还站在原地,一动不动,像傻子一样。 
        野猫又追上来了。   \chapter{第22集} 
        舒克驾驶直升机参战; 
        舒克把野猫吊离战场; 
        贝塔决定和咪丽一起回家   
        贝塔又装上一发炮弹,他掉转坦克,瞄准了为首的那只野猫的脑门。 
        贝塔按下了射击按钮,只见那只野猫大叫一声,蹦得老高。打中了! 
        可野猫毕竟不是麻雀,贝塔的石子炮弹打不伤他们。野猫们被激怒了,他们三个从不同的方向朝坦克扑过来。 
        “请求空中支援!请求空中支援!”贝塔对着话筒喊起来。 
        舒克早已做好了准备,他把两只弹弓枪都压上了子弹。 
        “贝塔,你掉头跑!”舒克说。 
        贝塔操纵坦克掉头就跑,野猫在后边追。 
        舒克驾驶直升机压在野猫的头顶上飞。直升机离野猫只有一尺的距离。 
        舒克一手握驾驶杆,另一只手搂住弹弓枪的扳机。 
        枪口几乎挨着一只野猫的后脑勺。舒克抠动了扳机。 
        那只野猫惨叫一声,在地上连着打了好几个滚儿。 
        另外两只野猫还是死咬住坦克不放。 
        静静的夜晚,在郊外发生着一场激烈的搏斗:一辆坦克在前边跑,两只野猫在后边追,一架直升机压在野猫头顶上飞。说起来也好笑,两只老鼠为一只猫打抱不平。 
        “舒克!用铁钩子钩野猫的耳朵。”贝塔想出一个办法。 
        “太棒了!”舒克忘了发挥铁钩的作用,经贝塔这么一提醒,他觉得铁钩子一定厉害。 
        野猫还在高速奔跑着。舒克的直升机与野猫保持着同等的速度,真是一场立体战争!直升机下边的铁钩子在一只野猫的耳边来回晃动着,要想钩住他的耳朵也不容易,舒克全神贯注地操纵飞机。 
        终于钩上了!直升机加足马力向天上飞去,可野猫太重了,飞机只能把他的两条腿吊离地面。这就够了,野猫被直升机拖着,疼得他大声喊“饶命”。 
        “把他拖远点儿!”贝塔说。 
        舒克驾驶直升机拖着大野猫朝远处飞去。 
        剩下的一只野猫不敢再追贝塔的坦克了,他跑到那只被击中后脑勺的野猫旁边,两只野猫商量了一会儿,溜走了。 
        贝塔驾驶坦克来到咪丽身边,咪丽感激地看着坦克。 
        “谢谢你救了我,贝塔。”咪丽肯定坦克里一定是贝塔。 
        贝塔不敢出来。他牢牢记着眯丽猛然回头咬他一口的教训。 
        “你怎么到野外来了?”贝塔在坦克里问。 
        “你走后不久,主人就把我从家里赶出来了。”眯丽委屈地说。 
        “为什么?”贝塔不明白。他不在了,咪丽应该生活得好呀! 
        “主人说,没有老鼠,养猫也就没用了。”咪丽伤心地说,“原来怪我不好,原谅我吧,贝塔!现在我懂了,没有你,丰人根本不会养我。” 
        贝塔简直不相信这是真的,猫是因为有老鼠才受到人的优待。 
        “我原来不该恨你,应该感谢你才对。”咪丽对着坦克说,“你出来吧,贝塔,我不会咬你了。” 
        贝塔半信半疑地从坦克里伸出头来,他做好了随时钻回去的准备。 
        “刚才他们干吗欺负你?”贝塔看到咪丽浑身是伤。 
        “我好不容易找到一点儿吃的,他们来抢,我不给,他们就咬我,我已经好几天没吃饭了。”咪丽一边说一边掉眼泪。 
        要不是亲眼看见,说什么贝塔也不会相信猫咬自己的同胞时比咬老鼠还狠。 
        贝塔钻回坦克,给咪丽拿出一根香肠。 
        “你吃吧!”贝塔把香肠递给咪丽。 
        咪丽想起从前自己不让贝塔吃饭,惭愧极了。 
        “吃吧!”贝塔又说了一遍。 
        咪丽大口大口吃起来。 
        “你以后怎么办?”贝塔问。 
        咪丽摇摇头。 
        “你在家里过惯了舒服日子,出来真够受罪的。”贝塔说。 
        咪丽哭了。 
        一个念头在贝塔脑子里产生了,他想帮助咪丽。 
        “我帮你再回到主人家怎么样?”贝塔问。 
        “再回到主人家?”咪丽摇摇头,不相信。 
        “我先回去,你在屋外等着。主人一看见我回来了不就又会收养你了吗?”贝塔说。 
        咪丽感动了。 
        咪丽和贝塔就这样决定了。   \chapter{第23集} 
        舒克决定去看妈妈; 
        贝塔和舒克约定一小时通过无线电台联络一次   
        “舒克!舒克!你在哪里?你在哪里?”贝塔呼叫舒克。 
        “我把大野猫扔到河里了,他洗了个澡!我马上回来!”贝塔的耳机里传来舒克兴奋的声音。 
        不一会儿,直升机出现在贝塔和咪丽的头顶上。 
        从空中看到贝塔和咪丽在一起,舒克吓了一跳。 
        “贝塔!注意安全!”舒克提醒贝塔。 
        “你着陆吧,没事。”贝塔说。 
        直升机在坦克旁边着陆了。舒克打开驾驶座旁边的玻璃窗,他不敢下来。 
        贝塔把咪丽介绍给舒克,并把他要帮咪丽回家的决定告诉舒克。 
        “和我们一起去吧!”贝塔说,“就一天。” 
        听说贝塔要回到原来住的地方去,舒克忽然想起了自己的妈妈。自从他开着直升机离开家后,舒克还从未见过妈妈。尽管自己的妈妈有着不光彩的名声,可她毕竟是妈妈。 
        “我想回家去看看妈妈。”舒克说。 
        “这样吧,你去看妈妈,我去帮咪丽回家,咱们随时用无线电台联系,争取明天下午会合,行吗?” 贝塔提议。 
        “好吧,一小时联络一次。”舒克说。 
        朋友要分手了,虽然只有一天,可心里还挺难受。他们互相告诉了地址,再次约定好每小时联络一次。 
        舒克钻进直升机,他冲贝塔和咪丽摆摆手。直升机起飞了。 
        “祝你一路平安!”贝塔说。 
        “祝你顺利!”耳机里传来舒克的祝愿。 
        “咱们走吧!”贝塔对咪丽说。 
        咪丽心里挺不好受,是她把贝塔从家里逼走的。而现在,却是贝塔送她回家。 
        贝塔心里挺得意,一想到没有他,主人就不养咪丽了,贝塔美滋滋的。 
        “贝塔!贝塔!你怎么样了?请回答。”耳机里传出舒克的询问声。 
        “我很好,请放心。你怎么样了?”贝塔问。 
        “我已经接近家了,正在寻找降落的地方。”舒克说。 
        “注意安全。多在空中观察一会儿。”贝塔嘱咐舒克。 
        “一小时后再联系。” 
        “好,一小时后再联系。” 
        从潜望镜里,贝塔已经看见他原来居住的那座房子了。 
        贝塔把坦克停在咪丽身旁,打开舱盖,探出半个身子。 
        “你在这儿等着。昕到主人在里边喊叫后,你就进去,保准主人对你好。”贝塔对眯丽说。 
        “你不会有危险吧?”眯丽有点儿替贝塔担心。 
        “没事儿!”贝塔钻回坦克,把舱盖锁牢。 
        坦克从咪丽出入的小门驶进了屋子。   \chapter{第24集} 
        贝塔大闹一场; 
        咪丽受到主人热情的欢迎; 
        贝塔教咪丽学老鼠叫; 
        舒克贝塔失去联系   
        屋里黑咕隆咚,主人睡觉了。 
        贝塔把坦克丹到床底下隐蔽好,他悄悄从坦克里钻出来。 
        这里的一切对贝塔来说太熟悉了。衣柜,写字台,电视机……几乎一点儿变化也没有。贝塔想起了他从前的生活。 
        贝塔走进他原来居住的洞里,他觉得这洞又黑又小,他感到奇怪,从前他住在这儿怎么一点也没觉得小。脚下一个东西差点儿绊了贝塔一个跟头,他低头一看,是他从前用来装香味的布口袋。 
        贝塔想起了自己昔日饿肚子时的难受劲儿,他可怜自己。 
        贝塔想起咪丽还在屋外等着,他准备行动了。 
        贝塔钻出洞,爬上食品柜。 
        食品柜上放着一盘油炸花生米,贝塔不客气地大吃起来,还故意把花生米撤了一地。 
        主人睡得挺香。 
        贝塔把一个铁缸子从柜子上推下去。 
        咣当! 
        主人被吵醒了。 
        贝塔趁机大叫起来。 
        主人打开电灯,看见了食品柜上的贝塔。 
        “老鼠!”主人一惊,掀开被子朝食品柜扑过来, 
        “抓住它!” 
        贝塔一溜烟儿钻到床底下。 
        “谁让你把咪丽轰跑了,看,老鼠又回来了吧!” 
        “这……” 
        “老鼠把花生米撤了一地!” 
        主人家里吵翻了天。 
        正在全家手足无措时,咪丽像天使般出现在主人面前。 
        “咪丽!”主人兴奋得大叫起来。 
        “咪丽回来了!” 
        “咪丽回来了!”主人全家一片欢呼。 
        “快去给咪丽拿吃的!” 
        “快去给咪丽洗澡!” 
        咪丽受到了最隆重的接待。 
        贝塔在床卜看着这一刹,心里有点儿那个。他也想象咪丽这样受到人们的欢迎,贝塔明白这是做梦。不过他很清楚,是他使咪丽受到这样隆重的欢迎的。然而他却扮演着一个不光彩的角色,用来换取咪丽在主人面前大放光彩的位置。 
        咪丽要感谢贝塔,她钻进床底下。 
        “咪丽去抓老鼠了!”主人喊道。 
        这话吓了贝塔一跳。他慌忙钻进坦克,锁好舱盖。准能保证咪丽不是来抓贝塔向主人献殷勤的呢? 
        “贝塔!贝塔!”咪丽站在坦克旁边叫。 
        “干吗?”贝塔问。 
        “谢谢你!”咪丽感激地说。 
        “……”贝塔投说话。 
        “你出来呀,我给你带来好吃的了。”咪丽说。 
        贝塔越想越不是滋味:咪丽为什么可以光明正大地受人宠爱,而他贝塔却要躲在这阴暗的床下。尤其使贝塔生气的是,咪丽还是打着抓他的旗号钻进床底下来的。 
        咪丽明白贝塔为什么伤心了,她哭着说:“贝塔你别伤心。我真想和你换换,让你当猫,我当老鼠。是你帮我回来的你却只能藏在床下挨骂。刚刚听主人骂你,我心里真难过。咱们走吧,贝塔,我宁愿去野外流浪。” 
        坦克舱盖打开了,贝塔钻出来。 
        “别哭了。净说傻话,这儿过得多舒服!我一点儿也不伤心,只要你不挨饿就行了。”贝塔一边说一边抹眼睛。 
        咪丽给贝塔食物。 
        “我明天就走了,你在这儿好好过吧。’贝塔告诉咪丽。 
        ‘你一走主人又该轰我了,”咪丽说。 
        贝塔觉得咪丽的话有道理。 
        “你就留在这儿吧,每天有吃有喝。”咪丽提议。 
        “那可不行,我得和舒克在一起。”贝塔不干。 
        咪丽挺惭愧,她觉得贝塔对朋友讲义气。 
        贝塔忽然想出了一个主意。 
        “咪丽,我想出一个办法,不过你别嫌脏。” 
        “什么办法?” 
        “我给你留点儿我的屎,就是耗子屎。你每天拿一点儿撒在主人的饭桌上,主人肯定就不会轰你走了。” 
        咪丽一想,这办法不错。 
        “可要是用光了呢?你每隔两天能给我送一次吗?”咪丽不放心。 
        “两天送一次?这我可做不到。”贝塔吐吐舌头。 
        “那老鼠屎用完了以后主人又该轰我了。”咪丽发愁。 
        贝塔眼睛一亮,对咪丽说:  “干脆我教你学老鼠叫吧!你学会了老鼠叫,每天晚上叫一会儿,主人听到这种声音就不会轰你了,天天还得给你好吃的。” 
        咪丽觉得这办法好。 
        “来,现在就教。”贝塔当老师。 
        “吱——”贝塔作示范。 
        “喵——”咪丽跟着发音。 
        “不是喵,是吱——” 
        “吃——” 
        “也不是吃,是吱——,你注意看我的口型。” 
        贝塔把牙露出来,嘴角向后撇。 
        咪丽模仿贝塔的口型。 
        “气——” 
        “不对,不对,是吱——” 
        “……”咪丽不敢发音了。 
        “别灰心,要想生活得好就得下功夫。你看主人学外语时不是也很费事吗?来,再试试,吱——” 
        “次——” 
        “好,快了!吱——” 
        “次——吱——” 
        “对!就这样!再来一遍。” 
        “吱——吱——吱——” 
        咪丽学会了老鼠叫。贝塔走后,主人还会继续宠爱她。 
        为了保险起见,贝塔决定让咪丽演习一次。 
        咪丽在床底下连续发出“吱——”的叫声,同时用爪子抓纸箱子。 
        主人被吵醒了,他打开灯。 
        “吱——吱——” 
        “喵——喵——” 
        “吱——吱——” 
        “喵——喵——” 
        咪丽一会儿学老鼠叫,一会儿发出猫叫的声音,同时在床下乱踢乱抓,好像床下正发生着一场猫鼠之问的恶战。 
        “你们快听,咪丽抓老鼠呢!”主人对家里人说。 
        床下越打越热闹。床上的主人高兴得止不住笑。衣柜上的贝塔手舞足蹈。 
        老鼠叫声没有了。猫叫声继续着。 
        “抓住了!抓住了!”主人兴奋地跳下床,趴在地上往床底下看。 
        咪丽一边抹着嘴一边从床底下钻出来。 
        主人喜爱地拍拍咪丽的头。 
        “这么快就把该死的老鼠吃了!”主人夸奖咪丽。 
        贝塔心里又有点儿不是滋味。但一会儿就过去了。 
        主人睡觉后,贝塔从衣柜上下来,钻进床底下。 
        “像真的一样!”贝塔认为咪丽可以毕业了。 
        “对小起,足你帮助了我,向主人每次夸奖我时都要骂你,真对不起。”咪丽心里很难过。 
        “投关系,我不怕骂。”贝塔安慰咪丽。 
        贝塔忽然一拍脑袋:“哎呀,该和舒克联系了!” 
        贝塔钻进坦克,戴上耳机。 
        “舒克!舒克!我是贝塔!请你回答!” 
        “……” 
        “舒克!舒克!我是贝塔!请你回答!” 
        “……” 
        “舒克!舒克!我是贝塔!……” 
        “……” 
        “糟糕,舒克一定出事了!”贝塔钻出坦克,焦急地对咪丽说,“我得马上走。再见了,咪丽。” 
        “我和你一起去!”咪丽连想都没想就说。和贝塔相处时间虽然不长,可咪丽的身上已经起了变化。 
        “你?不在这儿过舒服日子了?”贝塔问。 
        “说不定我能帮你们忙呢!”咪丽说。 
        贝塔觉得有一只猫跟着他的确安全些,就同意了。 
        咪丽和贝塔迅速离开屋子,用最快的速度朝舒克家奔去。   \chapter{第25集} 
        舒克的直升机在房项上着陆; 
        舒克在窗台上碰见蓝鹦鹉和绿鹦鹉; 
        绿鹦鹉和蓝鹦鹉反对舒克看妈妈; 
        舒克遇险   
        舒克和贝塔分手后,驾驶着直升机去看妈妈。 
        舒克飞到了自己熟悉的地方。他看见了和贝塔打仗的地方,看见了蜜蜂皇后宴请他的地方,还有小麻雀的家。 
        舒克很想见他们。自从他把贝塔的坦克吊走后,还一直没回来过,小麻雀他们一定急坏了。 
        舒克决定还是先去看妈妈。他已经看见了妈妈住的那座房子。直升机朝房子飞去。 
        为了安全起见,舒克把直升机停在房顶上。他把一根绳子拴在飞机上,另一头扔下来,绳子正好经过窗户。 
        舒克顺着绳子溜下来,落到窗台上。 
        窗户没插。舒克悄悄钻进屋里。 
        舒克借着月光一看,屋里变化挺大,床和桌子都挪了位置。舒克从窗台跳到桌子上。 
        “这不是舒克吗?”黑暗里传来一个声音。 
        舒克顺着声音传来的方同看去,是鸟笼里的蓝鹦鹉和绿鹦鹉。 
        “你们好!”舒克问候。 
        “你好!”蓝鹦鹉热情地说,“我们听说你现在变得可好了,净帮助别人。” 
        “听小麻雀说,你救过他的命。”绿鹦鹉说。 
        “应该做的。”舒克不好意思了。 
        “你来干什么?”蓝鹦鹉好奇地问。 
        “我来看妈妈。”舒克说。 
        “你妈妈还偷东西哪!”蓝鹦鹉提醒舒克。 
        “你不应该看她!有这样的妈妈真丢人!”绿鹦鹉说。 
        “可……她……是我的妈妈……”舒克说。他觉得妈妈就是妈妈,偷东西和不偷东西是另一回事。 
        绿鹦鹉和蓝鹦鹉开始撇嘴了,接着他俩又交头接耳地嘀咕起来。 
        “再见。”舒克说完从桌子上爬下来,朝自己家走去。 
        舒克的家没有变化,洞口还是老样子。 
        舒克趴在洞口昕听,里面没动静。他蹑手蹑脚地钻进去,生怕吓着妈妈。 
        一钻进洞里,舒克立刻想起了自己的童年,想起妈妈每天从外边带吃的回来喂他的情景。 
        “谁?”黑暗中传来颤抖的声音。 
        舒克定了定神,走过去一看,角落里躺着一只年迈的老鼠,正是他的妈妈。 
        “妈妈,我是舒克!”舒克简直不相信这是自己的妈妈,她老了,牙齿都快掉光了。 
        “舒克?舒克!”妈妈惊讶地欠起了身子,一把抓住舒克的胳膊,又躺下了。 
        “妈妈,你病了?”舒克问。 
        “我老了,不行了,好几天没吃上东西了。”妈妈有气无力地说,“听说你在外边混了个好名声,妈妈也就放心了,千万要保住这个好名声。妈妈知道,老鼠混个好名声不容易。” 
        望着饿得有气无力的妈妈,舒克忽然恨起自己来:为了自己出去混个好名声,把年迈的妈妈扔在家里不管。好名声到手了,可良心到哪儿去了?没有良心的好名声能算好名声吗? 
        “妈妈,我对不起你!是你把我养大的,我却……”舒克哭了。 
        “别这样说,你快走吧!妈妈能见上你一面,也就放心了。记住,保住好名声,保住好名声啊!”妈妈推开舒克的胳膊,闭上眼睛。 
        舒克恨死名声这个东西了。为了得到好名声,他抛弃了生他养他的妈妈,可谁也没有谴责过他,就因为他妈妈是老鼠!舒克真可怜自己的妈妈,她应该和猫的妈妈享有样的做母亲的权利呀! 
        舒克擦干眼泪,他决定留在妈妈身边,伺候妈妈。什么名声不名声,去他的吧!丧失良心的名声再好,舒克也不稀罕了。 
        舒克钻出洞,给妈妈找吃的。妈妈已经饿得昏过去了。 
        他来到食品柜旁边,食品柜锁着。舒克发现食品柜上放着一只碗。 
        他爬上食品柜,碗里是香肠。舒克拿了一根香肠,回到地上。他还没站稳,就觉得背后刮起一阵疾风,紧接着,舒克的肩膀被死死地抓住了。 
        舒克回头一看,是大花猫!皮皮鲁家的大花猫!   \chapter{第26集} 
        舒克又见到小麻雀他们; 
        朋友不理解舒克; 
        大花猫准备处决舒克; 
        咪丽和哥哥重逢   
        原来,自从舒克和贝塔离开皮皮鲁家后,大花猫越想越生气,他不但不感谢舒克和贝塔“包庇”了他一次,反而更恨舒克了。他悄悄离开皮皮鲁家,来到舒克家附近潜伏着,他下决心一定要抓住舒克。 
        果然,舒克开着直升机回来了。大花猫等舒克从窗户钻进屋子后,他也跟着钻了进去。 
        “你还有什么说的?装成飞行员,到处招摇撞骗,实际上是小偷!”大花猫冷笑了一下,死死抓住舒克不放。 
        舒克觉得肩膀像火烧一样疼,他请求大花猫:“让我把香肠给妈妈送去行吗?你别松开我,我把香肠塞进洞里就行,妈妈快饿死了。” 
        “我让你把香肠送给你的老鼠妈妈?老鼠也配当妈妈?笑话!这根香肠正是你的罪证!”大花猫不同意。 
        “咱们从前听到的舒克变好了的消息都是假的。”蓝鹦鹉对绿鹦鹉说。 
        ‘就是,他这么留恋他的妈妈,真不像话。”绿鹦鹉说。 
        一想到妈妈在家里饿得昏丁过去,听着刚才这些侮辱妈妈的话,舒克闭上了眼睛。 
        “走,去见见小麻雀他们,让大家认识认识你的真面目!”大花猫拎起舒克,拿着他的罪证香肠,从窗户跳了出去。 
        天,渐渐亮了。 
        大花猫押着舒克来到小树林里,这里的一草一木舒克都非常熟悉。 
        “大家快来看!我抓住了一个小偷!”大花猫扯着嗓子喊。 
        小麻雀飞来了。小蜜蜂飞来了。蚂蚁们跑来了。 
        “舒克!”朋友们惊喜地喊叫起来,自从舒克开直升机把贝塔的坦克吊走后,他们一直在找舒克。 
        “你干什么?”小麻雀生气地质问大花猫。 
        “他是小偷!”大花猫说完用力压丁压舒克的肩膀,舒克差点儿趴在地上。 
        “你胡说!”小麻雀火了。 
        “你放开他!”小蜜蜂飞到大花猫头上,准备蜇他。 
        “让他自己说,他是不是小偷?这香肠就是他偷的!”大花猫把香肠往大家面前一扔。 
        “舒克,这不是你偷的!”小麻雀说。 
        “是我偷的。”舒克说。 
        小麻雀们都愣了。 
        “不,不是你偷的!”小麻雀急了,他不相信舒克会偷东西。 
        “是我偷的。”舒克义重复了一遍。 
        舒克现在什么也不怕了。名声,面子,他统统不去想。舒克现在惟一惦记的是他的妈妈在挨饿。妈妈快饿死了,而食品柜上放着吃的,为什么不能去拿呢?管这叫偷也好,叫拿也好,反正舒克不能看着妈妈饿死。 
        “舒克,你干吗要去偷吃的呢?”小麻雀还是不信。 
        “我妈妈快饿死了。”舒克说。 
        “你妈妈!”小麻雀愣了一下,他头一次想到舒克的妈妈,一只老鼠。听见舒克管老鼠叫妈妈,小麻雀有点儿不习惯。 
        小蜜蜂不明白舒克为什么留恋一个不光彩的妈妈。 
        “我现在处决他!”大花猫看到大家认出了舒克的真面目,得意极了,他拎起舒克,朝草丛里走去。 
        小麻雀忍不住飞过去,但他又落在树枝上了。他心里很难过,舒克干吗要为那样一个妈妈而偷东西呢? 
        大花猫把舒克拖进草丛,正准备动手。忽然一颗石子炮弹打在他后脑勺上。 
        大花猫大叫一声。 
        一辆电动坦克朝大花猫撞过来。 
        大花猫定定神,认出是贝塔的坦克。他松开舒克,准备朝坦克扑过去。 
        “哥哥!”坦克后边传来一个熟悉的声音。 
        大花猫一看,是他分别已久的妹妹咪丽。 
        “咪丽!”大花猫顾不上坦克了,他跑到咪丽身边。 
        “咪丽,你从哪儿来?”大花猫激动得喘不过气来,自从小时候和妹妹分离后,他几乎天天在想妹妹。 
        “我来救舒克。”咪丽没想到抓舒克的是她的哥哥。 
        “救舒克?”大花猫吃了一惊。 
        咪丽点点头。 
        “舒克是老鼠!”大花猫把老鼠两个字说得特别重。 
        “我知道舒克是老鼠。”咪丽说。 
        “那你?”大花猫退后一步,仔细打量着自己的妹妹。 
        咪丽把舒克和贝塔怎么救她,主人因为没有老鼠就把她轰出来了以及贝塔怎么帮她回家等等都讲给哥哥听。 
        大花猫听着听着头慢慢地垂下来了。“没有老鼠,人不会养猫”,他觉得咪丽这话挺有道理。再说,舒克还救过妹妹的命! 
        大花猫走到舒克身边,给舒克拍拍身上的土,什么话也没说。 
        舒克扭头就跑。 
        “你去干吗?”贝塔急了。 
        “我去给妈妈送吃的。”舒克头也不回。 
        “我跟你去!”贝塔跳出坦克,跟着舒克跑。 
        小麻雀和小蜜蜂飞过来。 
        “怎么啦?”小麻雀问大花猫。 
        “舒克是我们的朋友,你们不该这样!”咪丽说。 
        “他……他妈妈……”小麻雀结结巴巴。 
        “舒克的妈妈也是妈妈,她也有生存的权利,就你的妈妈是妈妈!”咪丽不客气地训斥小麻雀,“人家还救过你的命呢!关键时刻不够朋友!” 
        “我……”看到猫都为舒克辩护,小麻雀惭愧了。就是,舒克心疼自己的妈妈,有什么不好呢?一个连妈妈都不爱的人,怎么会爱大家呢? 
        小蜜蜂脸也红了。 
        “咱们去给舒克道歉。”小麻雀提议。 
        舒克的妈妈醒过来了。舒克和贝塔把妈妈从家里抬出来,朋友们要慰问她。 
        看到舒克的妈妈骨瘦如柴的样子,大家心里都挺难受,都对舒克不满了,他怎么早没想起照看自己的妈妈呢! 
        “老鼠妈妈,您受苦了。”小麻雀说。 
        “老鼠妈妈,我们对不起您。”小蜜蜂说。 
        大花猫给舒克的妈妈找来许多吃的。 
        听到这么亲切的话,受到这样的尊敬,舒克的妈妈感到承受不了。她一辈子都是在歧视和侮辱中度过的。 
        “我也想体体面面地过日子,我也恨自己干吗是一只老鼠。我生舒克时也像你们的妈妈一样受罪,可为什么我的儿子只有抛弃了我才能混到好名声呢!”舒克的妈妈哭了。 
        从来不掉泪的大花猫也哭了。 
        经过商量,大家决定今后由咪丽把舒克的妈妈带回主人家抚养。这样一举两得。 
        舒克和贝塔决定成立舒克贝塔航空公司,为朋友们服务。   \chapter{第27集} 
        舒克贝塔航空公司成立; 
        首航运送客人遇故障   
        飞行员舒克和坦克兵贝塔决定成立舒克贝塔航空公司,他俩在一条小河旁选择了一块平地作为机场,朋友们都来帮助舒克和贝塔建造机场。蚂蚁挖地基,小麻雀运木料,咪丽盖塔台和候机室,小蜜蜂为餐厅准备食物。 
        经过一个月的努力,机场终于竣工了。漂亮的塔台矗立在停机坪和跑道之间,宽敞明亮的候机室四周绿草如茵。舒克的直升机停在停机坪上,机身在阳光的照射下闪闪发光。 
        舒克担任飞行员,贝塔担任地面指挥,航空公司还缺机械师和空中小姐,舒克和贝塔决定招聘机场工作人员。 
        招聘广告贴出后,有70多只小老鼠前来报名。 
        他们早就听说舒克和贝塔的人名,很是羡慕,也想摆脱“小偷”的坏名声。 
        舒克和贝塔坐在办公室里,他们见到这么多同胞,挺高兴,他们想帮助所有的老鼠同胞改邪归正,靠自己的劳动生活。 
        “咱们好像不需要这么多工作人员吧?”贝塔问舒克。 
        舒克感到为难,可他不忍心拒绝同胞。 
        “咱们算算。”舒克同贝塔商议,  “机务人员五个,空中小姐四个,扫跑道的八个,餐厅厨师两个……” 
        七算八算,70多名老鼠都收下了。经过智力测验,舒克和贝塔给他们分了工,有的当地勤机务人员、有的当空中小姐,有的当货运员,还有清洁工、广播员、警卫… 
        舒克为机务人员举办了训练班,教给他们怎么修理和维护直升机。五名机务人员中,舒克挑选了一个叫臭球的小老鼠担任机械师,全面负责直升机的维修。舒克觉得臭球脑子灵活,一点就通。 
        经过一个月的准备,这天,舒克贝塔航空公司正式营业。首航运送七只小松鼠去远方探亲。 
        机场上一派繁忙景象:餐厅往飞机上运送饮料,机务人员最后一次检查飞机,空中小姐招呼旅客登机,贝塔手持话筒站在塔台上,舒克钻进飞机驾驶舱。 
        朋友们都赶来祝贺舒克贝塔航空公司首航。咪丽和哥哥挥舞着彩旗,蚂蚁王和蜜蜂皇后送来精美的食品,小麻雀要为直升机护航。 
        “起飞!”贝塔发令。 
        直升机的螺旋桨慢慢旋转起来,它越转越快,逐渐形成一股巨大的魔力,把直升机拉上了天空。 
        “请各位系好安全带。”空中小姐关照旅客。 
        小松鼠们还是头一次坐飞机,他们感到新鲜和有趣,有一只老松鼠有点儿害怕,他问空中小姐会不会有危险。 
        “没问题,绝对安全!是我亲自检查的飞机。”臭球机械师走过来说。 
        松鼠点点头,放心了。 
        直升机朝目的地飞去。 
        舒克不时同贝塔保持联系。 
        “请报告飞行状态。”贝塔问。 
        “一切正常。”舒克回答。 
        就在这时,舒克忽然看见仪表盘上的发动机转速指示表的指针来回抖动,紧跟着,飞机开始急剧下降。 
        “不好,发动机停车!”舒克大喊一声。 
        “快采取措施迫降!”贝塔急了。 
        机舱里乱成一团,旅客们吓得面如土色,他们很清楚和飞机一同掉到地上是什么后果。空中小姐们也慌了,尽管她们已乘坐过几次飞机作适应性训练,可万万擞想到头一次飞航班就遇上了空难。空中机械师臭球后悔没带降落伞。 
        舒克打开了应急开关。想重新起动发动机,无效。飞机像秤砣一样往下掉。 
        “让飞机挂在树上!”舒克仗着自己经验多,大胆地操纵飞机朝一棵大树降下去。舒克明白,要是头一次飞行就出大事故,那以后谁也不敢乘坐舒克贝塔航空公司的飞机了。舒克要不惜一切代价保住公司的声誉。 
        飞机飞速下坠……   \chapter{第28集} 
        直升机脱险; 
        臭球机械师被停职; 
        首航成功   
        直升机迅速下坠,机舱里一片混乱。 
        飞行员舒克沉着地操纵飞机朝一棵大树降下去,飞机奇迹般地挂在了树枝上。 
        松鼠旅客们拥到机舱门口,拼命想往出挤,可谁也打不开门。 
        “别挤!别挤!”空中小姐喊道。她打开舱门,让松鼠们离开飞机。 
        松鼠们站到树枝上,都松了一口气。 
        “我以后再也不坐飞机了。”一只松鼠说。 
        “真可怕!”另一只松鼠说。 
        “谢天谢地,舒克的驾驶技术真不错。”又一只松鼠说。 
        “咱们怎么回家呀?”一只松鼠望着这陌生的地方,说。 
        这时,舒克在机舱里把臭球机械帅和空中小姐召集到一起,说:“咱们抓紧时间排除发动机故障,然后把旅客送到目的地。” 
        “飞机挂在树上,怎么排除故障?”臭球机械师说,“这飞机太老了,该换新的了。” 
        “别说挂在树上,我在烟筒里还开过飞机呢!”舒克瞪了臭球一眼。 
        臭球不吭气了,他知道舒克的厉害。 
        “你去照看旅客。”舒克对空中小姐说。“让他们放心,飞机一会儿就能修好。” 
        空中小姐离开机舱。 
        这时,话筒响了: 
        “舒克!舒克!我是贝塔,我是贝塔!请回答!” 
        舒克戴上耳机。 
        “我是舒克!我是舒克!请讲。” 
        “飞机现在何处?” 
        “发动机停车,我把飞机迫降在一棵大树上。乘员无伤亡。正准备排除故障。”舒克说。 
        “随时报告情况,祝你顺利。” 
        舒克摘下耳机,同臭球机械师一起检查发动机。 
        “起飞前你检查过发动机吗?”舒克一边开发动机的盖一边问臭球机械师。 
        “检查了好几遍。”臭球毫不含糊地说。 
        舒克从发动机里找出一把改锥。 
        “这工具是你丢在发动机里吧?”舒克的脸一沉。 
        臭球机械师傻眼了。的确,这改锥是他的,干活时马虎,留在发动机里了。 
        “胡闹!差点儿把大家的命都送了!”舒克火了,训斥臭球机械师。 
        “我……”臭球机械师无言以对。他总觉得自己聪明,是从几十位同胞中选拔出来当机械师的,没想到头一次出航就栽了。 
        “不让你当机械师了,回机场后,扫跑道去!”舒克解除了臭球的机械师职务。 
        臭球没意见,谁让自己粗心大意呢!他赶紧帮助舒克更换损坏的发动机零件。 
        经过一个小时的工作,发动机修好了。 
        直升机得从树枝上起飞,这很危险。舒克让臭球离开飞机。臭球不干。 
        “为什么不离开?”舒克问。 
        “要死一块死。”臭球说。 
        舒克心说,别看这小子粗心,还挺仗义,就同意了。他让臭球观察四周的情况。 
        空中小姐把旅客都领到地面上。 
        舒克按下了启动按钮。螺旋桨转起来了。树叶纷纷落下,树枝剧烈地摇晃着。 
        直升机挣脱了树枝的束缚,升到了空中。舒克驾驶飞机在空中转了一圈儿,平稳地落在地上。 
        臭球打开舱门,招呼旅客上飞机。可松鼠们谁也不敢上,都怕再出事。 
        “请你们相信我。”舒克对旅客说。 
        松鼠们望着舒克真诚的目光,他们感到必须信任舒克,必须信任这种目光。 
        乘客们登上了飞机。空中小姐关好舱门。 
        舒克戴上耳机,向贝塔请示:“飞机故障已排除,请求起飞。” 
        耳机里传来贝塔的声音: 
        “同意起飞!” 
        直升机稳健地升到空中,朝目的地飞去。 
        空中小姐给乘客端来了汽水。松鼠们一边喝饮料一边观看窗外的景色。 
        经过一个多小时的飞行,直升机终于到达了目的地。松鼠们的亲戚早就等在那里了。 
        舒克、臭球和空中小姐帮助旅客把行李搬下飞机。旅客依次紧紧握着舒克的手,感谢他临危镇静,保证了大家的安全。 
        舒克为航空公司赢得了荣誉和信任。 
        臭球惭愧地低下了头。 
        当直升机降落在舒克贝塔航空公司的机场时,受到机场全体工作人员的热烈欢迎,大家像迎接凯旋的勇上一样迎接舒克。 
        “我去扫跑道。”臭球小声对舒克和贝塔说。 
        “我看还让他当机械师吧!”舒克说。他相信,臭球不会再马虎了,他是聪明人,不会在同一件事上犯两次错误。 
        贝塔同意了。 
        臭球机械师乐了。   \chapter{第29集} 
        冰淇淋和牛奶; 
        舒克去奶牛场; 
        海盗袭击舒克和臭球   
        舒克贝塔航空公司白开航以来,十分繁忙。机场每天都是热闹非凡,旅客进进出出,飞机时起时落。 
        候机大厅里坐着许多等候登机的旅客。天气炎热,旅客们感到口渴。 
        一位刺猬旅客找到贝塔。 
        “我提个建议。”刺猬说。 
        “欢迎。”贝塔请刺猬坐在沙发上。 
        “候机大厅应该开设一个冷饮部。”刺猬说。 
        “这个建议很好。”贝塔同意了。 
        贝塔拨通了机场餐厅的电话。 
        “喂,是罗丘吗?”贝塔问。罗丘是餐厅主任。 
        “我是。”罗丘说。 
        “我是贝塔。你会制作冰淇淋吗?” 
        “冰淇淋?没做过。” 
        “候机大厅要开设一个冷饮部,这事交给你办,快点儿试试做冰淇淋或雪糕什么的。” 
        “是。” 
        餐厅主任罗丘放下电话,把手下的人召集到一起。 
        “谁会制作冰淇淋?”罗丘问。 
        “我吃过,真好吃。”一只小老鼠抹抹嘴。 
        “味很好,特甜。”端盘子的老鼠姑娘说。 
        “我不是问好吃不好吃,是问谁会做。咱们要开一个冷饮部。”罗丘说。 
        “听说冰淇淋要放牛奶。” 
        “还有(又鸟)蛋。” 
        “还得有冰箱才行。” 
        罗丘拿起电话听筒。 
        “是贝塔吗?做冰淇淋需要牛奶和(又鸟)蛋,可我们没有牛奶,也没有(又鸟)蛋。”罗丘说。 
        “我同舒克联系一下,让他去搞。”贝塔说。 
        贝塔放下电话听筒,问导航员:“舒克现在在哪里?” 
        “在飞往黑山寨的途中。” 
        “我和他通电话。” 
        导航员要通舒克。 
        “舒克,舒克,我是贝塔。” 
        “我是舒克,请讲。” 
        “咱们的候机大厅要增设一个冷饮部,需要牛奶和(又鸟)蛋,你能不能设法弄一些来?” 
        “行,我想想办法。” 
        “祝平安!” 
        “谢谢。” 
        舒克一边开飞机一边把臭球机械师叫到驾驶舱来。 
        “你知道哪儿有奶牛场吗?”舒克问。 
        “干吗?”臭球不明白。 
        “咱们的机场要设冷饮部.需要牛奶和(又鸟)蛋。”舒克调整了一下飞机的方向。 
        “做冰淇淋用?”臭球挺精通。 
        “对。”舒克点点头。 
        “我知道奶牛场在哪儿。”臭球朝地面望去。“翻过前边那座山,山脚下有座奶牛场。” 
        “咱们先把旅客送到目的地,再去奶牛场。”舒克说。 
        直升机穿过白云,穿过蓝天。 
        送完旅客,舒克驾驶直升机朝奶牛场飞去。 
        “就是那座山。”站在舒克身边的臭球机械师给舒克指路。 
        直升机飞临山旁,在奶牛场上空盘旋。 
        “你看,有多少奶牛!”臭球机械师指指下边,“那些铁桶里都是牛奶。” 
        “着陆。”舒克一推驾驶杆,直升机笔直地下降。 
        “注意观察地面!”舒克吩咐臭球。 
        臭球把脸贴在窗玻璃上,往下看。 
        “就在这座房子后边的草丛里着陆。”臭球对这一带还挺熟悉。 
        舒克操纵直升机平稳地降落在草丛里。 
        “我去弄牛奶。”臭球边说边离开驾驶舱。 
        “怎么弄?”舒克叫住了臭球。 
        “拿呀!”臭球机械师说。 
        “不行。那叫偷。”舒克皱了皱眉头。 
        “那你说怎么办?”臭球机械师一摊手。 
        “去跟奶牛要。”舒克说。 
        “老鼠跟奶牛要牛奶?笑话,人家才不会给呢!”臭球机械师觉得舒克太天真。 
        “咱们一起去。”舒克说完把空中小姐叫过来,“你看守飞机,把舱门从里边锁好,除了我们俩,谁来也别开门。” 
        空中小姐点点头。 
        臭球机械师从货舱里找了两个小桶,然后和舒克下了飞机。 
        他们沿着墙角往牛栏走。 
        “当心点儿,屋里有人。”臭球机械师提醒舒克。 
        舒克蹑手蹑脚地朝牛栏走去,臭球机械师同他保持着距离。 
        一头小奶牛先发现了舒克,她忙告诉妈妈。 
        “妈妈,老鼠又来了!”小奶牛叫道。 
        奶牛们顿时警惕起来,她们恨老鼠。老鼠经常来偷喝牛奶。 
        “你们好!”舒克站在牛栏外面说。 
        “还假装有礼貌呢!”一头奶牛撇撇嘴。 
        “黄鼠狼给(又鸟)拜年,没安好心。”另一头奶牛说。 
        “你们误会了,我是飞行员舒克,是舒克贝塔航空公司的飞行员,不是小偷。”舒克说。 
        “老鼠能当飞行员?”小奶牛不信。 
        “你们看,他还真穿着飞行服呢。”一头见过世面的奶牛说。 
        “说不定,是海盗他们耍的新花招儿。”另一头奶牛提醒大家。 
        “海盗?”舒克觉得好玩,这大山里哪来的海盗? 
        “海盗是一只老鼠的名字,他是这一带的老鼠头儿,很坏。”小奶牛说。 
        “别理他,他是装傻呢!”小奶牛的妈妈对女儿说。 
        “我跟海盗根本不认识。再说一遍,我是舒克贝塔航空公司的飞行员,我们机场要开设冷饮部,耍做冰淇淋,需要牛奶,一点儿就够。”舒克拍拍手中的小桶。 
        “什么叫冰淇淋?”小奶牛好奇地问。 
        “冰淇淋…就是……”舒克没吃过。 
        “冰淇淋就是白的……也有黄的,凉凉的,软软的,甜甜的那么一种食物,很好吃。”臭球机械师有幸吃过。 
        “机场和冰淇淋有什么关系?”一头奶牛问。 
        “就足,难道你们的飞机是靠冰淇淋作燃料飞行的吗?”见过世面的奶牛问。 
        “现在天气太热,旅客吃些冷饮,就凉快了。”舒克解释道。 
        “你们的旅客都是老鼠吗?”小奶牛问。 
        “你们把全世界的老鼠运来运去,这不是提供作案工具吗?”见过世面的奶牛还真掌握不少名词。 
        “我们的旅客有老鼠,可大多数是小动物,像松鼠啦,刺猬啦,蜗牛啦……再说,老鼠也不全是坏蛋。”舒克有些不耐烦了。 
        “妈妈,给他们一点儿牛奶吧,我看舒克不像坏蛋。”小奶牛的直觉起作用了。 
        女儿的话妈妈总是听的,奶牛们商量了一下,决定给舒克两小桶牛奶。 
        “真有你的!”臭球佩服舒克。 
        舒克和臭球机械师谢过奶牛们,拎着两桶牛奶朝飞机走去。 
        他俩拐过墙角,只听一声人喝:  “站住!” 
        舒克抬头一看,几十只老鼠把他和臭球围住了。 
        “干吗?”舒克预感到不妙了。 
        “干吗?这是我的地盘,谁让你们来的?收获还不小呀!”一只蓝眼睛的老鼠冷笑着说。 
        “你是谁?”舒克问。 
        “说话注意点!这是我们的大王,绰号海盗,威震天下。”一只老鼠说。 
        “我看你刚才对奶牛说话挺懂礼貌嘛,怎么,对自己的同胞倒不讲礼貌了?噢,对坏蛋是不能讲礼貌的,这样才能显出你好来,对吧?”海盗一边嚼着半根香肠一边说。 
        舒克感到这个对手不一般。 
        一只海盗的部下从臭球手中抢过牛奶桶,递给海盗。海盗一仰脖,喝了个痛快。 
        臭球气得直咬牙,无奈,寡不敌众。   \chapter{第30集} 
        海盗奔袭舒克贝塔航空公司; 
        舒克和贝塔大战海盗及其喽罗; 
        直升机把海盗吊到空中   
        海盗喝足了牛奶,抹抹嘴,问舒克:“刚才你和奶牛说,你是什么航空公司的飞行员?吹什么牛!不过你真有两下子,大模大样就骗来两桶奶,比我们高明!” 
        “报告大王,草丛里真有一架飞机!”一个小喽罗跑来禀报。 
        “噢?”海盗用异样的眼光看了看舒克,转身去草丛里看飞机。 
        舒克冲臭球使了个眼色,臭球撒腿往东跑,舒克往西跑。 
        “抓住他们!”小喽罗们喊起来。 
        海盗的部下太多了,舒克和臭球又被抓回来,这次是五花大绑。 
        海盗来到舒克面前。 
        “我要接管你们的飞机场,同意吗?不同意?那我就烧了你的飞机!如果同意,现在就运我们去。”海盗对舒克下了最后通牒。 
        舒克点点头,他不能眼看着海盗烧了他心爱的飞机。只要到了空中,就是舒克的天下,会有办法打败海盗的。 
        臭球机械师不解地看了舒克一眼,他明白,这一群强盗乘飞机降落在机场,毫无准备的贝塔和整个机场都会成为海盗们的俘虏。 
        舒克朝臭球使个眼色,示意他别胡来。 
        “给他俩松绑。”海盗下令。 
        舒克来到直升机跟前,叫空中小姐开门。 
        飞机舱门打开了,海盗和部下们一拥面上。舒克走进驾驶舱,臭球机械师检查发动机。空中小姐拒绝为海盗们服务,她躲进货舱。 
        海盗走进驾驶舱,他被仪表弄得眼花缭乱,由此倒生出几分对舒克的敬畏之意。 
        “起飞吧!”海盗下令。 
        “驾驶舱不能进外人,请去客舱。”舒克说。 
        “噢,我才不是傻瓜,你好想往哪儿开就往哪儿开,不行,我得看着你。”海盗说。 
        “你会看罗盘吗?你会看航行图吗?”舒克指指仪表盘上的罗盘表,又指指航行图,“不会看这个,上了天你连东西南北也分不清。” 
        海盗看看罗盘,义看看航行图,乖乖地回客舱了。 
        “发动机正常吗?”舒克问臭球机械师。 
        “一切正常,可以起飞。”臭球盖上发动机罩,钻进飞机。 
        舒克按下启动按钮,直升机升到空中。海盗和喽罗们惊叫起来,他们感到新奇,纷纷趴在窗口往外看。 
        “肃静!”海盗叫着,“听着,飞机一降落,你们马上冲下去,占领机场!” 
        海盗像个军事指挥官,给部下分工。 
        舒克回手关好驾驶舱的门,悄悄接通了电台。 
        “贝塔!贝塔!我是舒克!我是舒克!请回答!请回答!”舒克小声呼叫。 
        “我是贝塔。我是贝塔。请讲。”贝塔回话。 
        “飞机现在被一群老鼠强盗占领了,他们现在乘飞机去机场,要占领机场,请作好战斗准备。” 
        “他们有多少?”贝塔问。 
        “27只。”舒克早数好了。 
        “放心吧,我的坦克都呆烦了。”贝塔挂上耳机,拉响了警报。 
        机场各部门的负责鼠都来到贝塔的办公室。 
        “有一伙强盗乘飞机马上来咱们机场,大家赶紧作好准备。你带部下守住候机大楼;你带部下埋伏在停机坪四周;你带部下作好增援准备……”贝塔布置任务。 
        整座机场都忙碌起来,好多旅客也加入了保卫机场的行列。 
        贝塔来到车库,钻进他心爱的坦克。坦克里有充足的炮弹。贝塔把坦克开到停机坪旁的草丛里隐蔽起来。 
        直升机出现在机场上空。 
        整座机场鸦雀无声,只有飞机的发动机声。 
        海盗走进驾驶舱。 
        “你刚才同贝塔的通话我都听见了,大概你还不清楚我的部下的力量。来人!”海盗大喝一声。 
        一个小喽罗走进驾驶舱。 
        “把这根铁棍子窝成圆圈儿。”海盗发话。 
        小喽罗轻而易举地把一根铁棍子窝成了圆圈。 
        舒克愣住了。 
        “他们都会气功,你的同伙是打不过我们的,哈哈!”海盗得意极了。 
        舒克真想一推驾驶杆,来个机毁鼠亡。 
        就在这时,舒克看见草丛里的坦克。他在心里笑了。海盗的部下绝对打不过贝塔的坦克。 
        “做好准备!”海盗回到客舱,向部下发令。 
        小喽罗们个个摩拳擦掌.臭球机械师和空中小姐已被捆了起来塞进货舱。 
        直升机徐徐降落了。螺旋桨还在旋转,海盗就打开舱门,率领部下冲出飞机。 
        埋伏在停机坪四周的机场工作人员呼喊着朝强盗们包围过来。 
        海盗一挥手,小喽罗们四面迎战。 
        机场上作人员不是这伙强盗的对手,已有两名工作人员被摔倒在地上。 
        贝塔的坦克冲出草丛,朝强盗们撞去。 
        海盗弄不清坦克的威力,犹豫之间,已被坦克撞了个跟头。 
        只见他大喊一声,招呼过来几个部下,一起向坦克冲去。 
        贝塔腊准了其中一个小喽罗开炮。 
        炮弹射中了小喽罗的耳朵。耳朵被削去了一半,疼得他大叫不止。 
        毕竟是海盗,凶猛顽固。海盗命令一部分喽罗围攻坦克,另外一部分跟他去占领候机大楼。 
        这回贝塔傻眼了,他不能把坦克分成两辆。 
        在飞机上观战的舒克灵机一动,他跑进货舱放出臭球机械师和空中小姐。 
        “你们作好准备,货舱里有一箱子弹,咱们从空中打击他们。”舒克说完发动飞机。 
        直升机起飞了,擦着地面追赶企图去占领候机大楼的海盗们。 
        臭球机械师把子弹箱扛来了。直升机上有皮皮鲁安装的弹弓枪。 
        直升机追上海盗了。 
        “开火!”舒克命令。 
        臭球机械师接过空中小姐递来的子弹,装进弹弓枪,瞄准海盗的后脑勺.抠动了扳机。 
        打偏了,子弹擦着海盗的脑袋飞过去,打倒了他旁边的一个喽罗。 
        “瞄准海盗打!”舒克懂得擒贼先擒王。 
        臭球又装了一发子弹。 
        还是没打中。海盗真狡猾,拐着弯跑。 
        眼看海盗就要冲进候机大楼了。舒克急了,他要用飞机的起落架压海盗。 
        直升机擦着海盗的头飞。海盗一会儿往左躲,一会儿往右躲,飞机就是压不着他。 
        舒克吸了一口气,撞撞运气,这回就往左落。 
        飞机在海盗的头上飞。海盗知道飞机要从上往下压他,他突然往左一闪。上帝保佑,舒克也是往左一落,起落架牢牢地把海盗压在地面上。 
        “饶命!饶命!!”海盗吓坏了,只要舒克让飞机全部落地,海盗就一命呜呼了。 
        舒克见海盗的两只手死死抓住起落架,他突然一拉杆,直升机拔地而起,把海盗带上了天空。 
        海盗不敢松手。飞机飞得越高他越不敢松手,可又义爬不上去,就这样被吊在空中。 
        “哈哈,太棒啦!”臭球机械师乐了,他打开机舱门,跷着二郎腿逗海盗: 
        “累了吧?头儿!这叫健美锻炼,专练臂力肌肉。你不足会气功吗吗?” 
         海盗的威风全没了。 
        “舒克,来个急转弯,练练他的功。”臭球大声喊。 
        “好吧!”舒克操纵直升机来了个急转弯。 
        海盗的身体被风吹得和飞机平行了。 
        “再来一个俯冲!”臭球愈发得意,他要出出被绑的气。 
        “再来一个急降!”臭球还挺懂飞行姿态。   \chapter{第31集} 
        舒克贝塔航空公司战胜海盗; 
        贝塔想学开飞机; 
        海盗越狱逃跑   
        舒克驾驶直升机将海盗吊到空中,尽情地折腾他,眼看着海盗的臂力不支了,舒克把直升机悬停在空中。 
        “干吗不继续折腾他?”臭球问。 
        舒克不忍心把海盗从天上摔下去。 
        臭球钻进后舱,找出一根铁棍子。 
        “你要干什么?”舒克从驾驶舱探出头问。 
        “我把这强盗打下去。”臭球说完抡起棍子要往机舱外边打。 
        “住手!留着他有用。”舒克大喝一声。 
        臭球的棍子在空中停住了。空中小姐走过来夺走臭球手中的棍子。 
        舒克打开电台。 
        “贝塔,贝塔,我是舒克,请回答!”舒克呼叫。 
        “我是贝塔,我是贝塔,请讲!’贝塔在坦克里说话。 
        “你让海盗的部下马上投降,否则我就把他们的头儿从天上扔下去!”舒克说。 
        “明白。”贝塔关上电台,打开坦克舱盖儿,将头探出坦克。 
        机场上战斗仍在继续,海盗的喽罗们还挺顽固。 
        “海盗的部下们!”贝塔大声喊话,“你们往天上看!你们的头儿正吊在空中。如果你们不投降,我们就把他从天上扔下来!” 
        海盗的部下往上一看果然看见首领被吊在空中,他们只好纷纷投降。 
        贝塔吩咐将俘虏集中到一起,关进机场的库房。 
        “舒克,舒克,我是贝塔,战斗已经结束,请你着陆。”贝塔站在塔台上说。 
        “明白!请你布置人马准备活捉海盗!”舒克边说边操纵直升机下降。 
        停机坪上严阵以待。 
        海盗的两条腿还没落地,就被捆了起来。 
        直升机着陆了。舒克出现在机舱门口,大家像欢迎凯旋的英雄那样冲舒克欢呼鼓掌。 
        贝塔和舒克紧紧拥抱。 
        臭球把五花大绑的海盗关进仓库旁的一间小黑屋。 
        “怎么处置他们?”贝塔问舒克。 
        舒克一时答不上来。处死了于心不忍,都是同胞。放走?他们又要去干坏事。留下?不敢。 
        “你说呢?”舒克问。 
        贝塔耸耸肩膀,也想不出办法。 
        “先关几天,”舒克扭头叫来餐厅部主任:“派人给他们送点儿食物。” 
        餐厅部主任罗丘点点头,他忽然想起了什么,问舒克:“牛奶弄到了吗?我还等着做冰淇淋呢!” 
        舒克这才想起牛奶被海盗喝光了。 
        “我现在就去弄。”舒克转身朝飞机走去。 
        “歇会儿,你得吃点儿东西。”贝塔拉着舒克朝餐厅走去。 
        餐厅里有不少旅客在用餐,他们因海盗袭击机场而延误了起飞时间。 
        舒克看见这么多旅客滞留在机场,他隐隐约约感到光靠一架直升机运载旅客已力不从心。 
        贝塔给舒克端来一份丰美的饭菜。闻到香味儿.舒克才发现自己早就饿了。 
        “你去通知旅客,下一班次马上起飞。我送完这批旅客就去弄牛奶。”舒克边吃边对贝塔说。 
        “我看我该学开飞机了。”贝塔心疼舒克。再说,整个航空公司就一个飞行员,也显得少了点儿。 
        “过几天我教你。”舒克没意见。 
        半小时后,舒克的直升机满载着旅客起飞了。 
        贝塔坐在塔台里随时同舒克保持联系,不敢有一点儿疏忽。 
        两个小时后,直升机平安返航了。舒克将两桶牛奶递给餐厅主任罗丘。 
        “快去做冰淇淋吧。”贝塔对罗丘说。 
        “好,马上做,”罗丘拎着两桶牛奶走了。 
        “从明天开始,你教我开飞机。”贝塔坐在沙发上说。 
        “行。”舒克疲劳地躺在长沙发上。 
        臭球一阵风似地跑进舒克的办公室。 
        “海盗跑了!”臭球报告。 
        舒克和贝塔“腾”地从沙发上蹦起来。 
        “怎么跑的?”舒克不信,海盗被捆得很结实。 
        “绳子都断了,他把部下也都放跑了。”臭球后悔当初没有把海盗从天上扔下来。 
        舒克和贝塔感到了海盗的厉害。 
        “赶快搜索机场,加强直升机的警卫。”舒克下令。 
        “是!”臭球跑出去。 
        天已经黑了,探照灯在机场上扫来扫去。工作人员搜索机场的每一个角落。   \chapter{第32集} 
        舒克带罗丘去城里学做冰淇淋; 
        直升机落在冷饮店房顶上; 
        罗丘遇险   
        机场上没有海盗和他的喽罗们的踪影。 
        “没有潜伏在机场,确实跑了。他们大概也被坦克和飞机吓破胆了。”臭球分析。 
        “也许。”舒克点点头。 
        “咱们去看看冰淇淋。”臭球念念不忘。 
        舒克、贝塔和臭球来到餐厅,只见罗丘主任正冲着桌上的一个方盒子皱眉头。 
        “冰淇淋做好了?”贝塔凑过去看。 
        只见方盒子里冻着一块白颜色的冰块,硬得啃都啃不动。 
        “做不成。”餐厅主任泄气了。 
        “我带你去城里学学。”舒克拍拍罗丘的肩膀。 
        他觉得机场开设个冷饮部还是很必要的,何况正是为做冰淇淋才同海盗开了战,要是做不成冰淇淋,岂不太亏了。 
        “什么时候去?”罗丘来情绪了,他只要看一遍冰淇淋的制作过程,就能学会。 
        “趁着天黑,现在就去。”舒克也来劲了,他好久没进城了。 
        “我也去。”贝塔想去看看咪丽。 
        “你得在家值班。等你学会了开直升机,就可以自己进城了。”舒克说。 
        贝塔无奈,只得留在机场。 
        舒克、罗丘和臭球朝直升机走去。臭球打开发动机盖,检查发动机。 
        舒克和罗丘钻进机舱。 
        “怎么样?”舒克从驾驶舱伸出头来问臭球机械师。 
        “一切正常,可以起飞。”臭球盖好发动机盖,也钻进飞机。 
        “报告塔台,请求起飞。”舒克请示贝塔。 
        “可以起飞,注意安全,随时保持联系。”贝塔回答。 
        机场上灯火通明。 
        舒克好久没飞夜航了,他很兴奋。发动机开始运转,螺旋桨开始旋转,机身开始离地。整个机身转着圈地升到空巾,径直朝城市飞去。 
        罗丘和臭球把鼻子贴在舷窗上往下看。 
        “这是电影院。这是体育场。这是商店。”臭球对城市建筑挺精通。 
        “冷饮店!”罗丘喊道。 
        舒克往下一看,一座灯火闪烁的冷饮店出现在机身下方,店门口人来人往,热闹非凡。 
        “注意,飞机降落!”舒克告诉机上人员。 
        直升机缓缓地在冷饮店屋顶上着陆了。 
        “臭球,你看守飞机,我和罗丘去看看。”舒克说。 
        “嗯。”臭球不大情愿地点点头,他也想看冰淇淋是怎么做出来的。 
        舒克和罗丘沿着下水管道钻进冷饮店。他们来到冷饮店后边,这里是做冰淇淋的地方。几个穿白大褂的人在做冰淇淋。 
        舒克看见一张桌子上有一堆瓶瓶罐罐,他和罗丘躲列瓶瓶罐罐的后边,这里视野开阔,能看到整个房间。 
        罗丘的眼睛直勾勾地盯着那几个做冰淇淋的人。 
        一个人正往盆里打(又鸟)蛋。一个胖胖的人走过来。 
        “打这么多(又鸟)蛋!”胖子有些不满。 
        “经理,按规定做50公斤冰淇淋就得放这么多(又鸟)蛋。”打(又鸟)蛋的人说。 
        “以后少放一半儿(又鸟)蛋!”胖经理说。 
        “这……” 
        “人家吃不出来!”胖经理又对放牛奶的人说:“牛奶也要少放。多放色素,多放糖精。” 
        舒克和罗丘相互看看,无话可说。他们都知道糖精不是好东西,牛奶和(又鸟)蛋是好东西。舒克替门口那些掏钱买冰淇淋的人担心。 
        罗丘把制作冰淇淋的全部过程都记在心里。 
        “咱们走吧。”舒克一转身,碰翻丁桌上的一个小瓶子。瓶子滚到地上,碎了。 
        响声惊动了屋里的人,他们的视线“刷”地扫向桌子上。 
        “快跑!”罗丘和舒克撒腿就往外跑。 
        “老鼠!抓老鼠!!”人们喊起来。 
        一阵杂乱的脚步声尾随着舒克和罗丘。 
        舒克和罗丘跑进营业大厅,许多顾客在吃冷饮。人们一听说老鼠,纷纷站起来观察自己脚下。 
        “分头跑,你往左,我往右,到房顶集合!”舒克冲罗丘喊。 
        罗丘朝左边跑去。舒克往右边跑。 
        “堵住门口,别让它跑了!偷吃我的食物,真可恶!”胖经理怒不可遏。 
        顾客们齐心帮着店员抓老鼠。 
        舒克想起胖经理少往冰淇淋里放(又鸟)蛋的事,他觉得胖经理和老鼠差不多,可人却不抓他。 
        舒克毕竟是经验丰富,他成功地绕过无数只脚,逃出丁冷饮厅。 
        臭球正躺在飞机里睡觉呢,他被舒克剧烈的砸门声惊醒了。 
        “罗丘设有回来?”舒克劈头便问。 
        “罗丘?”臭球揉揉眼睛。 
        “糟糕!”舒克扭头就走。 
        “等等,出了什么事?”臭球抓住舒克问。 
        “罗丘大概被人抓住了!我去救他,你快同贝塔联系。”舒克说完便消失在夜色中。 
        果然,罗丘被人抓获,他被关在一个铁笼子中,全身打着哆嗦。   \chapter{第33集} 
        罗丘脱险; 
        舒克到皮皮鲁家作客; 
        舒克贝塔航空公司增添大型喷气客机   
        舒克眼见罗丘被人抓住,他束手无策。就在这时,舒克看见冷饮店门口出现了一个熟悉的身影。 
        舒克一愣,是皮皮鲁! 
        自从和皮皮鲁分手后,舒克经常想起他,他感激皮皮鲁对他和贝塔的友情。 
        舒克跑过去拽皮皮鲁的裤腿。 
        “我是舒克!”舒克扯着嗓子喊。 
        “舒克?”皮皮鲁惊讶。他蹲下去,借着灯光一看,果然是舒克。 
        舒克把自己来到城里以及罗丘怎么被抓住等等统统告诉了皮皮鲁。 
        “你藏在我兜里,我去救罗丘。”皮皮鲁把舒克装进口袋里,走进冷饮店。 
        “把这老鼠交给我吧,由我来处决他!我家有只猫。”皮皮鲁对大家说。 
        没有人反对。 
        皮皮鲁掀开铁笼子,用手抓起罗丘。 
        “这孩子,用手抓老鼠。” 
        “真不讲卫生!” 
        “……” 
        人们议论纷纷。 
        皮皮鲁旁若无人地走出冷饮店。 
        “谢谢你!皮皮鲁。”舒克感激地说。 
        “到我家去歇会儿吧!咱们得好好聊聊。”皮皮鲁邀请舒克。 
        “行!”舒克同意了,“告诉我怎么走,我开飞机去。” 
        皮皮鲁将方位和标志告诉舒克。 
        “你先走吧,我们随后就到。”舒克和罗丘说完顺着下水管爬上屋顶。 
        臭球正在驾驶舱里同贝塔通话。舒克接过话筒。 
        “贝塔,我是舒克,罗丘已脱险。” 
        “太好了,马上返航!,’ 
        “我先去趟皮皮鲁家。” 
        “皮皮鲁?你见到皮皮鲁了?” 
        “回去详谈。” 
        “注意安全。” 
        “明白。” 
        舒克摘下耳机,准备起飞。 
        臭球给罗丘包扎伤口。罗丘身上受了几处伤,是被人用扫帚砸的。 
        直升机准确地降落在皮皮鲁家的阳台上。皮皮鲁已经在阳台上恭候舒克了。 
        一顿丰盛的晚餐等待着舒克和伙伴们。舒克把臭球介绍给皮皮鲁。 
        舒克一边吃一边给皮皮鲁讲舒克贝塔航空公司的故事。 
        “一架直升机就能开航空公司?”皮皮鲁撇撇嘴。 
        “是少了点儿。旅客多,飞机少。”舒克承认。 
        “我送你们一架大型喷气式客机。”皮皮鲁说完从书柜里拿出一架极豪华的玩具大型客机,“这是我过生日时,舅舅送我的。” 
        “这……”舒克有点儿不好意思。 
        “送给你!放在我这儿也没用。”皮皮鲁豪爽地说。 
        “可我不会开呀!”舒克望着巨大的喷气机,为难地说。 
        “你当了这么长时间飞行员,大同小异,明天咱们到楼顶上的大平台试飞。”皮皮鲁说。 
        “谢谢你!”舒克顾不上吃饭了。 
        “咱们参观参观。”臭球提议。 
        “进去看吧!”皮皮鲁打开机舱门。 
        舒克、臭球和罗丘走进客舱,整座客舱富丽堂皇。绿色的地毯,舒适的高背椅,冷气设备,灯光设备、音响设备……靠近驾驶舱的是头等舱,头等舱里设备更齐全。 
        “还有二楼呢!”臭球指指上边。 
        “这是仿造波音747飞机。”皮皮鲁在外边告诉舒克他们。 
        “咱们把好消息告诉贝塔。”舒克走进驾驶室,打开电台。 
        “贝塔,贝塔,我是舒克,请回答。” 
        “我是贝塔,请讲!” 
        “皮皮鲁送给咱们一架大型喷气客机,我们现在在客机上同你讲话。” 
        “真的?太好啦!” 
        “我想给这架飞机定名为皮皮鲁号,行吗?” 
        “同意!” 
        “请你马上组织扩建机场跑道,我明天下午驾驶皮皮鲁号试航!” 
        “明白!”   \chapter{第34集} 
        皮皮鲁号安全抵达机场; 
        舒克和贝塔决定拍电影   
        贝塔和舒克通过话以后,立即组织扩建机场跑道的工作。航空公司全体人员出动,将跑道长度扩容了一倍。 
        第二天上午,贝塔通过无线电告诉舒克,机场跑道扩建完毕。 
        “我今天下午驾驶皮皮鲁号返回机场。”舒克说。 
        “直升机怎么办?”贝塔问。 
        舒克这才想起还有直升机。 
        “我把直升机开回去。”臭球在一边说。 
        “你?”舒克不放心。 
        “我看都看会了。”臭球的牛劲上来了。 
        舒克想想,也只好冒这个险了。 
        “我先训练你一下。”舒克说。 
        舒克和臭球钻进直升机,舒克给臭球作示范飞行,臭球脑子不笨,一会儿就能单独飞行了。 
        皮皮鲁在一旁看着,很开心。 
        下午,皮皮鲁将喷气式客机和直升机都拿到楼顶的太平台上。 
        舒克和罗丘钻进皮皮鲁号,臭球钻进直升机。 
        大型喷气式飞机开始在平台上滑行,舒克给飞机不断加大马力。 
        飞机的机头离开地面,紧跟着,整个机身都离地了。飞机上天了。 
        臭球也操纵直升机起飞。 
        皮皮鲁向他们挥手。 
        舒克的飞机升到了空中。飞机突然开始摇晃起来,舒克觉得喷气式飞机比直升机难驾驶,他努力体会驾驶窍门。 
        “舒克,舒克,我是贝塔,请回答。”耳机里传来贝塔的呼叫。 
        “我是舒克,我的飞机现在空中。” 
        “情况怎样?” 
        “有点儿摇摆,问题不大,放心吧。” 
        “祝你成功!” 
        “你再同臭球联系一下。”舒克还想着臭球。 
        飞机开始平稳飞行了。舒克知道,关键是着陆,弄不好就会机毁鼠亡。 
        机场出现在前方,舒克紧张地握着驾驶杆,眼睛盯着下边。 
        “皮皮鲁号请求着陆。”舒克请示塔台。 
        “同意着陆。”贝塔的声音也很紧张。 
        跑道旁边停着消防车和救护车。 
        巨大的皮皮鲁号离跑道越来越近。 
        “快拉起来!快!”贝塔对着话筒大叫。 
        舒克来不及问为什么,就在飞机与跑道尚未接触的一刹那,将飞机拉了起来。 
        “你忘了放起落架!”贝塔惊魂未定。 
        舒克出了一身冷汗。他把起落架放出机舱。 
        飞机绕场一圈,第二次对准了跑道。 
        成功了,皮皮鲁号平安着陆。大家涌向这架巨大的客机,一片欢呼。 
        臭球驾驶的直升机也安全着陆。 
        罗丘跑步去餐厅制作冰淇淋。 
        舒克带大家参观皮皮鲁号,大家都被皮皮鲁号的豪华和气势惊呆了。 
        舒克和贝塔决定成立皮皮鲁号机组,他们选出了七名精干的工作人员,分别担任空中小姐和机械师,臭球担任副驾驶,舒克担任机长。 
        舒克教贝塔学会了驾驶直升机。 
        这天清晨,皮皮鲁号首航运送客人。几百名旅客依次登上飞机,他们去南方旅游。 
        皮皮鲁号满载着旅客起飞了。它昂着头,插进云端。 
        舒克定好方位,打开自动驾驶仪。 
        “你在这儿值班,我去客舱看看。”舒克吩咐副驾驶臭球。 
        “放心吧。”臭球说。 
        舒克走进客舱,看见空中小姐正给旅客分发饮料和冰淇淋。有的旅客往舷窗外看,有的在打瞌睡。 
        “应该丰富旅客的旅途生活。”舒克想。他抬头看见了悬挂在客舱前方的电影银幕。 
        舒克回到驾驶舱。 
        “贝塔,贝塔,我是舒克,请回答。”舒克打开电台。 
        “我想在飞行中为旅客放电影解闷,可咱们没有电影片子,你准备一下,咱们自己拍电影。”舒克说。 
        “拍电影?行啊。”贝塔挺兴奋,“我去筹备,谁当导演呀?” 
        “你当吧。” 
        “编剧呢?” 
        “臭球当。” 
        “制片主任呢?” 
        “……” 
        美工呢?” 
        “……” 
        “摄影呢?” 
        “行啦行啦,我看你挺内行,就都包了吧!”舒克关上电台。 
        “到了。”臭球提醒机长。舒克这才意识到大飞机没有跑道是无法着陆的。他们忘了跑道的事。 
        皮皮鲁号在空中盘旋。   \chapter{第35集} 
        皮皮鲁号在公路上迫降; 
        险些同大卡车相撞; 
        艾丽担任故事片编剧   
        没有跑道,喷气式客机皮皮鲁号无法着陆。舒克埋怨自己粗心大意,现在后悔也晚了。 
        飞机在空中盘旋。 
        旅客们发现飞机老在原地打转,觉出不对头了,纷纷趴在窗口往外看。 
        “你看!”臭球让舒克往下看。 
        舒克看见地面上有一条宽大的公路,公路上行进着来往的车辆。 
        舒克眼睛一亮,对,在公路上迫降。 
        公路上车辆很多,得避开它们。 
        舒克来到客舱,对旅客们说: 
        “请大家原谅,由于我们的疏忽,忘记修跑道了。现在,我们要在一条公路上降落。希望大家坐在座位上不要动,系好安全带,保持飞机平稳。” 
        没有旅客起哄,也没人谴责舒克的粗心,既然人家已经承认了错误,何况现在是生死与共。 
        舒克回到驾驶舱,操纵飞机在公路上空盘旋,等候时机。 
        终于,公路上出现了一个空白带,没有车辆。 
        皮皮鲁号对准了公路。舒克一推驾驶杆,飞机朝公路逼近…… 
        机轮挨到了地面,迫降成功。飞机在公路上滑行。 
        “注意对面!”臭球尖叫一声。 
        舒克抬头一看,对面驶来一辆大卡车。 
        操纵飞机拐弯已经来不及了,惟一的出路是从卡车下边钻过去。 
        舒克双手紧握驾驶杆,两脚踩着转向舵。不断调整着飞机的方向。 
        卡车司机显然看见了皮皮鲁号,他来了个急刹车。皮皮鲁号从卡车下边钻了过去。 
        路旁是草丛。舒克操纵飞机钻进草丛隐蔽起来。卡车司机从车上跳下来,找那架玩具飞机。他揉揉眼睛,怀疑是自己眼花看错了。 
        皮皮鲁号里一片掌声,大家庆贺迫降成功。 
        舒克操纵飞机滑行到一座小山坡旁,他觉得在这儿修建飞机场挺合适。 
        旅客们离开飞机,他们站在飞机旁不走。 
        “怎么回事?”舒克问空中小姐。 
        空中小姐跑到飞机下边,同旅客们说着什么,然后爬进机舱告诉舒克: 
        “他们说,要帮咱们修完跑道再走。” 
        舒克原来还以为旅客要同他算帐,他感动得眼眶湿了。 
        说干就干。舒克和臭球跳下飞机,拿尺子测量土地。旅客们有的拔草,有的平地。 
        经过一天的紧张施工,跑道修好了。 
        几位旅客毛遂自荐当机场的工作人员,舒克同意了。 
        这天上午,皮皮鲁号缓缓滑上了新修的跑道。转眼间,喷气客机(禁止)云端。 
        当皮皮鲁号平安降落在舒克贝塔航空公司机场时,机场的工作人员都拥到飞机旁,迎接首航归来的勇土。 
        “咱们以后可得细心点儿,”舒克对贝塔说,“哪儿有不修跑道就搞空运的呀!这在世界航空史上也算奇迹了。” 
        贝塔耸耸肩膀。 
        “快去餐厅吃饭吧。”贝塔拉着舒克来到餐厅。 
        餐厅主任给舒克和贝塔端来丰盛的午餐。舒克大口大口吃起来。 
        “咱们商量商量拍电影的事。”舒克往嘴里塞了一块肉。 
        “我看场务组里有个负责扫跑道的叫艾丽的姑娘,平时喜欢诌两句,就让她当编剧吧。”贝塔对部下的特长挺了解。 
        “行,你让她快点儿把剧本写出来。”舒克抹抹嘴。 
        饭后贝塔打电话把艾丽叫来。 
        “咱们公司准备拍电影。”贝塔对艾丽说。 
        “拍电影?”艾丽觉得新鲜。 
        “决定由你当编剧。” 
        “我?”艾丽怀疑自己的耳朵。 
        “尽快把剧本写出来。舒克已派人去弄拍电影的器材了。”贝塔说。   \chapter{第36集} 
        审查小组通过了剧作家艾丽的电影剧本; 
        臭球当电影演员; 
        臭球借着拍电影大吃花生米   
        一个星期后,一部为舒克贝塔航空公司歌功颂德的剧本诞生了。 
        为此,公司专门成立丁一个审查小组,研究通过该公司的第一部故事片剧本。 
        贝塔担任组长。组员有舒克、臭球、罗丘,都是见过世面的人物。 
        这天,航空公司停飞一天,专门讨论剧本。 
        艾丽捧着剧本坐在会议室里,像等待审判一样。 
        “这是咱们公司的第一部电影,当然,主要是在飞机上放映。不过,如果拍好了,说不定也能去参加奥斯卡金像奖评选。”贝塔先说几句。 
        “什么叫奥斯卡?”罗丘问。 
        “奥斯卡 ……奥斯卡就是奥斯卡,世界上最权威的电影评奖,在外国。”贝塔解释。 
        ‘咱们老鼠拍的电影也能参加评选?”臭球表示怀疑。 
        “这……我想能吧。几十年评来评去都是评人拍的电影,突然来了一部老鼠拍的,大家准感兴趣。”贝塔说。 
        “让艾丽把剧本先念一遍。”舒克提议。 
        艾丽清清嗓子,开始念剧本。 
        “剧本的名字叫《会开直升机的老鼠》,是以舒克的经历为原型写的。”艾丽说。 
        “这电影名字长了点儿。”臭球发表意见。 
        “嗯,老鼠两个字出现在片名上有点儿那个。”罗丘谈自己的看法。 
        “改成《舒克和直升机》怎么样?”贝塔提议。 
        “我反对用真名。”舒克不同意。 
        艾丽感到为难,不知如何修改片名。 
        “先念剧本吧。”贝塔说。 
        艾丽把剧本念了一遍。 
        大家七嘴八舌地提修改意见。这个说主要人物性格不突出,那个说猫的形象太高大,另一个说某一个细节有丑化老鼠之嫌…… 
        艾丽记录下来的意见比原剧本的字数多出一倍。 
        “回去好好改改,尽快写好。”贝塔吩咐。 
        “嗯。”艾丽面有难色,但还是答应了。 
        “你会改好的。”舒克像个大学教授。 
        “语言再精炼些。”臭球像大作家。 
        “增加些悬念。”罗丘不甘落后。 
        大家都得到了极大的满足,都没想到自己这辈子还能审查电影剧本。说真的审查剧本比当编剧舒服多了,神气多了。 
        数日后,电影剧本审查小组再次开会研究艾丽写的电影剧本。 
        艾丽上次开完会后,回去把大家的意见看了一遍。认为按照这些意见无法修改剧本,许多意见都是自相矛盾的。她干脆一字不改。 
        “我按照你们的意见把电影剧本重新写了一遍,现在念给你们听听。”剧作家艾丽说。 
        大家冼耳恭听。 
        “改得好。”舒克听完后点头喝彩。 
        “不错,具备了获得奥斯卡奖的条件。”贝塔点头。 
        罗丘和臭球也是赞不绝口。 
        艾丽本来悬着的一颗心放下了。 
        为了奖赏剧作家,罗丘赠给她一碗试制成功的冰淇淋。 
        电影要正式开拍了。贝塔担任导演。让舒克演舒克最合适不过,可舒克要飞行,没时间。 
        “让臭球主演吧,他会开飞机。”舒克推荐。 
        就这样,臭球当上了电影演员。 
        这天下午,摄制组正式开机。 
        第一个镜头:舒克(臭球扮演)跟着妈妈出来找吃的,碰到一盘花生米。 
        妈妈由剧作家艾丽扮演。 
        “开机!”贝塔一声令下。 
        “妈妈”带着“舒克”从家里走出来。“舒克”比“妈妈”还神气,还对这个世界不屑一顾。 
        “停!”贝塔挥手。他走到臭球面前,“你是有生以来第一次见到这个世界,要惊讶,要东张西望,别老摆出电影天皇巨星的派头。” 
        臭球撇嘴。 
        “重拍!”导演对摄影师说。 
        摄影师名叫四黑.是从导航室抽调来的。 
        “舒克”跟着“妈妈”东张西望地从家里走出来,他们j来到一张桌子上。桌上放着一盘油炸花生米。 
        臭球一见到油炸花生米就忘了一切。他扑上去大吃起来,忘了说台词。 
        “台词!台词!”贝塔急了。 
        臭球还是不顾一切地吃。 
        “停!”贝塔冲上去拉拉臭球,“你怎么搞的?” 
        “我,我忘了台词。”臭球抹抹嘴。 
        “重拍!”贝塔下令。 
        臭球眼前一亮,刚才的花生米白吃了。对,就故意出错,直到把这盘花生米全吃完为止。 
        “台词错了,重拍!”贝塔说。 
        “又错了,重拍!”贝塔说。 
        “重拍!” 
        “重拍!” 
        盘中的花生米急剧减少。 
        当剩下最后五颗花生米时,拍摄成功。 
        “拍下一个镜头。”贝塔宣布。 
        臭球的肚子胀得像皮球,躺在地上一动不动。 
        经过一个月的紧张拍摄,故事片完成了。 
        舒克提出看看样片,摄影师四黑这才发现,拍摄过程中,摄影机里始终没装胶片。   \chapter{第37集} 
        舒克贝塔航空公司发展壮大; 
        皮皮鲁号遭受来历不明的歼击机袭击   
        当贝塔听说摄影机里忘了装胶片后,当即昏了过去。臭球倒挺高兴,他又可以大吃一番花生米了。 
        没办法,电影只得重拍一次。这回为了保险起见,摄影师四黑往摄影机里装了两副胶片。 
        故事片终于拍成。舒克看后挺满意,但觉得送奥斯卡奖评比还差一截,因为导演时不时也出现在画面上。 
        皮皮鲁号喷气机在飞行途中能为旅客放电影了,旅客们再不会感到旅途是无聊寂寞的了。 
        经过不断的翻修和扩建,舒克贝塔航空公司的主机场越来越宏伟,现代化设施星罗棋布,皮皮鲁号喷气机和直升机都投入空运。舒克又培训出六名飞行员。 
        在一个万里无云的晴天,皮皮鲁号客机送100多只青蛙去远方。 
        飞机平稳地飞行。客舱里,旅客们在看电影。 
        舒克和副驾驶臭球聚精会神地驾驶飞机。突然,飞机前方山现了三个小黑点。 
        “臭球,你看那是什么?”舒克说。 
        “是大鸟?”臭球拿不准。 
        舒克和臭球仔细看。 
        “是三架飞机!”臭球叫起来。 
        舒克定神一看,真是三架飞机,三架同皮皮鲁号比例一样的小飞机。 
        舒克没听说过附近还有小动物开飞机的。 
        “是歼击机!”臭球报告说。 
        三架涂得花花绿绿的歼击机朝皮皮鲁号逼近,它们摆开了三角队形,冲过来。 
        “快闪开!”臭球叫起来。 
        舒克忙操纵飞机降低高度。 
        三架歼击机同时开火了。皮皮鲁号的机身急剧晃动。 
        舒克和臭球慌了,他们弄不清这三架来历不明的歼击机干吗要袭击皮皮鲁号。向客机开火,是严重违反国际航空法的行为。 
        “你去告诉旅客,系好安全带。”舒克看见那三架飞机掉头义朝皮皮鲁号飞来,他对臭球说。 
        臭球跑进客舱,把紧急情况通报给人家。 
        舒克决定返航。他拿起话筒。 
        “贝塔,贝塔,我是舒克,请回答。” 
        “我是贝塔,请讲。” 
        “有三架来历不明的歼击机袭击我们,皮皮鲁号已经受伤,我现在驾机返航,请你做好准备。” 
        “歼击机?开炮打客机?”贝塔感到问题十分严重。 
        “歼击机又来了!”舒克握驾驶杆的手心出汗了。 
        “当心!”贝塔叮嘱舒克。他恨自己帮不上舒克的忙。 
        三架歼击机从皮皮鲁号上方压下来。 
        “嘿嘿!”舒克的耳机里传出一阵冷笑。 
        舒克感到这声音很熟悉。他知道这是从歼击机上传来的声音。 
        “你是谁?”舒克问,“干吗袭击我们?” 
        “我名叫海盗,上次你把我抓去,我越狱成功了。现在我是海盗飞行大队总队长,找你算账来了!” 
        海盗洋洋得意。 
        舒克和臭球傻眼了。 
        “请你不要朝客机开火,我们可以到地面上谈判。客机上有100多名旅客。”舒克说。 
        “我管你有多少旅客!僚机注意,目标,皮皮鲁号,开火!”海盗发狠了。 
          随着猛烈的炮火,皮皮鲁号的机身开始倾斜。 
        “报告机长,机舱左侧出现了一个窟窿,有一只青蛙受伤。”空中小姐闯进驾驶室。 
        “你们给他包扎伤口!告诉旅客不要紧张。”舒克一推驾驶杆,飞机快速下降高度。 
        “咱们到树林里去飞。歼击机速度快,会撞在树上的。”舒克告诉臭球。 
        “你注意驾驶,我观察敌机情况。”臭球说。 
        皮皮鲁号一头扎进一片树林。 
        海盗很狡猾,没有跟进来。 
        舒克松了口气。 
        “你去检查一下飞机损伤情况。”舒克吩咐臭球。 
        臭球来到客舱,只见左侧有几处弹孔,其中一个弹孔就挨着发动机,好险。 
        “舒克,舒克,我是贝塔!快回答!” 
        舒克的耳机里传出贝塔的紧急呼叫。 
        “我是舒克,快讲!” 
        “那三架歼击机来袭击咱们机场,还扔了两颗炸弹!”贝塔上气不接下气。   \chapter{第38集} 
        皮皮鲁号超低空飞行,撞在渔网上; 
        海盗袭击机场; 
        舒克引开海盗   
        “海盗去袭击咱们机场了!”舒克告诉臭球。 
        “咱们往哪儿飞?”臭球急得直揪自己的耳朵。 
        “先在这儿盘旋,我再同贝塔联系。”舒克按下电台上的通话按钮。“贝塔,贝塔,我是舒克,快讲话!” 
        没有回答。 
        “贝塔,贝塔,我是舒克,请回答!” 
        还是没有声音。 
        准是机场遭到了严重破坏。 
        “飞回去!”舒克决定冒险。 
        大型客机作超低空飞行是非常危险的。但只有超低空飞行能躲开海盗的歼击机。 
        “你驾驶一会儿,我去同旅客们说。”舒克对臭球说完来到客舱。 
        “各位旅客,请大家不要慌。”舒克镇静地说: 
        “我们的飞机遇到了空中强盗的拦截,现在这帮强盗又去袭击我们的机场。我们现在返航。为了保证大家的安全。要超低空飞行,请大家放心。” 
        “这些空中强盗真可恶!” 
        “皮皮鲁号上要是有炮就好了!” 
        旅客们议论纷纷。 
        舒克放心了。青蛙们没有胆怯。 
        舒克回到驾驶舱,对臭球说:“你注意观察,我驾驶。” 
        皮皮鲁号喷气客机开始降低高度,擦着地面飞行。舒克眼睛都不敢眨巴,生怕撞在什么东西上。 
        “注意,前方有树!”臭球站着说。他不敢坐下观察,怕发现障碍物太晚。 
        客机绕过大树。 
        前边有一张大渔网,晾在竹竿上,臭球和舒克都没看见渔网。皮皮鲁号撞在渔网上,把鱼网拉走了。 
        渔网罩在飞机上,把飞机包住了。飞机像是被装在网兜里。但没有影响飞行。 
        舒克出了一身冷汗。 
        前方是一片高高的草丛。 
        “海盗的飞机飞回来了!”臭球发现了上空的一架歼击机。 
        舒克一推驾驶杆,飞机钻进卓丛,在草丛里飞行。渔网挂上了许多草叶,飞机穿上了伪装服。 
        此时还有两架海盗的歼击机正在轮番攻击舒克贝塔航空公司的机场。海盗派了一架飞机去找皮皮鲁号。 
        海盗坐在座舱里,他得意极了。自从上次越狱逃跑后,海盗就发誓要成立一个飞行大队,他纠集了十几只老鼠家族中的亡命徒,弄来十几架歼击机,天天训练飞行。 
        海盗两次从天上掉下来,又奇迹般地死里逃生。终于,他练出了一手过硬的飞行本领,还培训出十几名歼击机飞行员。 
        现在,海盗报仇的机会来了。他忘不了自己被舒克吊在空中的耻辱。 
        “僚机跟上!”海盗又冲舒克贝塔的机场发起了一次进攻。 
        两架歼击机呼啸着俯冲下来。 
        海盗用瞄准具的光环套住了塔台,他使劲按下驾驶杆上的射击按钮。 
        一串炮弹射出去,塔台起火了。 
        贝塔的腿被打伤了。电台也被打坏了。 
        “快,快修电台!”贝塔倒在地上喊。 
        无线电员扑上去抢修电台。 
        急救车赶来抢救贝塔。 
        海盗的飞机又俯冲下来。 
        贝塔连一点儿招架的方法也想不出来,他气得直喘粗气。 
        一串炮弹打中了候机大楼。 
        “叫旅客都躲到地下室去。”贝塔命令。 
        “电台修好了。”无线电员报告。 
        “接舒克!”贝塔戴上耳机。 
        舒克听到了贝塔的呼叫。 
        “海盗的飞机还在袭击机场,机场损失惨重,跑道被炸了两个大坑,皮皮鲁号无法着陆。”贝塔说。 
        “我已接近机场,我想法把海盗引开,保护机场。”舒克听说机场损失严重,心疼极了。他决定用皮皮鲁号引开海盗。 
        舒克一拉驾驶杆,飞机冲上天空。 
        海盗派来的歼击机发现了披着伪装网的皮皮鲁号。他立即报告了海盗。 
        “盯住它,我们马上赶到。”海盗说完带领僚机离开了机场上空。   \chapter{第39集} 
        舒克调虎离山; 
        海盗的飞机被鹰击中; 
        海盗再次空袭舒克贝塔航空公司机场   
        那架歼击机死死咬住皮皮鲁号不放。 
        舒克减速,它也减速。舒克加速,它也加速。 
        客舱里不少旅客呕吐了。空中小姐忙着给他们收拾清洁袋。 
        “舒克舒克,我是贝塔。海盗已离开机场,你要当心。”贝塔通知舒克。 
        “明白。请赶快抢修跑道。”舒克说。 
        “明白。”贝塔带伤指挥机场工作人员抢修跑道。许多旅客也来帮忙。 
        皮皮鲁号摆脱不了歼击机的追逐。这时,海盗带着僚机来了。 
        三架歼击机压在皮皮鲁号上方。海盗还操纵飞机来回晃机翼,故意气舒克。 
        舒克忽然想起前边那座山头上有个鹰巢,他有办法了。 
        “咱们把飞机开到鹰巢旁边,把鹰引出来。”舒克对臭球说。 
        “这办法不错,也够危险的。”臭球点点头。 
        皮皮鲁号朝鹰巢飞去。 
        海盗觉得马上就把舒克的飞机打下来太便宜他了,他想折腾舒克,拿舒克开心。 
        舒克驾驶飞机擦着鹰巢飞过去,机翼尖碰掉了鹰巢的几根树枝。 
        鹰被激怒了,它“呼”地一下飞到空中,寻找挑衅者。 
        正好三架歼击机飞过来,它们自然成为鹰的攻击目标。 
        海盗正一边吹口哨一边开飞机,忽然觉得眼前一黑,原来是鹰用翅膀拍他的飞机。海盗的飞机失去了控制,螺旋着朝地面栽下去。 
        海盗的飞行技术堪称世界一流,他在飞机撞地的一刹那问,把飞机拉起来了。 
        鹰又冲过来。 
        “返航。快返航!”海盗招呼部下。 
        三架歼击机慌忙逃窜,躲避鹰的攻击。 
        皮皮鲁号降落在弹痕累累的机场上。 
        贝塔躺在担架上来接舒克。 
        看着被破坏的机场,看着受伤的朋友,舒克咬牙切齿: 
        “我非再把这个海盗吊到天上去不可!” 
        “也不知那鹰消灭海盗没有?”臭球说。 
        空中小姐跑过来问舒克: 
        “青蛙旅客们怎么办?” 
        “安排在机场宾馆住下,明天照常飞行。”舒克说。 
        空中小姐把旅客们带到宾馆去休息。 
        天黑了。舒克和臭球吃完饭后,带着机械师们修补飞机。 
        机场工作人员修整被炮弹打坏的房屋。 
        舒克、臭球和地勤人员把挂在飞机上的渔网“伪装服”摘下来,拿胶水补机身上的弹孔。 
        空中小姐清扫客舱里的垃圾。 
        飞机修好时已是深夜了。 
        “大家回去好好休息,明天早晨起飞。”舒克吩咐道。 
        第二天早晨,皮皮鲁号准备起航。舒克检查发动机。旅客们陆续登机。 
        突然,警报响了。 
        “怎么回事?”舒克从驾驶室里探出头来。 
        “海盗的战斗机又来了!”一名地勤人员指指天上 
        舒克一看,远处空中出现了几个黑点。那黑点越来越火。不是三架,而是几十架! 
        舒克急忙发动了飞机,把皮皮鲁号滑进停机坪的草丛里。 
        “臭球,快派人把伪装网盖在飞机身上。”舒克一边说一边跑下飞机。 
        空中小姐疏散旅客。 
        地勤人员飞速把伪装网盖在飞机身上。 
        海盗的飞机飞临机场上空了,黑压压一片排着整齐的队形。 
        其中的两架歼击机俯冲下来,一阵排炮,机场上硝烟骤起。 
        又是两架歼击机俯冲下来扫射。 
        海盗的飞行大队就是这样两架一组地轮番进攻机场。舒克没有招架和还手的可能。 
        机场上的不少建筑被摧毁了,一些工作人员和旅客受伤了。 
        海盗们打够了,就驾机在机场上空作飞行表演:翻筋斗,空中开花,拉烟…… 
        舒克气得咬牙切齿。 
        “舒克,贝塔叫你去一趟。”餐厅主任罗丘爬过来对草丛中的舒克说。 
        舒克来到伤员住的地下室,贝塔在这儿躲空袭。 
        “把坦克前边垫起来,炮就能往空中打了。”贝塔说。 
        舒克眼睛一亮,这倒是个办法。 
        “臭球,跟我来!”舒克招呼。 
        “抬着我去!”贝塔叫道,“你不会开坦克。” 
        “你的腿动不了,怎么开?我早看会了。”舒克跑出去。   \chapter{第40集} 
        舒克驾驶坦克险些撞墙; 
        坦克变成高射炮; 
        贝塔击中海盗   
        贝塔对医生说:“快抬着我去,他开不了坦克。别以为会开飞机就什么都会开。” 
        医生说:“你这伤可开不了坦克。” 
        贝塔急了:  “我坐在他旁边指挥他,快抬我去!” 
        医生忙叫护上抬着贝塔去追舒克。 
        轰炸后,扫射又开始了。 
        舒克和臭球好不容易来到停放坦克的库房旁,臭球打开库房门。 
        坦克已蒙上了一层尘土。 
        舒克打开坦克舱盖儿,钻进去。 
        臭球站在炮塔旁边,从舱口往里看舒克操纵。 
        舒克回忆着从前看贝塔开坦克的情景,他按了一个按钮。 
        坦克“呼”地一下子冲出车库,差点儿把臭球甩下来。舒克不知道怎么停,坦克朝候机大楼撞过去。 
        “按蓝色的按钮!”贝塔躺在担架上喊。 
        臭球转告给舒克。 
        坦克在距候机大楼1厘米的地方停住了。 
        贝塔坐着担架过来了。 
        “把我抬到坦克里边去。”贝塔对hushi说。 
        臭球帮助hushi把贝塔塞进坦克,坐在舒克身边。 
        “我指挥你。”贝塔冲舒克挤挤眼睛,“飞行员不一定什么都能开。” 
        舒克耸耸肩,无话可说。 
        “臭球,快去堆一个斜土坡,角度大点儿。”贝塔抬头对趴在舱盖上的臭球说。 
        “当心空袭!”舒克提醒臭球。 
        “这个按钮是倒车,这个按钮是启动,这个按钮是加速,这个是制动……”贝塔教舒克。 
        尽管舒克觉得坦克兵教飞行员有点那个,可这是战争时期,他也顾不上面子了。 
        “你试着开开。”贝塔说。 
        舒克觉得“试”字挺刺耳,他撇撇嘴。 
        坦克在原地来了个360度旋转,再来个180度转弯,向前开去。 
        贝塔在心里不得不佩服舒克,毕竟是飞行员出身,学起来就是快。 
        海盗的歼击机继续在机场上空横行霸道,跑道上布满了弹坑。 
        舒克通过潜望镜看见臭球他们已经把土坡筑好了。 
        坦克朝土坡开去。 
        “掌握好速度!”贝塔叮嘱舒克。 
        坦克驶上了土坡。车身几乎垂直。 
        臭球指挥机场上作人员拿石头顶在坦克的尾部。 
        海盗在天上发现了坦克。 
        “攻击那辆坦克!”海盗F令。 
        歼击机轮番向坦克俯冲。炮弹打在坦克四周。 
        贝塔也用瞄准镜瞄准了正在俯冲下来的海盗的飞机。 
        双方同时按下射击按钮。 
        坦克被击中了。海盗的飞机也被击中了。 
        飞机冒着烟往下掉。海盗不敢跳伞,他知道被抓住下场好不了。他一边招呼部下返航,一边强行操纵飞机向机场以外的地域滑翔。 
        坦克也起火了,舒克忙打开舱盖儿,可坦克是垂直的,贝塔腿上有伤,出不去。 
        消防车来了,扑灭了坦克的火。 
        救护车停在坦克旁边,hushi们把贝塔从坦克里抬出来。 
        舒克从坦克里钻出来,满脸是灰。 
        空中小姐跑过来说:“旅客们问什么时候通航?” 
        “机场暂时关闭。”舒克决定。他相信,海盗明天还会来捣乱。 
        “通知全体人员,到会议室开紧急会议。”舒克告诉臭球。 
        一听说开会商量对策,工作人员们争先恐后地赶到会议室。他们都希望早点儿想出办法来治治海盗。 
        “大家出出主意。”舒克说。 
        “咱们应该有高射炮。” 
        “咱们也应该有战斗机。” 
        “咱们……” 
        大家七嘴八舌。 
        “到哪儿去弄高射炮和战斗机呢?”舒克为难地说。 
        “去找皮皮鲁。”臭球提议。 
        舒克服睛一亮,对,去找皮皮鲁!   \chapter{第41集} 
        舒克和臭球开直升机到皮皮鲁家; 
        皮皮鲁送给舒克吸铁石; 
        海盗拦截直升机   
        趁着天黑,舒克和臭球驾驶直升机进城去找皮皮鲁。 
        直升机降落在皮皮鲁家的阳台上。 
        “你在飞机上等着,我进去看看。”舒克对臭球说。 
        “当心点儿。”臭球看看飞机外边。 
        舒克轻轻打开飞机舱门,蹑手蹑脚地溜下飞机。 
        阳台门关着。舒克顺着墙爬上窗台,他看见屋子里皮皮鲁正在看电视。 
        舒克使劲撞窗户,皮皮鲁听见声响回过头来,他看见了窗台上的舒克。 
        皮皮鲁“腾”地从椅子上蹦起来,打开阳台门。 
        “真想你呀!那架大飞机怎么样?运了几次旅客啦?”皮皮鲁提了一连串问题。 
        舒克走进屋里,叹了口气。 
        “怎么啦?”皮皮鲁看出舒克情绪不好。 
        “海盗不知从哪儿弄来一些战斗机,拦截我们的客机,他们还空袭机场。”舒克说。 
        “就是你上次跟我说过的那个海盗?”皮皮鲁问。 
        “嗯。”舒克点点头,“帮我们想想办法吧!” 
        皮皮鲁眼睛盯着电视,在想。 
        “帮我搞几架歼击机。”舒克说。 
        “现在我这儿没有,得等明天晚上。”皮皮鲁说,“可明天早晨海盗又会去袭击你们呀!” 
        “高射炮也行。”舒克说。 
        “高射炮现在也没有。”皮皮鲁摇头。 
        舒克绝望了。 
        “有办法了。”皮皮鲁一拍腿。 
        舒克兴奋。 
        “我这儿有几块吸铁石,你把它们带回去,安放在机场上,架起来。准能把海盗的飞机吸住。”皮皮鲁说。 
        “能行吗?”舒克不大相信。 
        皮皮鲁打开抽屉,拿出几块圆形的黑磁铁,放在地上,他又从图钉盒里掏出一把图钉,朝吸铁石扔过去。 
        图钉都被磁铁石吸过去了,牢牢地依附在上边。 
        舒克乐了。 
        “你用直升机把这几块吸铁石运回去,明天我去给你搞战斗机。”皮皮鲁说。 
        “这磁铁会把直升机吸住吧?”舒克担心。 
        “每次运一块,别装在飞机里。吊着,绳子放长点儿。”皮皮鲁说。 
        “现在就运。”舒克迫不及待。 
        “歇会儿,咱们聊聊天。”皮皮鲁不想让舒克现在就走。 
        “等我打败了海盗,来陪你聊三天。”舒克说。 
        皮皮鲁无奈,他从抽屉里找出一捆塑料绳。 
        舒克和皮皮鲁来到阳台上,皮皮鲁帮助舒克捆好吸铁石。 
        “你把飞机悬停在空中,我把绳子给你系在飞机上。”皮皮鲁说。 
        舒克钻进直升机。飞机升到空中,在离皮皮鲁鼻子不远的地方悬停住。 
        皮皮鲁把绳子系在机身下边的铁环上。 
        “起飞吧!”皮皮鲁招手,“一会儿再来运!” 
        舒克冲皮皮鲁招招手。直升机离开阳台,返航了。 
        吊着这么一块大磁铁,直升机飞得很吃力。加上绳子太长,只能慢慢飞。 
        “注意观察。”舒克叮嘱臭球。 
        臭球困得都快睁不开眼睛了,他使劲儿掐自己的耳朵。 
        “贝塔,贝塔,我是舒克!听见了吗?请回答!”舒克要了解一下机场的情况。贝塔虽然受伤了,但他的病床放在塔台上。 
        “我是贝塔。请讲。” 
        “我已接近机场,可以着陆吗?” 
        “可以。”贝塔说。 
        舒克操纵飞机朝机场飞去。 
        “注意!”臭球大叫一声。 
        舒克往前一看,一群星星在他眼前飘行。 
        “是什么?”舒克把飞机悬停住。 
        臭球揉揉眼睛。 
        “飞机!”臭球脱口而出。 
        海盗的飞机!他们在这里等着舒克。原来,海盗的飞机被击中后,他凭着高超的驾驶技艺,硬是把伤机开回了他的机场。逃跑中他没忘了留下一架飞机侦察舒克的情况。当他得到情报说舒克驾驶直升机进城后,就带领自己的飞行大队埋伏蹲守在空中,等候舒克。 
        “海盗还会飞夜航!”舒克咬咬牙。 
        “咱们快把磁铁扔了吧?”臭球边说边掏出小刀,准备割绳。 
        的确,吊着磁铁太不灵活。 
        “别割!”舒克制止臭球,  “说不定能吸住敌机!” 
        “吸住了咱们也吊不动呀!”臭球说。 
        “那就往下掉,反正是先摔它。”舒克说。 
        臭球觉得有道理。但是够冒险的。   \chapter{第42集} 
        吸铁石“击落”两架歼击机; 
        海盗飞行大队撤退; 
        舒克驾机平安着陆   
        海盗的飞机压过来了,它们对舒克的直升机形成了一个包围圈。 
        “臭球,你到后舱去拿两个伞包,万一不行咱们就跳伞”舒克说。 
        臭球站起身,摸黑来到后舱,取出两个降落伞包,拎到驾驶舱。 
        一架歼击机冲直升机开火了,炮弹擦着直升机飞过去。好险。 
        “下边飞过来一架!”臭球把脸贴在玻璃上往下看。 
        “你指挥!”舒克授权臭球指挥他。 
        “拉杆!”臭球根据下边的敌机与吸铁石的距离指挥舒克。 
        直升机向上升去。吸铁石与敌机平行了。 
        “去看看直升机下边吊的是什么东西?”海盗命令部下。 
        那架飞机朝吸铁石靠拢。突然,它感到自己失去了控制,身不由己地朝吸铁石贴过去。 
        吸铁石把歼击机吸住了。就在同时,由于重量猛增导致直升机急剧下降。 
        臭球慌了,忙背上伞包。 
        直升机迅速朝地面坠落,只听“轰”的一声,贴在吸铁石下边的歼击机撞地爆炸了。 
        就在爆炸的同时,舒克操纵直升机拉起了高度。 
        叉一架歼击机靠过来。舒克驾驶直升机主动迎上去。 
        “降低高度!”臭球干脆把头探出舷窗。 
        第二架歼击机又被吸铁石吸住了。 
        直升机急剧下降。“轰”的一声,吊在直升机下边的歼击机爆炸了。 
        直升机再次升到空中。 
        海盗傻眼了,他断定直升机装备了新式武器。一瞬间两架飞机报销了,海盗决定返航。 
        舒克和臭球看到敌机飞走了,高兴得哼起了进行曲。 
        “舒克,舒克,我是贝塔,请回答!”耳机里传出贝塔的呼叫。 
        “我是舒克,我是舒克,请讲!”舒克答话。 
        “出了什么事啦?怎么还没到机场?”贝塔不放心了。 
        “刚才我们遇到了海盗的拦截。我们打掉了他两架飞机!”舒克报捷。 
        “真的?拿什么打的?” 
        “拿磁铁!” 
        “磁铁?” 
        “就是吸铁石!” 
        “吸铁石?”贝塔还是不明白。 
        “回去你就知道了。我们现在返航。”舒克告诉贝塔。 
        lO分钟后,舒克的直升机出现在机场上空。 
        地勤人员指示出直升机着陆的地点。 
        “贝塔,你通知地勤,把绳子剪断。我再换个地方着陆。”舒克同贝塔联系。 
        贝塔立即让hushi通知地勤人员。 
        直升机吊着吸铁石成功地着陆了。   \chapter{第43集} 
        吸铁石构成防空网; 
        海盗向自己的飞机开炮; 
        海盗驾机同吸铁石展开空战   
        舒克和臭球驾驶直升机吊着吸铁石成功地在机场着陆后,贝塔躺在担架上赶到直升机旁边。 
        “要吸铁石干什么?”贝塔不明白。 
        “吸海盗的飞机!刚才我们就是靠吸铁石打掉海盗两架飞机的。”舒克得意极了。 
        贝塔恍然大悟。 
        “我们还得去皮皮鲁家运吸铁石,你指挥大家修几座圆柱形的高台子,把吸铁石固定上去。”舒克说完钻回直升机。 
        臭球和舒克驾驶直升机消失在夜色中。 
        贝塔立即将工程师叫来,吩咐他马上画图纸,设计吸铁石底座。 
        图纸画好了,施工开始。 
        天亮时,机场的四个角耸立起四座细长的建筑,每座建筑的顶部都安装着吸铁石。 
        舒克和臭球在餐厅用早餐,昨天夜里他俩整整空运了一夜吸铁石。 
        餐厅主任罗丘亲自给舒克和臭球端来了丰盛的饭莱。有面包、香肠、油炸花生米、银耳汤。 
        舒克和臭球大吃特吃。 
        “我从来没吃过这么香的饭。”臭球说。 
        “这得感谢海盗。”舒克抹抹嘴。 
        话音没落,空袭警报响了。 
        舒克兴奋得蹦起来,好像空投罐头的飞机来了似的。臭球也急不可待地往外跑。 
        天空中出现了海盗的机群,它们耀武扬威地在机场上空作着各种放肆的飞行动作。 
        “妈的,你们高兴不了几分钟了!”臭球跺跺脚。 
        海盗机群拉起了高度。两架歼击机编队向机场俯冲下来。 
        精彩的场面出现了。 
        飞在前面的长机突然偏离了航线,向右边栽下去,一头粘在吸铁石上。后边的僚机刚想去给长机保驾,忽然被一股看不见的神秘力量拉向左边,死死地钉在另一块吸铁石上。 
        在空中指挥袭击机场的海盗愣了,他在飞机里拼命呼叫: 
        “05——05——快拉起来!快拉起来!!” 
        “报告头儿,拉不起来!我被吸铁石吸住了!”05号机的飞行员回答。 
        “是吸铁石!”海盗明白了。 
        “他们来抓我了,快来救我!”两架被吸住的飞机的飞行员求救。 
        海盗往下一看,几十名机场工作人员朝两架被吸住的飞机跑去。 
        “不能让他们得到飞机,快摧毁它们!”海盗发狠了,命令部下自己打自己的飞机。 
        几架歼击机开始远距离地向粘在吸铁石上的飞机开火,可惜距离太远,打不中。 
        “近点儿!”海盗命令。 
        又一架飞机被吸上了。 
        海盗急红了眼,他亲自驾驶飞机俯冲。他不能眼看着自己的三架飞机落到舒克手里。 
        舒克认出了海盗的飞机,他知道海盗有高超的驾驶技术,忙招呼大家隐蔽。 
        果然,海盗驾驶的歼击机像喝醉了酒一样摇摇晃晃地俯冲下来,他利用这种摇晃来摆脱吸铁石的磁力。 
        海盗开炮了。 
        一架粘在吸铁石上的歼击机着火了。飞机里的飞行员慌忙逃出座舱,可飞机离地面很远,他只好跳下来,摔断了腿。 
        着火的飞机爆炸了。吸铁石也被炸碎了。 
        海盗又开始向另一架粘在吸铁石上的飞机进攻。 
        舒克原想缴获海盗的几架歼击机武装自己,现在眼看着愿望就要落空,急得他直揪自己的胡子。 
        候机大楼的玻璃窗反射的阳光晃了一下舒克的眼睛。舒克笑了。 
        “快去叫大家拿镜子晃海盗的飞机!”舒克大喊。 
        大家纷纷跑回宿舍,拿来自己的镜子。 
        当海盗再次俯冲时,他只觉得地面上有几十道强光直射他的眼睛,刺得他睁不开眼睛,看不见目标。 
        海盗失去了控制,身不由己地向下边坠落。 
        “糟糕,吸铁石!”海盗一惊,他拼尽全力扳驾驶杆,飞机和吸铁石之间展开了一场力的搏斗。 
        吸铁石想把飞机吸下来,飞机想摆脱吸铁石,双方僵持住了。 
        这是空战史上的奇迹。海盗的歼击机像直升机那样停留在空中。 
        海盗毕竟是海盗,靠他强大的臂力,把飞机一点儿一点儿向空中挪。 
        精彩的场面出现了。吸铁石离开了底座,升到空中,朝海盗的飞机扑去。 
        只听“通”的一声,吸铁石吸在海盗的飞机下边。 
        海盗的飞机急剧下降,在坠地的一刹那,又吃力地昂起机头。海盗艰难地拉起高度,率领着部下,返航了。   \chapter{第44集} 
        直升机吊歼击机; 
        舒克驾驶歼击机试飞; 
        两名飞行员俘虏瞠目结舌   
        海盗的机群败退了,被他们遗弃的两架歼击机孤零零地挂在吸铁石上。 
        “我和他们通话。”舒克跑到塔台上,打开无线电通讯设备。 
        “我是舒克,请回答!”舒克冲两架挂在吸铁石上的飞机里的飞行员喊话。 
        不回答。 
        “你们的头儿都冲你们开炮了,你们还这么死心塌地!”舒克说。 
        “你准备怎么处置我们?”飞行员答话了。 
        “放你们走,要你们的飞机。”舒克说。 
        “……”对方显然不信。 
        “我现在就驾驶直升机把你们的飞机吊到地面上,请你们配合。”舒克说完把话筒交给贝塔,“你指挥。” 
        舒克驾驶直升机升到空中,悬停在一架歼击机上空。臭球打开舱门,扔下绳索。 
        几名机场工作人员已爬上吸铁石塔,把绳索捆在歼击机上。 
        “上升!”贝塔在塔台指挥。 
        舒克对能不能把歼击机吊到空中心里没底,他使劲儿拉操纵杆,直升机纹丝不动。 
        “我来试试。”臭球从机舱走进驾驶舱。 
        舒克离开驾驶员的座位。 
        臭球把脸都憋红了,直升机还是不动。 
        舒克从驾驶舱探出头往下看,歼击机是头朝上粘在吸铁石上的。 
        “有办法了!”舒克拿起话筒。 
        “海盗的部下,我是舒克,请回答!” 
        “我是海盗的部下,请讲!”尽管海盗的部下不喜欢这个称呼,可也顾不上更正了。 
        “打开你的发动机!”舒克下令。 
        “干什么?”海盗的部下觉得这命令莫名其妙,粘在吸铁石上的飞机启动发动机干什么? 
        “让你开你就开!”舒克没时间向他解释。 
        歼击机的发动机启动了,喷气管向地面喷射出强大的气流。臭球明白了,原来舒克是让歼击机也一起往上使劲。 
        直升机把歼击机和磁铁一起吊到空中。地勤人员挥舞着红旗指挥直升机在草坪上着陆。 
        歼击机和磁铁先着陆了。地勤人员解开绳索。歼击机座舱里的飞行员被押离飞机。 
        舒克和臭球又把另一架歼击机吊到地面上。 
        两架歼击机完好无损地并排停在停机坪上,地勤们围着飞机看。 
        舒克来到歼击机旁,他钻进座舱。 
        歼击机的座舱设备与直升机和客机大同小异。舒克知道歼击机速度快,驾驶难度高,但他知道海盗一会儿还得来袭击机场,他想驾驶战斗机同海盗打空战。 
        “去把俘虏叫来。”舒克吩咐。 
        两名飞行员俘虏来了。 
        “说说这飞机的性能。”舒克说。 
        “……”俘虏不开口。 
        “说了就放你们走。”舒克说。 
        两名俘虏对看了一眼。 
        “说话算数。”臭球在一旁帮腔。 
        一名俘虏开口了,他把飞行速度、高度、航程、转弯半径等飞行数据告诉了舒克。 
        “等我平安着陆,就放你们走。”舒克关上座舱盖,他要试飞。 
        “我是贝塔!我是贝塔!请回答。”舒克的耳机里传出贝塔的声音。 
        “我是舒克,请求试飞。” 
        “太危险,你还不了解这飞机的性能。” 
        “我问过了。” 
        “不行!”贝塔坚决不同意。 
        “一会儿海盗还会来,咱们不能老是被动挨打呀!”舒克急了。 
        “干万当心!”贝塔只好同意。 
        “放心吧,我是老飞行员了。”舒克又摆出大飞行员主义。 
        歼击机徐徐滑上刚抢修好的跑道,稳稳地停在起飞线上。 
        “舒克请求起飞。”舒克向塔台报告。 
        “同意起飞。”贝塔回答。 
        舒克给发动机增大了转速,歼击机轰鸣着向前冲去.像离弦的箭。 
        机头昂起了,机轮离地了,机身腾空了。 
        机场上一片欢呼。 
        舒克觉得歼击机太灵活,他还不大习惯,他仔细地揣摩。 
        渐渐地,舒克掌握了驾驶要领,他准备翻个筋斗。当了这么长时问飞行员,舒克还没在天上驾机翻过筋斗。这两天看着海盗在天上翻来翻去,他早眼红了。 
        “舒克请求翻筋斗!”舒克请求贝塔。 
        “翻吧!”贝塔已认定舒克是飞行天才了。 
        一个漂亮的前滚翻。 
        连两个俘虏也佩服得五体投地,要知道这个动作他们整整练了一个月! 
        “舒克请求着陆!”舒克知道歼击机着陆难度非常大。 
        “同意着陆。”贝塔回答。 
        就在这时,舒克看见前方出现了海盗的机群。 
        海盗正准备袭击舒克的机场,忽然看见了自己部下的飞机,很是高兴。他之所以这么快回来,就是怕舒克学会开歼击机。   \chapter{第45集} 
        舒克同海盗飞行大队展开激烈的空战: 
        舒克假装被击落; 
        臭球驾机升空参战   
        舒克见海盗误会了,决定将计就计。 
        舒克在驾驶杆上找到了射击按钮,他做好了射击准备。 
        “归队!”海盗通过电台命令舒克。 
        “明白!”舒克假装答应。 
        舒克的歼击机朝海盗靠拢,他用瞄准具的光环套住了海盗的飞机,只要把海盗打掉,海盗大队就 垮了。 
        距离海盗越来越近。舒克已经看见海盗的头了。 
        海盗已经觉察到自己的这位部下不大对头,从飞行姿态上可以看出不是他训练出来的飞行员。 
        就在这时,舒克按下了射击按钮,一串炮弹拖着蓝色的尾烟向海盗的座机射去。 
        海盗猛一拉杆,飞机来了个筋斗.炮弹擦着飞机肚皮钻过去,击中了另一架飞机。 
        那架飞机螺旋式地朝下栽去。 
        舒克不放过机会,他对准机群猛打一顿,又有几架飞机被击中。 
        “快散开!”海盗在电台里拼命喊。 
        机群迅速散开了,它们对舒克形成了一个包围圈。 
        舒克现在是无路可逃。如果着陆,肯定被击落。舒克心里清楚,他打掉了海盗四架飞机,海盗不会饶了他的。 
        果然,海盗组织进攻了。 
        两架歼击机从左边飞过来.两架歼击机从右边飞来。舒克发现下方也有。 
        “舒克注意!后边也有!”贝塔在塔台里提醒舒克。 
        机场上的工作人员都为舒克捏了一把汗,这是一场力量悬殊的空战,况且舒克是头一次驾驶歼击机。 
        舒克背好伞包,作好了随时跳伞的准备。他严密注视着四面八方的敌机动态。 
        右边的那两架敌机进入了最佳射击距离和角度,舒克断定它们就要开火了。 
        舒克猛一推驾驶杆,飞机朝地面俯冲下去。在这同时,那两架敌机开炮了。 
        海盗和部下们还以为舒克被击中了,他们高兴得直抖机翼。 
        只见舒克的飞机在坠地之前拉了起来,机身几乎是垂直向上升去。 
        地面上的尘土被吹起老高。 
        舒克发现头顶上正有一架敌机,他果断地按下射击按钮。 
        敌机被击中了,冒着黑烟栽下去。 
        海盗勃然大怒。又上了舒克的当!他恨不得把舒克吃了,他要亲手打掉舒克。 
        海盗的飞机朝舒克逼过来。 
        舒克现在已经完全掌握了歼击机的驾驶要领,他自如地操纵着飞机,躲避着敌机的攻击。 
        地面上的臭球见舒克孤军奋战,决定助他一臂之力。 
        臭球跨进停机坪上另一架缴获的歼击机的座舱。 
        “贝塔,贝塔,臭球请求起飞!”臭球向塔台报告。 
        “你要干什么?”贝塔问。 
        “我去支援舒克。” 
        “可你不会开歼击机呀!” 
        “舒克也不会!” 
        “他比你……”,贝塔想说“他比你聪明”,后半截话咽回肚里了,太伤人。 
        臭球不管那么多了,他启动了发动机。 
        歼击机滑向跑道。 
        “背好伞包。”贝塔提醒臭球。 
        “谢谢。”臭球把屁股下边的伞包的带子系在肩上。 
        “臭球请求起飞。”臭球请示塔台。 
        “同意起飞,注意安全。”贝塔咬咬牙。 
        舒克在空中看见了臭球。 
        “握紧驾驶杆,慢慢往后拉!”舒克提示臭球。 
        海盗听见了舒克的话,他往下一看,糟糕,又有一架歼击机要来参战。 
        “打掉它!不让它升空!”海盗命令部下。 
        两架敌机呼啸着朝正在跑道上滑行的臭球扑去。 
        “快加速!”舒克冲着话筒喊。 
        臭球把发动机的转速增到最大,飞机像脱缰的野马一样在跑道上疾驰。天空中,两架敌机从上而下地压在臭球的飞机上空,阻止他起飞。 
        臭球的飞机的机轮离地了。 
        “打掉它!”海盗急了。 
        舒克趁海盗分心的时机,摆脱了他的追击,一推驾驶杆去增援臭球。臭球头顶上的两架敌机死死压住臭球。臭球急了,猛一拉杆,飞机突然上升,险些撞上那两架敌机。吓得两架敌机忙闪开。 
        “笨蛋!”海盗骂人了。 
        “好样的!”舒克表扬臭球勇敢。 
        臭球得意了,他想翻个筋斗,就像刚才舒克翻的那样。 
        海盗清点部下,还有14架飞机。 
        十几只老鼠,驾驶着歼击机在天上进行一场史无前例的空战。   \chapter{第46集} 
        臭球在空中现眼; 
        舒克保护臭球跳伞; 
        舒克的座舱盖被海盗击落   
        臭球想向大家证明一个道理:他的智商不比舒克低。他决定驾驶歼击机也在空中翻一个筋斗,给贝塔他们看看,也让海盗他们知道知道。 
        臭球一拉杆,飞机先是往上升,可没升多久,就往下掉,接着是头朝下进入螺旋状态。 
        “失速!”舒克大叫一声。飞行员都知道“失速”是飞行中最可怕的事情。 
        臭球的飞机急速下降。 
        “臭球,快跳伞!”舒克通过电台大喊。 
        臭球只好放弃了飞机,弹射跳伞。 
        歼击机撞地后发出巨响。 
        臭球的降落伞在空中张开了。 
        “打他!”海盗下令打悬在降落伞下的臭球。 
        几架飞机朝臭球逼过去。 
        舒克火了。海盗这足公然违反国际上有关空战的规定:跳伞后的飞行员失去了武装,不准向他射击。 
        舒克从左下方向袭击臭球的敌机开火了。一架飞机起了火,飞行员也跳伞了。 
        舒克喊话: 
        “海盗!如果你打我们的人,我就打你们的人!” 
        舒克威胁海盗。现在天上飘着两顶降落伞,一边一个。 
        “你打吧!”海盗恶狠狠地说,“反正我要打你的。” 
        舒克这回彻底知道了海盗的心有多狠,他决定正面同海盗展开空战,拼了。舒克在歼击机的飞行技术上一直怵海盗,现在他顾不了那么多了。 
        海盗见舒克的乜机向他扑来,高兴了。他就想让舒克跟他拼,他认定自己的飞行技术是世界一流水平。 
        “你们去打降落伞!”海盗命令部下。 
        奇怪的是海盗的部下都不去打臭球。他们知道自己也有跳伞的那天,都觉得头儿做得太绝情。 
        臭球平安着陆了。救护车开到他身边。 
        臭球满脸通红,羞愧万分。 
        天上,舒克正同海盗展开一场惊心动魄的空战。 
        海盗拿出江湖大侠的派头,命令部下在一边巡航,不得参战。他要同舒克单练。 
        舒克握驾驶杆的手心出汗了,他明白这是生死之战。即便是他被击中跳伞,海盗也会在空中击毙他。 
        “我让你先打我。”海盗嘲笑舒克。他的飞机故意飞在舒克的正前方。 
        舒克用瞄准具的光环套住海盗的飞机,他按下了射击按钮。 
        海盗轻轻一抬机翼,炮弹从机身下边飞过去。 
        “现在该我打你了!”海盗掉转机头。 
        舒克不得不飞到海盗前边让他打。 
        舒克认定海盗或向他的机身下边开炮或向他的机身上边开炮,往上躲还是往下躲呢?舒克犹豫着,这简直像足球比赛发点球时的守门员心理。 
        “往下。”舒克碰运气。 
        海盗开炮了。 
        与此同时,舒克操纵飞机下降高度。 
        一发炮弹掀掉了舒克的座舱盖,强大的气流吹得舒克睁不开眼睛。 
        海盗来劲儿了,他驾机继续向舒克进攻。 
        舒克一边把飞行帽上的风镜拉下来,一边躲避着海盗的攻击. 
        地面上的小老鼠们都急了。贝塔恨自己不能上天帮助舒克。餐厅主任罗丘挥舞着勺子冲着天上的海盗大骂。臭球在一旁揪自己的耳朵生自己的气。 
        舒克想,如果自己在机场上空被击落,海盗准还得袭击机场,不如干脆把海盗引开。 
        舒克驾驶飞机朝远处飞去,海盗紧追不舍。 
        “你快投降吧,舒克!”海盗向舒克喊话。 
        “有本事你打掉我!”舒克说。 
        海盗射出一串炮弹。 
        舒克回头看看,十几架敌机都跟在后边。 
        舒克的头被风吹得发晕,他只顾加速往前开,也不知开到哪儿了。 
        “你到城里去找死呀!”海盗骂舒克。 
        舒克低头一看,下边是城市,皮皮鲁居住的城市。 
        海盗怕城市。他曾经差点儿在城里送命。海盗决定在城外于掉舒克。 
        舒克的飞机由于失去了座舱盖,速度受到影响。海盗们追上来。 
        “所有飞机都瞄准敌机。”海盗下令。 
        十几架歼击机同时瞄准了舒克的飞机。 
        舒克左右摇晃飞机,给敌机的瞄准制造困难。 
        到城市上空了。 
        “开炮!”海盗不敢飞临城市上空,他下令。舒克的飞机被击中了好几处,冒烟了。舒克发现飞机操纵失灵,他跳伞了。 
        海盗断定舒克降落后一定会被人抓住处死或被猫吃掉。他招呼部下返航。 
        舒克操纵降落伞寻找合适的着陆点。   \chapter{第47集} 
        舒克降落在湖中; 
        舒克乘坐摩托车; 
        舒克当了空军司令   
        舒克看见下边是一个公园,公园里有一个湖。舒克觉得落在水里比落在地上安全。 
        他操纵降落伞朝湖中心降落。 
        在落水之前,舒克给救生艇充了气。 
        舒克的身体落在湖水里,他从降落伞下边钻出来,爬进救生艇。 
        四周静悄悄的。舒克趴在救生艇里向四面张望,没发现不安全因素。 
        救生艇靠岸了,舒克看看岸上没人,迅速爬上岸。 
        舒克爬上一棵大树,判断着皮皮鲁家的方位。过去他都是开飞机来,没从地面走过。 
        舒克看见了装有大钟的楼,他曾经和贝塔拨过那座钟的表针。舒克知道皮皮鲁家的方向了。 
        白天行走是危险的,但舒克一想到海盗还会去袭击机场,就豁出去了。他溜到树下,朝皮皮鲁家的方向跑去。 
        出了公园,来到大街上。舒克藏在一根水泥管子后边。街上人来人往。舒克寻找着机会。 
        一辆摩托车停在水泥管子旁边,骑车的小伙子下车买东西。 
        舒克无法断定这辆摩托车往哪边开,他只有碰运气了。舒克钻进摩托车侧盖里边,在机油箱上坐好。 
        摩托车启动了。舒克抓紧一根电线。 
        也不知开了多长时间,车停了。舒克听见两个人在谈话。 
        “修车吗?” 
        “嗯。” 
        “什么毛病?” 
        “烧机油。” 
        “把侧盖打开。” 
        有人用螺丝刀拧侧盖的螺钉。 
        舒克慌了,他无路可逃。眼看侧盖就要被打开了,舒克只得使劲儿钻进机油箱后边。这里的间隙很小,舒克的身体被压成了馅饼状。 
        侧盖打开了。 
        “这机油什么时候加的?” 
        “前天。” 
        “是够费的。放在这儿修吧。” 
        “什么时候取车?” 
        “后天。” 
        原来是修车店。 
        小伙子走了。舒克悄悄探出头,看见修理工去修另一辆摩托车。 
        舒克蹑手蹑脚地溜下摩托车,一步三回头地朝门口溜去。他光顾得回头看那修理工,不留意碰翻了一个玻璃瓶。 
        修理工一回头,看见了舒克。他大喝一声。 
        舒克“噌”的一下钻出门外,没命地跑。修理工根本就没追出来。 
        一场虚惊,舒克站住定定神。 
        当他看清自己周围的建筑物时,笑了。这是皮皮鲁家。 
        舒克藏进皮皮鲁家所在的单元门口的花坛里,等着皮皮鲁放学归来。 
        这两天真累。舒克睡着了。 
        舒克做了一个梦。他梦见自己当了空军司令,统率着几千架战斗机。当海盗的机群来犯时,他一声令下,几千架战斗机腾空而起,遮天蔽日。 
        止当他在空中指挥战斗时,耳机里传出了皮皮鲁的声音。 
        舒克睁开眼睛,透过草丛,他看见皮皮鲁正和同学们一边打闹一边朝单元门口走去。 
        舒克知道错过这个机会就完了,他冲出花坛,站在皮皮鲁面前。 
        “老鼠!”一位同学惊叫起来。 
        皮皮鲁定睛一看,是舒克!忙蹲下把舒克捡起来。 
        “你……你抓老鼠!”那位同学直皱眉头。 
        皮皮鲁冲他一笑,拿着舒克朝楼上跑去。 
        到了屋里,皮皮鲁把舒克放在桌子上。 
        “快说,吸铁石管用吗?”皮皮鲁一直挂念着舒克的机场。 
        “管用!吸住了海盗的好几架飞机。”舒克接着把今天的激烈空战讲给皮皮鲁听。 
        皮皮鲁听呆了。 
        “明天你就不用怕海盗了。”皮皮鲁说。 
        “为什么?”舒克预感到有好事。 
        “你看!”皮皮鲁打开他的抽屉。 
        抽屉里全是战斗机,足足有20架。 
        “这是歼击机。”皮皮鲁拿出几架放在桌上。 
        “这是强击机。” 
        “这是轰炸机。” 
        “这是侦察机。” 
        现在舒克毫不怀疑自己真要当空军司令了。 
        “这么多飞机,我怎么开回去?”舒克为难了。 
        “别急,先吃饭。一会儿我骑自行车送你回去,把飞机都运去。”皮皮鲁说。 
        舒克在皮皮鲁家美美地吃了一顿饭。 
        舒克贝塔航空公司和海盗飞行大队之问将展开一场真正的国际水平的空中大战。   \chapter{第48集} 
        皮皮鲁训练战斗机飞行大队; 
        舒克兼任战斗机飞行大队长; 
        臭球干着急   
        舒克在皮皮鲁家吃完饭后,皮皮鲁把书包里的课本倒出来,把战斗机都装进书包。 
        “委屈你在书包里呆会儿,我骑自行车把你和飞机送到你的机场去。”皮皮鲁对舒克说。 
        舒克告诉皮皮鲁机场的方位。 
        皮皮鲁趁爸爸妈妈还没下班,骑自行车直奔郊外。 
        舒克不时从书包里探出头给皮皮鲁指路。 
        在一座小山坡后边,皮皮鲁找到了舒克的机场。 
        “真壮观呀!”皮皮鲁看着脚下这座微型飞机场,直吐舌头。 
        “你看,那么多建筑都被海盗的飞机炸坏了。”舒克从书包里探出头来。 
        “没关系,你马上就能击败他!”皮皮鲁信心十足。 
        皮皮鲁打开书包.先把舒克放在地上,然后将战斗机一架一架放在机场的停机坪上。 
        舒克朝塔台跑去。 
        贝塔已经在塔台里看见了皮皮鲁,他的伤好些了,已经能下地行走。 
        “快扶我去见皮皮鲁。”贝塔对舒克说。 
        舒克扶着贝塔来到皮皮鲁身边。 
        “光荣负伤啊!”皮皮鲁蹲下对贝塔说。 
        贝塔耸肩。 
        “我来帮你们训练战斗机。”皮皮鲁说。 
        “太好了,我去挑选飞行员。”舒克跑回机场。 
        机场全体工作人员集合。 
        舒克讲话。 
        “现在,咱们要成立战斗机飞行大队,为运输机护航。想当战斗机飞行员的举手!” 
        “呼啦”,都举手了。 
        舒克挑选了十几名身强力壮的。 
        “怎么没我?”臭球对自己落选感到意外。 
        “你负责开运输机。”舒克拍拍臭球的肩膀。 
        “人选的飞行员留下,其他人各就各位,准备飞行的地面工作。”舒克一挥手,大家散开了。 
        场务组去扫跑道,油车、电瓶车穿梭不息,机务人员检查飞机,气象人员报告风向、风力,餐厅的大师傅们准备可口的饭菜……机场一派繁忙景象。 
        舒克给新飞行员简要地讲述了飞行原理和操作要领,然后分配飞机。 
        舒克兼任战斗机飞行大队队长,下设歼击机飞行中队,轰炸机飞行中队和强击机飞行中队。 
        训练开始了,几十架飞机整齐地排列在起飞线上。皮皮鲁站在机场旁边像一座巨塔,指挥飞行。 
        舒克钻进一架最新式的喷气式超音速可变翼歼击机里,他关上了座舱盖儿。 
        “舒克请求起飞!”舒克请示。 
        皮皮鲁对着话筒说:  “同意。” 
        歼击机滑上了跑道。舒克感到这飞机很好操纵。转眼间,歼击机插上蓝天。 
        “俯冲!” 
        “翻滚!” 
        “拉杆!” 
        皮皮鲁不时发出指令。舒克的歼击机作着各种高难度的飞行动作。 
        “这飞行技术,没治了!” 
        皮皮鲁连连点头。 
        “01号要求返航。”舒克请示。 
        “同意!” 
        舒克驾驶歼击机绕机场一周,通过四转弯降低高度,对准了跑道。 
        一个漂亮的着陆动作,机轮挨地时连烟都没冒。 
        第二架歼击机准备起飞了,这是新飞行员。飞机摇摆着冲上了跑道,像喝醉了酒似的在跑道上蹦跳着向前冲去。 
        飞机离开了地面,飞行员拼命往后拉驾驶杆,想让飞机快些升高,没想到拉杆过猛,造成了飞机失速。 
        飞机急剧向地面驶去。眼看就要机毁鼠亡。在这紧急关头,皮皮鲁伸手接住了飞机。 
        一场空难避免了。 
        舒克突然意识到了有皮皮鲁在的优点:皮皮鲁可以甩手扶着飞机飞行,不断纠正新飞行员的动作,还可以保证不摔飞机。 
        飞机训练开始了。飞机一架接一架地起飞,皮皮鲁一会儿纠正这架飞机错误的飞行动作,一会儿抢救那架出故障的飞机,忙得不可开交。 
        经过两个小时的训练,新飞行员们已经熟练地掌握了战斗机的驾驶技术。 
        臭球在地面心里痒痒得直搓手心。 
        紧接着又进行了编队、射击、轰炸投弹等科目训练。 
        天快黑时,皮皮鲁对舒克说: 
        “行啦,你的战斗机飞行大队可以参加空战了我该回家了,祝你们胜利!” 
        皮皮鲁骑车回城丁。   \chapter{第49集} 
        歼击机为客机护航: 
        海盗飞行大队与舒克战斗机大队展开空战; 
        舒克摆脱导弹   
        晚上,舒克召集全体机场工作人员开会,他宣布从明天起,舒克贝塔航空公司正常空运。 
        “由臭球负责空运,我负责护航和保卫机场,贝塔负责地面指挥。”舒克像个军事家。 
        “除了值班的,大家都早点儿休息。”贝塔关照大家。这几天,谁也没休息好。 
        第二天.当太阳刚露出笑脸,机场上就忙开了。 
        大客机停在候机大楼旁边,闻讯而来的旅客挤满了候机大厅,他们坐过飞机后,再坐什么都嫌慢。这几天关闭机场,把他们急坏了。 
        旅客们依次登上客机,空中小姐站在机舱门口为旅客提供服务。 
        战斗机也作好了起飞的准备。 
        贝塔在塔台上注视着机场和空中的情况。 
        臭球启动皮皮鲁号的发动机。 
        “皮皮鲁号请求起飞。”臭球请示塔台。 
        “请战斗机为皮皮鲁号护航。”贝塔说。 
        “战斗机明白!”舒克坐在歼击机座舱里说。 
        “同意起飞!”贝塔发令, 
        皮皮鲁号滑上了跑道。六架歼击机护卫在皮皮鲁号的两侧。 
        七架飞机同时起飞,从形紧密,声如雷鸣,壮观无比。 
        机群进人了航线,臭球高枕无忧地打开自动驾驶仪,听筒里传出舒克的声音: 
        “臭球注意!臭球注意!发现敌机!” 
        臭球往前一看,几架海盗的歼击机正扑过来。 
        “05、06号掩护皮皮鲁号,其他飞机跟我出击!”舒克一声令下。 
        四架歼击机迎着海盗的飞机飞过去。 
        海盗没见过这种飞机,有点儿纳闷:这是谁的飞机?正当他犹豫的时候,舒克开火了。一架敌机被击中。 
        海盗这才清醒过来,忙招呼部下反击。 
        一场真正的空战开始了。一群老鼠驾驶着世界上最先进的飞机在空中搏斗。 
        海盗一共是10架飞机,舒克除去为客机护航的两架,还有四架,敌我力量悬殊。 
        “贝塔,贝塔,我们在6号空域与海盗相遇,请求增援!”舒克打开电台。 
        “马上增援。”贝塔立即命令待命的歼击机统统起飞。 
        海盗没想到舒克降落在城里没有死,而且居然在一夜之间组建了战斗机群,他心里暗暗打了个哆嗦。必须置舒克于死地,趁着力量悬殊。海盗发令: 
        “冲散他们的队形,单个打!” 
        海盗的部下操纵飞机冲进舒克的机群,舒克的飞行员驾驶技术还不熟练,一下就乱了阵脚。舒克回头一看,他的僚机已经不见了。 
        “02,02,你在哪里?”舒克呼叫。 
        “我在你的上方。” 
        舒克抬头一看,僚机跑到了长机的上边。 
        “跟着我!”舒克加速朝前飞,僚机降低高度跟上了。 
        舒克寻找海盗的飞机。 
        “长机注意!右前方有敌机!”僚机报告。 
        舒克一看,一架敌机鬼头鬼脑地正在瞄准他的飞机。 
        舒克一推驾驶杆,飞机俯冲下去。接着一拉杆,飞机又向上冲去。现在,那架敌机的肚皮正处在舒克的射击圈内。 
        舒克按下了射击按钮,三门大炮同时开火,敌机在空中开了花。 
        就在同时,舒克的僚机被海盗击中了。原来这是海盗使用的调虎离山计。 
        “快跳伞!”舒克命令僚机。 
        僚机飞行员弹射跳伞了。 
        舒克寻找着自己的另外两架飞机。他看见它们被敌机包围了。舒克驾机冲过去给他们解围。 
        海盗感到胜利在握了,他要亲手打掉舒克。 
        “舒克,你敢跟我决斗吗?”海盗通过无线电台问舒克。 
        “来吧!”舒克不含糊。 
        “咱们头对头飞,看谁能把谁打下来!”海盗恶狠狠地说。 
        “来!”舒克驾机转了一个弯。 
        海盗的飞机也转了过来。现在他们的飞机头对头,处在一个高度上。 
        两架飞机对着头互相飞过去。这样射击想命中难度很高。舒克用光环套住了海盗的飞机。海盗也用光环套住了舒克的飞机,两人都开炮了,谁也没打中对方。 
        眼看两架飞机要撞上了,舒克一拉杆,海盗一推杆,错开了。 
        “再来!”海盗说。 
        舒克二活没说,转弯,掉头。 
        两架飞机又开始了头对头的飞行。 
        这回海盗想发坏了,他的机翼下挂着一对空对空导弹,他一直舍不得使用它们,要知道,用导弹打目标几乎是不用瞄准的,导弹能自己跟踪目标。海盗决定用导弹击落舒克。 
        舒克正用瞄准具的光环套海盗,忽然看见从海盗的机翼下钻出两条火龙。导弹!舒克大惊失色。说时迟,那时快,舒克一拉杆,飞机成90度角向上升去。导弹也向上升去,紧紧咬住舒克的飞机。 
        舒克急了,他打开加力系统,飞机进入了超音速。导弹在后边跟着,只差五米! 
        飞机甩不掉导弹,导弹也追不上飞机。舒克不停地翻滚,俯冲,上升,仍然甩不掉。 
        海盗兴奋得直吹口哨。 
        舒克忽然想起这种导弹是采用红外线制导的,专咬发动机排气管的火焰,如果关闭发动机,导弹就会迷航。但空中停车是十分危险的,可只有此路一条了。 
        舒克在一个俯冲过后,关闭了发动机。 
        导弹失去了目标,它们经过短暂的迷失后,重新咬住了海盗的两位部下。 
        几秒钟后,海盗的两架飞机被自己的导弹击碎了。 
        海盗气得把驾驶杆拧成了拐棍。 
        舒克成功地在空中重新启动了发动机。 
        这时,增援的飞机赶来了。双方的力量对比发生了有利于舒克的变化。 
        海盗的飞机一架接一架地冒烟。   \chapter{第50集} 
        海盗逃跑; 
        庆功宴会; 
        舒克轰炸海盗机场   
        舒克率领部下把海盗的飞机打了个落花流水。当空中还剩下一架敌机时,舒克才发现海盗已经逃跑了。 
        舒克感到惋惜,他把那架敌机打掉后,打开电台同臭球联系。 
        “臭球,臭球!”舒克呼叫。 
        “我是臭球,请讲。” 
        “你在哪里?” 
        “我已平安到达目的地,现正准备起飞返回机场。” 
        “我们去为你护航。” 
        当皮皮鲁号客机在十几架歼击机的护卫下出现在机场上空时,机场上一片欢腾。大家都知道空战打赢了。 
        餐厅主任罗丘早已准备好了庆功宴会,贝塔吩咐准备节目,召开联欢会。 
        空中的飞机一架接一架地降落在跑道上,机务人员迎上前去检查飞机。飞行员一跨出座舱就接到了姑娘们献上来的鲜花。 
        “海盗打掉了吗?”贝塔一见舒克就问。 
        “让他溜了!”舒克叹口气。 
        贝塔清楚只要海盗活着,他和舒克就不会安宁。 
        舒克一眼看见了停机坪上的轰炸机群。 
        “派一架侦察机去侦察海盗的机场位置,咱们去把他的机场炸了。”舒克说。 
        “好主意!”贝塔乐了。 
        侦察机起飞了。 
        舒克和飞行员们来到餐厅出席庆功宴会,宴会很丰盛。望着桌上的油炸花生米,舒克想起了第一次同妈妈离开洞出去寻找食物的情景。现在,舒克靠自己的劳动生活了,他为老鼠赢得了好名声。舒克想妈妈了,他决定过几天去看妈妈。 
        宴会开得十分热闹,大家要求舒克讲讲空战的经过。舒克来劲儿了,他绘声绘色地讲起了空战,一会儿拿杯子当飞机比划,一会儿拿嘴当机关炮开火,大家都听呆了。 
        “可惜是老鼠跟老鼠打,要是同别人打,多长威风!”不知谁说。 
        舒克的眼睛无光了,他耸耸肩膀。 
        “还记得白路吗?”舒克问贝塔。 
        贝塔点点头,说:“也不知他在发电厂生活得怎么样了?” 
        贝塔喝了口酒。 
        他想起了克里斯王国,想起了咪丽。 
        “我有个想法。”舒克小声对贝塔说。 
        贝塔把耳朵凑过去。 
        “等把海盗收拾了,把航空公司交给臭球经营,咱们再去开始新的生活。”舒克说。 
        “对,重新开始,从零开始,如果咱们再干成一件大事业,不就等于活了两辈子吗?”贝塔同意,而且有高见。 
        “就是,一般人干成了一件事,就拿它当自己的终身事业,挺傻。其实把旧事业扔了,再重新弄另一个新事业,才够味儿。”舒克说。 
        机场上传来轰鸣声,舒克朝窗外一看,侦察机着陆了。 
        舒克和贝塔来到停机坪,飞行员正从侦察机上下来。 
        “找到了,方位是……”侦察机飞行员报告说。 
        “去叫轰炸机中队准备起飞,多装炸弹!”舒克叫喊道。 
        机场上又忙碌起来。军械员们推着炸弹车往轰炸机的肚子里塞炸弹。 
        “你去吗?”贝塔问舒克。 
        “当然。我得出出气。”舒克对海盗极为不满。 
        “我也去。”贝塔的伤已经好了。 
        “行!”舒克答应了。反正轰炸机能坐好几个人。 
        舒克委派臭球担任地面指挥。 
        轰炸机中队整装待发。 
        舒克和贝塔钻进一架轰炸机。 
        “我当驾驶员,你当领航员。”舒克对贝塔说。 
        “行,反正好久没坐飞机了。”贝塔早已对机场有规律的生活厌倦了。他向往睡在坦克车里的生活。 
        “侦察机带路,准备起飞。”臭球在塔台发令。 
        侦察机起飞了。舒克的轰炸机紧跟着也起飞了。所有轰炸机都起飞了。 
        这是舒克贝塔航空公司头一次主动去揍海盗,大家都感到激动。 
        轰炸机群浩浩荡荡向海盗的机场飞去。 
        “等会儿把海盗收拾了,咱们就告别机场吧!”舒克边开飞机边对贝塔说。 
        “你可真是急性子,还吊着我的坦克去吗?”贝塔冲舒克笑笑。 
        “当然,如果你不反对的话。”舒克修正着航向。 
        贝塔回头看看,见轰炸机群的队形编得很整齐。 
        “请注意,目标已出现!”侦察机通报。 
        舒克往下一看,果然有一座机场,停机坪上停放着几架飞机。没错,正是海盗的飞机。 
        “准备轮番轰炸,跟着我来!”舒克说完驾驶轰炸机朝海盗的机场俯冲下去,他看准了目标,按下投弹按钮。 
        轰炸机肚皮下的弹舱盖打开了,炸弹成串地向地面坠落,转眼间机场变成了火海。 
        轰炸机一架接一架地投弹,海盗的机场被掀了个底朝天。有一架歼击机想强行起飞,在跑道中间被炸翻了。 
        海盗正在喝闷酒,这回他知道舒克的厉害了,他的机场被炸平了,飞机都炸坏了。海盗躲在防空工事里,他发誓要报仇。 
        “返航!”舒克一声令下,轰炸机群凯旋。   \chapter{第51集} 
        舒克和贝塔任命臭球为航空公司经理; 
        舒克和贝塔告别机场; 
        直升机又吊着坦克飞到空中   
        当天晚上,舒克贝塔航空公司就像过狂欢节一样热闹。 
        舒克和贝塔悄悄来到直升机和坦克旁边。 
        舒克拉开直升机的舱门,钻进去。机舱里弥漫着一股舒克熟悉的机器味儿。他把飞机检查了一遍。 
        贝塔擦掉坦克身上的灰尘,然后钻进坦克里。他看见了自己的软床,看见了石子炮弹。 
        “怎么样,正常吗?”舒克趴在舱口问。 
        “我试试。”贝塔发动了坦克,“完全正常。” 
        “咱们去同臭球和罗丘谈谈。”舒克说。 
        他们来到办公室,打电话把臭球和罗丘叫来。 
        “我们准备走了。”舒克开门见山。 
        “走?去哪儿?”臭球一愣。 
        罗丘也摸不着头脑。 
        “回去。”贝塔说。 
        “回哪儿?”臭球说。 
        “从哪儿来,回哪儿去。”舒克说。 
        “回地洞里去?!”罗丘不信。 
        “我们不愿意过这种躺在成功事业上的生活。我们喜欢冒险,喜欢干新的事业。”贝塔说。 
        “干什么?”臭球好奇地问。 
        “还不知道。”舒克一摊手。 
        “公司怎么办?”罗丘问。 
        “交给你们俩。”舒克说。 
        臭球和罗丘大眼瞪小眼。 
        “臭球当经理,罗丘当副经理。把空运搞好,大家需要飞机。你们把轰炸机改装成客机,歼击机留着,以防万一。”贝塔吩咐。 
        “这……”臭球一时还接受不了这个现实。 
        “别紧张,大胆干。”舒克给臭球打气。 
        “你们什么时候走?”罗丘问。 
        “马上就走。”贝塔说。 
        “这么急?” 
        “我先去看妈妈。”舒克说。 
        “以后有事还可以用无线电联系嘛!”贝塔提醒臭球和罗丘。 
        “我去给你妈妈准备些点心。”罗丘说。 
        听说舒克和贝塔要离开机场了,机场的全体空地勤人员都来到直升机和坦克旁边。 
        舒克戴上飞行帽,钻进直升机驾驶舱。 
        贝塔跨人坦克。 
        “再见了,朋友们,祝你们好运气!”舒克朝朋友们挥手。 
        “再见!”贝塔的眼睛湿了。 
        人群中哭泣声越来越大。 
        舒克赶紧起飞。直升机升到空中后,用钩子吊起了坦克。 
        直升机吊着坦克飞走了。舒克和贝塔不喜欢过有条不紊的生活,他们喜欢冒险,喜欢过不知道第二天会发生什么事的生活。   \chapter{第52集} 
        舒克见到妈妈; 
        贝塔见到咪丽; 
        舒克和贝塔去教堂见鼠王   
        直升机趁着夜色朝眯丽家飞去。 
        “喂,舒克!”贝塔在坦克里通过电台同舒克聊天。 
        “我又想起了同咪丽周旋的日子。”贝塔深有感触地说。 
        “吃不饱吧!”舒克笑了。 
        “我还拿个口袋贮存香味儿呢,饿了就打开闻闻。”贝塔鼻子开始发酸。 
        “别忆苦思甜了。”舒克注意观察地面。 
        地面上是个小操场,许多人坐在罩边,像是在开会。 
        “你看下边干什么呢?咱们看看吧?”舒克跟贝塔商量。 
        贝塔从炮塔里搽出头来。 
        “下去看看。”贝塔同意。 
        直升机悄悄地在操场旁的房顶上着陆了。 
        一个老太婆在讲话。 
        “这次灭鼠运动是全市统一行动,家家都要投放鼠药。”老太婆说。 
        “没有老鼠的家也放鼠药吗?”有人问。 
        “当然要放!”老人婆说。 
        “那毒谁呀?”又问。 
        “这是规定!”老太婆不再理他,继续讲话。 
        “听见没有?贝塔,要灭鼠呢!”舒克身上直出冷汗。 
        “你听你听,药里还掺和糖,引诱老鼠吃!”贝塔起了一身(又鸟)皮疙瘩。 
        老鼠再坏,也不该下毒药害死人家呀!舒克和贝塔不满了。再说你们是那么巨大的动物,跟小小的老鼠叫真,也太丢份子了。 
        “明天下午发鼠药,当心别让孩子吃了。”老太婆不知为什么掏出个红袖章戴上。 
        “怎么办?”贝塔问舒克。 
        “赶紧通知同胞们,叫他们别吃鼠药。”舒克想了个主意。 
        “太棒了!”贝塔一拍大腿。 
        “先去告诉妈妈。”舒克操纵飞机吊着坦克起飞了。 
        直升机降落在贝塔原先的主人家门口。舒克走下飞机,钻进坦克。 
        贝塔开着坦克驶进屋里。屋里的人都在看电视。贝塔驾驶坦克顺着墙根溜进床下。 
        咪丽正同舒克的妈妈在一个碗里吃饭。咪丽认出了坦克,她兴奋得扑了过去。 
        贝塔打开舱盖儿,钻出来。 
        “咪丽,真想你。”贝塔真诚地说。 
        “我也是。”咪丽也不假。 
        “妈妈——”舒克从坦克里钻出来,奔到妈妈身边。 
        见到了出名的儿子,妈妈流下了眼泪。 
        “谢谢你,咪丽,把我妈妈照看得这么好。”舒克觉得一只猫能照看老鼠,真太不容易了。 
        “我也得感谢你妈妈,没有她,主人就不要我了。”咪丽说。 
        “对了,”舒克告诉妈妈,“人间要灭鼠了,从明天下午开始,家家投放鼠药,您可要当心,千万别吃!” 
        妈妈点点头。 
        “我们要去通知整个老鼠家族。”舒克说。 
        “这么多老鼠,你们怎么通知得过来?”妈妈问。 
        舒克和贝塔为难了。 
        “听说这座城市的鼠王住在一座教堂里,你们去找找。只要告诉鼠王就行,他马上可以传达到每只老鼠。”咪丽提供了一个信息。 
        舒克把罗丘做的点心留给妈妈,他和贝塔去找鼠王。 
        直升机降落在一座教堂顶上,舒克和贝塔从飞机和坦克里出来。 
        “飞机和坦克就放在这儿,保险。咱们爬下去。”舒克说。 
        他俩顺着墙缝儿往下爬,来到教堂的院子里。 
        教堂里静极了,黑乎乎的。舒克和贝塔隐约感到这寂静后边隐藏着恐怖的冷笑。 
        果然,就在他俩转身的时候,从黑暗中冲出几十只鼠兵,把他们围住了。 
        “干什么的?”一只老鼠问。看样子是小头目。 
        “你们是干什么的?”舒克见世面见大了。 
        “还挺厉害!我们是鼠王的御林军!你们深夜来王宫干什么?”小头目说。 
        “找鼠王有事!”贝塔不耐烦了,“快去通报!” 
        “你们是干什么的?”小头目要对鼠王的安全负责。 
        “两只老鼠。”舒克没好气地说。 
        大概是小头目也觉出这两位同胞气质不一般,忙去通报。 
        片刻后,舒克和贝塔被带进了王宫。 
        王宫坐落在一个墙角的地洞内,里边灯火辉煌。 
        鼠王坐在王位上。 
        “什么事?”鼠王傲慢地问。 
        舒克把人类灭鼠的信息告诉鼠王。 
        “真有此事?”鼠王吃惊。 
        舒克点点头。 
        “快通知臣民,万万不得食用鼠药!”鼠王下旨。 
        “且慢!”宰相出来说话了。 
        “讲。”鼠王说。 
        “臣有一计,何不命令臣民们将鼠药放人人类的食物中,给他们点儿颜色看看。”宰相献计献策。 
        鼠王大喜,照此传旨。 
        “这可不行!”舒克急了。 
        “怎么?”鼠王问。 
        “这是要死人的呀!”舒克说。 
        “他们怎么能毒死我们呢?”鼠王问。 
        “这……无论如何不行!”舒克恨死那宰相了。 
        可鼠王的圣旨已经传下去了。   \chapter{第53集} 
        舒克和贝塔在夜幕的掩护下直飞皮皮鲁家; 
        皮皮鲁闻讯大吃一惊   
        舒克和贝塔没想到鼠王会向老鼠家族发布将鼠药放到人的食物里去的圣旨,他俩气疯了。 
        “你们立了大功,我赏你们食物。”鼠王对舒克和贝塔说。 
        “你留着自己吃吧。”舒克说完拉起贝塔就走。 
        他们来到直升机和坦克旁边。 
        “这帮家伙,心眼儿太坏。”舒克为自己的同胞素质太差感到羞愧。 
        “咱们得想个办法,明天就要投放鼠药了。”贝塔急得直搓手。 
        一家一家通知?不可能,第一家人就会把他俩抓起来。去告诉鼠同胞别这样干?也来不及了,明天下午就要投放鼠药了,全城有多少老鼠! 
        真要把全城的人都毒死,太可怕了。 
        几乎是同时,舒克和贝塔想到了皮皮鲁的安全。 
        “咱们先去通知皮皮鲁!”舒克和贝塔异口同声。 
        舒克钻进直升机,贝塔钻进坦克。 
        直升机吊着坦克起飞。 
        舒克辨别了一下皮皮鲁家的方向,驾机直飞皮皮鲁家。 
        贝塔打开坦克舱盖儿,把头探出坦克。 
        月光覆盖了整座城市,微风轻拂过建筑物,万家灯火富有生命力地闪烁着。城市呈现安祥,好像在梦中微笑。贝塔看不出潜藏着危机。 
        “舒克,你看这城市还挺美。”贝塔通过无线电台同舒克说话。 
        “明天晚上就不美了。”舒克往地面看了一眼,说。 
        “唉,干吗都想互相置对方于死地呢。”贝塔叹了口气。 
        “到皮皮鲁家了。”舒克通知贝塔,  “注意观察,我着陆了。” 
        直升机吊着坦克平稳地降落在皮皮鲁家的阳台上。 
        舒克钻出直升机,贝塔钻出坦克。 
        阳台门关着。 
        舒克爬上窗台。屋里漆黑,皮皮鲁躺在床上睡觉。 
        “皮皮鲁!皮皮鲁!”舒克叫。 
        可惜他声占太小,熟睡中的皮皮鲁无动于衷。 
        “把门上的纱窗撕开吧?”贝塔觉得不能等到天亮了。 
        “行,快去拿工具。”舒克说完从窗台上溜下来。 
        贝塔钻进坦克,拎出工具箱。 
        舒克从工具箱里拿出剪子,把纱窗剪开一个口。 
        贝塔先钻进去,舒克紧跟着也钻进去了。 
        他们来到皮皮鲁的枕头旁边。 
        舒克拽拽皮皮鲁的耳朵。 
        “别,别。”皮皮鲁翻了个身。 
        贝塔趴在皮皮鲁耳朵旁边,大声说:“皮皮鲁,快醒醒,考试啦!” 
        皮皮鲁“噌”地坐起来。 
        这一着真灵。 
        “皮皮鲁,别紧张,是我们,舒克贝塔。”舒克赶紧说,生怕考试吓坏了皮皮鲁。 
        皮皮鲁拉开床头灯,笑了。 
        “怎么进来的?”皮皮鲁问。 
        “从那儿。”贝塔指指纱窗门上的豁口。 
        “破门而入呀!”皮皮鲁说。 
        “有急事找你。”舒克说。 
        “海盗又来了?”皮皮鲁问。 
        “我们已经把海盗打败了。”贝塔得意地说。 
        “真的?太棒了!”皮皮鲁必奋了。 
        “我俩离开航空公司了。”舒克告诉皮皮鲁。 
        “为什么?”皮皮鲁以为出了事情。 
        “我们不愿意老在一个地方呆着,干成了的事情,就把它当成终身职业,没劲!”贝塔解释。 
        皮皮鲁赞同地点头。 
        “你知道要灭鼠吗?”舒克坐在皮皮鲁的枕头上说。 
        “鼠王已下令,让老鼠们把鼠药放到你们人类的食物里去。”贝塔说。 
        皮皮鲁愣了。 
        “我俩挺着急,又想不出办法通知所有的人,只好先来告诉你,你可千万注意!”舒克一口气说下来。 
        “谢谢你们。”皮皮鲁说。不过他关心着全城居民的生命安全,“这鼠王也太坏了。” 
        “嗯,也怪我们,不该去告诉他别吃鼠药。”贝塔说。 
        皮皮鲁这才意识到是人先要毒死老鼠的。 
        “怎么办?皮皮鲁,你快想个办法。”舒克急切地说。 
        皮皮鲁看了一下表,已经是凌晨两点钟了。今天下午全城就要发放鼠药。   \chapter{第54集} 
        舒克建议登报; 
        舒克、贝塔和皮皮鲁前往报社; 
        同编辑见面   
        “那你就赶快去通知邻居,告诉一家算一家!”舒克对皮皮鲁说。 
        “我去告诉人家,人家准不信。他们会问,你是从哪儿知道的?”皮皮鲁想到这个问题。 
        “这倒是。”贝塔点头。 
        舒克一眼看见了桌上的报纸。 
        “登报!”舒克脱口而出。 
        皮皮鲁眼腈一亮。 
        “这主意不错,我知道《晚报》杜在哪儿,《晚报》发行量大,几乎家家有,咱们去找《晚报》的编辑。”皮皮鲁说完用最快速度穿衣服。 
        “编辑会同意吗?他准不信。”贝塔提醒皮皮鲁。 
        “去试试,没别的路了。”皮皮鲁说。 
        “咱们一起去,我们到外面等你。”舒克说。 
        “行,我得悄悄出去,不能惊动爸爸妈妈。”皮皮鲁穿好衣服,关上灯。 
        舒克和贝塔钻到阳台上,直升机吊着坦克升到空中,然后降低高度,在单元门口附近等着皮皮鲁。 
        皮皮鲁顺利地出来了,他骑上自行车朝《晚报》社奔去。舒克的直升机跟在他身后飞。 
        “拐过前边那个路口就是。”皮皮鲁告诉舒克。 
        《晚报》社到了。 
        “你们从墙上飞进去。”皮皮鲁告诉舒克。 
        报社的大门关着。皮皮鲁敲门。 
        一位老大爷探头出来。 
        “什么事?这么晚!” 
        “我有急事找编辑。”皮皮鲁说。 
        “急事?先跟我说说。”老大爷要把关。 
        “人命关天的事。”皮皮鲁说。 
        “人命关天?报警呀!”老大爷说。 
        “关系到全城人的生命,得登报。”皮皮鲁说。 
        “你是在编童话吧!”老大爷眯着眼睛看皮皮鲁,“我可没时间听你瞎侃。” 
        “真的,您快让我进去吧!”皮皮鲁急了。 
        老大爷看看皮皮鲁,不像有意来调皮的孩子,再说,哪个孩子会半夜三更来报社捣乱呢? 
        “这样吧,我打电话让夜班编辑下来。”老大爷打开大门,让皮皮鲁进去了。 
        皮皮鲁坐在会客室里,他看见舒克和贝塔趴在窗台上往里看。 
        一位戴眼镜的中年男人走进会客室。 
        “你找编辑?”眼镜编辑对于这么小的读者深夜来访感到惊讶。 
        “您是编辑?”皮皮鲁问。 
        “嗯,有事对我说吧。”眼镜编辑坐在沙发里。 
        “是这样,”皮皮鲁措着词,  “今天下午全城要投放鼠药,您知道吗?” 
        眼镜编辑点点头,示意皮皮鲁继续往下说。 
        “老鼠世界有个鼠王,您知道吗?” 
        眼镜编辑笑了,他摇摇头。 
        从这一笑中,皮皮鲁感到这位编辑可以信任。 
        “鼠王知道了咱们要毒他们,他就下令让全体老鼠把鼠药放回到人类的食物里。”皮皮鲁一字一句地说。 
        眼镜编辑瞪大了眼睛,他在判断面前这个男孩子神经是否正常。 
        “你怎么知道?”编辑问。 
        “我有两只老鼠朋友,他们特意来告诉我。”皮皮鲁说。 
        “老鼠朋友?”眼镜编辑不信。他感到越发奇了。 
        “真的。”皮皮鲁脸真诚。 
        眼镜编辑不说话了,他注视着皮皮鲁。 
        皮皮鲁看出对面这个人心眼儿不坏,可以冒一下险,况且他也没本事抓住舒克和贝塔。 
        “我的老鼠朋友就在外边,我把他们叫进来,您看看?”皮皮鲁问。 
        “在外边?你能叫进来?”眼镜编辑显然不信。 
        “您必须答应一个条件。” 
        “你说吧。” 
        “为舒克和贝塔保密。” 
        “谁是舒克贝塔?” 
        “就是我的老鼠朋友呀!” 
        还有名字! 
        “行,我答应这个条件。”眼镜编辑有点儿信皮皮鲁了。 
        皮皮鲁走出会客室。 
        他在外面同舒克贝塔制定了应急措施后,领着他们走进会客室。 
        眼镜编辑见皮皮鲁真的领进来两只老鼠,而且老鼠身上还穿着飞行服和坦克兵服,他呆了。 
        “您信了吧?”皮皮鲁同。 
        “这……这是怎么回事?”编辑还反应不过来。 
        “鼠王真的下了道命令,让他的臣民把鼠药放回到人的食物里。”舒克说。 
        眼镜编辑差点儿从沙发上蹦起来,老鼠说话了。 
        他现在确信不疑关于老鼠要毒人类的事了。 
        “你们想怎么办?”他问。 
        “登报,告诉全城的人,千万提高警惕。”皮皮鲁说。 
        “登报!在报上说老鼠要毒人类?!谁信呀!”眼镜编辑嚷嚷道。太滑稽了。 
        “这是人命关天的事。”皮皮鲁提醒眼镜编辑。 
        “让人们每天检验食物里是否有毒?”眼镜编辑问。 
        皮皮鲁愣了。也是,谁家有天天检验食物是否含毒的设备? 
        “那就告诉大家别把鼠药投放出去。”皮皮鲁说。 
        “只有这么办了。可这不是干扰灭鼠吗?”眼镜编辑为难了。 
        “上百万人的性命重要。”皮皮鲁说。 
        “好,我同意!但还得同总编辑商量。”眼镜编辑站起来,“你在这儿等我。” 
        眼镜编辑去找总编辑了。 
        皮皮鲁让舒克和贝塔躲出去。   \chapter{第55集} 
        总编辑坚决反对刊登消息; 
        皮皮鲁说出校名也没用; 
        舒克和贝塔想出高招儿   
        眼镜编辑来到夜班总编辑办公室。他将老鼠要毒人类的紧急情况向总编辑汇报。 
        “什么什么?你是在做梦吧?”总编辑放下手中的红笔,惊诧地看着眼镜编辑。 
        “这是真的!”眼镜编辑认真地说。 
        “现在是上班时间,别开玩笑。”总编辑开始生气了。 
        “怎么是玩笑?!” 
        “你是说,全城的老鼠准备把人类毒它们的鼠药反过来毒人类?” 
        “千真万确。” 
        “简直是无稽之谈。” 
        “总编辑,你怎么能不相信!” 
        “我怎么能相信?!” 
        “那孩子就在传达室坐着。” 
        “好,你叫他上来。” 
         眼镜编辑用最快的速度跑到传达室的会客厅。 
        “总编辑不信,他叫你上去。”编辑对皮皮鲁说,“也难怪,谁信呀!” 
        “你没把舒克贝塔供出来吧?”皮皮鲁不放心。 
        “没有。”编辑恪守诺言。 
        “我去同总编辑说。”皮皮鲁站起来。 
        眼镜编辑领着皮皮鲁来到总编室。 
        总编辑注视了皮皮鲁半分钟,判断这孩子干吗深更半夜跑到报社来捣乱。 
        还没等总编辑开口,皮皮鲁先说了。 
        “这都是真的!” 
        “你是哪个学校的?”总编辑突然提出这样的问题。 
        皮皮鲁噎住了。 
        “哪个学校?”总编辑重复了一遍。 
        皮皮鲁清楚,这事要是通知了学校,挨处分是轻的。 
        “你看,他连学校都不敢说出来,这事能真吗?”总编辑问眼镜编辑。 
        “这……”眼镜编辑完全理解皮皮鲁为什么不愿意暴露自己的“单位。” 
        “好了,你回去吧,这次原谅你。以后如果再来报社调皮,我可不客气喽。”总编辑宽容地笑笑。 
        皮皮鲁说出了自己的校名。他觉得值得。 
        总编辑愣了一下,马上把校名记在台历上。 
        “可以登报了吗?”皮皮鲁问。 
        “当然不行。”总编辑毫不动摇。 
        “我负责。”皮皮鲁说。 
        “谁负我的责?这消息一登,我这总编辑就得被撤职。” 
        “这关系到全城人的性命!”皮皮鲁急切地说。 
        “够了!我没时间陪你胡闹了。领他出去吧!”总编辑示意眼镜编辑。 
        眼镜编辑还想作一次努力,当他看见总编辑的目光时,把话咽了回去。 
        “走吧。”眼镜编辑对皮皮鲁说。 
        皮皮鲁还想说什么,总编辑扬扬手。 
        皮皮鲁只好跟着眼镜编辑离开了总编室。 
        “没办法。”眼镜编辑站在报社门口,冲皮皮鲁耸耸肩膀。 
        “您多加注意,把食物保管好。”皮皮鲁很感谢这位编辑。 
        “我家那栋楼我包了。” 
        “再见。” 
        皮皮鲁离开报社。舒克和贝塔在外边等他。 
        “不行。”皮皮鲁对舒克和贝塔说。 
        “我们趴在窗户上听见了。”贝塔说。 
        天蒙蒙亮了。 
        “我们刚才在院子里转了一圈,看见印刷厂就在报社的院里。”贝塔说。 
        “这有什么用?”皮皮鲁找了块砖头坐下。 
        “印刷车问里放着《晚报》的版,咱们偷偷把消息放上去不就行了?”舒克说。 
        “总编辑要审稿的呀!”皮皮鲁说。 
        “咱们等他审完了,签上字,再去改!”舒克已经把报纸出版的程序摸清了。 
        “太棒了!”皮皮鲁一跃而起。 
        “我们打听了,报纸是上午11点钟付印。你把消息的内容写给我们,你就去上学,我们目标小,我们来改版。”贝塔振振有词。 
        “伟大!”皮皮鲁冒出这么一句,“你们向谁打听的?” 
        “向住在报社的老鼠呗。”舒克为自己的同胞遍布全城自豪了。 
        “他们不会向鼠王报告?” 
        “老鼠也不都坏。”贝塔说。 
        “当然!当然!”皮皮鲁深有感触。 
        “快写消息吧。”舒克催促。 
        皮皮鲁掏出纸和笔,垫在膝盖上写起来。 
        “嗯……先写个标题,就写紧急通知吧。”皮皮鲁边写边说。 
        “本报紧急消息,得知鼠王发布了一道圣旨,命令全城的老鼠把人类投放的鼠药再放回到人类的食物中去。为此,本报提醒全城市民千万别投放鼠药,并请互相转告。” 
        皮皮鲁写完后又看了一遍,然后交给舒克。 
        “就按这个排版。”皮皮鲁的口气像主编。   \chapter{第56集} 
        名叫头版的老鼠帮助舒克和贝塔; 
        印刷机印刷皮皮鲁写的消息   
        皮皮鲁回家了,舒克和贝塔将直升机和坦克隐蔽在报社院内的草丛中。 
        舒克把居住在报社的小老鼠叫到直升机旁边。 
        “真厉害,你还会开飞机。”那小老鼠惊叹道。 
        “这儿还有坦克呢!”贝塔对同胞的眼神不济感到遗憾。 
        小老鼠惊讶得张大了嘴巴合不拢。 
        “你叫什么名字?”舒克问。 
        “头版。”小老鼠说。 
        “头版?”舒克和贝塔异口同声,他们觉得这不像是老鼠的名字。 
        “我妈生我的时候,是垫着报纸的第一版生的我,所以给我取名叫头版。”头版解释说,“我妈说这名字不俗,没人叫。” 
        “我叫舒克,他叫贝塔,咱们是朋友了。”舒克对头版说,“谢谢你晚上帮助我们。” 
        “别客气,应该的。”头版爽快地说。 
        “咱们到飞机里坐会儿。”舒克邀请头版上飞机。 
        头版兴奋地钻进直升机。 
        舒克、贝塔和头版分别在皮椅上就座。 
        “你们到底要在这儿干什么?”头版好奇地问。 
        舒克和贝塔对视了一下,认为可以信任头版。 
        “你知道鼠王的圣旨吗?”贝塔问。 
        “是把鼠药放到人的食物里去的圣旨吗?”头版问。 
        “正是,我们想制止这样的惨事。”舒克看着头版说。 
        “你们要帮助人类?”头版站起来。 
        “对。”贝塔说。 
        “叛徒。”头版咬牙切齿地吐出两个字。 
        “人类要投鼠药的消息是我们告诉鼠王的。”舒克说。 
        “那你们干吗又反过来帮人类的忙?”头版不理解。 
        “我们希望大家都活在这个地球上,谁也别害谁。”贝塔说。 
        “人总是害我们!”头版提醒两位同胞。 
        “他们以后会明白的,真要是把老鼠消灭光了,世界将会变成什么样。可我们也不能把人都毒死呀!”舒克说,“有个叫皮皮鲁的男孩子,就对我们特别好。” 
        “你们想怎么办?”头版问。他已经动摇了。 
        “登报。提醒市民们注意,别投放鼠药了。”贝塔说。 
        “得请你帮忙。等总编辑签字后,你去把版上的铅字换了。”舒克把消息的稿子递给头版。 
        头版看了一遍稿子,同意了。 
        “这事千万别让你家的其他老鼠知道,要保密。”舒克告诫头版。 
        头版点点头。 
        “咱们制定一下计划……”舒克说。 
        三只老鼠躲在直升机里策划着修改《晚报》头版内容的步骤。 
        “好了,现在咱们睡觉,养精蓄锐。”舒克说。 
        “我给你们放哨。”头版要尽地主之谊。 
        舒克和贝塔躺在皮椅上睡着了。 
        “快到点了,醒醒!”头版叫舒克和贝塔。 
        舒克和贝塔一边揉眼睛一边坐起来。 
        “行动吧!”舒克打开飞机舱门。 
        三只老鼠离开直升机。 
        他们潜入印刷车间,躲在一台机器下边。 
        “看,那台子上放的就是。”头版指给朋友看。 
        一个人拿着一张纸走到版跟前。 
        “总编辑签了字,行动吧!”头版极有经验,他从生下来就看印刷出版,对这套把戏熟透了。 
        舒克从机器下边冲出来,故意从那人的鞋上溜过去。 
        “啊,老鼠!抓老鼠呀!”那人看见舒克,大喊起来。 
        整个车间沸腾了,人们抄起扫帚、拖把,蜂拥而来,又蜂拥而去。 
        “快,跟我来!”头版招呼贝塔来到捡字盒边,盒里盛着各种铅字。 
        头版飞快地从盒里抽出所需要的铅字,贝塔按顺序排好。那时的报社还采用铅字印刷。 
        等人们精疲力竭地回到岗位上时,版已被换了。 
        印刷机开始工作了。 
        震耳欲聋的轰鸣声仿佛要把房顶揭开。 
        头版偷了一张印好的报纸,给舒克和贝塔看。 
        “成功了!”舒克兴奋得在头版脸上亲了一下。 
        “我们走了,以后来找你玩!”贝塔把报纸收好。 
        “别忘了我!”头版恋恋不舍地同两位朋友告别。 
        舒克和贝塔分别钻进飞机和坦克。 
        直升机吊着坦克起飞了,朝皮皮鲁家飞去。   \chapter{第57集} 
        总编辑被撤职去当排字工; 
        皮皮鲁挨处分还挺高兴; 
        人鼠平安   
        下午,全城都在发放鼠药。 
        老鼠世界也作好了充分的准备,时刻准备将鼠药放进人类的食物里。 
        就在人们正要投放鼠药时,他们看到《晚报》头版上的《紧急通知》。 
        尽管人们对这条消息的真实性表示怀疑。可没人愿意拿自己的性命作试验。 
        鼠药都被小心翼翼地保管好。 
        市灭鼠委员会火了,他们打电话给《晚报》社。 
        “喂,是《晚报》总编室吗?” 
        “是的,什么事?” 
        “我是灭鼠委员会,我找总编辑。” 
        “我就是。” 
        “我们抗议!最强烈的抗议!!” 
        “抗议?为什么?” 
        “你们干扰灭鼠!” 
        “我们干扰灭鼠?” 
        “今天《晚报》的头版!” 
        “头版?你等等。” 
        总编辑从写字台右侧拿起当天的《晚报》,他的脸色渐渐白了。 
        “这,这……”总编辑说不出话来。 
        “我们要告你!” 
        灭鼠委员会的人挂了电话。 
        总编辑突然想起了那个深夜来访的男孩子。 
        他给眼镜编辑打电话。 
        “你看看头版!” 
        “我看见了。” 
        “那你为什么不告诉我?” 
        “我以为是您安排的稿子。” 
        “我安排的?!这到底是怎么回事?” 
        “我也不知道。你可以问问印刷车间,我根本就没去过。” 
        总编辑又打电话给印刷车间,答复是从未有人来过。 
        负责灭鼠的副市长来电话了,声称要追究法律责任。 
        总编辑慌了,他翻出台历上皮皮鲁的校名,又通知查号台查出了学校的电话号码。 
        电话通了。 
        “我找校长,教导主任也行。”总编辑说。 
        “我是教导丰任。” 
        “我是《晚报》总编辑。” 
        “哟,您好!”教导主任受宠若惊。 
        “昨天深夜,有贵校的一个男学生到我们报社来,声称老鼠要毒人类,你能帮助查一下是谁吗?” 
        “多大岁数?穿什么衣服?特征?” 
        总编辑把皮皮鲁的特征讲给教导主任听。 
        皮皮鲁在学校是著名人士,教导主任猜到是皮皮鲁。 
        “我现在去证实一下,马上答复您。”教导主任放卜电话。 
        教导主任蹬上自行车,直奔皮皮鲁家。 
        皮皮鲁全家正在吃饭。 
        敲门声。 
        皮皮鲁去开门。 
        “哟,教导主任!”皮皮鲁一愣。 
        “你昨天夜里去《晚报》社了吗?”教导主任劈头就问。 
        皮皮鲁知道是查灭鼠的事,他点点头。 
        “你惹了大乱子!”教导土任说。 
        “怎么回事?”皮皮鲁的爸爸皮威端着碗出来问。 
        教导主任把经过简要地讲给皮皮鲁的家长听。 
        “这,这可能吗?”皮威不相信。 
        “我去了,可总编辑根本不相信我的话,他说了不发呀!”皮皮鲁装傻。 
        “你从哪儿知道老鼠要毒人类?老鼠来告诉你了?”皮威生气了。不管怎么说,反正儿子是去报社了。 
        “我先去给报社总编辑打个电话。”教导主任转身下楼去了,那口气仿佛报社总编辑是他的老朋友。 
        市政府专门为此事件成立了调查组,总编辑即使浑身是嘴也说不清楚。调查组不信皮皮鲁这么小的男孩子能左右报社的版面。 
        “确实是他来说的。”总编辑有气无力。 
        “他说了,你就登报?”官员问。 
        “我不同意。” 
        “那报上怎么登了?” 
        “这……” 
        市政府决定撤销《晚报》总编辑的职务,下放到印刷厂当排字工。 
        学校也给皮皮鲁记过处分。 
        “值得,值得!”舒克说。 
        第二天的《晚报》头版上发表了检查,并动员市民们马上投放鼠药。 
        鼠药是投放了,可人们多了心眼儿,把自己家的食物都严加看管起来,以防老鼠发坏。 
        老鼠们无奈,无从下手。 
        老鼠不吃一粒鼠药,一只老鼠未毒死。 
        人将食物严加看管,一粒鼠药放不进去。 
        “这结局不错。”贝塔得意地说。 
        “咱们该庆祝庆祝。”舒克说。 
        “我可背了个处分。”皮皮鲁说。   \chapter{第58集} 
        海盗担任追捕舒克贝塔别动队队长; 
        舒克不知道陷阱正等待着他   
        自从鼠王颁发了那道圣旨后,他就梦想着全城的人都被毒死后的情景。 
        “那我就是全城的大王了。”鼠王得意洋洋地想。 
        “禀报鼠王,人类拿着鼠药不撒手。”一位大臣前来汇报。 
        “为什么?”鼠王不解。 
        “据说是一家什么报纸发表文章警告市民。”大臣说。 
        “有这样的事?快去弄一份来!”鼠王的梦碎了。 
        大臣立即派部下去找报纸。 
        半小时后,《晚报》到了鼠王手中。 
        鼠王不认识人类的文字,让一位大臣给他念。 
        “这是谁干的?!”鼠王听完大怒,他认定自己的部下中出了叛徒。 
        “禀报鼠王,外边有一只老鼠求见。”门卫进来说。 
        “让他进来。”鼠王没好气地说。 
        一只老鼠来到鼠王面前,他的胳膊上缠着绷带。 
        “你有什么事?”鼠王问。 
        “我知道是谁向人类告的密。”那老鼠慢悠悠地说。 
        “快说!”鼠王急不可待。 
        “是两只叫舒克和贝塔的老鼠干的。” 
        “舒克贝塔?”鼠王从记忆中搜寻着这两个名字。 
        “就是那两只来向您报告人类灭鼠消息的老鼠。” 
        “是他们!”鼠王想起来了。 
        “这两个家伙是我们老鼠家族的败类,他们利用直升机和坦克,干了一系列违背我们老鼠品质的坏事,还和猫勾结起来。” 
        “啊!”鼠王对此一无所知,“你这么了解他们?” 
        “我和他们打过空战!” 
        原来是海盗。 
        自从舒克和贝塔率领轰炸机炸平了海盗的机场后,海盗逃跑了。他一直追踪舒克和贝塔的足迹,寻机报仇。机会终于来了。 
        “通缉这两个叛徒!”鼠王下令。 
        “要抓活的,审判他们!”海盗建议, 
        “对,抓活的!”鼠王说。 
        “这两个家伙不好对付。”海盗提醒鼠王。 
        “嗯。成立一个特别行动队,由你担任队长。”鼠王封海盗为追捕舒克和贝塔别动队队长。 
        “愿为鼠王效劳!”海盗向鼠王叩首。 
        “行动吧!”鼠王说。 
        海盗当上了别动队长。追捕舒克和贝塔的天罗地网张开了。 
        这几天,舒克和贝塔一直在皮皮鲁家休息,皮皮鲁拿最好吃的食物款待他们。 
        “咱们该出去闯闯了吧?”贝塔不喜欢这种安宁的日子,没味儿。 
        “走。先去看看妈妈和咪丽。”舒克也不愿老呆着。 
        “等等,”皮皮鲁来到阳台上,递给舒克一把东西,又递给贝塔一把。 
        舒克和贝塔一看,是石头炮弹,他们收下了。 
        “起飞!”皮皮鲁一挥手。 
        直升机缓缓升到空中,坦克也离地了。 
        “你看,下边是报社。”舒克通过无线电台对贝塔说。 
        “咱们去看看头版吧?”贝塔提议。 
        “行。”舒克也很想见见头版。 
        直升机和坦克悄悄地降落在花丛中。 
        舒克跳下飞机。贝塔钻出坦克。 
        “你守在这儿,我去找。这样保险。”舒克说。 
        “你快点儿。”贝塔又爬进坦克。 
        舒克钻出花丛,朝头版家走去。 
        草丛中有一双眼睛盯着舒克,这是头版的哥哥,他认出了面前这只老鼠就是鼠王通缉的罪犯。 
        头版的哥哥飞速回家报信。他知道,抓住罪犯有重赏。 
        “快!快!”头版的哥哥气喘吁吁跑进洞。 
        “出什么事了?”头版的爸爸问。 
        “鼠王通缉的罪犯朝咱们家走来了!” 
        “真的?”头版的妈妈大喜。 
        “千真万确。” 
        “快布陷阱抓他!”爸爸觉得发财的时机到了。 
        头版慌了,他想去通知舒克和贝塔。 
        “你干什么去?”哥哥拦住了弟弟。 
        “我、我有点儿小事。”头版支支吾吾。 
        “不许出去!”哥哥厉声道。 
        一张大网悬在洞口上边,准备扣舒克。 
        “我出去引他来。”哥哥出去了。 
        头版的心提到了嗓子眼儿。   \chapter{第59集} 
        舒克洞口遇险; 
        贝塔炮击头版的爸爸; 
        头版入伙   
        舒克陕走到头版家的洞口时,迎面走来一只老鼠。 
        “请问头版住这儿吗?”舒克问。 
        头版的哥哥愣了一下,他不知道舒克是怎么认识他弟弟的。管他呢,先抓住舒克再说。“我是头版的哥哥,我带你去。” 
        “谢谢你啦!”舒克一点儿没怀疑。 
        “头版在家吗?”舒克边走边问。 
        “在家等你呢!” 
        “他怎么知道我来了?”舒克放慢了脚步,他感到奇怪。 
        “他,他……”头版的哥哥知道说漏了,忙往回找,“他刚才看见你了,先回去给你准备吃的,让我来接你。” 
        舒克点点头,又跟着头版的哥哥朝他家走去。 
        到洞口了。 
        “请。”头版的哥哥站在洞口旁请舒克进去。 
        舒克径直朝里走。 
        就在这时,只听洞里传出头版的喊声: 
        “舒克,快跑,他们要抓你!” 
        舒克心里已经有了准备,听到喊声,他转身就跑。 
        头版的哥哥扑上来,死死掐住舒克的脖子。 
        舒克使劲儿踢了他小腹一脚,头版的哥哥捂着肚子蹲下了。 
        这时,头版的爸爸、妈妈,姐姐、弟弟都从家里跑出来追捕舒克,舒克知道寡不敌众,忙朝花丛中跑去。 
        贝塔正懒洋洋地躺在坦克里哼歌,突然听见纷乱的脚步声。他把头探出坦克看,只见舒克身后有好几只老鼠在追他。 
        “怎么回事?”贝塔大声问。 
        “开炮打他们!”舒克钻进直升机。 
        贝塔钻进坦克,瞄准了跑在最前边的老鼠,那是头版的爸爸。 
        贝塔按下了射击按钮。 
        头版的爸爸大叫一声,倒下了。 
        老鼠们站住了,惊恐地望着面前的坦克和直升机。 
        贝塔接通了电台,问舒克: 
        “还打吗?” 
        “暂停。看看他们的动静。” 
        “怎么回事?” 
        “他们设好陷阱抓我。” 
        “头版家?” 
        “头版救了我!” 
        “他在哪儿?” 
        “对了,咱们应该找到他,问问是怎么回事。” 
        “我瞄准,你喊话,吓唬他们!” 
        舒克打开飞机舱门,冲老鼠们说: 
        “快去把头版叫来,不然我开炮了!” 
        老鼠们不知道什么是开炮,站着不动。 
        “开炮!”舒克对贝塔下令。 
        “嗵——” 
        头版的姐姐身边的花被打碎了。 
        她的脸吓白了。 
        “快去叫头版来!”舒克说。 
        头版的姐姐跑回去放头版出来,原来,他们已经把头版关禁闭了。 
        不一会儿,头版来了。 
        “快上飞机!”舒克对他说。 
        头版钻进直升机。 
        “我们可以走了吗?”头版的姐姐胆怯地朝直升机问话。 
        “呆会儿,先别动!”舒克说。 
        头版对舒克说: 
        “鼠王知道了是你和贝塔向人类告的密,现在正通缉抓你们!还专门成立了别动队。” 
        “原来是这样!”舒克明白了。 
        “我也回不了家了,他们会处决我的。”头版说。 
        “跟我们走吧。”舒克让头版先在客舱休息。 
        “贝塔,准备起飞。”舒克拿起话筒。 
        “明白。”贝塔回答。 
        直升机吊着坦克起飞。 
        头版的家人仰着头一边看天上一边发呆。他们这才知道自己根本不是舒克和贝塔的对手。 
        等飞机飞远后,头版的妈妈最先清醒过来,她大喊一声: 
        “快去报案!” 
        海盗接到报案后,立即率领别动队赶到现场。 
        “他们朝哪个方向飞的?”海盗问: 
        “东边。”头版的妈妈指给海盗看。 
        “你们逃不出我的手心!”海盗咬牙。 
        “但愿快点儿抓住。”头版的妈妈觉得直升机和坦克太可怕。 
        “这么说,你的儿子也入伙了!”海盗盯着头版的妈妈问。 
        “对。”头版的妈妈大义灭亲。 
        “连头版一起通缉。”海盗下命令。   \chapter{第60集} 
        舒克、贝塔和头版化装; 
        用花生米换情报; 
        电击鼠王卫兵   
        舒克和贝塔本来马上要离开这座城市去外边闯闯,现在他们知道了鼠王在通缉他们,他们决定不走了,跟鼠王的别动队较量较量。 
        舒克把直升机降落在一座楼顶上。 
        “咱们化一下装,下去玩玩。”舒克对伙伴们说。 
        “行。”贝塔把坦克服脱了。 
        舒克把飞行服也脱了,又往脸上涂了点土。 
        头版也打扮了一番。 
        “走!”舒克看看飞机和坦克挺安全后,说。 
        天渐渐黑了。 
        舒克、贝塔和头版从楼顶上爬下来,舒克看看四周,没有可疑迹象。 
        “走,去鼠王的王宫。”舒克说。 
        “咱们也认识认识别动队长。”贝塔说。 
        鼠王居住的那座教堂就在附近。 
        “咱们从地沟走。”舒克带头钻进地沟。 
        贝塔和头版跟在后边。 
        舒克知道,地沟是老鼠的天下。越是在老鼠聚集的地方,他们就越安全。 
        这段地沟是老鼠的集市,许多老鼠在这儿交换食物。有的用花生换香肠,有的拿面包换腊肉。 
        “这地沟准通王宫,咱们打听一下。”舒克小声对贝塔和头版说。 
        一只老鼠朝舒克他们走来。 
        “要咸菜吗?”那老鼠问。 
        “什么咸菜?”舒克装做感兴趣的样子。 
        “甜辣萝卜干。” 
        “有泡菜吗?”头版问,他爱吃泡菜。 
        “没有。”那老鼠摇摇头,“你们有什么?” 
        “花生米。”舒克说。 
        “换点儿吧?”那老鼠一听说花生米,咽了口唾沫。 
        “你拿什么换?”舒克问他。 
        “我只有……甜辣萝卜干。”那老鼠面带难色。 
        “你知道哪条路通王官吗?”舒克小声问。 
        “当然知道,干什么?”那老鼠看看舒克。 
        “你告诉我,我给你花生米。”舒克冲他挤挤眼睛。 
        “真的?”那老鼠不信有这么便宜的事。 
        “当然是真的。”贝塔拍拍他的肩膀。 
        “从这往前,过两个口,往右就是。不过进不去,有鼠兵把守。”那老鼠说。 
        舒克给了他一把花生米。 
        那老鼠一看是真的,兴奋了。一边往嘴里塞花生米一边问: 
        “你们还想打听什么路?” 
        “我们没那么多花生米。”贝塔耸耸肩膀。 
        舒克、贝塔和头版朝通向王宫的路口走去。果然,路口站着两名鼠王的卫队队员。 
        舒克假装不知道,径直往里走。 
        “站住!”卫兵喝道。 
        舒克吃惊地望着卫兵。 
        “这是王宫。”卫兵说。 
        “我就是去王宫。”舒克说。 
        这时,贝塔和头版分别靠近两名卫兵。 
        “去王宫干什么?”卫兵盘问。 
        “你看他们手里拿的什么?”舒克让卫兵看贝塔和头版。 
        卫兵的目光刚移过去,贝塔和头版打开了手中电棍的开关,将电棍触到卫兵的身体上。 
        两名卫兵顿时休克了。 
        其实,电棍里只有一节二号电池,这是舒克和贝塔创造的新式武器。 
        “快走!”舒克一挥手。 
        他们朝深处跑去。 
        走到无路可走时,贝塔指指上边,说: 
        “就在上边。” 
        三位勇士朝上爬去,他们看见了亮光。 
        “你们等一下,我先侦察侦察。”舒克对同伴说。 
        舒克从地沟的出口探出头去,有一名卫兵守在出口,正打瞌睡。 
        这是王宫的院子。 
        舒克冲贝塔招招手,贝塔过来了。舒克趴在他耳朵上说了几句。 
        舒克和贝塔同时冲出去,将卫兵拖进地沟里。头版冒充卫兵出去站岗。 
        “你……你们……”卫兵醒进来时,发现自己被劫持了。 
        “鼠王住在哪儿?”贝塔用电棍指着卫兵问。 
        “……”卫兵不说。 
        贝塔用电棍轻轻碰了碰他。卫兵马上说了。   \chapter{第61集} 
        舒克和贝塔到鼠王的床边; 
        贝塔拔鼠王的胡子; 
        海盗同老对手见面   
        贝塔掏出绳子,把卫兵捆上。舒克往卫兵嘴里塞布条,堵住他的发声通道。 
        舒克和贝塔将卫兵安置好后,钻出地沟。 
        “你在这儿替他站岗,我们去治治鼠王。”舒克对头版说。 
        “当心点儿。”头版说。 
        贝塔和舒克接近了鼠王的住处,洞口又有两名卫兵。 
        “讨厌,设这么多卫兵干什么!”舒克觉得官越大越怕死。 
        “你还记得克里斯国王吗?”舒克问。 
        “嗯,差不多。”贝塔说。 
        “准备电棍吧。”舒克掏出电棍。 
        贝塔也准备好。 
        他俩朝两名卫兵走去。 
        “站住。”卫兵喝道。 
        “给鼠王送东西。”舒克将打开了开关的电棍递给卫兵。 
        贝塔也如法炮制,将电棍递给另一名卫兵。 
        两位卫兵同时伸手接电棍,又同时凝固了。 
        舒克和贝塔推倒休克的卫兵,冲进鼠王的卧室。 
        鼠王正躺在床上吃香肠。 
        “你们好大胆,不禀报就闯进来!”鼠王大怒。 
        “禀报鼠王,我们是来报告舒克和贝塔的消息的。”舒克说。 
        “抓住了?!”鼠王一跃而起。 
        “嗯。”贝塔点头。 
        “太好了!”鼠王捋捋胡须,满面顿生春风。 
        “怎么处置?”舒克想知道他俩如果落到鼠王手中的下场。 
        “斩首!”鼠王毫不犹豫,“不过,这太便宜了他们。应该千刀万剐。还不过瘾。用开水烫。用开水烫?不解气,对,把他们的腿砍了,扔到大街上,让人去处置他们!谁让他们当叛徒呢!” 
        舒克和贝塔身上同时起了一层(又鸟)皮疙瘩。 
        “他们已经来了。”贝塔说。 
        “在哪儿?”鼠王看看洞里,没有。 
        舒克和贝塔把伪装拿掉,露出真容。 
        鼠王呆了,他刚想喊,贝塔冲上去按住他的脖子。 
        “我装鼠王,咱们见见那别动队长。”舒克开始化装成鼠王的模样,穿上鼠王的衣服。 
        “像吗?”舒克问贝塔。 
        “就是胡子不像。”贝塔左看右看。 
        “把他的胡子拽下来,粘在我嘴上。”舒克想了个好办法。 
        贝塔按住鼠王,把他的胡子一根一根揪下来。 
        气得鼠王直咬牙。 
        鼠王的胡子粘在了舒克的嘴上。 
        “真像!”贝塔说。 
        舒克把事先准备好的飞行服给鼠王穿上,贝塔把鼠王捆好,又将嘴堵上,再用一根黑布条把他的眼睛蒙上。谁也认不出这是鼠王。 
        “你冒充卫兵去叫别动队长,我在这儿等着。”舒克躺在床上。 
        鼠王坐在地上。 
        贝塔跑出去,先把两个昏迷的卫兵拖进来塞到床底下,然后去叫别动队长。 
        当贝塔找到别动队长时,他差点儿喊出来,是海盗。 
        海盗同贝塔见面不多,投认出来,他听说鼠王传他,忙跟贝塔去了。 
        贝塔的心已经提到嗓子眼儿,他生怕海盗认出舒克来。 
        走到鼠王的卧室洞口时,贝塔对海盗说: 
        “你等一下,我去禀报一声。” 
        贝塔先进去,告诉舒克别动队长的姓名。 
        “海盗!”舒克大吃一惊,真是冤家路窄。 
        “怎么办?”贝塔问。 
        “把蜡烛弄暗点儿。”舒克说,“让他进来.够刺激。” 
        贝塔来到洞口,对海盗说: 
        “鼠王有请。” 
        海盗不耐烦地往里走。他早想好了,只要一抓住舒克和贝塔,马上就搞政变,篡夺王位。   \chapter{第62集} 
        海盗鞭打鼠王; 
        鼠王咽不下这口气; 
        咪丽中计   
        海盗来到鼠王的床边。 
        “您有什么吩咐?”海盗问。 
        “我抓住舒克了。”舒克指指蹲在地上的鼠王。 
        “真的?”海盗乐了,尽管他觉得鼠王的发音不大对头,可他顾不上分析了。 
        海盗走到蹲在地上的鼠王旁边,他一眼认出舒克的飞行服。 
        “没错,就是他!”海盗抡圆了胳膊打了鼠王一记耳光。 
        “你这么恨他?”舒克坐在床上问。 
        “他是咱们老鼠家族的败类!”海盗说完又狠踢了鼠王一脚。 
        “把他交给你了!”舒克说。 
        “由我来惩治他!”海盗拎起鼠王朝洞外走去。 
        “快溜!”贝塔对舒克说。 
        舒克从床上跳下来,和贝塔离开了鼠王的卧室。 
        头版见舒克和贝塔回来了,问: 
        “怎么样?” 
        “完事了,快走!”舒克说完先钻进地沟。贝塔和头版跟着钻进去。 
        他们平安地回到直升机上。 
        舒克和贝塔给头版表演冒充鼠王的经过,逗得头版差点笑破了肚皮。 
        “海盗怎么跑到这儿来了?”舒克自言自语。 
        “这家伙不一般,咱们得留心点儿。”贝塔说。 
        “嗯。”舒克又想起同海盗空战的情景。 
        海盗将“舒克”押到别动队总部,他抄起一根皮鞭,劈头盖脑抡过去,“舒克”嘴被堵住了,喊不出声来,他拼命挣扎。 
        “让你知道知道我的厉害!”海盗使出全身力气抽“舒克”。 
      “舒克”在地上打滚儿。 
      “把他嘴里的布掏出来,我要问话。”海盗对部下说。 
        布刚从嘴里掏出来,鼠王立刻从地上爬起来,大喊:“混蛋!我是鼠王!” 
        “你还敢冒充鼠王!”海盗一鞭子抽过去,打得鼠王嘴角直淌血。 
        “我是鼠王,快给我松绑!”鼠王气急败坏地跺脚。 
        又是一鞭子。鼠王沉默了。 
        “把他的遮眼布摘了!”海盗命令。 
        遮眼布一摘,海盗傻眼了。 
        “快,快松绑!”海盗一边叫一边亲自去给鼠王松绑。 
        鼠王一脚把他踢出去老远。 
        “来人,把海盗抓起来!”鼠王要出这口气。 
        “大王息怒!”鼠王的一位大臣上前劝阻,“要想抓住逃犯,非他不可。再说他也是恨逃犯呀!” 
        鼠王摸着身上的伤处,他实在咽不下这口气。 
        “大王恕罪!”海盗跪在地上给鼠王磕头,“我三天之内非抓住舒克不可,不然拿我问罪!” 
        “好,给你三天时间!”鼠王一屁股坐在地上,他疼得站不住。 
        “我派出去的密探打听到舒克的妈妈就住在本城,咱们去把他妈妈抓来,舒克孝顺,还怕他不来上钩?”海盗用心毒辣。 
        “马上去抓!”鼠王下令。 
        海盗率领着别动队出发了,他们直奔咪丽的主人家。 
        咪丽和舒克的妈妈正在床底下进餐,咪丽忽然支起耳朵。 
        “有老鼠来了。’咪丽说。 
        “是舒克吗?” 
        “不像。”咪丽趴在地上往外看。 
        当海盗看见床下有猫时,他们立刻分散不敢靠近床。 
        “他们看见你,不敢过来。”舒克的妈妈对咪丽说。 
        “您去跟他们说。”咪丽往后退了几步。 
        舒克的妈妈走出去问:  “你们是舒克派来的吗?” 
        “正是。”海盗灵机一动,  “舒克让我们来接他妈妈。” 
        “我就是舒克的妈妈,舒克在哪儿?” 
        “舒克找了座新房子,让您看看。”海盗诓她。 
        “我去看看,一会儿回来。”舒克的妈妈对咪丽说。 
        咪丽同意了。   \chapter{第63集} 
        坦克直逼鼠王王宫; 
        贝塔用炮管抡鼠兵; 
        舒克遇险   
        当咪丽发觉上当时,已经是两天后的事了。 
        这天夜里,舒克和贝塔来到咪丽面前。 
        “我妈妈呢?”舒克问。 
        “不是被你接走了吗?”咪丽说。 
        “我接走了?”舒克看看贝塔。 
        贝塔两手一摊,耸耸肩膀。 
        “你没派两只老鼠来接你妈妈?”咪丽傻眼了。 
        舒克感到不妙了。“什么时候的事?”贝塔问。 
        “两天前。”咪丽把海盗的样子形容了一番。 
        “是海盗!”舒克一跺脚。 
        “什么海盗?”咪丽头一次听说。 
        贝塔把海盗的来历以及他们同海盗的仇怨告诉咪丽。 
        “中计了!中汁了!”咪丽气得直咬牙。 
        “我去救妈妈!”舒克转身准备出去。 
        “这肯定是海盗的圈套。”贝塔提醒说。 
        “我跟你去!”咪丽决定治治这些老鼠。 
        “他们是想用妈妈当诱饵抓我,咱们得提防点儿。”舒克挺清醒。 
        “这回不能从地沟走了,咱们开坦克去。”贝塔说。 
        舒克同意了。他将直升机藏在咪丽家的外边。 
        舒克、贝塔和头版钻进坦克。咪丽在一旁护卫,向鼠王的王官进发。 
        贝塔好长时间没开坦克了,他坐在驾驶员的座位上,聚精会神地操纵坦克行驶。 
        舒克坐在炮手的座位上。头版躺在贝塔的软床上。 
        “你准备好炮弹。”贝塔对舒克说。 
        舒克从弹箱里翻出几颗重型炮弹。他真后悔在王宫里没把海盗干掉。 
        夜,漆黑一片,街上没有行人。 
        坦克接近那座教堂了。 
        舒克从坦克里探出头来,对咪丽说: 
        “注意,前边就是!” 
        咪丽用最快的速度跑到教堂旁的阴影下躲起来。 
        舒克从坦克里钻出来,跟咪丽隐蔽在一起。 
        贝塔开始采用调虎离山计。他驾驶着坦克明目张胆地朝鼠王王官闯去。 
        头版坐在炮手的位置上,随时准备射击。 
        王宫被惊动了。 
        鼠王的卫队全副武装迎战。他们还不知道坦克的厉害,竟然手持刀棍朝坦克冲来。 
        “开炮!”贝塔发令。 
        头版瞄准冲在最前边的鼠兵开炮。 
        两名鼠兵应声倒在地上,疼得大叫。 
        剩下的卫兵毫无惧色,继续冲击。 
        “开炮!”贝塔喊。 
        头版不会装炮弹,刚才那发是舒克装的。 
        眼看鼠兵们接近坦克,如果他们爬上坦克,情况就不妙了。 
        装炮弹已经来不及了,贝塔决定转炮塔,用大炮的炮管抡鼠兵。 
        “坐好,抓紧!”贝塔对头版说, 
        贝塔从潜望镜里看见鼠兵靠近坦克,他猛地按下炮塔旋转按钮。炮塔飞速旋起来,炮管把几名鼠兵抡出去数米远。 
        其余的鼠兵吓坏了,朝王宫撤退。 
        这时,趁鼠兵们同贝塔交战之际,舒克和咪丽翻墙进了王宫。 
        王宫里很平静。舒克看见一名鼠兵在站岗。 
        舒克拿着电棍靠近他。 
        鼠兵发现了舒克,刚要喊,舒克用电棍碰了他一下,鼠兵不敢动了。 
        “舒克的妈妈关在哪儿?”舒克问。 
        鼠兵指指一个洞口。 
        舒克蹑手蹑脚朝那洞口走去,咪丽进不去,躲在一旁等他。 
        舒克钻进洞,一张大网从天而降,扣住了他。 
        ‘哈哈,你终于落到我的手里了!”海盗出现在舒克面前。 
        舒克明白中计了,他拼命挣扎,越动网子越紧。 
        咪丽在外边干着急,她进不去。所有的老鼠都躲进洞里,咪丽一只也抓不着。 
        “你和猫也勾结起来了。”海盗冷笑着对舒克说。   \chapter{第64集} 
        贝塔向皮皮鲁求援; 
        电子捕鼠器显示威力; 
        海盗又设圈套   
        咪丽救不出舒克,只好去向贝塔报信。 
        “啊?!舒克被抓啦?”贝塔差点儿从坦克炮塔上弹射出去。 
        “咱们中计了。海盗太狡猾。”咪丽真想一口把海盗吞了。 
        “坦克能冲进去吗?”贝塔问。 
        “进不去。直升机差不多。”咪丽望望教堂的高墙。 
        贝塔清楚,耽误一分钟,舒克就多一分危险。 
        “去向皮皮鲁求援!”咪丽提醒贝塔。 
        贝塔眼睛一亮:“对,皮皮鲁准有办法。” 
        “我在这儿守着,他们出来一只我抓一只,你们快去找皮皮鲁。”咪丽说。 
        贝塔和头版驾驶坦克朝皮皮鲁家开去。过去贝塔都是从空中去皮皮鲁家,他根本不认路。 
        坦克在城里转来转去,直到天快亮时才找到皮皮鲁家。 
        贝塔把坦克隐蔽在草丛里,对头版说: 
        “你守着坦克,我上去找皮皮鲁。” 
        头版点点头。 
        贝塔顺着下水管爬上皮皮鲁家的阳台,阳台门关着,纱窗上还留着那个洞口。贝塔从洞口钻进屋里。 
        皮皮鲁睡得正香,被贝塔叫醒了。 
        “什么,舒克被海盗抓住了?!”皮皮鲁一个鲤鱼打挺坐起来。 
        “你快想办法救他吧!”贝塔急切地说。 
        “这些坏老鼠,真该……”皮皮鲁想起贝塔也是老鼠,把“真该毒死它们”后半截话咽回去了。 
        “有办法吗?”贝塔等不及了。 
        “我想想。”皮皮鲁知道自己钻不进老鼠洞去救舒克。 
        贝塔注视着桌上闹钟的指针。 
        “我用电子捕鼠器治它们!”皮皮鲁说。 
        光“电子捕鼠器”这名字就让贝塔打了个哆嗦。 
        “你有吗?”贝塔问。 
        “学校有,一会儿我去拿。”皮皮鲁三下两下穿上衣服。 
        “我们呢?”贝塔问。 
        “我带你们一起去。”皮皮鲁背上书包,“把坦克装进我的书包。” 
        舒克现在被海盗结结实实地捆起来,同妈妈关在一起。 
        “我要亲手处决他!”鼠王要出这口气。 
        “不忙,等把贝塔抓住,一起处决才爽。”海盗慢悠悠地说。 
        “什么时候能抓住贝塔?”鼠王问。 
        “快了,贝塔准来救舒克,我已布置好了陷阱。”海盗胸有成竹。 
        舒克为贝塔捏了一把汗。 
        就在这时,老鼠们突然感觉到有一种无形的力量在呼唤他们。几只老鼠兵身不由己地出去了。 
        “怎么回事?”海盗感到奇怪。 
        “我出去看看。”鼠王也身不由己地想出去。 
        “等等,怎么都不回来了?”海盗觉出不妙了。 
        舒克认定这和贝塔有关。 
        “妈妈,准是贝塔来救咱们了。”舒克小声说。 
        “不好了!不好了!”一只鼠兵跑进来。 
        “怎么回事?”海盗急忙问。 
        “有一台仪器放在院里,咱们的鼠兵被那台仪器抓住好多只,可还是有不少鼠兵往那儿跑!”那鼠兵报告。 
        “电子捕鼠器!”海盗恍然大悟。 
        这时,一只被五花大绑的鼠兵跑进来。 
        “他放我回来,说要是不把舒克和他妈妈马上放了,就把咱们都抓起来。”五花大绑的鼠兵哭丧着脸说。 
        “他?他是谁?”海盗问。 
        “是个人,叫皮皮鲁。”鼠兵说。 
        “好啊,舒克和人也勾结起来了,我说你哪儿来那么多飞机!”海盗恶狠狠地看着舒克说。 
        舒克不理他。 
        “你去告诉他,晚上放舒克。”海盗对鼠兵说。 
        “不能放!”鼠王急了。 
        “这是计。”海盗小声对鼠王说着什么,鼠王笑   \chapter{第65集} 
        老鼠家族组成敢死队; 
        皮皮鲁被围困; 
        贝塔呼救   
        鼠兵将海盗的话转告皮皮鲁。 
        “好,我晚上来。如果到时候不放舒克,我就不客气了。”皮皮鲁看时间不早了,该去上学了,就答应了。 
        “我在这儿监视他们。”咪丽说。 
        “行,贝塔和头版坐在坦克里跟我上学去吧,这里危险。”皮皮鲁对贝塔说。 
        贝塔同意了。 
        皮皮鲁将电子捕鼠器收起来,把俘虏们关进事先准备好的笼子,交给咪丽看押。 
        贝塔和头版钻进坦克。皮皮鲁把坦克装进书包,去上学。 
        “咱们睡会儿。”贝塔在坦克里对头版说。 
        海盗得知皮皮鲁走了,立即召来两名部下,吩咐道:“门口有猫,你们从暗道出去跟着皮皮鲁,随时向我报告他的动向,” 
        两名鼠兵去跟踪皮皮鲁。 
        “大王,咱们得治治这个皮皮鲁。”海盗对鼠王说。 
        “嗯。你快想办法。”鼠王说。 
        “发动全城的老鼠去咬他,攻击他!”海盗说,“全城共有多少老鼠?” 
        掌管鼠数的大臣递上花名册。 
        “5000只就足够了。”海盗说。 
        “传我的命令,马上挑选5000只身强力壮的老鼠,准备同皮皮鲁较量。”鼠王说。 
        鼠王的圣旨立即被传达到全城。5000只老鼠精选出来了,他们磨尖了牙齿,随时待命准备同那个与老鼠为敌的皮皮鲁拼命。 
        “报告,皮皮鲁现在学校的大操场上。”侦察鼠向海盗报告。 
        “几个人?”海盗问。 
        “就他自己。” 
        “干什么呢?” 
        “好像在背书。” 
        “传令,敢死队出击!目标,皮皮鲁。”海盗全身披挂,亲自督战。他断定一个男孩子绝对不足5000只老鼠的对手。 
        下午放学后,皮皮鲁一个人在学校的大操场上背书,应付明天的考试。同时等天黑了去鼠王王宫接舒克。 
        贝塔和头版在坦克里睡觉。坦克在皮皮鲁的书包里。 
        5000只老鼠组成的敢死队从学校的下水道里冲出来了,他们朝皮皮鲁围过来。 
        皮皮鲁正在埋头背书,忽然感到远处的地上有什么东西在蠕动,他抬头一看,愣了。黑压压的一片老鼠正朝他冲过来。 
        皮皮鲁往后退了几步,一回头,身后全是老鼠。再往左右看,皮皮鲁明白是怎么回事了。密密麻麻的老鼠把皮皮鲁围在了中间。 
        大操场上空无一人。没有可以用来当武器的东西。皮皮鲁完全清楚这么多的老鼠能把他怎么样。 
        贝塔被同胞的脚步声惊动了。他从书包里探出头,傻了。 
        “皮皮鲁,快把坦克从书包里拿出来!”贝塔喊。 
        皮皮鲁明白贝塔的坦克起不了什么作用,老鼠太多了,但总比不抵抗强。皮皮鲁从书包里掏出坦克,放到地上。 
        “快装炮弹!”贝塔对头版说。 
        头版的手直哆嗦,炮弹装上了。 
        “打!”贝塔一挥手。 
        坦克开炮了。两名老鼠敢死队员倒下了,敢死队员踩着同胞的身体继续冲锋。 
        皮皮鲁手里只有书包,他用书包抡已经接近他的老鼠。 
        贝塔操纵坦克撞老鼠,坦克在老鼠群里横冲直撞。可老鼠太多了,已经有许多老鼠爬上了坦克,有的在使劲掀盖子。 
        皮皮鲁的身上已经爬上了几只老鼠。皮皮鲁把他们打下来,又有老鼠爬上去…… 
        情况万分紧急。 
        贝塔没想到海盗这么狠毒,他认定自己和舒克加上皮皮鲁都会葬送在海盗手里。 
        危急之中,贝塔看见了电台。他眼睛一亮。 
        贝塔戴上耳机,拿起话筒。 
        “罗丘!臭球!贝塔呼叫!贝塔呼叫!请回答!”贝塔太声喊着。 
        没有回声。 
        “臭球!臭球!我是贝塔!我是贝塔!”贝塔带着哭腔喊。   \chapter{第66集} 
        臭球率领战斗机群大败敢死队; 
        摄像记者激动得忘记放录像带; 
        舒克、贝塔和臭球相逢   
        “我是臭球!我是臭球!请讲!”贝塔的耳机里传出了臭球的声音。 
        “臭球,我是贝塔!我们遇到了险情,请立即派战斗机来支援!请立即派战斗机来支援!”贝塔喊。 
        “方位?” 
        贝塔将方位告诉臭球。 
        臭球拉响了机场的警报器。 
        民用机场转眼变成了军用机场。歼击机、强击机和轰炸机腾空而起,朝城市扑去。 
        此时皮皮鲁的身上已经爬上了几十只老鼠,他们开始咬皮皮鲁的衣服,咬皮皮鲁的手。 
        皮皮鲁一边抡书包一边把身上的老鼠往下赶,他大声呼喊。 
        学校的教导主任闻声赶来,这场面把他吓坏了。他想起皮皮鲁深夜去报社说老鼠要毒人的事,他信了。 
        教导主任不敢上前,他跑回去打电话。 
        贝塔的坦克已经被老鼠敢死队围死了,炮塔也转不动了。老鼠们正使劲翻坦克。 
        海盗神气活现地在一旁指挥,他终于知道了自己的力量,把人都治住了!他想好了,只要一抓住贝塔,就立即篡夺王位。 
        皮皮鲁已经精疲力竭,老鼠敢死队把对人类的仇恨都集中到皮皮鲁身上。 
        正当海盗认定稳操胜券的时候,天空中出现了战斗机群。 
        “贝塔,贝塔,我们到了,坚持住!”臭球在歼击机里喊话。 
        “快打!使劲儿打!”贝塔兴奋了。 
        臭球率领歼击机群开始轮番俯冲扫射,数十只老鼠敢死队员倒在地上。 
        强击机编队俯冲。 
        轰炸机时而俯冲投弹,时而水平投弹。 
        操场上爆炸声此起彼伏。 
        皮皮鲁趁机逃出操场,躲在树后观看老鼠之间的大战。 
        贝塔来劲儿了,他驾驶坦克横冲直撞。 
        “瞧,海盗!”头版从潜望镜里看见丁海盗。 
        “撞他!”贝塔驾驶坦克朝海盗轧过去。 
        敢死队员们被战斗机打得抱头鼠蹿。 
        臭球不断地下达攻击命令: 
        “一中队,截住往南逃的那股!” 
        “三中队,狠狠打西边那一小撮儿!” 
        “轰炸机5号,往东边投颗催泪弹!” 
        “强击机!压住北边那伙!” 
        学校里没回家的学生和老师都跑到操场上目睹这场老鼠之间的大战,他们看呆了。要不是亲眼所见,谁会相信! 
        不知是谁打电话把电视台的摄像记者叫来了。记者扛着摄像机猛拍,要不是因为他激动忘了放录像带,全世界都能欣赏这场大战。 
        贝塔的坦克死盯着海盗追。海盗拐着弯跑,他没想到形势变化这么快,海盗看见操场边有一个下水道口,他钻了进去。 
        贝塔没办法了。 
        “臭球,舒克被关在教堂下边的鼠王王宫里,你快派部下去救他!”贝塔通过电台喊叫。贝塔清楚,如果海盗先回王宫,舒克准倒霉。 
        臭球留下一些飞机继续攻击老鼠敢死队,他带领另一批飞机直飞教堂。 
        操场上的战斗已接近尾声,皮皮鲁跑进操场拿起贝塔的坦克,朝教堂跑去。 
        臭球的空降兵在教堂着陆了,他们在咪丽的指引下,杀进鼠王的王宫。 
        鼠王正在等待凯旋的海盗,忽闻宫外杀声骤起。 
        “不好了,人王,数百只鼠兵从天而降,杀进王宫来了!”卫队长慌忙跑来向鼠王禀报。 
        鼠王面如土色。 
        臭球的歼击机在教堂的围墙上着陆。 
        臭球率领部下生擒了鼠王。 
        “快说,舒克关在哪儿?”臭球问。 
        鼠王筛糠着交待。 
        舒克和妈妈得救了。 
        “谢谢你,臭球!”舒克和臭球拥抱。 
        贝塔和头版赶到了,大家十分激动。 
        “怎么处置他?”臭球指指鼠王。 
        “罢免他的王位,放他当普通老鼠。”舒克说。 
        鼠王感激不尽。 
        “抓住海盗了吗?”舒克穿上飞行服问。   \chapter{第67集} 
        舒克不同意搜捕海盗; 
        舒克和贝塔决定办报; 
        头版担任印刷厂厂长   
        “又让他跑了。”贝塔惋惜地说。 
        听说海盗又溜了,舒克直跺脚。 
        “咱们在全城搜捕他!”臭球提议。 
        舒克想了想,摇摇头,说:“算了,闹得大家惊慌不安。再说,抓住他又能把他怎么样呢?” 
        大家明白,舒克不会处死海盗。 
        “我觉得老鼠家族的品质应该改变改变。”舒克说,“单靠抓住一个海盗起不了作用。” 
        “这可不容易。”臭球耸耸肩膀。 
        舒克看见了头版,他眼睛一亮。 
        舒克把《晚报》发消息不让人类投放鼠药的故事讲给臭球听。 
        “报纸真厉害。”臭球说。 
        “咱们办张报纸,改变老鼠家族的品质。”舒克说。 
        “办报纸!” 
        “太好了!” 
        大家一致同意。 
        “办报不大容易吧?”贝塔问。 
        “其实不难,把三方的关系摆对就行。”头版说。 
        “哪三方?”舒克问。 
        “读者、编者和作者。”头版说,“读者是爷爷,编者是爸爸,作者是孙子。这样的关系,准能办好报。” 
        ‘如果倒过来呢?作者是爷爷,编者是爸爸,读者是孙子。”舒克问头版。 
        “世界上凡是没人看的报刊,准因为像你说的这样摆三方的关系。”头版一口咬定。 
        舒克连连点头。 
        “我们机场订一百份!”臭球当了老鼠报第一个订户。 
        “将来还得靠你的运输机给我们往各地送报。”贝塔拍拍臭球的肩膀。 
        “没问题,包在我身上。”臭球拍胸脯。 
        “说干就干,咱们先给报纸起个名字。”贝塔说。 
        “叫《老鼠日报》怎么样?”舒克说。 
        “一开始就出日报恐怕困难,”头版经验丰富,“其实叫《老鼠报》就行。” 
        “对,就叫《老鼠报》。”舒克同意了。 
        “舒克当总编辑。”贝塔提议。 
        没人反对。 
        “头版当印刷厂厂长。贝塔当发行部经理。”舒克任命。 
        “编辑部设在哪儿?”头版问。 
        “就设在直升机上。”舒克说。 
        “保准是全世界第一家空中报社。”臭球羡慕。 
        “咱们去告诉皮皮鲁。”舒克同大家来到外边。 
        皮皮鲁正在检阅他的飞机。 
        “办报?我也有希望发表作品啦!”皮皮鲁听说后乐得拍手。 
        “头版,你现在就去筹备印刷厂,有什么困难?”总编辑舒克问。 
        “我回家去动员我家里人一起干,他们都在行。”头版说。 
        “我送你去。”贝塔钻进坦克。 
        “我们先返航了,有事随时联系。”臭球说。 
        舒克拥抱臭球。 
        机群升空。 
        皮皮鲁也告辞。 
        “我得物色几名编辑和记者。”舒克躺在直升机里的软椅上想。 
        有人敲直升机的玻璃窗。 
        舒克坐起来往外看,是两只老鼠。 
        舒克打开舱门,问:“什么事?” 
        “我叫松果,她叫荷叶。我们的父母被海盗强迫去当敢死队员,都战死了,我们没有亲人了。”叫松果的老鼠说。 
        舒克叹了口气。 
        “你是舒克吧?”荷叶问。 
        “嗯。” 
        “你的本事大,你收留我们吧!”荷叶请求道。 
        舒克拿不定注意。   \chapter{第68集} 
        舒克招聘记者和编辑; 
        总编辑亲自撰写发刊词; 
        荷叶的文章通过了   
        “你们识字吗?”舒克问松果和荷叶。 
        他们点头。 
        “我们办了张《老鼠报》,你们给我当记者和编辑吧。”舒克说。 
        “记者?编辑?我们行吗?”松果怀疑自己的能力。 
        “听说记者和编辑都得大学毕业。”荷叶说。 
        “那是人类同自己过不去,大学是给那些没本事又想比别人活得好的人准备的。其实,有学历的笨蛋更麻烦。咱们不管那么多。你们试试。”舒克说完请松果和荷叶上飞机。 
        荷叶和松果登上直升机。 
        “这就是报社。来,咱们把机舱改装一下,像个编辑部的样子。”舒克找来工具箱。 
        他们一起动手,把皮椅移动位置,再安上办公的桌子,机舱变了样儿,像办公室。 
        “我来考考你们,看看你们适合干什么。”舒克坐在总编辑的座位上。 
        荷叶和松果坐在总编辑对面。 
        “你们谁知道记者是干什么的?”总编辑发问了。 
        “记者就是把事情记着。”松果说。 
        “记者就是……就是到处跑。”荷叶说。 
        舒克也不知道记者的确切含义。他又问:“编辑呢?” 
        “编辑就是把事情编起来。”荷叶先回答。 
        “大概是这样的。”松果对编辑一无所知。 
        “松果当记者。荷叶当编辑。”总编辑任命部下。 
        舒克找出几杆笔。分给记者和编辑。 
        这时,电台里传出贝塔的呼叫声。 
        舒克走进驾驶舱,戴上耳机。 
        “我是舒克。请讲。” 
        “我是贝塔。我们已到达头版家。头版的全家愿意筹办印刷厂。他们还向你道歉。现在我们正在工作。” 
        “很好,我也物色到了记者和编辑。什么时间可以印报?” 
        “再有两小时就可以,头版家的人很熟悉这项业务。” 
        “好,两小时后我给你送稿子去。”总编辑挂上耳机。 
        舒克叫来荷叶和松果。 
        “现在你立即出去采访。”总编辑对松果说。 
        “采访什么?”松果还摸不着头绪。 
        “到老鼠居民的家里去,问问他们对鼠王下台的看法什么的。”总编辑吩咐,“一小时后回来。” 
        松果离开飞机。 
        “我干什么?”荷叶编辑请求。 
        “你写一篇文章,说说鼠王下台的经过。”舒克边想边说,“还应该有一篇发刊词。” 
        “发刊词我可不会写,听说这样的文章都是总编亲自写。”荷叶说。 
        “发刊词我写。通讯你写。”总编辑开始埋头写作。 
        对于舒克来讲,并未觉得写文章比开飞机难。 
        很快,他将发刊词写好了,全文如下: 
        发刊词 
        我们想办《老鼠报》,我们就办了。想看的人就看,不愿看的人就别看。 
        后来贝塔说这是全世界最精彩的发刊词,没一句废话。大家也认定舒克是大手笔,有写作才能。 
        编辑的文章也写好了,交给总编辑审查。 
        舒克边阅边修改。荷叶站在一旁直冒汗。她很喜欢编辑这工作,她害怕失去。 
        “以后努力提高业务水平。”舒克说。 
        “通过了?”荷叶问。 
        “第一次写成这样就不错了。”总编辑点点头。 
        这时,记者松果回来了。 
        “怎么样?”舒克问。 
        松果擦擦头上的汗,说:“我还得整理一下。” 
        “快点儿,半小时后咱们起飞去印刷厂付印”总编辑检查飞机去了。   \chapter{第69集} 
        总编辑舒克将记者松果的文章改为《挑毛病专栏》; 
        印刷厂遭劫   
        松果趴在办公桌上写稿。 
        “我写的文章已经通过了。”荷叶沾沾自喜地告诉同事。 
        松果连头也顾不上抬。 
        舒克检查完飞机,走到松果身边。 
        “交稿。”记者将笔往桌上一扔,拿起稿件递给总编辑。 
        舒克边看边皱眉头。这篇采访写得驴唇不对马嘴,错别字连篇。 
        修改的时间没有了。舒克灵机一动,干脆来个“请读者挑毛病”专栏吧,让读者挑出这篇文章的毛病,不是还可以提高读者的阅读水平吗? 
        尽管松果有点儿伤自尊心,可也无可奈何。 
        总编辑将稿件收集好,驾驶直升机起飞。荷叶和松果头一次坐飞机,趴在窗户上往外瞧个没够。 
        直升机飞临《晚报》社上空,舒克看见草丛中贝塔为飞机着陆设置的标志,他操纵直升机缓缓降落。 
        贝塔等候在草丛中,坦克停在一边。 
        “印刷厂建好了。”贝塔说。 
        “真快!在哪儿?”总编辑对部下的工作效率感到满意。 
        “在《晚报》的印刷车间里,东西都是现成的。”贝塔像个经理。 
        “危险吗?” 
        “咱们的印刷厂藏在一堆铅字后边,一般发现不了。” 
        “现在就去开印。”总编辑拿上稿件,然后吩咐荷叶跟着去,留下松果看守飞机和坦克。 
        他们来到印刷车间,工人们还在干活。 
        “跟我来。”贝塔领着舒克和荷叶绕过人群,顺着墙根儿来到《老鼠报》的印刷厂。 
        头版欢迎总编辑的到来。 
        舒克见了头版的家人,并无成见。这使头版全家十分感动。 
        印刷设备隐蔽在暗处。舒克把稿件递给头版。 
        “还没划版?”头版是内行,“咱们不用打校样,直接排版。” 
        “你就负责排吧。”总编辑把排版的任务交给了印刷厂长。 
        头版忙起来。 
        这时,一位印刷工人到铅字堆旁取铅字,他无意中发现了一只老鼠。他悄悄地探头一看,啊,一群老鼠。 
        他就是由于刊登不让居民投放鼠药而被撤职的原《晚报》总编辑,他恨死老鼠了。 
        “抓老鼠呀!”原总编辑嚷嚷起来。 
        “不好,快分头跑!去草丛集合!”舒克大喊。 
        “印刷厂怎么办?”头版问。 
        “不要了!”舒克当机立断。 
        大家四散逃命。 
        车间里的工人们拿着扫帚拖把朝这边围过来。 
        “老鼠在哪儿?” 
        “藏哪儿了?” 
        “这是什么?” 
        工人们围着看《老鼠报》的印刷厂。 
        老鼠们溜了。 
        “它们还要办报?”原《晚报》总编辑有点相信老鼠会往人的碗里放毒药了。   \chapter{第70集} 
        印刷厂搬进飞机和坦克里; 
        总编辑签字付印创刊号   
        《老鼠报》社全体工作人员平安到达草丛里。 
        “贝塔,你认出来了吗?刚才那个工人就是《晚报》原来的总编。”舒克气喘吁吁地对贝塔说。 
        “没错,是他。”贝塔一边擦汗一边说。 
        “咱们的印刷厂完了。”头版伤心。 
        “没关系,重新建。”舒克充满信心。 
        “出报得推迟了。”贝塔有些惋惜。 
        “稿子在这儿。”荷叶把她抢出的稿子递给总编辑。 
        舒克这才想起稿子不在自己手里了。看到荷叶在危急关头还能注意到稿子,舒克对她刮目相看。他考虑在适当的时候提拔荷叶当编辑室主任。 
        “咱们不能在车间里建印刷厂了,太危险。”贝塔说。 
        “嗯。”舒克点点头,  “把印刷厂建在直升机和坦克里。” 
        “太棒了,在直升机里排字,在坦克里印刷。”头版兴奋。 
        “说于就干!”贝塔卷起袖子。 
        “我去侦察一下,看看他们的动静。”头版离开草丛,朝印刷车间跑去。 
        “当心点儿!”舒克叮嘱。 
        一会儿,头版回来了。 
        “他们又开始干活儿了,咱们的印刷厂被破坏了。”头版说。 
        “重建一座。”舒克说,“头版,你指挥。” 
        头版来劲儿了,他开始分工。 
        大家先将直升机和坦克收拾好。直升机舱里除了编辑部外,又隔出一间排字房。这样倒方便了,总编辑审完稿后,直接就进排字房排字。坦克舱里也收拾出安放印刷机的位置。 
        然后,头版带领几名老鼠去运印刷机,头版的哥哥带几名老鼠去运铅字。 
        经过几个小时的奋斗,印刷厂终于在直升机和坦克里竣工了。 
        “咱们离开这个地方,找个安全的地点印刷。”舒克招呼大家上飞机和坦克。 
        编辑和排字工上了直升机,贝塔和印刷工们进了坦克。 
        直升机吊着坦克起飞了。舒克感到驾驶杆很重,直升机还是头一次负荷这么大。 
        舒克寻找合适的着陆点。 
        “那座高楼上有个大平台。”贝塔通过无线电告诉舒克。 
        舒克看见了。他操纵直升机降落在高楼顶上。 
        贝塔打开坦克舱盖,探出头,看看四周没有危险,然后跳出坦克,招呼大家出来。 
        《老鼠报》社的工作人员们纷纷从直升机和坦克里钻出来。 
        “咱们抓紧出报,贝塔你负责警卫,其他人各就各位。”总编辑舒克说。 
        贝塔将稿件交给头版排字,头版的哥哥在排字间忙碌着。 
        版拼好了,头版印出小样让总编辑检查。 
        “如果没问题,请签字付印。”头版对舒克说。 
        舒克一本正经地检查了一遍,郑重地在校样上签了字。 
        版被运进坦克舱,印刷机转起来。 
        “印多少?”头版的爸爸从坦克里伸出头来问总编辑。 
        “5000份。”舒克说。 
        “坦克里装不下,怎么办?”头版的爸爸为难了。   \chapter{第71集} 
        《老鼠报》在空中印刷; 
        老鼠世界第一份报纸出版   
        贝塔想了个办法,他打开坦克底盘上的紧急出口。 
        “让报纸从紧急出口出来。”贝塔对印刷工说。 
        印刷机又开始转动了,报纸飞快地从坦克下边的出口飞出。不一会儿,坦克下边的报纸就把坦克顶了起来。 
        “停机!停机!”贝塔忙叫。 
        印刷机关闭了。 
        “怎么了?”舒克问贝塔。 
        “这么下去,我的坦克被越顶越高,会掉下来摔坏的。”贝塔爱惜他的坦克。 
        “又不能印了?”舒克惋惜地耸耸肩。 
        “用你的直升机把坦克吊起来再印。”贝塔想了个主意。 
        “不错。”舒克启动直升机。 
        直升机悬停在坦克上空,用钩于钩住坦克,将坦克吊到空中。 
        “开印。”贝塔坐在坦克里下令。 
        印刷机运转了。 
        由于坦克距离楼顶太远,印出的报纸被风刮走好多。 
        “低点儿!降低高度!”留在楼顶上的头版大喊。 
        直升机凋整高度,直至印出的报纸正好落在楼顶上。 
        舒克坐在驾驶舱里感到惬意。他望着地面上不断增多的报纸,他没想到自己还能当上总编辑。 
        耳机里传出贝塔的喊声,通知舒克报纸已全部印完,可以着陆了。 
        舒克和直升机着陆后,大家争先恐后地抢报纸看。 
        “真不错!”贝塔边看边夸。 
        “这是咱们老鼠家族的头一张报。”荷叶兴奋地说。 
        “咱们用这张报改变改变老鼠家族的品质。”舒克边说边将报头上“总编:舒克”几个字连看了100遍。 
        “下边就是发行的问题了。”头版提醒贝塔。 
        “没问题,我包了。”贝塔胸有成竹。 
        贝塔钻进坦克,要通了臭球的电台。 
        “我是贝塔!报纸已出版,快来拿!”贝塔说。 
        “什么报纸?”臭球已经忘光了。 
        “《老鼠报》呀!你不是汀了100份吗?”贝塔提醒臭球。 
        “噢,对!对!我马上亲自去取。方位?”臭球还来个亲自! 
        贝塔告诉他。 
        不一会儿,天空中出现了大型运输机。 
        “请导航。”臭球提要求。 
        舒克拿起话筒,指引臭球着陆。 
        “这家伙现在还真行,什么飞机都能开。”贝塔边看飞机着陆边说。 
        臭球从飞机上走下来。 
        “报纸出得真快。”臭球拿起一张报说。 
        “给你100份。”贝塔将捆好的报纸交给臭球。 
        “别急,”臭球顾不上接,还在埋头看报,“呆板了些,应该登些文艺作品,还有笑话什么的。” 
        舒克设想到臭球先提意见,虽然他心里不舒服,可还是觉得臭球说得有道理。 
        “第一期仓促了点儿,以后就好了。”舒克说。 
        “还没发现海盗?”臭球问。 
        “没有。”贝塔摇头。 
        “好,我走了。第二期出版后我再来拿。”臭球同机组人员一起将报纸运上飞机。   \chapter{第72集} 
        贝塔开坦克送报; 
        记者松果向总编辑报告《老鼠报》发行情况   
        “下边看你的啦!”舒克对贝塔说。 
        “天一黑我就出发。”贝塔信心十足。他吩咐头版:“把坦克里装满报纸。” 
        头版和印刷工们将一部分报纸装进坦克。 
        太阳慢慢落山了。 
        “咱们吃饭吧。”舒克从飞机里取出一些食物,分给大家。 
        大家都觉得这种生活很有意思。 
        吃完饭,天黑了。 
        “你得把我送到地面上去。”贝塔对舒克说。 
        舒克驾驶直升机将贝塔的坦克吊到地面上,贝塔从坦克里探出头来.说: 
        “你回去等好消息吧,我回来后用电台通知你。” 
        “当心,常通话。”舒克驾驶直升机返回楼顶。 
        贝塔把坦克舱盖儿关严,坐在驾驶座位上,操纵坦克向前驶去。 
        这是一片草坪,绿油油的草叶摇来晃去,贝塔透过草叶看见了一只小老鼠。 
        小老鼠听见坦克的声音,警惕地望着这个怪物。 
        贝塔把坦克停在小老鼠附近,他探出头来:“别怕,我也是老鼠。” 
        小老鼠感到惊奇。 
        “你家在哪儿?”贝塔问。 
        “干什么?”小老鼠起疑。 
        “别紧张。听说过舒克贝塔吗?”贝塔问。 
        “就是打败鼠王的那两个勇士?” 
        “对,就是那两个勇士!”贝塔很满意“勇士”这个头衔。 
        “当然知道。” 
        “我就是贝塔。” 
        小老鼠信了。 
        “我们办了一份报纸,送给你家几份。”贝塔边说边递给小老鼠一份报纸。 
        “用坦克送报纸?”小老鼠觉得有点儿那个。 
        “我没有别的交通工具。’贝塔也意识到全副武装送报纸挺吓人。 
        小老鼠翻看报纸。 
        “你家几只鼠?”贝塔问。 
        “17只。” 
        “给你五份,轮着看吧。”贝塔又递给小老鼠四份。 
        “不要东西换?自给?”小老鼠不信。 
        “当然是自给。这叫赠送。”贝塔准备走了。 
        “你一家一家送报?”小老鼠问。 
        “对。” 
        “那得多少时间!一个星期也送不完。” 
        贝塔一想也真够麻烦的。 
        “我让我们全家帮你送吧?”小老鼠热情地说。 
        “他们同意吗?”贝塔喜出望外。 
        “咱们去同他们谈谈。” 
        贝塔的坦克尾随着小老鼠来到他家门口。贝塔没想到非常顺利,小老鼠全家一致同意帮助贝塔发行报纸。 
        贝塔将一坦克报纸全部交给了他们。 
        贝塔驾驶坦克回到高楼下边。他拿起话筒呼叫舒克。 
        舒克开直升机将贝塔的坦克吊上楼顶。 
        “这么快?!”舒克不信。 
        “你能这么快出报,我就不能这么快发行?”贝塔得意。 
        第二天下午,记者松果向总编辑汇报,有老鼠在市场上高价出售《老鼠报》。 
        贝塔傻眼了,他断定是小老鼠全家干的。他感到给老鼠送报不光应该用坦克,还应该架着机关枪。   \chapter{第73集} 
        《老鼠报》社空中转移; 
        贝塔报告紧急情况   
        总编辑舒克问发行部经理贝塔:“怎么回事?” 
        “我委托了几位老鼠帮我发行,大概是他们把报纸拿去卖了。”贝塔挠后脑勺。 
        “走,找他们算账去!”舒克火了。 
        “我自己去就行了,你们编报吧。”贝塔说。 
        “我用直升机送你下去。”舒克说。 
        “咱们换个地方怎么样?呆在这么高的楼上,实在不方便。”贝塔提议。 
        “行,换个地方。”舒克对大家说:“各就各位,准备起飞。” 
        大家分别登上直升机和坦克,直升机吊着坦克离开了楼顶,升到空中。 
        “那边有一片草丛。”荷叶指给舒克看。 
        舒克驾驶直升机飞临那片草丛上空。 
        “观察一下。”舒克对荷叶和松果说。 
        荷叶和松果从不同的方向往下看,没发现危险因素。 
        直升机在草丛中着陆了。 
        大家纷纷从飞机和坦克里钻出来,呼吸新鲜空气。 
        “侦察一下。”舒克让大家分头看看地形。 
        这片草丛两面临水,两面通陆,十分安全。 
        “咱们就在这里安营扎寨。”舒克对这里挺满意,“贝塔,体去找那窝老鼠吧!” 
        贝塔开着坦克走了。 
        “咱们开始编第二期《老鼠报》。”总编辑对记者和编辑说。 
        “什么内容?”记者等候总编辑指示。 
        “你把卖报的事写篇通讯,狠狠挖苦一下这些利欲熏心的同胞。”总编辑说。 
        “我呢?写什么?”荷叶问。 
        “你写篇小说。”舒克想起臭球的建议。 
        “可我从来没写过小说呀!”荷叶为难地说。 
        “写一次就会了。”总编辑说。 
        “什么叫小说?”编辑向头儿请教。 
        “小说……小说就是……小说,就是不是诗,不是剧本也不是散文的一种东西。”总编辑给小说下定义了。 
        “……”编辑茫然。 
        “你就编一个故事,编得让大家爱看,就行了。”舒克开导部下。 
        舒克坐在总编辑的座位上,看着记者和编辑伏案书写,满意地点点头。 
        “头版,准备排字。”舒克透过机窗叫头版。 
        头版从草地上站起来,走进机舱。 
        舒克审查了荷叶写的小说,认为完全达到发表水平。荷叶激动得脸都红了。 
        “嗯.不错,你继续努力,完全可以得诺贝尔文学奖。”总编辑把稿子交给头版。 
        机舱外传来轰隆隆的卢音。 
        “怎么回事?”舒克警惕。 
        “水里在举行舰模比赛。”头版的哥哥来报告。 
        舒克跑到临水的那边一看,水面上行驶着几十艘航模舰艇,有军舰,有油轮,有汽艇,对面岸上不少人在操纵舰模。 
        “贝塔回来了,他说有紧急情况!”松果跑到舒克身边报告。 
        舒克忙跑回到直升机旁边,只见贝塔正气喘吁吁地擦汗。 
        “咱们……这些同胞……真没治了!”贝塔上气不接下气,  “咱们的报纸成了稀有物品,身价越来越高,现在已经成了老鼠世界的钱,开始流通使用了。” 
        “什么?报纸成了货币?”舒克对自己的同胞的品质太不了解了。   \chapter{第74集} 
        老鼠同胞来抢“钱”: 
        舒克的直升机无法起飞; 
        航空母舰情况危急   
        “刚才我去找那家老鼠,他们一见我就扑上来要报纸,瞧,衣服都被撕破了。我好不容易打散他们,跳上坦克就跑,他们还在后边追呢,一会儿准来。”贝塔说。 
        “报纸当钱用?”舒克还半信半疑。 
        “千真万确。那家老鼠卖了那批报纸后,又被别的老鼠以更高的价卖出。当然你知道,咱们老鼠世界没有钱,只有易货贸易。现在大家就拿报纸当钱了。他们要来抢钱呀!”贝塔说。 
        “头版。你爬到树上了望一下。”舒克吩咐。 
        头版刚爬到树上就大喊起来:“他们来了,黑压压一片!” 
        舒克爬上树一看,身上不由打了个哆嗦。他边往下跑边说:“快准备起飞,咱们不是他们的对手。” 
        舒克做梦也没想到给老鼠同胞办报办出这么个结果来。看来品质这东西是难以改变的。 
        大家急忙钻进坦克和飞机。 
        舒克打开开关,发动机不转!舒克再按应急开关,还是不转!糟糕,电池没电了。 
        “贝塔,快把坦克上的电池卸下来给我,我的电池没电了!”舒克拿起话筒。 
        “不是一个型号!”贝塔提醒舒克。 
        舒克傻眼了。 
        “怎么办?”贝塔问,“他们已经接近这边了。” 
        舒克透过草丛看见了水中的一艘军舰模型。舒克跳下飞机,对坦克里的贝塔说: 
        “我去想办法把那艘航空母舰弄过来,你用坦克把直升机拖到水边去,快!” 
        舒克朝水边跑去,头版跟在他身后。 
        这艘航空母舰模型很大,甲板宽阔甲坦,足足可以停放几百架舒克的直升机。它正平稳地行驶在水中。 
        舒克跳进水里,朝航空母舰游去。头版也跳入水中。 
        舒克够不着航空母舰的船舷,头版拼命在水中托起舒克。舒克爬上了甲板,他回身拉上头版。 
        “去驾驶室!”舒克朝驾驶室跑。 
        舒克钻进驾驶室,密密麻麻的仪表令他眼花缭乱。他试着转了转轮舵,军舰不听他摆布。 
        “头版,去把那根天线拿掉。”舒克对头版说,他断定是那根天线在接收岸上发出的遥控信号。 
        头版三下两下把天线拆了。 
        航空母舰听从舒克操纵了!舒克让航空母舰朝贝塔的坦克停着的方向靠拢。 
        贝塔的坦克拖着直升机停在岸边。来抢钱的老鼠们已经拥过来,他们看见了坦克和直升机。 
        “快开上来!”舒克冲贝塔大喊。 
        贝塔的坦克拖着直升机驶上了航空母舰的甲板。舒克操纵航空母舰驶离岸边,几只冲在前边的老鼠掉进水中。 
        航空母舰上一阵欢呼。 
        贝塔从坦克里钻出来,冲进驾驶室,拥抱舒克。 
        对岸上操纵舰模的人们被航空母舰的异常举动惊呆了,他们不明白这艘舰模怎么了。 
        一位选手借来望远镜。 
        “老鼠!还有坦克和直升机!”他刚举起望远镜就大叫起来。 
        “截住它!它要跑!”一个人大喊。 
        十几名选手操纵自己的军舰朝航空母舰追过来。 
        “我去试试舰尾的那门大炮。”贝塔跳出驾驶室。   \chapter{第75集} 
        一场惊心动魄的海战; 
        贝塔驾驶坦克朝水里开   
        十几艘军舰朝航空母舰逼过来。有驱逐舰,有巡洋舰,还有炮舰和鱼雷快艇等。 
        贝塔朝航空母舰的护尾炮跑去。 
        气氛空前紧张。 
        舒克一边操纵航空母舰加快速度,一边对着话筒大喊: 
        “头版,去帮助贝塔装炮弹!松果,了望海面!荷叶,组织其他人员进船舱隐蔽!” 
        贝塔跑到炮塔旁边,他看见大炮下边有一箱炮弹。 
        头版跑过来。 
        “我帮你装炮弹,你瞄准。”头版说。 
        贝塔寻找目标。 
        一艘黑色的驱逐舰跑在最前边,气焰十分嚣张。 
        贝塔决定拿驱逐舰开刀。 
        炮弹塞进了炮膛。 
        贝塔边瞄准边喊:“目标——驱逐舰.放!” 
        轰! 
        驱逐舰起火了。 
        “打中了!打中了!”负责了望海面的松果嚷嚷起来。他出于职业习惯还想记录下这场面,遗憾的是报纸不办了。 
        “贝塔加油——”躲在舰舱里的荷叶喊。 
        贝塔没激动,他认为这成绩对于神炮手太寻常了。 
        一艘炮艇飞速朝航空母舰冲过来,那神态活像敢死队员。 
        贝塔把准星钉在炮艇上。 
        炮艇狡猾极了,开始左右晃动,曲线行驶,贝塔无从下手。 
        炮艇开炮了。 
        炮弹落在航空母舰周围,溅起大水花。 
        岸上一阵欢呼。 
        航空母舰躯体庞大,无法躲避炮艇。舒克急出一头汗。 
        一发炮弹落在航空母舰甲板上爆炸了,松果被冲击波从了望台上震下来,摔伤了。 
        贝塔红眼了。他在炮艇向左晃的时候,瞄准右边按下射击按钮。 
        炮艇被打成了两截,沉了。 
        舒克一高兴,头撞在舱壁上。 
        荷叶跑上甲板把松果背进船舱。 
        贝塔来劲儿了,冲着后边的几艘军舰打起了连发。又有两艘军舰起火了。 
        “没炮弹了!”头版喊道。 
        贝塔傻了。一艘鱼雷快艇劈开水面冲过来,那架势分明是要和航空母舰同归于尽。 
        “必须挡住它!”舒克冲贝塔喊。 
        贝塔想起了他的坦克。 
        坦克停在甲板上。贝塔朝坦克跑去。 
        贝塔钻进坦克,驾驶坦克开到甲板后边。贝塔将一发炮弹塞进炮膛,通过潜望镜瞄准了鱼雷快艇。 
        鱼雷快艇正准备发射鱼雷,被贝塔一炮击中,爆炸了。 
        贝塔开心极了,他喜欢在航空母舰上用坦克炮打海战。他忘记了是在甲板上,开着坦克随心所欲地驰骋。 
        贝塔的坦克离船舷只有一米了,他还不知道。 
        “快停住!”头版急了。 
        贝塔没有听见,他想离那艘巡洋舰再近点儿。 
        头版不顾一切地跨上坦克,掀开坦克舱盖儿。 
        “停车!”头版闭着眼睛喊。 
        坦克的半截车身已经悬空了。头版认定自己将随坦克一起掉进水里。   \chapter{第76集} 
        子弹擦过贝塔的头; 
        航空母舰“投降”; 
        豪华的舰舱   
        贝塔毕竟是久经沙场的坦克兵,他控制住了坦克。坦克的半个身子悬空在甲板外边。 
        头版大喊一声:“贝塔万岁!” 
        贝塔缓慢地倒车。坦克回到了甲板上。 
        敌舰还是不放过航空母舰。 
        由于航空母舰速度慢,包围圈渐渐形成了。有三艘军舰已经出现在航空母舰前方。 
        舒克无法操纵航空母舰往前行驶了。 
        贝塔从坦克里探出头来,他看见航空母舰四周都是军舰。 
        贝塔迅速从坦克里跳出来,往驾驶室跑。 
        一排子弹朝贝塔射过来。贝塔赶忙趴在甲板上。子弹擦着贝塔的头飞过去。 
        贝塔匍匐前行进入驾驶室。 
        “怎么办?”贝塔问舒克。 
        “这大家伙太笨,要不然早跑没影了。”舒克在这紧急关头,首先想到的是自己的声誉。 
        “先说说怎么办吧!”贝塔望着窗外的军舰群说。 
        “叫臭球吧!”舒克没别的办法,尽管他觉得动不动就搬援兵挺丢人。 
        “有电台吗?”贝塔问。 
        舒克将驾驶室里的电台打开。 
        “臭球!臭球!我是贝塔,清回答!!”贝塔呼叫。 
        “……” 
        “臭球!臭球!我是贝塔,请回答!!”贝塔头上出汗了。 
        “我是臭球!我是臭球!请讲!” 
        “我们遇到了危险,请求空中支援!” 
        “机种?” 
        “歼击机10架。强击机20架。轰炸机20架。” 
        “有备降机场吗?” 
        “有!太有了!我们现在航空母舰上。” 
        “航空母舰上没飞机?哪儿找的破船!” 
        “哎呀别罗嗦了,快来吧!”贝塔将方位告诉臭球。 
        “坚持住!”臭球关上电台。 
        舒克和贝塔松了口气。 
        头版冲进来,说:“一艘巡洋舰向咱们打信号。” 
        贝塔往右边看。 
        “它说让咱们投降,不然就开炮了。”贝塔懂信号灯语。 
        “回答投降。”舒克用缓兵之计。 
        贝塔按动航空母舰上的信号灯开关。 
        “它说让咱们往回开。”贝塔将巡洋舰的灯语翻译给舒克。 
        “慢慢往回磨。”舒克操纵航空母舰掉头。 
        航空母舰在十几艘大小军舰的武装押送下,向岸边驶去。 
        “我去舱里转转,你再开慢点儿。”贝塔说完离开驾驶室。 
        航空母舰的船舱很大。豪华无比。有医院、餐厅、卧室、还有电影院。 
        贝塔眼睛都看花了。 
        荷叶看见贝塔来了,说:“松果在医院里躺着。” 
        贝塔问:“仓库在哪儿?” 
        荷叶指指前边。老鼠都有天生的找仓库的本领。 
        贝塔走进库房。他看见了弹药箱,还看见了舒克的直升机所需要的电池。 
        贝塔拿了几节电池。往甲板上跑。当他站在甲板上时,发现航空母舰已经快靠岸了。岸上的人们还挺激动。   \chapter{第77集} 
        舒克决定抛弃直升机和坦克; 
        臭球率领机群狂轰滥炸; 
        一架飞机险些误打自己人   
        贝塔跑进驾驶室。 
        “有电池了!”贝塔对舒克说。 
        “来不及装了。”舒克指指临近的岸。 
        “咱们不能让他们抓住呀!”贝塔急了。 
        “告诉大家拿救生圈准备跳水。”舒克说。 
        “直升机和坦克呢?’,贝塔问。 
        “以后再想办法弄回来吧!”舒克拍拍贝塔的肩膀。 
        贝塔舍不得,可也没办法。 
        “听!”舒克精神一振。 
        天空传来飞机的马达声。 
        “臭球!”贝塔一拳砸在驾驶台上。 
        舒克拿起话筒。 
        “臭球!臭球!我是舒克!” 
        “我是臭球!请指明目标!请指明空袭目标!!” 
        “除了航空母舰,随便打!打呀!!”舒克挥手。 
        双方势力立刻发生了显著变化。 
        臭球率领的机群对军舰们开始了灾难性的袭击。 
        歼击机俯冲扫射。轰炸机投弹。强击机低空点射。几乎是同时,军舰都起火了。 
        岸上的人们愣了。今天是怎么啦,先是舰模失控,接着又不知从哪儿冒出一群类似航模飞机似的战斗机,狂轰滥炸。 
        “打!打!!使劲儿打!!!”贝塔乐得直蹦。 
        一架飞机朝航空母舰俯冲。 
        “错了,自己人!!”贝塔握着话筒喊。 
        幸好飞机没开炮,避免了一起大水冲龙王庙的恶性事故。舒克操纵航空母舰掉头,从两艘被击沉的敌舰中间穿过去。 
        眼看着航空母舰走了,岸上的人们无计可施。 
        机群在空中为航空母舰护航。 
        舒克同臭球通话。 
        “我们返航了?”臭球问。 
        “别急,下来呆会儿,我还有事呢。”舒克说。 
        臭球从空中俯瞰,航空母舰已进人一个大湖泊,没什么危险了。 
        “07号机留在空中警戒,其他飞机跟我在航空母舰的跑道上着陆。” 
        臭球率领机群在航空母舰的跑道上鱼贯着陆。 
        “哪儿搞的这家伙,真漂亮!”臭球钻出座舱,看着航空母舰说。 
        “办报纸办出来的!”舒克幽了一默。 
        “酷!”臭球说。 
        “我想乘这艘航空母舰去大海逛逛,行吗?贝塔。”舒克问贝塔。 
        “太行了!”贝塔说。 
        “请你把我妈妈和咪丽接到你的机场去。”舒克对臭球说,“帮我照看一下。” 
        “没问题。”臭球说。 
        “再给我留六架飞机。两架歼击机,两架强击机,两架轰炸机。”舒克说。 
        “可以。”臭球当即选出了留下的六架飞机和六名飞行员。 
        贝塔领臭球参观航空母舰。臭球嫉妒得要死。 
        “你可以回去了。”舒克拍拍臭球,“有事再叫你!” 
        “你自己有了战斗机,别动不动就搬援兵了。”臭球说。 
        臭球的机群返航了。   \chapter{第78集} 
        无敌号航空母舰驶向大海; 
        荧光屏上奇怪的黑影; 
        一级战斗警报   
        “怎么样,去大海闯闯吧?”舒克问贝塔。 
        “你还想去哪儿?”贝塔显然非常同意。 
        “去太空。”舒克说。 
        “航海对咱们来说可是新鲜事儿,得准备准备吧?”贝塔说。 
        “咱们先给航空母舰起个名字,叫无敌号怎么样?”舒克说。 
        “够俗的。”贝塔撇撇嘴。 
        “你起一个。”舒克说。 
        “就叫无敌号吧,一时想不出别的。”贝塔作出无可奈何的样子。 
        “你当舰长,我当空军司令。”舒克提议。 
        “不错。”贝塔没想到自己这辈子还能当一回航空母舰舰长。 
        “咱们到会议室召集大伙分分工,准备启锚。”舒克整整飞行服。 
        航空母舰的会议室宽大明亮,豪华型沙发围成两圈。大家都来到会议室,连受伤的松果也来了。 
        “大家愿意开着这艘航空母舰去大海玩玩吗?”贝塔跷着二郎腿问。 
        “当然愿意!” 
        “太棒了!” 
        “……” 
        没人反对。 
        “我担任这艘无敌号航空母舰的舰长。舒克当空军司令。我任命头版当炮长,松果当雷达兵,荷叶当医生,头版的爸爸当大师傅,头版的妈妈当……”贝塔行使着舰长的人事大权,感到十分过瘾。 
        舒克给飞行员编了队。 
        “各就各位,准备启锚!”贝塔一声令下。 
        水兵们奔向自己的岗位。 
        舒克将电池装进直升机。飞行员们将战斗机固定在甲板上。贝塔将坦克开进隐蔽舱。餐厅里传出香味儿。 
        头版检查了舰上的几十门大炮。 
        贝塔站在驾驶室里,对着话筒下令: 
        “无敌号启锚,目标——大海!” 
        无敌号航空母舰徐徐开动了。 
        “大海在哪边?”掌握轮舵的贝塔想到方向问题,问舒克。 
        “哪儿宽阔往哪儿开。”舒克说。 
        “我开直升机升到空中看看。”舒克说完跑出驾驶室,钻进直升机。 
        直升机离开了航空母舰,升到空中。从空中看航空母舰,舒克觉得好玩。他没想到自己还能拥有一座水上机场。 
        “贝塔,贝塔,往南开。”舒克通过元线电台指挥航空母舰。 
        “我这个舰长还得听他指挥。”贝塔耸耸肩,操纵航空母舰往南开。 
        航空母舰终于开到大海里,这已经是一个星期以后的事了。舒克和贝塔见到了大海,他们一下子觉得自己渺小起来。他们后悔不应该办报而应该把老鼠同胞都带到航空母舰上来看大海。船员们都到甲板上看大海。 
        “哎呀,哎呀,哎呀!”头版连说三声哎呀,没别的词儿。 
        荷叶连一句话都说不出来。 
      “舰长,快到雷达室来!”松果通过扩音器喊贝塔。 
        贝塔跑进雷达室。 
        “你看,那是什么?”松果指着雷达扫描荧光屏上的黑点儿时贝塔说。 
        荧光屏上的黑点儿越来越大。显然是离航空母舰越来越近。 
        “一级战斗警报!”贝塔拉响了警笛。   \chapter{第79集} 
        舒克潜水; 
        黑色的物体是什么; 
        舒克遇险   
        “怎么回事?”舒克跑来问贝塔。 
        “海里有不明物体在靠近本舰。”贝塔指指荧光屏。 
        “在水下?”舒克挠挠后脑勺,他的空军技术无法发挥优势。 
        “咱们缺个潜水员。”舰长贝塔自知失职,分工时欠周全。 
        “我去。”舒克说,“快拿潜水服来!” 
        “也只好这样了,反正你是飞行员,身体素质好。”贝塔吩咐部下去仓库取潜水用具。 
        舒克穿戴好潜水服,准备人海。 
        “潜水头盔里有通话器,注意联络,听舰长指挥!”贝塔对舒克下令。 
        “可当了一回舰长。”舒克冲贝塔一笑,跳进大海。 
        “飞机随时准备起飞。”贝塔下令,  “大炮准备射击。” 
        飞行员钻进座舱,炮手坐上炮位。 
        舒克来到大海里,他擦着航空母舰往下潜水。许多巨大的鱼在舒克身边游来游去,不过对舒克都挺友好。 
        “舒克,舒克,我是贝塔,请回答。”舒克的耳机里传出贝塔的呼叫。 
        “我是舒克,请讲。”舒克说。 
        “报告你的方位。” 
        舒克看看潜水表上的罗盘,报告着自己的方位。 
        “你已经接近了那家伙,注意!请尽快报告观察结果。” 
        舒克一回头,看见水中有一个黑色的物体,正缓缓地游着。 
        “我看不清是什么,我靠近它。”舒克告诉贝塔。 
        “注意安全。”贝塔叮嘱道。 
        舒克悄悄地游到黑色的物体身边,他摸摸它,很硬。 
        舒克绕着它游了两圈儿。 
        “不是动物。也不是植物。”舒克随时向贝塔报告。 
        “是船?”贝塔问。 
        “可它全身都在水里!”舒克否定了。 
        “潜水艇!”贝塔大喊一声。 
        舒克被提醒了,没错,是一艘潜水艇。 
        “向它发信号,问它是谁?”舒克对贝塔说。 
        过了一会儿,贝塔告诉舒克,没回答。 
        “我进去看看。”舒克决定历险。 
        “祝你好运!”贝塔相信舒克的胆量和运气。 
        舒克很容易就找到潜水艇的潜水舱,他的身体 
        刚一钻进潜水舱,舱门就关上了。 
        “我中计了。”舒克通知贝塔。 
        只听一阵巨响,舒克的身体离开了潜水舱,进入潜水艇内。 
        舒克的胳膊被控制住了,眼睛被蒙上了黑布。他被带到一个舱里。 
        “解开。”一个熟悉的声音。 
        舒克眼前一亮,他看见海盗坐在对面的沙发上,穿着海魂衫。 
        “没想到吧?我一直跟踪你们,你的雷达兵该撤职。”海盗笑笑。 
        “够刺激!”舒克对这次历险的惊险度和出人意料度表示满意。 
        “这回刺激你回老家!皮皮鲁救不了你了吧?”海盗打了个榧子。 
        舒克看看四周,他身后站着四名水兵。 
        “你征兵还挺有办法。”舒克说。 
        “老鼠家族,找人干坏事还不是召之即来?”海盗盯着舒克笑。 
        舒克摇摇头,为同胞的素质遗憾。 
        “报纸办得有效果呀!”海盗不知从哪儿摸出一张舒克出版的报纸,扔给舒克。 
        “你想怎么办?”舒克问海盗。 
        “我一按电钮,鱼雷就能击沉你的航空母舰。”海盗一乐:“可我不这样干,太损了。我想当航空母舰舰长,统率这座水上机场,当大海的主人。” 
        舒克转眼珠。 
        “这回你跑不了,别瞎费脑子了。当然,为了不使我在发生意外后感到遗憾,我现在就把你的尾巴割下来。我正缺一根腰带。”海盗像大元帅看俘虏似地看着舒克。 
        “我要是帮助你得到航空母舰呢?”舒克认为尾巴比航空母舰重要。 
        “航空母舰和尾巴我都要。”海盗向部下招手,“拿刀来。”   \chapter{第80集} 
        尾巴捆住海盗的手; 
        舒克按了发射鱼雷的按钮   
        部下把刀子递给海盗。 
        “把尾巴伸过来。”海盗冲舒克飞了个吻。 
        舒克老老实实走到海盗跟前,把屁股转过来。, 
        海盗弯腰抓住舒克的尾巴,另一只手握刀准备割。 
        舒克飞快地用胳膊夹住海盗的脖子,另一只手夺过海盗的刀,然后用刀尖顶住海盗的太阳穴。 
        水兵们冲上来。 
        “别动,再往前走我就把刀捅进去了。”舒克警告水兵们。 
        水兵们等海盗的指示。 
        海盗挥手示意部下听舒克指挥。他真后悔自己麻痹大意,让舒克钻了空子。 
        “都出去。”舒克说。 
        水兵们退了出去。 
        舒克用海盗的尾巴捆住海盗的手。 
        “带我去驾驶舱。”舒克用刀顶着海盗的后脑勺。 
        海盗顺从地带着舒克去驾驶舱。走廊里的水兵们死盯着舒克手中的刀,谁也不敢轻举妄动。 
        “这就是驾驶舱。”海盗停在一座小圆门跟前。 
        舒克推开门,里边有两名水手在操纵潜水艇。 
        “出来!”舒克对驾驶舱里的水手说。 
        “你是谁?”水手没见过舒克。 
        “出来!”海盗喝了一声,他感觉到后脑勺上的刀尖有变化。 
        水兵老老实实出来了。 
        舒克押着海盗走进驾驶舱,把舱门关死。 
        荧光屏上显示出无敌号航空母舰。 
        “电台在哪儿?”舒克问海盗。 
        海盗十分不情愿地把电台的方位告诉舒克。 
        舒克打开电台,戴上耳机,他调整着频率。 
        “贝塔,贝塔,我是舒克.请回答!” 
        “我是贝塔。你怎么搞的,半天不吭声。”贝塔质问。 
        “你猜我在潜水艇里碰见谁了?” 
        “反正不会是海盗。” 
        “正是海盗。”舒克加重了语气。 
        “说着玩吧?” 
        “我让海盗跟你说话。”舒克把话筒递到海盗嘴边,“你说。” 
        海盗真想一口吃了舒克,这会儿还拿他开心。可他不得不说,刀尖又动了。 
        “我是海盗。”海盗对着话筒说。 
        贝塔听山是海盗,他吓出了一身汗。 
        “我缴获了他的潜水艇。”舒克拿回话筒说。 
        “我给你记功!”贝塔拿出舰长的口气。 
        舒克不稀罕贝塔的嘉奖。 
        “操纵潜水艇往后退的开关在哪儿?”舒克问海盗。 
        海盗把发射鱼雷的开关告诉舒克。 
        “贝塔.我现在操纵潜水艇往后退,离开航空母舰。”舒克说完往下按电钮。 
        一阵巨响。 
        “怎么回事?”舒克一惊,他隐约感到上了海盗的当。 
        “无敌号中弹了!”贝塔大叫。 
        “你按的是发射鱼雷的按钮。”海盗狂笑起来。 
        舒克一拳打在海盗头上,海盗倒下了。 
        “情况怎么样?”舒克问贝塔。 
        “有两个舱进水了!”贝塔说。 
        “快堵住!我把潜水艇开出海面,你让头版打沉它!”舒克说。 
        “那你呢?” 
        “我能跑出去。”舒克说。 
        舒克蹲下又打了海盗一拳,他确信海盗真昏过去后,才放心地站在操纵台前研究怎么开潜水艇。 
        舒克的耳机里不断传来贝塔指挥抢救航空母舰的声音。 
        舒克渐渐摸着了潜水艇的门路。潜水艇向上升去。 
        海盗醒了,他轻轻地站来,悄悄走到舒克背后。   \chapter{第81集} 
        舒克将潜水艇浮出水面; 
        头版击碎潜水艇; 
        贝塔当不成舰长了   
        舒克早已从操纵台上的反光镜里看见了海盗,他猛一回手,给了海盗一拳加一脚。继续操纵潜水艇。 
        海盗又昏过去了。 
        “贝塔,贝塔,潜水艇马上就要浮出水面。在航空母舰右侧,准各打。”舒克喊话。 
        “你离开潜水艇我就打。”贝塔说。 
        舒克确信潜水艇已经浮出水面后,叫醒了海盗。 
        “快叫你的部下跑吧.一分钟后潜水艇将爆炸!”舒克给海盗解开捆着手的尾巴。 
        海盗不解地看看舒克,他不明白舒克干吗不处决他。 
        “快点儿!”舒克大喝一声。 
        海盗打开舱门跑了。 
        “贝塔,一分钟后开炮!”舒克说完扔掉话筒,跑出驾驶舱。 
        贝塔指挥头版操纵大炮瞄准了水面上的潜水艇。 
        “先别打,还没看见舒克出来。”贝塔对头版说。 
        “打呀!”舒克从航空母舰的船舷上伸出头来喊,他已经回来了。 
        头版终于有了显示他具备射击天才的机会,他冲着浮在海面上的近在咫尺的潜水艇乱打一通。潜水艇被打得碎片横飞,粉身碎骨。 
        “行啦行啦!”贝塔制止头版。 
        “报告舰长,缺口堵不住了,舱里进了好多水!”松果跑来报告。 
        贝塔和舒克跑进舱里一看,水已经漏进舱里许多,荷叶正指挥水兵们堵缺口。 
        “看样子堵不住了。”舒克遗憾地说。 
        “这是你的杰作,自己打自己。”贝塔拍拍舒克的肩膀。 
        舒克苦笑。 
        “怎么办?”贝塔问舒克。 
        “不要航空母舰了,咱们都坐飞机走。”舒克说。 
        “我的舰长当不成了。”贝塔喜欢当舰长。挺过瘾。 
        舒克耸耸肩。 
        “飞机坐得下吗?”贝塔能上能下,不当舰长就不当。 
        “轰炸机的弹舱里可以坐几位。”舒克已经想好了。 
        航空母舰在迅速下沉。 
        “都到甲板上去,准备登飞机。”贝塔命令部下。 
        舒克召集飞行员下达任务。 
        “我驾直升机,吊着坦克。歼击机和强击机护航。轰炸机用弹舱装人,千万别按投弹按钮。”舒克对飞行员们说。 
        飞行员们奔向自己的飞机。 
        这时海面上传来一阵“救命”声。 
        舒克一看,是海盗和他的水兵。他们在海水里扑腾着.喊叫着。 
        “给他们抛救生圈。”舒克不忍心看自己的同胞遭受灭顶之灾。 
        “救上来怎么办?”贝塔问。 
        “管他呢。先救上来再说。”舒克一边说一边往海里扔救生圈。 
        潜水艇的水手们在抢救生圈。 
        “上来一个捆一个。”舒克说。 
       头版找来了绳子。 
       海里的潜水艇水手都陆续爬上了航空母舰,又都陆续被捆起来——那他们也愿意。 
        海盗一爬上航空母舰就被五花大绑。 
        “见到您真高兴。”贝塔冲海盗点点头。 
        海盗不说话。他看着下沉的航空母舰,心里高兴,反正大家同归于尽。   \chapter{第82集} 
        海盗被囚禁在孤岛上; 
        舒克率领机群返回陆地   
        “准备起飞!”舒克一声令下。 
        海盗愣了。他忘了这是航空母舰,有机场。 
        “我们怎么办?”海盗的部下哀求道。 
        “一起走。不过你们得老实点儿。”舒克说。 
        “海盗呢?”头版问。 
        “也走。不过不能让他回到陆地上去。”舒克看看海盗,  “咱们来时见过一座孤岛,把他扔到孤岛上。” 
        航空母舰的甲板快和海水平行了。 
        “快登机!”舒克大喊。 
        大家押着俘虏分头登上飞机。 
        “歼击机队,起飞!”舒克坐在直升机驾驶舱里指挥。 
        歼击机风驰电掣般插进云霄。 
        “强击机起飞!”舒克下令。 
        强击机闪电般甩开航空母舰。 
        海水涌上了甲板。 
        “轰炸机,快起飞!”舒克一边下令一边操纵直升机吊着坦克离开了甲板。 
        一架轰炸机趟着海水起飞了。 
        “报告司令,02号轰炸机出现故障!”另一架轰炸机的飞行员向空军司令舒克报告。 
        舒克从空中往下一看,海水已淹没了02号轰炸机的起落架。再耽搁一分钟。轰炸机将葬身海底。 
        “什么故障?”舒克问。 
        “发动机转速不够。”飞行员回答。 
        “强行起飞!”舒克命令。尽管他知道发动机转速不够是不能强行起飞的,可他没有别的办法。 
        轰炸机像船一样劈开海水疾驶,它离开甲板后很长一段时间还在海面上滑行。 
        “慢慢拉杆,别太猛!”舒克提醒轰炸机飞行员。 
        轰炸机终于吃力地离开了海面,艰难地跃上天空。 
        机群向着陆地飞行。 
        航空母舰沉没了。 
        贝塔从空中俯瞰海面,他真想让舒克把海盗从直升机上扔下去。贝塔没当够舰长。 
        舒克看见了那座孤岛。岛上有植物。海盗饿不死。 
        “各机组在空中盘旋,我把海盗放到孤岛上去。”舒克通过电台指挥机群。 
        直升机吊着坦克在孤岛上着陆了。 
        “你就留在这里吧!”舒克把海盗押离直升机。 
        “不!不!!”海盗抗议。 
        “你还想回陆地?”舒克问。 
        “……”海盗盯着舒克,不回答。 
        “你总想着霸占东西,呆在这里最合适,这座岛归你霸占了。”舒克给海盗松绑。 
        海盗一屁股坐在地上。他承认自己喜欢霸占,可如果就剩下他自己,他觉得霸占的东西再多也索然无味。霸占是占给别人看的。 
        直升机起飞了。 
        海盗绝望地大喊了一声。 
        “有机会我一定来看看他。”舒克想。 
        松果和荷叶问舒克:“咱们现在去哪儿?” 
        “到机场就把他们放了。”舒克想想说。 
        经过五个小时的飞行,机群飞临陆地上空。 
        舒克开始同臭球通话。 
        “我们请求在机场着陆。”舒克说。 
        “同意。”臭球回答。 
        “又要回家了。”贝塔想。 
        臭球指挥战斗机陆续在跑道上着陆。 
        舒克操纵直升机直接降落在停机坪上。贝塔从坦克里跳出来,他揉揉眼睛,机场变化真大。 
        候机大楼扩建了,塔台增高了,跑道加长加宽了。 
        臭球朝直升机跑过来,后边跟着罗丘和机场的好多工作人员。   \chapter{第83集} 
        罗丘在宴会厅举行盛大宴会欢迎舒克 
        和贝塔; 
        舒克和贝塔开始新的生活   
        “你们干得不错。”舒克看着机场对臭球和罗丘说。 
        “创业最难。”罗丘回赠舒克一顶高帽子。 
        “先去客厅休息吧?”臭球提议。 
        “我们带回来一些俘虏。先把他们放了。”舒克说。 
        “俘虏?”罗丘问。 
        “要不是海盗捣乱,我们现在还在海上航行呢!”贝塔不无遗憾地说。 
        “海盗追到海上去了?”臭球吃惊。 
        “我们把他留在一座孤岛上了。”舒克说。 
        头版跑来报告舒克,俘虏都释放了。有几名不愿意走,想留下找个工作干干。 
        “行。”臭球点头。 
        “咱们去客厅休息,”臭球扭头对舒克说,“你妈妈已经接来了,她现在客厅等你呢。” 
        “你怎么不早说!”舒克撒腿往客厅跑。 
        舒克见到了妈妈,贝塔见到咪丽。又是一番热闹场面。 
        罗丘在宴会厅为客人们操办了盛大的宴会,大家频频为机场的创建人舒克和贝塔干杯。 
        “下一步干什么?”贝塔喝了口酒小声问舒克。 
        “咱俩还是走吧?”舒克说。 
        “头版他们呢?” 
        “留在机场工作。” 
        “不错。” 
        “就这么定了。” 
        舒克举起酒杯。大家静下来。 
        “我和贝塔准备明天走了……”舒克刚说了一半,就被人家打断了。 
        “我呢?”头版问。 
        “我呢?”松果问。 
        “还有我!”荷叶嚷嚷。 
        “我……” 
        “我……” 
        “你们留在机场,让臭球给你们分配工作。”舒克说。 
        “我是编辑。”荷叶大声说,“机场不需要编辑。” 
        “机场马上办报。”臭球给舒克解围。 
        “贝塔需要炮手!”头版提醒贝塔别白培养他。 
        “机场有高射炮。”罗丘说。 
        “我们会经常来看你们的!”贝塔干了一杯。 
        大家知道,舒克和贝塔喜欢冒险,喜欢过有刺激性的生活。他们不愿意有负担。 
        “祝你们好运气!”荷叶举杯。 
        “谢谢,谢谢!”舒克和贝塔眼眶湿润了。 
        第二天一早,大家到停机坪上为舒克和贝塔送行。 
        罗丘给舒克和贝塔准备了充足的食物。臭球给贝塔的坦克储备了足够的弹药。 
        舒克跨进直升机驾驶舱。贝塔钻进坦克。他们心里挺难受。生活就是这样,一会儿分离,一会儿合聚,有喜有忧。乐在其中。 
        直升机的螺旋桨开始旋转了。 
        舒克和贝塔又开始了新的生活。   \chapter{第84集} 
        雷电袭击直升机; 
        舒克和贝塔跳伞; 
        火车顶上历险   
        舒克驾驶直升机吊着坦克升到空中,贝塔坐在舒克身边。 
        “还记得咱们头一次打交道吗?”舒克想起了第一次见贝塔时打仗的情景。 
        “当然。”贝塔感到亲切,“你从空中往下压我的坦克,够狠的。” 
        “你把我的飞机挂在树上,也不善。”舒克说。 
        “咱们现在去哪儿?”贝塔向地面了望。 
        “不知道。”舒克说。 
        “去看皮皮鲁吧?”贝塔提议。 
        “行。”舒克调整航向。 
        直升机朝皮皮鲁居住的城市飞去。 
        “好像要变天。”贝塔发现天上有乌云在活动。那乌云仿佛张开血盆大口,要吞噬蓝天和白云。 
        舒克把头伸出机舱,吐舌头。 
        “马上着陆。”舒克说。 
        乌云的速度比直升机的速度快,一道霹雳般的闪电把天空劈成两半,直升机一阵剧烈晃动后,起火了。 
        “雷击!”舒克大叫。 
        贝塔打开通向后舱的门,一阵浓烟扑进驾驶舱。 
        “飞机失控!”舒克惊呼。 
        “怎么办?”贝塔问。 
        “跳伞!”舒克离开座位。 
        “飞机?坦克?!''贝塔急了。 
        “命要紧!”舒克顶着火焰跑进后舱,拿出两个伞包,扔给贝塔一个。 
        舒克踢开舱门。 
        直升机急剧下降。离地面不远了。 
        “快背好伞包!跳!”舒克把贝塔推出机舱,自己也跟着跳出去。 
        两顶降落伞张开了。 
        直升机和坦克冒着烟坠毁了。 
        舒克调整着降落伞的方向,同贝塔平行。 
        “我的坦克!”贝塔心疼地大喊。 
        “注意地面!”舒克提醒贝塔,“坦克还会有的。” 
        贝塔往下看,地面是一条铁路。 
        豆大的雨点从天而降,降落伞被打湿了。舒克和贝塔下降的速度越来越快,马上就要着陆了。 
        就在这时,一列火车呼啸着开过来,把舒克和贝塔同大地隔开。 
        舒克落在火车顶上,降落伞挂在火车顶上的排气筒上。舒克趴在车顶上,巨大的气流从他的后背席卷而过。 
        雨越下越大。 
        舒克往四周看看,没有贝塔的影子。 
        “贝塔!贝塔!!”舒克喊叫。 
        没有回答。同暴风雨比起来,舒克的呼叫声太微弱了。 
        贝塔没落到火车上?被火车轧了?舒克猜测。失去直升机舒克不在乎,可失去贝塔使他神思恍惚。 
        舒克掏出小刀,割断伞绳,他决定在车顶上找贝塔。如果没有,就跳车沿路往回找。 
        在火车顶上行走很危险,何况还下着雨。舒克的身体紧贴着车顶,一点儿一点儿挪。 
        爬上两节车厢,舒克看见了惊心动魄的一幕:一顶降落伞缠绕在车厢顶部的金属丝上,降落伞的伞绳飞舞在车厢的外边,贝塔的身体在随风飘动,他够不着车厢! 
        贝塔已经绝望了。 
        舒克爬到降落伞旁,他用力往车厢这边收伞绳,力图使贝塔的身体够着火车。 
        一阵狂风袭来,舒克差点儿被吹下火车。他把自己的尾巴拴在伞绳上当安全带,继续努力。 
        贝塔发现了舒克,他开始配合舒克。正好一阵风刮来,贝塔借着风力,身体挨上了车厢,舒克伸手拉住了他。   \chapter{第85集} 
        在行李架上休息; 
        贝塔发现一辆小卧车; 
        舒克想开汽车   
        “抓紧!”舒克喊道。 
        贝塔趴在车顶上,死死抓住伞绳。 
        “你当伞兵太差。”舒克说。 
        “全世界也找不到比我技术高的伞兵了。”贝塔缓过劲儿来了。 
        “怎么样?”舒克抹脸上的雨水。 
        “够刺激。”贝塔全身精湿。 
        “咱们去车厢里暖和暖和。”舒克说。 
        “走。”贝塔从身上摸出伞刀割绳子。 
        舒克和贝塔钻进通风管道。管道里很黑,但暖和。 
        “那儿有亮光。”贝塔指指前边。 
        他们朝亮光走去。 
        舒克趴在亮光上往下看,是行李架。行李架下边是卧铺车厢。 
        “有人吗?”贝塔问。 
        “很多人。”舒克说。 
        “能下去吗?” 
        “小心点儿。” 
        舒克先下去了,他躲在一个旅行包后边。贝塔随后跳下去。 
        车厢里热气腾腾,对刚从雨中进来的舒克和贝塔来说是天堂。 
        “把衣服脱下来拧干。”贝塔说。 
        “完全必要。”舒克同意。 
        他们把飞行服和坦克服脱下来。 
        “没有飞机和坦克了,还穿这个干什么?”贝塔发牢骚。 
        “留个纪念嘛。”舒克说。 
        “注意。”贝塔警告道。 
        一个人伸手到旅行货架上找东西。舒克和贝塔忙躲在旅行包后边。 
        那人找完东西下去了。 
        “这火车往哪儿开?”贝塔嘀咕。 
        “天知道。”舒克伸了个懒腰。 
        贝塔往下看,车厢里有人在打扑克,有人看书,有人睡觉。 
        “也不知道他们都去干什么?”贝塔自言白语。 
        “忙呗。活着就是忙。”舒克说。 
        “你看那边!”贝塔的声音有些激动。 
        舒克顺着贝塔指的方向看,行李架上有一个华丽的玩具汽车盒子。 
        盒子上画的汽车神气极了。 
        “去看看。”舒克的手痒痒了。 
        他俩蹑手蹑脚地爬到汽车盒子旁边。盒子用塑料绳捆着。 
        “钻进去看。”舒克掀开盒盖的一角。 
        贝塔和舒克先后钻进去。 
        一辆黑色的超豪华小轿车呈现在舒克和贝塔眼前。 
        “真棒。”舒克脱口而出。 
        “是遥控的?”贝塔判断。 
        “进去看看。”舒克拉开车门,坐在驾驶员的位置上。 
        贝塔坐在舒克身边。 
        车里宽敞舒适,设备都是一流的。 
        舒克转转方向盘,挺过瘾。 
        “这比坦克好开多了。”贝塔冒出这么一句。 
        “跟飞机更没法比了。”舒克说。 
        这时,汽车忽然晃动起来。 
       “注意,他们要打开盒子!”舒克说。 
        “咱们这叫自投罗网。”贝塔低头看看座位下边,“这底下可以藏。” 
      盒子没有打开,却继续晃动着。 
      “汽车的主人下火车了。”舒克断定。   \chapter{第86集} 
        小个子和眼镜的交易; 
        舒克开汽车的技术不如贝塔; 
        舒克拒绝小个子的要求   
        舒克猜对了,汽车的主人到站下火车了。 
        “怎么办?”舒克问贝塔又像是在问自己。 
        “听天由命,反正现在跑不出去。”贝塔整整自己的坦克服。 
        汽车继续晃动,舒克练习想像力: 
        “现在上公共汽车了。现在下车了。现在快到家了。” 
        汽车终于停止了摇晃。 
        “你蒙得还挺准。”贝塔说完准备开车门下去。 
        “等会儿,听听动静。”舒克制止贝塔。 
        果然,盒盖被掀开了。一只大手伸进来将汽车从盒子里拿出去。 
        “就是这车,样子很棒,但性能不行。我们厂长说,请你多关照。”一个粗声音说。 
        舒克趴在窗玻璃上往外看,拿车的是一个三十多岁的男人,小个子。 
        “这回来参展的玩具很多,不好照顾呀。”另一个人说。 
        舒克顺着声音看去,是一个戴眼镜的中年男人。 
        “这是点儿小意思。”小个子塞给眼镜一个信封。 
        眼镜打开一看,是钱。笑了。 
        “我会尽力的。”眼镜改口道。 
        小个子把汽车放在地上,同眼镜聊天。 
        舒克和贝塔渐渐听明白了,这里要举行玩具博览订货会。许多厂家都将自己的产品送来参展,寻找销路。小个子的工厂生产的这种汽车外观华丽,但性能不好。他们收买了博览会的评委工作人员,弄虚作假。 
        “够邪的。”贝塔吹了声口哨。 
        “现在是机会,走吗?”舒克问。 
        “怎么着也得开车过一下瘾吧?急什么?大风大浪都闯过来了。”贝塔想开汽车。 
        “行,行。”舒克没意见,好在窗玻璃是茶色的,外边看不进来。 
        小个子送客去了。 
        舒克发动汽车。马达声很大。 
        “发动机质量够呛!”贝塔亮出行家的口气。 
        舒克很快就掌握了驾驶汽车的技术。 
        “人真蠢,听说考个驾驶执照要几个月时问,其实半个小时就能学会。自己折腾自己。”舒克边开边说。 
        “让我开一会儿。”贝塔说。 
        舒克把驾驶员的座位让给贝塔。 
        舒克不得不在心里承认,贝塔开汽车的技术比他高,毕竟是开坦克出身。 
        小个子送客回来,走到房间门口听见屋里汽车响,他悄悄把门推开一道缝儿往里看,愣了。 
        汽车自己在地上来回跑着,没有人操纵!而且动作灵活,令人眼花缭乱。 
        小个子突然推开门,闯进屋里从地上拿起汽车。 
        当舒克和贝塔反应过来时,车门已经被小个子打开了。 
        他惊讶极了,车里是两只穿着衣服的小老鼠! 
        贝塔冲小个子点点头,无可奈何地苦笑一下。毕竟是玩具工厂的推销员,小个子很快就接受了这个现实,这两只小老鼠会开汽车,而且开得很好,绝了!小个子心里萌生了一个伟大的念头。他后悔给眼镜塞钱塞早了。 
        “听着,我不抓你们。但明天你们帮我表演。懂吗?我拿着遥控器,你们开车,明白吗?”小个子对舒克和贝塔说。 
        舒克和贝塔明白了,小个子要他俩帮他作弊。小个子要在博览会上假装拿着遥控器遥控汽车。其实呢,汽车是由舒克和贝塔操纵。这样就能让人觉得汽车菲常灵活。 
        “做梦。”舒克忍不住冲口而出。 
        小个子没吃惊,他早就断定会开车的老鼠准会说话。他把车门一关,从外边用胶带封死了。   \chapter{第87集} 
        贝塔被扣为人质; 
        舒克在玩具博览订货会上表演开汽车   
        小个子还怕不保险,又把汽车塞进床头柜里,锁住。 
        舒克试了试,车门根本打不开。 
        “别管他,咱们先睡觉。”贝塔在后座上躺下。 
        “睡就睡。”舒克躺在前座上。 
        舒克和贝塔在睡梦中被一阵大喝声惊醒了。 
        “快起来,睡得真舒服呀!”小个子拉开车门喊。 
        舒克和贝塔坐起来。 
        小个子伸进两个手指,把贝塔从车里夹出去。舒克急了。 
        “别急。你今天好好表演,我不会亏待你们。如果捣乱,他就是人质,我把他交给猫。”小个子说完把贝塔塞进准备好的铁盒子里。 
        舒克惊讶小个子能想出这么恶毒的办法,他无计可施。 
        “现在就去参加订货会,我冲你打手势,你就开车,越灵活越好。”小个子给舒克下指令,“你的同伴就在我的提包里,你要不老实,他可要吃苦了。” 
        舒克想一口吃了小个子。 
        汽车又被装连纸盒子。舒克透过纸盒子上的玻璃纸窗,看见小个子走进一座富丽堂皇的展览馆。展览馆里有许多各种各样的玩具,许多围着玩具看,大概是准备订货。 
        小个子走到自己的展台前,从纸盒子里掏出汽车,放在展台上。 
        展台有两平方米,高出地面一米。 
        小个子拿出一盘磁带,插进录音机。 
        “这是本厂生产的最新式AQ——20型遥控汽车,性能良好,造型美观……”录音机的喇叭叫唤着。 
        围过来几个人。 
        “现在我给各位表演。”小个子假摸三刀地拿起遥控器。 
        人们的眼睛盯着汽车。 
        小个子冲舒克打了个手势。舒克不得不启动汽车,他怕贝塔受罪。 
        舒克驾驶汽车在展台上行驶着。他一会儿转弯,一会儿倒车。 
        “真灵活!” 
        “不错!” 
        观众赞叹着。 
        “我订五千辆!” 
        “我订两千辆!” 
        “我订一万!” 
        “……” 
        小个子应接不暇地签合同。 
        舒克试图打开车门逃跑,车门被小个子从外面封死了。 
        舒克坐在车里,看着那么多人轻而易举地上了小个子的当,而自己正是小个子的帮凶,他气坏了,越想越不甘心。 
        舒克想发动汽车从展台上捧下去,可又怕贝塔倒霉,只好老老实实坐在车里。 
        过了一会儿,小个子又让舒克表演。这回吸引了更多的人。人们惊叹这种遥控汽车的性能。 
        舒克整整开了一天车,小个子一点儿饭也不给他吃。 
        闭馆的时候,小个子从铁盒里拽出贝塔,把他塞进汽车。 
        “你表演得不错,你的朋友得谢谢你。”小个子替贝塔谢舒克,阴阳怪气。 
        舒克在心里骂了一句最难听的话。       \chapter{第88集} 
        舒克和贝塔撕毁合同书; 
        胖子和瘦子略施小计; 
        舒克和贝塔再次遇难   
        小个子把车门封死,然后将汽车塞进床头柜锁上。 
        汽车里一片黑暗。 
        舒克打开车灯。 
        “这家伙真坏,坑人。”舒克咬牙切齿。 
        “咱们得治治他。”贝塔掏出伞刀。 
        “把玻璃扎碎。”贝塔说。 
        玻璃碎了。 
        舒克爬出汽车。 
        “当心玻璃碴。”舒克提醒贝塔。 
        贝塔也从车里钻出来。 
        床头柜下边有一道缝儿,舒克试了试,太窄。 
        舒克和贝塔一起用刀扩张那条缝儿。 
        “行了。”贝塔收起刀,先钻出去。 
        舒克紧跟着钻出去。 
        房间里没人,小个子大概吃饭去了。 
        “把他订的合同都给撕了。”舒克说。 
        “太应该了。”贝塔说完爬上小个子的公文包。 
        公文包里是满满一包合同书,舒克和贝塔连撕带咬,合同书都粉身碎骨。 
        “还应该去告诉那些订货的人,别上当。”舒克提议。 
        “走!”贝塔同意。 
        舒克和贝塔从门缝儿底下钻到走廊上。这是一座旅馆,被参加玩具博览订货会的人包下了。 
        “咱们到每个房间去说。”贝塔提议。 
        “行。”舒克同意。 
        他们钻进第一个房间。 
        房间里两个人在煮方便面。 
        “今天那遥控汽车不错,我订了五千辆。”胖子说。 
        “我那儿子见了就不走了,我也订了不少。”瘦子说。 
        “那车质量不好,你们受骗了!”舒克大声说。 
        胖子和瘦子吓了一跳,看看门,关着。 
        胖子看看瘦子。瘦子看看胖子。 
        “你们上当了。”贝塔大喊一声。 
        胖子和瘦子往地上一看,两只老鼠。 
        闹鬼了! 
        胖子壮着胆问:“你们怎么知道?” 
        舒克把经过说了一遍。 
        胖子冲瘦子使个眼色,瘦子领会了胖子的意图,他朝门口走去。 
        舒克和贝塔的退路被堵死了。他俩还不知道等待自己的是什么。 
        “感谢你们来报信。”胖子蹲下来对舒克和贝塔说。 
        一条枕巾从天而降,扣住了舒克和贝塔。 
        瘦子得意极了,大叫:“抓住了!抓住了!” 
        “放哪儿?”胖子问。 
        “扣在玻璃杯里,一个杯子里一只。”瘦了说。 
        舒克和贝塔被分别扣在两只玻璃杯里,他们不明白这两个人于吗来这一手。 
        胖子和瘦子像看天外来客似地看舒克和贝塔。 
        “你说会说话的老鼠值多少钱?”胖子直接进人问题的实质。 
        “少不了。”瘦子说。 
        “咱们一人一只,怎么样?”胖子斜眼看瘦子。 
        瘦子点点头,没吭声。 
        舒克看着被关在对面杯子里的贝塔苦笑。       \chapter{第89集} 
        瘦子威胁舒克; 
        雷雷和舒克、贝塔交谈; 
        猫来了   
        “你们订的遥控汽车是假的!”贝塔在玻璃杯里冲胖子和瘦子大喊。 
        “现在什么不是假的?”瘦子反问贝塔。 
        “大惊小怪!”胖子笑笑,“假药,假酒,还少吗?你说玩具汽车是假的,又害不了人命,有什么关系?” 
        舒克和贝塔愣了,他们感到人和老鼠差不多,一个层次。 
        房间门开了,一个六岁左右的男孩子跑进来。 
        “雷雷,看爸爸抓了只什么?”瘦子叫儿子过来。 
        雷雷跑到桌子前一看,乐了。 
        “穿衣服的小老鼠!”雷雷兴奋了。 
        “还会说话呢!”胖子说。 
        “会说话?!”雷雷不信。 
        “说句话!”瘦子冲舒克说。 
        舒克不说。 
        “爸爸骗人!”雷雷给爸爸定性。 
        “你他妈说不说?”瘦子不能背这个黑锅,他一捋袖子,指着舒克骂道。 
        “你不说?我倒上汽油点了你!”胖子加强攻势。 
        舒克打了个哆嗦。 
        “别吓唬它们。”雷雷不满了。 
        “雷雷,你在这儿看着它们,我们出去一下。”瘦子对儿子说。 
        “去干吗?”雷雷问。 
        “小孩子不懂。”瘦子拍拍儿子的头,然后和胖子嘀咕了几句什么,两人出去了。 
        舒克听见了。瘦子说,既然汽车是假的,咱们订了不少,就应该敲小个子一下,让他给什么“回扣”。 
        雷雷趴在桌子旁,盯着舒克和贝塔看。 
        “你们真会说话吗?”雷雷问。 
        贝塔点点头。 
        雷雷惊讶了,瞪大了眼睛。 
        “你们怎么还穿着衣服?”雷雷问。 
        舒克觉得这个孩子的眼睛挺善良,决定和他谈谈。 
        “我是飞行员舒克,他是坦克兵贝塔。”舒克说,“我们是偶然来到这个展览会的……” 
        舒克把事情的经过讲给雷雷昕。 
        “我爸爸这么坏?”雷雷站起来。 
        舒克和贝塔看到了希望。 
        “我放你们走!”雷雷掀开了两只玻璃杯。 
        舒克和贝塔没想到瘦子生了这么个好儿子。真怪。 
        “谢谢你。”舒克说。 
        “再见!咱们还能见面吗?”看得出,雷雷舍不得和两只小老鼠分手。 
        “能见着!”贝塔说。 
        门外传来脚步声。 
        “快躲到门后去!”雷雷说。 
        舒克和贝塔藏到门后。 
        瘦子、胖子和小个子推门进来。 
        “老鼠呢?”瘦了一眼看见玻璃杯里的老鼠没了。 
        舒克和贝塔趁机溜出门外。他们听见“啪”的一记耳光。紧接着是雷雷的哭声。 
        脚步声。 
        “快进这个房间,他们追出来了!”贝塔拉了舒克一把,他们钻进临近的一个房间。 
        “跑不了,你快去餐厅把那只大猫抱来。我在这儿守着!”瘦子怒气冲冲地对胖子说。 
        小个子在一旁幸灾乐祸,当然他也希望抓住舒克和贝塔,出出气。 
        猫来了。   \chapter{第90集} 
        提包救了舒克和贝塔; 
        半夜里发生的事; 
        舒克和贝塔对世界失去信心   
        舒克和贝塔进的这间屋子黑着灯。他们钻到床底下。 
        走廊里一阵喧嚣。 
        猫嗅到了舒克和贝塔藏匿的房间门口。 
        “在这儿!”瘦子断定。 
        胖子敲门。 
        “找谁?”一个女人的声音从舒克和叭塔躲藏的床上发出。 
        “您的房间有老鼠!”瘦子说。 
        “老鼠?!”女人吓了一跳,从床上蹦起来,下地开灯。 
        “快躲进她的提包里!”舒克看见了靠在衣柜旁边的提包。 
        趁女人开门的机会,舒克和贝塔钻进提包。 
        猫冲进房间,在地上嗅着。它清楚老鼠就在屋里。 
        “对不起,除害嘛!”胖子冲女人笑笑。 
        “没关系,我最怕老鼠,也最讨厌老鼠。”女人不介意。 
        猫在床底下折腾了一阵,又跑出来,停在提包旁边。 
        猫围着提包绕了一圈。 
        “在提包里!”瘦子说。 
        女人的脸色变了,她怕别人开她的提包。 
        “不可能!这提包一直关着。”女人反对。 
        “打开看看?”胖子征求女人的意见。 
        “不行。”女人不干。 
        瘦子看看胖子,耸耸肩,无可奈何。 
        “还找吗?”女人问。 
        “算啦。”瘦子看了提包一眼,悻悻地说。 
        胖子抱起大猫。 
        他们走了。 
        舒克和贝塔在提包里松了一口气。 
        “也不知这包里装的是什么?”贝塔看着一个个纸包说。 
        “反正它们救了咱们。”舒克说。 
        “咱们在这里睡会儿,夜里再溜。”贝塔把身体挤进两个纸包中间。 
        “行。”舒克也选择了一个舒适的位置,睡了。 
        半夜,一阵晃动惊醒了舒克和贝塔。 
        提包被拉开了。一道刺眼的亮光射进提包里。 
        舒克和贝塔忙往里躲。 
        纸包一个一个被拿出去了。 
        跟看无处躲藏了,舒克和贝塔钻进了最后一个纸包。纸包里有一股令人窒息的气味。舒克想吐。贝塔捂着鼻子。 
        最后一个纸包被拿出提包,放在床上。 
        舒克探头一看,女人坐在床上,数着大把大把的钱。 
        提前睡觉,半夜点钱。舒克觉得这个世界被弄得充满了小家子气。 
        “你看她脸上。”贝塔小声对舒克说。 
        女人的脸上有哲学,有希望,有恐惧,有快感,有一切——当她数大把大把的钱的时候。 
        随着她的手指的移动,嘴唇的张合,舒克和贝塔对这个世界渐渐失去了信心。 
        “活着太难了。”舒克说。 
        “是。”贝塔同意。 
        他们想起了像海盗那样的同胞的霸占欲,想起了白路国王,想起了冷饮店的老板,还有小个子、胖子、瘦子…… 
        再加上这纸包里钞票的气味儿。 
        “咱们走吧。”贝塔说。 
        “去哪儿?”舒克无精打采。 
        “太空。” 
        “太空!怎么去?” 
        “我看见博览会上有宇宙飞船。”   \chapter{第91集} 
        贝塔开卡车去拉食物; 
        舒克检查宇宙飞船; 
        准备点火   
        舒克心里一震,乘宇宙飞船去太空?离开地球?! 
        的确,舒克和贝塔在这个星球生活得太难了。偷偷摸摸,躲躲藏藏,就因为有一个老鼠的外表。谁都可以正大光明名正言顺地欺负他们。舒克和贝塔渴望能堂堂正正地走在大街上,梦想能像其他动物一样同人类交往。 
        他们清楚这是不可能的。他们对在地球生活失去了信心。 
        “去太空!”舒克决定了,尽管他有些舍不得地球。地球是他的故乡,不能因为他是老鼠就剥夺了他有故乡的权利。 
        “怎么出去?”贝塔问。 
        “大摇大摆出去。现在她准不会大喊大叫。”舒克断定她数钱的时候不会叫。 
        正当女人准备数最后一包钱的时候,从纸包里钻出两只小老鼠。 
        她想喊,但自己用手捂住了自己的嘴。 
        两只老鼠当着她的面大摇大摆地出了屋子。 
        “咱们得谢谢钱。”贝塔在走廊里说。 
        “要是全世界的人每分钟都数钱,咱们就可以平安无事了。”舒克深有感触地说。 
        “快走,去找宇宙飞船。”贝塔说。 
        舒克和贝塔来到展览大厅,大厅里陈列着许多玩具。 
        一艘宇宙飞船醒目地耸立在大厅中央。 
        舒克和贝塔顺着扶梯爬上宇宙飞船,舒克打开舱门,钻进去。贝塔站在梯子上放哨。 
        宇宙飞船的座舱和直升机的大不一样。舒克凭自己的飞行经验判断着宇宙飞船的驾驶系统。 
        “怎么样,能开走吗?”贝塔把头伸进舱里问。 
        “没有问题。”舒克说。 
        “现在起飞?”贝塔迫不及待,他一刻也不想在地球上呆了。 
        “咱们到太空吃什么?”舒克想到食物问题。 
        “对,得弄足了食物。”贝塔拍拍脑袋。 
        “你去找食物,我在这儿熟悉一下操纵系统。”舒克吩咐。 
        贝塔顺着梯子下去,他找了一辆电动卡车,开着弄食物去了。 
        舒克检查了一遍宇宙飞船所有的舱,他对它的性能和设施很满意。 
        舒克把头伸出舱门,往上看。大厅的天花板是由玻璃组成的,必须打开玻璃窗,宇宙飞船才能发射出去。 
        贝塔的卡车拉着满满一车食物停在宇宙飞船旁。 
        “你真行。”舒克表扬贝塔。 
        “你在干什么?”贝塔抬头看舒克。 
        “不把上边打开,咱们出不去。”舒克指指天花板。 
        “玻璃挡不住宇宙飞船。”贝塔说。 
        “这倒是。”舒克一时想不出打开天花板的办法。 
        他们把食物运进宇宙飞船的储备舱。 
        贝塔也对宇宙飞船表示满意。 
        一切准备工作就绪。 
        “发射宇宙飞船不是每次都成功。”贝塔说。 
        “但愿咱们运气好。”舒克系安全带。 
        “就要离开地球了。”贝塔跟睛湿润了。 
        “……”舒克说不出话来,他的嗓子里像有一块东西堵着声道。 
        地球不容他们。 
         沉默了5分钟。 
        “点火啦?”舒克颤抖着声音说。 
        “点……点吧……”贝塔闭上眼睛。   \chapter{第92集} 
        舒克和贝塔飞向太空; 
        一顿神奇的饭; 
        不明飞行物出现   
        宇宙飞船点火了。 
        舒克和贝塔屏住呼吸,他们期望发射成功。 
        只听“轰”的一声巨响,宇宙飞船撞碎了展览大厅天花板上的玻璃,冲向夜空。 
        舒克和贝塔松了口气。 
        “和臭球他们告别。”舒克示意贝塔接通电台。 
        贝塔调整着频率。 
        “臭球,臭球,我是贝塔,我是贝塔,请回答!”贝塔呼叫。 
        “我是值班员,请等一下。”对方回答。 
        “请抓紧时间!”贝塔说。他清楚,宇宙飞船穿过大气层后,可能就通不上话了。 
        “我是臭球,我是臭球,请讲话!” 
        “我是贝塔。我和舒克现在驾驶宇宙飞船正离开地球,前往太空。” 
        “离开地球?”臭球吃了一惊。 
        “我们祝你们走运。请照顾好舒克的妈妈。请转告皮皮鲁。”贝塔说。 
        “为什么?为什么离开地球?还回来吗?”臭球的回话声越来越小。 
        “再见了,臭球!”贝塔哽咽了。 
        舒克咬着嘴唇。 
        宇宙飞船穿过大气层.进入了轨道,开始围绕着地球飞行。 
        “太空真美!”舒克从圆窗口往外看。 
        “宇宙太神秘了。”贝塔出神地说。 
        “现在不用驾驶了,宇宙飞船自己飞行了。咱们吃点儿东西吧。”舒克开始解安全带。 
        安全带一解开,舒克的身体就飘了起来。 
        “哎呀,失重!”舒克突然想起来了。 
        “真逗。”贝塔也飘起来。 
        “没有地球吸引力了。”舒克调整着身体飘向储备舱。 
        “等等我。”贝塔喊。 
        他们好不容易从储备舱里抓住了满天飞的食物,又好不容易飞回了驾驶舱。 
        “快,帮忙把我捆在椅子上。”舒克一边笑~边向贝塔求援。 
        “我捆上你,我怎么办?”贝塔也笑得喘不过气来。 
        舒克手中的一袋花生米撒了,花生米上下飞舞着。 
        贝塔和舒克张开嘴,追着花生米吃。 
        贝塔索性把手里的食物都扔了,舒克也扔了。他们同食物一起飘飞,抓住机会就吃。 
        舒克和贝塔玩得开心。太空里没人干涉他们,他们不用担惊受怕,可以尽情地大声喊叫,宣示生命的存在。 
        几天过去了,新鲜劲儿没有了。舒克和小塔开始感到寂寞。 
        “说点儿什么,”舒克说。 
        “话都说得差不多了。”贝塔说。 
        “这没有交往的滋味儿也挺难受。”舒克扒着圆窗往外看。 
        “被人歧视也是一种享受。”贝塔像是哲学家。 
        “快看,那是什么?”舒克叫起来。 
        贝塔往外一看,一个庞大的飞行物向他们的宇宙飞船飞过来。 
        “注意观察。”舒克打开操纵系统,准备应付突发事件。 
        不明飞行物继续向宇宙飞船靠近,显然它已经发现了舒克和贝塔的飞船。 
        “是什么东西?”舒克问贝塔。 
        贝塔揉揉眼睛。 
        “是宇宙飞船!人类的宇宙飞船!”贝塔大喊。 
        “摆脱它!”舒克说完开始操纵小宇宙飞船躲开大宇宙飞船。 
        晚了。   \chapter{第93集} 
        勇敢号字宙飞船发现“外星人”; 
        舒克和贝塔轰动地球   
        人类发往太空的勇敢号载人宇宙飞船在太空轨道运行的第49天发现了一架微型飞行器。 
        宇航员们争先恐后从荧光屏上看这神秘的飞行器。 
        “这么小!”一位宇航员叫道。 
        “请示地面!”机长命令负责通讯的宇航员。 
        “我是勇敢号,我是勇敢号。我们在太空发现不明飞行物。可能是外星生物,请指示。”宇航员向地球报告。 
        “设法接近它,争取将它带回地球。”地球指示勇敢号。 
        “明白。” 
        勇敢号做好了应付一切突然事变的准备后,开始小心翼翼地按近不明飞行物。 
        不明飞行物试图摆脱勇敢号的纠缠,但没有成功。 
        “贝塔,咱们的飞船进了它的肚子。”舒克无可奈何地说。 
        舒克和贝塔的宇宙飞船被勇敢号宇宙飞船“吸”进自己的舱内。 
        宇航员们看着这微型宇宙飞船,惊讶至极。 
        “我们和不明飞行物‘对接’成功!”勇敢号向地球报告。 
        “立即查明对方来历。”地球下达指令。 
        机长小心翼翼地抓住了微型宇宙飞船,他从圆窗往舱里看。 
        “外星人!”机长大叫。 
        舱内一片震惊。 
        宇航员们抢着先睹为快。 
        电波迅速将这一消息传给地球。地球在一瞬间目瞪口呆。 
        “立即转播实况!”地球清醒过来后迫不及待。要知道,她一直在寻找宇宙中的外星生命,地球太孤独了。 
        刹那问,整个地球都知道舒克和贝塔的光辉形象了。地球人类从电视荧光屏里一睹外星人的风采。 
        播音员说:“这两个外星人很像我们地球上的老鼠,当然他们和我们的老鼠有着质的区别,外形上也有很大不同。他们是高等动物,智慧生物,能驾驭现代科学技术。据悉,他们的服装的质地非常现代化,我们地球上还没有这种纺织品。据一位专家初步分析,可能是原子服装或超导服装。至于他们来自哪个星球,还有待于进一步考证。” 
        地球沸腾了。 
        播音员最后说:“告诉观众一个好消息,勇敢号宇宙飞船将把外星人带回地球。我们地球将以热情的姿态欢迎第一批光临地球的外星人!” 
        地球忙碌起来,打扫卫生,抢建宾馆;全世界的上千名字宙学家、历史学家、考古学家、心理学家、语言学家、教育学家、遗传学家……云集外星人将下榻的宾馆,准备研究外星人。同行是冤家,同行的科学家们剑拔弩张,要一决雌雄。 
        舒克和贝塔原以为等待他们的是歧视和侮辱,他们做梦也没想到自己到太空转了一圈,从根本上改变了自己的位置。 
        当勇敢号宇宙飞船带着舒克和贝塔返回地球时,这天成了地球的盛大节日。全球放假,几十亿双眼睛死盯着电视机。几亿根电视天线死咬着电视信号。 
        迎接他们的是鲜花和微笑,摄像机和照相机。   \chapter{第94集} 
        舒克和贝塔大出风头: 
        贝塔吐了一次舌头; 
        自有电视以来最昂贵的广告收费   
        舒克和贝塔一出机舱就被记者包围了。 
        上百名全副武装荷枪实弹的警察保护着舒克和贝塔的安全。 
        “怎么了?”贝塔小声问舒克。 
        舒克耸耸肩。 
        “他们交谈了!”电视播音员说,“据语言学家说,他们使用的是一种最简洁的语言,只有主语,没有谓语,没有形容词。这样可以节省大量时间。由此可见外星人是讲效率的。” 
        全球的观众十分羡慕外星人。 
        “把咱们当外星人了。天外来客!”舒克恍然大悟,对贝塔说。 
        “天哪!”贝塔吐了下舌头。 
        播音员对全球观众讲解:  “请大家注意!外星人吐舌头比我们有水平,比我们自然!时机也比我们掌握很好!请大家再看一次外星人吐舌头的慢镜头。瞧,分寸感多强!好,现在我们请著名的表情专家给我们分析一下外星人对吐舌头这一表情的运用。” 
        全球的人类跟着电视向舒克和贝塔学吐舌头。 
        舒克和贝塔乘坐超豪华小轿车在几百辆摩托车的护卫下驶向宾馆。 
        沿途的群众向车队抛掷鲜花和彩带。 
        舒克发现了几张熟悉的面孔,他仔细一看,是小个子,胖子和瘦子。他们起劲儿地扔鲜花和喊口号。 
        舒克和贝塔得意了,他们终于受到了人类的尊重和承认,他们决定享受这一现实。 
        舒克和贝塔分别向窗外招手致意。 
        电视播音员激动了: 
        “外星人终于向我们表态了!他们不沉默了!他们承认我们星球了!!!” 
        被承认的地球自豪了。地球的几千年文明发展被承认了!没白干。人类掉泪了。 
        在宾馆,当地最高首脑拜见了舒克和贝塔,他代表市民将本城的金钥匙赠送给外星人,并请舒克和贝塔担任该市的荣誉市民。 
        “我们非常荣幸。”贝塔张口说话了。 
        电视播音员几乎足喊起来: 
        “他们已经掌握了我们的语言!现在请著名语言学家白活博士分析!” 
        白活博士出现在荧光屏上。 
        “外星人只用了半个小时就掌握了我们的语言,可见他们的大脑结构比我们先进,比我们发达,比我们合理。他们的听觉和声带都是宇宙第一流的。我建议,应该让外星人帮我们改良人种,优生优育。”白活博士侃侃而谈。 
        立即有上千名妇女报名。 
        “能够来到地球,我们很高兴!”舒克也发话了,他不能让贝塔一人出风头,“地球比我们想像得要好一些。” 
        整个地球安静极了,洗耳恭听天外来客的教诲。 
        “请问你们的食物是什么?”一位负责外星人食宿的官员问。 
        “在我们的星球上,主要靠油炸花生米。当然,我们不叫花生米,叫巨豆。”贝塔回答。 
        所有商店出售花生米的牌子都更名为巨豆。 
        花生米立即在全球脱销。 
        “请问贵星球生命的历史?”一位记者问。 
        “比地球生命早四倍时间。”舒克说。 
        全球肃然起敬。 
        在电视实况转播采访外星人期间,播了一秒钟广告。该广告收费占全球生产总值的百分之四十。   \chapter{第95集} 
        有一个国家改名为舒克国; 
        舒克和贝塔觉得出名的滋味不好受; 
        皮皮鲁上主席台   
        从此,舒克和贝塔成了地球瞩目的中心。人类敬重他们,羡慕他们,崇拜他们。 
        研究舒克和贝塔的上万篇论文争先恐后地问世。根据舒克和贝塔为原型创作的电影、电视剧、小说、报告文学、连环画、戏剧充斥了影尉坛、文坛…… 
        舒克和贝塔住在豪华的饭店里,吃着山珍海味,享受着一流的服侍。 
        以舒克和贝塔的名字命名的街道、城市比比皆是,甚至有一个国家要改名为舒克国。 
        没有歧视,没有饥饿,不用为生存发愁,但舒克和贝塔觉得很累。 
        “有意思吗?”一天晚上,舒克问贝塔。 
        “够烦的。”贝塔说。 
        “我觉得恶心。咱们还是咱们,就因为当了外星人,待遇全变了。”舒克撇撇嘴,“我看人类也就那么回事,没一个人发现咱们和地球上的老鼠一模一样。” 
        “真可怜。”贝塔打心眼儿里同情人类。 
        “我看出名是最痛苦的事了。”舒克说。 
        “没错,出了名老得端着架子活。老想着怎么才能不辜负自己的名,活活能把人累死。这些天我自己都没了。” 
        “咱们走吧!” 
        “去哪儿?” 
        “不知道。”舒克为难了。的确,地球上是没法呆了,受尊重和受歧视都不好受,太空又太寂寞,唉,活着真难。 
        “咱们去找真正的外星人。”贝塔灵机一动。 
        “这主意不错。”舒克兴奋了。 
        “什么时候走?”贝塔问。 
        “咱们去看看皮皮鲁,还有臭球他们,然后就走。”舒克说。 
        “看皮皮鲁可以,看臭球会给他们带来危险。你想,咱们不能自由行动呀!走到哪儿都有人跟着,还有照相机和摄像机。”贝塔反对去见臭球他们。 
        舒克认为贝塔的话有道理。 
        “明天去见皮皮鲁。”舒克说。 
        “祝你做个好梦。”贝塔临睡前说。 
        “晚安。”舒克没说完就睡着了。 
        皮皮鲁早就从电视荧光屏上认出了舒克和贝塔,他要求见舒克和贝塔,但被有关方面拒绝了,理由是皮皮鲁的地位太低,现在只有国家元首级的人才能有幸亲眼见外星人。 
        这天,皮皮鲁正在上课,只见校长神情紧张而激动地把班主任从教室里叫出去了。 
        不一会儿,班主任叫皮皮鲁出去。 
        “又倒楣了。”皮皮鲁想。 
        “皮皮鲁,祝贺你!”校长伸出手来。 
        “……”皮皮鲁茫然。 
        “上边来通知,说外星人要见你!”校长满面春风。 
        皮皮鲁一蹦老高。 
        “中午1点接见,现在学校给你开欢送会。”校长不觉得恶心,也不知开哪门子欢送会,见外星人就在本市。 
        皮皮鲁终于坐上了学校会场的主席台。 
        欢送会后,皮皮鲁乘专车前往舒克和贝塔下榻的宾馆。 
        有关方而对外星人提出要见一个小学生感到吃惊,而且指名道姓。专家们更感到外星人具有遥感能力,能预知一切。 
        皮皮鲁和舒克、贝塔见面了,他们都很激动。在外星人的要求下,工作人员和记者都退下去丁。 
        舒克小声将事情的经过讲给皮皮鲁听。 
        皮皮鲁手舞足蹈。 
        “我们是来向你告别的。”贝塔说。 
        “告别?”皮皮鲁愣。   \chapter{第96集} 
        “外星人”离开地球; 
        一个布满大石头的星球; 
        食物告急   
        舒克把原因告诉皮皮鲁。 
        “那就走吧。”皮皮鲁像大人似地点点头。 
        “我们以后还会来看你。”贝塔说。 
        “我等着。”皮皮鲁相信。 
        会见结束。 
        贝塔按电铃叫工作人员进来。 
        皮皮鲁回学校参加学校为他召开的欢迎会去了。 
        舒克和贝塔当着记者宣布,他们将于明天返回自己的星球。 
        这一爆炸新闻立即传遍了全球。 
        地球忙起来了,它要用最隆重的仪式欢送外星人。各种欢送方案被送到“欢送外星人筹备委员会”,计算机辨别筛选最佳方案。 
        勇敢号宇宙飞船承担了将外星人送回太空的历史性重任。机组成员早已成为家喻户晓的传奇英雄。 
        舒克和贝塔就要离开地球了。他们回到地球时间不长,因研究他俩取得成果而晋升教授的就有五万人。因推销和他俩有关的商品而发财晋升为百万富翁的商人不下十万人。 
        人类决定将舒克和贝塔离开地球的这一天定为今后每年的“国际外星人日”,年年纪念。 
        勇敢号宇宙飞船就要起飞了,舒克和贝塔通过荧光屏看见了全世界欢送他俩的场面。 
        一想到整个地球在欢送两只货真价实的老鼠.舒克和贝塔就想笑。 
        勇敢号宇宙飞船点火了。 
        舒克和贝塔心里挺不是滋味儿。 
        地球越来越小。 
        勇敢号飞出了大气层,来到太空。 
        机长来到舒克和贝塔身边,恭敬地请示:“什么时间离机?” 
        “现在。”舒克说。 
        “祝你们顺利。”机长说。 
        “谢谢。”贝塔说。 
        舒克和贝塔的宇宙飞船离开了勇敢号,来到广袤无垠的太空。 
       “自由了!”贝塔大喊起来。 
        “咱们要摆脱地球吸引力,直飞外星。”舒克说。 
        “看你的了!”贝塔知道这需要很高的速度。 
        舒克全神贯注地驾驶宇宙飞船。 
        贝塔盯着仪表盘。 
        “再快一点儿!再快一点儿!还差一点儿!”贝塔给舒克加油。 
        宇宙飞船终于摆脱了地球的吸引力,像脱缰的野马,自由地向太空飞去。 
        舒克和贝塔心旷神怡。 
        “也不知哪颗星球上有生命。”舒克望着太空里众多的星球说。 
        “挨个找呗。”贝塔趴在圆窗上往外看。 
        “那儿有一颗。”舒克说。 
        “挺大,落上去看看。”贝塔一拍腿。 
        舒克驾驶宇宙飞船朝那颗星球逼近。 
        宇宙飞船在星球上着陆了。 
        “我出去看看。”贝塔离开飞船。 
        星球上全是石头,没有植物。更没有动物。 
        贝塔回到飞船上。 
        “没有。”贝塔耸耸肩膀。 
        “够荒凉的。”舒克往舱外望去。 
        “上帝也偏心,同样是星球。”贝塔想起地球上的景象。 
        “没有生命也不亏。”舒克说。 
        “那倒是。”贝塔赞同。 
        “再接着找。”舒克按动了起飞按钮。 
        宇宙飞船继续在太空寻找外星人。 
        舒克和贝塔又陆续在20多个星球上着陆,都没有生命。他们的食物不多了。   \chapter{第97集} 
        终于找到有生命的星球; 
        外星人见到外星人不吃惊; 
        外星人送给舒克和贝塔翻译机   
        “现在离地球很远了吧?”舒克问贝塔。 
        “太远了。”贝塔望望窗外。 
        “咱们的食物还够吃几天?” 
        “如果省着吃,还能吃三四天。” 
        “你不后悔吧?” 
        “不。你呢?” 
        “也不。” 
        “我去清点一下食物。” 
        贝塔来到储藏室,把剩下的一点儿食物清理了一下,还有两根香肠,一筒罐头,一包压缩饼干。 
        “贝塔,快来,又有一颗星球!”从驾驶舱传来舒克的喊声。 
        贝塔对新星球已没有兴趣了。他坚信根本没有外星人,宇宙里只有地球上有生命。 
        “快来!”舒克又叫。 
        ‘犬惊小怪。”贝塔边说边来到驾驶舱。 
        “快看!”舒克激动。 
        贝塔往窗外一看,一颗布满亮光的星球充满生机地出现在宁宙飞船的左侧。 
        “真漂亮。”贝塔情不白禁地喊。 
        “准备着陆。”舒克发令。 
        贝塔系好安全带,戴上耳机。 
        宇宙飞船靠近那颗美丽的星球。距离越来越近。 
        “有房子!”贝塔把安全带挣断了。 
        “扶好,我着陆了。”舒克看准了一片空地,驾驶宇宙飞船着陆。 
        宇宙飞船自信地着陆。 
        “他们该隆重欢迎我们。”贝塔预测。他想起地球人欢迎他们的情景。 
        “那可真够烦的。”舒克解开安全带。 
        字宙飞船四周是造型奇物的建筑,没有相同的房子。绿树环绕,碧水点缀,像仙镜。 
        舒克和贝塔从宇宙飞船里钻出来,他们伸伸胳膊踢踢腿,等待着外星人的出现。 
        “来了!”贝塔往左边指指。 
        两个外星人走过来,他们和地球人长得差不多,就是耳朵特别特别小。 
        他们看见舒克和贝塔没有表现出吃惊。他们像什么事也没发生那样,继续走路。 
        “请问,这是什么星球?”贝塔看外星人不理睬他们,追上去问。 
        外星人站住了,其中一个说: 
        “巴九红七蒲定川。” 
        “语言不通!”舒克对贝塔说。 
        贝塔冲外星人摇摇头,指指自己的耳朵。 
        外星人马上明白了舒克和贝塔听不懂他们的话。其中一个外星人掏出一个小仪器,调了调旋钮,他示意贝塔再说几句话。 
        贝塔又说了几句。 
        外星人凋旋钮。 
        “行了,可以对话了。”外星人说。声音是从小仪器里传出发来的。看来小仪器是一台翻译机。 
        “你们从哪儿来?”外星人同。 
        “地球。”贝塔回答。 
        “愿意在我们这里住住吗?”外星人问。 
        “愿意。”舒克说。 
        “这些房子都可以住,随便挑吧!”外星人说,“这个翻译机送给你们,再见。” 
        外星人走了。 
        “他们听说咱们从地球来一点儿也不吃惊。”贝塔感到有点遗憾。 
        “他们不因为咱们是外星人就抬高咱们的身价,和地球人不一样。”舒克说。 
        “走,咱们去那座房子看看。”贝塔指指飞船旁一座房子。 
      房子里没人,设备很好,房门上的一块牌子上写着:无人住。舒克把牌子翻过来,背面写着:有人住。 
        “咱们就住这儿吧?”舒克问贝塔。 
        “行。”贝塔同意。   \chapter{第98集} 
        双子星球上所有动物都有生存的权利; 
        舒克和贝塔望着窗外发呆; 
        返回太空   
        舒克和贝塔在小房子里定居下来,他们给小房子起名为宇宙公寓。 
        舒克和贝塔给他们的宇宙飞船搭了个棚子,以防风吹日晒。 
        宇宙公寓里有彩色电视机,电话,还有食物传送柜。你需要什么食物,只要按不同的按钮,食物就会从传送柜里送到你手中。 
        舒克洗了个痛快澡。贝塔顾不上洗澡,他被电视节目吸引了。 
        “这个星球叫双子星球,离地球远极了。”贝塔把他刚从电视上获得的信息告诉舒克。 
        “咱们吃完饭出去走走。”舒克穿着浴衣说。 
        “车库里有汽车。”贝塔说。 
        舒克和贝塔吃完饭,开车去外边转转。 
        双子星球给人以和缓的感觉。 
        双子星球人类的耳朵特别小,舒克和贝塔一打听,才知道双子星球上的人不在乎别人怎么说他,自己干自己的事,不理会别人对他的看法。时间长了,耳朵就退化了。 
        “地球上的人太重视别人对他的看法。”贝塔说。 
        “活得太累。”贝塔觉得身上最多余的器官就是耳朵。耳朵累人。 
        舒克想起自己为了争个好名声,抛弃了妈妈驾飞机出走的经历,他认定地球上的动物不如双子星球上的动物智商高。 
        双子星球上的所有动物都过着安适的生活,包括老鼠和苍蝇。双子星球认为所有来到这个星球上的动物都有平等的生存权利,苍蝇经过几代的消毒措施已经无病菌,老鼠有食物也不会去偷东西了。 
        舒克和贝塔喜欢双子星球,他们决定在这里永久定居,过舒心的生活。 
        他们结交了许多新朋友,有不少新经历。双子星球的食物是抗衰老的,该星球动物的平均寿命是八百岁。舒克和贝塔愉快地生活着。 
        30天过去了。 
        一天早晨,舒克起床后望着窗外发呆。 
        “你怎么了?”贝塔走到舒克身边。 
        “我想家了。”舒克说。 
        “我也是。”贝塔这几天心里感到怅然若失。 
        很怪,生活在这么好的地方,还要想地球。就因为地球是故乡。 
        “看来咱俩呆在哪儿也不行。”贝塔给自己和舒克下了个结论。 
        “说不定是不习惯,再住住看。”舒克确实不愿意再回去遭受歧视。 
        第二天,他们又站在窗前发呆。 
        他们这才知道还有比自尊更重要的东西。 
        居住在不属于自己的地方,生活得再好.也不会幸福。 
        “回吧?”舒克问贝塔, 
        “嗯。”贝塔点点头。 
        舒克和贝塔开始做返回地球的准备工作,他们拆除了宇宙飞船的棚子,往飞船里搬运食物。 
        发射飞船的时问定在下午两点整。 
        舒克将宇宙公寓门口的小牌子翻过来。 
        他们恋恋不舍地向双子星球告别。舒克和贝塔的朋友来为他们送行。 
        宇宙飞船点火了。 
        舒克和贝塔又回到了太空,他们驾驶飞船向地球驶去。   \chapter{第99集} 
        地球变了; 
        小老鼠的爷爷的爷爷的故事; 
        地球出了什么事   
        舒克和贝塔的宇宙飞船经过一段时间的航行,接近地球了。 
        “这回可别再被人类的宇宙飞船发现了。”贝塔提醒舒克。 
        “看咱们的运气了。”舒克说。 
        贝塔向太空了望。 
        宇宙飞船即将穿越大气层。 
        “注意!发现飞行物!”贝塔喊道。 
        舒克往外一看,一架飞行器正从右后侧接近宇宙飞船。 
        “摆脱它!”贝塔不想再忍受隆重欢迎。 
        舒克驾驶飞船变了个方向,那飞行器没有追过来。 
        贝塔松了口气。 
        “大概是卫星。”舒克说。 
        “谢天谢地。”贝塔说。 
        宇宙飞船穿过大气层。舒克和贝塔渐渐看见地球上的景物了。 
        “又回来了!”贝塔感到亲切。 
        舒克聚精会神地寻找着陆地点。 
        “地球好像变了!”舒克惊奇地发现。 
        贝塔往下看,的确,地球变了,树木变了,汽车的形状变了。 
        “才走了几十天,变化这么大?”贝塔不相信。 
        “下边好像是臭球的机场的位置,我着陆了。”舒克操纵宇宙飞船下降高度。 
        宇宙飞船平稳地下降。贝塔拿起话筒。 
        “臭球!臭球!我是贝塔,请回答!” 
        “……” 
        “臭球!臭球!我是贝塔,请回答!” 
        “……” 
        贝塔看看舒克,他预感到发生了什么事。 
        飞船降落在一片草丛里。 
        “这地方还挺熟悉,记得吗?”舒克下飞船后看看四周,对贝塔说。 
        “嗯,记得。”贝塔记得这里离他们的机场不远。 
        舒克和贝塔把宇宙飞船藏好,他俩朝机场走去。 
        哪里还有什么机场,只有空空的一片开阔地。 
        舒克和贝塔愣了。 
        “找错地方了?”舒克往好里想。 
        贝塔发现了地上的一个东西,他捡起来。 
        “皮皮鲁号运输机上的天线!”贝塔和舒克异口同声。 
        从前这里就是机场,可机场呢? 
        旁边的草丛里一阵颤动。 
        “谁?”舒克问。 
        “是我。”一只小老鼠钻出来。 
        “请问这里的机场呢?”舒克问。 
        “机场?”小老鼠纳闷。 
        “就是舒克贝塔航空公司。”贝塔迫不及待。 
        “航空公司?”小老鼠摸不着头脑。 
        舒克看看贝塔,贝塔看看舒克。 
        “你是刚来这里生活?路过?”舒克不信这一带的小动物不知道他的航空公司。 
        “我从一生下来就在这里。”小老鼠说。 
        “你从没见过或听过有许多小老鼠开飞机的事?”贝塔问。 
        “慢着,我想起来了,我爷爷给我讲过这事。他说原先这里有一座老鼠机场,不过那是古代的事情了。”小老鼠拍拍头。 
        “古代?”舒克和贝塔几乎蹦起来。 
        “我爷爷还是听我爷爷的爷爷说的呢!”小老鼠比划着。 
        “咱们才离开地球几十天呀!”舒克弄不清地球出了什么事儿。

\chapter{第100集 }

        三十年过去了; 
        臭球留下的信; 
        舒克和贝塔决定去看皮皮鲁   
        经过一阵惊愕,舒克和贝塔冷静下来。 
        “我从前好像听说过,天上一天,地上一年。咱们去的双子星球上的一天,说不定就是地球上的一年。”舒克说。 
        “准是这样。”贝塔断定。 
        在舒克和贝塔离开地球的这30多天中,地球度过了30年的漫长岁月。 
        “我带你们去见我爷爷,他知道得多一些,听说他还有传家宝呢。”小老鼠说。 
        “去看看。”舒克想多知道些关于机场的事。 
        小老鼠领着舒克和贝塔走进草丛,钻进一个地洞。 
        舒克和贝塔对地洞已经十分陌生了,贝塔还下意识地捂了一下鼻子,他受不了地洞里的潮气。 
        小老鼠的爷爷有气无力地躺着。舒克想起了自己的妈妈。 
        “我叫舒克,他叫贝塔。听说您知道关子机场的事,我们想听昕。”舒克说。 
        “舒克?!贝塔?!”老鼠爷爷激动地坐起来。 
        “你知道我们?”贝塔问。 
        “我的祖宗是臭球,他临去世前留下一封信,让后代转交给舒克和贝塔。这封信由我爷爷的爷爷的爷爷的爷爷交给我爷爷的爷爷的爷爷,又由我爷爷的……一直传到我手中。”老鼠爷爷说。 
        “我们就是舒克、贝塔,快把信给我们看看!”舒克急不可待。 
        “可你们……岁数比我还小呀!”老鼠爷爷不信。 
        “我们比您岁数大多了!”贝塔把去太空的事讲给他听。 
        老鼠爷爷从地底下挖出一个瓶子,从瓶子里拿出一封信。 
        舒克用颤抖的手打开这封30年前臭球写给他和贝塔的信。 
        亲爱的舒克和贝塔: 
        自从你们去太空后,机场一直很兴旺。 
        可就在一个月前,灾难从天而降。在一个夜晚,正当我们休息的时候,一支灭鼠队扫荡了机场。幸存者只有我。我对不起你们,把你们交给我的机场毁了,你们骂我吧! 
        还有一辆摩托车是完好的,留给你们,由我的儿子转交。祝你们好运。 
        臭球临终前 
        舒克把信递给贝塔,他看着墙,发呆。 
        贝塔看完信,一拳砸在地上。 
        他们的跟前出现了灭鼠队围歼机场上的老鼠的情最。飞机被毁坏,建筑被踏平。舒克想到妈妈。贝塔想到罗丘、荷叶、松果、头版…… 
        “摩托车在哪儿?”舒克问老鼠爷爷。 
        “是那箱东西吗?在外边的地下埋着。”老鼠爷爷说,“我带你们去。” 
        舒克和贝塔扶起臭球的后代,一同去找臭球留下的遗物。 
        埋藏在地下的箱子被挖出来了。 
        舒克和贝塔用工具打开箱子,一辆用塑料布包着的摩托车呈现在他们眼前。这是当年机场惟一的遗物。 
        贝塔掉泪了。 
        “咱们去看看皮皮鲁。”舒克拍拍贝塔的肩膀,强打精神地说。 
        “皮皮鲁有40多岁了吧?”贝塔擦干眼泪说。 
        “差不多。”舒克说。 
        舒克检查一遍摩托车.保存得很好。 
        “我记得你在机场开过摩托车,你发动一下试试。”舒克想骑摩托车去找皮皮鲁。 
        贝塔发动30年前的摩托车一次成功。 
        摩托车的排气管“突、突、突”地冒着热气,把积蓄了30年的寂寞一吐为快。 
        “你带着我,咱们去找皮皮鲁。”舒克看看天色,已近黄昏。 
        贝塔的性格好像突然之间变了,他沉默地跨上摩托车。舒克坐在后座上。 
        “再见,谢谢你们!”舒克深情地看看臭球的后代。 
        “再见!”臭球的后代终于完成了祖宗留下的遗愿。 
        贝塔驾驶着摩托车上路了。 
        夜色渐渐笼罩丁大地,月亮将它的光吝啬地投向原野。 
         舒克和贝塔辨认着道路。从前,他们是从空中去皮皮鲁家的。 
        “看,我原先居住的房子!”贝塔认出了他出生的地方,他和咪丽打仗的地方。贝塔想起他开着坦克出走的情景。 
        “停车去看看?”舒克建议。 
        “不。”贝塔继续开车,他也不说为什么。 
        摩托车直驶皮皮鲁家。 
        舒克和贝塔觉得自己长大了。一路上,他们想了许多,也什么都没想。 
        摩托车进入了城市,街道上行人稀少。 
        “看,钟楼!记得咱们为皮皮鲁拨表吗?”贝塔说。 
        “当然!皮皮鲁家在钟楼的东南方向。”舒克说。 
        摩托车停在皮皮鲁家的单元门口。舒克和贝塔攀着排水管爬上皮皮鲁家的阳台。 
        纱窗上的小窟窿还在! 
        “皮皮鲁还在这儿住!”舒克断定。 
        舒克和贝塔钻进屋里,他俩看见沙发上坐着一个中年男人,正在看电视。 
        “是皮皮鲁吗?”贝塔小声问舒克。 
        “不像。”舒克说。 
        “神态有刖有点儿像。”贝塔端详着。 
        “做好跑的准备,我冒一次险。”舒克说。 
        贝塔把退路看好。 
        舒克小心翼翼走到中年男子的脚下,他拽拽他的裤脚。 
        中年男子低头一看,愣了一下,好像是从大脑的记忆细胞里搜寻着什么。 
        见到老鼠而不吃惊,舒克断定他就是皮皮鲁! 
        “我是舒克!”舒克大叫。 
        “舒克?”中年男子一跃而起,好像变成了小孩子。 
        “你是皮皮鲁?”舒克问。 
        “正是!我是皮皮鲁!你还活着?都几十年啦!!!”皮皮鲁又激动又觉得不可思议。 
        贝塔从沙发后面跑过来。 
        ‘贝塔!!!”皮皮鲁大喊。 
        皮皮鲁把舒克和贝塔捧在手掌上,他看着自己童年时的朋友,眼睛湿润了。 
        舒克和贝塔把分别后的经历讲给皮皮鲁听。皮皮鲁也把自己这30年来走过的路告诉朋友。皮皮鲁现在是著名的物理学家,去年曾获得了诺贝尔奖的提名。 
        舒克和贝塔高兴极了,这是他们回地球后惟一高兴的事。 
        “祝贺你!”舒克和贝塔异口同声。 
        “谢谢。”皮皮鲁脸上有一丝愁云。 
        “你有不顺心的事?”贝塔问。 
        皮皮鲁点点头。 
        “在我的童年,整天就是上学,写作业,考试,根本没时间玩。幸亏认识了你们.要不然,我就没有童年了。”皮皮鲁说。 
        “可你换来了今天呀!你成名了呀!!”舒克提醒皮皮鲁。 
        “长大成就再辉煌,没有童年的人生也是不完整的人生。”皮皮鲁叹了口气。 
        舒克和贝塔不吭声了。 
        “你们看。”皮皮鲁指指书柜。 
        书柜里停放着一架绿色的直升机和一辆米黄色的坦克。 
        “这两样东西,我在书柜里放了30年。”皮皮鲁说,“现在送给你们。” 
        舒克和贝塔的心脏有力地跳动着,他们喜欢生命,尽管生命本身就是痛苦和欢乐的结合体,他们还是喜欢。 
        舒克和贝塔驾驶着直升机和坦克在屋里给皮皮鲁作飞行和行车表演。他们想补上皮皮鲁的童年,使他有完整的人生。 
        第二天,舒克和贝塔向皮皮鲁告别。他们要去海上的孤岛看看海盗。他们想,也许孤岛上的一天也等于陆地上的一年呢。反正他们非常想见从前同他们打过交道的人,不管是朋友还是敌人。 
        何况随着时间的推移,世界上没有敌人。 
        舒克驾驶直升机吊着贝塔的坦克,飞出了皮皮鲁家。 
        等待他们的,是新的一天。   

\backmatter
      
\end{document}
