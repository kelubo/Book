% 活着
% 活着.tex

\documentclass[12pt,UTF8]{ctexbook}

% 设置纸张信息。
% 纸张设置配置文件
% 用于定义书籍的页面尺寸和边距

\usepackage[a4paper,twoside]{geometry}
\geometry{
	left=25mm,
	right=20mm,
	top=25mm,
	bottom=25.4mm,
	headsep=1cm, 
    footskip=1cm,
	bindingoffset=10mm
}

% 设置字体,并解决显示难检字问题。
\xeCJKsetup{AutoFallBack=true}
\setCJKmainfont{SimSun}[BoldFont=SimHei, ItalicFont=KaiTi, FallBack=SimSun-ExtB]

% 目录 chapter 级别加点(.)。
\usepackage{titletoc}
\titlecontents{chapter}[0pt]{\vspace{3mm}\bf\addvspace{2pt}\filright}{\contentspush{\thecontentslabel\hspace{0.8em}}}{}{\titlerule*[8pt]{.}\contentspage}

% 设置 part 和 chapter 标题格式。
\ctexset{
	part/name= {第,卷},
	part/number={\chinese{part}},
	chapter/name={第,篇},
	chapter/number={\chinese{chapter}}
}

% 图片相关设置。
\usepackage{graphicx}
\graphicspath{{Images/}}

% 设置署名格式。
\newenvironment{shuming}{\hfill\zihao{4}}

% 注脚每页重新编号,避免编号过大。
\usepackage[perpage]{footmisc}

\title{\heiti\zihao{0} 私密处的奇幻旅程 打破所有女孩对身体的错误迷思}
\author{妮娜‧布罗克 艾伦‧斯托肯‧达尔}
\date{}

\begin{document}

\maketitle
\tableofcontents

\frontmatter

\chapter{前言}

我们在二〇一五的新年开始营运部落格《生殖器二三事》(Underlivet)。我们无法确定媒体对性健康、女性身体与性的胡诌程度有多少。无论好坏,在性方面我们比以往接触得更多。对儿童与青少年早期来说,性知识透过网路便唾手可得。若思考身体是否出状况的话,寻求Google医生的协助确实容易。但人们不是应该在学校的性教育课程中就该有所了解才对吗?

我们对于如何整合资源也无法给予明确的方案。是另外再开一个性专栏吗?还是再请一组天真的医学院学生告诉每个人是否健康、正常呢?

部落格成立的当周,我们因来访人次超过七百人而互相打电话给彼此欢呼。大多数的访客大概都是朋友与家人。而两年多后的现在,我们可以确信「那时候」真的有问题需要我们解答。无论认识与否,我们收到来自读者无数可爱的回复,现在文章已经有超过一百四十万的点阅数。

原本为青少年所设的部落格最后演变将读者群向外扩展。每一天,来自不同性别、年龄的读者疑问涌入部落格。我们也时常收到应该在中学课程就学到的基本问题。有时候,读者们最需要的是安慰他们身上出现的状况是「正常」的、本来的样貌没有什么不好。遗憾的是,这些族群大多皆为女性。

这本书就是为妳而写,写给每位不确定自己的行为、模样、感受是否正确的女性。希望这本书能够提供妳所需要的信心。也写给替自己感到开心与骄傲,同时想学习更多存在双腿之间美好器官之知识的妳。身体的私密部分非常奇妙,我们也相信良好性健康的关键在于了解身体是如何运作。

二〇一六年的秋天,报纸头条向大众揭示挪威高中的同侪过度性化事件。融入耍帅的残酷社交压力意即让十六岁的少女被迫跨越自身的性别界线⸺这些难以置信的案件令人作呕。去试想十八岁少年认为藉由社交地位让新生少女连续替十位男生口交没有什么不对的想法确实可怕。挪威VG报纸当时写道,现今已成为「你情我愿的性行为与性暴力界线濒临模糊危机」的文化。近年来,我们发现青少年文化的性化程度激增,尤其在少女身上更容易被发现,与甫成年人的安稳环境来比相去甚远。不幸的是,对多数人来说,让余生痛苦的不愉快性经验会伴随他们成长。不该是这样的。

当女人替自己的身体与性别做决定时,她们同时也在更大的环境下抉择。无论是避孕、堕胎、性别认同或性实践,文化、宗教和政治的力量控制了这些选择。

我们想让女性在公开的事实中独立抉择,让她们根据医学知识而非八卦、误解与恐惧做决定。奠定身体运作的良好知识基础将使女性更容易在自信、安心的状态下做选择。性必须浅显易懂,我们必须成为自己身体的主人。我们希望能提供机会帮助妳做出「适合自己」合理、受益的决定。

妳现在有可能坐着猜想:「为何我要看一本两个学生所写的医学书?」她们还没从大学毕业呢!同样的问题我们也问过自己好几遍。我们既不是受过完善训练的医生也非任何领域的专家,所以更怀着一定程度的敬意完成这本书。

我们从德国医学生茱莉亚‧恩德斯(Giulia Enders)的例子获得勇气。她的著作《肠保魅力》(Darm mit Charm),获得成功回响,使肠子与粪便成为黄金时段的谈话节目供人们讨论的议题。我们书名的押韵就是对她的致敬。她开创先河,让我们知道医学可以变得好懂、有趣,特别的是,我们谈论身体最私密部分也不会感到一丝尴尬。

因为身为医学院学生,我们拥有别人没有的优点:我们充满好奇、年轻又有勇气问一些「蠢」问题⸺经常是因为我们,或是朋友们,对于彼此都拥有的构造感到兴趣。没有危及职业名声的风险,也还没像一般医生一样,忘记直接了当,反而耗费许久时间向病患说明。希望有更多的年轻同仁付诸行动参与创作。

在写书的时候,好几次发现一些我们完全误解的观念。我们也一样会陷入女性性器官的迷思,而且拥有相同想法的人为数众多。存在最久的大概就属处女膜迷思,让女性在整个世界下持续受害,甚至挪威也有一样的情况。还有少数的医生搞不清楚身体的这个微小部位。也有医生提供患者替女性检查处女膜的服务,帮助她们延续迷思。在寻找解答时,我们发现资深的妇产科医师会以枯燥或避重就轻的回复闪避问题。处女膜掌握女人一生是件令人难以置信的事。透过本书,我们已经尽最大的所能破解处女膜的迷思。

另一个迷思是荷尔蒙避孕法既不健康又危险,结果导致上千名选择不安全避孕方式的少女意外怀孕。我们明白人们对副作用感到茫然与害怕,也受够医疗单位没给予适当的解释消除大众的疑虑。因此我们决定用一大篇幅来探讨避孕。反复阅读重要文献记载可能的副作用如情绪不稳及性欲降低后,为了让大家放心,我们在不确定的状态下会保持开放的立场。引发严重副作用案例相当罕见,且没有什么迹象表明情绪低落或性欲下降是影响大多数女性使用荷尔蒙避孕药的问题。凡事是总有例外,我们希望读完此书后,妳能够分辨正常与否。

其他迷思没有直接的伤害,却也反应男性主宰医疗研究领域行之有年的事实。当朋友抱怨从未有过「阴道高潮」时,代表我们对女性性行为的认知程度已被男性需求扭曲了好几个世纪。从来没有所谓的阴道高潮,只有高潮的形式不同,而每个方法都一样愉悦。我们希望女性能够停止自卑,因为她们需要其他方式的刺激,并非只是直接进入阴道的性行为。

上述探讨的主题为《私密处的奇幻旅程》众多内容里的一部分。从外阴到卵巢,期待妳一起加入探索女性生殖器官的旅程。如同写书过程时的我们,希望妳从中获益良多。对我们来说最重要的是妳看完这本书后能够得到放松。身体非常的单纯,每个人在生命中都拥有带来喜悦与挑战的身躯。我们要替身体的成就感到骄傲,在它奋斗的时候保持耐心。

\mainmatter

生殖器官

我们的生殖部位大概是身体最私密的地方了。打从母亲产道出生初见曙光的刹那,它就和我们成为最亲密的伙伴。在护校里我们喜欢称之为尿尿地方的里面与外面。随着青春期的开始,第一个展现的特征为胯下的毛发。无论充满骄傲或恐惧,每个人都不会忘记初经来潮。妳或许开始自慰并发现自己能够在蜷曲的身体中获得快感。紧接而来的,是初次性经验造成的伤害、好奇与欲望。也许妳有或曾经有过孩子,也体验过性高潮带来的巨大变化,以及其所伴随的各种惊奇展现。生殖器官是身体的一部分,是时候了解更多了。






阴户⸺私密部位的奇幻旅程





全裸站在镜子前面好好看着自己。妳的生殖部位处于肚子下方,包覆在髋骨最前方的脂肪区块。这片柔软的地带称为维纳斯丘,青春期的时候毛发会开始覆盖在上面。部分女性在维纳斯丘的脂肪垫会比他人要来得大,所以有些人的肚子接近阴部之区域会略微突出,反之另一群人会相对平坦。两者都是极为正常的。







从维纳斯丘往下看的话,妳会来到所谓的阴户(Vulva),或称之为妹妹、饼干罐、阴沟、阴道、屄等等。我们挪威人会叫它小老鼠。阴户可能不是世界上最常用的词汇,然而身为女人,妳往两腿中间一瞧,妳看见的东西就是阴户。

很多人认为女性性器官的可见之处叫作阴道。妳可能会说,「我的阴道上长了毛发」,或是「妳的阴道怎么那么美」,其实都是不正确的说法。阴道没有任何毛发,即使非常漂亮,也不容易被看到。阴道只是性器官的部分称呼,更准确来说,是当妳进行插入式性行为或生产时所使用的肌肉质管⸺换言之,就是通往子宫的管道。我们如此讲究术语是因为,无论我们从中得到多少欢愉,性器官不单只有阴道而已。大多数人偏好将女性生殖器用阴道来表示阴户,所以接下来我们将从这里开始介绍奇妙的女性性器官。

阴户的形状像一朵花,附有两层花瓣。信不信由妳,花的比喻并不是由我们发想。看着阴户的各个部位,我们认为从外而内开始介绍会比较恰当。







花瓣或阴唇(Labia,嘴唇的拉丁文)的目的在于保护更深处的敏感部位。比里层更加厚实的阴唇外部,充满着脂肪并以气囊、避震器的形式运作。外部阴唇或许长到足以覆盖内层,却也有狭窄的类型,有些人甚至只有围住其他阴唇部位的两片小凹坑而已。

大阴唇由正常皮肤覆盖。和其他身体部位的皮肤一样,布满皮脂腺、汗腺和毛囊。除了毛发这个好东西之外,大阴唇上也可能长出斑点与湿疹等坏东西。必须遗憾地说,这就是皮肤。并非绝对,不过小阴唇通常比大阴唇长,上面充满皱褶,像是公主的纱裙。当妳站在镜前端详,小阴唇可能会较为突出,也有可能要拨开大阴唇才能看见小阴唇。

相较充满脂肪的大阴唇,小阴唇更薄又高度敏感。敏感程度虽不像身体最敏感的部位阴蒂一样,却又交集了许多神经末端,所以碰触时会非常有感觉。

小阴唇没有一般皮肤的构造,而是由黏膜包覆⸺妳一定看过黏膜,例如脸上的眼球与嘴唇内部。亦即小阴唇充满保护及滋润的黏液。一般皮肤覆盖在角质层之下,角质层提供保护,让正常皮肤在干燥状况下得以成长;然而,黏膜没有角质层保护,因此不太能承受磨损与撕裂。举例来说,小阴唇长会因为与紧身裤摩擦感到疼痛。与一般皮肤不同的是,黏膜更加湿润。黏膜上没有毛发,也代表小阴唇里面没有毛发。

如果妳拨开小阴唇的话,会看到阴道前庭(Vestibulum)。阴道前庭来自于拉丁文,意即前庭,或位于建筑物入口与内部之间的区域。妳若是会去戏院、剧院的类型,前庭就是妳在中场休息时吃蛋糕、喝香槟的地方。是个有着柱子和柔软红丝绒幕帘的华丽入口大厅。女性阴道前庭没有任何的柱子,尽管如此却仍是个入口,我们还是认为它无瑕壮丽。接着妳会发现两个洞口:尿道及阴道口。尿道口位于阴唇与正前方的阴蒂的交会处,阴道则与肛门相邻。

即使我们每天使用好几次,还是有很多人对尿道口没有概念。事实上,有些人以为尿液没有单独的排出孔,与男性相同,一个孔有两种用途:精液与尿液。这是错误的认知:尿道有专属的洞口。我们不用阴道排尿,即使看过一大堆女性生殖器,却也很容易误解。就算从镜子看过去,尿道口仍难以被看见。尿道非常小,洞口周围有许多皮肤小皱褶,但只要试着寻找就会发现。







阴道⸺惊奇的扩张管道





与细小的尿道孔不同的是,我们很容易看到更大的阴道口。阴道是从阴部到子宫,长度为7〜10公分狭窄的肌肉质管。大多时候质管为扁平的状态,所以前后壁相互挤压。这有助于防水,想像一下!

当妳激起性欲时,阴道口会纵、横向扩展,在各方面极富弹性。阴道有点像打褶的裙子。用手指触摸的话,妳会感觉到皱褶。

阴道周围的肌肉非常强韧,当妳将一根手指放进阴道后它会突然紧缩。和其他肌肉一样,这些骨盆腔底的肌肉,透过锻炼会愈来愈强健。

阴道壁里面充满湿润的黏膜。大多黏液不是由腺体分泌,而是从身体内部渗出至阴道壁。阴道壁没有任何腺体,但有些分泌液来自于子宫颈。阴道一直保持着湿润的状态,在性欲高涨时会比以往更加潮湿。更多血液流至整个生殖部位,也就有更多液体从阴道壁渗出。阴蒂和小阴唇开始充血,所以妳会注意到有增量的血液流往生殖器官。黏液在性奋时产生,让妳在自慰或与他人进行性行为时降低阴道受到的摩擦。通常性交时阴道壁有如被连续击打一样,摩擦减少表示对阴道壁的伤害更少了。性行为结束后阴道壁因为些微撕裂而流血是正常的状况,妳也会感到有些疼痛。幸运的是,这不会造成伤害。阴道壁善于自我修复。

除了阴道壁分泌的液体,有些黏液来自前庭的两个腺体。它们位于阴道口两侧接近屁股的位置,称为巴多林氏腺(Bartholin's glands),以丹麦解剖学家卡斯柏‧巴多林(Casper Bartholin)为名。它们分泌出黏液帮助润滑阴道口。椭圆形的巴多林氏腺体积小若豌豆,却可能带来麻烦。输送黏液的小管堵塞的话会形成阴门囊肿,从阴户侧边便能感受到小型的硬块,彷彿像一颗小气球。一旦这类的囊肿感染,就会转变为麻烦的状况,不过透过微型手术即可解决问题。有些人对巴多林氏腺之于阴道润滑的重要性却持反对立场,因为囊肿感染而切除腺体的女性在性欲高涨时还是能感受到阴道润滑。

在阴道壁前端,换句话说就是靠近膀胱的位置,存在女性杂志性专栏里热门讨论的地方。我们讲的正是所谓的G点(G-spot)。以发现者,德国妇产科医生,恩斯特‧格雷芬贝格(Ernst Gräfenberg)之名命名。从一九四〇年代起,科学家不断探讨和研究G点,但争议层出不穷。科学家不确定它到底为何,也无从证实它存在与否。

在部分女性阴道中G点特别敏感,某些女性表示藉由刺激便能达到高潮。G点差不多位于阴道壁前端,也就是靠近胃部,且可以用「快过来啊」的挑逗手势来刺激它。幻想一下迪士尼女巫试着诱惑妳靠近她,就是那个姿势。根据几名女性的叙述,比起阴道任何地方,G点的刺激会有更好或是不同的感受。妳可能会注意到,相较阴户甚至是阴蒂,阴道本身不是特别敏感。敏感是阴道口最重要的关键,同时缓冲更进一步的动作。

媒体通常将G点看作分开的部位。妳若阅读过性专栏或性方面的书籍,一定对此印象深刻。二〇一二年英国的一篇综述文章在现有主张G点与阴道为不同部位的文献里得到论点薄弱的结论。大多G点的研究都是透过女性叙述的问卷调查而来。文章中还提到许多相信G点存在的女性不能指出确切位置。研究者同时表示,以成像技术为据的论文无法找到比阴蒂更能让女性得到高潮或快感的部位。

事实上,其中一个假设为G点并非分离的构造,却处于阴蒂最深处,在性交时才会受到刺激,直接经由阴道壁而来。二〇一〇年,一群学者发表了一项研究,他们观察女性进行阴道性行为时阴道壁前端的变化。透过超音波探测发生的过程与找寻G点的位置。他们没有找到,但认定阴蒂内部非常靠近阴道壁前端,所以阴蒂就是G点谜团的解答。

另个可能为G点与一群阴道壁前端的腺体相连,也就是广为人知的史基恩氏腺(Skene's glands),相当于女性的摄护腺,围绕男性部分尿道的腺体。史基恩氏腺结合女性射精或是潮射。有些研究表示G点是达到潮射的重要部位,然而现在都只是理论。我们已经知道部分女性会有潮射的状态,却还不清楚G点是否存在。

奇怪的是,和阴道壁一样的地方居然笼罩在神秘当中,尤其又那么多对于G点的谬误。我们屏息以待更多有关女性身体的优质研究。






阴蒂⸺一座冰山





当我们写到阴蒂内部时妳应该很讶异。内部的哪个地方?毕竟我们常常形容阴蒂的体积就像葡萄干,位于阴户的最顶端,精准对齐小阴唇交集的所在。但这个小纽扣却只是冰山的一角。藏在骨盆区最深处的这个器官,超越妳所有狂野的想像。

即使从十九世纪开始,解剖学家已知阴蒂是巨大的地下器官1 ,却和普遍知识大相径庭。解剖学课本里将男性阴茎描述得非常详尽,而阴蒂仍留下好奇的空间。一九四八年末,格雷氏解剖学决定不标示阴蒂的项目,连男性主宰的医学界也对阴蒂更进一步的研究没有兴趣。阴蒂确切的构造及如何运作还存在相左的看法。以医学的角度来看,实在令人震惊。







我们所知道的是,人们大多将阴蒂形容为骨盆所延伸的大型器官中的一小部分,且向下延伸至阴户两侧。透过X光检测的话,我们能够看到阴蒂整体形状像是倒Y字型。称为阴蒂头的小葡萄干,就在正上方。阴蒂长0.5〜3.5公分不等,由于被阴蒂包皮遮住一小部分的关系,所以看起来会更小。阴蒂头是唯一肉眼看得见的地方,而其下方为阴蒂体,向下延展与身体形成一个角度,模样近似回力镖,前面有一对阴蒂脚,被阴唇两侧包覆在下。







两脚皆有勃起组织阴蒂海绵体(corpus cavernosum),性欲激起时开始充血。两脚中间有额外的勃起组织前庭球(bulbi vestibuli),围绕于阴道及尿道口。

上生物课特别认真的妳,会对刚才的叙述有点印象⸺不过男性阴茎不是也有阴茎头、脚和勃起组织吗?女性高潮的主要来源为阴蒂,这是不为人知的秘密,至少显而易见的是,它和勃起的阴茎截然不同。或许会让人感到惊讶,阴蒂与阴茎为相同器官的两种版本。

男女性生殖道胚胎在子宫内约十二周时,长得完全相同,由一种像迷你阴茎(或是巨大阴蒂!)的形体主宰一切,称为生殖结节(genital tubercle)。它具有女性或是男性性器官的潜在发展。阴茎和阴蒂从相同的构造成长,因此两个器官有许多相似的形式与功能。

阴茎头和阴蒂头其实一样,所以两者都被赋予相同的名字,分泌腺(glans)。在两性身体中都是最敏感的地带。据统计,男女性腺头拥有超过八千多条感觉神经末梢。感觉神经末梢接受压力与触摸的资讯,传送大脑讯号,转换为疼痛或愉悦的感受。有愈多的神经末梢,大脑就接受愈多不同有力的讯号。然而,阴蒂头比阴茎头更加敏感,因为末梢神经集中在更小的部位:没错,集中度多了五十倍。

不幸的是,阴蒂为性欢愉开关的认知,让一些男性相信向它施予压力是正确不过的。若一点的施压没有满足欲望,他们只会使更大的力气。这不是阴蒂运作的方式。由于富含了许多神经末梢,即使最细微的变化它都能感觉得到。它提供了意想不到的刺激与快感,却也代表疼痛或完全麻痹的过度期极短。长期下来,过度加压造成神经末梢拒绝传送讯号到大脑:阴蒂按钮已切换至「静音」模式。一旦发生,直到准备好再次说话时阴蒂会保持平静。就好比搭讪一样:妳做得太过头,事情通常会搞砸。

男性勃起组织让阴茎开始充血变硬。更不用说女性的勃起组织也有一样的功能。当性欲被唤起时,整个阴蒂肿胀至原本的两倍大,完全是令人肃然起敬的勃起。因为阴蒂脚与前庭球在阴唇下及尿道、阴道口周围,让阴户在性奋时看起来更大。此外,阴道前庭与小阴唇由于血液聚集的关系呈现更深、紫红的颜色。

相似处还不止这样。男人最爱吹嘘的早晨「一柱擎天」与夜间勃起,我们女性也有。一九七〇年代佛罗里达大学展开两名阴蒂较大的女性与男性间不同的研究。研究发现女性夜间「勃起」的次数与熟睡男性相当。另一项研究指出女性一夜「勃起」可以达到8次,合计时间为80分钟!

将所有的资讯结合起来,妳会发现在生物课并没有学到很多关于阴蒂的内容。这个让人骄傲的器官一直被忽视、低估、遗忘已久。只有当我们理解阴蒂是如何延伸至整个骨盆腔时才会感受到自己拥有这个惊奇部位的喜悦。







* * *





1	解剖学家科贝尔特(Kobelt)于一八四〇年代描述阴蒂的内结构,并论定男女性的性器官架构相同。






落红贞洁





数千年来,不同的文化(包括挪威)极为在乎童贞,不是男性,而是女性的贞操。男性没有圣父或妓男、纯洁或污秽之分,但女性就有,「幸运」的是,从新婚之夜阴道流血便能分辨她是哪一类的女性。

有许多人使用这个说辞:「破她处。」彷彿没有性经验的女性就可以像开香槟那样被破处,就好像阴道在初次性行为前后的不同,和酩悦香槟有没有软木塞一样。妳应该感受到我们的语气,破处根本不是重点。

童贞的概念在主流文化中广为流传。对《嗜血真爱》(True Blood)里的吸血鬼洁西卡而言,每次的性行为都是第一次,她每一次都必须要流血。同样的疑团绕着《权力游戏》(Game of Thrones)的皇后玛格丽‧提利尔(Margaery Tyrell)打转。嫁给第三号国王后真的依然纯洁吗?

经典作品里也有提到童贞与落红。「该死!」我们在挪威文学课里所放的电影里看到克莉丝丁‧拉兰斯达特(Kristin Lavransdatter)的鲜血流到大腿时,可能曾经如此咒骂。不过她却说了这段话:「谁会想要被摘过的花呢?」她在情人艾伦(Erlend)的双臂里嚎啕大哭,而他完全不需要落下男儿泪。身为男人,艾伦根本没有操守好失去。

女性为无邪花朵,甚至取走她的贞洁等同摘下花朵的概念出现在医学用语。女性第一次发生性行为而流血的情况称为失去贞洁,整个事态实在是无法形容地古板。这好似来自不同文化与不同历史年代的男性聚在一起找出控制和限制女性性取向及替身体做决定的方法。







如果妳已经搞清楚上述内容,是时候讨论处女膜(hymen)了,这个在阴道口裡的神秘构造仍然让全世界女性付出名誉或使她们活在陈旧传统与误解当中。不敢置信的是至今仍以此分化两性。美好正面的性爱只毁掉女性却没对男性造成任何影响。妳想想看,在处女膜与流血之上,有迷思主宰一切,这整件事实在太愚蠢 。

自古以来处女膜代表贞操的象徵,如同迷思所叙述的,它会在女性第一次性交时破掉且流血,只有这个时候。流血被用来当作贞洁的证明⸺过去人们在新婚之夜后按例将染血的床单拿到外面挂起,让左邻右舍看到凡事都是如此进行。

处女膜的迷思为:如果妳性交后流血,人们会知道妳以前没有过性经验。如果没有流血就表示妳已经发生过性关系。然而和其他迷思一样,完全错误。

这个迷思的信念持续存在的原因在于对处女膜为一片薄膜的认知广泛流传。当妳听到「薄膜」一字时,妳大概想像一片干净的塑胶膜在妳戳一个洞后就破裂了。啵!但妳曾经用镜子看生殖器的话,妳会知道根本没有一张保鲜膜在妳的阴道上,就算妳没有发生过性关系。别让一个迷思被其他迷思给取代。最近我们听过许多关于「处女膜不存在」的言论。没有封住阴道的薄膜是正确的观念,但造成误解的形体仍然存在。

阴道口裡有个环形摺状的黏膜倚靠著阴道壁,像一枚戒指。这小戒指以前曾有阴道薄膜、贞洁等称呼。我们叫作处女膜。虽然这些名字都代表相同的意思,但阴道薄膜一词容易让人误解,最好不要使用。

所有女性生来便具有处女膜,然而对妳来说却没有任何用途。处女膜等同男性的乳头。从我们是胚胎时就是个没有功能又多余的部位。

处女膜具有厚度与宽度。换言之,它不是薄得跟塑胶膜一样,而是厚又坚固。它的外表在少女青春期时光滑,像中间有洞的甜甜圈。接着身体的荷尔蒙交响乐团登台,处女膜和其他身体部位相同,开始随之产生变化。在青春期结束后,通常会变为新月形。它的后端较宽,朝向肛门,位置依旧围绕在阴道壁,而中间洞口变得更大了。至少这是理论上的模样。事实上,处女膜并没有一定的形状。

多数女性的处女膜为中间开洞的圆形,然而不是每个人的都那么平滑。它通常既皱又有凹痕,也不是性行为的代表。有些人的处女膜上有延伸至阴道口的线状物,所以看起来较像「ø」而非「o」字型。其他人的就像筛子,中间没有一个大洞,而是有许多小孔在上面。又有一群人的处女膜看起来像小流苏一样长在阴道壁,而也有少数女孩的处女膜确实覆盖整个阴道口,她们的处女膜相当坚固,这是变异所造成的问题,因为经血一定要有地方排出才对!具有此类处女膜的女性往往在初经来临时才发现问题。若经血困在阴道裡,会造成剧痛且有可能需要进行手术。像密封一样,这个少见的变异型态是我们最接近处女膜迷思的地方。

无论何种样貌处女膜都具有弹性,除了少数包覆整个阴道口的例子。即使如此,处女膜仍是阴道最狭窄的地方。阴道具有惊人的延展、收缩能力:毕竟妳能够让婴儿从这里出来。所以处女膜应该也要能够延伸。虽然它有弹性,却不足以在性交时派上用场。有点像是把一条橡皮筋拉得很长,一旦太用力就会应声断开。

当妳进行初次阴道性交时,处女膜往剩余的阴道空间延伸。许多女性的处女膜有弹性到足以应付一切,但对其他人来说,处女膜会撕裂和流一点血。换句话说,有些女性第一次发生关系会流血,有些则不会,完全取决于处女膜的弹性。拥有横跨阴道口呈现ø字特别形状处女膜的女性,为了让阴茎或手指有进入的空间而经常觉得此处有撕裂感。

很难确认有多少女性的处女膜在第一次性交后流血。有几项研究记录了统计结果,但数值存在了变数。我们在两项不同研究中分别看到56%及40%的女性,在初次进行双方同意的阴道性行为后流血。虽然不是全体女性,比例却是极高。

这些研究访问女性关于第一次性交的经验。我们不能完全确定是处女膜流血,即便它是阴道最狭窄的地方,亦或血液是来自别处。在阴道的部分里我们有提到,如果是略为激烈的性行为、阴道不够湿润或紧张所造成的内部肌肉紧绷,阴道壁裡出现小裂痕而出血是正常、合理的。人们第一次发生关系或其他情况下都有可能发生。

另一个处女膜严重的迷思与处女检测有关。这些试验代表人们相信透过目视女性生殖器即可辨别她是否发生过性行为。圣母玛丽亚很明显地经历过处女测试,圣女贞德和一大堆在现代不同保守环境下的女性也都有相同的遭遇。

有时我们听闻挪威医生仍然受家属所托对年轻女性进行处女检测,证明自己女儿的贞操完好如初⸺儘管法医专家认为这些测试无关紧要。我们也听说有医生开立处女证明给担心新婚初夜没有流血而惊恐的女性。

然而结果为,通常不可能透过处女膜变化加以辨别女生有无性经验。这使整个处女检测变得荒谬。虽然处女膜可能在性交时因激烈延展而受损,造成的伤害却不一定永久。在许多案例上可以发现,处女膜能够不留任何伤痕复原。

许多处女膜与其变化方式的研究在一项女性初次经验来自性虐待的调查后受到改变。一篇挪威的综述文章指出现有影响幼童处女膜的因素(例如洞口宽大或边缘狭窄),已经认定为没有特殊发现也并非性虐待的证明。这些处女膜的变化可以从没有遭受性虐待的孩童上发现。附带一提,文章的作者还小心翼翼地表示缺乏相关文献并无法证明孩童没有曝露于性虐待的遭遇之下。

基本上,妳没办法从女性两腿之间看出她是否有无性经验。处女膜不是为了保护那些没有性经验的人、又或是发生过性行为的人、还是「处女」的人。与身体的其他部位一样,处女膜外观的不同是因人而异。抱歉,处女检测根本没用。

不幸的是,这个常识鲜为人知。女性仍求助手术确保她们能在新婚之夜流血⸺叫作处女膜整形术。直到二〇〇六年,挪威奥斯陆的沃尔沃特私人诊所仍提供这项手术,在寻求医学伦理的咨询后他们停止了处女膜整形术。议会反对这项手术,因为这会使贞洁问题有了快速修补或替代的解决方案:也就是文化改变。

处女膜整形术依然存在。妳能在网路上以30美元购入包含特效血浆的假黏膜,保证让妳「和黑历史说再见」,放心去结婚。顺带一提,埃及政客在二〇〇九年建议禁止进口此产品。

为什么我们选择寻求这些方法,而非告诉人们没有落红不代表没有贞操呢?又为何对我们来说,女性直到婚前保有「完好如初」的证明是如此重要?落红必须变得无意义,处女检测必须彻底废除,最重要的,我们必须屏弃贞操本身的重要性。

问题是找到处女膜可信的资讯很困难⸺尤其,分辨是非与否更难。对于处女膜的认知,我们找到的极少资讯对大多数人而言不易理解、无法取得,也不正确。即便发现了优秀的研究文献,但是医学院最常使用的妇产科课本对处女膜的叙述少之又少,有些迷思也不断反复出现。我们还存有超多疑问。更糟的是,获得的极少数资讯无法满足需要它的人。我们每个人为此都抱有神圣的任务,只是要不要开始的问题而已。






洞外有洞





当提到屁股的时候,我们会说,那是太阳不会照到的地方。这个褐色、充满皱褶的洞孔在探讨女性生殖器时常常被忽视,但分隔阴道与屁股的东西只有一座薄墙。位于身体如此末端的地带,屁股无可避免地与阴道、阴户,和许多女性性形象连接在一起。

屁股,又称肛门(anus),为巨大环状肌肉,用于排泄前储存粪便的地方。排便是自古以来是极其重要的任务,我们的身体有一对肛门括约肌。如果其中一块让我们失望的话,还有另外一块括约肌供我们使用。

肛门内括约肌受到所谓的自律神经系统管理,不受意识控制。当身体察觉直肠开始充满粪便,即对內括约肌下达放松的信号。这就是排便反射作用,我们在迫切找到最近的厕所时就能感受得到。

若我们只有原始反射作用的话,会像小孩一样随地便溺,但人类是社会性动物。肛门外括约肌⸺为手指放进屁股并夹紧后所摸到位於上方的东西。它属于随意肌,直到状况允许解放前确保妳能够撑住。如果妳维持夹紧的状态许久,身体会收到暗示,同时下意识知道要错过这次的排泄。粪便微微地回到肠子里并耐心等候更好的时机。我们喜欢戏称为大便出口的部位也会暂时关闭。

屁股是生殖器的暗处,幸运的是它不只有大便的功能。肛门周围与内部充满等候刺激的神经末梢。有些人发现肛门扩展了性生活的规模,如果他们让屁股一起加入狂欢的话。而其他人则藉由赞赏屁股为美丽部位感到满足,并不时对此传达爱意。






毛髮小常识





身为一名女性即代表胯下有毛发。就天性而言,便是如此。在青春期,稀疏的暗色毛发开始出现在妳的维纳斯丘与阴唇边缘。渐渐地,它们开始蔓延、增多,直到形成一片密集、三角形的毛发草原至妳的屁股,而通常会横跨著名的比基尼线长到大腿内侧一点。

近年来无毛或微整型过的阴户再次成为理想美感的潮流⸺也成为许多女性焦虑与问题的来源。许多人担心除毛后会长出更多、更深色的毛发,甚至长得更快速。我们在这几年也害怕如果使用剃刀不慎,比基尼线会不受控制地长出一大堆阴毛。同理可循,许多青少年经常借用父亲的刮胡刀剃掉他蹩脚的胡子,希望长得更加粗犷来遮住青春痘。对此我们感到开心,对少年们而言刚好相反,这一切实在太荒谬。

基因与荷尔蒙决定体毛量和生长的时间。出生时,妳就具备约五百万个伴你一生的毛囊。举例来说,部分的毛囊位于生殖器官及腋下,对荷尔蒙特别敏感。在青春期,我们身体的性荷尔蒙爆发,对荷尔蒙敏感的毛囊扩大并长出粗厚、暗沉的毛发。荷尔蒙敏感的形式根据个人与基因有所变化,这也解释了为何有些男性背后的毛发浓密而他人胸毛稀疏。虽然看似如此,但其实妳在青春期不会长出更多的毛发;只是会渐渐转变为「大人」的毛发。很多人认为剃毛刺激毛发生长的原因只是我们经常在换毛的时候整理它们。

同时有些人觉得除毛时毛发变得更粗、更硬或长得更快。剃掉后的隔天妳可能在坐下时会有摸到带刺豪猪的感觉,但这也不是事实。我们毛发主要以死亡细胞构成。事实上,所有皮肤上看得到的毛发都是死掉的蛋白质,唯一活着的物质都在毛囊下面。即使妳剃掉毛发,毛囊也不会知道,这些死亡的形体只会出现在《魔鬼剋星》(Ghosthunters)。现实世界中,毛囊和以往一样的速率持续长出毛发,然而一无所知的妳却残酷地割下它所安排的一切。

毛囊的大小也决定头发长出的厚度。无论剃了多少次,大小不会改变。也就是说,感觉到毛发变硬只是因为长出来时变短了。一般毛发离开原本的毛囊顶端后长得愈来愈细,这也是为何感觉柔软的原因。当剃毛的时候,我们在毛发最厚、离皮肤表面最近的时候割除,所以重新长出来后,它的尖端会变厚一阵子。

我们可能会咒骂(或珍惜)生长的毛发,但体毛的配置是命中注定的。如果妳决定对毛发做其他的打算,那是妳的选择。人体的毛发绝对有它的功能存在,却也没重要到必须留住它,如果妳想移除它的话。不过值得知道的是毛发有助提高我们对性的敏感。如果妳的伴侣轻抚妳的阴毛,弯下去的部分会对毛囊传达信号,将信息送至神经系统。我们的毛囊连接许多神经末梢,所以没有毛发我们会损失一些感知的体验。

历史上,不同形式的除毛对两性来说都是练习。现在,妳可以剃掉、上蜡、拔毛或使用脱毛霜来当作短期的解决方案。最重要的,即使它们各有优缺,这些选择还是与个人偏好有关。

拔毛与蜜蜡除毛会导致长出的毛发稀疏,因为毛囊在妳连根拔起后受到极大的伤害。它的坏处在于,稀疏的毛发变得较难穿透皮肤,可能造成倒插及毛囊发炎。而脱毛霜是藉由破坏蛋白质结构来「移除」皮肤表层的毛发。既然毛囊没受影响,比起使用其他方法,人们自然也少了毛发倒插的问题。

除毛有许多主要的问题:剃毛肿块、毛发倒插与毛囊炎(pseudofolliculitis barbae)。除毛时,尤其是卷曲的毛发,重新长出来时可能会往回长进皮肤裡。身体将倒插毛发视为外来个体并触发毛囊发炎,看起来像是一个小点。妳若不幸运或挖开小点肿块的话,可能一并得到细菌感染。它会变得更痛更肿,通常会留下疤痕。

媒体上充斥着无肿块除毛的建议,我们完全相信美容专家的建议:毕竟,刮干净的胯下有倒插的毛与斑点实在有碍观瞻。但妳真的需要除毛美容店推销给妳、一瓶65欧元的乳霜吗?或是一把5美元的吉列维纳斯亲肤敏感肌专用除毛刀吗?

不幸的是,妳正在虚掷金钱。真的为倒插毛发与毛囊感染而困扰的话,试试以脱毛霜代替其他方式。如果妳偏好拔毛、蜜蜡或是剃毛,就非常需要注意卫生。在妳开始前需要将除毛区域清洗干净。有毛囊感染风险的人需要杀菌液冲洗或在除完后使用杀菌乳液。妳可以在药局柜台购买这些产品,比起美容院贩卖精美瓶身的专门产品还便宜。

最后,非常重要的一点,若妳有毛发倒插或感染的问题,妳应该避免挤压,因为会造成皮肤疤痕。还有,最坏的状况是感染区域可能会扩散。更有可能造成毛囊严重感染形成一颗葡萄大小的肿块。这样的话,妳得寻求能够温柔排出脓肿并在必要时给予抗生素药方的医生协助。






除毛五诫







1.不要直接除毛或拉扯皮肤



如果妳拉紧皮肤、直接除毛的话,会因剃掉表层的毛发而拥有最光滑、最柔软平面。然而遗憾的是,这个方法更容易让头发生长时嵌入肌肤,造成毛囊发炎。








2.永远使用干净、锋利的除毛刀,最好是新的



因为除毛刀太贵所以很想多用几次,这是假节约的行为。锐利的刀片才能将毛发割除得更干净,更不容易倒插。妳也能用更少的力气除毛,帮助预防刺激及肿块的出现。此外,使用过的刀片充满细菌,会导致毛囊受到感染。








3.使用(便宜)单一刀头的除毛刀



除毛刀永远有着新颖、细致的版本以及增加的刀头数,结果价格跟着高涨。而上头标语往往都是「更彻底除毛」,或许会令人讶异,额外的刀头会切掉肌肤表层下的毛发,所以造成更多倒插的毛发。再者,高价意味着更多人不会经常更换剃刀,使刀子变钝充满细菌,妳最好不要这么做。男性除毛刀通常较为便宜,所以值得买来使用。








4.使用大量温水



无论如何必须避免直接乾剃,干燥的毛发坚硬因此难以割除。妳必须花费更多力气来达到目的,这会更伤害皮肤,增加红肿和发炎。温水是让毛发柔软最有效的方式。如果在剃毛前五分钟使用除毛泡也有同样的效用,即使效果不大,却是大多数人使用的方式(快速涂上,快速剃下)。








5.温和去角质



用画圆的方式温柔清洗除毛部位,同时使用去角质手套或颗粒去角质霜,帮助倒插毛发离开肌肤。切记勿过度使用,因为会造成更多伤害及皮肤发炎。






内部生殖器官⸺潜藏在内的宝物





