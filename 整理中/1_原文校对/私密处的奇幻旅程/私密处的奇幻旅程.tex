% 活着
% 活着.tex

\documentclass[12pt,UTF8]{ctexbook}

% 设置纸张信息。
% 纸张设置配置文件
% 用于定义书籍的页面尺寸和边距

\usepackage[a4paper,twoside]{geometry}
\geometry{
	left=25mm,
	right=20mm,
	top=25mm,
	bottom=25.4mm,
	headsep=1cm, 
    footskip=1cm,
	bindingoffset=10mm
}

% 设置字体,并解决显示难检字问题。
\xeCJKsetup{AutoFallBack=true}
\setCJKmainfont{SimSun}[BoldFont=SimHei, ItalicFont=KaiTi, FallBack=SimSun-ExtB]

% 目录 chapter 级别加点(.)。
\usepackage{titletoc}
\titlecontents{chapter}[0pt]{\vspace{3mm}\bf\addvspace{2pt}\filright}{\contentspush{\thecontentslabel\hspace{0.8em}}}{}{\titlerule*[8pt]{.}\contentspage}

% 设置 part 和 chapter 标题格式。
\ctexset{
	part/name= {第,卷},
	part/number={\chinese{part}},
	chapter/name={第,篇},
	chapter/number={\chinese{chapter}}
}

% 图片相关设置。
\usepackage{graphicx}
\graphicspath{{Images/}}

% 设置署名格式。
\newenvironment{shuming}{\hfill\zihao{4}}

% 注脚每页重新编号,避免编号过大。
\usepackage[perpage]{footmisc}

\title{\heiti\zihao{0} 私密处的奇幻旅程 打破所有女孩对身体的错误迷思}
\author{妮娜‧布罗克 艾伦‧斯托肯‧达尔}
\date{}

\begin{document}

\maketitle
\tableofcontents

\frontmatter

\chapter{前言}

我们在二〇一五的新年开始营运部落格《生殖器二三事》(Underlivet)。我们无法确定媒体对性健康、女性身体与性的胡诌程度有多少。无论好坏,在性方面我们比以往接触得更多。对儿童与青少年早期来说,性知识透过网路便唾手可得。若思考身体是否出状况的话,寻求Google医生的协助确实容易。但人们不是应该在学校的性教育课程中就该有所了解才对吗?

我们对於如何整合資源也无法給予明确的方案。是另外再开一個性專欄吗?還是再請一組天真的医学院学生告訴每個人是否健康、正常呢?

部落格成立的當週,我们因來訪人次超过七百人而互相打電话給彼此歡呼。大多數的訪客大概都是朋友与家人。而兩年多後的現在,我们可以确信「那時候」真的有問題需要我们解答。无论認识与否,我们收到來自讀者无數可愛的回覆,現在文章已經有超过一百四十萬的點閱數。

原本為青少年所設的部落格最後演變將讀者群向外擴展。每一天,來自不同性別、年齡的讀者疑問湧入部落格。我们也時常收到应该在中学课程就学到的基本問題。有時候,讀者们最需要的是安慰他们身上出現的状况是「正常」的、本來的樣貌沒有什麼不好。遺憾的是,這些族群大多皆為女性。

這本書就是為妳而寫,寫給每位不确定自己的行為、模樣、感受是否正确的女性。希望這本書能夠提供妳所需要的信心。也寫給替自己感到开心与驕傲,同時想学習更多存在雙腿之間美好器官之知识的妳。身体的私密部分非常奇妙,我们也相信良好性健康的關鍵在於了解身体是如何運作。

二〇一六年的秋天,報紙頭條向大眾揭示挪威高中的同儕过度性化事件。融入耍帥的殘酷社交壓力意即讓十六歲的少女被迫跨越自身的性別界線⸺這些難以置信的案件令人作嘔。去試想十八歲少年認為藉由社交地位讓新生少女連續替十位男生口交沒有什麼不对的想法确实可怕。挪威VG報紙當時寫道,現今已成為「你情我願的性行為与性暴力界線瀕臨模糊危機」的文化。近年來,我们發現青少年文化的性化程度激增,尤其在少女身上更容易被發現,与甫成年人的安穩環境來比相去甚遠。不幸的是,对多數人来说,讓餘生痛苦的不愉快性經驗會伴隨他们成長。不该是這樣的。

當女人替自己的身体与性別做決定時,她们同時也在更大的環境下抉擇。无论是避孕、墮胎、性別認同或性实踐,文化、宗教和政治的力量控制了這些選擇。

我们想讓女性在公开的事实中獨立抉擇,讓她们根據医学知识而非八卦、誤解与恐懼做決定。奠定身体運作的良好知识基礎將使女性更容易在自信、安心的狀態下做選擇。性必須淺顯易懂,我们必須成為自己身体的主人。我们希望能提供機會幫助妳做出「適合自己」合理、受益的決定。

妳現在有可能坐著猜想:「為何我要看一本兩個学生所寫的医学書?」她们還沒從大学畢業呢!同樣的問題我们也問过自己好幾遍。我们既不是受过完善訓練的医生也非任何領域的專家,所以更懷著一定程度的敬意完成這本書。

我们從德國医学生茱莉亞‧恩德斯(Giulia Enders)的例子獲得勇氣。她的著作《腸保魅力》(Darm mit Charm),獲得成功迴響,使腸子与糞便成為黃金時段的談话節目供人们討论的議題。我们書名的押韻就是对她的致敬。她开創先河,讓我们知道医学可以變得好懂、有趣,特別的是,我们談论身体最私密部分也不會感到一絲尷尬。

因為身為医学院学生,我们擁有別人沒有的優點:我们充滿好奇、年輕又有勇氣問一些「蠢」問題⸺經常是因為我们,或是朋友们,对於彼此都擁有的構造感到興趣。沒有危及職業名聲的風險,也還沒像一般医生一樣,忘記直接了當,反而耗費許久時間向病患說明。希望有更多的年輕同仁付諸行動參与創作。

在寫書的時候,好幾次發現一些我们完全誤解的觀念。我们也一樣會陷入女性性器官的迷思,而且擁有相同想法的人為數眾多。存在最久的大概就屬處女膜迷思,讓女性在整個世界下持續受害,甚至挪威也有一樣的情況。還有少數的医生搞不清楚身体的這個微小部位。也有医生提供患者替女性檢查處女膜的服務,幫助她们延續迷思。在寻找解答時,我们發現資深的婦產科医師會以枯燥或避重就輕的回覆閃避問題。處女膜掌握女人一生是件令人難以置信的事。透过本書,我们已經盡最大的所能破解處女膜的迷思。

另一個迷思是荷爾蒙避孕法既不健康又危險,結果導致上千名選擇不安全避孕方式的少女意外懷孕。我们明白人们对副作用感到茫然与害怕,也受夠医療單位沒給予適當的解釋消除大眾的疑慮。因此我们決定用一大篇幅來探討避孕。反覆閱讀重要文獻記載可能的副作用如情緒不穩及性慾降低後,為了讓大家放心,我们在不确定的狀態下會保持开放的立場。引發嚴重副作用案例相當罕見,且沒有什麼跡象表明情緒低落或性慾下降是影響大多數女性使用荷爾蒙避孕藥的問題。凡事是總有例外,我们希望讀完此書後,妳能夠分辨正常与否。

其他迷思沒有直接的傷害,卻也反應男性主宰医療研究領域行之有年的事实。當朋友抱怨從未有过「陰道高潮」時,代表我们对女性性行為的認知程度已被男性需求扭曲了好幾世紀。從來沒有所謂的陰道高潮,只有高潮的形式不同,而每個方法都一樣愉悅。我们希望女性能夠停止自卑,因為她们需要其他方式的刺激,並非只是直接進入陰道的性行為。

上述探討的主題為《私密處的奇幻旅程》眾多內容裡的一部分。從外陰到卵巢,期待妳一起加入探索女性生殖器官的旅程。如同寫書过程時的我们,希望妳從中獲益良多。对我们来说最重要的是妳看完這本書後能夠得到放鬆。身体非常的單純,每個人在生命中都擁有帶來喜悅与挑戰的身軀。我们要替身体的成就感到驕傲,在它奮鬥的時候保持耐心。

\mainmatter

生殖器官

我们的生殖部位大概是身体最私密的地方了。打從母親產道出生初見曙光的剎那,它就和我们成為最親密的夥伴。在護校裡我们喜歡稱之為尿尿地方的裡面与外面。隨著青春期的开始,第一個展現的特徵為胯下的毛髮。无论充滿驕傲或恐懼,每個人都不會忘記初經來潮。妳或許开始自慰並發現自己能夠在蜷曲的身体中獲得快感。緊接而來的,是初次性經驗造成的傷害、好奇与慾望。也許妳有或曾經有过孩子,也体驗过性高潮帶來的巨大變化,以及其所伴隨的各種驚奇展現。生殖器官是身体的一部分,是時候了解更多了。





陰戶⸺私密部位的奇幻旅程




全裸站在鏡子前面好好看著自己。妳的生殖部位處於肚子下方,包覆在髖骨最前方的脂肪區塊。這片柔軟的地帶稱為維納斯丘,青春期的時候毛髮會开始覆蓋在上面。部分女性在維納斯丘的脂肪墊會比他人要來得大,所以有些人的肚子接近陰部之區域會略微突出,反之另一群人會相对平坦。兩者都是極為正常的。





從維納斯丘往下看的话,妳會來到所謂的陰戶(Vulva),或稱之為妹妹、餅乾罐、陰溝、陰道、屄等等。我们挪威人會叫它小老鼠。陰戶可能不是世界上最常用的詞彙,然而身為女人,妳往兩腿中間一瞧,妳看見的東西就是陰戶。

很多人認為女性性器官的可見之處叫作陰道。妳可能會說,「我的陰道上長了毛髮」,或是「妳的陰道怎麼那麼美」,其实都是不正确的說法。陰道沒有任何毛髮,即使非常漂亮,也不容易被看到。陰道只是性器官的部分稱呼,更準确来说,是當妳進行插入式性行為或生產時所使用的肌肉質管⸺換言之,就是通往子宮的管道。我们如此講究術語是因為,无论我们從中得到多少歡愉,性器官不單只有陰道而已。大多數人偏好將女性生殖器用陰道來表示陰戶,所以接下來我们將從這裡开始介紹奇妙的女性性器官。

陰戶的形狀像一朵花,附有兩層花瓣。信不信由妳,花的比喻並不是由我们發想。看著陰戶的各個部位,我们認為從外而內开始介紹會比較恰當。





花瓣或陰唇(Labia,嘴唇的拉丁文)的目的在於保護更深處的敏感部位。比裡層更加厚实的陰唇外部,充滿著脂肪並以氣囊、避震器的形式運作。外部陰唇或許長到足以覆蓋內層,卻也有狹窄的類型,有些人甚至只有圍住其他陰唇部位的兩片小凹坑而已。

大陰唇由正常皮膚覆蓋。和其他身体部位的皮膚一樣,布滿皮脂腺、汗腺和毛囊。除了毛髮這個好東西之外,大陰唇上也可能長出斑點与濕疹等坏東西。必須遺憾地說,這就是皮膚。並非絕对,不过小陰唇通常比大陰唇長,上面充滿皺摺,像是公主的紗裙。當妳站在鏡前端詳,小陰唇可能會較為突出,也有可能要撥开大陰唇才能看見小陰唇。

相較充滿脂肪的大陰唇,小陰唇更薄又高度敏感。敏感程度雖不像身体最敏感的部位陰蒂一樣,卻又交集了許多神經末端,所以碰触時會非常有感覺。

小陰唇沒有一般皮膚的構造,而是由黏膜包覆⸺妳一定看过黏膜,例如臉上的眼球与嘴唇內部。亦即小陰唇充滿保護及滋潤的黏液。一般皮膚覆蓋在角質層之下,角質層提供保護,讓正常皮膚在乾燥状况下得以成長;然而,黏膜沒有角質層保護,因此不太能承受磨損与撕裂。舉例来说,小陰唇長會因為与緊身褲摩擦感到疼痛。与一般皮膚不同的是,黏膜更加濕潤。黏膜上沒有毛髮,也代表小陰唇裡面沒有毛髮。

如果妳撥开小陰唇的话,會看到陰道前庭(Vestibulum)。陰道前庭來自於拉丁文,意即前庭,或位於建築物入口与內部之間的區域。妳若是會去戲院、劇院的類型,前庭就是妳在中場休息時吃蛋糕、喝香檳的地方。是個有著柱子和柔軟紅絲絨幕簾的華麗入口大廳。女性陰道前庭沒有任何的柱子,儘管如此卻仍是個入口,我们還是認為它无瑕壯麗。接著妳會發現兩個洞口:尿道及陰道口。尿道口位於陰唇与正前方的陰蒂的交會處,陰道則与肛門相鄰。

即使我们每天使用好幾次,還是有很多人对尿道口沒有概念。事实上,有些人以為尿液沒有單獨的排出孔,与男性相同,一個孔有兩種用途:精液与尿液。這是错误的認知:尿道有專屬的洞口。我们不用陰道排尿,即使看过一大堆女性生殖器,卻也很容易誤解。就算從鏡子看过去,尿道口仍難以被看見。尿道非常小,洞口周圍有許多皮膚小皺摺,但只要試著寻找就會發現。





陰道⸺驚奇的擴張管道




与細小的尿道孔不同的是,我们很容易看到更大的陰道口。陰道是從陰部到子宮,長度為7〜10公分狹窄的肌肉質管。大多時候質管為扁平的狀態,所以前後壁相互擠壓。這有助於防水,想像一下!

當妳激起性慾時,陰道口會縱、橫向擴展,在各方面極富彈性。陰道有點像打褶的裙子。用手指触摸的话,妳會感覺到皺褶。

陰道周圍的肌肉非常強韌,當妳將一根手指放進陰道後它會突然緊縮。和其他肌肉一樣,這些骨盆腔底的肌肉,透过鍛鍊會愈來愈強健。

陰道壁裡面充滿濕潤的黏膜。大多黏液不是由腺体分泌,而是從身体內部滲出至陰道壁。陰道壁沒有任何腺体,但有些分泌液來自於子宮頸。陰道一直保持著濕潤的狀態,在性慾高漲時會比以往更加潮濕。更多血液流至整個生殖部位,也就有更多液体從陰道壁滲出。陰蒂和小陰唇开始充血,所以妳會注意到有增量的血液流往生殖器官。黏液在性奮時產生,讓妳在自慰或与他人進行性行為時降低陰道受到的摩擦。通常性交時陰道壁有如被連續擊打一樣,摩擦減少表示对陰道壁的傷害更少了。性行為結束後陰道壁因為些微撕裂而流血是正常的状况,妳也會感到有些疼痛。幸運的是,這不會造成傷害。陰道壁善於自我修復。

除了陰道壁分泌的液体,有些黏液來自前庭的兩個腺体。它们位於陰道口兩側接近屁股的位置,稱為巴多林氏腺(Bartholin’s glands),以丹麥解剖学家卡斯柏‧巴多林(Casper Bartholin)為名。它们分泌出黏液幫助潤滑陰道口。橢圓形的巴多林氏腺体積小若碗豆,卻可能帶來麻煩。輸送黏液的小管堵塞的话會形成陰門囊腫,從陰戶側邊便能感受到小型的硬塊,彷彿像一顆小氣球。一旦這類的囊腫感染,就會轉變成麻煩的状况,不过透过微型手術即可解決問題。有些人对巴多林氏腺之於陰道潤滑的重要性卻持反对立場,因為囊腫感染而切除腺体的女性在性慾高漲時還是能感受到陰道潤滑。

在陰道壁前端,換句话說就是靠近膀胱的位置,存在女性雜誌性專欄裡熱門討论的地方。我们講的正是所謂的G點(G-spot)。以發現者,德國婦產科医生,恩斯特‧格雷芬貝格(Ernst Gräfenberg)之名命名。從一九四〇年代起,科学家不斷探討和研究G點,但爭議層出不窮。科学家不确定它到底為何,也无從證实它存在与否。

在部分女性陰道中G點特別敏感,某些女性表示藉由刺激便能達到高潮。G點差不多位於陰道壁前端,也就是靠近胃部,且可以用「快过來啊」的挑逗手勢來刺激它。幻想一下迪士尼女巫試著誘惑妳靠近她,就是那個姿勢。根據幾名女性的敘述,比起陰道任何地方,G點的刺激會有更好或是不同的感受。妳可能會注意到,相較陰戶甚至是陰蒂,陰道本身不是特別敏感。敏感是陰道口最重要的關鍵,同時緩衝更近一步的動作。

媒体通常將G點看作分开的部位。妳若閱讀过性專欄或性方面的書籍,一定对此印象深刻。二〇一二年英國的一篇綜述文章在現有主張G點与陰道為不同部位的文獻裡得到论點薄弱的結论。大多G點的研究都是透过女性敘述的問卷調查而來。文章中還提到許多相信G點存在的女性不能指出确切位置。研究者同時表示,以成像技術為據的论文无法找到比陰蒂更能讓女性得到高潮或快感的部位。

事实上,其中一個假設為G點並非分離的構造,卻處於陰蒂最深處,在性交時才會受到刺激,直接經由陰道壁而來。二〇一〇年,一群学者發表了一項研究,他们觀察女性進行陰道性行為時陰道壁前端的變化。透过超音波探測發生的过程与找寻G點的位置。他们沒有找到,但認定陰蒂內部非常靠近陰道壁前端,所以陰蒂就是G點謎團的解答。

另個可能為G點与一群陰道壁前端的腺体相連,也就是廣為人知的史基恩氏腺(Skene's glands),相當於女性的攝護腺,圍繞男性部分尿道的腺体。史基恩氏腺結合女性射精或是潮射。有些研究表示G點是達到潮射的重要部位,然而現在都只是理论。我们已經知道部分女性會有潮射的狀態,卻還不清楚G點是否存在。

奇怪的是,和陰道壁一樣的地方居然籠罩在神祕當中,尤其又那麼多对於G點的謬誤。我们屏息以待更多有關女性身体的優質研究。





陰蒂⸺一座冰山




當我们寫到陰蒂內部時妳应该很訝異。內部的哪個地方?畢竟我们常常形容陰蒂的体積就像葡萄乾,位於陰戶的最頂端,精準对齊小陰唇交集的所在。但這個小紐扣卻只是冰山的一角。藏在骨盆區最深處的這個器官,超越妳所有狂野的想像。

即使從十九世紀开始,解剖学家已知陰蒂是巨大的地下器官1 ,卻和普遍知识大相徑庭。解剖学课本裡將男性陰莖描述得非常詳盡,而陰蒂仍留下好奇的空間。一九四八年末,格雷氏解剖学決定不標示陰蒂的項目,連男性主宰的医学界也对陰蒂更進一步的研究沒有興趣。陰蒂确切的構造及如何運作還存在相左的看法。以医学的角度來看,实在令人震驚。





我们所知道的是,人们大多將陰蒂形容為骨盆所延伸的大型器官中的一小部分,且向下延伸至陰戶兩側。透过X光檢測的话,我们能夠看到陰蒂整体形狀像是倒Y字型。稱為陰蒂頭的小葡萄乾,就在正上方。陰蒂長0.5〜3.5公分不等,由於被陰蒂包皮遮住一小部分的關係,所以看起來會更小。陰蒂頭是唯一肉眼看得見的地方,而其下方為陰蒂体,向下延展与身体形成一個角度,模樣近似迴力鏢,前面有一对陰蒂腳,被陰唇兩側包覆在下。





兩腳皆有勃起組織陰蒂海綿体(corpus cavernosum),性慾激起時开始充血。兩腳中間有額外的勃起組織前庭球(bulbi vestibuli),圍繞於陰道及尿道口。

上生物课特別認真的妳,會对剛才的敘述有點印象⸺不过男性陰莖不是也有陰莖頭、腳和勃起組織吗?女性高潮的主要來源為陰蒂,這是不為人知的祕密,至少顯而易見的是,它和勃起的陰莖截然不同。或許會讓人感到驚訝,陰蒂与陰莖為相同器官的兩種版本。

男女性生殖道胚胎在子宮內約十二週時,長得完全相同,由一種像迷你陰莖(或是巨大陰蒂!)的形体主宰一切,稱為生殖結節(genital tubercle)。它具有女性或是男性性器官的潛在發展。陰莖和陰蒂從相同的構造成長,因此兩個器官有許多相似的形式与功能。

陰莖頭和陰蒂頭其实一樣,所以兩者都被賦予相同的名字,分泌腺(glans)。在兩性身体中都是最敏感的地帶。據統計,男女性腺頭擁有超过八千多條感覺神經末梢。感覺神經末梢接受壓力与触摸的資訊,傳送大腦訊號,轉換為疼痛或愉悅的感受。有愈多的神經末梢,大腦就接受愈多不同有力的訊號。然而,陰蒂頭比陰莖頭更加敏感,因為末梢神經集中在更小的部位:沒錯,集中度多了五十倍。

不幸的是,陰蒂為性歡愉开關的認知,讓一些男性相信向它施予壓力是正确不过的。若一點的施壓沒有滿足慾望,他们只會使更大的力氣。這不是陰蒂運作的方式。由於富含了許多神經末梢,即使最細微的變化它都能感覺得到。它提供了意想不到的刺激与快感,卻也代表疼痛或完全麻痺的过度期極短。長期下來,过度加壓造成神經末梢拒絕傳送訊號到大腦:陰蒂按鈕已切換至「靜音」模式。一旦發生,直到準備好再次說话時陰蒂會保持平靜。就好比搭訕一樣:妳做得太过頭,事情通常會搞砸。

男性勃起組織讓陰莖开始充血變硬。更不用說女性的勃起組織也有一樣的功能。當性慾被喚起時,整個陰蒂腫脹至原本的兩倍大,完全是令人肅然起敬的勃起。因為陰蒂腳与前庭球在陰唇下及尿道、陰道口周圍,讓陰戶在性奮時看起來更大。此外,陰道前庭与小陰唇由於血液聚集的關係呈現更深、紫紅的顏色。

相似處還不止這樣。男人最愛吹噓的早晨「一柱擎天」与夜間勃起,我们女性也有。一九七〇年代佛罗里達大学展开兩名陰蒂較大的女性与男性間不同的研究。研究發現女性夜間「勃起」的次數与熟睡男性相當。另一項研究指出女性一夜「勃起」可以達到8次,合計時間為80分鐘!

將所有的資訊結合起來,妳會發現在生物课並沒有学到很多關於陰蒂的內容。這個讓人驕傲的器官一直被忽視、低估、遺忘已久。只有當我们理解陰蒂是如何延伸至整個骨盆腔時才會感受到自己擁有這個驚奇部位的喜悅。





* * *



1	解剖学家科貝爾特(Kobelt)於一八四〇年代描述陰蒂的內結構,並论定男女性的性器官架構相同。





落紅貞潔




數千年以來,不同的文化(包括挪威)極為在乎童貞,不是男性,而是女性的貞操。男性沒有聖父或妓男、純潔或污穢之分,但女性就有,「幸運」的是,從新婚之夜陰道流血便能分辨她是哪一類的女性。

有許多人使用這個說詞:「破她處。」彷彿沒有性經驗的女性就可以像开香檳那樣被破處,就好像陰道在初次性行為前後的不同,和酩悦香檳有沒有軟木塞一樣。妳应该感受到我们的語氣,破處根本不是重點。

童貞的概念在主流文化中廣為流傳。对《嗜血真愛》(True Blood)裡的吸血鬼潔西卡而言,每次的性行為都是第一次,她每一次都必須要流血。同樣的疑團繞著《權力遊戲》(Game of Thrones)的皇后瑪格麗‧提利爾(Margaery Tyrell)打轉。嫁給第三號國王後真的依然純潔吗?

經典作品裡也有提到童貞与落紅。「该死!」我们在挪威文学课裡所放的電影裡看到克莉絲丁‧拉蘭斯達特(Kristin Lavransdatter)的鮮血流到大腿時,可能曾經如此咒罵。不过她卻說了這段话:「誰會想要被摘过的花呢?」她在情人艾伦(Erlend)的雙臂裡嚎啕大哭,而他完全不需要落下男儿淚。身為男人,艾伦根本沒有操守好失去。

女性為无邪花朵,甚至取走她的貞潔等同摘下花朵的概念出現在医学用語。女性第一次發生性行為而流血的情況稱為失去貞潔,整個事態实在是无法形容地古板。這好似來自不同文化与不同歷史年代的男性聚在一起找出控制和限制女性性取向及替身体做決定的方法。





如果妳已經搞清楚上述內容,是時候討论處女膜(hymen)了,這個在陰道口裡的神祕構造仍然讓全世界女性付出名譽或使她们活在陳舊傳統与誤解當中。不敢置信的是至今仍以此分化兩性。美好正面的性愛只毀掉女性卻沒对男性造成任何影響。妳想想看,在處女膜与流血之上,有迷思主宰一切,這整件事实在太愚蠢 。

自古以來處女膜代表貞操的象徵,如同迷思所敘述的,它會在女性第一次性交時破掉且流血,只有這個時候。流血被用來當作貞潔的證明⸺过去人们在新婚之夜後按例將染血的床單拿到外面掛起,讓左鄰右舍看到凡事都是如此進行。

處女膜的迷思為:如果妳性交後流血,人们會知道妳以前沒有过性經驗。如果沒有流血就表示妳已經發生过性關係。然而和其他迷思一樣,完全错误。

這個迷思的信念持續存在的原因在於对處女膜為一片薄膜的認知廣泛流傳。當妳聽到「薄膜」一字時,妳大概想像一片乾淨的塑膠膜在妳戳一個洞後就破裂了。啵!但妳曾經用鏡子看生殖器的话,妳會知道根本沒有一張保鮮膜在妳的陰道上,就算妳沒有發生过性關係。別讓一個迷思被其他迷思給取代。最近我们聽过許多關於「處女膜不存在」的言论。沒有封住陰道的薄膜是正确的觀念,但造成誤解的形体仍然存在。

陰道口裡有個環形摺狀的黏膜倚靠著陰道壁,像一枚戒指。這小戒指以前曾有陰道薄膜、貞潔等稱呼。我们叫作處女膜。雖然這些名字都代表相同的意思,但陰道薄膜一詞容易讓人誤解,最好不要使用。

所有女性生來便具有處女膜,然而对妳来说卻沒有任何用途。處女膜等同男性的乳頭。從我们是胚胎時就是個沒有功能又多餘的部位。

處女膜具有厚度与寬度。換言之,它不是薄得跟塑膠膜一樣,而是厚又堅固。它的外表在少女青春期時光滑,像中間有洞的甜甜圈。接著身体的荷爾蒙交響樂團登台,處女膜和其他身体部位相同,开始隨之產生變化。在青春期結束後,通常會變為新月形。它的後端較寬,朝向肛門,位置依舊圍繞在陰道壁,而中間洞口變得更大了。至少這是理论上的模樣。事实上,處女膜並沒有一定的形狀。

多數女性的處女膜為中間开洞的圓形,然而不是每個人的都那麼平滑。它通常既皺又有凹痕,也不是性行為的代表。有些人的處女膜上有延伸至陰道口的線狀物,所以看起來較像「ø」而非「o」字型。其他人的就像篩子,中間沒有一個大洞,而是有許多小孔在上面。又有一群人的處女膜看起來像小流蘇一樣長在陰道壁,而也有少數女孩的處女膜确实覆蓋整個陰道口,她们的處女膜相當堅固,這是變異所造成的問題,因為經血一定要有地方排出才对!具有此類處女膜的女性往往在初經來臨時才發現問題。若經血困在陰道裡,會造成劇痛且有可能需要進行手術。像密封一樣,這個少見的變異型態是我们最接近處女膜迷思的地方。

无论何種樣貌處女膜都具有彈性,除了少數包覆整個陰道口的例子。即使如此,處女膜仍是陰道最狹窄的地方。陰道具有驚人的延展、收縮能力:畢竟妳能夠讓嬰儿從這裡出來。所以處女膜应该也要能夠延伸。雖然它有彈性,卻不足以在性交時派上用場。有點像是把一條橡皮筋拉得很長,一旦太用力就會應聲斷开。

當妳進行初次陰道性交時,處女膜往剩餘的陰道空間延伸。許多女性的處女膜有彈性到足以應付一切,但对其他人来说,處女膜會撕裂和流一點血。換句话說,有些女性第一次發生關係會流血,有些則不會,完全取決於處女膜的彈性。擁有橫跨陰道口呈現ø字特別形狀處女膜的女性,為了讓陰莖或手指有進入的空間而經常覺得此處有撕裂感。

很難确認有多少女性的處女膜在第一次性交後流血。有幾項研究記錄了統計結果,但數值存在了變數。我们在兩項不同研究中分別看到56%及40%的女性,在初次進行雙方同意的陰道性行為後流血。雖然不是全体女性,比例卻是極高。

這些研究訪問女性關於第一次性交的經驗。我们不能完全确定是處女膜流血,即便它是陰道最狹窄的地方,亦或血液是來自別處。在陰道的部分裡我们有提到,如果是略為激烈的性行為、陰道不夠濕潤或緊張所造成的內部肌肉緊繃,陰道壁裡出現小裂痕而出血是正常、合理的。人们第一次發生關係或其他情況下都有可能發生。

另一個處女膜嚴重的迷思与處女檢測有關。這些試驗代表人们相信透过目視女性生殖器即可辨別她是否發生过性行為。聖母瑪利亞很明顯地經歷过處女測試,聖女貞德和一大堆在現代不同保守環境下的女性也都有相同的遭遇。

有時我们聽聞挪威医生仍然受家屬所託对年輕女性進行處女檢測,證明自己女儿的貞操完好如初⸺儘管法医專家認為這些測試无關緊要。我们也聽說有医生开立處女證明給擔心新婚初夜沒有流血而驚恐的女性。

然而結果為,通常不可能透过處女膜變化加以辨別女生有无性經驗。這使整個處女檢測變得荒謬。雖然處女膜可能在性交時因激烈延展而受損,造成的傷害卻不一定永久。在許多案例上可以發現,處女膜能夠不留任何傷痕復原。

許多處女膜与其變化方式的研究在一項女性初次經驗來自性虐待的調查後受到改變。一篇挪威的綜述文章指出現有影響幼童處女膜的因素(例如洞口寬大或邊緣狹窄),已經認定為沒有特殊發現也並非性虐待的證明。這些處女膜的變化可以從沒有遭受性虐待的孩童上發現。附帶一提,文章的作者還小心翼翼地表示缺乏相關文獻並无法證明孩童沒有曝露於性虐待的遭遇之下。

基本上,妳沒辦法從女性兩腿之間看出她是否有无性經驗。處女膜不是為了保護那些沒有性經驗的人、又或是發生过性行為的人、還是「處女」的人。与身体的其他部位一樣,處女膜外觀的不同是因人而異。抱歉,處女檢測根本沒用。

不幸的是,這個常识鮮為人知。女性仍求助手術确保她们能在新婚之夜流血⸺叫作處女膜整形術。直到二〇〇六年,挪威奧斯陸的沃爾沃特私人診所仍提供這項手術,在寻求医学伦理的諮詢後他们停止了處女膜整形術。議會反对這項手術,因為這會使貞潔問題有了快速修補或替代的解決方案:也就是文化改變。

處女膜整形術依然存在。妳能在网路上以30美元購入包含特效血漿的假黏膜,保證讓妳「和黑歷史說再見」,放心去結婚。順帶一提,埃及政客在二〇〇九年建議禁止進口此產品。

為什麼我们選擇寻求這些方法,而非告訴人们沒有落紅不代表沒有貞操呢?又為何对我们来说,女性直到婚前保有「完好如初」的證明是如此重要?落紅必須變得无意義,處女檢測必須徹底廢除,最重要的,我们必須屏棄貞操本身的重要性。

問題是找到處女膜可信的資訊很困難⸺尤其,分辨是非与否更難。对於處女膜的認知,我们找到的極少資訊对大多數人而言不易理解、无法取得,也不正确。即便發現了優秀的研究文獻,但是医学院最常使用的婦產科课本对處女膜的敘述少之又少,有些迷思也不斷反覆出現。我们還存有超多疑問。更糟的是,獲得的極少數資訊无法滿足需要它的人。我们每個人為此都抱有神聖的任務,只是要不要开始的問題而已。





洞外有洞




當提到屁股的時候,我们會說,那是太陽不會照到的地方。這個褐色、充滿皺褶的洞孔在探討女性生殖器時常常被忽視,但分隔陰道与屁股的東西只有一座薄牆。位於身体如此末端的地帶,屁股无可避免地与陰道、陰戶,和許多女性性形象連結在一起。

屁股,又稱肛門(anus),為巨大環狀肌肉,用於排泄前儲存糞便的地方。排便是自古以來是極其重要的任務,我们的身体有一对肛門括約肌。如果其中一塊讓我们失望的话,還有另外一塊括約肌供我们使用。

肛門內括約肌受到所謂的自律神經系統管理,不受意识控制。當身体察覺到直腸开始充滿糞便,即对內括約肌下達放鬆的訊號。這就是排便反射作用,我们在迫切找到最近的廁所時就能感受得到。

若我们只有原始反射作用的话,會像小孩一樣隨地便溺,但人類是社會性動物。肛門外括約肌⸺為手指放進屁股並夾緊後所摸到位於上方的東西。它屬於隨意肌,直到状况允許解放前确保妳能夠撐住。如果妳維持夾緊的狀態許久,身体會收到暗示,同時下意识知道要錯过這次的排泄。糞便微微地回到腸子裡並耐心等候更好的時機。我们喜歡戲稱為大便出口的部位也會暫時關閉。

屁股是生殖器的暗處,幸運的是它不只有大便的功能。肛門周圍与內部充滿等候刺激的神經末梢。有些人發現肛門擴展了性生活的規模,如果他们讓屁股一起加入狂歡的话。而其他人則藉由讚賞屁股為美麗部位感到滿足,並不時对此傳達愛意。





毛髮小常识




身為一名女性即代表胯下有毛髮。就天性而言,便是如此。在青春期,稀疏的暗色毛髮开始出現在妳的維納斯丘与陰唇邊緣。漸漸地,它们开始蔓延、增多,直到形成一片密集、三角形的毛髮草原至妳的屁股,而通常會橫跨著名的比基尼線長到大腿內側一點。

近年來无毛或微整型过的陰戶再次成為理想美感的潮流⸺也成為許多女性焦慮与問題的來源。許多人擔心除毛後會長出更多、更深色的毛髮,甚至長得更快速。我们在這幾年也害怕如果使用剃刀不慎,比基尼線會不受控制地長出一大堆陰毛。同理可循,許多青少年經常借用父親的刮鬍刀剃掉他蹩腳的鬍子,希望長得更加粗獷來遮住青春痘。对此我们感到开心,对少年们而言剛好相反,這一切实在太荒謬。

基因与荷爾蒙決定体毛量和生長的時間。出生時,妳就具備約五百萬個伴你一生的毛囊。舉例来说,部分的毛囊位於生殖器官及腋下,对荷爾蒙特別敏感。在青春期,我们身体的性荷爾蒙爆發,对荷爾蒙敏感的毛囊擴大並長出粗厚、暗沉的毛髮。荷爾蒙敏感的形式根據個人与基因有所變化,這也解釋了為何有些男性背後的毛髮濃密而他人胸毛稀疏。雖然看似如此,但其实妳在青春期不會長出更多的毛髮;只是會漸漸轉變為「大人」的毛髮。很多人認為剃毛刺激毛髮生長的原因只是我们經常在換毛的時候整理它们。

同時有些人覺得除毛時毛髮變得更粗、更硬或長得更快。剃掉後的隔天妳可能在坐下時會有摸到帶刺豪豬的感覺,但這也不是事实。我们毛髮主要以死亡細胞構成。事实上,所有皮膚上看得到的毛髮都是死掉的蛋白質,唯一活著的物質都在毛囊下面。即使妳剃掉毛髮,毛囊也不會知道,這些死亡的形体只會出現在《魔鬼剋星》(Ghosthunters)。現实世界中,毛囊和以往一樣的速率持續長出毛髮,然而一无所知的妳卻殘酷地割下它所安排的一切。

毛囊的大小也決定頭髮長出的厚度。无论剃了多少次,大小不會改變。也就是說,感覺到毛髮變硬只是因為長出來時變短了。一般毛髮離开原本的毛囊頂端後長得愈來愈細,這也是為何感覺柔軟的原因。當剃毛的時候,我们在毛髮最厚、離皮膚表面最近的時候割除,所以重新長出來後,它的尖端會變厚一陣子。

我们可能會咒罵(或珍惜)生長的毛髮,但体毛的配置是命中註定的。如果妳決定对毛髮做其他的打算,那是妳的選擇。人体的毛髮絕对有它的功能存在,卻也沒重要到必須留住它,如果妳想移除它的话。不过值得知道的是毛髮有助提高我们对性的敏感。如果妳的伴侶輕撫妳的陰毛,彎下去的部分會对毛囊傳達訊號,將訊息送至神經系統。我们的毛囊連結許多神經末梢,所以沒有毛髮我们會損失一些感知的体驗。

歷史上,不同形式的除毛对兩性来说都是練習。現在,妳可以剃掉、上蠟、拔毛或使用脫毛霜來當作短期的解決方案。最重要的,即使它们各有優缺,這些選擇還是与個人偏好有關。

拔毛与蜜蠟除毛會導致長出的毛髮稀疏,因為毛囊在妳連根拔起後受到極大的傷害。它的坏處在於,稀疏的毛髮變得較難穿透皮膚,可能造成倒插及毛囊發炎。而脫毛霜是藉由破坏蛋白質結構來「移除」皮膚表層的毛髮。既然毛囊沒受影響,比起使用其他方法,人们自然也少了毛髮倒插的問題。

除毛有許多主要的問題:剃毛腫塊、毛髮倒插与毛囊炎(pseudofolliculitis barbae)。除毛時,尤其是捲曲的毛髮,重新長出時可能會往回長進皮膚裡。身体將倒插毛髮視為外來個体並触發毛囊發炎,看起來像是一個小點。妳若不幸運或挖开小點腫塊的话,可能一併得到細菌感染。它會變得更痛更腫,通常會留下疤痕。

媒体上充斥著无腫塊除毛的建議,我们完全相信美容專家的建議:畢竟,刮乾淨的胯下有倒插的毛与斑點实在有礙觀瞻。但妳真的需要除毛美容店推銷給妳、一瓶65歐元的乳霜吗?或是一把5美元的吉列維納斯親膚敏感肌專用除毛刀吗?

不幸的是,妳正在虛擲金錢。真的為倒插毛髮与毛囊感染而困擾的话,試試以除毛霜代替其他方式。如果妳偏好拔毛、蜜蠟或是剃毛,就非常需要注意衛生。在妳开始前需要將除毛區域清洗乾淨。有毛囊感染風險的人需要殺菌液沖洗或在除完後使用殺菌乳液。妳可以在藥局櫃台購買這些產品,比起美容院販賣精美瓶身的專門產品還便宜。

最後,非常重要的一點,若妳有毛髮倒插或感染的問題,妳应该避免擠壓,因為會造成皮膚疤痕。還有,最坏的状况是感染區域可能會擴散。更有可能造成毛囊嚴重感染形成一顆葡萄大小的腫塊。這樣的话,妳得寻求能夠溫柔排出膿腫並在必要時給予抗生素藥方的医生协助。





除毛五誡





1.不要直接除毛或拉扯皮膚


如果妳拉緊皮膚、直接除毛的话,會因剃掉表層的毛髮而擁有最光滑、最柔軟平面。然而遺憾的是,這個方法更容易讓頭髮生長時嵌入肌膚,造成毛囊發炎。





2.永遠使用乾淨、鋒利的除毛刀,最好是新的


因為除毛刀太貴所以很想多用幾次,這是假節約的行為。銳利的刀片才能將毛髮割除得更乾淨,更不容易倒插。妳也能用更少的力氣除毛,幫助預防刺激及腫塊的出現。此外,使用过的刀片充滿細菌,會導致毛囊受到感染。





3.使用(便宜)單一刀頭的除毛刀


除毛刀永遠有著新穎、細緻的版本以及增加的刀頭數,結果價格跟著高漲。而上頭標語往往都是「更徹底除毛」,或許會令人訝異,額外的刀頭會切掉肌膚表層下的毛髮,所以造成更多倒插的毛髮。再者,高價意味著更多人不會經常更換剃刀,使刀子變鈍充滿細菌,妳最好不要這麼做。男性除毛刀通常較為便宜,所以值得買來使用。





4.使用大量溫水


无论如何必須避免直接乾剃,乾燥的毛髮堅硬因此難以割除。妳必須花費更多力氣來達到目的,這會更傷害皮膚,增加紅腫和發炎。溫水是讓毛髮柔軟最有效的方式。如果在剃毛前五分鐘使用除毛泡也有同樣的效用,即使效果不大,卻是大多數人使用的方式(快速塗上,快速剃下)。





5.溫和去角質


用畫圓的方式溫柔清洗除毛部位,同時使用去角質手套或顆粒去角質霜,幫助倒插毛髮離开肌膚。切記勿过度使用,因為會造成更多傷害及皮膚發炎。





內部生殖器官⸺潛藏在內的寶物




大家很容易忘記女性生殖器官不只有陰戶与陰道,還有皮膚底下的脂肪、肌肉存在於柔軟、看不到的部位,包括內部生殖器官。

讓我们开始這趟旅程吧。若用一根手指進入陰道,妳會感受到約長7〜10公分,有著一樣厚度、形狀像是鼻尖⸺再大一點的柔軟小凸出物。那是子宮的脖子或稱作子宮頸(cervix),為子宮的入口。從陰道开始,子宮頸看起來像扁平半球。第一眼看起來,它並不像出口或通道,但在正中央卻有個叫作子宮頸口的小洞。這個开口長約2〜3公分、極為狹窄的通道帶我们通往子宮內部,為經血流出的通道,分泌物也會從這裡流出。事实上,這個小小的道路是製造大多分泌物的場所。





很多人以為陰道往子宮的通道是張开的,我们也常常被問到以下問題:懷孕時如果做愛的话會不會讓陰莖撞到寶寶呢?有許多人对性行為与子宮的關係感到好奇。讀过村上春樹的小說《海邊的卡夫卡》(Kafka on the Shore)的话,妳应该挺享受那段女子感受到男人的精子灑在她子宮壁的橋段,就好像他射精時陰莖在她的子宮裡一樣。妳不可能讓陰莖進到子宮裡,子宮頸並非开放的氣室,它是封閉的。任何状况下,陰道在長、寬上具有彈性所以深到能夠容納大部分的陰莖,但也完全沒有必要再更深入進去。

我们的認知是大多女性並不注意自己的子宮頸,也的确不意外。妳既看不到也不一定感受、意识到它的存在。但為了健康著想,子宮頸确实需要妳的重視。子宮頸是年輕女性得到癌症的部位之一,此外,它也是許多性傳染疾病出現症狀的地方。

子宮頸很重要,然而也只是更大型器官⸺子宮(uterus)的一小部分。子宮一般為拳頭大小的器官,一旦懷孕就會擴張得非常離譜,畢竟在孕期裡需要大到足以裝下一個(或多個)成長胚胎的地方。更年期前的女性,子宮大約7.5公分長,重約70克以下。子宮的外型類似一顆倒过來的梨子,而子宮頸就是最纖細的根莖部分。

女性的子宮多為前傾,朝向肚臍,与陰道大約呈90度的距離。這也是另一個陰莖无法進入子宮的原因:陰莖在勃起時不能彎曲,如果彎曲,它就會斷掉,陰莖不是雜技演員!有20%的女性子宮後傾,但運作方式和前傾的完全相同;就和有些人是藍眼睛而有人的是棕色一樣,他们的眼睛都看得到。

子宮是空心的,卻也不像桶子那樣,因為它沒有空氣。子宮的前後壁与陰道壁一樣相互緊壓,夾在其中的為一小層液体。

子宮擁有非常厚的肌肉壁,這些肌肉非常重要,例如在凝結的經血往極窄的子宮頸排出時得仰賴它们。子宮肌肉收縮就好像一塊被擰緊的菜瓜布。當妳經痛時所感覺到的疼痛和腹部或背部絞痛一樣,但痛處卻來自於將經血与黏液推出的子宮本身。

子宮壁有許多層,而最裡面的子宮內膜為黏膜的一種。它大幅改變經期的过程,並在其中扮演核心的角色。子宮內膜每個月變大和增厚,如果沒有懷孕便會從子宮排出。值得記住這個名字的原因是它和困擾許多女性甚鉅的病症相似:子宮內膜異位(endometriosis)。這是子宮內壁長到身体其他地方所造成的疾病。在其他引發的症狀裡,它會產生更嚴重的經痛。妳稍後會学到更多關於子宮內膜異位的內容。

將子宮想成一個三角形,其中一個角朝下而其他兩角分別有細細的管子延伸出去。這是大家所知道的輸卵管(fallopian tubes),兩側都有十公分長,目的在於將卵子從卵巢(ovaries)向下送至子宮。輸卵管末端有著布滿小型指狀物的輸卵管傘(fimbriae),向卵巢延伸並接收它排出的卵子。在輸卵管受精的卵子會往子宮移動,為了生長而著床於子宮內膜。

我们擁有兩個像小袋子或壘包一樣的卵巢,位於子宮的兩側,而且它们有兩個任務。第一為儲藏並使女性的性細胞成熟成卵子。与男性不同的是,女性在一生中不會製造新的性細胞。打從一出生我们就只擁有三十萬顆卵子,不过它们皆尚未成熟。生來所擁有的卵子其实都是受精卵的前身。這些前身在胚胎生長的六天就已經形成。直到青春期开始月經週期時,它们才會為下一個任務做準備,再來一批批的成熟。但由於沒收到大腦排卵的訊號,最後以大規模的形式就此滅亡。到達青春期後,我们為了演練,早已流失了三分之一的卵子,只剩約莫十八萬顆。而二十五歲的時候,我们剩六萬五千顆。這些卵子必須耐心等候,在每次月經週期時成熟釋出。

妳現在应该覺得莫名其妙,我们在青春期开始擁有十八萬顆卵子。一生中我们不會有許多生理期,那要數萬顆卵子做什麼呢?事实上⸺這也同時讓人感到訝異,我们每個月可以用掉一千顆卵子,不是單單只有一顆。換言之,卵子与男性精子的不同並非在於經常大量製造。对女性,也对男性来说,多個性細胞相互競爭就為了对的匹配及生成嬰儿。每個月有一千顆卵子成熟,卻只有一顆能夠跨越封鎖線被卵巢選擇並排出。剩下的就會被殘忍地淘汰、摧毀。

有好多次,我们一直想到一個關於荷爾蒙避孕的問題:避孕會阻止排卵讓卵子和生育力維持得更長久吗?对身体来说,保存卵子到準備懷孕時使用而非每個月透过生理期拋棄它们是非常值得的,畢竟聽起來很有邏輯,但它卻不是這麼運作。荷爾蒙避孕每個月只會避掉一顆被卵巢選上的卵子,不能預防那一千顆成熟的卵子。无论避孕多少次,妳每個月還是會失去許多卵子。

到了四十五至五十五歲,我们進入更年期的年紀,女性身体已經歷許多如同青春期所遇到戲劇性轉變的階段。最重要的改變是停止受孕,我们已經用完庫存的卵子。每位女性更年期的年齡有所不同,且开始的時間點主要由基因決定。

另外,有些女性天生擁有比他人更多的卵子。而男性還會持續製造精子直到心臟停止跳動的時候⸺一天最多至數百萬條。即使每年的精子品質下降,他们的生育力卻沒有保存期限2 。七十二歲的滾石樂團主唱米克‧傑格(Mick Jagger),目前正等待他与年輕模特儿女友的第八名孩子。這個世界实在不公平。

卵巢的第二項任務為製造荷爾蒙,最重要也最廣為人知的就是雌激素(oestrogen)与黃体素(progesterone)。這些激素在生命不同階段改變我们的身体,並与包含大腦在內的幾個不同部位的激素控制月經週期。不过我们之後再回來談论這段內容。





* * *



2	精子細胞的品質隨著年齡降低。換句话說,男性的歲數影響伴侶的生育力与孩子先天疾病的風險。





他与她与亻她




对許多人而言,性別一詞相當於:女性与男性、女孩与男孩。當妳聽到:「什麼是男人?」或「什麼是女人?」的問題時,也許會覺得小事一樁;因為理所當然的,男人就是擁有男性身体的人,女人就是擁有女性身体的人。舉例来说,《陰緣際會》一書,是一本關於人類所擁有的陰道与其他女性生殖器的書,所以這一定是本關於女性的書,沒錯吧?

這麼想的话确实不令妳意外,但卻沒那麼單純。无论我们是女性或男性,並非只憑生殖器官或体態來判斷。此外,兩性身体上的不同比妳想像的要來得少。

在接下來的部分,我们會把重點放在三個決定性別的因素:我们的染色体(chromosomes),在這裡稱為基因性別(genetic gender);我们的身体,或稱生理性別(biological gender);以及心理因素,或叫作心理性別(psychological gender)。我们沒有說這些就是構成「性別」的完全因素,當然也可以探討社會和文化因素。不过既然這是本医学書,我们決定著重於基因、身体与心理方面。





基因性別⸺烹飪書




妳曾經看过DNA螺旋圖吗?從高倍率顯微鏡拉近觀看,它就像一個扭曲成螺旋形狀的梯子。而DNA螺旋梯上的橫擋並不同於妳用來換公寓燈泡所踩的階梯。以小到只能從顯微鏡看到的寬度来说,DNA螺旋梯極長且擁有非常特別的橫擋。





DNA螺旋梯上的橫擋由不同物質組成,我们可以當作字母來看。每個橫擋為兩個字母,合在一起能夠看成一組密碼或是小食譜。每個食譜所編列的蛋白質會在身体執行特別的任務。組合起來,我们將這些蛋白質的密碼稱作基因(gene)。我们的基因決定我们擁有藍色或棕色的眼睛、兩條或三條腿、翅膀及尾巴或腦袋的大小。這些密碼有點像是集結不同食譜的烹飪書,記載每一個必備的物質來塑造我们。這類烹飪書有個別緻的名字叫作基因体(genome),我们的基因体就是整個基因的食譜。

人類身体裡的每一個細胞都包含一本專屬的烹飪書,每個細胞裡面有平均約3公尺的DNA雙股螺旋,為警察透过血液、精子、指甲或皮膚細胞找到犯人的依據。如果妳從別人身上隨便取出一個細胞,例如挪威首相艾娜‧瑟爾貝克(Erna Solberg),這個細胞,照理来说,會包含所有打造全新的她,也就是複製品所需要的資訊。但是一個3公尺的烹飪書,是如何擠進像細胞那麼小的物質呢?DNA雙股螺旋會相互緊密交纏成條,像毛線一樣,所以什麼東西都可以進得去。每個細胞有46條,將整個基因密碼,也就是烹飪書組合起來,這每一條便是所謂的染色体(chromosomes)。

染色体是成对的,所以我们有23对、46條染色体,每一对裡的每一條分別來自於母親与父親。

講到性別的话,唯一能決定的就是:我们的第23对性染色体。就基因上来说,這兩條決定我们是男生還是女生。性染色体有兩種,意即大家所知的X与Y。女性的組合為相同種類,代號為XX,而男性的是X及Y變体,代號為XY。

回想一下,我们是由母親的一個細胞(卵子細胞)和父親的一個細胞(精子細胞)而來。每個細胞包含半套染色体組,也就是23條或是半本烹飪書。在懷孕時,妳把來自母親的半本烹飪書及另外半本來自父親的烹飪書合而為一,給了孩子一整本有著獨特食譜的烹飪書。

因為基因上為女性的人沒有Y染色体,只有兩個X,所以卵細胞只有性染色体X。這是母親对胚胎的第23对染色体的影響,她永遠无法提供Y染色体。然而,父親的精子細胞,就包含X或Y染色体,精子細胞中的X、Y染色体的比例各占一半。与卵子結合的精子細胞為Y染色体的话,則組合密碼為XY,胚胎裡就會是男孩。若和卵子結合的精子細胞為X染色体,則胚胎為女孩,組合密碼為XX。

因此,一直都是由男性「決定」孩子的性別。自古以來女性受到嚴重的「生男孩」壓力,妳可能讀过關於失望的國王期盼皇后誕下合適繼承者的小說,而生下的孩子當然必須是男生。

現在我们懂得更多了,孩子是女或男的機率單純各占一半3 ,取決於其中一種男性精子細胞与卵子做結合,女性卵細胞对孩子性別沒有任何影響。

總結来说:若第23对染色体有兩個X染色体,胚胎的烹飪書會下達「變成女性」的指令。如果分別為X、Y染色体,則烹飪書會下達「變成男性」的指令。

一切看起來完整又簡單,有著這些食譜,妳會建立起性別只是「非此即彼」的印象。之後妳會發現,不單單只是這樣。事实上,男性及女性的生殖器官極為相像,在器官成熟的过程之中有許多現象發生。我们往往太过專注於兩者間的不同,然而,我们兩腿間所擁有的並非只是「洞或棒狀物」而已。

无论染色体或基因,皆有可能在DNA裡的某個地方出錯,而使食譜也造成错误的結果。食譜上的错误代表結果也會跟著不同⸺有點像是本來该加一公斤的糖卻加成胡椒一樣。或許嚐起來還不錯,卻跟妳所想得完全不同。

人们有可能生下來就有过多或过少的性染色体,那麼這樣會變成哪個性別呢?是X、XXX,還是XXY呢?這真是個好問題。可能妳現在才意识到,其实根本沒有所謂的YY染色体組合,因為兩個精子細胞是无法結合而成一個嬰儿胚胎。

為了追根究柢,我们需要探討一些關於生殖器官是如何發展的內容,因此,這也變成介紹第二個性別面向的好時機:生理性別。





生理性別⸺身体与性器官




目前為止,我们已經知道卵細胞會与精子細胞結合。沒發生問題的话,我们就有XX或是XY食譜⸺女人或男人。儘管如此,男、女生的胚胎在一开始完全沒有差別。事实上在初期,无论染色体的組合為何,胚胎是完全一樣的。胚胎一直都是從不分性別的生殖器开始發展,變成女性或男性生殖器官的機率皆有可能,而內生殖器官也會隨之變為睪丸与卵巢。

為了方便,我们著重在外生殖器官就好。它们一开始是長這樣的:





生殖部位的最頂部為生殖結節。看起來有點像小陰莖,不是吗?還是像陰蒂呢?生殖結節确实都能變成兩者。

為了讓中性胚胎生殖器發展為男性生殖器官,胚胎在懷孕初期需要所有在重要時程中具備的要素。胚胎必須在正确時刻受到男性性荷爾蒙的影響,在這場遊戲裡擔任要角的荷爾蒙為睪固酮(testosterone),只有胚胎有Y染色体才能製造出來。附有Y染色体的胚胎沒有受到睪固酮影響的话,主要原因為一個至多個的胚胎基因發生错误,而生殖部位也自然變為陰戶,因此導致基因上為男孩但生殖器官卻為女孩的形式。

換句话說,除非發生反向指令的特別状况,陰戶是所有胚胎具備的部位。有些男性會理解為男性有「額外的部位」,而女性就很普通⸺好比說將白色T恤与花俏的派对上衣做比較,妳可以自行解讀想法。妳也可以直接說女性是第一性,男性卻是變体,為第二性。等等⋯⋯這樣不也是女性吗?

看看性別發展的圖形,我们在之前提过,胚胎生殖部位頂端的小結,生殖結節,能變為陰莖或陰蒂。如果妳对陰莖稍微了解,又在前面讀过關於陰蒂的章節,一定會明白兩者有許多共通處。

对於受陰蒂頭大小所苦的女性而言尤其重要。陰蒂長得像小巧可愛的鈕扣是我们以前接收到的知识,但它的外表可能會長到延伸出來。這不代表妳更像男人!陰蒂和陰莖一樣有不同的大小,長度介於七到二十公分,所以小陰莖也不會讓男性更像女人。

回到我们的胚胎,男性的尿道与陰莖合為一体,而女性的尿道是分开的部位。成長中的陰蒂⸺陰莖旁邊形成皺褶,這些皺褶會變為男性的陰囊(scrotum)或女性的大陰唇(labia majora)。要形成陰囊的话,就必須要在中間結合。若要形成陰唇,雖然不需要結合,卻還要再成長一些。

如果妳不相信我们所說男性外生殖器官和我们的非常類似,下次遇到裸身的男性妳可以仔細瞧瞧他的兩腿中間。妳會看到,他的陰囊是由一條整齊、細長的線,像裂縫一樣分成兩邊。妳知道吗?那就是裂縫!這是陰唇合而為一所變成的陰囊!陰莖沒有變化,但过度生長的陰蒂有著內藏的尿道:想像一下,將它大幅縮小,尿道下移,陰囊一分為二,就會變成另一種陰戶了。

哇!看起來实在很酷,但千萬不要切开妳所認识的可愛男性,男人需要陰囊存放睪丸。這与外科医生將男性身体改造成女性的性別确認手術(gender confirmation surgery)非常相似,我们稍後再來討论那個內容。

現在我们回到染色体错误的問題。所有缺乏Y染色体的胚胎在生理上會變為女性,反之有著睪固酮影響的Y染色体胎儿會在生理上成為男性;或是像人氣漫畫集《世上最後一個男人》(Y: The Last Man)一樣,全都被消滅了⸺並非如此。

這些只是理论上的案例,若將胎儿編為X或XXX,烹飪書會說這是女人。如果是Y或XXY的话,食譜會朝男性开始成長。然而在其他烹飪書內,結果並非總是和食譜寫的一樣。以身体来说,即使基因上為男性,還是有可能長成女性⸺反之亦然!

有些胎儿在基因上為男性,卻无法对身体產生的睪固酮有反應。少了睪固酮,他们會在外表上變成女性,兩腿間有陰戶而非陰莖与陰囊。而漸變的状况也的确存在,即使有陰戶,有人出生沒有子宮,肚子裡沒有卵巢但是兩腿間卻有睪丸。外生殖器也有可能在最後發展為陰莖⸺睪丸的組合(男性生殖器官)与陰戶兼具的情形。

每年都會有接生員在孩子出生時被父母問性別問到想破頭的事情。事实上,他们无法給予明确的答案。這類的診斷可以稱為雙性人(intersex)4 ,意即「介於兩者之間的性別」。

之前我们提到的案例,基因性別与外生殖器官沒有对應關係,就是雙性的一種。雙性有許多形式,可能是外生殖器与性別不符,或內、外生殖器对應不同性別或是兼具。

許多雙性孩童從一出生便接受手術,帶給我们遺憾的歷史教訓。以前,所有与生俱來「不明确」外生殖器的孩子會透过手術變為女性。第一,人们覺得合理是因為性別与後天教養有關。只要以指定的性別養育孩童,他们就會認為自己便是那個性別。譬如給他们娃娃及粉色衣服,也是一樣的手法。這就是俗话所說的,先天之於後天的教養。

再者,外科医生認為打造陰戶比陰莖与睪丸的成果要來得好。他们本身即是男性,便會覺得男人沒辦法在小又只有一半功能的陰莖下生活,但对女性来说只有一半功能的陰戶並不會造成問題。畢竟,性对男性是最重要的。結果造成孩子们在生理上為女孩,在基因及心理上,卻還是男孩。許多生命就這樣被毀滅。

因此有很多外科医生大幅改變了做法。現在使用更深入的檢查來判定性別,确保孩童在出生前為「正确性別」。雖然不再讓嬰儿出生時動手術,但是通常需要花好幾年的時間來檢驗。

這類的做法有一些反对意見。許多人覺得這些孩子不该被動手術,应该要讓他们成年時決定自己所要的選擇。這派看法認為人類必須往男孩或女孩的模樣發展是错误的原則。為什麼不能接受兩者之間呢?為什麼我们不能以「雙性」養育孩子,讓他们慢慢探索自己的性別認同呢?這也帶我们前往第三個性別面向:心理性別。





心理性別⸺認知的問題




心理性別比起生理又更難以解釋,因為我们心理性別是關於認同的問題:我们对自己的想法、我们是誰。這是非常主觀的,只有妳知道適合自己的是什麼。

許多事情被忽略的原因是我们对於「正常」這件事想得太複雜。对多數人來講,一個性別有三個因素。我们覺得是女人,我们兩腿間看起來是女人,而我们的基因也确定我们是女人。实際上我们大多經歷过的事情对每個人来说不代表一樣⸺這是人類不斷反覆学習的课題。

當你的儿子說他是女生,只想要穿洋裝,比起火車組合与足球更喜歡姐姐的芭比收藏。要強調這只是过渡期非常容易,然而事实並非如此。為了成為女生並不一定要「陰柔」或喜歡娃娃勝过足球。心理性別与個性不同,也不須根據傳統性別角色來判斷。儘管如此,人類的心理性別很有可能不同於生殖及基因性別。

我们經常使用跨性別或「生錯身体」來描述不同性別的人,所以什麼是跨性別呢?跨性別來自拉丁語,意即「穿过」、「跨越」或「改變」,和超越有一樣的意思。它用來形容性別認同与其所屬基因、生理性別相反的人士。如果沒有特定認同性別的话他们也會稱自己為跨性別:並不是所有人都需要這類的標籤。跨性別通常會以星字號標示(trans*),表達涵蓋許多層面的廣義用詞。舉例来说,可以問一個跨性別人士希望被如何稱呼:是他、她或是他们呢?還是其他完全不同的稱呼呢?妳不需要事前知道,所以好奇的话就問吧。

非跨性別人士稱為順性別(cis),同樣來自拉丁語,即「跨越」的反義詞。而順性別一詞有「留在同一邊」的含義。

跨性別女性為以男性身体出生但卻是女性,並希望改變她的身体讓身、心理性別相符的人。而跨性別男性是以女性身体出生卻認定自己為男性的人。

許多跨性別人士從童年开始便知道自己不屬於生理上的性別。和未知事物看似可怕的道理一樣,对家長来说可能會受到驚嚇。因此我们認為探討跨性別与喚起其意识非常重要。如果人们懷疑自己的孩子「出生在错误的身体」,可以交由儿科医師判斷。倘若确实,孩童可以透过荷爾蒙与手術的輔助接受性別确認治療。

幸運的是,多數人漸漸在主流文化下使用跨性別一詞。二〇一三年電視劇《勁爆女子監獄》(Orange is the New Black)的演員拉維恩‧考克斯(Laverne Cox)与卡戴珊家族的凱特琳‧詹納(Caitlyn Jenner),為近年來大力推廣跨性別議題的代表人物。挪威影集《錯位人生》(Born in the Wrong Body)吸引了許多關注,許多挪威跨性別者也積極參与社會辯论。資深医師艾斯本‧艾斯特‧皮耶里‧本尼斯泰德(Esben Esther Pirelli Benestad)則主張自己是流性人(genderfluid),並偏好使用他们一詞。而不久前,跨性別男性盧卡‧戴伦‧艾斯貝沙(Luca Dalen Espseth)也挺身而出告訴挪威孩童与年輕跨性別族群他们並不孤單。





結论




決定我们所屬的性別(至少)有三個因素,分別為基因、生理与心理性別。性別不需要二分法,我们染色体的缺陷可能代表我们不需要典型的染色体組合XX与XY。基因缺陷有可能使我们在生殖器官的發展中介於女性与男性之間的狀態。而心理性別也可能与妳出生的生殖及基因性別有所不同。換句话說,性別並非看起來如此簡單。我们希望這個概述能激起你的好奇,讓你更能接受性別帶來的多元可能。





* * *



3	事实上機率並非完全各占一半,新生儿男性多过女性的比例較高。



4	雙性人:对於這個名詞有許多不同的看法,有可能用來描述一群有著這樣的医療状况或身分的人。我们認為在男女性身体變異上會是個不錯的詞彙,但我们知道有其他人傾向使用不同名稱來稱呼自己。





分泌物、月經与血栓





如同我们身体的其他洞口,陰道也是一個出口,並不只是讓東西放進去的地方。從那裡出來的有尖叫的嬰儿、血液、黏液与血栓,因此陰道成為充滿喜悅与尷尬的根源,也是我们發現身体異狀的一個管道。而主導這一切的重要物質,就是荷爾蒙。是時候探討我们生殖器官中不太明顯的部分了。





私密處沖洗器与氣味




分泌物,這個詞琅琅上口。看起來是會不禁想起水管系統或是污水管的奇怪字眼。我们最熟悉的分泌物為黏滑、乳狀或黃白色污垢,青春期之後時常出現在我们的內褲上,讓我们的的內褲變髒。也許分泌物很難令我们驚訝的原因在於它不是熱門或高談闊论的话題,而且看起來又髒兮兮的。此外,在大多異性戀男性眼裡,濕潤陰道最重要的是能夠把陰莖放進去而已。那麼什麼是分泌物呢?下体裡不同液狀物質之間有什麼差別呢?為什麼我们要在一开始留意分泌物呢?

所有歷經青春期的健康女生都會在內褲上發現分泌物,每一天都有。分泌物是從青春期第一天开始,受到雌激素(oestrogen)影響導致陰道持續滲出的液態物質。有些分泌物是來自子宮頸的腺体。陰道本身沒有任何的腺体,卻有許多結合子宮頸、陰道口,包括巴多林氏腺所分泌之液体透过陰道壁流出。

基本上一天會有半匙到一茶匙的分泌物滲出,依照女性身体与生理週期有所變化,而有些使用荷爾蒙避孕藥及懷孕的女性會發現分泌物增加。從液態到黏稠,分泌物的濃稠度也會不一樣,在排卵期前則會出現蛋白細絲狀分泌物。

分泌物不單只是正常的現象,而是必須的。它讓陰道變成自淨管道,目的在於保持陰道乾淨並除掉真菌、細菌,連同表層黏膜的坏死細胞等不速之客。此外,分泌物還富有大量良性乳酸菌,稱為乳酸桿菌(lactobacilli)。這些物質,对,相信妳已經猜到,就是乳酸菌讓分泌物嚐起來、聞起來有些微的酸味。

更重要的是,乳酸菌所帶來的低酸鹼值对健康陰道是不可或缺的。多數造成疾病的細菌无法在酸性的環境下生存。再者,彼此在空間及養分上的需求相同,所以乳酸菌能夠阻止潛在有害菌体找到生長的環境,最終達到預防感染的目的。簡單来说,分泌物會維持陰道的健康。

同一時間,它能潤滑黏膜保持滋潤。乾燥的黏膜容易撕裂,這種問題一旦發生,各種麻煩馬上接踵而至。只要想想妳的嘴巴沒有口水就好,沒有分泌物,陰道黏膜會撕裂,妳會感到些許疼痛。性交跟著變成夢靨,而身体屏障遭到破坏使性行為傳染風險增加。換言之,分泌物並不是应该要從陰道沖掉的髒東西,而是我们重要的夥伴。

問題在於,人们認為分泌物很髒,是不潔或衛生不好的象徵。女生很少把穿过的內褲丟在或掛在浴室裡。在某些社會裡,有人竟然認為应该將陰道上的分泌物沖洗乾淨。妳大概沒想过羞辱人的「混蛋」(douchebag)一詞從何而來。妮娜也是,直到她搬到美國,把在店裡買的一瓶私密處清潔劑留在宿舍公共淋浴間內以後才明白。过一陣子,有位竊笑的学生告訴她应该要把洗劑拿走,因為關於帶著沖洗器的挪威女孩的謠言已經滿天飛。

「沖洗器?」妮娜帶有一絲困惑地問著。她馬上被告知每個人都以為她用一種燈泡狀的注射器擠出帶有香味的肥皂水到她的陰道裡⸺大多是性工作者及其他女性常用的東西。妮娜試著解釋這只是普通的pH3.5陰道清洗劑,但隨即放棄說服其他同学。好女孩千萬不要,拜託,把重點放在生殖部位需要經常沖洗這件事。即使承認妳覺得清洗生殖器是個禁忌,就好像放棄分泌物這個好東西一樣。妮娜最後將瓶子留在淋浴間。

我们的生殖器官最喜歡溫水或溫和的親密部位清洗皂。妳不应该用一般的肥皂,因為很容易造成下体脆弱的黏膜乾燥或受損。通常使用太刺激的洗劑或者过於用力清潔會造成搔癢或灼熱感。无论如何,妳不该用错误方法清洗陰道,增加它受到感染的風險。

女性會在什麼理由下覺得她们需要沖洗自己的陰道呢?对大多數人来说,大概和氣味有關。許多和我们对談过的女性对自己下体聞起來是否「正常」感到焦慮。她们擔心开會時坐在隔壁的同事是否聞得到陰道的味道,或是拒絕讓性伴侶往私處探頭以免聞到味道而失去性致。

健康生殖部位的味道,就是它本來的味道。新鮮的分泌物含有乳酸的關係所以聞起來、嚐起來帶有一點酸味。此外,陰戶与鼠蹊部布滿許多汗腺,緊身褲、合成纖維的內褲及翹腳的姿勢在雙腿間製造了溫暖的環境。一整天下來,妳也自然而然流了一堆汗。一天的分泌物和汗水在与殘留的尿液作用後形成特別的氣味。在我们女性朋友的圈子裡,我们常常使用挪威话「discomus」,意即舞廳老鼠,來描述在舞池歷經長夜或在健身房運動後生殖部位(妳的小老鼠)所排出的特別味道。味道聞起來沒那麼糟糕,只是氣味比較重。





分泌物的味道与分量依據生理週期有所不同。我们的性荷爾蒙有影響身体除掉惡氣物質三甲胺(trimethylamine)的能力,也就是隨處可聞、腐敗的魚臭味。經研究發現,生理前、中期的健康女性身体至少有60%至70%的能力可以排除此物質,也同時解釋了即使是健康女性在生理期時生殖部位也會有魚腥味。

生殖部位的味道是最靠近我们的氣味之一。妳融會貫通的话會發現,其实有味道是正常的,特別是在一天結束之後。如果妳明白我们意思的话,它们通常聞起來不會太糟糕。難聞的氣味可能是感染的徵兆,最好该去看個医生。倘若檢查过後妳的氣味問題不是感染造成,或許穿著寬鬆的褲子或裙子、時常更換貼身衣物以及多多注意個人衛生(但也不要太过頭!)會比較好。

明白的话,分泌物与性器官的狀態密不可分,所以一點也不意外,只要小小地觀察,就能知道私密處的状况。分泌物可能導致感染及陰道菌落(vaginal flora)的不平衡,同時在一般生理週期也會發生大幅變化的情況。

換句话說,了解正常分泌物的氣味、顏色与濃稠度為何重要?一天當中有些人分泌得少而有些人大量分泌以至必須更換內褲,兩者都是正常的。最重要的在於明白什麼樣的狀態对妳而言才正常。這樣一來,妳不止在身体出状况或何時该看医生有初步的認知,還能意會到自己處於生理週期的哪個階段。為了給予妳一點幫助,我们列出了分泌物指南。





需要向医生求診的分泌物




‧大量水狀分泌物呈灰白色且帶有魚腥味⸺可能是陰道菌落不平衡導致的細菌性陰道炎(bacterial vaginosis)。

‧黏稠、白色塊狀、氣味正常的分泌物有可能為陰道酵母菌感染(yeast infection)的徵兆。

‧分泌物流量增加,普遍呈現黃白色⸺可能是披衣菌(chlamydia)、黴漿菌(mycoplasma)或淋病(gonorrhoea)感染,而淋病比前兩項症狀更常產生黃綠色分泌物。

‧・大量黃綠色、呈水狀發泡並帶有惡臭的分泌物⸺可能為滴蟲病(trichomoniasis),在挪威相當罕見5 。

‧大量白色、顆粒狀、氣味正常的分泌物⸺可能是乳酸桿菌过度分泌,會特別容易有搔癢感及鼠蹊部疼痛的状况。

‧・非生理期卻帶血的分泌物(任何呈現棕色、粉色、暗沉或是鮮血色小點的分泌物)可能為性傳染或子宮頸不正常細胞導致的状况。發生任何不明原因的出血妳应该去找医生諮詢。





不會造成疑慮的正常分泌物變化




‧黏狀蛋白,可透过手指展开的分泌物⸺可能是排卵期將近。

‧大量与平常相同之氣味、顏色和黏稠度的分泌物⸺可能為荷爾蒙避孕或懷孕所造成的情況。





* * *



5	陰道滴蟲(trichomonas vaginalis)為造成滴蟲病的小型寄生蟲。在挪威是罕見的疾病,但卻是全球最常見性傳染疾病之一。有些人的陰戶与陰道會伴隨強烈搔癢感与惡臭,碰水時有灼熱感,而有些人並不會注意到這些症狀。滴蟲感染不會造成危險,可使用抗生素甲硝唑(metronidazole)予以治療。





生理期⸺过度失血也不會死亡的狀態




基本上每個月會有一次生理期。有時候會覺得痛,有時候突然報到讓妳感到尷尬,不过大多時候都是順利的。雖然每個月陰道沒有流血我们也能生活得不錯,但生理週期的出現在特別状况下可以說是極大的安慰:呼!妳這次沒有懷孕。

月經在我们生命中占了極大的分量。如果妳每個月都有五天的生理期,代表每年有六十天的時間在流血。如果妳的月經週期維持四十年的话,表示一生有兩千四百天的生理期⸺等於六年半的時間!妳猜到的话,現在我们应该更深入談论這個话題,尤其包含一堆難受的挑戰,例如經前症候群(pre-menstrual syndrome,PMS,我们稍後會提到)、尷尬的情況及激烈的疼痛。

雖然相較活在人類發明衛生棉條、月經杯、衛生棉和止痛藥之前的姐妹们,挪威現代女性面臨的難題不大,但這些挑戰還真夠糟的。當年的挪威女性用鉤針或棒針編織棉質衛生巾,甚至每次使用完畢後必須下水煮沸並且晾乾。在世界各地,月經仍然是個極大的挑戰。當妳知道对於世界的某處来说,有人因每個月流血而放棄上学,或沒有乾淨、可替換的用品所以使用髒布而得到感染的情況視為理所當然,經前症候群也變得无足輕重了。月經在全球女性真正的平等上是經常被忽視的阻礙。下次買棉條的時候妳可以好好思考一下。

讓我们將主題聚焦在流血本身,大多數人知道這与生育有關。月經代表妳身体內部擁有一定的週期而且有能力孕育孩子。不过真正流出來的東西是什麼,傷口又在哪裡呢?為什麼經血的顏色從棕色到紅色都有,又為何是塊狀的呢?

經血來自於接收受精卵失敗因而等待下次著床的子宮。子宮藉由增厚內膜或黏膜來準備懷孕,意即子宮內壁或內層。受精卵會附著在內層上,提供母親血液作為細小成長胚胎的養分。當沒有受精卵著床時,身体也就不需要厚層黏膜,因此全部都流血排出,而這也是經血黏稠的原因。有些血栓只是排出黏膜的碎屑,流出來的並非從子宮而來的純血。

許多女性在發現經血的顏色或濃稠度与上次不同時紛紛感到擔心,但不管血液是鮮紅、棕色或有血塊都很正常。因為血液有凝聚作用,經血顏色或濃稠度的變化在每個週期或同一週期間有所不同,代表血液離开我们血管後會改變顏色与濃稠度。剛流出的鮮血是紅色、液狀的。當經血呈亮紅液狀時,表示血液快速從子宮流出,尚未凝結。同樣地,棕色、帶有塊狀物的血液就稍微放置了一段時間。倘若妳有大量出血过,會發現血液通常看起來很新鮮,是因為子宮很容易將血液擠出。如果妳是輕微流血,血液可能還留在子宮,同時凝結了一些,不过最後仍會全部排出。不代表血液會累積在妳身体裡。

月經並非不衛生或是危險,它包含了血液及黏膜,所以妳的感受由妳決定。如果妳想,沒有人能夠阻撓妳在生理期時做愛,但一定要記得做好預防措施。目的並不是要避免懷孕或性傳染疾病的感染,而是因為妳正在流血。

既然妳对月經有概念了,妳可能也知道我们為什麼在懷孕的時候不會流血。因為經血包含未來受精卵的新家,子宮內黏膜。在懷孕的時候,我们當然會想留住它,也因此胚胎不會就此流出。這是由一種名為黃体素(progesterone)的荷爾蒙來幫助黏膜著床,妳很快會讀到這個部分。

等等,雖然妳学到了關於月經的一切,但是我们真的需要它吗?也許妳注意到了,其他雌性動物不會每個月流血。許多人認為雌狗在發情期有月經,不过出血位置卻不一樣。

和我们從子宮出血的方式不同,她们在排卵時是從陰道出血,接著受孕。事实上月經非常罕見,只有我们与兩種人猿及其他稀有生物⸺包括蝙蝠才有。

也就是說,繁衍後代不一定要有月經。

实在是毫无意義,為什麼我们每個月還要特地花力氣製造新的子宮內膜,最後眼看它化成血流掉呢?达尔文你說說看啊。

妳大概聽过演化与物競天擇說。在歷代物種的演變裡,具有優勢基因特徵的個体會成功轉移他们的基因給後代,最後這些特徵在後世中占有主要地位,這就是千年以來人類和動物的發展。不像大多哺乳類動物,人類擁有月經;這代表月經本身是对我们有利的特徵吗?生物学家迪娜‧艾瑪拉(Deena Emera)卻不這麼認為。她的理论裡,月經是非適應結果(non-adaptive consequence),而並不具適應優勢(adaptive advantage)。

艾瑪拉認為月經是和我们日常生活裡尚未意识之適應優勢有關的結果:我们稱為自發性黏膜生長(spontaneous mucous membrane growth)6 。妳現在也知道,子宮內膜增生,是為了提供受精卵養分及住所。沒有月經的動物,黏膜只會在受精卵形成後生長。換句话說,母体收到受精卵的求救訊號後建立子宮內膜讓受精卵入住。然而,人類卻有點不同。在我们身体裡,即使沒有受精卵,黏膜仍每個月自行生長,這成為我们的優勢。

因為維持我们不必要的多餘組織必須耗費能量,所以當人類子宮內膜与其他月經物質沒有接收到受精卵後便會排出。於是我们有了月經,或者可以說是黏膜自發生長的結果。而每個月不用排出多餘組織的動物也就沒有月經,牠们只有需要的時候才會製造子宮內膜。

那麼自發性黏膜增生的優勢為何呢?艾瑪拉提出的理论是,母親与胎儿並不只是分享好處而已。事实上,透过我们演化的过程來看,胎儿發展了從母体身上取得更多資源的特徵,便能想像母親与胎儿處於長期競爭關係。從母親發展的角度來看,這些特質讓她得以為了生存保留所需要的資源。

在這個環境下,艾瑪拉提出兩個论點解釋為什麼自發性黏膜生長对人類有益。

首先,子宮內膜的生長保護母親抵抗外來侵入的胎儿;沒錯,妳猜到了:和沒有月經的物種相比,有經期的物種其胎儿非常具有侵略性。這些胎儿肆无忌憚,他们發狂似地生長,像寄生蟲一樣闖入母体只為了將能量与養分弄到手。

人類已經事先製造黏膜,因此看起來有強大保護力能夠对付入侵的胎儿。

妳可以想成是母親為了更有效控制胎儿可以取得与替自己所保留的資源而準備好的防護所。

另個论點為母親能夠意识到受精卵附著在完成黏膜上的狀態。妳會在這本書後面讀到更多所有從受精卵發展至嬰儿的內容。因為基因上的問題,許多胎儿在初期會出現自發性流產的情形。浪費力氣撫育不能長大的胎儿对母親而言簡直愚蠢至極。如果她能透过子宮內膜察覺,就能藉由早期排出有缺陷的胎儿來保留珍貴的力氣。

月經的優勢不在於它本身,而是自發性黏膜生長造成的結果。黏膜的生長,不是我们每個月需要的東西,其实是為了懷孕所建立的地方。許多人認為月經很重要,有月經是健康的,但事实並非如此。我们扣除每個月黏膜生長的理由後,就不再需要月經。它只是結果,而且經血本身不代表健康,只是每個月所流失的血液罷了。

記者隆恩‧法蘭克(Lone Frank)在文章中就艾瑪拉的研究提出現代人和我们萬年前祖先的月經發展相去甚遠。現代女性一生約有五百次生理週期,遠古時期的女性可能只有一百次。為什麼呢?因為她们大部分的時間都在懷孕或哺乳,也缺乏可靠的避孕方式。

比起不生兩個以上的孩子,藉由避孕來阻隔月經对我们而言正常不过。我们現在能夠選擇生孩子与否,也能控制生育數。对現代女性而言月經並沒有內涵的生理價值。

關於月經有許多的迷思,尤其有很多月經決定妳能做或不能做什麼事的话題。

月經对妳及日常生活来说到底是什麼呢?有妳应该要避开的事情吗?舉例来说,妳的瑜伽老師在妳流最多血的時候建議妳不要做頭倒式是对的吗?

我们向瑜伽老師詢問了為何妳不应该在生理期做頭倒式動作。「血液流回肚子是不好的。」老師這麼回答,就某方面来说他是对的。生理期有少量的經血從輸卵管流回肚子似乎是不正常的情況。許多緊張的外科医師替生理期女性开刀時會發現腹部有血卻沒有任何傷口。經血流到腹部其实不太危險,因為妳的身体很快就能處理一切。

許多人相信特定活動,如倒立,會讓妳流更多的血,這也是不正确的。月經是子宮內膜的排出物,倒立、性行為或到處跑都不會讓子宮內膜增加或減少。

生理期內,唯一流出來的只有子宮內壁,而內壁的厚度与最後排出的量可能有時不一樣。

除非因為一些疼痛对妳帶來困擾的特定活動,其实在生理期妳想做什麼都可以。倒立、跑馬拉松、游泳或發生性行為⸺由妳決定,甚至有些女性認為運動能夠舒緩經痛。

但是發生性行為真的不會流更多血吗?當我们在奧斯陸的咖啡店寫到這個章節時,想起女性朋友对我们訴說充滿既戲劇又痛苦難忘,讓她们措手不及的流血故事。她们在流最多血的一次生理期中躺在男伴的懷裡。一個女生,躺在血泊中,被受到驚嚇,不知道她是死是活的愛人叫醒。哈囉!哈囉囉囉囉!?我应该叫救護車吗?他的家裡發生慘案⸺連同他的床單,本來還是白色的。而另一個女生,則是在过程當中突然流血,讓人想起屠宰場或是一九七二年的砍殺電影場景。到底發生什麼事了?我们必須探討一下。

這些瘋狂流血的現象无從解釋,不过妳对身体運作方式稍微了解一點的话,下面的理论或許會合理一點。

第一個理论是我们所謂的痙攣,子宮的肌肉收縮會釋出經血,但痙攣除了月經之外還有可能是其他因素造成。有時候子宮痙攣也不是坏事,我们在這裡指的是性高潮,整個性器官包含子宮在內有節奏性地收縮。高潮可能是造成月經湧出的触發點。

第二個理论是荷爾蒙。當發生性關係,身体會釋放一種催產素(oxytocin)的愉悅荷爾蒙。催產素在身体許多过程中擔任重要的角色,其中,它和触發女性生產有關。催產素刺激收縮,所以是非常重要的東西。如果高潮不足,催產素也能使子宮收縮,因而出血。

第三個可能的解釋為有些經血累積在陰道內,只會在性行為發生時從「水門」洩洪。也許妳還記得,陰道有許多皺褶,所以能夠將血液聚集。此外,當妳放鬆的時候,陰道並非敞开而是呈現前後壁互相擠壓的狀態。

另一個自一九七〇年代起便流傳的有趣迷思是長期生活在同一個屋簷下的女性月經會同步一致。我们身体可能會有心電感應的力量讓我们对痙攣及对巧克力的渴望產生共鳴。這是哈佛心理学家研究同住宿舍的美國女大学生月經週期的情況所得到的论點。演化生物学者抓住這點不放,認為女性同時來月經与排卵有一個好處:男性不會引起劈腿的念頭,反而形成穩定的情侶關係。顯然有80%的女性相信月經同步的迷思。

无论聽起來多可愛,有更多研究顯示我们确实如此。研究指出,住在宿舍的女同性伴侶、中國女子及住在「月經小屋」的西非女性並无共時性的結果。雖然我们看似同時發生,但這是因為女性的生理週期長度變化相當大。如果妳和摯友同時來月經的话,大概只是巧合,必須遺憾地說,不是妳们之間有特別的連結。





* * *



6	這是艾瑪拉在研究中对自發蛻膜化(spontaneous decidualisation)一詞的簡易說法,蛻膜化的过程並非只有黏膜增生而已。





不要在沙發上流血!關於衛生棉、棉條与月經杯




只要妳還在使用衛生產品,每個月的生理期就不會妨礙妳喜歡或想要做的事情。如果妳有用東西抑制血流的话,經血流在朋友沙發上的風險也會明顯降低。

最常見的衛生用品屬於拋棄式的衛生棉与棉條。近年來,月經杯在眾多女性的喜好度中向前攀升。其中的考量有很多,包含經濟、環境与舒適度,完全由妳決定,這是關係著妳喜好及状况的問題。

自從我们悄悄離开文明的發源地後,女性至今已使用过不同種類的衛生棉。有個關於衛生棉古老(又有趣)的說法出自於人類所知第一名女數学家的故事。公元四百年的希臘人希帕提亞(Hypatia),據說厭惡咄咄逼人的追求者因而朝向对方丟擲沾血的布巾,最後有沒有驅趕成功无從得知。

現代衛生棉的底部有自黏貼條,以便附著在妳的內褲上並吸收從陰道流出的經血。有許多不同尺寸的衛生棉,從小型的護墊到大型、柔軟的夜用衛生棉。和棉條相比,衛生棉的好處在於不會有陰道細菌滋生的風險。建議使用衛生棉是因為陰道受到感染的風險特別高⸺也就是子宮口張得更开,例如使用避孕環、墮胎或是生產後的情形更容易讓細菌進入。

棉條是生理期裡放進陰道的小型、子彈狀的吸收材質。使用陰道內月經防護產品的好處是可以更輕鬆運動,尤其是游泳的時候。雖然這個從法語而來的字,「tampon」,有塞子的意思,但它不能讓血液保留在陰道裡,而是吸收經血將它们聚集起來。棉條絕非是最新的發明,但它的材質也並非總是用單獨的塑料包裝。想想以前古埃及女性使用軟紙莎草放入陰道當作月經防護品。

現在,棉條分成有无導管的種類還有不同的尺寸,妳可以根據流量選擇大小。然而使用大型棉條來免除經常更換的麻煩是毫无意義的:棉條应该要定期更換,一般建議的時間為三到八小時之間。避免細菌滋生,更換棉條前徹底洗手很重要。

多年來我们聽过眾多棉條的故事,其中經典的一則与放入兩個棉條或將棉條「遺忘」在陰道內有關。救命啊,許多人會這麼想,棉條要消失在身体了。棉條自行往胃裡移動的想法就和不當配戴隱形眼鏡的话會流竄至大腦的迷思一樣。就妳現在所知,陰道是近似封閉的管道。經过子宮頸及子宮的通道非常狹窄,即使最小的棉條也不可能進入子宮。通往子宮的子宮頸不是开放式氣室,所以任何東西无法進入陰道並消失在胃裡。奇怪的是,陰道內的裂縫是能夠藏住東西的,也因此棉條附上繩子好讓妳再次將它拉出來。

如果妳擔心棉條會在陰道裡不見的话,妳可以試著將它推出。就像要大便一樣蹲下然後試著擠出,用妳的手指找到它的位置。因為陰道不到七公分長,所以通常都能自行掏出棉條。如果沒辦法找到的话,妳需要去看家医科,越快越好。任何進入陰道的東西都必須要出來才行。妳認為妳是第一個因為這個問題去找女医生的话,那妳可以放心了。

月經杯是專門聚集經血而非吸收的衛生產品,材質為柔軟矽膠量杯,將它折疊並放入陰道中。置入後,杯子自動展开後口端朝向子宮頸底部,讓血液聚集於此。杯緣与陰道壁緊密貼合,維持在一定位置。由於月經杯並非拋棄式產品,所以衛生格外重要。必須清空、沖洗,最好每十二小時至少以溫和陰道清洗劑沖洗一次。每次生理期,建議熱水煮沸月經杯來消滅所有細菌。

月經杯的主要優點在於妳可以使用比棉條更久的時間。同時,因為月經杯放置在陰道內,所以非常適合在運動及游泳時使用。妳可以用好多年,最多十年,長期下來是個便宜又環保的選擇。而且一個月經杯能夠取代最後落入垃圾掩埋場的好幾千個棉條与衛生棉。

此外,妳一定看过使用棉條的一些警語。每個棉條包裝盒內都有一本小冊子提醒可能發生的可怕疾病⸺中毒性休克症候群(toxic shock syndrome,TSS),又稱棉條病。使用棉條真的會得到嚴重的疾病吗?

中毒性休克症候群是一種攻擊整個身体的細菌感染疾病。使用棉條就是得到TSS的風險之一,因為陰道內溫暖、吸血的棉條是細菌的溫床。置入棉條時忽略自身清潔,同時久未取出,妳就有可能遭遇如此不幸。這也是棉條放置時間最好不要超过八小時的原因。細菌孳生並進入身体需要時間,如果妳忘記棉條還在陰道內的话,這樣的情況會大幅提升。妥善使用棉條不會造成危險。

倘若得到TSS,妳會知道渾身不对勁。可能會有高燒、起疹、喉嚨痛、嘔吐、腹瀉及暈眩的症狀,妳會覺得非常難受。順帶一提,妳应该時常注意任何嚴重及突發的病症。妳覺得自己得到TSS的话,必須趕緊去看医生,因為感染久而久之變得更嚴重,而且會快速蔓延,最糟的状况還可能致命。

不要因為TSS与棉條有關就認為使用棉條很危險,TSS是嚴重的疾病但是卻非常少見。自從高吸收型的棉條下架後,使用棉條得到TSS的比例大幅下降。現在,只有一半的案例与生理期有關,嚴重創傷及手術後的感染也都也可能引發TSS。換言之,不使用棉條也是有可能造成TSS,同時男性也有可能得到,所以「棉條病」不是最好的代稱。

關於TSS与月經杯,由於這些主題的研究少之又少所以我们懂得並不多,相对来说月經杯是個新現象。目前為止,全球最少只有一個TSS案例与月經杯有關。因此從TSS角度來看,我们還不知道月經杯是否比棉條好。无论如何,最好還是注意衛生清潔。





經前症候群⸺疼痛与謀殺症候群




「怎麼了?⸺月經來了吗?」這是典型的詢問手法。有時候我们難以斷言女性是因為軟弱、荷爾蒙分泌过多或是滿腹牢騷而无視我们。「月經手法」不只是歧視女性的批判字眼,以完全生理学的角度來看同樣也是错误的。為了教育大眾,這些错误必須釐清乾淨。人们如果堅持使用糟糕操控手法对待我们,他们也只有在生理学上的觀念正确。妳是否留意到自己身体因為月經週期而受到心理方面影響最大的時候並不是流血的那幾天,這個症狀实際上是在月經开始前。沒錯,正是廣為人知的⸺經前症候群(PMS)。

PMS或經前症候群(意思等同經前緊張症,premenstrual tension)也許煩人但大致上我们可以与它共存。雖然會造成一些不算太嚴重的問題,但PMS不应该是无視女性的合理原因。女性並非滿腹牢騷、軟弱或是「荷爾蒙分泌过多」,因為我们有月經週期。无论妳認定自己的性別為何,一切可能令人覺得震驚又不專業,不过我们沒有要爭论,畢竟這完全是不同回事。

PMS是所有可能在經期出現的疾病總稱,它幾乎涵蓋所有生理及心理上的症狀:疼痛、易怒、憂鬱、脹氣、情緒波動、哭泣、焦慮与冒痘痘,等多種症狀。同時人们現有的病症會更加嚴重,例如偏頭痛、癲癇或氣喘。這些問題出現在排卵及月經階段的中間,我们稱為經前或黃体期(luteal phase)。當月經來了,在流血的第一天因壓力釋放,同時症狀也消失得无影无蹤。

沒有特別的檢查方式能夠診斷PMS。舉例来说,医生在婦科檢查時不會告訴妳有PMS。這讓診斷變得有點困難。經前的一些症狀不用接受診斷,光憑妳遇到的状况就能知道妳有沒有PMS。大概所有女性在經前有过一些症狀,並不是因為妳有女性的身体才需要診斷。85%至95%的女性在生理期前有輕微類似PMS的症狀。

為了診斷出PMS,那麼症狀必須相當嚴重,嚴重到生理或心理上妨礙到妳的日常生活。當然,這取決於個人对妨礙程度的認定。妳可以預料一些症狀,但能預料的仍然有限。有些女性因為這些症狀完全喪失行為能力,本來不该是這樣的。除了嚴重的症狀,還要有一定的週期,意即妳每個月通常都會有這些症狀。再者,PMS一定有固定的起始及終止時間:必須是經前开始,在月經來的時候結束。大約有20%至30%女性的症狀屬於輕微或是一般的PMS。

情況最嚴重的女性通常會診斷為另一個比PMS還要嚴苛的病症,雖然兩者症狀相同。這個病症稱為經前不悅症(premenstrual dysphoric disorder,PMDD),此類的症狀從完全脫離能夠控制的範圍變為无法忍受的狀態,有3%到8%的女性受到此病症的影響。此外還有另一種症狀叫作經前憂鬱症,有些女性在每次生理期前承受抱有自殺想法的嚴重憂鬱,实在是非常危險。這三個症狀的內容有些相互重疊。

雖然月經會從青春期持續到更年期,但PMS不會存在那麼久。它的症狀通常從二十多歲开始,所以大多數人在問題逐漸出現前有好幾年的時間不會經歷PMS的状况。症狀變得愈來愈嚴重也很正常,許多人到三、四十歲前不會特別寻求医療上的幫助。

我们不知道造成PMS的原因為何,從高度敏感、身体荷爾蒙變化到神經学甚至是文化的說法,在不同理论中都有提出。所有女性在生理週期時會受到荷爾蒙變化的影響,至於為什麼有些人會有PMS或PMDD,而其他人卻沒有這些症狀卻无從得知,或許我们以後會找到原因。

人们大多不用治療PMS,況且最重要的則是避免透过医学方式治療因自然荷爾蒙變化所造成的疾病。我们稍早之前提过,多數人的疾病並沒有生命危險。一般来说,妳可以和PMS共存,面对不舒服的症狀時也有替代的解決方案可以選擇。

遇到問題嚴重的人,其療法會針对個人状况有相當大的變化。比起嚴重疼痛,倘若妳有憂鬱或焦慮的状况,可以採用不同的療法。对一些人来说雌激素避孕能夠幫助她们完全抑制生理期,其他飽受心理疾病所苦的女性使用抗憂鬱藥物也是不錯的方式,容易經痛的族群則可以用止痛藥。

回到那些透过歧視性言论与女性說话的人们。无论你相信什麼,有PMS的女性在經期前會喪失理智或无法作出理性回覆都不是正确觀念。如果面对生理期的女性你決定敷衍回應,不要說:「怎麼了?⸺月經來了吗?」或是「怎麼了?再过幾天月經要來吗?」這聽起來不是很好,但重要的是如果你堅持羞辱別人的话,先正确理解這方面的觀念。





永恆之輪⸺荷爾蒙与生理週期




每個月,大多具有生育力的女性會經歷身体內部、荷爾蒙影響的週期,我们說的正是月經週期。大部分的人知道得很少:在某個時間或是另一顆卵子成熟時,我们如果在正确(或错误)的時間發生性行為就有可能懷孕,而有月經有來就代表我们沒有懷孕。

我们真的需要知道更多吗?我们看过許多医学院学生讀到月經週期的章節就突然把書闔起,所以為什麼妳应该要知道呢?首先,這对妳来说很实用;第二,這是非常驚奇的事情;第三,我们保證把這些內容弄得比妳看过的其他课本都更容易去理解。

如果我们都能了解更多細小訊號物質⸺荷爾蒙在月經週期是如何影響我们的话,便可以更容易知道所有和女性日常生活有關的事物。我们常會收到這些問題:荷爾蒙避孕是怎麼運作的呢?什麼是受孕窗口(fertile window),而且它什麼時候會出現呢?控制我们月經的是什麼?眾多女性疾病又是如何運作的呢?





荷爾蒙⸺駕馭我们的物質




我们在內生殖器官章節內提到的卵巢,它所分泌的雌激素与黃体素為女性的性荷爾蒙,在這一章该是講得更詳細的時候了。

雌激素最近蒙受不公正的惡評,我们聽到的都是血栓、情緒起伏、乳癌及其他可怕症狀的風險。然而雌激素是個奇妙的荷爾蒙,它主要掌管和一般女性特徵有關的部位,胸部、屁股、臀部⸺都和雌激素有關。雌激素維持陰道壁的濕潤及厚度,讓性行為變得愉悅順暢,也能夠讓子宮為懷孕做準備,同時也能抑制汗毛及斑點生長。事实上,跨性別女性會使用雌激素療法將原本男性身体的脂肪分布改為女性的身体。少了男性的大肚子,多了乳房与屁股。小小的荷爾蒙能做到這樣实在很驚人。

如果妳有語感的话,大概可以明白黃体素(progesterone)是什麼東西。「pro」代表為了,而「gestation」表示懷孕,所以意思是「為了懷孕」。每個月當身体準備接收受精卵時,我们需要大量的黃体素。黃体素阻擋子宮收縮及推出潛在的受精卵。再者,它將子宮內膜打造成適合居住的地方,以大量血液和腺体分泌的黏膜給予我们未來的後代營養。

其他兩個荷爾蒙則控制我们的月經週期,它们來自大腦豌豆般的構造,形狀像陰囊一樣,稱為腦垂体(pituitary gland)。身為性作家,我们所見之處都是性器官。

大腦的兩個生殖系統荷爾蒙分別叫作濾泡刺激素(follicle-stimulating hormone,FSH)及黃体成長激素(luteinising hormone,LH)。簡單来说,濾泡刺激素与卵子成熟有關。卵子其实存在於一群細胞中,稱為濾泡(follicle),在濾泡刺激素裡面出現过這個詞。黃体成長激素則是以触發排卵著稱。男性大腦其实也製造一樣的荷爾蒙,不过由於是在女性身体上所展現的功能於是以此命名。因為在世界医学上非常少見,所以我们覺得特別新穎。

到目前為止都還不錯,現在妳必須認识荷爾蒙這個大明星,接下來要來看看它的循環。





月經週期⸺不斷循環的28天!




為了理解月經週期,畫個圓形的時間表會更有幫助。雖然每個女性週期的長度与下次月經的間隔有所不同,為了方便我们就以28天當作範本,而28天也剛好可以分成四週。不过,正常週期的幅度是介於23〜35天之間。





圓狀圖的頂端為新週期的起始,同時也是終止點。在這個地方同時標示象徵週期重新來过的0,与表示經期結束的第28天及上一週期的結束點。一個週期的开始永遠和另一週期的結束同步,妳的月經週期真的是個永恆之輪!

許多人會覺得難以理解,經期的开始跟結束怎麼會在同時發生呢?倘若將月經週期与我们非常熟悉的物品相比的话會易於理解。它和時鐘的運轉完全一模一樣,我们就這樣一天接著一天度过。

當時鐘來到午夜,數位時鐘上一樣顯示24:00,用來表示一天最後一個小時,同時00:00則表示新的开始。時鐘從一天的开始運轉到隔天,在午夜時分之際,妳正身處同一天當中。兩天之中並沒有間隙,而月經週期也是如此。

新週期的开始會特別容易察覺,因為那是妳开始流血的第一天。通常流血的天數會持續一個禮拜,正好是週期的前7天。

為了徹底釐清,月經週期通常分為兩個階段。當妳开始新的月經週期時,妳正處於所謂的濾泡期(follicular phase),這時候容納卵子的濾泡成熟並準備排卵。大約到了第14天,週期的末端开始排卵,來到第二階段,我们稱為黃体期(luteal phase)。整個週期已經过了一半。而接下來直到第28天的兩週內不會有其他明顯的現象。28天後,如妳所知,我们又回到原點,新的週期正式开始。

現在我们稍微把事情變得複雜一點並想像一下妳的週期有30天。這樣的话,排卵日會差不多出現在第16天。什麼?為什麼不是第15天呢?妳可能會這麼問。畢竟30除以2是15。答案是排卵期与下個經期的第一天差不多經过14天,身体在這個時候需要确認是否要开始懷孕。週期多於或少於28天,主要是影響排卵期前的月經來潮長度。如果妳的週期很短,雖然妳不會在月經的第一天排卵,但排卵和經血同時發生的状况确实有可能發生。妳的週期不規律的话,只有在流血的第一天妳才會知道妳不會排卵。

我们現在有基本的概述了,接著可以开始真正有趣的環節:隨著週期起舞的荷爾蒙。我们從圓形圖的頂部开始。月經來潮第一階段的第一天,也就是濾泡期。這個期間的反應不只和子宮有關,同時會在卵巢及大腦裡的陰囊,也就是眾所皆知的腦垂体中作用。當子宮內膜連同集希望於一身的受精卵剝落時,腦垂体开始製造濾泡刺激素(FSH)。大腦永遠不會放棄,即使正在經期仍然會準備新的卵子与下次受孕的機會。回想一下,卵巢內所有的卵子稱為濾泡,受到FSH刺激後才會开始成熟。而成熟的濾泡正是第一階段的濾泡期的由來。

濾泡接收從大腦來的FSH成長,接著濾泡开始製造雌激素。當濾泡逐漸成長,血液中雌激素的數量开始劇烈增加。濾泡越多,雌激素的數量也就越多。接下來,雌激素影響子宮內膜,讓它增厚。在子宮結束流血後,再次重建內部,完全沒有難过的時間。子宮是個固執的小淘氣,即使每個月會受挫也絕不放棄任何接收受精卵的機會。

濾泡和子宮內膜生長時,我们來到第14天的排卵期与第二階段的过渡期。濾泡改變了形狀,變成膨脹、充滿液体的氣球,像一顆即將爆炸的水球。濾泡分泌許多雌激素,讓雌激素達到最高峰,這正是腦垂体在等候的訊號。

腦垂体开始製造黃体成長素(LH)來回應強烈雌激素的訊號。此時的LH並非少量,而是像火箭一樣突然上升。妳曾經嘗試懷孕的话,应该对LH的急劇上升並不陌生。排卵試紙會從妳的尿液中捕捉上升的LH,所以試紙呈現陽性反應的话,就能知道LH已經开始分泌,而排卵日也即將到來。大量流動的LH靠近濾泡造成它破掉,使卵子破繭而出離开卵巢。不久後,卵子離开卵巢四處飄移到輸卵管的触手前,也就是輸卵管傘,快速附著並与可能在輸卵管上等待的任一精子細胞結合後开始一趟旅程。我们的月經週期經过了一半,同時也确实來到了排卵期。

現在看起來是時候休息一下來回應幾個我们沒在國中生物课裡学到的一些東西,卵子細胞。妳大概還記得勇猛的精子細胞之間如史詩般的戰鬥或競賽,瘋狂地想成為第一個游向等待受精的被動卵子。第一點:卵子並不是站在那邊等待。她沒有在酒吧裡緊張地來回踱步等著精子細胞。卵子是天后,和其他天后一樣,她會為了華麗現身派对而遲到。妳可以在懷孕的章節裡可以讀到更多,想要懷孕的最佳性行為時機是在排卵期前。所以卵子一點也不被動,至少和精子細胞一樣積極。不像精子游向卵子那樣,而是上下擺動与等待的精子接触。他们通常會等她等上好幾天⋯⋯

第二點:卵子細胞之間同樣也進行著史詩般的戰鬥,不过基於一些原因,我们不在学校课程裡提及這件事。濾泡刺激素每個月並非只影響一個卵子濾泡。就妳現在所知,每個月大約有一千顆濾泡开始長大与成熟,但這麼多之中只有一個會破掉並釋放卵子。其他卵子不會有任何与精子細胞相遇的機會接著萎縮死亡。妳大概覺得一千顆濾泡不像精子一樣競爭激烈⸺畢竟,它们是數百萬個競爭者相互競爭。記得一點,男性每天製造上百萬精子細胞,而女性的卵子則在出生時擁有一定的數量,然後被消耗殆盡。

為什麼我们在完全沒有对照真实的状况下就自然而然認為(來自女性的)卵子細胞被動,(來自男性的)精子細胞則是主動的呢?只是隨口問問⋯⋯

回到月經週期。我们處於第二階段,也就是第15〜28天,或稱為黃体期。由於濾泡內所有雌激素的關係,卵子釋出而子宮內膜增厚得恰到好處。在第二階段,黃体素為主要的荷爾蒙,反之在第一階段,則是促使子宮內膜生長的雌激素。黃体素的分泌來自破掉的濾泡,而剩餘的濾泡改變外形及顏色,變成一小團的黃体(corpus luteum),意即拉丁文裡的黃色身体,因此以黃色來命名,有時候就是那麼簡單。

更早之前我们提到黃体素代表「為了懷孕」,所以身体現在準備接收結合的卵子和精子細胞。黃体素避免子宮收縮及排出子宮內膜,也确保子宮內膜非常適合受精卵居住。

同一時間,腦垂体抑制分泌FSH或LH,也就是製造和成長新卵子的荷爾蒙。畢竟我们等著即將到來的受精卵,所以並不需要新的成熟卵子。黃体裡的黃体素正是阻止腦垂体做這件事。

(对黃体来说)不幸的是,月經週期的第二階段通常會以悲壯的自殺收尾,妳待會就會知道。誠如我们之前說过的,黃体內的黃体素阻止腦垂体製造任何的FSH与LH,但問題是黃体需要這兩種荷爾蒙才得以生存。換言之,黃体阻撓了自身的救生圈,而只有在受精卵著床的時候才能獲救。因此黃体常常為了維持潛在受精卵生存的利他行為而淪落為犧牲者。沒有受精現象,黃体和黃体素會逐漸消失、死亡。

黃体消失後,再也沒有任何的黃体素抑制腦垂体最會做的事:分泌荷爾蒙。血液中的FSH和LH濃度再次上升,卵巢裡的濾泡再次活動,準備新的成熟機會、破掉並且讓卵子与被選中的精子細胞結合。少了黃体素,再也沒有東西能維持厚实的子宮內膜或預防子宮收縮。我们知道接下來出來的會是什麼:那就是月經。來到了流血的第一天,我们再次回到圓形圖的頂端。週期已經結束,緊接著新的週期又準備开始。





什麼時候才會真的懷孕?




性愛被認為是女性為了懷孕而做的事,但除此之外,仍舊存在許多不确定性。「如果她懷孕了该怎麼辦?」在实境節目《樂園飯店》(Paradise Hotel)的其中一集裡,眾人在早餐桌上熱烈討论其中兩名參与者發生无防護措施的性行為。有些人堅信女方才剛來生理期所以一定沒事,而其他人則認為女性在經期後最容易受孕。為了打破眾人的疑問,最後由節目單位TV3電視台出資讓參賽者緊急避孕來解決問題。懷孕這件事並沒有那麼簡單。

懷孕是女性人生裡的分水嶺。我们的反應從驚恐、耗費許多腦力在如何避免懷孕,到由衷祈禱不會太快懷孕都有可能。最坏及最好的情況也都有可能發生,取決於我们所處於的人生階段与在一起的对象。為有這些想法的人寫出關於懷孕的章節看起來很了不起,但這一切其实非常單純。關於如何懷孕的知识对想要避免懷孕或希望懷孕的妳来说才是最佳處方,那麼该怎麼服用呢?

我们就從已知的事实开始。妳不會因為肛交、口交或坐在有精子的馬桶上(唉呃!)而懷孕。妳必須透过陰道性交的方式才行,也就是在陰道裡發生的性行為。接下來事情變得有點複雜。

當男性高潮時,會射出上百萬條精子細胞進到女性陰道中。大多數精子在不久後死亡,它们主要經由性交後射出或游進陰道深處。只有非常少數的精子細胞才能找到子宮頸口,完全是時機的問題。

事实上,大部分的時間,子宮頸是由身体自然製造的高濃度厚实膠狀黏液荷爾蒙,黃体素所封住。只有在生理期接近排卵的時候才能讓黏液分解,並打开与子宮腔室之間的通道。排卵期的前幾天,妳可能從彈性絲狀黏膜分泌物的變化注意到這個現象!這種和蛋白相似的黏膜,如果妳想要嘗試,可以用兩隻手指將它延伸至令人訝異的長度。

當排卵期接近,黃体素濃度降低,身体开始製造更多雌激素。雌激素造成子宮頸口張开,取代膠狀黏液製造出水狀液体,讓精子細胞能夠游進子宮。同樣的,妳可以透过水狀、乳白色的分泌物來觀察。這是妳在排卵、最能夠受孕時的情況。

現在假設妳在排卵期時發生无防護措施的陰道性行為,而子宮頸為开放的狀態。有一小群上百個精子細胞試著找到通往妳子宮的路,它们會花上2〜7小時的時間從子宮移動到其中一條輸卵管,經由子宮細微且有規律的動作幫助,加上製造波動的輸卵管,讓它们可以乘浪前進。它们所選的方向非常重要,因為卵巢一次只會排出一顆卵子。到達輸卵管後,精子細胞稍作休息並等待可能現身的卵子⸺畢竟,妳知道的,卵子很明顯是整場派对的天后,她會讓精子細胞等待著。雖然性交後的5〜7天仍然找得到活体精子,然而通常位於子宮或輸卵管的精子只能存活約48小時。是個有耐心的傢伙啊,精子!

排卵後,卵子會沿著輸卵管朝向等候的精子跳下去。一個精子細胞与一個卵子在輸卵管結合後受精現象就此發生,並共同創造胎儿的前体,意即受精卵(zygote)。有時候排卵期出現了兩個卵子,那麼妳可能就有懷了雙卵雙胞胎的状况。通常在女性老年時期及先天遺傳更容易發生,所以有些家庭會有許多組雙胞胎。在罕見的案例中,也有單卵雙胞胎的情況。雙胞胎的形成發生在与精子結合後受精卵快速分裂為兩個單獨個体的時候。

受精後的第一天,受精卵依然沿著其中一條輸卵管裡漂流著,而接下來細胞即將开始分裂。即使如此,仍无法保證妳將要懷孕。确保順利懷孕,生長的細胞必須找到下面子宮的方向並立即附著在充滿黏膜的子宮壁上。此外,身体必須透过一種荷爾蒙,人類絨毛膜促性腺激素(human chorionic gondadotropin,hCG)來接收子宮的訊號告知細胞團著床完畢⸺和驗尿懷孕的指標荷爾蒙一樣。這個荷爾蒙确保我们在前面部分提到的黃体存活並繼續製造黃体素。如果沒有經过此过程,受精卵會於下一個經期在妳尚未發覺的時候排出。

受精卵細胞附著於子宮內膜需要7〜10天的時間,只有在這個時候妳才能确定懷孕。接下來九個月的龐大旅程我们選擇略而不談,畢竟市面上有許多懷孕的書籍供妳閱讀。

回到那对《樂園旅館》的情侶,女方會因為在生理期當中就可能要懷孕了吗?在試著懷孕的伴侶研究當中,只有在接近排卵期的六天空窗期中進行性行為的人懷孕,也就是排卵期前五天加上排卵日當天。在排卵期前或排卵日當天發生性行為的伴侶只有30%的機會懷孕,排卵期的五天以前則是10%。

即使在排卵日前好幾天進行性行為而懷孕的人也是有很多。我们稍早之前提到,精子細胞死亡前能在女性体內存活至少一週,理论上,妳在排卵期前7天的經期及後一天是能夠受孕的,也就是總共有8天。換句话說,我们有8天的受孕窗口(fertile window)時間。我们大多不太注意自己的排卵期,所以要知道《樂園飯店》的參与者是否在危險期的關鍵就要測量她的月經週期並了解為期多久才行。

我们在月經週期的部分提过,排卵期通常都在下個經期的前14天。如果妳是完全規律的28天週期,那麼排卵期會在週期的中間,第14天,或是最後一次來月經後的兩週。我们已經知道有8天的受孕窗口,代表妳有可能在週期的第8〜15天懷孕。

我们就假定《樂園旅館》的參賽者有規律的28天週期和7天的經期,也就是整個週期的第1至第7天。這表示她只有一天,一天的時間在經期結束後有懷孕的機會。五天後月經來了,就有很大的機會确定懷孕。

在這種週期中,正在經期中的她發生毫无防護措施的性行為絕对不安全。然而等待下次經期的前一週,在第21〜28天發生就會安全。对於沒有《樂園旅館》寶寶,我们可以感謝緊急避孕措施或是純粹的好運氣。

如果每個週期只有8天能夠受孕,那麼找到安全期好像聽起來很簡單一樣,問題是只有非常少數的女性有完全穩定的週期。妳大概注意到自己也是一樣不穩定,因為妳无法事先得知這個月比正常排卵時間早或晚,所以妳必須延長受孕窗口的天數。如果排卵日落在前後兩天而已,那麼危險期就會延長至12天,許多女性變化的幅度比這個大了許多。此外,如果你是不喜歡在女方生理期時發生性行為的好人,等個幾天你就能進行沒有任何避孕的性行為而且還能有自信地認定不會懷孕。換言之,採取避孕措施永遠是明智的選擇。





性愛二三事





自古以來人類所只擁有的一個共通點,那就是性。无论是自己或与其他人,我们大多都曾有或將會發生性關係。少了性,地球上再也不會有人類,我们認為人類也會过著无趣的生活。性是我们能做的最原始的事情之一,即使不同形式的性(同性或異性)和其他動物都是一樣的。

不同的地方在於人類是唯一对性深感羞愧的物種。礙於社會規範,行房的時候我们會遮遮掩掩的。如此不坦率的現象表示性一直籠罩在不确定性的烏雲下。我们不知道其他人是怎麼想的,我们不知道自身的慾望是否正常,我们也无法确信自己是否達到標準。弔詭的是,即使是兩人進行的性愛,性仍然是非常寂寞的一件事。

尤其在妳性生活的最开端⸺青春期。

近年來与性有關的書籍包罗萬象,年輕人花上數小時觀看色情作品。色情影片被分享至各個社群媒体,青少年傳送勃起陰莖与乳頭的快照給彼此的伴侶。也許有人會主張我们活在有史以來最开放的性社會當中。

這個現象創造了特別的雙重性。对於性衝動、慾望和身体的啟發及見解,我们有自己專屬的門路,知识僅僅是滑鼠點一下便唾手可得。同時,這種开放並不會使我们充滿自信,正好相反。

事实上我们所呈現的是光彩的一面。我们对性的理想已經逐漸升高,同時心中的不安依舊存在著。當我们喚起性致時還不時地想要遮掩,但社會卻告訴我们任何事都该开誠布公。懸殊的对比讓人不知所措。我们相信這樣的結果會讓許多女性覺得自己的性慾太低,刺激的性愛和高潮太少。

這個社會需要对現实有新的見解。在本書這個章節裡,我们想要探討人们所謂正常的性生活。當然我们使用「正常」一詞,並不代表任何偏離內容所講的就是错误的或应该感到羞恥,只是大多數人不會做的事情而已。性有千種類型,只有妳知道適合自己的方式。我们希望在人们正常看待性的方面能夠盡一份心力並提供一些方法讓妳找到滿意又舒適的性生活。





第一次性行為




人生裡總是有幾個極為神祕的經驗,例如:做那檔事,第一次的性行為。妳对自己或伴侶表現的期許,可能像天一樣高,發生何事实在難以想像。

結果,造成有些人在初体驗時对自己或伴侶感到失望。妳沒有高潮吗?從妳看过的姿勢進入會不會很難?妳男朋友的陰莖十秒後就軟掉了吗?她沒碰到你的陰蒂吗?

勇敢一點!性和人生中的其他事物一樣。妳沒有練習自然不會在行,而你的伴侶也是。必須要記住,第一次不會如此完美,如果妳降低一點期望的话,可能還是會有不錯的經驗,畢竟人都有第一次。我们收集了一些資訊盡可能讓第一次体驗變得更美好。

我们隨著二〇〇四年的電影《赤裸青春》(Bare Bea/Only Bea),看著一群奧斯陸高中一年級的朋友。小碧是裡面唯一一個還沒有性經驗的人。她们團体的一項儀式,是在成為破處的一分子後,要去當地的糕餅店吃一塊杏仁蛋糕。現年十六歲又九個月的小碧,覺得所有的一切取決於她是否能夠上床。糕餅店櫥窗裡的杏仁蛋糕正在呼喚著她。

小碧並不是第一個認為「其他人都做过了」而將自己趕上床去的女性。當這些念頭迸出時,擁有一些相關知识是非常有用的。

挪威女性初次性經驗的平均年齡約為十七歲,但只是平均數值,並非上限。有些人开始的早,也有些人晚了一點。事实上,只有20\%的年輕人在十六歲以前开始性行為。所以有四分之五的人在念高中一年級時還沒有过性經驗。換句话說,小碧根本不急著品嚐那塊蛋糕。

雖然在平均年齡經歷初次性行為也許不錯,但請不要忘記,第一次与妳及妳的伴侶關係密切,妳应该在雙方準備好的時候才能开始。當妳感受到慾望(腦海裡浮現慾望)和受到吸引或是存在於身体裡的性慾被激起時才算萬事俱備。有時候,妳的想法跟身体不一定同步,這樣的话,再等一下也許是不錯的選擇。我们被激起性慾的時間、誰讓我们有這樣的感覺因人而異。有些人在國中就蓄勢待發,部分則是在高中時期,有些人到了二十或三十歲甚至更年長一些才整裝完畢。

沒有一定的標準,不过許多挪威人的第一次是和同年齡的对象發生。有些人和女友或男友;有些人是透过一夜情;也有人是和同伴或女性友人進行性關係。至於地點,有些人是在房間,其他人的话,則是在音樂節的流動廁所後面。只要彼此是真心參与,這一切就沒有所謂的对与錯。

只要記得小荳蔻法則7 :妳和伴侶現在可能性慾高漲並迫切地想要做愛,但是最佳的場合与時機要以不去打擾他人為主才是。例如,搭飛機時坐在正在做愛的情侶隔壁实在很糟。曾經在飛往紐約的航班上經歷过「機震」的艾伦可以作證。這对情侶鸚鵡很明顯來自克里斯提安桑(Kristiansand,挪威南方的一個小鎮),卻故意不講英語或挪威語,实在讓人受不了。尊重一下好吗?

許多人非常在意是否能說出有过性經驗,以及處女所代表的意義。舉例来说,進行一些性行為後還有可能是處女吗?如果肛交过還會是處女吗?那麼口交或指交呢?真正的性行為是什麼?我们還沒有答案,但是我们認為將焦點放在貼上標籤和性的本質相去甚遠。性沒有对錯,无论如何,也沒有「真正」或「不太真实」的性。只有妳才能定義自己的性生活。第一次性經驗可以有多種形式,因為性包含了口交、指交、陰道性交与肛交。少了傳統陰道性交妳也可以有很棒的性經驗。畢竟,認為和男性進行陰道性交之前的女同性戀者都還是處女這個想法实在非常荒謬。

現在的孩子大多对性一知半解。這不应该是性教育该負起的責任,因為他们幾乎都透过色情片的形式看过性愛畫面。儘管(或因為?)如此,許多人擔心他们在第一次發生前是否完美。

第一次性經驗,妳可以預料會發生一連串的状况。无论妳做了什麼,都不會像色情電影一樣。如同其他電影,有些色情片運用特效讓原本所做的事變得不一樣,所以在現实生活中妳不會做出所有在色情片裡看到的事情,即使色情片的場景是根據一些真实的內容改編。這有點像是哈比人電影:真实世界裡有許多座山,卻不代表有龍住在裡面。即使有,也不會是班尼狄克‧康伯拜區(Benedict Cumberbatch)的聲音。

同時也不要忘了色情演員应该要被當作極限運動員才是。他们早就做过所有事情,名副其实。知名滑雪選手琳賽‧沃恩(Lindsey Vonn)從高山滑降看似輕鬆,但如果妳是初次站上滑雪板並試著像她一樣,妳有可能會傷到脖子。

不要期待自己能夠有知名色情演員史托雅(Stoya)一樣的表現。初次嘗試的妳也无法做到慾經(Kama Sutra)裡的高難度姿勢。妳大概也不能完全掌握一切,這沒有關係:妳不需要為了美好的性經驗而去做一堆事情。第一次會非常笨拙,但這就是原本该有的模樣,而且也是魅力的一種。妳大概會覺得礙手礙腳,不过藉由練習會愈來愈好的。

不只降低对自身表現的期許很重要,也請記得不要对你的伴侶太过嚴苛。第一次做愛的時候妳的男伴或女伴不會知道你喜歡什麼,至少他们也和妳一樣緊張。无论如何,結束後最好溝通一下,來個事後檢討。哪些是有用的?會有下一次吗?如果有,妳应该做什麼改變呢?





我该如何達陣呢?




我们只有寫到性的包罗萬象,卻忽略另一群人並过分聚焦於陰道性交上。畢竟,性不單是男女之間的事,雖然在我们異性戀社會當中可能很常見。在挪威,約有十分之一的女性与同性別的对象有过性經驗。不过我们仍然會多提一些第一次陰道性行為的部分。並非因為它是唯一進行的方式,而是因為我们对於這方面的疑問是最多的。

在陰道性交之前,詢問自己以下問題的女生數量令人難以致信。我會流血吗?會痛吗?許多女性擔心自己的陰道太緊。我该如何把那東西放進去呢?我甚至无法放棉條啊!

把大如陰莖的東西放進陰道看起來很嚇人,但是陰道還有很多空間。妳的陰道十分具有彈性,在妳激起性慾時能夠四處擴張。很多人認為尚未有过性行為的女性陰道會比有經驗的人更緊实。妳大概有聽說做过愈多次妳的陰道會變得愈來愈鬆弛,這不是事实。

陰道是有力的肌肉管腔,妳可以控制它的緊实度。无论有多少陰莖或人造假陰莖在陰道內仍然能調節運作。如果妳确实放鬆,會很容易讓陰莖滑入,但如果妳繃緊身子,可能會讓任何東西不得其門而入。就算妳有过好幾次性經驗,仍然可以緊縮陰道讓它變窄。性愛時積極使用陰道肌肉的话,可以調節陰道与陰莖之間的摩擦。試試看就知道了!

很多女生在初体驗前會緊張,受到期許的壓力,這不意外。有點緊張是完全正常的,太过緊張的话,可能會造成一個不愉快的經驗。妳如果太緊張,會容易下意识縮緊陰道肌肉,讓所有東西變得難以進去,甚至會有點痛。

當女性性慾高漲時,生殖器官會分泌更多液体,這個黏液是身体自然的潤滑劑8 。倘若妳非常焦慮,會沒辦法提起性趣讓下体濕潤。就算妳真的決定要做愛也有可能發生這種情形。就某種程度来说,緊張感會阻止身体隨心所欲。

如果太乾燥或是不由自主縮緊陰道,很容易造成陰道壁撕裂並流血。雖然不危險,卻會有不愉悅及刺痛感。第一次的要點就是在於放輕鬆。花點時間在接吻及前戲,這樣會讓妳的肌肉更容易放鬆。給自己一點時間勾起性致,妳才會分泌更多液体。

就算想要做愛,放鬆了,在前戲上多花了一點時間,不管怎麼試,有些女生的下体就是不會變濕。而且,也有些女性沒有性慾高漲仍然濕潤。大腦和生殖器並不是一直都有關聯。好在除了自然產出的陰道黏液外還有其他的選擇,使用口水或是超市、藥局購入的潤滑液也同樣好用。潤滑液能夠改善許多人的性經驗,當妳不知道身体會有什麼反應時,在第一次使用一些會是不錯的想法 。

接下來提到處女膜,陰道薄膜。我们已經花了整個章節談论陰道最狹小的部分,但我们還是會重申幾個重點。妳的處女膜在進行第一次性行為時不一定會流血。不會流血和會流血的機率差不多是一樣的。而且沒有人能夠在完事後透过妳的生殖器來判斷妳是否有过性經驗。妳的陰道不會被薄膜完全覆蓋,也不會有弄破陰道薄膜的情形,它只是一個有彈性的環狀組織。別浪費力氣擔心妳的處女膜,不如花時間擔心像是環境危機、難民問題還有学校性教育不完全的嚴重議題。妳的處女膜不值得賠上留在对方家过夜的機會。





建議与竅門




妳現在知道了不少陰道在性行為當中會發生的變化,那麼实際上妳该怎麼去做呢?第一次和男人(或男孩)上床妳可以怎麼做?從完全技術面的立場來看,我们有兩種建議,但是妳也可以有第三種選擇。畢竟,這是妳的陰道。雖然有不同的方式,不过都是不錯的選擇。

第一種非常傳統,但絕对值得考慮。傳教士姿勢不常在色情片裡使用,因為看不太到性器官(沒有露出性器官的色情片又會變成什麼樣子呢?)在現实中,傳教士是第一次性行為裡最常見的姿勢。妳(女方)會在這個姿勢裡躺著,而男方倚身在妳的兩腿中間,因此你们的胸部及腹部會互相面对面。接著陰莖隨著男方前後擺動進入妳的陰道。從妳的角度來看,這不是個積極的姿勢,不过有許多原因可以解釋它是好的起步點。可以充分進入及掌控彼此身体,還可以繼續接吻,而且,最重要的,是妳能夠觀察彼此的反應,所以妳總是可以得知对方樂在其中。对都很緊張的你们来说,第一次格外重要。如果覺得有太多的眼神交流,妳可以閉上眼睛。

比起主導,有些人更害怕放棄控制權。許多人对於被別人載往高速公路時感到无比驚恐,而想要成為指手畫腳的人。妳是這樣的人吗?如果是,妳會更喜歡掌握主導權,我们讓妳坐在上面。男生會在一开始躺著,而妳會在他的上面,有點像是顛倒版的傳教士姿勢。把妳的膝蓋放在他的臀部兩側,坐在他的陰莖上。如果妳想,也可以將手臂或手掌放在床上支撐自己。妳完全不必像騎馬一樣坐在那邊,即使人们常說這個姿勢像是女方騎著男方或稱作女牛仔姿勢。覺得坐直挺挺感覺很赤裸,妳可以往前傾。妳就喜歡那樣的话,那就騎著。現在妳是動最多的人了。妳自己能夠控制方向、陰莖進入的深度,以及搖動的速度,這就是坐在上面的優點!

當妳坐在上面時就和傳教士姿勢一樣,可以非常清楚地看到彼此的臉。是有一點可怕,不过這個姿勢比較方便溝通。

雖然色情片或好萊塢電影給妳的印象是如此,但不是所有的性愛都以高潮結束,男女都一樣。高潮需要練習,並非是第一次進行性行為的妳或伴侶该期待的事。為了達到高潮,徹底了解自己的身体還有感到安全很重要。也因為如此,有些与穩定交往对象進行性行為的女性更容易高潮。另一個了解自己身体的重要方式則是自慰,对很多人来说,可能需要花上好多年的時間才有高潮:而且通常自己來更容易達成。不过熟能生巧!我们之後再來討论這個部分。

和伴侶溝通也很重要,一定要說出妳要的是什麼,不要指望伴侶替妳解決高潮的問題。親自解決絕对既正常又合理。和伴侶做愛不代表妳就无法同時留意自身的狀態。畢竟,妳可以向伴侶展示當下的行為,而对方也能对喜歡与否作出回應。

性是有趣的,然而如同其他有趣的事物,存在著一定的風險。和安全帶及自行車專用安全帽降低嚴重受傷的風險一樣,避孕措施能夠降低性傳染疾病与懷孕的風險。

避孕絕对是共同的責任。性行為要雙方才能完成的话,那麼也就需要共同處理避孕的事情。然而,不能總是認定妳的伴侶也會預備好。我们对妳的建議是凡事親力而為,這也是我们給所有看到這個章節的男性的建議。倘若妳的伴侶也準備了,那倒是好現象,代表這個人很聰明。

避孕需要計畫,所以在第一次性行為之前要知道如何妥當使用。去向医師或護士諮詢,同時詳讀這本書的避孕章節,任何妳需要知道的內容都在裡面。我们建議採用保險套加上其中一種避孕方式,才能提高避孕的防護效力。目前,現行的避孕措施幾乎為女性使用,不过好消息是,男性專用的方式正在开發中。保險套是唯一避免性傳染疾病的避孕方式。无论如何一定要使用保險套,但是也要照著我们後面的保險套教学來确認不會在性行為中間受傷。為了避免發生意外,準備事後避孕藥也是不錯的方式。妳很快就會学到更多。

如果妳想要做愛也準備好避孕措施,那麼放手去做吧。只有妳才知道準備好了沒。儘管如此,我们对於第一次性行為最重要的建議則是:就把它當做第一次看待。同樣是第一次的人有很多,妳也會變得更熟練,一切都會變得更好。





* * *



7	小荳蔻法則出自於托比約‧艾格納(Thorbjørn Egner)的童書《小豆蔻城裡的人民和強盜》(When the Robbers Came to Cardamom Town)意思約為:不要打擾其他人,要善良、仁慈,此外,不要猶豫去做任何想到的事。



8	這不適用於所有女性。很有可能會是性慾高漲但下体卻不濕潤的情況,反之亦然,沒有感受任何慾望也能夠使下体濕潤。妳可以在談论慾望的段落中讀到更多。





肛交




我们「洞外有洞」的部分結束在讓人心癢癢的地方:布滿肛門周圍及內部的神經末端正等著接受刺激。如果有人讓臀部一起加入狂歡的话,他们會發現肛門擴展了性生活的規模。

好極了⸺我们的肛門有大量的神經末梢,但我们该如何刺激它们呢?你大概認為「讓臀部一起加入狂歡」聽起來太过樂觀。和鞭打及蒙眼相提並论,許多人覺得肛交聽起來既可怕又有點髒。「什麼?我该把東西伸進去那裡吗?我们该在和大便同個洞口上性交吗?」

肛交毫无疑問屬於「性愛高階课程」。妳沒有特別喜歡的话,它也不是必須要做的事情。然而,肛交在異性戀情侶中漸漸變得稀鬆平常。去年幾乎有五分之一的十六歲到二十四歲英國年輕人有过肛交的經驗,其他地區的年輕人和我们也沒有什麼不同。

人们开始肛交,但是他们常常因為错误的理由而做。不幸的是,我们可以看到女性常常被迫進行肛交,而她们都擁有不愉快或疼痛的經驗。肛交是女生必須「学會享受」的概念廣泛流傳,這不该是肛交的用意。肛交应该是自願去做而且本來应该充滿樂趣。如果妳沒有興趣,就不要強求。設好妳的底線。

好奇的话,這個部分就是為妳而寫。有許多女性喜歡肛交,而肛交有不同的形式。肛交一詞涵蓋了所有種類的肛門刺激,可能需要陰莖或人造假陰莖進行插入式性交,也可能透过指交或口交,舔著肛門的周圍,也就是所謂的舔肛。妳不喜歡陰莖在屁股裡並不代表妳无法藉由其他方式在肛門中得到快感。

這個部分所給的建議是与插入式肛交有關,也就是使用手指、陰莖或其他物品。所以有些事情需要在妳开始之前理解。

妳可能還記得之前章節提过的內容,肛門有兩個強壯、相鄰的括約肌:一個是自然運作,而另一個是隨意肌。它们非常实用,表示我们不用經常為了上大號而衝進廁所。括約肌以百褶裙似的皺褶形式保持肛門緊实及隱藏環形肌。

許多人認為肛管与直腸非常狹窄,比陰道更緊实。這可能是吸引男性的魔力之一,但事实卻只有一部分正确。实際上直腸像是汽球一樣:在底部打個結。位於最底部的括約肌強力壓迫腸子末端。這表示它的末端非常狹窄,一旦通过括約肌後就會有很大的空間。另一方面,陰道是從洞口到子宮頸都充滿肌肉的質管。所以陰道可以保持一路狹窄,而直腸卻只有底部狹窄。此外,括約肌並非一直如此狹小。在妳進入沒多久後就會开始放鬆,而直腸也不再那麼狹小了。

氣球上的結則代表肛交有幾個非常特別的挑戰。當妳進行陰道性交時,我们曾提到讓身体放鬆,使骨盆肌肉減少收縮,避免讓性交變得困難。肛門裡的括約肌卻不是這麼運作。就你所知,在完全放鬆的時候,妳的屁股仍然是封閉的狀態。睡覺或是深度冥想時也一直保持緊閉。這是不隨意環狀肌正在運行。妳无法自行藉由放鬆讓肛門口變大,妳能做的只有防止隨意括約肌收縮。妳不能控制不隨意括約肌,但是,就像我们說的,它會因為刺激變得鬆弛。

最重要的一點,則是放輕鬆开始。如果妳的肛門從來沒有東西進入过,就不要從太过粗硬的陰莖或是巨型人工假陰莖开始。括約肌需要時間才能放鬆,一开始要經过可以掌控放鬆程度的隨意肌,接著不隨意括約肌才得以收到提示。試著從手指或是小型情趣用品等小東西开始習慣。大多數人在準備好以前需要許久的時間熱身準備。

如果妳進行得太快,肛門容易造成小型撕裂傷,可能呈現好幾天極度疼痛的狀態。在沒有完全了解的情況下,任何打算往躺平的伴侶陰道進入的人,最後都會找到错误的洞口,這樣會受傷的。如果妳打算肛交,妳必須完全準備好。同時妳的伴侶也必須要有耐心,直接开始是沒有用的。

只要妳一起步,事情會變得更容易。妳的肛門會愈來愈鬆弛⸺這讓很多人感到害怕。結束过後氣球結沒有封起來。「噢不!我的環形肌要永遠下垂了吗?」還差得遠呢,放輕鬆。肌肉會慢慢地再次緊实,只是需要花一點時間。

那倒是真的,括約肌和身体任何部分一樣,都可能會有永久性的傷害,但是妳必須要連續猛搥才能造成。別忘了肛門能容納比一般尺寸的陰莖還大的東西。慢慢开始,小心進行,覺得不对勁就停止:這樣一切都會順利。

肛交的另一個要點是濕潤度。陰道通常在妳性慾高漲時會自己變濕,所以妳必須使用潤滑劑或其他類型的人工液体進行肛交。沒有潤滑劑,任何東西都難以進入,而且如果太乾,會造成許多摩擦。摩擦增加撕裂及輕微流血的風險。

事实上直腸腺体也會分泌一些液体,不过与你性慾高漲与否无關。腸子裡就像陰道与口腔內部一樣,有黏膜存在。黏膜的特徵就是製造黏液:如同口腔裡的口水和陰道裡的分泌物。當直腸黏膜受到如陰莖的刺激,會製造黏液保護自身不受到傷害。因此性交會促使体液分泌,但還是不夠,妳同樣需要潤滑劑。

接著來到了大問題:大便。我们都聽过有女性在肛交時突然大便在伴侶身上的都市傳說。雖然对多數人来说沒有什麼吸引力,卻也无法改變糞便在直腸裡的事实⸺畢竟,這就是腸子的功能。即使妳不覺得自己需要去廁所,糞便會累積在腸子裡直到飽和。直腸是糞便流向外頭世界前的儲藏地,這表示糞便可能會落在陰莖、情趣用品或手指上,如果妳事先沒想过,可能會造成一點驚慌。真的發生的话,也沒有什麼不对,妳也不该感到羞愧。如果妳打算進行和腸子有關的性交,那這就是遊戲的一部分。

然而,的确有方法能降低大便的風險。有些人選擇在藥局購入的少量灌腸劑清理腸子以解決問題。其他人則是在开始前先去廁所一趟。

妳當然不能透过肛交懷孕,但絕对有可能得到性傳染疾病。許多人忘記這一點,或認為他们不太可能有肛門感染的情況。事实上,完全相反。部分性傳染疾病更容易透过肛交傳遞。如果妳和新伴侶發生性行為,直到对方去檢查前,保險套的使用就變得无比重要。无论什麼類型的性交都一樣。

如妳所知,妳和伴侶做过性病檢查後可以不用保險套進行陰道性交,但是屁股包含了腸道菌群,因此衛生非常重要!妳不會想要讓腸道細菌進入陰道或是尿道這些不屬於它们的地方,因為最後有可能造成感染。這一樣會發生在男性身上。所以從肛交換成陰道性交時要特別留意,透过手指或陰莖也是。如果妳想接著進行陰道性交的话,那麼在肛交時使用保險套,之後再拿掉是不錯的方法。記得也要清洗使用在肛門裡的情趣用品。

順帶一提,有些玩具是為肛門設計,通常在底部會有一個插塞以防落入直腸裡。因為陰道長度不會大於7〜10公分而且頂端是封閉的,所以沒有任何東西能夠消失在裡面。但是,腸道沒有盡頭。因為玩具卡在裡面而去急診室拿掉的感覺实在糟糕,卻也真的發生过。医生们透过互相分享掏出病患屁股裡最奇怪的物品獲得極大的娛樂:厚实的蠟燭、玩具車、iPod或是瓶子。医生想必也感受到他们的樂趣。

這些确認事項是給想要嘗試肛交的人。正确地進行,就能讓女方与男方感到驚奇,但是前提是女性必須停止肛交為了男性而做的想法。肛交,和其他形式性交一樣,必須你情我願才能進行。





完全正常的性生活




當二〇一五年影集《女孩我最大》(Girls)占據我们的電視螢幕造成轟動時,許多人因為終於看到普通女性進行一般性行為而將此描述為革命性的突破⸺不管裡面的劇情為何。除了許多在廚房角落發生的高潮与火辣性愛以外,我们看到了笨拙、冷場以及不斷嘗試以性感內衣現身在男友家裡失敗的場景。這些呈現了主流文化下的女性費盡心思去实踐理想的性生活,在《Elle》雜誌最新文章裡的淫聲浪語与拍打屁股看起來很性感,然而在亞當与莉娜於現实生活中嘗試時,卻變得像是讓人尷尬的電視節目。《女孩我最大》呈現了理想与現实之間的衝突。

《女孩我最大》反應了性成為公共財的事实。幾杯紅酒下肚後人们高談闊论朋友性生活的最私密細節。女性已經掌握性的自主權。性慾高漲是好事,明白自己想要的也是好事。对那些实現的人来说,再好不过了。

不幸的是,对於性生活应该要如何的期許成了精神包袱的一部分。我们的性生活變成另一個应该要去展現自我的平台。只有在女性好友之間的私密对话中才能提出更害羞的問題:每隔一週才做愛這樣正常吗?妳平常做愛都會替对方口交吗?如果我在做愛時只能靠撫摸自己來高潮會不正常吗?





因為,正常性生活的組成到底為何?諸如此類的疑問使我们对正常的性愛進行探索。

當人们評估自身性生活時,通常做愛次數會是与他人相比最簡單的指標。雖然數量是如此主觀,卻容易計算。如果妳問異性戀的人通常多久做一次,妳會在大部分西方世界裡得到相同的答案:異性戀情侶每週會有1或2次。而同居伴侶比已婚伴侶更多次,單身人士則是最少。我们对同性戀男性及女性知道的甚少,但是有些數據顯示女同性戀伴侶的性愛次數和異性戀差不多。

挪威人非常与眾不同。在一項介於二十三歲至六十七歲伴侶的研究指出,大約40%的人上個月每週做愛次數會有1至2次。只有10%激烈的族群,一週會有3、4甚至更多次,和完全沒有做的比例一樣多。剩下的人則是每隔週1次或更少。

這項研究中,令人意外的是,不同年齡層性愛頻率沒有很大的差異。只有年滿五十歲伴侶的頻率較低,儘管如此,其中還是有40%的人每週有1、2次甚至更多。不过,我们從一系列的研究得知年齡是情侶做愛頻率的重要因素之一。這是因為身体性功能劣化与年紀有關。性慾降低,男性有勃起問題而女性可能會發現雌激素降低,造成陰道黏膜薄弱、易損,性愛过後會變得更不舒服。然而,除了年齡以外還有其他影響我们做愛頻率的因素,其中一個則是相愛程度。

新關係的第一階段彷彿像在泡泡當中。大腦充滿神經傳導介質,傳遞快樂、滿足与慾望。沉浸在相愛的感覺,妳會忘記任何你们兩個以外的事物。性變得比睡、吃与朋友重要,同時也成為彼此的共通語言來傳達所有妳不敢說出的话:現在只有你和我⸺只有這件事最要緊。

性每天總是有辦法在最後悄悄接近妳身旁。某天晚上,妳盯著時鐘,發現此時有一隻迫切的手緩緩地伸入妳的彈性內褲裡。「我们可以稍微愛撫一下就好吗?我必須要早點起床。」妳帶著歉意的笑容回答。如果妳突然不再日日夜夜想要做愛的话代表妳的關係出了什麼問題吗?還是這只是自然的發展呢?

德國对1900名二十多歲穩定交往中的学生進行研究,發現情侶交往的時間与做愛的頻率有明顯的關聯。平均来说,熱戀中的情侶做愛的頻率為一個月10次,或是一週2.5次。70%的人則是一個月多於7次。交往第一年过後,頻率开始下降。當達到一至三年的時間,一週有2次以上的人數不到一半。五年之後,跌入谷底。此時性交頻率減半,一個月為5至10次不等。這些結果在其他女同性伴侶間的研究裡也能看到。

換句话說,不是只有妳一個認為自己比以往更少做愛。所以到底是怎麼一回事?這項德國研究得到一些有趣的結果。在關係初期,女性与男性对性渴望程度相同,也同樣擁有对親密与近距離接触的慾望。接著奇怪的事發生了。男性在三年过後仍然性慾高漲,但研究顯示女性在交往第一年後的性渴望急劇降低。第一年裡,有四分之三的女性承認想要經常做愛。三年过後比例降至四分之一。比起交往初期有兩倍的人,達到9%至17%的比例,表示他们經常感受不到对性的慾望。

另一項調查為交往中的男女在想要做愛時被拒絕的比例。先前提到的挪威研究裡,有半數的男性表示有時候會遭到拒絕,10%的人則是經常被拒絕。而女性的比例剛好相反,90%的女性表示她们從未或是甚少被男性拒絕。

隨著交往時間的增長,有項指標卻沒有降低,就是女性对親密与近距離接触的需求。然而对男性来说,擁抱的慾望則是隨著時間降低。或許世俗的認知比我们所想的還更加真实:女性想要擁抱,而男性想要做愛。為什麼?我们无從而知。德國研究的參与人員認為最佳的解釋則是人類演化。女性下意识將性當作綁住男性的手段,一旦目的達成,男性中計後便會喪失興趣。其他人則認為答案存在於生物的性驅動程度不同(我们會在之後提到性驅動的影響有多遠)。而他们甚至指出社會上有著所謂的性「劇本」,記載著男女应该要如何扮演各自的角色。人们認為性慾是男子氣概的展現,而女性表達相同程度的慾望則有失女子風範。這可能造成女性更容易比男性進入沒有性慾的狀態,但也可能會使男性因為失去性趣而羞愧的程度上升。

到目前為止,我们已經知道妳交往得越久,做愛的次數就會越少。同時,我们也知道最幸福的情侶是擁有做愛次數最多的族群。可以感到欣慰的是,幸福程度還是有一定的極限。

一項樣本數為三萬人的加拿大研究發現,每週做愛一次的人幸福程度並不會上升。因此人類是有可能找到屬於自己每週幾次的黃金比例!

那麼除了頻率之外哪個面向還會影響我们对性生活的滿意与否呢?再次強調,答案顯而易見:關係的品質。我们对關係的滿意与性生活的品質有著密不可分的關係。簡單来说,好的性生活就是好關係。我们不知道是好的性生活讓我们对關係滿意,還是好的關係創造好的性生活,我想大概都有吧?

良好的關係与溝通極為有關。在性与情感方面妳必須与对方互相討论。喔,天啊,也太瞎了吧?

為什麼我们一定要談到性?這不就活生生證明妳的性生活完蛋了吗?一夜情与新關係最迷人的地方完全在於不用对话。人们比起忘記保險套更害怕溝通破坏了氣氛。簡單的談话对神祕与刺激感造成分崩離析的威脅。

顯而易見的是,能夠藉由談论想法、需求与期許來達到情感上親密的情侶,長期下來更會滿意他们的關係与性生活。此外,聊性的情侶不止更滿意彼此,做愛的次數甚至也會增加。

情感關係上有很多事情能夠摧毀对性的慾望:壓力、缺乏相互分享的寶貴時光、沒有做愛的想法、負面的自我形象与身体意识不足。如果妳覺得和伴侶的性需求不同,妳可能很快會進入其中一人永遠掌握主權而另一人常常拒絕的惡性循環。拒絕別人一點也不有趣。妳有罪惡感是因為妳无法活在他人的期許當中,妳可能开始焦慮对方最終因為厭倦而離开。妳擔心得愈多,对性的慾望就會愈少。最後,妳害怕对方期待更多甚至避免了單純的擁抱或接吻。

這通常是情侶停止正常性交的潛在危機。認為自己能夠不透过相互溝通來擺脫困境的想法实在天真。如果有更多的情侶察覺到情況不对勁便勇於溝通的话,他们之間就能避免許多的問題。所以和伴侶坐下來,放下手機來場徹底的談话吧。搞不好妳可以得到更多、更好的性生活。

妳現在大概在想次數不是一切,我们也絕对認同。一週做愛兩次固然不錯,但內容才是重點。人们通常都進行哪一類的性交呢?畢竟性可以涵蓋許多種類。它包含了吸与舔的動作,陰道或是肛門的性交;人们會不會達到高潮,也可能在雙人床、沙發或是飯店電梯內做愛。对某些人来说,固定的做愛生活是他们的死对頭。他们懷念刺激感、單身生活的不可預測性,或是交往初期的感覺。

二〇〇六年澳洲的一項研究調查一萬九千人最近一次做愛的組合。得到的答案是有12%的人只進行陰道性交,50%的人在陰道性交時會透过雙手刺激彼此的性器官,而約30%的人也會進行口交。研究發現透过雙手与舌頭的次數愈多,女性更有可能達到高潮,這樣的結果讓人不足為奇。

良好性生活的概念伴隨許多的期望。現实是正常的性生活,真的,非常普通。只有非常少人像兔子一樣喜歡頻繁做愛。人们在熱戀感消失及日常生活和他们的性生活變得密不可分後开始有點厭倦。少數人在做愛時會替对方口交,不过大部分都還是非常滿意。如果妳想要讓一切變得更好,那麼只要去做一件事:彼此溝通。





慾望的消失




激起性慾不再是禁忌。甚至在年輕女性當中性慾是理想的典範。這個完美的概念包括享受性愛、主導性愛及实驗性愛。如果妳的慾望消失或者從一开始就沒有出現过的话我们该怎麼辦呢?這可能讓人们留下可怕的拒絕感。

二〇一五年冬天,妮娜非常榮幸与特別迷人的奶奶見面。當時一百歲的雪莉‧祖斯曼(Shirley Zussman),是名略為駝背,有著豐唇和水汪汪大眼的女士。她可以說是性革命的先驅。她曾与「探索」女性高潮為名,HBO影集《性愛大師》(Masters of Sex)的原型威廉‧麥斯特(William Masters)与維吉尼亞‧E‧強生(Virginia E. Johnson)一同研究。自一九六〇年起,祖斯曼在紐約以一名性治療家從業。

五十年後,她仍然在紐約上東區的辦公室替患者治療。辦公室裡布滿了花卉,而書架上裝飾了不同性姿勢的木雕。這些擺飾一直以來在性問題發展方面給她獨特的觀點:「以前,我的患者曾經來找我求助和高潮有關問題,像是早洩或是達不到高潮的状况,可是現在就只有失去激情上的困擾。」她說道。比起一九六〇年代,人们絕对擁有更好的性生活,祖斯曼表示,无法引起人们做愛的興致才是問題。她將原因歸咎於科技与職場的高度壓力。「前來求診的女性疲憊到寧願看著那些该死的iPhone也不願留一點時間維持親密關係。我们忘記去互相撫摸、看著彼此的雙眼。」

祖斯曼博士極有可能是对的,看起來缺乏慾望是女性的新疾病。二〇一三年的一項重大研究顯示,上一年度有三分之一的英國女性飽受性慾缺乏所苦。介於十六至二十四歲的族群中,四分之一的人缺乏做愛的興趣。這項結果讀起來令人難过。

所以女性缺乏慾望的標準是以什麼做為比較呢?從一九六〇年代起,部分研究採用了一種骨牌模型,其中提到四個性反應的階段:慾望—興奮—高潮—消退。其中,將慾望定義為希望進行性行為,包括幻想与思考。慾望是純粹的心理过程:我現在超想做愛!然而,興奮,是兼具歡愉感与純粹的生理反應,此外,還有性器官充血、陰道濕潤与擴張、脈搏數与血壓升高与呼吸急促的情況。

直到近代才有研究人員开始質疑這個模組。事实上,調查指出,有三分之一的女性甚少擁有性慾,也就是說他们沒有感受到所謂的「自發性慾望」。然而,大多數人仍經歷性所帶來身体上的刺激与享受。或許聽起來很奇怪。真的有那麼多女性因為這些嚴重問題感到困擾吗?





不,有愈來愈多人會這麼說。对許多女性来说,慾望其实有反應性,也就是說,會昇華成親密接触或是性行為狀態。身体上的刺激地位高於慾望,妳可能會說,所以這些女人比較仰賴前戲和親密互動來触動开關囉?反應性慾望的女性对性不太有興趣而且在床上也不太擁有主導權,但她们性致來了仍然有能力進行美好的性愛。只要小心翼翼,慾望是可以養成的。

肩負了教育女性有關反應性慾望責任的性研究者艾蜜莉‧納高斯基(Emily Nagoski),在她的著作《性愛好科学:掙脫迷思、用自己的方式高潮

》(Come As You Are)裡提到,接近三分之一的女性对性慾是有反應的。另一方面,我们發現15%的人屬於「典型」的自發性慾望,意即妳會突然感受到性慾上升。其他女性則是介於兩者之間。有時候,她们不太清楚自己想要做愛的原因,但其他時間,性聽起來又有點像是累贅,直到她们感受身体起了反應,接下來大腦才慢慢地一起加入同樂,只有5%的少數族群完全缺乏自發性或反應性慾望。

反應性慾望的模組与主流文化应该展現的性有明顯的分叉。我们遇过許多的女孩与女人无法分辨這種形象。她们納悶著如果和「其他人」不同,对性不感到興趣的话,她们是否就不正常。她们被男友洗腦,認為她们非常无趣,甚至因為对做愛不主動感到罪惡。对這些廣大女性来说,另一種解釋這樣現象的模組可能讓她们鬆一口氣。許多現象顯示反應性慾望純粹只是女性性慾的正常變化並非缺陷或是疾病9 。

我们認為另一部分的原因是自發性慾望為男性对慾望的正常主宰。納戈斯基表示,大約有四分之三的男性明顯屬於自發性慾望,出於一些奇怪的原因,我们認為男性与女性的性慾是以相同的方式運作。我们之後就會知道,男女之間大概不是。

另一個疑問的來源是人類与生俱來性驅力的迷思。也就是我们一出生就擁有性慾。驅力就像幫助我们生存的本能。它造成我们口渴、飢餓、疲倦。我们大腦完全下意识地傳送是時候去做特定事情讓身体為持平衡的訊息,例如睡覺、進食或是飲水。如果我们擁有性驅力,它會告訴我们对性有需求就好比我们对食物、睡眠与保暖的需求一樣。這樣的话,性成為我们生存的必須要素。當性被如此定義的時候,我们也就沒必要將性慾想得太複雜了。10 另外,你们对一切感到懷疑的话想想,沒有人死於缺乏性慾。性不是驅力而是獎勵。

只要性持續帶給我们享受与歡愉,就會像毒品对大腦的影響一樣:我们會想要更多。慾望受到刺激,接著我们开始寻找可以得到性滿足的情況。這就來到了納戈斯基重要的觀點:如果性沒有帶來獎勵的效果,例如因為痛苦、早期受到侵犯帶來的影響或是單純厭倦,妳的慾望便會消失。只有當性作為獎勵的時候這個機制才會運行。換言之,我们不是一出生就有性慾,而是漸漸變得慾火焚身。

我们從這個觀點得到兩個結论。第一,擁有較少性慾的女性(与男性,即使一般来说他们只經歷过反應性慾望)並非不正常或是生病。就像有些人喜歡巧克力,而其他人不喜歡。就算大部分的大腦对脂肪和糖分的這類的愉悅組合非常有反應,我们也不會認為不喜歡巧克力的人有什麼不对的地方。然後,順帶一提,我们有沒有將那些人稱為有病這件事情為什麼這麼重要呢?如果妳被當作一個活生生的怪人,那麼体內剩餘的一點性慾真的會完全被消滅。

第二,這表示性慾並非恆久不變。我们生來具有變得性慾高漲的潛力,但是程度範圍依據性帶給我们多少的快樂与滿足以及我们平常生活状况為何,並隨著時間有所變化。此外,我们的性史,也就是我们的性經驗,幫助我们形成性慾。

這個论點解釋了為何性慾會隨著我们的人生与關係高漲及降低,同時也在慾望的影響上給我们特別的意義。如果我们明白一切的運作,就能夠操縱大腦的獎勵系統。這個觀點告訴我们男性与女性最大的不同。

性研究者想出一些非常奇怪的點子。在大量的实驗中,將男性的陰莖与女性的陰道上裝有測量儀器,觀察性器官的血液流量。這個研究方式能夠從生理上得知人類的興奮程度,這些是人類无法靠意识控制的自然反應。实驗裡,实驗对象可能要觀看不同類型的色情片:異性戀、同性戀、擁抱、暴力,甚至是猩猩之間的性交。滿足各種偏好,也就是這個意思。看完影片後他们接著回報自己感受到的性慾程度,而在這個階段產生了非常有趣的發現。

在男性當中,約有65%的人陰莖勃起状况會对應所感受到的性慾程度。所以大腦通常會隨著男性性器官有著相同的自動反應。啊哈,我勃起了所以我一定是想要做愛,男性會這麼認為。(當然,這只是個大概。男性即使沒有任何像是做愛的慾望也能勃起,就像是大家都知道的晨間勃起現象,或是青少年在黑板前展示計算过程時的勃起。)男性的慾望与陰莖的惡作劇密不可分,因此當男性想要「站起來」的時候,威而鋼之類的藥丸就能完全發揮功效。威而鋼並不在大腦運作,而是單純讓血管將血液帶離陰莖造成收縮,讓陰莖變得更硬,使其充血。如果你讓陰莖在对的位置上,那麼也差不多大功告成,只要知道這一點就夠了。

然而,在女性实驗者裡發現,只有25%的人大腦与性器官重疊運作。兩者之間的關聯如此薄弱,因此根本不可能說女性感受到性慾的程度是取決於她的性器官有多濕或充血。有位女性的生殖器官在看到男性雙腿之間及猩猩的性交後完全腫脹与濕潤,但她完全沒有被激起性慾。女性生殖器官同時对女同性戀性愛的反應強烈,頻率比起異性戀性愛來得高。令人震驚的是,女性在被侵犯時能夠同時達到身体上的興奮与高潮。這代表什麼呢?女性的确喜歡猩猩性交,還是有些女生喜歡被侵犯呢?

不,不是,不是這樣的!這表示女性和男性不同,擁有性研究者所稱更高程度的「性興奮不一致」或是「主觀生殖器一致(不一致)」。這些複雜的名詞,其实表示在慾望當前大腦和下体區域之間的反應不一致。兩個身体部位明顯地沒有共同的想法,而慾望非常低的女性卻在這方面得到最高分。她们的大腦幾乎无法接收生殖器傳來的訊號。

女性慾望主要座落在大腦當中。光是有魅力的人躺在我们床上,或是像男性常有的變濕与勃起,還是不夠。我们需要更多,因為需要刺激的是大腦,而非生殖器官。這就是即使盡了許多努力,威而鋼卻只对極為少數女性有用的原因。為了讓女性性慾得以受到藥物的影響,必須改變大腦錯綜複雜的迴路,這才是提升到全新層次藥物该有的技術。

針对女性性慾的「粉色藥丸」歷經許多努力終於被研發出來。其中一項实驗是給予女性睪固酮,因為這個性荷爾蒙被認為是性慾的中樞。問題是,具有生育力的女性服用睪固酮卻不是個好主意,如果女性懷孕的话會对胎儿造成潛在的危害。所以大部分研究會選擇因為癌症手術或是更年期而完全缺乏睪固酮的女性作為实驗对象。在這些案例當中發現,睪固酮濃度提升对性慾有正向影響。以三十五到四十六歲略為年輕女性為对象的重要研究裡,卻出現性慾程度沒有上升的情況。然而,經过一個月的時間,接收中等劑量睪固酮的女性与服用安慰劑的女性相比,在「滿意的性活動」次數中增加了0.8%。

研究成果顯示了一旦超过最低水平,更多的睪固酮不會造成極大影響。事实上睪固酮对性慾影響的研究沒有任何重要結果能夠誇耀。无论性致高低与否,它似乎不能預測妳所位於的性慾階段。看起來性荷爾蒙无法对女性性慾有強烈的影響。

其他藥物也經过了測試。以「芭比藥丹」廣為人知的人工荷爾蒙美拉諾坦(Melanotan),由於挪威知名部落客蘇菲‧伊麗絲‧艾薩克森(Sophie Elise Isachsen)在青少女當中引領网上非法購入的風潮,曾經一度引起媒体極大的注意。

美拉諾坦模仿身体其中一種荷爾蒙,讓我们皮膚變黑並製造斑點。起初,美拉諾坦是用來幫助我们不須憑藉太陽曬黑,也就是一種助曬藥丸。後來發現美拉諾坦的副作用包含食慾降低,以及有提升性慾功效的可能。因此它具備完美女性的夢幻條件:古銅肌膚、身材纖細与性慾高漲。我们也從而得知,藥廠看見了商機。

然而問題在於,使用美拉諾坦最終會造成潛在生命危險的副作用。所有相關实驗也就此暫停。接著藥商發現了危險因子較低的藥物布雷默浪丹(Bremelanotide)。經过幾年实驗後,藥物現在正接受最終測試,看起來未來得以批准上市。但是這個昂貴的藥物需要使用注射器,再者效用也沒有特別大。平均來看,该藥物的使用者回報每個月「滿意的性活動」次數比起使用安慰劑的人多了半次。既然如此,也沒什麼可以說的了。

另一種藥物,氟班色林(Flibanserin),起初當作抗憂鬱藥物使用,於二〇一五年八月开放給低性慾的人使用。這項藥物一樣極其昂貴,一個月需花費好幾百元英磅,而且必須每天使用。由於會引起低血壓造成攸關生命的風險,妳也不能在服用藥物時飲酒。諸如噁心、暈眩与疲倦等副作用也經常發生。同樣的,這個藥品的效用沒有非常驚人。平均每個月使用者「滿意的性活動」次數增加了0.4%到1%之間。

換句话說,藥丸目前无法成為治療患者的希望。上述所提的藥物將副作用、價格与結果納入考量的话,沒有一個是適切的選擇。然而,這類的研究著重於性慾与滿足对我们的感受影響有多大。在一些研究裡,安慰劑效用确实有著極高的效果,幾乎比其他「藥物」來得高。此外在一項威而鋼的实驗中,40%拿到糖果藥丸的女性其性慾有改善的情況。藉由服用藥丸,她们進入新模式与新角色,改變了原先認為大家都不想跟自己做愛的舊有、根深蒂固的思維模式。

安慰劑效應告訴我们:性慾就在我们腦中,而且是能夠被操縱的。但是要怎麼做呢?

性研究者艾蜜莉‧納高斯基解釋得非常到位。想像大腦是個在身体頂部支配一切的敏感指揮官。身体的指揮官經常接受來自身体的訊號,他解讀的環境幫助這些訊號形成更合適的意象。我们神經系統和傳遞至大腦的訊號構造非常簡單,有點像電腦程式,裡面都是0与1。當中有一個訊號告訴我们「加速」,意即刺激,另一個則告訴我们「減速」,或稱抑制。刺激与抑制兩個訊號之間的平衡決定大腦在任意時間对身体的運作。如果妳減速了,同時踩了一點油門的话不會造成任何變化。總效應才是最重要的。

无论是有意识或是潛意识,想像每個阻止妳想要做愛的原因,將這一些壓力放到減速器上。原因可能包含壓力、憂鬱、身体形象差、罪惡感与无法達到高潮的恐懼。所有減速器上的微小壓力逐漸增加能夠使其壓迫地面,讓事情完全暫停。為了紓解減速的沉重壓力,大腦需要接受更有力的訊號來告訴我们加速,例如愛与歡愉。獎勵必須比努力要來得好。有時候,它自己會突然發生,例如當我们戀愛時,除此之外,我们的任務是确保「加速」的訊號能夠主宰一切而盡可能減少減速的功能。聽起來有點模糊,但其实一點竅門也沒有。第一步是承認性慾不是自己憑空而來,或是妳与生俱來不變的特質。之後,妳必須坐下來思考哪些東西會讓妳性慾冷感或高漲。照著納高斯基所說的:列一張清單。

什麼事情會澆熄我的性慾?在睡前做愛,因為我會擔心隔天无法好好休息、覺得心情低弱或傷心、害怕我的伴侶在我不想要的時候提議做愛,然後我又得拒絕他们、对感情的不确定、嫉妒、完全知道接下來會發生何事的規律做愛、為了讓伴侶覺得自己是個好情人,我必須達到高潮的期許、我今天应该要做的事情但卻沒時間去完成的壓力或擔憂、覺得自己好醜、當我還沒洗澡或覺得身上很髒的時候、當我们在床上看手機的時候。

什麼事情會點燃我的性慾?知道我们有許多時間也不用急著把事情完成、做愛速戰速決,不要廢话、对高潮的念頭、对自己的身体感覺良好、情色書籍或電影,或者是單純的色情片、運動後腦內啡湧出,熱血沸騰的時候做愛、在大白天做愛、一片漆黑,籠罩在黑暗當中、乾淨的寢具、覺得被愛、讚美、新環境、安全的環境、看到伴侶自由自在、做我所擅長的事情、我的背部被挑逗、勇於在床上嘗試新事物、當我确定每次在床上所做的一切絕对是伴侶認為最棒的事。

在妳寫下清單後,真正要做的事情來了。妳必須妥善安排這些事情,讓中間的平衡慢慢傾往「加速」的方向。這表示盡可能消除更多減速的原因,同時創造更多触發性慾的开關。

在感情當中妳非常難去獨自完成這件事。妳必須要和伴侶一起,告訴她(他)哪些事情會讓妳性慾高漲以及妳要的是什麼。關係變得一成不變的時候,性治療師通常會建議妳暫停做愛一陣子,或是建立做愛的原則,例如決定特定的日子与時間做愛,排开妳當天的行程。聽起來不太性感,但是卻有道理。藉由屏除所有对性的期待,妳在慾望主動回歸時獲得了喘息的空間。畢竟妳不能強迫慾望到來。妳应该要有慾望的念頭只是另一個造成減速的因子。

但不代表妳应该停止与对方親密。事实上,对許多人来说,效果卻是反其道而行。因為他们少了任何尚未準備好的壓力,多了能夠擁抱与變得親密的空間。妳应该要对自己寬容、有耐心。如果妳的伴侶不認為這很重要,那麼妳大概知道問題在哪了。

祖斯曼奶奶,憑著百年的經驗,領會到了重要的觀點。性慾並非憑空發生。慾望緊密交織在我们生活的情感關係裡,不單是我们与關係之間,也沒有快速修復的方法。然而大部分的人能夠感受到慾望的存在。





* * *



9	這些慾望的差別同樣明顯適用於男性身上。男性也可能同時擁有反應性慾望,但是在男性主要慾望形式中有點少見。根據納戈斯基所言,約有75%的男性主要經歷的是自發性慾望,相反地,女性則是15%。5%的男性屬於反應性慾望,而另一方面女性則為30%。



10	缺乏或喪失性慾确实在國際疾病傷害及死因分類標準(ICD-10)當中屬於心理与行為疾患的病症。即使擁有性快感与刺激,仍然有可能被診斷出來。美國診斷系統《精神疾病診斷与統計手冊(第五版)》(DSM-V)上的相同症狀現在也已被修正。





會見大圓滿




高潮是美好、了不起的現象。它和身体為了維持生活機能所進行枯燥乏味的例行工作不同。心臟跳動將血液注入我们全身,腸子翻攪消化提供我们養分,大腦神經訊號顫動使身体开始規劃与行動,而高潮有著非常特別的作用。高潮簡單来说就是包含腳趾蜷曲、頭皮發麻、呻吟的福報,是我们小小的獎勵。

人们試著对高潮下不同的定義,不过学者们卻是不完全認同。傳統医学对高潮的認知是劇烈性快感短暫達到高峰的狀態,連同規律的骨盆腔區域肌肉收縮。

現代性研究者認為這個定義太狹隘。女性高潮的感受因人而異,此外,身体上是有可能經歷不愉快的高潮或无性高潮,例如受到侵犯或是睡覺的時候。事实上,有三分之一的女性有过睡覺中高潮的經驗。因此研究者们开始認定高潮比較像是突然發生的事,是无意识釋放的性張力,就像釋放拉緊的弓弦一樣。

因此人可以在沒有愉悅感、沒有生殖器的接触或是陰道痙攣收縮的情況下高潮。有人將高潮描述為溫暖、顫抖的感覺席捲全身,有著非常明顯的「到達」感。簡單来说就是在高潮的當下妳會知道。如果妳不知道有沒有高潮过,那麼妳就是沒有。模糊卻非常簡單。

我们如果糾結在傳統的高潮概念,也就是最常見的那一種,那麼高潮意即性反應達到最高峰的狀態。當女性开始身体上的興奮,小陰唇与陰蒂內部會和男性陰莖變硬一樣充血。事实上,整個陰蒂部位在性慾高漲時會變成兩倍大。在生殖器受到刺激後的10至30秒,陰道往往开始變得濕潤,同時會變寬、變長至少一公分。妳愈接近高潮,妳的脈搏就跳得更快,妳的呼吸加速而血壓也會升高。很多人同時感受到其他部位的肌肉緊繃,外加手指与腳趾在床單上蜷曲。這個現象有個美妙的名字叫作:腕足痙攣(carpopedal spasms)。

最後,高潮來了。幸福的感覺從頭到腳貫穿全身。感覺就像妳的生殖器官爆炸,骨盆腔肌肉接二連三收縮。收縮的狀態從陰道的最低點开始,接著向上擴展到整個陰道与子宮。尿道周圍的肌肉与肛門也會有感覺。女性的高潮平均可以維持17秒。當高潮結束後,血液會开始離开生殖器官,就如同男性陰莖達到高潮後开始變得鬆軟。這個時候身体來到性消退期(resolution phase),所有部位都緩慢地回到正常狀態。

与男性不同,女性如果持續刺激自己可以達到好幾次高潮。女性高潮的世界紀錄不得而知。或許出於某種原因,金氏世界紀錄中並沒有記載,卻能在网站上搜寻到其他恐怖刺激的性紀錄,像是「最頻繁性愛」。如果妳对此充滿好奇,澳洲的鱗蟋蟀(Ornebius aperta)是紀錄保持者,在3〜4小時內可以進行50次性交。這個小坏蛋。

非官方統計最高的高潮次數發生在所謂的自慰馬拉松(Masturbate-a-Thon),藉由令人驚訝的自慰比賽來籌集善款。締造的最高紀錄發生於二〇〇九年的丹麥自慰馬拉松,第一名在極為冗長的馬拉松賽制時間內,達到222次高潮。這讓大部分的人有個目標可以挑戰……。

現在,妳大概訝異我们一般所提到的高潮,因為高潮有陰蒂高潮、陰道高潮、G點高潮、譚崔式高潮、潮吹高潮,以及舔腳趾高潮。不是吗?

其实所有高潮都是一樣的:就是高潮。身体上与心理上的反應也都相同。唯一的不同在於触發的地方。我们全身都是性敏感地帶。能夠刺激並帶來快感的神經末梢遍布身体。只要想像別人親吻妳的脖子、撥弄妳的頭髮或是撫摸妳的大腿會有多誘人就好。我们也有遇过一整天、每一天都會自發性高潮的女性,沒有任何的身体刺激,只要呼吸就能高潮。

雖然陰道高潮和陰蒂高潮其实沒有什麼不同之處,但是這兩個名詞仍然廣泛使用。我们現在知道陰蒂是大型器官,並不只是陰戶前方的小肉塊。圍繞尿道及陰道的陰蒂內部,通常能從陰戶与陰道間接收到刺激。關於「陰蒂高潮」与「陰道高潮」的說法不太确切,因為陰蒂和陰道性交是同時進行的。陰道本身非常不敏感。之後妳會知道,每個女性的陰蒂頭位置不一樣。依據位置的不同,有人認為在陰道性交時可能會讓女性更難或更容易達到高潮。

古老傳說掩蓋了潮吹高潮,女性射精或潮吹的內容,自亞里斯多德時代开始超过兩千年,這些詞才在文学當中被提及。大多女性認為,即使尿道位於陰蒂頭与陰道的前端,但卻不屬於性生活所触及的器官。儘管如此,有些女性發現尿道在高潮時會發生特別的現象,造成女性与研究者一頭霧水。當女性高潮時,尿道口噴發透明或乳白色的液体。有些人噴出相當於一個牛奶杯的量,而有些女性則是好幾毫升。這種高潮到底是什麼?

我们不知道多少女性有过潮吹高潮,但我们知道确实有這個現象,而許多人都曾在网路上看过。二〇一四年英國禁止色情片描述女性潮吹的劇情。我们不知道為什麼女性射精會比男性射精還要來得糟糕,看起來人们覺得女性射精格外唐突,他们大概認為那是排尿吧?但确实是如此吗?

至今仍然不确定潮吹液体的成分,有些研究的论點主張來自小型腺体,斯基恩氏腺(Skene's glands)。這些腺体位於陰道壁前端,圍繞尿道下方。看起來並非所有女性都有该腺体,而大小也因人而異,不过至少解釋了只有特定女性擁有潮吹高潮的現象。斯基恩氏腺等同男性製造精液的前列腺,在高潮時會將分泌物排至尿道。這個理论是根據女性射精液体中所發現的前列腺物質。然而在二〇一五年的一項研究當中,透过超音波檢測七名自慰的女性,發現雖然有少量的前列腺物質在射精液体裡,但主要的成分為尿液。幾名研究者認為我们會有兩種不同的現象:有些女性從斯基恩氏腺射出少量白色液体,而其他人則是從膀胱噴出大量透明液体。无论如何,分泌物的物質不是那麼重要。因為高潮对許多女性来说是一種自然現象。

我们回到陰蒂与陰道高潮。連同透过一般陰道插入方式触發到最高點的陰道高潮,女性長期受到高潮層次的困擾。她们認為沒有透过《發條橘子》(A Clockwork Orange)的主角艾力克斯‧迪拉吉常說的「抽送」獲得高潮就好像哪裡出錯一樣。也認為如果藉由手指或是舔吮的幫助達到高潮就好像欺騙一樣。

這很奇怪。不只是因為高潮就是高潮,无论妳怎麼看待。同時,這種達到高潮的方式对女性来说确实不正常。女性高潮的詭異階級制度到底是怎麼來的11 ?无论如何,它不是從古老時代留下的刻板印象。啟蒙運動時期前的人民認為女性只要高潮就會懷孕。如果妳真的想要懷孕,那麼男女雙方应该要同時高潮。近年來,隨著嬰儿死亡率急遽攀升,大量增加孩童數量是重點目標。如果男性想要有後代的话那麼讓女性高潮就成為必須展現的技術。女性高潮的關鍵就是直接刺激陰蒂頭。

在一七四〇年,医生向奧地利公主提議「神聖的陰戶女王陛下应该要在性交前受到挑逗。」現代的医生可能是從這裡得到啟發。比起被告知要活得健康一點,卻聽到妳应该要更常讓私密處受刺激,試想一下這個畫面。這就是現在我们所稱的主流健康趨勢!

因為上述的由來,所以十八世紀的男性就算对世界上更多事物完全不懂,卻仍明白整個高潮的運作。而籠罩自卑情結的陰蒂高潮,它的源頭就離我们生活的年代更近許多。我们必須回溯到二十世紀。

陰道与陰蒂高潮的差別,与陰道高潮才是真正高潮的這些概念純粹是現代男性的發明。心理分析之父西格蒙德‧佛洛伊德,在一九〇五年提出一個新理论,將陰蒂高潮認定為年輕女性不成熟的高潮形式。只有在小女孩房間裡才會發生的那種類型。只要女孩嗅到男性的氣息,她对陰蒂的興趣會消失,取而代之燃起了对插入式性行為的慾望。男女之間的交合是性唯一的健康形式,也是唯一給予女性歡愉的形式。根據佛洛伊德所說,真正的女性,必定体驗过陰道高潮。

佛洛伊德是從哪裡知道的?當然是從他腦袋裡啊!即使眾多女性非常不同意也沒關係。因為她们病了。她们處於一個叫作性冷感(frigidity)的模糊状况,起初她们很難從男性最雄偉的地方得到该有的歡愉。即使妳同意他的看法或是妳為此感到生氣,這完全是一種心理操控手法。

據佛洛伊德所述,女性如果認為撫摸自己的陰蒂的感覺很好,或是(但願不要發生)无法与丈夫在陰道性交時達到高潮的话,应该立即寻求心理医生的幫助,當然对男性来说這很棒。倘若女性无法高潮,並不是身為伴侶的資格遭到質疑,反而女性需要為自己做一些努力。現在男性正式得到允許做愛,接著完事,然後开心地關燈轉身入睡。女性的歡愉是屬於她個人的責任。

佛洛伊德不是普通人,他的理论得到大量的支持。因此千年以來女性的經驗瞬間被判定為女性神經疾病。幾世紀以來被當作女性性快感核心的陰蒂,在解剖書上遭到遺忘且隱沒。將近六十年沒有人敢反駁這一切。

一九六〇年代,美國華盛頓大学医院掀起一場寂靜革命。婦產科医生威廉‧麥斯特与研究夥伴維吉尼亞‧E‧強生开始对女性性慾感到興趣,並著手一系列以現在標準來看相當瘋狂的实驗。他们招募了情侶在实驗室做愛,替他们裝上測量儀器,研究人員同時在一旁認真觀察。他们甚至製作了塑膠震動陰莖,並將攝影機放在頂端,觀察女性高潮時陰道的變化。他们的研究結果被認定為驚人的医学發現:陰蒂的确是女性高潮的要塞。很震驚吗?看來真的很令人震驚。

現在,我们知道不到三分之一的女性經常單靠陰道性交獲得高潮,即使如此,仍然有許多人認為陰蒂才是核心。有些学者認為這些女性得到解剖樂透的頭獎。因為從陰蒂的尺寸与位置而言,她们看起來有一定的優勢。第一位在该領域進行科学研究的是另一位公主,法國的瑪麗‧波拿巴(Marie Bonaparte),即使她对性与伴侶有極高的品味,但因為无法透过陰道高潮所以一直无法滿足自己的性生活。波拿巴公主与現代研究者都認同一件事:愈大的陰蒂頭和陰蒂至陰道的距離愈短更容易高潮。因為陰蒂的表面与內部在插入式性交時同時獲得大幅度的間接刺激。波拿巴採取了激烈的方式,透过手術將陰蒂往下移,卻不幸以失敗收場。

我们希望波拿巴明白和男性進行一般性交卻沒有高潮並非不正常,這是正常的。但是由於男性長期主宰女性性研究以及对性的公共论述,所以導致許多人將這種觀念流傳下來。陰道裡的陰莖,性完全成為男性插入活動的同義詞。的确,大家會說少了男性高潮,性交就並非「完美」。如果只有女性高潮,性交在理论上就不完整,造成性交中斷。女性在這個階段就此淡出。

公平来说,对雙方来说日常性愛应该充滿歡愉与高潮,所以異性戀做愛,照理来说,舔吮和插入的比例不应该要是各半吗?女同性戀做愛達到高潮的頻率比異性戀女性高,因此擴展個人的技能很明顯是件好事。將女性高潮當作純粹的額外福利是错误的,高潮应该是一種習慣,对女性来说也是如此。

儘管如此,女性仍然无法擺脫比男性更難達到高潮的事实。女性得到性感缺失病(anorgasmia),也就是從未獨自或与对方性交獲得高潮的比例,介於5%到10%。对男性而言,則是相反:他们飽受太快高潮的困擾。英國一項大型研究發現,有21%介於十六至二十四歲的女性做愛時很難高潮。大多女性發現自己屬於「偶爾高潮」的類別。

有些幸運的女性不知道我们在說什麼。我们都有一個討厭的朋友,告訴我们她常常高潮,每次做愛通常會有3〜4次。妳問她到底有什麼祕訣?不幸的是,她可能沒辦法幫你。雖然祕訣可能會有一點貢獻,但我们之間有多容易高潮的差距确实存在,我们也无能為力。這些其中的差距,被認為是受到我们基因的影響。非常少數的人會去想到這是我们父母做愛後的結果,但他们的性生活卻多少和妳有一點關係。如果妳是高潮女王,妳大概要感謝妳的父母。

研究人員藉由研究雙胞胎發現我们的基因可以解釋三分之一人们做愛高潮頻率的變異。聽起來大概沒有很多,但是在基因学的範疇下卻不可輕視。他们同時觀察自慰高潮的頻率,此時遺傳扮演更重要的角色。研究实際顯示,我们的基因解釋了一半自慰高潮的變異。起初基因如何分別影響性与自慰的不同看起來很奇怪,然而,在妳消除更大程度的心理不确定与伴侶之間的性相互作用時,自慰被認為是身体高潮能力更真实的反應。

另一件对女性高潮能力有著極大影響的,則是我们做愛的情境。幾乎所有女性很難從一夜情達到高潮。只有10%初次与新伴侶上床的美國大学生擁有高潮,同時穩定交往超过六個月的女生擁有高潮的比例至少有70%。

我们達到高潮的容易度有著遺傳性的差異,不过振奮人心的消息是,只要妳想要,大部分的女性皆能擁有高潮。從跨出獨自完成高潮的一小步,接著偶爾達到,最後做到幾乎每次高潮,艱辛的挑戰就在這裡。我们不認為這很容易,或是特別強調擁有高潮的重要性,但只要妳願意付出努力就有可能達到。以下是我们的高潮寶典,靈感來自於无法高潮而接受治療的女性。





* * *



11	下述歷史敘述是根據漫畫家家麗芙‧史托姆奎斯特(Liv Strömquist)的圖像小說《知识的果实》(Kunskapens frukt)為靈感來源。





高潮寶典




一、 熟能生巧

如果妳不曾自慰,是時候空出一點時間了。自慰真的有用。在女性未曾有过高潮的研究裡,經过五、六週的規律訓練後,60%至90%的人藉由自己和伴侶達到高潮。我们保證這是医生建議的最有趣的運動形式。運用妳的手指或是買一個按摩棒。做任何能激起性慾、讓妳有心情去做愛的事情。妳喜歡情色文学、色情片還是幻想呢?最重要的是妳絕对不要把高潮當成目標,而忽略去探索妳喜歡的做法。練習感受歡愉並抱持开放的態度。練習清空腦袋所有煩人的念頭,无论是肚子上的肥肉還是妳獲得的高潮有多棒,之後跟伴侶就有可能會有更好的高潮。同時記得,和伴侶在做愛的時候用自己的雙手絕对不會錯。只要你们有個火辣辣的時光,誰该做什麼一點都不重要。



二、 主張妳的權利

妳的伴侶必須參与「高潮計畫」。重要的是不要激怒別人,而是讓這個計畫變得愉快。妳沒有高潮不是对方的錯,除非他(她)完全拒絕付出努力讓妳滿意。事实上,妳必須花點力氣做一些準備工作。妳的生殖器官沒有使用手冊,所以在沒有指南的狀態下,妳的伴侶可能要花上一年的時間才明白妳會如何高潮。最簡單的工作就是在一开始自己來,在做愛或是自慰的同時撫摸自己,过一段時間後妳可以指導伴侶妳是怎麼做的。許多人覺得很尷尬,但是卻只有這個方法。不要抱持一次到位的期待。伴侶每次做对的時候稱讚他们並对他们有耐心,在這之後妳就調教出一個優秀的对象。



三、 訓練自己用貓式体位

做愛姿勢有很多種,妳現在学到的,很少能讓女性高潮。不过,有一種姿勢卻很特別:貓式体位。因為傳教士姿勢的變化型,直進性交法(coital alignment technique,CAT),特別能讓女性高潮。這個姿勢需要練習与互相配合,但是妳付出的耐心會在各方面得到回報。

在貓式体位,你的伴侶不靠手掌,必須靠手臂支撐並盡量保持身体与妳服貼。和正常的進出抽插不同,对方应该將身体靠在妳身上並与你平行,直到他的生殖器從正上方進入陰道。同時,妳应该擠壓妳的胯下,就像海浪撞擊岸邊一樣(雖然很老調,但确实如此)。他的屁股应该朝下,這樣恥骨和陰莖頭才能与妳胯下摩擦。妳就會知道他是否做对了。接著妳应该保持兩腿打直並且盡可能夾住对方,這樣妳的腳踝會放在他的小腿肚頂端。相較一般傳教士姿勢屬於急速衝刺型,貓式体位則是以傳統經典的摩擦方式進行。陰莖不會非常深入妳的陰道,但是會对陰道幾公分外的末梢神經地帶給予極大的刺激,同時妳的陰蒂會頻繁接触。一旦掌握訣竅後,妳可以嘗試貓式体位的另一個版本,換成由妳在上面。這樣妳就掌握了主導權,同時能控制妳想要的陰蒂刺激節奏与力道。



四、 不要放鬆!

放鬆、放輕鬆,妳通常會聽到這句话。這大概是世界上最好与最糟的建議了。的确,妳应该試著讓大腦放鬆,但是妳光躺著不動就希望高潮突如其來湧上的话,妳就大錯特錯了。繃緊自己的身体很重要。夾緊臀部並試著縮緊生殖器的肌肉,最好是先縮緊後放鬆,第一,這會讓妳的生殖器充血,讓自己性慾高漲。第二,集中注意目前所進行的動作,當作一種心智鍛鍊。如果妳喜歡可以試試,但是真的很難邊想著晚餐要吃肉丸同時邊夾緊骨盆肌肉。起初,真的很難与維持這些肌肉。畢竟,妳家附近的健身房並不會有《健身趣》(陰道版)的遊戲。不过其他地方应该會有。許多常做骨盆運動的女性更容易達到更激烈的高潮,生殖部位的收縮更頻繁。此外,它能夠預防漏尿或是骨盆器官脫垂,也可能會改善做愛時的疼痛。妳可以在任何地方進行骨盆運動,在公車上或是睡覺前。雖然沒有絕对必要,但妳也可以使用陰道球。



五、 去跑個步

做個運動,尤其在做愛前,更容易讓妳性慾高漲以及能夠增加高潮的機率。



六、 穿上襪子

這個建議看起來有點好笑卻是認真的。重點是,我们大腦持續接收身体傳來我们感受的訊號。這些訊號,和煽動的想法,博取我们的注意力。當心裡想的都是其他事而不是生殖器官裡發生何事的時候就很難達到高潮,舉例来说,妳的雙腳冰冷。女性特別受到這些事情分心,周遭的事情也一樣。性研究者阿爾弗雷德‧金賽(Alfred Kinsey)從老鼠上觀察到,不同於公鼠,母鼠在性交時容易受到誘人的起司屑影響而分心。這個研究告訴我们,妳必須做任何能夠完全集中注意力並創造火熱時光的情境。這表示所有的燈都要關掉,妳想要穿著T恤做愛,或者,妳必須穿上襪子的话,那就聆聽自己內在的聲音吧。对自己好一點,只有身、心同時覺得很舒適時才會高潮,這樣妳才能把所有煩心的事情拋到九霄雲外。這大概是最難的课題,也是最多人遺忘的方式。





避孕





當女性与男性發生性關係時,可能會懷孕。因為這是普遍的結果,所以不应该感到震驚。孩子從來不是由鸛鳥送过來的。性愛很美好,大多數人希望做愛的次數比他们計劃生孩子數量多。如果妳是異性戀者,想要不會懷孕的陰道性交的话,那麼妳得使用某種形式的避孕方法。

藉由避孕的方式,我们指的是所有可以降低性交導致懷孕風險的方法。換句话說,也就是性交中斷,或稱為体外射精,這是一種避孕方法⸺雖然不是我们推薦的方法之一。

避孕並不是一項新發明,但是隨著医学的發展,更周全的方式已經進入市場。儘管如此,目前有許多避孕方法存在悠久的歷史。保險套沒有任何創新之處,但以前的材質是用動物皮而不是乳膠所製成。據說在四千年前的古希臘,婦女在陰道裡放了蜂蜜和葉子的混合物,使精子細胞離开子宮。這讓人聯想到現代放在子宮頸上的塑料圓盤避孕隔膜(diaphragm),用來阻擋精子細胞。雖然避孕膈膜在挪威已經有很長一段時間沒有被使用,但是在瑞典等其他國家變得更為常見,可能是因為反荷爾蒙避孕的趨勢。所以避孕也是一個風潮問題。總之体外射精幾乎不算是新發現。甚至在《創世紀》(Genesis)裡有一則与它有關的故事(有個叫作俄南的男子),妳可以非常肯定今晚某個地方會有情侶這樣做。或者是現在。

人類实際上已經嘗試了大部分的事情,然而現在的優勢在於妳有機會選擇。我们有很多經过嘗試与測試的備選方案,我们也知道這些方案有效。妳可以找到一個適合妳自己、健康与生活方式的可靠方法。

也許妳認為理所當然,但是避孕实在令人不可思議。它讓妳選擇是否要生孩子,同時也不會影響妳的性生活。如果妳想生孩子,妳可以選擇時間、对象和多少孩子。体外射精、避孕隔膜及植物与蜂蜜的組合可能有一定的效果,但与一九五〇年代所引進的避孕藥有很大的不同。

這是一場革命。當時的口服避孕藥是一種有效的避孕方式,比現在更厲害,因為經过多年的嘗試和測試。避孕藥改變了女性選擇何種關係的能力。她们可以控制自己的性生活並規劃適合自己職涯与經濟状况的家庭組織。從此,許多新的避孕方法开發,包括植入式避孕棒和荷爾蒙子宮環(intrauterine device,IUD)等長效方法。挪威衛生當局現在推薦這兩種方法作為他们的首要選擇。

以上讓你对歷史背景有一點認知。現在是時候談談這個情況了,我们必須要完全坦白。避孕這些事情真的枯燥乏味,非常无趣。在本書的這個章節,我们盡力解釋不同避孕方法之間的差別。我们寫了一些如何使用的方式,並提供了一些技巧和竅門當作額外獎賞。而這全都是非常技術性的內容。很抱歉但我们不得不說,避孕章節的第一部分可能对你们来说會是這本書最无聊的地方。即便如此,我们仍決定納入書中。為什麼?嗯,大概是因為這是我们必須去寫的東西當中最重要的內容。

畢竟,我们知道年輕女性在想什麼,而其中有許多人对避孕有些複雜的疑問。這並不意外,因為所有女性,出於某種因素,被要求在沒有指導的情況下,憑直覺去理解這個複雜的事情。我们也知道,關於避孕的迷思有難以置信的數量存在並流傳著,許多人由於胡亂使用而飽受不必要的副作用,或者因為資訊不足而感到不安。我们不知道是不是开立避孕處方的医療專業人員所提供的資訊太糟還是太少,又或是因為一次採取太多的措施所導致這些状况。

這個章節我们的目的在於給妳一個避孕的基本介紹,讓妳有方法替自己做選擇。避孕正處於不斷發展的过程,我们建議妳聽取医療專業人員的建議,他们可能对妳所感興趣的避孕方法有更新、更詳細的知识。





荷爾蒙避孕法




防止懷孕的荷爾蒙避孕法是什麼呢?當妳每天早上吞下避孕藥,每三週插入一次陰道環,或將植入式避孕棒放進妳的手臂時,進入妳身体系統的到底是什麼呢?

与卵巢產生的激素相同,荷爾蒙避孕含有非常低劑量的激素,与控制月經週期有關。所有類型的荷爾蒙避孕都含有黃体製劑(progestin)。這是身体製造黃体素的合成版。有些避孕方法甚至包含另一種荷爾蒙,雌激素。這些被稱為複合避孕法,而只含黃体製劑的藥物被稱為純黃体製劑產品。





雌激素避孕




複合避孕法分為三種:複合避孕藥、陰道環与避孕貼。複合避孕的優點在於雌激素可以控制妳的出血。缺點是並非每個人都可以使用雌激素,妳在之後的部分可以讀到更多內容。

複合避孕藥是最常用的複合避孕法,種類很多,每種都略有不同。首先,這些藥丸使用不同類型的雌激素和黃体製劑。再者,不同的複合藥丸有不同的黃体製劑和雌激素劑量。這兩種不同都會影響妳所經歷的副作用,无论是正面還是負面的影響,但是妳无法事先知道该類型的複合藥丸对妳是否造成效果。直到找到適合妳的品牌前,需要反覆嘗試。如果妳开始服用複合避孕藥,挪威衛生當局會建議妳嘗試含有左炔諾孕酮(levonorgestrel)的黃体製劑,也就是樂盈肌(Loette)、欣无妊(Microgynon)和同等產品。

複合藥丸有兩大類,分為多相和單相。但究竟是什麼意思呢?

大多數的藥丸屬於單相。如果妳使用單相類型,那麼從藥丸包裝的哪個位置打开服用並不重要,因為每顆的劑量相同。換句话說,包裝裡所有藥丸都是一樣的。大多數單相藥丸的設計,可以讓妳在某種程度上建立一個固定天數的人造月經週期。大部分類型屬於28天的週期。因此妳服用21天週期藥丸的期間內,妳不會有任何出血的情形。最後7天是所謂的无藥丸週。接著妳也可以服用一些包裝裡的糖衣錠或暫時停止。在沒有荷爾蒙的日子裡,子宮通常會排出子宮內膜,所以妳會流血。欣无妊与優思明(Yasmin)就是使用21+7天的單相藥丸模式。有些單相藥丸已經排列妥當,所以妳可以連續服用24顆,然後休息4天。優悅(Yaz)和佐宜(Zoely)屬於使用24+4天組合藥丸的例子。如果妳不想要任何出血,妳可以不用間斷直接服用新的一包藥。之後我们會提到更多。

多相藥丸裡的每一顆藥丸所含的荷爾蒙劑量不同,關於服用藥丸的天數和出血天數,每個品牌都有自己設計的週期。所以妳不可能隨時使用多相藥丸,妳必須仔細參照說明。如果妳使用多相藥丸,閱讀服用資訊並正确使用特別重要,尤其是妳打算避开經期的话。目前在挪威銷售的多相藥丸僅有Synfase和Qlaira。

當妳使用複合藥丸時,即使是不服藥的期間,也可以保護妳避免懷孕。所以妳可以隨時做愛,不必使用其他避孕措施來預防懷孕。但是,如果妳沒有準時服用藥丸,可能會失去防護力。妳能夠錯过藥丸的數量与可能懷孕的機率取決於藥丸的種類,所以請參考服用資訊与医生、護士或助產士的指示。錯过服用藥丸時機導致預防懷孕的劑量不足,我们稱為避孕失敗。

陰道環是一個插入妳陰道的塑料環。它看起來像是一個義大利麵細度的環形柔軟甜甜圈。目前在挪威只出售一種陰道環⸺舞悠(NuvaRing)。妳只須用兩根手指一起按下陰道環,將其推入就能插進妳的陰道。當妳釋放握把時,陰道環會彈回原來的形狀,調整至貼合陰道內壁,並保持原狀。要拿掉的话,只須用中指輕輕取出即可。

陰道環同時含有雌激素与黃体製劑。荷爾蒙通过陰道黏膜,最後進到血液當中。很多人認為有東西圍繞著陰道聽起來不太舒服。她们也在猜想陰道環是否會消失在陰道裡。

一旦插入陰道環,妳就不应该注意到它的存在,就像使用衛生棉條一樣。但還是要小心!雖然這種情況並不常見,卻有一些陰道環脫落掉入廁所的例子。我们一位女性友人就發生這個情況,在她晚上到市區玩樂的時候。她告訴我们直到隔天下午她才注意到這件事。當妳喝酒的時候,很容易變得比平常少一點警覺性,接著可能會發生倒霉的事。現在養成將手指放入陰道的習慣,檢查一下陰道環的位置,會是個不錯的方式。

与大多數單相藥丸一樣,妳应该連續使用陰道環21天,也就是連續三週。妳可以使用相同的陰道環三週,接著停用7天讓經血出來。如果妳想避开經血,妳也可以直接放一個新的陰道環,不要中斷使用。

雖然妳不會注意到陰道環的位置,但妳的伴侶可能會在陰道性交時感受到它。所以有些女性喜歡在性行為前將其取出。這麼做完全是安全的。妳一次可以拿出陰道環3個小時,但是務必記得要在3小時後把它放回去,否則妳將失去避孕的防護。

荷爾蒙貼片是直接放在皮膚上,讓荷爾蒙穿透妳的皮膚進入血液。挪威販賣的貼片叫作以芙(Evra)。妳可以每週使用1個貼片,如同陰道環和大多單相藥丸,妳应该連續21天讓荷爾蒙進入体內。所以妳必須使用3個貼片,每週1次,最後停用7天。如果妳忘記及時更換貼片或是貼片脫落,可能會導致避孕失敗。



複合避孕法如何預防懷孕?

我们体內已有的荷爾蒙可能會阻止懷孕,看起來有點奇怪,不过複合產品中的黃体製劑与雌激素有著非常好的效果。

複合避孕藥的主要作用是防止月經週期中的排卵期發生,大約是每個月一次。如果妳在排卵前5天左右(包括排卵的那一天)發生无防護性行為,妳可能會懷孕。這個時期稱為受孕窗口。

使用荷爾蒙避孕可以与懷孕的情況做对照。當妳懷孕時,妳的月經週期停止,就好像有人按了暫停鈕。如果妳的月經週期停止,沒有排卵,而沒有排卵也就沒有受孕窗口或受精的可能。

當妳懷孕的時候,体內自然產生的黃体素是造成月經停止的原因。黃体素告訴大腦的腦垂体(看起來像陰囊的那個地方),不再產生濾泡刺激素(FSH)和黃体生成素(LH)。妳可能還記得,這些荷爾蒙是維持月經週期的必要條件。沒有FSH就沒有濾泡期,而沒有LH也就沒有排卵期。

荷爾蒙避孕中的黃体製劑和懷孕時身体裡的黃体素一樣。黃体製劑告訴大腦是時候停止月經週期一陣子。在某種程度上,妳可以說複合避孕法使計讓身体以為懷孕了。

複合避孕以好幾種方式防止懷孕,不只是停止排卵而已。經过性交後,精子細胞必須經过子宮頸游向子宮。子宮頸裡有黏液。複合避孕中的黃体製劑使黏液變稠,使精子細胞更難游入子宮。此外,子宮內膜比平常來得更薄。這讓受精卵難以緊貼在子宮內膜上。

雌激素通常負責子宮壁或子宮內膜的生長,而這些內膜後來會成為妳的月經。複合避孕法中的雌激素讓子宮內膜每個月增長一些,所以大多數使用複合避孕的女性在使用後短暫休息7天或7天以內時也會出現月經。





无雌激素避孕




无雌激素避孕的優點是任何人都可以使用,甚至是因為某種原因不能服用雌激素的婦女。如荷爾蒙子宮環和植入式避孕棒的長效避孕方法,都沒有雌激素,而且提供最有效的避孕防護方式。這就是為什麼挪威衛生當局建議大家作為首選。使用无雌激素避孕的缺點是妳可能无法像使用複合避孕一樣控制出血。換句话說,如果妳使用无雌激素避孕,妳不能決定經期的時間。一般来说,使用各種形式的荷爾蒙避孕法,出血會比平常要少得多。在我们的印象裡,一些使用植入式避孕棒的女性有持續性出血異常的問題,但是這对使用荷爾蒙子宮環的女性来说不是問題。再者,這個問題与实驗及错误有關。

植入式避孕棒是含有黃体製劑的塑料棒。挪威所販售的品牌為Nexplanon。它使用一種注射器放入上臂的皮下組織,可以維持三年的時間。同時不斷釋放一點荷爾蒙,讓血液中的量達到穩定的低水平。植入式避孕棒目前是市場上最安全的避孕方法。一旦在妳的手臂上,就不會出差錯。只要它還在妳的手臂裡,植入物中的黃体製劑會停止妳的月經週期,讓妳停止排卵。

荷爾蒙子宮環是一個小型T狀物体,由經过訓練的医療專業人員將其放入子宮。子宮環主要在生殖部位局部釋放低劑量的荷爾蒙,雖然量少卻能通过子宮黏膜並被吸收進血液當中。在血液裡循環的荷爾蒙劑量非常低,許多經歷其他避孕方法副作用的人可能因為改用子宮環而減少使用上的問題。妳可以依據所選擇的子宮環類型維持三到五年之久。目前挪威市面上有三種類型。其中之一,可以使用五年,叫作蜜蕊娜(Mirena)。它是荷爾蒙劑量最高的子宮環,所以特別適合想要少量出血的女性。許多女性發現,使用蜜蕊娜後月經完全停止。

接下來是Kyleena,同樣也能維持五年,但是它專門為沒有生產过的婦女而設計。它比蜜蕊娜小,荷爾蒙的劑量也更低。最後一種,可以維持三年的小蜜(Jaydess),同樣也有非常低的荷爾蒙劑量,而且很小。儘管小蜜与Kyleena特別販售給沒有生產过的女性,但是年輕女性也可以選擇使用蜜蕊娜。

比起其他兩種子宮環,蜜蕊娜相对来说有點大,有些人可能會在插入後感到不適,而另一方面,能夠讓妳更好控制出血,此外妳身体裡的荷爾蒙劑量与其他避孕方法相比還要更低。荷爾蒙子宮環只適合已經生產过的婦女,這只是一個老舊的迷思!

有些女性可能會發現使用子宮環後就不再排卵,但並非所有的子宮環都有這種現象。這當然是暫時的,也取決於子宮環中荷爾蒙的劑量。當妳使用蜜蕊娜時,因為劑量略高,所以會比較常見到停止排卵的情況。通常小蜜和Kyleena含有的黃体劑量太低,无法影響大腦的腦垂体,但這並不意味著它们不能發揮作用。事实上,子宮環最重要的功效在於局部:黃体製劑讓子宮頸內的黏液不能穿透精子細胞。同時,子宮內膜變薄,讓任何受精卵難以生存。

三種荷爾蒙子宮環都提供不錯的避孕方法,以及長效、可靠的避孕防護。大多數女性會比以往經歷更輕微的出血与更少的經痛,而且許多人還會發現,由於荷爾蒙劑量低,与其他荷爾蒙避孕法相比,其副作用更少或不再嚴重。最常見的副作用,尤其是使用小蜜和Kyleena,則是長斑和不規律出血。

如果妳擔心插入子宮環會很痛,那麼在1小時前服用止痛藥可能是一個好主意,因為有些人會在插入後的一段時間有經痛感,但是疼痛感很快就消卻。在那之後,妳不會注意到它的存在,除了在陰道最深處妳能感覺到兩根小小的線從妳的子宮頸延伸。這是定期更換子宮環時用來讓医生移除的牽引線。

无雌激素避孕藥是一種必須每天服用的避孕藥。你永遠沒有停止服用而造成月經恢復的時候。也沒有必要每天在同一時間服用。只有在服用最後一顆藥丸後的36小時內,才有懷孕的風險。无雌激素荷爾蒙丸裡黃体製劑的作用方式与植入式避孕棒相同:它影響大腦的腦垂体防止排卵。此外,子宮頸黏液變得難以穿透,子宮內膜也變薄。

迷你丸(mini pills)也是一種每天服用的避孕藥,沒有暫停服用導致月經恢復的情況。迷你丸的黃体劑量低於无雌激素避孕藥,所以妳每天必須在同一時間服用避孕藥會比較好。妳會有3小時的受孕窗口,所以很容易造成不正确服用藥丸与懷孕的風險。





避孕針(contraceptive injection)需要由医生或其他医療專業人員施打,最遲要在使用上一針的十二週以內注射。所以妳必須每3個月去找一次医療專業人員接受新的注射。荷爾蒙注射含有大量的黃体劑,足以防止排卵。它也適用於子宮頸黏液和讓子宮內膜變薄。一般情況下,不推薦二十五歲以下的女性使用荷爾蒙注射劑,因為荷爾蒙劑量太高,會影響身体內骨骼的增長。





非荷爾蒙避孕




妳是否想要一種不含荷爾蒙的替代品?

每個不含荷爾蒙的避孕方法間幾乎沒有共同之處,所以人们有很多使用的選擇。有些女性會受到荷爾蒙避孕的副作用所苦或恐懼副作用,而影響她们的決定。預防性傳染疾病是使用保險套的一個好理由。其他女性則因為对伴侶或家人隱瞞使用避孕產品感到擔憂,所以更希望能像以前一樣繼續月經週期。





保險套




保險套能阻止精子細胞進入子宮。

保險套有阻礙的功能,因此被稱為障礙避孕法(barrier method)。目前,保險套是男性唯一可以使用的避孕方法。

保險套是一種由乳膠或類似材料製成,套在陰莖上的袋子,並在男性射精時收集精子。性交後,陰莖撤出時应该要把保險套固定住,這樣保險套与精子都不會留在陰道內。一旦結束後,只需脫下保險套,在上面打個結,然後直接扔到垃圾桶裡。不要把保險套扔進馬桶。在妳沒有料到的時候,它们會有浮起來的特性,无论在共同的住處,還是在父母的房子裡,都不會是件好玩的事。

保險套是唯一能保護妳免於性傳染感染的避孕方式。換句话說,保險套可以保護妳免於疾病和懷孕。聽起來好像应该放棄其他避孕產品一樣,只要一直使用保險套就好。但不幸的是,只用保險套的人卻發生不少意外。它们可能會裂开、脫落或毀損,所以許多人會選擇使用保險套同時結合其他避孕措施。

很多人使用保險套的方式不正确,這代表著出錯的可能性更大了。考慮到這一點,以下是我们正确使用保險套的方法。





保險套课程




1. 檢查日期戳印,舊的保險套更容易破掉。

2. 小心打开保險套包裝。注意,不要用鋒利的指甲、牙齒或珠寶 以免劃破保險套。

3. 一旦陰莖變硬,將保險套像墨西哥帽一樣放在它的頂部。





4. 擠壓保險套頂部推出空氣。空氣可能會導致保險套裂开。

5. 從陰道抽出陰莖時,請將保險套牢牢固定,否則精子可能會流入陰道。

6. 保險套应该保持在性行為時全程中使用,以防止懷孕或性傳染疾病,而且只能使用一次。



還有其他類型的障礙避孕法可供女性使用。我们已經提到的避孕隔膜,亦即在瑞典流行的葉子和蜂蜜的現代版。雖然可以在网上訂購,但是挪威的避孕隔膜並不是特別容易取得。還有一種反向保險套,就像一個放在陰道內的袋子,而不是在陰莖周圍。這稱為女用保險套,也提供保護並防止疾病。據我们所知,它在挪威市場上无法買到,在我们地區也甚少使用。





安全期⸺找到妳的受孕窗口





月經週期裡妳可以懷孕的時間稱為受孕窗口。為了避免因性行為而懷孕的機會,有些避孕方式与找到受孕窗口有關。

找到受孕窗口有不同的方式。妳可以使用月經日曆、每天早上測量体溫或檢查自己的子宮頸粘液。人们經常將這些方法結合起來以提升更高的可靠性。

這些都不是好的避孕方式。我们認為,对於絕对不想懷孕的女性来说,這麼做太不可靠了。根據世界衛生組織(World Health Organisation,WHO)統計,所有使用基礎体溫法的女性,有25%的人會在一年內懷孕,換言之,就是四分之一。數量非常多,但是如果对妳而言完全不危險的话,妳可以評估妳是否願意承擔懷孕的風險。

鑑於許多人对這些避孕方法的興趣增加,我们希望能夠对這些原理提供一個簡短的說明,儘管我们不推薦它们。雖然這些方法不能提供可靠的防護,但的确具有一定的價值。嘗試懷孕的女性也可以使用這些方法來判斷自己的受孕窗口,變得更容易受孕。

透过經期日曆計算排卵日的人首先會以月經週期的資訊當作標準。排卵日通常在每個週期的同一時間發生,也就是在月經前14天左右。

那些使用基礎体溫法的人,首先會以月經週期內我们略有改變的体溫當作依據。

确切来说是0.3度!妳可能還記得,月經週期有兩個階段。第二階段前的1〜2天,妳的体溫會上升0.3度,並持續10天左右。第二階段开始時,大量的LH從大腦釋放到血液中。LH的急劇增加引發排卵,通常在荷爾蒙濃度上升後1〜2天發生。換句话說排卵日會在体溫升高後2〜4天之間。每天持續測量妳的体溫,可以發現在月經週期中,妳通常會以排卵日為基準,推算受孕的最佳時間。

事实上,在排卵時,妳也可以觀察子宮頸黏液的變化。為了讓這個方式有效,妳必須每天檢查分泌物的情況,並找出變化。就在排卵日前,妳的分泌物變得光滑、粘稠,妳可以透过手指伸展,長度通常是好幾公分。剛排卵的時候,妳的分泌物會變為乳白色。這種方法需要妳非常熟悉分泌物的状况,並花時間研究它在整個週期过程中的變化。妳应该知道,除了週期以外,影響分泌物變化還有其他的原因。例如各種疾病會影響濃稠度,使妳很難判自己身處週期中的哪個階段。

也許妳現在認為聽起來很複雜⸺的确這也是個問題。這其中有很多的計算、推測和記錄,同時也有好幾天的誤差,所以會有很多搞錯的機會。除了对打算使用這個方法避孕的女性非常嚴格以外,也有條件地要求女性經期必須完全規律而且每次週期只能有一次排卵。由於上述的條件,這些方法並不會特別安全。

正常来说,一次月經週期只有一次排卵,然而可能會有多次排卵的情況發生。在排卵前至排卵後一天的1〜5天內發生性行為,有可能會懷孕。

月經週期也可能藉由外部因素如壓力、体重變化与疾病改變。通常年輕女性的月經週期比老年女性更不規律。所以比起年長女性,這個方法較不適用於年輕女性。





含銅子宮內避孕器




含銅子宮內避孕器是一種不含荷爾蒙的替代品,我们會推薦妳使用。在所有使用含銅子宮內避孕器的女性當中,只有不到1%的人在一年內懷孕。像荷爾蒙子宮避孕器一樣,含銅子宮內避孕器也是由医生或其他医療專業人員放入子宮的小型T形物体。不同之處在於含銅避孕器有銅線包覆。妳可以將含銅避孕器存放在子宮長達五年,且它在整個过程中提供良好的防護。含銅避孕器的底部有兩條線懸掛,並附著在子宮頸开口,以便妳可以用手指檢查避孕器是否還在原位。荷爾蒙避孕器也是相同的構造。當含銅避孕器需要移除或更換的時候,医生會透过那兩條線移除更換。

含銅避孕器有好幾種類型,在品質上的差別並不大。然而卻有價格的差異,最便宜的約為350克朗。如果將避孕器放置五年,每個月的成本就低於6克朗,与其他可靠的避孕方法相比,含銅避孕器極其便宜。

我们不知道含銅避孕器防止懷孕的原因与運作。我们所知道的是含銅避孕器造成子宮輕微發炎,因而改變了裡面的環境。不知為何,這樣的情況可以防止懷孕。有一種理论認為,子宮开始排放殺精子物質,可能是因為發炎,或者銅本身可能會殺死精子。另一種理论則是含銅避孕器防止任何受精卵附著在子宮壁上。

和使用荷爾蒙避孕法的人不同,使用含銅避孕器的女性每個月都會正常排卵。含銅避孕器对大腦或卵巢沒有任何的影響,它只在子宮有局部效應。

含銅避孕器不會產生荷爾蒙副作用,但這並不意味著沒有副作用。許多女性經歷比以前更嚴重的出血和更嚴重的經痛。由於這些状况,100名女性當中有2〜10人決定在第一年取出避孕器,因此,含銅避孕器通常不適合有這類問題的女性。

含銅避孕器有許多的迷思。最普遍的一種是,如果妳以前沒有生產过,就不能使用它。就算沒有孩子,使用荷爾蒙避孕器或是含銅避孕器完全不會產生問題,即使妳很年輕也歡迎妳來嘗試含銅避孕器。含銅避孕器是行之有年的避孕方式,而且近年來也變得更小、更可靠。

從純粹的实用觀點來看含銅避孕器可能也會有不舒服的感受,因為它必須通过狹窄的通道進入子宮放置。有許多人体驗过嚴重、短期的經痛感。可能得事先服用止痛藥。同時嘗試与真正放鬆也很重要。与替妳安裝避孕器的医生討论一下。

如果避孕器底部的兩條線突然消失了,妳应该与医生聯繫。這可能表示含銅避孕器被推出妳的子宮,妳不再有避孕的防護。顯然有5%〜10%的使用者發現含銅避孕器就這麼掉出來。在極少數的情況下,妳找不到線頭的话代表妳可能已經懷孕。如果懷孕的话,那兩條線其实已經被拉進子宮裡了。





緊急避孕⸺慌亂狀態




星期天上午,妳在昨天晚上發生了性行為也沒使用可靠的避孕方式。妳沒有特別渴望懷孕,所以令人如此害怕,讓妳开始胃痛。妳不是第一個經歷這種状况的人,也不會是最後一個。有時候事情就是會出錯,也是為什麼我们有緊急避孕措施。妳可以在无防護性行為後,或是避孕失敗時使用。

避孕失敗的定義因避孕的種類而異。可能是錯过服用藥丸的時機,陰道環脫落或者保險套裂开。重要的是要熟悉妳所使用的避孕方法,這樣在避孕失敗的當下才能立即意识到。兩顆避孕藥的間隔超过多久才會被認定是避孕失敗呢?陰道環停留在陰道外面多久才算避孕失敗?問問妳的医生、護士或接生員關於妳避孕方法失敗的一些規則。

當妳使用荷爾蒙避孕法失敗(例如錯过服用藥丸),通常會導致排卵。很多人在避孕失敗後不做緊急避孕措施,是因為她们不明白可能會有懷孕的風險。從發生性行為後可能已經过了好幾天的時間她们才發現忘記吃藥。但是記住,精子細胞為了等待卵子可以在子宮裡存活五天。如果妳避孕失敗導致現在排卵的话,這代表妳可能在五天前透过性行為懷孕了。

在挪威,人们稱緊急避孕為「後悔藥」。我们应该停止稱呼這個名字。「後悔藥」是一個會聯想到撅起嘴唇和揚起眉毛模樣的呆板說法。這意味著妳做了应该感到後悔的事⸺但妳真的沒有。妳才剛發生性行為,如果這是一個正向的經驗,我们就沒有理由後悔。另外,當妳拿起包裝服用今天的藥丸時,發現妳在过去一週已經錯过三顆藥的心情並非後悔:而是恐慌。這就是為什麼我们選擇在這本書當中稱為恐慌丸。

我们也沒有特別拘泥這個詞⸺「事後丸」。這聽起來不錯,又方便進行,就好像妳在每次性行為後的早晨吃下,而不用其他避孕措施。但重要的是,不要太容易依賴恐慌丸。它不像一般避孕法有效,雖然不是那麼危險,卻仍有一些副作用。只有在其他避孕方式失敗時,才能使用緊急避孕。它不应该取代正常避孕方式。

緊急避孕措施有三種:兩個不同的藥丸与含銅避孕器。第一類避孕藥含有一種左炔諾孕酮的物質,屬於黃体製劑的一種。換句话說它包含与荷爾蒙避孕相同的物質,黃体製劑的量要高得許多。第二類藥丸含有醋酸烏利司他(ulipristal)的物質。這種物質影響身体自然黃体素的運作。





第一類恐慌丸:左炔諾孕酮




含左炔諾孕酮孕的避孕藥是挪威最常販售的緊急避孕藥。在藥局的櫃檯、加油站和一些超市皆能購入。在挪威,這種恐慌丸是以后安錠(Norlevo)的品牌名進行販售。





第一類恐慌丸的作用是推遲排卵。問題是,如果妳已經排卵或者即將排卵的话就沒有效了。妳可能還記得在月經週期的章節裡提过,女性在排卵前會經歷LH急劇增加的情形。一旦LH开始上升,含左炔諾孕酮藥丸將无法阻止排卵。

妳很難知道自己是否已經排卵。排卵通常一個週期發生一次,但它可以發生很多次,也只有週期完全規律的女性知道自己排卵的多寡。

因此,這種藥並非完全可靠,雖然它确实會降低妳懷孕的機會,所以服用它絕对是明智的選擇。妳愈早服用效果更好。最好在无防護性行為或避孕失敗的24小時內服用。儘管如此,恐慌丸在无防護性行為或避孕失敗後三天(72小時)內是有效的。避孕藥降低效用的機率會隨著更多時間过去提升,所以隨時放一顆在妳的盥洗包裡是個好主意。

單次生理週期中多次服用左炔諾孕酮藥丸完全沒有大礙。



優點:取得容易、不影響其他避孕措施、可以在月經週期裡多次 服用

缺點:不太可靠

注意事項:三週後要進行驗孕!





第二類恐慌丸:醋酸烏利司他




在挪威,含有醋酸烏利司他的藥丸是以艾伊樂(EllaOne)的名稱進行販售。有效的防護時間為无防護性行為或避孕失敗後的五天(120小時)內。艾伊樂在藥局的櫃檯上販售,卻尚未在商店或加油站裡販賣。





和左炔諾孕酮的恐慌丸一樣,醋酸烏利司他藥丸也能延後排卵。這兩種類型的恐慌丸之間的差別在於,這類的藥丸可以在接近排卵期服用,而且仍然有效。妳可以在快要排卵時服用。然而,如果妳已經排卵的话這種藥丸就會失去效用。換句话說,醋酸烏利司他藥丸即使身体的LH濃度上升仍然有效。這讓藥丸變得更加有用,它會預防更多懷孕的次數。

當然,這種藥丸也有一個主要的缺點。主要的問題在於,它与荷爾蒙避孕有嚴重反應。首先,它會在服用後影響一般避孕方法在妳身体的運作模式。這意味著在服用恐慌丸後妳只能使用保險套,因為妳的荷爾蒙避孕法可能无法運作。妳必須按照這個模式去進行的時間,取決於妳採取什麼樣的避孕措施。

同時妳所使用的荷爾蒙避孕方式也會影響緊急避孕藥的效果。因此,它是雙向的。這表示妳不应该服用艾伊樂後使用荷爾蒙避孕。事实上,新的研究顯示,使用荷爾蒙避孕可能破坏藥丸对排卵的影響,阻止延遲排卵的功能。服用艾伊樂後妳应该經过五天再开始或繼續使用荷爾蒙避孕。

醋酸烏利司他緊急避孕藥只能在每次月經週期中服用一次,因為從未有研究進行一個週期內使用多次的測試。

這並不代表避孕藥危險,只是沒有人知道它是否在一個週期內發揮作用多次。避孕藥會影響其他荷爾蒙避孕法,如果妳試著在服用醋酸烏利司他後使用左炔諾孕酮的话,它也能影響第一類藥丸的效用。如果妳已經使用醋酸烏利司他,避孕失敗的话最好使用含銅避孕器。



優點:比左炔諾孕酮丸有更好、更久的效果

缺點:使用荷爾蒙避孕藥會有嚴重反應

注意事項:三週後要進行驗孕!





含銅子宮避孕器




雖然含銅避孕器是緊急避孕最安全的形式,但很少使用。如果妳需要緊急避孕,我们建議妳考慮含銅避孕器,因為它有99%的效用。它能阻止受精卵附著在子宮上。

含銅避孕器由医療專業人員放入子宮,所以在妳有无防護性行為後,可以向妳医生掛急診並解釋發生何事。妳還可以求助緊急手術或到年輕人會去的診所。含銅避孕器有效的防護時間為无防護性行為或避孕失敗後的五天內(120小時)。它會有效是因為在排卵的第六天裡受精卵不會附著在子宮壁上,因此在一些性交後超过五天的案例中使用含銅避孕器當作緊急避孕措施是可行的,前提是妳知道排卵的時間。

含銅避孕器最晚必須在排卵後的第五天使用。

關於含銅避孕器好處,除了作為緊急避孕非常有效之外,妳還可以把它留在子宮內當作一般避孕方式。如果妳不想把它當作一般避孕方式,可以快速地將它取出。



優點:高可靠性,未來五年能夠以避孕方式運行

缺點:不易取得、需要處方、必須由医生、護士或接生員放入





注意事項




很多人認為在服用恐慌丸後就絕对安全,但事实並非如此!

緊急避孕措施減少懷孕的風險,卻不像一般避孕方式來得有效。緊急避孕後進行驗孕非常重要。无论是否來月經我们建議妳去做驗孕測試。如果妳的伴侶或是有人服用了恐慌丸,妳能提醒她去做驗孕就太好了。

驗孕測試要有效的话,妳必須在使用緊急避孕措施後等待至少三週。在緊急避孕後馬上驗孕是毫无意義的,因為不可能那麼快就檢測到妳是否懷孕。

緊急避孕措施有副作用。最常見的是不規律出血。恐慌丸延遲排卵時間,同時也延遲妳的經期。不規則出血不會有危險,但是卻造成麻煩。幸運的是,這不是長期的問題,而且會沒事。有些人還發現,恐慌丸會讓她们覺得噁心。如果妳服用避孕藥後沒多久就嘔吐的话,妳必須使用另外一種。遵從藥丸的服用資訊与医生的指示進行。

含銅避孕器不含荷爾蒙,然而即便如此,它通常在一开始就改變妳的月經週期模式。如果妳想維持含銅避孕器當作一般避孕与經歷月經的變化,那麼我们建議妳三個月後看看有什麼樣的改變。經血規律通常會隨著一段時間慢慢穩定下來。





有其他更好的避孕方法吗?




我们談过許多關於自身的不同与各種避孕方法分別適合哪些女性,但並不表示所有的避孕方法都是一樣的好。而放在陰道裡的葉子与蜂蜜組合已經不再流行,以及根據安全期來避孕的方式導致許多意外懷孕的原因,就是使用方法的本身。

研究人員認為,目前世界上最好的避孕方法,就是植入式避孕,緊接在後的則是荷爾蒙避孕器。使用植入式避孕棒的女性懷孕的機率是最低的。很多人不知道避孕品質是如何測量的。我们该如何确定哪些避孕方法是最好的呢?植入式避孕棒比藥丸更好到底表示什麼意思呢?讓我们解釋一下:當我们說「最好」的時候,我们單純是指避孕方式運作有多好,也就是在防止懷孕方面上有多好。我们不是在談论副作用或者有多少人喜歡哪種避孕方法。无论喜歡与否都是主觀的。但是,避孕的效果有多好是客觀的,只要透过研究觀察有多少女性使用特定避孕方法而懷孕就能衡量。很難判定妳客觀上所偏好的選擇會是最好的避孕方式。我们的目標在於找到盡可能可靠而妳也开心的避孕方法

研究人員使用一種叫做佩爾指數(Pearl Index)的標準,用來評估和比較避孕方法的不同。佩爾指數是衡量避孕方法有效程度的指標。有效的評估,我们指的是避孕的效果⸺不是危不危險的問題,避孕不危險。

所以,佩爾指數是衡量有多少女性使用特定避孕方法而懷孕的指標。佩爾指數的确切、具体的條件為100名女性在一年當中使用避孕而懷孕的次數12 。

例如,如果妳想測試一種新型避孕藥的效果,妳請一群女性來測試藥丸,然後看看她们有沒有懷孕。根據許多研究結果,統計人員可以根據效果有多好來排名這些避孕方式。但是,是什麼原因導致各種避孕方式之間有差別呢?

有兩種因素能夠判斷避孕方法是否有效。第一個是使用的方式。因為有些避孕方法可能會不當使用,使得比正确使用效果來得較差。

就拿体外射精舉例。其目的是男性在高潮前將陰莖從女性陰道中拔出,最後讓精子射在床墊上、她的胸部或是其他有趣的地方。然而我们許多人會發現,通常拔出的時間點很容易發生在在高潮後,而非在高潮前。一時衝動的情形下,晚一點從陰道拔出实在很誘人,如果搞砸了一次,可能也足以讓妳懷孕。错误使用讓体外射精變得難以依靠,而且還遠遠不受医療專業人士与不希望懷孕的人青睞。即使非常妥善又有效使用,人類的犯錯能力總是令人唏噓。

避孕藥,避孕最常用的方法之一,當講到使用者错误時,它也是罪犯之一。在事实上使用者错误只是它的中間名。人们非常容易就會錯过一、兩顆藥丸;在某人床上醒來,離自己牙刷和藥包很遠的每個女性都可以感同身受。很多女性在无藥丸週仍服用藥丸也因此懷孕。她们擺脫每天服用藥丸的習慣,然後她们搞不清楚停用的時間应该要多久。任何人都有可能發生錯过服藥的情況。我们都有心不在焉的一天,但有些人卻是每一天都心不在焉。

而在另一方面,植入式避孕更加有效的原因在於它直接在妳的手臂裡發揮作用,妳什麼事也不用做。除了妳忘記更換以外,根本很難去忘記做过植入手術,而且僅僅三年才換一次。因此,植入方式不會有使用者错误。它運作得非常完美,不管妳的例行事項和記憶。

另一個判斷避孕方法是否良好的要素則是人们如何使用,而非方法本身。我们把這種要素稱為使用者错误。

有些人可能會認為不公平,表示避孕方法不好只是因為使用者搞錯;畢竟,並不是方法本身的錯不是吗?妳可能會這麼認為,但我们認為尊重避孕方式那不存在的心情一點意義也沒有。研究顯示,只要有這樣的可能性,往往會讓我们在最後做錯事,而這对避孕方法的有效性產生影響。在任何情況下,唯一的目的就是阻止妳懷孕,並不是讓妳最喜歡的避孕方式贏得人氣競賽。

确定有效性的第二個要素為避孕方法的实際品質。很多人認為結紮是不希望有(更多)小孩的最有效方式。當一個女人進行結紮,輸卵管會被切除,使卵子不能從卵巢傳送到子宮,雖然結紮後200名女性會有一名在隔年懷孕。无论是植入式避孕和荷爾蒙避孕器都比結紮更有效。這種類型的错误,和避孕本身的方法有關,而非使用它的人,因此稱為方法错误。

使用植入式避孕的人幾乎无人懷孕,不过在医学當中沒有所謂的对与錯。只要有人的地方就會懷孕,无论她使用哪一種方法。不幸的是,只要妳是与男性做愛的女性就无法說不可能;頂多可以說「幾乎不可能」,還比較可信。

由於有兩種不同類型的错误和避孕方式有關,使用者错误和方法错误,其效果也以兩種不同的方式進行測量。我们以「完美使用」和「实際使用」來區分避孕方法。完美使用意即使用避孕方法的人完全沒有發生失誤。沒有使用者错误,沒有錯过服用藥丸,沒有延遲陰莖從陰道拔出的時間,陰道環沒有在城裡喝醉時掉進馬桶。另一方面,实際使用的結果則是女性竭盡所能正确使用避孕方法,就和一般避孕使用者一樣,但還是會在过程中造成一些错误。

完美使用与实際使用的不同,取決於使用该避孕方式犯的错误有多少,在重要和不存在的範圍間判定結果。

如果妳的生活作息良好,如果妳沒有一絲輕率或心不在焉,如果妳有鋼鐵般的意志,例如,避孕藥,那麼妳會懷孕的風險在佩爾指數裡會比起「实際使用」更靠近「完美使用」。

只有妳最了解自己。但是如果妳的生活方式稍微難捉摸,值得考慮一種无论做了多少错误還是有用的避孕方法。沒有使用者错误的避孕方法,例如植入式避孕棒和含銅避孕器/荷爾蒙避孕器,在完美和实際使用的指標下一樣有效,因為无須做任何努力,实際使用仍然能完全發揮功效。

那麼,哪些避孕方法最好?妳會看到列有不同選擇方法的表格。這些數據是由世界衛生組織(WHO)提供。這些資料在二〇一五年更新,但有可能因為研究人員發現新的避孕方式或对現有方式進行新的研究而有所改變。

當妳在做決定時,它有可能幫助妳了解各種避孕方法的效果有多好。不过,我们建議許多女性可以嘗試的最有效方法則是:具有長效作用,沒有使用者错误的方式。





避孕方式的有效果





基礎体溫法与效果




我们特別想对避孕列表裡的其中一個方法提出評论。基礎体溫的方法因為自然循環(Natural Cycles)透过高調的部落客宣傳,已經在挪威与瑞典部落格圈引起熱烈討论。自然循環是一款提供以手機紀錄每日体溫測量來計算安全期的應用程式。

以此為依據,自然循環宣稱有效程度為99.9%。

我们不相信這個數字,而且也不是只有我们這麼認為。挪威医藥局發言人对廠商与宣傳自然循環的部落客採取強硬的態度回應。

從表格中妳可以看到根據WHO提供有效程度高達99.95%的唯一方式只有植入式避孕⸺是世界上最有效的避孕方法。下一個最有效的方法,荷爾蒙避孕環,有效度為99.8%。

根據WHO的資料,以实際使用來看,自然循環在基礎体溫方式上的最佳結果只有75%;換句话說一年當中100名女性當中有25人使用基礎体溫法而懷孕。這種方法從完美使用的角度來看可以提供達到99%的效度,100名女性當中只有1位懷孕;但是請記住,完美使用只是理论上的概念,不可能在廣大女性中達成。

即使是設計精美的應用程式,完美使用基礎体溫法的概念和自然循環对使用者的眾多要求实在難以期待不會犯出錯。事实上這個方法充斥著潛在的使用者错误,最明顯的是在错误的日子進行性行為的可能性。我们知道,有些使用此程式的情侶最終以懷孕收場,正是因為他们无法控制自己的一時衝動。

許多方法错误造成:為什麼有些女性永遠无法達成高效益,即使她们善於正确使用避孕方式。

如果妳有發燒、月經不規律或妳的月經週期过程中有額外的排卵情形,基礎体溫測量和安全期方法也就无效。這是妳无法控制的因素。

儘管如此,透过藉由應用程式的輔助实際使用基礎体溫法有可能降低懷孕的機率。

畢竟,應用程式消除了一些誤判和類似的可能性,並使用較早的溫度記錄、計算懷孕的可能。由自然循環資助的一項研究發現,應用程式在实際使用的效果從75%增加至92.5%。所以在一年的使用过程中有7.5%的女性懷孕。如果他们的數據是正确的,這差不多相當於避孕藥的实際使用效果。如果妳問我们,為什麼公司不在廣告上使用這個數據而宣稱99%的比例,我们也覺得有點奇妙。

希望使用自然循環的人首先需要有規律的月經週期,生活在極為有序的作息,有足夠的時間,每天早上測量溫度。必須要有鋼鐵般的意志,拒絕在错误的時間發生性行為(或必須使用保險套),同時必須做好可能懷孕的打算。如果這聽起來像妳一樣,沒有什麼能阻止妳嘗試自然循環或其他基礎体溫測量方式。如果妳想不惜一切成本避免懷孕的话,我们會建議妳選擇別的方法。





* * *



12	通常大家會有個誤解,理论上最高的指數跟百分比一樣為100。但是所有參与研究的女性懷孕的话,佩爾指數會來到1200。有點令人困惑但一點也不重要。除非妳是斤斤計較的書呆子,就和我们一樣。





荷爾蒙避孕对月經的影響




荷爾蒙避孕會影響妳的月經週期。因為每月的出血状况會改變,所以妳會注意到它。

大多數女性的經血量變得更少量或經期變短,但並非每個人都會如此。出血也可能變得不規律或完全消失。因為失去或避开月經的迷思讓很多女性覺得有點可怕。許多人認為,難道出血的現象不就是因為是自然的;我们的身体難道需要它吗?我们真的要对大自然這麼隨便吗?

如果妳還記得月經章節的內容,那麼經血对你来说有好處。如果妳正在使用荷爾蒙避孕的话的确是如此。使用荷爾蒙避孕,妳的月經週期不再正常,大多類型的荷爾蒙避孕會完全停止月經週期。所以,這種情況發生的出血也不再是正常月經出血,我们會稱為停藥性出血(withdrawal bleeding)。

我们先從使用複合藥丸時,妳的月經會發生什麼變化开始說起。五十年前設計避孕藥的研究人員規劃了每個月一次的无藥丸週讓女性可以有停藥性出血。他们認為如果荷爾蒙製造類似每四週出血的月經週期模式,那麼就更容易讓人们接受服用藥丸當作避孕的方式,但是即使模仿自然週期,卻一點也不「自然」。出血不自然,避开週期的用意也完全不自然。

通常是雌激素引起子宮內膜生長,這些黏膜就變成妳之後的經血。複合產品的雌激素使子宮內膜每個月增加一點點,所以即使沒有正常月經週期,使用複合產品的女性在停用荷爾蒙藥丸、避孕貼片或是陰道環7天或更少天數的情況下會發生停藥性出血。子宮內膜生比平常增生得少,這就是為什麼經血量不需和暫時停用避孕時一樣。对許多女性来说一個月一次可能就太多餘了。

只要妳隨心所欲就可以使用複合產品避开月經多次,甚至連續服用藥丸或讓它出血,只要適合妳都行。

這既不危險又有效。複合產品的黃体製劑約束了子宮內膜,所以經血不會流出。

如果妳常常使用複合產品並避开停藥性出血,妳最後有可能得到所謂的突破性出血(breakthrough bleeding)。

黃体製劑盡可能約束子宮內膜,但最後太过頭了。突破性出血意味著妳在荷爾蒙作用的情況下流血⸺換句话說,就是在使用避孕藥、陰道環或是避孕貼片的短期停藥之外發生的出血情況。可能會是斑點狀,即不規則輕微出血(通常是妳內褲上的斑點),或更嚴重像月經一樣的出血量。這是正常的,而它的意思是:是時候休息7天的時間啦!之後妳可以回去繼續避开停藥性出血。

許多女性認為每個月的出血在使用荷爾蒙避孕時可以知道自己是否懷孕,而避开太久可能會降低潛在懷孕的機率。這個想法並非完全正确。事实上即使在中途开始无藥週,它還是有可能完全停止出血。這不代表妳懷孕了。更重要的反而是,它可能會在懷孕期間輕微出血。荷爾蒙避孕出血通常和一般月經不同,是輕微的出血。所以,即使妳在无藥丸週出血也有可能懷孕。重點在於妳应该仰賴妳正在使用的避孕方法。如果使用正确,複合避孕法是有效的,但是如果有任何的變化讓妳懷疑自己懷孕了,唯一的方法就是去做驗孕測試。

很多女性深受頻繁突破性出血受苦,這可能會成為長期的折磨。有些人可能發現改變自己的避孕方法會有改善的效果。如果妳在服用避孕藥,從低劑量的雌激素到稍微高一點的劑量就有可能改善。最高劑量的雌激素藥丸最擅長控制出血。例如很多女性會感受到欣无妊或奧諾康(Oralcon)的控制出血效果比樂婷錠(Loette 28)要來得好。妳可以和医生討论应该換成哪種方式。

使用黃体製劑避孕与複合避孕下的月經之間有很大的不同。主要的區別在於,妳不能決定自己的月經週期,也不能改變或一直控制。這是因為妳每天服用相同劑量的荷爾蒙,沒有休息过。如果妳稍作休息,就會失去防護。這意味著,黃体製劑不再約束子宮內膜,妳隨時會流血。當妳使用黃体製劑避孕卻一直流血的话,实際上,就是突破性出血,因為沒有出現停藥性出血的間隔時間。

黃体製劑約束子宮內膜,讓經血變得更難滲出。同時黏膜變得比平常更薄。由於黃体製劑當中沒有雌激素,所以沒有任何東西可以傳達生長的指令給子宮內膜。最後轉變成沒有完全不出血的可能性,許多女性确实會出血。畢竟,雌激素在体內自然發生了。

當妳开始使用黃体製劑避孕時會有點像把月經週期當成俄罗斯輪盤在玩。妳无法事先得知結果,但是它會有三種選擇:經常出血、不會出血或是不規則出血。

很多人認為无论使用植入式避孕或荷爾蒙避孕器來停止月經的女性,都是因為可以停止經血而選擇這種避孕方式。但是這並不完全正确。很多女性最後沒有出血,但也有可能會是極不規則的出血或是以完全正常的週期結束。无论如何出血量都會小於不使用荷爾蒙避孕的方法。

至於複合產品,在妳使用黃体製劑避孕仍流血時並不能排除懷孕的可能。我们收到每三個月規律地做驗孕測試女孩所提出的疑問,她们因為避孕的效果而停止了月經。這是不必要又很花錢的一件事。使用黃体製劑避孕時出血並非驗孕的好指標。如果妳避孕失敗或不确定是否受到避孕防護才要進行驗孕測試。

雖然含銅避孕器不是荷爾蒙的形式,妳卻有可能發生和月經有關的副作用。不像荷爾蒙避孕導致輕微的出血,很多人會發現當她们使用含銅避孕器時出血加重,經痛也更嚴重。对先前受过嚴重、長期或疼痛出血的女性来说更是如此。多達十分之一的女性在使用含銅避孕器第一年因為這些問題才決定移除。





我要如何避开月經呢?




有時候,月經週期不太方便。有可能是因為妳要去海灘度假,和伴侶一起去小屋滑雪之旅,或是因為在考試前一週妳无法承受經血与經痛帶來的麻煩。這些都是所有有月經的女性會發生的煩惱,特別是那些受到嚴重出血和劇痛的人。當你覺得不方便的時候,可以盡量讓子宮延遲出血。

使用複合避孕法一直是最容易延遲出血的方式⸺也就是複合藥丸、避孕貼片或陰道環。其他人可能使用處方藥物來延遲月經。



以下是使用複合產品的注意事項:



單相型複合藥丸:通常妳會服用含有荷爾蒙的藥丸21〜24天,接著休息7或4天,依據你所使用的單相藥丸種類來決定天數。在停用藥丸的日子裡妳就會流血。如果妳想避开出血,一旦妳完成目前循環的所有荷爾蒙藥丸後可以直接服用新的一包。所以,如果妳使用的是21天份的荷爾蒙藥丸(例如欣无妊或樂婷錠),就不會有无藥丸週。如果包含糖衣錠在內,數量達到28顆的话,妳可以把剩下的扔掉。如果妳用的是24天份荷爾蒙藥丸加上為期4天停藥日的優悅或佐宜,妳可以跳过休息間隔,直接再服用新的24天份。如果妳正在服用多相藥丸如Synphase和Qlaira,妳也可以避开經期,但在這種情況下,妳會需要稍微更詳細的說明。我们鼓勵使用這些藥丸的人如果有任何問可以去看医生或護士,並詳讀服用資訊。

陰道環:通常使用三週休息一週,我们可以稱停用的日子為无環或无荷爾蒙週。在這一週當中妳會出血。如果妳想避开,可以在三週後跳过无環插入新的避孕環到陰道裡。

避孕貼片:避孕貼片通常每週更換一次並維持三週,第四週為无貼片週。在這一週裡妳會出血。要跳过出血的第四週只要換上新的貼片就好。



如果妳不使用複合產品的话會是這麼運作:



Primolut-N藥丸含有延遲月經的荷爾蒙。這可能是不希望使用含有雌激素避孕卻想要避开月經一、兩週女性的解決方案。

在月經來的前三天妳开始服用Primolut-N:每次一錠,一天3次。這代表妳在月經來臨時必須徹底監督。在沒有正常月經週期下使用Primolut-N很難成功發揮效果。之後,只要妳想延遲月經,妳需要每天服用三顆,最多用到14天。一旦停止服用藥丸,妳會在2天後出血。換句话說,妳不能无限延後月經。

Primolut-N可以透过医師處方獲得。除了一些女性,大多數人可以使用Primolut-N;妳的医生會向妳解釋。當妳服用Primolut-N時使用保險套來保護妳避免懷孕很重要,因為這種藥沒有避孕的效果。一次療程的費用為65克朗。





使用避孕藥的最佳方式?




避孕藥可能有很多麻煩,卻仍然是受歡迎的方式。正如妳在前面章節中所看到,使用避孕藥是有可能懷孕的,主要是因為很容易使用不當。

最酷的地方是,使用避孕藥懷孕的風險較低,降低異常出血的機率与出血量更輕微。這個方法適用於所有複合產品。也就是說,妳使用避孕貼片和植入式避孕棒會有一樣的效果。使用多相藥丸的人必須遵照医生、護士或接生員的指令。

只要妳正确使用避孕藥与其他複合產品,它们會一直保有效果。就妳所知,複合避孕法有設置停用期間。妳可以使用荷爾蒙三週(21天),接著无论不服用任何藥丸或是服用糖衣錠停用荷爾蒙一週(7天)。在這7天裡,妳會有停藥性出血。如果使用優悅或佐宜妳會服用荷爾蒙藥丸24天,並有為期4天的休息期間。

在提到複合避孕法時,21+7或24+4是非常重要的數字,因為這些組合列出了兩個重要的限制。

當妳使用複合避孕時,為了使避孕方式有效,妳必須服用至少21或24天的荷爾蒙藥丸。如果妳連續使用少於21或24天的藥丸(例如,如果妳忘記最後兩顆,總共才服用19或22天,而不是21或24天)妳會失去防護並开始排卵。然後,妳有可能懷孕。因此21或24天的荷爾蒙藥丸意味著至少服用21或24天。延長使用荷爾蒙的時間沒有任何的問題。只要結束妳規定服用的天數,可以連續服用30、50或100的藥丸。這完全取決於妳。

7會是一個限制天數(如果妳使用佐宜或優悅的话則是4),這代表停用的休息期間最多為7或4天。不得超过這些天數。如果停用更久的時間,妳會失去避孕的防護。休息大約3天的時間則沒有問題。如果,例如,妳有2天的短暫出血,妳可以休息2天後重新开始服用荷爾蒙藥丸。但妳絕对不能休息超过7或4天。如果這麼做,妳可能會排卵,然後陷入懷孕的風險。

只要妳使用荷爾蒙藥丸至少21或24天,並達到單相藥丸、避孕貼片或陰道環7或4天的最大休息限制,妳可以使用任何的複合產品。既然有這麼多因為搞混无藥丸週而意外懷孕的状况,降低无藥週的天數會是個好辦法。实際上這麼做會讓避孕更加有效。

一旦妳避开一定的月經時間,妳可能會發生突破性出血。需要的话妳可以藉由持續使用複合藥丸並休息一陣子來解決這個問題。

這樣一來,妳可以量身打造適合自己的月經週期,盡可能減少出血量。

持續服用荷爾蒙直到出血,接著休息一下撐过出血的日子。停藥的休息時間很有可能短於7或4天。休息过後,重新开始服用荷爾蒙直到出現下一次突破性的出血。只要妳不要服用少於21或24顆藥就絕对沒事。如果妳在服用十幾天的荷爾蒙藥丸後出血,為了避孕的果妳必須繼續服用藥丸直到第21或24顆為止。





荷爾蒙避孕⸺是不是很危險?




妳可能注意到多年以前關於「自然」的新理想。像是排毒、对羥基苯甲酸酯(parabens)、蔬果汁和超級食物等奇怪的名詞已經不足為奇。那些自稱健康大師所說的话是非常明确:「人造」添加物对妳的身体沒有好處。妳不应该招惹他们。

一夜之間,蔬果汁已成為最熱門的時尚配件,在同一時間,荷爾蒙避孕也不再流行。年輕女性因為擔心可怕的副作變得害怕使用避孕藥。我们聽到有愈來愈多人說她们有服用荷爾蒙避孕藥的不適症狀,就好像过敏一樣。還有人詢問停用荷爾蒙休息、排毒、沖掉身体內的非自然物質是否健康。

在愈來愈注重純淨和自然的時代,很多人仍然覺得医生沒有認真看待她们对副作用的擔憂⸺医学專家忽視她们的問題或者試著掩蓋一切。結果造成許多女性对避孕方法安全与否感到无法放心,因而從不可靠的來源寻求資訊。

大約三分之一的女性在开始使用避孕藥的六個月後便停止服用。其中,約有一半的人會這麼做的原因在於她们經歷了一些副作用。如果妳不明白為什麼會發生或意義為何的话,它會讓妳因為感受身体的變化而恐懼。我们認為妳应该要有正面与反面的荷爾蒙避孕資訊,以便替妳身体做出正确的選擇。知识孕育自信。

同時在近期出現的可怕現象中帶來了一些不同。現在媒体能夠給大眾的印象是,我们不知道荷爾蒙避孕的副作用為何,就像我们以年輕女性的健康為賭注來玩俄罗斯輪盤一樣。好在這是错误的。妳可以自信地認為,妳在藥局拿起的藥丸是世界上最認真研究的藥物之一。研究人員自一九六〇年代起以驚人的統計數據研究數百萬地球上廣大女性服用藥丸的情況。

尤其是當妳想到第一顆上市的藥丸大概是我们現在的五倍大時,倘若荷爾蒙避孕存在潛在未知效果的话,可能早在很久以前就已經被發現了。





什麼是副作用?




在我们可以开始談论個別副作用前,我们需要了解何謂副作用。藥物的目的在於对身体具有特定的效果,這就是為什麼我们會去服用。在荷爾蒙避孕當中,我们服用的原因是希望能夠預防懷孕。副作用是所有其他藥物对身体的影響,可能是正面与負面的影響。例如很多女性發現自己使用荷爾蒙避孕時斑點減少很多,這被認為是一個正向的副作用。而在另一方面沒有人會希望得到血栓的副作用。

在一九九八年的電影《雙面情人》(Sliding Doors)裡我们跟著葛妮絲‧帕特洛(Gwyneth Paltrow)的活在兩個平行時空的命運當中:在一個場景裡,某天早上她趕上列車去工作;而在另一個時空裡她錯过了。這個小細節深深影響她生命原來的運作模式。這也是我们身体運作的方式。我们的身体是如此的複雜,複雜到不可能同時在缺乏身体其他部位的連鎖反應下影響單一的功能。副作用不代表藥物就是有害。它代表著本身的運作效果。如果有人曾聲稱藥物或療法沒有副作用,那麼妳应该对此抱持半信半疑的態度。這表示他们不是在撒謊就是该物質沒有任何的效果。

医生和衛生當局都十分關心副作用。我们知道它们是不得已的產物,不过我们的目的是讓副作用盡可能處於低水平的效果。這就是為什麼藥物極難得到銷售批准。首先生產者必須證明藥物的正面影響比負面影響有更大價值的可能性。正是因為我们必須知道在妳服用藥物時,能夠預期它的副作用到底為何,所以任何新藥物的背後都有多年的研究和实驗掌控一切。

上市的藥物經过中立於医藥產業的挪威藥品局嚴格監控,所以才能提早偵測到任何未知的副作用。如果妳受到副作用的影響,妳和妳的医生可以將向挪威藥品局回報(我们絕对鼓勵妳這麼做)。如果有嚴重的副作用被忽視了(比如多年來使用避孕藥引發癌症的人),藥品局將會展开全新的調查。

這麼做往往將服用藥物的廣大族群与不使用藥物的相當族群互相比較。然後研究看看在那些使用藥物的人當中是否有更多的人有潛在的副作用。如果事实證明這兩個族群裡有很多人得到癌症的话,他们就會知道這些藥丸不會導致這種類型的癌症,因為在有副作用的情況下,只會看到更多服用藥物的人罹患癌症。





反安慰劑效應




為什麼當許多女性回報相同的藥物副作用時會沒有人下意识相信呢?當女性表示受到副作用時,健保局難道不相信她们吗?為什麼他们在沒有更進一步調查下不能接受副作用确实存在的事实,是因為一種叫作反安慰劑效應(nocebo effect)的現象。

大多數人都聽說过安慰劑⸺意即人们在事物沒有确实運作下体驗到真实、正向的影響,因為他们期望它會發生。舉例来说,有一個原因解釋了許多藥物為何以鮮豔的膠囊包裝:研究發現,如果人们服用外觀精緻的藥丸會變得更健康!這也是為什麼医生會穿上白袍,通常將聽診器掛在脖子上的原因之一。外套和聽診器对患者創造了治療与專業能力的聯想。這有助於改善患者的健康。

反安慰劑效應一詞,出自於拉丁語「會傷害我」的意思,卻与運作方式恰好相反。因為妳相信糖衣錠內含正向物質所以會導致生理上的問題。事实上有四分之一的患者在接受安慰劑治療時(也就是沒有任何治療)經歷負面的副作用影響。如果一個医生告訴病人该藥物可能有特定負面影響的话,同樣的事情仍會發生。比平常有更多人回報這個影響,但实際上卻是藥物的錯。可能是人们將原因一味歸咎於藥物的一般症狀。賴登堡和洛文塔爾(Reidenberg and Lowenthal)的一項研究發現,只有19%沒有服用任何藥物的健康受測者在前三天完全沒有任何問題。然而,之後有39%的人經歷疲勞,有14%的人有頭痛的現象,而5%的人則感到眩暈。

從耶魯大学的一項研究發現,受过高等教育的女性容易高估荷爾蒙避孕的危險。同時,她们不知道荷爾蒙避孕提供了所有正向的健康優點⸺例如,降低卵巢癌和子宮內膜癌的風險。這些負面的預期可以成為自我实現的預言。

考慮到這一點,可能更容易理解為什麼医生會在很多女性突然回報舊藥,如避孕藥有新的副作用時抱持著懷疑立場。這可能只是太多負面宣傳所導致的結果。經过更多的研究才能找出该現象是否發現到真正的副作用或單純只是反安慰劑效應。





凡事都有風險




如果妳使用荷爾蒙避孕,會先從取出服用資訊的冊子开始。在那裡,妳會發現一長串的副作用列表,根據常見次數做排序。第一個是最常見的,影響程度介於10%到1%之間。這些副作用包括了頭痛、情緒不穩与乳房疼痛。之後的副作用則介於1%到0.1%之間。隨著往下閱讀妳手中的列表,愈來愈令人不安。

首先要知道,妳閱讀的服用資訊是誰寫的:藥物製造商。人们接著會想到,也許他们正試圖向我们隱瞞其中的副作用,但事实正好相反。當提到可能出現的副作用時,他们會过度誇飾,這樣就不會被不滿意的消費者告上法院。有些在服用資訊冊內所描述的副作用与使用荷爾蒙避孕的女性回報的相符,但尚未被證实是由藥物引起的現象。我们之後會回來提到這個部分。其他我们知道的副作用,是由荷爾蒙避孕引起的。

妳必須清楚的另一件事則是对風險一詞的理解。當我们聽到風險這個詞很容易認為是危險的東西,然而实際上只是出事的機率。

到了簡易統計课程的時間。當我们提到副作用,也就是所謂的相对風險,往往獲得所有的注意力。相对風險是以妳服用与不服用藥物之間作比較來計算副作用增加的機率。

所以舉例来说妳可能閱讀过使用避孕藥比沒有使用的人造成血栓的機率是二到四倍以上的相關研究。這聽起來頗有戲劇性。想像一下小報上的頭條寫著:「造成生命危險的避孕藥!造成血栓的可能性為四倍!」但实際上卻一點也无戲劇性可言。

对我们人類来说最有趣的真相稱為絕对風險(absolute risk)。不过小報对絕对風險的數字不感興趣,因為它往往造成无聊的新聞頭條:「避孕藥造成血栓的可能性極小!倒楣少女意外得到血栓!」絕对風險單純是妳使用後实際會有的副作用,以避孕藥為例,就不會与沒有服用的人相做比較。這使妳更能理解接所触到的危險和真实的情況。

妳服用避孕藥造成血栓的可能性有多大?雖然相对風險顯示避孕藥的使用者造成血栓的風險比不服用的人高於二到四倍,确实造成血栓的可能性,意即絕对風險,每年則是介於0.0005%〜0.001%。

這代表服用避孕藥的女性每年10萬名就有50到100人會造成血栓。換句话說,即使服用避孕藥妳還是會發生血栓這種令人難以置信的不幸。





荷爾蒙避孕的一般副作用




現在,我们对副作用的背景有一點了解,我们可以开始著重在荷爾蒙避孕的地方。讓我们先從最常見的事情开始:副作用如頭痛、頭暈及乳房疼痛影響的幅度為1%至10%。這些都不是危險的副作用,但仍然是件麻煩事。沒有人會得到所有的副作用,許多女性也不會經歷全部的副作用。实際上只有1〜10個人才有這些副作用,同時意味著有90〜99人沒有這種情況。

同樣重要的是,要明白常見的副作用与危險副作用之間沒有任何關聯性。如果妳受到常見副作用的影響,那麼妳就不屬於危險副作用的高風險族群。

常見的副作用往往經过數個月的時間才會發揮效果,所以我们建議放棄現行方式,嘗試新的避孕方式三個月再換下一種方式。如果妳還是覺得受到副作用的困擾,妳可以嘗試其他品牌或避孕方式。

事实上,人们对各種品牌和方法的反應不同。帶給妳朋友強烈頭痛的產品可能非常適合妳。只有自己試試看才會知道好不好用。正如我们前面所解釋的,不同的黃体製劑產品,在我们身上的運作模式也略有不同。此外還有使用包含純黃体製劑的避孕方式,如荷爾蒙避孕器与植入式避孕器,或是含有雌激素的複合產品都有不同的差異。即使妳在一個產品上得到許多副作用,並不表示妳对一般荷爾蒙避孕法具有「不耐性」。不會对妳造成影響的其他方法存在著極高的可能性。妳只需要确保自己選擇了不同種類的黃体製劑避孕方式;妳的医生可以幫助妳判斷。

含有雌激素的避孕藥特別會有常見的副作用。其实這些都很容易在妳懷孕時可以感受到!首先列表上有噁心、頭暈等症狀。至於孕婦很快就能適應,但如果妳在一开始的症狀就感到非常困擾的话,那麼在用餐或睡前吃藥可能是明智的選擇。

雌激素還會導致分泌物增加。看起來或聞起來和一般的沒有什麼不一樣;只是數量變得更多。少數人腿還會抽筋。我们不知道為什麼會出現這種情況,但我们知道它不危險。乳頭溢出少量的乳汁也是其中一個不太常見的副作用。

雌激素避孕的另一個副作用則是色素沉澱。雖然這是使用雌激素避孕的女性才會經歷的副作用,但主要原因可能是避孕藥裡的黃体製劑所引起。色素沉澱,嚴格来说稱為肝斑(melasma),為皮膚上顯現之較深的棕色斑塊。无论戶外或是日曬機,只要曬太陽就會產生斑塊。正常来说懷孕期間才會色素沉澱,同樣也是由荷爾蒙引起。如果妳有這方面的問題,高防曬係數乳液有助於防止色素沉澱。另一種替代方案是嘗試含有不同黃体製劑的避孕藥來看看是否有幫助。

雌激素同樣也有正向的影響。妳可能聽別人說过,懷孕婦女容光煥發。皮膚更加透亮其实是雌激素的正向效果。如果妳有痘痘的問題,複合產品可以提供幫助。然而,純黃体製劑的避孕藥卻產生相反的效果,反而造成皮膚、頭髮油膩与痘痘。对有些選擇避孕方式的人来说可能是個重要的因素。

含有雌激素的避孕藥,其实經常被用來治療多囊卵巢綜合症(polycystic ovary syndrome)的女性,我们會在後面的章節談论這個非常常見的情況。

雌激素避孕藥的另一個正向副作用是,它们讓妳控制自己的月經週期。這表示經痛減緩,在衛生棉條的花費也降低了,再也不是因為經前症候群而歇斯底里的月經潑婦或是亂吃巧克力的胡鬧嬰儿。

另一種早期更常見的副作用為水腫(oedema),這是腫脹的医学術語。簡單地說,就是身体裡的積水。雌激素和黃体製劑可能都是罪魁禍首,因此所有的荷爾蒙產品都有這樣的效果,並非只有複合產品。女性开始使用荷爾蒙避孕法時會覺得自己体重增加,液体滯留是其中一個原因,但妳就本沒有變胖:妳只需要喝下額外的水即可!

服用荷爾蒙避孕藥會導致妳發胖的迷思是存在的。在這些理由當中,這個迷思引起很多女性开始在日常生活中避孕,特別是身体正在經歷劇烈變化的階段:青春期。另一個原因可能是很多女性找到伴侶後体重增加了一些。然後,她们認為這些幸福肥是因為避孕的關係,完全忘記她们花了更多時間在沙發上相擁,並在腿上放一袋零食看著五季的《權力遊戲》(Game of Thrones)。其实妳不會因為服用荷爾蒙避孕藥而發胖,只是這一切都太容易怪罪到避孕藥身上。

妳的乳房也會有液体堆積,而且變得愈大愈敏感。另一個有點奇怪的副作用則是配戴隱形眼鏡的人可能會發現鏡片突然不合適。這是因為一些額外的水分累積在眼睛裡,所以改變角膜的形狀。体內水分的增加也可能導致頭痛。

許多使用避孕藥、貼片或陰道環的女性在出血的那週只會有頭痛的症狀,也就是停止使用荷爾蒙的那一週。這是很常見的,有點像妳早上還沒有喝到習慣會喝的咖啡前的頭痛。頭痛的徵兆表示妳失去平常會得到的東西,在這個情況下就是荷爾蒙。為了減少這些痛苦,妳可以直接跳过或縮短无荷爾蒙日的天數。正如我们前面提到的,沒有特別規定妳应该要休息7天。只要妳不超过7天,妳就能自行決定。妳使用的是黃体製劑產品就不會有這個選項。

如果妳使用的避孕措施僅包含黃体製劑,例如植入式避孕棒、荷爾蒙避孕器和无雌激素避孕藥,妳不會得到我们之前提到的雌激素所帶來的副作用。妳也不會有任何的雌激素的正向效果,例如更明亮的皮膚与經期的控制。黃体製劑实際上可能造成膚況不佳,而且在某些情況下,還會增加毛髮的生長。

所有使用這些避孕方式的女性所經歷最重要的副作用大概是出血變化。

這是完全无害的,但還是有人覺得困擾。有關變化的幅度,則依據個人与所使用黃体製劑避孕類型而有所不同。直到試过之前妳根本不會知道自己會不會起反應。有些女性會完全停經,而其他人則可能有更頻繁的輕微出血或不規則出血。多數人比平常經歷更輕微的出血,可能會持續更多或更少天。當妳使用避孕措施三到六個月後狀態會趨於穩定,所以妳得学會認识屬於自己的避孕方式型態。

儘管植入式避孕和荷爾蒙避孕器經常造成出血變化,卻仍然是我们強烈推薦的兩種避孕方法。它们在佩爾指數得到最佳的分數,因此是避孕最有效的方式。荷爾蒙避孕器与其他形式的避孕方法相比,有著低得令人難以置信的荷爾蒙劑量。有些人認為荷爾蒙避孕器因為已經行之有年,所以能夠提供身体更多的荷爾蒙,但事实並非如此。使用最小的荷爾蒙避孕器時,血液中的荷爾蒙濃度居然是如此之低,這相當於每隔一週服用一顆小藥丸!有些人認為低濃度的荷爾蒙能降低發生副作用的機會,但目前尚未得到證实。即便如此仍然值得嘗試,尤其妳无法使用其他避孕方式的话。





罕見副作用




現在我们直接來到服用資訊冊上的副作用列表。這些副作用一年可以印在小報的頭版好幾次,沒有別的,報紙有的是疾病和死亡的恐懼。嗯,大概除了性愛。如果妳懷疑的话,医生和製藥公司之間沒有所謂用荷爾蒙來威脅健康年輕女孩生命的陰謀。甚至還有研究測試可以證明!哈佛大学耗時三十六年研究12萬名女性服用避孕藥的長期影響。他们的結论是經常使用避孕藥的女性(或者妳很喜歡使用),很少死亡,機率和不使用荷爾蒙避孕的女性一樣。在任何情況下,我们可以把死亡踢出我们的擔憂名單。





血栓




儘管如此,使用含有雌激素的避孕方式确实有嚴重的副作用,雖然它们極為罕見,但我们仍須說明。通常最受關注的是血栓的發生。

血栓是當我们的血液凝固,在血管上出現一個或多個腫塊。血管內的腫塊停止血液流動⸺最常出現在腿部和骨盆的大靜脈當中。靜脈,与動脈相反,是血管從妳的器官和四肢攜帶血液回流到心臟。医生稱這種状况為深靜脈血栓形成(deep vein thrombosis)。

我们的腿會得到血栓的原因是,當血液被輸送回心臟時需要花很大的力氣抵抗重力。血液仰賴肌肉收縮的輔助來增加速度,有點像是幫浦。比方說我们在飛機上的航程坐了很長一段時間,血流速度可能會过慢。如果妳真的不走運,可能會开始凝固。如果腿部有血栓,妳會發現它腫起來,呈現紅色並有疼痛感。

為什麼人们害怕腿部血栓的最主要的原因是,部分血塊可能會鬆動。接著經由血液衝回心臟,然後,進入肺部。由於肺部血管較小,血塊可能會卡在那裡,導致呼吸道問題。這個現象稱為肺栓塞(pulmonary embolism)。雖然可能會很嚴重,但很少致人於死地。肺部血栓的一個跡象是,如果妳胸部出現突然的刺痛感,吸氣也會變得更糟糕。我们有時候胸口會有點刺痛,通常是因為肋骨壓迫肌肉所致,但肺栓塞造成的疼痛不會消失。同時,妳可能會呼吸急促並引發咳嗽。如果妳懷疑有血栓,記得立刻到急診室或進行緊急手術治療。

正如之前学到,避孕藥所含的荷爾蒙類型有所不同。只有含雌激素的避孕藥才有增加血栓的風險,包括避孕藥、避孕貼片和陰道環。我们在風險的部分提到,妳使用複合避孕得到血栓的風險會上升兩到四倍。為什麼我们說兩到四倍的原因是取決於妳使用的類型?現在以雌激素為基底的避孕藥在市面上流通,一種是得到血栓的可能性最小、含有左炔諾孕酮的黃体製劑。挪威市面上有三種不同類型的左炔諾孕酮藥丸:欣无妊、奧諾康和樂婷錠。奧諾康和欣无妊完全相同,而樂婷錠的雌激素劑量稍低。如果妳第一次打算使用避孕藥,我们會建議在這些類型選擇一種。

因為有增加血液凝塊的風險,所以有些女性不應使用含雌激素的避孕方式。最重要的族群則是基因上缺陷影響血液凝固功能的女性,例如被稱為萊頓突變(Leiden mutation)的情況。這就是為什麼医生會在妳开始使用複合避孕法時詢問妳的父母或兄弟姐妹是否有过血栓。

我们在前面提过,健康的年輕女性得到血栓的風險是非常小,不管她们是否使用雌激素的避孕法,絕对風險小。如果10萬名女性在一年當中服用避孕藥,會有40〜100人得到血栓。如果她们沒服用,仍然有20〜50人得到血栓13 。這是不正确的,避孕藥中的雌激素比在体內的「自然」雌激素更危險。製造大量雌激素的孕婦比避孕藥使用者的血液凝塊風險更大。為了比較,10萬名女性中高達200人在懷孕或生產後得到血栓。

換句话說,妳意外懷孕比妳正在使用避孕藥得到血栓的可能性更大。懷孕時身体自然增加的荷爾蒙比我们為了防止懷孕所增加得更多。這就是為什麼我们使用避孕藥時应该接受血栓風險略微增加的最重要原因之一。懷孕更加危險。





中風与心臟病




雌激素避孕藥另外的嚴重副作用是中風和心臟病。這些是影響動脈的疾病,也就是血管帶著富含氧氣的血液從心臟流到我们的器官。當血流停止,无论是血栓或血管破裂,血管組織可能會因為缺氧而陣亡。這意味著心臟的一部分缺氧而死。顯然,這種傷害的後果可能會相當大。

一項於一九九五年到二〇〇九年之間对所有丹麥女性的研究發現,使用雌激素避孕藥的中風和心臟病風險約為兩倍。然而,請記得相对和絕对風險之間的區別:雖然增加了一倍聽起來很誇張,但這些都是很少發生在年輕女性身上的疾病。即使風險加倍,妳會中風的可能性還是最小。

為了說明,我们回到相同的研究。一年裡使用避孕藥的10萬女性中,大約20位中風,10位得到心臟病。這包括所有使用避孕藥的丹麥女性的類型:胖与瘦、吸煙者和非吸煙者、老年与年輕。如果只是調查健康的年輕女性,風險甚至更低。

有些女性不應使用含雌激素的避孕藥以減少中風和心臟病的風險。例如三十五歲以上患有高血壓或心臟疾病的吸菸女性,或罹患糖尿病二十年以上。不應使用雌激素避孕藥的另一族群則是有前兆性偏頭痛的女性。但是,如果妳的偏頭痛沒有前兆性的话,只要妳在三十五歲以下,便能使用雌激素避孕。

如果妳接触到太多可能導致中風和心臟病的風險因素(例如超重、高膽固醇和吸煙)妳的医生在安全上可能會建議妳選擇另一種避孕方式。長话短說,如果妳還年輕、健康,即使使用雌激素避孕,也沒有必要擔心中風与心臟病。





癌症




我们要談的最後一個副作用為癌症。在某些圈子裡,其实還有人相信避孕藥會致癌。我们先從強調使用避孕藥等其他荷爾蒙避孕法不會增加妳人生罹癌的可能性开始。從整体上來看,避孕藥其实降低患癌症的風險。它们似乎保護我们抵抗腸、膀胱、子宮和卵巢的癌症。這些類型的癌症常常發生在女性身上。

在妳停止服用藥丸後,避孕藥可以預防卵巢癌長達三十年之久。如果這個數字正确的,研究人員認為避孕藥在的未來幾十年內每年可以阻止三萬件卵巢癌的案例!以族群為基礎的研究顯示,避孕藥可以防止子宮內膜癌至少十五年,得到這種癌症的風險比沒有使用荷爾蒙避孕的女性相比,減少了一半。有些研究人員已經清楚傳達其中的意涵:避孕藥可以預防婦科癌症,這個正向副作用勝过所有的負面影響。

然而,避孕藥似乎會增加子宮頸癌的風險。该領域最優秀的研究顯示,使用十年避孕藥後,子宮頸癌的發生率每千名女性當中從3.8%提升至4.5%。風險隨著妳使用避孕藥的時間增加,但一旦妳停止了會再次下跌。停止服用避孕藥的十年後,風險回到和开始時一樣的水平。

問題是,不可能肯定地說,避孕藥本身增加罹患癌症的風險,因為使用的女性更容易感染HPV⸺也就是導致子宮頸癌的病毒。更容易病毒感染的原因是許多服用荷爾蒙避孕藥的女性与新伴侶進行性行為時,在使用保險套上變得更加寬鬆。甚至發現,使用這種避孕方式的女性性行為次數更多⸺畢竟,這就是為什麼她们會使用避孕藥的首要原因。

乳癌是在提到癌症与服用避孕藥之間關聯時人们想知道的最後一種病症。我们知道有些特定類型的乳癌稱為「荷爾蒙敏感型」⸺這代表它们像雌激素一樣,為了癌細胞生長所需。複合避孕藥的确含有雌激素,這可能會使妳認為雌激素避孕藥幫助「飼養」這種類型的癌症。

幸運的是,運作的效果並不完全。多數大型研究在乳癌和服用避孕藥之間尚未找到任何的關聯,只有少數例外。個別研究發現一九六〇和七〇年代使用最高劑量避孕藥的女性罹患風險略微增加。不过,專家認為,現在的避孕藥和其他含有這種低劑量荷爾蒙的複合產品影響乳癌風險的可能性很小。

總結:避孕藥等複合產品的出現能夠保護女性免於一些普通或嚴重類型的癌症所苦。如果妳正在仔細評估荷爾蒙避孕的整体重點,這是值得考慮使用的。不幸的是,這些重要、正向的副作用与罕見、危險的副作用相比,得到媒体的關注太少。





我们不确定的事物




如果妳讀过避孕藥的服用資訊冊,也許會感到驚訝,我们忽略了兩個重要的副作用:情緒波動与性慾降低。我们還沒有提到的原因是我们認為不重要⸺正好相反。這些副作用是研究人員最不确定的東西。然而這兩種可能的副作用在近幾年已獲得女性愈來愈多的關注,所以我们認為它们应该得到充分研究。

天然性荷爾蒙影響大腦調節情緒水平和性慾的區域。女性的情緒會根據月經週期的荷爾蒙變化波動,這是一個眾所周知的事实。有些女性發現她们接近排卵期時性慾特別高漲。甚至觀察發現,女性在排卵前後更加不忠!

考慮到這一點,改變性荷爾蒙平衡的避孕方式同樣也能对心理和性慾產生影響並不奇怪。女性和許多医生之間也逐步形成廣泛的共识,服用荷爾蒙避孕藥會引起情緒波動、煩躁,在最坏的情況下,則會罹患憂鬱症。心理和其他非特異性副作用為女孩放棄服用避孕藥的主要原因。

即使女性之間存在這項共识,研究人員仍繼續努力。一些研究試圖證明服用荷爾蒙避孕藥对女孩的情緒有負面影響,但沒有成功。以下有幾個可能的解釋。





第一個可能的解釋:研究還不夠完善

避孕藥的研究數量龐大到令人難以置信,在过去幾十年間,文獻數已經超过四萬篇。問題是,許多研究往往缺乏品質,尤其是關於副作用的部分。儘管質量低,也不太可能導致荷爾蒙避孕的副作用遭到忽視或低估。也許這聽起來很奇怪,但妳通常會在不好的研究中找到最多副作用,而非好的论文裡。大多數不錯的研究認同整体趨勢而不是顯示很少或沒有副作用。所以關於很多不好的避孕藥副作用研究可能对我们过度誇張這些副作用的規模与嚴重性。

許多不良研究的問題在於他们通常會找使用荷爾蒙避孕藥的女性,向她们詢問副作用時,不去詢問不使用荷爾蒙避孕藥的女性。當妳這麼做的時候,妳根本不能得出任何結论,因為妳所做的一切極有可能是衡量這些症狀在一般人群當中有多常見。

想像一下,例如所有女性中有10%的人每個月會頭痛一次,但通常不會有任何特別的想法。如果有人問她们多久頭痛一次,她们就會加以猜測。接著她们參与一項研究,每天服用避孕藥並寫下所有可能的副作用。因此在這項研究中,有10%的人會自動回報頭痛的現象,即使和避孕藥无關。因為沒有与不使用避孕藥的女性比較所以不會察覺。取而代之的是,看起來就好像是避孕藥引起的頭痛。這些類型的研究盛行,也是最常看到荷爾蒙避孕对心理和性慾有副作用的文獻。

在医学領域上有一種研究被認為是最好的,也就是絕对黃金標準。當然它有一個奇特的名字:隨機对照研究。一群人被隨機分為接受治療与沒接受治療兩組,不接受治療的人為对照組。盲測是研究的理想情況,意即患者(最好医生和研究員也一樣)不知道正在接受何種治療。只有這種模式的研究才可能會有因果關係,也就是證明藥物是否為造成症狀的原因。

據我们所知,以隨機对照方式進行避孕藥和如情緒變化的非特異性副作用研究目前只有四項14 。其中兩項研究發現,有无服用藥丸的人在情緒變化上並无顯著變化。另一項研究則發現,服用避孕藥導致憂鬱症症狀改善。

在过去愛丁堡和馬尼拉女性的研究裡,發現服用迷妳丸的女性憂鬱症狀減少,而收到安慰劑与避孕藥的人憂鬱症狀輕微增幅。

唯一的例外是一項瑞典的小型研究。一組研究員在烏普薩拉(Uppsala)邀請了一群曾經因為避孕藥經歷过心理副作用的女性參加安慰劑对照研究。在不知道自己組別下,一半的患者收到避孕藥,而另一半則是糖衣錠。研究發現,平均上,收到避孕藥比沒有收到的人經歷了更大的智力減退。此外拍攝的女性大腦圖像有感情喚起的傾向。有些服用避孕藥的女性裡,大腦運作在情感中樞的地方發生變化。

然而,有一個很大的「但是」:這只適用於三分之一的藥丸使用者。三分之二的服用避孕藥的女性沒有經歷智力衰退或大腦活動的變化,即使如此,他们本身对荷爾蒙避孕有不利反應的傾向。這些發現可能顯示,避孕藥对少數女性有負面心理影響。但是比起覺得就是如此的人數,適用這個結果到女性卻少了許多。這讓我们來到下一個可能的解釋:機會的力量。





第二種可能的解釋:機會的力量

人類都有一個喜歡維持世界秩序和系統的大腦。我们藉由想像不存在事件的連結來試圖理解我们時而混亂的環境。如果兩個事件同時連結,我们得出相互影響的結论。例如,妳开始服用避孕藥三個月後,突然發現妳的情緒有點低落。我能肯定不是避孕藥的關係吗?畢竟,就妳回想而知,妳從來就沒有經歷过這樣的情況。

但是,這並不是一定的理由。憂鬱症是人類極為常見的疾病。大約五分之一的女性在人生中經歷適當的低潮,有更多的人經歷抑鬱的感受和想法。憂鬱症是包含許多複雜原因的疾病。個性類型、大腦的生理變化、遺傳和生活中的問題參和其中。因為很多因素在內,幾乎不可能指向一個具体的原因。

憂鬱、情緒變化与煩躁,在人群裡是如此普遍的現象,也可能是偶發的状况。此外,如果妳聽說避孕藥能引起情緒變化和憂鬱症,如同我们稍早談到的反安慰劑效應,妳更可能會得出這樣的結论。网路论壇上情緒變化的謠言在女性网友間滿天飛,妳突然看到一道新的曙光。

這是由許多大型研究構成,在芬蘭、澳洲与美國,已經進行过這種研究,並得到負面的結果。澳洲的研究追蹤一萬名女性三年的時間,有无使用避孕藥之間得到憂鬱症狀的機率沒有任何區別。此外,研究發現,隨著使用避孕藥的時間越長,憂鬱的念頭也就越少。美國的研究在一九九四年至二〇〇八年間追蹤七千名女性。事实上,研究發現,上一年度使用避孕藥的女性比起不使用荷爾蒙避孕得到更少的憂鬱症狀,企圖自殺的可能性也更低。研究人員也發現同樣的現象存在於芬蘭的研究:使用荷爾蒙避孕的女性比其他女性較不憂鬱。

這些研究的問題在於,有无服用避孕藥的女性之間潛在的差異。且有可能是陷入惡劣情緒的女性停止服用避孕藥,而繼續服用的女性沒有負面反應的情況。負面影響因而被掩蓋。

鑑於這種評论,二〇〇〇年至二〇一三年間,哥本哈根的研究人員以十五至三十四歲的一百萬名丹麥女性進行大型人口基礎研究。研究發現,使用避孕藥与其他荷爾蒙避孕法的人比不使用的人需要抗憂鬱藥或是診斷出憂鬱症的風險增加得多。

在十五至十九歲之間的年輕女孩中效果最明顯,一旦滿二十歲,風險明顯下降並隨著年紀增長而持續下降。超过三十歲使用荷爾蒙避孕的女性,服用抗抑鬱藥或憂鬱症的發病率幾乎沒有增加。研究人員認為,大腦隨著年齡增長对荷爾蒙的波動變得不太敏感。

這項研究還發現,憂鬱症和服用抗抑鬱藥的風險隨著女性服用荷爾蒙避孕愈久而穩定降低。風險最高的族群月經為使用六個月的女性,之後再开始下降。使用四年的荷爾蒙避孕後,繼續使用与不使用的人之間罹患憂鬱症的風險並无區別。

研究人員同時得到不同種類的避孕法之間存在這差異的結果。避孕藥是使用抗抑鬱藥的風險最小的種類,反之,例如迷妳丸、陰道環和長效方法的風險更大。雖然不可能只根據一種研究來斷定,不过研究強調了女性在經歷不良副作用時应该要有更換其他避孕方法的对策。避孕方法帶給女性的副作用不同,嘗試顯得更為重要。

话雖如此,我们建議解讀這項研究時应该要謹慎一些。目前丹麥有很多恐慌宣傳警告女性抵制荷爾蒙避孕,因為會導致憂鬱症。信不信由妳,根據研究,妳不能將這個結果奉為圭臬。研究顯示的是,使用荷爾蒙避孕的女孩开始服用抗抑鬱藥比那些不使用的人要來得多。沒有人可以證明,荷爾蒙避孕是造成憂鬱的原因。聽起來像是吹毛求疵,但這是一個很重要的差別。為了能夠說出因果關聯的結果,妳必須使用完全不同的研究方法:隨機对照研究。正如我们之前討论过,這類研究目前沒有得到任何近似的結果。丹麥的研究很完全,有可能會在该領域進行更詳細的研究,然而直到我们有更多顯示同樣結果的研究前,我们不能斷定荷爾蒙避孕在某些女性身上造成憂鬱症。

我们同時也无法擺脫相对与絕对風險之間的对立。一些与丹麥研究有關的報章雜誌表示,青少女得到憂鬱症的風險高達80%。這聽起來很可怕,表示妳很可能在高中來月經後,因為服用避孕藥而變得憂鬱。事实上完全相反。每年,100名不使用荷爾蒙避孕的丹麥少女當中有1位开始服用抗抑鬱藥。相較之下,使用荷爾蒙避孕的100名少女則有1.8位服用抗憂鬱藥。我们要講的是增加幅度根本不到一人。98位使用荷爾蒙避孕的少女沒有憂鬱症,而只有1位會在任何情況下得到。妳应该記住這些數據,而非頭條上那個驚人的80%。一旦事实了然可見,妳仍可自行將它當成不使用賀爾蒙避孕的理由。我们不會干預這個選擇。

現在,我们經歷了許多研究,並提出相反的結果。可能很難以消化這一切,我们非常清楚這一點。即便如此,我们認為有可能從這些研究中得到一個重要的結论:荷爾蒙避孕不可能对大多數女性的心靈造成任何重大負面影響。如果這樣的副作用确实存在,它容易出現在由於某種原因容易对荷爾蒙產生不良反應的人。我们希望將來能夠好好了解這些女性。如果妳的家庭有很多人一直飽受憂鬱症所苦或是妳过去有憂鬱症傾向的话,或許值得謹慎看待。

对於此外的族群,是時候停止擔心⸺當我们聽到荷爾蒙避孕心理副作用的可怕故事時完全不用盡信。感覺不一定就是事实。

我们使用荷爾蒙避孕是為了有无憂无慮的性生活,但如果使用後造成性生活无趣呢?避孕藥扼殺性慾是真的吗?許多女性似乎這麼認為。在瑞典的調查中,近30%使用荷爾蒙避孕藥的女性認為其中一個副作用就是性慾降低。

荷爾蒙避孕和性慾之間的關聯在二〇一三年進行了一項最大規模的研究。結合了36項研究,共計1萬3千名女性,其中8千人服用避孕藥。大多數女性發現自己的性慾在服用藥丸後維持不變(64%)或确实增加(22%)。同時一些研究發現服用避孕藥會導致性慾增加;因為避孕消除可能會懷孕的焦慮⸺世界上最大的女性激情殺手。正如我们在慾望的部分討论过,性慾,簡單地說,是減速与加速之間平衡的作用。因此,研究人員不認為荷爾蒙會直接提高性慾。另一方面,15%使用荷爾蒙避孕的女性性慾降低。我们不能肯定地說荷爾蒙是否就是罪魁禍首。

然而,已知的是,体內現有的睪固酮濃度在使用荷爾蒙避孕時降低。正如我们所知,睪固酮是男性最優秀的賀爾蒙,但女性也會產生一小劑量。服用睪固酮來增加肌肉的健美先生經常會性慾高漲(通常伴隨陰莖細小和劣質精子的討厭組合)。荷爾蒙避孕的女性有可能因為少了睪固酮造成性慾喪失的反效果吗?

睪固酮降低的程度依每個女性与我们使用避孕的類型有所不同。荷爾蒙避孕含有不同的黃体製劑,对睪固酮有不同的影響。那些含有屈螺酮(drospirenone)的產品,例如優思明,會降低睪固酮的濃度。可能會使青春痘變少,而且,有可能也會造成性慾降低。然而,包含左炔諾孕酮的黃体製劑樂婷錠、欣无妊和荷爾蒙避孕器,有更多的睪固酮效果,因此不太可能導致性慾降低。

睪固酮理论的問題在於,血液中的睪固酮濃度和所經歷的性慾程度之間沒有明确的關聯。有些較高睪固酮濃度的女性飽受性慾所苦,而其他低睪固酮的女性卻沒有感覺。性慾与睪固酮濃度顯然不成正比。即便如此,人们曾試圖給予女性睪固酮來提高性慾但是卻沒有任何神奇效果15 。平均而言,她们每個月在「滿意的性活動」次數上比別人多一次(這些人在研究世界裡实在很會开黃腔)。

儘管如此,女性的性慾還有很多我们不知道的事。關於荷爾蒙避孕对性慾的影響,我们对能否找到一個很好的答案沒有把握。因為对妳来说什麼是真正的性慾?因為性慾沒有好的評估方式所以難以研究。更重要的是,性慾是受到很多生活因素影響,很難分出到底是因為避孕藥還是感情淡掉所致。

妳可能已經明白,研究的世界充滿不确定性。然而,我们可以說,很少有研究提出荷爾蒙避孕对大多女性的性慾有強烈的副作用。

可能妳的避孕方式降低妳的性慾,但並不常見。更常見的是性慾隨著情感關係的時間消退,或是壓力剝奪我们对性的樂趣和花招所需的多餘能量。

我们的建議是,在把妳的避孕藥丟到垃圾桶,或是預約移除植入式避孕棒之前,評估生活中是否有其他導致妳性慾降低的其他方面。妳也可以嘗試切換不同黃体製劑的避孕方法。





是時候來做荷爾蒙排毒?





性对我们大多數人来说並不一直都是好的。當妳在一個穩定的關係時,也許妳每週做愛好幾次;但結束之後,妳的單身生活並不像妳所憧憬的《慾望城市》(Sex and the City)劇情那樣。妳开始覺得自己像是草原上的大象,在旱季高峰寻找水源。眼前沒有時尚雜誌、養眼的对象也不會有陰莖。避孕藥成為每天妳非自願禁慾的殘忍提醒,似乎從浴室櫃子嘲笑妳:「哈!妳今天也不會發生性行為!」

同時,妳大概也聽說荷爾蒙对妳身体不好,它们是不自然的物質。為什麼妳要在沒有獲得性作為補償的狀態下讓身体受到邪惡的荷爾蒙支配呢?妳認為:「這段單身的時間就當作排毒、淨化和養身!是時候暫停使用荷爾蒙了!」

等等,這实際上不像聽起來那麼聰明。如果妳發現了一個適合妳的荷爾蒙避孕方式,如果因為妳單身就停用就太傻了。大多數人开始使用荷爾蒙避孕會有特定的副作用,但是通常會在幾個月後變得緩和。身体會調整到一個新的荷爾蒙平衡並穩定下來。當妳停下來,這需要時間讓妳的身体恢復到一個新的平衡點,下一次再使用時只會再經歷一次完全相同的副作用。

血栓,是我们不建議暫停服用荷爾蒙避孕的主要理由。一些研究表示,开始服用避孕藥後的前幾個月造成血栓的風險最大,隨著時間的推移急劇下降。如果妳每一見一個新对象就开始和暫停使用避孕措施,妳的身体不會有時間回到平衡。結果造成妳理想对象不只會讓妳緊張,得到血栓的風險也變高了。

如果血栓是暫停荷爾蒙的危險卻罕見的副作用的话,那麼另一種,就是更加普遍。情人与妳不期而遇,而妳的医生不能提供全天候服務。如果還考慮到一個事实,就是挪威人是世界最不愛用保險套的人,這一點也不奇怪,停藥後最終往往給妳比妳討價還價還更多的排毒。实際上是九個月排毒。四分之一停用避孕六個月的女孩最後在半年內意外懷孕。非常自然!

有些女性擔心長期使用荷爾蒙避孕,可能在以後的生活中難以懷孕。好在這些都是胡說八道,雖然使用特定荷爾蒙避孕可能需要幾個月的時間才會再次排卵。事实上使用荷爾蒙避孕的女性當中不孕的可能性較低,得到性傳染疾病的话,她们骨盆發炎的機會更小。不幸的是女性(和男人)无法懷孕有很多種原因。問題是,直到停止使用避孕措施並試著生孩子之前妳不會知道自己是否是其中之一。如果妳三十五歲卻懷孕失敗的话很容易把原因歸咎於從十五歲开始使用避孕藥的決定。研究表示,使用避孕藥对女性生育能力沒有影響,无论是否已經使用一年或是十年。然而,老化确实有很大的關係。





為荷爾蒙避孕辯護




最近,在挪威關於荷爾蒙避孕的複雜面向有許多的公共討论。沒有更多的避孕方法可供選擇确实遺憾,我们完全同意,我们非常想看到市面上有更好的男性專用避孕方式。但事实是,性行為會導致懷孕,因此避孕是女性不可或缺的決定。不管我们多麼不喜歡它,這個事实並不會消失。當然,我们想要做愛。

避孕的世界還不盡理想,我们在結束這個部分前必須替荷爾蒙避孕發聲,提出一段言论作為辯護。因為許多荷爾蒙避孕的正向因素往往遭到忽視。荷爾蒙避孕,和含銅避孕器及結紮一樣,是我们最有效維持避孕的方式。部分使用荷爾蒙避孕女性所体會到的无害副作用,和大多懷孕女性的經歷相比算不了什麼:与懷孕有關的骨盆疼痛、大量分泌物、雙腿腫脹、痔瘡与妊娠紋。更不用說危險罕見的副作用。懷孕時血栓的風險,比使用荷爾蒙避孕時更高。太少人知道荷爾蒙避孕的正面影響。我们已經提过,不过重複幾次並无坏處。



• 荷爾蒙避孕似乎能預防一些女性最常見和最危險的癌症形式:大腸癌、卵巢癌和子宮內膜癌。

• 荷爾蒙避孕能減少經痛,縮短經期及降低血量,更可以降低得到貧血的機率。对許多女性而言這是嚴重的問題。

• 使用複合避孕產品,妳可以控制適合自己的經血模式。

• 荷爾蒙避孕防止骨盆腔感染⸺对无子女的女性造成很大的影響⸺藉由增厚子宮頸黏液,讓細菌難以進入。

• 得到良性乳房腫塊⸺導致許多年輕女性焦慮同時必須接受手術治療的機率降低。

• 荷爾蒙避孕也擅於治療兩種常見又麻煩的女性疾病:多囊卵巢綜合症和子宮內膜異位。



當人们將荷爾蒙避孕看作是女性的天敵時,記住這個列表可能是個好主意。避孕藥在女性平權中已經並將繼續成為世界上最重要的發現之一。





避孕指南




妳覺得很難選擇避孕方式吗?我们有十一種方式可供選擇。但是不要絕望:我们已經替妳準備避孕指南。由於有效的避孕方式需要處方籤,所以妳必須和医生、接生員或護士諮詢。不过先建立一些想法是不錯的。根據对妳而言最重要的方式,妳現在可以選擇適合妳的避孕方法,並找出妳最好避开的有哪些。妳可能对以下幾個組合有興趣,所以妳的問題就是選擇最佳的替代品。





对我来说,最重要的是避免懷孕


如果对妳最重要的事情是避免懷孕,妳应该選擇避孕最有效的方法⸺也就是所謂的長效方法。在列表的頂部,妳會發現植入式避孕棒和荷爾蒙的避孕器,緊接在後的是含銅避孕器。複合產品,如避孕藥,如果妳正确地使用它们也是有效的。

適合:具有低佩爾指數的長效避孕:植入式避孕棒、荷爾蒙避孕器和含銅避孕器。

不適合:具有較高佩爾指數的方法,尤其是基於安全期的方式。





我是血栓、中風或心臟病高風險族群


如果妳是這些疾病的高危險族群,妳必須避免使用雌激素。妳仍然可以選擇最擅長防止懷孕的避孕方法⸺即黃体製劑產品,如植入式避孕棒和荷爾蒙避孕器。如果妳喜歡服用避孕藥,挪威市面上也有販售无雌激素藥丸如Cerazette。

適合:无雌激素的方法:避孕植入物,荷爾蒙避孕器,无雌激素避孕藥和含銅避孕器

不適合:複合產品:複合藥丸、避孕貼片和陰道環





我想更少的出血量


月經可以很痛苦,特別是对擁有嚴重疼痛与出血的女性。有些女性是如此難受,她们流血到自己貧血,或者因為疼痛每個月得花一個星期躺在床上。如果這聽起來很像是妳的情況,那麼知道某種避孕方法可以減少出血是很有幫助的。所有荷爾蒙避孕的普遍特徵是血量較小。為了找到哪一個最適合妳,妳应该去嘗試,透过試驗和错误,再与妳的医生諮詢。含銅避孕器同時會增加出血和疼痛,所以不建議妳使用。

適合:一般荷爾蒙避孕,特別是荷爾蒙避孕器和複合產品

不適合:含銅避孕器





我想控制出血


正如妳所記得先前提到「荷爾蒙避孕对月經的影響」的部分,可以使用含有雌激素的避孕藥來控制妳的出血。黃体製劑產品不提供任何月經血量控制。如果妳已經在使用雌激素避孕卻沒有正向結果,可以將低劑量的雌激素產品換到略高劑量的產品。例如,從樂婷錠切換到奧諾康或欣无妊。這種變化不會增加血栓的風險。

適合:複合產品:複合避孕藥,避孕貼片和陰道環

不適合:黃体製劑產品





我有痘痘的煩惱


如果妳有痘痘煩惱,雌激素可以幫助改善;換句话說,妳可能會諮詢妳的医生考慮使用複合產品。黃体製劑常常引發痘痘。如果妳已經使用複合產品,妳可以嘗試改用另一個包含不同類型的黃体製劑或更高劑量的雌激素。記住,它往往需要三個月的時間,才能看到效果。

適合:複合產品:複合藥丸、避孕貼片和陰道環

不適合:妳嘗試过的黃体製劑產品





我想要对其他人隱藏我的避孕方法


对於一些女性来说,隱瞞使用避孕的事实是很重要的。有些避孕方式如植入式避孕棒、含銅避孕器、荷爾蒙避孕器或是避孕針,因為在妳的身体裡面所以看不到。如果妳擔心如何对妳的伴侶或家人隱瞞妳在避孕的话,不妨利用不會改變月經規律的方法,因為月經變化會影響妳的性生活或意味著妳需要比平常用更多或更少的衛生棉和衛生棉條。可以使用複合產品或含銅避孕器作為替代方案。這些經常讓妳有固定的經期,雖然出血量可能會改變。如果懷孕对妳来说不是危機的话,妳也可以嘗試計算安全期來減少懷孕的風險。但是要記住,使用這種避孕方法的女性有四分之一在一年內懷孕。

適合:隱形避孕,如植入式避孕棒和荷爾蒙避孕器,或者給妳固定週期的避孕方式,例如複合產品

不適合:這取決於妳想隱藏避孕的方式





我要保護自己免於受到性傳播感染


保險套是保護妳免於性病的唯一避孕方法。直到妳和妳的伴侶已做过性病測試前,我们建議妳一起使用保險套与另一種避孕方式。

適合:同時使用保險套与另一種避孕方法

不適合:不使用保險套





我服用其他藥品⸺我這樣可以使用荷爾蒙避孕吗?


藥物會相互影響。如果妳正在服用諸如癲癇或精神疾病的藥物,這可能會影響妳的避孕。妳的医生會追蹤這一點,也許她可以給妳量身定做的解決方案。

適合:如果妳正在服用其他藥品,妳的医生會幫助妳找到最好的解決辦法。





我有子宮內膜異位症


如果妳有子宮內膜異位症或懷疑自己可能有嚴重的疼痛,荷爾蒙避孕是妳治療的第一步。既然目的是停經,妳就不能休息。

適合:連續使用複合產品或荷爾蒙避孕器。





我有多囊卵巢綜合症或是月經極為不規律


如果妳在一年不使用荷爾蒙避孕的状况下月經少於四次的话,妳应该开始使用荷爾蒙避孕來驅逐定期的子宮內膜。如果月經是極為罕見,妳可能會有子宮內膜过度生長的問題,長期下來对妳不利。一旦妳在使用荷爾蒙避孕時發生幾次突破性出血,問題就能解決,而妳可以繼續如妳所願避开月經。

適合:複合產品:避孕藥、避孕貼片和陰道環





我使用的避孕措施降低我的性慾


无法确定荷爾蒙避孕是否會引起性慾降低,如果是這樣,機制才是罪魁禍首。有一種理论認為,這是睪固酮不活躍造成的。不同類型的黃体製劑对睪固酮有不同的影響。含有屈螺酮,例如優思明的藥丸可以降低睪固酮濃度。可以減少痘痘,但是性慾也會同時下降。然而,像樂婷錠、欣无妊和荷爾蒙避孕器裡的左炔諾孕酮黃体製劑具有更類似於睪固酮,因此不太可能降低妳的性慾。

適合:含有左炔諾孕酮黃体製劑,如樂婷錠、欣无妊和荷爾蒙避孕器,不含荷爾蒙的避孕藥,如含銅避孕器。

不適合:含有屈螺酮孕荷爾蒙如優思明的產品。





* * *



13	數據依研究而有所不同,並取決於研究的年齡層与人口類型。年齡、体重的增加与吸菸族群所隱含的血栓風險有明顯的上升。



14	這些研究的其中一個弱點是,他们以非避孕目的而使用荷爾蒙避孕法的族群為研究对象,例如有嚴重痘痘或經痛問題的人。因此,可以想像這些女性与使用荷爾蒙避孕的女性有所不同,而結果也受到影響。舉例而言,有更多痘痘困擾的女性會更加憂鬱吗?



15	睪固酮的增加主要是对絕經後或因為癌症切除卵巢的女性進行測試。鮮為人知的是使用睪固酮的長期風險,如果女性在懷孕時服用睪固酮,可能會对胎儿造成傷害。在一些对略為年輕女性(三十五至四十六歲)的隨機研究中,增加睪固酮对性慾的影響很小或根本沒有影響。然而,安慰劑效應卻有很大的影響。





墮胎




墮胎,故意終止懷孕的做法。一方面,在身体上,是關於女性選擇自己是否要生下一個孩子的權利。另一方面,墮胎是一個新生活的开始,一旦懷孕,這個孩子应该擁有什麼樣的權利?墮胎的問題並沒有簡單、道德的答案;總會有輸的一方,无论是孕婦、孩子的父親、進行人工流產的医療專業人員或是胎儿。对我们来说,女人的權利占有舉足輕重的地位⸺因為女性在懷孕与生產中會經歷身体和心理上的壓力。

一直以來總是留給女性提供照顧及支持的責任。一個孩子的出現會導致女性更多的情緒、經濟和社會的震動,而女性在一开始擁有的最少,最後受到的衝擊卻往往最大。因為受到如此大的衝擊,女性应该要決定自己是否想承擔這一切。當我们迫使女性生下她不希望擁有的孩子時,我们認為沒有一個地方的政策能夠接受对人民施加大量的個人成本,來滿足社會道德規範。也就是必須要有所限制。大多數人在懷孕上都同意一些觀點,墮胎不能僅基於女性的選擇為考量,胎儿不再是胎儿,而是一個孩子,還有擁有比偏好更有價值的權利以及孕婦本身的權利。這樣的限制設置隨著國家有所不同,但是不管上限為何卻甚少受到挑戰。在墮胎合法的大多國家裡,多數墮胎發生在懷孕初期,而少數的晚期墮胎為人所知的原因是,胎儿有嚴重或危及生命的異常状况或是為了保住母親的性命。

舉例来说,墮胎規範有非常不一樣的方式⸺從完全禁止的智利与馬爾他,以及女性有權在懷孕十二週內選擇墮胎的挪威,到沒有墮胎相關法律反而屬於女性和医生之間医療行為的加拿大。可以使用墮胎的程度也有許多區別:儘管有沒有遭到禁止,有可能是因為非常昂貴或开放的地方太少,讓許多女性无法如願。舉例来说,美國許多州就有這種情況。无论妳对墮胎的個人感受如何,禁止墮胎或將墮胎流程複雜化无助於降低墮胎數,這是個不爭的事实。我们經常會發現,最嚴格的立法國家墮胎率同時也是最高,而那些容易獲得合法墮胎的國家流產率往往較低。儘管懲罰和社會排斥的威脅,所有的世代和世界上每個角落意外懷孕的女性都選擇自行處理⸺更不用說暴露於嚴重傷害或死亡的風險。生下不想要的孩子的想法可以是如此讓人難受,遠遠超过法律起訴的危險与威脅。織針、女祭司、陡峭樓梯和毒藥仍是世界上某些地方的女性在墮胎是違法或无法实施的情況所下使用的最後手段。每年有二十萬女性認為有必要進行不安全墮胎⸺幾乎是全球懷孕女性的十分之一。在這些女性中,就有五萬起完全不必要的死亡。約六百九十萬名女性需要透过医療服務治療危險墮胎造成的併發症。安全的墮胎會使她们免於這些傷害。換句话說,合法和安全的墮胎,对保障女性的健康極為重要。禁止墮胎不會留著任何孩子,只會傷害絕望的女性。然而,墮胎沒有捷徑。我们認定少數女性想要墮胎或有意识地用來當作避孕的替代方案。這是因為經常遭遇在错误的時間下進行无防護性行為、避孕失敗、无法使用現代避孕方式或者(在最糟的情況下)受到侵犯和性暴力等不幸。如果目標是維持墮胎率低,那麼最有效的措施是确保容易取得、有效的避孕措施,並提供良好的性教育。

不幸的是,我们經常看到,限制墮胎的法律正好与這兩個難以在健保服務中進行墮胎的面向息息相關。這就像鴕鳥堅持把頭埋在沙子裡認為問題將會消失,只是因為沒有看的必要。

不管妳是否生活在墮胎方不方便的國家,妳會慶幸多少知道墮胎在医療系統下是如何進行。墮胎進行的方式⸺无论是医院或專科門診,与適用的規矩,依國家而異。但是這些方法都是一樣的,如果妳發現自己正處於意外懷孕的狀態,能夠將妳的想法專注在更重要的事情而非找出实務上的方法會是件好事。





我懷孕多久了?

談到墮胎,常見的疑惑之一就是妳懷孕多久了。許多國家的墮胎法律都有時間限制;例如十二週以內的墮胎是受到允許的。但是,真正懷孕十二週是什麼時候呢?妳會認為是從發生未防護性行為那天开始計算,令人難以置信的是,事实並非如此。反而是從最後一次月經的第一天开始計算。因為這是确定妳沒有懷孕的最後一刻。從這個角度看,在性交懷孕後,法律認為妳「懷孕」了兩個星期。雖然不完全合乎邏輯,卻是規定運作的方式。在妳墮胎之前,大多數医生會替妳進行超音波檢查。厚度大約是一根細胡蘿蔔的小探針插入妳的陰道,看看妳懷孕了幾週。假如子宮內的胎儿長度多於6.6公分,就會被認定為超过十二週大。檢查可以知道一切,因為很多女性有月經不規則或不記得自己最後一次月經時間的情況。此外,也讓医生确保妳說的是实话。如果有任何的疑問,超音波檢查是妳懷孕多久最合理的解答。





墮胎的兩種方法

墮胎的方式有兩種:藥丸或是小手術。

使用藥丸的墮胎稱為藥物流產,而另一種方法則被稱為人工流產手術或子宮刮除術。





藥物墮胎

藥物墮胎的流程通常從妳在医院或在医生面前服用藥丸开始進行。藥丸中含有一種叫米非司酮(mifepristone)的物質,讓身体以為妳不再懷孕。所有複雜的过程都在确保受精卵長成胚胎,接著中斷變成胎儿後的運作。墮胎已經开始,卻並非完全運作,因此胎儿留在妳的子宮。雖然过程不完整,這並不代表妳在服藥後可以反悔⸺照理来说,胎儿不會有更進一步的成長。

當妳服用藥丸後,必須等待一到兩天。這段期間有輕度噁心、輕微出血和經痛現象完全正常,除此之外,妳可以正常作息。然後,墮胎經过大約兩天的時間會完全結束。如果妳是懷孕不到九〜十週的健康女性,可以在家裡進行。如果妳懷孕達到此週數的话,有個成年人在妳身旁很重要,例如妳的朋友或伴侶。這是基於安全考量以防出現併發症⸺雖然併發症非常罕見。在許多地方,如果妳希望,或如果妳已經懷孕超过九〜十週,以門診病人的身分到医院或診所,服用最後一輪的墮胎藥。

不管妳在哪裡進行,方法都是一樣的。將四顆米索前列醇(misprostol)放入妳的陰道或舌頭下。在墮胎是違法的國家,女性漸漸愈來愈普遍在网上或透过其他方式取得米索前列醇進行墮胎。藥丸引起子宮收縮並擠出裡面的東西⸺排出形式和妳的月經有點像,只是這一次妳子宮裡的小小胚胎會隨著血液一起出來。

一旦墮胎進行,妳的出血會比正常月經更加嚴重。出來的血液會凝結成紅色。如果妳害怕看到胚胎,只能說妳愈早墮胎,看到的機會也就愈少。挪威墮胎的時機大多在懷孕第九週前,胎儿為1.5公分長的透明蝌蚪狀,被黏液和血液包覆。任何妳在网路上看过的可愛的迷你嬰儿圖片完全誤導大家,造成女性对墮胎感到內疚。

对於95%至98%的女性而言,墮胎只是幾個小時的事情。 妳必須記得按照医生的指示服用止痛藥,因為可能會感到疼痛。如果妳墮胎後仍然有劇痛、發燒或嚴重出血的情形,必須打電话給医院或去急診室。大家常說,如果妳在不到兩個小時內出血超过一個夜用型衛生棉的量,妳就应该向医生聯繫。

墮胎後,連續二〜三週中度出血与感受到一點疼痛非常正常。在這種情況下,為了預防感染,使用衛生棉而非衛生棉條很重要。此外,妳不应该在出血的時候有性行為。只要妳有流血,這代表子宮仍在排出懷孕的殘留物,同時任何細菌會透过陰道輕鬆進到妳的身体系統當中。墮胎後受到感染並不常見,但仍然需要採取預防措施。

妳常常在媒体上看到許多女性進行藥物墮胎,卻發現自己還仍然懷孕數個月的恐怖故事。如果妳遵照医生的指示,就不可能會發生這種状况。一百名患者在藥物流產後只有一位可能會保持懷孕的狀態。妳會被告知這種情況的發生,是因為妳在陰道裡放入最後一輪的藥丸後會有不正常的出血。如果出現這種情況,妳应该再次迅速聯繫医院。服用藥丸停止懷孕後,若子宮裡還有殘留物不是好事。所有墮胎後的女性应该在一個月後進行驗孕測試,以确保懷孕完全終止。此外,如果出血停止後,妳的月經還沒有在四〜六週恢復,应该向医生聯絡。





手術墮胎

手術墮胎的过程稍微不一樣,而且必須在医院或診所進行。妳通常會拿到兩顆藥,在安排流產的當天早上將它们放入陰道,這些藥丸會導致子宮頸擴張。如果妳打算在墮胎过程麻醉,妳必須從手術前一晚禁食。這代表妳不应该吃、喝或吸菸。很多地方的手術只使用局部麻醉進行墮胎。

手術本身大約十分鐘,手術器具從陰道進入,接著來到子宮頸。緊接著,医生使用小型吸引器吸出胎儿与胎盤,然後輕輕刮下子宮內膜,确保都已經被移除。墮胎後,妳必須留在医院幾個小時,這樣医生可以檢查一切進展順利。在那之後,妳可以在當天回家。

如同藥物墮胎,經过一陣疼痛後,妳可能會出血。也一樣要遵守使用衛生棉与不能性行為的規定,同樣地,如果妳變得不適、出血嚴重或六週後月經沒有再次开始,妳应该要与医生聯繫。

和所有手術一樣,麻醉或手術本身有關併發症的危險性不大,包括子宮、膀胱或尿道的損坏。這些非常罕見的併發症也就是為何許多國家會建議進行藥物墮胎的原因。避免手術墮胎一直都是上策,但由医療專業人員進行的手術是非常安全的。很多女性喜歡手術墮胎,不想經歷过程冗長的藥物墮胎。

有些人可能聽說过手術墮胎後可能會更難懷孕,這種印象是來自一個名為阿休曼症候群(Asherman's Syndrome)的罕見疾病。如果医生必須從子宮刮出大量組織的话,最後會破坏子宮內膜的最深的一層。然後,妳可能會有子宮傷口和黏連的現象,這可能使妳之後難以懷孕。現在的婦科医生都會害怕這種事發生,所以會盡一切所能以保安全。換句话說,一個簡單的子宮刮除會对妳以後的懷孕機會產生任何影響是不可能的。但只要妳子宮刮除的次數愈多,風險就愈大。 這是為什麼墮胎不应该被用來當作避孕手段的原因之一。

發現自己意外懷孕可能是一個震撼的体驗。當然,对於一些人来说會是一種驚喜,但是仍會在不寻常中感到一絲恐慌。在妳可能還沒有準備好時,懷孕會触發許多情感过程。如果發生了,有人可以訴說就再好不过了。无论妳選擇怎麼做,医療服務裡遇到的每個人都有保密的義務,並可以為妳提供指導⸺不管妳最後不得不墮胎、留下孩子或去送養。不管妳選擇怎麼做,和妳的伴侶、朋友和家人徵求意見和關注也是明智的作法。





私密處的疾病





我们的生殖器就像我们身体其他部分一樣。只要一切正常,我们就不會多加留意。一旦出現問題,可能就會成為傷神傷腦的事情。任何有嚴重酵母感染的女性,或因經痛所苦的女性都能明白。在這種日子裡,我们可能會咒罵自己為何身為女性。為什麼不能把我们每月一次的痙攣換到睪丸上發作呢?

本書的這一部分將提到所有可能对我们下体帶來麻煩的疾病。我们非常确信大多數女性在生命中會遇到一些這樣的情況。幸運的是,像子宮頸癌等症狀其实非常罕見。

當我们在寫到這一部分的時候,我们發現自己也不确定是否會在最後造成更多焦慮。藉由談论症狀模糊的罕見危險疾病,我们會不會讓女性暴露於不必要的新擔憂呢?

我们希望、同時也相信事实並非如此。記住,妳的身体會一直發出健康或疾病的小訊號。我们应该注意到我们還活著的事实⸺畢竟我们不是機器。但是,我们當中的某一族群比其他人更加在意這些訊號,這可能導致健康上的焦慮。我们認為這種焦慮的最佳藥方就是獲得更多的知识。更多的知识可以給妳安全感,然而用Google搜索模糊、常見的症狀嚇自己只會讓恐懼變得更嚴重。祕訣在於區分我们現在和以後經常會遇到的一些現象,以及哪些可能是更為嚴重的跡象。

在我们身為性健康作家的工作裡,我们發現女性普遍缺乏婦科疾病的知识。很多女性都在為周遭從來沒聽过的疾病糾結,她们常常感到孤獨和不信任。許多人不知道去哪寻求幫助。例如,我们在开始念医学系前從來沒聽过子宮內膜異位症。即便如此,十分之一的女性正遊走這種疾病中,有許多人努力使自己的日常生活適應這種痛苦,不应该是這樣的。想像一下,如果每十名男性中就有一人每個月因為睪丸受到極大痛苦而不得不休假一個星期呢?這是一個全國性的問題,也是每所学校课程裡會有的內容。

換句话說,是時候談到我们的問題了。這是确保人们獲得所需幫助的唯一途徑。也許可以將更多資源分配到女性疾病的研究,讓我们在未來找到更好的治療方式。我们可以寄予希望。

我们將從最常見的問題开始:出血性疾病。





出血異常⸺月經毀了




对於大多數女性而言,月經是生命中很重要的一部分。從青春期开始,直到四十五到五十五歲之間(可能更久或更短),我们的月經週期都是每個月的循環,加上一些例外。

當妳的月經出現問題,週期与妳应该的認知不一樣時,擔心和困惑是正常的。可惡,妳會這麼想⸺妳也不是唯一有這個念頭的人。奇怪的是,血液和黏液在子宮的變化雖令人擔憂,但女性们卻更容易相信自己的私密部位出了什麼問題。妳的想法全都糾結在妳的腦海裡,同時陷入危機。我有什麼問題吗?按照計畫,從現在开始的十年後,我能不能有孩子呢?是癌症吗?這是一種疾病吗?我是認真的⸺誰來幫幫我!

出血異常有許多不同的類型,可能和疼痛、不規律、流量問題有關,或者妳的月經可能就此停止。我们來討论一下最常見的問題。





當妳的月經停止時




最常見也最可怕的事情就是妳的月經消失得无影无踪,或有跡可循。儘管平時的經血已消失在稀薄的空氣中,但是有時候妳仍會發現微量或點狀出血。

如果过去月經規律,但月經突然消失超过三個月或者月經不規律達到九個月的女性,我们稱為閉經(amenorrhoea)。我们所說的規律是指,妳的月經週期時間每次都是一樣,而且每次都在同一時間來潮,這樣妳就能透过經期月曆來預測月經的時間。希臘語的閉經意味著「沒有月經」,完全等同字面上的意思。

女性停止月經是常有的事。十六至二十四歲的女性中,每年有8%的人經歷這種情況,而且可能有不同的原因。月經停止時妳首要考慮的事情,就是自己是否懷孕。但是我用了保險套,難道沒用吗?當妳月經遲來三天時,妳會這麼想。妳現在還沒有準備好接受孩子,恐慌即將浮上心頭。

在適當的時候進行驗孕測試可以排除懷孕的可能。如果有任何的可能,檢查自己是否懷孕非常重要。避孕失敗吗?錯过服用避孕藥的時間吗?妳是否依賴体外射精或安全期避孕呢?買一支驗孕棒⸺在无防護性行為或避孕失敗後三週內便能得到可靠的結果。如果妳沒有發生性行為,或正在使用安全的避孕方法(例如植入式避孕棒或荷爾蒙避孕器),那麼就是別的状况了。當妳有任何疑問,就進行驗孕測試。但是,可能是其他的原因造成妳的月經消失了。

旅行是閉經一個罕見卻有趣的原因。我们不知道為何會發生這種情況,但長時間的飛行,特別是跨越好幾個時區,可能會影響妳的月經週期,導致出血的時間错误,就像時差一樣。

但失去月經的兩個更常見的原因是体重變化和大量運動。体重變化的幅度難以定義,也无從得知需要達到多少運動量才會發生。專業運動員往往有閉經的情況,但妳不必為了當專業運動員而耗掉妳的月經。根據最嚴格的厭食症診斷標準,閉經是其中的項目,這並不表示因体重變化失去月經,就一定是厭食症。

心理壓力也是常見病因的一種。妳的情緒狀態影響妳的月經。也許妳学校课業太繁重了,或者妳經歷过諸如戰爭、事故或家庭死亡等重大心理創傷。

簡而言之,妳的月經是妳有力氣的象徵。為了讓妳懷孕,妳的身体必須足夠強壯才能生產。懷孕是一種緊張狀態,如果因為種種原因,妳失去了生下孩子所需的能量,妳的月經往往會停止,阻止妳懷孕,因為妳還沒有準備好。一切都息息相關。身体、心理和月經无一例外。如果妳的月經停止,而妳不知道發生了什麼事,看医生是完全合理的。

導致月經喪失的疾病包括多囊卵巢綜合症和代謝疾病。記住避孕是如何影響月經也許是一件好事。黃体製劑產品,如荷爾蒙避孕器、避孕注射劑,无雌激素藥丸和植入式避孕棒往往會導致月經停止。這很正常,並不代表有什麼不妥的地方。使用避孕藥時的出血不是正常的月經,而是我们所說的停藥性出血。不像正常的月經,這並非儲備能量的象徵。如果妳因荷爾蒙避孕而出血,那麼妳就不是閉經。

最後,自有經期後的前兩年裡,亂經的現象絕对正常。包含短暫的月經停止。妳的荷爾蒙需要一點時間才能達到平衡,讓排卵按月進行。它會自行調整。





好痛啊!




半數以上的人經歷嚴重的經痛:在我们的下腹有令人不適的痙攣疼痛。只要妳排除任何特殊原因的可能,例如導致更嚴重經痛的疾病,這就是所謂的原發性痛經(primary dysmenorrhea)。如果疼痛有根本的原因,就被稱為繼發性痛經(secondary dysmenorrhea)。痛經意味著「經痛」。有些女性在背部,大腿或陰道也會有疼痛感。疼痛在月經的頭幾天最嚴重,並時常伴隨其他症狀,如噁心、嘔吐和腹瀉。每六名女性中就有一人遭受如此嚴重的疼痛,每個月都需要休息幾天。

經痛是由子宮收縮所引起,在每次週期結束時,這些小小的空心肌肉緊縮並排出子宮內膜,最後以月經的形式出現。

子宮很強壯⸺也許有點过於強大。它緊緊地壓縮,以至於无法屏住呼吸,而且很痛!妳的子宮當然不會呼吸(只有妳的肺部能這麼做)但是妳身体的所有細胞都需要氧氣。沒有這些,它们會窒息而死。氧氣挾帶於血液當中,子宮卻緊緊地夾緊肌肉,並在這個过程裡完全切斷自己的血液供應。因為它急於擺脫舊的子宮內膜。組織缺乏氧氣就是導致疼痛的原因。

但是,等等⸺妳有沒有聽过這樣的事情?如果妳在衛生服務部門工作,或者妳有一個患有心絞痛(angina)的爺爺,這種疼痛的情況聽起來十分熟悉。事实上,由缺氧引起的疼痛正是人们在心臟血管阻塞時會有的状况。他们會在運動中体驗到胸痛。如果爺爺上樓,他的心臟需要更多的氧氣,但是他狹窄的血管无法快速輸送血液。接著心臟遭受「缺氧的痛苦」。完全相同的事情發生在妳的子宮收縮時。

妳也會從心臟病引發胸部疼痛。在這種情況下,氧氣很少,妳的心臟就會窒息死亡。如果妳現在开始有點擔心,我们向妳保證:經痛与心臟病發作不一樣⸺它们並不危險!雖然在相同的狀態下認為缺氧是導致疼痛的原因有點奇怪,但妳不會因為痙攣失去妳子宮的一部分。不一樣,卻有點相似。

那麼為什麼有些人會這麼痛苦,而另外一些人卻覺得彷彿像微風吹过呢?

答案在於妳体內的酶(enzymes)有多活躍。酶是小型的蛋白質,确保妳身体中所有的化学作用都遵循正确的过程。有一組叫做環氧合酶(COX enzymes)的物質,与製造前列腺素(prostaglandins)有關。其中,前列腺素是引發孕婦生產的物質。他们造成子宮收縮,反过來導致我们剛剛談到的缺氧現象。

有些專家認為,那些特別痛的女性,她们的環氧合酶會特別活躍。結果卻是,她们的前列腺素比其他人更多。這會導致子宮收縮強度增加,在放鬆与收縮之間遊走。前列腺素還會使生殖器區域的神經过度疼痛。

如果妳想知道妳的痛苦門檻是否很低,或當妳描述妳的痛苦時發現人们不相信妳,這裡有一些生產的比較,应该可以讓大多數人閉嘴。根據觀察,痛經女性的子宮收縮可以達到相當於150〜180毫米汞柱的壓力。或許現在来说对妳沒有任何意義,但相較之下,在生產推擠階段的壓力大約是120毫米汞柱。生產期間,女性每十分鐘有3〜4次的子宮收縮。經期裡,同樣状况,痛經的女性可能會發生4〜5次。換句话說,可怕的經痛壓力至少和推擠階段的壓力一樣高,而陣痛間隔稍微短些。所以現在妳可以明白,這可能會痛苦到死。幸運的是,這些可怕的疼痛通常會隨著時間緩解。

妳可以使用止痛藥來治療月經痙攣,但重要的是要正确使用它们。布洛芬(Ibuprofen)能夠直接抑制環氧合酶,确保產生較少的前列腺素。這就是為什麼布洛芬和類似的藥物,被稱為非類固醇消炎藥(Non-Steroidal Anti-Inflammatory Drug,NSAID),是月經痛最有效的藥物。如果妳經常有嚴重的經痛,妳应该在月經开始的前一天服用布洛芬,或者至少在意识到輕微疼痛跡象發生時服用。之後,妳应该在月經的頭幾天裡每6〜8小時服用一次止痛藥。有太多人等到真的痛了才吃止痛藥,不幸的是,由於前列腺素已經產生,所以止痛效果較差。

此外,大多形式的荷爾蒙避孕也对經痛有很好的效果。避孕藥也是一個更長期的解決方案,因為妳會不斷地使用。

最後,我们必須指出,有些人可能有不同、潛在的經痛原因。对於發現疼痛隨時間變化、突然增加、悄悄來襲的女性来说尤其如此。以前不是這樣的。這可能表示妳有子宮肌肉結,也就是子宮肌瘤(fibroids)或子宮內膜異位症,在子宮外產生多餘的子宮內膜。含銅避孕器也可能導致疼痛增加。如果妳有這種状况,現在是時候改用另一種避孕方法了。

如果妳遇到突如其來的劇烈疼痛,妳可能會認為是更嚴重的急性状况。例如,子宮外孕。如果受精卵沒有按照正常的方式進入子宮,就會發生這種情況。接著胎儿开始在輸卵管裡發育,但是那裡卻沒有足夠的受孕空間。子宮外孕可能會以嚴重的經痛呈現,一般會集中在同一側。在這種情況下,妳就得去一趟急診室了。





月經不規律




在妳初經後的第一年与停經前幾年,或者使用荷爾蒙避孕藥的時候,妳的月經有點不規律是正常的。开始月經之後,妳的週期需要一段時間才能穩定,當妳使用荷爾蒙避孕藥時,妳不再有正常的月經,因為妳的週期不像以前一樣。除了這些情況之外,妳的週期应该會穩定下來,或多或少,介於21〜35天之間。

但是如果妳已經有好多年的月經,出血情況仍然(或者突然變得)像《控制》(Gone Girl)的劇情一樣不可預測,那麼妳应该注意了。不規律出血可能与許多状况有關。可能是无預期點狀出血(在每次月經之間的小滴血)在性行為當中,或者与其相關的出血。

除了月經消失之外,壓力、体重變化或过度運動也會讓妳的月經延遲或突然來潮。這些事情影響著我们的荷爾蒙。其他原因有可能是潛在疾病,例如多囊卵巢綜合症或代謝疾病。

子宮頸癌或性感染疾病(STI)可能會導致子宮頸輕微流血。如果發生這種情況,性行為过程或結束後可能會触發輕微出血。正因為如此,妳应该讓医生檢查与性行為相關的出血。

如果妳正在使用複合避孕法(避孕藥、避孕貼片或陰道環),並出現不規律出血,和妳的医生或護士討论會是一個好主意。換成更多雌激素的產品也許會有所改善。避孕藥中有兩種不同劑量的雌激素。例如,樂婷錠是一種低劑量的藥丸,而欣无妊和奧諾康則含有較多的雌激素。除此之外,它们都是一樣的。當許多女性改用含有更多雌激素的產品時,會發現不規律出血状况停止。





太多血了!





妳在超市的衛生棉條架上看到不同的尺寸,而妳的女性友人不一定會像妳一樣出血。对於部分女性来说,即使是黃色迷你衛生棉條也會覺得太多。那些血量最少的女性只需要在短褲上黏上一張舒潔衛生紙就能解決問題。其他人每隔幾個小時不得不更換一次綠色大流量衛生棉條,而且因為出血的恐懼使她们渴望更高的吸收效果⸺超級增增增增增量。相信妳腦海中有畫面了。





每位女性失血量往往會相差很多,不过平均介於25〜30毫升之間,也就是說,在當地的一家咖啡館裡,大約只有一杯濃縮咖啡的量。雙倍濃縮咖啡的量也算正常。

妳是這些正在嘲笑他人的其中一員吗?一杯濃縮咖啡?在整個經期中?哈哈—太可憐了!至少一天兩杯吧!

有些人的月經量比起當地咖啡館更像巴托里夫人(Lady Báthory)的浴缸。巴托里夫人是外西凡尼亞(Transylvania)的一名連續殺人犯,據說她為了保持自己的青春而沐浴在處女的血液當中。但事实上不會有那麼多,儘管有種出血不會停止,而妳也會有血液流过衛生棉條、內褲、褲子,直接流到婆婆白色沙發上的感覺,但沒有人在會經期內流滿整個浴缸的血。实際上,大概7名女性一生的經血才能填滿20公升左右的浴缸。然而,還是有很多女性因為貧血而服用鐵劑。她们變得呆滯和蒼白,經常頭痛,不能去做她们喜歡的事情而煩惱。月經真的可以讓妳失去妳的鋒芒!

如果每個週期的出血時間超过8天,或者超过80毫升⸺超过兩杯半的濃縮咖啡,那麼就會被認為非常嚴重。不完全是一個浴缸的量,但大量的出血都是一樣的。

在初經來潮後的初期,年輕女孩血量較多的情況很普遍。可以隨著時間的推移而改善,而且甚少引起擔憂。然而,有些女孩會有極大的血量,檢查是否為潛在疾病引起的現象是明智的選擇。某些血液疾病实際上可能讓妳血量比其他人多而且更容易出血,不过這種情況卻很少見。

含銅避孕器是嚴重出血常見的罪魁禍首。許多女性發現這種避孕方法運作良好,而也有人發現她们的經血量与疼痛增加。对於以前有大量出血的女性更是如此。複合避孕藥可以治療大量出血,因為它们能夠有效控制出血。黃体製劑產品,像荷爾蒙避孕器,往往造成妳的月經消失,或大幅減少血量,同樣也是常勝軍。

來經一段時間並且逐漸有大量出血問題的女性可能罹患一種潛在的疾病,像是混淆妳荷爾蒙的多囊卵巢綜合症。大量出血也可能是因為子宮肌瘤,子宮壁上的肌肉結所致。妳可以在本章後面閱讀更多關於這些疾病的內容。





子宮內膜異位症⸺血腥的特許假期




經痛是女性们認為理所當然的事情,但有些人的經痛,嚴重到她们不得不把整個生活擱在一旁。 她们每個月有幾天抱著熱水瓶蜷縮在沙發上,像吃糖果一樣地服用止痛藥。不应该是這樣的。如果妳也是如此,妳可能受到一種叫做子宮內膜異位症的疾病所苦,這種情況影響約十分之一的女性。約有三分之一下腹和生殖器与這種疼痛奮戰的女性患有子宮內膜異位症16 。當然,這並不適用於陰戶本身的疼痛,我们會再回來提到。

正如妳可能從名字中了解意思,子宮內膜異位症与子宮內膜⸺也就是子宮內部的黏膜有關。這是每次週期在妳的子宮準備接收受精卵時會形成的黏膜。如果妳沒有懷孕,它會以月經的形式從子宮中排出,不过妳也早就知道了。子宮內膜異位症不同的地方是患者在子宮腔外也有子宮內膜。在某些情況下,子宮內膜誤入子宮肌層中,這個情況稱為子宮腺肌症(adenomyosis)。

无法确定這種子宮內膜是如何出現在子宮外面。主要的一個理论是,月經運作的方式错误,也就是說,它會從輸卵管而不是在子宮頸外增生,最後留在胃裡。這種情況發生在所有月經來潮中的女性身上,但似乎還有一些易感体質的女性无法自体清除。當發生這種情況時,一小群黏膜細胞就會誤會它们所屬的地方,例如:卵巢、骨盆、腸道或腹部的其他部位,接著停留於此。

這些子宮內膜細胞通常在靠近生殖器官內部被發現,但是在一些非常罕見的情況下,它们可以遠至包圍肺囊。這促使一些人懷疑除了誤入歧途的月經之外,是否還有其他機制引起子宮內膜異位症。也許是一種幹細胞(也就是可以成為任何所想的細胞)生長在错误的地方?或者是從子宮內膜透过血液輸送到其他部位的細胞呢?我们大概會在幾年內找到确切的答案。

就像太陽海岸的英國退休人士一樣,儘管它们已經找到一個新家,但是這群子宮內膜並沒有忘記它们來自哪裡。它们的一舉一動就彷彿住在子宮裡。這代表它们与普通的子宮內膜相同,对月經週期中的荷爾蒙產生反應。令人難以置信的是,這意味著每個月,妳的子宮外還有一個小小的經期。

錯位的月經並不是一件普遍的事情。子宮內膜在安靜有序的社區定居下來時,与免疫系統非常倔強地打了一仗。因為身体本身在某處对应该發生的事情有嚴格的規定。當這些子宮內膜在不屬於它们的地方开始流血時,叛亂迅速發生。新鄰居突然被一陣意外的血淋淋衝擊,所以自然而然地打電话給警察(我们的免疫細胞)以最快的速度到達現場處理。結果妳子宮內膜周圍的組織开始發炎,並因此受傷。





由於子宮內膜通常位於靠近子宮的地帶,大多數人很難將這些疼痛与嚴重卻正常的經痛做區隔,儘管有些女性也會發現她们在奇怪的地方有疼痛感。例如,如果內膜群在尿道附近定居下來,排尿時可能會疼痛,或者它们待在直腸裡最快樂的话,那麼排便時就會很痛。

這些類型的疼痛都有一個共同點,那就是它们有循環⸺意即它们遵循固定的模式。它们經常在月經前一、兩天,也可能會在結束後連續幾天發作。与平常經痛相比的差別在於,它们通常在妳初次开始月經的幾年後逐漸發展。有些人在十幾歲時就經歷过這種疼痛,但這種情況並不常見。因此,直到十九歲後,才會被診斷出子宮內膜異位症。

隨著時間的推移,內膜群周圍的發炎可能會導致疤痕和身体內部的沾黏。例如,膀胱可能黏附到鄰居,子宮。這些內部傷疤會導致不同類型的疾病,如慢性疼痛。生殖部位的慢性疼痛是子宮內膜異位症患者常見的問題。許多人在性交过程中也經歷了極度刺痛。疼痛會發生在妳腹部的最底部,而不是在妳的陰道或陰戶。

另一個問題是許多子宮內膜異位症患者難以懷孕。大約四分之一的非自願不孕案例的原因是子宮內膜異位症。我们不清楚人们為何有生育問題。疤痕和沾黏可能會損害輸卵管和卵巢,但看起來其他身体機制也是罪魁禍首。免疫系統和荷爾蒙似乎也都有所關連。如果妳正在努力懷孕,同時有子宮內膜異位症,人工授精可能會有幫助。除了人工授精之外,也可以透过手術的方式。子宮外的子宮內膜群切除手術幫助一些女性能夠自然地透过人工授精來懷孕。我们建議手術進行一次就好,而且应该將這個機會保留到打算懷孕時使用。

我们不知道為什麼有些女性會罹患子宮內膜異位症。某種程度上這是遺傳性疾病,但似乎還有許多其他因素。據我们所知,妳沒有辦法避开子宮內膜異位症

。因此這只是一個運氣不好的問題。就好比一些爺爺奶奶喜歡太陽海岸而其他人喜歡農村,有人喜歡夏天就有人喜歡冬天一樣。同樣地,有些人的子宮內膜想要搬到子宮外面。

子宮內膜異位症的問題是,它无法經由簡單的檢查被診斷出來。血液檢查、婦科檢查与核磁共振等影像檢查很難或无法告訴我们這些耍特權的子宮內膜是否存在。我们唯一能确定或排除女性是否患有子宮內膜異位症的方法,就是打开肚子看看裡面。通过內視鏡手術⸺用相機從小孔窺視胃部就能完成。和所有的手術一樣,可能會出現併發症,所以除非有嚴重或造成其他疼痛的問題,否則就不會這麼做。

除了進行這項手術外,医生經常會做的則是嘗試子宮內膜異位症的治療,看看是否有效。幸運的是,大多女性的治療很簡單。這也是无害的:不間斷服用避孕藥,或使用荷爾蒙避孕器和止痛藥也能鎮定發炎的状况。布洛芬就是屬於這類藥物的一種。持續服用避孕藥,防止子宮內膜群出血,有可能導致它们隨著時間而縮小。布洛芬对疼痛有效,並同時降低發炎現象。這種療法的目的並非消除子宮內膜群,不过問題會得到解決。

如果這種方式起不了作用,治療子宮內膜異位症還有其他更複雜的方法,譬如說手術或者更強效的荷爾蒙治療。這是涉及專業的工作。不幸的是,子宮內膜異位症是一種直到更年期才能結束的慢性疾病。療法不能治好這種疾病。即使手術切除後,子宮內膜群也會隨著時間的推移而復發。儘管如此,妳应该要知道減輕痛苦的辦法。





* * *



16	很難知道有多少人受影響,因為許多女性沒有明顯症狀,且診斷方式只有透过手術才能判定。





多囊卵巢綜合症⸺未知的女性疾病




正如我们一位女性朋友喜歡說的:「唯一比月經更糟的事情就是沒有月經。」很多女性擔心,她们的月經消失或是流量一次比一次更少。月經不規律或經血量不多的常見原因是多囊卵巢綜合症(polycystic ovary syndrome,PCOS)造成的影響。妳以前沒有聽說过吗?那麼,妳不是唯一的一個,我们应该要更加注意這種疾病有許多充分的理由。实際上,這是來到生育年紀的女性中最常見的荷爾蒙失調症,影響程度在4%〜12%之間,其中有許多人並不知道自己有這個症狀。

疾病的名稱源於多囊卵巢綜合症是經常在卵巢上發現的囊腫。這些充滿了透明液体的小水泡,使卵巢看起來有點像一串串的葡萄。与其他類型的卵巢囊腫不同,這些囊腫很小,不會破裂,所以妳不會注意到它们在那裡。

雖然這是多囊卵巢綜合症最著名的特質,但卻只是疾病的一小部分。多囊卵巢綜合症是一種症候群,意即由許多不同的疾病組成,這些疾病通常不會一起發生。而這些症狀是由一些荷爾蒙系統疾病所引起。它们不只沾黏卵巢,還包括胰腺、消化系統以及位於大腦中形狀如小型陰囊的腦垂体。

卵巢的任務是儲存所有的卵子,並确保每個月都有排卵。如果妳有多囊卵巢綜合症,這些任務可能會發生問題,因為大腦的腦垂体和卵巢都會產生控制月經週期的错误荷爾蒙濃度。結果造成妳少量排卵或是根本沒有排卵。妳會在日常生活中注意到這一點,因為妳的經血將會更少或完全消失。

為了懷孕,排卵是必要的,許多多囊卵巢綜合症女性將耗費比平常更長的時間懷孕,或者需要藉由幫助才能達成。多囊卵巢綜合症是女性懷孕問題最常見的原因之一。流產和妊娠性糖尿病(gestational diabetes)等懷孕併發症的風險也較高。

令人懷疑的是,未治療多囊卵巢綜合症的女性在晚年面臨子宮內膜癌的風險更高;這是西方世界女性最常見的生殖器癌症。在一項研究發現,健康女性一生中罹患子宮內膜癌的風險為3%,未經治療的多囊卵巢綜合症女性則為9%。

未治療多囊卵巢綜合症導致罹患子宮內膜癌風險較高的其中一個原因是,女性的子宮內膜与多囊卵巢綜合症一直增生,卻无法形成月經並脫落。因此,子宮內膜的細胞變「舊」,开始行為異常。為了确保女性一年有超过三、四次月經出血,藉由避孕藥或荷爾蒙療程的幫助便能更容易預防。

這邊要澄清一點,舊子宮內膜發生的状况和使用荷爾蒙避孕暫停月經是不一樣的。多囊卵巢綜合症裡,子宮內膜連續接收,不斷傳達生長的信號,而荷爾蒙避孕則是防止子宮內膜的生長。雖然這兩種情況下的結果都是月經變得更少,但是形成機制卻有很大的不同。

除了所有对排卵的爭论外,卵巢(以及脂肪組織和腎上腺)可能產生过多的雄性荷爾蒙,即所謂的雄激素。所有女性會產生一些雄激素,但濃度維持平衡往往对女性有利。如果雄激素占上風,妳可能發現臉上會長鬍子,或者有一條厚实的「快樂小徑」⸺妳的肚子上長出廣泛的毛髪。這就是所謂的多毛症,半數以上有多囊卵巢綜合症女性為此感到困擾。很多患有多囊卵巢綜合症的女性也有持續冒痘的問題,持續時間遠遠超出青春期。她们發胖的方式也受到影響。女性身体往往會呈現梨狀(大多數的脂肪圍繞在她们的臀部和大腿),但多囊卵巢綜合症患者的男性性荷爾蒙導致女性体型呈現蘋果形狀⸺脂肪圍繞在肚子上。妳甚至會有最不健康的脂肪類型之一,啤酒肚。然而,性荷爾蒙也有不易察覺的作用。例如,妳的血液中可能會有高含量的膽固醇和脂肪酸,這对我们的血管壁沒有好處。

多囊卵巢綜合症常常表現異常的第三個地方是胰臟。這是消化系統產生分解食物的荷爾蒙物質⸺胰島素的器官。胰島素是飯後發號施令和發送触發攝取和消耗体內血糖的訊號細胞。50%〜70%的多囊卵巢綜合症女性,細胞对胰臟的胰島素信號沒有反應。女性產生抗胰島素作用,因此胰臟產生更多的胰島素來補償,希望信號最後能夠傳遞出去。沒有人因妳講的笑话發笑吗?那就講大聲點!

高濃度的胰島素对身体沒有好處。如果妳抗胰島素状况沒有獲得控制,妳可能會漸漸發展成第二型糖尿病。多囊卵巢綜合症患者罹患糖尿病的可能性比其他相同体重和生活方式的女性要高得多。美國研究表示,多囊卵巢綜合症患者有20%〜40%的人位於糖尿病的初期或在四十歲時發展為第二型糖尿病。

抗胰島、血脂濃度異常和周圍腹部脂肪增加是心血管疾病的關鍵因素。當妳年紀大了,這些變化可能會導致心血管疾病增加的風險。

正如之後所理解的,妳应该認真看待多囊卵巢綜合症。如果妳有月經不規律的情況,多囊卵巢綜合症可能就是原因。要檢查是否有多囊卵巢綜合症,医生會衡量妳的荷爾蒙濃度並用超音波檢查妳的卵巢是否有囊腫。如果妳變成多囊卵巢綜合症的其中一員,考慮一些确保妳未來健康的事很重要。

对多囊卵巢綜合症女性最重要的建議与控制体重和改變生活方式有關。如果妳过重,減肥可能會減少問題。如果妳的体重是正常的,妳當然不需要考慮這個。減肥說比做更容易,不过任何運動与健康飲食會改善妳的健康!因為多達五分之四的过重女性,僅僅降低5%的体重(例如從80〜76公斤)就足以恢復正常的排卵。此外,也可以降低抗胰島素、糖尿病和心血管疾病的可能性。頭髮增生和痘痘的問題也會減少,因為过重增加了男性性荷爾蒙的生長。

我们甚至建議妳开始与对多囊卵巢綜合症非常了解的医師討论如避孕藥、避孕貼片或陰道環等複合產品。這是多囊卵巢綜合症治療裡最重要的部分之一。避孕藥的雌激素會減少卵巢內男性性荷爾蒙的產生和活動,有助於改善頭髮增生和痘痘的問題。此外,它可以進一步減少囊腫的發展和子宮內膜癌的風險。因為血栓而不能服用雌激素的人,可以使用无雌激素避孕法,如荷爾蒙避孕器或植入式避孕棒。但不幸的是,這些对男性荷爾蒙不會造成影響。

思考看看妳是否想要孩子,如果妳想,不要長時間使用是明智的選擇。許多多囊卵巢綜合症患者可能需要懷孕的幫助,而這個过程需要時間。事先準備好會是一個好主意。





肌瘤⸺帶球的子宮




妳最後一次見婦科医生有不愉快的經驗吗?很多人的子宮內都有良性腫瘤,也就是肌瘤。毫不意外,當妳聽到腫瘤一詞用在自己身体上時,妳的血液瞬間降溫。但是在這種情況下,妳可以鬆一口氣。只要躺在婦產科的椅子上深呼吸就好。細胞生長在子宮肌壁的子宮肌瘤是良性的腫瘤,和癌症一點關係也沒有。它们不是癌症,也永遠不會變成癌症。医生可能會將子宮肌瘤稱作肌瘤或「肌肉結」,這樣应该更容易明白良性和略不良性腫瘤之間的差異。

子宮肌瘤是由我们所謂的平滑肌構成,換句话說就是肌肉。就像我们的腸道和胃,我们不能有意识地控制它。子宮肌瘤往往以彈性球体呈現。如果妳桌上有一個,妳可以用刀子將它一分為二,會看到它裡面实際上是像珍珠一樣的白色,而不是妳所想的紅色。子宮肌瘤看起來有點像珍珠⸺在海底生長的牡蠣。

肌瘤可在子宮不同的地方生長,壁內、壁外或突出至子宮腔。有些女性的子宮肌瘤只有一個,但常見的有多達六、七個。它们可能很小或者在更糟的情況下,增大為一個葡萄柚的大小。子宮肌瘤不一定隨著時間增長。有些人可能會在很短的時間內增長非常多,當它们長到一公分長時又會縮水,並自行消失。

子宮肌瘤是女性更年期很常見的症狀。像許多和生殖器有關的東西一樣,它们对雌激素有反應,所以只會出現在青春期後,通常在更年期後消失。多達四分之一的女性會自行發覺体內的肌瘤。可能有更多人擁有肌瘤,但往往因為太小,所以沒有人會注意到。由於肌瘤只是良性腫瘤,沒有必要為了看它们是否存在而去檢查。只要不給妳添任何麻煩,擁有它们也不成問題。

妳可能會有嚴重或長期的月經出血,尤其當它们正在成長進入子宮腔時,但大多數肌瘤不會有其他症狀。

經期間出血在子宮肌瘤中不太常見。而疼痛也不是肌瘤的典型症狀,但如果長得非常大的话,有一些女性會遭受生殖器受到壓迫的疼痛感。另一個例外則是肌瘤因血液供應不良等因素开始破裂、陣亡。這可能會極為痛苦,它可以很可怕(特別是在懷孕期間發生的话)但並不危險。

如果妳想像子宮充滿了六、七顆网球大小肌瘤的畫面,很容易理解肌瘤為什麼會造成問題。例如,它们可能壓迫子宮前面的膀胱,讓妳有排尿的衝動。它们還會給妳沉重、臃腫的感覺,讓人聯想到略微懷孕的模樣,妳的胃可以長成讓妳看起來像懷有好幾個月身孕的人。

肌瘤最為諷刺的是,在最坏的情況下可能難以懷孕。還好,這只適用於少數得到肌瘤的女性,但仍然造成1%、2%難以懷孕的女性不育。不能肯定女性肌瘤是怎麼抑制懷孕,然而比起大小,位置似乎是主要原因。子宮肌瘤可能很難讓受精卵本身附著,還可能封住進入輸卵管的入口,使精子无法与卵子會合,只能焦急地等著和不錯的約會对象結合。如果懷疑肌瘤是不育的原因,可能就會進行切除,但是效果不一定如此顯著。

我们更不确定的是女性一旦想要懷孕,肌瘤會如何影響她们。同樣地,似乎是長到子宮腔裡的肌瘤導致最大的問題。一些研究表示,肌瘤向內生成時,流產的風險增加22%〜47%。除了因為肌瘤擋住產道可能造成剖腹生產更頻繁之外,肌瘤似乎沒有对懷孕產生任何重大不利的影響。因此,在有孩子之前,沒有理由透过手術移除它们。

要怎麼做才能限制肌瘤的生長呢?簡單的解決辦法就是盡量使用長效黃体製劑產品,如避孕藥、植入式避孕棒或荷爾蒙避孕器。如果受到大量出血所苦,服用荷爾蒙避孕藥也可以解決這個問題。使用低劑量的雌激素避孕方法不會導致肌瘤生長,所以妳沒有理由拒絕這些產品。

对子宮来说肌瘤有點像雀斑:妳可能會有幾個或是很多,形狀或大或小,但它们不會引起任何麻煩。沒必要因為它们的存在就將其切除。妳只需要在它们造成問題時切除就好。請記住:肌瘤不會變成癌症。





陰唇痛症⸺生殖器不明原因疼痛




妳的生殖部位疼痛,而医生和其他医療專業人員找不到可以解釋疼痛的原因吗?妳不孤單,這些缺乏真相的疼痛令人沮喪。可以肯定的是,疼痛就在那裡。它们在妳的日常生活中產生不好的影響,讓性行為難以進行⸺但它们是從哪裡來的呢?現在,我们知道的不多。

總而言之,生殖器部位疼痛的原因很多。有很多的因素會对私處造成傷害。酵母菌感染和其他生殖器疾病造成持續性燒灼和瘙癢;性傳播疾病可能導致性交時的疼痛;令我们感到疼痛的皮膚疾病,可能會影響陰戶;更罕見的是,生殖器癌也可能引起疼痛;巴多林氏腺可能會發炎而且極為疼痛⸺這樣的例子不勝枚舉。所有情況的共通點在於它们通常是顯而易見的。如果妳把疼痛的現象告訴妳的医生,他會馬上替妳檢查,並找出疼痛的原因。如果妳的皰疹經常發作,那麼生殖器疼痛也沒有什麼好奇怪的,但是如果找医生檢查,卻找不到原因呢?

如果妳生殖部位有任何疼痛,不能找到任何明确的理由往往就是陰唇痛症(vulvodynia)所致。痛症來自希臘文「dynia」的疼痛。因此陰唇痛症指的就是陰戶疼痛。

有一件事我们從一开始就強調,陰唇痛症的疼痛絕对是真实的,即使医生无法找到任何原因。很多有這種症狀的女性都有不被認真对待的感覺,得不到任何明确的答案告訴她们為何會這樣。也許她们接受大量的檢查,陸續找了好多医生卻沒有人發現問題。這是否代表疼痛只存在她们的腦海裡呢?絕对不是:疼痛感是真实的,我们認真看待妳的症狀。

陰唇痛症有幾種不同的形式,這可能意味著兩件事情:第一,導致陰戶疼痛存在一些未知的條件,但是我们知道得不多,因此將它们放在陰唇痛症的總稱下;第二,陰戶疼痛的不明原因可能隨著每個人產生不同的症狀。

這件事情的真相到底是什麼(尤其是,是什麼樣的原因導致疼痛)?我们不知道。在相關領域看到更進一步研究的话实在會令人興奮,因為幸運的是,藥物的發展一直穩步推進。在中古世紀,人们相信所有的疾病是由於体液失衡,而放血(也許用的是水蛭)這個奇妙的想法同時能奇蹟般地治癒任何從憂鬱症到癌症的疾病。給妳一個年代稍微近一點的例子,医生認為生活習慣,例如壓力和咖啡是引起胃潰瘍的因素。然而,一個叫做幽門螺桿菌(helicobacter pylori)的特殊細菌竟然才是罪魁禍首。

同理可以用在陰唇痛症下。它是一種神經疾病吗?是一種細菌或病毒引起的感染吗?是另一種療法的对應症吗?我们之後就會知道。

女性的陰唇痛症可能會經歷不同類型的疼痛:她们的生殖器可能有自發性的灼熱感,或是我们所說的触摸疼痛和痛感过敏。触摸疼痛,是通常不會感到痛覺的刺激,但只要受到輕微壓力或触摸,就會突然變得疼痛。例如手指的触摸就能夠触發陰戶的燒灼痛感。触摸疼痛通常發生在已經以某種方式受到損傷的地方。我们不能确定這是否能解釋生殖器部位的触摸疼痛。痛覺过敏指的是往往變得更加疼痛的刺激。

例如,妳一般想要擺脫的針眼可能會導致劇烈的疼痛。這兩種痛覺过敏和触摸疼痛屬於神經性疼痛(neuropathic pains)。這意味著它们會發生在傷口或周邊神經的疾病中⸺也就是大腦和脊髓外部的神經。

灼熱的疼痛和神經性疼痛是陰唇痛症最常見的形式,但我们不能肯定其他形式的疼痛不會出現。疼痛可能因人而異,如先前所述,我们不知道陰唇痛症的所有实例是否為相同疾病。另一個重要因素是我们对疼痛的解釋不同。

這適用於所有的痛苦,而不只有陰戶疼痛。例如,有些人可能會有瘙癢的不適,並認為這是由他们以前熟悉的東西,例如酵母感染所引起的症狀。即使酵母菌不是确切的原因,卻可能會使人们更頻繁地接受抗真菌治療。

疼痛的部位也會造成不同差異,也是歸類陰唇痛症的要素之一。有些人經歷整個陰戶的疼痛⸺即陰道口、陰蒂周圍和陰唇。這稱為廣泛性陰唇痛症(generalized vulvodynia),稍微年長的女性當中更為常見。其他人則是在外陰的特定位置產生局部疼痛。這就是所謂的局部陰唇痛症(localized vulvodynia),是年輕女性中最常見的。陰蒂或旁邊的陰道口的疼痛最常見,這個區域稱為前庭(vestibulum),所以這兩個局部陰唇痛症都有屬於自己的名字:陰蒂痛症(clitorodynia)以及前庭痛症(vestibulodynia)。

陰唇痛症,特別是前庭痛症,以前被稱為前庭炎(vestibulitis),這個術語你可能在媒体上聽过或讀过。當一個医学術語中以「炎」當作字尾,這代表我们正在談论發炎。舉例来说,陰道炎,和陰道發炎一樣。當女性有陰唇痛症時,由於沒有人能夠證明生殖器存在任何炎症,所以医生選擇停止使用前庭炎這個名稱。稱為陰唇痛症或是單純陰戶疼痛會更為精确。

疼痛所表現的方式有所不同。有些女性有所謂的誘發性疼痛(provoked pain),而另一些人則是自發性疼痛(spontaneous pain)。誘發性疼痛通常包括神經性疼痛,也就是痛覺过敏或触摸疼痛。誘發性疼痛意味著我们直接接触生殖器所產生的疼痛。可能會以稍微不同的形式發生,輕触或一般不會造成傷害的壓力可能會導致巨大的疼痛。例子包括自行車車座、性交、使用棉塞与陰蒂直接接触產生的壓力。妳可能變得非常敏感,即使与寬鬆的衣物或內褲接触就能引起疼痛。医生經常用來了解妳是否正遭受誘發性疼痛所苦的一項測試則是用棉花棒按壓疼痛的部位。例如,他们可能會在陰道口施加壓力。

自發性疼痛指的是疼痛突然發生,沒有与任何東西有所接触。這種疼痛往往屬於燒灼痛感。妳也可能會同時經歷誘發性与自發性疼痛。有些女性一直都有灼熱感,而其他人卻是偶爾感到疼痛。一般来说,局部陰唇痛症,如前庭痛症,最常發生誘發性疼痛,而廣泛性陰唇痛症最常發生自發性疼痛,以及透过衣物接触產生的疼痛。

陰唇痛症及其他生殖器官疾病,例如性感染疾病之間並沒有發現明确的關聯。

然而,一項主流理论表示,酵母感染療法与陰唇痛症之間有關。這並不一定表示使用抗真菌治療就會得到陰唇痛症。正如我们前面所寫,許多人認為她们經歷由真菌感染引起的陰戶不適,很自然地,就使用抗真菌治療來解決問題。這下變得難以确定是不是因為治療造成問題還是問題導致治療。

有項研究發現酵母菌感染復發与陰唇痛症之間的關係,但实驗是在老鼠身上進行,所以在兩足動物上很難得出任何結论。研究中小老鼠經歷了触摸疼痛。同樣的研究也發現受影響的區域有變得特別敏感的趨勢。能夠感知疼痛的神經末梢數量已經上升。在這項研究為基礎,看起來可能是酵母菌反覆感染影響了小老鼠純粹神經視角感受疼痛的能力。

其他研究則指出陰唇痛症的女性神經分布改變。看起來有些女性的陰唇痛症像是變為更加疼痛敏感的神經纖維。目前仍不清楚導致這些變化的原因。





好女孩症候群?




如果妳在媒体上讀过陰唇痛症,妳可能注意到很多人將疾病著重於潛在的心理層面。

許多人如此看待,尤其是性学家研究心理和性之間的相互作用,在与患者接触時也同樣強調這一點。也許是陰唇痛症是实際上不想發生性行為的女性所導致的症狀吗?難道「好女孩」是那些受到某種影響,或在过去有不好或痛苦性經驗的女性呢?那麼遭受侵犯或虐待的女性呢?這些說明已用於解釋不明原因的陰唇痛症。但是,站得住腳吗?





在身体因素无法立即辨识的状况下很容易貼上「心理因素」的標籤,对此我们应该要非常謹慎。如果女性无法分辨自己是否符合這些敘述,可能會導致混亂和憤怒。特別是「好女孩」一詞可能建立错误的印象,認為女性本身,或是她的個性,就是造成疼痛的原因。所以能力与勤勞应该是造成妨礙妳存在的身体疾病,這不是建設性的問題。也就是說,生殖部位疼痛可能对一些女性造成極大的心理壓力,但卻不是恥辱的來源。

很多患者會透过談话療法當作治療陰唇痛症的一部分。這個方式可能會產生效用,不只是因為可以改善疼痛的潛在心理因素,甚至陰唇痛症本身是人们可能需要寻求协助的極大心理負擔。

我们知道,各種疼痛与心理緊密連結在一起。很多經歷疼痛的人會逐漸養成迴避行為和緊張心態,這麼做可能會加劇潛在的問題,讓患者陷入惡性循環。舉例而言,預期性交時會感到痛苦,妳會不自覺縮緊妳的陰道來保護自己,然後在嘗試性交時會更加受傷。

此外一項廣為人知的疼痛研究顯示,當人们長時間与疼痛共存時,大腦对新疼痛的刺激變得更敏感。

疼痛只會滋生疼痛。在這兩種情況下,放鬆技巧和心理治療可以幫助人们擺脫痛苦的循環。然而,聲稱陰唇痛症從一开始絕对是心理因素所導致是不同回事。

據我们所知,沒有研究指出陰唇痛症与过往的侵犯或虐待之間有明确關係。即便如此,這樣的經歷可能是一些女性的潛在因素。研究比較了有无陰唇痛症女性的心理輪廓並得到不同的結果。一項对各240位有无陰唇痛症女性的研究中,結果指出陰唇痛症的女性先前經歷焦慮相關症狀的状况更加常見。

而另一項研究比較的群体數較少,在有无陰唇痛症女性的心理輪廓中並无發現差異。到底陰唇痛症屬於一種心理原因的疾病有多接近,是個具有爭議的問題。沒有受过心理煎熬或性暴力經驗仍舊非常有可能得到陰唇痛症。

我们对造成陰唇痛症的原因知道得不多,而療法卻是複雜又處於实驗狀態。希望改善其他疼痛症候群的不同方法,也能对陰唇痛症有所幫助。儘管如此,第一步還是要找到一個在该領域具有專業知识的医生。妳可以找到具有陰唇痛症專業的婦科医生和家庭医生。妳也可以寻求轉診至奧斯陸或特隆赫姆專門治療陰戶疼痛的診所,但不幸的是,掛號人數可能會非常多。

正如我们前面提到,某種形式的陰唇痛症都与神經性疼痛有關,在這種情況下,可以使用一些不錯的藥物如抗抑鬱与癲癇藥物。這些類型的藥物有助於对抗神經疼痛,且已經證实对一些陰唇痛症的女性發揮作用。其他人可能會發現雌激素也有效果,好比像使用陰道環等避孕方式。雌激素會影響陰道黏膜,使之增厚。同時鎮痛凝膠也能減少疼痛,而有誘發性疼痛仍想進行性行為的女性可以透过此款凝膠受惠。除了談话療法,很多人發現物理治療會有所幫助。妳可以学習專門的運動,會更容易放鬆骨盆底的肌肉。許多誘發性陰唇痛症的女性也受到其他如頸部、肩部疼痛或緊張性頭痛等肌肉緊繃問題。

經常給陰唇痛症女性的一般建議之一則是不要做任何導致疼痛的事情。例如,如果會痛就不要強迫自己發生性行為。儘管如此,如果妳想要的话可以和伴侶嘗試其他不會引起疼痛的行為。性学家在這方面可以提供不錯的建議与指導,如果妳有伴侶,帶去診療會是不錯的想法。同時在自己生殖器上要謹慎使用香水、香皂和乳霜類產品,因為在最坏情況可能會造成疼痛加劇。





陰道痙攣




提到陰唇痛症時,許多人會与陰道痙攣(vaginismus)相提並论⸺又是另一個困難、頗具爭議的症狀。陰道痙攣是女性骨盆底肌肉所圍繞的陰道口不由自主地收縮或緊繃。這些女性通常拒絕陰道插入(无论是性行為或是婦科檢查)因為會造成或預計使她们遭受或疼痛与不適。換言之,陰道痙攣會是一個兼具性行為、棉條使用及身体檢查的複雜症狀。

有些人認為陰道痙攣屬於非自願肌肉痙攣,使陰道變得非常狹窄。在挪威有時候會將陰道痙攣翻譯為陰道抽蓄。使用儀器測量肌肉活動的研究發現,並沒有明确的證據能夠表示陰道痙攣的女性會有這類的「肌肉痙攣」,而肌肉可能与陰道痙攣有關的主張也沒有得到專業人士的認同。

前庭痛症和陰道痙攣的症狀重疊。陰道痙攣經常被描述為与前庭痛症有相同或相似的疼痛。疼痛大多位於陰道口,因此与有子宮內膜異位症或因性病導致子宮頸而劇烈疼痛的女性不同。无论這兩種症狀是一体兩面,或是兩種單獨卻經常同時發生的疾病,都難以分辨。

陰道痙攣的療法与陰唇痛症幾乎相同。对於陰道痙攣的女性,往往會額外訓練她们能夠承受有東西在陰道裡;通常從女性自行插入非常薄的物体,如擴張器开始,接著逐漸增大尺寸。鎮痛凝膠持續在插入过程中使用,以致不會过於疼痛。這種療法可以由婦科医生、性学家或物理治療師共同進行。

陰道痙攣和陰唇痛症令人難以置信地限制患者的生活樂趣与性生活。对於許多女性来说,只要症狀仍然存在,不可能會有正常的性生活,而与伴侶的關係會惡化或以失敗收場。很多人擔心她们是否永遠不會有伴侶或孩子,或是她们不得不在剩下的人生裡獨自生活。她们覺得自己无法克服。這些症狀可能導致的痛苦,以及許多女性在医療体系上可能遭遇的傷害,我们的認知並不多。一個令人感到小小的安慰及更進一步的資訊是,大多數女性都會有所改善,甚至許多人會變得完全健康⸺即使陰唇痛症往往是持續時間長的慢性疾病。





衣原体、淋病及它的遠房親戚




當然,我们是实境節目《樂園飯店》的大粉絲,甚至因為其中一位男性參賽者聲稱他光看女生就知道对方有沒有性病,因此他從來不使用保險套時放聲大笑。我们不知道他受到什麼樣的祝福。或許他在霍格華茲学院得到證書或和其中一位電視靈媒有所關聯呢?不过,有一件事可以肯定:沒有人可以光看女性(或男性)就能分辨他们是否有性病。很多人甚至不知道他们已經感染了⸺這就是問題的核心。人们即使有了性病仍然不使用保險套進行性行為。妳不知道的疾病就此蔓延。

我们一般稱性交傳染的疾病,為性傳染疾病或簡稱為性病。當妳与其他已經感染的人從事性行為或性接触時就有可能感染性病。性病是由不同類型的微生物,如細菌、病毒和寄生蟲所引起。有些性病只能通过体液如血液和精子傳遞。其他疾病則可以透过皮膚和黏膜之間的接触來傳遞。

某些性病極為常見,有的在這些章節裡則是比較罕見。妳在人生中有可能得到一種或多種性病,這是做愛的缺點之一。

由於性慾一直与羞恥和罪惡感有關(特別是女性),性病也是如此。即便現在,很少人能夠开誠布公談论自己得到尖銳濕疣(genital warts)与衣原体感染(chlamydia)等疾病。雖然這些情況甚是常見,有時很難防範,很多人都有了应该要減少非正式性行為的次數避免讓自己的伴侶曝露於感染風險的想法。我们希望性病相關的知识和規範可以消除一些難受的羞恥感。感染是不當使用保險套最重要的問題,接著才是運氣好坏。這並不是妳的「個人性道德」問題。有些人和好幾百人上床不使用保險套,卻像奇蹟一樣沒有得到感染,而其他人可能一夜情一次就得到尖銳濕疣。妳的性生活同樣也會發生鳥事。

在我们有現代藥物与抗生素前,有些性病比丟臉還要更糟糕。還會造成嚴重的症狀,在最坏的情況下,則是死亡。有很長一段時間,淋病(gonorrhoea)經常造成儿童在出生時受到母親感染而失明。這種情形普遍到在挪威所有新生儿出生時都要滴入治療淋病的眼藥水。易卜生(Ibsen)於一八八一年的戲劇《群鬼》(Ghosts)中,將梅毒(syphilis)當作為一個單獨的角色可能就是一種詮釋。注入藝術靈魂的梅毒,在劇中最終攻擊儿子奧斯瓦爾德(Osvald)的大腦与中樞神經系統。現在,我们可以用青黴素(penicillin)消除梅毒,使感染的人重返身体健康。在《群鬼》發表時的一八八一年是不可能做到的,許多人遭受像奧斯瓦爾德的窘況並死於這種疾病。

儘管医学進步,性病仍然是全球公共健康的主要障礙。自一九八〇年起,當愛滋病奪走數千名年輕男同性戀者的性命,但新聞上卻甚少提及病情。有充分的理由,愛滋病,或稱後天免疫缺乏症候群(Acquired Immune Deficiency Syndrome,AIDS),是導致免疫防禦(意即人体抵禦細菌、病毒和其他垃圾的保護罩)崩潰的疾病。

愛滋病病毒的微生物与人類免疫缺陷病毒(Human Immunodeficiency Virus,HIV)有關。在二〇一五年,有110萬人死於与HIV相關的疾病,現在超过3670萬人与病毒共存。自疫情开始蔓延,有3500萬人失去了生命。一旦妳感染了HIV,就沒有辦法擺脫。在挪威,HIV呈陽性的人接受良好治療後現在幾乎可以正常生活。有了徹底的治療,它们將不再具有傳染性。因此有藥物能夠控制病毒,但遺憾的是,世界上受到感染的人只有一半有機會獲得這些藥物。

現在的挪威,无论是梅毒或是HIV已經不再蔓延,但疾病仍确实存在。二〇一五年,有221人被診斷為HIV,其中189人在二〇一四年診斷出梅毒。与其他性病相比數字是令人難以置信的小,但也幾乎可以認定為流行性疾病。妳可以透过血液檢查診斷自己是否得到HIV和梅毒,但因為罕見,除非妳特別曝露在受到感染的風險下,不然沒有必要定期檢查。

挪威最常見的細菌性疾病稱為衣原体。二〇一四年有292‚772次衣原体測試,其中有248‚117個呈現陽性⸺也就是多達8%!陽性結果代表發現了衣原体。在十五至十九歲与二十至四十歲的組別,陽性測試的數量比所有其他年齡組別還高。十五至十九歲的女生呈陽性的比率為13.6%,而男生則是16.1%。二十至四十歲族群中,呈現陽性的女性為10.6%,男性為16.3%。因為很多人沒有進行測試,所以我们必須假設未統計資料的數量極多。

呈現陽性的大多為女性:不會少於60%。這並不表示女性得到更多衣原体,只是比較容易被檢測到而已。妳可以從十五至十九歲和二十至四十歲的組別得知,擁有衣原体的男生比例,在做实際測試時更高。這意味著更多的男生比女生更容易帶有衣原体,只是我们不知道罷了。

看起來有些男孩好像拿上女孩服用測試的護理的機會,並承擔他们會打個電话,如果过去的合作夥伴測試陽性。這個策略不太上道,更不用說離无懈可擊還差得很遠。即使妳的性伴侶檢驗結果呈現陰性,妳可能也會有衣原体。每次性接触的感染風險並非100%,因此伴侶雙方应该要去做檢查。換句话說,樂園飯店裡不用保險套的參賽者和認同他的一大群人必須要改變他们的習慣。即使妳自己有做檢驗,當妳和新对象發生性關係時,使用保險套始終是明智的舉動。无法确保妳的伴侶會和妳一樣聰明。此外,我们偶爾會忘記使用保險套而導致木已成舟的遺憾。如果妳忘記使用避孕套,那麼做檢驗很重要。

有點像衣原体的兩種細菌性疾病分別為黴漿菌(mycoplasma)和淋病。保險套能有效地保護妳免於受到上述三種疾病的感染。淋病在挪威比衣原体感染更為罕見。二〇一四年,有682人被診斷出淋病。其中只有119位是異性戀女性,但是感染人數卻愈來愈多。

黴漿菌是經常被家医科忽視的疾病。它有點像是衣原体的胞弟。它们非常相似,具有相同的症狀,可能也有相同的後遺症⸺我们會回过頭來提到。然而,除非患者有症狀,不然不用定期做黴漿菌檢查。即使如此,許多医生做檢驗時也不會想到它。黴漿菌感染的治療方式与衣原体不同,所以找出疾病很重要。如果妳有症狀,但衣原体檢驗結果呈現陰性的话,要求進行黴漿菌測試是明智的作法。

衣原体、黴漿菌及淋病最常見的症狀為分泌物改變或增加、排尿時感到刺痛、生殖器不適或生殖器、尿道及肛門有騷癢感,取決於感染的位置而定。這三種細菌性疾病經常攻擊子宮頸,造成紅腫。同時使性交不愉快或疼痛,由於壓迫疼痛的子宮頸,有些人可能在性交後或途中會發現稍微出血的現象。一般情況下,在我们不知道原因時,应该对任何的陰道出血保持警覺⸺尤其是与性行為有關的出血。例如,月經或使用荷爾蒙避孕可能是出血的原因之一,但是不明原因的出血可能是性病或其他疾病引起,所以应该由医生進行檢查。

然而,並不是每個人都有症狀。事实上,只有一半,最少三分之一的女性得到衣原体的症狀。黴漿菌的症狀也不常見,也有人感染淋病卻沒有相關症狀,所以我们為什麼要替根本不會注意到的東西操心呢?嗯,首先,細菌性疾病極具傳染性。无防護性行為感染衣原体的風險是20%。其次是長期損害的危險。

如果細菌有機會,它们可以自行經由子宮頸,最後停留在子宮及輸卵管。它们可以在那裡引起發炎。這稱為骨盆腔炎疾病,妳可以從衣原体、黴漿菌和淋病感染得到17 。根據統計,未經治療的衣原体會引起10%〜15%的人發展為急性骨盆腔炎。危險的是,發炎後可能導致輸卵管留疤,造成阻塞。這是女性難以懷孕的常見原因,而除了這一點,它也會引發慢性疼痛。

如果妳得到骨盆腔炎,很常感到噁心与不適,而妳經常會下腹劇痛、陰道出血、發燒和分泌物增加。通常疼痛不會減輕或好轉,只會更嚴重。這些類型的症狀应该要認真看待並盡快找医生檢查,如果有需要則必須進行緊急手術。

雖然不常見,不过的确有可能得到沒有症狀的骨盆腔炎,只會在進行不孕治療後幾年發現。

所以,就算有其他原因,更換性伴侶後記得去做定期檢查。

衣原体、黴漿菌及淋病可以用抗生素治療。現在,受到感染的人大多能完全恢復健康,不會有長期的損害,但是抗生素的抗藥性,特別是对黴漿菌及淋病逐漸上升令人感到擔憂。抗生素的抗藥性是指細菌对某些類型的抗生素免疫,因此需要更強的藥物來擺脫它们。換句话說,避免感染的最好辦法就是在一开始使用保險套。

有些性病是甚至比衣原体更為常見:皰疹和HPV,這兩者都是病毒型疾病。HPV意即人類乳突病毒(Human Papillomavirus),它有許多不同的形式。某些類型引起尖銳濕疣。其他的可以導致子宮頸癌。皰疹与感冒瘡相同,而且是會在皮膚上形成小水泡的疾病。

皰疹和HPV是透过皮膚和黏膜之間的接触傳播。我们不知道到底有多少人感染了這些疾病,但兩者蔓延的情況都非常嚴重,人们往往不會注意到自己已經受到感染。

因為沒有任何症狀,許多人被不知道自己已經感染的伴侶傳染,造成感染難以防止。同時也不能肯定保險套就能提供足夠的保護。舉例来说,如果陰莖根部有尖形濕疣或皰疹的男性,即使使用保險套仍然會傳染給他的伴侶。保險套无法完全覆蓋感染區域。

妳可以接種疫苗抵抗HPV,有些疫苗能夠給予病毒引起的尖銳濕疣和子宮頸癌的防護。如果妳得到尖銳濕疣,可以用冷凍療法(用液氮冷凍)

進行治療,或用不同的藥物擦拭讓它们消失。換句话說,和治療妳在游泳池淋浴時所得到的疣非常類似。尖銳濕疣沒有危險,也不會造成罹癌的風險。疣和癌症分別屬於不同類型的HPV病毒。

HPV感染往往會自行消失。疣也一樣,但是有些疣會反覆生長造成困擾。

皰疹,同時是妳无法擺脫的病毒。一旦遭到感染,病毒會處於一種休眠狀態永遠留在妳的神經細胞。妳可能好幾次爆發,可以透过處方藥物降低次數。然而,皰疹並不危險而且問題往往隨著時間的推移不斷減少。





我该如何保護自己不受性傳播疾病傳染?




保險套提供HIV、衣原体、黴漿菌及淋病良好的保護。然而,HPV和皰疹能夠透过皮膚接触傳播,所以妳可能會從沒有保險套覆蓋的地方受到感染。

幫女性口交時,可以使用口交膜片⸺一款輕薄、透明的乳膠片,可以放在陰戶上面。這將會防止皰疹從嘴部到生殖器或是從生殖器到嘴部感染。口交膜片並非特別实用(而且不太能固定住),所以沒有被廣泛使用。如果妳想仍然使用它们,妳可以剪斷保險套頂端,從圓筒裁切並攤开,妳就能得到一張大又透明的方形薄膜。





我什麼時候该去做檢驗?




即使妳沒有任何症狀,每次与新伴侶進行无防護性行為後進行衣原体檢驗是聰明的想法。在關係初期儘早和妳的伴侶一起檢查也是一個好主意。既然妳可以在察覺不到異狀,而得到性病很長一段時間,那妳实際上可能也有衣原体而不自知。如果妳沒有症狀,通常足以在妳的医生或年輕人常去的診所透过尿液或使用棉花棒对陰道或肛門採樣來進行自篩。

如果妳有过无防護肛交,卻沒有進行肛門檢查的话,很難确定是否受到感染。這樣的话,妳应该進行肛門檢查。

如果妳有症狀,妳可能還需要進行生殖器檢查,由妳的医生決定。重要的是,如果妳排尿有刺痛、有瘙癢、分泌物變化、起疹子、水泡、不寻常的出血或是妳發現有任何其他的東西就要聯繫妳的医生。

除非患者有特別的症狀,不然不用進行皰疹或HPV病毒檢查。

要注意,只有在可能暴露於感染威脅後的兩個禮拜進行衣原体測試才是有效的。這意味著如果發生性行為後兩週或以上進行測試,妳可能會得到陰性反應。妳當然可以在初期自行測試。很多人在早在兩週前就測到陽性反應,這是好事,因為這意味著她们可以即早治療。如果妳在兩週後得到陽性結果,妳可以肯定妳得到衣原体感染。但是,如果早期測試的結果是陰性,至少在妳可能暴露於感染威脅後的兩個禮拜進行新測試前妳都无法得到完全肯定的結果。

兩週法則也同樣適用於黴漿菌及淋病檢查。





度假性交的風險与危險




現在,我们已經討论一長串的性病,不过主要集中在衣原体的檢查。那麼其他的疾病呢?有些女性去医院要求医生進行「全部」的檢查,但是真的沒有必要每次讓自己接受所有的測試。妳应该要做的檢查必須和妳的医生共同決定,而這將取決於妳有可能感染到的性病。

如果妳是生活在挪威的異性戀女性,只和居住在挪威的異性戀男性發生性行為,比起在泰國度假与性工作者進行无防護性行為得到HIV、梅毒与淋病的風險相对較低,不言自明。在挪威的異性戀者當中,衣原体絕对是最常見的性病,而且通常能夠被診斷出來。

但是,如果妳在國外度假有无防護的性行為,必須要告訴妳的医生。医生經常會忘記詢問,所以不要指望他们主動开口。即使妳只有在滿月派对上与可愛的瑞典男孩發生關係,還是去做個測試吧。妳根本不知道在他们背包假期當中除了妳還和什麼人發生性行為。如果妳和從性病很嚴重的國家旅行回來的人進行性行為也一樣得去檢查。為了以正視聽,不只是前往泰國才需要留意;許多挪威人在,德國、西班牙和波蘭得到感染,這些國家的性病發病率和我们非常不同。

如果妳有賣淫或買春,妳要採取的檢查範圍必須更廣。這也同樣適用於注射藥物或与注射藥物的人有过性行為。

在挪威,男性与男性間的性行為(men who have sex with men,MSM)族群得到更嚴重性病的風險最高18 。

MSM比異性戀人口更常得到淋病和梅毒感染。 讓這些男性進行檢查也就格外重要。必須記住這也適用於与MSM進行性行為的女性。如果妳最後一夜情的男性也和男性做愛,那麼得到性病的風險比只与其他女性做愛還高。針对MSM並非羞辱⸺這只是單純的統計問題。无论妳和女性、男性或与男性發生性行為的男性做愛,妳的運氣好坏參半。

自己去做較不常見的疾病檢查不會有任何害處,但如果風險不是特別高,妳就不需要每次都去做。盡可能地經常檢查,根據風險進行檢查,並且使用保險套。





* * *



17	業界不認同黴漿菌會導致骨盆腔發炎。该領域的研究仍然稀少,但有少數小型研究認為的确屬实。畢竟小心駛得萬年船。



18	比起「同性戀」我们傾向使用「男性与男性之間性行為」。男性有可能在尚未确認同性戀的情況下与男性性交。性取向和妳發生關係的对象為何不同。





皰疹⸺妳的性生活完蛋了吗?




微小、疼痛的水泡出現在妳的嘴唇或生殖器上聽起來一點也不好玩,但皰疹(herpes)比妳想像得更常見。皰疹具有傳染性,既麻煩又難以抵抗,但幸運的是它是无害的。即便如此,皰疹似乎很多人最害怕的性病。

許多人害怕无法擺脫皰疹的困擾。一旦妳被感染,病毒會在一直留在妳身体裡。因此造成了很多疑問。舉例来说,這代表妳一直都有傳染性,所以永遠不能和任何人進行无套性行為了吗?

皰疹突然在關係中出現也造成很多的不信任与不确定。是誰感染了誰?在一起三年的伴侶对妳不忠吗?

關於皰疹有很多、非常多誤解。无论对被感染或害怕感染的人来说,因為皰疹而感到焦慮是正常的。

皰疹是一種影響皮膚和黏膜的病毒性疾病。兩種略微不同的病毒可能就是罪魁禍首:第一型

單純皰疹病毒(HSV-1)和第二型單純皰疹病毒(HSV-2)。皰疹病毒是透过与皮膚或黏膜如接吻或性行為接触進行傳播,它也可以間接傳遞。典型的例子則是幼稚園孩子彼此吸吮相同的塑料恐龍後得到感染。超过一半的人在幼儿時期嘴巴可能感染过HSV-1病毒。

沒有任何的紀錄所以我们不知道到底有多少人感染了皰疹。但是這一次可以幾乎正确的說,每個人都有得过,不像那次妳試圖說服爸爸說別人都有GameBoy,妳也必須有一個了。據悉,多達70%的人感染了HSV-1病毒,而40%的人得到HSV-2。妳可能感染這兩種或其中一種病毒。最重要的是,大部分人口都可能感染过皰疹,被誰感染並非主要的問題。

只要停下來稍微想想這些數字。畢竟,這代表感染比沒有感染頻率更高而已。即便如此,很多人認為皰疹就是世界末日。但是超过70%的挪威人生活還沒毀滅且還能做愛呢!





等等,口腔皰疹和生殖器皰疹是兩種不同的疾病,不是吗?那麼,為什麼我们談论它们,好像是同件事情一樣。性傳播感染和感冒瘡完全不同,不是吗⋯⋯?

事实上,无论皰疹長在妳身体哪裡都是一樣。以前,人们認為HSV-1病毒主要与口腔皰疹有關,而HSV-2病毒則是和生殖器皰疹有所關聯。假設這是妳所感染的部位,HSV-1病毒很容易在生殖器上爆發,同時HSV-2病毒也很容易長滿嘴巴。妳也可能在肛門周圍、手指或(如果妳真的不走運)在妳的眼睛上長出皰疹。也就是說,HSV-1比HSV-2在生殖器上所展現的症狀較少也比較溫和。

所以生殖器皰疹也是感冒瘡,而口腔皰疹可能透过性傳播感染造成。從生殖器傳染到嘴唇是可能發生的事情,而且從嘴巴傳染的方式更為常見。現在大多得到生殖器皰疹的年輕女性都是經由嘴部感染HSV-1病毒的伴侶口交導致,比率多達80%。

既然有這麼多人得到皰疹卻不自知,這代表在实際上許多年輕女性被不知道自己已經有皰疹的伴侶傳染。所以妳该如何保護自己呢?

一旦妳被感染,病毒會在幾天內爆發,但也有可能在沒注意到任何情況下遭受感染。妳感染後,皰疹病毒群會從神經向上移動至造成感染的皮膚區域。它们會定居在妳身体略為深處的神經細胞內睡覺,就像進入冬眠的熊一樣。它们在妳的餘生都會繼續待在那邊。有時候,病毒透过神經向下移動後又重新回到妳的皮膚上。接著造成新的爆發,和上次一樣在同一個地方形成水泡。同時也可能有隱藏的爆發⸺可能是皮膚上妳沒注意到的病毒。一隻看不見的熊從冬眠中醒來。

有形皰疹發作的不適感會以刺痛的形式开始,在妳的生殖器或嘴唇皮膚引起燒灼感。然後出現長成一團的小水泡,幾個幾個堆在一起。过了幾天,水泡乾掉結痂,最終脫落。

第一次爆發通常是最嚴重的。通常稱為初次皰疹爆發,會对某些人造成非常嚴重的状况。因為生殖器的急劇疼痛,可能會有發燒或排尿問題。如同每件事一樣,如果妳得到嚴重的症狀卻不知道出了什麼問題,妳就应该去看医生。初次皰疹的爆發比起其他疾病持續的時間較長。妳可能會在一到兩週內得到新的水泡,結痂會在三、四週後完全消失。如果妳有一個戲劇性的初次爆發,認為下一次的爆發不會那麼糟糕也許能讓妳舒坦一點⸺事实上,妳不會有下一次爆發。許多人初次得到之後就再也沒有过了。

如果妳有新的爆發,總是會發生在妳第一次感染的同個地方。爆發的數量通常會隨年減少。沒有藥物能夠擺脫皰疹,如果妳很迫切,藥錠可以有鎮靜和縮短爆發的作用。每年特別麻煩的情況都与許多爆發有關,妳可以長期使用藥物治療來抑制症狀。

新的爆發往往在妳免疫力低的時候出現。這就是為什麼口腔皰疹的常見名稱叫作感冒瘡。妳經常會在生病,例如感冒的時候得到。壓力、月經或陽光也可能触動皰疹爆發,刺激皮膚⸺例如內褲摩擦、蜜蠟除毛或剃毛也會造成。

皰疹和HPV病毒一樣沒有疫苗,但实際上也不需要有。皰疹就像是对抗自身的疫苗:如果妳在孩童時期已經感染了一次,妳之後就不會被相同的病毒感染到身体的其他部位。病毒触發妳的免疫系統,這樣它们就會识別相同的病毒,並防止病毒定居在新的神經細胞。這代表每個病毒只會感染一次。如果妳的嘴巴遭到感染,那麼妳的生殖器就得到防護,反之亦然。

但是,就如妳目前所知,皰疹病毒有兩種。如果妳之前已經感染了HSV-1,妳不會有HSV-2的防護。理论上,如果是兩種不同的病毒,妳會在兩個地方得到皰疹。然而,必須要說的是,妳會有一定程度的交叉保護。如果妳感染了第二型病毒,妳的症狀通常會較輕或沒有症狀。

由於皰疹的運作有點像疫苗,所以妳不會自己感染自己。如果妳有生殖器皰疹,病毒就无法移動到身体的其他地方。但是要注意!只有在免疫系統受到触發時才有用。然而,妳的免疫系統需要一點時間來適應皰疹,所以妳实際上可能在其中一種皰疹病毒初次爆發時讓自己受到感染。所以妳第一次爆發時要特別注意洗手和衛生。當妳病毒在妳的手指上時不要揉眼睛。不要這樣做!

即使在初次爆發後妳不會讓自己受到感染,卻還是可以傳染給別人。我们最常遇到有關皰疹的問題是:我什麼時候具有傳染力?很自然地,有生殖器皰疹的人會害怕傳染給別人⸺那妳又该怎麼知道妳是安全的呢?藥物治療可以防止感染吗?還是有不应该進行性行為的特殊時間點呢?

皮膚或黏膜必須要有病毒存在才能透过皮膚和黏膜接触造成的傳遞。由於皰疹往往深藏於体內的神經細胞冬眠,所以妳通常不會具有傳染性。病毒必須從神經向上移動到皮膚,這樣妳才能夠感染其他人。這是妳爆發時會發生的情況。爆發前一週与爆發當下是妳傳染性最強的時候,因為病毒都聚集在皮膚上。水泡裡充滿了病毒。在妳感覺到皰疹即將爆發時,避免性行為可能是明智的作法⸺通常發生在水泡出現的前幾天。不过當然,在前一週有可能也難以确定這次的爆發是否嚴重。

同時還有一些隱藏的爆發。病毒可以在妳完全沒注意到的時候漫布到皮膚上,也不會出現任何的水泡。即便如此,妳還是具有傳染力。实際上這代表妳不具有一般的傳染性,但妳可以在任何時候有傳染性。妳永遠不能肯定妳沒有傳染性。沒有安全的時候。現在,妳可能會想:但這是完全的危機啊!实在很難确定妳不會感染別人,這也是受到感染最困難層面。但是再想一想。

比方說妳生殖器得到HSV-1病毒卻想要和新对象發生性關係。妳的潛在伴侶已經感染病毒的機率有70%,因此會对新感染造成防護而不自知,這樣就降低了極大的風險。此外,如果妳伴侶的嘴巴長了感冒瘡,妳幾乎可以肯定妳不會傳染給对方,因為口腔皰疹通常是由HSV-1病毒引起。如果妳的伴侶已經感染了HSV-1病毒,他们对妳帶來的新感染形成了防護。

另一種分辨方式則是大多數人无论早晚都會感染。如果妳沒有感染他们,別人以後也會感染他们。皰疹是无害的,而大多數人卻很難注意到自己受到病毒感染。

總之,我们需要討论一下和皰疹有關的一個棘手問題:皰疹对關係的影響。比方說,妳和妳的伴侶以前沒有得过皰疹。妳的嘴,妳的生殖器上也都沒有。妳们已經在一起三年,並擁有一個夢幻般的關係。然後皰疹就此發生。妳的生殖器上爆發嚴重的水泡,而妳有了最糟糕的念頭。妳沒有和其他人發生關係,所以一定是妳的伴侶,難道不是吗?

妳現在知道了,妳也不一定知道自己有皰疹。妳受到感染時不一定會爆發水泡。妳得到皰疹可能有很長一段時間卻沒有可見的水泡爆發。當然也很有可能是妳被妳无形水泡爆發的伴侶的感染。換句话說,不必考慮不忠的事了!皰疹是常見的,而妳也不一定知道妳有。我们已經看到皰疹爆發後許多沒有根據的不忠指控毀了好多關係。

當然,不忠可能已經介入,但皰疹不能證明這一點。如果妳沒有任何理由懷疑妳的伴侶,那麼皰疹不应该是撒下不信任種子的因素。

背負不傳染性病給伴侶的責任实在很偉大。如果我们講到的是衣原体,我们會大聲鼓掌,但是如果是皰疹,往往只會讓我们感到難过。沒有必要讓人因為皰疹而害怕做愛。皰疹不是HIV病毒,即使它们都是妳无法擺脫的病毒。皰疹完全无害。感染上生殖器皰疹並非世界末日。

妳是許多人裡的其中一人。沒錯,事实上,是多數人之一。皰疹非常有可能在妳生活中造成極少的麻煩。如果妳确实遇到問題,它们是有機會補救的。如果妳是有很多爆發的幾個不幸的人之一,還有治療方法可以採取。





劇烈瘙癢和腐爛的魚腥味⸺妳肯定會遇到的生殖部位問題




妳的雙腿之間有東西即將發生。它是紅色,氣味特殊,或者它非常癢讓妳无法在晚上睡覺。酵母菌感染和細菌性陰道炎是非經由性傳播感染所引起的常見生殖器官疾病。大多數女性在生活中會得到其中一種,或兩者都有。這兩個症狀是无害的,但會是令人難以置信的麻煩。因為妳可能會遇到這些生殖器疾病,所以对它们了解更多可能相當值得。

微生物,如細菌和真菌通常触發負向關聯以及他们非常嚮往肥皂及廚房噴霧。有誰沒聽过細菌在抹布繁殖的速度有多快,或者看过真菌是如何在潮濕的地下室散播出去?這足以讓妳顫抖⸺但不是所有的微生物都是有害的。

有些細菌对我们身体運作不可或缺,例如幫助我们消化的腸道細菌。事实上,我们身体裡的細菌數大約是其他細胞的十倍,這並不代表我们生病了。

陰戶与陰道的黏膜由所謂的生殖器正常菌群構成的微生物所覆蓋。正常菌群藉由支持免疫系統对抗外來微生物有助於保持妳的陰道健康与維持陰道內環境的平衡。妳可能還記得,陰道有自淨功能,如果使用肥皂,特別是沖洗劑會消除生殖器的天然保護。

陰道正常菌群會根據妳所在的生命階段而變。在妳進入青春期和更年期後,正常菌群主要由皮膚和腸道細菌組成,但是當妳受孕時,妳的身体被雌激素所影響。這使得黏膜變厚、活躍,而正常菌群对生殖器来说也就變得獨一无二。它不同於身体其他部位的正常菌群。

育齡女性的正常菌群包括不同類型,依賴雌激素的營養來生存的乳酸菌与乳酸桿菌。乳酸桿菌產生的酸就像優格一樣。乳酸确保陰道維持在大約4.5的低pH值,因此創造了一個对坏細菌類型不友善的環境,因為它在酸性環境下會感到不適。此外,裡面還有一些其他的細菌:些微的酵母菌和病毒。所有的微生物都在爭奪相同的食物与住處,由於有這麼多不同的類型,它们沒有一個能占上風。加上身体的免疫系統,不同的微生物會相互制約。當具有保護力的正常菌群失去平衡,生殖部位變得脆弱且很容易造成問題。





陰道酵母菌感染




我们先從酵母菌感染开始。大約20%的女性有一種被稱為白色念珠菌(candida albicans)的酵母菌,為陰道正常菌群的一部分。許多人的屁股有這種酵母菌,尤其增生機會很大的话,它可能會轉移到陰道。多達50%的懷孕女性陰道裡有酵母菌存在。可能是因為白色念珠菌喜歡雌激素,而懷孕時身体充滿了雌激素。白色念珠菌是挪威酵母菌感染最主要的起因。

等等,酵母⸺妳是說我们加進捲餅和麵包的東西吗?差不多是了!和妳在超市看到的酵母是不完全一樣的類型,但卻相似。其实,二〇一五年十一月有一名酵母菌感染的女性曾使用她陰道的酵母製造老麵,使她成了网路大紅人。她的作法是用人工假陰莖收集她的分泌物。她將發酵後的老麵烤成麵包,並吃了下去。她說味道「真是该死的有夠好吃」。

如果妳是陰道內一直充滿酵母菌的20%其中之一,並不代表妳有酵母菌感染;只有當酵母菌引起黏膜發炎時才算。換句话說,當妳得到感染時,妳完全會知道。

酵母菌感染,亦稱作鵝口瘡,能夠影響陰道內部与小陰唇。搔癢感可能十分強烈,而一些女性下体會感到刺痛或灼熱。一樣的,這也會使性交疼痛,或在妳排尿時引起陰戶刺痛。受感染的黏膜變得有些紅腫。有些女性同時也有乳白色塊狀分泌物,像是低脂乾酪一樣,而其他人則是液態分泌物。

有些女性發現自己得到陰道酵母菌感染時,男性伴侶的陰莖也會有如起疹或瘙癢感的症狀。然而,我们必須強調酵母菌感染並非是一種性傳播感染。即使得到感染或正在接受治療,妳仍然可以發生性關係。男性的症狀通常不需要個別處理。妳有能力可以擺脫感染,而他的症狀也會消失。

由於酵母感染或鵝口瘡非常普遍,因此藥店專櫃上會販售治療藥品。治療的藥物有許多不同的類型,每種都同樣有效。包括藥膏和陰道藥劑,或是口服抗真菌藥錠。如果妳使用陰道藥劑,妳应该在妳上床睡覺前放入,讓藥效可以在晚上運作。如果沒有,藥丸可能會溶解並迅速流到妳的內褲。如果妳使用的是藥膏,妳需要從小陰唇开始塗抹薄薄一層,一路經过陰蒂最後擦到肛門。在月經來潮時避免使用陰道藥劑會是好主意,血液可能會將藥物帶離陰道⸺也可以說是將它沖出來。

當女性有類似鵝口瘡症狀時這些非處方治療有效降低了她们診斷和治療自己的門檻。

問題來了,並非所有的搔癢就是鵝口瘡!如果下面搔癢,只有50%的可能性為鵝口瘡。不同的生殖器疾病可能會彼此類似。因此,我们強烈建議有新症狀的女性应该去拜訪医生。瘙癢和分泌物的變化可能是任何疾病引起的模糊的症狀,例如像衣原体、淋病等性感染疾病,這些問題值得儘早分辨。不同類型的濕疹和生殖器的刺激症狀也很常見,可能是因為內褲殘留的洗滌劑,或者使用香芬皂或濕紙巾。

即使女性得过鵝口瘡,她们仍不善於區分鵝口瘡和其他生殖器疾病。女性正确診斷出鵝口瘡的機率只有三分之一。如果在這些情況下,她们選擇非處方治療,而不是去看医生,會導致了很多沒有意義、不正确的治療,无助於消除疾病。非必要使用抗真菌治療也可以延遲实際疾病的發現,造成額外症狀發生。事实上,廣泛使用抗真菌藥物會刺激黏膜,讓人聯想到酵母感染。換句话說,出門看医生确認是不是鵝口瘡一點也不愚蠢⸺至少在第一時間妳發現問題,或者妳發現症狀不斷反覆出現。

當妳被診斷患有鵝口瘡並使用抗真菌的藥物時,按照妳的医生或藥劑師建議的方式很重要。即使症狀已經消失,妳仍必須完成療程。症狀消失後繼續使用藥膏至少兩天,否則鵝口瘡可能很容易再長回來。如果妳太早完成治療,可能導致少量鵝口瘡累積,當妳停止服用後感染就會再次爆發。

酵母菌感染是常見的,我们知道有四分之三的女性在生命中會遇到這個問題。是什麼樣的原因造成呢?实際上很難有确切答案。我们知道的很多。我们容易得到酵母菌感染。我们知道使用抗生素後,許多女性會得到鵝口瘡,或是因為她们清洗自己的生殖器过於頻繁。畢竟,肥皂和抗生素有助於減少維持我们的生殖器健康的正常菌群。我们也知道,雌激素和酵母菌感染有關。青春期前和更年期後的女性很少有鵝口瘡的問題,因為生殖器不會受到性荷爾蒙的影響,而孕婦可能經常感到困擾。我们知道鵝口瘡通常會出現在月經週期的某個時間點。不像細菌性陰道炎,女性最容易在月經前得到,我们很快會提到這點。

糖尿病患者特別容易得到鵝口瘡,尤其是血糖控制不良的人。我们也看到女孩變得性積極後更容易得到鵝口瘡,而一個月多次性行為的人也是如此。

有些長期受到酵母菌感染的女性症狀永遠不會消失,对這些女性可能會造成一大阻礙。有3%〜5%的女性一年會得到4次以上的酵母菌感染。如果妳很容易感染,必須要告訴妳的医生,因為妳可能需進行適當的檢查和比非處方產品藥效更強的抗真菌治療。

不幸的是,沒有有效的方法可以防範鵝口瘡。然而,无论是网路上民間偏方或找医生動手術都非常盛行。其中一項常見建議則是吃優格補充妳陰道的乳酸菌,无论是藥丸或喝大量的乳酸飲料,如Actimel。然而,這種治療尚未證实有效,所以這也許是在浪費金錢,除非妳真的非常喜歡Actimel。

除此之外,鵝口瘡喜歡潮濕、溫暖的環境,因此人们一般被要求保持自己的生殖器部位乾燥。這代表妳应该避免穿著合成纖維的內褲和緊身長褲,在絕对必要時只能使用護墊。因為透氣極佳所以建議穿上棉質內褲,而裸睡能夠讓妳的生殖器得以喘息。沒有一種方法有科学效用的記載,但如果妳真有鵝口瘡困擾,都值得一試。畢竟,妳可以自由選擇,而且也沒有副作用。





細菌性陰道炎




現在,我们將移動到另一個也是極其常見的生殖器疾病:細菌性陰道炎,或簡稱BV(Bacterial vaginosis,BV)。妳聽过有人用和魚相關的詞⸺蝦子嘉年華或墨西哥魚捲,來描述女性生殖器吗?事实上健康的生殖器不应该有腥味,但BV就是這種味道的罪魁禍首。

BV是由生殖器正常菌群不平衡所引起。有保護力的乳酸菌減少,而環境中引起麻煩的其他類型細菌卻蓬勃發展。乳酸菌維持妳陰道環境呈現酸性,而酸性对陰道来说是好的物質。當妳得到BV時,妳陰道的酸性略為降低,變得偏向鹼性。這就是為什麼當妳有生殖器問題,酸鹼值是医師用來判斷妳是否得到BV感染的項目之一。

沒有任何一種類型的細菌与BV有關:它就像是不同種類參雜而成的雞尾酒。有些細菌通常以正常菌群生活在陰道或身体的其他部位。問題在於它们已經轉移,或是數量太多。

大多數專家認為,只有發生性行為的女性才會得到BV,而且風險隨著性伴侶數增加,以及減少使用保險套的次數上升。女性之間或与男性發生性行為也一樣。伴侶越多,得到BV的機率也就越高。所以妳可能會覺得有些細菌是來自妳的性伴侶,但是,這不代表BV是一種性傳染疾病。請記住,BV是由許多不同的細菌所引起。它不是一種如衣原体具有傳染性与殺傷力的細菌疾病。將它看作妳的正常菌群与好幾個擁有与妳稍微不同細菌組合的人混合在一起。廚師太多燒坏一鍋湯,在這種情況下,則是破坏了平衡。

沒有多位性伴侶的女性也會得到BV,但她们一定有过性行為。BV被認為是无害的,所以沒有理由在妳治療期間使用保險套或放棄性行為來保護伴侶免於受到感染。即使妳有多位伴侶,使用保險套永遠是明智的選擇,這是因為性傳播感染的風險,而非BV的風險。

除了特殊氣味被描述為腐爛的魚之外,得到BV的女性分泌物比平常還大量。許多人將分泌物形容為灰色、極為液態,每天需要經常更換自己的內褲數次。氣味會如此強烈到光從衣物就能聞到。

很多女性在陰道性交或生理期後會產生魚腥味或是味道更加惡化。這是否代表月經和性行為讓妳得到BV呢?不是這樣的,如果妳有BV,月經和精子只會加重症狀。

事实上氣味變得更加強烈是因為妳的生殖器偏向鹼性的環境。這表示如果妳陰道的乳酸菌較少或鹼性物質增加,妳的陰道會變得更糟。血液和精液在陰道環境中會更偏向鹼性,因此會增加腥味。如果妳在月經或性交後聞到魚腥味,可能代表妳的陰道環境只是不再偏向酸性,而BV症狀也沒有那麼嚴重。

也許聽起來很容易辨認,但得到鵝口瘡後,妳不一定可以藉由症狀來判斷是否得到BV。得到BV的女性經常會有發癢或其他可能讓她们聯想到鵝口瘡的症狀。分泌物是不同性傳播感染的常見症狀,並記住一次可能會同時得到好幾項症狀!分辨生殖器疾病的不同一直都很困難。由此得到啟發,如果妳的生殖器不正常的话,妳必須去看医生做檢查。分泌物、瘙癢或刺痛感讓妳很不舒服吗?去看医生吧。

雖然很多人在聞到氣味時會這麼認為,但細菌性陰道炎並不代表妳的生殖器很髒。如果藉由清洗來擺脫問題,妳只會洗去維持陰道酸性環境的良好細菌,讓事情變得更糟。BV可以自行消失,但進行治療會更好。因為BV是由所謂的細菌、抗生素或抗菌治療引起。含有乳酸菌的陰道膠囊对BV也同樣有用,並維持陰道的環境。不幸的是,沒有研究能證明這種療法有任何影響效果。





排尿疼痛




尿道感染被稱為排尿鐵絲网並非巧合。尿道感染的感覺很糟,而身為一名女性,妳特別容易得到它们。

我们的尿道偏短難辭其咎。事实上,我们的肛門距離我们尿道口很近。如果細菌留在屬於自己的地方,那麼它们會在我们的屁股上發揮最好的效果,但是实在很難限制細菌的行動。它们很容易爬進尿道口,經过尿道,最後定居在尿道和膀胱的黏膜。一到那裡,它们會引起發炎。

妳會因為尿尿疼而發現尿道感染。會有蜇傷、燒傷感,而且感覺好像從鐵絲网爬出來一樣。尤其在排尿的最後,膀胱本身排空和肌肉壁互相擠壓時特別疼痛。此外,妳會發現有頻繁的排尿衝動,但一次只能排出一點點。妳可能會注意到妳的尿液氣味奇怪,或者帶有一點血。

尿道感染的絕大多數年輕女性(多達95%)是我们所說的无併發症族群。代表感染被認為較不危險,只需簡單或根本不用治療。早期,因為人们認為感染會透过泌尿系統爬上腎臟,引起腎盂炎(pyelitis),所有的尿道感染使用抗生素治療,但結果它卻是不必要的方式。大多數尿道感染在沒有使用抗生素的情況下,幾天後會自行消失,並在必要時服用一些止痛藥。

當然,妳应该總是警惕任何的惡化現象。如果妳有發燒或更嚴重的疼痛,特別是疼痛往妳的背部移動,妳必須盡快去看医生,如果有需要則採取緊急手術。這可能是細菌引起的腎盂炎,在最坏的情況下可能導致腎臟受損。

如果妳懷孕的话必須認真看待尿道感染。在這種情況下會被認為是複雜的状况,同時妳需要特殊的抗生素治療。如果妳有頻繁的尿道感染也會被認為是麻煩的情況。通常需要更仔細調查其中出現什麼樣的細菌,有時會檢查是否有讓妳容易感染的潛在症狀。话雖如此,不知道為什麼,有些女性仍一次又一次得到尿道感染。根據推測,這些女性尿道黏膜的免疫系統可能有點不同,讓細菌更容易站穩腳步。

許多女性設法避免得到尿道感染。蔓越莓果汁或藥丸已經是使用好幾百年的常見民間偏方。蔓越莓包含防止細菌依附在膀胱黏膜的物質。然而,著名的考科藍資料庫(Cochrane Library)的一個主要研究顯示,蔓越莓病沒有保護作用。但是,再次強調:如果妳喜歡蔓越梅果汁,沒有什麼能夠阻止妳嘗試。蔓越莓汁沒有副作用。其他建議則是飲用大量的水沖洗尿道,並在需要排尿時盡快清空妳的膀胱,在排便後當然保持從前面往後面擦拭的習慣。

我们所知道的是,性會增加感染尿道感染的機率。在性交过程中大量水分往往在生殖器中累積,細菌變得更容易移動到另一個地方,在同一時間所有生殖器之間的摩擦和推力可以把細菌帶進错误的洞口。我们知道三十歲以下的族群性交後頭兩天得到尿道感染的風險比平常還多出六十倍。

妳可能聽說过的一個時下建議則是,如果做愛後排尿妳可以降低排尿刺痛的機會。這是很棒的建議。做愛後排尿,妳會沖洗掉自行向上進入尿道的腸道細菌,擺脫它们設法侵入妳的黏膜因而導致麻煩。

儘管性可能參与其中,一般的尿道感染並不是性傳播感染的一種⸺只是正常臀部細菌出現在错误的地方而已。但是衣原体、淋病和黴漿菌也是排便時刺痛的常見原因。所以,妳应该提高警覺。然而,這些細菌的表現略有不同。性病的細菌會在尿道口衍生,不像臀部細菌位於膀胱。當妳有性傳播感染時,妳排尿的最後不會有特別的疼痛。也不會有常見又頻繁排尿衝動。即便如此,自己也不太容易發現其中的差別。尿道感染和衣原体相似,而衣原体也可以像尿道感染一樣。如果妳真的不走運,妳可能會一次同時獲得這兩種感染。





滴滴滴⸺關於漏尿




當妳十九歲半也沒有子女,在商店購買大量的添寧護妳墊時一點也不有趣,然而老太太和擁有一堆孩子的女性並非遭受漏尿所苦的唯一族群。漏尿的專門術語為尿失禁(urinary incontinence),是女性中普遍存在的問題。

老化、生產,連同高BMI,是漏尿的最高危險因子,這表示有愈來愈多的女性开始隨著歲月流逝遭受這些困擾。這可能也是許多人認為生產前漏尿是不正常的原因,但是所有年齡的女性都可能會受到影響。

很難得知有多少女性确实有漏尿的困擾。每項研究的數據各不相同,但都認為只有不到一半的尿失禁女性會去医院求診,其中也可能有未統計資料的數字。挪威的一項研究發現,30%的女性有漏尿的現象,而另一項針对產後三個月女性的研究則發現20%〜30%的人受到影響。國外的一些研究顯示比率從10%〜60%不等,取決於漏尿的嚴重程度。

我们对年輕、沒有孩子的女性与确实存在的極大數字差距所知甚少。一項針对十六至三十歲沒有子女的澳洲女性的研究結果發現多達12.6%的女性經歷过漏尿。而瑞典的研究卻有完全不同的結果:二十至二十九歲的女性約有3%擁有漏尿的情況。

无论哪項研究的最接近事实,我们可以有把握地說,漏尿在年輕、沒有子女的女性中很少見。尿失禁有好幾種方式,我们區分為所謂應力性尿失禁(stress incontinence)、急迫性尿失禁(urge incontinence)与結合前面兩種的混合型尿失禁。

應力性尿失禁是最常見的,大約影響50%的漏尿女性。應力性尿失禁是引起上腹部壓力增加的漏尿現象,例如,咳嗽或打噴嚏、大笑、跳躍、奔跑或其他相似行為。与急迫性尿失禁相比,涉及的範疇雖小,但嚴重程度有很大差別。漏尿的頻率与漏尿量之間的差異可能有所不同。

急迫性尿失禁和需求有關。有這種尿失禁形式的女性會有突然、強烈的感覺需要馬上「排尿」,接著伴隨大漏尿的現象。擁有尿失禁的女性有10%〜15%會是這種形式。急迫性尿失禁的女性通常具有膀胱过動症(overactive bladder),這代表它们通常具有強烈的排尿慾望,卻不一定會漏尿。有膀胱过動症的女性通常比其他女性更常在半夜起床排尿。

介於35%〜50%的女性有混合型尿失禁的模式,也就是同時具有壓力性和急迫性尿失禁。換句话說,漏尿的形式會有所變化。有時候,妳跳躍或打噴嚏時就會漏尿,或在其他時候,妳會有強大的排尿衝動和大量漏尿。

漏尿可以由許多原因引起。如果妳喝的水比妳所需的更多,那麼減少飲水量可能會是一個好主意。很多人認為喝大量的水本身是健康的,但除非妳運動量很大或住在非常炎熱的氣候,妳不需要在每24小時飲用超过2公升的水。妳可以從食物獲得一些水分。通常沒有必要每24小時喝超过1.5〜2到兩公升的水。減少飲用利尿的飲料,例如咖啡和茶,也會是個好主意。

有時漏尿可能是其他疾病的症狀。有些女性在尿道感染和一些神經系統疾病時可能會引起漏尿。因此,如果妳找不到任何明确的原因,例如,妳生完孩子或者突然开始每天飲用5公升的水後开始漏尿,与妳的医生討论可能是明智選擇。妳的医生可以給妳指導和幫助妳找到解決方式。

妳會漏尿並不代表妳的下半身注定要穿黑色衣物來盡可能隱藏漏尿,或者在剩下的人生中放棄奔跑与歡笑。幸運的是,妳可以有所作為。試著停止漏尿的第一件事需要一點積極性。很多應力性尿失禁的人會有這種漏尿情況是因為她们的骨盆底肌肉太弱⸺舉例而言,她们可能在出生後受到影響。骨盆底肌肉是妳排尿時用來阻擋尿流或是緊縮陰道。如果妳的骨盆底肌肉較為強壯的话在妳的腹部壓力增加時更可以容易防止无意识的漏尿。有幾種方法可以訓練妳的骨盆底肌肉,但主要与間隔收縮妳的陰部肌肉有關,和妳在健身房鍛煉身体其他肌肉一樣。許多女性從她们的家庭医生或物理治療師得到幫助。妳可以遵循一些特殊的運動項目,包括專為訓練骨盆底肌肉設計的應用程式。妳也可以嘗試陰道球或類似的工具。陰道球的重點在於藉由使用骨盆底肌肉,來維持它的位置。不管妳如何訓練,妳會希望看到自己漸漸變得更強壯並減少漏尿的現象。

骨盆底鍛煉可能对受到急迫性尿失禁影響的女性有一定的影響,但膀胱訓練的过程為更重要。对於那些有急迫性尿失禁的人来说,問題並不在於肌肉。膀胱肌肉在妳沒有任何控制的情況下於错误的時間收縮。這就是為什麼人们有急迫性尿失禁時會經常大量排尿。膀胱訓練是訓練自己不經常排尿。問題的關鍵是按照時間計劃表,而不是根據需求排尿。舉例来说,妳可以從限制排尿开始,每隔一小時都這麼做。如果在不能排尿的時間裡有突然的衝動,妳必須忍住不能去廁所。过了一陣子,妳可以逐漸增加允許排尿的時間至兩小時、三小時、四小時等。隨著時間的推移,這將往往对急迫性尿失禁有幫助。

在某些情況下藥物治療或手術可用於治療尿失禁。一些女性只要在門診進行簡單的治療就能有不一樣的世界,而对於其他人来说運動會有效果。幫助妳最多的事情取決於自己想要的是什麼以及漏尿的問題有多嚴重。





痔瘡及肛門皮膚贅瘤




如果妳稍微看一眼妳的屁股,妳很快就會發現它有好多皺摺。人们常常把它叫做氣球節有一個原因。皺紋是由夾緊洞口的括約肌構成。它必須能夠擴張得非常大,而它的大直徑被有點像是百褶裙的結構給隱藏住。一般情況下,皺摺均勻圍繞洞口,形成一個相对平坦的表面。因此它可以在妳突然發現屁股外面掛著新的和外來的東西時使妳心驚膽顫。妳感覺新的突起物好像在尖叫引起注意,而大家的目光往妳的肛門看去,很多女性試著完全忘記這回事。這種感覺可能是肛門皮膚贅瘤(anal skin tag)或是痔瘡(haemorrhoid)造成,而兩種都是无害的症狀。

痔瘡对男女性来说皆是極為常見的疾病。雖然沒有足夠的理由能夠當作一般晚餐话題,然而实際上大約有三分之一的成年人得到痔瘡。痔瘡同時長在直腸或直腸外的肛門周圍是可能發生的;不过我们談论肛門外部的痔瘡就好。无论如何,痔瘡就只是痔瘡。

痔瘡是由於肛門靜脈曲張,在外觀上呈現氣球形狀的紫藍色突起物。不像肛門皮膚贅瘤,妳幾乎能再次把它推回原位,但隨後會在妳排便或是做了特別費力的深蹲時再次彈出。痔瘡經常會有搔癢感也可能一触即痛。有時唯一的問題可能是當妳擦後面的時妳會發現衛生紙上有鮮血。造成的原因單純是痔瘡的血管錯位。通常圍繞直腸口的血管由結締組織和黏膜的支撐,所以我们看不到它们的存在。隨著年齡的增長,這些支撐結構變得鬆弛而骨盆受到的壓力增加(例如如廁用力、提重物、懷孕和生產)可能引起一小部分的血管被推出來,就像一個打結的澆花水管一樣。這個扭結的根部很容易承受壓力,引起血液累積並形成一顆小氣球。這氣球就是我们所說的痔瘡。

圍繞在屁股的痔瘡不危險,但卻也能造成真正的麻煩。血管不喜歡被這種方式对待,因此輕微發炎就很容易發生在痔瘡周遭。接著妳會發現有點黏液,或者感到疼痛、發癢,讓單純的坐下(更不用說排便)變成煩人的一件事。有些人也會發現自己有出血的現象无论流量很少或相當多。

幸運的是,解決方法很簡單。最重要的事,其实很平凡,就是确保自己擁有良好的如廁習慣。喝足夠的水保持妳糞便柔軟,當妳感到廁急時去上洗手間,還有避免緊張。我们還會建議妳把報紙留在廚房的桌子上。如果妳坐在廁所很長一段時間,圍繞痔瘡的壓力會增加,這會讓問題更嚴重。良好的如廁習慣往往會讓所有痔瘡自己回到原位。用手指將彈出的痔瘡推回原位也是好方法,可能會有機會找回了正确的位置。用妳的手指戳妳的屁股可能會讓人覺得有點怪,如果這麼說能帶來安慰,医生每星期都要对完全的陌生人這麼做。

妳也可以在藥店買到各種效果良好的痔瘡軟膏。如果沒有效的话,妳的医生也可以提供很多不錯的治療方案來幫助妳,包括手術在內。正如妳現在可能明白,妳的医生对這些事早就習慣了!

如果突出妳屁股的不是痔瘡,那麼有可能是肛門皮膚贅瘤。它是存在於肛門的一個略大的皮膚皺摺,通常是由破掉的痔瘡所產生。當痔瘡自行壓迫時,可能會在肛門環造成一些皮膚皺摺從原本的地方脫落。後來,當痔瘡消退時,它们會結合而形成稍大的皺摺從表面略微突起。雖然妳可能因為皮膚褶皺受到丁字褲摩擦、頻繁排泄或類似情況而刺激,造成短暫的瘙癢和分泌物,但一個或兩個肛門皮膚贅瘤很少會引發大問題。有些人甚至會發現難保持屁股乾淨。

然而,有些人覺得肛門皮膚贅瘤有礙觀瞻。贅瘤可以藉由手術移除,不过妳应该在選擇手術前謹慎思考,因為併發症的風險總是存在。同時也值得注意術後疼痛。妳會在妳的屁股中間得到一條疤痕,而不幸的是因為妳剛做完手術所以會憋不住糞便。除非它们造成的許多問題,不然我们的建議是放輕鬆並和肛門皮膚贅瘤和平共處。





子宮頸癌与如何避免




子宮頸是子宮与陰道之間的通道。妳能感覺到妳的子宮頸在妳陰道的最上部,像是鼻尖的塞子一樣,中間有個小小的洞。這條狹窄道路是精子細胞到達子宮所經过的通道。妳的月經從這裡離开,當妳生產時妳的子宮頸可以擴張到足以讓整個嬰儿通过。也正是這裡可以讓妳得到子宮頸癌。

子宮頸癌在癌症當中格外特別。早在十九世紀,人们發現這種類型的癌症在每個人身上的表現有所不同。妓女比已婚女性更常得到,而修女傳播這種疾病的機率很低。難道這是对放蕩女性的一種神聖的懲罰吗?

現在我们知道,上帝的懲罰和這個疾病沒有關係。子宮頸癌單純是藉由性行為傳播所造成的疾病!我们先前提过這個和性傳播感染有關的病毒:人類乳突病毒(HPV)。

HPV是一種大家族的病毒,其中幾個病毒會造成尖銳濕疣。它们大多數都是无害的,例如普通的皮膚疣是由一種類型所引起。某些HPV類型會在生殖器茁壯成長。它们是經由性接触傳播,而且大多數性活躍的人會在生命中感染其中一種。到了五十歲,有过病毒感染的人超过80%。HPV被認為是最常見的性病,过半數二十到二十四歲的人在任何時間帶著感染四處遊走。

无須擔心HVP,不像皰疹感染,妳的身体通常會自行擺脫病毒,運作的方式和感冒相同。我们知道這是因為确診HVP女性的病毒會隨著時間變換病毒類型。這表示感染是短暫的,而女性在更換性伴侶時會重新感染新病毒的類型。

然而,某些類型的HPV与其他病毒的不同之處在於,它们可能造成一些人的子宮頸長期感染。這些類型就是所謂的高危險病毒,最常見的類型是HPV16和18。如果妳運氣不好的话,感染隨著時間可能會發展成癌症。16號就超过子宮頸癌的病例一半,也可能會導致口腔和咽喉癌,以及陰道、陰戶和肛門癌。然而,不只會得到一種感染。感染HPV16很常見,但只會有極少數人得到癌症。這表示其他因素是決定癌症發展的關鍵,例如,容易受到感染的人,或是像是有吸菸習慣等環境因素。這些其他因素是什麼,我们也還不知道。

稍微有點不同,幾乎所有子宮頸癌的患者會得到HPV病毒造成的感染,但很少有人因為感染而罹患癌症。





從性到癌症的漫長之路




幸運的是癌症不會在一夜之間發展。首先,病毒會造成妳子宮頸細胞變化,在專業術語上稱作發育不良(dysplasia)。小缺陷和異常細胞會阻止正常細胞運作。這些患病細胞在一开始的時候都只是略有不同,但如果免疫系統讓它们遠離和平,它们才可能真正开始脫穎而出。細胞直到完全无法辨認前會隨著時間出現愈來愈多改變,並开始在不应该存在的地方生長。只有在這時才會成為癌細胞。

在大多數情況下從无害變化細胞到全面的子宮頸癌至少需要十至十五年的時間。在這些期間,它们會經过不同階段的變化。在每一個階段,細胞可能改變想法或被免疫系統破坏。

這些類型的細胞之變化,可能是癌前的病變階段,最好能儘早發現。藉由每三年定期掃描和細胞檢查,可以即時捕捉變化並在它们構成任何威脅之前去除。這是有效对抗子宮頸癌防禦方式。

當妳生病時細胞改變和子宮頸癌在末期前很少會有症狀或徵兆。這就是為什麼子宮頸的定期檢查非常重要。子宮頸癌的症狀可包括出血異常,如月經週期之間或与性行為有關聯的出血。有些女性的生殖器或下腹在性行為間或日常生活中會出現疼痛。其他人可能會發現分泌物开始發臭並帶有血跡。

換句话說,子宮頸癌的症狀非常不明确:它们存在於許多生殖器常見和危害較小的疾病當中。如果妳有這些症狀,妳絕对要去找妳的医生進行檢查,但不需要擔心會是癌症。最有可能的原因則是性病、避孕的副作用或性交時導致的疼痛;最重要的還是要去做檢查。





做檢查




妳很快就要二十五歲了吗?那麼妳就會收到癌症登記中心的邀請,進行子宮頸抹片檢查19 。如果有收到,妳真的应该去做檢查,就是這樣。進行定期抹片檢查的女性一生中罹患子宮頸癌的風險可以減少70%。這就是我们所說的便宜得難以置信的人壽保險!

儘管二十五至三十四歲之間的挪威女性中,幾乎有一半的人選擇把信丟到垃圾桶,進行篩檢的比例已經從71%下降到57%。即使比以往更加暴露在子宮頸癌的風險,年輕人仍是不做檢查的族群之一。這樣造成了不良的後果,在挪威有更多的年輕女性比以前更容易得到子宮頸癌。根據癌症登記中心記載四十歲以下女性的癌症病例數在近幾年上升了30%。其原因為在更多年輕女性受到致癌的HPV感染時,仍很少有女性轉而定期進行子宮頸檢查。

因此抹片檢查是預防子宮頸癌的簡單解決方案。挪威女性在二十五歲時會納入檢查名單,接著會建議她们直到六十九歲应该要每三年做新的抹片檢查。妳在檢查後屆滿三年時會收到預約新檢查的提醒通知。

抹片檢查需要与家庭医生預約。即使還沒有收到邀請妳也能去做。在婦產科医生那邊也可以做抹片檢查,在這種情況下,妳通常會需要得到医生的轉診許可。

妳不应该在月經期間進行抹片檢查,同時也最好不要在檢查前兩天進行陰道性交。進行婦科檢查只需要幾分鐘的時間。医生會以一種漏斗狀的擴張器撐开妳的陰道,窺視妳的子宮頸,並以小刷子進行採樣。刷子會靠在子宮頸上輕輕摩擦並鬆動一些細胞,接著可以在实驗室的顯微鏡下檢查。如果子宮頸細胞顯示變化,妳會在幾週內從妳的医生那邊獲得通知。如果一切正常,妳一般不會收到任何消息。





細胞變化並不代表妳有癌症




抹片檢查後,妳可能會從妳的医生那邊得到討厭、難以理解的檢查報告信函:妳的細胞不正常⸺但這到底是什麼意思呢 ?

我们所遇到的女性當中最常重複的话題是她们对医生傳達子宮頸周遭細胞變化的过程不充分而感到沮喪和焦慮。大多發現有異常細胞的年輕女性都感到自己非常健康,從來沒有想过自己會得到癌症。所以檢查報告帶來的衝擊可能比医療專業人員知道的還更多。

許多女性在被告知發現細胞變化後會認為自己得到癌症,且有可能死去而感到害怕。我们要对這些女性強調,年輕与性活躍的女性子宮頸細胞發生些微變化非常常見。任何HPV感染,即使是低風險的病毒,也可能會導致變化。這就是為什麼二十五歲以下的女性不用檢查⸺令人難以置信的數字將會成為不必要的焦慮,最終可能會在沒有增進我们發現新癌症案例的能力下受到过度治療。

在絕大多數情況下,子宮頸細胞變化在沒有任何形式的治療下會自行消失。像其他的病毒一樣,往往自己消失。人体自身的免疫系統會自己整理一切,实在是太棒了!妳的医生知道這一點,因此這可以解釋為什麼在妳想到的結果全是「癌症」時她會看起來一派從容。

讓妳得到多一點安慰:每年有兩萬五千名挪威女性在抹片檢查發現細胞異常,只有三千人需要接受嚴重癌前治療。後來發展成子宮頸癌的甚至更少,大約三百人而已。

但是,讓我们來看看從妳医生得到的檢查結果信函。妳在抹片檢查後發生了什麼事?從妳的子宮頸刷下的細胞被送到实驗室。在那裡,医生將細胞染色並將它们放在顯微鏡下,寻找出現異常的細胞。根據細胞外觀不寻常的程度和數量多寡,將細胞變化以輕度、中度和重度做分類。即使是嚴重的細胞變化也可以自行消失,但追蹤所有細胞的變化仍然很重要。

觀察細胞之外,实驗室會將切片拿去進行HPV檢查。當涉及下列發生的情況時,細胞變化的規模,与HPV檢查的結果會是決定性的因素:





細胞切片呈現不确定或低度變化




妳只需要在六個月後回到医生那邊做追蹤檢查。異常的細胞往往在病毒攻擊後修復,或被妳的免疫系統殺害。如果細胞的變化已經消失而HPV呈現陰性,妳就跟以前一樣健康,三年後妳才需要做另一次抹片檢查。如果經过追蹤檢查,而HPV呈現陽性後妳的細胞仍然有變化,妳會被送到婦產科進行更深入的檢查,如下面所述。





細胞切片呈現高度或嚴重變化




妳的家庭医生會替妳轉診至婦科医生,她將進行兩件事情。首先,當妳坐在診療椅上時,她會先用特殊放大鏡工具來看看妳的子宮頸。這種檢查稱為陰道鏡檢查(colposcopy),是為了寻找黏膜的變化。接著医生將會從妳的子宮頸採集組織樣本(活体組織切片,biopsy)讓專家(病理学家)透过顯微鏡進行檢查。在細胞檢查中只有少數細胞會從黏膜表面上被刷掉,但活体組織切片裡會移除一小片皮膚去調查黏膜深處是否有異常細胞。整個黏膜架構都會進行檢查。

通常妳的子宮頸會被施打局部麻醉,因此在活体組織切片結束後可能會感到疼痛。在檢查前服用一些布洛芬可能是一個好主意。同時檢查过程中流血也是正常情況,所以大多數女性在這幾天需要使用衛生棉(不是衛生棉條!)。

當病理学家從顯微鏡檢查組織切片時,這些變化將再次根據輕度、中度到重度變化分類。這些都不是癌症。只有在不正常細胞直接自行穿过黏膜才算是子宮頸癌。

如果陰道鏡檢查及組織切片顯示完全正常或僅有輕微的變化,那麼妳可以鬆口氣。但是,妳必須在六至十二個月內去找妳的家庭医生進行新的抹片和HPV檢查來确保一切正常。在十分之九的案例中,這些變化將會消失或在沒有任何形式的惡化下維持穩定20 。

如果任何的檢查确定有中度至重度的癌前病變階段,妳將按照規定,會被送至医院接受小手術。

這個过程被稱為錐形活体組織切除(cone biopsy)。通常會以線圈電刀或吊帶切除部分子宮頸的外部。一开始医生會用刀進行錐形活体組織切除,而被切除的部分看上去像一個倒置的冰淇淋甜筒,這也解釋了這個手術的名稱。現在被去除的組織部分看起來更像一個扁平甜甜圈。

雖然有少數人可以進行全身麻醉,但錐形活体組織切除通常是在局部麻醉下進行。這是一個簡單的过程,但除非有必要不然可以不用這樣做。這是因為經歷錐形活体組織切除的女性在往後懷孕被認為有較高的早產或流產風險。

絕大多數有过錐形活体組織切除的女性,約90%會痊癒。為了可以100%肯定自己沒事,她们在六個月或在手術後十二至十八個月到家庭医科進行新的抹片和HPV檢查來确保一切无事。

如果細胞的變化已經自行消失或藉由錐形活体組織切除移除,就不必再擔心子宮頸癌。就像蛇梯棋一樣:妳會直接回到原點!

不过需要特別記住的是,妳可能會再次感染HPV病毒,所以接種HPV疫苗可能是明智的選擇。我们待會再說。妳還必須繼續在餘生每三年進行抹片檢查。然而總体来说,妳应该專注在放鬆身心。





对抗癌症的疫苗




現在,我们談了很多關於妳应该如何与HPV感染和細胞變化共處,但是想像一下,妳是否可以事先預防致癌病毒的感染!事实上,是可以的,完全有可能。幾年前可能會像科幻小說一樣,但現在居然還有能預防癌症的疫苗。這是個医学奇蹟。

正如我们前面所解釋的,人類乳突病毒(HPV)有超过一百個不同類型,但能夠致癌的只有其中幾個。加衛苗(Gardasil)和保蓓(Cervarix)這兩種HPV疫苗可以抵抗最危險的HPV16和18類型感染。在它们之間,這兩個高危病毒引起70%的子宮頸癌。針对這些病毒的疫苗可以讓妳幾乎100%抵抗感染,以及細胞變化与這些種類病毒導致的子宮頸癌的防護。近期一種可以防止九種不同病毒類型的HPV疫苗受到核准上市。它可以防止90%的子宮頸癌,但挪威健保局尚未給付這支疫苗。

加衛苗也可以預防造成尖銳濕疣的HPV6和HPV11類型的感染。一些研究表明保蓓也能提供防止尖銳濕疣的部分防護。重要的是要明白尖銳濕疣与生殖器癌之間沒有任何的關係,但是能夠避免尖銳濕疣仍然是好事。如果沒有疫苗,10%的挪威人口會得到尖銳濕疣,約10%的挪威女性將有嚴重的細胞變化,而1%的女性會在剩下的日子裡接受子宮頸癌治療。

自二〇〇九年以來,HPV疫苗已經變成挪威儿童接種疫苗計畫的一部分。換句话說,提供所有十一至十二歲的女孩免費接種。此外,出生於一九九一年或以後未接受疫苗的女孩從二〇一六年十一月可以免費接種疫苗。

疫苗分為三劑並為期半年施打。

疫苗並非藥物,但卻能防止病毒在妳体內定居,避免妳在之後得到感染生病。疫苗刺激免疫系統辨识病毒,在病毒出現時準備最快速、最有效的粉碎病毒作戰計畫。如果妳已經得到HPV16或18的感染,疫苗不會消除妳身体中的病毒。這就是為什麼要讓年輕女孩施打疫苗。我们要在她们开始進行性行為或得到潛在病毒感染前保護她们。

疫苗被核准用於九至二十六歲的女性和男性,並已證明有效年齡範圍達到四十五歲。其中有兩個原因。首先,我们當中很少有人同時感染HPV16和18,如果妳還沒有被這些類型的病毒感染,疫苗將有一定的保護作用。其次,如同我们提过,多數HPV感染會自行消失。不幸的是,对抗HPV的自然免疫已被發現效力薄弱。這代表即使妳較早得到HPV感染,後續不一定能從其他性伴侶再次感染的風險中得到防護。但HPV疫苗可以幫助妳抵抗這種再次感染。

現在男性已經不在疫苗接種計畫當中,但挪威公共衛生协會(Norway’s Public Health Institute)則建議女孩和男孩均須接種HPV疫苗。疫苗对男人也同樣有效(对尖銳濕疣、陰莖、肛門和咽喉、口腔的HPV相關腫瘤)对女性也是如此。有些人可能已經注意到,男性罹患咽喉和口腔癌的風險增加了。據猜測是因為口交變得愈來愈普遍。疫苗也可以預防這類的感染和癌症生成。尤其同性戀男性將會受惠於疫苗,因為他们不像女性有通常稱作群体免疫(herd immunity)的直接疫苗保護。

不幸的是,出生於一九九一年前的男性与女性必須自費施打,但是妳应该考慮從口袋掏錢支付疫苗費用吗?

对於妳擁有的每個性伴侶,HPV感染的風險是10%左右。即使妳已經感染了一種或多種類型,妳非常可能不會被HPV16或18感染。如果妳接種疫苗,妳能防止自己被未來新性伴侶感染。正如我们提到的,研究表示疫苗的有效年齡可以達到四十五歲。既然妳必須為此從自己口袋裡掏錢出來支付費用超过三千克朗的疫苗,妳应该与得到感染的風險相互衡量。簡單地說,這表示妳有过的性伴侶數量很重要。先前的伴侶數愈少,疫苗对妳有效的可能性也就更大。妳有过的性伴侶總數在將來也會有一定的作用。伴侶數愈多,感染的潛在風險就更大,疫苗的好處同樣也就更多。此外,接受異常細胞治療的女性因為施打过HPV疫苗,復發的風險也會很低。





安全有效的疫苗




現在挪威七年級的女孩有四分之一選擇不接受HPV疫苗。我们不知道為什麼人们要選擇不注射疫苗,而害怕副作用似乎是普遍的現象。也有一些家長認為自己十二歲的女儿不會發生性行為,所以HPV疫苗是不必要的。在丹麥有許多媒体非常關注可能出現的副作用,而這也導致了接種女生的比例大幅度降低。同時最近挪威媒体上有一些關於疫苗恐怖故事。其中有一點原因在內。

在挪威有16萬名女孩已獲得近50萬劑疫苗。其中有645件可能副作用的通報案例,92%的人認為不是很嚴重。這只是短暫的症狀,例如接種疫苗部位腫脹、發燒、噁心及腹瀉 。

自二〇〇九年以來,在52個通報案例當中有嚴重副作用的有10例為慢性疲勞症候群(chronic fatigue syndrome,CFS/myalgic encephalomyelitis,ME),而姿勢性直立心搏过速症候群(postural orthostatic tachycardia syndrome,POTS)則有5例。POTS是當妳站起來時會導致脈搏數升高的状况,以及血壓不穩定、疲勞与頭暈。根據挪威藥品局報告,在這個年齡組中有或沒有接種疫苗的病例數不高於人们所預期的狀態。換句话說,疫苗不被認為是造成這些問題的因素。

然而這類關於可能嚴重副作用的回報一直被嚴重看待。在接種疫苗後,許多包括POTS的症狀案例在丹麥頻繁出現,歐洲藥物管理局決定進行安全審查。調查的結果在二〇一五年十一月出爐,結论是,沒有數據證明HPV疫苗与POTS或稱為CRPS的另一個症候群(複雜區域疼痛症候群,complex regional pain syndrome)之間有因果關係。這些是罕見的情況,且接種疫苗的女性其發生率也不會高於其他人口。也沒有發現疫苗和慢性疲勞症候群之間有任何的關聯。

到目前為止世界各地約有180多萬名女性接種HPV疫苗,施打後並沒有發生嚴重的安全問題。使用藥物和疫苗總是會有副作用的可能性,但是這些往往是輕度、暫時性的問題,然而生殖器癌卻並非如此。





* * *



19	有時試驗計畫會在挪威的特定城市進行,會以HPV檢查取代抹片檢查。感染HPV 16或18的女性後來被招回進行抹片檢查。未來最快幾年內,這可能會成為挪威实施的篩檢方式。這代表挪威女性可以免除不必要的婦科檢驗与抹片檢查,只有異常細胞風險增加的人才要接受檢查。



20	60%的些微變化會自發性消失,而30%會維持不變。只有10%會發展為嚴重改變,而1%則是發展為癌症。





流產⸺從Facebook回到現实




二〇一五年夏天,Facebook創辦人,馬克·祖克柏向他的3300萬好友更新一則極為特別的貼文。他和他的医生妻子宣布,他们正高興地期待他们的第一個孩子,是個女孩,並為了她準備讓世界變得更美好。无聊到要打哈欠了,妳可能會這麼認為,接著自動按讚。這些各種各樣的個人公告是Facebook賴以為生的東西,這個地方已經變成謙虛吹牛和製造各式各樣形象的代名詞。

然而祖克柏並沒有就此罷手。他選擇向他的追蹤者介紹懷孕的崎嶇歷程,最後幸福的結局留下160萬個讚⸺是一個我们平常不會提到的故事。這对夫妻經歷三次流產与數年的努力,終於成為父母。四次的懷孕才得到一個孩子。

流產是懷孕二十週以前受精卵停止生長或胎儿在子宮中死亡的非自願終止孕期現象。妳通常會因為疼痛及陰道出血才意识到妳流產了。也就是說,懷孕期間出血有沒有什麼不寻常的地方。雖然只有十分之一的機率會發生流產,但大約四分之一的孕婦在前三個月有出血的情況。即便如此,如果妳在懷孕期間出血,妳应该聯繫妳的医生並作檢查。

流產是早期懷孕最常見的併發症。女性本身意识到懷孕的臨床性懷孕(clinical pregnancy)流產的機率約為五分之一。同時也有在驗孕測試前就發生流產的情況。這些類型的懷孕通常被稱為化学性懷孕(chemical pregnancy)。將化学性懷孕納入考量,會認定只有一半的受精卵能造成可行的懷孕。換句话說流產和成功懷孕一樣普遍。

現在的驗孕測試極為敏銳,能夠在非常早期檢測到妳懷孕了,但如果妳非常渴望陽性反應結果,使用這個方式不一定是非常明智選擇。因為大多數的流產發生的時間從受精後前幾個星期,到妳的下一個月經來臨前。由於早期懷孕以流產結束是如此的普遍,妳可以省點讓自己失望的時間,等到妳下次月經預計到來時再做驗孕。如果多等個兩週,直到懷孕的第六週,流產的風險已經下降到10%〜15%。因此,在這個時間點得到陽性的結果意味著妳可能會在八個月後當媽媽了。八週过後,風險下降到只剩3%。一旦三個月的標準过去,風險穩定在0.6%左右的低水平當中。每过一週,一切都會平安的機率將愈來愈高。

流產的恐懼是孕婦往往選擇等到三個月过後才會告訴大家的原因。這個祕密背後的想法主要是避免孕婦出了差錯。失去渴望已久的孩子已經夠糟,根本不需要向周遭親友打電话收回這個好消息。三個月的限制是否明确值得商榷,如果妳必須要有限制的话,妳也可以將門檻往前設置一個月前,大約是第八週。

不幸的是這種保密的結果會讓很多夫妻对於流產感到可恥。經常聽到人们对於流產後評论則是:「嗯,太快告訴大家好像有點奇怪」,彷彿只因為公开就殺了妳肚子裡的胎儿一樣。這是相當荒謬的想法。祖克伯將流產描述為一種孤獨的經歷:「大多數人不討论流產,因為妳擔心妳的問題造成自己的距離或是讓妳檢討自己⸺就好像妳有缺陷或者做了什麼事才導致如此。所以,妳會一個人獨自掙扎。」

祖克柏所描述的心情並非只有他一人感受到。發表在美國研究期刊《婦產科学雜誌》(Obstetrics & Gynecology)當中,近一半流產的人表示,她们有遭受指責,或自己做錯事的感覺。她们感到孤獨与羞愧。讀起來令人格外難过,尤其是因為流產的普遍誤解所造成自責。在相同的美國研究裡發現,近四分之一的人認為生活方式的選擇,如吸菸、酗酒和毒品,是常見的流產原因。許多人還認為,繁重的工作和壓力會導致流產。网絡上的媽媽和寶寶论壇,將飲用咖啡和洗泡泡浴列為其他可能的原因。

在現实中,流產很少是母親(或父親)失誤的結果。流產最常見的原因是胎儿的染色体嚴重異常,也就是已經确定受孕的遺傳密碼當中發生错误。在妳知道懷孕前,忘記酗酒、不健康的飲食或吸菸吧。

母親和父親的遺傳物質只會為一個人,按照順序合併成一本共同的食譜,複雜的讓人難以想像。错误沒有原因,不斷發生也不足為奇。流產是身体的控制機制並以自己的方式确保我们擁有能夠好好生活的健康孩子。若以這種形式流產可能會感到極具痛苦,但妳的身体实際上在做对的事情。

只有當妳已經連續發生兩次或三次時,才应该考慮調查母親(或父親)身上是否有什麼状况導致流產。在此之前都會被當作是一個很正常的現象發生。女性經歷反覆流產,問題從解剖畸變和賀爾蒙失調到自体免疫疾病和遺傳性血液疾病都有可能。沒有人可以指責這些疾病,但卻有望可以治療。

單純運氣不好是流產最常見的原因,但我们知道有幾件事情會增加流產風險。最重要的因素是母親的年齡,丹麥的一項研究發現,三十五至三十九歲的孕婦有25%最後流產,相較之下,二十五至二十九歲的人則是12%。四十歲當中只有一半的人能夠順利生產,因為卵子的質量开始變差,造成染色体和基因發生错误,使胎儿不能存活的情況更加頻繁。

我们都知道懷孕不能吸菸。妳应该在知道懷孕後儘快戒菸。但是在妳發現之前该怎麼辦呢?在還不知道的時候,妳在一個聚會上抽菸了,又该如何是好?大型的研究調查發現吸菸与流產之間有明确的關聯。如果100名非吸菸者和100名吸菸者懷孕後,非吸菸組會有20人流產,而吸菸組大約為26名21 。根據統計,每十件流產案例約有一件是由吸菸引起,但抽菸量看起來好像要很大(每天超过十支)才會提高明顯的風險。所以在懷孕的前幾週有少量的社交吸菸不會是巨大的內疚或焦慮的理由。

一定程度的酒精量也是一樣。酒精对胎儿的危害極大,但我们不知道要達到多少才會造成傷害。孕婦可以在胎儿受到傷害或死亡前喝多少的量实在很難确認。要一組孕婦在懷孕期間喝酒,調查多少酒量能引起流產或胎儿的傷害當然很不人道。由於我们不知道極限,所以挪威衛生當局則建議完全避免酒精。這樣一來妳就可以能夠處於安全的狀態。

然而,並不是每個人都同意戒酒是唯一正确的方式,在妳懷孕時甚至會造成混亂。妮娜在懷孕時發現這個情況,許多医生告訴她偶爾一杯紅酒完全不會有問題。世界著名經濟学家艾蜜莉‧奧斯特(Emily Oster)厭倦了這些綜合訊息,並決定深入調查這項建議背後的研究經过。在她二〇一三年出版的書,《期待好孕:為什麼傳統懷孕的智慧是錯的,妳真正需要的是什麼?》(Expecting Better: Why the Conventional Pregnancy Wisdom Is Wrong and What You Really Need),她聲稱很少有人支持在孕期中絕对迴避酒精的官方建議。她的研究表示,每週喝一到兩單位的酒精,一週兩天飲用一小杯葡萄酒或一杯啤酒完全安全。這不會对孩子的行為或智力產生長期影響。在她看來,要完全戒酒的官方建議是对女性不能克制自己的臆測而來:如果妳在生日中喝了一杯葡萄酒,可能會迅速喝掉一瓶。我们同意奧斯特的看法,但這種看法低估了女性的自律;畢竟我们大多數人都能在九個月的孕期內克制不飲酒。如果妳好奇或者懷疑的话妳可以閱讀她的書籍,看看自己是否被說服了。

在妳驗孕得到陽性反應時,一杯晚餐的紅酒或許不會讓妳擔心。許多女性在發現懷孕時,因為在幾週前參加一兩場酒精派对,喝下超过一兩杯酒而感到緊張。丹麥從二〇一二年开始的人口基礎研究發現,如果女性在懷孕頭三個月每週飲酒四次或以上,流產的風險會增加一倍。所以理论上,在發現妳懷孕前幾週飲酒过度可能會導致流產,但絕不代表必然發生。如果确实如此,根本无法將原因歸咎於大量飲酒。流產可能隨時發生,只要知道它是多麼常見就好!

而現在這些謠言在网絡上比比皆是:搬重物、壓力和正常的咖啡量攝取不會導致流產。看來在可能造成風險前妳每天可以喝到十杯咖啡。越野滑雪冠軍瑪麗特·比約根(Marit Bjørgen)在懷孕期間一天訓練六小時,最後仍平安產下孩子。維他命或其他類似補充品看起來不能形成对流產的防護,但是在發現懷孕時妳应该开始補服用維生素B葉酸⸺而且最好是在嘗試懷孕的時候就服用。它可以防止孩子的神經系統受到損害。

馬克‧祖克伯一直是鼓勵人们在社交媒体上分享流產經歷的人之一。很多人認為這些經歷太过私人和有失体面因此拒絕在公开場合談论,但這名男子仍傳達了一項重要的訊息。公开流產的消息是為要讓大家清楚流產其实是很正常的現象,同時這項事实是會影響到所有的人。对於流產无須感到羞恥,它通常不是任何人的錯。而流產帶來的其中一項好消息是,絕大多數流產的人之後會有完全健康的孩子。

我们前面提到的三個月法則是為了保護女性免於承受告知別流產的痛苦,但這條法則实際上或許弊大於利。它延續了誤解和歧視,而非讓人有普遍的認知与理解。結果造成許多女性有更孤立、羞恥和內疚不合理感,在這個時候她们最需要來自周遭的溫暖和体貼。因此,請大家告訴大家!





* * *



21	在懷孕期間流產的相对風險吸菸者比非吸菸者多了1.32%。這個例子讓我们推论出非吸菸者流產的機率為20%。比例可能太高,但這是以大眾可以理解為目的選擇用相对風險來展示。





滴答作響的時鐘⸺妳還能推遲懷孕多久?




當妳接近三十歲,奇怪的是就連素昧平生的陌生人都覺得自己有權干涉妳的私人生活。「時鐘在轉動,親愛的!妳是不是该开始考慮生孩子的時間了?」不管妳是單身,處於新的關係或者和工作結婚,完全与他们无關。他们更想看到妳拋下妳所做的一切,強迫妳和能夠倚靠的男性迅速生儿育女。

是的,想想生孩子這件事。很多女性想像了,也同時想像沒有孩子的畫面。即使妳想生孩子(這絕对不是既定的想法)仍然存在大量的潛在障礙。最明顯的是找到一個妳可以想像能夠与妳生孩子,也準備好与妳生孩子的人。奇怪的是許多男性在聽到酒吧裡的可愛的女孩开始談论嬰儿車,在喝第二輪酒時眼裡浮現穩定下來的念頭後紛紛落荒而逃。

不幸的是,我们不能幫妳找到完美的父親,我们能做的就是給妳一點東西拿去对那些不斷催妳懷孕的好事者反擊,或者在妳开始感到壓力的话給妳一點安慰。因為三十歲往往被當作是一個神奇的門檻,不过事实並非如此。

讓我们從幾個事实开始說起。大約有75%的情侶在交往六個月後开始嘗試懷孕。這一年結束以前,介於85%〜90%的人得以懷孕。一年定期從事无防護性行為後沒有懷孕的現象會被定義為不孕症(Infertility)。這適用於大約10%〜15%的夫婦,但這並不是結束。被貼上不育的夫婦中,有一半會在第二年嘗試懷孕的过程中自然懷孕。他们事实上应该被稱為低生育率(sub-fertile)。他们很難懷孕卻嘗試非常長的時間,最後卻得以实現。所以高達95%的異性戀者會耗費很長一段時間藉由進行定期性行為來懷孕。

再來就是年齡的問題。隨著女性進入勞動力市場,生下第一胎的平均年齡逐步上升。二〇一四年奧斯陸女性生下第一胎的平均年齡為三〇‧八歲。女性因為長期求学並希望能夠建立自己的職涯,所以比以往想要等更久一點的時間再來懷孕。同時,医学界透过數據強調生育率急劇下降,並向我们發出警告,希望我们在推遲懷孕前再三思考。其中有幾個很好的理由⸺懷孕的併發症和畸形儿童的風險會隨著母親變老上升,我们會回來提到。現在的問題是,我们是否过於誇大年齡屆滿三十會有懷孕困難的風險呢?

一些最近的研究調查健康女性和她们懷孕的可能性。雖然很少有女性在上了年紀後會主動懷孕,但這些數字可能比妳想像的還不意外。一項研究追蹤了782对夫嘗試懷孕的夫婦。十九至二十六歲的女性顯然是最具生育力的年齡組(有92%在一年內懷孕了)之後,這個趨勢有所下降。二十歲末和三十歲初期女性的之間的生育力並沒有太大的差異。二十七到三十四歲的女性有86%在一年內懷孕。同時相較之下,三十五到三十九歲的女性為82%。其他研究也發現了類似的數字。在三千名丹麥女性的研究裡,三十五至四十歲在一年內懷孕的女性有72%,而按照排卵期性交而懷孕的人有78%。而三十至四十歲的比率則是87%。

我们從這裡得到什麼結果?如果所有從高中畢業的女孩試圖懷孕,十個當中會有一個失敗。二十年後,這個數字會上升到十分之二或三。然而,好處是,大部分的女性在三十多歲時還能夠懷孕!如果我们必須談到年齡的限制,三十五歲會更接近事实。

对於大多數難以懷孕的人来说,年齡不是直接原因。首先,我们要指出的是,問題出在男性的情況為三分之一,這是因為男性的年齡也有重要的影響。女性是主要的問題,或是問題的一部分,才是剩下的三分之二。到底發生什麼問題呢?不孕的最大因素是控制排卵的賀爾蒙失調。經常會歸咎於改變原有賀爾蒙平衡的多囊卵巢綜合症。下一個最常見的原因是輸卵管的損坏。可能是由於如衣原体等性傳播感染,細菌引起輸卵管發炎和留下傷痕。這些問題也可能由子宮內膜生長在错误地方的子宮內膜異位症導致。最後,肌瘤(也就是子宮肌瘤)能夠阻饒懷孕。這些都是懷孕最常見的問題,而不是年齡。

然而,年齡的增加的問題導致流產的風險更高。正如我们前面提到,超过三十五歲女性流產的風險會高出一倍。這意味著那些期盼懷孕的女性比年輕時期懷孕的女性經歷流產的次數更加頻繁。

年齡对妳懷孕的機會、流產的風險、懷孕的併發症和染色体错误如唐氏症增加有明顯的負面影響。但大多數的女性想要在三十歲擁有「舊派思想下的」健康孩子不會有任何的問題。妳有可能是難以懷孕的其中一名女性,當然不可能根據統計數據來确定,但是即使妳在二十八歲仍一直嘗試卻无法懷孕就有可能屬於難以懷孕的族群。如果妳懷疑妳有子宮內膜異位症或多囊卵巢綜合症,或者如果妳有过幾次衣原体感染,不要推遲懷孕太久可能是明智的選擇。妳可能還需要一點額外的幫助与時間來成功懷孕。





生殖器割禮




每年都有數以百萬的女孩遭到切割。她们的生殖器被針頭切割、縫合或刺穿。生殖器官切割是存在於世界好幾個角落的文化習俗,幸運的是它愈來愈少常見。現在最常發生在非洲、中東和亞洲某些國家的部分地區,但西方以前也有施行过生殖器官切割。從十九世紀中葉,美國和英國的許多婦科医生切掉女性陰蒂來治療自慰,因為自慰,當然有可能導致歇斯底里、癲癇和低智商。切割女性的生殖器一直是持續控制女性的殘酷方式。

在挪威,在防止具有移民背景的女童遭到切割上付出許多努力,這些心力似乎得到成效。但是对於很多女性而言傷害已經造成。

世界衛生組織將生殖器切割分為四類。第一類為去掉全部或部分陰蒂,或是陰蒂頭,而陰蒂通常會被切除。同時陰蒂如果不切除可能會長成像陰莖一樣的東西的错误認知存在已久,但是真正无法擺脫的事实是,切除或損坏陰蒂,妳會消除女性性快感的主要來源,這是在企圖控制我们性慾。由於陰蒂主要位於皮膚表層,有些女性會保有得到高潮的感覺与能力22 。其他女性也會發現,陰蒂所產生的疤痕組織會產生持續疼痛。

切割生殖器官的第二種形式与切掉小陰唇有關,通常結合各種形式的陰蒂損傷。小陰唇在我们進入青春期時並隨著青少年時期的性覺醒增長。也許人们看到了生殖器成長与对性好奇之間的連接。藉由去除陰唇,她们能保持幼稚純真的幻想。

女性生殖器切割的第三種形式經常得到最多的關注,因為它是生殖器最具侵略性的改變。在這種形式下會將外陰唇大略縫合在一起,因此,上方剩下的开口就是通往陰道的小洞口。有時候小陰唇和陰蒂甚至會被切除。而尿液与經血同時從這個人工洞口流出。我们拜訪的一位挪威裔索馬利族女性,告訴我们第一次在挪威公廁排尿得到的震撼⸺挪威女性排尿就像大象一樣!她以前花費長達二十分鐘的時間排空她的膀胱,所以她的尿流是稀疏的。月經期間可能也會出現同樣的問題,經血可能會在陰道內積聚。這使得陰道變成細菌的溫床,讓女性暴露在生殖器和尿道感染。

人工打造的洞口对性交而言过小,因此成為女性婚前沒有經歷陰道性交的一種保證。在她因為初次發生性關係而不得不用剪刀或刀子,或由男性的陰莖打开,問題當然就此發生。有些女性有一個大到足夠進行插入式性行為的孔,但生產時仍必須要切开。陰道周圍的疤痕組織无法擴大讓一名嬰儿經过,如果沒有切开,她们會有遭受无法控制的撕裂傷,造成大量出血和損坏直腸的風險。

生殖器切割的最後一種形式則包括上述三種形式未提及的生殖器損傷。例如包括將熱針插進陰蒂⸺一種女性性慾的殺害儀式。

所有形式的生殖器切割可能會造成生殖器長期的問題。此外,這個过程本身是造成感染和出血的主要危險,更不用說心理創傷。世界的大部分地區嚴格禁止生殖器切割有幾個很好的理由。在挪威,即使女孩或女性仍想要這麼做,導致生殖器長期損害的所有形式女性生殖器切割將會受到懲罰,包括允許自己的孩子在海外完成手術。

然而,在生殖器切割沒有完全禁止。如果妳之前進行过切割手術發生問題,妳可以得到幫助。在医院,医生可以進行重建手術,試圖讓妳的生殖器回到正常功能。雖然不能給妳和當初一樣的生殖器,但他们可以盡可能減少妳的日常問題。





* * *



22	在瑪莉‧罗奇(Mary Roach)的著作《砰!科学与性之間奇妙的組合》(Bonk–the curious coupling of science and sex)裡提到研究者瑪莉‧波拿巴公主与埃及女性接受过生殖器切割,但仍透过陰蒂疤痕自慰。





生殖器設計師⸺為什麼我们要对自己的陰戶下刀?




女性(或男性)透过手術選擇改變他们的外表已經不足為奇。隆乳、隆鼻、抽脂,整容⸺有些人為了滿足自己的審美理想,花了很長的時間整形。然而,經由私密處整形手術改變妳的陰戶是相对来说較新的趨勢。

私密處整形手術是外生殖器官改變手術的總稱。可能包含注射脂肪,整理和消除脂肪,減少和擴大。很多都是有可能的,而最常見的形式是陰唇整型術(labioplasty)。這是在特定的小陰唇上進行的手術,使它们變得更短。

我们認為趨勢逐漸上升的私密處手術會造成問題。不,我们不打算寫這個部分,因為我们喜歡輕視女性自己所做出的選擇,或者是因為我们認為女性不具備自行決定如何處理自己身体的權利。當然,妳应该自己決定⸺但這是不同的議題。我们是因為害怕一群年輕女性基於错误的認知而選擇進行私密處整形。根據我们的經驗,有正常、健康生殖器的女性會因為覺得奇怪而選擇私密處整形。這種誤解需要被糾正,要做到這一點,我们必須回去剖析。

我们將選擇陰唇整型術的原因,在医療和美学上做出區別。妳因為鼻子呼吸困難而選擇動刀,和因為妳不喜歡妳鼻子的樣子而動刀之間不一樣。同理,因為妳有疼痛或性交困難而修剪妳的陰唇,和因為妳不覺得妳的生殖器看起來不錯也是不一樣的想法。如果它们对妳造成問題的话,小陰唇的長度只是一個医学問題。這並不一定代表由於審美因素而希望做手術有什麼不妥,但如果妳選擇走這一步,妳做出的選擇是根據知识,而不是誤解是很重要的。

許多女性認為自己小陰唇应该要被大陰唇包覆在內,但成年女性的小陰唇比大陰唇長是正常的。其实女性看待下体的角度沒有一定的方式。但我们确实有一個共同點,形成我们陰戶的有這些部位:大、小陰唇、陰蒂、尿道口以及陰道口。但是,這些部位在每個女性都有不同的差異;其數量令人難以置信。即使如此,小陰唇应该要短小而且被包覆的想法仍然意外存在於許多女性心裡。澳洲研究訪問了十八至二十八歲的女性大眾所謂的「理想陰戶」為何,所有女性皆選出小陰唇被遮住的无毛陰戶照片。

有這麼多的完美、變化多端的生殖器,到底這種想法從何而來?連同其他形式的身体形象壓力,我们可能還會考慮到流行文化和色情文化上的理想。无论如何,它们可能是問題的一部分。當提到審美角度,特別是理想的陰戶時,這個問題就非常麻煩,現实上很難确認這些想法源自何處,或是它们是否不太真实。一旦有人建立正常生殖器只能有短小陰唇的想法,這個想法會比正常肚子是平的還更加強烈。畢竟,我们每天都會看到肚子;我们知道它们有各種形狀和尺寸,因此很容易打掉這種想法。但是,我们不常得到窺探女性雙腿之間的機會,尤其是現在的年輕女性和女孩因為害怕被看見裸体選擇在公共淋浴間穿著泳衣与內褲。裸体不再是自然。裸体經常都与性有所關聯,而对於很多女性来说,展露她们的身体与羞恥相互連結。

我们相信大家对內陰唇的誤解有一部分來自於学校课程提到的青春期發育有極度的落差。就像身体的其他部位,女性生殖器在青春期裡的變化很大,但我们无法隨時記得當初在课堂上聽到生殖器在青春期的改變。在学校裡,我们聽到陰莖如何生長,我们聽到乳房如何成長,我们聽到身体不同部位逐漸長出毛髮覆蓋。我们学到非常多,但是我们沒有学到小陰唇從我们的童年到成年會發生什麼樣的變化。

事实上所有女孩的大陰唇會覆蓋小陰唇。換句话說,我们在儿時都熟悉和習慣了這種樣式的生殖器。但在青春期小陰唇开始增長。对於許多女性来说,她们的小陰唇會變長,並突出外陰唇,甚至往往厚度、皺褶不均。

如果妳的生殖器一直有大陰唇包住小陰唇的情況,特別是如果沒有人告訴妳它會發生而且是正常的话,這種突然改變可能會讓人们大吃一驚。妳告訴朋友自己的小陰唇不明顯,覺得不对勁的想法會增強。然而,這兩種類型都很正常。

換句话說,一些女性認為只有正常或「正确」的生殖器形狀就像我们在儿童時期有的那樣。如果年輕女孩和女性早在幼稚園学到她们的生殖器會發生改變;如果她们对自己成年時兩腿之間的變化想要有更多的了解,也許我们可能不會在現在看到這麼多的私密處整形案例。如果我们早知道生殖器的形式可以有這麼驚人的數量,而其中絕大多數是正常、健康的,那麼會有更少的女性因為誤解而白白挨刀了。

重要的是要記住小陰唇的功能和削減它们代表的意義。由於小陰唇有性功能,它们充滿神經末梢而且触感很好。當妳切除妳的陰唇時,妳正移除生殖器重要与敏感的部分,而所有的手術都涉及風險。在最坏的情況下,有可能會造成組織留疤變得難看,並造成永久性的疼痛,這就是為什麼妳应该在接受手術前仔細考慮。問題大到值得冒這個險吗?







\backmatter

後記





我们的旅程在此結束。希望妳学到許多,我们也希望,讓妳更注意雙腿間的東西,變得对自己的生殖器官更好奇、更有興趣。就算擁有全部的知识,永遠都有更多要去了解。此外,医学不斷進步。我们今天所寫的內容也許不到一個月就过時了。我们不會停止学習。女性生殖器官很奇妙。我们真的希望妳能以擁有它们為傲。

不幸的是,我们的生殖器官也是問題的源頭。許多生殖器的疾病可能會潛在地影響它们。我们精密的生殖系統被用來承受這一切,即使我们不用擔心自己的睪丸會被踢到,但我们有時候太过脆弱。生殖器疾病及問題可能特別會讓人感到太私密与羞愧。很少人可以完全像談到喉嚨感染或椎間盤突出那樣討论這些問題,因此許多女性在事情並非如原本相同時感到孤單与焦慮。我们希望這本書能夠給妳足夠的知识与一點自信,讓妳抬頭挺胸去看医生,妳也會知道何時需要緊張,何時又可以淡定看待。

我们也希望妳能拋棄对妳的生殖器或性生活產生过的負面想法。我们遇过許多覺得自己不正常的女性,只因為她们无法透过陰道插入得到高潮,或是認為自己得到的生殖器或陰戶皰疹就跟解剖学课本上的插圖一樣。看完這本書妳就會知道,這些真的,非常正常。

在我们日常性生活當中時常容易忘記我们的身体不僅僅只有外觀与行為而已,而裸露的身体也並非都和性有關。妳在床上所做的事很容易建立自身的自我價值,而不單指妳外觀的模樣。我们認為自己所缺乏的東西通常會引起強烈的感覺。妳不应该認為他人对妳有所期待而進行性行為。重要的是自己与身体都要学會享受,无论妳是一個人、有一個或三個伴侶,妳就是妳自己。每個人不一定要做所有的事,而每個人也並非看起來都一樣。提到下体時,身体就只是身体,但因為妳只有一個所以顯得更加珍貴。





致謝





我们想要特別感謝一些人。身為男性与医生都同樣優秀的馬略‧喬韓森(Marius Johansen)在書裡提到的医学面向做了很棒的品質認證。我们希望這不會是最後一次的合作。其他優秀的業界人士也提供他们專業的知识。謝謝查坦‧莫(Kjartan Moe)、特朗德‧迪賽斯(Trond Diseth)、 卡利‧歐姆史戴德(Kari Ormstad)、 史懷農‧W‧瑟維(Sveinung W. Sørbye)、 約伦·瑟林(Jorun Thørring)、 安·麗莎‧赫格森(Anne Lise Helgesen)、 安德斯‧瑞能伯格(Anders Røyneberg)、艾瑟斯特‧范奇(Eszter Vanki)、 博利特‧奧斯特維(Berit Austveg)以及萊頓‧佛德(Reidun Førde)你们的对談、觀後感和評论。我们也必須感謝奧斯陸医学院的教授,在不知情的状况下,无论在演講或课堂間的对话裡仍然耐心地替我们解惑。我们再次強調這本書上的任何错误完全是我们的責任。

我们同時謝謝奧斯陸性教育医師團隊、SUSS電信基金會、性与社會中心以及奧拉維亞診所的前任与現任同仁,打造優質、激發学習動力的環境。我们也非常感謝我们親愛的朋友与同事讀过及討论書裡的內容,在我们糾結難懂的解釋的時,給予一針見血的建議。

謝謝每個閱讀我们部落格,提供主題的建議,詢問好問題並鼓勵我们的妳们。我们這本書是為妳而寫。

特別感謝我们的編輯,阿思豪格出版社的娜茲尼恩‧韓艾斯丁(Nazneen Khan-Østrem)。与妳從月經聊到龐克搖滾讓我们感到非常开心。知道妳在後面守護我们,讓我们很有安全感。謝謝畫出最棒插圖的畫家漢娜(TegneHanne),漢娜‧喜比約森(Hanne Sigbjørnsen)。擁有如此有趣的護士在我们團隊实在是個賞賜。

現在,在尾聲的階段,无可避免得提到我们的家人。

妮娜:構思這本書的同時,麥斯也來到這個世上。最有耐心与貼心的男友,弗雷德里克,你是獨一无二的。麥斯,妳是我的一道陽光,我确定妳以後讀到媽媽的書會超級尷尬。我會試著不在餐桌上講太多女性私處的事。媽媽、爸爸還有赫爾奇⸺你们是人人所希望的最棒家人。

艾伦:感謝世界上最棒的家庭,很有耐心地聽著一長串關於處女膜、陰戶痛、皰疹与其他廢话⸺有時候在公共和不洽當的場合當中的媽媽、爸爸及赫爾格。同時感謝將我们看作世界衛生組織共同創辦人卡爾‧伊旺(Karl Evang)的祖父。我对你的愛无法衡量。最重要的,我想要謝謝黑寧,理由可不只是我想要寫出來而已。



閱讀愉快!

妮娜与艾伦

二〇一六年十一月十五日,於奧斯陸





私密處的奇幻旅程


打破所有女孩对身体的错误迷思


GLEDEN MED SKJEDEN




作  者:妮娜‧布罗克(Nina Brochmann)、 艾伦‧斯托肯‧达尔(Ellen Støkken Dahl)

插  圖:漢娜‧希格比喬恩森(Hanne Sigbjørnsen)

譯  者:林怡汎

編  輯:曾盈慈

封面設計:高鍾琪

美術設計:陳柔含

出 版 者:晨星出版有限公司

地  址:台中市407工業區30路1號

電  话:04-23595820

傳  真:04-23550581

网  址:www.morningstar.com.tw

電子郵件:service@morningstar.com.tw

出版年月:2018年11月

定  價:224元

\end{document}