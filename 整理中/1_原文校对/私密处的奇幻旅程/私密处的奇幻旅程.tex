\documentclass[12pt,UTF8]{ctexbook}

% 设置纸张信息。
% 纸张设置配置文件
% 用于定义书籍的页面尺寸和边距

\usepackage[a4paper,twoside]{geometry}
\geometry{
	left=25mm,
	right=20mm,
	top=25mm,
	bottom=25.4mm,
	headsep=1cm, 
    footskip=1cm,
	bindingoffset=10mm
}

% 设置字体,并解决显示难检字问题。
\xeCJKsetup{AutoFallBack=true}
\setCJKmainfont{SimSun}[BoldFont=SimHei, ItalicFont=KaiTi, FallBack=SimSun-ExtB]

% 目录 chapter 级别加点(.)。
\usepackage{titletoc}
\titlecontents{chapter}[0pt]{\vspace{3mm}\bf\addvspace{2pt}\filright}{\contentspush{\thecontentslabel\hspace{0.8em}}}{}{\titlerule*[8pt]{.}\contentspage}

% 设置 part 和 chapter 标题格式。
\ctexset{
	part/name= {第,卷},
	part/number={\chinese{part}},
	chapter/name={第,篇},
	chapter/number={\chinese{chapter}}
}

% 图片相关设置。
\usepackage{graphicx}
\graphicspath{{Images/}}

% 设置署名格式。
\newenvironment{shuming}{\hfill\zihao{4}}

% 注脚每页重新编号,避免编号过大。
\usepackage[perpage]{footmisc}

\title{\heiti\zihao{0} 私密处的奇幻旅程 打破所有女孩对身体的错误迷思}
\author{妮娜‧布罗克 艾伦‧斯托肯‧达尔}
\date{}

\begin{document}

\maketitle
\tableofcontents

\frontmatter

\mainmatter

与细小的尿道孔不同的是,我们很容易看到更大的阴道口。阴道是从阴部到子宫,长度为7〜10公分狭窄的肌肉质管。大多时候质管为扁平的状态,所以前后壁相互挤压。这有助于防水,想象一下!

当你激起性欲时,阴道口会纵、横向扩展,在各方面极富弹性。阴道有点像打褶的裙子。用手指触摸的话,你会感觉到皱褶。

阴道周围的肌肉非常强韧,当你将一根手指放进阴道后它会突然紧缩。和其他肌肉一样,这些骨盆腔底的肌肉,通过锻炼会越来越强健。

阴道壁里面充满湿润的黏膜。大多黏液不是由腺体分泌,而是从身体内部渗出至阴道壁。阴道壁没有任何腺体,但有些分泌液来自于子宫颈。阴道一直保持着湿润的状态,在性欲高涨时会比以往更加潮湿。更多血液流至整个生殖部位,也就有更多液体从阴道壁渗出。阴蒂和小阴唇开始充血,所以你会注意到有增量的血液流往生殖器官。黏液在性奋时产生,让你在自慰或与他人进行性行为时降低阴道受到的摩擦。通常性交时阴道壁有如被连续击打一样,摩擦减少表示对阴道壁的伤害更少了。性行为结束后阴道壁因为些微撕裂而流血是正常的状况,你也会感到有些疼痛。幸运的是,这不会造成伤害。阴道壁善于自我修复。

除了阴道壁分泌的液体,有些黏液来自前庭的两个腺体。它们位于阴道口两侧接近屁股的位置,称为巴多林氏腺(Bartholin’s glands),以丹麦解剖学家卡斯柏‧巴多林(Casper Bartholin)为名。它们分泌出黏液帮助润滑阴道口。椭圆形的巴多林氏腺体积小若豌豆,却可能带来麻烦。输送黏液的小管堵塞的话会形成阴门囊肿,从阴户侧边便能感受到小型的硬块,仿佛像一颗小气球。一旦这类的囊肿感染,就会转变成麻烦的状况,不过透过微型手术即可解决问题。有些人对巴多林氏腺之于阴道润滑的重要性却持反对立场,因为囊肿感染而切除腺体的女性在性欲高涨时还是能感受到阴道润滑。

在阴道壁前端,换句话说就是靠近膀胱的位置,存在女性杂志性专栏里热门讨论的地方。我们讲的正是所谓的G点(G-spot)。以发现者,德国妇产科医生,恩斯特‧格雷芬贝格(Ernst Gräfenberg)之名命名。从一九四〇年代起,科学家不断探讨和研究G点,但争议层出不穷。科学家不确定它到底为何,也无从证实它存在与否。

在部分女性阴道中G点特别敏感,某些女性表示借由刺激便能达到高潮。G点差不多位于阴道壁前端,也就是靠近胃部,且可以用"快过来啊"的挑逗手势来刺激它。幻想一下迪士尼女巫试着诱惑你靠近她,就是那个姿势。根据几名女性的叙述,比起阴道任何地方,G点的刺激会有更好或是不同的感受。你可能会注意到,相较阴户甚至是阴蒂,阴道本身不是特别敏感。敏感是阴道口最重要的关键,同时缓冲更近一步的动作。

媒体通常将G点看作分开的部位。你若阅读过性专栏或性方面的书籍,一定对此印象深刻。二〇一二年英国的一篇综述文章在现有主张G点与阴道为不同部位的文献里得到论点薄弱的结论。大多G点的研究都是通过女性叙述的问卷调查而来。文章中还提到许多相信G点存在的女性不能指出确切位置。研究者同时表示,以成像技术为据的论文无法找到比阴蒂更能让女性得到高潮或快感的部位。

事实上,其中一个假设为G点并非分离的构造,却处于阴蒂最深处,在性交时才会受到刺激,直接经由阴道壁而来。二〇一〇年,一群学者发表了一项研究,他们观察女性进行阴道性行为时阴道壁前端的变化。通过超声波探测发生的过程与找寻G点的位置。他们没有找到,但认定阴蒂内部非常靠近阴道壁前端,所以阴蒂就是G点谜团的解答。

另个可能为G点与一群阴道壁前端的腺体相连,也就是广为人知的史基恩氏腺(Skene's glands),相当于女性的前列腺,围绕男性部分尿道的腺体。史基恩氏腺结合女性射精或是潮射。有些研究表示G点是达到潮射的重要部位,然而现在都只是理论。我们已经知道部分女性会有潮射的状态,却还不清楚G点是否存在。

奇怪的是,和阴道壁一样的地方居然笼罩在神秘当中,尤其又那么多对于G点的谬误。我们屏息以待更多有关女性身体的优质研究。





阴蒂⸺一座冰山




当我们写到阴蒂内部时你应该很讶异。内部的哪个地方?毕竟我们常常形容阴蒂的体积就像葡萄干,位于阴户的最顶端,精准对齐小阴唇交集的所在。但这个小纽扣却只是冰山的一角。藏在骨盆区最深处的这个器官,超越你所有狂野的想象。

即使从十九世纪开始,解剖学家已知阴蒂是巨大的地下器官1 ,却和普遍知识大相径庭。解剖学课本里将男性阴茎描述得非常详尽,而阴蒂仍留下好奇的空间。一九四八年末,格雷氏解剖学决定不标示阴蒂的项目,连男性主宰的医学界也对阴蒂更进一步的研究没有兴趣。阴蒂确切的构造及如何运作还存在相左的看法。以医学的角度来看,实在令人震惊。





我们所知道的是,人们大多将阴蒂形容为骨盆所延伸的大型器官中的一小部分,且向下延伸至阴户两侧。通过X光检测的话,我们能够看到阴蒂整体形状像是倒Y字型。称为阴蒂头的小葡萄干,就在正上方。阴蒂长0.5〜3.5公分不等,由于被阴蒂包皮遮住一小部分的关系,所以看起来会更小。阴蒂头是唯一肉眼看得见的地方,而其下方为阴蒂体,向下延展与身体形成一个角度,模样近似回力镖,前面有一对阴蒂脚,被阴唇两侧包覆在下。





两脚皆有勃起组织阴蒂海绵体(corpus cavernosum),性欲激起时开始充血。两脚中间有额外的勃起组织前庭球(bulbi vestibuli),围绕于阴道及尿道口。

上生物课特别认真的你,会对刚才的叙述有点印象⸺不过男性阴茎不是也有阴茎头、脚和勃起组织吗?女性高潮的主要来源为阴蒂,这是不为人知的秘密,至少显而易见的是,它和勃起的阴茎截然不同。或许会让人感到惊讶,阴蒂与阴茎为相同器官的两种版本。

男女性生殖道胚胎在子宫内约十二周时,长得完全相同,由一种像迷你阴茎(或是巨大阴蒂!)的形体主宰一切,称为生殖结节(genital tubercle)。它具有女性或是男性性器官的潜在发展。阴茎和阴蒂从相同的构造成长,因此两个器官有许多相似的形式与功能。

阴茎头和阴蒂头其实一样,所以两者都被赋予相同的名字,分泌腺(glans)。在两性身体中都是最敏感的地带。据统计,男女性腺头拥有超过八千多条感觉神经末梢。感觉神经末梢接受压力与触摸的信息,传送大脑讯号,转换为疼痛或愉悦的感受。有愈多的神经末梢,大脑就接受愈多不同有力的讯号。然而,阴蒂头比阴茎头更加敏感,因为末梢神经集中在更小的部位:没错,集中度多了五十倍。

不幸的是,阴蒂为性欢愉开关的认知,让一些男性相信向它施予压力是正确不过的。若一点的施压没有满足欲望,他们只会使更大的力气。这不是阴蒂运作的方式。由于富含了许多神经末梢,即使最细微的变化它都能感觉得到。它提供了意想不到的刺激与快感,却也代表疼痛或完全麻痹的过度期极短。长期下来,过度加压造成神经末梢拒绝传送讯号到大脑:阴蒂按钮已切换至"静音"模式。一旦发生,直到准备好再次说话时阴蒂会保持平静。就好比搭讪一样:你做得太过头,事情通常会搞砸。

男性勃起组织让阴茎开始充血变硬。更不用说女性的勃起组织也有一样的功能。当性欲被唤起时,整个阴蒂肿胀至原本的两倍大,完全是令人肃然起敬的勃起。因为阴蒂脚与前庭球在阴唇下及尿道、阴道口周围,让阴户在性奋时看起来更大。此外,阴道前庭与小阴唇由于血液聚集的关系呈现更深、紫红的颜色。

相似处还不止这样。男人最爱吹嘘的早晨"一柱擎天"与夜间勃起,我们女性也有。一九七〇年代佛罗里达大学展开两名阴蒂较大的女性与男性间不同的研究。研究发现女性夜间"勃起"的次数与熟睡男性相当。另一项研究指出女性一夜"勃起"可以达到8次,合计时间为80分钟!

将所有的信息结合起来,你会发现在生物课并没有学到很多关于阴蒂的内容。这个让人骄傲的器官一直被忽视、低估、遗忘已久。只有当我们理解阴蒂是如何延伸至整个骨盆腔时才会感受到自己拥有这个惊奇部位的喜悦。





* * *



1	解剖学家科贝尔特(Kobelt)于一八四〇年代描述阴蒂的内结构,并论定男女性的性器官架构相同。





落红贞洁




数千年来,不同的文化(包括挪威)极为在乎童贞,不是男性,而是女性的贞操。男性没有圣父或妓男、纯洁或污秽之分,但女性就有,"幸运"的是,从新婚之夜阴道流血便能分辨她是哪一类的女性。

有许多人使用这个说词:"破她处。"仿佛没有性经验的女性就可以像开香槟那样被破处,就好像阴道在初次性行为前后的不同,和酩悦香槟有没有软木塞一样。你应该感受到我们的语气,破处根本不是重点。

童贞的概念在主流文化中广为流传。对《嗜血真爱》(True Blood)里的吸血鬼洁西卡而言,每次的性行为都是第一次,她每一次都必须要流血。同样的疑团绕着《权力游戏》(Game of Thrones)的皇后玛格丽‧提利尔(Margaery Tyrell)打转。嫁给第三号国王后真的依然纯洁吗?

经典作品里也有提到童贞与落红。"该死!"我们在挪威文学课里所放的电影里看到克莉丝丁‧拉兰斯达特(Kristin Lavransdatter)的鲜血流到大腿时,可能曾经如此咒骂。不过她却说了这段话:"谁会想要被摘过的花呢?"她在情人艾伦(Erlend)的双臂里嚎啕大哭,而他完全不需要落下男儿泪。身为男人,艾伦根本没有操守好失去。

女性为无邪花朵,甚至取走她的贞洁等同摘下花朵的概念出现在医学用语。女性第一次发生性行为而流血的情况称为失去贞洁,整个事态实在是无法形容地古板。这好似来自不同文化与不同历史年代的男性聚在一起找出控制和限制女性性取向及替身体做决定的方法。




如果你已经搞清楚上述内容,是时候讨论处女膜(hymen)了,这个在阴道口里的神秘构造仍然让全世界女性付出名誉或使她们活在陈旧传统与误解当中。不敢置信的是至今仍以此分化两性。美好正面的性爱只毁掉女性却没对男性造成任何影响。你想想看,在处女膜与流血之上,有迷思主宰一切,这整件事实在太愚蠢。

自古以来处女膜代表贞操的象征,如同迷思所叙述的,它会在女性第一次性交时破掉且流血,只有这个时候。流血被用来当作贞洁的证明⸺过去人们在新婚之夜后按例将染血的床单拿到外面挂起,让左邻右舍看到凡事都是如此进行。

处女膜的迷思为:如果你性交后流血,人们会知道你以前没有过性经验。如果没有流血就表示你已经发生过性关系。然而和其他迷思一样,完全错误。

这个迷思的信念持续存在的原因在于对处女膜为一片薄膜的认知广泛流传。当你听到"薄膜"一字时,你大概想象一片干净的塑胶膜在你戳一个洞后就破裂了。啵!但你曾经用镜子看生殖器的话,你会知道根本没有一张保鲜膜在你的阴道上,就算你没有发生过性关系。别让一个迷思被其他迷思给取代。最近我们听过许多关于"处女膜不存在"的言论。没有封住阴道的薄膜是正确的观念,但造成误解的形体仍然存在。

阴道口里有个环形折状的黏膜倚靠着阴道壁,像一枚戒指。这小戒指以前曾有阴道薄膜、贞洁等称呼。我们叫作处女膜。虽然这些名字都代表相同的意思,但阴道薄膜一词容易让人误解,最好不要使用。

所有女性生来便具有处女膜,然而对你来说却没有任何用途。处女膜等同男性的乳头。从我们是胚胎时就是个没有功能又多余的部位。

处女膜具有厚度与宽度。换言之,它不是薄得跟塑胶膜一样,而是厚又坚固。它的外表在少女青春期时光滑,像中间有洞的甜甜圈。接着身体的荷尔蒙交响乐团登台,处女膜和其他身体部位相同,开始随之产生变化。在青春期结束后,通常会变为新月形。它的后端较宽,朝向肛门,位置依旧围绕在阴道壁,而中间洞口变得更大了。至少这是理论上的模样。事实上,处女膜并没有一定的形状。

多数女性的处女膜为中间开洞的圆形,然而不是每个人的都那么平滑。它通常既皱又有凹痕,也不是性行为的代表。有些人的处女膜上有延伸至阴道口的线状物,所以看起来较像"ø"而非"o"字型。其他人的就像筛子,中间没有一个大洞,而是有许多小孔在上面。又有一群人的处女膜看起来像小流苏一样长在阴道壁,而也有少数女孩的处女膜确实覆盖整个阴道口,她们的处女膜相当坚固,这是变异所造成的问题,因为经血一定要有地方排出才对!具有此类处女膜的女性往往在初经来临时才发现问题。若经血困在阴道里,会造成剧痛且有可能需要进行手术。像密封一样,这个少见的变异型态是我们最接近处女膜迷思的地方。

无论何种样貌处女膜都具有弹性,除了少数包覆整个阴道口的例子。即使如此,处女膜仍是阴道最狭窄的地方。阴道具有惊人的延展、收缩能力:毕竟你能够让婴儿从这里出来。所以处女膜应该也要能够延伸。虽然它有弹性,却不足以在性交时派上用场。有点像是把一条橡皮筋拉得很长,一旦太用力就会应声断开。

当你进行初次阴道性交时,处女膜往剩余的阴道空间延伸。许多女性的处女膜有弹性到足以应付一切,但对其他人来说,处女膜会撕裂和流一点血。换句话说,有些女性第一次发生关系会流血,有些则不会,完全取决于处女膜的弹性。拥有横跨阴道口呈现ø字特别形状处女膜的女性,为了让阴茎或手指有进入的空间而经常觉得此处有撕裂感。

很难确认有多少女性的处女膜在第一次性交后流血。有几项研究记录了统计结果,但数值存在了变数。我们在两项不同研究中分别看到56\%及40\%的女性,在初次进行双方同意的阴道性行为后流血。虽然不是全体女性,比例却是极高。

这些研究访问女性关于第一次性交的经验。我们不能完全确定是处女膜流血,即便它是阴道最狭窄的地方,亦或血液是来自别处。在阴道的部分里我们有提到,如果是略为激烈的性行为、阴道不够湿润或紧张所造成的内部肌肉紧绷,阴道壁里出现小裂痕而出血是正常、合理的。人们第一次发生关系或其他情况下都有可能发生。

另一个处女膜严重的迷思与处女检测有关。这些试验代表人们相信透过目视女性生殖器即可辨别她是否发生过性行为。圣母玛利亚很明显地经历过处女测试,圣女贞德和一大堆在现代不同保守环境下的女性也都有相同的遭遇。

有时我们听闻挪威医生仍然受家属所托对年轻女性进行处女检测,证明自己女儿的贞操完好如初⸺尽管法医专家认为这些测试无关紧要。我们也听说有医生开立处女证明给担心新婚初夜没有流血而惊恐的女性。

然而结果为,通常不可能透过处女膜变化加以辨别女生有无性经验。这使整个处女检测变得荒谬。虽然处女膜可能在性交时因激烈延展而受损,造成的伤害却不一定永久。在许多案例上可以发现,处女膜能够不留任何伤痕复原。

许多处女膜与其变化方式的研究在一项女性初次经验来自性虐待的调查后受到改变。一篇挪威的综述文章指出现有影响幼童处女膜的因素(例如洞口宽大或边缘狭窄),已经认定为没有特殊发现也并非性虐待的证明。这些处女膜的变化可以从没有遭受性虐待的孩童上发现。附带一提,文章的作者还小心翼翼地表示缺乏相关文献并无法证明孩童没有曝露于性虐待的遭遇之下。

基本上,您没办法从女性两腿之间看出她是否有无性经验。处女膜不是为了保护那些没有性经验的人、又或是发生过性行为的人、还是"处女"的人。与身体的其他部位一样,处女膜外观的不同是因人而异。抱歉,处女检测根本没用。

不幸的是,这个常识鲜为人知。女性仍求助手术确保她们能在新婚之夜流血⸺叫作处女膜整形术。直到二〇〇六年,挪威奥斯陆的沃尔沃特私人诊所仍提供这项手术,在寻求医学伦理的咨询后他们停止了处女膜整形术。议会反对这项手术,因为这会使贞洁问题有了快速修补或替代的解决方案:也就是文化改变。

处女膜整形术依然存在。您能在网络上以30美元购入包含特效血浆的假黏膜,保证让您"和黑历史说再见",放心去结婚。顺带一提,埃及政客在二〇〇九年建议禁止进口此产品。

为什么我们选择寻求这些方法,而非告诉人们没有落红不代表没有贞操呢?又为何对我们来说,女性直到婚前保有"完好如初"的证明是如此重要?落红必须变得无意义,处女检测必须彻底废除,最重要的,我们必须屏弃贞操本身的重要性。

问题是找到处女膜可信的资讯很困难⸺尤其,分辨是非与否更难。对于处女膜的认知,我们找到的极少资讯对大多数人而言不易理解、无法取得,也不正确。即便发现了优秀的研究文献,但是医学院最常使用的妇产科课本对处女膜的叙述少之又少,有些迷思也不断反复出现。我们还存有超多疑问。更糟的是,获得的极少数资讯无法满足需要它的人。我们每个人为此都抱有神圣的任务,只是要不要开始的问题而已。





洞外有洞



当提到屁股的时候,我们会说,那是太阳不会照到的地方。这个褐色、充满皱褶的洞孔在探讨女性生殖器时常常被忽视,但分隔阴道与屁股的东西只有一座薄墙。位于身体如此末端的地带,屁股无可避免地与阴道、阴户,和许多女性性形象连在一起。

屁股,又称肛门(anus),为巨大环状肌肉,用于排泄前储存粪便的地方。排便是自古以来是极其重要的任务,我们的身体有一对肛门括约肌。如果其中一块让我们失望的话,还有另外一块括约肌供我们使用。

肛门内括约肌受到所谓的自律神经系统管理,不受意识控制。当身体察觉到直肠开始充满粪便,即对内括约肌下达放松的信号。这就是排便反射作用,我们在迫切找到最近的厕所时就能感受得到。

若我们只有原始反射作用的话,会像小孩一样随地便溺,但人类是社会性动物。肛门外括约肌—为手指放进屁股并夹紧后所摸到位于上方的东西。它属于随意肌,直到状况允许解放前确保您能够撑住。如果您维持夹紧的状态许久,身体会收到暗示,同时下意识知道要错过这次的排泄。粪便微微地回到肠子里并耐心等候更好的时机。我们喜欢戏称为大便出口的部位也会暂时关闭。

屁股是生殖器的暗处,幸运的是它不只有大便的功能。肛门周围与内部充满等候刺激的神经末梢。有些人发现肛门扩展了性生活的规模,如果他们让屁股一起加入狂欢的话。而其他人则借由赞赏屁股为美丽部位感到满足,并不时对此传达爱意。





毛发小常识



身为一名女性即代表胯下有毛发。就天性而言,便是如此。在青春期,稀疏的暗色毛发开始出现在您的维纳斯丘与阴唇边缘。渐渐地,它们开始蔓延、增多,直到形成一片密集、三角形的毛发草原至您的屁股,而通常会横跨著名的比基尼线长到大腿内侧一点。

近年来无毛或微整型过的阴户再次成为理想美感的潮流⸺也成为许多女性焦虑与问题的来源。许多人担心除毛后会长出更多、更深色的毛发,甚至长得更快速。我们在这几年也害怕如果使用剃刀不慎,比基尼线会不受控制地长出一大堆阴毛。同理可循,许多青少年经常借用父亲的刮胡刀剃掉他蹩脚的胡子,希望长得更加粗犷来遮住青春痘。对此我们感到开心,对少年们而言刚好相反,这一切实在太荒谬。

基因与荷尔蒙决定体毛量和生长的时间。出生时,您就具备约五百万个伴你一生的毛囊。举例来说,部分的毛囊位于生殖器官及腋下,对荷尔蒙特别敏感。在青春期,我们身体的性荷尔蒙爆发,对荷尔蒙敏感的毛囊扩大并长出粗厚、暗沉的毛发。荷尔蒙敏感的形式根据个人与基因有所变化,这也解释了为何有些男性背后的毛发浓密而他人胸毛稀疏。虽然看似如此,但其实您在青春期不会长出更多的毛发;只是会渐渐转变为"大人"的毛发。很多人认为剃毛刺激毛发生长的原因只是我们经常在换毛的时候整理它们。

同时有些人觉得除毛时毛发变得更粗、更硬或长得更快。剃掉后的隔天您可能在坐下时会有摸到带刺豪猪的感觉,但这也不是事实。我们毛发主要以死亡细胞构成。事实上,所有皮肤上看得到的毛发都是死掉的蛋白质,唯一活着的物质都在毛囊下面。即使您剃掉毛发,毛囊也不会知道,这些死亡的形体只会出现在《魔鬼克星》(Ghosthunters)。现实世界中,毛囊和以往一样的速率持续长出毛发,然而一无所知的您却残酷地割下它所安排的一切。

毛囊的大小也决定头发长出的厚度。无论剃了多少次,大小不会改变。也就是说,感觉到毛发变硬只是因为长出来时变短了。一般毛发离开原本的毛囊顶端后长得愈来愈细,这也是为何感觉柔软的原因。当剃毛的时候,我们在毛发最厚、离皮肤表面最近的时候割除,所以重新长出来后,它的尖端会变厚一阵子。

我们可能会咒骂(或珍惜)生长的毛发,但体毛的配置是命中注定的。如果您决定对毛发做其他的打算,那是您的选择。人体的毛发绝对有它的功能存在,却也没重要到必须留住它,如果您想移除它的话。不过值得知道的是毛发有助提高我们对性的敏感。如果您的伴侣轻抚您的阴毛,弯下去的部分会对毛囊传达信号,将信息送至神经系统。我们的毛囊连接许多神经末梢,所以没有毛发我们会损失一些感知的体验。

历史上,不同形式的除毛对两性来说都是练习。现在,您可以剃掉、上蜡、拔毛或使用脱毛霜来当作短期的解决方案。最重要的,即使它们各有优缺,这些选择还是与个人偏好有关。

拔毛与蜜蜡除毛会导致长出的毛发稀疏,因为毛囊在您连根拔起后受到极大的伤害。它的坏处在于,稀疏的毛发变得较难穿透皮肤,可能造成倒插及毛囊发炎。而脱毛霜是借由破坏蛋白质结构来"移除"皮肤表层的毛发。既然毛囊没受影响,比起使用其他方法,人们自然也少了毛发倒插的问题。

除毛有许多主要的问题:剃毛肿块、毛发倒插与毛囊炎(pseudofolliculitis barbae)。除毛时,尤其是卷曲的毛发,重新长出时可能会往回长进皮肤里。身体将倒插毛发视为外来个体并触发毛囊发炎,看起来像是一个小点。你若不幸运或挖开小点肿块的话,可能一并得到细菌感染。它会变得更痛更肿,通常会留下疤痕。

媒体上充斥着无肿块除毛的建议,我们完全相信美容专家的建议:毕竟,刮干净的胯下有倒插的毛与斑点实在有碍观瞻。但你真的需要除毛美容店推销给你、一瓶65欧元的乳霜吗?或是一把5美元的吉列维纳斯亲肤敏感肌专用除毛刀吗?

不幸的是,你正在虚掷金钱。真的为倒插毛发与毛囊感染而困扰的话,试试以除毛霜代替其他方式。如果你偏好拔毛、蜜蜡或是剃毛,就非常需要注意卫生。在你开始前需要将除毛区域清洗干净。有毛囊感染风险的人需要杀菌液冲洗或在除完后使用杀菌乳液。你可以在药店柜台购买这些产品,比起美容院贩卖精美瓶身的专门产品还便宜。

最后,非常重要的一点,若你有毛发倒插或感染的问题,你应该避免挤压,因为会造成皮肤疤痕。还有,最坏的状况是感染区域可能会扩散。更有可能造成毛囊严重感染形成一颗葡萄大小的肿块。这样的话,你得寻求能够温柔排出脓肿并在必要时给予抗生素药方的医生协助。





除毛五诫





1.不要直接除毛或拉扯皮肤



如果你拉紧皮肤、直接除毛的话,会因剃掉表层的毛发而拥有最光滑、最柔软平面。然而遗憾的是,这个方法更容易让头发生长时嵌入肌肤,造成毛囊发炎。





2.永远使用干净、锋利的除毛刀,最好是新的



因为除毛刀太贵所以很想多用几次,这是假节约的行为。锐利的刀片才能将毛发割除得更干净,更不容易倒插。你也能用更少的力气除毛,帮助预防刺激及肿块的出现。此外,使用过的刀片充满细菌,会导致毛囊受到感染。





3.使用(便宜)单一刀头的除毛刀



除毛刀永远有着新颖、细致的版本以及增加的刀头数,结果价格跟着高涨。而上头标语往往都是"更彻底除毛",或许会令人讶异,额外的刀头会切掉肌肤表层下的毛发,所以造成更多倒插的毛发。再者,高价意味着更多人不会经常更换剃刀,使刀子变钝充满细菌,你最好不要这么做。男性除毛刀通常较为便宜,所以值得买来使用。





4.使用大量温水



无论如何必须避免直接干剃,干燥的毛发坚硬因此难以割除。你必须花费更多力气来达到目的,这会更伤害皮肤,增加红肿和发炎。温水是让毛发柔软最有效的方式。如果在剃毛前五分钟使用除毛泡也有同样的效用,即使效果不大,却是大多数人使用的方式(快速涂上,快速剃下)。





5.温和去角质



用画圆的方式温柔清洗除毛部位,同时使用去角质手套或颗粒去角质霜,帮助倒插毛发离开肌肤。切记勿过度使用,因为会造成更多伤害及皮肤发炎。





内部生殖器官⸺潜藏在内的宝物



大家很容易忘记女性生殖器官不只有阴户与阴道,还有皮肤底下的脂肪、肌肉存在于柔软、看不到的部位,包括内部生殖器官。

让我们开始这趟旅程吧。若用一根手指进入阴道,您会感受到约长7〜10公分,有着一样厚度、形状像是鼻尖—再大一点的柔软小凸出物。那是子宫的脖子或称作子宫颈(cervix),为子宫的入口。从阴道开始,子宫颈看起来像扁平半球。第一看起来,它并不像出口或通道,但在正中央却有个叫作子宫颈口的小洞。这个开口长约2〜3公分、极为狭窄的通道带我们通往子宫内部,为经血流出的通道,分泌物也会从这里流出。事实上,这个小小的道路是制造大多分泌物的场所。





很多人以为阴道往子宫的通道是张开的,我们也常常被问到以下问题:怀孕时如果做爱的话会不会让阴茎撞到宝宝呢?有许多人对性行为与子宫的关系感到好奇。读过村上春树的小说《海边的卡夫卡》(Kafka on the Shore)的话,你应该挺享受那段女子感受到男人的精子洒在她子宫壁的桥段,就好像他射精时阴茎在她的子宫里一样。你不可能让阴茎进到子宫里,子宫颈并非开放的气室,它是封闭的。任何状况下,阴道在长、宽上具有弹性所以深到能够容纳大部分的阴茎,但也完全没有必要再更深入进去。

我们的认知是大多女性并不注意自己的子宫颈,也的确不意外。你既看不到也不一定感受、意识到它的存在。但为了健康着想,子宫颈确实需要你的重视。子宫颈是年轻女性得到癌症的部位之一,此外,它也是许多性传染病出现症状的地方。

子宫颈很重要,然而也只是更大型器官⸺子宫(uterus)的一小部分。子宫一般为拳头大小的器官,一旦怀孕就会扩张得非常离谱,毕竟在孕期里需要大到足以装下一个(或多个)成长胚胎的地方。更年期前的女性,子宫大约7.5公分长,重约70克以下。子宫的外型类似一颗倒过来的梨子,而子宫颈就是最纤细的根茎部分。

女性的子宫多为前倾,朝向肚脐,与阴道大约呈90度的距离。这也是另一个阴茎无法进入子宫的原因:阴茎在勃起时不能弯曲,如果弯曲,它就会断掉,阴茎不是杂技演员!有20\%的女性子宫后倾,但运作方式和前倾的完全相同;就和有些人是蓝眼睛而有人的是棕色一样,他们的眼睛都看得到。

子宫是空心的,却也不像桶子那样,因为它没有空气。子宫的前后壁与阴道壁一样相互紧压,夹在其中的为一小层液体。

子宫拥有非常厚的肌肉壁,这些肌肉非常重要,例如在凝结的经血往极窄的子宫颈排出时得仰赖它们。子宫肌肉收缩就好像一块被拧紧的菜瓜布。当你经痛时所感觉到的疼痛和腹部或背部绞痛一样,但痛处却来自于将经血与黏液推出的子宫本身。

子宫壁有许多层,而最里面的子宫内膜为黏膜的一种。它大幅改变经期的过程,并在其中扮演核心的角色。子宫内膜每个月变大和增厚,如果没有怀孕便会从子宫排出。值得记住这个名字的原因是它和困扰许多女性甚钜的病症相似:子宫内膜异位(endometriosis)。这是子宫内壁长到身体其他地方所造成的疾病。在其他引发的症状里,它会产生更严重的经痛。你稍后会学到更多关于子宫内膜异位的内容。

将子宫想成一个三角形,其中一个角朝下而其他两角分别有细细的管子延伸出去。这是大家所知道的输卵管(fallopian tubes),两侧都有十公分长,目的在于将卵子从卵巢(ovaries)向下送至子宫。输卵管末端有着布满小型指状物的输卵管伞(fimbriae),向卵巢延伸并接收它排出的卵子。在输卵管受精的卵子会往子宫移动,为了生长而着床于子宫内膜。

我们拥有两个像小袋子或垒包一样的卵巢,位于子宫的两侧,而且它们有两个任务。第一为储藏并使女性的性细胞成熟成卵子。与男性不同的是,女性在一生中不会制造新的性细胞。打从一出生我们就只拥有三十万颗卵子,不过它们皆尚未成熟。生来所拥有的卵子其实都是受精卵的前身。这些前身在胚胎生长的六天就已经形成。直到青春期开始月经周期时,它们才会为下一个任务做准备,再来一批批的成熟。但由于没收到大脑排卵的信号,最后以大规模的形式就此灭亡。到达青春期后,我们为了演练,早已流失了三分之一的卵子,只剩约莫十八万颗。而二十五岁的时候,我们剩六万五千颗。这些卵子必须耐心等候,在每次月经周期时成熟释出。

你现在应该觉得莫名其妙,我们在青春期开始拥有十八万颗卵子。一生中我们不会有许多生理期,那要数万颗卵子做什么呢?事实上⸺这也同时让人感到讶异,我们每个月可以用掉一千颗卵子,不是单单只有一颗。换言之,卵子与男性精子的不同并非在于经常大量制造。对女性,也对男性来说,多个性细胞相互竞争就为了对的匹配及生成婴儿。每个月有一千颗卵子成熟,却只有一颗能够跨越封锁线被卵巢选择并排出。剩下的就会被残忍地淘汰、摧毁。

有好多次,我们一直想到一个关于荷尔蒙避孕的问题:避孕会阻止排卵让卵子和生育力维持得更长久吗?对身体来说,保存卵子到准备怀孕时使用而非每个月透过生理期抛弃它们是非常值得的,毕竟听起来很有逻辑,但它却不是这么运作。荷尔蒙避孕每个月只会避掉一颗被卵巢选上的卵子,不能预防那一千颗成熟的卵子。无论避孕多少次,你每个月还是会失去许多卵子。

到了四十五至五十五岁,我们进入更年期的年龄,女性身体已经历许多如同青春期所遇到戏剧性转变的阶段。最重要的改变是停止受孕,我们已经用完库存的卵子。每位女性更年期的年龄有所不同,且开始的时间点主要由基因决定。

另外,有些女性天生拥有比他人更多的卵子。而男性还会持续制造精子直到心脏停止跳动的时候⸺一天最多至数百万条。即使每年的精子品质下降,他们的生育力却没有保存期限2 。七十二岁的滚石乐团主唱米克‧杰格(Mick Jagger),目前正等待他与年轻模特儿女友的第八名孩子。这个世界实在不公平。

卵巢的第二项任务为制造荷尔蒙,最重要也最广为人知的就是雌激素(oestrogen)与黄体素(progesterone)。这些激素在生命不同阶段改变我们的身体,并与包含大脑在内的几个不同部位的激素控制月经周期。不过我们之后再回来谈论这段内容。





* * *



2	精子细胞的品质随着年龄降低。换句话说,男性的岁数影响伴侣的生育力与孩子先天疾病的风险。





他与她与亻她



对许多人而言,性别一词相当于:女性与男性、女孩与男孩。当你听到:"什么是男人?"或"什么是女人?"的问题时,也许会觉得小事一桩;因为理所当然的,男人就是拥有男性身体的人,女人就是拥有女性身体的人。举例来说,《阴缘际会》一书,是一本关于人类所拥有的阴道与其他女性生殖器的书,所以这一定是本关于女性的书,没错吧?

这么想的话确实不令你意外,但却没那么单纯。无论我们是女性或男性,并非只凭生殖器官或体态来判断。此外,两性身体上的不同比你想象的要来得少。

在接下来的部分,我们会把重点放在三个决定性别因素:我们的染色体(chromosomes),在这里称为基因性别(genetic gender);我们的身体,或称生理性别(biological gender);以及心理因素,或叫作心理性别(psychological gender)。我们没有说这些就是构成"性别"的完全因素,当然也可以探讨社会和文化因素。不过既然这是本医学书,我们决定着重于基因、身体与心理方面。





基因性别⸺烹饪书



你曾经看过DNA螺旋图吗?从高倍率显微镜拉近观看,它就像一个扭曲成螺旋形状的梯子。而DNA螺旋梯上的横挡并不同于你用来换公寓灯泡所踩的阶梯。以小到只能从显微镜看到的宽度来说,DNA螺旋梯极长且拥有非常特别的横挡。





DNA螺旋梯上的横挡由不同物质组成,我们可以当作字母来看。每个横挡为两个字母,合在一起能够看成一组密码或是小食谱。每个食谱所编列的蛋白质会在身体执行特别的任务。组合起来,我们将这些蛋白质的密码称作基因(gene)。我们的基因决定我们拥有蓝色或棕色的眼睛、两条或三条腿、翅膀及尾巴或脑袋的大小。这些密码有点像是集结不同食谱的烹饪书,记载每一个必备的物质来塑造我们。这类烹饪书有个别致的名字叫作基因体(genome),我们的基因体就是整个基因的食谱。

人类身体里的每一个细胞都包含一本专属的烹饪书,每个细胞里面有平均约3公尺的DNA双股螺旋,为警察透过血液、精子、指甲或皮肤细胞找到犯人的依据。如果你从别人身上随便取出一个细胞,例如挪威首相艾娜‧瑟尔贝克(Erna Solberg),这个细胞,照理来说,会包含所有打造全新的她,也就是复制品所需要的资讯。但是一个3公尺的烹饪书,是如何挤进像细胞那么小的物质呢?DNA双股螺旋会相互紧密交缠成条,像毛线一样,所以什么东西都可以进得去。每个细胞有46条,将整个基因密码,也就是烹饪书组合起来,这每一条便是所谓的染色体(chromosomes)。

染色体是成对的,所以我们有23对、46条染色体,每一对里的每一条分别来自于母亲与父亲。

讲到性别的话,唯一能决定的就是:我们的第23对性染色体。就基因上来说,这两条决定我们是男生还是女生。性染色体有两种,意即大家所知的X与Y。女性的组合为相同种类,代号为XX,而男性的是X及Y变体,代号为XY。

回想一下,我们是由母亲的一个细胞(卵子细胞)和父亲的一个细胞(精子细胞)而来。每个细胞包含半套染色体组,也就是23条或是半本烹饪书。在怀孕时,你把来自母亲的半本烹饪书及另外半本来自父亲的烹饪书合而为一,给了孩子一整本有着独特食谱的烹饪书。

因为基因上为女性的人没有Y染色体,只有两个X,所以卵细胞只有性染色体X。这是母亲对胚胎的第23对染色体的影响,她永远无法提供Y染色体。然而,父亲的精子细胞,就包含X或Y染色体,精子细胞中的X、Y染色体的比例各占一半。与卵子结合的精子细胞为Y染色体的话,则组合密码为XY,胚胎里就会是男孩。若和卵子结合的精子细胞为X染色体,则胚胎为女孩,组合密码为XX。

因此,一直都是由男性"决定"孩子的性别。自古以来女性受到严重的"生男孩"压力,你可能读过关于失望的国王期盼皇后诞下合适继承者的小说,而生下的孩子当然必须是男生。

现在我们懂得更多了,孩子是女或男的几率单纯各占一半3 ,取决于其中一种男性精子细胞与卵子做结合,女性卵细胞对孩子性别没有任何影响。

总结来说:若第23对染色体有两个X染色体,胚胎的烹饪书会下达"变成女性"的指令。如果分别为X、Y染色体,则烹饪书会下达"变成男性"的指令。

一切看起来完整又简单,有着这些食谱,你会建立起性别只是"非此即彼"的印象。之后你会发现,不单单只是这样。事实上,男性及女性的生殖器官极为相像,在器官成熟的过程之中有许多现象发生。我们往往太过专注于两者间的不同,然而,我们两腿间所拥有的并非只是"洞或棒状物"而已。

无论染色体或基因,皆有可能在DNA里的某个地方出错,而使食谱也造成错误的结果。食谱上的错误代表结果也会跟着不同—有点像是本来该加一公斤的糖却加成胡椒一样。或许尝起来还不错,却跟您所想得完全不同。

人们有可能生下来就有过多或过少的性染色体,那么这样会变成哪个性别呢?是X、XXX,还是XXY呢?这真是个好问题。可能您现在才意识到,其实根本没有所谓的YY染色体组合,因为两个精子细胞是无法结合而成一个婴儿胚胎。

为了追根究柢,我们需要探讨一些关于生殖器官是如何发展的内容,因此,这也变成介绍第二个性别面向的好时机:生理性别。





生理性别⸺身体与性器官



目前为止,我们已经知道卵细胞会与精子细胞结合。没发生问题的话,我们就有XX或是XY食谱⸺女人或男人。尽管如此,男、女生的胚胎在一开始完全没有差别。事实上在初期,无论染色体的组合为何,胚胎是完全一样的。胚胎一直都是从不分性别的生殖器开始发展,变成女性或男性生殖器官的概率皆有可能,而内生殖器官也会随之变为睾丸与卵巢。

为了方便,我们着重在外生殖器官就好。它们一开始是长这样的:







生殖部位的最顶部为生殖结节。看起来有点像小阴茎,不是吗?还是像阴蒂呢?生殖结节确实都能变成两者。

为了让中性胚胎生殖器发展为男性生殖器官,胚胎在怀孕初期需要所有在重要时程中具备的要素。胚胎必须在正确时刻受到男性性荷尔蒙的影响,在这场游戏里担任要角的荷尔蒙为睾酮(testosterone),只有胚胎有Y染色体才能制造出来。附有Y染色体的胚胎没有受到睾酮影响的话,主要原因为一个至多个的胚胎基因发生错误,而生殖部位也自然变为阴户,因此导致基因上为男孩但生殖器官却为女孩的形式。

换句话说,除非发生反向指令的特别状况,阴户是所有胚胎具备的部位。有些男性会理解为男性有"额外的部位",而女性就很普通—好比说将白色T恤与花俏的派对上衣做比较,您可以自行解读想法。您也可以直接说女性是第一性,男性却是变体,为第二性。等等⋯⋯这样不也是女性吗?

看看性别发展的图形,我们在之前提过,胚胎生殖部位顶端的小结,生殖结节,能变为阴茎或阴蒂。如果您对阴茎稍微了解,又在前面读过关于阴蒂的章节,一定会明白两者有许多共通处。

对于受阴蒂头大小所苦的女性而言尤其重要。阴蒂长得像小巧可爱的钮扣是我们以前接收到的知识,但它的外表可能会长到延伸出来。这不代表您更像男人!阴蒂和阴茎一样有不同的大小,长度介于七到二十公分,所以小阴茎也不会让男性更像女人。

回到我们的胚胎,男性的尿道与阴茎合为一体,而女性的尿道是分开的部位。成长中的阴蒂⸺阴茎旁边形成皱褶,这些皱褶会变为男性的阴囊(scrotum)或女性的大阴唇(labia majora)。要形成阴囊的话,就必须要在中间结合。若要形成阴唇,虽然不需要结合,却还要再成长一些。

如果您不相信我们所说男性外生殖器官和我们的非常类似,下次遇到裸身的男性您可以仔细瞧瞧他的两腿中间。您会看到,他的阴囊是由一条整齐、细长的线,像裂缝一样分成两边。您知道吗?那就是裂缝!这是阴唇合而为一所变成的阴囊!阴茎没有变化,但过度生长的阴蒂有着内藏的尿道:想像一下,将它大幅缩小,尿道下移,阴囊一分为二,就会变成另一种阴户了。

哇!看起来实在很酷,但千万不要切开您所认识的可爱男性,男人需要阴囊存放睾丸。这与外科医生将男性身体改造成女性的性别确认手术(gender confirmation surgery)非常相似,我们稍后再来讨论那个内容。

现在我们回到染色体错误的问题。所有缺乏Y染色体的胚胎在生理上会变为女性,反之有着睾酮影响的Y染色体胎儿会在生理上成为男性;或是像人气漫画集《世上最后一个男人》(Y: The Last Man)一样,全都被消灭了⸺并非如此。

这些只是理论上的案例,若将胎儿编为X或XXX,烹饪书会说这是女人。如果是Y或XXY的话,食谱会朝男性开始成长。然而在其他烹饪书内,结果并非总是和食谱写的一样。以身体来说,即使基因上为男性,还是有可能长成女性⸺反之亦然!

有些胎儿在基因上为男性,却无法对身体产生的睾酮有反应。少了睾酮,他们会在外表上变成女性,两腿间有阴户而非阴茎与阴囊。而渐变的状况也的确存在,即使有阴户,有人出生没有子宫,肚子里没有卵巢但是两腿间却有睾丸。外生殖器也有可能在最后发展为阴茎—睾丸的组合(男性生殖器官)与阴户兼具的情形。

每年都会有接生员在孩子出生时被父母问性别问到想破头的事情。事实上,他们无法给予明确的答案。这类的诊断可以称为双性人(intersex)4 ,意即"介于两者之间的性别"。

之前我们提到的案例,基因性别与外生殖器官没有对应关系,就是双性的一种。双性有许多形式,可能是外生殖器与性别不符,或内、外生殖器对应不同性别或是兼具。

许多双性孩童从一出生便接受手术,带给我们遗憾的历史教训。以前,所有与生俱来"不明确"外生殖器的孩子会透过手术变为女性。第一,人们觉得合理是因为性别与后天教养有关。只要以指定的性别养育孩童,他们就会认为自己便是那个性别。譬如给他们娃娃及粉色衣服,也是一样的手法。这就是俗话所说的,先天之于后天的教养。

再者,外科医生认为打造阴户比阴茎与睾丸的成果要来得好。他们本身即是男性,便会觉得男人没办法在小又只有一半功能的阴茎下生活,但对女性来说只有一半功能的阴户并不会造成问题。毕竟,性对男性是最重要的。结果造成孩子们在生理上为女孩,在基因及心理上,却还是男孩。许多生命就这样被毁灭。

因此有很多外科医生大幅改变了做法。现在使用更深入的检查来判定性别,确保孩童在出生前为"正确性别"。虽然不再让婴儿出生时动手术,但是通常需要花好几年的时间来检验。

这类的做法有一些反对意见。许多人觉得这些孩子不该被动手术,应该要让他们成年时决定自己所要的选择。这派看法认为人类必须往男孩或女孩的模样发展是错误的原则。为什么不能接受两者之间呢?为什么我们不能以"双性"养育孩子,让他们慢慢探索自己的性别认同呢?这也带我们前往第三个性别面向:心理性别。





心理性别⸺认知的问题



心理性别比起生理又更难以解释,因为我们心理性别是关于认同的问题:我们对自己的想法、我们是谁。这是非常主观的,只有您知道适合自己的是什么。

许多事情被忽略的原因是我们对于"正常"这件事想得太复杂。对多数人来讲,一个性别有三个因素。我们觉得是女人,我们两腿间看起来是女人,而我们的基因也确定我们是女人。实际上我们大多经历过的事情对每个人来说不代表一样⸺这是人类不断反复学习的课题。

当你的儿子说他是女生,只想要穿洋装,比起火车组合与足球更喜欢姐姐的芭比收藏。要强调这只是过渡期非常容易,然而事实并非如此。为了成为女生并不一定要"阴柔"或喜欢娃娃胜过足球。心理性别与个性不同,也不须根据传统性别角色来判断。尽管如此,人类的心理性别很有可能不同于生殖及基因性别。

我们经常使用跨性别或"生错身体"来描述不同性别的人,所以什么是跨性别呢?跨性别来自拉丁语,意即"穿过"、"跨越"或"改变",和超越有一样的意思。它用来形容性别认同与其所属基因、生理性别相反的人士。如果没有特定认同性别的话他们也会称自己为跨性别:并不是所有人都需要这类的标签。跨性别通常会以星字号标示(trans*),表达涵盖许多层面的广义用词。举例来说,可以问一个跨性别人士希望被如何称呼:是他、她或是他们呢?还是其他完全不同的称呼呢?您不需要事前知道,所以好奇的话就问吧。

非跨性别人士称为顺性别(cis),同样来自拉丁语,即"跨越"的反义词。而顺性别一词有"留在同一边"的含义。

跨性别女性为以男性身体出生但却是女性,并希望改变她的身体让身、心理性别相符的人。而跨性别男性是以女性身体出生却认定自己为男性的人。

许多跨性别人士从童年开始便知道自己不属于生理上的性别。和未知事物看似可怕的道理一样,对家长来说可能会受到惊吓。因此我们认为探讨跨性别与唤起其意识非常重要。如果人们怀疑自己的孩子"出生在错误的身体",可以交由儿科医师判断。倘若确实,孩童可以透过荷尔蒙与手术的辅助接受性别确认治疗。

幸运的是,多数人渐渐在主流文化下使用跨性别一词。二〇一三年电视剧《劲爆女子监狱》(Orange is the New Black)的演员拉维恩‧考克斯(Laverne Cox)与卡戴珊家族的凯特琳‧詹纳(Caitlyn Jenner),为近年来大力推广跨性别议题的代表人物。挪威影集《错位人生》(Born in the Wrong Body)吸引了许多关注,许多挪威跨性别者也积极参与社会辩论。资深医师艾斯本‧艾斯特‧皮耶里‧本尼斯泰德(Esben Esther Pirelli Benestad)则主张自己是流性人(genderfluid),并偏好使用他们一词。而不久前,跨性别男性卢卡‧戴伦‧艾斯贝沙(Luca Dalen Espseth)也挺身而出告诉挪威孩童与年轻跨性别族群他们并不孤单。





结论



决定我们所属的性别(至少)有三个因素,分别为基因、生理与心理性别。性别不需要二分法,我们染色体的缺陷可能代表我们不需要典型的染色体组合XX与XY。基因缺陷有可能使我们在生殖器官的发展中介于女性与男性之间的状态。而心理性别也可能与您出生的生殖及基因性别有所不同。换句话说,性别并非看起来如此简单。我们希望这个概述能激起你的好奇,让你更能接受性别带来的多元可能。





* * *



3	事实上机率并非完全各占一半,新生儿男性多过女性的比例较高。



4	双性人:对于这个名词有许多不同的看法,有可能用来描述一群有著这样的医疗状况或身分的人。我们认为在男女性身体变异上会是个不错的词汇,但我们知道有其他人倾向使用不同名称来称呼自己。





分泌物、月经与血栓





如同我们身体的其他洞口,阴道也是一个出口,并不只是让东西放进去的地方。从那里出来的有尖叫的婴儿、血液、黏液与血栓,因此阴道成为充满喜悦与尴尬的根源,也是我们发现身体异状的一个管道。而主导这一切的重要物质,就是荷尔蒙。是时候探讨我们生殖器官中不太明显的部分了。





私密处冲洗器与气味



分泌物,这个词琅琅上口。看起来是会不禁想起水管系统或是污水管的奇怪字眼。我们最熟悉的分泌物为黏滑、乳状或黄白色污垢,青春期之后时常出现在我们的内裤上,让我们的的内裤变脏。也许分泌物很难令我们惊讶的原因在于它不是热门或高谈阔论的话题,而且看起来又脏兮兮的。此外,在大多异性恋男性眼里,湿润阴道最重要的是能够把阴茎放进去而已。那么什么是分泌物呢?下体里不同液状物质之间有什么差别呢?为什么我们要在一开始留意分泌物呢?

所有历经青春期的健康女生都会在内裤上发现分泌物,每一天都有。分泌物是从青春期第一天开始,受到雌激素(oestrogen)影响导致阴道持续渗出的液态物质。有些分泌物是来自子宫颈的腺体。阴道本身没有任何的腺体,却有许多结合子宫颈、阴道口,包括巴多林氏腺所分泌之液体透过阴道壁流出。

基本上一天会有半匙到一茶匙的分泌物渗出,依照女性身体与生理周期有所变化,而有些使用荷尔蒙避孕药及怀孕的女性会发现分泌物增加。从液态到黏稠,分泌物的浓稠度也会不一样,在排卵期前则会出现蛋白细丝状分泌物。

分泌物不单只是正常的现象,而是必须的。它让阴道变成自净管道,目的在于保持阴道干净并除掉真菌、细菌,连同表层黏膜的坏死细胞等不速之客。此外,分泌物还富有大量良性乳酸菌,称为乳酸杆菌(lactobacilli)。这些物质,对,相信您已经猜到,就是乳酸菌让分泌物尝起来、闻起来有些微的酸味。

更重要的是,乳酸菌所带来的低酸碱值对健康阴道是不可或缺的。多数造成疾病的细菌无法在酸性的环境下生存。再者,彼此在空间及养分上的需求相同,所以乳酸菌能够阻止潜在有害菌体找到生长的环境,最终达到预防感染的目的。简单来说,分泌物会维持阴道的健康。

同一时间,它能润滑黏膜保持滋润。干燥的黏膜容易撕裂,这种问题一旦发生,各种麻烦马上接踵而至。只要想想您的嘴巴没有口水就好,没有分泌物,阴道黏膜会撕裂,您会感到些许疼痛。性交跟着变成梦魇,而身体屏障遭到破坏使性行为传染风险增加。换言之,分泌物并不是应该要从阴道冲掉的脏东西,而是我们重要的伙伴。

问题在于,人们认为分泌物很脏,是不洁或卫生不好的象征。女生很少把穿过的内裤丢在或挂在浴室里。在某些社会里,有人竟然认为应该将阴道上的分泌物冲洗干净。您大概没想过羞辱人的"混蛋"(douchebag)一词从何而来。妮娜也是,直到她搬到美国,把在店里买的一瓶私密处清洁剂留在宿舍公共淋浴间内以后才明白。过一阵子,有位窃笑的学生告诉她应该要把洗剂拿走,因为关于带着冲洗器的挪威女孩的谣言已经满天飞。

"冲洗器?"妮娜带有一丝困惑地问着。她马上被告知每个人都以为她用一种灯泡状的注射器挤出带有香味的肥皂水到她的阴道里—大多是性工作者及其他女性常用的东西。妮娜试着解释这只是普通的pH3.5阴道清洗剂,但随即放弃说服其他同学。好女孩千万不要,拜托,把重点放在生殖部位需要经常冲洗这件事。即使承认您觉得清洗生殖器是个禁忌,就好像放弃分泌物这个好东西一样。妮娜最后将瓶子留在淋浴间。

我们的生殖器官最喜欢温水或温和的亲密部位清洗皂。您不应该用一般的肥皂,因为很容易造成下体脆弱的黏膜干燥或受损。通常使用太刺激的洗剂或者过于用力清洁会造成搔痒或灼热感。无论如何,您不该用错误方法清洗阴道,增加它受到感染的风险。

女性会在什么理由下觉得她们需要冲洗自己的阴道呢?对大多数人来说,大概和气味有关。许多和我们对谈过的女性对自己下体闻起来是否"正常"感到焦虑。她们担心开会时坐在隔壁的同事是否闻得到阴道的味道,或是拒绝让性伴侣往私处探头以免闻到味道而失去兴致。

健康生殖部位的味道,就是它本来的味道。新鲜的分泌物含有乳酸的关系所以闻起来、尝起来带有一点酸味。此外,阴户与鼠蹊部布满许多汗腺,紧身裤、合成纤维的内裤及翘脚的姿势在双腿间制造了温暖的环境。一整天下来,你也自然而然流了一堆汗。一天的分泌物和汗水在与残留的尿液作用后形成特别的气味。在我们女性朋友的圈子里,我们常常使用挪威话"discomus",意即舞厅老鼠,来描述在舞池历经长夜或在健身房运动后生殖部位(你的小老鼠)所排出的特别味道。味道闻起来没那么糟糕,只是气味比较重。





分泌物的味道与分量依据生理周期有所不同。我们的性荷尔蒙有影响身体除掉恶气物质三甲胺(trimethylamine)的能力,也就是随处可闻、腐败的鱼臭味。经研究发现,生理前、中期的健康女性身体至少有60\%至70\%的能力可以排除此物质,也同时解释了即使是健康女性在生理期时生殖部位也会有鱼腥味。

生殖部位的味道是最靠近我们的气味之一。您融会贯通的话会发现,其实有味道是正常的,特别是在一天结束之后。如果您明白我们意思的话,它们通常闻起来不会太糟糕。难闻的气味可能是感染的征兆,最好该去看个医生。倘若检查过后您的气味问题不是感染造成,或许穿着宽松的裤子或裙子、时常更换贴身衣物以及多多注意个人卫生(但也不要太过头!)会比较好。

明白的话,分泌物与性器官的状态密不可分,所以一点也不意外,只要小小地观察,就能知道私密处的状况。分泌物可能导致感染及阴道菌落(vaginal flora)的不平衡,同时在一般生理周期也会发生大幅变化的情况。

换句话说,了解正常分泌物的气味、颜色与浓稠度为何重要?一天当中有些人分泌得少而有些人大量分泌以至必须更换内裤,两者都是正常的。最重要的在于明白什么样的状态对您而言才正常。这样一来,您不止在身体出状况或何时该看医生有初步的认知,还能意会到自己处于生理周期的哪个阶段。为了给予您一点帮助,我们列出了分泌物指南。





需要向医生求诊的分泌物



‧大量水状分泌物呈灰白色且带有鱼腥味⸺可能是阴道菌落不平衡导致的细菌性阴道炎(bacterial vaginosis)。

‧黏稠、白色块状、气味正常的分泌物有可能为阴道酵母菌感染(yeast infection)的征兆。

‧分泌物流量增加,普遍呈现黄白色⸺可能是披衣菌(chlamydia)、霉浆菌(mycoplasma)或淋病(gonorrhoea)感染,而淋病比前两项症状更常产生黄绿色分泌物。

‧・大量黄绿色、呈水状发泡并带有恶臭的分泌物⸺可能为滴虫病(trichomoniasis),在挪威相当罕见5 。

‧大量白色、颗粒状、气味正常的分泌物⸺可能是乳酸杆菌过度分泌,会特别容易有搔痒感及鼠蹊部疼痛的状况。

‧・非生理期却带血的分泌物(任何呈现棕色、粉色、暗沉或是鲜红色小点的分泌物)可能为性传染或子宫颈不正常细胞导致的状况。发生任何不明原因的出血您应该去找医生咨询。





不会造成疑虑的正常分泌物变化



‧黏状蛋白,可透过手指展开的分泌物⸺可能是排卵期将近。

‧大量与平常相同之气味、颜色和黏稠度的分泌物⸺可能为荷尔蒙避孕或怀孕所造成的情况。





* * *



5	阴道滴虫(trichomonas vaginalis)为造成滴虫病的小型寄生虫。在挪威是罕见的疾病,但却是全球最常见性传播疾病之一。有些人的阴户与阴道会伴随强烈瘙痒感与恶臭,碰水时有灼热感,而有些人并不会注意到这些症状。滴虫感染不会造成危险,可使用抗生素甲硝唑(metronidazole)予以治疗。





生理期⸺过度失血也不会死亡的状态



基本上每个月会有一次生理期。有时候会觉得痛,有时候突然报到让您感到尴尬,不过大多时候都是顺利的。虽然每个月阴道没有流血我们也能生活得不错,但生理周期的出现在特别状况下可以说是极大的安慰:呼!您这次没有怀孕。

月经在我们生命中占了极大的分量。如果您每个月都有五天的生理期,代表每年有六十天的时间在流血。如果您的月经周期维持四十年的话,表示一生有两千四百天的生理期—等于六年半的时间!您猜到的话,现在我们应该更深入谈论这个话题,尤其包含一堆难受的挑战,例如经前症候群(pre-menstrual syndrome,PMS,我们稍后会提到)、尴尬的情况及激烈的疼痛。

虽然相较活在人类发明卫生棉条、月经杯、卫生棉和止痛药之前的姐妹们,挪威现代女性面临的难题不大,但这些挑战还真够糟的。当年的挪威女性用钩针或棒针编织棉质卫生巾,甚至每次使用完毕后必须下水煮沸并且晾干。在世界各地,月经仍然是个极大的挑战。当您知道对于世界的某处来说,有人因每个月流血而放弃上学,或没有干净、可替换的用品所以使用脏布而得到感染的情况视为理所当然,经前症候群也变得无足轻重了。月经在全球女性真正的平等上是经常被忽视的阻碍。下次买棉条的时候您可以好好思考一下。

让我们将主题聚焦在流血本身,大多数人知道这与生育有关。月经代表您身体内部拥有一定的周期而且有能力孕育孩子。不过真正流出来的东西是什么,伤口又在哪里呢?为什么经血的颜色从棕色到红色都有,又为何是块状的呢?

经血来自于接收受精卵失败因而等待下次着床的子宫。子宫借由增厚内膜或黏膜来准备怀孕,意即子宫内壁或内层。受精卵会附着在内层上,提供母亲血液作为细小成长胚胎的养分。当没有受精卵着床时,身体也就不需要厚层黏膜,因此全部都流血排出,而这也是经血黏稠的原因。有些血栓只是排出黏膜的碎屑,流出来的并非从子宫而来的纯血。

许多女性在发现经血的颜色或浓稠度与上次不同时纷纷感到担心,但不管血液是鲜红、棕色或有血块都很正常。因为血液有凝聚作用,经血颜色或浓稠度的变化在每个周期或同一周期间有所不同,代表血液离开我们血管后会改变颜色与浓稠度。刚流出的鲜血是红色、液状的。当经血呈亮红液状时,表示血液快速从子宫流出,尚未凝结。同样地,棕色、带有块状物的血液就稍微放置了一段时间。倘若您有大量出血过,会发现血液通常看起来很新鲜,是因为子宫很容易将血液挤出。如果您是轻微流血,血液可能还留在子宫,同时凝结了一些,不过最后仍会全部排出。不代表血液会累积在您身体里。

月经并非不卫生或是危险,它包含了血液及黏膜,所以您的感受由您决定。如果您想,没有人能够阻挠您在生理期时做爱,但一定要记得做好预防措施。目的并不是要避免怀孕或性传染疾病的感染,而是因为您正在流血。

既然您对月经有概念了,您可能也知道我们为什么在怀孕的时候不会流血。因为经血包含未来受精卵的新家,子宫内黏膜。在怀孕的时候,我们当然会想留住它,也因此胚胎不会就此流出。这是由一种名为黄体素(progesterone)的荷尔蒙来帮助黏膜着床,您很快会读到这个部分。

等等,虽然您学到了关于月经的一切,但是我们真的需要它吗?也许您注意到了,其他雌性动物不会每个月流血。许多人认为雌狗在发情期有月经,不过出血位置却不一样。

和我们从子宫出血的方式不同,她们在排卵时是从阴道出血,接着受孕。事实上月经非常罕见,只有我们与两种人猿及其他稀有生物⸺包括蝙蝠才有。

也就是说,繁衍后代不一定要有月经。

实在是毫无意义,为什么我们每个月还要特地花力气制造新的子宫内膜,最后眼看它化成血流掉呢?达尔文你说说看啊。

您大概听过演化与物竞天择说。在历代物种的演变里,具有优势基因特征的个体会成功转移他们的基因给后代,最后这些特征在后世中占有主要地位,这就是千年以来人类和动物的发展。不像大多哺乳类动物,人类拥有月经;这代表月经本身是对我们有利的特征吗?生物学家迪娜‧艾玛拉(Deena Emera)却不这么认为。她的理论里,月经是非适应结果(non-adaptive consequence),而并不具适应优势(adaptive advantage)。

艾玛拉认为月经是和我们日常生活里尚未意识之适应优势有关的结果:我们称为自发性黏膜生长(spontaneous mucous membrane growth)6 。您现在也知道,子宫内膜增生,是为了提供受精卵养分及住所。没有月经的动物,黏膜只会在受精卵形成后生长。换句话说,母体收到受精卵的求救信号后建立子宫内膜让受精卵入住。然而,人类却有点不同。在我们身体里,即使没有受精卵,黏膜仍每个月自行生长,这成为我们的优势。

因为维持我们不必要的多余组织必须耗费能量,所以当人类子宫内膜与其他月经物质没有接收到受精卵后便会排出。于是我们有了月经,或者可以说是黏膜自发生长的结果。而每个月不用排出多余组织的动物也就没有月经,牠们只有需要的时候才会制造子宫内膜。

那么自发性黏膜增生的优势为何呢?艾玛拉提出的理论是,母亲与胎儿并不只是分享好处而已。事实上,透过我们演化的过程来看,胎儿发展了从母体身上取得更多资源的特征,便能想像母亲与胎儿处于长期竞争关系。从母亲发展的角度来看,这些特质让她得以为了生存保留所需要的资源。

在这个环境下,艾玛拉提出两个论点解释为什么自发性黏膜生长对人类有益。

首先,子宫内膜的生长保护母亲抵抗外来侵入的胎儿;没错,您猜到了:和没有月经的物种相比,有经期的物种其胎儿非常具有侵略性。这些胎儿肆无忌惮,他们发狂似地生长,像寄生虫一样闯入母体只为了将能量与养分弄到手。

人类已经事先制造黏膜,因此看起来有强大保护力能够对付入侵的胎儿。

您可以想成是母亲为了更有效控制胎儿可以取得与替自己所保留的资源而准备好的防护所。

另个论点为母亲能够意识到受精卵附着在完成黏膜上的状态。您会在这本书后面读到更多所有从受精卵发展至婴儿的内容。因为基因上的问题,许多胎儿在初期会出现自发性流产的情形。浪费力气抚育不能长大的胎儿对母亲而言简直愚蠢至极。如果她能透过子宫内膜察觉,就能借由早期排出有缺陷的胎儿来保留珍贵的力气。

月经的优势不在于它本身,而是自发性黏膜生长造成的结果。黏膜的生长,不是我们每个月需要的东西,其实是为了怀孕所建立的地方。许多人认为月经很重要,有月经是健康的,但事实并非如此。我们扣除每个月黏膜生长的理由后,就不再需要月经。它只是结果,而且经血本身不代表健康,只是每个月所流失的血液罢了。

记者隆恩‧法兰克(Lone Frank)在文章中就艾玛拉的研究提出现代人和我们万年前祖先的月经发展相去甚远。现代女性一生约有五百次生理周期,远古时期的女性可能只有一百次。为什么呢?因为她们大部分的时间都在怀孕或哺乳,也缺乏可靠的避孕方式。

比起不生两个以上的孩子,借由避孕来阻隔月经对我们而言正常不过。我们现在能够选择生孩子与否,也能控制生育数。对现代女性而言月经并没有内涵的生理价值。

关于月经有许多的迷思,尤其有很多月经决定您能做或不能做什么事的话题。

月经对您及日常生活来说到底是什么呢?有您应该要避开的事情吗?举例来说,您的瑜伽老师在您流最多血的时候建议您不要做头倒式是对的吗?

我们向瑜伽老师询问了为何您不应该在生理期做头倒式动作。"血液流回肚子是不好的。"老师这么回答,就某方面来说他是对的。生理期有少量的经血从输卵管流回肚子似乎是不正常的情况。许多紧张的外科医师替生理期女性开刀时会发现腹部有血却没有任何伤口。经血流到腹部其实不太危险,因为您的身体很快就能处理一切。

许多人相信特定活动,如倒立,会让您流更多的血,这也是不正确的。月经是子宫内膜的排出物,倒立、性行为或到处跑都不会让子宫内膜增加或减少。

生理期内,唯一流出来的只有子宫内壁,而内壁的厚度与最后排出的量可能有时不一样。

除非因为一些疼痛对您带来困扰的特定活动,其实在生理期您想做什么都可以。倒立、跑马拉松、游泳或发生性行为—由您决定,甚至有些女性认为运动能够舒缓经痛。

但是发生性行为真的不会流更多血吗?当我们在奥斯陆的咖啡店写到这个章节时,想起女性朋友对我们诉说充满既戏剧又痛苦难忘,让她们措手不及的流血故事。她们在流最多血的一次生理期中躺在男伴的怀里。一个女生,躺在血泊中,被受到惊吓,不知道她是死是活的爱人叫醒。哈啰!哈啰啰啰啰!?我应该叫救护车吗?他的家里发生惨案⸺连同他的床单,本来还是白色的。而另一个女生,则是在过程当中突然流血,让人想起屠宰场或是一九七二年的砍杀电影场景。到底发生什么事了?我们必须探讨一下。

这些疯狂流血的现象无从解释,不过您对身体运作方式稍微了解一点的话,下面的理论或许会合理一点。

第一个理论是我们所谓的痉挛,子宫的肌肉收缩会释出经血,但痉挛除了月经之外还有可能是其他因素造成。有时候子宫痉挛也不是坏事,我们在这里指的是性高潮,整个性器官包含子宫在内有节奏性地收缩。高潮可能是造成月经涌出的触发点。

第二个理论是荷尔蒙。当发生性关系,身体会释放一种催产素(oxytocin)的愉悦荷尔蒙。催产素在身体许多过程中担任重要的角色,其中,它和触发女性生产有关。催产素刺激收缩,所以是非常重要的东西。如果高潮不足,催产素也能使子宫收缩,因而出血。

第三个可能的解释为有些经血累积在阴道内,只会在性行为发生时从"水门"泄洪。也许您还记得,阴道有许多皱褶,所以能够将血液聚集。此外,当您放松的时候,阴道并非敞开而是呈现前后壁互相挤压的状态。

另一个自一九七〇年代起便流传的有趣迷思是长期生活在同一个屋檐下的女性月经会同步一致。我们身体可能会有心电感应的力量让我们对痉挛及对巧克力的渴望产生共鸣。这是哈佛心理学家研究同住宿舍的美国女大学生月经周期的情况所得到的论点。演化生物学者抓住这点不放,认为女性同时来月经与排卵有一个好处:男性不会引起劈腿的念头,反而形成稳定的情侣关系。显然有80%的女性相信月经同步的迷思。

无论听起来多可爱,有更多研究显示我们确实如此。研究指出,住在宿舍的女同性伴侣、中国女子及住在"月经小屋"的西非女性并无共时性的结果。虽然我们看似同时发生,但这是因为女性的生理周期长度变化相当大。如果您和挚友同时来月经的话,大概只是巧合,必须遗憾地说,不是您们之间有特别的连接。





* * *



6	这是艾玛拉在研究中对自发蜕膜化(spontaneous decidualisation)一词的简易说法,蜕膜化的过程并非只有黏膜增生而已。





不要在沙发上流血!关于卫生棉、棉条与月经杯



只要您还在使用卫生产品,每个月的生理期就不会妨碍您喜欢或想要做的事情。如果您有用东西抑制血流的话,经血流在朋友沙发上的风险也会明显降低。

最常见的卫生用品属于抛弃式的卫生棉与棉条。近年来,月经杯在众多女性的喜好度中向前攀升。其中的考量有很多,包含经济、环境与舒适度,完全由您决定,这是关系着您喜好及状况的问题。

自从我们悄悄离开文明的发源地后,女性至今已使用过不同种类的卫生棉。有个关于卫生棉古老(又有趣)的说法出自于人类所知第一名女数学家的故事。公元四百年的希腊人希帕提亚(Hypatia),据说厌恶咄咄逼人的追求者因而朝向对方丢掷沾血的布巾,最后有没有驱赶成功无从得知。

现代卫生棉的底部有自黏贴条,以便附着在您的内裤上并吸收从阴道流出的经血。有许多不同尺寸的卫生棉,从小型的护垫到大型、柔软的夜用卫生棉。和棉条相比,卫生棉的好处在于不会有阴道细菌滋生的风险。建议使用卫生棉是因为阴道受到感染的风险特别高—也就是子宫口张得更开,例如使用避孕环、堕胎或是生产后的情形更容易让细菌进入。

棉条是生理期里放进阴道的小型、子弹状的吸收材质。使用阴道内月经防护产品的好处是可以更轻松运动,尤其是游泳的时候。虽然这个从法语而来的字,"tampon",有塞子的意思,但它不能让血液保留在阴道里,而是吸收经血将它们聚集起来。棉条绝非是最新的发明,但它的材质也并非总是用单独的塑料包装。想想以前古埃及女性使用软纸莎草放入阴道当作月经防护品。

现在,棉条分成有无导管的种类还有不同的尺寸,您可以根据流量选择大小。然而使用大型棉条来免除经常更换的麻烦是毫无意义的:棉条应该要定期更换,一般建议的时间为三到八小时之间。避免细菌滋生,更换棉条前彻底洗手很重要。

多年来我们听过众多棉条的故事,其中经典的一则与放入两个棉条或将棉条"遗忘"在阴道内有关。救命啊,许多人会这么想,棉条要消失在身体了。棉条自行往胃里移动的想法就和不当配戴隐形眼镜的话会流窜至大脑的迷思一样。就你现在所知,阴道是近似封闭的管道。经过子宫颈及子宫的通道非常狭窄,即使最小的棉条也不可能进入子宫。通往子宫的子宫颈不是开放式气室,所以任何东西无法进入阴道并消失在胃里。奇怪的是,阴道内的裂缝是能够藏住东西的,也因此棉条附上绳子好让你再次将它拉出来。

如果你担心棉条会在阴道里不见的话,你可以试着将它推出。就像要大便一样蹲下然后试着挤出,用你的手指找到它的位置。因为阴道不到七公分长,所以通常都能自行掏出棉条。如果没办法找到的话,你需要去看家医科,越快越好。任何进入阴道的东西都必须要出来才行。你认为你是第一个因为这个问题去找女医生的话,那你可以放心了。

月经杯是专门聚集经血而非吸收的卫生产品,材质为柔软硅胶量杯,将它折叠并放入阴道中。置入后,杯子自动展开后口端朝向子宫颈底部,让血液聚集于此。杯缘与阴道壁紧密贴合,维持在一定位置。由于月经杯并非抛弃式产品,所以卫生格外重要。必须清空、冲洗,最好每十二小时至少以温和阴道清洗剂冲洗一次。每次生理期,建议热水煮沸月经杯来消灭所有细菌。

月经杯的主要优点在于你可以使用比棉条更久的时间。同时,因为月经杯放置在阴道内,所以非常适合在运动及游泳时使用。你可以用好多年,最多十年,长期下来是个便宜又环保的选择。而且一个月经杯能够取代最后落入垃圾掩埋场的好几千个棉条与卫生棉。

此外,你一定看过使用棉条的一些警语。每个棉条包装盒内都有一本小册子提醒可能发生的可怕疾病⸺中毒性休克症候群(toxic shock syndrome,TSS),又称棉条病。使用棉条真的会得到严重的疾病吗?

中毒性休克症候群是一种攻击整个身体的细菌感染疾病。使用棉条就是得到TSS的风险之一,因为阴道内温暖、吸血的棉条是细菌的温床。置入棉条时忽略自身清洁,同时久未取出,你就有可能遭遇如此不幸。这也是棉条放置时间最好不要超过八小时的原因。细菌孳生并进入身体需要时间,如果你忘记棉条还在阴道内的话,这样的情况会大幅提升。妥善使用棉条不会造成危险。

倘若得到TSS,你会知道浑身不对劲。可能会有高烧、起疹、喉咙痛、呕吐、腹泻及晕眩的症状,你会觉得非常难受。顺带一提,你应该时常注意任何严重及突发的病症。你觉得自己得到TSS的话,必须赶紧去看医生,因为感染久而久之变得更严重,而且会快速蔓延,最糟的状况还可能致命。

不要因为TSS与棉条有关就认为使用棉条很危险,TSS是严重的疾病但是却非常少见。自从高吸收型的棉条下架后,使用棉条得到TSS的比例大幅下降。现在,只有一半的案例与生理期有关,严重创伤及手术后的感染也都也可能引发TSS。换言之,不使用棉条也是有可能造成TSS,同时男性也有可能得到,所以"棉条病"不是最好的代称。

关于TSS与月经杯,由于这些主题的研究少之又少所以我们懂得并不多,相对来说月经杯是个新现象。目前为止,全球最少只有一个TSS案例与月经杯有关。因此从TSS角度来看,我们还不知道月经杯是否比棉条好。无论如何,最好还是注意卫生清洁。





经前症候群⸺疼痛与谋杀症候群




"怎么了?⸺月经来了吗?"这是典型的询问手法。有时候我们难以断言女性是因为软弱、荷尔蒙分泌过多或是满腹牢骚而无视我们。"月经手法"不只是歧视女性的批判字眼,以完全生理学的角度来看同样也是错误的。为了教育大众,这些错误必须厘清干净。人们如果坚持使用糟糕操控手法对待我们,他们也只有在生理学上的观念正确。你是否留意到自己身体因为月经周期而受到心理方面影响最大的时候并不是流血的那几天,这个症状实际上是在月经开始前。没错,正是广为人知的⸺经前症候群(PMS)。

PMS或经前症候群(意思等同经前紧张症,premenstrual tension)也许烦人但大致上我们可以与它共存。虽然会造成一些不算太严重的问题,但PMS不应该是无视女性的合理原因。女性并非满腹牢骚、软弱或是"荷尔蒙分泌过多",因为我们有月经周期。无论你认定自己的性别为何,一切可能令人觉得震惊又不专业,不过我们没有要争论,毕竟这完全是不同回事。

PMS是所有可能在经期出现的疾病总称,它几乎涵盖所有生理及心理上的症状:疼痛、易怒、忧郁、胀气、情绪波动、哭泣、焦虑与冒痘痘,等多种症状。同时人们现有的病症会更加严重,例如偏头痛、癫痫或气喘。这些问题出现在排卵及月经阶段的中间,我们称为经前或黄体期(luteal phase)。当月经来了,在流血的第一天因压力释放,同时症状也消失得无影无踪。

没有特别的检查方式能够诊断PMS。举例来说,医生在妇科检查时不会告诉您有PMS。这让诊断变得有点困难。经前的一些症状不用接受诊断,光凭您遇到的状况就能知道您有没有PMS。大概所有女性在经前有过一些症状,并不是因为您有女性的身体才需要诊断。85\%至95\%的女性在生理期前有轻微类似PMS的症状。

为了诊断出PMS,那么症状必须相当严重,严重到生理或心理上妨碍到您的日常生活。当然,这取决于个人对妨碍程度的认定。您可以预料一些症状,但能预料的仍然有限。有些女性因为这些症状完全丧失行为能力,本来不该是这样的。除了严重的症状,还要有一定的周期,意即您每个月通常都会有这些症状。再者,PMS一定有固定的起始及终止时间:必须是经前开始,在月经来的时候结束。大约有20\%至30\%女性的症状属于轻微或是一般的PMS。

情况最严重的女性通常会诊断为另一个比PMS还要严苛的病症,虽然两者症状相同。这个病症称为经前不悦症(premenstrual dysphoric disorder,PMDD),此类的症状从完全脱离能够控制的范围变为无法忍受的状态,有3\%到8\%的女性受到此病症的影响。此外还有另一种症状叫作经前忧郁症,有些女性在每次生理期前承受抱有自杀想法的严重忧郁,实在是非常危险。这三个症状的内容有些相互重叠。

虽然月经会从青春期持续到更年期,但PMS不会存在那么久。它的症状通常从二十多岁开始,所以大多数人在问题逐渐出现前有好几年的时间不会经历PMS的状况。症状变得愈来愈严重也很正常,许多人到三、四十岁前不会特别寻求医疗上的帮助。

我们不知道造成PMS的原因为何,从高度敏感、身体荷尔蒙变化到神经学甚至是文化的说法,在不同理论中都有提出。所有女性在生理周期时会受到荷尔蒙变化的影响,至于为什么有些人会有PMS或PMDD,而其他人却没有这些症状却无从得知,或许我们以后会找到原因。

人们大多不用治疗PMS,况且最重要的则是避免透过医学方式治疗因自然荷尔蒙变化所造成的疾病。我们稍早之前提过,多数人的疾病并没有生命危险。一般来说,您可以和PMS共存,面对不舒服的症状时也有替代的解决方案可以选择。

遇到问题严重的人,其疗法会针对个人状况有相当大的变化。比起严重疼痛,倘若您有忧郁或焦虑的状况,可以采用不同的疗法。对一些人来说雌激素避孕能够帮助她们完全抑制生理期,其他饱受心理疾病所苦的女性使用抗忧郁药物也是不错的方式,容易经痛的族群则可以用止痛药。

回到那些透过歧视性言论与女性说话的人们。无论你相信什么,有PMS的女性在经期前会丧失理智或无法作出理性回复都不是正确观念。如果面对生理期的女性你决定敷衍回应,不要说:"怎么了?—月经来了吗?"或是"怎么了?再过几天月经要来吗?"这听起来不是很好,但重要的是如果你坚持羞辱别人的话,先正确理解这方面的观念。





永恒之轮—荷尔蒙与生理周期

每个月,大多具有生育力的女性会经历身体内部、荷尔蒙影响的周期,我们说的正是月经周期。大部分的人知道得很少:在某个时间或是另一颗卵子成熟时,我们如果在正确(或错误)的时间发生性行为就有可能怀孕,而有月经有来就代表我们没有怀孕。

我们真的需要知道更多吗?我们看过许多医学院学生读到月经周期的章节就突然把书阖起,所以为什么您应该要知道呢?首先,这对您来说很实用;第二,这是非常惊奇的事情;第三,我们保证把这些内容弄得比您看过的其他课本都更容易去理解。

如果我们都能了解更多细小信号物质—荷尔蒙在月经周期是如何影响我们的话,便可以更容易知道所有和女性日常生活有关的事物。我们常会收到这些问题:荷尔蒙避孕是怎么运作的呢?什么是受孕窗口(fertile window),而且它什么时候会出现呢?控制我们月经的是什么?众多女性疾病又是如何运作的呢?





荷尔蒙—驾驭我们的物质

我们在内生殖器官章节内提到的卵巢,它所分泌的雌激素与黄体素为女性的性荷尔蒙,在这一章该是讲得更详细的时候了。

雌激素最近蒙受不公正的恶评,我们听到的都是血栓、情绪起伏、乳癌及其他可怕症状的风险。然而雌激素是个奇妙的荷尔蒙,它主要掌管和一般女性特征有关的部位,胸部、屁股、臀部⸺都和雌激素有关。雌激素维持阴道壁的湿润及厚度,让性行为变得愉悦顺畅,也能够让子宫为怀孕做准备,同时也能抑制汗毛及斑点生长。事实上,跨性别女性会使用雌激素疗法将原本男性身体的脂肪分布改为女性的身体。少了男性的大肚子,多了乳房与屁股。小小的荷尔蒙能做到这样实在很惊人。

如果您有语感的话,大概可以明白黄体素(progesterone)是什么东西。"pro"代表为了,而"gestation"表示怀孕,所以意思是"为了怀孕"。每个月当身体准备接收受精卵时,我们需要大量的黄体素。黄体素阻挡子宫收缩及推出潜在的受精卵。再者,它将子宫内膜打造成适合居住的地方,以大量血液和腺体分泌的黏膜给予我们未来的后代营养。

其他两个荷尔蒙则控制我们的月经周期,它们来自大脑豌豆般的构造,形状像阴囊一样,称为脑垂体(pituitary gland)。身为性作家,我们所见之处都是性器官。

大脑的两个生殖系统荷尔蒙分别叫作滤泡刺激素(follicle-stimulating hormone,FSH)及黄体成长激素(luteinising hormone,LH)。简单来说,滤泡刺激素与卵子成熟有关。卵子其实存在于一群细胞中,称为滤泡(follicle),在滤泡刺激素里面出现过这个词。黄体成长激素则是以触发排卵著称。男性大脑其实也制造一样的荷尔蒙,不过由于是在女性身体上所展现的功能于是以此命名。因为在世界医学上非常少见,所以我们觉得特别新颖。

到目前为止都还不错,现在您必须认识荷尔蒙这个大明星,接下来要来看它的循环。





月经周期—不断循环的28天!

为了理解月经周期,画个圆形的时间表会更有帮助。虽然每个女性周期的长度与下次月经的间隔有所不同,为了方便我们就以28天当作范本,而28天也刚好可以分成四周。不过,正常周期的幅度是介于23~35天之间。





圆状图的顶端为新周期的起始,同时也是终止点。在这个地方同时标示象征周期重新来过的0,与表示经期结束的第28天及上一周期的结束点。一个周期的开始永远和另一周期的结束同步,您的月经周期真的是个永恒之轮!

许多人会觉得难以理解,经期的开始跟结束怎么会在同时发生呢?倘若将月经周期与我们非常熟悉的物品相比的话会易于理解。它和时钟的运转完全一模一样,我们就这样一天接着一天度过。

当时钟来到午夜,数字时钟上一样显示24:00,用来表示一天最后一个小时,同时00:00则表示新的开始。时钟从一天的开始运转到隔天,在午夜时分之际,您正身处同一天当中。两天之中并没有间隙,而月经周期也是如此。

新周期的开始会特别容易察觉,因为那是您开始流血的第一天。通常流血的天数会持续一个礼拜,正好是周期的前7天。

为了彻底厘清,月经周期通常分为两个阶段。当您开始新的月经周期时,您正处于所谓的滤泡期(follicular phase),这时候容纳卵子的滤泡成熟并准备排卵。大约到了第14天,周期的末端开始排卵,来到第二阶段,我们称为黄体期(luteal phase)。整个周期已经过了一半。而接下来直到第28天的两周内不会有其他明显的现象。28天后,如您所知,我们又回到原点,新的周期正式开始。

现在我们稍微把事情变得复杂一点并想象一下您的周期有30天。这样的话,排卵日会差不多出现在第16天。什么?为什么不是第15天呢?您可能会这么问。毕竟30除以2是15。答案是排卵期与下个经期的第一天差不多经过14天,身体在这个时候需要确认是否要开始怀孕。周期多于或少于28天,主要是影响排卵期前的月经来潮长度。如果您的周期很短,虽然您不会在月经的第一天排卵,但排卵和经血同时发生的状况确实有可能发生。您的周期不规律的话,只有在流血的第一天您才会知道您不会排卵。

我们现在有基本的概述了,接着可以开始真正有趣的环节:随着周期起舞的荷尔蒙。我们从圆形图的顶部开始。月经来潮第一阶段的第一天,也就是滤泡期。这个期间的反应不只和子宫有关,同时会在卵巢及大脑里的阴囊,也就是众所皆知的脑垂体中作用。当子宫内膜连同集希望于一身的受精卵剥落时,脑垂体开始制造滤泡刺激素(FSH)。大脑永远不会放弃,即使正在经期仍然会准备新的卵子与下次受孕的机会。回想一下,卵巢内所有的卵子称为滤泡,受到FSH刺激后才会开始成熟。而成熟的滤泡正是第一阶段的滤泡期的由来。

滤泡接收从大脑来的FSH成长,接着滤泡开始制造雌激素。当滤泡逐渐成长,血液中雌激素的数量开始剧烈增加。滤泡越多,雌激素的数量也就越多。接下来,雌激素影响子宫内膜,让它增厚。在子宫结束流血后,再次重建内部,完全没有难过的时间。子宫是个固执的小淘气,即使每个月会受挫也绝不放弃任何接收受精卵的机会。

滤泡和子宫内膜生长时,我们来到第14天的排卵期与第二阶段的过渡期。滤泡改变了形状,变成膨胀、充满液体的气球,像一颗即将爆炸的水球。滤泡分泌许多雌激素,让雌激素达到最高峰,这正是脑垂体在等候的讯号。

脑垂体开始制造黄体成长素(LH)来回应强烈雌激素的讯号。此时的LH并非少量,而是像火箭一样突然上升。您曾经尝试怀孕的话,应该对LH的急剧上升并不陌生。排卵试纸会从您的尿液中捕捉上升的LH,所以试纸呈现阳性反应的话,就能知道LH已经开始分泌,而排卵日也即将到来。大量流动的LH靠近滤泡造成它破掉,使卵子破茧而出离开卵巢。不久后,卵子离开卵巢四处飘移到输卵管的触手前,也就是输卵管伞,快速附着并与可能在输卵管上等待的任一精子细胞结合后开始一趟旅程。我们的月经周期经过了一半,同时也确实来到了排卵期。

现在看起来是时候休息一下来回应几个我们没在国中生物课里学到的一些东西,卵子细胞。您大概还记得勇猛的精子细胞之间如史诗般的战斗或竞赛,疯狂地想成为第一个游向等待受精的被动卵子。第一点:卵子并不是站在那边等待。她没有在酒吧里紧张地来回踱步等着精子细胞。卵子是天后,和其他天后一样,她会为了华丽现身派对而迟到。您可以在怀孕的章节里可以读到更多,想要怀孕的最佳性行为时机是在排卵期前。所以卵子一点也不被动,至少和精子细胞一样积极。不像精子游向卵子那样,而是上下摆动与等待的精子接触。他们通常会等她等上好几天⋯⋯

第二点:卵子细胞之间同样也进行着史诗般的战斗,不过基于一些原因,我们不在学校课程里提及这件事。滤泡刺激素每个月并非只影响一个卵子滤泡。就您现在所知,每个月大约有一千颗滤泡开始长大与成熟,但这么多之中只有一个会破掉并释放卵子。其他卵子不会有任何与精子细胞相遇的机会接着萎缩死亡。您大概觉得一千颗滤泡不像精子一样竞争激烈—毕竟,它们是数百万个竞争者相互竞争。记得一点,男性每天制造上百万精子细胞,而女性的卵子则在出生时拥有一定的数量,然后被消耗殆尽。

为什么我们在完全没有对照真实的状况下就自然而然认为(来自女性的)卵子细胞被动,(来自男性的)精子细胞则是主动的呢?只是随口问问⋯⋯

回到月经周期。我们处于第二阶段,也就是第15~28天,或称为黄体期。由于滤泡内所有雌激素的关系,卵子释出而子宫内膜增厚得恰到好处。在第二阶段,黄体素为主要的荷尔蒙,反之在第一阶段,则是促使子宫内膜生长的雌激素。黄体素的分泌来自破掉的滤泡,而剩余的滤泡改变外形及颜色,变成一小团的黄体(corpus luteum),意即拉丁文里的黄色身体,因此以黄色来命名,有时候就是那么简单。

更早之前我们提到黄体素代表"为了怀孕",所以身体现在准备接收结合的卵子和精子细胞。黄体素避免子宫收缩及排出子宫内膜,也确保子宫内膜非常适合受精卵居住。

同一时间,脑垂体抑制分泌FSH或LH,也就是制造和成长新卵子的荷尔蒙。毕竟我们等着即将到来的受精卵,所以并不需要新的成熟卵子。黄体里的黄体素正是阻止脑垂体做这件事。

(对黄体来说)不幸的是,月经周期的第二阶段通常会以悲壮的自杀收尾,您待会就会知道。诚如我们之前说过的,黄体内的黄体素阻止脑垂体制造任何的FSH与LH,但问题是黄体需要这两种荷尔蒙才得以生存。换言之,黄体阻挠了自身的救生圈,而只有在受精卵着床的时候才能获救。因此黄体常常为了维持潜在受精卵生存的利他行为而沦落为牺牲者。没有受精现象,黄体和黄体素会逐渐消失、死亡。

黄体消失后,再也没有任何的黄体素抑制脑垂体最会做的事:分泌荷尔蒙。血液中的FSH和LH浓度再次上升,卵巢里的滤泡再次活动,准备新的成熟机会、破掉并且让卵子与被选中的精子细胞结合。少了黄体素,再也没有东西能维持厚实的子宫内膜或预防子宫收缩。我们知道接下来出来的会是什么:那就是月经。来到了流血的第一天,我们再次回到圆形图的顶端。周期已经结束,紧接着新的周期又准备开始。





什么时候才会真的怀孕?




性爱被认为是女性为了怀孕而做的事,但除此之外,仍旧存在许多不确定性。"如果她怀孕了该怎么办?"在实境节目《乐园饭店》(Paradise Hotel)的其中一集里,众人在早餐桌上热烈讨论其中两名参与者发生无防护措施的性行为。有些人坚信女方才刚来生理期所以一定没事,而其他人则认为女性在经期后最容易受孕。为了打破众人的疑问,最后由节目单位TV3电视台出资让参赛者紧急避孕来解决问题。怀孕这件事并没有那么简单。

怀孕是女性人生里的分水岭。我们的反应从惊恐、耗费许多脑力在如何避免怀孕,到由衷祈祷不会太快怀孕都有可能。最坏及最好的情况也都有可能发生,取决于我们所处于的人生阶段与在一起的对象。为有这些想法的人写出关于怀孕的章节看起来很了不起,但这一切其实非常单纯。关于如何怀孕的知识对想要避免怀孕或希望怀孕的您来说才是最佳处方,那么该怎么服用呢?

我们就从已知的事实开始。你不会因为肛交、口交或坐在有精子的马桶上(唉呃!)而怀孕。你必须透过阴道性交的方式才行,也就是在阴道里发生的性行为。接下来事情变得有点复杂。

当男性高潮时,会射出上百万条精子细胞进到女性阴道中。大多数精子在不久后死亡,它们主要经由性交后射出或游进阴道深处。只有非常少数的精子细胞才能找到子宫颈口,完全是时机的问题。

事实上,大部分的时间,子宫颈是由身体自然制造的高浓度厚实胶状黏液荷尔蒙,黄体素所封住。只有在生理期接近排卵的时候才能让黏液分解,并打开与子宫腔室之间的通道。排卵期的前几天,您可能从弹性丝状黏膜分泌物的变化注意到这个现象!这种和蛋白相似的黏膜,如果您想要尝试,可以用两只手指将它延伸至令人讶异的长度。

当排卵期接近,黄体素浓度降低,身体开始制造更多雌激素。雌激素造成子宫颈口张开,取代胶状黏液制造出水状液体,让精子细胞能够游进子宫。同样的,您可以透过水状、乳白色的分泌物来观察。这是您在排卵、最能够受孕时的情况。

现在假设你在排卵期时发生无防护措施的阴道性行为,而子宫颈为开放的状态。有一小群上百个精子细胞试着找到通往你子宫的路,它们会花上2〜7小时的时间从子宫移动到其中一条输卵管,经由子宫细微且有规律的动作帮助,加上制造波动的输卵管,让它们可以乘浪前进。它们所选的方向非常重要,因为卵巢一次只会排出一颗卵子。到达输卵管后,精子细胞稍作休息并等待可能现身的卵子⸺毕竟,你知道的,卵子很明显是整场派对的天后,她会让精子细胞等待着。虽然性交后的5〜7天仍然找得到活体精子,然而通常位于子宫或输卵管的精子只能存活约48小时。是个有耐心的家伙啊,精子!

排卵后,卵子会沿着输卵管朝向等候的精子跳下去。一个精子细胞与一个卵子在输卵管结合后受精现象就此发生,并共同创造胎儿的前体,意即受精卵(zygote)。有时候排卵期出现了两个卵子,那么您可能就有怀了双卵双胞胎的状况。通常在女性老年时期及先天遗传更容易发生,所以有些家庭会有许多组双胞胎。在罕见的案例中,也有单卵双胞胎的情况。双胞胎的形成发生在与精子结合后受精卵快速分裂为两个单独个体的时候。

受精后的第一天,受精卵依然沿着其中一条输卵管里漂流着,而接下来细胞即将开始分裂。即使如此,仍无法保证您将要怀孕。确保顺利怀孕,生长的细胞必须找到下面子宫的方向并立即附着在充满黏膜的子宫壁上。此外,身体必须透过一种荷尔蒙,人类绒毛膜促性腺激素(human chorionic gondadotropin,hCG)来接收子宫的信号告知细胞团着床完毕—和验尿怀孕的指标荷尔蒙一样。这个荷尔蒙确保我们在前面部分提到的黄体存活并继续制造黄体素。如果没有经过此过程,受精卵会于下一个经期在您尚未发觉的时候排出。

受精卵细胞附着于子宫内膜需要7〜10天的时间,只有在这个时候您才能确定怀孕。接下来九个月的庞大旅程我们选择略而不谈,毕竟市面上有许多怀孕的书籍供您阅读。

回到那对《乐园旅馆》的情侣,女方会因为在生理期当中就可能要怀孕了吗?在试著怀孕的伴侣研究当中,只有在接近排卵期的六天空窗期中进行性行为的人怀孕,也就是排卵期前五天加上排卵日当天。在排卵期前或排卵日当天发生性行为的伴侣只有30%的机会怀孕,排卵期的五天以前则是10%。

即使在排卵日前好几天进行性行为而怀孕的人也是有很多。我们稍早之前提到,精子细胞死亡前能在女性体内存活至少一周,理论上,您在排卵期前7天的经期及后一天是能够受孕的,也就是总共有8天。换句话说,我们有8天的受孕窗口(fertile window)时间。我们大多不太注意自己的排卵期,所以要知道《乐园饭店》的参与者是否在危险期的关键就要测量她的月经周期并了解为期多久才行。

我们在月经周期的部分提过,排卵期通常都在下个经期的前14天。如果妳是完全规律的28天周期,那么排卵期会在周期的中间,第14天,或是最后一次来月经后的两周。我们已经知道有8天的受孕窗口,代表妳有可能在周期的第8〜15天怀孕。

我们就假定《乐园旅馆》的参赛者有规律的28天周期和7天的经期,也就是整个周期的第1至第7天。这表示她只有一天,一天的时间在经期结束后有怀孕的机会。五天后月经来了,就有很大的机会确定怀孕。

在这种周期中,正在经期中的她发生毫无防护措施的性行为绝对不安全。然而等待下次经期的前一周,在第21〜28天发生就会安全。对于没有《乐园旅馆》宝宝,我们可以感谢紧急避孕措施或是纯粹的好运气。

如果每个周期只有8天能够受孕,那么找到安全期好像听起来很简单一样,问题是只有非常少数的女性有完全稳定的周期。妳大概注意到自己也是一样不稳定,因为妳无法事先得知这个月比正常排卵时间早或晚,所以妳必须延长受孕窗口的天数。如果排卵日落在前后两天而已,那么危险期就会延长至12天,许多女性变化的幅度比这个大了许多。此外,如果你是不喜欢在女方生理期时发生性行为的好人,等个几天你就能进行没有任何避孕的性行为而且还能有自信地认定不会怀孕。换言之,采取避孕措施永远是明智的选择。

性爱二三事





自古以来人类所只拥有的一个共通点,那就是性。无论是自己或与其他人,我们大多都曾有或将会发生性关系。少了性,地球上再也不会有人类,我们认为人类也会过著无趣的生活。性是我们能做的最原始的事情之一,即使不同形式的性(同性或异性)和其他动物都是一样的。

不同的地方在于人类是唯一对性深感羞愧的物种。碍于社会规范,行房的时候我们会遮遮掩掩的。如此不坦率的现象表示性一直笼罩在不确定性的乌云下。我们不知道其他人是怎么想的,我们不知道自身的欲望是否正常,我们也无法确信自己是否达到标准。吊诡的是,即使是两人进行的性爱,性仍然是非常寂寞的一件事。

尤其在妳性生活的最开端⸺青春期。

近年来与性有关的书籍包罗万象,年轻人花上数小时观看色情作品。色情影片被分享至各个社群媒体,青少年传送勃起阴茎与乳头的快照给彼此的伴侣。也许有人会主张我们活在有史以来最开放的性社会当中。

这个现象创造了特别的双重性。对于性冲动、欲望和身体的启发及见解,我们有自己专属的门路,知识仅仅是滑鼠点一下便唾手可得。同时,这种开放并不会使我们充满自信,正好相反。

事实上我们所呈现的是光彩的一面。我们对性的理想已经逐渐升高,同时心中的不安依旧存在著。当我们唤起性致时还不时地想要遮掩,但社会却告诉我们任何事都该开诚布公。悬殊的对比让人不知所措。我们相信这样的结果会让许多女性觉得自己的性欲太低,刺激的性爱和高潮太少。

这个社会需要对现实有新的见解。在本书这个章节里,我们想要探讨人们所谓正常的性生活。当然我们使用"正常"一词,并不代表任何偏离内容所讲的就是错误的或应该感到羞耻,只是大多数人不会做的事情而已。性有千种类型,只有妳知道适合自己的方式。我们希望在人们正常看待性的方面能够尽一份心力并提供一些方法让妳找到满意又舒适的性生活。





第一次性行为




人生里总是有几个极为神秘的经验,例如:做那档事,第一次的性行为。妳对自己或伴侣表现的期许,可能像天一样高,发生何事实在难以想像。

结果,造成有些人在初体验时对自己或伴侣感到失望。妳没有高潮吗?从妳看过的姿势进入会不会很难?妳男朋友的阴茎十秒后就软掉了吗?她没碰到你的阴蒂吗?

勇敢一点!性和人生中的其他事物一样。妳没有练习自然不会在行,而你的伴侣也是。必须要记住,第一次不会如此完美,如果妳降低一点期望的话,可能还是会有不错的经验,毕竟人都有第一次。我们收集了一些资讯尽可能让第一次体验变得更美好。

我们随著二〇〇四年的电影《赤裸青春》(Bare Bea/Only Bea),看著一群奥斯陆高中一年级的朋友。小碧是里面唯一一个还没有性经验的人。她们团体的一项仪式,是在成为破处的一分子后,要去当地的糕饼店吃一块杏仁蛋糕。现年十六岁又九个月的小碧,觉得所有的一切取决于她是否能够上床。糕饼店橱窗里的杏仁蛋糕正在呼唤著她。

小碧并不是第一个认为"其他人都做过了"而将自己赶上床去的女性。当这些念头迸出时,拥有一些相关知识是非常有用的。

挪威女性初次性经验的平均年龄约为十七岁,但只是平均数值,并非上限。有些人开始的早,也有些人晚了一点。事实上,只有20\%的年轻人在十六岁以前开始性行为。所以有四分之五的人在念高中一年级时还没有过性经验。换句话说,小碧根本不急著品尝那块蛋糕。

虽然在平均年龄经历初次性行为也许不错,但请不要忘记,第一次与妳及妳的伴侣关系密切,妳应该在双方准备好的时候才能开始。当妳感受到欲望(脑海里浮现欲望)和受到吸引或是存在于身体里的性欲被激起时才算万事俱备。有时候,妳的想法跟身体不一定同步,这样的话,再等一下也许是不错的选择。我们被激起性欲的时间、谁让我们有这样的感觉因人而异。有些人在国中就蓄势待发,部分则是在高中时期,有些人到了二十或三十岁甚至更年长一些才整装完毕。

没有一定的标准,不过许多挪威人的第一次是和同年龄的对象发生。有些人和女友或男友;有些人是透过一夜情;也有人是和同伴或女性友人进行性关系。至于地点,有些人是在房间,其他人的话,则是在音乐节的流动厕所后面。只要彼此是真心参与,这一切就没有所谓的对与错。

只要记得小荳蔻法则7 :妳和伴侣现在可能性欲高涨并迫切地想要做爱,但是最佳的场合与时机要以不去打扰他人为主才是。例如,搭飞机时坐在正在做爱的情侣隔壁实在很糟。曾经在飞往纽约的航班上经历过"机震"的艾伦可以作证。这对情侣鹦鹉很明显来自克里斯提安桑(Kristiansand,挪威南方的一个小镇),却故意不讲英语或挪威语,实在让人受不了。尊重一下好吗?

许多人非常在意是否能说出有过性经验,以及处女所代表的意义。举例来说,进行一些性行为后还可能是处女吗?如果肛交过还会是处女吗?那么口交或指交呢?真正的性行为是什么?我们还没有答案,但是我们认为将焦点放在贴上标签和性的本质相去甚远。性没有对错,无论如何,也没有"真正"或"不太真实"的性。只有你才能定义自己的性生活。第一次性经验可以有多种形式,因为性包含了口交、指交、阴道性交与肛交。少了传统阴道性交你也可以有很棒的性经验。毕竟,认为和男性进行阴道性交之前的女同性恋者都还是处女这个想法实在非常荒谬。

现在的孩子大多对性一知半解。这不应该是性教育该负起的责任,因为他们几乎都透过色情片的形式看过性爱画面。尽管(或因为?)如此,许多人担心他们在第一次发生前是否完美。

第一次性经验,妳可以预料会发生一连串的状况。无论妳做了什么,都不会像色情电影一样。如同其他电影,有些色情片运用特效让原本所做的事变得不一样,所以在现实生活中妳不会做出所有在色情片里看到的事情,即使色情片的场景是根据一些真实的内容改编。这有点像是哈比人电影:真实世界里有许多座山,却不代表有龙住在里面。即使有,也不会是班尼狄克‧康伯拜区(Benedict Cumberbatch)的声音。

同时也不要忘了色情演员应该要被当作极限运动员才是。他们早就做过所有事情,名副其实。知名滑雪选手琳赛‧沃恩(Lindsey Vonn)从高山滑降看似轻松,但如果妳是初次站上滑雪板并试著像她一样,妳有可能会伤到脖子。

不要期待自己能够有知名色情演员史托雅(Stoya)一样的表现。初次尝试的妳也无法做到欲经(Kama Sutra)里的高难度姿势。妳大概也不能完全掌握一切,这没有关系:妳不需要为了美好的性经验而去做一堆事情。第一次会非常笨拙,但这就是原本该有的模样,而且也是魅力的一种。妳大概会觉得碍手碍脚,不过借由练习会愈来愈好的。

不只降低对自身表现的期许很重要,也请记得不要对你的伴侣太过严苛。第一次做爱的时候妳的男伴或女伴不会知道你喜欢什么,至少他们也和妳一样紧张。无论如何,结束后最好沟通一下,来个事后检讨。哪些是有用的?会有下一次吗?如果有,妳应该做什么改变呢?





我该如何达阵呢?




我们只有写到性的包罗万象,却忽略另一群人并过分聚焦于阴道性交上。毕竟,性不单是男女之间的事,虽然在我们异性恋社会当中可能很常见。在挪威,约有十分之一的女性与同性别的对象有过性经验。不过我们仍然会多提一些第一次阴道性行为的部分。并非因为它是唯一进行的方式,而是因为我们对于这方面的疑问是最多的。

在阴道性交之前,询问自己以下问题的女生数量令人难以置信。我会流血吗?会痛吗?许多女性担心自己的阴道太紧。我该如何把那东西放进去呢?我甚至无法放棉条啊!

把大如阴茎的东西放进阴道看起来很吓人,但是阴道还有很多空间。你的阴道十分具有弹性,在你激起性欲时能够四处扩张。很多人认为尚未有过性行为的女性阴道会比有经验的人更紧实。你大概有听说做过愈多次你的阴道会变得愈来愈松弛,这不是事实。

阴道是有力的肌肉管腔,你可以控制它的紧实度。无论有多少阴茎或人造假阴茎在阴道内仍然能调节运作。如果你确实放松,会很容易让阴茎滑入,但如果你绷紧身子,可能会让任何东西不得其门而入。就算你有过好多次性经验,仍然可以紧缩阴道让它变窄。性爱时积极使用阴道肌肉的话,可以调节阴道与阴茎之间的摩擦。试试看就知道了!

很多女生在初体验前会紧张,受到期许的压力,这不意外。有点紧张是完全正常的,太过紧张的话,可能会造成一个不愉快的经验。你如果太紧张,会容易下意识紧缩阴道肌肉,让所有东西变得难以进去,甚至会有点痛。

当女性性欲高涨时,生殖器官会分泌更多液体,这个黏液是身体自然的润滑剂8 。倘若妳非常焦虑,会没办法提起性趣让下体湿润。就算妳真的决定要做爱也有可能发生这种情形。就某种程度来说,紧张感会阻止身体随心所欲。

如果太干燥或是不由自主紧缩阴道,很容易造成阴道壁撕裂并流血。虽然不危险,却会有不愉快及刺痛感。第一次的要点就是在于放轻松。花点时间在接吻及前戏,这样会让你的肌肉更容易放松。给自己一点时间勾起兴致,你才会分泌更多液体。

就算想要做爱,放松了,在前戏上多花了一点时间,不管怎么试,有些女生的下体就是不会变湿。而且,也有些女性没有性欲高涨仍然湿润。大脑和生殖器并不是一直都有关联。好在除了自然产出的阴道黏液外还有其他的选择,使用口水或是超市、药店购入的润滑液也同样好用。润滑液能够改善许多人的性经验,当你不知道身体会有什么反应时,在第一次使用一些会是不错的想法 。

接下来提到处女膜,阴道薄膜。我们已经花了整个章节谈论阴道最狭小的部分,但我们还是会重申几个重点。你的处女膜在进行第一次性行为时不一定会流血。不会流血和会流血的机率差不多是一样的。而且没有人能够在完事后透过你的生殖器来判断你是否有过性经验。你的阴道不会被薄膜完全覆盖,也不会有弄破阴道薄膜的情形,它只是一个有弹性的环状组织。别浪费力气担心你的处女膜,不如花时间担心像是环境危机、难民问题还有学校性教育不完全的严重议题。你的处女膜不值得赔上留在对方家过夜的机会。





建议与窍门




妳现在知道了不少阴道在性行为当中会发生的变化,那么实际上妳该怎么去做呢?第一次和男人(或男孩)上床妳可以怎么做?从完全技术面的立场来看,我们有两种建议,但是妳也可以有第三种选择。毕竟,这是妳的阴道。虽然有不同的方式,不过都是不错的选择。

第一种非常传统,但绝对值得考虑。传教士姿势不常在色情片里使用,因为看不太到性器官(没有露出性器官的色情片又会变成什么样子呢?)在现实中,传教士是第一次性行为里最常见的姿势。妳(女方)会在这个姿势里躺著,而男方倚身在妳的两腿中间,因此你们的胸部及腹部会互相面对面。接著阴茎随著男方前后摆动进入妳的阴道。从妳的角度来看,这不是个积极的姿势,不过有许多原因可以解释它是好的起步点。可以充分进入及掌控彼此身体,还可以继续接吻,而且,最重要的,是妳能够观察彼此的反应,所以妳总是可以得知对方乐在其中。对都很紧张的你们来说,第一次格外重要。如果觉得有太多的眼神交流,妳可以闭上眼睛。

比起主导,有些人更害怕放弃控制权。许多人对于被别人载往高速公路时感到无比惊恐,而想要成为指手画脚的人。妳是这样的人吗?如果是,妳会更喜欢掌握主导权,我们让妳坐在上面。男生会在一开始躺著,而妳会在他的上面,有点像是颠倒版的传教士姿势。把妳的膝盖放在他的臀部两侧,坐在他的阴茎上。如果妳想,也可以将手臂或手掌放在床上支撑自己。妳完全不必像骑马一样坐在那边,即使人们常说这个姿势像是女方骑著男方或称作女牛仔姿势。觉得坐直挺挺感觉很赤裸,妳可以往前倾。妳就喜欢那样的话,那就骑著。现在妳是动最多的人了。妳自己能够控制方向、阴茎进入的深度,以及摇动的速度,这就是坐在上面的优点!

当妳坐在上面时就和传教士姿势一样,可以非常清楚地看到彼此的脸。是有一点可怕,不过这个姿势比较方便沟通。

虽然色情片或好莱坞电影给妳的印象是如此,但不是所有的性爱都以高潮结束,男女都一样。高潮需要练习,并非是第一次进行性行为的妳或伴侣该期待的事。为了达到高潮,彻底了解自己的身体还有感到安全很重要。也因为如此,有些与稳定交往对象进行性行为的女性更容易高潮。另一个了解自己身体的重要方式则是自慰,对很多人来说,可能需要花上好多年的时间才有高潮:而且通常自己来更容易达成。不过熟能生巧!我们之后再来讨论这个部分。

和伴侣沟通也很重要,一定要说出妳要的是什么,不要指望伴侣替妳解决高潮的问题。亲自解决绝对既正常又合理。和伴侣做爱不代表妳就无法同时留意自身的状态。毕竟,妳可以向伴侣展示当下的行为,而对方也能对喜欢与否作出回应。

性是有趣的,然而如同其他有趣的事物,存在著一定的风险。和安全带及自行车专用安全帽降低严重受伤的风险一样,避孕措施能够降低性传染疾病与怀孕的风险。

避孕绝对是共同的责任。性行为要双方才能完成的话,那么也就需要共同处理避孕的事情。然而,不能总是认定妳的伴侣也会预备好。我们对妳的建议是凡事亲力而为,这也是我们给所有看到这个章节的男性的建议。倘若妳的伴侣也准备了,那倒是好现象,代表这个人很聪明。

避孕需要计划,所以在第一次性行为之前要知道如何妥当使用。去向医师或护士咨询,同时详读这本书的避孕章节,任何妳需要知道的内容都在里面。我们建议采用保险套加上其中一种避孕方式,才能提高避孕的防护效力。目前,现行的避孕措施几乎为女性使用,不过好消息是,男性专用的方式正在开发中。保险套是唯一避免性传染疾病的避孕方式。无论如何一定要使用保险套,但是也要照著我们后面的保险套教学来确认不会在性行为中间受伤。为了避免发生意外,准备事后避孕药也是不错的方式。妳很快就会学到更多。

如果妳想要做爱也准备好避孕措施,那么放手去做吧。只有妳才知道准备好了没。尽管如此,我们对于第一次性行为最重要的建议则是:就把它当做第一次看待。同样是第一次的人有很多,妳也会变得更熟练,一切都会变得更好。





* * *



7	小荳蔻法则出自于托比约‧艾格纳(Thorbjørn Egner)的童书《小豆蔻城里的人民和强盗》(When the Robbers Came to Cardamom Town)意思约为:不要打扰其他人,要善良、仁慈,此外,不要犹豫去做任何想到的事。



8	这不适用于所有女性。很有可能会是性欲高涨但下体却不湿润的情况,反之亦然,没有感受任何欲望也能够使下体湿润。妳可以在谈论欲望的段落中读到更多。





肛交




我们"洞外有洞"的部分结束在让人心痒痒的地方:布满肛门周围及内部的神经末端正等著接受刺激。如果有人让臀部一起加入狂欢的话,他们会发现肛门扩展了性生活的规模。

好极了⸺我们的肛门有大量的神经末梢,但我们该如何刺激它们呢?你大概认为"让臀部一起加入狂欢"听起来太过乐观。和鞭打及蒙眼相提并论,许多人觉得肛交听起来既可怕又有点脏。"什么?我该把东西伸进去那里吗?我们该在和大便同个洞口上性交吗?"

肛交毫无疑问属于"性爱高阶课程"。妳没有特别喜欢的话,它也不是必须要做的事情。然而,肛交在异性恋情侣中渐渐变得稀松平常。去年几乎有五分之一的十六岁到二十四岁英国年轻人有过肛交的经验,其他地区的年轻人和我们也没有什么不同。

人们开始肛交,但是他们常常因为错误的理由而做。不幸的是,我们可以看到女性常常被迫进行肛交,而她们都拥有不愉快或疼痛的经验。肛交是女生必须"学会享受"的概念广泛流传,这不该是肛交的用意。肛交应该是自愿去做而且本来应该充满乐趣。如果妳没有兴趣,就不要强求。设好妳的底线。

好奇的话,这个部分就是为妳而写。有许多女性喜欢肛交,而肛交有不同的形式。肛交一词涵盖了所有种类的肛门刺激,可能需要阴茎或人造假阴茎进行插入式性交,也可能透过指交或口交,舔著肛门的周围,也就是所谓的舔肛。妳不喜欢阴茎在屁股里并不代表妳无法借由其他方式在肛门中得到快感。

这个部分所给的建议是与插入式肛交有关,也就是使用手指、阴茎或其他物品。所以有些事情需要在妳开始之前理解。

妳可能还记得之前章节提过的内容,肛门有两个强壮、相邻的括约肌:一个是自然运作,而另一个是随意肌。它们非常实用,表示我们不用经常为了上大号而冲进厕所。括约肌以百褶裙似的皱褶形式保持肛门紧实及隐藏环形肌。

许多人认为肛管与直肠非常狭窄,比阴道更紧实。这可能是吸引男性的魔力之一,但事实却只有一部分正确。实际上直肠像是汽球一样:在底部打个结。位于最底部的括约肌强力压迫肠子末端。这表示它的末端非常狭窄,一旦通过括约肌后就会有很大的空间。另一方面,阴道是从洞口到子宫颈都充满肌肉的质管。所以阴道可以保持一路狭窄,而直肠却只有底部狭窄。此外,括约肌并非一直如此狭小。在妳进入没多久后就会开始放松,而直肠也不再那么狭小了。

气球上的结则代表肛交有几个非常特别的挑战。当妳进行阴道性交时,我们曾提到让身体放松,使骨盆肌肉减少收缩,避免让性交变得困难。肛门里的括约肌却不是这么运作。就你所知,在完全放松的时候,妳的屁股仍然是封闭的状态。睡觉或是深度冥想时也一直保持紧闭。这是不随意环状肌正在运行。妳无法自行借由放松让肛门口变大,妳能做的只有防止随意括约肌收缩。妳不能控制不随意括约肌,但是,就像我们说的,它会因为刺激变得松弛。

最重要的一点,则是放轻松开始。如果妳的肛门从来没有东西进入过,就不要从太过粗硬的阴茎或是巨型人工假阴茎开始。括约肌需要时间才能放松,一开始要经过可以掌控放松程度的随意肌,接著不随意括约肌才得以收到提示。试著从手指或是小型情趣用品等小东西开始习惯。大多数人在准备好以前需要许久的时间热身准备。

如果妳进行得太快,肛门容易造成小型撕裂伤,可能呈现好几天极度疼痛的状态。在没有完全了解的情况下,任何打算往躺平的伴侣阴道进入的人,最后都会找到错误的洞口,这样会受伤的。如果妳打算肛交,妳必须完全准备好。同时妳的伴侣也必须要有耐心,直接开始是没有用的。

只要妳一起步,事情会变得更容易。妳的肛门会愈来愈松弛⸺这让很多人感到害怕。结束过后气球结没有封起来。"噢不!我的环形肌要永远下垂了吗?"还差得远呢,放轻松。肌肉会慢慢地再次紧实,只是需要花一点时间。

那倒是真的,括约肌和身体任何部分一样,都可能会有永久性的伤害,但是妳必须要连续猛搥才能造成。别忘了肛门能容纳比一般尺寸的阴茎还大的东西。慢慢开始,小心进行,觉得不对劲就停止:这样一切都会顺利。

肛交的另一个要点是湿润度。阴道通常在妳性欲高涨时会自己变湿,所以妳必须使用润滑剂或其他类型的人工液体进行肛交。没有润滑剂,任何东西都难以进入,而且如果太干,会造成许多摩擦。摩擦增加撕裂及轻微流血的风险。

事实上直肠腺体也会分泌一些液体,不过与你性欲高涨与否无关。肠子里就像阴道与口腔内部一样,有黏膜存在。黏膜的特征就是制造黏液:如同口腔里的口水和阴道里的分泌物。当直肠黏膜受到如阴茎的刺激,会制造黏液保护自身不受到伤害。因此性交会促使体液分泌,但还是不够,妳同样需要润滑剂。

接著来到了大问题:大便。我们都听过有女性在肛交时突然大便在伴侣身上的都市传说。虽然对多数人来说没有什么吸引力,却也无法改变粪便在直肠里的事实⸺毕竟,这就是肠子的功能。即使妳不觉得自己需要去厕所,粪便会累积在肠子里直到饱和。直肠是粪便流向外头世界前的储藏地,这表示粪便可能会落在阴茎、情趣用品或手指上,如果妳事先没想过,可能会造成一点惊慌。真的发生的话,也没有什么不对,妳也不该感到羞愧。如果妳打算进行和肠子有关的性交,那这就是游戏的一部分。

然而,的确有方法能降低大便的风险。有些人选择在药局购入的少量灌肠剂清理肠子以解决问题。其他人则是在开始前先去厕所一趟。

妳当然不能透过肛交怀孕,但绝对有可能得到性传染疾病。许多人忘记这一点,或认为他们不太可能有肛门感染的情况。事实上,完全相反。部分性传染疾病更容易透过肛交传递。如果妳和新伴侣发生性行为,直到对方去检查前,保险套的使用就变得无比重要。无论什么类型的性交都一样。

如妳所知,妳和伴侣做过性病检查后可以不用保险套进行阴道性交,但是屁股包含了肠道菌群,因此卫生非常重要!妳不会想要让肠道细菌进入阴道或是尿道这些不属于它们的地方,因为最后有可能造成感染。这一样会发生在男性身上。所以从肛交换成阴道性交时要特别留意,透过手指或阴茎也是。如果妳想接著进行阴道性交的话,那么在肛交时使用保险套,之后再拿掉是不错的方法。记得也要清洗使用在肛门里的情趣用品。

顺带一提,有些玩具是为肛门设计,通常在底部会有一个插塞以防落入直肠里。因为阴道长度不会大于7〜10公分而且顶端是封闭的,所以没有任何东西能够消失在里面。但是,肠道没有尽头。因为玩具卡在里面而去急诊室拿掉的感觉实在糟糕,却也真的发生过。医生们透过互相分享掏出病患屁股里最奇怪的物品获得极大的娱乐:厚实的蜡烛、玩具车、iPod或是瓶子。医生想必也感受到他们的乐趣。

这些确认事项是给想要尝试肛交的人。正确地进行,就能让女方与男方感到惊奇,但是前提是女性必须停止肛交为了男性而做的想法。肛交,和其他形式性交一样,必须你情我愿才能进行。





完全正常的性生活




当二〇一五年影集《女孩我最大》(Girls)占据我们的电视萤幕造成轰动时,许多人因为终于看到普通女性进行一般性行为而将此描述为革命性的突破⸺不管里面的剧情为何。除了许多在厨房角落发生的高潮与火辣性爱以外,我们看到了笨拙、冷场以及不断尝试以性感内衣现身在男友家里失败的场景。这些呈现了主流文化下的女性费尽心思去实践理想的性生活,在《Elle》杂志最新文章里的淫声浪语与拍打屁股看起来很性感,然而在亚当与莉娜于现实生活中尝试时,却变得像是让人尴尬的电视节目。《女孩我最大》呈现了理想与现实之间的冲突。

《女孩我最大》反应了性成为公共财的事实。几杯红酒下肚后人们高谈阔论朋友性生活的最私密细节。女性已经掌握性的自主权。性欲高涨是好事,明白自己想要的也是好事。对那些实现的人来说,再好不过了。

不幸的是,对于性生活应该要如何的期许成了精神包袱的一部分。我们的性生活变成另一个应该要去展现自我的平台。只有在女性好友之间的私密对话中才能提出更害羞的问题:每隔一周才做爱这样正常吗?妳平常做爱都会替对方口交吗?如果我在做爱时只能靠抚摸自己来高潮会不正常吗?





因为,正常性生活的组成到底为何?诸如此类的疑问使我们对正常的性爱进行探索。

当人们评估自身性生活时,通常做爱次数会是与他人相比最简单的指标。虽然数量是如此主观,却容易计算。如果妳问异性恋的人通常多久做一次,妳会在大部分西方世界里得到相同的答案:异性恋情侣每周会有1或2次。而同居伴侣比已婚伴侣更多次,单身人士则是最少。我们对同性恋男性及女性知道的甚少,但是有些数据显示女同性恋伴侣的性爱次数和异性恋差不多。

挪威人非常与众不同。在一项介于二十三岁至六十七岁伴侣的研究指出,大约40%的人上个月每周做爱次数会有1至2次。只有10%激烈的族群,一周会有3、4甚至更多次,和完全没有做的比例一样多。剩下的人则是每隔周1次或更少。

这项研究中,令人意外的是,不同年龄层性爱频率没有很大的差异。只有年满五十岁伴侣的频率较低,尽管如此,其中还是有40%的人每周有1、2次甚至更多。不过,我们从一系列的研究得知年龄是情侣做爱频率的重要因素之一。这是因为身体性功能劣化与年纪有关。性欲降低,男性有勃起问题而女性可能会发现雌激素降低,造成阴道黏膜薄弱、易损,性爱过后会变得更不舒服。然而,除了年龄以外还有其他影响我们做爱频率的因素,其中一个则是相爱程度。

新关系的第一阶段仿佛像在泡泡当中。大脑充满神经传导介质,传递快乐、满足与欲望。沉浸在相爱的感觉,妳会忘记任何你们两个以外的事物。性变得比睡、吃与朋友重要,同时也成为彼此的共通语言来传达所有妳不敢说出的话:现在只有你和我⸺只有这件事最要紧。

性每天总是有办法在最后悄悄接近妳身旁。某天晚上,妳盯著时钟,发现此时有一只迫切的手缓缓地伸入妳的弹性内裤里。"我们可以稍微爱抚一下就好吗?我必须要早点起床。"妳带著歉意的笑容回答。如果妳突然不再日日夜夜想要做爱的话代表妳的关系出了什么问题吗?还是这只是自然的发展呢?

德国对1900名二十多岁稳定交往中的学生进行研究,发现情侣交往的时间与做爱的频率有明显的关联。平均来说,热恋中的情侣做爱的频率为一个月10次,或是一周2.5次。70%的人则是一个月多于7次。交往第一年过后,频率开始下降。当达到一至三年的时间,一周有2次以上的人数不到一半。五年之后,跌入谷底。此时性交频率减半,一个月为5至10次不等。这些结果在其他女同性伴侣间的研究里也能看到。

换句话说,不是只有妳一个认为自己比以往更少做爱。所以到底是怎么一回事?这项德国研究得到一些有趣的结果。在关系初期,女性与男性对性渴望程度相同,也同样拥有对亲密与近距离接触的欲望。接著奇怪的事发生了。男性在三年过后仍然性欲高涨,但研究显示女性在交往第一年后的性渴望急剧降低。第一年里,有四分之三的女性承认想要经常做爱。三年过后比例降至四分之一。比起交往初期有两倍的人,达到9%至17%的比例,表示他们经常感受不到对性的欲望。

另一项调查为交往中的男女在想要做爱时被拒绝的比例。先前提到的挪威研究里,有半数的男性表示有时候会遭到拒绝,10%的人则是经常被拒绝。而女性的比例刚好相反,90%的女性表示她们从未或是甚少被男性拒绝。

随著交往时间的增长,有项指标却没有降低,就是女性对亲密与近距离接触的需求。然而对男性来说,拥抱的欲望则是随著时间降低。或许世俗的认知比我们所想的还更加真实:女性想要拥抱,而男性想要做爱。为什么?我们无从而知。德国研究的参与人员认为最佳的解释则是人类演化。女性下意识将性当作绑住男性的手段,一旦目的达成,男性中计后便会丧失兴趣。其他人则认为答案存在于生物的性驱动程度不同(我们会在之后提到性驱动的影响有多远)。而他们甚至指出社会上有著所谓的性"剧本",记载著男女应该要如何扮演各自的角色。人们认为性欲是男子气概的展现,而女性表达相同程度的欲望则有失女子风范。这可能造成女性更容易比男性进入没有性欲的状态,但也可能会使男性因为失去性趣而羞愧的程度上升。

到目前为止,我们已经知道妳交往得越久,做爱的次数就会越少。同时,我们也知道最幸福的情侣是拥有做爱次数最多的族群。可以感到欣慰的是,幸福程度还是有一定的极限。

一项样本数为三万人的加拿大研究发现,每周做爱一次的人幸福程度并不会上升。因此人类是有可能找到属于自己每周几次的黄金比例!

那么除了频率之外哪个面向还会影响我们对性生活的满意与否呢?再次强调,答案显而易见:关系的品质。我们对关系的满意与性生活的品质有著密不可分的关系。简单来说,好的性生活就是好关系。我们不知道是好的性生活让我们对关系满意,还是好的关系创造好的性生活,我想大概都有吧?

良好的关系与沟通极为有关。在性与情感方面你必须与对方互相讨论。喔,天啊,也太瞎了吧?

为什么我们一定要谈到性?这不就活生生证明你的性生活完蛋了吗?一夜情与新关系最迷人的地方完全在于不用对话。人们比起忘记保险套更害怕沟通破坏了气氛。简单的谈话对神秘与刺激感造成分崩离析的威胁。

显而易见的是,能够借由谈论想法、需求与期许来达到情感上亲密的情侣,长期下来更会满意他们的关系与性生活。此外,聊性的情侣不止更满意彼此,做爱的次数甚至也会增加。

情感关系上有很多事情能够摧毁对性的欲望:压力、缺乏相互分享的宝贵时光、没有做爱的想法、负面的自我形象与身体意识不足。如果你觉得和伴侣的性需求不同,你可能很快会进入其中一人永远掌握主权而另一人常常拒绝的恶性循环。拒绝别人一点也不有趣。你有罪恶感是因为你无法活在他人的期许当中,你可能开始焦虑对方最终因为厌倦而离开。你担心得愈多,对性的欲望就会愈少。最后,你害怕对方期待更多甚至避免了单纯的拥抱或接吻。

这通常是情侣停止正常性交的潜在危机。认为自己能够不透过相互沟通来摆脱困境的想法实在天真。如果有更多的情侣察觉到情况不对劲便勇于沟通的话,他们之间就能避免许多的问题。所以和伴侣坐下来,放下手机来场彻底的谈话吧。搞不好你可以得到更多、更好的性生活。

你现在大概在想次数不是一切,我们也绝对认同。一周做爱两次固然不错,但内容才是重点。人们通常都进行哪一类的性交呢?毕竟性可以涵盖许多种类。它包含了吸与舔的动作,阴道或是肛门的性交;人们会不会达到高潮,也可能在双人床、沙发或是饭店电梯内做爱。对某些人来说,固定的做爱生活是他们的死对头。他们怀念刺激感、单身生活的不可预测性,或是交往初期的感觉。

二〇〇六年澳洲的一项研究调查一万九千人最近一次性爱的组合。得到的答案是有12、%的人只进行阴道性交,50%的人在阴道性交时会透过双手刺激彼此的性器官,而约30%的人也会进行口交。研究发现透过双手与舌头的次数愈多,女性更有可能达到高潮,这样的结果让人不足为奇。

良好性生活的概念伴随许多的期望。现实是正常的性生活,真的,非常普通。只有非常少人像兔子一样喜欢频繁做爱。人们在热恋感消失及日常生活和他们的性生活变得密不可分后开始有点厌倦。少数人在做爱时会替对方口交,不过大部分都还是非常满意。如果你想要让一切变得更好,那么只要去做一件事:彼此沟通。





欲望的消失




激起性欲不再是禁忌。甚至在年轻女性当中性欲是理想的典范。这个完美的概念包括享受性爱、主导性爱及实验性爱。如果你的欲望消失或者从一开始就没有出现过的话我们该怎么办呢?这可能让人们留下可怕的拒绝感。

二〇一五年冬天,妮娜非常荣幸与特别迷人的奶奶见面。当时一百岁的雪莉‧祖斯曼(Shirley Zussman),是名略为驼背,有着丰唇和水汪汪大眼的女士。她可以说是性革命的先驱。她曾与"探索"女性高潮为名,HBO影集《性爱大师》(Masters of Sex)的原型威廉‧麦斯特(William Masters)与维吉尼亚‧E‧强生(Virginia E. Johnson)一同研究。自一九六〇年起,祖斯曼在纽约以一名性治疗家从业。

五十年后,她仍然在纽约上东区的办公室替患者治疗。办公室里布满了花卉,而书架上装饰了不同性姿势的木雕。这些摆饰一直以来在性问题发展方面给她独特的观点:"以前,我的患者曾经来找我求助和高潮有关问题,像是早泄或是达不到高潮的状况,可是现在就只有失去激情上的困扰。"她说道。比起一九六〇年代,人们绝对拥有更好的性生活,祖斯曼表示,无法引起人们做爱的兴致才是问题。她将原因归咎于科技与职场的高度压力。"前来求诊的女性疲惫到宁愿看着那些该死的iPhone也不愿留一点时间维持亲密关系。我们忘记去互相抚摸、看着彼此的双眼。"

祖斯曼博士极有可能是对的,看起来缺乏欲望是女性的新疾病。二〇一三年的一项重大研究显示,上一年度有三分之一的英国女性饱受性欲缺乏所苦。介于十六至二十四岁的族群中,四分之一的人缺乏做爱的兴趣。这项结果读起来令人难过。

所以女性缺乏欲望的标准是以什么作为比较呢?从一九六〇年代起,部分研究采用了一种骨牌模型,其中提到四个性反应的阶段:欲望—兴奋—高潮—消退。其中,将欲望定义为希望进行性行为,包括幻想与思考。欲望是纯粹的心理过程:我现在超想做爱!然而,兴奋,是兼具欢愉感与纯粹的生理反应,此外,还有性器官充血、阴道湿润与扩张、脉搏数与血压升高与呼吸急促的情况。

直到近代才有研究人员开始质疑这个模组。事实上,调查指出,有三分之一的女性甚少拥有性欲,也就是说他们没有感受到所谓的"自发性欲望"。然而,大多数人仍经历性所带来身体上的刺激与享受。或许听起来很奇怪。真的有那么多女性因为这些严重问题感到困扰吗?





不,有愈来愈多人会这么说。对许多女性来说,欲望其实有反应性,也就是说,会升华成亲密接触或是性行为状态。身体上的刺激地位高于欲望,你可能会说,所以这些女人比较仰赖前戏和亲密互动来触动开关啰?反应性欲望的女性对性不太有兴趣而且在床上也不太拥有主导权,但她们性致来了仍然有能力进行美好的性爱。只要小心翼翼,欲望是可以养成的。

肩负了教育女性有关反应性欲望责任的性研究者艾蜜莉‧纳高斯基(Emily Nagoski),在她的著作《性爱好科学:挣脱迷思、用自己的方式高潮

》(Come As You Are)里提到,接近三分之一的女性对性欲是有反应的。另一方面,我们发现15\%的人属于"典型"的自发性欲望,意即你会突然感受到性欲上升。其他女性则是介于两者之间。有时候,她们不太清楚自己想要做爱的原因,但其他时间,性听起来又有点像是累赘,直到她们感受身体起了反应,接下来大脑才慢慢地一起加入同乐,只有5\%的少数族群完全缺乏自发性或反应性欲望。

反应性欲望的模组与主流文化应该展现的性有明显的分叉。我们遇过许多的女孩与女人无法分辨这种形象。她们纳闷着如果和"其他人"不同,对性不感到兴趣的话,她们是否就不正常。她们被男友洗脑,认为她们非常无趣,甚至因为对做爱不主动感到罪恶。对这些广大女性来说,另一种解释这样现象的模组可能让她们松一口气。许多现象显示反应性欲望纯粹只是女性性欲的正常变化并非缺陷或是疾病9 。

我们认为另一部分的原因是自发性欲望为男性对欲望的正常主宰。纳戈斯基表示,大约有四分之三的男性明显属于自发性欲望,出于一些奇怪的原因,我们认为男性与女性的性欲是以相同的方式运作。我们之后就会知道,男女之间大概不是。

另一个疑问的来源是人类与生俱来性驱力的迷思。也就是我们一出生就拥有性欲。驱力就像帮助我们生存的本能。它造成我们口渴、饥饿、疲倦。我们大脑完全下意识地传送是时候去做特定事情让身体为持平衡的讯息,例如睡觉、进食或是饮水。如果我们拥有性驱力,它会告诉我们对性有需求就好比我们对食物、睡眠与保暖的需求一样。这样的话,性成为我们生存的必须要素。当性被如此定义的时候,我们也就没必要将性欲想得太复杂了。10 另外,你们对一切感到怀疑的话想想,没有人死于缺乏性欲。性不是驱力而是奖励。

只要性持续带给我们享受与欢愉,就会像毒品对大脑的影响一样:我们会想要更多。欲望受到刺激,接著我们开始寻找可以得到性满足的情况。这就来到了纳戈斯基重要的观点:如果性没有带来奖励的效果,例如因为痛苦、早期受到侵犯带来的影响或是单纯厌倦,妳的欲望便会消失。只有当性作为奖励的时候这个机制才会运行。换言之,我们不是一出生就有性欲,而是渐渐变得欲火焚身。

我们从这个观点得到两个结论。第一,拥有较少性欲的女性(与男性,即使一般来说他们只经历过反应性欲望)并非不正常或是生病。就像有些人喜欢巧克力,而其他人不喜欢。就算大部分的大脑对脂肪和糖分的这类的愉悦组合非常有反应,我们也不会认为不喜欢巧克力的人有什么不对的地方。然后,顺带一提,我们有没有将那些人称为有病这件事情为什么这么重要呢?如果妳被当作一个活生生的怪人,那么体内剩余的一点性欲真的会完全被消灭。

第二,这表示性欲并非恒久不变。我们生来具有变得性欲高涨的潜力,但是程度范围依据性带给我们多少的快乐与满足以及我们平常生活状况为何,并随著时间有所变化。此外,我们的性史,也就是我们的性经验,帮助我们形成性欲。

这个论点解释了为何性欲会随著我们的人生与关系高涨及降低,同时也在欲望的影响上给我们特别的意义。如果我们明白一切的运作,就能够操纵大脑的奖励系统。这个观点告诉我们男性与女性最大的不同。

性研究者想出一些非常奇怪的点子。在大量的实验中,将男性的阴茎与女性的阴道上装有测量仪器,观察性器官的血液流量。这个研究方式能够从生理上得知人类的兴奋程度,这些是人类无法靠意识控制的自然反应。实验里,实验对象可能要观看不同类型的色情片:异性恋、同性恋、拥抱、暴力,甚至是猩猩之间的性交。满足各种偏好,也就是这个意思。看完影片后他们接著回报自己感受到的性欲程度,而在这个阶段产生了非常有趣的发现。

在男性当中,约有65\%的人阴茎勃起状况会对应所感受到的性欲程度。所以大脑通常会随著男性性器官有著相同的自动反应。啊哈,我勃起了所以我一定是想要做爱,男性会这么认为。(当然,这只是个大概。男性即使没有任何像是做爱的欲望也能勃起,就像是大家都知道的晨间勃起现象,或是青少年在黑板前展示计算过程时的勃起。)男性的欲望与阴茎的恶作剧密不可分,因此当男性想要"站起来"的时候,威而钢之类的药丸就能完全发挥功效。威而钢并不在大脑运作,而是单纯让血管将血液带离阴茎造成收缩,让阴茎变得更硬,使其充血。如果你让阴茎在对的位置上,那么也差不多大功告成,只要知道这一点就够了。

然而,在女性实验者里发现,只有25\%的人大脑与性器官重叠运作。两者之间的关联如此薄弱,因此根本不可能说女性感受到性欲的程度是取决于她的性器官有多湿或充血。有位女性的生殖器官在看到男性双腿之间及猩猩的性交后完全肿胀与湿润,但她完全没有被激起性欲。女性生殖器官同时对女同性恋性爱的反应强烈,频率比起异性恋性爱来得高。令人震惊的是,女性在被侵犯时能够同时达到身体上的兴奋与高潮。这代表什么呢?女性的确喜欢猩猩性交,还是有些女生喜欢被侵犯呢?

不,不是,不是这样的!这表示女性和男性不同,拥有性研究者所称更高程度的"性兴奋不一致"或是"主观生殖器一致(不一致)"。这些复杂的名词,其实表示在欲望当前大脑和下体区域之间的反应不一致。两个身体部位明显地没有共同的想法,而欲望非常低的女性却在这方面得到最高分。她们的大脑几乎无法接收生殖器传来的讯号。

女性欲望主要座落在大脑当中。光是有魅力的人躺在我们床上,或是像男性常有的变湿与勃起,还是不够。我们需要更多,因为需要刺激的是大脑,而非生殖器官。这就是即使尽了许多努力,威而钢却只对极为少数女性有用的原因。为了让女性性欲得以受到药物的影响,必须改变大脑错综复杂的回路,这才是提升到全新层次药物该有的技术。

针对女性性欲的"粉色药丸"历经许多努力终于被研发出来。其中一项实验是给予女性睪固酮,因为这个性荷尔蒙被认为是性欲的中枢。问题是,具有生育力的女性服用睪固酮却不是个好主意,如果女性怀孕的话会对胎儿造成潜在的危害。所以大部分研究会选择因为癌症手术或是更年期而完全缺乏睪固酮的女性作为实验对象。在这些案例当中发现,睪固酮浓度提升对性欲有正向影响。以三十五到四十六岁略为年轻女性为对象的重要研究里,却出现性欲程度没有上升的情况。然而,经过一个月的时间,接收中等剂量睪固酮的女性与服用安慰剂的女性相比,在"满意的性活动"次数中增加了0.8%。

研究成果显示了一旦超过最低水平,更多的睪固酮不会造成极大影响。事实上睪固酮对性欲影响的研究没有任何重要结果能够夸耀。无论性致高低与否,它似乎不能预测妳所位于的性欲阶段。看起来性荷尔蒙无法对女性性欲有强烈的影响。

其他药物也经过了测试。以"芭比药丹"广为人知的人工荷尔蒙美拉诺坦(Melanotan),由于挪威知名部落客苏菲‧伊丽丝‧艾萨克森(Sophie Elise Isachsen)在青少女当中引领网上非法购入的风潮,曾经一度引起媒体极大的注意。

美拉诺坦模仿身体其中一种荷尔蒙,让我们皮肤变黑并制造斑点。起初,美拉诺坦是用来帮助我们不须凭借太阳晒黑,也就是一种助晒药丸。后来发现美拉诺坦的副作用包含食欲降低,以及有提升性欲功效的可能。因此它具备完美女性的梦幻条件:古铜肌肤、身材纤细与性欲高涨。我们也从而得知,药厂看见了商机。

然而问题在于,使用美拉诺坦最终会造成潜在生命危险的副作用。所有相关实验也就此暂停。接著药商发现了危险因子较低的药物布雷默浪丹(Bremelanotide)。经过几年实验后,药物现在正接受最终测试,看起来未来得以批准上市。但是这个昂贵的药物需要使用注射器,再者效用也没有特别大。平均来看,该药物的使用者回报每个月"满意的性活动"次数比起使用安慰剂的人多了半次。既然如此,也没什么可以说的了。

另一种药物,氟班色林(Flibanserin),起初当作抗忧郁药物使用,于二〇一五年八月开放给低性欲的人使用。这项药物一样极其昂贵,一个月需花费好几百元英磅,而且必须每天使用。由于会引起低血压造成攸关生命的风险,妳也不能在服用药物时饮酒。诸如恶心、晕眩与疲倦等副作用也经常发生。同样的,这个药品的效用没有非常惊人。平均每个月使用者"满意的性活动"次数增加了0.4%到1%之间。

换句话说,药丸目前无法成为治疗患者的希望。上述所提的药物将副作用、价格与结果纳入考量的话,没有一个是适切的选择。然而,这类的研究著重于性欲与满足对我们的感受影响有多大。在一些研究里,安慰剂效用确实有著极高的效果,几乎比其他"药物"来得高。此外在一项威而钢的实验中,40%拿到糖果药丸的女性其性欲有改善的情况。借由服用药丸,她们进入新模式与新角色,改变了原先认为大家都不想跟自己做爱的旧有、根深蒂固的思维模式。

安慰剂效应告诉我们:性欲就在我们脑中,而且是能够被操纵的。但是要怎么做呢?

性研究者艾蜜莉‧纳高斯基解释得非常到位。想像大脑是个在身体顶部支配一切的敏感指挥官。身体的指挥官经常接受来自身体的讯号,他解读的环境帮助这些讯号形成更合适的意象。我们神经系统和传递至大脑的讯号构造非常简单,有点像电脑程式,里面都是0与1。当中有一个讯号告诉我们"加速",意即刺激,另一个则告诉我们"减速",或称抑制。刺激与抑制两个讯号之间的平衡决定大脑在任意时间对身体的运作。如果妳减速了,同时踩了一点油门的话不会造成任何变化。总效应才是最重要的。

无论是有意识或是潜意识,想像每个阻止妳想要做爱的原因,将这一些压力放到减速器上。原因可能包含压力、忧郁、身体形象差、罪恶感与无法达到高潮的恐惧。所有减速器上的微小压力逐渐增加能够使其压迫地面,让事情完全暂停。为了纾解减速的沉重压力,大脑需要接受更有力的讯号来告诉我们加速,例如爱与欢愉。奖励必须比努力要来得好。有时候,它自己会突然发生,例如当我们恋爱时,除此之外,我们的任务是确保"加速"的讯号能够主宰一切而尽可能减少减速的功能。听起来有点模糊,但其实一点窍门也没有。第一步是承认性欲不是自己凭空而来,或是妳与生俱来不变的特质。之后,妳必须坐下来思考哪些东西会让妳性欲冷感或高涨。照著纳高斯基所说的:列一张清单。

什么事情会浇熄我的性欲?在睡前做爱,因为我会担心隔天无法好好休息、觉得心情低弱或伤心、害怕我的伴侣在我不想要的时候提议做爱,然后我又得拒绝他们、对感情的不确定、嫉妒、完全知道接下来会发生何事的规律做爱、为了让伴侣觉得自己是个好情人,我必须达到高潮的期许、我今天应该要做的事情但却没时间去完成的压力或担忧、觉得自己好丑、当我还没洗澡或觉得身上很脏的时候、当我们在床上看手机的时候。

什么事情会点燃我的性欲?知道我们有许多时间也不用急著把事情完成、做爱速战速决,不要废话、对高潮的念头、对自己的身体感觉良好、情色书籍或电影,或者是单纯的色情片、运动后脑内啡涌出,热血沸腾的时候做爱、在大白天做爱、一片漆黑,笼罩在黑暗当中、干净的寝具、觉得被爱、赞美、新环境、安全的环境、看到伴侣自由自在、做我所擅长的事情、我的背部被挑逗、勇于在床上尝试新事物、当我确定每次在床上所做的一切绝对是伴侣认为最棒的事。

在妳写下清单后,真正要做的事情来了。妳必须妥善安排这些事情,让中间的平衡慢慢倾往"加速"的方向。这表示尽可能消除更多减速的原因,同时创造更多触发性欲的开关。

在感情当中妳非常难去独自完成这件事。妳必须要和伴侣一起,告诉她(他)哪些事情会让妳性欲高涨以及妳要的是什么。关系变得一成不变的时候,性治疗师通常会建议妳暂停做爱一阵子,或是建立做爱的原则,例如决定特定的日子与时间做爱,排开妳当天的行程。听起来不太性感,但是却有道理。借由屏除所有对性的期待,妳在欲望主动回归时获得了喘息的空间。毕竟妳不能强迫欲望到来。妳应该要有欲望的念头只是另一个造成减速的因子。

但不代表妳应该停止与对方亲密。事实上,对许多人来说,效果却是反其道而行。因为他们少了任何尚未准备好的压力,多了能够拥抱与变得亲密的空间。妳应该要对自己宽容、有耐心。如果妳的伴侣不认为这很重要,那么妳大概知道问题在哪了。

祖斯曼奶奶,凭著百年的经验,领会到了重要的观点。性欲并非凭空发生。欲望紧密交织在我们生活的情感关系里,不单是我们与关系之间,也没有快速修复的方法。然而大部分的人能够感受到欲望的存在。





* * *



9	这些欲望的差别同样明显适用于男性身上。男性也可能同时拥有反应性欲望,但是在男性主要欲望形式中有点少见。根据纳戈斯基所言,约有75%的男性主要经历的是自发性欲望,相反地,女性则是15%。5%的男性属于反应性欲望,而另一方面女性则为30%。



10	缺乏或丧失性欲确实在国际疾病伤害及死因分类标准(ICD-10)当中属于心理与行为疾患的病症。即使拥有性快感与刺激,仍然有可能被诊断出来。美国诊断系统《精神疾病诊断与统计手册(第五版)》(DSM-V)上的相同症状现在也已被修正。





会见大圆满




高潮是美好、了不起的现象。它和身体为了维持生活机能所进行枯燥乏味的例行工作不同。心脏跳动将血液注入我们全身,肠子翻搅消化提供我们养分,大脑神经讯号颤动使身体开始规划与行动,而高潮有著非常特别的作用。高潮简单来说就是包含脚趾蜷曲、头皮发麻、呻吟的福报,是我们小小的奖励。

人们试著对高潮下不同的定义,不过学者们却是不完全认同。传统医学对高潮的认知是剧烈性快感短暂达到高峰的状态,连同规律的骨盆腔区域肌肉收缩。

现代性研究者认为这个定义太狭隘。女性高潮的感受因人而异,此外,身体上是有可能经历不愉快的高潮或无性高潮,例如受到侵犯或是睡觉的时候。事实上,有三分之一的女性有过睡觉中高潮的经验。因此研究者们开始认定高潮比较像是突然发生的事,是无意识释放的性张力,就像释放拉紧的弓弦一样。

因此人可以在没有愉悦感、没有生殖器的接触或是阴道痉挛收缩的情况下高潮。有人将高潮描述为温暖、颤抖的感觉席卷全身,有著非常明显的"到达"感。简单来说就是在高潮的当下妳会知道。如果妳不知道有没有高潮过,那么妳就是没有。模糊却非常简单。

我们如果纠结在传统的高潮概念,也就是最常见的那一种,那么高潮意即性反应达到最高峰的状态。当女性开始身体上的兴奋,小阴唇与阴蒂内部会和男性阴茎变硬一样充血。事实上,整个阴蒂部位在性欲高涨时会变成两倍大。在生殖器受到刺激后的10至30秒,阴道往往开始变得湿润,同时会变宽、变长至少一公分。妳愈接近高潮,妳的脉搏就跳得更快,妳的呼吸加速而血压也会升高。很多人同时感受到其他部位的肌肉紧绷,外加手指与脚趾在床单上蜷曲。这个现象有个美妙的名字叫作:腕足痉挛(carpopedal spasms)。

最后,高潮来了。幸福的感觉从头到脚贯穿全身。感觉就像妳的生殖器官爆炸,骨盆腔肌肉接二连三收缩。收缩的状态从阴道的最低点开始,接著向上扩展到整个阴道与子宫。尿道周围的肌肉与肛门也会有感觉。女性的高潮平均可以维持17秒。当高潮结束后,血液会开始离开生殖器官,就如同男性阴茎达到高潮后开始变得松软。这个时候身体来到性消退期(resolution phase),所有部位都缓慢地回到正常状态。

与男性不同,女性如果持续刺激自己可以达到好几次高潮。女性高潮的世界纪录不得而知。或许出于某种原因,金氏世界纪录中并没有记载,却能在网站上搜寻到其他恐怖刺激的性纪录,像是"最频繁性爱"。如果妳对此充满好奇,澳洲的鳞蟋蟀(Ornebius aperta)是纪录保持者,在3〜4小时内可以进行50次性交。这个小坏蛋。

非官方统计最高的高潮次数发生在所谓的自慰马拉松(Masturbate-a-Thon),借由令人惊讶的自慰比赛来筹集善款。缔造的最高纪录发生于二〇〇九年的丹麦自慰马拉松,第一名在极为冗长的马拉松赛制时间内,达到222次高潮。这让大部分的人有个目标可以挑战……。

现在,妳大概讶异我们一般所提到的高潮,因为高潮有阴蒂高潮、阴道高潮、G点高潮、谭崔式高潮、潮吹高潮,以及舔脚趾高潮。不是吗?

其实所有高潮都是一样的:就是高潮。身体上与心理上的反应也都相同。唯一的不同在于触发的地方。我们全身都是性敏感地带。能够刺激并带来快感的神经末梢遍布身体。只要想像别人亲吻妳的脖子、拨弄妳的头发或是抚摸妳的大腿会有多诱人就好。我们也有遇过一整天、每一天都会自发性高潮的女性,没有任何的身体刺激,只要呼吸就能高潮。

虽然阴道高潮和阴蒂高潮其实没有什么不同之处,但是这两个名词仍然广泛使用。我们现在知道阴蒂是大型器官,并不只是阴户前方的小肉块。围绕尿道及阴道的阴蒂内部,通常能从阴户与阴道间接收到刺激。关于"阴蒂高潮"与"阴道高潮"的说法不太确切,因为阴蒂和阴道性交是同时进行的。阴道本身非常不敏感。之后妳会知道,每个女性的阴蒂头位置不一样。依据位置的不同,有人认为在阴道性交时可能会让女性更难或更容易达到高潮。

古老传说掩盖了潮吹高潮,女性射精或潮吹的内容,自亚里斯多德时代开始超过两千年,这些词才在文学当中被提及。大多女性认为,即使尿道位于阴蒂头与阴道的前端,但却不属于性生活所触及的器官。尽管如此,有些女性发现尿道在高潮时会发生特别的现象,造成女性与研究者一头雾水。当女性高潮时,尿道口喷发透明或乳白色的液体。有些人喷出相当于一个牛奶杯的量,而有些女性则是好几毫升。这种高潮到底是什么?

我们不知道多少女性有过潮吹高潮,但我们知道确实有这个现象,而许多人都曾在网路上看过。二〇一四年英国禁止色情片描述女性潮吹的剧情。我们不知道为什么女性射精会比男性射精还要来得糟糕,看起来人们觉得女性射精格外唐突,他们大概认为那是排尿吧?但确实是如此吗?

至今仍然不确定潮吹液体的成分,有些研究的论点主张来自小型腺体,斯基恩氏腺(Skene's glands)。这些腺体位于阴道壁前端,围绕尿道下方。看起来并非所有女性都有该腺体,而大小也因人而异,不过至少解释了只有特定女性拥有潮吹高潮的现象。斯基恩氏腺等同男性制造精液的前列腺,在高潮时会将分泌物排至尿道。这个理论是根据女性射精液体中所发现的前列腺物质。然而在二〇一五年的一项研究当中,透过超音波检测七名自慰的女性,发现虽然有少量的前列腺物质在射精液体里,但主要的成分为尿液。几名研究者认为我们会有两种不同的现象:有些女性从斯基恩氏腺射出少量白色液体,而其他人则是从膀胱喷出大量透明液体。无论如何,分泌物的物质不是那么重要。因为高潮对许多女性来说是一种自然现象。

我们回到阴蒂与阴道高潮。连同透过一般阴道插入方式触发到最高点的阴道高潮,女性长期受到高潮层次的困扰。她们认为没有透过《发条橘子》(A Clockwork Orange)的主角艾力克斯‧迪拉吉常说的"抽送"获得高潮就好像哪里出错一样。也认为如果借由手指或是舔吮的帮助达到高潮就好像欺骗一样。

这很奇怪。不只是因为高潮就是高潮,无论妳怎么看待。同时,这种达到高潮的方式对女性来说确实不正常。女性高潮的诡异阶级制度到底是怎么来的11 ?无论如何,它不是从古老时代留下的刻板印象。启蒙运动时期前的人民认为女性只要高潮就会怀孕。如果妳真的想要怀孕,那么男女双方应该要同时高潮。近年来,随著婴儿死亡率急遽攀升,大量增加孩童数量是重点目标。如果男性想要有后代的话那么让女性高潮就成为必须展现的技术。女性高潮的关键就是直接刺激阴蒂头。

在一七四〇年,医生向奥地利公主提议"神圣的阴户女王陛下应该要在性交前受到挑逗。"现代的医生可能是从这里得到启发。比起被告知要活得健康一点,却听到妳应该要更常让私密处受刺激,试想一下这个画面。这就是现在我们所称的主流健康趋势!

因为上述的由来,所以十八世纪的男性就算对世界上更多事物完全不懂,却仍明白整个高潮的运作。而笼罩自卑情结的阴蒂高潮,它的源头就离我们生活的年代更近许多。我们必须回溯到二十世纪。

阴道与阴蒂高潮的差别,与阴道高潮才是真正高潮的这些概念纯粹是现代男性的发明。心理分析之父西格蒙德‧佛洛伊德,在一九〇五年提出一个新理论,将阴蒂高潮认定为年轻女性不成熟的高潮形式。只有在小女孩房间里才会发生的那种类型。只要女孩嗅到男性的气息,她对阴蒂的兴趣会消失,取而代之燃起了对插入式性行为的欲望。男女之间的交合是性唯一的健康形式,也是唯一给予女性欢愉的形式。根据佛洛伊德所说,真正的女性,必定体验过阴道高潮。

佛洛伊德是从哪里知道的?当然是从他脑袋里啊!即使众多女性非常不同意也没关系。因为她们病了。她们处于一个叫作性冷感(frigidity)的模糊状况,起初她们很难从男性最雄伟的地方得到该有的欢愉。即使妳同意他的看法或是妳为此感到生气,这完全是一种心理操控手法。

据佛洛伊德所述,女性如果认为抚摸自己的阴蒂的感觉很好,或是(但愿不要发生)无法与丈夫在阴道性交时达到高潮的话,应该立即寻求心理医生的帮助,当然对男性来说这很棒。倘若女性无法高潮,并不是身为伴侣的资格遭到质疑,反而女性需要为自己做一些努力。现在男性正式得到允许做爱,接著完事,然后开心地关灯转身入睡。女性的欢愉是属于她个人的责任。

佛洛伊德不是普通人,他的理论得到大量的支持。因此千年以来女性的经验瞬间被判定为女性神经疾病。几世纪以来被当作女性性快感核心的阴蒂,在解剖书上遭到遗忘且隐没。将近六十年没有人敢反驳这一切。

一九六〇年代,美国华盛顿大学医院掀起一场寂静革命。妇产科医生威廉‧麦斯特与研究伙伴维吉尼亚‧E‧强生开始对女性性欲感到兴趣,并著手一系列以现在标准来看相当疯狂的实验。他们招募了情侣在实验室做爱,替他们装上测量仪器,研究人员同时在一旁认真观察。他们甚至制作了塑胶震动阴茎,并将摄影机放在顶端,观察女性高潮时阴道的变化。他们的研究结果被认定为惊人的医学发现:阴蒂的确是女性高潮的要塞。很震惊吗?看来真的很令人震惊。

现在,我们知道不到三分之一的女性经常单靠阴道性交获得高潮,即使如此,仍然有许多人认为阴蒂才是核心。有些学者认为这些女性得到解剖乐透的头奖。因为从阴蒂的尺寸与位置而言,她们看起来有一定的优势。第一位在该领域进行科学研究的是另一位公主,法国的玛丽‧波拿巴(Marie Bonaparte),即使她对性与伴侣有极高的品味,但因为无法透过阴道高潮所以一直无法满足自己的性生活。波拿巴公主与现代研究者都认同一件事:愈大的阴蒂头和阴蒂至阴道的距离愈短更容易高潮。因为阴蒂的表面与内部在插入式性交时同时获得大幅度的间接刺激。波拿巴采取了激烈的方式,透过手术将阴蒂往下移,却不幸以失败收场。

我们希望波拿巴明白和男性进行一般性交却没有高潮并非不正常,这是正常的。但是由于男性长期主宰女性性研究以及对性的公共论述,所以导致许多人将这种观念流传下来。阴道里的阴茎,性完全成为男性插入活动的同义词。的确,大家会说少了男性高潮,性交就并非"完美"。如果只有女性高潮,性交在理论上就不完整,造成性交中断。女性在这个阶段就此淡出。

公平来说,对双方来说日常性爱应该充满欢愉与高潮,所以异性恋做爱,照理来说,舔吮和插入的比例不应该要是各半吗?女同性恋做爱达到高潮的频率比异性恋女性高,因此扩展个人的技能很明显是件好事。将女性高潮当作纯粹的额外福利是错误的,高潮应该是一种习惯,对女性来说也是如此。

尽管如此,女性仍然无法摆脱比男性更难达到高潮的事实。女性得到性感缺失病(anorgasmia),也就是从未独自或与对方性交获得高潮的比例,介于5%到10%。对男性而言,则是相反:他们饱受太快高潮的困扰。英国一项大型研究发现,有21%介于十六至二十四岁的女性做爱时很难高潮。大多女性发现自己属于"偶尔高潮"的类别。

有些幸运的女性不知道我们在说什么。我们都有一个讨厌的朋友,告诉我们她常常高潮,每次做爱通常会有3〜4次。妳问她到底有什么秘诀?不幸的是,她可能没办法帮你。虽然秘诀可能会有一点贡献,但我们之间有多容易高潮的差距确实存在,我们也无能为力。这些其中的差距,被认为是受到我们基因的影响。非常少数的人会去想到这是我们父母做爱后的结果,但他们的性生活却多少和妳有一点关系。如果妳是高潮女王,妳大概要感谢妳的父母。

研究人员借由研究双胞胎发现我们的基因可以解释三分之一人们做爱高潮频率的变异。听起来大概没有很多,但是在基因学的范畴下却不可轻视。他们同时观察自慰高潮的频率,此时遗传扮演更重要的角色。研究实际显示,我们的基因解释了一半自慰高潮的变异。起初基因如何分别影响性与自慰的不同看起来很奇怪,然而,在妳消除更大程度的心理不确定与伴侣之间的性相互作用时,自慰被认为是身体高潮能力更真实的反应。

另一件对女性高潮能力有著极大影响的,则是我们做爱的情境。几乎所有女性很难从一夜情达到高潮。只有10%初次与新伴侣上床的美国大学生拥有高潮,同时稳定交往超过六个月的女生拥有高潮的比例至少有70%。

我们达到高潮的容易度有著遗传性的差异,不过振奋人心的消息是,只要妳想要,大部分的女性皆能拥有高潮。从跨出独自完成高潮的一小步,接著偶尔达到,最后做到几乎每次高潮,艰辛的挑战就在这里。我们不认为这很容易,或是特别强调拥有高潮的重要性,但只要妳愿意付出努力就有可能达到。以下是我们的高潮宝典,灵感来自于无法高潮而接受治疗的女性。





* * *



11	下述历史叙述是根据漫画家家丽芙‧史托姆奎斯特(Liv Strömquist)的图像小说《知识的果实》(Kunskapens frukt)为灵感来源。





高潮宝典




一、 熟能生巧

如果妳不曾自慰,是时候空出一点时间了。自慰真的有用。在女性未曾有过高潮的研究里,经过五、六周的规律训练后,60%至90%的人借由自己和伴侣达到高潮。我们保证这是医生建议的最有趣的运动形式。运用妳的手指或是买一个按摩棒。做任何能激起性欲、让妳有心情去做爱的事情。妳喜欢情色文学、色情片还是幻想呢?最重要的是妳绝对不要把高潮当成目标,而忽略去探索妳喜欢的做法。练习感受欢愉并抱持开放的态度。练习清空脑袋所有烦人的念头,无论是肚子上的肥肉还是妳获得的高潮有多棒,之后跟伴侣就有可能会有更好的高潮。同时记得,和伴侣在做爱的时候用自己的双手绝对不会错。只要你们有个火辣辣的时光,谁该做什么一点都不重要。



二、 主张妳的权利

妳的伴侣必须参与"高潮计划"。重要的是不要激怒别人,而是让这个计划变得愉快。妳没有高潮不是对方的错,除非他(她)完全拒绝付出努力让妳满意。事实上,妳必须花点力气做一些准备工作。妳的生殖器官没有使用手册,所以在没有指南的状态下,妳的伴侣可能要花上一年的时间才明白妳会如何高潮。最简单的工作就是在一开始自己来,在做爱或是自慰的同时抚摸自己,过一段时间后妳可以指导伴侣妳是怎么做的。许多人觉得很尴尬,但是却只有这个方法。不要抱持一次到位的期待。伴侣每次做对的时候称赞他们并对他们有耐心,在这之后妳就调教出一个优秀的对象。



三、 训练自己用猫式体位

做爱姿势有很多种,妳现在学到的,很少能让女性高潮。不过,有一种姿势却很特别:猫式体位。因为传教士姿势的变化型,直进性交法(coital alignment technique,CAT),特别能让女性高潮。这个姿势需要练习与互相配合,但是妳付出的耐心会在各方面得到回报。

在猫式体位,你的伴侣不靠手掌,必须靠手臂支撑并尽量保持身体与妳服贴。和正常的进出抽插不同,对方应该将身体靠在妳身上并与你平行,直到他的生殖器从正上方进入阴道。同时,妳应该挤压妳的胯下,就像海浪撞击岸边一样(虽然很老调,但确实如此)。他的屁股应该朝下,这样耻骨和阴茎头才能与妳胯下摩擦。妳就会知道他是否做对了。接著妳应该保持两腿打直并且尽可能夹住对方,这样妳的脚踝会放在他的小腿肚顶端。相较一般传教士姿势属于急速冲刺型,猫式体位则是以传统经典的摩擦方式进行。阴茎不会非常深入妳的阴道,但是会对阴道几公分外的末梢神经地带给予极大的刺激,同时妳的阴蒂会频繁接触。一旦掌握诀窍后,妳可以尝试猫式体位的另一个版本,换成由妳在上面。这样妳就掌握了主导权,同时能控制妳想要的阴蒂刺激节奏与力道。



四、 不要放松!

放松、放轻松,妳通常会听到这句话。这大概是世界上最好与最糟的建议了。的确,妳应该试著让大脑放松,但是妳光躺著不动就希望高潮突如其来涌上的话,妳就大错特错了。绷紧自己的身体很重要。夹紧臀部并试著缩紧生殖器的肌肉,最好是先缩紧后放松,第一,这会让妳的生殖器充血,让自己性欲高涨。第二,集中注意目前所进行的动作,当作一种心智锻炼。如果妳喜欢可以试试,但是真的很难边想著晚餐要吃肉丸同时边夹紧骨盆肌肉。起初,真的很难与维持这些肌肉。毕竟,妳家附近的健身房并不会有《健身趣》(阴道版)的游戏。不过其他地方应该会有。许多常做骨盆运动的女性更容易达到更激烈的高潮,生殖部位的收缩更频繁。此外,它能够预防漏尿或是骨盆器官脱垂,也可能会改善做爱时的疼痛。妳可以在任何地方进行骨盆运动,在公车上或是睡觉前。虽然没有绝对必要,但妳也可以使用阴道球。



五、 去跑个步

做个运动,尤其在做爱前,更容易让妳性欲高涨以及能够增加高潮的机率。



六、 穿上袜子

这个建议看起来有点好笑却是认真的。重点是,我们大脑持续接收身体传来我们感受的讯号。这些讯号,和煽动的想法,博取我们的注意力。当心里想的都是其他事而不是生殖器官里发生何事的时候就很难达到高潮,举例来说,妳的双脚冰冷。女性特别受到这些事情分心,周遭的事情也一样。性研究者阿尔弗雷德‧金赛(Alfred Kinsey)从老鼠上观察到,不同于公鼠,母鼠在性交时容易受到诱人的起司屑影响而分心。这个研究告诉我们,妳必须做任何能够完全集中注意力并创造火热时光的情境。这表示所有的灯都要关掉,妳想要穿著T恤做爱,或者,妳必须穿上袜子的话,那就聆听自己内在的声音吧。对自己好一点,只有身、心同时觉得很舒适时才会高潮,这样妳才能把所有烦心的事情抛到九霄云外。这大概是最难的课题,也是最多人遗忘的方式。





避孕





当女性与男性发生性关系时,可能会怀孕。因为这是普遍的结果,所以不应该感到震惊。孩子从来不是由鹳鸟送过来的。性爱很美好,大多数人希望做爱的次数比他们计划生孩子数量多。如果妳是异性恋者,想要不会怀孕的阴道性交的话,那么妳得使用某种形式的避孕方法。

借由避孕的方式,我们指的是所有可以降低性交导致怀孕风险的方法。换句话说,也就是性交中断,或称为体外射精,这是一种避孕方法⸺虽然不是我们推荐的方法之一。

避孕并不是一项新发明,但是随著医学的发展,更周全的方式已经进入市场。尽管如此,目前有许多避孕方法存在悠久的历史。保险套没有任何创新之处,但以前的材质是用动物皮而不是乳胶所制成。据说在四千年前的古希腊,妇女在阴道里放了蜂蜜和叶子的混合物,使精子细胞离开子宫。这让人联想到现代放在子宫颈上的塑料圆盘避孕隔膜(diaphragm),用来阻挡精子细胞。虽然避孕膈膜在挪威已经有很长一段时间没有被使用,但是在瑞典等其他国家变得更为常见,可能是因为反荷尔蒙避孕的趋势。所以避孕也是一个风潮问题。总之体外射精几乎不算是新发现。甚至在《创世纪》(Genesis)里有一则与它有关的故事(有个叫作俄南的男子),妳可以非常肯定今晚某个地方会有情侣这样做。或者是现在。

人类实际上已经尝试了大部分的事情,然而现在的优势在于妳有机会选择。我们有很多经过尝试与测试的备选方案,我们也知道这些方案有效。妳可以找到一个适合妳自己、健康与生活方式的可靠方法。

也许妳认为理所当然,但是避孕实在令人不可思议。它让妳选择是否要生孩子,同时也不会影响妳的性生活。如果妳想生孩子,妳可以选择时间、对象和多少孩子。体外射精、避孕隔膜及植物与蜂蜜的组合可能有一定的效果,但与一九五〇年代所引进的避孕药有很大的不同。

这是一场革命。当时的口服避孕药是一种有效的避孕方式,比现在更厉害,因为经过多年的尝试和测试。避孕药改变了女性选择何种关系的能力。她们可以控制自己的性生活并规划适合自己职涯与经济状况的家庭组织。从此,许多新的避孕方法开发,包括植入式避孕棒和荷尔蒙子宫环(intrauterine device,IUD)等长效方法。挪威卫生当局现在推荐这两种方法作为他们的首要选择。

以上让你对历史背景有一点认知。现在是时候谈谈这个情况了,我们必须要完全坦白。避孕这些事情真的枯燥乏味,非常无趣。在本书的这个章节,我们尽力解释不同避孕方法之间的差别。我们写了一些如何使用的方式,并提供了一些技巧和窍门当作额外奖赏。而这全都是非常技术性的内容。很抱歉但我们不得不说,避孕章节的第一部分可能对你们来说会是这本书最无聊的地方。即便如此,我们仍决定纳入书中。为什么?嗯,大概是因为这是我们必须去写的东西当中最重要的内容。

毕竟,我们知道年轻女性在想什么,而其中有许多人对避孕有些复杂的疑问。这并不意外,因为所有女性,出于某种因素,被要求在没有指导的情况下,凭直觉去理解这个复杂的事情。我们也知道,关于避孕的迷思有难以置信的数量存在并流传著,许多人由于胡乱使用而饱受不必要的副作用,或者因为资讯不足而感到不安。我们不知道是不是开立避孕处方的医疗专业人员所提供的资讯太糟还是太少,又或是因为一次采取太多的措施所导致这些状况。

这个章节我们的目的在于给妳一个避孕的基本介绍,让妳有方法替自己做选择。避孕正处于不断发展的过程,我们建议妳听取医疗专业人员的建议,他们可能对妳所感兴趣的避孕方法有更新、更详细的知识。





荷尔蒙避孕法




防止怀孕的荷尔蒙避孕法是什么呢?当妳每天早上吞下避孕药,每三周插入一次阴道环,或将植入式避孕棒放进妳的手臂时,进入妳身体系统的到底是什么呢?

与卵巢产生的激素相同,荷尔蒙避孕含有非常低剂量的激素,与控制月经周期有关。所有类型的荷尔蒙避孕都含有黄体制剂(progestin)。这是身体制造黄体素的合成版。有些避孕方法甚至包含另一种荷尔蒙,雌激素。这些被称为复合避孕法,而只含黄体制剂的药物被称为纯黄体制剂产品。





雌激素避孕




复合避孕法分为三种:复合避孕药、阴道环与避孕贴。复合避孕的优点在于雌激素可以控制妳的出血。缺点是并非每个人都可以使用雌激素,妳在之后的部分可以读到更多内容。

复合避孕药是最常用的复合避孕法,种类很多,每种都略有不同。首先,这些药丸使用不同类型的雌激素和黄体制剂。再者,不同的复合药丸有不同的黄体制剂和雌激素剂量。这两种不同都会影响妳所经历的副作用,无论是正面还是负面的影响,但是妳无法事先知道该类型的复合药丸对妳是否造成效果。直到找到适合妳的品牌前,需要反复尝试。如果妳开始服用复合避孕药,挪威卫生当局会建议妳尝试含有左炔诺孕酮(levonorgestrel)的黄体制剂,也就是乐盈肌(Loette)、欣无妊(Microgynon)和同等产品。

复合药丸有两大类,分为多相和单相。但究竟是什么意思呢?

大多数的药丸属于单相。如果妳使用单相类型,那么从药丸包装的哪个位置打开服用并不重要,因为每颗的剂量相同。换句话说,包装里所有药丸都是一样的。大多数单相药丸的设计,可以让妳在某种程度上建立一个固定天数的人造月经周期。大部分类型属于28天的周期。因此妳服用21天周期药丸的期间内,妳不会有任何出血的情形。最后7天是所谓的无药丸周。接著妳也可以服用一些包装里的糖衣锭或暂时停止。在没有荷尔蒙的日子里,子宫通常会排出子宫内膜,所以妳会流血。欣无妊与优思明(Yasmin)就是使用21+7天的单相药丸模式。有些单相药丸已经排列妥当,所以妳可以连续服用24颗,然后休息4天。优悦(Yaz)和佐宜(Zoely)属于使用24+4天组合药丸的例子。如果妳不想要任何出血,妳可以不用间断直接服用新的一包药。之后我们会提到更多。

多相药丸里的每一颗药丸所含的荷尔蒙剂量不同,关于服用药丸的天数和出血天数,每个品牌都有自己设计的周期。所以妳不可能随时使用多相药丸,妳必须仔细参照说明。如果妳使用多相药丸,阅读服用资讯并正确使用特别重要,尤其是妳打算避开经期的话。目前在挪威销售的多相药丸仅有Synfase和Qlaira。

当妳使用复合药丸时,即使是不服药的期间,也可以保护妳避免怀孕。所以妳可以随时做爱,不必使用其他避孕措施来预防怀孕。但是,如果妳没有准时服用药丸,可能会失去防护力。妳能够错过药丸的数量与可能怀孕的机率取决于药丸的种类,所以请参考服用资讯与医生、护士或助产士的指示。错过服用药丸时机导致预防怀孕的剂量不足,我们称为避孕失败。

阴道环是一个插入妳阴道的塑料环。它看起来像是一个义大利面细度的环形柔软甜甜圈。目前在挪威只出售一种阴道环⸺舞悠(NuvaRing)。妳只须用两根手指一起按下阴道环,将其推入就能插进妳的阴道。当妳释放握把时,阴道环会弹回原来的形状,调整至贴合阴道内壁,并保持原状。要拿掉的话,只须用中指轻轻取出即可。

阴道环同时含有雌激素与黄体制剂。荷尔蒙通过阴道黏膜,最后进到血液当中。很多人认为有东西围绕著阴道听起来不太舒服。她们也在猜想阴道环是否会消失在阴道里。

一旦插入阴道环,妳就不应该注意到它的存在,就像使用卫生棉条一样。但还是要小心!虽然这种情况并不常见,却有一些阴道环脱落掉入厕所的例子。我们一位女性友人就发生这个情况,在她晚上到市区玩乐的时候。她告诉我们直到隔天下午她才注意到这件事。当妳喝酒的时候,很容易变得比平常少一点警觉性,接著可能会发生倒霉的事。现在养成将手指放入阴道的习惯,检查一下阴道环的位置,会是个不错的方式。

与大多数单相药丸一样,妳应该连续使用阴道环21天,也就是连续三周。妳可以使用相同的阴道环三周,接著停用7天让经血出来。如果妳想避开经血,妳也可以直接放一个新的阴道环,不要中断使用。

虽然妳不会注意到阴道环的位置,但妳的伴侣可能会在阴道性交时感受到它。所以有些女性喜欢在性行为前将其取出。这么做完全是安全的。妳一次可以拿出阴道环3个小时,但是务必记得要在3小时后把它放回去,否则妳将失去避孕的防护。

荷尔蒙贴片是直接放在皮肤上,让荷尔蒙穿透妳的皮肤进入血液。挪威贩卖的贴片叫作以芙(Evra)。妳可以每周使用1个贴片,如同阴道环和大多单相药丸,妳应该连续21天让荷尔蒙进入体内。所以妳必须使用3个贴片,每周1次,最后停用7天。如果妳忘记及时更换贴片或是贴片脱落,可能会导致避孕失败。



复合避孕法如何预防怀孕?

我们体内已有的荷尔蒙可能会阻止怀孕,看起来有点奇怪,不过复合产品中的黄体制剂与雌激素有著非常好的效果。

复合避孕药的主要作用是防止月经周期中的排卵期发生,大约是每个月一次。如果妳在排卵前5天左右(包括排卵的那一天)发生无防护性行为,妳可能会怀孕。这个时期称为受孕窗口。

使用荷尔蒙避孕可以与怀孕的情况做对照。当妳怀孕时,妳的月经周期停止,就好像有人按了暂停钮。如果妳的月经周期停止,没有排卵,而没有排卵也就没有受孕窗口或受精的可能。

当妳怀孕的时候,体内自然产生的黄体素是造成月经停止的原因。黄体素告诉大脑的脑垂体(看起来像阴囊的那个地方),不再产生滤泡刺激素(FSH)和黄体生成素(LH)。妳可能还记得,这些荷尔蒙是维持月经周期的必要条件。没有FSH就没有滤泡期,而没有LH也就没有排卵期。

荷尔蒙避孕中的黄体制剂和怀孕时身体里的黄体素一样。黄体制剂告诉大脑是时候停止月经周期一阵子。在某种程度上,妳可以说复合避孕法使计让身体以为怀孕了。

复合避孕以好几种方式防止怀孕,不只是停止排卵而已。经过性交后,精子细胞必须经过子宫颈游向子宫。子宫颈里有黏液。复合避孕中的黄体制剂使黏液变稠,使精子细胞更难游入子宫。此外,子宫内膜比平常来得更薄。这让受精卵难以紧贴在子宫内膜上。

雌激素通常负责子宫壁或子宫内膜的生长,而这些内膜后来会成为妳的月经。复合避孕法中的雌激素让子宫内膜每个月增长一些,所以大多数使用复合避孕的女性在使用后短暂休息7天或7天以内时也会出现月经。





无雌激素避孕




无雌激素避孕的优点是任何人都可以使用,甚至是因为某种原因不能服用雌激素的妇女。如荷尔蒙子宫环和植入式避孕棒的长效避孕方法,都没有雌激素,而且提供最有效的避孕防护方式。这就是为什么挪威卫生当局建议大家作为首选。使用无雌激素避孕的缺点是妳可能无法像使用复合避孕一样控制出血。换句话说,如果妳使用无雌激素避孕,妳不能决定经期的时间。一般来说,使用各种形式的荷尔蒙避孕法,出血会比平常要少得多。在我们的印象里,一些使用植入式避孕棒的女性有持续性出血异常的问题,但是这对使用荷尔蒙子宫环的女性来说不是问题。再者,这个问题与实验及错误有关。

植入式避孕棒是含有黄体制剂的塑料棒。挪威所贩售的品牌为Nexplanon。它使用一种注射器放入上臂的皮下组织,可以维持三年的时间。同时不断释放一点荷尔蒙,让血液中的量达到稳定的低水平。植入式避孕棒目前是市场上最安全的避孕方法。一旦在妳的手臂上,就不会出差错。只要它还在妳的手臂里,植入物中的黄体制剂会停止妳的月经周期,让妳停止排卵。

荷尔蒙子宫环是一个小型T状物体,由经过训练的医疗专业人员将其放入子宫。子宫环主要在生殖部位局部释放低剂量的荷尔蒙,虽然量少却能通过子宫黏膜并被吸收进血液当中。在血液里循环的荷尔蒙剂量非常低,许多经历其他避孕方法副作用的人可能因为改用子宫环而减少使用上的问题。妳可以依据所选择的子宫环类型维持三到五年之久。目前挪威市面上有三种类型。其中之一,可以使用五年,叫作蜜蕊娜(Mirena)。它是荷尔蒙剂量最高的子宫环,所以特别适合想要少量出血的女性。许多女性发现,使用蜜蕊娜后月经完全停止。

接下来是Kyleena,同样也能维持五年,但是它专门为没有生产过的妇女而设计。它比蜜蕊娜小,荷尔蒙的剂量也更低。最后一种,可以维持三年的小蜜(Jaydess),同样也有非常低的荷尔蒙剂量,而且很小。尽管小蜜与Kyleena特别贩售给没有生产过的女性,但是年轻女性也可以选择使用蜜蕊娜。

比起其他两种子宫环,蜜蕊娜相对来说有点大,有些人可能会在插入后感到不适,而另一方面,能够让妳更好控制出血,此外妳身体里的荷尔蒙剂量与其他避孕方法相比还要更低。荷尔蒙子宫环只适合已经生产过的妇女,这只是一个老旧的迷思!

有些女性可能会发现使用子宫环后就不再排卵,但并非所有的子宫环都有这种现象。这当然是暂时的,也取决于子宫环中荷尔蒙的剂量。当妳使用蜜蕊娜时,因为剂量略高,所以会比较常见到停止排卵的情况。通常小蜜和Kyleena含有的黄体剂量太低,无法影响大脑的脑垂体,但这并不意味著它们不能发挥作用。事实上,子宫环最重要的功效在于局部:黄体制剂让子宫颈内的黏液不能穿透精子细胞。同时,子宫内膜变薄,让任何受精卵难以生存。

三种荷尔蒙子宫环都提供不错的避孕方法,以及长效、可靠的避孕防护。大多数女性会比以往经历更轻微的出血与更少的经痛,而且许多人还会发现,由于荷尔蒙剂量低,与其他荷尔蒙避孕法相比,其副作用更少或不再严重。最常见的副作用,尤其是使用小蜜和Kyleena,则是长斑和不规律出血。

如果妳担心插入子宫环会很痛,那么在1小时前服用止痛药可能是一个好主意,因为有些人会在插入后的一段时间有经痛感,但是疼痛感很快就消却。在那之后,妳不会注意到它的存在,除了在阴道最深处妳能感觉到两根小小的线从妳的子宫颈延伸。这是定期更换子宫环时用来让医生移除的牵引线。

无雌激素避孕药是一种必须每天服用的避孕药。你永远没有停止服用而造成月经恢复的时候。也没有必要每天在同一时间服用。只有在服用最后一颗药丸后的36小时内,才有怀孕的风险。无雌激素荷尔蒙丸里黄体制剂的作用方式与植入式避孕棒相同:它影响大脑的脑垂体防止排卵。此外,子宫颈黏液变得难以穿透,子宫内膜也变薄。

迷你丸(mini pills)也是一种每天服用的避孕药,没有暂停服用导致月经恢复的情况。迷你丸的黄体剂量低于无雌激素避孕药,所以妳每天必须在同一时间服用避孕药会比较好。妳会有3小时的受孕窗口,所以很容易造成不正确服用药丸与怀孕的风险。





避孕针(contraceptive injection)需要由医生或其他医疗专业人员施打,最迟要在使用上一针的十二周以内注射。所以妳必须每3个月去找一次医疗专业人员接受新的注射。荷尔蒙注射含有大量的黄体剂,足以防止排卵。它也适用于子宫颈黏液和让子宫内膜变薄。一般情况下,不推荐二十五岁以下的女性使用荷尔蒙注射剂,因为荷尔蒙剂量太高,会影响身体内骨骼的增长。





非荷尔蒙避孕




妳是否想要一种不含荷尔蒙的替代品?

每个不含荷尔蒙的避孕方法间几乎没有共同之处,所以人们有很多使用的选择。有些女性会受到荷尔蒙避孕的副作用所苦或恐惧副作用,而影响她们的决定。预防性传染疾病是使用保险套的一个好理由。其他女性则因为对伴侣或家人隐瞒使用避孕产品感到担忧,所以更希望能像以前一样继续月经周期。





保险套




保险套能阻止精子细胞进入子宫。

保险套有阻碍的功能,因此被称为障碍避孕法(barrier method)。目前,保险套是男性唯一可以使用的避孕方法。

保险套是一种由乳胶或类似材料制成,套在阴茎上的袋子,并在男性射精时收集精子。性交后,阴茎撤出时应该要把保险套固定住,这样保险套与精子都不会留在阴道内。一旦结束后,只需脱下保险套,在上面打个结,然后直接扔到垃圾桶里。不要把保险套扔进马桶。在妳没有料到的时候,它们会有浮起来的特性,无论在共同的住处,还是在父母的房子里,都不会是件好玩的事。

保险套是唯一能保护妳免于性传染感染的避孕方式。换句话说,保险套可以保护妳免于疾病和怀孕。听起来好像应该放弃其他避孕产品一样,只要一直使用保险套就好。但不幸的是,只用保险套的人却发生不少意外。它们可能会裂开、脱落或毁损,所以许多人会选择使用保险套同时结合其他避孕措施。

很多人使用保险套的方式不正确,这代表著出错的可能性更大了。考虑到这一点,以下是我们正确使用保险套的方法。





保险套课程




1. 检查日期戳印,旧的保险套更容易破掉。

2. 小心打开保险套包装。注意,不要用锋利的指甲、牙齿或珠宝 以免划破保险套。

3. 一旦阴茎变硬,将保险套像墨西哥帽一样放在它的顶部。





4. 挤压保险套顶部推出空气。空气可能会导致保险套裂开。

5. 从阴道抽出阴茎时,请将保险套牢牢固定,否则精子可能会流入阴道。

6. 保险套应该保持在性行为时全程中使用,以防止怀孕或性传染疾病,而且只能使用一次。



还有其他类型的障碍避孕法可供女性使用。我们已经提到的避孕隔膜,亦即在瑞典流行的叶子和蜂蜜的现代版。虽然可以在网上订购,但是挪威的避孕隔膜并不是特别容易取得。还有一种反向保险套,就像一个放在阴道内的袋子,而不是在阴茎周围。这称为女用保险套,也提供保护并防止疾病。据我们所知,它在挪威市场上无法买到,在我们地区也甚少使用。





安全期⸺找到妳的受孕窗口





月经周期里妳可以怀孕的时间称为受孕窗口。为了避免因性行为而怀孕的机会,有些避孕方式与找到受孕窗口有关。

找到受孕窗口有不同的方式。妳可以使用月经日历、每天早上测量体温或检查自己的子宫颈粘液。人们经常将这些方法结合起来以提升更高的可靠性。

这些都不是好的避孕方式。我们认为,对于绝对不想怀孕的女性来说,这么做太不可靠了。根据世界卫生组织(World Health Organisation,WHO)统计,所有使用基础体温法的女性,有25%的人会在一年内怀孕,换言之,就是四分之一。数量非常多,但是如果对妳而言完全不危险的话,妳可以评估妳是否愿意承担怀孕的风险。

鉴于许多人对这些避孕方法的兴趣增加,我们希望能够对这些原理提供一个简短的说明,尽管我们不推荐它们。虽然这些方法不能提供可靠的防护,但的确具有一定的价值。尝试怀孕的女性也可以使用这些方法来判断自己的受孕窗口,变得更容易受孕。

透过经期日历计算排卵日的人首先会以月经周期的资讯当作标准。排卵日通常在每个周期的同一时间发生,也就是在月经前14天左右。

那些使用基础体温法的人,首先会以月经周期内我们略有改变的体温当作依据。

确切来说是0.3度!妳可能还记得,月经周期有两个阶段。第二阶段前的1〜2天,妳的体温会上升0.3度,并持续10天左右。第二阶段开始时,大量的LH从大脑释放到血液中。LH的急剧增加引发排卵,通常在荷尔蒙浓度上升后1〜2天发生。换句话说排卵日会在体温升高后2〜4天之间。每天持续测量妳的体温,可以发现在月经周期中,妳通常会以排卵日为基准,推算受孕的最佳时间。

事实上,在排卵时,妳也可以观察子宫颈黏液的变化。为了让这个方式有效,妳必须每天检查分泌物的情况,并找出变化。就在排卵日前,妳的分泌物变得光滑、粘稠,妳可以透过手指伸展,长度通常是好几公分。刚排卵的时候,妳的分泌物会变为乳白色。这种方法需要妳非常熟悉分泌物的状况,并花时间研究它在整个周期过程中的变化。妳应该知道,除了周期以外,影响分泌物变化还有其他的原因。例如各种疾病会影响浓稠度,使妳很难判自己身处周期中的哪个阶段。

也许妳现在认为听起来很复杂⸺的确这也是个问题。这其中有很多的计算、推测和记录,同时也有好几天的误差,所以会有很多搞错的机会。除了对打算使用这个方法避孕的女性非常严格以外,也有条件地要求女性经期必须完全规律而且每次周期只能有一次排卵。由于上述的条件,这些方法并不会特别安全。

正常来说,一次月经周期只有一次排卵,然而可能会有多次排卵的情况发生。在排卵前至排卵后一天的1〜5天内发生性行为,有可能会怀孕。

月经周期也可能借由外部因素如压力、体重变化与疾病改变。通常年轻女性的月经周期比老年女性更不规律。所以比起年长女性,这个方法较不适用于年轻女性。





含铜子宫内避孕器




含铜子宫内避孕器是一种不含荷尔蒙的替代品,我们会推荐妳使用。在所有使用含铜子宫内避孕器的女性当中,只有不到1%的人在一年内怀孕。像荷尔蒙子宫避孕器一样,含铜子宫内避孕器也是由医生或其他医疗专业人员放入子宫的小型T形物体。不同之处在于含铜避孕器有铜线包覆。妳可以将含铜避孕器存放在子宫长达五年,且它在整个过程中提供良好的防护。含铜避孕器的底部有两条线悬挂,并附著在子宫颈开口,以便妳可以用手指检查避孕器是否还在原位。荷尔蒙避孕器也是相同的构造。当含铜避孕器需要移除或更换的时候,医生会透过那两条线移除更换。

含铜避孕器有好几种类型,在品质上的差别并不大。然而却有价格的差异,最便宜的约为350克朗。如果将避孕器放置五年,每个月的成本就低于6克朗,与其他可靠的避孕方法相比,含铜避孕器极其便宜。

我们不知道含铜避孕器防止怀孕的原因与运作。我们所知道的是含铜避孕器造成子宫轻微发炎,因而改变了里面的环境。不知为何,这样的情况可以防止怀孕。有一种理论认为,子宫开始排放杀精子物质,可能是因为发炎,或者铜本身可能会杀死精子。另一种理论则是含铜避孕器防止任何受精卵附著在子宫壁上。

和使用荷尔蒙避孕法的人不同,使用含铜避孕器的女性每个月都会正常排卵。含铜避孕器对大脑或卵巢没有任何的影响,它只在子宫有局部效应。

含铜避孕器不会产生荷尔蒙副作用,但这并不意味著没有副作用。许多女性经历比以前更严重的出血和更严重的经痛。由于这些状况,100名女性当中有2〜10人决定在第一年取出避孕器,因此,含铜避孕器通常不适合有这类问题的女性。

含铜避孕器有许多的迷思。最普遍的一种是,如果妳以前没有生产过,就不能使用它。就算没有孩子,使用荷尔蒙避孕器或是含铜避孕器完全不会产生问题,即使妳很年轻也欢迎妳来尝试含铜避孕器。含铜避孕器是行之有年的避孕方式,而且近年来也变得更小、更可靠。

从纯粹的实用观点来看含铜避孕器可能也会有不舒服的感受,因为它必须通过狭窄的通道进入子宫放置。有许多人体验过严重、短期的经痛感。可能得事先服用止痛药。同时尝试与真正放松也很重要。与替妳安装避孕器的医生讨论一下。

如果避孕器底部的两条线突然消失了,妳应该与医生联系。这可能表示含铜避孕器被推出妳的子宫,妳不再有避孕的防护。显然有5%〜10%的使用者发现含铜避孕器就这么掉出来。在极少数的情况下,妳找不到线头的话代表妳可能已经怀孕。如果怀孕的话,那两条线其实已经被拉进子宫里了。





紧急避孕⸺慌乱状态




星期天上午,妳在昨天晚上发生了性行为也没使用可靠的避孕方式。妳没有特别渴望怀孕,所以令人如此害怕,让妳开始胃痛。妳不是第一个经历这种状况的人,也不会是最后一个。有时候事情就是会出错,也是为什么我们有紧急避孕措施。妳可以在无防护性行为后,或是避孕失败时使用。

避孕失败的定义因避孕的种类而异。可能是错过服用药丸的时机,阴道环脱落或者保险套裂开。重要的是要熟悉妳所使用的避孕方法,这样在避孕失败的当下才能立即意识到。两颗避孕药的间隔超过多久才会被认定是避孕失败呢?阴道环停留在阴道外面多久才算避孕失败?问问妳的医生、护士或接生员关于妳避孕方法失败的一些规则。

当妳使用荷尔蒙避孕法失败(例如错过服用药丸),通常会导致排卵。很多人在避孕失败后不做紧急避孕措施,是因为她们不明白可能会有怀孕的风险。从发生性行为后可能已经过了好几天的时间她们才发现忘记吃药。但是记住,精子细胞为了等待卵子可以在子宫里存活五天。如果妳避孕失败导致现在排卵的话,这代表妳可能在五天前透过性行为怀孕了。

在挪威,人们称紧急避孕为"后悔药"。我们应该停止称呼这个名字。"后悔药"是一个会联想到撅起嘴唇和扬起眉毛模样的呆板说法。这意味著妳做了应该感到后悔的事⸺但妳真的没有。妳才刚发生性行为,如果这是一个正向的经验,我们就没有理由后悔。另外,当妳拿起包装服用今天的药丸时,发现妳在过去一周已经错过三颗药的心情并非后悔:而是恐慌。这就是为什么我们选择在这本书当中称为恐慌丸。

我们也没有特别拘泥这个词⸺"事后丸"。这听起来不错,又方便进行,就好像妳在每次性行为后的早晨吃下,而不用其他避孕措施。但重要的是,不要太容易依赖恐慌丸。它不像一般避孕法有效,虽然不是那么危险,却仍有一些副作用。只有在其他避孕方式失败时,才能使用紧急避孕。它不应该取代正常避孕方式。

紧急避孕措施有三种:两个不同的药丸与含铜避孕器。第一类避孕药含有一种左炔诺孕酮的物质,属于黄体制剂的一种。换句话说它包含与荷尔蒙避孕相同的物质,黄体制剂的量要高得许多。第二类药丸含有醋酸乌利司他(ulipristal)的物质。这种物质影响身体自然黄体素的运作。





第一类恐慌丸:左炔诺孕酮




含左炔诺孕酮孕的避孕药是挪威最常贩售的紧急避孕药。在药局的柜台、加油站和一些超市皆能购入。在挪威,这种恐慌丸是以后安锭(Norlevo)的品牌名进行贩售。





第一类恐慌丸的作用是推迟排卵。问题是,如果妳已经排卵或者即将排卵的话就没有效了。妳可能还记得在月经周期的章节里提过,女性在排卵前会经历LH急剧增加的情形。一旦LH开始上升,含左炔诺孕酮药丸将无法阻止排卵。

妳很难知道自己是否已经排卵。排卵通常一个周期发生一次,但它可以发生很多次,也只有周期完全规律的女性知道自己排卵的多寡。

因此,这种药并非完全可靠,虽然它确实会降低妳怀孕的机会,所以服用它绝对是明智的选择。妳愈早服用效果更好。最好在无防护性行为或避孕失败的24小时内服用。尽管如此,恐慌丸在无防护性行为或避孕失败后三天(72小时)内是有效的。避孕药降低效用的机率会随著更多时间过去提升,所以随时放一颗在妳的盥洗包里是个好主意。

单次生理周期中多次服用左炔诺孕酮药丸完全没有大碍。



优点:取得容易、不影响其他避孕措施、可以在月经周期里多次 服用

缺点:不太可靠

注意事项:三周后要进行验孕!





第二类恐慌丸:醋酸乌利司他




在挪威,含有醋酸乌利司他的药丸是以艾伊乐(EllaOne)的名称进行贩售。有效的防护时间为无防护性行为或避孕失败后的五天(120小时)内。艾伊乐在药局的柜台上贩售,却尚未在商店或加油站里贩卖。





和左炔诺孕酮的恐慌丸一样,醋酸乌利司他药丸也能延后排卵。这两种类型的恐慌丸之间的差别在于,这类的药丸可以在接近排卵期服用,而且仍然有效。妳可以在快要排卵时服用。然而,如果妳已经排卵的话这种药丸就会失去效用。换句话说,醋酸乌利司他药丸即使身体的LH浓度上升仍然有效。这让药丸变得更加有用,它会预防更多怀孕的次数。

当然,这种药丸也有一个主要的缺点。主要的问题在于,它与荷尔蒙避孕有严重反应。首先,它会在服用后影响一般避孕方法在妳身体的运作模式。这意味著在服用恐慌丸后妳只能使用保险套,因为妳的荷尔蒙避孕法可能无法运作。妳必须按照这个模式去进行的时间,取决于妳采取什么样的避孕措施。

同时妳所使用的荷尔蒙避孕方式也会影响紧急避孕药的效果。因此,它是双向的。这表示妳不应该服用艾伊乐后使用荷尔蒙避孕。事实上,新的研究显示,使用荷尔蒙避孕可能破坏药丸对排卵的影响,阻止延迟排卵的功能。服用艾伊乐后妳应该经过五天再开始或继续使用荷尔蒙避孕。

醋酸乌利司他紧急避孕药只能在每次月经周期中服用一次,因为从未有研究进行一个周期内使用多次的测试。

这并不代表避孕药危险,只是没有人知道它是否在一个周期内发挥作用多次。避孕药会影响其他荷尔蒙避孕法,如果妳试著在服用醋酸乌利司他后使用左炔诺孕酮的话,它也能影响第一类药丸的效用。如果妳已经使用醋酸乌利司他,避孕失败的话最好使用含铜避孕器。



优点:比左炔诺孕酮丸有更好、更久的效果

缺点:使用荷尔蒙避孕药会有严重反应

注意事项:三周后要进行验孕!





含铜子宫避孕器




虽然含铜避孕器是紧急避孕最安全的形式,但很少使用。如果妳需要紧急避孕,我们建议妳考虑含铜避孕器,因为它有99\%的效用。它能阻止受精卵附著在子宫上。

含铜避孕器由医疗专业人员放入子宫,所以在妳有无防护性行为后,可以向妳医生挂急诊并解释发生何事。妳还可以求助紧急手术或到年轻人会去的诊所。含铜避孕器有效的防护时间为无防护性行为或避孕失败后的五天内(120小时)。它会有效是因为在排卵的第六天里受精卵不会附著在子宫壁上,因此在一些性交后超过五天的案例中使用含铜避孕器当作紧急避孕措施是可行的,前提是妳知道排卵的时间。

含铜避孕器最晚必须在排卵后的第五天使用。

关于含铜避孕器好处,除了作为紧急避孕非常有效之外,妳还可以把它留在子宫内当作一般避孕方式。如果妳不想把它当作一般避孕方式,可以快速地将它取出。



优点:高可靠性,未来五年能够以避孕方式运行

缺点:不易取得、需要处方、必须由医生、护士或接生员放入





注意事项




很多人认为在服用恐慌丸后就绝对安全,但事实并非如此!

紧急避孕措施减少怀孕的风险,却不像一般避孕方式来得有效。紧急避孕后进行验孕非常重要。无论是否来月经我们建议妳去做验孕测试。如果妳的伴侣或是有人服用了恐慌丸,妳能提醒她去做验孕就太好了。

验孕测试要有效的话,妳必须在使用紧急避孕措施后等待至少三周。在紧急避孕后马上验孕是毫无意义的,因为不可能那么快就检测到妳是否怀孕。

紧急避孕措施有副作用。最常见的是不规律出血。恐慌丸延迟排卵时间,同时也延迟妳的经期。不规则出血不会有危险,但是却造成麻烦。幸运的是,这不是长期的问题,而且会没事。有些人还发现,恐慌丸会让她们觉得恶心。如果妳服用避孕药后没多久就呕吐的话,妳必须使用另外一种。遵从药丸的服用资讯与医生的指示进行。

含铜避孕器不含荷尔蒙,然而即便如此,它通常在一开始就改变妳的月经周期模式。如果妳想维持含铜避孕器当作一般避孕与经历月经的变化,那么我们建议妳三个月后看看有什么样的改变。经血规律通常会随著一段时间慢慢稳定下来。





有其他更好的避孕方法吗?




我们谈过许多关于自身的不同与各种避孕方法分别适合哪些女性,但并不表示所有的避孕方法都是一样的好。而放在阴道里的叶子与蜂蜜组合已经不再流行,以及根据安全期来避孕的方式导致许多意外怀孕的原因,就是使用方法的本身。

研究人员认为,目前世界上最好的避孕方法,就是植入式避孕,紧接在后的则是荷尔蒙避孕器。使用植入式避孕棒的女性怀孕的机率是最低的。很多人不知道避孕品质是如何测量的。我们该如何确定哪些避孕方法是最好的呢?植入式避孕棒比药丸更好到底表示什么意思呢?让我们解释一下:当我们说"最好"的时候,我们单纯是指避孕方式运作有多好,也就是在防止怀孕方面上有多好。我们不是在谈论副作用或者有多少人喜欢哪种避孕方法。无论喜欢与否都是主观的。但是,避孕的效果有多好是客观的,只要透过研究观察有多少女性使用特定避孕方法而怀孕就能衡量。很难判定妳客观上所偏好的选择会是最好的避孕方式。我们的目标在于找到尽可能可靠而妳也开心的避孕方法

研究人员使用一种叫做佩尔指数(Pearl Index)的标准,用来评估和比较避孕方法的不同。佩尔指数是衡量避孕方法有效程度的指标。有效的评估,我们指的是避孕的效果⸺不是危不危险的问题,避孕不危险。

所以,佩尔指数是衡量有多少女性使用特定避孕方法而怀孕的指标。佩尔指数的确切、具体的条件为100名女性在一年当中使用避孕而怀孕的次数12 。

例如,如果妳想测试一种新型避孕药的效果,妳请一群女性来测试药丸,然后看看她们有没有怀孕。根据许多研究结果,统计人员可以根据效果有多好来排名这些避孕方式。但是,是什么原因导致各种避孕方式之间有差别呢?

有两种因素能够判断避孕方法是否有效。第一个是使用的方式。因为有些避孕方法可能会不当使用,使得比正确使用效果来得较差。

就拿体外射精举例。其目的是男性在高潮前将阴茎从女性阴道中拔出,最后让精子射在床垫上、她的胸部或是其他有趣的地方。然而我们许多人会发现,通常拔出的时间点很容易发生在在高潮后,而非在高潮前。一时冲动的情形下,晚一点从阴道拔出实在很诱人,如果搞砸了一次,可能也足以让妳怀孕。错误使用让体外射精变得难以依靠,而且还远远不受医疗专业人士与不希望怀孕的人青睐。即使非常妥善又有效使用,人类的犯错能力总是令人唏嘘。

避孕药,避孕最常用的方法之一,当讲到使用者错误时,它也是罪犯之一。在事实上使用者错误只是它的中间名。人们非常容易就会错过一、两颗药丸;在某人床上醒来,离自己牙刷和药包很远的每个女性都可以感同身受。很多女性在无药丸周仍服用药丸也因此怀孕。她们摆脱每天服用药丸的习惯,然后她们搞不清楚停用的时间应该要多久。任何人都有可能发生错过服药的情况。我们都有心不在焉的一天,但有些人却是每一天都心不在焉。

而在另一方面,植入式避孕更加有效的原因在于它直接在妳的手臂里发挥作用,妳什么事也不用做。除了妳忘记更换以外,根本很难去忘记做过植入手术,而且仅仅三年才换一次。因此,植入方式不会有使用者错误。它运作得非常完美,不管妳的例行事项和记忆。

另一个判断避孕方法是否良好的要素则是人们如何使用,而非方法本身。我们把这种要素称为使用者错误。

有些人可能会认为不公平,表示避孕方法不好只是因为使用者搞错;毕竟,并不是方法本身的错不是吗?妳可能会这么认为,但我们认为尊重避孕方式那不存在的心情一点意义也没有。研究显示,只要有这样的可能性,往往会让我们在最后做错事,而这对避孕方法的有效性产生影响。在任何情况下,唯一的目的就是阻止妳怀孕,并不是让妳最喜欢的避孕方式赢得人气竞赛。

确定有效性的第二个要素为避孕方法的实际品质。很多人认为结扎是不希望有(更多)小孩的最有效方式。当一个女人进行结扎,输卵管会被切除,使卵子不能从卵巢传送到子宫,虽然结扎后200名女性会有一名在隔年怀孕。无论是植入式避孕和荷尔蒙避孕器都比结扎更有效。这种类型的错误,和避孕本身的方法有关,而非使用它的人,因此称为方法错误。

使用植入式避孕的人几乎无人怀孕,不过在医学当中没有所谓的对与错。只要有人的地方就会怀孕,无论她使用哪一种方法。不幸的是,只要妳是与男性做爱的女性就无法说不可能;顶多可以说"几乎不可能",还比较可信。

由于有两种不同类型的错误和避孕方式有关,使用者错误和方法错误,其效果也以两种不同的方式进行测量。我们以"完美使用"和"实际使用"来区分避孕方法。完美使用意即使用避孕方法的人完全没有发生失误。没有使用者错误,没有错过服用药丸,没有延迟阴茎从阴道拔出的时间,阴道环没有在城里喝醉时掉进马桶。另一方面,实际使用的结果则是女性竭尽所能正确使用避孕方法,就和一般避孕使用者一样,但还是会在过程中造成一些错误。

完美使用与实际使用的不同,取决于使用该避孕方式犯的错误有多少,在重要和不存在的范围间判定结果。

如果妳的生活作息良好,如果妳没有一丝轻率或心不在焉,如果妳有钢铁般的意志,例如,避孕药,那么妳会怀孕的风险在佩尔指数里会比起"实际使用"更靠近"完美使用"。

只有妳最了解自己。但是如果妳的生活方式稍微难捉摸,值得考虑一种无论做了多少错误还是有用的避孕方法。没有使用者错误的避孕方法,例如植入式避孕棒和含铜避孕器/荷尔蒙避孕器,在完美和实际使用的指标下一样有效,因为无须做任何努力,实际使用仍然能完全发挥功效。

那么,哪些避孕方法最好?妳会看到列有不同选择方法的表格。这些数据是由世界卫生组织(WHO)提供。这些资料在二〇一五年更新,但有可能因为研究人员发现新的避孕方式或对现有方式进行新的研究而有所改变。

当妳在做决定时,它有可能帮助妳了解各种避孕方法的效果有多好。不过,我们建议许多女性可以尝试的最有效方法则是:具有长效作用,没有使用者错误的方式。





避孕方式的有效果





基础体温法与效果




我们特别想对避孕列表里的其中一个方法提出评论。基础体温的方法因为自然循环(Natural Cycles)透过高调的部落客宣传,已经在挪威与瑞典部落格圈引起热烈讨论。自然循环是一款提供以手机纪录每日体温测量来计算安全期的应用程式。

以此为依据,自然循环宣称有效程度为99.9%。

我们不相信这个数字,而且也不是只有我们这么认为。挪威医药局发言人对厂商与宣传自然循环的部落客采取强硬的态度回应。

从表格中妳可以看到根据WHO提供有效程度高达99.95%的唯一方式只有植入式避孕⸺是世界上最有效的避孕方法。下一个最有效的方法,荷尔蒙避孕环,有效度为99.8%。

根据WHO的资料,以实际使用来看,自然循环在基础体温方式上的最佳结果只有75%;换句话说一年当中100名女性当中有25人使用基础体温法而怀孕。这种方法从完美使用的角度来看可以提供达到99%的效度,100名女性当中只有1位怀孕;但是请记住,完美使用只是理论上的概念,不可能在广大女性中达成。

即使是设计精美的应用程式,完美使用基础体温法的概念和自然循环对使用者的众多要求实在难以期待不会犯出错。事实上这个方法充斥著潜在的使用者错误,最明显的是在错误的日子进行性行为的可能性。我们知道,有些使用此程式的情侣最终以怀孕收场,正是因为他们无法控制自己的一时冲动。

许多方法错误造成:为什么有些女性永远无法达成高效益,即使她们善于正确使用避孕方式。

如果妳有发烧、月经不规律或妳的月经周期过程中有额外的排卵情形,基础体温测量和安全期方法也就无效。这是妳无法控制的因素。

尽管如此,透过借由应用程式的辅助实际使用基础体温法有可能降低怀孕的机率。

毕竟,应用程式消除了一些误判和类似的可能性,并使用较早的温度记录、计算怀孕的可能。由自然循环资助的一项研究发现,应用程式在实际使用的效果从75%增加至92.5%。所以在一年的使用过程中有7.5%的女性怀孕。如果他们的数据是正确的,这差不多相当于避孕药的实际使用效果。如果妳问我们,为什么公司不在广告上使用这个数据而宣称99%的比例,我们也觉得有点奇妙。

希望使用自然循环的人首先需要有规律的月经周期,生活在极为有序的作息,有足够的时间,每天早上测量温度。必须要有钢铁般的意志,拒绝在错误的时间发生性行为(或必须使用保险套),同时必须做好可能怀孕的打算。如果这听起来像妳一样,没有什么能阻止妳尝试自然循环或其他基础体温测量方式。如果妳想不惜一切成本避免怀孕的话,我们会建议妳选择别的方法。





* * *



12	通常大家会有个误解,理论上最高的指数跟百分比一样为100。但是所有参与研究的女性怀孕的话,佩尔指数会来到1200。有点令人困惑但一点也不重要。除非妳是斤斤计较的书呆子,就和我们一样。





荷尔蒙避孕对月经的影响




荷尔蒙避孕会影响妳的月经周期。因为每月的出血状况会改变,所以妳会注意到它。

大多数女性的经血量变得更少量或经期变短,但并非每个人都会如此。出血也可能变得不规律或完全消失。因为失去或避开月经的迷思让很多女性觉得有点可怕。许多人认为,难道出血的现象不就是因为是自然的;我们的身体难道需要它吗?我们真的要对大自然这么随便吗?

如果妳还记得月经章节的内容,那么经血对你来说有好处。如果妳正在使用荷尔蒙避孕的话的确是如此。使用荷尔蒙避孕,妳的月经周期不再正常,大多类型的荷尔蒙避孕会完全停止月经周期。所以,这种情况发生的出血也不再是正常月经出血,我们会称为停药性出血(withdrawal bleeding)。

我们先从使用复合药丸时,妳的月经会发生什么变化开始说起。五十年前设计避孕药的研究人员规划了每个月一次的无药丸周让女性可以有停药性出血。他们认为如果荷尔蒙制造类似每四周出血的月经周期模式,那么就更容易让人们接受服用药丸当作避孕的方式,但是即使模仿自然周期,却一点也不"自然"。出血不自然,避开周期的用意也完全不自然。

通常是雌激素引起子宫内膜生长,这些黏膜就变成妳之后的经血。复合产品的雌激素使子宫内膜每个月增加一点点,所以即使没有正常月经周期,使用复合产品的女性在停用荷尔蒙药丸、避孕贴片或是阴道环7天或更少天数的情况下会发生停药性出血。子宫内膜生比平常增生得少,这就是为什么经血量不需和暂时停用避孕时一样。对许多女性来说一个月一次可能就太多余了。

只要妳随心所欲就可以使用复合产品避开月经多次,甚至连续服用药丸或让它出血,只要适合妳都行。

这既不危险又有效。复合产品的黄体制剂约束了子宫内膜,所以经血不会流出。

如果妳常常使用复合产品并避开停药性出血,妳最后有可能得到所谓的突破性出血(breakthrough bleeding)。

黄体制剂尽可能约束子宫内膜,但最后太过头了。突破性出血意味著妳在荷尔蒙作用的情况下流血⸺换句话说,就是在使用避孕药、阴道环或是避孕贴片的短期停药之外发生的出血情况。可能会是斑点状,即不规则轻微出血(通常是妳内裤上的斑点),或更严重像月经一样的出血量。这是正常的,而它的意思是:是时候休息7天的时间啦!之后妳可以回去继续避开停药性出血。

许多女性认为每个月的出血在使用荷尔蒙避孕时可以知道自己是否怀孕,而避开太久可能会降低潜在怀孕的机率。这个想法并非完全正确。事实上即使在中途开始无药周,它还是有可能完全停止出血。这不代表妳怀孕了。更重要的反而是,它可能会在怀孕期间轻微出血。荷尔蒙避孕出血通常和一般月经不同,是轻微的出血。所以,即使妳在无药丸周出血也有可能怀孕。重点在于妳应该仰赖妳正在使用的避孕方法。如果使用正确,复合避孕法是有效的,但是如果有任何的变化让妳怀疑自己怀孕了,唯一的方法就是去做验孕测试。

很多女性深受频繁突破性出血受苦,这可能会成为长期的折磨。有些人可能发现改变自己的避孕方法会有改善的效果。如果妳在服用避孕药,从低剂量的雌激素到稍微高一点的剂量就有可能改善。最高剂量的雌激素药丸最擅长控制出血。例如很多女性会感受到欣无妊或奥诺康(Oralcon)的控制出血效果比乐婷锭(Loette 28)要来得好。妳可以和医生讨论应该换成哪种方式。

使用黄体制剂避孕与复合避孕下的月经之间有很大的不同。主要的区别在于,妳不能决定自己的月经周期,也不能改变或一直控制。这是因为妳每天服用相同剂量的荷尔蒙,没有休息过。如果妳稍作休息,就会失去防护。这意味著,黄体制剂不再约束子宫内膜,妳随时会流血。当妳使用黄体制剂避孕却一直流血的话,实际上,就是突破性出血,因为没有出现停药性出血的间隔时间。

黄体制剂约束子宫内膜,让经血变得更难渗出。同时黏膜变得比平常更薄。由于黄体制剂当中没有雌激素,所以没有任何东西可以传达生长的指令给子宫内膜。最后转变成没有完全不出血的可能性,许多女性确实会出血。毕竟,雌激素在体内自然发生了。

当妳开始使用黄体制剂避孕时会有点像把月经周期当成俄罗斯轮盘在玩。妳无法事先得知结果,但是它会有三种选择:经常出血、不会出血或是不规则出血。

很多人认为无论使用植入式避孕或荷尔蒙避孕器来停止月经的女性,都是因为可以停止经血而选择这种避孕方式。但是这并不完全正确。很多女性最后没有出血,但也有可能会是极不规则的出血或是以完全正常的周期结束。无论如何出血量都会小于不使用荷尔蒙避孕的方法。

至于复合产品,在妳使用黄体制剂避孕仍流血时并不能排除怀孕的可能。我们收到每三个月规律地做验孕测试女孩所提出的疑问,她们因为避孕的效果而停止了月经。这是不必要又很花钱的一件事。使用黄体制剂避孕时出血并非验孕的好指标。如果妳避孕失败或不确定是否受到避孕防护才要进行验孕测试。

虽然含铜避孕器不是荷尔蒙的形式,妳却有可能发生和月经有关的副作用。不像荷尔蒙避孕导致轻微的出血,很多人会发现当她们使用含铜避孕器时出血加重,经痛也更严重。对先前受过严重、长期或疼痛出血的女性来说更是如此。多达十分之一的女性在使用含铜避孕器第一年因为这些问题才决定移除。





我要如何避开月经呢?




有时候,月经周期不太方便。有可能是因为妳要去海滩度假,和伴侣一起去小屋滑雪之旅,或是因为在考试前一周妳无法承受经血与经痛带来的麻烦。这些都是所有有月经的女性会发生的烦恼,特别是那些受到严重出血和剧痛的人。当你觉得不方便的时候,可以尽量让子宫延迟出血。

使用复合避孕法一直是最容易延迟出血的方式⸺也就是复合药丸、避孕贴片或阴道环。其他人可能使用处方药物来延迟月经。



以下是使用复合产品的注意事项:



单相型复合药丸:通常妳会服用含有荷尔蒙的药丸21〜24天,接著休息7或4天,依据你所使用的单相药丸种类来决定天数。在停用药丸的日子里妳就会流血。如果妳想避开出血,一旦妳完成目前循环的所有荷尔蒙药丸后可以直接服用新的一包。所以,如果妳使用的是21天份的荷尔蒙药丸(例如欣无妊或乐婷锭),就不会有无药丸周。如果包含糖衣锭在内,数量达到28颗的话,妳可以把剩下的扔掉。如果妳用的是24天份荷尔蒙药丸加上为期4天停药日的优悦或佐宜,妳可以跳过休息间隔,直接再服用新的24天份。如果妳正在服用多相药丸如Synphase和Qlaira,妳也可以避开经期,但在这种情况下,妳会需要稍微更详细的说明。我们鼓励使用这些药丸的人如果有任何问可以去看医生或护士,并详读服用资讯。

阴道环:通常使用三周休息一周,我们可以称停用的日子为无环或无荷尔蒙周。在这一周当中妳会出血。如果妳想避开,可以在三周后跳过无环插入新的避孕环到阴道里。

避孕贴片:避孕贴片通常每周更换一次并维持三周,第四周为无贴片周。在这一周里妳会出血。要跳过出血的第四周只要换上新的贴片就好。



如果妳不使用复合产品的话会是这么运作:



Primolut-N药丸含有延迟月经的荷尔蒙。这可能是不希望使用含有雌激素避孕却想要避开月经一、两周女性的解决方案。

在月经来的前三天妳开始服用Primolut-N:每次一锭,一天3次。这代表妳在月经来临时必须彻底监督。在没有正常月经周期下使用Primolut-N很难成功发挥效果。之后,只要妳想延迟月经,妳需要每天服用三颗,最多用到14天。一旦停止服用药丸,妳会在2天后出血。换句话说,妳不能无限延后月经。

Primolut-N可以透过医师处方获得。除了一些女性,大多数人可以使用Primolut-N;妳的医生会向妳解释。当妳服用Primolut-N时使用保险套来保护妳避免怀孕很重要,因为这种药没有避孕的效果。一次疗程的费用为65克朗。





使用避孕药的最佳方式?




避孕药可能有很多麻烦,却仍然是受欢迎的方式。正如妳在前面章节中所看到,使用避孕药是有可能怀孕的,主要是因为很容易使用不当。

最酷的地方是,使用避孕药怀孕的风险较低,降低异常出血的机率与出血量更轻微。这个方法适用于所有复合产品。也就是说,妳使用避孕贴片和植入式避孕棒会有一样的效果。使用多相药丸的人必须遵照医生、护士或接生员的指令。

只要妳正确使用避孕药与其他复合产品,它们会一直保有效果。就妳所知,复合避孕法有设置停用期间。妳可以使用荷尔蒙三周(21天),接著无论不服用任何药丸或是服用糖衣锭停用荷尔蒙一周(7天)。在这7天里,妳会有停药性出血。如果使用优悦或佐宜妳会服用荷尔蒙药丸24天,并有为期4天的休息期间。

在提到复合避孕法时,21+7或24+4是非常重要的数字,因为这些组合列出了两个重要的限制。

当妳使用复合避孕时,为了使避孕方式有效,妳必须服用至少21或24天的荷尔蒙药丸。如果妳连续使用少于21或24天的药丸(例如,如果妳忘记最后两颗,总共才服用19或22天,而不是21或24天)妳会失去防护并开始排卵。然后,妳有可能怀孕。因此21或24天的荷尔蒙药丸意味著至少服用21或24天。延长使用荷尔蒙的时间没有任何的问题。只要结束妳规定服用的天数,可以连续服用30、50或100的药丸。这完全取决于妳。

7会是一个限制天数(如果妳使用佐宜或优悦的话则是4),这代表停用的休息期间最多为7或4天。不得超过这些天数。如果停用更久的时间,妳会失去避孕的防护。休息大约3天的时间则没有问题。如果,例如,妳有2天的短暂出血,妳可以休息2天后重新开始服用荷尔蒙药丸。但妳绝对不能休息超过7或4天。如果这么做,妳可能会排卵,然后陷入怀孕的风险。

只要妳使用荷尔蒙药丸至少21或24天,并达到单相药丸、避孕贴片或阴道环7或4天的最大休息限制,妳可以使用任何的复合产品。既然有这么多因为搞混无药丸周而意外怀孕的状况,降低无药周的天数会是个好办法。实际上这么做会让避孕更加有效。

一旦妳避开一定的月经时间,妳可能会发生突破性出血。需要的话妳可以借由持续使用复合药丸并休息一阵子来解决这个问题。

这样一来,妳可以量身打造适合自己的月经周期,尽可能减少出血量。

持续服用荷尔蒙直到出血,接著休息一下撑过出血的日子。停药的休息时间很有可能短于7或4天。休息过后,重新开始服用荷尔蒙直到出现下一次突破性的出血。只要妳不要服用少于21或24颗药就绝对没事。如果妳在服用十几天的荷尔蒙药丸后出血,为了避孕的果妳必须继续服用药丸直到第21或24颗为止。





荷尔蒙避孕⸺是不是很危险?




妳可能注意到多年以前关于"自然"的新理想。像是排毒、对羟基苯甲酸酯(parabens)、蔬果汁和超级食物等奇怪的名词已经不足为奇。那些自称健康大师所说的话是非常明确:"人造"添加物对妳的身体没有好处。妳不应该招惹他们。

一夜之间,蔬果汁已成为最热门的时尚配件,在同一时间,荷尔蒙避孕也不再流行。年轻女性因为担心可怕的副作变得害怕使用避孕药。我们听到有愈来愈多人说她们有服用荷尔蒙避孕药的不适症状,就好像过敏一样。还有人询问停用荷尔蒙休息、排毒、冲掉身体内的非自然物质是否健康。

在愈来愈注重纯净和自然的时代,很多人仍然觉得医生没有认真看待她们对副作用的担忧⸺医学专家忽视她们的问题或者试著掩盖一切。结果造成许多女性对避孕方法安全与否感到无法放心,因而从不可靠的来源寻求资讯。

大约三分之一的女性在开始使用避孕药的六个月后便停止服用。其中,约有一半的人会这么做的原因在于她们经历了一些副作用。如果妳不明白为什么会发生或意义为何的话,它会让妳因为感受身体的变化而恐惧。我们认为妳应该要有正面与反面的荷尔蒙避孕资讯,以便替妳身体做出正确的选择。知识孕育自信。

同时在近期出现的可怕现象中带来了一些不同。现在媒体能够给大众的印象是,我们不知道荷尔蒙避孕的副作用为何,就像我们以年轻女性的健康为赌注来玩俄罗斯轮盘一样。好在这是错误的。妳可以自信地认为,妳在药局拿起的药丸是世界上最认真研究的药物之一。研究人员自一九六〇年代起以惊人的统计数据研究数百万地球上广大女性服用药丸的情况。

尤其是当妳想到第一颗上市的药丸大概是我们现在的五倍大时,倘若荷尔蒙避孕存在潜在未知效果的话,可能早在很久以前就已经被发现了。





什么是副作用?




在我们可以开始谈论个别副作用前,我们需要了解何谓副作用。药物的目的在于对身体具有特定的效果,这就是为什么我们会去服用。在荷尔蒙避孕当中,我们服用的原因是希望能够预防怀孕。副作用是所有其他药物对身体的影响,可能是正面与负面的影响。例如很多女性发现自己使用荷尔蒙避孕时斑点减少很多,这被认为是一个正向的副作用。而在另一方面没有人会希望得到血栓的副作用。

在一九九八年的电影《双面情人》(Sliding Doors)里我们跟著葛妮丝‧帕特洛(Gwyneth Paltrow)的活在两个平行时空的命运当中:在一个场景里,某天早上她赶上列车去工作;而在另一个时空里她错过了。这个小细节深深影响她生命原来的运作模式。这也是我们身体运作的方式。我们的身体是如此的复杂,复杂到不可能同时在缺乏身体其他部位的连锁反应下影响单一的功能。副作用不代表药物就是有害。它代表著本身的运作效果。如果有人曾声称药物或疗法没有副作用,那么妳应该对此抱持半信半疑的态度。这表示他们不是在撒谎就是该物质没有任何的效果。

医生和卫生当局都十分关心副作用。我们知道它们是不得已的产物,不过我们的目的是让副作用尽可能处于低水平的效果。这就是为什么药物极难得到销售批准。首先生产者必须证明药物的正面影响比负面影响有更大价值的可能性。正是因为我们必须知道在妳服用药物时,能够预期它的副作用到底为何,所以任何新药物的背后都有多年的研究和实验掌控一切。

上市的药物经过中立于医药产业的挪威药品局严格监控,所以才能提早侦测到任何未知的副作用。如果妳受到副作用的影响,妳和妳的医生可以将向挪威药品局回报(我们绝对鼓励妳这么做)。如果有严重的副作用被忽视了(比如多年来使用避孕药引发癌症的人),药品局将会展开全新的调查。

这么做往往将服用药物的广大族群与不使用药物的相当族群互相比较。然后研究看看在那些使用药物的人当中是否有更多的人有潜在的副作用。如果事实证明这两个族群里有很多人得到癌症的话,他们就会知道这些药丸不会导致这种类型的癌症,因为在有副作用的情况下,只会看到更多服用药物的人罹患癌症。





反安慰剂效应




为什么当许多女性回报相同的药物副作用时会没有人下意识相信呢?当女性表示受到副作用时,健保局难道不相信她们吗?为什么他们在没有更进一步调查下不能接受副作用确实存在的事实,是因为一种叫作反安慰剂效应(nocebo effect)的现象。

大多数人都听说过安慰剂⸺意即人们在事物没有确实运作下体验到真实、正向的影响,因为他们期望它会发生。举例来说,有一个原因解释了许多药物为何以鲜艳的胶囊包装:研究发现,如果人们服用外观精致的药丸会变得更健康!这也是为什么医生会穿上白袍,通常将听诊器挂在脖子上的原因之一。外套和听诊器对患者创造了治疗与专业能力的联想。这有助于改善患者的健康。

反安慰剂效应一词,出自于拉丁语"会伤害我"的意思,却与运作方式恰好相反。因为妳相信糖衣锭内含正向物质所以会导致生理上的问题。事实上有四分之一的患者在接受安慰剂治疗时(也就是没有任何治疗)经历负面的副作用影响。如果一个医生告诉病人该药物可能有特定负面影响的话,同样的事情仍会发生。比平常有更多人回报这个影响,但实际上却是药物的错。可能是人们将原因一味归咎于药物的一般症状。赖登堡和洛文塔尔(Reidenberg and Lowenthal)的一项研究发现,只有19%没有服用任何药物的健康受测者在前三天完全没有任何问题。然而,之后有39%的人经历疲劳,有14%的人有头痛的现象,而5%的人则感到眩晕。

从耶鲁大学的一项研究发现,受过高等教育的女性容易高估荷尔蒙避孕的危险。同时,她们不知道荷尔蒙避孕提供了所有正向的健康优点⸺例如,降低卵巢癌和子宫内膜癌的风险。这些负面的预期可以成为自我实现的预言。

考虑到这一点,可能更容易理解为什么医生会在很多女性突然回报旧药,如避孕药有新的副作用时抱持著怀疑立场。这可能只是太多负面宣传所导致的结果。经过更多的研究才能找出该现象是否发现到真正的副作用或单纯只是反安慰剂效应。





凡事都有风险




如果妳使用荷尔蒙避孕,会先从取出服用资讯的册子开始。在那里,妳会发现一长串的副作用列表,根据常见次数做排序。第一个是最常见的,影响程度介于10%到1%之间。这些副作用包括了头痛、情绪不稳与乳房疼痛。之后的副作用则介于1%到0.1%之间。随著往下阅读妳手中的列表,愈来愈令人不安。

首先要知道,妳阅读的服用资讯是谁写的:药物制造商。人们接著会想到,也许他们正试图向我们隐瞒其中的副作用,但事实正好相反。当提到可能出现的副作用时,他们会过度夸饰,这样就不会被不满意的消费者告上法院。有些在服用资讯册内所描述的副作用与使用荷尔蒙避孕的女性回报的相符,但尚未被证实是由药物引起的现象。我们之后会回来提到这个部分。其他我们知道的副作用,是由荷尔蒙避孕引起的。

妳必须清楚的另一件事则是对风险一词的理解。当我们听到风险这个词很容易认为是危险的东西,然而实际上只是出事的机率。

到了简易统计课程的时间。当我们提到副作用,也就是所谓的相对风险,往往获得所有的注意力。相对风险是以妳服用与不服用药物之间作比较来计算副作用增加的机率。

所以举例来说妳可能阅读过使用避孕药比没有使用的人造成血栓的机率是二到四倍以上的相关研究。这听起来颇有戏剧性。想像一下小报上的头条写著:"造成生命危险的避孕药!造成血栓的可能性为四倍!"但实际上却一点也无戏剧性可言。

对我们人类来说最有趣的真相称为绝对风险(absolute risk)。不过小报对绝对风险的数字不感兴趣,因为它往往造成无聊的新闻头条:"避孕药造成血栓的可能性极小!倒楣少女意外得到血栓!"绝对风险单纯是妳使用后实际会有的副作用,以避孕药为例,就不会与没有服用的人相做比较。这使妳更能理解接所触到的危险和真实的情况。

妳服用避孕药造成血栓的可能性有多大?虽然相对风险显示避孕药的使用者造成血栓的风险比不服用的人高于二到四倍,确实造成血栓的可能性,意即绝对风险,每年则是介于0.0005%〜0.001%。

这代表服用避孕药的女性每年10万名就有50到100人会造成血栓。换句话说,即使服用避孕药妳还是会发生血栓这种令人难以置信的不幸。





荷尔蒙避孕的一般副作用




现在,我们对副作用的背景有一点了解,我们可以开始著重在荷尔蒙避孕的地方。让我们先从最常见的事情开始:副作用如头痛、头晕及乳房疼痛影响的幅度为1%至10%。这些都不是危险的副作用,但仍然是件麻烦事。没有人会得到所有的副作用,许多女性也不会经历全部的副作用。实际上只有1〜10个人才有这些副作用,同时意味著有90〜99人没有这种情况。

同样重要的是,要明白常见的副作用与危险副作用之间没有任何关联性。如果妳受到常见副作用的影响,那么妳就不属于危险副作用的高风险族群。

常见的副作用往往经过数个月的时间才会发挥效果,所以我们建议放弃现行方式,尝试新的避孕方式三个月再换下一种方式。如果妳还是觉得受到副作用的困扰,妳可以尝试其他品牌或避孕方式。

事实上,人们对各种品牌和方法的反应不同。带给妳朋友强烈头痛的产品可能非常适合妳。只有自己试试看才会知道好不好用。正如我们前面所解释的,不同的黄体制剂产品,在我们身上的运作模式也略有不同。此外还有使用包含纯黄体制剂的避孕方式,如荷尔蒙避孕器与植入式避孕器,或是含有雌激素的复合产品都有不同的差异。即使妳在一个产品上得到许多副作用,并不表示妳对一般荷尔蒙避孕法具有"不耐性"。不会对妳造成影响的其他方法存在著极高的可能性。妳只需要确保自己选择了不同种类的黄体制剂避孕方式;妳的医生可以帮助妳判断。

含有雌激素的避孕药特别会有常见的副作用。其实这些都很容易在妳怀孕时可以感受到!首先列表上有恶心、头晕等症状。至于孕妇很快就能适应,但如果妳在一开始的症状就感到非常困扰的话,那么在用餐或睡前吃药可能是明智的选择。

雌激素还会导致分泌物增加。看起来或闻起来和一般的没有什么不一样;只是数量变得更多。少数人腿还会抽筋。我们不知道为什么会出现这种情况,但我们知道它不危险。乳头溢出少量的乳汁也是其中一个不太常见的副作用。

雌激素避孕的另一个副作用则是色素沉淀。虽然这是使用雌激素避孕的女性才会经历的副作用,但主要原因可能是避孕药里的黄体制剂所引起。色素沉淀,严格来说称为肝斑(melasma),为皮肤上显现之较深的棕色斑块。无论户外或是日晒机,只要晒太阳就会产生斑块。正常来说怀孕期间才会色素沉淀,同样也是由荷尔蒙引起。如果妳有这方面的问题,高防晒系数乳液有助于防止色素沉淀。另一种替代方案是尝试含有不同黄体制剂的避孕药来看看是否有帮助。

雌激素同样也有正向的影响。妳可能听别人说过,怀孕妇女容光焕发。皮肤更加透亮其实是雌激素的正向效果。如果妳有痘痘的问题,复合产品可以提供帮助。然而,纯黄体制剂的避孕药却产生相反的效果,反而造成皮肤、头发油腻与痘痘。对有些选择避孕方式的人来说可能是个重要的因素。

含有雌激素的避孕药,其实经常被用来治疗多囊卵巢综合症(polycystic ovary syndrome)的女性,我们会在后面的章节谈论这个非常常见的情况。

雌激素避孕药的另一个正向副作用是,它们让妳控制自己的月经周期。这表示经痛减缓,在卫生棉条的花费也降低了,再也不是因为经前症候群而歇斯底里的月经泼妇或是乱吃巧克力的胡闹婴儿。

另一种早期更常见的副作用为水肿(oedema),这是肿胀的医学术语。简单地说,就是身体里的积水。雌激素和黄体制剂可能都是罪魁祸首,因此所有的荷尔蒙产品都有这样的效果,并非只有复合产品。女性开始使用荷尔蒙避孕法时会觉得自己体重增加,液体滞留是其中一个原因,但妳就本没有变胖:妳只需要喝下额外的水即可!

服用荷尔蒙避孕药会导致妳发胖的迷思是存在的。在这些理由当中,这个迷思引起很多女性开始在日常生活中避孕,特别是身体正在经历剧烈变化的阶段:青春期。另一个原因可能是很多女性找到伴侣后体重增加了一些。然后,她们认为这些幸福肥是因为避孕的关系,完全忘记她们花了更多时间在沙发上相拥,并在腿上放一袋零食看著五季的《权力游戏》(Game of Thrones)。其实妳不会因为服用荷尔蒙避孕药而发胖,只是这一切都太容易怪罪到避孕药身上。

妳的乳房也会有液体堆积,而且变得愈大愈敏感。另一个有点奇怪的副作用则是配戴隐形眼镜的人可能会发现镜片突然不合适。这是因为一些额外的水分累积在眼睛里,所以改变角膜的形状。体内水分的增加也可能导致头痛。

许多使用避孕药、贴片或阴道环的女性在出血的那周只会有头痛的症状,也就是停止使用荷尔蒙的那一周。这是很常见的,有点像妳早上还没有喝到习惯会喝的咖啡前的头痛。头痛的征兆表示妳失去平常会得到的东西,在这个情况下就是荷尔蒙。为了减少这些痛苦,妳可以直接跳过或缩短无荷尔蒙日的天数。正如我们前面提到的,没有特别规定妳应该要休息7天。只要妳不超过7天,妳就能自行决定。妳使用的是黄体制剂产品就不会有这个选项。

如果妳使用的避孕措施仅包含黄体制剂,例如植入式避孕棒、荷尔蒙避孕器和无雌激素避孕药,妳不会得到我们之前提到的雌激素所带来的副作用。妳也不会有任何的雌激素的正向效果,例如更明亮的皮肤与经期的控制。黄体制剂实际上可能造成肤况不佳,而且在某些情况下,还会增加毛发的生长。

所有使用这些避孕方式的女性所经历最重要的副作用大概是出血变化。

这是完全无害的,但还是有人觉得困扰。有关变化的幅度,则依据个人与所使用黄体制剂避孕类型而有所不同。直到试过之前妳根本不会知道自己会不会起反应。有些女性会完全停经,而其他人则可能有更频繁的轻微出血或不规则出血。多数人比平常经历更轻微的出血,可能会持续更多或更少天。当妳使用避孕措施三到六个月后状态会趋于稳定,所以妳得学会认识属于自己的避孕方式型态。

尽管植入式避孕和荷尔蒙避孕器经常造成出血变化,却仍然是我们强烈推荐的两种避孕方法。它们在佩尔指数得到最佳的分数,因此是避孕最有效的方式。荷尔蒙避孕器与其他形式的避孕方法相比,有著低得令人难以置信的荷尔蒙剂量。有些人认为荷尔蒙避孕器因为已经行之有年,所以能够提供身体更多的荷尔蒙,但事实并非如此。使用最小的荷尔蒙避孕器时,血液中的荷尔蒙浓度居然是如此之低,这相当于每隔一周服用一颗小药丸!有些人认为低浓度的荷尔蒙能降低发生副作用的机会,但目前尚未得到证实。即便如此仍然值得尝试,尤其妳无法使用其他避孕方式的话。





罕见副作用




现在我们直接来到服用资讯册上的副作用列表。这些副作用一年可以印在小报的头版好几次,没有别的,报纸有的是疾病和死亡的恐惧。嗯,大概除了性爱。如果妳怀疑的话,医生和制药公司之间没有所谓用荷尔蒙来威胁健康年轻女孩生命的阴谋。甚至还有研究测试可以证明!哈佛大学耗时三十六年研究12万名女性服用避孕药的长期影响。他们的结论是经常使用避孕药的女性(或者妳很喜欢使用),很少死亡,机率和不使用荷尔蒙避孕的女性一样。在任何情况下,我们可以把死亡踢出我们的担忧名单。





血栓




尽管如此,使用含有雌激素的避孕方式确实有严重的副作用,虽然它们极为罕见,但我们仍须说明。通常最受关注的是血栓的发生。

血栓是当我们的血液凝固,在血管上出现一个或多个肿块。血管内的肿块停止血液流动⸺最常出现在腿部和骨盆的大静脉当中。静脉,与动脉相反,是血管从妳的器官和四肢携带血液回流到心脏。医生称这种状况为深静脉血栓形成(deep vein thrombosis)。

我们的腿会得到血栓的原因是,当血液被输送回心脏时需要花很大的力气抵抗重力。血液仰赖肌肉收缩的辅助来增加速度,有点像是帮浦。比方说我们在飞机上的航程坐了很长一段时间,血流速度可能会过慢。如果妳真的不走运,可能会开始凝固。如果腿部有血栓,妳会发现它肿起来,呈现红色并有疼痛感。

为什么人们害怕腿部血栓的最主要的原因是,部分血块可能会松动。接著经由血液冲回心脏,然后,进入肺部。由于肺部血管较小,血块可能会卡在那里,导致呼吸道问题。这个现象称为肺栓塞(pulmonary embolism)。虽然可能会很严重,但很少致人于死地。肺部血栓的一个迹象是,如果妳胸部出现突然的刺痛感,吸气也会变得更糟糕。我们有时候胸口会有点刺痛,通常是因为肋骨压迫肌肉所致,但肺栓塞造成的疼痛不会消失。同时,妳可能会呼吸急促并引发咳嗽。如果妳怀疑有血栓,记得立刻到急诊室或进行紧急手术治疗。

正如之前学到,避孕药所含的荷尔蒙类型有所不同。只有含雌激素的避孕药才有增加血栓的风险,包括避孕药、避孕贴片和阴道环。我们在风险的部分提到,妳使用复合避孕得到血栓的风险会上升两到四倍。为什么我们说两到四倍的原因是取决于妳使用的类型?现在以雌激素为基底的避孕药在市面上流通,一种是得到血栓的可能性最小、含有左炔诺孕酮的黄体制剂。挪威市面上有三种不同类型的左炔诺孕酮药丸:欣无妊、奥诺康和乐婷锭。奥诺康和欣无妊完全相同,而乐婷锭的雌激素剂量稍低。如果妳第一次打算使用避孕药,我们会建议在这些类型选择一种。

因为有增加血液凝块的风险,所以有些女性不应使用含雌激素的避孕方式。最重要的族群则是基因上缺陷影响血液凝固功能的女性,例如被称为莱顿突变(Leiden mutation)的情况。这就是为什么医生会在妳开始使用复合避孕法时询问妳的父母或兄弟姐妹是否有过血栓。

我们在前面提过,健康的年轻女性得到血栓的风险是非常小,不管她们是否使用雌激素的避孕法,绝对风险小。如果10万名女性在一年当中服用避孕药,会有40〜100人得到血栓。如果她们没服用,仍然有20〜50人得到血栓13 。这是不正确的,避孕药中的雌激素比在体内的"自然"雌激素更危险。制造大量雌激素的孕妇比避孕药使用者的血液凝块风险更大。为了比较,10万名女性中高达200人在怀孕或生产后得到血栓。

换句话说,妳意外怀孕比妳正在使用避孕药得到血栓的可能性更大。怀孕时身体自然增加的荷尔蒙比我们为了防止怀孕所增加得更多。这就是为什么我们使用避孕药时应该接受血栓风险略微增加的最重要原因之一。怀孕更加危险。





中风与心脏病




雌激素避孕药另外的严重副作用是中风和心脏病。这些是影响动脉的疾病,也就是血管带著富含氧气的血液从心脏流到我们的器官。当血流停止,无论是血栓或血管破裂,血管组织可能会因为缺氧而阵亡。这意味著心脏的一部分缺氧而死。显然,这种伤害的后果可能会相当大。

一项于一九九五年到二〇〇九年之间对所有丹麦女性的研究发现,使用雌激素避孕药的中风和心脏病风险约为两倍。然而,请记得相对和绝对风险之间的区别:虽然增加了一倍听起来很夸张,但这些都是很少发生在年轻女性身上的疾病。即使风险加倍,妳会中风的可能性还是最小。

为了说明,我们回到相同的研究。一年里使用避孕药的10万女性中,大约20位中风,10位得到心脏病。这包括所有使用避孕药的丹麦女性的类型:胖与瘦、吸烟者和非吸烟者、老年与年轻。如果只是调查健康的年轻女性,风险甚至更低。

有些女性不应使用含雌激素的避孕药以减少中风和心脏病的风险。例如三十五岁以上患有高血压或心脏疾病的吸烟女性,或罹患糖尿病二十年以上。不应使用雌激素避孕药的另一族群则是有前兆性偏头痛的女性。但是,如果妳的偏头痛没有前兆性的话,只要妳在三十五岁以下,便能使用雌激素避孕。

如果妳接触到太多可能导致中风和心脏病的风险因素(例如超重、高胆固醇和吸烟)妳的医生在安全上可能会建议妳选择另一种避孕方式。长话短说,如果妳还年轻、健康,即使使用雌激素避孕,也没有必要担心中风与心脏病。





癌症




我们要谈的最后一个副作用为癌症。在某些圈子里,其实还有人相信避孕药会致癌。我们先从强调使用避孕药等其他荷尔蒙避孕法不会增加妳人生罹癌的可能性开始。从整体上来看,避孕药其实降低患癌症的风险。它们似乎保护我们抵抗肠、膀胱、子宫和卵巢的癌症。这些类型的癌症常常发生在女性身上。

在妳停止服用药丸后,避孕药可以预防卵巢癌长达三十年之久。如果这个数字正确的,研究人员认为避孕药在的未来几十年内每年可以阻止三万件卵巢癌的案例!以族群为基础的研究显示,避孕药可以防止子宫内膜癌至少十五年,得到这种癌症的风险比没有使用荷尔蒙避孕的女性相比,减少了一半。有些研究人员已经清楚传达其中的意涵:避孕药可以预防妇科癌症,这个正向副作用胜过所有的负面影响。

然而,避孕药似乎会增加子宫颈癌的风险。该领域最优秀的研究显示,使用十年避孕药后,子宫颈癌的发生率每千名女性当中从3.8%提升至4.5%。风险随著妳使用避孕药的时间增加,但一旦妳停止了会再次下跌。停止服用避孕药的十年后,风险回到和开始时一样的水平。

问题是,不可能肯定地说,避孕药本身增加罹患癌症的风险,因为使用的女性更容易感染HPV⸺也就是导致子宫颈癌的病毒。更容易病毒感染的原因是许多服用荷尔蒙避孕药的女性与新伴侣进行性行为时,在使用保险套上变得更加宽松。甚至发现,使用这种避孕方式的女性性行为次数更多⸺毕竟,这就是为什么她们会使用避孕药的首要原因。

乳癌是在提到癌症与服用避孕药之间关联时人们想知道的最后一种病症。我们知道有些特定类型的乳癌称为"荷尔蒙敏感型"⸺这代表它们像雌激素一样,为了癌细胞生长所需。复合避孕药的确含有雌激素,这可能会使妳认为雌激素避孕药帮助"饲养"这种类型的癌症。

幸运的是,运作的效果并不完全。多数大型研究在乳癌和服用避孕药之间尚未找到任何的关联,只有少数例外。个别研究发现一九六〇和七〇年代使用最高剂量避孕药的女性罹患风险略微增加。不过,专家认为,现在的避孕药和其他含有这种低剂量荷尔蒙的复合产品影响乳癌风险的可能性很小。

总结:避孕药等复合产品的出现能够保护女性免于一些普通或严重类型的癌症所苦。如果妳正在仔细评估荷尔蒙避孕的整体重点,这是值得考虑使用的。不幸的是,这些重要、正向的副作用与罕见、危险的副作用相比,得到媒体的关注太少。





我们不确定的事物




如果妳读过避孕药的服用资讯册,也许会感到惊讶,我们忽略了两个重要的副作用:情绪波动与性欲降低。我们还没有提到的原因是我们认为不重要⸺正好相反。这些副作用是研究人员最不确定的东西。然而这两种可能的副作用在近几年已获得女性愈来愈多的关注,所以我们认为它们应该得到充分研究。

天然性荷尔蒙影响大脑调节情绪水平和性欲的区域。女性的情绪会根据月经周期的荷尔蒙变化波动,这是一个众所周知的事实。有些女性发现她们接近排卵期时性欲特别高涨。甚至观察发现,女性在排卵前后更加不忠!

考虑到这一点,改变性荷尔蒙平衡的避孕方式同样也能对心理和性欲产生影响并不奇怪。女性和许多医生之间也逐步形成广泛的共识,服用荷尔蒙避孕药会引起情绪波动、烦躁,在最坏的情况下,则会罹患忧郁症。心理和其他非特异性副作用为女孩放弃服用避孕药的主要原因。

即使女性之间存在这项共识,研究人员仍继续努力。一些研究试图证明服用荷尔蒙避孕药对女孩的情绪有负面影响,但没有成功。以下有几个可能的解释。





第一个可能的解释:研究还不够完善

避孕药的研究数量庞大到令人难以置信,在过去几十年间,文献数已经超过四万篇。问题是,许多研究往往缺乏品质,尤其是关于副作用的部分。尽管质量低,也不太可能导致荷尔蒙避孕的副作用遭到忽视或低估。也许这听起来很奇怪,但妳通常会在不好的研究中找到最多副作用,而非好的论文里。大多数不错的研究认同整体趋势而不是显示很少或没有副作用。所以关于很多不好的避孕药副作用研究可能对我们过度夸张这些副作用的规模与严重性。

许多不良研究的问题在于他们通常会找使用荷尔蒙避孕药的女性,向她们询问副作用时,不去询问不使用荷尔蒙避孕药的女性。当妳这么做的时候,妳根本不能得出任何结论,因为妳所做的一切极有可能是衡量这些症状在一般人群当中有多常见。

想像一下,例如所有女性中有10%的人每个月会头痛一次,但通常不会有任何特别的想法。如果有人问她们多久头痛一次,她们就会加以猜测。接著她们参与一项研究,每天服用避孕药并写下所有可能的副作用。因此在这项研究中,有10%的人会自动回报头痛的现象,即使和避孕药无关。因为没有与不使用避孕药的女性比较所以不会察觉。取而代之的是,看起来就好像是避孕药引起的头痛。这些类型的研究盛行,也是最常看到荷尔蒙避孕对心理和性欲有副作用的文献。

在医学领域上有一种研究被认为是最好的,也就是绝对黄金标准。当然它有一个奇特的名字:随机对照研究。一群人被随机分为接受治疗与没接受治疗两组,不接受治疗的人为对照组。盲测是研究的理想情况,意即患者(最好医生和研究员也一样)不知道正在接受何种治疗。只有这种模式的研究才可能会有因果关系,也就是证明药物是否为造成症状的原因。

据我们所知,以随机对照方式进行避孕药和如情绪变化的非特异性副作用研究目前只有四项14 。其中两项研究发现,有无服用药丸的人在情绪变化上并无显著变化。另一项研究则发现,服用避孕药导致忧郁症症状改善。

在过去爱丁堡和马尼拉女性的研究里,发现服用迷妳丸的女性忧郁症状减少,而收到安慰剂与避孕药的人忧郁症状轻微增幅。

唯一的例外是一项瑞典的小型研究。一组研究员在乌普萨拉(Uppsala)邀请了一群曾经因为避孕药经历过心理副作用的女性参加安慰剂对照研究。在不知道自己组别下,一半的患者收到避孕药,而另一半则是糖衣锭。研究发现,平均上,收到避孕药比没有收到的人经历了更大的智力减退。此外拍摄的女性大脑图像有感情唤起的倾向。有些服用避孕药的女性里,大脑运作在情感中枢的地方发生变化。

然而,有一个很大的"但是":这只适用于三分之一的药丸使用者。三分之二的服用避孕药的女性没有经历智力衰退或大脑活动的变化,即使如此,他们本身对荷尔蒙避孕有不利反应的倾向。这些发现可能显示,避孕药对少数女性有负面心理影响。但是比起觉得就是如此的人数,适用这个结果到女性却少了许多。这让我们来到下一个可能的解释:机会的力量。





第二种可能的解释:机会的力量

人类都有一个喜欢维持世界秩序和系统的大脑。我们借由想像不存在事件的连结来试图理解我们时而混乱的环境。如果两个事件同时连结,我们得出相互影响的结论。例如,妳开始服用避孕药三个月后,突然发现妳的情绪有点低落。我能肯定不是避孕药的关系吗?毕竟,就妳回想而知,妳从来就没有经历过这样的情况。

但是,这并不是一定的理由。忧郁症是人类极为常见的疾病。大约五分之一的女性在人生中经历适当的低潮,有更多的人经历抑郁的感受和想法。忧郁症是包含许多复杂原因的疾病。个性类型、大脑的生理变化、遗传和生活中的问题参和其中。因为很多因素在内,几乎不可能指向一个具体的原因。

忧郁、情绪变化与烦躁,在人群里是如此普遍的现象,也可能是偶发的状况。此外,如果妳听说避孕药能引起情绪变化和忧郁症,如同我们稍早谈到的反安慰剂效应,妳更可能会得出这样的结论。网路论坛上情绪变化的谣言在女性网友间满天飞,妳突然看到一道新的曙光。

这是由许多大型研究构成,在芬兰、澳洲与美国,已经进行过这种研究,并得到负面的结果。澳洲的研究追踪一万名女性三年的时间,有无使用避孕药之间得到忧郁症状的机率没有任何区别。此外,研究发现,随著使用避孕药的时间越长,忧郁的念头也就越少。美国的研究在一九九四年至二〇〇八年间追踪七千名女性。事实上,研究发现,上一年度使用避孕药的女性比起不使用荷尔蒙避孕得到更少的忧郁症状,企图自杀的可能性也更低。研究人员也发现同样的现象存在于芬兰的研究:使用荷尔蒙避孕的女性比其他女性较不忧郁。

这些研究的问题在于,有无服用避孕药的女性之间潜在的差异。且有可能是陷入恶劣情绪的女性停止服用避孕药,而继续服用的女性没有负面反应的情况。负面影响因而被掩盖。

鉴于这种评论,二〇〇〇年至二〇一三年间,哥本哈根的研究人员以十五至三十四岁的一百万名丹麦女性进行大型人口基础研究。研究发现,使用避孕药与其他荷尔蒙避孕法的人比不使用的人需要抗忧郁药或是诊断出忧郁症的风险增加得多。

在十五至十九岁之间的年轻女孩中效果最明显,一旦满二十岁,风险明显下降并随著年纪增长而持续下降。超过三十岁使用荷尔蒙避孕的女性,服用抗抑郁药或忧郁症的发病率几乎没有增加。研究人员认为,大脑随著年龄增长对荷尔蒙的波动变得不太敏感。

这项研究还发现,忧郁症和服用抗抑郁药的风险随著女性服用荷尔蒙避孕愈久而稳定降低。风险最高的族群月经为使用六个月的女性,之后再开始下降。使用四年的荷尔蒙避孕后,继续使用与不使用的人之间罹患忧郁症的风险并无区别。

研究人员同时得到不同种类的避孕法之间存在这差异的结果。避孕药是使用抗抑郁药的风险最小的种类,反之,例如迷妳丸、阴道环和长效方法的风险更大。虽然不可能只根据一种研究来断定,不过研究强调了女性在经历不良副作用时应该要有更换其他避孕方法的对策。避孕方法带给女性的副作用不同,尝试显得更为重要。

话虽如此,我们建议解读这项研究时应该要谨慎一些。目前丹麦有很多恐慌宣传警告女性抵制荷尔蒙避孕,因为会导致忧郁症。信不信由妳,根据研究,妳不能将这个结果奉为圭臬。研究显示的是,使用荷尔蒙避孕的女孩开始服用抗抑郁药比那些不使用的人要来得多。没有人可以证明,荷尔蒙避孕是造成忧郁的原因。听起来像是吹毛求疵,但这是一个很重要的差别。为了能够说出因果关联的结果,妳必须使用完全不同的研究方法:随机对照研究。正如我们之前讨论过,这类研究目前没有得到任何近似的结果。丹麦的研究很完全,有可能会在该领域进行更详细的研究,然而直到我们有更多显示同样结果的研究前,我们不能断定荷尔蒙避孕在某些女性身上造成忧郁症。

我们同时也无法摆脱相对与绝对风险之间的对立。一些与丹麦研究有关的报章杂志表示,青少女得到忧郁症的风险高达80%。这听起来很可怕,表示妳很可能在高中来月经后,因为服用避孕药而变得忧郁。事实上完全相反。每年,100名不使用荷尔蒙避孕的丹麦少女当中有1位开始服用抗抑郁药。相较之下,使用荷尔蒙避孕的100名少女则有1.8位服用抗忧郁药。我们要讲的是增加幅度根本不到一人。98位使用荷尔蒙避孕的少女没有忧郁症,而只有1位会在任何情况下得到。妳应该记住这些数据,而非头条上那个惊人的80%。一旦事实了然可见,妳仍可自行将它当成不使用贺尔蒙避孕的理由。我们不会干预这个选择。

现在,我们经历了许多研究,并提出相反的结果。可能很难以消化这一切,我们非常清楚这一点。即便如此,我们认为有可能从这些研究中得到一个重要的结论:荷尔蒙避孕不可能对大多数女性的心灵造成任何重大负面影响。如果这样的副作用确实存在,它容易出现在由于某种原因容易对荷尔蒙产生不良反应的人。我们希望将来能够好好了解这些女性。如果妳的家庭有很多人一直饱受忧郁症所苦或是妳过去有忧郁症倾向的话,或许值得谨慎看待。

对于此外的族群,是时候停止担心⸺当我们听到荷尔蒙避孕心理副作用的可怕故事时完全不用尽信。感觉不一定就是事实。

我们使用荷尔蒙避孕是为了有无忧无虑的性生活,但如果使用后造成性生活无趣呢?避孕药扼杀性欲是真的吗?许多女性似乎这么认为。在瑞典的调查中,近30%使用荷尔蒙避孕药的女性认为其中一个副作用就是性欲降低。

荷尔蒙避孕和性欲之间的关联在二〇一三年进行了一项最大规模的研究。结合了36项研究,共计1万3千名女性,其中8千人服用避孕药。大多数女性发现自己的性欲在服用药丸后维持不变(64%)或确实增加(22%)。同时一些研究发现服用避孕药会导致性欲增加;因为避孕消除可能会怀孕的焦虑⸺世界上最大的女性激情杀手。正如我们在欲望的部分讨论过,性欲,简单地说,是减速与加速之间平衡的作用。因此,研究人员不认为荷尔蒙会直接提高性欲。另一方面,15%使用荷尔蒙避孕的女性性欲降低。我们不能肯定地说荷尔蒙是否就是罪魁祸首。

然而,已知的是,体内现有的睪固酮浓度在使用荷尔蒙避孕时降低。正如我们所知,睪固酮是男性最优秀的贺尔蒙,但女性也会产生一小剂量。服用睪固酮来增加肌肉的健美先生经常会性欲高涨(通常伴随阴茎细小和劣质精子的讨厌组合)。荷尔蒙避孕的女性有可能因为少了睪固酮造成性欲丧失的反效果吗?

睪固酮降低的程度依每个女性与我们使用避孕的类型有所不同。荷尔蒙避孕含有不同的黄体制剂,对睪固酮有不同的影响。那些含有屈螺酮(drospirenone)的产品,例如优思明,会降低睪固酮的浓度。可能会使青春痘变少,而且,有可能也会造成性欲降低。然而,包含左炔诺孕酮的黄体制剂乐婷锭、欣无妊和荷尔蒙避孕器,有更多的睪固酮效果,因此不太可能导致性欲降低。

睪固酮理论的问题在于,血液中的睪固酮浓度和所经历的性欲程度之间没有明确的关联。有些较高睪固酮浓度的女性饱受性欲所苦,而其他低睪固酮的女性却没有感觉。性欲与睪固酮浓度显然不成正比。即便如此,人们曾试图给予女性睪固酮来提高性欲但是却没有任何神奇效果15 。平均而言,她们每个月在"满意的性活动"次数上比别人多一次(这些人在研究世界里实在很会开黄腔)。

尽管如此,女性的性欲还有很多我们不知道的事。关于荷尔蒙避孕对性欲的影响,我们对能否找到一个很好的答案没有把握。因为对妳来说什么是真正的性欲?因为性欲没有好的评估方式所以难以研究。更重要的是,性欲是受到很多生活因素影响,很难分出到底是因为避孕药还是感情淡掉所致。

妳可能已经明白,研究的世界充满不确定性。然而,我们可以说,很少有研究提出荷尔蒙避孕对大多女性的性欲有强烈的副作用。

可能妳的避孕方式降低妳的性欲,但并不常见。更常见的是性欲随著情感关系的时间消退,或是压力剥夺我们对性的乐趣和花招所需的多余能量。

我们的建议是,在把妳的避孕药丢到垃圾桶,或是预约移除植入式避孕棒之前,评估生活中是否有其他导致妳性欲降低的其他方面。妳也可以尝试切换不同黄体制剂的避孕方法。





是时候来做荷尔蒙排毒?





性对我们大多数人来说并不一直都是好的。当妳在一个稳定的关系时,也许妳每周做爱好几次;但结束之后,妳的单身生活并不像妳所憧憬的《欲望城市》(Sex and the City)剧情那样。妳开始觉得自己像是草原上的大象,在旱季高峰寻找水源。眼前没有时尚杂志、养眼的对象也不会有阴茎。避孕药成为每天妳非自愿禁欲的残忍提醒,似乎从浴室柜子嘲笑妳:"哈!妳今天也不会发生性行为!"

同时,妳大概也听说荷尔蒙对妳身体不好,它们是不自然的物质。为什么妳要在没有获得性作为补偿的状态下让身体受到邪恶的荷尔蒙支配呢?妳认为:"这段单身的时间就当作排毒、净化和养身!是时候暂停使用荷尔蒙了!"

等等,这实际上不像听起来那么聪明。如果妳发现了一个适合妳的荷尔蒙避孕方式,如果因为妳单身就停用就太傻了。大多数人开始使用荷尔蒙避孕会有特定的副作用,但是通常会在几个月后变得缓和。身体会调整到一个新的荷尔蒙平衡并稳定下来。当妳停下来,这需要时间让妳的身体恢复到一个新的平衡点,下一次再使用时只会再经历一次完全相同的副作用。

血栓,是我们不建议暂停服用荷尔蒙避孕的主要理由。一些研究表示,开始服用避孕药后的前几个月造成血栓的风险最大,随著时间的推移急剧下降。如果妳每一见一个新对象就开始和暂停使用避孕措施,妳的身体不会有时间回到平衡。结果造成妳理想对象不只会让妳紧张,得到血栓的风险也变高了。

如果血栓是暂停荷尔蒙的危险却罕见的副作用的话,那么另一种,就是更加普遍。情人与妳不期而遇,而妳的医生不能提供全天候服务。如果还考虑到一个事实,就是挪威人是世界最不爱用保险套的人,这一点也不奇怪,停药后最终往往给妳比妳讨价还价还更多的排毒。实际上是九个月排毒。四分之一停用避孕六个月的女孩最后在半年内意外怀孕。非常自然!

有些女性担心长期使用荷尔蒙避孕,可能在以后的生活中难以怀孕。好在这些都是胡说八道,虽然使用特定荷尔蒙避孕可能需要几个月的时间才会再次排卵。事实上使用荷尔蒙避孕的女性当中不孕的可能性较低,得到性传染疾病的话,她们骨盆发炎的机会更小。不幸的是女性(和男人)无法怀孕有很多种原因。问题是,直到停止使用避孕措施并试著生孩子之前妳不会知道自己是否是其中之一。如果妳三十五岁却怀孕失败的话很容易把原因归咎于从十五岁开始使用避孕药的决定。研究表示,使用避孕药对女性生育能力没有影响,无论是否已经使用一年或是十年。然而,老化确实有很大的关系。





为荷尔蒙避孕辩护




最近,在挪威关于荷尔蒙避孕的复杂面向有许多的公共讨论。没有更多的避孕方法可供选择确实遗憾,我们完全同意,我们非常想看到市面上有更好的男性专用避孕方式。但事实是,性行为会导致怀孕,因此避孕是女性不可或缺的决定。不管我们多么不喜欢它,这个事实并不会消失。当然,我们想要做爱。

避孕的世界还不尽理想,我们在结束这个部分前必须替荷尔蒙避孕发声,提出一段言论作为辩护。因为许多荷尔蒙避孕的正向因素往往遭到忽视。荷尔蒙避孕,和含铜避孕器及结扎一样,是我们最有效维持避孕的方式。部分使用荷尔蒙避孕女性所体会到的无害副作用,和大多怀孕女性的经历相比算不了什么:与怀孕有关的骨盆疼痛、大量分泌物、双腿肿胀、痔疮与妊娠纹。更不用说危险罕见的副作用。怀孕时血栓的风险,比使用荷尔蒙避孕时更高。太少人知道荷尔蒙避孕的正面影响。我们已经提过,不过重复几次并无坏处。



• 荷尔蒙避孕似乎能预防一些女性最常见和最危险的癌症形式:大肠癌、卵巢癌和子宫内膜癌。

• 荷尔蒙避孕能减少经痛,缩短经期及降低血量,更可以降低得到贫血的机率。对许多女性而言这是严重的问题。

• 使用复合避孕产品,妳可以控制适合自己的经血模式。

• 荷尔蒙避孕防止骨盆腔感染⸺对无子女的女性造成很大的影响⸺借由增厚子宫颈黏液,让细菌难以进入。

• 得到良性乳房肿块⸺导致许多年轻女性焦虑同时必须接受手术治疗的机率降低。

• 荷尔蒙避孕也擅于治疗两种常见又麻烦的女性疾病:多囊卵巢综合症和子宫内膜异位。



当人们将荷尔蒙避孕看作是女性的天敌时,记住这个列表可能是个好主意。避孕药在女性平权中已经并将继续成为世界上最重要的发现之一。





避孕指南




妳觉得很难选择避孕方式吗?我们有十一种方式可供选择。但是不要绝望:我们已经替妳准备避孕指南。由于有效的避孕方式需要处方签,所以妳必须和医生、接生员或护士咨询。不过先建立一些想法是不错的。根据对妳而言最重要的方式,妳现在可以选择适合妳的避孕方法,并找出妳最好避开的有哪些。妳可能对以下几个组合有兴趣,所以妳的问题就是选择最佳的替代品。





对我来说,最重要的是避免怀孕


如果对妳最重要的事情是避免怀孕,妳应该选择避孕最有效的方法⸺也就是所谓的长效方法。在列表的顶部,妳会发现植入式避孕棒和荷尔蒙的避孕器,紧接在后的是含铜避孕器。复合产品,如避孕药,如果妳正确地使用它们也是有效的。

适合:具有低佩尔指数的长效避孕:植入式避孕棒、荷尔蒙避孕器和含铜避孕器。

不适合:具有较高佩尔指数的方法,尤其是基于安全期的方式。





我是血栓、中风或心脏病高风险族群


如果妳是这些疾病的高危险族群,妳必须避免使用雌激素。妳仍然可以选择最擅长防止怀孕的避孕方法⸺即黄体制剂产品,如植入式避孕棒和荷尔蒙避孕器。如果妳喜欢服用避孕药,挪威市面上也有贩售无雌激素药丸如Cerazette。

适合:无雌激素的方法:避孕植入物,荷尔蒙避孕器,无雌激素避孕药和含铜避孕器

不适合:复合产品:复合药丸、避孕贴片和阴道环





我想更少的出血量


月经可以很痛苦,特别是对拥有严重疼痛与出血的女性。有些女性是如此难受,她们流血到自己贫血,或者因为疼痛每个月得花一个星期躺在床上。如果这听起来很像是妳的情况,那么知道某种避孕方法可以减少出血是很有帮助的。所有荷尔蒙避孕的普遍特征是血量较小。为了找到哪一个最适合妳,妳应该去尝试,透过试验和错误,再与妳的医生咨询。含铜避孕器同时会增加出血和疼痛,所以不建议妳使用。

适合:一般荷尔蒙避孕,特别是荷尔蒙避孕器和复合产品

不适合:含铜避孕器





我想控制出血


正如妳所记得先前提到"荷尔蒙避孕对月经的影响"的部分,可以使用含有雌激素的避孕药来控制妳的出血。黄体制剂产品不提供任何月经血量控制。如果妳已经在使用雌激素避孕却没有正向结果,可以将低剂量的雌激素产品换到略高剂量的产品。例如,从乐婷锭切换到奥诺康或欣无妊。这种变化不会增加血栓的风险。

适合:复合产品:复合避孕药,避孕贴片和阴道环

不适合:黄体制剂产品





我有痘痘的烦恼


如果妳有痘痘烦恼,雌激素可以帮助改善;换句话说,妳可能会咨询妳的医生考虑使用复合产品。黄体制剂常常引发痘痘。如果妳已经使用复合产品,妳可以尝试改用另一个包含不同类型的黄体制剂或更高剂量的雌激素。记住,它往往需要三个月的时间,才能看到效果。

适合:复合产品:复合药丸、避孕贴片和阴道环

不适合:妳尝试过的黄体制剂产品





我想要对其他人隐藏我的避孕方法


对于一些女性来说,隐瞒使用避孕的事实是很重要的。有些避孕方式如植入式避孕棒、含铜避孕器、荷尔蒙避孕器或是避孕针,因为在妳的身体里面所以看不到。如果妳担心如何对妳的伴侣或家人隐瞒妳在避孕的话,不妨利用不会改变月经规律的方法,因为月经变化会影响妳的性生活或意味著妳需要比平常用更多或更少的卫生棉和卫生棉条。可以使用复合产品或含铜避孕器作为替代方案。这些经常让妳有固定的经期,虽然出血量可能会改变。如果怀孕对妳来说不是危机的话,妳也可以尝试计算安全期来减少怀孕的风险。但是要记住,使用这种避孕方法的女性有四分之一在一年内怀孕。

适合:隐形避孕,如植入式避孕棒和荷尔蒙避孕器,或者给妳固定周期的避孕方式,例如复合产品

不适合:这取决于妳想隐藏避孕的方式





我要保护自己免于受到性传播感染


保险套是保护妳免于性病的唯一避孕方法。直到妳和妳的伴侣已做过性病测试前,我们建议妳一起使用保险套与另一种避孕方式。

适合:同时使用保险套与另一种避孕方法

不适合:不使用保险套





我服用其他药品⸺我这样可以使用荷尔蒙避孕吗?


药物会相互影响。如果妳正在服用诸如癫痫或精神疾病的药物,这可能会影响妳的避孕。妳的医生会追踪这一点,也许她可以给妳量身定做的解决方案。

适合:如果妳正在服用其他药品,妳的医生会帮助妳找到最好的解决办法。





我有子宫内膜异位症


如果妳有子宫内膜异位症或怀疑自己可能有严重的疼痛,荷尔蒙避孕是妳治疗的第一步。既然目的是停经,妳就不能休息。

适合:连续使用复合产品或荷尔蒙避孕器。





我有多囊卵巢综合症或是月经极为不规律


如果妳在一年不使用荷尔蒙避孕的状况下月经少于四次的话,妳应该开始使用荷尔蒙避孕来驱逐定期的子宫内膜。如果月经是极为罕见,妳可能会有子宫内膜过度生长的问题,长期下来对妳不利。一旦妳在使用荷尔蒙避孕时发生几次突破性出血,问题就能解决,而妳可以继续如妳所愿避开月经。

适合:复合产品:避孕药、避孕贴片和阴道环





我使用的避孕措施降低我的性欲


无法确定荷尔蒙避孕是否会引起性欲降低,如果是这样,机制才是罪魁祸首。有一种理论认为,这是睪固酮不活跃造成的。不同类型的黄体制剂对睪固酮有不同的影响。含有屈螺酮,例如优思明的药丸可以降低睪固酮浓度。可以减少痘痘,但是性欲也会同时下降。然而,像乐婷锭、欣无妊和荷尔蒙避孕器里的左炔诺孕酮黄体制剂具有更类似于睪固酮,因此不太可能降低妳的性欲。

适合:含有左炔诺孕酮黄体制剂,如乐婷锭、欣无妊和荷尔蒙避孕器,不含荷尔蒙的避孕药,如含铜避孕器。

不适合:含有屈螺酮孕荷尔蒙如优思明的产品。





* * *



13	数据依研究而有所不同,并取决于研究的年龄层与人口类型。年龄、体重的增加与吸烟族群所隐含的血栓风险有明显的上升。



14	这些研究的其中一个弱点是,他们以非避孕目的而使用荷尔蒙避孕法的族群为研究对象,例如有严重痘痘或经痛问题的人。因此,可以想像这些女性与使用荷尔蒙避孕的女性有所不同,而结果也受到影响。举例而言,有更多痘痘困扰的女性会更加忧郁吗?



15	睪固酮的增加主要是对绝经后或因为癌症切除卵巢的女性进行测试。鲜为人知的是使用睪固酮的长期风险,如果女性在怀孕时服用睪固酮,可能会对胎儿造成伤害。在一些对略为年轻女性(三十五至四十六岁)的随机研究中,增加睪固酮对性欲的影响很小或根本没有影响。然而,安慰剂效应却有很大的影响。





堕胎




堕胎,故意终止怀孕的做法。一方面,在身体上,是关于女性选择自己是否要生下一个孩子的权利。另一方面,堕胎是一个新生活的开始,一旦怀孕,这个孩子应该拥有什么样的权利?堕胎的问题并没有简单、道德的答案;总会有输的一方,无论是孕妇、孩子的父亲、进行人工流产的医疗专业人员或是胎儿。对我们来说,女人的权利占有举足轻重的地位⸺因为女性在怀孕与生产中会经历身体和心理上的压力。

一直以来总是留给女性提供照顾及支持的责任。一个孩子的出现会导致女性更多的情绪、经济和社会的震动,而女性在一开始拥有的最少,最后受到的冲击却往往最大。因为受到如此大的冲击,女性应该要决定自己是否想承担这一切。当我们迫使女性生下她不希望拥有的孩子时,我们认为没有一个地方的政策能够接受对人民施加大量的个人成本,来满足社会道德规范。也就是必须要有所限制。大多数人在怀孕上都同意一些观点,堕胎不能仅基于女性的选择为考量,胎儿不再是胎儿,而是一个孩子,还有拥有比偏好更有价值的权利以及孕妇本身的权利。这样的限制设置随著国家有所不同,但是不管上限为何却甚少受到挑战。在堕胎合法的大多国家里,多数堕胎发生在怀孕初期,而少数的晚期堕胎为人所知的原因是,胎儿有严重或危及生命的异常状况或是为了保住母亲的性命。

举例来说,堕胎规范有非常不一样的方式⸺从完全禁止的智利与马尔他,以及女性有权在怀孕十二周内选择堕胎的挪威,到没有堕胎相关法律反而属于女性和医生之间医疗行为的加拿大。可以使用堕胎的程度也有许多区别:尽管有没有遭到禁止,有可能是因为非常昂贵或开放的地方太少,让许多女性无法如愿。举例来说,美国许多州就有这种情况。无论妳对堕胎的个人感受如何,禁止堕胎或将堕胎流程复杂化无助于降低堕胎数,这是个不争的事实。我们经常会发现,最严格的立法国家堕胎率同时也是最高,而那些容易获得合法堕胎的国家流产率往往较低。尽管惩罚和社会排斥的威胁,所有的世代和世界上每个角落意外怀孕的女性都选择自行处理⸺更不用说暴露于严重伤害或死亡的风险。生下不想要的孩子的想法可以是如此让人难受,远远超过法律起诉的危险与威胁。织针、女祭司、陡峭楼梯和毒药仍是世界上某些地方的女性在堕胎是违法或无法实施的情况所下使用的最后手段。每年有二十万女性认为有必要进行不安全堕胎⸺几乎是全球怀孕女性的十分之一。在这些女性中,就有五万起完全不必要的死亡。约六百九十万名女性需要透过医疗服务治疗危险堕胎造成的并发症。安全的堕胎会使她们免于这些伤害。换句话说,合法和安全的堕胎,对保障女性的健康极为重要。禁止堕胎不会留著任何孩子,只会伤害绝望的女性。然而,堕胎没有捷径。我们认定少数女性想要堕胎或有意识地用来当作避孕的替代方案。这是因为经常遭遇在错误的时间下进行无防护性行为、避孕失败、无法使用现代避孕方式或者(在最糟的情况下)受到侵犯和性暴力等不幸。如果目标是维持堕胎率低,那么最有效的措施是确保容易取得、有效的避孕措施,并提供良好的性教育。

不幸的是,我们经常看到,限制堕胎的法律正好与这两个难以在健保服务中进行堕胎的面向息息相关。这就像鸵鸟坚持把头埋在沙子里认为问题将会消失,只是因为没有看的必要。

不管妳是否生活在堕胎方不方便的国家,妳会庆幸多少知道堕胎在医疗系统下是如何进行。堕胎进行的方式⸺无论是医院或专科门诊,与适用的规矩,依国家而异。但是这些方法都是一样的,如果妳发现自己正处于意外怀孕的状态,能够将妳的想法专注在更重要的事情而非找出实务上的方法会是件好事。





我怀孕多久了?

谈到堕胎,常见的疑惑之一就是妳怀孕多久了。许多国家的堕胎法律都有时间限制;例如十二周以内的堕胎是受到允许的。但是,真正怀孕十二周是什么时候呢?妳会认为是从发生未防护性行为那天开始计算,令人难以置信的是,事实并非如此。反而是从最后一次月经的第一天开始计算。因为这是确定妳没有怀孕的最后一刻。从这个角度看,在性交怀孕后,法律认为妳"怀孕"了两个星期。虽然不完全合乎逻辑,却是规定运作的方式。在妳堕胎之前,大多数医生会替妳进行超音波检查。厚度大约是一根细胡萝卜的小探针插入妳的阴道,看看妳怀孕了几周。假如子宫内的胎儿长度多于6.6公分,就会被认定为超过十二周大。检查可以知道一切,因为很多女性有月经不规则或不记得自己最后一次月经时间的情况。此外,也让医生确保妳说的是实话。如果有任何的疑问,超音波检查是妳怀孕多久最合理的解答。





堕胎的两种方法

堕胎的方式有两种:药丸或是小手术。

使用药丸的堕胎称为药物流产,而另一种方法则被称为人工流产手术或子宫刮除术。





药物堕胎

药物堕胎的流程通常从妳在医院或在医生面前服用药丸开始进行。药丸中含有一种叫米非司酮(mifepristone)的物质,让身体以为妳不再怀孕。所有复杂的过程都在确保受精卵长成胚胎,接著中断变成胎儿后的运作。堕胎已经开始,却并非完全运作,因此胎儿留在妳的子宫。虽然过程不完整,这并不代表妳在服药后可以反悔⸺照理来说,胎儿不会有更进一步的成长。

当妳服用药丸后,必须等待一到两天。这段期间有轻度恶心、轻微出血和经痛现象完全正常,除此之外,妳可以正常作息。然后,堕胎经过大约两天的时间会完全结束。如果妳是怀孕不到九〜十周的健康女性,可以在家里进行。如果妳怀孕达到此周数的话,有个成年人在妳身旁很重要,例如妳的朋友或伴侣。这是基于安全考量以防出现并发症⸺虽然并发症非常罕见。在许多地方,如果妳希望,或如果妳已经怀孕超过九〜十周,以门诊病人的身分到医院或诊所,服用最后一轮的堕胎药。

不管妳在哪里进行,方法都是一样的。将四颗米索前列醇(misprostol)放入妳的阴道或舌头下。在堕胎是违法的国家,女性渐渐愈来愈普遍在网上或透过其他方式取得米索前列醇进行堕胎。药丸引起子宫收缩并挤出里面的东西⸺排出形式和妳的月经有点像,只是这一次妳子宫里的小小胚胎会随著血液一起出来。

一旦堕胎进行,妳的出血会比正常月经更加严重。出来的血液会凝结成红色。如果妳害怕看到胚胎,只能说妳愈早堕胎,看到的机会也就愈少。挪威堕胎的时机大多在怀孕第九周前,胎儿为1.5公分长的透明蝌蚪状,被黏液和血液包覆。任何妳在网路上看过的可爱的迷你婴儿图片完全误导大家,造成女性对堕胎感到内疚。

对于95%至98%的女性而言,堕胎只是几个小时的事情。 妳必须记得按照医生的指示服用止痛药,因为可能会感到疼痛。如果妳堕胎后仍然有剧痛、发烧或严重出血的情形,必须打电话给医院或去急诊室。大家常说,如果妳在不到两个小时内出血超过一个夜用型卫生棉的量,妳就应该向医生联系。

堕胎后,连续二〜三周中度出血与感受到一点疼痛非常正常。在这种情况下,为了预防感染,使用卫生棉而非卫生棉条很重要。此外,妳不应该在出血的时候有性行为。只要妳有流血,这代表子宫仍在排出怀孕的残留物,同时任何细菌会透过阴道轻松进到妳的身体系统当中。堕胎后受到感染并不常见,但仍然需要采取预防措施。

妳常常在媒体上看到许多女性进行药物堕胎,却发现自己还仍然怀孕数个月的恐怖故事。如果妳遵照医生的指示,就不可能会发生这种状况。一百名患者在药物流产后只有一位可能会保持怀孕的状态。妳会被告知这种情况的发生,是因为妳在阴道里放入最后一轮的药丸后会有不正常的出血。如果出现这种情况,妳应该再次迅速联系医院。服用药丸停止怀孕后,若子宫里还有残留物不是好事。所有堕胎后的女性应该在一个月后进行验孕测试,以确保怀孕完全终止。此外,如果出血停止后,妳的月经还没有在四〜六周恢复,应该向医生联络。





手术堕胎

手术堕胎的过程稍微不一样,而且必须在医院或诊所进行。妳通常会拿到两颗药,在安排流产的当天早上将它们放入阴道,这些药丸会导致子宫颈扩张。如果妳打算在堕胎过程麻醉,妳必须从手术前一晚禁食。这代表妳不应该吃、喝或吸烟。很多地方的手术只使用局部麻醉进行堕胎。

手术本身大约十分钟,手术器具从阴道进入,接著来到子宫颈。紧接著,医生使用小型吸引器吸出胎儿与胎盘,然后轻轻刮下子宫内膜,确保都已经被移除。堕胎后,妳必须留在医院几个小时,这样医生可以检查一切进展顺利。在那之后,妳可以在当天回家。

如同药物堕胎,经过一阵疼痛后,妳可能会出血。也一样要遵守使用卫生棉与不能性行为的规定,同样地,如果妳变得不适、出血严重或六周后月经没有再次开始,妳应该要与医生联系。

和所有手术一样,麻醉或手术本身有关并发症的危险性不大,包括子宫、膀胱或尿道的损坏。这些非常罕见的并发症也就是为何许多国家会建议进行药物堕胎的原因。避免手术堕胎一直都是上策,但由医疗专业人员进行的手术是非常安全的。很多女性喜欢手术堕胎,不想经历过程冗长的药物堕胎。

有些人可能听说过手术堕胎后可能会更难怀孕,这种印象是来自一个名为阿休曼症候群(Asherman's Syndrome)的罕见疾病。如果医生必须从子宫刮出大量组织的话,最后会破坏子宫内膜的最深的一层。然后,妳可能会有子宫伤口和黏连的现象,这可能使妳之后难以怀孕。现在的妇科医生都会害怕这种事发生,所以会尽一切所能以保安全。换句话说,一个简单的子宫刮除会对妳以后的怀孕机会产生任何影响是不可能的。但只要妳子宫刮除的次数愈多,风险就愈大。 这是为什么堕胎不应该被用来当作避孕手段的原因之一。

发现自己意外怀孕可能是一个震撼的体验。当然,对于一些人来说会是一种惊喜,但是仍会在不寻常中感到一丝恐慌。在妳可能还没有准备好时,怀孕会触发许多情感过程。如果发生了,有人可以诉说就再好不过了。无论妳选择怎么做,医疗服务里遇到的每个人都有保密的义务,并可以为妳提供指导⸺不管妳最后不得不堕胎、留下孩子或去送养。不管妳选择怎么做,和妳的伴侣、朋友和家人征求意见和关注也是明智的作法。





私密处的疾病





我们的生殖器就像我们身体其他部分一样。只要一切正常,我们就不会多加留意。一旦出现问题,可能就会成为伤神伤脑的事情。任何有严重酵母感染的女性,或因经痛所苦的女性都能明白。在这种日子里,我们可能会咒骂自己为何身为女性。为什么不能把我们每月一次的痉挛换到睾丸上发作呢?

本书的这一部分将提到所有可能对我们下体带来麻烦的疾病。我们非常确信大多数女性在生命中会遇到一些这样的情况。幸运的是,像子宫颈癌等症状其实非常罕见。

当我们在写到这一部分的时候,我们发现自己也不确定是否会在最后造成更多焦虑。借由谈论症状模糊的罕见危险疾病,我们会不会让女性暴露于不必要的新担忧呢?

我们希望、同时也相信事实并非如此。记住,妳的身体会一直发出健康或疾病的小讯号。我们应该注意到我们还活著的事实⸺毕竟我们不是机器。但是,我们当中的某一族群比其他人更加在意这些讯号,这可能导致健康上的焦虑。我们认为这种焦虑的最佳药方就是获得更多的知识。更多的知识可以给妳安全感,然而用Google搜索模糊、常见的症状吓自己只会让恐惧变得更严重。秘诀在于区分我们现在和以后经常会遇到的一些现象,以及哪些可能是更为严重的迹象。

在我们身为性健康作家的工作里,我们发现女性普遍缺乏妇科疾病的知识。很多女性都在为周遭从来没听过的疾病纠结,她们常常感到孤独和不信任。许多人不知道去哪寻求帮助。例如,我们在开始念医学系前从来没听过子宫内膜异位症。即便如此,十分之一的女性正游走这种疾病中,有许多人努力使自己的日常生活适应这种痛苦,不应该是这样的。想像一下,如果每十名男性中就有一人每个月因为睪丸受到极大痛苦而不得不休假一个星期呢?这是一个全国性的问题,也是每所学校课程里会有的内容。

换句话说,是时候谈到我们的问题了。这是确保人们获得所需帮助的唯一途径。也许可以将更多资源分配到女性疾病的研究,让我们在未来找到更好的治疗方式。我们可以寄予希望。

我们将从最常见的问题开始:出血性疾病。





出血异常⸺月经毁了




对于大多数女性而言,月经是生命中很重要的一部分。从青春期开始,直到四十五到五十五岁之间(可能更久或更短),我们的月经周期都是每个月的循环,加上一些例外。

当妳的月经出现问题,周期与妳应该的认知不一样时,担心和困惑是正常的。可恶,妳会这么想⸺妳也不是唯一有这个念头的人。奇怪的是,血液和黏液在子宫的变化虽令人担忧,但女性们却更容易相信自己的私密部位出了什么问题。妳的想法全都纠结在妳的脑海里,同时陷入危机。我有什么问题吗?按照计划,从现在开始的十年后,我能不能有孩子呢?是癌症吗?这是一种疾病吗?我是认真的⸺谁来帮帮我!

出血异常有许多不同的类型,可能和疼痛、不规律、流量问题有关,或者妳的月经可能就此停止。我们来讨论一下最常见的问题。





当妳的月经停止时




最常见也最可怕的事情就是妳的月经消失得无影无踪,或有迹可循。尽管平时的经血已消失在稀薄的空气中,但是有时候妳仍会发现微量或点状出血。

如果过去月经规律,但月经突然消失超过三个月或者月经不规律达到九个月的女性,我们称为闭经(amenorrhoea)。我们所说的规律是指,妳的月经周期时间每次都是一样,而且每次都在同一时间来潮,这样妳就能透过经期月历来预测月经的时间。希腊语的闭经意味著"没有月经",完全等同字面上的意思。

女性停止月经是常有的事。十六至二十四岁的女性中,每年有8%的人经历这种情况,而且可能有不同的原因。月经停止时妳首要考虑的事情,就是自己是否怀孕。但是我用了保险套,难道没用吗?当妳月经迟来三天时,妳会这么想。妳现在还没有准备好接受孩子,恐慌即将浮上心头。

在适当的时候进行验孕测试可以排除怀孕的可能。如果有任何的可能,检查自己是否怀孕非常重要。避孕失败吗?错过服用避孕药的时间吗?妳是否依赖体外射精或安全期避孕呢?买一支验孕棒⸺在无防护性行为或避孕失败后三周内便能得到可靠的结果。如果妳没有发生性行为,或正在使用安全的避孕方法(例如植入式避孕棒或荷尔蒙避孕器),那么就是别的状况了。当妳有任何疑问,就进行验孕测试。但是,可能是其他的原因造成妳的月经消失了。

旅行是闭经一个罕见却有趣的原因。我们不知道为何会发生这种情况,但长时间的飞行,特别是跨越好几个时区,可能会影响妳的月经周期,导致出血的时间错误,就像时差一样。

但失去月经的两个更常见的原因是体重变化和大量运动。体重变化的幅度难以定义,也无从得知需要达到多少运动量才会发生。专业运动员往往有闭经的情况,但妳不必为了当专业运动员而耗掉妳的月经。根据最严格的厌食症诊断标准,闭经是其中的项目,这并不表示因体重变化失去月经,就一定是厌食症。

心理压力也是常见病因的一种。妳的情绪状态影响妳的月经。也许妳学校课业太繁重了,或者妳经历过诸如战争、事故或家庭死亡等重大心理创伤。

简而言之,妳的月经是妳有力气的象征。为了让妳怀孕,妳的身体必须足够强壮才能生产。怀孕是一种紧张状态,如果因为种种原因,妳失去了生下孩子所需的能量,妳的月经往往会停止,阻止妳怀孕,因为妳还没有准备好。一切都息息相关。身体、心理和月经无一例外。如果妳的月经停止,而妳不知道发生了什么事,看医生是完全合理的。

导致月经丧失的疾病包括多囊卵巢综合症和代谢疾病。记住避孕是如何影响月经也许是一件好事。黄体制剂产品,如荷尔蒙避孕器、避孕注射剂,无雌激素药丸和植入式避孕棒往往会导致月经停止。这很正常,并不代表有什么不妥的地方。使用避孕药时的出血不是正常的月经,而是我们所说的停药性出血。不像正常的月经,这并非储备能量的象征。如果妳因荷尔蒙避孕而出血,那么妳就不是闭经。

最后,自有经期后的前两年里,乱经的现象绝对正常。包含短暂的月经停止。妳的荷尔蒙需要一点时间才能达到平衡,让排卵按月进行。它会自行调整。





好痛啊!




半数以上的人经历严重的经痛:在我们的下腹有令人不适的痉挛疼痛。只要妳排除任何特殊原因的可能,例如导致更严重经痛的疾病,这就是所谓的原发性痛经(primary dysmenorrhea)。如果疼痛有根本的原因,就被称为继发性痛经(secondary dysmenorrhea)。痛经意味著"经痛"。有些女性在背部,大腿或阴道也会有疼痛感。疼痛在月经的头几天最严重,并时常伴随其他症状,如恶心、呕吐和腹泻。每六名女性中就有一人遭受如此严重的疼痛,每个月都需要休息几天。

经痛是由子宫收缩所引起,在每次周期结束时,这些小小的空心肌肉紧缩并排出子宫内膜,最后以月经的形式出现。

子宫很强壮⸺也许有点过于强大。它紧紧地压缩,以至于无法屏住呼吸,而且很痛!妳的子宫当然不会呼吸(只有妳的肺部能这么做)但是妳身体的所有细胞都需要氧气。没有这些,它们会窒息而死。氧气挟带于血液当中,子宫却紧紧地夹紧肌肉,并在这个过程里完全切断自己的血液供应。因为它急于摆脱旧的子宫内膜。组织缺乏氧气就是导致疼痛的原因。

但是,等等⸺妳有没有听过这样的事情?如果妳在卫生服务部门工作,或者妳有一个患有心绞痛(angina)的爷爷,这种疼痛的情况听起来十分熟悉。事实上,由缺氧引起的疼痛正是人们在心脏血管阻塞时会有的状况。他们会在运动中体验到胸痛。如果爷爷上楼,他的心脏需要更多的氧气,但是他狭窄的血管无法快速输送血液。接著心脏遭受"缺氧的痛苦"。完全相同的事情发生在妳的子宫收缩时。

妳也会从心脏病引发胸部疼痛。在这种情况下,氧气很少,妳的心脏就会窒息死亡。如果妳现在开始有点担心,我们向妳保证:经痛与心脏病发作不一样⸺它们并不危险!虽然在相同的状态下认为缺氧是导致疼痛的原因有点奇怪,但妳不会因为痉挛失去妳子宫的一部分。不一样,却有点相似。

那么为什么有些人会这么痛苦,而另外一些人却觉得仿佛像微风吹过呢?

答案在于妳体内的酶(enzymes)有多活跃。酶是小型的蛋白质,确保妳身体中所有的化学作用都遵循正确的过程。有一组叫做环氧合酶(COX enzymes)的物质,与制造前列腺素(prostaglandins)有关。其中,前列腺素是引发孕妇生产的物质。他们造成子宫收缩,反过来导致我们刚刚谈到的缺氧现象。

有些专家认为,那些特别痛的女性,她们的环氧合酶会特别活跃。结果却是,她们的前列腺素比其他人更多。这会导致子宫收缩强度增加,在放松与收缩之间游走。前列腺素还会使生殖器区域的神经过度疼痛。

如果妳想知道妳的痛苦门槛是否很低,或当妳描述妳的痛苦时发现人们不相信妳,这里有一些生产的比较,应该可以让大多数人闭嘴。根据观察,痛经女性的子宫收缩可以达到相当于150〜180毫米汞柱的压力。或许现在来说对妳没有任何意义,但相较之下,在生产推挤阶段的压力大约是120毫米汞柱。生产期间,女性每十分钟有3〜4次的子宫收缩。经期里,同样状况,痛经的女性可能会发生4〜5次。换句话说,可怕的经痛压力至少和推挤阶段的压力一样高,而阵痛间隔稍微短些。所以现在妳可以明白,这可能会痛苦到死。幸运的是,这些可怕的疼痛通常会随著时间缓解。

妳可以使用止痛药来治疗月经痉挛,但重要的是要正确使用它们。布洛芬(Ibuprofen)能够直接抑制环氧合酶,确保产生较少的前列腺素。这就是为什么布洛芬和类似的药物,被称为非类固醇消炎药(Non-Steroidal Anti-Inflammatory Drug,NSAID),是月经痛最有效的药物。如果妳经常有严重的经痛,妳应该在月经开始的前一天服用布洛芬,或者至少在意识到轻微疼痛迹象发生时服用。之后,妳应该在月经的头几天里每6〜8小时服用一次止痛药。有太多人等到真的痛了才吃止痛药,不幸的是,由于前列腺素已经产生,所以止痛效果较差。

此外,大多形式的荷尔蒙避孕也对经痛有很好的效果。避孕药也是一个更长期的解决方案,因为妳会不断地使用。

最后,我们必须指出,有些人可能有不同、潜在的经痛原因。对于发现疼痛随时间变化、突然增加、悄悄来袭的女性来说尤其如此。以前不是这样的。这可能表示妳有子宫肌肉结,也就是子宫肌瘤(fibroids)或子宫内膜异位症,在子宫外产生多余的子宫内膜。含铜避孕器也可能导致疼痛增加。如果妳有这种状况,现在是时候改用另一种避孕方法了。

如果妳遇到突如其来的剧烈疼痛,妳可能会认为是更严重的急性状况。例如,子宫外孕。如果受精卵没有按照正常的方式进入子宫,就会发生这种情况。接著胎儿开始在输卵管里发育,但是那里却没有足够的受孕空间。子宫外孕可能会以严重的经痛呈现,一般会集中在同一侧。在这种情况下,妳就得去一趟急诊室了。





月经不规律




在妳初经后的第一年与停经前几年,或者使用荷尔蒙避孕药的时候,妳的月经有点不规律是正常的。开始月经之后,妳的周期需要一段时间才能稳定,当妳使用荷尔蒙避孕药时,妳不再有正常的月经,因为妳的周期不像以前一样。除了这些情况之外,妳的周期应该会稳定下来,或多或少,介于21〜35天之间。

但是如果妳已经有好多年的月经,出血情况仍然(或者突然变得)像《控制》(Gone Girl)的剧情一样不可预测,那么妳应该注意了。不规律出血可能与许多状况有关。可能是无预期点状出血(在每次月经之间的小滴血)在性行为当中,或者与其相关的出血。

除了月经消失之外,压力、体重变化或过度运动也会让妳的月经延迟或突然来潮。这些事情影响著我们的荷尔蒙。其他原因有可能是潜在疾病,例如多囊卵巢综合症或代谢疾病。

子宫颈癌或性感染疾病(STI)可能会导致子宫颈轻微流血。如果发生这种情况,性行为过程或结束后可能会触发轻微出血。正因为如此,妳应该让医生检查与性行为相关的出血。

如果妳正在使用复合避孕法(避孕药、避孕贴片或阴道环),并出现不规律出血,和妳的医生或护士讨论会是一个好主意。换成更多雌激素的产品也许会有所改善。避孕药中有两种不同剂量的雌激素。例如,乐婷锭是一种低剂量的药丸,而欣无妊和奥诺康则含有较多的雌激素。除此之外,它们都是一样的。当许多女性改用含有更多雌激素的产品时,会发现不规律出血状况停止。





太多血了!





妳在超市的卫生棉条架上看到不同的尺寸,而妳的女性友人不一定会像妳一样出血。对于部分女性来说,即使是黄色迷你卫生棉条也会觉得太多。那些血量最少的女性只需要在短裤上黏上一张舒洁卫生纸就能解决问题。其他人每隔几个小时不得不更换一次绿色大流量卫生棉条,而且因为出血的恐惧使她们渴望更高的吸收效果⸺超级增增增增增量。相信妳脑海中有画面了。





每位女性失血量往往会相差很多,不过平均介于25〜30毫升之间,也就是说,在当地的一家咖啡馆里,大约只有一杯浓缩咖啡的量。双倍浓缩咖啡的量也算正常。

妳是这些正在嘲笑他人的其中一员吗?一杯浓缩咖啡?在整个经期中?哈哈—太可怜了!至少一天两杯吧!

有些人的月经量比起当地咖啡馆更像巴托里夫人(Lady Báthory)的浴缸。巴托里夫人是外西凡尼亚(Transylvania)的一名连续杀人犯,据说她为了保持自己的青春而沐浴在处女的血液当中。但事实上不会有那么多,尽管有种出血不会停止,而妳也会有血液流过卫生棉条、内裤、裤子,直接流到婆婆白色沙发上的感觉,但没有人在会经期内流满整个浴缸的血。实际上,大概7名女性一生的经血才能填满20公升左右的浴缸。然而,还是有很多女性因为贫血而服用铁剂。她们变得呆滞和苍白,经常头痛,不能去做她们喜欢的事情而烦恼。月经真的可以让妳失去妳的锋芒!

如果每个周期的出血时间超过8天,或者超过80毫升⸺超过两杯半的浓缩咖啡,那么就会被认为非常严重。不完全是一个浴缸的量,但大量的出血都是一样的。

在初经来潮后的初期,年轻女孩血量较多的情况很普遍。可以随著时间的推移而改善,而且甚少引起担忧。然而,有些女孩会有极大的血量,检查是否为潜在疾病引起的现象是明智的选择。某些血液疾病实际上可能让妳血量比其他人多而且更容易出血,不过这种情况却很少见。

含铜避孕器是严重出血常见的罪魁祸首。许多女性发现这种避孕方法运作良好,而也有人发现她们的经血量与疼痛增加。对于以前有大量出血的女性更是如此。复合避孕药可以治疗大量出血,因为它们能够有效控制出血。黄体制剂产品,像荷尔蒙避孕器,往往造成妳的月经消失,或大幅减少血量,同样也是常胜军。

来经一段时间并且逐渐有大量出血问题的女性可能罹患一种潜在的疾病,像是混淆妳荷尔蒙的多囊卵巢综合症。大量出血也可能是因为子宫肌瘤,子宫壁上的肌肉结所致。妳可以在本章后面阅读更多关于这些疾病的内容。





子宫内膜异位症⸺血腥的特许假期




经痛是女性们认为理所当然的事情,但有些人的经痛,严重到她们不得不把整个生活搁在一旁。 她们每个月有几天抱著热水瓶蜷缩在沙发上,像吃糖果一样地服用止痛药。不应该是这样的。如果妳也是如此,妳可能受到一种叫做子宫内膜异位症的疾病所苦,这种情况影响约十分之一的女性。约有三分之一下腹和生殖器与这种疼痛奋战的女性患有子宫内膜异位症16 。当然,这并不适用于阴户本身的疼痛,我们会再回来提到。

正如妳可能从名字中了解意思,子宫内膜异位症与子宫内膜⸺也就是子宫内部的黏膜有关。这是每次周期在妳的子宫准备接收受精卵时会形成的黏膜。如果妳没有怀孕,它会以月经的形式从子宫中排出,不过妳也早就知道了。子宫内膜异位症不同的地方是患者在子宫腔外也有子宫内膜。在某些情况下,子宫内膜误入子宫肌层中,这个情况称为子宫腺肌症(adenomyosis)。

无法确定这种子宫内膜是如何出现在子宫外面。主要的一个理论是,月经运作的方式错误,也就是说,它会从输卵管而不是在子宫颈外增生,最后留在胃里。这种情况发生在所有月经来潮中的女性身上,但似乎还有一些易感体质的女性无法自体清除。当发生这种情况时,一小群黏膜细胞就会误会它们所属的地方,例如:卵巢、骨盆、肠道或腹部的其他部位,接著停留于此。

这些子宫内膜细胞通常在靠近生殖器官内部被发现,但是在一些非常罕见的情况下,它们可以远至包围肺囊。这促使一些人怀疑除了误入歧途的月经之外,是否还有其他机制引起子宫内膜异位症。也许是一种干细胞(也就是可以成为任何所想的细胞)生长在错误的地方?或者是从子宫内膜透过血液输送到其他部位的细胞呢?我们大概会在几年内找到确切的答案。

就像太阳海岸的英国退休人士一样,尽管它们已经找到一个新家,但是这群子宫内膜并没有忘记它们来自哪里。它们的一举一动就仿佛住在子宫里。这代表它们与普通的子宫内膜相同,对月经周期中的荷尔蒙产生反应。令人难以置信的是,这意味著每个月,妳的子宫外还有一个小小的经期。

错位的月经并不是一件普遍的事情。子宫内膜在安静有序的社区定居下来时,与免疫系统非常倔强地打了一仗。因为身体本身在某处对应该发生的事情有严格的规定。当这些子宫内膜在不属于它们的地方开始流血时,叛乱迅速发生。新邻居突然被一阵意外的血淋淋冲击,所以自然而然地打电话给警察(我们的免疫细胞)以最快的速度到达现场处理。结果妳子宫内膜周围的组织开始发炎,并因此受伤。





由于子宫内膜通常位于靠近子宫的地带,大多数人很难将这些疼痛与严重却正常的经痛做区隔,尽管有些女性也会发现她们在奇怪的地方有疼痛感。例如,如果内膜群在尿道附近定居下来,排尿时可能会疼痛,或者它们待在直肠里最快乐的话,那么排便时就会很痛。

这些类型的疼痛都有一个共同点,那就是它们有循环⸺意即它们遵循固定的模式。它们经常在月经前一、两天,也可能会在结束后连续几天发作。与平常经痛相比的差别在于,它们通常在妳初次开始月经的几年后逐渐发展。有些人在十几岁时就经历过这种疼痛,但这种情况并不常见。因此,直到十九岁后,才会被诊断出子宫内膜异位症。

随著时间的推移,内膜群周围的发炎可能会导致疤痕和身体内部的沾黏。例如,膀胱可能黏附到邻居,子宫。这些内部伤疤会导致不同类型的疾病,如慢性疼痛。生殖部位的慢性疼痛是子宫内膜异位症患者常见的问题。许多人在性交过程中也经历了极度刺痛。疼痛会发生在妳腹部的最底部,而不是在妳的阴道或阴户。

另一个问题是许多子宫内膜异位症患者难以怀孕。大约四分之一的非自愿不孕案例的原因是子宫内膜异位症。我们不清楚人们为何有生育问题。疤痕和沾黏可能会损害输卵管和卵巢,但看起来其他身体机制也是罪魁祸首。免疫系统和荷尔蒙似乎也都有所关连。如果妳正在努力怀孕,同时有子宫内膜异位症,人工授精可能会有帮助。除了人工授精之外,也可以透过手术的方式。子宫外的子宫内膜群切除手术帮助一些女性能够自然地透过人工授精来怀孕。我们建议手术进行一次就好,而且应该将这个机会保留到打算怀孕时使用。

我们不知道为什么有些女性会罹患子宫内膜异位症。某种程度上这是遗传性疾病,但似乎还有许多其他因素。据我们所知,妳没有办法避开子宫内膜异位症

。因此这只是一个运气不好的问题。就好比一些爷爷奶奶喜欢太阳海岸而其他人喜欢农村,有人喜欢夏天就有人喜欢冬天一样。同样地,有些人的子宫内膜想要搬到子宫外面。

子宫内膜异位症的问题是,它无法经由简单的检查被诊断出来。血液检查、妇科检查与核磁共振等影像检查很难或无法告诉我们这些耍特权的子宫内膜是否存在。我们唯一能确定或排除女性是否患有子宫内膜异位症的方法,就是打开肚子看看里面。通过内视镜手术⸺用相机从小孔窥视胃部就能完成。和所有的手术一样,可能会出现并发症,所以除非有严重或造成其他疼痛的问题,否则就不会这么做。

除了进行这项手术外,医生经常会做的则是尝试子宫内膜异位症的治疗,看看是否有效。幸运的是,大多女性的治疗很简单。这也是无害的:不间断服用避孕药,或使用荷尔蒙避孕器和止痛药也能镇定发炎的状况。布洛芬就是属于这类药物的一种。持续服用避孕药,防止子宫内膜群出血,有可能导致它们随著时间而缩小。布洛芬对疼痛有效,并同时降低发炎现象。这种疗法的目的并非消除子宫内膜群,不过问题会得到解决。

如果这种方式起不了作用,治疗子宫内膜异位症还有其他更复杂的方法,譬如说手术或者更强效的荷尔蒙治疗。这是涉及专业的工作。不幸的是,子宫内膜异位症是一种直到更年期才能结束的慢性疾病。疗法不能治好这种疾病。即使手术切除后,子宫内膜群也会随著时间的推移而复发。尽管如此,妳应该要知道减轻痛苦的办法。





* * *



16	很难知道有多少人受影响,因为许多女性没有明显症状,且诊断方式只有透过手术才能判定。





多囊卵巢综合症⸺未知的女性疾病




正如我们一位女性朋友喜欢说的:"唯一比月经更糟的事情就是没有月经。"很多女性担心,她们的月经消失或是流量一次比一次更少。月经不规律或经血量不多的常见原因是多囊卵巢综合症(polycystic ovary syndrome,PCOS)造成的影响。妳以前没有听说过吗?那么,妳不是唯一的一个,我们应该要更加注意这种疾病有许多充分的理由。实际上,这是来到生育年纪的女性中最常见的荷尔蒙失调症,影响程度在4%〜12%之间,其中有许多人并不知道自己有这个症状。

疾病的名称源于多囊卵巢综合症是经常在卵巢上发现的囊肿。这些充满了透明液体的小水泡,使卵巢看起来有点像一串串的葡萄。与其他类型的卵巢囊肿不同,这些囊肿很小,不会破裂,所以妳不会注意到它们在那里。

虽然这是多囊卵巢综合症最著名的特质,但却只是疾病的一小部分。多囊卵巢综合症是一种症候群,意即由许多不同的疾病组成,这些疾病通常不会一起发生。而这些症状是由一些荷尔蒙系统疾病所引起。它们不只沾黏卵巢,还包括胰腺、消化系统以及位于大脑中形状如小型阴囊的脑垂体。

卵巢的任务是储存所有的卵子,并确保每个月都有排卵。如果妳有多囊卵巢综合症,这些任务可能会发生问题,因为大脑的脑垂体和卵巢都会产生控制月经周期的错误荷尔蒙浓度。结果造成妳少量排卵或是根本没有排卵。妳会在日常生活中注意到这一点,因为妳的经血将会更少或完全消失。

为了怀孕,排卵是必要的,许多多囊卵巢综合症女性将耗费比平常更长的时间怀孕,或者需要借由帮助才能达成。多囊卵巢综合症是女性怀孕问题最常见的原因之一。流产和妊娠性糖尿病(gestational diabetes)等怀孕并发症的风险也较高。

令人怀疑的是,未治疗多囊卵巢综合症的女性在晚年面临子宫内膜癌的风险更高;这是西方世界女性最常见的生殖器癌症。在一项研究发现,健康女性一生中罹患子宫内膜癌的风险为3%,未经治疗的多囊卵巢综合症女性则为9%。

未治疗多囊卵巢综合症导致罹患子宫内膜癌风险较高的其中一个原因是,女性的子宫内膜与多囊卵巢综合症一直增生,却无法形成月经并脱落。因此,子宫内膜的细胞变"旧",开始行为异常。为了确保女性一年有超过三、四次月经出血,借由避孕药或荷尔蒙疗程的帮助便能更容易预防。

这边要澄清一点,旧子宫内膜发生的状况和使用荷尔蒙避孕暂停月经是不一样的。多囊卵巢综合症里,子宫内膜连续接收,不断传达生长的信号,而荷尔蒙避孕则是防止子宫内膜的生长。虽然这两种情况下的结果都是月经变得更少,但是形成机制却有很大的不同。

除了所有对排卵的争论外,卵巢(以及脂肪组织和肾上腺)可能产生过多的雄性荷尔蒙,即所谓的雄激素。所有女性会产生一些雄激素,但浓度维持平衡往往对女性有利。如果雄激素占上风,妳可能发现脸上会长胡子,或者有一条厚实的"快乐小径"⸺妳的肚子上长出广泛的毛髪。这就是所谓的多毛症,半数以上有多囊卵巢综合症女性为此感到困扰。很多患有多囊卵巢综合症的女性也有持续冒痘的问题,持续时间远远超出青春期。她们发胖的方式也受到影响。女性身体往往会呈现梨状(大多数的脂肪围绕在她们的臀部和大腿),但多囊卵巢综合症患者的男性性荷尔蒙导致女性体型呈现苹果形状⸺脂肪围绕在肚子上。妳甚至会有最不健康的脂肪类型之一,啤酒肚。然而,性荷尔蒙也有不易察觉的作用。例如,妳的血液中可能会有高含量的胆固醇和脂肪酸,这对我们的血管壁没有好处。

多囊卵巢综合症常常表现异常的第三个地方是胰脏。这是消化系统产生分解食物的荷尔蒙物质⸺胰岛素的器官。胰岛素是饭后发号施令和发送触发摄取和消耗体内血糖的讯号细胞。50%〜70%的多囊卵巢综合症女性,细胞对胰脏的胰岛素信号没有反应。女性产生抗胰岛素作用,因此胰脏产生更多的胰岛素来补偿,希望信号最后能够传递出去。没有人因妳讲的笑话发笑吗?那就讲大声点!

高浓度的胰岛素对身体没有好处。如果妳抗胰岛素状况没有获得控制,妳可能会渐渐发展成第二型糖尿病。多囊卵巢综合症患者罹患糖尿病的可能性比其他相同体重和生活方式的女性要高得多。美国研究表示,多囊卵巢综合症患者有20%〜40%的人位于糖尿病的初期或在四十岁时发展为第二型糖尿病。

抗胰岛、血脂浓度异常和周围腹部脂肪增加是心血管疾病的关键因素。当妳年纪大了,这些变化可能会导致心血管疾病增加的风险。

正如之后所理解的,妳应该认真看待多囊卵巢综合症。如果妳有月经不规律的情况,多囊卵巢综合症可能就是原因。要检查是否有多囊卵巢综合症,医生会衡量妳的荷尔蒙浓度并用超音波检查妳的卵巢是否有囊肿。如果妳变成多囊卵巢综合症的其中一员,考虑一些确保妳未来健康的事很重要。

对多囊卵巢综合症女性最重要的建议与控制体重和改变生活方式有关。如果妳过重,减肥可能会减少问题。如果妳的体重是正常的,妳当然不需要考虑这个。减肥说比做更容易,不过任何运动与健康饮食会改善妳的健康!因为多达五分之四的过重女性,仅仅降低5%的体重(例如从80〜76公斤)就足以恢复正常的排卵。此外,也可以降低抗胰岛素、糖尿病和心血管疾病的可能性。头发增生和痘痘的问题也会减少,因为过重增加了男性性荷尔蒙的生长。

我们甚至建议妳开始与对多囊卵巢综合症非常了解的医师讨论如避孕药、避孕贴片或阴道环等复合产品。这是多囊卵巢综合症治疗里最重要的部分之一。避孕药的雌激素会减少卵巢内男性性荷尔蒙的产生和活动,有助于改善头发增生和痘痘的问题。此外,它可以进一步减少囊肿的发展和子宫内膜癌的风险。因为血栓而不能服用雌激素的人,可以使用无雌激素避孕法,如荷尔蒙避孕器或植入式避孕棒。但不幸的是,这些对男性荷尔蒙不会造成影响。

思考看看妳是否想要孩子,如果妳想,不要长时间使用是明智的选择。许多多囊卵巢综合症患者可能需要怀孕的帮助,而这个过程需要时间。事先准备好会是一个好主意。





肌瘤⸺带球的子宫




妳最后一次见妇科医生有不愉快的经验吗?很多人的子宫内都有良性肿瘤,也就是肌瘤。毫不意外,当妳听到肿瘤一词用在自己身体上时,妳的血液瞬间降温。但是在这种情况下,妳可以松一口气。只要躺在妇产科的椅子上深呼吸就好。细胞生长在子宫肌壁的子宫肌瘤是良性的肿瘤,和癌症一点关系也没有。它们不是癌症,也永远不会变成癌症。医生可能会将子宫肌瘤称作肌瘤或"肌肉结",这样应该更容易明白良性和略不良性肿瘤之间的差异。

子宫肌瘤是由我们所谓的平滑肌构成,换句话说就是肌肉。就像我们的肠道和胃,我们不能有意识地控制它。子宫肌瘤往往以弹性球体呈现。如果妳桌上有一个,妳可以用刀子将它一分为二,会看到它里面实际上是像珍珠一样的白色,而不是妳所想的红色。子宫肌瘤看起来有点像珍珠⸺在海底生长的牡蛎。

肌瘤可在子宫不同的地方生长,壁内、壁外或突出至子宫腔。有些女性的子宫肌瘤只有一个,但常见的有多达六、七个。它们可能很小或者在更糟的情况下,增大为一个葡萄柚的大小。子宫肌瘤不一定随著时间增长。有些人可能会在很短的时间内增长非常多,当它们长到一公分长时又会缩水,并自行消失。

子宫肌瘤是女性更年期很常见的症状。像许多和生殖器有关的东西一样,它们对雌激素有反应,所以只会出现在青春期后,通常在更年期后消失。多达四分之一的女性会自行发觉体内的肌瘤。可能有更多人拥有肌瘤,但往往因为太小,所以没有人会注意到。由于肌瘤只是良性肿瘤,没有必要为了看它们是否存在而去检查。只要不给妳添任何麻烦,拥有它们也不成问题。

妳可能会有严重或长期的月经出血,尤其当它们正在成长进入子宫腔时,但大多数肌瘤不会有其他症状。

经期间出血在子宫肌瘤中不太常见。而疼痛也不是肌瘤的典型症状,但如果长得非常大的话,有一些女性会遭受生殖器受到压迫的疼痛感。另一个例外则是肌瘤因血液供应不良等因素开始破裂、阵亡。这可能会极为痛苦,它可以很可怕(特别是在怀孕期间发生的话)但并不危险。

如果妳想像子宫充满了六、七颗网球大小肌瘤的画面,很容易理解肌瘤为什么会造成问题。例如,它们可能压迫子宫前面的膀胱,让妳有排尿的冲动。它们还会给妳沉重、臃肿的感觉,让人联想到略微怀孕的模样,妳的胃可以长成让妳看起来像怀有好几个月身孕的人。

肌瘤最为讽刺的是,在最坏的情况下可能难以怀孕。还好,这只适用于少数得到肌瘤的女性,但仍然造成1%、2%难以怀孕的女性不育。不能肯定女性肌瘤是怎么抑制怀孕,然而比起大小,位置似乎是主要原因。子宫肌瘤可能很难让受精卵本身附著,还可能封住进入输卵管的入口,使精子无法与卵子会合,只能焦急地等著和不错的约会对象结合。如果怀疑肌瘤是不育的原因,可能就会进行切除,但是效果不一定如此显著。

我们更不确定的是女性一旦想要怀孕,肌瘤会如何影响她们。同样地,似乎是长到子宫腔里的肌瘤导致最大的问题。一些研究表示,肌瘤向内生成时,流产的风险增加22%〜47%。除了因为肌瘤挡住产道可能造成剖腹生产更频繁之外,肌瘤似乎没有对怀孕产生任何重大不利的影响。因此,在有孩子之前,没有理由透过手术移除它们。

要怎么做才能限制肌瘤的生长呢?简单的解决办法就是尽量使用长效黄体制剂产品,如避孕药、植入式避孕棒或荷尔蒙避孕器。如果受到大量出血所苦,服用荷尔蒙避孕药也可以解决这个问题。使用低剂量的雌激素避孕方法不会导致肌瘤生长,所以妳没有理由拒绝这些产品。

对子宫来说肌瘤有点像雀斑:妳可能会有几个或是很多,形状或大或小,但它们不会引起任何麻烦。没必要因为它们的存在就将其切除。妳只需要在它们造成问题时切除就好。请记住:肌瘤不会变成癌症。





阴唇痛症⸺生殖器不明原因疼痛




妳的生殖部位疼痛,而医生和其他医疗专业人员找不到可以解释疼痛的原因吗?妳不孤单,这些缺乏真相的疼痛令人沮丧。可以肯定的是,疼痛就在那里。它们在妳的日常生活中产生不好的影响,让性行为难以进行⸺但它们是从哪里来的呢?现在,我们知道的不多。

总而言之,生殖器部位疼痛的原因很多。有很多的因素会对私处造成伤害。酵母菌感染和其他生殖器疾病造成持续性烧灼和瘙痒;性传播疾病可能导致性交时的疼痛;令我们感到疼痛的皮肤疾病,可能会影响阴户;更罕见的是,生殖器癌也可能引起疼痛;巴多林氏腺可能会发炎而且极为疼痛⸺这样的例子不胜枚举。所有情况的共通点在于它们通常是显而易见的。如果妳把疼痛的现象告诉妳的医生,他会马上替妳检查,并找出疼痛的原因。如果妳的疱疹经常发作,那么生殖器疼痛也没有什么好奇怪的,但是如果找医生检查,却找不到原因呢?

如果妳生殖部位有任何疼痛,不能找到任何明确的理由往往就是阴唇痛症(vulvodynia)所致。痛症来自希腊文"dynia"的疼痛。因此阴唇痛症指的就是阴户疼痛。

有一件事我们从一开始就强调,阴唇痛症的疼痛绝对是真实的,即使医生无法找到任何原因。很多有这种症状的女性都有不被认真对待的感觉,得不到任何明确的答案告诉她们为何会这样。也许她们接受大量的检查,陆续找了好多医生却没有人发现问题。这是否代表疼痛只存在她们的脑海里呢?绝对不是:疼痛感是真实的,我们认真看待妳的症状。

阴唇痛症有几种不同的形式,这可能意味著两件事情:第一,导致阴户疼痛存在一些未知的条件,但是我们知道得不多,因此将它们放在阴唇痛症的总称下;第二,阴户疼痛的不明原因可能随著每个人产生不同的症状。

这件事情的真相到底是什么(尤其是,是什么样的原因导致疼痛)?我们不知道。在相关领域看到更进一步研究的话实在会令人兴奋,因为幸运的是,药物的发展一直稳步推进。在中古世纪,人们相信所有的疾病是由于体液失衡,而放血(也许用的是水蛭)这个奇妙的想法同时能奇迹般地治愈任何从忧郁症到癌症的疾病。给妳一个年代稍微近一点的例子,医生认为生活习惯,例如压力和咖啡是引起胃溃疡的因素。然而,一个叫做幽门螺杆菌(helicobacter pylori)的特殊细菌竟然才是罪魁祸首。

同理可以用在阴唇痛症下。它是一种神经疾病吗?是一种细菌或病毒引起的感染吗?是另一种疗法的对应症吗?我们之后就会知道。

女性的阴唇痛症可能会经历不同类型的疼痛:她们的生殖器可能有自发性的灼热感,或是我们所说的触摸疼痛和痛感过敏。触摸疼痛,是通常不会感到痛觉的刺激,但只要受到轻微压力或触摸,就会突然变得疼痛。例如手指的触摸就能够触发阴户的烧灼痛感。触摸疼痛通常发生在已经以某种方式受到损伤的地方。我们不能确定这是否能解释生殖器部位的触摸疼痛。痛觉过敏指的是往往变得更加疼痛的刺激。

例如,妳一般想要摆脱的针眼可能会导致剧烈的疼痛。这两种痛觉过敏和触摸疼痛属于神经性疼痛(neuropathic pains)。这意味著它们会发生在伤口或周边神经的疾病中⸺也就是大脑和脊髓外部的神经。

灼热的疼痛和神经性疼痛是阴唇痛症最常见的形式,但我们不能肯定其他形式的疼痛不会出现。疼痛可能因人而异,如先前所述,我们不知道阴唇痛症的所有实例是否为相同疾病。另一个重要因素是我们对疼痛的解释不同。

这适用于所有的痛苦,而不只有阴户疼痛。例如,有些人可能会有瘙痒的不适,并认为这是由他们以前熟悉的东西,例如酵母感染所引起的症状。即使酵母菌不是确切的原因,却可能会使人们更频繁地接受抗真菌治疗。

疼痛的部位也会造成不同差异,也是归类阴唇痛症的要素之一。有些人经历整个阴户的疼痛⸺即阴道口、阴蒂周围和阴唇。这称为广泛性阴唇痛症(generalized vulvodynia),稍微年长的女性当中更为常见。其他人则是在外阴的特定位置产生局部疼痛。这就是所谓的局部阴唇痛症(localized vulvodynia),是年轻女性中最常见的。阴蒂或旁边的阴道口的疼痛最常见,这个区域称为前庭(vestibulum),所以这两个局部阴唇痛症都有属于自己的名字:阴蒂痛症(clitorodynia)以及前庭痛症(vestibulodynia)。

阴唇痛症,特别是前庭痛症,以前被称为前庭炎(vestibulitis),这个术语你可能在媒体上听过或读过。当一个医学术语中以"炎"当作字尾,这代表我们正在谈论发炎。举例来说,阴道炎,和阴道发炎一样。当女性有阴唇痛症时,由于没有人能够证明生殖器存在任何炎症,所以医生选择停止使用前庭炎这个名称。称为阴唇痛症或是单纯阴户疼痛会更为精确。

疼痛所表现的方式有所不同。有些女性有所谓的诱发性疼痛(provoked pain),而另一些人则是自发性疼痛(spontaneous pain)。诱发性疼痛通常包括神经性疼痛,也就是痛觉过敏或触摸疼痛。诱发性疼痛意味著我们直接接触生殖器所产生的疼痛。可能会以稍微不同的形式发生,轻触或一般不会造成伤害的压力可能会导致巨大的疼痛。例子包括自行车车座、性交、使用棉塞与阴蒂直接接触产生的压力。妳可能变得非常敏感,即使与宽松的衣物或内裤接触就能引起疼痛。医生经常用来了解妳是否正遭受诱发性疼痛所苦的一项测试则是用棉花棒按压疼痛的部位。例如,他们可能会在阴道口施加压力。

自发性疼痛指的是疼痛突然发生,没有与任何东西有所接触。这种疼痛往往属于烧灼痛感。妳也可能会同时经历诱发性与自发性疼痛。有些女性一直都有灼热感,而其他人却是偶尔感到疼痛。一般来说,局部阴唇痛症,如前庭痛症,最常发生诱发性疼痛,而广泛性阴唇痛症最常发生自发性疼痛,以及透过衣物接触产生的疼痛。

阴唇痛症及其他生殖器官疾病,例如性感染疾病之间并没有发现明确的关联。

然而,一项主流理论表示,酵母感染疗法与阴唇痛症之间有关。这并不一定表示使用抗真菌治疗就会得到阴唇痛症。正如我们前面所写,许多人认为她们经历由真菌感染引起的阴户不适,很自然地,就使用抗真菌治疗来解决问题。这下变得难以确定是不是因为治疗造成问题还是问题导致治疗。

有项研究发现酵母菌感染复发与阴唇痛症之间的关系,但实验是在老鼠身上进行,所以在两足动物上很难得出任何结论。研究中小老鼠经历了触摸疼痛。同样的研究也发现受影响的区域有变得特别敏感的趋势。能够感知疼痛的神经末梢数量已经上升。在这项研究为基础,看起来可能是酵母菌反复感染影响了小老鼠纯粹神经视角感受疼痛的能力。

其他研究则指出阴唇痛症的女性神经分布改变。看起来有些女性的阴唇痛症像是变为更加疼痛敏感的神经纤维。目前仍不清楚导致这些变化的原因。





好女孩症候群?




如果妳在媒体上读过阴唇痛症,妳可能注意到很多人将疾病著重于潜在的心理层面。

许多人如此看待,尤其是性学家研究心理和性之间的相互作用,在与患者接触时也同样强调这一点。也许是阴唇痛症是实际上不想发生性行为的女性所导致的症状吗?难道"好女孩"是那些受到某种影响,或在过去有不好或痛苦性经验的女性呢?那么遭受侵犯或虐待的女性呢?这些说明已用于解释不明原因的阴唇痛症。但是,站得住脚吗?





在身体因素无法立即辨识的状况下很容易贴上"心理因素"的标签,对此我们应该要非常谨慎。如果女性无法分辨自己是否符合这些叙述,可能会导致混乱和愤怒。特别是"好女孩"一词可能建立错误的印象,认为女性本身,或是她的个性,就是造成疼痛的原因。所以能力与勤劳应该是造成妨碍妳存在的身体疾病,这不是建设性的问题。也就是说,生殖部位疼痛可能对一些女性造成极大的心理压力,但却不是耻辱的来源。

很多患者会透过谈话疗法当作治疗阴唇痛症的一部分。这个方式可能会产生效用,不只是因为可以改善疼痛的潜在心理因素,甚至阴唇痛症本身是人们可能需要寻求协助的极大心理负担。

我们知道,各种疼痛与心理紧密连结在一起。很多经历疼痛的人会逐渐养成回避行为和紧张心态,这么做可能会加剧潜在的问题,让患者陷入恶性循环。举例而言,预期性交时会感到痛苦,妳会不自觉缩紧妳的阴道来保护自己,然后在尝试性交时会更加受伤。

此外一项广为人知的疼痛研究显示,当人们长时间与疼痛共存时,大脑对新疼痛的刺激变得更敏感。

疼痛只会滋生疼痛。在这两种情况下,放松技巧和心理治疗可以帮助人们摆脱痛苦的循环。然而,声称阴唇痛症从一开始绝对是心理因素所导致是不同回事。

据我们所知,没有研究指出阴唇痛症与过往的侵犯或虐待之间有明确关系。即便如此,这样的经历可能是一些女性的潜在因素。研究比较了有无阴唇痛症女性的心理轮廓并得到不同的结果。一项对各240位有无阴唇痛症女性的研究中,结果指出阴唇痛症的女性先前经历焦虑相关症状的状况更加常见。

而另一项研究比较的群体数较少,在有无阴唇痛症女性的心理轮廓中并无发现差异。到底阴唇痛症属于一种心理原因的疾病有多接近,是个具有争议的问题。没有受过心理煎熬或性暴力经验仍旧非常有可能得到阴唇痛症。

我们对造成阴唇痛症的原因知道得不多,而疗法却是复杂又处于实验状态。希望改善其他疼痛症候群的不同方法,也能对阴唇痛症有所帮助。尽管如此,第一步还是要找到一个在该领域具有专业知识的医生。妳可以找到具有阴唇痛症专业的妇科医生和家庭医生。妳也可以寻求转诊至奥斯陆或特隆赫姆专门治疗阴户疼痛的诊所,但不幸的是,挂号人数可能会非常多。

正如我们前面提到,某种形式的阴唇痛症都与神经性疼痛有关,在这种情况下,可以使用一些不错的药物如抗抑郁与癫痫药物。这些类型的药物有助于对抗神经疼痛,且已经证实对一些阴唇痛症的女性发挥作用。其他人可能会发现雌激素也有效果,好比像使用阴道环等避孕方式。雌激素会影响阴道黏膜,使之增厚。同时镇痛凝胶也能减少疼痛,而有诱发性疼痛仍想进行性行为的女性可以透过此款凝胶受惠。除了谈话疗法,很多人发现物理治疗会有所帮助。妳可以学习专门的运动,会更容易放松骨盆底的肌肉。许多诱发性阴唇痛症的女性也受到其他如颈部、肩部疼痛或紧张性头痛等肌肉紧绷问题。

经常给阴唇痛症女性的一般建议之一则是不要做任何导致疼痛的事情。例如,如果会痛就不要强迫自己发生性行为。尽管如此,如果妳想要的话可以和伴侣尝试其他不会引起疼痛的行为。性学家在这方面可以提供不错的建议与指导,如果妳有伴侣,带去诊疗会是不错的想法。同时在自己生殖器上要谨慎使用香水、香皂和乳霜类产品,因为在最坏情况可能会造成疼痛加剧。





阴道痉挛




提到阴唇痛症时,许多人会与阴道痉挛(vaginismus)相提并论⸺又是另一个困难、颇具争议的症状。阴道痉挛是女性骨盆底肌肉所围绕的阴道口不由自主地收缩或紧绷。这些女性通常拒绝阴道插入(无论是性行为或是妇科检查)因为会造成或预计使她们遭受或疼痛与不适。换言之,阴道痉挛会是一个兼具性行为、棉条使用及身体检查的复杂症状。

有些人认为阴道痉挛属于非自愿肌肉痉挛,使阴道变得非常狭窄。在挪威有时候会将阴道痉挛翻译为阴道抽蓄。使用仪器测量肌肉活动的研究发现,并没有明确的证据能够表示阴道痉挛的女性会有这类的"肌肉痉挛",而肌肉可能与阴道痉挛有关的主张也没有得到专业人士的认同。

前庭痛症和阴道痉挛的症状重叠。阴道痉挛经常被描述为与前庭痛症有相同或相似的疼痛。疼痛大多位于阴道口,因此与有子宫内膜异位症或因性病导致子宫颈而剧烈疼痛的女性不同。无论这两种症状是一体两面,或是两种单独却经常同时发生的疾病,都难以分辨。

阴道痉挛的疗法与阴唇痛症几乎相同。对于阴道痉挛的女性,往往会额外训练她们能够承受有东西在阴道里;通常从女性自行插入非常薄的物体,如扩张器开始,接著逐渐增大尺寸。镇痛凝胶持续在插入过程中使用,以致不会过于疼痛。这种疗法可以由妇科医生、性学家或物理治疗师共同进行。

阴道痉挛和阴唇痛症令人难以置信地限制患者的生活乐趣与性生活。对于许多女性来说,只要症状仍然存在,不可能会有正常的性生活,而与伴侣的关系会恶化或以失败收场。很多人担心她们是否永远不会有伴侣或孩子,或是她们不得不在剩下的人生里独自生活。她们觉得自己无法克服。这些症状可能导致的痛苦,以及许多女性在医疗体系上可能遭遇的伤害,我们的认知并不多。一个令人感到小小的安慰及更进一步的资讯是,大多数女性都会有所改善,甚至许多人会变得完全健康⸺即使阴唇痛症往往是持续时间长的慢性疾病。





衣原体、淋病及它的远房亲戚




当然,我们是实境节目《乐园饭店》的大粉丝,甚至因为其中一位男性参赛者声称他光看女生就知道对方有没有性病,因此他从来不使用保险套时放声大笑。我们不知道他受到什么样的祝福。或许他在霍格华兹学院得到证书或和其中一位电视灵媒有所关联呢?不过,有一件事可以肯定:没有人可以光看女性(或男性)就能分辨他们是否有性病。很多人甚至不知道他们已经感染了⸺这就是问题的核心。人们即使有了性病仍然不使用保险套进行性行为。妳不知道的疾病就此蔓延。

我们一般称性交传染的疾病,为性传染疾病或简称为性病。当妳与其他已经感染的人从事性行为或性接触时就有可能感染性病。性病是由不同类型的微生物,如细菌、病毒和寄生虫所引起。有些性病只能通过体液如血液和精子传递。其他疾病则可以透过皮肤和黏膜之间的接触来传递。

某些性病极为常见,有的在这些章节里则是比较罕见。妳在人生中有可能得到一种或多种性病,这是做爱的缺点之一。

由于性欲一直与羞耻和罪恶感有关(特别是女性),性病也是如此。即便现在,很少人能够开诚布公谈论自己得到尖锐湿疣(genital warts)与衣原体感染(chlamydia)等疾病。虽然这些情况甚是常见,有时很难防范,很多人都有了应该要减少非正式性行为的次数避免让自己的伴侣曝露于感染风险的想法。我们希望性病相关的知识和规范可以消除一些难受的羞耻感。感染是不当使用保险套最重要的问题,接著才是运气好坏。这并不是妳的"个人性道德"问题。有些人和好几百人上床不使用保险套,却像奇迹一样没有得到感染,而其他人可能一夜情一次就得到尖锐湿疣。妳的性生活同样也会发生鸟事。

在我们有现代药物与抗生素前,有些性病比丢脸还要更糟糕。还会造成严重的症状,在最坏的情况下,则是死亡。有很长一段时间,淋病(gonorrhoea)经常造成儿童在出生时受到母亲感染而失明。这种情形普遍到在挪威所有新生儿出生时都要滴入治疗淋病的眼药水。易卜生(Ibsen)于一八八一年的戏剧《群鬼》(Ghosts)中,将梅毒(syphilis)当作为一个单独的角色可能就是一种诠释。注入艺术灵魂的梅毒,在剧中最终攻击儿子奥斯瓦尔德(Osvald)的大脑与中枢神经系统。现在,我们可以用青霉素(penicillin)消除梅毒,使感染的人重返身体健康。在《群鬼》发表时的一八八一年是不可能做到的,许多人遭受像奥斯瓦尔德的窘况并死于这种疾病。

尽管医学进步,性病仍然是全球公共健康的主要障碍。自一九八〇年起,当爱滋病夺走数千名年轻男同性恋者的性命,但新闻上却甚少提及病情。有充分的理由,爱滋病,或称后天免疫缺乏症候群(Acquired Immune Deficiency Syndrome,AIDS),是导致免疫防御(意即人体抵御细菌、病毒和其他垃圾的保护罩)崩溃的疾病。

爱滋病病毒的微生物与人类免疫缺陷病毒(Human Immunodeficiency Virus,HIV)有关。在二〇一五年,有110万人死于与HIV相关的疾病,现在超过3670万人与病毒共存。自疫情开始蔓延,有3500万人失去了生命。一旦妳感染了HIV,就没有办法摆脱。在挪威,HIV呈阳性的人接受良好治疗后现在几乎可以正常生活。有了彻底的治疗,它们将不再具有传染性。因此有药物能够控制病毒,但遗憾的是,世界上受到感染的人只有一半有机会获得这些药物。

现在的挪威,无论是梅毒或是HIV已经不再蔓延,但疾病仍确实存在。二〇一五年,有221人被诊断为HIV,其中189人在二〇一四年诊断出梅毒。与其他性病相比数字是令人难以置信的小,但也几乎可以认定为流行性疾病。妳可以透过血液检查诊断自己是否得到HIV和梅毒,但因为罕见,除非妳特别曝露在受到感染的风险下,不然没有必要定期检查。

挪威最常见的细菌性疾病称为衣原体。二〇一四年有292‚772次衣原体测试,其中有248‚117个呈现阳性⸺也就是多达8%!阳性结果代表发现了衣原体。在十五至十九岁与二十至四十岁的组别,阳性测试的数量比所有其他年龄组别还高。十五至十九岁的女生呈阳性的比率为13.6%,而男生则是16.1%。二十至四十岁族群中,呈现阳性的女性为10.6%,男性为16.3%。因为很多人没有进行测试,所以我们必须假设未统计资料的数量极多。

呈现阳性的大多为女性:不会少于60%。这并不表示女性得到更多衣原体,只是比较容易被检测到而已。妳可以从十五至十九岁和二十至四十岁的组别得知,拥有衣原体的男生比例,在做实际测试时更高。这意味著更多的男生比女生更容易带有衣原体,只是我们不知道罢了。

看起来有些男孩好像拿上女孩服用测试的护理的机会,并承担他们会打个电话,如果过去的合作伙伴测试阳性。这个策略不太上道,更不用说离无懈可击还差得很远。即使妳的性伴侣检验结果呈现阴性,妳可能也会有衣原体。每次性接触的感染风险并非100\%,因此伴侣双方应该要去做检查。换句话说,乐园饭店里不用保险套的参赛者和认同他的一大群人必须要改变他们的习惯。即使妳自己有做检验,当妳和新对象发生性关系时,使用保险套始终是明智的举动。无法确保妳的伴侣会和妳一样聪明。此外,我们偶尔会忘记使用保险套而导致木已成舟的遗憾。如果妳忘记使用避孕套,那么做检验很重要。

有点像衣原体的两种细菌性疾病分别为支原体(mycoplasma)和淋病。保险套能有效地保护妳免于受到上述三种疾病的感染。淋病在挪威比衣原体感染更为罕见。二〇一四年,有682人被诊断出淋病。其中只有119位是异性恋女性,但是感染人数却愈来愈多。

支原体是经常被家医科忽视的疾病。它有点像是衣原体的胞弟。它们非常相似,具有相同的症状,可能也有相同的后遗症⸺我们会回过头来提到。然而,除非患者有症状,不然不用定期做支原体检查。即使如此,许多医生做检验时也不会想到它。霉浆菌感染的治疗方式与衣原体不同,所以找出疾病很重要。如果妳有症状,但衣原体检验结果呈现阴性的话,要求进行霉浆菌测试是明智的作法。

衣原体、霉浆菌及淋病最常见的症状为分泌物改变或增加、排尿时感到刺痛、生殖器不适或生殖器、尿道及肛门有骚痒感,取决于感染的位置而定。这三种细菌性疾病经常攻击子宫颈,造成红肿。同时使性交不愉快或疼痛,由于压迫疼痛的子宫颈,有些人可能在性交后或途中会发现稍微出血的现象。一般情况下,在我们不知道原因时,应该对任何的阴道出血保持警觉⸺尤其是与性行为有关的出血。例如,月经或使用荷尔蒙避孕可能是出血的原因之一,但是不明原因的出血可能是性病或其他疾病引起,所以应该由医生进行检查。

然而,并不是每个人都有症状。事实上,只有一半,最少三分之一的女性得到衣原体的症状。霉浆菌的症状也不常见,也有人感染淋病却没有相关症状,所以我们为什么要替根本不会注意到的东西操心呢?嗯,首先,细菌性疾病极具传染性。无防护性行为感染衣原体的风险是20%。其次是长期损害的危险。

如果细菌有机会,它们可以自行经由子宫颈,最后停留在子宫及输卵管。它们可以在那里引起发炎。这称为骨盆腔炎疾病,妳可以从衣原体、霉浆菌和淋病感染得到17 。根据统计,未经治疗的衣原体会引起10\%〜15\%的人发展为急性骨盆腔炎。危险的是,发炎后可能导致输卵管留疤,造成阻塞。这是女性难以怀孕的常见原因,而除了这一点,它也会引发慢性疼痛。

如果妳得到骨盆腔炎,很常感到恶心与不适,而妳经常会下腹剧痛、阴道出血、发烧和分泌物增加。通常疼痛不会减轻或好转,只会更严重。这些类型的症状应该要认真看待并尽快找医生检查,如果有需要则必须进行紧急手术。

虽然不常见,不过的确有可能得到没有症状的骨盆腔炎,只会在进行不孕治疗后几年发现。

所以,就算有其他原因,更换性伴侣后记得去做定期检查。

衣原体、霉浆菌及淋病可以用抗生素治疗。现在,受到感染的人大多能完全恢复健康,不会有长期的损害,但是抗生素的抗药性,特别是对霉浆菌及淋病逐渐上升令人感到担忧。抗生素的抗药性是指细菌对某些类型的抗生素免疫,因此需要更强的药物来摆脱它们。换句话说,避免感染的最好办法就是在一开始使用保险套。

有些性病是甚至比衣原体更为常见:疱疹和HPV,这两者都是病毒型疾病。HPV意即人类乳突病毒(Human Papillomavirus),它有许多不同的形式。某些类型引起尖锐湿疣。其他的可以导致子宫颈癌。疱疹与感冒疮相同,而且是会在皮肤上形成小水泡的疾病。

疱疹和HPV是透过皮肤和黏膜之间的接触传播。我们不知道到底有多少人感染了这些疾病,但两者蔓延的情况都非常严重,人们往往不会注意到自己已经受到感染。

因为没有任何症状,许多人被不知道自己已经感染的伴侣传染,造成感染难以防止。同时也不能肯定保险套就能提供足够的保护。举例来说,如果阴茎根部有尖形湿疣或疱疹的男性,即使使用保险套仍然会传染给他的伴侣。保险套无法完全覆盖感染区域。

妳可以接种疫苗抵抗HPV,有些疫苗能够给予病毒引起的尖锐湿疣和子宫颈癌的防护。如果妳得到尖锐湿疣,可以用冷冻疗法(用液氮冷冻)

进行治疗,或用不同的药物擦拭让它们消失。换句话说,和治疗妳在游泳池淋浴时所得到的疣非常类似。尖锐湿疣没有危险,也不会造成罹癌的风险。疣和癌症分别属于不同类型的HPV病毒。

HPV感染往往会自行消失。疣也一样,但是有些疣会反复生长造成困扰。

疱疹,同时是妳无法摆脱的病毒。一旦遭到感染,病毒会处于一种休眠状态永远留在妳的神经细胞。妳可能好几次爆发,可以透过处方药物降低次数。然而,疱疹并不危险而且问题往往随著时间的推移不断减少。





我该如何保护自己不受性传播疾病传染?




保险套提供HIV、衣原体、霉浆菌及淋病良好的保护。然而,HPV和疱疹能够透过皮肤接触传播,所以妳可能会从没有保险套覆盖的地方受到感染。

帮女性口交时,可以使用口交膜片⸺一款轻薄、透明的乳胶片,可以放在阴户上面。这将会防止疱疹从嘴部到生殖器或是从生殖器到嘴部感染。口交膜片并非特别实用(而且不太能固定住),所以没有被广泛使用。如果妳想仍然使用它们,妳可以剪断保险套顶端,从圆筒裁切并摊开,妳就能得到一张大又透明的方形薄膜。





我什么时候该去做检验?




即使妳没有任何症状,每次与新伴侣进行无防护性行为后进行衣原体检验是聪明的想法。在关系初期尽早和妳的伴侣一起检查也是一个好主意。既然妳可以在察觉不到异状,而得到性病很长一段时间,那妳实际上可能也有衣原体而不自知。如果妳没有症状,通常足以在妳的医生或年轻人常去的诊所透过尿液或使用棉花棒对阴道或肛门采样来进行自筛。

如果妳有过无防护肛交,却没有进行肛门检查的话,很难确定是否受到感染。这样的话,妳应该进行肛门检查。

如果妳有症状,妳可能还需要进行生殖器检查,由妳的医生决定。重要的是,如果妳排尿有刺痛、有瘙痒、分泌物变化、起疹子、水泡、不寻常的出血或是妳发现有任何其他的东西就要联系妳的医生。

除非患者有特别的症状,不然不用进行疱疹或HPV病毒检查。

要注意,只有在可能暴露于感染威胁后的两个礼拜进行衣原体测试才是有效的。这意味著如果发生性行为后两周或以上进行测试,妳可能会得到阴性反应。妳当然可以在初期自行测试。很多人在早在两周前就测到阳性反应,这是好事,因为这意味著她们可以即早治疗。如果妳在两周后得到阳性结果,妳可以肯定妳得到衣原体感染。但是,如果早期测试的结果是阴性,至少在妳可能暴露于感染威胁后的两个礼拜进行新测试前妳都无法得到完全肯定的结果。

两周法则也同样适用于霉浆菌及淋病检查。





度假性交的风险与危险




现在,我们已经讨论一长串的性病,不过主要集中在衣原体的检查。那么其他的疾病呢?有些女性去医院要求医生进行"全部"的检查,但是真的没有必要每次让自己接受所有的测试。妳应该要做的检查必须和妳的医生共同决定,而这将取决于妳有可能感染到的性病。

如果妳是生活在挪威的异性恋女性,只和居住在挪威的异性恋男性发生性行为,比起在泰国度假与性工作者进行无防护性行为得到HIV、梅毒与淋病的风险相对较低,不言自明。在挪威的异性恋者当中,衣原体绝对是最常见的性病,而且通常能够被诊断出来。

但是,如果妳在国外度假有无防护的性行为,必须要告诉妳的医生。医生经常会忘记询问,所以不要指望他们主动开口。即使妳只有在满月派对上与可爱的瑞典男孩发生关系,还是去做个测试吧。妳根本不知道在他们背包假期当中除了妳还和什么人发生性行为。如果妳和从性病很严重的国家旅行回来的人进行性行为也一样得去检查。为了以正视听,不只是前往泰国才需要留意;许多挪威人在,德国、西班牙和波兰得到感染,这些国家的性病发病率和我们非常不同。

如果妳有卖淫或买春,妳要采取的检查范围必须更广。这也同样适用于注射药物或与注射药物的人有过性行为。

在挪威,男性与男性间的性行为(men who have sex with men,MSM)族群得到更严重性病的风险最高18 。

MSM比异性恋人口更常得到淋病和梅毒感染。 让这些男性进行检查也就格外重要。必须记住这也适用于与MSM进行性行为的女性。如果妳最后一夜情的男性也和男性做爱,那么得到性病的风险比只与其他女性做爱还高。针对MSM并非羞辱⸺这只是单纯的统计问题。无论妳和女性、男性或与男性发生性行为的男性做爱,妳的运气好坏参半。

自己去做较不常见的疾病检查不会有任何害处,但如果风险不是特别高,妳就不需要每次都去做。尽可能地经常检查,根据风险进行检查,并且使用保险套。





* * *



17	业界不认同霉浆菌会导致骨盆腔发炎。该领域的研究仍然稀少,但有少数小型研究认为的确属实。毕竟小心驶得万年船。



18	比起"同性恋"我们倾向使用"男性与男性之间性行为"。男性有可能在尚未确认同性恋的情况下与男性性交。性取向和妳发生关系的对象为何不同。





疱疹⸺妳的性生活完蛋了吗?




微小、疼痛的水泡出现在妳的嘴唇或生殖器上听起来一点也不好玩,但疱疹(herpes)比妳想像得更常见。疱疹具有传染性,既麻烦又难以抵抗,但幸运的是它是无害的。即便如此,疱疹似乎很多人最害怕的性病。

许多人害怕无法摆脱疱疹的困扰。一旦妳被感染,病毒会在一直留在妳身体里。因此造成了很多疑问。举例来说,这代表妳一直都有传染性,所以永远不能和任何人进行无套性行为了吗?

疱疹突然在关系中出现也造成很多的不信任与不确定。是谁感染了谁?在一起三年的伴侣对妳不忠吗?

关于疱疹有很多、非常多误解。无论对被感染或害怕感染的人来说,因为疱疹而感到焦虑是正常的。

疱疹是一种影响皮肤和黏膜的病毒性疾病。两种略微不同的病毒可能就是罪魁祸首:第一型

单纯疱疹病毒(HSV-1)和第二型单纯疱疹病毒(HSV-2)。疱疹病毒是透过与皮肤或黏膜如接吻或性行为接触进行传播,它也可以间接传递。典型的例子则是幼稚园孩子彼此吸吮相同的塑料恐龙后得到感染。超过一半的人在幼儿时期嘴巴可能感染过HSV-1病毒。

没有任何的纪录所以我们不知道到底有多少人感染了疱疹。但是这一次可以几乎正确的说,每个人都有得过,不像那次妳试图说服爸爸说别人都有GameBoy,妳也必须有一个了。据悉,多达70%的人感染了HSV-1病毒,而40%的人得到HSV-2。妳可能感染这两种或其中一种病毒。最重要的是,大部分人口都可能感染过疱疹,被谁感染并非主要的问题。

只要停下来稍微想想这些数字。毕竟,这代表感染比没有感染频率更高而已。即便如此,很多人认为疱疹就是世界末日。但是超过70\%的挪威人生活还没毁灭且还能做爱呢!





等等,口腔疱疹和生殖器疱疹是两种不同的疾病,不是吗?那么,为什么我们谈论它们,好像是同件事情一样。性传播感染和感冒疮完全不同,不是吗⋯⋯?

事实上,无论疱疹长在妳身体哪里都是一样。以前,人们认为HSV-1病毒主要与口腔疱疹有关,而HSV-2病毒则是和生殖器疱疹有所关联。假设这是妳所感染的部位,HSV-1病毒很容易在生殖器上爆发,同时HSV-2病毒也很容易长满嘴巴。妳也可能在肛门周围、手指或(如果妳真的不走运)在妳的眼睛上长出疱疹。也就是说,HSV-1比HSV-2在生殖器上所展现的症状较少也比较温和。

所以生殖器疱疹也是感冒疮,而口腔疱疹可能透过性传播感染造成。从生殖器传染到嘴唇是可能发生的事情,而且从嘴巴传染的方式更为常见。现在大多得到生殖器疱疹的年轻女性都是经由嘴部感染HSV-1病毒的伴侣口交导致,比率多达80%。

既然有这么多人得到疱疹却不自知,这代表在实际上许多年轻女性被不知道自己已经有疱疹的伴侣传染。所以妳该如何保护自己呢?

一旦妳被感染,病毒会在几天内爆发,但也有可能在没注意到任何情况下遭受感染。妳感染后,疱疹病毒群会从神经向上移动至造成感染的皮肤区域。它们会定居在妳身体略为深处的神经细胞内睡觉,就像进入冬眠的熊一样。它们在妳的余生都会继续待在那边。有时候,病毒透过神经向下移动后又重新回到妳的皮肤上。接著造成新的爆发,和上次一样在同一个地方形成水泡。同时也可能有隐藏的爆发⸺可能是皮肤上妳没注意到的病毒。一只看不见的熊从冬眠中醒来。

有形疱疹发作的不适感会以刺痛的形式开始,在妳的生殖器或嘴唇皮肤引起烧灼感。然后出现长成一团的小水泡,几个几个堆在一起。过了几天,水泡干掉结痂,最终脱落。

第一次爆发通常是最严重的。通常称为初次疱疹爆发,会对某些人造成非常严重的状况。因为生殖器的急剧疼痛,可能会有发烧或排尿问题。如同每件事一样,如果妳得到严重的症状却不知道出了什么问题,妳就应该去看医生。初次疱疹的爆发比起其他疾病持续的时间较长。妳可能会在一到两周内得到新的水泡,结痂会在三、四周后完全消失。如果妳有一个戏剧性的初次爆发,认为下一次的爆发不会那么糟糕也许能让妳舒坦一点⸺事实上,妳不会有下一次爆发。许多人初次得到之后就再也没有过了。

如果妳有新的爆发,总是会发生在妳第一次感染的同个地方。爆发的数量通常会随年减少。没有药物能够摆脱疱疹,如果妳很迫切,药锭可以有镇静和缩短爆发的作用。每年特别麻烦的情况都与许多爆发有关,妳可以长期使用药物治疗来抑制症状。

新的爆发往往在妳免疫力低的时候出现。这就是为什么口腔疱疹的常见名称叫作感冒疮。妳经常会在生病,例如感冒的时候得到。压力、月经或阳光也可能触动疱疹爆发,刺激皮肤⸺例如内裤摩擦、蜜蜡除毛或剃毛也会造成。

疱疹和HPV病毒一样没有疫苗,但实际上也不需要有。疱疹就像是对抗自身的疫苗:如果妳在孩童时期已经感染了一次,妳之后就不会被相同的病毒感染到身体的其他部位。病毒触发妳的免疫系统,这样它们就会识别相同的病毒,并防止病毒定居在新的神经细胞。这代表每个病毒只会感染一次。如果妳的嘴巴遭到感染,那么妳的生殖器就得到防护,反之亦然。

但是,就如妳目前所知,疱疹病毒有两种。如果妳之前已经感染了HSV-1,妳不会有HSV-2的防护。理论上,如果是两种不同的病毒,妳会在两个地方得到疱疹。然而,必须要说的是,妳会有一定程度的交叉保护。如果妳感染了第二型病毒,妳的症状通常会较轻或没有症状。

由于疱疹的运作有点像疫苗,所以妳不会自己感染自己。如果妳有生殖器疱疹,病毒就无法移动到身体的其他地方。但是要注意!只有在免疫系统受到触发时才有用。然而,妳的免疫系统需要一点时间来适应疱疹,所以妳实际上可能在其中一种疱疹病毒初次爆发时让自己受到感染。所以妳第一次爆发时要特别注意洗手和卫生。当妳病毒在妳的手指上时不要揉眼睛。不要这样做!

即使在初次爆发后妳不会让自己受到感染,却还是可以传染给别人。我们最常遇到有关疱疹的问题是:我什么时候具有传染力?很自然地,有生殖器疱疹的人会害怕传染给别人⸺那妳又该怎么知道妳是安全的呢?药物治疗可以防止感染吗?还是有不应该进行性行为的特殊时间点呢?

皮肤或黏膜必须要有病毒存在才能透过皮肤和黏膜接触造成的传递。由于疱疹往往深藏于体内的神经细胞冬眠,所以妳通常不会具有传染性。病毒必须从神经向上移动到皮肤,这样妳才能够感染其他人。这是妳爆发时会发生的情况。爆发前一周与爆发当下是妳传染性最强的时候,因为病毒都聚集在皮肤上。水泡里充满了病毒。在妳感觉到疱疹即将爆发时,避免性行为可能是明智的作法⸺通常发生在水泡出现的前几天。不过当然,在前一周有可能也难以确定这次的爆发是否严重。

同时还有一些隐藏的爆发。病毒可以在妳完全没注意到的时候漫布到皮肤上,也不会出现任何的水泡。即便如此,妳还是具有传染力。实际上这代表妳不具有一般的传染性,但妳可以在任何时候有传染性。妳永远不能肯定妳没有传染性。没有安全的时候。现在,妳可能会想:但这是完全的危机啊!实在很难确定妳不会感染别人,这也是受到感染最困难层面。但是再想一想。

比方说妳生殖器得到HSV-1病毒却想要和新对象发生性关系。妳的潜在伴侣已经感染病毒的机率有70\%,因此会对新感染造成防护而不自知,这样就降低了极大的风险。此外,如果妳伴侣的嘴巴长了感冒疮,妳几乎可以肯定妳不会传染给对方,因为口腔疱疹通常是由HSV-1病毒引起。如果妳的伴侣已经感染了HSV-1病毒,他们对妳带来的新感染形成了防护。

另一种分辨方式则是大多数人无论早晚都会感染。如果妳没有感染他们,别人以后也会感染他们。疱疹是无害的,而大多数人却很难注意到自己受到病毒感染。

总之,我们需要讨论一下和疱疹有关的一个棘手问题:疱疹对关系的影响。比方说,妳和妳的伴侣以前没有得过疱疹。妳的嘴,妳的生殖器上也都没有。妳们已经在一起三年,并拥有一个梦幻般的关系。然后疱疹就此发生。妳的生殖器上爆发严重的水泡,而妳有了最糟糕的念头。妳没有和其他人发生关系,所以一定是妳的伴侣,难道不是吗?

妳现在知道了,妳也不一定知道自己有疱疹。妳受到感染时不一定会爆发水泡。妳得到疱疹可能有很长一段时间却没有可见的水泡爆发。当然也很有可能是妳被妳无形水泡爆发的伴侣的感染。换句话说,不必考虑不忠的事了!疱疹是常见的,而妳也不一定知道妳有。我们已经看到疱疹爆发后许多没有根据的不忠指控毁了好多关系。

当然,不忠可能已经介入,但疱疹不能证明这一点。如果妳没有任何理由怀疑妳的伴侣,那么疱疹不应该是撒下不信任种子的因素。

背负不传染性病给伴侣的责任实在很伟大。如果我们讲到的是衣原体,我们会大声鼓掌,但是如果是疱疹,往往只会让我们感到难过。没有必要让人因为疱疹而害怕做爱。疱疹不是HIV病毒,即使它们都是妳无法摆脱的病毒。疱疹完全无害。感染上生殖器疱疹并非世界末日。

妳是许多人里的其中一人。没错,事实上,是多数人之一。疱疹非常有可能在妳生活中造成极少的麻烦。如果妳确实遇到问题,它们是有机会补救的。如果妳是有很多爆发的几个不幸的人之一,还有治疗方法可以采取。





剧烈瘙痒和腐烂的鱼腥味⸺妳肯定会遇到的生殖部位问题




妳的双腿之间有东西即将发生。它是红色,气味特殊,或者它非常痒让妳无法在晚上睡觉。酵母菌感染和细菌性阴道炎是非经由性传播感染所引起的常见生殖器官疾病。大多数女性在生活中会得到其中一种,或两者都有。这两个症状是无害的,但会是令人难以置信的麻烦。因为妳可能会遇到这些生殖器疾病,所以对它们了解更多可能相当值得。

微生物,如细菌和真菌通常触发负向关联以及他们非常向往肥皂及厨房喷雾。有谁没听过细菌在抹布繁殖的速度有多快,或者看过真菌是如何在潮湿的地下室散播出去?这足以让妳颤抖⸺但不是所有的微生物都是有害的。

有些细菌对我们身体运作不可或缺,例如帮助我们消化的肠道细菌。事实上,我们身体里的细菌数大约是其他细胞的十倍,这并不代表我们生病了。

阴户与阴道的黏膜由所谓的生殖器正常菌群构成的微生物所覆盖。正常菌群借由支持免疫系统对抗外来微生物有助于保持妳的阴道健康与维持阴道内环境的平衡。妳可能还记得,阴道有自净功能,如果使用肥皂,特别是冲洗剂会消除生殖器的天然保护。

阴道正常菌群会根据妳所在的生命阶段而变。在妳进入青春期和更年期后,正常菌群主要由皮肤和肠道细菌组成,但是当妳受孕时,妳的身体被雌激素所影响。这使得黏膜变厚、活跃,而正常菌群对生殖器来说也就变得独一无二。它不同于身体其他部位的正常菌群。

育龄女性的正常菌群包括不同类型,依赖雌激素的营养来生存的乳酸菌与乳酸杆菌。乳酸杆菌产生的酸就像优格一样。乳酸确保阴道维持在大约4.5的低pH值,因此创造了一个对坏细菌类型不友善的环境,因为它在酸性环境下会感到不适。此外,里面还有一些其他的细菌:些微的酵母菌和病毒。所有的微生物都在争夺相同的食物与住处,由于有这么多不同的类型,它们没有一个能占上风。加上身体的免疫系统,不同的微生物会相互制约。当具有保护力的正常菌群失去平衡,生殖部位变得脆弱且很容易造成问题。





阴道酵母菌感染




我们先从酵母菌感染开始。大约20\%的女性有一种被称为白色念珠菌(candida albicans)的酵母菌,为阴道正常菌群的一部分。许多人的屁股有这种酵母菌,尤其增生机会很大的话,它可能会转移到阴道。多达50%的怀孕女性阴道里有酵母菌存在。可能是因为白色念珠菌喜欢雌激素,而怀孕时身体充满了雌激素。白色念珠菌是挪威酵母菌感染最主要的起因。

等等,酵母⸺妳是说我们加进卷饼和面包的东西吗?差不多是了!和妳在超市看到的酵母是不完全一样的类型,但却相似。其实,二〇一五年十一月有一名酵母菌感染的女性曾使用她阴道的酵母制造老面,使她成了网路大红人。她的作法是用人工假阴茎收集她的分泌物。她将发酵后的老面烤成面包,并吃了下去。她说味道"真是该死的有够好吃"。

如果妳是阴道内一直充满酵母菌的20\%其中之一,并不代表妳有酵母菌感染;只有当酵母菌引起黏膜发炎时才算。换句话说,当妳得到感染时,妳完全会知道。

酵母菌感染,亦称作鹅口疮,能够影响阴道内部与小阴唇。搔痒感可能十分强烈,而一些女性下体会感到刺痛或灼热。一样的,这也会使性交疼痛,或在妳排尿时引起阴户刺痛。受感染的黏膜变得有些红肿。有些女性同时也有乳白色块状分泌物,像是低脂干酪一样,而其他人则是液态分泌物。

有些女性发现自己得到阴道酵母菌感染时,男性伴侣的阴茎也会有如起疹或瘙痒感的症状。然而,我们必须强调酵母菌感染并非是一种性传播感染。即使得到感染或正在接受治疗,妳仍然可以发生性关系。男性的症状通常不需要个别处理。妳有能力可以摆脱感染,而他的症状也会消失。

由于酵母感染或鹅口疮非常普遍,因此药店专柜上会贩售治疗药品。治疗的药物有许多不同的类型,每种都同样有效。包括药膏和阴道药剂,或是口服抗真菌药锭。如果妳使用阴道药剂,妳应该在妳上床睡觉前放入,让药效可以在晚上运作。如果没有,药丸可能会溶解并迅速流到妳的内裤。如果妳使用的是药膏,妳需要从小阴唇开始涂抹薄薄一层,一路经过阴蒂最后擦到肛门。在月经来潮时避免使用阴道药剂会是好主意,血液可能会将药物带离阴道⸺也可以说是将它冲出来。

当女性有类似鹅口疮症状时这些非处方治疗有效降低了她们诊断和治疗自己的门槛。

问题来了,并非所有的搔痒就是鹅口疮!如果下面搔痒,只有50\%的可能性为鹅口疮。不同的生殖器疾病可能会彼此类似。因此,我们强烈建议有新症状的女性应该去拜访医生。瘙痒和分泌物的变化可能是任何疾病引起的模糊的症状,例如像衣原体、淋病等性感染疾病,这些问题值得尽早分辨。不同类型的湿疹和生殖器的刺激症状也很常见,可能是因为内裤残留的洗涤剂,或者使用香芬皂或湿纸巾。

即使女性得过鹅口疮,她们仍不善于区分鹅口疮和其他生殖器疾病。女性正确诊断出鹅口疮的机率只有三分之一。如果在这些情况下,她们选择非处方治疗,而不是去看医生,会导致了很多没有意义、不正确的治疗,无助于消除疾病。非必要使用抗真菌治疗也可以延迟实际疾病的发现,造成额外症状发生。事实上,广泛使用抗真菌药物会刺激黏膜,让人联想到酵母感染。换句话说,出门看医生确认是不是鹅口疮一点也不愚蠢⸺至少在第一时间妳发现问题,或者妳发现症状不断反复出现。

当妳被诊断患有鹅口疮并使用抗真菌的药物时,按照妳的医生或药剂师建议的方式很重要。即使症状已经消失,妳仍必须完成疗程。症状消失后继续使用药膏至少两天,否则鹅口疮可能很容易再长回来。如果妳太早完成治疗,可能导致少量鹅口疮累积,当妳停止服用后感染就会再次爆发。

酵母菌感染是常见的,我们知道有四分之三的女性在生命中会遇到这个问题。是什么样的原因造成呢?实际上很难有确切答案。我们知道的很多。我们容易得到酵母菌感染。我们知道使用抗生素后,许多女性会得到鹅口疮,或是因为她们清洗自己的生殖器过于频繁。毕竟,肥皂和抗生素有助于减少维持我们的生殖器健康的正常菌群。我们也知道,雌激素和酵母菌感染有关。青春期前和更年期后的女性很少有鹅口疮的问题,因为生殖器不会受到性荷尔蒙的影响,而孕妇可能经常感到困扰。我们知道鹅口疮通常会出现在月经周期的某个时间点。不像细菌性阴道炎,女性最容易在月经前得到,我们很快会提到这点。

糖尿病患者特别容易得到鹅口疮,尤其是血糖控制不良的人。我们也看到女孩变得性积极后更容易得到鹅口疮,而一个月多次性行为的人也是如此。

有些长期受到酵母菌感染的女性症状永远不会消失,对这些女性可能会造成一大阻碍。有3\%〜5\%的女性一年会得到4次以上的酵母菌感染。如果妳很容易感染,必须要告诉妳的医生,因为妳可能需进行适当的检查和比非处方产品药效更强的抗真菌治疗。

不幸的是,没有有效的方法可以防范鹅口疮。然而,无论是网路上民间偏方或找医生动手术都非常盛行。其中一项常见建议则是吃优格补充妳阴道的乳酸菌,无论是药丸或喝大量的乳酸饮料,如Actimel。然而,这种治疗尚未证实有效,所以这也许是在浪费金钱,除非妳真的非常喜欢Actimel。

除此之外,鹅口疮喜欢潮湿、温暖的环境,因此人们一般被要求保持自己的生殖器部位干燥。这代表妳应该避免穿著合成纤维的内裤和紧身长裤,在绝对必要时只能使用护垫。因为透气极佳所以建议穿上棉质内裤,而裸睡能够让妳的生殖器得以喘息。没有一种方法有科学效用的记载,但如果妳真有鹅口疮困扰,都值得一试。毕竟,妳可以自由选择,而且也没有副作用。





细菌性阴道炎




现在,我们将移动到另一个也是极其常见的生殖器疾病:细菌性阴道炎,或简称BV(Bacterial vaginosis,BV)。妳听过有人用和鱼相关的词⸺虾子嘉年华或墨西哥鱼卷,来描述女性生殖器吗?事实上健康的生殖器不应该有腥味,但BV就是这种味道的罪魁祸首。

BV是由生殖器正常菌群不平衡所引起。有保护力的乳酸菌减少,而环境中引起麻烦的其他类型细菌却蓬勃发展。乳酸菌维持妳阴道环境呈现酸性,而酸性对阴道来说是好的物质。当妳得到BV时,妳阴道的酸性略为降低,变得偏向碱性。这就是为什么当妳有生殖器问题,酸碱值是医师用来判断妳是否得到BV感染的项目之一。

没有任何一种类型的细菌与BV有关:它就像是不同种类参杂而成的鸡尾酒。有些细菌通常以正常菌群生活在阴道或身体的其他部位。问题在于它们已经转移,或是数量太多。

大多数专家认为,只有发生性行为的女性才会得到BV,而且风险随著性伴侣数增加,以及减少使用保险套的次数上升。女性之间或与男性发生性行为也一样。伴侣越多,得到BV的机率也就越高。所以妳可能会觉得有些细菌是来自妳的性伴侣,但是,这不代表BV是一种性传染疾病。请记住,BV是由许多不同的细菌所引起。它不是一种如衣原体具有传染性与杀伤力的细菌疾病。将它看作妳的正常菌群与好几个拥有与妳稍微不同细菌组合的人混合在一起。厨师太多烧坏一锅汤,在这种情况下,则是破坏了平衡。

没有多位性伴侣的女性也会得到BV,但她们一定有过性行为。BV被认为是无害的,所以没有理由在妳治疗期间使用保险套或放弃性行为来保护伴侣免于受到感染。即使妳有多位伴侣,使用保险套永远是明智的选择,这是因为性传播感染的风险,而非BV的风险。

除了特殊气味被描述为腐烂的鱼之外,得到BV的女性分泌物比平常还大量。许多人将分泌物形容为灰色、极为液态,每天需要经常更换自己的内裤数次。气味会如此强烈到光从衣物就能闻到。

很多女性在阴道性交或生理期后会产生鱼腥味或是味道更加恶化。这是否代表月经和性行为让妳得到BV呢?不是这样的,如果妳有BV,月经和精子只会加重症状。

事实上气味变得更加强烈是因为妳的生殖器偏向碱性的环境。这表示如果妳阴道的乳酸菌较少或碱性物质增加,妳的阴道会变得更糟。血液和精液在阴道环境中会更偏向碱性,因此会增加腥味。如果妳在月经或性交后闻到鱼腥味,可能代表妳的阴道环境只是不再偏向酸性,而BV症状也没有那么严重。

也许听起来很容易辨认,但得到鹅口疮后,妳不一定可以借由症状来判断是否得到BV。得到BV的女性经常会有发痒或其他可能让她们联想到鹅口疮的症状。分泌物是不同性传播感染的常见症状,并记住一次可能会同时得到好几项症状!分辨生殖器疾病的不同一直都很困难。由此得到启发,如果妳的生殖器不正常的话,妳必须去看医生做检查。分泌物、瘙痒或刺痛感让妳很不舒服吗?去看医生吧。

虽然很多人在闻到气味时会这么认为,但细菌性阴道炎并不代表妳的生殖器很脏。如果借由清洗来摆脱问题,妳只会洗去维持阴道酸性环境的良好细菌,让事情变得更糟。BV可以自行消失,但进行治疗会更好。因为BV是由所谓的细菌、抗生素或抗菌治疗引起。含有乳酸菌的阴道胶囊对BV也同样有用,并维持阴道的环境。不幸的是,没有研究能证明这种疗法有任何影响效果。





排尿疼痛




尿道感染被称为排尿铁丝网并非巧合。尿道感染的感觉很糟,而身为一名女性,妳特别容易得到它们。

我们的尿道偏短难辞其咎。事实上,我们的肛门距离我们尿道口很近。如果细菌留在属于自己的地方,那么它们会在我们的屁股上发挥最好的效果,但是实在很难限制细菌的行动。它们很容易爬进尿道口,经过尿道,最后定居在尿道和膀胱的黏膜。一到那里,它们会引起发炎。

妳会因为尿尿疼而发现尿道感染。会有蜇伤、烧伤感,而且感觉好像从铁丝网爬出来一样。尤其在排尿的最后,膀胱本身排空和肌肉壁互相挤压时特别疼痛。此外,妳会发现有频繁的排尿冲动,但一次只能排出一点点。妳可能会注意到妳的尿液气味奇怪,或者带有一点血。

尿道感染的绝大多数年轻女性(多达95\%)是我们所说的无并发症族群。代表感染被认为较不危险,只需简单或根本不用治疗。早期,因为人们认为感染会透过泌尿系统爬上肾脏,引起肾盂炎(pyelitis),所有的尿道感染使用抗生素治疗,但结果它却是不必要的方式。大多数尿道感染在没有使用抗生素的情况下,几天后会自行消失,并在必要时服用一些止痛药。

当然,妳应该总是警惕任何的恶化现象。如果妳有发烧或更严重的疼痛,特别是疼痛往妳的背部移动,妳必须尽快去看医生,如果有需要则采取紧急手术。这可能是细菌引起的肾盂炎,在最坏的情况下可能导致肾脏受损。

如果妳怀孕的话必须认真看待尿道感染。在这种情况下会被认为是复杂的状况,同时妳需要特殊的抗生素治疗。如果妳有频繁的尿道感染也会被认为是麻烦的情况。通常需要更仔细调查其中出现什么样的细菌,有时会检查是否有让妳容易感染的潜在症状。话虽如此,不知道为什么,有些女性仍一次又一次得到尿道感染。根据推测,这些女性尿道黏膜的免疫系统可能有点不同,让细菌更容易站稳脚步。

许多女性设法避免得到尿道感染。蔓越莓果汁或药丸已经是使用好几百年的常见民间偏方。蔓越莓包含防止细菌依附在膀胱黏膜的物质。然而,著名的考科蓝资料库(Cochrane Library)的一个主要研究显示,蔓越莓病没有保护作用。但是,再次强调:如果妳喜欢蔓越梅果汁,没有什么能够阻止妳尝试。蔓越莓汁没有副作用。其他建议则是饮用大量的水冲洗尿道,并在需要排尿时尽快清空妳的膀胱,在排便后当然保持从前面往后面擦拭的习惯。

我们所知道的是,性会增加感染尿道感染的机率。在性交过程中大量水分往往在生殖器中累积,细菌变得更容易移动到另一个地方,在同一时间所有生殖器之间的摩擦和推力可以把细菌带进错误的洞口。我们知道三十岁以下的族群性交后头两天得到尿道感染的风险比平常还多出六十倍。

妳可能听说过的一个时下建议则是,如果做爱后排尿妳可以降低排尿刺痛的机会。这是很棒的建议。做爱后排尿,妳会冲洗掉自行向上进入尿道的肠道细菌,摆脱它们设法侵入妳的黏膜因而导致麻烦。

尽管性可能参与其中,一般的尿道感染并不是性传播感染的一种⸺只是正常臀部细菌出现在错误的地方而已。但是衣原体、淋病和霉浆菌也是排便时刺痛的常见原因。所以,妳应该提高警觉。然而,这些细菌的表现略有不同。性病的细菌会在尿道口衍生,不像臀部细菌位于膀胱。当妳有性传播感染时,妳排尿的最后不会有特别的疼痛。也不会有常见又频繁排尿冲动。即便如此,自己也不太容易发现其中的差别。尿道感染和衣原体相似,而衣原体也可以像尿道感染一样。如果妳真的不走运,妳可能会一次同时获得这两种感染。





滴滴滴⸺关于漏尿




当妳十九岁半也没有子女,在商店购买大量的添宁护妳垫时一点也不有趣,然而老太太和拥有一堆孩子的女性并非遭受漏尿所苦的唯一族群。漏尿的专门术语为尿失禁(urinary incontinence),是女性中普遍存在的问题。

老化、生产,连同高BMI,是漏尿的最高危险因子,这表示有愈来愈多的女性开始随著岁月流逝遭受这些困扰。这可能也是许多人认为生产前漏尿是不正常的原因,但是所有年龄的女性都可能会受到影响。

很难得知有多少女性确实有漏尿的困扰。每项研究的数据各不相同,但都认为只有不到一半的尿失禁女性会去医院求诊,其中也可能有未统计资料的数字。挪威的一项研究发现,30%的女性有漏尿的现象,而另一项针对产后三个月女性的研究则发现20%〜30%的人受到影响。国外的一些研究显示比率从10%〜60%不等,取决于漏尿的严重程度。

我们对年轻、没有孩子的女性与确实存在的极大数字差距所知甚少。一项针对十六至三十岁没有子女的澳洲女性的研究结果发现多达12.6%的女性经历过漏尿。而瑞典的研究却有完全不同的结果:二十至二十九岁的女性约有3%拥有漏尿的情况。

无论哪项研究的最接近事实,我们可以有把握地说,漏尿在年轻、没有子女的女性中很少见。尿失禁有好几种方式,我们区分为所谓应力性尿失禁(stress incontinence)、急迫性尿失禁(urge incontinence)与结合前面两种的混合型尿失禁。

应力性尿失禁是最常见的,大约影响50\%的漏尿女性。应力性尿失禁是引起上腹部压力增加的漏尿现象,例如,咳嗽或打喷嚏、大笑、跳跃、奔跑或其他相似行为。与急迫性尿失禁相比,涉及的范畴虽小,但严重程度有很大差别。漏尿的频率与漏尿量之间的差异可能有所不同。

急迫性尿失禁和需求有关。有这种尿失禁形式的女性会有突然、强烈的感觉需要马上"排尿",接著伴随大漏尿的现象。拥有尿失禁的女性有10\%〜15\%会是这种形式。急迫性尿失禁的女性通常具有膀胱过动症(overactive bladder),这代表它们通常具有强烈的排尿欲望,却不一定会漏尿。有膀胱过动症的女性通常比其他女性更常在半夜起床排尿。

介于35\%〜50\%的女性有混合型尿失禁的模式,也就是同时具有压力性和急迫性尿失禁。换句话说,漏尿的形式会有所变化。有时候,妳跳跃或打喷嚏时就会漏尿,或在其他时候,妳会有强大的排尿冲动和大量漏尿。

漏尿可以由许多原因引起。如果妳喝的水比妳所需的更多,那么减少饮水量可能会是一个好主意。很多人认为喝大量的水本身是健康的,但除非妳运动量很大或住在非常炎热的气候,妳不需要在每24小时饮用超过2公升的水。妳可以从食物获得一些水分。通常没有必要每24小时喝超过1.5〜2到两公升的水。减少饮用利尿的饮料,例如咖啡和茶,也会是个好主意。

有时漏尿可能是其他疾病的症状。有些女性在尿道感染和一些神经系统疾病时可能会引起漏尿。因此,如果妳找不到任何明确的原因,例如,妳生完孩子或者突然开始每天饮用5公升的水后开始漏尿,与妳的医生讨论可能是明智选择。妳的医生可以给妳指导和帮助妳找到解决方式。

妳会漏尿并不代表妳的下半身注定要穿黑色衣物来尽可能隐藏漏尿,或者在剩下的人生中放弃奔跑与欢笑。幸运的是,妳可以有所作为。试著停止漏尿的第一件事需要一点积极性。很多应力性尿失禁的人会有这种漏尿情况是因为她们的骨盆底肌肉太弱⸺举例而言,她们可能在出生后受到影响。骨盆底肌肉是妳排尿时用来阻挡尿流或是紧缩阴道。如果妳的骨盆底肌肉较为强壮的话在妳的腹部压力增加时更可以容易防止无意识的漏尿。有几种方法可以训练妳的骨盆底肌肉,但主要与间隔收缩妳的阴部肌肉有关,和妳在健身房锻炼身体其他肌肉一样。许多女性从她们的家庭医生或物理治疗师得到帮助。妳可以遵循一些特殊的运动项目,包括专为训练骨盆底肌肉设计的应用程式。妳也可以尝试阴道球或类似的工具。阴道球的重点在于借由使用骨盆底肌肉,来维持它的位置。不管妳如何训练,妳会希望看到自己渐渐变得更强壮并减少漏尿的现象。

骨盆底锻炼可能对受到急迫性尿失禁影响的女性有一定的影响,但膀胱训练的过程为更重要。对于那些有急迫性尿失禁的人来说,问题并不在于肌肉。膀胱肌肉在妳没有任何控制的情况下于错误的时间收缩。这就是为什么人们有急迫性尿失禁时会经常大量排尿。膀胱训练是训练自己不经常排尿。问题的关键是按照时间计划表,而不是根据需求排尿。举例来说,妳可以从限制排尿开始,每隔一小时都这么做。如果在不能排尿的时间里有突然的冲动,妳必须忍住不能去厕所。过了一阵子,妳可以逐渐增加允许排尿的时间至两小时、三小时、四小时等。随著时间的推移,这将往往对急迫性尿失禁有帮助。

在某些情况下药物治疗或手术可用于治疗尿失禁。一些女性只要在门诊进行简单的治疗就能有不一样的世界,而对于其他人来说运动会有效果。帮助妳最多的事情取决于自己想要的是什么以及漏尿的问题有多严重。





痔疮及肛门皮肤赘瘤




如果妳稍微看一眼妳的屁股,妳很快就会发现它有好多皱折。人们常常把它叫做气球节有一个原因。皱纹是由夹紧洞口的括约肌构成。它必须能够扩张得非常大,而它的大直径被有点像是百褶裙的结构给隐藏住。一般情况下,皱折均匀围绕洞口,形成一个相对平坦的表面。因此它可以在妳突然发现屁股外面挂著新的和外来的东西时使妳心惊胆颤。妳感觉新的突起物好像在尖叫引起注意,而大家的目光往妳的肛门看去,很多女性试著完全忘记这回事。这种感觉可能是肛门皮肤赘瘤(anal skin tag)或是痔疮(haemorrhoid)造成,而两种都是无害的症状。

痔疮对男女性来说皆是极为常见的疾病。虽然没有足够的理由能够当作一般晚餐话题,然而实际上大约有三分之一的成年人得到痔疮。痔疮同时长在直肠或直肠外的肛门周围是可能发生的;不过我们谈论肛门外部的痔疮就好。无论如何,痔疮就只是痔疮。

痔疮是由于肛门静脉曲张,在外观上呈现气球形状的紫蓝色突起物。不像肛门皮肤赘瘤,妳几乎能再次把它推回原位,但随后会在妳排便或是做了特别费力的深蹲时再次弹出。痔疮经常会有搔痒感也可能一触即痛。有时唯一的问题可能是当妳擦后面的时妳会发现卫生纸上有鲜血。造成的原因单纯是痔疮的血管错位。通常围绕直肠口的血管由结缔组织和黏膜的支撑,所以我们看不到它们的存在。随著年龄的增长,这些支撑结构变得松弛而骨盆受到的压力增加(例如如厕用力、提重物、怀孕和生产)可能引起一小部分的血管被推出来,就像一个打结的浇花水管一样。这个扭结的根部很容易承受压力,引起血液累积并形成一颗小气球。这气球就是我们所说的痔疮。

围绕在屁股的痔疮不危险,但却也能造成真正的麻烦。血管不喜欢被这种方式对待,因此轻微发炎就很容易发生在痔疮周遭。接著妳会发现有点黏液,或者感到疼痛、发痒,让单纯的坐下(更不用说排便)变成烦人的一件事。有些人也会发现自己有出血的现象无论流量很少或相当多。

幸运的是,解决方法很简单。最重要的事,其实很平凡,就是确保自己拥有良好的如厕习惯。喝足够的水保持妳粪便柔软,当妳感到厕急时去上洗手间,还有避免紧张。我们还会建议妳把报纸留在厨房的桌子上。如果妳坐在厕所很长一段时间,围绕痔疮的压力会增加,这会让问题更严重。良好的如厕习惯往往会让所有痔疮自己回到原位。用手指将弹出的痔疮推回原位也是好方法,可能会有机会找回了正确的位置。用妳的手指戳妳的屁股可能会让人觉得有点怪,如果这么说能带来安慰,医生每星期都要对完全的陌生人这么做。

妳也可以在药店买到各种效果良好的痔疮软膏。如果没有效的话,妳的医生也可以提供很多不错的治疗方案来帮助妳,包括手术在内。正如妳现在可能明白,妳的医生对这些事早就习惯了!

如果突出妳屁股的不是痔疮,那么有可能是肛门皮肤赘瘤。它是存在于肛门的一个略大的皮肤皱折,通常是由破掉的痔疮所产生。当痔疮自行压迫时,可能会在肛门环造成一些皮肤皱折从原本的地方脱落。后来,当痔疮消退时,它们会结合而形成稍大的皱折从表面略微突起。虽然妳可能因为皮肤褶皱受到丁字裤摩擦、频繁排泄或类似情况而刺激,造成短暂的瘙痒和分泌物,但一个或两个肛门皮肤赘瘤很少会引发大问题。有些人甚至会发现难保持屁股干净。

然而,有些人觉得肛门皮肤赘瘤有碍观瞻。赘瘤可以借由手术移除,不过妳应该在选择手术前谨慎思考,因为并发症的风险总是存在。同时也值得注意术后疼痛。妳会在妳的屁股中间得到一条疤痕,而不幸的是因为妳刚做完手术所以会憋不住粪便。除非它们造成的许多问题,不然我们的建议是放轻松并和肛门皮肤赘瘤和平共处。





子宫颈癌与如何避免




子宫颈是子宫与阴道之间的通道。妳能感觉到妳的子宫颈在妳阴道的最上部,像是鼻尖的塞子一样,中间有个小小的洞。这条狭窄道路是精子细胞到达子宫所经过的通道。妳的月经从这里离开,当妳生产时妳的子宫颈可以扩张到足以让整个婴儿通过。也正是这里可以让妳得到子宫颈癌。

子宫颈癌在癌症当中格外特别。早在十九世纪,人们发现这种类型的癌症在每个人身上的表现有所不同。妓女比已婚女性更常得到,而修女传播这种疾病的机率很低。难道这是对放荡女性的一种神圣的惩罚吗?

现在我们知道,上帝的惩罚和这个疾病没有关系。子宫颈癌单纯是借由性行为传播所造成的疾病!我们先前提过这个和性传播感染有关的病毒:人类乳突病毒(HPV)。

HPV是一种大家族的病毒,其中几个病毒会造成尖锐湿疣。它们大多数都是无害的,例如普通的皮肤疣是由一种类型所引起。某些HPV类型会在生殖器茁壮成长。它们是经由性接触传播,而且大多数性活跃的人会在生命中感染其中一种。到了五十岁,有过病毒感染的人超过80%。HPV被认为是最常见的性病,过半数二十到二十四岁的人在任何时间带著感染四处游走。

无须担心HVP,不像疱疹感染,妳的身体通常会自行摆脱病毒,运作的方式和感冒相同。我们知道这是因为确诊HVP女性的病毒会随著时间变换病毒类型。这表示感染是短暂的,而女性在更换性伴侣时会重新感染新病毒的类型。

然而,某些类型的HPV与其他病毒的不同之处在于,它们可能造成一些人的子宫颈长期感染。这些类型就是所谓的高危险病毒,最常见的类型是HPV16和18。如果妳运气不好的话,感染随著时间可能会发展成癌症。16号就超过子宫颈癌的病例一半,也可能会导致口腔和咽喉癌,以及阴道、阴户和肛门癌。然而,不只会得到一种感染。感染HPV16很常见,但只会有极少数人得到癌症。这表示其他因素是决定癌症发展的关键,例如,容易受到感染的人,或是像是有吸烟习惯等环境因素。这些其他因素是什么,我们也还不知道。

稍微有点不同,几乎所有子宫颈癌的患者会得到HPV病毒造成的感染,但很少有人因为感染而罹患癌症。





从性到癌症的漫长之路




幸运的是癌症不会在一夜之间发展。首先,病毒会造成妳子宫颈细胞变化,在专业术语上称作发育不良(dysplasia)。小缺陷和异常细胞会阻止正常细胞运作。这些患病细胞在一开始的时候都只是略有不同,但如果免疫系统让它们远离和平,它们才可能真正开始脱颖而出。细胞直到完全无法辨认前会随著时间出现愈来愈多改变,并开始在不应该存在的地方生长。只有在这时才会成为癌细胞。

在大多数情况下从无害变化细胞到全面的子宫颈癌至少需要十至十五年的时间。在这些期间,它们会经过不同阶段的变化。在每一个阶段,细胞可能改变想法或被免疫系统破坏。

这些类型的细胞之变化,可能是癌前的病变阶段,最好能尽早发现。借由每三年定期扫描和细胞检查,可以即时捕捉变化并在它们构成任何威胁之前去除。这是有效对抗子宫颈癌防御方式。

当妳生病时细胞改变和子宫颈癌在末期前很少会有症状或征兆。这就是为什么子宫颈的定期检查非常重要。子宫颈癌的症状可包括出血异常,如月经周期之间或与性行为有关联的出血。有些女性的生殖器或下腹在性行为间或日常生活中会出现疼痛。其他人可能会发现分泌物开始发臭并带有血迹。

换句话说,子宫颈癌的症状非常不明确:它们存在于许多生殖器常见和危害较小的疾病当中。如果妳有这些症状,妳绝对要去找妳的医生进行检查,但不需要担心会是癌症。最有可能的原因则是性病、避孕的副作用或性交时导致的疼痛;最重要的还是要去做检查。





做检查




妳很快就要二十五岁了吗?那么妳就会收到癌症登记中心的邀请,进行子宫颈抹片检查19 。如果有收到,妳真的应该去做检查,就是这样。进行定期抹片检查的女性一生中罹患子宫颈癌的风险可以减少70%。这就是我们所说的便宜得难以置信的人寿保险!

尽管二十五至三十四岁之间的挪威女性中,几乎有一半的人选择把信丢到垃圾桶,进行筛检的比例已经从71\%下降到57\%。即使比以往更加暴露在子宫颈癌的风险,年轻人仍是不做检查的族群之一。这样造成了不良的后果,在挪威有更多的年轻女性比以前更容易得到子宫颈癌。根据癌症登记中心记载四十岁以下女性的癌症病例数在近几年上升了30%。其原因为在更多年轻女性受到致癌的HPV感染时,仍很少有女性转而定期进行子宫颈检查。

因此抹片检查是预防子宫颈癌的简单解决方案。挪威女性在二十五岁时会纳入检查名单,接著会建议她们直到六十九岁应该要每三年做新的抹片检查。妳在检查后届满三年时会收到预约新检查的提醒通知。

抹片检查需要与家庭医生预约。即使还没有收到邀请妳也能去做。在妇产科医生那边也可以做抹片检查,在这种情况下,妳通常会需要得到医生的转诊许可。

妳不应该在月经期间进行抹片检查,同时也最好不要在检查前两天进行阴道性交。进行妇科检查只需要几分钟的时间。医生会以一种漏斗状的扩张器撑开妳的阴道,窥视妳的子宫颈,并以小刷子进行采样。刷子会靠在子宫颈上轻轻摩擦并松动一些细胞,接著可以在实验室的显微镜下检查。如果子宫颈细胞显示变化,妳会在几周内从妳的医生那边获得通知。如果一切正常,妳一般不会收到任何消息。





细胞变化并不代表妳有癌症




抹片检查后,妳可能会从妳的医生那边得到讨厌、难以理解的检查报告信函:妳的细胞不正常⸺但这到底是什么意思呢 ?

我们所遇到的女性当中最常重复的话题是她们对医生传达子宫颈周遭细胞变化的过程不充分而感到沮丧和焦虑。大多发现有异常细胞的年轻女性都感到自己非常健康,从来没有想过自己会得到癌症。所以检查报告带来的冲击可能比医疗专业人员知道的还更多。

许多女性在被告知发现细胞变化后会认为自己得到癌症,且有可能死去而感到害怕。我们要对这些女性强调,年轻与性活跃的女性子宫颈细胞发生些微变化非常常见。任何HPV感染,即使是低风险的病毒,也可能会导致变化。这就是为什么二十五岁以下的女性不用检查⸺令人难以置信的数字将会成为不必要的焦虑,最终可能会在没有增进我们发现新癌症案例的能力下受到过度治疗。

在绝大多数情况下,子宫颈细胞变化在没有任何形式的治疗下会自行消失。像其他的病毒一样,往往自己消失。人体自身的免疫系统会自己整理一切,实在是太棒了!妳的医生知道这一点,因此这可以解释为什么在妳想到的结果全是"癌症"时她会看起来一派从容。

让妳得到多一点安慰:每年有两万五千名挪威女性在抹片检查发现细胞异常,只有三千人需要接受严重癌前治疗。后来发展成子宫颈癌的甚至更少,大约三百人而已。

但是,让我们来看看从妳医生得到的检查结果信函。妳在抹片检查后发生了什么事?从妳的子宫颈刷下的细胞被送到实验室。在那里,医生将细胞染色并将它们放在显微镜下,寻找出现异常的细胞。根据细胞外观不寻常的程度和数量多寡,将细胞变化以轻度、中度和重度做分类。即使是严重的细胞变化也可以自行消失,但追踪所有细胞的变化仍然很重要。

观察细胞之外,实验室会将切片拿去进行HPV检查。当涉及下列发生的情况时,细胞变化的规模,与HPV检查的结果会是决定性的因素:





细胞切片呈现不确定或低度变化




妳只需要在六个月后回到医生那边做追踪检查。异常的细胞往往在病毒攻击后修复,或被妳的免疫系统杀害。如果细胞的变化已经消失而HPV呈现阴性,妳就跟以前一样健康,三年后妳才需要做另一次抹片检查。如果经过追踪检查,而HPV呈现阳性后妳的细胞仍然有变化,妳会被送到妇产科进行更深入的检查,如下面所述。





细胞切片呈现高度或严重变化




妳的家庭医生会替妳转诊至妇科医生,她将进行两件事情。首先,当妳坐在诊疗椅上时,她会先用特殊放大镜工具来看看妳的子宫颈。这种检查称为阴道镜检查(colposcopy),是为了寻找黏膜的变化。接著医生将会从妳的子宫颈采集组织样本(活体组织切片,biopsy)让专家(病理学家)透过显微镜进行检查。在细胞检查中只有少数细胞会从黏膜表面上被刷掉,但活体组织切片里会移除一小片皮肤去调查黏膜深处是否有异常细胞。整个黏膜架构都会进行检查。

通常妳的子宫颈会被施打局部麻醉,因此在活体组织切片结束后可能会感到疼痛。在检查前服用一些布洛芬可能是一个好主意。同时检查过程中流血也是正常情况,所以大多数女性在这几天需要使用卫生棉(不是卫生棉条!)。

当病理学家从显微镜检查组织切片时,这些变化将再次根据轻度、中度到重度变化分类。这些都不是癌症。只有在不正常细胞直接自行穿过黏膜才算是子宫颈癌。

如果阴道镜检查及组织切片显示完全正常或仅有轻微的变化,那么妳可以松口气。但是,妳必须在六至十二个月内去找妳的家庭医生进行新的抹片和HPV检查来确保一切正常。在十分之九的案例中,这些变化将会消失或在没有任何形式的恶化下维持稳定20 。

如果任何的检查确定有中度至重度的癌前病变阶段,妳将按照规定,会被送至医院接受小手术。

这个过程被称为锥形活体组织切除(cone biopsy)。通常会以线圈电刀或吊带切除部分子宫颈的外部。一开始医生会用刀进行锥形活体组织切除,而被切除的部分看上去像一个倒置的冰淇淋甜筒,这也解释了这个手术的名称。现在被去除的组织部分看起来更像一个扁平甜甜圈。

虽然有少数人可以进行全身麻醉,但锥形活体组织切除通常是在局部麻醉下进行。这是一个简单的过程,但除非有必要不然可以不用这样做。这是因为经历锥形活体组织切除的女性在往后怀孕被认为有较高的早产或流产风险。

绝大多数有过锥形活体组织切除的女性,约90\%会痊愈。为了可以100\%肯定自己没事,她们在六个月或在手术后十二至十八个月到家庭医科进行新的抹片和HPV检查来确保一切无事。

如果细胞的变化已经自行消失或借由锥形活体组织切除移除,就不必再担心子宫颈癌。就像蛇梯棋一样:妳会直接回到原点!

不过需要特别记住的是,妳可能会再次感染HPV病毒,所以接种HPV疫苗可能是明智的选择。我们待会再说。妳还必须继续在余生每三年进行抹片检查。然而总体来说,妳应该专注在放松身心。





对抗癌症的疫苗




现在,我们谈了很多关于妳应该如何与HPV感染和细胞变化共处,但是想像一下,妳是否可以事先预防致癌病毒的感染!事实上,是可以的,完全有可能。几年前可能会像科幻小说一样,但现在居然还有能预防癌症的疫苗。这是个医学奇迹。

正如我们前面所解释的,人类乳突病毒(HPV)有超过一百个不同类型,但能够致癌的只有其中几个。加卫苗(Gardasil)和保蓓(Cervarix)这两种HPV疫苗可以抵抗最危险的HPV16和18类型感染。在它们之间,这两个高危病毒引起70%的子宫颈癌。针对这些病毒的疫苗可以让妳几乎100%抵抗感染,以及细胞变化与这些种类病毒导致的子宫颈癌的防护。近期一种可以防止九种不同病毒类型的HPV疫苗受到核准上市。它可以防止90%的子宫颈癌,但挪威健保局尚未给付这支疫苗。

加卫苗也可以预防造成尖锐湿疣的HPV6和HPV11类型的感染。一些研究表明保蓓也能提供防止尖锐湿疣的部分防护。重要的是要明白尖锐湿疣与生殖器癌之间没有任何的关系,但是能够避免尖锐湿疣仍然是好事。如果没有疫苗,10%的挪威人口会得到尖锐湿疣,约10%的挪威女性将有严重的细胞变化,而1%的女性会在剩下的日子里接受子宫颈癌治疗。

自二〇〇九年以来,HPV疫苗已经变成挪威儿童接种疫苗计划的一部分。换句话说,提供所有十一至十二岁的女孩免费接种。此外,出生于一九九一年或以后未接受疫苗的女孩从二〇一六年十一月可以免费接种疫苗。

疫苗分为三剂并为期半年施打。

疫苗并非药物,但却能防止病毒在妳体内定居,避免妳在之后得到感染生病。疫苗刺激免疫系统辨识病毒,在病毒出现时准备最快速、最有效的粉碎病毒作战计划。如果妳已经得到HPV16或18的感染,疫苗不会消除妳身体中的病毒。这就是为什么要让年轻女孩施打疫苗。我们要在她们开始进行性行为或得到潜在病毒感染前保护她们。

疫苗被核准用于九至二十六岁的女性和男性,并已证明有效年龄范围达到四十五岁。其中有两个原因。首先,我们当中很少有人同时感染HPV16和18,如果妳还没有被这些类型的病毒感染,疫苗将有一定的保护作用。其次,如同我们提过,多数HPV感染会自行消失。不幸的是,对抗HPV的自然免疫已被发现效力薄弱。这代表即使妳较早得到HPV感染,后续不一定能从其他性伴侣再次感染的风险中得到防护。但HPV疫苗可以帮助妳抵抗这种再次感染。

现在男性已经不在疫苗接种计划当中,但挪威公共卫生协会(Norway’s Public Health Institute)则建议女孩和男孩均须接种HPV疫苗。疫苗对男人也同样有效(对尖锐湿疣、阴茎、肛门和咽喉、口腔的HPV相关肿瘤)对女性也是如此。有些人可能已经注意到,男性罹患咽喉和口腔癌的风险增加了。据猜测是因为口交变得愈来愈普遍。疫苗也可以预防这类的感染和癌症生成。尤其同性恋男性将会受惠于疫苗,因为他们不像女性有通常称作群体免疫(herd immunity)的直接疫苗保护。

不幸的是,出生于一九九一年前的男性与女性必须自费施打,但是妳应该考虑从口袋掏钱支付疫苗费用吗?

对于妳拥有的每个性伴侣,HPV感染的风险是10\%左右。即使妳已经感染了一种或多种类型,妳非常可能不会被HPV16或18感染。如果妳接种疫苗,妳能防止自己被未来新性伴侣感染。正如我们提到的,研究表示疫苗的有效年龄可以达到四十五岁。既然妳必须为此从自己口袋里掏钱出来支付费用超过三千克朗的疫苗,妳应该与得到感染的风险相互衡量。简单地说,这表示妳有过的性伴侣数量很重要。先前的伴侣数愈少,疫苗对妳有效的可能性也就更大。妳有过的性伴侣总数在将来也会有一定的作用。伴侣数愈多,感染的潜在风险就更大,疫苗的好处同样也就更多。此外,接受异常细胞治疗的女性因为施打过HPV疫苗,复发的风险也会很低。





安全有效的疫苗




现在挪威七年级的女孩有四分之一选择不接受HPV疫苗。我们不知道为什么人们要选择不注射疫苗,而害怕副作用似乎是普遍的现象。也有一些家长认为自己十二岁的女儿不会发生性行为,所以HPV疫苗是不必要的。在丹麦有许多媒体非常关注可能出现的副作用,而这也导致了接种女生的比例大幅度降低。同时最近挪威媒体上有一些关于疫苗恐怖故事。其中有一点原因在内。

在挪威有16万名女孩已获得近50万剂疫苗。其中有645件可能副作用的通报案例,92\%的人认为不是很严重。这只是短暂的症状,例如接种疫苗部位肿胀、发烧、恶心及腹泻 。

自二〇〇九年以来,在52个通报案例当中有严重副作用的有10例为慢性疲劳症候群(chronic fatigue syndrome,CFS/myalgic encephalomyelitis,ME),而姿势性直立心搏过速症候群(postural orthostatic tachycardia syndrome,POTS)则有5例。POTS是当妳站起来时会导致脉搏数升高的状况,以及血压不稳定、疲劳与头晕。根据挪威药品局报告,在这个年龄组中有或没有接种疫苗的病例数不高于人们所预期的状态。换句话说,疫苗不被认为是造成这些问题的因素。

然而这类关于可能严重副作用的回报一直被严重看待。在接种疫苗后,许多包括POTS的症状案例在丹麦频繁出现,欧洲药物管理局决定进行安全审查。调查的结果在二〇一五年十一月出炉,结论是,没有数据证明HPV疫苗与POTS或称为CRPS的另一个症候群(复杂区域疼痛症候群,complex regional pain syndrome)之间有因果关系。这些是罕见的情况,且接种疫苗的女性其发生率也不会高于其他人口。也没有发现疫苗和慢性疲劳症候群之间有任何的关联。

到目前为止世界各地约有180多万名女性接种HPV疫苗,施打后并没有发生严重的安全问题。使用药物和疫苗总是会有副作用的可能性,但是这些往往是轻度、暂时性的问题,然而生殖器癌却并非如此。





* * *



19	有时试验计划会在挪威的特定城市进行,会以HPV检查取代抹片检查。感染HPV 16或18的女性后来被招回进行抹片检查。未来最快几年内,这可能会成为挪威实施的筛检方式。这代表挪威女性可以免除不必要的妇科检验与抹片检查,只有异常细胞风险增加的人才要接受检查。



20	60\%的些微变化会自发性消失,而30\%会维持不变。只有10\%会发展为严重改变,而1%则是发展为癌症。





流产⸺从Facebook回到现实




二〇一五年夏天,Facebook创办人,马克·祖克柏向他的3300万好友更新一则极为特别的贴文。他和他的医生妻子宣布,他们正高兴地期待他们的第一个孩子,是个女孩,并为了她准备让世界变得更美好。无聊到要打哈欠了,妳可能会这么认为,接著自动按赞。这些各种各样的个人公告是Facebook赖以为生的东西,这个地方已经变成谦虚吹牛和制造各式各样形象的代名词。

然而祖克柏并没有就此罢手。他选择向他的追踪者介绍怀孕的崎岖历程,最后幸福的结局留下160万个赞⸺是一个我们平常不会提到的故事。这对夫妻经历三次流产与数年的努力,终于成为父母。四次的怀孕才得到一个孩子。

流产是怀孕二十周以前受精卵停止生长或胎儿在子宫中死亡的非自愿终止孕期现象。妳通常会因为疼痛及阴道出血才意识到妳流产了。也就是说,怀孕期间出血有没有什么不寻常的地方。虽然只有十分之一的机率会发生流产,但大约四分之一的孕妇在前三个月有出血的情况。即便如此,如果妳在怀孕期间出血,妳应该联系妳的医生并作检查。

流产是早期怀孕最常见的并发症。女性本身意识到怀孕的临床性怀孕(clinical pregnancy)流产的机率约为五分之一。同时也有在验孕测试前就发生流产的情况。这些类型的怀孕通常被称为化学性怀孕(chemical pregnancy)。将化学性怀孕纳入考量,会认定只有一半的受精卵能造成可行的怀孕。换句话说流产和成功怀孕一样普遍。

现在的验孕测试极为敏锐,能够在非常早期检测到妳怀孕了,但如果妳非常渴望阳性反应结果,使用这个方式不一定是非常明智选择。因为大多数的流产发生的时间从受精后前几个星期,到妳的下一个月经来临前。由于早期怀孕以流产结束是如此的普遍,妳可以省点让自己失望的时间,等到妳下次月经预计到来时再做验孕。如果多等个两周,直到怀孕的第六周,流产的风险已经下降到10\%〜15\%。因此,在这个时间点得到阳性的结果意味著妳可能会在八个月后当妈妈了。八周过后,风险下降到只剩3\%。一旦三个月的标准过去,风险稳定在0.6\%左右的低水平当中。每过一周,一切都会平安的机率将愈来愈高。

流产的恐惧是孕妇往往选择等到三个月过后才会告诉大家的原因。这个秘密背后的想法主要是避免孕妇出了差错。失去渴望已久的孩子已经够糟,根本不需要向周遭亲友打电话收回这个好消息。三个月的限制是否明确值得商榷,如果妳必须要有限制的话,妳也可以将门槛往前设置一个月前,大约是第八周。

不幸的是这种保密的结果会让很多夫妻对于流产感到可耻。经常听到人们对于流产后评论则是:"嗯,太快告诉大家好像有点奇怪",仿佛只因为公开就杀了妳肚子里的胎儿一样。这是相当荒谬的想法。祖克伯将流产描述为一种孤独的经历:"大多数人不讨论流产,因为妳担心妳的问题造成自己的距离或是让妳检讨自己⸺就好像妳有缺陷或者做了什么事才导致如此。所以,妳会一个人独自挣扎。"

祖克柏所描述的心情并非只有他一人感受到。发表在美国研究期刊《妇产科学杂志》(Obstetrics & Gynecology)当中,近一半流产的人表示,她们有遭受指责,或自己做错事的感觉。她们感到孤独与羞愧。读起来令人格外难过,尤其是因为流产的普遍误解所造成自责。在相同的美国研究里发现,近四分之一的人认为生活方式的选择,如吸烟、酗酒和毒品,是常见的流产原因。许多人还认为,繁重的工作和压力会导致流产。网络上的妈妈和宝宝论坛,将饮用咖啡和洗泡泡浴列为其他可能的原因。

在现实中,流产很少是母亲(或父亲)失误的结果。流产最常见的原因是胎儿的染色体严重异常,也就是已经确定受孕的遗传密码当中发生错误。在妳知道怀孕前,忘记酗酒、不健康的饮食或吸烟吧。

母亲和父亲的遗传物质只会为一个人,按照顺序合并成一本共同的食谱,复杂的让人难以想像。错误没有原因,不断发生也不足为奇。流产是身体的控制机制并以自己的方式确保我们拥有能够好好生活的健康孩子。若以这种形式流产可能会感到极具痛苦,但妳的身体实际上在做对的事情。

只有当妳已经连续发生两次或三次时,才应该考虑调查母亲(或父亲)身上是否有什么状况导致流产。在此之前都会被当作是一个很正常的现象发生。女性经历反复流产,问题从解剖畸变和贺尔蒙失调到自体免疫疾病和遗传性血液疾病都有可能。没有人可以指责这些疾病,但却有望可以治疗。

单纯运气不好是流产最常见的原因,但我们知道有几件事情会增加流产风险。最重要的因素是母亲的年龄,丹麦的一项研究发现,三十五至三十九岁的孕妇有25\%最后流产,相较之下,二十五至二十九岁的人则是12\%。四十岁当中只有一半的人能够顺利生产,因为卵子的质量开始变差,造成染色体和基因发生错误,使胎儿不能存活的情况更加频繁。

我们都知道怀孕不能吸烟。妳应该在知道怀孕后尽快戒烟。但是在妳发现之前该怎么办呢?在还不知道的时候,妳在一个聚会上抽烟了,又该如何是好?大型的研究调查发现吸烟与流产之间有明确的关联。如果100名非吸烟者和100名吸烟者怀孕后,非吸烟组会有20人流产,而吸烟组大约为26名21 。根据统计,每十件流产案例约有一件是由吸烟引起,但抽烟量看起来好像要很大(每天超过十支)才会提高明显的风险。所以在怀孕的前几周有少量的社交吸烟不会是巨大的内疚或焦虑的理由。

一定程度的酒精量也是一样。酒精对胎儿的危害极大,但我们不知道要达到多少才会造成伤害。孕妇可以在胎儿受到伤害或死亡前喝多少的量实在很难确认。要一组孕妇在怀孕期间喝酒,调查多少酒量能引起流产或胎儿的伤害当然很不人道。由于我们不知道极限,所以挪威卫生当局则建议完全避免酒精。这样一来妳就可以能够处于安全的状态。

然而,并不是每个人都同意戒酒是唯一正确的方式,在妳怀孕时甚至会造成混乱。妮娜在怀孕时发现这个情况,许多医生告诉她偶尔一杯红酒完全不会有问题。世界著名经济学家艾蜜莉‧奥斯特(Emily Oster)厌倦了这些综合讯息,并决定深入调查这项建议背后的研究经过。在她二〇一三年出版的书,《期待好孕:为什么传统怀孕的智慧是错的,妳真正需要的是什么?》(Expecting Better: Why the Conventional Pregnancy Wisdom Is Wrong and What You Really Need),她声称很少有人支持在孕期中绝对回避酒精的官方建议。她的研究表示,每周喝一到两单位的酒精,一周两天饮用一小杯葡萄酒或一杯啤酒完全安全。这不会对孩子的行为或智力产生长期影响。在她看来,要完全戒酒的官方建议是对女性不能克制自己的臆测而来:如果妳在生日中喝了一杯葡萄酒,可能会迅速喝掉一瓶。我们同意奥斯特的看法,但这种看法低估了女性的自律;毕竟我们大多数人都能在九个月的孕期内克制不饮酒。如果妳好奇或者怀疑的话妳可以阅读她的书籍,看看自己是否被说服了。

在妳验孕得到阳性反应时,一杯晚餐的红酒或许不会让妳担心。许多女性在发现怀孕时,因为在几周前参加一两场酒精派对,喝下超过一两杯酒而感到紧张。丹麦从二〇一二年开始的人口基础研究发现,如果女性在怀孕头三个月每周饮酒四次或以上,流产的风险会增加一倍。所以理论上,在发现妳怀孕前几周饮酒过度可能会导致流产,但绝不代表必然发生。如果确实如此,根本无法将原因归咎于大量饮酒。流产可能随时发生,只要知道它是多么常见就好!

而现在这些谣言在网络上比比皆是:搬重物、压力和正常的咖啡量摄取不会导致流产。看来在可能造成风险前妳每天可以喝到十杯咖啡。越野滑雪冠军玛丽特·比约根(Marit Bjørgen)在怀孕期间一天训练六小时,最后仍平安产下孩子。维他命或其他类似补充品看起来不能形成对流产的防护,但是在发现怀孕时妳应该开始补服用维生素B叶酸⸺而且最好是在尝试怀孕的时候就服用。它可以防止孩子的神经系统受到损害。

马克‧祖克伯一直是鼓励人们在社交媒体上分享流产经历的人之一。很多人认为这些经历太过私人和有失体面因此拒绝在公开场合谈论,但这名男子仍传达了一项重要的讯息。公开流产的消息是为要让大家清楚流产其实是很正常的现象,同时这项事实是会影响到所有的人。对于流产无须感到羞耻,它通常不是任何人的错。而流产带来的其中一项好消息是,绝大多数流产的人之后会有完全健康的孩子。

我们前面提到的三个月法则是为了保护女性免于承受告知别流产的痛苦,但这条法则实际上或许弊大于利。它延续了误解和歧视,而非让人有普遍的认知与理解。结果造成许多女性有更孤立、羞耻和内疚不合理感,在这个时候她们最需要来自周遭的温暖和体贴。因此,请大家告诉大家!





* * *



21	在怀孕期间流产的相对风险吸烟者比非吸烟者多了1.32\%。这个例子让我们推论出非吸烟者流产的机率为20\%。比例可能太高,但这是以大众可以理解为目的选择用相对风险来展示。





滴答作响的时钟⸺妳还能推迟怀孕多久?




当妳接近三十岁,奇怪的是就连素昧平生的陌生人都觉得自己有权干涉妳的私人生活。"时钟在转动,亲爱的!妳是不是该开始考虑生孩子的时间了?"不管妳是单身,处于新的关系或者和工作结婚,完全与他们无关。他们更想看到妳抛下妳所做的一切,强迫妳和能够倚靠的男性迅速生儿育女。

是的,想想生孩子这件事。很多女性想像了,也同时想像没有孩子的画面。即使妳想生孩子(这绝对不是既定的想法)仍然存在大量的潜在障碍。最明显的是找到一个妳可以想像能够与妳生孩子,也准备好与妳生孩子的人。奇怪的是许多男性在听到酒吧里的可爱的女孩开始谈论婴儿车,在喝第二轮酒时眼里浮现稳定下来的念头后纷纷落荒而逃。

不幸的是,我们不能帮妳找到完美的父亲,我们能做的就是给妳一点东西拿去对那些不断催妳怀孕的好事者反击,或者在妳开始感到压力的话给妳一点安慰。因为三十岁往往被当作是一个神奇的门槛,不过事实并非如此。

让我们从几个事实开始说起。大约有75\%的情侣在交往六个月后开始尝试怀孕。这一年结束以前,介于85\%〜90\%的人得以怀孕。一年定期从事无防护性行为后没有怀孕的现象会被定义为不孕症(Infertility)。这适用于大约10\%〜15\%的夫妇,但这并不是结束。被贴上不育的夫妇中,有一半会在第二年尝试怀孕的过程中自然怀孕。他们事实上应该被称为低生育率(sub-fertile)。他们很难怀孕却尝试非常长的时间,最后却得以实现。所以高达95\%的异性恋者会耗费很长一段时间借由进行定期性行为来怀孕。

再来就是年龄的问题。随著女性进入劳动力市场,生下第一胎的平均年龄逐步上升。二〇一四年奥斯陆女性生下第一胎的平均年龄为三〇‧八岁。女性因为长期求学并希望能够建立自己的职涯,所以比以往想要等更久一点的时间再来怀孕。同时,医学界透过数据强调生育率急剧下降,并向我们发出警告,希望我们在推迟怀孕前再三思考。其中有几个很好的理由⸺怀孕的并发症和畸形儿童的风险会随著母亲变老上升,我们会回来提到。现在的问题是,我们是否过于夸大年龄届满三十会有怀孕困难的风险呢?

一些最近的研究调查健康女性和她们怀孕的可能性。虽然很少有女性在上了年纪后会主动怀孕,但这些数字可能比妳想像的还不意外。一项研究追踪了782对夫尝试怀孕的夫妇。十九至二十六岁的女性显然是最具生育力的年龄组(有92\%在一年内怀孕了)之后,这个趋势有所下降。二十岁末和三十岁初期女性的之间的生育力并没有太大的差异。二十七到三十四岁的女性有86\%在一年内怀孕。同时相较之下,三十五到三十九岁的女性为82\%。其他研究也发现了类似的数字。在三千名丹麦女性的研究里,三十五至四十岁在一年内怀孕的女性有72\%,而按照排卵期性交而怀孕的人有78\%。而三十至四十岁的比率则是87\%。

我们从这里得到什么结果?如果所有从高中毕业的女孩试图怀孕,十个当中会有一个失败。二十年后,这个数字会上升到十分之二或三。然而,好处是,大部分的女性在三十多岁时还能够怀孕!如果我们必须谈到年龄的限制,三十五岁会更接近事实。

对于大多数难以怀孕的人来说,年龄不是直接原因。首先,我们要指出的是,问题出在男性的情况为三分之一,这是因为男性的年龄也有重要的影响。女性是主要的问题,或是问题的一部分,才是剩下的三分之二。到底发生什么问题呢?不孕的最大因素是控制排卵的贺尔蒙失调。经常会归咎于改变原有贺尔蒙平衡的多囊卵巢综合症。下一个最常见的原因是输卵管的损坏。可能是由于如衣原体等性传播感染,细菌引起输卵管发炎和留下伤痕。这些问题也可能由子宫内膜生长在错误地方的子宫内膜异位症导致。最后,肌瘤(也就是子宫肌瘤)能够阻饶怀孕。这些都是怀孕最常见的问题,而不是年龄。

然而,年龄的增加的问题导致流产的风险更高。正如我们前面提到,超过三十五岁女性流产的风险会高出一倍。这意味著那些期盼怀孕的女性比年轻时期怀孕的女性经历流产的次数更加频繁。

年龄对妳怀孕的机会、流产的风险、怀孕的并发症和染色体错误如唐氏症增加有明显的负面影响。但大多数的女性想要在三十岁拥有"旧派思想下的"健康孩子不会有任何的问题。妳有可能是难以怀孕的其中一名女性,当然不可能根据统计数据来确定,但是即使妳在二十八岁仍一直尝试却无法怀孕就有可能属于难以怀孕的族群。如果妳怀疑妳有子宫内膜异位症或多囊卵巢综合症,或者如果妳有过几次衣原体感染,不要推迟怀孕太久可能是明智的选择。妳可能还需要一点额外的帮助与时间来成功怀孕。





生殖器割礼




每年都有数以百万的女孩遭到切割。她们的生殖器被针头切割、缝合或刺穿。生殖器官切割是存在于世界好几个角落的文化习俗,幸运的是它愈来愈少常见。现在最常发生在非洲、中东和亚洲某些国家的部分地区,但西方以前也有施行过生殖器官切割。从十九世纪中叶,美国和英国的许多妇科医生切掉女性阴蒂来治疗自慰,因为自慰,当然有可能导致歇斯底里、癫痫和低智商。切割女性的生殖器一直是持续控制女性的残酷方式。

在挪威,在防止具有移民背景的女童遭到切割上付出许多努力,这些心力似乎得到成效。但是对于很多女性而言伤害已经造成。

世界卫生组织将生殖器切割分为四类。第一类为去掉全部或部分阴蒂,或是阴蒂头,而阴蒂通常会被切除。同时阴蒂如果不切除可能会长成像阴茎一样的东西的错误认知存在已久,但是真正无法摆脱的事实是,切除或损坏阴蒂,妳会消除女性性快感的主要来源,这是在企图控制我们性欲。由于阴蒂主要位于皮肤表层,有些女性会保有得到高潮的感觉与能力22 。其他女性也会发现,阴蒂所产生的疤痕组织会产生持续疼痛。

切割生殖器官的第二种形式与切掉小阴唇有关,通常结合各种形式的阴蒂损伤。小阴唇在我们进入青春期时并随著青少年时期的性觉醒增长。也许人们看到了生殖器成长与对性好奇之间的连接。借由去除阴唇,她们能保持幼稚纯真的幻想。

女性生殖器切割的第三种形式经常得到最多的关注,因为它是生殖器最具侵略性的改变。在这种形式下会将外阴唇大略缝合在一起,因此,上方剩下的开口就是通往阴道的小洞口。有时候小阴唇和阴蒂甚至会被切除。而尿液与经血同时从这个人工洞口流出。我们拜访的一位挪威裔索马利族女性,告诉我们第一次在挪威公厕排尿得到的震撼⸺挪威女性排尿就像大象一样!她以前花费长达二十分钟的时间排空她的膀胱,所以她的尿流是稀疏的。月经期间可能也会出现同样的问题,经血可能会在阴道内积聚。这使得阴道变成细菌的温床,让女性暴露在生殖器和尿道感染。

人工打造的洞口对性交而言过小,因此成为女性婚前没有经历阴道性交的一种保证。在她因为初次发生性关系而不得不用剪刀或刀子,或由男性的阴茎打开,问题当然就此发生。有些女性有一个大到足够进行插入式性行为的孔,但生产时仍必须要切开。阴道周围的疤痕组织无法扩大让一名婴儿经过,如果没有切开,她们会有遭受无法控制的撕裂伤,造成大量出血和损坏直肠的风险。

生殖器切割的最后一种形式则包括上述三种形式未提及的生殖器损伤。例如包括将热针插进阴蒂⸺一种女性性欲的杀害仪式。

所有形式的生殖器切割可能会造成生殖器长期的问题。此外,这个过程本身是造成感染和出血的主要危险,更不用说心理创伤。世界的大部分地区严格禁止生殖器切割有几个很好的理由。在挪威,即使女孩或女性仍想要这么做,导致生殖器长期损害的所有形式女性生殖器切割将会受到惩罚,包括允许自己的孩子在海外完成手术。

然而,在生殖器切割没有完全禁止。如果妳之前进行过切割手术发生问题,妳可以得到帮助。在医院,医生可以进行重建手术,试图让妳的生殖器回到正常功能。虽然不能给妳和当初一样的生殖器,但他们可以尽可能减少妳的日常问题。





* * *



22	在玛莉‧罗奇(Mary Roach)的著作《砰!科学与性之间奇妙的组合》(Bonk–the curious coupling of science and sex)里提到研究者玛莉‧波拿巴公主与埃及女性接受过生殖器切割,但仍透过阴蒂疤痕自慰。





生殖器设计师⸺为什么我们要对自己的阴户下刀?




女性(或男性)透过手术选择改变他们的外表已经不足为奇。隆乳、隆鼻、抽脂,整容⸺有些人为了满足自己的审美理想,花了很长的时间整形。然而,经由私密处整形手术改变妳的阴户是相对来说较新的趋势。

私密处整形手术是外生殖器官改变手术的总称。可能包含注射脂肪,整理和消除脂肪,减少和扩大。很多都是有可能的,而最常见的形式是阴唇整型术(labioplasty)。这是在特定的小阴唇上进行的手术,使它们变得更短。

我们认为趋势逐渐上升的私密处手术会造成问题。不,我们不打算写这个部分,因为我们喜欢轻视女性自己所做出的选择,或者是因为我们认为女性不具备自行决定如何处理自己身体的权利。当然,妳应该自己决定⸺但这是不同的议题。我们是因为害怕一群年轻女性基于错误的认知而选择进行私密处整形。根据我们的经验,有正常、健康生殖器的女性会因为觉得奇怪而选择私密处整形。这种误解需要被纠正,要做到这一点,我们必须回去剖析。

我们将选择阴唇整型术的原因,在医疗和美学上做出区别。妳因为鼻子呼吸困难而选择动刀,和因为妳不喜欢妳鼻子的样子而动刀之间不一样。同理,因为妳有疼痛或性交困难而修剪妳的阴唇,和因为妳不觉得妳的生殖器看起来不错也是不一样的想法。如果它们对妳造成问题的话,小阴唇的长度只是一个医学问题。这并不一定代表由于审美因素而希望做手术有什么不妥,但如果妳选择走这一步,妳做出的选择是根据知识,而不是误解是很重要的。

许多女性认为自己小阴唇应该要被大阴唇包覆在内,但成年女性的小阴唇比大阴唇长是正常的。其实女性看待下体的角度没有一定的方式。但我们确实有一个共同点,形成我们阴户的有这些部位:大、小阴唇、阴蒂、尿道口以及阴道口。但是,这些部位在每个女性都有不同的差异;其数量令人难以置信。即使如此,小阴唇应该要短小而且被包覆的想法仍然意外存在于许多女性心里。澳洲研究访问了十八至二十八岁的女性大众所谓的"理想阴户"为何,所有女性皆选出小阴唇被遮住的无毛阴户照片。

有这么多的完美、变化多端的生殖器,到底这种想法从何而来?连同其他形式的身体形象压力,我们可能还会考虑到流行文化和色情文化上的理想。无论如何,它们可能是问题的一部分。当提到审美角度,特别是理想的阴户时,这个问题就非常麻烦,现实上很难确认这些想法源自何处,或是它们是否不太真实。一旦有人建立正常生殖器只能有短小阴唇的想法,这个想法会比正常肚子是平的还更加强烈。毕竟,我们每天都会看到肚子;我们知道它们有各种形状和尺寸,因此很容易打掉这种想法。但是,我们不常得到窥探女性双腿之间的机会,尤其是现在的年轻女性和女孩因为害怕被看见裸体选择在公共淋浴间穿著泳衣与内裤。裸体不再是自然。裸体经常都与性有所关联,而对于很多女性来说,展露她们的身体与羞耻相互连结。

我们相信大家对内阴唇的误解有一部分来自于学校课程提到的青春期发育有极度的落差。就像身体的其他部位,女性生殖器在青春期里的变化很大,但我们无法随时记得当初在课堂上听到生殖器在青春期的改变。在学校里,我们听到阴茎如何生长,我们听到乳房如何成长,我们听到身体不同部位逐渐长出毛发覆盖。我们学到非常多,但是我们没有学到小阴唇从我们的童年到成年会发生什么样的变化。

事实上所有女孩的大阴唇会覆盖小阴唇。换句话说,我们在儿时都熟悉和习惯了这种样式的生殖器。但在青春期小阴唇开始增长。对于许多女性来说,她们的小阴唇会变长,并突出外阴唇,甚至往往厚度、皱褶不均。

如果妳的生殖器一直有大阴唇包住小阴唇的情况,特别是如果没有人告诉妳它会发生而且是正常的话,这种突然改变可能会让人们大吃一惊。妳告诉朋友自己的小阴唇不明显,觉得不对劲的想法会增强。然而,这两种类型都很正常。

换句话说,一些女性认为只有正常或"正确"的生殖器形状就像我们在儿童时期有的那样。如果年轻女孩和女性早在幼稚园学到她们的生殖器会发生改变;如果她们对自己成年时两腿之间的变化想要有更多的了解,也许我们可能不会在现在看到这么多的私密处整形案例。如果我们早知道生殖器的形式可以有这么惊人的数量,而其中绝大多数是正常、健康的,那么会有更少的女性因为误解而白白挨刀了。

重要的是要记住小阴唇的功能和削减它们代表的意义。由于小阴唇有性功能,它们充满神经末梢而且触感很好。当妳切除妳的阴唇时,妳正移除生殖器重要与敏感的部分,而所有的手术都涉及风险。在最坏的情况下,有可能会造成组织留疤变得难看,并造成永久性的疼痛,这就是为什么妳应该在接受手术前仔细考虑。问题大到值得冒这个险吗?







\backmatter

后记





我们的旅程在此结束。希望妳学到许多,我们也希望,让妳更注意双腿间的东西,变得对自己的生殖器官更好奇、更有兴趣。就算拥有全部的知识,永远都有更多要去了解。此外,医学不断进步。我们今天所写的内容也许不到一个月就过时了。我们不会停止学习。女性生殖器官很奇妙。我们真的希望妳能以拥有它们为傲。

不幸的是,我们的生殖器官也是问题的源头。许多生殖器的疾病可能会潜在地影响它们。我们精密的生殖系统被用来承受这一切,即使我们不用担心自己的睪丸会被踢到,但我们有时候太过脆弱。生殖器疾病及问题可能特别会让人感到太私密与羞愧。很少人可以完全像谈到喉咙感染或椎间盘突出那样讨论这些问题,因此许多女性在事情并非如原本相同时感到孤单与焦虑。我们希望这本书能够给妳足够的知识与一点自信,让妳抬头挺胸去看医生,妳也会知道何时需要紧张,何时又可以淡定看待。

我们也希望妳能抛弃对妳的生殖器或性生活产生过的负面想法。我们遇过许多觉得自己不正常的女性,只因为她们无法透过阴道插入得到高潮,或是认为自己得到的生殖器或阴户疱疹就跟解剖学课本上的插图一样。看完这本书妳就会知道,这些真的,非常正常。

在我们日常性生活当中时常容易忘记我们的身体不仅仅只有外观与行为而已,而裸露的身体也并非都和性有关。妳在床上所做的事很容易建立自身的自我价值,而不单指妳外观的模样。我们认为自己所缺乏的东西通常会引起强烈的感觉。妳不应该认为他人对妳有所期待而进行性行为。重要的是自己与身体都要学会享受,无论妳是一个人、有一个或三个伴侣,妳就是妳自己。每个人不一定要做所有的事,而每个人也并非看起来都一样。提到下体时,身体就只是身体,但因为妳只有一个所以显得更加珍贵。





致谢


我们想要特别感谢一些人。身为男性与医生都同样优秀的马略‧乔韩森(Marius Johansen)在书里提到的医学面向做了很棒的品质认证。我们希望这不会是最后一次的合作。其他优秀的业界人士也提供他们专业的知识。谢谢查坦‧莫(Kjartan Moe)、特朗德‧迪赛斯(Trond Diseth)、 卡利‧欧姆史戴德(Kari Ormstad)、 史怀农‧W‧瑟维(Sveinung W. Sørbye)、 约伦·瑟林(Jorun Thørring)、 安·丽莎‧赫格森(Anne Lise Helgesen)、 安德斯‧瑞能伯格(Anders Røyneberg)、艾瑟斯特‧范奇(Eszter Vanki)、 博利特‧奥斯特维(Berit Austveg)以及莱顿‧佛德(Reidun Førde)你们的对谈、观后感和评论。我们也必须感谢奥斯陆医学院的教授,在不知情的状况下,无论在演讲或课堂间的对话里仍然耐心地替我们解惑。我们再次强调这本书上的任何错误完全是我们的责任。

我们同时谢谢奥斯陆性教育医师团队、SUSS电信基金会、性与社会中心以及奥拉维亚诊所的前任与现任同仁,打造优质、激发学习动力的环境。我们也非常感谢我们亲爱的朋友与同事读过及讨论书里的内容,在我们纠结难懂的解释的时,给予一针见血的建议。

谢谢每个阅读我们部落格,提供主题的建议,询问好问题并鼓励我们的妳们。我们这本书是为妳而写。

特别感谢我们的编辑,阿思豪格出版社的娜兹尼恩‧韩艾斯丁(Nazneen Khan-Østrem)。与妳从月经聊到庞克摇滚让我们感到非常开心。知道妳在后面守护我们,让我们很有安全感。谢谢画出最棒插图的画家汉娜(TegneHanne),汉娜‧喜比约森(Hanne Sigbjørnsen)。拥有如此有趣的护士在我们团队实在是个赏赐。

现在,在尾声的阶段,无可避免得提到我们的家人。

妮娜:构思这本书的同时,麦斯也来到这个世上。最有耐心与贴心的男友,弗雷德里克,你是独一无二的。麦斯,妳是我的一道阳光,我确定妳以后读到妈妈的书会超级尴尬。我会试著不在餐桌上讲太多女性私处的事。妈妈、爸爸还有赫尔奇⸺你们是人人所希望的最棒家人。

艾伦:感谢世界上最棒的家庭,很有耐心地听著一长串关于处女膜、阴户痛、疱疹与其他废话⸺有时候在公共和不洽当的场合当中的妈妈、爸爸及赫尔格。同时感谢将我们看作世界卫生组织共同创办人卡尔‧伊旺(Karl Evang)的祖父。我对你的爱无法衡量。最重要的,我想要谢谢黑宁,理由可不只是我想要写出来而已。



阅读愉快!

妮娜与艾伦

二〇一六年十一月十五日,于奥斯陆





私密处的奇幻旅程


打破所有女孩对身体的错误迷思


GLEDEN MED SKJEDEN



\end{document}

