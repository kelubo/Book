% 恶魔的饱食
% 一.tex

\documentclass[a4paper,12pt,UTF8,twoside]{ctexbook}

% 设置纸张信息。
\RequirePackage[a4paper]{geometry}
\geometry{
	%textwidth=138mm,
	%textheight=215mm,
	%left=27mm,
	%right=27mm,
	%top=25.4mm, 
	%bottom=25.4mm,
	%headheight=2.17cm,
	%headsep=4mm,
	%footskip=12mm,
	%heightrounded,
	inner=1in,
	outer=1.25in
}

% 设置字体,并解决显示难检字问题。
\xeCJKsetup{AutoFallBack=true}
\setCJKmainfont{SimSun}[BoldFont=SimHei, ItalicFont=KaiTi, FallBack=SimSun-ExtB]

% 目录 chapter 级别加点(.)。
\usepackage{titletoc}
\titlecontents{chapter}[0pt]{\vspace{3mm}\bf\addvspace{2pt}\filright}{\contentspush{\thecontentslabel\hspace{0.8em}}}{}{\titlerule*[8pt]{.}\contentspage}

% 设置 part 和 chapter 标题格式。
\ctexset{
	chapter/name={第,章},
	chapter/number={\arabic{chapter}},
	section/name={},
	section/number={}
}

% 设置署名格式。
\newenvironment{shuming}{\hfill\bfseries\zihao{4}}

\title{\heiti\zihao{0} 恶魔的饱食\\ \zihao{1}——日本731细菌战部队揭秘\\ 第一集}
\author{[日]森村诚一}
\date{}

\begin{document}

\maketitle
\tableofcontents

\frontmatter

\chapter{前言}

因披露大量机密档案资料而备受关注的《恶魔的饱食——日本731细菌战部队揭秘》全三卷,由日本著名作家森村诚一所著。作者倾尽十多年心力,冒着生命危险采访了原 731 部队人员,还越洋渡海前往美国,费尽周折挖掘出美国、日本等密不外宣的大量档案资料,并赴中国进行现场查证,彻底揭开了关东军满洲 731 细菌战部队在中国进行活体实验以及细菌战的恐怖的全貌,引起了国际社会的极大震惊。

森村诚一先生承受着外界难以想像的巨大压力完成了这部著作,他正直、勇敢和坚韧的精神令人钦佩。事实证明,世界各地都有中国人民的朋友在努力还原历史真相,包括日本本国的朋友。

第一集中译本在 2003 年已经付梓,第二集和第三集也分别在 2004 年、2005 年译完,但一直没有奉献给读者。原因在于:最初译者曾取得作者森村诚一先生的授权,后来出现一个除邮箱外未留下任何真实信息的“代理人”,声称该授权无效。此后,尽管学苑出版社想尽一切办法与森村诚一先生联系版权的授予事宜,但是,可能出于日本国内的某些压力,这个问题一直未能明确解决。面对各种各样的困难和阻扰,学苑出版社坚持认为中国人民有权知道 731 部队在中国犯下的罪恶。于是,从 2008 年年底开始,不仅承担所有的出版相关费用,并决定将《恶魔的饱食》中译本免费赠送给一些大学图书馆、抗战纪念馆、有关历史研究部门、日本研究机构、长期关注日本问题的专家、学者和翻译工作者。

在将《恶魔的饱食》赠送至上述机构和个人后,学苑出版社还将此套图书赠送给了部分在该社网站、博客上留下真知灼见和真实联系方式的热心网友。作为出版社,让尽可能多的中国读者,尤其是青少年读者了解历史,记住历史,不让历史重演,是出版社应尽的义务。

翻译:骆为龙 陈耐轩

图源:hezhibin

OCR校对:hezhibin

出版:学苑出版社

时间:2008年3月第2版

印数:1-2000册

字数:476千字(全三卷)

ISBN:9787507720624

定价:学苑出版社免费赠阅

\chapter{作者简介}

森村诚一,日本著名推理小说作家,与高木彬光、江户川乱步、佐野洋、和横沟正史并称日本推理文坛五虎将。

1933年1月2日生于日本琦玉县,父亲是个商人。他从中学到大学,一帆风顺。1958年毕业于青山学院英美文学系。他是英语科班出身,对欧美小说读得甚多。他最崇拜的作家是罗曼·罗兰,《约翰·克里斯朵夫》是引导他走上文学道路的“圣经”。

在八十年代初,森村诚一创作了长篇纪实小说《恶魔的饱食》。这部纪实小说以大量的事实揭露了第二次世界大战期间日本侵略中国东北时,以中国人为生物实验对象来试验细菌生化武器。作品在《赤旗报》上连载,引起轰动,出书300万册一销而空。由于此书的出版,日本右翼团体视森村诚一为眼中钉。一些好心人劝森村诚一不要冒政治风险,不要招惹是非,因为森村诚一此时已功成名就,不如写点纯技巧性的推理小说安稳度日。但森村诚一毫不惧怕,他以无畏的精神表达了自己文学创作的目的:“当一个只知版税与稿酬的作家,我是无法容忍的。”、“当一个作家应当关注社会问题,以反省历史来揭露社会弊端,追求人生的真谛,这才是我写作的目的,是我生存的意义。”

八十年代中期,《恶魔的饱食》在他修订下出版了第二版与第三版。森村诚一又以推理小说的形式,写出了《新人性的证明》,以一个中国女翻译被谋杀,再次揭露日本七三一部队当年犯下的杀人事实,使日本右翼人士哑口无言。

更难能可贵的是,森村诚一为了写作《恶魔的饱食》一书,不惜工本,花了近2000万日元作私下调查。在作品发表后,因为误用照片,日本军国主义者和右翼团体借机大做文章,恶意攻击森村诚一。森村诚一在受到围攻与恐吓之后,泰然自若,他说:“如果我就此退缩,以后我还有何面目以作家自居?”他在自己的作品中援引日本宪法条文据理作出反驳,辑为《日本国宪法的证明》一书,由德间文库出版发行,这更表明森村诚一是一个具有良知和正义感的作家,也奠定了他在世界文坛的地位。

\chapter{译者的话}

日本著名作家森村诚一所著《恶魔的饱食——“关东军满洲731石井细菌部队”恐怖的全貌》一书先后于1981年11月和1982年7月在日本出版发行。作者以纪实的手法,用第一手资料系统地揭露了日本细菌部队——731部队在中国哈尔滨东南郊平房镇等地建立大规模细菌战研究基地,利用3000多名中国、苏联等国的战俘和平民进行活体解剖,研制细菌武器并在中国和中苏边境进行细菌战的历史事实,引起了日本国内外的极大震惊。这部著作的发行量一度超过180万册,成为经久不衰的畅销书。而后,作者在征求了包括原731部队人员在内的广大读者的意见之后,又在日本国内、美国和中国做了进一步的深入采访,对原著进行了较大的删改和重要补充,由日本角川文库出版了三卷本的新版《恶魔的饱食——日本细菌战部队揭秘》。

呈现在中国读者面前的这套中文版,是根据角川文库出版的下述原著全文译出的:《恶魔的饱食》1998年5月30日出版(第28版),《恶魔的饱食》(续集)1994年4月10日出版(第4版),《恶魔的饱食》第三部2000年6月20日出版(第11版)。

这部著作的新版出版,花费了十几年的时间,主要是核实历史事实和补充新资料。因此,新版比初版更加准确和充实。第一部主要采访原731部队人员,揭露了这支被称作“恶魔”的细菌部队的黑暗内幕;第二部增加了第一部出版后新查明的有关731部队内幕的事实,还利用在美国保存的资料,揭露了美苏两国围绕着细菌战部队问题进行的争执以及美、日两国就日本研究细菌武器和实施细菌战的技术情报进行交易的内幕;第三部重点披露了在中国采访中国受害者对日军细菌战罪行的控诉及在现场搜集到的各种物证等。这套著作不仅是日本侵华战争的有力罪证,而且也是对广大青少年进行正确的历史教育的好教材。

731部队是日本在中国建立的许多支细菌战部队之一,这支“恶魔”部队利用活体解剖,进行细菌实验,制造细菌武器,进行灭绝人性的细菌战,是日本公然违反国际法的严重的国家犯罪。中国是日军细菌战最大的受害国,中国中央档案馆出版的有关侵华日军细菌战挡案资料证实,仅据有案可查的记录,中国各地在侵华日军细菌战中受害人数超过27万人,这个数字不包括军人死亡人数,也不包括扩大传染后死亡的人数。中国研究侵华日军细菌战的专家认为,731部队以活体实验杀害的中国人、苏联人、蒙古人和朝鲜人等,远不止3000多人,这个数字仅仅是1940年至1945年期间在哈尔滨平房镇内杀害的人数。实际上,在1932年至1939年期间已进行过大量的人体实验,其所杀害的人数并未计算在内。

日本的细菌战部队不只是731部队,日军利用解剖活体研制细菌武器的基地也不只是中国东北的平房地区。据调查,当年日军共有7个大的细菌战基地部队,除平房镇的731部队外,在北京、南京、东南亚和日本东京等地都建立了细菌战部队,都做过大量的活体实验。据档案记载,侵华日军曾在中国二十几个省市内进行过细菌战。无论在进攻、退却或扫荡中都使用过细菌武器。许多地区由于日军使用细菌进行大屠杀的暴行,形成疫病大流行,惨绝人寰,不少人惨死,有的地方至今还残留着细菌源。据我国有关专家估计,由于侵华日军的细菌战,使中国受害人数不下200万人。

据原731部队人员在远东军事法庭上的供述,这支“恶魔”部队的建立与扩充,都是根据日本天皇裕仁的敕令实施的;侵华日军细菌战部队的每一个研制细菌武器的计划和细菌战的作战行动都是依据日本陆军参谋本部或关东军司令部的命令进行的。防卫厅防卫研究所图书馆中发现的大本营参谋本部作战课参谋的《业务日志》等“具有一级证据价值”。但是,从战败之日起,日本当局就千方百计地销毁证据,隐瞒这一罪行的真相,既不追究这些暴行的责任,也不公开有关细菌战部队的档案,甚至还隐匿了美国交还日本的731部队的罪行资料,为战争受害者的申诉设置了重重障碍。

731部队细菌战的受害者及其遗属的代表自1995年起向东京地方法院两次起诉,要求日本国家谢罪和赔偿。该法院的判决虽不得不承认“对于731部队的存在及其进行活体实验,没有怀疑的余地”,也认定了日本731部队在中国进行过细菌战,并杀害无数中国人的事实,但是又炮制种种谬论驳回中国原告提出的认罪和赔偿的正当要求。看来斗争是复杂的,还将持续下去。

尊重历史,正视历史,进行正确的历史教育,是我们翻译这套新著的目的,我们也想通过如实地揭露侵华日军的这段罪恶历史,对细菌战受害者申诉的正义斗争进行声援。我们在东京工作时,森村诚一先生在接受采访时曾说过:“《恶魔的饱食》是从加害者的角度来记述这一段历史,反映了历史的真实,许多人尤其是年轻一代读完这本书以后受到很大冲击。他们的反应是:震惊、愤怒、反省,下决心不允许这种暴行重演,而日本政府却竭力掩盖过去犯下的罪行。承认不承认《恶魔的饱食》的记述,是一个是否承认过去的侵略战争历史和侵略罪行的根本问题。”他强调:“只有承认这些错误,才能防止重演这种错误。”我们在翻译新版著作的过程中,耳边又响起森村先生十年前语意深长的这一席话。我们认为,在炭疽菌等细菌又作为生物战剂重新威胁人类的今天,向读者介绍这部新版著作,具有特殊重要的现实意义。

值本书出版之际,承蒙学苑出版社社长孟白及编辑潘占伟等同志的大力支持和帮助,在此一并表示诚挚的谢意。

译者

2007年10月于北京

\chapter{新版序言}

此次根据1982年9月访华以及尔后调查中查明的新事实,还有搜集到的新资料,对本书第一、二部进行补充核实之后出版了新版。在新版中,删除了对证词中有争议的部分,增补了查明的新事实,也吸收了读者提出的意见。关于有争议的证词内容,我打算把采访笔记、原731部队队员本身的演讲录音等整理之后另行出版。

自从发生误用照片事件\footnote{1982年2月,一个叫“竹内”的人向本书作者的助手提供了石井四郎的亲笔信、日记、地图、军帽和一批“有关731部队的照片”。作者请原731部队人员过目后认为无误,就使用了,但后经查实,这些照片中有若干张“同731部队无关”,纯属误用。因此,在“新版”中均未使用。}(1982年9月14日)以来,《恶魔的饱食》和作者以及助手下里正树遭到社会上严厉的责难,受到猛烈的攻击。从这部实录内容的份量之重和社会影响之大来看,这是理所当然的。以误用照片问题为契机,日本那些希望复活军国主义,不喜欢揭发这些罪行的势力乘机钻了进来。

部分媒体抓住误用照片问题不放,加之《恶魔的饱食》最早是在《赤旗》报上连载发表的,遂就此认为这是日本共产党的谋略,把它用作攻击该党的武器。我是同该党毫无关系的一个作家。《恶魔的饱食》纯属我个人的作品。一名作家写的纪实被用作政治、思想斗争的工具,实属罕见。

我执笔创作《恶魔的饱食》的真正意图,并非仅仅暴露侵略军的残酷性,并揭发其罪行本身,而是要把真相传给不了解战争的下一代人,以防止日本人重蹈覆辙。我相信这就是战争体检者的义务。

731部队纪实,并非只揭露一支部队局部的战争犯罪,而是通过该部队说明在战争中旨在拥有比敌人更强的武器,不择手投的机制,以及平时善良的市民一旦被战争集团狂人所控制时的可怕性。

有人批评说,731部队不光是黑暗面,如果不写它在医学和防疫给水方面的贡献,那是不公平的。的确,731部队在医学和防疫给水领域里做出的贡献是巨大的。但是,我们绝不能忘记:创设731部队的基础是侵略,日本闯进满洲,决不是受中国邀请而去的,而是日本担心赶不上欧美各国瓜分非洲之后竞相侵略中国的公共汽车,籍口用亚洲人的手保卫亚洲的名义,在中国的领土上任意划定日本的生命线而进行的便略。日本企图成为亚洲的盟主,在中国建设“王道乐土”,在整个亚洲建立“大东亚共荣圈”。侵略者无论在侵略地区为被侵略国国民做出什么样的贡献,那只不过是为了本国的利益而把别国作为牺牲品的殖民政策的一环而已。

如果改变一下立场,中国打着建设以中囯为盟主的王道乐土或大东亚共荣圈的旗号闯进日本的话,那会怎么样呢?假定说,作为殖民政策的一环,对日本做出某些贡献的话,作为日本人对于这种做法能够加以评价吗?所谓公平,对等的双方才能够说公平。无论包括731部队在内的日本军队在过去的被侵略国家里做出了什么样的贡献,也丝毫不能抵消他的侵略罪行。

《恶鷹的饱食》凝聚着无数战争牺牲者的怨恨和绝不让那种悲剧重演的誓愿。

本书曾一度绝版,现在又以更加充实的内容由角川书店再度复活,令我不胜喜悦。如果本书能够为日本和平与民主的大厦增添一砖一瓦,我将感到无比荣幸。

对于本书的新生给予巨大支持的角川书店的有关各位人士以及在困难时期不断给我温暖有力支援和激励的各位读者,我表示深切的谢意。

最后,对于在《恶魔的饱食》(续集)中误用了作者不明、同731部队无关的照片一事,我深表歉意,至于误用的始末,将在“笫三部”中详述。

森村城一

\chapter{序:今天窥视到的恐怖真面目}

——写作731部队纪实的必要性

我在《赤旗》报星期日版上刊登连载小说《死器》的过程中,有机会接触到原满洲731部队即日本陆军细菌战部队的许多生存者。

这支世界上规模最大细菌战部队集中了日本全国优秀的医生和科学家,以3000名以上的战俘为对象进行惨无人道的活体实验,生产了大量细菌武器。

后来,美国生物化学部队获得了这种技术和技术秘诀,但是,关于731部队的记录,在战争结束的同时,被彻底地销毁了,变成战争史中的一段空白。生存者们也像商量过似地都守口如瓶,不愿谈论有关情况。

在哈巴罗夫斯克远东军事法庭,曾审判过731部队。以这次审判记录为基础写的作品有岛村乔著的《三千人的活体实验》(1967年原书房出版),山田清三郎著的《细菌战军事审判》(1974年东邦出版社出版)等精心创作。

我接触原部队队员之后,窥视到这支都队恐怖的真相。731部队是日本陆军产下的恶魔部队。他们把生物学和医学转用为武器,并实施国际法上禁止的细菌战。队员们对自己的经历严守秘密,其中许多人隐姓埋名地活着。

战争本来是很残酷的。对于从事在战争中都认为是惨无人道而被禁止的细菌战的人们来说,尔后的人生必定是很沉重的。但是,我们必须真实地把真相记录下来,历史决不能留下空白。经过我们拼命的说服,这些人好不容易才开始张开了紧闭的嘴。

开始时,我想把这些材料写进小说《死器》中去,但是,有关731都队的情况都是客观存在的事实。我感到有必要把听到的情况原原本本地写下来,而不做小说般的修饰和加工。

对于把自己的青春或人生中最能取得硕果的时期献给731部队的人们来说,731部队究竟为何物呢?战后这些人度过了沉重而苛刻的人生,记录这些真实情况的笔也是很沉重的。

对在这次采访中给予协助的731部队的生存者们深表谢意的同时,我打算切入正题。

首先,让我们概观一下731部队1933年在哈尔滨市近郊背阴河延生前后满洲和日本所处的国际环境。

当时的日本,在中日甲午战争、日俄战争和第一次世界大战中相继获胜,资本主义和军事力量取得了飞跃发展。这些战胜并非来自日本自身的国力,而是借助于清朝末期的软弱和俄国国内革命等,但是,却把日本的国际地位推上到同欧美列强并肩而立的地步。

明治政府提出的国家意志是,改变由于锁国而带来的落后,建立一个与欧美列强并肩而立的大国。这种国家的意志,由于得助于三次侥幸的胜仗,使日本的野心膨胀起来,乘自19世纪末期欧美列强在亚洲各地进行帝国主义侵略之机一举扩大了它的版图。

由于欧美列强的侵略,中国到处被蚕食,日本在日俄战争中取胜后,把俄国势力从满洲赶出去,从而増强了他在满洲的优势。

但是,日本迅速的抬头,遂步与欧美列强产生了摩擦,尤其同不喜欢日本在满洲占居优势的美英之间的关系变得险恶起来。

当时的日本军部有一种强烈的意识:即满洲是日本官兵用鲜血换得的领地。这同突破日本资本主义所产生的危机对策联系在一起,从对满洲的侵略贪婪地扩大到对中国大陆的侵略。

从清朝末年(1900年)至1937年实现“国共合作”,全中国进入团结一致抗日态势之前,中国处于群魔乱舞状态之中,欧洲列强竞相侵略中国,各地军阀割据,混战不休。日军乘机在满洲扩张势力。在这些势力之间,土匪里、马贼和秘密结社活动猖獗,加之满蒙独立运动混入其中,进一步加剧了动乱。

中国这种极其动乱的状态,大大地刺激了当时日本青年的野心。在“大陆雄飞”之梦的煽动下,年轻人竞相进入中国。在日本无法谋生的浪人,冒充忧国之士进入大陆。伊达顺之助、松本要之助和川岛芳子等人都活跃在这一时期。

在这一土壤之上,日本的派遣军部——关东军于1931年9月18日发动了柳条湖事件,在正当防卫的借口下,无视政府不扩大战争的方针独断专行,占领了整个满洲地区。接着,于1937年7月又制造了成为日中战争导火线的卢沟桥事变。

在这种时代背景下,731部队确定在哈尔滨郊外平房地区建设他的根据地。

\mainmatter

\chapter{哈尔滨市以南20公里的军事特区}

\section{加茂部队的秘密——731部队的前身}
731部队决定把根据地设在哈尔滨,它是今天中国黑龙江省的省会。1982年7月,人口为254.4万人,是一座重工业城市,位于中国东北平原的中央。它是沿着黑龙江的支流松花江发展起来的。据哈尔滨市人民政府东北史学会的关成和先生说,“HARUBIN”的语源是从满族女真族的“家荣”转译过来的,汉译为“阿勒锦”,意思是荣誉。

1898年,帝俄为了实现其侵略中国东北地方的野心,把这一地区作为铺设东清铁路的根据地。同年6月9日,作为铁路建设团第9分团团长的朱柯夫亲王被看作是哈尔滨的创始人来到这里。他一度把阿勒锦改为松花江市,但是遭到市民们的强烈反对。结果于1903年7月14日又改称为哈尔滨。

哈尔滨在满语中是“渔网晒场”,在蒙古语中是“平地”,在俄语中误译为“大坟墓”等,附加的这些意思,均无历史和语言的根据。以莫斯科为样版而建设的哈尔滨在街道整齐的市中心区内欧洲风格的建筑栉比鳞次,显示出帝俄在这里曾经进行过正式的城市建设。

从历史上看,实际的称呼是“HARUBIN”。这是正确的。但在本书中,作为731部队的根据地,仍使用当时日本方面通用的名称——“HARUPIN”\footnote{本书中无论“HARUBIN”或“HARUPIN”,均按中国现行行政区划正式名称,使用哈尔滨。——译者}。

继第一次世界大战之后,发生了俄国(十月)革命,因此,这座美丽的城市一度回到中国政府的管辖之下,但是,不久后于1931年9月由于发生了以谋略性的柳条湖事件为发端的满洲事变,哈尔滨的命运发生了巨大的变化。

以柳条湖事件为契机,早就一直企图占领满洲南部地区的日本关东军雪崩般地开始了侵略。

关东军占领奉天(今沈阳)之后,尽管若槻内阁提出了不扩大战线的方针,但是仍然不断地扩大军事占领地域,占领了北满的吉林、齐齐哈尔,直到辽西的锦州。1932年3月成立了满洲国,制造了傀儡政权。

在这个过程中,哈尔滨陷入了关东军的手中,直到第二次世界大战结束时为止,把行政上的称呼,改为滨江省的省会。

1933年在哈尔滨设置了细菌战秘密研究所——后来的关东军防疫给水部总部(通称为石井部队)。开始设置在哈尔滨市东南方的拉宾线上的小站背阴河。为了保密,称作“加茂部队”(也有人说开始时称作东乡部队)。“加茂”这两个字,是解开创设这支部队之谜的关键。

“加茂部队”到1938年已变成一支大规模的秘密部队。

同年6月13日,将距哈尔滨市中心区往南约20公里地区,当时称作滨江省平房镇,划定为关东军的军事特区。

在平房附近有三屯、四屯和五屯三个村落,军事特区大致设在这三个村落的中心位置。目前,关于其所在位置,曾有种种说法和写法,但满洲731部队的位置,这里是准确的所在地。

在禁止入内的约6平方公里的宽阔的军事特区内建设大规模的军事设施花费了一年多的时间。设施包括宿舍群、发电站、铁路支线、训练设施。平时可关押80-100人的监狱、大大小小的许多研究室、训练用的马场、大礼堂、运动场和神社。

军事特区周围是架设着高压电线的土墙和壕沟。“加茂部队”从背阴河和滨江迁进了这个用铁丝网围起来的大军事设施之中,那是1939年,但是由于是分多批进驻的,准确的日子不得而知。“加茂部队”在这里一时改称“东乡部队”。在发生诺门坎事件(1939年5月-8月)两年以后的1941年8月改用秘密番号“满洲731部队”。

在哈尔滨市以南约20公里地区建成的这一军事设施内,究竟发生了些什么事,进行了什么样的研究,当时是绝密中的绝密,是被封锁在军队厚厚的帷幕之中的。“即使是友军的飞机,擅自飞入上空时,也可以击落”。731部队甚至拥有专用的战斗机。

1949年12月在哈巴罗夫斯克远东军事法庭才揭露出这一部分的秘密。

本书发表的“要图”(本书后面的插页),是由笔者在《赤旗》报星期日版上执笔写连载小说《死器》时接触过的几位原731部队人员绘制的。它显示了关东军防疫给水部总部设施的全貌及部队部署。

这张“要图”标明了每一个编制名称和班名,并涉及了它的研究内容,指明了设施内的任务和分布。无论在战前或战后,这张图都是首次公开发表。

\section{恐怖的课题组}

根据前731部队人员们一致的证词,为了去平房附近设置关东军防疫给水部本部,必定要经过哈尔滨市内的秘密联络站。

哈尔滨市分为新旧两大区。在新市区吉林街有一幢叫做“白桦寮”的红砖建筑,它是三层的高大建筑(部分是两层),这里就是联接部队外部和内部的秘密联络站。

白桦寮和法国电影中常常出现的那种公寓相似,是一幢设有内部庭院的“口”字型的建筑,在入口处有大门,供部队的军用卡车和大轿车出入。队员们前往哈尔滨市时,先乘大轿车或卡车到白桦寮,在内部庭院里换上便装再出大门,前往哈尔滨市内。返回731部队时则按相反的顺序来进行。

哈尔滨市内当时设有花园小学、桃园小学、哈尔滨中学、哈尔滨女高以及作为培训满洲国官吏机构的哈尔滨高等学院等日本人学校以及许多教育设施。在这里上学的队员子弟们也是乘军用大轿车前往白桦寮内部庭院的。

白桦寮是一种像公用大厦的建筑。除此之外,也作为若干与军队有关的机关、团体的办公室、投宿、就餐来使用,表面上是满洲国政府接收的一幢建筑物,实际上是731部队的秘密联络站。不过,即使是哈尔滨市市内的日本人中也只有极少数人知道这一点。

从白桦寮出发,穿过哈尔滨市内,在左手方向可以看见忠灵塔,乘坐大轿车沿着宽阔的农村道路,摇晃约近一小时,通过新发屯,过五屯,来到右手方向可以看到四屯的地方就看到了与被叫做8372部队的航空队专用机场相邻接的、用铁丝网和土墙弯弯曲曲包围着的宽阔的一角。这就是731部队的根据地。

部队的设施大体上可划分为以下六个区:

(一)从其形状看,叫“口”字楼\footnote{口字楼的口是日语字母RO,由于该建筑呈四方形,又称方字楼。——译者},是集中设置第一部、第四部的各部门的主要大楼;

(二)同“口”字楼相毗邻的是总务部、器材部等所在的建筑,其中设有总务部和医疗部的大楼,叫“1号楼”;

(三)设有食堂和放映厅的大礼堂;

(四)负责进行少年兵和从全中国各地、日本各部队集中来的卫生兵等人员训练的教育部大楼;

(五)部队人员及其家属居住的宿舍区,叫“东乡村”,其中包括单身宿舍和部队修建的东乡神社;

(六)在哈尔滨市滨江车站附近,还有一座第三部和诊疗所的大楼(通常叫“南楼”)。

在上述设施中,前两区是由布满高压电铁丝网的土墙围起来的,这两栋楼通常叫“总部”。(“要图”中用黑粗线标志)

在“要图”中,虽然巳做了标记,但在这里再次说明一下731部队的组织概况:

部队长:石井中将(1936-1942年,1945年3月至战争结束两段时间任职,在1942-1945年2月期间为北野少将)

总务部:部长中留中佐(后由太田大佐接替)

第一部:研究细菌,部长菊地少将

第二部:研究实战,部长太田澄大佐(兼)

第三部:制造滤水器,部长江口中佐

第四部:制造细菌,部长川岛清少将

教育部:训练士兵,部长园田大佐(后由西中佐接替)

器材部:实验器材,部长大谷少将

诊疗部:附属医院,部长永山大佐

此外,731部队沿苏满边境还设有四个支部和一个实验机场。即海拉尔、林口、孙吴、牡丹江支部和安达实验机场。

在大连还有一个由安东工程师(将官)率领的满铁卫生研究所,它直辖于关东军,在同731部队密切合作下,制造菌苗,并进行各种实验,实际上也可以说是731部队的支部。

战后,人们把731部队通称为“石井部队”,是由于这里的大规模设施和组织,都是根据部队长石井中将的创议而设置的缘故。

从1939年建成一系列设施起,约有2600余人在731部队从事细菌战的研究工作。其中大部分人是由日本内地大学医学部、医科大学或民间研究所派遣来的研究人员和学者。他们的身份是军队职员或工程师。

部队里设有19个令人毛骨悚然的研究作业班,按照现在的说法,就是“课题研究组”,过去任何书刊都没有发表过它的全貌。整个组织情况如下:

特别班:负责管理“马鲁太”\footnote{“马鲁太”系731部队内对在押活体实验用人员的称呼,日语原意为“圆木”,在这里把人视为任其随意使用的实验材料。一旦被送进731部队,无论哪国人,都不再称呼人的姓名,而称XX号“马魯太”了。——译者}

第一部:

笠原班,研究病毒

田中班,研究昆虫

吉村班,研究冻伤

高桥班,研究鼠疫

江岛班,研究赤痢

太田班,研究炭疽

凑 班,研究霍乱

冈本班,研究病理

石川班,研究病理

内海班,研究血清

田部班,研究伤寒

二木班,研究结核

草味班,研究药理

野口班,研究立克次氏体(斑疹、伤寒等病原体)

第二部:

八木泽班,研究植物

烧成班,制造炸弹

第四部:

柄泽班,制造细菌

朝比奈班,制造伤寒及疫苗

在东乡部队时,原来已经确定了这些研究班的正式编制名称,如“第一部细菌课”、“第一部病理课”等。但是东乡部队改称731部队以后,完全隐蔽了正式课名,从保密要求出发,只称呼班名。

这些班名是部队内部使用的一种“隐语”,正式编制名称是第一部下属14个课,如第一部田部班是第一课;凑班是第二课;江岛班是第四课;高桥班是第五课;石川班是第七课;吉村班是第八课;二木班是第十一课等。

在“课题研究组”编制表最前边列出了一个“特别班”负责管理“马鲁太”。

“马鲁太”是什么呢?

\section{被剥夺了人格的“人”}

所谓“马鲁太”就是指关东军宪兵队和特务机关以及它所管辖下的“哈尔滨保护院”里关押的苏联人、中国人和蒙古人俘虏(也包括朝鲜人)。

关东军宪兵和特务机关逮捕了潜入中国各地的苏联红军情报军官、在作战中成为俘虏的中国红军(八路军)干部及战士、为反对日本帝国主义的侵略而参加抗日运动的中国记者以及学者、工人、学生乃至他们的家属等许多人。

关东军把捕获的俘虏叫“马魯太”,一律以特种转移的方式由各地送往731部队“口”字楼内的特设监狱之中。

被关东军逮捕的爱国者们,遭受非人的待遇,仅仅是被当做一块圆木头来使用的“马鲁太”。

由于是“马鲁太”,也就无须有人的姓名。凡是送往731部队的“马魯太”,每个人都编上一个三位数字的号码。然后按编号分给前面谈到的各个班,归各班所有。根据研究目的的需要,他们把“马鲁太”当做进行活体实验的“材料”来使用。

对部队的各个班来说,“马鲁太”的经历、人格自不待言,就连年龄也是毫无意义了。

在被送往部队之前,无论宪兵的审问是多么地残酷,“马魯太”仍然是会开口说话的人。

但是,自从他们作为“马鲁太”被关进部队之日起,全部都变成了无法生还的实验材料。

也有一些女的“马鲁太”,她们是以反日分子的嫌疑而被逮捕的俄国女人和中国的女学生。女的“马鲁太”主要用来作为性病的实验材料使用。

在部队“口”字楼的中间,有一幢混凝土的二层建筑,周围是用长廊围起来的。这里有许多单间牢房,每间牢房都有一个小窗户。这幢混凝土建筑同各研究班直接相联接,叫“马鲁太小屋”(即731部队特设监狱)。

特别班管辖的“马鲁太小屋”,左右分为两个部分,通常称做“7号楼”和“8号楼”。从移驻平房初期起,各建筑物就从1号起按顺序以编号称呼。原则上“7号楼”是关押男的“马鲁太”,“8号楼”是关押女的“马鲁太”,由于女的“马鲁太”少,有时“8号楼”也关押男的“马鲁太”。

据被告人川岛在哈巴罗夫斯克军事法庭上所作的证词,731部队经常关押着二三百名“马鲁太”。实际数字没有记录。

根据各班的实验研究的不同目的,“马鲁太”被转移到单间,或以三至十人为单位被转移到杂居房。

在送入731部队监狱之前,他(她)们一直受到关东军宪兵队夜以继日的拷问,但是,一旦被送进了731部队,就停止进行任何拷问和虐待,也不强制他(她)们服苦役。

不仅如此,他们供应“马鲁太”最好的伙食,给予充分的睡眠,甚至还发给他们维他命药剂,以便早日恢复已经衰弱了的体力,恢复为健康的肉体——这就是赋予被关押的“马鲁太”的任务。

由于获得了充分的营养,除了做完冻伤实验后的人以外,其他“马鲁太”都很胖,每天什么事情也不做。到接近供做实验时,肯定是死,或者说等待他们的是地狱般的痛苦,但是直到走上实验台的前一天,他们每天的生活是无所事事,表面上是十分无聊的。

作为“马鲁太”而关押的中国女教师做纸捻,并以它作材料编制小的中国鞋或其他工艺品。这就是“马鲁太”的日常生活。

但是,营养丰富的日常生活是短暂的。“马鲁太”的新旧更替是十分频繁的,大致是按照两天三个人的比例被当做实验材料来用掉。

据后来川岛被告在哈巴罗夫斯克举行的远东军事法庭上的供述,在1940年至1945年期间,731部队“消耗”掉的“马鲁太”达3000人以上,但是原部队人员一致做出的证词是“恐怕比这个数字还要多”。

关东军十分重视731部队秘密领受的特殊任务,为他们更加容易进行这种研究实验提供了各种方便。

“方便”之一就是源源不断地供应“马鲁太”。

他们按照实验的程序来使用“马鲁太”,给他们注射鼠疫、霍乱、梅毒、螺旋体等菌苗,或者在他们的饮料或馒头等食物中渗入或人工“移植”这些菌苗。有的也被用来进行冻伤实验、枪杀实验或瓦斯坏疽实验。

\section{残酷的给养}

关押在731部队里的“马鲁太”获得充分的营养。

养肥“马鲁太”有四重意义:

其目的之一是为了获得完善的“实脸材料”,如果“马鲁太”身体衰弱或患疾病,就失去了“实验”的完善条件。

731部队一直全面负责进行细菌战准备的研究和实际作战。为了进行细菌战,他们需要有关细菌传染的准确资料。

健康的人体在何种条件下会患鼠疫和霍乱,经过一个什么样的过程才致死或得救呢?

为了彻底了解各种细菌感染、发病的过程,部队通过大规模的临床实验,来搜集资料。只有通过连续使用完善的“实验材料”才能取得完善的资料。“马鲁太”必须是健康的,而且随时可以进行补充。

养肥“马鲁太”的另一个目的是研究预防和治疗各种传染病的方法。

为了进行细菌战,必须深入敌后或在前线散布大量的病菌。

这种作战自然是由日军部队(731部队)来实施。届时,日军士兵由于不注意,皮肤和口腔接触病菌的可能性极大。另外,在实施细菌战之后,友军也可能会进入细菌污染地区。

由于这种缘故,必须掌握预防和治疗的方法,否则,马上就会有导致产生“悲剧”的可能。如果由于进行细菌战,而使友军蒙受损失的话,那么,开展细菌战就毫无意义了。

彻底打击敌人,认真保护自己,是进行细菌战的要诀。因此,必须“研究”针对鼠疫、霍乱、伤寒等病毒的菌苗,并研究血清疗法或利用其他药物的化学疗法等。

在大规模的细菌战中,需要大量预防用的菌苗。部队为了生产大量的细菌,同时,还要开发研究预防和治疗用的菌苗。

为了制造菌苗,必须进行许多实验,需要大量的血清。为此,有必要使“马鲁太”处于健康状态,并发胖。

有一原731部队的人员回忆说:“部分‘马鲁太’被注射细菌发病以后,以当时731部队最好的疗法,来控制‘马魯太’病情发展,因为他们想尽可能再次使用‘马鲁太’,也想从患病而尚未致死、又复原了的‘马鲁太’身上抽血或作为标本来使用,这种‘马鲁太’作为研究材料来说是十分宝责的。”

保证“马鲁太”获得充分营养的第三个目的是研制细菌战的“武器”。

731部队把老鼠和寄生于老鼠的跳蚤作为传播鼠疫病菌的有力媒体来加以研究。研究如何让老鼠和跳蚤传染上病菌,并让它们接触或聚在人体上,研究用什么方法在敌后和前线大量放出活老鼠和跳蚤……而利用“马鲁太”来解决这些难题是有用的。

细菌武器不仅仅采用小动物和昆虫,把炭疽菌和伤寒菌等放入食物中或投入井水和饮水之中,就是最好的“武器”。他们研制了滲入细菌的巧克力糖和馒头,用来进行实验。为了进行这种实验,也需要健康的“马鲁太”,又丝毫不承认他们的人性。

部队研制了自来水笔式的细菌手枪和手杖式的细菌枪,来实验这些“枪”的效果,马鲁太是最有用的。

养肥“马魯太”的最后一个原因是为了研究防治中国东北地区的地方病。

当时驻苏满边境的日本军队的部分官兵中,蔓延着一种原因不明的流行性出血热。据估计,这种疾病是由某种病毒或立克次氏体引起的。731部队曾使用“马鲁太”研究流行性出血热。

\section{“马魯太”与“圆木”之间}

731部队关押的俘虏,一律被叫做“马鲁太”,这一事实是1949年12月25日至30日在哈巴罗夫斯克举行的军事法庭上查明的。

让我们引用当时的公审记录——《关于目前日本军人被起诉准备和使用细菌武器事件的公审文件》(1950年,莫斯科外语图书出版社出版)的一段吧。

出席法庭的是前军医少将川岛。他在731部队中历任总务部长、第一部长(负责研究细菌的部门)、第四部长(负责制造细菌的部门)等职,是部队中的一个资深军官,战后曾被苏军逮捕。

以下是国家检察官斯米尔诺夫提出的问题和川岛的回答:

问:你们为何不在日本而在满洲进行细菌战的准备呢?

答:满洲是同苏联相邻接的国家,战争开始时,从满洲地区使用细菌武器比较容易和方便,而且在满洲进行有关研究细菌武器的实验非常方便。

问:在满洲进行实验的“方便性”究竟表现在什么地方?

答:之所以说满洲非常方便,是因为在那里有充分的实验材料的缘故。

问:“实验材料”指何而言?是否是为供实验使用而送往部队的人?

答:就是这个意思。

问:部队里使用何种隐语来称呼实验牺牲者?

答:把他们叫做“马鲁太”(MARUTA)。

问:在监狱里你们是按照他们的姓名关押的吗?

答:不,他们都有编号。

问:那么,这些人最终都必须死去吗?

答:是的。

问:你作为专门研究细菌的人,采用蔓延杀人性的传染病作为战争的手段,难道不知道会带来可怕的灾难吗?

答:是的,我知道这一点。

问:你难道不知道鼠疫及其他传染病引起的可怕的灾难也会蔓延到中立国家去吗?

答:是的,我知道这一点。(以下略)

山田清三郎著《细菌战军事审判》记录当时公审法庭上有关“马鲁太”问题的对话,记述如下:

问:部队使用何种隐语来称呼实验牺牲者?

答:把他们叫做“马鲁太”(MARUTA),是材料的意思。(带点的‘字’是作者注的)

这是同一次公审法庭,同一场面的对话记录,但是却漏掉了山田清三郎著《细菌战军事审判》中提到的“‘MARUTA’是材料的意思”这一句。

从编写《细菌战军事审判》的前后情况和山田的经历来看,可以认为山田的叙述是准确的。据原731部队人员一致的证词,关押在特设监狱中的俘虏,都一律叫做“马鲁太”。

记录各种实验时,按其性别,在表格上单纯地填上阳性“马鲁太”或阴性“马鲁太”。这些都是731部队中表明实验材料的专用语。

“马鲁太”就是“马鲁太”,它是否意味着什么“圆木”材料,队员们是不知道的。

“公审文件”中漏掉的前军医少将川岛关于“‘MARUTA’是‘圆木’材料的意思”的附带说明,具有重要的意义。

被告人川岛在公审法庭上如实地解释了在部队内部表明实验材料的“马鲁太”这个词正如山田所写的,是“材料”。

可是,后来在把公审记录译为日文时,译者使用了可以切削、搬运甚至可以燃烧的“圆木”这个日语词汇,估计是那时删掉了附带说明部分。我想法庭上的实际对话可能是这样的:

问:部队里使用何种隐语来称呼实验的牺牲者?

答:我们把他们叫做“马鲁太”。“马鲁太”是实验材料的意思。

731部队中使用的隐语“马鲁太”,在《公审文件》出版以后变成了“圆木”。“圆木”是植物,而“马鲁太”是被剥夺了人格的人。把活生生的人叫做“材料”,这就是731部队的恶魔性之所在。另外,“马鲁太”也有作为“实验动物”的野山羊的意思。

自此以后,正如《公审文件》中那样,其他文章中也决定使用“圆木”。

\section{审俘要领}

“马鲁太”从关押之日起,“人”这个固有名词就被抹掉了,而是用编号来称呼。但是,其中也有个别人的姓名流传下来。

牡丹江的老铁路工人孙朝山、木匠吴殿兴、修理工朱志敏、沈阳的爱国人士王英、大连的商业公司职员钟民慈、山东省的中国共产党员邱德思和乌克兰的苏联红军战士德姆契柯……

留下姓名的人都是在关东军宪兵队、哈尔滨特务机关逮捕和审讯中坚持斗争到底的人。

例如中国共产党党员邱德思,“在日本宪兵队的严刑拷打面前坚贞不屈,直到最后忠于自己的人民,未做叛徒”。(引自《公审文件》)最后,被送往731部队遭惨杀。

苏联红军战士德姆契柯顽强地拒绝提供有关苏联的任何情报,坚持不开口回答问题。尽管哈尔滨特务机关“把他的手脚捆绑起来,吊在屋梁上,严刑拷打”,但是德姆契柯始终坚持未供出任何情况。(引自《公审文件》)因此,被送往731部队。

对于拒绝审讯、坚持斗争的俘虏进行的拷打是很残酷的。这里有一份关东军宪兵队的手抄文件。这份题为《审俘要领》的文件表明当时这些“马鲁太”在731部队关押之前,遭到了如何残酷的拷打。在“总则”部分:

第1节\ 本审讯是基于搜集情报的目的进行的,不包括旨在调查犯罪行为的审讯在内。

第2节\ 对于投降者、逃兵、逮捕的敌方间谍、非法越境者、流浪者、迫降者、被俘又逃回我军的人员、新占领地带的居民、从敌区逃回的居民等进行的审问,除特殊情况外,均可参照审俘方法执行。

731部队创设时的哈尔滨则相当于这里所说的“我方新占领地带”。对该地区的居民参照本审俘要领。

在第一篇——对俘虏的调查的“通则”部分的第三节:

第3节\ 通过审俘获取情报,比使用间谍或其他手段搜集情报更加容易和迅速,常常可以获得利用其他方法所难以取得的重要情报,各部队及各级司令部均应致力于审俘,这是至关重要的。

审俘要领说,从俘虏口中获取情报,比利用间谍和其他手段获取情报更加容易和迅速,而且重要情报多,因此要重视审俘。(第4节至第59节未收录)

第60节\ 对怀疑我方措施或相信敌方宣传而顽固拒绝供述真情者,可给予绝对保护(具体给予生命安全或前途光明等“保证”),揭露敌方宣传的欺骗性,提高对敌我态势的认识(总之,做出有利于我方的证明)。一旦思想发生转向而自供者,常常有利于用来说服后来获得同样性质的俘虏。

关东军宪兵队逮捕的“抗日分子”是抱有坚定信念的犯人,应让他们了解日本是不可战胜的,使之转向。一旦转向的人,在逮捕到同样的“抗日分子”时,可以作为说服人员发挥作用。

第61节\ 被审讯者为保护自己,有时希望在“内部”说明自己了解的情况,尤其是关于军事力量的情况等。所以应通过用套话的策略或以赞杨的方法来获得可靠的资料。

第62节\ 根据情况,有时进行拷问有利,但不少情况下又往往伴随着弊害。所以,事先要研究是否使用拷问的形式,不致于以后对这种方法感到遗憾。

在这里解释一下审问要领中的拷问方法。

\section{恶魔的送终水}

第63节\ 持续进行拷问,给俘虏在肉体上造成痛苦,使之除陈述真实情况外,别无消除痛苦的方法。

所以,对于为了加快审讯速度而持续进行的拷问,意志薄弱者比较容易吐露真情,但他们也有可能为了迎合拷问或为摆脱眼前的痛苦而说谎。意志坚强者有时会增加反抗情绪,或在审讯后对帝国留下极坏的感情。

审俘要领说,由于给俘虏造成肉体上痛苦,使之感到除自供外,别无逃脱拷问的方法。俘虏为了逃脱痛苦,有时会做出迎合性的陈述,这反而使我们弄不清其实情况。对那些意志坚强者进行拷问,会促使其增强反抗心和反日感情。所以,要做到毫无遗憾。

第64节\ 通常在两种情况下实施拷问:对已掌握确凿证据者,只让俘虏就内容供出真情,而且通过拷问完全有可能取得情报;对意志薄弱者,估计完全有可能屈服于拷问。

第65节\ 实施拷问的手段,应着眼于容易实施、没有残忍感、痛苦持续性大且不留下伤痕。但是,需要使对方感到生命危险时,则不应顾及伤害如何,而应保持“持续性”。

简言之,就是最好采取:容易实施,表面看不残酷但痛苦的时间持续长且又不留下伤痕的拷问手段。有时有必要使俘虏感到“再这样坚持下去就将被杀害时”,应毫不踌躇地留下伤痕。持续进行拷问是必要的——也就是说,如果判断必须这样做时,就要毫不客气地拷问下去。

让我们列举一个进行拷问的实例:

1.让俘虏跪坐;

2.用几枝铅笔夹在各手指根部的夹缝里,然后用绳子或皮条紧缚指头,使之发生动摇;

3.让俘虏仰卧(腿略抬高)向鼻子和嘴里灌水。

4.让俘虏横卧地上,踩其踝骨;

5.让俘虏站在比身子低矮的棚子下面。

第66节\ 如发生误伤时,要从有利于国家的大局考虑,做出断然处置。

跪坐、夹铅笔、灌水、踩踝骨、长时间屈身站立,如果造成俘虏伤害,为了保护日军的利益,就要断然处置。送731部队者,就属于这一类。

第67节\ 通过拷问取得供词时,对那些为了逃脱痛苦而做出的迎合性的供词,要研究其真实性,并取得证据。

第68节\ (略)

第69节\ 进行拷问时,不得让无关人员,尤其不能让其他俘虏知道。“审俘要领”中提到届时要特别注意,不要让拷问时的痛苦呻吟声外传,还要留意观察俘虏。

第70节\ 在审讯中,应注意被审讯者的脸色、姿态、眼神、声音变化以及嘴唇动作等细微部位的变化。以观察其供述语言以外的心理状态,要经常注意从喜、怒、哀、乐、羞耻、恐惧以及震惊等面部表情发现其有无隐瞒。

1.审讯时说口渴要求给水喝者多半是自供前内心发生苦闷。

2.严密注视着被审讯者的态度和脸色等,这种人企图从中看审讯者对供词的反应,他们多半仍隐藏着秘密。

以上是“审俘要领”文件的结束。俘虏开始要水喝是自供的前兆。

\section{中央集中供热部队}

731部队的设施,是在当时设在长春的关东军司令部的直接监督下,经施工部门特殊设计,由军队的御用企业(除日本特殊工业大林组公司外)施工建成的。

在当时军队的有关设施中,特殊设计施工是最奢侈的。一说奢侈,人们的脑海里就会浮现出豪华的高级建筑的形象,但是,731部队的设施却“奢侈”在超群的清洁方面。

在大致分为五个区的建筑物中,除教育部和卫兵所以外,其他如大礼堂、宿舍、“口”字楼(总部所在地)等建筑物都修建有抽水式厕所。仅仅这一点在当时就是令人吃惊的。

负责准备、研究并实施细菌战的731部队制造大量的细菌,不断进行大规模的实验。关东军最怕的是部队内部的细菌感染。

731部队的全部设施,直至3000名队员及家属的居住区都使用抽水式厕所就是为了预防细菌感染。

731部设施的另一个“奢侈”的地方,按现在的说法,就是建立了中央集中供暖系统。除教育部的建筑外,供暖系统几乎遍及了每一幢建筑。

从宿舍到总部大楼,所有房间都安装暖气,有热水供应系统。无论在哪一间房里,打开水龙头就会流出热水,像高级旅馆一样。
在部队院内的一角设有三台利用塔库马式锅炉的两台发电机,热水和暖气都由这里的锅炉来供应。在距塔库马式锅炉不远的地方,设有瓦斯罐,供应部队专用的煤气,也利用它制造出高温蒸气来做饭。

各研究室内部都设有洋式个人专用洗澡间,宿舍区也有公共浴池,约三分之一的宿舍里设有家庭澡塘。

饮用水和研究用水都是从731部队院内的水井中抽出的地下水。由于水井很深,抽出的是硬水,要经过发电站加工成软水以后才进行供应。

队员们把731部队夸耀为“满洲首屈一指清洁的军队”。

恶魔般的细菌部队却拥有最清洁而且现代化的设备,真成了二律背反。但是,这里是有它的道理的。

如上所述,整个院内建立完备的抽水式厕所和下水道,是为了预防细菌感染。“中央集中供热系统”是二十四小时大量制造细菌所必不可缺的设备。进行实验和研究,需要充分的电力供应,热水供应也是必不可缺的。

一名原队员说:“给人留下的印象是无论哪一幢楼,不分昼夜,总是灯火通明。”

731部队设施还有一个特点,就是建筑物的面积和房间设计比较宽敞。

楼内各层的天花板都很高。本部大楼(“口”字楼)等虽说是三层楼,但整个建筑物的高度都相当于现在的五层楼左右。“口”字楼的外壁,是在混凝土构造上,又镶上乳白色的瓷砖。从哈尔滨市内乘公共汽车驶近这一地区时,给人这样一种感觉:在辽阔的平原上突然出现了一幢被土墙包围着的白色建筑。

在三层的“口”字楼里,设有手动式电梯,可以一直通往楼顶。

教育部大楼是一幢砖楼,但宿舍却全部是坚固的混凝土建筑。

关押“马鲁太”的特设监狱,是一幢被“口”字楼包围起来的二层建筑,修建得特别坚固。设计是特殊的,只要一扳动阀门,整个狱内就会充满氰酸瓦斯。部队人员说:“一旦有事时,扳动队长室里的阀门,‘马鲁太’就会全部死掉。”

总部大楼“口”字楼的一层都叫“地下”。在一层走廊正中间,铺设一圈供小型四轮手推车用的滑轨(轨距50公分)。柄泽班把大量制造好的细菌装在一种特殊容器之中,由“工厂”运往细菌仓库收藏。这种金属制的特殊容器很重,所以才设计了这种手推车,并在走廊中铺设手推车用的铁轨。

731部队这种尽善尽美的设施,存在着一个难以解决的问题:那就是有一股十分厉害的刺鼻的腐臭气味,笼罩着整个总部大楼,有时甚至飘到院外。

笼罩着731部队的令人作呕的腐臭气味,使人们真正“嗅”到了这支部队整个设施的性质。这股腐臭气味究竟来自哪里呢?

\section{腐臭气味的元凶}

笼罩着731部队的腐臭气味,主要是“琼胶\footnote{琼胶也称“寒天”,是一种细苗培养基。——译者}”的烂臭味。

731部队”口“字楼的整个一层,全部被第四部属下的柄泽班所占用。柄泽班负责主持细菌制造工厂。臭味就是来自这里。

731部队制造细菌是使用石井四郎亲自发明的细菌培养器(加压式培养器)来进行的。这是为了在短时间内大量培养和繁殖细菌的一种创造发明。其原理很简单。

细菌喜欢肉汁、糖分等营养丰富的东西。利用这些东西制造出无菌的培养基,这是利用琼胶和胨\footnote{胨(Peptone):是一种用来分解蛋白质的物质。——译者}繁殖细菌的基础,然后只要保持一定的温度和暗度,细菌就会迅速繁殖起来,并聚集在细菌培养基的表面上。通常细菌是看不到的微小生物,但是,经过多次繁殖后,最后琼胶上呈现一种糊状乳白色的薄层,这就是制造出的细菌群体。

第四部的柄泽班利用特殊的传送带自动回收使鼠疫菌、霍乱菌等繁殖的细菌培养基,再利用特殊的刮取器把细菌刮取下来,然后把琼胶培养基收集起来,放入高压灭菌器中。

再向细菌刮取完毕的琼胶培养基通入高压蒸气,使它的表面恢复无菌状态,以便再次用来制造细菌。

收入高压灭菌器后,细菌培养基散发出十分厉害的臭气,它像果酱变质后的臭味,充满了整个731部队总部大楼,随着风向的变化,有时也被吹到旁边的大礼堂去。

731部队还有一个“臭气来源”:那就是解剖室及其附近的焚尸炉。

每两天”消费“三名”马鲁太“,他们的尸体就在这里火化。731部队为了消除战争犯罪的一切痕迹,所有尸体都在焚尸炉中烧掉,剩下的骨头全部扔入叫”骨坟“的大坑里。焚尸炉的烟囱虽然很高,而且采用高温焚烧方式,排烟较少,但是,由于风向的关系仍然散发出恶臭味。“马鲁太”的尸体一般都是这样处理的。但是,也有例外情况。那就是1945年8月9日731部队撤退时,对“马鲁太”的处理并未能这样做。关于这件事将在后面加以叙述。

部队里有一个由特别班负责管理的动物饲养室。在这里,饲养着兔子、豚鼠、老鼠和跳蚤。

原部队人员一致强调,进行细菌战着眼于老鼠和跳蚤,表现了指挥者石井四郎的恶魔“天才”。

石井中将一直把鼠疫和霍乱两大菌种置于构成细菌战基本内容的”两大基本武器“的地位。大量制造鼠疫和霍乱两大”武器“的特殊系统,就是根据石井式细菌培养法制造的。

为了利用制造出来的细菌进行细菌战,需要有传染的媒体。

鼠疫最有效的媒体就是跳蚤。饲养大量的跳蚤,利用细菌将它们污染后,放到目的地去,短期间内就能够使鼠疫蔓延开来。

“石井这老头子和他的助手发现了通过将鼠疫菌寄生在跳蚤体内,使其在跳蚤的保护膜的保护下不断繁殖的理想的循环途径。这老头子恐怕是世界上最精通跳蚤的专家,也是鼠疫菌的研究家”。这是一个原731部队人员的证词。

为了大量获得被污染了的跳蚤,就必须确保有大量的啮齿类的动物(老鼠等)。

为了研究和实践如何在短期内繁殖老鼠,731部队纹尽了脑汁。

负责管理“马鲁太”的石井“特别班”,同时负责管理动物饲养室里的跳蚤和老鼠。

在动物饲养室里,有胖得像小狗那么大的豚鼠。人们一走近它,它就瞪大眼睛发出尖叫声。石井动物饲养室的老鼠,受到他们特别的爱护。

对731部队来说,“马鲁太”是比较容易补充的“材料”,而却不能让老鼠轻易地死掉,因为它们是重要的“武器”。

\section{出现幽灵的细菌工厂}

第四部柄泽班负责管理的“口”字楼的一层,这是凝聚着731部队的经验和技术精华的大规模“细菌制造工厂”。

制造细菌的工序,是绝密中的绝密。柄泽班以外的部队人员,除特别重要的事情以外,严禁进入“工厂”。

在一楼的后面,有一个镶有瓷砖的消毒槽和更衣室。柄泽班成员首先更衣,必须“洗澡”之后才能到办公室上班。在更衣室里脱掉身上的全部衣服,换上白色工作服,带上七、八层厚的纱布口罩和白色的帽子,以及一个从脖子直到脚尖的橡胶制的前围裙,再穿上高到膝盖下面的长胶靴子,还要带上橡皮手套和特制的眼镜才算装备完毕。然后,以这身打扮走进浴室。浴室里一个比较浅的浴池里放满了石炭酸溶液,他们哗啦哗啦地趟过消毒液才算结束。这时膝盖以下部分变成了无菌状态,这是一条消毒用的“小河”。

制造细菌的工序完全是、流水作业。走进“口”字楼一层,向左拐第一间就是细菌培养基室。在这里放着四台巨大的蒸汽锅和培养基。在蒸汽锅里把琼胶溶化以后,再放入培养基中,然后将它放入中央走廊右侧的高压锅里,高压锅的温度高达180°C-250°C,使溶化了的琼胶处于完全无菌状态。

然后,把经过杀菌后的培养基琼胶放进冷却室内冷冻,再把形成的琼胶培养基送进无菌室,在这里把被培养的细菌涂在琼胶上。
无菌室是一间约三十叠榻榻米大小的玻璃房间。部队人员进入无菌室之前,必须首先通过“灭菌室”。从这个7平方米的“灭菌室”的天花板上喷出雾状消毒液,给部队人员全身消毒,以防止被培养的细菌以外的东西附着在琼胶上。

向琼胶植菌,是使用一根长五十公分的像铅笔那样粗的“棉棍”。先在“棉棍”端上卷棉花,再让棉花沾满菌母,然后迅速地涂在琼胶片上,由于一次就得把菌母植上去而毫无浪费,所以这是一项需要相当熟练技术的劳动。

在无菌室作业的人员除戴上厚口罩之外,还需要戴上圆型的白帽和大眼镜,整个身体几乎都被包得严严的,所以分不清谁是谁。为了防止吸入菌母,作业时不说一句话,全靠手势来表达自己的意思。

由于这是一项繁重的劳动,作业时汗流浃背。旁边的人拿着纱布协助擦汗。柄泽班人员头戴白帽,身着白衣,前挂围裙,戴着眼镜默不作声地缓慢动作着的样子,具有一种异常扣人心弦的力量。

植菌结束后,培养基被运进培养室。培养室内全部墙壁都贴上了铜板。这是一间很大的房间,天花板上只吊着两个电灯,令人觉得好像是一间暗室。培养室的温度一般保持在20°C-80°C之间。操作门口的仪表,可以自由调节室温。根据培养的菌种不同有必要调整室温。不同的菌种,繁殖的时间也不一,有的一天就够了,有的则需要一个星期。在培养细菌期间,严禁开关门户。在适当的暗度和温度的条件下,细菌从琼胶中获得营养,于是在培养基的表面上形成一层粘糊状的白色乳液而持续增殖。柄泽班的人员认为时机成熟就开始进行刮菌作业。

刮菌作业是利用一个用硬铝制作的,长50公分,顶端有一个5公分至7公分的小竹刀的,被称作“刮棍”的工具来进行的。利用它把聚集在培养基上的细菌刮到一个直径约10公分、高约30公分的特殊玻璃容器里。

一个原731部队人员(柄泽班勤务)回忆说:“玻璃容器底下聚集的菌母,让人联想起甜酒糟来。”

柄泽班“工厂”里制造出鼠疫、伤寒、霍乱、赤病、破伤风、结核、炭疽、癞病等各种细菌。

把经过刮菌以后的培养基,再次放入高压锅中彻底杀菌之后,扔掉已经溶解的琼胶。至此,完成了制造细菌的一个周期。

如果杀菌后的琼胶还可以使用,再生之后,作为培养基再次使用。琼胶通常使用三次后就失去了再生的能力。

制造细菌是一项需要体力和精神集中的极其危险的劳动。有时作业时一不小心滑倒,就会从上到下沾满细菌。

无论多么小心,在制造过程中,变成空中浮游物的细菌,也难免不会进入口中。

由于这一缘故,柄泽班“工厂”的各个房间里堆满了红苹果,队员们进行的作业告一段落之后,就不停地啃苹果,然后再吐出来。这样,可以让苹果把嘴里的细菌吸收掉。

在制造细菌的过程中,柄泽班的许多成员倒下去,丧失了生命。伸向三个方向的无窗走廊里,灯光十分微弱,“口”字楼一层的“工厂”,即使在白天也是阴暗的,部队人员有人悄悄地说:“出现了幽灵……”

\section{充满怨恨的陈列品}

在731部队里,同关押“马鲁太”的特设监狱相并列,有一间“恐怖的房间”,除少数有关人员外,其他人严禁进入这一房间。

一间位于第一幢(总务部)二层的左端,是面积很大的“陈列室”;另一间是总部一层高桥班(负责研究鼠疫)左侧的“解剖室”。

关于“解剖室”以后再谈,在此先介绍一下“陈列室”的情景。

“陈列室”位于总务部对面的左端(参照要图)。顺便提一下,在哈巴罗夫斯克进行的远东军事法庭审判的“公审文件”中,把“总务部”误译为“庶务部”,这是一个错误。

虽然叫“总务部”,但是它并非一个单纯管理人事、会计等的事务部门。负责拍摄许多利用“马鲁太”进行实验情况的照片及16毫米电影片的“摄影班”也归总务部管辖。

还有一个“印刷班”,也归总务部管辖。这个班负责把利用“马鲁太”进行实验证明的结果等医学上的发明和各种研究资料编成小册子,或者以报告文件的形式进行印刷。这里集中了731部队所进行的各种活体实验的资料。

在731部队里,还有一个收藏进行细菌战作战所需要的大量地图的图库。

不仅仅是一般的地图,而且还有详细记载饮用水、江河、水井等细菌污染“战术目标”的地图。有许多对苏满边境或苏联境内、蒙古境内的军事设施经过彻底的研究后,详细标有搜集到的情报的地图和报告。这些资料都是由总务部兵要地志班编制的。

总务部还同宪兵室保持密切的联系,它担负着重要的防谍任务,严防731部队的罪恶勾当泄露到外部去。

从某种意义来说,总务部才是731部队的中枢神经。总务部集中掌握着有关731部队的一切信息。

问题是“陈列室”。它设于总务部二层的左端。虽说叫“室”,但它所占的面积相当于总务部内的庶务、会计和人事三个课所占面积的总和。经过总务部的走廊,就来到“陈列室”,一打开大门,一股福尔马林液的气味就钻进人们的鼻子,视神经受到刺激,使人睁不开眼睛。

原部队一个人员回忆说:“第一次看到陈列室的人,不禁会吓得瘫软,甚至会吓得坐在地下。”

沿着白墙排列着三排高约60公分、宽约40公分装满福尔马林液的玻璃容器以及三层玻璃拒。

在福尔马林液中,放着人头。泡在这些容器中的人头,有的睁着眼睛凝视着上方;有的头发被裹成一团,紧闭着眼睛;有的面部被破坏得像石榴;有的被刀剑从头部到耳朵后边劈成两半;有的被锯锯开露出脑浆;有的面部严重溃烂,分不清眼睛、鼻子和嘴巴;有的皮肤上生满了红色、蓝色和黑色的斑点,呆呆地张着嘴。人头的主人主要是中国人、蒙古人和苏联人。不同种族男女老少都有。这些泡在浅咖啡色的溶液中的头向着走进室内的人,充满怨恨地无声质问:“为什么把我放在这里!”在“陈列室”陈列的不仅是人头,还有从大腿部切下来的人腿,也有没有头颅而四肢弯曲的人体,肾脏、肠等卷成一团泡在液中。还有妇女的子宫和胎儿。人体的所有部分都被泡在大小不同的容器中。这是人体各部位的“陈列室”。一位731部队人员说:“尽管部队的领导人解释说这是从诺门坎事件的战场上取回的标本。但是没有一个人相信这一点。因为解剖‘马鲁太’的结果造成标本的数量一直在增加,添了不少新的标本。”

据说在这些陈列物中有一个奇怪的标本,那是一只从肘部切断的手腕。这个手腕的主人是731部队中的人员。据说,他每个月都到“陈列室”里去看一次自己的手腕,一直不厌其烦地看它。

这间房子并非一间单纯的“陈列室”,同时又是发表各种研究成果的“大厅”。在许多“人头”包围之下,发表研究成果的医生们的形象,比“人头”更令人恐惧。

\section{“安灵室”与“焚尸炉”}

“安灵室”设于总务部庶务课和企划课之间。室内正面设有一个祭坛,并放置有许多照片和灯。悬挂的照片主要是731部队的死亡人员。这些人是在研究、准备和实验细菌战的过程中死去的。部队人员走过“安灵室”前,一定要鞠躬行礼。原部队人员说:“虽然一般人不知道殉职了多少部队人员,仅据我所知,因患鼠疫而死去的人就有30多名。在制造危险的细菌和分离克次氏体的过程中,许多研究人员由于被细菌感染而住院。每当发生这种情况时,就由各班抽调人员组成新的小组,对各种设施反复进行消毒,而且给部队人员及家属注射预防疫苗。”

在研究细菌战的过程中,究竟死了多少人呢?过去一直没有公布,所以说不知道。但是,另一位原部队人员做出证言说:“应该说每年不下30人。”许多人都曾经亲眼看到:官员殉职宿舍的一角,被绳子围了起来,有严禁入内的字样。

在731部队里,有不少家属也参加了工作。部队人员定额为3000人。部队为经常缺额五六百人而苦恼。因此,一般军队职员的家属们当女职员在部队里工作。

在731部队的诊疗部,工作的护士多数是女职员,她们不允许进入“口”字楼。部队人员和家属有一个共同的义务:那就是向部队提出一份写明“自己死亡后,无论什么原因,都同意解剖自己的尸体”的誓约书。被进行遗体解剖的部队人员和家属的人数不详,但不在少数。原部队人员间流传着这种说法:

有一位美貌的女职员死亡,其死因不明。按誓约书的要求,对她的遗体进行了解剖。但是由于她是一个妙龄美女,参加解剖的队员在队内传开了她遗体的特征。据说,有关的消息让领导感到吃惊,对有关人员下令不准谈论。由于部队人员和家属等殉职人员增多,甚至要在部队内建立一座寺庙,作为部队的设施。

与此形成对照的是女“马鲁太”的待遇。在部队领导人中,悄悄地传阅过几十张关于女“马鲁太”的照片。这些照片都是摄影班拍摄的。拍摄的都是被脱去衣服的女“马鲁太”下半身的各种姿态,都是特写镜头的照片。干部淫笑着让部分部队人员观看这些照片。原中国女教师、苏联女学生和被捕的“马鲁太”的妻女们带着手铐,在无法抵抗的情况下在单间牢房里,首先由部队人员用照相机进行凌辱。

有一名女“马鲁太”在狱中生了孩子。在摄影班中把女“马鲁太”妊娠的对象究竟是什么人当成话题。这充满好奇心的话题立即传播开来。每当女“马鲁太”妊娠,几十张照片和猥亵的会话结合在一起,到处充满了淫靡的气氛。而另一方面则有如下的证词:

有一名妊娠的“马鲁太”被捕以后在狱中生了孩子。她为了救孩子一命,表示同意自己被用作任何“实验”。她流着眼泪,每天恳求看守人员救救这个孩子。只有部分部队人员知道这件事……但是在731部队看来,“马鲁太”的母亲只会生“马鲁太仔”。由于“马鲁太”是材料,那么,她的孩子也只不过像饲养老鼠一样地被饲养着而已。不消说,母子都遭杀害了……

部队人员死后,准备了“安灵室”。可是等待女“马鲁太”母子的却仅仅是印上编号的卡片和焚尸炉。

\section{“特别移送”的内幕}

“我们在日本占领的满洲领域内从未感到供应石井部队实验用的人体的不足……每年都有600人左右是以‘特别移送’的名义送到我们那里。”这是731部队第四部(制造细菌)的负责人川岛后来在哈巴罗夫斯克进行的远东军事法庭中作出的供词。

“马鲁太”是在抗日战争中作为特嫌被宣布死刑的人,是“反正要杀掉的家伙”。这就是731部队大规模进行人体实验的“合理的”口实。

在满洲建设王道乐土,建立五族协和(蒙古人、满洲人、汉人、朝鲜人和日本人)的理想国家——这种宣传蒙蔽了绝大部分日本国民,军部本身也陶醉于这种歪理之中。然而,无论日本军部怎样任意制造出种种歪理,但是,在中国人看来,日军侵入他们的国家,用武力扩大占领地区,只能说是侵略者。

为反抗异民族的武装侵略而斗争,对这个国家的国民来说是理所当然的事情。把充满爱国心和民族独立观念的抗日游击队员、中国工人、农民以及学生们,当做“马鲁太”来对待,这当然是没有道理的。

然而,当时在日本国民中这种理所当然的正当逻辑也完全被压制下去。凡是持有这种看法的日本人就被视为“非国民”,是“国贼”。

原部队人员说:“由于我深信:我们进行的作战,是使贫穷的日本富裕起来,而且有助于亚洲的和平,因此,也就认为‘马鲁太’等不是人,他们的存在成了连畜生也不如了……派往731部队的研究人员和学者,可以说没有一个人同情‘马魯太’,而731部队的全体军队职员和军人认为杀掉‘马鲁太’是理所当然的。”

但是,在731部队中却也关押着许多同反日抗战毫无关系的“马鲁太”。

哈尔滨原来是俄国人建设起来的一座城市,具有亲苏反日情绪的市民比较多。哈尔滨宪兵队和特务机关人员(配置在警察中的关东军间谍学校“绿阴学院”毕业生等)认为是“反日分子”者,不论有无实际行动,把他们统统抓起来,予以“特别移送处理”。

远东军事法庭审判记录中也记载了女“马鲁太”在731部队的特设监狱中生孩子的事,但是也有没有提到的事实。

女“马鲁太”生孩子,并非仅仅一次。据有关人员说,有数名女“马鲁太”在特设监狱里生了孩子。

当女“马鲁太”生孩子的消息在731部队里传开后,部分人员开玩笑地说:“该不是通过人工授精而生的孩子吧!”然而,这是发生在集中了现代医学精华的“关东军防疫给水部总部”内的事情。这里是否曾经进行过人工授精的研究,是个疑问。在部分人中产生这种传闻,并非没有道理。

但是,没有进行人工授精的研究,似乎是事实。那么,女“马鲁太”怎么会生孩子呢?有两种可能:一种可能是逮捕了怀孕期的俄国妇女、中国妇女,并作为“马鲁太”关押着;另一种可能是女“马鲁太”在狱中怀孕而生的孩子。关于这一真相是被封锁的。

据说,女“马鲁太”全部都作为治疗性病感染实验材料来使用。部队中有数名军人和职员,战后在东海地方开设妇产科医院。当年在731部队时,队员们看到他们到口字楼去时就议论说:“那家伙又去看女‘马鲁太’了!”虽然仅此一次,但部队内的《会报》曾刊载“公告”说:“该人使妇女发生妊娠,予以免职惩戒。”

\section{作者的插话——写作过程中的留言}

在执笔写这部纪实的过程中,我收到了许多来信。

许多原731部队人员向我提供了新的情况。其间,让我觉得好像周围全都是关东军防疫给水部总部的人员似的。

在本文中出现笔者,有些冒昧,但是,由于反应过分强烈,其中也有一些人诚心诚意地提出了责备和批评,所以,我想在这里写封回信。

有人希望看看“关东军防疫给水部总部满洲731部队要图”的原图。

关于这一问题,我在前面巳经提到,无论战前战后,这张要图是首次公布。关于731部队的设施,迄今,在《特殊部队731》(三一书店出版)一书中,作者秋山浩提供的“图”是惟一的线索。那张图可以说是依靠记忆,一部分是靠想像补充的,是非常不准确的,但是,在战后36年的漫长岁月中一直通用,并为各种书刊所引用,至今仍认为是准确的,到处通行。在这种情况的后面有这样一种现象:原部队人员通过沉默严守秘密。

但是,另一方面却有一张由原部队人员悄悄保存的“绝密要图”。“要图”毕竟是“要图”,而并非“全图”。但是,根据后来我的调查,判明这张图大致是准确的。经过对持有“要图”的人一再进行说服,终于在本书中把它公布于世。

遗憾的是原图不能原封不动地公布,这是由于制作者的笔迹和制图的痕迹一清二楚的缘故。

原731部队人员的纪律仍然很严格。战争结束后他们曾发誓“要共同背着731部队的秘密走进坟墓”。公布原图的结果,不知会给提供人带来什么样的麻烦。因此,我在此郑重表示拒绝。

关于首次公布的731部队各班(课题研究组)的研究内容和负责人的姓名问题,有人提出“为什么不公布全名?”关于这一问题,我有两个理由:

第一个理由是,各课题研究组的“班长”都是当时日本医学界具有代表性的学者和研究人员,虽然有的人已经逝世,但是,不少人目前正活跃于医学界的第一线,其社会影响很大。我的意图是写出填补历史空白的满洲731部队的准确纪实,并非追究个人的责任。

第二个理由同第一个理由有所关连,部队人员的经历,并非是同当时的日本人完全脱离的特殊经历。日军侵略中国东北之后,许多“开拓”满洲的农民、商人、学者、工人及作家接踵而至。他们只听说“可以雄飞大陆”,心想在这一名义下到了满洲总会有办法的。但是结果却是无法维持生计,成了满洲游民。当时日本的国家舆论认为满洲是日本生存下去的生命线,几乎所有的日本人把中国的领土满洲错误地认为似乎是本国的领土一样,而毫无侵略的意识。与其说是继承俄国人的侵略,掠夺中国人的土地和财产,毋宁说是一种旨在保卫日本的正当行为。这巳成为当时的一种时代潮流。批判这种潮流的政党和个人,以违反治安维持法的名义,作为“国贼”而遭到当局的残酷镇压,被关进监狱,拷打致死。

在分配到731部队的学者和研究人员中,有许多人由于对数百名“马鲁太”进行活体解剖练得了“本事”,在战后医学界取得了地位。731部队的各种活体实验是作为集体的日本人根据组织命令进行的。

还有人问:在课题研究组中,“田部班”是否“田部井班”之误。这并非弄错。731部队中,有一直从事伤寒研究的田部中校和曾任第一部负责人的田部井部长两个人。如果写出全名,就很清楚,但是根据上述理由,我回避了。

有人对纪实中多次出现的地名提出了批评。他们说为什么把HARUBIN(哈尔滨)写成HARUPIN呢;五屯和四屯等地名一旁注上日语字母的七号小铅字是汉语和日语低俗的掺和;笔者对中国的地名几乎是无知的,等等。

关于这一点,我在执笔前进行了各种调查,结果,“决定作为纪实采用当时731部队人员一直使用的称呼”。对战后的日本人来说,是HARUBIN,但是在之前,HARUPIN是街道名称。有的意思说,既然把黑龙江写成了阿穆尔河,那么,松花江也应写成SUNGARI。从地名辞典来看,这一意见是正确的,但是对当时住在哈尔滨市的许多日本人来说,SUNGARI还是“松花江”,对731部队人员来说,五屯还是UTON。

从这一点来说,731部队公布的支部名中林口、孙吴和牡丹江等注上了日语读法,但海拉尔并未如此注明。很难统一按国语(日本)注上日语字母的七号小铅字。

同样,例如在下面即将出现的诺门坎、三不管、傅家甸等地名或地区名称,有的按日语读法,有的按俄语读法,有的按汉语读法。统一按汉字(日语)读音记载是困难的。

另外,关于满洲的插图,按照原部队人员提供的原图那样,使用了“满洲”和“满洲囯”,其他地方则使用了“满州”和“满州国”。

\chapter{残酷的大检阅——让人产生梦魇的标本}

\section{“马鲁太”的用途}

如上所述,731部队里有“研究病理”(冈本、石川班)、“研究药理”(草味班)、“研究冻伤”(吉村班)等课题的研究组。

问题在于这种课题研究组的内幕。

据原部队人员作出的证词,例如“研究药理”的研究课题之一就是研究各种药物(毒物)对人体的影响以及解毒的方法。
无论在过去,还是今天,医学界研究剧毒药物对生物的危害,最多只使用植物或小动物来做实验材料。

可是731部队却使用了“马鲁太”。送入特设监狱的“马鲁太”,多数是因反日抗战而被捕的俄国人和中国人,而且是经审讯、拷打以后,巳经被决定要被枪毙或处以绞刑的人。不执行死刑,而强使他们遭受成为“马鲁太”的命运,也就是在活体实验的名义下执行死刑。

读者也许还记得战争结束不久的1948年1月26日曾发生过一个“帝银事件”:在东京丰岛区长崎一丁目33番地的帝国银行推名町支店里,有16名银行职员在一个自称“来自东京都防疫部门的人员”逼迫下服毒,其中12人当即死亡,16.4万日元现金和面额为1.7万日元的支票被抢走。当时的一名幸存者说,制造这一事件的犯人是一名“年约四十五六岁,身高五尺二寸左右,瘦小的光头。作为医生来说,罪犯又显得有些粗鲁。”

这一事件具有几个特点:

首先,犯罪使用的毒物是“氰酸化合物”(而不是氰酸鉀),罪犯使用的吸管(玻璃制的医疗用具,用以正确测定剧毒等液量)以及医药箱都是旧日军部分研究所进行细菌实验用的特殊用具。

其次,从犯罪前后的行为看,罪犯对“氰酸化合物”及它的“解毒药”似乎有着丰富的知识。

第三,据判断,罪犯在灌毒药的方法方面有非同一般的经验。

过去几位学者都指出,这里所谓旧日军部分研究所,就是指满洲731部队。

这次我所接触的原部队人员一致做出证词说:“帝银事件发生后,马上有刑事警察来查问过。”“我被叫到警察署去,让我看模拟照片”,他们问我“你知道名叫XX的人吗?”有不少原部队人员被拘留了很长的时间。帝银事件的真正罪犯究竟是不是战后33年的漫长岁月中一直在狱中呻吟的平泽贞通呢?关于这一点,我没有具体的材料,也没有说明的义务。

我想说明的是许多搜查人员从帝银事件中使用过的毒物、吸管、罪犯的行动以及年龄和容貌等各方面,对原部队人员一个个地进行调查的这一事实。当时有一个受到刑事警察盘问的原部队人员(中尉)说,他们一见面,刑事警察们就说:“噢,真像,和帝银事件的罪犯长得一模一样。”

原部队队员聚在一起时,至今还在说:“制造帝银事件的真正罪犯并非平泽。无论从手法或使用的毒物来看,肯定是731部队的家伙。”(战后原部队人员让部队长石井四郎任顾问,有时悄悄地召集秘密聚会,这一事实将在后面加以介绍)

搜查当局以原731部队的“研究药理”班人员为中心,一个个地对原部队人员进行调查,自有一定的道理。

731部队为了研究杀人用的氰酸化合物及其解毒药,使用了许多“马鲁太”。

并非单纯研究毒药和制造解毒药品,而是为了确认制造出来的毒药的效果。他们以“马鲁太”为对象,详细地研究了它的最低致死量以及旨在谋杀的“灌毒方法”。

\section{让“绘图兵”来报到}

让我写一下731部队对关押中的“马鲁太”进行残酷实验的一个小故事吧!

731部队里有一个擅长于画日本画的人,他出生于石川县金泽市,原来是一个描绘加贺友禅①底图的画师。

①“加贺友掸”是日本金泽地区的一种著名的染色的方法,是日本传统工艺“友掸染”中的一个流浓。利用这种方法进行和服衣料染花时,首先要用毛笔在农料上画底图,而后才进入其他工序。——译者

在战时“奢侈就是敌人”的口号流行的时代,一些名流妇女特地将做礼服用的和服袖子剪掉,然后报纸大量予以报道。订做友禅染和服的人显著减少。画师们逐渐失业或觉得前途无望而转业。

画师就请求军队给自己安排一个能够发挥绘画才能的职业。经过军队文职职员的考试,他被录取后分配到满洲731部队。那是1942年1月的事儿。他到总部报到任职时,那里的人在了解他的履历以后问道:“能制图吗?”

当时部队绘制的地图、建筑物略图等许多图纸都是用钢笔画成的。可是,他一直描绘加贺友禅图,却从未拿过钢笔进行工作。画师描绘加贺友禅底画的本事,是利用几支细毛笔描成图案。

“我虽然没有用过钢笔和鸭嘴笔,但是只要有细毛笔,总能设法描出图来!”他回答道。

人事部门的人听后十分感兴趣,说这倒很有趣,画画看吧!于是拿出一张地图放在他的面前,看看他能否用细毛笔照样画出来。
他在一张很结实的日本纸上仔细地挥动着细毛笔开始绘制地图。绘制道路、河流的细微变化,标出散布在旱田和丘陵中间的村子等,确是一件十分复杂的作业,但是运用绘友禅底图的纤细技术,并不难完成。他手中的细毛笔尖自由自在地描绘出了线条又画制了一幅地图。地图和文字约花了三个多小时。部队里有专门编制地图的人。当时在兵要地志班的部门里有一名叫做T的能干的绘图员。他画的地图,被送到兵要地志班T绘图员那里。绘图员看到这幅仅使用细毛笔,而没有使用其他制图辅助工具而巧妙地画成的地图以后,感到非常惊讶:“唔……我还是第一次看到用毛笔画成的地图,真不错!”

据说,T拿着那张地图到处给别人看。一位有绘画才能、技术高超的人参加了731部队,这个消息不久传到了部队领导人的耳中。
他逐渐习惯了731部队的工作。一天,他接到了领导的传呼。命令的内容是“拿着画笔、绘画工具到吉村班报到”。

吉村班是一个主要研究冻伤的课题研究组。该班的冻伤实脸主要在冬季(11月-翌年3月)进行。哈尔滨的冬天是很冷的,夜间气温常常降至零下40度以下。晚上十一二点,吉村班把“马鲁太”带到特别处置室,把他们的手脚泡在装满冷水的桶里,人为地制造冻伤。

把手脚泡过水的“马鲁太”带到户外,让他们暴露在零度以下的室外,经过一定的时间,造成冻伤。他们估计已经发生了冻伤,就把“马鲁太”带回室内,进行“治疗”。当时,在侵略中国东北地方的关东军士兵中,不少人由于在极寒地带作战而患冻疮。731部队为了研究冻伤原理和改进治疗方法,一直在加紧搜集有关极寒条件下细菌感染途径等资料。这位画师带着绘画工具去报到是白天,几个小时以后,他面色铁青,带着一幅十分憔悴的面容回到了自己所属的某班,他的腋下夹着十几张已经画好了的画。他坐下来,在桌子上打开了画。这是极为残酷的彩色画。任何人看后都会情不自禁把头扭过去,不忍看下去。

\section{邪道的色彩}

这是些十分残酷而又令人讨厌的画。不,与其说是画,不如说是在用墨画的素描之上,又用作日本画所用的工具加上淡淡色彩的写生。画的是“马鲁太”严重冻伤的手脚。

在严寒中,只要把冷水泡过的手脚一放在室外,眼看着就会发生冻伤。室外气温-40°C,“马鲁太”外露的皮肤开始时由于贫血,变白,接着由于淤血由红变成紫色,在肿胀的皮肤上产生水泡,然后破裂发生溃烂,变成紫黑色,肌肉组织坏死。冷水浸泡过的皮肤在极寒条件下外露,在极短的时间内,就会从轻度冻伤发展成完全冻伤。

吉村班人员为了确认是否巳经完全冻伤,用四棱木棍毒打“马鲁太”的手脚。打下去,如果还有“痛”的感觉,那么,就证明尚未完全冻伤。完全冻伤是坏疽性冻伤。病患部位变成坏疽,肌肉组织全部坏死,皮肤呈暗黑色,并脱落开来。吉村班确认已经完全冻伤之后,把“马魯太”带进屋内。

这时,再给巳冻伤的“马鲁太”进行治疗。把他们的四肢放在低温水中浸泡,逐步提高水温。有时,突然把他们的四肢放进15°C的温水中,看“马鲁太”的冻伤会有什么反应;有时,在保持水温不变情况下,观察轻度冻伤与严重冻伤的患部又都会怎样,研究治疗时水温与冻伤程度之间的关系。他们变换着外部条件,彻底地进行实验。

“吉村班实验的目的并非给‘马鲁太’进行治疗,而是为了搜集用什么方法才能准确地测定皮肤表面的温度、细胞变成坏死状态所需的时间以及整个过程等有关资料。……把‘马魯太’的手脚泡在温水中,也是为了寻找治疗冻伤最适当的水温以及泡入温水中的时间幅度等。……但是,一旦‘马魯太’完全冻伤的四肢突然泡入热水之中,那部分的整个肌肉组织就会一下子全部脱落掉……露出白骨来,尔后只能锯掉四肢,否则,‘马鲁太’是无法活命的”,一个原部队人员这样说道。

绘画师描绘下来的正是这些“马魯太”变了形、坏死了的四肢。

有一张画上画着“马鲁太”的双手,从指关节到指尖的肉巳经完全烂掉。另一张画上描绘的是“马鲁太”的两只脚,巳没有踝骨以下部分。还有一张画上,从脚膀子到大腿都露出了白骨。还有的画上画的四肢像海豹一样短。发生水泡时冻伤的画,描绘着肌肉组织逐步变成紫黑色过程。他根据命令描绘的彩色画,就是冻伤实验的记录。当时,吉村班对利用“马鲁太”进行的冻伤实验都细致地拍了照片,收入了黑白胶卷中,还拍了记录影片。但是,这些实验记录中惟一的缺点是“没有颜色”。当时也没有研究出彩色胶卷,拍摄的照片,除了事后着色之外,没有办法制作出“有颜色的照片来”。所以,吉村班才看中了他的绘画才能。他运用描绘友禅染底图锻炼出来的技术和眼光,在短时间内就可以完成对“马鲁太”的某些部分的写生,

并再现在画册上,使之成为一张彩色画。它是彩色胶卷的代用品。过去用来描绘美丽的友禅染底图——设计装饰姑娘们的和服长袖盛装的设计技术,现在被用来记录残酷的实验。他的内心是否痛苦,对吉村班来说是无所谓的,他们需要的是原封不动地再现那些腐烂了的“马鲁太”手脚的彩色画,需要的是描绘那些东西的绘画天才。

从那以后,他多次接到“带着绘画工具前来报到”的传呼。同事们看到,每次他离开时面部表情总是变得慌慌张张。他变得沉默起来。不久,传呼他的不仅是吉村班了。

他陷入了不得不描绘使用“马鲁太”进行的各种细菌实验结果的处境。对他提出的要求是局部彩色写生。他很想知道那个“局部”究竟进行了什么样的实验?在近距离写生是否会有什么感染的危险?但是他没有得到任何说明。他在战争结束复员之后,从未参加过原部队人员的聚会。每当他得知举行这种聚会时,他就嘟哝着:“真是多余……在731部队里留下的都是令人讨厌的记忆,没有任何一件事值得原部队人员聚集在一起进行回忆的。”旧战友中知道他住处的人也极少。

吉村班当时的负责人吉村寿人于1982年11月4日在接受《每日新闻》采访时做了如下的回答:

问:据揭露,你们利用冷冻装置进行过冻伤的活体实验。

答:当时还没有那种装置。为了推行对苏战略,曾研究防寒用具能否经受住零下70°C低温的问题。为此,制造了冷冻装置。由于器材来的太晚,快到战争结束时才开始试用。可是,由于苏联参战而把它炸毁了。恐怕你们是从少年兵那里听到这些话的吧!
问:听说你们按民族分别进行过预防冻伤研究。

答:把中指放在冰水里研究其反应的方法称作摆动反应(Hunting)。这是今天也使用的一种方法。当时根据关东军的调查,不到-4°C,不会发生冻伤。因此,我们在0°C的条件下进行实验。而且并非利用“马鲁太”来进行的,而是在当地人的合作下进行研究的,并非活体实验。我尽量不接近看管“马鲁太”的特别班。后来我也进行过冻伤治疗研究,是让我的部下军医中尉进行的。虽说他提出过报告,但是我没怎么过问。他干了些什么,我不太知道。

问:利用婴儿进行过预防冻伤研究是事实吗?

答:1972年,这件事在学术会议上成为问题,曾对一起进行研究的技师(1973年去世)进行过质询。他写信回答说:“使用了自己的婴儿。”至今我还保留着这封信。当时认为生命轻如鸿毛,能够帮助军队是一种荣誉。利用自己的孩子做实验,并不成为问题。詹纳(Jenner)①首次种痘不也是在自己儿子身上进行的吗?
①詹纳系英国医生,种痘的创始人。

问:你为什么进入石井部队?

答:大学的恩师说:“你去满洲吧!”我曾表示拒绝。但他说,如果不去就开除你。

问:虽说你本身没有进行活体实验,但你不是仍负有对部下的监督责任吗?

答:也许负有监督责任。但是,既然已参加军队,那是没办法的事情。

但是,就是这位吉村曾就自己进行冻伤实验问题发表过学术论文。关于这件事,追查731部队干部战后足迹的高杉晋吾先生写的《追查石井细菌部队残存干部》论文中有详细的记载。让我们从中引用部分的内容。

这篇论文是吉村寿人战后发表在日本生理学会的英文刊物《日本生理学季刊》上的,是一篇关于寒冷生理学的论文。

正像前面提到过的渡边在M报纸发表的那样,吉村对别的民族(包括中国人、满洲的蒙古人和鄂伦春人)进行过冷冻实验,同样也对100多名旅满日本学生(18岁至28岁)和中国苦力(对劳工的蔑称)进行过活体实验。为对不同年龄层的差别进行调查,对7岁至14岁的中国小学生也进行过活体实验。

更令人感到震惊的是,他还进行过把出生第三天、一个月、和六个月的婴儿的中指放进冷冻水中浸泡30分钟的活体实验。

读者一看就明白,这些实验并非志愿者乐意进行的。出生第三天的婴儿绝不可能乐意把自己的手指提供给他们做冷冻实验。更何况一般家庭的父母是不会乐意把出生才三天的婴儿供做冷冻实验的。把手指浸泡在冷水中30分钟,婴儿一定会声嘶力竭地哭叫。
由于这种情况,只有在密室中专门修造一个能够强制婴儿进行实验的设施,才有可能。当然这种实验只有在731部队的设施里才能进行。

对我来说,这种情况虽是十分凄惨的,但是那些接受这种论文的日本学者的感觉才是令人难以想像的。

在哈巴罗夫斯克军事审判记录中,有许多证人证实了吉村进行的冻伤研究。现将其中部分证词引用如下。

首先引用原731部队教习生古都证人的供述:

问:731部队内是否进行过冻伤实验?

答:是的。我见过这种实验。

问:这种实验是在哪位研究员指导下进行的?

答:吉村研究员。

问:你谈一下活体冷冻实验的有关情况。

答:每年在全年最寒冷的月份——11月、12月、1月和2月,在部队内进行。这种实验的方法如下:在夜晚11时左右把被实验者带到极寒的户外,让他们把双手放入装有冷水的桶内,然后把手拿出来,让他们双手湿淋淋地长时间站在寒风中。有时也这样做:虽然穿着衣服,但却赤着脚,然后带到户外,在夜间最冷的时候,让他站在寒风中。

这些人冻伤以后,再带进室内,让他们把脚放进5°C左右的温水中,然后逐步升高水温。就是利用这种方法研究冻伤的治疗方法。我没有直接看到被实验者被带进室内以后的实验情况。我值夜班时,只看到把他们带到户外进行冻伤时的情景。让他们把双手放到水中进行室内的实验,是听目睹者讲的。(以下从略)

下面是原731部队教育部长西中佐的供述:

问:关于部队内进行过冻伤实验,你知道些什么?

答:听吉村研究员说,在严寒的条件下,把他们从监狱中带出来,空手站立,然后用人造风(电扇)吹,使手冻伤。然后用一支小棍不断敲打冻伤的手,直到能够听到像敲打小板时的声音为止。

下面是原731部队内宪兵的供述:

“(前略)我经过监狱实验室时,看到有5名中国人坐在长椅子上。他们中有两人没有手指,双手都变黑了,有三个人的手已露出了白骨。虽说还有手指,但只剩下骨头了。听吉村说,这些都是对他们进行冻伤实验的结果。(以下从略)

\section{绝望的解剖室}

731部队“口”字楼内设有一条秘密的地下道。这条地下道同关押“马鲁太”的特设监狱7号楼和8号楼相连,由“口”字楼一层西北角开始,向前直走,向左拐,沿着一个没有扶手的混凝土楼梯走下去,就是它的“入口”。

没有扶手的楼梯往下走,右拐弯约走半分钟,有地下通道,再顺混凝土楼梯往上走,这上行的楼梯也没有扶手。在楼梯口有一个向外开的铁门。这里就是地下道的“出口”。

“出口”是一间混凝土的大房间。高高的天花板上吊着一个特大的聚集型照明灯(相当于今天的无影灯)。下面设有一个铁制的手术台,乍一看,似乎像大学附属医院里的一间手术室,但是不同的是,除铁床(手术台)之外,找不到一件类似医疗器械的东西,而仅仅放着几个水桶和装有福尔马林液的供装标本用的大型玻璃容器。

这里就是731部队的解剖室。这座解剖室仅仅在靠近天花板的墙壁上开有一个采光用的小窗,它设于距“口”字楼的各个研究室——课题研究组最近的地方。

打开通往解剖室的大门,长廊的一侧就是负责研究鼠疫的高桥班的各个研究室。沿这条长廊走,一会儿就可以走到同纵贯“口”字楼内的中央走廊交叉的地方。交叉部分很陡,整个尽头部分呈逆坡状态,好像刨掉了一块似的(像山上的陡壁似地向前突出起来)。

这种设有逆坡的走廊结构,是731部队独特的设计,部队里的人员称此为“鼠回头”。

所谓“鼠回头”是怎么一回事呢?前面已经谈到,“口”字楼一层沿着长方形走廊的整个层就是第四部属下柄泽班的细菌工厂。这个工厂在高桥班——繁殖利用鼠疫菌污染的鼠疫跳蚤的地方附近。

繁殖鼠疫跳蚤,需要使用大量的老鼠。首先给老鼠注射鼠疫菌,将一二只老鼠固定在石油罐中,使之处于无法活动的状态,然后放入跳蚤,让它吸老鼠身上的血。直到剩下骨头为止,同时进行繁殖。

跳蚤吸了已经染有鼠疫菌的老鼠血,保持老鼠的体温,在黑暗中,跳蚤就会拼命地繁殖。
731部队中约有4500个这种饲育鼠疫跳蚤的器具,在两个多月中,可以“制造”几十公斤的鼠疫跳蚤,不是一千或一万只,而是几十公斤的鼠疫跳蚤

假定是50公斤跳蚤的话,据专家的计算,其数量就会有几千万只之多。这是一套在两个多月内就能生产大量鼠疫跳蚤的设备。
写到这里,修建高桥班各研究室前的走廊尽头的“鼠回头”,其目的就不言而喻了。万一发生事故,这些经过鼠疫菌注射的研究用老鼠,从高桥班研究室里逃出来,即使穿过走廊,也必定会被逆坡的陡壁所阻拦,而无法再向前逃走。“鼠回头”的名字也就是由此而来的。

从解剖室的大门走出来,一直向前走,走到同中央走廊交叉处“向左拐,走到最深处就是笠原班(研究病毒)研究室。从解剖室走到那里仅需几分钟的时间。

如果按相反的方向向右拐,沿楼梯向上走,就来到”口“字楼二层。在二层楼上,是拥有功率极大的冷冻室和保温室的吉村班研究室,以及凑班(研究霍乱),冈本班、石川班(均研究病理)。还有江岛班(研究赤痢)、太田班(研究炭疽)以及内海班(研究血清)。

从解剖室到二楼的这些研究室步行只需几分钟的时间。

解剖室通过秘密的地下通道同关押”马鲁太“的特设监狱相连,而且距各研究室很近。

设有铁制手术台的这间房子,就是731部队对”马鲁太“秘密进行活体解剖的地方。

据说从1939年731部队进驻平房附近的新设军事区后直到1945年夏季这支部队垮台为止,六年期间,通过这个秘密地下道送入解剖室的活”马鲁太“就有数百名。战后,驻日盟军总司令部调查时,高级队员冈本供称1945年解剖的”马鲁太“约在1000人以下。

\section{来自恶魔的预约}

哈巴罗夫斯克远东军事法庭未能弄清731部队所进行的无数活体解剖(据部分原部队人员说,刚死的人称“战死前解剖”——从法律上说,死体还有体温,不认为是尸体,解剖这种尸体,不是尸体解剖,而是称活体解剖)的真相。
这是由于受审的部队人员小心翼翼地掩盖着事实,包庇他们头目的缘故。
从哈巴罗夫斯克审判的记录中,到处可以看到被告人为隐蔽或缩小事实而费尽心机的措词。也就是说,被告人敏感地察知了苏军当局的审讯意图和审讯的锋芒,然后在回答审讯中反复地供述已经暴露出来的犯罪行为,始终采取了一种“没有问到的事情,一句也不讲”的态度。
因此,在哈巴罗夫斯克所记录的“公审文件”本身,虽然是日本陆军进行细菌战犯罪行为的详细记录,但是却又会产生这样一种危险:即把731部队的全部所作所为只限定于“公审文件”上所列的事实来认识。
资料、文件以及出版文件,都是利用铅字印刷的,从这些东西中,读者容易产生对铅字的“迷信”,但是为了揭露731部队的真相,仅靠涉猎印刷资料或硬搬已经提到了的事实,是不充分的。依靠那么一点证据就会陷入一种玄学式的表面调查研究之中。那只不过是现成资料的堆积而巳。继承、扩大现成资料的错误,也是有危险的。
山田清三郎写的《细菌战军事审判》和岛村撰写的《三千人的活体实验》,作为记录来说,是很不错的,因为他们虽然根据的是哈巴罗夫斯克军事审判的“公审记录”,但还通过笔者亲自的调查进行了证实。
上面提到的“口”字楼二、三层的各个班都利用设于部队一角的解剖室。
我在前面谈到“马鲁太”时曾写道:“马鲁太”是“按编号……作为731部队各班所有物分配的,成了他们根据不同研究目的进行活体实验的”材料“。
“马鲁太”为什么必须“作为各班的所有物分配”呢?其最大的原因之一就在于解剖室里的活体解剖。
从活着的人身上釆集新鲜的“标本”时,必须事先确定这些标本是哪个班的“所有物”。
据原部队人员的证词,在实际进行活体解剖时,解剖“马鲁太”人体的执刀和进行实验的权利,属于拥有这个“马鲁太”的那个班。执刀解剖和进行实验完毕以后,人体的内脏,根据各研究班的要求进行分配。

他们事先把解剖活体和进行实验的计划通知各个研究班。“解剖以后把小肠和胰交A班”,“B班要脑子”,“C班要心脏”……从那时起就开始进行预约。这是对被活生生地解剖了的人体部件的“预先订货”。
731部队进行活体解剖,大致有两个目的:
第一个目的是釆集标本。人患传染病时,心脏是否会肥大?肝脏是否会变色?感染各个时期的变化情况如何?在人活着的时候,查明各个部分的变化情况,解剖活体是最“理想的”方法。
并不只是采集感染疾病的标本,活体解剖的价值在于可以研究“马鲁太”服用一种药物后,随着时间的推移,与此有关的内脏发生的各种变化。
为了达到这一目的,给“马鲁太”“注射”了人们想得到相关结果的一切物质。从“马鲁太”的静脉注射进空气,观察身体的各种器官是经过一个什么样的过程才窒息的?部队人员虽然知道注射空气会导致人的死亡,但是他们对更加详细的经过抱有兴趣。
他们还把“马鲁太”倒吊起来,进行实验,看看多少小时多少分钟死亡,身体各个部分会发生什么样的变化?或者把“马鲁太”放入一个巨大的离心分离器内反复进行高速旋转实验,直到“马鲁太”死亡为止。
把尿、马血注入肾脏,人的身体会发生什么反应呢?他们用猴血、马血和人血进行交替的实验。究竟能从“马鲁太”身上抽出多少血液呢?他们利用针管进行过多次抽血实验。这是一种不折不扣的榨取。
把大量的烟送入人肺中会怎么样?如果以毒气来代替烟,又会怎么样?毒气或糜烂性气体进入人的胃,会出现何种变化呢?
试用这种药物,不,使用那种物质时……那些平时甚至连念头本身都认为是一种邪念而禁忌的事情,在731部队里却满不在乎地付诸实施。利用X光射线长时间照射破坏肝脏,也是活体实验的内容之一。据说,也包括一些在医学界是早巳判明的毫无意义的实验。
解剖活体的手术刀,主要由研究班里具有助手资格的人(雇员)掌握……采集标本的想法,由班长一级提出。各班的班长是当时著名的学者或医生。只对特别有兴趣的“马鲁太”,他们才直接动手。通常绝不玷污自己的手。一切事情都让部下来干。各班人员对活体解剖丝毫没有罪恶感,毋宁说各班里充满了一种这次能够采集到何种标本的期待气氛。
这是原部队人员做出的证词。经过全身麻醉或局部麻醉的“马鲁太”,一小时以后已经变成了一个“最好的活标本”。

\section{手术的“自助餐”}

据原部队人员说,在这种解剖活体“实验”中,他们曾经接受过哈尔滨医科大学日本人教授和当时满洲国首都新京(今长春)大学的“委托研究”。
根据研究课题的需要,虽然是只有很少几次,但是,大学教授也曾来过731部队。每次都戒备森严,下车时都是把教授的眼睛严严地蒙起来,进入大楼以后才取掉蒙眼布。
有一次,前往哈尔滨外游的“皇族人士”悄悄地来到了731部队,石井四郎部队长根据“无关东军司令官的许可,严禁任何人入内”的规定,让这位“皇族人士”在大门外等了很久,说了一番好话之后,才领他参观了设施。这段小插曲在有关人员中是人人皆知的。
石井四郎队长(军医中将)认为解剖活体是一种具有吸引力的“实验”,并把它作为对此有兴趣的日本医学人员参加部队的“诱饵”,不少教授既是731部队人员,又在当时的哈尔滨医科大学里教书。
“某有名国立大学的教授在战后日本医学上做过许多疑难的手术,博得高名,获得了政府的勋章……这位先生怎么掌握如此高明的外科手术技术的呢?如果那么难的手术遭到失败是不得了的!难道那位先生没有失败过吗?不,他的手术有过几十次失败的经验,……他在什么地方积累这些经验的呢?都是在731部队!”
在关西,我见到了原部队的一个队员,他说,以“马鲁太”为对象进行过多次疑难手术的“实验”。
“马鲁太”——是人,又并非是人。因为每个“马鲁太”都没有姓名,只有编号的管理卡片。当“马鲁太”被“消费”掉之后,就把他的编号改用在新“进货”的“马鲁太”身上。
但是,被731部队进行活体解剖的人,并非都是“反日分子”。现在让我们介绍一个原部队人员当时目睹的一个实例吧!
1943年的某一天,他们把一个中国的少年带进了解剖室。据原部队人员说,这个少年并非“马鲁太”,估计可能是从哪里拐骗来的,详情不得而知。这个少年仿佛已经绝望,蹲在解剖室的角落里。站在解剖台周围的十几个身着白色上衣的队员,只露出经过消毒的双手。有一个人说了句简短的话,命令这个少年爬上解剖台。

\section{可以“再造个活人”}

中国少年按照命令脱光了上身,躺在解剖台上。这位少年还不知道自己身上即将发生什么事情。然后脱掉他的裤子。少年生殖器周围还没有长毛,也许中国东北地区的人体毛较少,从生殖器和其周围的情况来推测,这位少年年龄约为十二三岁。

他们首先把浸透了哥罗仿(麻醉药)的脱脂棉捂在那个躺着的中国少年的嘴和鼻子上进行了全身麻醉。然后再用酒精擦干净少年的身体。

一位资深的雇员从围绕着解剖台的田部班成员中走出来,手握手术刀靠近这个少年,然后他沿着少年的胸腔用手术刀开出了一个Y字型。再用止血钳进行止血,鲜血不停地流出,露出了白色的脂肪,活体解剖便开始了。

“少年并不是‘马鲁太’……孩子并没有进行什么抗日运动。后来,我才知道解剖他是为了取得一个健康的男少年的内脏。由于这个缘故,这个少年就活活地被解剖了……”,后来,一个原731部队人员回忆当时解剖情景时这样说道。

从这个沉睡中的少年身上依次取出肠、胰、肝、肾、胃等各种内脏,分别计量之后把它们丢进了桶里。放在计量器上的内脏还在蠕动,所以指针在摇摆,队员很难看准刻度。接着他们又把丢进桶里的内脏放到一个装有福尔马林液的大玻璃容器里,盖上盖子。沾满少年体液的手术刀闪闪发光。由于雇员熟练的“执刀”,少年的上半身在流血中几乎变得空无一物了。取出的内脏,泡在福尔马林液中,还在不断地抽动,进行着收缩运动。

“喂,还活着呢……”

不知是谁这样说道,这可以再造一个活人。取掉胃,切除肺部之后,中国少年只剩下头部,一个小小的光头。凑班的一个人把它固定在解剖台上,在耳部到鼻子之间,横切了一刀。在剥开头皮之后,开始锯头,头盖骨被错成三角形之后取了下来,露出了脑子。部队人员用手插入柔软的保护膜,像取豆腐般地把少年的脑子取了出来,又迅速地放入装有福尔马林液的容器中,解剖台上的少年只剩下四肢和一副空躯壳了。到此,解剖结束。

“拿走!”

呆在一旁的人员把装有少年内脏的容器一个个地拿走,而对这个被迫死去的少年没有一点怜悯之心。在他们看来,甚至连判刑都不需要。少年只不过是摆在恶魔餐桌上的一块肉而已。队员双手捧着玻璃容器在走廊上一走,由于摇晃,内脏在溶液里不时作响,收缩了起来。由于容器重,生怕摔倒,他们使出全身的力气,捧着它,缓慢地走着……

将要进入青春期的这个中国少年的姓名,恐怕同无数“马鲁太”一样,至今也无人知晓,他本人也不会知道自己被活生生地解剖的理由。在被迫短短的假寐状态中,他丧失了一切:

鲜血流如注,
活体解剖躯尽空,
五脏秤上动。

1940年9月,在浙江省杭州市郊外笕桥逮捕了一名便衣队人员(中国游击队员),在蒋介石的中央航校旧址斩首后,尸体由731部队冈本班解剖。原731部队I.N氏目睹了活体解剖现场。他在回忆当时情景时写下了这首短诗。



\backmatter

\end{document}