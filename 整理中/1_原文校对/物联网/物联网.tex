% 物联网
% 物联网.tex

\documentclass[12pt,UTF8]{ctexbook}

% 设置纸张信息。
% 纸张设置配置文件
% 用于定义书籍的页面尺寸和边距

\usepackage[a4paper,twoside]{geometry}
\geometry{
	left=25mm,
	right=20mm,
	top=25mm,
	bottom=25.4mm,
	headsep=1cm, 
    footskip=1cm,
	bindingoffset=10mm
}

% 设置字体,并解决显示难检字问题。
\xeCJKsetup{AutoFallBack=true}
\setCJKmainfont{SimSun}[BoldFont=SimHei, ItalicFont=KaiTi, FallBack=SimSun-ExtB]

% 目录 chapter 级别加点(.)。
\usepackage{titletoc}
\titlecontents{chapter}[0pt]{\vspace{3mm}\bf\addvspace{2pt}\filright}{\contentspush{\thecontentslabel\hspace{0.8em}}}{}{\titlerule*[8pt]{.}\contentspage}

% 设置 part 和 chapter 标题格式。
\ctexset{
	chapter/name={第,章},
	chapter/number={\arabic{chapter}}
}

% 图片相关设置。
\usepackage{graphicx}
\graphicspath{{Images/}}

% 设置署名格式。
\newenvironment{shuming}{\hfill\zihao{4}}

% 注脚每页重新编号,避免编号过大。
\usepackage[perpage]{footmisc}

\title{\heiti\zihao{0} 物联网}
\author{佚名}
\date{}

\begin{document}

\maketitle
\tableofcontents

\frontmatter

\mainmatter

\chapter{物联网的基础知识}

首先我们来了解一下学习物联网所需的基础知识。

\section{物联网入门}

\subsection{物联网}

大家在听到物联网时,脑海中会出现一个什么样的印象呢?

物联网的英语是 Internet of Things,缩写为 IoT,这里的“物”指的是我们身边一切能与网络相连的物品。例如您身上穿着的衣服、戴着的手表、家里的家用电器和汽车,或者是房屋本身,甚至正在读的这本书,只要能与网络相连,就都是物联网说的“物”。

就像我们用互联网在彼此之间传递信息一样,物联网就是“物”之间通过连接互联网来共享信息并产生有用的信息,而且无需人为管理就能运行的机制。这样一来,就创造出了一直未能实现的魔法般的世界。

\subsection{物联网的相关动向}

ICT\footnote{信息、通信和技术三个英文单词的首字母组合(Information Communication Technology,简称 ICT)。} 市场调查公司的 IDC(Internet Data Center,互联网数据中心)调查结果显示,2013 年日本国内物联网市场的市场份额约有 11 万亿日元,预测这个数字在 2018 年大约会增至 2013 年的两倍,即 21 万亿日元左右。

物联网市场是由若干个市场形成的,包括作为“物”的设备市场,掌管物与物之间联系的网络市场,还有运营管理类的平台市场,分析采集到的数据的分析处理市场等(图 1.1)。

\begin{figure}[htbp]
	\centering
	\includegraphics[width=1\linewidth]{1}
	\caption{物联网的相关市场}
	\label{fig:1}
\end{figure}

说起创建物联网市场的要素,那就要提到通信模块价格趋向低廉以及云服务的普及。英特尔公司在 2014 年 10 月将一款名为英特尔 Edison 的单板计算机投入了市场。这款单板机在一个只有邮票大小的模块上搭
载了双核双线程的 CPU 和 1 GB 内存、4 GB 的存储空间、双频的 Wi-Fi 以及蓝牙 4.0。除此之外,微软还公布了名为 Microsoft Azure Intelligent Systems Service(Azure 智能系统服务)的解决方案,它负责用云技术实现数据管理和处理,以及通信管理等功能。

此外,在平台、分析处理和网络安全等方面,针对物联网的产品和服务也已经开始投入市场。物联网市场今后的重点在于跟熟悉各垂直市场的从业者加强合作,积极提供试验环境以及开发贴近用户生活的服务项目。

\section{物联网所实现的世界}

\subsection{“泛在网络”社会}

在讲物联网所实现的世界之前,我们先从“连接网络”的观点来回顾一下历史。

20 世纪 90 年代初,过去以大型机为中心的集中式处理逐渐向以客户端服务器为中心的分布式处理转移。自 20 世纪 90 年代后期起,新型集中式处理围绕着以互联网和 Web 为代表的网络形成了一股发展趋势。这就是 Web 计算的概念。以互联网为媒介,人们可以轻松实现 PC、服务器、移动设备之间的信息交换。

21世纪初,一个名为“泛在网络”的概念开始受到人们的关注。泛在网络的理念在于使人们能够通过“随时随地”连接互联网等网络来利用多种多样的服务(图 1.2)。近年来,通过智能手机和平板电脑,甚至游戏机、电视机等一些过去无法连接到网络的“物”,就可以随时随地访问互联网。

\begin{figure}[htbp]
	\centering
	\includegraphics[width=1\linewidth]{2}
	\caption{泛在网络可以让人们随时随地访问网络}
	\label{fig:1}
\end{figure}

\subsection{“物”的互联网连接}

随着宽带的普及,泛在网络社会日益得到实现。此外,能搭载在机器上的超低功耗传感器投入市场、无线通信技术进步等,都促使除了电脑、服务器和智能手机等传统连接互联网的 IT 相关设备以外,各种各样的“物”也可以连接互联网(图 1.3)。以汽车、家用电器以及房屋为开端,近来,眼镜和手表、饰品这些戴在身上的“物”也连接上了互联网并开始得到应用,如 Google Glass 和 Apple Watch 。

\begin{figure}[htbp]
	\centering
	\includegraphics[width=1\linewidth]{3}
	\caption{连接互联网的各种各样的“物”}
	\label{fig:1}
\end{figure}

形形色色的“物”都能与互联网相连,这一点大家都已经了解了。那么这种“相连”会产生什么呢?它又是如何给人们的生活带来方便的呢?下面,就来看看物联网带给我们的世界吧。

\subsection{机器对机器通信所实现的事}

在物联网的实现方面,近年来机器对机器通信等关键技术备受人们关注(图 1.4)。物联网和机器对机器通信在很多方面可以视作同一个意思,但从严格意义上来说二者是不同的。机器对机器通信是不经人为控制的、机器和机器之间的通信;也就是说,多数情况下它表示的是机器和机器自动交换信息的整体系统。另一方面,物联网则大多含有给信息接收者提供服务的含义,它比机器对机器通信的概念范围更广。

\begin{figure}[htbp]
	\centering
	\includegraphics[width=1\linewidth]{4}
	\caption{机器对机器通信所实现的社会}
	\label{fig:1}
\end{figure}

泛在计算的世界是一个所有的“物”都内置计算机中,随时随地可以得到计算机帮助的世界。而机器对机器通信支撑着泛在计算的世界,并通过支撑社会的基础设施——智能社区和智能电网等形式逐步得到实现。

此外,机器对机器通信不仅可以通过 3G 和 LTE 电路的信息系统实现,还可以通过本地网络中的无线通信和有线通信来实现。

除了企业内的信息和互联网的信息以外,我们还能够灵活应用来自机器的信息。这样一来也就掌握了现实世界中的情况变动,尤其是提高了企业中的信息应用度。

\subsection{物联网实现的世界}

大家已经知道,我们可以借助机器对机器通信采集和积累信息,并灵活运用从信息中分析出的数据来方便我们的生活。那么,如果在此基础上把数百亿台设备都连接上物联网,又会如何呢?

以前,人们通过让少数昂贵的工业机械通信,来实现对“物”的远程控制。今后,人们将更多地以低廉的价格大量生产面向用户的机器,并让这些机器通信。也正因应用了从这些“物”中获取的数据,各种各样的服务才如雨后春笋般涌现出来。此外,先进感测技术的普及实现了人类对现实世界的掌握和预测,通过实时且海量地搜集人、物、社会和环境的数据,也有望进行新型社会基础设施的构建,例如强化产业竞争力、建设都市和社会制度、监测灾害等异常情况。

除了那些一眼就能看明白的设备,具有连通性(机器和系统间的互联性和关联性)的设备也在不断地随处增加。物联网的趋势指的就是这一现象。通过本章,我们再深入地看一下物联网所实现的是一个怎样的世界。

1. 智能设备

2. 具备连通性的“物”

3. 网络

4. Web 系统

5. 数据分析技术

大家认为把这些因素组合到一起,将会产生出一个怎样的充满革新性的服务呢?

举个例子,市面上已经出现了很多叫作智能家居的设备,其用途是控制智能住宅。飞利浦 Hue 是一款能通过 IP 网络来控制自身亮度和光色的灯泡。Nest 是一种机器控制器,它能学习如何控制空调等机器以及如何设定这些机器的目标值。如果把它们与 Web 系统和可穿戴设备等智能设备组合在一起,还能实现由住宅主动根据人的动作和身体状况来调整环境(图 1.5)。

\begin{figure}[htbp]
	\centering
	\includegraphics[width=1\linewidth]{5}
	\caption{根据人体状况自动控制环境——以智能家居为例}
	\label{fig:1}
\end{figure}

可以说,当下的趋势之一就是不停留在单纯的控制层面,而是像“凭借短距离通信实现自主控制和自动化”及“通过机器学习实现自动判断”这样,给事物增添附加价值。

\subsection{蓬勃发展的标准化活动}

除 IETE\footnote{The Institution of Electronics and Telecommunication Engineers
电子与电信工程师协会。} 、3GPP\footnote{Third Generation Partnership Project
第三代合作伙伴计划。} 、ITU\footnote{International Telecommunication Union
国际电信联盟。} 等标准化团体以外,民间企业也围绕物联网积极地开展了活动。

2013 年 12 月,在美国高通公司的支持下,家电厂商的横向性物联网推进联盟 AllSeen Alliance 成立了。该联盟的意图在于越过厂商这道高墙,规划一种统一规格,让冰箱、烤箱及电灯等所有电器都能通过互联网实现协作。

2014 年 7 月,在英特尔和三星的推动下,物联网联盟 OIC\footnote{Open Interconnect Consortium
开放互联联盟。} 成立了。该联盟旨在为物联网相关机器的规格和认证设立标准。

可想而知,今后物联网普及的关键在于各厂商是否采用这种开放性规格。作为从事物联网的工程师,在选定产品时还得把这种标准化动向考虑进去,这一点是重中之重。

\section{实现物联网的技术要素}

要实现物联网,需要很多技术要素。除了传感器等电子零件和电子电路以外,还包括 Web 应用中经常用到的技术,以及数据分析等。本书将会为大家整体解说这些技术。个别详细内容在第 2 章及以后的章节中会提到,这里我们先来总览一下本书将会讲解的全部内容。

\subsection{设备}

物联网与以往的 Web 服务不同,设备在其中担任着重要的作用。设备指的是一种“物”,它上面装有一种名为传感器的电子零件,并与网络相连接。比如大家拿着的智能手机和平板电脑就是设备的一种。家电产品、我们时刻戴着的手表以及伞等,只要能满足上述条件,就是设备(图 1.6)。

\begin{figure}[htbp]
	\centering
	\includegraphics[width=1\linewidth]{6}
	\caption{与网络连接的设备}
	\label{fig:1}
\end{figure}

这些设备起着两个作用:感测和反馈。下面我们分别说明它们各自的作用。

\subsubsection{感测的作用}

感测指的是搜集设备本身的状态和周边环境的状态并通知系统(图 1.7)。这里说的状态包括房门的开闭状态、房间的温度和湿度、房间里面有没有人,等等。设备是利用传感器这种电子零件来实现感测的。

打个比方,如果伞上有用于检测其开合的传感器并具备连接网络的功能,那么多把伞的开合状态就可以被检测到。利用这一点就能调查出是否在下雨。在这种情况下,如果一个地区有多把伞打开,就可以推测出该地区正在下雨。反过来,就能推断出大多数伞都合着的地区没有在下雨。此外,通过感测设备周边的环境还能搜集温度和湿度等信息。

\begin{figure}[htbp]
	\centering
	\includegraphics[width=1\linewidth]{7}
	\caption{感测的作用}
	\label{fig:1}
\end{figure}

\subsubsection{反馈的作用}

设备的另外一个作用是接收从系统发来的通知,显示信息或执行指定操作(图 1.8)。系统会基于从传感器处搜集到的信息进行一些反馈,并针对现实世界采取行动。

\begin{figure}[htbp]
	\centering
	\includegraphics[width=1\linewidth]{8}
	\caption{反馈的作用}
	\label{fig:1}
\end{figure}

反馈有多种方法。大体分成如图 1.9 所示的 3 种方法,分别是可视化、通知,以及控制。

\begin{figure}[htbp]
	\centering
	\includegraphics[width=1\linewidth]{9}
	\caption{反馈的 3 种方法}
	\label{fig:1}
\end{figure}

比方说,用户通过“可视化”就能使用电脑和智能手机上的 Web 浏览器浏览物联网服务搜集到的信息。虽然最终采取行动的是用户,不过这是最简单的一个反馈的例子。只要把房间的当前温度和湿度可视化,人就能将环境控制在最适宜的条件下。

利用“推送通知”,系统就能检测到“物”的状态和某些活动,并将其通知给设备。例如从服务器给用户的智能手机推送通知,使其显示消息。近年来,Facebook 和 Twitter 等 SNS 社交应用就在贴心地向我们的智能手机频繁推送朋友们吃饭和旅行的消息。如果你去逛超市时,推送通知能告诉你冰箱的牛奶过了保质期,洗涤用品卖完了,这个世界岂不就更方便了吗?

利用“控制”,系统就可以直接控制设备的运转,而无需借助人工。假设在某个夏天的傍晚,你正在从离家最近的车站往家走,你的智能手机会用 GPS 确定你现在的位置和前进的方向,用加速度传感器把你的步速通知给物联网服务。这样一来,服务就能分析出你正在回家的路上,进而从你的移动速度预测你到家的时间,然后发出指示调节家里空调的温度并令其开始运转。这样当你回到家的时候,家里就已经很舒服了。

\subsection{传感器}

要想像前文说的那样搜集设备和环境的状态,就需要利用一个叫作传感器的电子零件。

传感器负责把物理现象用电子信号的形式输出。例如有的传感器可以把温度和湿度作为电子信号输出,还有的传感器能把超声波和红外线等人类难以感知的现象转换成电子信号输出。

数码相机上使用的图像传感器也能把进入镜头的光线捕捉成 3 种颜色的光源,并将其转换成电子信号。因此它也可以归在传感器的分类里。传感器的种类如图 1.10 所示。关于这些传感器的种类和它们各自的结构,我们会在第 3 章详细介绍。

\begin{figure}[htbp]
	\centering
	\includegraphics[width=1\linewidth]{10}
	\caption{具有代表性的传感器的种类}
	\label{fig:1}
\end{figure}

通过传感器输出的电子信号,系统就能够获取现实世界的“物”的状态和环境的状态。

人们很少单独利用这些传感器,通常都是将它们置入各种各样的“物”里来加以利用。最近的智能手机和平板电脑就内置了很多传感器,例如用于检测画面倾斜度的陀螺仪传感器和加速度传感器,采集语音的麦克风,用于拍摄照片的相机,具备指南针功能的磁场传感器。

还有一种东西叫作传感器节点,它把传感器本身置入环境中搜集信息。传感器节点是集蓝牙和 Wi-Fi 等无线通信装置与电池为一体的传感器。我们把这些传感器连接到一种叫作网关的专用无线路由器来进行传感器数据的搜集(图 1.11)。

\begin{figure}[htbp]
	\centering
	\includegraphics[width=1\linewidth]{11}
	\caption{传感器节点和网关}
	\label{fig:1}
\end{figure}

比如,在农场测量栽培植物的环境时,或是检测家里房间的温度和湿度时,就可以利用这些传感器节点。除此之外,市面上还有各种各样用于医疗保健的可穿戴设备,这些设备上装有加速度传感器和脉搏计,人们可以利用这些设备管理自己的生活节奏和健康状况。

这样一来,物联网服务就能利用传感器获取设备、环境、人这些“物”的状态。自己想实现的服务都需要哪些信息,为此应该利用哪种传感器和设备,这些都需要我们仔细分析。

\subsection{网络}

在把设备连接到物联网服务时,网络是不可或缺的。不仅要把设备连接到物联网服务,还得把设备连接到其他设备。物联网使用的网络大体上分为两种:一种是把设备连接到其他设备的网络,另一种是把设备连接到物联网服务的网络(图 1.12)。

\begin{figure}[htbp]
	\centering
	\includegraphics[width=1\linewidth]{12}
	\caption{用于物联网的两种网络}
	\label{fig:1}
\end{figure}

\subsubsection{把设备连接到其他设备的网络}

无法直接连接到互联网的设备也是存在的。我们通过把设备连接到其他设备,就能通过其他设备把这些不能连接到互联网的设备连接到互联网。前面我们介绍的传感器节点和网关正是两个典型的例子。此外,还有通过智能手机把可穿戴设备采集到的数据发送给物联网服务这一办法。

蓝牙和 ZigBee 是两种具有代表性的网络标准。它们是用无线连接的,利用的通信协议也是固定的。这些协议的特征有采用擅长近距离通信的无线连接、低功耗、易于嵌入嵌入式设备等。

要把设备连接到其他设备,除了 1 对 1 之外,还可以采用 1 对 N、N 对 N 的方式连接。特别是 N 对 N 连接的情况,我们称这种情况为网状网络(图 1.13)。

\begin{figure}[htbp]
	\centering
	\includegraphics[width=1\linewidth]{13}
	\caption{设备之间的网络连接}
	\label{fig:1}
\end{figure}

有一种与网状网络对应的通信标准,名为 ZigBee。通过采用 N 对 N 的通信方式,设备可以一边接管其他的设备,一边进行远程通信。除此之外它还有一个优点,那就是即使有一台设备发生故障无法通信,其他设备也会代替它来执行通信。

关于上述设备的通信规格我们会在第 3 章讲解。

\subsubsection{把设备连接到服务器的网络}

把设备连接到物联网服务的网络时,会用到互联网线路。3G 和 LTE 等移动线路最为常用。

除了现在 Web 服务中广泛使用的 HTTP 和 WebSocket 协议以外,还有一些专为机器对机器通信和物联网而产生的轻量级协议,如 MQTT 等。关于该协议,我们会在第 2 章进行详细说明。

\subsection{物联网服务}

物联网服务有两个作用:一是从设备接收数据以及发送数据给设备;二是处理和保存数据(图 1.14)。

\begin{figure}[htbp]
	\centering
	\includegraphics[width=1\linewidth]{14}
	\caption{Web 系统的作用}
	\label{fig:1}
\end{figure}

我们来具体看一下这两个作用。

\subsubsection{数据交换}

通常的 Web 服务会根据 Web 浏览器发送的 HTTP 请求发送 HTML,然后用 Web 浏览器显示。物联网服务则不采用 Web 浏览器,而是接收从设备直接发来的数据。设备发来的数据内容包括设备搭载的传感器所采集到的信息,以及用户对设备进行的操作。设备和物联网服务的通信方法大致分为两种:同步传输和异步传输(图 1.15)。

在同步传输的情况下,设备发送数据时会把数据发送给物联网服务。接下来直到物联网服务接收完数据之前,不管设备向物联网服务发送多少次数据,都算作一次传输。反过来,物联网服务在执行对设备的反馈时,则是先由设备向物联网服务发送请求消息,然后物联网服务会响应请求并将消息发送给设备。就这种方法而言,直到设备发送请求之前,物联网服务都不能把消息发送给设备。但是这种方法只适用于不知
道设备 IP 地址的情况,因为就算不知道设备的 IP 地址,只要设备发送了请求,物联网服务就能把消息发送给设备。

在异步传输中,设备会把数据发送给物联网服务,每发送一次,就算作一次传输。此外,从物联网服务向设备进行传输时,无需等待设备发来的请求,可以在任意时间点执行发送。采用这个方法能在物联网服
务规定的任意一个时刻发送消息。但是,物联网服务需要预先知道发送消息的设备的 IP 地址。

\begin{figure}[htbp]
	\centering
	\includegraphics[width=1\linewidth]{15}
	\caption{Web 系统和设备的通信}
	\label{fig:1}
\end{figure}

第 2 章会用一些实际使用的协议来讲解这种通信。

\subsubsection{处理和保存数据}

就如大家在图 1.14 看到的那样,处理和保存数据的操作包括把从设备接收到的数据保存到数据库,以及从接收到的数据来判断如何控制设备。

从设备接收到的数据不只有能用计算机简单处理的数值型数据,根据要实现的内容,还包含图像、语音、自然语言这些很难直接用计算机处理、没有被结构化的数据。我们把这种数据叫作非结构化数据。处理
时,有时也会把那些易于用计算机处理的数据从非结构化数据中提取出来,例如把表示图像和语音特征的值提取出来。这些信息会被保存到数据库中。

设备按照所提取数据的判断逻辑来决定反馈的内容,例如基于某个房间的温度数据来决定空调的开关状态和目标温度。这些处理和保存的方法大体上分为两种:一种是对保存的数据定期进行采集和处理的批处
理,另一种是将收到的数据逐次进行处理的流处理(图 1.16)。

\begin{figure}[htbp]
	\centering
	\includegraphics[width=1\linewidth]{16}
	\caption{保存和处理数据的时机}
	\label{fig:1}
\end{figure}

根据房间的温度变化来调整空调的运转时,从向空调发出指示到温度发生变化,这中间会需要一段时间。这种情况下就适合采用批处理来持续记录每隔一定时间的温度值,并定期执行处理。此外,如果希望回到房间之后再打开空调,那么就适合采用能立即执行操作的流处理。

\subsection{数据分析}

前一节我们以“温度传感器和空调运转的关系”为例进行了说明。那么我们能像这个例子那样,轻松实现“根据房间温度控制空调”这一目的吗?

要实现这一目的,需要决定控制空调开 / 关的房间温度值,也就是决定温度的阈值。这种情况下,阈值会根据使用者目的而有所不同。举个例子,把空调的功耗降到最小所需要的阈值和保持令人体感舒适的温
度所需要的阈值就是两个不同的值。此外,为了能准确判断房间里有没有人,需要从多个传感器的值所包含的关联性来判断人在或不在房间里。人类很难光凭经验去摸索和决定这种值。这就凸显出了数据分析的
重要性。

数据分析的代表性方法有两种,分别是统计分析和机器学习。这里就来看看我们用这两种方法能办到什么(图 1.17)。

\begin{figure}[htbp]
	\centering
	\includegraphics[width=1\linewidth]{17}
	\caption{数据分析的两种方法}
	\label{fig:1}
\end{figure}

\subsubsection{统计分析}

统计分析是用数学手法通过搜集到的大量数据来明确事物的联系性的方法。比如为了实现给空调节能的目的,我们调查了空调在某个固定的温度下运转时,房间的温度和空调的耗电量,并将这些数据制成了表格(图 1.18)。

\begin{figure}[htbp]
	\centering
	\includegraphics[width=1\linewidth]{18}
	\caption{空调的电力和室温的关系示例}
	\label{fig:1}
\end{figure}

从这个关系中可以推导出在室温下把空调温度设定在多少才能最省电,由此就能决定阈值了。

上述示例采用的是先填表再分析的方法,除此之外还有一种叫作回归分析的统计方法,此方法我们会在第 6 章详细说明。

\subsubsection{机器学习}

统计分析基于大量数据之间的联系性,明确当前数据间形成的关联。机器学习则不仅仅能进行分析,还能预测今后的发展状况。

机器学习就如它的字面意思一样,计算机会按照程序决定的算法,机械性地学习所给数据之间的联系性。当给出未知数据时,也会输出与其对应的值。

机器学习分为两个阶段:学习阶段和识别阶段(图 1.19)。在学习阶段,一个名为学习器的程序会基于一些训练数据,机械性地掌握这些数据之间的联系。作为学习阶段的结果,计算机会根据机器学习的算法输出参数,然后以这个参数为基础创建叫作鉴别器(discriminator)的程序。只要把未知的数据给这个鉴别器,就能输出最适合这个值的结果。

\begin{figure}[htbp]
	\centering
	\includegraphics[width=1\linewidth]{19}
	\caption{机器学习示例}
	\label{fig:1}
\end{figure}

举个例子,假设我们想使用若干种传感器来识别房间里有没有人。这种情况下需要准备两种数据,即房间里有人时的传感器数据(正面例子)和房间里没人时的传感器数据(反面例子)。计算机通过把这两种数据分别交给学习器,可以获取制作鉴别器用的参数。对于以参数为基准制作的鉴别器而言,只要输入从各个感测设备接收到的数据,鉴别器就能输出结果,告诉我们现在房间里是否有人。

上述内容属于机器学习的示例之一,被称作分类问题。在用于执行数据分类的机器学习算法中有很多途径,如用于垃圾邮件过滤器的贝叶斯过滤器和用于分类文档及图像的支持向量机(Support Vector Machine,SVM)等。此外,除了分类问题以外,机器学习还能解决很多领域的问题。

\chapter{物联网的架构}

\section{物联网的整体结构}

实现物联网时,物联网服务大体上发挥着两个作用。

第一是把从设备收到的数据保存到数据库,并对采集的数据进行分析。

第二是向设备发送指令和信息。

本章将会为大家介绍如何构建物联网服务,以及用于实现物联网的重要要素。

\subsection{整体结构}

物联网大体上有 3 个构成要素,如图 2.1 所示。一个是设备,另一个是网关,再来就是服务器。关于设备的基本结构和使用的技术,我们会在第 3 章详细说明。因此本章并不涉及设备。我们来详细看一下用怎样的机制才能实现网关和服务器。

\begin{figure}[htbp]
	\centering
	\includegraphics[width=1\linewidth]{20}
	\caption{物联网的整体结构}
	\label{fig:1}
\end{figure}

\subsection{网关}

如图 2.1 左下所示,物联网使用的设备中,有 3 台设备不能直接连接到互联网。网关就负责把这些设备转发到互联网。

网关指的是能连接多台设备,并具备直接连接到互联网的功能的机器和软件(图 2.2)。如今,市面上有很多种网关。在多数情况下,网关凭借 Linux 操作系统来运行。

\begin{figure}[htbp]
	\centering
	\includegraphics[width=1\linewidth]{21}
	\caption{选择网关的标准}
	\label{fig:1}
\end{figure}

选择网关时有几项重要的标准,我们来一起看一下。

\subsubsection{接口}

第一重要的是用于连接网关和设备的接口。网关的接口决定了能连接的设备,因此重点在于选择一个适配设备的接口。

有线连接方式包括串行通信和 USB 连接。串行通信中经常用的是一种叫作 D-SUB 9 针(pin)的连接器,而 USB 连接中用到的 USB 连接器则种类繁多。

无线连接中用的接口是蓝牙和 Wi-Fi(IEEE 802.11)。此外,还有采用 920 MHz 频段的 Zigbee 标准,以及各制造商们的专属协议。第 3 章会详细讲解这些规格各自的特征,重点在于根据设备对应的标准来选择接口。

\subsubsection{网络接口}

我们用以太网或是 Wi-Fi、3G/LTE 来连接外部网络。网络接口会影响到网关的设置场所。以太网采用有线连接,通信环境稳定。然而正因为采用的是有线连接,所以必须把 LAN 电缆布线到网关的设置场所。因此,在设置场所方面就会在某种程度上受到限制。

对 3G/LTE 连接而言,设置场所就比较自由了,但通信的质量会受信号强弱影响,所以通信不如有线连接稳定。因此,有时很难在信号不良的大楼和工厂等封闭环境中设置。不过,3G/LTE 连接有个好处,即
只使用网关就能完成和外部的通信,因此操作起来很简单。此外,想使用 3G/LTE 时,需要和电信运营商签订协议并获取 SIM 卡,这点就跟使用手机一样。

\subsubsection{硬件}

相对于一般计算机而言,网关在 CPU 和内存这些硬件的性能方面比较受限。我们需要确定让网关做哪些事情,也需要考虑到它的硬件性能。

\subsubsection{软件}

人们主要使用 Linux 操作系统来运行网关。虽然有很多种用于服务器的 Linux,不过,网关上搭载的 Linux 是面向嵌入式的。

此外,还有一个叫作 BusyBox 的软件,它运行起来占用内存少,集成了标准的 Linux 命令工具。它用于在硬件资源匮乏的时候运行网关。除此之外,还要考虑是否有用于控制网关功能的程序库,以及与这种程序库对应的语言等。

\subsubsection{电源}

说起来,电源很容易被人们遗忘。网关基本上都是使用 AC 适配器当电源的,因此需要事先在设置网关的场所准备好电源。如果网关本身搭载有电池,那么就不需要准备电源了,不过需要进行充电等维护工作。

\subsection{服务器的结构}

在功能方面,物联网服务大体上可分为 3 个部分,本书分别称它们为前端部分、处理部分,以及数据库部分(图 2.3)。

\begin{figure}[htbp]
	\centering
	\includegraphics[width=1\linewidth]{22}
	\caption{物联网服务的 3 个功能}
	\label{fig:1}
\end{figure}

首先,前端部分包括数据接收服务器和数据发送服务器。数据接收服务器接收设备和网关发来的数据,转交给后续的处理部分。数据发送服务器则刚好相反,它负责把从处理服务器接收到的内容发送给设备。

通常情况下,Web 服务的前端部分只接受 HTTP 协议。而物联网服务的前端部分则需要根据连接设备的不同来匹配 HTTP 以外的协议。使用者需要考虑到协议的实时性和通信的轻量化,以及能否以服务器为起
点发送数据。我们会在 2.2 节重新讲解这些协议。

处理部分负责处理从前端部分接收到的数据。这里的“处理”指的是分解数据、存储数据、分析数据、生成发给设备的通知内容,等等。数据处理包括批处理和流处理等,批处理即把数据存入数据库之后一并进行处理,而流处理是逐次处理从前端部分收到的数据。使用者需要根据处理内容和数据特性来灵活使用这些“处理”。

最后是数据库。这里的数据库不只会用到关系数据库,还会用到 NoSQL 数据库。当然,使用者需要根据想存储的数据和想使用的方法来选择数据库。

\section{采集数据}

\subsection{网关的作用}

就如我们前面说的那样,网关是一台用于把不能直接连接到互联网的设备转发连接到互联网的设备。再往细了说,网关是由 3 种功能构成的(图 2.4)。

\begin{figure}[htbp]
	\centering
	\includegraphics[width=1\linewidth]{23}
	\caption{网关的功能}
	\label{fig:1}
\end{figure}

这 3 种功能分别是连接设备功能、数据处理功能和向服务器发送数据的功能。此外,实际使用网关执行应用时,还需要其他的管理应用功能。管理应用功能会在第 5 章单独介绍。

接下来就来详细看看网关的 3 种功能。

\subsubsection{连接设备}

设备和网关是通过各种各样的接口连接的。当通过传感器终端连接时,多数情况下是传感器单方面持续向服务器发送数据。根据设备不同,也存在设备申请从外部获取数据时,服务器向设备发送数据的情况,这时就需要通过网关申请数据。

\subsubsection{生成要发送的数据}

接下来把从设备接收到的数据转化成能发送给服务器的格式。在表示从设备发送到网关的数据时,也有把 4 位二进制数(如二进制数据和 BCD 码)替换成一位十进制数数据来表示的(图 2.5)。这样的数据不会被直接发给服务器,而是在网关处被转化成数值数据和字符串的格式。

\begin{figure}[htbp]
	\centering
	\includegraphics[width=1\linewidth]{24}
	\caption{BCD 码}
	\label{fig:1}
\end{figure}

还存在下面这种情况:不把每台设备发来的数据直接发送给服务器,而是将大量数据整合在一起再发送给服务器。这么做有以下两个原因。

第一,通过整合数据能减少数据的附加信息,减少数据量。第二,通过一并发送数据能减轻访问物联网服务时对服务器造成的负担。

\subsubsection{发送数据给服务器}

向物联网服务发送数据。此时,需要根据服务器来决定发送数据的时间间隔和发送数据的协议。另外,为了能从物联网的服务器接收消息,还得事先准备好这种功能。

\section{接收数据}

\subsection{数据接收服务器的作用}

数据接收服务器就跟它的字面意思一样,负责接收从设备发送来的数据。它在设备和系统之间起着桥梁作用。有很多种方法可以从设备把数据发送给服务器,其中具有代表性的包括以下两种方法。

● 准备一个使用了 HTTP 协议的 Web API 来访问设备(如通常的 Web 系统)

● 执行语音和视频的实时通信(如 WebSocket 和 WebRTC)

除此之外,还出现了一种名为 MQTT 的、专门针对物联网的新型通信协议。

本章将为大家介绍 HTTP 协议、WebSocket、MQTT 这几个典型协议。

\subsection{HTTP 协议}

HTTP 协议提供的是最大众化且最简易的方法。使用一般的 Web 框架就可以制作数据接收服务器。设备用 HTTP 的 GET 方法和 POST 方法访问服务器,把数据存入请求参数和 BODY 并发送(图 2.6)。

\begin{figure}[htbp]
	\centering
	\includegraphics[width=1\linewidth]{25}
	\caption{通过 HTTP 协议发送和接收数据}
	\label{fig:1}
\end{figure}

HTTP 协议是 Web 的标准协议,这一点自不用说。因此 HTTP 协议和 Web 的兼容性非常强。此外,因为 HTTP 协议有非常多的技术诀窍,所以我们必须在制作实际系统时审视服务器的结构,应用程序的架构以及安全性等。关于这点,有很多事例值得参考。另外,HTTP 协议还准备了 OSS 的框架,方便人们使用。

\subsubsection{REST API}

设备应该如何访问物联网服务呢?用 HTTP 协议访问的时候,也得从 GET 和 POST 中选择一种合适的方法来访问。除了物联网服务,一般 Web 服务中公开的 API 也应格外重视这个问题。

在 Web 服务的世界里,有一种思路叫作 RESTful。REST 是一种接口,它为特定的 URL 指定参数并执行访问,作为其响应来获取结果。它通过用多个 HTTP 方法访问一个 URL,来对一个 URL 执行获取和注册数据。这样一来,URL 的作用就易于理解了。

例如,使用 GET 方法访问 /sensor/temperature 就能获取温度传感器的值。使用 POST 方法一并访问传感器数据,就会追加新的传感器数据。

如果想用除了 RESTful 以外的方法实现同样的功能,就需要生成用于获取以往数据的 URL 和追加数据的 URL,并决定其分别用 GET 方法访问还是用 POST 方法访问。RESTful 的思路保证了 URL 设计的简单性,请大家务必审视一下 RESTful 的思路。

\subsection{WebSocket}

WebSocket 是一种通信协议,用于在互联网上实现套接字通信。它实现了 Web 浏览器和 Web 服务器间的数据双向连续传输。

就 HTTP 协议而言,每次发送数据都必须生成发送数据用的通信路径及连接。此外,一般情况下,客户端没有发出申请就不能进行通信。

相对而言,WebSocket 就不同了。只要一开始根据客户端发出的连接申请确立了连接,就能持续用同一个连接传输数据。另外,只要确立了连接,就算客户端没有发出申请,服务器也能给客户端发送数据(图
2.7)。

\begin{figure}[htbp]
	\centering
	\includegraphics[width=1\linewidth]{26}
	\caption{通过 WebSocket 协议传输数据}
	\label{fig:1}
\end{figure}

这样一来,在发送语音数据等连续的数据,以及发生与服务器的相互交换时,就能使用 WebSocket 了。WebSocket 自身只提供服务器与客户端的数据交换,因此需要使用者另外决定在应用层上使用的协议。

\subsection{MQTT}

MQTT(MQ Telemetry Transport,消息队列遥测传输)是近年来出现的一种新型协议,物联网领域会将其作为标准协议。MQTT 原本是 IBM 公司开发的协议,现在则开源了,被人们不断开发着。

MQTT 是一种能实现一对多通信(人们称之为发布或订阅型)的协议。它由 3 种功能构成,分别是中介(broker)、发布者(publisher)和订阅者(subscriber)(图 2.8)。

\begin{figure}[htbp]
	\centering
	\includegraphics[width=1\linewidth]{27}
	\caption{通过 MQTT 传输消息}
	\label{fig:1}
\end{figure}

中介承担着转发 MQTT 通信的服务器的作用。相对而言,发布者和订阅者则起着客户端的作用。发布者是负责发送消息的客户端,而订阅者是负责接收消息的客户端。MQTT 交换的消息都附带“主题”地址,各个客户端把这个“主题”视为收信地址,对其执行传输消息的操作。形象地比喻一下,中介就是接收邮件的邮箱。

再来详细看一下 MQTT 通信的机制(图 2.9)。首先,中介在等待各个客户端对其进行连接。订阅者连接中介,把自己想订阅的主题名称告诉中介。这就叫作订阅。

\begin{figure}[htbp]
	\centering
	\includegraphics[width=1\linewidth]{28}
	\caption{MQTT 通信的机制}
	\label{fig:1}
\end{figure}

然后发布者连接中介,以主题为收信地址发送消息。这就是发布。

发布者一发布主题,中介就会把消息传递给订阅了该主题的订阅者。如图 2.9 所示,如果订阅者订阅了主题 A,那么只有在发布者发布了主题 A 的情况下,中介才会把消息传递给订阅者。订阅者和中介总是处于连接状态,而发布者则只需在发布时建立连接,不过要在短期内数次发布时,就需要保持连接状态了。因为中介起着转发消息的作用,所以各个客户端彼此之间没有必要知道对方的 IP 地址等网络上的收信地址。

又因为多个客户端可以订阅同一个主题,所以发布者和订阅者是一对多的关系。在设备和服务器的通信中,设备相当于发布者,服务器则相当于订阅者。

主题采用的是分层结构。用“\#”和“+”这样的符号能指定多个主题。如图 2.10 所示,/Sensor/temperature/\# 中使用了“\#”符号,这样就能指定所有开头为 /Sensor/temperature/ 的主题。此外,/Sensor/+/room1 中使用了符号“+”,这样一来就能指定所有开头是 /Sensor/、结尾是 /room1 的主题。

\begin{figure}[htbp]
	\centering
	\includegraphics[width=1\linewidth]{29}
	\caption{MQTT 的主题示例}
	\label{fig:1}
\end{figure}

像这样借助于中介的发布 / 订阅型通信,MQTT 就能实现物联网服务与多台设备之间的通信。另外,MQTT 还实现了轻量型协议。因此它还能在网络带宽低、可靠性低的环境下运行;又因为消息小、协议机制简单,所以在硬件资源(设备、CPU 和内存等)受限的条件下也能运行,可以说是为物联网量身定做的协议。MQTT 本身还具备特殊的机制,下面我们会对其逐一说明。

\subsubsection{QoS}

QoS\footnote{Quality of Service	服务质量,指一个网络能够使用各种基础技术,为指定的网络	通信提供更好的服务能力,是网络的一种安全机制,也是用来解决网络延迟和阻塞等问题的一种技术。} 是 Quality of Service(服务质量)的简称。这个词在网络领域表示的是通信线路的品质保证。MQTT 里存在 3 个等级的 QoS。“发布者和中介之间”以及“中介和订阅者之间”都分别定义了不同的 QoS 等级,以异步的方式运行。此外,当“中介与订阅者之间”指定的 QoS 小于“发布者和中介之间”交换的 QoS 时,“中介与订阅者之间”的 QoS 会被降级到指定的 QoS。QoS 0 指的是最多发送一次消息(at most once)(图 2.11),发送要遵循 TCP/IP 通信的“尽力服务”\footnote{Best Effort 尽力服务,标准的因特网服务模式。在网络接口发生拥塞时,不顾及用户或应用,马上丢弃数据包,直到业务量有所减少为止。}。消息分两种情况,即到达了一次中介处,或没有到达中介处。

\begin{figure}[htbp]
	\centering
	\includegraphics[width=1\linewidth]{30}
	\caption{QoS 0(最多只能发送一次)}
	\label{fig:1}
\end{figure}

接下来的 QoS 1 是至少发送一次消息(at least once)(图 2.12)。

中介一接收到消息就会向发布者发送一个叫作“PUBACK 消息”的响应,除此之外还会根据订阅者指定的 QoS 发送消息。当发生故障,或经过一定时间后仍没能确认 PUBACK 消息时,发布者会重新发送消息。
如果中介接收了发布者发来的消息却没有返回 PUBACK,那么中介会重复收到消息。

\begin{figure}[htbp]
	\centering
	\includegraphics[width=1\linewidth]{31}
	\caption{QoS 1( 至少发送一次消息 )}
	\label{fig:1}
\end{figure}

最后是 QoS 2,它指的是准确发送一次消息(exactly once)。把它跟 QoS 1 合在一起使用,就能避免接收到重复的消息(图 2.13)。用QoS 2 发送的消息里面含有消息 ID。中介收到消息后会将消息保存,然后给发布者发送 PUBREC 消息。发布者再给中介发送 PUBREL 消息,然后中介会给发布者发送 PUBCOMP 消息。接下来中介才会依据订阅者指定的 QoS,向订阅者传递接收到的消息。

\begin{figure}[htbp]
	\centering
	\includegraphics[width=1\linewidth]{32}
	\caption{QoS 2( 只发送一次消息 )}
	\label{fig:1}
\end{figure}

此外,就 QoS 2 而言,有时使用的中介会影响消息的传递时间。

人们通常使用的是 QoS 0,只有要确保信息发送成功时才使用 QoS 1 和 QoS 2,这样一来可以减少网络的负担。后文将会讲到 Clean session,其中 QoS 的设定也是非常重要的。

\subsubsection{Retain}

订阅者只能接收在订阅之后发布的消息,但如果发布者事先发布了带有 Retain 标志的消息,那么订阅者就能在订阅后马上收到消息。

当发布者发布了带有 Retain 标志的消息时,中介会把消息传递给订阅了主题的订阅者,同时保存带有 Retain 标志的最新的消息。此时,若别的订阅者订阅了主题,就能马上收到带有 Retain 标志的新消息
(图 2.14)。

\begin{figure}[htbp]
	\centering
	\includegraphics[width=1\linewidth]{33}
	\caption{Retain}
	\label{fig:1}
\end{figure}

\subsubsection{Will}

Will 有“遗言”的意思。由于中介的 I/O 错误或网络故障等情况,发布者可能会突然从中介断开,Will 就是专门针对于这种情况的一个机
构,它用于定义中介向订阅者发送的消息(图 2.15)。

\begin{figure}[htbp]
	\centering
	\includegraphics[width=1\linewidth]{34}
	\caption{Will}
	\label{fig:1}
\end{figure}

发布者在连接中介时会用到 CONNECT(连接)消息,连接时对其指定 Will 标志、要发送的消息以及 QoS。这样一来,如果连接意外断开,Will 消息就会被传递给订阅者。另外,还有一个标志叫作 Will
Retain。通过指定这个标志,就能跟前面说的 Retain 达到同样的效果,即在中介处保存消息。

当发布者使用 DISCONNECT(断开连接)消息明确表明连接已断开时,Will 消息就不会被发送给订阅者。

\subsubsection{Clean session}

Clean session 用于指定中介是否保留了订阅者的已订阅状态。用CONNECT 消息连接时,订阅者把 Clean session 标志设定为 0 或 1。0是保留 session,1 是不保留 session。

若指定 Clean session 为 0 且中介已经连接上了订阅者,则中介需要在订阅者断开连接后保留订阅的消息。另外,如果订阅者的连接已经断开,且发布者已经发布了 QoS 1、QoS 2 的消息给已订阅的主题时,中介则会把消息保存,等订阅者再次连接时发送给订阅者(图 2.16)。

\begin{figure}[htbp]
	\centering
	\includegraphics[width=1\linewidth]{35}
	\caption{Clean session}
	\label{fig:1}
\end{figure}

若指定 Clean session 为 1 并连接,中介就会废弃以往保留的客户端信息,将其当成一次“干净”的连接来看待。此外,订阅者断开连接时,中介也会废弃所有的信息。

我们可以用表 2.1 所示的几种产品来实现 MQTT。是否支持前文介绍的功能则取决于中介的种类。

\begin{table}[!ht] 
\centering
\caption{MQTT 的实现}
\begin{tabular}{|l|l|l|l|l|l|}
	\hline
	实现 & QoS & Retain & Will & Clean session & 其他 \\
	\hline
	ActiveMQ 5.10.0(支持插件) & 0、1、2 & 支持 & 不支持 & 不支持 & 有独立的指定主题的方法  \\
	\hline
	Apolo 1.7 &	0、1、2 &	支持 & 支持 & 支持 & —— \\
	\hline
	mosquitto 1.3.5 & 0、1、2 & 支持 & 支持 &	支持 & —— \\
	\hline
	RabitoMQ 3.4.3(支持插件) & 只有0和1 & 不支持 & 支持 & 支持 & —— \\
	\hline
\end{tabular}
\end{table}

除此之外,一个叫作 Paho 的库还公开了发布者和订阅者等客户端功能。不仅 Java、JavaScript、Python 配备了 Paho,连 C 语言和 C++ 都配备了 Paho。因此,我们能够将其与设备结合起来并加以使用。

\subsection{数据格式}

前面我们围绕用于接收数据的通信过程,即协议进行了讲解。事实上,数据就是通过协议来进行交换的。当然,就如我们前文所说,这条规则在物联网的世界里也是不变的。数据要经过协议进行交换,而数据的格式也很重要。通过 Web 协议来使用的数据格式中,具有代表性的包括 XML 和 JSON(图 2.17)。

\begin{figure}[htbp]
	\centering
	\includegraphics[width=1\linewidth]{36}
	\caption{具有代表性的格式}
	\label{fig:1}
\end{figure}

从物联网的角度来说,使用者也能很方便地使用 XML 和 JSON 。举个例子,假设设备要发送传感器的值,此时除了发送传感器的值以外,还要一并发送数据接收时间、设备的机器信息以及用户信息等数据。自然,设备还会通知多个传感器的值和机器的状态。这样一来,使用者就需要好好地把从设备发送来的数据结构化。

图 2.18 用 XML 和 JSON 分别表示了两台传感器的信息、设备的状态、获取数据的时间,以及发送数据的设备名称等。

\begin{figure}[htbp]
	\centering
	\includegraphics[width=1\linewidth]{37}
	\caption{传感器信息的示例(XML 和 JSON)}
	\label{fig:1}
\end{figure}

比较二者可知,XML 的格式比 JSON 更容易理解。然而 XML 的字符数较多,数据量较大。相对而言,JSON 比 XML 字符数少,数据量也小。

XML 和 JSON 这两种数据格式都在每种语言中实现了各自的库,使用者通过程序就能很轻松地使用这些库。那么到底使用哪种才好呢?关于这点我们不能一概而论,不过 JSON 数据量小,更适合使用移动线路等低速线路通信的情况。

设备传来的数据和 Web 不一样,大多是传感器、图像、语音等数值数据。相较于文本而言,这样的数据更适合用二进制来处理。不过,我们前文介绍的 XML 和 JSON 都是用文本格式来处理数据的。

基于物联网服务处理这些格式时,要把文本数据转换成数值数据和二进制数据。因此需要进行两项工作,即解析 XML 和 JSON 格式,以及把解析结果从文本格式转换到二进制形式。这样一来,就需要分两步来处理。

如果能直接以二进制形式接收数据,是不是就能更迅速地处理数据了呢?由此,一种数据格式应运而生,它就是 MessagePack(图 2.19)。

\begin{figure}[htbp]
	\centering
	\includegraphics[width=1\linewidth]{38}
	\caption{使用 MessagePack 格式的传感器数据示例以及与 JSON 的对比}
	\label{fig:1}
\end{figure}

MessagePack 的数据格式虽然跟 JSON 相似,其数据却保留了二进制的形式。因此,虽然这种数据格式不方便人们直接阅读,但计算机却能很容易地处理。

又因为 MessagePack 发送的是二进制数据,所以比起以文本形式发送数据的 JSON,数据更加紧凑。MessagePack 跟 XML 和 JSON 一样,都提供了面向多种编程语言的库,另外,近年来多个 OSS(开源软件)也都采用了 MessagePack。

我们不能一口咬定哪种格式好,哪种格式不好,请各位根据要发送的数据的特性,来选择符合目的的数据格式。

\subsubsection{图像、语音、视频数据的处理}

传感器数据、文本数据”和“图像、语音、视频”的数据格式差别很大。拿图像、语音、视频来说,一条数据之巨大远远超过传感器数据,而且这些数据是二进制数据,很难转换成字符串,所以就很难用前面介绍的 XML 和 JSON 格式对它们进行处理。

用 HTTP 发送图像数据时,可以用 XML 或 JSON 格式记录拍摄时间和设备的信息,用 multi-part/form-data 格式来发送图像数据。然而,换成语音和视频时,就是一种时间上连续的数据。因此,我们在发送语音和视频数据时需要下一番工夫。

例如,需要把语音和视频分割成一个个小文件来发送。在用 HTTP 协议进行这项操作时,每次发送一个小数据都会生成一个会话。这样一来就能通过有效应用 WebSocket 等协议来减轻给物联网服务造成的负担了。这种情况下,使用者或许需要使用 MessagePack,或是定义一个专门用于处理二进制的格式。再或者,还能以用物联网服务进行语音和数据分析为前提,只在设备处提取用于分析的特征并发送,而不是把所有数据一并进行发送。大家在试图实现包含语音和视频数据的服务时,不妨考虑一下本专栏的思路。

\section{处理数据}

\subsection{处理服务器的作用}

很显然,处理服务器就是处理接收到的数据的地方。“处理”是一个抽象的词语,例如保存数据,以及转换数据以使其看上去更易懂,还有从多台传感器的数据中发现新的数据,这些都是处理。使用者的目的不同,处理服务器的内容也各异。不过说到数据的处理方法,它可以归纳成以下 4 种:数据分析、数据加工、数据保存以及向设备发出指令(图 2.20)。

\begin{figure}[htbp]
	\centering
	\includegraphics[width=1\linewidth]{39}
	\caption{数据的处理}
	\label{fig:1}
\end{figure}

关于数据的分析和加工,有两种典型的处理方式,分别叫作“批处理”和“流处理”。首先就来说说这个“批处理”和“流处理”。

\subsection{批处理}

批处理的方法是隔一段时间就分批处理一次积攒的数据。一般情况下是先把数据存入数据库里,隔一段时间就从数据库获取数据,执行处理。批处理的重点在于要在规定时间内处理所有数据。因此,数据的数量越多,执行处理的机器性能就得越好。

今后设备的数量将会增加,这一点在第一章已经解释过了。人们需要处理从数量庞大的设备发来的传感器数据和图像等大型数据,这被称为“大数据”。不过,通过使用一种叫作分布式处理平台的平台软件,就能高效地处理数兆、数千兆这种大型数据了。具有代表性的分布式处理平台包括 Hadoop 和 Spark。

\subsubsection{Apache Hadoop}

Apache Hadoop 是一个对大规模数据进行分布式处理的开源框架。Hadoop 有一种叫作 MapReduce 的机制,用来高效处理数据。MapReduce 是一种专门用于在分布式环境下高效处理数据的机制,它基本由 Map、Shuffle、Reduce 这 3 种处理构成(图 2.21)。

\begin{figure}[htbp]
	\centering
	\includegraphics[width=1\linewidth]{40}
	\caption{通过 Hadoop MapReduce 执行批处理}
	\label{fig:1}
\end{figure}

Hadoop 对于每个被称为节点的服务器执行 MapReduce,并统计结果。首先是分割数据,这里的数据指的是各个服务器的处理对象。最初负责分割数据的是 Map。Map 对于每条数据反复执行同一项处理,通过 Map 而发生变更的数据会被移送到下一项处理,即 Shuffle。Shuffle 会跨 Hadoop 的节点来把同种类的数据进行分类。最后,Reduce 把分类好的数据汇总。

也就是说,MapReduce 是一种类似于收集硬币,按种类给硬币分类后再点数的方法。用 Hadoop 执行处理的时候,为了能用 MapReduce 实现处理内容,使用者需要下一番工夫。

另外,Hadoop 还有一种叫分布式文件系统(HDFS)的机制,用于在分布式环境下运行 Hadoop。HDFS 把数据分割并存入多个磁盘里,读取数据时,就从多个磁盘里同时读取分割好的数据。这样一来,跟从一台磁盘里读出巨大的文件相比,这种方法更能高速地进行读取。如上所述,如果使用 MapReduce 和 HDFS 这两种机制,Hadoop 就能高速处理巨型数据。

\subsubsection{Apache Spark}

Apache Spark 也和 Hadoop 一样,是一个分布式处理大规模数据的开源框架。Spark 用一种叫作 RDD(Resilient Distributed Dataset,弹性分布数据集)的数据结构来处理数据(图 2.22)。

\begin{figure}[htbp]
	\centering
	\includegraphics[width=1\linewidth]{41}
	\caption{通过 Spark 执行批处理}
	\label{fig:1}
\end{figure}

RDD 能够把数据放在内存上,不经过磁盘访问也能处理数据。而且 RDD 使用的内存不能被写入,所以要在新的内存上展开处理结果。

通过保持内存之间的关系,就能从必要的时间点开始计算,即使再次计算也不用从头算起。根据这些条件,Spark 在反复处理同一数据时(如机器学习等),就能非常高速地运行了。

对物联网而言,传输的数据都是一些像传感器数据、语音、图像这种比较大的数据。批处理能够存储这些数据,然后导出当天的设备使用情况,以及通过图像处理从拍摄的图像来调查环境的变化。随着设备的增加,想必今后这样的大型数据会越来越多。因此,重要的是学会在批处理中使用我们介绍的分布式处理平台。

\subsection{流处理}

批处理是把数据攒起来,一次性进行处理的方法。相对而言,流处理是不保存数据,按照到达处理服务器的顺序对数据依次进行处理。

想实时对数据做出反应时,流处理是一个很有效的处理方法。因为批处理是把数据积攒之后隔一段时间进行处理,所以从数据到达之后到处理完毕为止,会出现时间延迟。因此,流处理这种把到达的数据逐次进行处理的思路就变得很重要了。此外,流处理基本上是不会保存数据的。只要是被使用过的数据,如果没必要保存,就会直接丢弃。

举个例子,假设有个系统,这个系统会对道路上行驶的车辆的当前位置和车辆雨刷的运转情况进行搜集。

仅凭搜集那些雨刷正在运转的车辆的当前位置,就能够实时确定哪片地区正在下雨。此时,使用者可能想保存下过雨的地区的数据,这时候只要保存处理结果就好,所以原来的传感器数据可以丢掉不要,流处理正适用于这种情况。用流处理平台就能实现流处理。

流处理和批处理一样,也准备了框架。在这里就给大家介绍一下 Spark Streaming 和 Apache Storm 这两个框架。

\subsubsection{Spark Streaming}

Spark Streaming 是作为 Apache Spark(在“批处理”部分介绍过)的库被公开的。通过 Spark Streaming,就能够把 Apache Spark 拿到流处理中来使用(图 2.23)。

\begin{figure}[htbp]
	\centering
	\includegraphics[width=1\linewidth]{42}
	\caption{通过 Spark Streaming 执行流处理}
	\label{fig:1}
\end{figure}

Spark Streaming 是用 RDD 分割数据行的,它通过对分割的数据执行小批量的批处理来实现流处理。输入的数据会被转换成一种叫作 DStream 的细且连续的 RDD。先对一个 RDD 执行 Spark 的批处理,将其转换成别的 RDD,然后按顺序对所有 RDD 反复执行上述处理来实现流处理。

\subsubsection{Apache Storm}

Apache Storm 是用于实现流处理的框架,结构如图 2.24 所示。

\begin{figure}[htbp]
	\centering
	\includegraphics[width=1\linewidth]{43}
	\caption{Apache Storm 的结构}
	\label{fig:1}
\end{figure}

用 Storm 处理的数据叫作 Tuple,这个 Tuple 的流程叫作 Streams。

Storm 的处理过程由 Spout 和 Bolts 两项处理构成,这种结构叫作 Topology。Spout 从其他处理接收到数据的时候,Storm 处理就开始了。Spout 把接收到的数据分割成 Tuple,然后将其流入 Topology 来生成Streams,这就形成了流处理的入口。接下来,Bolts 接收 Spout 以及从其他 Bolts 输出的 Streams,并以 Tuple 为单位处理收到的 Streams,然
后将其作为新的 Streams 输出。可以自由组合 Bolts 之间的连接,也可以根据想执行的处理自由组合 Topology,还可以随意决定 Tuple 使用的数据类型,以及使用 JSON 等数据格式。

\section{存储数据}

\subsection{数据库的作用}

数据库的作用是保存并灵活运用数据(图 2.25)。除此之外,其作用还包括从保存的数据中找出与所指定条件相符的数据。另外,数据库还能把多条数据连在一起,把它们作为一个数据取出。

\begin{figure}[htbp]
	\centering
	\includegraphics[width=1\linewidth]{44}
	\caption{数据库的作用}
	\label{fig:1}
\end{figure}

打个比方,已知与特定传感器相关的 ID,测量时间,以及温度传感器的值。光凭这些数据,是无法理解数据指的是哪个房间的温度的。因此就需要传感器的 ID 以及跟房间名字有关的数据。把这两条数据加在一起,才能知道某房间的温度。

图 2.25 展示的是一个叫作 RDB(关系数据库)的数据库。最近,除了 RDB 以外还出现了一种叫作 NoSQL 的数据库。

RDB 用一种叫作 SQL 的专门用来操作数据库的语言来保存和提取数据。另一方面,NoSQL 则是用 SQL 以外的各种方法来操作数据库。本书还会介绍键值存储(Key-Value Store,简称 KVS)和文档型数据库等种类的数据库。

\subsection{数据库的种类和特征}

这里我们一并为大家说明数据库的种类和特征,以及为了实现物联网服务而处理设备数据时的要点。

\subsubsection{关系数据库}

关系数据库是人们用得最普遍的数据库。如图 2.25 所示,关系数据库具备一种叫作表格的表格型数据结构,其用途在于存储数据库,使用者用 SQL 语言来对其执行数据的提取、插入以及删除。

SQL 是一种非常强大的语言,它能用非常简洁的表述写出命令,来把多个表格联系到一起,搜索符合目标条件的数据。此外,使用者还能通过多种多样的编程语言来使用 SQL。不过一旦确定了表格,就很难更改其结构了。因此,需要仔细考虑设备传来的数据性质再决定结构。

举个例子,假设由于传感器和设备的增加而导致一些必须保存的数据增多,此时,如果表格结构如图 2.26 所示,那么就很难再追加新的数据了。

\begin{figure}[htbp]
	\centering
	\includegraphics[width=1\linewidth]{45}
	\caption{以 RDB 的表格结构示例}
	\label{fig:1}
\end{figure}

在 A 表这种情况下,我们就必须变更表格的条目。而换成 B 表就没必要更改表格本身。不过,这样一来就需要生成一个新的表格。

因此,如图 2.27 所示,要生成一个结构来把所有传感器数据插入同一个字段里。采用这个结构时,即使来了新的传感器数据,也没有必要更改表格结构或是追加新的表格。不过传感器数据的类型必须是统一的,而且,这样一来就会在同一个表格里注册大量的数据。这种情况下,有时就得花一段时间才能从表格里检索到我们需要的数据。为了解决这个麻烦,数据库提供了一个叫作索引的机制。

\begin{figure}[htbp]
	\centering
	\includegraphics[width=1\linewidth]{46}
	\caption{用于保存传感器信息的表格结构示例}
	\label{fig:1}
\end{figure}

以上列举的表格就是一个例子。关于用哪种方法构成表格更好,我们不能一概而论,而是需要先考虑注册的是怎样的数据,以后又会积累多少数据,然后再下决定。

关系数据库也不擅长保存图像和语音等二进制形式的数据。虽然能够用一种叫作 BLOB(Binary Large Object,二进制大对象)的数据形式来达到保存的目的,不过,这也需要另费一番工夫,因为根据用途,有时需要把图像直接保存为文件,把图像的路径单独保存在 RDB 里(图 2.28)。

\begin{figure}[htbp]
	\centering
	\includegraphics[width=1\linewidth]{47}
	\caption{用 RDB 处理图像和语音}
	\label{fig:1}
\end{figure}

数据库把数据保存到硬盘,因此经常会发生对硬盘的访问(磁盘 I/O)。这样一来,这步处理就比其他处理要慢。就系统中而言,这是处理速度方面容易产生瓶颈的一个地方。除了介绍的内容之外,还有一些需要大家注意的地方,希望大家加深对这部分内容的理解并将其灵活运用。

\subsubsection{键值存储}

键值存储属于 NoSQL 数据库的一种。NoSQL 是一种不使用 SQL 的数据库的统称。键值存储,就是把一种叫作“值”(value)的数据值,和能够一对一特定“值”的“键”(key)的集合保存在一起。

此外,还有把数据保存在内存里的键值存储,以及把数据保存在硬盘里的键值存储。前者一方面能够高速保存数据,而另一方面,因为数据是放在内存上的,所以软件停止运行的时候,原先保存的内容就会丢失。因此前者适合作为缓存来使用。而后者保存数据的速度虽然不及前者,但即使软件停止运行,数据也不会丢失。

有一种叫作 Redis 的键值存储,它具备前后两者的性质,在通常情况下它是把数据存储在内存上的,但在任何时间都能够把数据保存到硬盘。因此,它既能够高速执行存储,也能永久保存数据。

\subsubsection{文档型数据库}

文档型数据库和键值存储一样,都属于 NoSQL 数据库的一种。文档型数据库能以 XML 和 JSON 这种结构化文档的格式保存数据。特别是近年来,有一种叫作 MongoDB 的文档型数据库很受欢迎,它以JSON 的格式保存数据(图 2.29)。

\begin{figure}[htbp]
	\centering
	\includegraphics[width=1\linewidth]{48}
	\caption{文档型数据库 MongoDB}
	\label{fig:1}
\end{figure}

MongoDB 能够直接保存 JSON 格式的数据,还能用 JSON 的值进行检索。这样一来,在用 JSON 交换传感器的信息时,就能直接对数据进行保存和使用。即使增加了新的数据条目或是新增了设备,也能直接以JSON 格式保存数据,因此,不需要像 RDB 那样考虑表格的结构。非常
适合用于无法读出设备的数量和数据的种类等情况,以及保存传感器等设备的数据。

\section{控制设备}

\subsection{发送服务器的作用}

发送服务器的目的在于向设备发送数据并控制设备。发送服务器可以使用 2.3 节介绍过的 HTTP、WebSocket、MQTT 协议和数据格式。

发送服务器靠在 1.3.4 节提到过的两种方法来运行,一种是通过设备申请来发送数据的同步传输;另一种是由发送服务器在任意时间发送数据的异步传输。那么,就用 HTTP、WebSocket、MQTT 协议来看看如何实现同步和异步传输。

\subsection{使用 HTTP 发送数据}

要实现数据发送,HTTP 是最简单的方法。在这个方法里,发送服务器是等待接收 HTTP 请求的 Web 服务器。设备向这台服务器申请发送数据,作为响应,服务器把数据发给设备(图 2.30)。

\begin{figure}[htbp]
	\centering
	\includegraphics[width=1\linewidth]{49}
	\caption{通过 HTTP 发送数据}
	\label{fig:1}
\end{figure}

使用者需要定期从设备执行轮询连接。采用此方法的原因主要有以下两个。

原因一:无法确定唯一地址,例如无法给设备设定全局 IP 地址等。这种情况下,发送服务器就不知道应该把数据发送给哪台设备了。

原因二:考虑到设备频繁断电和移动线路的传输费用。此时,设备没有持续连接网络。即使设备已经连接过网络,但只要没有持续连接,那么,即使发送服务器执行了发送数据的操作,也发不到设备那里去(图 2.31)。



\subsection{使用 WebSocket 发送数据}……55
2.6.4
\subsection{使用 MQTT 发送数据}……55
\subsubsection{事例:面向植物工厂的环境控制系统}……56

\chapter{物联网设备……59}
3.1
设备——通向现实世界的接口……60
3.1.1
为什么要学习设备的相关知识……60
3.1.2
连通性带来的变化……60
3.2
物联网设备的结构……63
3.2.1
基本结构……63
3.2.2
微控制器主板的类型和选择方法……68
开源硬件的兴起……80
目
录
IX
3.3
连接“云”与现实世界……80
3.3.1
与全球网络相连接……80
3.3.2
与网关设备的通信方式……81
3.3.3
有线连接……82
3.3.4
无线连接……84
3.3.5
获得电波认证……89
3.4
采集现实世界的信息……89
3.4.1
传感器是什么……89
3.4.2
传感器的机制……90
3.4.3
传感器的利用过程……94
3.4.4
放大传感器的信号……95
3.4.5
把模拟信号转换成数字信号……96
3.4.6
传感器的校准……98
3.4.7
如何选择传感器……100
3.5
反馈给现实世界……103
3.5.1
使用输出设备时的重要事项……103
3.5.2
驱动的作用……104
3.5.3
制作正确的电源……107
3.5.4
把数字信号转换成模拟信号……108
3.6
硬件原型设计……110
3.6.1
原型设计的重要性……110
3.6.2
硬件原型设计的注意事项……111
3.6.3
硬件原型设计的工具……114
挑战制作电路板!……115
3.6.4
原型制作结束之后……116

\chapter{先进的感测技术……119}
4.1
逐步扩张的传感器世界……120
4.2
先进的感测设备……120
X
目
录
4.2.1
RGB-D 传感器……122
4.2.2
自然用户界面……129
4.3
先进的感测系统……132
4.3.1
卫星定位系统……133
4.3.2
准天顶卫星……144
4.3.3
IMES……145
4.3.4
使用了 Wi-Fi 的定位技术……147
4.3.5
Beacon……150
4.3.6
位置信息和物联网的关系……152

\chapter{物联网服务的系统开发……153}
5.1
物联网和系统开发……154
5.1.1
物联网系统开发的问题……154
5.1.2
物联网系统开发的特征……155
5.2
物联网系统开发的流程……157
5.2.1
验证假设阶段……158
5.2.2
系统开发阶段……159
5.2.3
维护应用阶段……159
收益共享……160
5.3
物联网服务的系统开发案例……161
5.3.1
楼层环境监控系统……161
5.3.2
节能监控系统……164
5.4
物联网服务开发的重点……166
5.4.1
设备……167
5.4.2
处理方式设计……175
5.4.3
网络……183
5.4.4
安全性……185
5.4.5
应用与维护……192
5.5
面向物联网服务的系统开发……195

\chapter{物联网与数据分析……197}
6.1
传感器数据与分析……198
6.2
可视化……200
6.3
高级分析……207
6.3.1
高级分析的基础……207
机器学习和数据挖掘……216
6.3.2
用分析算法来发现和预测……216
6.3.3
预测……217
6.4
分析所需要的要素……221
6.4.1
数据分析的基础架构……221
6.4.2
CEP……224
6.4.3
Jubatus……225
分析的难度……227

\chapter{物联网与可穿戴设备……229}
7.1
可穿戴设备的基础……230
7.1.1
物联网和可穿戴设备的关系……230
7.1.2
可穿戴设备市场……233
7.1.3
可穿戴设备的特征……237
7.2
可穿戴设备的种类……239
7.2.1
可穿戴设备的分类……239
7.2.2
眼镜型……243
7.2.3
手表型……248
7.2.4
饰品型……250
7.2.5
按照目的来选择……253
7.3
可穿戴设备的应用……261
7.3.1
可穿戴设备的方便之处……261
7.3.2
消费者应用情景……262
XII
目
录
7.3.3
用于企业领域……265
硬件开发的近期动向……268

\chapter{物联网与机器人……271}
8.1
由设备到机器人……272
8.1.1
机器人——设备的延续……272
8.1.2
机器人的实用范围正在扩大……273
8.1.3
构建机器人系统的关键……274
8.2
利用机器人专用中间件……275
8.2.1
机器人专用中间件的作用……275
8.2.2
RT 中间件……276
8.2.3
ROS……278
8.3
连接到云端的机器人……280
8.3.1
云机器人……280
8.3.2
UNR-PF……281
8.3.3
RoboEarth……284
8.4
物联网和机器人的未来……287
后记……289


\backmatter



\end{document}