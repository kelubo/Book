% PIXIE
% PIXIE.tex

\documentclass[12pt,UTF8]{ctexbook}

% 设置纸张信息。
% 纸张设置配置文件
% 用于定义书籍的页面尺寸和边距

\usepackage[a4paper,twoside]{geometry}
\geometry{
	left=25mm,
	right=20mm,
	top=25mm,
	bottom=25.4mm,
	headsep=1cm, 
    footskip=1cm,
	bindingoffset=10mm
}

% 设置字体,并解决显示难检字问题。
\xeCJKsetup{AutoFallBack=true}
\setCJKmainfont{SimSun}[BoldFont=SimHei, ItalicFont=KaiTi, FallBack=SimSun-ExtB]

% 目录 chapter 级别加点(.)。
\usepackage{titletoc}
\titlecontents{chapter}[0pt]{\vspace{3mm}\bf\addvspace{2pt}\filright}{\contentspush{\thecontentslabel\hspace{0.8em}}}{}{\titlerule*[8pt]{.}\contentspage}

% 设置 part 和 chapter 标题格式。
\ctexset{
	chapter/name={第,章},
	chapter/number={\arabic{chapter}}
}

% 列表项向右偏移。
\usepackage{enumitem}

% 图片相关设置。
\usepackage{graphicx}
\graphicspath{{Images/}}

% 设置署名格式。
\newenvironment{shuming}{\hfill\zihao{4}}

% 注脚每页重新编号,避免编号过大。
\usepackage[perpage]{footmisc}

\title{\heiti\zihao{0} PIXIE}
\author{}
\date{}

\begin{document}

\maketitle
\tableofcontents

\frontmatter

\mainmatter

\chapter{概述}

Pixie 2 CW Transceiver 由美国 HAM W7AMX 设计,并由 BD6CR 引进国内。

Pixie 意为“小精灵、小仙子”,BD6CR 音译为“皮鞋”。

可能是世界上最简单的电报收发信机,只是 2 个常用三极管与一个常见的 LM386 音频功放 IC 就完成了 80 米业余波段(3.5MHz--3.9MHz)或 40 米业余波段(7.0MHz--7.1MHz)200mW 发信机与直放接收机的组合。(可通过更换晶体振荡器和电感电容改变波段。)输出功率 100--300mW 。

Pixie 是一款典型的 QRP(低功率)业余无线电设备。QRP 是指使用 5W 以下的功率进行业余无线电通信的活动,其核心是用尽可能小的功率实现最远的通信距离。这种方式不仅能节省能源,减少电磁干扰,还能锻炼操作者的技能,是业余无线电爱好者中非常流行的活动形式。

利用这么简陋的机器,不少美国和日本的爱好者成功与几百公里外的业余无线电台进行了双向通信并进行了 QSL 卡片确认。山东威海的 BD4OS 于 10 月 15 日晚 10 点左右使用改进过的“皮鞋”以 600mW 左右的输出功率与几百公里外济南的 BA4IN 完成了 CW/SSB 混合模式双向通信。

\begin{figure}[htbp]
	\centering
	\includegraphics[width=0.7\linewidth]{pixie2}
	\caption{Pixie 机照片与电路图}
	\label{fig:1}
\end{figure}

Original “Pixie 2” 
c. 1995 NorCal QRP Club 

NorCal QRP Club(北加州低功率俱乐部)是世界上最具影响力的QRP爱好者组织之一,专注于设计和推广低成本、高性能的低功率业余无线电设备。Pixie 2是该俱乐部最成功的设计之一,因其简单的电路、低廉的成本和可靠的性能而广受欢迎,成为QRP爱好者入门的经典机型。

Based on “Micro-80” design 
c. 1992 RV3GM 

Loosely based on “Foxx” design 
c. 1982 GM3OXX

\chapter{原理}

Pixie 2 由振荡器、混频器、π型滤波器和功率放大器组成。原理图如下:

\textbf{\begin{figure}[htbp]
		\centering
		\includegraphics[width=1\linewidth]{Pix2sch}
		\caption{原理图}
		\label{fig:1}
\end{figure}}

\section{发射机}

一个简单的 CW 发射机:

\begin{figure}[htbp]
	\centering
	\includegraphics[width=0.7\linewidth]{CW1}
	\caption{CW 发射机}
	\label{fig:1}
\end{figure}

缺点:

\begin{itemize}[leftmargin=2cm]
\item 元件发热可能导致漂移 Component heating can cause drift
\item 天线变化可能引发频偏 Antenna changes can pull frequency
\item 容易产生啁啾现象 Prone to Chirp
\end{itemize}

改良型 CW 发射机 - MOPA

\begin{figure}[htbp]
	\centering
	\includegraphics[width=0.7\linewidth]{CW2}
	\caption{改良型 CW 发射机}
	\label{fig:1}
\end{figure}

\section{接收机}

一个简单的无线电接收器

\begin{figure}[htbp]
	\centering
	\includegraphics[width=0.7\linewidth]{CW3}
	\caption{简单的无线电接收器}
	\label{fig:1}
\end{figure}

这被称为直接变频接收机。

\begin{figure}[htbp]
	\centering
	\includegraphics[width=0.7\linewidth]{CW4}
	\caption{直接变频接收机工作原理}
	\label{fig:1}
\end{figure}

\section{电路分析}

\textbf{\begin{figure}[htbp]
		\centering
		\includegraphics[width=1\linewidth]{Pix2sch_1}
		\caption{电路图}
		\label{fig:1}
\end{figure}}

这一条线至关重要,是理解 Pixie 收发器工作原理的关键。按键闭合 = 接地(0伏特),按键开启 ≈ 7.75 伏特。

当电键闭合时,

\textbf{\begin{figure}[htbp]
		\centering
		\includegraphics[width=1\linewidth]{Pix2sch_2}
		\caption{电路图}
		\label{fig:1}
\end{figure}}



\textbf{\begin{figure}[htbp]
		\centering
		\includegraphics[width=1\linewidth]{Pix2sch_3}
		\caption{电路图}
		\label{fig:1}
\end{figure}}

\textbf{\begin{figure}[htbp]
		\centering
		\includegraphics[width=1\linewidth]{Pix2sch_4}
		\caption{电路图}
		\label{fig:1}
\end{figure}}

\textbf{\begin{figure}[htbp]
		\centering
		\includegraphics[width=1\linewidth]{Pix2sch_5}
		\caption{电路图}
		\label{fig:1}
\end{figure}}

\chapter{改进版本}

\chapter{测试}

\begin{figure}[htbp]
	\centering
	\includegraphics[width=0.7\linewidth]{hfpm}
	\caption{带 1W 假负载的高频功率表}
	\label{fig:1}
\end{figure}

对于QRP设备的测试,需要特别关注低功率输出的准确性和稳定性:

1. **输出功率测试**:
   - 使用高精度的功率计,如上图所示的带1W假负载的高频功率表
   - 测试时确保使用50Ω匹配的假负载,避免功率反射
   - Pixie的输出功率通常在100-300mW之间,测试时注意选择合适的功率量程

2. **频率准确性测试**:
   - 使用频率计或频谱分析仪检查工作频率是否准确
   - 与标准频率源(如GPS时钟或已知准确的电台)进行对比
   - QRP通信对频率准确性要求较高,特别是在拥挤的波段中

3. **发射波形质量测试**:
   - 使用示波器或频谱分析仪检查发射信号的波形和频谱
   - 确保CW信号的键控波形干净,没有明显的失真或杂散发射
   - QRP设备由于功率小,杂散发射相对较少,但仍需符合无线电管理规定

4. **接收灵敏度测试**:
   - 使用信号源输入微弱信号,测试接收机的灵敏度
   - Pixie作为直放式接收机,灵敏度相对较低,但对于QRP通信来说已经足够
   - 可以通过与已知性能的接收机对比来评估接收性能

5. **QRP通信测试**:
   - 在实际通联中测试设备的性能,从近距离开始,逐步尝试更远距离
   - 注意选择合适的工作时间和波段,利用良好的传播条件
   - 记录通联结果,包括距离、信号强度、传播模式等

QRP通信的关键是合理利用天线和传播条件,即使是功率很小的设备,也能通过良好的天线和合适的传播模式实现远距离通信。


\backmatter



\end{document}