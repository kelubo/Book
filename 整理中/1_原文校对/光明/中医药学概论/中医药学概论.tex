% 中医药学概论
% 中医药学概论.tex

\documentclass[a4paper,12pt,UTF8,twoside]{ctexbook}

% 设置纸张信息。
\RequirePackage[a4paper]{geometry}
\geometry{
	%textwidth=138mm,
	%textheight=215mm,
	%left=27mm,
	%right=27mm,
	%top=25.4mm, 
	%bottom=25.4mm,
	%headheight=2.17cm,
	%headsep=4mm,
	%footskip=12mm,
	%heightrounded,
	inner=1in,
	outer=1.25in
}

% 设置字体,并解决显示难检字问题。
\xeCJKsetup{AutoFallBack=true}
\setCJKmainfont{SimSun}[BoldFont=SimHei, ItalicFont=KaiTi, FallBack=SimSun-ExtB]

% 目录 chapter 级别加点(.)。
\usepackage{titletoc}
\titlecontents{chapter}[0pt]{\vspace{3mm}\bf\addvspace{2pt}\filright}{\contentspush{\thecontentslabel\hspace{0.8em}}}{}{\titlerule*[8pt]{.}\contentspage}

% 设置 part 和 chapter 标题格式。
\ctexset{
	chapter/name={第,章},
	chapter/number={\chinese{chapter}},
	section/name= {第,节},
	section/number={\chinese{section}},
	subsection/name= {},
	subsection/number={\chinese{subsection}},
	subsubsection/name= {(,)},
	subsubsection/number={\chinese{subsubsection}},
}

% 设置古文原文格式。
\newenvironment{yuanwen}{\bfseries\zihao{4}}

% 设置署名格式。
\newenvironment{shuming}{\hfill\bfseries\zihao{4}}

\title{\heiti\zihao{0} 中医药学概论}
\author{}
\date{}

\begin{document}

\maketitle
\tableofcontents

\frontmatter

光明中医函授大学 主编

白永波 主编

白永波 刘景源 姜典华 樊正伦 编

史常永 审阅

\chapter{导言}

中医教育学,是一门古老而崭新的科学。中医教育的历史,若从师徒授受和医籍编纂算起,已有两千余年。近代史上的中医教育,首推一八八五年浙江陈虬创立的利济医学堂。新中国诞生不久,创办了北京、上海、广州和成都四所中医学院,从而揭开了当代中医教育的序幕,至现在,全国已发展到二十三所。但是,如果把我国中医教育的实践经验加以分析、研究、总结和提炼,升华,揭示它的规律,使之成为一门专门的学科——中医教育学的话,那么,它还处在再创阶段。这就是说,中医教育及其规律存在的历史是悠久的,但论述中医教育及其规律的学科却是崭新的。因此,中医教育工作需要进行探索和研究。

在探索和创建适合我国国情的中医教育的时候,我们必须植根于我们民族文化的肥沃土壤之中,充分重视中医典籍在培育和造就历代医家中的伟大作用。事实上,在长期的历史发展中,逐渐形成了具有中华民族特色的中医药理论体系,它既有丰富临床经验,又有高深的理论基础。历代医学家就是把这些道理传授给他们的弟子,其中部分人经过刻苦自学和临床实践,成为医术高超的医学家,这是我国历代医学家成才之路,亦是中医教育史上培养人才的宝贵经验。这就是我们民族中医教育事业的光辉历史。

在新的历史时期,作为中医教育工作来说,既要给学生打好传统医学的基本功,又要使他们掌握一些新兴的科学知识,使继承与发展得到统一。根据这种认识,我们十分认真地研究和设计了光明中医函授大学的教学计划、教材内容、教学方法与教学手段。归结起来即是:注重打好中医基本功,注意提高中医基本理论水平和培养临床诊治技能,着力培养辨证论治的思维方法,竭诚发挥中医在防病治病中的特长。并在这个基础上,扩大学员知识面。我们把这些要求与思想,全面体现在本校的教材建设中。其目的是使中医人才的知识结构更加合理,以便能担负起继承和发扬祖国医药学防病治病的光荣任务。

在回顾中华医学教育历史,展望现代医学教育的发展趋势以及总结三十多年正反两方面经验的基础上,我们认为,要培养出适合四化需要的合格中医人才,对中医教育的课程设置和教材内容,就要进行必要的改革,建立起为新形势所需要的中医教材。我们正在朝这一方向努力。在认真研究高等中医院校教材和广泛征询中医专家、学者和医务人员意见的基础上,新编了这套较为完整的中医教材,定名为《高等中医函授教材》(包括了二十八门课程)教材的编写人员,由本校选聘知名教授、学者和学有专长者担任,编写时,我们力求各门教材要有鲜明的针对性,在内容上富有实用性,在文字表达上深入浅出、简明易懂,以利便于自学或函授,此外,我们还将根据需要,选编一些辅导材料,以帮助学员(读者)理解教材内容,更好地学取中医知识。

由于教材编写时间仓促,又竭力于继承与创新,不足之处在所难免,敬希学员和广大读者惠赐宝贵意见,以便在再版时修订。

光明中医函授大学教育研究室

一九八五年十月四日

\chapter{编者的话}

《中医药学概论》一书,旨在介绍中医药学的基本概念和基本理论,从“高等中医函授教材”中的顺序看,它是普通基础课向专业基础课过渡的一门桥梁课,属于专业基础课程之一。是深入学习中医基础理论和中医临床各科的基础。内容包括除绪论、附录外,主要有阴阳五行、藏象、经络、病因与病机、诊法、辨证、治则与治法、中药学概述、方剂学概述、中医药学在医疗上的优势等十章。

通过本课程的学习,要求学员能系统地掌握中医基础理论的基本概念及基本内容。具体要求则按教材各章内容统一划分为三级:第一级“掌握”为重点内容。第二级“熟悉”为次重点内容;第三级“了解”为非重点内容。学员在自学过程中,宜按此三级要求,运用唯物论和辩证法的观点,充分发挥独立思考能力,尽可能地理论联系实际,做到分别主次,循序渐进。在此基础上,结合复习思考题进行自我练习和测试,以巩固已经掌握的学习内容,提髙自学效果。本课程计划安排自学120学时,面授30学时,详见各章学习进度安排项内。

任何学科要建立科学理论,都必须形成概念,并运用概念进行判断推理,中医是一门自然科学,当然也不例外。中医学把形式逻辑与辩证逻辑结合起来运用,形成具有特色的思维方法,并凭借这一成功的思维方法,将丰富的临床经验上升为一般规律性的认识而形成理论,并有效地指导着临床实践,因此,中医学的理论,是符合客观实际的正确理论。假若学习中医和从事中医专业的人们,不遵循中医理论体系,不学习、不研究中医的思维方法,不运用中医概念进行判断推理,是掌握不了这门科学的。假若把中医理论只停留在纸上和嘴上,不在临床实践中运用理、法、方、药,不从临床疗效上理解中医理论的正确性,也是掌握不了这门科学的。从中医学的形成与发展的历史来看,吸收其它学科的知识(包括吸收现代医学的知识)能否促进中医的发展,则应以是否符合中医学术自身的发展规律为准则来检验其得失。对初学中医者来说,先把中医学的基本概念弄清楚是十分重要的。

本书在编写过程中、力求保持和发扬中医特色,正确处理继承与发扬的关系,紧紧围绕“培养临床中医师”这一目标。凡属与中医临床密切相关的理论问题,予以详尽论述;凡有益于建立中医临证思维方法的学术观点,则反复阐释。本书在写法上,力求简明扼要,朴实无华,为学习与领会中医学的理论和知识提供阅读上的方便。

本书的编写工作,是在光明中医函授大学教材编辑室的主持下进行的,从选题、设计到取材和写法,经过反复推敲而成。在本节編写过程中,曾得到中医同道刘广洲、韩世湧、王琦、王立等同志的鼓励与帮助,谨此一并致谢。

由于编写时间匆促,加之水平有限,书中错漏之处在所难免,敬请批评指正。

编者

一九八六年七月一日于北京



\mainmatter

% 增加空行
~\\

% 增加字间间隔,适用于三字经、诗文等。
 \qquad  

\chapter{绪 论}

[自学时数] 2学时

[目的要求]

1.了解中医学理论体系的形成发展和它具有的唯物辩证观。

2.掌握中医学的基本特点。

中医药学有数千年的历史,是中华民族长期同疾病斗争的丰富经验的总结,是我国优秀的科技文化的一部分。中医药学是在古代唯物论和辩证法思想的影响和指导下,总结了长期的医疗实践,逐步形成、发展成为具有独特理论体系的一门自然科学。它为中国人民的保健事业和中华民族的繁衍昌盛做出了巨大的贡献。

\section{中医学理论体系的形成和发展}

中医学是以阴阳五行学说、脏象经络学说、气血津液学说等为基础,以整体观为主导思想,以辨证论治为诊疗特点的,研究人体生理、病理,以及疾病的诊断、治疗与预防的一门科学。

春秋战国时期,社会政治、经济、文化都有了显著发展,学术思想也出现了活跃的局面。我国现存的最早的一部医学经典著作——《黄帝内经》(以下简称《内经》),就基本成书于这一时期。《内经》一书主要总结了春秋战国以前的医学理论和临床经验,创立了中医学独特的理论体系,为中医药学的发展奠定了坚实的基础。

《内经》一书,由《素问》和《灵枢》两个部分(各八十一篇)组成。它系统地阐述了人体生理、病理,以及疾病的诊断、治疗和预防等,为中医学理论奠定了基础。主要包括阴阳五行、藏象、经络、病因、病机、诊法、辨证、治则,以及解剖、针灸、养生等方面的内容。《内经》中的许多内容已大大超越了当时的世界医学发展水平。在解剖学方面,关于人体骨骼、血脉长度、内脏器官的大小和容量等的记载,基本上是符合实际情况的,如食管与肠的长度之比,《内经》记载为1 : 35,与现代解剖学描述的1:37,非常接近。在血液循环方面,提出了“心主身之血脉”(《素问·痿论》)的观点,并认识到血液在脉管内是“流行不止,环周不休”(《素问·举痛论》)的。这些认识比英国哈维氏在公元1628年发现血液循环要早一千多年。

《难经》是在《内经》之后出现的又一部医学经典著作,成书于汉以前。《难经》和《内经》一样,都不是出自一时一人之手,而是不同时代的医家的集体创作。《难经》一书,内容也十分丰富,它包括人体生理、病理、诊断、治疗等各个方面,其主要内容是对《内经》的补充和发展。

两汉时期,中医学有了显著的进步和发展。代表这一时期医学发展水平的是东汉末年著名医家张仲景 (公元150 - 219?)的《伤寒杂病论》。《伤寒杂病论》是张仲景运用《内经》、《难经》所阐述的基本理论于临床实际的经验总结,是中医临床学的奠基之作。《伤寒杂病论》即后世之《伤寒论》和《金匮要略》。《伤寒论》和《金匮要略》所确立的六经证治、脏腑证治,开中医学辨证论治的先河,并使汉以前中医学的发展达到了新的高度。张仲景所确立的外感疾病的六经证治,内伤杂病的脏腑证治,基本上形成了中医临床医学的辨证论治的理论体系,为后世临床医学的发展奠定了基础。

汉以后的医家在《内经》、《伤寒杂病论》的基础上,从不同的方面发展了祖国医学。晋·王叔和《脉经》关于脉诊的论述,是对中医诊断学的重要发展。晋·皇甫\footnote{f\v{u}}谧\footnote{m\`i}的《针灸甲乙经》,基本确立了针灸医学的理论和临床体系。隋·巢元方等的《诸病源候论》是中医学第一部关于病因病机及证候学的专书。宋·陈无择的《三因极一病证方论》,发展了《内经》的病因学说,明确提出“内因”、“外因”、“不内外因”的“三因”致病说。宋·钱乙的《小儿药证直诀》,发展了脏腑证治学说。金元时期的百家争鸣、流派纷争的活跃局面,有力地促进了中医学的发展,先后出现了以刘完素、张从正、李杲\footnote{g\v{a}o}、朱震亨为代表的著名医家,即“金元四大家”。刘完素以火热立论,倡“六气皆从火化”、“五志过极 皆能生火” 之说,用药以寒凉为主,后世称为“寒凉派”。他的学术观点给后世温病学说的形成和发展以很大启示。张从正认为病由邪生,“邪去则正安”,主张以攻邪为主,倡汗、吐、下三法,后世称为“攻下派”。李杲认为“内伤脾胃,百病由生”,故治疗以补益脾胃为主,后世称为“补土派”。朱震亨倡“相火论”,认为人体“阳常有余,阴常不足”,治疗以滋阴降火为主,后世称为“养阴派”。总之,刘、张、李、朱四大家,都从不同方面丰富和发展了中医学。明代赵献可、张景岳发展了命门学说,为脏象学说增添了新的内容。

明清时期中医学的突出发展是温病学说的形成。温病学是研究四时温病发生、发展规律,及其诊治方法的一门临床学科。温病学理论源于《内经》、《难经》、《伤寒杂病论》等书,经过汉以后历代医家的发展,至清代温病学家叶天士、吴鞠通,逐步形成一门具有完整理论体系的独立学科。明·吴又可对温疫病原,提出 “非风非寒、非暑非湿,乃天地间别有一种异气所感”。吴又可称这种“异气”为“戾气”或“杂气”,并指出其受邪途径不是从肌表而入,是从口鼻而入。这对温病(特别是温疫)的病因学是一个重大发展。至清代温病学的理论和临床日趋完善,叶天士、吴鞠通等温病学家,创立了以卫气营血和三焦为中心的温病辨证论治体系。

至清代中医学基本上形成了比较系统和完整的临床各科。

新中国成立后,中西医学工作者在中西医结合治疗急腹症、中医治疗急证,以及应用现代科学技术方法研究脏腑、经络实质及针刺麻醉、中药麻醉等方面,都取得了可喜的进展。

\section{中医学的唯物辩证观}

对于哲学和自然科学的关系,恩格斯曾经做过精辟地论述,他在《自然辩证法》一书中说:“不管自然科学家采取什么样的态度,他们总还是在哲学的支配之下”。做为自然科学之一的中国医药学也不例外。中国医药学在其形成和发展的过程中,就直接地渗透着中国古代唯物辩证法—阴阳五行学说的深刻影响。因此,唯物辩证观贯穿于整个中医药学理论体系之中。

\subsection{中医学的唯物观}

\subsubsection{(一)\ 对生命认识的唯物观}

中医学接受了阴阳学说的唯物观,认为世界是物质的,是阴阳二气相互作用的结果。“清阳为天,浊阴为地”(《素问·阴阳应象大论》),就说明天地是物质的,是阴阳之气运动变化形成的。更由于“天地合气,六节分而万物化生矣”(《素问·至真要大论》),万物当然包括人在内,所以“天地合气,命之曰人”(《素问·宝命全形论》、“人以天地之气生,四时之法成”(《素问·宝命全形论》)。中医学把人看成物质世界的一部分,肯定了生命的物质性。这种对生命的朴素的唯物主义的认识,虽然不能也不可能象现代科学那样解决生命起源问题,但在数千年前就具有这种唯物主义的认识,确是难能可贵的。

中医学在认为生命是物质的同时,认为精(气)是生命的本源物质。这种精气先身而生,具有遗传性。《素问·金匮真言论》云:“夫精者,身之本也”,《灵枢·本神》云:“故生之来谓之精,两精相搏谓之神”。这里的精气指父母之精,即先天之精。由于父母之精气相合,才产生新的生命。新的生命产生之后,同样要依靠精气才能生长发育。《灵枢·经脉》云:“人始生,先成精,精成而脑髓生,骨为干,脉为营,筋为刚,肉为墙,皮肤坚而毛髮长”。人生下之后,维持身体发育的精气,称为后天之精。所以,精气是构成和维持生命活动的基本物质。

气是维持生命活动的物质基础,气的运动变化及其伴随发生的能量转化过程称之为气化。气化运动是生命的基本特征,没有气化就没有生命。气化的实质是机体内部阴阳消长转化的矛盾运动。“升降出入,无器不有”、“故非出入,则无以生长壮老已;非升降,则无以生长化收藏”、“出入废则神机化灭,升降息则气立孤危”(《素问·六微旨大论》)。 由此可见,升降出入是气化运动的基本形式,生命活动当然不能例外,生命活动就寓于升降出入的矛盾运动之中。

\subsubsection{(二)\ 形神统一的唯物观}

形神学说是中医学基础理论的重要内容,是在唯物辩证自然观的基础上形成的。形即形体。神,广义是指人体生命活动的外在表现的总称,包括生理、病理外现的征象;狭义是指人的精神思维活动。神的概念很广泛,其含义有三:一指自然界(包括生命)物质运动变化的能量。荀子说:“万物各得其和以生,各得其养以成,不见其事,而见其功,夫谓之神”(《荀子·天论》),自然界万物的化生是“神”的作用,而这种“神”看不见摸不着,但是人们却能认识到它的存在,即在“神”的作用下万物的生成发展与变化;二指人体生命活动。中医学认为人体本身就是一个阴阳对立统一体,阴阳之气的对立统一,决定了生命的运动变化,而生命活动的机能也称之为神。得神者昌,失神者亡。神去则气化停止,生命即告完结。可见,神是人体生命的根本基础;三指人的精神思维活动。《内经》中论述人的意识与思维活动,如:“心者,君主之官,神明出焉”(《素问·灵兰秘典论》),“积神于心,以知往今”(《灵枢·五色》)和“所以任物者谓之心,心有所忆谓之意,意有所存谓之志,因志而存变谓之思,因思而远慕谓之虑,因虑而处物谓之智”(《灵枢·本神》),以及关于心藏神、肺藏魄、肝藏魂、脾藏意、肾藏志等五神脏的论述,指的就是人的精神思维活动之神。

中医学对形神关系的认识。实质就是对物质和精神二者相互关系的认识。形即形体。形体是第一性的,精神是第二性的。形体是本,神是生命的作用。有形体才有生命,有生命才有神产生。神的物质基础是气血,“血气者,人之神”(《素问·八正神明论》),气血又是构成人体的基本物质,而人体脏腑等的功能活动,气血的运行,又受神的主宰。神附于形,形主于神,形存则神存,形灭则神亡。反之,神亡则形亦灭。所以,形与神是相互依附而不可分割的,即“形与神俱”和“形神相得”的形神关系,这是对形神关系的唯物主义认识。

\subsubsection{(三)\ 疾病可知、可防、可治的唯物观}

(三) 疾病可知、可防、可治的唯物观

中医学对人体采取了 唯物主义的可知论观。它认为正常的生命过程是阴阳二气对立统一运动的动态平衡过程,即“阴平阳秘,精神乃治”(《素问·生气通天论》),而一切疾病发生的根本原因是阴阳平衡协调关系的破坏,即阴阳失调。“阴胜则阳病,阳胜则阴病;阳胜则热,阴胜则寒”(《素问·阴阳应象大论》),就是中医学对疾病发生的原因和机理的最高度、最原则地概括。所以,生命是可知的,疾病也是可知的,“言不可知者,未得其术也”(《灵枢·九针十二原》)。同时也认为疾病可防,即“治未病”的预防思想。主张未病之前,重视形体和精神的调养,应当顺应四时阴阳的变化,调整生活起居,还应节制情志活动等,以使人体阴阳气血保持相对协调,从而使正气充盛,防止病邪侵入。“治未病”不仅表现在未病先防,而且主张有病早治,以防传变。

二、中医学的辩证观

二、中医学的辩证观

中医学不仅认为一切事物都有着共同的物质根源—气 (阴阳),而且还认为一切事物都不是一成不变的,各个事物之间也不是孤立的,它们之间是相互联系和相互制约的。因此,中医学中包含着辩证法观点。

人体是一个不断运动变化的有机整体。人的生命活动过程,就是人体阴阳对立双方在不断地矛盾运动中取得统一的过程。在这一过程中,阴阳是“变化之父母,生杀之本始”。中医学认为人与自然界是一个统一整体,与自然界有着极为密切的联系,无论是人体生理还是病理,都直接或间接地与自然界变化影响有关。同时,人体内各脏腑器官组织之间是相互联系、相互影响的统一整体,从而形成了中医学对人体认识的辩证整体观。

精神意识对人体的反作用。中医学一方面认为精神依附于形体而存在,另一方面也强调精神(情志)活动对人体健康与疾病的反作用。在养生保健方面,中医学重视精神调养对健康的积极意义,以及不注意精神调养对人体的危害。同时,指出情志过极对脏腑功能活动的不良影响:“怒伤肝”、“喜伤心”、“思伤脾”、“忧伤肺”、“恐伤肾”(《素问·阴阳应象大论》)。总之,精神对形体的反作用这一认识,即使以现代哲学来衡量,也是正确的辩证的认识论。

治疗学中的辩证观。中医治疗学中包含着辩证法原则。如标本缓急、正治反治、异法方宜、同病异治与异病同治等。

标本缓急:标与本是相对的一组概念,其涵义也比较广泛。标本是一个整体,就其本义来说:本,是根本,本质; 标,是现象,是事物本质所反映的现象。中医学中所谓疾病的标本,则包涵有疾病的本质和现象、原因和结果、原生和派生等矛盾关系的意义。治疗学在认识疾病之“标本”,从而决定论治之“缓急”中,已经涉及到辩证法中的根本矛盾、主要矛盾和次要矛盾的关系问题。一般来说,以正邪而论,人体正气为本,邪气为标,以病因和证候论,则病因为本,证候为标;以发病先后论,则先病、旧病为本,后病、新病为标:以原发续发论,则原发病为本,续发病为标;以病变部位论,病在内、在脏腑为本,病在外、在经络肌表为标。以“标本”论的原则去认识疾病,医生就容易抓住疾病的本质,以及疾病治疗的关键。治疗学的总原则是“治病必求其本”,疾病的标本既明,则具体治疗原则的确立就不难了。即“知标本者,万举万当;不知标本,是谓妄行”(《素问·标本病传论》)。对于疾病标本的缓急论治的一般原则是“急则治其标,缓则治其本”,即标急者,先治标而后治本;标不急者,先治其本后治其标;若标本俱急,则又应标本同治。总之,对于标本的先后缓急,应视疾病实际情况而定。而“标本缓急”的原则,正是符合辩证法原理。

正治反治:正治是针对疾病本质的寒、热、虚、实性质,而分别采用“寒者热之”、“热者寒之”、“虚者补之”、“实者泻之”的治疗方法。反治是指所采用的治法与病证所表现的假象症状性质一致,是针对病证的假象而言,如“寒因寒用”、“热因热用”、“塞因塞用”、“通因通用”(详见本书“治则”)等。反治法,就其本质而言,仍属正治法。这里的正治反治,正是体现了矛盾对立统一的辩证法原理。

异法方宜:中医治疗学认为疾病是复杂多样、变化多端的。即使同一种疾病,由于方域、气候、季节、生活、环境、职业,以及病人的性别、年龄、体质、病程等不同,治法就有区别。如同样是热证病人,由于体质有偏阴偏阳之不同,所用寒凉药的剂量,或作用强度就有区别:偏阳体质者,剂量可大,用药时间可长;偏阴体质者,剂量宜小,用药时间宜适可而止。否则,就会热证虽去又添寒证。清·喻嘉言指出:“凡治病不察五方风气、衣食居处各不相同,一概施治,药不中窍,医之过也”(《医门法律·申明内经法律》)。“异法方宜”的治疗原则,体现了具体问题具体分析的原则,说明中医治疗学已经注意到矛盾的共性与个性,普遍性与特殊性问题,只是未明确提出这一哲学概念。

病治异同:病治异同指同病异治和异病同治两方面。同一疾病,因人、因时、因地的不同,或由于病情的发展、病机的变化、邪正消长的差异,而采取不同的治法,谓之同病异治。不同的疾病,在其病程发展的某一阶段,出现了相同的病机变化,则采取相同的治法,谓之异病同治。“病治异同”的原则,反映了中医治疗学是从事物运动变化的观点,从事物间相互联系的观点,去认识疾病的辩证法思想。

\section{中医学的基本特点}

第三节 中医学的基本特点

中医学是在唯物论和辩证法思想指导下,经过长期的临床实践,逐渐形成的,它具有一整套完整的、独特的理论体系。其基本特点有二:一是整体观, 二是辨证论治。

一、整体观

一、整体观

整体观就是统一性和完整性。中医学的整体观表现在:中医学客观地认识到并非常重视人体本身,及其与自然界(天地)的统一性、完整性。它认为人体是一个有机的整体,构成人体的各组成部分之间,在结构上、生理作用上、病理变化上,是相互联系、相互协调、相互影响的。同时也认识到人体与自然环境是息息相关的,人体的生长发育、健康与疾病,都直接或间接地受到自然环境的一定影响。这种人体自身的整体性和人体与自然环境的统一性的思想,称为整体观。整体观思想贯穿于中医学的生理、病理、诊法、辨证、治疗等各个方面,成为中医学重要特点之一。

(一) 人体是一个有机整体

(一) 人体是一个有机整体

人体是由脏腑、组织、器官所组成。各个脏腑、组织、器官,都有着各自不同的生理作用。中医学通过长期地对人体进行观察研究,认为人体是一个以五脏为中心,并通过经络系统,把六腑、九窍、四肢百骸等全身组织器官联系成有机的整体,并通过精、神、气、血、津液的作用,完成人体的整体功能活动。正因为脏腑、组织、器官之间存在着这种实质性的联系,因而决定了它们之间在生理上的相互联系,病理上的相互影响,即人体在生理、病理上的整体性。从这一认识出发,就形成了中医学对疾病认识和治疗上的整体观。

1.生理上的整体观

在整体观思想指导下,中医学认为人体的脏腑、组织、器官之间,在生理上是一个统一整体。特别是脏腑之间,在生理功能上是相辅相成和相反相成的,即五行学说所说的生克关系。正是这种生克关系维持着脏腑,及脏腑所属组织器官之间协调的动态平衡。同时,任何一个脏腑功能的太过和不及,都将破坏这种整体的平衡联系。而这种整体联系,又是在心的统一指挥下完成的,“心者,君主之官”、“主明则下安”、“主不明则十二官危”(《素问·灵兰秘典论》。经络系统联络全身内外上下,把脏腑、经络、肢体、九窍等联系成为一个有机整体。而气血津液理论和形体与精神统一的学说,则反映了功能与形体的整体性。“阴平阳秘,精神乃治”和“亢则害,承乃制,制则生化”等理论,则说明阴阳学说论述了人体内部阴阳双方的对立统一的相互关系,五行学说阐述了五脏六腑之间的生克制化关系,这些都反映了中医学对人体生理的整体观思想。

2.病理上的整体观

中医学不仅从整体观思想出发去探索人体正常生命活动规律,而且以整体观去认识研究人体疾病的病理机制。由于人体脏腑、组织、器官在生理上的密切联系,必然决定了它们之间在病理变化上的相互影响,即每一脏腑、组织、器官的病变都不是孤立的存在。每一局部的病变都会影响于整体,而整体病变又可反映于某一局部,即某一局部病变可能是整体病变的一部分。这一病理变化上的相互影响,如同生理上的相互联系一样,是通过经络系统、气血津液系统实现的。如某一肌表、经络的病变,可能是与其相关的脏腑病变的反映,或由脏腑病变延及其所属经络、器官。又如肝病可影响于脾、肺,使脾、肺发病。如此等等,都说明了人体疾病病理变化的整体性,同时证明中医学在人体病理变化上的整体观是正确的。

3.诊法与辨证上的整体观

前已述及,人体的内外上下各部是一个有机的统一整体。那么,体内的病变就可反映于体表,这一脏腑的病变就可以影响于与其相关的另一脏腑。因此,就可以通过体表五官、形体、色脉的外在变化,推知体内的病变,这就是中医“以外揣内”的诊法的依据。同时,做为诊法理论基础的脏象学说,更是在整体观指导下形成的。

4.治疗原则上的整体观

人体是一个有机的整体、中医在疾病治疗上,是把局部病变做为整体的一部分去考虑的。既考虑局部病变,又不忘其与整体的关系。在治疗上往往是局部与整体兼顾。如外科阳证疮疡的治疗,不是单纯在疮疡局部用药,而是从整体着眼,进行清热解毒治疗。又如口舌糜烂,心开窍于舌,心与小肠相表里,因此,治疗口舌糜烂,不是只着眼于舌的局部,是从清心泻小肠火的治法入手治疗口舌糜烂。其他,如“从阴引阳,从阳引阴;以右治左,以左治右”(《素问·阴阳应象大论》),“病在上者下取之,病在下者高取之”(《灵枢·终始》)等等,都是在整体观指导下确定的治疗原则。

(二)人与自然界的统一性

(二)人与自然界的统一性

人与自然界的统一性(即人与天地相应,或人与天地相参的天地人一体)是中医学整体观的又一重要内容。人类生活于自然界中,自然界存在着人类赖以生存的必要条件。同时,自然界的变化又直接或间接地影响于人体,使人体产生一定程度的相应的反映。以《内经》为例,就用了大量的篇幅专门论述了“人与天地相应”的整体观思想。

自然界的变化对人体生理的影响

(1)季节气候对人体的影响 一年之中有春温、夏热、长夏湿、秋凉、冬寒的正常季节气候变化。生物在这种季节气候变化之中,则为春生、夏长、长夏化、秋收、冬藏的相应变化。人体也毫不例外地受着季节气候变化的影响,而出现相应的生理性变化。如人在夏天皮腠就疏松而多汗少尿,冬天皮腠固闭而少汗多尿。《灵枢·五癃津液别》说:“天暑、衣厚则腠理开,故汗出”,“天寒腠理闭,气湿不行,水下留于膀胱,则为溺与气”。又如,脉象也随着季节变化的不同而有一定的变化:春脉多弦,夏脉多洪、秋脉多浮、冬脉多沉。就总体来说,春夏脉象浮大,秋冬脉象多沉小。脉象这种浮沉大小的变化,就是人体受到季节变化的影响发生的生理性改变。再如人体气血运行也与季节气候变化有关:春夏温热则气血流行较快而流利,秋冬寒凉则气血运行较慢而涩滞。《素问·八正神明论》说:“天温日明,则人血淖液而卫气浮,故血易泻,气易行;天寒日阴,则人血凝泣而卫气沉”。

(2)昼夜对人体的影响 在昼夜晨昏的阴阳变化过程中,人体也发生相应变化。如人体阳气白天趋向于表,夜间趋向于里。《素问·生气通天论》说:“故阳气者,一日而主外,平旦人气生,日中而阳气隆,日西而阳气已虚,气门乃闭”。又如人体卫气,昼行于阳经,夜行于阴经。《灵枢·大惑论》说:“夫卫气者,昼日常行于阳,夜行于阴”。上述论述,说明了昼夜变化对于人体生理的影响。

(3)地区方域对人体的影响 因地区气候的差异,地理环境和生活习惯的不同,在一定程度上,也影响着人体的生理活动。如江南多湿热,人体腠理多稀疏;北方多燥寒,人体腠理多致密。这是长期生活在这样的环境中,人体对方域影响的反映。同时,长期生活在一地的人们,一旦易地而居,开始总有些不适应之感,经过一定时间后,人体就会发生适应新环境的变化。这也是方域对人体影响所致。

此外,天体(日月)运行,对人体生理过程也有一定影响。中医学认为,人与天地相应。自然界对人体生理产生影响这一事实,是客观存在的。但中医学同时认为,人类只要提高健康水平,减少疾病,就能提高适应自然的能力,能动地改造自然,和自然作斗争。

2.自然界的变化对人体疾病的影响

(1) 季节气候对疾病的影响 四时气候的变化,是生物(包括人类)生存的重要条件之一,但有时也会成为生物生存的不利因素。人类适应外界变化的能力是有限的,如果气候出现异常变化(太过或不及),或人体调节机能低下,而不能对气候变化作出适应性调节,就会发生疾病,或使已发生的疾病出现相应的变化。如季节性多发病,或时令病的出现,就带有明显的季节性。又如某些宿疾,往往在一定的气候条件,或季节交接之际发作,如痺证、哮喘等。中医运气学说更明确提出某气当令之年,有某病流行的预测。

(2)昼夜的变化对疾病的影响 大多数疾病,白天减轻,夜晚加重。因此,《灵枢·顺气一日分为四时》说:“夫百病者,多以旦慧昼安,夕加夜甚”。有些疾病,常在午后或夜间定时周期性发热(潮热),或在上午头痛加重。总之,某些疾病病情的变化与昼夜密切相关。

此外,地域对疾病也有一定影响。如地方性疾病等。

由于中医学中的天人一体观的指导思想,使中医学不仅在认识人体生理、病理时,考虑自然界对人体的影响,而且在确定治疗原则和选方用药时,同样考虑到外界环境的影响。如《素问·八正神明论》就指出应当“因天时而调血气”的原则:“是以天寒无刺,天温无疑,月生无泻,月满无补,月廊空无治,是谓得时而调之”。这句话译成口语是:因此,天气寒冷,不要针刺;天气温和,不要迟疑;月亮初生的时候,不可用泻法;月亮正圆的时候,不可用补法;月黑无光的时候,不可针刺;这就是所谓顺着天时而调治气血的法则。这说明中医学早已注意到天体运行对人体的影响,这一思想是十分宝贵的。

二、辨证论治

二、辨证论治

辨证论治是中医学认识疾病和治疗疾病的基本原则,是中医学对疾病的一种特殊的研究和处理方法,也是中医学的基本特点之一。

证,是机体在疾病发展过程中的某一阶段的病变部位、原因、性质,以及邪正关系的概括。因此,证是反映疾病发展过程中,某一阶段的病变的本质,所以它比症状对疾病的揭示更全面、更深刻、更正确。

所谓辨证,就是运用中医学基本理论,将四诊(望、闻、问、切)所收集的资料(症状、体征),通过分析综合,对疾病进行诊断的过程。论治,又称施治,是根据辨证的结论,确定相应的治疗方法。辨证是确定治疗方法的前提和依据,论治是治疗疾病的手段和方法。辨证论治的过程,就是认识疾病和解决疾病的过程,是中医学理法方药在临床上的具体运用,也是临床医生所掌握的基础理论和治疗经验的综合运用过程。

中医认识和治疗疾病, 是既辨病又辨证。辨证明确,才能抓住病的本质,从而正确的施治。例如感冒病,见发热恶寒,头身疼痛等症状,病属在表,但由于致病邪气和人体正气抗邪能力不同,又可有风寒感冒和风热感冒两种不同的证。辨明证属风寒还是风热,才能确定运用辛温或辛凉解表方法,予以恰当治疗。同时可以看出,辨证施治是不同于一般对症治疗的。

辨证论治作为指导临床诊治疾病的基本法则,由于它能辩证地认识病与证的关系,既可看到一种病可以包括几种不同的证,又看到几种不同的病在其发展过程中可以出现同一种证。因此,在临床治疗时,还可以在辨证论治的原则指导下,采取“同病异治”和“异病同治”的方法来处理。如同是感冒病,但由于发病季节不同,则有冬季感冒和暑季感冒之不同,同时其致病邪气也不完全一样,冬季感冒常感于风寒,暑季感冒则常夹有暑湿之邪,因而治法不同,即所谓“同病异治”。又如久痢脱肛、子宫下垂等,病虽不同,但均属中气下陷之证,故可采用相同的治法,即所谓“异病同治”。由此可见,中医治病主要的不是着眼于“病”的异同,而是注重于“证”的区别。相同的证,可采用基本相同的治法,不同的证,治法就不同,所谓“证同治亦同,证异治亦异”。这种针对疾病发展过程中,不同质的矛盾用不同的方法去解决的法则,就是辨证论治的精髓。

复习思考题

1、简述中医学理论体系的形成和发展过程。

2、中医学中的唯物辩证观表现在哪些方面?

3、简述中医学的基本特点是什么?

第一章 阴阳五行学说



\chapter{第一章 阴阳五行学说}

[自学时数] 8学时

[面授时数] 2学时

[目的要求]

1.了解阴阳五行学说的形成、发展及其在中医学中的应用。

2.掌握阴阳五行的概念及基本内容。

阴阳学说和五行学说,是我国古代哲学基本理论——朴素的唯物论和自发的辩证法的思想基础,是古人通过长期的生活和生产实践,对自然界观察和认识的总结和概括。古人认为,形形色色和千变万化的自然界,都是“阴阳”和“五行”运动变化的结果,阴阳和五行是自然界的根本和源泉。因此,阴阳和五行就成为古人认识自然界和解释自然界的世界观和方法论。《类经附翼·医易》说:“天地之道,以阴阳二气以造化万物;人生之理,以阴阳二气而长养百骸”。

阴阳学说认为自然界是物质的,是在阴阳二气相互作用的情况下,发生、发展和变化的,没有阴阳就没有自然界。五行学说认为,木、火、土、金、水是构成自然界的五种基本物质。形形色色的自然界,就其构成物质来说,都可以高度抽象、概括、归纳为五个方面,而此五个方面又是相互作用、运动变化的。

作为自然科学的中国医学,是在吸收了当时的先进的唯物主义的哲学理论——阴阳五行学说之后,才形成了具有独特理论体系的中医药学。阴阳五行学说已经成为中医学认识和阐释人体生命过程和疾病过程,并指导诊断治疗(辨证施治)的理论和说理工具。因此,学习和研究中医学,必须首先通晓阴阳五行学说的基本原理。

\section{第一节 阴阳学说}

第一节 阴阳学说

古人在长期的生活实践中,逐渐认识到这样一种规律,即自然界的每一种事物都有与其相对应的另一方面事物存在,如男女、水火、昼夜、日月、寒暑、晴阴等,因而形成了最早的阴阳概念。在成书于西周时的《易经》中就出现了明确的阴阳概念,—(阳爻)和--(阴爻),并由此演化为八卦。不过此时并没有使用阴阳两字。至西周末期才使用阴和阳二字来表示相联系而又对立的两方面事物。如《国语·周语》中就记载了伯阳父用阴阳运动失去平衡来解释地震产生的原因:“阳伏而不能出,阴迫而不能蒸,于是有地震”。《左传·昭公元年》还记载了秦国医生医和,以阴阳代表寒热,他说:“阴淫寒疾,阳淫热疾”。在此期间并有一些学者,阐述了阴阳的对立与相互转化关系。如《国语·越语》说:“阳至而阴,阴至而阳。日困而还,月盈而匡”。《荀子》说:“天地之变,阴阳之化”。《易传》在总结自然界运动变化规律时提出:“一阴一阳之谓道”,等等。总之,阴阳学说在春秋战国时期已经形成并盛行,对于当时的哲学和自然科学产生了广泛的影响。

一、阴阳的基本属性

一、阴阳的基本属性

阴阳二字的本义是指日光的向背,即向日者为阳,背日者为阴。《谷梁传·僖公二十八年》说:“水北为阳,山南为阳;水之南,山之北为阴”。但作为阴阳学说中的“阴阳”,已经不是“日光向背”的涵义了,而是一类事物属性的抽象和概括。因此,它们指的不是某一个特定的具体的事物,而是抽象的表示事物属性形态特征的概念。《灵枢·阴阳系日月》已经明确指出:“且夫阴阳者,有名而无形”。明·赵献可也说:“阴阳者,虚名也”(《医贯·阴阳论》)。

阴阳学说认为,自然界一切事物(包括人体)均可根据它们各自的特性,按阴阳属性进行归类。根据前人的论述,阴阳的基本属性可归纳如下:凡是动的、升的、浮的、上的、外的、热的、明的、无形的等等属阳;凡是静的、降的、沉的、下的、内的、寒的、暗的、有形的等等属阴。即天在上、无形属阳;地在下、有形属阴。日是明亮的、热的属阳;月是晦暗的、寒冷的属阴。疾病中的热证、实证、表证属阳;寒证、虚证、里证属阴。阴阳的基本属性是绝对的、固定的,但事物本身的阴阳属性却是相对的、变动的。这种事物本身的阴阳属性的相对性和变动性,一方面表现为在一定条件下相互关联的相对事物之间可以互相转化,即阴可以转化为阳、阳可以转化为阴;另一方面表现为一事物内部阴阳属性的无限可分性,即“阴中有阳,阳中有阴,阴阳之中复各有阴阳”(《类经·阴阳类》)。我们在明确了阴阳学说中阴阳的属性(即阴阳质的规定性)和具体事物本身的阴阳属性是不同的两回事之后,就能容易理解为什么某一总体属阳的事物,而又包含阴的属性,总体属阴的事物又包含阳的属性;及某事物由于其相对方面的变化,其阴阳属性也随之而发生相应的转化的原理。

二、阴阳的相互关系

二、阴阳的相互关系

阴阳的相互关系,概括起来,可以归纳为对立和统一两个方面。阴阳学说认为自然界的一切都是由阴阳二气所构成。而事物的发展变化是阴阳二气对立统一的表现形式,即《素问·阴阳应象大论》所概括的:“阴阳者,天地之道也,万物之纲纪,变化之父母,生杀之本始,神明之府也”。这一观点,正是阴阳学说宝贵的唯物主义思想所在。

阴阳两方面的相互对立,主要表现在阴阳之间的相互制约、相互斗争。阴阳两方面的统一,主要表现在阴阳之间的相互依存、相互为用、相互转化,以及阴阳的相互包涵和无限可分性。
(一)阴阳的相互对立与消长平衡

阴阳学说认为自然界一切事物无不存在着相互对立的阴阳两个方面。如上与下、南与北、动与静、寒与热,以及动物的雌与雄等等,它们之间虽然或者为一事物内部的两个方面,或为相关联的两个事物,但它们有一个共同点即相互对立。上下对立、南北对立、动静对立、寒热对立,雌雄对立。但恰恰是这些对立,才维持着事物的存在——发生、发展、变化。没有上下对立、南北对立,就没有空间方位;没有动静对立,就没有事物的运动状态;没有寒热、就没有四季;没有雌雄,就没有动物的繁衍等等。这些事例都充分证明了,阴阳学说认为自然界一切事物都存在阴阳对立的两个方面的观点是完全正确的。又如:四季的变化就是阴阳(寒凉温热)对立的结果,没有阴阳(寒热,温凉)的对立,也就没有四季的差别。同样,人体的生命过程,也包含着一个阴阳对立制约的过程。就生命过程的总体来说,“阴平阳秘”(《素问·生气通天论》),即阴阳平衡,才是无病的常人的根本保证。但是,这种“平衡”并不是静止的、一成不变的,而是动态中的平衡。由于阴阳二气的对立制约,使对立双方中不致出现一方过亢的现象,使阴阳二气的变化水平维持在一个正常生命过程的允许范围之内。如果这种对立制约关系被破坏(或内因、或外因)即双方互不受对方制约,就必然出现一方偏盛或偏衰(或阳盛、或阴盛,从而导致阴不胜其阳,或阳不胜其阴),使阴阳失去平衡状态(阴阳不和),或阳胜则阴病,或阴胜则阳病,疾病于是产生。就每一个具体脏腑来说,同样是阴阳对立制约,才得以维持脏腑正常的气化功能。如肝阴肝阳,肾阴肾阳等,正常情况下,阴阳双方互相制约从而不亢不衰。如果外感于六淫,或内伤于七情,使阴阳一方偏衰,就会失去制约对方的能力,就会出现另一方的偏胜。肝阴不足必然肝阳偏亢;肾阴不足,必然导致肾阳偏亢。已经偏亢的肾阳、肝阳,又反过来制约已经不足的肾阴、肝阴。于是,出现盛者更盛,衰者更衰。还有临床上常见的热证病人,必然出现舌苔黄燥、口干口渴、喜饮、便燥等津液不足的症状,这更是典型的阳盛伤阴的事例。

治疗疾病的过程,也是一个阴阳对立制约的过程。如用阳热性药物治疗阴寒证,即是以热(药)治寒(证),以阳制阴的例证。

由此可见,阴阳的对立制约无论在自然界及人体,是普遍存在的。在人体,无论是正常生命过程,还是疾病及治疗过程等,都包涵着阴阳的对立制约。

阴阳的相互对立,还包含着阴阳的互为消长。因为阴衰(消)必然失去对阳的制约,致使阳亢盛(长);阳衰(消),必然导致阴亢盛(长)。即“阳长则阴消,阳退则阴进”、“一阴一阳互为进退”(《类经·阴阳类》)。如春、夏、秋、冬四季,有温、热、凉、寒的变化,春夏之温热,是由于春夏阳气逐渐亢盛(长),抑制(对立制约)了秋冬之阴气,使秋冬阴寒之气衰退(消);秋冬之所以寒冷,是因为秋冬时节,阴气逐渐亢盛,抑制(制约)了春夏阳热之气,使春夏阳热之气衰退(消)的结果。《素问·脉要精微论》所记载的“冬至四十五日,阳气微上,阴气微下;夏至四十五日,阴气微上,阳气微下”,讲的就是一年四季中阴阳互为消长的渐变过程。一天之中阴阳消长的变化也一样,夜半之后阴气渐退,阳气渐进,至中午,阳气最盛达到极点,阳气便开始衰退,阴气开始复升,至夜半,阴气最盛达到极点之后,又开始衰退,如此周而复始。人体疾病过程更是如此,阴寒过盛(阴长)的病人,必然出现阳热不足(阳消)之证;火热亢盛(阳长)的病人,必然出现阴津不足(阴消)之证。

因此,阴阳的相互对立、相互制约、互为消长,是自然界以及人体生命和疾病过程中的一个普遍规律。尽管阴阳平衡是相对的,对立制约是绝对的,但在人体为了某种目的和需要还要维护这种平衡。医学家的最终任务是千方百计地恢复和稳定这种“平衡”,以达到健康和延长寿命的目的。如果不能恢复和维护人体的这种平衡,就意味着疾病的不可挽回和死亡的到来。
(二) 阴阳的互根互用

阴阳是相互对立的,但又是相互依存的,即二者都不能脱离对方而独立存在。没有阳,就没有阴;没有阴,就没有阳。而阴阳之间的相互依存——互根互用,则是阴阳对立的统一。所谓互根互用,即是说阴阳双方互以对方为存在条件。如:上是对下而言,没有下,就无所谓上;外是对于内而言,没有内,就无所谓外;实是对于虚而言,没有虚,就无所谓实。上下、内外、虚实,没有一方,另一方就不存在了。这说明阴阳二者是相对立而存在的。不仅如此,阴阳又是相互包涵的。张景岳引朱子曰:“阴气流行则为阳,阳气凝集则为阴”。如水(阴)蒸发则化为气(阳),气(阳)凝结则化为水(阴)。这就是“阴根于阳,阳根于阴”(《类经图翼·运气上》)和阴可化为阳,阳可化为阴,阳从阴中化,阴从阳中生的阴阳互相包涵,并互根互用的道理。再如人体气血,气为阳,血为阴。气血之间的正常关系是血载气,气行血,气为血之帅,血为气之舍。血的运行靠气的推动作用,而气又要依附于血而存。所以,临床上气病或血病患者,在经过一定阶段后,往往出现气血并病的证候。气虚者,血亦多虚;气滞者,多见血瘀;血瘀者,必气滞等。至于骤然气脱或血脱者,更多见气血并病了。因此,气血(阴阳)是相互依存、互根互用的。再以血病治疗为例,中医有“治血必先理气”和“血脱益气”的治疗原则。如血虚证,补血可以不用四物汤,而用当归补血汤,当归补血汤由黄芪一两为君,以补气为主;当归二钱为臣,以补血为辅。以补气达到生血的目的。其理论根据就是“阳生阴长”。所以,张景岳总结说:“阴无阳不生,阳无阴不成,而阴阳之气,本同一体”(《类经图翼·运气上》)。

对于阴阳的相互依存、互根互用的关系,《内经》作了明确地阐述,《素问·阴阳应象大论》说“阴在内,阳之守也;阳在外,阴之使也”,《素问·生气通天论》说:“阴者,藏精而起亟也;阳者,卫外而为固也”。就是说阴精藏于内,有赖于在外之阳气的卫护,方能固藏而不外泄;而在外之阳气,又须阴精不断地转化为气予以补充。由于阴阳是相互依存的,所以,一方的太过或不及,都将直接影响到另一方的正常存在。临床上的阴盛于内,格阳于外(阴盛格阳),是由于阴寒过盛、破坏了阴阳的依存关系,从而逼阳外达所致。例如,阳虚冷汗,是由于阳虚导致卫外之气不能固表,使阴津不能安守于内而外泄;大汗亡阳,是由于阴津大亏,阴不敛阳,阳无所依而外脱。治疗的原则,前者补阳以固阴,后者固阴以敛阳。他如,形(阴)与神(阳),精(阴)与气(阳)等也同样存在着相互依存的密切关系,形存则神在,形亡则神灭;精化生气,气又生精,无精则不化气,无气则不生精。
(三) 阴阳的相互转化

阴阳的相互转化是指对立的阴阳双方,在一定条件下可以分别向对方转化。即阴可以转化为阳,阳可以转化为阴。阴阳的相互转化,在自然界(天地间),及自然界中的人,都普遍存在着这一过程。自然界中一年四季,寒凉、温热的变化,以及一天中的“昼夜”的变化,就是一个阴阳转化的过程。春夏为阳,秋冬为阴,由春夏而至秋冬,是由阳转化为阴的过程;由秋冬而至春夏,则是由阴转化为阳的过程。同样道理,白昼为阳,黑夜为阴,昼夜的变化,也是阴阳的转化。

人体的正常生命过程和疾病过程,同样是一个普遍的阴阳转化过程。如生命过程中,需要有精(阴)和气(阳),阴精是生命的物质基础,而阳气是功能活动(这里的阴精和阳气都是相对而言)阴精和阳气既有互根互用的关系,又有一个相互转化的关系。饮食入胃,在阳气(脾气)的作用下,方能转化为阴精,而阳气又是由阴精所化生,并从阴精化生中得到不断地补充。所以《素问·阴阳应象大论》说:“气归精”,“精化为气”。

疾病过程中,阳证可以转化为阴证;热证可以转化为寒证,实证可以转化为虚证;表证可以转化为里证等等。反之亦然。如:外感寒邪,初时表现为发热、恶寒、无汗、脉浮紧等,属表寒证,由于未能及时治疗,则寒邪进一步由表入里化热,出现但热不寒,口渴苔黄,脉数等证候,则为里热证。又如:中医常说“久病必虚”,初病时,由于正气尚较充盛,因而病属实证,病久则正气渐耗,病即由实转虚。上述由寒转热,由实转虚的例子,证明了疾病过程中的阴阳是可以相互转化的。

《内经》已经明确认识并充分肯定了阴阳的相互转化问题。《灵枢·论疾诊尺》说:“四时之变,寒暑之盛。重阴必阳,重阳必阴,故阴主寒,阳主热。故寒甚则热,热甚则寒。故曰:寒生热、热生寒,此阴阳之变也”,《素问·阴阳应象大论》说:“寒极生热、热极生寒”,以及《素问·生气通天论》中所说:“冬伤于寒,春必温病”等,都说明了一个阴阳的相互转化问题。

《内经》对阴阳转化的论述中,强调了一个“物极必反”的观点,上述引文中的“重”、“甚”、“极”,意义是一个,即“极”,是说事物发展到极点时,就会向其相反方面转化。但是,这种相互转化是在一定条件下的转化,没有一定的条件,是不能转化的。而“极”是转化的原因,它不是条件。就疾病过程来说,阴阳转化的条件应当是人体正气盛衰、禀赋的偏阴偏阳、治疗得是否及时和恰当等。如邪气初中人体,表现为实证,在人体正气充盛、并治疗及时而得当的情况下,可以始终停留于实证阶段并逐渐向愈;如果正气不足,又治疗不当,则实证即可转化为虚证。又如,同样为水饮(阴)之邪中人,但由于个体素日禀赋不同,可出现不同的证候,素日禀赋偏阳的人,水饮之邪可从阳化热而为热证,素日禀赋偏阴的人,水饮即从阴化寒而为寒证。所以,根据这一原则,临床上的任何治疗手段,其实质都在于改变或促进阴阳的相互转化方向,使之向有利于人体健康和解除疾病痛苦方面转化。

阴阳学说强调了“物极必反”,即《内经》中所阐述的“重阴必阳,重阳必阴”、“寒极生热、热极生寒”的观点,在一定情况下,是正确的。但是,“变”的产生,不一定都是在“阴极”、“阳极”时发生,如上述的情况,并没有“阴极”、“阳极”,但由于外部条件(正气、禀赋、治疗等)发生变化,阴阳亦同样会出现转化。
(四) 阴阳是无限可分的

阴阳是无限可分的。阴中有阴阳,阳中有阴阳,阴阳之中各有阴阳。一句话,就是在事物的不同层次中均各有阴阳。如一天之中,昼为阳,则平旦至日中为阳中之阳;日中至黄昏为阳中之阴;夜为阴,则合夜至鸡鸣为阴中之阴,鸡鸣至平旦为阴中之阳。又如五脏六腑中,五脏属阴,但由于心肺居于膈上故又属阳,因而是阴中之阳;肝脾肾居于膈下,下为阴,故又为阴中之阴。再就心肺之间来说,心属阳,肺属阴等等。

再如,按阴阳的基本属性,外为阳、内为阴。表属阳,寒属阴,故表寒证为阳中之阴;表属阳,热为阳,故表热证为阳中之阳;内属阴,热属阳,而里热证即为阴中之阳;里属阴,寒属阴,里寒证即为阴中之阴。诸如此类不胜枚举。因此说:“内有阴阳,外亦有阴阳”(《灵枢·寿夭刚柔》),“阴中有阴,阳中有阳”(《素问·金匮真言论》),“阴阳之中,又有阴阳”(《类经图翼·运气》),“此阴阳之道,所以无穷”(《类经·阴阳类》)。

三、阴阳学说在中医学中的应用

三、阴阳学说在中医学中的应用

阴阳学说,贯穿于中医学理论和临床体系的各个方面,用以认识和阐述人体的形体、脏腑、经络、生命过程、疾病过程、以及辨证治疗等等。下面仅就主要方面,作一简要说明。
(一)概括形体、脏腑的部位特点及脏腑功能特点

人体是一个统一的整体,根据阴阳对立统一的观点,人体内外上下无不充满着阴阳对立的两方面。所以,不仅人体这一整体,而且组成这一整体的各部分,均可按不同的阴阳属性,划分为若干相互对立的阴阳两部分。

以部位概括形体的阴阳属性。就整体来说,以上下而论,人体上半部属阳,下半部属阴。以内外而论,体内属阴,肌表为阳;皮肤为阳,筋骨为阴。以局部来说,背为阳、腹为阴;四肢外侧为阳,内侧为阴。

以功能和部位特点概括脏腑的阴阳属性:肝、心、脾、肺、肾五脏,总的功能特点是贮藏精气,而没有传送饮食的作用,故主静,属阴;胆、胃、大肠、小肠、膀胱、三焦六腑,总的功能特点是传送并消化饮食,故主动,属阳。因此,中医学中有迳以阴、阳代称脏腑的,如:《灵枢·阴阳清浊》说:“清者注阴,浊者注阳”,此处阴、阳即指五脏和六腑而言。

总之,人体的上下、内外均可按不同的阴阳属性加以概括和说明。
(二)概括和阐述人体的生命过程

阴阳学说认为,“人生有形,不离阴阳”(《素问·宝命全形论》),“人生之理,以阴阳二气而长养百骸”(《类经附翼·医易》)。就是说,人的有生命的形体产生于阴阳二气,所以说:“生之本,本于阴阳”(《素问·生气通天论》),而生长发育同样是阴阳二气的作用,这就把人体整个生命过程概括为阴阳二气运动变化的过程,是阴阳二气对立统一、协调平衡的结果。“阴平阳秘,精神乃治”(《素问·生气通天论》)。同时认为,生命的结束,是阴阳离决的结果,“阴阳离决,精气乃绝”(《素问·生气通天论》),这是阴阳学说对生命的总观点。

阴阳学说不仅认为阴阳二气是人体整个生命的根本基础,而且认为人体中的每一脏腑器官都有阴阳二气存在,这就是阴精和阳气。如心阴心阳、肝阴肝阳、脾阴脾阳、肺阴肺阳、肾阴肾阳、及六腑等各存在阴阳二气。同时,又将先天得之于父母的阴阳二气,称为元阴元阳;此外,肾阴、肾阳又称为真阴真阳。这里,无论整个人体的阴阳二气,还是脏腑器官等的阴阳二气,阴精是指生命的物质基础,阳气指的是生命过程的气化功能,即“阳化气,阴成形”(《素问·阴阳应象大论》)。

以阴阳阐述人体生理过程:在概括人体阴阳二气的生理作用时,中医学认为:清阳出上窍,浊阴出下窍。清阳发腠理, 浊阴走五脏。清阳实四肢,浊阴归六腑。阴精藏于内,阳气卫护于外。营血行于脉中,卫气行于脉外。以阴阳概括某些生命物质和生命现象:营、血、精、津、液等有形物质属阴;卫、气、神等生命活动为阳。
(三) 概括和说明病机病证的阴阳属性

中医学认为,疾病是阴阳双方动态平衡关系的失调。人体疾病尽管千变万化,并有内、外、妇、儿之分,寒、热、虚、实之别,脏腑气血之异,归根结底,都可以用阴阳失调来加以分析、归纳。即,或阴盛,或阳盛,或阴衰,或阳衰,或阴阳两衰,或阴阳俱盛。就是说,不管致病邪气是外感六淫,还是内伤七情,都是邪气中人后,导致了阴阳平衡失调,而引起疾病的发生。

1.阴阳偏盛

即阴胜、阳胜、阴阳俱胜。人体是阴阳二气所构成;但是,如果因某种因素使阴阳二气亢盛,即成为危害人体的邪气,即过则为害。《素问·阴阳应象大论》说:“少火生气”、“壮火散气”,李东垣认为:气有余便是火。正常之火与气(阳),是人体气化过程不可缺少的,但超过一定限度(壮、有余),就会反过来损伤人体正气。因此,阴胜、阳胜、阴阳俱胜,是阴阳二气亢盛超过了正常水平的病变。阴阳偏盛,既是致病邪气引起的病变的性质,又是进一步损伤人体的邪气。《素问·阴阳应象大论》对于阴阳偏盛所导致的病变做了如下的概括:“阴胜则阳病,阳胜则阴病。阳胜则热,阴胜则寒“。由于阴阳的对立制约,所以,阴胜、阳胜,必然导致相应的阳衰、阴衰。阴主生寒,阳主生热。阴胜则阳衰,阳衰则不能生热以制寒,故阴胜则为寒;阳胜则阴衰,阴衰则不能生寒以制热,故阳胜则为热。

阴胜、阳胜,有绝对和相对之分。所谓绝对,是指对立的一方不衰,而另一方独盛;所谓相对,是指对立的一方衰退,而使另一方相对亢盛。所以,临床上,阴胜、阳胜所致的寒热,有虚实之异。即阴或阳某一方绝对亢盛及其所致之寒或热,为实;阴或阳某一方相对亢盛及其所致的寒或热,为虚。

2.阴阳偏衰

即阴亏(虚)、阳亏(虚)、阴阳两亏(虚)。从常见病、多发病的所见证候分析,证属阴阳偏胜的可有绝对和相对之分,但证属阴阳偏衰的却多见绝对偏衰,而相对偏衰是很短暂的过程。因为阴的绝对偏盛必定耗伤阳气,阳的绝对偏胜亦必定耗伤阴气,而阴和阳的相对偏胜又是以阴和阳的绝对偏衰为基础的。

所以,一般来说,阴阳偏衰是阴阳低于人体正常水平以下的病变。而所谓正常水平,又是因人而异的。由于阴阳之间存在着对立制约关系,所以,一方偏衰,即失去对另一方的制约,而使另一方偏胜。对于阴阳偏衰所导致的病变,《素问·调经论》指出:“阳虚生外寒,阴虚生内热”。阳虚不能制阴,则阴相对偏亢,故生寒;阴虚不能制阳,则阳便相对偏亢,故生热。这里的寒、热是属虚寒、虚热。与前所述阴、阳绝对偏胜之实寒、实热自然是不同的。

根据阴阳对立制约和阴阳互根的规律,阴阳一方的亏损,在经过一定阶段后,必然导致另一方的亏损,此即所谓“阴损及阳,阳损及阴”,最后导致阴阳两亏。如临床上的阳虚水饮证候,反而出现口干,口渴,但渴不欲多饮,或喜热饮等表现,是因为阳虚不能化水为津,津液不足,故出现口干等现象。最后阳气与阴津均出现不足之象,是属阳损及阴。同样,阴虚发展到一定程度,亦可导致阳虚。如:阴血虚脱的大汗,使阴液外泄,以致阳无所依,而造成阳气泄越于外的亡阳,是属阴损及阳,最后导致阴阳两亡。

阴阳两亏并不完全是阴阳双方处于低水平的动态平衡,而是或侧重于阳亏,或侧重于阴亏,临床上应视具体证候加以辨别。

3.阴阳转化

阴阳转化,是指疾病过程中,属阴的病变,在一定条件下,可以转化为属阳的病变;同样,属阳的病变,在一定条件下,可以转化为阴的病变。即阴证可以转化为阳证;阳证可以转化为阴证(参见“阴阳的相互关系”一节)。
(四)概括辨证论治原则

1.用阴阳归纳四诊资料

用阴阳归纳四诊所获取的疾病资料,有利于认识疾病表现的本质,是进行辨证的必要准备。疾病的证候表现千端万绪,如何将纷繁的病情表现,理出一个头绪,从中找出带规律性的本质性的方面来,这对于每个医生,是天天都要碰到的问题。而用阴阳去概括归纳这些通过四诊得到的资料,就会使医生执简驭繁,准确地认识每一证候表现所反映的证候本质。

中医学根据阴阳的基本属性,对四诊资料的归纳举例说明如下:

望诊:病人烦躁不安者属阳,闭目静卧者属阴。色泽:润泽明亮属阳,枯涩晦暗属阴。面色:赤、黄属阳,青、黑、白属阴。水肿:身半以上先肿为阳,身半以下先肿为阴。疮疡:红肿热痛为阳,肤色青暗、疮面下陷、不热为阴。小便:尿赤、黄而短为阳,青白而长为阴等等。

闻诊:语声宏亮、呼吸气粗为阳,语声低微、呼吸无力、气息微弱、不相接续为阴。

问诊:恶寒为阴,恶热为阳。渴喜热饮为阴,渴喜冷饮为阳。身热为阳,身凉为阴。

切诊:脉诊中,以部位分,寸脉为阳,尺脉为阴。以脉象分,浮、数、滑、洪、大、弦、实等为阳;沉、迟、涩、微、小、芤、结等为阴。

以阴阳为纲,分析、归纳四诊资料,是正确认识疾病本质和辨证的基础。所以阴阳学说的理论原则,对临床诊断具有十分重要的意义。正如张景岳所说:“凡诊病施治,必须先审阴阳,乃为医道之纲领,阴阳无谬,治焉有差?医道虽繁,而可以一言敝之者,曰阴阳而已”(《景岳全书·传忠录》)。

2.阴阳为八纲辨证的总纲,亦是诸证的总纲

临床上无论是外感或内伤所引起的疾病,都可以用阴阳、表里、虚实、寒热八证(即八纲)加以概括归类。而表里、虚实、寒热又可以按阴阳的基本属性统于阴阳两证之下,即表、实、热为阳,里、虚、寒为阴。所以,阴阳又是八纲辨证的总纲,即阴阳统帅诸证。因此,临床上的病证尽管多种多样(或外感、或内伤;或脏腑经络,或气血津液等),概括起来,不过阴阳两大类。同时,阴阳还用于病证命名,如伤寒六经病证的太阳病、阳明病、少阳病(三阳证),太阴病、少阴病、厥阴病(三阴病)等。它即是对病证的命名,又是病证属性的概括。

总之,阴阳在分析归纳四诊资料的性质,概括病证属性方面,都有着十分重要的作用,是分析和认识疾病的纲领。

3.确立论治原则

中医学认为阴阳二气的协调平衡,是维持人体正常生命活动的根本因素。尽管临床具体治疗原则种种不一,方药变化多端,其根本目的在于调整、恢复阴阳的平衡。即所谓:“谨察阴阳而调之,以平为期”(《素问·至真要大论》)。

阴阳的失调,不外乎阴阳偏盛或偏衰。偏盛即为有余,偏衰即为不足。有余者去之,不足者补之。根据阴阳相互对立制约的规律,对于阴盛者(阴盛则寒),则以阳制阴,如阴寒证,用阳热类药物,以祛除阴寒之气;对于阳偏盛者(阳盛则热),则以阴制阳,即以阴寒(寒凉)类药物,祛除其阳热之气。这就是“寒者热之,热者寒之”的治疗原则。然而,由于阴阳盛衰有相对的一面,所以阴盛则阳病,阳盛则阴病。如果临床上遇有阴阳偏盛(邪)之证时,要考虑与辨别是否同时存在阴阳相对偏衰(正)的一方。假若阴阳偏胜偏衰同时存在,治疗时应据情兼顾。如阳盛则热,热胜则伤阴,因此,阳热亢盛之证,每有伤阴兼证,治疗时,一方面以泻火为主,另一方面,亦须适当补阴。

阴阳偏衰者,属于不足之证,包括阳虚,阴虚,阴阳两虚。不足者补之,相应的治疗原则即:补阳、补阴、或阴阳两补。在阴或阳偏衰(正气)的证候中,又当考虑病人是否存在阴阳相对偏盛(邪气)的一面,如阴虚不能制阳而致阳亢者,属虚热证,须补(养)阴以制阳,即“壮水之主,以制阳光”(《素问·至真要大论》王冰注)。又如温病中之暑证或杂病中之虚损病等的气阴两伤,阴阳两亏,又当阴阳兼补。

总之,中医对疾病治疗的根本原则是有余者,损之,泻之;不足者,益之,补之。阴阳偏盛则泻其有余,阴阳偏衰则补其不足。
(五) 概括归纳病因

中医学对致病邪气的阴阳归类,基本上反映了该致病邪气的某些方面的特点,有的可直接反映出该致病邪气所引起病证的本质属性。因此,对疾病邪气的阴阳归类法,是有其实际意义的。如《内经》曾将邪气从其来源而分为阴阳两大类:“夫邪之生,或生于阴,或生于阳。其生于阳者,得之风雨寒暑。其生于阴者,得之饮食居处,阴阳喜怒”(《素问·调经论》)。《内经》中的“生于阴”、“生于阳”中之阴阳,指天地而言,即是说邪气或来源于天之“风雨寒暑”,或来源于地之“饮食居处,阴阳(此处指男女房事)喜怒(指情志)”。这是从邪气来源于天地之不同而分阴阳两大类。随着医学的进步,人们对致病邪气认识逐渐深化,又将天之六淫邪气(风、寒、暑、湿、燥、火)及人体内生之六邪(即内风、内寒、内湿、内燥、内火、内热)根据它们的致病特点,又进一步用阴阳加以归类。即风、火、燥、热、暑为阳邪;寒、湿为阴邪。从这种阴阳归类法,大体可以反映出这两类邪气的不同致病特点:阳邪致病多为热证,易于伤及阴津;阴邪致病多见寒证,易伤阳气。这里仅就一般而言,不是完全如此。如内伤之寒邪,为寒证,而外感之寒邪可为热证。又暑邪,即可伤阴,又可伤阳。燥邪,又有温燥,凉燥之分(详见本书“病因与病机”一章)。
(六) 用于经络命名与归类

经络学根据经络循行部位和所连属脏腑等不同,将经络分为阴经、阳经、阴络、阳络,并以阴阳命名。如循行于四肢内侧,并连属于五脏者,称为阴经(手太阴肺经、手少阴心经、手厥阴心包经、足太阴脾经、足少阴肾经、足厥阴肝经);循行于四肢外侧,并与六腑相连属者,称为阳经(手太阳小肠经、手少阳三焦经、手阳明大肠经、足太阳膀胱经、足少阳胆经、足阳明胃经)。其他,任脉行于身之前,属阴经;督脉行于身背,属阳经。以阴阳命名的还有阴蹻、阳蹻、阴维、阳维、以及根据分布深浅,将全身络脉分为阴络、阳络。
(七)概括药物性味功能特点

中药学根据药物有四气(性)一一寒、热、温、凉,五味——酸、苦、甘、辛、咸,以及升、降、浮、沉的不同作用特点,按阴阳基本属性,分别将药物归为阴阳两大类。以四气言,寒凉性药物属阴,而多用于阳证、热证;温热性药物属阳,而多用于阴证、寒证。以五味言,具有辛、甘、淡味药物属阳,具有酸、苦、咸味药物属阴。《素问·至真要大论》对五味(实为六味)及其作用,以阴阳属性做了概括:“辛甘发散为阳,酸苦涌泄为阴,咸味涌泄为阴,淡味渗泄为阳”。以升、降、浮、沉言,具有升浮作用的药物属阳,具有沉降作用的药物属阴。升、浮是指药物具有上行、升提和向外发散作用;沉、降是指药物具有下行、向内的作用(如潜镇、降逆、收敛、渗泻等)。因此,了解了药物的阴阳分类,就基本上能掌握药物的总的作用特点。
(八) 确立养生保健原则

阴阳学说认为,自然界和人都是由阴阳二气所构成。同时,人体和自然界是一个统一整体(即天人一体观)。《内经》认为宇宙是一个大天地,人体是一个“小天地”,而且人与天地相参应。《灵枢·邪客》说:“天有日月,人有两目;地有九州,人有九窍;天有风雨,人有喜怒”,“天有四时,人有四肢;天有五音,人有五脏;天有六律,人有六腑”等等。尽管这种联系,从表面上看是缺乏科学依据的,但我们应从其反映的精神实质去研究分析。即古人是采用取类比象的方法,把人体置于与自然界息息相关的地位,认为人体与自然界是一个统一的整体。这一认识是完全正确的。所以,自然界的阴阳变化,必将影响到人体阴阳的运动变化,从而对人体的健康与否发生影响。因此,中医学在养生保健方面十分重视如何适应自然界阴阳的变化规律。《内经》认为:“阴阳四时者,万物之终始也,死生之本也。逆之则灾害生,从之则苛疾不起”(《素问·四气调神大论》),并进一步指出人类应当“春夏养阳,秋冬养阴,以从其根”。就是说,人在春夏要养护阳气,秋冬要养护阴气,以符合于阴阳自身所固有的运动变化规律。《内经》还指出,凡是健康长寿的人,都善于“和于阴阳”、“法于阴阳”、“把握阴阳”(《素问·上古天真论》)。即按阴阳运动规律,调整自己的生活起居。不仅如此,其他如情志等过极,如:“暴怒伤阴,暴喜伤阳”(《素问·阴阳应象大论》),都会直接影响人体阴阳平衡。中医学是把人体如何适应自然界阴阳运动规律和维护、调整人体自身阴阳平衡,做为指导养生保健的最高原则。
(九) 概括体质禀赋

人体在正常生理状态下,虽然阴阳二气保持着相对的平衡状态,但这种平衡不是阴阳二气之间一对一的对等关系,而是有或偏于阳,或偏于阴的差别。这一点中医学早就有明确的论述,并将其用于指导临床辨证和治疗。

《内经》时代,就已将人群分为阴阳两大类,“黄帝问于少师曰:余尝闻人有阴阳,何谓阴人,何谓阳人”(《灵枢·通天》)?在这篇论述中,不仅把人群按体质禀赋差别分为阴阳二类,而且具体区别为太阴之人,少阴之人,太阳之人,少阳之人,阴阳和平之人五种,并指出这五种人在生理、病理和治疗上的不同之处。“太阴之人,多阴而无阳,其阴血浊,其卫气涩,阴阳不和,缓筋而厚皮,不之疾泻,不能移之。少阴之人,多阴少阳,小胃而大肠,六腑不调,其阳明脉小,而太阳脉大,必审而调之,其血易脱,其气易败也。太阳之人,多阳而少阴,必谨调之,无脱其阴,而泻其阳。阳重脱者,易狂,阴阳皆脱者,暴死不知人也。少阳之人,多阳而少阴,经小而络大,血在中而气外,实阴而虚阳,独泻其络脉则强,气脱而疾,中气不足,病不起也。阴阳和平之人,其阴阳之气和,血脉调,宜谨诊其阴阳,视其邪正,安其容仪,审有余不足,盛则泻之,虚则补之,不盛不虚,以经取之。此所以调阴阳,别五态之人者也”(《灵枢·通天》)。此外,《内经》还提出“重阳之人,其神易动,其气易往也”,并且“阳气滑盛而扬,故神动而气先行”,“多阳者多喜,多阴者多怒”(《灵枢·行针》)等,以及不同阴阳禀赋的人,对针刺反映也不同。

后世医家对人体禀赋阴阳的不同,在病机转化趋向,以及在辨证和治疗上的意义都有一定的认识。张景岳早已提出不同阴阳体质的人,所患病证性质有个基本趋向。他说:“阳脏之人多热,阴脏之人多寒。阳脏者必平生喜冷畏热,即朝夕食冷一无所病,此阳之有余也;阴脏者一犯寒凉,脾肾必伤,此其阳之不足也”。这里所说的阴脏阳脏,说的是平素脏腑偏阴、偏阳,即指体质而言。后世如叶天士、华云岫等都十分重视人的体质阴阳的不同,在致病和治疗上的意义。清·叶天士说:“平素体质不可不论”。同样感受湿邪,阳性体质之人,则化热;阴性体质之人,则化寒。所以,中医学素有“从阳化热,从阴化寒”之说。在治疗上,对于不同阴阳体质的人,用药也须斟酌,寒热之药,应用时要有尺度,不可太过。如阴性体质的病人和阳性体质的病人,若是同样的热证,用寒凉药的药物、剂量、疗程也不同,若阴性体质的病人服寒凉药过量,就会产生热邪刚去又生寒证的后果。

\section{第二节 五行学说}
一、五行学说的基本内容
(一) 五行的基本属性
(二)五行的相互关系
二、五行学说在中医学中的应用
\section{第三节 阴阳学说和五行学说的关系}

\chapter{第二章 藏象}
一、藏象的概念
二、藏象学说的主要内容
三、藏象学说的形成
四、藏象学说的特点
第一节 脏腑
一、五脏六腑
(一)心、小肠(附心胞络)
(二)肺、大肠
(三)脾、胃
(四) 肝、胆
(五)肾、膀胱(附命门)
(六)三焦
二、五脏之间的相互关系
三、六腑之间的相互关系
四、奇恒之腑
第二节 精、气、血、津液
一、气
二、血
(一)血的生成
(二)血的功能
(三)血的运行
三、津液
四、气血津液之间的相互关系

\chapter{第三章 经络}
第一节 经络与经络系统的组成
第二节 十二经脉
一、十二经脉的命名
二、十二经脉的循行走向和交接规律
三、十二经脉的表里关系
四、十二经脉气血流注次序
五、十二经脉的循行部位
附]《灵枢·经脉》十二经脉原文
第三节 奇经八脉
第四节 十二经别
第五节 十五别络
第六节 十二经筋
第七节 皮部
第八节 经络系统的作用
第九节 经络学说的临床应用
一、指导临床辨证
〔附〕《内经》有关经络病证的原文(仅供临床辨证参考)
二、指导临床治疗

\chapter{第四章 病因与病机}
第一节 病因
一、六淫
(一)风
(二)寒
(三)暑
(四)湿
(五)燥
(六)火
二、疫疠
三、情志
四、饮食、劳倦
五、外伤及虫兽伤害
六、寄生虫
七、痰饮、瘀血
第二节 病机

\chapter{第五章 诊法}
第一节 望诊
一、望神
二、望面
三、望目
四、望舌
五、望形体姿态
六、望排泄物
七、望头、髮、五窍
八、望皮肤
九、望小儿指纹
第二节 闻诊
一、听声音
二、嗅气味
第三节 问诊
一、问一般情况
二、问现在病情
(一)问寒热
(二) 问汗
(三)问疼痛
(四)问饮食
(五)问二便
(六)问睡眠
(七)问经带胎产
(八)问小儿
第四节 切诊
一、脉诊
二、按诊

\chapter{第六章 辨证}
第一节八纲辨证
一、八纲辨证的基本内容
二、八纲之间的相互关系
第二节 气血津液辨证
第三节 脏腑辨证
第四节 六经辨证
第五节 卫气营血辨证
第六节 三焦辨证
第七节 经络辨证

\chapter{第七章 治则与治法}
第一节 防治原则
一、预防原则
二、治疗原则
(一)治病求本
(二)扶正祛邪
(三)调整阴阳
(四)因时、因地、因人制宜
第二节 治疗方法
一、内治法
(一) 汗法
(二) 吐法
(三) 下法
(四) 和法
(五) 温法
(六) 清法
(七) 补法
(八) 消法
(九) 八法的配合使用
二、外治法
\chapter{第八章 中药学概述}
\section{第一节 中药的性能}
\section{第二节 中药的归经}
\section{第三节 中药的炮制}
\section{第四节 中药的配伍}
\section{第五节 中药的有毒、无毒与用药禁忌}
\section{第六节 中药的服用法}

\chapter{第九章 方剂学概述}
\section{第一节 方剂的组成}
\section{第二节 方剂的分类}
\section{第三节 方剂的剂型及应用特点}

\chapter{第十章 中医药学在医疗上的优势}
\section{一、医疗思想的先进性}
\section{二、治疗手段的优越性}


\backmatter

附:形体骨骼
一、头面颈项部
(一) 前、侧面观(见图48、49)
(二) 背面观(见图50)
二、躯干部
(一) 前、侧面观(见图51、52)
(二)背面观(见图53)
三、四肢部(见图54、55、56)
(一)上肢(见图54)
(二) 下肢(见图55、56)


\end{document}