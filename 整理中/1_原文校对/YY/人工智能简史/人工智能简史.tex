\documentclass{book}
\usepackage[UTF8]{ctex}
\usepackage{geometry}
\usepackage{graphicx}
\usepackage{hyperref}
\usepackage{amsmath}
\usepackage{amssymb}
\usepackage{booktabs}
\usepackage{listings}

\geometry{a4paper, top=3cm, bottom=3cm, left=3cm, right=3cm}

\title{人工智能简史}
\author{刘韩}
\date{2026}

\begin{document}

\maketitle
\tableofcontents

\chapter{引言}
\section{人工智能的起源}
\section{本书的结构}

\chapter{群星闪耀达特茅斯会议,香农大神见证人工智能的诞生}
\section{达特茅斯会议}
\section{克劳德·香农}
\section{西蒙与纽厄尔}
\section{麦卡锡与明斯基}

\chapter{国际象棋与围棋,人工智能最先攻破的堡垒}
\section{AlphaGo横空出世}
\section{国际象棋程序的逐步演进}
\section{"深蓝"挑战世界冠军卡斯帕罗夫}
\section{AlphaGo的故事}

\chapter{深度学习,掀起人工智能的新高潮}
\section{早期的人工神经网络}
\section{一代宗师杰弗里·辛顿}
\section{延恩·乐存与卷积神经网络}
\section{GPU与海量训练数据}
\section{深度学习的应用}
\section{对天才少年的一点建议}

\chapter{从汇编语言到TensorFlow,人工智能的开发语言和工具的演化}
\section{冯·诺依曼结构与汇编语言}
\section{Lisp语言与Prolog语言}
\section{UNIX操作系统与C语言}
\section{Python语言}
\section{TensorFlow深度学习框架}
\section{畅想未来:"超级人工智能"可能使用的编程语言}

\chapter{专家系统、知识图谱与人机对话,各种人工智能软件系统}
\section{专家系统}
\section{大百科全书项目}
\section{谷歌知识图谱}
\section{人机对话软件}

\chapter{机器人,电影与现实}
\section{电影中的几个机器人}
\section{阿西莫夫三定律}
\section{工业机器人}
\section{移动机器人}
\section{电影与现实中的"钢铁侠"}

\chapter{数学家的贡献,从牛顿到哥德尔}
\section{牛顿}
\section{莱布尼茨}
\section{费马}
\section{贝叶斯定理与贝叶斯网络}
\section{数理逻辑的演化}
\section{哥德尔}

\chapter{怀念先知,冯·诺伊曼、图灵和香农}
\section{冯·诺伊曼}
\section{阿兰·图灵}
\section{先知的传承与未来的展望}

\chapter{附录1 将"良知"注入机器人"内心"的初步思考}
\chapter{附录2 人工智能大事年表}
\chapter{附录3 人工智能先驱者的学术谱系}
\chapter{附录4 术语释义汇编}
\chapter{附录5 参考文献}

\end{document}