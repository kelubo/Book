\documentclass{book}
\usepackage[UTF8]{ctex}
\usepackage{geometry}
\usepackage{graphicx}
\usepackage{hyperref}
\usepackage{amsmath}
\usepackage{amssymb}
\usepackage{booktabs}

\geometry{a4paper, top=3cm, bottom=3cm, left=3cm, right=3cm}

\title{逊色:科学对女性做错了什么?}
\author{安吉拉·萨伊尼 (Angela Saini)}
\date{2026}

\begin{document}

\maketitle
\tableofcontents

\chapter{引言}
\section{科学中的性别偏见}
\section{本书的研究方法}

\chapter{历史中的女性科学家}
\section{被忽视的女性贡献}
\section{科学史的性别叙事}

\chapter{生物学与性别差异}
\section{大脑的性别差异研究}
\section{荷尔蒙与行为的研究}
\section{进化心理学的性别假设}

\chapter{科学研究中的偏见}
\section{研究设计的性别偏见}
\section{数据分析中的性别偏见}
\section{发表和引用中的性别偏见}

\chapter{医学研究中的性别差异}
\section{药物测试中的性别忽视}
\section{疾病诊断中的性别偏见}
\section{女性健康研究的不足}

\chapter{ STEM 领域的性别不平等}
\section{教育中的性别差异}
\section{职场中的性别歧视}
\section{女性在科学领导中的缺失}

\chapter{科学与社会性别规范}
\section{科学如何强化性别刻板印象}
\section{性别规范对科学的影响}
\section{性别与科学认同}

\chapter{走向性别公正的科学}
\section{性别包容的研究方法}
\section{科学教育的改革}
\section{职场文化的转变}

\chapter{结论}
\section{科学的未来}
\section{性别公正的重要性}

\end{document}