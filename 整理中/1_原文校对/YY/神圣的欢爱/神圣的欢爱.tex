\documentclass{book}
\usepackage[UTF8]{ctex}
\usepackage{geometry}
\usepackage{graphicx}
\usepackage{hyperref}
\usepackage{amsmath}
\usepackage{amssymb}
\usepackage{booktabs}

\geometry{a4paper, top=3cm, bottom=3cm, left=3cm, right=3cm}

\title{神圣的欢爱:性、神话与女性肉体的政治学}
\author{理安·艾斯勒 (Riane Eisler)}
\date{2026}

\begin{document}

\maketitle
\tableofcontents

\chapter{引言}
\section{性与权力的关系}
\section{本书的研究视角}

\chapter{神话与象征}
\section{古代神话中的女性形象}
\section{性在神话中的象征意义}

\chapter{文化与社会结构}
\section{伙伴关系文化}
\section{统治关系文化}
\section{两种文化的对比}

\chapter{女性肉体的政治学}
\section{女性身体的社会建构}
\section{性与权力的交织}
\section{女性的性自主权}

\chapter{历史的演变}
\section{古代社会中的女性地位}
\section{父权制的兴起}
\section{女性运动的发展}

\chapter{现代社会的挑战}
\section{性别不平等的现状}
\section{性暴力与压迫}
\section{媒体对女性的塑造}

\chapter{未来的可能性}
\section{伙伴关系的复兴}
\section{性别平等的实现}
\section{神圣欢爱的回归}

\chapter{结论}
\section{性、神话与权力的重新思考}
\section{走向更加公正的社会}

\end{document}