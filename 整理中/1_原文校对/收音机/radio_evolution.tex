% 收音机的演变:从矿石机到七管超外差
% 使用xelatex编译

\documentclass[12pt,a4paper,twoside]{ctexbook}

% 页面设置
% 纸张设置配置文件
% 用于定义书籍的页面尺寸和边距

\usepackage[a4paper,twoside]{geometry}
\geometry{
	left=25mm,
	right=20mm,
	top=25mm,
	bottom=25.4mm,
	headsep=1cm, 
    footskip=1cm,
	bindingoffset=10mm
}

% 字体设置
\usepackage{xeCJK}
\usepackage{fontspec}
\usepackage{microtype}

% 设置中文字体
\setCJKmainfont{SimSun}[
    BoldFont=SimHei,
    ItalicFont=KaiTi
]
\setCJKsansfont{SimHei}
\setCJKmonofont{SimSun}
\setCJKfamilyfont{kai}[
    BoldFont=KaiTi
]{KaiTi}
\setCJKfamilyfont{fs}[
    BoldFont=FangSong
]{FangSong}

% 常用字体命令
\newcommand{\song}{\CJKfamily{zhsong}}
\newcommand{\hei}{\CJKfamily{zhhei}}
\newcommand{\kai}{\CJKfamily{kai}}
\newcommand{\fs}{\CJKfamily{fs}}

% 标题格式设置
\ctexset{
    part/name={第,卷},
    part/number={\chinese{part}},
    chapter/name={第,章},
    chapter/number={\chinese{chapter}},
    section/name={第,节},
    section/number={\arabic{section}},
    subsection/number={\arabic{section}.\arabic{subsection}},
    chapter/format={\centering\hei\zihao{2}},
    section/format={\hei\zihao{4}},
    subsection/format={\hei\zihao{5}}
}

% 目录设置
\usepackage{titletoc}
\titlecontents{chapter}[0pt]{\vspace{10pt}\bfseries\zihao{-4}}{\contentspush{\thecontentslabel\hspace{1em}}}{}{\titlerule*[8pt]{.}\contentspage}
\titlecontents{section}[2.5em]{\vspace{5pt}\zihao{5}}{\contentspush{\thecontentslabel\hspace{1em}}}{}{\titlerule*[8pt]{.}\contentspage}
\titlecontents{subsection}[5em]{\vspace{3pt}\zihao{5}}{\contentspush{\thecontentslabel\hspace{1em}}}{}{\titlerule*[8pt]{.}\contentspage}

% 页眉页脚设置
\usepackage{fancyhdr}
\pagestyle{fancy}
\fancyhf{}
\fancyhead[LE,RO]{\zihao{5}\thepage}
\fancyhead[LO]{\zihao{5}\leftmark}
\fancyhead[RE]{\zihao{5}\rightmark}
\renewcommand{\chaptermark}[1]{\markboth{\chaptername\ \thechapter\ #1}{}}
\renewcommand{\sectionmark}[1]{\markright{\thesection\ #1}}
\fancyfoot[C]{\zihao{5} \thepage}
\renewcommand{\headrulewidth}{0.4pt}
\renewcommand{\footrulewidth}{0pt}

% 插图设置
\usepackage{graphicx}
\usepackage{float}
\usepackage{subfigure}
\graphicspath{{images/}}
\floatstyle{plaintop}
\restylefloat{figure}

% 表格设置
\usepackage{tabularx}
\usepackage{booktabs}
\usepackage{longtable}

% 数学公式设置
\usepackage{amsmath, amssymb, amsthm}
\usepackage{mathrsfs}

% 定理环境
\newtheorem{theorem}{定理}[chapter]
\newtheorem{definition}{定义}[chapter]
\newtheorem{lemma}{引理}[chapter]
\newtheorem{corollary}{推论}[chapter]
\newtheorem{example}{例}[chapter]

% 引用设置
\usepackage{hyperref}
\hypersetup{
    colorlinks=true,
    linkcolor=blue,
    citecolor=blue,
    urlcolor=blue,
    pdftitle={收音机的演变:从矿石机到七管超外差},
    pdfauthor={作者名},
    pdfsubject={收音机技术发展},
    pdfkeywords={收音机 \quad 电子管 \quad 晶体管 \quad 场效应管 \quad 超外差}
}

% 目录深度
\setcounter{tocdepth}{3}
\setcounter{secnumdepth}{3}

% 标题页设置
\usepackage{titling}

% 封面信息
\title{\hei\zihao{0} 收音机的演变:从矿石机到七管超外差}
\author{\song\zihao{2} 作者名}
\date{\song\zihao{4} \today}

\begin{document}

% 封面
\begin{titlepage}
    \begin{center}
        \vspace*{6cm}
        \hei\zihao{0} 收音机的演变:从矿石机到七管超外差
        \vspace*{3cm}
        \song\zihao{2} 作者名
        \vspace*{3cm}
        \song\zihao{4} \today
    \end{center}
\end{titlepage}

% 版权页
\newpage
\thispagestyle{empty}
\begin{center}
    \vspace*{8cm}
    \song\zihao{5} 版权所有 \textcopyright\ 2026 作者名
    \vspace*{1cm}
    \song\zihao{5} 出版社名称
\end{center}

% 目录
\newpage
\tableofcontents

% 正文开始
\mainmatter

\part{收音机基础理论}

\chapter{收音机发展的历史背景}

收音机的发展起源于19世纪末的无线电报技术。1895年,意大利物理学家马可尼(Guglielmo Marconi)成功实现了无线电信号的远距离传输,为收音机的发明奠定了基础。20世纪初,真空电子管的发明(1904年弗莱明发明二极管,1906年德福雷斯特发明三极管)使得无线电信号的放大和检波成为可能,标志着现代收音机时代的开始。

早期的收音机主要采用矿石检波器,结构简单但灵敏度低。随着真空电子管技术的发展,收音机逐渐从矿石收音机过渡到电子管收音机,并经历了从单管到多管、从再生式到超外差式的演变过程。20世纪中叶,晶体管和场效应管的发明进一步推动了收音机的小型化和普及化。

\chapter{调谐电路设计与LC值计算}

调谐电路是所有类型收音机的核心组成部分,无论是矿石收音机、晶体管收音机、电子管收音机还是场效应管收音机,都基于相同的LC谐振原理。本章将详细介绍调谐电路的设计方法和LC值的计算技巧。

\subsection{AM广播频率范围}

在设计调谐电路之前,首先需要了解广播频率范围:
- 中波(MW):530kHz - 1600kHz
- 短波(SW):3MHz - 30MHz

大多数收音机接收中波广播,因此调谐电路需要覆盖530kHz - 1600kHz的频率范围。

\subsection{LC值计算方法}

设计调谐电路时,需要确定电感L的值和可变电容C的容量范围。

\subsubsection{步骤1:确定可变电容的容量范围}

常见的空气可变电容容量范围:
- 小型:40pF - 360pF
- 中型:50pF - 450pF
- 大型:60pF - 550pF

选择可变电容时,需要考虑容量范围和品质因数(Q值),Q值越高,选择性越好。

\subsubsection{步骤2:计算电感值}

在调谐电路设计中,有两种常用的电感计算方法:**最低频率法**和**几何平均值法**。

\paragraph{方法一:最低频率法}

最低频率法是最常用的计算方法,以最低频率和最大电容计算电感值,确保低端频率一定能被覆盖。

以中波广播为例,假设使用360pF的可变电容,需要覆盖530kHz - 1600kHz的频率范围。

根据谐振频率公式:
\[L = \frac{1}{(2\pi f)^2 C}\]

计算接收最低频率(530kHz)时所需的电感值:
- $f = 530\,\text{kHz} = 530000\,\text{Hz}$
- $C = 360\,\text{pF} = 360 \times 10^{-12}\text{F}$

\[L = \frac{1}{(2\pi \times 530000)^2 \times 360 \times 10^{-12}} ≈ 240\,\mu\text{H}\]

验证接收最高频率(1600kHz)时的电容值:
- $f = 1600\,\text{kHz} = 1600000\,\text{Hz}$
- $L = 240\,\mu\text{H} = 240 \times 10^{-6}\text{H}$

\[C = \frac{1}{(2\pi \times 1600000)^2 \times 240 \times 10^{-6}} ≈ 40\,\text{pF}\]

\paragraph{方法二:几何平均值法}

几何平均值法利用频率和电容的几何平均值来计算电感,实现更均匀的频率覆盖。

\subparagraph{计算步骤}
1. **计算频率的几何平均值**:\( f_0 = \sqrt{f_{\text{min}} × f_{\text{max}}} \)
2. **计算电容的几何平均值**:\( C_0 = \sqrt{C_{\text{min}} × C_{\text{max}}} \)
3. **计算电感值**:\( L = \frac{1}{(2\pi f_0)^2 C_0} \)

\subparagraph{示例计算}
- 频率几何平均值:\( f_0 = \sqrt{530000 × 1600000} ≈ 918\,\text{kHz} \)
- 电容几何平均值:\( C_0 = \sqrt{40 × 10^{-12} × 360 × 10^{-12}} ≈ 120\,\text{pF} \)
- 电感值:\( L = \frac{1}{(2\pi × 918000)^2 × 120 × 10^{-12}} ≈ 250\,\mu\text{H} \)

\paragraph{两种方法的对比}

| **对比项**       | **最低频率法**       | **几何平均值法**                           |
|----------------|-------------------|----------------------------------------|
| **计算依据**     | 以最低频率和最大电容计算电感    | 以频率和电容的几何平均值计算电感                    |
| **频率覆盖特性**   | 频率分布不均匀:低端密集,高端稀疏  | 频率分布更均匀:整个频段内覆盖间隔相近                 |
| **电感值差异**     | 示例中为240μH(中波)     | 示例中为250μH(中波)                       |
| **适用场景**     | - 强调接收低端频率<br>- 可变电容范围较小<br>- 对频率均匀性要求不高 | - 要求全频段均匀覆盖<br>- 可变电容范围较大<br>- 追求调谐手感均匀 |
| **实际调整难度**   | 可能需要在高端频率微调      | 调谐过程中频率变化更线性,调整更顺畅                 |

\paragraph{实际应用建议}

在收音机的实际制作中,两种方法可结合使用:

1. **初步计算**:优先使用几何平均值法计算电感,获得更均匀的频率覆盖
2. **实际调整**:制作线圈后,通过接收已知频率的电台进行微调
3. **修正分布电容**:若实际调谐时频率偏移,可通过增减线圈匝数或并联小电容来调整

因此,使用240-250μH的电感和40pF - 360pF的可变电容,可以覆盖530kHz - 1600kHz的中波广播频率范围。

\subsection{电感线圈的设计与制作}

\subsubsection{电感线圈的计算}

电感线圈的电感值可以通过以下公式计算:

\[L = \frac{\mu_0 N^2 A}{l}\]

其中:
- $L$:电感量(H)
- $\mu_0$:真空磁导率,$\mu_0 = 4\pi \times 10^{-7} \text{ H/m}$
- $N$:线圈匝数
- $A$:线圈横截面积(m²)
- $l$:线圈长度(m)

\subsubsection{电感线圈的制作}

以$240\,\mu\text{H}$电感为例,制作方法如下:

1. **选择骨架**:使用直径5cm的纸筒或塑料管
2. **计算匝数**:
   - 骨架直径:$5\,\text{cm} = 0.05\,\text{m}$
   - 骨架横截面积:$A = \pi \times (0.05/2)^2 ≈ 1.96 \times 10^{-3} \,\text{m}^2$
   - 线圈长度:$l = 0.08\,\text{m}$(8cm)
   - 代入公式计算匝数N:
     \[N = \sqrt{\frac{L l}{\mu_0 A}} ≈ \sqrt{\frac{240 \times 10^{-6} \times 0.08}{4\pi \times 10^{-7} \times 1.96 \times 10^{-3}}} ≈ 250\ \text{匝}\]

3. **绕制线圈**:
   - 使用0.2mm漆包线
   - 紧密绕制250匝
   - 线圈两端引出接线端子

\subsection{调谐电路的实际调整}

在实际制作收音机时,由于线圈的分布电容和其他因素的影响,计算值可能与实际值有差异,需要通过实际调整来达到最佳效果。

\subsubsection{调整方法}

1. **接收已知频率的电台**:选择一个已知频率的强信号电台
2. **调整可变电容**:找到该电台的最佳接收点
3. **测量实际频率**:使用频率计测量调谐电路的谐振频率
4. **调整电感**:如果频率偏差较大,调整线圈匝数
5. **微调**:反复调整直到达到最佳接收效果

\subsection{调谐电路的Q值与选择性}

调谐电路的Q值(品质因数)直接影响收音机的选择性,Q值越高,选择性越好,相邻电台的干扰越小。

\subsubsection{Q值的影响因素}

1. **线圈的Q值**:取决于线圈的材质、结构和工艺
2. **电容的Q值**:空气可变电容的Q值较高
3. **负载的影响**:检波器和耳机的阻抗会降低Q值

\subsubsection{提高Q值的方法}

1. **使用高Q值的线圈**:使用粗铜线、多股线或镀银线绕制线圈
2. **使用空气可变电容**:空气可变电容的Q值高于塑料介质电容
3. **减少负载影响**:使用高阻抗耳机或添加阻抗匹配电路
4. **优化线圈结构**:使用蜂房式绕法或分段绕法

通过合理设计调谐电路的LC值和提高Q值,可以显著改善收音机的接收效果和选择性。

\part{经典收音机设计}

\chapter{矿石收音机}

矿石收音机是最简单的收音机类型,不需要电源,仅利用矿石作为检波器来接收无线电信号。矿石收音机的发明可以追溯到20世纪初,是无线电技术发展的重要里程碑。1906年,美国发明家格林利夫·惠蒂尔·皮卡德(Greenleaf Whittier Pickard)发现了硅晶体的检波特性,随后各种矿石检波器被广泛应用于早期无线电接收设备中。

\subsection{工作原理}

矿石收音机的工作原理基于半导体材料的单向导电性,主要包括以下几个部分:

\subsubsection{基本原理}

1. **天线接收**:天线接收空间中的无线电波,将电磁波转换为高频交流电信号
2. **调谐回路**:由电感线圈和可变电容组成的LC并联谐振回路,选择所需频率的广播信号
3. **矿石检波**:利用矿石晶体的单向导电性,将高频调幅信号转换为音频信号
4. **耳机输出**:将音频信号转换为声音

\subsubsection{检波原理}

矿石检波器的核心是半导体晶体的单向导电特性。当高频调幅信号通过矿石检波器时:

- 高频载波被滤除,只保留音频包络信号
- 检波后的音频信号驱动耳机发声

常用的矿石检波材料包括:
- 方铅矿(PbS):最常见的矿石检波材料
- 黄铁矿(FeS₂):检波效果较好
- 硅晶体:性能稳定,易于获得
- 锗晶体:检波效率高

\subsubsection{调谐原理}

调谐回路利用LC并联谐振原理选择接收频率:

谐振频率公式:
\[f = \frac{1}{2\pi\sqrt{LC}}\]

其中:
- f:谐振频率(Hz)
- L:电感量(H)
- C:电容量(F)

通过调节可变电容C的容量,可以改变谐振频率,从而选择不同的广播电台。



\subsection{典型电路结构}

\subsubsection{基本电路图}

```
[天线] ---> [调谐回路(L1 + C1)] ---> [矿石检波器(D1)] ---> [耳机(SPK)]
                       |
                       |
                       ---> [地线]
```

\subsubsection{电路说明}

- **天线**:接收空间中的无线电波,通常使用5-10米长的导线
- **调谐回路**:由电感线圈L1和可变电容C1组成,用于选择接收频率
- **矿石检波器D1**:将高频调幅信号转换为音频信号
- **耳机SPK**:将音频信号转换为声音,需要高阻抗耳机(2000Ω以上)
- **地线**:提供参考电位,改善接收效果

\subsubsection{改进型电路}

为了提高接收效果,可以采用以下改进措施:

1. **增加耦合电容**:在天线和调谐回路之间增加耦合电容,改善阻抗匹配
2. **使用双调谐回路**:提高选择性,减少邻近电台干扰
3. **增加检波负载电阻**:改善检波效果

改进型电路图:

```
[天线] ---> [耦合电容(C0)] ---> [调谐回路(L1 + C1)] ---> [矿石检波器(D1)] ---> [耳机(SPK)]
                                       |                         |
                                       |                         v
                                       |                    [负载电阻(R1)]
                                       |                         |
                                       |                         v
                                       |-----------------------> [地线]
```

\subsection{元器件清单}

\subsubsection{基本型矿石收音机}

\begin{table}[H]
    \centering
    \caption{矿石收音机元器件清单}
    \label{tab:crystal_radio_parts}
    \begin{tabular}{ccc}
        \toprule
        元件名称 & 型号/规格 & 用途 \\ 
        \midrule
        调谐线圈 & L1:200-300匝,0.2mm漆包线,直径5cm & 接收信号调谐 \\
        可变电容 & C1:270pF空气可变电容 & 调谐选台 \\
        矿石检波器 & 方铅矿或黄铁矿 & 检波 \\
        耳机 & 2000Ω高阻抗耳机 & 音频输出 \\
        天线 & 5-10米导线 & 接收无线电波 \\
        地线 & 金属水管或接地棒 & 接地 \\
        \bottomrule
    \end{tabular}
\end{table}

\subsubsection{改进型矿石收音机}

\begin{table}[H]
    \centering
    \caption{改进型矿石收音机元器件清单}
    \label{tab:crystal_radio_improved_parts}
    \begin{tabular}{ccc}
        \toprule
        元件名称 & 型号/规格 & 用途 \\ 
        \midrule
        调谐线圈 & L1:200-300匝,0.2mm漆包线,直径5cm & 接收信号调谐 \\
        可变电容 & C1:270pF空气可变电容 & 调谐选台 \\
        耦合电容 & C0:10-50pF瓷片电容 & 阻抗匹配 \\
        矿石检波器 & 方铅矿、黄铁矿或硅晶体 & 检波 \\
        负载电阻 & R1:100kΩ-1MΩ碳膜电阻 & 改善检波效果 \\
        耳机 & 2000Ω-4000Ω高阻抗耳机 & 音频输出 \\
        天线 & 5-10米导线 & 接收无线电波 \\
        地线 & 金属水管或接地棒 & 接地 \\
        \bottomrule
    \end{tabular}
\end{table}

\subsubsection{制作要点}

1. **调谐线圈制作**:
   - 使用直径5cm的纸筒或塑料管作为线圈骨架
   - 用0.2mm漆包线紧密绕制200-300匝
   - 线圈两端引出接线端子

2. **矿石检波器制作**:
   - 选择纯净的方铅矿或黄铁矿晶体
   - 用细铜丝作为触针,轻轻接触矿石表面
   - 调节触针位置,找到最佳检波点

3. **天线安装**:
   - 天线应尽量远离建筑物和金属物体
   - 天线高度越高,接收效果越好
   - 天线与收音机之间应使用绝缘导线连接

4. **地线安装**:
   - 地线应连接到金属水管或专门的接地棒
   - 良好的接地可以显著提高接收效果

\subsection{性能特点}

\subsubsection{优点}

1. **结构简单**:元件数量少,制作容易
2. **无需电源**:完全依靠无线电波能量工作
3. **成本低廉**:材料易得,制作成本低
4. **教育价值**:是学习无线电技术的绝佳入门实验
5. **可靠性高**:没有有源元件,故障率低

\subsubsection{缺点}

1. **灵敏度低**:只能接收近距离的强信号
2. **选择性差**:容易受到其他频率信号的干扰
3. **音量小**:只能驱动高阻抗耳机,不能驱动扬声器
4. **调节困难**:矿石检波器需要仔细调节才能获得最佳效果
5. **受环境影响大**:接收效果受天线、地线、天气等因素影响

\subsection{历史意义}

矿石收音机在无线电技术发展史上具有重要意义:

1. **技术启蒙**:让普通大众首次接触到无线电技术
2. **普及广播**:为无线电广播的普及奠定了基础
3. **教育价值**:培养了无数无线电爱好者
4. **技术发展**:为后来电子管收音机的发展提供了技术积累

尽管现代收音机已经非常先进,但矿石收音机仍然具有独特的魅力,它不仅是无线电技术的活化石,也是电子爱好者制作和收藏的经典项目。

\chapter{单管收音机}

单管收音机是收音机发展史上的重要里程碑,它使用一个有源器件(电子管、晶体管或场效应管)来实现信号的放大和检波。与矿石收音机相比,单管收音机具有更高的灵敏度和输出功率,能够驱动扬声器,是现代收音机的雏形。

\subsection{电子管单管收音机}

电子管单管收音机是20世纪20-30年代的主流收音机类型,使用真空电子管作为放大器件。电子管单管收音机通常采用再生式电路,通过正反馈来提高灵敏度和选择性。

\subsubsection{再生式电子管单管收音机}

再生式电子管单管收音机利用电子管的放大特性和正反馈原理,将部分输出信号反馈到输入端,从而提高电路的增益和选择性。

\paragraph{工作原理}

再生式电路的核心是正反馈机制。电子管放大后的信号通过再生线圈反馈到调谐回路,与输入信号叠加,形成再生作用。再生量的调节是关键,再生过小则灵敏度不足,再生过大则会产生自激振荡。

\paragraph{电路图结构}

```
[天线] ---> [调谐回路(L1 + C1)] ---> [电子管栅极(G)]
                       |                  |
                       |                  v
                       |           [电子管阳极(A)]
                       |                  |
                       |                  v
                       |           [再生线圈(L2)] ---
                       |                  |          |
                       |                  v          |
                       |           [检波器(D1)]     |
                       |                  |          |
                       |                  v          |
                       |           [音频变压器(T1)] |
                       |                  |          |
                       |                  v          |
                       |-----------[耳机(SPK)] <-----
```

\paragraph{电路说明}

- **天线**:接收空间中的无线电波
- **调谐回路(L1 + C1)**:选择所需频率的广播信号
- **电子管**:通常使用三极管(如1A2、2P2、3A2),负责信号放大和检波
- **再生线圈(L2)**:与L1耦合,提供正反馈信号
- **检波器(D1)**:将高频信号转换为音频信号
- **音频变压器(T1)**:匹配电子管输出与耳机阻抗
- **耳机(SPK)**:音频输出设备

\subsubsection{元器件清单}

\begin{table}[H]
    \centering
    \caption{电子管单管收音机元器件清单}
    \label{tab:vacuum_tube_single_radio_parts}
    \begin{tabular}{ccc}
        \toprule
        元件名称 & 型号/规格 & 用途 \\ 
        \midrule
        电子管 & 1A2(或2P2、3A2) & 高频放大与再生 \\
        调谐线圈 & L1:200-300匝,0.2mm漆包线 & 接收信号调谐 \\
        再生线圈 & L2:10-20匝,0.2mm漆包线 & 提供再生反馈 \\
        可变电容 & C1:270pF空气可变电容 & 调谐选台 \\
        固定电容 & C2:0.01μF瓷片电容 & 耦合电容 \\
        固定电容 & C3:10μF电解电容 & 滤波电容 \\
        电阻 & R1:200kΩ碳膜电阻 & 栅极电阻 \\
        电阻 & R2:1MΩ碳膜电阻(电位器) & 再生调节电阻 \\
        电阻 & R3:1kΩ碳膜电阻 & 阴极电阻 \\
        二极管 & D1:2AP9(或1N4148) & 检波 \\
        音频变压器 & T1:300Ω/8Ω & 阻抗匹配 \\
        耳机 & 8Ω高阻抗耳机 & 音频输出 \\
        电池 & 6V甲电 + 1.5V乙电 & 电源 \\
        \bottomrule
    \end{tabular}
\end{table}

\subsubsection{工作原理}

电子管单管收音机的工作过程可以分为以下几个步骤:

1. **信号接收**:天线接收空间中的无线电波,通过调谐回路选择所需频率的信号
2. **信号放大**:信号进入电子管的栅极,经电子管放大后从阳极输出
3. **再生反馈**:一部分放大后的信号通过再生线圈反馈回调谐回路,增强原始信号
4. **信号检波**:放大后的高频信号通过检波器转换为音频信号
5. **音频输出**:音频信号经耳机转换为声音

\paragraph{再生原理}

再生是提高收音机灵敏度和选择性的关键技术。再生量的大小决定了电路的工作状态:

- **欠再生**:反馈量不足,灵敏度较低
- **临界再生**:最佳工作状态,灵敏度和选择性达到最佳平衡
- **过再生**:产生自激振荡,电路变为振荡器

\subsubsection{结构特点}

1. **单管设计**:仅使用一个电子管,电路简单
2. **再生电路**:采用正反馈提高性能
3. **电池供电**:使用甲电(灯丝电源)和乙电(阳极电源)
4. **小型化**:相比多管收音机,体积较小
5. **低功耗**:电子管功耗较低,适合便携使用

\subsubsection{性能特点}

\paragraph{优点}

1. **灵敏度较高**:再生作用显著提高了接收灵敏度
2. **电路简单**:元件数量少,制作容易
3. **成本低廉**:相比多管收音机,成本较低
4. **体积小巧**:适合便携使用
5. **功耗较低**:电池供电,使用方便

\paragraph{缺点}

1. **选择性较差**:容易受到其他频率信号的干扰
2. **再生调节困难**:需要仔细调节才能获得最佳效果
3. **稳定性差**:容易受到电源电压、温度等因素影响
4. **输出功率小**:只能驱动耳机,不能驱动扬声器
5. **噪声较大**:电子管本身会产生一定噪声

\paragraph{主要接收频段}

- **中波(MW)**:530-1600kHz,主要接收本地广播
- **短波(SW)**:部分机型可接收短波广播
- **长波(LW)**:少数机型支持长波接收

\subsection{晶体管单管收音机}

晶体管单管收音机是20世纪50-60年代随着晶体管技术发展而出现的新型收音机。相比电子管收音机,晶体管收音机具有体积小、功耗低、寿命长等优点,是收音机小型化的重要里程碑。

\subsubsection{再生式晶体管单管收音机}

再生式晶体管单管收音机利用晶体管的放大特性和正反馈原理,实现信号的放大和检波。与电子管电路相比,晶体管电路具有更高的效率和稳定性。

\paragraph{工作原理}

晶体管单管收音机的工作原理与电子管单管收音机类似,但使用晶体管作为放大器件。晶体管通常工作在共发射极或共基极组态,通过发射极或集电极的反馈线圈实现再生作用。

\paragraph{电路图结构}

```
[天线] ---> [调谐回路(L1 + C1)] ---> [晶体管基极(B)]
                       |                  |
                       |                  v
                       |           [晶体管集电极(C)]
                       |                  |
                       |                  v
                       |           [再生线圈(L2)] ---
                       |                  |          |
                       |                  v          |
                       |           [检波器(D1)]     |
                       |                  |          |
                       |                  v          |
                       |           [耦合电容(C2)]   |
                       |                  |          |
                       |                  v          |
                       |-----------[耳机(SPK)] <-----
                       |
                       v
                  [发射极电阻(R1)]
                       |
                       v
                      [地]
```

\paragraph{电路说明}

- **天线**:接收空间中的无线电波
- **调谐回路(L1 + C1)**:选择所需频率的广播信号
- **晶体管**:通常使用NPN型高频晶体管(如3AG1、3DG6),负责信号放大和检波
- **再生线圈(L2)**:与L1耦合,提供正反馈信号
- **检波器(D1)**:将高频信号转换为音频信号
- **耦合电容(C2)**:将音频信号耦合到耳机
- **耳机(SPK)**:音频输出设备
- **发射极电阻(R1)**:提供发射极偏置

\subsubsection{元器件清单}

\begin{table}[H]
    \centering
    \caption{晶体管单管收音机元器件清单}
    \label{tab:transistor_single_radio_parts}
    \begin{tabular}{ccc}
        \toprule
        元件名称 & 型号/规格 & 用途 \\ 
        \midrule
        晶体管 & 3AG1(或3DG6、9018) & 高频放大与再生 \\
        调谐线圈 & L1:200-300匝,0.2mm漆包线 & 接收信号调谐 \\
        再生线圈 & L2:10-20匝,0.2mm漆包线 & 提供再生反馈 \\
        可变电容 & C1:270pF空气可变电容 & 调谐选台 \\
        耦合电容 & C2:0.01μF瓷片电容 & 音频信号耦合 \\
        旁路电容 & C3:10μF电解电容 & 发射极旁路 \\
        偏置电阻 & R1:100kΩ碳膜电阻 & 基极偏置 \\
        发射极电阻 & R2:1kΩ碳膜电阻 & 发射极偏置 \\
        再生电阻 & R3:10kΩ电位器 & 再生强度调节 \\
        二极管 & D1:2AP9(或1N4148) & 检波 \\
        耳机 & 8Ω高阻抗耳机 & 音频输出 \\
        电池 & 9V层叠电池 & 电源 \\
        \bottomrule
    \end{tabular}
\end{table}

\subsubsection{工作原理}

晶体管单管收音机的工作过程:

1. **信号接收**:天线接收空间中的无线电波,通过调谐回路选择所需频率的信号
2. **信号放大**:信号进入晶体管的基极,经晶体管放大后从集电极输出
3. **再生反馈**:一部分放大后的信号通过再生线圈反馈到调谐回路,增强原始信号
4. **信号检波**:放大后的高频信号通过检波器转换为音频信号
5. **音频输出**:音频信号经耳机转换为声音

\paragraph{晶体管工作原理}

晶体管是三端半导体器件,具有三个电极:基极(B)、集电极(C)、发射极(E)。在共发射极组态下:

- 基极电流微小的变化可以控制集电极电流的较大变化
- 电流放大倍数(β)通常在几十到几百之间
- 高频晶体管具有良好的高频特性,适合射频放大

\subsubsection{结构特点}

1. **单管设计**:仅使用一个晶体管,电路简单
2. **低电压工作**:通常使用9V或更低电压
3. **小型化**:晶体管体积小,整机体积大幅减小
4. **低功耗**:功耗远低于电子管收音机
5. **长寿命**:晶体管寿命长,可靠性高

\subsubsection{性能特点}

\paragraph{优点}

1. **体积小巧**:便于携带,适合户外使用
2. **功耗极低**:电池使用寿命长
3. **启动迅速**:无需预热,开机即用
4. **抗振动**:晶体管耐振动,适合移动使用
5. **成本低廉**:晶体管价格低,制作成本低

\paragraph{缺点}

1. **选择性较差**:容易受到其他频率信号的干扰
2. **再生调节困难**:需要仔细调节才能获得最佳效果
3. **输出功率小**:只能驱动耳机,不能驱动扬声器
4. **受温度影响**:晶体管参数受温度影响较大
5. **噪声较大**:晶体管本身会产生一定噪声

\paragraph{主要接收频段}

- **中波(MW)**:530-1600kHz,主要接收本地广播
- **短波(SW)**:部分机型可接收短波广播
- **调频(FM)**:少数机型支持调频接收(需要额外电路)

\subsection{场效应管单管收音机}

场效应管单管收音机是利用场效应管(FET)作为放大器件的单管收音机。场效应管具有高输入阻抗、低噪声、线性好等优点,特别适合用于高灵敏度收音机的前级放大。

\subsubsection{再生式场效应管单管收音机}

再生式场效应管单管收音机利用场效应管的电压控制特性和正反馈原理,实现信号的放大和检波。与晶体管相比,场效应管具有更高的输入阻抗,对调谐回路的负载更小。

\paragraph{工作原理}

场效应管单管收音机的工作原理与晶体管单管收音机类似,但使用场效应管作为放大器件。场效应管通常工作在共源极组态,通过源极或漏极的反馈线圈实现再生作用。

\paragraph{电路图结构}

```
[天线] ---> [调谐回路(L1 + C1)] ---> [场效应管栅极(G)]
                       |                  |
                       |                  v
                       |           [场效应管漏极(D)]
                       |                  |
                       |                  v
                       |           [再生线圈(L2)] ---
                       |                  |          |
                       |                  v          |
                       |           [检波器(D1)]     |
                       |                  |          |
                       |                  v          |
                       |           [耦合电容(C2)]   |
                       |                  |          |
                       |                  v          |
                       |-----------[耳机(SPK)] <-----
                       |
                       v
                  [源极电阻(R1)]
                       |
                       v
                      [地]
```

\paragraph{电路说明}

- **天线**:接收空间中的无线电波
- **调谐回路(L1 + C1)**:选择所需频率的广播信号
- **场效应管**:通常使用结型场效应管(如3DJ6、2SK241),负责信号放大和检波
- **再生线圈(L2)**:与L1耦合,提供正反馈信号
- **检波器(D1)**:将高频信号转换为音频信号
- **耦合电容(C2)**:将音频信号耦合到耳机
- **耳机(SPK)**:音频输出设备
- **源极电阻(R1)**:提供源极偏置

\subsubsection{元器件清单}

\begin{table}[H]
    \centering
    \caption{场效应管单管收音机元器件清单}
    \label{tab:fet_single_radio_parts}
    \begin{tabular}{ccc}
        \toprule
        元件名称 & 型号/规格 & 用途 \\ 
        \midrule
        场效应管 & 3DJ6(或2SK241、BF245) & 高频放大与再生 \\
        调谐线圈 & L1:200-300匝,0.2mm漆包线 & 接收信号调谐 \\
        再生线圈 & L2:10-20匝,0.2mm漆包线 & 提供再生反馈 \\
        可变电容 & C1:270pF空气可变电容 & 调谐选台 \\
        耦合电容 & C2:0.01μF瓷片电容 & 音频信号耦合 \\
        旁路电容 & C3:10μF电解电容 & 源极旁路 \\
        栅极电阻 & R1:1MΩ碳膜电阻 & 栅极偏置 \\
        源极电阻 & R2:1kΩ碳膜电阻 & 源极偏置 \\
        再生电阻 & R3:10kΩ电位器 & 再生强度调节 \\
        二极管 & D1:2AP9(或1N4148) & 检波 \\
        耳机 & 8Ω高阻抗耳机 & 音频输出 \\
        电池 & 9V层叠电池 & 电源 \\
        \bottomrule
    \end{tabular}
\end{table}

\subsubsection{工作原理}

场效应管单管收音机的工作过程:

1. **信号接收**:天线接收空间中的无线电波,通过调谐回路选择所需频率的信号
2. **信号放大**:信号进入场效应管的栅极,经场效应管放大后从漏极输出
3. **再生反馈**:一部分放大后的信号通过再生线圈反馈到调谐回路,增强原始信号
4. **信号检波**:放大后的高频信号通过检波器转换为音频信号
5. **音频输出**:音频信号经耳机转换为声音

\paragraph{场效应管工作原理}

场效应管是三端半导体器件,具有三个电极:栅极(G)、漏极(D)、源极(S)。在共源极组态下:

- 栅极电压的变化可以控制漏极电流的变化
- 跨导(gm)表示栅极电压对漏极电流的控制能力
- 场效应管具有极高的输入阻抗,对信号源几乎没有负载效应
- 结型场效应管(JFET)具有良好的高频特性和低噪声特性

\subsubsection{结构特点}

1. **单管设计**:仅使用一个场效应管,电路简单
2. **高输入阻抗**:对调谐回路负载小,Q值高
3. **低噪声**:场效应管噪声系数低,适合弱信号接收
4. **线性好**:场效应管具有良好的线性特性
5. **低功耗**:功耗与晶体管相当,低于电子管

\subsubsection{性能特点}

\paragraph{优点}

1. **高输入阻抗**:对调谐回路影响小,选择性更好
2. **低噪声**:适合接收远距离弱信号
3. **线性好**:失真小,音质更好
4. **稳定性高**:场效应管参数受温度影响较小
5. **动态范围大**:可以处理较大信号而不失真

\paragraph{缺点}

1. **选择性较差**:容易受到其他频率信号的干扰
2. **再生调节困难**:需要仔细调节才能获得最佳效果
3. **输出功率小**:只能驱动耳机,不能驱动扬声器
4. **成本较高**:场效应管价格高于普通晶体管
5. **受静电影响**:场效应管容易受到静电损坏

\paragraph{主要接收频段}

- **中波(MW)**:530-1600kHz,主要接收本地广播
- **短波(SW)**:部分机型可接收短波广播
- **调频(FM)**:少数机型支持调频接收(需要额外电路)

\chapter{二管收音机}

二管收音机在单管收音机的基础上增加了一个有源器件,通常用于分离高频放大和低频放大功能,显著提高了收音机的性能。二管收音机可以驱动扬声器,实现外放功能,是收音机发展史上的重要进步。

\subsection{电子管二管收音机}

电子管二管收音机是20世纪30-40年代的普及型收音机,使用两个电子管分别负责高频放大和低频放大。相比单管收音机,二管收音机具有更高的输出功率和更好的音质。

\subsubsection{典型电路结构}

电子管二管收音机的典型电路包括:
1. **第一管(高频管)**:负责高频放大和检波
2. **第二管(低频管)**:负责低频放大,驱动扬声器
3. **调谐回路**:选择所需频率的广播信号
4. **检波器**:将高频信号转换为音频信号
5. **音频变压器**:匹配放大管和扬声器的阻抗

\subsubsection{电路图结构}

```
[天线] ---> [调谐回路(L1 + C1)] ---> [高频管栅极(V1-G)]
                       |                  |
                       |                  v
                       |           [高频管阳极(V1-A)]
                       |                  |
                       |                  v
                       |           [再生线圈(L2)]
                       |                  |
                       |                  v
                       |           [检波器(D1)]
                       |                  |
                       |                  v
                       |           [耦合电容(C2)]
                       |                  |
                       |                  v
                       |           [低频管栅极(V2-G)]
                       |                  |
                       |                  v
                       |           [低频管阳极(V2-A)]
                       |                  |
                       |                  v
                       |-----------[扬声器(SPK)]
```

\paragraph{电路说明}

- **V1**:高频管,通常使用三极管(如1A2),负责高频放大和再生
- **V2**:低频管,通常使用功率三极管(如3A2、2P2),负责音频放大
- **L1 + C1**:调谐回路,选择接收频率
- **L2**:再生线圈,提供再生反馈,增强接收信号
- **D1**:检波二极管,将高频信号转换为音频信号
- **C2**:耦合电容,将音频信号耦合到低频管

\subsubsection{元器件清单}

\begin{table}[H]
    \centering
    \caption{电子管二管收音机元器件清单}
    \label{tab:vacuum_tube_two_radio_parts}
    \begin{tabular}{ccc}
        \toprule
        元件名称 & 型号/规格 & 用途 \\ 
        \midrule
        高频电子管 & 1A2(或2P2) & 高频放大与再生 \\
        低频电子管 & 3A2(或5Y3、6P1) & 音频功率放大 \\
        调谐线圈 & L1:200-300匝,0.2mm漆包线 & 接收信号调谐 \\
        再生线圈 & L2:10-20匝,0.2mm漆包线 & 提供再生反馈 \\
        可变电容 & C1:270pF空气可变电容 & 调谐选台 \\
        耦合电容 & C2:0.01μF瓷片电容 & 音频信号耦合 \\
        滤波电容 & C3:10μF电解电容 & 电源滤波 \\
        栅极电阻 & R1:200kΩ碳膜电阻 & 高频管栅极电阻 \\
        再生电阻 & R2:1MΩ碳膜电阻(电位器) & 再生强度调节 \\
        阴极电阻 & R3:1kΩ碳膜电阻 & 高频管阴极电阻 \\
        阴极电阻 & R4:200Ω碳膜电阻 & 低频管阴极电阻 \\
        阳极电阻 & R5:10kΩ碳膜电阻 & 高频管阳极负载 \\
        检波二极管 & D1:2AP9(或1N4148) & 检波 \\
        扬声器 & 8Ω/0.5W动圈扬声器 & 音频输出 \\
        电池 & 6V甲电 + 4.5V乙电 & 电源 \\
        \bottomrule
    \end{tabular}
\end{table}

\subsubsection{性能提升}

与单管收音机相比,二管收音机的主要性能提升包括:

\paragraph{输出功率增加}
由于增加了专门的低频放大管,可以驱动小型扬声器,实现外放功能。输出功率从单管的毫瓦级提升到几百毫瓦,音量显著增大。

\paragraph{灵敏度提高}
高频放大和低频放大分离,优化了各部分的工作状态。高频管专注于高频信号的放大,低频管专注于音频信号的放大,整体灵敏度得到提升。

\paragraph{音质改善}
低频放大电路的加入使得音频信号的放大更加充分,音质更加清晰。音频信号的动态范围扩大,失真减小。

\paragraph{稳定性增强}
减少了自激振荡的可能性,工作更加稳定。高频和低频电路分离,相互干扰减小,工作稳定性提高。

\subsection{晶体管二管收音机}

晶体管二管收音机是20世纪60-70年代的普及型收音机,使用两个晶体管分别负责高频放大和低频放大。相比电子管二管收音机,晶体管二管收音机具有体积小、功耗低、寿命长等优点。

\subsubsection{典型电路结构}

晶体管二管收音机的典型电路包括:
1. **第一管(高频管)**:负责高频放大和检波
2. **第二管(低频管)**:负责低频放大,驱动扬声器
3. **调谐回路**:选择所需频率的广播信号
4. **检波器**:将高频信号转换为音频信号
5. **输出变压器**:匹配晶体管和扬声器的阻抗

\subsubsection{电路图结构}

```
[天线] ---> [调谐回路(L1 + C1)] ---> [高频管基极(Q1-B)]
                       |                  |
                       |                  v
                       |           [高频管集电极(Q1-C)]
                       |                  |
                       |                  v
                       |           [再生线圈(L2)]
                       |                  |
                       |                  v
                       |           [检波器(D1)]
                       |                  |
                       |                  v
                       |           [耦合电容(C2)]
                       |                  |
                       |                  v
                       |           [低频管基极(Q2-B)]
                       |                  |
                       |                  v
                       |           [低频管集电极(Q2-C)]
                       |                  |
                       |                  v
                       |-----------[扬声器(SPK)]
```

\paragraph{电路说明}

- **Q1**:高频管,通常使用NPN型高频晶体管(如3AG1、3DG6),负责高频放大和再生
- **Q2**:低频管,通常使用NPN型功率晶体管(如3AX31、3DG12),负责音频放大
- **L1 + C1**:调谐回路,选择接收频率
- **L2**:再生线圈,提供再生反馈,增强接收信号
- **D1**:检波二极管,将高频信号转换为音频信号
- **C2**:耦合电容,将音频信号耦合到低频管

\subsubsection{元器件清单}

\begin{table}[H]
    \centering
    \caption{晶体管二管收音机元器件清单}
    \label{tab:transistor_two_radio_parts}
    \begin{tabular}{ccc}
        \toprule
        元件名称 & 型号/规格 & 用途 \\ 
        \midrule
        高频晶体管 & 3AG1(或3DG6、9018) & 高频放大与再生 \\
        低频晶体管 & 3AX31(或3DG12、8050) & 音频功率放大 \\
        调谐线圈 & L1:200-300匝,0.2mm漆包线 & 接收信号调谐 \\
        再生线圈 & L2:10-20匝,0.2mm漆包线 & 提供再生反馈 \\
        可变电容 & C1:270pF空气可变电容 & 调谐选台 \\
        耦合电容 & C2:0.01μF瓷片电容 & 音频信号耦合 \\
        旁路电容 & C3:10μF电解电容 & 电源滤波 \\
        偏置电阻 & R1:100kΩ碳膜电阻 & 高频管基极偏置 \\
        再生电阻 & R2:10kΩ电位器 & 再生强度调节 \\
        发射极电阻 & R3:1kΩ碳膜电阻 & 高频管发射极电阻 \\
        偏置电阻 & R4:100kΩ碳膜电阻 & 低频管基极偏置 \\
        发射极电阻 & R5:100Ω碳膜电阻 & 低频管发射极电阻 \\
        检波二极管 & D1:2AP9(或1N4148) & 检波 \\
        扬声器 & 8Ω/0.5W动圈扬声器 & 音频输出 \\
        电池 & 9V层叠电池 & 电源 \\
        \bottomrule
    \end{tabular}
\end{table}

\subsubsection{性能提升}

与单管收音机相比,晶体管二管收音机的主要性能提升包括:

\paragraph{输出功率增加}
可以驱动小型扬声器,实现外放功能。输出功率从单管的毫瓦级提升到几百毫瓦,音量显著增大。

\paragraph{灵敏度提高}
高频放大和低频放大分离,优化了各部分的工作状态。整体灵敏度得到提升,可以接收更远的电台。

\paragraph{音质改善}
低频放大电路的加入使得音频信号的放大更加充分,音质更加清晰。音频信号的动态范围扩大,失真减小。

\paragraph{稳定性增强}
晶体管工作稳定,受温度影响较小。高频和低频电路分离,相互干扰减小,工作稳定性提高。

\subsection{场效应管二管收音机}

场效应管二管收音机利用两个场效应管分别负责高频放大和低频放大。相比晶体管二管收音机,场效应管二管收音机具有更高的输入阻抗和更低的噪声,适合高灵敏度接收。

\subsubsection{典型电路结构}

场效应管二管收音机的典型电路包括:
1. **第一管(高频管)**:负责高频放大和检波
2. **第二管(低频管)**:负责低频放大,驱动扬声器
3. **调谐回路**:选择所需频率的广播信号
4. **检波器**:将高频信号转换为音频信号
5. **输出变压器**:匹配场效应管和扬声器的阻抗

\subsubsection{电路图结构}

```
[天线] ---> [调谐回路(L1 + C1)] ---> [高频管栅极(Q1-G)]
                       |                  |
                       |                  v
                       |           [高频管漏极(Q1-D)]
                       |                  |
                       |                  v
                       |           [再生线圈(L2)]
                       |                  |
                       |                  v
                       |           [检波器(D1)]
                       |                  |
                       |                  v
                       |           [耦合电容(C2)]
                       |                  |
                       |                  v
                       |           [低频管栅极(Q2-G)]
                       |                  |
                       |                  v
                       |           [低频管漏极(Q2-D)]
                       |                  |
                       |                  v
                       |-----------[扬声器(SPK)]
```

\paragraph{电路说明}

- **Q1**:高频管,通常使用结型场效应管(如3DJ6、2SK241),负责高频放大和再生
- **Q2**:低频管,通常使用功率场效应管(如2SK135、IRF510),负责音频放大
- **L1 + C1**:调谐回路,选择接收频率
- **L2**:再生线圈,提供再生反馈,增强接收信号
- **D1**:检波二极管,将高频信号转换为音频信号
- **C2**:耦合电容,将音频信号耦合到低频管

\subsubsection{元器件清单}

\begin{table}[H]
    \centering
    \caption{场效应管二管收音机元器件清单}
    \label{tab:fet_two_radio_parts}
    \begin{tabular}{ccc}
        \toprule
        元件名称 & 型号/规格 & 用途 \\ 
        \midrule
        高频场效应管 & 3DJ6(或2SK241、BF245) & 高频放大与再生 \\
        低频场效应管 & 2SK135(或IRF510、IRF540) & 音频功率放大 \\
        调谐线圈 & L1:200-300匝,0.2mm漆包线 & 接收信号调谐 \\
        再生线圈 & L2:10-20匝,0.2mm漆包线 & 提供再生反馈 \\
        可变电容 & C1:270pF空气可变电容 & 调谐选台 \\
        耦合电容 & C2:0.01μF瓷片电容 & 音频信号耦合 \\
        旁路电容 & C3:10μF电解电容 & 电源滤波 \\
        栅极电阻 & R1:1MΩ碳膜电阻 & 高频管栅极偏置 \\
        再生电阻 & R2:10kΩ电位器 & 再生强度调节 \\
        源极电阻 & R3:1kΩ碳膜电阻 & 高频管源极电阻 \\
        栅极电阻 & R4:1MΩ碳膜电阻 & 低频管栅极偏置 \\
        源极电阻 & R5:100Ω碳膜电阻 & 低频管源极电阻 \\
        检波二极管 & D1:2AP9(或1N4148) & 检波 \\
        扬声器 & 8Ω/0.5W动圈扬声器 & 音频输出 \\
        电池 & 9V层叠电池 & 电源 \\
        \bottomrule
    \end{tabular}
\end{table}

\subsubsection{性能提升}

与单管收音机相比,场效应管二管收音机的主要性能提升包括:

\paragraph{输出功率增加}
可以驱动小型扬声器,实现外放功能。输出功率从单管的毫瓦级提升到几百毫瓦,音量显著增大。

\paragraph{灵敏度提高}
场效应管的高输入阻抗对调谐回路负载小,Q值高,选择性更好。整体灵敏度得到提升,可以接收更远的电台。

\paragraph{音质改善}
场效应管的低噪声特性和良好的线性度,使得音质更加清晰。音频信号的动态范围扩大,失真减小。

\paragraph{稳定性增强}
场效应管参数受温度影响较小,工作稳定。高频和低频电路分离,相互干扰减小,工作稳定性提高。

\chapter{三管收音机}

三管收音机进一步完善了电路结构,通常采用超外差式电路的基本架构,开始具备现代收音机的雏形。三管收音机包括变频管、中放管和低放管,性能相比二管收音机有了显著提升。

\subsection{电子管三管收音机}

电子管三管收音机是20世纪40-50年代的主流收音机类型,采用超外差式电路架构。三管收音机包括变频管、中放管和低放管,实现了高灵敏度、高选择性的接收性能。

\subsubsection{电路结构特点}

三管收音机的典型电路包括三个主要部分:
1. **变频管**:将接收到的高频信号转换为固定频率的中频信号(通常为465kHz)
2. **中放管**:对中频信号进行放大,提高灵敏度和选择性
3. **低放管**:对检波后的音频信号进行放大,驱动扬声器

\subsubsection{电路图结构}

```
[天线] ---> [输入回路(L1 + C1)] ---> [变频管栅极(V1-G)]
                                   |          |
                                   |          v
                                   |   [变频管本机振荡(V1-Osc)]
                                   |          |
                                   |          v
                                   |   [混频输出中频(465kHz)]
                                   |          |
                                   |          v
                                   |   [中频变压器(T1)]
                                   |          |
                                   |          v
                                   |   [中放管栅极(V2-G)]
                                   |          |
                                   |          v
                                   |   [中放管阳极(V2-A)]
                                   |          |
                                   |          v
                                   |   [检波器(D1)]
                                   |          |
                                   |          v
                                   |   [耦合电容(C2)]
                                   |          |
                                   |          v
                                   |   [低放管栅极(V3-G)]
                                   |          |
                                   |          v
                                   |   [低放管阳极(V3-A)]
                                   |          |
                                   |          v
                                   |---[扬声器(SPK)]
```

\paragraph{电路说明}

- **V1**:变频管,通常使用变频三极管(如1A2、6A2),同时负责高频放大和本机振荡
- **V2**:中放管,通常使用高增益三极管(如3AG1、6K4),负责中频信号放大
- **V3**:低放管,通常使用功率三极管(如3A2、6P1),负责音频功率放大
- **L1 + C1**:输入调谐回路,选择接收频率
- **T1**:中频变压器(中周),调谐在465kHz,提高选择性
- **D1**:检波二极管,将中频信号转换为音频信号
- **C2**:耦合电容,将音频信号耦合到低放管

\subsubsection{元器件清单}

\begin{table}[H]
    \centering
    \caption{电子管三管收音机元器件清单}
    \label{tab:vacuum_tube_three_radio_parts}
    \begin{tabular}{ccc}
        \toprule
        元件名称 & 型号/规格 & 用途 \\ 
        \midrule
        变频电子管 & 6A2(或1A2、6SA7) & 高频放大与变频 \\
        中放电子管 & 6K4(或3AG1、6SK7) & 中频信号放大 \\
        低放电子管 & 6P1(或3A2、5Y3) & 音频功率放大 \\
        输入线圈 & L1:200-300匝,0.2mm漆包线 & 接收信号调谐 \\
        可变电容 & C1:270pF空气可变电容 & 调谐选台 \\
        本机振荡线圈 & L2:100-150匝,0.2mm漆包线 & 产生本机振荡信号 \\
        中频变压器 & T1:465kHz中周 & 中频信号选频与耦合 \\
        耦合电容 & C2:0.01μF瓷片电容 & 音频信号耦合 \\
        滤波电容 & C3:10μF电解电容 & 电源滤波 \\
        栅极电阻 & R1:200kΩ碳膜电阻 & 变频管栅极电阻 \\
        栅极电阻 & R2:1MΩ碳膜电阻 & 中放管栅极电阻 \\
        阴极电阻 & R3:1kΩ碳膜电阻 & 变频管阴极电阻 \\
        阴极电阻 & R4:500Ω碳膜电阻 & 中放管阴极电阻 \\
        阴极电阻 & R5:200Ω碳膜电阻 & 低放管阴极电阻 \\
        阳极电阻 & R6:10kΩ碳膜电阻 & 中放管和变频管阳极负载 \\
        检波二极管 & D1:2AP9(或1N4148) & 中频检波 \\
        扬声器 & 8Ω/1W动圈扬声器 & 音频输出 \\
        电源变压器 & 220V/6.3V×2 + 12V & 提供电子管灯丝和阳极电压 \\
        \bottomrule
    \end{tabular}
\end{table}

\subsubsection{超外差式电路的优势}

三管收音机开始采用超外差式电路,这是收音机发展史上的重要里程碑。超外差式电路的优势包括:

\paragraph{高选择性}
通过固定中频的选频回路,可以有效滤除其他频率的干扰信号。中频变压器的Q值很高,选择性远优于再生式电路。

\paragraph{高灵敏度}
中频放大器可以针对固定频率进行优化设计,提高放大效率。固定频率的放大器可以获得更高的增益和更好的稳定性。

\paragraph{性能稳定}
电路工作状态受信号频率变化的影响较小。无论接收哪个频率的电台,中频放大器都工作在相同的频率下,性能稳定。

\paragraph{易于调试}
各部分电路可以独立设计和调试。变频、中放、低放各部分相对独立,调试和维护更加方便。

\subsubsection{性能特点}

三管收音机的性能比二管收音机有了显著提升:

\paragraph{接收频段扩展}
除中波外,部分三管收音机还可以接收短波(SW)。通过增加波段开关,可以实现多波段接收。

\paragraph{选择性大幅提高}
可以清晰接收相邻频率的广播信号。超外差式电路的选择性远优于再生式电路,抗干扰能力强。

\paragraph{输出功率增加}
可以驱动更大的扬声器,声音更加洪亮。三管收音机的输出功率通常在1W左右。

\paragraph{使用便利性提升}
部分机型开始使用交流电源,减少了电池消耗。使用电源变压器供电,更加经济方便。

\subsection{晶体管三管收音机}

晶体管三管收音机是20世纪70-80年代的普及型收音机,采用超外差式电路架构。相比电子管三管收音机,晶体管三管收音机具有体积小、功耗低、寿命长等优点。

\subsubsection{电路结构特点}

三管收音机的典型电路包括三个主要部分:
1. **变频管**:将接收到的高频信号转换为固定频率的中频信号
2. **中放管**:对中频信号进行放大,提高灵敏度和选择性
3. **低放管**:对检波后的音频信号进行放大,驱动扬声器

\subsubsection{电路图结构}

```
[天线] ---> [输入回路(L1 + C1)] ---> [变频管基极(Q1-B)]
                                   |          |
                                   |          v
                                   |   [变频管本机振荡(Q1-Osc)]
                                   |          |
                                   |          v
                                   |   [混频输出中频(465kHz)]
                                   |          |
                                   |          v
                                   |   [中频变压器(T1)]
                                   |          |
                                   |          v
                                   |   [中放管基极(Q2-B)]
                                   |          |
                                   |          v
                                   |   [中放管集电极(Q2-C)]
                                   |          |
                                   |          v
                                   |   [检波器(D1)]
                                   |          |
                                   |          v
                                   |   [耦合电容(C2)]
                                   |          |
                                   |          v
                                   |   [低放管基极(Q3-B)]
                                   |          |
                                   |          v
                                   |   [低放管集电极(Q3-C)]
                                   |          |
                                   |          v
                                   |---[扬声器(SPK)]
```

\paragraph{电路说明}

- **Q1**:变频管,通常使用NPN型高频晶体管(如3DG6、9018),负责变频
- **Q2**:中放管,通常使用NPN型高频晶体管(如3DG6、9013),负责中频放大
- **Q3**:低放管,通常使用NPN型功率晶体管(如3DG12、8050),负责音频放大
- **L1 + C1**:输入调谐回路,选择接收频率
- **T1**:中频变压器(中周),调谐在465kHz,提高选择性
- **D1**:检波二极管,将中频信号转换为音频信号
- **C2**:耦合电容,将音频信号耦合到低放管

\subsubsection{元器件清单}

\begin{table}[H]
    \centering
    \caption{晶体管三管收音机元器件清单}
    \label{tab:transistor_three_radio_parts}
    \begin{tabular}{ccc}
        \toprule
        元件名称 & 型号/规格 & 用途 \\ 
        \midrule
        变频晶体管 & 3DG6(或9018、3AG1) & 高频放大与变频 \\
        中放晶体管 & 3DG6(或9013、3DG201) & 中频信号放大 \\
        低放晶体管 & 3DG12(或8050、3DD15) & 音频功率放大 \\
        输入线圈 & L1:200-300匝,0.2mm漆包线 & 接收信号调谐 \\
        可变电容 & C1:270pF空气可变电容 & 调谐选台 \\
        本机振荡线圈 & L2:100-150匝,0.2mm漆包线 & 产生本机振荡信号 \\
        中频变压器 & T1:465kHz中周 & 中频信号选频与耦合 \\
        耦合电容 & C2:0.01μF瓷片电容 & 音频信号耦合 \\
        旁路电容 & C3:10μF电解电容 & 电源滤波 \\
        偏置电阻 & R1:100kΩ碳膜电阻 & 变频管基极偏置 \\
        偏置电阻 & R2:100kΩ碳膜电阻 & 中放管基极偏置 \\
        偏置电阻 & R3:100kΩ碳膜电阻 & 低放管基极偏置 \\
        发射极电阻 & R4:1kΩ碳膜电阻 & 变频管发射极电阻 \\
        发射极电阻 & R5:500Ω碳膜电阻 & 中放管发射极电阻 \\
        发射极电阻 & R6:100Ω碳膜电阻 & 低放管发射极电阻 \\
        检波二极管 & D1:2AP9(或1N4148) & 中频检波 \\
        扬声器 & 8Ω/1W动圈扬声器 & 音频输出 \\
        电池 & 9V层叠电池 & 电源 \\
        \bottomrule
    \end{tabular}
\end{table}

\subsubsection{超外差式电路的优势}

晶体管三管收音机采用超外差式电路,具有与电子管三管收音机相同的优势:

\paragraph{高选择性}
通过固定中频的选频回路,可以有效滤除其他频率的干扰信号。

\paragraph{高灵敏度}
中频放大器可以针对固定频率进行优化设计,提高放大效率。

\paragraph{性能稳定}
电路工作状态受信号频率变化的影响较小。

\paragraph{易于调试}
各部分电路可以独立设计和调试。

\subsubsection{性能特点}

晶体管三管收音机的性能特点:

\paragraph{接收频段扩展}
除中波外,部分三管收音机还可以接收短波(SW)。

\paragraph{选择性大幅提高}
可以清晰接收相邻频率的广播信号。

\paragraph{输出功率增加}
可以驱动更大的扬声器,声音更加洪亮。

\paragraph{使用便利性提升}
电池供电,便于携带使用。

\subsection{场效应管三管收音机}

场效应管三管收音机利用三个场效应管分别负责变频、中放和低放。相比晶体管三管收音机,场效应管三管收音机具有更高的输入阻抗和更低的噪声,适合高灵敏度接收。

\subsubsection{电路结构特点}

三管收音机的典型电路包括三个主要部分:
1. **变频管**:将接收到的高频信号转换为固定频率的中频信号
2. **中放管**:对中频信号进行放大,提高灵敏度和选择性
3. **低放管**:对检波后的音频信号进行放大,驱动扬声器

\subsubsection{电路图结构}

```
[天线] ---> [输入回路(L1 + C1)] ---> [变频管栅极(Q1-G)]
                                   |          |
                                   |          v
                                   |   [变频管本机振荡(Q1-Osc)]
                                   |          |
                                   |          v
                                   |   [混频输出中频(465kHz)]
                                   |          |
                                   |          v
                                   |   [中频变压器(T1)]
                                   |          |
                                   |          v
                                   |   [中放管栅极(Q2-G)]
                                   |          |
                                   |          v
                                   |   [中放管漏极(Q2-D)]
                                   |          |
                                   |          v
                                   |   [检波器(D1)]
                                   |          |
                                   |          v
                                   |   [耦合电容(C2)]
                                   |          |
                                   |          v
                                   |   [低放管栅极(Q3-G)]
                                   |          |
                                   |          v
                                   |   [低放管漏极(Q3-D)]
                                   |          |
                                   |          v
                                   |---[扬声器(SPK)]
```

\paragraph{电路说明}

- **Q1**:变频管,通常使用结型场效应管(如3DJ6、2SK241),负责变频
- **Q2**:中放管,通常使用结型场效应管(如3DJ6、2SK241),负责中频放大
- **Q3**:低放管,通常使用功率场效应管(如2SK135、IRF510),负责音频放大
- **L1 + C1**:输入调谐回路,选择接收频率
- **T1**:中频变压器(中周),调谐在465kHz,提高选择性
- **D1**:检波二极管,将中频信号转换为音频信号
- **C2**:耦合电容,将音频信号耦合到低放管

\subsubsection{元器件清单}

\begin{table}[H]
    \centering
    \caption{场效应管三管收音机元器件清单}
    \label{tab:fet_three_radio_parts}
    \begin{tabular}{ccc}
        \toprule
        元件名称 & 型号/规格 & 用途 \\ 
        \midrule
        变频场效应管 & 3DJ6(或2SK241、BF245) & 高频放大与变频 \\
        中放场效应管 & 3DJ6(或2SK241、BF245) & 中频信号放大 \\
        低放场效应管 & 2SK135(或IRF510、IRF540) & 音频功率放大 \\
        输入线圈 & L1:200-300匝,0.2mm漆包线 & 接收信号调谐 \\
        可变电容 & C1:270pF空气可变电容 & 调谐选台 \\
        本机振荡线圈 & L2:100-150匝,0.2mm漆包线 & 产生本机振荡信号 \\
        中频变压器 & T1:465kHz中周 & 中频信号选频与耦合 \\
        耦合电容 & C2:0.01μF瓷片电容 & 音频信号耦合 \\
        旁路电容 & C3:10μF电解电容 & 电源滤波 \\
        栅极电阻 & R1:1MΩ碳膜电阻 & 变频管栅极偏置 \\
        栅极电阻 & R2:1MΩ碳膜电阻 & 中放管栅极偏置 \\
        栅极电阻 & R3:1MΩ碳膜电阻 & 低放管栅极偏置 \\
        源极电阻 & R4:1kΩ碳膜电阻 & 变频管源极电阻 \\
        源极电阻 & R5:500Ω碳膜电阻 & 中放管源极电阻 \\
        源极电阻 & R6:100Ω碳膜电阻 & 低放管源极电阻 \\
        检波二极管 & D1:2AP9(或1N4148) & 中频检波 \\
        扬声器 & 8Ω/1W动圈扬声器 & 音频输出 \\
        电池 & 9V层叠电池 & 电源 \\
        \bottomrule
    \end{tabular}
\end{table}

\subsubsection{超外差式电路的优势}

场效应管三管收音机采用超外差式电路,具有与电子管和晶体管三管收音机相同的优势:

\paragraph{高选择性}
通过固定中频的选频回路,可以有效滤除其他频率的干扰信号。

\paragraph{高灵敏度}
中频放大器可以针对固定频率进行优化设计,提高放大效率。

\paragraph{性能稳定}
电路工作状态受信号频率变化的影响较小。

\paragraph{易于调试}
各部分电路可以独立设计和调试。

\subsubsection{性能特点}

场效应管三管收音机的性能特点:

\paragraph{接收频段扩展}
除中波外,部分三管收音机还可以接收短波(SW)。

\paragraph{选择性大幅提高}
可以清晰接收相邻频率的广播信号。场效应管的高输入阻抗使得调谐回路Q值更高,选择性更好。

\paragraph{输出功率增加}
可以驱动更大的扬声器,声音更加洪亮。

\paragraph{使用便利性提升}
电池供电,便于携带使用。场效应管的低噪声特性使得接收效果更好。

\chapter{四管收音机}

四管收音机是超外差式收音机的完善阶段,通过增加第四个有源器件,进一步优化了收音机的各项性能指标。四管收音机通常采用变频管、二级中放管和低放管的配置,性能相比三管收音机有了显著提升。

\subsection{电子管四管收音机}

电子管四管收音机是20世纪50-60年代的高档收音机类型,采用二级中频放大电路。四管收音机包括变频管、二级中放管和低放管,实现了更高的灵敏度和选择性。

\subsubsection{电路结构特点}

四管收音机的典型电路包括四个主要部分:
1. **变频管**:将接收到的高频信号转换为固定频率的中频信号(通常为465kHz)
2. **第一中放管**:对中频信号进行第一级放大
3. **第二中放管**:对中频信号进行第二级放大,进一步提高灵敏度和选择性
4. **低放管**:对检波后的音频信号进行放大,驱动扬声器

\subsubsection{电路图结构}

```
[天线] ---> [输入回路(L1 + C1)] ---> [变频管栅极(V1-G)]
                                   |          |
                                   |          v
                                   |   [变频管本机振荡(V1-Osc)]
                                   |          |
                                   |          v
                                   |   [混频输出中频(465kHz)]
                                   |          |
                                   |          v
                                   |   [第一中频变压器(T1)]
                                   |          |
                                   |          v
                                   |   [第一中放管栅极(V2-G)]
                                   |          |
                                   |          v
                                   |   [第一中放管阳极(V2-A)]
                                   |          |
                                   |          v
                                   |   [第二中频变压器(T2)]
                                   |          |
                                   |          v
                                   |   [第二中放管栅极(V3-G)]
                                   |          |
                                   |          v
                                   |   [第二中放管阳极(V3-A)]
                                   |          |
                                   |          v
                                   |   [检波器(D1)]
                                   |          |
                                   |          v
                                   |   [耦合电容(C2)]
                                   |          |
                                   |          v
                                   |   [低放管栅极(V4-G)]
                                   |          |
                                   |          v
                                   |   [低放管阳极(V4-A)]
                                   |          |
                                   |          v
                                   |---[扬声器(SPK)]
```

\paragraph{电路说明}

- **V1**:变频管,通常使用变频三极管(如6A2),负责变频
- **V2**:第一中放管,通常使用高增益三极管(如6K4),负责第一级中频放大
- **V3**:第二中放管,通常使用高增益三极管(如6K4),负责第二级中频放大
- **V4**:低放管,通常使用功率三极管(如6P1),负责音频功率放大
- **L1 + C1**:输入调谐回路,选择接收频率
- **T1、T2**:中频变压器(中周),调谐在465kHz,提高选择性
- **D1**:检波二极管,将中频信号转换为音频信号
- **C2**:耦合电容,将音频信号耦合到低放管

\subsubsection{元器件清单}

\begin{table}[H]
    \centering
    \caption{电子管四管收音机元器件清单}
    \label{tab:vacuum_tube_four_radio_parts}
    \begin{tabular}{ccc}
        \toprule
        元件名称 & 型号/规格 & 用途 \\ 
        \midrule
        变频电子管 & 6A2(或6SA7、1A2) & 高频放大与变频 \\
        第一中放电子管 & 6K4(或6SK7、3AG1) & 第一级中频放大 \\
        第二中放电子管 & 6K4(或6SK7、3AG1) & 第二级中频放大 \\
        低放电子管 & 6P1(或6P14、6V6) & 音频功率放大 \\
        输入线圈 & L1:200-300匝,0.2mm漆包线 & 接收信号调谐 \\
        可变电容 & C1:270pF空气可变电容 & 调谐选台 \\
        本机振荡线圈 & L2:100-150匝,0.2mm漆包线 & 产生本机振荡信号 \\
        第一中频变压器 & T1:465kHz中周 & 第一级中频选频与耦合 \\
        第二中频变压器 & T2:465kHz中周 & 第二级中频选频与耦合 \\
        耦合电容 & C2:0.01μF瓷片电容 & 音频信号耦合 \\
        滤波电容 & C3:10μF电解电容 & 电源滤波 \\
        栅极电阻 & R1:200kΩ碳膜电阻 & 变频管栅极电阻 \\
        栅极电阻 & R2:1MΩ碳膜电阻 & 第一中放管栅极电阻 \\
        栅极电阻 & R3:1MΩ碳膜电阻 & 第二中放管栅极电阻 \\
        阴极电阻 & R4:1kΩ碳膜电阻 & 变频管阴极电阻 \\
        阴极电阻 & R5:500Ω碳膜电阻 & 第一中放管阴极电阻 \\
        阴极电阻 & R6:500Ω碳膜电阻 & 第二中放管阴极电阻 \\
        阴极电阻 & R7:200Ω碳膜电阻 & 低放管阴极电阻 \\
        阳极电阻 & R8:10kΩ碳膜电阻 & 中放管阳极负载 \\
        检波二极管 & D1:2AP9(或1N4148) & 中频检波 \\
        扬声器 & 8Ω/2W动圈扬声器 & 音频输出 \\
        电源变压器 & 220V/6.3V×3 + 12V & 提供电子管灯丝和阳极电压 \\
        \bottomrule
    \end{tabular}
\end{table}

\subsubsection{性能提升}

与三管收音机相比,四管收音机的主要性能提升包括:

\paragraph{灵敏度大幅提高}
二级中频放大显著提高了收音机的灵敏度。中频增益的增加使得收音机能够接收更远的电台和更弱的信号。

\paragraph{选择性显著改善}
二级中频放大和双中频变压器大大提高了选择性。可以清晰接收相邻频率的广播信号,抗干扰能力显著增强。

\paragraph{音质更加清晰}
中频放大的增加使得音频信号的提取更加充分,音质更加清晰,失真更小。

\paragraph{稳定性增强}
二级中频放大使得电路工作更加稳定,受外界干扰的影响更小。

\subsection{晶体管四管收音机}

晶体管四管收音机是20世纪70-80年代的高档收音机类型,采用二级中频放大电路。相比电子管四管收音机,晶体管四管收音机具有体积小、功耗低、寿命长等优点。

\subsubsection{电路结构特点}

四管收音机的典型电路包括四个主要部分:
1. **变频管**:将接收到的高频信号转换为固定频率的中频信号
2. **第一中放管**:对中频信号进行第一级放大
3. **第二中放管**:对中频信号进行第二级放大
4. **低放管**:对检波后的音频信号进行放大,驱动扬声器

\subsubsection{电路图结构}

```
[天线] ---> [输入回路(L1 + C1)] ---> [变频管基极(Q1-B)]
                                   |          |
                                   |          v
                                   |   [变频管本机振荡(Q1-Osc)]
                                   |          |
                                   |          v
                                   |   [混频输出中频(465kHz)]
                                   |          |
                                   |          v
                                   |   [第一中频变压器(T1)]
                                   |          |
                                   |          v
                                   |   [第一中放管基极(Q2-B)]
                                   |          |
                                   |          v
                                   |   [第一中放管集电极(Q2-C)]
                                   |          |
                                   |          v
                                   |   [第二中频变压器(T2)]
                                   |          |
                                   |          v
                                   |   [第二中放管基极(Q3-B)]
                                   |          |
                                   |          v
                                   |   [第二中放管集电极(Q3-C)]
                                   |          |
                                   |          v
                                   |   [检波器(D1)]
                                   |          |
                                   |          v
                                   |   [耦合电容(C2)]
                                   |          |
                                   |          v
                                   |   [低放管基极(Q4-B)]
                                   |          |
                                   |          v
                                   |   [低放管集电极(Q4-C)]
                                   |          |
                                   |          v
                                   |---[扬声器(SPK)]
```

\paragraph{电路说明}

- **Q1**:变频管,通常使用NPN型高频晶体管(如3DG6、9018),负责变频
- **Q2**:第一中放管,通常使用NPN型高频晶体管(如3DG6、9013),负责第一级中频放大
- **Q3**:第二中放管,通常使用NPN型高频晶体管(如3DG6、9013),负责第二级中频放大
- **Q4**:低放管,通常使用NPN型功率晶体管(如3DG12、8050),负责音频放大
- **L1 + C1**:输入调谐回路,选择接收频率
- **T1、T2**:中频变压器(中周),调谐在465kHz,提高选择性
- **D1**:检波二极管,将中频信号转换为音频信号
- **C2**:耦合电容,将音频信号耦合到低放管

\subsubsection{元器件清单}

\begin{table}[H]
    \centering
    \caption{晶体管四管收音机元器件清单}
    \label{tab:transistor_four_radio_parts}
    \begin{tabular}{ccc}
        \toprule
        元件名称 & 型号/规格 & 用途 \\ 
        \midrule
        变频晶体管 & 3DG6(或9018、3AG1) & 高频放大与变频 \\
        第一中放晶体管 & 3DG6(或9013、3DG201) & 第一级中频放大 \\
        第二中放晶体管 & 3DG6(或9013、3DG201) & 第二级中频放大 \\
        低放晶体管 & 3DG12(或8050、3DD15) & 音频功率放大 \\
        输入线圈 & L1:200-300匝,0.2mm漆包线 & 接收信号调谐 \\
        可变电容 & C1:270pF空气可变电容 & 调谐选台 \\
        本机振荡线圈 & L2:100-150匝,0.2mm漆包线 & 产生本机振荡信号 \\
        第一中频变压器 & T1:465kHz中周 & 第一级中频选频与耦合 \\
        第二中频变压器 & T2:465kHz中周 & 第二级中频选频与耦合 \\
        耦合电容 & C2:0.01μF瓷片电容 & 音频信号耦合 \\
        旁路电容 & C3:10μF电解电容 & 电源滤波 \\
        偏置电阻 & R1:100kΩ碳膜电阻 & 变频管基极偏置 \\
        偏置电阻 & R2:100kΩ碳膜电阻 & 第一中放管基极偏置 \\
        偏置电阻 & R3:100kΩ碳膜电阻 & 第二中放管基极偏置 \\
        偏置电阻 & R4:100kΩ碳膜电阻 & 低放管基极偏置 \\
        发射极电阻 & R5:1kΩ碳膜电阻 & 变频管发射极电阻 \\
        发射极电阻 & R6:500Ω碳膜电阻 & 第一中放管发射极电阻 \\
        发射极电阻 & R7:500Ω碳膜电阻 & 第二中放管发射极电阻 \\
        发射极电阻 & R8:100Ω碳膜电阻 & 低放管发射极电阻 \\
        检波二极管 & D1:2AP9(或1N4148) & 中频检波 \\
        扬声器 & 8Ω/2W动圈扬声器 & 音频输出 \\
        电池 & 9V层叠电池 & 电源 \\
        \bottomrule
    \end{tabular}
\end{table}

\subsubsection{性能提升}

与三管收音机相比,晶体管四管收音机的主要性能提升包括:

\paragraph{灵敏度大幅提高}
二级中频放大显著提高了收音机的灵敏度。可以接收更远的电台和更弱的信号。

\paragraph{选择性显著改善}
二级中频放大和双中频变压器大大提高了选择性。可以清晰接收相邻频率的广播信号。

\paragraph{音质更加清晰}
中频放大的增加使得音频信号的提取更加充分,音质更加清晰。

\paragraph{稳定性增强}
晶体管工作稳定,二级中频放大使得电路工作更加稳定。

\subsection{场效应管四管收音机}

场效应管四管收音机利用四个场效应管分别负责变频、二级中放和低放。相比晶体管四管收音机,场效应管四管收音机具有更高的输入阻抗和更低的噪声,适合高灵敏度接收。

\subsubsection{电路结构特点}

四管收音机的典型电路包括四个主要部分:
1. **变频管**:将接收到的高频信号转换为固定频率的中频信号
2. **第一中放管**:对中频信号进行第一级放大
3. **第二中放管**:对中频信号进行第二级放大
4. **低放管**:对检波后的音频信号进行放大,驱动扬声器

\subsubsection{电路图结构}

```
[天线] ---> [输入回路(L1 + C1)] ---> [变频管栅极(Q1-G)]
                                   |          |
                                   |          v
                                   |   [变频管本机振荡(Q1-Osc)]
                                   |          |
                                   |          v
                                   |   [混频输出中频(465kHz)]
                                   |          |
                                   |          v
                                   |   [第一中频变压器(T1)]
                                   |          |
                                   |          v
                                   |   [第一中放管栅极(Q2-G)]
                                   |          |
                                   |          v
                                   |   [第一中放管漏极(Q2-D)]
                                   |          |
                                   |          v
                                   |   [第二中频变压器(T2)]
                                   |          |
                                   |          v
                                   |   [第二中放管栅极(Q3-G)]
                                   |          |
                                   |          v
                                   |   [第二中放管漏极(Q3-D)]
                                   |          |
                                   |          v
                                   |   [检波器(D1)]
                                   |          |
                                   |          v
                                   |   [耦合电容(C2)]
                                   |          |
                                   |          v
                                   |   [低放管栅极(Q4-G)]
                                   |          |
                                   |          v
                                   |   [低放管漏极(Q4-D)]
                                   |          |
                                   |          v
                                   |---[扬声器(SPK)]
```

\paragraph{电路说明}

- **Q1**:变频管,通常使用结型场效应管(如3DJ6、2SK241),负责变频
- **Q2**:第一中放管,通常使用结型场效应管(如3DJ6、2SK241),负责第一级中频放大
- **Q3**:第二中放管,通常使用结型场效应管(如3DJ6、2SK241),负责第二级中频放大
- **Q4**:低放管,通常使用功率场效应管(如2SK135、IRF510),负责音频放大
- **L1 + C1**:输入调谐回路,选择接收频率
- **T1、T2**:中频变压器(中周),调谐在465kHz,提高选择性
- **D1**:检波二极管,将中频信号转换为音频信号
- **C2**:耦合电容,将音频信号耦合到低放管

\subsubsection{元器件清单}

\begin{table}[H]
    \centering
    \caption{场效应管四管收音机元器件清单}
    \label{tab:fet_four_radio_parts}
    \begin{tabular}{ccc}
        \toprule
        元件名称 & 型号/规格 & 用途 \\ 
        \midrule
        变频场效应管 & 3DJ6(或2SK241、BF245) & 高频放大与变频 \\
        第一中放场效应管 & 3DJ6(或2SK241、BF245) & 第一级中频放大 \\
        第二中放场效应管 & 3DJ6(或2SK241、BF245) & 第二级中频放大 \\
        低放场效应管 & 2SK135(或IRF510、IRF540) & 音频功率放大 \\
        输入线圈 & L1:200-300匝,0.2mm漆包线 & 接收信号调谐 \\
        可变电容 & C1:270pF空气可变电容 & 调谐选台 \\
        本机振荡线圈 & L2:100-150匝,0.2mm漆包线 & 产生本机振荡信号 \\
        第一中频变压器 & T1:465kHz中周 & 第一级中频选频与耦合 \\
        第二中频变压器 & T2:465kHz中周 & 第二级中频选频与耦合 \\
        耦合电容 & C2:0.01μF瓷片电容 & 音频信号耦合 \\
        旁路电容 & C3:10μF电解电容 & 电源滤波 \\
        栅极电阻 & R1:1MΩ碳膜电阻 & 变频管栅极偏置 \\
        栅极电阻 & R2:1MΩ碳膜电阻 & 第一中放管栅极偏置 \\
        栅极电阻 & R3:1MΩ碳膜电阻 & 第二中放管栅极偏置 \\
        栅极电阻 & R4:1MΩ碳膜电阻 & 低放管栅极偏置 \\
        源极电阻 & R5:1kΩ碳膜电阻 & 变频管源极电阻 \\
        源极电阻 & R6:500Ω碳膜电阻 & 第一中放管源极电阻 \\
        源极电阻 & R7:500Ω碳膜电阻 & 第二中放管源极电阻 \\
        源极电阻 & R8:100Ω碳膜电阻 & 低放管源极电阻 \\
        检波二极管 & D1:2AP9(或1N4148) & 中频检波 \\
        扬声器 & 8Ω/2W动圈扬声器 & 音频输出 \\
        电池 & 9V层叠电池 & 电源 \\
        \bottomrule
    \end{tabular}
\end{table}

\subsubsection{性能提升}

与三管收音机相比,场效应管四管收音机的主要性能提升包括:

\paragraph{灵敏度大幅提高}
二级中频放大显著提高了收音机的灵敏度。场效应管的高输入阻抗使得调谐回路Q值更高,灵敏度更好。

\paragraph{选择性显著改善}
二级中频放大和双中频变压器大大提高了选择性。可以清晰接收相邻频率的广播信号。

\paragraph{音质更加清晰}
场效应管的低噪声特性和良好的线性度,使得音质更加清晰,失真更小。

\paragraph{稳定性增强}
场效应管参数受温度影响较小,二级中频放大使得电路工作更加稳定。

\chapter{五管收音机}

五管收音机在四管收音机的基础上进一步完善电路,通常采用变频管、二级中放管、检波管和推挽低放管的配置。五管收音机是超外差式收音机的成熟阶段,性能相比四管收音机有了显著提升。

\subsection{电子管五管收音机}

电子管五管收音机是20世纪50-60年代的高档收音机类型,采用推挽功率放大电路。五管收音机包括变频管、二级中放管、检波管和推挽低放管,实现了更高的输出功率和更好的音质。

\subsubsection{电路结构特点}

五管收音机的典型电路包括五个主要部分:
1. **变频管**:将接收到的高频信号转换为固定频率的中频信号(通常为465kHz)
2. **第一中放管**:对中频信号进行第一级放大
3. **第二中放管**:对中频信号进行第二级放大
4. **检波管**:使用专用二极管进行检波,提高检波效率
5. **推挽低放管**:使用两个电子管组成推挽功放电路,提高输出功率和音质

\subsubsection{电路图结构}

```
[天线] ---> [输入回路(L1 + C1)] ---> [变频管栅极(V1-G)]
                                   |          |
                                   |          v
                                   |   [变频管本机振荡(V1-Osc)]
                                   |          |
                                   |          v
                                   |   [混频输出中频(465kHz)]
                                   |          |
                                   |          v
                                   |   [第一中频变压器(T1)]
                                   |          |
                                   |          v
                                   |   [第一中放管栅极(V2-G)]
                                   |          |
                                   |          v
                                   |   [第一中放管阳极(V2-A)]
                                   |          |
                                   |          v
                                   |   [第二中频变压器(T2)]
                                   |          |
                                   |          v
                                   |   [第二中放管栅极(V3-G)]
                                   |          |
                                   |          v
                                   |   [第二中放管阳极(V3-A)]
                                   |          |
                                   |          v
                                   |   [检波管(V4)]
                                   |          |
                                   |          v
                                   |   [耦合电容(C2)]
                                   |          |
                                   |          v
                                   |   [推挽低放管栅极(V5-G1、V5-G2)]
                                   |          |
                                   |          v
                                   |   [输出变压器(T3)]
                                   |          |
                                   |          v
                                   |---[扬声器(SPK)]
```

\paragraph{电路说明}

- **V1**:变频管,通常使用变频三极管(如6A2),负责变频
- **V2**:第一中放管,通常使用高增益三极管(如6K4),负责第一级中频放大
- **V3**:第二中放管,通常使用高增益三极管(如6K4),负责第二级中频放大
- **V4**:检波管,通常使用双二极管(如6H2),负责中频检波和AGC
- **V5**:推挽低放管,通常使用功率三极管(如6P1×2),负责音频功率放大
- **L1 + C1**:输入调谐回路,选择接收频率
- **T1、T2**:中频变压器(中周),调谐在465kHz,提高选择性
- **T3**:输出变压器,匹配推挽功放和扬声器阻抗
- **C2**:耦合电容,将音频信号耦合到低放管

\subsubsection{元器件清单}

\begin{table}[H]
    \centering
    \caption{电子管五管收音机元器件清单}
    \label{tab:vacuum_tube_five_radio_parts}
    \begin{tabular}{ccc}
        \toprule
        元件名称 & 型号/规格 & 用途 \\ 
        \midrule
        变频电子管 & 6A2(或6SA7、1A2) & 高频放大与变频 \\
        第一中放电子管 & 6K4(或6SK7、3AG1) & 第一级中频放大 \\
        第二中放电子管 & 6K4(或6SK7、3AG1) & 第二级中频放大 \\
        检波电子管 & 6H2(或6H1、6N2) & 中频检波与AGC \\
        推挽低放电子管 & 6P1×2(或6P14×2) & 推挽音频功率放大 \\
        输入线圈 & L1:200-300匝,0.2mm漆包线 & 接收信号调谐 \\
        可变电容 & C1:270pF空气可变电容 & 调谐选台 \\
        本机振荡线圈 & L2:100-150匝,0.2mm漆包线 & 产生本机振荡信号 \\
        第一中频变压器 & T1:465kHz中周 & 第一级中频选频与耦合 \\
        第二中频变压器 & T2:465kHz中周 & 第二级中频选频与耦合 \\
        耦合电容 & C2:0.01μF瓷片电容 & 音频信号耦合 \\
        滤波电容 & C3:10μF电解电容 & 电源滤波 \\
        栅极电阻 & R1:200kΩ碳膜电阻 & 变频管栅极电阻 \\
        栅极电阻 & R2:1MΩ碳膜电阻 & 第一中放管栅极电阻 \\
        栅极电阻 & R3:1MΩ碳膜电阻 & 第二中放管栅极电阻 \\
        阴极电阻 & R4:1kΩ碳膜电阻 & 变频管阴极电阻 \\
        阴极电阻 & R5:500Ω碳膜电阻 & 第一中放管阴极电阻 \\
        阴极电阻 & R6:500Ω碳膜电阻 & 第二中放管阴极电阻 \\
        阴极电阻 & R7:200Ω碳膜电阻 & 推挽低放管阴极电阻 \\
        阳极电阻 & R8:10kΩ碳膜电阻 & 中放管阳极负载 \\
        扬声器 & 8Ω/3W动圈扬声器 & 音频输出 \\
        电源变压器 & 220V/6.3V×4 + 12V & 提供电子管灯丝和阳极电压 \\
        \bottomrule
    \end{tabular}
\end{table}

\subsubsection{性能提升}

与四管收音机相比,五管收音机的主要性能提升包括:

\paragraph{输出功率大幅增加}
推挽功率放大电路显著提高了输出功率。输出功率从四管的2W左右提升到5W以上,声音更加洪亮。

\paragraph{音质显著改善}
推挽电路的对称性和专用检波管使得音质更加清晰,失真更小。音频信号的动态范围进一步扩大。

\paragraph{自动增益控制(AGC)}
专用检波管可以实现自动增益控制功能,使收音机在接收强弱不同的信号时保持输出音量的稳定。

\paragraph{稳定性增强}
推挽电路的工作稳定性更好,受外界干扰的影响更小。

\subsection{晶体管五管收音机}

晶体管五管收音机是20世纪70-80年代的高档收音机类型,采用推挽功率放大电路。相比电子管五管收音机,晶体管五管收音机具有体积小、功耗低、寿命长等优点。

\subsubsection{电路结构特点}

五管收音机的典型电路包括五个主要部分:
1. **变频管**:将接收到的高频信号转换为固定频率的中频信号
2. **第一中放管**:对中频信号进行第一级放大
3. **第二中放管**:对中频信号进行第二级放大
4. **检波管**:使用专用二极管进行检波
5. **推挽低放管**:使用两个晶体管组成推挽功放电路,提高输出功率

\subsubsection{电路图结构}

```
[天线] ---> [输入回路(L1 + C1)] ---> [变频管基极(Q1-B)]
                                   |          |
                                   |          v
                                   |   [变频管本机振荡(Q1-Osc)]
                                   |          |
                                   |          v
                                   |   [混频输出中频(465kHz)]
                                   |          |
                                   |          v
                                   |   [第一中频变压器(T1)]
                                   |          |
                                   |          v
                                   |   [第一中放管基极(Q2-B)]
                                   |          |
                                   |          v
                                   |   [第一中放管集电极(Q2-C)]
                                   |          |
                                   |          v
                                   |   [第二中频变压器(T2)]
                                   |          |
                                   |          v
                                   |   [第二中放管基极(Q3-B)]
                                   |          |
                                   |          v
                                   |   [第二中放管集电极(Q3-C)]
                                   |          |
                                   |          v
                                   |   [检波器(D1)]
                                   |          |
                                   |          v
                                   |   [耦合电容(C2)]
                                   |          |
                                   |          v
                                   |   [推挽低放管基极(Q4-B、Q5-B)]
                                   |          |
                                   |          v
                                   |   [输出变压器(T3)]
                                   |          |
                                   |          v
                                   |---[扬声器(SPK)]
```

\paragraph{电路说明}

- **Q1**:变频管,通常使用NPN型高频晶体管(如3DG6、9018),负责变频
- **Q2**:第一中放管,通常使用NPN型高频晶体管(如3DG6、9013),负责第一级中频放大
- **Q3**:第二中放管,通常使用NPN型高频晶体管(如3DG6、9013),负责第二级中频放大
- **D1**:检波二极管,将中频信号转换为音频信号
- **Q4、Q5**:推挽低放管,通常使用NPN型功率晶体管(如3DG12×2、8050×2),负责音频功率放大
- **L1 + C1**:输入调谐回路,选择接收频率
- **T1、T2**:中频变压器(中周),调谐在465kHz,提高选择性
- **T3**:输出变压器,匹配推挽功放和扬声器阻抗
- **C2**:耦合电容,将音频信号耦合到低放管

\subsubsection{元器件清单}

\begin{table}[H]
    \centering
    \caption{晶体管五管收音机元器件清单}
    \label{tab:transistor_five_radio_parts}
    \begin{tabular}{ccc}
        \toprule
        元件名称 & 型号/规格 & 用途 \\ 
        \midrule
        变频晶体管 & 3DG6(或9018、3AG1) & 高频放大与变频 \\
        第一中放晶体管 & 3DG6(或9013、3DG201) & 第一级中频放大 \\
        第二中放晶体管 & 3DG6(或9013、3DG201) & 第二级中频放大 \\
        检波二极管 & D1:2AP9(或1N4148) & 中频检波 \\
        推挽低放晶体管 & 3DG12×2(或8050×2) & 推挽音频功率放大 \\
        输入线圈 & L1:200-300匝,0.2mm漆包线 & 接收信号调谐 \\
        可变电容 & C1:270pF空气可变电容 & 调谐选台 \\
        本机振荡线圈 & L2:100-150匝,0.2mm漆包线 & 产生本机振荡信号 \\
        第一中频变压器 & T1:465kHz中周 & 第一级中频选频与耦合 \\
        第二中频变压器 & T2:465kHz中周 & 第二级中频选频与耦合 \\
        耦合电容 & C2:0.01μF瓷片电容 & 音频信号耦合 \\
        旁路电容 & C3:10μF电解电容 & 电源滤波 \\
        偏置电阻 & R1:100kΩ碳膜电阻 & 变频管基极偏置 \\
        偏置电阻 & R2:100kΩ碳膜电阻 & 第一中放管基极偏置 \\
        偏置电阻 & R3:100kΩ碳膜电阻 & 第二中放管基极偏置 \\
        偏置电阻 & R4:100kΩ碳膜电阻 & 推挽低放管基极偏置 \\
        发射极电阻 & R5:1kΩ碳膜电阻 & 变频管发射极电阻 \\
        发射极电阻 & R6:500Ω碳膜电阻 & 第一中放管发射极电阻 \\
        发射极电阻 & R7:500Ω碳膜电阻 & 第二中放管发射极电阻 \\
        发射极电阻 & R8:100Ω碳膜电阻 & 推挽低放管发射极电阻 \\
        扬声器 & 8Ω/3W动圈扬声器 & 音频输出 \\
        电池 & 9V层叠电池 & 电源 \\
        \bottomrule
    \end{tabular}
\end{table}

\subsubsection{性能提升}

与四管收音机相比,晶体管五管收音机的主要性能提升包括:

\paragraph{输出功率大幅增加}
推挽功率放大电路显著提高了输出功率。输出功率从四管的2W左右提升到5W以上。

\paragraph{音质显著改善}
推挽电路的对称性使得音质更加清晰,失真更小。

\paragraph{稳定性增强}
晶体管工作稳定,推挽电路的工作稳定性更好。

\subsection{场效应管五管收音机}

场效应管五管收音机利用五个场效应管分别负责变频、二级中放、检波和推挽低放。相比晶体管五管收音机,场效应管五管收音机具有更高的输入阻抗和更低的噪声,适合高灵敏度接收。

\subsubsection{电路结构特点}

五管收音机的典型电路包括五个主要部分:
1. **变频管**:将接收到的高频信号转换为固定频率的中频信号
2. **第一中放管**:对中频信号进行第一级放大
3. **第二中放管**:对中频信号进行第二级放大
4. **检波管**:使用专用二极管进行检波
5. **推挽低放管**:使用两个场效应管组成推挽功放电路,提高输出功率

\subsubsection{电路图结构}

```
[天线] ---> [输入回路(L1 + C1)] ---> [变频管栅极(Q1-G)]
                                   |          |
                                   |          v
                                   |   [变频管本机振荡(Q1-Osc)]
                                   |          |
                                   |          v
                                   |   [混频输出中频(465kHz)]
                                   |          |
                                   |          v
                                   |   [第一中频变压器(T1)]
                                   |          |
                                   |          v
                                   |   [第一中放管栅极(Q2-G)]
                                   |          |
                                   |          v
                                   |   [第一中放管漏极(Q2-D)]
                                   |          |
                                   |          v
                                   |   [第二中频变压器(T2)]
                                   |          |
                                   |          v
                                   |   [第二中放管栅极(Q3-G)]
                                   |          |
                                   |          v
                                   |   [第二中放管漏极(Q3-D)]
                                   |          |
                                   |          v
                                   |   [检波器(D1)]
                                   |          |
                                   |          v
                                   |   [耦合电容(C2)]
                                   |          |
                                   |          v
                                   |   [推挽低放管栅极(Q4-G、Q5-G)]
                                   |          |
                                   |          v
                                   |   [输出变压器(T3)]
                                   |          |
                                   |          v
                                   |---[扬声器(SPK)]
```

\paragraph{电路说明}

- **Q1**:变频管,通常使用结型场效应管(如3DJ6、2SK241),负责变频
- **Q2**:第一中放管,通常使用结型场效应管(如3DJ6、2SK241),负责第一级中频放大
- **Q3**:第二中放管,通常使用结型场效应管(如3DJ6、2SK241),负责第二级中频放大
- **D1**:检波二极管,将中频信号转换为音频信号
- **Q4、Q5**:推挽低放管,通常使用功率场效应管(如2SK135×2、IRF510×2),负责音频功率放大
- **L1 + C1**:输入调谐回路,选择接收频率
- **T1、T2**:中频变压器(中周),调谐在465kHz,提高选择性
- **T3**:输出变压器,匹配推挽功放和扬声器阻抗
- **C2**:耦合电容,将音频信号耦合到低放管

\subsubsection{元器件清单}

\begin{table}[H]
    \centering
    \caption{场效应管五管收音机元器件清单}
    \label{tab:fet_five_radio_parts}
    \begin{tabular}{ccc}
        \toprule
        元件名称 & 型号/规格 & 用途 \\ 
        \midrule
        变频场效应管 & 3DJ6(或2SK241、BF245) & 高频放大与变频 \\
        第一中放场效应管 & 3DJ6(或2SK241、BF245) & 第一级中频放大 \\
        第二中放场效应管 & 3DJ6(或2SK241、BF245) & 第二级中频放大 \\
        检波二极管 & D1:2AP9(或1N4148) & 中频检波 \\
        推挽低放场效应管 & 2SK135×2(或IRF510×2) & 推挽音频功率放大 \\
        输入线圈 & L1:200-300匝,0.2mm漆包线 & 接收信号调谐 \\
        可变电容 & C1:270pF空气可变电容 & 调谐选台 \\
        本机振荡线圈 & L2:100-150匝,0.2mm漆包线 & 产生本机振荡信号 \\
        第一中频变压器 & T1:465kHz中周 & 第一级中频选频与耦合 \\
        第二中频变压器 & T2:465kHz中周 & 第二级中频选频与耦合 \\
        耦合电容 & C2:0.01μF瓷片电容 & 音频信号耦合 \\
        旁路电容 & C3:10μF电解电容 & 电源滤波 \\
        栅极电阻 & R1:1MΩ碳膜电阻 & 变频管栅极偏置 \\
        栅极电阻 & R2:1MΩ碳膜电阻 & 第一中放管栅极偏置 \\
        栅极电阻 & R3:1MΩ碳膜电阻 & 第二中放管栅极偏置 \\
        栅极电阻 & R4:1MΩ碳膜电阻 & 推挽低放管栅极偏置 \\
        源极电阻 & R5:1kΩ碳膜电阻 & 变频管源极电阻 \\
        源极电阻 & R6:500Ω碳膜电阻 & 第一中放管源极电阻 \\
        源极电阻 & R7:500Ω碳膜电阻 & 第二中放管源极电阻 \\
        源极电阻 & R8:100Ω碳膜电阻 & 推挽低放管源极电阻 \\
        扬声器 & 8Ω/3W动圈扬声器 & 音频输出 \\
        电池 & 9V层叠电池 & 电源 \\
        \bottomrule
    \end{tabular}
\end{table}

\subsubsection{性能提升}

与四管收音机相比,场效应管五管收音机的主要性能提升包括:

\paragraph{输出功率大幅增加}
推挽功率放大电路显著提高了输出功率。输出功率从四管的2W左右提升到5W以上。

\paragraph{音质显著改善}
场效应管的低噪声特性和推挽电路的对称性,使得音质更加清晰,失真更小。

\paragraph{稳定性增强}
场效应管参数受温度影响较小,推挽电路的工作稳定性更好。

\subsection{电子管五管收音机}

\subsubsection{电路结构特点}

\subsubsection{电路图结构}

\subsubsection{元器件清单}

\subsubsection{性能提升}

\subsection{晶体管五管收音机}

\subsubsection{电路结构特点}

\subsubsection{电路图结构}

\subsubsection{元器件清单}

\subsubsection{性能提升}

\subsection{场效应管五管收音机}

\subsubsection{电路结构特点}

\subsubsection{电路图结构}

\subsubsection{元器件清单}

\subsubsection{性能提升}

\chapter{六管收音机}

六管收音机在五管收音机的基础上进一步完善电路,通常采用变频管、二级中放管、检波管、前置低放管和推挽低放管的配置。六管收音机是超外差式收音机的完善阶段,性能相比五管收音机有了进一步提升。

\subsection{电子管六管收音机}

电子管六管收音机是20世纪60-70年代的高档收音机类型,采用前置低放和推挽功率放大电路。六管收音机包括变频管、二级中放管、检波管、前置低放管和推挽低放管,实现了更高的输出功率、更好的音质和更稳定的性能。

\subsubsection{电路结构特点}

六管收音机的典型电路包括六个主要部分:
1. **变频管**:将接收到的高频信号转换为固定频率的的中频信号(通常为465kHz)
2. **第一中放管**:对中频信号进行第一级放大
3. **第二中放管**:对中频信号进行第二级放大
4. **检波管**:使用专用二极管进行检波,提供AGC控制电压
5. **前置低放管**:对检波后的音频信号进行电压放大
6. **推挽低放管**:使用两个电子管组成推挽功放电路,提高输出功率和音质

与五管收音机相比,六管收音机增加了一个前置低放管,使得音频信号在进入推挽功放之前先进行电压放大,从而提高了整体的放大倍数和音质。

\subsubsection{电路图结构}

```
[天线] ---> [输入回路(L1 + C1)] ---> [变频管栅极(V1-G)]
                                   |          |
                                   |          v
                                   |   [变频管本机振荡(V1-Osc)]
                                   |          |
                                   |          v
                                   |   [混频输出中频(465kHz)]
                                   |          |
                                   |          v
                                   |   [第一中频变压器(T1)]
                                   |          |
                                   |          v
                                   |   [第一中放管栅极(V2-G)]
                                   |          |
                                   |          v
                                   |   [第一中放管阳极(V2-A)]
                                   |          |
                                   |          v
                                   |   [第二中频变压器(T2)]
                                   |          |
                                   |          v
                                   |   [第二中放管栅极(V3-G)]
                                   |          |
                                   |          v
                                   |   [第二中放管阳极(V3-A)]
                                   |          |
                                   |          v
                                   |   [检波管(V4)]
                                   |          |
                                   |          v
                                   |   [耦合电容(C2)]
                                   |          |
                                   |          v
                                   |   [前置低放管栅极(V5-G)]
                                   |          |
                                   |          v
                                   |   [前置低放管阳极(V5-A)]
                                   |          |
                                   |          v
                                   |   [耦合电容(C3)]
                                   |          |
                                   |          v
                                   |   [推挽低放管栅极(V6-G1、V6-G2)]
                                   |          |
                                   |          v
                                   |   [输出变压器(T3)]
                                   |          |
                                   |          v
                                   |---[扬声器(SPK)]
```

\paragraph{电路说明}

- **V1**:变频管,通常使用变频三极管(如6A2),负责变频
- **V2**:第一中放管,通常使用高增益三极管(如6K4),负责第一级中频放大
- **V3**:第二中放管,通常使用高增益三极管(如6K4),负责第二级中频放大
- **V4**:检波管,通常使用双二极管(如6H2),负责中频检波和AGC
- **V5**:前置低放管,通常使用高增益三极管(如6N2),负责音频电压放大
- **V6**:推挽低放管,通常使用功率三极管(如6P1×2),负责音频功率放大
- **L1 + C1**:输入调谐回路,选择接收频率
- **T1、T2**:中频变压器(中周),调谐在465kHz,提高选择性
- **T3**:输出变压器,匹配推挽功放和扬声器阻抗
- **C2、C3**:耦合电容,将音频信号耦合到低放管

\subsubsection{元器件清单}

\begin{table}[H]
    \centering
    \caption{电子管六管收音机元器件清单}
    \label{tab:vacuum_tube_six_radio_parts}
    \begin{tabular}{ccc}
        \toprule
        元件名称 & 型号/规格 & 用途 \\ 
        \midrule
        变频电子管 & 6A2(或6SA7、1A2) & 高频放大与变频 \\
        第一中放电子管 & 6K4(或6SK7、3AG1) & 第一级中频放大 \\
        第二中放电子管 & 6K4(或6SK7、3AG1) & 第二级中频放大 \\
        检波电子管 & 6H2(或6H1、6N2) & 中频检波与AGC \\
        前置低放电子管 & 6N2(或6J1、6AU6) & 音频电压放大 \\
        推挽低放电子管 & 6P1×2(或6P14×2) & 推挽音频功率放大 \\
        输入线圈 & L1:200-300匝,0.2mm漆包线 & 接收信号调谐 \\
        可变电容 & C1:270pF空气可变电容 & 调谐选台 \\
        本机振荡线圈 & L2:100-150匝,0.2mm漆包线 & 产生本机振荡信号 \\
        第一中频变压器 & T1:465kHz中周 & 第一级中频选频与耦合 \\
        第二中频变压器 & T2:465kHz中周 & 第二级中频选频与耦合 \\
        耦合电容 & C2:0.01μF瓷片电容 & 检波输出耦合 \\
        耦合电容 & C3:0.1μF瓷片电容 & 前置低放耦合 \\
        滤波电容 & C4:10μF电解电容 & 电源滤波 \\
        栅极电阻 & R1:200kΩ碳膜电阻 & 变频管栅极电阻 \\
        栅极电阻 & R2:1MΩ碳膜电阻 & 第一中放管栅极电阻 \\
        栅极电阻 & R3:1MΩ碳膜电阻 & 第二中放管栅极电阻 \\
        栅极电阻 & R4:500kΩ碳膜电阻 & 前置低放管栅极电阻 \\
        阴极电阻 & R5:1kΩ碳膜电阻 & 变频管阴极电阻 \\
        阴极电阻 & R6:500Ω碳膜电阻 & 第一中放管阴极电阻 \\
        阴极电阻 & R7:500Ω碳膜电阻 & 第二中放管阴极电阻 \\
        阴极电阻 & R8:1kΩ碳膜电阻 & 前置低放管阴极电阻 \\
        阴极电阻 & R9:200Ω碳膜电阻 & 推挽低放管阴极电阻 \\
        阳极电阻 & R10:100kΩ碳膜电阻 & 前置低放管阳极负载 \\
        阳极电阻 & R11:10kΩ碳膜电阻 & 中放管阳极负载 \\
        扬声器 & 8Ω/5W动圈扬声器 & 音频输出 \\
        电源变压器 & 220V/6.3V×6 + 12V & 提供电子管灯丝和阳极电压 \\
        \bottomrule
    \end{tabular}
\end{table}

\subsubsection{性能提升}

与五管收音机相比,六管收音机的主要性能提升包括:

\paragraph{输出功率进一步增加}
由于增加了前置低放级,推挽功放电路的输入信号幅度增大,输出功率从五管的5W左右提升到8W以上,声音更加洪亮有力。

\paragraph{音质显著改善}
前置低放管提供了额外的电压放大,使得音频信号的动态范围更大,失真更小。高音和低音的表现更加出色。

\paragraph{灵敏度提高}
前置低放管的存在使得整个音频放大链路的增益更高,能够更好地放大微弱的音频信号,提高了收音机的整体灵敏度。

\paragraph{稳定性增强}
由于音频信号在进入推挽功放之前已经经过前置放大,推挽功放的工作更加稳定,受外界干扰的影响更小。

\paragraph{音量控制更平滑}
前置低放管的存在使得音量控制可以在前置级进行,音量调节更加平滑,不会出现五管收音机在低音量时音质变差的问题。

\subsection{晶体管六管收音机}

晶体管六管收音机是20世纪70-80年代的高档收音机类型,采用前置低放和推挽功率放大电路。相比电子管六管收音机,晶体管六管收音机具有体积小、功耗低、寿命长等优点。

\subsubsection{电路结构特点}

六管收音机的典型电路包括六个主要部分:
1. **变频管**:将接收到的高频信号转换为固定频率的中频信号
2. **第一中放管**:对中频信号进行第一级放大
3. **第二中放管**:对中频信号进行第二级放大
4. **检波管**:使用专用二极管进行检波
5. **前置低放管**:对检波后的音频信号进行电压放大
6. **推挽低放管**:使用两个晶体管组成推挽功放电路,提高输出功率

与五管收音机相比,六管收音机增加了一个前置低放管,使得音频信号在进入推挽功放之前先进行电压放大,从而提高了整体的放大倍数和音质。

\subsubsection{电路图结构}

```
[天线] ---> [输入回路(L1 + C1)] ---> [变频管基极(Q1-B)]
                                   |          |
                                   |          v
                                   |   [变频管本机振荡(Q1-Osc)]
                                   |          |
                                   |          v
                                   |   [混频输出中频(465kHz)]
                                   |          |
                                   |          v
                                   |   [第一中频变压器(T1)]
                                   |          |
                                   |          v
                                   |   [第一中放管基极(Q2-B)]
                                   |          |
                                   |          v
                                   |   [第一中放管集电极(Q2-C)]
                                   |          |
                                   |          v
                                   |   [第二中频变压器(T2)]
                                   |          |
                                   |          v
                                   |   [第二中放管基极(Q3-B)]
                                   |          |
                                   |          v
                                   |   [第二中放管集电极(Q3-C)]
                                   |          |
                                   |          v
                                   |   [检波二极管(D1)]
                                   |          |
                                   |          v
                                   |   [耦合电容(C2)]
                                   |          |
                                   |          v
                                   |   [前置低放管基极(Q4-B)]
                                   |          |
                                   |          v
                                   |   [前置低放管集电极(Q4-C)]
                                   |          |
                                   |          v
                                   |   [耦合电容(C3)]
                                   |          |
                                   |          v
                                   |   [推挽低放管基极(Q5-B、Q6-B)]
                                   |          |
                                   |          v
                                   |   [输出变压器(T3)]
                                   |          |
                                   |          v
                                   |---[扬声器(SPK)]
```

\paragraph{电路说明}

- **Q1**:变频管,通常使用高频小功率晶体管(如3DG6、9018),负责变频
- **Q2**:第一中放管,通常使用高增益晶体管(如3DG6、9013),负责第一级中频放大
- **Q3**:第二中放管,通常使用高增益晶体管(如3DG6、9013),负责第二级中频放大
- **D1**:检波二极管,通常使用锗二极管(如2AP9),负责中频检波
- **Q4**:前置低放管,通常使用高增益晶体管(如3DG12、9014),负责音频电压放大
- **Q5、Q6**:推挽低放管,通常使用功率晶体管(如3DG12×2、8050×2),负责音频功率放大
- **L1 + C1**:输入调谐回路,选择接收频率
- **T1、T2**:中频变压器(中周),调谐在465kHz,提高选择性
- **T3**:输出变压器,匹配推挽功放和扬声器阻抗
- **C2、C3**:耦合电容,将音频信号耦合到低放管

\subsubsection{元器件清单}

\begin{table}[H]
    \centering
    \caption{晶体管六管收音机元器件清单}
    \label{tab:transistor_six_radio_parts}
    \begin{tabular}{ccc}
        \toprule
        元件名称 & 型号/规格 & 用途 \\ 
        \midrule
        变频晶体管 & 3DG6(或9018、3AG1) & 高频放大与变频 \\
        第一中放晶体管 & 3DG6(或9013、3DG201) & 第一级中频放大 \\
        第二中放晶体管 & 3DG6(或9013、3DG201) & 第二级中频放大 \\
        检波二极管 & D1:2AP9(或1N4148) & 中频检波 \\
        前置低放晶体管 & 3DG12(或9014、3DG201) & 音频电压放大 \\
        推挽低放晶体管 & 3DG12×2(或8050×2) & 推挽音频功率放大 \\
        输入线圈 & L1:200-300匝,0.2mm漆包线 & 接收信号调谐 \\
        可变电容 & C1:270pF空气可变电容 & 调谐选台 \\
        本机振荡线圈 & L2:100-150匝,0.2mm漆包线 & 产生本机振荡信号 \\
        第一中频变压器 & T1:465kHz中周 & 第一级中频选频与耦合 \\
        第二中频变压器 & T2:465kHz中周 & 第二级中频选频与耦合 \\
        耦合电容 & C2:0.01μF瓷片电容 & 检波输出耦合 \\
        耦合电容 & C3:0.1μF瓷片电容 & 前置低放耦合 \\
        旁路电容 & C4:10μF电解电容 & 电源滤波 \\
        偏置电阻 & R1:100kΩ碳膜电阻 & 变频管基极偏置 \\
        偏置电阻 & R2:100kΩ碳膜电阻 & 第一中放管基极偏置 \\
        偏置电阻 & R3:100kΩ碳膜电阻 & 第二中放管基极偏置 \\
        偏置电阻 & R4:100kΩ碳膜电阻 & 前置低放管基极偏置 \\
        偏置电阻 & R5:100kΩ碳膜电阻 & 推挽低放管基极偏置 \\
        发射极电阻 & R6:1kΩ碳膜电阻 & 变频管发射极电阻 \\
        发射极电阻 & R7:500Ω碳膜电阻 & 第一中放管发射极电阻 \\
        发射极电阻 & R8:500Ω碳膜电阻 & 第二中放管发射极电阻 \\
        发射极电阻 & R9:1kΩ碳膜电阻 & 前置低放管发射极电阻 \\
        发射极电阻 & R10:100Ω碳膜电阻 & 推挽低放管发射极电阻 \\
        集电极电阻 & R11:10kΩ碳膜电阻 & 前置低放管集电极负载 \\
        集电极电阻 & R12:2kΩ碳膜电阻 & 中放管集电极负载 \\
        扬声器 & 8Ω/5W动圈扬声器 & 音频输出 \\
        电池 & 9V层叠电池 & 电源 \\
        \bottomrule
    \end{tabular}
\end{table}

\subsubsection{性能提升}

与五管收音机相比,晶体管六管收音机的主要性能提升包括:

\paragraph{输出功率进一步增加}
由于增加了前置低放级,推挽功放电路的输入信号幅度增大,输出功率从五管的3W左右提升到5W以上,声音更加洪亮有力。

\paragraph{音质显著改善}
前置低放管提供了额外的电压放大,使得音频信号的动态范围更大,失真更小。高音和低音的表现更加出色。

\paragraph{灵敏度提高}
前置低放管的存在使得整个音频放大链路的增益更高,能够更好地放大微弱的音频信号,提高了收音机的整体灵敏度。

\paragraph{稳定性增强}
由于音频信号在进入推挽功放之前已经经过前置放大,推挽功放的工作更加稳定,受外界干扰的影响更小。

\paragraph{音量控制更平滑}
前置低放管的存在使得音量控制可以在前置级进行,音量调节更加平滑,不会出现五管收音机在低音量时音质变差的问题。

\paragraph{功耗低}
晶体管六管收音机相比电子管六管收音机功耗更低,可以使用电池供电,便于携带。

\subsection{场效应管六管收音机}

场效应管六管收音机利用六个场效应管分别负责变频、二级中放、检波、前置低放和推挽低放。相比晶体管六管收音机,场效应管六管收音机具有更高的输入阻抗和更低的噪声,适合高灵敏度接收。

\subsubsection{电路结构特点}

六管收音机的典型电路包括六个主要部分:
1. **变频管**:将接收到的高频信号转换为固定频率的中频信号
2. **第一中放管**:对中频信号进行第一级放大
3. **第二中放管**:对中频信号进行第二级放大
4. **检波管**:使用专用二极管进行检波
5. **前置低放管**:对检波后的音频信号进行电压放大
6. **推挽低放管**:使用两个场效应管组成推挽功放电路,提高输出功率

与五管收音机相比,六管收音机增加了一个前置低放管,使得音频信号在进入推挽功放之前先进行电压放大,从而提高了整体的放大倍数和音质。

\subsubsection{电路图结构}

```
[天线] ---> [输入回路(L1 + C1)] ---> [变频管栅极(Q1-G)]
                                   |          |
                                   |          v
                                   |   [变频管本机振荡(Q1-Osc)]
                                   |          |
                                   |          v
                                   |   [混频输出中频(465kHz)]
                                   |          |
                                   |          v
                                   |   [第一中频变压器(T1)]
                                   |          |
                                   |          v
                                   |   [第一中放管栅极(Q2-G)]
                                   |          |
                                   |          v
                                   |   [第一中放管漏极(Q2-D)]
                                   |          |
                                   |          v
                                   |   [第二中频变压器(T2)]
                                   |          |
                                   |          v
                                   |   [第二中放管栅极(Q3-G)]
                                   |          |
                                   |          v
                                   |   [第二中放管漏极(Q3-D)]
                                   |          |
                                   |          v
                                   |   [检波二极管(D1)]
                                   |          |
                                   |          v
                                   |   [耦合电容(C2)]
                                   |          |
                                   |          v
                                   |   [前置低放管栅极(Q4-G)]
                                   |          |
                                   |          v
                                   |   [前置低放管漏极(Q4-D)]
                                   |          |
                                   |          v
                                   |   [耦合电容(C3)]
                                   |          |
                                   |          v
                                   |   [推挽低放管栅极(Q5-G、Q6-G)]
                                   |          |
                                   |          v
                                   |   [输出变压器(T3)]
                                   |          |
                                   |          v
                                   |---[扬声器(SPK)]
```

\paragraph{电路说明}

- **Q1**:变频管,通常使用高频场效应管(如3DJ6、BF494),负责变频
- **Q2**:第一中放管,通常使用高增益场效应管(如3DJ6、2N3819),负责第一级中频放大
- **Q3**:第二中放管,通常使用高增益场效应管(如3DJ6、2N3819),负责第二级中频放大
- **D1**:检波二极管,通常使用锗二极管(如2AP9),负责中频检波
- **Q4**:前置低放管,通常使用高增益场效应管(如3DJ7、MPF102),负责音频电压放大
- **Q5、Q6**:推挽低放管,通常使用功率场效应管(如IRF540×2、IRF9540×2),负责音频功率放大
- **L1 + C1**:输入调谐回路,选择接收频率
- **T1、T2**:中频变压器(中周),调谐在465kHz,提高选择性
- **T3**:输出变压器,匹配推挽功放和扬声器阻抗
- **C2、C3**:耦合电容,将音频信号耦合到低放管

\subsubsection{元器件清单}

\begin{table}[H]
    \centering
    \caption{场效应管六管收音机元器件清单}
    \label{tab:fet_six_radio_parts}
    \begin{tabular}{ccc}
        \toprule
        元件名称 & 型号/规格 & 用途 \\ 
        \midrule
        变频场效应管 & 3DJ6(或BF494、2N3819) & 高频放大与变频 \\
        第一中放场效应管 & 3DJ6(或2N3819、MPF102) & 第一级中频放大 \\
        第二中放场效应管 & 3DJ6(或2N3819、MPF102) & 第二级中频放大 \\
        检波二极管 & D1:2AP9(或1N4148) & 中频检波 \\
        前置低放场效应管 & 3DJ7(或MPF102、2N3819) & 音频电压放大 \\
        推挽低放场效应管 & IRF540×2(或IRF9540×2) & 推挽音频功率放大 \\
        输入线圈 & L1:200-300匝,0.2mm漆包线 & 接收信号调谐 \\
        可变电容 & C1:270pF空气可变电容 & 调谐选台 \\
        本机振荡线圈 & L2:100-150匝,0.2mm漆包线 & 产生本机振荡信号 \\
        第一中频变压器 & T1:465kHz中周 & 第一级中频选频与耦合 \\
        第二中频变压器 & T2:465kHz中周 & 第二级中频选频与耦合 \\
        耦合电容 & C2:0.01μF瓷片电容 & 检波输出耦合 \\
        耦合电容 & C3:0.1μF瓷片电容 & 前置低放耦合 \\
        旁路电容 & C4:10μF电解电容 & 电源滤波 \\
        栅极电阻 & R1:1MΩ碳膜电阻 & 变频管栅极电阻 \\
        栅极电阻 & R2:1MΩ碳膜电阻 & 第一中放管栅极电阻 \\
        栅极电阻 & R3:1MΩ碳膜电阻 & 第二中放管栅极电阻 \\
        栅极电阻 & R4:1MΩ碳膜电阻 & 前置低放管栅极电阻 \\
        栅极电阻 & R5:1MΩ碳膜电阻 & 推挽低放管栅极电阻 \\
        源极电阻 & R6:1kΩ碳膜电阻 & 变频管源极电阻 \\
        源极电阻 & R7:500Ω碳膜电阻 & 第一中放管源极电阻 \\
        源极电阻 & R8:500Ω碳膜电阻 & 第二中放管源极电阻 \\
        源极电阻 & R9:1kΩ碳膜电阻 & 前置低放管源极电阻 \\
        源极电阻 & R10:100Ω碳膜电阻 & 推挽低放管源极电阻 \\
        漏极电阻 & R11:10kΩ碳膜电阻 & 前置低放管漏极负载 \\
        漏极电阻 & R12:2kΩ碳膜电阻 & 中放管漏极负载 \\
        扬声器 & 8Ω/5W动圈扬声器 & 音频输出 \\
        电池 & 9V层叠电池 & 电源 \\
        \bottomrule
    \end{tabular}
\end{table}

\subsubsection{性能提升}

与五管收音机相比,场效应管六管收音机的主要性能提升包括:

\paragraph{输出功率进一步增加}
由于增加了前置低放级,推挽功放电路的输入信号幅度增大,输出功率从五管的3W左右提升到5W以上,声音更加洪亮有力。

\paragraph{音质显著改善}
前置低放管提供了额外的电压放大,使得音频信号的动态范围更大,失真更小。高音和低音的表现更加出色。

\paragraph{灵敏度提高}
前置低放管的存在使得整个音频放大链路的增益更高,能够更好地放大微弱的音频信号,提高了收音机的整体灵敏度。

\paragraph{稳定性增强}
由于音频信号在进入推挽功放之前已经经过前置放大,推挽功放的工作更加稳定,受外界干扰的影响更小。

\paragraph{音量控制更平滑}
前置低放管的存在使得音量控制可以在前置级进行,音量调节更加平滑,不会出现五管收音机在低音量时音质变差的问题。

\paragraph{噪声更低}
场效应管的高输入阻抗特性使得电路的噪声更低,适合接收微弱信号,提高了收音机的信噪比。

\paragraph{功耗低}
场效应管六管收音机相比电子管六管收音机功耗更低,可以使用电池供电,便于携带。

\chapter{七管超外差收音机}

七管超外差收音机是超外差式收音机的完善阶段,通常采用变频管、二级中放管、检波管、前置低放管、推动管和推挽低放管的配置。七管收音机相比六管收音机增加了一个推动管,使得推挽功放电路的驱动能力更强,输出功率更大,音质更加出色。

\subsection{电子管七管超外差收音机}

电子管七管超外差收音机是20世纪70-80年代的高档收音机类型,采用前置低放、推动和推挽功率放大电路。七管收音机包括变频管、二级中放管、检波管、前置低放管、推动管和推挽低放管,实现了最高的输出功率、最佳的音质和最稳定的性能。

\subsubsection{电路结构特点}

七管收音机的典型电路包括七个主要部分:
1. **变频管**:将接收到的高频信号转换为固定频率的中频信号(通常为465kHz)
2. **第一中放管**:对中频信号进行第一级放大
3. **第二中放管**:对中频信号进行第二级放大
4. **检波管**:使用专用二极管进行检波,提供AGC控制电压
5. **前置低放管**:对检波后的音频信号进行第一级电压放大
6. **推动管**:对前置低放输出的音频信号进行第二级电压放大,为推挽功放提供足够的驱动信号
7. **推挽低放管**:使用两个电子管组成推挽功放电路,提高输出功率和音质

与六管收音机相比,七管收音机增加了一个推动管,使得音频信号在进入推挽功放之前经过两级电压放大,从而提供了更强的驱动能力和更大的输出功率。

\subsubsection{电路图结构}

```
[天线] ---> [输入回路(L1 + C1)] ---> [变频管栅极(V1-G)]
                                   |          |
                                   |          v
                                   |   [变频管本机振荡(V1-Osc)]
                                   |          |
                                   |          v
                                   |   [混频输出中频(465kHz)]
                                   |          |
                                   |          v
                                   |   [第一中频变压器(T1)]
                                   |          |
                                   |          v
                                   |   [第一中放管栅极(V2-G)]
                                   |          |
                                   |          v
                                   |   [第一中放管阳极(V2-A)]
                                   |          |
                                   |          v
                                   |   [第二中频变压器(T2)]
                                   |          |
                                   |          v
                                   |   [第二中放管栅极(V3-G)]
                                   |          |
                                   |          v
                                   |   [第二中放管阳极(V3-A)]
                                   |          |
                                   |          v
                                   |   [检波管(V4)]
                                   |          |
                                   |          v
                                   |   [耦合电容(C2)]
                                   |          |
                                   |          v
                                   |   [前置低放管栅极(V5-G)]
                                   |          |
                                   |          v
                                   |   [前置低放管阳极(V5-A)]
                                   |          |
                                   |          v
                                   |   [耦合电容(C3)]
                                   |          |
                                   |          v
                                   |   [推动管栅极(V6-G)]
                                   |          |
                                   |          v
                                   |   [推动管阳极(V6-A)]
                                   |          |
                                   |          v
                                   |   [耦合变压器(T4)]
                                   |          |
                                   |          v
                                   |   [推挽低放管栅极(V7-G1、V7-G2)]
                                   |          |
                                   |          v
                                   |   [输出变压器(T3)]
                                   |          |
                                   |          v
                                   |---[扬声器(SPK)]
```

\paragraph{电路说明}

- **V1**:变频管,通常使用变频三极管(如6A2),负责变频
- **V2**:第一中放管,通常使用高增益三极管(如6K4),负责第一级中频放大
- **V3**:第二中放管,通常使用高增益三极管(如6K4),负责第二级中频放大
- **V4**:检波管,通常使用双二极管(如6H2),负责中频检波和AGC
- **V5**:前置低放管,通常使用高增益三极管(如6N2),负责第一级音频电压放大
- **V6**:推动管,通常使用高增益三极管(如6N1),负责第二级音频电压放大
- **V7**:推挽低放管,通常使用功率三极管(如6P1×2),负责音频功率放大
- **L1 + C1**:输入调谐回路,选择接收频率
- **T1、T2**:中频变压器(中周),调谐在465kHz,提高选择性
- **T3**:输出变压器,匹配推挽功放和扬声器阻抗
- **T4**:耦合变压器,将推动管输出耦合到推挽功放
- **C2、C3**:耦合电容,将音频信号耦合到低放管

\subsubsection{元器件清单}

\begin{table}[H]
    \centering
    \caption{电子管七管超外差收音机元器件清单}
    \label{tab:vacuum_tube_seven_radio_parts}
    \begin{tabular}{ccc}
        \toprule
        元件名称 & 型号/规格 & 用途 \\ 
        \midrule
        变频电子管 & 6A2(或6SA7、1A2) & 高频放大与变频 \\
        第一中放电子管 & 6K4(或6SK7、3AG1) & 第一级中频放大 \\
        第二中放电子管 & 6K4(或6SK7、3AG1) & 第二级中频放大 \\
        检波电子管 & 6H2(或6H1、6N2) & 中频检波与AGC \\
        前置低放电子管 & 6N2(或6J1、6AU6) & 第一级音频电压放大 \\
        推动电子管 & 6N1(或6J1、6AU6) & 第二级音频电压放大 \\
        推挽低放电子管 & 6P1×2(或6P14×2) & 推挽音频功率放大 \\
        输入线圈 & L1:200-300匝,0.2mm漆包线 & 接收信号调谐 \\
        可变电容 & C1:270pF空气可变电容 & 调谐选台 \\
        本机振荡线圈 & L2:100-150匝,0.2mm漆包线 & 产生本机振荡信号 \\
        第一中频变压器 & T1:465kHz中周 & 第一级中频选频与耦合 \\
        第二中频变压器 & T2:465kHz中周 & 第二级中频选频与耦合 \\
        耦合电容 & C2:0.01μF瓷片电容 & 检波输出耦合 \\
        耦合电容 & C3:0.1μF瓷片电容 & 前置低放耦合 \\
        滤波电容 & C4:10μF电解电容 & 电源滤波 \\
        栅极电阻 & R1:200kΩ碳膜电阻 & 变频管栅极电阻 \\
        栅极电阻 & R2:1MΩ碳膜电阻 & 第一中放管栅极电阻 \\
        栅极电阻 & R3:1MΩ碳膜电阻 & 第二中放管栅极电阻 \\
        栅极电阻 & R4:500kΩ碳膜电阻 & 前置低放管栅极电阻 \\
        栅极电阻 & R5:500kΩ碳膜电阻 & 推动管栅极电阻 \\
        阴极电阻 & R6:1kΩ碳膜电阻 & 变频管阴极电阻 \\
        阴极电阻 & R7:500Ω碳膜电阻 & 第一中放管阴极电阻 \\
        阴极电阻 & R8:500Ω碳膜电阻 & 第二中放管阴极电阻 \\
        阴极电阻 & R9:1kΩ碳膜电阻 & 前置低放管阴极电阻 \\
        阴极电阻 & R10:1kΩ碳膜电阻 & 推动管阴极电阻 \\
        阴极电阻 & R11:200Ω碳膜电阻 & 推挽低放管阴极电阻 \\
        阳极电阻 & R12:100kΩ碳膜电阻 & 前置低放管阳极负载 \\
        阳极电阻 & R13:50kΩ碳膜电阻 & 推动管阳极负载 \\
        阳极电阻 & R14:10kΩ碳膜电阻 & 中放管阳极负载 \\
        耦合变压器 & T4:5kΩ/5kΩ & 推动级耦合 \\
        扬声器 & 8Ω/10W动圈扬声器 & 音频输出 \\
        电源变压器 & 220V/6.3V×7 + 12V & 提供电子管灯丝和阳极电压 \\
        \bottomrule
    \end{tabular}
\end{table}

\subsubsection{性能提升}

与六管收音机相比,七管收音机的主要性能提升包括:

\paragraph{输出功率大幅增加}
由于增加了推动级,推挽功放电路的输入信号幅度进一步增大,输出功率从六管的8W左右提升到10W以上,声音更加洪亮有力,能够驱动更大的扬声器。

\paragraph{音质显著改善}
推动管提供了额外的电压放大,使得音频信号的动态范围更大,失真更小。高音和低音的表现更加出色,音质更加纯净。

\paragraph{灵敏度进一步提高}
推动管的存在使得整个音频放大链路的增益更高,能够更好地放大微弱的音频信号,提高了收音机的整体灵敏度。

\paragraph{稳定性显著增强}
由于音频信号在进入推挽功放之前已经经过两级放大,推挽功放的工作更加稳定,受外界干扰的影响更小。

\paragraph{音量控制更加平滑}
推动管的存在使得音量控制可以在推动级进行,音量调节更加平滑,不会出现六管收音机在低音量时音质变差的问题。

\paragraph{驱动能力更强}
推动管能够为推挽功放提供更强的驱动信号,使得推挽功放能够更好地工作,输出功率更大,音质更好。

\subsection{晶体管七管超外差收音机}

晶体管七管超外差收音机是20世纪70-80年代的高档收音机类型,采用前置低放、推动和推挽功率放大电路。相比电子管七管收音机,晶体管七管收音机具有体积小、功耗低、寿命长等优点。

\subsubsection{电路结构特点}

七管收音机的典型电路包括七个主要部分:
1. **变频管**:将接收到的高频信号转换为固定频率的中频信号
2. **第一中放管**:对中频信号进行第一级放大
3. **第二中放管**:对中频信号进行第二级放大
4. **检波管**:使用专用二极管进行检波
5. **前置低放管**:对检波后的音频信号进行第一级电压放大
6. **推动管**:对前置低放输出的音频信号进行第二级电压放大,为推挽功放提供足够的驱动信号
7. **推挽低放管**:使用两个晶体管组成推挽功放电路,提高输出功率

与六管收音机相比,七管收音机增加了一个推动管,使得音频信号在进入推挽功放之前经过两级电压放大,从而提供了更强的驱动能力和更大的输出功率。

\subsubsection{电路图结构}

```
[天线] ---> [输入回路(L1 + C1)] ---> [变频管基极(Q1-B)]
                                   |          |
                                   |          v
                                   |   [变频管本机振荡(Q1-Osc)]
                                   |          |
                                   |          v
                                   |   [混频输出中频(465kHz)]
                                   |          |
                                   |          v
                                   |   [第一中频变压器(T1)]
                                   |          |
                                   |          v
                                   |   [第一中放管基极(Q2-B)]
                                   |          |
                                   |          v
                                   |   [第一中放管集电极(Q2-C)]
                                   |          |
                                   |          v
                                   |   [第二中频变压器(T2)]
                                   |          |
                                   |          v
                                   |   [第二中放管基极(Q3-B)]
                                   |          |
                                   |          v
                                   |   [第二中放管集电极(Q3-C)]
                                   |          |
                                   |          v
                                   |   [检波二极管(D1)]
                                   |          |
                                   |          v
                                   |   [耦合电容(C2)]
                                   |          |
                                   |          v
                                   |   [前置低放管基极(Q4-B)]
                                   |          |
                                   |          v
                                   |   [前置低放管集电极(Q4-C)]
                                   |          |
                                   |          v
                                   |   [耦合电容(C3)]
                                   |          |
                                   |          v
                                   |   [推动管基极(Q5-B)]
                                   |          |
                                   |          v
                                   |   [推动管集电极(Q5-C)]
                                   |          |
                                   |          v
                                   |   [耦合变压器(T4)]
                                   |          |
                                   |          v
                                   |   [推挽低放管基极(Q6-B、Q7-B)]
                                   |          |
                                   |          v
                                   |   [输出变压器(T3)]
                                   |          |
                                   |          v
                                   |---[扬声器(SPK)]
```

\paragraph{电路说明}

- **Q1**:变频管,通常使用高频小功率晶体管(如3DG6、9018),负责变频
- **Q2**:第一中放管,通常使用高增益晶体管(如3DG6、9013),负责第一级中频放大
- **Q3**:第二中放管,通常使用高增益晶体管(如3DG6、9013),负责第二级中频放大
- **D1**:检波二极管,通常使用锗二极管(如2AP9),负责中频检波
- **Q4**:前置低放管,通常使用高增益晶体管(如3DG12、9014),负责第一级音频电压放大
- **Q5**:推动管,通常使用高增益晶体管(如3DG12、9014),负责第二级音频电压放大
- **Q6、Q7**:推挽低放管,通常使用功率晶体管(如3DG12×2、8050×2),负责音频功率放大
- **L1 + C1**:输入调谐回路,选择接收频率
- **T1、T2**:中频变压器(中周),调谐在465kHz,提高选择性
- **T3**:输出变压器,匹配推挽功放和扬声器阻抗
- **T4**:耦合变压器,将推动管输出耦合到推挽功放
- **C2、C3**:耦合电容,将音频信号耦合到低放管

\subsubsection{元器件清单}

\begin{table}[H]
    \centering
    \caption{晶体管七管超外差收音机元器件清单}
    \label{tab:transistor_seven_radio_parts}
    \begin{tabular}{ccc}
        \toprule
        元件名称 & 型号/规格 & 用途 \\ 
        \midrule
        变频晶体管 & 3DG6(或9018、3AG1) & 高频放大与变频 \\
        第一中放晶体管 & 3DG6(或9013、3DG201) & 第一级中频放大 \\
        第二中放晶体管 & 3DG6(或9013、3DG201) & 第二级中频放大 \\
        检波二极管 & D1:2AP9(或1N4148) & 中频检波 \\
        前置低放晶体管 & 3DG12(或9014、3DG201) & 第一级音频电压放大 \\
        推动晶体管 & 3DG12(或9014、3DG201) & 第二级音频电压放大 \\
        推挽低放晶体管 & 3DG12×2(或8050×2) & 推挽音频功率放大 \\
        输入线圈 & L1:200-300匝,0.2mm漆包线 & 接收信号调谐 \\
        可变电容 & C1:270pF空气可变电容 & 调谐选台 \\
        本机振荡线圈 & L2:100-150匝,0.2mm漆包线 & 产生本机振荡信号 \\
        第一中频变压器 & T1:465kHz中周 & 第一级中频选频与耦合 \\
        第二中频变压器 & T2:465kHz中周 & 第二级中频选频与耦合 \\
        耦合电容 & C2:0.01μF瓷片电容 & 检波输出耦合 \\
        耦合电容 & C3:0.1μF瓷片电容 & 前置低放耦合 \\
        旁路电容 & C4:10μF电解电容 & 电源滤波 \\
        偏置电阻 & R1:100kΩ碳膜电阻 & 变频管基极偏置 \\
        偏置电阻 & R2:100kΩ碳膜电阻 & 第一中放管基极偏置 \\
        偏置电阻 & R3:100kΩ碳膜电阻 & 第二中放管基极偏置 \\
        偏置电阻 & R4:100kΩ碳膜电阻 & 前置低放管基极偏置 \\
        偏置电阻 & R5:100kΩ碳膜电阻 & 推动管基极偏置 \\
        偏置电阻 & R6:100kΩ碳膜电阻 & 推挽低放管基极偏置 \\
        发射极电阻 & R7:1kΩ碳膜电阻 & 变频管发射极电阻 \\
        发射极电阻 & R8:500Ω碳膜电阻 & 第一中放管发射极电阻 \\
        发射极电阻 & R9:500Ω碳膜电阻 & 第二中放管发射极电阻 \\
        发射极电阻 & R10:1kΩ碳膜电阻 & 前置低放管发射极电阻 \\
        发射极电阻 & R11:1kΩ碳膜电阻 & 推动管发射极电阻 \\
        发射极电阻 & R12:100Ω碳膜电阻 & 推挽低放管发射极电阻 \\
        集电极电阻 & R13:10kΩ碳膜电阻 & 前置低放管集电极负载 \\
        集电极电阻 & R14:5kΩ碳膜电阻 & 推动管集电极负载 \\
        集电极电阻 & R15:2kΩ碳膜电阻 & 中放管集电极负载 \\
        耦合变压器 & T4:5kΩ/5kΩ & 推动级耦合 \\
        扬声器 & 8Ω/10W动圈扬声器 & 音频输出 \\
        电池 & 9V层叠电池 & 电源 \\
        \bottomrule
    \end{tabular}
\end{table}

\subsubsection{性能提升}

与六管收音机相比,晶体管七管收音机的主要性能提升包括:

\paragraph{输出功率大幅增加}
由于增加了推动级,推挽功放电路的输入信号幅度进一步增大,输出功率从六管的5W左右提升到8W以上,声音更加洪亮有力,能够驱动更大的扬声器。

\paragraph{音质显著改善}
推动管提供了额外的电压放大,使得音频信号的动态范围更大,失真更小。高音和低音的表现更加出色,音质更加纯净。

\paragraph{灵敏度进一步提高}
推动管的存在使得整个音频放大链路的增益更高,能够更好地放大微弱的音频信号,提高了收音机的整体灵敏度。

\paragraph{稳定性显著增强}
由于音频信号在进入推挽功放之前已经经过两级放大,推挽功放的工作更加稳定,受外界干扰的影响更小。

\paragraph{音量控制更加平滑}
推动管的存在使得音量控制可以在推动级进行,音量调节更加平滑,不会出现六管收音机在低音量时音质变差的问题。

\paragraph{驱动能力更强}
推动管能够为推挽功放提供更强的驱动信号,使得推挽功放能够更好地工作,输出功率更大,音质更好。

\paragraph{功耗低}
晶体管七管收音机相比电子管七管收音机功耗更低,可以使用电池供电,便于携带。

\subsection{场效应管七管超外差收音机}

场效应管七管超外差收音机利用七个场效应管分别负责变频、二级中放、检波、前置低放、推动和推挽低放。相比晶体管七管收音机,场效应管七管收音机具有更高的输入阻抗和更低的噪声,适合高灵敏度接收。

\subsubsection{电路结构特点}

七管收音机的典型电路包括七个主要部分:
1. **变频管**:将接收到的高频信号转换为固定频率的中频信号
2. **第一中放管**:对中频信号进行第一级放大
3. **第二中放管**:对中频信号进行第二级放大
4. **检波管**:使用专用二极管进行检波
5. **前置低放管**:对检波后的音频信号进行第一级电压放大
6. **推动管**:对前置低放输出的音频信号进行第二级电压放大,为推挽功放提供足够的驱动信号
7. **推挽低放管**:使用两个场效应管组成推挽功放电路,提高输出功率

与六管收音机相比,七管收音机增加了一个推动管,使得音频信号在进入推挽功放之前经过两级电压放大,从而提供了更强的驱动能力和更大的输出功率。

\subsubsection{电路图结构}

```
[天线] ---> [输入回路(L1 + C1)] ---> [变频管栅极(Q1-G)]
                                   |          |
                                   |          v
                                   |   [变频管本机振荡(Q1-Osc)]
                                   |          |
                                   |          v
                                   |   [混频输出中频(465kHz)]
                                   |          |
                                   |          v
                                   |   [第一中频变压器(T1)]
                                   |          |
                                   |          v
                                   |   [第一中放管栅极(Q2-G)]
                                   |          |
                                   |          v
                                   |   [第一中放管漏极(Q2-D)]
                                   |          |
                                   |          v
                                   |   [第二中频变压器(T2)]
                                   |          |
                                   |          v
                                   |   [第二中放管栅极(Q3-G)]
                                   |          |
                                   |          v
                                   |   [第二中放管漏极(Q3-D)]
                                   |          |
                                   |          v
                                   |   [检波二极管(D1)]
                                   |          |
                                   |          v
                                   |   [耦合电容(C2)]
                                   |          |
                                   |          v
                                   |   [前置低放管栅极(Q4-G)]
                                   |          |
                                   |          v
                                   |   [前置低放管漏极(Q4-D)]
                                   |          |
                                   |          v
                                   |   [耦合电容(C3)]
                                   |          |
                                   |          v
                                   |   [推动管栅极(Q5-G)]
                                   |          |
                                   |          v
                                   |   [推动管漏极(Q5-D)]
                                   |          |
                                   |          v
                                   |   [耦合变压器(T4)]
                                   |          |
                                   |          v
                                   |   [推挽低放管栅极(Q6-G、Q7-G)]
                                   |          |
                                   |          v
                                   |   [输出变压器(T3)]
                                   |          |
                                   |          v
                                   |---[扬声器(SPK)]
```

\paragraph{电路说明}

- **Q1**:变频管,通常使用高频场效应管(如3DJ6、BF494),负责变频
- **Q2**:第一中放管,通常使用高增益场效应管(如3DJ6、2N3819),负责第一级中频放大
- **Q3**:第二中放管,通常使用高增益场效应管(如3DJ6、2N3819),负责第二级中频放大
- **D1**:检波二极管,通常使用锗二极管(如2AP9),负责中频检波
- **Q4**:前置低放管,通常使用高增益场效应管(如3DJ7、MPF102),负责第一级音频电压放大
- **Q5**:推动管,通常使用高增益场效应管(如3DJ7、MPF102),负责第二级音频电压放大
- **Q6、Q7**:推挽低放管,通常使用功率场效应管(如IRF540×2、IRF9540×2),负责音频功率放大
- **L1 + C1**:输入调谐回路,选择接收频率
- **T1、T2**:中频变压器(中周),调谐在465kHz,提高选择性
- **T3**:输出变压器,匹配推挽功放和扬声器阻抗
- **T4**:耦合变压器,将推动管输出耦合到推挽功放
- **C2、C3**:耦合电容,将音频信号耦合到低放管

\subsubsection{元器件清单}

\begin{table}[H]
    \centering
    \caption{场效应管七管超外差收音机元器件清单}
    \label{tab:fet_seven_radio_parts}
    \begin{tabular}{ccc}
        \toprule
        元件名称 & 型号/规格 & 用途 \\ 
        \midrule
        变频场效应管 & 3DJ6(或BF494、2N3819) & 高频放大与变频 \\
        第一中放场效应管 & 3DJ6(或2N3819、MPF102) & 第一级中频放大 \\
        第二中放场效应管 & 3DJ6(或2N3819、MPF102) & 第二级中频放大 \\
        检波二极管 & D1:2AP9(或1N4148) & 中频检波 \\
        前置低放场效应管 & 3DJ7(或MPF102、2N3819) & 第一级音频电压放大 \\
        推动场效应管 & 3DJ7(或MPF102、2N3819) & 第二级音频电压放大 \\
        推挽低放场效应管 & IRF540×2(或IRF9540×2) & 推挽音频功率放大 \\
        输入线圈 & L1:200-300匝,0.2mm漆包线 & 接收信号调谐 \\
        可变电容 & C1:270pF空气可变电容 & 调谐选台 \\
        本机振荡线圈 & L2:100-150匝,0.2mm漆包线 & 产生本机振荡信号 \\
        第一中频变压器 & T1:465kHz中周 & 第一级中频选频与耦合 \\
        第二中频变压器 & T2:465kHz中周 & 第二级中频选频与耦合 \\
        耦合电容 & C2:0.01μF瓷片电容 & 检波输出耦合 \\
        耦合电容 & C3:0.1μF瓷片电容 & 前置低放耦合 \\
        旁路电容 & C4:10μF电解电容 & 电源滤波 \\
        栅极电阻 & R1:1MΩ碳膜电阻 & 变频管栅极电阻 \\
        栅极电阻 & R2:1MΩ碳膜电阻 & 第一中放管栅极电阻 \\
        栅极电阻 & R3:1MΩ碳膜电阻 & 第二中放管栅极电阻 \\
        栅极电阻 & R4:1MΩ碳膜电阻 & 前置低放管栅极电阻 \\
        栅极电阻 & R5:1MΩ碳膜电阻 & 推动管栅极电阻 \\
        栅极电阻 & R6:1MΩ碳膜电阻 & 推挽低放管栅极电阻 \\
        源极电阻 & R7:1kΩ碳膜电阻 & 变频管源极电阻 \\
        源极电阻 & R8:500Ω碳膜电阻 & 第一中放管源极电阻 \\
        源极电阻 & R9:500Ω碳膜电阻 & 第二中放管源极电阻 \\
        源极电阻 & R10:1kΩ碳膜电阻 & 前置低放管源极电阻 \\
        源极电阻 & R11:1kΩ碳膜电阻 & 推动管源极电阻 \\
        源极电阻 & R12:100Ω碳膜电阻 & 推挽低放管源极电阻 \\
        漏极电阻 & R13:10kΩ碳膜电阻 & 前置低放管漏极负载 \\
        漏极电阻 & R14:5kΩ碳膜电阻 & 推动管漏极负载 \\
        漏极电阻 & R15:2kΩ碳膜电阻 & 中放管漏极负载 \\
        耦合变压器 & T4:5kΩ/5kΩ & 推动级耦合 \\
        扬声器 & 8Ω/10W动圈扬声器 & 音频输出 \\
        电池 & 9V层叠电池 & 电源 \\
        \bottomrule
    \end{tabular}
\end{table}

\subsubsection{性能提升}

与六管收音机相比,场效应管七管收音机的主要性能提升包括:

\paragraph{输出功率大幅增加}
由于增加了推动级,推挽功放电路的输入信号幅度进一步增大,输出功率从六管的5W左右提升到8W以上,声音更加洪亮有力,能够驱动更大的扬声器。

\paragraph{音质显著改善}
推动管提供了额外的电压放大,使得音频信号的动态范围更大,失真更小。高音和低音的表现更加出色,音质更加纯净。

\paragraph{灵敏度进一步提高}
推动管的存在使得整个音频放大链路的增益更高,能够更好地放大微弱的音频信号,提高了收音机的整体灵敏度。

\paragraph{稳定性显著增强}
由于音频信号在进入推挽功放之前已经经过两级放大,推挽功放的工作更加稳定,受外界干扰的影响更小。

\paragraph{音量控制更加平滑}
推动管的存在使得音量控制可以在推动级进行,音量调节更加平滑,不会出现六管收音机在低音量时音质变差的问题。

\paragraph{驱动能力更强}
推动管能够为推挽功放提供更强的驱动信号,使得推挽功放能够更好地工作,输出功率更大,音质更好。

\paragraph{噪声更低}
场效应管的高输入阻抗特性使得电路的噪声更低,适合接收微弱信号,提高了收音机的信噪比。

\paragraph{功耗低}
场效应管七管收音机相比电子管七管收音机功耗更低,可以使用电池供电,便于携带。

\section{FM收音机}

FM(Frequency Modulation,频率调制)收音机是收音机技术的重要发展,相比AM(Amplitude Modulation,幅度调制)收音机,FM收音机具有更高的音质和抗干扰能力。FM收音机的工作原理是通过改变载波的频率来传输音频信号,而不是像AM收音机那样改变载波的幅度。

\subsection{FM收音机的工作原理}

FM收音机的基本工作原理包括以下几个部分:

1. **高频接收**:通过天线接收FM广播信号(通常在88-108MHz频段)
2. **高频放大**:对接收的FM信号进行高频放大
3. **变频**:将FM信号转换为固定频率的中频信号(通常为10.7MHz)
4. **中频放大**:对10.7MHz中频信号进行放大
5. **鉴频**:将频率调制的中频信号转换为音频信号
6. **音频放大**:对鉴频后的音频信号进行放大
7. **扬声器输出**:将音频信号转换为声音

与AM收音机相比,FM收音机需要使用专用的鉴频器来解调FM信号,而不是简单的检波器。此外,FM收音机的中频频率更高(10.7MHz vs 465kHz),需要使用专门的中频变压器和滤波器。

\subsection{电子管FM收音机}

电子管FM收音机是20世纪50-60年代的高档收音机类型,采用专用的FM高频头和鉴频器。电子管FM收音机通常包括高频放大管、变频管、中频放大管、鉴频管和音频放大管。

\subsubsection{电路结构特点}

电子管FM收音机的典型电路包括以下主要部分:
1. **高频放大管**:对FM广播信号进行高频放大
2. **变频管**:将FM信号转换为10.7MHz的中频信号
3. **第一中放管**:对10.7MHz中频信号进行第一级放大
4. **第二中放管**:对10.7MHz中频信号进行第二级放大
5. **鉴频管**:使用专用鉴频器将FM信号转换为音频信号
6. **音频放大管**:对鉴频后的音频信号进行放大
7. **功率放大管**:对音频信号进行功率放大,驱动扬声器

与AM收音机相比,FM收音机需要使用专用的FM高频头和10.7MHz中频变压器,以及专门的鉴频器电路。

\subsubsection{电路图结构}

```
[天线] ---> [FM高频头(L1 + C1)] ---> [高频放大管栅极(V1-G)]
                                   |          |
                                   |          v
                                   |   [高频放大管阳极(V1-A)]
                                   |          |
                                   |          v
                                   |   [变频管栅极(V2-G)]
                                   |          |
                                   |          v
                                   |   [变频管本机振荡(V2-Osc)]
                                   |          |
                                   |          v
                                   |   [混频输出中频(10.7MHz)]
                                   |          |
                                   |          v
                                   |   [第一中频变压器(T1)]
                                   |          |
                                   |          v
                                   |   [第一中放管栅极(V3-G)]
                                   |          |
                                   |          v
                                   |   [第一中放管阳极(V3-A)]
                                   |          |
                                   |          v
                                   |   [第二中频变压器(T2)]
                                   |          |
                                   |          v
                                   |   [第二中放管栅极(V4-G)]
                                   |          |
                                   |          v
                                   |   [第二中放管阳极(V4-A)]
                                   |          |
                                   |          v
                                   |   [鉴频器(V5)]
                                   |          |
                                   |          v
                                   |   [耦合电容(C2)]
                                   |          |
                                   |          v
                                   |   [音频放大管栅极(V6-G)]
                                   |          |
                                   |          v
                                   |   [功率放大管栅极(V7-G)]
                                   |          |
                                   |          v
                                   |   [输出变压器(T3)]
                                   |          |
                                   |          v
                                   |---[扬声器(SPK)]
```

\paragraph{电路说明}

- **V1**:高频放大管,通常使用高频三极管(如6BA6、6SJ7),负责FM信号的高频放大
- **V2**:变频管,通常使用变频三极管(如6BE6、6BA7),负责将FM信号转换为10.7MHz中频
- **V3**:第一中放管,通常使用高增益三极管(如6BA6、6SJ7),负责第一级中频放大
- **V4**:第二中放管,通常使用高增益三极管(如6BA6、6SJ7),负责第二级中频放大
- **V5**:鉴频管,通常使用双三极管(如6AL5、6H2),负责将FM信号转换为音频信号
- **V6**:音频放大管,通常使用高增益三极管(如6N2、6J5),负责音频电压放大
- **V7**:功率放大管,通常使用功率三极管(如6P1、6V6),负责音频功率放大
- **T1、T2**:10.7MHz中频变压器,用于中频信号的选频和耦合
- **T3**:输出变压器,匹配功率放大管和扬声器阻抗

\subsubsection{元器件清单}

\begin{table}[H]
    \centering
    \caption{电子管FM收音机元器件清单}
    \label{tab:vacuum_tube_fm_radio_parts}
    \begin{tabular}{ccc}
        \toprule
        元件名称 & 型号/规格 & 用途 \\ 
        \midrule
        高频放大电子管 & 6BA6(或6SJ7、6SK7) & FM信号高频放大 \\
        变频电子管 & 6BE6(或6BA7、6AJ8) & FM信号变频 \\
        第一中放电子管 & 6BA6(或6SJ7、6SK7) & 第一级中频放大 \\
        第二中放电子管 & 6BA6(或6SJ7、6SK7) & 第二级中频放大 \\
        鉴频电子管 & 6AL5(或6H2、6AV6) & FM信号鉴频 \\
        音频放大电子管 & 6N2(或6J5、6CG7) & 音频电压放大 \\
        功率放大电子管 & 6P1(或6V6、6AQ5) & 音频功率放大 \\
        FM高频头 & 88-108MHz & FM信号接收和调谐 \\
        本机振荡线圈 & L2:10.7MHz & 产生本机振荡信号 \\
        第一中频变压器 & T1:10.7MHz中周 & 第一级中频选频与耦合 \\
        第二中频变压器 & T2:10.7MHz中周 & 第二级中频选频与耦合 \\
        耦合电容 & C2:0.01μF瓷片电容 & 音频信号耦合 \\
        滤波电容 & C3:10μF电解电容 & 电源滤波 \\
        栅极电阻 & R1:200kΩ碳膜电阻 & 高频放大管栅极电阻 \\
        栅极电阻 & R2:1MΩ碳膜电阻 & 中放管栅极电阻 \\
        阴极电阻 & R3:1kΩ碳膜电阻 & 高频放大管阴极电阻 \\
        阴极电阻 & R4:500Ω碳膜电阻 & 中放管阴极电阻 \\
        阴极电阻 & R5:200Ω碳膜电阻 & 功率放大管阴极电阻 \\
        阳极电阻 & R6:10kΩ碳膜电阻 & 中放管阳极负载 \\
        扬声器 & 8Ω/5W动圈扬声器 & 音频输出 \\
        电源变压器 & 220V/6.3V×7 + 12V & 提供电子管灯丝和阳极电压 \\
        \bottomrule
    \end{tabular}
\end{table}

\subsubsection{性能特点}

电子管FM收音机的主要性能特点包括:

\paragraph{高音质}
FM调制方式本身具有比AM更高的音质,电子管的温暖音色进一步提升了音质表现。高频响应可达15kHz以上,失真小于1%。

\paragraph{抗干扰能力强}
FM调制方式对幅度干扰不敏感,具有较强的抗干扰能力,特别适合城市等电磁环境复杂的地区。

\paragraph{立体声接收}
电子管FM收音机可以通过添加立体声解码电路,实现FM立体声广播的接收。

\paragraph{灵敏度高}
专用的FM高频头和多级中频放大使得电子管FM收音机具有较高的灵敏度,可以接收微弱的FM信号。

\paragraph{体积较大}
由于使用了多个电子管和专用的FM高频头,电子管FM收音机的体积相对较大,通常为台式设计。

\paragraph{功耗较高}
电子管需要灯丝加热和较高的阳极电压,因此电子管FM收音机的功耗相对较高,通常需要使用交流电源。

\subsection{晶体管FM收音机}

晶体管FM收音机是20世纪60-80年代的主流收音机类型,采用晶体管替代电子管,具有体积小、功耗低、寿命长等优点。晶体管FM收音机通常包括高频放大管、变频管、中频放大管、鉴频管和音频放大管。

\subsubsection{电路结构特点}

晶体管FM收音机的典型电路包括以下主要部分:
1. **高频放大管**:对FM广播信号进行高频放大
2. **变频管**:将FM信号转换为10.7MHz的中频信号
3. **第一中放管**:对10.7MHz中频信号进行第一级放大
4. **第二中放管**:对10.7MHz中频信号进行第二级放大
5. **鉴频管**:使用专用鉴频器将FM信号转换为音频信号
6. **音频放大管**:对鉴频后的音频信号进行放大
7. **功率放大管**:对音频信号进行功率放大,驱动扬声器

与电子管FM收音机相比,晶体管FM收音机具有体积小、功耗低、寿命长等优点,但在音色的温暖度方面可能略有不足。

\subsubsection{电路图结构}

```
[天线] ---> [FM高频头(L1 + C1)] ---> [高频放大管基极(Q1-B)]
                                   |          |
                                   |          v
                                   |   [高频放大管集电极(Q1-C)]
                                   |          |
                                   |          v
                                   |   [变频管基极(Q2-B)]
                                   |          |
                                   |          v
                                   |   [变频管本机振荡(Q2-Osc)]
                                   |          |
                                   |          v
                                   |   [混频输出中频(10.7MHz)]
                                   |          |
                                   |          v
                                   |   [第一中频变压器(T1)]
                                   |          |
                                   |          v
                                   |   [第一中放管基极(Q3-B)]
                                   |          |
                                   |          v
                                   |   [第一中放管集电极(Q3-C)]
                                   |          |
                                   |          v
                                   |   [第二中频变压器(T2)]
                                   |          |
                                   |          v
                                   |   [第二中放管基极(Q4-B)]
                                   |          |
                                   |          v
                                   |   [第二中放管集电极(Q4-C)]
                                   |          |
                                   |          v
                                   |   [鉴频器(D1)]
                                   |          |
                                   |          v
                                   |   [耦合电容(C2)]
                                   |          |
                                   |          v
                                   |   [音频放大管基极(Q5-B)]
                                   |          |
                                   |          v
                                   |   [功率放大管基极(Q6-B)]
                                   |          |
                                   |          v
                                   |   [输出变压器(T3)]
                                   |          |
                                   |          v
                                   |---[扬声器(SPK)]
```

\paragraph{电路说明}

- **Q1**:高频放大管,通常使用高频晶体管(如2SC3355、BF495),负责FM信号的高频放大
- **Q2**:变频管,通常使用高频晶体管(如2SC3355、BF494),负责将FM信号转换为10.7MHz中频
- **Q3**:第一中放管,通常使用高增益晶体管(如2SC945、BC547),负责第一级中频放大
- **Q4**:第二中放管,通常使用高增益晶体管(如2SC945、BC547),负责第二级中频放大
- **D1**:鉴频器,通常使用专用FM鉴频器(如TA7343P、LA1235),负责将FM信号转换为音频信号
- **Q5**:音频放大管,通常使用高增益晶体管(如2SC1815、BC547),负责音频电压放大
- **Q6**:功率放大管,通常使用功率晶体管(如2SD882、TIP31),负责音频功率放大
- **T1、T2**:10.7MHz中频变压器,用于中频信号的选频和耦合
- **T3**:输出变压器,匹配功率放大管和扬声器阻抗

\subsubsection{元器件清单}

\begin{table}[H]
    \centering
    \caption{晶体管FM收音机元器件清单}
    \label{tab:transistor_fm_radio_parts}
    \begin{tabular}{ccc}
        \toprule
        元件名称 & 型号/规格 & 用途 \\ 
        \midrule
        高频放大晶体管 & 2SC3355(或BF495、2SC2712) & FM信号高频放大 \\
        变频晶体管 & 2SC3355(或BF494、2SC2712) & FM信号变频 \\
        第一中放晶体管 & 2SC945(或BC547、2N3904) & 第一级中频放大 \\
        第二中放晶体管 & 2SC945(或BC547、2N3904) & 第二级中频放大 \\
        鉴频器 & TA7343P(或LA1235、TA7021P) & FM信号鉴频 \\
        音频放大晶体管 & 2SC1815(或BC547、2N3904) & 音频电压放大 \\
        功率放大晶体管 & 2SD882(或TIP31、2N3055) & 音频功率放大 \\
        FM高频头 & 88-108MHz & FM信号接收和调谐 \\
        本机振荡线圈 & L2:10.7MHz & 产生本机振荡信号 \\
        第一中频变压器 & T1:10.7MHz中周 & 第一级中频选频与耦合 \\
        第二中频变压器 & T2:10.7MHz中周 & 第二级中频选频与耦合 \\
        耦合电容 & C2:0.01μF瓷片电容 & 音频信号耦合 \\
        滤波电容 & C3:10μF电解电容 & 电源滤波 \\
        偏置电阻 & R1:100kΩ碳膜电阻 & 高频放大管基极偏置 \\
        偏置电阻 & R2:100kΩ碳膜电阻 & 中放管基极偏置 \\
        发射极电阻 & R3:1kΩ碳膜电阻 & 高频放大管发射极电阻 \\
        发射极电阻 & R4:500Ω碳膜电阻 & 中放管发射极电阻 \\
        发射极电阻 & R5:100Ω碳膜电阻 & 功率放大管发射极电阻 \\
        集电极电阻 & R6:10kΩ碳膜电阻 & 中放管集电极负载 \\
        扬声器 & 8Ω/3W动圈扬声器 & 音频输出 \\
        电池 & 9V层叠电池 & 电源 \\
        \bottomrule
    \end{tabular}
\end{table}

\subsubsection{性能特点}

晶体管FM收音机的主要性能特点包括:

\paragraph{体积小}
晶体管的小体积使得晶体管FM收音机可以做得很小,甚至可以做成便携式口袋收音机。

\paragraph{功耗低}
晶体管的功耗远低于电子管,晶体管FM收音机可以使用电池供电,电池寿命可达几十小时。

\paragraph{高音质}
虽然晶体管的音色不如电子管温暖,但FM调制方式本身具有较高的音质,晶体管FM收音机的音质仍然非常出色。

\paragraph{抗干扰能力强}
与电子管FM收音机一样,晶体管FM收音机也具有较强的抗干扰能力。

\paragraph{立体声接收}
现代晶体管FM收音机通常内置立体声解码电路,可以直接接收FM立体声广播。

\paragraph{灵敏度高}
专用的FM高频头和多级中频放大使得晶体管FM收音机具有较高的灵敏度。

\paragraph{可靠性高}
晶体管的寿命远长于电子管,晶体管FM收音机的可靠性更高,故障率更低。

\subsection{场效应管FM收音机}

场效应管FM收音机是FM收音机的高级版本,利用场效应管的高输入阻抗和低噪声特性,实现了更高的灵敏度和更低的噪声。场效应管FM收音机通常包括高频放大管、变频管、中频放大管、鉴频管和音频放大管。

\subsubsection{电路结构特点}

场效应管FM收音机的典型电路包括以下主要部分:
1. **高频放大管**:使用场效应管对FM广播信号进行高频放大
2. **变频管**:使用场效应管将FM信号转换为10.7MHz的中频信号
3. **第一中放管**:对10.7MHz中频信号进行第一级放大
4. **第二中放管**:对10.7MHz中频信号进行第二级放大
5. **鉴频管**:使用专用鉴频器将FM信号转换为音频信号
6. **音频放大管**:对鉴频后的音频信号进行放大
7. **功率放大管**:对音频信号进行功率放大,驱动扬声器

与晶体管FM收音机相比,场效应管FM收音机在高频放大和变频级使用场效应管,利用其高输入阻抗和低噪声特性,提高了收音机的灵敏度和信噪比。

\subsubsection{电路图结构}

```
[天线] ---> [FM高频头(L1 + C1)] ---> [高频放大管栅极(Q1-G)]
                                   |          |
                                   |          v
                                   |   [高频放大管漏极(Q1-D)]
                                   |          |
                                   |          v
                                   |   [变频管栅极(Q2-G)]
                                   |          |
                                   |          v
                                   |   [变频管本机振荡(Q2-Osc)]
                                   |          |
                                   |          v
                                   |   [混频输出中频(10.7MHz)]
                                   |          |
                                   |          v
                                   |   [第一中频变压器(T1)]
                                   |          |
                                   |          v
                                   |   [第一中放管栅极(Q3-G)]
                                   |          |
                                   |          v
                                   |   [第一中放管漏极(Q3-D)]
                                   |          |
                                   |          v
                                   |   [第二中频变压器(T2)]
                                   |          |
                                   |          v
                                   |   [第二中放管栅极(Q4-G)]
                                   |          |
                                   |          v
                                   |   [第二中放管漏极(Q4-D)]
                                   |          |
                                   |          v
                                   |   [鉴频器(D1)]
                                   |          |
                                   |          v
                                   |   [耦合电容(C2)]
                                   |          |
                                   |          v
                                   |   [音频放大管栅极(Q5-G)]
                                   |          |
                                   |          v
                                   |   [功率放大管栅极(Q6-G)]
                                   |          |
                                   |          v
                                   |   [输出变压器(T3)]
                                   |          |
                                   |          v
                                   |---[扬声器(SPK)]
```

\paragraph{电路说明}

- **Q1**:高频放大管,通常使用高频场效应管(如BF998、2SK30A),负责FM信号的高频放大
- **Q2**:变频管,通常使用高频场效应管(如BF998、2SK30A),负责将FM信号转换为10.7MHz中频
- **Q3**:第一中放管,通常使用高增益场效应管(如2SK170、MPF102),负责第一级中频放大
- **Q4**:第二中放管,通常使用高增益场效应管(如2SK170、MPF102),负责第二级中频放大
- **D1**:鉴频器,通常使用专用FM鉴频器(如TA7343P、LA1235),负责将FM信号转换为音频信号
- **Q5**:音频放大管,通常使用高增益场效应管(如2SK170、MPF102),负责音频电压放大
- **Q6**:功率放大管,通常使用功率场效应管(如IRF540、IRF640),负责音频功率放大
- **T1、T2**:10.7MHz中频变压器,用于中频信号的选频和耦合
- **T3**:输出变压器,匹配功率放大管和扬声器阻抗

\subsubsection{元器件清单}

\begin{table}[H]
    \centering
    \caption{场效应管FM收音机元器件清单}
    \label{tab:fet_fm_radio_parts}
    \begin{tabular}{ccc}
        \toprule
        元件名称 & 型号/规格 & 用途 \\ 
        \midrule
        高频放大场效应管 & BF998(或2SK30A、BF245) & FM信号高频放大 \\
        变频场效应管 & BF998(或2SK30A、BF245) & FM信号变频 \\
        第一中放场效应管 & 2SK170(或MPF102、BF245) & 第一级中频放大 \\
        第二中放场效应管 & 2SK170(或MPF102、BF245) & 第二级中频放大 \\
        鉴频器 & TA7343P(或LA1235、TA7021P) & FM信号鉴频 \\
        音频放大场效应管 & 2SK170(或MPF102、BF245) & 音频电压放大 \\
        功率放大场效应管 & IRF540(或IRF640、IRF840) & 音频功率放大 \\
        FM高频头 & 88-108MHz & FM信号接收和调谐 \\
        本机振荡线圈 & L2:10.7MHz & 产生本机振荡信号 \\
        第一中频变压器 & T1:10.7MHz中周 & 第一级中频选频与耦合 \\
        第二中频变压器 & T2:10.7MHz中周 & 第二级中频选频与耦合 \\
        耦合电容 & C2:0.01μF瓷片电容 & 音频信号耦合 \\
        滤波电容 & C3:10μF电解电容 & 电源滤波 \\
        栅极电阻 & R1:1MΩ碳膜电阻 & 高频放大管栅极电阻 \\
        栅极电阻 & R2:1MΩ碳膜电阻 & 中放管栅极电阻 \\
        源极电阻 & R3:1kΩ碳膜电阻 & 高频放大管源极电阻 \\
        源极电阻 & R4:500Ω碳膜电阻 & 中放管源极电阻 \\
        源极电阻 & R5:100Ω碳膜电阻 & 功率放大管源极电阻 \\
        漏极电阻 & R6:10kΩ碳膜电阻 & 中放管漏极负载 \\
        扬声器 & 8Ω/3W动圈扬声器 & 音频输出 \\
        电池 & 9V层叠电池 & 电源 \\
        \bottomrule
    \end{tabular}
\end{table}

\subsubsection{性能特点}

场效应管FM收音机的主要性能特点包括:

\paragraph{低噪声}
场效应管的低噪声特性使得场效应管FM收音机具有更低的噪声水平,提高了信噪比。

\paragraph{高灵敏度}
场效应管的高输入阻抗特性使得场效应管FM收音机具有更高的灵敏度,可以接收更微弱的FM信号。

\paragraph{高音质}
场效应管的线性特性较好,场效应管FM收音机的音质更加清晰、自然。

\paragraph{抗干扰能力强}
与其他类型的FM收音机一样,场效应管FM收音机也具有较强的抗干扰能力。

\paragraph{立体声接收}
场效应管FM收音机通常内置立体声解码电路,可以直接接收FM立体声广播。

\paragraph{体积小}
与晶体管FM收音机一样,场效应管FM收音机的体积也可以做得很小。

\paragraph{功耗低}
场效应管的功耗与晶体管相当,场效应管FM收音机可以使用电池供电。

\paragraph{可靠性高}
场效应管的寿命长,场效应管FM收音机的可靠性高,故障率低。

\section{收音机技术发展的总结与展望}

收音机技术的发展经历了从矿石机到单管收音机,到多管超外差收音机,再到FM收音机的演进过程。每一次技术突破都带来了性能的显著提升:

1. **矿石收音机**:最简单的收音机,无需电源,结构简单,但灵敏度低、音量小。

2. **单管收音机**:使用一个有源器件(电子管、晶体管或场效应管),实现了信号放大,提高了灵敏度和音量。

3. **多管超外差收音机**:从两管到七管,通过增加放大级和改进电路设计,实现了更高的灵敏度、更好的选择性和更强大的功率输出。

4. **FM收音机**:利用频率调制技术,实现了更高的音质和更强的抗干扰能力。

随着电子技术的不断发展,现代收音机已经进入了数字化时代,数字调谐、数字信号处理、网络收音机等新技术不断涌现。但传统的模拟收音机仍然具有独特的魅力,特别是电子管收音机的温暖音色,仍然受到许多无线电爱好者的喜爱。

未来,收音机技术将继续融合现代电子技术的发展成果,在保持传统优势的同时,不断提升性能和功能,为人们带来更好的听觉体验。

\end{document}
