\documentclass{book}
\usepackage[utf8]{inputenc}
\usepackage[T1]{fontenc}
\usepackage{graphicx}
\usepackage{geometry}
\usepackage{titlesec}
\usepackage{hyperref}
\graphicspath{{Images/}}

\geometry{a4paper, margin=2cm}
\titleformat{\chapter}[display]
  {\bfseries\Large}
  {Chapter \thechapter}
  {20pt}
  {\Huge}

\title{Radio Builder's Book}
\author{Burkhard Kainka}
\date{}

\begin{document}

\maketitle

\tableofcontents

\chapter*{Preface}
\addcontentsline{toc}{chapter}{Preface}

Discover the captivating world of radio technology and unlock the secrets of radio set construction using this comprehensive guide. From the early days of the humble crystal set to the modern wonders of Software-Defined Radios (SDRs), this book takes you on a journey through time and technology. With detailed instructions and step-by-step illustrations, you'll learn how to build and assemble various receivers to understand their various strengths and weaknesses. Practical antenna design and amateur radio rigs are also covered in this inclusive handbook.

For many years in the early history of electronics, the construction of homebrew radio receivers was the most common entry point to electronics. Nowadays, there are many other routes in, especially through the use of computers, microcontrollers, and digital technology. The analogue roots of electronics are now often overlooked but radio technology is particularly well suited as an introduction to electronics because you will be rewarded with early success from even the most basic circuit. The connection to modern digital technology is also obvious when it comes to modern tuning methods and the use of highly integrated PLL, DDS and DSP radios.

This book aims to provide an overview and present a collection of simple projects to encourage budding engineers along the path of discovery. Now in its second edition, many new projects have been added which include important circuits and cover most recent developments. With this book by your side, you will go on to develop your own ideas and design and test your own receivers.

Wishing you every success and crystal clear reception!
Burkhard Kainka, DK7JD
www.elektronik-labor.de

\chapter{Introduction}

Building radios is an old hobby that has seen something of a renaissance recently. In addition to the classic and dead simple 'Foxhole' crystal radio set and more sophisticated vacuum tube receivers right up to the more recent software-defined radio projects, there are many aspects to the technology for newcomers to get their teeth into. Recent improvements in semiconductor technology and integrated circuits have allowed sophisticated features such as Direct Digital Synthesis (DDS) and integrated PLL technology to be incorporated into home brew receiver designs to produce radios with surprisingly good specifications.

This book provides an overview of radio technology and clearly explains the basics of radio receiver design. Using numerous circuits and building plans it guides you step by step along the way. If you want to cook up a simple crystal set or vacuum tube regenerative type receiver, you'll find all the recipes here. As your knowledge and confidence grows you will want to develop your own circuits. That's why the basics of resonant circuits, oscillator configurations and antenna design are all explained clearly here.

Tuned Radio Frequency and Audion type receivers are a step up from the basic crystal detector type receiver. They originally used a single vacuum tube to demodulate the received signal and amplify the resulting baseband output. Receivers like this and simple transistor radios for medium or shortwave reception can be built quickly and easily. They are a lot of fun to play with and are a good way to gain knowledge of RF technology. Modern direct mixing concepts using ring mixers and DDS or PLLs, or simple software-defined radios allow the construction of universal receivers for amateur radio and digital operation. Many things have become easier to build thanks to highly integrated modern chips, but it helps if you also have the essential background information. In this book we only use components that are easily obtainable. Elektor magazine has also created board layouts or finished assemblies and devices that can be sourced from its online store.

My own interest in RF technology comes from an early fascination with amateur radio; I spent many hours in my youth building and using my own transmitters and receivers. After a long break and following the introduction of new digital broadcasting standards such as DAB and DRM my interest has been rekindled. The challenge for me now was to design a receiver sufficiently stable to decode DRM signals reliably on the short and medium wave bands. I spent time tinkering with vacuum tubes and studied how they were used back in the early days. Although the anode HT was usually high voltage DC, here you experiment with lower anode voltages starting from just 6 V to simplify experiments and make tinkering less hazardous. Other topics explored here are IQ mixer design and building a software-defined radio.

Personally, my interest in ham radio has also undergone a revival. For a long time, I had accepted it would not be feasible to install a useful amateur radio antenna in the apartment where I live now. With the help of newer techniques and improved measurement technology, I have been able to build simple and inconspicuous antennas that allow reasonably interference-free reception and can also be used for low power transmitting.

Radio technology has always been a theme of my professional work. More recently I have been involved in the development of kits for the Kosmos-Verlag and the Franzis-Verlag, as well as articles and projects for \textit{Elektor Magazine}. Work for the AK Modul-Bus company brought new challenges and resulted in projects for school teaching programs and hobby electronics. Some radio receiver projects on topics such as vacuum tube technology, DRM reception, software-defined radio and DSP radio have been developed using AK Modul-Bus products and turned into \textit{Elektor Magazine} projects.

\chapter{Detector Radios}

Tuning in to radio broadcasts without a battery or any other power source is only possible with a crystal radio receiver. This simplest of all radio circuits has not lost any of its charm over the decades. In the early days of radio technology, the crystal radio or Foxhole receiver was a widely used concept. Today, just as 90 years ago, it serves as a great introduction to RF technology. You don't necessarily have to recreate authentic historical devices or use homemade crystal detectors. Using a germanium or Schottky diode detector simplifies the process of recovering the baseband signal. You also won't even need an extremely long antenna or highly sensitive headphones. Using an existing speaker amplifier, such as a set of PC active speakers makes building your first radio a piece of cake.

\begin{figure}[htbp]
\centering
\includegraphics[width=0.8\textwidth]{cover}
\caption{Cover Image}
\end{figure}

\section{The Diode Radio}

The simplest receiver you can build consists of a long length of wire for use as an antenna, a ground connection, a germanium (Ge) diode and a high-impedance headphone. The germanium diode may be difficult to source so a modern Schottky diode can be substituted. The radio needs no external power supply because the signal picked up by the antenna provides all the energy necessary, which is why it needs to be relatively long. Usually, a 10 meter length of wire will do the job. This design assumes a high-impedance headphone of 2 kΩ, but it will work just as well with a 600 Ω type. Standard low-impedance headphones of modern design are usually 32 Ω but they can also be used with a suitable transformer (see section 2.2) to provide impedance matching.

\begin{figure}[htbp]
\centering
\includegraphics[width=0.6\textwidth]{fig2-1}
\caption{Simple Diode Radio Circuit}
\end{figure}

You can use any Ge diodes from type AA112 to AA144 or any Schottky diodes from BAT41 to BAT86. This simple radio is not selective, which means it receives all strong stations at the same time. Unless a strong local station is overpowering all the others, you should be able to hear some stations with fluctuating volume, especially at dusk.

To achieve desired selectivity, a resonant circuit consisting of a coil and tuning capacitor can be added to the circuit. Using a tuning capacitor of up to 320 pF and a coil of 300 µH, will allow the entire medium-wave band can be covered. The coil consists of 90 turns of wire wound onto a 4 cm diameter cardboard roll to make the necessary air-cored coil.

\begin{figure}[htbp]
\centering
\includegraphics[width=0.6\textwidth]{fig2-2}
\caption{Resonant Circuit for Medium Wave}
\end{figure}

This radio is not very selective and doesn't achieve much in terms of output volume. It's important to carefully adjust the antenna and rectifier; this can be achieved by adding tap or connection points along the receiving coil. In section 2.2, a medium wave receiver is described which uses adjustable matching.

You may wonder why a diode is necessary in a receiver circuit. To answer that one you need to delve into a little bit of radio theory. A transmitter broadcasts high frequency electromagnetic waves into free space via a transmitting mast. The broadcast radiates in all directions and induces a small signal in an antenna at the receiving location. Transmitters that send on the medium wave band transfer their information, such as speech and music, in the form of amplitude modulation (AM) of the carrier frequency. The radio frequency carrier amplitude changes in time with the low-frequency (baseband) voice or music signals.

The received radio signal remains inaudible even in headphones because our ears are only sensitive to sound pressure waves up to about 20 kHz. The low-frequency signal carrying the voice and music information needs to be recovered from the radio carrier wave. This is where the diode comes in; using just one diode you can demodulate the RF signal. The average current of the rectified signal corresponds to the original modulated AF signal.

\begin{figure}[htbp]
\centering
\includegraphics[width=0.6\textwidth]{fig2-3}
\caption{Demodulation Process}
\end{figure}

The first detector radios used crystal detectors. Lead sulfide (galena) or a piece of pyrite crystal was used for this purpose. Both are sulfur compounds and occur in nature as ores (lead ore; iron ore).

The crystal can be used successfully to build a diode radio. Numerous strong stations can be heard without the need for any additional amplifier. Even today, it is possible to build a detector using these natural minerals. Pyrite forms regular, gold-colored cuboid crystals in rock. Lead sulfide is black with areas of metallic shiny facets on its surface. A sewing needle can be used as the cat's whisker detector. You will need to test various points on the crystal surface until contact achieves a good rectification characteristic.

\section{Headphone Adapter}

Vintage circuit diagrams for detector radios assume that headphones shown on the circuit will be high-impedance types with 2000 Ω driver coils. These were standard back then. Nowadays a typical set of headphones will use 32 Ω driver coils which will be too low to function properly in the original circuit. You can, however, use a small transformer to provide the necessary impedance matching. A transformer salvaged from a small mains adapter can be used here. If the mains adapter has switchable taps, (3/4.5/6/9/12 V) on the secondary winding you may be able to use these to optimize the impedance match. Remove the transformer and connect the secondary winding to the headphones and the primary winding to the circuit where the high impedance phones would normally be connected.

\begin{figure}[htbp]
\centering
\includegraphics[width=0.6\textwidth]{fig2-4}
\caption{Headphone Adapter Circuit}
\end{figure}

In a diode radio, correct antenna matching is the key to success because you cannot afford to waste any of the received RF energy. The receiver coil, therefore, has several tap points. Using a total of 80 turns of 'Litz' wire on a 10 mm diameter ferrite rod, makes sure you will be able to cover the entire medium wave band. Long antennas should be connected to a lower tap of the coil to not overly dampen the resonant circuit at the input. Try connecting the long antenna to each of the winding taps to find which one gives the best reception. Two coupling capacitors are also shown connected at the coil end. Experiment with the aerial connection, a higher value of capacitance results in stronger coupling.

\begin{figure}[htbp]
\centering
\includegraphics[width=0.6\textwidth]{fig2-5}
\caption{Antenna Matching Circuit}
\end{figure}

For such a simple radio, a good antenna is crucial. If your house is fitted with metal rainwater guttering, this can make a good antenna. The guttering should not have a connection to ground potential. A zinc gutter will often be cemented into a drainage pipe near the ground and thereby will be insulated. All you need now is a connection wire, and that should make a really good antenna. In case the reception is still too quiet for headphones, you can connect the output to a set of PC's active speakers.

Another good antenna is sometimes the heating system of an apartment. Although the pipes are usually grounded at some point, the total length of all the piping can effectively act as a loop antenna. In many cases, this can result in high received signal levels.

\section{A Detector for Shortwave}

Looking at old "plans" to build detector radios, they are usually designed to receive signals from local stations in the medium wave band. These stations are becoming rarer and may even be unavailable now as more countries shut down their medium wave transmitters. Some countries such as the UK, Italy, France, and Spain however still broadcast in the band. Transmitting on shortwave has the advantage of covering much greater distances. Many countries have their own international broadcasting services designed to inform and entertain overseas listeners. The tried-and-true AM radio is, therefore, just as active on shortwave as ever.

Tuning in to higher frequencies requires smaller coils which are much easier to make. While a good medium wave coil needs a ferrite rod and a coil wound from hard-to-find 'Litz' wire, on shortwave, you can use standard insulated copper wire. A special coil former with a ferrite core is not required; you can use any insulated wire.

\begin{figure}[htbp]
\centering
\includegraphics[width=0.6\textwidth]{fig2-6}
\caption{Shortwave Coil Construction}
\end{figure}

For the first attempt, a coil with a total of 25 turns with four taps should be wound. I used the plastic body of a banana plug which measures 8 mm diameter but you could use a ball-point pen body. Two holes spaced 1 cm apart help to fix the wire ends. Then, wind 5 turns, make a tap point, and apply the next turns. The finished coil connections can be soldered to a 6-way pinheader strip.

\begin{figure}[htbp]
\centering
\includegraphics[width=0.6\textwidth]{fig2-7}
\caption{Shortwave Detector Circuit}
\end{figure}

The entire radio can be built on an experimental plug board. Pins have been soldered to the variable connections so that it can be easily plugged into the prototyping plug board. The advantage of this construction method is that it allows for easy experimentation to try out other circuit mods.

The respective taps for the antenna connection and the diode on the coil can be experimentally adjusted. This tuning capacitor is dual gang with both halves connected in parallel. If only the upper range above 10 MHz is to be received, the lower-valued half of the tuning capacitor rated at 80 pF will be sufficient.

The radio requires a high-impedance headphone, such as a piezoceramic crystal earpiece or dynamic headphones with a 2 kΩ resistor placed in series with each capsule. Low-impedance 32 Ω types cannot be used directly and require an impedance transformer (see Section 2.2). Medium-impedance headphones with 600 Ω can also be used directly.

A germanium or a Schottky diode can be used as the detector. Both of these diodes have a low forward threshold voltage. A germanium diode also has reasonably low conductivity in the reverse direction, which is important when using a high-impedance crystal earpiece.

\section{Silicon Diode Detector}

Germanium diodes are rarely used nowadays. Silicon diodes are very popular for all sorts of applications and the 1N4148 is the most commonly used universal diode. The circuit shown uses a silicon diode with an additional bias voltage applied. In addition, a coupling capacitor is used here to connect the signal to an amplifier input.

\begin{figure}[htbp]
\centering
\includegraphics[width=0.6\textwidth]{fig2-8}
\caption{Silicon Diode Detector with Bias}
\end{figure}

A silicon diode requires a forward voltage bias of around 0.5 V before any significant current starts to flow. Since the received RF signal at the resonant circuit only rarely reaches such high levels, you may not hear any recovered signal. Germanium diodes, however, have a forward voltage threshold below 0.2 V so that smaller signals can be recovered. To overcome the high threshold of silicon diodes, you can set up a small DC current of about 10 µA to flow through the diode. This will forward bias the junction allowing received signals below 100 mV to be demodulated.

Although the circuit can be operated directly with headphones, it works better with a speaker amplifier. A set of active PC speakers, for example, will work well here. The built-in audio amplifier provides sufficient amplification and a high input resistance in the order of 100 kΩ. This results in less damping of the resonant circuit and gives better selectivity. In addition, compared to using headphones, you can connect the antenna to a lower tap on the coil to improve selectivity and boost volume level.

\section{Coils and Resonant Circuits}

In order to build each circuit described here we've provided the necessary inductance and specific measurements. However, sometimes you may need to modify the circuit or use a different coil body; in that case, you'll need to determine the number of turns yourself. It's also possible that you have some old coils salvaged from redundant equipment that you can modify and adapt. Regardless, it's useful to know how to calculate coils yourself.

\begin{figure}[htbp]
\centering
\includegraphics[width=0.6\textwidth]{fig2-9}
\caption{Air-core Coil Construction}
\end{figure}

There are basically two types of coils: those wound on a magnetizable core (like ferrite or iron powder) and those without a core, known as air-core coils. Let's focus on air-core coils first. For instance, a coil for a shortwave resonant circuit has 20 turns, a diameter of 16 mm, and a coil length of 35 mm. It has an inductance of approximately 3 µH and, when combined with a variable capacitor up to 300 pF, can reach a lower frequency limit of around 5.3 MHz. We'll show you how to calculate this and introduce a simple tool that can make the process easier.

\begin{figure}[htbp]
\centering
\includegraphics[width=0.6\textwidth]{fig2-10}
\caption{Coil Winding Tool}
\end{figure}

In general, the following formula applies to a long coil where \textit{l} > \textit{D} where \textit{n} is the number of turns, \textit{A} is the cross-sectional area in square meters, and \textit{l} is the length in meters:

\[ L = \mu_0 \times n^2 \times A / l \]

where the magnetic field constant \(\mu_0\) equals:

4π × 10^–7 Vs/Am = 1.2466 × 10^-6 Vs/Am

This formula actually only applies to an infinitely long coil but can be used as a useful approximation up to a length of \textit{l = D}. With a short coil of the same number of turns, the magnetic coupling between individual turns increases, resulting in a higher inductance. Conversely, stretching out the turns reduces inductance, which can sometimes be used to adjust coils.

The above formula can be simplified for a circular coil cross-section, where the diameter D and length l of the coil are given in mm, to the following approximation formula:

\[ L = 1\,nH \times n^2 \times D^2/\text{mm}^2 / (l/\text{mm}) \]

This formula uses the approximation of π × π = 10 which introduces an error of approximately 1.3\%. This is generally an acceptable simplification; you cannot expect high accuracy since the shape of the coil, especially the ratio of length and thickness, wire thickness, and even the location where the coil is mounted, all influence the final value of inductance achieved. In practice you can expect to achieve accuracy within 10\% for an air-core calculation.

RF coil formers with ferrite screw cores are often used. The coil inductance increases by up to four times or more when a ferrite core is used. By changing the insertion depth of the screw-in core the coil value can be adjusted. Ferrite cores are manufactured for use with certain frequency bands in which they have lowest energy losses.

Much larger inductances can be achieved by using closed cores with or without an air gap. The air gap reduces the inductance of the coil, but allows for greater magnetization, i.e., the core itself only reaches magnetic saturation at higher currents. Common types of cores include ring cores, transformer cores in E-I shape, and closed pot cores.

The inductance depends heavily on the number of turns, the material used, and the geometry of the core. A theoretical calculation, like for the air-core coil, is not so simple. The manufacturer will instead provide an AL value in nH/\textit{n²} for each type of core.

\[ L = A_L \times n^2 \]

For example, an Amidon T37-2 ring core has an inductance of 40 µH at 100 turns, which corresponds to an AL value of 4 nH/n². Reducing the winding to 30 turns, the inductance becomes:

\[ L = 30 \times 30 \times 4\,nH = 3600\,nH = 3.6\,µH. \]

The ring core coil is suitable for building an RF resonant circuit, just like an air-core coil. Besides the AL value, the intended frequency range of a core is also important. The Amidon type xxx-2 has the color code red which indicates it is suitable for frequencies up to 30 MHz.

\section{Resonant Frequency and Bandwidth}

If you connect a coil and a capacitor, a resonant circuit is created. Electrical energy can oscillate back and forth between the coil and capacitor, similar to the decaying swing of a pendulum, the period of the swing indicates the resonant frequency \textit{f}. The electrical circuit responds to a short pulse of current with a diminishing oscillatory voltage waveform.

The formula for calculating the resonance frequency is:

\[ f = 1 / (2\pi\sqrt{LC}) \]

Tuned circuits are often used in electrical circuits that process a range of different signal frequencies or mixed frequencies. Current and voltages that flow in such circuits will vary according to the signal frequency. The parallel resonant circuit has a complex impedance Z with a sharp maximum value at the resonant frequency f0. At this frequency, RC = RL and the currents through the coil and capacitor cancel out exactly due to their total phase difference of 180 degrees. An ideal oscillating circuit with no losses would have infinitely large impedance at the resonant frequency.

In practice however, damping of the oscillation occurs because of energy losses in the resistance of the coil wire, magnetic losses of the coil core, and electromagnetic radiation, resulting in a finite resonant resistance. To simplify you can add all the losses together and assign them as a parallel loss resistance R.

Each resonant circuit has a property called the Quality factor or just Q which is inversely proportional to the bandwidth of the circuit. Q can be easily determined when the parallel damping resistance R is related to the inductive resistance RL = 2πfL or to the capacitive resistance RC = 1 / (2πfC) at the resonant frequency.

\[ Q = R / R_L \text{ or } Q = R / R_C \]

If a resonant circuit is excited with a constant alternating current I of variable frequency, or through an alternating current source with high internal resistance, then the resonant circuit voltage is proportional to the magnitude of the complex impedance Z. At resonance, the voltage is highest. The smaller the damping of the vibration due to energy losses of any kind, or the larger the quality of the resonant circuit, the higher the resonant voltage rises.

On both sides of the resonant frequency, points on the resonance curve can be determined at which the voltage has dropped to a factor of 1 / √2 = 0.707 = –3 dB. The frequency separation of these points is referred to as the bandwidth b of the circuit. Between the resonant frequency f0, bandwidth b, and quality factor Q of the circuit, the relationship is

\[ b = f_0 / Q. \]

The damping of the circuit, and therefore its quality, is practically always caused by intrinsic series and parallel resistances. The series resistance is due to the wire winding, but for a certain frequency, it is greater than the DC resistance due to the skin effect. The parallel resistance is determined by the connection impedance in the circuit. However, an iron or ferrite core also has losses that can be represented by a parallel resistor. With the same inductance, a coil with a core requires fewer turns and therefore incurs lower copper losses. At the same time there are now losses in the core to consider. At very high frequencies of around 100 MHz, pure air coils made of thick, silver-plated wire perform better, while at medium frequencies of around 10 MHz, the best quality is achieved with a closed core such as a toroidal core. Air coils, on the other hand, are an alternative down to about 1 MHz. Coils and transformers used in the audio frequency range, however, are almost always built with a core.

You can expect to get a quality factor \textit{Q} of up to 100 by being careful with coil construction. A resonant circuit is however also damped by the external circuitry to which it is connected to or by an antenna. This damping effect can to some extent be mitigated by ensuring a loose coupling of the resonant circuit by using a small auxiliary winding, a tap point on the coil, or a suitable coupling capacitor. When a coil connects directly to the input of an amplifier, its input impedance should be very high to lessen the damping effect.

A small Visual Basic program can be found on the author's website called LCFR which has been written to simplify the calculation of coils and resonant circuits. The program calculates the inductance of air coils and coils with a known \textit{A_L} value. In addition, the resonance frequency, and the inductive resistance RL of the coil at this frequency can be determined if a value of capacitance is given in addition to the inductance.

Here are a few examples:

To wind a coil with 300 µH for a medium-wave detector radio on a cardboard roll with a diameter of 42 mm, assuming a wire diameter of 0.5 mm, you would need about 90 turns. The tuning capacitor must be at least 320 pF to tune the medium-wave range starting from 530 kHz.

For higher frequencies, you need fewer turns. For example, a coil in a shortwave receiver has 25 turns, a diameter of 8 mm, and a coil length of 10 mm, resulting in an inductance of 3 µH. With a capacitance of 320 pF, you can tune down to 4.4 MHz.

The previous examples used air coils. But how can you use a ferrite core? Often you don't have exact data on the core material properties, so you have to estimate by how much the core increases the inductance or decreases the frequency. For a coil in the shortwave range, with \textit{n} = 18 turns, \textit{L} = 12 mm, and \textit{D} = 8 mm, you can estimate an inductance of about 1.7 µH and a frequency of 7.3 MHz using a capacitance of 275 pF for an air coil. With a variable capacitor of 275 pF and a fully inserted ferrite core, experiments show a lower frequency limit of 3.7 MHz, or an inductance of about 6.8 µH. Using an adjustable screw core, the frequency can be halved, and the inductance can be increased up to four times.

Similarly, a longer ferrite rod for the medium-wave range can increase the inductance by about ten times. Roughly speaking, for a coil on a ferrite rod to achieve the same inductance requires only about one-third of the turns of an air coil of the same size.

The resonant frequency of a resonant circuit can change significantly when installed in the circuit. Especially at higher frequencies, line capacitances, for example, can have an effect. For this reason you often have to make corrections afterwards or plan for adjustment options from the beginning, such as using screw cores or trimmers. For large changes, the following rules of thumb, which can be derived directly from the formulae given and can be simulated with the LCFR program, often help: doubling the number of turns causes quadruple the inductance and half the frequency with the same value of capacitance. So, the frequency is inversely proportional to the number of turns. On the other hand, the frequency is inversely proportional to the square of the capacitance. Therefore, you can double the frequency with a quarter of the capacitance value. For example, to tune a frequency range from 1 to 3 using a variable capacitor, the capacitance ratio must be at least 1 to 9.

Let's take a closer look at the achievable bandwidth and quality with an example. Let's say you have a shortwave resonant circuit with \textit{L} = 3 µH, \textit{C} = 240 pF, \textit{f} = 5.9 MHz, \textit{RL} = \textit{RC} = 112 Ω. With a thick wire or a good ferrite core, you can achieve an unloaded quality factor (\textit{Q}) of 100, which means the unloaded bandwidth would be about \textit{b} = 6000 kHz / 100 = 60 kHz. The resonant resistance would be 112 Ω × 100 = 11.2 kΩ, say roughly 10 kΩ.

The actual losses, caused by copper resistance, are around 1 Ω, while the DC resistance is much lower. However, the effective loss resistance increases due to skin effect, where the RF current migrates to the thin outer layer of the conductor. To reduce losses, coils for the medium and long wave bands are usually wound with multi-stranded, individually insulated copper wires called "RF-Litz" wire.

In a crystal receiver, the working \textit{Q} factor shouldn't be set too high to get a good output volume. By loading the circuit with about 10 kΩ in parallel, you can achieve good power matching and volume, with a working value of \textit{Q} = 50 and a bandwidth of 120 kHz. This example indicates that neighboring stations within a band will not be separated. The actual value of Q also depends on the antenna used and its coupling.

\section{The Vacuum Tube Detector}

In addition to semiconductor diodes, there are also tubes that can serve the same function. A typical tube diode that can be used as an RF detector is the EAA91. This is a dual diode with a heater voltage of 6.3 V/0.3 A. Unlike a germanium or Schottky diode detector, a radio built using this type of detector consumes a continuous 1.8 W of power just to recover the signal.

Unlike a germanium or silicon diode or even the crystal detector made from galena, the tube diode does not require the signal to exceed a minimum voltage threshold before current starts to flow. Even without a positive anode voltage, some electrons will find their way to the anode. A short-circuit current of about 30 µA can be measured. With a load resistance of 1 MΩ, the tube has a grid voltage of 0.5 V, thereby creating its own appropriate bias voltage.

Using the two diodes in the tube together with a twin gang tuning capacitor (one range value covers shortwave while the other is suitable for medium wave) and a few other components, you can build a dual-band radio covering both medium wave and shortwave. This essentially builds two completely independent radios, with band switching occurring after rectification. The output signal can then be fed to, for example, a set of PC active speakers. The selectivity is good in both frequency bands because the rectifier circuit has very high impedance.

The two-band radio is coupled to the antenna through small coupling capacitors of around 30 pF. With a sufficiently long antenna, you can receive numerous stations. Distant European stations can be heard on shortwave and medium wave, especially in the evening.

\section{Diode Radio with Active Regeneration}

You have seen that a simple shortwave detector is not very sensitive or selective but with the help of a regeneration circuit, its performance can be significantly improved. The additional circuit can compensate for losses in the oscillator circuit. With this design, the RF signal is amplified by a transistor and fed back (in phase) into the oscillating receive coil. With carefully adjusted amplification, the feedback can compensate for all losses. In this state the oscillating circuit will be optimally damped and have a very high quality factor up to about \textit{Q} = 1000. This high \textit{Q} factor means that broadcast stations only 10 kHz apart can be separated and very weak stations will be easily picked up.

The regeneration feedback is produced here using an NPN transistor but you could also use a tube. Even with this regeneration circuit the radio is still classed as a detector receiver. It's only when the tube or transistor itself also replaces the function of the detector diode that the radio becomes an Audion receiver. Audion refers back to Lee De Forest's patent which turned out to be the prototype of the triode vacuum tube. Up until then crystal detectors were all you had to demodulate radio signals. De Forest's Audion tube took care of demodulation and amplification of the recovered baseband signal.

The feedback circuit used here could essentially be almost any oscillator circuit. Here, a Hartley oscillator with emitter feedback is used because the necessary coil taps are already available on the receiver coil. An alternative is the circuit shown further down in section 4.4 which uses two PNP transistors.

This circuit can be easily connected to an amplifier, such as a set of active PC speakers, making it a great shortwave radio. The antenna doesn't need to be very long — a roughly one meter long whip should do. To use the receiver, you tune it to a station and adjust the regeneration level until good output volume is achieved. If you crank up the regen pot too far, the set will burst into oscillation and what you thought was a receiver has now become a small carrier wave transmitter. The coil providing the feedback to the receiver coil is also known as a 'tickler' coil and can be electrically separate from the main coil.

When properly adjusted, the regenerative receiver can hold its own against any conventional shortwave radio. The audio is quite pleasant, and unlike a simple superheterodyne radio, there are no interference noises due to poor image rejection. During periods of strong signal fading, there isn't any unpleasant sound distortion, just temporary reductions in signal strength.

For those who think that a detector receiver using a battery and amplifier is cheating, don't worry — you can remove the battery and connect a crystal earphone instead. The radio still functions without the active regeneration signal, but with far less sensitivity.

\section{Regeneration using Tubes}

Instead of a transistor, a tube can also be used to reinforce the oscillations in the resonant circuit. Figure 2.21 shows just such a circuit using an EC92 tube. In addition to the anode voltage of 12 V, a filament voltage of 6 V is also required. The circuit shows a great similarity to the corresponding transistor circuit shown in Figure 2.19. The RF signal is applied to the grid via an RC network. The amplified RF current is coupled back into the resonant circuit via a tap. To adjust the regeneration level here, you can change the anode voltage.

In this circuit, te triode EC92 is operated with a low anode voltage of maximum 12 V, although it is actually intended for operation with voltages from 100 V upwards. The achievable amplification is however still sufficient for our purpose here. At low anode voltage, the tube has only a small usable slope range. Therefore, it must be placed directly in the resonant circuit, i.e. the grid is at the hot end of the resonant circuit, and the cathode is a tap approximately in the middle of the coil.

Almost any other triode can be substituted for the EC92. Double triodes such as the ECC81 or ECC82, which can be operated with a heater voltage of 12 V, are also well suited. If both halves are connected in parallel, the slope characteristic doubles. If the amplification is not sufficient with one tube and no oscillation can be observed even with the feedback control turned fully up, a smaller grid resistor should be used. The tube then enters the range of lower grid bias and higher anode currents.

A pentode can also be used; Figure 2.22 shows a circuit using an EF95. Here, the anode is fixed at the operating voltage, while the screen grid voltage is adjusted to change the characteristic slope. A relatively small 10 kΩ grid resistor guarantees safe oscillation even with strong antenna coupling. The tube circuit is already largely similar to the regenerative tube receiver described in section 3.5.

\chapter{Tube Regenerative Receivers}

Listening to distant radio stations or 'DXing' is a hobby which gained popularity after the introduction of the regenerative receiver design. This type of receiver was usually built using one or two tubes together with a tuning capacitor to adjust the reception frequency and a feedback control to control the oscillating regeneration circuit. Those who became experts at tweaking the controls were able to draw down the faintest of signals from the ether. During the subsequent era of superheterodyne receivers, it was easy to forget just how well the simpler regenerative receiver can perform. A good shortwave receiver working on this principle can be built using just one or two tubes.

\section{Triode AM Receivers}

As mentioned above, an Audion receiver is characterized by the fact that both RF demodulation and signal amplification are performed with just one active device. This provides better sensitivity and more amplification overall. The one valve shortwave radio shown in figure 3.1, therefore, has significantly more sensitivity than the tube detector receiver. The circuit can be used with either an EC92 or an ECC81 vacuum tube. For this simple circuit, a high-impedance earpiece of at least 600 Ω is required

Two different processes contribute to the demodulation of the RF signal. Grid rectification is based on the rising grid current at positive input voltage. Anode rectification is based on the curvature of the \textit{I}\textsubscript{a} / \textit{V}\textsubscript{g} characteristic curve of the tube. Depending on the tube and its operating point, one of the two rectification processes predominates.

\subsection{Anode rectification}

The characteristic slope of a tube depends on the grid voltage and increases with increasing input voltage. The non-linearity of the characteristic curve causes the average anode current to increase with increasing RF amplitude. It is easy to see that effective demodulation is only possible if the input voltage swing is not too narrow. Very small signal excursions will only drive the tube over a more linear region of the curve so that the rectification effect will be smaller and the recovered baseband will have lower amplitude.

\subsection{Grid Rectification}

The grid-cathode structure forms a rectifier. As the RF input voltage increases a larger grid current flows, which charges the grid capacitance negatively. This reduces the average anode current. The output signal is thus exactly out of phase with anode rectification. Again, a non-linear characteristic curve is responsible, this time the grid current characteristic curve.

The limited non-linear region ensures, as with a semiconductor diode, that significant demodulation only occurs at sufficiently large signal amplitudes.

It is not easy to predict which of the two effects predominates in practice. However, a assessment can be made based on the phase of the output signal. Figure 3.3 shows a measurement of an AM receiver using the ECC82. The lower channel shows an amplitude modulated RF signal directly at the control grid, and the upper channel shows the voltage at the anode. It can be seen that the anode voltage drops when the RF amplitude is large. Here, the tube is therefore operating in anode rectification mode, i.e., the grid current plays no significant role.

Figure 3.4 on the other hand displays the results using an EF95 pentode in an Audion configuration. As you can see, the average grid voltage actually decreases with larger RF signal levels, meaning that the grid capacitor is charged negatively. Consequently, the anode current decreases and the anode voltage increases. In other words, the tube functions as a grid rectifier.

In a given circuit, the type of demodulation depends on the tube, its operating point, the grid resistor, the grid capacitor, and the current modulation frequency. If the grid time constant is large, the grid voltage will not be able to keep up with a high modulation frequency. There may therefore be frequency ranges where both effects cancel each other out, resulting in little or no signal demodulation.

Both operating attributes of the Audion based on non linear characteristics make it clear why the audio amplitude increases disproportionately with the RF amplitude. As a result, weaker stations are practically inaudible because the carrier amplitude is not sufficient to produce any significant demodulation. This weakness can be overcome by adding feedback to the circuit as described in section 3.3.

\section{A Two-stage Receiver}

Without using any deliberate feedback, good results can be achieved by increasing the overall gain of the receiver by just adding a second stage. Figure 3.5 shows a tested circuit in the medium-wave range with a low-voltage ECC86 tube, which can operate with an anode voltage of only 6 V. A high-impedance earpiece was used for testing.

\begin{figure}[h]
\centering
\includegraphics[width=0.8\textwidth]{fig3-5.png}
\caption{A 2-stage Audion using an ECC86.}
\end{figure}

Similar Audion circuits can be built with other types of multiple tubes. Some of the more suitable are the combined Triode/Pentode type ECFxx or ECLxx family of devices, or their US equivalents. Figure 3.6 shows an example using an ECF12 steel tube. In this example, the pentode section of the tube is used to provide AF amplification. The output volume is strongly dependent on the anode voltage. Using this design to receive signals on the short-wave band and with an anode HT of 24 V provides an adequate volume level.

\begin{figure}[h]
\centering
\includegraphics[width=0.8\textwidth]{fig3-6.png}
\caption{An Audion built with a single ECF12.}
\end{figure}

A circuit without feedback cannot be as selective as a regenerative receiver in the short-wave range. However, it can provide good volume and sound quality.

\begin{figure}[h]
\centering
\includegraphics[width=0.8\textwidth]{fig3-7.png}
\caption{Experimental Audion using a steel vacuum tube type ECF12.}
\end{figure}

\section{Audion Receiver with Regeneration}

The design of a typical vacuum tube Audion type radio will most likely incorporate regeneration. This combines the demodulation technique described in section 3.1 with the oscillator circuit regeneration used in the detector receiver described in section 2.9. Altogether this makes a single tube responsible for three tasks at the same time: regeneration, demodulation, and amplification. The result is good reception performance with little outlay.

The ECC81, like the ECC82, is not specifically intended for low-voltage operation but works well from a single 12 V supply to power the heater and also the anode HT. Both these tubes can have their two heaters wired in parallel for 6.3 V / 0.3 A operation or in series from 12.6 V / 0.15 A. When the heaters are connected in series, a single 12 V battery is sufficient for heating and the anode HT voltage. In this regard, the ECC81/82 is superior to the ECC86 low-voltage tube, which only works with the heaters in parallel. The ECC83, which can also use 12 V for the heater, is not suitable for the receiver due to its very low anode current.

Feedback has already been introduced in the active regeneration of the oscillator circuit in the detector receiver and also leads to better selectivity and sensitivity with an Audion. Feedback means that the RF signal is amplified, phase-corrected and fed back to the receive coil circuit. The same principle is also used in a conventional oscillator circuit to maintain oscillation. In the radio receiver, the feedback signal reinforces the receive signal via a variable coupling to the coil or by an adjustable gain circuit so that the amount of feedback can be finely adjusted. As the feedback is increased and just before the onset of self-oscillation, any weak RF signal received on that frequency will undergo maximum amplification.

In practice the resonant circuit is damped by various energy losses in the circuit. These include losses due to wire resistance, the finite input resistance of the tube and the damping effect of the antenna. All of these losses can be compensated for by feedback, theoretically you get a resonant circuit with infinite \textit{Q} factor. In practice, a \textit{Q} factor of up to 10,000 can be achieved, resulting in a narrow bandwidth of, for example, 600 Hz in the 6 MHz band.

This leads to a reduction of the sidebands or an increase in the lower modulation frequencies. In fact, with the help of regeneration it is possible to achieve selective amplification of the carrier signal, which gives improved demodulation sensitivity.

The feedback arrangement can be accomplished in many ways. Practically any oscillator circuit can be used to supply the regeneration signal. Here, the particularly simple three-point circuit with feedback via the cathode is used, which only requires a tap point on the resonant circuit coil. The degree of coupling depends on the position of the tap and is a maximum when the tap is at the midpoint of the total coil length. With sufficient amplification by the tube, you only need about 1/10th of the total number of turns of the coil for the feedback winding. However, in order to achieve oscillation even with low anode voltage and low tube gain, the tap should be placed closer to the hot end of the coil.

Whether a tube with a given slope achieves enough amplification to sustain oscillation can be illustrated by an example. At 6 MHz and 300 pF, the capacitive resistance will be approximately 90 Ω. With an unloaded \textit{Q} of 100, the resonant resistance of the resonant circuit is 9 kΩ. The impedance at the cathode tapping is only a quarter of that, or about 2.25 kΩ, if the tapping is exactly at the midpoint. In order for amplification greater than one to be possible, the slope must be at least S = 1/2.25 kΩ, or at least about S = 0.44 mA/V. Especially in tube circuits with reduced anode voltage, a high unloaded \textit{Q} of the resonant circuit is therefore important. In particular, the shortwave coil should be wound with thick wire approximately 1 mm in diameter.

\begin{figure}[h]
\centering
\includegraphics[width=0.8\textwidth]{fig3-8.png}
\caption{An Audion with feedback.}
\end{figure}

The level of feedback is adjusted by changing the anode voltage which affects the characteristic slope and tube amplification. The 100 kΩ feedback potentiometer allows for very precise control and the circuit achieves smooth oscillation onset. If the antenna coupling is too strong however, the tube may not have enough amplification to provide adequate feedback. In this case, weaker antenna coupling or a smaller grid resistor may provide more amplification.

The \textit{Q} factor achieved after the effect of regeneration can be estimated by determining the bandwidth. At optimal settings, the Audion can easily separate stations with a channel spacing of just 10 kHz. The bandwidth is about 6 kHz, and higher modulation frequencies are noticeably suppressed. In the 6 MHz band, this results in an effective quality factor of 1000 and a resonant resistance of about 100 kΩ. The damping effects of the antenna and the tube's internal resistance are compensated for by the feedback. The actual \textit{Q} factor depends on the exact feedback adjustment. Near the oscillation threshold, an even higher \textit{Q} factor can be achieved, but the volume stops increasing at a certain point because the sidebands are no longer boosted.

This Audion with regeneration already provides reception capabilities that can compete with a simple superheterodyne receiver, especially because it does not generate an image frequency during demodulation and has good rejection of strong receive signals. For a listener the Audion is a really enjoyable receiver, especially in the evening when excellent long-range reception can be achieved with just a 3-meter length antenna. With interchangeable plug-in coils, multiple wavebands or spread shortwave bands can be implemented. Receivers for individual amateur radio bands are also possible. With feedback engaged CW and SSB signals can be received, which is only otherwise possible by using a superheterodyne receiver with a beat frequency oscillator (BFO).

An SSB (Single Sideband) transmitter doesn't transmit a carrier wave, but only the lower or upper sideband. When you listen to it on an AM (Amplitude Modulation) receiver, you'll hear an indistinct noise. The receiver has to add the carrier wave back at the exact frequency. If it's not tuned precisely, voices will sound distorted a bit like Mickey Mouse. In the 80- and 40-meter band the lower sideband is used, so the receiver needs to be tuned slightly above the receiving frequency. You'll usually have the best results in the 40-meter band at around 7050 kHz.

At the lower end of the band, around 7000 kHz, you'll hear CW (Carrier Wave) transmitters. These use unmodulated carriers that are turned on and off to transmit Morse code messages. To hear anything, you need to superimpose the input signal with a second signal. To do this, you need to detune the receiver using a feedback frequency about 500 Hz to 1 kHz below or above the receive signal. The exact adjustment determines the audible pitch. A deviation of 800 Hz also creates a beat frequency of 800 Hz. When you tune it exactly to the received signal carrier, nothing will be audible.

\begin{figure}[h]
\centering
\includegraphics[width=0.8\textwidth]{fig3-9.png}
\caption{The ECC81-Audion built on a wooden breadboard.}
\end{figure}

An important characteristic of a good regenerative receiver is soft onset of the feedback state. A harsh, sudden onset makes tuning difficult and produces distortion. For a gentle onset, it's important to have effective automatic gain control for high signal amplitude at resonance. Current flow in the grid increases the grid's negative charge. It's often the case that self-oscillations stop just above the critical point when tuned to a strong signal. The large signal amplitude of the useful signal itself reduces the gain. However, this only works with a relatively short grid time constant.

If the grid capacitance is increased, for example, to a value of 1 µF, fine tuning becomes impossible. It leads to swings in the oscillator gain with crackling noises. With each new oscillation attempt, the tube reaches a working point with higher anode current and higher slope, which amplifies the feedback. The oscillation then becomes larger until it gets cut off by a strong grid charge. Using a small grid capacitor below 1 nF ensures the gain is throttled back quickly enough to stabilize the amplitude.

Both processes responsible for demodulation in the tube, namely grid rectification and anode rectification, work together here. While the anode current characteristic curve increases the current with modulation, the rising grid current reduces the anode current by negatively charging the grid. This regulation even works like automatic gain control (AGC) to some extent and partially regulates field strength fluctuations automatically.

The circuit shown here was used with minor changes and an ECC82 in the anniversary edition of the Radiomann 2004 experiment kit from Kosmos. They have paid particular attention to recreate a feeling of the 1940s with the look and feel of this kit. Plug-in coils allow for band switching. A cylindrical coil is used for the shortwave band, and a flat coil for medium wave reception.

\begin{figure}[h]
\centering
\includegraphics[width=0.8\textwidth]{fig3-10.png}
\caption{The Kosmos-Radiomann kit fitted with a medium wave coil installed.}
\end{figure}

\section{Loudspeaker Operation}

The following circuit was developed as an extension of the Kosmos-Radiomann receiver. The original receiver design with an ECC82 and headphones output can be converted to speaker output with minimal changes. This was achieved using a PCL86 running at an anode voltage of 60 V. The triode section acts as the audio stage. Due to the higher anode voltage, the anode resistor should be increased to 100 kΩ, which achieves greater voltage gain. Like the original, this receiver is classified as a 0V1, which means it's a receiver without an RF preamp, with an Audion stage and an AF stage.

\begin{figure}[h]
\centering
\includegraphics[width=0.8\textwidth]{fig3-11.png}
\caption{Loudspeaker operation using a PCL86.}
\end{figure}

The power pentode in the PCL86 composite tube delivers around 5 mA of anode current at 60 V, which is adequate for speaker operation. The 20:1 ratio output transformer can be used as is. The output tube then operates into an output impedance of 8 Ω × 400 = 3.2 kΩ. While the original radio was powered by batteries, a simple external power supply needs to be used here. The PCL86 has a heater rating of 13 V/300 mA, so you could salvage a power transformer from an old 12 V halogen desk lamp, for example. The anode voltage is generated by a voltage multiplication network. To achieve the best possible hum-free operation, the screen grid voltage and the operating voltage for the audio stage are smoothed with an additional RC filter. The receiver provides good sensitivity and volume on both bands.

\section{A Regenerative Receiver using two EF95 Tubes}

Usually, when a pentode is used in an Audion receiver design it results in greater gain levels than when using a triode. That's why the EF95 is used here. This design uses the second EF95 as an AF (Audio Frequency) amplifier.

\begin{figure}[h]
\centering
\includegraphics[width=0.8\textwidth]{fig3-12.png}
\caption{EF95 Audion with AF-Stage.}
\end{figure}

The audio tube operates in triode mode with capacitor coupling to the headphones or an external audio amplifier. Using high-impedance headphones, the volume is sufficient; even 32 Ω headphones can be driven directly.

For more volume, the audio stage impedance matching needs to be improved. A small mains transformer is connected to the output to act as a signal impedance transformer. For example, a small power transformer with a rating of 230 V/24 V (120 V/12 V in the US) will be suitable. The winding ratio is about 10:1, so the impedance ratio is 100:1. If both drivers in the headphones are connected in parallel, their combined impedance is 16 Ω. From the tubes point of view it is now operating into a 1.6 kΩ load. This produces better matching to the high internal resistance of the tube output and leads to improved output power. A multi-tap output mains transformer is suitable to use as this impedance transformer. A power transformer salvaged from a plug-in mains adapter with switchable output voltages between 3 V and 12 V could also be used.

\begin{figure}[h]
\centering
\includegraphics[width=0.8\textwidth]{fig3-13.png}
\caption{Impedance matching using an output transformer.}
\end{figure}

The radio was built on the RT100 tube experimentation system from AK Modul-Bus. This system allows for experiments to be carried out on a plugboard. A variable capacitor, potentiometer, power supply connection, and audio jacks are included in the kit. For the shortwave range, a wire coil with two sets of 10 turns of single-strand wire is sufficient, which can be free wound without a supporting coil former.

\begin{figure}[h]
\centering
\includegraphics[width=0.8\textwidth]{fig3-14.png}
\caption{The regeneration stage wiring.}
\end{figure}

The variable capacitor connections terminate at the screw terminals and gold-plated 2 mm sockets. This conveniently allows coils for different frequency bands to be terminated in 2 mm plugs and plugged in as required, without generating any additional losses.

\begin{figure}[h]
\centering
\includegraphics[width=0.8\textwidth]{fig3-15.png}
\caption{Plug-in coils for short and medium wave operation.}
\end{figure}

Figure 3.15 shows two plug-in coils mounted on perforated circuit boards. The shortwave coil has an adjustable ferrite core and a total of 20 turns with a center tap. The medium wave coil uses a small ferrite rod with 80 turns of RF 'Litz' wire. On top of this is a small 10-turn coupling coil of enamel-coated copper wire.

\section{A Shortwave Audion Type 0V2}

This vacuum tube Audion regenerative receiver design was developed with the aim of building a sensitive shortwave receiver for speaker operation whilst using a safe anode voltage. This particular design is termed '0V2', which classifies it as a receiver with one regenerative stage and two AF stages. The same circuit functioned using almost the same component values but with an EF80 tube running on only 12 V. At 70 V, however, much more volume is achieved. At the higher voltage, the grid resistor must have a higher value. The EF80 was replaced by an EF183 which offers more gain and higher maximum frequency operation. Now it is a fully-fledged radio that performs rather well when compared with a tube superheterodyne radio. The 70 V anode voltage is considered safe, so no special precautions are necessary.

\begin{figure}[h]
\centering
\includegraphics[width=0.8\textwidth]{fig3-16.png}
\caption{The three-stage Audion receiver.}
\end{figure}

Often, the feedback signal is picked up from the anode contact and fed to a special coupling coil. However, with an indirectly heated tube, it's easier to connect the cathode to a tap on the resonant circuit coil. This gives us the basic circuit of a classic 'cathode 3 point oscillator' or, if it's a pentode, the so-called ECO circuit (Electron Coupled Oscillator). The amount of feedback applied is controlled by varying the characteristic slope. With a triode, this is achieved by changing the anode voltage; with a pentode it's done by changing the screen grid voltage. The audio stages don't need a cathode resistor here because the operating voltage is relatively low. The negative grid bias is automatically generated by the grid current as a voltage drop across the grid resistor.

From the 1950s to 1970s, the ECO Audion was very popular receiver design among radio amateurs as the classic "beginner's project". Plug-in coils were usually fitted to cover the various amateur radio bands. The sensitivity of the design is so good you can easily pick up long-distance stations, even with the relatively low anode voltage of just 70 V.

The coil has 20 turns with a tap at the 4th and 7th turn. The frequency range can be tweaked by adjusting the ferrite screw slug. The receiver covers the 49-m to 31-m band. However, you can also hear SSB voice communication and telegraphy in the 40-m amateur radio band. The antenna can be connected either directly to the lower tap or with loose coupling via an additional antenna capacitor. Any wire over one meter long can be used as an antenna.

The volume is more than adequate. With stronger transmitters, the volume needs to be turned down. You can listen to a station all day long. In the evening, the range increases and more distant overseas stations will also be received. The feedback oscillation onset is very soft and only lightly dependent on the frequency. So, you can find some settings where only the tuning capacitor needs to be adjusted. However, if you want to receive a weak station in amongst stronger ones, you have to maximize the gain and selectivity by setting the feedback as close to the oscillation threshold as possible.

\section{A 6 V Tube Regenerative Receiver}

The goal of this circuit is to create a regenerative receiver that operates from a single HT voltage of only 6 volts. For this purpose, an EL95 tube will be used, which can often be easier to find than the low-voltage tube types ECC86 or EF98. The EL95 also works very well with a low anode voltage of only 6 volts.

Good results have also been produced with a simple Electron Coupled Oscillator (ECO) circuit using a low anode voltage. The slope characteristic of the tube is controlled by adjusting the screen grid voltage. In this case, the screen grid can be grounded via a capacitor for both RF and audio frequencies because the demodulated audio signal feeds out from the tube's anode. The system of cathode-control grid-screen grid can be thought of as a triode system that is only responsible for adding regeneration to the receiving circuit and for demodulation.

\begin{figure}[h]
\centering
\includegraphics[width=0.8\textwidth]{fig3-17.png}
\caption{Single stage ECO Audion.}
\end{figure}

At the low anode voltage of 6 V, the tube has a low transconductance. Therefore, the Audion can only be brought into oscillation if the resonant circuit has a very high unloaded \textit{Q} factor. This requires a large air-core coil made with thick wire. Similar care is required as with free-running oscillators in amateur radio equipment. Also make sure the antenna coil is loosely coupled.

\begin{figure}[h]
\centering
\includegraphics[width=0.8\textwidth]{fig3-18.png}
\caption{Shortwave Audion using an EL95.}
\end{figure}

This regenerative receiver can be used to receive Digital Radio Mondial (DRM) broadcasts where available. In this application, good frequency stability is crucial for reliable reception, and this all comes down to the oscillator circuit design. To achieve the required level of stability, you need a large 20 turn coil of 1.5 mm diameter wire wound onto a length of 18 mm diameter PVC pipe. All connections must be as short as possible especially between the coil and the air-spaced variable tuning capacitor. With careful construction a \textit{Q} factor of over 300 was obtained. All other connections should be kept mechanically secure. Nothing can be allowed to wobble or physically vibrate. Even the tube is physically secured on its glass top spike to reduce vibrations.

The receiver is best operated with an external audio amplifier or connected to a PC sound card. It is also possible to listen in directly using a high impedance headphone. With just one stage, clear reception of numerous shortwave stations can be achieved.

\section{Cascode Triode Regeneration}

The cascode regenerative receiver has proven itself particularly useful in the field of amateur radio. Back in the 1950s and 1960s, it was a popular radio design and proved a good introduction into receiver technology for many budding radio enthusiasts. Often, an ECC81 was used, with the receiver running at 250 V.

In general, triodes are less noisy than pentodes. The additional grid structures in a pentode such as the screen and suppressor grid can introduce more noise into the circuit. Pentodes, however, have a higher voltage gain and lower feedback capacitance. The cascode configuration combines the advantages of both characteristics. For this reason, it was often used for VHF preamplifiers in television receivers, where high input sensitivity is crucial. What applies for an RF preamplifier also applies for a regenerative receiver.

\begin{figure}[h]
\centering
\includegraphics[width=0.8\textwidth]{fig3-19.png}
\caption{The cascode amplier configuration.}
\end{figure}

The cascode regenerative receiver design uses the cascode configuration with feedback applied via the cathode. The feedback potentiometer changes the grid voltage of the upper tube, and thus the anode voltage of the lower triode, which changes the characteristic slope of the circuit. The feedback can be adjusted to just on the verge of oscillation, so that the input circuit is optimally stimulated. An antenna signal of just a few microvolts can be amplied to several hundred millivolts of RF signal, resulting in high selectivity and bandwidths of around 5 kHz with high gain.

\begin{figure}[h]
\centering
\includegraphics[width=0.8\textwidth]{fig3-20.png}
\caption{The cascode Audion regenerative configuration.}
\end{figure}

When using tube receivers with low anode voltages, triodes (ECC82 in the case of radios) or pentodes (EF95 in the RT100) are often employed. At first glance, the cascode configuration has a significant disadvantage: it halves the already-low anode voltage. That's why a special tube such as the ECC88 is required that can handle this.

Most tubes are designed for use with anode voltages between 100 V and 300 V but the legendary ECC86 was developed to operate at low anode voltages such as 6.3 V or 12.6 V. Back in the day, before transistors became commonplace, the goal was to design a car radio that didn't need a voltage inverter to generate the high voltages normally used for the tube HT supply. Operating with an anode voltage of, say, 12 V, the tube still achieves a mutual conductance characteristic slope of 4.6 mA/V.

\begin{table}[h]
\centering
\begin{tabular}{|l|l|l|l|}
\hline
\textbf{ECC86} & \textbf{Heater rating} & \textbf{Maximum Operating conditions} & \textbf{Operating with VA at 6.3 V} \\
\hline
For RF, amplier, VHF Mixer & 6.3 V/ 0.33 A & PA = 0.6 W & VA = 12.6 V \\
& Ic = 20 mA & VA = 6.3 V & IA = 2.5 mA \\
& VA = 30 V & VG = 0 V & IA = 0.9 mA \\
& S = 4.6 mA/V & S = 2.6 mA/V & VG = 0 V \\
\hline
\end{tabular}
\caption{Key features of the ECC86 dual triode.}
\end{table}

After many experiments it turns out that many other tubes can also operate successfully with an anode voltage of just 12 V. In general, the useful operating range shifts down to very low grid biases. This means that a non-negligible grid current flows, so you cannot expect to achieve extremely high input impedance. However, an Audion has always been driven with grid current, so there is hardly any disadvantage here.

On inspection, what stands out physically about the ECC88 is the special shape of its triode structure. Small notches in the anode plates provide extremely close proximity to the cathode. This same structure can also be found in the low-voltage ECC86 tube. It's clear that a voltage grid allows an even shorter distance between cathode and grid. Short distances equals' low voltage: this rule also applies in reverse, as can be seen in giant vacuum tubes used by high power broadcast transmitters. In any case, the ECC88 looks suitable for low voltage operation from a cursory visual inspection. The single triode EC88 is also built very similarly and has proven to be a good low-voltage triode.

More widespread than the ECC88 tube was the PCC88 which could often be found in VHF front ends of television sets. It was used in a cascode configuration, with the two tube sections connected in series. Each section receives only half of the operating voltage. Therefore, the tube is already designed for lower anode voltages from the start. With a VA = 90 V and Vg = –1.3 V, a characteristic slope of 12.5 mA/V is achieved. That's a good sign and gives hope that it will work well at even lower voltages. This tube is also popular among Hi-Fi enthusiasts in its E variant with a 6.3 V heater voltage.

\begin{table}[h]
\centering
\begin{tabular}{|l|l|l|l|}
\hline
\textbf{ECC88} & \textbf{Heater} & \textbf{Max Value} & \textbf{Operating Value} \\
\hline
For VHF input stages & 6.3 V/ 0.365 A & PA = 1.8 W & VA = 90 V \\
Cascode circuits & IK = 25 mA & IA = 15 mA & VG = –1.3 V \\
& VA = 130 V & S = 12.5 mA/V & \\
\hline
\end{tabular}
\caption{Key features of the ECC88.}
\end{table}

An easy measurement method is available to test the suitability of a tube for operation at low voltages. In this method, the grid bias is generated by using different value grid leak resistors. The voltage amlication is measured using an audio-frequency generator and an oscilloscope. A 1 kΩ anode resistor is used in the circuit so a gain of 1 corresponds to a transconductance of 1 mA/V.

\begin{figure}[h]
\centering
\includegraphics[width=0.8\textwidth]{fig3-21.png}
\caption{The test circuit.}
\end{figure}

\begin{table}[h]
\centering
\begin{tabular}{|l|l|l|l|}
\hline
\textbf{ECC86} & \textbf{RG} & \textbf{VG} & \textbf{IA} & \textbf{S} \\
\hline
VA=12 V & 0 & 0 & 2.2 mA & 1 kΩ \\
& 1 kΩ & -30 mV & 2.1 mA & 3.3 mA/V \\
& 10 kΩ & -200 mV & 1.45 mA & 2.8 mA/V \\
& 100 kΩ & -510 mV & 0.58 mA & 1.6 mA/V \\
& 1 MΩ & -710 mV & 0.28 mA & 1.0 mA/V \\
\hline
\end{tabular}
\caption{Test measurements of the ECC86}
\end{table}

When it comes finding triodes that can operate at low anode voltages, you need to compare their characteristics with the ECC86. Measurements indicate that the ECC88 is very similar. Although the anode current IA of the ECC88 with VA = 12 V and VG = 0 is only about half as large as that of the ECC86, the value of transconductance is about the same. Therefore, it can be assumed that the ECC88 can be considered a good replacement for the now scarce ECC86.

\begin{table}[h]
\centering
\begin{tabular}{|l|l|l|l|}
\hline
\textbf{ECC88} & \textbf{RG} & \textbf{VG} & \textbf{IA} & \textbf{S} \\
\hline
VA=12 V & 0 & 0 & 1.2 mA & 1 kΩ \\
& 1 kΩ & –25 mV & 1.1 mA & 3.3 mA/V \\
& 10 kΩ & –80 mV & 1.0 mA & 3.0 mA/V \\
& 100 kΩ & –210 mV & 0.60 mA & 2.0 mA/V \\
& 1 MΩ & –380 mV & 0.29 mA & 1.0 mA/V \\
\hline
\end{tabular}
\caption{Test measurements of the ECC88.}
\end{table}

This receiver uses two double triodes type ECC88, one for the Audion stage and one for the two-stage AF amplier. It's classed as a 0-V-2 receiver. The heaters of both tubes are connected in series so that the heater voltage is 12 V and the whole receiver can be powered with a simple 12 V power supply.

\begin{figure}[h]
\centering
\includegraphics[width=0.8\textwidth]{fig3-22.png}
\caption{The receiver circuit diagram.}
\end{figure}

When building an Audion, it's best to use a high value of grid resistance to keep the grid current low, the resonant circuit will then be lightly loaded and loosely coupled to the antenna. Usually, a grid resistance value of 1 MΩ will be used when working with high anode voltages and 100 kΩ when working with lower operating voltages. Through experimentation, a resistance of 270 kΩ was found to be optimal for use with the ECC88, which has a relatively high characteristic slope even at low anode voltages and currents. The anode resistance of the cascade stage can also be designed to be quite high at 27 kΩ, which results in a good level of amlication. The feedback setting is approximately mid position of the potentiometer (P2 on the RT100), which provides the first triode with an effective anode voltage of 6 V and an anode current of approximately 0.1 mA. These settings allow for loose antenna coupling and give a high open circuit \textit{Q} for the input circuit.

The antenna coil is wound on a 5 mm diameter coil former with a screw core and has a total of 20 turns of 0.2 mm diameter copper wire. The antenna tap is located at two turns, and the feedback tap is located at a total of seven turns. With these coil parameters, the receiver covers the range from 5 to 12 MHz approximately and the screw ferrite core allows for tuning adjustments. The coil former connection pins have been replaced with longer versions that fit into a plug board.

\begin{figure}[h]
\centering
\includegraphics[width=0.8\textwidth]{fig3-23.png}
\caption{The antenna coil.}
\end{figure}

The two-stage audio amplier uses a small ferrite output transformer with a winding ratio of 10:1. If the connected headphones have an impedance of 64 Ω with both drivers wired in series, the external resistance for the final amplier tube is 6.4 kΩ. The total amlication provided by this amplier is so high that you need to introduce a volume control (P1).

\begin{figure}[h]
\centering
\includegraphics[width=0.8\textwidth]{fig3-24.png}
\caption{The finished receiver.}
\end{figure}

The receiver was operated using a 10-meter long wire antenna. All radio stations in the 49, 41, and 31-meter bands are received loud and clear. The sound tone is very pleasant and clear. Reception is enjoyable, especially because of the more than ample volume. The louder stations must be turned down quite significantly. In the 40-meter amateur radio band, CW and SSB stations can also be received with feedback engaged. Instead of headphones, a PC active speaker set can also be used for listening via a loudspeaker.

The sensitivity of the Audion is in no way inferior to a superheterodyne receiver, and the sound is sometimes even better. To test the sensitivity, you can listen with feedback engaged once with and once without an antenna connected. Without an antenna, only a low level of noise is audible, but with an antenna connected, significant noise and crackling can be heard. The inherent noise floor is thus clearly lower than the antenna noise level. More sensitivity will not bring any benefit in the signal to noise ratio. The concept of the particularly low-noise cascode Audion also proves its worth here.

Another possible source of interference is mains hum, which can be picked up by the high-impedance input stage if the circuit layout is badly implemented. The RT100, however, has a continuous ground plane under the patch panel, which functions as an effective shield. For the same reason, the receiver also shows hardly any sensitivity to hand movements close to the set, which would otherwise cause frequency shifts.

Once a circuit has been developed on the RT100, it can easily be replicated and then built more conventionally, for example, on an aluminum chassis. However, the plug board layout has distinct advantages if you want to try out small changes. For example, you could use a band spread for the 49-meter band or for the narrow 40-meter band between 7.0 and 7.1 MHz. Trying out various value parallel and series capacitors available in the bottom of your junk box is faster than calculating and planning. Figure 3.25 shows a variant for the 40 meter band.

\begin{figure}[h]
\centering
\includegraphics[width=0.8\textwidth]{fig3-25.png}
\caption{Bandspread control for the 40 m band.}
\end{figure}

\section{The 6J1 Tube Radio}

This radio kit from Franzis Verlag uses the 6J1 tube which is a far-eastern clone of the EF95 tube. This nostalgic shortwave radio is a genuine tube radio, just like the ones built in the pioneering days of radio technology. An RF tube in the receiving section ensures excellent reception performance, while a modern amplier IC provides all the necessary room filling volume. The radio operates from a 15 V anode voltage.

\begin{figure}[h]
\centering
\includegraphics[width=0.8\textwidth]{fig3-26.png}
\caption{The Franzis tube radio.}
\end{figure}

In this design, the tube performs three tasks: amlication, resonant circuit tuning and RF signal demodulation. The 6J1 pentode is operated in triode mode with a connection between the screen grid and anode. The grid resistor R1 is connected to the anode, which increases the grid bias. This gives a sufficiently large anode current at low anode voltage. With the cathode connected to the center tap of the resonance circuit, amplied RF energy is fed back into the circuit. The tube operates in a Hartley oscillator configuration, which amplies an incoming signal. At the same time, the grid diode rectifies the RF signal to perform demodulation.

\begin{figure}[h]
\centering
\includegraphics[width=0.8\textwidth]{fig3-27.png}
\caption{The 6J1 Audion radio with output amplier IC.}
\end{figure}

By adjusting the anode voltage with the feedback controller P1, the level of amlication can be chosen so that the oscillator is just on the verge of oscillating. At this working point, the tube compensates for all losses that occur in the resonance circuit. The \textit{Q} factor can be increased from about 50 to over 1000. At a reception frequency of 6 MHz, the bandwidth is about 6 kHz, which means that closely spaced transmitters can be effectively separated. The tuning peak also simultaneously leads to an increase in signal amplitude. Therefore, RF signals of several hundred millivolts can occur at the control grid of the tube. The AM signals are demodulated at the grid diode by an increase in grid current with greater RF signal amplitude and a decrease in grid voltage. The demodulated AF signal appears at the same time at the grid and modulates the anode current. The AF signal therefore appears at the anode resistor R2. T2 acts as an AF preamplifier for the integrated amplier IC1.

The radio uses two battery packs: four AA batteries supply a total of 6 V for the tube heater and the AF amplier. The second is a 9 V battery connected in series with the heater battery so together they can supply an anode voltage of up to 15 V. The volume control has a power switch which only has one contact. This disconnects the heater battery and turns off T1 to disconnect the anode battery. The anode, screen grid, and control grid remain at 9 V when the device is switched off but the tube cathode will be cold so no current flows in this state. When the power is switched on, T1 conducts and connects the lower end of P2 to ground. The operating current of the anode battery is less than 1 mA, so this will usually last longer than the heater battery.

\begin{figure}[h]
\centering
\includegraphics[width=0.8\textwidth]{fig3-28.png}
\caption{The coil and IC amp mounted on the PCB.}
\end{figure}

The entire circuit fits onto a very compact PCB. All wired components and the coil are located on the assembly side. The tube socket is installed on the reverse side.

\begin{figure}[h]
\centering
\includegraphics[width=0.8\textwidth]{fig3-29.png}
\caption{The tube socket mounted on the reverse side.}
\end{figure}

\begin{figure}[h]
\centering
\includegraphics[width=0.8\textwidth]{fig3-30.png}
\caption{The PCB wiring plan.}
\end{figure}

\begin{figure}[h]
\centering
\includegraphics[width=0.8\textwidth]{fig3-31.png}
\caption{All the wiring in the radio set.}
\end{figure}

In the final configuration, the circuit board is held in place by the wires to the tuning capacitor which is mounted on the front face. The tube lines up behind an opening in the case so that users can benefit from the warm glow of nostalgia.

\section{A Tube Regen for the 80 m Band}

The specification for an amateur radio receiver in terms of frequency stability and sensitivity are quite high. The goal here is to design an amateur receiver built with vacuum tube devices which meets the spec. A design for a crystal controlled transmitter using an EL95 tube can be found in Section 5.7. This receiver design should complete the rig.

\begin{figure}[h]
\centering
\includegraphics[width=0.8\textwidth]{fig3-32.png}
\caption{A tube-based shortwave rig.}
\end{figure}

For use in amateur radio, the Franzis tube radio has been tweaked for use as an 80-meter amateur radio receiver. The crucial change relates to the oscillator circuit. Bandwidth tuning is important so that the receiver can be precisely tuned. With five parallel 56 pF capacitors and the 20 pF FM tuning capacitor, a band from 3500 kHz to 3620 kHz has been achieved. This includes the entire CW band and the beginning of the SSB band. With a long antenna, the receiver is sufficiently sensitive, and the frequency stability is also good. In this regard, the tube radio is significantly superior to its transistor radio equivalent.

\begin{figure}[h]
\centering
\includegraphics[width=0.8\textwidth]{fig3-33.png}
\caption{The tweaked receiver.}
\end{figure}

When I first started experimenting with the tube receiver together with the tube transmitter to build a ham radio rig, it proved really tricky to switch between send and receive. I had to flip the send switch, turn back the feedback regulator on the receiver, and adjust the volume to a level where I could just hear a slight tone. Then, after making the CQ call, I needed to quickly readjust everything to optimal settings, so I wouldn't miss any reply. To improve usability I decided to add an automatic mute function to the receiver. Now when I switch to send, the transmitter applies a DC voltage to the antenna input, which turns on two transistors. One acts as an audio gain control and switches a small resistor in parallel with the volume pot. The other connects a resistor to the feedback potentiometer's wiper. This reduces the feedback below the oscillation threshold but does not completely turn it off. This mute function is slightly delayed by a parallel base capacitor. With this mod you can hear the crystal oscillator's whistle for a very brief moment after switching. This is quite useful because it allows you to estimate if you're still on the same frequency when transmitting and receiving.

On top of this I soldered two capacitors across the volume potentiometer to filter out some RF interference that was being picked up when I touched the volume control knob while Morse keying. This has now been eliminated thanks to this low-pass filter. The sound quality during reception is also slightly improved. However, a faint hum is still audible. This can be traced back to ripple on the transmitter anode supply. Since the feedback still has some effect and there is still some amlication, the Audion now operates as an AM receiver to demodulate the transmitter's 100 Hz residual hum.

\chapter{The Transistor Audion}

Compared to a vacuum tube Audion, a transistor version can be built smaller and with less effort. Its receive sensitivity will not necessarily be any worse than its vacuum tube equivalent. In the age of semiconductors the vintage Audion design is still relevant and its simple construction makes it an ideal practical introduction to radio frequency technology.

\section{A One Transistor Radio}

The circuit in Figure 4.1 shows a receiver without feedback, consisting of just one transistor and one 1.5 V battery. A low-impedance set of headphones can be used, preferably with the left and right drivers connected in series, resulting in a working resistance of 64 Ω. The headphone jack also serves as the on/off switch, as the power supply is disconnected when the headphone is unplugged.

In this circuit, the transistor performs both demodulation and signal amlication. The sensitivity of this receiver is so good you only need a 2 m length of wire as an antenna. The tap on the coil should be at about 1/5 of the total number of turns of the resonant circuit coil. The circuit is suitable for all AM bands from long wave to shortwave. A shortwave coil can be made of 25 turns with four taps, as described in the detector receiver in Section 2.3.

\begin{figure}[h]
\centering
\includegraphics[width=0.8\textwidth]{fig4-1.png}
\caption{A 1-transistor receiver without feedback}
\end{figure}

This transistor receiver works in similar way to the vacuum tube design. Once again, the RF signal at the input shifts the operating point to recover the modulation signal. In this case, the base-emitter diode also shifts the average input voltage lower when a high amplitude signal is received. The time constant of the base circuit is so large that audio frequency signals at the base are shorted to ground. The collector current increases with the instantaneous value of the signal amplitude.

The most important difference compared to the vacuum tube radio is that the transistor circuit operates at much lower impedance levels. The input resistance is about 5 kΩ and depends on the collector current and the current amlication factor. The input must therefore be connected to a tap on the receiver coil that will not damp the resonant circuit too much.

\section{A Shortwave PC Radio}

The majority of radio receiver dongles or hardware for PCs, which pick up over-the-air radio broadcasts, are designed to cover the FM band. There is no reason however why you can't build one which covers the medium or shortwave bands.

The untuned receiver shown in Figure 4.2 is designed for direct connection to the microphone input of a PC's sound card. A supply voltage of 2.5 V is provided by the PC at the sound input to power the usual electret type microphones that plug into the sound card, a resistor of 2 to 3 kΩ provides a DC path to this supply. This means you already have supply voltage and a collector load resistor to build a simple transistor receiver.

\begin{figure}[h]
\centering
\includegraphics[width=0.8\textwidth]{fig4-2.png}
\caption{An untuned PC radio.}
\end{figure}

In operation the diode at the base-emitter transistor junction demodulates the incoming RF signal. The supply voltage via the resistor forward biases the junction so that it only takes RF signals of just a few millivolts to produce a demodulated baseband signal at collector. The circuit is therefore much more sensitive than a simple diode detector design.

To hear the radio output on the PC speakers, you will of course need to turn the microphone input on. The corresponding control provides volume adjustment. Among the advanced settings of the microphone input, there is an additional switchable preamplifier that should also be turned on. The radio never needs to be disconnected from the sound card since it can be turned off by clicking on the "mute" tickbox.

This receiver design hasn't got any variable capacitor tuning and its reception is therefore extremely broadband. All strong stations from the 49 to the 19-meter band are received simultaneously. The coil has an inductance of approximately 2 µH and consists of 15 turns in two layers using a pencil as a former. The resonant circuit capacitance of about 100 pF consists of the transistor's base-emitter capacitance together with the antenna capacitance, resulting in a resonant frequency of approximately 11 MHz. The low input impedance of the transistor dampens the resonant circuit to such an extent that a \textit{Q} factor of one is obtained, meaning that the bandwidth is also approximately 11 MHz. The reception frequency ranges between 6 and 17 MHz.

Without any of the usual frequency selection control you can get some really surprising results. The special propagation conditions of shortwave radio signals cause one station or another to stand out stronger at different times of the day. You will be able to hear news broadcasts simultaneously in multiple languages, music from classical to pop, or folk songs from distant countries. Without any tuning, you can just sit back and listen in to the entire shortwave band.

To restrict the number of stations received simultaneously you can introduce a tuned resonant circuit to the radio design. In order to achieve a high \textit{Q} factor the transistor must be coupled via a low tap on the coil. The coil data and antenna coupling are the same as in the single-transistor radio from the previous section.

\begin{figure}[h]
\centering
\includegraphics[width=0.8\textwidth]{fig4-3.png}
\caption{The PC Audion with variable capacitor tuning.}
\end{figure}

\section{Regenerative Receiver}

The simple transistor receivers described so far do not yet match up to the reception performance of a good tube radio. A transistor radio without feedback already has relatively good sensitivity and can work with a short whip antenna. The sensitivity can be increased significantly by using RF feedback. This involves feeding back a portion of the amplied RF signal to the resonant circuit. This compensates for losses and increases the RF amplitude while also significantly improving selectivity. It is important to feed back just the right amount of energy, as feeding back too much can cause self-oscillations that result in a whistling sound. The level of feedback needs to be adjustable.

The circuit diagram in Figure 4.4 bears a strong resemblance to the corresponding tube radio circuit from Section 3.3. The feedback is also taken from a lower tap on the resonant circuit, but the transistor coupling has lower impedance. Either headphones or an audio amplier can be directly connected to the output.

\begin{figure}[h]
\centering
\includegraphics[width=0.8\textwidth]{fig4-4.png}
\caption{An Audion with feedback.}
\end{figure}

The receiver circuit achieves good sensitivity and volume but adjusting the feedback can be very tricky. The feedback loop is quite harsh and prone to oscillation. Additionally, strong signals can cause distortion, requiring the feedback to be pegged back. This results in a relatively good reception of strong signals, but it's difficult to find the optimal feedback setting. During modulation peaks, the transistor enters a region of higher gain slope characteristic, causing self-oscillation. This results in oscillations that periodically start and stop the feedback loop. The oscillation frequency is much lower than the modulation frequency and sounds like an unpleasant rattling noise. While you can adjust the feedback control to reduce this and obtain clear reception, the unpredictable behavior at the feedback threshold doesn't make for easy listening.

A softer feedback reaction requires self-regulation of the feedback, as in a well-designed tube radio. In the initial approach, you can try reducing the base resistor and capacitor. The shorter time constant allows for a fast enough self-regulation through the base current. The circuit now closely resembles active regeneration according to Section 2.8, except that the same transistor is now also used for signal demodulation.

\begin{figure}[h]
\centering
\includegraphics[width=0.8\textwidth]{fig4-5.png}
\caption{Feedback using a smaller time constant at the base circuit.}
\end{figure}

The circuit now exhibits a smoother feedback operation. Continuous oscillations (motor boating) can still occur but only when the control is turned up very high. Now the self-oscillation frequencies are higher at around 20 kHz. A loud noise can be heard and the circuit can be used in the upper shortwave band as a sensitive super regenerative receiver.

The smaller base capacitor of 100 pF improves the feedback operation but leads to lower volume. The AF gain drops to a minimum because the coupling at the base attenuates lower frequencies.

As with the vacuum tube Audion (see Section 3.1), two opposing processes are at work here. A high RF signal causes an increase in the collector current (anode demodulation) because of the transistor gain slope characteristic. At the same time, however, the base current negatively charges the base capacitor, thus reducing the collector current (grid demodulation). Both affects partially cancel each other out and lead to a low AF output voltage.

It has therefore been shown that optimizing the component values of an Audion stage built with a transistor poses greater difficulties than one with a vacuum tube. One solution is to divide the tasks. Separate transistors can be used for the feedback and the demodulator stages. The feedback stage is given a small time constant in the interest of a smooth feedback control. The Audion stage, on the other hand, uses a large base capacitor that shorts the base to ground even for AF signals.

\begin{figure}[h]
\centering
\includegraphics[width=0.8\textwidth]{fig4-6.png}
\caption{An Audion using separated feedback stages.}
\end{figure}

This circuit actually delivers reception results that come close to a good tube radio. Good sensitivity and volume, a smooth feedback loop, and good sound even when receiving stronger stations provide the basis for a powerful shortwave receiver.

\section{Separated Feedback Paths}

It has proven to be effective to use a separate amplier for tuning the oscillator circuit and to separate it from the Audion stage. The circuit in Figure 4.7 shows a differential amplier with two PNP transistors in an oscillator circuit. The amlication can be adjusted within wide limits by controlling the emitter current with the potentiometer. One advantage of the circuit is that the oscillator coil requires just a single tap. It should be placed at approximately one third of the total number of coil turns to avoid influencing the circuit too much.

\begin{figure}[h]
\centering
\includegraphics[width=0.8\textwidth]{fig4-7.png}
\caption{Audion with separate regeneration circuit.}
\end{figure}

The differential amplier stage has good linearity because the nonlinear gain characteristics of both transistors largely cancel each other out. That's why, in this circuit, there is no shift in the operating point with increasing input amplitude. Capacitor coupling is not required, resulting in a regeneration circuit that operates with a largely flat frequency response.

\begin{figure}[h]
\centering
\includegraphics[width=0.8\textwidth]{fig4-8.png}
\caption{The transistor Audion circuit board.}
\end{figure}

In the November 2000 issue of \textit{Elektor Magazine}, a circuit for a shortwave Audion radio was described along with a PCB designed for the circuit. A TDA7052 audio amplier IC takes care of the AF stage and the circuit employs an NPN transistor in an Audion detector configuration, using with two PNP transistors in a separate regeneration circuit. The voltage at the feedback potentiometer is stabilized by a forward biased LED. The coil has multiple taps, allowing for adjustment of the coupling between the antenna of the audio input and the regeneration circuit. Typically, the device covers a frequency range of about 5 to 12 MHz. An additional 300 pF capacitor switches the receive frequency to the 80-meter amateur radio band (3.5 to 3.8 MHz), where you can pick up SSB and CW transmissions.

\begin{figure}[h]
\centering
\includegraphics[width=0.8\textwidth]{fig4-9.png}
\caption{The receiver circuit diagram.}
\end{figure}

The two PNP transistors in a differential amplier configuration practically work as an oscillator. Therefore, you can add the missing carrier for SSB and CW signal reception. For AM reception, however, you set the current so that all losses are just compensated for and no oscillations occur. With optimal regeneration, the circuit provides very good amlication and selectivity. The circuit does not suffer from large signal and intermodulation products that affect many other types of receiver design because only the desired signal is amplied by regeneration. In practice, this simple circuit can outperform some lower-priced PLL world band receivers in terms of sound and sensitivity.

T1 and T2 form a differential amplier, with the input (base of T2) and output (collector of T1) both connect to the coil. This configuration acts like a negative differential resistance and adds the regeneration signal to the resonant circuit. The circuit can be connected to the hot end or to a tap point on the coil. For coils with high damping, the lowest tap may have too small an impedance, so that no oscillations can occur. The resonant circuit is theoretically best when the loss resistance is exactly compensated for by the negative resistance generator circuit. The gain is set by the emitter current. The antenna also dampens circuit; a long antenna should be connected to a lower tap point.

The coil has four times five turns, i.e., 20 turns with three tap points. A separate antenna coupling coil is not needed since the resonant circuit coil has multiple taps. Try each tap point for the most favorable match.

A 50 cm to 3 m length of wire is all you need for the antenna. You can easily listen to all the strong stations in the 49 m and 41 m bands. With feedback engaged, you can hear CW signals in the 40 m amateur radio band. The 80 m band can also be monitored with jumper JP1 in place.

\section{Regeneration using an Emitter Follower}

Using an emitter-follower configuration has often proven to be effective. This circuit operates similar to a diode circuit with biasing, where the input resistance is increased by the current amlication of the transistor. While you can expect a lower voltage gain here, it can be easily compensated for by subsequent stages.

\begin{figure}[h]
\centering
\includegraphics[width=0.8\textwidth]{fig4-10.png}
\caption{MW Audion in collector circuit.}
\end{figure}

One advantage of this circuit is the use of a simple coil without any tap points. This is possible because the collector circuit has a high input resistance. Regeneration is also achieved without tapping the coil. The RF voltage is coupled into the resonant circuit via a capacitive voltage divider provided by the transistor base-emitter capacitance Cbe and the emitter capacitor. Amlication is adjusted by controlling the collector voltage. This results in an easily adjustable regeneration control with soft oscillation onset. This circuit is suitable for a wide frequency range from about 50 kHz to 4 MHz i.e., from the longwave band to the lower shortwave band. By switching coils, multiple bands can also be covered.

The circuit can also be used without any regeneration regulation by using a lower operating voltage. Figure 4.11 shows a four-stage medium wave radio with a speaker that operates with only 1.5 V. With a current consumption of only 10 mA the life for an alkaline AA battery is approximately 200 hours. The radio works well with the internal ferrite antenna but adding a wire of about 2 meters as an additional antenna allows for the reception of more distant stations.

\begin{figure}[h]
\centering
\includegraphics[width=0.8\textwidth]{fig4-11.png}
\caption{The mediumwave radio receiver with loudspeaker.}
\end{figure}

The four-stage circuit has a significant overall gain, which means there is a potential risk of instability due to unwanted feedback of audio or RF signals. To prevent this, the Audion stage has its own supply smoothing capacitor, which prevents signals from it coupling to the power supply. The following audio stages operate with reduced cutoff frequency to prevent self-oscillation due to parasitic capacitances.

\section{A Medium Wave Receiver using the TA7642}

The integrated medium-wave receiver module ZN414 from Ferranti was later replaced by the MK484 and is now available as the TA7642. This 3-pin integrated module is packaged in a TO92 outline, designed to operate from a 1.5 V supply. Figure 4.12 shows a basic circuit where it is used to build a simple medium-wave receiver with a ferrite rod and variable capacitor for station tuning. The receiver has good sensitivity and selectivity and is comparable in reception performance to simple superheterodyne receivers. In the evening, reception across Europe is possible.

\begin{figure}[h]
\centering
\includegraphics[width=0.8\textwidth]{fig4-12.png}
\caption{A medium wave receiver using a ferrite rod antenna.}
\end{figure}

This circuit has a large overall gain and stimulates the resonant circuit depending on the supply voltage through a negative input resistance. Circuit stability is not guaranteed with a supply voltage higher than 1.5 V. If there are self-oscillations with a high-\textit{Q} resonant circuit, a 200 kΩ to 1 MΩ resistor can be connected in parallel to the coil to ensure stability.

The circuit contains a simple gain control so that weak and strong stations are received with almost the same volume. Large RF input signals increase the current consumption and voltage drop across the working resistor. This reduces the supply voltage at pin 3 which decreases the gain.

Figure 4.13 shows the internal circuitry of the TA7642. You can see an emitter follower with T1 as a high-impedance input stage. T4, T5, T7, and T9 form four RF amplier stages. T10 is the actual demodulator. All other transistors form auxiliary circuits to stabilize the operating points. The collector of the output stage T10 is also connected to the operating voltage of all preceding stages and reduces the gain at high RF input voltages. Only very small values of capacitor can be fabricated in integrated circuits so this receiver only works with high gain from around 500 kHz. The upper cut-off frequency is determined by the high-impedance design and the junction capacitances. The IC can be used but with certain limitations in the longwave and in the lower shortwave bands.

\begin{figure}[h]
\centering
\includegraphics[width=0.8\textwidth]{fig4-13.png}
\caption{The TA7642 Block diagram.}
\end{figure}

The receiver chip is well-suited to build a medium-wave PC radio, which, like the shortwave radio described in Section 4.2, is powered through the sound card's microphone input. The higher operating voltage of 2.5 V necessitates the use of an additional series resistor of 10 kΩ for stability purposes. The resistor should be bridged with an electrolytic capacitor to allow the full audio signal to be applied to the microphone input.

\begin{figure}[h]
\centering
\includegraphics[width=0.8\textwidth]{fig4-14.png}
\caption{Powered from the PC's sound card.}
\end{figure}

The TA7642 is currently not stocked by many suppliers so it would be useful if you try to replicate its function using discrete components. A block diagram of the chip shows its working principle so it's not so difficult to reduce the circuit to its essential components. Figure 4.15 shows an almost equivalent replacement circuit using only three BC548C NPN transistors. This substitute consists of one emitter follower input stage, only one RF stage, and the demodulator stage.

\begin{figure}[h]
\centering
\includegraphics[width=0.8\textwidth]{fig4-15.png}
\caption{A replica of the TA7642 built using discrete components.}
\end{figure}

The replica circuit shows very similar characteristics to the original but tends to be more prone to bursting into oscillation due to the larger emitter current in the input stage. This means that the regeneration effect is stronger. In principle, this is the same oscillator circuit as the emitter follower used in Section 4.5. If you use a coil with low self-damping, it may be necessary to use a resistor of value 200 kΩ to 1 MΩ in parallel with the resonant circuit, which partially compensates for the negative input resistance. Another advantage of this substitute circuit is that the larger coupling capacitor also allows operation in the long-wave band.

\section{A Retro Medium Wave Radio}

The design of Franzis-Radio (which uses a TA7642) is reminiscent of valve radio sets that were popular in the 1950s. In 2008 this kit originally included a signal strength analogue meter but later versions were housed in a smaller case without the meter. More recently, broadcastors have turned to more contemporary methods of sending out program material and in a lot of countries many medium wave stations are now silent. The kit is no longer available but the concept is still relevant; you can still listen to many interesting stations from all over Europe, especially during the evenings.

\begin{figure}[h]
\centering
\includegraphics[width=0.8\textwidth]{fig4-16.png}
\caption{The medium wave Franzis set oozes nostalgia.}
\end{figure}

The resonant circuit which serves as the receiving antenna consists of a ferrite-core coil and variable capacitor. The RF signal is taken out at a tap point on the coil and fed to the input of the receiver IC (Pin 1). At the output (Pin 3), both the demodulated audio signal and a control voltage for automatic gain control are present. This voltage drops from 1.2 V without a signal to below 1 V with a strong signal. The control voltage is fed back to the input via R4 and adjusts the receiver's amlication. This feedback ensures that strong and weak stations appear to sound almost equally loud.

\begin{figure}[h]
\centering
\includegraphics[width=0.8\textwidth]{fig4-17.png}
\caption{Schematic of the medium wave radio.}
\end{figure}

A regulated voltage between approximately 1 and 1.2 V is supplied via the volume control to the base of the amplier transistor T1. The operating point at around 20 mA is thus largely independent of the battery voltage and variations in the transitor's current gain, but sensitive to changes in the received signal strength. The display pointer shows the emitter voltage and thus also the emitter current of T1. The current is reduced by about 5 mA when the volume control is turned down because an additional base resistor of up to 10 kΩ reduces the base current. The gauge indicator pointer shows all changes in the emitter current and thus also the state of the battery, the set volume, and the signal strength of the selected station.

The circuit is particularly efficient and requires only a single 1.5 V battery. A typical alkaline battery with a capacity of 2000 mAh will power it for 100 hours at high volume. Turning down the volume will extend battery life.

The pointer deflection decreases when a station is received, which helps with more precise tuning. In old tube radios this function was often realized with a "magic eye" vacuum tube.

The ferrite rod antenna is quite directional; the received signal strength will be at a maximum when the ferrite rod axis is perpendicular to the signal source. This directional property is useful if you need to suppress a strong signal from a transmitter interfering with another station you want to listen to. Just rotate the radio so that the interfering signal drops to a minimum. At this point you know the ferrite rod axis is directed at the location of the interfering signal transmitter.

\begin{figure}[h]
\centering
\includegraphics[width=0.8\textwidth]{fig4-18.png}
\caption{The populated PCB.}
\end{figure}

The small PCB is easy to assemble. In addition to the receiver IC, there is a transitor which works as a power amplier. When fully assembled, the PCB is quite light and can be held in place just by the wires soldered to the tuning capacitor which is mounted to the front panel.

\begin{figure}[h]
\centering
\includegraphics[width=0.8\textwidth]{fig4-19.png}
\caption{It all fits in the case.}
\end{figure}

\section{Shortwave Regen using an Emitter Follower}

A shortwave radio using just two transistors and powered by a 1.5 V battery is a great way to get started with shortwave reception technology. You can connect the receiver to a set of active PC speakers to provide impressive reception performance.

The circuit has a unique feature. The BC558C PNP transitor works as an Audion in a collector circuit (emitter follower, see Section 4.5). Only a very low emitter current is required to achieve oscillation. Using the potentiometer, for AM reception you can adjust the Audion feedback level to the point just before oscillation starts and for CW and SSB reception, just after oscillation has begun.

\begin{figure}[h]
\centering
\includegraphics[width=0.8\textwidth]{fig4-20.png}
\caption{The 2-transistor Audion receiver.}
\end{figure}

The entire circuit was soldered onto the cut-out lid of a coffee can. A 100 pF trimmer is used as the tuning capacitor but you can also change station by adjusting the core in the coil former. Even though you will need to use a screwdriver to select a station, it's easy to tune in.

\begin{figure}[h]
\centering
\includegraphics[width=0.8\textwidth]{fig4-21.png}
\caption{Experimental construction.}
\end{figure}

The Audion features smooth feedback behavior and is relatively easy to use. The receiver frequency range covers from the 49-meter band to the 31-meter band. You can also listen to amateur radio in CW and SSB between 7.0 and 7.1 MHz. A wire antenna with a minimum length of 3 meters works well with this design.

\begin{figure}[h]
\centering
\includegraphics[width=0.8\textwidth]{fig4-22.png}
\caption{Expanding the design using 4 transistors.}
\end{figure}

This emitter-follower Audion design was adapted without any tuning and for loudspeaker operation. The idea is that anyone should be able to build it without the need for any special RF components; it doesn't even use a tuning capacitor or shortwave coil. All that's required is a handful of standard components, a speaker, an old tin lid and a 1.5 V battery.

\begin{figure}[h]
\centering
\includegraphics[width=0.8\textwidth]{fig4-23.png}
\caption{Loudspeaker operation.}
\end{figure}

Without a trimming capacitor, how can the receiver be tuned? The Audion uses a number of fixed value capacitors soldered together wired in parallel with the home-brew variometer coil that achieves fine-tuning by squashing or stretching the coil turns. To make this coil 17 turns of wire are tightly wound around an AA battery. When the coil is removed from the battery it expands slightly leaving 15 turns with a diameter of 17 mm.

In use its necessary to find the desired frequency and antenna coupling, then you can turn up the potentiometer near the feedback loop, until you hear an increase in noise which indicates an increased sensitivity. The coil can now be carefully squashed or stretched until the desired station is clearly audible. In the first attempt, I managed to pick up a strong station in the 49-meter band. Many other stations will also be available at dusk. To simplify tuning, the radio was equipped with a lever mechanism at the end to adjust the coil length. With a little skill, the frequency can be set as precisely as with a tuning capacitor.

\begin{figure}[h]
\centering
\includegraphics[width=0.8\textwidth]{fig4-24.png}
\caption{Tuning using a DIY variometer coil adjuster.}
\end{figure}

\section{A Shortwave Retro Radio}

The Franzis shortwave radio is a transitor Audion for the range of 3.5 to 9.5 MHz. The emitter-follower circuit was also used here, but in addition, an audio amplier using the LM386 was added.

\begin{figure}[h]
\centering
\includegraphics[width=0.8\textwidth]{fig4-25.png}
\caption{The Shortwave Transistor Audion radio.}
\end{figure}

Transistor T1 operates as an emitter follower in an Audion configuration performing three tasks: amlication, adding regenerative feedback to the resonant circuit and demodulating the RF signal. C2 and the internal base-emitter capacitance of about 5 pF form a capacitive voltage divider. Together with the resonant circuit, a Colpitts oscillator is formed. By adjusting the emitter current appropriately, the gain can be chosen such that the oscillator is just about to start oscillating. At this operating point, the transitor compensates for all losses incurred in the resonant circuit. The quality factor \textit{Q} can be increased from about 50 to over 1000. Correctly adjusted this can provide the radio with a receive bandwidth of about 6 kHz on a 6 MHz radio signal to give good separation of adjacent broadcast stations.

The regenerative feedback leads to an increase in the received signal amplitude. RF signals up to about 100 mV can therefore occur at the transitor base. The AM signals are demodulated by the non-linear transitor characteristics. The AF signal then appears at the emitter. R1 and C2 form a low-pass filter that removes any RF signal remnants. T2 forms an AF preamplifier for the integrated amplier IC1. The AF stage also uses a PNP transitor to avoid any possible mix up during construction.

\begin{figure}[h]
\centering
\includegraphics[width=0.8\textwidth]{fig4-26.png}
\caption{Transistor Audion with an IC output amp.}
\end{figure}

One special feature of this Audion circuit is the direct coupling of the transitor to the oscillator circuit. T1 operates with a collector-emitter voltage of only about 0.6 V. The base-emitter junction has a capacitance of about 5 pF which strongly affects the oscillating circuit. The close coupling ensures that the transitor also acts like a capacitance diode or varicap, allowing fine-tuning of the frequency via feedback control. The regeneration and onset of oscillation is quite soft so the frequency can be pulled by several kilohertz, which is advantageous for receiving SSB and CW stations.

The LM386 speaker amplier operates directly from a 9 V battery. Power consumption depends heavily on the volume setting. At low volume, the entire receiver only consumes about 5 mA. The LED serves not only to show the receiver is on but also for voltage stabilization as the LED has a constant 1.8 V forward voltage drop. As a result, the two transitor stages always receive a stable operating voltage.

\begin{figure}[h]
\centering
\includegraphics[width=0.8\textwidth]{fig4-27.png}
\caption{The PCB and components mounted in the case.}
\end{figure}

When tuning the frequency, you will find several shortwave bands with multiple stations. On shortwave, you can achieve a high range during the day, but many stations only broadcast in the evening. Below 4 MHz is the 75-meter band, which is often overlooked on many shortwave receivers. On this band you can pick up a few interesting stations in the evening. The 49-meter band at 6 MHz is densely occupied by numerous European stations. Some frequencies are used consecutively by different stations. The 41-meter band above 7 MHz is also very busy in the evening. This receiver also reaches parts of the 31-meter band above 9 MHz. Generally, you can achieve a better range at higher frequencies. Often, you can also receive stations from outside Europe. Between broadcast bands, there are numerous stations in CW (Morse telegraphy), SSB (single-sideband voice radio), RTTY (radio teletype), and weatherfax (weather picture radio). All of these stations can only be heard with activated regeneration.

\section{A 40-m Shortwave Audion with Regeneration}

The Franzis shortwave radio featured in the previous section with its open construction is well-suited for shortwave radio reception. It does, however, suffer from poor frequency stability when it comes to receiving CW and SSB signals on the amateur radio bands. It's also difficult to set the frequency accurately. The radio can be modified to make it more suitable as an amateur radio receiver specifically for the 40m band. The crucial improvement made here is to add some shielding to the circuits on the PCB. This reduces its sensitivity to the proximity of the radio operator's hands and helps reduce any tendency for unwanted circuit oscillations.

\begin{figure}[h]
\centering
\includegraphics[width=0.8\textwidth]{fig4-28.png}
\caption{RF shielding plate.}
\end{figure}

The thin tinplate used for the shielding is salvaged from a food container of cappuccino coffee powder. I've always kept such sheets and used them as a base for experimental circuits. The material is very thin and can be cut with regular scissors. I first cut out a cardboard template to determine the shielding plate dimensions. The final shield was then cut, bent, and soldered together.

\begin{figure}[h]
\centering
\includegraphics[width=0.8\textwidth]{fig4-29.png}
\caption{Earthing the shield.}
\end{figure}

I swapped the connections to the variable capacitor so that now the VHF section is being used. The ground tag of the AM side is soldered to the shield. A fairly thick ground wire also connects the variable capacitor ground terminal on the board to the metal shield, which also sets the mounting depth. Overall, there are several ground connections to the metal sheet. I also soldered some wire support struts to hold the free ends of the sheet and prevent mechanical vibrations. All of these measures have been very successful and the radio set is now totally unaffected by the radio operator's hand movements.

\begin{figure}[h]
\centering
\includegraphics[width=0.8\textwidth]{fig4-30.png}
\caption{Capacitors in the resonant circuit.}
\end{figure}

The band spread uses a 56 pF capacitor across the resonant circuit with another 56 pF capacitor in series on one side of the 20 pF VHF variable capacitor and its parallel 10 pF trimmer. The dimensions were not precisely calculated but by experimentation, using whatever materials were available at the time. As a result, the receiver now covers the entire 40-meter band and the start of the 41-meter broadcast band. The base of the Audion transitor is no longer at the hot end of the coil but at the center tap that was previously used as the antenna connection A1.

Some distortion is still audible when receiving CW and SSB stations due to changes in the transitor's junction capacitance. Depending on the signal, the operating point changes, and so does the input capacitance. This results in additional frequency modulation and unpleasant audio effects. To reduce this effect, I installed an additional 12 pF capacitor between the base and emitter.

\begin{figure}[h]
\centering
\includegraphics[width=0.8\textwidth]{fig4-31.png}
\caption{The modified schematic.}
\end{figure}

With this modification, the influence of the transitor's base capacitance is reduced. The regeneration is also increased, allowing for larger feedback signals to be set. The transitor then operates more like a direct mixer. When the regeneration signals are significantly larger than the received signals, the operating point shifts less strongly. There is now a larger range of regeneration regulation beyond the oscillation threshold, which can be used to help fine-tune the frequency. This makes exact tuning of an SSB signal much easier.

Another improvement in the signal-to-noise ratio was achieved by reducing the audio bandwidth. To do this, I just soldered a 22 nF capacitor across the volume control. In the end, audio volume was a little lacking but the LM386 still has something in reserve. I placed a 10 µF capacitor in series with 470 Ω resistor between pins 1 and 8. The Audion receiver in this form is suitable for amateur radio use. It's fun to use it to listen to CW transmissions, and the Audion is also well suited as a practice receiver.

\section{Shortwave Regen Optimization}

Is it possible to optimize a transitor Audion so that it can be used for real radio operation on amateur radio bands, as was practiced in the early days of radio? The basic idea is that if a stable and well-shielded oscillating circuit with the highest possible \textit{Q} factor is loosely coupled, it should be possible to build a very stable and sensitive Audion regenerative receiver. My test setup on a section of copper-clad board does not yet meet these requirements but it allows for any other oscillating circuits to be connected easily. At least the continuous ground plane has a positive effect on circuit stability.

\begin{figure}[h]
\centering
\includegraphics[width=0.8\textwidth]{fig4-32.png}
\caption{Prototype layout built on a copper-clad board.}
\end{figure}

Over the years, two Audion regen circuit designs have proven particularly effective so you can take a closer look at them here. The first, simple version is used in the Franzis shortwave radio. A single PNP transitor is responsible for regeneration and demodulation.

\begin{figure}[h]
\centering
\includegraphics[width=0.8\textwidth]{fig4-33.png}
\caption{Audion regen in PNP-Emitter follower configuration.}
\end{figure}

While this circuit works very well with careful layout, it has some disadvantages in terms of easy repeatability for those building it themselves. The transitor is tightly coupled to the circuit, so the feedback regulator affects the frequency. This can also be an advantage if used skillfully for fine-tuning. Overall, the gain of the regeneration circuit is low, so the antenna must only be loosely coupled. Correct adjustment is therefore not easy. Another disadvantage is that the Audion stage has no gain, so an AF amplier stage with a high level of gain is required.

The second proven circuit was also used in the Elektor shortwave Audion. Here, the actual Audion stage operates in emitter configuration which has high gain. Regeneration is taken care of separately using two PNP transistors in a differential amplier configuration.

\begin{figure}[h]
\centering
\includegraphics[width=0.8\textwidth]{fig4-34.png}
\caption{Audion with separate regeneration.}
\end{figure}

I based my first build on this variant. The regeneration stage has so much gain that it practically brings any oscillation circuit to resonance. Therefore, the circuit is suitable for experiments with different oscillation circuits and also for band switching. Unlike the original, I operated everything from 9 V this time and used an LM386 amplier in the final stage. Additionally, there is a 5 V voltage regulator for the Audion.

\begin{figure}[h]
\centering
\includegraphics[width=0.8\textwidth]{fig4-35.png}
\caption{Experimental air-spaced coil with coil tap points.}
\end{figure}

The free-standing coil was wound using one meter of insulated single-strand wire. Although this construction method produces a coil which is mechanically quite loose it is stable enough for our purposes and is easy to replicate at home without any special tools. The tap points can be easily added using a hot soldering iron. The Audion stage is connected to the fifth turn tap and the antenna at the first turn. Because a long wire antenna was used here, it has to be very loosely coupled. Alternatively, a large wire-loop antenna can also be used.

\begin{figure}[h]
\centering
\includegraphics[width=0.8\textwidth]{fig4-36.png}
\caption{Audion stage with separate regeneration circuit.}
\end{figure}

The radio was built on an all copper clad board. The individual components are soldered onto short sections of a strip board. Figure 4.36 shows the NPN Audion stage and the two PNP transistors for the regeneration circuit. The continuous copper surface is useful to promote operational stability and shield against electrical interference.

\begin{figure}[h]
\centering
\includegraphics[width=0.8\textwidth]{fig4-37.png}
\caption{A Loop antenna provides the resonant circuit.}
\end{figure}

This regen can be used to add regeneration to different circuits. The circuit can be connected to the low tap for loose coupling, or to the hot end of the circuit for greater signal voltage. When connecting a long outdoor antenna, tap points low down in the winding are used for both the regeneration and the antenna connections. However, a small loop antenna can be connected directly to the hot end. Both variants were tested with the two circuit halves.

The loop consists of a total of two meters of Litz wire, simply hung over a stool. This provides good reception indoors because the loop is sensitive to the magnetic field component of the electromagnetic signal. Adding signal regeneration to the loop achieves the same effect as using a much larger loop. A croc clip was used for simple hook up to test circuits.

\begin{figure}[h]
\centering
\includegraphics[width=0.8\textwidth]{fig4-38.png}
\caption{Test setup with a loop antenna and loudspeaker.}
\end{figure}

The variable capacitor is a 4-gang version and two of the sections have not been used yet. I wanted to test the stability of an iron powder toroidal core, so I wound some wire around a T50-2 core and added a few tap points to the coil. I also included two ceramic capacitors, each with a capacitance of 56 pF, in parallel with one 20 pF section of the variable capacitor. This 4 gang variable capacitor has 2 × 266 pF tunable sections for use with AM bands and 2 × 20 pF tunable sections for the FM band. On my first attempt, I was able to pick up many radio stations very clearly on the 49-meter band. The total capacitance value ranges between 112 and 132 pF. This gives a capacitance ratio of 1 to 1.18. The frequency ratio is the square root of this, which is 1.085. This gives a tuning range of 500 kHz in the 49-meter band.

\begin{figure}[h]
\centering
\includegraphics[width=0.8\textwidth]{fig4-39.png}
\caption{Toroidal-core coil for 40 m.}
\end{figure}

To be able to work on the 40-meter band, I carefully unwound some turns. This brought me close to the beginning of the band at 7 MHz. The receiver now covers the 40-meter band and part of the 41-meter band. The Audion needs to be connected to the tap at about half the number of turns. Thanks to the band spread, SSB and CW stations can be easily tuned with precision. The frequency stability is quite good, although there is still some sensitivity to hand movements due to the open construction of the circuit. It should be possible to build a useful Audion using a toroidal core for amateur radio use inside a shielded enclosure.

\begin{figure}[h]
\centering
\includegraphics[width=0.8\textwidth]{fig4-40.png}
\caption{Test setup using a JFET Audion.}
\end{figure}

At the end of the series of experiments, I wanted to try using a JFET as well. Previous attempts had not been very successful, and I attributed this to the fact that a JFET's characteristic curve exhibits greater linearity compared to a bipolar transitor. Direct comparisons have indeed shown that a JFET circuit delivers a lower audio signal. For this reason, an additional audio stage was added.

\begin{figure}[h]
\centering
\includegraphics[width=0.8\textwidth]{fig4-41.png}
\caption{The JFET Audion.}
\end{figure}

The JFET is operated with very little source current and only very small capacitance at the input. Nevertheless, the amlication is sufficient for active regeneration.

\begin{figure}[h]
\centering
\includegraphics[width=0.8\textwidth]{fig4-42.png}
\caption{Construction of the active stage.}
\end{figure}

The new Audion circuit was now built on a small section of perf board. The circuit is connected to the desired resonance circuit and volume potentiometer with two crocodile clips. This way makes it easier to switch back to the old bipolar circuit.

The comparison clearly shows that the FET Audion is the better circuit. The frequency stability is significantly higher, and the almost complete lack of frequency dependence on regeneration settings is convincing. For CW and SSB reception, regeneration can be more strongly over driven without losing AF volume. This is where the more linear characteristic curve of the FET is advantageous. Overall, the circuit has a very large dynamic range and can process weak CW stations as well as strong broadcasting stations.

All in all, there's nothing like the good old Audion. Compared to a direct mixer, an Audion is much simpler to build because it only needs a single resonant circuit. Additionally, essential amlication is achieved through the sharply tuned regeneration. The circuit has a quality factor of up to 1000, which simultaneously provides good selection and reduces the risk of intermodulation distortion or strong broadcast stations breaking through. On top of this, the Audion can also receive AM clearly, which is hardly possible with a direct mixer.

\chapter{RF Oscillators}

The quality of a receiver or a transmitter depends heavily on the type of oscillator circuit used. While free-running oscillators with tuning capacitors require a lot of effort to achieve good stability, a quartz crystal oscillator can provide ideal properties but at a fixed frequency. Good stability of a quartz oscillator and tuning capability can be achieved using a PLL circuit. The current state-of-the-art oscillator technology is represented by DDS generators, which provide accuracy of a quartz oscillator together with an adjustable frequency output signal with very fine frequency resolution.

\section{LC Oscillators}

Common basic circuits for RF oscillators have already been described in connection with Audion, as every Audion with regeneration works like an oscillator. Many results with Audion also showed some potential weaknesses of the oscillator. Frequency instability can be a result of external influences such as varying capacitive coupling from hand movements close to the receiver, voltage fluctuations or interference signals. In the interest of high stability, all of these influences must be kept to a minimum. The following factors are crucial:

• A high \textit{Q} resonant circuit
• Low temperature coefficients of the coil and capacitors
• Loose coupling to the oscillator circuit
• Oscillator shielding
• Decoupling and stabilization of the operating voltage
• Load decoupling using buffer stages

\begin{figure}[htbp]
\centering
\includegraphics[width=0.6\textwidth]{fig5-1}
\caption{VFO with Capacitive Coupling}
\end{figure}

Figure 5.1 shows the basic circuit of a stable VFO (Variable Frequency Oscillator) with oscillator stage and buffer amplifier. The first transistor is loosely coupled to the resonant circuit through a capacitive voltage divider. The optimal values for capacitors and resistors depend on the desired frequency range and should be determined by experimentation. A subsequent buffer stage should minimize feedback signal on the resonant circuit.

The long-term stability of a circuit also depends on the temperature behavior of the components that determine the frequency. Styroflex (polystyrene) capacitors are well-suited for this purpose because their temperature coefficient (TC) is close to zero. However, even standard ferrite cores have a certain temperature coefficient. It takes a lot of careful consideration to compensate for temperature induced drift by using ceramic capacitors with opposing TCs.

\begin{figure}[htbp]
\centering
\includegraphics[width=0.6\textwidth]{fig5-2}
\caption{Clapp Oscillator for the 40-m Band}
\end{figure}

Figure 5.2 shows the popular Clapp oscillator circuit often used in amateur radio equipment with component values suitable for use in the 40-meter amateur radio band and the 41-meter broadcasting band. In this form, the oscillator can be used for direct conversion receivers to listen to CW and SSB signals, as well as for digital radio broadcasting (DRM). Mechanical stability and good shielding are important factors to consider.

The Clapp oscillator configuration uses capacitive coupling and has a wide operating frequency range which makes it popular with amateur radio enthusiasts. If you need to cover even larger frequency ranges, you can use a tap point coupling to the resonant circuit coil as shown in Figure 5.3.

\begin{figure}[htbp]
\centering
\includegraphics[width=0.6\textwidth]{fig5-3}
\caption{Stable VFO with Wide Operating Range}
\end{figure}

\section{Crystal Oscillators}

The electrical characteristics of quartz crystal are equivalent to a resonant circuit with an extremely high value of \textit{Q} or quality. Therefore, quartz oscillators with good stability can be easily built without the need for any inductors. Figure 5.4 shows the standard Pierce circuit used to build a crystal oscillator.

\begin{figure}[htbp]
\centering
\includegraphics[width=0.6\textwidth]{fig5-4}
\caption{The Pierce Oscillator}
\end{figure}

A transistor can be used in an oscillator collector circuit using a capacitor voltage divider network. This circuit is used, for example, in the integrated mixer and oscillator IC type NE612. A capacitor trimmer allows for fine tuning and can adjust or 'pull' the quartz frequency by up to about 3 kHz.

\begin{figure}[htbp]
\centering
\includegraphics[width=0.6\textwidth]{fig5-5}
\caption{Quartz Crystal Oscillator Using Common Collector Amplifier}
\end{figure}

Another popular oscillator configuration uses a CMOS gate instead of a transistor. The circuit shown in Figure 5.6 is used in integrated oscillators in microcontrollers, as well as in CMOS circuits such as the oscillator and divider IC 4060 or the faster HC4060.

\begin{figure}[htbp]
\centering
\includegraphics[width=0.6\textwidth]{fig5-6}
\caption{A Crystal Oscillator Using a CMOS Gate}
\end{figure}

For frequencies below 1 MHz, a ceramic resonator is often used instead of a quartz crystal. These components are less expensive than quartz crystals, but they don't achieve the same temperature stability and accuracy.

\section{Amplitude Modulation}

The small medium wave AM transmitter described here can be used to broadcast programs to a nearby medium wave radio. The transmitting coil wound on a ferrite core sends signals which couple directly to the ferrite rod in a radio receiver. The transmitter operating frequency is derived from a 976 kHz ceramic resonator, which can be salvaged, for example, from an old TV remote control unit. Some degree of fine frequency tuning is possible using the 30 pF trimmer. A likely weak station in the background can be tuned to the beat frequency null, for example, at 981 kHz.

\begin{figure}[htbp]
\centering
\includegraphics[width=0.6\textwidth]{fig5-7}
\caption{The AM Modulator}
\end{figure}

The modulator works as an emitter follower and modulates the power amplifier supply voltage. You can only send mono signals on medium wave so both left and right input channels are combined. The potentiometer is used to adjust for the lowest distortion and best sound quality. The RF amplifier stage is designed to operate at low power to limit the signal range.

A waveform showing the amplitude modulated RF signal output can be seen in Figure 5.8.

\begin{figure}[htbp]
\centering
\includegraphics[width=0.6\textwidth]{fig5-8}
\caption{Waveform of the RF Output Signal}
\end{figure}

This medium wave RF modulator can now be placed on top of a regular MW radio receiver. An audio cable hookup to a CD player or any other audio source will now give you one more strong station transmitting on the medium wave band. Not only does this one have particularly good modulation purity but it's also guaranteed to play great tunes!

\section{Crystal-stabilized Medium Wave Modulator}

A medium wave AM modulator should be as stable as possible and, above all, not have any FM components in its output signal. A good solution would be to use a crystal oscillator. Unfortunately, quartz crystals that operate at medium wave frequencies are rare and expensive. To get over this you can use a divide-by-ten circuit. This allows any quartz in the range of 5 MHz to 16 MHz to be used to transmit on medium wave frequencies between 500 kHz and 1.6 MHz.

\begin{figure}[htbp]
\centering
\includegraphics[width=0.6\textwidth]{fig5-9}
\caption{Using a Microcontroller}
\end{figure}

Here an ATtiny25 microcontroller is used as a frequency divider. Its quartz oscillator can run up to 20 MHz. A very short program controls the microcontroller's DDRB.1 register. At B1, there is not a usual CMOS output driver, but only the lower port FET with an open drain output stage. The controller thus forms the oscillator, frequency divider, driver, and transmission power amplifier. The power amplifier could easily deliver up to 30 mA, but it is operated here with greatly reduced power to comply with legal requirements for inductive transmission.

\begin{figure}[htbp]
\centering
\includegraphics[width=0.6\textwidth]{fig5-10}
\caption{Test Build Using a Ferrite-rod Antenna}
\end{figure}

The modulation input can be connected to any headphone output of a CD player, MP3 player, DAB radio, etc. The modulation level can be adjusted using the volume control. The modulation is absolutely pure and distortion-free and can be turned up to a modulation level of 100\% without any problems. The sound quality is comparable to that of a real medium wave transmitter. Instead of the ferrite choke, a wire loop can also be used, which can then be placed near the target radio receiver. Now you can try different quartz crystals from your hobby box to find one that operates on an unoccupied frequency.

The AM modulator is available as a pre-assembled board from AK Modul-Bus and operates at 900 kHz. To operate on any frequency in the medium wave band a crystal socket can be fitted allowing other crystals to be easily swapped out. Long wave operation has also been successfully tested.

\begin{figure}[htbp]
\centering
\includegraphics[width=0.6\textwidth]{fig5-11}
\caption{The AM Modulator}
\end{figure}

A wire loop antenna from the modulator placed directly behind a radio receiver works well. This new station will now be a good substitute for any shut-down local transmitter. In evening, you will be able to tune into your own station broadcasting your own program along with many weaker signals from far away.

\section{The ICS307-2 PLL Clock Generator}

While searching for a possible alternative to the no longer available CY27EE16 PLL chip, the ICS307-2 was discovered. This clock generator is somewhat simpler and offers fewer options. This IC is quite compact and comes packaged in a 16-pin SOIC outline with a pin spacing of 1.27 mm.

\begin{figure}[htbp]
\centering
\includegraphics[width=0.6\textwidth]{fig5-12}
\caption{Testing the ICS207-2}
\end{figure}

The ICS307-2 uses an SPI interface to connect with a PC. This requires three 10 kohm resistors plus a voltage supply of in the range of 3.3 to 5 V.

\begin{figure}[htbp]
\centering
\includegraphics[width=0.6\textwidth]{fig5-13}
\caption{The Clock Generator Schematic}
\end{figure}

A small VB program is available so that you can control the chip using a PC via the serial interface. The output Clk2 can generate frequencies between 2 MHz and 120 MHz, allowing an IQ mixer (see Section 8) to be operated between 500 kHz and 30 MHz. Other values displayed were used for investigation into frequency deviations and settings during program development.

The IC has three internal dividers with relatively little scope, resulting in lower frequency resolution compared to the CY27EE16. Only three bytes are transferred to program chip.

In order to program the correct values into the internal count registers to achieve the desired output frequency takes a little trial and error, by calculating all allowed settings to find the best match. This results in either the exact desired frequency or a neighboring frequency within 1 kHz or, at a few critical points, within 5 kHz to be achieved.

\begin{figure}[htbp]
\centering
\includegraphics[width=0.6\textwidth]{fig5-14}
\caption{Control Software in VB}
\end{figure}

AK Modul-Bus offers a preassembled board for the ICS307-2, which also includes the required software. All connections to the PCB are via screw terminal blocks. The board can be used as a general-purpose clock oscillator for microcontroller applications and digital electronics, as well as for high-frequency applications. The clock generator can also be used as an inexpensive VFO for shortwave receivers.

\begin{figure}[htbp]
\centering
\includegraphics[width=0.6\textwidth]{fig5-15}
\caption{The Populated PCB}
\end{figure}

\section{A Programable Crystal Oscillator}

If you ever need a quartz crystal for a very specific frequency, it's often not available. Some frequencies can't be purchased at all. Custom-made crystals are always an option, but they are very expensive as a one-off. What is required in this case is a programmable quartz oscillator. The CY27EE16 chip is ideal for this application and was used, for example, in the Elektor-SDR (Section 8.6). Unfortunately, this chip is no longer manufactured. Modul-Bus produce a programmable clock oscillator board using the ICS307-2 chip which outputs a frequency in the range from 2 to 120 MHz. This chip, however, lacks the ability to store a selected frequency. All you need to remedy this shortfall is to add a tiny low-cost microcontroller like the ATiny13 which can be used to store a setting and use it on the next power-up. That will now give you a usable and programmable quartz generator.

\begin{figure}[htbp]
\centering
\includegraphics[width=0.6\textwidth]{fig5-16}
\caption{Controlled by an ATiny13}
\end{figure}

The ATiny13 here directly generates the three interface signals Data, SCLK, and Strobe of the clock generator PCB. In our initial test, the DB9 plug was used, i.e., the port lines directly replace the corresponding output lines of the RS232. Now, the ATiny13 must send exactly the data that was previously supplied by the PC. Altogether, the frequency required is contained in three control bytes, giving 24 bits in total. These bytes are either directly received and clocked into the generator or loaded into the EEPROM of the ATiny13, where they are read out at the next start.

Data transfer to the clock oscillator PCB is performed by the handy Bascom command Shiftout to transfer the three relevant bytes. After that, a strobe pulse is generated with Pulseout. A software UART running at 9600 baud takes care of data reception. A command word is required to be prefixed to the three data bytes B1, B2, and B3. The command word tells the processor what to do with the 3 data bytes. The value 65 indicates the data is for direct output while 66 indicates it's for storage in the EEPROM. These three bytes stored at addresses 10, 11, and 12 in the EEPROM are read at each restart and clocked into oscillator chip.

\begin{figure}[htbp]
\centering
\includegraphics[width=0.6\textwidth]{fig5-17}
\caption{Setting Up the Desired Frequency}
\end{figure}

The original program for setting the output frequency now needs to be modified to send four bytes over the RS232 interface. The example shown here sets a frequency of 27.12 MHz, for which the three control bytes 60, 165, and 176 were determined. The frequency can be set using the slider control or by entering the desired frequency into the field and transferring it to the slider via the Set button. The VB software and ATiny13 firmware are available on the author's website.

\begin{figure}[htbp]
\centering
\includegraphics[width=0.6\textwidth]{fig5-18}
\caption{The ATiny13 and Crystal}
\end{figure}

In order to use the programmable quartz oscillator in the same way as regular quartz oscillators with four pins in a DIP14 package, it was placed on a small piece of double-sided perf board (a.k.a. stripboard or Veroboard). The standard pin assignment was adopted, but now the RXD input is connected to Pin 1. Only this one pin is needed to reprogram generator. The microcontroller is located on the top side, and the PLL chip is on the bottom side of the board.

\section{CW Transmitter with an EL95}

You may wonder what sort of rigs amateur radio enthusiasts were using back in the 1950s when vacuum tubes were the order of the day. Well, many radio amateur rigs at that time were made up of a tube Audion type regen receiver (0V2) and a small transmitter, with an EL84 tube in the final output stage. I built something similar to that myself. By the time I sat my license exam, I had already skipped a few steps beyond the basic receivers and transmitters. As a result, I never actually got to use a 0V2 and a small tube transmitter. To make up for the gap in my ham radio apprenticeship I thought it was about time I took a look at the design, just to see how well it performs.

\begin{figure}[htbp]
\centering
\includegraphics[width=0.6\textwidth]{fig5-19}
\caption{The Homebrew Tube Transmitter}
\end{figure}

At first, I thought it would be easy to modify an old tube radio for that purpose. A small Grundig radio was already on my radar, but the set still works so beautifully I couldn't bring myself to cannibalize it. I then decided to build the necessary power supply myself. I already had two suitable transformers. By connecting the two secondary windings together the first transformer gets the mains voltage down to 12 V and the second produces an output of about 200 V at its primary winding to use for the HT supply. Using this method, you don't need to find a special transformer and everything can be installed in the Franzis shortwave radio enclosure.

\begin{figure}[htbp]
\centering
\includegraphics[width=0.6\textwidth]{fig5-20}
\caption{The HT Voltage Supply}
\end{figure}

A prototyping test board was used to build the tube stage. The board is designed for seven-pin miniature sockets which can accept a 6AQ5A (= EL90). Despite its small size this tube can handle an anode power dissipation of 12 W, just like the much larger EL84. With this tube, I was able to generate up to 5 W at 3.5 MHz. The EL95 is pin-compatible and delivers 3.5 W using the same circuit configuration.

\begin{figure}[htbp]
\centering
\includegraphics[width=0.6\textwidth]{fig5-21}
\caption{The Output Stage}
\end{figure}

The complete crystal-stabilized transmitter circuit shown above uses cathode keying. The carrier signal which I set to 3560 kHz is generated by the programmable crystal generator described in the last section. When 5 V is applied to VCC, the RF signal is generated at the output. The tuned grid circuit boosts the signal to about 20 Vpp.

A small neon lamp indicates the anode HT voltage. An LED lights when the toggle switch is set to transmit. The relay then pulls in and applies 5 V to the quartz oscillator. During transmission, a DC voltage is also applied to the antenna output which activates an external relay to switch the transmitting antenna. A DC voltage is also applied to the antenna cable to the receiver, which is used for signal muting.

\begin{figure}[htbp]
\centering
\includegraphics[width=0.6\textwidth]{fig5-22}
\caption{Schematic of the CW Transmitter}
\end{figure}

The coil in the anode circuit is made up of 20 turns on a toroidal ferrite core with a secondary winding of 3 turns to ensure antenna matching and necessary isolation from the anode HT voltage. An additional Pi-filter at the output is for matching and provides further harmonic suppression. This is useful when other frequency bands are used.

\begin{figure}[htbp]
\centering
\includegraphics[width=0.6\textwidth]{fig5-23}
\caption{The Transmitter in Operation}
\end{figure}

A small 5 V/0.4 A filament lamp (vintage bicycle headlamp) primarily serves to dissipate some of the heat generated inside the case. The transformer outputs 12 V, which means the voltage regulator needs to dissipate a lot of power to provide 6.3 V at 450 mA for the tube filament.

\section{AM Tube Transmitter}

This small medium wave transmitter uses a tunable oscillator with a ferrite antenna. Two EF95 type HF pentodes are used in this design. One tube acts as a tunable oscillator, other as a modulation amplifier. To use a simple plug-in power supply, both tube heaters were connected in series to 12 V and the anode HT voltage was also limited to 12 V.

\begin{figure}[htbp]
\centering
\includegraphics[width=0.6\textwidth]{fig5-24}
\caption{A tunable medium wave transmitter.}
\end{figure}

This free-running oscillator is modulated via the screen grid. The preceding modulation amplifier operates in triode mode to achieve a sufficiently large, distortion-free modulation despite the low anode voltage. The operating point is adjusted with a trimmer to minimize distortion. The characteristic curves of both stages are oppositely curved due to the phase shift of the modulation amplifier. With an optimal adjustment, the resulting distortion is largely cancelled out, allowing for a large modulation range of up to about 50\%. There is a stereo jack at the input where, for example, a PC sound card output can be connected as a modulation source. Both channels are combined to provide a mono signal to the transmitter.

The resonant circuit coil is wound on a ferrite rod which also serves as the transmitting antenna. A nearby radio should be able to tune into the signal. A wire antenna could also be used as an alternative but this increases the risk of the transmitted signal being picked up by others in the neighborhood.

\begin{figure}[htbp]
\centering
\includegraphics[width=0.6\textwidth]{fig5-25}
\caption{Construction of the medium wave AM transmitter.}
\end{figure}

Figure 5.25 shows the transmitter built using the tube experimentation system RT100 from AK Modul-Bus. This system contains all the necessary connections, a variable capacitor, and the required potentiometer. All connections, resistors, and capacitors have been connected up on the plug board area.

\section{A DDS Generator using the AD9835}

DDS oscillators meet the highest demands in terms of frequency stability and noise immunity. They are also much easier to build than PLL VFOs or free-running oscillators, and unlike a PLL, they allow almost any frequency to be synthesized with fractional-Hertz resolution.

\begin{figure}[htbp]
\centering
\includegraphics[width=0.6\textwidth]{fig5-26}
\caption{Block diagram of the AD9835.}
\end{figure}

The term DDS stands for 'Direct Digital Synthesis' and describes the digital generation of a repetitive waveform at a defined frequency. To generate a sine wave the core of a DDS oscillator uses a table of sine waveform values stored in ROM. A DAC takes the sine wave values and converts them into an output voltage. A phase accumulator serves as an address pointer to the sine table and keeps track of the instantaneous phase of the waveform. The desired output frequency is represented by a digital value called the frequency tuning word. This value determines the rate at which the phase accumulator increments to give the desired output frequency.

The quality of the output signal depends on the resolution of the DAC and the size of the sine table. The AD9835 uses a 10-bit converter and a sine table with 4096 support values. The phase accumulator has a width of 32 bits, with only the upper 12 bits determining the address of the current output value.

\begin{figure}[htbp]
\centering
\includegraphics[width=0.6\textwidth]{fig5-27}
\caption{Programmable Oscillator with serial interface.}
\end{figure}

Figure 5.27 shows the complete circuit diagram of the DDS generator with power supply and interface. A 7805 type voltage regulator provides 5 V for the AD9835 chip from the 9 V input supply to the board. The DDS requires a 50 MHz clock signal which is generated by an integrated quartz oscillator IC3. The remaining circuitry is mainly limited to supply voltage bypass capacitors and two resistors. R1 is located at pin FSADJUST and determines the output current IOUT at pin 14. This current produces a voltage drop across resistor R2. Here you can find the sinusoidal output voltage, superimposed upon a DC offset voltage. A simple low-pass pi filter attenuates frequency components above 22 MHz.

\begin{figure}[htbp]
\centering
\includegraphics[width=0.6\textwidth]{fig5-28}
\caption{The fully populated DDS oscillator PCB.}
\end{figure}

The DDS chip is only available in the TSPOP outline with a pin spacing of 0.65 mm. This tiny outline is necessary to achieve sufficiently short signal line lengths and good decoupling of the power supply. SMD assembly is not very easy and requires a certain level of skill. To smooth over any possible hassle, a ready-assembled DDS PCB is available from the company AK Modul-Bus.

\begin{figure}[htbp]
\centering
\includegraphics[width=0.6\textwidth]{fig5-29}
\caption{Control software and COM1 to COM4 port selection.}
\end{figure}

Figure 5.29 shows a Visual Basic program from the manufacturer's website for controlling the DDS oscillator. It can be easily adapted and expanded for your own application requirements. The user interface allows you to define the output frequency between 0 and 24 MHz with a step size of 1 kHz and additional fine tuning in 100 Hz and 10 Hz steps. If required, an offset of 455 kHz can be chosen so that the frequency display shows the receiving frequency in a superhet with a 455 kHz IF. In addition, there are frequency sweep functions which are particularly useful for making RF measurements and displaying filter characteristics.

\section{The SI5351 PLL}

The SI5351 uses a 25 MHz crystal oscillator and contains two PLLs that can operate at a frequency between 600 and 900 MHz. The PLL dividers are used to multiply the input frequencies to a high frequency intermediate clock while the second stage of synthesis uses high resolution MultiSynth fractional dividers to generate the desired output frequency. This provides two options for generating the desired frequency: The PLL can be set to a fixed frequency, for example, 900 MHz, and then divided down using fractional numbers. Alternatively, the PLL can be adjusted in small steps and then divided down using integer values to generate the final frequency.

\begin{figure}[htbp]
\centering
\includegraphics[width=0.6\textwidth]{fig5-30}
\caption{Block diagram of the SI5351.}
\end{figure}

The original software-defined radio for shortwave up to 30 MHz (see Section 8.6) was an interesting project, but there were issues sourcing an alternative for the discontinued PLL chip. Then the SI5351 clock chip from Silicon Labs came along. The Adafruit breakout board was used for the initial experiments. There is also a useful Arduino library to support it.

\begin{figure}[htbp]
\centering
\includegraphics[width=0.6\textwidth]{fig5-31}
\caption{The Adafruit SI5351 board.}
\end{figure}

With all this help and an Arduino Uno, it was possible to build on the Elektor SDR project and expand its capabilities. The SI5351 is now a key component in the Elektor SDR Shield (Section 8.7). Thanks to its excellent characteristics, numerous possibilities are now available, including digital data transmission, for example, with a WSPR transmitter or for HF measurement applications.

\begin{figure}[htbp]
\centering
\includegraphics[width=0.6\textwidth]{fig5-32}
\caption{The SI5351 (IC1) mounted on the Elektor SDR shield.}
\end{figure}

It's not necessary to always use an Arduino board to control the SI5351 it can of course be interfaced to much smaller microcontroller. In fact, Andrew Woodfield, ZL2PD, has produced a program written in Bascom for the ATtiny85. Using his code, it is possible to use two of the PLL outputs of the SI5351 simultaneously by driving both internal PLLs. Any slight change in the frequency is tracked by the PLL while the divide-ratio settings of the following dividers remain unchanged. This method promises to keep phase noise to a minimum. The source code for this project can be found on the author's website.

\begin{figure}[htbp]
\centering
\includegraphics[width=0.6\textwidth]{fig5-33}
\caption{Controlled by a ATiny85.}
\end{figure}

\chapter{Direct Mixers}

Direct mixers consist of an oscillator and a mixer and convert an RF input signal to a baseband (usually audio) signal in one step. They are suitable for receiving SSB and CW broadcasts as well as digital broadcasts such as DRM. Compared to a regenerative receiver mixer achieves better frequency stability. An oscillator with low phase noise is important for error-free reception of DRM broadcasts.

\section{Mixer Types}

Mixer circuits convert input frequencies to other frequency ranges. In a superheterodyne receiver, the mixer converts the received signal to an intermediate frequency (IF) (see Figure 6.1). In contrast, a direct-conversion mixer or zero-IF mixer converts the signal directly to audio frequency (AF) range. Such receivers are used, for example, in simple amateur radio applications. For DRM reception, the signal is converted to the 12 kHz range, and it is only a question of definition whether to call the output signal an AF or an IF.

\begin{figure}[htbp]
\centering
\includegraphics[width=0.6\textwidth]{fig6-1}
\caption{Block diagram of a medium wave superhet radio.}
\end{figure}

A direct mixer directly converts the received signal to the audio frequency range, without the use of an intermediate frequency. It consists of an oscillator and a mixer and is commonly used in simple amateur radio applications. Due to the finite Q factor of the resonant circuit, image frequency suppression is only possible at low frequencies (long wave).

In this text, mixer types are described in connection to their use in a superheterodyne type of receiver with an IF amplifier. However, the same circuit can also be used as a direct mixer by tapping off the baseband signal instead of the intermediate frequency.

One particularly simple and commonly used mixer is the multiplying mixer stage which uses a single transistor. When two sinusoidal signals are applied to a multiplier, the output signal contains the sum and difference frequencies of the two input signals, in addition to the two original signals. In this case, multiplication means that the gain of one signal is directly controlled by the instantaneous value of the second signal. A variable gain amplifier is therefore necessary, the gain of which can be adjusted directly by an oscillator signal.

A single transistor can be used to mix multiple frequencies when biased at the appropriate operating point and both signals are applied to the base. This method of signal processing is also known as an additive mixer. Considering the gain of the transistor as the product of its transconductance and external resistance and noting that the transconductance is proportional to the collector current, a change in the collector current due to the oscillator signal is sufficient to multiply the input signal with the oscillator signal. The input signal should be small enough so that it only operates over a narrow region of the transconductance slope, i.e., being kept below 1 mV. The oscillator signal, on the other hand, should modulate the collector current as linearly as possible and without overloading it.

\begin{figure}[htbp]
\centering
\includegraphics[width=0.6\textwidth]{fig6-2}
\caption{A slope multiplier used as a mixer.}
\end{figure}

Figure 6.2 shows a simple mixer using an NPN transistor in a Superheterodyne application. The low impedance oscillator signal is coupled to the emitter and directly modulates the collector current. The current feedback using emitter coupling ensures good linearity. The input signal is fed directly to the transistor base. This signal must be relatively small to ensure the transistor is not driven beyond its narrow effective operating region. The collector current therefore contains the mixed down converted signal, from which the intermediate frequency can be filtered out to recover the desired baseband signal. The mixer operating point should be stabilized to ensure consistent results.

In principle, both signals could also be applied to the transistor base. Figure 6.3 shows a simplified version of the mixer, with the operating point stabilization omitted for clarity of principle. It's important that the oscillator signal is large enough to modulate the collector current and the input signal is small enough not to create distortion. In simple receivers, self-oscillating mixer stages are often used, where the oscillator transistor also serves as the mixer. A regenerative transistor stage can also be regarded as a mixer of this type.

\begin{figure}[htbp]
\centering
\includegraphics[width=0.6\textwidth]{fig6-3}
\caption{A simplified mixer.}
\end{figure}

The simple mixer circuit shown in Figure 6.3 is practically no different from the basic circuit for a single transistor emitter amplifier. This means that practically any amplifier can also become a mixer if you supply it with two signals of different frequencies and appropriate signal amplitudes. This also poses a danger because most of the time, the frequency mix needs to undergo further processing. The input amplifier in a shortwave receiver often has to deal with very strong signals alongside very weak signals. It can happen that one of the stronger signals acts like an oscillator signal and generates mixing products. This phenomenon is called intermodulation or cross-modulation. This generates numerous interference signals that disturb the reception of weak signals.

The same applies to the transistor in a mixing stage such as in Figure 6.2. In addition to the oscillator signal, other high levels input signals can also lead to mixing products. The interference immunity of such a mixing stage is therefore not particularly high. They are still used in simple radio receivers where high sensitivity or immunity to high input signals are not important properties. For better performance more sophisticated mixers must be used. A shortwave receiver, as used in amateur radio applications, should be able to cope with signals in the sub 1 µV range as well as much stronger signals of up to 100 mV without distortion. This wide dynamic range is only possible if the mixer characteristics are extremely linear for the entire input signal range.

Dual-gate field-effect transistors (DG-MOSFETs) such as the BF961 can be used to build a mixer with a good dynamic range. Used as amplifiers, they have a linear characteristic curve and therefore produce low distortion. The steepness of the transistor characteristic can be modulated via the second gate. Using appropriate adjustment of the gate voltages and a suitable oscillator level, good dynamic range is achievable. Dual-gate MOSFETs are commonly used in shortwave and FM receivers mainly because they lead to reduced circuit complexity.

\begin{figure}[htbp]
\centering
\includegraphics[width=0.6\textwidth]{fig6-4}
\caption{Mixer stage using a dual gate MOSFET.}
\end{figure}

A single diode can also be used as a mixer. In the circuit shown in Figure 6.5, both signals are first added together. The envelope of the resulting mixed signal contains the desired mixing products. A rectifier is all you need to extract them.

\begin{figure}[htbp]
\centering
\includegraphics[width=0.6\textwidth]{fig6-5}
\caption{A diode mixer.}
\end{figure}

An additive mixer has good ability to handle large signals. One disadvantage is that more than two frequency products are created in the mixing process. The oscillator signal is now no longer a sinusoid but appears as a square wave because of the diode switching characteristics. The signal now includes odd harmonics of the oscillator signal i.e., 3 fOSC, 5 fOSC, etc. Each of these harmonics also generates corresponding mixed products. For a demodulator, this is not significant, as signals with multiples of the carrier frequency are far enough away from the wanted baseband signal so that they can be removed with a low-pass filter that will be required anyway. At the receiver front-end they can be suppressed by filters.

In principle, a transistor or field-effect transistor such as the BF245 can also be used as the active element in a mixer. A JFET has the advantage of very fast switching times and lower levels of distortion. In the circuit shown in Figure 6.6, a negative bias voltage for the field-effect transistor is automatically established. The FET acts like a switch that repeatedly shorts the input signal. This type of large-signal mixer is used, for example, as a second mixer in the DRM receiver covered in Section 7.4.

\begin{figure}[htbp]
\centering
\includegraphics[width=0.6\textwidth]{fig6-6}
\caption{A JFET mixer.}
\end{figure}

A special form of diode mixer is the fully symmetrical ring mixer using four diodes. The diodes work as switches that are controlled in sync with the oscillator frequency. Schottky diodes are usually used because they have particularly fast switching times and generate low distortion. At the output of the ring mixer, neither the oscillator signal nor the input signal appears if the broadband transformers and the diodes are well balanced. A ring mixer is used in the input of the DRM receiver covered in Section 7.4 and in the direct mixer in Section 6.4.

\begin{figure}[htbp]
\centering
\includegraphics[width=0.6\textwidth]{fig6-7}
\caption{A wideband diode mixer.}
\end{figure}

Integrated mixers often work as fully symmetric multipliers. One simple mixer with a built-in oscillator is the NE612. This IC requires only minimal external circuitry and works at frequencies up to 300 MHz. It is suitable for battery operation and works with a supply voltage between 4.5 and 8.5 V.

\begin{figure}[htbp]
\centering
\includegraphics[width=0.6\textwidth]{fig6-8}
\caption{The NE612 with internal oscillator.}
\end{figure}

\begin{figure}[htbp]
\centering
\includegraphics[width=0.6\textwidth]{fig6-9}
\caption{Block diagram of the NE612.}
\end{figure}

The internal circuitry of the NE612 provides all the necessary bias voltages, so input signals can be AC coupled with capacitors. Collector resistors are also built in. The mixer has an input and output impedance of 1.5 kΩ and is suitable for balanced or unbalanced operation (see Figure 6.10). The internal oscillator can be configured for an external crystal, a tuned tank network or as a buffer to an external local oscillator.

\begin{figure}[htbp]
\centering
\includegraphics[width=0.6\textwidth]{fig6-10}
\caption{The NE612 used as a symmetrical and unsymmetrical mixer.}
\end{figure}

\section{Direct Mixer using a BF245}

A low-cost, home-brew direct mixer design often relies on a freely tunable oscillator to avoid the cost of a custom crystal or professional VFO. Ideally a direct mixer should have the largest possible dynamic range to allow for the use of a better antenna in weak reception conditions. Experience with simple regenerative receivers has shown that the use of a long antenna can lead to receiver front end overload and generation of intermodulation products. A passive mixer without any mixing amplification inserted directly behind the receiver front end can provide a solution if it has good interference rejection.

A JFET mixer is particularly simple and exhibits good large signal stability. If the FET is operated without a DC voltage, it essentially operates as a controlled resistor. An ideal passive resistor has a linear characteristic that does not generate any signal distortion. The FET comes relatively close to this ideal. Therefore, even with less than optimal control, an FET is a very good switch that can handle relatively large signals without producing intermodulation products in a mixer configuration.

\begin{figure}[htbp]
\centering
\includegraphics[width=0.6\textwidth]{fig6-11}
\caption{The FET Direct Mixer.}
\end{figure}

The circuit shown in Figure 6.11 was designed for use across the entire shortwave range from 5.8 MHz upwards. The oscillator and input circuit use separate tuning capacitors, so the circuits do not have to be adjusted for optimal synchronization. If necessary, the tuning range can be narrowed down to a specific band to allow for finer tuning. The oscillator coil has taps at around 20\% and 50\% of the total number of winding turns. A relatively loose coupling with the transistor results in good frequency stability.

An oscillator signal of approximately 1 to 2 Vpp is sufficient at the gate of the BF245. The additional gate resistor prevents excessive damping of the oscillator network when the JFET input diode enters a conductive state. The FET acts as a switch that provides a short to ground in sync with the oscillator signal for the received signal from the coupled coil. This produces the audio signal at the output filter. The coupling coil has approximately 20\% of the number of turns at the antenna resonant coil. By changing this transformer ratio, an optimal match can be found, with a small coupling factor resulting in a narrow input circuit bandwidth and improved noise immunity.

The FET mixer offers good large-signal immunity but intermodulation products could be generated further down the line in the audio stage. A low-pass filter is therefore inserted to reduce the signal bandwidth. Signals above 20 kHz are now sufficiently attenuated, so they will not generate intermodulation products. The circuit was mainly used for DRM reception in the 49 and 41-m bands. It showed no evidence of intermodulation effects even at high field strengths using a long antenna. This allows for the use of an outdoor antenna which suffers less from the effects of general domestic electrical interference.

The passive FET mixer does not provide any amplification on its own. The resulting audio signals are therefore way down in the microvolt range. A single audio stage boosts the signal level high enough for the microphone input of a PC sound card. Its overall sensitivity depends mainly on the antenna properties. The circuit has very effective selectivity, allowing the reception of weak amateur radio signals in the 40-m band even in the immediate vicinity of high power broadcast signals.

\section{Diode Ring Mixer}

In this test setup, a DDS signal generator is used to provide the local oscillator signal to a 4-diode ring mixer. The direct mixer uses a Mini Circuits TDM2 diode ring mixer and TUF-1 can also be used in its place. The DDS oscillator output signal level at 0 dBm was raised by about 7 dBm using a BF494 transistor as a broadband amplifier. The baseband signal recovered by the mixer is amplified by the low-noise B548C AF stage at the mixer output. This provides sufficient signal level to directly drive the microphone input of a PC sound card even with a poor RF. The mixer is terminated with about 50 Ω, resulting in good large-signal stability. For best reception a long length of wire setup high outdoors can be used for the antenna. If that is impractical a one to three meters length of wire stretched out around the room can also do the job. There are no tuned resonant circuits or RF preamplifiers used in this design so construction is not particularly critical.

\begin{figure}[htbp]
\centering
\includegraphics[width=0.6\textwidth]{fig6-12}
\caption{Direct mixer using 50 Ω termination.}
\end{figure}

A ready-made Schottky diode ring mixer is not cheap but you can easily build one yourself. The mixer consists of four identical Schottky diodes type BAR28 and two broadband transformers. The RF transformers are wound on Amidon FT37-77 ferrite toroidal cores. The primary winding consists of a trifilar winding of 10 turns of 0.2 mm CuL wires wound through the toroid. To make this, three lengths wires are laid alongside each other and wound through the core ten times.

\begin{figure}[htbp]
\centering
\includegraphics[width=0.6\textwidth]{fig6-13}
\caption{Windings on the ferrite toroidal cores.}
\end{figure}

Then ends of the coils can now be scraped clean and tinned with solder. A continuity tester can be used to identify the ends of the three coils. Two of the windings are connected in series to form the two-phase winding for connection to the diodes. The third winding forms the oscillator or signal input. The four diodes can now be soldered on correctly and mixer will be ready to go. Results from testing the homemade mixer in the circuit shown in Figure 6.12 indicates that its performance is not too far away from commercially produced ones.

\section{Direct Mixer using an NE612}

A widely used and inexpensive integrated mixer is the NE612. This IC contains an internal oscillator and a fully symmetrical mixer. If an external oscillator is used, the IC must be driven at Pin 6 with a signal level from 200 mVpp to a maximum of 300 mVpp.

\begin{figure}[htbp]
\centering
\includegraphics[width=0.6\textwidth]{fig6-14}
\caption{External control from a DDS-Generator.}
\end{figure}

This receiver design operates without any input selector and uses a long wire antenna. At the input, there is only a small, fixed value inductor. Using a long wire antenna, DRM signals can be received with an SNR of up to 20 dB. The mixer used in this receiver, however, is not quite as good as a diode mixer, possibly due to its poorer performance with large signals. Overloading the mixer input will generate intermodulation products, which then interfere with the DRM signal. One advantage of the NE612 mixer is its mixer gain, which gives it greater sensitivity. The NE612 circuit has better performance than a diode mixer when using short whip antennas and with low RF signal levels.

\begin{figure}[htbp]
\centering
\includegraphics[width=0.6\textwidth]{fig6-15}
\caption{An integrated direct mixer.}
\end{figure}

Figure 6.15 shows a freely-tunable receiver for the 40-meter band. The internal oscillator of the NE612 exhibits good stability and low phase noise, even when operating without a tuned circuit. The tuned input circuit provides enough preselection to avoid overload from strong signals in other bands.

Using a coil of approximately 20 turns wound on an 8 mm diameter coil former, both 6 MHz and 7 MHz can be received by adjusting the screw-in ferrite slug. A small FM tuning capacitor with triple reduction drive provides sensitive tuning control. The input circuit is not critical and can also be tuned by soldering a fixed capacitor, such as a 120 pF capacitor across the adjustable coil.

\begin{figure}[htbp]
\centering
\includegraphics[width=0.6\textwidth]{fig6-16}
\caption{Installed in the enclosure of the VHF tuning capacitor.}
\end{figure}

The shielded enclosure from an old FM tuner with a 3-gang tuning capacitor was used for the test setup. With the shielding lid closed, it provides impressive isolation from external influences. Using Styroflex capacitors in the oscillator resonant circuit provides good frequency stability and easy tuning. Once tuned, the receiver remains on station for many hours.

An easier alternative to manual tuning is to use a direct mixer with a quartz crystal. This will of course limit the number stations you can receive to just one but stability is no longer a problem. The circuit in Figure 6.17 uses a standard 6 MHz quartz crystal that can be pulled to 6002 kHz using the 20 pF trimmer capacitor. This allows reception of DRM-RTL 2 on 5990 kHz with an inverted spectrum. The circuit was featured in the German magazine Funkamateur in April 2004 as a PCB project using a special 6107 kHz quartz crystal to receive transmissions on 6095 kHz. Since then, the station has shut down. You can still, however, receive AM stations using the appropriate SDR software or switch to the 40-meter amateur radio band by using a different value quartz crystal.

\begin{figure}[htbp]
\centering
\includegraphics[width=0.6\textwidth]{fig6-17}
\caption{A fixed-frequency receiver for 5990 kHz.}
\end{figure}

\chapter{The AM-Superheterodyne Receiver}

Regenerative and direct conversion receivers are now really only of interest as hobby projects, almost all radios you can buy today work according to the superheterodyne principle. A superheterodyne achieves better selectivity and also allows for the use of automatic gain control (AGC). Here you will take a closer look at the technology and build some simple superhet receiver designs.

\section{An AM Shortwave Receiver using the TCA440}

The TCA440 integrated circuit simplifies the construction of an AM superhet receiver, as all stages are combined in one IC. Figure 7.1 shows the typical structure of a freely tunable receiver using a ceramic filter in the IF stage. The IC contains a regulated front-end and a regulated IF amplifier. The control voltage is obtained via a separate diode at the output of the IF amplifier. The received signal strength can also be displayed on an S-meter. A suitable audio amplifier stage is still required at the audio output.

\begin{figure}[htbp]
\centering
\includegraphics[width=0.6\textwidth]{fig7-1}
\caption{Shortwave Superhet using the TCA440.}
\end{figure}

Building and tuning the oscillator and input circuits is the most challenging task when building a superhet. Both circuits must be finely tuned around the intermediate frequency with good coherence.

Figure 7.2 shows a variant of the receiver using varicap diodes for tuning. A potentiometer or an external PLL such as the SAA1057 can be used to provide the tuning control voltage. An additional second mixer allows the receiver to be used as a DRM receiver.

\begin{figure}[htbp]
\centering
\includegraphics[width=0.6\textwidth]{fig7-2}
\caption{A receiver with PLL tuning}
\end{figure}

The RF coils used come from IF filters for 10.7 MHz and each has an additional 10-turn coupler winding. The frequency range and coupling coefficient were determined experimentally. Both circuits are tuned using a BB204 dual varicap diode. The oscillator circuit is adjusted via the coil's screw-in slug which allows the PLL to lock in to the desired frequency range. The input circuit is then adjusted for maximum output signal amplitude. The circuit has a high Q and provides adequate image frequency rejection. Good synchronization of both circuits is achieved without difficulty in the range of 5.8 to 7.5 MHz if the input circuit is tuned precisely in the middle range.

The IF filters use ready-made coil filters for 455 kHz and the ceramic filter CFW455F. The second oscillator is controlled by a CSB470 low cost ceramic resonator. Using the component values shown, the frequency is pulled by 3 kHz to 467 kHz. The oscillator is sufficiently stable and precise to within 1 kHz without the need for any special adjustment.

The demodulator uses a simple additive mixer using a germanium diode, which works well here because the IF voltage at the output of the TCA440 is already at around 1 V. For distortion-free demodulation, it is important that the oscillator amplitude is greater than the signal amplitude at the coupling coil. An advantage of this simple demodulator is that it also works seamlessly for AM when the supply voltage to the second oscillator is switched off.

The TCA440 has significant overall gain, requiring the use of automatic gain control (AGC). The control voltage is obtained using a second germanium diode, which coupled to the output circuit via a 22 kΩ resistor, so that it does not load the signal and give rise to any distortion. This also results in a softer AGC characteristic so that a single interfering noise spike will not immediately reduce the gain. This slow reaction is more favorable for DRM reception than a fast AGC response.

\section{AM/FM Radio using the CD2003GP}

While searching for highly integrated radio ICs for AM and FM bands I found the CD2003GP chip. This IC can often be found inside domestic radios such as bedside clock radio receivers. It is also available from Modul-Bus.

\begin{figure}[htbp]
\centering
\includegraphics[width=0.6\textwidth]{fig7-3}
\caption{Block diagram of the CD2003.}
\end{figure}

A test circuit diagram shows the basic application. Interestingly, the design completely dispenses with coil filters in the IF stage, making adjustment very easy. Selection depends solely on the ceramic filters.

\begin{figure}[htbp]
\centering
\includegraphics[width=0.6\textwidth]{fig7-4}
\caption{External components to build an FM/AM radio using the CD2003.}
\end{figure}

The very simple external circuitry uses coils without the need for any tap points. This makes the IC an ideal component for personal experimentation. Figure 7.5 shows the construction of a complete AM/FM receiver. The input circuit for medium wave uses a ferrite rod antenna, and the oscillator circuit uses a fixed 100 µH inductor. For FM, free standing air-cored coils are used, which can be tuned by slightly stretching or squashing the turns together. The concept allows for improvements to the IF filters. In this case, two AM filters are used with an intermediate circuit and a fixed inductor.

\begin{figure}[htbp]
\centering
\includegraphics[width=0.6\textwidth]{fig7-5}
\caption{A complete AM/FM Radio.}
\end{figure}

With the circuit constructed, everything can be fitted into the case of a retro radio. An LM386 amplifier drives a large speaker, resulting in very good sound quality. The ALC information was taken from pin 5 of the radio IC via an emitter follower stage to display on the S meter. This helps find the optimal tuning point.

\begin{figure}[htbp]
\centering
\includegraphics[width=0.6\textwidth]{fig7-6}
\caption{Installed in a case.}
\end{figure}

\section{DRM Receiver}

In the 3/2004 issue of Elektor Magazine, a DRM superhet receiver (using a 455 kHz IF) was described and proved to be a good introduction for many readers interested in digital radio systems. Unlike an IQ or zero-IF receiver this design uses a steep IF filter. The 12 kHz output signal from the receiver is fed to a PC via a mono audio input channel. An older laptop with just a microphone input may be suitable machine to run the necessary software.

The main goal of the development of this DRM receiver was to build a radio with good reception performance that does not require any adjustments. No special coils or tuning capacitors are needed in this design, only readily available fixed inductors. This is a bonus for those who feel more at home with the black and white world of digital electronics rather than the dark arts of RF technology. Altogether the design works with no tweaking, no special measuring equipment, just a very simple software adjustment to compensate for tolerances in the oscillator frequencies.

\begin{figure}[htbp]
\centering
\includegraphics[width=0.6\textwidth]{fig7-7}
\caption{The finished receiver.}
\end{figure}

Basically, the receiver can be seen as a DRM interface for the PC. As shown in Figure 7.8, the DRM receiver has two connections to the computer: the first is via the RS232 interface, where the receiver inputs digital control information for tuning the receiver to the frequency of the desired DRM transmitter.

\begin{figure}[htbp]
\centering
\includegraphics[width=0.6\textwidth]{fig7-8}
\caption{Functional diagram of the DRM receiver and PC.}
\end{figure}

The second interface supplies the received signal information to the PC. Unlike a regular radio, the output of the DRM receiver is not an audio signal that can be heard through speakers or headphones. Instead, the DRM receiver mixes the signal of the DRM transmitter down to an intermediate frequency of 12 kHz. The receiver output supplies a mixture of modulated carrier frequencies that together transmit the audio signal as a digital data stream. This DRM spectrum, a frequency mixture with a bandwidth of 10 kHz, is connected to the line input of the PC's sound card. The sound card digitizes the signal, and a DRM receiving program, containing a DRM software demodulator/decoder as its core component, is responsible for both demodulating the DRM signal and decoding the received data stream. The audio signal is then available at the output of the sound card in stereo Hi-Fi quality for playback through the PC's speakers.

The block diagram can be easily found in the circuit diagram after Figure 7.9. The DDS oscillator with IC2 (see Section 5.9) supplies its signal via T1 to the first mixer (MIX1), a diode-ring mixer. The intermediate frequency of 455 kHz passes through a steep-slope ceramic filter (Fl1) with a bandwidth of 12 kHz. A IF amplifier stage with a BF494 (T2) boosts the level by about 20 dB before the signal is fed to the second mixer, a passive FET mixer with a BF245 (T4). The second oscillator is stabilized by a ceramic resonator CSB470, which is 'pulled' by 3 kHz to 467 kHz. The resulting 12 kHz IF signal passes through a simple bandpass filter and is again amplified and buffered by two opamps (IC3), before it is ready at the output for connection to the PC sound card.

The most important characteristic for good DRM reception is phase purity of the mixer oscillator. The DRM receiver meets the highest demands here: the DDS VFO generates an extremely phase-pure oscillator signal. Another important characteristic is the receiver's large-signal rejection performance. The mixers used in this design offer excellent performance in this regard so that with a simple wire antenna connected to the receiver input the DRM software achieves 30 dB quieting.

\begin{figure}[htbp]
\centering
\includegraphics[width=0.6\textwidth]{fig7-9}
\caption{The circuit diagram.}
\end{figure}

Some properties of a receiver's design are important for AM reception but are not so critical for DRM reception so these excellent results have been achieved here despite the receiver's simplified design and alignment-free setup.

The dynamic range of the PC sound card, together with the DRM software, is large enough to cope with variations of the input signal of up to 30 dB. This eliminates the need for an automatic gain control (AGC). High sensitivity is also not an issue for DRM. Even very weak DRM signals of about 10 µV cannot be improved by increasing the overall gain, because the actual signal to noise ratio is not sufficient with the large bandwidth of 10 kHz. More gain would only raise the noise floor. It has also been shown that the receiver does not require a tuned preselector. On the one hand, the image frequency at a distance of 910 kHz (2 × 455 kHz) is almost always outside neighboring broadcast bands, and on the other hand, interfering signals are surprisingly well tolerated by the DRM decoder.

The antenna input with an impedance of approximately 50 Ω is directly connected to the diode ring mixer TUF-1, which is designed for a frequency range of 2 to 600 MHz. In practice, however, the receiver can also work in the medium-wave range down to 500 kHz without any problems. If an active antenna or a low-impedance preamplifier is used, successful operation can also be achieved in the long-wave range. At the output of the ring mixer, a broadband matching network is used for 455 kHz. The impedance is raised by a resonant circuit with capacitive tapping to approximately 1.5 kΩ to match the input resistance of the ceramic filter CFW455F. The circuit is operated with a low Q-factor (<10), which results in a bandwidth of approximately 50 kHz and avoids component tolerance issues. The matching circuit also contributes to the remote signal rejection of the IF filter.

The filter CFW455F has a bandwidth of 12 kHz, of which 10 kHz are required for DRM. The additional bandwidth is not detrimental in fact, having a slightly wider bandwidth is important to handle certain frequency deviations of the second oscillator. If the second oscillator is not exactly at 467 kHz but, for example, at 467.5 kHz, the first IF shifts to 455.5 kHz. The software then has to tune the first oscillator 500 Hz higher. In the end, however, a signal of 12 kHz appears as required. The slightly shifted first IF still passes through the IF filter. This made it possible to avoid an expensive special crystal in the second oscillator. Instead, the second oscillator at 467 kHz uses an inexpensive ceramic resonator type CSB470. The frequency is pulled down by 3 kHz due to the large capacitance of the oscillator and reaches a maximum deviation of about 1 kHz.

After the IF filter, there is a single unregulated amplifier stage that raises the signal level by about 20 dB. Since there is no pre-amplification or mixing amplification and the IF filter causes additional signal attenuation, the signal levels are sufficiently small to safely avoid overloading.

A passive FET mixer converts the signal to 12 kHz. The JFET BF245 works like an RF switch shorting the signal in sync with the oscillator. This simple mixer has a large dynamic range and processes signals up to over 100 mV without noticeable distortion. The subsequent audio amplifier with the dual op-amp LM358 raises the level by about 20 dB again and contains a simple bandpass filter.

\begin{figure}[htbp]
\centering
\includegraphics[width=0.6\textwidth]{fig7-10}
\caption{The component mounting plan.}
\end{figure}

The receiver PCB is populated with conventional through-hole components on the top side, while the DDS is in its SMD outline is mounted on the underside. Some SMD capacitors are also mounted on the underside to reduce lead length and inductance.

There are only a few active DRM stations at this time, but the receiver can still be used with appropriate software for shortwave broadcast reception, digital stations of all kinds, and amateur radio.

\chapter{IQ Mixers and Software Defined Radio}

An IQ mixer is a double mixer with a 90 degrees phase difference between the two oscillator signals. This makes it possible to suppress unwanted image frequencies and reduce the need for complex pre-selection measures for the received signal. This is an essential principle of most SDR concepts.

\section{SDRadio}

Back in the day, it was only high-end world receivers that came with features like a flat screen display, a wide range of selectable receive bands, and similar luxuries. More recently however, more and more of a receiver's functions are off loaded into software, while the hardware becomes ever more Spartan. This trend has resulted in a concept called 'Software Defined Radio' (SDR) and is especially relevant in amateur radio.

\begin{figure}[htbp]
\centering
\includegraphics[width=0.6\textwidth]{fig8-1}
\caption{The SDRadio GUI.}
\end{figure}

One of the first and simplest programs for working with SDRadio was created by Italian amateur radio operator Alberto, I2PHD. This PC program, together with a sound card and a simple IQ mixer as an HF frontend, creates an excellent shortwave receiver working in all modes from AM to SSB. Without having to retune the receiver, a range up to 48 kHz can be tuned solely with the mouse. You can always see what is happening on the neighboring frequencies and can flexibly respond to interference by adjusting the receiver bandwidth, for example.

The required hardware is an IQ mixer, essentially a direct mixer in the form of a two-stage mixer with phase-shifted oscillator signals. This effectively achieves the suppression of image frequencies.

\section{Image Frequency Rejection}

Every simple mixer generates, in addition to the desired frequency, a signal at the image frequency, which often requires a lot of effort to filter out. The I/Q mixer, on the other hand, consists of two mixing stages and provides its own image frequency suppression. This principle can be used for very simple receivers and is particularly useful in the context of software-based receivers.

The most commonly used types of receivers are the direct conversion receiver, the superheterodyne receiver, and the direct mixer receiver. With a direct conversion receiver, a resonant circuit at the input provides the only method to select a particular station. An example is the regenerative receiver where active regeneration provides the necessary selectivity. A direct conversion receiver does not suffer from image frequency problems but has relatively low attenuation of adjacent channels. The superheterodyne receiver, on the other hand, uses several intermediate frequency circuits to achieve good selectivity. However, now the image frequency comes into play. A superheterodyne with an intermediate frequency of 455 kHz has a secondary reception point at a distance of 2 × 455 kHz = 910 kHz. On medium wave, a pre-selector is sufficient to suppress this unwanted signal. Many shortwave receivers, however, actually show a significant image frequency.

The direct mixer is a particularly simple receiver that can, for example, produce good results for anyone starting out in amateur radio. Without going through an intermediate frequency, it mixes the received signal with a local oscillator running at a frequency very close to the received signal to directly produce the baseband signal. The principle has also been successfully used for very simple DRM receivers, where the 'AF signal' is actually a 12 kHz intermediate frequency. In both cases, the image frequency is so close to the target frequency there is no chance of filtering it out. Figure 8.2 illustrates the problem with an example. A signal at 3990 kHz is to be downmixed to 12 kHz. The mixer oscillator operates at 3990 kHz + 12 kHz = 4002 kHz. This creates the image frequency of 4002 kHz + 12 kHz = 4014 kHz. Now you have to hope there is no strong signal on this 'wrong' frequency.

\begin{figure}[htbp]
\centering
\includegraphics[width=0.6\textwidth]{fig8-2}
\caption{Image frequency generation.}
\end{figure}

An I/Q mixer solves the problem of the unwanted spurious frequency by using the concept of quadrature signals. Two identical mixers are used here, which use the same local oscillator signal, but with a phase difference of 90 degrees. The RF received signal is now mixed with the two LO signals to produce an I (in-phase) and Q (phase shifted) output. The signals must now be phase-shifted again and then sent to an adder. Here, the image frequencies will cancel each other out, while the desired signal will be amplified. The reverse procedure is used to generate SSB signals (Figure 8.3). The same job is performed here: mixing without an image frequency, which in this case corresponds to the other sideband. The technique is known as the 'phasing method' in amateur radio circles.

\begin{figure}[htbp]
\centering
\includegraphics[width=0.6\textwidth]{fig8-3}
\caption{SSB signal generation using the phasing method.}
\end{figure}

The difficulty with the phasing method, however, is to uniformly rotate an entire frequency band from 300 to 3000 Hz in phase. In the so-called 'third method' using the 'Weaver modulation' method, two additional mixers are used, which also receive phase-shifted oscillator signals to recover the audio signal.

\begin{figure}[htbp]
\centering
\includegraphics[width=0.6\textwidth]{fig8-4}
\caption{Signal processing according to the 'Third Method'.}
\end{figure}

Nowadays, the conditions for using an I/Q mixer have become even more favorable because signal processing via software has made tremendous progress. For simple experiments, there are excellent programs like SDRadio or SDR\# available. All you need to do is provide two signals mixed with a 90-degree phase difference to the left and right channels of the PC sound card. The software takes care of everything else.

\section{The IQ Mixer}

The simplest way to generate a phase shifted waveform from an oscillator signal is to use some digital circuitry. Two D-type flip-flops such as the 74AC74 can be used to divide an input frequency by four and simultaneously produce two output waveforms shifted by exactly 90 degrees.

\begin{figure}[htbp]
\centering
\includegraphics[width=0.6\textwidth]{fig8-5}
\caption{The IQ-Mixer.}
\end{figure}

Originally, a programmable quartz oscillator based on the CY27EE16 was used as a clock oscillator. This chip is no longer available and can be substituted with the SI5351. In principle, you could also use a tunable oscillator running at four times the reception frequency, but the necessary stability could only be achieved after careful design.

The mixers are constructed using four analog switches contained in the 74HC4066 IC, which offer good synchronization and can handle high signal levels with good isolation between channels. Two analog switches are controlled by the clock generator, to produce a balanced mixer. Figure 8.6 shows an I/Q direct mixer for the frequency range from 500 kHz to about 30 MHz.

\begin{figure}[htbp]
\centering
\includegraphics[width=0.6\textwidth]{fig8-6}
\caption{A wideband receiver.}
\end{figure}

A broadband transformer with 10:20+20 turns was wound on a small ferrite core. Simple low-pass filters are used at the mixer outputs. The subsequent 20 dB amplifier gain improves the receiver sensitivity. Figure 8.7 shows a prototype setup of the circuit.

\begin{figure}[htbp]
\centering
\includegraphics[width=0.6\textwidth]{fig8-7}
\caption{Prototype layout of the complete receiver.}
\end{figure}

The I/Q mixer can achieve a spurious signal rejection of up to about 40 dB. Connecting only one of the two channels to the sound card will produce the typical result for a simple direct mixer (Figure 8.8). A signal of 11 kHz appears both at +11 kHz and at -11 kHz. However, with both inputs separated by a 90 degrees phase shift, the desired signal is amplified and the image signal suppressed (Figure 8.9). A bandwidth of 48 kHz can therefore be tuned by software alone.

\begin{figure}[htbp]
\centering
\includegraphics[width=0.6\textwidth]{fig8-8}
\caption{Using one mixer generates an image frequency.}
\end{figure}

\begin{figure}[htbp]
\centering
\includegraphics[width=0.6\textwidth]{fig8-9}
\caption{Using two mixers showing image frequency suppression.}
\end{figure}

This simple receiver shows surprisingly good reception results on medium wave and shortwave bands, especially high sensitivity, and good frequency stability, as well as excellent selectivity, which is achieved solely through software.

A fundamental weakness of the simple circuit is that the mixer is driven at the odd harmonics of the oscillator frequency. Some of the local oscillator switching signals can leak into the output signal path to generate intermodulation products, a low-pass filter or resonant circuit can be useful to remove them.

\section{Circuit Optimization}

The IQ mixer consists of two identical mixing stages driven with a signal phase-shifted by 90 degrees. It is important for the two signals to have exactly the same gain and excellent linearity. CMOS analog switches have proven to be effective mixers. Originally, a programmable quartz oscillator based on the CY27EE16 served as the oscillator, which has now been replaced by the SI5351. This allows for an operating frequency of up to about 30 MHz.

The circuit uses analog switches of the 74HC4066 type and a digital divider using 2 flip flops using the 74AC74, which divides the oscillator frequency by four and creates the necessary phase shift. To make it easy and get a head start the company Modul-Bus has developed a board that provides the entire mixer/oscillator as a PCB module. Connections to the I and Q mixer signals are available at the board edge.

\begin{figure}[htbp]
\centering
\includegraphics[width=0.6\textwidth]{fig8-10}
\caption{The IQ mixer PCB.}
\end{figure}

Here, the different circuit variants will be studied in more detail. For the first attempt, an extremely simple receiver is built with just one additional resistor. It will receive strong radio signals with a sufficiently long wire antenna.

\begin{figure}[htbp]
\centering
\includegraphics[width=0.6\textwidth]{fig8-11}
\caption{The first mixer design.}
\end{figure}

This basic receiver can be improved by using a balanced mixer. Here a small transformer with a center tap is used and achieves an image frequency rejection of about 40 dB.

\begin{figure}[htbp]
\centering
\includegraphics[width=0.6\textwidth]{fig8-12}
\caption{Symmetrical Mixer.}
\end{figure}

In a balanced mixer the RF transformer can be a source of asymmetry. Here we will look at how to build a symmetric mixer with four audio output channels. The signals will be combined using two differential amplifiers. The idea is to design the circuit so that the mixer outputs all have a loading of precisely 10 kΩ. The initial circuit is shown in figure 8.13.

The 2.2 nF capacitors at the outputs of the analogue switches provide a low-pass cutoff frequency well above the band limit of 24 kHz, which ensures that passband tolerances do not cause additional phase shifts. These simple low-pass filters are only used to isolate high level RF signals from the operational amplifiers. The actual filtering is left to the anti-aliasing filter in the PC sound card.

\begin{figure}[htbp]
\centering
\includegraphics[width=0.6\textwidth]{fig8-13}
\caption{Mixer without carrier.}
\end{figure}

The circuit achieves a good image rejection of more than 40 dB. The high overall gain of about 40 dB is effective for picking up weak signals, but for extremely strong radio stations an antenna attenuator may be required. A weakness of the circuit is the low upper cut-off frequency. Above about 12 MHz, sensitivity drops sharply. An experiment showed that the two 1 kΩ resistors in front of the mixers are the source of the problem. If both are replaced by shorts, sensitivity is maintained above 25 MHz but this slightly reduces the image frequency rejection.

Although the circuit works relatively well, there is still room for improvement. The following detailed circuit diagram shows the differential amplifier used. It can be seen that this stage is actually not completely symmetrical. This means that the differential amplifier does not have high common-mode signal rejection. This does not cause problems with image rejection but may result in intermodulation products from the RF stage below 24 kHz directly entering the IF path via the mixer.

\begin{figure}[htbp]
\centering
\includegraphics[width=0.6\textwidth]{fig8-14}
\caption{The differential amplifier configuration.}
\end{figure}

A conventional differential amplifier was tested and showed good common-mode rejection but image rejection performance was poor.

\begin{figure}[htbp]
\centering
\includegraphics[width=0.6\textwidth]{fig8-15}
\caption{A symmetrical differential amplifier.}
\end{figure}

The problem here is in the different input resistance of the two inputs, even though the circuit appears to be symmetrical at first glance. The inverting input has an impedance of about 0.5 kΩ, while the non-inverting input has an impedance of 11 kΩ. Using rounded voltage values in the basic amplifier configuration (Figure 8.16) and driving the circuit with +1 V and -1 V. The non-inverting input sets the voltage at both op-amp inputs because it is a voltage divider to ground with no negative feedback. The upper 1 kΩ resistor is between +1 V and -1 V, resulting in an input impedance of only 0.5 kΩ. This violates the most important rule of IQ circuit technology, which is that all four phases should be equally loaded.

\begin{figure}[htbp]
\centering
\includegraphics[width=0.6\textwidth]{fig8-16}
\caption{Unsymmetrical input impedance.}
\end{figure}

There is a possible solution in the form of the so-called 'instrumentation amplifier'. For this, two fully differential op-amps are used as impedance converters. The circuit appears to have the same theoretically infinite impedance at the front end. This variant was also tested. The result was again good image rejection. However, the receiver overall showed more noise, lower sensitivity, and more distortion.

\begin{figure}[htbp]
\centering
\includegraphics[width=0.6\textwidth]{fig8-17}
\caption{An instrumentation amplifier.}
\end{figure}

Theory and practice do not always go hand in hand. The reason here is that when the signal gets up to a reasonably high frequency of 24 kHz standard operational amplifier begins to run out of steam. An LM324, for example, has a gain-bandwidth product of 1 MHz so at 20 kHz, the open-loop gain is only about 50. The difference between both inputs will no longer be almost zero. A fully compensated operational amplifier has to struggle to follow the input voltage. At higher frequencies, distortion occurs. Signals above 20 kHz can be present at the mixer output, and their distortion products may partly lie in the passband. These considerations led to the decision to reject the instrumentation amplifier for this application.

The experiments have clearly shown that the symmetrical input resistance of the circuit is more important than a high common-mode rejection ratio. If the RF input circuit operates very linearly and a high-pass filter of 24 kHz is also used, there are actually no problems with IF breakthrough. So, you returned to the original circuit using slightly different values. The effective input resistance is now about 5 kΩ at both inputs. A series resistor of 100 Ω offers a good compromise between high RF cutoff frequency and good decoupling between I and Q mixers. In addition, the TL084 is now used, which is not quite as noise-free but has a gain bandwidth product of 4 MHz and can run from a simple 5 V supply voltage.

\begin{figure}[htbp]
\centering
\includegraphics[width=0.6\textwidth]{fig8-18}
\caption{Now with matched input impedances.}
\end{figure}

The overall circuit is not optimized for highest sensitivity, but rather for high immunity to large signals and low distortion even with strong radio signals, including strong DRM stations. An SNR of well over 30 dB has been observed using this setup. The high sensitivity typically found in amateur radio receiver specifications is not achieved here. Selective preamplifiers can however be used to address these requirements as well.

\begin{figure}[htbp]
\centering
\includegraphics[width=0.6\textwidth]{fig8-19}
\caption{The optimized mixer design.}
\end{figure}

Figure 8.19 shows the latest version of the receiver. What's still missing here for a practical design are some low-pass or band-pass filters for specific frequency bands. For example, if you want to receive medium wave without a pre-filter, shortwave stations working at three or five times the frequency will break through. Suppressing unwanted signals in the IQ receiver is easy because there is no image frequency to worry about, only harmonic mixing products. In principle, low-pass filters alone will do the job.

\begin{figure}[htbp]
\centering
\includegraphics[width=0.6\textwidth]{fig8-20}
\caption{The populated IQ mixer PCB.}
\end{figure}

Now with the benefits of the latest optimized circuit the software-defined radio is more sensitive with better image signal rejection. In the end, everything was fitted onto a single PCB.

\begin{figure}[htbp]
\centering
\includegraphics[width=0.6\textwidth]{fig8-21}
\caption{The RF front end.}
\end{figure}

At the RF front end an FET is configured as a source follower which provides a low impedence signal to drive the mixers. There are three options for the antenna coupling: a wideband setting, a high-pass filter for shortwave or a low-pass filter for medium wave. The low-pass filter has a cutoff frequency of around 1.6 MHz to help suppress harmonic mixing products and interference from shortwave when receiving medium wave stations.

\begin{figure}[htbp]
\centering
\includegraphics[width=0.6\textwidth]{fig8-22}
\caption{Two-stage IQ amplifier.}
\end{figure}

The IF amplifier uses two stages to give a total gain of 100. The input is optimized for equal input impedance of all four phases, which improves image rejection. The receiver achieves an overall image rejection of 40 dB or more.

\begin{figure}[htbp]
\centering
\includegraphics[width=0.6\textwidth]{fig8-23}
\caption{Connected to the mixer PCB.}
\end{figure}

The circuit board fits directly onto the IQ mixer. Together with the programmable quartz oscillator, it forms a complete receiver with tuning via the serial interface. Power is now supplied from the right-hand board via a voltage regulator. The screw terminals blocks on the IQ mixer are unused but 5 V can be taken from here to power any additional circuits.

\begin{figure}[htbp]
\centering
\includegraphics[width=0.6\textwidth]{fig8-24}
\caption{An alternative oscillator.}
\end{figure}

The tried and tested receiver design was now also tested with a different VFO. The ICS307-2 programmable oscillator is powered via the 2-way 5 V power terminal block on the mixer board. Its output clock signal is connected using a small length of wire to the appropriate socket position on the DIL socket.

\begin{figure}[htbp]
\centering
\includegraphics[width=0.6\textwidth]{fig8-25}
\caption{Oscillator setup in software.}
\end{figure}

To tune the receiver to a specific frequency, you can use the ICS703-2.exe program, which communicates with the receiver through the COM1 serial port. For example, to receive an AM station at 6155 kHz, you would set the receiver oscillator to 6145 kHz. This allows you to receive the station using SDRadio which gives excellent audio quality.

\begin{figure}[htbp]
\centering
\includegraphics[width=0.6\textwidth]{fig8-26}
\caption{AM station reception in SDRadio.}
\end{figure}

\section{Software Defined Radio with USB Interface}

Back in 2007, Elektor Magazine developed and produced this receiver board based on the earlier explorations into the emerging field of digital radio. The aim of the project is to provide beginners with easy access to the topic of Software Defined Radio.

A Software Defined Radio (SDR) requires little hardware but sophisticated software. This SDR project aimed to show what is achievable with minimal effort. The goal was to create a universal receiver working from 150 kHz to 30 MHz, optimized for DRM and AM reception, but also allowing reception of amateur radio bands.

The objective of this project was to create a receiver with medium sensitivity, but with the highest linearity and phase purity. The development focused on properties that are important for a top-notch DRM receiver. In fact, the receiver achieves an excellent signal to noise performance.

\begin{figure}[htbp]
\centering
\includegraphics[width=0.6\textwidth]{fig8-27}
\caption{The Elektor Software Defined Radio with USB port.}
\end{figure}

Regarding sensitivity and overload resistance, the receiver can't compete with top of the range amateur radio equipment. Experiments have however shown that on the lower bands up to 20 meters, the atmospheric noise is usually so strong that greater sensitivity is not really an advantage. A comparison with an older Yaesu FT-7B rig showed about the same results when receiving CW and SSB stations on the 80, 40, and 20-meter bands using the same antenna. However, the SDR scored points with its advanced software capabilities. Features such as continuously adjustable bandwidth and spectrum display are otherwise only available in much more expensive receivers.

The receiver is controlled via USB from where it also sources its power. No additional power supply is needed. The FT232R was chosen as the USB interface. This modern USB-to-serial converter does not require a quartz crystal because it has an internal high-precision RC oscillator. The component is used here in its bit-bang mode, like a fast parallel port. Eight data lines are available and can be controlled as desired. Two of the signals are used as I²C bus to control the receiver frequency. Three signals are used to switch the input multiplexer to select one of eight antenna inputs with and without filters. Two more inputs are used to switch the receiver's IF amplifier gain. This control interface ensures the receiver can be fully managed via software.

\begin{figure}[htbp]
\centering
\includegraphics[width=0.6\textwidth]{fig8-28}
\caption{The receiver schematic in brilliant Elektor style.}
\end{figure}

Special attention was paid to decoupling the power supply. One of the reasons for this is that the FT232RL USB chip uses internal clock signals which are at frequencies that will also be received via the antenna. You don't want any of this unwanted RF noise to leak across from one stage to another. The FT232R contains exactly what the programmable clock generator CY27EE16 needs with a voltage regulator of 3.3 V. Therefore, no additional voltage regulator is needed. The remaining part of the circuit operates at 5 V, several supplies are provided to power specific functions on the board with appropriate decoupling to reduce crosstalk and noise. Keep in mind that the 5 V from the USB ultimately comes from a PC power supply. The same power supply powers the entire PC, and load changes can cause some noise on the supply line. This is particularly critical for the RF preamplifier of the receiver, which ultimately couples via the mixers to the IF branch. Therefore, a large capacitor provides stability at this point (VCC\_HF).

The SDR requires an oscillator frequency that is four times higher than the signal received so that it can be divided by four with the required phase shift. If you are aiming to receive signals up to 30 MHz, the oscillator needs to run up to 120 MHz. DDS oscillators are often used in advanced RF projects but a DDS able to run at this speed will work out expensive, power-hungry, and difficult to control. For this reason a programmable clock oscillator with an internal PLL is used here. Although the CY27EE16 was originally designed as a clock oscillator for digital electronics and processors, it has proven itself in many RF applications. Although the achievable frequency resolution is not as good as a DDS, the relatively modest power consumption is important for this project, as you cannot draw too much power from the standard USB port.

The chip is programmed via the I²C bus using the SCL and SDA lines. A VCO operates internally in the frequency range of 100 to 400 MHz. The VCO is stabilized using a 10 MHz crystal and a PLL. Its output signal is passed via dividers to reach the desired outputs. The clock output Clock5 was chosen here. There is a VFO signal between 600 kHz and 120 MHz that passes to the 74AC74 chain of dividers.

The exact phase shift of 90 degrees between the two oscillator signals is important. Deviations lead to less effective suppression of image frequencies. Since the divider 74AC74 is connected as a synchronous divider, you would not expect to find any phase error here. In fact, the receiver shows a constant mirror suppression of about 40 dB up to about 15 MHz. From about 20 MHz, this value decreases noticeably, but this is tolerable due to the lower occupancy in this frequency range.

The receiver has several inputs that are switched via the input multiplexer 74HC4051. The antenna input Ant1 is fed by way of filters to the first three inputs. The first switch position (wide) uses only an input choke, to short any low-frequency signals at the input to ground. In the second position (medium wave), there is a low-pass filter with a cut-off frequency of 1.6 MHz, where the resistor R12 dampens a resonance peak. This filter prevents medium wave reception from being disturbed by harmonic mixing with stations in the shortwave range. The third position uses a simple RC high-pass filter, which is intended to attenuate strong medium wave signals.

Another input (PC1) can be selected if you wish to connect external tuned input circuits or preamplifiers. Three more inputs are provided for future expansion. The input filters on the board can be regarded as a kind of basic equipment that is sufficient in most cases. However, it would be possible to add further steep low-pass filters or special band-pass filters, which would safely attenuate harmonic mixing components in all situations.

From the input multiplexer, the RF signal goes to a BF245C JFET which functions as an impedence converter. The input is relatively high-impedance at 100 kohms, so, for example, a high-Q resonant circuit can also be connected to the input In2. At the low-impedance output of the source follower, a voltage of about 2.5 V is established, which is passed through the mixers and the following operational amplifiers to the output. Therefore, it is important that there are no low-frequency signal residues at the source. For example, the purity of the supply voltage Vcc\_HF is critical and therefore has a high level of filtering. The FET itself provides additional decoupling of the supply voltage. But nothing should come from its gate that reaches the IF range below 24 kHz. For this reason an RF choke is placed directly at the antenna input to shunt any 50 Hz (60 Hz) hum signals, for example.

The IF amplifier consists of two exactly equal branches, each of which provides a total gain of up to 40 dB. The TL084 was chosen here because it has a good gain-bandwidth product of 10 MHz at a supply voltage of 5 V. This is important in order to supply a gain of 10 without phase errors to signals at around 20 kHz.

The final stage has a gain of 10 (20 dB), but this can be reduced to unity gain via the analog switches. A total of three attenuation steps are available: 0 dB, -10 dB, and -20 dB. So, if excessively strong signals lead to overload, the gain can be reduced by software. The attenuator is not located at the input of the receiver because there is already high overload resistance built in. On the other hand, at full gain, with a long antenna and high field strengths, the output can be over driven. The attenuation therefore applies to the output driver stage and corresponds approximately to the gain control in an IF amplifier.

\begin{figure}[htbp]
\centering
\includegraphics[width=0.6\textwidth]{fig8-29}
\caption{The receive spectrum obtained with G8JCFSDR.}
\end{figure}

\section{Arduino SDR Shield}

Elektor built its first Software Defined Radio with a USB interface back in 2007, using a conventional board and only a few SMD components. Since then, there have been thoughts about updating the design and when the PLL chip used in the original was phased out it was time to find a new solution. The original concept was recreated using the SI5351 Silicon Labs PLL chip which is a CMOS clock generator with an I2C interface that generates signals ranging from 8 kHz to 160 MHz.

Initial tests with this new chip using a breakout board from Adafruit (Section 5.10) were successful. The existing software examples were written for the Arduino, so the first steps were taken towards the Arduino environment. The new VFO was connected to the existing SDR board, and it proved to be functional. Then came the question; why not build the entire receiver as an Arduino shield? That way the power supply and USB interface would be taken care of.

\begin{figure}[htbp]
\centering
\includegraphics[width=0.6\textwidth]{fig8-30}
\caption{The SDR Shield with an Arduino Uno.}
\end{figure}

The current version of the receiver, which was last updated in 2019, is delivered as a fully assembled board with included header strips. You still need to solder the headers onto the board so that the SDR board can be plugged onto the Arduino. Once that's done you will need to install some software, which can be found on the Elektor website and the author's homepage. Finally to make it all work together you can use an audio cable to establish a connection to your PC's sound card, attach an antenna, and you're good to go.

\begin{figure}[htbp]
\centering
\includegraphics[width=0.6\textwidth]{fig8-31}
\caption{The SDR shield and its header-socket strips.}
\end{figure}

The Arduino itself doesn't really have much to do - it receives the desired frequency from the PC and adjusts the VFO as required. This means that there is even a real chance to build a standalone receiver, as the Arduino can handle the tuning all by itself. This opens up unlimited possibilities, especially since the Arduino is widely used and many people can program it.

\begin{figure}[htbp]
\centering
\includegraphics[width=0.6\textwidth]{fig8-32}
\caption{The shield schematic.}
\end{figure}

Looking at the circuit diagram (Figure 8.34), you can see the individual components. The SI5351 chip provides the oscillator signal which is tuned to four times the desired receive station frequency. Two D-type flip flops type 74AC74 (IC2B) provide a divide-by-4 function to produce two clocks at the desired frequency with a 90 degree phase shift difference. These two clock signals are used to control the analog switches in the 74HC4066 (IC3) which functions as a mixer. It alternately connects the RF signal to the inverting and non-inverting inputs of the TS914 operational amplifier (IC4), downconverting the signal to recover the baseband signal. After minor filtering and amplification, the signal goes to the audio output. The RF input stage is a source follower with the BF545B JFET, the SMD equivalent of the BF245B.

The input is broadband and protected against overvoltage by two limiting diodes, which is sufficient for shortwave reception with a wire antenna. The overvoltage protection is based on the experience that input stages can be damaged during a thunderstorm. For critical tasks, external filters and preamplifiers can still be used.

The latest version, V2\_0, is technically identical but has additional connection points for additional PLL outputs and DC-coupled signal outputs. This expansion facilitates experimental use of the shield and simplifies external expansions for measuring instruments or shortwave transceivers.

To use the receiver, you need a USB connection to the Arduino, an audio cable to the PC sound card and a suitable antenna. Additionally, an Arduino sketch must be loaded to set the VFO chip to the desired frequency.

\begin{figure}[htbp]
\centering
\includegraphics[width=0.6\textwidth]{fig8-33}
\caption{The receiver with LC-Display mounted.}
\end{figure}

The shield is designed to be used together with the Elektor LCD shield. It can be used to display the current frequency and is also useful for measuring purposes or standalone applications.

\begin{figure}[htbp]
\centering
\includegraphics[width=0.6\textwidth]{fig8-34}
\caption{The received station displayed using SDR\#.}
\end{figure}

\chapter{Shortwave Antenna Design}

When starting out building a simple crystal radio or an SDR receiver, it's often enough to use a simple stick antenna or even a test lead on the lab bench to act as an aerial. For better reception try something like a 5 m length of wire hung around the room or even just resting on the floor. However, with this type of layout, the antenna will also pick up any background electrical noise from domestic appliances and interference from the mains power network. For indoor use, antennas sensitive to the magnetic field have proven to be more effective. With a good outdoor antenna, especially on shortwave, you can achieve so much more. The special appeal of listening in to the shortwave bands lies in the chance of picking up broadcasts from great distances (DX reception) thanks to sky bounce.

\section{Radio Wave Propagation}

In the VHF radio frequency band, radio waves behave similar to light waves and this quasi-optical propagation pattern limits their reception range to about 100 km depending on antenna height. However, radio waves below 30 MHz behave completely differently and allow for a much greater range. Nevertheless, the complex propagation mechanisms in this range also lead to special problems such as dependence on the time of day, fluctuating field strength (fading) and selective fading.

The crucial role in the propagation of shortwave radio signals is played by ionized, weakly conducting layers of air in the upper atmosphere, created by solar particle and gamma radiation that ionize the air molecules, stripping electrons from them. These free electrons act like a mirror to certain frequency bands and under certain RF wave angles of incidence.

The ionosphere is, however, transparent to high angles of incidence and high frequencies.

\begin{figure}[htbp]
\centering
\includegraphics[width=0.6\textwidth]{fig9-1}
\caption{Skywave and dead zones affecting shortwave propagation.}
\end{figure}

In shortwave radio communication, the range of a transmitter can be limited to about 30 to 100 km by ground wave propagation, depending on the height of the antenna. Beyond this distance, the signal disappears over the horizon and direct line of sight communication is not possible. However, at a certain minimum distance, waves reflected by the ionosphere can reach the receiver (skywave propagation). There is a dead zone between the ground wave range and the reflected wave range, where neither wave can be received. This dead zone varies for each frequency and depends on the time of day and level of solar activity.

Higher frequencies allow for a flatter reflection, resulting in longer ranges. A dead zone is therefore larger and extends up to about 200 km at 6 MHz and up to about 1000 km at 15 MHz during the day. At night, dead zones expand and the range increases. This can lead to a situation where a specific transmitter is heard clearly in the evening, but suddenly drops out, having entered a dead zone. Listeners may be able to overcome this by switching to a lower frequency band if the same program is broadcast on multiple bands.

Most of the time, radio waves reach the receiver through multiple paths. These paths have different lengths, causing phase differences that can result in partial amplification or cancellation of the signal. In the shortwave range, field strength can fluctuate rapidly, leading to selective fading, which can cause unpleasant distortion in AM radio broadcasts.

However, DRM (digital radio mondiale) is more robust against partial data loss caused by fading. Despite the deep notches in the DRM spectrum caused by the cancellation of certain frequencies, the reception is usually not disturbed thanks to effective error correction.

\begin{figure}[htbp]
\centering
\includegraphics[width=0.6\textwidth]{fig9-2}
\caption{Selective Fading.}
\end{figure}

\section{The Longwire Antenna}

If you are only interested in listening to strong local shortwave broadcasts all that's necessary is a short whip antenna less than a meter in length. Under favorable conditions, you can test this on the bench by using a length of cable as the antenna. For long-distance (DX) reception however, a long wire antenna outside the house in free space is a better bet. More important than the shape of the antenna is its elevated position far enough away from houses to avoid domestic electrical interference.

Suspended long wire antennas are a good solution for shortwave reception. In theory, resonance occurs at a quarter wavelength of the received radio signal; a good ground wire to the antenna can help to reduce the effects of unwanted noise and interference and can also improve the antenna's efficiency by providing a more stable and consistent ground reference. In practice, wire antennas of around 10 meters in length usually give good results. If the receiver is close to a window or an outer wall of your house, all you need is to connect the end of the antenna directly to the center pole of the coaxial aerial connector of your receiver. If the antenna feed needs to travel a longer distance inside the house, a coaxial cable should be used, and a ground connection should be made near the antenna feed point. It doesn't matter whether a 50-ohm or 75-ohm cable is used since the antenna has a variable characteristic impedance depending on the receive frequency and a complex impedance with varying capacitive and inductive components. The length of coax also has its own resonances which affect the overall antenna impedance, so that other resonances can occur outside those predicted by the antenna length itself. However, this is barely noticeable at the receiver because signal differences of around 10 dB are hardly significant.

\begin{figure}[htbp]
\centering
\includegraphics[width=0.6\textwidth]{fig9-3}
\caption{A longwire antenna using a coaxial cable feed.}
\end{figure}

When planning a longwire antenna, it's common to use copper wire with a decent sized cross-section to achieve both good mechanical stability and low ohmic losses. A good option is to use the type of cable used for standard mains wiring with a cross-section of 0.75 mm² to 1.5 mm², but thinner wires can also be used. For example, a test with thin coil wire with a diameter of only 0.3 mm produced usable results when stretched about 10 meters outdoors and then another 10 meters inside an apartment. Indoors it was rigged above head height to remain reasonably inconspicuous. The additional section running inside the building picks up local interference and adds some additional signal attenuation. Despite this, a usable makeshift antenna was created and remained unnoticed and almost invisible while still pulling in far flung stations.

If building your own antenna seems daunting, you might be able to reuse some existing installations and cables. A typical rooftop antenna installation delivers not only TV and FM signals, but also the entire AM range from longwave to shortwave. It's worth trying to see what can be received. In many cases, a rooftop antenna provides better results than an indoor antenna. Often, old antennas are no longer in use, or have been swept off the roof by a passing gale. In this case, the coax feed may still be in place; just by shorting the outer shield together with the inner core you now have a useful vertical antenna. The cable usually leads to the roof of the house, providing greater height than a horizontal longwire antenna. Better results can be achieved with this set up, especially on higher frequencies above 15 MHz, than with a longwire antenna.

\section{Using a Preselector}

In many cases, the performance of a receiver can be improved by adding a tuned preselector circuit between the aerial and front end of the receiver. This can often prevent overload caused by strong nearby signals outside the reception band. Whilst preselection is not always necessary, it can be a good solution in some specific situations. In amateur radio or shortwave listening, antenna matching devices are used that provide both optimal impedance matching and some selectivity. This often results in a significant attenuation of the unwanted image signal.

A simple solution for the 49-meter band is to use a 6.0 MHz ceramic bandpass IF filter type SFE 6, which was originally used in the audio carrier path of television sets. Although its nominal -3 dB bandwidth of approximately 100 kHz is somewhat narrow and its 600 ohms impedance is not optimal, it still works well. The low impedance of the antenna and receiver input flattens the filter response. The -6 dB corner frequencies were measured at 5850 kHz and 6150 kHz, allowing all the important frequencies in the 49-meter band to pass through. Figure 9.4 shows the filter at the receiver input. A bypass switch allows for easy comparison of results with and without the filter. Additionally, the receiver can be easily switched to a wider bandwidth when receiving stations outside the 49-meter band.

\begin{figure}[htbp]
\centering
\includegraphics[width=0.6\textwidth]{fig9-4}
\caption{Using a ceramic IF filter.}
\end{figure}

To build a preselector for use with other shortwave bands as well, is best to use an adjustable resonant circuit. For this you can start by winding an air core coil made up of 20 turns wound around an 8 mm diameter plastic former. The winding should measure 10 mm along the length of the former to produce an inductance of 2.5 µH. A tuning capacitor of 320 pF achieves a lower frequency of about 5.6 MHz. Therefore, the 49 m band and higher bands up to about 16 MHz can be tuned. A tap at the second turn of this coil provides the appropriate impedance for connection to the receiver input. The antenna can be connected via a coupling coil made up of two to four turns. If the coupling coil is designed so that it can be shifted along the axis of the first coil then some variable coupling will be possible. This allows you to find the optimal match. A tighter coupling results in a higher signal voltage, but also a lower Q factor and thereby less attenuation of the image frequency. If a short antenna such as a stick antenna is to be used, the coupling must be arranged more tightly. The antenna can then be connected directly to the hot end of the resonant circuit.

\begin{figure}[htbp]
\centering
\includegraphics[width=0.6\textwidth]{fig9-5}
\caption{An adjustable band pass filter adds preselection.}
\end{figure}

The resonant circuit has a Q factor of about 50, which results in a bandwidth of 120 kHz at 6 MHz. Therefore, the tuning capacitor needs to be adjusted quite precisely. If the preselector is housed in a case, it's a good idea to mark the most important frequencies on a scale. A typical tuning capacitor has a tuning range of not much more than 1 to 10, including all circuit capacitances. This results in a frequency range ratio of 1 to 3. To cover larger frequency bands, multiple coils can be used, and a selector switch can be used to choose between them.

An alternative solution comes from the world of amateur radio, where the same problem occurs with the usual amateur radio bands (80 m to 10 m, 3.5 MHz to 29.7 MHz) which require a preselector with a tuning range of 1 to 10. Here the answer is to use coupled circuits, which have two tracked resonances. Figure 9.6 shows a proven circuit with a twin ganged tuning capacitor. Although there are two pass frequencies for each setting, the "wrong" one is far away from the desired frequency.

\begin{figure}[htbp]
\centering
\includegraphics[width=0.6\textwidth]{fig9-6}
\caption{Tuning from 3 MHz to 30 MHz.}
\end{figure}

Instead of a variable capacitor, a variable capacitance diode or varicap like the BB112 can also be used for tuning. It is important to have a stable and well-smoothed tuning voltage, otherwise phase modulation of the received signals could affect reception. Figure 9.7 shows a preselector using a BB112 varicap.

\begin{figure}[htbp]
\centering
\includegraphics[width=0.6\textwidth]{fig9-7}
\caption{A varicap is used for tuning.}
\end{figure}

A fixed-frequency tuned circuit at the receiver front end may also be a useful solution in some situations. In the medium-wave band for example, there may be only one usable reception frequency. Even with a relatively large relative bandwidth of the input circuit, good selection is achieved due to the low frequency. The circuit in Figure 9.8 therefore employs a fixed inductor. The fixed-frequency band pass filter used here was designed for 1296 kHz.

\begin{figure}[htbp]
\centering
\includegraphics[width=0.6\textwidth]{fig9-8}
\caption{An antenna filter for 1296 kHz.}
\end{figure}

\section{Tuned Magnetic Field Antennas}

A long wire antenna receives electrical energy from both the electric and magnetic field components of the RF field but smaller antennas such as whip antenna are only sensitive to the electric field component, which results in higher noise in the received signal, especially in domestic environments. Electrical appliances and power line noise couple capacitively to the receiving antenna, so in this environment it would be advantageous to pick up the magnetic field component instead. In principle, a wire loop or coil is sufficient for this purpose. A commonly used antenna design for this purpose is a frame antenna onto which a few turns of wire or simple loops (aka magnetic loops) are wound. Tuned loops with high Q are highly effective. For example, a length of copper pipe bent into a one-meter diameter loop can be used, or a wide length of aluminum foil wrapped around a correspondingly large cardboard box also produces good results. Tuning the loop with a variable capacitor up to 500 pF produces a resonant circuit with an extremely high Q, resulting in significantly more voltage at the antenna than might be expected for an aerial of this size. The receiver is loosely coupled using a small coupling coil to avoid excessively damping the loop. The optimal size and location of the coupling coil is best determined experimentally. Thanks to the high Q of the antenna, an additional preselector is unnecessary in this design.

\begin{figure}[htbp]
\centering
\includegraphics[width=0.6\textwidth]{fig9-9}
\caption{A Magnetic Loop Antenna.}
\end{figure}

A magnetic loop antenna can also be built using simple wire, although this results in lower Q factor together with lower antenna voltage and a wider bandwidth. If the antenna needs to be physically smaller, two or more turns of insulated wire can be used.

In the simplest case, a shielded loop can be built using a length of coaxial cable. This antenna can be discreetly placed on a bookshelf and provides a relatively good signal-to-noise ratio. The resonance frequency is determined by the size of the loop and the value of the tuning capacitance. With a total of four meters of coaxial cable and a 500 pF tuning capacitor, resonant frequencies below 6 MHz can be achieved. The broadband transformer should have a higher value of inductance on the primary inner-wire loop side than on secondary outer-shield loop side. Good results can be achieved with about 20 turns on a ferrite or toroidal core. The tuned circuit should not be overly damped for high quality. Therefore, the secondary side of the transformer should only have two to four turns, the best coupling coil match should be determined experimentally.

\begin{figure}[htbp]
\centering
\includegraphics[width=0.6\textwidth]{fig9-10}
\caption{A Tunable Shielded Loop Antenna.}
\end{figure}

\section{An Active Indoor Antenna}

Sometimes there is simply no opportunity to rig up an outdoor antenna. A solution to this situation could be a small table-top antenna with a two-stage preamp. The antenna described here consists of a shielded magnetic loop with an additional telescopic antenna. The loop is about 30 cm diameter and the telescopic antenna extends to about 75 cm. While the shielded loop antenna is very insensitive to electrical near-field interference, the antenna provides a increased electric field signal as required.

The loop antenna has a clear directional characteristic with two maxima in the longitudinal plane. By rotating the antenna you can search for a maximum of the wanted signal or optionally suppress an interfering signal. The telescopic antenna has a circular radiation pattern without any directional effect on its own. However, when operated together with the loop, both signals add together. Due to the phase shift between the electric (E) and magnetic field (H), their sum produces a single maximum. To produce a distinct minimum for unwanted signals, the telescopic antenna needs to be adjusted experimentally to balance both signals. The best directional effect is achieved with a telescopic length of about 40 cm.

\begin{figure}[htbp]
\centering
\includegraphics[width=0.6\textwidth]{fig9-11}
\caption{A tabletop antenna with preamplifier.}
\end{figure}

The antenna itself only delivers very small signal levels compared to a long wire antenna. The two-stage preamplifier compensates for this signal loss almost completely. The limit to the maximum preamplifier gain setting is when you begin to hear the noise generated by the first transistor in the preamp stage. However, this is not yet the case, especially in the shortwave band, i.e., atmospheric noise prevails despite the small antenna. A second limit of amplification arises due to the possible occurrence of intermodulation from strong signals. With three stages, strong overloading of the final transistor and clear intermodulation products can occur. You can recognize this condition when there are practically no free frequencies available and the background noise increases sharply. A two-stage preamp however works quite well with this size of antenna.

Both stages use feedback to adjust the operating point and reduce distortion. The input stage is designed with high impedance, while the output stage is optimized for high output with a collector resistor of 150 ohms.

The suggested BFR96T transistors were specially developed for UHF preamplifiers. However, in this application in the shortwave band, there is really no need to use such a high frequency component; you could substitute another high frequency transistor such as a BF494. Experiments using a low-noise low-frequency transistor such as a BC548 also produced good results, although a drop off in the RF gain was apparent above 10 MHz. The antenna also works quite well in the medium-wave band when general purpose transistors are fitted.

\begin{figure}[htbp]
\centering
\includegraphics[width=0.6\textwidth]{fig9-12}
\caption{Active Antenna construction.}
\end{figure}

The loop antenna has an extremely low radiation resistance and poor matching to the first amplifier stage, which results in only small signal voltages and a low signal-to-noise ratio, especially for reception on longwave and the VLF band below 150 kHz. By using an RF transformer you can improve matching. A ferrite core transformer with an AL value of over 1000 nH/n² is suitable for this application and should have 4 turns on the primary and 40 turns on the secondary. This increases sensitivity by about 20 dB, allowing reception of the entire band from about 50 kHz up to the medium wave band. Using only the magnetic loop antenna (without the rod antenna) provides good directional characteristics and excellent shielding against electric fields from nearby electrical devices. The active antenna achieves much better reception performance than a long wire antenna. Signals from the DCF77 time signal transmitter at 77.5 kHz and numerous other stations operating below 100 kHz can be received clearly.

\begin{figure}[htbp]
\centering
\includegraphics[width=0.6\textwidth]{fig9-13}
\caption{Adaptation for the longwave band.}
\end{figure}

\section{Antenna Preamplifier}

Typically, an SDR receiver is designed for long antennas and will not overload even with the higher signal levels of a long wire antenna. However, it's worth testing to see if a short whip antenna will do the job also. For this purpose, a small preamplifier is required. The following circuit works across the entire shortwave range. The antenna used here is only 30 cm long and consists of 0.5 mm diameter single strand wire. The recovered signal levels were comparable to those of the long wire antenna but interference from domestic appliances was more apparent so that the noise floor was worse, resulting in more dropouts.

\begin{figure}[htbp]
\centering
\includegraphics[width=0.6\textwidth]{fig9-14}
\caption{Preamplifier for short antennas.}
\end{figure}

In the medium-wave range, a ferrite rod antenna works well as they are relatively insensitive to electrical interference, similar to magnetic loops. Figure 9.15 shows a ferrite antenna with an impedance converter. The resonant circuit generates relatively high resonance signals, even from distant transmitters.

\begin{figure}[htbp]
\centering
\includegraphics[width=0.6\textwidth]{fig9-15}
\caption{Active Ferrite antenna.}
\end{figure}

\chapter{VHF Radios}

Listening to distant shortwave broadcasts can be exciting and a real challenge, but most people who simply want to listen to music or current affairs end up using an FM radio on the more common VHF frequency band. This type of radio that sits on the kitchen worktop is fairly ubiquitous. It's easy to forget that you could fairly easily build one of them from scratch. In this article, we will look at some tried and tested designs that will help you make your own FM radio.

\section{A Superregenerative Receiver}

The simplest FM radio receiver circuit is the superregenerative (superregen) receiver. You can build such a receiver with just two transistors.

\begin{figure}[htbp]
\centering
\includegraphics[width=0.6\textwidth]{fig10-1}
\caption{A 2 transistor superregenerative receiver.}
\end{figure}

To ensure stable operation, a large ground plane is necessary for this design. To build this experimental setup in the lab I used a cut-out tin lid of a coffee can. These sorts of cans are often used to package dry edible goods; they normally have a cardboard tube crimped onto a thin circular tin-plate base. For our purposes the cardboard can be cut away with a sharp knife. The lid here is slightly domed and provides a stable base which takes solder very well. A piece of perforated or strip board is used as a wiring field.

The tuning coil can be made of copper wire or, better still, silver-plated copper wire with a gauge of 0.8 mm winding 5 turns on an 8 mm diameter former. Keep the interconnections short, especially to the tuning capacitor. The tuning capacitor used here is a trimmer capacitor mounted directly on the ground plane. The second coil in the circuit has 20 turns of 0.2 mm CuL wound directly on the body of a quarter-watt, 10 kΩ resistor.

\begin{figure}[htbp]
\centering
\includegraphics[width=0.6\textwidth]{fig10-2}
\caption{Circuit build using a tinplated earth plane.}
\end{figure}

The antenna should not be too long to avoid interfering with other radio listeners via the regeneration process. The circuit is very sensitive and works with a 10 cm long antenna made simply from a piece of wire. The headphone should ideally be a 400 Ω high-impedance type. A 32 Ω stereo headphone will also work but may be relatively quiet.

At turn on the receiver initially makes a loud noise. You can use a screwdriver on the coil slug to adjust the frequency and when you find an FM station, the noise becomes quieter or completely silent. To hear the FM signal clearly, you need to tune in precisely. This requires some practice and skill but once you've found your favorite station on FM, you don't need to touch the dial again.

The sound quality of this simple receiver is admittedly rather poor. But at least it works with just two transistors. In the early days of FM radio, the superregen design was widely used. Back then, the circuit was built using vacuum tubes. This design, however, eventually fell into disrepute because it simultaneously receives and transmits and can interfere with your neighbors listening pleasure. This also applies to the version of this receiver built here. It is doubtful that you would get the CE stamp of conformity for such a radio. The whole thing is more of an interesting experiment rather than a suggested replacement for the proven superhet. On the other hand, you often find the superregen principle still used in receivers working in simple remote control receivers, radio-controlled sockets, and wireless thermometers.

\begin{figure}[htbp]
\centering
\includegraphics[width=0.6\textwidth]{fig10-3}
\caption{Battery operation.}
\end{figure}

The quench oscillator is just a normal oscillator. Every time the oscillator output releases the tuned VHF circuit, oscillations start building up beginning from almost zero. Thermal noise in the front end helps initiate the tuned circuit oscillation. This stimulation effect sometimes works faster and sometimes more slowly. The individual build-up process therefore takes different lengths of time, which leads to an increase in collector current noise overall. This noise is audible in a superregen receiver when it is not tuned to any station.

The waveform shown in Figure 10.4 triggers on the first left-most edge; noise on waveform can be seen as increasing fuzzyness as the trace moves across to the right side of the screen.

\begin{figure}[htbp]
\centering
\includegraphics[width=0.6\textwidth]{fig10-4}
\caption{Quench waveform with noise.}
\end{figure}

When a received signal is present at the set frequency, this helps to initiate the next envelope of RF oscillations. So it starts a little faster every time. The quenching frequency will therefore increase when receiving a signal. An unmodulated receive signal will produce a stable quenching oscillation with no noise at the output. An amplitude modulated signal will provide differing levels of oscillator start up assistance, which will be reflected in the average value of the change in collector current. An FM signal can be demodulated by tuning to the edge of the oscillator signal to produce an amplitude modulated signal so that both types of modulation can be accommodated. The resulting waveforms can be viewed on an oscilloscope. A Sawtooth waveform at the emitter resistor indicates when a station is being received. The sensitivity of this receiver is so good it can actually work without an antenna! The RF resonant circuit on its own absorbs enough energy for operation.

\section{Vacuum Tube Super Regen Receiver}

The Franzis tube radio (Section 3.9) is a shortwave regenerative receiver. Such circuits will burst into oscillation (motor boating) if the feedback control is turned up too far at higher frequencies. An experiment was carried out to find out whether the radio could be converted into a VHF superregen receiver, thereby removing the need for manual feedback tweaking.

\begin{figure}[htbp]
\centering
\includegraphics[width=0.6\textwidth]{fig10-5}
\caption{VHF superregen with a 6J1.}
\end{figure}

In the first test, I removed the shortwave coil and replaced it with a smaller three turn coil more suitable for use in the VHF band. Connections to the tuning capacitor are also changed, so that the 20 pF range is now used. Instead of the feedback adjustment potentiometer I connected a 0 to 60 V power supply. The 100 kΩ grid resistor no longer connects to the anode, because that would cause negative feedback and dampen oscillations. Now it connects to +6 V at the heater. Some regeneration oscillations could already be observed, but at too low a frequency.

\begin{figure}[htbp]
\centering
\includegraphics[width=0.6\textwidth]{fig10-6}
\caption{The 3-turn VHF coil.}
\end{figure}

The long tracks on the circuit board were causing a problem. I wound an improved coil using thicker wire made up of three turns using the shank of an 8 mm drill bit as a former and soldered the coil very close to the tuning capacitor. The tap point to the cathode is made with a short wire, and the grid is connected to the 100 pF capacitor using short leads. The tracks at the grid and cathode are also cut. Now with all these changes made I was then able to tune the radio across the entire FM band.

When measuring with the oscilloscope, I saw strong evidence of the quenching oscillation signal at the collector of the AF preamp. For this reason I soldered a 100 nF capacitor between the collector and emitter of transistor T2 (see Figure 10.5). For the preliminary testing I just soldered a 470 kΩ resistor between P4 and P5 instead of the volume pot. With these changes, the FM radio worked. I need to supply the anode voltage between 30 and 40 V from a lab power supply. The circuit will not function on 15 V alone like the shortwave version of the radio does.

\section{VHF Receiver using the TDA7088}

This FM radio from Franzis can receive stations in the range of 87.5 MHz to 108 MHz and provides good reception quality. Thanks to the TDA7088 integrated receiver module, you can listen to strong local stations with good sound quality. The receiver's sensitivity is also good enough to pull in distant stations.

\begin{figure}[htbp]
\centering
\includegraphics[width=0.6\textwidth]{fig10-7}
\caption{The VHF FM retro radio.}
\end{figure}

The design of this radio set is reminiscent of portable radios from the 1960s. Back in those days semiconductor devices were replacing vacuum tubes in more and more applications. Transistors consumed less energy and allowed devices like radio sets to be made smaller, battery powered and portable. Apart from that, the principles of a radio receiver design were very similar to that of older tube radios.

\begin{figure}[htbp]
\centering
\includegraphics[width=0.6\textwidth]{fig10-8}
\caption{The TDA7088 uses a pot for tuning.}
\end{figure}

Thanks to the highly integrated nature of the TDA7088 receiver IC, building your own FM radio is now very easy. The single-ended audio amplifier function is more similar to the historical predecessor of a tube radio. The vintage radio uses a two-stage transistor amplifier and gives a moderate output volume powered from two AA cells which will last for up to 200 hours.

\begin{figure}[htbp]
\centering
\includegraphics[width=0.6\textwidth]{fig10-9}
\caption{The kit components.}
\end{figure}

Most FM superheterodyne receivers use an intermediate frequency of 10.7 MHz. The received frequency is first converted to the intermediate frequency and then filtered, amplified, and demodulated. This FM radio is also a superhet that converts its received signal to an intermediate frequency. However, the intermediate frequency is much lower at about 70 kHz. This means that the IF filters do not require matched coils. The FM demodulator is simplified and much more immune to distortion. All the essential stages are included in a single SMD IC, the 16-pin TDA7088. Instead of an air-vaned tuning capacitor like you see in older radio receivers, this radio uses a varactor or varicap diode D1. As the DC voltage across the diode increases its depletion zone becomes wider and its capacitance value decreases. This translates into a higher receive frequency. The only adjustment point is coil SP1, which sets the oscillator frequency lower limit.

\begin{figure}[htbp]
\centering
\includegraphics[width=0.6\textwidth]{fig10-10}
\caption{The fully populated PCB.}
\end{figure}

The circuit board is designed in such a way that all components around the TDA7088 receiver chip have SMD outline. This makes the construction easier. In this radio, the two coils need to be wound by hand using the wire provided and then during setup the coil turns can be stretched out or pressed together slightly to perform fine-tuning.

\begin{figure}[htbp]
\centering
\includegraphics[width=0.6\textwidth]{fig10-11}
\caption{All the parts mounted in the case.}
\end{figure}

The audio power amplifier is a simple Class-A amplifier with the two transistors T1 and T2. The idle current is about 20 mA. The circuit still works with good sound quality when supply voltage falls to 2.2 V.

\begin{figure}[htbp]
\centering
\includegraphics[width=0.6\textwidth]{fig10-10}
\caption{The fully populated PCB.}
\end{figure}

The circuit board is designed in such a way that all components around the TDA7088 receiver chip have SMD outline. This makes the construction easier. In this radio, the two coils need to be wound by hand using the wire provided and then during setup the coil turns can be stretched out or pressed together slightly to perform fine-tuning.

\begin{figure}[htbp]
\centering
\includegraphics[width=0.6\textwidth]{fig10-11}
\caption{All the parts mounted in the case.}
\end{figure}

The audio power amplifier is a simple Class-A amplifier with the two transistors T1 and T2. The idle current is about 20 mA. The circuit still works with good sound quality when supply voltage falls to 2.2 V.

Some of the wired components can be exchanged to change certain properties of the radio. R1 determines the tunable frequency range. A lower resistance will increase the tuning range. This is useful, for example, if you plan to operate the radio with NiMH batteries at 2.4 V. R2 determines the width of the AFC capture range. If you want to receive weak stations in the vicinity of stronger stations, it may be useful to increase R2 up to 1 MΩ to reduce the capture range.

The two connections RE1 and SC1 of the board are not used initially and are intended for later expansion. The TDA7088 was originally developed for push-button tuning. The circuit diagram shows the two push-button switches for reset and scan. If you want to modify receiver accordingly, the PT2\_2 connection to the slider of the frequency controller should be disconnected. At this point, you may wish to install a switch so that the receiver can be tuned either via pushbuttons or the potentiometer.

\section{Stereo Signal Decoding}

The TDA7040 stereo decoder chip is perfect for converting the Franzis FM radio to stereo output. To achieve the necessary bandwidth at the receiver output, the 680 pF SMD capacitor C10 must be removed from the circuit. You just need to unsolder one side of C10, that way it won't get lost, you never know, you may need it again.

\begin{figure}[htbp]
\centering
\includegraphics[width=0.6\textwidth]{fig10-12}
\caption{External components for the TDA7040.}
\end{figure}

Figure 10.13 shows my first attempt at hooking up the TDA7040 decoder to the output of the TDA7088. A set of high-impedance stereo headphones with (2 × 300 ohms) without any filter capacitors are shown at the left and right output pins. This is not ideal, but it is enough for initial testing.

\begin{figure}[htbp]
\centering
\includegraphics[width=0.6\textwidth]{fig10-13}
\caption{Operation into stereo headphones.}
\end{figure}

Using the potentiometer, the oscillator frequency is adjusted to the appropriate level. On the oscilloscope, you can see how the correct 38 kHz subcarrier signal is detected when a stereo signal is present. The capture range is so wide you can just replace the pot with a fixed resistor. At its mid-point the pot measures 50 kΩ if you add that value to the 100 kΩ fixed resistor it gives a value of 150 kΩ. The scope waveform also shows that the decoder is still being overloaded, so the signal here will need to be attenuated. The result is, however, quite impressive: a clear stereo audio signal can be heard from the headphones. It's relatively quiet, but the circuit functions correctly.

To drive speakers at a reasonable volume a small amplifier type TDA7050 is very easy to install and also operates from 3 V. No additional capacitors are required. A 27 kΩ resistor has now been placed between the radio IC output and the stereo decoder to prevent the overloading mentioned above. A twin-gang stereo potentiometer directs the L and R signals to the final amplifier. All of this can be fitted onto a small square of perfboard.

\begin{figure}[htbp]
\centering
\includegraphics[width=0.6\textwidth]{fig10-14}
\caption{Adding an output amplifier.}
\end{figure}

\begin{figure}[htbp]
\centering
\includegraphics[width=0.6\textwidth]{fig10-15}
\caption{VHF radio with loudspeaker and stereo headphone output.}
\end{figure}

The radio now has two volume knobs, one for the mono speaker, which has now been swapped for a higher (32 Ω) impedance coil to give better volume, and one for the stereo amplifier which outputs to the stereo jack socket. A set of headphones can be plugged in here or alternatively there is enough power to drive two 32 Ω speakers. There is now one volume knob for the mono speaker and one for the stereo headphones. This adds flexibility to the way you listen to programs.

\begin{figure}[htbp]
\centering
\includegraphics[width=0.6\textwidth]{fig10-16}
\caption{Decoder and stereo amplifier.}
\end{figure}

If you think building the circuit on a perfboard is too risky, there are other options available. Both ICs can be placed on a shared SMD adapter board. Only a few additional components need to be added to the board. Coupling and filtering capacitors will be installed as part of the wiring. As mentioned above the 100 kΩ trimmer pot connected to pin 3 on the TDA 7040 is unnecessary and was replaced with a fixed 160 kΩ resistor to ground.

\section{A Plug-in VHF Module}

The Franzis-supplied kit "Build your own FM radio" uses a pre-assembled PCB which contains the TDA7088 FM receiver chip together the necessary coils printed on the PCB. A 6-way pinheader strip provides connections for the board to the supply, tuning voltage, antenna, and AF output.

\begin{figure}[htbp]
\centering
\includegraphics[width=0.6\textwidth]{fig10-17}
\caption{The plug-in PCB fitted with the TDA7088/CD9088.}
\end{figure}

A 3 V voltage regulator ensures more stability when tuning. An integrated speaker amplifier provides a good level of volume.

\begin{figure}[htbp]
\centering
\includegraphics[width=0.6\textwidth]{fig10-18}
\caption{VHF radio with voltage regulator and output amplifier.}
\end{figure}

The PCB is fitted with a 6-way pinheader strip and all the other components have flying leads attached with ends that plug into a prototyping plug board so you won't need a soldering iron to assemble this kit. This TDA7088 receiver PCB is also suitable for building simple FM radios to incorporate in your own projects.

\begin{figure}[htbp]
\centering
\includegraphics[width=0.6\textwidth]{fig10-19}
\caption{The complete VHF receiver fits neatly into an enclosure.}
\end{figure}

\section{'Tube Sound' VHF Radio}

The Franzis Retro Radio Deluxe combines a TDA7088 FM receiver with a tube audio amplifier stage and an integrated LM386 power amplifier.

\begin{figure}[htbp]
\centering
\includegraphics[width=0.6\textwidth]{fig10-20}
\caption{Schematic of the VHF receiver with tube audio stage.}
\end{figure}

The PCB contains many SMD components already mounted on the board, including the TDA7088 receiver IC, 15 capacitors, and one resistor. The components with connecting wires, such as all the parts of the audio amplifier, the tube socket, and the coils and components around the radio's diode tuning are the only items that need to be soldered.

\begin{figure}[htbp]
\centering
\includegraphics[width=0.6\textwidth]{fig10-21}
\caption{The compact PCB contains the VHF receiver and tube socket.}
\end{figure}

\begin{figure}[htbp]
\centering
\includegraphics[width=0.6\textwidth]{fig10-22}
\caption{Installation in the case.}
\end{figure}

This radio has a special feature called the 'sound switch'. When you turn it on, it activates the tube and gives the radio a fuller sound. If you just want to casually listen to the news, you can turn off this tube and save power. The sound switch interrupts the tube's heating circuit. When the heater is off, anode current flow will stop. Part of the audio signal is then directed past the tube to the final amplifier. With an active tube, you get more volume and the distinctive changes in sound due to the nonlinearity of the tube's characteristic curve.

The radio is designed so that you can see a red glow from the cathode from the front of the set.

\begin{figure}[htbp]
\centering
\includegraphics[width=0.6\textwidth]{fig10-23}
\caption{Front view showing the tube port top left.}
\end{figure}

\section{The SI4735 DSP Radio}

The SI4735 is a chip measuring 3mm × 3mm that contains a complete radio able to tune to one FM and three AM bands. The company Modul-Bus has developed an adapter PCB on which the chip is mounted. This board can be conveniently used together with a prototyping plug board for carrying out tests to build a radio. In addition, for experimentation a USB/serial converter board type UM232R is used to provide the interface between a PC and the receiver chip via USB. The chip also provides the required 3.3 V operating voltage with its in-built LDO regulator.

\begin{figure}[htbp]
\centering
\includegraphics[width=0.6\textwidth]{fig10-24}
\caption{The SI4735 block diagram.}
\end{figure}

From the block diagram you can see that this radio chip functions as an IQ type receiver, similar to the shortwave receiver design we described earlier (Section 8.6). The main difference here is that all the signal decoding does not need any external PC software because the SI4735 contains a digital signal processor (DSP) which takes care of these tasks. A PC or microcontroller is, however, still required for tuning. The SI4735 has various digital interfaces for communication, and in this case, the I2C bus is used. This uses signals SDA and SCL along with the chip's reset input.

\begin{figure}[htbp]
\centering
\includegraphics[width=0.6\textwidth]{fig10-25}
\caption{Connections to the outside world.}
\end{figure}

The circuit diagram shows a minimal setup for the initial test. Only a short piece of wire was used as an FM antenna. The stereo outputs R and L do not have coupling capacitors because they are already included in the internal amplifier input. A 32 kHz crystal provides the clock signal. The interface to the UM232R requires three resistors. Two of the lines could be connected directly, but this provides greater fault tolerance.

The chip requires a supply voltage of 3.3 V at VDD and VIO. Note that no more than 3.6 V is allowed at VDD. It is important to make sure that the UM232R interface adapter is not accidentally jumpered to 5 V. Unfortunately one chip was damaged during the initial trials due to my own carelessness. Jumper S1 must be in the upper position to provide 3.3 V to VIO.

\begin{figure}[htbp]
\centering
\includegraphics[width=0.6\textwidth]{fig10-26}
\caption{The clock crystal, RS232 interface and SI4735 on a breadboard.}
\end{figure}

In this project, the FT232R chip is used as a serial interface, like COM1 or COM2. The TTL levels with 3.3 V are inverted compared to a real RS232. The DTR and RTS lines form an I2C bus with the additional input line CTS. To control it, a small test program has been written in VB, which can be downloaded from elexs.de.

\begin{figure}[htbp]
\centering
\includegraphics[width=0.6\textwidth]{fig10-27}
\caption{The control interface.}
\end{figure}

You will need a length of wire about 10 cm long to use as a simple antenna to receive FM broadcasts. After starting the program initialize the receiver by clicking on the FM button. The 32 kHz crystal will now start oscillating. Another test for successful initialization is the voltage level at the R and L audio outputs, which should now rise to about 1 V. The FM radio tunes to the first strong station and the stereo signal appears at the output. You can enter another frequency in the frequency field or start a scan. At the top right there is also a volume control slider.

For AM reception, you will need to connect an appropriate antenna or preselector, such as a ferrite antenna. After AM initialization, tuning works similarly to FM, either by direct input or by using the scan function.

The SI4735 module has been used in various projects, including the PC radio and home radio from Modul-Bus, as well as the Elektor DSP radio with a microcontroller and LCD.

\section{PC Radio from a USB Port}

The PC radio allows for control of all functions of the SI4735 via the USB interface. The perforated grid area provides enough space for additional circuitry.

\begin{figure}[htbp]
\centering
\includegraphics[width=0.6\textwidth]{fig10-28}
\caption{The PC Radio.}
\end{figure}

\begin{figure}[htbp]
\centering
\includegraphics[width=0.6\textwidth]{fig10-29}
\caption{Schematic with USB port.}
\end{figure}

Software to interface with and control the radio is available from elexs.de. The standard program is called Si4735Radio5.exe. It was written in Delphi and includes handling of the chip's RDS function.

\begin{figure}[htbp]
\centering
\includegraphics[width=0.6\textwidth]{fig10-30}
\caption{The SI4735 radio with RDS information.}
\end{figure}

\section{The VHF FM Home Radio}

This FM radio has been designed to make it easy and intuitive to use especially for those who are less confident with tasks that require a good level of manual dexterity. Firstly any station in the complete FM band can be tuned using the tuning potentiometer so that the stations are distributed around a rotation angle of 270 degrees. The user would typically only listen to just a few of these stations. Say for example three stations are programmed during setup. Now the tuning knob will now select between only these three stations so that the first 90 degrees of rotation selects station 1 and the next 90 degrees selects station 2, etc. This makes tuning a doddle, no more squinting at a tuning dial or fiddling with band selection. This level of convenience is made possible by a small microcontroller type ATtiny25.

\begin{figure}[htbp]
\centering
\includegraphics[width=0.6\textwidth]{fig10-31}
\caption{Operation with a microcontroller.}
\end{figure}

The small standalone board also includes a simple mono speaker amplifier. If you want to use it, both jumpers must be in position. You can install the board, for example, into an existing speaker cabinet and make a custom radio. It is often useful not to mount the potentiometers on the board but to install some with long spindles elsewhere in case. The stereo jack output can also be used to connect the radio to PC active speakers, for example. The provided housing offers space for a 9 V battery or power can be supplied via the power supply jack.

\begin{figure}[htbp]
\centering
\includegraphics[width=0.6\textwidth]{fig10-32}
\caption{The VHF FM home radio with all controls and mounted components.}
\end{figure}

\begin{figure}[htbp]
\centering
\includegraphics[width=0.6\textwidth]{fig10-33}
\caption{The radio fitted into a case.}
\end{figure}

There are two ways to program the radio: using the pushbutton on the board or via a connected PC.

Programming using the pushbutton: When turning on the radio, the pushbutton must be held down for more than a second to enter programming mode. The radio then immediately searches from 87.5 MHz for the first station. Now with each press of the button, the radio scans to the next station. To save a station, the button must be held down for more than a second. A short press of the button skips the last station heard and searches for the next one. Up to 20 stations can be saved, which are then distributed over the entire 270 degrees rotation of the tuning dial during normal operation. After programming, the radio should be turned off and then back on again.

To program via a PC, a serial cable can be connected to GND and COM (TXD pin). For operation, any terminal program working in text input mode is sufficient. The transfer rate is 1200 baud. The radio can be switched to PC mode at any time during normal operation and can then only be controlled by the PC until the next restart. Using the terminal, you can tune the receiver and assign frequencies between 65 MHz and 108 MHz to the individual memory locations. If, for example, ten stations were previously stored and now only four stations are required, a special end marker must be written to location 5.

Enter: Start the PC mode

8880: Tune to 88.8 MHz

10280: Tune to 102.8 MHz

1: Set to memory location 1

20000: Use as end marker for frequencies > 108 MHz

5: Use end marker in memory location 5

After finishing the programming process, the radio needs to be restarted for the settings to take effect. Alternatively, the receiver can just be used as a PC radio by default. It is also possible to create custom applications with special firmware, such as a kitchen radio for parents with children. A station chosen by one of the children will automatically switch back to the station preferred by a parent after 30 minutes. Or a sleep radio, which turns off automatically after a predetermined time. Those who wish can modify the firmware to do exactly as they please. The possibilities are endless.

In Figure 10.34, an alternative installation suggestion is shown using a retro radio case. The potentiometers on the board have been replaced with ones mounted on the case. The meter monitors the battery voltage. The built-in speaker and the large housing provide a full sound.

\begin{figure}[htbp]
\centering
\includegraphics[width=0.6\textwidth]{fig10-34}
\caption{Using an existing enclosure.}
\end{figure}

\begin{figure}[htbp]
\centering
\includegraphics[width=0.6\textwidth]{fig10-35}
\caption{All the wiring inside using the existing pots.}
\end{figure}

Another possibility would be to use a different custom enclosure. How about a vintage tube radio, for example? There must be loads of these hanging around in junk shops that most people don't have the time or inclination to bring back to life. If you are planning a retro vibe for your home decoration and think a particular vintage set would be the cherry on the cake why not bring the radio back to life by installing a home radio? It doesn't have to be final, but you could simply use the original speaker and just retire the vacuum tube chassis. The result would most likely be a particularly beautiful sound, almost like in the old days, but without the crackles and distortion that dogged radio reception in the early days. One thing is clear; for sound quality, none of these old radios would be able to compete with modern FM broadcasts.

It may also be possible to actually use the output tube EL84 and volume control of the original set, but first you would need to remove all the RF tubes from their sockets. Then, all that's missing is where to position the tuning potentiometer for the home radio. One solution might be a mechanical coupling with the tuning capacitor. It may be less hassle if one of the tone controls were repurposed. Whatever, you would certainly end up with something quite unique!

\section{The Elektor DSP Radio}

A world receiver which works on all the FM, LW, MW, and SW bands, but doesn't have any of the traditional tuning circuitry or controls can be built using Digital Signal Processing (DSP) principles. In the design shown here all the essential functional groups are housed in the tiny 3 mm × 3 mm Si4735 DSP radio chip. In addition to this the radio has a control unit with an LCD, a stereo audio amplifier, and the necessary interfaces to allow the receiver to be controlled by a PC, if desired.

Many radio enthusiasts actually find they need two receivers, one for portable use and one as a stationary receiver with PC control. The Elektor DSP radio shown here can do both. Thanks to the USB interface, PC control is possible at any time, and the entire receiver can be powered via USB. The audio output can also be connected to PC active speakers. The receiver can also be powered from a 6 V battery pack and the circuit has its own integrated audio amplifiers and one (or two) speakers.

\begin{figure}[htbp]
\centering
\includegraphics[width=0.6\textwidth]{fig10-36}
\caption{Stuffed prototype of the Elektor DSP Radio.}
\end{figure}

When it comes to a universal receiver, the first thing I look for is a clean FM reception, preferably in stereo and with RDS station display, so I can see what I am listening to. This receiver offers these features with excellent FM sensitivity and sound quality. It uses the SI4735 chip so RDS is also included.

The second requirement is that the radio's shortwave performance should have the ability to pick up distant AM stations. Here, too, the receiver excels with excellent shortwave reception characteristics, with very high sensitivity combined with good large signal tolerance, allowing the use of long antennas. An effective Automatic Level Control (ALC) brings the received signal into the optimal range, so that low gain antennas can be used without much loss in performance. This receiver's selectivity is also outstanding, and you can freely choose the bandwidth in several stages, which is usually only available with top-end receivers.

This receiver also covers the medium and long wave bands. An antenna input allows for the connection of an external antenna for any frequency bands. If a simple whip antenna or some other indoor aerial is fitted it will usually pick up too much domestic interference so you can alternatively connect to a ferrite antenna here.

\begin{figure}[htbp]
\centering
\includegraphics[width=0.6\textwidth]{fig10-37}
\caption{The receiver schematic.}
\end{figure}

At first glance, the receiver's circuit doesn't show much evidence of typical RF technology or VHF receiver design. That's because all essential functions are integrated into the Si4735. Only the antenna input circuitry reveals the RF nature of this design. The antenna signal from the BNC socket K4 or screw terminal K3 first passes through a diode limiter with D4 and D5. L2 is the FM coil with a value of 0.1 µH. The jumper JP1 is normally in position 3-2, connecting the bottom end of the FM coil to the AM input.

What you can't see in the circuit diagram is that in FM mode, the receiver sets its internal AM 'variable capacitor' to 500 pF, which effectively shorts the FM coil to ground. In AM mode, however, the antenna signal now passes through L2 as an extension coil to the AM resonant circuit made up of L3, L4, L5 and the automatically tuned 'variable capacitor' inside the Si4735 at pin 4 (AMI). The diode switch with D6 and D7 determines which fixed inductances are effective. If necessary, a portion of the coils are be shorted to ground via the 1N4148 diodes. In normal operation, the three jumpers JP2 to JP4 are closed, but alternative input circuits or a ferrite antenna can be connected via the jumper pins. For example, a medium wave ferrite antenna can be connected to JP3, and a shortwave loop to JP2. If a whip antenna is only used for FM, JP1 is set to short pins 1-2.

The stereo output signal of the Si4735 is led to a stereo jack socket as an audio output via C28 and C29, for connection to an external amplifier or powered speakers. The output is short-circuit-proof with an output impedance of 10 k at 80 mVeff approx. Two LM386 ICs are used as audio power amplifiers, allowing speakers to be connected at K5. The maximum power into 8 Ω is about 300 mW. A stereo volume potentiometer is not required in the circuit. The microcontroller IC3 (ATmega168) controls the volume of both channels and all other functions of the Si4735 via software using the I2C bus with its two control signals SDA and SCL. It reads the voltage at the linear potentiometer P1 via the analog input ADC0 and converts it into corresponding commands for the Si4735. Tuning control is implemented via a rotary encoder (ENC1) which connects to two input pins. The four pushbuttons S2 to S5 are additional controls. To show received signal strength a PWM output for connecting an optional S-meter, generates a 500 Hz square wave signal with variable duty cycle and a median voltage between 0 and 3.3 V. Almost any measuring device up to about 1 mA can be connected to it with a suitable resistor. The ATmega168 microcontroller is clocked at 8 MHz, which is independent of the receiver's actual frequency. The receiver derives its reception frequency from a connected clock crystal which runs at 32.768 kHz.

There are three options for powering the radio: through the USB port, a 6 V mains adapter or a battery pack with four cells (4.8 to 6 V). This voltage VIN is applied to the two LM386 amplifiers, the LCD backlight, and the input of the voltage regulator IC1 (LP2950-3.3), which regulates it down to 3.3 V for the radio chip, microcontroller, and LCD. The power switch S1 on the board only switches the voltage from K1 (battery or power supply), while the voltage from the USB port remains on. If you want to save power, you can turn off the LCD backlight by removing the link at JP5. The LCD is still readable without the backlight. The receiver consumes around 50 mA and works down to a voltage of 4.0V so you can expect good battery life.

\begin{figure}[htbp]
\centering
\includegraphics[width=0.6\textwidth]{fig10-38}
\caption{The LC-Display in operation.}
\end{figure}

The display shows the tuning frequency, the antenna voltage in dBµV, and the signal-to-noise ratio (SNR in dB). In FM mode, the lower line displays the station identifier and time sent via RDS.

\section{The BK1079/1068 FM Radio Chip}

A new type of radio chip designed for use in small scanning headphone radios was introduced only a few years ago. The BK1068 from the Chinese company Beken bears a strong resemblance to the BK1079 from the same company; it comes in a slightly larger housing with the unusual 1 mm pin spacing.

\begin{figure}[htbp]
\centering
\includegraphics[width=0.6\textwidth]{fig10-39}
\caption{A small 8-way carrier PCB with the mounted BK1068.}
\end{figure}

\begin{figure}[htbp]
\centering
\includegraphics[width=0.6\textwidth]{fig10-40}
\caption{Block diagram of the BK1079.}
\end{figure}

This IC seems to strongly resemble the DSP radios chips from Silicon Labs like the SI4735, indicating that the BK1079/1068 is actually a DSP radio. This explains its high quality, as the output signal is absolutely clean and shows no traces of the stereo subcarrier signal. Another advantage over the TDA7088 is that the volume can be adjusted internally.

The Seek input and the Volume input have about half the operating voltage in their idle state. The chip detects when the inputs are pulled to GND or VDD potential. Additionally, there is a Reset function that sets the lowest frequency and a Power-Down input (PDN) which turns the chip on and off. This allows the radio to operate without a switch to the battery. When the radio is turned off, it consumes hardly any power and retains the last used settings.

An adapter board for this IC is available from Modul-Bus. This brings the small SMD IC to a handy DIP8 format. The ten connections are reduced to eight pins because GND appears twice (pin 5 and pin 7) and the unused RCLK input is connected to Vdd. This board makes experimenting easy. All you need is a 3 V battery and a few control pushbuttons to build a high-quality radio.

The circuit shows a typical application with pushbuttons for all the functions. The Scan and Vol button inputs are actually tristate with a middle level at half the operating voltage. Therefore, switching the pin to either GND or Vdd assigns two functions to one input.

\begin{figure}[htbp]
\centering
\includegraphics[width=0.6\textwidth]{fig10-41}
\caption{External component connections}
\end{figure}

This module requires minimal external components if some of the switch functions are ignored. The IC starts with maximum volume. A single scan button is sufficient because it automatically switches back to the beginning at the upper end of the band, allowing the IC to scan in a loop. The on/off button switches the IC to the power-down mode and back to the active mode, while keeping all current settings, such as frequency and volume. This means that unlike older scanning radios, you don't need to scan for your preferred station every time. Another advantage is that you can begin scanning in either direction.

In Figure 10.42, there is a setup using six pushbuttons on a test board. A 470 Ω resistor is placed between the two up/down pushbuttons to prevent a short circuit if both pushbuttons are accidentally pressed simultaneously.

\begin{figure}[htbp]
\centering
\includegraphics[width=0.6\textwidth]{fig10-42}
\caption{Test setup testing all of the switch possibilities.}
\end{figure}

The IC is actually intended to drive headphones with 16 Ω impedance but delivers more than ample volume and excellent sound quality. Tests have also shown that an 8 Ω speaker can also be used. An output voltage of up to about 1 Vpp was measured. This volume level is sufficient for most domestic environments without an additional power amplifier. The module is also ideal for sprucing up an old tube radio, either using its own power amplifier or feeding into the existing tube power amp.

\end{document}