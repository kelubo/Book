\documentclass[12pt,UTF8]{ctexbook}

% 设置纸张信息。
% 纸张设置配置文件
% 用于定义书籍的页面尺寸和边距

\usepackage[a4paper,twoside]{geometry}
\geometry{
	left=25mm,
	right=20mm,
	top=25mm,
	bottom=25.4mm,
	headsep=1cm, 
    footskip=1cm,
	bindingoffset=10mm
}

% 设置字体,并解决显示难检字问题。
\xeCJKsetup{AutoFallBack=true}
% 注意:ctexbook类已默认设置SimSun为CJKrmdefault
% 以下设置用于确保字体回退和扩展字体可用
% 仅设置扩展字体以避免与默认设置冲突
\setCJKfamilyfont{hei}{SimHei}
\setCJKfamilyfont{kai}{KaiTi}

% 目录 chapter 级别加点(.)。
\usepackage{titletoc}
\titlecontents{chapter}[0pt]{\vspace{3mm}\bf\addvspace{2pt}\filright}{\contentspush{\thecontentslabel\hspace{0.8em}}}{}{\titlerule*[8pt]{.}\contentspage}

% 支持目录点击跳转
\usepackage[colorlinks,linkcolor=blue,citecolor=blue,urlcolor=blue]{hyperref}

% 设置 part 和 chapter 标题格式。
\ctexset{
	part/name= {第,卷},
	part/number={\chinese{part}},
	chapter/name={第,篇},
	chapter/number={\arabic{chapter}}
}

% 图片相关设置。
\usepackage{graphicx}
\graphicspath{{Images/}}

% 设置署名格式。
\newenvironment{shuming}{\hfill\zihao{4}}

% 注脚每页重新编号,避免编号过大。
\usepackage[perpage]{footmisc}

% 设置古文原文格式。
\newenvironment{yuanwen}{\bfseries\zihao{4}}

% 列表项向右偏移。
\usepackage{enumitem}
% 双语左右对照
\usepackage{paracol}
\columnratio{0.5,0.5} % 设置左右栏比例各占一半
\setlength{\columnsep}{1em} % 设置栏间距
\setlength{\columnseprule}{0.4pt} % 可选:添加栏分隔线
\globalcounter{section} % 同步计数器
\sloppy % 避免换行问题

\title{Radio Builder's Book \\ From Detector to Software Defined Radio \\ 收音机制作之书:从检波器到软件定义无线电}
\author{Burkhard Kainka,DK7JD}
\date{}

\begin{document}

\maketitle

\tableofcontents

\frontmatter

\begin{figure}[htbp]
\centering
\includegraphics[width=0.8\textwidth]{cover}
\caption{Cover Image}
\end{figure}

\chapter{版权页}

\begin{paracol}{2}
This is an Elektor Publication. Elektor is the media brand of Elektor International Media B.V.

PO Box 11, NL-6114-ZG Susteren, The Netherlands

Phone: +31 46 4389444

\switchcolumn
这是一本Elektor出版物。Elektor是Elektor International Media B.V.的媒体品牌。

PO Box 11, NL-6114-ZG Susteren,荷兰

电话:+31 46 4389444

\switchcolumn*
All rights reserved. No part of this book may be reproduced in any material form, including photocopying, or storing in any medium by electronic means and whether or not transiently or incidentally to some other use of this publication, without the written permission of the copyright holder except in accordance with the provisions of the Copyright Designs and Patents Act 1988 or under the terms of a licence issued by the Copyright Licencing Agency Ltd., 90 Tottenham Court Road, London, England W1P 9HE. Applications for the copyright holder's permission to reproduce any part of the publication should be addressed to the publishers.

\switchcolumn
保留所有权利。未经版权所有者的书面许可,不得以任何物质形式(包括影印)复制本书的任何部分,也不得以电子方式将本书的任何部分存储在任何介质中,无论是否临时或偶然用于本出版物的其他用途,除非符合1988年版权、设计和专利法的规定或根据版权许可机构有限公司颁发的许可条款,地址90 Tottenham Court Road, London, England W1P 9HE。申请版权所有者许可复制本出版物的任何部分应致函出版商。

\switchcolumn*
Declaration

The author, editor, and publisher have used their best efforts in ensuring the correctness of the information contained in this book. They do not assume, and hereby disclaim, any liability to any party for any loss or damage caused by errors or omissions in this book, whether such errors or omissions result from negligence, accident or any other cause.

\switchcolumn
声明

作者、编辑和出版商已尽最大努力确保本书中包含的信息的正确性。他们不承担,并特此声明,对任何方因本书中的错误或遗漏造成的任何损失或损害承担任何责任,无论此类错误或遗漏是由于疏忽、事故还是任何其他原因造成的。

\switchcolumn*
All the programs given in the book are Copyright of the Author and Elektor International Media. These programs may only be used for educational purposes. Written permission from the Author or Elektor must be obtained before any of these programs can be used for commercial purposes.

\switchcolumn
本书中提供的所有程序均为作者和Elektor International Media的版权。这些程序只能用于教育目的。在任何这些程序用于商业目的之前,必须获得作者或Elektor的书面许可。

\switchcolumn*
British Library Cataloguing in Publication Data

A catalogue record for this book is available from the British Library

ISBN 978-3-89576-565-0 Print
ISBN 978-3-89576-566-7 eBook

Copyright 2023: Elektor International Media B.V.

Translator: Martin Cooke
Editor: Jan Buiting
Prepress Production: D-Vision, Julian van den Berg

\switchcolumn
英国图书馆出版数据编目

本书的目录记录可从英国图书馆获得。

ISBN 978-3-89576-565-0(印刷版)
ISBN 978-3-89576-566-7(电子版)

版权所有 © 2023: Elektor International Media B.V.

译者:Martin Cooke
编辑:Jan Buiting
印前制作:D-Vision, Julian van den Berg
\end{paracol}

\chapter{前言}

\begin{paracol}{2}
Discover the captivating world of radio technology and unlock the secrets of radio set construction using this comprehensive guide. From the early days of the humble crystal set to the modern wonders of Software-Defined Radios (SDRs), this book takes you on a journey through time and technology. With detailed instructions and step-by-step illustrations, you'll learn how to build and assemble various receivers to understand their various strengths and weaknesses. Practical antenna design and amateur radio rigs are also covered in this inclusive handbook.

\switchcolumn
发现无线电技术的迷人世界,并使用这本综合指南揭开收音机结构的秘密。从早期简陋的矿石机到现代软件定义无线电(SDR)的奇迹,本书将带您穿越时空和技术。通过详细的说明和分步插图,您将学习如何构建和组装各种接收机,以了解它们各自的优势和劣势。实用的天线设计和业余无线电设备也包含在这本综合手册中。
\switchcolumn*
For many years in the early history of electronics, the construction of homebrew radio receivers was the most common entry point to electronics. Nowadays, there are many other routes in, especially through the use of computers, microcontrollers, and digital technology. The analogue roots of electronics are now often overlooked but radio technology is particularly well suited as an introduction to electronics because you will be rewarded with early success from even the most basic circuit. The connection to modern digital technology is also obvious when it comes to modern tuning methods and the use of highly integrated PLL, DDS and DSP radios.

\switchcolumn
在电子学的早期历史中,自制无线电接收机的构建是最常见的入门途径。如今,有许多其他途径,特别是通过计算机、微控制器和数字技术的使用。电子学的模拟根源现在经常被忽视,但无线电技术特别适合作为电子学的入门,因为即使是最基本的电路,您也会获得早期的成功回报。当涉及到现代调谐方法和高度集成的PLL、DDS和DSP收音机的使用时,与现代数字技术的联系也是显而易见的。

\switchcolumn*
This book aims to provide an overview and present a collection of simple projects to encourage budding engineers along the path of discovery. Now in its second edition, many new projects have been added which include important circuits and cover most recent developments. With this book by your side, you will go on to develop your own ideas and design and test your own receivers.

\switchcolumn
本书旨在提供概述并展示一系列简单的项目,以鼓励初学者沿着发现之路前进。现在是第二版,增加了许多新项目,其中包括重要的电路并涵盖了最新的发展。有了这本书在您身边,您将继续发展自己的想法,并设计和测试自己的接收机。

\switchcolumn*
Wishing you every success and crystal clear reception!

\switchcolumn
祝您一切成功,接收清晰!

\switchcolumn*
Burkhard Kainka, DK7JD

www.elektronik-labor.de

\switchcolumn
Burkhard Kainka, DK7JD

www.elektronik-labor.de
\end{paracol}

\mainmatter

\chapter{Introduction 引言}

\begin{paracol}{2}
Building radios is an old hobby that has seen something of a renaissance recently. In addition to the classic and dead simple 'Foxhole' crystal radio set and more sophisticated vacuum tube receivers right up to the more recent software-defined radio projects, there are many aspects to the technology for newcomers to get their teeth into. Recent improvements in semiconductor technology and integrated circuits have allowed sophisticated features such as Direct Digital Synthesis (DDS) and integrated PLL technology to be incorporated into home brew receiver designs to produce radios with surprisingly good specifications.

\switchcolumn
制作收音机是一个古老的爱好,最近又看到了某种复兴。除了经典的极其简单的"战壕"矿石收音机和更复杂的电子管接收机,直到最近的软件定义无线电项目,这项技术有许多方面可供新手深入探索。半导体技术和集成电路的最新改进使得直接数字合成(DDS)和集成PLL技术等复杂功能能够被纳入自制接收机设计中,从而生产出规格惊人的收音机。

\switchcolumn*
This book provides an overview of radio technology and clearly explains the basics of radio receiver design. Using numerous circuits and building plans it guides you step by step along the way. If you want to cook up a simple crystal set or vacuum tube regenerative type receiver, you'll find all the recipes here. As your knowledge and confidence grows you will want to develop your own circuits. That's why the basics of resonant circuits, oscillator configurations and antenna design are all explained clearly here.

\switchcolumn
本书提供了无线电技术的概述,并清楚地解释了无线电接收机设计的基础知识。通过大量的电路和构建计划,它一步步地指导您。如果您想制作一个简单的矿石机或电子管再生式接收机,您将在这里找到所有的配方。随着您的知识和信心的增长,您将想要开发自己的电路。这就是为什么谐振电路、振荡器配置和天线设计的基础都在这里清楚地解释。

\switchcolumn*
Tuned Radio Frequency and Audion type receivers are a step up from the basic crystal detector type receiver. They originally used a single vacuum tube to demodulate the received signal and amplify the resulting baseband output. Receivers like this and simple transistor radios for medium or shortwave reception can be built quickly and easily. They are a lot of fun to play with and are a good way to gain knowledge of RF technology. Modern direct mixing concepts using ring mixers and DDS or PLLs, or simple software-defined radios allow the construction of universal receivers for amateur radio and digital operation. Many things have become easier to build thanks to highly integrated modern chips, but it helps if you also have the essential background information. In this book we only use components that are easily obtainable. Elektor magazine has also created board layouts or finished assemblies and devices that can be sourced from its online store.

\switchcolumn
调谐射频和电子管类型的接收机是基本晶体检测器类型接收机的升级。它们最初使用单个电子管来解调接收信号并放大产生的基带输出。像这样的接收机以及用于中波或短波接收的简单晶体管收音机可以快速轻松地构建。它们很有趣,是获得射频技术知识的好方法。使用环形混频器和DDS或PLL的现代直接混频概念,或简单的软件定义无线电允许为业余无线电和数字操作构建通用接收机。由于高度集成的现代芯片,许多东西变得更容易构建,但如果您也有基本的背景信息,这将有所帮助。在本书中,我们只使用容易获得的组件。Elektor杂志还创建了可以从其在线商店采购的电路板布局或成品组件和设备。

\switchcolumn*
My own interest in RF technology comes from an early fascination with amateur radio; I spent many hours in my youth building and using my own transmitters and receivers. After a long break and following the introduction of new digital broadcasting standards such as DAB and DRM my interest has been rekindled. The challenge for me now was to design a receiver sufficiently stable to decode DRM signals reliably on the short and medium wave bands. I spent time tinkering with vacuum tubes and studied how they were used back in the early days. Although the anode HT was usually high voltage DC, here you experiment with lower anode voltages starting from just 6 V to simplify experiments and make tinkering less hazardous. Other topics explored here are IQ mixer design and building a software-defined radio.

\switchcolumn
我对射频技术的兴趣来自早期对业余无线电的着迷;我在年轻时花了几个小时构建和使用自己的发射机和接收机。经过长时间的休息,随着DAB和DRM等新数字广播标准的引入,我的兴趣重新点燃。现在对我来说,挑战是设计一个足够稳定的接收机,以便在短波和中波波段可靠地解码DRM信号。我花了一些时间摆弄电子管,并研究它们在早期是如何使用的。虽然阳极高压通常是高压直流电,但在这里您可以从仅6V的较低阳极电压开始实验,以简化实验并减少摆弄的危险。这里探索的其他主题是IQ混频器设计和构建软件定义无线电。

\switchcolumn*
Personally, my interest in ham radio has also undergone a revival. For a long time, I had accepted it would not be feasible to install a useful amateur radio antenna in the apartment where I live now. With the help of newer techniques and improved measurement technology, I have been able to build simple and inconspicuous antennas that allow reasonably interference-free reception and can also be used for low power transmitting.

\switchcolumn
就我个人而言,我对业余无线电的兴趣也经历了复兴。很长一段时间以来,我接受了一个事实,即在我现在居住的公寓里安装一个有用的业余无线电天线是不可行的。在较新技术和改进测量技术的帮助下,我已经能够构建简单且不显眼的天线,这些天线允许相对无干扰的接收,也可以用于低功率发射。

\switchcolumn*
Radio technology has always been a theme of my professional work. More recently I have been involved in the development of kits for the Kosmos-Verlag and the Franzis-Verlag, as well as articles and projects for \textit{Elektor Magazine}. Work for the AK Modul-Bus company brought new challenges and resulted in projects for school teaching programs and hobby electronics. Some radio receiver projects on topics such as vacuum tube technology, DRM reception, software-defined radio and DSP radio have been developed using AK Modul-Bus products and turned into \textit{Elektor Magazine} projects.

\switchcolumn
无线电技术一直是我专业工作的主题。最近,我参与了Kosmos-Verlag和Franzis-Verlag的套件开发,以及Elektor杂志的文章和项目。为AK Modul-Bus公司的工作带来了新的挑战,并导致了学校教学计划和业余电子学的项目。一些关于电子管技术、DRM接收、软件定义无线电和DSP收音机等主题的无线电接收机项目已经使用AK Modul-Bus产品开发,并转变为Elektor杂志项目。
\end{paracol}

\chapter{Detector Radios 检测器收音机}

\begin{paracol}{2}
Tuning in to radio broadcasts without a battery or any other power source is only possible with a crystal radio receiver. This simplest of all radio circuits has not lost any of its charm over the decades. In the early days of radio technology, the crystal radio or Foxhole receiver was a widely used concept. Today, just as 90 years ago, it serves as a great introduction to RF technology. You don't necessarily have to recreate authentic historical devices or use homemade crystal detectors. Using a germanium or Schottky diode detector simplifies the process of recovering the baseband signal. You also won't even need an extremely long antenna or highly sensitive headphones. Using an existing speaker amplifier, such as a set of PC active speakers makes building your first radio a piece of cake.

\switchcolumn
在没有电池或任何其他电源的情况下收听无线电广播,只有矿石收音机才能实现。这是所有无线电电路中最简单的,几十年来并没有失去它的魅力。在无线电技术的早期,矿石收音机或战壕收音机是一个广泛使用的概念。今天,就像90年前一样,它是射频技术的绝佳入门。您不一定非要复制真实的历史设备或使用自制的晶体检测器。使用锗或肖特基二极管检测器简化了恢复基带信号的过程。您甚至不需要极长的天线或高灵敏度的耳机。使用现有的扬声器放大器,例如一组PC有源扬声器,使构建您的第一台收音机变得轻而易举。
\end{paracol}

\section{The Diode Radio 二极管收音机}

\begin{paracol}{2}
The simplest receiver you can build consists of a long length of wire for use as an antenna, a ground connection, a germanium (Ge) diode and a high-impedance headphone. The germanium diode may be difficult to source so a modern Schottky diode can be substituted. The radio needs no external power supply because the signal picked up by the antenna provides all the energy necessary, which is why it needs to be relatively long. Usually, a 10 meter length of wire will do the job. This design assumes a high-impedance headphone of 2 kΩ, but it will work just as well with a 600 Ω type. Standard low-impedance headphones of modern design are usually 32 Ω but they can also be used with a suitable transformer (see section 2.2) to provide impedance matching.

\switchcolumn
您可以构建的最简单的接收机由一根长导线作为天线、接地连接、锗(Ge)二极管和高阻抗耳机组成。锗二极管可能很难找到,因此可以用现代肖特基二极管代替。收音机不需要外部电源,因为天线接收的信号提供了所有必要的能量,这就是为什么它需要相对较长。通常10米长的导线就可以完成这项工作。该设计假设使用2 kΩ的高阻抗耳机,但它与600 Ω类型的效果一样好。现代设计的标准低阻抗耳机通常32 Ω,但它们也可以与合适的变压器一起使用(2.2节)以提供阻抗匹配。
\end{paracol}

\begin{figure}[htbp]
\centering
\includegraphics[width=0.6\textwidth]{fig2-1}
\caption{The Diode Radio}
\end{figure}

\begin{paracol}{2}
You can use any Ge diodes from type AA112 to AA144 or any Schottky diodes from BAT41 to BAT86. This simple radio is not selective, which means it receives all strong stations at the same time. Unless a strong local station is overpowering all the others, you should be able to hear some stations with fluctuating volume, especially at dusk.

\switchcolumn
您可以使用从AA112到AA144的任何锗二极管或从BAT41到BAT86的任何肖特基二极管。这个简单的收音机没有选择性,这意味着它同时接收所有强电台。除非一个强大的本地电台压倒了所有其他电台,否则您应该能够听到一些音量波动的电台,特别是在黄昏时分。

\switchcolumn*
To achieve desired selectivity, a resonant circuit consisting of a coil and tuning capacitor can be added to the circuit. Using a tuning capacitor of up to 320 pF and a coil of 300 µH, will allow the entire medium-wave band can be covered. The coil consists of 90 turns of wire wound onto a 4 cm diameter cardboard roll to make the necessary air-cored coil.

\switchcolumn
为了实现所需的选择性,可以在电路中添加由线圈和调谐电容器组成的谐振电路。使用高320 pF的调谐电容器300 µH的线圈,可以覆盖整个中波波段。线圈由90圈导线绕4厘米直径的纸板卷上,以制作必要的空心线圈。
\end{paracol}

\begin{figure}[htbp]
\centering
\includegraphics[width=0.6\textwidth]{fig2-2}
\caption{Diode receiver circuit with resonant circuit.}
\end{figure}

\begin{paracol}{2}
This radio is not very selective and doesn't achieve much in terms of output volume. It's important to carefully adjust the antenna and rectifier; this can be achieved by adding tap or connection points along the receiving coil. In section 2.2, a medium wave receiver is described which uses adjustable matching.

\switchcolumn
这个收音机不是很有选择性,在输出音量方面也没有太大表现。重要的是仔细调整天线和整流器;这可以通过沿接收线圈添加抽头或连接点来实现。在2.2节中,描述了一个使用可调匹配的中波接收机。

\switchcolumn*
You may wonder why a diode is necessary in a receiver circuit. To answer that one you need to delve into a little bit of radio theory. A transmitter broadcasts high frequency electromagnetic waves into free space via a transmitting mast. The broadcast radiates in all directions and induces a small signal in an antenna at the receiving location. Transmitters that send on the medium wave band transfer their information, such as speech and music, in the form of amplitude modulation (AM) of the carrier frequency. The radio frequency carrier amplitude changes in time with the low-frequency (baseband) voice or music signals.

\switchcolumn
您可能想知道为什么接收机电路中需要二极管。要回答这个问题,您需要深入研究一点无线电理论。发射机通过发射塔将高频电磁波广播到自由空间。广播向各个方向辐射,并在接收位置的天线中感应出一个小信号。在中波波段发送的发射机以载波频率的幅度调制(AM)的形式传输它们的信息,如语音和音乐。射频载波幅度随低频(基带)语音或音乐信号随时间变化。
\end{paracol}

\begin{figure}[htbp]
\centering
\includegraphics[width=0.6\textwidth]{fig2-3}
\caption{Amplitude modulation}
\end{figure}

\begin{paracol}{2}
The received radio signal remains inaudible even in headphones because our ears are only sensitive to sound pressure waves up to about 20 kHz. The low-frequency signal carrying the voice and music information needs to be recovered from the radio carrier wave. This is where the diode comes in; using just one diode you can demodulate the RF signal. The average current of the rectified signal corresponds to the original modulated AF signal.

\switchcolumn
接收到的无线电信号即使在耳机中也是听不见的,因为我们的耳朵只对高达20 kHz的声压波敏感。携带语音和音乐信息的低频信号需要从无线电载波中恢复。这就是二极管发挥作用的地方;仅使用一个二极管,您就可以解调射频信号。整流信号的平均电流对应于原始调制音频信号。
\end{paracol}

\begin{figure}[htbp]
\centering
\includegraphics[width=0.6\textwidth]{fig2-4}
\caption{Demodulation using a rectifier}
\end{figure}

\begin{paracol}{2}
The first detector radios used crystal detectors. Lead sulfide (galena) or a piece of pyrite crystal was used for this purpose. Both are sulfur compounds and occur in nature as ores (lead ore; iron ore).

\switchcolumn
最早的检测器收音机使用晶体检测器。硫化铅(方铅矿)或一块黄铁矿晶体用于此目的。两者都是硫化合物,在自然界中以矿石形式存在(铅矿石;铁矿石)。

\switchcolumn*
Figure 2.5 shows a crystal holder with a lead sulfide crystal from the early days of radio technology. A spiral spring made of steel wire known as a cat’s whisker contacts the crystal surface. The characteristics of the semiconductor junction formed at the crystal surface can be tested with an oscilloscope component tester. You will need to experiment a bit to find a suitable spot on the crystal surface and to recognize a typical diode characteristic curve on the component tester trace.

\switchcolumn
图2.5展示了无线电技术早期带有硫化铅晶体的晶体支架。由钢丝制成的螺旋弹簧(称为猫须)接触晶体表面。晶体表面形成的半导体结的特性可以用示波器元件测试仪进行测试。您需要稍微实验一下,在晶体表面找到合适的位置,并在元件测试仪的轨迹上识别出典型的二极管特性曲线。

\end{paracol}

\begin{figure}[htbp]
\centering
\includegraphics[width=0.6\textwidth]{fig2-5}
\caption{An original detector crystal mount}
\end{figure}

\begin{paracol}{2}
The crystal can be used successfully to build a diode radio. Numerous strong stations can be heard without the need for any additional amplifier. Even today, it is possible to build a detector using these natural minerals. Pyrite forms regular, gold-colored cuboid crystals in rock. Lead sulfide is black with areas of metallic shiny facets on its surface. A sewing needle can be used as the cat's whisker detector. You will need to test various points on the crystal surface until contact achieves a good rectification characteristic.

\switchcolumn
晶体可以成功地用于构建二极管收音机。无需任何额外的放大器即可听到许多强电台。即使在今天,也可以使用这些天然矿物构建检测器。黄铁矿在岩石中形成规则的、金色的长方体晶体。硫化铅是黑色的,表面有金属光泽的切面区域。缝纫针可以用作猫须检测器。您需要测试晶体表面的各个点,直到接触实现良好的整流特性。
\end{paracol}

\begin{figure}[htbp]
\centering
\includegraphics[width=0.6\textwidth]{fig2-6}
\caption{Naturally formed Pyrite and Lead sulfide.}
\end{figure}

\section{Headphone Adapter 耳机适配器}

\begin{paracol}{2}
Vintage circuit diagrams for detector radios assume that headphones shown on the circuit will be high-impedance types with 2000 Ω driver coils. These were standard back then. Nowadays a typical set of headphones will use 32 Ω driver coils which will be too low to function properly in the original circuit. You can, however, use a small transformer to provide the necessary impedance matching. A transformer salvaged from a small mains adapter can be used here. If the mains adapter has switchable taps, (3/4.5/6/9/12 V) on the secondary winding you may be able to use these to optimize the impedance match. Remove the transformer and connect the secondary winding to the headphones and the primary winding to the circuit where the high impedance phones would normally be connected.

\switchcolumn
检测器收音机的旧电路图假设电路中显示的耳机将是具有2000 Ω驱动线圈的高阻抗类型。这些在当时是标准的。如今,典型的耳机将使用32 Ω驱动线圈,这在原始电路中太低而无法正常工作。但是,您可以使用一个小变压器来提供必要的阻抗匹配。这里可以使用从小型电源适配器中拆下的变压器。如果电源适配器在次级绕组上有可切换的抽头(3/4.5/6/9/12 V)),您可能能够使用这些抽头来优化阻抗匹配。拆下变压器,将次级绕组连接到耳机,将初级绕组连接到通常连接高阻抗耳机的电路。

\switchcolumn*
In a diode radio, correct antenna matching is the key to success because you cannot afford to waste any of the received RF energy. The receiver coil, therefore, has several tap points. Using a total of 80 turns of 'Litz' wire on a 10 mm diameter ferrite rod, makes sure you will be able to cover the entire medium wave band. Long antennas should be connected to a lower tap of the coil to not overly dampen the resonant circuit at the input. Try connecting the long antenna to each of the winding taps to find which one gives the best reception. Two coupling capacitors are also shown connected at the coil end. Experiment with the aerial connection, a higher value of capacitance results in stronger coupling.

\switchcolumn
在二极管收音机中,正确的天线匹配是成功的关键,因为您不能浪费任何接收到的射频能量。因此,接收线圈有几个抽头点。在10毫米直径的磁棒上使用总共80圈利兹线,确保您能够覆盖整个中波波段。长天线应该连接到线圈的较低抽头,以免过度抑制输入端的谐振电路。尝试将长天线连接到每个绕组抽头,找出哪个提供最佳接收。线圈端还显示了两个耦合电容器。尝试天线连接,较高的电容值会导致更强的耦合。
\end{paracol}

\begin{figure}[htbp]
\centering
\includegraphics[width=0.6\textwidth]{fig2-7}
\caption{Low impedance headphones with transformer impedance matching.}
\end{figure}

\begin{paracol}{2}
For such a simple radio, a good antenna is crucial. If your house is fitted with metal rainwater guttering, this can make a good antenna. The guttering should not have a connection to ground potential. A zinc gutter will often be cemented into a drainage pipe near the ground and thereby will be insulated. All you need now is a connection wire, and that should make a really good antenna. In case the reception is still too quiet for headphones, you can connect the output to a set of PC's active speakers.

\switchcolumn
对于这样一个简单的收音机,一个好的天线至关重要。如果您的房子装有金属雨水槽,这可以成为一个很好的天线。雨水槽不应与地电位有连接。锌制雨水槽通常会用水泥固定在靠近地面的排水管中,从而被绝缘。现在您只需要一根连接线,这应该能成为一个非常好的天线。如果接收对于耳机来说仍然太安静,您可以将输出连接到一套PC有源扬声器。

\switchcolumn*
Another good antenna is sometimes the heating system of an apartment. Although the pipes are usually grounded at some point, the total length of all the piping can effectively act as a loop antenna. In many cases, this can result in high received signal levels.

\switchcolumn
另一个好的天线有时是公寓的供暖系统。虽然管道通常在某处接地,但所有管道的总长度可以有效地充当环形天线。在许多情况下,这会导致高接收信号电平。
\end{paracol}

\section{A Detector for Shortwave 短波检测器}

\begin{paracol}{2}
Looking at old "plans" to build detector radios, they are usually designed to receive signals from local stations in the medium wave band. These stations are becoming rarer and may even be unavailable now as more countries shut down their medium wave transmitters. Some countries such as the UK, Italy, France, and Spain however still broadcast in the band. Transmitting on shortwave has the advantage of covering much greater distances. Many countries have their own international broadcasting services designed to inform and entertain overseas listeners. The tried-and-true AM radio is, therefore, just as active on shortwave as ever.

\switchcolumn
查看构建检测器收音机的"计划",它们通常设计用于接收中波波段本地电台的信号。随着越来越多的国家关闭其中波发射机,这些电台变得越来越罕见,甚至可能现在无法收听。然而,一些国家如英国、意大利、法国和西班牙仍然在该波段广播。在短波上广播具有覆盖更远距离的优势。许多国家都有自己的国际广播服务,旨在为海外听众提供信息和娱乐。因此,久经考验的AM收音机在短波上仍然和以前一样活跃。

\switchcolumn*
Tuning in to higher frequencies requires smaller coils which are much easier to make. While a good medium wave coil needs a ferrite rod and a coil wound from hard-to-find 'Litz' wire, on shortwave, you can use standard insulated copper wire. A special coil former with a ferrite core is not required; you can use any insulated wire.

\switchcolumn
调谐到更高的频率需要更小的线圈,这更容易制作。虽然一个好的中波线圈需要磁棒和用难以找到的"利兹"线绕制的线圈,但在短波上,您可以使用标准的绝缘铜线。不需要带有磁芯的特殊线圈骨架;您可以使用任何绝缘线。

\switchcolumn*
For the first attempt, a coil with a total of 25 turns with four taps should be wound. I used the plastic body of a banana plug which measures 8 mm diameter but you could use a ball-point pen body. Two holes spaced 1 cm apart help to fix the wire ends. Then, wind 5 turns, make a tap point, and apply the next turns. The finished coil connections can be soldered to a 6-way pinheader strip.

\switchcolumn
对于第一次尝试,应该绕制一个总共25圈带有四个抽头的线圈。我使用了直8毫米的香蕉插头的塑料体,但您可以使用圆珠笔的笔身。两个相1厘米的孔有助于固定导线端。然后,5圈,做一个抽头点,然后绕下一圈。完成的线圈连接可以焊接6路排针条上。
\end{paracol}

\begin{figure}[htbp]
\centering
\includegraphics[width=0.6\textwidth]{fig2-8}
\caption{Dual trimmer capacitor and shortwave coil assembly}
\end{figure}

\begin{paracol}{2}
The entire radio can be built on an experimental plug board. Pins have been soldered to the variable connections so that it can be easily plugged into the prototyping plug board. The advantage of this construction method is that it allows for easy experimentation to try out other circuit mods.

\switchcolumn
整个收音机可以在实验插板上构建。引脚已焊接到可变连接上,以便可以轻松插入到原型插板中。这种构建方法的优点是允许轻松实验以尝试其他电路修改。
\end{paracol}

\begin{figure}[htbp]
\centering
\includegraphics[width=0.6\textwidth]{fig2-9}
\caption{Testing using a plug board.}
\end{figure}

\begin{paracol}{2}
The respective taps for the antenna connection and the diode on the coil can be experimentally adjusted. This tuning capacitor is dual gang with both halves connected in parallel. If only the upper range above 10 MHz is to be received, the lower-valued half of the tuning capacitor rated at 80 pF will be sufficient.

\switchcolumn
天线连接和线圈上二极管的相应抽头可以实验性地调整。这个调谐电容器是双联的,两半并联连接。如果只接收10 MHz以上的较高范围,额定值为80 pF的调谐电容器的较低值半部分就足够了。
\end{paracol}

\begin{figure}[htbp]
\centering
\includegraphics[width=0.6\textwidth]{fig2-10}
\caption{Schematic of the shortwave detector.}
\end{figure}

\begin{paracol}{2}
The radio requires a high-impedance headphone, such as a piezoceramic crystal earpiece or dynamic headphones with a 2 kΩ resistor placed in series with each capsule. Low-impedance 32 Ω types cannot be used directly and require an impedance transformer (see Section 2.2). Medium-impedance headphones with 600 Ω can also be used directly.

\switchcolumn
收音机需要高阻抗耳机,例如压电陶瓷晶体耳机或每个耳机串联2 kΩ电阻的动圈耳机。低阻抗32 Ω类型不能直接使用,需要阻抗变压器(见2.2节)600 Ω的中等阻抗耳机也可以直接使用。

\switchcolumn*
A germanium or a Schottky diode can be used as the detector. Both of these diodes have a low forward threshold voltage. A germanium diode also has reasonably low conductivity in the reverse direction, which is important when using a high-impedance crystal earpiece.

\switchcolumn
可以使用锗二极管或肖特基二极管作为检测器。这两种二极管都具有较低的正向阈值电压。锗二极管在反向方向也具有相当低的导电性,这在高阻抗晶体耳机时很重要。
\end{paracol}

\begin{paracol}{2}
With a Schottky diode, the earpiece can become electrically charged like a capacitor. As the charge builds up the diode is completely reverse biased. In this case, an additional 100 kΩ resistor should be connected in parallel with the earphone to dissipate any charge build up to ground.

\switchcolumn
使用肖特基二极管时,耳机可能会像电容器一样带电。随着电荷的积累,二极管会完全反向偏置。在这种情况下,应该在耳机两端并联一个额外的100 kΩ电阻,将积累的电荷释放到地。

\switchcolumn*
For best reception you need an aerial wire a good 10 m long suspended as high as possible. But even a short 3 m length of wire hooked up high up around a room will provide reasonable signal strength for initial trials. With careful tweaking of the tuning capacitor, several stations will be heard with sufficient volume, especially in the evening. Often you will hear two or three stations at the same time and tuning. The usual fluctuations in field strength on shortwave reception mean that one station may be clearer than another at different times of day. While individual radio bands are clear, nearby stations cannot be clearly separated. The selectivity of this radio design is not yet optimal.

\switchcolumn
为了获得最佳接收效果,您需要一根至少10米长的天线,尽可能悬挂得高一些。但即使是3米长的短线,挂在房间高处,也能为初步试验提供合理的信号强度。仔细调整调谐电容器,就能听到几个音量足够的电台,尤其是在晚上。通常你会同时听到两三个电台,需要仔细调谐。短波接收时场强的常见波动意味着在一天的不同时间,某个电台可能比另一个更清晰。虽然各个无线电波段是清晰的,但邻近的电台无法清楚分离。这种收音机设计的选择性还不是最佳的。

\switchcolumn*
The taps to the coil shown in the circuit diagram are only rough guidelines. Here you can try to find the optimum balance between volume and selectivity, which is easily achievable with the plug board construction method used here. The following rules of thumb apply:

• Lower taps points for the antenna and diode improve the receiver’s selectivity but reduce output volume.

• Long antennas should be connected to one of the lower tap points. Connecting to points higher up results in reduced volume and lower selectivity.

These relationships can be easily verified experimentally. Later on you will take a closer look at the theory to support these general rules.

\switchcolumn
电路图中显示的线圈抽头只是粗略的指导。在这里你可以尝试找到音量和选择性之间的最佳平衡,使用这里的实验插板构建方法很容易实现。适用以下经验法则:

• 天线和二极管的较低抽头点可以提高接收器的选择性,但会降低输出音量。

• 长天线应该连接到较低的抽头点之一。连接到较高的点会导致音量降低和选择性下降。

这些关系可以通过实验轻松验证。稍后你将更深入地了解支持这些通用规则的理论。
\end{paracol}

%\section{Silicon Diode Detector}
%\textbf{中文:硅二极管检测器}
%
%Germanium diodes are rarely used nowadays. Silicon diodes are very popular for all sorts of applications and the 1N4148 is the %most commonly used universal diode. The circuit shown uses a silicon diode with an additional bias voltage applied. In addition, a %coupling capacitor is used here to connect the signal to an amplifier input.
%
%\textbf{中文:锗二极管现在很少使用。硅二极管在各种应用中非常流行,1N4148是最常用的通用二极管。所示的电路使用硅二极管并施加了额外的偏置电压。此%外,这里使用耦合电容器将信号连接到放大器输入。}
%
%\begin{figure}[htbp]
%\centering
%%%\includegraphics[width=0.6\textwidth]{fig2-8}
%\caption{Silicon Diode Detector with Bias}
%\textbf{中文:带有偏置的硅二极管检测器}
%\end{figure}
%
%A silicon diode requires a forward voltage bias of around 0.5 V before any significant current starts to flow. Since the received %RF signal at the resonant circuit only rarely reaches such high levels, you may not hear any recovered signal. Germanium diodes, %however, have a forward voltage threshold below 0.2 V so that smaller signals can be recovered. To overcome the high threshold of %silicon diodes, you can set up a small DC current of about 10 µA to flow through the diode. This will forward bias the junction %allowing received signals below 100 mV to be demodulated.
%
%\textbf{中文:硅二极管需要约0.5 V的正向电压偏置才能开始有显著的电流流动。由于谐振电路接收到的射频信号很少达到如此高的电平,您可能听不到任何恢复%的信号。然而,锗二极管的正向阈值电压低0.2 V,因此可以恢复较小的信号。为了克服硅二极管的高阈值,您可以设置约10 µA的小直流电流流过二极管。这将正向%偏置结,允许解调低100 mV的接收信号。}
%
%Although the circuit can be operated directly with headphones, it works better with a speaker amplifier. A set of active PC %speakers, for example, will work well here. The built-in audio amplifier provides sufficient amplification and a high input %resistance in the order of 100 kΩ. This results in less damping of the resonant circuit and gives better selectivity. In addition, %compared to using headphones, you can connect the antenna to a lower tap on the coil to improve selectivity and boost volume level.
%
%\textbf{中文:虽然电路可以直接用耳机操作,但使用扬声器放大器效果更好。例如,一套PC有源扬声器在这里会工作得很好。内置音频放大器提供足够的放大和%100 kΩ的高输入电阻。这导致谐振电路的阻尼更少,并提供更好的选择性。此外,与使用耳机相比,您可以将天线连接到线圈的较低抽头,以改善选择性并提高音%量。}
%
%\section{Coils and Resonant Circuits}
%\textbf{中文:线圈和谐振电路}
%
%In order to build each circuit described here we've provided the necessary inductance and specific measurements. However, %sometimes you may need to modify the circuit or use a different coil body; in that case, you'll need to determine the number of %turns yourself. It's also possible that you have some old coils salvaged from redundant equipment that you can modify and adapt. %Regardless, it's useful to know how to calculate coils yourself.
%
%\textbf{中文:为了构建这里描述的每个电路,我们提供了必要的电感和具体测量值。然而,有时您可能需要修改电路或使用不同的线圈体;在这种情况下,您需要%自己确定圈数。也有可能您有一些从多余设备中拆下的旧线圈,可以修改和适应。无论如何,知道如何自己计算线圈是有用的。}
%
%\begin{figure}[htbp]
%\centering
%%%\includegraphics[width=0.6\textwidth]{fig2-9}
%\caption{Air-core Coil Construction}
%\textbf{中文:空心线圈构造}
%\end{figure}
%
%There are basically two types of coils: those wound on a magnetizable core (like ferrite or iron powder) and those without a core, %known as air-core coils. Let's focus on air-core coils first. For instance, a coil for a shortwave resonant circuit has 20 turns, %a diameter of 16 mm, and a coil length of 35 mm. It has an inductance of approximately 3 µH and, when combined with a variable %capacitor up to 300 pF, can reach a lower frequency limit of around 5.3 MHz. We'll show you how to calculate this and introduce a %simple tool that can make the process easier.
%
%\textbf{中文:基本上有两种类型的线圈:绕在可磁化磁芯(如铁氧体或铁粉)上的线圈和没有磁芯的线圈,称为空心线圈。让我们先关注空心线圈。例如,短波谐%振电路的线圈有20圈,直径16毫米,线圈长35毫米。它的电感约3 µH,当与高300 pF的可变电容器结合时,可以达到5.3 MHz的下限频率。我们将向您展示如何计算%这一点,并介绍一个可以使过程更简单的简单工具。}
%
%\begin{figure}[htbp]
%\centering
%%%\includegraphics[width=0.6\textwidth]{fig2-10}
%\caption{Coil Winding Tool}
%\textbf{中文:线圈绕制工具}
%\end{figure}
%
%In general, the following formula applies to a long coil where \textit{l} > \textit{D} where \textit{n} is the number of turns, %\textit{A} is the cross-sectional area in square meters, and \textit{l} is the length in meters:
%
%\[ L = \mu_0 \times n^2 \times A / l \]
%
%where the magnetic field constant \(\mu_0\) equals:
%
%\[ 4\pi \times 10^{-7} \text{Vs/Am} = 1.2466 \times 10^{-6} \text{Vs/Am} \]
%
%\textbf{中文:一般来说,以下公式适用于长线圈,其中\textit{l} > \textit{D},其中\textit{n}是圈数,\textit{A}是以平方米为单位的横截面积,%\textit{l}是以米为单位的长度:
%
%\[ L = \mu_0 \times n^2 \times A / l \]
%
%其中磁场常数\(\mu_0\)等于
%
%\[ 4\pi \times 10^{-7} \text{Vs/Am} = 1.2466 \times 10^{-6} \text{Vs/Am} \]
%
%This formula actually only applies to an infinitely long coil but can be used as a useful approximation up to a length of \textit%{l = D}. With a short coil of the same number of turns, the magnetic coupling between individual turns increases, resulting in a %higher inductance. Conversely, stretching out the turns reduces inductance, which can sometimes be used to adjust coils.
%
%\textbf{中文:这个公式实际上只适用于无限长的线圈,但可以用作有用的近似值,直到长度为\textit{l = D}。对于具有相同圈数的短线圈,各个圈之间的磁耦%合增加,导致更高的电感。相反,拉伸圈会减少电感,这有时可以用来调整线圈。}
%
%The above formula can be simplified for a circular coil cross-section, where the diameter D and length l of the coil are given in %mm, to the following approximation formula:
%
%\[ L = 1\,\text{nH} \times n^2 \times D^2/\text{mm}^2 / (l/\text{mm}) \]
%
%This formula uses the approximation of π × π = 10 which introduces an error of approximately 1.3\%. This is generally an %acceptable simplification; you cannot expect high accuracy since the shape of the coil, especially the ratio of length and %thickness, wire thickness, and even the location where the coil is mounted, all influence the final value of inductance achieved. %In practice you can expect to achieve accuracy within 10\% for an air-core calculation.
%
%\textbf{中文:上述公式可以简化为圆形线圈横截面,其中线圈的直径D和长度l以毫米为单位,简化为以下近似公式:}
%
%\[ L = 1\,\text{nH} \times n^2 \times D^2/\text{mm}^2 / (l/\text{mm}) \]
%
%\textbf{中文:这个公式使用π × π = 10的近似值,引入了约1.3\%的误差。这通常是一个可接受的简化;您不能期望高精度,因为线圈的形状,特别是长度和厚度%的比例、导线厚度,甚至线圈安装的位置,都会影响最终实现的电感值。在实践中,您可以期望空心线圈计算达到10\%以内的精度。}
%
%RF coil formers with ferrite screw cores are often used. The coil inductance increases by up to four times or more when a ferrite %core is used. By changing the insertion depth of the screw-in core the coil value can be adjusted. Ferrite cores are manufactured %for use with certain frequency bands in which they have lowest energy losses.
%
%\textbf{中文:带有铁氧体螺钉磁芯的射频线圈骨架经常使用。当使用铁氧体磁芯时,线圈电感增加多达四倍或更多。通过改变螺钉磁芯的插入深度,可以调整线圈%值。铁氧体磁芯是为在具有最低能量损耗的某些频段中使用而制造的。}
%
%Much larger inductances can be achieved by using closed cores with or without an air gap. The air gap reduces the inductance of %the coil, but allows for greater magnetization, i.e., the core itself only reaches magnetic saturation at higher currents. Common %types of cores include ring cores, transformer cores in E-I shape, and closed pot cores.
%
%\textbf{中文:通过使用有或没有气隙的闭合磁芯可以实现更大的电感。气隙减少了线圈的电感,但允许更大的磁化,即磁芯本身仅在更高电流下达到磁饱和。常见%的磁芯类型包括环形磁芯、E-I形状的变压器磁芯和闭合罐形磁芯。}
%
%The inductance depends heavily on the number of turns, the material used, and the geometry of the core. A theoretical calculation, %like for the air-core coil, is not so simple. The manufacturer will instead provide an AL value in nH/\textit{n²} for each type of %core.
%
%\[ L = A_L \times n^2 \]
%
%\textbf{中文:电感在很大程度上取决于圈数、使用的材料和磁芯的几何形状。像空心线圈那样的理论计算并不那么简单。制造商将为每种类型的磁芯提供以nH/n²%为单位的AL值:}
%
%\[ L = A_L \times n^2 \]
%
%For example, an Amidon T37-2 ring core has an inductance of 40 µH at 100 turns, which corresponds to an AL value of 4 nH/n². %Reducing the winding to 30 turns, the inductance becomes:
%
%\[ L = 30 \times 30 \times 4\,\text{nH} = 3600\,\text{nH} = 3.6\,\mu\text{H}. \]
%
%\textbf{中文:例如,Amidon T37-2环形磁芯200圈时具有40 µH的电感,对应4 nH/n²的AL值。将绕组减少30圈,电感变为:}
%
%\[ L = 30 \times 30 \times 4\,\text{nH} = 3600\,\text{nH} = 3.6\,\mu\text{H}. \]
%
%The ring core coil is suitable for building an RF resonant circuit, just like an air-core coil. Besides the AL value, the intended %frequency range of a core is also important. The Amidon type xxx-2 has the color code red which indicates it is suitable for %frequencies up to 30 MHz.
%
%\textbf{中文:环形磁芯线圈适合构建射频谐振电路,就像空心线圈一样。除了AL值,磁芯的预期频率范围也很重要。Amidon xxx-2类型有红色代码,表示它适用%于高30 MHz的频率。}
%
%\section{Resonant Frequency and Bandwidth}
%\textbf{中文:谐振频率和带宽}
%
%If you connect a coil and a capacitor, a resonant circuit is created. Electrical energy can oscillate back and forth between the %coil and capacitor, similar to the decaying swing of a pendulum, the period of the swing indicates the resonant frequency \textit%{f}. The electrical circuit responds to a short pulse of current with a diminishing oscillatory voltage waveform.
%
%\textbf{中文:如果您连接线圈和电容器,就会创建一个谐振电路。电能可以在线圈和电容器之间来回振荡,类似于钟摆的衰减摆动,摆动的周期表示谐振频率%\textit{f}。电路以衰减振荡电压波形响应短电流脉冲。}
%
%The formula for calculating the resonance frequency is:
%
%\[ f = 1 / (2\pi\sqrt{LC}) \]
%
%\textbf{中文:计算谐振频率的公式是:}
%
%\[ f = 1 / (2\pi\sqrt{LC}) \]
%
%Tuned circuits are often used in electrical circuits that process a range of different signal frequencies or mixed frequencies. %Current and voltages that flow in such circuits will vary according to the signal frequency. The parallel resonant circuit has a %complex impedance Z with a sharp maximum value at the resonant frequency f0. At this frequency, RC = RL and the currents through %the coil and capacitor cancel out exactly due to their total phase difference of 180 degrees. An ideal oscillating circuit with no %losses would have infinitely large impedance at the resonant frequency.
%
%\textbf{中文:调谐电路经常用于处理一系列不同信号频率或混合频率的电路中。在这样的电路中流动的电流和电压将根据信号频率而变化。并联谐振电路在谐振频%率f0处具有尖锐最大值的复阻抗Z。在这个频率下,RC = RL,通过线圈和电容器的电流由于它们的总相位差180度而完全抵消。没有损耗的理想振荡电路在谐振频率%处将具有无限大的阻抗。}
%
%In practice however, damping of the oscillation occurs because of energy losses in the resistance of the coil wire, magnetic %losses of the coil core, and electromagnetic radiation, resulting in a finite resonant resistance. To simplify you can add all the %losses together and assign them as a parallel loss resistance R.
%
%\textbf{中文:然而在实践中,由于线圈导线电阻的能量损耗、线圈磁芯的磁损耗和电磁辐射,振荡的阻尼会发生,导致有限的谐振电阻。为了简化,您可以将所有%损耗加在一起,并将它们指定为并联损耗电阻R。}
%
%Each resonant circuit has a property called the Quality factor or just Q which is inversely proportional to the bandwidth of the %circuit. Q can be easily determined when the parallel damping resistance R is related to the inductive resistance RL = 2πfL or to %the capacitive resistance RC = 1 / (2πfC) at the resonant frequency.
%
%\[ Q = R / R_L \text{ or } Q = R / R_C \]
%
%\textbf{中文:每个谐振电路都有一个称为品质因数或简称Q的属性,它与电路的带宽成反比。当并联阻尼电阻R与谐振频率下的感抗RL = 2πfL或容抗RC = 1 / %(2πfC)相关时,可以容易地确定Q:}
%
%\[ Q = R / R_L \text{ or } Q = R / R_C \]
%
%If a resonant circuit is excited with a constant alternating current I of variable frequency, or through an alternating current %source with high internal resistance, then the resonant circuit voltage is proportional to the magnitude of the complex impedance %Z. At resonance, the voltage is highest. The smaller the damping of the vibration due to energy losses of any kind, or the larger %the quality of the resonant circuit, the higher the resonant voltage rises.
%
%\textbf{中文:如果谐振电路由可变频率的恒定交流电流I激励,或者通过具有高内阻的交流电流源,那么谐振电路电压与复阻抗Z的大小成正比。在谐振时,电压最%高。由于任何种类的能量损耗导致的振动阻尼越小,或者谐振电路的品质越大,谐振电压上升得越高。}
%
%On both sides of the resonant frequency, points on the resonance curve can be determined at which the voltage has dropped to a %factor of 1 / 2 = 0.707 = -3 dB. The frequency separation of these points is referred to as the bandwidth b of the circuit. %Between the resonant frequency f0, bandwidth b, and quality factor Q of the circuit, the relationship is
%
%\[ b = f_0 / Q. \]
%
%\textbf{中文:在谐振频率的两侧,可以确定谐振曲线上的点,在这些点上电压已下降1 / 2 = 0.707 = -3 dB的系数。这些点的频率间隔被称为电路的带宽b。在%谐振频率f0、带宽b和电路的品质因数Q之间,关系是}
%
%\[ b = f_0 / Q. \]
%
%The damping of the circuit, and therefore its quality, is practically always caused by intrinsic series and parallel resistances. %The series resistance is due to the wire winding, but for a certain frequency, it is greater than the DC resistance due to the %skin effect. The parallel resistance is determined by the connection impedance in the circuit. However, an iron or ferrite core %also has losses that can be represented by a parallel resistor. With the same inductance, a coil with a core requires fewer turns %and therefore incurs lower copper losses. At the same time there are now losses in the core to consider. At very high frequencies %of around 100 MHz, pure air coils made of thick, silver-plated wire perform better, while at medium frequencies of around 10 MHz, %the best quality is achieved with a closed core such as a toroidal core. Air coils, on the other hand, are an alternative down to %about 1 MHz. Coils and transformers used in the audio frequency range, however, are almost always built with a core.
%
%\textbf{中文:电路的阻尼,因此其品质,实际上总是由内在的串联和并联电阻引起的。串联电阻是由于导线绕组,但对于特定频率,由于趋肤效应,它大于直流电%阻。并联电阻由电路中的连接阻抗决定。然而,铁或铁氧体磁芯也有损耗,可以用并联电阻表示。对于相同的电感,带有磁芯的线圈需要更少的圈数,因此产生较低%的铜损耗。同时,现在还要考虑磁芯中的损耗。在100 MHz的非常高的频率下,由粗的镀银线制成的纯空心线圈表现更好,而在10 MHz的中等频率下,使用闭合磁芯%(如环形磁芯)可以实现最佳品质。另一方面,空心线圈是低至1 MHz的替代方案。然而,在音频频率范围内使用的线圈和变压器几乎总是带有磁芯构建。}
%
%You can expect to get a quality factor \textit{Q} of up to 100 by being careful with coil construction. A resonant circuit is %however also damped by the external circuitry to which it is connected to or by an antenna. This damping effect can to some extent %be mitigated by ensuring a loose coupling of the resonant circuit by using a small auxiliary winding, a tap point on the coil, or %a suitable coupling capacitor. When a coil connects directly to the input of an amplifier, its input impedance should be very high %to lessen the damping effect.
%
%\textbf{中文:通过仔细的线圈构造,您可以期望获得高300的品质因数\textit{Q}。然而,谐振电路也被其连接到的外部电路或天线阻尼。这种阻尼效应可以通过%使用小辅助绕组、线圈上的抽头点或合适的耦合电容器来确保谐振电路的松散耦合,从而在一定程度上减轻。当线圈直接连接到放大器的输入时,其输入阻抗应该非%常高,以减少阻尼效应。}
%
%A small Visual Basic program can be found on the author's website called LCFR which has been written to simplify the calculation %of coils and resonant circuits. The program calculates the inductance of air coils and coils with a known \textit{A_L} value. In %addition, the resonance frequency, and the inductive resistance RL of the coil at this frequency can be determined if a value of %capacitance is given in addition to the inductance.
%
%\textbf{中文:可以在作者的网站上找到一个名为LCFR的小型Visual Basic程序,该程序已编写用于简化线圈和谐振电路的计算。该程序计算空心线圈和具有已知%\textit{A_L}值的线圈的电感。此外,如果除了电感之外还给出了电容值,则可以确定谐振频率以及线圈在该频率下的感抗RL。}
%
%Here are a few examples:
%\textbf{中文:这里有几个例子:}
%
%To wind a coil with 300 µH for a medium-wave detector radio on a cardboard roll with a diameter of 42 mm, assuming a wire diameter %of 0.5 mm, you would need about 90 turns. The tuning capacitor must be at least 320 pF to tune the medium-wave range starting from %530 kHz.
%
%\textbf{中文:要在直径2毫米的纸板卷上为中等波检测器收音机绕400 µH的线圈,假设导线直径0.5毫米,您需要大约300圈。调谐电容器必须至少320 pF,以%530 kHz开始调谐中波范围。}
%
%For higher frequencies, you need fewer turns. For example, a coil in a shortwave receiver has 25 turns, a diameter of 8 mm, and a %coil length of 10 mm, resulting in an inductance of 3 µH. With a capacitance of 320 pF, you can tune down to 4.4 MHz.
%
%\textbf{中文:对于更高的频率,您需要更少的圈数。例如,短波接收机中的线圈有25圈,直径8毫米,线圈长12毫米,导线3 µH的电感。使用320 pF的电容,您可%以调谐低5.4 MHz。}
%
%The previous examples used air coils. But how can you use a ferrite core? Often you don't have exact data on the core material %properties, so you have to estimate by how much the core increases the inductance or decreases the frequency. For a coil in the %shortwave range, with \textit{n} = 18 turns, \textit{L} = 12 mm, and \textit{D} = 8 mm, you can estimate an inductance of about 1.%7 µH and a frequency of 7.3 MHz using a capacitance of 275 pF for an air coil. With a variable capacitor of 275 pF and a fully %inserted ferrite core, experiments show a lower frequency limit of 3.7 MHz, or an inductance of about 6.8 µH. Using an adjustable %screw core, the frequency can be halved, and the inductance can be increased up to four times.
%
%\textbf{中文:前面的例子使用了空心线圈。但是如何使用铁氧体磁芯呢?通常您没有关于磁芯材料属性的精确数据,因此您必须估计磁芯增加电感或降低频率的程%度。对于短波范围内的线圈,\textit{n} = 18圈,\textit{L} = 12毫米,\textit{D} = 8毫米,使用275 pF的电容,您可以估计空心线圈约1.7 µH的电感和%7.3 MHz的频率。使用275 pF的可变电容器和完全插入的铁氧体磁芯,实验显示下限频率3.7 MHz,或6.8 µH的电感。使用可调螺钉磁芯,频率可以减半,电感可以%增加多达四倍。}
%
%Similarly, a longer ferrite rod for the medium-wave range can increase the inductance by about ten times. Roughly speaking, for a %coil on a ferrite rod to achieve the same inductance requires only about one-third of the turns of an air coil of the same size.
%
%\textbf{中文:同样,用于中波范围的较长铁氧体棒可以将电感增加约十倍。粗略地说,对于铁氧体棒上的线圈,要实现相同的电感,只需要相同尺寸空心线圈圈数%的大约三分之一。}
%
%The resonant frequency of a resonant circuit can change significantly when installed in the circuit. Especially at higher %frequencies, line capacitances, for example, can have an effect. For this reason you often have to make corrections afterwards or %plan for adjustment options from the beginning, such as using screw cores or trimmers. For large changes, the following rules of %thumb, which can be derived directly from the formulae given and can be simulated with the LCFR program, often help: doubling the %number of turns causes quadruple the inductance and half the frequency with the same value of capacitance. So, the frequency is %inversely proportional to the number of turns. On the other hand, the frequency is inversely proportional to the square of the %capacitance. Therefore, you can double the frequency with a quarter of the capacitance value. For example, to tune a frequency %range from 1 to 3 using a variable capacitor, the capacitance ratio must be at least 1 to 9.
%
%\textbf{中文:谐振电路的谐振频率在安装在电路中时可能会显著变化。特别是在较高频率下,线路电容,例如,可能会产生影响。由于这个原因,您经常必须事后%进行更正或从一开始就计划调整选项,例如使用螺钉磁芯或微调电容器。对于大的变化,以下经验法则,可以直接从给出的公式推导出来,并且可以用LCFR程序模%拟,通常有帮助:圈数加倍会导致电感四倍,并且在相同电容值下频率减半。因此,频率与圈数成反比。另一方面,频率与电容的平方成反比。因此,您可以用电容%值的四分之一将频率加倍。例如,要使用可变电容器将频率范围从1调谐3,电容比必须至少1:9。}
%
%Let's take a closer look at the achievable bandwidth and quality with an example. Let's say you have a shortwave resonant circuit %with \textit{L} = 3 µH, \textit{C} = 240 pF, \textit{f} = 5.9 MHz, \textit{RL} = \textit{RC} = 112 Ω. With a thick wire or a good %ferrite core, you can achieve an unloaded quality factor (\textit{Q}) of 100, which means the unloaded bandwidth would be about %\textit{b} = 6000 kHz / 100 = 60 kHz. The resonant resistance would be 112 Ω × 100 = 11.2 kΩ, say roughly 10 kΩ.
%
%\textbf{中文:让我们通过一个例子更仔细地看看可实现的带宽和品质。假设您有一个短波谐振电路,\textit{L} = 3 µH,\textit{C} = 240 pF,\textit%{f} = 5.9 MHz,\textit{RL} = \textit{RC} = 112 Ω。使用粗导线或良好的铁氧体磁芯,您可以实现300的无载品质因数(\textit{Q}),这意味着无载带宽%约为\textit{b} = 6000 kHz / 100 = 60 kHz。谐振电阻将为112 Ω × 100 = 11.2 kΩ,大约为10 kΩ。}
%
%The actual losses, caused by copper resistance, are around 1 Ω, while the DC resistance is much lower. However, the effective loss %resistance increases due to skin effect, where the RF current migrates to the thin outer layer of the conductor. To reduce losses, %coils for the medium and long wave bands are usually wound with multi-stranded, individually insulated copper wires called %"RF-Litz" wire.
%
%\textbf{中文:由铜电阻引起的实际损耗约1 Ω,而直流电阻要低得多。然而,由于趋肤效应,有效损耗电阻增加,其中射频电流迁移到导体的薄外层。为了减少损%耗,中波和长波波段的线圈通常用多股、单独绝缘的铜线绕制,称为"RF-Litz"线。}
%
%In a crystal receiver, the working \textit{Q} factor shouldn't be set too high to get a good output volume. By loading the circuit %with about 10 kΩ in parallel, you can achieve good power matching and volume, with a working value of \textit{Q} = 50 and a %bandwidth of 120 kHz. This example indicates that neighboring stations within a band will not be separated. The actual value of Q %also depends on the antenna used and its coupling.
%
%\textbf{中文:在晶体接收机中,工作\textit{Q}因数不应设置得太高以获得良好的输出音量。通过在并联中用约10 kΩ加载电路,您可以实现良好的功率匹配和音%量,工作值为\textit{Q} = 50,带宽为120 kHz。这个例子表明,波段内的相邻电台将不会被分离。Q的实际值也取决于使用的天线及其耦合。}
%
%\section{The Vacuum Tube Detector}
%\textbf{中文:电子管检测器}
%
%In addition to semiconductor diodes, there are also tubes that can serve the same function. A typical tube diode that can be used %as an RF detector is the EAA91. This is a dual diode with a heater voltage of 6.3 V/0.3 A. Unlike a germanium or Schottky diode %detector, a radio built using this type of detector consumes a continuous 1.8 W of power just to recover the signal.
%
%\textbf{中文:除了半导体二极管外,还有可以发挥相同功能的电子管。可以用作射频检测器的典型电子管二极管是EAA91。这是一个双二极管,加热电压6.3 V/0.%3 A。与锗或肖特基二极管检测器不同,使用这种检测器构建的收音机仅为了恢复信号就连续消耗0.8 W的功率。}
%
%Unlike a germanium or silicon diode or even the crystal detector made from galena, the tube diode does not require the signal to %exceed a minimum voltage threshold before current starts to flow. Even without a positive anode voltage, some electrons will find %their way to the anode. A short-circuit current of about 30 µA can be measured. With a load resistance of 1 MΩ, the tube has a %grid voltage of 0.5 V, thereby creating its own appropriate bias voltage.
%
%\textbf{中文:与锗或硅二极管甚至由方铅矿制成的晶体检测器不同,电子管二极管不需要信号超过最小电压阈值才开始有电流流动。即使没有正阳极电压,一些电%子也会找到通往阳极的路径。可以测量到30 µA的短路电流。使用1 MΩ的负载电阻,电子管具有-0.5 V的栅极电压,从而创建其自己的适当偏置电压。}
%
%Using the two diodes in the tube together with a twin gang tuning capacitor (one range value covers shortwave while the other is %suitable for medium wave) and a few other components, you can build a dual-band radio covering both medium wave and shortwave. %This essentially builds two completely independent radios, with band switching occurring after rectification. The output signal %can then be fed to, for example, a set of PC active speakers. The selectivity is good in both frequency bands because the %rectifier circuit has very high impedance.
%
%\textbf{中文:使用电子管中的两个二极管以及双联调谐电容器(一个范围值覆盖短波,另一个适合中波)和其他几个组件,您可以构建一个覆盖中波和短波的双波%段收音机。这本质上构建了两个完全独立的收音机,波段切换发生在整流之后。然后可以将输出信号馈送到,例如,一套PC有源扬声器。由于整流电路具有非常高的%阻抗,两个频段的选择性都很好。}
%
%The two-band radio is coupled to the antenna through small coupling capacitors of around 30 pF. With a sufficiently long antenna, %you can receive numerous stations. Distant European stations can be heard on shortwave and medium wave, especially in the evening.
%
%\textbf{中文:双波段收音机通过30 pF的小耦合电容器耦合到天线。使用足够长的天线,您可以接收许多电台。在短波和中波上可以听到遥远的欧洲电台,特别是%在晚上。}
%
%\section{Diode Radio with Active Regeneration}
%\textbf{中文:具有有源再生的二极管收音机}
%
%You have seen that a simple shortwave detector is not very sensitive or selective but with the help of a regeneration circuit, its %performance can be significantly improved. The additional circuit can compensate for losses in the oscillator circuit. With this %design, the RF signal is amplified by a transistor and fed back (in phase) into the oscillating receive coil. With carefully %adjusted amplification, the feedback can compensate for all losses. In this state the oscillating circuit will be optimally damped %and have a very high quality factor up to about \textit{Q} = 1000. This high \textit{Q} factor means that broadcast stations only %10 kHz apart can be separated and very weak stations will be easily picked up.
%
%\textbf{中文:您已经看到,简单的短波检测器不是很灵敏或选择性不强,但借助再生电路,其性能可以显著提高。附加电路可以补偿振荡电路中的损耗。使用这种%设计,射频信号由晶体管放大并反馈(同相)到振荡接收线圈。通过仔细调整放大,反馈可以补偿所有损耗。在这种状态下,振荡电路将得到最佳阻尼,并且具有非%常高的品质因数,高达约\textit{Q} = 1000。这个高\textit{Q}因数意味着仅相差10 kHz的广播电台可以被分离,并且非常弱的电台将很容易被接收到。}
%
%The regeneration feedback is produced here using an NPN transistor but you could also use a tube. Even with this regeneration %circuit the radio is still classed as a detector receiver. It's only when the tube or transistor itself also replaces the function %of the detector diode that the radio becomes an Audion receiver. Audion refers back to Lee De Forest's patent which turned out to %be the prototype of the triode vacuum tube. Up until then crystal detectors were all you had to demodulate radio signals. De %Forest's Audion tube took care of demodulation and amplification of the recovered baseband signal.
%
%\textbf{中文:这里使用NPN晶体管产生再生反馈,但您也可以使用电子管。即使有这个再生电路,收音机仍然被归类为检测器接收机。只有当电子管或晶体管本身%也取代检测器二极管的功能时,收音机才成为Audion接收机。Audion指的是Lee De Forest的专利,该专利原来是三极电子管的原型。在此之前,晶体检测器是您解%调无线电信号的唯一选择。De Forest的Audion电子管负责解调和放大恢复的基带信号。}
%
%The feedback circuit used here could essentially be almost any oscillator circuit. Here, a Hartley oscillator with emitter %feedback is used because the necessary coil taps are already available on the receiver coil. An alternative is the circuit shown %further down in section 4.4 which uses two PNP transistors.
%
%\textbf{中文:这里使用的反馈电路本质上可以是几乎任何振荡电路。这里使用具有发射极反馈的Hartley振荡器,因为接收线圈上已经提供了必要的线圈抽头。另%一种选择3.4节中显示的电路,它使用两个PNP晶体管。}
%
%This circuit can be easily connected to an amplifier, such as a set of active PC speakers, making it a great shortwave radio. The %antenna doesn't need to be very long—a roughly one meter long whip should do. To use the receiver, you tune it to a station and %adjust the regeneration level until good output volume is achieved. If you crank up the regen pot too far, the set will burst into %oscillation and what you thought was a receiver has now become a small carrier wave transmitter. The coil providing the feedback %to the receiver coil is also known as a 'tickler' coil and can be electrically separate from the main coil.
%
%\textbf{中文:这个电路可以很容易地连接到放大器,例如一套PC有源扬声器,使其成为一个很棒的短波收音机。天线不需要很长——大约一米长的鞭状天线就足够%了。要使用接收机,您将其调谐到一个电台并调整再生水平,直到实现良好的输出音量。如果您将再生电位器调得太高,设备将突然振荡,您认为的接收机现在已成%为一个小型载波发射机。向接收线圈提供反馈的线圈也称为"激磁线圈",可以在电气上与主线圈分开。}
%
%When properly adjusted, the regenerative receiver can hold its own against any conventional shortwave radio. The audio is quite %pleasant, and unlike a simple superheterodyne radio, there are no interference noises due to poor image rejection. During periods %of strong signal fading, there isn't any unpleasant sound distortion, just temporary reductions in signal strength.
%
%\textbf{中文:当正确调整时,再生接收机可以与任何常规短波收音机相媲美。音频相当令人愉快,并且与简单的超外差收音机不同,由于镜像抑制不良而没有干扰%噪声。在强信号衰落期间,没有任何令人不愉快的声音失真,只是信号强度的暂时降低。}
%
%For those who think that a detector receiver using a battery and amplifier is cheating, don't worry—you can remove the battery and %connect a crystal earphone instead. The radio still functions without the active regeneration signal, but with far less %sensitivity.
%
%\textbf{中文:对于那些认为使用电池和放大器的检测器接收机是作弊的人,别担心——您可以移除电池并改接晶体耳机。收音机在没有主动再生信号的情况下仍然工%作,但灵敏度要低得多。}
%
%\section{Regeneration using Tubes}
%\textbf{中文:使用电子管的再生}
%
%Instead of a transistor, a tube can also be used to reinforce the oscillations in the resonant circuit. Figure 2.21 shows just %such a circuit using an EC92 tube. In addition to the anode voltage of 12 V, a filament voltage of 6 V is also required. The %circuit shows a great similarity to the corresponding transistor circuit shown in Figure 2.19. The RF signal is applied to the %grid via an RC network. The amplified RF current is coupled back into the resonant circuit via a tap. To adjust the regeneration %level here, you can change the anode voltage.
%
%\textbf{中文:除了晶体管之外,电子管也可以用来加强谐振电路中的振荡。图2.21显示了使用EC92电子管的这样一个电路。除了12 V的阳极电压外,还需要6 V的%灯丝电压。该电路与图2.19中显示的相应晶体管电路显示出极大的相似性。射频信号通过RC网络施加到栅极。放大的射频电流通过抽头耦合回谐振电路。要在这里调%整再生水平,您可以改变阳极电压。}
%
%In this circuit, te triode EC92 is operated with a low anode voltage of maximum 12 V, although it is actually intended for %operation with voltages from 100 V upwards. The achievable amplification is however still sufficient for our purpose here. At low %anode voltage, the tube has only a small usable slope range. Therefore, it must be placed directly in the resonant circuit, i.e. %the grid is at the hot end of the resonant circuit, and the cathode is a tap approximately in the middle of the coil.
%
%\textbf{中文:在这个电路中,三极管EC92以最大12 V的低阳极电压运行,尽管它实际上旨在以100 V及以上的电压运行。然而,可实现的放大对于我们的目的来说%仍然足够。在低阳极电压下,电子管只有很小的可用斜率范围。因此,它必须直接放置在谐振电路中,即栅极位于谐振电路的热端,阴极是线圈中间附近的抽头。}
%
%Almost any other triode can be substituted for the EC92. Double triodes such as the ECC81 or ECC82, which can be operated with a %heater voltage of 12 V, are also well suited. If both halves are connected in parallel, the slope characteristic doubles. If the %amplification is not sufficient with one tube and no oscillation can be observed even with the feedback control turned fully up, a %smaller grid resistor should be used. The tube then enters the range of lower grid bias and higher anode currents.
%
%\textbf{中文:几乎任何其他三极管都可以替代EC92。可以在12 V加热电压下运行的双三极管(如ECC81或ECC82)也非常适合。如果两半部分并联连接,斜率特性%会加倍。如果一个电子管的放大不足,并且即使反馈控制完全向上也无法观察到振荡,则应使用较小的栅极电阻。然后电子管进入较低栅极偏置和较高阳极电流的范%围。}
%
%A pentode can also be used; Figure 2.22 shows a circuit using an EF95. Here, the anode is fixed at the operating voltage, while %the screen grid voltage is adjusted to change the characteristic slope. A relatively small 10 kΩ grid resistor guarantees safe %oscillation even with strong antenna coupling. The tube circuit is already largely similar to the regenerative tube receiver %described in section 3.5.
%
%\textbf{中文:五极管也可以使用;2.22显示了使用EF95的电路。在这里,阳极固定在工作电压,而屏幕栅极电压被调整以改变特性斜率。相对较小的10 kΩ栅极电%阻即使在天线强耦合的情况下也能保证安全振荡。电子管电路已经在3.5节中描述的再生电子管接收机在很大程度上相似。}
%
%\chapter{Tube Regenerative Receivers}
%\textbf{中文:电子管再生接收机}
%
%Listening to distant radio stations or 'DXing' is a hobby which gained popularity after the introduction of the regenerative %receiver design. This type of receiver was usually built using one or two tubes together with a tuning capacitor to adjust the %reception frequency and a feedback control to control the oscillating regeneration circuit. Those who became experts at tweaking %the controls were able to draw down the faintest of signals from the ether. During the subsequent era of superheterodyne %receivers, it was easy to forget just how well the simpler regenerative receiver can perform. A good shortwave receiver working on %this principle can be built using just one or two tubes.
%
%\textbf{中文:收听遥远的电台"DXing"是一种在再生接收机设计引入后获得流行的爱好。这种类型的接收机通常使用一个或两个电子管以及一个调谐电容器来调整%接收频率和一个反馈控制来控制振荡再生电路。那些成为调整控制专家的人能够从以太中提取最微弱的信号。在随后的超外差接收机时代,很容易忘记简单的再生接%收机可以表现得多么出色。一个基于此原理工作的良好短波接收机可以仅使用一个或两个电子管构建。}
%
%\section{Triode AM Receivers}
%\textbf{中文:三极管AM接收机}
%
%As mentioned above, an Audion receiver is characterized by the fact that both RF demodulation and signal amplification are %performed with just one active device. This provides better sensitivity and more amplification overall. The one valve shortwave %radio shown in figure 3.1, therefore, has significantly more sensitivity than the tube detector receiver. The circuit can be used %with either an EC92 or an ECC81 vacuum tube. For this simple circuit, a high-impedance earpiece of at least 600 Ω is required
%
%\textbf{中文:如上所述,Audion接收机的特点是射频解调和信号放大都仅由一个有源器件执行。这提供了更好的灵敏度和更多的整体放大。因此,3.1中显示的单%电子管短波收音机比电子管检测器接收机具有显著更高的灵敏度。该电路可以使用EC92或ECC81电子管。对于这个简单的电路,需要至200 Ω的高阻抗耳机}
%
%Two different processes contribute to the demodulation of the RF signal. Grid rectification is based on the rising grid current at %positive input voltage. Anode rectification is based on the curvature of the \textit{I}\textsubscript{a} / \textit{V}\textsubscript%{g} characteristic curve of the tube. Depending on the tube and its operating point, one of the two rectification processes %predominates.
%
%\textbf{中文:两个不同的过程有助于射频信号的解调。栅极整流基于正输入电压下的上升栅极电流。阳极整流基于电子管的\textit{I}\textsubscript{a} / %\textit{V}\textsubscript{g}特性曲线的曲率。根据电子管及其工作点,两个整流过程中的一个占主导地位。}
%
%\subsection{Anode rectification}
%
%The characteristic slope of a tube depends on the grid voltage and increases with increasing input voltage. The non-linearity of %the characteristic curve causes the average anode current to increase with increasing RF amplitude. It is easy to see that %effective demodulation is only possible if the input voltage swing is not too narrow. Very small signal excursions will only drive %the tube over a more linear region of the curve so that the rectification effect will be smaller and the recovered baseband will %have lower amplitude.
%
%\textbf{中文:电子管的特性斜率取决于栅极电压,并随着输入电压的增加而增加。特性曲线的非线性导致平均阳极电流随着射频幅度的增加而增加。很容易看出,%只有当输入电压摆动不太窄时,有效解调才可能。非常小的信号偏移只会驱动电子管通过曲线的更线性区域,因此整流效应将更小,恢复的基带将具有较低的幅度。}
%
%\subsection{Grid Rectification}
%
%The grid-cathode structure forms a rectifier. As the RF input voltage increases a larger grid current flows, which charges the %grid capacitance negatively. This reduces the average anode current. The output signal is thus exactly out of phase with anode %rectification. Again, a non-linear characteristic curve is responsible, this time the grid current characteristic curve.
%
%The limited non-linear region ensures, as with a semiconductor diode, that significant demodulation only occurs at sufficiently %large signal amplitudes.
%
%\textbf{中文:栅极-阴极结构形成一个整流器。随着射频输入电压的增加,更大的栅极电流流动,这使栅极电容负向充电。这降低了平均阳极电流。因此,输出信%号与阳极整流完全反相。同样,非线性特性曲线是原因,这次是栅极电流特性曲线。有限的非线性区域确保,与半导体二极管一样,只有足够大的信号幅度才会发生%显著的解调。}
%
%It is not easy to predict which of the two effects predominates in practice. However, a assessment can be made based on the phase %of the output signal. Figure 3.3 shows a measurement of an AM receiver using the ECC82. The lower channel shows an amplitude %modulated RF signal directly at the control grid, and the upper channel shows the voltage at the anode. It can be seen that the %anode voltage drops when the RF amplitude is large. Here, the tube is therefore operating in anode rectification mode, i.e., the %grid current plays no significant role.
%
%\textbf{中文:在实践中预测这两种效应中哪一种占主导地位并不容易。然而,可以根据输出信号的相位进行评估。图3.3显示了使用ECC82的AM接收机的测量结%果。下方通道显示控制栅极处的幅度调制射频信号,上方通道显示阳极处的电压。可以看出,当射频幅度较大时,阳极电压下降。在这里,电子管因此以阳极整流模%式运行,即栅极电流不起重要作用。}
%
%Figure 3.4 on the other hand displays the results using an EF95 pentode in an Audion configuration. As you can see, the average %grid voltage actually decreases with larger RF signal levels, meaning that the grid capacitor is charged negatively. Consequently, %the anode current decreases and the anode voltage increases. In other words, the tube functions as a grid rectifier.
%
%\textbf{中文:另一方面,图3.4显示了在Audion配置中使用EF95五极管的结果。如您所见,平均栅极电压实际上随着更大的射频信号水平而降低,这意味着栅极电%容被负向充电。因此,阳极电流降低,阳极电压升高。换句话说,电子管作为栅极整流器运行。}
%
%In a given circuit, the type of demodulation depends on the tube, its operating point, the grid resistor, the grid capacitor, and %the current modulation frequency. If the grid time constant is large, the grid voltage will not be able to keep up with a high %modulation frequency. There may therefore be frequency ranges where both effects cancel each other out, resulting in little or no %signal demodulation.
%
%\textbf{中文:在给定的电路中,解调类型取决于电子管、其工作点、栅极电阻、栅极电容和电流调制频率。如果栅极时间常数很大,栅极电压将无法跟上高调制频%率。因此可能存在两种效应相互抵消的频率范围,导致很少或没有信号解调。}
%
%Both operating attributes of the Audion based on non linear characteristics make it clear why the audio amplitude increases %disproportionately with the RF amplitude. As a result, weaker stations are practically inaudible because the carrier amplitude is %not sufficient to produce any significant demodulation. This weakness can be overcome by adding feedback to the circuit as %described in section 3.3.
%
%\textbf{中文:基于非线性特性的Audion的两种操作属性清楚地说明了为什么音频幅度与射频幅度不成比例地增加。因此,较弱的电台实际上听不见,因为载波幅度%不足以产生任何显著的解调。这个弱点可以通过向电路添加反馈来克服,如3.3节中所述。}
%
%\section{A Two-stage Receiver}
%\textbf{中文:两级接收机}
%
%Without using any deliberate feedback, good results can be achieved by increasing the overall gain of the receiver by just adding %a second stage. Figure 3.5 shows a tested circuit in the medium-wave range with a low-voltage ECC86 tube, which can operate with %an anode voltage of only 6 V. A high-impedance earpiece was used for testing.
%
%\textbf{中文:不使用任何故意的反馈,可以通过仅添加第二级来增加接收机的整体增益,从而获得良好的结果。图3.5显示了在中波范围内使用低压ECC86电子管的%测试电路,该电子管可以仅6 V的阳极电压运行。测试中使用了高阻抗耳机。}
%
%\begin{figure}[h]
%\centering
%%%\includegraphics[width=0.8\textwidth]{fig3-5.png}
%\caption{A 2-stage Audion using an ECC86.}
%\textbf{中文:使用ECC86的两级Audion。}
%\end{figure}
%
%Similar Audion circuits can be built with other types of multiple tubes. Some of the more suitable are the combined Triode/Pentode %type ECFxx or ECLxx family of devices, or their US equivalents. Figure 3.6 shows an example using an ECF12 steel tube. In this %example, the pentode section of the tube is used to provide AF amplification. The output volume is strongly dependent on the anode %voltage. Using this design to receive signals on the short-wave band and with an anode HT of 24 V provides an adequate volume %level.
%
%\textbf{中文:类似的Audion电路可以用其他类型的多电子管构建。一些更合适的是组合三极管/五极管类型的ECFxx或ECLxx系列设备,或它们的美国等效产品。图%3.6显示了使用ECF12钢电子管的示例。在这个示例中,电子管的五极管部分用于提供音频放大。输出音量在很大程度上取决于阳极电压。使用此设计在短波波段接收%信号并使用24 V的阳极高压可以提供足够的音量水平。}
%
%\begin{figure}[h]
%\centering
%%%\includegraphics[width=0.8\textwidth]{fig3-6.png}
%\caption{An Audion built with a single ECF12.}
%\textbf{中文:使用单个ECF12构建的Audion。}
%\end{figure}
%
%A circuit without feedback cannot be as selective as a regenerative receiver in the short-wave range. However, it can provide good %volume and sound quality.
%
%\textbf{中文:在短波范围内,没有反馈的电路不能像再生接收机那样具有选择性。然而,它可以提供良好的音量和音质。}
%
%\begin{figure}[h]
%\centering
%%%\includegraphics[width=0.8\textwidth]{fig3-7.png}
%\caption{Experimental Audion using a steel vacuum tube type ECF12.}
%\textbf{中文:使用钢制电子管类型ECF12的实验性Audion。}
%\end{figure}
%
%\section{Audion Receiver with Regeneration}
%\textbf{中文:具有再生的Audion接收机}
%
%The design of a typical vacuum tube Audion type radio will most likely incorporate regeneration. This combines the demodulation %technique described in section 3.1 with the oscillator circuit regeneration used in the detector receiver described in section 2.%9. Altogether this makes a single tube responsible for three tasks at the same time: regeneration, demodulation, and %amplification. The result is good reception performance with little outlay.
%
%\textbf{中文:典型的电子管Audion类型收音机的设计很可能会包含再生。这结合3.1节中描述的解调技术和2.9节中描述的检测器接收机中使用的振荡电路再生。%总的来说,这使得单个电子管同时负责三个任务:再生、解调和放大。结果是良好的接收性能,而开销很小。}
%
%The ECC81, like the ECC82, is not specifically intended for low-voltage operation but works well from a single 12 V supply to %power the heater and also the anode HT. Both these tubes can have their two heaters wired in parallel for 6.3 V / 0.3 A operation %or in series from 12.6 V / 0.15 A. When the heaters are connected in series, a single 12 V battery is sufficient for heating and %the anode HT voltage. In this regard, the ECC81/82 is superior to the ECC86 low-voltage tube, which only works with the heaters in %parallel. The ECC83, which can also use 12 V for the heater, is not suitable for the receiver due to its very low anode current.
%
%\textbf{中文:ECC81与ECC82一样,并非专门用于低压操作,但从单一12 V电源工作良好,以为加热器和阳极高压供电。这两种电子管都可以将其两个加热器并联%以进行6.3 V / 0.3 A操作,或12.6 V / 0.15 A串联。当加热器串联连接时,单一12 V电池足以用于加热和阳极高压电压。在这方面,ECC81/82优于ECC86低压%电子管,后者仅在加热器并联时工作。ECC83也可以使用12 V用于加热器,但由于其非常低的阳极电流而不适合用于接收机。}
%
%Feedback has already been introduced in the active regeneration of the oscillator circuit in the detector receiver and also leads %to better selectivity and sensitivity with an Audion. Feedback means that the RF signal is amplified, phase-corrected and fed back %to the receive coil circuit. The same principle is also used in a conventional oscillator circuit to maintain oscillation. In the %radio receiver, the feedback signal reinforces the receive signal via a variable coupling to the coil or by an adjustable gain %circuit so that the amount of feedback can be finely adjusted. As the feedback is increased and just before the onset of %self-oscillation, any weak RF signal received on that frequency will undergo maximum amplification.
%
%\textbf{中文:反馈已经在检测器接收机中的振荡电路的主动再生中引入,并且也导致Audion具有更好的选择性和灵敏度。反馈意味着射频信号被放大、相位校正并%反馈回接收线圈电路。同样的原理也用于常规振荡电路以维持振荡。在无线电接收机中,反馈信号通过到线圈的可变耦合或通过可调增益电路增强接收信号,以便可%以精细调整反馈量。随着反馈的增加,就在自振荡开始之前,在该频率上接收的任何微弱射频信号都将经历最大放大。}
%
%In practice the resonant circuit is damped by various energy losses in the circuit. These include losses due to wire resistance, %the finite input resistance of the tube and the damping effect of the antenna. All of these losses can be compensated for by %feedback, theoretically you get a resonant circuit with infinite \textit{Q} factor. In practice, a \textit{Q} factor of up to 10,%000 can be achieved, resulting in a narrow bandwidth of, for example, 600 Hz in the 6 MHz band.
%
%\textbf{中文:在实践中,谐振电路被电路中的各种能量损耗阻尼。这些包括由于导线电阻、电子管的有限输入电阻和天线的阻尼效应引起的损耗。所有这些损耗都%可以通过反馈来补偿,理论上您获得具有无限\textit{Q}因数的谐振电路。在实践中,可以实现高达10,000的\textit{Q}因数,导致窄带宽,例如,6 MHz波段中%为600 Hz。}
%
%This leads to a reduction of the sidebands or an increase in the lower modulation frequencies. In fact, with the help of %regeneration it is possible to achieve selective amplification of the carrier signal, which gives improved demodulation %sensitivity.
%
%\textbf{中文:这导致边带的减少或较低调制频率的增加。事实上,在再生的帮助下,可以实现载波信号的选择性放大,这提供了改进的解调灵敏度。}
%
%The feedback arrangement can be accomplished in many ways. Practically any oscillator circuit can be used to supply the %regeneration signal. Here, the particularly simple three-point circuit with feedback via the cathode is used, which only requires %a tap point on the resonant circuit coil. The degree of coupling depends on the position of the tap and is a maximum when the tap %is at the midpoint of the total coil length. With sufficient amplification by the tube, you only need about 1/10th of the total %number of turns of the coil for the feedback winding. However, in order to achieve oscillation even with low anode voltage and low %tube gain, the tap should be placed closer to the hot end of the coil.
%
%\textbf{中文:反馈安排可以通过多种方式完成。实际上,任何振荡电路都可以用来提供再生信号。这里,使用特别简单的通过阴极反馈的三点电路,这只需要在谐%振电路线圈上有一个抽头点。耦合程度取决于抽头的位置,当抽头位于总线圈长度的中点时最大。通过电子管的充分放大,反馈绕组只需要线圈总圈数的大约1/10。%然而,为了即使在低阳极电压和低电子管增益的情况下也实现振荡,抽头应该更靠近线圈的热端放置。}
%
%Whether a tube with a given slope achieves enough amplification to sustain oscillation can be illustrated by an example. At 6 MHz %and 300 pF, the capacitive resistance will be approximately 90 Ω. With an unloaded \textit{Q} of 100, the resonant resistance of %the resonant circuit is 9 kΩ. The impedance at the cathode tapping is only a quarter of that, or about 2.25 kΩ, if the tapping is %exactly at the midpoint. In order for amplification greater than one to be possible, the slope must be at least S = 1/2.25 kΩ, or %at least about S = 0.44 mA/V. Especially in tube circuits with reduced anode voltage, a high unloaded \textit{Q} of the resonant %circuit is therefore important. In particular, the shortwave coil should be wound with thick wire approximately 1 mm in diameter.
%
%\textbf{中文:具有给定斜率的电子管是否实现足够的放大以维持振荡可以通过一个例子来说明。在6 MHz和300 pF时,容抗将约90 Ω。具有100的无载Q,谐振电%路的谐振电阻9 kΩ。如果抽头恰好位于中点,阴极抽头处的阻抗只有那个的四分之一,或约2.25 kΩ。为了使大于一的放大成为可能,斜率必须至少为S = 1/2.25 %kΩ,或至少约S = 0.44 mA/V。特别是在具有降低的阳极电压的电子管电路中,谐振电路的高无载Q因此很重要。特别是,短波线圈应该用直径约1毫米的粗导线绕%制。}
%
%\begin{figure}[h]
%\centering
%%%\includegraphics[width=0.8\textwidth]{fig3-8.png}
%\caption{An Audion with feedback.}
%\end{figure}
%
%The level of feedback is adjusted by changing the anode voltage which affects the characteristic slope and tube amplification. The %100 kΩ feedback potentiometer allows for very precise control and the circuit achieves smooth oscillation onset. If the antenna %coupling is too strong however, the tube may not have enough amplification to provide adequate feedback. In this case, weaker %antenna coupling or a smaller grid resistor may provide more amplification.
%
%\textbf{中文:反馈水平通过改变阳极电压来调整,这会影响特性斜率和电子管放大100 kΩ反馈电位器允许非常精确的控制,并且电路实现平滑的振荡开始。然而,%如果天线耦合太强,电子管可能没有足够的放大来提供足够的反馈。在这种情况下,较弱的天线耦合或较小的栅极电阻可能会提供更多的放大。}
%
%The \textit{Q} factor achieved after the effect of regeneration can be estimated by determining the bandwidth. At optimal %settings, the Audion can easily separate stations with a channel spacing of just 10 kHz. The bandwidth is about 6 kHz, and higher %modulation frequencies are noticeably suppressed. In the 6 MHz band, this results in an effective quality factor of 1000 and a %resonant resistance of about 100 kΩ. The damping effects of the antenna and the tube's internal resistance are compensated for by %the feedback. The actual \textit{Q} factor depends on the exact feedback adjustment. Near the oscillation threshold, an even %higher \textit{Q} factor can be achieved, but the volume stops increasing at a certain point because the sidebands are no longer %boosted.
%
%\textbf{中文:再生效应后实现的Q因数可以通过确定带宽来估计。在最佳设置下,Audion可以容易地分离仅相隔10 kHz的电台。带宽约6 kHz,较高的调制频率被%明显抑制。在6 MHz波段中,这导致有效的品质因数1000,谐振电阻约100 kΩ。天线和电子管内部电阻的阻尼效应通过反馈得到补偿。实际的Q因数取决于精确的反馈%调整。在振荡阈值附近,可以实现更高的Q因数,但音量在某个点停止增加,因为边带不再被增强。}
%
%This Audion with regeneration already provides reception capabilities that can compete with a simple superheterodyne receiver, %especially because it does not generate an image frequency during demodulation and has good rejection of strong receive signals. %For a listener the Audion is a really enjoyable receiver, especially in the evening when excellent long-range reception can be %achieved with just a 3-meter length antenna. With interchangeable plug-in coils, multiple wavebands or spread shortwave bands can %be implemented. Receivers for individual amateur radio bands are also possible. With feedback engaged CW and SSB signals can be %received, which is only otherwise possible by using a superheterodyne receiver with a beat frequency oscillator (BFO).
%
%\textbf{中文:这种带再生的Audion已经提供了可以与简单的超外差接收机竞争的接收能力,特别是因为它在解调期间不会产生镜像频率,并且对强接收信号有良好%的抑制。对于听众来说,Audion是一个真正令人愉快的接收机,特别是在晚上,仅3米长的天线就可以实现出色的远程接收。使用可互换的插拔式线圈,可以实现多个%波段或扩展的短波波段。单个业余无线电波段的接收机也是可能的。通过接通反馈,可以接收CW和SSB信号,否则只能通过使用带有拍频振荡器(BFO)的超外差接收%机来实现。}
%
%An SSB (Single Sideband) transmitter doesn't transmit a carrier wave, but only the lower or upper sideband. When you listen to it %on an AM (Amplitude Modulation) receiver, you'll hear an indistinct noise. The receiver has to add the carrier wave back at the %exact frequency. If it's not tuned precisely, voices will sound distorted a bit like Mickey Mouse. In the 80- and 40-meter band %the lower sideband is used, so the receiver needs to be tuned slightly above the receiving frequency. You'll usually have the best %results in the 40-meter band at around 7050 kHz.
%
%\textbf{中文:SSB(单边带)发射机不传输载波,只传输下边带或上边带。当您在AM(幅度调制)接收机上收听时,您会听到不清晰的噪音。接收机必须在精确频%率处重新添加载波。如果调谐不精确,声音听起来会有点像米老鼠那样失真。在80米和40米波段中使用下边带,因此接收机需要调谐到略高于接收频率。您通常会在%40米波段中7050 kHz左右获得最佳结果。}
%
%At the lower end of the band, around 7000 kHz, you'll hear CW (Carrier Wave) transmitters. These use unmodulated carriers that are %turned on and off to transmit Morse code messages. To hear anything, you need to superimpose the input signal with a second %signal. To do this, you need to detune the receiver using a feedback frequency about 500 Hz to 1 kHz below or above the receive %signal. The exact adjustment determines the audible pitch. A deviation of 800 Hz also creates a beat frequency of 800 Hz. When you %tune it exactly to the received signal carrier, nothing will be audible.
%
%\textbf{中文:在波段的低端,大约7000 kHz,您会听到CW(载波)发射机。这些使用未调制的载波,通过开启和关闭来传输摩尔斯电码消息。为了听到任何东%西,您需要将输入信号与第二个信号叠加。为此,您需要使用低于或高于接收信号500 Hz到1 kHz的反馈频率来失谐接收机。精确调整决定了可听音调。800 Hz的偏%差也产生800 Hz的拍频。当您精确调谐到接收到的信号载波时,什么也听不到。}
%
%\begin{figure}[h]
%\centering
%%%\includegraphics[width=0.8\textwidth]{fig3-9.png}
%\caption{The ECC81-Audion built on a wooden breadboard.}
%\end{figure}
%
%An important characteristic of a good regenerative receiver is soft onset of the feedback state. A harsh, sudden onset makes %tuning difficult and produces distortion. For a gentle onset, it's important to have effective automatic gain control for high %signal amplitude at resonance. Current flow in the grid increases the grid's negative charge. It's often the case that %self-oscillations stop just above the critical point when tuned to a strong signal. The large signal amplitude of the useful %signal itself reduces the gain. However, this only works with a relatively short grid time constant.
%
%\textbf{中文:良好的再生接收机的一个重要特征是反馈状态的软开始。严厉、突然的开始使调谐困难并产生失真。为了温和的开始,重要的是在谐振时对高信号幅%度有有效的自动增益控制。栅极中的电流流动增加了栅极的负电荷。通常情况是,当调谐到强信号时,自振荡恰好在临界点上方停止。有用信号的大信号幅度本身降%低了增益。然而,这只在相对较短的栅极时间常数下工作。}
%
%If the grid capacitance is increased, for example, to a value of 1 µF, fine tuning becomes impossible. It leads to swings in the %oscillator gain with crackling noises. With each new oscillation attempt, the tube reaches a working point with higher anode %current and higher slope, which amplifies the feedback. The oscillation then becomes larger until it gets cut off by a strong grid %charge. Using a small grid capacitor below 1 nF ensures the gain is throttled back quickly enough to stabilize the amplitude.
%
%\textbf{中文:如果栅极电容增加,例如,增加到1 µF的值,精细调谐变得不可能。这导致振荡器增益的波动,伴有噼啪声。每次新的振荡尝试,电子管达到一个具%有更高阳极电流和更高斜率的工作点,这放大了反馈。然后振荡变得更大,直到被强栅极电荷切断。使用低于1 nF的小栅极电容确保增益被足够快地节流以稳定幅%度。}
%
%Both processes responsible for demodulation in the tube, namely grid rectification and anode rectification, work together here. %While the anode current characteristic curve increases the current with modulation, the rising grid current reduces the anode %current by negatively charging the grid. This regulation even works like automatic gain control (AGC) to some extent and partially %regulates field strength fluctuations automatically.
%
%\textbf{中文:负责电子管中解调的两个过程,即栅极整流和阳极整流,在这里一起工作。虽然阳极电流特性曲线随着调制增加电流,但上升的栅极电流通过负向充%电栅极来降低阳极电流。这种调节甚至在某种程度上像自动增益控制(AGC)一样工作,并部分自动调节场强波动。}
%
%The circuit shown here was used with minor changes and an ECC82 in the anniversary edition of the Radiomann 2004 experiment kit %from Kosmos. They have paid particular attention to recreate a feeling of the 1940s with the look and feel of this kit. Plug-in %coils allow for band switching. A cylindrical coil is used for the shortwave band, and a flat coil for medium wave reception.
%
%\textbf{中文:这里显示的电路在Kosmos的Radiomann 2004实验套件的周年纪念版中进行了微小的更改并使用了ECC82。他们特别注意用这个套件的外观和感受来%重现1940年代的感觉。插拔式线圈允许波段切换。圆柱形线圈用于短波波段,扁平线圈用于中波接收。}
%
%\begin{figure}[h]
%\centering
%%%\includegraphics[width=0.8\textwidth]{fig3-10.png}
%\caption{The Kosmos-Radiomann kit fitted with a medium wave coil installed.}
%\end{figure}
%
%\section{Loudspeaker Operation}
%\textbf{中文:扬声器操作}
%
%The following circuit was developed as an extension of the Kosmos-Radiomann receiver. The original receiver design with an ECC82 %and headphones output can be converted to speaker output with minimal changes. This was achieved using a PCL86 running at an anode %voltage of 60 V. The triode section acts as the audio stage. Due to the higher anode voltage, the anode resistor should be %increased to 100 kΩ, which achieves greater voltage gain. Like the original, this receiver is classified as a 0V1, which means %it's a receiver without an RF preamp, with an Audion stage and an AF stage.
%
%\textbf{中文:以下电路是作为Kosmos-Radiomann接收机的扩展而开发的。带有ECC82和耳机输出的原始接收机设计可以通过最小的更改为扬声器输出。这是通过%使用60 V阳极电压下运行的PCL86实现的。三极管部分作为音频级。由于较高的阳极电压,阳极电阻应该增加到100 kΩ,这实现了更大的电压增益。与原始设计一%样,这个接收机被归类为0V1,这意味着它是一个没有射频前置放大器的接收机,具有Audion级和音频级。}
%
%\begin{figure}[h]
%\centering
%%%\includegraphics[width=0.8\textwidth]{fig3-11.png}
%\caption{Loudspeaker operation using a PCL86.}
%\end{figure}
%
%The power pentode in the PCL86 composite tube delivers around 5 mA of anode current at 60 V, which is adequate for speaker %operation. The 20:1 ratio output transformer can be used as is. The output tube then operates into an output impedance of 8 Ω × %400 = 3.2 kΩ. While the original radio was powered by batteries, a simple external power supply needs to be used here. The PCL86 %has a heater rating of 13 V/300 mA, so you could salvage a power transformer from an old 12 V halogen desk lamp, for example. The %anode voltage is generated by a voltage multiplication network. To achieve the best possible hum-free operation, the screen grid %voltage and the operating voltage for the audio stage are smoothed with an additional RC filter. The receiver provides good %sensitivity and volume on both bands.
%
%\textbf{中文:PCL86复合电子管中的功率五极管60 V时提供约5 mA的阳极电流,这对于扬声器操作是足够的。20:1比率的输出变压器可以按原样使用。输出电子管%然后工作在8 Ω × 400 = 3.2 kΩ的输出阻抗。虽然原始收音机由电池供电,但这里需要使用简单的外部电源。PCL86的加热器额定值为13 V/300 mA,因此您可以从%旧12 V卤素台灯中,例如,抢救电源变压器。阳极电压由电压倍增网络产生。为了实现尽可能好的无嗡嗡声操作,屏幕栅极电压和音频级的工作电压用额外的RC滤波%器平滑。接收机在两个波段上都提供良好的灵敏度和音量。}
%
%\section{A Regenerative Receiver using two EF95 Tubes}
%\textbf{中文:使用两个EF95电子管的再生接收机}
%
%Usually, when a pentode is used in an Audion receiver design it results in greater gain levels than when using a triode. That's %why the EF95 is used here. This design uses the second EF95 as an AF (Audio Frequency) amplifier.
%
%\textbf{中文:通常,当在Audion接收机设计中使用五极管时,它导致比使用三极管更高的增益水平。这就是为什么这里使用EF95。这个设计使用第二个EF95作为%AF(音频频率)放大器。}
%
%\begin{figure}[h]
%\centering
%%%\includegraphics[width=0.8\textwidth]{fig3-12.png}
%\caption{EF95 Audion with AF-Stage.}
%\end{figure}
%
%The audio tube operates in triode mode with capacitor coupling to headphones or an external audio amplifier. Using high-impedance %headphones, the volume is sufficient; even 32 Ω headphones can be driven directly.
%
%\textbf{中文:音频电子管以三极管模式运行,通过电容耦合到耳机或外部音频放大器。使用高阻抗耳机,音量是足够的;甚至32 Ω耳机也可以直接驱动。}
%
%For more volume, the audio stage impedance matching needs to be improved. A small mains transformer is connected to the output to %act as a signal impedance transformer. For example, a small power transformer with a rating of 230 V/24 V (120 V/12 V in the US) %will be suitable. The winding ratio is about 10:1, so the impedance ratio is 100:1. If both drivers in the headphones are %connected in parallel, their combined impedance is 16 Ω. From the tubes point of view it is now operating into a 1.6 kΩ load. This %produces better matching to the high internal resistance of the tube output and leads to improved output power. A multi-tap output %mains transformer is suitable to use as this impedance transformer. A power transformer salvaged from a plug-in mains adapter with %switchable output voltages between 3 V and 12 V could also be used.
%
%\textbf{中文:为了获得更多音量,音频级阻抗匹配需要改进。一个小型电源变压器连接到输出,作为信号阻抗变压器。例如,额定值为230 V/24 V(美国为120 %V/12 V)的小型电源变压器将是合适的。绕组比约为10:1,因此阻抗比为100:1。如果耳机中的两个驱动器并联连接,它们的组合阻抗为16 Ω。从电子管的角度来%看,它现在工作在1.6 kΩ负载。这产生与电子管输出的高内阻更好的匹配,并导致改进的输出功率。多抽头输出电源变压器适合用作这个阻抗变压器。从输出电压可%在3 V到12 V之间切换的插拔式电源适配器中抢救的电源变压器也可以使用。}
%
%\begin{figure}[h]
%\centering
%%%\includegraphics[width=0.8\textwidth]{fig3-13.png}
%\caption{Impedance matching using an output transformer.}
%\end{figure}
%
%The radio was built on the RT100 tube experimentation system from AK Modul-Bus. This system allows for experiments to be carried %out on a plugboard. A variable capacitor, potentiometer, power supply connection, and audio jacks are included in the kit. For the %shortwave range, a wire coil with two sets of 10 turns of single-strand wire is sufficient, which can be free wound without a %supporting coil former.
%
%\textbf{中文:收音机是在AK Modul-Bus的RT100电子管实验系统上构建的。该系统允许在插板上进行实验。可变电容器、电位器、电源连接和音频插孔都包含在套%件中。对于短波范围,一个具有两组10圈单股导线的线圈就足够了,可以自由绕制而无需支撑线圈骨架。}
%
%\begin{figure}[h]
%\centering
%%%\includegraphics[width=0.8\textwidth]{fig3-14.png}
%\caption{The regeneration stage wiring.}
%\end{figure}
%
%The variable capacitor connections terminate at the screw terminals and gold-plated 2 mm sockets. This conveniently allows coils %for different frequency bands to be terminated in 2 mm plugs and plugged in as required, without generating any additional losses.
%
%\textbf{中文:可变电容器连接在螺丝端子和镀金2 mm插座处终止。这方便地允许不同波段的线圈在2 mm插头中终止,并根据需要插入,而不会产生任何额外的损%耗。}
%
%\begin{figure}[h]
%\centering
%%%\includegraphics[width=0.8\textwidth]{fig3-15.png}
%\caption{Plug-in coils for short and medium wave operation.}
%\end{figure}
%
%Figure 3.15 shows two plug-in coils mounted on perforated circuit boards. The shortwave coil has an adjustable ferrite core and a %total of 20 turns with a center tap. The medium wave coil uses a small ferrite rod with 80 turns of RF 'Litz' wire. On top of this %is a small 10-turn coupling coil of enamel-coated copper wire.
%
%\textbf{中文:图3.15显示了安装在穿孔电路板上的两个插拔式线圈。短波线圈具有可调铁氧体磁芯和总共20圈,带有中心抽头。中波线圈使用带80圈RF"Litz"线%的小型铁氧体棒。在这个上面是一10圈的漆包铜线耦合线圈。}
%
%\section{A Shortwave Audion Type 0V2}
%\textbf{中文:短波Audion类型0V2}
%
%This vacuum tube Audion regenerative receiver design was developed with the aim of building a sensitive shortwave receiver for %speaker operation whilst using a safe anode voltage. This particular design is termed '0V2', which classifies it as a receiver %with one regenerative stage and two AF stages. The same circuit functioned using almost the same component values but with an EF80 %tube running on only 12 V. At 70 V, however, much more volume is achieved. At the higher voltage, the grid resistor must have a %higher value. The EF80 was replaced by an EF183 which offers more gain and higher maximum frequency operation. Now it is a %fully-fledged radio that performs rather well when compared with a tube superheterodyne radio. The 70 V anode voltage is %considered safe, so no special precautions are necessary.
%
%\textbf{中文:这种电子管Audion再生接收机设计是怀着使用安全阳极电压构建用于扬声器操作的灵敏短波接收机的目标而开发的。这个特殊设计被称为"0V2",将%其归类为具有一个再生级和两个音频级的接收机。相同的电路使用几乎相同的元件值工作,但使用仅12 V上运行的EF80电子管。然而,70 V时,实现了更多的音量。%在更高的电压下,栅极电阻必须具有更高的值。EF80被EF183取代,后者提供更多的增益和更高的最大频率操作。现在它是一个成熟的收音机,当与电子管超外差收%音机相比时表现相当好。70 V阳极电压被认为是安全的,因此不需要特殊的预防措施。}
%
%\begin{figure}[h]
%\centering
%%%\includegraphics[width=0.8\textwidth]{fig3-16.png}
%\caption{The three-stage Audion receiver.}
%\end{figure}
%
%Often, the feedback signal is picked up from the anode contact and fed to a special coupling coil. However, with an indirectly %heated tube, it's easier to connect the cathode to a tap on the resonant circuit coil. This gives us the basic circuit of a %classic 'cathode 3 point oscillator' or, if it's a pentode, the so-called ECO circuit (Electron Coupled Oscillator). The amount of %feedback applied is controlled by varying the characteristic slope. With a triode, this is achieved by changing the anode voltage; %with a pentode it's done by changing the screen grid voltage. The audio stages don't need a cathode resistor here because the %operating voltage is relatively low. The negative grid bias is automatically generated by the grid current as a voltage drop %across the grid resistor.
%
%\textbf{中文:通常,反馈信号从阳极接触处拾取并馈送到特殊的耦合线圈。然而,对于间接加热的电子管,将阴极连接到谐振电路线圈上的抽头更容易。这给了我%们经典'阴极三点振荡器'的基本电路,或者如果是五极管,就是所谓的ECO电路(电子耦合振荡器)。应用的反馈量通过改变特性斜率来控制。对于三极管,这是通过%改变阳极电压来实现的;对于五极管,这是通过改变屏幕栅极电压来完成的。音频级在这里不需要阴极电阻,因为工作电压相对较低。负栅极偏置由栅极电流作为栅%极电阻上的电压降自动产生。}
%
%From the 1950s to 1970s, the ECO Audion was very popular receiver design among radio amateurs as the classic "beginner's project". %Plug-in coils were usually fitted to cover the various amateur radio bands. The sensitivity of the design is so good you can %easily pick up long-distance stations, even with the relatively low anode voltage of just 70 V.
%
%\textbf{中文:从1950年代到1970年代,ECO Audion在无线电业余爱好者中是非常流行的接收机设计,作为经典的"初学者项目"。插拔式线圈通常被安装以覆盖各%种业余无线电波段。该设计的灵敏度非常好,您可以轻松接收到长距离电台,即使只有相对较低的70 V阳极电压。}
%
%The coil has 20 turns with a tap at the 4th and 7th turn. The frequency range can be tweaked by adjusting the ferrite screw slug. %The receiver covers the 49-m to 31-m band. However, you can also hear SSB voice communication and telegraphy in the 40-m amateur %radio band. The antenna can be connected either directly to the lower tap or with loose coupling via an additional antenna %capacitor. Any wire over one meter long can be used as an antenna.
%
%\textbf{中文:线圈有20圈,在第4圈和7圈处有抽头。频率范围可以通过调整铁氧体螺钉塞来调整。接收机覆盖49米到31米波段。然而,您也可以40米业余无线电%波段听到SSB语音通信和电报。天线可以直接连接到下部抽头,或者通过额外的天线电容器进行松散耦合。任何超过一米长的导线都可以用作天线。}
%
%The volume is more than adequate. With stronger transmitters, the volume needs to be turned down. You can listen to a station all %day long. In the evening, the range increases and more distant overseas stations will also be received. The feedback oscillation %onset is very soft and only lightly dependent on the frequency. So, you can find some settings where only the tuning capacitor %needs to be adjusted. However, if you want to receive a weak station in amongst stronger ones, you have to maximize the gain and %selectivity by setting the feedback as close to the oscillation threshold as possible.
%
%\textbf{中文:音量是绰绰有余的。对于更强的发射机,音量需要调低。您可以整天收听一个电台。在晚上,范围增加,更多遥远的海外电台也将被接收到。反馈振%荡开始非常柔和,并且仅轻微地依赖于频率。因此,您可以找到一些设置,只需要调整调谐电容器。然而,如果您想在较强的电台中接收到一个弱电台,您必须通过%将反馈设置得尽可能接近振荡阈值来最大化增益和选择性。}
%
%\section{A 6 V Tube Regenerative Receiver}
%\textbf{中文:6 V电子管再生接收机}
%
%The goal of this circuit is to create a regenerative receiver that operates from a single HT voltage of only 6 volts. For this %purpose, an EL95 tube will be used, which can often be easier to find than the low-voltage tube types ECC86 or EF98. The EL95 also %works very well with a low anode voltage of only 6 volts.
%
%\textbf{中文:这个电路的目标是创建一个仅从单一6伏特高压电压运行的再生接收机。为此,将使用EL95电子管,它通常比低压电子管类型ECC86或EF98更容易找%到。EL95也仅6伏特的低阳极电压下工作得非常好。}
%
%Good results have also been produced with a simple Electron Coupled Oscillator (ECO) circuit using a low anode voltage. The slope %characteristic of the tube is controlled by adjusting the screen grid voltage. In this case, the screen grid can be grounded via a %capacitor for both RF and audio frequencies because the demodulated audio signal feeds out from the tube's anode. The system of %cathode-control grid-screen grid can be thought of as a triode system that is only responsible for adding regeneration to the %receiving circuit and for demodulation.
%
%\textbf{中文:使用低阳极电压的简单电子耦合振荡器(ECO)电路也产生了良好的结果。电子管的斜率特性通过调整屏幕栅极电压来控制。在这种情况下,屏幕栅%极可以通过电容对射频和音频频率接地,因为解调的音频信号从电子管的阳极输出。阴极-控制栅极-屏幕栅极系统可以被认为是一个三极管系统,仅负责向接收电路%添加再生和解调。}
%
%\begin{figure}[h]
%\centering
%%%\includegraphics[width=0.8\textwidth]{fig3-17.png}
%\caption{Single stage ECO Audion.}
%\end{figure}
%
%At the low anode voltage of 6 V, the tube has a low transconductance. Therefore, the Audion can only be brought into oscillation %if the resonant circuit has a very high unloaded \textit{Q} factor. This requires a large air-core coil made with thick wire. %Similar care is required as with free-running oscillators in amateur radio equipment. Also make sure the antenna coil is loosely %coupled.
%
%\textbf{中文:在6 V的低阳极电压下,电子管具有低跨导。因此,只有当谐振电路具有非常高的无载\textit{Q}因数时,Audion才能被带入振荡。这需要一个用%粗导线制成的大型空心线圈。需要与业余无线电设备中的自由运行振荡器类似的护理。还要确保天线线圈是松散耦合的。}
%
%\begin{figure}[h]
%\centering
%%%\includegraphics[width=0.8\textwidth]{fig3-18.png}
%\caption{Shortwave Audion using an EL95.}
%\end{figure}
%
%This regenerative receiver can be used to receive Digital Radio Mondial (DRM) broadcasts where available. In this application, %good frequency stability is crucial for reliable reception, and this all comes down to the oscillator circuit design. To achieve %the required level of stability, you need a large 20 turn coil of 1.5 mm diameter wire wound onto a length of 18 mm diameter PVC %pipe. All connections must be as short as possible especially between the coil and the air-spaced variable tuning capacitor. With %careful construction a \textit{Q} factor of over 300 was obtained. All other connections should be kept mechanically secure. %Nothing can be allowed to wobble or physically vibrate. Even the tube is physically secured on its glass top spike to reduce %vibrations.
%
%\textbf{中文:这种再生接收机可以用于接收数字广播(DRM)广播,如果有的话。在这个应用中,良好的频率稳定性对于可靠的接收至关重要,这完全取决于振荡%器电路设计。为了实现所需的稳定性水平,您需要一个用1.5毫米直径导线绕制的大约20圈线圈,绕在18毫米直径的PVC管上。所有连接必须尽可能短,特别是线圈和%空气间隔可变调谐电容器之间的连接。通过仔细的构造,获得了超过300的\textit{Q}因数。所有其他连接应该保持机械牢固。不允许任何东西摇摆或物理振动。甚%至电子管也在其玻璃顶部尖峰上物理固定以减少振动。}
%
%The receiver is best operated with an external audio amplifier or connected to a PC sound card. It is also possible to listen in %directly using a high impedance headphone. With just one stage, clear reception of numerous shortwave stations can be achieved.
%
%\textbf{中文:接收机最好使用外部音频放大器操作或连接到PC声卡。也可以直接使用高阻抗耳机收听。仅用一个阶段,就可以实现许多短波电台的清晰接收。}
%
%\section{Cascode Triode Regeneration}
%\textbf{中文:共栅共阴三极管再生}
%
%The cascode regenerative receiver has proven itself particularly useful in the field of amateur radio. Back in the 1950s and %1960s, it was a popular radio design and proved a good introduction into receiver technology for many budding radio enthusiasts. %Often, an ECC81 was used, with the receiver running at 250 V.
%
%\textbf{中文:共栅共阴再生接收机在业余无线电领域被证明特别有用。早在1950年代到1960年代,它是一个流行的收音机设计,并为许多萌芽的无线电爱好者证明%是接收机技术的良好介绍。通常使用ECC81,接收机在250 V下运行。}
%
%In general, triodes are less noisy than pentodes. The additional grid structures in a pentode such as the screen and suppressor %grid can introduce more noise into the circuit. Pentodes, however, have a higher voltage gain and lower feedback capacitance. The %cascode configuration combines the advantages of both characteristics. For this reason, it was often used for VHF preamplifiers in %television receivers, where high input sensitivity is crucial. What applies for an RF preamplifier also applies for a regenerative %receiver.
%
%\textbf{中文:一般来说,三极管比五极管更安静。五极管中的附加栅极结构,如屏幕栅极和抑制栅极,可以向电路中引入更多的噪声。然而,五极管具有更高的电%压增益和更低的反馈电容。共栅共阴配置结合了两种特性的优点。因此,它经常用于电视接收机中的VHF前置放大器,其中高输入灵敏度至关重要。适用于射频前置放%大器的也适用于再生接收机。}
%
%\begin{figure}[h]
%\centering
%%%\includegraphics[width=0.8\textwidth]{fig3-19.png}
%\caption{The cascode amplier configuration.}
%\end{figure}
%
%The cascode regenerative receiver design uses the cascode configuration with feedback applied via the cathode. The feedback %potentiometer changes the grid voltage of the upper tube, and thus the anode voltage of the lower triode, which changes the %characteristic slope of the circuit. The feedback can be adjusted to just on the verge of oscillation, so that the input circuit %is optimally stimulated. An antenna signal of just a few microvolts can be amplied to several hundred millivolts of RF signal, %resulting in high selectivity and bandwidths of around 5 kHz with high gain.
%
%\textbf{中文:共栅共阴再生接收机设计使用共栅共阴配置,反馈通过阴极应用。反馈电位器改变上部电子管的栅极电压,从而改变下部三极管的阳极电压,这改变%了电路的特性斜率。反馈可以调整到刚好在振荡的边缘,以便输入电路被最佳地刺激。仅几微伏的天线信号可以被放大到几百毫伏的射频信号,导致高选择性和5 kHz%的带宽,具有高增益。}
%
%\begin{figure}[h]
%\centering
%%%\includegraphics[width=0.8\textwidth]{fig3-20.png}
%\caption{The cascode Audion regenerative configuration.}
%\end{figure}
%
%When using tube receivers with low anode voltages, triodes (ECC82 in the case of radios) or pentodes (EF95 in the RT100) are often %employed. At first glance, the cascode configuration has a significant disadvantage: it halves the already-low anode voltage. %That's why a special tube such as the ECC88 is required that can handle this.
%
%\textbf{中文:当使用低阳极电压的电子管接收机时,通常使用三极管(在收音机的情况下为ECC82)或五极管(在RT100中为EF95)。乍一看,共栅共阴配置有一%个显著的缺点:它将已经很低的阳极电压减半。这就是为什么需要一个特殊的电子管,如ECC88,来处理这个问题。}
%
%Most tubes are designed for use with anode voltages between 100 V and 300 V but the legendary ECC86 was developed to operate at %low anode voltages such as 6.3 V or 12.6 V. Back in the day, before transistors became commonplace, the goal was to design a car %radio that didn't need a voltage inverter to generate the high voltages normally used for the tube HT supply. Operating with an %anode voltage of, say, 12 V, the tube still achieves a mutual conductance characteristic slope of 4.6 mA/V.
%
%\textbf{中文:大多数电子管设计用于在100 V到300 V之间的阳极电压下使用,但传奇的ECC86被开发用于在低阳极电压下操作,如6.3 V或12.6 V。在过去,在晶%体管变得普遍之前,目标是设计一个不需要电压逆变器来产生通常用于电子管高压电源的高电压的汽车收音机。在12 V的阳极电压下操作,电子管仍然实现4.6 mA/V%的互导特性斜率。}
%
%\begin{table}[h]
%\centering
%\begin{tabular}{|l|l|l|l|l|}
%\hline
%\textbf{ECC86} & \textbf{Heater rating} & \textbf{Maximum Operating conditions} & \textbf{Operating with VA at 6.3 V} \\
%\hline
%For RF, amplier, VHF Mixer & 6.3 V/ 0.33 A & PA = 0.6 W & VA = 12.6 V \\
%& Ic = 20 mA & VA = 6.3 V & IA = 2.5 mA \\
%& VA = 30 V & VG = 0 V & IA = 0.9 mA \\
%& S = 4.6 mA/V & S = 2.6 mA/V & VG = 0 V \\
%\hline
%\end{tabular}
%\caption{Key features of the ECC86 dual triode.}
%\end{table}
%
%After many experiments it turns out that many other tubes can also operate successfully with an anode voltage of just 12 V. In %general, the useful operating range shifts down to very low grid biases. This means that a non-negligible grid current flows, so %you cannot expect to achieve extremely high input impedance. However, an Audion has always been driven with grid current, so there %is hardly any disadvantage here.
%
%\textbf{中文:经过许多实验,结果发现许多其他电子管也可以在仅12 V的阳极电压下成功操作。一般来说,有用的操作范围向下移动到非常低的栅极偏置。这意味%着不可忽略的栅极电流流动,因此您不能期望实现极高的输入阻抗。然而,Audion一直用栅极电流驱动,所以这里几乎没有缺点。}
%
%On inspection, what stands out physically about the ECC88 is the special shape of its triode structure. Small notches in the anode %plates provide extremely close proximity to the cathode. This same structure can also be found in the low-voltage ECC86 tube. It's %clear that a voltage grid allows an even shorter distance between cathode and grid. Short distances equals' low voltage: this rule %also applies in reverse, as can be seen in giant vacuum tubes used by high power broadcast transmitters. In any case, the ECC88 %looks suitable for low voltage operation from a cursory visual inspection. The single triode EC88 is also built very similarly and %has proven to be a good low-voltage triode.
%
%\textbf{中文:在检查时,ECC88在物理上突出的是其三极管结构的特殊形状。阳极板中的小凹口提供了与阴极的极近距离。这种相同结构也可以在低压ECC86电子%管中找到。很明显,电压栅极允许阴极和栅极之间更短的距离。短距离等于低电压:这个规则也反过来适用,正如在高功率广播发射机中使用的大型真空电子管中可%以看到的那样。无论如何,从粗略的视觉检查来看,ECC88看起来适合低压操作。单个三极管EC88也建造得非常相似,并已被证明是一个良好的低压三极管。}
%
%More widespread than the ECC88 tube was the PCC88 which could often be found in VHF front ends of television sets. It was used in %a cascode configuration, with the two tube sections connected in series. Each section receives only half of the operating voltage. %Therefore, the tube is already designed for lower anode voltages from the start. With a VA = 90 V and Vg = -1.3 V, a %characteristic slope of 12.5 mA/V is achieved. That's a good sign and gives hope that it will work well at even lower voltages. %This tube is also popular among Hi-Fi enthusiasts in its E variant with a 6.3 V heater voltage.
%
%\textbf{中文:比ECC88电子管更广泛的是PCC88,它经常可以在电视机的VHF前端中找到。它在共栅共阴配置中使用,两个电子管部分串联连接。每个部分只接收操%作电压的一半。因此,电子管从一开始就设计用于较低的阳极电压。在VA = 90 V和Vg = -1.3 V的情况下,实现了12.5 mA/V的特性斜率。这是一个好迹象,并给%人希望它将在更低的电压下工作得很好。这种电子管在其E变体中,具有6.3 V加热器电压,在Hi-Fi爱好者中也很受欢迎。}
%
%\begin{table}[h]
%\centering
%\begin{tabular}{|l|l|l|l|l|}
%\hline
%\textbf{ECC88} & \textbf{Heater} & \textbf{Max Value} & \textbf{Operating Value} \\
%\hline
%For VHF input stages & 6.3 V/ 0.365 A & PA = 1.8 W & VA = 90 V \\
%Cascode circuits & IK = 25 mA & IA = 15 mA & Vg = -1.3 V \\
%& VA = 130 V & S = 12.5 mA/V & \\
%\hline
%\end{tabular}
%\caption{Key features of the ECC88.}
%\end{table}
%
%An easy measurement method is available to test the suitability of a tube for operation at low voltages. In this method, the grid %bias is generated by using different value grid leak resistors. The voltage amlication is measured using an audio-frequency %generator and an oscilloscope. A 1 kΩ anode resistor is used in the circuit so a gain of 1 corresponds to a transconductance of 1 %mA/V.
%
%\textbf{中文:有一种简单的测量方法可以测试电子管在低电压下操作的适用性。在这种方法中,栅极偏置通过使用不同值的栅漏电阻器产生。电压放大使用音频频%率发生器和示波器测量。电路中使用1 kΩ阳极电阻器,因此增益1对应1 mA/V的跨导。}
%
%\begin{figure}[h]
%\centering
%%%\includegraphics[width=0.8\textwidth]{fig3-21.png}
%\caption{The test circuit.}
%\end{figure}
%
%\begin{table}[h]
%\centering
%\begin{tabular}{|l|l|l|l|l|}
%\hline
%\textbf{ECC86} & \textbf{RG} & \textbf{VG} & \textbf{IA} & \textbf{S} \\
%\hline
%VA=12 V & 0 & 0 & 2.2 mA & 1 kΩ \\
%& 1 kΩ & -30 mV & 2.1 mA & 3.3 mA/V \\
%& 10 kΩ & -200 mV & 1.45 mA & 2.8 mA/V \\
%& 100 kΩ & -510 mV & 0.58 mA & 1.6 mA/V \\
%& 1 MΩ & -710 mV & 0.28 mA & 1.0 mA/V \\
%\hline
%\end{tabular}
%\caption{Test measurements of the ECC86.}
%\end{table}
%
%When it comes finding triodes that can operate at low anode voltages, you need to compare their characteristics with the ECC86. %Measurements indicate that the ECC88 is very similar. Although the anode current IA of the ECC88 with VA = 12 V and VG = 0 is only %about half as large as that of the ECC86, the value of transconductance is about the same. Therefore, it can be assumed that the %ECC88 can be considered a good replacement for the now scarce ECC86.
%
%\textbf{中文:在寻找可以在低阳极电压下操作的三极管时,您需要将它们的特性与ECC86进行比较。测量表明ECC88非常相似。虽然ECC88在VA = 12 V和VG = 0%时的阳极电流IA只有ECC86的大约一半,但跨导值大约相同。因此,可以假设ECC88可以被认为是现在稀缺的ECC86的良好替代品。}
%
%\begin{table}[h]
%\centering
%\begin{tabular}{|l|l|l|l|l|}
%\hline
%\textbf{ECC88} & \textbf{RG} & \textbf{VG} & \textbf{IA} & \textbf{S} \\
%\hline
%VA=12 V & 0 & 0 & 1.2 mA & 1 kΩ \\
%& 1 kΩ & -25 mV & 1.1 mA & 3.3 mA/V \\
%& 10 kΩ & -50 mV & 1.0 mA & 3.0 mA/V \\
%& 100 kΩ & -110 mV & 0.60 mA & 2.0 mA/V \\
%& 1 MΩ & -180 mV & 0.29 mA & 1.0 mA/V \\
%\hline
%\end{tabular}
%\caption{Test measurements of the ECC88.}
%\end{table}
%
%This receiver uses two double triodes type ECC88, one for the Audion stage and one for the two-stage AF amplier. It's classed as a %0-V-2 receiver. The heaters of both tubes are connected in series so that the heater voltage is 12 V and the whole receiver can be %powered with a simple 12 V power supply.
%
%\textbf{中文:这个接收机使用两个双三极管类型ECC88,一个用于Audion级,一个用于两级音频放大器。它被归类为0-V-2接收机。两个电子管的加热器串联连%接,因此加热器电压12 V,整个接收机可以用简单的12 V电源供电。}
%
%\begin{figure}[h]
%\centering
%%%\includegraphics[width=0.8\textwidth]{fig3-22.png}
%\caption{The receiver circuit diagram.}
%\end{figure}
%
%When building an Audion, it's best to use a high value of grid resistance to keep the grid current low, the resonant circuit will %then be lightly loaded and loosely coupled to the antenna. Usually, a grid resistance value of 1 MΩ will be used when working with %high anode voltages and 100 kΩ when working with lower operating voltages. Through experimentation, a resistance of 270 kΩ was %found to be optimal for use with the ECC88, which has a relatively high characteristic slope even at low anode voltages and %currents. The anode resistance of the cascade stage can also be designed to be quite high at 27 kΩ, which results in a good level %of amlication. The feedback setting is approximately mid position of the potentiometer (P2 on the RT100), which provides the first %triode with an effective anode voltage of 6 V and an anode current of approximately 0.1 mA. These settings allow for loose antenna %coupling and give a high open circuit \textit{Q} for the input circuit.
%
%\textbf{中文:在构建Audion时,最好使用高值的栅极电阻来保持栅极电流低,谐振电路将轻载并松散耦合到天线。通常,在使用高阳极电压时将使用1 MΩ的栅极%电阻值,在使用较低的操作电压时使用100 kΩ。通过实验,发现270 kΩ的电阻对于与ECC88一起使用是最佳的,即使在低阳极电压和电流下,它也具有相对较高的特%性斜率。级联级的阳极电阻也可以设计得相当高,为27 kΩ,这导致良好的放大水平。反馈设置大约是电位器的中间位置(RT100上的P2),这为第一个三极管提供6 %V的有效阳极电压和0.1 mA的阳极电流。这些设置允许松散的天线耦合,并为输入电路提供高的开路\textit{Q}。}
%
%The antenna coil is wound on a 5 mm diameter coil former with a screw core and has a total of 20 turns of 0.2 mm diameter copper %wire. The antenna tap is located at two turns, and the feedback tap is located at a total of seven turns. With these coil %parameters, the receiver covers the range from 5 to 12 MHz approximately and the screw ferrite core allows for tuning adjustments. %The coil former connection pins have been replaced with longer versions that fit into a plug board.
%
%\textbf{中文:天线线圈绕在带有螺钉磁芯的5毫米直径线圈骨架上,总共20圈0.2毫米直径的铜线。天线抽头位于2圈,反馈抽头总共位于7圈。使用这些线圈参数,%接收机覆盖大约5到12 MHz的范围,螺钉铁氧体磁芯允许进行调谐调整。线圈骨架连接引脚已被更长的版本替换,适合插入插板。}
%
%\begin{figure}[h]
%\centering
%%%\includegraphics[width=0.8\textwidth]{fig3-23.png}
%\caption{The antenna coil.}
%\end{figure}
%
%The two-stage audio amplier uses a small ferrite output transformer with a winding ratio of 10:1. If the connected headphones have %an impedance of 64 Ω with both drivers wired in series, the external resistance for the final amplier tube is 6.4 kΩ. The total %amlication provided by this amplier is so high that you need to introduce a volume control (P1).
%
%\textbf{中文:两级音频放大器使用一个具有10:1绕组比的小型铁氧体输出变压器。如果连接的耳机具有64 Ω的阻抗,两个驱动器串联接线,最终放大器电子管的外%部电阻6.4 kΩ。这个放大器提供的总放大非常高,以至于您需要引入音量控制(P1)。}
%
%\begin{figure}[h]
%\centering
%%%\includegraphics[width=0.8\textwidth]{fig3-24.png}
%\caption{The finished receiver.}
%\end{figure}
%
%The receiver was operated using a 10-meter long wire antenna. All radio stations in the 49, 41, and 31-meter bands are received %loud and clear. The sound tone is very pleasant and clear. Reception is enjoyable, especially because of the more than ample %volume. The louder stations must be turned down quite significantly. In the 40-meter amateur radio band, CW and SSB stations can %also be received with feedback engaged. Instead of headphones, a PC active speaker set can also be used for listening via a %loudspeaker.
%
%\textbf{中文:接收机使用10米长的导线天线操作,49米、41米和31米波段的所有无线电台都响亮而清晰地被接收到。声音音调非常愉快和清晰。接收是令人愉快%的,特别是因为绰绰有余的音量。较大的电台必须显著调低。在40米业余无线电波段,CW和SSB电台也可以在反馈接通的情况下被接收到。代替耳机,PC有源扬声器%组也可以用于通过扬声器收听。}
%
%The sensitivity of the Audion is in no way inferior to a superheterodyne receiver, and the sound is sometimes even better. To test %the sensitivity, you can listen with feedback engaged once with and once without an antenna connected. Without an antenna, only a %low level of noise is audible, but with an antenna connected, significant noise and crackling can be heard. The inherent noise %floor is thus clearly lower than the antenna noise level. More sensitivity will not bring any benefit in the signal to noise %ratio. The concept of the particularly low-noise cascode Audion also proves its worth here.
%
%\textbf{中文:Audion的灵敏度绝不逊色于超外差接收机,声音有时甚至更好。为了测试灵敏度,您可以接通反馈一次,一次不连接天线收听。没有天线,只能听到%低水平的噪声,但连接天线后,可以听到显著的噪声和噼啪声。固有的噪声底因此明显低于天线噪声水平。更多的灵敏度在信噪比方面不会带来任何好处。特别低噪%声的共栅共阴Audion的概念也在这里证明了它的价值。}
%
%Another possible source of interference is mains hum, which can be picked up by the high-impedance input stage if the circuit %layout is badly implemented. The RT100, however, has a continuous ground plane under the patch panel, which functions as an %effective shield. For the same reason, the receiver also shows hardly any sensitivity to hand movements close to the set, which %would otherwise cause frequency shifts.
%
%\textbf{中文:另一个可能的干扰源是电源嗡嗡声,如果电路布局实施不当,它可以被高阻抗输入级拾取。然而,RT100在插板下有一个连续的接地平面,它作为有%效的屏蔽。出于同样的原因,接收机也几乎没有对靠近设备的手部运动的敏感性,否则会导致频率偏移。}
%
%Once a circuit has been developed on the RT100, it can easily be replicated and then built more conventionally, for example, on an %aluminum chassis. However, the plug board layout has distinct advantages if you want to try out small changes. For example, you %could use a band spread for the 49-meter band or for the narrow 40-meter band between 7.0 and 7.1 MHz. Trying out various value %parallel and series capacitors available in the bottom of your junk box is faster than calculating and planning. Figure 3.25 shows %a variant for the 40 meter band.
%
%\textbf{中文:一旦在RT100上开发了电路,它就可以轻松复制,然后更传统地构建,例如,在铝底盘上。然而,如果您想尝试小的更改,插板布局有明显的优势。%例如,您可以为49米波段或7.0到7.1 MHz之间的窄40米波段使用波段扩展。尝试在您的垃圾箱底部可用的各种值并联和串联电容器比计算和规划更快。图3.25显示%40米波段的变体。}
%
%\begin{figure}[h]
%\centering
%%%\includegraphics[width=0.8\textwidth]{fig3-25.png}
%\caption{Bandspread control for the 40 m band.}
%\end{figure}
%
%\section{The 6J1 Tube Radio}
%\textbf{中文:6J1电子管收音机}
%
%This radio kit from Franzis Verlag uses the 6J1 tube which is a far-eastern clone of the EF95 tube. This nostalgic shortwave radio %is a genuine tube radio, just like the ones built in the pioneering days of radio technology. An RF tube in the receiving section %ensures excellent reception performance, while a modern amplier IC provides all the necessary room filling volume. The radio %operates from a 15 V anode voltage.
%
%\textbf{中文:这个来自Franzis Verlag的收音机套件使用6J1电子管,它是EF95电子管的远东克隆。这种怀旧短波收音机是一个真正的电子管收音机,就像在无%线电技术的先驱时代建造的那些一样。接收部分的射频电子管确保了卓越的接收性能,而现代放大器IC提供了所有必要的填充房间的音量。收音机在15 V阳极电压操%作。}
%
%\begin{figure}[h]
%\centering
%%%\includegraphics[width=0.8\textwidth]{fig3-26.png}
%\caption{The Franzis tube radio.}
%\end{figure}
%
%In this design, the tube performs three tasks: amlication, resonant circuit tuning and RF signal demodulation. The 6J1 pentode is %operated in triode mode with a connection between the screen grid and anode. The grid resistor R1 is connected to the anode, which %increases the grid bias. This gives a sufficiently large anode current at low anode voltage. With the cathode connected to the %center tap of the resonance circuit, amplied RF energy is fed back into the circuit. The tube operates in a Hartley oscillator %configuration, which amplies an incoming signal. At the same time, the grid diode rectifies the RF signal to perform demodulation.
%
%\textbf{中文:在这个设计中,电子管执行三个任务:放大、谐振电路调谐和射频信号解调。6J1五极管以三极管模式操作,屏幕栅极和阳极之间有连接。栅极电阻%器R1连接到阳极,这增加了栅极偏置。这在低阳极电压下给出了足够大的阳极电流。随着阴极连接到谐振电路的中心抽头,放大的射频能量被反馈到电路中。电子管%在Hartley振荡器配置中操作,它放大输入信号。同时,栅极二极管整流射频信号以执行解调。}
%
%\begin{figure}[h]
%\centering
%%%\includegraphics[width=0.8\textwidth]{fig3-27.png}
%\caption{The 6J1 Audion radio with output amplier IC.}
%\end{figure}
%
%By adjusting the anode voltage with the feedback controller P1, the level of amlication can be chosen so that the oscillator is %just on the verge of oscillating. At this working point, the tube compensates for all losses that occur in the resonance circuit. %The \textit{Q} factor can be increased from about 50 to over 1000. At a reception frequency of 6 MHz, the bandwidth is about 6 %kHz, which means that closely spaced transmitters can be effectively separated. The tuning peak also simultaneously leads to an %increase in signal amplitude. Therefore, RF signals of several hundred millivolts can occur at the control grid of the tube. The %AM signals are demodulated at the grid diode by an increase in grid current with greater RF signal amplitude and a decrease in %grid voltage. The demodulated AF signal appears at the same time at the grid and modulates the anode current. The AF signal %therefore appears at the anode resistor R2. T2 acts as an AF preamplifier for the integrated amplier IC1.
%
%\textbf{中文:通过反馈控制器P1调整阳极电压,可以选择放大水平,以便振荡器刚好在振荡的边缘。在这个工作点,电子管补偿谐振电路中发生的所有损耗。%\textit{Q}因数可以从大约30增加到超过1000。在6 MHz的接收频率下,带宽约为1 kHz,这意味着可以有效地分离紧密间隔的发射机。调谐峰值也同时导致信号幅%度的增加。因此,几百毫伏的射频信号可以出现在电子管的控制栅极处。AM信号在栅极二极管处通过栅极电流增加和栅极电压减少而随着更大的射频信号幅度被解%调。解调的AF信号同时出现在栅极处并调制阳极电流。AF信号因此出现在阳极电阻器R2处。T2作为集成放大器IC1的AF前置放大器。}
%
%The radio uses two battery packs: four AA batteries supply a total of 6 V for the tube heater and the AF amplier. The second is a %9 V battery connected in series with the heater battery so together they can supply an anode voltage of up to 15 V. The volume %control has a power switch which only has one contact. This disconnects the heater battery and turns off T1 to disconnect the %anode battery. The anode, screen grid, and control grid remain at 9 V when the device is switched off but the tube cathode will be %cold so no current flows in this state. When the power is switched on, T1 conducts and connects the lower end of P2 to ground. The %operating current of the anode battery is less than 1 mA, so this will usually last longer than the heater battery.
%
%\textbf{中文:收音机使用两个电池组:四节AA电池总共提供6 V用于电子管加热器和音频放大器。第二个是9 V电池,与加热器电池串联连接,因此它们可以一起提%供高达15 V的阳极电压。音量控制具有只有一个触点的电源开关。这断开加热器电池并关闭T1以断开阳极电池。当设备关闭时,阳极、屏幕栅极和控制栅极保持9 %V,但电子管阴极将是冷的,因此在这种状态下没有电流流动。当电源打开时,T1导通并将P2的下端连接到地。阳极电池的工作电流小于1 mA,因此这通常比加热器%电池持续时间更长。}
%
%\begin{figure}[h]
%\centering
%%%\includegraphics[width=0.8\textwidth]{fig3-28.png}
%\caption{The coil and IC amp mounted on the PCB.}
%\end{figure}
%
%The entire circuit fits onto a very compact PCB. All wired components and the coil are located on the assembly side. The tube %socket is installed on the reverse side.
%
%\textbf{中文:整个电路适合一个非常紧凑的PCB。所有布线组件和线圈位于装配侧。电子管插座安装在反面。}
%
%\begin{figure}[h]
%\centering
%%%\includegraphics[width=0.8\textwidth]{fig3-29.png}
%\caption{The tube socket mounted on the reverse side.}
%\end{figure}
%
%\begin{figure}[h]
%\centering
%%%\includegraphics[width=0.8\textwidth]{fig3-30.png}
%\caption{The PCB wiring plan.}
%\end{figure}
%
%\begin{figure}[h]
%\centering
%%%\includegraphics[width=0.8\textwidth]{fig3-31.png}
%\caption{All the wiring in the radio set.}
%\end{figure}
%
%In the final configuration, the circuit board is held in place by the wires to the tuning capacitor which is mounted on the front %face. The tube lines up behind an opening in the case so that users can benefit from the warm glow of nostalgia.
%
%\textbf{中文:在最终配置中,电路板通过连接到安装在前面板上的调谐电容器的导线保持在适当位置。电子管排列在机箱开口后面,以便用户可以受益于怀旧的温%暖光芒。}
%
%\section{A Tube Regen for the 80 m Band}
%\textbf{中文:40米波段的电子管再生接收机}
%
%The specification for an amateur radio receiver in terms of frequency stability and sensitivity are quite high. The goal here is %to design an amateur receiver built with vacuum tube devices which meets the spec. A design for a crystal controlled transmitter %using an EL95 tube can be found in Section 5.7. This receiver design should complete the rig.
%
%\textbf{中文:业余无线电接收机在频率稳定性和灵敏度方面的规格相当高。这里的目标是设计一个用真空电子管设备构建的业余接收机,它符合规格。使用EL95电%子管的晶体控制发射机的设计可以在第5.7节中找到。这个接收机设计应该完成设备。}
%
%\begin{figure}[h]
%\centering
%%%\includegraphics[width=0.8\textwidth]{fig3-32.png}
%\caption{A tube-based shortwave rig.}
%\end{figure}
%
%For use in amateur radio, the Franzis tube radio has been tweaked for use as an 80-meter amateur radio receiver. The crucial %change relates to the oscillator circuit. Bandwidth tuning is important so that the receiver can be precisely tuned. With five %parallel 56 pF capacitors and the 20 pF FM tuning capacitor, a band from 3500 kHz to 3620 kHz has been achieved. This includes the %entire CW band and the beginning of the SSB band. With a long antenna, the receiver is sufficiently sensitive, and the frequency %stability is also good. In this regard, the tube radio is significantly superior to its transistor radio equivalent.
%
%\textbf{中文:为了在业余无线电中使用,Franzis电子管收音机已经被调整用作80米业余无线电接收机。关键的改变与振荡器电路有关。带宽调谐很重要,以便接%收机可以被精确调谐。使用五个并联的56 pF电容器和20 pF FM调谐电容器,已经实现了从3500 kHz到3620 kHz的波段。这包括整个CW波段和SSB波段的开始。使用%长天线,接收机足够灵敏,频率稳定性也很好。在这方面,电子管收音机明显优于其晶体管收音机等效物。}
%
%\begin{figure}[h]
%\centering
%%%\includegraphics[width=0.8\textwidth]{fig3-33.png}
%\caption{The tweaked receiver.}
%\end{figure}
%
%When I first started experimenting with the tube receiver together with the tube transmitter to build a ham radio rig, it proved %really tricky to switch between send and receive. I had to flip the send switch, turn back the feedback regulator on the receiver, %and adjust the volume to a level where I could just hear a slight tone. Then, after making the CQ call, I needed to quickly %readjust everything to optimal settings, so I wouldn't miss any reply. To improve usability I decided to add an automatic mute %function to the receiver. Now when I switch to send, the transmitter applies a DC voltage to the antenna input, which turns on two %transistors. One acts as an audio gain control and switches a small resistor in parallel with the volume pot. The other connects a %resistor to the feedback potentiometer's wiper. This reduces the feedback below the oscillation threshold but does not completely %turn it off. This mute function is slightly delayed by a parallel base capacitor. With this mod you can hear the crystal %oscillator's whistle for a very brief moment after switching. This is quite useful because it allows you to estimate if you're %still on the same frequency when transmitting and receiving.
%
%\textbf{中文:当我第一次开始实验电子管接收机和电子管发射机一起构建业余无线电设备时,证明在发送和接收之间切换真的非常棘手。我必须翻转发送开关,将%接收机上的反馈调节器转回,并将音量调整到一个我只能听到轻微音调的水平。然后,在发出CQ呼叫后,我需要快速将所有内容重新调整到最佳设置,这样我就不会%错过任何回复。为了提高可用性,我决定向接收机添加一个自动静音功能。现在当我切换到发送时,发射机向天线输入施加直流电压,这打开两个晶体管。一个作为%音频增益控制,并在音量电位器并联切换一个小电阻。另一个将电阻连接到反馈电位器的滑动触头。这将反馈降低到振荡阈值以下,但不会完全关闭它。这个静音功%能被并联基极电容器稍微延迟。通过这个修改,您可以在切换后听到晶体振荡器的哨声非常短暂的时刻。这非常有用,因为它允许您估计在发送和接收时是否仍然在%相同的频率上。}
%
%On top of this I soldered two capacitors across the volume potentiometer to filter out some RF interference that was being picked %up when I touched the volume control knob while Morse keying. This has now been eliminated thanks to this low-pass filter. The %sound quality during reception is also slightly improved. However, a faint hum is still audible. This can be traced back to ripple %on the transmitter anode supply. Since the feedback still has some effect and there is still some amlication, the Audion now %operates as an AM receiver to demodulate the transmitter's 100 Hz residual hum.
%
%\textbf{中文:除此之外,我在音量电位器上焊接了两个电容器,以过滤掉一些在Morse键控时触摸音量控制旋钮时拾取的射频干扰。由于这个低通滤波器,这现在%已经被消除。接收期间的音质也略有改善。然而,仍然可以听到微弱的嗡嗡声。这可以追溯到发射机阳极电源上的纹波。由于反馈仍然有一些影响,并且仍然有一些%放大,Audion现在作为AM接收机操作,以解调发射机100 Hz残余嗡嗡声。}
%
%\chapter{The Transistor Audion}
%\textbf{中文:晶体管Audion}
%
%Compared to a vacuum tube Audion, a transistor version can be built smaller and with less effort. Its receive sensitivity will not %necessarily be any worse than its vacuum tube equivalent. In the age of semiconductors the vintage Audion design is still relevant %and its simple construction makes it an ideal practical introduction to radio frequency technology.
%
%\textbf{中文:与真空电子管Audion相比,晶体管版本可以更小且更少努力地构建。其接收灵敏度不一定比其真空电子管等效物更差。在半导体时代,复古Audion%设计仍然相关,其简单的构建使其成为射频技术的理想实用介绍。}
%
%\section{A One Transistor Radio}
%\textbf{中文:单晶体管收音机}
%
%The circuit in Figure 4.1 shows a receiver without feedback, consisting of just one transistor and one 1.5 V battery. A %low-impedance set of headphones can be used, preferably with the left and right drivers connected in series, resulting in a %working resistance of 64 Ω. The headphone jack also serves as the on/off switch, as the power supply is disconnected when the %headphone is unplugged.
%
%\textbf{中文:图4.1中的电路显示了一个没有反馈的接收机,仅由一个晶体管和一节1.5 V电池组成。可以使用低阻抗耳机,最好将左右驱动器串联连接,导致64 %Ω的工作电阻。耳机插孔也用作开/关开关,因为当耳机拔出时电源断开。}
%
%In this circuit, the transistor performs both demodulation and signal amlication. The sensitivity of this receiver is so good you %only need a 2 m length of wire as an antenna. The tap on the coil should be at about 1/5 of the total number of turns of the %resonant circuit coil. The circuit is suitable for all AM bands from long wave to shortwave. A shortwave coil can be made of 25 %turns with four taps, as described in the detector receiver in Section 2.3.
%
%\textbf{中文:在这个电路中,晶体管执行解调和信号放大。这个接收机的灵敏度非常好,您只需2米长的导线作为天线。线圈上的抽头应该大约在谐振电路线圈总%圈数的1/5处。该电路适用于从长波到短波的所有AM波段。短波线圈可以由25圈和四个抽头制成,如2.3节中描述的检波器接收机所述。}
%
%\begin{figure}[h]
%\centering
%%%\includegraphics[width=0.8\textwidth]{fig4-1.png}
%\caption{A 1-transistor receiver without feedback}
%\end{figure}
%
%This transistor receiver works in similar way to the vacuum tube design. Once again, the RF signal at the input shifts the %operating point to recover the modulation signal. In this case, the base-emitter diode also shifts the average input voltage lower %when a high amplitude signal is received. The time constant of the base circuit is so large that audio frequency signals at the %base are shorted to ground. The collector current increases with the instantaneous value of the signal amplitude.
%
%\textbf{中文:这个晶体管接收机以类似于真空电子管设计的方式工作。再次,输入处的射频信号移动工作点以恢复调制信号。在这种情况下,当接收到高幅度信号%时,基极-发射极二极管也将平均输入电压降低。基极电路的时间常数如此之大,以至于基极处的音频频率信号被短路到地。集电极电流随着信号幅度的瞬时值而增%加。}
%
%The most important difference compared to the vacuum tube radio is that the transistor circuit operates at much lower impedance %levels. The input resistance is about 5 kΩ and depends on the collector current and the current amlication factor. The input must %therefore be connected to a tap on the receiver coil that will not damp the resonant circuit too much.
%
%\textbf{中文:与真空电子管收音机相比最重要的区别是,晶体管电路在低得多的阻抗水平下操作。输入电阻约为5 kΩ,并取决于集电极电流和电流放大因数。因%此,输入必须连接到接收机线圈上的一个抽头,该抽头不会过度衰减谐振电路。}
%
%\section{A Shortwave PC Radio}
%\textbf{中文:短波PC收音机}
%
%The majority of radio receiver dongles or hardware for PCs, which pick up over-the-air radio broadcasts, are designed to cover the %FM band. There is no reason however why you can't build one which covers the medium or shortwave bands.
%
%\textbf{中文:大多数PC的无线电接收机加密狗或硬件,它们接收空中无线电广播,被设计为覆盖FM波段。然而,没有理由为什么您不能构建一个覆盖中波或短波波%段的。}
%
%The untuned receiver shown in Figure 4.2 is designed for direct connection to the microphone input of a PC's sound card. A supply %voltage of 2.5 V is provided by the PC at the sound input to power the usual electret type microphones that plug into the sound %card, a resistor of 2 to 3 kΩ provides a DC path to this supply. This means you already have supply voltage and a collector load %resistor to build a simple transistor receiver.
%
%\textbf{中文:图4.2中显示的未调谐接收机设计用于直接连接到PC声卡的麦克风输入。PC在声音输入处提供2.5 V的电源电压,为插入声卡的普通驻极体类型麦克%风供电,2.2 kΩ的电阻器提供到这个电源的直流路径。这意味着您已经有了电源电压和集电极负载电阻器来构建一个简单的晶体管接收机。}
%
%\begin{figure}[h]
%\centering
%%%\includegraphics[width=0.8\textwidth]{fig4-2.png}
%\caption{An untuned PC radio.}
%\end{figure}
%
%In operation the diode at the base-emitter transistor junction demodulates the incoming RF signal. The supply voltage via the %resistor forward biases the junction so that it only takes RF signals of just a few millivolts to produce a demodulated baseband %signal at collector. The circuit is therefore much more sensitive than a simple diode detector design.
%
%\textbf{中文:在操作中,基极-发射极晶体管结处的二极管解调输入的射频信号。通过电阻器的电源电压正向偏置该结,以便它只需要几毫伏的射频信号来在集电%极处产生解调的基带信号。因此,该电路比简单的二极管检波器设计敏感得多。}
%
%To hear the radio output on the PC speakers, you will of course need to turn the microphone input on. The corresponding control %provides volume adjustment. Among the advanced settings of the microphone input, there is an additional switchable preamplifier %that should also be turned on. The radio never needs to be disconnected from the sound card since it can be turned off by clicking %on the "mute" tickbox.
%
%\textbf{中文:要在PC扬声器上听到收音机输出,您当然需要打开麦克风输入。相应的控制提供音量调整。在麦克风输入的高级设置中,有一个额外的可切换前置放%大器也应该打开。收音机永远不需要从声卡断开,因为可以通过点击"静音"复选框来关闭它。}
%
%This receiver design hasn't got any variable capacitor tuning and its reception is therefore extremely broadband. All strong %stations from the 49 to the 19-meter band are received simultaneously. The coil has an inductance of approximately 2 µH and %consists of 15 turns in two layers using a pencil as a former. The resonant circuit capacitance of about 100 pF consists of the %transistor's base-emitter capacitance together with the antenna capacitance, resulting in a resonant frequency of approximately 11 %MHz. The low input impedance of the transistor dampens the resonant circuit to such an extent that a \textit{Q} factor of one is %obtained, meaning that the bandwidth is also approximately 11 MHz. The reception frequency ranges between 6 and 17 MHz.
%
%\textbf{中文:这个接收机设计没有任何可变电容器调谐,因此其接收是极其宽带的。从49米到19米波段的所有强电台都被同时接收到。线圈具有大约100 µH的电%感,15圈两层组成,使用铅笔作为骨架。大约100 pF的谐振电路电容由晶体管的基极-发射极电容和天线电容组成,导致大约11 MHz的谐振频率。晶体管的低输入阻%抗将谐振电路衰减到这样的程度,即获得了1的\textit{Q}因数,这意味着带宽也大约为11 MHz。接收频率范围在6到7 MHz之间。}
%
%Without any of the usual frequency selection control you can get some really surprising results. The special propagation %conditions of shortwave radio signals cause one station or another to stand out stronger at different times of the day. You will %be able to hear news broadcasts simultaneously in multiple languages, music from classical to pop, or folk songs from distant %countries. Without any tuning, you can just sit back and listen in to the entire shortwave band.
%
%\textbf{中文:没有通常的任何频率选择控制,您可以得到一些真正令人惊讶的结果。短波无线电信号的特殊传播条件导致一个电台或另一个在一天的不同时间突出%更强。您将能够同时听到多种语言的新闻广播,从古典到流行音乐,或来自遥远国家的民歌。没有任何调谐,您可以坐下来收听整个短波波段。}
%
%To restrict the number of stations received simultaneously you can introduce a tuned resonant circuit to the radio design. In %order to achieve a high \textit{Q} factor the transistor must be coupled via a low tap on the coil. The coil data and antenna %coupling are the same as in the single-transistor radio from the previous section.
%
%\textbf{中文:为了限制同时接收的电台数量,您可以在收音机设计中引入调谐谐振电路。为了实现高\textit{Q}因数,晶体管必须通过线圈上的低抽头耦合。线%圈数据和天线耦合与前一节中的单晶体管收音机相同。}
%
%\begin{figure}[h]
%\centering
%%%\includegraphics[width=0.8\textwidth]{fig4-3.png}
%\caption{The PC Audion with variable capacitor tuning.}
%\end{figure}
%
%\section{Regenerative Receiver}
%\textbf{中文:再生接收机}
%
%The simple transistor receivers described so far do not yet match up to the reception performance of a good tube radio. A %transistor radio without feedback already has relatively good sensitivity and can work with a short whip antenna. The sensitivity %can be increased significantly by using RF feedback. This involves feeding back a portion of the amplied RF signal to the resonant %circuit. This compensates for losses and increases the RF amplitude while also significantly improving selectivity. It is %important to feed back just the right amount of energy, as feeding back too much can cause self-oscillations that result in a %whistling sound. The level of feedback needs to be adjustable.
%
%\textbf{中文:到目前为止描述的简单晶体管接收机还没有达到良好电子管收音机的接收性能。没有反馈的晶体管收音机已经具有相对良好的灵敏度,并且可以使用%短鞭形天线工作。通过使用射频反馈,灵敏度可以显著增加。这涉及将一部分放大的射频信号反馈到谐振电路。这补偿损耗并增加射频幅度,同时也显著改善选择%性。重要的是反馈正好适量的能量,因为反馈太多可能导致自振荡,导致哨声。反馈水平需要可调。}
%
%The circuit diagram in Figure 4.4 bears a strong resemblance to the corresponding tube radio circuit from Section 3.3. The %feedback is also taken from a lower tap on the resonant circuit, but the transistor coupling has lower impedance. Either %headphones or an audio amplier can be directly connected to the output.
%
%\textbf{中文:图4.4中的电路图与3.3节中相应的电子管收音机电路有强烈的相似性。反馈也从谐振电路上的下部抽头获取,但晶体管耦合具有更低的阻抗。耳机或%音频放大器都可以直接连接到输出。}
%
%\begin{figure}[h]
%\centering
%%%\includegraphics[width=0.8\textwidth]{fig4-4.png}
%\caption{An Audion with feedback.}
%\end{figure}
%
%The receiver circuit achieves good sensitivity and volume but adjusting the feedback can be very tricky. The feedback loop is %quite harsh and prone to oscillation. Additionally, strong signals can cause distortion, requiring the feedback to be pegged back. %This results in a relatively good reception of strong signals, but it's difficult to find the optimal feedback setting. During %modulation peaks, the transistor enters a region of higher gain slope characteristic, causing self-oscillation. This results in %oscillations that periodically start and stop the feedback loop. The oscillation frequency is much lower than the modulation %frequency and sounds like an unpleasant rattling noise. While you can adjust the feedback control to reduce this and obtain clear %reception, the unpredictable behavior at the feedback threshold doesn't make for easy listening.
%
%\textbf{中文:接收机电路实现了良好的灵敏度和音量,但调整反馈可能非常棘手。反馈循环相当严厉,容易振荡。此外,强信号可能导致失真,需要将反馈调回。%这导致强信号的相对良好接收,但很难找到最佳反馈设置。在调制峰值期间,晶体管进入更高增益斜率特性的区域,导致自振荡。这导致周期性启动和停止反馈循环%的振荡。振荡频率远低于调制频率,听起来像不愉快的嘎嘎噪声。虽然您可以调整反馈控制来减少这一点并获得清晰的接收,但反馈阈值处的不可预测行为不容易收%听。}
%
%A softer feedback reaction requires self-regulation of the feedback, as in a well-designed tube radio. In the initial approach, %you can try reducing the base resistor and capacitor. The shorter time constant allows for a fast enough self-regulation through %the base current. The circuit now closely resembles active regeneration according to Section 2.8, except that the same transistor %is now also used for signal demodulation.
%
%\textbf{中文:更柔和的反馈反应需要反馈的自我调节,就像在良好设计的电子管收音机中一样。在最初的方法中,您可以尝试减少基极电阻器和电容器。更短的时%间常数允许通过基极电流进行足够快的自我调节。该电路现在非常类似于第2.8节中的有源再生,除了相同的晶体管现在也用于信号解调。}
%
%\begin{figure}[h]
%\centering
%%%\includegraphics[width=0.8\textwidth]{fig4-5.png}
%\caption{Feedback using a smaller time constant at the base circuit.}
%\end{figure}
%
%The circuit now exhibits a smoother feedback operation. Continuous oscillations (motor boating) can still occur but only when the %control is turned up very high. Now the self-oscillation frequencies are higher at around 20 kHz. A loud noise can be heard and %the circuit can be used in the upper shortwave band as a sensitive super regenerative receiver.
%
%\textbf{中文:该电路现在表现出更平滑的反馈操作。连续振荡(马达船声)仍然可能发生,但只有当控制调得非常高时。现在自振荡频率更高,大约为20 kHz。可%以听到大声噪声,该电路可以用作上短波波段的灵敏超再生接收机。}
%
%The smaller base capacitor of 100 pF improves the feedback operation but leads to lower volume. The AF gain drops to a minimum %because the coupling at the base attenuates lower frequencies.
%
%\textbf{中文:100 pF的更小基极电容器改善了反馈操作,但导致更低的音量。AF增益降到最小,因为基极处的耦合衰减较低频率。}
%
%As with the vacuum tube Audion (see Section 3.1), two opposing processes are at work here. A high RF signal causes an increase in %the collector current (anode demodulation) because of the transistor gain slope characteristic. At the same time, however, the %base current negatively charges the base capacitor, thus reducing the collector current (grid demodulation). Both affects %partially cancel each other out and lead to a low AF output voltage.
%
%\textbf{中文:与真空电子管Audion一样(见第3.1节),这里有两个相反的过程在工作。高射频信号导致集电极电流增加(阳极解调),因为晶体管增益斜率特%性。然而,同时,基极电流负向充电基极电容器,从而减少集电极电流(栅极解调)。两种效应部分相互抵消,导致低的AF输出电压。}
%
%It has therefore been shown that optimizing the component values of an Audion stage built with a transistor poses greater %difficulties than one with a vacuum tube. One solution is to divide the tasks. Separate transistors can be used for the feedback %and the demodulator stages. The feedback stage is given a small time constant in the interest of a smooth feedback control. The %Audion stage, on the other hand, uses a large base capacitor that shorts the base to ground even for AF signals.
%
%\textbf{中文:因此已经表明,用晶体管构建的Audion级的元件值优化比用真空电子管的更困难。一个解决方案是划分任务。可以使用单独的晶体管用于反馈和解调%器级。反馈级被赋予小的时间常数,以实现平滑的反馈控制。另一方面,Audion级使用大的基极电容器,即使对于AF信号也将基极短路到地。}
%
%\begin{figure}[h]
%\centering
%%%\includegraphics[width=0.8\textwidth]{fig4-6.png}
%\caption{An Audion using separated feedback stages.}
%\end{figure}
%
%This circuit actually delivers reception results that come close to a good tube radio. Good sensitivity and volume, a smooth %feedback loop, and good sound even when receiving stronger stations provide the basis for a powerful shortwave receiver.
%
%\textbf{中文:该电路实际上提供了接近良好电子管收音机的接收结果。良好的灵敏度和音量,平滑的反馈循环,以及即使在接收较强电台时也良好的声音,为强大%的短波接收机提供了基础。}
%
%\section{Separated Feedback Paths}
%\textbf{中文:分离的反馈路径}
%
%It has proven to be effective to use a separate amplier for tuning the oscillator circuit and to separate it from the Audion %stage. The circuit in Figure 4.7 shows a differential amplier with two PNP transistors in an oscillator circuit. The amlication %can be adjusted within wide limits by controlling the emitter current with the potentiometer. One advantage of the circuit is that %the oscillator coil requires just a single tap. It should be placed at approximately one third of the total number of coil turns %to avoid influencing the circuit too much.
%
%\textbf{中文:已经证明使用单独的放大器用于调谐振荡器电路并将其与Audion级分离是有效的。图4.7中的电路显示了一个差分放大器,在振荡器电路中有两个%PNP晶体管。放大可以通过电位器控制发射极电流在宽范围内调整。该电路的一个优点是,振荡器线圈只需要一个抽头。它应该放置在大约线圈总圈数的三分之一处,%以避免过度影响电路。}
%
%\begin{figure}[h]
%\centering
%%%\includegraphics[width=0.8\textwidth]{fig4-7.png}
%\caption{Audion with separate regeneration circuit.}
%\end{figure}
%
%The differential amplier stage has good linearity because the nonlinear gain characteristics of both transistors largely cancel %each other out. That's why, in this circuit, there is no shift in the operating point with increasing input amplitude. Capacitor %coupling is not required, resulting in a regeneration circuit that operates with a largely flat frequency response.
%
%\textbf{中文:差分放大器级具有良好的线性度,因为两个晶体管的非线性增益特性在很大程度上相互抵消。这就是为什么,在这个电路中,随着输入幅度的增加,%工作点没有偏移。不需要电容器耦合,导致一个以很大程度上平坦的频率响应操作的再生电路。}
%
%\begin{figure}[h]
%\centering
%%%\includegraphics[width=0.8\textwidth]{fig4-8.png}
%\caption{The transistor Audion circuit board.}
%\end{figure}
%
%In the November 2000 issue of \textit{Elektor Magazine}, a circuit for a shortwave Audion radio was described along with a PCB %designed for the circuit. A TDA7052 audio amplier IC takes care of the AF stage and the circuit employs an NPN transistor in an %Audion detector configuration, using with two PNP transistors in a separate regeneration circuit. The voltage at the feedback %potentiometer is stabilized by a forward biased LED. The coil has multiple taps, allowing for adjustment of the coupling between %the antenna of the audio input and the regeneration circuit. Typically, the device covers a frequency range of about 5 to 12 MHz. %An additional 300 pF capacitor switches the receive frequency to the 80-meter amateur radio band (3.5 to 3.8 MHz), where you can %pick up SSB and CW transmissions.
%
%\textbf{中文:在2000年1月的《Elektor杂志》中,描述了一个短波Audion收音机的电路,以及为该电路设计的PCB。TDA7052音频放大器IC负责AF级,该电路在%Audion检波器配置中使用NPN晶体管,在单独的再生电路中使用两个PNP晶体管。反馈电位器处的电压由正向偏置的LED稳定。线圈有多个抽头,允许调整音频输入的%天线和再生电路之间的耦合。通常,该设备覆盖大约5到12 MHz的频率范围。一个额外的300 pF电容器将接收频率切换到40米业余无线电波段(7.5到7.8 MHz),在%那里您可以接收SSB和CW传输。}
%
%\begin{figure}[h]
%\centering
%%%\includegraphics[width=0.8\textwidth]{fig4-9.png}
%\caption{The receiver circuit diagram.}
%\end{figure}
%
%The two PNP transistors in a differential amplier configuration practically work as an oscillator. Therefore, you can add the %missing carrier for SSB and CW signal reception. For AM reception, however, you set the current so that all losses are just %compensated for and no oscillations occur. With optimal regeneration, the circuit provides very good amlication and selectivity. %The circuit does not suffer from large signal and intermodulation products that affect many other types of receiver design because %only the desired signal is amplied by regeneration. In practice, this simple circuit can outperform some lower-priced PLL world %band receivers in terms of sound and sensitivity.
%
%\textbf{中文:差分放大器配置中的两个PNP晶体管实际上作为振荡器工作。因此,您可以为SSB和CW信号接收添加缺失的载波。然而,对于AM接收,您设置电流,%以便所有损耗都得到补偿,并且没有振荡发生。在最佳再生下,该电路提供非常好的放大和选择性。该电路不受影响许多其他类型接收机设计的大信号和互调产物的%影响,因为只有所需的信号通过再生被放大。在实践中,这个简单的电路在声音和灵敏度方面可以超越一些较低价格的PLL世界波段接收机。}
%
%T1 and T2 form a differential amplier, with the input (base of T2) and output (collector of T1) both connect to the coil. This %configuration acts like a negative differential resistance and adds the regeneration signal to the resonant circuit. The circuit %can be connected to the hot end or to a tap point on the coil. For coils with high damping, the lowest tap may have too small an %impedance, so that no oscillations can occur. The resonant circuit is theoretically best when the loss resistance is exactly %compensated for by the negative resistance generator circuit. The gain is set by the emitter current. The antenna also dampens %circuit; a long antenna should be connected to a lower tap point.
%
%\textbf{中文:T1和T2形成一个差分放大器,输入(T2的基极)和输出(T1的集电极)都连接到线圈。这个配置像负差分电阻一样,将再生信号添加到谐振电路。该%电路可以连接到热端或线圈上的抽头点。对于高阻尼的线圈,最低的抽头可能具有太小的阻抗,因此不会发生振荡。谐振电路在理论上最佳,当损耗电阻由负电阻发%生器电路精确补偿时。增益由发射极电流设置。天线也衰减电路;长天线应该连接到较低的抽头点。}
%
%The coil has four times five turns, i.e., 20 turns with three tap points. A separate antenna coupling coil is not needed since the %resonant circuit coil has multiple taps. Try each tap point for the most favorable match.
%
%\textbf{中文:线圈有四乘五圈,即20圈,有三个抽头点。不需要单独的天线耦合线圈,因为谐振电路线圈有多个抽头。尝试每个抽头点以获得最有利的匹配。}
%
%A 50 cm to 3 m length of wire is all you need for the antenna. You can easily listen to all the strong stations in the 49 m and 41 %m bands. With feedback engaged, you can hear CW signals in the 40 m amateur radio band. The 80 m band can also be monitored with %jumper JP1 in place.
%
%\textbf{中文:10厘米到1米长的导线是您天线所需的全部。您可以轻松收听49米和41米波段的所有强电台。接通反馈后,您可以听到40米业余无线电波段的CW信%号,80米波段也可以通过放置跳线JP1来监控。}
%
%\section{Regeneration using an Emitter Follower}
%\textbf{中文:使用发射极跟随器的再生}
%
%Using an emitter-follower configuration has often proven to be effective. This circuit operates similar to a diode circuit with %biasing, where the input resistance is increased by the current amlication of the transistor. While you can expect a lower voltage %gain here, it can be easily compensated for by subsequent stages.
%
%\textbf{中文:使用发射极跟随器配置已被证明是有效的。该电路类似于具有偏置的二极管电路,其中输入电阻通过晶体管的电流放大而增加。虽然您可以预期这里%电压增益较低,但可以通过后续级轻松补偿。}
%
%\begin{figure}[h]
%\centering
%%%\includegraphics[width=0.8\textwidth]{fig4-10.png}
%\caption{MW Audion in collector circuit.}
%\end{figure}
%
%One advantage of this circuit is the use of a simple coil without any tap points. This is possible because the collector circuit %has a high input resistance. Regeneration is also achieved without tapping the coil. The RF voltage is coupled into the resonant %circuit via a capacitive voltage divider provided by the transistor base-emitter capacitance Cbe and the emitter capacitor. %Amlication is adjusted by controlling the collector voltage. This results in an easily adjustable regeneration control with soft %oscillation onset. This circuit is suitable for a wide frequency range from about 50 kHz to 4 MHz i.e., from the longwave band to %the lower shortwave band. By switching coils, multiple bands can also be covered.
%
%\textbf{中文:该电路的一个优点是使用没有任何抽头点的简单线圈。这是可能的,因为集电极电路具有高输入电阻。再生也无需抽头线圈即可实现。射频电压通过%由晶体管基极-发射极电容Cbe和发射极电容器提供的电容性电压分压器耦合到谐振电路。放大通过控制集电极电压来调整。这导致一个易于调整的再生控制,具有柔%和的振荡开始。该电路适用于从大约50 kHz到3 MHz的宽频率范围,即从长波波段到下短波波段。通过切换线圈,也可以覆盖多个波段。}
%
%The circuit can also be used without any regeneration regulation by using a lower operating voltage. Figure 4.11 shows a %four-stage medium wave radio with a speaker that operates with only 1.5 V. With a current consumption of only 10 mA the life for %an alkaline AA battery is approximately 200 hours. The radio works well with the internal ferrite antenna but adding a wire of %about 2 meters as an additional antenna allows for the reception of more distant stations.
%
%\textbf{中文:该电路也可以通过使用较低的操作电压来使用,而无需任何再生调节。图4.11显示了一个四级中波收音机,带有扬声器,仅1.5 V操作。在10 mA的%电流消耗下,碱性AA电池的寿命大约为200小时。收音机与内部铁氧体天线工作良好,但添加2米的导线作为额外天线允许接收更远的电台。}
%
%\begin{figure}[h]
%\centering
%%%\includegraphics[width=0.8\textwidth]{fig4-11.png}
%\caption{The mediumwave radio receiver with loudspeaker.}
%\end{figure}
%
%The four-stage circuit has a significant overall gain, which means there is a potential risk of instability due to unwanted %feedback of audio or RF signals. To prevent this, the Audion stage has its own supply smoothing capacitor, which prevents signals %from it coupling to the power supply. The following audio stages operate with reduced cutoff frequency to prevent self-oscillation %due to parasitic capacitances.
%
%\textbf{中文:四级电路具有显著的总体增益,这意味着由于音频或射频信号的不需要的反馈,存在不稳定的潜在风险。为了防止这一点,Audion级有自己的电源平%滑电容器,这防止信号耦合到电源。后续音频级以降低的截止频率操作,以防止由于寄生电容导致的自振荡。}
%
%\section{A Medium Wave Receiver using the TA7642}
%\textbf{中文:使用TA7642的中波接收机}
%
%The integrated medium-wave receiver module ZN414 from Ferranti was later replaced by the MK484 and is now available as the TA7642. %This 3-pin integrated module is packaged in a TO92 outline, designed to operate from a 1.5 V supply. Figure 4.12 shows a basic %circuit where it is used to build a simple medium-wave receiver with a ferrite rod and variable capacitor for station tuning. The %receiver has good sensitivity and selectivity and is comparable in reception performance to simple superheterodyne receivers. In %the evening, reception across Europe is possible.
%
%\textbf{中文:来自Ferranti的集成中波接收机模块ZN414后来被MK484取代,现在作为TA7642提供。这是一个3引脚集成模块封装在TO92外形中,设计为从1.5 V%电源操作。图4.12显示了一个基本电路,其中它用于构建一个简单的中波接收机,带有铁氧体棒和可变电容器用于电台调谐。接收机具有良好的灵敏度和选择性,在%接收性能上可与简单的超外差接收机相媲美。在晚上,可以接收整个欧洲。}
%
%\begin{figure}[h]
%\centering
%%%\includegraphics[width=0.8\textwidth]{fig4-12.png}
%\caption{A medium wave receiver using a ferrite rod antenna.}
%\end{figure}
%
%This circuit has a large overall gain and stimulates the resonant circuit depending on the supply voltage through a negative input %resistance. Circuit stability is not guaranteed with a supply voltage higher than 1.5 V. If there are self-oscillations with a %high-\textit{Q} resonant circuit, a 200 kΩ to 1 MΩ resistor can be connected in parallel to the coil to ensure stability.
%
%\textbf{中文:该电路具有大的总体增益,并通过负输入电阻根据电源电压激励谐振电路。对于高于1.5 V的电源电压,电路稳定性不保证。如果高Q谐振电路存在自%振荡,可以将100 kΩ到4.7 MΩ的电阻器并联连接到线圈以确保稳定性。}
%
%The circuit contains a simple gain control so that weak and strong stations are received with almost the same volume. Large RF %input signals increase the current consumption and voltage drop across the working resistor. This reduces the supply voltage at %pin 3 which decreases the gain.
%
%\textbf{中文:该电路包含一个简单的增益控制,以便弱和强电台以几乎相同的音量接收。大的射频输入信号增加电流消耗和工作电阻器两端的电压降。这减少了引%脚3处的电源电压,从而降低增益。}
%
%Figure 4.13 shows the internal circuitry of the TA7642. You can see an emitter follower with T1 as a high-impedance input stage. %T4, T5, T7, and T9 form four RF amplier stages. T10 is the actual demodulator. All other transistors form auxiliary circuits to %stabilize the operating points. The collector of the output stage T10 is also connected to the operating voltage of all preceding %stages and reduces the gain at high RF input voltages. Only very small values of capacitor can be fabricated in integrated %circuits so this receiver only works with high gain from around 500 kHz. The upper cut-off frequency is determined by the %high-impedance design and the junction capacitances. The IC can be used but with certain limitations in the longwave and in the %lower shortwave bands.
%
%\textbf{中文:图4.13显示了TA7642的内部电路。您可以看到一个发射极跟随器,T1作为高阻抗输入级。T4、T5、T7和T9形成四个射频放大级。T10是实际的解调%器。所有其他晶体管形成辅助电路以稳定工作点。输出级T10的集电极也连接到所有前级的操作电压,并在高射频输入电压下降低增益。在集成电路中只能制造非常小%的电容值,因此该接收机仅在500 kHz以上具有高增益时工作。上限截止频率由高阻抗设计和结电容决定。IC可以在长波和下短波波段使用,但有一定限制。}
%
%\begin{figure}[h]
%\centering
%%%\includegraphics[width=0.8\textwidth]{fig4-13.png}
%\caption{The TA7642 Block diagram.}
%\end{figure}
%
%The receiver chip is well-suited to build a medium-wave PC radio, which, like the shortwave radio described in Section 4.2, is %powered through the sound card's microphone input. The higher operating voltage of 2.5 V necessitates the use of an additional %series resistor of 10 kΩ for stability purposes. The resistor should be bridged with an electrolytic capacitor to allow the full %audio signal to be applied to the microphone input.
%
%\textbf{中文:接收机芯片非常适合构建中波PC收音机,就像4.2节中描述的短波收音机一样,通过声卡的麦克风输入供电。1.5 V的较高操作电压需要使用额外的%10 kΩ串联电阻器以确保稳定性。电阻器应该用电解电容器桥接,以便将完整的音频信号施加到麦克风输入。}
%
%\begin{figure}[h]
%\centering
%%%\includegraphics[width=0.8\textwidth]{fig4-14.png}
%\caption{Powered from the PC's sound card.}
%\end{figure}
%
%The TA7642 is currently not stocked by many suppliers so it would be useful if you try to replicate its function using discrete %components. A block diagram of the chip shows its working principle so it's not so difficult to reduce the circuit to its %essential components. Figure 4.15 shows an almost equivalent replacement circuit using only three BC548C NPN transistors. This %substitute consists of one emitter follower input stage, only one RF stage, and the demodulator stage.
%
%\textbf{中文:TA7642目前没有被许多供应商库存,因此如果您尝试使用分立元件复制其功能,那将是有用的。芯片的框图显示了它的工作原理,因此将电路减少到%其基本组件并不那么困难。图4.15显示了一个几乎等效的替代电路,仅使用三个BC548C NPN晶体管。这个替代品由一个发射极跟随器输入级、仅一个射频级和解调器%级组成。}
%
%\begin{figure}[h]
%\centering
%%%\includegraphics[width=0.8\textwidth]{fig4-15.png}
%\caption{A replica of the TA7642 built using discrete components.}
%\end{figure}
%
%The replica circuit shows very similar characteristics to the original but tends to be more prone to bursting into oscillation due %to the larger emitter current in the input stage. This means that the regeneration effect is stronger. In principle, this is the %same oscillator circuit as the emitter follower used in Section 4.5. If you use a coil with low self-damping, it may be necessary %to use a resistor of value 200 kΩ to 1 MΩ in parallel with the resonant circuit, which partially compensates for the negative %input resistance. Another advantage of this substitute circuit is that the larger coupling capacitor also allows operation in the %long-wave band.
%
%\textbf{中文:复制品电路显示出与原始电路非常相似的特性,但由于输入级中较大的发射极电流,倾向于更容易爆发振荡。这意味着再生效应更强。原则上,这与%第4.5节中使用的发射极跟随器是相同的振荡器电路。如果您使用具有低自阻尼的线圈,可能需要使用100 kΩ到4.7 MΩ的电阻器与谐振电路并联,这部分补偿负输入%电阻。这个替代电路的另一个优点是,较大的耦合电容器也允许在长波波段操作。}
%
%\section{A Retro Medium Wave Radio}
%\textbf{中文:复古中波收音机}
%
%The design of Franzis-Radio (which uses a TA7642) is reminiscent of valve radio sets that were popular in the 1950s. In 2008 this %kit originally included a signal strength analogue meter but later versions were housed in a smaller case without the meter. More %recently, broadcastors have turned to more contemporary methods of sending out program material and in a lot of countries many %medium wave stations are now silent. The kit is no longer available but the concept is still relevant; you can still listen to %many interesting stations from all over Europe, especially during the evenings.
%
%\textbf{中文:Franzis-Radio(使用TA7642)的设计让人想起1950年代流行的电子管收音机。在2008年,这个套件最初包括一个信号强度模拟仪表,但后来的版%本被安置在一个没有仪表的更小外壳中。最近,广播公司已经转向更现代的发送节目材料的方法,在许多国家,许多中波电台现在已经沉默。该套件不再可用,但概%念仍然相关;您仍然可以收听来自整个欧洲的许多有趣的电台,特别是在晚上。}
%
%\begin{figure}[h]
%\centering
%%%\includegraphics[width=0.8\textwidth]{fig4-16.png}
%\caption{The medium wave Franzis set oozes nostalgia.}
%\end{figure}
%
%The resonant circuit which serves as the receiving antenna consists of a ferrite-core coil and variable capacitor. The RF signal %is taken out at a tap point on the coil and fed to the input of the receiver IC (Pin 1). At the output (Pin 3), both the %demodulated audio signal and a control voltage for automatic gain control are present. This voltage drops from 1.2 V without a %signal to below 1 V with a strong signal. The control voltage is fed back to the input via R4 and adjusts the receiver's %amlication. This feedback ensures that strong and weak stations appear to sound almost equally loud.
%
%\textbf{中文:作为接收天线的谐振电路由铁氧体芯线圈和可变电容器组成。射频信号在线圈上的抽头点处取出,并馈送到接收机IC的输入(引脚1)。在输出(引%脚3)处,存在解调后的音频信号和用于自动增益控制的控制电压。该电压在没有信号时约1.2 V下降到强信号时低约0.3 V。控制电压通过R4反馈到输入并调整接收%机的放大。这种反馈确保强和弱电台听起来几乎同样响亮。}
%
%\begin{figure}[h]
%\centering
%%%\includegraphics[width=0.8\textwidth]{fig4-17.png}
%\caption{Schematic of the medium wave radio.}
%\end{figure}
%
%A regulated voltage between approximately 1 and 1.2 V is supplied via the volume control to the base of the amplier transistor T1. %The operating point at around 20 mA is thus largely independent of the battery voltage and variations in the transitor's current %gain, but sensitive to changes in the received signal strength. The display pointer shows the emitter voltage and thus also the %emitter current of T1. The current is reduced by about 5 mA when the volume control is turned down because an additional base %resistor of up to 10 kΩ reduces the base current. The gauge indicator pointer shows all changes in the emitter current and thus %also the state of the battery, the set volume, and the signal strength of the selected station.
%
%\textbf{中文:通过音量控制向放大器晶体管T1的基极提供大约0到1.2 V之间的稳定电压。因此,大约20 mA的工作点在很大程度上独立于电池电压和晶体管电流增%益的变化,但对接收信号强度的变化敏感。显示指针显示发射极电压,因此也显示T1的发射极电流。当音量控制调低时,电流减少到约5 mA,因为高达470 kΩ的额外%基极电阻器减少基极电流。仪表指示器指针显示发射极电流的所有变化,因此也显示电池状态、设置的音量和所选电台的信号强度。}
%
%The circuit is particularly efficient and requires only a single 1.5 V battery. A typical alkaline battery with a capacity of 2000 %mAh will power it for 100 hours at high volume. Turning down the volume will extend battery life.
%
%\textbf{中文:该电路特别高效,只需要一节1.5 V电池。容量为2000 mAh的典型碱性电池将在高音量下为其供电100小时。调低音量将延长电池寿命。}
%
%The pointer deflection decreases when a station is received, which helps with more precise tuning. In old tube radios this %function was often realized with a "magic eye" vacuum tube.
%
%\textbf{中文:当接收到电台时,指针偏转减少,这有助于更精确的调谐。在旧的电子管收音机中,这个功能通常通过"魔眼"电子管实现。}
%
%The ferrite rod antenna is quite directional; the received signal strength will be at a maximum when the ferrite rod axis is %perpendicular to the signal source. This directional property is useful if you need to suppress a strong signal from a transmitter %interfering with another station you want to listen to. Just rotate the radio so that the interfering signal drops to a minimum. %At this point you know the ferrite rod axis is directed at the location of the interfering signal transmitter.
%
%\textbf{中文:铁氧体棒天线具有相当的方向性;当铁氧体棒轴垂直于信号源时,接收信号强度将最大。这种方向性属性很有用,如果您需要抑制来自发射机的强信%号干扰您想要收听的另一个电台。只需旋转收音机,使干扰信号降到最小。此时您知道铁氧体棒轴指向干扰信号发射机的位置。}
%
%\begin{figure}[h]
%\centering
%%%\includegraphics[width=0.8\textwidth]{fig4-18.png}
%\caption{The populated PCB.}
%\end{figure}
%
%The small PCB is easy to assemble. In addition to the receiver IC, there is a transitor which works as a power amplier. When fully %assembled, the PCB is quite light and can be held in place just by the wires soldered to the tuning capacitor which is mounted to %the front panel.
%
%\textbf{中文:小型PCB易于组装。除了接收机IC外,还有一个晶体管作为功率放大器。完全组装后,PCB相当轻,可以仅通过焊接到调谐电容器的导线保持在适当位%置,调谐电容器安装在前面板上。}
%
%\begin{figure}[h]
%\centering
%%%\includegraphics[width=0.8\textwidth]{fig4-19.png}
%\caption{It all fits in the case.}
%\end{figure}
%
%\section{Shortwave Regen using an Emitter Follower}
%\textbf{中文:使用发射极跟随器的短波再生接收机}
%
%A shortwave radio using just two transistors and powered by a 1.5 V battery is a great way to get started with shortwave reception %technology. You can connect the receiver to a set of active PC speakers to provide impressive reception performance.
%
%\textbf{中文:仅使用两个晶体管并1.5 V电池供电的短波收音机是开始短波接收技术的好方法。您可以将接收机连接到一组有源PC扬声器,以提供令人印象深刻的%接收性能。}
%
%The circuit has a unique feature. The BC558C PNP transitor works as an Audion in a collector circuit (emitter follower, see %Section 4.5). Only a very low emitter current is required to achieve oscillation. Using the potentiometer, for AM reception you %can adjust the Audion feedback level to the point just before oscillation starts and for CW and SSB reception, just after %oscillation has begun.
%
%\textbf{中文:该电路具有独特的功能。BC558C PNP晶体管在集电极电路(发射极跟随器,见4.5节)中作为Audion工作。只需要非常低的发射极电流即可实现振%荡。使用电位器,对于AM接收,您可以将Audion反馈水平调整到振荡开始之前的点,对于CW和SSB接收,就在振荡开始之后。}
%
%\begin{figure}[h]
%\centering
%%%\includegraphics[width=0.8\textwidth]{fig4-20.png}
%\caption{The 2-transistor Audion receiver.}
%\end{figure}
%
%The entire circuit was soldered onto the cut-out lid of a coffee can. A 100 pF trimmer is used as the tuning capacitor but you can %also change station by adjusting the core in the coil former. Even though you will need to use a screwdriver to select a station, %it's easy to tune in.
%
%\textbf{中文:整个电路被焊接到咖啡罐的切盖上。使用330 pF微调电容器作为调谐电容器,但您也可以通过调整线圈架中的磁心来更改电台。即使您需要使用螺丝%刀选择电台,也很容易调谐。}
%
%\begin{figure}[h]
%\centering
%%%\includegraphics[width=0.8\textwidth]{fig4-21.png}
%\caption{Experimental construction.}
%\end{figure}
%
%The Audion features smooth feedback behavior and is relatively easy to use. The receiver frequency range covers from the 49-meter %band to the 31-meter band. You can also listen to amateur radio in CW and SSB between 7.0 and 7.1 MHz. A wire antenna with a %minimum length of 3 meters works well with this design.
%
%\textbf{中文:Audion具有平滑的反馈行为,相对易于使用。接收机频率范围覆盖49米波段到31米波段。您还可以在7.0到7.1 MHz之间收听业余无线电CW和SSB。%最小长度为3米的导线天线在此设计上工作良好。}
%
%\begin{figure}[h]
%\centering
%%%\includegraphics[width=0.8\textwidth]{fig4-22.png}
%\caption{Expanding the design using 4 transistors.}
%\end{figure}
%
%This emitter-follower Audion design was adapted without any tuning and for loudspeaker operation. The idea is that anyone should %be able to build it without the need for any special RF components; it doesn't even use a tuning capacitor or shortwave coil. All %that's required is a handful of standard components, a speaker, an old tin lid and a 1.5 V battery.
%
%\textbf{中文:这个发射极跟随器Audion设计被改编,没有任何调谐,用于扬声器操作。想法是,任何人都应该能够构建它,而不需要任何特殊的射频组件;它甚至%不使用调谐电容器或短波线圈。所需要的只是一些标准组件、一个扬声器、一个旧锡盖和一节1.5 V电池。}
%
%\begin{figure}[h]
%\centering
%%%\includegraphics[width=0.8\textwidth]{fig4-23.png}
%\caption{Loudspeaker operation.}
%\end{figure}
%
%Without a trimming capacitor, how can the receiver be tuned? The Audion uses a number of fixed value capacitors soldered together %wired in parallel with the home-brew variometer coil that achieves fine-tuning by squashing or stretching the coil turns. To make %this coil 17 turns of wire are tightly wound around an AA battery. When the coil is removed from the battery it expands slightly %leaving 15 turns with a diameter of 17 mm.
%
%\textbf{中文:没有微调电容器,如何调谐接收机?Audion使用许多固定值电容器焊接在一起,与自制变感器线圈并联,通过挤压或拉伸线圈圈数实现精细调谐。为%了制作这个线圈,17圈导线紧密缠绕在AA电池周围。当线圈从电池上移除时,它稍微膨胀,留下15圈,直径17 mm。}
%
%In use its necessary to find the desired frequency and antenna coupling, then you can turn up the potentiometer near the feedback %loop, until you hear an increase in noise which indicates an increased sensitivity. The coil can now be carefully squashed or %stretched until the desired station is clearly audible. In the first attempt, I managed to pick up a strong station in the %49-meter band. Many other stations will also be available at dusk. To simplify tuning, the radio was equipped with a lever %mechanism at the end to adjust the coil length. With a little skill, the frequency can be set as precisely as with a tuning %capacitor.
%
%\textbf{中文:在使用时,有必要找到所需的频率和天线耦合,然后您可以调高反馈循环附近电位器,直到您听到噪声增加,这表明灵敏度增加。现在可以小心地挤%压或拉伸线圈,直到所需的电台清晰可听。在第一次尝试中,我设法在49米波段接收到一个强电台。在黄昏时,许多其他电台也将可用。为了简化调谐,收音机在末%端配备了杠杆机构来调整线圈长度。有一点技巧,频率可以像使用调谐电容器一样精确设置。}
%
%\begin{figure}[h]
%\centering
%%%\includegraphics[width=0.8\textwidth]{fig4-24.png}
%\caption{Tuning using a DIY variometer coil adjuster.}
%\end{figure}
%
%\section{A Shortwave Retro Radio}
%\textbf{中文:短波复古收音机}
%
%The Franzis shortwave radio is a transitor Audion for the range of 3.5 to 9.5 MHz. The emitter-follower circuit was also used %here, but in addition, an audio amplier using the LM386 was added.
%
%\textbf{中文:Franzis短波收音机是一个用于2.5到12.5 MHz范围的晶体管Audion。这里也使用了发射极跟随器电路,但此外,还添加了使用LM386的音频放大%器。}
%
%\begin{figure}[h]
%\centering
%%%\includegraphics[width=0.8\textwidth]{fig4-25.png}
%\caption{The Shortwave Transistor Audion radio.}
%\end{figure}
%
%Transistor T1 operates as an emitter follower in an Audion configuration performing three tasks: amlication, adding regenerative %feedback to the resonant circuit and demodulating the RF signal. C2 and the internal base-emitter capacitance of about 5 pF form a %capacitive voltage divider. Together with the resonant circuit, a Colpitts oscillator is formed. By adjusting the emitter current %appropriately, the gain can be chosen such that the oscillator is just about to start oscillating. At this operating point, the %transitor compensates for all losses incurred in the resonant circuit. The quality factor \textit{Q} can be increased from about %50 to over 1000. Correctly adjusted this can provide the radio with a receive bandwidth of about 6 kHz on a 6 MHz radio signal to %give good separation of adjacent broadcast stations.
%
%\textbf{中文:晶体管T1在Audion配置中作为发射极跟随器工作,执行三个任务:放大、向谐振电路添加再生反馈和解调射频信号。C2和大约4 pF的内部基极-发射%极电容形成一个电容性电压分压器。与谐振电路一起,形成Colpitts振荡器。通过适当调整发射极电流,可以选择增益,使得振荡器即将开始振荡。在这个工作点,%晶体管补偿谐振电路中发生的所有损耗。品质因数Q可以从大约50增加到超过1000。正确调整这可以为收音机提供1 MHz射频信号上大约1 kHz的接收带宽,以良好地%分离相邻广播电台。}
%
%The regenerative feedback leads to an increase in the received signal amplitude. RF signals up to about 100 mV can therefore occur %at the transitor base. The AM signals are demodulated by the non-linear transitor characteristics. The AF signal then appears at %the emitter. R1 and C2 form a low-pass filter that removes any RF signal remnants. T2 forms an AF preamplifier for the integrated %amplier IC1. The AF stage also uses a PNP transitor to avoid any possible mix up during construction.
%
%\textbf{中文:再生反馈导致接收信号幅度增加。因此,高达100 mV的射频信号可能出现在晶体管基极处。AM信号通过非线性晶体管特性被解调。AF信号然后出现%在发射极处。R1和C2形成一个低通滤波器,去除任何射频信号残余。T2为集成放大器IC1形成AF前置放大器。AF级也使用PNP晶体管,以避免在构建期间可能的混%淆。}
%
%\begin{figure}[h]
%\centering
%%%\includegraphics[width=0.8\textwidth]{fig4-26.png}
%\caption{Transistor Audion with an IC output amp.}
%\end{figure}
%
%One special feature of this Audion circuit is the direct coupling of the transitor to the oscillator circuit. T1 operates with a %collector-emitter voltage of only about 0.6 V. The base-emitter junction has a capacitance of about 5 pF which strongly affects %the oscillating circuit. The close coupling ensures that the transitor also acts like a capacitance diode or varicap, allowing %fine-tuning of the frequency via feedback control. The regeneration and onset of oscillation is quite soft so the frequency can be %pulled by several kilohertz, which is advantageous for receiving SSB and CW stations.
%
%\textbf{中文:这个Audion电路的一个特殊功能是晶体管与振荡器电路的直接耦合。T1仅以大约0.6 V的集电极-发射极电压操作。基极-发射极结具有大约5 pF的%电容,这强烈影响振荡电路。紧密耦合确保晶体管也像电容二极管或变容二极管一样,允许通过反馈控制精细调谐频率。再生和振荡的开始相当柔和,因此频率可以%被拉动几千赫兹,这对于接收SSB和CW电台是有利的。}
%
%The LM386 speaker amplier operates directly from a 9 V battery. Power consumption depends heavily on the volume setting. At low %volume, the entire receiver only consumes about 5 mA. The LED serves not only to show the receiver is on but also for voltage %stabilization as the LED has a constant 1.8 V forward voltage drop. As a result, the two transitor stages always receive a stable %operating voltage.
%
%\textbf{中文:LM386扬声器放大器直接从9 V电池操作。功耗很大程度上取决于音量设置。在低音量下,整个接收机仅消耗约5 mA。LED不仅显示接收机开启,还用%于电压稳定,因为LED具有恒定1.8 V正向电压降。因此,两个晶体管级总是接收稳定的操作电压。}
%
%\begin{figure}[h]
%\centering
%%%\includegraphics[width=0.8\textwidth]{fig4-27.png}
%\caption{The PCB and components mounted in the case.}
%\end{figure}
%
%When tuning the frequency, you will find several shortwave bands with multiple stations. On shortwave, you can achieve a high %range during the day, but many stations only broadcast in the evening. Below 4 MHz is the 75-meter band, which is often overlooked %on many shortwave receivers. On this band you can pick up a few interesting stations in the evening. The 49-meter band at 6 MHz is %densely occupied by numerous European stations. Some frequencies are used consecutively by different stations. The 41-meter band %above 7 MHz is also very busy in the evening. This receiver also reaches parts of the 31-meter band above 9 MHz. Generally, you %can achieve a better range at higher frequencies. Often, you can also receive stations from outside Europe. Between broadcast %bands, there are numerous stations in CW (Morse telegraphy), SSB (single-sideband voice radio), RTTY (radio teletype), and %weatherfax (weather picture radio). All of these stations can only be heard with activated regeneration.
%
%\textbf{中文:当调谐频率时,您会发现几个短波波段有多个电台。在短波上,您可以在白天实现高范围,但许多电台只在晚上广播。低于3 MHz的120米波段,这在%许多短波接收机上经常被忽视。在这个波段上,您可以在晚上接收到一些有趣的电台。6 MHz的49米波段被许多欧洲电台密集占据。一些频率被不同的电台连续使用。%9 MHz以上的31米波段在晚上也非常繁忙。该接收机也达到9 MHz以上31米波段的部分。通常,您可以在更高的频率上实现更好的范围。通常,您还可以接收来自欧洲%以外的电台。在广播波段之间,有CW(莫尔斯电报)、SSB(单边带语音无线电)、RTTY(无线电电传)和weatherfax(天气图片无线电)的众多电台。所有这些电%台只能在激活再生的情况下听到。}
%
%\section{A 40-m Shortwave Audion with Regeneration}
%\textbf{中文:具有再生的40米短波Audion}
%
%The Franzis shortwave radio featured in the previous section with its open construction is well-suited for shortwave radio %reception. It does, however, suffer from poor frequency stability when it comes to receiving CW and SSB signals on the amateur %radio bands. It's also difficult to set the frequency accurately. The radio can be modified to make it more suitable as an amateur %radio receiver specifically for the 40m band. The crucial improvement made here is to add some shielding to the circuits on the %PCB. This reduces its sensitivity to the proximity of the radio operator's hands and helps reduce any tendency for unwanted %circuit oscillations.
%
%\textbf{中文:前一节中介绍的Franzis短波收音机以其开放式结构非常适合短波接收。然而,在业余无线电波段接收CW和SSB信号时,它确实存在频率稳定性差的%问题。也很难准确设置频率。收音机可以修改,使其更适合作为专门用于40米波段的业余无线电接收机。这里所做的关键改进是给PCB上的电路添加一些屏蔽。这减少%了它对无线电操作员双手靠近的敏感性,并有助于减少任何不需要的电路振荡倾向。}
%
%\begin{figure}[h]
%\centering
%%%\includegraphics[width=0.8\textwidth]{fig4-28.png}
%\caption{RF shielding plate.}
%\end{figure}
%
%The thin tinplate used for the shielding is salvaged from a food container of cappuccino coffee powder. I've always kept such %sheets and used them as a base for experimental circuits. The material is very thin and can be cut with regular scissors. I first %cut out a cardboard template to determine the shielding plate dimensions. The final shield was then cut, bent, and soldered %together.
%
%\textbf{中文:用于屏蔽的薄马口铁是从卡布奇诺咖啡粉的食品容器中回收的。我一直保留这样的薄片,并将它们用作实验电路的基础。材料非常薄,可以用普通剪%刀切割。我首先切出一个纸板模板来确定屏蔽板的尺寸。然后切割、弯曲并焊接最终屏蔽。}
%
%\begin{figure}[h]
%\centering
%%%\includegraphics[width=0.8\textwidth]{fig4-29.png}
%\caption{Earthing the shield.}
%\end{figure}
%
%I swapped the connections to the variable capacitor so that now the VHF section is being used. The ground tag of the AM side is %soldered to the shield. A fairly thick ground wire also connects the variable capacitor ground terminal on the board to the metal %shield, which also sets the mounting depth. Overall, there are several ground connections to the metal sheet. I also soldered some %wire support struts to hold the free ends of the sheet and prevent mechanical vibrations. All of these measures have been very %successful and the radio set is now totally unaffected by the radio operator's hand movements.
%
%\textbf{中文:我交换了可变电容器的连接,以便现在使用VHF部分。AM侧的接地标签焊接到屏蔽上。一条相当粗的接地线还将板上的可变电容器接地端子连接到金%属屏蔽,这也设置了安装深度。总的来说,有多个到金属片的接地连接。我还焊接了一些导线支撑支柱来保持片的自由端并防止机械振动。所有这些措施都非常成%功,收音机现在完全不受无线电操作员手部移动的影响。}
%
%\begin{figure}[h]
%\centering
%%%\includegraphics[width=0.8\textwidth]{fig4-30.png}
%\caption{Capacitors in the resonant circuit.}
%\end{figure}
%
%The band spread uses a 56 pF capacitor across the resonant circuit with another 56 pF capacitor in series on one side of the 20 pF %VHF variable capacitor and its parallel 10 pF trimmer. The dimensions were not precisely calculated but by experimentation, using %whatever materials were available at the time. As a result, the receiver now covers the entire 40-meter band and the start of the %41-meter broadcast band. The base of the Audion transitor is no longer at the hot end of the coil but at the center tap that was %previously used as the antenna connection A1.
%
%\textbf{中文:波段扩展使用56 pF电容器跨接谐振电路,30 pF VHF可变电容器的一侧串联另一个56 pF电容器及其并联30 pF微调器。尺寸没有精确计算,而是通%过实验,使用当时可用的任何材料。因此,接收机现在覆盖整个40米波段和41米广播波段的开始。Audion晶体管的基极不再位于线圈的热端,而是位于之前用作天线%连接A1的中心抽头处。}
%
%Some distortion is still audible when receiving CW and SSB stations due to changes in the transitor's junction capacitance. %Depending on the signal, the operating point changes, and so does the input capacitance. This results in additional frequency %modulation and unpleasant audio effects. To reduce this effect, I installed an additional 12 pF capacitor between the base and %emitter.
%
%\textbf{中文:由于晶体管结电容的变化,在接收CW和SSB电台时仍然可以听到一些失真。取决于信号,工作点会发生变化,输入电容也会发生变化。这导致额外的%频率调制和不愉快的音频效果。为了减少这种影响,我在基极和发射极之间安装了一个额外的12 pF电容器。}
%
%\begin{figure}[h]
%\centering
%%%\includegraphics[width=0.8\textwidth]{fig4-31.png}
%\caption{The modified schematic.}
%\end{figure}
%
%With this modification, the influence of the transitor's base capacitance is reduced. The regeneration is also increased, allowing %for larger feedback signals to be set. The transitor then operates more like a direct mixer. When the regeneration signals are %significantly larger than the received signals, the operating point shifts less strongly. There is now a larger range of %regeneration regulation beyond the oscillation threshold, which can be used to help fine-tune the frequency. This makes exact %tuning of an SSB signal much easier.
%
%\textbf{中文:通过这种修改,晶体管基极电容的影响减少了。再生也增加了,允许设置更大的反馈信号。然后晶体管更像直接混频器一样操作。当再生信号明显大%于接收信号时,工作点偏移不那么强烈。现在在振荡阈值之外有更大的再生调节范围,可以用来帮助精细调谐频率。这使得SSB信号的精确调谐容易得多。}
%
%Another improvement in the signal-to-noise ratio was achieved by reducing the audio bandwidth. To do this, I just soldered a 22 nF %capacitor across the volume control. In the end, audio volume was a little lacking but the LM386 still has something in reserve. I %placed a 10 µF capacitor in series with 470 Ω resistor between pins 1 and 8. The Audion receiver in this form is suitable for %amateur radio use. It's fun to use it to listen to CW transmissions, and the Audion is also well suited as a practice receiver.
%
%\textbf{中文:通过减少音频带宽,信噪比的另一个改进得以实现。为此,我只是在音量控制上焊接了一个22 nF电容器。最后,音频音量有点不足,但LM386仍有%储备。我在引脚1和8之间串联放置了一个10 µF电容器和470 Ω电阻器。这种形式的Audion接收机适合业余无线电使用。用它来收听CW传输很有趣,Audion也非常适%合作为练习接收机。}
%
%\section{Shortwave Regen Optimization}
%\textbf{中文:短波再生接收机优化}
%
%Is it possible to optimize a transitor Audion so that it can be used for real radio operation on amateur radio bands, as was %practiced in the early days of radio? The basic idea is that if a stable and well-shielded oscillating circuit with the highest %possible \textit{Q} factor is loosely coupled, it should be possible to build a very stable and sensitive Audion regenerative %receiver. My test setup on a section of copper-clad board does not yet meet these requirements but it allows for any other %oscillating circuits to be connected easily. At least the continuous ground plane has a positive effect on circuit stability.
%
%\textbf{中文:是否可以优化晶体管Audion,使其可以用于业余无线电波段上的实际无线电操作,就像无线电早期实践中那样?基本思想是,如果具有最高可能Q因%子的稳定且良好屏蔽的振荡电路被松散耦合,应该可以构建非常稳定和灵敏的Audion再生接收机。我在一段覆铜板上的测试设置尚未满足这些要求,但它允许轻松连%接任何其他振荡电路。至少连续接地平面对电路稳定性有积极影响。}
%
%\begin{figure}[h]
%\centering
%%%\includegraphics[width=0.8\textwidth]{fig4-32.png}
%\caption{Prototype layout built on a copper-clad board.}
%\end{figure}
%
%Over the years, two Audion regen circuit designs have proven particularly effective so you can take a closer look at them here. %The first, simple version is used in the Franzis shortwave radio. A single PNP transitor is responsible for regeneration and %demodulation.
%
%\textbf{中文:多年来,两种Audion再生电路设计已被证明特别有效,因此您可以在这里仔细查看它们。第一个简单版本用于Franzis短波收音机。单个PNP晶体管%负责再生和解调。}
%
%\begin{figure}[h]
%\centering
%%%\includegraphics[width=0.8\textwidth]{fig4-33.png}
%\caption{Audion regen in PNP-Emitter follower configuration.}
%\end{figure}
%
%While this circuit works very well with careful layout, it has some disadvantages in terms of easy repeatability for those %building it themselves. The transitor is tightly coupled to the circuit, so the feedback regulator affects the frequency. This can %also be an advantage if used skillfully for fine-tuning. Overall, the gain of the regeneration circuit is low, so the antenna must %only be loosely coupled. Correct adjustment is therefore not easy. Another disadvantage is that the Audion stage has no gain, so %an AF amplier stage with a high level of gain is required.
%
%\textbf{中文:虽然这个电路在仔细布局的情况下工作得很好,但对于自己构建的人来说,在易于重复性方面有一些缺点。晶体管与电路紧密耦合,因此反馈调节器%影响频率。如果巧妙地用于精细调谐,这也可能是一个优势。总的来说,再生电路的增益很低,因此天线必须只能松散耦合。因此正确的调整并不容易。另一个缺点%是Audion级没有增益,因此需要具有高增益水平的AF放大器级。}
%
%The second proven circuit was also used in the Elektor shortwave Audion. Here, the actual Audion stage operates in emitter %configuration which has high gain. Regeneration is taken care of separately using two PNP transistors in a differential amplier %configuration.
%
%\textbf{中文:第二种经过验证的电路也用于Elektor短波Audion。在这里,实际的Audion级在发射极配置中操作,该配置具有高增益。再生使用差分放大器配置%中的两个PNP晶体管单独处理。}
%
%\begin{figure}[h]
%\centering
%%%\includegraphics[width=0.8\textwidth]{fig4-34.png}
%\caption{Audion with separate regeneration.}
%\end{figure}
%
%I based my first build on this variant. The regeneration stage has so much gain that it practically brings any oscillation circuit %to resonance. Therefore, the circuit is suitable for experiments with different oscillation circuits and also for band switching. %Unlike the original, I operated everything from 9 V this time and used an LM386 amplier in the final stage. Additionally, there is %a 5 V voltage regulator for the Audion.
%
%\textbf{中文:我的第一次构建基于这个变体。再生级具有如此多的增益,以至于实际上将任何振荡电路带到谐振。因此,该电路适合于不同振荡电路的实验,也适%合波段切换。与原始版本不同,这次我用9 V操作所有东西,并在最终级使用LM386放大器。此外,还有一个用于Audion的5 V电压调节器。}
%
%\begin{figure}[h]
%\centering
%%%\includegraphics[width=0.8\textwidth]{fig4-35.png}
%\caption{Experimental air-spaced coil with coil tap points.}
%\end{figure}
%
%The free-standing coil was wound using one meter of insulated single-strand wire. Although this construction method produces a %coil which is mechanically quite loose it is stable enough for our purposes and is easy to replicate at home without any special %tools. The tap points can be easily added using a hot soldering iron. The Audion stage is connected to the fifth turn tap and the %antenna at the first turn. Because a long wire antenna was used here, it has to be very loosely coupled. Alternatively, a large %wire-loop antenna can also be used.
%
%\textbf{中文:独立线圈使用一米绝缘单股导线缠绕。虽然这种构造方法产生的线圈在机械上相当松散,但对于我们的目的来说足够稳定,并且在家里无需任何特殊%工具即可轻松复制。可以使用热烙铁轻松添加抽头点。Audion级连接到第五圈抽头,天线连接到第一圈。因为这里使用了长导线天线,所以必须非常松散耦合。或%者,也可以使用大的导线环形天线。}
%
%\begin{figure}[h]
%\centering
%%%\includegraphics[width=0.8\textwidth]{fig4-36.png}
%\caption{Audion stage with separate regeneration circuit.}
%\end{figure}
%
%The radio was built on an all copper clad board. The individual components are soldered onto short sections of a strip board. %Figure 4.36 shows the NPN Audion stage and the two PNP transistors for the regeneration circuit. The continuous copper surface is %useful to promote operational stability and shield against electrical interference.
%
%\textbf{中文:收音机是在全覆铜板上构建的。各个组件焊接到条板的短段上。图4.36显示了NPN Audion级和用于再生电路的两个PNP晶体管。连续铜表面有助于促%进操作稳定性并屏蔽电气干扰。}
%
%\begin{figure}[h]
%\centering
%%%\includegraphics[width=0.8\textwidth]{fig4-37.png}
%\caption{A Loop antenna provides the resonant circuit.}
%\end{figure}
%
%This regen can be used to add regeneration to different circuits. The circuit can be connected to the low tap for loose coupling, %or to the hot end of the circuit for greater signal voltage. When connecting a long outdoor antenna, tap points low down in the %winding are used for both the regeneration and the antenna connections. However, a small loop antenna can be connected directly to %the hot end. Both variants were tested with the two circuit halves.
%
%\textbf{中文:这个再生电路可以用来向不同电路添加再生。电路可以连接到低抽头以进行松散耦合,或连接到电路的热端以获得更大的信号电压。当连接长室外天%线时,绕组下方的低抽头点用于再生和天线连接。然而,小型环形天线可以直接连接到热端。两种变体都用两个电路半部分进行了测试。}
%
%The loop consists of a total of two meters of Litz wire, simply hung over a stool. This provides good reception indoors because %the loop is sensitive to the magnetic field component of the electromagnetic signal. Adding signal regeneration to the loop %achieves the same effect as using a much larger loop. A croc clip was used for simple hook up to test circuits.
%
%\textbf{中文:环形天线总共由两米Litz导线组成,简单地挂在凳子上。这在室内提供良好的接收,因为环形天线对电磁信号的磁场分量敏感。向环形天线添加信号%再生实现了与使用大得多的环形天线相同的效果。使用鳄鱼夹进行简单的连接以测试电路。}
%
%\begin{figure}[h]
%\centering
%%%\includegraphics[width=0.8\textwidth]{fig4-38.png}
%\caption{Test setup with a loop antenna and loudspeaker.}
%\end{figure}
%
%The variable capacitor is a 4-gang version and two of the sections have not been used yet. I wanted to test the stability of an %iron powder toroidal core, so I wound some wire around a T50-2 core and added a few tap points to the coil. I also included two %ceramic capacitors, each with a capacitance of 56 pF, in parallel with one 20 pF section of the variable capacitor. This 4 gang %variable capacitor has 2 × 266 pF tunable sections for use with AM bands and 2 × 20 pF tunable sections for the FM band. On my %first attempt, I was able to pick up many radio stations very clearly on the 49-meter band. The total capacitance value ranges %between 112 and 132 pF. This gives a capacitance ratio of 1 to 1.18. The frequency ratio is the square root of this, which is 1.%085. This gives a tuning range of 500 kHz in the 49-meter band.
%
%\textbf{中文:可变电容器是一个4联版本,其中两个部分尚未使用。我想测试铁粉环形磁芯的稳定性,所以我在T50-2磁芯上缠绕了一些导线,并向线圈添加了几个%抽头点。我还包括两个陶瓷电容器,每个电容为56 pF,与可变电容器的一个20 pF部分并联。这个4联可变电容器具有2 × 266 pF可调部分用于AM波段和2 × 20 pF%可调部分用于FM波段。在第一次尝试中,我能够在49米波段非常清晰地接收到许多电台。总电容值在112到132 pF之间。这给出1到1.18的电容比。频率比是这个的%平方根,即1.085。这在49米波段给出了500 kHz的调谐范围。}
%
%\begin{figure}[h]
%\centering
%%%\includegraphics[width=0.8\textwidth]{fig4-39.png}
%\caption{Toroidal-core coil for 40 m.}
%\end{figure}
%
%To be able to work on the 40-meter band, I carefully unwound some turns. This brought me close to the beginning of the band at 7 %MHz. The receiver now covers the 40-meter band and part of the 41-meter band. The Audion needs to be connected to the tap at about %half the number of turns. Thanks to the band spread, SSB and CW stations can be easily tuned with precision. The frequency %stability is quite good, although there is still some sensitivity to hand movements due to the open construction of the circuit. %It should be possible to build a useful Audion using a toroidal core for amateur radio use inside a shielded enclosure.
%
%\textbf{中文:为了能够在40米波段上工作,我仔细地拆下了一些圈。这使我接近7 MHz波段的开始。接收机现在覆盖40米波段和41米波段的一部分。Audion需要连%接到大约圈数一半的抽头。由于波段扩展,SSB和CW电台可以精确调谐。频率稳定性相当好,尽管由于电路的开放式构造,对手部移动仍然有一些敏感性。应该可以在%屏蔽外壳内使用环形磁芯为业余无线电使用构建有用的Audion。}
%
%\begin{figure}[h]
%\centering
%%%\includegraphics[width=0.8\textwidth]{fig4-40.png}
%\caption{Test setup using a JFET Audion.}
%\end{figure}
%
%At the end of the series of experiments, I wanted to try using a JFET as well. Previous attempts had not been very successful, and %I attributed this to the fact that a JFET's characteristic curve exhibits greater linearity compared to a bipolar transitor. %Direct comparisons have indeed shown that a JFET circuit delivers a lower audio signal. For this reason, an additional audio stage %was added.
%
%\textbf{中文:在一系列实验结束时,我也想尝试使用JFET。以前的尝试不是很成功,我将此归因于JFET的特性曲线与双极晶体管相比表现出更大的线性。直接比较%确实表明,JFET电路提供较低的音频信号。因此,添加了一个额外的音频级。}
%
%\begin{figure}[h]
%\centering
%%%\includegraphics[width=0.8\textwidth]{fig4-41.png}
%\caption{The JFET Audion.}
%\end{figure}
%
%The JFET is operated with very little source current and only very small capacitance at the input. Nevertheless, the amlication is %sufficient for active regeneration.
%
%\textbf{中文:JFET以非常小的源电流和输入处仅非常小的电容操作。尽管如此,放大对于有源再生是足够的。}
%
%\begin{figure}[h]
%\centering
%%%\includegraphics[width=0.8\textwidth]{fig4-42.png}
%\caption{Construction of the active stage.}
%\end{figure}
%
%The new Audion circuit was now built on a small section of perf board. The circuit is connected to the desired resonance circuit %and volume potentiometer with two crocodile clips. This way makes it easier to switch back to the old bipolar circuit.
%
%\textbf{中文:新的Audion电路现在构建在穿孔板的小段上。电路通过两个鳄鱼夹连接到所需的谐振电路和音量电位器。这种方式使得更容易切换回旧的双极电%路。}
%
%The comparison clearly shows that the FET Audion is the better circuit. The frequency stability is significantly higher, and the %almost complete lack of frequency dependence on regeneration settings is convincing. For CW and SSB reception, regeneration can be %more strongly over driven without losing AF volume. This is where the more linear characteristic curve of the FET is advantageous. %Overall, the circuit has a very large dynamic range and can process weak CW stations as well as strong broadcasting stations.
%
%\textbf{中文:比较清楚地表明,FET Audion是更好的电路。频率稳定性明显更高,并且几乎完全缺乏对再生设置的频率依赖性是令人信服的。对于CW和SSB接收,%再生可以更强地过驱动而不会失去AF音量。这就是FET更线性特性曲线的优势所在。总的来说,该电路具有非常大的动态范围,可以处理弱CW电台以及强广播电台。}
%
%All in all, there's nothing like the good old Audion. Compared to a direct mixer, an Audion is much simpler to build because it %only needs a single resonant circuit. Additionally, essential amlication is achieved through the sharply tuned regeneration. The %circuit has a quality factor of up to 1000, which simultaneously provides good selection and reduces the risk of intermodulation %distortion or strong broadcast stations breaking through. On top of this, the Audion can also receive AM clearly, which is hardly %possible with a direct mixer.
%
%\textbf{中文:总的来说,没有什么比好的旧Audion更好。与直接混频器相比,Audion构建起来简单得多,因为它只需要单个谐振电路。此外,基本放大是通过尖%锐调谐的再生实现的。该电路具有高达1000的品质因数,这同时提供良好的选择性并减少互调失真或强广播电台突破的风险。最重要的是,Audion还可以清晰地接收%AM,这对于直接混频器来说几乎是不可能的。}
%
%\chapter{RF Oscillators}
%\textbf{中文:RF振荡器}
%
%The quality of a receiver or a transmitter depends heavily on the type of oscillator circuit used. While free-running oscillators %with tuning capacitors require a lot of effort to achieve good stability, a quartz crystal oscillator can provide ideal properties %but at a fixed frequency. Good stability of a quartz oscillator and tuning capability can be achieved using a PLL circuit. The %current state-of-the-art oscillator technology is represented by DDS generators, which provide accuracy of a quartz oscillator %together with an adjustable frequency output signal with very fine frequency resolution.
%
%\textbf{中文:接收机或发射机的质量在很大程度上取决于所使用的振荡器电路类型。虽然带有调谐电容器的自由运行振荡器需要大量努力才能实现良好的稳定性,%但石英晶体振荡器可以提供理想的特性,但频率固定。可以使用PLL电路实现石英振荡器的良好稳定性和调谐能力。当前最先进的振荡器技术由DDS发生器代表,它们%提供石英振荡器的精度以及具有非常精细频率分辨率的可调频率输出信号。}
%
%\section{LC Oscillators}
%\textbf{中文:LC振荡器}
%
%Common basic circuits for RF oscillators have already been described in connection with Audion, as every Audion with regeneration %works like an oscillator. Many results with Audion also showed some potential weaknesses of the oscillator. Frequency instability %can be a result of external influences such as varying capacitive coupling from hand movements close to the receiver, voltage %fluctuations or interference signals. In the interest of high stability, all of these influences must be kept to a minimum. The %following factors are crucial:
%
%- A high \textit{Q} resonant circuit
%- Low temperature coefficients of the coil and capacitors
%- Loose coupling to the oscillator circuit
%- Oscillator shielding
%- Decoupling and stabilization of the operating voltage
%- Load decoupling using buffer stages
%
%\textbf{中文:射频振荡器的常见基本电路已经结合Audion进行了描述,因为每个带有再生的Audion都像振荡器一样工作。Audion的许多结果也显示了振荡器的一%些潜在弱点。频率不稳定性可能是外部影响的结果,例如由于手部靠近接收机而变化的电容性耦合、电压波动或干扰信号。为了高稳定性,必须将所有这些影响保持%在最低限度。以下因素至关重要:}
%
%- 高Q谐振电路
%- 线圈和电容器的低温度系数
%- 与振荡器电路的松散耦合
%- 振荡器屏蔽
%- 操作电压的去耦和稳定
%- 使用缓冲级的负载去耦
%
%\begin{figure}[htbp]
%\centering
%%%\includegraphics[width=0.6\textwidth]{fig5-1}
%\caption{VFO with Capacitive Coupling}
%\end{figure}
%
%Figure 5.1 shows the basic circuit of a stable VFO (Variable Frequency Oscillator) with oscillator stage and buffer amplifier. The %first transistor is loosely coupled to the resonant circuit through a capacitive voltage divider. The optimal values for %capacitors and resistors depend on the desired frequency range and should be determined by experimentation. A subsequent buffer %stage should minimize feedback signal on the resonant circuit.
%
%\textbf{中文:图5.1显示了稳定VFO(可变频率振荡器)的基本电路,带有振荡器级和缓冲放大器。第一个晶体管通过电容性电压分压器松散耦合到谐振电路。电容%器和电阻器的最佳值取决于所需的频率范围,应通过实验确定。后续缓冲级应最小化谐振电路上的反馈信号。}
%
%The long-term stability of a circuit also depends on the temperature behavior of the components that determine the frequency. %Styroflex (polystyrene) capacitors are well-suited for this purpose because their temperature coefficient (TC) is close to zero. %However, even standard ferrite cores have a certain temperature coefficient. It takes a lot of careful consideration to compensate %for temperature induced drift by using ceramic capacitors with opposing TCs.
%
%\textbf{中文:电路的长期稳定性也取决于决定频率的组件的温度行为。Styroflex(聚苯乙烯)电容器非常适合此目的,因为它们的温度系数(TC)接近于零。然%而,即使是标准铁氧体磁芯也有一定的温度系数。需要大量仔细考虑,通过使用具有相反TC的陶瓷电容器来补偿温度引起的漂移。}
%
%\begin{figure}[htbp]
%\centering
%%%\includegraphics[width=0.6\textwidth]{fig5-2}
%\caption{Clapp Oscillator for the 40-m Band}
%\end{figure}
%
%Figure 5.2 shows the popular Clapp oscillator circuit often used in amateur radio equipment with component values suitable for use %in the 40-meter amateur radio band and the 41-meter broadcasting band. In this form, the oscillator can be used for direct %conversion receivers to listen to CW and SSB signals, as well as for digital radio broadcasting (DRM). Mechanical stability and %good shielding are important factors to consider.
%
%\textbf{中文:图5.2显示了流行的Clapp振荡器电路,经常在业余无线电设备中使用,具有适合用于40米业余无线电波段和41米广播波段的组件值。在这种形式%中,振荡器可以用于直接转换接收机以收听CW和SSB信号,以及用于数字无线电广播(DRM)。机械稳定性和良好屏蔽是考虑的重要因素。}
%
%The Clapp oscillator configuration uses capacitive coupling and has a wide operating frequency range which makes it popular with %amateur radio enthusiasts. If you need to cover even larger frequency ranges, you can use a tap point coupling to the resonant %circuit coil as shown in Figure 5.3.
%
%\textbf{中文:Clapp振荡器配置使用电容性耦合,具有宽的操作频率范围,这使其在业余无线电爱好者中受欢迎。如果您需要覆盖甚至更大的频率范围,您可以使%用抽头点耦合到谐振电路线圈,如图5.3所示。}
%
%\begin{figure}[htbp]
%\centering
%%%\includegraphics[width=0.6\textwidth]{fig5-3}
%\caption{Stable VFO with Wide Operating Range}
%\end{figure}
%
%\section{Crystal Oscillators}
%\textbf{中文:晶体振荡器}
%
%The electrical characteristics of quartz crystal are equivalent to a resonant circuit with an extremely high value of \textit{Q} %or quality. Therefore, quartz oscillators with good stability can be easily built without the need for any inductors. Figure 5.4 %shows the standard Pierce circuit used to build a crystal oscillator.
%
%\textbf{中文:石英晶体的电特性等效于具有极高Q值或品质的谐振电路。因此,可以轻松构建具有良好稳定性的石英振荡器,而无需任何电感器。图5.4显示了用于%构建晶体振荡器的标准Pierce电路。}
%
%\begin{figure}[htbp]
%\centering
%%%\includegraphics[width=0.6\textwidth]{fig5-4}
%\caption{The Pierce Oscillator}
%\end{figure}
%
%A transistor can be used in an oscillator collector circuit using a capacitor voltage divider network. This circuit is used, for %example, in the integrated mixer and oscillator IC type NE612. A capacitor trimmer allows for fine tuning and can adjust or 'pull' %the quartz frequency by up to about 3 kHz.
%
%\textbf{中文:晶体管可以在振荡器集电极电路中使用电容性电压分压器网络。例如,该电路用于集成混频器和振荡器IC类型NE612。电容器微调器允许精细调谐,%并且可以调整和"拉动"石英频率高达10 kHz。}
%
%\begin{figure}[htbp]
%\centering
%%%\includegraphics[width=0.6\textwidth]{fig5-5}
%\caption{Quartz Crystal Oscillator Using Common Collector Amplifier}
%\end{figure}
%
%Another popular oscillator configuration uses a CMOS gate instead of a transistor. The circuit shown in Figure 5.6 is used in %integrated oscillators in microcontrollers, as well as in CMOS circuits such as the oscillator and divider IC 4060 or the faster %HC4060.
%
%\textbf{中文:另一种流行的振荡器配置使用CMOS门而不是晶体管。图5.6中显示的电路用于微控制器中的集成振荡器,以及CMOS电路,例如振荡器和分频器IC %4060或更快的HC4060。}
%
%\begin{figure}[htbp]
%\centering
%%%\includegraphics[width=0.6\textwidth]{fig5-6}
%\caption{A Crystal Oscillator Using a CMOS Gate}
%\end{figure}
%
%For frequencies below 1 MHz, a ceramic resonator is often used instead of a quartz crystal. These components are less expensive %than quartz crystals, but they don't achieve the same temperature stability and accuracy.
%
%\textbf{中文:对于低于30 MHz的频率,通常使用陶瓷谐振器而不是石英晶体。这些组件比石英晶体便宜,但它们不能达到相同的温度稳定性和精度。}
%
%\section{Amplitude Modulation}
%\textbf{中文:幅度调制}
%
%The small medium wave AM transmitter described here can be used to broadcast programs to a nearby medium wave radio. The %transmitting coil wound on a ferrite core sends signals which couple directly to the ferrite rod in a radio receiver. The %transmitter operating frequency is derived from a 976 kHz ceramic resonator, which can be salvaged, for example, from an old TV %remote control unit. Some degree of fine frequency tuning is possible using the 30 pF trimmer. A likely weak station in the %background can be tuned to the beat frequency null, for example, at 981 kHz.
%
%\textbf{中文:这里描述的小型中波AM发射机可以用来向附近的中波收音机广播节目。缠绕在铁氧体磁芯上的发射线圈发送信号,这些信号直接耦合到收音机接收机%中的铁氧体棒。发射机操作频率源自976 kHz陶瓷谐振器,该谐振器可以回收,例如,从旧的电视遥控器单元。使用30 pF微调器可以实现一定程度的精细频率调谐。%背景中可能的弱电台可以调谐到拍频零点,例如,在981 kHz处。}
%
%\begin{figure}[htbp]
%\centering
%%%\includegraphics[width=0.6\textwidth]{fig5-7}
%\caption{The AM Modulator}
%\end{figure}
%
%The modulator works as an emitter follower and modulates the power amplifier supply voltage. You can only send mono signals on %medium wave so both left and right input channels are combined. The potentiometer is used to adjust for the lowest distortion and %best sound quality. The RF amplifier stage is designed to operate at low power to limit the signal range.
%
%\textbf{中文:调制器作为发射极跟随器工作,并调制功率放大器电源电压。您只能在中波上发送单声道信号,因此左右输入通道被组合。电位器用于调整以获得最%低失真和最佳音质。射频放大器级设计为在低功率下操作以限制信号范围。}
%
%A waveform showing the amplitude modulated RF signal output can be seen in Figure 5.8.
%
%\textbf{中文:显示幅度调制射频信号输出的波形可以在图5.8中看到。}
%
%\begin{figure}[htbp]
%\centering
%%%\includegraphics[width=0.6\textwidth]{fig5-8}
%\caption{Waveform of the RF Output Signal}
%\end{figure}
%
%This medium wave RF modulator can now be placed on top of a regular MW radio receiver. An audio cable hookup to a CD player or any %other audio source will now give you one more strong station transmitting on the medium wave band. Not only does this one have %particularly good modulation purity but it's also guaranteed to play great tunes!
%
%\textbf{中文:这个中波射频调制器现在可以放置在常规MW收音机接收机的顶部。连接到CD播放器或任何其他音频源的音频电缆现在将为您提供一个在中波波段上发%射的更强电台。这个不仅具有特别良好的调制纯度,而且保证播放美妙的音乐!}
%
%\section{Crystal-stabilized Medium Wave Modulator}
%\textbf{中文:晶体稳定的中波调制器}
%
%A medium wave AM modulator should be as stable as possible and, above all, not have any FM components in its output signal. A good %solution would be to use a crystal oscillator. Unfortunately, quartz crystals that operate at medium wave frequencies are rare and %expensive. To get over this you can use a divide-by-ten circuit. This allows any quartz in the range of 5 MHz to 16 MHz to be used %to transmit on medium wave frequencies between 500 kHz and 1.6 MHz.
%
%\textbf{中文:中波AM调制器应该尽可能稳定,最重要的是,在其输出信号中没有任何FM分量。一个好的解决方案是使用晶体振荡器。不幸的是,在中波频率下操作%的石英晶体稀有且昂贵。为了克服这一点,您可以使用十分频电路。这允许5 MHz到16 MHz范围内的任何石英用于500 kHz到1.6 MHz之间的中波频率上传输。}
%
%\begin{figure}[htbp]
%\centering
%%%\includegraphics[width=0.6\textwidth]{fig5-9}
%\caption{Using a Microcontroller}
%\end{figure}
%
%Here an ATtiny25 microcontroller is used as a frequency divider. Its quartz oscillator can run up to 20 MHz. A very short program %controls the microcontroller's DDRB.1 register. At B1, there is not a usual CMOS output driver, but only the lower port FET with %an open drain output stage. The controller thus forms the oscillator, frequency divider, driver, and transmission power amplifier. %The power amplifier could easily deliver up to 30 mA, but it is operated here with greatly reduced power to comply with legal %requirements for inductive transmission.
%
%\textbf{中文:这里使用ATtiny25微控制器作为频率分频器。其石英振荡器可以运行高30 MHz。一个非常短的程序控制微控制器的DDRB.1寄存器。在B1处,没有通%常的CMOS输出驱动器,只有具有开漏输出级的下部端口FET。控制器因此形成振荡器、频率分频器、驱动器和传输功率放大器。功率放大器可以轻松提供高达30 mA,%但这里以大大降低的功率操作,以符合电感传输的法律要求。}
%
%\begin{figure}[htbp]
%\centering
%%%\includegraphics[width=0.6\textwidth]{fig5-10}
%\caption{Test Build Using a Ferrite-rod Antenna}
%\end{figure}
%
%The modulation input can be connected to any headphone output of a CD player, MP3 player, DAB radio, etc. The modulation level can %be adjusted using the volume control. The modulation is absolutely pure and distortion-free and can be turned up to a modulation %level of 100\% without any problems. The sound quality is comparable to that of a real medium wave transmitter. Instead of the %ferrite choke, a wire loop can also be used, which can then be placed near the target radio receiver. Now you can try different %quartz crystals from your hobby box to find one that operates on an unoccupied frequency.
%
%\textbf{中文:调制输入可以连接到CD播放器、MP3播放器、DAB收音机等的任何耳机输出。调制电平可以使用音量控制进行调整。调制绝对纯净且无失真,并且可以%调高到100\%的调制电平而没有任何问题。音质与真正的中波发射机相当。可以使用导线环路代替铁氧体扼流圈,然后可以将其放置在目标收音机接收机附近。现在您%可以尝试从您的爱好箱中尝试不同的石英晶体,找到一个在未占用频率上操作的石英。}
%
%The AM modulator is available as a pre-assembled board from AK Modul-Bus and operates at 900 kHz. To operate on any frequency in %the medium wave band a crystal socket can be fitted allowing other crystals to be easily swapped out. Long wave operation has also %been successfully tested.
%
%\textbf{中文:AM调制器作为预组装板从AK Modul-Bus提供,并在900 kHz下操作。为了在中波波段的任何频率上操作,可以安装晶体插座,允许轻松更换其他晶%体。长波操作也已成功测试。}
%
%\begin{figure}[htbp]
%\centering
%%%\includegraphics[width=0.6\textwidth]{fig5-11}
%\caption{The AM Modulator}
%\end{figure}
%
%A wire loop antenna from the modulator placed directly behind a radio receiver works well. This new station will now be a good %substitute for any shut-down local transmitter. In evening, you will be able to tune into your own station broadcasting your own %program along with many weaker signals from far away.
%
%\textbf{中文:来自调制器的导线环路天线直接放置在收音机接收机后面效果很好。这个新电台现在将是任何关闭的本地发射机的好替代品。在晚上,您将能够调谐%到您自己的电台广播您自己的节目,以及来自远方的许多较弱信号。}
%
%\section{The ICS307-2 PLL Clock Generator}
%\textbf{中文:ICS307-2 PLL时钟发生器}
%
%While searching for a possible alternative to the no longer available CY27EE16 PLL chip, the ICS307-2 was discovered. This clock %generator is somewhat simpler and offers fewer options. This IC is quite compact and comes packaged in a 16-pin SOIC outline with %a pin spacing of 1.27 mm.
%
%\textbf{中文:在寻找不再可用的CY27EE16 PLL芯片的可能替代品时,发现了ICS307-2。这个时钟发生器更简单一些,提供的选项更少。这个IC非常紧凑,采用16%引脚SOIC封装,引脚间距为1.27 mm。}
%
%\begin{figure}[htbp]
%\centering
%%%\includegraphics[width=0.6\textwidth]{fig5-12}
%\caption{Testing the ICS207-2}
%\end{figure}
%
%The ICS307-2 uses an SPI interface to connect with a PC. This requires three 10 kohm resistors plus a voltage supply of in the %range of 3.3 to 5 V.
%
%\textbf{中文:ICS307-2使用SPI接口与PC连接。这需要三个10 kΩ电阻加上3.3到5 V范围内的电压供应。}
%
%\begin{figure}[htbp]
%\centering
%%%\includegraphics[width=0.6\textwidth]{fig5-13}
%\caption{The Clock Generator Schematic}
%\end{figure}
%
%A small VB program is available so that you can control the chip using a PC via the serial interface. The output Clk2 can generate %frequencies between 2 MHz and 120 MHz, allowing an IQ mixer (see Section 8) to be operated between 500 kHz and 30 MHz. Other %values displayed were used for investigation into frequency deviations and settings during program development.
%
%\textbf{中文:有一个小的VB程序可用,以便您可以通过串行接口使用PC控制芯片。输出Clk2可以生成2 MHz到120 MHz之间的频率,允许IQ混频器(见第8节)在%500 kHz到30 MHz之间操作。显示的其他值用于程序开发期间的频率偏差和设置调查。}
%
%The IC has three internal dividers with relatively little scope, resulting in lower frequency resolution compared to the CY27EE16. %Only three bytes are transferred to program chip.
%
%\textbf{中文:该IC具有三个内部分频器,范围相对较小,导致与CY27EE16相比频率分辨率较低。仅传输三个字节来编程芯片。}
%
%In order to program the correct values into the internal count registers to achieve the desired output frequency takes a little %trial and error, by calculating all allowed settings to find the best match. This results in either the exact desired frequency or %a neighboring frequency within 1 kHz or, at a few critical points, within 5 kHz to be achieved.
%
%\textbf{中文:为了将正确的值编程到内部计数寄存器中以实现所需的输出频率,需要一些试错,通过计算所有允许的设置以找到最佳匹配。这导致要么实现精确的%所需频率,要么在1 kHz内的相邻频率,或者在几个关键点内,在5 kHz内实现。}
%
%\begin{figure}[htbp]
%\centering
%%%\includegraphics[width=0.6\textwidth]{fig5-14}
%\caption{Control Software in VB}
%\end{figure}
%
%AK Modul-Bus offers a preassembled board for the ICS307-2, which also includes the required software. All connections to the PCB %are via screw terminal blocks. The board can be used as a general-purpose clock oscillator for microcontroller applications and %digital electronics, as well as for high-frequency applications. The clock generator can also be used as an inexpensive VFO for %shortwave receivers.
%
%\textbf{中文:AK Modul-Bus为ICS307-2提供预组装板,还包括所需的软件。PCB的所有连接都通过螺钉端子块进行。该板可以用作微控制器应用和数字电子设备%的通用时钟振荡器,以及用于高频应用。时钟发生器还可以用作短波接收机的廉价VFO。}
%
%\begin{figure}[htbp]
%\centering
%%%\includegraphics[width=0.6\textwidth]{fig5-15}
%\caption{The Populated PCB}
%\end{figure}
%
%\section{A Programable Crystal Oscillator}
%\textbf{中文:可编程晶体振荡器}
%
%If you ever need a quartz crystal for a very specific frequency, it's often not available. Some frequencies can't be purchased at %all. Custom-made crystals are always an option, but they are very expensive as a one-off. What is required in this case is a %programmable quartz oscillator. The CY27EE16 chip is ideal for this application and was used, for example, in the Elektor-SDR %(Section 8.6). Unfortunately, this chip is no longer manufactured. Modul-Bus produce a programmable clock oscillator board using %the ICS307-2 chip which outputs a frequency in the range from 2 to 120 MHz. This chip, however, lacks the ability to store a %selected frequency. All you need to remedy this shortfall is to add a tiny low-cost microcontroller like the ATiny13 which can be %used to store a setting and use it on the next power-up. That will now give you a usable and programmable quartz generator.
%
%\textbf{中文:如果您需要特定频率的石英晶体,通常不可用。某些频率根本无法购买。定制晶体总是一个选项,但作为一次性产品非常昂贵。在这种情况下需要的%是可编程石英振荡器。CY27EE16芯片非常适合此应用,并已用于Elektor-SDR(第8.6节)。不幸的是,该芯片不再生产。Modul-Bus生产使用ICS307-2芯片的可编%程时钟振荡器板,输出频率范围为2到120 MHz。然而,该芯片缺乏存储所选频率的能力。为了弥补这一不足,您需要添加一个微小的低成本微控制器,如ATiny13,%它可以用于存储设置并在下次启动时使用。这将为您提供可用且可编程的石英发生器。}
%
%\begin{figure}[htbp]
%\centering
%%%\includegraphics[width=0.6\textwidth]{fig5-16}
%\caption{Controlled by an ATiny13}
%\end{figure}
%
%The ATiny13 here directly generates the three interface signals Data, SCLK, and Strobe of the clock generator PCB. In our initial %test, the DB9 plug was used, i.e., the port lines directly replace the corresponding output lines of the RS232. Now, the ATiny13 %must send exactly the data that was previously supplied by the PC. Altogether, the frequency required is contained in three %control bytes, giving 24 bits in total. These bytes are either directly received and clocked into the generator or loaded into the %EEPROM of the ATiny13, where they are read out at the next start.
%
%\textbf{中文:这里的ATiny13直接生成时钟发生器PCB的三个接口信号Data、SCLK和Strobe。在我们的初始测试中,使用了DB9插头,即端口线直接替换RS232的%相应输出线。现在,ATiny13必须发送以前由PC提供的确切数据。总的来说,所需的频率包含在三个控制字节中,总共给出24位。这些字节要么直接接收并时钟到发%生器中,要么加载到ATiny13的EEPROM中,在下次启动时读出。}
%
%Data transfer to the clock oscillator PCB is performed by the handy Bascom command Shiftout to transfer the three relevant bytes. %After that, a strobe pulse is generated with Pulseout. A software UART running at 9600 baud takes care of data reception. A %command word is required to be prefixed to the three data bytes B1, B2, and B3. The command word tells the processor what to do %with the 3 data bytes. The value 65 indicates the data is for direct output while 66 indicates it's for storage in the EEPROM. %These three bytes stored at addresses 10, 11, and 12 in the EEPROM are read at each restart and clocked into oscillator chip.
%
%\textbf{中文:数据传输到时钟振荡器PCB是通过方便的Bascom命令Shiftout传输三个相关字节来执行的。之后,使用Pulseout生成选通脉冲。以9600波特运行的%软件UART负责数据接收。需要将命令字前缀到三个数据字节B1、B2和B3。命令字告诉处理器如何处理3个数据字节。65表示数据用于直接输出,66表示用于存储在%EEPROM中。这些存储在EEPROM中地址10、11、12的三个字节在每次重启时读出并时钟到振荡器芯片中。}
%
%\begin{figure}[htbp]
%\centering
%%%\includegraphics[width=0.6\textwidth]{fig5-17}
%\caption{Setting Up the Desired Frequency}
%\end{figure}
%
%The original program for setting the output frequency now needs to be modified to send four bytes over the RS232 interface. The %example shown here sets a frequency of 27.12 MHz, for which the three control bytes 60, 165, and 176 were determined. The %frequency can be set using the slider control or by entering the desired frequency into the field and transferring it to the %slider via the Set button. The VB software and ATiny13 firmware are available on the author's website.
%
%\textbf{中文:用于设置输出频率的原始程序现在需要修改以通过RS232接口发送四个字节。这里显示的示例设置27.12 MHz的频率,为此确定了三个控制字60, %165, 176。可以使用滑块控制设置频率,或者通过将所需频率输入字段并通过Set按钮将其传输到滑块来设置。VB软件和ATiny13固件可在作者的网站上获得。}
%
%\begin{figure}[htbp]
%\centering
%%%\includegraphics[width=0.6\textwidth]{fig5-18}
%\caption{The ATiny13 and Crystal}
%\end{figure}
%
%In order to use the programmable quartz oscillator in the same way as regular quartz oscillators with four pins in a DIP14 %package, it was placed on a small piece of double-sided perf board (a.k.a. stripboard or Veroboard). The standard pin assignment %was adopted, but now the RXD input is connected to Pin 1. Only this one pin is needed to reprogram generator. The microcontroller %is located on the top side, and the PLL chip is on the bottom side of the board.
%
%\textbf{中文:为了像带有四个引脚的DIP14封装的常规石英振荡器一样使用可编程石英振荡器,它被放置在一小块双面穿孔板(也称为条形板或Veroboard)上。%采用了标准引脚分配,但现在RXD输入连接到引脚1。只需要这一个引脚来重新编程发生器。微控制器位于板的顶部,PLL芯片位于板的底部。}
%
%\section{CW Transmitter with an EL95}
%\textbf{中文:使用EL95的CW发射机}
%
%You may wonder what sort of rigs amateur radio enthusiasts were using back in the 1950s when vacuum tubes were the order of the %day. Well, many radio amateur rigs at that time were made up of a tube Audion type regen receiver (0V2) and a small transmitter, %with an EL84 tube in the final output stage. I built something similar to that myself. By the time I sat my license exam, I had %already skipped a few steps beyond the basic receivers and transmitters. As a result, I never actually got to use a 0V2 and a %small tube transmitter. To make up for the gap in my ham radio apprenticeship I thought it was about time I took a look at the %design, just to see how well it performs.
%
%\textbf{中文:您可能想知道业余无线电爱好者在1950年代使用什么设备,当时真空管是主流。嗯,当时许多业余无线电设备由电子管Audion型再生接收机(0V2)%和小型发射机组成,最终输出级使用EL84电子管。我自己构建了类似的东西。当我参加执照考试时,我已经跳过了基本接收机和发射机的几个步骤。因此,我从未真%正使用过0V2和小型电子管发射机。为了弥补我业余无线电学徒期的空白,我认为是时候看看这个设计了,看看它表现如何。}
%
%\begin{figure}[htbp]
%\centering
%%%\includegraphics[width=0.6\textwidth]{fig5-19}
%\caption{The Homebrew Tube Transmitter}
%\end{figure}
%
%At first, I thought it would be easy to modify an old tube radio for that purpose. A small Grundig radio was already on my radar, %but the set still works so beautifully I couldn't bring myself to cannibalize it. I then decided to build the necessary power %supply myself. I already had two suitable transformers. By connecting the two secondary windings together the first transformer %gets the mains voltage down to 12 V and the second produces an output of about 200 V at its primary winding to use for the HT %supply. Using this method, you don't need to find a special transformer and everything can be installed in the Franzis shortwave %radio enclosure.
%
%\textbf{中文:起初,我认为为此目的修改旧电子管收音机会很容易。我已经注意到一台小型Grundig收音机,但这台收音机仍然工作得如此美妙,我无法将其拆%解。然后我决定自己构建必要的电源。我已经有两个合适的变压器。通过将两个次级绕组连接在一起,第一个变压器将市电电压降到12 V,第二个变压器在其初级绕%组产生约200 V的输出,用于HT供应。使用这种方法,您不需要找到特殊的变压器,所有东西都可以安装在Franzis短波收音机外壳中。}
%
%\begin{figure}[htbp]
%\centering
%%%\includegraphics[width=0.6\textwidth]{fig5-20}
%\caption{The HT Voltage Supply}
%\end{figure}
%
%A prototyping test board was used to build the tube stage. The board is designed for seven-pin miniature sockets which can accept %a 6AQ5A (= EL90). Despite its small size this tube can handle an anode power dissipation of 12 W, just like the much larger EL84. %With this tube, I was able to generate up to 5 W at 3.5 MHz. The EL95 is pin-compatible and delivers 3.5 W using the same circuit %configuration.
%
%\textbf{中文:使用原型测试板构建电子管级。该板设计用于七针微型插座,可以接受6AQ5A(= EL90)。尽管其尺寸小,该电子管可以处理12 W的阳极功耗,就像%更大的EL84一样。使用这个电子管,我能够在3.5 MHz下产生5 W的功率。EL95引脚兼容,并使用相同的电路配置提供3.5 W。}
%
%\begin{figure}[htbp]
%\centering
%%%\includegraphics[width=0.6\textwidth]{fig5-21}
%\caption{The Output Stage}
%\end{figure}
%
%The complete crystal-stabilized transmitter circuit shown above uses cathode keying. The carrier signal which I set to 3560 kHz is %generated by the programmable crystal generator described in the last section. When 5 V is applied to VCC, the RF signal is %generated at the output. The tuned grid circuit boosts the signal to about 20 Vpp.
%
%\textbf{中文:上面显示的完整晶体稳定发射机电路使用阴极键控。我设置3560 kHz的载波信号由上一节描述的可编程晶体发生器生成。当5 V施加到VCC时,在输%出处生成RF信号。调谐栅极电路将信号提升到约20 Vpp。}
%
%A small neon lamp indicates the anode HT voltage. An LED lights when the toggle switch is set to transmit. The relay then pulls in %and applies 5 V to the quartz oscillator. During transmission, a DC voltage is also applied to the antenna output which activates %an external relay to switch the transmitting antenna. A DC voltage is also applied to the antenna cable to the receiver, which is %used for signal muting.
%
%\textbf{中文:一个小氖灯指示阳极HT电压。当切换开关设置为发射时,LED亮起。继电器然后吸合并向石英振荡器施加5 V。在发射期间,DC电压也施加到天线输%出,激活外部继电器以切换发射天线。DC电压也施加到接收器的天线电缆,用于信号静音。}
%
%\begin{figure}[htbp]
%\centering
%%%\includegraphics[width=0.6\textwidth]{fig5-22}
%\caption{Schematic of the CW Transmitter}
%\end{figure}
%
%The coil in the anode circuit is made up of 20 turns on a toroidal ferrite core with a secondary winding of 3 turns to ensure %antenna matching and necessary isolation from the anode HT voltage. An additional Pi-filter at the output is for matching and %provides further harmonic suppression. This is useful when other frequency bands are used.
%
%\textbf{中文:阳极电路中的线圈由环形铁氧体磁芯上20匝组成,次级绕组3匝,以确保天线匹配和与阳极HT电压的必要隔离。输出处的额外Pi滤波器用于匹配并提%供进一步的谐波抑制。当使用其他频段时,这很有用。}
%
%\begin{figure}[htbp]
%\centering
%%%\includegraphics[width=0.6\textwidth]{fig5-23}
%\caption{The Transmitter in Operation}
%\end{figure}
%
%A small 5 V/0.4 A filament lamp (vintage bicycle headlamp) primarily serves to dissipate some of the heat generated inside the %case. The transformer outputs 12 V, which means the voltage regulator needs to dissipate a lot of power to provide 6.3 V at 450 mA %for the tube filament.
%
%\textbf{中文:一个小5 V/0.4 A灯丝灯(老式自行车头灯)主要用于散发外壳内产生的一些热量。变压器输出12 V,这意味着电压调节器需要消耗大量功率才能为%电子管灯丝提供6.3 V、450 mA的电流。}
%
%\section{AM Tube Transmitter}
%\textbf{中文:AM电子管发射机}
%
%This small medium wave transmitter uses a tunable oscillator with a ferrite antenna. Two EF95 type HF pentodes are used in this %design. One tube acts as a tunable oscillator, other as a modulation amplifier. To use a simple plug-in power supply, both tube %heaters were connected in series to 12 V and the anode HT voltage was also limited to 12 V.
%
%\textbf{中文:这个小型中波发射机使用带有铁氧体天线的可调振荡器。设计中使用了两个EF95型高频五极管。一个电子管用作可调振荡器,另一个用作调制放大%器。为了使用简单的插入式电源,两个电子管灯丝串联连接到12 V,阳极HT电压也限制为12 V。}
%
%\begin{figure}[htbp]
%\centering
%%%\includegraphics[width=0.6\textwidth]{fig5-24}
%\caption{A tunable medium wave transmitter.}
%\end{figure}
%
%This free-running oscillator is modulated via the screen grid. The preceding modulation amplifier operates in triode mode to %achieve a sufficiently large, distortion-free modulation despite the low anode voltage. The operating point is adjusted with a %trimmer to minimize distortion. The characteristic curves of both stages are oppositely curved due to the phase shift of the %modulation amplifier. With an optimal adjustment, the resulting distortion is largely cancelled out, allowing for a large %modulation range of up to about 50\%. There is a stereo jack at the input where, for example, a PC sound card output can be %connected as a modulation source. Both channels are combined to provide a mono signal to the transmitter.
%
%\textbf{中文:这个自由运行的振荡器通过帘栅极进行调制。前面的调制放大器工作在三极管模式,尽管阳极电压低,但仍能实现足够大的无失真调制。工作点通过%微调器调整以最小化失真。由于调制放大器的相移,两个级的特性曲线相反弯曲。通过最佳调整,产生的失真在很大程度上被抵消,允许高达约50\%的大调制范围。%输入处有一个立体声插孔,例如,可以连接PC声卡输出作为调制源。两个通道被组合以向发射机提供单声道信号。}
%
%The resonant circuit coil is wound on a ferrite rod which also serves as the transmitting antenna. A nearby radio should be able %to tune into the signal. A wire antenna could also be used as an alternative but this increases the risk of the transmitted signal %being picked up by others in the neighborhood.
%
%\textbf{中文:谐振电路线圈缠绕在铁氧体棒上,该棒也用作发射天线。附近的收音机应该能够调谐到信号。也可以使用导线天线作为替代方案,但这会增加发射信%号被附近其他人接收到的风险。}
%
%\begin{figure}[htbp]
%\centering
%%%\includegraphics[width=0.6\textwidth]{fig5-25}
%\caption{Construction of the medium wave AM transmitter.}
%\end{figure}
%
%Figure 5.25 shows the transmitter built using the tube experimentation system RT100 from AK Modul-Bus. This system contains all %the necessary connections, a variable capacitor, and the required potentiometer. All connections, resistors, and capacitors have %been connected up on the plug board area.
%
%\textbf{中文:图5.25显示了使用AK Modul-Bus的电子管实验系统RT100构建的发射机。该系统包含所有必要的连接、可变电容器和所需的电位器。所有连接、电%阻器和电容器都已在插件板区域连接起来。}
%
%\section{A DDS Generator using the AD9835}
%\textbf{中文:使用AD9835的DDS发生器}
%
%DDS oscillators meet the highest demands in terms of frequency stability and noise immunity. They are also much easier to build %than PLL VFOs or free-running oscillators, and unlike a PLL, they allow almost any frequency to be synthesized with %fractional-Hertz resolution.
%
%\textbf{中文:DDS振荡器满足频率稳定性和抗噪声方面的最高要求。它们也比PLL VFO或自由运行振荡器更容易构建,并且与PLL不同,它们允许几乎任何频率以分%数赫兹分辨率合成。}
%
%\begin{figure}[htbp]
%\centering
%%%\includegraphics[width=0.6\textwidth]{fig5-26}
%\caption{Block diagram of the AD9835.}
%\end{figure}
%
%The term DDS stands for 'Direct Digital Synthesis' and describes the digital generation of a repetitive waveform at a defined %frequency. To generate a sine wave the core of a DDS oscillator uses a table of sine waveform values stored in ROM. A DAC takes %the sine wave values and converts them into an output voltage. A phase accumulator serves as an address pointer to the sine table %and keeps track of the instantaneous phase of the waveform. The desired output frequency is represented by a digital value called %the frequency tuning word. This value determines the rate at which the phase accumulator increments to give the desired output %frequency.
%
%\textbf{中文:DDS一词代表"直接数字合成",描述了在定义频率下重复波形的数字生成。为了生成正弦波,DDS振荡器的核心使用存储在ROM中的正弦波形值表。%DAC获取正弦波值并将其转换为输出电压。相位累加器作为正弦表的地址指针,并跟踪波形的瞬时相位。所需的输出频率由称为频率调谐字的数字值表示。该值决定了%相位累加器递增的速率,以产生所需的输出频率。}
%
%The quality of the output signal depends on the resolution of the DAC and the size of the sine table. The AD9835 uses a 10-bit %converter and a sine table with 4096 support values. The phase accumulator has a width of 32 bits, with only the upper 12 bits %determining the address of the current output value.
%
%\textbf{中文:输出信号的质量取决于DAC的分辨率和正弦表的大小。AD9835使用10位转换器和带4096个支持值的正弦表。相位累加器的宽度为32位,只有12位决%定当前输出值的地址。}
%
%\begin{figure}[htbp]
%\centering
%%%\includegraphics[width=0.6\textwidth]{fig5-27}
%\caption{Programmable Oscillator with serial interface.}
%\end{figure}
%
%Figure 5.27 shows the complete circuit diagram of the DDS generator with power supply and interface. A 7805 type voltage regulator %provides 5 V for the AD9835 chip from the 9 V input supply to the board. The DDS requires a 50 MHz clock signal which is generated %by an integrated quartz oscillator IC3. The remaining circuitry is mainly limited to supply voltage bypass capacitors and two %resistors. R1 is located at pin FSADJUST and determines the output current IOUT at pin 14. This current produces a voltage drop %across resistor R2. Here you can find the sinusoidal output voltage, superimposed upon a DC offset voltage. A simple low-pass pi %filter attenuates frequency components above 22 MHz.
%
%\textbf{中文:图5.27显示了带有电源和接口的DDS发生器的完整电路图。7805型电压调节器从9 V输入电源向AD9835芯片提供5 V。DDS需要50 MHz时钟信号,该%信号由集成石英振荡器IC3生成。其余电路主要限于电源电压旁路电容器和两个电阻器。R1位于FSADJUST引脚,决定引脚14处的输出电流IOUT。该电流在电阻器R2上%产生电压降。在这里您可以找到正弦输出电压,叠加在DC偏移电压上。简单的低通pi滤波器衰减22 MHz以上的频率分量。}
%
%\begin{figure}[htbp]
%\centering
%%%\includegraphics[width=0.6\textwidth]{fig5-28}
%\caption{The fully populated DDS oscillator PCB.}
%\end{figure}
%
%The DDS chip is only available in the TSPOP outline with a pin spacing of 0.65 mm. This tiny outline is necessary to achieve %sufficiently short signal line lengths and good decoupling of the power supply. SMD assembly is not very easy and requires a %certain level of skill. To smooth over any possible hassle, a ready-assembled DDS PCB is available from the company AK Modul-Bus.
%
%\textbf{中文:DDS芯片仅以引脚间距0.65 mm的TSPOP封装提供。这种微小的封装对于实现足够短的信号线长度和良好的电源去耦是必要的。SMD组装不是很容易,%需要一定的技能水平。为了消除任何可能的麻烦,AK Modul-Bus公司提供了现成组装的DDS PCB。}
%
%\begin{figure}[htbp]
%\centering
%%%\includegraphics[width=0.6\textwidth]{fig5-29}
%\caption{Control software and COM1 to COM4 port selection.}
%\end{figure}
%
%Figure 5.29 shows a Visual Basic program from the manufacturer's website for controlling the DDS oscillator. It can be easily %adapted and expanded for your own application requirements. The user interface allows you to define the output frequency between 0 %and 24 MHz with a step size of 1 kHz and additional fine tuning in 100 Hz and 10 Hz steps. If required, an offset of 455 kHz can %be chosen so that the frequency display shows the receiving frequency in a superhet with a 455 kHz IF. In addition, there are %frequency sweep functions which are particularly useful for making RF measurements and displaying filter characteristics.
%
%\textbf{中文:图5.29显示了制造商网站上用于控制DDS振荡器的Visual Basic程序。它可以轻松适应和扩展以满足您自己的应用需求。用户界面允许您定义0到%24 MHz之间的输出频率,步长1 kHz,并100 Hz和10 Hz步长中进行额外的微调。如果需要,可以选择455 kHz的偏移,以便频率显示在具有455 kHz IF的超外差中%显示接收频率。此外,还有频率扫描功能,对于进行RF测量和显示滤波器特性特别有用。}
%
%\section{SI5351 PLL}
%
%
%The SI5351 uses a 25 MHz crystal oscillator and contains two PLLs that can operate at a frequency between 600 and 900 MHz. The PLL %dividers are used to multiply the input frequencies to a high frequency intermediate clock while the second stage of synthesis %uses high resolution MultiSynth fractional dividers to generate the desired output frequency. This provides two options for %generating the desired frequency: The PLL can be set to a fixed frequency, for example, 900 MHz, and then divided down using %fractional numbers. Alternatively, the PLL can be adjusted in small steps and then divided down using integer values to generate %the final frequency.
%
%\textbf{中文:SI5351使用25 MHz晶体振荡器,包含两个可以在600到900 MHz之间频率操作的PLL。PLL分频器用于将输入频率乘以高频中间时钟,而合成的第二%级使用高分辨率MultiSynth分数分频器生成所需的输出频率。这提供了两种生成所需频率的选项:PLL可以设置为固定频率,例如900 MHz,然后使用分数进行分%频。或者,PLL可以以小步长调整,然后使用整数值分频以生成最终频率。}
%
%\begin{figure}[htbp]
%\centering
%%%\includegraphics[width=0.6\textwidth]{fig5-30}
%\caption{Block diagram of the SI5351.}
%\end{figure}
%
%The original software-defined radio for shortwave up to 30 MHz (see Section 8.6) was an interesting project, but there were issues %sourcing an alternative for the discontinued PLL chip. Then the SI5351 clock chip from Silicon Labs came along. The Adafruit %breakout board was used for the initial experiments. There is also a useful Arduino library to support it.
%
%\textbf{中文:原始的短波高达30 MHz的软件定义无线电(见8.6节)是一个有趣的项目,但在寻找已停产PLL芯片的替代品时存在问题。然后Silicon Labs的%SI5351时钟芯片出现了。Adafruit breakout板用于初始实验。还有一个有用的Arduino库来支持它。}
%
%\begin{figure}[htbp]
%\centering
%%%\includegraphics[width=0.6\textwidth]{fig5-31}
%\caption{The Adafruit SI5351 board.}
%\end{figure}
%
%With all this help and an Arduino Uno, it was possible to build on the Elektor SDR project and expand its capabilities. The SI5351 %is now a key component in the Elektor SDR Shield (Section 8.7). Thanks to its excellent characteristics, numerous possibilities %are now available, including digital data transmission, for example, with a WSPR transmitter or for HF measurement applications.
%
%\textbf{中文:有了所有这些帮助和Arduino Uno,就有可能在Elektor SDR项目的基础上构建并扩展其功能。SI5351现在是Elektor SDR Shield(第8.7节)中%的关键组件。由于其出色的特性,现在有许多可能性,包括数字数据传输,例如,使用WSPR发射机或用于HF测量应用。}
%
%\begin{figure}[htbp]
%\centering
%%%\includegraphics[width=0.6\textwidth]{fig5-32}
%\caption{The SI5351 (IC1) mounted on the Elektor SDR shield.}
%\end{figure}
%
%It's not necessary to always use an Arduino board to control the SI5351 it can of course be interfaced to much smaller %microcontroller. In fact, Andrew Woodfield, ZL2PD, has produced a program written in Bascom for the ATtiny85. Using his code, it %is possible to use two of the PLL outputs of the SI5351 simultaneously by driving both internal PLLs. Any slight change in the %frequency is tracked by the PLL while the divide-ratio settings of the following dividers remain unchanged. This method promises %to keep phase noise to a minimum. The source code for this project can be found on the author's website.
%
%\textbf{中文:不一定要总是使用Arduino板来控制SI5351,它当然可以与更小的微控制器接口。事实上,Andrew Woodfield,ZL2PD,已经为ATtiny85制作了一%个用Bascom编写的程序。使用他的代码,可以通过驱动两个内部PLL同时使用SI5351的两个PLL输出。频率的任何微小变化都由PLL跟踪,而后续分频器的分频比设置%保持不变。这种方法有望将相位噪声保持在最低水平。该项目的源代码可以在作者的网站上找到。}
%
%\begin{figure}[htbp]
%\centering
%%%\includegraphics[width=0.6\textwidth]{fig5-33}
%\caption{Controlled by a ATiny85.}
%\end{figure}
%
%\chapter{Direct Mixers}
%\textbf{中文:直接混频器}
%
%Direct mixers consist of an oscillator and a mixer and convert an RF input signal to a baseband (usually audio) signal in one %step. They are suitable for receiving SSB and CW broadcasts as well as digital broadcasts such as DRM. Compared to a regenerative %receiver mixer achieves better frequency stability. An oscillator with low phase noise is important for error-free reception of %DRM broadcasts.
%
%\textbf{中文:直接混频器由振荡器和混频器组成,一步将RF输入信号转换为基带(通常是音频)信号。它们适用于接收SSB和CW广播以及数字广播,如DRM。与再%生接收机相比,混频器实现更好的频率稳定性。具有低相位噪声的振荡器对于DRM广播的无差错接收很重要。}
%
%\section{Mixer Types}
%\textbf{中文:混频器类型}
%
%Mixer circuits convert input frequencies to other frequency ranges. In a superheterodyne receiver, the mixer converts the received %signal to an intermediate frequency (IF) (see Figure 6.1). In contrast, a direct-conversion mixer or zero-IF mixer converts the %signal directly to audio frequency (AF) range. Such receivers are used, for example, in simple amateur radio applications. For DRM %reception, the signal is converted to the 12 kHz range, and it is only a question of definition whether to call the output signal %an AF or an IF.
%
%\textbf{中文:混频器电路将输入频率转换为其他频率范围。在超外差接收机中,混频器将接收到的信号转换为中频(IF)(见图6.1)。相比之下,直接转换混频%器或零IF混频器直接将信号转换为音频频率(AF)范围。这种接收机用于例如简单的业余无线电应用。对于DRM接收,信号被转换到12 kHz范围,这只是定义的问%题,是否将输出信号称为AF或IF。}
%
%\begin{figure}[htbp]
%\centering
%%%\includegraphics[width=0.6\textwidth]{fig6-1}
%\caption{Block diagram of a medium wave superhet radio.}
%\end{figure}
%
%A direct mixer directly converts the received signal to the audio frequency range, without the use of an intermediate frequency. %It consists of an oscillator and a mixer and is commonly used in simple amateur radio applications. Due to the finite Q factor of %the resonant circuit, image frequency suppression is only possible at low frequencies (long wave).
%
%\textbf{中文:直接混频器直接将接收到的信号转换为音频频率范围,无需使用中频。它由振荡器和混频器组成,通常用于简单的业余无线电应用。由于谐振电路的%有限Q因子,镜像频率抑制仅在低频(长波)下可能。}
%
%In this text, mixer types are described in connection to their use in a superheterodyne type of receiver with an IF amplifier. %However, the same circuit can also be used as a direct mixer by tapping off the baseband signal instead of the intermediate %frequency.
%
%\textbf{中文:在本文中,混频器类型是结合它们在带有IF放大器的超外差接收机中的使用来描述的。然而,通过提取基带信号而不是中频,同一电路也可以用作直%接混频器。}
%
%One particularly simple and commonly used mixer is the multiplying mixer stage which uses a single transistor. When two sinusoidal %signals are applied to a multiplier, the output signal contains the sum and difference frequencies of the two input signals, in %addition to the two original signals. In this case, multiplication means that the gain of one signal is directly controlled by the %instantaneous value of the second signal. A variable gain amplifier is therefore necessary, the gain of which can be adjusted %directly by an oscillator signal.
%
%\textbf{中文:一种特别简单且常用的混频器是使用单个晶体管的乘法混频器级。当两个正弦信号应用于乘法器时,输出信号除了两个原始信号外,还包含两个输入%信号的和频和差频。在这种情况下,乘法意味着一个信号的增益直接由第二个信号的瞬时值控制。因此,需要一个可变增益放大器,其增益可以通过振荡器信号直接%调节。}
%
%A single transistor can be used to mix multiple frequencies when biased at the appropriate operating point and both signals are %applied to the base. This method of signal processing is also known as an additive mixer. Considering the gain of the transistor %as the product of its transconductance and external resistance and noting that the transconductance is proportional to the %collector current, a change in the collector current due to the oscillator signal is sufficient to multiply the input signal with %the oscillator signal. The input signal should be small enough so that it only operates over a narrow region of the %transconductance slope, i.e., being kept below 1 mV. The oscillator signal, on the other hand, should modulate the collector %current as linearly as possible and without overloading it.
%
%\textbf{中文:当晶体管偏置在适当的工作点并且两个信号都施加到基极时,单个晶体管可以用于混合多个频率。这种信号处理方法也被称为加法混频器。考虑到晶%体管的增益是其跨导和外部电阻的乘积,并注意到跨导与集电极电流成正比,由振荡器信号引起的集电极电流变化足以将输入信号与振荡器信号相乘。输入信号应该%足够小,以便它只在跨导斜率的狭窄区域内工作,即保持1 mV以下。另一方面,振荡器信号应该尽可能线性地调制集电极电流,而不会使其过载。}
%
%\begin{figure}[htbp]
%\centering
%%%\includegraphics[width=0.6\textwidth]{fig6-2}
%\caption{A slope multiplier used as a mixer.}
%\end{figure}
%
%Figure 6.2 shows a simple mixer using an NPN transistor in a Superheterodyne application. The low impedance oscillator signal is %coupled to the emitter and directly modulates the collector current. The current feedback using emitter coupling ensures good %linearity. The input signal is fed directly to the transistor base. This signal must be relatively small to ensure the transistor %is not driven beyond its narrow effective operating region. The collector current therefore contains the mixed down converted %signal, from which the intermediate frequency can be filtered out to recover the desired baseband signal. The mixer operating %point should be stabilized to ensure consistent results.
%
%\textbf{中文:图6.2显示了在超外差应用中使用NPN晶体管的简单混频器。低阻抗振荡器信号耦合到发射极并直接调制集电极电流。使用发射极耦合的电流反馈确保%良好的线性度。输入信号直接馈送到晶体管基极。该信号必须相对较小,以确保晶体管不会被驱动超出其狭窄的有效工作区域。因此,集电极电流包含下变频信号,%从中可以滤除中频以恢复所需的基带信号。混频器工作点应稳定以确保一致的结果。}
%
%In principle, both signals could also be applied to the transistor base. Figure 6.3 shows a simplified version of the mixer, with %the operating point stabilization omitted for clarity of principle. It's important that the oscillator signal is large enough to %modulate the collector current and the input signal is small enough not to create distortion. In simple receivers, %self-oscillating mixer stages are often used, where the oscillator transistor also serves as the mixer. A regenerative transistor %stage can also be regarded as a mixer of this type.
%
%\textbf{中文:原则上,两个信号也可以施加到晶体管基极。图6.3显示了混频器的简化版本,为了原理清晰起见,省略了工作点稳定。重要的是振荡器信号足够大%以调制集电极电流,并且输入信号足够小以不产生失真。在简单接收机中,经常使用自振荡混频器级,其中振荡器晶体管也用作混频器。再生晶体管级也可以被视为%这种类型的混频器。}
%
%\begin{figure}[htbp]
%\centering
%%%\includegraphics[width=0.6\textwidth]{fig6-3}
%\caption{A simplified mixer.}
%\end{figure}
%
%The simple mixer circuit shown in Figure 6.3 is practically no different from the basic circuit for a single transistor emitter %amplifier. This means that practically any amplifier can also become a mixer if you supply it with two signals of different %frequencies and appropriate signal amplitudes. This also poses a danger because most of the time, the frequency mix needs to %undergo further processing. The input amplifier in a shortwave receiver often has to deal with very strong signals alongside very %weak signals. It can happen that one of the stronger signals acts like an oscillator signal and generates mixing products. This %phenomenon is called intermodulation or cross-modulation. This generates numerous interference signals that disturb the reception %of weak signals.
%
%\textbf{中文:图6.3所示的简单混频器电路实际上与单个晶体管发射极放大器的基本电路没有区别。这意味着,实际上任何放大器都可以成为混频器,只要你向它%提供两个不同频率的信号和适当的信号幅度。这也带来了危险,因为大多数时候,频率混合需要进行进一步处理。短波接收机中的输入放大器经常需要处理非常强的%信号和非常弱的信号。可能发生的是,其中一个较强的信号像振荡器信号一样工作并生成混合产物。这种现象称为互调或交叉调制。这会生成许多干扰信号,干扰弱%信号的接收。}
%
%The same applies to the transistor in a mixing stage such as in Figure 6.2. In addition to the oscillator signal, other high %levels input signals can also lead to mixing products. The interference immunity of such a mixing stage is therefore not %particularly high. They are still used in simple radio receivers where high sensitivity or immunity to high input signals are not %important properties. For better performance more sophisticated mixers must be used. A shortwave receiver, as used in amateur %radio applications, should be able to cope with signals in the sub 1 µV range as well as much stronger signals of up to 100 mV %without distortion. This wide dynamic range is only possible if the mixer characteristics are extremely linear for the entire %input signal range.
%
%\textbf{中文:同样的情况适用于图6.2中的混频器级中的晶体管。除了振荡器信号外,其他高电平输入信号也会导致混合产物。因此,这种混频器级的抗干扰能力%不是特别高。它们仍然用于简单的无线电接收机中,在这些接收机中,高灵敏度或对高输入信号的抗扰度不是重要特性。为了获得更好的性能,必须使用更复杂的混%频器。业余无线电应用中使用的短波接收机应该能够处理低1 µV范围的信号以及高100 mV的更强信号而不失真。只有当混频器特性在整个输入信号范围内极其线性%时,这种宽动态范围才有可能实现。}
%
%Dual-gate field-effect transistors (DG-MOSFETs) such as the BF961 can be used to build a mixer with a good dynamic range. Used as %amplifiers, they have a linear characteristic curve and therefore produce low distortion. The steepness of the transistor %characteristic can be modulated via the second gate. Using appropriate adjustment of the gate voltages and a suitable oscillator %level, good dynamic range is achievable. Dual-gate MOSFETs are commonly used in shortwave and FM receivers mainly because they %lead to reduced circuit complexity.
%
%\textbf{中文:双栅极场效应晶体管(DG-MOSFET)如BF961可用于构建具有良好动态范围的混频器。用作放大器时,它们具有线性特性曲线,因此产生低失真。晶%体管特性的陡度可以通过第二栅极调制。通过适当调整栅极电压和合适的振荡器电平,可以实现良好的动态范围。双栅极MOSFET通常用于短波和FM接收机,主要是因%为它们导致电路复杂度降低。}
%
%\begin{figure}[htbp]
%\centering
%%%\includegraphics[width=0.6\textwidth]{fig6-4}
%\caption{Mixer stage using a dual gate MOSFET.}
%\end{figure}
%
%A single diode can also be used as a mixer. In the circuit shown in Figure 6.5, both signals are first added together. The %envelope of the resulting mixed signal contains the desired mixing products. A rectifier is all you need to extract them.
%
%\textbf{中文:单个二极管也可以用作混频器。在图6.5所示的电路中,两个信号首先相加。所得混合信号的包络包含所需的混合产物。您只需要一个整流器来提取%它们。}
%
%\begin{figure}[htbp]
%\centering
%%%\includegraphics[width=0.6\textwidth]{fig6-5}
%\caption{A diode mixer.}
%\end{figure}
%
%An additive mixer has good ability to handle large signals. One disadvantage is that more than two frequency products are created %in the mixing process. The oscillator signal is now no longer a sinusoid but appears as a square wave because of the diode %switching characteristics. The signal now includes odd harmonics of the oscillator signal i.e., 3 fOSC, 5 fOSC, etc. Each of these %harmonics also generates corresponding mixed products. For a demodulator, this is not significant, as signals with multiples of %the carrier frequency are far enough away from the wanted baseband signal so that they can be removed with a low-pass filter that %will be required anyway. At the receiver front-end they can be suppressed by filters.
%
%\textbf{中文:加法混频器具有处理大信号的良好能力。一个缺点是在混合过程中会产生两个以上的频率产物。由于二极管开关特性,振荡器信号不再是正弦波,而%是表现为方波。信号现在包括振荡器信号的奇次谐波,即3 fOSC、5 fOSC等。每个这些谐波也会生成相应的混合产物。对于解调器来说,这并不重要,因为具有载波%频率倍数的信号离所需的基带信号足够远,因此可以通过无论如何都需要的低通滤波器去除。在接收机前端,它们可以被滤波器抑制。}
%
%In principle, a transistor or field-effect transistor such as the BF245 can also be used as the active element in a mixer. A JFET %has the advantage of very fast switching times and lower levels of distortion. In the circuit shown in Figure 6.6, a negative bias %voltage for the field-effect transistor is automatically established. The FET acts like a switch that repeatedly shorts the input %signal. This type of large-signal mixer is used, for example, as a second mixer in the DRM receiver covered in Section 7.4.
%
%\textbf{中文:原则上,晶体管或场效应晶体管如BF245也可以用作混频器中的有源元件。JFET具有非常快的开关时间和较低失真水平的优势。在图6.6所示的电路%中,场效应晶体管的负偏置电压自动建立。FET的作用就像一个开关,反复短路输入信号。这种类型的大信号混频器例如用作7.4节中介绍的DRM接收机中的第二混频%器。}
%
%\begin{figure}[htbp]
%\centering
%%%\includegraphics[width=0.6\textwidth]{fig6-6}
%\caption{A JFET mixer.}
%\end{figure}
%
%A special form of diode mixer is the fully symmetrical ring mixer using four diodes. The diodes work as switches that are %controlled in sync with the oscillator frequency. Schottky diodes are usually used because they have particularly fast switching %times and generate low distortion. At the output of the ring mixer, neither the oscillator signal nor the input signal appears if %the broadband transformers and the diodes are well balanced. A ring mixer is used in the input of the DRM receiver covered in %Section 7.4 and in the direct mixer in Section 6.4.
%
%\textbf{中文:二极管混频器的一种特殊形式是使用四个二极管的完全对称环形混频器。二极管作为开关工作,与振荡器频率同步控制。通常使用肖特基二极管,因%为它们具有特别快的开关时间并产生低失真。如果宽带变压器和二极管平衡良好,环形混频器的输出既不会出现振荡器信号也不会出现输入信号。环形混频器用于7.4%节中介绍的DRM接收机的输入和第6.4节中的直接混频器。}
%
%\begin{figure}[htbp]
%\centering
%%%\includegraphics[width=0.6\textwidth]{fig6-7}
%\caption{A wideband diode mixer.}
%\end{figure}
%
%Integrated mixers often work as fully symmetric multipliers. One simple mixer with a built-in oscillator is the NE612. This IC %requires only minimal external circuitry and works at frequencies up to 300 MHz. It is suitable for battery operation and works %with a supply voltage between 4.5 and 8.5 V.
%
%\textbf{中文:集成混频器通常作为完全对称乘法器工作。一种带有内置振荡器的简单混频器是NE612。这个IC只需要最少的外部电路,工作频率高达300 MHz。它%适合电池操作,工作电压在4.5到8.5 V之间。}
%
%\begin{figure}[htbp]
%\centering
%%%\includegraphics[width=0.6\textwidth]{fig6-8}
%\caption{The NE612 with internal oscillator.}
%\end{figure}
%
%\begin{figure}[htbp]
%\centering
%%%\includegraphics[width=0.6\textwidth]{fig6-9}
%\caption{Block diagram of the NE612.}
%\end{figure}
%
%The internal circuitry of the NE612 provides all the necessary bias voltages, so input signals can be AC coupled with capacitors. %Collector resistors are also built in. The mixer has an input and output impedance of 1.5 kΩ and is suitable for balanced or %unbalanced operation (see Figure 6.10). The internal oscillator can be configured for an external crystal, a tuned tank network or %as a buffer to an external local oscillator.
%
%\textbf{中文:NE612的内部电路提供所有必要的偏置电压,因此输入信号可以通过电容器AC耦合。集电极电阻也内置。混频器的输入和输出阻抗1.5 kΩ,适合平衡%或不平衡操作(见图6.10)。内部振荡器可以配置为外部晶体、调谐tank网络或作为外部本地振荡器的缓冲器。}
%
%\begin{figure}[htbp]
%\centering
%%%\includegraphics[width=0.6\textwidth]{fig6-10}
%\caption{The NE612 used as a symmetrical and unsymmetrical mixer.}
%\end{figure}
%
%\section{Direct Mixer using a BF245}
%\textbf{中文:使用BF245的直接混频器}
%
%A low-cost, home-brew direct mixer design often relies on a freely tunable oscillator to avoid the cost of a custom crystal or %professional VFO. Ideally a direct mixer should have the largest possible dynamic range to allow for the use of a better antenna %in weak reception conditions. Experience with simple regenerative receivers has shown that the use of a long antenna can lead to %receiver front end overload and generation of intermodulation products. A passive mixer without any mixing amplification inserted %directly behind the receiver front end can provide a solution if it has good interference rejection.
%
%\textbf{中文:低成本的自制直接混频器设计通常依赖于可自由调谐的振荡器,以避免定制晶体或专业VFO的成本。理想情况下,直接混频器应具有尽可能大的动态%范围,以允许在弱接收条件下使用更好的天线。使用简单再生接收机的经验表明,使用长天线会导致接收机前端过载和生成互调产物。如果具有良好的干扰抑制能%力,直接插入接收机前端后面的无混合放大的无源混频器可以提供解决方案。}
%
%A JFET mixer is particularly simple and exhibits good large signal stability. If the FET is operated without a DC voltage, it %essentially operates as a controlled resistor. An ideal passive resistor has a linear characteristic that does not generate any %signal distortion. The FET comes relatively close to this ideal. Therefore, even with less than optimal control, an FET is a very %good switch that can handle relatively large signals without producing intermodulation products in a mixer configuration.
%
%\textbf{中文:JFET混频器特别简单,并表现出良好的大信号稳定性。如果FET在没有DC电压的情况下操作,它本质上作为受控电阻器工作。理想的无源电阻器具有%不产生任何信号失真的线性特性。FET相对接近这一理想。因此,即使在控制不够理想的情况下,FET也是一个非常好的开关,可以处理相对较大的信号而不在混频器%配置中产生互调产物。}
%
%\begin{figure}[htbp]
%\centering
%%%\includegraphics[width=0.6\textwidth]{fig6-11}
%\caption{The FET Direct Mixer.}
%\end{figure}
%
%The circuit shown in Figure 6.11 was designed for use across the entire shortwave range from 5.8 MHz upwards. The oscillator and %input circuit use separate tuning capacitors, so the circuits do not have to be adjusted for optimal synchronization. If %necessary, the tuning range can be narrowed down to a specific band to allow for finer tuning. The oscillator coil has taps at %around 20\% and 50\% of the total number of winding turns. A relatively loose coupling with the transistor results in good %frequency stability.
%
%\textbf{中文:图6.11所示的电路设计用于整个短波范围,从5.8 MHz向上。振荡器和输入电路使用单独的调谐电容器,因此电路不需要调整以获得最佳同步。如果%需要,调谐范围可以缩小到特定频段以允许更精细的调谐。振荡器线圈在总绕组匝数的20\%和50\%处有抽头。与晶体管的相对松散耦合导致良好的频率稳定性。}
%
%An oscillator signal of approximately 1 to 2 Vpp is sufficient at the gate of the BF245. The additional gate resistor prevents %excessive damping of the oscillator network when the JFET input diode enters a conductive state. The FET acts as a switch that %provides a short to ground in sync with the oscillator signal for the received signal from the coupled coil. This produces the %audio signal at the output filter. The coupling coil has approximately 20\% of the number of turns at the antenna resonant coil. %By changing this transformer ratio, an optimal match can be found, with a small coupling factor resulting in a narrow input %circuit bandwidth and improved noise immunity.
%
%\textbf{中文:BF245栅极处约1到2 Vpp的振荡器信号就足够了。当JFET输入二极管进入导通状态时,额外的栅极电阻可防止振荡器网络过度阻尼。FET作为开关,%为来自耦合线圈的接收信号提供与振荡器信号同步的接地短路。这在输出滤波器处产生音频信号。耦合线圈的匝数约为天线谐振线圈的20\%。通过改变此变压器比%率,可以找到最佳匹配,小耦合因子会导致窄输入电路带宽和改善的抗噪声能力。}
%
%The FET mixer offers good large-signal immunity but intermodulation products could be generated further down the line in the audio %stage. A low-pass filter is therefore inserted to reduce the signal bandwidth. Signals above 20 kHz are now sufficiently %attenuated, so they will not generate intermodulation products. The circuit was mainly used for DRM reception in the 49 and 41-m %bands. It showed no evidence of intermodulation effects even at high field strengths using a long antenna. This allows for the use %of an outdoor antenna which suffers less from the effects of general domestic electrical interference.
%
%\textbf{中文:FET混频器提供良好的大信号抗扰度,但互调产物可能在音频级的后续阶段生成。因此,插入低通滤波器以减少信号带宽。现20 kHz以上的信号被充%分衰减,因此它们不会生成互调产物。该电路主要用于49和41米波段的DRM接收。即使在使用长天线的高场强下,也没有显示互调效应的证据。这允许使用户外天线,%其受一般家庭电气干扰的影响较小。}
%
%The passive FET mixer does not provide any amplification on its own. The resulting audio signals are therefore way down in the %microvolt range. A single audio stage boosts the signal level high enough for the microphone input of a PC sound card. Its overall %sensitivity depends mainly on the antenna properties. The circuit has very effective selectivity, allowing the reception of weak %amateur radio signals in the 40-m band even in the immediate vicinity of high power broadcast signals.
%
%\textbf{中文:无源FET混频器本身不提供任何放大。因此,产生的音频信号处于微伏范围内。单个音频级将信号电平提升到足够高,以适应PC声卡的麦克风输入。%其整体灵敏度主要取决于天线特性。该电路具有非常有效的选择性,即使在高功率广播信号的紧邻区域,也能接收40米波段的微弱业余无线电信号。}
%
%\section{Diode Ring Mixer}
%\textbf{中文:二极管环形混频器}
%
%In this test setup, a DDS signal generator is used to provide the local oscillator signal to a 4-diode ring mixer. The direct %mixer uses a Mini Circuits TDM2 diode ring mixer and TUF-1 can also be used in its place. The DDS oscillator output signal level %at 0 dBm was raised by about 7 dBm using a BF494 transistor as a broadband amplifier. The baseband signal recovered by the mixer %is amplified by the low-noise B548C AF stage at the mixer output. This provides sufficient signal level to directly drive the %microphone input of a PC sound card even with a poor RF. The mixer is terminated with about 50 Ω, resulting in good large-signal %stability. For best reception a long length of wire setup high outdoors can be used for the antenna. If that is impractical a one %to three meters length of wire stretched out around the room can also do the job. There are no tuned resonant circuits or RF %preamplifiers used in this design so construction is not particularly critical.
%
%\textbf{中文:在这个测试设置中,DDS信号发生器用于向4二极管环形混频器提供本地振荡器信号。直接混频器使用Mini Circuits TDM2二极管环形混频器,也可%以使用TUF-1代替。使用BF494晶体管作为宽带放大器,将0 dBm的DDS振荡器输出信号电平提高约7 dBm。混频器恢复的基带信号由混频器输出处的低噪声B548C AF%级放大。这提供了足够的信号电平,即使在RF较差的情况下也能直接驱动PC声卡的麦克风输入。混频器端接50 Ω,导致良好的大信号稳定性。为了获得最佳接收效%果,可以使用室外高处设置的长导线作为天线。如果这不切实际,在房间周围伸展1到3米长的导线也可以完成工作。该设计中没有使用调谐谐振电路或RF前置放大%器,因此构造不是特别关键。}
%
%\begin{figure}[htbp]
%\centering
%%%\includegraphics[width=0.6\textwidth]{fig6-12}
%\caption{Direct mixer using 50 Ω termination.}
%\end{figure}
%
%A ready-made Schottky diode ring mixer is not cheap but you can easily build one yourself. The mixer consists of four identical %Schottky diodes type BAR28 and two broadband transformers. The RF transformers are wound on Amidon FT37-77 ferrite toroidal cores. %The primary winding consists of a trifilar winding of 10 turns of 0.2 mm CuL wires wound through the toroid. To make this, three %lengths wires are laid alongside each other and wound through the core ten times.
%
%\textbf{中文:现成的肖特基二极管环形混频器并不便宜,但你可以轻松自己制作一个。混频器由四个相同的BAR28型肖特基二极管和两个宽带变压器组成。RF变压%器缠绕在Amidon FT37-77铁氧体环形磁芯上。初级绕组由10匝0.2 mm CuL线的三线并绕组成,穿过环形磁芯。要制作这个,将三根长度的线并排放在一起,穿过磁%芯缠绕十次。}
%
%\begin{figure}[htbp]
%\centering
%%%\includegraphics[width=0.6\textwidth]{fig6-13}
%\caption{Windings on the ferrite toroidal cores.}
%\end{figure}
%
%Then ends of the coils can now be scraped clean and tinned with solder. A continuity tester can be used to identify the ends of %the three coils. Two of the windings are connected in series to form the two-phase winding for connection to the diodes. The third %winding forms the oscillator or signal input. The four diodes can now be soldered on correctly and mixer will be ready to go. %Results from testing the homemade mixer in the circuit shown in Figure 6.12 indicates that its performance is not too far away %from commercially produced ones.
%
%\textbf{中文:现在可以将线圈的末端刮干净并用焊锡镀锡。可以使用连续性测试仪来识别三个线圈的末端。两个绕组串联连接,形成用于连接到二极管的两相绕%组。第三个绕组形成振荡器或信号输入。现在可以正确焊接四个二极管,混频器就可以使用了。在图6.12所示电路中测试自制混频器的结果表明,其性能与商业生产%的混频器相差不远。}
%
%\section{Direct Mixer using an NE612}
%\textbf{中文:使用NE612的直接混频器}
%
%A widely used and inexpensive integrated mixer is the NE612. This IC contains an internal oscillator and a fully symmetrical %mixer. If an external oscillator is used, the IC must be driven at Pin 6 with a signal level from 200 mVpp to a maximum of 300 %mVpp.
%
%\textbf{中文:一种广泛使用且价格低廉的集成混频器是NE612。该IC包含内部振荡器和完全对称的混频器。如果使用外部振荡器,IC必须在引脚6处以200 mVpp至%最大300 mVpp的信号电平驱动。}
%
%\begin{figure}[htbp]
%\centering
%%%\includegraphics[width=0.6\textwidth]{fig6-14}
%\caption{External control from a DDS-Generator.}
%\end{figure}
%
%This receiver design operates without any input selector and uses a long wire antenna. At the input, there is only a small, fixed %value inductor. Using a long wire antenna, DRM signals can be received with an SNR of up to 20 dB. The mixer used in this %receiver, however, is not quite as good as a diode mixer, possibly due to its poorer performance with large signals. Overloading %the mixer input will generate intermodulation products, which then interfere with the DRM signal. One advantage of the NE612 mixer %is its mixer gain, which gives it greater sensitivity. The NE612 circuit has better performance than a diode mixer when using %short whip antennas and with low RF signal levels.
%
%\textbf{中文:这种接收机设计没有任何输入选择器,使用长导线天线。在输入处,只有一个小的固定值电感器。使用长导线天线,可以接收SNR高达20 dB的DRM信%号。然而,这种接收机中使用的混频器不如二极管混频器好,可能是因为它在大信号下的性能较差。混频器输入过载会生成互调产物,然后干扰DRM信号。NE612混频%器的一个优点是其混频器增益,这使其具有更高的灵敏度。当使用短鞭状天线和低RF信号电平时,NE612电路的性能优于二极管混频器。}
%
%\begin{figure}[htbp]
%\centering
%%%\includegraphics[width=0.6\textwidth]{fig6-15}
%\caption{An integrated direct mixer.}
%\end{figure}
%
%Figure 6.15 shows a freely-tunable receiver for the 40-meter band. The internal oscillator of the NE612 exhibits good stability %and low phase noise, even when operating without a tuned circuit. The tuned input circuit provides enough preselection to avoid %overload from strong signals in other bands.
%
%\textbf{中文:图6.15显示了一个用于40米波段的可自由调谐接收机。NE612的内部振荡器即使在没有调谐电路的情况下也表现出良好的稳定性和低相位噪声。调谐%输入电路提供足够的预选,以避免来自其他波段强信号的过载。}
%
%Using a coil of approximately 20 turns wound on an 8 mm diameter coil former, both 6 MHz and 7 MHz can be received by adjusting %the screw-in ferrite slug. A small FM tuning capacitor with triple reduction drive provides sensitive tuning control. The input %circuit is not critical and can also be tuned by soldering a fixed capacitor, such as a 120 pF capacitor across the adjustable %coil.
%
%\textbf{中文:使用缠绕在8 mm直径线圈骨架上的20匝线圈,通过调整旋入式铁氧体磁芯,可以接收6 MHz到7 MHz。带有三倍减速驱动的小型FM调谐电容器提供灵%敏的调谐控制。输入电路并不关键,也可以通过在可调线圈两端焊接固定电容器(如120 pF电容器)来调谐。}
%
%\begin{figure}[htbp]
%\centering
%%%\includegraphics[width=0.6\textwidth]{fig6-16}
%\caption{Installed in the enclosure of the VHF tuning capacitor.}
%\end{figure}
%
%The shielded enclosure from an old FM tuner with a 3-gang tuning capacitor was used for the test setup. With the shielding lid %closed, it provides impressive isolation from external influences. Using Styroflex capacitors in the oscillator resonant circuit %provides good frequency stability and easy tuning. Once tuned, the receiver remains on station for many hours.
%
%\textbf{中文:测试设置使用了带有3连调谐电容器的旧FM调谐器的屏蔽外壳。关闭屏蔽盖后,它提供了令人印象深刻的隔离外部影响的能力。在振荡器谐振电路中%使用Styroflex电容器提供了良好的频率稳定性和易于调谐性。一旦调谐,接收机可以保持在一个电台数小时。}
%
%An easier alternative to manual tuning is to use a direct mixer with a quartz crystal. This will of course limit the number %stations you can receive to just one but stability is no longer a problem. The circuit in Figure 6.17 uses a standard 6 MHz quartz %crystal that can be pulled to 6002 kHz using the 20 pF trimmer capacitor. This allows reception of DRM-RTL 2 on 5990 kHz with an %inverted spectrum. The circuit was featured in the German magazine Funkamateur in April 2004 as a PCB project using a special 6107 %kHz quartz crystal to receive transmissions on 6095 kHz. Since then, the station has shut down. You can still, however, receive AM %stations using the appropriate SDR software or switch to the 40-meter amateur radio band by using a different value quartz crystal.
%
%\textbf{中文:手动调谐的一个更简单的替代方案是使用带有石英晶体的直接混频器。当然,这会将你可以接收的电台数量限制为一个,但稳定性不再是问题。图6.%17中的电路使用标准6 MHz石英晶体,可以通过20 pF微调电容器拉到6002 kHz。这允许以反转频谱接收5990 kHz上的DRM-RTL 2。该电路2004年4月在德国杂志%Funkamateur上作为PCB项目展示,使用特殊的6107 kHz石英晶体接收6095 kHz上的传输。从那时起,该电台已经关闭。然而,你仍然可以使用适当的SDR软件接收%AM电台,或者通过使用不同值的石英晶体切换40米业余无线电波段。}
%
%\begin{figure}[htbp]
%\centering
%%%\includegraphics[width=0.6\textwidth]{fig6-17}
%\caption{A fixed-frequency receiver for 5990 kHz.}
%\end{figure}
%
%\chapter{The AM-Superheterodyne Receiver}
%\textbf{中文:AM超外差接收机}
%
%Regenerative and direct conversion receivers are now really only of interest as hobby projects, almost all radios you can buy %today work according to the superheterodyne principle. A superheterodyne achieves better selectivity and also allows for the use %of automatic gain control (AGC). Here you will take a closer look at the technology and build some simple superhet receiver %designs.
%
%\textbf{中文:再生和直接转换接收机现在实际上只作为业余项目感兴趣,几乎所有你今天能买到的收音机都根据超外差原理工作。超外差实现了更好的选择性,并%且还允许使用自动增益控制(AGC)。在这里,你将更仔细地研究这项技术并构建一些简单的超外差接收机设计。}
%
%\section{An AM Shortwave Receiver using the TCA440}
%\textbf{中文:使用TCA440的AM短波接收机}
%
%The TCA440 integrated circuit simplifies the construction of an AM superhet receiver, as all stages are combined in one IC. Figure %7.1 shows the typical structure of a freely tunable receiver using a ceramic filter in the IF stage. The IC contains a regulated %front-end and a regulated IF amplifier. The control voltage is obtained via a separate diode at the output of the IF amplifier. %The received signal strength can also be displayed on an S-meter. A suitable audio amplifier stage is still required at the audio %output.
%
%\textbf{中文:TCA440集成电路简化了AM超外差接收机的构造,因为所有级都集成在一个IC中。图7.1显示了在IF级使用陶瓷滤波器的可自由调谐接收机的典型结%构。该IC包含一个稳压前端和一个稳压IF放大器。控制电压通过IF放大器输出处的单独二极管获得。接收信号强度也可以在S表上显示。在音频输出处仍然需要合适的%音频放大器级。}
%
%\begin{figure}[htbp]
%\centering
%%%\includegraphics[width=0.6\textwidth]{fig7-1}
%\caption{Shortwave Superhet using the TCA440.}
%\end{figure}
%
%Building and tuning the oscillator and input circuits is the most challenging task when building a superhet. Both circuits must be %finely tuned around the intermediate frequency with good coherence.
%
%\textbf{中文:构建和调谐振荡器和输入电路是构建超外差接收机时最具挑战性的任务。两个电路必须围绕中频进行精细调谐,具有良好的一致性。}
%
%Figure 7.2 shows a variant of the receiver using varicap diodes for tuning. A potentiometer or an external PLL such as the SAA1057 %can be used to provide the tuning control voltage. An additional second mixer allows the receiver to be used as a DRM receiver.
%
%\textbf{中文:图7.2显示了使用变容二极管进行调谐的接收机变体。可以使用电位器或外部PLL(如SAA1057)来提供调谐控制电压。额外的第二混频器允许接收机%用作DRM接收机。}
%
%\begin{figure}[htbp]
%\centering
%%%\includegraphics[width=0.6\textwidth]{fig7-2}
%\caption{A receiver with PLL tuning}
%\end{figure}
%
%The RF coils used come from IF filters for 10.7 MHz and each has an additional 10-turn coupler winding. The frequency range and %coupling coefficient were determined experimentally. Both circuits are tuned using a BB204 dual varicap diode. The oscillator %circuit is adjusted via the coil's screw-in slug which allows the PLL to lock in to the desired frequency range. The input circuit %is then adjusted for maximum output signal amplitude. The circuit has a high Q and provides adequate image frequency rejection. %Good synchronization of both circuits is achieved without difficulty in the range of 5.8 to 7.5 MHz if the input circuit is tuned %precisely in the middle range.
%
%\textbf{中文:使用的RF线圈来自10.7 MHz的IF滤波器,每个都有一个额外的10匝耦合绕组。频率范围和耦合系数是通过实验确定的。两个电路都使用BB204双变%容二极管进行调谐。振荡器电路通过线圈的旋入式磁芯进行调整,这允许PLL锁定到所需的频率范围。然后调整输入电路以获得最大输出信号幅度。该电路具有高Q%值,并提供足够的镜像频率抑制。如果输入电路在中间范围精确调谐,两个电路在5.8到7.5 MHz范围内可以毫不困难地实现良好的同步。}
%
%The IF filters use ready-made coil filters for 455 kHz and the ceramic filter CFW455F. The second oscillator is controlled by a %CSB470 low cost ceramic resonator. Using the component values shown, the frequency is pulled by 3 kHz to 467 kHz. The oscillator %is sufficiently stable and precise to within 1 kHz without the need for any special adjustment.
%
%\textbf{中文:IF滤波器使用现成的455 kHz线圈滤波器和陶瓷滤波器CFW455F。第二振荡器由CSB470低成本陶瓷谐振器控制。使用所示的元件值,频率被拉3 kHz%到467 kHz。振荡器足够稳定和精确,误差1 kHz以内,无需任何特殊调整。}
%
%The demodulator uses a simple additive mixer using a germanium diode, which works well here because the IF voltage at the output %of the TCA440 is already at around 1 V. For distortion-free demodulation, it is important that the oscillator amplitude is greater %than the signal amplitude at the coupling coil. An advantage of this simple demodulator is that it also works seamlessly for AM %when the supply voltage to the second oscillator is switched off.
%
%\textbf{中文:解调器使用一个简单的使用锗二极管的加法混频器,在这里工作良好,因为TCA440输出处的IF电压已经在约1 V左右。对于无失真解调,重要的是振%荡器幅度大于耦合线圈处的信号幅度。这种简单解调器的一个优点是,当第二振荡器的供电电压关闭时,它也可以无缝地用于AM。}
%
%The TCA440 has significant overall gain, requiring the use of automatic gain control (AGC). The control voltage is obtained using %a second germanium diode, which coupled to the output circuit via a 22 kΩ resistor, so that it does not load the signal and give %rise to any distortion. This also results in a softer AGC characteristic so that a single interfering noise spike will not %immediately reduce the gain. This slow reaction is more favorable for DRM reception than a fast AGC response.
%
%\textbf{中文:TCA440具有显著的整体增益,需要使用自动增益控制(AGC)。控制电压通过第二个锗二极管获得,该二极管通过22 kΩ电阻耦合到输出电路,这样%它就不会加载信号并引起任何失真。这也导致更柔和的AGC特性,使得单个干扰噪声尖峰不会立即降低增益。这种缓慢的反应比快速的AGC响应更有利于DRM接收。}
%
%\section{AM/FM Radio using the CD2003GP}
%\textbf{中文:使用CD2003GP的AM/FM收音机}
%
%While searching for highly integrated radio ICs for AM and FM bands I found the CD2003GP chip. This IC can often be found inside %domestic radios such as bedside clock radio receivers. It is also available from Modul-Bus.
%
%\textbf{中文:在寻找用于AM和FM波段的高度集成无线电IC时,我发现了CD2003GP芯片。这种IC经常可以在国内收音机(如床头时钟收音机)中找到。它也可以从%Modul-Bus获得。}
%
%\begin{figure}[htbp]
%\centering
%%%\includegraphics[width=0.6\textwidth]{fig7-3}
%\caption{Block diagram of the CD2003.}
%\end{figure}
%
%A test circuit diagram shows the basic application. Interestingly, the design completely dispenses with coil filters in the IF %stage, making adjustment very easy. Selection depends solely on the ceramic filters.
%
%\textbf{中文:测试电路图显示了基本应用。有趣的是,该设计完全省去了IF级中的线圈滤波器,使得调整非常容易。选择完全取决于陶瓷滤波器。}
%
%\begin{figure}[htbp]
%\centering
%%%\includegraphics[width=0.6\textwidth]{fig7-4}
%\caption{External components to build an FM/AM radio using the CD2003.}
%\end{figure}
%
%The very simple external circuitry uses coils without the need for any tap points. This makes the IC an ideal component for %personal experimentation. Figure 7.5 shows the construction of a complete AM/FM receiver. The input circuit for medium wave uses a %ferrite rod antenna, and the oscillator circuit uses a fixed 100 µH inductor. For FM, free standing air-cored coils are used, %which can be tuned by slightly stretching or squashing the turns together. The concept allows for improvements to the IF filters. %In this case, two AM filters are used with an intermediate circuit and a fixed inductor.
%
%\textbf{中文:非常简单的外部电路使用线圈,不需要任何抽头点。这使得该IC成为个人实验的理想组件。图7.5显示了完整AM/FM接收机的构造。中波输入电路使%用铁氧体棒天线,振荡器电路使用固定的100 µH电感器。对于FM,使用独立的空心线圈,可以通过轻微拉伸或挤压线圈匝来调谐。该概念允许改进IF滤波器。在这种%情况下,使用两个AM滤波器,带有中间电路和固定电感器。}
%
%\begin{figure}[htbp]
%\centering
%%%\includegraphics[width=0.6\textwidth]{fig7-5}
%\caption{A complete AM/FM Radio.}
%\end{figure}
%
%With the circuit constructed, everything can be fitted into the case of a retro radio. An LM386 amplifier drives a large speaker, %resulting in very good sound quality. The ALC information was taken from pin 5 of the radio IC via an emitter follower stage to %display on the S meter. This helps find the optimal tuning point.
%
%\textbf{中文:电路构建完成后,所有东西都可以装入复古收音机的外壳中。LM386放大器驱动大型扬声器,产生非常好的音质。ALC信息通过发射极跟随器级从无线%电IC的引脚5获取,以在S表上显示。这有助于找到最佳调谐点。}
%
%\begin{figure}[htbp]
%\centering
%%%\includegraphics[width=0.6\textwidth]{fig7-6}
%\caption{Installed in a case.}
%\end{figure}
%
%\section{DRM Receiver}
%\textbf{中文:DRM接收机}
%
%In the 3/2004 issue of Elektor Magazine, a DRM superhet receiver (using a 455 kHz IF) was described and proved to be a good %introduction for many readers interested in digital radio systems. Unlike an IQ or zero-IF receiver this design uses a steep IF %filter. The 12 kHz output signal from the receiver is fed to a PC via a mono audio input channel. An older laptop with just a %microphone input may be suitable machine to run the necessary software.
%
%\textbf{中文:在Elektor杂志2004年第3期中,描述了一个DRM超外差接收机(使用455 kHz IF),并被证明是对许多对数字无线电系统感兴趣的读者的良好介%绍。与IQ或零IF接收机不同,该设计使用陡峭的IF滤波器。来自接收机12 kHz输出信号通过单声道音频输入通道馈送到PC。只有麦克风输入的旧笔记本电脑可能是运%行必要软件的合适机器。}
%
%The main goal of the development of this DRM receiver was to build a radio with good reception performance that does not require %any adjustments. No special coils or tuning capacitors are needed in this design, only readily available fixed inductors. This is %a bonus for those who feel more at home with the black and white world of digital electronics rather than the dark arts of RF %technology. Altogether the design works with no tweaking, no special measuring equipment, just a very simple software adjustment %to compensate for tolerances in the oscillator frequencies.
%
%\textbf{中文:开发这个DRM接收机的主要目标是构建一个具有良好接收性能且不需要任何调整的收音机。该设计中不需要特殊的线圈或调谐电容器,只需要现成的%固定电感器。对于那些觉得在数字电子技术的黑白世界中比在RF技术的黑暗艺术中更自在的人来说,这是一个额外的好处。总的来说,该设计无需调整,无需特殊测%量设备,只需一个非常简单的软件调整来补偿振荡器频率的容差。}
%
%\begin{figure}[htbp]
%\centering
%%%\includegraphics[width=0.6\textwidth]{fig7-7}
%\caption{The finished receiver.}
%\end{figure}
%
%Basically, the receiver can be seen as a DRM interface for the PC. As shown in Figure 7.8, the DRM receiver has two connections to %the computer: the first is via the RS232 interface, where the receiver inputs digital control information for tuning the receiver %to the frequency of the desired DRM transmitter.
%
%\textbf{中文:基本上,接收机可以被视为PC的DRM接口。如图7.8所示,DRM接收机与计算机有两个连接:第一个是通过RS232接口,接收机在此输入数字控制信%息,以将接收机调谐到所需DRM发射机的频率。}
%
%\begin{figure}[htbp]
%\centering
%%%\includegraphics[width=0.6\textwidth]{fig7-8}
%\caption{Functional diagram of the DRM receiver and PC.}
%\end{figure}
%
%The second interface supplies the received signal information to the PC. Unlike a regular radio, the output of the DRM receiver is %not an audio signal that can be heard through speakers or headphones. Instead, the DRM receiver mixes the signal of the DRM %transmitter down to an intermediate frequency of 12 kHz. The receiver output supplies a mixture of modulated carrier frequencies %that together transmit the audio signal as a digital data stream. This DRM spectrum, a frequency mixture with a bandwidth of 10 %kHz, is connected to the line input of the PC's sound card. The sound card digitizes the signal, and a DRM receiving program, %containing a DRM software demodulator/decoder as its core component, is responsible for both demodulating the DRM signal and %decoding the received data stream. The audio signal is then available at the output of the sound card in stereo Hi-Fi quality for %playback through the PC's speakers.
%
%\textbf{中文:第二个接口将接收到的信号信息提供给PC。与普通收音机不同,DRM接收机的输出不是可以通过扬声器或耳机听到的音频信号。相反,DRM接收机将%DRM发射机的信号下变频到12 kHz的中频。接收机输出提供调制载波频率的混合,这些频率一起将音频信号作为数字数据流传输。这个DRM频谱,一个带宽为10 kHz%的频率混合,连接到PC声卡的线路输入。声卡将信号数字化,DRM接收程序包含DRM软件解调器/解码器作为其核心组件,负责解调DRM信号和解码接收到的数据流。然%后,音频信号以立体声Hi-Fi质量在声卡输出处可用,通过PC的扬声器播放。}
%
%The block diagram can be easily found in the circuit diagram after Figure 7.9. The DDS oscillator with IC2 (see Section 5.9) %supplies its signal via T1 to the first mixer (MIX1), a diode-ring mixer. The intermediate frequency of 455 kHz passes through a %steep-slope ceramic filter (Fl1) with a bandwidth of 12 kHz. A IF amplifier stage with a BF494 (T2) boosts the level by about 20 %dB before the signal is fed to the second mixer, a passive FET mixer with a BF245 (T4). The second oscillator is stabilized by a %ceramic resonator CSB470, which is 'pulled' by 3 kHz to 467 kHz. The resulting 12 kHz IF signal passes through a simple bandpass %filter and is again amplified and buffered by two opamps (IC3), before it is ready at the output for connection to the PC sound %card.
%
%\textbf{中文:框图可以在图7.9之后的电路图中轻松找到。带有IC2的DDS振荡器(见第5.9节)通过T1将其信号提供给第一混频器(MIX1),这是一个二极管环形%混频器。455 kHz的中频通过带宽12 kHz的陡峭斜率陶瓷滤波器(Fl1)。带有BF494(T2)的IF放大器级将电平提升约20 dB,然后信号被馈送到第二混频器,这是%一个带有BF245(T4)的无源FET混频器。第二振荡器由陶瓷谐振器CSB470稳定,该谐振器被拉3 kHz到467 kHz。产生的12 kHz IF信号通过简单的带通滤波器,然%后由两个运算放大器(IC3)再次放大和缓冲,最后在输出处准备好连接到PC声卡。}
%
%The most important characteristic for good DRM reception is phase purity of the mixer oscillator. The DRM receiver meets the %highest demands here: the DDS VFO generates an extremely phase-pure oscillator signal. Another important characteristic is the %receiver's large-signal rejection performance. The mixers used in this design offer excellent performance in this regard so that %with a simple wire antenna connected to the receiver input the DRM software achieves 30 dB quieting.
%
%\textbf{中文:良好DRM接收的最重要的特性是混频器振荡器的相位纯度。DRM接收机在这里满足最高要求:DDS VFO产生极其相位纯的振荡器信号。另一个重要特性%是接收机的大信号抑制性能。该设计中使用的混频器在这方面提供出色的性能,因此将简单的导线天线连接到接收机输入时,DRM软件可以实现30 dB的静噪。}
%
%\begin{figure}[htbp]
%\centering
%%%\includegraphics[width=0.6\textwidth]{fig7-9}
%\caption{The circuit diagram.}
%\end{figure}
%
%Some properties of a receiver's design are important for AM reception but are not so critical for DRM reception so these excellent %results have been achieved here despite the receiver's simplified design and alignment-free setup.
%
%\textbf{中文:接收机设计的一些特性对AM接收很重要,但对DRM接收并不那么关键,因此尽管接收机设计简化和免调整设置,但在这里取得了这些出色的结果。}
%
%The dynamic range of the PC sound card, together with the DRM software, is large enough to cope with variations of the input %signal of up to 30 dB. This eliminates the need for an automatic gain control (AGC). High sensitivity is also not an issue for %DRM. Even very weak DRM signals of about 10 µV cannot be improved by increasing the overall gain, because the actual signal to %noise ratio is not sufficient with the large bandwidth of 10 kHz. More gain would only raise the noise floor. It has also been %shown that the receiver does not require a tuned preselector. On the one hand, the image frequency at a distance of 910 kHz (2 × %455 kHz) is almost always outside neighboring broadcast bands, and on the other hand, interfering signals are surprisingly well %tolerated by the DRM decoder.
%
%\textbf{中文:PC声卡的动态范围,加上DRM软件,足够处理高30 dB的输入信号变化。这消除了对自动增益控制(AGC)的需求。高灵敏度对DRM也不是问题。即使%约10 µV的非常弱的DRM信号也无法通过增加整体增益来改善,因为10 kHz的大带宽下,实际信噪比不足。更多的增益只会提高噪声底。还表明接收机不需要调谐预选%器。一方面,距910 kHz(2 × 455 kHz)的镜像频率几乎总是在相邻广播波段之外,另一方面,干扰信号被DRM解码器惊人地很好地容忍。}
%
%The antenna input with an impedance of approximately 50 Ω is directly connected to the diode ring mixer TUF-1, which is designed %for a frequency range of 2 to 600 MHz. In practice, however, the receiver can also work in the medium-wave range down to 500 kHz %without any problems. If an active antenna or a low-impedance preamplifier is used, successful operation can also be achieved in %the long-wave range. At the output of the ring mixer, a broadband matching network is used for 455 kHz. The impedance is raised by %a resonant circuit with capacitive tapping to approximately 1.5 kΩ to match the input resistance of the ceramic filter CFW455F. %The circuit is operated with a low Q-factor (<10), which results in a bandwidth of approximately 50 kHz and avoids component %tolerance issues. The matching circuit also contributes to the remote signal rejection of the IF filter.
%
%\textbf{中文:阻抗约50 Ω的天线输入直接连接到二极管环形混频器TUF-1,该混频器设计用于2到600 MHz的频率范围。然而,实际上,接收机也可以在低到500 %kHz的中波范围内工作而没有任何问题。如果使用有源天线或低阻抗前置放大器,也可以在长波范围内成功运行。在环形混频器的输出处,使用宽带匹配网络用于455 %kHz。阻抗通过具有电容抽头的谐振电路提高到1.5 kΩ,以匹配陶瓷滤波器CFW455F的输入电阻。该电路以低Q因子(<10)运行,导致50 kHz的带宽,并避免元件容%差问题。匹配电路也有助于IF滤波器的远程信号抑制。}
%
%The filter CFW455F has a bandwidth of 12 kHz, of which 10 kHz are required for DRM. The additional bandwidth is not detrimental in %fact, having a slightly wider bandwidth is important to handle certain frequency deviations of the second oscillator. If the %second oscillator is not exactly at 467 kHz but, for example, at 467.5 kHz, the first IF shifts to 455.5 kHz. The software then %has to tune the first oscillator 500 Hz higher. In the end, however, a signal of 12 kHz appears as required. The slightly shifted %first IF still passes through the IF filter. This made it possible to avoid an expensive special crystal in the second oscillator. %Instead, the second oscillator at 467 kHz uses an inexpensive ceramic resonator type CSB470. The frequency is pulled down by 3 kHz %due to the large capacitance of the oscillator and reaches a maximum deviation of about 1 kHz.
%
%\textbf{中文:滤波器CFW455F的带宽为12 kHz,其中DRM需要10 kHz。额外的带宽实际上没有害处,具有稍宽的带宽对于处理第二振荡器的某些频率偏差很重要。%如果第二振荡器不完全在467 kHz,而是在例如467.5 kHz,第一IF偏移到455.5 kHz。然后软件必须将第一振荡器调高500 Hz。然而,最终会出现所需的12 kHz信%号。略微偏移的第一IF仍然通过IF滤波器。这使得避免在第二振荡器中使用昂贵的特殊晶体成为可能。相反,467 kHz的第二振荡器使用廉价的陶瓷谐振器CSB470。%由于振荡器的大电容,频率被拉低3 kHz,达到约1 kHz的最大偏差。}
%
%After the IF filter, there is a single unregulated amplifier stage that raises the signal level by about 20 dB. Since there is no %pre-amplification or mixing amplification and the IF filter causes additional signal attenuation, the signal levels are %sufficiently small to safely avoid overloading.
%
%\textbf{中文:在IF滤波器之后,有一个单级未稳压放大器级,将信号电平提高约20 dB。由于没有前置放大或混合放大,并且IF滤波器导致额外的信号衰减,信号%电平足够小,可以安全地避免过载。}
%
%A passive FET mixer converts the signal to 12 kHz. The JFET BF245 works like an RF switch shorting the signal in sync with the %oscillator. This simple mixer has a large dynamic range and processes signals up to over 100 mV without noticeable distortion. The %subsequent audio amplifier with the dual op-amp LM358 raises the level by about 20 dB again and contains a simple bandpass filter.
%
%\textbf{中文:无源FET混频器将信号转换到12 kHz。JFET BF245的工作方式类似于RF开关,与振荡器同步短路信号。这个简单的混频器具有大动态范围,可以处%理高达100 mV以上的信号而不会产生明显的失真。随后的带有双运算放大器LM358的音频放大器再次将电平提高约20 dB,并包含一个简单的带通滤波器。}
%
%\begin{figure}[htbp]
%\centering
%%%\includegraphics[width=0.6\textwidth]{fig7-10}
%\caption{The component mounting plan.}
%\end{figure}
%
%The receiver PCB is populated with conventional through-hole components on the top side, while the DDS is in its SMD outline is %mounted on the underside. Some SMD capacitors are also mounted on the underside to reduce lead length and inductance.
%
%\textbf{中文:接收机PCB在顶部装有常规通孔元件,而DDS以其SMD外形安装在底部。一些SMD电容器也安装在底部,以减少引线长度和电感。}
%
%There are only a few active DRM stations at this time, but the receiver can still be used with appropriate software for shortwave %broadcast reception, digital stations of all kinds, and amateur radio.
%
%\textbf{中文:目前只有少数活跃的DRM电台,但接收机仍然可以使用适当的软件用于短波广播接收、各种数字电台和业余无线电。}
%
%\chapter{IQ Mixers and Software Defined Radio}
%\textbf{中文:IQ混频器和软件定义无线电}
%
%An IQ mixer is a double mixer with a 90 degrees phase difference between the two oscillator signals. This makes it possible to %suppress unwanted image frequencies and reduce the need for complex pre-selection measures for the received signal. This is an %essential principle of most SDR concepts.
%
%\textbf{中文:IQ混频器是一个双混频器,两个振荡器信号之间有90度的相位差。这使得抑制不需要的镜像频率并减少对接收信号的复杂预选措施成为可能。这是大%多数SDR概念的基本原理。}
%
%\section{SDRadio}
%\textbf{中文:SDRadio软件}
%
%Back in the day, it was only high-end world receivers that came with features like a flat screen display, a wide range of %selectable receive bands, and similar luxuries. More recently however, more and more of a receiver's functions are off loaded into %software, while the hardware becomes ever more Spartan. This trend has resulted in a concept called 'Software Defined Radio' (SDR) %and is especially relevant in amateur radio.
%
%\textbf{中文:在过去,只有高端世界接收机才配备平面屏幕显示器、宽范围的可选接收波段和类似的奢侈功能。然而,最近,越来越多的接收机功能被卸载到软件%中,而硬件变得越来越简单。这一趋势导致了一个称为软件定义无线电(SDR)的概念,这在业余无线电中特别相关。}
%
%\begin{figure}[htbp]
%\centering
%%%\includegraphics[width=0.6\textwidth]{fig8-1}
%\caption{The SDRadio GUI.}
%\end{figure}
%
%One of the first and simplest programs for working with SDRadio was created by Italian amateur radio operator Alberto, I2PHD. This %PC program, together with a sound card and a simple IQ mixer as an HF frontend, creates an excellent shortwave receiver working in %all modes from AM to SSB. Without having to retune the receiver, a range up to 48 kHz can be tuned solely with the mouse. You can %always see what is happening on the neighboring frequencies and can flexibly respond to interference by adjusting the receiver %bandwidth, for example.
%
%\textbf{中文:最早和最简单的SDRadio程序之一是由意大利业余无线电操作员Alberto,I2PHD创建的。这个PC程序,加上声卡和简单的IQ混频器作为HF前端,创%建了一个出色的短波接收机,可以在从AM到SSB的所有模式下工作。无需重新调谐接收机,仅用鼠标就可以调谐高48 kHz的范围。你总是可以看到相邻频率上发生的%事情,并通过调整接收机带宽灵活地响应干扰,例如。}
%
%The required hardware is an IQ mixer, essentially a direct mixer in the form of a two-stage mixer with phase-shifted oscillator %signals. This effectively achieves the suppression of image frequencies.
%
%\textbf{中文:所需的硬件是IQ混频器,本质上是具有相移振荡器信号的两级混频器形式的直接混频器。这有效地实现了镜像频率的抑制。}
%
%\section{Image Frequency Rejection}
%\textbf{中文:镜像频率抑制}
%
%Every simple mixer generates, in addition to the desired frequency, a signal at the image frequency, which often requires a lot of %effort to filter out. The I/Q mixer, on the other hand, consists of two mixing stages and provides its own image frequency %suppression. This principle can be used for very simple receivers and is particularly useful in the context of software-based %receivers.
%
%\textbf{中文:每个简单的混频器除了产生所需的频率外,还会在镜像频率处产生信号,这通常需要很多努力来滤除。另一方面,I/Q混频器由两个混频级组成,并%提供自己的镜像频率抑制。这个原理可以用于非常简单的接收机,在基于软件的接收机背景下特别有用。}
%
%The most commonly used types of receivers are the direct conversion receiver, the superheterodyne receiver, and the direct mixer %receiver. With a direct conversion receiver, a resonant circuit at the input provides the only method to select a particular %station. An example is the regenerative receiver where active regeneration provides the necessary selectivity. A direct conversion %receiver does not suffer from image frequency problems but has relatively low attenuation of adjacent channels. The %superheterodyne receiver, on the other hand, uses several intermediate frequency circuits to achieve good selectivity. However, %now the image frequency comes into play. A superheterodyne with an intermediate frequency of 455 kHz has a secondary reception %point at a distance of 2 × 455 kHz = 910 kHz. On medium wave, a pre-selector is sufficient to suppress this unwanted signal. Many %shortwave receivers, however, actually show a significant image frequency.
%
%\textbf{中文:最常用的接收机类型是直接转换接收机、超外差接收机和直接混频器接收机。对于直接转换接收机,输入处的谐振电路提供了选择特定电台的唯一方%法。一个例子是再生接收机,其中主动再生提供了必要的选择性。直接转换接收机不会受到镜像频率问题的影响,但对相邻信道的衰减相对较低。另一方面,超外差%接收机使用几个中频电路来实现良好的选择性。然而,现在镜像频率开始起作用。中频为455 kHz的超外差接收机在2 × 455 kHz = 910 kHz的距离处有次级接收%点。在中波上,预选器足以抑制这个不需要的信号。然而,许多短波接收机实际上显示出显著的镜像频率。}
%
%The direct mixer is a particularly simple receiver that can, for example, produce good results for anyone starting out in amateur %radio. Without going through an intermediate frequency, it mixes the received signal with a local oscillator running at a %frequency very close to the received signal to directly produce the baseband signal. The principle has also been successfully used %for very simple DRM receivers, where the 'AF signal' is actually a 12 kHz intermediate frequency. In both cases, the image %frequency is so close to the target frequency there is no chance of filtering it out. Figure 8.2 illustrates the problem with an %example. A signal at 3990 kHz is to be downmixed to 12 kHz. The mixer oscillator operates at 3990 kHz + 12 kHz = 4002 kHz. This %creates the image frequency of 4002 kHz + 12 kHz = 4014 kHz. Now you have to hope there is no strong signal on this 'wrong' %frequency.
%
%\textbf{中文:直接混频器是一个特别简单的接收机,例如,可以为业余无线电初学者产生良好的结果。它不经过中频,将接收到的信号与以非常接近接收信号频率%运行的本地振荡器混合,直接产生基带信号。该原理也已成功用于非常简单的DRM接收机,其中"AF信号"实际上是12 kHz的中频。在这两种情况下,镜像频率都如此%接近目标频率,没有机会将其滤除。图8.2用一个例子说明了这个问题。3990 kHz的信号要下变频到12 kHz。混频器振荡器运行在3990 kHz + 12 kHz = 4002 %kHz。这产生4002 kHz + 12 kHz = 4014 kHz的镜像频率。现在你必须希望在这个"错误"频率上没有强信号。}
%
%\begin{figure}[htbp]
%\centering
%%%\includegraphics[width=0.6\textwidth]{fig8-2}
%\caption{Image frequency generation.}
%\end{figure}
%
%An I/Q mixer solves the problem of the unwanted spurious frequency by using the concept of quadrature signals. Two identical %mixers are used here, which use the same local oscillator signal, but with a phase difference of 90 degrees. The RF received %signal is now mixed with the two LO signals to produce an I (in-phase) and Q (phase shifted) output. The signals must now be %phase-shifted again and then sent to an adder. Here, the image frequencies will cancel each other out, while the desired signal %will be amplified. The reverse procedure is used to generate SSB signals (Figure 8.3). The same job is performed here: mixing %without an image frequency, which in this case corresponds to the other sideband. The technique is known as the 'phasing method' %in amateur radio circles.
%
%\textbf{中文:I/Q混频器通过使用正交信号的概念解决了不需要的杂散频率问题。这里使用两个相同的混频器,它们使用相同的本地振荡器信号,但有90度的相位%差。接收到的RF信号现在与两个LO信号混合,产生I(同相)和Q(相移)输出。信号现在必须再次相移,然后发送到加法器。在这里,镜像频率将相互抵消,而所需%的信号将被放大。相反的过程用于生成SSB信号(图8.3)。这里执行相同的工作:没有镜像频率的混频,在这种情况下对应于另一个边带。该技术在业余无线电圈子%中被称为相位法。}
%
%\begin{figure}[htbp]
%\centering
%%%\includegraphics[width=0.6\textwidth]{fig8-3}
%\caption{SSB signal generation using the phasing method.}
%\end{figure}
%
%The difficulty with the phasing method, however, is to uniformly rotate an entire frequency band from 300 to 3000 Hz in phase. In %the so-called 'third method' using the 'Weaver modulation' method, two additional mixers are used, which also receive %phase-shifted oscillator signals to recover the audio signal.
%
%\textbf{中文:然而,相位法的困难是在相位中均匀旋转300到3000 Hz的整个频带。在使用"Weaver调制"方法的所谓的"第三法"中,使用了两个额外的混频器,它%们也接收相移振荡器信号以恢复音频信号。}
%
%\begin{figure}[htbp]
%\centering
%%%\includegraphics[width=0.6\textwidth]{fig8-4}
%\caption{Signal processing according to the 'Third Method'.}
%\end{figure}
%
%Nowadays, the conditions for using an I/Q mixer have become even more favorable because signal processing via software has made %tremendous progress. For simple experiments, there are excellent programs like SDRadio or SDR\# available. All you need to do is %provide two signals mixed with a 90-degree phase difference to the left and right channels of the PC sound card. The software %takes care of everything else.
%
%\textbf{中文:如今,使用I/Q混频器的条件变得更加有利,因为通过软件进行信号处理取得了巨大进步。对于简单的实验,有像SDRadio或SDR\#这样的优秀程序可%用。你只需要做的就是提供两个与90度相位差混合的信号到PC声卡的左右声道。软件负责处理其他所有事情。}
%
%\section{The IQ Mixer}
%\textbf{中文:IQ混频器}
%
%The simplest way to generate a phase shifted waveform from an oscillator signal is to use some digital circuitry. Two D-type %flip-flops such as the 74AC74 can be used to divide an input frequency by four and simultaneously produce two output waveforms %shifted by exactly 90 degrees.
%
%\textbf{中文:从振荡器信号生成相移波形的最简单方法是使用一些数字电路。两个D型触发器,如74AC74,可以用于将输入频率除以四,并同时产生两个精确相移%90度的输出波形。}
%
%\begin{figure}[htbp]
%\centering
%%%\includegraphics[width=0.6\textwidth]{fig8-5}
%\caption{The IQ-Mixer.}
%\end{figure}
%
%Originally, a programmable quartz oscillator based on the CY27EE16 was used as a clock oscillator. This chip is no longer %available and can be substituted with the SI5351. In principle, you could also use a tunable oscillator running at four times the %reception frequency, but the necessary stability could only be achieved after careful design.
%
%\textbf{中文:最初,基于CY27EE16的可编程石英振荡器用作时钟振荡器。该芯片不再可用,可以用SI5351代替。原则上,你也可以使用以接收频率四倍运行的可%调谐振荡器,但必要的稳定性只能在精心设计后实现。}
%
%The mixers are constructed using four analog switches contained in the 74HC4066 IC, which offer good synchronization and can %handle high signal levels with good isolation between channels. Two analog switches are controlled by the clock generator, to %produce a balanced mixer. Figure 8.6 shows an I/Q direct mixer for the frequency range from 500 kHz to about 30 MHz.
%
%\textbf{中文:混频器使用74HC4066 IC中包含的四个模拟开关构建,它们提供良好的同步性,并可以在通道间良好隔离的情况下处理高信号电平。两个模拟开关由%时钟发生器控制,以产生平衡混频器。图8.6显示了一个用于500 kHz到约30 MHz频率范围的I/Q直接混频器。}
%
%\begin{figure}[htbp]
%\centering
%%%\includegraphics[width=0.6\textwidth]{fig8-6}
%\caption{A wideband receiver.}
%\end{figure}
%
%A broadband transformer with 10:20+20 turns was wound on a small ferrite core. Simple low-pass filters are used at the mixer %outputs. The subsequent 20 dB amplifier gain improves the receiver sensitivity. Figure 8.7 shows a prototype setup of the circuit.
%
%\textbf{中文:宽带变压器具有10:20+20匝,绕制在小铁氧体磁芯上。简单的低通滤波器用于混频器输出。随后的20 dB放大器增益提高了接收机灵敏度。图8.7显%示了电路的原型设置。}
%
%\begin{figure}[htbp]
%\centering
%%%\includegraphics[width=0.6\textwidth]{fig8-7}
%\caption{Prototype layout of the complete receiver.}
%\end{figure}
%
%The I/Q mixer can achieve a spurious signal rejection of up to about 40 dB. Connecting only one of the two channels to the sound %card will produce the typical result for a simple direct mixer (Figure 8.8). A signal of 11 kHz appears both at +11 kHz and at -11 %kHz. However, with both inputs separated by a 90 degrees phase shift, the desired signal is amplified and the image signal %suppressed (Figure 8.9). A bandwidth of 48 kHz can therefore be tuned by software alone.
%
%\textbf{中文:I/Q混频器可以实现高达约40 dB的杂散信号抑制。仅将两个通道中的一个连接到声卡将产生简单直接混频器的典型结果(图8.8)。11 kHz的信号同%时出现在+11 kHz和-11 kHz处。然而,当两个输入由90度相移分隔时,所需信号被放大,镜像信号被抑制(图8.9)。因此,48 kHz的带宽可以仅由软件调谐。}
%
%\begin{figure}[htbp]
%\centering
%%%\includegraphics[width=0.6\textwidth]{fig8-8}
%\caption{Using one mixer generates an image frequency.}
%\end{figure}
%
%\begin{figure}[htbp]
%\centering
%%%\includegraphics[width=0.6\textwidth]{fig8-9}
%\caption{Using two mixers showing image frequency suppression.}
%\end{figure}
%
%This simple receiver shows surprisingly good reception results on medium wave and shortwave bands, especially high sensitivity, %and good frequency stability, as well as excellent selectivity, which is achieved solely through software.
%
%\textbf{中文:这个简单的接收机在中波和短波波段显示出令人惊讶的良好接收结果,特别是高灵敏度、良好的频率稳定性以及出色的选择性,这些完全通过软件实%现。}
%
%A fundamental weakness of the simple circuit is that the mixer is driven at the odd harmonics of the oscillator frequency. Some of %the local oscillator switching signals can leak into the output signal path to generate intermodulation products, a low-pass %filter or resonant circuit can be useful to remove them.
%
%\textbf{中文:简单电路的一个基本弱点是混频器由振荡器频率的奇次谐波驱动。一些本地振荡器开关信号可能泄漏到输出信号路径以产生互调产物,低通滤波器或%谐振电路可以用于去除它们。}
%
%\section{Circuit Optimization}
%\textbf{中文:电路优化}
%
%The IQ mixer consists of two identical mixing stages driven with a signal phase-shifted by 90 degrees. It is important for the two %signals to have exactly the same gain and excellent linearity. CMOS analog switches have proven to be effective mixers. %Originally, a programmable quartz oscillator based on the CY27EE16 served as the oscillator, which has now been replaced by the %SI5351. This allows for an operating frequency of up to about 30 MHz.
%
%\textbf{中文:IQ混频器由两个相同的混频级组成,由相移90度的信号驱动。重要的是两个信号具有完全相同的增益和出色的线性。CMOS模拟开关已被证明是有效的%混频器。最初,基于CY27EE16的可编程石英振荡器用作振荡器,现在已被SI5351取代。这允许高达30 MHz的工作频率。}
%
%The circuit uses analog switches of the 74HC4066 type and a digital divider using 2 flip flops using the 74AC74, which divides the %oscillator frequency by four and creates the necessary phase shift. To make it easy and get a head start the company Modul-Bus has %developed a board that provides the entire mixer/oscillator as a PCB module. Connections to the I and Q mixer signals are %available at the board edge.
%
%\textbf{中文:电路使用74HC4066型的模拟开关和使用74AC74两个触发器的数字分频器,它将振荡器频率除以四并创建必要的相移。为了使其简单并快速开始,%Modul-Bus公司开发了一个板,提供整个混频器/振荡器作为PCB模块。I和Q混频器信号的连接在板边缘可用。}
%
%\begin{figure}[htbp]
%\centering
%%%\includegraphics[width=0.6\textwidth]{fig8-10}
%\caption{The IQ mixer PCB.}
%\end{figure}
%
%Here, the different circuit variants will be studied in more detail. For the first attempt, an extremely simple receiver is built %with just one additional resistor. It will receive strong radio signals with a sufficiently long wire antenna.
%
%\textbf{中文:在这里,将更详细地研究不同的电路变体。对于第一次尝试,构建了一个极其简单的接收机,只有一个额外的电阻器。它将用足够长的导线天线接收%强无线电信号。}
%
%\begin{figure}[htbp]
%\centering
%%%\includegraphics[width=0.6\textwidth]{fig8-11}
%\caption{The first mixer design.}
%\end{figure}
%
%This basic receiver can be improved by using a balanced mixer. Here a small transformer with a center tap is used and achieves an %image frequency rejection of about 40 dB.
%
%\textbf{中文:这个基本接收机可以通过使用平衡混频器来改进。这里使用带有中心抽头的小型变压器,实现了约40 dB的镜像频率抑制。}
%
%\begin{figure}[htbp]
%\centering
%%%\includegraphics[width=0.6\textwidth]{fig8-12}
%\caption{Symmetrical Mixer.}
%\end{figure}
%
%In a balanced mixer the RF transformer can be a source of asymmetry. Here we will look at how to build a symmetric mixer with four %audio output channels. The signals will be combined using two differential amplifiers. The idea is to design the circuit so that %the mixer outputs all have a loading of precisely 10 kΩ. The initial circuit is shown in figure 8.13.
%
%\textbf{中文:在平衡混频器中,RF变压器可能是不对称性的来源。在这里,我们将看看如何构建具有四个音频输出通道的对称混频器。信号将使用两个差分放大器%组合。设计电路的想法是使混频器输出都具有精确10 kΩ的负载。初始电路如图8.13所示。}
%
%The 2.2 nF capacitors at the outputs of the analogue switches provide a low-pass cutoff frequency well above the band limit of 24 %kHz, which ensures that passband tolerances do not cause additional phase shifts. These simple low-pass filters are only used to %isolate high level RF signals from the operational amplifiers. The actual filtering is left to the anti-aliasing filter in the PC %sound card.
%
%\textbf{中文:模拟开关输出处的2.2 nF电容器提供远高于24 kHz频带限制的低通截止频率,这确保通带容差不会引起额外的相移。这些简单的低通滤波器仅用于%将高电平RF信号与运算放大器隔离。实际滤波留给PC声卡中的抗混叠滤波器。}
%
%\begin{figure}[htbp]
%\centering
%%%\includegraphics[width=0.6\textwidth]{fig8-13}
%\caption{Mixer without carrier.}
%\end{figure}
%
%The circuit achieves a good image rejection of more than 40 dB. The high overall gain of about 40 dB is effective for picking up %weak signals, but for extremely strong radio stations an antenna attenuator may be required. A weakness of the circuit is the low %upper cut-off frequency. Above about 12 MHz, sensitivity drops sharply. An experiment showed that the two 1 kΩ resistors in front %of the mixers are the source of the problem. If both are replaced by shorts, sensitivity is maintained above 25 MHz but this %slightly reduces the image frequency rejection.
%
%\textbf{中文:该电路实现了超过40 dB的良好镜像抑制。约40 dB的高整体增益对于拾取弱信号有效,但对于极强的无线电台,可能需要天线衰减器。电路的一个%弱点是低上限截止频率。在12 MHz以上,灵敏度急剧下降。实验表明,混频器前面的两个1 kΩ电阻器是问题的来源。如果两者都被短路替换,灵敏度在25 MHz以上保%持,但这略微降低了镜像频率抑制。}
%
%Although the circuit works relatively well, there is still room for improvement. The following detailed circuit diagram shows the %differential amplifier used. It can be seen that this stage is actually not completely symmetrical. This means that the %differential amplifier does not have high common-mode signal rejection. This does not cause problems with image rejection but may %result in intermodulation products from the RF stage below 24 kHz directly entering the IF path via the mixer.
%
%\textbf{中文:虽然电路相对良好地工作,但仍有改进空间。以下详细电路图显示了所使用的差分放大器。可以看出,这一级实际上不完全对称。这意味着差分放大%器不具有高共模信号抑制。这不会引起镜像抑制问题,但可能导致来自RF级的互调产物低于24 kHz直接通过混频器进入IF路径。}
%
%\begin{figure}[htbp]
%\centering
%%%\includegraphics[width=0.6\textwidth]{fig8-14}
%\caption{The differential amplifier configuration.}
%\end{figure}
%
%A conventional differential amplifier was tested and showed good common-mode rejection but image rejection performance was poor.
%
%\textbf{中文:测试了常规差分放大器,显示出良好的共模抑制,但镜像抑制性能较差。}
%
%\begin{figure}[htbp]
%\centering
%%%\includegraphics[width=0.6\textwidth]{fig8-15}
%\caption{A symmetrical differential amplifier.}
%\end{figure}
%
%The problem here is in the different input resistance of the two inputs, even though the circuit appears to be symmetrical at %first glance. The inverting input has an impedance of about 0.5 kΩ, while the non-inverting input has an impedance of 11 kΩ. Using %rounded voltage values in the basic amplifier configuration (Figure 8.16) and driving the circuit with +1 V and -1 V. The %non-inverting input sets the voltage at both op-amp inputs because it is a voltage divider to ground with no negative feedback. %The upper 1 kΩ resistor is between +1 V and -1 V, resulting in an input impedance of only 0.5 kΩ. This violates the most important %rule of IQ circuit technology, which is that all four phases should be equally loaded.
%
%\textbf{中文:这里的问题在于两个输入的不同输入电阻,尽管电路乍看起来是对称的。反相输入的阻抗约为0.5 kΩ,而同相输入的阻抗11 kΩ。在基本放大器配置%(图8.16)中使用舍入电压值,并用+1 V和-1 V驱动电路。同相输入设置两个运算放大器输入的电压,因为它是到地的分压器,没有负反馈。上面的1 kΩ电阻器在%+1 V和-1 V之间,导致输入阻抗仅0.5 kΩ。这违反了IQ电路技术最重要的规则,即所有四个相位应该同等负载。}
%
%\begin{figure}[htbp]
%\centering
%%%\includegraphics[width=0.6\textwidth]{fig8-16}
%\caption{Unsymmetrical input impedance.}
%\end{figure}
%
%There is a possible solution in the form of the so-called 'instrumentation amplifier'. For this, two fully differential op-amps %are used as impedance converters. The circuit appears to have the same theoretically infinite impedance at the front end. This %variant was also tested. The result was again good image rejection. However, the receiver overall showed more noise, lower %sensitivity, and more distortion.
%
%\textbf{中文:有一个可能的解决方案,即所谓的"仪表放大器"。为此,使用两个完全差分运算放大器作为阻抗转换器。电路在前端似乎具有相同的理论无限阻抗。%这个变体也被测试了。结果再次是良好的镜像抑制。然而,接收机整体显示出更多噪声、更低灵敏度和更多失真。}
%
%\begin{figure}[htbp]
%\centering
%%%\includegraphics[width=0.6\textwidth]{fig8-17}
%\caption{An instrumentation amplifier.}
%\end{figure}
%
%Theory and practice do not always go hand in hand. The reason here is that when the signal gets up to a reasonably high frequency %of 24 kHz standard operational amplifier begins to run out of steam. An LM324, for example, has a gain-bandwidth product of 1 MHz %so at 20 kHz, the open-loop gain is only about 50. The difference between both inputs will no longer be almost zero. A fully %compensated operational amplifier has to struggle to follow the input voltage. At higher frequencies, distortion occurs. Signals %above 20 kHz can be present at the mixer output, and their distortion products may partly lie in the passband. These %considerations led to the decision to reject the instrumentation amplifier for this application.
%
%\textbf{中文:理论和实践并不总是齐头并进。这里的原因是,当信号达到相当高24 kHz频率时,标准运算放大器开始力不从心。例如,LM324的增益带宽积1 %MHz,因此在20 kHz时,开环增益仅为约50。两个输入之间的差异将不再几乎为零。完全补偿的运算放大器必须努力跟随输入电压。在较高频率下,会出现失真。高%于20 kHz的信号可能存在于混频器输出处,并且它们的失真产物可能部分位于通带中。这些考虑导致决定在此应用中拒绝仪表放大器。}
%
%The experiments have clearly shown that the symmetrical input resistance of the circuit is more important than a high common-mode %rejection ratio. If the RF input circuit operates very linearly and a high-pass filter of 24 kHz is also used, there are actually %no problems with IF breakthrough. So, you returned to the original circuit using slightly different values. The effective input %resistance is now about 5 kΩ at both inputs. A series resistor of 100 Ω offers a good compromise between high RF cutoff frequency %and good decoupling between I and Q mixers. In addition, the TL084 is now used, which is not quite as noise-free but has a gain %bandwidth product of 4 MHz and can run from a simple 5 V supply voltage.
%
%\textbf{中文:实验清楚地表明,电路的对称输入电阻比高共模抑制比更重要。如果RF输入电路非常线性地工作并且也使用24 kHz的高通滤波器,那么实际上没有%IF突破的问题。因此,你回到了使用稍微不同值的原始电路。有效输入电阻现在在两个输入处都约为5 kΩ。200 Ω的串联电阻器在高RF截止频率和I和Q混频器之间的%良好去耦之间提供了良好的折衷。此外,现在使用TL084,它虽然不是完全无噪声,但具有4 MHz的增益带宽积,并且可以从简单的5 V电源电压运行。}
%
%\begin{figure}[htbp]
%\centering
%%%\includegraphics[width=0.6\textwidth]{fig8-18}
%\caption{Now with matched input impedances.}
%\end{figure}
%
%The overall circuit is not optimized for highest sensitivity, but rather for high immunity to large signals and low distortion %even with strong radio signals, including strong DRM stations. An SNR of well over 30 dB has been observed using this setup. The %high sensitivity typically found in amateur radio receiver specifications is not achieved here. Selective preamplifiers can %however be used to address these requirements as well.
%
%\textbf{中文:整体电路不是针对最高灵敏度进行优化,而是针对对大信号的高抗扰度和低失真,即使有强无线电信号,包括强DRM电台。使用此设置已观察到超过%30 dB的信噪比。这里没有达到业余无线电接收机规格中典型的高灵敏度。然而,可以使用选择性前置放大器来满足这些要求。}
%
%\begin{figure}[htbp]
%\centering
%%%\includegraphics[width=0.6\textwidth]{fig8-19}
%\caption{The optimized mixer design.}
%\end{figure}
%
%Figure 8.19 shows the latest version of the receiver. What's still missing here for a practical design are some low-pass or %band-pass filters for specific frequency bands. For example, if you want to receive medium wave without a pre-filter, shortwave %stations working at three or five times the frequency will break through. Suppressing unwanted signals in the IQ receiver is easy %because there is no image frequency to worry about, only harmonic mixing products. In principle, low-pass filters alone will do %the job.
%
%\textbf{中文:图8.19显示了接收机的最新版本。对于实际设计,这里仍然缺少一些针对特定频带的低通或带通滤波器。例如,如果你想在没有预滤波器的情况下接%收中波,工作在频率三倍或五倍的短波电台将会突破。在IQ接收机中抑制不需要的信号很容易,因为没有镜像频率需要担心,只有谐波混频产物。原则上,仅低通滤%波器就可以完成这项工作。}
%
%\begin{figure}[htbp]
%\centering
%%%\includegraphics[width=0.6\textwidth]{fig8-20}
%\caption{The populated IQ mixer PCB.}
%\end{figure}
%
%Now with the benefits of the latest optimized circuit the software-defined radio is more sensitive with better image signal %rejection. In the end, everything was fitted onto a single PCB.
%
%\textbf{中文:现在,由于最新优化电路的好处,软件定义无线电更加灵敏,具有更好的镜像信号抑制。最终,所有东西都安装在单个PCB上。}
%
%\begin{figure}[htbp]
%\centering
%%%\includegraphics[width=0.6\textwidth]{fig8-21}
%\caption{The RF front end.}
%\end{figure}
%
%At the RF front end an FET is configured as a source follower which provides a low impedence signal to drive the mixers. There are %three options for the antenna coupling: a wideband setting, a high-pass filter for shortwave or a low-pass filter for medium wave. %The low-pass filter has a cutoff frequency of around 1.6 MHz to help suppress harmonic mixing products and interference from %shortwave when receiving medium wave stations.
%
%\textbf{中文:在RF前端,FET配置为源极跟随器,它提供低阻抗信号来驱动混频器。天线耦合有三个选项:宽带设置、短波的高通滤波器或中波的低通滤波器。低%通滤波器的截止频率约1.6 MHz,以帮助抑制谐波混频产物和来自短波的干扰,当接收中波电台时。}
%
%\begin{figure}[htbp]
%\centering
%%%\includegraphics[width=0.6\textwidth]{fig8-22}
%\caption{Two-stage IQ amplifier.}
%\end{figure}
%
%The IF amplifier uses two stages to give a total gain of 100. The input is optimized for equal input impedance of all four phases, %which improves image rejection. The receiver achieves an overall image rejection of 40 dB or more.
%
%\textbf{中文:IF放大器使用两个级来提供100的总增益。输入针对所有四个相等的输入阻抗进行优化,这改善了镜像抑制。接收机实现40 dB或更高的整体镜像抑%制。}
%
%\begin{figure}[htbp]
%\centering
%%%\includegraphics[width=0.6\textwidth]{fig8-23}
%\caption{Connected to the mixer PCB.}
%\end{figure}
%
%The circuit board fits directly onto the IQ mixer. Together with the programmable quartz oscillator, it forms a complete receiver %with tuning via the serial interface. Power is now supplied from the right-hand board via a voltage regulator. The screw terminals %blocks on the IQ mixer are unused but 5 V can be taken from here to power any additional circuits.
%
%\textbf{中文:电路板直接安装在IQ混频器上。与可编程石英振荡器一起,它形成了一个完整的接收机,通过串行接口进行调谐。电源现在通过电压调节器从右侧板%供应。IQ混频器上的螺丝端子块未使用,但可以从此处获取5 V来为任何附加电路供电。}
%
%\begin{figure}[htbp]
%\centering
%%%\includegraphics[width=0.6\textwidth]{fig8-24}
%\caption{An alternative oscillator.}
%\end{figure}
%
%The tried and tested receiver design was now also tested with a different VFO. The ICS307-2 programmable oscillator is powered via %the 2-way 5 V power terminal block on the mixer board. Its output clock signal is connected using a small length of wire to the %appropriate socket position on the DIL socket.
%
%\textbf{中文:经过尝试和测试的接收机设计现在也用不同的VFO进行了测试。ICS307-2可编程振荡器通过混频器板上的2路5 V电源端子块供电。其输出时钟信号使%用一小段导线连接到DIL插座上适当的插座位置。}
%
%\begin{figure}[htbp]
%\centering
%%%\includegraphics[width=0.6\textwidth]{fig8-25}
%\caption{Oscillator setup in software.}
%\end{figure}
%
%To tune the receiver to a specific frequency, you can use the ICS703-2.exe program, which communicates with the receiver through %the COM1 serial port. For example, to receive an AM station at 6155 kHz, you would set the receiver oscillator to 6145 kHz. This %allows you to receive the station using SDRadio which gives excellent audio quality.
%
%\textbf{中文:要将接收机调谐到特定频率,您可以使用ICS703-2.exe程序,该程序通过COM1串行端口与接收机通信。例如,要接收6155 kHz的AM电台,您可以将%接收机振荡器设置为6145 kHz。这允许您使用SDRadio接收电台,该软件提供出色的音频质量。}
%
%\begin{figure}[htbp]
%\centering
%%%\includegraphics[width=0.6\textwidth]{fig8-26}
%\caption{AM station reception in SDRadio.}
%\end{figure}
%
%\section{Software Defined Radio with USB Interface}
%\textbf{中文:具有USB接口的软件定义无线电}
%
%Back in 2007, Elektor Magazine developed and produced this receiver board based on earlier explorations into emerging field of %digital radio. The aim of project is to provide beginners with easy access to topic of Software Defined Radio.
%
%\textbf{中文:早在2007年,Elektor杂志基于对数字无线电新兴领域的早期探索开发并生产了这个接收机板。该项目的目标是为初学者提供对软件定义无线电主题%的轻松访问。}
%
%A Software Defined Radio (SDR) requires little hardware but sophisticated software. This SDR project aimed to show what is %achievable with minimal effort. The goal was to create a universal receiver working from 150 kHz to 30 MHz, optimized for DRM and %AM reception, but also allowing reception of amateur radio bands.
%
%\textbf{中文:软件定义无线电(SDR)需要很少的硬件但复杂的软件。这个SDR项目旨在展示用最小的努力可以实现什么。目标是创建一个从150 kHz到30 MHz工%作的通用接收机,针对DRM和AM接收进行优化,但也允许业余无线电波段的接收。}
%
%The objective of this project was to create a receiver with medium sensitivity, but with the highest linearity and phase purity. %The development focused on properties that are important for a top-notch DRM receiver. In fact, the receiver achieves an excellent %signal to noise performance.
%
%\textbf{中文:该项目的目标是创建一个具有中等灵敏度的接收机,但具有最高的线性和相位纯度。开发专注于对顶级DRM接收机重要的属性。事实上,接收机实现%了出色的信噪性能。}
%
%\begin{figure}[htbp]
%\centering
%%%\includegraphics[width=0.6\textwidth]{fig8-27}
%\caption{The Elektor Software Defined Radio with USB port.}
%\end{figure}
%
%Regarding sensitivity and overload resistance, the receiver can't compete with top of the range amateur radio equipment. %Experiments have however shown that on the lower bands up to 20 meters, the atmospheric noise is usually so strong that greater %sensitivity is not really an advantage. A comparison with an older Yaesu FT-7B rig showed about the same results when receiving CW %and SSB stations on the 80, 40, and 20-meter bands using the same antenna. However, the SDR scored points with its advanced %software capabilities. Features such as continuously adjustable bandwidth and spectrum display are otherwise only available in %much more expensive receivers.
%
%\textbf{中文:关于灵敏度和过载电阻,接收机无法与顶级业余无线电设备竞争。然而,实验表明,在低至20米的波段上,大气噪声通常如此强烈,以至于更高的灵%敏度并不是真正的优势。与较旧的Yaesu FT-7B设备的比较显示,在使用相同天线接收80、40和20米波段的CW和SSB电台时,结果大致相同。然而,SDR凭借其先进的%软件功能获得了优势。诸如连续可调带宽和频谱显示等功能通常只在更昂贵的接收机中可用。}
%
%The receiver is controlled via USB from where it also sources its power. No additional power supply is needed. The FT232R was %chosen as the USB interface. This modern USB-to-serial converter does not require a quartz crystal because it has an internal %high-precision RC oscillator. The component is used here in its bit-bang mode, like a fast parallel port. Eight data lines are %available and can be controlled as desired. Two of the signals are used as I²C bus to control the receiver frequency. Three %signals are used to switch the input multiplexer to select one of eight antenna inputs with and without filters. Two more inputs %are used to switch the receiver's IF amplifier gain. This control interface ensures the receiver can be fully managed via software.
%
%\textbf{中文:接收机通过USB从其也获取电源的地方进行控制。不需要额外的电源。FT232R被选为USB接口。这种现代USB到串行转换器不需要石英晶体,因为它具%有内部高精度RC振荡器。该组件在这里以其位脉冲模式使用,就像快速并行端口。八条数据线可用,可以按需控制。其中两个信号用作I²C总线来控制接收机频率。三%个信号用于切换输入多路复用器,以选择八个天线输入中的一个,有或没有滤波器。还有两个输入用于切换接收机的IF放大器增益。此控制接口确保接收机可以通过%软件完全管理。}
%
%\begin{figure}[htbp]
%\centering
%%%\includegraphics[width=0.6\textwidth]{fig8-28}
%\caption{The receiver schematic in brilliant Elektor style.}
%\end{figure}
%
%Special attention was paid to decoupling the power supply. One of the reasons for this is that the FT232RL USB chip uses internal %clock signals which are at frequencies that will also be received via the antenna. You don't want any of this unwanted RF noise to %leak across from one stage to another. The FT232R contains exactly what the programmable clock generator CY27EE16 needs with a %voltage regulator of 3.3 V. Therefore, no additional voltage regulator is needed. The remaining part of the circuit operates at 5 %V, several supplies are provided to power specific functions on the board with appropriate decoupling to reduce crosstalk and %noise. Keep in mind that the 5 V from the USB ultimately comes from a PC power supply. The same power supply powers the entire PC, %and load changes can cause some noise on the supply line. This is particularly critical for the RF preamplifier of the receiver, %which ultimately couples via the mixers to the IF branch. Therefore, a large capacitor provides stability at this point (VCC\_HF).
%
%\textbf{中文:特别注意电源的去耦。这样做的原因之一是FT232RL USB芯片使用内部时钟信号,这些信号的频率也将通过天线接收。您不希望任何这种不需要的RF%噪声从一个阶段泄漏到另一个阶段。FT232R恰好包含可编程时钟发生器CY27EE16所需的3.3 V电压调节器。因此,不需要额外的电压调节器。电路的其余部分5 V下%工作,提供了几个电源来为板上的特定功能供电,并具有适当的去耦以减少串扰和噪声。请记住,来自USB5 V最终来自PC电源。同一个电源为整个PC供电,负载变化%可能会在电源线上引起一些噪声。这对于接收机的RF前置放大器特别关键,它最终通过混频器耦合到IF分支。因此,一个大电容器在此点(VCC\_HF)提供稳定性。}
%
%The SDR requires an oscillator frequency that is four times higher than the signal received so that it can be divided by four with %the required phase shift. If you are aiming to receive signals up to 30 MHz, the oscillator needs to run up to 120 MHz. DDS %oscillators are often used in advanced RF projects but a DDS able to run at this speed will work out expensive, power-hungry, and %difficult to control. For this reason a programmable clock oscillator with an internal PLL is used here. Although the CY27EE16 was %originally designed as a clock oscillator for digital electronics and processors, it has proven itself in many RF applications. %Although the achievable frequency resolution is not as good as a DDS, the relatively modest power consumption is important for %this project, as you cannot draw too much power from the standard USB port.
%
%\textbf{中文:SDR需要一个比接收信号高四倍的振荡器频率,以便可以将其除以四并获得所需的相位偏移。如果您打算接收高达30 MHz的信号,振荡器需要运行到%120 MHz。DDS振荡器经常在高级RF项目中使用,但能够以这种速度运行的DDS将变得昂贵、功耗大且难以控制。因此,这里使用了一个具有内部PLL的可编程时钟振%荡器。虽然CY27EE16最初是为数字电子和处理器设计的时钟振荡器,但它在许多RF应用中证明了自己。虽然可实现的频率分辨率不如DDS,但相对适中的功耗对于这%个项目很重要,因为您不能从标准USB端口汲取太多功率。}
%
%The chip is programmed via the I²C bus using the SCL and SDA lines. A VCO operates internally in the frequency range of 100 to 400 %MHz. The VCO is stabilized using a 10 MHz crystal and a PLL. Its output signal is passed via dividers to reach the desired %outputs. The clock output Clock5 was chosen here. There is a VFO signal between 600 kHz and 120 MHz that passes to the 74AC74 %chain of dividers.
%
%\textbf{中文:该芯片通过I²C总线使用SCL和SDA线进行编程。VCO在100到400 MHz的频率范围内内部运行。VCO使用10 MHz晶体和PLL进行稳定。其输出信号通过%分频器传递以达到所需的输出。这里选择了时钟输出Clock5。有600 kHz到120 MHz之间的VFO信号传递到74AC74分频器链。}
%
%The exact phase shift of 90 degrees between the two oscillator signals is important. Deviations lead to less effective suppression %of image frequencies. Since the divider 74AC74 is connected as a synchronous divider, you would not expect to find any phase error %here. In fact, the receiver shows a constant mirror suppression of about 40 dB up to about 15 MHz. From about 20 MHz, this value %decreases noticeably, but this is tolerable due to the lower occupancy in this frequency range.
%
%\textbf{中文:两个振荡器信号之间精确90度相位偏移很重要。偏差导致对镜像频率的抑制效果较差。由于分频器74AC74连接为同步分频器,您不会期望在这里发现%任何相位误差。事实上,接收机在大约15 MHz以下显示出大约40 dB的恒定镜像抑制。从大约20 MHz开始,这个值明显下降,但由于该频率范围内的占用率较低,这%是可以接受的。}
%
%The receiver has several inputs that are switched via the input multiplexer 74HC4051. The antenna input Ant1 is fed by way of %filters to the first three inputs. The first switch position (wide) uses only an input choke, to short any low-frequency signals %at the input to ground. In the second position (medium wave), there is a low-pass filter with a cut-off frequency of 1.6 MHz, %where the resistor R12 dampens a resonance peak. This filter prevents medium wave reception from being disturbed by harmonic %mixing with stations in the shortwave range. The third position uses a simple RC high-pass filter, which is intended to attenuate %strong medium wave signals.
%
%\textbf{中文:接收机有几个输入,通过输入多路复用器74HC4051进行切换。天线输入Ant1通过滤波器馈送到前三个输入。第一个开关位置(宽频)仅使用输入扼%流圈,将输入处的任何低频信号短路到地。在第二个位置(中波),有一个截止频率为1.6 MHz的低通滤波器,其中电阻R12抑制谐振峰值。该滤波器防止中波接收被%与短波范围内的电台的谐波混合所干扰。第三个位置使用简单的RC高通滤波器,旨在衰减强的中波信号。}
%
%Another input (PC1) can be selected if you wish to connect external tuned input circuits or preamplifiers. Three more inputs are %provided for future expansion. The input filters on the board can be regarded as a kind of basic equipment that is sufficient in %most cases. However, it would be possible to add further steep low-pass filters or special band-pass filters, which would safely %attenuate harmonic mixing components in all situations.
%
%\textbf{中文:如果您希望连接外部调谐输入电路或前置放大器,可以选择另一个输入(PC1)。还提供了三个输入供将来扩展。板上的输入滤波器可以被视为一种%基本设备,在大多数情况下是足够的。然而,可以添加更陡峭的低通滤波器或特殊的带通滤波器,这将在所有情况下安全地衰减谐波混合成分。}
%
%From the input multiplexer, the RF signal goes to a BF245C JFET which functions as an impedence converter. The input is relatively %high-impedance at 100 kohms, so, for example, a high-Q resonant circuit can also be connected to the input In2. At the %low-impedance output of the source follower, a voltage of about 2.5 V is established, which is passed through the mixers and the %following operational amplifiers to the output. Therefore, it is important that there are no low-frequency signal residues at the %source. For example, the purity of the supply voltage Vcc\_HF is critical and therefore has a high level of filtering. The FET %itself provides additional decoupling of the supply voltage. But nothing should come from its gate that reaches the IF range below %24 kHz. For this reason an RF choke is placed directly at the antenna input to shunt any 50 Hz (60 Hz) hum signals, for example.
%
%\textbf{中文:从输入多路复用器,RF信号进入BF245C JFET,它作为阻抗转换器。输入在100 kohms时相对高阻抗,因此,例如,高Q谐振电路也可以连接到输入%In2。在源极跟随器的低阻抗输出处,建立了大约2.5 V的电压,该电压通过混频器和随后的运算放大器传递到输出。因此,重要的是源极没有低频信号残留。例如,%电源电压Vcc\_HF的纯度是关键的,因此具有高水平的滤波。FET本身提供了电源电压的额外去耦。但是,不应该有任何来自其栅极的信号到达24 kHz以下的IF范%围。因此,RF扼流圈直接放置在天线输入处,以分流任何50 Hz(60 Hz)嗡嗡声信号。}
%
%The IF amplifier consists of two exactly equal branches, each of which provides a total gain of up to 40 dB. The TL084 was chosen %here because it has a good gain-bandwidth product of 10 MHz at a supply voltage of 5 V. This is important in order to supply a %gain of 10 without phase errors to signals at around 20 kHz.
%
%\textbf{中文:IF放大器由两个完全相同的分支组成,每个分支提供高达40 dB的总增益。这里选择了TL084,因为它在5 V电源电压下具有良好的10 MHz增益带宽%积。这对于20 kHz左右的信号提供10的增益而没有相位误差很重要。}
%
%The final stage has a gain of 10 (20 dB), but this can be reduced to unity gain via the analog switches. A total of three %attenuation steps are available: 0 dB, -10 dB, and -20 dB. So, if excessively strong signals lead to overload, the gain can be %reduced by software. The attenuator is not located at the input of the receiver because there is already high overload resistance %built in. On the other hand, at full gain, with a long antenna and high field strengths, the output can be over driven. The %attenuation therefore applies to the output driver stage and corresponds approximately to the gain control in an IF amplifier.
%
%\textbf{中文:最后一级具有10(20 dB)的增益,但这可以通过模拟开关降低到单位增益。总共提供三个衰减步骤:0 dB、-10 dB和-20 dB。因此,如果过强的%信号导致过载,可以通过软件降低增益。衰减器不位于接收机的输入端,因为已经内置了高过载电阻。另一方面,在满增益下,使用长天线和高场强,输出可能会过%驱动。因此,衰减适用于输出驱动级,大约对应于IF放大器中的增益控制。}
%
%\begin{figure}[htbp]
%\centering
%%%\includegraphics[width=0.6\textwidth]{fig8-29}
%\caption{The receive spectrum obtained with G8JCFSDR.}
%\end{figure}
%
%\section{Arduino SDR Shield}
%\textbf{中文:Arduino SDR屏蔽板}
%
%Elektor built its first Software Defined Radio with a USB interface back in 2007, using a conventional board and only a few SMD %components. Since then, there have been thoughts about updating the design and when the PLL chip used in the original was phased %out it was time to find a new solution. The original concept was recreated using the SI5351 Silicon Labs PLL chip which is a CMOS %clock generator with an I2C interface that generates signals ranging from 8 kHz to 160 MHz.
%
%\textbf{中文:Elektor早在2007年就使用传统板和少量SMD组件构建了其第一个带有USB接口的软件定义无线电。从那时起,一直在考虑更新设计,当原版中使用的%PLL芯片被淘汰时,是时候找到新的解决方案了。使用SI5351 Silicon Labs PLL芯片重新创建了原始概念,该芯片是一个CMOS时钟发生器,具有I2C接口,可生成%8 kHz到160 MHz范围内的信号。}
%
%Initial tests with this new chip using a breakout board from Adafruit (Section 5.10) were successful. The existing software %examples were written for the Arduino, so the first steps were taken towards the Arduino environment. The new VFO was connected to %the existing SDR board, and it proved to be functional. Then came the question; why not build the entire receiver as an Arduino %shield? That way the power supply and USB interface would be taken care of.
%
%\textbf{中文:使用来自Adafruit的转接板(第5.10节)对这个新芯片的初步测试是成功的。现有的软件示例是为Arduino编写的,因此向Arduino环境迈出了第一%步。新的VFO连接到现有的SDR板,并被证明是功能性的。然后问题来了;为什么不将整个接收机构建为Arduino屏蔽板?这样电源和USB接口就可以得到处理。}
%
%\begin{figure}[htbp]
%\centering
%%%\includegraphics[width=0.6\textwidth]{fig8-30}
%\caption{The SDR Shield with an Arduino Uno.}
%\end{figure}
%
%The current version of the receiver, which was last updated in 2019, is delivered as a fully assembled board with included header %strips. You still need to solder the headers onto the board so that the SDR board can be plugged onto the Arduino. Once that's %done you will need to install some software, which can be found on the Elektor website and the author's homepage. Finally to make %it all work together you can use an audio cable to establish a connection to your PC's sound card, attach an antenna, and you're %good to go.
%
%\textbf{中文:接收机的当前版本最后一次更新于2019年,作为完全组装的板交付,包含排针。您仍然需要将排针焊接到板上,以便SDR板可以插入到Arduino上。%完成后,您需要安装一些软件,可以在Elektor网站和作者的主页上找到。最后,为了使其全部协同工作,您可以使用音频电缆建立与PC声卡的连接,连接天线,就%可以开始了。}
%
%\begin{figure}[htbp]
%\centering
%%%\includegraphics[width=0.6\textwidth]{fig8-31}
%\caption{The SDR shield and its header-socket strips.}
%\end{figure}
%
%The Arduino itself doesn't really have much to do - it receives the desired frequency from the PC and adjusts the VFO as required. %This means that there is even a real chance to build a standalone receiver, as the Arduino can handle the tuning all by itself. %This opens up unlimited possibilities, especially since the Arduino is widely used and many people can program it.
%
%\textbf{中文:Arduino本身并没有太多事情要做——它从PC接收所需的频率并根据需要调整VFO。这意味着甚至有机会构建一个独立的接收机,因为Arduino可以独%自处理调谐。这开辟了无限的可能性,特别是因为Arduino被广泛使用,许多人可以对其进行编程。}
%
%\begin{figure}[htbp]
%\centering
%%%\includegraphics[width=0.6\textwidth]{fig8-32}
%\caption{The shield schematic.}
%\end{figure}
%
%Looking at the circuit diagram (Figure 8.34), you can see the individual components. The SI5351 chip provides the oscillator %signal which is tuned to four times the desired receive station frequency. Two D-type flip flops type 74AC74 (IC2B) provide a %divide-by-4 function to produce two clocks at the desired frequency with a 90 degree phase shift difference. These two clock %signals are used to control the analog switches in the 74HC4066 (IC3) which functions as a mixer. It alternately connects the RF %signal to the inverting and non-inverting inputs of the TS914 operational amplifier (IC4), downconverting the signal to recover %the baseband signal. After minor filtering and amplification, the signal goes to the audio output. The RF input stage is a source %follower with the BF545B JFET, the SMD equivalent of the BF245B.
%
%\textbf{中文:查看电路图(图8.34),您可以看到各个组件。SI5351芯片提供振荡器信号,该信号被调谐到所需接收电台频率的四倍。两个D型触发器类型74AC74%(IC2B)提供除以4的功能,以在所需频率下产生两个具有90度相位偏移差的时钟。这两个时钟信号用于控制74HC4066(IC3)中的模拟开关,该开关作为混频器起%作用。它交替地将RF信号连接到TS914运算放大器(IC4)的反相和非反相输入,下变频信号以恢复基带信号。经过轻微的滤波和放大后,信号进入音频输出。RF输入%级是带有BF545B JFET的源极跟随器,这是BF245B的SMD等效器件。}
%
%The input is broadband and protected against overvoltage by two limiting diodes, which is sufficient for shortwave reception with %a wire antenna. The overvoltage protection is based on the experience that input stages can be damaged during a thunderstorm. For %critical tasks, external filters and preamplifiers can still be used.
%
%\textbf{中文:输入是宽带,并通过两个限流二极管防止过压,这对于使用线天线的短波接收是足够的。过压保护基于输入级可能在雷暴期间损坏的经验。对于关键%任务,仍然可以使用外部滤波器和前置放大器。}
%
%The latest version, V2\_0, is technically identical but has additional connection points for additional PLL outputs and DC-coupled %signal outputs. This expansion facilitates experimental use of the shield and simplifies external expansions for measuring %instruments or shortwave transceivers.
%
%\textbf{中文:最新版本V2\_0在技术上相同,但具有额外的连接点用于额外的PLL输出和直流耦合信号输出。这种扩展促进了屏蔽板的实验使用,并简化了测量仪器%或短波收发器的外部扩展。}
%
%To use the receiver, you need a USB connection to the Arduino, an audio cable to the PC sound card and a suitable antenna. %Additionally, an Arduino sketch must be loaded to set the VFO chip to the desired frequency.
%
%\textbf{中文:要使用接收机,您需要与Arduino的USB连接,与PC声卡的音频电缆以及合适的天线。此外,必须加载Arduino草图以将VFO芯片设置为所需的频%率。}
%
%\begin{figure}[htbp]
%\centering
%%%\includegraphics[width=0.6\textwidth]{fig8-33}
%\caption{The receiver with LC-Display mounted.}
%\end{figure}
%
%The shield is designed to be used together with the Elektor LCD shield. It can be used to display the current frequency and is %also useful for measuring purposes or standalone applications.
%
%\begin{figure}[htbp]
%\centering
%%%\includegraphics[width=0.6\textwidth]{fig8-34}
%\caption{The received station displayed using SDR\#.}
%\end{figure}
%
%\chapter{Shortwave Antenna Design}
%\textbf{中文:短波天线设计}
%
%When starting out building a simple crystal radio or an SDR receiver, it's often enough to use a simple stick antenna or even a %test lead on the lab bench to act as an aerial. For better reception try something like a 5 m length of wire hung around the room %or even just resting on the floor. However, with this type of layout, the antenna will also pick up any background electrical %noise from domestic appliances and interference from the mains power network. For indoor use, antennas sensitive to the magnetic %field have proven to be more effective. With a good outdoor antenna, especially on shortwave, you can achieve so much more. The %special appeal of listening in to the shortwave bands lies in the chance of picking up broadcasts from great distances (DX %reception) thanks to sky bounce.
%
%\textbf{中文:在开始构建简单的晶体收音机或SDR接收机时,通常足以使用简单的棒状天线,甚至实验室工作台上的测试引线作为天线。为了获得更好的接收,可%以尝试在房间周围悬挂5米长的电线,甚至只是放在地板上。然而,使用这种类型的布局,天线也会拾取来自家用电器的背景电噪声和来自电力网络的干扰。对于室内%使用,对磁场敏感的天线已被证明更有效。使用良好的室外天线,特别是在短波上,您可以获得更多。收听短波波段的特殊吸引力在于由于天波反射,有机会接收到%来自远距离的广播(DX接收)。}
%
%\section{Radio Wave Propagation}
%\textbf{中文:无线电波传播}
%
%In the VHF radio frequency band, radio waves behave similar to light waves and this quasi-optical propagation pattern limits their %reception range to about 100 km depending on antenna height. However, radio waves below 30 MHz behave completely differently and %allow for a much greater range. Nevertheless, the complex propagation mechanisms in this range also lead to special problems such %as dependence on the time of day, fluctuating field strength (fading) and selective fading.
%
%\textbf{中文:在VHF射频波段,无线电波的行为类似于光波,这种准光学传播模式根据天线高度将其接收范围限制在大约100公里。然而,30 MHz以下的无线电波%的行为完全不同,并允许更大的范围。尽管如此,该范围内的复杂传播机制也导致特殊问题,例如对一天中时间的依赖、波动的场强(衰落)和选择性衰落。}
%
%The crucial role in the propagation of shortwave radio signals is played by ionized, weakly conducting layers of air in the upper %atmosphere, created by solar particle and gamma radiation that ionize the air molecules, stripping electrons from them. These free %electrons act like a mirror to certain frequency bands and under certain RF wave angles of incidence.
%
%The ionosphere is, however, transparent to high angles of incidence and high frequencies.
%
%\textbf{中文:短波无线电信号传播中的关键作用是由上层大气中的电离、弱导电空气层发挥的,这些空气层由太阳粒子和伽马辐射产生,这些辐射电离空气分子,%从中剥离电子。这些自由电子对某些频带和某些射频波入射角起着镜子的作用。然而,电离层对高入射角和高频率是透明的。}
%
%\begin{figure}[htbp]
%\centering
%%%\includegraphics[width=0.6\textwidth]{fig9-1}
%\caption{Skywave and dead zones affecting shortwave propagation.}
%\end{figure}
%
%In shortwave radio communication, the range of a transmitter can be limited to about 30 to 100 km by ground wave propagation, %depending on the height of the antenna. Beyond this distance, the signal disappears over the horizon and direct line of sight %communication is not possible. However, at a certain minimum distance, waves reflected by the ionosphere can reach the receiver %(skywave propagation). There is a dead zone between the ground wave range and the reflected wave range, where neither wave can be %received. This dead zone varies for each frequency and depends on the time of day and level of solar activity.
%
%\textbf{中文:在短波无线电通信中,发射机的范围可以通过地波传播限制在大约30到100公里,具体取决于天线的高度。超过这个距离,信号消失在地平线之外,%直接视线通信是不可能的。然而,在一定的最小距离处,由电离层反射的波可以到达接收机(天波传播)。在地波范围和反射波范围之间存在一个死区,两个波都无%法被接收。这个死区对于每个频率都不同,并且取决于一天中的时间和太阳活动的水平。}
%
%Higher frequencies allow for a flatter reflection, resulting in longer ranges. A dead zone is therefore larger and extends up to %about 200 km at 6 MHz and up to about 1000 km at 15 MHz during the day. At night, dead zones expand and the range increases. This %can lead to a situation where a specific transmitter is heard clearly in the evening, but suddenly drops out, having entered a %dead zone. Listeners may be able to overcome this by switching to a lower frequency band if the same program is broadcast on %multiple bands.
%
%\textbf{中文:较高的频率允许更平坦的反射,导致更长的范围。因此,死区更大,在白天6 MHz时延伸到大约200公里,在15 MHz时延伸到大约1000公里。在夜%间,死区扩大,范围增加。这可能导致一种情况,即在晚上清楚地听到特定的发射机,但突然消失,因为进入了死区。如果同一节目在多个频段广播,听众可以通过%切换到较低的频段来克服这个问题。}
%
%Most of the time, radio waves reach the receiver through multiple paths. These paths have different lengths, causing phase %differences that can result in partial amplification or cancellation of the signal. In the shortwave range, field strength can %fluctuate rapidly, leading to selective fading, which can cause unpleasant distortion in AM radio broadcasts.
%
%\textbf{中文:大多数时候,无线电波通过多条路径到达接收机。这些路径具有不同的长度,导致相位差异,这可能导致信号的部分放大或抵消。在短波范围内,场%强可以快速波动,导致选择性衰落,这可能在AM无线电广播中引起不愉快的失真。}
%
%However, DRM (digital radio mondiale) is more robust against partial data loss caused by fading. Despite the deep notches in the %DRM spectrum caused by the cancellation of certain frequencies, the reception is usually not disturbed thanks to effective error %correction.
%
%\textbf{中文:然而,DRM(数字广播)对由衰落引起的部分数据丢失更加稳健。尽管由于某些频率的抵消,DRM频谱中存在深陷波,但由于有效的纠错,接收通常不%会受到干扰。}
%
%\begin{figure}[htbp]
%\centering
%%%\includegraphics[width=0.6\textwidth]{fig9-2}
%\caption{Selective Fading.}
%\end{figure}
%
%\section{The Longwire Antenna}
%\textbf{中文:长线天线}
%
%If you are only interested in listening to strong local shortwave broadcasts all that's necessary is a short whip antenna less %than a meter in length. Under favorable conditions, you can test this on the bench by using a length of cable as the antenna. For %long-distance (DX) reception however, a long wire antenna outside the house in free space is a better bet. More important than the %shape of the antenna is its elevated position far enough away from houses to avoid domestic electrical interference.
%
%\textbf{中文:如果您只对收听强的本地短波广播感兴趣,那么所需的就是一根长度不到一米的短鞭状天线。在有利条件下,您可以在工作台上使用一段电缆作为天%线进行测试。然而,对于远距离(DX)接收,房屋外自由空间中的长线天线是一个更好的选择。比天线形状更重要的是其高架位置,距离房屋足够远以避免家用电气%干扰。}
%
%Suspended long wire antennas are a good solution for shortwave reception. In theory, resonance occurs at a quarter wavelength of %the received radio signal; a good ground wire to the antenna can help to reduce the effects of unwanted noise and interference and %can also improve the antenna's efficiency by providing a more stable and consistent ground reference. In practice, wire antennas %of around 10 meters in length usually give good results. If the receiver is close to a window or an outer wall of your house, all %you need is to connect the end of the antenna directly to the center pole of the coaxial aerial connector of your receiver. If the %antenna feed needs to travel a longer distance inside the house, a coaxial cable should be used, and a ground connection should be %made near the antenna feed point. It doesn't matter whether a 50-ohm or 75-ohm cable is used since the antenna has a variable %characteristic impedance depending on the receive frequency and a complex impedance with varying capacitive and inductive %components. The length of coax also has its own resonances which affect the overall antenna impedance, so that other resonances %can occur outside those predicted by the antenna length itself. However, this is barely noticeable at the receiver because signal %differences of around 10 dB are hardly significant.
%
%\textbf{中文:悬挂的长线天线是短波接收的良好解决方案。理论上,谐振发生在接收无线电信号的四分之一波长处;天线良好的接地线可以帮助减少不需要的噪声%和干扰的影响,并通过提供更稳定和一致的接地参考来提高天线的效率。实际上,长度约10米的线天线通常提供良好的结果。如果接收机靠近窗户或房屋的外墙,您%只需要将天线的末端直接连接到接收机的同轴天线连接器的中心极。如果天线馈线需要在房屋内传输更长的距离,应该使用同轴电缆,并且应该在天线馈点附近建立%接地连接。使用50欧姆还是75欧姆电缆并不重要,因为天线具有取决于接收频率的可变特性阻抗,以及具有变化的电容和电感分量的复杂阻抗。同轴电缆的长度也有%其自己的谐振,这会影响整体天线阻抗,因此在天线长度本身预测的谐振之外可能会发生其他谐振。然而,这在接收机处几乎不可察觉,因为大约10 dB的信号差异几%乎不重要。}
%
%\begin{figure}[htbp]
%\centering
%%%\includegraphics[width=0.6\textwidth]{fig9-3}
%\caption{A longwire antenna using a coaxial cable feed.}
%\end{figure}
%
%When planning a longwire antenna, it's common to use copper wire with a decent sized cross-section to achieve both good mechanical %stability and low ohmic losses. A good option is to use the type of cable used for standard mains wiring with a cross-section of 0.%75 mm² to 1.5 mm², but thinner wires can also be used. For example, a test with thin coil wire with a diameter of only 0.3 mm %produced usable results when stretched about 10 meters outdoors and then another 10 meters inside an apartment. Indoors it was %rigged above head height to remain reasonably inconspicuous. The additional section running inside the building picks up local %interference and adds some additional signal attenuation. Despite this, a usable makeshift antenna was created and remained %unnoticed and almost invisible while still pulling in far flung stations.
%
%\textbf{中文:在规划长线天线时,通常使用具有相当大截面的铜线以实现良好的机械稳定性和低欧姆损耗。一个好的选择是使用用于标准电源布线的电缆,截面积%0.75 mm²到1.5 mm²,但也可以使用更细的电线。例如,使用直径仅为0.3 mm的细线圈电线进行的测试在户外拉伸10米然后在公寓内再拉伸10米时产生了可用的结%果。在室内,它被安装在头部高度以上以保持相当不显眼。在建筑物内运行的额外部分拾取本地干扰并增加一些额外的信号衰减。尽管如此,创建了一个可用的临时%天线,并保持未被注意且几乎不可见,同时仍然接收远处的电台。}
%
%If building your own antenna seems daunting, you might be able to reuse some existing installations and cables. A typical rooftop %antenna installation delivers not only TV and FM signals, but also the entire AM range from longwave to shortwave. It's worth %trying to see what can be received. In many cases, a rooftop antenna provides better results than an indoor antenna. Often, old %antennas are no longer in use, or have been swept off the roof by a passing gale. In this case, the coax feed may still be in %place; just by shorting the outer shield together with the inner core you now have a useful vertical antenna. The cable usually %leads to the roof of the house, providing greater height than a horizontal longwire antenna. Better results can be achieved with %this set up, especially on higher frequencies above 15 MHz, than with a longwire antenna.
%
%\textbf{中文:如果您觉得构建自己的天线令人生畏,您可能能够重用一些现有的安装和电缆。典型的屋顶天线安装不仅提供电视和FM信号,还提供从长波到短波的%整个AM范围。值得尝试看看可以接收什么。在许多情况下,屋顶天线比室内天线提供更好的结果。通常,旧天线不再使用,或者被经过的大风从屋顶吹走。在这种情%况下,同轴馈线可能仍然存在;只需将外屏蔽层与内芯短路,您就有了一个有用的垂直天线。电缆通常通向房屋的屋顶,提供比水平长线天线更大的高度。使用这种%设置可以获得比长线天线更好的结果,特别是在15 MHz以上的较高频率上。}
%
%\section{Using a Preselector}
%\textbf{中文:使用预选器}
%
%In many cases, the performance of a receiver can be improved by adding a tuned preselector circuit between the aerial and front %end of the receiver. This can often prevent overload caused by strong nearby signals outside the reception band. Whilst %preselection is not always necessary, it can be a good solution in some specific situations. In amateur radio or shortwave %listening, antenna matching devices are used that provide both optimal impedance matching and some selectivity. This often results %in a significant attenuation of the unwanted image signal.
%
%\textbf{中文:在许多情况下,可以通过在天线和接收机前端之间添加调谐预选器电路来改善接收机的性能。这通常可以防止由接收频段外的强附近信号引起的过%载。虽然预选并不总是必要的,但在某些特定情况下它可能是一个好的解决方案。在业余无线电或短波收听中,使用天线匹配设备,这些设备提供最佳阻抗匹配和一%些选择性。这通常导致不需要的镜像信号的显著衰减。}
%
%A simple solution for the 49-meter band is to use a 6.0 MHz ceramic bandpass IF filter type SFE 6, which was originally used in %the audio carrier path of television sets. Although its nominal -3 dB bandwidth of approximately 100 kHz is somewhat narrow and %its 600 ohms impedance is not optimal, it still works well. The low impedance of the antenna and receiver input flattens the %filter response. The -6 dB corner frequencies were measured at 5850 kHz and 6150 kHz, allowing all the important frequencies in %the 49-meter band to pass through. Figure 9.4 shows the filter at the receiver input. A bypass switch allows for easy comparison %of results with and without the filter. Additionally, the receiver can be easily switched to a wider bandwidth when receiving %stations outside the 49-meter band.
%
%\textbf{中文:对49米波段的一个简单解决方案是使用6.0 MHz陶瓷带通IF滤波器类型SFE 6,该滤波器最初用于电视机的音频载波路径。虽然其大约100 kHz的标%称-3 dB带宽有些窄,并且600欧姆阻抗不是最佳的,但它仍然工作良好。天线和接收机输入的低阻抗使滤波器响应平坦。-6 dB拐角频率在5850 kHz和6150 kHz处%测量,允许49米波段的所有重要频率通过。图9.4显示了接收机输入处的滤波器。旁路开关允许轻松比较有滤波器和没有滤波器的结果。此外,当接收49米波段外的电%台时,接收机可以轻松切换到更宽的带宽。}
%
%\begin{figure}[htbp]
%\centering
%%%\includegraphics[width=0.6\textwidth]{fig9-4}
%\caption{Using a ceramic IF filter.}
%\end{figure}
%
%To build a preselector for use with other shortwave bands as well, is best to use an adjustable resonant circuit. For this you can %start by winding an air core coil made up of 20 turns wound around an 8 mm diameter plastic former. The winding should measure 10 %mm along the length of the former to produce an inductance of 2.5 µH. A tuning capacitor of 320 pF achieves a lower frequency of %about 5.6 MHz. Therefore, the 49 m band and higher bands up to about 16 MHz can be tuned. A tap at the second turn of this coil %provides the appropriate impedance for connection to the receiver input. The antenna can be connected via a coupling coil made up %of two to four turns. If the coupling coil is designed so that it can be shifted along the axis of the first coil then some %variable coupling will be possible. This allows you to find the optimal match. A tighter coupling results in a higher signal %voltage, but also a lower Q factor and thereby less attenuation of the image frequency. If a short antenna such as a stick antenna %is to be used, the coupling must be arranged more tightly. The antenna can then be connected directly to the hot end of the %resonant circuit.
%
%\textbf{中文:要构建一个也可用于其他短波波段的前置选择器,最好使用可调谐振电路。为此,您可以开始绕制一个由20圈导线绕8毫米直径塑料骨架上的空心线%圈。绕组应沿着骨架的长度测量10毫米,以产生2.5 µH的电感,320 pF的调谐电容器实现约5.6 MHz的下限频率。因此,可以调谐49米波段和高达16 MHz的更高波%段。该线圈第二圈的抽头为连接到接收机输入提供了适当的阻抗。天线可以通过由两到四圈组成的耦合线圈连接。如果耦合线圈设计为可以沿着第一个线圈的轴线移%动,那么一些可变耦合将是可能的。这允许您找到最佳匹配。更紧密的耦合导致更高的信号电压,但也导致更低的Q因数,从而对镜像频率的衰减更少。如果要使用短%天线(如鞭状天线),则必须更紧密地安排耦合。然后天线可以直接连接到谐振电路的热端。}
%
%\begin{figure}[htbp]
%\centering
%%%\includegraphics[width=0.6\textwidth]{fig9-5}
%\caption{An adjustable band pass filter adds preselection.}
%\end{figure}
%
%The resonant circuit has a Q factor of about 50, which results in a bandwidth of 120 kHz at 6 MHz. Therefore, the tuning capacitor %needs to be adjusted quite precisely. If the preselector is housed in a case, it's a good idea to mark the most important %frequencies on a scale. A typical tuning capacitor has a tuning range of not much more than 1 to 10, including all circuit %capacitances. This results in a frequency range ratio of 1 to 3. To cover larger frequency bands, multiple coils can be used, and %a selector switch can be used to choose between them.
%
%\textbf{中文:谐振电路具有约50的Q因数,这导致6 MHz时的带宽120 kHz。因此,调谐电容器需要相当精确地调整。如果前置选择器装在机箱中,最好在刻度上标%记最重要的频率。典型的调谐电容器具有不超过1到10的调谐范围,包括所有电路电容。这导致1到3的频率范围比。为了覆盖更大的频率波段,可以使用多个线圈,并%可以使用选择开关在它们之间进行选择。}
%
%An alternative solution comes from the world of amateur radio, where the same problem occurs with the usual amateur radio bands %(80 m to 10 m, 3.5 MHz to 29.7 MHz) which require a preselector with a tuning range of 1 to 10. Here the answer is to use coupled %circuits, which have two tracked resonances. Figure 9.6 shows a proven circuit with a twin ganged tuning capacitor. Although there %are two pass frequencies for each setting, the "wrong" one is far away from the desired frequency.
%
%\textbf{中文:另一种解决方案来自业余无线电世界,在那里,通常的业余无线电波段80米到10米,3.5 MHz到29.7 MHz)也会出现同样的问题,这需要具有1到10%调谐范围的前置选择器。这里的答案是使用耦合电路,它具有两个跟踪谐振。图9.6显示了一个经过验证的电路,使用双联调谐电容器。虽然每个设置有两个通过频%率,但"错误"的那个离所需频率很远。}
%
%\begin{figure}[htbp]
%\centering
%%%\includegraphics[width=0.6\textwidth]{fig9-6}
%\caption{Tuning from 3 MHz to 30 MHz.}
%\end{figure}
%
%Instead of a variable capacitor, a variable capacitance diode or varicap like the BB112 can also be used for tuning. It is %important to have a stable and well-smoothed tuning voltage, otherwise phase modulation of the received signals could affect %reception. Figure 9.7 shows a preselector using a BB112 varicap.
%
%\textbf{中文:除了可变电容器之外,可变电容二极管或变容二极管(如BB112)也可以用于调谐。重要的是要有稳定且平滑的调谐电压,否则接收信号的相位调制%可能会影响接收。图9.7显示了使用BB112变容二极管的前置选择器。}
%
%\begin{figure}[htbp]
%\centering
%%%\includegraphics[width=0.6\textwidth]{fig9-7}
%\caption{A varicap is used for tuning.}
%\end{figure}
%
%A fixed-frequency tuned circuit at the receiver front end may also be a useful solution in some situations. In the medium-wave %band for example, there may be only one usable reception frequency. Even with a relatively large relative bandwidth of the input %circuit, good selection is achieved due to the low frequency. The circuit in Figure 9.8 therefore employs a fixed inductor. The %fixed-frequency band pass filter used here was designed for 1296 kHz.
%
%\textbf{中文:在接收机前端使用固定频率调谐电路在某些情况下也可能是一个有用的解决方案。例如,在中波波段,可能只有一个可用的接收频率。即使输入电路%具有相对较大的相对带宽,由于低频率,也能实现良好的选择性。因此,图9.8中的电路使用了一个固定电感器。这里使用的固定频率带通滤波器是为1296 kHz设计%的。}
%
%\begin{figure}[htbp]
%\centering
%%%\includegraphics[width=0.6\textwidth]{fig9-8}
%\caption{An antenna filter for 1296 kHz.}
%\end{figure}
%
%\section{Tuned Magnetic Field Antennas}
%\textbf{中文:调谐磁场天线}
%
%A long wire antenna receives electrical energy from both the electric and magnetic field components of the RF field but smaller %antennas such as whip antenna are only sensitive to the electric field component, which results in higher noise in the received %signal, especially in domestic environments. Electrical appliances and power line noise couple capacitively to the receiving %antenna, so in this environment it would be advantageous to pick up the magnetic field component instead. In principle, a wire %loop or coil is sufficient for this purpose. A commonly used antenna design for this purpose is a frame antenna onto which a few %turns of wire or simple loops (aka magnetic loops) are wound. Tuned loops with high Q are highly effective. For example, a length %of copper pipe bent into a one-meter diameter loop can be used, or a wide length of aluminum foil wrapped around a correspondingly %large cardboard box also produces good results. Tuning the loop with a variable capacitor up to 500 pF produces a resonant circuit %with an extremely high Q, resulting in significantly more voltage at the antenna than might be expected for an aerial of this %size. The receiver is loosely coupled using a small coupling coil to avoid excessively damping the loop. The optimal size and %location of the coupling coil is best determined experimentally. Thanks to the high Q of the antenna, an additional preselector is %unnecessary in this design.
%
%\textbf{中文:长线天线从射频场的电场和磁场分量接收电能,但较小的天线(如鞭状天线)仅对电场分量敏感,这导致接收信号中的噪声更高,特别是在家庭环境%中。电器和电源线噪声通过电容耦合到接收天线,因此在这种环境下,拾取磁场分量将是有利的。原则上,导线环或线圈足以用于此目的。为此目的常用的天线设计%是框架天线,上面绕有几圈导线或简单的环(也称为磁环)。具有高Q的调谐环非常有效。例如,可以使用弯曲成一米直径环的铜管,或者包裹在相应大的纸板箱上的%宽铝箔也能产生良好的结果。使用高达500 pF的可变电容器调谐环会产生一个具有极高Q的谐振电路,导致天线处的电压明显高于这种尺寸的天线所预期的。接收机使%用小耦合线圈松散耦合,以避免过度阻尼环。耦合线圈的最佳尺寸和位置最好通过实验确定。由于天线的高Q,这种设计不需要额外的前置选择器。}
%
%\begin{figure}[htbp]
%\centering
%%%\includegraphics[width=0.6\textwidth]{fig9-9}
%\caption{A Magnetic Loop Antenna.}
%\end{figure}
%
%A magnetic loop antenna can also be built using simple wire, although this results in lower Q factor together with lower antenna %voltage and a wider bandwidth. If the antenna needs to be physically smaller, two or more turns of insulated wire can be used.
%
%\textbf{中文:磁环天线也可以使用简单导线构建,虽然这导致较低的Q因数以及较低的天线电压和更宽的带宽。如果天线需要在物理上更小,可以使用两圈或更多%圈的绝缘导线。}
%
%In the simplest case, a shielded loop can be built using a length of coaxial cable. This antenna can be discreetly placed on a %bookshelf and provides a relatively good signal-to-noise ratio. The resonance frequency is determined by the size of the loop and %the value of the tuning capacitance. With a total of four meters of coaxial cable and a 500 pF tuning capacitor, resonant %frequencies below 6 MHz can be achieved. The broadband transformer should have a higher value of inductance on the primary %inner-wire loop side than on secondary outer-shield loop side. Good results can be achieved with about 20 turns on a ferrite or %toroidal core. The tuned circuit should not be overly damped for high quality. Therefore, the secondary side of the transformer %should only have two to four turns, the best coupling coil match should be determined experimentally.
%
%\textbf{中文:在最简单的情况下,可以使用一段同轴电缆构建屏蔽环。这种天线可以隐蔽地放置在书架上,并提供相对良好的信噪比。谐振频率由环的大小和调谐%电容的值决定。使用总共四米的同轴电缆和500 pF的调谐电容器,可以实现低于6 MHz的谐振频率。宽带变压器在初级内导线环侧应比在次级外屏蔽环侧具有更高的%电感值。在铁氧体或环形磁芯上绕20圈可以获得良好的结果。调谐电路不应过度阻尼以获得高质量。因此,变压器的次级侧应只有两到四圈,最佳耦合线圈匹配应通%过实验确定。}
%
%\begin{figure}[htbp]
%\centering
%%%\includegraphics[width=0.6\textwidth]{fig9-10}
%\caption{A Tunable Shielded Loop Antenna.}
%\end{figure}
%
%\section{An Active Indoor Antenna}
%\textbf{中文:有源室内天线}
%
%Sometimes there is simply no opportunity to rig up an outdoor antenna. A solution to this situation could be a small table-top %antenna with a two-stage preamp. The antenna described here consists of a shielded magnetic loop with an additional telescopic %antenna. The loop is about 30 cm diameter and the telescopic antenna extends to about 75 cm. While the shielded loop antenna is %very insensitive to electrical near-field interference, the antenna provides a increased electric field signal as required.
%
%\textbf{中文:有时根本没有机会架设室外天线。这种情况的解决方案可能是一个带有两级前置放大器的小型桌面天线。这里描述的天线由屏蔽磁环和附加的伸缩天%线组成。环的直径约30厘米,伸缩天线延伸至75厘米。虽然屏蔽环天线对电场近场干扰非常不敏感,但天线根据需要提供了增加的电场信号。}
%
%The loop antenna has a clear directional characteristic with two maxima in the longitudinal plane. By rotating the antenna you can %search for a maximum of the wanted signal or optionally suppress an interfering signal. The telescopic antenna has a circular %radiation pattern without any directional effect on its own. However, when operated together with the loop, both signals add %together. Due to the phase shift between the electric (E) and magnetic field (H), their sum produces a single maximum. To produce %a distinct minimum for unwanted signals, the telescopic antenna needs to be adjusted experimentally to balance both signals. The %best directional effect is achieved with a telescopic length of about 40 cm.
%
%\textbf{中文:环天线在纵向平面上具有两个最大值的清晰方向特性。通过旋转天线,您可以搜索所需信号的最大值,或者选择性地抑制干扰信号。伸缩天线具有圆%形辐射图案,本身没有任何方向效应。然而,当与环一起操作时,两个信号相加。由于电场(E)和磁场(H)之间的相移,它们的总和产生一个最大值。为了对不需%要的信号产生明显的最小值,需要通过实验调整伸缩天线以平衡两个信号。最佳方向效应是通过40厘米的伸缩长度实现的。}
%
%\begin{figure}[htbp]
%\centering
%%%\includegraphics[width=0.6\textwidth]{fig9-11}
%\caption{A tabletop antenna with preamplifier.}
%\end{figure}
%
%The antenna itself only delivers very small signal levels compared to a long wire antenna. The two-stage preamplifier compensates %for this signal loss almost completely. The limit to the maximum preamplifier gain setting is when you begin to hear the noise %generated by the first transistor in the preamp stage. However, this is not yet the case, especially in the shortwave band, i.e., %atmospheric noise prevails despite the small antenna. A second limit of amplification arises due to the possible occurrence of %intermodulation from strong signals. With three stages, strong overloading of the final transistor and clear intermodulation %products can occur. You can recognize this condition when there are practically no free frequencies available and the background %noise increases sharply. A two-stage preamp however works quite well with this size of antenna.
%
%Both stages use feedback to adjust the operating point and reduce distortion. The input stage is designed with high impedance, %while the output stage is optimized for high output with a collector resistor of 150 ohms.
%
%The suggested BFR96T transistors were specially developed for UHF preamplifiers. However, in this application in the shortwave %band, there is really no need to use such a high frequency component; you could substitute another high frequency transistor such %as a BF494. Experiments using a low-noise low-frequency transistor such as a BC548 also produced good results, although a drop off %in the RF gain was apparent above 10 MHz. The antenna also works quite well in the medium-wave band when general purpose %transistors are fitted.
%
%\begin{figure}[htbp]
%\centering
%%%\includegraphics[width=0.6\textwidth]{fig9-12}
%\caption{Active Antenna construction.}
%\end{figure}
%
%The loop antenna has an extremely low radiation resistance and poor matching to the first amplifier stage, which results in only %small signal voltages and a low signal-to-noise ratio, especially for reception on longwave and the VLF band below 150 kHz. By %using an RF transformer you can improve matching. A ferrite core transformer with an AL value of over 1000 nH/n² is suitable for %this application and should have 4 turns on the primary and 40 turns on the secondary. This increases sensitivity by about 20 dB, %allowing reception of the entire band from about 50 kHz up to the medium wave band. Using only the magnetic loop antenna (without %the rod antenna) provides good directional characteristics and excellent shielding against electric fields from nearby electrical %devices. The active antenna achieves much better reception performance than a long wire antenna. Signals from the DCF77 time %signal transmitter at 77.5 kHz and numerous other stations operating below 100 kHz can be received clearly.
%
%\begin{figure}[htbp]
%\centering
%%%\includegraphics[width=0.6\textwidth]{fig9-13}
%\caption{Adaptation for the longwave band.}
%\end{figure}
%
%\section{Antenna Preamplifier}
%\textbf{中文:天线前置放大器}
%
%Typically, an SDR receiver is designed for long antennas and will not overload even with the higher signal levels of a long wire %antenna. However, it's worth testing to see if a short whip antenna will do the job also. For this purpose, a small preamplifier %is required. The following circuit works across the entire shortwave range. The antenna used here is only 30 cm long and consists %of 0.5 mm diameter single strand wire. The recovered signal levels were comparable to those of the long wire antenna but %interference from domestic appliances was more apparent so that the noise floor was worse, resulting in more dropouts.
%
%\begin{figure}[htbp]
%\centering
%%%\includegraphics[width=0.6\textwidth]{fig9-14}
%\caption{Preamplifier for short antennas.}
%\end{figure}
%
%In the medium-wave range, a ferrite rod antenna works well as they are relatively insensitive to electrical interference, similar %to magnetic loops. Figure 9.15 shows a ferrite antenna with an impedance converter. The resonant circuit generates relatively high %resonance signals, even from distant transmitters.
%
%\begin{figure}[htbp]
%\centering
%%%\includegraphics[width=0.6\textwidth]{fig9-15}
%\caption{Active Ferrite antenna.}
%\end{figure}
%
%\chapter{VHF Radios}
%\textbf{中文:VHF收音机}
%
%Listening to distant shortwave broadcasts can be exciting and a real challenge, but most people who simply want to listen to music %or current affairs end up using an FM radio on the more common VHF frequency band. This type of radio that sits on the kitchen %worktop is fairly ubiquitous. It's easy to forget that you could fairly easily build one of them from scratch. In this article, we %will look at some tried and tested designs that will help you make your own FM radio.
%
%\textbf{中文:收听远处的短波广播可能令人兴奋且具有真正的挑战性,但大多数只想听音乐或时事的人最终在更常见的VHF频段上使用FM收音机。这种放在厨房台%面上的收音机相当普遍。很容易忘记您可以相当容易地从零开始构建一个。在本文中,我们将看一些经过尝试和测试的设计,这将帮助您制作自己的FM收音机。}
%
%\section{A Superregenerative Receiver}
%\textbf{中文:超再生接收机}
%
%The simplest FM radio receiver circuit is the superregenerative (superregen) receiver. You can build such a receiver with just two %transistors.
%
%\textbf{中文:最简单的FM收音机接收机电路是超再生(superregen)接收机。您可以仅用两个晶体管构建这样的接收机。}
%
%\begin{figure}[htbp]
%\centering
%%%\includegraphics[width=0.6\textwidth]{fig10-1}
%\caption{A 2 transistor superregenerative receiver.}
%\end{figure}
%
%To ensure stable operation, a large ground plane is necessary for this design. To build this experimental setup in the lab I used %a cut-out tin lid of a coffee can. These sorts of cans are often used to package dry edible goods; they normally have a cardboard %tube crimped onto a thin circular tin-plate base. For our purposes the cardboard can be cut away with a sharp knife. The lid here %is slightly domed and provides a stable base which takes solder very well. A piece of perforated or strip board is used as a %wiring field.
%
%\textbf{中文:为了确保稳定运行,该设计需要大的接地平面。为了在实验室中构建这个实验装置,我使用了一个咖啡罐的切割锡盖。这种类型的罐子通常用于包装%干食品;它们通常有一个纸板管压接到薄的圆形锡板底座上。对于我们的目的,纸板可以用锋利的刀切掉。这里的盖子稍微呈圆顶状,并提供了一个稳定的底座,非%常容易焊接。一块穿孔板或条板用作接线区域。}
%
%The tuning coil can be made of copper wire or, better still, silver-plated copper wire with a gauge of 0.8 mm winding 5 turns on %an 8 mm diameter former. Keep the interconnections short, especially to the tuning capacitor. The tuning capacitor used here is a %trimmer capacitor mounted directly on the ground plane. The second coil in the circuit has 20 turns of 0.2 mm CuL wound directly %on the body of a quarter-watt, 10 kΩ resistor.
%
%\textbf{中文:调谐线圈可以用铜线制成,或者更好的是镀银铜线,线径0.8 mm,在8 mm直径的骨架上5圈。保持互连短,特别是到调谐电容器的连接。这里使用的%调谐电容器是一个直接安装在接地平面上的微调电容器。电路中的第二个线圈20圈0.2 mm CuL,直接绕在四分之一瓦10 kΩ电阻器的主体上。}
%
%\begin{figure}[htbp]
%\centering
%%%\includegraphics[width=0.6\textwidth]{fig10-2}
%\caption{Circuit build using a tinplated earth plane.}
%\end{figure}
%
%The antenna should not be too long to avoid interfering with other radio listeners via the regeneration process. The circuit is %very sensitive and works with a 10 cm long antenna made simply from a piece of wire. The headphone should ideally be a 400 Ω %high-impedance type. A 32 Ω stereo headphone will also work but may be relatively quiet.
%
%\textbf{中文:天线不应太长,以避免通过再生过程干扰其他无线电听众。电路非常敏感,可以使用仅由一段电线制成的10 cm长的天线工作。耳机理想情况下应该%是400欧姆高阻抗类型。32欧姆立体声耳机也可以工作,但可能相对安静。}
%
%At turn on the receiver initially makes a loud noise. You can use a screwdriver on the coil slug to adjust the frequency and when %you find an FM station, the noise becomes quieter or completely silent. To hear the FM signal clearly, you need to tune in %precisely. This requires some practice and skill but once you've found your favorite station on FM, you don't need to touch the %dial again.
%
%\textbf{中文:打开接收机时最初会发出很大的噪音。您可以使用螺丝刀在线圈芯上调整频率,当您找到一个FM电台时,噪音会变得更安静或完全静音。要清楚地听%到FM信号,您需要精确调谐。这需要一些练习和技巧,但一旦您在FM上找到了您喜欢的电台,您就不需要再触摸刻度盘了。}
%
%The sound quality of this simple receiver is admittedly rather poor. But at least it works with just two transistors. In the early %days of FM radio, the superregen design was widely used. Back then, the circuit was built using vacuum tubes. This design, %however, eventually fell into disrepute because it simultaneously receives and transmits and can interfere with your neighbors %listening pleasure. This also applies to the version of this receiver built here. It is doubtful that you would get the CE stamp %of conformity for such a radio. The whole thing is more of an interesting experiment rather than a suggested replacement for the %proven superhet. On the other hand, you often find the superregen principle still used in receivers working in simple remote %control receivers, radio-controlled sockets, and wireless thermometers.
%
%\textbf{中文:这个简单接收机的音质确实相当差。但至少它仅用两个晶体管就能工作。在FM收音机的早期,超再生设计被广泛使用。那时,电路是使用真空管构建%的。然而,这种设计最终失去了声誉,因为它同时接收和发射,并可能干扰您邻居的收听乐趣。这也适用于这里构建的接收机版本。对于这样的收音机,您不太可能%获得CE合格印章。整个事情更像是一个有趣的实验,而不是对经过验证的超外差的建议替代品。另一方面,您经常发现超再生原理仍然在简单的遥控接收机、无线电%控制插座和无线温度计中使用的接收机中使用。}
%
%\begin{figure}[htbp]
%\centering
%%%\includegraphics[width=0.6\textwidth]{fig10-3}
%\caption{Battery operation.}
%\end{figure}
%
%The quench oscillator is just a normal oscillator. Every time the oscillator output releases the tuned VHF circuit, oscillations %start building up beginning from almost zero. Thermal noise in the front end helps initiate the tuned circuit oscillation. This %stimulation effect sometimes works faster and sometimes more slowly. The individual build-up process therefore takes different %lengths of time, which leads to an increase in collector current noise overall. This noise is audible in a superregen receiver %when it is not tuned to any station.
%
%\textbf{中文:淬灭振荡器只是一个普通振荡器。每次振荡器输出释放调谐的VHF电路时,振荡开始从几乎零开始建立。前端的热噪声有助于启动调谐电路振荡。这%种刺激效应有时工作得更快,有时更慢。因此,单独的建立过程需要不同的时间长度,这导致集电极电流噪声总体增加。当超再生接收机未调谐到任何电台时,这种%噪声是可听到的。}
%
%The waveform shown in Figure 10.4 triggers on the first left-most edge; noise on waveform can be seen as increasing fuzzyness as %the trace moves across to the right side of the screen.
%
%\textbf{中文:图10.4中显示的波形在最左侧边缘触发;波形上的噪声可以看作是随着迹线移动到屏幕右侧而增加的模糊度。}
%
%\begin{figure}[htbp]
%\centering
%%%\includegraphics[width=0.6\textwidth]{fig10-4}
%\caption{Quench waveform with noise.}
%\end{figure}
%
%When a received signal is present at the set frequency, this helps to initiate the next envelope of RF oscillations. So it starts %a little faster every time. The quenching frequency will therefore increase when receiving a signal. An unmodulated receive signal %will produce a stable quenching oscillation with no noise at the output. An amplitude modulated signal will provide differing %levels of oscillator start up assistance, which will be reflected in the average value of the change in collector current. An FM %signal can be demodulated by tuning to the edge of the oscillator signal to produce an amplitude modulated signal so that both %types of modulation can be accommodated. The resulting waveforms can be viewed on an oscilloscope. A Sawtooth waveform at the %emitter resistor indicates when a station is being received. The sensitivity of this receiver is so good it can actually work %without an antenna! The RF resonant circuit on its own absorbs enough energy for operation.
%
%\textbf{中文:当在设定频率处存在接收信号时,这有助于启动下一个RF振荡包络。因此,它每次都开始得稍快一些。因此,当接收信号时,淬灭频率会增加。未调%制的接收信号将产生稳定的淬灭振荡,输出端没有噪声。调幅信号将提供不同水平的振荡器启动辅助,这将反映在集电极电流变化的平均值中。FM信号可以通过调谐%到振荡器信号的边缘来解调,以产生调幅信号,从而可以适应两种类型的调制。结果波形可以在示波器上查看。发射极电阻器处的锯齿波形指示何时正在接收电台。%这个接收机的灵敏度如此之好,以至于它实际上可以在没有天线的情况下工作!RF谐振电路本身吸收足够的能量用于运行。}
%
%\section{Vacuum Tube Super Regen Receiver}
%\textbf{中文:电子管超再生接收机}
%
%The Franzis tube radio (Section 3.9) is a shortwave regenerative receiver. Such circuits will burst into oscillation (motor %boating) if the feedback control is turned up too far at higher frequencies. An experiment was carried out to find out whether the %radio could be converted into a VHF superregen receiver, thereby removing the need for manual feedback tweaking.
%
%\textbf{中文:Franzis电子管收音机(第3.9节)是一个短波再生接收机。如果在较高频率上将反馈控制调得太高,这种电路会突然开始振荡(马达船声)。进行%了一个实验,以确定收音机是否可以转换为VHF超再生接收机,从而消除手动反馈调整的需要。}
%
%\begin{figure}[htbp]
%\centering
%%%\includegraphics[width=0.6\textwidth]{fig10-5}
%\caption{VHF superregen with a 6J1.}
%\end{figure}
%
%In the first test, I removed the shortwave coil and replaced it with a smaller three turn coil more suitable for use in the VHF %band. Connections to the tuning capacitor are also changed, so that the 20 pF range is now used. Instead of the feedback %adjustment potentiometer I connected a 0 to 60 V power supply. The 100 kΩ grid resistor no longer connects to the anode, because %that would cause negative feedback and dampen oscillations. Now it connects to +6 V at the heater. Some regeneration oscillations %could already be observed, but at too low a frequency.
%
%\textbf{中文:在第一次测试中,我移除了短波线圈,并用一个更小的三圈线圈替换它,这更适合在VHF波段中使用。到调谐电容器的连接也改变了,因此现在使用%20 pF范围。我没有连接反馈调整电位器,而是连接了0到60 V的电源。100 kΩ栅极电阻器不再连接到阳极,因为那会导致负反馈并抑制振荡。现在它连接到加热器%+6 V。已经可以观察到一些再生振荡,但频率太低。}
%
%\begin{figure}[htbp]
%\centering
%%%\includegraphics[width=0.6\textwidth]{fig10-6}
%\caption{The 3-turn VHF coil.}
%\end{figure}
%
%The long tracks on the circuit board were causing a problem. I wound an improved coil using thicker wire made up of three turns %using the shank of an 8 mm drill bit as a former and soldered the coil very close to the tuning capacitor. The tap point to the %cathode is made with a short wire, and the grid is connected to the 100 pF capacitor using short leads. The tracks at the grid and %cathode are also cut. Now with all these changes made I was then able to tune the radio across the entire FM band.
%
%\textbf{中文:电路板上的长迹线引起了问题。我使用更粗的电线绕制了一个改进的线圈,使用8 mm钻头的柄作为骨架绕三圈,并将线圈非常靠近调谐电容器焊接。%到阴极的抽头点用短电线制作,栅极使用短引线连接100 pF电容器。栅极和阴极处的迹线也被切断。现在,进行了所有这些更改后,我就能够调谐整个FM波段的收音%机了。}
%
%When measuring with the oscilloscope, I saw strong evidence of the quenching oscillation signal at the collector of the AF preamp. %For this reason I soldered a 100 nF capacitor between the collector and emitter of transistor T2 (see Figure 10.5). For the %preliminary testing I just soldered a 470 kΩ resistor between P4 and P5 instead of the volume pot. With these changes, the FM %radio worked. I need to supply the anode voltage between 30 and 40 V from a lab power supply. The circuit will not function on 15 %V alone like the shortwave version of the radio does.
%
%\textbf{中文:使用示波器测量时,我在AF前置放大器的集电极处看到了淬灭振荡信号的强有力证据。因此,我在晶体管T2的集电极和发射极之间焊接了100 nF电%容器(见图10.5)。对于初步测试,我在P4和P5之间焊接了470 kΩ电阻器,而不是音量电位器。进行了这些更改后,FM收音机工作了。我需要从实验室电源供30到%40 V之间的阳极电压。电路不能像收音机的短波版本那样仅用15 V工作。}
%
%\section{VHF Receiver using the TDA7088}
%\textbf{中文:使用TDA7088的VHF接收机}
%
%This FM radio from Franzis can receive stations in the range of 87.5 MHz to 108 MHz and provides good reception quality. Thanks to %the TDA7088 integrated receiver module, you can listen to strong local stations with good sound quality. The receiver's %sensitivity is also good enough to pull in distant stations.
%
%\textbf{中文:来自Franzis的这个FM收音机可以接收87.5 MHz到108 MHz范围内的电台,并提供良好的接收质量。由于TDA7088集成接收机模块,您可以以良好的%音质收听强的本地电台。接收机的灵敏度也足够好,可以接收远处的电台。}
%
%\begin{figure}[htbp]
%\centering
%%%\includegraphics[width=0.6\textwidth]{fig10-7}
%\caption{The VHF FM retro radio.}
%\end{figure}
%
%The design of this radio set is reminiscent of portable radios from the 1960s. Back in those days semiconductor devices were %replacing vacuum tubes in more and more applications. Transistors consumed less energy and allowed devices like radio sets to be %made smaller, battery powered and portable. Apart from that, the principles of a radio receiver design were very similar to that %of older tube radios.
%
%\textbf{中文:这个收音机的设计让人想起1960年代的便携式收音机。在那些日子里,半导体设备在越来越多的应用中取代真空管。晶体管消耗更少的能量,并允许%收音机等设备变得更小、电池供电和便携。除此之外,收音机接收机设计的原理与较旧的电子管收音机非常相似。}
%
%\begin{figure}[htbp]
%\centering
%%%\includegraphics[width=0.6\textwidth]{fig10-8}
%\caption{The TDA7088 uses a pot for tuning.}
%\end{figure}
%
%Thanks to the highly integrated nature of the TDA7088 receiver IC, building your own FM radio is now very easy. The single-ended %audio amplifier function is more similar to the historical predecessor of a tube radio. The vintage radio uses a two-stage %transistor amplifier and gives a moderate output volume powered from two AA cells which will last for up to 200 hours.
%
%\textbf{中文:由于TDA7088接收机IC的高度集成性质,现在构建自己的FM收音机非常容易。单端音频放大器功能更类似于电子管收音机的历史前身。复古收音机使%用两级晶体管放大器,并由两个AA电池供电,提供适中的输出音量,可持续长达200小时。}
%
%\begin{figure}[htbp]
%\centering
%%%\includegraphics[width=0.6\textwidth]{fig10-9}
%\caption{The kit components.}
%\end{figure}
%
%Most FM superheterodyne receivers use an intermediate frequency of 10.7 MHz. The received frequency is first converted to the %intermediate frequency and then filtered, amplified, and demodulated. This FM radio is also a superhet that converts its received %signal to an intermediate frequency. However, the intermediate frequency is much lower at about 70 kHz. This means that the IF %filters do not require matched coils. The FM demodulator is simplified and much more immune to distortion. All the essential %stages are included in a single SMD IC, the 16-pin TDA7088. Instead of an air-vaned tuning capacitor like you see in older radio %receivers, this radio uses a varactor or varicap diode D1. As the DC voltage across the diode increases its depletion zone becomes %wider and its capacitance value decreases. This translates into a higher receive frequency. The only adjustment point is coil SP1, %which sets the oscillator frequency lower limit.
%
%\textbf{中文:大多数FM超外差接收机使用10.7 MHz的中频。接收频率首先转换到中频,然后滤波、放大和解调。这个FM收音机也是一个超外差,将其接收信号转%换到中频。然而,中频要低得多,大约为70 kHz。这意味着IF滤波器不需要匹配线圈。FM解调器被简化,并且对失真更加免疫。所有基本阶段都包含在一个SMD IC%中,16引脚TDA7088。与您在较旧的收音机接收机中看到的空气叶片调谐电容器不同,这个收音机使用变容二极管或变容二极管D1。随着二极管两端的直流电压增%加,其耗尽区变宽,电容值减小。这转换为更高的接收频率。唯一的调整点是线圈SP1,它设置振荡器频率下限。}
%
%\begin{figure}[htbp]
%\centering
%%%\includegraphics[width=0.6\textwidth]{fig10-10}
%\caption{The fully populated PCB.}
%\end{figure}
%
%The circuit board is designed in such a way that all components around the TDA7088 receiver chip have SMD outline. This makes the %construction easier. In this radio, the two coils need to be wound by hand using the wire provided and then during setup the coil %turns can be stretched out or pressed together slightly to perform fine-tuning.
%
%\begin{figure}[htbp]
%\centering
%%%\includegraphics[width=0.6\textwidth]{fig10-11}
%\caption{All the parts mounted in the case.}
%\end{figure}
%
%The audio power amplifier is a simple Class-A amplifier with the two transistors T1 and T2. The idle current is about 20 mA. The %circuit still works with good sound quality when supply voltage falls to 2.2 V.
%
%\begin{figure}[htbp]
%\centering
%%%\includegraphics[width=0.6\textwidth]{fig10-10}
%\caption{The fully populated PCB.}
%\end{figure}
%
%The circuit board is designed in such a way that all components around the TDA7088 receiver chip have SMD outline. This makes the %construction easier. In this radio, the two coils need to be wound by hand using the wire provided and then during setup the coil %turns can be stretched out or pressed together slightly to perform fine-tuning.
%
%\begin{figure}[htbp]
%\centering
%%%\includegraphics[width=0.6\textwidth]{fig10-11}
%\caption{All the parts mounted in the case.}
%\end{figure}
%
%The audio power amplifier is a simple Class-A amplifier with the two transistors T1 and T2. The idle current is about 20 mA. The %circuit still works with good sound quality when supply voltage falls to 2.2 V.
%
%Some of the wired components can be exchanged to change certain properties of the radio. R1 determines the tunable frequency %range. A lower resistance will increase the tuning range. This is useful, for example, if you plan to operate the radio with NiMH %batteries at 2.4 V. R2 determines the width of the AFC capture range. If you want to receive weak stations in the vicinity of %stronger stations, it may be useful to increase R2 up to 1 MΩ to reduce the capture range.
%
%The two connections RE1 and SC1 of the board are not used initially and are intended for later expansion. The TDA7088 was %originally developed for push-button tuning. The circuit diagram shows the two push-button switches for reset and scan. If you %want to modify receiver accordingly, the PT2\_2 connection to the slider of the frequency controller should be disconnected. At %this point, you may wish to install a switch so that the receiver can be tuned either via pushbuttons or the potentiometer.
%
%\section{Stereo Signal Decoding}
%\textbf{中文:立体声信号解码}
%
%The TDA7040 stereo decoder chip is perfect for converting the Franzis FM radio to stereo output. To achieve the necessary %bandwidth at the receiver output, the 680 pF SMD capacitor C10 must be removed from the circuit. You just need to unsolder one %side of C10, that way it won't get lost, you never know, you may need it again.
%
%\begin{figure}[htbp]
%\centering
%%%\includegraphics[width=0.6\textwidth]{fig10-12}
%\caption{External components for the TDA7040.}
%\end{figure}
%
%Figure 10.13 shows my first attempt at hooking up the TDA7040 decoder to the output of the TDA7088. A set of high-impedance stereo %headphones with (2 × 300 ohms) without any filter capacitors are shown at the left and right output pins. This is not ideal, but %it is enough for initial testing.
%
%\begin{figure}[htbp]
%\centering
%%%\includegraphics[width=0.6\textwidth]{fig10-13}
%\caption{Operation into stereo headphones.}
%\end{figure}
%
%Using the potentiometer, the oscillator frequency is adjusted to the appropriate level. On the oscilloscope, you can see how the %correct 38 kHz subcarrier signal is detected when a stereo signal is present. The capture range is so wide you can just replace %the pot with a fixed resistor. At its mid-point the pot measures 50 kΩ. If you add that value to the 100 kΩ fixed resistor it %gives a value of 150 kΩ. The scope waveform also shows that the decoder is still being overloaded, so the signal here will need to %be attenuated. The result is, however, quite impressive: a clear stereo audio signal can be heard from the headphones. It's %relatively quiet, but the circuit functions correctly.
%
%To drive speakers at a reasonable volume a small amplifier type TDA7050 is very easy to install and also operates from 3 V. No %additional capacitors are required. A 27 kΩ resistor has now been placed between the radio IC output and the stereo decoder to %prevent the overloading mentioned above. A twin-gang stereo potentiometer directs the L and R signals to the final amplifier. All %of this can be fitted onto a small square of perfboard.
%
%\begin{figure}[htbp]
%\centering
%%%\includegraphics[width=0.6\textwidth]{fig10-14}
%\caption{Adding an output amplifier.}
%\end{figure}
%
%\begin{figure}[htbp]
%\centering
%%%\includegraphics[width=0.6\textwidth]{fig10-15}
%\caption{VHF radio with loudspeaker and stereo headphone output.}
%\end{figure}
%
%The radio now has two volume knobs, one for the mono speaker, which has now been swapped for a higher (32 Ω impedance coil to give %better volume, and one for the stereo amplifier which outputs to the stereo jack socket. A set of headphones can be plugged in %here or alternatively there is enough power to drive two 32 Ω speakers. There is now one volume knob for the mono speaker and one %for the stereo headphones. This adds flexibility to the way you listen to programs.
%
%\begin{figure}[htbp]
%\centering
%%%\includegraphics[width=0.6\textwidth]{fig10-16}
%\caption{Decoder and stereo amplifier.}
%\end{figure}
%
%If you think building the circuit on a perfboard is too risky, there are other options available. Both ICs can be placed on a %shared SMD adapter board. Only a few additional components need to be added to the board. Coupling and filtering capacitors will %be installed as part of the wiring. As mentioned above the 100 kΩ trimmer pot connected to pin 3 on the TDA 7040 is unnecessary %and was replaced with a fixed 160 kΩ resistor to ground.
%
%\section{A Plug-in VHF Module}
%\textbf{中文:插入式VHF模块}
%
%The Franzis-supplied kit "Build your own FM radio" uses a pre-assembled PCB which contains the TDA7088 FM receiver chip together %the necessary coils printed on the PCB. A 6-way pinheader strip provides connections for the board to the supply, tuning voltage, %antenna, and AF output.
%
%\textbf{中文:Franzis提供的"构建自己的FM收音机"套件使用预组装的PCB,其中包含TDA7088 FM接收机芯片以及PCB上印刷的必要线圈。6路排针条为板提供到%电源、调谐电压、天线和AF输出的连接。}
%
%\begin{figure}[htbp]
%\centering
%%%\includegraphics[width=0.6\textwidth]{fig10-17}
%\caption{The plug-in PCB fitted with the TDA7088/CD9088.}
%\end{figure}
%
%A 3 V voltage regulator ensures more stability when tuning. An integrated speaker amplifier provides a good level of volume.
%
%\textbf{中文:3 V电压调节器确保调谐时更加稳定。集成扬声器放大器提供良好的音量水平。}
%
%\begin{figure}[htbp]
%\centering
%%%\includegraphics[width=0.6\textwidth]{fig10-18}
%\caption{VHF radio with voltage regulator and output amplifier.}
%\end{figure}
%
%The PCB is fitted with a 6-way pinheader strip and all the other components have flying leads attached with ends that plug into a %prototyping plug board so you won't need a soldering iron to assemble this kit. This TDA7088 receiver PCB is also suitable for %building simple FM radios to incorporate in your own projects.
%
%\textbf{中文:PCB配有6路排针条,所有其他组件都有飞线连接,端头插入原型插件板,因此您不需要电烙铁来组装这个套件。这个TDA7088接收机PCB也适合构建%简单的FM收音机,以集成到您自己的项目中。}
%
%\begin{figure}[htbp]
%\centering
%%%\includegraphics[width=0.6\textwidth]{fig10-19}
%\caption{The complete VHF receiver fits neatly into an enclosure.}
%\end{figure}
%
%\section{'Tube Sound' VHF Radio}
%\textbf{中文:电子管音色VHF收音机}
%
%The Franzis Retro Radio Deluxe combines a TDA7088 FM receiver with a tube audio amplifier stage and an integrated LM386 power %amplifier.
%
%\begin{figure}[htbp]
%\centering
%%%\includegraphics[width=0.6\textwidth]{fig10-20}
%\caption{Schematic of the VHF receiver with tube audio stage.}
%\end{figure}
%
%The PCB contains many SMD components already mounted on the board, including the TDA7088 receiver IC, 15 capacitors, and one %resistor. The components with connecting wires, such as all the parts of the audio amplifier, the tube socket, and the coils and %components around the radio's diode tuning are the only items that need to be soldered.
%
%\begin{figure}[htbp]
%\centering
%%%\includegraphics[width=0.6\textwidth]{fig10-21}
%\caption{The compact PCB contains the VHF receiver and tube socket.}
%\end{figure}
%
%\begin{figure}[htbp]
%\centering
%%%\includegraphics[width=0.6\textwidth]{fig10-22}
%\caption{Installation in the case.}
%\end{figure}
%
%This radio has a special feature called the 'sound switch'. When you turn it on, it activates the tube and gives the radio a %fuller sound. If you just want to casually listen to the news, you can turn off this tube and save power. The sound switch %interrupts the tube's heating circuit. When the heater is off, anode current flow will stop. Part of the audio signal is then %directed past the tube to the final amplifier. With an active tube, you get more volume and the distinctive changes in sound due %to the nonlinearity of the tube's characteristic curve.
%
%The radio is designed so that you can see a red glow from the cathode from the front of the set.
%
%\begin{figure}[htbp]
%\centering
%%%\includegraphics[width=0.6\textwidth]{fig10-23}
%\caption{Front view showing the tube port top left.}
%\end{figure}
%
%\section{The SI4735 DSP Radio}
%\textbf{中文:SI4735 DSP收音机}
%
%The SI4735 is a chip measuring 3mm × 3mm that contains a complete radio able to tune to one FM and three AM bands. The company %Modul-Bus has developed an adapter PCB on which the chip is mounted. This board can be conveniently used together with a %prototyping plug board for carrying out tests to build a radio. In addition, for experimentation a USB/serial converter board type %UM232R is used to provide the interface between a PC and the receiver chip via USB. The chip also provides the required 3.3 V %operating voltage with its in-built LDO regulator.
%
%\begin{figure}[htbp]
%\centering
%%%\includegraphics[width=0.6\textwidth]{fig10-24}
%\caption{The SI4735 block diagram.}
%\end{figure}
%
%From the block diagram you can see that this radio chip functions as an IQ type receiver, similar to the shortwave receiver design %we described earlier (Section 8.6). The main difference here is that all the signal decoding does not need any external PC %software because the SI4735 contains a digital signal processor (DSP) which takes care of these tasks. A PC or microcontroller is, %however, still required for tuning. The SI4735 has various digital interfaces for communication, and in this case, the I2C bus is %used. This uses signals SDA and SCL along with the chip's reset input.
%
%\begin{figure}[htbp]
%\centering
%%%\includegraphics[width=0.6\textwidth]{fig10-25}
%\caption{Connections to the outside world.}
%\end{figure}
%
%The circuit diagram shows a minimal setup for the initial test. Only a short piece of wire was used as an FM antenna. The stereo %outputs R and L do not have coupling capacitors because they are already included in the internal amplifier input. A 32 kHz %crystal provides the clock signal. The interface to the UM232R requires three resistors. Two of the lines could be connected %directly, but this provides greater fault tolerance.
%
%The chip requires a supply voltage of 3.3 V at VDD and VIO. Note that no more than 3.6 V is allowed at VDD. It is important to %make sure that the UM232R interface adapter is not accidentally jumpered to 5 V. Unfortunately one chip was damaged during the %initial trials due to my own carelessness. Jumper S1 must be in the upper position to provide 3.3 V to VIO.
%
%\begin{figure}[htbp]
%\centering
%%%\includegraphics[width=0.6\textwidth]{fig10-26}
%\caption{The clock crystal, RS232 interface and SI4735 on a breadboard.}
%\end{figure}
%
%In this project, the FT232R chip is used as a serial interface, like COM1 or COM2. The TTL levels with 3.3 V are inverted compared %to a real RS232. The DTR and RTS lines form an I2C bus with the additional input line CTS. To control it, a small test program has %been written in VB, which can be downloaded from elexs.de.
%
%\begin{figure}[htbp]
%\centering
%%%\includegraphics[width=0.6\textwidth]{fig10-27}
%\caption{The control interface.}
%\end{figure}
%
%You will need a length of wire about 10 cm long to use as a simple antenna to receive FM broadcasts. After starting the program %initialize the receiver by clicking on the FM button. The 32 kHz crystal will now start oscillating. Another test for successful %initialization is the voltage level at the R and L audio outputs, which should now rise to about 1 V. The FM radio tunes to the %first strong station and the stereo signal appears at the output. You can enter another frequency in the frequency field or start %a scan. At the top right there is also a volume control slider.
%
%\textbf{中文:您需要一根长10厘米的电线作为简单天线来接收FM广播。启动程序后,通过单击FM按钮初始化接收机。32 kHz晶体现在将开始振荡。成功初始化的%另一个测试是R和L音频输出处的电压水平,现在应该上升到大约1 V。FM收音机调谐到第一个强电台,立体声信号出现在输出端。您可以在频率字段中输入另一个频率%或开始扫描。右上角还有一个音量控制滑块。}
%
%For AM reception, you will need to connect an appropriate antenna or preselector, such as a ferrite antenna. After AM %initialization, tuning works similarly to FM, either by direct input or by using the scan function.
%
%\textbf{中文:对于AM接收,您需要连接适当的天线或预选器,例如铁氧体天线。AM初始化后,调谐与FM类似,可以通过直接输入或使用扫描功能。}
%
%The SI4735 module has been used in various projects, including the PC radio and home radio from Modul-Bus, as well as the Elektor %DSP radio with a microcontroller and LCD.
%
%\textbf{中文:SI4735模块已在各种项目中使用,包括来自Modul-Bus的PC收音机和家用收音机,以及带有微控制器和LCD的Elektor DSP收音机。}
%
%\section{PC Radio from a USB Port}
%\textbf{中文:来自USB端口的PC收音机}
%
%The PC radio allows for control of all functions of the SI4735 via the USB interface. The perforated grid area provides enough %space for additional circuitry.
%
%\textbf{中文:PC收音机允许通过USB接口控制SI4735的所有功能。穿孔网格区域为额外的电路提供足够的空间。}
%
%\begin{figure}[htbp]
%\centering
%%%\includegraphics[width=0.6\textwidth]{fig10-28}
%\caption{The PC Radio.}
%\end{figure}
%
%\begin{figure}[htbp]
%\centering
%%%\includegraphics[width=0.6\textwidth]{fig10-29}
%\caption{Schematic with USB port.}
%\end{figure}
%
%Software to interface with and control the radio is available from elexs.de. The standard program is called Si4735Radio5.exe. It %was written in Delphi and includes handling of the chip's RDS function.
%
%\textbf{中文:用于接口和控制收音机的软件可从elexs.de获得。标准程序称为Si4735Radio5.exe。它是用Delphi编写的,包括处理芯片的RDS功能。}
%
%\begin{figure}[htbp]
%\centering
%%%\includegraphics[width=0.6\textwidth]{fig10-30}
%\caption{The SI4735 radio with RDS information.}
%\end{figure}
%
%\section{The VHF FM Home Radio}
%\textbf{中文:VHF FM家用收音机}
%
%This FM radio has been designed to make it easy and intuitive to use especially for those who are less confident with tasks that %require a good level of manual dexterity. Firstly any station in the complete FM band can be tuned using the tuning potentiometer %so that the stations are distributed around a rotation angle of 270 degrees. The user would typically only listen to just a few of %these stations. Say for example three stations are programmed during setup. Now the tuning knob will now select between only these %three stations so that the first 90 degrees of rotation selects station 1 and the next 90 degrees selects station 2, etc. This %makes tuning a doddle, no more squinting at a tuning dial or fiddling with band selection. This level of convenience is made %possible by a small microcontroller type ATtiny25.
%
%\textbf{中文:这个FM收音机被设计得易于直观使用,特别是对于那些对需要良好手动灵巧性的任务不太自信的人。首先,可以使用调谐电位器调谐整个FM波段中的%任何电台,以便电台分布在270度的旋转角度周围。用户通常只会收听这些电台中的几个。例如,在设置期间编程三个电台。现在,调谐旋钮将仅在这三个电台之间选%择,因此前90度的旋转选择电台1,接下来90度选择电台2,依此类推。这使得调谐变得轻而易举,不再需要眯着眼看调谐刻度盘或摆弄波段选择。这种便利性水平是%由小型微控制器ATtiny25实现的。}
%
%\begin{figure}[htbp]
%\centering
%%%\includegraphics[width=0.6\textwidth]{fig10-31}
%\caption{Operation with a microcontroller.}
%\end{figure}
%
%The small standalone board also includes a simple mono speaker amplifier. If you want to use it, both jumpers must be in position. %You can install the board, for example, into an existing speaker cabinet and make a custom radio. It is often useful not to mount %the potentiometers on the board but to install some with long spindles elsewhere in case. The stereo jack output can also be used %to connect the radio to PC active speakers, for example. The provided housing offers space for a 9 V battery or power can be %supplied via the power supply jack.
%
%\textbf{中文:小型独立板还包括一个简单的单声道扬声器放大器。如果您想使用它,两个跳线都必须就位。您可以将板安装,例如,到现有的扬声器柜中,并制作%一个自定义收音机。通常有用的是不要将电位器安装在板上,而是在其他地方安装一些长轴的电位器。立体声插孔输出也可以用于将收音机连接到PC有源扬声器,例%如。提供的外壳提供9 V电池的空间,或者可以通过电源插孔供电。}
%
%\begin{figure}[htbp]
%\centering
%%%\includegraphics[width=0.6\textwidth]{fig10-32}
%\caption{The VHF FM home radio with all controls and mounted components.}
%\end{figure}
%
%\begin{figure}[htbp]
%\centering
%%%\includegraphics[width=0.6\textwidth]{fig10-33}
%\caption{The radio fitted into a case.}
%\end{figure}
%
%There are two ways to program the radio: using the pushbutton on the board or via a connected PC.
%
%\textbf{中文:有两种方式来编程收音机:使用板上的按钮或通过连接的PC。}
%
%Programming using the pushbutton: When turning on the radio, the pushbutton must be held down for more than a second to enter %programming mode. The radio then immediately searches from 87.5 MHz for the first station. Now with each press of the button, the %radio scans to the next station. To save a station, the button must be held down for more than a second. A short press of the %button skips the last station heard and searches for the next one. Up to 20 stations can be saved, which are then distributed over %the entire 270 degrees rotation of the tuning dial during normal operation. After programming, the radio should be turned off and %then back on again.
%
%\textbf{中文:使用按钮编程:打开收音机时,必须按住按钮超过一秒钟才能进入编程模式。收音机然后立即从87.5 MHz搜索第一个电台。现在,每次按下按钮,收%音机扫描到下一个电台。要保存电台,必须按住按钮超过一秒钟。短按按钮跳过最后听到的电台并搜索下一个。最多可以保存20个电台,这些电台在正常操作期间分%布在整个270度的调谐刻度盘旋转上。编程后,收音机应该关闭然后重新打开。}
%
%To program via a PC, a serial cable can be connected to GND and COM (TXD pin). For operation, any terminal program working in text %input mode is sufficient. The transfer rate is 1200 baud. The radio can be switched to PC mode at any time during normal operation %and can then only be controlled by the PC until the next restart. Using the terminal, you can tune the receiver and assign %frequencies between 65 MHz and 108 MHz to the individual memory locations. If, for example, ten stations were previously stored %and now only four stations are required, a special end marker must be written to location 5.
%
%\textbf{中文:要通过PC编程,可以将串行电缆连接到GND和COM(TXD引脚)。对于操作,任何在文本输入模式下工作的终端程序就足够了。传输速率1200波特。收%音机可以在正常操作期间的任何时候切换到PC模式,然后只能由PC控制,直到下次重启。使用终端,您可以调谐接收机并将65 MHz到108 MHz之间的频率分配给各个%存储位置。例如,如果之前存储了十个电台,现在只需要四个电台,则必须将特殊的结束标记写入位置5。}
%
%Enter: Start the PC mode
%
%8880: Tune to 88.8 MHz
%
%10280: Tune to 102.8 MHz
%
%1: Set to memory location 1
%
%20000: Use as end marker for frequencies > 108 MHz
%
%5: Use end marker in memory location 5
%
%After finishing the programming process, the radio needs to be restarted for the settings to take effect. Alternatively, the %receiver can just be used as a PC radio by default. It is also possible to create custom applications with special firmware, such %as a kitchen radio for parents with children. A station chosen by one of the children will automatically switch back to the %station preferred by a parent after 30 minutes. Or a sleep radio, which turns off automatically after a predetermined time. Those %who wish can modify the firmware to do exactly as they please. The possibilities are endless.
%
%\textbf{中文:完成编程过程后,收音机需要重新启动才能使设置生效。或者,接收机可以默认用作PC收音机。也可以创建具有特殊固件的自定义应用程序,例如为%有孩子的父母提供的厨房收音机。一个孩子选择的电台将在30分钟后自动切换回父母偏好的电台。或者睡眠收音机,它在预定时间后自动关闭。那些愿意的人可以修%改固件以完全按照他们的意愿做。可能性是无限的。}
%
%In Figure 10.34, an alternative installation suggestion is shown using a retro radio case. The potentiometers on the board have %been replaced with ones mounted on the case. The meter monitors the battery voltage. The built-in speaker and the large housing %provide a full sound.
%
%\textbf{中文:在图10.34中,显示了使用复古收音机外壳的替代安装建议。板上的电位器已被安装在外壳上的电位器替换。仪表监视电池电压。内置扬声器和大型%外壳提供完整的声音。}
%
%\begin{figure}[htbp]
%\centering
%%%\includegraphics[width=0.6\textwidth]{fig10-34}
%\caption{Using an existing enclosure.}
%\end{figure}
%
%\begin{figure}[htbp]
%\centering
%%%\includegraphics[width=0.6\textwidth]{fig10-35}
%\caption{All the wiring inside using the existing pots.}
%\end{figure}
%
%Another possibility would be to use a different custom enclosure. How about a vintage tube radio, for example? There must be loads %of these hanging around in junk shops that most people don't have the time or inclination to bring back to life. If you are %planning a retro vibe for your home decoration and think a particular vintage set would be the cherry on the cake why not bring %the radio back to life by installing a home radio? It doesn't have to be final, but you could simply use the original speaker and %just retire the vacuum tube chassis. The result would most likely be a particularly beautiful sound, almost like in the old days, %but without the crackles and distortion that dogged radio reception in the early days. One thing is clear; for sound quality, none %of these old radios would be able to compete with modern FM broadcasts.
%
%\textbf{中文:另一种可能性是使用不同的自定义外壳。例如,复古电子管收音机怎么样?在旧货店里肯定有很多这样的收音机,大多数人没有时间或意愿让它们复%活。如果您正在为家居装饰规划复古氛围,并认为某个复古套装是锦上添花,为什么不通过安装家用收音机让收音机复活呢?它不一定是最终的,但您可以简单地使%用原始扬声器,只是让电子管底盘退役。结果很可能是一种特别美丽的声音,几乎就像过去的日子,但没有早期困扰收音机接收的爆裂声和失真。有一点是清楚的;%就音质而言,这些旧收音机都无法与现代FM广播竞争。}
%
%It may also be possible to actually use the output tube EL84 and volume control of the original set, but first you would need to %remove all the RF tubes from their sockets. Then, all that's missing is where to position the tuning potentiometer for the home %radio. One solution might be a mechanical coupling with the tuning capacitor. It may be less hassle if one of the tone controls %were repurposed. Whatever, you would certainly end up with something quite unique!
%
%\textbf{中文:也可能实际上使用原始套装的输出管EL84和音量控制,但首先您需要从其插座中移除所有RF管。然后,缺少的是在哪里定位家用收音机的调谐电位%器。一种解决方案可能是与调谐电容器的机械耦合。如果重新利用音量控制之一,可能会减少麻烦。无论如何,您肯定会得到一些独特的东西!}
%
%\section{The Elektor DSP Radio}
%\textbf{中文:Elektor DSP收音机}
%
%A world receiver which works on all the FM, LW, MW, and SW bands, but doesn't have any of the traditional tuning circuitry or %controls can be built using Digital Signal Processing (DSP) principles. In the design shown here all the essential functional %groups are housed in the tiny 3 mm × 3 mm Si4735 DSP radio chip. In addition to this the radio has a control unit with an LCD, a %stereo audio amplifier, and the necessary interfaces to allow the receiver to be controlled by a PC, if desired.
%
%\textbf{中文:一个在所有FM、LW、MW和SW波段上工作的世界接收机,但没有任何传统的调谐电路或控制,可以使用数字信号处理(DSP)原理构建。在这里显示的%设计中,所有基本功能组都包含在微小的3 mm × 3 mm Si4735 DSP收音机芯片中。除此之外,收音机还具有带有LCD的控制单元、立体声音频放大器,以及必要的接%口,以允许在需要时由PC控制接收机。}
%
%Many radio enthusiasts actually find they need two receivers, one for portable use and one as a stationary receiver with PC %control. The Elektor DSP radio shown here can do both. Thanks to the USB interface, PC control is possible at any time, and the %entire receiver can be powered via USB. The audio output can also be connected to PC active speakers. The receiver can also be %powered from a 6 V battery pack and the circuit has its own integrated audio amplifiers and one (or two) speakers.
%
%\textbf{中文:许多收音机爱好者实际上发现他们需要两个接收机,一个用于便携式使用,一个作为带有PC控制的固定接收机。这里显示的Elektor DSP收音机可以%两者兼得。由于USB接口,PC控制随时可用,整个接收机可以通过USB供电。音频输出也可以连接到PC有源扬声器。接收机也可以由6 V电池组供电,电路具有自己的%集成音频放大器和一个(或两个)扬声器。}
%
%\begin{figure}[htbp]
%\centering
%%%\includegraphics[width=0.6\textwidth]{fig10-36}
%\caption{Stuffed prototype of the Elektor DSP Radio.}
%\end{figure}
%
%When it comes to a universal receiver, the first thing I look for is a clean FM reception, preferably in stereo and with RDS %station display, so I can see what I am listening to. This receiver offers these features with excellent FM sensitivity and sound %quality. It uses the SI4735 chip so RDS is also included.
%
%\textbf{中文:当涉及到通用接收机时,我首先寻找的是干净的FM接收,最好是立体声并带有RDS电台显示,这样我可以看到我正在听什么。这个接收机提供了这些%功能,具有出色的FM灵敏度和音质。它使用SI4735芯片,因此也包括RDS。}
%
%The second requirement is that the radio's shortwave performance should have the ability to pick up distant AM stations. Here, %too, the receiver excels with excellent shortwave reception characteristics, with very high sensitivity combined with good large %signal tolerance, allowing the use of long antennas. An effective Automatic Level Control (ALC) brings the received signal into %the optimal range, so that low gain antennas can be used without much loss in performance. This receiver's selectivity is also %outstanding, and you can freely choose the bandwidth in several stages, which is usually only available with top-end receivers.
%
%\textbf{中文:第二个要求是收音机的短波性能应该能够接收远处的AM电台。在这里,接收机也以出色的短波接收特性而卓越,具有非常高的灵敏度结合良好的大信%号容限,允许使用长天线。有效的自动电平控制(ALC)将接收信号带入最佳范围,因此可以使用低增益天线而不会损失太多性能。这个接收机的选择性也是出色的,%您可以自由地在几个阶段中选择带宽,这通常只有高端接收机才具有。}
%
%This receiver also covers the medium and long wave bands. An antenna input allows for the connection of an external antenna for %any frequency bands. If a simple whip antenna or some other indoor aerial is fitted it will usually pick up too much domestic %interference so you can alternatively connect to a ferrite antenna here.
%
%\textbf{中文:这个接收机也覆盖中波和长波波段。天线输入允许为任何频段连接外部天线。如果安装了简单的鞭状天线或其他室内天线,它通常会接收到太多的家%庭干扰,因此您可以在这里连接到铁氧体天线。}
%
%\begin{figure}[htbp]
%\centering
%%%\includegraphics[width=0.6\textwidth]{fig10-37}
%\caption{The receiver schematic.}
%\end{figure}
%
%At first glance, the receiver's circuit doesn't show much evidence of typical RF technology or VHF receiver design. That's because %all essential functions are integrated into the Si4735. Only the antenna input circuitry reveals the RF nature of this design. The %antenna signal from the BNC socket K4 or screw terminal K3 first passes through a diode limiter with D4 and D5. L2 is the FM coil %with a value of 0.1 µH. The jumper JP1 is normally in position 3-2, connecting the bottom end of the FM coil to the AM input.
%
%\textbf{中文:乍一看,接收机的电路没有显示太多典型RF技术或VHF接收机设计的证据。这是因为所有基本功能都集成到Si4735中。只有天线输入电路揭示了这种%设计的RF性质。来自BNC插座K4或螺钉端子K3的天线信号首先通过带有D4和D5的二极管限幅器。L2是FM线圈,值为0.1 µH。跳线JP1通常处于位置3-2,将FM线圈的%底端连接到AM输入。}
%
%What you can't see in the circuit diagram is that in FM mode, the receiver sets its internal AM 'variable capacitor' to 500 pF, %which effectively shorts the FM coil to ground. In AM mode, however, the antenna signal now passes through L2 as an extension coil %to the AM resonant circuit made up of L3, L4, L5 and the automatically tuned 'variable capacitor' inside the Si4735 at pin 4 %(AMI). The diode switch with D6 and D7 determines which fixed inductances are effective. If necessary, a portion of the coils are %be shorted to ground via the 1N4148 diodes. In normal operation, the three jumpers JP2 to JP4 are closed, but alternative input %circuits or a ferrite antenna can be connected via the jumper pins. For example, a medium wave ferrite antenna can be connected to %JP3, and a shortwave loop to JP2. If a whip antenna is only used for FM, JP1 is set to short pins 1-2.
%
%The stereo output signal of the Si4735 is led to a stereo jack socket as an audio output via C28 and C29, for connection to an %external amplifier or powered speakers. The output is short-circuit-proof with an output impedance of 10 k at 80 mVeff approx. Two %LM386 ICs are used as audio power amplifiers, allowing speakers to be connected at K5. The maximum power into 8 Ω is about 300 mW. %A stereo volume potentiometer is not required in the circuit. The microcontroller IC3 (ATmega168) controls the volume of both %channels and all other functions of the Si4735 via software using the I2C bus with its two control signals SDA and SCL. It reads %the voltage at the linear potentiometer P1 via the analog input ADC0 and converts it into corresponding commands for the Si4735. %Tuning control is implemented via a rotary encoder (ENC1) which connects to two input pins. The four pushbuttons S2 to S5 are %additional controls. To show received signal strength a PWM output for connecting an optional S-meter, generates a 500 Hz square %wave signal with variable duty cycle and a median voltage between 0 and 3.3 V. Almost any measuring device up to about 1 mA can be %connected to it with a suitable resistor. The ATmega168 microcontroller is clocked at 8 MHz, which is independent of the %receiver's actual frequency. The receiver derives its reception frequency from a connected clock crystal which runs at 32.768 kHz.
%
%\textbf{中文:Si4735的立体声输出信号通过C28和C29引导到立体声插孔插座作为音频输出,用于连接到外部放大器或有源扬声器。输出是短路保护的,输出阻抗%10 k,大约80 mVeff。两个LM386 IC用作音频功率放大器,允许在K5处连接扬声器。进入8 Ω的最大功率约300 mW。电路中不需要立体声音量电位器。微控制器IC3%(ATmega168)通过软件使用I2C总线及其两个控制信号SDA和SCL控制两个通道的音量和Si4735的所有其他功能。它通过模拟输入ADC0读取线性电位器P1处的电压,%并将其转换为Si4735的相应命令。调谐控制通过旋转编码器(ENC1)实现,该编码器连接到两个输入引脚。四个按钮S2到S5是附加控制。为了显示接收信号强度,用%于连接可选S表的PWM输出生成500 Hz方波信号,具有可变占空比0和3.3 V之间的中值电压。几乎任何高达约1 mA的测量设备都可以通过适当的电阻连接到它。%ATmega168微控制器以8 MHz时钟运行,这与接收机的实际频率无关。接收机从连接的时钟晶体导出其接收频率,该晶体以32.768 kHz运行。}
%
%There are three options for powering the radio: through the USB port, a 6 V mains adapter or a battery pack with four cells (4.8 %to 6 V). This voltage VIN is applied to the two LM386 amplifiers, the LCD backlight, and the input of the voltage regulator IC1 %(LP2950-3.3), which regulates it down to 3.3 V for the radio chip, microcontroller, and LCD. The power switch S1 on the board only %switches the voltage from K1 (battery or power supply), while the voltage from the USB port remains on. If you want to save power, %you can turn off the LCD backlight by removing the link at JP5. The LCD is still readable without the backlight. The receiver %consumes around 50 mA and works down to a voltage of 4.0V so you can expect good battery life.
%
%\textbf{中文:有三种选项为收音机供电:通过USB端口、6 V电源适配器或带有四个电池的电池组(4.8到6 V)。这个电压VIN施加到两个LM386放大器、LCD背光%和电压调节器IC1(LP2950-3.3)的输入,后者将其调节到3.3 V,用于收音机芯片、微控制器和LCD。板上的电源开关S1仅切换来自K1(电池或电源)的电压,而来%自USB端口的电压保持开启。如果您想节省电源,可以通过移除JP5处的连接来关闭LCD背光。LCD在没有背光的情况下仍然可读。接收机消耗大约50 mA,并且可以工%作到4.0 V的电压,因此您可以期望良好的电池寿命。}
%
%\begin{figure}[htbp]
%\centering
%%%\includegraphics[width=0.6\textwidth]{fig10-38}
%\caption{The LC-Display in operation.}
%\end{figure}
%
%The display shows the tuning frequency, the antenna voltage in dBµV, and the signal-to-noise ratio (SNR in dB). In FM mode, the %lower line displays the station identifier and time sent via RDS.
%
%\textbf{中文:显示显示调谐频率、以dBµV为单位的天线电压以及信噪比(SNR,以dB为单位)。在FM模式下,下行显示通过RDS发送的电台标识符和时间。}
%
%\section{The BK1079/1068 FM Radio Chip}
%\textbf{中文:BK1079/1068 FM收音机芯片}
%
%A new type of radio chip designed for use in small scanning headphone radios was introduced only a few years ago. The BK1068 from %the Chinese company Beken bears a strong resemblance to the BK1079 from the same company; it comes in a slightly larger housing %with the unusual 1 mm pin spacing.
%
%\textbf{中文:一种专为用于小型扫描耳机收音机而设计的新型收音机芯片仅在几年前推出。来自中国公司Beken的BK1068与同公司的BK1079非常相似;它采用稍大%的外壳,具有不寻常的1 mm引脚间距。}
%
%\begin{figure}[htbp]
%\centering
%%%\includegraphics[width=0.6\textwidth]{fig10-39}
%\caption{A small 8-way carrier PCB with the mounted BK1068.}
%\end{figure}
%
%\begin{figure}[htbp]
%\centering
%%%\includegraphics[width=0.6\textwidth]{fig10-40}
%\caption{Block diagram of the BK1079.}
%\end{figure}
%
%This IC seems to strongly resemble the DSP radios chips from Silicon Labs like the SI4735, indicating that the BK1079/1068 is %actually a DSP radio. This explains its high quality, as the output signal is absolutely clean and shows no traces of the stereo %subcarrier signal. Another advantage over the TDA7088 is that the volume can be adjusted internally.
%
%\textbf{中文:这个IC似乎与来自Silicon Labs的DSP收音机芯片(如SI4735)非常相似,表明BK1079/1068实际上是一个DSP收音机。这解释了它的高质量,因%为输出信号绝对干净,没有立体声副载波信号的痕迹。与TDA7088相比的另一个优点是音量可以在内部调整。}
%
%The Seek input and the Volume input have about half the operating voltage in their idle state. The chip detects when the inputs %are pulled to GND or VDD potential. Additionally, there is a Reset function that sets the lowest frequency and a Power-Down input %(PDN) which turns the chip on and off. This allows the radio to operate without a switch to the battery. When the radio is turned %off, it consumes hardly any power and retains the last used settings.
%
%\textbf{中文:Seek输入和Volume输入在空闲状态下具有大约一半的工作电压。芯片检测输入何时被拉到GND或VDD电位。此外,还有一个Reset功能,它设置最低%频率,以及一个Power-Down输入(PDN),它打开和关闭芯片。这允许收音机在没有开关到电池的情况下运行。当收音机关闭时,它几乎不消耗任何功率并保留最后%使用的设置。}
%
%An adapter board for this IC is available from Modul-Bus. This brings the small SMD IC to a handy DIP8 format. The ten connections %are reduced to eight pins because GND appears twice (pin 5 and pin 7) and the unused RCLK input is connected to Vdd. This board %makes experimenting easy. All you need is a 3 V battery and a few control pushbuttons to build a high-quality radio.
%
%\textbf{中文:这个IC的适配器板可从Modul-Bus获得。这将小型SMD IC带到方便的DIP8格式。十个连接减少到八个引脚,因为GND出现两次(引脚5和引脚7),未%使用的RCLK输入连接到Vdd。这个板使实验变得容易。您只需要一个3 V电池和几个控制按钮来构建一个高质量的收音机。}
%
%The circuit shows a typical application with pushbuttons for all the functions. The Scan and Vol button inputs are actually %tristate with a middle level at half the operating voltage. Therefore, switching the pin to either GND or Vdd assigns two %functions to one input.
%
%\textbf{中文:电路显示了所有功能的典型应用程序,带有按钮。Scan和Vol按钮输入实际上是三态的,具有工作电压一半的中间电平。因此,将引脚切换到GND或%Vdd将两个功能分配给一个输入。}
%
%\begin{figure}[htbp]
%\centering
%%%\includegraphics[width=0.6\textwidth]{fig10-41}
%\caption{External component connections}
%\end{figure}
%
%This module requires minimal external components if some of the switch functions are ignored. The IC starts with maximum volume. A %single scan button is sufficient because it automatically switches back to the beginning at the upper end of the band, allowing %the IC to scan in a loop. The on/off button switches the IC to the power-down mode and back to the active mode, while keeping all %current settings, such as frequency and volume. This means that unlike older scanning radios, you don't need to scan for your %preferred station every time. Another advantage is that you can begin scanning in either direction.
%
%\textbf{中文:如果忽略一些开关功能,这个模块只需要最少的外部组件。IC以最大音量启动。单个扫描按钮就足够了,因为它在波段的上端自动切换回开始,允许%IC循环扫描。开/关按钮将IC切换到断电模式并返回到活动模式,同时保持所有当前设置,例如频率和音量。这意味着与较旧的扫描收音机不同,您不需要每次都扫描%您偏好的电台。另一个优点是您可以开始向任一方向扫描。}
%
%In Figure 10.42, there is a setup using six pushbuttons on a test board. A 470 Ω resistor is placed between the two up/down %pushbuttons to prevent a short circuit if both pushbuttons are accidentally pressed simultaneously.
%
%\textbf{中文:在图10.42中,有一个在测试板上使用六个按钮的设置。在两个上下按钮之间放置一个470 Ω电阻,以防止两个按钮同时意外按下时发生短路。}
%
%\begin{figure}[htbp]
%\centering
%%%\includegraphics[width=0.6\textwidth]{fig10-42}
%\caption{Test setup testing all of the switch possibilities.}
%\end{figure}
%
%The IC is actually intended to drive headphones with 16 \(\Omega\) impedance but delivers more than ample volume and excellent %sound quality. Tests have also shown that an 8 \(\Omega\) speaker can also be used. An output voltage of up to about 1 Vpp was %measured. This volume level is sufficient for most domestic environments without an additional power amplifier. The module is also %ideal for sprucing up an old tube radio, either using its own power amplifier or feeding into the existing tube power amp.
%
%\textbf{中文:这个IC实际上旨在驱动具有16 \(\Omega\)阻抗的耳机,但提供超过充足的音量和出色的音质。测试还表明,也可以使用8 \(\Omega\)扬声器。测%量了高达1 Vpp的输出电压。这个音量水平对于大多数家庭环境来说足够了,无需额外的功率放大器。这个模块也非常适合翻新旧电子管收音机,无论是使用自己的功%率放大器还是馈送到现有的电子管功率放大器。}

\backmatter

\end{document}



