% 收音机
% 收音机.tex

\documentclass[12pt,UTF8]{ctexbook}

% 设置纸张信息。
% 纸张设置配置文件
% 用于定义书籍的页面尺寸和边距

\usepackage[a4paper,twoside]{geometry}
\geometry{
	left=25mm,
	right=20mm,
	top=25mm,
	bottom=25.4mm,
	headsep=1cm, 
    footskip=1cm,
	bindingoffset=10mm
}

% 设置字体,并解决显示难检字问题。
\xeCJKsetup{AutoFallBack=true}
% 注意:ctexbook类已默认设置SimSun为CJKrmdefault,此处不再重复设置
% 仅设置扩展字体以避免冲突
\setCJKfamilyfont{hei}{SimHei}
\setCJKfamilyfont{kai}{KaiTi}

% 目录 chapter 级别加点(.)。
\usepackage{titletoc}
\titlecontents{chapter}[0pt]{\vspace{3mm}\bf\addvspace{2pt}\filright}{\contentspush{\thecontentslabel\hspace{0.8em}}}{}{\titlerule*[8pt]{.}\contentspage}

% 设置 part 和 chapter 标题格式。
\ctexset{
	chapter/name={第,章},
	chapter/number={\chinese{chapter}}
}

% 图片相关设置。
\usepackage{graphicx}
\graphicspath{{Images/}}

% 数学公式支持
\usepackage{amsmath}

% 设置署名格式。
\newenvironment{shuming}{\hfill\zihao{4}}

% 注脚每页重新编号,避免编号过大。
\usepackage[perpage]{footmisc}

\title{\heiti\zihao{0} 收音机}
\author{佚名}
\date{}

\begin{document}

\maketitle
\tableofcontents

\frontmatter

\mainmatter

\part{基础元件}

\chapter{天线}

广播电台会将音频信号(声音)搭载在高频载波上,以电磁波的形式向空间发射。天线接收空间中传播的无线电波(如 AM 中波广播信号),这些电波会在天线中感应出微弱的高频交变电流。

天线的符号和实物见图2-2。它是一根水平张挂着的导线,用引入线通到室内,和收音机电路连接。在离电台5公里左右的市区内,只要用一根长5~7米的导线,通向室外,悬挂在3米左右高的地方就可以了。离电台较远的市郊,天线就应适当地加长、加高。离电台四、五十公里以外,尤其是有山岭阻隔的地方,本章所介绍的简单装置是很难收到电台播音的。

\begin{figure}[htbp]
	\centering
	\includegraphics[width=0.7\linewidth]{14}
	\caption{}
	\label{fig:1}
\end{figure}

\section{什么叫天线,为什么要用天线}

无线电收音机收到的远地广播电台的播音,是靠一种看不见、嗅不出、摸不到的所谓“无电波”(或者叫电磁波)来传递的。这种电波从广播电台向四周围发射出来,就好像声波在空气中向四周扩散一样。

\begin{figure}[htbp]
	\centering
	\includegraphics[width=0.7\linewidth]{1}
	\caption{}
	\label{fig:1}
\end{figure}

无线电波虽然漫天遍地都是,但是我们的感觉器官却无法直接感受,无法捕捉。“天线”就是用来捕捉那“来无影去无踪”的无线电波的。打个比方,正好像高挂着的蛛网,又好像昆虫用以探索物体的触须。所以无线电发明者 AC.波波夫就叫它为“Antenna”\footnote{Antenna是从希腊文“触须”一字转借来的,是法国物理学家布隆德尔在波波夫发明天线时写给波波夫信中第一次提出的。},无线电波虽然看不见、嗅不出、摸不着,但它一碰到金属等导电物体就会在这物体中感应出来相应的电压。所以天线是用良导体(如铜线等)做的。

虽然无线电波是无孔不入的,但它也会被高山和高大的建筑物等所阻挡或减弱,所以天线一般都高架在空中。

因此,可以简单地为天线下个定义:“天线\footnote{这里光指收信天线。}是用来捕捉(接收)无线电波的,高张在空中的金属线”。

因为天线是收音机的第一道门户,所以天线的好坏对收音机工作的好坏有非常密切的关系,尤其对比较简单的收音机来说更为重要。例如矿石收音机和单管机,如果没有一根比较好的天线,就不能顺利地收音。

\section{天线的种类}

天线的种类很多,有的以天线外形来分(如T形,Г形等);有的用它的工作原理来分(如行波天线、同相天线等);有用它的工作效能来分的(有定向及非定向等);有以用途来分的。

以用途分大体可分接收天线和发射天线,接收天线就是指用于收音机收报机上的,是用来接收电波的。发射天线是用于发射机的,用以发射电波。

简单的常用接收天线有定向非定向,有T形、Г形、垂直、刷形、环形,有防干扰用的特种天等;其中Г形及T形天线,业余无线电爱好者用得最多。

Г形天线又叫倒L形天线,主要是由水平悬挂的天线和引下接至收音机用的引下线以绝缘子、拉线等附属装置组成。因为它的外形象俄文字母Γ字(或倒的英文字母L)而得名。

\begin{figure}[htbp]
	\centering
	\includegraphics[width=0.7\linewidth]{2}
	\caption{}
	\label{fig:1}
\end{figure}

Γ形天线略有方向性,他接收由引下线一端来的电波的能力最强。但这个性能不很显著,基本上属于非方向性天线。

T形天线和Γ形天线很相像,所不同的是它的引下线不是从天线的一端接出,而是从它的中间接出。这种天线虽然接收来自两头的电波的能力稍强一些,但也不明显,所以也是属于非方向性的。

\begin{figure}[htbp]
	\centering
	\includegraphics[width=0.7\linewidth]{3}
	\caption{}
	\label{fig:1}
\end{figure}

垂直天线及倾斜天线如下图所示。这两种天线没有水平部分,只有一条重直挂着的或斜挂着的铜线。

\begin{figure}[htbp]
	\centering
	\includegraphics[width=0.7\linewidth]{4}
	\caption{}
	\label{fig:1}
\end{figure}

\begin{figure}[htbp]
	\centering
	\includegraphics[width=0.7\linewidth]{5}
	\caption{}
	\label{fig:1}
\end{figure}

刷形天线或叫集中电容式天线。它用电容很大的刷状或螺状导线束来代替Г形、T形等的水平部分。

\begin{figure}[htbp]
	\centering
	\includegraphics[width=0.7\linewidth]{6}
	\caption{}
	\label{fig:1}
\end{figure}

室内天线及代用天线。最简单的室内天线为一条拖在收音机外面的几尺长的线段,稍考究些的可在天花板下面拉上一段导线,或螺旋形天线。

\begin{figure}[htbp]
	\centering
	\includegraphics[width=0.7\linewidth]{7}
	\caption{}
	\label{fig:1}
\end{figure}

\begin{figure}[htbp]
	\centering
	\includegraphics[width=0.7\linewidth]{8}
	\caption{}
	\label{fig:1}
\end{figure}

至于代用天线,它的种类形式就更多了。因为上面已经说过,无线电波不单能在挂着的导线上激起电压(或电流),在一切的导体上都能产生电压。所以如电话机、电灯、铁皮屋顶以至在铁床,铜网纱窗等都可当作代用天线。金属导体的面积愈大,与地的绝缘愈好(对高频电流而言),那么代天线的效果也就愈好。

\section{怎样架设天线}

架设哪一种天线好,这是初学的业余无电爱好者所迫切需要解决的问题,可是对于这问题也很难作出一个“放之四海而皆准”的答复。因为采用那一种天线,要看你所用的收普机和所处的环境来决定。一般来说,装一条Г形天线对于任何收音机都能适用。

\subsection{Γ形天线}

Г形天线是由一根长约10-30米的,高悬着的(约10-20米)多股铜线和引下线组成。当然,单纯从收音的音量强度和收音距离的观点来看,天线愈长愈高,效果也就愈好。但是太长太高了会带来很大的天电干扰、工业干扰及其他妨碍收音的杂音。若附近有大电台时更将会引起夹音,反而不能很好收听。所以天线的长短、高低要看具体环境而定。

一般地说,在附近没有大电台,没有工业干扰(如在村),收音机比较简单,那就应将天线架得长些、高些;若在工业干扰很大、附近电台林立(如在大城市)的地区,天线就应短些、低些。

对一般矿石收音机来说可用长25--30米的天线;二、三管再生式收音机可用15--20米的;超外差收音机可用8--10米,或甚至只用一条几米的短线段。

天线的导线是用由多股0.5--0.7公厘裸线绞合成的2--3公厘的绞合线,或不小于1.5--2公厘的单股铜(16号或14号紫铜线)。不可用黄铜线或铝线,因为它们会很快氧化而变成非常脆弱。在真正没有办法时用1.5--2公厘的镀锌铁线,也可勉强应用。

天线最好是用整条的线,若条件不许可时也可将几段接起来,但必须加以焊接(不要用带酸性的焊剂)。

除导线外,还有一种主要材料是“绝缘子”。绝缘子是不导电的,可以防止天线上的高频电流经杆子等漏入地中。

绝缘子有好多种(可在无线电料商店或电料行中购得),按照制造它们的材料来分,有玻璃的和瓷的,玻璃的比较好,两端孔是穿线用的。瓷绝缘子形如蛋形,故叫蛋形绝缘子,有大有小,收音机天线用直径一寸的就可以了。假如上面两种绝缘子都没有,也可用普通装电灯线的双孔瓷夹板或瓷壶(也叫鼓形白料,)来代替,不得已时也可用玻璃瓶,或甚至用白腊浸煮过的硬木头、纱线圈。

\begin{figure}[htbp]
	\centering
	\includegraphics[width=0.7\linewidth]{9}
	\caption{}
	\label{fig:1}
\end{figure}

若引下线要通过墙或窗口,那么还需要几个瓷套管或硬橡胶套管。

\begin{figure}[htbp]
	\centering
	\includegraphics[width=0.7\linewidth]{10}
	\caption{}
	\label{fig:1}
\end{figure}

此外两根高的10-20米的木杆或竹杆,以及若于2.4--4公厘的铁丝(作拉紧天线杆用)和一些螺钉、钉子等,若有一小滑轮就更好了。

在动手架设天线之前必须首先选择好场地。在选择时要考虑到便于利用邻近的房屋、树木等,应尽可能使天线水平部分不跨越屋顶、树木等,使天线水平部分有足够的空间,并且使它尽可能地高些。

其次应法意下列事项:

(1)不使天线和屋篇、墙壁、树木、自来水管、煤气管等相碰

(2)天线不要和电灯线、电话靠得太近或平行,也不要横越这些线上,特别是不要架在高压电力线的上面或下面,因为这样不但容易受到干扰,而且也非常危险。

(3)引人线最好不要贴着墙壁走得太长,也不要拐许多弯,引入线愈短愈好,这样可以减小损失。

为了减短天线杆的长度,可将天线杆固定在屋顶上、大树上或其他高建筑物上。若屋顶上不易装,那么就只好将杆子立在地上了。

为了防止天线上的高频电流经过杆子漏掉,在天线和杆子之间必须要加几个绝缘子,一般每端用两个到三个。它的接法如图。

\begin{figure}[htbp]
	\centering
	\includegraphics[width=0.7\linewidth]{11}
	\caption{}
	\label{fig:1}
\end{figure}

但不能如下图那样。因为那样接时会使绝缘子受到拉力而破碎。

\begin{figure}[htbp]
	\centering
	\includegraphics[width=0.7\linewidth]{12}
	\caption{}
	\label{fig:1}
\end{figure}

=====================================================================================================



\section{特种天线}


\chapter{地线}

地线的符号和装置方法见图2-3,它是一条把收音机电路和大地连接起来的导线。用长约1米的铁棒,插入地里,再用导线引入室内和收音机相接。插铁棒的地方还应该浇一点水,使铁棒和大地间导电良好。在有自来水设备的地方装置地线就更方便了,只要用一根导线把收音机电路和水管连接起来就可以了。

\begin{figure}[htbp]
	\centering
	\includegraphics[width=0.7\linewidth]{15}
	\caption{}
	\label{fig:1}
\end{figure}

天线和地线是用来接收无线电波的。如果把天线和地线连接起来,在无线电波的作用下,天线和地线之间便会有高频电流流动着。

\section{怎样装地线}

\chapter{耳机}

图2-4是耳机的符号、外形和构造。在耳机里有一个马蹄形的永久磁铁,在磁铁上绕着线圈,磁铁的磁极前有一片薄铁片。当通过线圈电流的方向和强度发生变化时,磁铁吸引铁片的力量也就发生变化,因此铁片振动起来,推动耳机内的空气,造成声波,声波传到我们的耳朵里,便听到了声音。

\begin{figure}[htbp]
	\centering
	\includegraphics[width=0.7\linewidth]{16}
	\caption{}
	\label{fig:1}
\end{figure}

耳机内的磁铁,磁性越强,发出来的声音就越响、越逼真。

耳机由于线圈绕线的粗细和圈数不同,可分高阻抗(如800欧)和低阻抗(如8欧)两种,本书用的都是高阻抗耳机。

耳机是否损坏,可用万用电表测量电阻的Rx100档来测试(图2-5a),在正常的情况下,测得高阻抗耳机的线圈电阻,应该是800欧左右,如果表针不动,那就说明已经断线了。更简便的方法是用一节干电池和耳机串联起来,把耳机引出线的一头在电池的电极上刮来刮去,耳机内能发出嚓嚓的声音(图2-5b)。如果一点声音也没有,就说明线圈或引出线已经断了。

\begin{figure}[htbp]
	\centering
	\includegraphics[width=0.7\linewidth]{17}
	\caption{}
	\label{fig:1}
\end{figure}

\chapter{二极管}

作为检波器,利用半导体的“单向导电性”,对天线接收到的高频信号进行‌检波‌(将调幅信号的高频载波去掉,提取音频信号)。

早期是使用矿石,这就是矿石收音机中“矿石”二字的来源。以前的检波器叫矿石检波器,有活动矿石和固定矿石两种。活动矿石和固定矿石结构类似,核心就是矿石。常见有黄铁矿、方铅矿、红锌矿、自然铜等的金属矿石,结构如图。

\begin{figure}[htbp]
	\centering
	\includegraphics[width=0.7\linewidth]{矿石}
	\caption{}
	\label{fig:1}
\end{figure}

目前没有必要,价格较高,调整太麻烦,且效果不理想,采用晶体二极管替代。型号可选 2AP9 、1N60P 等,根据经验,1N60P 是个不错的选择。

\begin{figure}[htbp]
	\centering
	\includegraphics[width=0.7\linewidth]{二极管}
	\caption{}
	\label{fig:1}
\end{figure}

二极管的符号和外形如图2-6。二极管的种类很多,这里用的是2AP9型检波二极管。二极管的符号形象地说明了它单方向导电的性能:符号箭杆这一端是正极(+),箭头所指这一端是负极(-),电流只能从正极流向负极。

\begin{figure}[htbp]
	\centering
	\includegraphics[width=0.7\linewidth]{18}
	\caption{}
	\label{fig:1}
\end{figure}

在电路里,二极管就担负起“指挥交通”的任务,使原来应该一来一往流动的交流电,变成只来不往的单方向电流。

二极管质量的好坏可以用万用表的RX100档来测试。把正表棒(红色表棒)接二极管的负极,负表棒(黑色表棒)接二极管的正极,测得的“正向电阻”应该在500欧姆以下;反过来把正表棒接二极管的正极,负表棒接二极管的负极,测得的“反向电阻”应该在200千欧以上;这样的管子才是合用的。如果测得的正、反向电阻都很大或都很小,则说明管子质量很差,不能应用。应用万用表,还可以判别二极管的正负极:测得正向电阻的那一次,负表棒所接触的是二极管的正极,正表棒接触的是负极(图2-7)。

\begin{figure}[htbp]
	\centering
	\includegraphics[width=0.7\linewidth]{19}
	\caption{}
	\label{fig:1}
\end{figure}

如果没有万用表,应用电池和耳机,也可以大致判别二极管的好坏:先把二极管、电池和耳机像图2-8a那样连接起来,用耳机的一条引出线去刮二极管的负极,耳机里当能发出较响的嚓嚓声;再把二极管的正负极反过来,像图2-8b那样连接,用耳机的一条引出线去刮二极管的正极时,耳机里只能发出极轻的声音来。这样的二极管才是好的。如果两次试验声音都很响或都极轻,这个管子就不能用了。同样,这种方法也可以用来判别二极管的极性:在声较响的那次试验中,电池正极所接的正是二极管的正极。

\begin{figure}[htbp]
	\centering
	\includegraphics[width=0.7\linewidth]{20}
	\caption{}
	\label{fig:1}
\end{figure}

\part{经典收音机设计}

\chapter{矿石收音机}

矿石收音机是一种最简单,最济的收音机,需用器材不多,制作容易,不需要维持费用。这种简单而又经济的收音机,在装置上和检修上都不需要特殊的技术,最合于一般初学的无线电爱好者研究之用。

虽然矿石收音机也有它的缺点,如收程不远,声音小等,但是我国各省现在都已普遍地设立了人民广播电台,故在很大的地区内,矿石收音机仍可使用的。

\section{最简单收音机}

初次动手,先来试装一个十分简单的收音机。图2-1是它的电路图和实体接线图,它由天线、地线、耳机和晶体二极管组成。电路非常简单,一装就响。它不需要外接电源,完全依靠天线接收的无线电波能量工作。适用于接收 AM(调幅)电台的信号,AM波段是530kHz - 1600kHz 。

\begin{figure}[htbp]
\centering
\includegraphics[width=0.7\linewidth]{13}
\caption{}
\label{fig:1}
\end{figure}

\subsubsection{各部件功能}

以下是最简单收音机各部件的简要功能(详细介绍请参考前面的专门章节):

1. \textbf{天线}
   接收空间中传播的无线电波(如 AM 中波广播信号),在天线中感应出微弱的高频交变电流。对于最简单收音机,可在两个绝缘竿子之间横拉一根电线,高度越高越好,但需要注意防雷等问题。楼上的话,用个鱼竿,甩下一根电线既可,离开墙体越远越好。

2. \textbf{晶体二极管}
   作为检波器,利用半导体的"单向导电性"对高频信号进行检波,提取音频信号。早期使用矿石,这就是矿石收音机中"矿石"二字的来源,现在采用晶体二极管替代,型号可选 2AP9 、1N60P 等。

3. \textbf{耳机}
   将检波后的微弱音频信号转化为声波。需要使用高阻耳机(阻抗通常≥2000Ω),若使用低阻耳机需配合阻抗匹配变压器。

4. \textbf{地线}
   连接到大地,作为电路的回路参考点,使天线感应的电流形成完整通路,同时过滤部分杂波干扰。可通过插钢筋到地里、连接金属自来水管或卫生间等电位端子排实现。

在最简单收音机中,这四个部件直接连接,不需要额外的调谐电路或电源,依靠天线接收的无线电波能量工作。

\subsubsection{工作原理}

\paragraph{基本原理概述}

高频载波是高频交变电流(电流方向快速交替变化),音频信号以 "调幅" 的方式加载在载波上,表现为载波的振幅随音频信号变化。当电台发出的无线电波经过时,便会在架设的天、地线之间感应出一个高频信号电压来,里面包含着音频信号。在天、地线之间仅接有一个耳机时,由于信号的频率大大地高出音频范围,因此听不到声音。

在加接二极管后,高频电流从地线流向天线时,二极管正向导通,电流从二极管这一边流过;高频电流从天线流向地线时,二极管反向截止,电流只能从耳机这一边的电路里流过。就把音频信号从高频载波上"剥离"下来。流过耳机这一边的是一个方向不变,强度随时间而变的脉动高频电流。这个电流的包络线形状和音频信号完全一致。当音频信号电流流过耳机时,耳机的线圈具有一定的电感特性,对高频电流的阻抗很大,高频电流很难通过,相当于被"过滤"掉了。低频音频电流可以顺利通过线圈,最终发出人耳能听到的声音。

\paragraph{详细工作过程}

\subparagraph{1. 信号接收过程}

当广播电台发射的无线电波(AM调幅信号)经过天线时,会在天线中感应出微弱的高频交变电流。这些电流的频率与广播信号的载波频率相同(中波波段通常为535-1605kHz),但其振幅随音频信号变化而变化(调幅特性)。

天线和地线之间形成了一个接收回路,无线电波在这个回路中感应出高频电压,其大小取决于:
- 电台信号的强度
- 天线的长度和高度
- 天线与地线的质量
- 接收地点与电台的距离

\subparagraph{2. 检波原理(核心环节)}

二极管在这里起到了关键的\textbf{检波}作用:
- 二极管具有单向导电性,只允许电流从正极流向负极
- 当高频调幅信号加在二极管两端时,正半周信号能够通过,负半周信号被阻断
- 这样,双向的高频交变电流就变成了单向的脉动电流
- 这个脉动电流的包络线形状与原始音频信号完全一致

\textbf{检波过程示意图}:
1. 输入:高频调幅信号(载波频率很高,振幅随音频变化)
2. 经过二极管后:只有正半周信号通过,负半周被截断
3. 输出:单向脉动电流(包含音频成分和残留的高频成分)

\subparagraph{3. 音频信号的提取与转换}

脉动电流通过耳机时:
- 耳机的线圈具有电感特性,对高频电流的阻抗很大,相当于"过滤"掉了高频成分
- 音频频率相对较低(20Hz-20kHz),能够顺利通过耳机线圈
- 音频电流使耳机的音圈振动,带动膜片发出声音,还原出广播节目

\subparagraph{4. 能量来源}

这个电路的一个显著特点是\textbf{无源设计}:
- 整个电路没有任何电源,声音能量完全来自于天线接收的无线电波
- 因此,音量大小取决于:
  - 电台信号的强度
  - 天线的接收效率
  - 二极管的检波效率
  - 耳机的灵敏度

\subparagraph{5. 地线的作用}

地线在电路中起到至关重要的作用:
- 提供电路的参考电位,使天线感应的电流能够形成完整通路
- 增强接收效果,相当于"镜像天线",扩展了接收面积
- 过滤掉一部分杂波干扰,提高接收质量

\paragraph{信号流程总结}

1. \textbf{信号发射}:广播电台将音频信号调制到高频载波上,通过天线发射为无线电波
2. \textbf{信号接收}:接收天线感应到无线电波,产生高频调幅电流
3. \textbf{信号检波}:二极管将高频调幅电流转换为单向脉动电流(提取音频成分)
4. \textbf{信号转换}:耳机将脉动电流中的音频成分转换为声音
5. \textbf{能量传递}:从无线电波→天线电流→检波后电流→耳机声音

\paragraph{技术要点}

1. \textbf{二极管的选择}:
   - 应选择适合高频检波的二极管,如2AP9、1N60P等
   - 这些二极管具有较低的正向压降和较高的截止频率

2. \textbf{耳机的选择}:
   - 必须使用高阻耳机(阻抗通常≥2000Ω)
   - 低阻耳机(如8Ω、32Ω)需要配合阻抗匹配变压器使用

3. \textbf{天线的架设}:
   - 天线应尽可能长且高,离开墙体和其他导体
   - 最好使用多股铜线,以减少高频损耗
   - 天线两端应使用绝缘子,避免信号泄漏

4. \textbf{地线的安装}:
   - 地线应尽可能短且粗
   - 接地端应保持湿润,以降低接地电阻
   - 最好连接到金属水管或专门的接地装置

\paragraph{工作原理的数学模型}

从电路理论角度分析:

1. \textbf{接收信号}:天线感应的电压可表示为:
   \[ V_{ant} = V_0 \cos(2\pi f_c t) + V_a \cos(2\pi f_a t) \cos(2\pi f_c t) \]
   其中,\( f_c \) 是载波频率,\( f_a \) 是音频频率

2. \textbf{检波过程}:经过二极管后,输出电流为:
   \[ I_{out} = I_0 + I_a \cos(2\pi f_a t) \]
   其中,\( I_a \) 与音频信号成正比

3. \textbf{耳机响应}:耳机对高频成分呈现高阻抗,对音频成分呈现低阻抗,因此主要响应音频成分。

\subsubsection{制作步骤}

整块印制板(图1-2),划分为五区,这次试装在左下角的第2 区内进行。

(1)取长约1米的软接线一段,一端剥出2毫米左右的线头,焊接在电路板上,另一端剥去20毫米左右的线头与天线绞接(图 2-9a)。

(2)取长约1米的软接线,照图2-9b接入印制板,另一端与地线绞接。

(3)把耳机接入印制板(图 2-9c)。试听一下,耳机里寂然无声。

(4)把二极管的两条接脚用小刀刮清爽,搪好锡,根据印制板上的安装位置(图2-9d),弯折好接脚,焊入印制板。再来试听一下,哦!听到电台的播音了。

\begin{figure}[htbp]
\centering
\includegraphics[width=0.7\linewidth]{21}
\caption{}
\label{fig:1}
\end{figure}

\subsubsection{优缺点}

该收音装置实在太简单了,所以费用最低,且无需电源。以下是详细的优缺点分析:

\paragraph{优点}
- \textbf{结构简单}:仅需四个基本组件,制作容易
- \textbf{无需电源}:不消耗电能,环保节能
- \textbf{成本低廉}:所需元件价格便宜,甚至可以自制
- \textbf{学习价值}:是理解无线电原理的最佳入门电路

\paragraph{缺点}
- \textbf{选择性差}:无法筛选特定频率的电台,会同时收到多个频率的信号,声音混杂。
- \textbf{灵敏度低}:只能接收本地或附近强功率的电台
- \textbf{音量小}:能量完全来自天线接收的无线电波,音量有限
- \textbf{依赖环境}:接收效果受天线质量、地线质量和地理位置影响很大

\subsubsection{实际应用中的改进}

为了克服上述缺点,实际应用中常进行以下改进:

1. \textbf{增加调谐电路}:在天线和二极管之间加入LC并联谐振回路,实现选台功能
2. \textbf{使用高阻耳机}:提高音频转换效率,获得更大音量
3. \textbf{优化天线和地线}:使用更长的天线和更好的地线,提高接收效果
4. \textbf{添加放大电路}:使用晶体管或集成芯片对检波后的信号进行放大(需要电源)

早期的矿石收音机,会在天线和二极管之间加入可变电容 + 电感的调谐回路,通过调节电容改变谐振频率,从而选择想要收听的电台。

\subsection{电路原理}

我们用的电灯是交流电。交流电在电路里是一来一去地流动着的,电流的方向来去变化一次所需的时间叫做周期T,周期的长短是用秒做单位来计算的。交流电(市电)的交变周期是1/50秒。交流电每秒钟来去方向交变的次数叫做频率f,用赫兹做单位来计算,交流电的频率是50赫兹,就是说电流来去方向的变化是每秒钟50次。

\begin{figure}[htbp]
	\centering
	\includegraphics[width=0.7\linewidth]{22}
	\caption{}
	\label{fig:1}
\end{figure}

利用话简,可以把声波转换成相应的交流电,这种交流电通过耳机时,就能发出原来那样的声音来。人们的耳朵只能听到每秒钟振动20次到2万次的声波,振动得太快或太慢都是听不见的。因此把频率为20赫兹到20千赫兹的交流电流称做音频电流。

音频电流只能沿着导线流动。打电话时,电话线一断,电话就不通了。

\begin{figure}[htbp]
	\centering
	\includegraphics[width=0.7\linewidth]{23}
	\caption{}
	\label{fig:1}
\end{figure}

频率在100千赫兹以上的电流,叫做高频电流。高频电流能使周围的空间里产生无线电波,向四面八方传播开去,所以高频信号可以用来进行无线电通信。

\begin{figure}[htbp]
	\centering
	\includegraphics[width=0.7\linewidth]{24}
	\caption{}
	\label{fig:1}
\end{figure}

在无线电广播电台里,是把音频信号加于高频信号上,使高频信号电流的峰值随着音频信号而作忽大忽小的变化——这种信号叫做高频调幅信号。用这样的办法,叫高频信号带着音频信号,发射成为无线电波,由近及远地传播开去。

\begin{figure}[htbp]
	\centering
	\includegraphics[width=0.7\linewidth]{25}
	\caption{}
	\label{fig:1}
\end{figure}

当电台发出的无线电波经过我们这里时,便会在我们架设的天、地线之间感应出一个高频信号电压来。在天、地线之间仅接有一个耳机时,由于信号的频率大大地高出音频范围,因此听不到声音。

在加接二极管后,高频电流以地线流向天线时,二极管正向导通,电流从二极管这一边流过;高频电流从天线流向地线时,二极管反向截止。因此,电流只能从耳机这一边的电路里流过。流过耳机这一边的是一个方向不变,强度随时间而变的脉动电流。这脉动电流里,就包含了音频信号的成分。当音信号电流流过耳机时,耳机里就发出声音来了。这里二极管起着从高频信号电流中分离出音频信号来的作用,这种作用称做“检波”。

\begin{figure}[htbp]
	\centering
	\includegraphics[width=0.7\linewidth]{26}
	\caption{}
	\label{fig:1}
\end{figure}

这个收音装置毕竟太简单了,只有一个检波器,没有选择电台的能力。在耳机里时常夹杂着几个电台的播音声,使我们喜爱的节目也听不清楚了。解决这个问题的方法是在检波电路前加装「调谐电路」。

\begin{figure}[htbp]
	\centering
	\includegraphics[width=0.7\linewidth]{27}
	\caption{}
	\label{fig:1}
\end{figure}

\section{可调谐的最简单收音机}

为了解决简单收音机无法筛选特定频率电台的问题,需要在其基础上增加调谐电路。

\begin{figure}[htbp]
	\centering
	\includegraphics[width=0.7\linewidth]{单回路矿石机}
	\caption{}
	\label{fig:1}
\end{figure}

\subsection{电路组成元件}

| 元件 | 符号 | 功能 |
|------|------|------|
| 天线 | A | 接收空间中的无线电波信号 |
| 电感 | L | 与可变电容$C_1$组成调谐回路 |
| 可变电容 | $C_1$ | 调节回路谐振频率,用于选择不同电台 |
| 二极管 | D | 作为检波器,将高频信号转换为音频信号 |
| 固定电容 | $C_2$ | 滤波电容,去除检波后的高频成分 |
| 耳机 | E | 将音频信号转换为声音 |

\subsection{工作原理详解}

\subsubsection{信号接收与调谐(L-$C_1$回路)}

\textbf{天线接收}:天线A收集空间中的无线电波(交变电磁场),在天线导体中感应出微弱的交变电流,频率与广播电台的载波频率相同。

\textbf{调谐原理}:电感L和可变电容$C_1$组成串联谐振回路,其谐振频率公式为:
\[ f = \frac{1}{2\pi\sqrt{LC}} \]
当调节$C_1$时,回路的谐振频率随之改变。当谐振频率与某个广播电台的载波频率一致时,该频率的信号会在回路中产生最大的电流响应,而其他频率的信号则被抑制,从而实现选台功能。

\textbf{谐振特性}:谐振回路的Q值(品质因数)决定了选择性和灵敏度。Q值越高,选择性越好(区分相邻电台的能力强),但通频带越窄,可能影响音质。

\subsubsection{检波过程(二极管D)}

\textbf{调幅信号}:广播电台发送的是调幅(AM)信号,即音频信号(信息)被调制到高频载波上,表现为载波幅度随音频信号变化。

\textbf{二极管检波}:二极管D利用其单向导电性进行检波。当高频调幅信号施加到二极管两端时:
- 正半周:二极管导通,电流通过
- 负半周:二极管截止,无电流通过

这样,高频信号被"切割"成单向的脉动电流,其包络线恰好对应原始的音频信号。

\textbf{检波效率}:检波效率取决于二极管的正向导通电阻和反向漏电流。理想的检波二极管应具有低正向电阻和高反向电阻,锗二极管(如1N60)是常见选择。

\subsubsection{滤波与音频输出($C_2$与E)}

\textbf{滤波原理}:电容$C_2$的作用是滤除检波后残留的高频成分。根据电容的阻抗特性:
\[ X_C = \frac{1}{2\pi fC} \]
- 对高频成分:$f$高,$X_C$小,相当于短路
- 对音频成分:$f$低,$X_C$大,相当于开路

因此,高频成分被$C_2$旁路到地,而音频成分则流向耳机。

\textbf{耳机工作}:耳机E是一个电磁换能器,其内部有一个线圈和振膜。当音频电流通过线圈时,会产生交变磁场,与永久磁铁相互作用,带动振膜振动,从而发出声音。

\textbf{阻抗匹配}:耳机的阻抗应与检波后的信号源阻抗匹配,以获得最大的功率传输。矿石收音机通常使用高阻抗耳机(如2000Ω以上),因为检波输出的电压较高但电流较小。

\subsubsection{能量转换过程}

矿石收音机的核心特点是\textbf{无外部电源},完全依靠接收的无线电波能量工作:
1. 天线接收无线电波 → 转换为高频电流
2. 调谐回路选择特定频率 → 增强该频率信号
3. 二极管检波 → 将高频能量转换为音频能量
4. 耳机将音频能量 → 转换为声能

\subsection{元件参数选择与影响}

| 元件 | 参数选择 | 影响 |
|------|----------|------|
| 天线 | 越长越好 | 接收效果更好,但需考虑安装空间和安全性 |
| 电感L | 中波约100-200匝 | 电感量越大,谐振频率越低,适合接收中波低端频率 |
| 可变电容$C_1$ | 270pF(中波)或365pF(中波+短波) | 容量范围决定了接收频段 |
| 二极管D | 锗二极管(如1N60) | 正向压降小(约0.3V),检波效率高于硅二极管(约0.7V) |
| 滤波电容$C_2$ | 0.01μF-0.1μF | 容量过大可能导致音频信号损失,过小则滤波效果差 |
| 耳机E | 高阻抗(2000Ω-8000Ω) | 低阻抗耳机(如8Ω)需要变压器匹配 |

\subsection{实际制作注意事项}

1. \textbf{接地}:良好的接地(地线)对接收效果至关重要,可连接到金属水管或专门的接地极
2. \textbf{线圈绕制}:电感L通常用漆包线绕在绝缘骨架上,中波线圈约需100-200匝
3. \textbf{元件布局}:尽量减少元件间的分布电容,特别是调谐回路部分
4. \textbf{灵敏度提升}:可增加天线调谐回路(天线与主调谐回路耦合)以提高接收灵敏度
5. \textbf{环境影响}:附近的电力线、电子设备可能产生干扰,应选择远离干扰源的位置

\subsection{信号流程总结}

1. \textbf{信号接收}:天线A接收无线电波 → 转换为高频电流
2. \textbf{调谐选台}:L-$C_1$回路选择特定频率 → 增强该频率信号
3. \textbf{检波解调}:二极管D将高频调幅信号 → 转换为音频脉动电流
4. \textbf{滤波处理}:电容$C_2$滤除高频成分 → 保留纯净音频信号
5. \textbf{声音输出}:耳机E将音频信号 → 转换为声音

\subsection{技术特点}

- \textbf{无电源设计}:完全依靠无线电波的能量工作,节能环保
- \textbf{结构简单}:仅由几个基本元件组成,易于理解和制作
- \textbf{灵敏度较低}:由于没有放大电路,接收信号的能力有限,通常需要较长的天线
- \textbf{单声道接收}:只能接收调幅(AM)广播信号
- \textbf{教学价值}:清晰展示了无线电接收的基本原理,是电子学入门的理想实验项目

\subsection{LC 电路元器件参数详解}

调谐电路由可变电容 $C_1$ 和电感线圈 L 组合。根据业内前辈的经验,电容和电感的选择需要匹配中波广播频率范围(535-1605kHz)。下面详细介绍LC相关的计算方法:

\paragraph{核心公式}
LC 谐振频率公式决定谐振频率的公式是:
\[ f=\frac{1}{2\pi\sqrt{LC}} \]
其中:
-  f  是频率,单位赫兹 (Hz)
-  L  是电感,单位亨利 (H)
-  C  是电容,单位法拉 (F)

我们的目标是通过已知的  f  和  C  来求解  L  。将公式变形为:
\[ L=\frac{1}{(2\pi f)^{2}C} \]

\paragraph{参数选择}

\textbf{频率  f  的选择}
中波波段从 525kHz 到 1605kHz。为了在整个波段内获得均匀的调谐体验(即旋钮转动角度与频率变化大致线性),不取算术平均值,而是取几何平均值作为设计中心频率。
\[ f_{center}=\sqrt{525\times1605} \text{ kHz} \approx \sqrt{842625} \text{ kHz}\approx918 \text{ kHz} \]
在实际工程中,为了方便计算和留有余量,通常会取一个接近的整数值,例如 1000kHz (1MHz) 作为设计基准。这样能确保高低两端都能较好地覆盖。

\textbf{电容  C  的选择}
可变电容从  $C_{min}$  (~15pF) 变化到  $C_{max}$  (360pF)。同样,为了匹配频率的几何中心,电容也应取几何平均值(或近似值)作为与中心频率对应的设计中心电容。
\[ C_{center}\approx \sqrt{C_{min}\times C_{max}} = \sqrt{15\times360} \text{ pF}  = \sqrt{5400} \text{ pF} \approx 73.5 \text{ pF} \]
在实际电路中,由于存在线路分布电容、微调电容等,这个值会更大一些。一个更常用的经验中心值大约是 100pF 到 120pF。

\paragraph{代入计算}
采用经典的工程参数: $f = 1 \text{ MHz}$  和  $C = 100 \text{ pF}$  来进行计算。

将单位转换为标准单位:
-  $f = 1 \text{ MHz} = 1 \times 10^6 \text{ Hz}$ 
-  $C = 100 \text{ pF} = 100 \times 10^{-12} \text{ F} = 1 \times 10^{-10} \text{ F}$ 

代入公式:
\[ L=\frac{1}{(2\pi\times1\times10^{6})^{2}\times(1\times10^{−10})} =\frac{1}{(6.2832\times10^{6})^{2}\times10^{−10}} =\frac{1}{(3.9478\times10^{13})\times10^{−10}}=\frac{1}{3.9478\times10^{3}} \approx 2.533\times10^{−4} \text{ H} \]

将亨利 (H) 转换为微亨 (μH):
\[ 2.533\times10^{−4} \text{ H} = 253.3 \text{ μH} \]

计算结果约 253μH,非常接近经典的 270μH。

\paragraph{为什么最终是 270μH?}

1. \textbf{标准化与余量}:270μH 是一个标准化的、常见的电感值。它比我们计算出的 253μH 略大一点,这会使整个调谐曲线向低频端略有偏移,确保能完全覆盖从 525kHz 开始的最低端频率。

2. \textbf{分布电容的影响}:实际电路中,线圈本身匝间、线路、元器件都会引入额外的"分布电容"(可能高达十几到几十 pF)。这个分布电容相当于并联在可变电容两端,使总电容增大。为了补偿这个影响,需要将电感值适当增大一些(因为 L 增大和 C 增大对频率的影响方向相反)。

3. \textbf{微调电容的配合}:在实际收音机中,与可变电容并联的还有一个半可调的微调电容器(约 5-30pF),与电感并联的还有一个磁芯可调的线圈。通过调整它们,可以将覆盖范围精确地"校准"到标准的 525kHz - 1605kHz。

\paragraph{总结}
270μH 的电感值不是凭空而来的,它是通过以下步骤确定的:
1. \textbf{目标}:覆盖中波波段 (525-1605kHz)。
2. \textbf{原理}:利用 LC 谐振公式  $f=\frac{1}{2\pi\sqrt{LC}}$ 。
3. \textbf{计算}:选取中心频率约 1MHz 和中心电容约 100pF 作为设计点,计算出电感约 253μH。
4. \textbf{工程化}:根据元件标准化、分布电容影响和留出调整余量的需要,将电感值定为 270μH 这个经典值。

这个"360pF 可变电容 + 270μH 电感"的组合,是几十年来中波收音机调谐回路的一个黄金标准配方。

\subsection{常见变种与改进}

- \textbf{抽头线圈}:为提高匹配效率,L 常绕制多个抽头,天线接在不同位置以调整阻抗。
- \textbf{倍压检波}:使用两个二极管和两个电容组成倍压检波电路,可提高输出音量。
- \textbf{加简易放大}:在检波后增加一个晶体管放大电路(单管机),可大幅提高灵敏度和音量,但需电源供电。

\subsection{调试要点}

调整 $C_1$ 选台,若信号弱或噪声大,可尝试优化天线/地线,或微调 L 的抽头位置(线圈 L 可以有抽头)。

\section{双回路收音机}

在二极管检波电路前,加装一个由线圈和可变电容器构成的调谐电路,就成为一具可以选择电台的收音机了。这具收音机的特点是用不着装干电池,所以称做无电源收音机。


图3-1是本机电路图和实体接线图。

\begin{figure}[htbp]
	\centering
	\includegraphics[width=0.7\linewidth]{28}
	\caption{}
	\label{fig:1}
\end{figure}

\subsection{元件介绍与制作}

\subsubsection{线圈}

$L_{1}$和$L_{2}$是同绕在一条直径10毫米,长140毫米磁棒上的两个线圈。线圈的符号,形象地表示了它是用导线一圈一圈地绕制而成的。旁边的一段虚线,表示穿在线圈内的磁棒(或其他形状的磁芯)。

制作方法:先用牛皮纸或信封纸做两个内径11毫米、长约40毫米的纸管,套在磁棒上,要能随意前后移动。用φ0.07x7丝包线(是一种用直径0.07毫米的漆包线7股合成的丝包线),$L_{1}$绕60匝,$L_{2}$绕40匝。绕制时可按图3-2所示的方法,在线圈下预垫一条纱线(或纸条),用以固定线圈头尾两端,绕好后才不致松散开来。

\begin{figure}[htbp]
	\centering
	\includegraphics[width=0.7\linewidth]{29}
	\caption{}
	\label{fig:1}
\end{figure}

通过线圈的电流发生变化时,会在线圈里感应出一个电动势来阻碍电流的变化。表示线圈内电流的变化率为每秒1安培时,能感应出多少伏电动势来的量,叫做电感,单位亨利,简称亨(H)。无线电电路里所用的线圈,电感都较小,常用毫亨(mH)、微亨($\mu$H)做单位来计算:

1亨=1000毫亨,
1毫亨=1000微亨。

一般地说,线圈的圈数较多,电感较大,线圈里加入磁芯,可以大大地增强电感。套在磁棒上的线圈,移近磁棒的中点,电感增大,移向磁棒的一头,电感减小。$L_{1}$的电感约为300微亨,$L_{2}$约为130微亨。

\begin{figure}[htbp]
	\centering
	\includegraphics[width=0.7\linewidth]{30}
	\caption{}
	\label{fig:1}
\end{figure}

两个靠近的线圈,其中一个线圈内的电流发生变化时,会在另一个线圈里感应出电动势来。把这种作用称做互感。两个线圈间互感的大小,也用亨利做单位来计算。把线圈接在直流电路里,通过线圈的电流,方向和强度都不变,不会感应出电动势来。它的作用,等于用一条导线把电路接通,直流电流可以十分顺利地通过线圈。

把线圈接在交流电路里,情况就不同了。因为通过线圈的电流时刻在变化,所以会感应出一个和电源电压相反的电动势来阻碍电流的通过。把这种作用称做电感电抗(简称感抗)。交流电的频率越高,线圈的电感越大,所产生的感抗也越大,交流电流就越难于通过。

\begin{figure}[htbp]
	\centering
	\includegraphics[width=0.7\linewidth]{31}
	\caption{}
	\label{fig:1}
\end{figure}

\subsubsection{电容器}

电容器的种类很多,形状也是各样的,但它们的结构是一样的,都是由两片导电的极片当中夹一层介质(不导电的物质)构成的。

\begin{figure}[htbp]
	\centering
	\includegraphics[width=0.7\linewidth]{32}
	\caption{}
	\label{fig:1}
\end{figure}

顾名思义,电容器就是能贮藏电荷的容器。如果把电容器的两极片与直流电源的正负极相接,电容器里就充入了电荷。这时移去电源,电容器里的电荷也不会马上消失。如果把两极片短路一下,电容器里的电荷就会很快地释放掉。

\begin{figure}[htbp]
	\centering
	\includegraphics[width=0.7\linewidth]{33}
	\caption{}
	\label{fig:1}
\end{figure}

表示电容器两端加每伏电压能充入多少库仑电荷的量,叫做电容量。单位法拉,简称法(F),在无线电电路里,常用更小的单位微法($\mu$F),微微法(pF)来计算:

1法拉=1000000微法,
1微法=1000000微微法。

直流电流不能通过电容器,因为它的两极片间隔着一层介质。

交流电可以通过电容器。但电容器对交流电的通过有一定的阻碍作用——叫做电容电抗,简称容抗。对于同频率的信号来说,电容越大,容抗越小,越容易让信号电流通过;对于同样大小的电容来说,信号的频率越高,容抗越小,越容易让信号电流通过。

\begin{figure}[htbp]
	\centering
	\includegraphics[width=0.7\linewidth]{34}
	\caption{}
	\label{fig:1}
\end{figure}

在本机里$C_{0}$、$C_{2}$是两个瓷片电容。瓷片电容两极片间的介质是一种特殊的陶瓷。应用这种陶瓷作为介质,电容器的体积可以做得很小。$C_{0}$的电容量是20pF,$C_{2}$的电容量是0.01$\mu$F。这两个电容器的电容量是固定不变的。$C_{1}$是可变电容器,它有两组极片:一组叫定片,是固定不动的;一组叫动片,和转轴相连,转动轴柄,动片就连着转动。就用这样的方法来调节两组极片重合部分的面积(图3-7画有斜线部分),重合部分的面积越多,电容量就越大;重合部分的面积越少,电容量就越小。两组极片完全重合时电容量最大,为270pF。

\begin{figure}[htbp]
	\centering
	\includegraphics[width=0.7\linewidth]{35}
	\caption{}
	\label{fig:1}
\end{figure}

测量1$\mu$F以下小电容的电容量,要有专用的仪器。装置简易收音机用的电容,对电容量的正确性要求不高,只要极间不漏电、内部没有短路或开路就可用了。是否漏电或短路,可用万用电表的Rx1K档来测试,表针不动,说明没有漏电和短路。内部是否开路,可利用前次装置的检波电路来测试:把被测电容串联在天线和检波电路间,如照常能收到广播,即说明电容器内部没有开路(图 3-8)。

\begin{figure}[htbp]
	\centering
	\includegraphics[width=0.7\linewidth]{36}
	\caption{}
	\label{fig:1}
\end{figure}

\subsection{装置和调试}

在装置前先要把新加入的元件逐一测试过,确知没有损坏时才可加以应用。双联可变电容和磁棒都装置在印制板第1区内,检波电路仍装在第2区。焊接前先做好下列三件准备工作:

(1)把前次接到印制板上去的耳机和天线、地线的接线拆下来,二极管仍让它装在板上不动。

(2)把双联可变电容用螺丝固定在底板上。螺丝长度应在5毫米以下,螺丝太长,旋进去时会把电容器的极片顶坏。三条双联极片的接脚穿过底板上的小孔,等待焊接。

(3)把塑料磁棒架装在底板上,插好磁棒。把两个线圈的线头都搪好锡,然后套在磁棒上。

\begin{figure}[htbp]
	\centering
	\includegraphics[width=0.7\linewidth]{37}
	\caption{}
	\label{fig:1}
\end{figure}

焊接分下面三个步骤进行(图3-10和图3-11):

\begin{figure}[htbp]
	\centering
	\includegraphics[width=0.7\linewidth]{38}
	\caption{}
	\label{fig:1}
\end{figure}

(1)焊接调谐回路

把从小孔中穿过来的三条双联可变电容器的接脚弯过来,紧贴铜箔表面,并用焊锡焊牢;再把$L_{1}$的两端分别与双联可变电容器左面的接脚($C_{1}$的定片)及中间的接脚($C_{1}$的动片)相焊接。双联中右边的接脚(即还有一联可变电容器的定片)让它空着,留待后用。

(2)焊接检波电路

把$L_{2}$的一端接二极管正极,另一端就近接“地”。然后再接入一个0.01$\mu$F的电容$C_{2}$和耳机EJ。

(3)接天线、地线

把一只22pF的电容$C_{0}$接于$C_{1}$定片和接天线的焊点间。取一段长约1米的导线,一端与$C_{0}$相接,另一端与天线绞接;再取一段长约1米的导线,一端焊接在$C_{1}$动片接点附近,另一端与地线绞接(图3-11)。

\begin{figure}[htbp]
	\centering
	\includegraphics[width=0.7\linewidth]{39}
	\caption{}
	\label{fig:1}
\end{figure}

至此,焊接工作已全部完成。旋动可变电容器的轴柄,就能选择收听的电台。如果声音较响但夹有别的电台的播音声,可把$L_{1}$和$L_{2}$之间的距离拉开一点试试。如果声音太轻可把$L_{1}$和$L_{2}$之间的距离靠近一点试试。

\begin{figure}[htbp]
	\centering
	\includegraphics[width=0.7\linewidth]{40}
	\caption{}
	\label{fig:1}
\end{figure}

\subsection{电路原理详解}

\subsubsection{电路组成元件}

| 元件 | 符号 | 参数 | 功能 |
|------|------|------|------|
| 天线 | TX | - | 接收空间中的无线电波信号 |
| 地线 | DX | - | 提供参考电位,形成电流回路 |
| 微调电容 | $C_0$ | 20pF | 微调调谐回路,补偿分布电容 |
| 可变电容 | $C_1$ | 270pF | 调节主调谐回路的谐振频率,实现选台 |
| 主线圈 | $L_1$ | - | 与$C_1$组成主调谐回路,选择特定频率信号 |
| 次级线圈 | $L_2$ | - | 与$L_1$耦合,感应并传递信号到检波电路 |
| 检波二极管 | D | 2AP9 | 利用单向导电性将高频信号转换为音频信号 |
| 滤波电容 | $C_2$ | 0.01μF | 滤除检波后残留的高频成分 |
| 耳机 | EJ | 高阻抗 | 将音频信号转换为声音 |

\subsubsection{工作原理详解}

\paragraph{1. 信号接收与主调谐回路($L_1$-$C_1$-$C_0$)}

\textbf{天线接收}:天线TX接收空间中的无线电波,在天线中感应出微弱的高频交变电流,包含多个广播电台的信号。

\textbf{主调谐回路}:$L_1$与$C_1$、$C_0$组成并联谐振回路,其谐振频率公式为:
\[ f = \frac{1}{2\pi\sqrt{L_1(C_1+C_0)}} \]

当调节$C_1$时,回路的谐振频率随之改变。当谐振频率与某个广播电台的载波频率一致时,该频率的信号会在$L_1$中产生最大的感应电流,而其他频率的信号则被抑制,从而实现选台功能。

\textbf{微调电容$C_0$}:用于微调谐振频率,补偿电路中的分布电容,确保调谐精度。

\paragraph{2. 信号耦合($L_1$-$L_2$)}

\textbf{互感耦合}:$L_1$和$L_2$绕在同一磁棒上,通过互感作用将$L_1$中选择的高频信号耦合到$L_2$。这种耦合方式的优点是:
- 实现了天线回路与检波回路的隔离
- 可以通过调整线圈位置改变耦合度
- 提高了选择性和灵敏度

\textbf{耦合系数}:两线圈的耦合系数取决于它们的相对位置和匝数比。耦合系数越大,信号传输效率越高,但选择性可能下降。

\paragraph{3. 检波过程(二极管D)}

\textbf{调幅信号处理}:从$L_2$感应的是高频调幅信号,其振幅随音频信号变化。

\textbf{二极管检波}:2AP9锗二极管利用单向导电性进行检波:
- 正半周:二极管导通,电流通过
- 负半周:二极管截止,无电流通过

这样,高频调幅信号被转换为单向脉动电流,其包络线对应原始的音频信号。

\textbf{检波效率}:2AP9二极管正向压降小(约0.3V),适合微弱信号检波,检波效率较高。

\paragraph{4. 滤波与音频输出($C_2$与EJ)}

\textbf{滤波原理}:$C_2$的作用是滤除检波后残留的高频成分。根据电容阻抗特性:
\[ X_C = \frac{1}{2\pi fC} \]
- 对高频成分:$f$高,$X_C$小,相当于短路
- 对音频成分:$f$低,$X_C$大,相当于开路

因此,高频成分被$C_2$旁路到地,而音频成分则流向耳机。

\textbf{耳机工作}:EJ是高阻抗耳机(通常2000Ω以上),将音频电流转换为声音。高阻抗设计匹配检波输出的高电压低电流特性。

\subsubsection{能量转换过程}

双回路矿石收音机同样采用无电源设计,能量转换过程为:
1. 天线接收无线电波 → 转换为高频电流
2. 主调谐回路选择特定频率 → 增强该频率信号
3. 线圈耦合 → 将信号传递到检波回路
4. 二极管检波 → 将高频能量转换为音频能量
5. 耳机将音频能量 → 转换为声能

\subsubsection{元件参数选择分析}

\paragraph{调谐回路参数}
- \textbf{可变电容$C_1$(270pF)}:与$L_1$配合,覆盖中波广播频段(535-1605kHz)
- \textbf{微调电容$C_0$(20pF)}:补偿分布电容,微调谐振点
- \textbf{线圈$L_1$和$L_2$}:通常绕在磁棒上,$L_1$约270μH,$L_2$匝数较少(约为$L_1$的1/3-1/4)

\paragraph{检波与滤波参数}
- \textbf{二极管D(2AP9)}:锗二极管,正向压降小,适合高频检波
- \textbf{滤波电容$C_2$(0.01μF)}:容量适中,既能滤除高频,又能保留音频成分

\subsubsection{技术特点与优势}

1. \textbf{双回路设计}:相比单回路矿石收音机,具有更好的选择性,能更清晰地分离相邻电台
2. \textbf{磁棒线圈}:使用磁棒提高了线圈的电感量和Q值,增强了接收灵敏度
3. \textbf{微调电容}:提高了调谐精度,便于精确选台
4. \textbf{线圈耦合}:实现了回路隔离,减少了负载对调谐回路的影响
5. \textbf{无电源设计}:节能环保,结构简单,易于制作

\subsubsection{实际制作与调试要点}

1. \textbf{线圈绕制}:
   - $L_1$:在磁棒上用0.1mm漆包线绕约80-100匝
   - $L_2$:在$L_1$旁边绕约20-30匝,两者间距可调整以改变耦合度

2. \textbf{元件布局}:
   - 线圈应远离金属物体,减少能量损失
   - 可变电容应安装在方便调节的位置
   - 二极管和电容应尽量靠近,减少引线电感

3. \textbf{调试方法}:
   - 调整$C_1$选择电台
   - 调整$L_1$和$L_2$的相对位置改变耦合度
   - 调整$C_0$微调谐振点,获得最佳音量
   - 优化天线和地线,提高接收效果

4. \textbf{常见问题与解决}:
   - 音量小:检查天线接地,调整线圈耦合度
   - 选择性差:增加$L_1$匝数,减小$L_2$匝数,降低耦合度
   - 杂音大:检查地线连接,增加天线高度

\subsubsection{信号流程总结}

1. **信号接收**:天线TX接收无线电波 → 感应出高频电流
2. **调谐选台**:$L_1$-$C_1$-$C_0$回路谐振选择特定频率 → 增强该频率信号
3. **信号耦合**:$L_1$通过互感将信号传递给$L_2$ → 隔离负载影响
4. **检波解调**:二极管D将高频调幅信号 → 转换为音频脉动电流
5. **滤波处理**:电容$C_2$滤除高频成分 → 保留纯净音频信号
6. **声音输出**:耳机EJ将音频信号 → 转换为声音

\subsubsection{性能对比}

| 性能指标 | 单回路矿石机 | 双回路矿石机 |
|----------|--------------|--------------|
| 选择性 | 一般 | 良好 |
| 灵敏度 | 一般 | 较高 |
| 音量 | 较小 | 较大 |
| 结构复杂度 | 简单 | 中等 |
| 调试难度 | 容易 | 中等 |

双回路矿石收音机通过增加次级线圈和微调电容,显著提高了选择性和灵敏度,是矿石收音机的经典设计之一,非常适合无线电爱好者入门制作和学习。

\subsection{原有电路原理说明}

(1)调谐电路

$L_{1}$和$C_{1}$组成的回路,叫做调谐电路,也叫选择器。

选择器怎样选择电台呢?

在我们周围有许多电台在发射不同频率的电磁波,这些电磁波都会在收音机的天线上感应出各种不同频率的信号电流。只有当L、C组成的调谐回路的振荡频率与外来的信号频率一致时(这种情况叫做谐振),这个频率的电台信号才能优先通过,并得到增强后检波,还原成声音。其他不是这个频率的无线电波就被禁止通行。只要改变调谐回路的频率,就可以分别和各种电台的频率信号一一谐振,这就起到了选择电台的作用。

在$L_{1}$、$C_{1}$的回路里,谐振频率与电感、电容的大小有关:电感、电容越大,谐振频率越低;电感、电容越小,谐振频率越高。$L_{1}$向磁棒中心移,电感增大,回路的谐振频率减低;$L_{1}$向磁棒的一头移,电感减小,回路的谐振频率升高(图3-13)。$C_{1}$的轴柄顺时针方向旋转时,电容增大,回路的谐振频率减低;$C_{1}$的轴柄逆时针方向旋转时,电容减小,回路的谐振频率升高(图3-14)。和回路谐振的信号得到了加强,不谐振的信号遭到了削弱。因此在收音机里加入了调谐电路,不但可以用来选择电台,还起着提高灵敏度的作用。

\begin{figure}[htbp]
	\centering
	\includegraphics[width=0.7\linewidth]{41}
	\caption{}
	\label{fig:1}
\end{figure}

\begin{figure}[htbp]
	\centering
	\includegraphics[width=0.7\linewidth]{42}
	\caption{}
	\label{fig:1}
\end{figure}

(2)整机工作过程

天、地线接收到各电台信号,经电容$C_{0}$传到由$L_{1}$、$C_{1}$构成的调谐电路。天、地线之间也存在着一定大小的电容,如果把天、地线直接和调谐电路相接,就会影响调谐电路的谐振频率,所以这里用一个较小的电容$C_{0}$,把天、地线之间的电容与调谐电路隔开。在调谐电路里,只有被选中的那个电台信号与

电路谐振,能产生较强的电流。那些不谐振的电台信号,被衰减到微不足道的程度。由于$L_{1}$和$L_{2}$之间有互感作用,被选中的电台信号就传递到了$L_{2}$。$L_{2}$输出的高频信号电压,通过电容$C_{2}$加于二极管D的两端,经过二极管检波,得到一音频信号电流。$C_{2}$是一个0.01$\mu$F的小电容,对音频电流的容抗很大,所以音频电流只能从耳机EJ里通过,这时耳机里就发出电台的播音声来了(图3-15)。

\begin{figure}[htbp]
	\centering
	\includegraphics[width=0.7\linewidth]{43}
	\caption{}
	\label{fig:1}
\end{figure}

\section{变压器耦合矿石收音机}

变压器耦合矿石收音机是矿石收音机的进阶设计,通过使用变压器实现调谐回路与检波回路的隔离,相比直接耦合的双回路矿石机具有更好的选择性和灵敏度。

\subsection{电路组成元件}

| 元件 | 符号 | 参数 | 功能 |
|------|------|------|------|
| 天线 | TX | - | 接收空间中的无线电波信号 |
| 地线 | DX | - | 提供参考电位,形成电流回路 |
| 微调电容 | $C_0$ | 20pF | 微调调谐回路,补偿分布电容 |
| 可变电容 | $C_1$ | 270pF | 调节调谐回路的谐振频率,实现选台 |
| 初级线圈 | $L_1$ | - | 与$C_1$组成调谐回路,选择特定频率信号 |
| 次级线圈 | $L_2$ | - | 通过变压器耦合,感应并传递信号到检波电路 |
| 检波二极管 | D | 2AP9 | 利用单向导电性将高频信号转换为音频信号 |
| 滤波电容 | $C_2$ | 0.01μF | 滤除检波后残留的高频成分 |
| 耳机 | EJ | 高阻抗 | 将音频信号转换为声音 |

\subsection{工作原理详解}

\subsubsection{信号接收与调谐($L_1$-$C_1$-$C_0$)}

\textbf{天线接收}:天线TX接收空间中的无线电波,在天线中感应出微弱的高频交变电流,包含多个广播电台的信号。

\textbf{调谐回路}:$L_1$与$C_1$、$C_0$组成串联谐振回路,其谐振频率公式为:
\[ f = \frac{1}{2\pi\sqrt{L_1(C_1+C_0)}} \]

当调节$C_1$时,回路的谐振频率随之改变。当谐振频率与某个广播电台的载波频率一致时,该频率的信号会在$L_1$中产生最大的感应电流,而其他频率的信号则被抑制,从而实现选台功能。

\textbf{微调电容$C_0$}:用于微调谐振频率,补偿电路中的分布电容,确保调谐精度。

\subsubsection{信号耦合($L_1$-$L_2$)}

\textbf{变压器耦合}:$L_1$和$L_2$绕在同一磁芯上,构成一个变压器。通过电磁感应原理,将$L_1$中选择的高频信号耦合到$L_2$。这种耦合方式的优点是:
- 实现了天线回路与检波回路的隔离
- 可以通过变压器的匝数比调整信号电平
- 提高了选择性和灵敏度

\textbf{耦合系数}:两线圈的耦合系数取决于它们的相对位置、匝数比和磁芯特性。耦合系数越大,信号传输效率越高,但选择性可能下降。

\subsubsection{检波过程(二极管D)}

\textbf{调幅信号处理}:从$L_2$感应的是高频调幅信号,其振幅随音频信号变化。

\textbf{二极管检波}:2AP9锗二极管利用单向导电性进行检波:
- 正半周:二极管导通,电流通过
- 负半周:二极管截止,无电流通过

这样,高频调幅信号被转换为单向脉动电流,其包络线对应原始的音频信号。

\textbf{检波效率}:2AP9二极管正向压降小(约0.3V),适合微弱信号检波,检波效率较高。

\subsubsection{滤波与音频输出($C_2$与EJ)}

\textbf{滤波原理}:$C_2$的作用是滤除检波后残留的高频成分。根据电容阻抗特性:
\[ X_C = \frac{1}{2\pi fC} \]
- 对高频成分:$f$高,$X_C$小,相当于短路
- 对音频成分:$f$低,$X_C$大,相当于开路

因此,高频成分被$C_2$旁路到地,而音频成分则流向耳机。

\textbf{耳机工作}:EJ是高阻抗耳机(通常2000Ω以上),将音频电流转换为声音。高阻抗设计匹配检波输出的高电压低电流特性。

\subsection{能量转换过程}

变压器耦合矿石收音机同样采用无电源设计,能量转换过程为:
1. 天线接收无线电波 → 转换为高频电流
2. 调谐回路选择特定频率 → 增强该频率信号
3. 变压器耦合 → 将信号传递到检波回路
4. 二极管检波 → 将高频能量转换为音频能量
5. 耳机将音频能量 → 转换为声能

\subsection{元件参数选择分析}

\subsubsection{调谐回路参数}
- \textbf{可变电容$C_1$(270pF)}:与$L_1$配合,覆盖中波广播频段(535-1605kHz)
- \textbf{微调电容$C_0$(20pF)}:补偿分布电容,微调谐振点
- \textbf{线圈$L_1$和$L_2$}:通常绕在磁芯上,$L_1$约270μH,$L_2$匝数较少(约为$L_1$的1/3-1/4)

\subsubsection{检波与滤波参数}
- \textbf{二极管D(2AP9)}:锗二极管,正向压降小,适合高频检波
- \textbf{滤波电容$C_2$(0.01μF)}:容量适中,既能滤除高频,又能保留音频成分

\subsection{技术特点与优势}

1. \textbf{变压器耦合设计}:相比直接耦合的矿石收音机,具有更好的选择性和灵敏度
2. \textbf{磁芯变压器}:使用磁芯提高了线圈的电感量和Q值,增强了接收效果
3. \textbf{微调电容}:提高了调谐精度,便于精确选台
4. \textbf{回路隔离}:变压器实现了调谐回路与检波回路的隔离,减少了负载对调谐的影响
5. \textbf{无电源设计}:节能环保,结构简单,易于制作

\subsection{实际制作与调试要点}

\subsubsection{变压器制作}
1. **使用高磁导率的磁芯**:如铁氧体或坡莫合金
2. **$L_1$和$L_2$的匝数比根据需要调整**:通常$L_1$匝数多于$L_2$
3. **线圈应紧密绕制**:减少匝间电容

\subsubsection{元件布局}
1. **线圈应远离金属物体**:减少能量损失
2. **可变电容应安装在方便调节的位置**
3. **二极管和电容应尽量靠近**:减少引线电感

\subsubsection{调试方法}
1. **调整$C_1$选择电台**
2. **调整$C_0$微调谐振点**:获得最佳音量
3. **优化天线和地线**:提高接收效果
4. **如信号过强**:可调整变压器耦合度

\subsubsection{常见问题与解决}
1. **音量小**:检查天线接地,调整变压器耦合度
2. **选择性差**:增加$L_1$匝数,提高Q值
3. **杂音大**:检查地线连接,增加天线高度

\subsection{信号流程总结}

1. **信号接收**:天线TX接收无线电波 → 感应出高频电流
2. **调谐选台**:$L_1$-$C_1$-$C_0$回路谐振选择特定频率 → 增强该频率信号
3. **信号耦合**:$L_1$通过变压器将信号传递给$L_2$ → 隔离负载影响
4. **检波解调**:二极管D将高频调幅信号 → 转换为音频脉动电流
5. **滤波处理**:电容$C_2$滤除高频成分 → 保留纯净音频信号
6. **声音输出**:耳机EJ将音频信号 → 转换为声音

\subsection{性能对比}

| 性能指标 | 直接耦合矿石机 | 变压器耦合矿石机 |
|----------|----------------|------------------|
| 选择性 | 一般 | 良好 |
| 灵敏度 | 一般 | 较高 |
| 音量 | 较小 | 较大 |
| 结构复杂度 | 简单 | 中等 |
| 调试难度 | 容易 | 中等 |

变压器耦合矿石收音机通过使用变压器实现了回路隔离和信号耦合,显著提高了选择性和灵敏度,是矿石收音机的进阶设计,适合对接收效果有更高要求的无线电爱好者。

\subsection{两种电路连接方式对比}

在变压器耦合矿石收音机的设计中,有两种常见的电路连接方式,各有优劣:

\subsubsection{第一种电路:$C_1$在左边与$L_1$组成调谐电路(双回路矿石机)}

\textbf{电路连接}:
```
天线 → $C_0$ → $L_1$-$C_1$调谐回路 → $L_2$ → 检波电路
```

\textbf{电路特点}:
- $L_1$与$C_1$组成并联谐振回路,直接参与调谐
- 天线通过$C_0$耦合到$L_1$-$C_1$回路
- $L_1$和$L_2$通过互感耦合

\textbf{线圈参数}:
- **$L_1$**:约80-100匝,电感量约270μH(与$C_1$配合调谐)
- **$L_2$**:约20-30匝,电感量约130μH(耦合线圈)

\textbf{优点}:
1. \textbf{灵敏度高}:天线直接耦合到调谐回路,信号损失小
2. \textbf{结构简单}:只需要一个调谐回路
3. \textbf{调试容易}:调整$L_1$、$L_2$间距即可改变耦合度

\textbf{缺点}:
1. \textbf{选择性一般}:天线负载影响调谐回路的Q值
2. \textbf{天线影响调谐}:天线电容的变化会影响谐振频率
3. \textbf{阻抗匹配}:天线阻抗与调谐回路阻抗匹配较难

\subsubsection{第二种电路:$C_1$在右边与LC组成调谐电路,$L_1$只接天地线}

\textbf{电路连接}:
```
天线 → $L_1$(天线线圈) → $L_2$-$C_1$调谐回路 → 检波电路
```

\textbf{电路特点}:
- $L_1$作为天线线圈,只连接天地线,不参与调谐
- $L_2$与$C_1$组成调谐回路
- $L_1$和$L_2$通过互感耦合

\textbf{线圈参数}:
- **$L_1$**:约20-30匝,电感量较小(作为天线线圈)
- **$L_2$**:约80-100匝,电感量约270μH(与$C_1$配合调谐)

\textbf{优点}:
1. \textbf{选择性好}:天线与调谐回路隔离,调谐回路Q值高
2. \textbf{调谐稳定}:天线电容不影响调谐频率
3. \textbf{阻抗匹配}:$L_1$可以优化天线阻抗匹配

\textbf{缺点}:
1. \textbf{灵敏度较低}:信号经过两次耦合,损失较大
2. \textbf{结构复杂}:需要更精确的线圈设计
3. \textbf{调试困难}:耦合度调整更复杂

\subsubsection{性能对比}

| 性能指标 | 第一种($L_1$-$C_1$调谐) | 第二种($L_2$-$C_1$调谐) |
|----------|---------------------|---------------------|
| 灵敏度 | 高 | 中等 |
| 选择性 | 一般 | 良好 |
| 调谐稳定性 | 一般 | 好 |
| 天线影响 | 较大 | 较小 |
| 结构复杂度 | 简单 | 中等 |
| 调试难度 | 容易 | 中等 |

\subsubsection{线圈参数差异}

**第一种电路($L_1$-$C_1$调谐)**:
- $L_1$(调谐线圈):匝数80-100匝,电感约270μH,作用是与$C_1$组成调谐回路
- $L_2$(耦合线圈):匝数20-30匝,电感约130μH,作用是将信号耦合到检波电路

**第二种电路($L_2$-$C_1$调谐)**:
- $L_1$(天线线圈):匝数20-30匝,电感较小(约50-100μH),作用是天线耦合和阻抗匹配
- $L_2$(调谐线圈):匝数80-100匝,电感约270μH,作用是与$C_1$组成调谐回路

\subsubsection{适用场景建议}

**第一种电路适合**:
- 信号较弱地区:需要高灵敏度
- 初学者:制作和调试简单
- 近距离接收:选择性要求不高

**第二种电路适合**:
- 信号较强地区:需要高选择性
- 电台密集地区:需要清晰分离相邻电台
- 进阶爱好者:追求更好的性能

\subsubsection{设计原理总结}

这两种电路设计代表了不同的设计哲学:

1. **第一种电路**:优先考虑灵敏度,适合信号较弱的环境
2. **第二种电路**:优先考虑选择性,适合信号较强但电台密集的环境

线圈参数确实不同,主要是**哪个线圈参与调谐**决定了它的电感量大小。参与调谐的线圈需要约270μH的电感量来配合270pF的电容覆盖中波频段(535-1605kHz)。

\chapter{单管收音机}

在无电源收音机的检波器后面,加装一级音频放大器,把输出功率放大,耳机里发出来的声音就响得多了。在离电台十公里以内的地区,用5-7米长的天线来收听,声音竟有些震耳呢!根据实际的需要,在这具单管机里也装上了控制音量的电位器。

图4-1是本机的电路图和实体接线图。

\begin{figure}[htbp]
	\centering
	\includegraphics[width=0.7\linewidth]{44}
	\caption{}
	\label{fig:1}
\end{figure}

\begin{figure}[htbp]
	\centering
	\includegraphics[width=0.7\linewidth]{45}
	\caption{}
	\label{fig:1}
\end{figure}

\section{元件介绍}

(1)$R_{1}$是5.1千欧的碳膜电阻。

(2)$R_{2}$也是一个碳膜电阻,它的阻值约100千欧,要由实验来决定。

$R_{1}$、$R_{2}$电阻的阻值是否正确,可用万用表来测试(图4-2):先把万用表的分选开关拨到适当的倍率档(例如测几千欧的电阻时应该拨到Rx1K档);然后把两条表棒互相紧密接触,转动零位调节旋钮,使表针指着0欧姆处;再把两根表棒分别与电阻两端接触,从$\Omega$栏刻度中读出数值,再乘以倍率数。

\begin{figure}[htbp]
	\centering
	\includegraphics[width=0.7\linewidth]{46}
	\caption{}
	\label{fig:1}
\end{figure}

(3)W 是4.7千欧的连开关电位器。在圆形的小壳内,装着一片弧形的碳膜电阻和一个滑臂,滑臂和转轴相连,旋动转轴可使滑臂在碳膜片上移动。图4-3中a、b是弧形电阻片两端的引出接脚,c是滑臂的接脚。一个简单的开关就装在壳子的下面。电位器的好坏要分两次测试:第一次测a-c间的电阻,要求转动滑臂时,万用表指针始终缓缓偏转,不会产生突然的跳动。第二次测试c-b间的电阻,要求也是这样。

\begin{figure}[htbp]
	\centering
	\includegraphics[width=0.7\linewidth]{47}
	\caption{}
	\label{fig:1}
\end{figure}

(4)$C_{3}$是33$\mu$F的电解电容器(图4-4)。电解电容器的两极片,是有正、负极性之分的。正极片的引出线长一些,在外壳上近正极处有一个“+”的记号;负极片引出线比较短一些。在线路图中,正极片的符号是一个矩形方框,负极片仍是一段直线。接到电路里去时,正负极性不可搞错,搞错了电容器容易击穿。

\begin{figure}[htbp]
	\centering
	\includegraphics[width=0.7\linewidth]{48}
	\caption{}
	\label{fig:1}
\end{figure}

测试这种电容器好坏的方法如下:测试前先把电容器的两条引出线短接一下,释放掉里面的残余电荷,然后把万用表(拨到Rx1K档)的黑表棒接电容器的正极片,红表棒接电容器的负极片(手指不要和表棒、电容器引出线接触),表针应当很快地挥向右面,然后缓缓退回无穷大处($\infty$处)。

测试时,表针向右挥动的角度越大,表示电容量越大。大约:30$\mu$F的电容可挥到7千欧处。100$\mu$F的电容可挥到2千欧处。如果不会挥动,则表示这个电容内部开路,或已经失效了。挥动的角度不够大,则说明电容量不足。如果表针退不到原处,说明电容器漏电。我们从表针停留的位置上可读出它的“漏电电阻”,漏电电阻大于200千欧的电容器,在简易收音机里还是可以应用的。

(5)$BG_{1}$是3AX31型低频三极管。三极管有三个电极:发射极e,基极b和集电极c(图4-5)。发射极的符号是一个箭头,它表明了电流的方向。电流从发射极流进三极管,这一路电流称做发射极电流$I_{e}$;发射极电流在三极管内分成两股:一股经基极流出三极管,叫做基极电流$I_{b}$;另一股经集电极流出三极管,叫做集电极电流$I_{c}$。

\begin{figure}[htbp]
	\centering
	\includegraphics[width=0.7\linewidth]{49}
	\caption{}
	\label{fig:1}
\end{figure}

$I_{b}$和$I_{c}$成一定的比例,一般$I_{c}$是$I_{b}$的20倍到200倍,倍数的大小,随管子而异。这个倍数叫做“电流放大系数”,用一个希腊字母$\beta$来表示。例如有一个3AX31型三极管,它的$\beta$是50,在$I_{b}$是0.1毫安时,$I_{c}$就是5毫安;$I_{b}$是0.2毫安时,$I_{c}$就是10毫安。但是$I_{b}$等于零时,$I_{c}$却还存在一点点电流,这是从发集极穿过基极“漏”到集电极去的,叫做“穿透电流”$I_{ceo}$(图 4-6)。

\begin{figure}[htbp]
	\centering
	\includegraphics[width=0.7\linewidth]{50}
	\caption{}
	\label{fig:1}
\end{figure}

图4-7说明了应用万用表检查3AX31,3AG1等三极管好坏的方法:把分选开关拨到Rx100的位置,进行如下的测试:

①把万用表的红表棒接三极管的b极,用黑棒去接e极和c极,测得的电阻都应该在100到500欧之间;

②把万用表的黑表棒去接三极管的b极,用红表棒去接c极和e极,测得的电阻都应该在200千欧以上。

测试下来符合上述情况的,表示管子没有损坏。如果c极或e极到b极的电阻,两次测下来,都是零欧姆或无穷大(表针不动),就说明这个管子已经损坏了。

\begin{figure}[htbp]
	\centering
	\includegraphics[width=0.7\linewidth]{51}
	\caption{}
	\label{fig:1}
\end{figure}

利用万用表10mA档作指示器,再加三个电阻,便可制成一个测试三极管$\beta$和$I_{ceo}$近似值的简单仪器(图 4-8)。其中,按钮开关AN可用小片铜皮自制,电池E可利用准备好的收音机电源。把被测的三极管插入电路(e、b、c三个电极的位置不能插错),万用表指示的就是这个管子的穿透电流,可以从0~100的这栏刻度上读出,刻度100处是10毫安。穿透电流要求小于1毫安。按下按钮开关AN,输入约0.04毫安的基极电流,集电极就相应增大,从表面上0~25这栏刻度上可读得$\beta$的数值,刻度25处表示$\beta$为250。$\beta$自50至150之间的管子都可以用,但以选70~80之间的最适当。在测试过程中,如万用表指针一直在漂移,这种管子性能很不稳定,只能剔去不用。

\begin{figure}[htbp]
	\centering
	\includegraphics[width=0.7\linewidth]{52}
	\caption{}
	\label{fig:1}
\end{figure}

\section{装置和调试}

有关元件都装置在印制板第2区和第3区里,可按下面的步骤进行:

(1)把原来装在印制板上的耳机拆下。

(2)安装音量控制电位器W:

电位器安装在印制板左下角无铜箔的那一面。印制板装电位器的地方有七个小孔(图4-9),把a、b、c三个小孔对准电位器的三条焊片,K、K'两个小孔对准开关的两条焊片,那么其余两个小孔正好对准电位器底部的两个螺丝孔。用两枚小螺丝插入这两个小孔,旋进电位器底部的螺丝孔内,把电位器紧固在印制板上。然后用裸导线把电位器上K、K'和b三条焊片和背面的铜接通。

\begin{figure}[htbp]
	\centering
	\includegraphics[width=0.7\linewidth]{53}
	\caption{}
	\label{fig:1}
\end{figure}

另两条焊片,则根据图4-11,用单芯塑料绝缘导线连接:一条自焊片c穿过小孔接至$C_{3}$的负极,另一条自焊片a穿过小孔接至$BG_{1}$的基极。

(3)根据图4-10的印制板元件图,把$R_{1}$、$C_{3}$、$BG_{1}$等三个元件插入印制板,一一焊牢。要注意,$C_{3}$的正负极接脚和$BG_{1}$的c、b、e三条接脚的位置不可插错,如果在$BG_{1}$的三条接脚上各套一个不同颜色的塑套管,就便于识别,不致搞错了。$R_{2}$和耳机暂缓接入。

\begin{figure}[htbp]
	\centering
	\includegraphics[width=0.7\linewidth]{54}
	\caption{}
	\label{fig:1}
\end{figure}

(4)连接电源:把四节干电池串联起来(参看图4-1),正极接开关K点,负极接第3区上方的角尺形长条铜。

(5)调整$BG_{1}$的偏置电流:把万用表拨到10mA档,接入耳机的位置,红表棒接$BG_{1}$的c极,黑表棒接电源负极;把一个220千欧的电位器$W_{0}$和一个22千欧的电阻$R_{0}$串联起来,接在$R_{2}$的位置上(图4-11)。然后接通电源,调节220千欧的电位器,使万用表的读数为1毫安左右。调好后,切断电源,拆下$W_{0}$、$R_{0}$的串联电路和万用表,用万用表测出这段串联电路的阻值,再找一个与这阻值最接近的电阻器接入$R_{2}$的位置,$BG_{1}$的偏置电流就调好了。注意调整偏置电流时,音量控制电位器W应该逆时针方向旋到音量最小的位置。

\begin{figure}[htbp]
	\centering
	\includegraphics[width=0.7\linewidth]{55}
	\caption{}
	\label{fig:1}
\end{figure}

(6)试调音量控制电路。接入耳机,开启电源,顺势把音量控制电位器W顺时针方向旋到底——开到音量最大位置。再旋动$C_{2a}$,选择电台收听,就可以感到声音比原来要大二三倍。在近电台地区收听,一定会嫌声音太大。可把音量电位器朝逆时针方向旋转,耳机里发出来的声音就会跟着渐渐轻下去。逆时针方向旋到底而电源开关还未关断时,耳机里应该是完全听不到播音声了(图4-12)。如果W逆时针方向旋到底,而耳机里还有声音,那可能是W的接地端没有焊好或在它内部b端已经开路的缘故。

\begin{figure}[htbp]
	\centering
	\includegraphics[width=0.7\linewidth]{56}
	\caption{}
	\label{fig:1}
\end{figure}

\section{电路原理}

检波电流中的音频成分流过$R_{1}$时,在$R_{1}$的两端形成一个音频信号电压,通过$C_{3}$接到$BG_{1}$的基极。$C_{3}$是一个30$\mu$F的大电容,对音频信号的阻抗很小,它在电路里起着隔断直流电,传递音频信号的作用。

在没有音频信号送来时,$BG_{1}$的基极电流$I_{b}$是一股直流电流——基极偏置电流,它的集电极电流$I_{c}$当然也是一股直流电流——集电极偏置电流,不过$I_{c}$比$I_{b}$要大得多,是$I_{b}$的$\beta$倍。

当有音频信号送来时,$BG_{1}$的$I_{b}$随着音频信号电压产生忽强忽弱的变化。它的$I_{c}$也随着作忽强忽弱的变化,不过$I_{c}$的变化要比$I_{b}$的变化大得多——是$I_{b}$变化的$\beta$倍(图4-13)。这股变化较大的电流通过耳机时,耳机里就发出比以前响得多的声音来了。

\begin{figure}[htbp]
	\centering
	\includegraphics[width=0.7\linewidth]{57}
	\caption{}
	\label{fig:1}
\end{figure}

\chapter{二管收音机(高放式)}

在单管机的检波器前,再插入一级高频放大器,就成为一具高放式二管收音机。在距离电台几十公里以内的地区,不接天地线就可以收音了。如果加接天线,那就可以收到更远的电台。

图5-1是本机的电路图和实体接线图。粗线部分是新加入的高放部分的线路。

\begin{figure}[htbp]
	\centering
	\includegraphics[width=0.7\linewidth]{58}
	\caption{}
	\label{fig:1}
\end{figure}

\begin{figure}[htbp]
	\centering
	\includegraphics[width=0.7\linewidth]{59}
	\caption{}
	\label{fig:1}
\end{figure}

\section{元件介绍}

(1)电阻器 $R_{3}$是5.1千欧、1/8瓦的碳膜电阻,$R_{4}$是100千欧左右、1/8瓦碳膜电阻,它的阻值与$BG_{2}$的电流放大系数有关,在调试时决定。$R_{5}$是1千欧、1/8瓦的碳膜电阻。

(2)电容器$C_{4}$是0.022微法的瓷片电容,$C_{5}$是0.047微法的瓷片电容。

(3)高频变压器 $B_{0}$是自制的高频变压器(图5-2)。在直径3毫米左右的中频磁芯上,初级用直径0.06毫米的漆包线绕200匝,次级用同号线绕100匝。如果自制有困难,用TTF-2-9型中频变压器来代替,效果也不差。

\begin{figure}[htbp]
	\centering
	\includegraphics[width=0.7\linewidth]{60}
	\caption{}
	\label{fig:1}
\end{figure}

(4)高频三极管 $BG_{2}$是3AG1型高频三极管。它的符号和外形,和3AX31没有什么两样。虽然它们都有把信号放大的能力,但各有专长:3AG1善于放大高频信号,但允许通过的电流只有10毫安,通过的电流太大时,管子就会发热烧毁。3AX31不善于放大高频信号,但允许通过100多毫安的电流,可用来放大较强的信号(图5-3)。

\begin{figure}[htbp]
	\centering
	\includegraphics[width=0.7\linewidth]{61}
	\caption{}
	\label{fig:1}
\end{figure}

这个管子的$\beta$,自50--100的都可以应用,穿透电流选0.5毫安以下的。

\section{装置和调试}

高频放大器装置在印制板的第1区内,如图5-4所示。

\begin{figure}[htbp]
	\centering
	\includegraphics[width=0.7\linewidth]{62}
	\caption{}
	\label{fig:1}
\end{figure}

按下述步骤进行装置、调试:

(1)把$L_{2}$从磁棒上取下,拆剩6圈,照原样装在磁棒上,两个线头照图5-4 所示,改接到$BG_{2}$基极及$R_{3}$、$R_{4}$的连接点上去。

(2)把高频变压器的四个线头用小刀刮清爽,搪好锡,然后把它的磁芯插入印制板第1区下排的直径为3毫米的安装孔内,用胶水胶牢。初级线圈的两端分别接$BG_{2}$集电极及电源负极(角尺形长条),次级线圈始端接第2区内二极管正极,终接“地”。

(3)把$BG_{2}$、$R_{3}$、$R_{5}$、$C_{4}$、$C_{5}$插入印制板,一一焊好。$R_{4}$暂缓接入。

(4)调整$BG_{2}$的偏置电流:因为发射极电流约等于集电极电流,所以只要测出发射极电阻$R_{5}$两端的电压,就可以推算出集电极电流来。$R_{5}$的阻值是1千欧,通过1毫安电流时,两端的电压正好是1伏。把万用表拨到直流2.5伏档,并联在$R_{5}$两端,红表棒接“地”,黑表棒接$BG_{2}$的发射极。取一个22千欧的电阻和一个220千欧的电位器串联起来,接在$R_{4}$的位置上(图5-5)。开启电源,调节220千欧的电位器到万用表指示1伏处。再试听耳机里的播音声,并稍稍旋动这个220千欧的电位器,使播音声最响亮清晰为止。如果可变电容$C_{1}$旋到某些位置,耳机会产生啸叫,那末应该把$R_{5}$两端的电压再调小些,即把偏流减弱些。试听下来比较满意后,关掉电源,拆下临时接到印制板上去的电阻和电位器,测出这段串联电路的阻值,选一个与这阻值接近的碳膜电阻接入$R_{4}$的位置。

\begin{figure}[htbp]
	\centering
	\includegraphics[width=0.7\linewidth]{63}
	\caption{}
	\label{fig:1}
\end{figure}

(5)试听:离电台不太远的地方,已不必接天、地线,只用磁性天线(磁棒能聚集在它周围的无线电波,绕在磁棒上的线圈可代替天线来接收无线电信号,所以称做磁性天线),就可以收到电台的播音。不过磁性天线有很强的方向性,磁棒应水平放置,并取与电台方向成正交的位置,这时收到的信号最强;垂直放置或取与电台方向一致的位置时收到的信号最弱,甚至收不到信号(图5-6)。利用磁性天线的方向性,可以把来自不同方向的、频率十分接近的电台信号分隔开来。

\begin{figure}[htbp]
	\centering
	\includegraphics[width=0.7\linewidth]{64}
	\caption{}
	\label{fig:1}
\end{figure}

接着,应重新调节$L_{1}$与$L_{2}$之间的距离,使收音机既有较高的灵敏度,又有较好的选择性。调好后,可滴少许熔蜡于纸管和磁棒间,以固定线圈的位置。

\section{电路原理}

(1)偏置电路

偏置电路是直流电路。线圈极容易让直流电通过,$L_{2}$和$B_{0}$的初级线圈可以认为是短路的;电容器不能让直流电通过,$C_{4}$和$C_{5}$可以认为是开路的。所以和$BG_{2}$的偏置有关的,实际上只有$R_{3}$、$R_{4}$、$R_{5}$三个电阻(图5-7)。

\begin{figure}[htbp]
	\centering
	\includegraphics[width=0.7\linewidth]{65}
	\caption{}
	\label{fig:1}
\end{figure}

$R_{4}$把$BG_{2}$的基极和电源负极连接起来,构成基极电流的通路,同时也限制了基极电流的大小,$R_{4}$越大基极电流就越小。$R_{3}$和$BG_{2}$的基极相并联,加接了这个电阻可以使偏置电流稳定些。为了进一步稳定偏置电流,还给$BG_{2}$加接一个发射极电阻$R_{5}$,使在电源电压或环境温度稍有变化时,$BG_{2}$的偏置电流基本上稳定不变。

(2)高频信号放大过程

对于高频信号来说,$C_{4}$和$C_{5}$的容抗很小,不妨认为是直通的。$C_{4}$并联在$R_{3}$两端,等于把$R_{3}$短路了。$C_{5}$并联在$R_{5}$两端,等于把$R_{5}$短路了。$L_{2}$输出的高频信号,直接输入$BG_{2}$的基极,通过放大,在$BG_{2}$集电极电路里输出一较强的高频信号,经高频变压器$B_{0}$送到检波器去(图5-8)。

\begin{figure}[htbp]
	\centering
	\includegraphics[width=0.7\linewidth]{66}
	\caption{}
	\label{fig:1}
\end{figure}

在高频放大器里,有两处电阻和电容并联的地方:$R_{3}$和$C_{4}$并联,$R_{5}$和$C_{5}$并联。把图5-7和图5-8 比较一下,可知电阻是直流电的通路,直流电通过电阻时,在电阻两端产生一个电压降。和电阻并联的电容(称做旁路电容)是专为交流信号设置的一条捷径。旁路电容的电容量必须足够大,使它对交流信号的阻抗几乎等于零。那么交流信号可不经电阻,很方便地从这条捷径通过了。如果不设这两个旁路电容,输入信号通过$R_{3}$、$R_{5}$时,将遭到损失。输出信号在$R_{5}$上产生电压降,将会使放大器对信号的放大倍数大打折扣。读者如有兴趣,可拆去这两个电容试听,一定会感到耳机里的声音显著地减轻了。

\begin{figure}[htbp]
	\centering
	\includegraphics[width=0.7\linewidth]{67}
	\caption{}
	\label{fig:1}
\end{figure}

\chapter{三管收音机(音频放大)}

在高放式二管收音机后面,再加装一级音频放大器,就可以去掉耳机,用扬声器来放音了。发出来的声音,在十多平方米的房间里,三四个人围坐在收音机旁收听,可以听得很清楚了。整机线路见图6-1,粗线部分是这一次加进去的音频放大器电路。

\begin{figure}[htbp]
	\centering
	\includegraphics[width=0.7\linewidth]{68}
	\caption{}
	\label{fig:1}
\end{figure}

\begin{figure}[htbp]
	\centering
	\includegraphics[width=0.7\linewidth]{69}
	\caption{}
	\label{fig:1}
\end{figure}

\section{元件介绍}

(1)电阻题

$R_{6}$是2.2千欧1/8瓦碳膜电阻,$R_{7}$是5.1千欧1/8瓦碳膜电阻,$R_{8}$也是1/8瓦的碳膜电阻,它的阻值和$BG_{3}$的$\beta$有关,由调试时实验决定。

(2)电容器

$C_{6}$是30微法的电解电容器。

(3)三极管

$BG_{3}$是3AX31型低频三极管。

(4)输出变压器$B_{1}$

输出变压器的符号和实物见图6-2。符号的左面是初级线圈,有一个中心抽头,右面是次级线圈,初次级圈数之比约为6:1。符号中间的粗直线,表示线圈里的铁芯。铁芯由EI型硅钢片叠成。当初级线圈里有音频信号电流通过时,在次级线圈里就会感应出信号电动势来。

\begin{figure}[htbp]
	\centering
	\includegraphics[width=0.7\linewidth]{70}
	\caption{}
	\label{fig:1}
\end{figure}

输出变压器的损坏,无非是线圈的断头。可以用万用表的Rx1档或Rx10档来测试:初级线圈1--2间的电阻在数十欧到近百欧之间。2--3端的电阻应该和1--2端的相同。次级线圈4--5端间的电阻约1欧姆。测试下来如果有一个头不通,这个输出变压器便不能用了。

输入变压器的外型和输出变压器相象,辨认的方法也是测量线圈的直流电阻:输出变压器的初、次线圈圈数少、用线粗,线圈的电阻都比较小;输入变压器的初、次线圈圈数多、用线细,线圈电阻较大,一般都在100欧姆以上。

\begin{figure}[htbp]
	\centering
	\includegraphics[width=0.7\linewidth]{71}
	\caption{}
	\label{fig:1}
\end{figure}

(5)扬声器Y

这里采用2.5英寸8欧姆的纸盆扬声器。它的符号,正像纸盆扬声器的侧视图。纸盆扬声器的构造见图6-4。由磁铁、场芯柱、导磁板、音圈、弹簧、纸盆等部件构成。音圈套在场芯柱上,可以上下运动,它的位置处于场芯柱和导磁板之间的空隙里。在这个空隙里有很强的磁场,当音圈里通有音频电流时,音圈受磁力的作用,便一上一下地振动起来。音圈和纸盆相连,带着纸盆一起振动,在周围的空气中激起声波,于是就发出声音来了。

\begin{figure}[htbp]
	\centering
	\includegraphics[width=0.7\linewidth]{72}
	\caption{}
	\label{fig:1}
\end{figure}

扬声器在音圈断线或脱焊时,就不会发出声音来了。音圈是否断线,可用万用表的Rx1档或Rx10档来测试,测得的阻值应在6~7欧姆之间。如果阻值很大或指针不动,那就说明音圈已经断线了。如果没有万用表,可用一节干电池,一个电极与音圈引出线的一端连接,一极去试触音圈引出线的另一端,如果扬声器能发出“咔嚓”声,说明音圈没有断线。

音圈与场芯柱或导磁板相碰时,发出来的声音又沙哑又轻。检查的方法是用手指轻轻地按动纸盆,把耳朵凑近纸盆听听,如果音圈与场芯柱或导磁板相碰,可以听到“嚓嚓”的摩擦声。

\begin{figure}[htbp]
	\centering
	\includegraphics[width=0.7\linewidth]{73}
	\caption{}
	\label{fig:1}
\end{figure}

\section{装置和调试}

这一级放大器内的元件,大部装在印制板第4区内(图6-6)。

\begin{figure}[htbp]
	\centering
	\includegraphics[width=0.7\linewidth]{74}
	\caption{}
	\label{fig:1}
\end{figure}

装制和调试的步骤如下:

(1)把上次装置的高放式二管机线路中的耳机拆去,焊下电源接线。

(2)把输出变压器$B_{1}$插入印制板,焊好。

(3)把$R_{6}$、$C_{6}$、$R_{7}$、$BG_{3}$四个元件插入印制板,一一焊好。注意$C_{6}$的正负极性和$BG_{3}$三个电极的位置切不可插错。$R_{8}$暂不接入。

(4)在印制电路板零件图上,有A、B、C三条附加接线,先用单股接线接好A、B两条,另一条C,留待以后再接。

(5)调整$BG_{3}$的偏置电流:把一个22千欧的电阻和一个220千欧的电位器串联起来,接入$R_{8}$的位置。把万用表拨到10mA档,红表棒接$BG_{3}$集电极,黑表棒接输出变压器$B_{1}$初级的一端;然后接上扬声器和电源。核对一下接线无误时,开启电源(电位器W留在音量最小的位置),调整220千欧的电位器,使万用表指示3毫安左右的读数。

(6)试听:旋转电位器W,把音量开至最大,然后旋动可变电容器$C_{1}$,收听声音最响的那个电台。唱得最响时如果万用表指针要向右挥动,可把偏流再调大一些,调到指针不会随声强的变化而挥动或挥动的幅度最小为止。

(7)关掉电源,拆下万用表和调偏流接入的电阻、电位器。测出电阻和电位器串联电路的阻值,选一个与这阻值最接近的电阻接在$R_{8}$的位置里。再用一段单股接线,把$BG_{3}$的集电极和输出变压器$B_{1}$初级的一端连接起来(见图6-6中的接线C)。现在装置和调试工作就可告一段落。

\begin{figure}[htbp]
	\centering
	\includegraphics[width=0.7\linewidth]{75}
	\caption{}
	\label{fig:1}
\end{figure}

\section{电路原理}

这一级音频放大器的电路,基本上和第四章里$BG_{1}$接耳机时的电路一样。所不同的是,前者耳机是直接接在$BG_{1}$集电极电路里的,而后者,扬声器却是通过输出变压器$B_{1}$和$BG_{3}$的集电极电路相耦合的。

\begin{figure}[htbp]
	\centering
	\includegraphics[width=0.7\linewidth]{76}
	\caption{}
	\label{fig:1}
\end{figure}

耳机的线圈匝数多,阻抗高,只要通过十几微安的音频电流就能听出声音来了。但经不起强电流的驱动,也不能发出较响的声音来。$BG_{1}$输出的音频电流很微弱,把耳机直接接在集电极电路里,恰好能发挥它的作用。

扬声器的音圈匝数少,阻抗低,至少要十毫安左右的音频电流才推得动它。但它经得起几百毫安电流的驱动,而发出响亮的声音来。

\begin{figure}[htbp]
	\centering
	\includegraphics[width=0.7\linewidth]{77}
	\caption{}
	\label{fig:1}
\end{figure}

$BG_{3}$输出的音频信号电流只有几毫安,不能用来直接推动扬声器。

输出变压器$B_{1}$的初级匝数比次级匝数多,音频电流通过初级线圈时,在它的两端形成一音频信号电压;传到次级线圈,电压按两线圈的匝数比变低了,而电流却能按匝数的反比变大了--已经有10毫安左右的音频电流,能够驱动扬声器发出声音来了。输出变压器的这种作用,和力学中应用杠杆原理,以小力克服大力的情形十分相像。

\begin{figure}[htbp]
	\centering
	\includegraphics[width=0.7\linewidth]{78}
	\caption{}
	\label{fig:1}
\end{figure}

\chapter{五管收音机(乙类推挽式功率放大器)}

本书前面所讲的装置,要加入了乙类推挽式功率放大器后,才完成一台有实用意义的收音机。乙类推挽式功率放大器具有效率高、省电、输出功率大等特点。这台收音机在收听近距离电台广播时,输出功率可达一、二百毫瓦,供家庭收听广播已够响亮了。

图7-1是本机的电路图和实体接线图

\begin{figure}[htbp]
	\centering
	\includegraphics[width=0.7\linewidth]{79}
	\caption{}
	\label{fig:1}
\end{figure}

\begin{figure}[htbp]
	\centering
	\includegraphics[width=0.7\linewidth]{80}
	\caption{}
	\label{fig:1}
\end{figure}

\section{元件介绍}

(1)电阻器

$R_{9}$至$R_{11}$是三个1/8瓦的碳膜电阻,$R_{9}$、$R_{11}$的阻值都是100欧姆,$R_{10}$的阻值约3千欧,在调试时实验决定。

(2)电容器

$C_{7}$和$C_{8}$都是100微法耐压6伏的电解电容器。

(3)输入变压器$B_{2}$

它的符号和外形都和输出变压器相象,辨认的方法是测试线圈的电阻。具体方法在第六章里已讲过了。它的初级线圈匝数比次级多,而且次级线圈有一个中心抽头,把次级线圈均分为两组。

\begin{figure}[htbp]
	\centering
	\includegraphics[width=0.7\linewidth]{81}
	\caption{}
	\label{fig:1}
\end{figure}

(4)三极管

$BG_{4}$和$BG_{5}$都是3AX31型低频三极管,这两个管子在放大器里是要相互配合,才能很好地进行工作的。因此,它们的性能应力求一致,否则放大后的输出信号就要失真。两个管子的$\beta$最好大小一样,最低限度它们的差数不超过10\%。例如一个管子的$\beta$是100,另一个管子的$\beta$只要在 90--110之间,就可以凑合着应用。

\section{装置和调试}

(1)拆掉三管机里的加接线C和B(就是从$BG_{3}$集电极到$B_{1}$初级一端的一条接线和从电源负极接到$R_{2}$、$R_{4}$、$R_{6}$、$R_{8}$等元件去的一条接线),加接线A让它留着不可拆掉。

(2)把输入变压器$B_{2}$插入印制板,焊好。

(3)把$R_{9}$、$R_{11}$两个电阻、$C_{7}$、$C_{8}$两个电解电容和$BG_{4}$、$BG_{5}$两个三极管插入印制板,一一焊好,$R_{10}$暂缓接入。注意$C_{7}$、$C_{8}$的正负极性和$BG_{4}$、$BG_{5}$三个电极的位置切不可插错,否则就会损坏元件。

(4)照图7-3右上角所示,从$BG_{3}$集电极到$B_{2}$初级间,加焊一段导线。

\begin{figure}[htbp]
	\centering
	\includegraphics[width=0.7\linewidth]{82}
	\caption{}
	\label{fig:1}
\end{figure}

(5)调整$BG_{4}$、$BG_{5}$的偏流:临时把输出变压器$B_{1}$初级中心接点与电源负极相连处的铜箔割断,把万用表拨到10mA档,红表棒接$B_{1}$初级中心,黑表棒接电源负极。取一个1.5千欧的电阻和一个4.7千欧的电位器串联起来,接到$R_{10}$的位置上去。开启电源(电位器W留在音量最小的位置),调整临时接入的电位器使万用表指示3毫安左右的读数(图7-4)。

\begin{figure}[htbp]
	\centering
	\includegraphics[width=0.7\linewidth]{83}
	\caption{}
	\label{fig:1}
\end{figure}

(6)试听:把接在电路里的万用表拨到100毫安的位置,将音量开大,选择一个电台信号试听,扬声器里能发出比较响亮的播音声来。如果手头有现成的机壳,把扬声器装到机壳里,听起来会更响亮、更动听。如果没有现成的机壳,可找一个空纸盒,在纸盒上方开一个比扬声器口径小一些的孔,把扬声器复在孔口试听。这时,试看万用表的指针,它正随着声音的起伏。一挥一挥地向右摆动。声音越响,摆动的角度越大,声音最响时,可摆到60--80毫安处;一点声音也没有时,指针仍回到3毫安左右的地方。这一现象,说明了这种放大器,信号越强,通过的电流越大。

接下来转动W,逐渐减小音量,试听在声很轻时,是否会出现失真(声音闷塞、沙哑)的现象。乙类推挽放大器放音很轻时所出现的失真,叫做“交越失真”。试把偏流稍稍调大些,就能消除这种失真。

\begin{figure}[htbp]
	\centering
	\includegraphics[width=0.7\linewidth]{84}
	\caption{}
	\label{fig:1}
\end{figure}

\section{电路原理}

在这台收音机里,由$BG_{1}$、$BG_{2}$、$BG_{3}$构成的放大器,属于同一种类型,称做甲类放大器。甲类放大器的特点是:基极偏置电压应大于输入信号电压的峰值,在信号整个周期内都有电流通过晶体管,不过是随着输入信号电压在一强一弱地变化罢了,通过晶体管的平均电流始终等于偏置电流,与信号的强弱无关。这种放大器效率低(就是说从电源取得的电能多,输出讯号小),因此只能用来做信号很弱的前几级放大器。

最后一级由$BG_{4}$、$BG_{5}$两个晶体管构成的放大器,属于另一种类型,称做乙类放大器。典型的乙类放大器如图7-6所示,是不加偏置的(即偏流等于零)。

\begin{figure}[htbp]
	\centering
	\includegraphics[width=0.7\linewidth]{85}
	\caption{}
	\label{fig:1}
\end{figure}

在没有信号输入时,$BG_{A}$、$BG_{B}$都处于截止状态,不耗一点电。当有信号输入时:设在信号的负半周内,输入变压器次级A端的电位比C点低,B端的电位比C点高,$BG_{A}$导通而工作,$BG_{B}$仍被截止着;在信号的正半周内,输入变压器次级A端电位比C点高,B端电位比C点低,$BG_{A}$被截止着,$BG_{B}$导通而工作。在输出变压器初级里,$BG_{A}$集电极电流是从A'端流向中点C'的,$BG_{B}$的集电极电流却是从B'端流向中点C'的,两者的方向正好相反,在输出变压器次级里感应出来的正好是整个周期的信号电势。这两个管子分工合作,轮流着工作、休息,分别负责放大半周期的信号,合起来才成为一个完整的波形。这种情形很象两个木工合作锯一段木材一推一拉地工作着,所以我们把这种放大器称做推挽(拉)式放大器。

\begin{figure}[htbp]
	\centering
	\includegraphics[width=0.7\linewidth]{86}
	\caption{}
	\label{fig:1}
\end{figure}

乙类放大器里的晶体管只在信号的半周期内有电流通过。信号越强,电流越大;信号越弱,电流越小;没有信号就没有电流。通过两个三极管的平均电流合起来,也不过输出信号电流峰值的2/3。这种放大器的效率高(从电源取得的电能大部分可转化成为输出信号),因此收音机末级功率放大器大都采用这种电路。

这里,却给$BG_{4}$、$BG_{5}$设置了一定的偏流,那是因为3AX31这一类型的晶体管,基极电压小于-0.2伏时是截止着的,要到-0.3伏左右才导通而正常工作。如果不加偏置,输入信号电压在-0.2伏以下这一段时间里,两管都截止着不工作,出现了“交接班脱节”的现象,输出信号里缺少这一段时间内的信号,因此产生了“交越失真”。给这两个管子的基极加上-0.3伏左右的偏置电压,让它们处于刚导通的状态(两个管子合起来,只要有几毫安的集电极电流就可以了),那么只要输入信号一到,它们就会配合着轮流工作,“交接班”也就不会“脱节”,交越失真的现象就消除了。

因为$BG_{4}$、$BG_{5}$是分工合作,各负责放大半周期的信号的,如果这两个管子的电流放大系数$\beta$不同,输出信号的正、负两个半周的大小也就不一样,这时扬声器里发出来的声音也就失真了。所以在装置时,一定要选两个特性一致的管子来“配对”使用。

\begin{figure}[htbp]
	\centering
	\includegraphics[width=0.7\linewidth]{87}
	\caption{}
	\label{fig:1}
\end{figure}

\section{整机电路结构和收音过程}

整机电路是由:输入电路、高频放大器、检波器、第一级音频放大器、第二级音频放大器、推挽式功率放大器和电源等七个部分组成的(图7-7)。

\begin{figure}[htbp]
	\centering
	\includegraphics[width=0.7\linewidth]{88}
	\caption{}
	\label{fig:1}
\end{figure}

输入电路包括磁性天线和调谐电路,磁性天线收到各个电台的信号后,通过调谐电路选择其中一个电台的信号,送到高频放大器去进行放大。放大了的高频信号,送到检波器去,通过检波,得到音频信号,送到音频放大器去,经过二级音频放大器的放大,再送到功率放大器去。功率放大器输出一个有较大功率的音频信号去推动扬声器,扬声器就发出响亮的声音来了。电源供给各级放大器工作时需要的电能,电源切断时,各级放大器立刻停止工作。

本机工作过程的特点是:先直接把电台高频信号进行放大,然后通过检波得到音频信号。无线电接收程式的分类中,把这类收音机称做高放式或直放式(即直接把电台信号进行放大的意思)收音机。

==============================================================================每一个无电爱好者的研究工作,差不多都是从石收机开始的,本母就是为了具体帮助初学者去制造矿石收音机而寫。里面說明了接收原理;矿石收音机主要零件的 機造 和 性能;收晋机的制作和稚修。采用的零件都是容易贸到或可以自制的,读者只要依酰明安装;就是从來没有学督过无线电的人也能成功。
这是一个无线电爱好者从事制作的开端,在装管碳石收机獲得煦之后,就可以在这个基碰上,选一步去装电学管收管机了。
如果讀者还想在理論方面和制作方面作深一步的研,下面这几本者是比较适合的:
1.'初等电工学
苏联 ·耶列柏卓夫著



\backmatter

\chapter{示例电路板}

用一块75x140平方毫米的单面铜箔板,采用刀刻法来制作:先把透明薄纸覆在图1-1上,用铅笔照样描下来,再用复写纸复印在铜箔板上,然后用斜口小刀刻去线条上的铜箔。刻的时候右手拿刀,刀口向前,刀尖紧抵铜箔,左手拇指顶住刀背略往前推,右手却稍往后拉,利用杠杆作用,使刀口向前刻划。

\begin{figure}[htbp]
	\centering
	\includegraphics[width=0.7\linewidth]{pcb1}
	\caption{}
	\label{fig:1}
\end{figure}

刻出来的线条,宽约3毫米,粗细要均匀。可在线条的一边先刻一道划痕,再在另一边刻一道划痕,并把线条两端刻断,然后用刀尖在一端挑起边角,就可把应刻去的铜箔撕下来(图1-3)。如果第一遍刻的划痕太浅,可连刻二遍、三遍。

\begin{figure}[htbp]
	\centering
	\includegraphics[width=0.7\linewidth]{fl_1}
	\caption{}
	\label{fig:1}
\end{figure}

钻孔前,先要用冲头(或钢针)在孔的中心凿一个凹痕,这样钻孔时钻头才不会滑动,装插一般元件接脚的孔径是1.2毫米,装输入、输出变压器的孔径是1.5毫米左右,装螺丝的孔径是3毫米。如果没有适当大小的钻头,可先钻一个小孔,用斜口小刀把它适当扩大就行。装双连可变电容器的大孔,孔径是10毫米,还须用尖头木锉或圆锉来进一步加工(图1-4)。

\begin{figure}[htbp]
	\centering
	\includegraphics[width=0.7\linewidth]{fl_2}
	\caption{}
	\label{fig:1}
\end{figure}

钻好孔后,用细砂皮轻轻打磨铜箔,除去污物和氧化层,使表面光洁明亮,然后在铜箔面均匀地涂刷一层松香溶液(图1)。

\begin{figure}[htbp]
	\centering
	\includegraphics[width=0.7\linewidth]{fl_3}
	\caption{}
	\label{fig:1}
\end{figure}

松香溶液配制的方法是:在墨水瓶里盛半瓶95\%的酒精,放入六、七颗蚕豆大小的松香,用筷子搅拌,使它溶解。这种松香溶液涂在铜箔上,其中的酒精很快地蒸发掉,松香在铜表面形成一层薄膜,它能保护铜箔表面,防止氧化。在焊接时,松香还起着助焊的作用,使铜箔容易上锡。余下的松香溶液,留着在焊接时作助焊剂。

\begin{figure}[htbp]
	\centering
	\includegraphics[width=0.7\linewidth]{dl_1}
	\caption{}
	\label{fig:1}
\end{figure}

\chapter{元件表}

\begin{figure}[htbp]
	\centering
	\includegraphics[width=0.7\linewidth]{yj1}
	\caption{}
	\label{fig:1}
\end{figure}

\begin{figure}[htbp]
	\centering
	\includegraphics[width=0.7\linewidth]{yj2}
	\caption{}
	\label{fig:1}
\end{figure}

\chapter{怎样焊接}

对于初学者来说,焊接工作的好坏,是装置收音机成败的关键。装置晶体管收音机,应该用45瓦以下小功率的铁。烙铁的功率过大或焊接的时间过长,都会使元件因受热而变值、损坏。一般用20瓦内热式电烙铁比较合适。它的构造如图1-6,烙铁芯装在金属管里,它是一条有夹层的瓷管,夹层里绕着电热丝通电后就会发热。烙铁头用纯铜制成,套在金属管外面。小功率的电烙铁,像图1-6那样拿,焊接起来最方便。焊接时,烙铁头的温度过低,焊锡熔不开、焊不牢;温度太高,烙铁头又常常被烧死(即表层氧化发黑,吃不上锡)。烙铁头被烧死时,可用锉刀把它表面的氧化层锉掉,立刻蘸上松香、焊锡,便可继续使用。

\begin{figure}[htbp]
	\centering
	\includegraphics[width=0.7\linewidth]{fl_4}
	\caption{}
	\label{fig:1}
\end{figure}

焊接前,先要在元件接脚或接线的线头上搪一层锡。方法,用斜口小刀把元件接脚(或线头)表面充分刮消爽,涂上一点松香溶液,用尖头钳夹持元件接脚的根部,烙铁头上饱蘸熔锡,接触元件接脚,自左向右,搪上一层焊锡。用尖头钳夹持接脚根部,可把热量引走,以免传入元件内部,使元件因过热而损坏(图1-7)。

\begin{figure}[htbp]
	\centering
	\includegraphics[width=0.7\linewidth]{fl_5}
	\caption{}
	\label{fig:1}
\end{figure}

接线的线头也要预先搪锡。加接在印制板上的接线,应该用质地较硬的单芯塑料绝缘接线,装好后比较挺括牢靠。从线路板引出的接线(如与电源、扬声器连接的导线)应该用多股线芯、质地柔软的塑料绝缘接线,才经得起多次弯折。搪锡前先要剥出2毫米左右的接头。剥线头的方法是:利用热的烙铁头把塑料层熔断,趁热用手把割断的那段塑料皮拉去(图1-8)。刚去皮的新接线很容易上锡,不须刮削,只要涂上一点助焊剂,就可搪锡。

\begin{figure}[htbp]
	\centering
	\includegraphics[width=0.7\linewidth]{fl_6}
	\caption{}
	\label{fig:1}
\end{figure}

元件接脚搪好锡后,根据印制板来弯折接脚,使几条接脚刚好能插入印制板相应的位置。弯脚时要留意:使元件插入线路板后,标有数值的一面朝上,以便于装机时核对。

\part{收音机进阶知识}

\chapter{无线电波传播}

无线电波是一种电磁波,它在空间中以光速传播。不同频率的无线电波具有不同的传播特性,这对收音机的设计和使用有着重要的影响。

\section{中波与短波传播特性}

中波(MW)的频率范围是530kHz到1600kHz,主要依靠地波和天波传播。在白天,中波主要通过地波传播,传播距离一般在几百公里以内;在夜晚,天波反射增强,可以传播到更远的距离。

短波(SW)的频率范围是3MHz到30MHz,主要依靠电离层反射的天波传播。短波可以通过多次反射,实现全球范围的通信。

\section{天线理论与匹配}

天线是收音机接收无线电波的关键部件,其性能直接影响接收效果。天线的长度、形状和安装位置都会影响其接收特性。

天线匹配是指天线与接收机之间的阻抗匹配,良好的匹配可以减少信号损失,提高接收效率。常用的匹配方法包括使用匹配变压器和调整天线长度。

\section{接收环境与干扰防治}

接收环境对收音机的性能有很大影响,主要的干扰源包括:

1. **天电干扰**:如雷电、静电等自然现象产生的干扰
2. **工业干扰**:如电力线、电机、电焊机等产生的干扰
3. **邻频干扰**:相邻频率的电台信号干扰

防治干扰的方法包括:使用定向天线、增加屏蔽、改进电路设计等。

\chapter{调谐电路优化}

调谐电路是收音机的核心部分,其性能直接影响收音机的灵敏度、选择性和频率覆盖范围。

\section{几何平均值法与最小频率法比较}

在设计调谐电路时,有两种常用的电感计算方法:

1. **最小频率法**:使用最低频率和最大电容来计算电感
2. **几何平均值法**:使用频率和电容的几何平均值来计算电感

几何平均值法可以获得更均匀的频率覆盖,而最小频率法则计算更简单。

\section{可变电容与电感的选用}

可变电容是调谐电路的关键元件,其容量范围和质量直接影响调谐性能。常用的可变电容有空气介质和固体介质两种。

电感的设计需要考虑线圈的匝数、直径、长度等参数,以及磁芯的选用。高频应用中,通常使用磁棒或空心线圈。

\section{频率覆盖与统调方法}

频率覆盖是指收音机能够接收的频率范围,AM收音机通常覆盖中波波段(530kHz-1600kHz)。

统调是指调整收音机的调谐电路,使其在整个频率范围内都能良好接收信号。统调包括调整电感和电容,以及调整中频变压器。

\chapter{音质与功率放大}

音质是收音机的重要指标,而功率放大则决定了收音机的输出功率和音量。

\section{音频放大电路设计}

音频放大电路的设计需要考虑增益、频率响应、失真等因素。常用的音频放大电路包括:

1. **单管放大电路**:结构简单,适用于小功率放大
2. **推挽放大电路**:效率高,失真小,适用于功率放大
3. **集成放大电路**:体积小,性能稳定,使用方便

\section{扬声器与耳机匹配}

扬声器和耳机是收音机的输出设备,其阻抗和功率需要与放大器匹配。常见的扬声器阻抗有8Ω、16Ω等,耳机阻抗则有8Ω、32Ω、800Ω等多种。

匹配的方法包括使用输出变压器和调整放大电路的输出阻抗。

\section{音质改善技巧}

改善音质的方法包括:

1. **增加音调控制电路**:调节高低音比例
2. **使用负反馈电路**:减少失真,改善频率响应
3. **选择高品质元件**:如使用聚丙烯电容、金属膜电阻等
4. **优化电路布局**:减少干扰,提高信噪比

\part{实践与应用}

\chapter{收音机调试技巧}

调试是制作收音机过程中的重要环节,良好的调试可以显著提高收音机的性能。

\section{仪器使用}

常用的调试仪器包括:

1. **万用表**:测量电压、电流、电阻等参数
2. **信号发生器**:产生标准的高频信号
3. **示波器**:观察信号波形
4. **频率计**:测量信号频率

\section{调谐电路调试}

调谐电路的调试主要包括:

1. **调整频率覆盖**:确保能够接收整个波段的信号
2. **调整统调点**:使收音机在整个波段内都有良好的接收效果
3. **调整灵敏度**:提高收音机的接收能力

\section{中频调整与统调}

中频调整是指调整中频变压器的谐振频率,使其固定在465kHz(或其他标准中频)。统调则是调整收音机的调谐电路,使其在整个频率范围内都能与中频电路良好配合。

\chapter{常见故障排查}

收音机在使用过程中可能会出现各种故障,及时排查和修复故障是保持收音机良好性能的关键。

\section{无台故障分析}

无台故障的可能原因包括:

1. **电源故障**:电池没电、电源电路损坏
2. **天线故障**:天线断线、接触不良
3. **调谐电路故障**:线圈断线、电容失效
4. **放大电路故障**:晶体管损坏、电阻变值

\section{杂音与失真处理}

杂音和失真的可能原因包括:

1. **元件老化**:电容漏电、晶体管性能下降
2. **电路故障**:焊接不良、元件损坏
3. **干扰**:外部干扰、内部自激
4. **统调不良**:频率偏移、中频失谐

\section{元件老化与替换}

元件老化是收音机故障的常见原因,特别是电解电容、晶体管等元件。定期检查和替换老化的元件可以延长收音机的使用寿命。

替换元件时,应选择与原元件参数相近的替代品,并注意安装方向和焊接质量。

\chapter{DIY 收音机项目}

DIY收音机是无线电爱好者的重要活动,通过制作不同类型的收音机,可以深入了解无线电原理和电路设计。

\section{简单矿石机制作}

矿石机是最简单的收音机,不需要电源,只需要天线、地线、线圈、电容、矿石(或二极管)和耳机就可以制作。

制作步骤包括:设计电路、绕制线圈、选择元件、组装调试。

\section{单管机改进方案}

单管机是入门级的有源收音机,可以通过以下方法改进性能:

1. **增加再生电路**:提高灵敏度
2. **改进天线**:使用磁性天线或外接天线
3. **增加音量控制**:使用电位器调节音量
4. **优化电路布局**:减少干扰,提高稳定性

\section{超外差机设计实例}

超外差机是性能较好的收音机,其设计包括:

1. **变频电路**:将高频信号转换为中频信号
2. **中频放大电路**:放大中频信号
3. **检波电路**:从中频信号中提取音频信号
4. **音频放大电路**:放大音频信号,驱动扬声器

\part{收音机发展历史}

\chapter{收音机的诞生与演进}

收音机的发展经历了漫长的历史过程,从最早的矿石收音机到现代的数字收音机,每一步都代表了无线电技术的重大进步。

\section{从矿石检波器到电子管}

1887年,赫兹发现了电磁波;1895年,马可尼和波波夫各自独立发明了无线电报;1906年,费森登进行了第一次无线电广播;1910年左右,矿石收音机开始出现;1915年,电子管收音机问世。

\section{晶体管革命}

1947年,贝尔实验室发明了晶体管;1954年,第一台晶体管收音机问世;1958年,集成电路发明;1960年代,晶体管收音机普及。

\section{数字时代的收音机}

1980年代,数字调谐收音机出现;1990年代,DSP(数字信号处理)技术应用于收音机;2000年代,网络收音机和数字广播(DAB)出现;2010年代,智能手机和移动应用成为新的收音方式。

\chapter{重要发明与人物}

收音机的发展离不开众多科学家和工程师的贡献,他们的发明和创新推动了无线电技术的进步。

\section{先驱者的贡献}

1. **詹姆斯·克拉克·麦克斯韦**:提出电磁波理论
2. **海因里希·赫兹**:证实电磁波存在
3. **古列尔莫·马可尼**:发明实用无线电报
4. **亚历山大·波波夫**:独立发明无线电报
5. **雷金纳德·费森登**:进行第一次无线电广播
6. **李·德福雷斯特**:发明三极管

\section{电子管与晶体管的发明者}

1. **约翰·安布罗斯·弗莱明**:发明二极管
2. **李·德福雷斯特**:发明三极管
3. **威廉·肖克利**、**约翰·巴丁**、**沃尔特·布拉顿**:发明晶体管

\section{收音机技术的里程碑}

1. **1906年**:第一次无线电广播
2. **1910年**:矿石收音机出现
3. **1915年**:电子管收音机问世
4. **1920年**:商业广播开始
5. **1947年**:晶体管发明
6. **1954年**:第一台晶体管收音机问世
7. **1958年**:集成电路发明
8. **1980年代**:数字调谐收音机出现
9. **1990年代**:DSP技术应用
10. **2000年代**:数字广播(DAB)出现

\chapter{收音机的未来发展}

随着科技的不断进步,收音机也在不断发展和创新。

\section{数字化与网络化}

数字广播(如DAB、HD Radio)提供了更高的音质和更多的功能,网络收音机则通过互联网实现了全球范围的广播接收。

\section{智能化与个性化}

智能收音机结合了人工智能技术,可以根据用户的喜好推荐节目,自动调整音质和音量,甚至可以与其他智能设备互联。

\section{环保与可持续发展}

未来的收音机将更加注重环保和可持续发展,使用更节能的电路设计,采用可回收材料,减少对环境的影响。

收音机作为一种重要的信息传播工具,虽然面临着电视、互联网等新媒体的挑战,但它仍然具有独特的优势,如便携性、低功耗、实时性等。在未来,收音机将继续发展,为人们提供更加优质的广播服务。

\begin{figure}[htbp]
	\centering
	\includegraphics[width=0.7\linewidth]{fl_7}
	\caption{}
	\label{fig:1}
\end{figure}

元件接脚从线路板无铜箔的一面(简称A面)插入,穿出有铜箔的一面(简称B面),留下1.5毫米左右的接脚,把多余部分剪掉。然后用烙铁饱蘸焊锡,进行焊接。焊接时烙铁头应和焊点紧密接触,使熔锡充分浸润接脚和铜箔,并且吃牢在接脚和铜箔上。提起烙铁时,用手指弹击一下线路板,可使焊点圆浑不起毛刺。焊接的时间要掌握在2秒钟以内,时间过长,会使元件过热而损坏。焊接的过程如图1-9所示。焊点要求圆浑光亮不起毛刺,焊锡和接脚、铜箔要吃得很牢。用锡多少适宜,焊点不过大或过小(图 1-10)。

\begin{figure}[htbp]
	\centering
	\includegraphics[width=0.7\linewidth]{fl_8}
	\caption{}
	\label{fig:1}
\end{figure}

焊接得不好时,元件的接脚只是被焊锡含住,或锡粒只是靠松香粘结在铜箔上,没有真正的焊牢,这样的焊点,称做“假焊”。假焊的焊点往往有很大的电阻,有的时通时断,有的索性不通,所以会使电路出现许多莫名其妙的故障,个别接点的假焊,还有导致元件损坏的危险(如三极管的下偏流电阻)。

\begin{figure}[htbp]
	\centering
	\includegraphics[width=0.7\linewidth]{fl_9}
	\caption{}
	\label{fig:1}
\end{figure}

造成假焊的原因常有下列几个方面:

一是元件的接脚或线头没有刮清爽,连搪锡也没有搪好。

二是印制板铜箔表面染有污物或搁置日久表面氧化。

三是烙铁头温度太高或太低。

四是熔锡氧化已不适宜用来焊接,刚蘸上烙铁头的熔锡,表面光亮流动性很好,及时用来焊接,就容易和接脚、铜吃牢。如搁置时间过长,熔锡氧化,表面灰暗粗糙,失去了流动性,这样的熔锡就不能再和接脚、铜箔的表面结合,就会造成假焊。如果熔锡已经氧化,只要再蘸上点松香,就能使它还原,仍旧可以用来焊接。

五是烙铁头表面氧化或沾有污物,影响焊接。

六是烙铁头在焊点上停留的时间太短,焊锡还来不及和接脚、铜箔吃牢。

\chapter{外壳}

钢琴式机壳的制作

收音机电路调试好后,要把它安装到适当的机壳里去,才便于使用。少年们的科技作品,贵在具有少年的特色,机壳用什么材料来制作、采取怎样的式样,读者尽可根据自己的条件、爱好,发挥创造才能,自己去设计、制作。这里介绍一种小钢琴式机壳的制作方案,供读者参考。
请你先仔细看一看机壳的立体图(图附-1),和俯视图、侧视图(图附-2a、b)相对照,就可以明白它大体的结构了。图附-3 中画出了盖板、底板、四侧面板、键盘和琴脚等每一块材料的样子。图里,每一小格表示1平方厘米,请你先把这一张图放大成与实物一样大小的工作图,然后照着你自己画的工作图进行制作:
盖板--用三夹板制作,右下方的小方孔,是用来显示刻度盘上的频率读数的,它的具体位置和大小要根据刻度盘来决定。用厚8毫米的木板,锯四块边长为10毫米的正方形小片,用胶水胶在盖板下装印制板位置的四角,将来装印制板的螺丝就钉在这四块小木片上。

底板--用三夹板制作。三个装琴脚的孔直径为8毫米。左、右和后侧面板--都是用厚8毫米的木板制作。三块
板接合处的角度根据图附-2b虚线所示,角度要削正确,否则拼起来就不能密合。右侧面板上安装拨盘的孔,它的位置和大小,要根据拨盘的尺寸来决定。
前侧面板-一用厚约5毫米的木板制成。
键盘--用15毫米厚的木板条制作。
琴脚--用木条削成的圆台体,榫头直径为8毫米。
上述部件制好后,参考图附-1所示进行装配:
(1)把左、右、前、后四块侧面板用胶水胶在底板上,可适当地钉几只小铁钉加固。然后用胶水把键盘胶在底板的最前面。2)待胶水干透后,用细砂纸磨光琴体表面,漆上彩色磁漆,键盘上画出一个个的琴键,盖板上还可以画一点有意义的图案。
(3)漆干后,把印制板装在盖板下,扬声器、电池装在底板上。盖好盖板,用三只小木螺丝把盖板固定在左、右、后三方的侧面板上。最后,把三只琴脚装在底板下,用胶水胶牢。现在,这台精致美观的钢琴式收音机已全面完工了。



\part{收音机基础元件(补充)}

\chapter{电源系统}

电源是收音机的重要组成部分,为各电路提供稳定的工作电压。不同类型的收音机需要不同的电源系统。

\section{电池类型与选用}

收音机常用的电池类型包括:

1. **干电池**:如AA、AAA、9V电池,适合便携式收音机
2. **充电电池**:如镍镉、镍氢、锂离子电池,可重复使用
3. **外接电源**:如AC适配器,适合台式收音机

选用电池时需要考虑容量、电压、尺寸和成本等因素。

\section{电源滤波与稳压电路}

电源滤波电路用于去除电源中的交流成分,常用的滤波元件包括电容和电感。稳压电路则用于保持输出电压的稳定,常用的稳压元件包括稳压二极管和集成稳压器。

\section{低功耗设计技巧}

低功耗设计对于便携式收音机尤为重要,常用的低功耗设计技巧包括:

1. **选用低功耗元件**:如低功耗晶体管、CMOS集成电路
2. **优化电路设计**:如使用高效率的放大电路
3. **合理的电源管理**:如设置电源开关、自动关机功能

\chapter{测量仪器与工具}

正确使用测量仪器和工具是制作和调试收音机的关键。

\section{万用表的使用方法}

万用表是最常用的电子测量仪器,可以测量电压、电流、电阻等参数。使用万用表时需要注意:

1. **选择正确的量程**:根据测量对象选择合适的量程
2. **正确的测量方法**:电压并联测量,电流串联测量
3. **安全操作**:避免过载,注意测量高压时的安全

\section{信号发生器与示波器的应用}

信号发生器用于产生标准的电信号,是调试收音机的重要工具。示波器则用于观察电信号的波形,帮助分析电路的工作状态。

\section{常用工具与辅料介绍}

制作收音机需要的常用工具包括:

1. **焊接工具**:电烙铁、焊锡、助焊剂
2. **装配工具**:螺丝刀、钳子、剪刀
3. **调试工具**:无感螺丝刀、频率计

常用的辅料包括:

1. **绝缘材料**:热缩管、绝缘胶带
2. **导热材料**:导热硅脂、导热垫
3. **固定材料**:螺丝、螺母、垫片

\part{经典收音机设计(补充)}

\chapter{来复式收音机}

来复式收音机是一种结构简单、性能较好的收音机,它利用三极管的非线性特性,在一个三极管中同时完成高频放大和音频放大。

\section{来复再生式电路原理}

来复再生式电路的核心是利用三极管的发射结作为检波器,同时利用正反馈产生再生作用,提高收音机的灵敏度。

\section{制作与调试方法}

来复式收音机的制作步骤包括:

1. **设计电路**:选择合适的电路拓扑
2. **准备元件**:三极管、电阻、电容、线圈等
3. **组装电路**:按照电路图焊接元件
4. **调试再生**:调整再生电位器,使收音机达到最佳灵敏度

\section{性能优化技巧}

优化来复式收音机性能的技巧包括:

1. **调整再生量**:避免自激振荡的同时提高灵敏度
2. **优化天线**:使用合适的天线提高接收效果
3. **改善音频放大**:增加音频放大级数,提高音量

\chapter{场效应管收音机}

场效应管收音机利用场效应管的高输入阻抗、低噪声特性,具有较好的接收效果。

\section{场效应管的特性与应用}

场效应管的主要特性包括:

1. **高输入阻抗**:适合作为收音机的第一级放大
2. **低噪声**:提高接收弱信号的能力
3. **电压控制**:控制方式简单

\section{场效应管收音机电路设计}

场效应管收音机的电路设计需要考虑:

1. **偏置电路**:为场效应管提供合适的工作点
2. **输入匹配**:与天线进行阻抗匹配
3. **级间耦合**:合理设计级间耦合电路

\section{与晶体管收音机的比较}

场效应管收音机与晶体管收音机的比较:

1. **灵敏度**:场效应管收音机通常具有更高的灵敏度
2. **噪声**:场效应管收音机的噪声更低
3. **成本**:场效应管的成本通常高于晶体管
4. **稳定性**:晶体管收音机的稳定性通常更好

\part{收音机进阶知识(补充)}

\chapter{中频放大器设计}

中频放大器是超外差式收音机的核心部分,负责放大中频信号,提高收音机的灵敏度和选择性。

\section{中频变压器的设计与绕制}

中频变压器的设计需要考虑:

1. **谐振频率**:通常为465kHz(AM)或10.7MHz(FM)
2. **匝数比**:根据前后级的阻抗匹配要求确定
3. **磁芯选择**:根据工作频率选择合适的磁芯材料

绕制中频变压器时需要注意:

1. **线圈的绕制方向**:保持一致
2. **线圈的绝缘**:确保匝间绝缘良好
3. **磁芯的调整**:预留磁芯调整的空间

\section{中放电路的稳定性优化}

中放电路的稳定性优化需要考虑:

1. **防止自激振荡**:合理设计电路,避免正反馈
2. **降低噪声**:选用低噪声元件,优化电路设计
3. **温度稳定性**:使用温度补偿元件,优化偏置电路

\section{多级中放的级联设计}

多级中放的级联设计需要考虑:

1. **增益分配**:合理分配各级的增益
2. **阻抗匹配**:确保级间阻抗匹配
3. **带宽控制**:通过中频变压器的设计控制带宽

\chapter{自动增益控制(AGC)}

自动增益控制(AGC)电路用于自动调整收音机的增益,使收音机在接收强弱不同的信号时都能保持合适的音量。

\section{AGC电路原理与作用}

AGC电路的基本原理是:

1. **检测信号强度**:通过检波器检测输出信号的强度
2. **产生控制电压**:根据信号强度产生相应的控制电压
3. **调整增益**:利用控制电压调整中放或高放的增益

AGC的主要作用是:

1. **保持音量稳定**:接收强信号时降低增益,接收弱信号时提高增益
2. **防止过载**:避免强信号时电路过载失真
3. **提高抗干扰能力**:减少邻频干扰的影响

\section{不同类型的AGC电路}

常见的AGC电路类型包括:

1. **简单AGC**:直接利用检波输出控制增益
2. **延迟AGC**:弱信号时不启动AGC,强信号时才启动
3. **反向AGC**:控制电压增加时增益降低
4. **正向AGC**:控制电压增加时增益提高

\section{AGC的调整与优化}

AGC的调整与优化需要考虑:

1. **AGC起控点**:调整AGC开始起作用的信号强度
2. **AGC范围**:调整AGC能够控制的增益范围
3. **AGC时间常数**:调整AGC的响应速度

\part{实践与应用(补充)}

\chapter{收音机套件制作}

收音机套件是一种预先设计好的收音机制作材料包,包含所有必要的元件和电路板,适合初学者制作。

\section{套件的选择与购买}

选择收音机套件时需要考虑:

1. **难度级别**:根据自己的经验选择合适的难度
2. **电路类型**:矿石机、单管机、超外差机等
3. **配件齐全**:确保包含所有必要的元件和工具
4. **售后服务**:选择有良好售后服务的供应商

\section{组装步骤与注意事项}

组装收音机套件的步骤包括:

1. **检查配件**:确认所有元件都已齐全
2. **阅读说明书**:仔细阅读组装说明
3. **准备工具**:准备好必要的工具
4. **组装电路**:按照说明书的步骤组装电路
5. **调试电路**:按照说明书的方法调试电路

组装时需要注意:

1. **元件的极性**:确保元件的极性正确
2. **焊接质量**:确保焊接牢固,避免假焊
3. **安全操作**:注意用电安全,避免短路

\section{调试与故障排查}

调试收音机套件时需要注意:

1. **电源检查**:确保电源连接正确
2. **信号检查**:使用信号发生器或电台信号进行测试
3. **故障排查**:根据症状分析故障原因

常见的故障包括:

1. **无电源**:电池连接错误,开关故障
2. **无信号**:天线连接错误,调谐电路故障
3. **杂音大**:接地不良,元件损坏

\chapter{收音机的数字化改造}

随着科技的发展,传统收音机可以通过数字化改造,增加现代功能,提高使用体验。

\section{添加数字频率显示}

添加数字频率显示可以使收音机更加直观,便于用户调谐。常用的数字频率显示模块包括:

1. **频率计数器模块**:直接显示当前接收的频率
2. **微控制器+显示屏**:使用Arduino等微控制器控制显示屏

\section{安装蓝牙模块}

安装蓝牙模块可以使收音机与手机、电脑等设备连接,播放数字音乐。常用的蓝牙模块包括:

1. **HC-05/HC-06**:通用蓝牙串口模块
2. **专用蓝牙音频模块**:支持A2DP协议

\section{与现代设备的互联}

收音机与现代设备的互联方式包括:

1. **蓝牙连接**:与手机、电脑等设备连接
2. **WiFi连接**:连接到家庭网络,收听网络电台
3. **USB连接**:通过USB接口充电、传输数据

\part{收音机发展历史(补充)}

\chapter{中国收音机发展史}

中国收音机的发展经历了从进口到仿制,再到自主研发的过程,见证了中国电子工业的发展。

\section{早期收音机在中国的引入}

20世纪初,收音机开始传入中国。最早的收音机是从国外进口的,主要在沿海城市使用。1923年,中国第一座广播电台在上海建立,标志着中国广播事业的开始。

\section{新中国收音机工业的发展}

新中国成立后,收音机工业得到了快速发展:

1. **1950年代**:开始仿制苏联和东欧国家的收音机
2. **1960年代**:自主研发晶体管收音机
3. **1970年代**:普及半导体收音机
4. **1980年代**:发展集成电路收音机
5. **1990年代至今**:数字化、网络化发展

\section{国产经典收音机型号介绍}

国产经典收音机型号包括:

1. **红灯711**:电子管收音机的经典型号
2. **牡丹941**:晶体管收音机的代表产品
3. **飞跃J-303**:便携式收音机的经典型号
4. **德生R-9700**:现代数字调谐收音机的代表

\chapter{收音机收藏与鉴赏}

收音机收藏是一种有趣的爱好,不仅可以了解无线电技术的发展历史,还可以欣赏不同时期的设计风格。

\section{收音机收藏的分类与价值}

收音机收藏可以按照以下方式分类:

1. **按年代**:早期矿石机、电子管收音机、晶体管收音机等
2. **按品牌**:国内外知名品牌的产品
3. **按类型**:台式、便携式、汽车收音机等
4. **按功能**:AM、FM、短波收音机等

收音机的收藏价值主要取决于:

1. **稀有程度**:产量少、存世量少的机型
2. **历史意义**:具有重要历史意义的机型
3. **品相**:保存状态良好的机型
4. **功能**:功能完整、能正常工作的机型

\section{经典机型的识别与评估}

识别经典收音机机型时需要注意:

1. **品牌标识**:查看机身上的品牌标识
2. **型号标识**:查看机身上的型号标识
3. **生产日期**:查看机身上的生产日期
4. **内部结构**:打开机壳查看内部结构

评估经典收音机价值时需要考虑:

1. **完整性**:原装配件是否齐全
2. **工作状态**:是否能正常工作
3. **外观状态**:外观是否完好,有无损坏
4. **历史背景**:是否有特殊的历史背景

\section{收藏与保养技巧}

收藏收音机时需要注意:

1. **环境控制**:避免潮湿、高温、阳光直射的环境
2. **定期清洁**:定期清洁收音机的外观和内部
3. **定期通电**:定期通电,避免元件老化
4. **正确存放**:使用合适的包装材料存放

保养收音机时需要注意:

1. **元件的更换**:及时更换老化的元件
2. **机械部分的润滑**:定期润滑机械部分
3. **避免过度使用**:避免长时间连续使用

\part{特殊用途收音机}

\chapter{业余无线电接收机}

业余无线电接收机是业余无线电爱好者使用的专用接收机,用于接收业余无线电波段的信号。

\section{业余无线电的波段与规范}

业余无线电的波段包括:

1. **高频波段**:1.8MHz、3.5MHz、7MHz、14MHz、21MHz、28MHz等
2. **甚高频波段**:144MHz、430MHz等
3. **特高频波段**:1200MHz等

业余无线电的操作需要遵守相关的法规和规范,包括:

1. **执照要求**:需要取得业余无线电操作执照
2. **频率使用**:遵守频率分配规定
3. **功率限制**:遵守功率限制规定

\section{业余接收机的特点与设计}

业余接收机的特点包括:

1. **宽频率覆盖**:覆盖多个业余无线电波段
2. **高灵敏度**:能够接收微弱的业余无线电信号
3. **高选择性**:能够区分邻近频率的信号
4. **多功能**:支持多种调制方式,如AM、SSB、CW等

业余接收机的设计需要考虑:

1. **前端电路**:提高灵敏度和选择性
2. **中频电路**:提供足够的增益和带宽
3. **解调电路**:支持多种调制方式
4. **音频电路**:提供清晰的音频输出

\section{天线与接收系统的搭建}

业余无线电接收系统的搭建需要考虑:

1. **天线的选择**:根据接收波段选择合适的天线
2. **天线的安装**:选择合适的安装位置和高度
3. **馈线的选择**:选择低损耗的馈线
4. **接地系统**:建立良好的接地系统

\chapter{单边带(SSB)接收机}

单边带(SSB)是一种高效的调制方式,广泛应用于业余无线电、航空通信等领域。

\section{SSB调制原理与优势}

SSB调制的基本原理是:

1. **抑制载波**:去除调幅信号中的载波成分
2. **抑制边带**:去除调幅信号中的一个边带

SSB调制的优势包括:

1. **频谱效率高**:占用的频带宽度只有调幅的一半
2. **功率效率高**:所有功率都用于传输有用信号
3. **抗干扰能力强**:减少了噪声和干扰的影响

\section{SSB接收机的电路设计}

SSB接收机的电路设计需要考虑:

1. **变频器**:将高频信号转换为中频信号
2. **滤波器**:选择需要的边带,抑制不需要的边带
3. **解调器**:解调SSB信号,恢复原始音频
4. **自动频率控制(AFC)**:补偿频率偏移

\section{调试与使用技巧}

调试SSB接收机时需要注意:

1. **频率校准**:确保接收机的频率准确
2. **滤波器调整**:调整滤波器的中心频率和带宽
3. **AGC调整**:调整AGC的起控点和范围

使用SSB接收机时需要注意:

1. **天线的选择**:使用高效的天线提高接收效果
2. **接地的改善**:建立良好的接地系统减少噪声
3. **干扰的避免**:远离干扰源,使用屏蔽措施

\chapter{调频(FM)收音机}

调频(FM)是一种广泛应用于广播的调制方式,具有音质好、抗干扰能力强等优点。

\section{FM广播的特点与优势}

FM广播的特点包括:

1. **频率范围**:87.5MHz-108MHz
2. **调制方式**:频率调制
3. **带宽**:每个频道占用200kHz的带宽

FM广播的优势包括:

1. **音质好**:可以传输高保真音频信号
2. **抗干扰能力强**:对噪声和干扰不敏感
3. **立体声传输**:支持立体声音频传输

\section{FM接收机的电路设计}

FM接收机的电路设计需要考虑:

1. **高频头**:接收和解调FM信号
2. **中频电路**:放大和处理中频信号
3. **立体声解码**:解码立体声音频信号
4. **音频放大**:放大音频信号,驱动扬声器

\section{立体声解码电路}

FM立体声解码电路的基本原理是:

1. **导频信号**:接收19kHz的导频信号
2. **相位锁定**:使用锁相环锁定导频信号
3. **信号分离**:将复合信号分离为左、右声道信号
4. **矩阵解码**:使用矩阵电路解码立体声信号

\part{无线电实验与项目}

\chapter{无线电测向}

无线电测向是一种利用无线电波的传播特性,确定发射源方向的技术,是业余无线电活动的重要组成部分。

\section{测向原理与方法}

无线电测向的基本原理是:

1. **信号强度比较**:比较不同方向的信号强度
2. **相位比较**:比较不同位置的信号相位

常用的测向方法包括:

1. **幅度比较法**:使用定向天线比较不同方向的信号强度
2. **相位比较法**:使用多天线阵列比较信号的相位
3. **多普勒法**:利用多普勒效应测量信号频率的变化

\section{测向机的制作与调试}

制作测向机的步骤包括:

1. **设计电路**:选择合适的测向电路
2. **准备元件**:三极管、电阻、电容、天线等
3. **组装电路**:按照电路图焊接元件
4. **调试电路**:调整电路参数,提高测向精度

调试测向机时需要注意:

1. **天线的调整**:确保天线的方向性良好
2. **接收机的调整**:确保接收机的灵敏度和选择性
3. **指示器的调整**:确保指示器的准确性

\section{测向竞赛与活动}

无线电测向竞赛是一种常见的业余无线电活动,参与者使用测向机寻找隐藏的发射源。

常见的测向竞赛包括:

1. **短距离测向**:在较小的范围内寻找发射源
2. **长距离测向**:在较大的范围内寻找发射源
3. **快速测向**:计时比赛,看谁先找到所有发射源

参加测向竞赛时需要注意:

1. **熟悉规则**:了解竞赛的规则和要求
2. **准备装备**:准备好必要的装备,如测向机、地图、指南针
3. **训练技能**:提高测向的速度和准确性
4. **安全第一**:注意安全,避免发生意外

\chapter{简易信号发生器}

信号发生器是一种产生标准电信号的仪器,是调试收音机的重要工具。

\section{高频信号发生器的制作}

制作高频信号发生器的步骤包括:

1. **设计电路**:选择合适的振荡电路,如LC振荡、晶体振荡
2. **准备元件**:三极管、电阻、电容、电感、晶体等
3. **组装电路**:按照电路图焊接元件
4. **调试电路**:调整电路参数,产生稳定的信号

高频信号发生器的主要技术指标包括:

1. **频率范围**:能够产生的频率范围
2. **频率稳定性**:频率的稳定程度
3. **输出幅度**:输出信号的强度
4. **波形质量**:输出信号的波形纯度

\section{音频信号发生器的设计}

音频信号发生器用于产生音频范围内的电信号,是调试收音机音频电路的重要工具。

制作音频信号发生器的步骤包括:

1. **设计电路**:选择合适的振荡电路,如RC振荡、函数发生器
2. **准备元件**:运放、电阻、电容、电位器等
3. **组装电路**:按照电路图焊接元件
4. **调试电路**:调整电路参数,产生稳定的信号

音频信号发生器的主要技术指标包括:

1. **频率范围**:能够产生的频率范围
2. **频率精度**:频率的准确程度
3. **输出幅度**:输出信号的强度
4. **波形种类**:能够产生的波形种类,如正弦波、方波、三角波

\section{信号发生器的校准与使用}

校准信号发生器时需要注意:

1. **频率校准**:使用频率计校准输出信号的频率
2. **幅度校准**:使用万用表或示波器校准输出信号的幅度
3. **波形校准**:使用示波器校准输出信号的波形

使用信号发生器时需要注意:

1. **正确连接**:确保信号发生器与被测电路正确连接
2. **合适的输出**:选择合适的输出频率和幅度
3. **安全操作**:注意用电安全,避免短路

\chapter{无线电遥控实验}

无线电遥控是一种利用无线电波控制远程设备的技术,是无线电爱好者的重要实验项目。

\section{简单遥控电路的设计}

设计简单遥控电路的步骤包括:

1. **选择调制方式**:如调幅、调频、调相
2. **设计发射电路**:产生调制后的无线电信号
3. **设计接收电路**:接收和解调无线电信号
4. **设计控制电路**:根据解调后的信号控制设备

简单遥控电路的主要技术指标包括:

1. **控制距离**:能够控制的最远距离
2. **抗干扰能力**:抵抗干扰的能力
3. **控制精度**:控制的准确程度
4. **可靠性**:工作的可靠程度

\section{编码与解码电路}

为了提高遥控电路的抗干扰能力和安全性,通常需要使用编码与解码电路。

常用的编码与解码电路包括:

1. **固定编码**:使用固定的编码信号
2. **滚动编码**:使用不断变化的编码信号
3. **学习编码**:能够学习和存储编码信号

\section{遥控模型的应用}

无线电遥控技术广泛应用于模型飞机、模型汽车、模型船等遥控模型中。

制作遥控模型时需要注意:

1. **选择合适的遥控系统**:根据模型的类型选择合适的遥控系统
2. **正确安装设备**:确保遥控设备正确安装在模型上
3. **测试和调试**:在使用前进行充分的测试和调试
4. **安全操作**:注意安全,避免发生意外

\part{收音机维修与故障排除}

\chapter{常见故障分析与排查}

收音机在使用过程中可能会遇到各种故障,掌握常见故障的分析与排查方法,对于收音机爱好者来说非常重要。

\section{故障分类与症状}

收音机常见故障可以分为以下几类:

1. **接收故障**:收不到台、灵敏度低、选择性差等
2. **音频故障**:声音小、失真、噪声大等
3. **电源故障**:无法开机、电池消耗快等
4. **机械故障**:调谐旋钮失灵、开关接触不良等

\section{故障排查步骤}

故障排查的基本步骤包括:

1. **观察症状**:仔细观察收音机的故障症状
2. **初步检查**:检查电源、天线、连接线等基本部件
3. **电路分析**:根据电路图分析可能的故障点
4. **测试验证**:使用万用表等工具测试相关元件
5. **故障修复**:更换或修复故障元件
6. **功能验证**:修复后验证收音机的各项功能

\section{常见故障实例分析}

\subsection{收不到台的故障排查}

可能的原因:
- 天线连接不良
- 调谐电路故障
- 变频电路故障
- 中频放大电路故障

排查方法:
- 检查天线连接是否正确
- 测试调谐电路的电压和电流
- 检查变频管是否正常工作
- 测试中频放大电路的信号传输

\subsection{声音小的故障排查}

可能的原因:
- 音量电位器故障
- 音频放大电路故障
- 扬声器故障
- 电源电压不足

排查方法:
- 检查音量电位器是否接触良好
- 测试音频放大电路的输出电压
- 检查扬声器是否正常
- 测量电源电压是否正常

\chapter{元件检测与替换}

正确检测和替换故障元件是收音机维修的关键技能。

\section{常用检测工具}

常用的电子元件检测工具包括:

1. **万用表**:测量电压、电流、电阻等
2. **示波器**:观察电信号的波形
3. **信号发生器**:产生测试信号
4. **频率计**:测量信号频率

\section{元件检测方法}

\subsection{电阻器的检测}

- 使用万用表的电阻档测量电阻值
- 对比测量值与标称值
- 检查电阻是否开路或短路

\subsection{电容器的检测}

- 使用万用表的电容档测量电容值
- 检查电容是否漏电
- 检查电容是否开路或短路

\subsection{电感器的检测}

- 使用万用表的电阻档测量电感的直流电阻
- 检查电感是否开路
- 使用电感表测量电感值

\subsection{二极管的检测}

- 使用万用表的二极管档测量正向和反向电阻
- 检查二极管是否导通或击穿

\subsection{三极管的检测}

- 使用万用表的二极管档测量三极管的各极间电阻
- 检查三极管的放大能力
- 检查三极管是否损坏

\section{元件替换原则}

替换元件时应遵循以下原则:

1. **型号匹配**:尽量使用与原元件型号相同的元件
2. **参数相近**:如果无法找到相同型号,应使用参数相近的元件
3. **质量可靠**:选择质量可靠的元件
4. **安装正确**:确保元件安装方向正确

\chapter{电路故障诊断技巧}

掌握一些电路故障诊断技巧,可以提高故障排查的效率。

\section{信号注入法}

信号注入法是一种常用的故障诊断方法,通过向电路中注入测试信号,观察电路的反应,确定故障位置。

\subsection{信号注入点的选择}

- 从后向前注入:从音频输出级开始,逐步向前级注入
- 从前向后注入:从天线输入开始,逐步向后级注入

\subsection{信号注入的注意事项}

- 注入信号的幅度要适当
- 注入信号的频率要合适
- 注入信号时要注意安全,避免损坏电路

\section{电压测量法}

电压测量法是通过测量电路中各点的电压,与正常电压值进行比较,确定故障位置。

\subsection{关键测试点的选择}

- 电源电压测试点
- 晶体管的各极电压测试点
- 集成电路的各脚电压测试点

\subsection{电压测量的注意事项}

- 正确选择万用表的量程
- 确保测量时表笔接触良好
- 注意测量安全,避免短路

\section{电流测量法}

电流测量法是通过测量电路中的电流,与正常电流值进行比较,确定故障位置。

\subsection{电流测量的方法}

- 串联测量法:将万用表串联在电路中
- 间接测量法:通过测量电阻两端的电压,计算电流

\subsection{电流测量的注意事项}

- 正确选择万用表的量程
- 确保测量时电路连接正确
- 注意测量安全,避免短路

\chapter{收音机的日常维护与保养}

正确的日常维护与保养可以延长收音机的使用寿命,保持其良好的工作状态。

\section{清洁与除尘}

- 定期清洁收音机的外部
- 使用干燥的软布擦拭机身
- 使用毛刷清除旋钮和缝隙中的灰尘
- 避免使用湿布或有机溶剂清洁

\section{电池的使用与维护}

- 使用符合规格的电池
- 定期检查电池是否漏液
- 长期不使用时应取出电池
- 注意电池的正负极方向

\section{天线的维护}

- 保持天线的清洁
- 避免天线受到机械损伤
- 定期检查天线的连接是否牢固
- 对于外置天线,应注意防雷

\section{存放与保管}

- 存放时应避免潮湿环境
- 避免阳光直射
- 避免高温环境
- 避免剧烈震动

\part{收音机与现代技术的融合}

\chapter{软件定义无线电(SDR)基础}

软件定义无线电(SDR)是一种将硬件功能通过软件实现的无线电技术,代表了现代无线电技术的发展方向。

\section{SDR的基本原理}

SDR的基本原理是:

1. **模数转换**:将射频信号转换为数字信号
2. **数字信号处理**:使用软件对数字信号进行处理
3. **数模转换**:将处理后的数字信号转换为模拟信号

\section{SDR的优势}

SDR相比传统无线电具有以下优势:

1. **灵活性高**:通过软件修改可以实现不同的功能
2. **升级方便**:只需更新软件即可实现功能升级
3. **成本降低**:硬件标准化,降低了成本
4. **性能优越**:可以实现更复杂的信号处理算法

\section{SDR在收音机中的应用}

SDR技术在收音机中的应用包括:

1. **多波段接收**:通过软件配置实现多波段接收
2. **数字解调**:使用软件实现各种调制方式的解调
3. **信号处理**:使用软件进行降噪、均衡等处理
4. **数据显示**:通过软件实现各种数据的显示

\chapter{数字信号处理(DSP)在收音机中的应用}

数字信号处理(DSP)技术的应用,大大提高了收音机的性能。

\section{DSP的基本概念}

DSP是一种使用数字方法处理信号的技术,包括:

1. **信号采样**:将连续信号转换为离散信号
2. **信号量化**:将模拟信号转换为数字信号
3. **数字滤波**:使用数字方法对信号进行滤波
4. **快速傅里叶变换**:将时域信号转换为频域信号

\section{DSP在收音机中的具体应用}

\subsection{数字滤波}

- 中频数字滤波
- 音频数字滤波
- 带通滤波
- 陷波滤波

\subsection{降噪处理}

- 噪声抑制
- 信号增强
- 自适应降噪
- 语音增强

\subsection{自动增益控制}

- 数字AGC算法
- 自适应增益控制
- 动态范围压缩

\subsection{解调技术}

- 数字AM解调
- 数字FM解调
- 数字SSB解调
- 数字解调算法优化

\chapter{收音机的数字化改造}

将传统收音机进行数字化改造,可以提升其性能和功能。

\section{改造方案设计}

\subsection{局部数字化改造}

- 音频部分数字化
- 调谐部分数字化
- 显示部分数字化

\subsection{全数字化改造}

- 替换为SDR接收器
- 添加数字信号处理模块
- 增加数字控制界面

\section{改造实例}

\subsection{传统收音机添加数字调谐}

- 设计数字调谐电路
- 安装微控制器
- 增加显示屏
- 编写控制软件

\subsection{收音机添加蓝牙功能}

- 设计蓝牙音频模块
- 与收音机音频电路集成
- 增加蓝牙控制功能

\subsection{收音机添加USB供电}

- 设计USB电源转换电路
- 与收音机电源系统集成
- 增加充电功能

\chapter{互联网广播与传统收音机的结合}

互联网广播的兴起,为传统收音机带来了新的发展机遇。

\section{互联网广播的特点}

互联网广播相比传统广播具有以下特点:

1. **内容丰富**:可以提供更多的广播频道
2. **不受地域限制**:可以收听全球的广播节目
3. **交互性强**:可以与广播电台进行互动
4. **多媒体功能**:可以提供音频、视频等多媒体内容

\section{网络收音机的设计}

\subsection{硬件设计}

- 网络连接模块
- 音频处理模块
- 显示控制模块
- 电源管理模块

\subsection{软件设计}

- 网络协议栈
- 音频解码软件
- 用户界面软件
- 应用程序

\section{传统收音机与互联网的融合方案}

- 添加网络收音机模块
- 设计混合接收系统
- 实现传统广播与网络广播的无缝切换
- 增加网络功能的同时保留传统功能

\part{无线电法规与安全}

\chapter{无线电频率管理与分配}

了解无线电频率的管理与分配,对于收音机爱好者来说非常重要。

\section{无线电频率的划分}

无线电频率按照波长和频率的不同,可以划分为不同的波段:

1. **长波**:30kHz-300kHz
2. **中波**:300kHz-3MHz
3. **短波**:3MHz-30MHz
4. **超短波**:30MHz-300MHz
5. **微波**:300MHz以上

\section{频率分配原则}

频率分配的基本原则包括:

1. **国际协调**:由国际电信联盟(ITU)协调全球频率分配
2. **国家管理**:各国负责国内频率的具体分配和管理
3. **合理利用**:确保频率资源的合理利用
4. **避免干扰**:避免不同用户之间的干扰

\section{我国的频率分配}

我国的无线电频率由工业和信息化部负责管理和分配,主要包括:

1. **广播频率**:中波、短波、调频广播频率
2. **电视频率**:电视广播频率
3. **通信频率**:移动通信、卫星通信等频率
4. **业余无线电频率**:业余无线电爱好者使用的频率

\chapter{业余无线电执照制度}

业余无线电执照是业余无线电爱好者合法使用无线电设备的凭证。

\section{业余无线电执照的分类}

我国的业余无线电执照分为以下几类:

1. **A类**:允许使用VHF和UHF频段
2. **B类**:允许使用HF、VHF和UHF频段
3. **C类**:允许使用更多频段和更高功率

\section{执照申请条件}

申请业余无线电执照的基本条件包括:

1. **年龄要求**:年满18周岁
2. **知识要求**:通过业余无线电操作技术能力考试
3. **设备要求**:使用符合规定的业余无线电设备
4. **守法要求**:遵守无线电管理法规

\section{执照的使用与管理}

业余无线电执照的使用与管理要求包括:

1. **合法使用**:按照执照规定的频段和功率使用
2. **定期核验**:执照需要定期核验
3. **变更登记**:设备变更时需要进行登记
4. **遵守法规**:遵守国家有关无线电管理的法规

\chapter{无线电发射设备的法规要求}

使用无线电发射设备,必须遵守国家的相关法规要求。

\section{设备型号核准}

无线电发射设备必须取得型号核准证书,才能在中国境内销售和使用。

\section{设备技术指标要求}

无线电发射设备的技术指标要求包括:

1. **频率范围**:符合国家规定的频率范围
2. **发射功率**:符合国家规定的功率限制
3. **杂散发射**:杂散发射必须符合国家规定的限值
4. **调制方式**:符合国家规定的调制方式

\section{设备使用要求}

使用无线电发射设备的要求包括:

1. **持证使用**:必须持有相应的无线电执照
2. **合法用途**:用于合法的无线电业务
3. **避免干扰**:避免对其他无线电业务造成干扰
4. **安全操作**:确保设备的安全操作

\chapter{无线电操作安全指南}

安全操作无线电设备,对于保护人身安全和设备安全非常重要。

\section{用电安全}

- 使用符合规定的电源
- 避免电源短路
- 注意电池的安全使用
- 避免触电事故

\section{辐射安全}

- 了解无线电波的辐射特性
- 遵守辐射安全限值
- 合理使用天线
- 避免长时间靠近发射天线

\section{操作规范}

- 遵守无线电管理法规
- 尊重其他无线电用户
- 保持良好的通联礼仪
- 正确使用无线电术语

\section{应急处理}

- 设备故障的应急处理
- 干扰事件的应急处理
- 自然灾害中的应急通信
- 安全事故的应急处理

\part{收音机电路仿真与设计工具}

\chapter{SPICE仿真软件的使用}

SPICE是一种常用的电路仿真软件,可以帮助收音机设计者验证电路设计的正确性。

\section{SPICE的基本概念}

SPICE(Simulation Program with Integrated Circuit Emphasis)是一种通用的电路仿真程序,用于模拟电路的行为。

\section{常用SPICE软件}

常用的SPICE软件包括:

1. **LTspice**:Linear Technology公司开发的免费SPICE软件
2. **PSpice**:OrCAD公司开发的SPICE软件
3. **Multisim**:National Instruments公司开发的电路仿真软件
4. **TINA-TI**:Texas Instruments公司开发的免费电路仿真软件

\section{SPICE仿真的基本步骤}

SPICE仿真的基本步骤包括:

1. **绘制电路图**:在仿真软件中绘制电路图
2. **设置元件参数**:设置电路中各元件的参数
3. **设置仿真类型**:选择合适的仿真类型,如直流仿真、交流仿真、瞬态仿真等
4. **运行仿真**:执行仿真操作
5. **分析结果**:分析仿真结果,验证电路设计的正确性

\section{收音机电路的SPICE仿真实例}

\subsection{调谐电路的仿真}

- 绘制调谐电路原理图
- 设置元件参数
- 进行交流仿真
- 分析频率响应曲线

\subsection{中频放大电路的仿真}

- 绘制中频放大电路原理图
- 设置元件参数
- 进行交流仿真
- 分析增益和频率响应

\subsection{音频放大电路的仿真}

- 绘制音频放大电路原理图
- 设置元件参数
- 进行瞬态仿真
- 分析输出波形和失真

\chapter{电路设计软件介绍}

除了SPICE仿真软件外,还有许多专门的电路设计软件,可以帮助收音机设计者完成电路设计。

\section{原理图设计软件}

常用的原理图设计软件包括:

1. **Eagle**:CadSoft公司开发的电子设计自动化软件
2. **KiCad**:开源的电子设计自动化软件
3. **Altium Designer**:Altium公司开发的电子设计自动化软件
4. **OrCAD**:Cadence公司开发的电子设计自动化软件

\section{PCB设计软件}

常用的PCB设计软件包括:

1. **Eagle**:支持PCB设计
2. **KiCad**:支持PCB设计
3. **Altium Designer**:支持PCB设计
4. **PADS**:Mentor Graphics公司开发的PCB设计软件

\section{收音机电路设计实例}

\subsection{原理图设计}

- 绘制收音机的整体原理图
- 设计各个功能模块的电路
- 标注元件参数
- 生成物料清单

\subsection{PCB设计}

- 设计PCB布局
- 设计PCB布线
- 进行DRC检查
- 生成Gerber文件

\chapter{射频电路设计工具}

射频电路设计需要专门的工具和软件,以确保电路的性能。

\section{射频电路设计的特点}

射频电路设计相比低频电路设计具有以下特点:

1. **分布参数影响**:需要考虑分布参数的影响
2. **阻抗匹配**:需要考虑阻抗匹配
3. **电磁兼容**:需要考虑电磁兼容性
4. **高频效应**:需要考虑高频效应

\section{射频电路设计工具}

常用的射频电路设计工具包括:

1. **ADS**:Agilent公司开发的射频电路设计软件
2. **HFSS**:Ansys公司开发的高频结构仿真软件
3. **CST Microwave Studio**:CST公司开发的微波电路仿真软件
4. **Qucs**:开源的射频电路设计软件

\section{射频电路设计实例}

\subsection{天线设计}

- 选择天线类型
- 设计天线结构
- 仿真天线性能
- 优化天线设计

\subsection{射频前端设计}

- 设计低噪声放大器
- 设计混频器
- 设计滤波器
- 优化射频前端性能

\chapter{仿真结果与实际电路的对比分析}

仿真结果与实际电路之间可能存在差异,需要进行对比分析,以提高设计的准确性。

\section{差异原因分析}

仿真结果与实际电路之间的差异可能由以下原因引起:

1. **元件模型误差**:元件模型与实际元件之间的差异
2. **寄生参数**:实际电路中存在的寄生参数
3. **工艺误差**:PCB制作工艺的误差
4. **测量误差**:测量工具和方法的误差

\section{模型修正方法}

为了提高仿真的准确性,可以采取以下模型修正方法:

1. **元件参数测量**:实际测量元件的参数
2. **模型参数调整**:根据实际测量结果调整模型参数
3. **寄生参数提取**:提取实际电路中的寄生参数
4. **工艺参数考虑**:考虑PCB制作工艺的影响

\section{设计优化流程}

基于仿真结果和实际测试结果,可以建立以下设计优化流程:

1. **初始设计**:根据理论分析进行初始设计
2. **仿真验证**:使用仿真软件验证设计
3. **原型制作**:制作实际的电路原型
4. **测试分析**:测试原型电路的性能
5. **模型修正**:根据测试结果修正仿真模型
6. **设计优化**:基于修正后的模型优化设计
7. **最终验证**:制作最终的电路并验证性能

\part{高端收音机设计}

\chapter{专业广播接收机}

专业广播接收机是一种高性能的收音机,主要用于广播电台、监测站等专业领域。

\section{专业接收机的特点}

专业广播接收机相比普通收音机具有以下特点:

1. **高性能**:具有更高的灵敏度、选择性和稳定性
2. **多功能**:支持多种调制方式和波段
3. **高可靠性**:采用高品质的元件和设计
4. **专业接口**:具有更多的专业接口和控制功能

\section{专业接收机的电路设计}

\subsection{射频前端设计}

- 高性能低噪声放大器
- 高选择性调谐电路
- 抗干扰设计
- 宽动态范围设计

\subsection{中频处理电路设计}

- 多级中频放大
- 晶体滤波器
- 自动增益控制
- 中频AGC电路

\subsection{解调电路设计}

- 高性能解调器
- 多种调制方式支持
- 低失真设计
- 解调信号质量监测

\subsection{音频处理电路设计}

- 专业音频处理
- 音频均衡器
- 噪声 reduction电路
- 音频质量控制

\chapter{业余无线电高级接收机}

业余无线电高级接收机是为业余无线电爱好者设计的高性能接收机。

\section{业余无线电接收机的特点}

业余无线电接收机的特点包括:

1. **多波段**:支持多个业余无线电波段
2. **多模式**:支持多种调制模式,如AM、FM、SSB、CW等
3. **高性能**:具有较高的灵敏度和选择性
4. **可扩展性**:支持各种扩展功能

\section{业余无线电接收机的电路设计}

\subsection{接收系统设计}

- 多波段接收前端
- 宽带变频器
- 数字信号处理
- 软件定义无线电技术

\subsection{控制系统设计}

- 微控制器控制
- 数字调谐系统
- 存储功能
- 通信接口

\subsection{显示系统设计}

- 多功能显示屏
- 频谱显示
- 信号强度指示
- 操作状态显示

\chapter{高性能调谐电路设计}

调谐电路是收音机的核心部分,高性能调谐电路可以提高收音机的接收性能。

\section{调谐电路的基本原理}

调谐电路的基本原理是利用LC谐振回路的选频特性,选择所需的 radio frequency信号。

\section{高性能调谐电路的设计要点}

\subsection{Q值优化}

- 选择高品质因数的元件
- 优化电路设计
- 减少损耗
- 提高选择性

\subsection{跟踪调整}

- 设计跟踪调整电路
- 实现各波段的跟踪
- 提高调谐精度
- 确保整个波段的性能一致

\subsection{温度稳定性}

- 选择温度稳定性好的元件
- 设计温度补偿电路
- 提高电路的温度稳定性
- 确保在不同温度下的性能一致

\subsection{抗干扰设计}

- 屏蔽设计
- 滤波设计
- 接地设计
- 减少干扰的耦合

\chapter{低噪声放大技术}

低噪声放大技术是提高收音机灵敏度的关键技术之一。

\section{噪声的基本概念}

- 热噪声
- 散粒噪声
-  flicker噪声
- 噪声系数

\section{低噪声放大器的设计}

\subsection{晶体管选择}

- 选择低噪声系数的晶体管
- 考虑晶体管的频率特性
- 选择合适的晶体管类型

\subsection{偏置电路设计}

- 设计稳定的偏置电路
- 优化工作点
- 考虑温度稳定性

\subsection{匹配网络设计}

- 输入匹配网络设计
- 输出匹配网络设计
- 噪声匹配
- 功率匹配

\subsection{级联设计}

- 多级放大器的级联设计
- 噪声系数的计算
- 增益分配
- 稳定性考虑

\section{低噪声放大器的测试与优化}

- 噪声系数的测量
- 增益的测量
- 频率响应的测量
- 稳定性的测试
- 性能优化

\part{无线电通信基础}

\chapter{调制与解调原理}

调制与解调是无线电通信的基本技术,了解其原理对于收音机设计者来说非常重要。

\section{调制的基本概念}

调制是将信息信号加载到载波上的过程,其目的是:

1. **便于发射**:将低频信号转换为高频信号,便于天线发射
2. **实现多路复用**:不同的信号可以使用不同的载波频率,实现多路复用
3. **提高抗干扰能力**:调制可以提高信号的抗干扰能力

\section{常见的调制方式}

\subsection{调幅(AM)调制}

- AM的基本原理
- AM的调制电路
- AM的解调电路
- AM的优缺点

\subsection{调频(FM)调制}

- FM的基本原理
- FM的调制电路
- FM的解调电路
- FM的优缺点

\subsection{调相(PM)调制}

- PM的基本原理
- PM的调制电路
- PM的解调电路
- PM的优缺点

\subsection{单边带(SSB)调制}

- SSB的基本原理
- SSB的调制电路
- SSB的解调电路
- SSB的优缺点

\chapter{通信协议基础}

通信协议是通信双方约定的规则,对于无线电通信来说非常重要。

\section{通信协议的基本概念}

通信协议包括:

1. **物理层协议**:规定物理传输媒介的特性
2. **数据链路层协议**:规定数据的封装和传输
3. **网络层协议**:规定网络地址和路由
4. **应用层协议**:规定应用程序之间的通信规则

\section{无线电通信协议}

\subsection{AM广播协议}

- 载波频率范围
- 调制方式
- 带宽限制
- 发射功率限制

\subsection{FM广播协议}

- 载波频率范围
- 调制方式
- 带宽限制
- 立体声广播标准

\subsection{业余无线电协议}

- 业余无线电频段分配
- 业余无线电通信规则
- 业余无线电数字通信协议
- 业余无线电卫星通信协议

\chapter{数据传输技术}

数据传输技术是现代无线电通信的重要组成部分。

\section{数字调制技术}

数字调制技术包括:

1. **振幅键控(ASK)**
2. **频移键控(FSK)**
3. **相移键控(PSK)**
4. **正交振幅调制(QAM)**

\section{数字编码技术}

数字编码技术包括:

1. **脉冲编码调制(PCM)**
2. **增量调制(DM)**
3. **差分脉冲编码调制(DPCM)**
4. **自适应差分脉冲编码调制(ADPCM)**

\section{差错控制技术}

差错控制技术包括:

1. **奇偶校验**
2. **循环冗余校验(CRC)**
3. **前向纠错编码(FEC)**
4. **自动重传请求(ARQ)**

\chapter{无线电网络基础}

无线电网络是现代通信的重要组成部分,了解其基础对于收音机设计者来说非常重要。

\section{无线电网络的基本概念}

无线电网络的基本概念包括:

1. **网络拓扑**:网络的结构和连接方式
2. **多址接入**:多个用户共享同一无线电频道的方式
3. **路由协议**:数据包在网络中的传输路径选择
4. **网络协议栈**:网络通信的分层协议

\section{常见的无线电网络}

\subsection{蜂窝网络}

- GSM网络
- CDMA网络
- 3G网络
- 4G网络
- 5G网络

\subsection{无线局域网}

- WiFi网络
- 蓝牙网络
- ZigBee网络
- Z-Wave网络

\subsection{卫星通信网络}

- 同步卫星通信
- 低轨道卫星通信
- 卫星广播
- 卫星导航

\section{软件无线电与网络}

软件无线电技术在无线电网络中的应用:

- 多标准兼容
- 网络协议的软件实现
- 动态频谱管理
- 认知无线电

\part{收音机的未来发展趋势}

\chapter{数字广播技术发展}

数字广播技术是广播技术的发展方向,代表了收音机技术的未来。

\section{数字广播的优势}

数字广播相比传统模拟广播具有以下优势:

1. **音质好**:可以提供CD质量的音频
2. **抗干扰能力强**:数字信号具有更好的抗干扰能力
3. **频谱效率高**:可以在相同的频谱内传输更多的内容
4. **多功能**:可以提供数据服务、交互服务等多种功能

\section{常见的数字广播标准}

\subsection{DAB/DAB+}

- 欧洲数字广播标准
- 技术特点
- 覆盖范围
- 应用情况

\subsection{DRM}

- 数字调幅广播标准
- 技术特点
- 覆盖范围
- 应用情况

\subsection{HD Radio}

- 美国数字广播标准
- 技术特点
- 覆盖范围
- 应用情况

\subsection{CMMB}

- 中国移动多媒体广播标准
- 技术特点
- 覆盖范围
- 应用情况

\section{数字广播接收机的设计}

\subsection{硬件设计}

- 数字信号处理芯片
- 射频前端
- 解调电路
- 音频处理电路

\subsection{软件设计}

- 数字解调算法
- 音频解码软件
- 用户界面软件
- 数据服务处理软件

\chapter{智能收音机的兴起}

智能收音机是将人工智能技术与传统收音机相结合的产物,代表了收音机的未来发展方向。

\section{智能收音机的特点}

智能收音机相比传统收音机具有以下特点:

1. **语音控制**:支持语音指令控制
2. **智能推荐**:根据用户的喜好推荐广播节目
3. **联网功能**:可以连接互联网,收听网络广播
4. **多媒体功能**:支持音频、视频等多媒体内容
5. **智能家居集成**:可以与智能家居系统集成

\section{智能收音机的硬件设计}

\subsection{核心处理器}

- 高性能微处理器
- 语音处理芯片
- 网络连接模块
- 多媒体处理芯片

\subsection{传感器}

- 语音识别传感器
- 环境传感器
- 运动传感器
- 位置传感器

\subsection{接口设计}

- 触摸屏
- 语音接口
- 网络接口
- 音频接口
- 智能家居接口

\section{智能收音机的软件设计}

\subsection{操作系统}

- 嵌入式操作系统
- 实时操作系统
- 移动操作系统

\subsection{应用程序}

- 语音识别软件
- 智能推荐算法
- 网络广播应用
- 智能家居控制应用

\subsection{云服务}

- 语音助手服务
- 内容推荐服务
- 用户数据存储
- 远程控制服务

\chapter{绿色环保收音机设计}

绿色环保是现代产品设计的重要理念,收音机设计也不例外。

\section{绿色环保设计的理念}

绿色环保设计的理念包括:

1. **节能设计**:减少能源消耗
2. **材料环保**:使用环保材料
3. **可回收设计**:便于回收利用
4. **低污染设计**:减少对环境的污染

\section{节能设计}

- 低功耗电路设计
- 智能电源管理
- 高效电源转换
- 待机功耗优化

\section{材料选择}

- 环保材料的选择
- 可降解材料的使用
- 无毒无害材料
- 可再生材料

\section{可回收设计}

- 模块化设计
- 易于拆卸的结构
- 材料标识
- 回收指南

\section{低污染设计}

- 减少有害物质的使用
- 降低电磁辐射
- 减少噪声污染
- 环保生产工艺

\chapter{收音机在物联网中的应用}

物联网是互联网的延伸,收音机在物联网中可以发挥重要的作用。

\section{物联网的基本概念}

物联网(Internet of Things, IoT)是指通过各种信息传感设备,实时采集任何需要监控、连接、互动的物体或过程,采集其声、光、热、电、力学、化学、生物、位置等各种需要的信息,通过各类可能的网络接入,实现物与物、物与人的泛在连接,实现对物品和过程的智能化感知、识别和管理。

\section{收音机在物联网中的角色}

收音机在物联网中可以扮演以下角色:

1. **信息接收终端**:接收物联网中的广播信息
2. **环境监测节点**:监测环境参数并上传数据
3. **智能控制终端**:接收控制指令并执行相应的操作
4. **紧急通信设备**:在网络故障时提供紧急通信

\section{物联网收音机的设计}

\subsection{硬件设计}

- 网络连接模块
- 传感器模块
- 控制模块
- 电源管理模块

\subsection{软件设计}

- 物联网协议栈
- 数据处理软件
- 云服务接口
- 应用程序

\subsection{应用场景}

- 智能家庭
- 智能城市
- 智能农业
- 智能交通
- 环境监测

\chapter{参考书籍}

看图学装收音机  朱蔼初  少年儿童出版社 1984年,上海

矿石收音机 冯报本

无电源收音机	陈鹏飞	黑龙江科学技术出版社 1985	15217.170

\end{document}