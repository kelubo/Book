% 收音机
% 收音机.tex

\documentclass[12pt,UTF8]{ctexbook}

% 设置纸张信息。
% 纸张设置配置文件
% 用于定义书籍的页面尺寸和边距

\usepackage[a4paper,twoside]{geometry}
\geometry{
	left=25mm,
	right=20mm,
	top=25mm,
	bottom=25.4mm,
	headsep=1cm, 
    footskip=1cm,
	bindingoffset=10mm
}

% 设置字体,并解决显示难检字问题。
\xeCJKsetup{AutoFallBack=true}
\setCJKmainfont{SimSun}[BoldFont=SimHei, ItalicFont=KaiTi, FallBack=SimSun-ExtB]

% 目录 chapter 级别加点(.)。
\usepackage{titletoc}
\titlecontents{chapter}[0pt]{\vspace{3mm}\bf\addvspace{2pt}\filright}{\contentspush{\thecontentslabel\hspace{0.8em}}}{}{\titlerule*[8pt]{.}\contentspage}

% 设置 part 和 chapter 标题格式。
\ctexset{
	chapter/name={第,章},
	chapter/number={\chinese{chapter}}
}

% 图片相关设置。
\usepackage{graphicx}
\graphicspath{{Images/}}

% 设置署名格式。
\newenvironment{shuming}{\hfill\zihao{4}}

% 注脚每页重新编号,避免编号过大。
\usepackage[perpage]{footmisc}

\title{\heiti\zihao{0} 收音机}
\author{佚名}
\date{}

\begin{document}

\maketitle
\tableofcontents

\frontmatter

\mainmatter

\chapter{矿石收音机}

矿石收音机是一种最简单,最济的收音机,需用器材不多,制作容易,不需要维持费用。这种简单而又经济的收音机,在装置上和检修上都不需要特殊的技术,最合于一般初学的无线电爱好者研究之用。

虽然矿石收音机也有它的缺点,如收程不远,声音小等,但是我国各省现在都已普遍地设立了人民广播电台,故在很大的地区内,矿石收音机仍可使用的。

==============================================================================每一个无电爱好者的研究工作,差不多都是从石收机开始的,本母就是为了具体帮助初学者去制造矿石收音机而寫。里面說明了接收原理;矿石收音机主要零件的 機造 和 性能;收晋机的制作和稚修。采用的零件都是容易贸到或可以自制的,读者只要依酰明安装;就是从來没有学督过无线电的人也能成功。
这是一个无线电爱好者从事制作的开端,在装管碳石收机獲得煦之后,就可以在这个基碰上,选一步去装电学管收管机了。
如果讀者还想在理論方面和制作方面作深一步的研,下面这几本者是比较适合的:
1.'初等电工学
苏联 ·耶列柏卓夫著

\backmatter

矿石收音机 冯报本

\end{document}