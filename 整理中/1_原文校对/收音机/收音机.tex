% 收音机
% 收音机.tex

\documentclass[12pt,UTF8]{ctexbook}

% 设置纸张信息。
% 纸张设置配置文件
% 用于定义书籍的页面尺寸和边距

\usepackage[a4paper,twoside]{geometry}
\geometry{
	left=25mm,
	right=20mm,
	top=25mm,
	bottom=25.4mm,
	headsep=1cm, 
    footskip=1cm,
	bindingoffset=10mm
}

% 设置字体,并解决显示难检字问题。
\xeCJKsetup{AutoFallBack=true}
\setCJKmainfont{SimSun}[BoldFont=SimHei, ItalicFont=KaiTi, FallBack=SimSun-ExtB]

% 目录 chapter 级别加点(.)。
\usepackage{titletoc}
\titlecontents{chapter}[0pt]{\vspace{3mm}\bf\addvspace{2pt}\filright}{\contentspush{\thecontentslabel\hspace{0.8em}}}{}{\titlerule*[8pt]{.}\contentspage}

% 设置 part 和 chapter 标题格式。
\ctexset{
	chapter/name={第,章},
	chapter/number={\chinese{chapter}}
}

% 图片相关设置。
\usepackage{graphicx}
\graphicspath{{Images/}}

% 设置署名格式。
\newenvironment{shuming}{\hfill\zihao{4}}

% 注脚每页重新编号,避免编号过大。
\usepackage[perpage]{footmisc}

\title{\heiti\zihao{0} 收音机}
\author{佚名}
\date{}

\begin{document}

\maketitle
\tableofcontents

\frontmatter

\mainmatter



\chapter{天线}

天线的符号和实物见图2-2。它是一根水平张挂着的导线,用引入线通到室内,和收音机电路连接。在离电台5公里左右的市区内,只要用一根长5~7米的导线,通向室外,悬挂在3米左右高的地方就可以了。离电台较远的市郊,天线就应适当地加长、加高。离电台四、五十公里以外,尤其是有山岭阻隔的地方,本章所介绍的简单装置是很难收到电台播音的。

\begin{figure}[htbp]
	\centering
	\includegraphics[width=0.7\linewidth]{14}
	\caption{}
	\label{fig:1}
\end{figure}

\section{什么叫天线,为什么要用天线}

无线电收音机收到的远地广播电台的播音,是靠一种看不见、嗅不出、摸不到的所谓“无电波”(或者叫电磁波)来传递的。这种电波从广播电台向四周围发射出来,就好像声波在空气中向四周扩散一样。

\begin{figure}[htbp]
	\centering
	\includegraphics[width=0.7\linewidth]{1}
	\caption{}
	\label{fig:1}
\end{figure}

无线电波虽然漫天遍地都是,但是我们的感觉器官却无法直接感受,无法捕捉。“天线”就是用来捕捉那“来无影去无踪”的无线电波的。打个比方,正好像高挂着的蛛网,又好像昆虫用以探索物体的触须。所以无线电发明者 AC.波波夫就叫它为“Antenna”\footnote{Antenna是从希腊文“触须”一字转借来的,是法国物理学家布隆德尔在波波夫发明天线时写给波波夫信中第一次提出的。},无线电波虽然看不见、嗅不出、摸不着,但它一碰到金属等导电物体就会在这物体中感应出来相应的电压。所以天线是用良导体(如铜线等)做的。

虽然无线电波是无孔不入的,但它也会被高山和高大的建筑物等所阻挡或减弱,所以天线一般都高架在空中。

因此,可以简单地为天线下个定义:“天线\footnote{这里光指收信天线。}是用来捕捉(接收)无线电波的,高张在空中的金属线”。

因为天线是收音机的第一道门户,所以天线的好坏对收音机工作的好坏有非常密切的关系,尤其对比较简单的收音机来说更为重要。例如矿石收音机和单管机,如果没有一根比较好的天线,就不能顺利地收音。

\section{天线的种类}

天线的种类很多,有的以天线外形来分(如T形,Г形等);有的用它的工作原理来分(如行波天线、同相天线等);有用它的工作效能来分的(有定向及非定向等);有以用途来分的。

以用途分大体可分接收天线和发射天线,接收天线就是指用于收音机收报机上的,是用来接收电波的。发射天线是用于发射机的,用以发射电波。

简单的常用接收天线有定向非定向,有T形、Г形、垂直、刷形、环形,有防干扰用的特种天等;其中Г形及T形天线,业余无线电爱好者用得最多。

Г形天线又叫倒L形天线,主要是由水平悬挂的天线和引下接至收音机用的引下线以绝缘子、拉线等附属装置组成。因为它的外形象俄文字母Γ字(或倒的英文字母L)而得名。

\begin{figure}[htbp]
	\centering
	\includegraphics[width=0.7\linewidth]{2}
	\caption{}
	\label{fig:1}
\end{figure}

Γ形天线略有方向性,他接收由引下线一端来的电波的能力最强。但这个性能不很显著,基本上属于非方向性天线。

T形天线和Γ形天线很相像,所不同的是它的引下线不是从天线的一端接出,而是从它的中间接出。这种天线虽然接收来自两头的电波的能力稍强一些,但也不明显,所以也是属于非方向性的。

\begin{figure}[htbp]
	\centering
	\includegraphics[width=0.7\linewidth]{3}
	\caption{}
	\label{fig:1}
\end{figure}

垂直天线及倾斜天线如下图所示。这两种天线没有水平部分,只有一条重直挂着的或斜挂着的铜线。

\begin{figure}[htbp]
	\centering
	\includegraphics[width=0.7\linewidth]{4}
	\caption{}
	\label{fig:1}
\end{figure}

\begin{figure}[htbp]
	\centering
	\includegraphics[width=0.7\linewidth]{5}
	\caption{}
	\label{fig:1}
\end{figure}

刷形天线或叫集中电容式天线。它用电容很大的刷状或螺状导线束来代替Г形、T形等的水平部分。

\begin{figure}[htbp]
	\centering
	\includegraphics[width=0.7\linewidth]{6}
	\caption{}
	\label{fig:1}
\end{figure}

室内天线及代用天线。最简单的室内天线为一条拖在收音机外面的几尺长的线段,稍考究些的可在天花板下面拉上一段导线,或螺旋形天线。

\begin{figure}[htbp]
	\centering
	\includegraphics[width=0.7\linewidth]{7}
	\caption{}
	\label{fig:1}
\end{figure}

\begin{figure}[htbp]
	\centering
	\includegraphics[width=0.7\linewidth]{8}
	\caption{}
	\label{fig:1}
\end{figure}

至于代用天线,它的种类形式就更多了。因为上面已经说过,无线电波不单能在挂着的导线上激起电压(或电流),在一切的导体上都能产生电压。所以如电话机、电灯、铁皮屋顶以至在铁床,铜网纱窗等都可当作代用天线。金属导体的面积愈大,与地的绝缘愈好(对高频电流而言),那么代天线的效果也就愈好。

\section{怎样架设天线}

架设哪一种天线好,这是初学的业余无电爱好者所迫切需要解决的问题,可是对于这问题也很难作出一个“放之四海而皆准”的答复。因为采用那一种天线,要看你所用的收普机和所处的环境来决定。一般来说,装一条Г形天线对于任何收音机都能适用。

\subsection{Γ形天线}

Г形天线是由一根长约10-30米的,高悬着的(约10-20米)多股铜线和引下线组成。当然,单纯从收音的音量强度和收音距离的观点来看,天线愈长愈高,效果也就愈好。但是太长太高了会带来很大的天电干扰、工业干扰及其他妨碍收音的杂音。若附近有大电台时更将会引起夹音,反而不能很好收听。所以天线的长短、高低要看具体环境而定。

一般地说,在附近没有大电台,没有工业干扰(如在村),收音机比较简单,那就应将天线架得长些、高些;若在工业干扰很大、附近电台林立(如在大城市)的地区,天线就应短些、低些。

对一般矿石收音机来说可用长25--30米的天线;二、三管再生式收音机可用15--20米的;超外差收音机可用8--10米,或甚至只用一条几米的短线段。

天线的导线是用由多股0.5--0.7公厘裸线绞合成的2--3公厘的绞合线,或不小于1.5--2公厘的单股铜(16号或14号紫铜线)。不可用黄铜线或铝线,因为它们会很快氧化而变成非常脆弱。在真正没有办法时用1.5--2公厘的镀锌铁线,也可勉强应用。

天线最好是用整条的线,若条件不许可时也可将几段接起来,但必须加以焊接(不要用带酸性的焊剂)。

除导线外,还有一种主要材料是“绝缘子”。绝缘子是不导电的,可以防止天线上的高频电流经杆子等漏入地中。

绝缘子有好多种(可在无线电料商店或电料行中购得),按照制造它们的材料来分,有玻璃的和瓷的,玻璃的比较好,两端孔是穿线用的。瓷绝缘子形如蛋形,故叫蛋形绝缘子,有大有小,收音机天线用直径一寸的就可以了。假如上面两种绝缘子都没有,也可用普通装电灯线的双孔瓷夹板或瓷壶(也叫鼓形白料,)来代替,不得已时也可用玻璃瓶,或甚至用白腊浸煮过的硬木头、纱线圈。

\begin{figure}[htbp]
	\centering
	\includegraphics[width=0.7\linewidth]{9}
	\caption{}
	\label{fig:1}
\end{figure}

若引下线要通过墙或窗口,那么还需要几个瓷套管或硬橡胶套管。

\begin{figure}[htbp]
	\centering
	\includegraphics[width=0.7\linewidth]{10}
	\caption{}
	\label{fig:1}
\end{figure}

此外两根高的10-20米的木杆或竹杆,以及若于2.4--4公厘的铁丝(作拉紧天线杆用)和一些螺钉、钉子等,若有一小滑轮就更好了。

在动手架设天线之前必须首先选择好场地。在选择时要考虑到便于利用邻近的房屋、树木等,应尽可能使天线水平部分不跨越屋顶、树木等,使天线水平部分有足够的空间,并且使它尽可能地高些。

其次应法意下列事项:

(1)不使天线和屋篇、墙壁、树木、自来水管、煤气管等相碰

(2)天线不要和电灯线、电话靠得太近或平行,也不要横越这些线上,特别是不要架在高压电力线的上面或下面,因为这样不但容易受到干扰,而且也非常危险。

(3)引人线最好不要贴着墙壁走得太长,也不要拐许多弯,引入线愈短愈好,这样可以减小损失。

为了减短天线杆的长度,可将天线杆固定在屋顶上、大树上或其他高建筑物上。若屋顶上不易装,那么就只好将杆子立在地上了。

为了防止天线上的高频电流经过杆子漏掉,在天线和杆子之间必须要加几个绝缘子,一般每端用两个到三个。它的接法如图。

\begin{figure}[htbp]
	\centering
	\includegraphics[width=0.7\linewidth]{11}
	\caption{}
	\label{fig:1}
\end{figure}

但不能如下图那样。因为那样接时会使绝缘子受到拉力而破碎。

\begin{figure}[htbp]
	\centering
	\includegraphics[width=0.7\linewidth]{12}
	\caption{}
	\label{fig:1}
\end{figure}

=====================================================================================================



\section{特种天线}


\chapter{地线}

地线的符号和装置方法见图2-3,它是一条把收音机电路和大地连接起来的导线。用长约1米的铁棒,插入地里,再用导线引入室内和收音机相接。插铁棒的地方还应该浇一点水,使铁棒和大地间导电良好。在有自来水设备的地方装置地线就更方便了,只要用一根导线把收音机电路和水管连接起来就可以了。

\begin{figure}[htbp]
	\centering
	\includegraphics[width=0.7\linewidth]{15}
	\caption{}
	\label{fig:1}
\end{figure}

天线和地线是用来接收无线电波的。如果把天线和地线连接起来,在无线电波的作用下,天线和地线之间便会有高频电流流动着。

\section{怎样装地线}

\chapter{耳机}

图2-4是耳机的符号、外形和构造。在耳机里有一个马蹄形的永久磁铁,在磁铁上绕着线圈,磁铁的磁极前有一片薄铁片。当通过线圈电流的方向和强度发生变化时,磁铁吸引铁片的力量也就发生变化,因此铁片振动起来,推动耳机内的空气,造成声波,声波传到我们的耳朵里,便听到了声音。

\begin{figure}[htbp]
	\centering
	\includegraphics[width=0.7\linewidth]{16}
	\caption{}
	\label{fig:1}
\end{figure}

耳机内的磁铁,磁性越强,发出来的声音就越响、越逼真。

耳机由于线圈绕线的粗细和圈数不同,可分高阻抗(如800欧)和低阻抗(如8欧)两种,本书用的都是高阻抗耳机。

耳机是否损坏,可用万用电表测量电阻的Rx100档来测试(图2-5a),在正常的情况下,测得高阻抗耳机的线圈电阻,应该是800欧左右,如果表针不动,那就说明已经断线了。更简便的方法是用一节干电池和耳机串联起来,把耳机引出线的一头在电池的电极上刮来刮去,耳机内能发出嚓嚓的声音(图2-5b)。如果一点声音也没有,就说明线圈或引出线已经断了。

\begin{figure}[htbp]
	\centering
	\includegraphics[width=0.7\linewidth]{17}
	\caption{}
	\label{fig:1}
\end{figure}

\chapter{二极管}

二极管的符号和外形如图2-6。二极管的种类很多,这里用的是2AP9型检波二极管。二极管的符号形象地说明了它单方向导电的性能:符号箭杆这一端是正极(+),箭头所指这一端是负极(-),电流只能从正极流向负极。

\begin{figure}[htbp]
	\centering
	\includegraphics[width=0.7\linewidth]{18}
	\caption{}
	\label{fig:1}
\end{figure}

在电路里,二极管就担负起“指挥交通”的任务,使原来应该一来一往流动的交流电,变成只来不往的单方向电流。

二极管质量的好坏可以用万用表的RX100档来测试。把正表棒(红色表棒)接二极管的负极,负表棒(黑色表棒)接二极管的正极,测得的“正向电阻”应该在500欧姆以下;反过来把正表棒接二极管的正极,负表棒接二极管的负极,测得的“反向电阻”应该在200千欧以上;这样的管子才是合用的。如果测得的正、反向电阻都很大或都很小,则说明管子质量很差,不能应用。应用万用表,还可以判别二极管的正负极:测得正向电阻的那一次,负表棒所接触的是二极管的正极,正表棒接触的是负极(图2-7)。

\begin{figure}[htbp]
	\centering
	\includegraphics[width=0.7\linewidth]{19}
	\caption{}
	\label{fig:1}
\end{figure}

如果没有万用表,应用电池和耳机,也可以大致判别二极管的好坏:先把二极管、电池和耳机像图2-8a那样连接起来,用耳机的一条引出线去刮二极管的负极,耳机里当能发出较响的嚓嚓声;再把二极管的正负极反过来,像图2-8b那样连接,用耳机的一条引出线去刮二极管的正极时,耳机里只能发出极轻的声音来。这样的二极管才是好的。如果两次试验声音都很响或都极轻,这个管子就不能用了。同样,这种方法也可以用来判别二极管的极性:在声较响的那次试验中,电池正极所接的正是二极管的正极。

\begin{figure}[htbp]
	\centering
	\includegraphics[width=0.7\linewidth]{20}
	\caption{}
	\label{fig:1}
\end{figure}

\chapter{矿石收音机}

矿石收音机是一种最简单,最济的收音机,需用器材不多,制作容易,不需要维持费用。这种简单而又经济的收音机,在装置上和检修上都不需要特殊的技术,最合于一般初学的无线电爱好者研究之用。

虽然矿石收音机也有它的缺点,如收程不远,声音小等,但是我国各省现在都已普遍地设立了人民广播电台,故在很大的地区内,矿石收音机仍可使用的。

\section{最简单收音机}

初次动手,先来试装一个十分简单的收音机。图2-1是它的电路图和实体接线图,它由天线、地线、耳机和晶体二极管组成。电路非常简单,一装就响。

\begin{figure}[htbp]
	\centering
	\includegraphics[width=0.7\linewidth]{13}
	\caption{}
	\label{fig:1}
\end{figure}

整块印制板(图1-2),划分为五区,这次试装在左下角的第2 区内进行。

(1)取长约1米的软接线一段,一端剥出2毫米左右的线头,焊接在电路板上,另一端剥去20毫米左右的线头与天线绞接(图 2-9a)。

(2)取长约1米的软接线,照图2-9b接入印制板,另一端与地线绞接。

(3)把耳机接入印制板(图 2-9c)。试听一下,耳机里寂然无声。

(4)把二极管的两条接脚用小刀刮清爽,搪好锡,根据印制板上的安装位置(图2-9d),弯折好接脚,焊入印制板。再来试听一下,哦!听到电台的播音了。

\begin{figure}[htbp]
	\centering
	\includegraphics[width=0.7\linewidth]{21}
	\caption{}
	\label{fig:1}
\end{figure}

\subsection{电路原理}

我们用的电灯是交流电。交流电在电路里是一来一去地流动着的,电流的方向来去变化一次所需的时间叫做周期T,周期的长短是用秒做单位来计算的。交流电(市电)的交变周期是1/50秒。交流电每秒钟来去方向交变的次数叫做频率f,用赫兹做单位来计算,交流电的频率是50赫兹,就是说电流来去方向的变化是每秒钟50次。

\begin{figure}[htbp]
	\centering
	\includegraphics[width=0.7\linewidth]{22}
	\caption{}
	\label{fig:1}
\end{figure}

利用话简,可以把声波转换成相应的交流电,这种交流电通过耳机时,就能发出原来那样的声音来。人们的耳朵只能听到每秒钟振动20次到2万次的声波,振动得太快或太慢都是听不见的。因此把频率为20赫兹到20千赫兹的交流电流称做音频电流。

音频电流只能沿着导线流动。打电话时,电话线一断,电话就不通了。

\begin{figure}[htbp]
	\centering
	\includegraphics[width=0.7\linewidth]{23}
	\caption{}
	\label{fig:1}
\end{figure}

频率在100千赫兹以上的电流,叫做高频电流。高频电流能使周围的空间里产生无线电波,向四面八方传播开去,所以高频信号可以用来进行无线电通信。

高频电流
在无线电广播电台里,是把音频信号加于高频信号上,使高频信号电流的峰值随着音频信号而作忽
大忽小的变化--这种信号叫做高频调幅信号。用这样的办法,叫高频信号带着音频信号,发射成为无线电波,由近及远地传播开去。当电台发出的无线电波经过我们这里时,便会在我们架设的天、地线之间感应出一个高频信号电压来。在天、地线之间仅接有一个耳机时,由于信号的频率大大地高出音频范围,因此听不到声音。
在加接二极管后,高频电流以地线流向天线时,二极管正向直导通,电流从二极管这一边流过;高频电流从天线流向地线时:二极管反向截止。因此,电流只能从耳机这一边的电路里流过。流过耳机这一边的是一个方向不变,强度随时间而变的脉动电流。这脉动电流里,就包含了音频信号的成分。当音信号电流流过耳机时,耳机里就发出声音来了。这里二极管起着从高

频信号电流中分离出音频信号来的作用,这种作用称做“检波”这个收音装置毕竟太简单了,只有一个检波器,没有选择电台的能力。在耳机里时常夹杂着几个电台的播音声,使我们喜爱的节目也听不清楚了。解决这个问题的方法是在检波电路前加装“调谐电路”,在第三章里我们将详细讲解调谐电路的装置方法和原理。

在二极管检波电路前,加装一个由线圈和可变电容器构成的调谐电路,就成为一具可以选择电台的收音机了。这具收音机的特点是用不着装干电池,所以称做无电源收音机。图3-1是本机电路图和实体接线图。



==============================================================================每一个无电爱好者的研究工作,差不多都是从石收机开始的,本母就是为了具体帮助初学者去制造矿石收音机而寫。里面說明了接收原理;矿石收音机主要零件的 機造 和 性能;收晋机的制作和稚修。采用的零件都是容易贸到或可以自制的,读者只要依酰明安装;就是从來没有学督过无线电的人也能成功。
这是一个无线电爱好者从事制作的开端,在装管碳石收机獲得煦之后,就可以在这个基碰上,选一步去装电学管收管机了。
如果讀者还想在理論方面和制作方面作深一步的研,下面这几本者是比较适合的:
1.'初等电工学
苏联 ·耶列柏卓夫著



\backmatter

\chapter{示例电路板}

用一块75x140平方毫米的单面铜箔板,采用刀刻法来制作:先把透明薄纸覆在图1-1上,用铅笔照样描下来,再用复写纸复印在铜箔板上,然后用斜口小刀刻去线条上的铜箔。刻的时候右手拿刀,刀口向前,刀尖紧抵铜箔,左手拇指顶住刀背略往前推,右手却稍往后拉,利用杠杆作用,使刀口向前刻划。

\begin{figure}[htbp]
	\centering
	\includegraphics[width=0.7\linewidth]{pcb1}
	\caption{}
	\label{fig:1}
\end{figure}

刻出来的线条,宽约3毫米,粗细要均匀。可在线条的一边先刻一道划痕,再在另一边刻一道划痕,并把线条两端刻断,然后用刀尖在一端挑起边角,就可把应刻去的铜箔撕下来(图1-3)。如果第一遍刻的划痕太浅,可连刻二遍、三遍。

\begin{figure}[htbp]
	\centering
	\includegraphics[width=0.7\linewidth]{fl_1}
	\caption{}
	\label{fig:1}
\end{figure}

钻孔前,先要用冲头(或钢针)在孔的中心凿一个凹痕,这样钻孔时钻头才不会滑动,装插一般元件接脚的孔径是1.2毫米,装输入、输出变压器的孔径是1.5毫米左右,装螺丝的孔径是3毫米。如果没有适当大小的钻头,可先钻一个小孔,用斜口小刀把它适当扩大就行。装双连可变电容器的大孔,孔径是10毫米,还须用尖头木锉或圆锉来进一步加工(图1-4)。

\begin{figure}[htbp]
	\centering
	\includegraphics[width=0.7\linewidth]{fl_2}
	\caption{}
	\label{fig:1}
\end{figure}

钻好孔后,用细砂皮轻轻打磨铜箔,除去污物和氧化层,使表面光洁明亮,然后在铜箔面均匀地涂刷一层松香溶液(图1)。

\begin{figure}[htbp]
	\centering
	\includegraphics[width=0.7\linewidth]{fl_3}
	\caption{}
	\label{fig:1}
\end{figure}

松香溶液配制的方法是:在墨水瓶里盛半瓶95\%的酒精,放入六、七颗蚕豆大小的松香,用筷子搅拌,使它溶解。这种松香溶液涂在铜箔上,其中的酒精很快地蒸发掉,松香在铜表面形成一层薄膜,它能保护铜箔表面,防止氧化。在焊接时,松香还起着助焊的作用,使铜箔容易上锡。余下的松香溶液,留着在焊接时作助焊剂。

\begin{figure}[htbp]
	\centering
	\includegraphics[width=0.7\linewidth]{dl_1}
	\caption{}
	\label{fig:1}
\end{figure}

\chapter{怎样焊接}

对于初学者来说,焊接工作的好坏,是装置收音机成败的关键。装置晶体管收音机,应该用45瓦以下小功率的铁。烙铁的功率过大或焊接的时间过长,都会使元件因受热而变值、损坏。一般用20瓦内热式电烙铁比较合适。它的构造如图1-6,烙铁芯装在金属管里,它是一条有夹层的瓷管,夹层里绕着电热丝通电后就会发热。烙铁头用纯铜制成,套在金属管外面。小功率的电烙铁,像图1-6那样拿,焊接起来最方便。焊接时,烙铁头的温度过低,焊锡熔不开、焊不牢;温度太高,烙铁头又常常被烧死(即表层氧化发黑,吃不上锡)。烙铁头被烧死时,可用锉刀把它表面的氧化层锉掉,立刻蘸上松香、焊锡,便可继续使用。

\begin{figure}[htbp]
	\centering
	\includegraphics[width=0.7\linewidth]{fl_4}
	\caption{}
	\label{fig:1}
\end{figure}

焊接前,先要在元件接脚或接线的线头上搪一层锡。方法,用斜口小刀把元件接脚(或线头)表面充分刮消爽,涂上一点松香溶液,用尖头钳夹持元件接脚的根部,烙铁头上饱蘸熔锡,接触元件接脚,自左向右,搪上一层焊锡。用尖头钳夹持接脚根部,可把热量引走,以免传入元件内部,使元件因过热而损坏(图1-7)。

\begin{figure}[htbp]
	\centering
	\includegraphics[width=0.7\linewidth]{fl_5}
	\caption{}
	\label{fig:1}
\end{figure}

接线的线头也要预先搪锡。加接在印制板上的接线,应该用质地较硬的单芯塑料绝缘接线,装好后比较挺括牢靠。从线路板引出的接线(如与电源、扬声器连接的导线)应该用多股线芯、质地柔软的塑料绝缘接线,才经得起多次弯折。搪锡前先要剥出2毫米左右的接头。剥线头的方法是:利用热的烙铁头把塑料层熔断,趁热用手把割断的那段塑料皮拉去(图1-8)。刚去皮的新接线很容易上锡,不须刮削,只要涂上一点助焊剂,就可搪锡。

\begin{figure}[htbp]
	\centering
	\includegraphics[width=0.7\linewidth]{fl_6}
	\caption{}
	\label{fig:1}
\end{figure}

元件接脚搪好锡后,根据印制板来弯折接脚,使几条接脚刚好能插入印制板相应的位置。弯脚时要留意:使元件插入线路板后,标有数值的一面朝上,以便于装机时核对。

\begin{figure}[htbp]
	\centering
	\includegraphics[width=0.7\linewidth]{fl_7}
	\caption{}
	\label{fig:1}
\end{figure}

元件接脚从线路板无铜箔的一面(简称A面)插入,穿出有铜箔的一面(简称B面),留下1.5毫米左右的接脚,把多余部分剪掉。然后用烙铁饱蘸焊锡,进行焊接。焊接时烙铁头应和焊点紧密接触,使熔锡充分浸润接脚和铜箔,并且吃牢在接脚和铜箔上。提起烙铁时,用手指弹击一下线路板,可使焊点圆浑不起毛刺。焊接的时间要掌握在2秒钟以内,时间过长,会使元件过热而损坏。焊接的过程如图1-9所示。焊点要求圆浑光亮不起毛刺,焊锡和接脚、铜箔要吃得很牢。用锡多少适宜,焊点不过大或过小(图 1-10)。

\begin{figure}[htbp]
	\centering
	\includegraphics[width=0.7\linewidth]{fl_8}
	\caption{}
	\label{fig:1}
\end{figure}

焊接得不好时,元件的接脚只是被焊锡含住,或锡粒只是靠松香粘结在铜箔上,没有真正的焊牢,这样的焊点,称做“假焊”。假焊的焊点往往有很大的电阻,有的时通时断,有的索性不通,所以会使电路出现许多莫名其妙的故障,个别接点的假焊,还有导致元件损坏的危险(如三极管的下偏流电阻)。

\begin{figure}[htbp]
	\centering
	\includegraphics[width=0.7\linewidth]{fl_9}
	\caption{}
	\label{fig:1}
\end{figure}

造成假焊的原因常有下列几个方面:

一是元件的接脚或线头没有刮清爽,连搪锡也没有搪好。

二是印制板铜箔表面染有污物或搁置日久表面氧化。

三是烙铁头温度太高或太低。

四是熔锡氧化已不适宜用来焊接,刚蘸上烙铁头的熔锡,表面光亮流动性很好,及时用来焊接,就容易和接脚、铜吃牢。如搁置时间过长,熔锡氧化,表面灰暗粗糙,失去了流动性,这样的熔锡就不能再和接脚、铜箔的表面结合,就会造成假焊。如果熔锡已经氧化,只要再蘸上点松香,就能使它还原,仍旧可以用来焊接。

五是烙铁头表面氧化或沾有污物,影响焊接。

六是烙铁头在焊点上停留的时间太短,焊锡还来不及和接脚、铜箔吃牢。

\chapter{参考书籍}

看图学装收音机  朱蔼初  少年儿童出版社 1984年,上海

矿石收音机 冯报本

无电源收音机	陈鹏飞	黑龙江科学技术出版社 1985	15217.170

\end{document}