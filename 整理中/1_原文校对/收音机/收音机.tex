% 收音机
% 收音机.tex

\documentclass[12pt,UTF8]{ctexbook}

% 设置纸张信息。
% 纸张设置配置文件
% 用于定义书籍的页面尺寸和边距

\usepackage[a4paper,twoside]{geometry}
\geometry{
	left=25mm,
	right=20mm,
	top=25mm,
	bottom=25.4mm,
	headsep=1cm, 
    footskip=1cm,
	bindingoffset=10mm
}

% 设置字体,并解决显示难检字问题。
\xeCJKsetup{AutoFallBack=true}
\setCJKmainfont{SimSun}[BoldFont=SimHei, ItalicFont=KaiTi, FallBack=SimSun-ExtB]

% 目录 chapter 级别加点(.)。
\usepackage{titletoc}
\titlecontents{chapter}[0pt]{\vspace{3mm}\bf\addvspace{2pt}\filright}{\contentspush{\thecontentslabel\hspace{0.8em}}}{}{\titlerule*[8pt]{.}\contentspage}

% 设置 part 和 chapter 标题格式。
\ctexset{
	chapter/name={第,章},
	chapter/number={\chinese{chapter}}
}

% 图片相关设置。
\usepackage{graphicx}
\graphicspath{{Images/}}

% 设置署名格式。
\newenvironment{shuming}{\hfill\zihao{4}}

% 注脚每页重新编号,避免编号过大。
\usepackage[perpage]{footmisc}

\title{\heiti\zihao{0} 收音机}
\author{佚名}
\date{}

\begin{document}

\maketitle
\tableofcontents

\frontmatter

\mainmatter



\chapter{天线}

天线的符号和实物见图2-2。它是一根水平张挂着的导线,用引入线通到室内,和收音机电路连接。在离电台5公里左右的市区内,只要用一根长5~7米的导线,通向室外,悬挂在3米左右高的地方就可以了。离电台较远的市郊,天线就应适当地加长、加高。离电台四、五十公里以外,尤其是有山岭阻隔的地方,本章所介绍的简单装置是很难收到电台播音的。

\begin{figure}[htbp]
	\centering
	\includegraphics[width=0.7\linewidth]{14}
	\caption{}
	\label{fig:1}
\end{figure}

\section{什么叫天线,为什么要用天线}

无线电收音机收到的远地广播电台的播音,是靠一种看不见、嗅不出、摸不到的所谓“无电波”(或者叫电磁波)来传递的。这种电波从广播电台向四周围发射出来,就好像声波在空气中向四周扩散一样。

\begin{figure}[htbp]
	\centering
	\includegraphics[width=0.7\linewidth]{1}
	\caption{}
	\label{fig:1}
\end{figure}

无线电波虽然漫天遍地都是,但是我们的感觉器官却无法直接感受,无法捕捉。“天线”就是用来捕捉那“来无影去无踪”的无线电波的。打个比方,正好像高挂着的蛛网,又好像昆虫用以探索物体的触须。所以无线电发明者 AC.波波夫就叫它为“Antenna”\footnote{Antenna是从希腊文“触须”一字转借来的,是法国物理学家布隆德尔在波波夫发明天线时写给波波夫信中第一次提出的。},无线电波虽然看不见、嗅不出、摸不着,但它一碰到金属等导电物体就会在这物体中感应出来相应的电压。所以天线是用良导体(如铜线等)做的。

虽然无线电波是无孔不入的,但它也会被高山和高大的建筑物等所阻挡或减弱,所以天线一般都高架在空中。

因此,可以简单地为天线下个定义:“天线\footnote{这里光指收信天线。}是用来捕捉(接收)无线电波的,高张在空中的金属线”。

因为天线是收音机的第一道门户,所以天线的好坏对收音机工作的好坏有非常密切的关系,尤其对比较简单的收音机来说更为重要。例如矿石收音机和单管机,如果没有一根比较好的天线,就不能顺利地收音。

\section{天线的种类}

天线的种类很多,有的以天线外形来分(如T形,Г形等);有的用它的工作原理来分(如行波天线、同相天线等);有用它的工作效能来分的(有定向及非定向等);有以用途来分的。

以用途分大体可分接收天线和发射天线,接收天线就是指用于收音机收报机上的,是用来接收电波的。发射天线是用于发射机的,用以发射电波。

简单的常用接收天线有定向非定向,有T形、Г形、垂直、刷形、环形,有防干扰用的特种天等;其中Г形及T形天线,业余无线电爱好者用得最多。

Г形天线又叫倒L形天线,主要是由水平悬挂的天线和引下接至收音机用的引下线以绝缘子、拉线等附属装置组成。因为它的外形象俄文字母Γ字(或倒的英文字母L)而得名。

\begin{figure}[htbp]
	\centering
	\includegraphics[width=0.7\linewidth]{2}
	\caption{}
	\label{fig:1}
\end{figure}

Γ形天线略有方向性,他接收由引下线一端来的电波的能力最强。但这个性能不很显著,基本上属于非方向性天线。

T形天线和Γ形天线很相像,所不同的是它的引下线不是从天线的一端接出,而是从它的中间接出。这种天线虽然接收来自两头的电波的能力稍强一些,但也不明显,所以也是属于非方向性的。

\begin{figure}[htbp]
	\centering
	\includegraphics[width=0.7\linewidth]{3}
	\caption{}
	\label{fig:1}
\end{figure}

垂直天线及倾斜天线如下图所示。这两种天线没有水平部分,只有一条重直挂着的或斜挂着的铜线。

\begin{figure}[htbp]
	\centering
	\includegraphics[width=0.7\linewidth]{4}
	\caption{}
	\label{fig:1}
\end{figure}

\begin{figure}[htbp]
	\centering
	\includegraphics[width=0.7\linewidth]{5}
	\caption{}
	\label{fig:1}
\end{figure}

刷形天线或叫集中电容式天线。它用电容很大的刷状或螺状导线束来代替Г形、T形等的水平部分。

\begin{figure}[htbp]
	\centering
	\includegraphics[width=0.7\linewidth]{6}
	\caption{}
	\label{fig:1}
\end{figure}

室内天线及代用天线。最简单的室内天线为一条拖在收音机外面的几尺长的线段,稍考究些的可在天花板下面拉上一段导线,或螺旋形天线。

\begin{figure}[htbp]
	\centering
	\includegraphics[width=0.7\linewidth]{7}
	\caption{}
	\label{fig:1}
\end{figure}

\begin{figure}[htbp]
	\centering
	\includegraphics[width=0.7\linewidth]{8}
	\caption{}
	\label{fig:1}
\end{figure}

至于代用天线,它的种类形式就更多了。因为上面已经说过,无线电波不单能在挂着的导线上激起电压(或电流),在一切的导体上都能产生电压。所以如电话机、电灯、铁皮屋顶以至在铁床,铜网纱窗等都可当作代用天线。金属导体的面积愈大,与地的绝缘愈好(对高频电流而言),那么代天线的效果也就愈好。

\section{怎样架设天线}

架设哪一种天线好,这是初学的业余无电爱好者所迫切需要解决的问题,可是对于这问题也很难作出一个“放之四海而皆准”的答复。因为采用那一种天线,要看你所用的收普机和所处的环境来决定。一般来说,装一条Г形天线对于任何收音机都能适用。

\subsection{Γ形天线}

Г形天线是由一根长约10-30米的,高悬着的(约10-20米)多股铜线和引下线组成。当然,单纯从收音的音量强度和收音距离的观点来看,天线愈长愈高,效果也就愈好。但是太长太高了会带来很大的天电干扰、工业干扰及其他妨碍收音的杂音。若附近有大电台时更将会引起夹音,反而不能很好收听。所以天线的长短、高低要看具体环境而定。

一般地说,在附近没有大电台,没有工业干扰(如在村),收音机比较简单,那就应将天线架得长些、高些;若在工业干扰很大、附近电台林立(如在大城市)的地区,天线就应短些、低些。

对一般矿石收音机来说可用长25--30米的天线;二、三管再生式收音机可用15--20米的;超外差收音机可用8--10米,或甚至只用一条几米的短线段。

天线的导线是用由多股0.5--0.7公厘裸线绞合成的2--3公厘的绞合线,或不小于1.5--2公厘的单股铜(16号或14号紫铜线)。不可用黄铜线或铝线,因为它们会很快氧化而变成非常脆弱。在真正没有办法时用1.5--2公厘的镀锌铁线,也可勉强应用。

天线最好是用整条的线,若条件不许可时也可将几段接起来,但必须加以焊接(不要用带酸性的焊剂)。

除导线外,还有一种主要材料是“绝缘子”。绝缘子是不导电的,可以防止天线上的高频电流经杆子等漏入地中。

绝缘子有好多种(可在无线电料商店或电料行中购得),按照制造它们的材料来分,有玻璃的和瓷的,玻璃的比较好,两端孔是穿线用的。瓷绝缘子形如蛋形,故叫蛋形绝缘子,有大有小,收音机天线用直径一寸的就可以了。假如上面两种绝缘子都没有,也可用普通装电灯线的双孔瓷夹板或瓷壶(也叫鼓形白料,)来代替,不得已时也可用玻璃瓶,或甚至用白腊浸煮过的硬木头、纱线圈。

\begin{figure}[htbp]
	\centering
	\includegraphics[width=0.7\linewidth]{9}
	\caption{}
	\label{fig:1}
\end{figure}

若引下线要通过墙或窗口,那么还需要几个瓷套管或硬橡胶套管。

\begin{figure}[htbp]
	\centering
	\includegraphics[width=0.7\linewidth]{10}
	\caption{}
	\label{fig:1}
\end{figure}

此外两根高的10-20米的木杆或竹杆,以及若于2.4--4公厘的铁丝(作拉紧天线杆用)和一些螺钉、钉子等,若有一小滑轮就更好了。

在动手架设天线之前必须首先选择好场地。在选择时要考虑到便于利用邻近的房屋、树木等,应尽可能使天线水平部分不跨越屋顶、树木等,使天线水平部分有足够的空间,并且使它尽可能地高些。

其次应法意下列事项:

(1)不使天线和屋篇、墙壁、树木、自来水管、煤气管等相碰

(2)天线不要和电灯线、电话靠得太近或平行,也不要横越这些线上,特别是不要架在高压电力线的上面或下面,因为这样不但容易受到干扰,而且也非常危险。

(3)引人线最好不要贴着墙壁走得太长,也不要拐许多弯,引入线愈短愈好,这样可以减小损失。

为了减短天线杆的长度,可将天线杆固定在屋顶上、大树上或其他高建筑物上。若屋顶上不易装,那么就只好将杆子立在地上了。

为了防止天线上的高频电流经过杆子漏掉,在天线和杆子之间必须要加几个绝缘子,一般每端用两个到三个。它的接法如图。

\begin{figure}[htbp]
	\centering
	\includegraphics[width=0.7\linewidth]{11}
	\caption{}
	\label{fig:1}
\end{figure}

但不能如下图那样。因为那样接时会使绝缘子受到拉力而破碎。

\begin{figure}[htbp]
	\centering
	\includegraphics[width=0.7\linewidth]{12}
	\caption{}
	\label{fig:1}
\end{figure}

=====================================================================================================



\section{特种天线}


\chapter{地线}

地线的符号和装置方法见图2-3,它是一条把收音机电路和大地连接起来的导线。用长约1米的铁棒,插入地里,再用导线引入室内和收音机相接。插铁棒的地方还应该浇一点水,使铁棒和大地间导电良好。在有自来水设备的地方装置地线就更方便了,只要用一根导线把收音机电路和水管连接起来就可以了。

\begin{figure}[htbp]
	\centering
	\includegraphics[width=0.7\linewidth]{15}
	\caption{}
	\label{fig:1}
\end{figure}

天线和地线是用来接收无线电波的。如果把天线和地线连接起来,在无线电波的作用下,天线和地线之间便会有高频电流流动着。

\section{怎样装地线}

\chapter{耳机}

图2-4是耳机的符号、外形和构造。在耳机里有一个马蹄形的永久磁铁,在磁铁上绕着线圈,磁铁的磁极前有一片薄铁片。当通过线圈电流的方向和强度发生变化时,磁铁吸引铁片的力量也就发生变化,因此铁片振动起来,推动耳机内的空气,造成声波,声波传到我们的耳朵里,便听到了声音。

\begin{figure}[htbp]
	\centering
	\includegraphics[width=0.7\linewidth]{16}
	\caption{}
	\label{fig:1}
\end{figure}

耳机内的磁铁,磁性越强,发出来的声音就越响、越逼真。

耳机由于线圈绕线的粗细和圈数不同,可分高阻抗(如800欧)和低阻抗(如8欧)两种,本书用的都是高阻抗耳机。

耳机是否损坏,可用万用电表测量电阻的Rx100档来测试(图2-5a),在正常的情况下,测得高阻抗耳机的线圈电阻,应该是800欧左右,如果表针不动,那就说明已经断线了。更简便的方法是用一节干电池和耳机串联起来,把耳机引出线的一头在电池的电极上刮来刮去,耳机内能发出嚓嚓的声音(图2-5b)。如果一点声音也没有,就说明线圈或引出线已经断了。

\begin{figure}[htbp]
	\centering
	\includegraphics[width=0.7\linewidth]{17}
	\caption{}
	\label{fig:1}
\end{figure}

\chapter{二极管}

二极管的符号和外形如图2-6。二极管的种类很多,这里用的是2AP9型检波二极管。二极管的符号形象地说明了它单方向导电的性能:符号箭杆这一端是正极(+),箭头所指这一端是负极(-),电流只能从正极流向负极。

\begin{figure}[htbp]
	\centering
	\includegraphics[width=0.7\linewidth]{18}
	\caption{}
	\label{fig:1}
\end{figure}

在电路里,二极管就担负起“指挥交通”的任务,使原来应该一来一往流动的交流电,变成只来不往的单方向电流。

二极管质量的好坏可以用万用表的RX100档来测试。把正表棒(红色表棒)接二极管的负极,负表棒(黑色表棒)接二极管的正极,测得的“正向电阻”应该在500欧姆以下;反过来把正表棒接二极管的正极,负表棒接二极管的负极,测得的“反向电阻”应该在200千欧以上;这样的管子才是合用的。如果测得的正、反向电阻都很大或都很小,则说明管子质量很差,不能应用。应用万用表,还可以判别二极管的正负极:测得正向电阻的那一次,负表棒所接触的是二极管的正极,正表棒接触的是负极(图2-7)。

\begin{figure}[htbp]
	\centering
	\includegraphics[width=0.7\linewidth]{19}
	\caption{}
	\label{fig:1}
\end{figure}

如果没有万用表,应用电池和耳机,也可以大致判别二极管的好坏:先把二极管、电池和耳机像图2-8a那样连接起来,用耳机的一条引出线去刮二极管的负极,耳机里当能发出较响的嚓嚓声;再把二极管的正负极反过来,像图2-8b那样连接,用耳机的一条引出线去刮二极管的正极时,耳机里只能发出极轻的声音来。这样的二极管才是好的。如果两次试验声音都很响或都极轻,这个管子就不能用了。同样,这种方法也可以用来判别二极管的极性:在声较响的那次试验中,电池正极所接的正是二极管的正极。

\begin{figure}[htbp]
	\centering
	\includegraphics[width=0.7\linewidth]{20}
	\caption{}
	\label{fig:1}
\end{figure}

\chapter{矿石收音机}

矿石收音机是一种最简单,最济的收音机,需用器材不多,制作容易,不需要维持费用。这种简单而又经济的收音机,在装置上和检修上都不需要特殊的技术,最合于一般初学的无线电爱好者研究之用。

虽然矿石收音机也有它的缺点,如收程不远,声音小等,但是我国各省现在都已普遍地设立了人民广播电台,故在很大的地区内,矿石收音机仍可使用的。

\section{最简单收音机}

初次动手,先来试装一个十分简单的收音机。图2-1是它的电路图和实体接线图,它由天线、地线、耳机和晶体二极管组成。电路非常简单,一装就响。

\begin{figure}[htbp]
	\centering
	\includegraphics[width=0.7\linewidth]{13}
	\caption{}
	\label{fig:1}
\end{figure}

整块印制板(图1-2),划分为五区,这次试装在左下角的第2 区内进行。

(1)取长约1米的软接线一段,一端剥出2毫米左右的线头,焊接在电路板上,另一端剥去20毫米左右的线头与天线绞接(图 2-9a)。

(2)取长约1米的软接线,照图2-9b接入印制板,另一端与地线绞接。

(3)把耳机接入印制板(图 2-9c)。试听一下,耳机里寂然无声。

(4)把二极管的两条接脚用小刀刮清爽,搪好锡,根据印制板上的安装位置(图2-9d),弯折好接脚,焊入印制板。再来试听一下,哦!听到电台的播音了。

\begin{figure}[htbp]
	\centering
	\includegraphics[width=0.7\linewidth]{21}
	\caption{}
	\label{fig:1}
\end{figure}

\subsection{电路原理}

我们用的电灯是交流电。交流电在电路里是一来一去地流动着的,电流的方向来去变化一次所需的时间叫做周期T,周期的长短是用秒做单位来计算的。交流电(市电)的交变周期是1/50秒。交流电每秒钟来去方向交变的次数叫做频率f,用赫兹做单位来计算,交流电的频率是50赫兹,就是说电流来去方向的变化是每秒钟50次。

\begin{figure}[htbp]
	\centering
	\includegraphics[width=0.7\linewidth]{22}
	\caption{}
	\label{fig:1}
\end{figure}

利用话简,可以把声波转换成相应的交流电,这种交流电通过耳机时,就能发出原来那样的声音来。人们的耳朵只能听到每秒钟振动20次到2万次的声波,振动得太快或太慢都是听不见的。因此把频率为20赫兹到20千赫兹的交流电流称做音频电流。

音频电流只能沿着导线流动。打电话时,电话线一断,电话就不通了。

\begin{figure}[htbp]
	\centering
	\includegraphics[width=0.7\linewidth]{23}
	\caption{}
	\label{fig:1}
\end{figure}

频率在100千赫兹以上的电流,叫做高频电流。高频电流能使周围的空间里产生无线电波,向四面八方传播开去,所以高频信号可以用来进行无线电通信。

\begin{figure}[htbp]
	\centering
	\includegraphics[width=0.7\linewidth]{24}
	\caption{}
	\label{fig:1}
\end{figure}

在无线电广播电台里,是把音频信号加于高频信号上,使高频信号电流的峰值随着音频信号而作忽大忽小的变化——这种信号叫做高频调幅信号。用这样的办法,叫高频信号带着音频信号,发射成为无线电波,由近及远地传播开去。

\begin{figure}[htbp]
	\centering
	\includegraphics[width=0.7\linewidth]{25}
	\caption{}
	\label{fig:1}
\end{figure}

当电台发出的无线电波经过我们这里时,便会在我们架设的天、地线之间感应出一个高频信号电压来。在天、地线之间仅接有一个耳机时,由于信号的频率大大地高出音频范围,因此听不到声音。

在加接二极管后,高频电流以地线流向天线时,二极管正向导通,电流从二极管这一边流过;高频电流从天线流向地线时,二极管反向截止。因此,电流只能从耳机这一边的电路里流过。流过耳机这一边的是一个方向不变,强度随时间而变的脉动电流。这脉动电流里,就包含了音频信号的成分。当音信号电流流过耳机时,耳机里就发出声音来了。这里二极管起着从高频信号电流中分离出音频信号来的作用,这种作用称做“检波”。

\begin{figure}[htbp]
	\centering
	\includegraphics[width=0.7\linewidth]{26}
	\caption{}
	\label{fig:1}
\end{figure}

这个收音装置毕竟太简单了,只有一个检波器,没有选择电台的能力。在耳机里时常夹杂着几个电台的播音声,使我们喜爱的节目也听不清楚了。解决这个问题的方法是在检波电路前加装“调谐电路”。

\begin{figure}[htbp]
	\centering
	\includegraphics[width=0.7\linewidth]{27}
	\caption{}
	\label{fig:1}
\end{figure}

\section{无电源收音机}

在二极管检波电路前,加装一个由线圈和可变电容器构成的调谐电路,就成为一具可以选择电台的收音机了。这具收音机的特点是用不着装干电池,所以称做无电源收音机。

图3-1是本机电路图和实体接线图。

\begin{figure}[htbp]
	\centering
	\includegraphics[width=0.7\linewidth]{28}
	\caption{}
	\label{fig:1}
\end{figure}

\subsection{元件介绍与制作}

\subsubsection{线圈}

$L_{1}$和$L_{2}$是同绕在一条直径10毫米,长140毫米磁棒上的两个线圈。线圈的符号,形象地表示了它是用导线一圈一圈地绕制而成的。旁边的一段虚线,表示穿在线圈内的磁棒(或其他形状的磁芯)。

制作方法:先用牛皮纸或信封纸做两个内径11毫米、长约40毫米的纸管,套在磁棒上,要能随意前后移动。用φ0.07x7丝包线(是一种用直径0.07毫米的漆包线7股合成的丝包线),$L_{1}$绕60匝,$L_{2}$绕40匝。绕制时可按图3-2所示的方法,在线圈下预垫一条纱线(或纸条),用以固定线圈头尾两端,绕好后才不致松散开来。

\begin{figure}[htbp]
	\centering
	\includegraphics[width=0.7\linewidth]{29}
	\caption{}
	\label{fig:1}
\end{figure}

通过线圈的电流发生变化时,会在线圈里感应出一个电动势来阻碍电流的变化。表示线圈内电流的变化率为每秒1安培时,能感应出多少伏电动势来的量,叫做电感,单位亨利,简称亨(H)。无线电电路里所用的线圈,电感都较小,常用毫亨(mH)、微亨($\mu$H)做单位来计算:

1亨=1000毫亨,
1毫亨=1000微亨。

一般地说,线圈的圈数较多,电感较大,线圈里加入磁芯,可以大大地增强电感。套在磁棒上的线圈,移近磁棒的中点,电感增大,移向磁棒的一头,电感减小。$L_{1}$的电感约为300微亨,$L_{2}$约为130微亨。

\begin{figure}[htbp]
	\centering
	\includegraphics[width=0.7\linewidth]{30}
	\caption{}
	\label{fig:1}
\end{figure}

两个靠近的线圈,其中一个线圈内的电流发生变化时,会在另一个线圈里感应出电动势来。把这种作用称做互感。两个线圈间互感的大小,也用亨利做单位来计算。把线圈接在直流电路里,通过线圈的电流,方向和强度都不变,不会感应出电动势来。它的作用,等于用一条导线把电路接通,直流电流可以十分顺利地通过线圈。

把线圈接在交流电路里,情况就不同了。因为通过线圈的电流时刻在变化,所以会感应出一个和电源电压相反的电动势来阻碍电流的通过。把这种作用称做电感电抗(简称感抗)。交流电的频率越高,线圈的电感越大,所产生的感抗也越大,交流电流就越难于通过。

\begin{figure}[htbp]
	\centering
	\includegraphics[width=0.7\linewidth]{31}
	\caption{}
	\label{fig:1}
\end{figure}

\subsubsection{电容器}

电容器的种类很多,形状也是各样的,但它们的结构是一样的,都是由两片导电的极片当中夹一层介质(不导电的物质)构成的。

\begin{figure}[htbp]
	\centering
	\includegraphics[width=0.7\linewidth]{32}
	\caption{}
	\label{fig:1}
\end{figure}

顾名思义,电容器就是能贮藏电荷的容器。如果把电容器的两极片与直流电源的正负极相接,电容器里就充入了电荷。这时移去电源,电容器里的电荷也不会马上消失。如果把两极片短路一下,电容器里的电荷就会很快地释放掉。

\begin{figure}[htbp]
	\centering
	\includegraphics[width=0.7\linewidth]{33}
	\caption{}
	\label{fig:1}
\end{figure}

表示电容器两端加每伏电压能充入多少库仑电荷的量,叫做电容量。单位法拉,简称法(F),在无线电电路里,常用更小的单位微法($\mu$F),微微法(pF)来计算:

1法拉=1000000微法,
1微法=1000000微微法。

直流电流不能通过电容器,因为它的两极片间隔着一层介质。

交流电可以通过电容器。但电容器对交流电的通过有一定的阻碍作用——叫做电容电抗,简称容抗。对于同频率的信号来说,电容越大,容抗越小,越容易让信号电流通过;对于同样大小的电容来说,信号的频率越高,容抗越小,越容易让信号电流通过。

\begin{figure}[htbp]
	\centering
	\includegraphics[width=0.7\linewidth]{34}
	\caption{}
	\label{fig:1}
\end{figure}

在本机里$C_{0}$、$C_{2}$是两个瓷片电容。瓷片电容两极片间的介质是一种特殊的陶瓷。应用这种陶瓷作为介质,电容器的体积可以做得很小。$C_{0}$的电容量是20pF,$C_{2}$的电容量是0.01$\mu$F。这两个电容器的电容量是固定不变的。$C_{1}$是可变电容器,它有两组极片:一组叫定片,是固定不动的;一组叫动片,和转轴相连,转动轴柄,动片就连着转动。就用这样的方法来调节两组极片重合部分的面积(图3-7画有斜线部分),重合部分的面积越多,电容量就越大;重合部分的面积越少,电容量就越小。两组极片完全重合时电容量最大,为270pF。

\begin{figure}[htbp]
	\centering
	\includegraphics[width=0.7\linewidth]{35}
	\caption{}
	\label{fig:1}
\end{figure}

测量1$\mu$F以下小电容的电容量,要有专用的仪器。装置简易收音机用的电容,对电容量的正确性要求不高,只要极间不漏电、内部没有短路或开路就可用了。是否漏电或短路,可用万用电表的Rx1K档来测试,表针不动,说明没有漏电和短路。内部是否开路,可利用前次装置的检波电路来测试:把被测电容串联在天线和检波电路间,如照常能收到广播,即说明电容器内部没有开路(图 3-8)。

\begin{figure}[htbp]
	\centering
	\includegraphics[width=0.7\linewidth]{36}
	\caption{}
	\label{fig:1}
\end{figure}

\subsection{装置和调试}

在装置前先要把新加入的元件逐一测试过,确知没有损坏时才可加以应用。双联可变电容和磁棒都装置在印制板第1区内,检波电路仍装在第2区。焊接前先做好下列三件准备工作:

(1)把前次接到印制板上去的耳机和天线、地线的接线拆下来,二极管仍让它装在板上不动。

(2)把双联可变电容用螺丝固定在底板上。螺丝长度应在5毫米以下,螺丝太长,旋进去时会把电容器的极片顶坏。三条双联极片的接脚穿过底板上的小孔,等待焊接。

(3)把塑料磁棒架装在底板上,插好磁棒。把两个线圈的线头都搪好锡,然后套在磁棒上。

\begin{figure}[htbp]
	\centering
	\includegraphics[width=0.7\linewidth]{37}
	\caption{}
	\label{fig:1}
\end{figure}

焊接分下面三个步骤进行(图3-10和图3-11):

\begin{figure}[htbp]
	\centering
	\includegraphics[width=0.7\linewidth]{38}
	\caption{}
	\label{fig:1}
\end{figure}

(1)焊接调谐回路

把从小孔中穿过来的三条双联可变电容器的接脚弯过来,紧贴铜箔表面,并用焊锡焊牢;再把$L_{1}$的两端分别与双联可变电容器左面的接脚($C_{1}$的定片)及中间的接脚($C_{1}$的动片)相焊接。双联中右边的接脚(即还有一联可变电容器的定片)让它空着,留待后用。

(2)焊接检波电路

把$L_{2}$的一端接二极管正极,另一端就近接“地”。然后再接入一个0.01$\mu$F的电容$C_{2}$和耳机EJ。

(3)接天线、地线

把一只22pF的电容$C_{0}$接于$C_{1}$定片和接天线的焊点间。取一段长约1米的导线,一端与$C_{0}$相接,另一端与天线绞接;再取一段长约1米的导线,一端焊接在$C_{1}$动片接点附近,另一端与地线绞接(图3-11)。

\begin{figure}[htbp]
	\centering
	\includegraphics[width=0.7\linewidth]{39}
	\caption{}
	\label{fig:1}
\end{figure}

至此,焊接工作已全部完成。旋动可变电容器的轴柄,就能选择收听的电台。如果声音较响但夹有别的电台的播音声,可把$L_{1}$和$L_{2}$之间的距离拉开一点试试。如果声音太轻可把$L_{1}$和$L_{2}$之间的距离靠近一点试试。

\begin{figure}[htbp]
	\centering
	\includegraphics[width=0.7\linewidth]{40}
	\caption{}
	\label{fig:1}
\end{figure}

\subsection{电路原理}

(1)调谐电路

$L_{1}$和$C_{1}$组成的回路,叫做调谐电路,也叫选择器。

选择器怎样选择电台呢?

在我们周围有许多电台在发射不同频率的电磁波,这些电磁波都会在收音机的天线上感应出各种不同频率的信号电流。只有当L、C组成的调谐回路的振荡频率与外来的信号频率一致时(这种情况叫做谐振),这个频率的电台信号才能优先通过,并得到增强后检波,还原成声音。其他不是这个频率的无线电波就被禁止通行。只要改变调谐回路的频率,就可以分别和各种电台的频率信号一一谐振,这就起到了选择电台的作用。

在$L_{1}$、$C_{1}$的回路里,谐振频率与电感、电容的大小有关:电感、电容越大,谐振频率越低;电感、电容越小,谐振频率越高。$L_{1}$向磁棒中心移,电感增大,回路的谐振频率减低;$L_{1}$向磁棒的一头移,电感减小,回路的谐振频率升高(图3-13)。$C_{1}$的轴柄顺时针方向旋转时,电容增大,回路的谐振频率减低;$C_{1}$的轴柄逆时针方向旋转时,电容减小,回路的谐振频率升高(图3-14)。和回路谐振的信号得到了加强,不谐振的信号遭到了削弱。因此在收音机里加入了调谐电路,不但可以用来选择电台,还起着提高灵敏度的作用。

\begin{figure}[htbp]
	\centering
	\includegraphics[width=0.7\linewidth]{41}
	\caption{}
	\label{fig:1}
\end{figure}

\begin{figure}[htbp]
	\centering
	\includegraphics[width=0.7\linewidth]{42}
	\caption{}
	\label{fig:1}
\end{figure}

(2)整机工作过程

天、地线接收到各电台信号,经电容$C_{0}$传到由$L_{1}$、$C_{1}$构成的调谐电路。天、地线之间也存在着一定大小的电容,如果把天、地线直接和调谐电路相接,就会影响调谐电路的谐振频率,所以这里用一个较小的电容$C_{0}$,把天、地线之间的电容与调谐电路隔开。在调谐电路里,只有被选中的那个电台信号与

电路谐振,能产生较强的电流。那些不谐振的电台信号,被衰减到微不足道的程度。由于$L_{1}$和$L_{2}$之间有互感作用,被选中的电台信号就传递到了$L_{2}$。$L_{2}$输出的高频信号电压,通过电容$C_{2}$加于二极管D的两端,经过二极管检波,得到一音频信号电流。$C_{2}$是一个0.01$\mu$F的小电容,对音频电流的容抗很大,所以音频电流只能从耳机EJ里通过,这时耳机里就发出电台的播音声来了(图3-15)。

\begin{figure}[htbp]
	\centering
	\includegraphics[width=0.7\linewidth]{43}
	\caption{}
	\label{fig:1}
\end{figure}

\chapter{单管收音机}

在无电源收音机的检波器后面,加装一级音频放大器,把输出功率放大,耳机里发出来的声音就响得多了。在离电台十公里以内的地区,用5-7米长的天线来收听,声音竟有些震耳呢!根据实际的需要,在这具单管机里也装上了控制音量的电位器。

图4-1是本机的电路图和实体接线图。

\begin{figure}[htbp]
	\centering
	\includegraphics[width=0.7\linewidth]{44}
	\caption{}
	\label{fig:1}
\end{figure}

\begin{figure}[htbp]
	\centering
	\includegraphics[width=0.7\linewidth]{45}
	\caption{}
	\label{fig:1}
\end{figure}

\section{元件介绍}

(1)$R_{1}$是5.1千欧的碳膜电阻。

(2)$R_{2}$也是一个碳膜电阻,它的阻值约100千欧,要由实验来决定。

$R_{1}$、$R_{2}$电阻的阻值是否正确,可用万用表来测试(图4-2):先把万用表的分选开关拨到适当的倍率档(例如测几千欧的电阻时应该拨到Rx1K档);然后把两条表棒互相紧密接触,转动零位调节旋钮,使表针指着0欧姆处;再把两根表棒分别与电阻两端接触,从$\Omega$栏刻度中读出数值,再乘以倍率数。

\begin{figure}[htbp]
	\centering
	\includegraphics[width=0.7\linewidth]{46}
	\caption{}
	\label{fig:1}
\end{figure}

(3)W 是4.7千欧的连开关电位器。在圆形的小壳内,装着一片弧形的碳膜电阻和一个滑臂,滑臂和转轴相连,旋动转轴可使滑臂在碳膜片上移动。图4-3中a、b是弧形电阻片两端的引出接脚,c是滑臂的接脚。一个简单的开关就装在壳子的下面。电位器的好坏要分两次测试:第一次测a-c间的电阻,要求转动滑臂时,万用表指针始终缓缓偏转,不会产生突然的跳动。第二次测试c-b间的电阻,要求也是这样。

\begin{figure}[htbp]
	\centering
	\includegraphics[width=0.7\linewidth]{47}
	\caption{}
	\label{fig:1}
\end{figure}

(4)$C_{3}$是33$\mu$F的电解电容器(图4-4)。电解电容器的两极片,是有正、负极性之分的。正极片的引出线长一些,在外壳上近正极处有一个“+”的记号;负极片引出线比较短一些。在线路图中,正极片的符号是一个矩形方框,负极片仍是一段直线。接到电路里去时,正负极性不可搞错,搞错了电容器容易击穿。

\begin{figure}[htbp]
	\centering
	\includegraphics[width=0.7\linewidth]{48}
	\caption{}
	\label{fig:1}
\end{figure}

测试这种电容器好坏的方法如下:测试前先把电容器的两条引出线短接一下,释放掉里面的残余电荷,然后把万用表(拨到Rx1K档)的黑表棒接电容器的正极片,红表棒接电容器的负极片(手指不要和表棒、电容器引出线接触),表针应当很快地挥向右面,然后缓缓退回无穷大处($\infty$处)。

测试时,表针向右挥动的角度越大,表示电容量越大。大约:30$\mu$F的电容可挥到7千欧处。100$\mu$F的电容可挥到2千欧处。如果不会挥动,则表示这个电容内部开路,或已经失效了。挥动的角度不够大,则说明电容量不足。如果表针退不到原处,说明电容器漏电。我们从表针停留的位置上可读出它的“漏电电阻”,漏电电阻大于200千欧的电容器,在简易收音机里还是可以应用的。

(5)$BG_{1}$是3AX31型低频三极管。三极管有三个电极:发射极e,基极b和集电极c(图4-5)。发射极的符号是一个箭头,它表明了电流的方向。电流从发射极流进三极管,这一路电流称做发射极电流$I_{e}$;发射极电流在三极管内分成两股:一股经基极流出三极管,叫做基极电流$I_{b}$;另一股经集电极流出三极管,叫做集电极电流$I_{c}$。

\begin{figure}[htbp]
	\centering
	\includegraphics[width=0.7\linewidth]{49}
	\caption{}
	\label{fig:1}
\end{figure}

$I_{b}$和$I_{c}$成一定的比例,一般$I_{c}$是$I_{b}$的20倍到200倍,倍数的大小,随管子而异。这个倍数叫做“电流放大系数”,用一个希腊字母$\beta$来表示。例如有一个3AX31型三极管,它的$\beta$是50,在$I_{b}$是0.1毫安时,$I_{c}$就是5毫安;$I_{b}$是0.2毫安时,$I_{c}$就是10毫安。但是$I_{b}$等于零时,$I_{c}$却还存在一点点电流,这是从发集极穿过基极“漏”到集电极去的,叫做“穿透电流”$I_{ceo}$(图 4-6)。

\begin{figure}[htbp]
	\centering
	\includegraphics[width=0.7\linewidth]{50}
	\caption{}
	\label{fig:1}
\end{figure}

图4-7说明了应用万用表检查3AX31,3AG1等三极管好坏的方法:把分选开关拨到Rx100的位置,进行如下的测试:

①把万用表的红表棒接三极管的b极,用黑棒去接e极和c极,测得的电阻都应该在100到500欧之间;

②把万用表的黑表棒去接三极管的b极,用红表棒去接c极和e极,测得的电阻都应该在200千欧以上。

测试下来符合上述情况的,表示管子没有损坏。如果c极或e极到b极的电阻,两次测下来,都是零欧姆或无穷大(表针不动),就说明这个管子已经损坏了。

\begin{figure}[htbp]
	\centering
	\includegraphics[width=0.7\linewidth]{51}
	\caption{}
	\label{fig:1}
\end{figure}

利用万用表10mA档作指示器,再加三个电阻,便可制成一个测试三极管$\beta$和$I_{ceo}$近似值的简单仪器(图 4-8)。其中,按钮开关AN可用小片铜皮自制,电池E可利用准备好的收音机电源。把被测的三极管插入电路(e、b、c三个电极的位置不能插错),万用表指示的就是这个管子的穿透电流,可以从0~100的这栏刻度上读出,刻度100处是10毫安。穿透电流要求小于1毫安。按下按钮开关AN,输入约0.04毫安的基极电流,集电极就相应增大,从表面上0~25这栏刻度上可读得$\beta$的数值,刻度25处表示$\beta$为250。$\beta$自50至150之间的管子都可以用,但以选70~80之间的最适当。在测试过程中,如万用表指针一直在漂移,这种管子性能很不稳定,只能剔去不用。

\begin{figure}[htbp]
	\centering
	\includegraphics[width=0.7\linewidth]{52}
	\caption{}
	\label{fig:1}
\end{figure}

\section{装置和调试}

有关元件都装置在印制板第2区和第3区里,可按下面的步骤进行:

(1)把原来装在印制板上的耳机拆下。

(2)安装音量控制电位器W:

电位器安装在印制板左下角无铜箔的那一面。印制板装电位器的地方有七个小孔(图4-9),把a、b、c三个小孔对准电位器的三条焊片,K、K'两个小孔对准开关的两条焊片,那么其余两个小孔正好对准电位器底部的两个螺丝孔。用两枚小螺丝插入这两个小孔,旋进电位器底部的螺丝孔内,把电位器紧固在印制板上。然后用裸导线把电位器上K、K'和b三条焊片和背面的铜接通。

\begin{figure}[htbp]
	\centering
	\includegraphics[width=0.7\linewidth]{53}
	\caption{}
	\label{fig:1}
\end{figure}

另两条焊片,则根据图4-11,用单芯塑料绝缘导线连接:一条自焊片c穿过小孔接至$C_{3}$的负极,另一条自焊片a穿过小孔接至$BG_{1}$的基极。

(3)根据图4-10的印制板元件图,把$R_{1}$、$C_{3}$、$BG_{1}$等三个元件插入印制板,一一焊牢。要注意,$C_{3}$的正负极接脚和$BG_{1}$的c、b、e三条接脚的位置不可插错,如果在$BG_{1}$的三条接脚上各套一个不同颜色的塑套管,就便于识别,不致搞错了。$R_{2}$和耳机暂缓接入。

\begin{figure}[htbp]
	\centering
	\includegraphics[width=0.7\linewidth]{54}
	\caption{}
	\label{fig:1}
\end{figure}

(4)连接电源:把四节干电池串联起来(参看图4-1),正极接开关K点,负极接第3区上方的角尺形长条铜。

(5)调整$BG_{1}$的偏置电流:把万用表拨到10mA档,接入耳机的位置,红表棒接$BG_{1}$的c极,黑表棒接电源负极;把一个220千欧的电位器$W_{0}$和一个22千欧的电阻$R_{0}$串联起来,接在$R_{2}$的位置上(图4-11)。然后接通电源,调节220千欧的电位器,使万用表的读数为1毫安左右。调好后,切断电源,拆下$W_{0}$、$R_{0}$的串联电路和万用表,用万用表测出这段串联电路的阻值,再找一个与这阻值最接近的电阻器接入$R_{2}$的位置,$BG_{1}$的偏置电流就调好了。注意调整偏置电流时,音量控制电位器W应该逆时针方向旋到音量最小的位置。

\begin{figure}[htbp]
	\centering
	\includegraphics[width=0.7\linewidth]{55}
	\caption{}
	\label{fig:1}
\end{figure}

(6)试调音量控制电路。接入耳机,开启电源,顺势把音量控制电位器W顺时针方向旋到底——开到音量最大位置。再旋动$C_{2a}$,选择电台收听,就可以感到声音比原来要大二三倍。在近电台地区收听,一定会嫌声音太大。可把音量电位器朝逆时针方向旋转,耳机里发出来的声音就会跟着渐渐轻下去。逆时针方向旋到底而电源开关还未关断时,耳机里应该是完全听不到播音声了(图4-12)。如果W逆时针方向旋到底,而耳机里还有声音,那可能是W的接地端没有焊好或在它内部b端已经开路的缘故。

\begin{figure}[htbp]
	\centering
	\includegraphics[width=0.7\linewidth]{56}
	\caption{}
	\label{fig:1}
\end{figure}

\section{电路原理}

检波电流中的音频成分流过$R_{1}$时,在$R_{1}$的两端形成一个音频信号电压,通过$C_{3}$接到$BG_{1}$的基极。$C_{3}$是一个30$\mu$F的大电容,对音频信号的阻抗很小,它在电路里起着隔断直流电,传递音频信号的作用。

在没有音频信号送来时,$BG_{1}$的基极电流$I_{b}$是一股直流电流——基极偏置电流,它的集电极电流$I_{c}$当然也是一股直流电流——集电极偏置电流,不过$I_{c}$比$I_{b}$要大得多,是$I_{b}$的$\beta$倍。

当有音频信号送来时,$BG_{1}$的$I_{b}$随着音频信号电压产生忽强忽弱的变化。它的$I_{c}$也随着作忽强忽弱的变化,不过$I_{c}$的变化要比$I_{b}$的变化大得多——是$I_{b}$变化的$\beta$倍(图4-13)。这股变化较大的电流通过耳机时,耳机里就发出比以前响得多的声音来了。

\begin{figure}[htbp]
	\centering
	\includegraphics[width=0.7\linewidth]{57}
	\caption{}
	\label{fig:1}
\end{figure}

\chapter{二管收音机(高放式)}

在单管机的检波器前,再插入一级高频放大器,就成为一具高放式二管收音机。在距离电台几十公里以内的地区,不接天地线就可以收音了。如果加接天线,那就可以收到更远的电台。

图5-1是本机的电路图和实体接线图。粗线部分是新加入的高放部分的线路。

\begin{figure}[htbp]
	\centering
	\includegraphics[width=0.7\linewidth]{58}
	\caption{}
	\label{fig:1}
\end{figure}

\begin{figure}[htbp]
	\centering
	\includegraphics[width=0.7\linewidth]{59}
	\caption{}
	\label{fig:1}
\end{figure}

\section{元件介绍}

(1)电阻器 $R_{3}$是5.1千欧、1/8瓦的碳膜电阻,$R_{4}$是100千欧左右、1/8瓦碳膜电阻,它的阻值与$BG_{2}$的电流放大系数有关,在调试时决定。$R_{5}$是1千欧、1/8瓦的碳膜电阻。

(2)电容器$C_{4}$是0.022微法的瓷片电容,$C_{5}$是0.047微法的瓷片电容。

(3)高频变压器 $B_{0}$是自制的高频变压器(图5-2)。在直径3毫米左右的中频磁芯上,初级用直径0.06毫米的漆包线绕200匝,次级用同号线绕100匝。如果自制有困难,用TTF-2-9型中频变压器来代替,效果也不差。

\begin{figure}[htbp]
	\centering
	\includegraphics[width=0.7\linewidth]{60}
	\caption{}
	\label{fig:1}
\end{figure}

(4)高频三极管 $BG_{2}$是3AG1型高频三极管。它的符号和外形,和3AX31没有什么两样。虽然它们都有把信号放大的能力,但各有专长:3AG1善于放大高频信号,但允许通过的电流只有10毫安,通过的电流太大时,管子就会发热烧毁。3AX31不善于放大高频信号,但允许通过100多毫安的电流,可用来放大较强的信号(图5-3)。

\begin{figure}[htbp]
	\centering
	\includegraphics[width=0.7\linewidth]{61}
	\caption{}
	\label{fig:1}
\end{figure}

这个管子的$\beta$,自50--100的都可以应用,穿透电流选0.5毫安以下的。

\section{装置和调试}

高频放大器装置在印制板的第1区内,如图5-4所示。

\begin{figure}[htbp]
	\centering
	\includegraphics[width=0.7\linewidth]{62}
	\caption{}
	\label{fig:1}
\end{figure}

按下述步骤进行装置、调试:

(1)把$L_{2}$从磁棒上取下,拆剩6圈,照原样装在磁棒上,两个线头照图5-4 所示,改接到$BG_{2}$基极及$R_{3}$、$R_{4}$的连接点上去。

(2)把高频变压器的四个线头用小刀刮清爽,搪好锡,然后把它的磁芯插入印制板第1区下排的直径为3毫米的安装孔内,用胶水胶牢。初级线圈的两端分别接$BG_{2}$集电极及电源负极(角尺形长条),次级线圈始端接第2区内二极管正极,终接“地”。

(3)把$BG_{2}$、$R_{3}$、$R_{5}$、$C_{4}$、$C_{5}$插入印制板,一一焊好。$R_{4}$暂缓接入。

(4)调整$BG_{2}$的偏置电流:因为发射极电流约等于集电极电流,所以只要测出发射极电阻$R_{5}$两端的电压,就可以推算出集电极电流来。$R_{5}$的阻值是1千欧,通过1毫安电流时,两端的电压正好是1伏。把万用表拨到直流2.5伏档,并联在$R_{5}$两端,红表棒接“地”,黑表棒接$BG_{2}$的发射极。取一个22千欧的电阻和一个220千欧的电位器串联起来,接在$R_{4}$的位置上(图5-5)。开启电源,调节220千欧的电位器到万用表指示1伏处。再试听耳机里的播音声,并稍稍旋动这个220千欧的电位器,使播音声最响亮清晰为止。如果可变电容$C_{1}$旋到某些位置,耳机会产生啸叫,那末应该把$R_{5}$两端的电压再调小些,即把偏流减弱些。试听下来比较满意后,关掉电源,拆下临时接到印制板上去的电阻和电位器,测出这段串联电路的阻值,选一个与这阻值接近的碳膜电阻接入$R_{4}$的位置。

\begin{figure}[htbp]
	\centering
	\includegraphics[width=0.7\linewidth]{63}
	\caption{}
	\label{fig:1}
\end{figure}

(5)试听:离电台不太远的地方,已不必接天、地线,只用磁性天线(磁棒能聚集在它周围的无线电波,绕在磁棒上的线圈可代替天线来接收无线电信号,所以称做磁性天线),就可以收到电台的播音。不过磁性天线有很强的方向性,磁棒应水平放置,并取与电台方向成正交的位置,这时收到的信号最强;垂直放置或取与电台方向一致的位置时收到的信号最弱,甚至收不到信号(图5-6)。利用磁性天线的方向性,可以把来自不同方向的、频率十分接近的电台信号分隔开来。

\begin{figure}[htbp]
	\centering
	\includegraphics[width=0.7\linewidth]{64}
	\caption{}
	\label{fig:1}
\end{figure}

接着,应重新调节$L_{1}$与$L_{2}$之间的距离,使收音机既有较高的灵敏度,又有较好的选择性。调好后,可滴少许熔蜡于纸管和磁棒间,以固定线圈的位置。

\section{电路原理}

BG:
偏置电路偏置电路是直流电路。线圈极容易让直流电通过,工和B。的初级线圈可以认为是短路的;电容器不能让直流电通过,C和C;可以认为是开路的。所以和BG,的偏置有关的,实际上只有R、R、R;三个电阻(图5-7)。R把BG:的基极和电源负极连接起来,构成基极电流的通路,同时也限制了基极电流的大小,R越大基极电流就越小。R:和BG,的基极相并联,加接了这个电阻可以使偏置电流稳定些。为了进一步稳定偏置电流,还给BG:加接一个发射极电阻R,使在电源电压或环境温度稍有变化时,BG,的偏置电流基本上稳定不变(2)高频信号放大过程对于高频信号来说,C和C,的容抗很小,不妨认为是直通的。C.并联在Ra两端,等于把R;短路了。C;并联在 R;两端,等于把 Rs短路了。L₂输WAAAV出的高频信号,直接输入BG:的基极,通过放大,在BG。集电极电路里输出一较强的高频信号,经高频变压器B送到检波器去(图5-
二0:R.
图5-8

在高频放大器里,有两处电阻和电容并联的地方:R和C,并联,Rs和C,并联。把图5-7和图5-8 比较一下,可知电阻是直流电的通路,直流电通过电阻时,在电阻两端产生一个电压降。和电阻并联的电容(称做旁路电容)是专为交流信号设置的一条捷径。旁路电容的电容量必须足够大,使它对交流信号的阻抗几乎等于零。那么交流信号可不经电阻,很方便地从这条捷径通过了。如果不设这两个旁路电容,输入信号通过R、R时,将遭到损失。输出信号在R上产生电压降,将会使放大器对信号的放大倍数大打折扣。读者如有兴趣,可拆去这两个电容试听,一定会感到耳机里的声音显著地减轻了。

==============================================================================每一个无电爱好者的研究工作,差不多都是从石收机开始的,本母就是为了具体帮助初学者去制造矿石收音机而寫。里面說明了接收原理;矿石收音机主要零件的 機造 和 性能;收晋机的制作和稚修。采用的零件都是容易贸到或可以自制的,读者只要依酰明安装;就是从來没有学督过无线电的人也能成功。
这是一个无线电爱好者从事制作的开端,在装管碳石收机獲得煦之后,就可以在这个基碰上,选一步去装电学管收管机了。
如果讀者还想在理論方面和制作方面作深一步的研,下面这几本者是比较适合的:
1.'初等电工学
苏联 ·耶列柏卓夫著



\backmatter

\chapter{示例电路板}

用一块75x140平方毫米的单面铜箔板,采用刀刻法来制作:先把透明薄纸覆在图1-1上,用铅笔照样描下来,再用复写纸复印在铜箔板上,然后用斜口小刀刻去线条上的铜箔。刻的时候右手拿刀,刀口向前,刀尖紧抵铜箔,左手拇指顶住刀背略往前推,右手却稍往后拉,利用杠杆作用,使刀口向前刻划。

\begin{figure}[htbp]
	\centering
	\includegraphics[width=0.7\linewidth]{pcb1}
	\caption{}
	\label{fig:1}
\end{figure}

刻出来的线条,宽约3毫米,粗细要均匀。可在线条的一边先刻一道划痕,再在另一边刻一道划痕,并把线条两端刻断,然后用刀尖在一端挑起边角,就可把应刻去的铜箔撕下来(图1-3)。如果第一遍刻的划痕太浅,可连刻二遍、三遍。

\begin{figure}[htbp]
	\centering
	\includegraphics[width=0.7\linewidth]{fl_1}
	\caption{}
	\label{fig:1}
\end{figure}

钻孔前,先要用冲头(或钢针)在孔的中心凿一个凹痕,这样钻孔时钻头才不会滑动,装插一般元件接脚的孔径是1.2毫米,装输入、输出变压器的孔径是1.5毫米左右,装螺丝的孔径是3毫米。如果没有适当大小的钻头,可先钻一个小孔,用斜口小刀把它适当扩大就行。装双连可变电容器的大孔,孔径是10毫米,还须用尖头木锉或圆锉来进一步加工(图1-4)。

\begin{figure}[htbp]
	\centering
	\includegraphics[width=0.7\linewidth]{fl_2}
	\caption{}
	\label{fig:1}
\end{figure}

钻好孔后,用细砂皮轻轻打磨铜箔,除去污物和氧化层,使表面光洁明亮,然后在铜箔面均匀地涂刷一层松香溶液(图1)。

\begin{figure}[htbp]
	\centering
	\includegraphics[width=0.7\linewidth]{fl_3}
	\caption{}
	\label{fig:1}
\end{figure}

松香溶液配制的方法是:在墨水瓶里盛半瓶95\%的酒精,放入六、七颗蚕豆大小的松香,用筷子搅拌,使它溶解。这种松香溶液涂在铜箔上,其中的酒精很快地蒸发掉,松香在铜表面形成一层薄膜,它能保护铜箔表面,防止氧化。在焊接时,松香还起着助焊的作用,使铜箔容易上锡。余下的松香溶液,留着在焊接时作助焊剂。

\begin{figure}[htbp]
	\centering
	\includegraphics[width=0.7\linewidth]{dl_1}
	\caption{}
	\label{fig:1}
\end{figure}

\chapter{怎样焊接}

对于初学者来说,焊接工作的好坏,是装置收音机成败的关键。装置晶体管收音机,应该用45瓦以下小功率的铁。烙铁的功率过大或焊接的时间过长,都会使元件因受热而变值、损坏。一般用20瓦内热式电烙铁比较合适。它的构造如图1-6,烙铁芯装在金属管里,它是一条有夹层的瓷管,夹层里绕着电热丝通电后就会发热。烙铁头用纯铜制成,套在金属管外面。小功率的电烙铁,像图1-6那样拿,焊接起来最方便。焊接时,烙铁头的温度过低,焊锡熔不开、焊不牢;温度太高,烙铁头又常常被烧死(即表层氧化发黑,吃不上锡)。烙铁头被烧死时,可用锉刀把它表面的氧化层锉掉,立刻蘸上松香、焊锡,便可继续使用。

\begin{figure}[htbp]
	\centering
	\includegraphics[width=0.7\linewidth]{fl_4}
	\caption{}
	\label{fig:1}
\end{figure}

焊接前,先要在元件接脚或接线的线头上搪一层锡。方法,用斜口小刀把元件接脚(或线头)表面充分刮消爽,涂上一点松香溶液,用尖头钳夹持元件接脚的根部,烙铁头上饱蘸熔锡,接触元件接脚,自左向右,搪上一层焊锡。用尖头钳夹持接脚根部,可把热量引走,以免传入元件内部,使元件因过热而损坏(图1-7)。

\begin{figure}[htbp]
	\centering
	\includegraphics[width=0.7\linewidth]{fl_5}
	\caption{}
	\label{fig:1}
\end{figure}

接线的线头也要预先搪锡。加接在印制板上的接线,应该用质地较硬的单芯塑料绝缘接线,装好后比较挺括牢靠。从线路板引出的接线(如与电源、扬声器连接的导线)应该用多股线芯、质地柔软的塑料绝缘接线,才经得起多次弯折。搪锡前先要剥出2毫米左右的接头。剥线头的方法是:利用热的烙铁头把塑料层熔断,趁热用手把割断的那段塑料皮拉去(图1-8)。刚去皮的新接线很容易上锡,不须刮削,只要涂上一点助焊剂,就可搪锡。

\begin{figure}[htbp]
	\centering
	\includegraphics[width=0.7\linewidth]{fl_6}
	\caption{}
	\label{fig:1}
\end{figure}

元件接脚搪好锡后,根据印制板来弯折接脚,使几条接脚刚好能插入印制板相应的位置。弯脚时要留意:使元件插入线路板后,标有数值的一面朝上,以便于装机时核对。

\begin{figure}[htbp]
	\centering
	\includegraphics[width=0.7\linewidth]{fl_7}
	\caption{}
	\label{fig:1}
\end{figure}

元件接脚从线路板无铜箔的一面(简称A面)插入,穿出有铜箔的一面(简称B面),留下1.5毫米左右的接脚,把多余部分剪掉。然后用烙铁饱蘸焊锡,进行焊接。焊接时烙铁头应和焊点紧密接触,使熔锡充分浸润接脚和铜箔,并且吃牢在接脚和铜箔上。提起烙铁时,用手指弹击一下线路板,可使焊点圆浑不起毛刺。焊接的时间要掌握在2秒钟以内,时间过长,会使元件过热而损坏。焊接的过程如图1-9所示。焊点要求圆浑光亮不起毛刺,焊锡和接脚、铜箔要吃得很牢。用锡多少适宜,焊点不过大或过小(图 1-10)。

\begin{figure}[htbp]
	\centering
	\includegraphics[width=0.7\linewidth]{fl_8}
	\caption{}
	\label{fig:1}
\end{figure}

焊接得不好时,元件的接脚只是被焊锡含住,或锡粒只是靠松香粘结在铜箔上,没有真正的焊牢,这样的焊点,称做“假焊”。假焊的焊点往往有很大的电阻,有的时通时断,有的索性不通,所以会使电路出现许多莫名其妙的故障,个别接点的假焊,还有导致元件损坏的危险(如三极管的下偏流电阻)。

\begin{figure}[htbp]
	\centering
	\includegraphics[width=0.7\linewidth]{fl_9}
	\caption{}
	\label{fig:1}
\end{figure}

造成假焊的原因常有下列几个方面:

一是元件的接脚或线头没有刮清爽,连搪锡也没有搪好。

二是印制板铜箔表面染有污物或搁置日久表面氧化。

三是烙铁头温度太高或太低。

四是熔锡氧化已不适宜用来焊接,刚蘸上烙铁头的熔锡,表面光亮流动性很好,及时用来焊接,就容易和接脚、铜吃牢。如搁置时间过长,熔锡氧化,表面灰暗粗糙,失去了流动性,这样的熔锡就不能再和接脚、铜箔的表面结合,就会造成假焊。如果熔锡已经氧化,只要再蘸上点松香,就能使它还原,仍旧可以用来焊接。

五是烙铁头表面氧化或沾有污物,影响焊接。

六是烙铁头在焊点上停留的时间太短,焊锡还来不及和接脚、铜箔吃牢。

\chapter{参考书籍}

看图学装收音机  朱蔼初  少年儿童出版社 1984年,上海

矿石收音机 冯报本

无电源收音机	陈鹏飞	黑龙江科学技术出版社 1985	15217.170

\end{document}