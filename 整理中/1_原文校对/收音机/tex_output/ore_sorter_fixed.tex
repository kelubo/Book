\documentclass[12pt,a4paper]{article}
\usepackage[UTF8]{ctex}
\usepackage{graphicx}
\usepackage{geometry}
\graphicspath{{images/}}

% 设置页面边距
\geometry{left=2.5cm, right=2.5cm, top=2.5cm, bottom=2.5cm}

\begin{document}
	
	\title{经典矿石机鉴赏与现代矿石机制作}
	\author{徐蜀}
	\date{2016年}
	\maketitle
\section{文件 1}

%未知 目录 封面 扉页 版权 编委会 前言 第一章 国外矿石机的历史回顾 一、早期的线圈调感式矿石机 抽头式线圈调感矿石机 1.法国壁挂式矿石机 2.加拿大“普通人”矿石机 3.美国“宝宝”袖珍矿石机 4.德国WISI.NR.56袖珍矿石机 5.德国具有减震功能的袖珍矿石机 6.法国迷宫式线圈袖珍矿石机 线圈滑动抽头调感式矿石机 1.法国早期滑动抽头线圈矿石机3种 2.法国早期格勒诺布尔矿石机 3.法国便携式邮政矿石机 4.美国JR矿石机 5.德律风根ZA矿石机 6.德律风根RDE矿石机 7.美国BIG 4矿石机 8.美国火星矿石机 9.美国“小宝石”矿石机 10.美国“瓢虫”陶瓷矿石机 11.英国“汤姆叔叔”陶瓷矿石机 12.法国的“美洲豹”“热带鱼”和“羚羊”陶瓷矿石机 13.美国“啤酒瓶”矿石机 14.英国书本式矿石机 15.美国Metro.jr矿石机 16.美国PACENT袖珍矿石机 17.德国德律风根板式矿石机 线圈耦合调感式矿石机 1.早期线圈调感器4种 2.德国德律风根公司E04矿石机 3.美国Westinghouse矿石机 4.美国ER-753A便携式矿石机 5.美国Atwater Kent板式矿石机 6.英国BTH-C双矿石检波器矿石机 7.瑞典爱立信L.M矿石机 8.德国罗兰士Lorenz EDAT25型双矿石检波矿石机 9.德国“狐狸”矿石机 10.英国The Electron 双线圈调感矿石机 11.美国纽约丹齐格琼斯公司出品的线圈耦合调感式三检波器矿石机 12.德国Dyhr耳机矿石机 二、早期的线圈与可变电容器耦合调谐矿石机 1.美国RCA AR1300矿石机 2.美国Betts\&Betts板式矿石机 3.美国吉卜林豪华版矿石机 4.美国直立板式矿石机 5.法国三线圈矿石机 6.法国双回路调谐矿石机 7.德国1001矿石机 8.美国劳伦斯矿石机 9.英国马可尼矿石机 10.奥地利GEWES矿石机 三、另类矿石机(附两款火花无线电发射机) 1.法国1913年出品的E·POFERL矿石机 2.美国1917年出品的无线电火花发射机 3.美国BC-14A型军用矿石接收机 4.美国BC-15A军用火花发射机 5.英国MARK Ⅲ型军用矿石接收机 6.英国B.T.H.一灯矿石收音机(Valve-Crystal Receiver) 7.美国DeForest D-10矿石检波四管直流接收机 四、20世纪40年代后的传统矿石收音机 1.德国1946年出品的德律风根袖珍矿石收音机 2.德国1946年出品的蓝宝矿石收音机 3.20世纪40年代美国出品的铁壳袖珍自带扬声器矿石机 4.德国20世纪50年代初出品的欧米茄袖珍矿石收音机 5.德国20世纪50年代出品的德律风根D3型矿石收音机 6.20世纪50年代后期Miniman袖珍矿石机 7.1958年美国Hearever公司出品的Hearever火箭模型矿石机 8.20世纪50年代后期日本出品 的EM-TONE型矿石机 9.20世纪50年代后期日本出品的LARK CRYSTAL 矿石机 10.20世纪50年代初美国出品的 Philmore VC-1000矿石机套件 11.1958年美国希斯有限公司出品的Crystal Receiver CR-1型矿石机套件 五、前苏联“共青团员”牌矿石机 第二章 国产矿石机的历史回顾 一、民国时期的矿石收音机 1.一台我国20世纪30年代的矿石收音机 张明律收藏并撰文 2.一台天津无线电行组装的矿石机 3.一台抗战时期八路军缴获的“华北标准型矿石受信机” 4.一台标明了型号、厂名、出厂日期和制作者的“日本”矿石机 二、新中国成立后的矿石收音机 1.20世纪50年代初天津天华工业社出品的矿石收音机 2.“牡丹”矿石机 3.与“牡丹”矿石机近似的“上海造” 4.提手上铸有完税标记的矿石机 5.带提把的蓝色老矿石收音机 6.一台成品矿石机的机芯 7.一台红色木制机箱矿石机 8.济南“胜利”矿石收音机 毕冠超收藏并撰文 9.“万明”矿石机 10.“中山公园”矿石机 11.滑动抽头线圈调感式袖珍矿石机 12.象牌101型袖珍矿石收音机 13.教学示范用矿石机 第三章 国外矿石机的研究和进展 第四章 新时期国内矿石机的研究和进展 一、器材篇 高阻耳机和匹配变压器 高灵敏度舌簧耳机介绍及超高阻耳机改造实例 矿石机匹配变压器作用以及简单的设计方法 寻找3DQ双栅场效应管花絮 怎样制作高Q值线圈 二、测量篇 3DQ、3DP的测试 用万用表判断MOS管的方法 场效应管小信号检波计算 平方检波原理与Rd 矿石机的实测 1.万用表测量二极管的方法和测试实例 2.二极管Rd的测试实例 3.矿石机调谐回路输出电压的测试实例 4.线圈Q值的测试实例 5.可变电容Q值的测试实例 6.耳机阻抗的测试实例 7.匹配变压器阻抗及效率的测试实例 8.矿石机灵敏度测试实例 9.用指针式万用表简单判断MOS管是否可用于矿石机检波 三、整机篇 复古矿石机 电罗经的复古矿石机 蛛网线圈双回路矿石机 三回路蛛网线圈矿石收音机 现代矿石机 二极管、MOSFET、再生三用矿石机 场效应管“远程”矿石收音机 高灵敏度MOS管矿石机 中波3DQ矿石机/太阳能再生机 直径0.8m的场效应管检波中波大环矿石收音机 无天地线箱式便携中波场管矿石机 实验无天地线磁棒矿石机 便携式矿石收音机 中短波磁环小型矿石收音机 短波大环矿石收音机 专为远程接收设计的变耦合度矿石收音机 调频矿石机 “重型”矿石机

\section{文件 3}

%版权 图书在版编目(CIP)数据 经典矿石机鉴赏与现代矿石机制作/徐蜀主编.--北京:人民邮电出版社,2016.2 ISBN 978-7-115-40672-9 Ⅰ.①经… Ⅱ.①徐… Ⅲ.①矿石收音机—介绍—世界 Ⅳ.①TN859 中国版本图书馆CIP数据核字(2015)第277537号 内容提要 矿石机是无电源、用矿石进行检波、再加上调谐电路的无线电接收机,至今已有百余年历史。 本书主要分为两个部分,一部分介绍了数十款国内外经典的矿石机,一部分介绍了现代优秀的矿石机制作案例。这些内容不仅让我们了解了早期无线电接收装备的发展历史,还让我们看到了在高新技术飞速发展的今天矿石机的设计与制作水平,怀旧与创新并存,历史与现代交融。 希望本书对收音机收藏爱好者、关注无线电技术发展史的朋友和有深厚无线电DIY情怀的人士开卷有益。 ◆主编 徐蜀 责任编辑 房桦 责任印制 周昇亮 ◆人民邮电出版社出版发行 北京市丰台区成寿寺路11号 邮编 100164 电子邮件 315@ptpress.com.cn 网址 http://www.ptpress.com.cn 北京市雅迪彩色印刷有限公司印刷 ◆开本:787×1092 1/16 印张:18.5 2016年2月第1版 字数:334千字 2016年2月北京第1次印刷 定价:128.00元 读者服务热线:(010)81055339 印装质量热线:(010)81055316 反盗版热线:(010)81055315 广告经营许可证:京崇工商广字第0021号

\section{文件 5}
%前言 前言 徐蜀 矿石机是无电源、用矿石进行检波、再加上调谐电路的无线电接收机,至今已有百余年历史。20世纪20年代初,随着公众广播电台的出现和快速发展,专门为此而设计制造的民用无线电接收机问世,它们被统称为收音机。故严格意义上的矿石收音机,始于20世纪20年代初,至今也已近百年。 20世纪50年代开始,半导体二极管等现代电子器件取代了天然矿石,但人们依旧称使用半导体二极管制作出来的收音机为矿石机。矿石机的电路虽然极其简单,且工作效率低下,但是纵观收音机的百年历史,我们却惊讶地发现,矿石机的生命力最为顽强!电子管收音机大约在20世纪70年代退出了历史舞台,生命周期总共60年;后起之秀的晶体管收音机、集成电路收音机,至今也不过60余年的时间。但是矿石机从20世纪初诞生,到现在 100多年了,乐此不疲的仍然大有人在。 矿石机生命力之所以如此旺盛,主要有两个原因。首先,矿石机简单、成本低、门槛也低,在经济不发达的历史时期,是人们收听广播的首选。即使欧美经济较发达地区, 20世纪20年代至30年代,矿石机也有着广阔的市场。这是因为当时电子管才发明不久,不仅机器价格昂贵,而且使用起来电池费用也不低,再有就是早期电子管的性能不是十分稳定。矿石机价格要便宜得多,还没有电源的拖累和费用;接收近距离的无线电信号时效果很好,故障率也不高。因此20世纪10年代,法国的邮政通信大量使用矿石机,英美等国家甚至在军事接收设备中也应用矿石机。直至20世纪40年代,日本军队仍然在使用矿石无线电接收机。其实,大量的文献资料和实物证明,在欧美等国家,矿石机不仅拥有大批业余爱好者,即使成品收音机的厂家也一直生产矿石机至20世纪90年代。例如德国Niefern bei Pforzheim的WISI品牌的矿石机,起码生产到了1962年;美国的Crystal Receiver CR-1 也生产到20世纪60年代初;德国的NF-Verstärkung系列矿石机一直生产到21世纪初。 在经济相对落后的中国,直至20世纪60年代,矿石收音机依然是许多家庭唯一的收听(广播)工具。而且在中国,50岁以上的男性,很多人在少年时代都有过组装矿石机的经历。艰难时期和儿时的记忆最为深刻,因此矿石收音机相比起其他收音机,更容易勾起人们的思念与怀旧之情。当今的老收音机收藏市场,矿石机的相对价格较高,也与此有关。 以上说的矿石收音机的历史,是其生命力顽强的第一个原因。 其次,20世纪60年代矿石机退出生产销售和家庭使用领域后,并没有一蹶不振,相反,众多的业余爱好者饱含热情地投入其中,去研究、DIY,使矿石收音机又焕发出第二次、而且是更加充满活力的青春。这一时期的矿石机DIY活动,与过去的以实用(用来收听广播)成分居多不同,娱乐消遣、丰富生活,乃是其动力与目的。业余无线电爱好者之所以看中了矿石机,是因为其原理简单、效果差,从而有着无限的改进空间,这样一来爱好者们就能够挑战自我,获得极大的成就感。 当前的矿石机DIY者,大致分作仿古与技术两大流派。仿古派比较注重矿石机古典风格的造型,注重传统手艺技法,比如木工、钳工、车工技术。我有一位在大学从事管理工作的朋友,将近退休的年龄了,为了“修成正果”,特地报名参加木工培训班,购买了小型机床,苦练基本功。数月后,便能够以假乱真地仿造出老矿石机中几乎所有的元器件。他的一台仿古矿石机作品,机箱、蛋型单连、线圈、活动矿石、彩色调谐度盘、耳机天地线标牌,甚至每一颗螺丝钉都是自己亲手仿真制作的。为了让矿石机形象逼真,他还学会了让铜“生锈”、木头长“包浆”等技巧,令人叹为观止。 技术派的想象力和动手能力同样令人折服。不少爱好者是无线电外行,为了做出性能优异的矿石收音机,他们首先要自学无线电知识,进而钻研、改进和设计矿石机电路,反复试验。他们所做的试验包括:甩掉矿石机的天地线还能够收听;除了中波,还要收听短波和调频广播;要用扬声器放出比较洪亮的声音等。在改造矿石机电路的同时,他们还在元器件上下了很大工夫。例如自己设计制作高性能耳机和耳机组件,绕制高性能线圈,不遗余力地寻找高性能的检波器等。他们对矿石机所有的元器件和附件都精挑细选或精心制作,毫不马虎。他们总结出一套行之有效的矿石收音机器件及整机性能指标的测试项目和数据,还为此自制了专用仪表。他们在学习和反复实验的过程中,理论和实践技能都得到了升华,很多方面大大超过了他们的前辈。 我非常幸运地结识了一批精通现代矿石收音机制作的朋友,使我多年来试图编撰一本有关矿石机图书的愿望更加坚定,并得以最终实现。在此,我对他们表示衷心的感谢! 这本书的出版,还要感谢《无线电》编辑部,尤其是主编房桦女士和资深编辑邓晨女士,她们二位一直在鼓励我们,支持我们,使我们克服重重困难,完成了本书的资料搜集和编撰工作,直至出版。\begin{figure}[htb]  \centering  \includegraphics[width=0.8\linewidth]{Image00055.jpg}  \caption{图片 1: Image00055.jpg}\end{figure}\begin{figure}[htb]  \centering  \includegraphics[width=0.8\linewidth]{Image00282.jpg}  \caption{图片 2: Image00282.jpg}\end{figure}\section{文件 6}\
%第一章 国外矿石机的历史回顾 第一章 国外矿石机的历史回顾 一、早期的线圈调感式矿石机 美国科学家皮卡德1906年11月20日取得了硅晶体探测器的专利,1910年发现硅晶体的整流作用,用作无线电检波器。因此使用矿石作检波器的商品收音机,应该出现在1910年之后。20世纪10年代以及20年代的很长时间内,矿石检波器多数是开放型、触点可调的所谓“活动矿石”,这一点在美国表现得更为突出。早期矿石收音机源自矿石无线电报接收机,其结构、电路,甚至元器件都基本相同。早期无线收报机的“感应线圈”“自感线圈”等很多老元件经常被使用。 20世纪20年代初期,民用广播出现后,适于家用的矿石机逐渐成型。由于当时广播电台数量不多,功率较小,相互干扰不严重,最初的家用矿石机对选择性的要求不高,调谐电路多数采用线圈抽头与分线器配合调感、线圈滑动抽头调感,或者双(多)线圈调感等方式,可变电容器使用不多。由于调感线圈制作灵活,体积、形状没有固定的模式,非常适宜制作各种特殊造型的矿石收音机,例如动物、书本形式的等,以下分别予以介绍。 抽头式线圈调感矿石机 抽头式线圈调感矿石机的线圈有多组抽头,借助分线器调整电感,接收不同频率的广播信号,它的优点是简单易行,而且经久耐用,无线电报时期,抽头调感线圈就已初见端倪了;它的缺点,因不是“无级调感”,所以调谐不够精细。直到20世纪60年代,我国的部分成品矿石收音机仍在使用这种分离式分线器,当然,已经与可变电容器配合使用,性能有了很大提高。早期的业余爱好者DIY矿石机时,也愿意采用这种方式。此类机器的分线器并非一个整体,是接线钉与旋转轴分离的,均匀地划分加工接线钉的位置,十分考验DIY者的技术和耐心。 1.法国壁挂式矿石机 图1-1所示的这台法国20世纪20年代初出品的壁挂式矿石收音机,就是非常典型的一款双回路抽头式调感机,线圈用多股纱包线绕在木质机箱外侧。机箱正面的活动矿石,方便使用者调节。此机由早期的壁挂式电话机(见图1-2)演变而来,它与普通电话机最大的不同是只能收听不能对话。据资料记载,该矿石机专用于收听附近电台播放的各类信息,如天气预报、交通情况、股市行情等。但这台壁挂式矿石收音机与通常的壁挂式电话机一样,要站着并手持耳机收听,不适合长时间使用,还没有脱离有线电话机信息传递的巢臼,不具备娱乐欣赏的功能。由于早期的接收机(收音机)来源于无线电话机,因此不少产品是由电报电话公司制造的。 图1-1 20世纪20年代初法国出品的壁挂式矿石机 图1-2 早期的壁挂式电话 2.加拿大“普通人”矿石机 再介绍一台1923年由加拿大电报电话公司出品的“普通人”牌抽头式线圈调感矿石收音机(见图1-3)。该机器系木制箱式便携机,尺寸为25cm×18.3cm×22.2cm。“普通人”最早为美国德福雷斯特无线电电话电报公司(DeForest Radio Telephone \& Telegraph Co.)的产品,加拿大电报电话公司次年购买其专利,仿制了该机,机器的体积和某些细节与原型机稍有差别。“普通人”机箱的掀盖很有特色,两面均有合叶,可以完全脱离机器,使用起来非常方便。电路部分,在两组可调抽头线圈中串联了一个插接式蜂房线圈,该蜂房线圈有3组,其中两组为长波线圈,1组为短波线圈,这样就拓宽了矿石机的接收频率。 图1-3 1923年由加拿大电报电话公司出品的“普通人”抽头式线圈调感矿石收音机 3.美国“宝宝”袖珍矿石机 图1-4所示的是美国20世纪20年代出品的双回路线圈抽头袖珍式,也称“宝宝”或“口袋”矿石收音机。此机尺寸只有9.7cm×5.5cm×3.5cm,它小巧可爱,并且有一个漆布面的函套,外观很像是一部中国古书。此机机箱是一整块木头雕琢而成,无骨架抽头式线圈,设计巧妙,制作精美。 图1-4 20世纪20年代美国出品的“宝宝”双回路线圈抽头袖珍式矿石收音机 4.德国WISI.NR.56袖珍矿石机 与这台美国“宝宝”矿石收音机形状差不多的,有一台1933年德国出品的WISI.NR.56型矿石机(见图1-5)。该机也属袖珍机,体积为9.9cm×6.6cm×3cm。该机由硬纸机壳包覆黑色漆布,天、地线及耳机接口为插头式,未采用体积较大的接线柱。巧妙的是,该机天线有4个插口,分别为线圈的4个抽头,天线插在不同的位置可以改变接收频率。 图1-5 德国1933年出品的WISI.NR.56型矿石收音机 5.德国具有减震功能的袖珍矿石机 图1-6所示的这台20世纪20年代德国出品的袖珍矿石收音机就更加精彩了。该矿石收音机机壳为酚醛材质,整机尺寸为10cm×7cm×3cm,机芯尺寸为6.3cm×5cm,十分小巧。机芯虽然很小,但线圈的抽头却多达13个,因而能够比较精细地调谐接收电台频率。更为巧妙的是,该机的机芯由4根小弹簧悬挂在机盒中,因而可以有效地起到防震保护作用。 图1-6 德国20世纪20年代出品具有机芯减震功能的袖珍矿石收音机 6.法国迷宫式线圈袖珍矿石机 还有一台调谐方式与此机相近,那是1925年法国出品的迷宫式双调谐线圈矿石机(见图1-7),它制作更加精美,两组线圈的抽头多达60余个,赛璐珞面板上镶嵌着蝴蝶形状的银色金属图案,显得格外精致典雅。该机也属于袖珍型,木质机盒尺寸为13cmX9.5cmX5.5cm。 图1-7 1925年法国出品的迷宫式双调谐线圈矿石机 线圈滑动抽头调感式矿石机 1.法国早期滑动抽头线圈矿石机3种 说起早期的矿石收音机,线圈滑动抽头调感式应该是采用最多的一种方式。这种线圈是用一长臂形铜片做成活动抽头,紧压在线圈上作纵向滑动,铜片所及处的漆包线或纱包线是裸露导电的,等同于抽头,铜片改变位置就改变了线圈的电感量。滑动抽头调感线圈的历史也可以追溯到无线电报时期,其好处是调谐频率近似于无级调整,比较精细;另一个好处是结构简单紧凑,形式多样,适合各种体积、形状的机型安装。缺点是线圈与动片之间容易接触不良。比较传统的样式是图1-8和图1-9所示的这三台法国矿石收音机,其生产时间为20世纪20年代。可以看得出来,这些机器只是在老式的滑动抽头调感线圈底板上加装了一个活动矿石检波器而已。 图1-8 早期的滑动抽头调感线圈 图1-9 20世纪20年代法国出品的滑动抽头调感式矿石机 2.法国早期格勒诺布尔矿石机 下面是一架法国早期出品的格勒诺布尔矿石机(见图1-10),看起来很原始古朴,依靠滑动抽头线圈调谐频率。 图1-10 法国早期出品的格勒诺布尔矿石机 3.法国便携式邮政矿石机 1920年前后,一些高档的矿石机也采用滑动线圈抽头调感的方式。图1-11所示这款是法国1914年出品的便携式邮政矿石机,调谐机构是滑动线圈,检波为矿石和电解液双检波器。法国是擅长制造奢侈品的国度,还在收音机的萌芽阶段,它便已充分显现出这方面的优势。据相关文献记载,这台矿石机可以接收艾非尔铁塔上无线电台发布的标准授时、股票行情及其他信息。电解液检波器是电子管的发明者德弗雷斯特20世纪10年代初,针对马可尼无线电报机“金属屑检波器”的灵敏度太差,严重影响收发效果而发明的。电解液检波器需要电源,该机设有转换开关,负责切换两种检波器,并关断电源。矿石机机箱采用高级柚木制成,保存百年毫无变形开裂。箱体附件的合叶、提梁、锁扣为黄铜精制。调谐线圈的刻度标尺、晶体检波器支架、接线柱,以及其他连接件、螺丝等材料,也为黄铜制造。机器无论内外均给人以华贵、秀美的感觉。 图1-11 1914年法国出品的便携式邮政矿石机 4.美国JR矿石机 滑动抽头调感线圈矿石收音机电路虽然极其简单,但在早期矿石收音机的发展中却有着不俗的表现。除了上面介绍过的法国便携邮政矿石收音机外,美国联邦电话电报公司于1922年也出品了一款“JR”型精品矿石收音机(见图1-12)。JR矿石机采用铁制机箱,整机尺寸为21.6cm×22.4cm×15.2cm。JR造型类似双面天平,机身两面标有刻度,配以指针形调谐兼频率指示器;顶部安装活动矿石;箱体为黑色漆面,镀铬指针和矿石架形成强烈反差,给人以精密、复杂的感觉。其实该机内部只有一个单回路线圈,中间部分纱包线裸露供滑动调谐,机器两面的指针具有同样的调节功能。JR体积虽不大,由于机壳的铁板较厚,因此分量很重,拨动指针调台时非常稳当。 图1-12 美国联邦电话电报公司1922年出品的“JR”型矿石收音机 5.德律风根ZA矿石机 6.德律风根RDE矿石机 德国德律风根公司20世纪20年代也出品了两款比较经典的线圈滑动抽头调感式矿石机。第一种是ZA型垂直式(见图1-13),它高18cm,直径10cm。第二种为“RDE”型(见图1-14),它出品于1925年,外形为金属圆形机盒,直径14cm,高6cm。 图1-13 德国德律风根ZA矿石机 图1-14 德律风根RDE矿石机 7.美国BIG 4矿石机 8.美国火星矿石机 9.美国“小宝石”矿石机 除此之外,20世纪20年代美国还生产了3种销路较广的线圈滑动抽头调感式矿石机。前两种为系列产品,仅外观有所区别。第一种为MARCONI ERA OLD MARTIAN BIG 4(见图1-15),它高20cm,是马可尼时期的著名矿石机,美国早期多种经典收音机图录中都介绍过此机。第二种与前一种同属于“火星”系列产品(见图1-16),只是没有3条支架,安装在铁质底座上而已。第三种号称“小宝石”(见图1-17),与第一种BIG 4的造型和结构相近,尺寸略小,金属部分全部镀铬,线圈用皮革包裹,用料和制作工艺都好于BIG 4。 图1-15 美国20世纪20年代出品的MARCONI ERA OLD MARTIAN BIG 4矿石机 图1-16 火星系列的另一种矿石机 图1-17 美国20世纪20年代出品的“小宝石”矿石机 10.美国“瓢虫”陶瓷矿石机 由于滑动抽头线圈调感的结构灵活多变,因此能够做出形形色色的矿石收音机,如图1-18所示为一台美国20世纪20年代的瓢虫矿石收音机。该机由美国Brush真品陶瓷公司于1927—1930年期间生产。1927年,真品陶瓷公司决定设计生产一系列独特的陶瓷收音机,借以吸引世界各地的收音机爱好者,这款矿石收音机后来被称作“无线电瓢虫”。虽然号称“瓢虫”,但矿石机的尺寸却很大,长度达到了23.5cm,宽14cm,高7.5cm。这台瓢虫矿石机的形态非常逼真,制作者巧妙地将活动矿石、接线柱布置成虫子的触须,滑动调感线圈藏在肚子里,通过背部的滑竿调节接收频率。 图1-18 美国20世纪20年代的瓢虫矿石机 11.英国“汤姆叔叔”陶瓷矿石机 1924年英国也生产了一款陶瓷材质,人物坐像造型的“汤姆叔叔”矿石机(见图1-19)。线圈绕在汤姆叔叔的大礼帽上面,通过划片调谐电台频率。 图1-19 1924年英国生产的陶瓷材质,人物坐像造型的“汤姆叔叔”矿石机 12.法国的“美洲豹”“热带鱼”和“羚羊”陶瓷矿石机 13.美国“啤酒瓶”矿石机 上面提到了两款陶瓷材质的矿石机,顺便再介绍20世纪40年代法国出品的3款陶瓷矿石机。这3种陶瓷矿石机制作精美,具有很高的观赏价值,它们分别为美洲豹、热带鱼和羚羊造型(见图1-20)。美国20世纪30年代出品的“啤酒瓶”矿石机,也采用的这种调谐方式(见图1-21)。 图1-20 20世纪50年代法国出品的陶瓷矿石机 图1-21 美国20世纪30年代的“啤酒瓶”矿石机 14.英国书本式矿石机 下面言归正传,介绍一架1925年由英国Kenmac无线电有限公司出品的书本式滑动抽头线圈调感矿石机(见图1-22)。此机的体积很小,只有12.1cm×8.9cm×2.5cm,形状就像一本袖珍精装书。收音机的颜色有仿玳瑁、红色、蓝色和绿色,而且封面上文字的字体可以免费订制。这架书本式矿石机的构思与制作十分精巧,打开书口的折页,活动矿石、电台调谐与耳机、天地线接口等机构,一应俱全。由于构件材料考究、制作精良,这架构造复杂的袖珍矿石机,比起其他多数早期的袖珍矿石机,要结实耐用得多。 图1-22 1925年英国Kenmac无线电有限公司出品的书本式滑动抽头线圈调感矿石机 15.美国Metro.jr矿石机 接下来是1923年美国纽约大都会广播公司出品的metro jr型铝制模压成型的矿石机(见图1-23),它采用的也是滑动抽头线圈调感方式。metro jr造型别致、小巧,尺寸为17.8cm×6cm×12.7cm。由于该机主体模压制作一次成型,因此其成本低、效率高,在当年价格很便宜,深受用户欢迎,并拥有很高的产量,至今仍有不少品相如新的机器存世。 图1-23 1923年美国纽约大都会广播公司出品的metro jr型铝制模压成型的矿石机 16.美国PACENT袖珍矿石机 还有一款采用滑动抽头线圈调感方式制作的实木小型矿石收音机,也非常经典,它具有20世纪20年代典型的美国矿石机风范(见图1-24)。此机主体为木材雕刻而成,机箱尺寸为12.5cm×8cm×4.5cm。面板采用胶木材质,大型立式活动矿石系美国PACENT公司的产品。从整体材料与工艺水平看,应该是20世纪20年代的厂制机,很有可能就是PACENT公司组装的,理由是该公司是美国当时著名的无线电元器件制造商。 图1-24 20世纪20年代美国出品的实木机壳矿石收音机 17.德国德律风根板式矿石机 再欣赏一台德国著名无线电厂商德律风根20世纪20年代出品的滑动抽头线圈调感矿石机(见图1-25)。这台矿石机系开放型板式机,木制底座长22cm,宽12cm,底座上嵌有德文“德律风根专利”字样铜牌,该机具有欧式活动矿石架、金属接线柱(分别为天地线和耳机)及功能金属标牌。比较有特色的是,滑动抽头线圈串接了两组花篮线圈。机器底座下粘有一张印着“Friho”字样的纸质标签,我猜想大概是机器所有者做的标记。 图1-25 20世纪20年代德国德律风根出品的滑动抽头线圈调感矿石机 线圈耦合调感式矿石机 1.早期线圈调感器4种 在早期矿石机的调谐方式中,第三种方式是线圈耦合调感式,即由两组或更多组线圈组成调谐电路,通过改变线圈间的位置而改变电感量,以接收不同频率的电台广播。上述线圈既可以是一般筒式线圈,也可以是花篮式、蛛网式。早期还流行一种所谓“松耦合”式调谐器,即一组固定的长筒形线圈中,放置一直径略小的活动筒形线圈,调节小线圈的位置即可改变电感量。一些较高级的松耦合线圈还同时具备了抽头线圈和滑动调感线圈的功能,使调谐的功能更加完善。以上方式中还分为自感式和互感式两种。自感式为串联型,两组线圈是相通的,与自耦式电源变压器的原理相同。线圈耦合调感式电路的优点是效率高,并兼顾到了机器的选择性,早期的成品矿石收音机多乐于采用。20世纪30年代之前,由于线圈耦合调感器结构比较复杂,很多产品都做成一个独立的组件,厂家或爱好者可以用它们去组装矿石机及电子管接收机。当时一些独立调谐器制作异常精美,具有很强的装饰色彩。下面先领略一下早期各类线圈耦合调感式调谐器的风采(见图1-26~图1-28)。 图1-26 20世纪20年代出品美国的三组蜂房式调谐器 图1-27 20世纪20年代初美国出品的木制调谐器 图1-28 20世纪20年代初美国出品的具有抽头的双耦合线圈调谐器 2.德国德律风根公司E04矿石机 这里介绍的第一款线圈耦合调感式矿石机,是1905年左右德国德律风根公司出品的E04(见图1-29)。E04是非常早期的无线电接收机,那一时期由于无线电发射功率较低,主要接收火花发射机的信号,对接收机的效率要求很高,电路和结构因此异常复杂坚固。此机底板26cm×21cm,总高度41cm。我这台E04是德国无线电爱好者从旧货市场买到E04残破的主体构件后,自己DIY复制而成的。据原收藏者说,接上天地线后与功率放大器,这台矿石机的接收灵敏度和选择性都相当不错,并且可以推动扬声器,发出宏亮的声音。 图1-29 德律风根E04 图1-29 德律风根E04(续) 图1-29 德律风根E04(续) 3.美国Westinghouse矿石机 下面一款线圈耦合调感式矿石机,是1921年美国出品的Westinghouse矿石机(见图1-30)。这款Westinghouse矿石机是20世纪20年代前期美国非常经典的一款矿石收音机,在当时的无线电类杂志或图书中经常可以看到它的身影。直至今日,保存下来的还有不少。Westinghouse电器公司是美国历史悠久的名牌企业,该公司最早购买交流电的专利,开启了美国交流电发电和利用的先河。而且Westinghouse电器公司还是美国无线电广播事业的先驱, 1920年,位于东匹兹堡Westinghouse电气厂屋顶的发射天线,播出的KDKA电台的广播信号,昭示着世界广播事业的正式诞生。到了20世纪10年代末,因财务状况及专利纠纷等方面的原因,Westinghouse部分无线电业务加入了美国通用电器公司。 图1-30 美国1921年出品的Westinghouse矿石机 图1-30 美国1921年出品的Westinghouse矿石机(续) 4.美国ER-753A便携式矿石机 同一时期,双线圈旋转调感式矿石机的另一款名机,是1922年美国RCA公司出品的ER-753A型双检波器高档便携式矿石收音机(见图1-31)。该矿石机同时设置了红锌矿石和方铅矿石两种检波器,使用者可根据不同的情况选择使用。ER-753A在收音机发展史上有着十分重要的位置,因为它从形式上奠定了家庭用便携式收音机的基础。RCA公司于1922年推出ER-753A的同时,还上市了电路完全一样的台式机—ER-753。显而易见,前者是提供给人们外出旅行时使用的。20世纪20年代初期的民用收音机大多是台式机,而且很多还是采用元器件裸露在外的开放式,这些收音机仅仅适合固定在家庭环境使用。ER-753A的体积为26cm×14.5cm×16cm,这样的体型虽然算不上袖珍,但外出旅行携带也还算方便。顶部的皮制可伸缩提手,有利于携带,这也成为此后便携式收音机的标志。便携式收音机不仅体积不可太大,有提手便于使用,而且还要具备完善的防碰撞和防震动性能。这一点该机也做得很到位。ER-753A的密闭式机箱为锁控两开门:第一道门开启后供使用者调谐电台及矿石的触点;第二道门内是机器的内部结构,供检修用,而且线圈还有一个硬纸板模压保护套,以加强防护。收音机机箱的两面装有防震脚垫,可以根据需要将收音机立着或卧着摆放。 图1-31 美国RCA公司1922年出品的ER-753A型双检波器高档便携式矿石机 图1-31 美国RCA公司1922年出品的ER-753A型双检波器高档便携式矿石机(续) 5.美国Atwater Kent板式矿石机 美国著名的无线电制造商阿特沃特肯特(Atwater Kent)制造公司,曾出品了一款板式矿石机套件(见图1-32),它采用了线圈耦合调感式电路。阿特沃特肯特公司位于费城,成立于1919年,1921年起涉足广播事业,1922年推出了首款Atwater Kent 39XX系列板式电子管收音机。Atwater Kent板机也被称作AK机,是首批具有明显装饰功能的家用收音机。该机共出品了2~6管的数种机型,并出售1管机和矿石收音机板机套件,供爱好者自行组装。AK板机最大的特点是,主要部件均为该公司自制的胶木壳组件,如电子管组件、调谐线圈组件等。这些组件装配的板机不仅性能可靠,还具有极好的装饰性,在收音机造型“仪器”化的20世纪20年代,不禁使人耳目一新。时至今日,AK板机仍然是人们孜孜以求的珍贵藏品。 6.英国BTH-C双矿石检波器矿石机 再有就是1922年英国BTH汤森休斯顿有限公司生产和销售的C型高级双矿石检波器矿石机(见图1-33),该机也是采用的线圈调感式调谐电路。需要注意的是,早年在英国生产销售广播收音机,是要到邮政部门注册、批准的。得到授权后,通常要在收音机机箱上打上“BBC”的标识,这并不是收音机的品牌。本机正面圆形图标中便有BBC的字样。厂标“BTH”则位于机箱上方金属提梁的下方。该机器体积为29cm×20.3cm×13.2cm,采用全封闭箱式,面板上装有大型玻璃罩双活动矿石,可以通过转换开关选择矿石1或矿石2。此机也是便携式矿石机,可供人们外出旅行使用。 图1-32 美国阿特沃特肯特(Atwater Kent)板式矿石机 图1-33 1922年英国BTH汤森休斯顿有限公司生产和销售的C型高级双矿石检波器矿石机 图1-33 1922年英国BTH汤森休斯顿有限公司生产和销售的C型高级双矿石检波器矿石机(续) 7.瑞典爱立信L.M矿石机 20世纪20年代,线圈耦合调感式矿石机形式非常之多,下面这台是1926年瑞典爱立信公司生产的L.M.矿石收音机(见图1-34)。此机采用圆形机箱,直径16.2cm,高4cm。调感机构由8个小线圈分为上下两组,转动旋钮时,上下两组线圈的位置会改变,以此来获取不同的电感量。 图1-34 1926年瑞典爱立信公司生产的L.M.矿石收音机 8.德国罗兰士Lorenz EDAT25型双矿石检波矿石机 接下来是1925年德国柏林出品的罗兰士Lorenz EDAT25型双矿石检波收音机(见图1-35)。 图1-35 1925年德国柏林出品的罗兰士Lorenz EDAT25型双矿石检波收音机 图1-35 1925年德国柏林出品的罗兰士Lorenz EDAT25型双矿石检波收音机(续) 9.德国“狐狸”矿石机 还有一款早期德国制造的线圈耦合调感式矿石机,系木制狐狸造型(见图1-36)。我曾经见到过多台这种矿石机,它们的木制结构大体一样,仅细节部分略有出入。该矿石机底座长20cm,宽10cm,整机高23.5cm,利用调节两组花篮式线圈的相互位置改变接收频率。 图1-36 德国20世纪20年代出品的狐狸造型矿石机 10.英国The Electron 双线圈调感矿石机 另一款是1925年英国出品的The Electron 双线圈调感矿石机,它的尺寸为19cm×15cm,该机造型古朴,调谐简易直观(见图1-37)。 图1-37 1925年英国出品的The Electron 双线圈调感矿石机 11.美国纽约丹齐格琼斯公司出品的线圈耦合调感式三检波器矿石机 再有一款美国纽约丹齐格琼斯公司出品的线圈耦合调感式、三检波器矿石机(见图1-38)。该机的构造与材料很特殊,是采用铸铝框架,玻璃面板和纸质线圈架,共有4组线圈,其中两组为活动可调式。此机的另一个特点是,标注了具体的出厂时间。 图1-38 美国纽约丹齐格琼斯公司出品的线圈耦合调感式、三检波器矿石机 图1-38 美国纽约丹齐格琼斯公司出品的线圈耦合调感式、三检波器矿石机(续) 12.德国Dyhr耳机矿石机 最后介绍的是1926年德国出品的Zentler System“Dyhr”耳机矿石机。此机配备了几个不同电感量的小型线圈,使用时分别插在一个耳机的背面插口处,从而调谐接收不同的电台。在那只耳机中还安装了微型活动矿石。耳机背面还有一个插口,接驳在家中的暖气管、自来水管或者晾衣架上,即可收听(见图1-39)。 图1-39 Zenter Syste“n Dyhr”耳机矿石机 图1-39 Zenter Systen“Dyhr”耳机矿石机(续)\begin{figure}[htb]  \centering  \includegraphics[width=0.8\linewidth]{Image00317.jpg}  \caption{图片 3: Image00317.jpg}\end{figure}\begin{figure}[htb]  \centering  \includegraphics[width=0.8\linewidth]{Image00306.jpg}  \caption{图片 4: Image00306.jpg}\end{figure}\begin{figure}[htb]  \centering  \includegraphics[width=0.8\linewidth]{Image00232.jpg}  \caption{图片 5: Image00232.jpg}\end{figure}\begin{figure}[htb]  \centering  \includegraphics[width=0.8\linewidth]{Image00326.jpg}  \caption{图片 6: Image00326.jpg}\end{figure}\begin{figure}[htb]  \centering  \includegraphics[width=0.8\linewidth]{Image00045.jpg}  \caption{图片 7: Image00045.jpg}\end{figure}\begin{figure}[htb]  \centering  \includegraphics[width=0.8\linewidth]{Image00253.jpg}  \caption{图片 8: Image00253.jpg}\end{figure}\begin{figure}[htb]  \centering  \includegraphics[width=0.8\linewidth]{Image00124.jpg}  \caption{图片 9: Image00124.jpg}\end{figure}\begin{figure}[htb]  \centering  \includegraphics[width=0.8\linewidth]{Image00133.jpg}  \caption{图片 10: Image00133.jpg}\end{figure}\begin{figure}[htb]  \centering  \includegraphics[width=0.8\linewidth]{Image00073.jpg}  \caption{图片 11: Image00073.jpg}\end{figure}\begin{figure}[htb]  \centering  \includegraphics[width=0.8\linewidth]{Image00101.jpg}  \caption{图片 12: Image00101.jpg}\end{figure}\begin{figure}[htb]  \centering  \includegraphics[width=0.8\\linewidth]{Image00261.jpg}  \caption{图片 13: Image00261.jpg}\end{figure}\begin{figure}[htb]  \centering  \includegraphics[width=0.8\\linewidth]{Image00072.jpg}  \caption{图片 14: Image00072.jpg}\end{figure}\begin{figure}[htb]  \centering  \includegraphics[width=0.8\\linewidth]{Image00153.jpg}  \caption{图片 15: Image00153.jpg}\end{figure}\begin{figure}[htb]  \centering  \includegraphics[width=0.8\\linewidth]{Image00143.jpg}  \caption{图片 16: Image00143.jpg}\end{figure}\begin{figure}[htb]  \centering  \includegraphics[width=0.8\\linewidth]{Image00049.jpg}  \caption{图片 17: Image00049.jpg}\end{figure}\begin{figure}[htb]  \centering  \includegraphics[width=0.8\\linewidth]{Image00329.jpg}  \caption{图片 18: Image00329.jpg}\end{figure}\begin{figure}[htb]  \centering  \includegraphics[width=0.8\\linewidth]{Image00054.jpg}  \caption{图片 19: Image00054.jpg}\end{figure}\begin{figure}[htb]  \centering  \includegraphics[width=0.8\\linewidth]{Image00165.jpg}  \caption{图片 20: Image00165.jpg}\end{figure}\begin{figure}[htb]  \centering  \includegraphics[width=0.8\\linewidth]{Image00097.jpg}  \caption{图片 21: Image00097.jpg}\end{figure}\begin{figure}[htb]  \centering  \includegraphics[width=0.8\\linewidth]{Image00334.jpg}  \caption{图片 22: Image00334.jpg}\end{figure}\begin{figure}[htb]  \centering  \includegraphics[width=0.8\\linewidth]{Image00275.jpg}  \caption{图片 23: Image00275.jpg}\end{figure}\begin{figure}[htb]  \centering  \includegraphics[width=0.8\\linewidth]{Image00351.jpg}  \caption{图片 24: Image00351.jpg}\end{figure}\begin{figure}[htb]  \centering  \includegraphics[width=0.8\\linewidth]{Image00005.jpg}  \caption{图片 25: Image00005.jpg}\end{figure}\begin{figure}[htb]  \centering  \includegraphics[width=0.8\\linewidth]{Image00138.jpg}  \caption{图片 26: Image00138.jpg}\end{figure}\begin{figure}[htb]  \centering  \includegraphics[width=0.8\\linewidth]{Image00267.jpg}  \caption{图片 27: Image00267.jpg}\end{figure}\begin{figure}[htb]  \centering  \includegraphics[width=0.8\\linewidth]{Image00354.jpg}  \caption{图片 28: Image00354.jpg}\end{figure}\begin{figure}[htb]  \centering  \includegraphics[width=0.8\\linewidth]{Image00114.jpg}  \caption{图片 29: Image00114.jpg}\end{figure}\begin{figure}[htb]  \centering  \includegraphics[width=0.8\\linewidth]{Image00332.jpg}  \caption{图片 30: Image00332.jpg}\end{figure}\begin{figure}[htb]  \centering  \includegraphics[width=0.8\\linewidth]{Image00266.jpg}  \caption{图片 31: Image00266.jpg}\end{figure}\begin{figure}[htb]  \centering  \includegraphics[width=0.8\\linewidth]{Image00053.jpg}  \caption{图片 32: Image00053.jpg}\end{figure}\begin{figure}[htb]  \centering  \includegraphics[width=0.8\\linewidth]{Image00362.jpg}  \caption{图片 33: Image00362.jpg}\end{figure}\begin{figure}[htb]  \centering  \includegraphics[width=0.8\\linewidth]{Image00339.jpg}  \caption{图片 34: Image00339.jpg}\end{figure}\begin{figure}[htb]  \centering  \includegraphics[width=0.8\\linewidth]{Image00193.jpg}  \caption{图片 35: Image00193.jpg}\end{figure}\begin{figure}[htb]  \centering  \includegraphics[width=0.8\\linewidth]{Image00159.jpg}  \caption{图片 36: Image00159.jpg}\end{figure}\begin{figure}[htb]  \centering  \includegraphics[width=0.8\\linewidth]{Image00297.jpg}  \caption{图片 37: Image00297.jpg}\end{figure}\begin{figure}[htb]  \centering  \includegraphics[width=0.8\\linewidth]{Image00196.jpg}  \caption{图片 38: Image00196.jpg}\end{figure}\begin{figure}[htb]  \centering  \includegraphics[width=0.8\\linewidth]{Image00043.jpg}  \caption{图片 39: Image00043.jpg}\end{figure}\begin{figure}[htb]  \centering  \includegraphics[width=0.8\\linewidth]{Image00341.jpg}  \caption{图片 40: Image00341.jpg}\end{figure}\begin{figure}[htb]  \centering  \includegraphics[width=0.8\\linewidth]{Image00298.jpg}  \caption{图片 41: Image00298.jpg}\end{figure}\begin{figure}[htb]  \centering  \includegraphics[width=0.8\\linewidth]{Image00110.jpg}  \caption{图片 42: Image00110.jpg}\end{figure}\begin{figure}[htb]  \centering  \includegraphics[width=0.8\\linewidth]{Image00000.jpg}  \caption{图片 43: Image00000.jpg}\end{figure}\begin{figure}[htb]  \centering  \includegraphics[width=0.8\\linewidth]{Image00283.jpg}  \caption{图片 44: Image00283.jpg}\end{figure}\begin{figure}[htb]  \centering  \includegraphics[width=0.8\\linewidth]{Image00383.jpg}  \caption{图片 45: Image00383.jpg}\end{figure}\begin{figure}[htb]  \centering  \includegraphics[width=0.8\\linewidth]{Image00166.jpg}  \caption{图片 46: Image00166.jpg}\end{figure}\begin{figure}[htb]  \centering  \includegraphics[width=0.8\\linewidth]{Image00371.jpg}  \caption{图片 47: Image00371.jpg}\end{figure}\begin{figure}[htb]  \centering  \includegraphics[width=0.8\\linewidth]{Image00211.jpg}  \caption{图片 48: Image00211.jpg}\end{figure}\begin{figure}[htb]  \centering  \includegraphics[width=0.8\\linewidth]{Image00076.jpg}  \caption{图片 49: Image00076.jpg}\end{figure}\section{文件 7}\
%第一章 国外矿石机的历史回顾 二、早期的线圈与可变电容器耦合调谐矿石机 可变电容器很早便在矿石机中使用了。线圈与可变电容器配合使用,可以取得更加优良的接收效果。当然,在矿石机中加入可变电容器,成本会相应提高,这在20世纪20年代,在民用无线电领域,无论厂家还是业余爱好者,都是要仔细斟酌的一件事情。下面就按照年代的顺序,介绍早期线圈与可变电容器耦合调谐频率的矿石机。 1.美国RCA AR1300矿石机 先介绍一台1921年美国通用电气公司出品的RCA AR1300型矿石机(见图1-40)。该矿石机外壳为军绿色铁质机壳,而且铁皮相当厚,所用线圈、可变电容器的材料都很厚重,据说最早是军用或船上使用的,随着广播的快速兴起,才开始进入民用领域。与AR1300配套使用的AR1400三管功率放大器(见图1-41)也同时被推出,它可以推动扬声器供多人收听广播。 图1-40 1921年美国通用电气公司出品的RCA AR1300型矿石机 图1-40 1921年美国通用电气公司出品的RCA AR1300型矿石机(续) 图1-41 与RCA AR1300型矿石机配套使用的AR1400三管功率放大器 图1-41 与RCA AR1300型矿石机配套使用的AR1400三管功率放大器(续) 2.美国Betts\\&amp;Betts板式矿石机 再介绍一台1924年由美国贝茨\\&amp;贝茨(Betts \\&amp; Betts)公司出品的板式矿石机(见图1-42)。贝茨\\&amp;贝茨公司早年以生产各类无线电元器件而著称,后来也制造了少量成品收音机。这台矿石机虽然结构简单,但元件的选用和工艺制作却十分考究。矿石机装备了早期无线电报机经常使用的木质自耦式可调谐线圈(系该厂生产的产品),密封式空气单连和密封式活动矿石。底板的制作非常精良,背面的焊点和走线采用打眼刻槽方式,以保证电路长期使用的可靠性。 图1-42 美国贝茨\\&amp;贝茨(Betts \\&amp; Betts)公司1924年出品的板式矿石机 3.美国吉卜林豪华版矿石机 1920年美国出品的吉卜林豪华版矿石机(见图1-43),此机有简装和豪华两种版本,后者为双盒,大小盒之间可放置耳机和线圈。机器介于早期无线电接收机与后期民用收音机的过渡期,安装有帮助校准频率的蜂鸣器,因此要使用电池。 图1-43 美国1920年出品的吉卜林豪华版矿石机 图1-43 美国1920年出品的吉卜林豪华版矿石机(续) 4.美国直立板式矿石机 接下来是一台美国20世纪20年代出品的直立板式矿石机(见图1-44),它采用旋转可调线圈结合可变电容器调谐方式。该矿石机比起一般的板式机,更节省摆放的空间,机体顶部有一个扣手,因此很方便移动。 图1-44 美国20世纪20年代出品的直立板式矿石机 5.法国三线圈矿石机 下面这台法国20世纪20年代出品的矿石机则又是一种风格,如图1-45所示,它的机箱尺寸为18.5cm×13.3cm×14.3cm。这台机器已经比较高级,由3组可调电感线圈加上可变电容器构成整机的调谐电路。3组线圈位于机箱上部,中间的为固定线圈,两边的为可调线圈,它们通过各自的手柄来调整角度,以达到改变电感量的目的。这台矿石机线圈组的风格具有浓郁的法国格调,此后许多法国古典电子管收音机都曾采用此种风格。 图1-45 20世纪20年代法国出品的三组可调线圈与可变电容器耦合的矿石机 6.法国双回路调谐矿石机 图1-46所示为法国20世纪20年代出品的双回路调谐矿石机。此机没有厂牌和型号,但明显系厂制量产机,它设计新颖,做工精细。机器为两级台阶式,底座尺寸为25cm×15cm,前高11cm,后高16.8cm。机箱有4块木板和4块胶木板组成,木板均由子母槽连接,这使得箱体能够保证长期不变形。本机电路为双回路式,频率调谐分别由顶部可调线圈(可插接不同的线圈及调整两线圈之间的距离)和机内两个独立的可变电容器完成。该矿石机有一个波段转换开关,用以控制接收长波或短波。 7.德国1001矿石机 再看一台1926年德国出品的1001型矿石机(见图1-47)。该机箱底板尺寸为12.5cm×16.5cm,侧面呈梯形,前后高度分别为7cm和9.5cm。它内部有一架花篮线圈,同外面可更换蜂房线圈进行串联,并与空气可变电容器构成频率调谐回路。 图1-46 法国20世纪20年代出品的双回路调谐矿石机 图1-46 法国20世纪20年代出品的双回路调谐矿石机(续) 图1-47 1926年德国出品的1001型矿石机 图1-47 1926年德国出品的1001型矿石机(续) 8.美国劳伦斯矿石机 下面介绍的是1923年美国劳伦斯公园广播公司出品的矿石机(见图1-48)。劳伦斯公园广播公司位于匹兹堡,它是一家生产简单电子管收音机的小公司。此机是该公司生产的唯一一种矿石机。机器体积为33cm×15.2cm×12.7cm,造型中规中矩,尺寸大,具有当时电子管收音机的风范。该机电路非常经典:采用抽头并旋转式电感可调线圈,并配有可变电容器。别出心裁的是,该机采用了一颗外置固定矿石。固定矿石的优点是密封性好,触点一次性调好后,经久不变。当然,固定矿石更换成本比活动矿石要高。 图1-48 1923年美国劳伦斯公园广播公司出品的矿石机 图1-48 1923年美国劳伦斯公园广播公司出品的矿石机(续) 9.英国马可尼矿石机 图1-49所示是20世纪20年代英国马可尼公司出品的,具有旋转式电感可调线圈、可变电容器调谐电路、防震密封式活动矿石的矿石机。该机器用料考究,做工精细,无愧为马可尼公司的产品。该机器尺寸为16cm×16cm×13cm,采用木箱胶木面板。本机的最大亮点是那个密封并具有防震功能的活动矿石,该活动矿石的外壳是由金属压制而成,壳壁很厚,矿石由铜制套子和弹簧组成,具有很好的防尘、防震作用。 图1-49 20世纪20年代英国马可尼公司出品的具有旋转式电感可调线圈、可变电容器调谐的矿石机 图1-49 20世纪20年代英国马可尼公司出品的具有旋转式电感可调线圈、可变电容器调谐的矿石机(续) 10.奥地利GEWES矿石机 还有一台奥地利维也纳GEWES公司出品的电木圆形机箱矿石机,如图1-50所示,该机直径为13cm,高6.5cm。此机体积小巧,内部有一架花篮线圈,线圈中间是可变电容器。 图1-50 1927年奥地利维也纳GEWES公司出品的电木圆形机箱矿石机 图1-50 1927年奥地利维也纳GEWES公司出品的电木圆形机箱矿石机(续)\begin{figure}[htb]  \centering  \includegraphics[width=0.8\\linewidth]{Image00274.jpg}  \caption{图片 50: Image00274.jpg}\end{figure}\begin{figure}[htb]  \centering  \includegraphics[width=0.8\\linewidth]{Image00299.jpg}  \caption{图片 51: Image00299.jpg}\end{figure}\begin{figure}[htb]  \centering  \includegraphics[width=0.8\\linewidth]{Image00118.jpg}  \caption{图片 52: Image00118.jpg}\end{figure}\begin{figure}[htb]  \centering  \includegraphics[width=0.8\\linewidth]{Image00130.jpg}  \caption{图片 53: Image00130.jpg}\end{figure}\begin{figure}[htb]  \centering  \includegraphics[width=0.8\\linewidth]{Image00318.jpg}  \caption{图片 54: Image00318.jpg}\end{figure}\begin{figure}[htb]  \centering  \includegraphics[width=0.8\\linewidth]{Image00089.jpg}  \caption{图片 55: Image00089.jpg}\end{figure}\begin{figure}[htb]  \centering  \includegraphics[width=0.8\\linewidth]{Image00258.jpg}  \caption{图片 56: Image00258.jpg}\end{figure}\begin{figure}[htb]  \centering  \includegraphics[width=0.8\\linewidth]{Image00086.jpg}  \caption{图片 57: Image00086.jpg}\end{figure}\begin{figure}[htb]  \centering  \includegraphics[width=0.8\\linewidth]{Image00301.jpg}  \caption{图片 58: Image00301.jpg}\end{figure}\begin{figure}[htb]  \centering  \includegraphics[width=0.8\\linewidth]{Image00058.jpg}  \caption{图片 59: Image00058.jpg}\end{figure}\begin{figure}[htb]  \centering  \includegraphics[width=0.8\\linewidth]{Image00088.jpg}  \caption{图片 60: Image00088.jpg}\end{figure}\begin{figure}[htb]  \centering  \includegraphics[width=0.8\\linewidth]{Image00093.jpg}  \caption{图片 61: Image00093.jpg}\end{figure}\begin{figure}[htb]  \centering  \includegraphics[width=0.8\\linewidth]{Image00257.jpg}  \caption{图片 62: Image00257.jpg}\end{figure}\begin{figure}[htb]  \centering  \includegraphics[width=0.8\\linewidth]{Image00107.jpg}  \caption{图片 63: Image00107.jpg}\end{figure}\begin{figure}[htb]  \centering  \includegraphics[width=0.8\\linewidth]{Image00117.jpg}  \caption{图片 64: Image00117.jpg}\end{figure}\begin{figure}[htb]  \centering  \includegraphics[width=0.8\\linewidth]{Image00194.jpg}  \caption{图片 65: Image00194.jpg}\end{figure}\begin{figure}[htb]  \centering  \includegraphics[width=0.8\\linewidth]{Image00388.jpg}  \caption{图片 66: Image00388.jpg}\end{figure}\begin{figure}[htb]  \centering  \includegraphics[width=0.8\\linewidth]{Image00363.jpg}  \caption{图片 67: Image00363.jpg}\end{figure}\begin{figure}[htb]  \centering  \includegraphics[width=0.8\\linewidth]{Image00162.jpg}  \caption{图片 68: Image00162.jpg}\end{figure}\section{文件 8}\
%第一章 国外矿石机的历史回顾 三、另类矿石机(附两款火花无线电发射机) 这里所指的另类矿石机,是早期欧美用于通信方面的专业矿石机,以及矿石、电子管两用机。这足以证明矿石检波器在无线电发展的初级阶段,曾经担负着相当的重任。以下结合具体的机器分别予以介绍。 1.法国1913年出品的E·POFERL矿石机 首先介绍的是1913年法国里昂制造的E·POFERL矿石机(见图1-51)。该机系20世纪10年代无线电通信使用的接收机,主要接收火花型无线电发射机信号。机器的尺寸较大,底板为60cm×45cm,具有4组抽头式及滑动抽头式调谐线圈,密封罩式活动矿石检波器,以及防雷电装置。最为可贵的是,这台矿石接收机附有一张原机主早年与其他“火腿”通联时的器材照片(见图1-52),照片上此机的旁边摆放着同一时期的火花式无线电发射机。照片上的器材,应该是那一时期无线电爱好者(火腿)比较讲究的通信设备。当然,该机也具备接收连续波,即广播信号的功能,是一款标准的矿石接收机。 图1-51 法国里昂制造的E·POFERL矿石机 图1-52 法国里昂制造的E·POFERL矿石机照片(附发射机) 2.美国1917年出品的无线电火花发射机 这里有一架1917年美国出品的(AMRAD)无线电火花发射机(见图1-53)。它使用Amrad C型电火花点火线圈,Amrad 2834发送和接收开关。发射机工作电压为6~10V,电流为10A,可以发送莫尔斯电码。发射机尺寸约为55cm×23cm×30cm。这架发射机现在仍然可以稳定地发射无线电信号。 图1-53 1917年美国出品的(AMRAD)无线电火花发射机 3.美国BC-14A型军用矿石接收机 图1-54所示是一台1918年出品的美国一战时期的BC-14A型军用矿石接收机。这台机器很完整,状态也不错,我亲自测试过。用一根6m长的多股电线搭到树上,作为天线,然后将一把15cm长的螺丝刀插入土中,当作地线,稍微调整一下活动矿石的触点,便能清楚地收听到广播,而且声音相当清晰。BC-14A号称战壕晶体接收机,是美国一战时期大量生产,并准备广泛使用的军用无线电接收机,它会接收气球和飞机上小火花发射器发射的信号,在战壕中为炮兵指引射击的目标。 图1-54 1918年出品的美国一战时期的BC-14A型军用矿石接收机 图1-54 1918年出品的美国一战时期的BC-14A型军用矿石接收机(续) 这台接收机采用可变耦合线圈的双调谐电路,天线回路和检波回路都有抽头式线圈和可变电容器。带玻璃罩的“猫须检波器”(活动矿石)安装在面板右上角,旁边有检波器备用插口。此机有一个蜂鸣器,通过按钮驱动来生成一个信号,用于校准频率,因此要安装一节电池。整机的尺寸为32cm×22cm×21.5cm,它分量很重。其实,这就是一台电路异常复杂的矿石机。20世纪20年代的不少电子管收音机,电路甚至也没有如此复杂。 1918年的美国,电子管已经发明并广泛使用了,但由于当时的电子管机器稳定性差,电池体积大、容量低、充电不便等原因,在野外反而不如矿石机可靠,今日看起来未免有些可笑和不可思议。虽然,这台接收机的性能指标与现在的通信设备无法相比,但其工艺水准与材料的货真价实,绝对敢和今日的大多数机器一比高低。 BC-14A生产的时期,一战即将结束,因此多数机器并未派上用场,后来不少机器从库房直接流向旧货市场。2013年5月,EBAY上曾经出现过一台全品,带内外包装和原始操作手册的BC-14A,要价较高(现在看来并不高)。我当时刚上EBAY,犹豫了两天,痛失良机,后来此机价格一路攀升,至今令我懊恼不已。但我此番买到的BC-14A属于极低序列号—934号,应该是经历过实战考验的机器,算是十分难得的。之前我见到的BC-14A,序列号均在13万号以上。 我们再来对比一下那台大连制造的矿石机(见本书第二章《一台标明了型号、厂名、出厂日期和制作者的“日本”矿石机》),很明显,那就是一台不折不扣的军用接收机。BC-14A美军信号兵命名的型号是SCR-54,这与大连矿石收音机同时具有“GSR-375C受信机”和“GTR-375C通信机”型号的情况相似。BC-14A没有使用电子管,而是依靠矿石检波器接收信号。在那个年代,电子管技术尚不十分成熟,非但性能远不如后来,且体积大、耗电高、防震差,不大适应野外环境下的长时间工作。相反,矿石检波器无须电源、体积小、性能可靠,适合野外环境下使用。东北地区的侵华日军主要活动于山区,交通不便,距离有交流市电的地方较远,矿石检波的军机当然更加可靠和实用,最起码可以作为备用通信机,以备不时之需。BC-14A虽然生产的年代比大连矿石机早了20来年,但其工艺和材料比后者却要强不少,证明美国的科技和经济实力远超日本。但大连矿石机却也具有结构简单、体积小巧、便于携带的优点。 4.美国BC-15A军用火花发射机 BC-15A军用火花发射机(见图1-55)为美国康涅狄格电话\\&amp;电气有限公司出品,同时还有其他一些公司也在生产。BC-15A,1918年出品,被用于侦察飞机发射无线电信号,指挥地面炮兵部队射击。该机为木制机箱,其尺寸为16.2cm×11.5cm×20cm。 图1-55 康涅狄格电话\\&amp;电气有限公司出品的BC-15A军用火花发射机 5.英国MARK Ⅲ型军用矿石接收机 英国MARK Ⅲ型军用矿石接收机(见图1-56),是1915年由英国马可尼的无线电报公司开发的,1916—1918年多家公司参与生产。该机接收频率为100~700m,主要用于接收飞机侦察敌情后发布的莫尔斯无线电信号,然后指挥炮兵校正方位,提高炮击命中率。因大量在战壕中使用,该机也被称作战壕接收机。这种方式现在看起来很原始,但一战时期却发挥了巨大的作用。此机与美国的BC-14A一样,有一个蜂鸣器,通过按钮驱动来生成一个信号,以利于快速校准接收频率。MARK Ⅲ的尺寸为35cm×30cm×20cm,该机为木质机箱,外包绿色漆布,重约7.2kg。MARK Ⅲ比BC-14A的体积要大,分量也要重不少。从元器件的质量和制作工艺看,前者也明显优于后者。 图1-56 英国MARK Ⅲ型军用矿石接收机 图1-56 英国MARK Ⅲ型军用矿石接收机(续) 图1-56 英国MARK Ⅲ型军用矿石接收机(续) 6.英国B.T.H.一灯矿石收音机(Valve-Crystal Receiver) 20世纪20年代初期正是矿石收音机盛行,电子管收音机快速发展的时期,前者成本低,接收效果差;后者成本高,接收效果好。当时一般的收音机使用者对电子管本身及较高的运行成本是很在意的。为此,一些厂家给矿石收音机加上一级电子管放大,接收条件好时只用矿石检波器;后级电子管放大供接收条件不好时使用。图1-57所示的这台B.T.H.一灯矿石收音机,便是在B.T.H.C型高级双矿石检波矿石收音机的基础上制作出来的。本机1923年出品,体积为29cm×21.1cm×27.2cm。 图1-57 B.T.H.一灯矿石机 图1-57 B.T.H.一灯矿石机(续) 与此机情况相仿的还有一台,如图1-58所示,也是英国生产的。 图1-58 另一台英国制造的一灯矿石机 7.美国DeForest D-10矿石检波四管直流接收机 这台DeForest D-10矿石检波四管直流接收机(见图1-59),是美国纽约德福雷斯特无线电话和电报公司出品,采用4只DV3型三极管和一只活动矿石检波器。本机大概也是具有军用或其他野外收讯的功能,因为其机箱顶部有安插蛛网式移动天线的接口。很显然,本机与前述某些矿石或矿石电子管混装接收机一样,都还在充分利用矿石机不用电源,简便易用的长处。 图1-59 美国DeForest D-10矿石检波四管直流接收机 图1-59 美国DeForest D-10矿石检波四管直流接收机(续)\begin{figure}[htb]  \centering  \includegraphics[width=0.8\\linewidth]{Image00035.jpg}  \caption{图片 69: Image00035.jpg}\end{figure}\begin{figure}[htb]  \centering  \includegraphics[width=0.8\\linewidth]{Image00083.jpg}  \caption{图片 70: Image00083.jpg}\end{figure}\begin{figure}[htb]  \centering  \includegraphics[width=0.8\\linewidth]{Image00259.jpg}  \caption{图片 71: Image00259.jpg}\end{figure}\begin{figure}[htb]  \centering  \includegraphics[width=0.8\\linewidth]{Image00367.jpg}  \caption{图片 72: Image00367.jpg}\end{figure}\begin{figure}[htb]  \centering  \includegraphics[width=0.8\\linewidth]{Image00197.jpg}  \caption{图片 73: Image00197.jpg}\end{figure}\begin{figure}[htb]  \centering  \includegraphics[width=0.8\\linewidth]{Image00120.jpg}  \caption{图片 74: Image00120.jpg}\end{figure}\begin{figure}[htb]  \centering  \includegraphics[width=0.8\\linewidth]{Image00311.jpg}  \caption{图片 75: Image00311.jpg}\end{figure}\begin{figure}[htb]  \centering  \includegraphics[width=0.8\\linewidth]{Image00188.jpg}  \caption{图片 76: Image00188.jpg}\end{figure}\begin{figure}[htb]  \centering  \includegraphics[width=0.8\\linewidth]{Image00134.jpg}  \caption{图片 77: Image00134.jpg}\end{figure}\begin{figure}[htb]  \centering  \includegraphics[width=0.8\\linewidth]{Image00100.jpg}  \caption{图片 78: Image00100.jpg}\end{figure}\begin{figure}[htb]  \centering  \includegraphics[width=0.8\\linewidth]{Image00147.jpg}  \caption{图片 79: Image00147.jpg}\end{figure}\begin{figure}[htb]  \centering  \includegraphics[width=0.8\\linewidth]{Image00384.jpg}  \caption{图片 80: Image00384.jpg}\end{figure}\begin{figure}[htb]  \centering  \includegraphics[width=0.8\\linewidth]{Image00167.jpg}  \caption{图片 81: Image00167.jpg}\end{figure}\begin{figure}[htb]  \centering  \includegraphics[width=0.8\\linewidth]{Image00345.jpg}  \caption{图片 82: Image00345.jpg}\end{figure}\section{文件 9}\
%第一章 国外矿石机的历史回顾 四、20世纪40年代后的传统矿石收音机 1945年第二次世界大战结束之后的一段时间,包括德国、前苏联在内的欧洲国家经济凋敝,民众贫困,矿石收音机出现一个小规模的回潮,诸如罗兰士、德律风根、蓝宝等品牌收音机制造商,都推出了小型廉价的矿石收音机,以供穷人收听广播之用。1950年后逐渐用晶体二极管取代了矿石。下面就介绍几款。 1.德国1946年出品的德律风根袖珍矿石收音机 1946年德国出品的德律风根袖珍矿石收音机(见图1-60),采用小型蜂房线圈,整机尺寸为10.5cm×7.2cm×3.21cm。 图1-60 德律风根袖珍矿石收音机 2.德国1946年出品的蓝宝矿石收音机 1946年德国出品的蓝宝矿石收音机(见图1-61),此机采用木质机盒,造型简洁,可以挂在墙上使用。 图1-61 蓝宝矿石收音机 图1-61 蓝宝矿石收音机(续) 3.20世纪40年代美国出品的铁壳袖珍自带扬声器矿石机 铁壳袖珍自带扬声器矿石机(见图1-62),是1950年左右美国的产品,该机尺寸为8cm×5.2cm×2.4cm。黑色铁制机壳,线圈滑动抽头调感,机内装有一个高灵敏度舌簧(耳机)扬声器。接上天地线便可清晰的收音。 图1-62 铁壳袖珍自带扬声器矿石机 4.德国20世纪50年代初出品的欧米茄袖珍矿石收音机 德国20世纪50年代初出品的欧米茄袖珍矿石机,采用塑料机盒,精致美观(见图1-63)。此时矿石机的实用性逐渐让位于怀旧与玩赏。 图1-63 欧米茄袖珍矿石机 5.德国20世纪50年代出品的德律风根D3型矿石收音机 德国20世纪50年代出品的德律风根D3型矿石收音机,此机制作工艺非常精良,可以挂在墙上使用(见图1-64)。 图1-64 德律风根D3型矿石收音机 6.20世纪50年代后期Miniman袖珍矿石机 Miniman袖珍矿石机,20世纪50年代后期日本出品(见图1-65)。在20世纪50年代和60年代,日本大力发展半导体收音机,受此影响也出品了很多模仿晶体管机造型的袖珍矿石机。此机的尺寸为6.5cm×4.8 cm×2.6cm,非常小巧,但在如此狭窄的空间中,却安装了一个普通尺寸大小的空气可变电容器,这对保证矿石机的接收质量大有好处,而且调谐手感也远比其他类型要好。本机的线圈为磁性线圈,配有二极管检波、双耳塞机,而且配有一个乳白色塑料机套。 图1-65 Miniman袖珍矿石机 7.1958年美国Hearever公司出品的Hearever火箭模型矿石机 1958年,美国Hearever公司出品了一款Hearever火箭模型矿石机(见图1-66)。此款造型矿石机的出现,与20世纪50年代美国的航天航空事业大发展有关。机器整体长9.5cm,“火箭”尖头可以伸缩,带动磁芯在线圈中移动,以调谐频率。此款类型的矿石机日本也曾大量生产。 图1-66 Hearever火箭模型矿石机 8.20世纪50年代后期日本出品 的EM-TONE型矿石机 20世纪50年代后期,日本出品了一款EM-TONE型矿石机(见图1-67)。机器体积为8cm×5.3cm×2cm,采用二极管检波,线圈内磁芯移动调谐频率。 图1-67 EM-TONE型矿石机 9.20世纪50年代后期日本出品的LARK CRYSTAL 矿石机 20世纪50年代后期,日本制造了一款LARK CRYSTAL 矿石机(见图1-68),整机体积为10.7cm×6cm×2cm。此机的构造比较特殊,磁性线圈为蜂房式分段绕制,橡胶轮摩擦带动磁棒移动,以调谐频率。而且该机没有像同时期的多数矿石机用二极管检波,它依然采用了老式的固定矿石检波器,很有些复古的味道。 图1-68 LARK CRYSTAL 矿石机 10.20世纪50年代初美国出品的 Philmore VC-1000矿石机套件 20世纪50年代初期,美国出品了一款Philmore VC-1000矿石机套件(见图1-69)。Philmore无线电公司制造矿石收音机有着悠久的历史,远在20世纪20年代就出品了不少型号的机器,其中一款于20世纪30年代初被引进中国,就是著名的“亚美1001”。此款矿石机为开放型板式机,底板尺寸16.5cm×11.5cm。 图1-69 Philmore VC-1000矿石机 11.1958年美国希斯有限公司出品的Crystal Receiver CR-1型矿石机套件 1958年美国希斯有限公司出品了Crystal Receiver CR-1型矿石机套件(见图1-70)。此机堪称美国新时期(20世纪50年代)厂制矿石机(或套件)的终结者,无论电路的复杂、元器件的正规,均是那个时期其他矿石机不可攀比的。CR-1的机箱为黑色电器胶木,面板的材质是铝板。旋钮、接线柱,还有整个面板的布局、白色文字刻度标记,散发着强烈的现代气息。内部的双空气单连、磁性线圈、转换开关等,都是用料扎实、做工完美的产物。 图1-70 Crystal Receiver CR-1型矿石机套件\begin{figure}[htb]  \centering  \includegraphics[width=0.8\\linewidth]{Image00373.jpg}  \caption{图片 83: Image00373.jpg}\end{figure}\begin{figure}[htb]  \centering  \includegraphics[width=0.8\\linewidth]{Image00199.jpg}  \caption{图片 84: Image00199.jpg}\end{figure}\begin{figure}[htb]  \centering  \includegraphics[width=0.8\\linewidth]{Image00390.jpg}  \caption{图片 85: Image00390.jpg}\end{figure}\begin{figure}[htb]  \centering  \includegraphics[width=0.8\\linewidth]{Image00171.jpg}  \caption{图片 86: Image00171.jpg}\end{figure}\begin{figure}[htb]  \centering  \includegraphics[width=0.8\\linewidth]{Image00050.jpg}  \caption{图片 87: Image00050.jpg}\end{figure}\begin{figure}[htb]  \centering  \includegraphics[width=0.8\\linewidth]{Image00137.jpg}  \caption{图片 88: Image00137.jpg}\end{figure}\begin{figure}[htb]  \centering  \includegraphics[width=0.8\\linewidth]{Image00094.jpg}  \caption{图片 89: Image00094.jpg}\end{figure}\begin{figure}[htb]  \centering  \includegraphics[width=0.8\\linewidth]{Image00164.jpg}  \caption{图片 90: Image00164.jpg}\end{figure}\begin{figure}[htb]  \centering  \includegraphics[width=0.8\\linewidth]{Image00066.jpg}  \caption{图片 91: Image00066.jpg}\end{figure}\begin{figure}[htb]  \centering  \includegraphics[width=0.8\\linewidth]{Image00040.jpg}  \caption{图片 92: Image00040.jpg}\end{figure}\begin{figure}[htb]  \centering  \includegraphics[width=0.8\\linewidth]{Image00358.jpg}  \caption{图片 93: Image00358.jpg}\end{figure}\begin{figure}[htb]  \centering  \includegraphics[width=0.8\\linewidth]{Image00008.jpg}  \caption{图片 94: Image00008.jpg}\end{figure}\section{文件 10}\
%第一章 国外矿石机的历史回顾 五、前苏联“共青团员”牌矿石机 1947年至20世纪50年代初期,前苏联生产了一款共青团员(Комсомолец)牌的矿石机(见图1-71)。因政治及历史原因,前苏联的收音机对中国1949—1960年期间的收音机设计制造,产生了很大影响。而且中国还进口了数量众多的苏制收音机,如著名的“莫斯科人”,几乎家喻户晓。“共青团员”在国内也有着很高的知名度,是当年留苏人员回国时非常乐于携带的礼品之一。前苏联的收音机可以说是“半个”国产机,因此特设立专门章节介绍共青团员牌的矿石收音机。“共青团员”黑色电木机壳,尺寸为18cm×4.2cm×9cm。该机调谐电路采用磁芯电感式,即在线圈中插入一圆棒状磁芯,转动调节旋钮时带动磁芯在线圈内活动,以改变电感量。这种调谐方式在20世纪40年代后期至50年代比较流行,其好处是结构简单,能够有效降低机器的成本。正因为该款矿石收音机物美价廉,深受百姓欢迎,故前苏联在短短几年内便生产了上百万台。 “共青团员”的矿石检波器比较特殊,是插拔式的,可以方便地取下来,因此也容易丢失。好在不少活动矿石的安装尺寸与其相同,因此能够直接替换。 图1-71 1947年至20世纪50年代初期前苏联生产的共青团员(Комсомолец)牌的矿石机 图1-71 1947年至20世纪50年代初期前苏联生产的共青团员(Комсомолец)牌的矿石机(续)\begin{figure}[htb]  \centering  \includegraphics[width=0.8\linewidth]{Image00080.jpg}  \caption{图片 95: Image00080.jpg}\end{figure}\begin{figure}[htb]  \centering  \includegraphics[width=0.8\linewidth]{Image00191.jpg}  \caption{图片 96: Image00191.jpg}\end{figure}\section{文件 11}\
%第二章 国产矿石机的历史回顾 第二章 国产矿石机的历史回顾 一、民国时期的矿石收音机 1923年1月,美国人奥斯邦利用旅日华侨的资金,在上海建立了中国第一家广播电台,并借机开始推销收音机。根据文献记载,当时上海地区约有500架收音机,全部是洋货。此后又有永安、新孚、开洛数家电台开张,其目的均不离销售收音机。如1924年5月14日《申报》发布开洛电台《申报》馆分台即将播音的消息时,便大做收音机的广告: “本馆自明日起,用无线电话报告新闻,每日两次,上午九时四十五分至十时一刻,晚间七时至八时三十分。所报告者,上午为汇兑市价、钱庄兑现价格、小菜上市等;晚间为重要新闻及百代公司留声机新片。凡本外埠各处装有无线电收音机者,均可听得本馆之报告。 “本馆此项无线电话机,系由江西路六十二号开洛公司装置,异常灵捷便利,发音亦复清晰。该公司制有最新式大号收音机,形如电动留声机,所用检波器及两级增波器之全副机器,可以接收远近所发大小无线电之波浪。又有J字五百零二号之收音机,亦有检波及两级增波器之全副机件。据美国报告,以上二种收音机可接到三千英里(约中国九千里)以外之电浪。另有一种最小最廉之晶石(矿石)收音机,可接收二十至四十英里(即六十至一百二十华里)所发之电浪,价格不过三四十元。” 20世纪20年代中期,中国主要的广播电台及收音机均在上海。当时有美、英、法、德、日诸国收音机进口,美国货占据份额最多。一些洋行贪图中国劳力便宜,便进口散件进行组装。后来甚至委托中国的无线电作坊生产不重要的配件。一些不法商人则采用低廉的元器件组装,然后冠以洋行的字号卖高价。收音机的形式,洋人基本用电子管收音机,但电路以再生式为主,超外差机极少;国人因财力有限,多数喜用矿石收音机。矿石收音机虽然构造简单,价格低廉,但一直到20世纪30年代,市场上出售的成品机仍然以利用进口元器件组装为主。据文献记载,1933年,上海的亚美无线电公司推出了亚美1001号矿石收音机(见图2-1)。1001号矿石收音机是亚美公司的第一台矿石收音机,但同样是用美国菲尔默(philmore)矿石收音机套件组装的,区别只是在机器面板上增加了“亚美”二字的英文标记。 图2-1 1933年上海亚美无线电公司出品的亚美1001号矿石收音机广告 我的好友,上海著名老收音机收藏家张明律(网名“张无线电”)先生收藏了一台亚美牌矿石收音机。以下是他对这台矿石收音机收藏经过及机器性能等方面的描述。 1.一台我国20世纪30年代的矿石收音机 张明律收藏并撰文 这台矿石收音机我是在1995年从一个地摊上收来的(见图2-2)。由于我当时初涉旧货市场,又有点识货,看到这台矿石收音机立刻表现出爱不释手的样子。卖家肯定是个老手,看到我的表情知道赚钱的机会来了,开口要价1000元,一点还价的余地也没有,搞了一个多小时,一分钱也没有还掉,硬是花了1000元买下了它。当时一台品相很好的红灯711只要5~10元就可以买到了。随着时间的流逝,这类机器越来越难见到,现在想想还是很值的。在我的收藏品中,它已经归入珍稀品种的范围了。 情况介绍: 一、外壳材料:木质,外表以大红漆覆盖。 二、外形尺寸:宽28cm×高26cm×厚20cm,分上下两部分。 上半部分:高15cm,放置矿石收音机。 下半部分:高11cm,有一个小抽屉,放置耳机。 三、矿石类型:固定矿石,正向电阻2.2kΩ,反向电阻75kΩ,还可以使用。 它的标贴是: 意思是M商标的调幅固定晶体检波器,专利号为105848。 固定矿石中间都有一个小孔用来调节触针,而我的这个没有。 四、电路类型:三回路,线圈间距可调。 五、线圈情况: 六、可变电容器:28~365pF 七、云母固定电容器:2000pF 八、耳机:3000Ω 九、收听情况: 天线:220V市电的地线 地线:镀锌白铁自来水管 由于上海交通广播电台发射功率特别强大,所以在它广播时,只能收到它的节目,输出电流达30mA,只有在它停止广播时才能够收到其他几个台的广播。 矿石收音机的收听效果与天地线的质量有密切的关系,天地线的质量不同,收听效果相差悬殊,因此无法讨论评价。 图2-2 20世纪30年代的亚美矿石机 图2-2 20世纪30年代的亚美矿石机(续) 图2-2 20世纪30年代的亚美矿石机(续) 张先生的这台亚美矿石机,应该是亚美公司的产品,但那块固定矿石肯定是美国货。机器的结构、线圈,包括机箱,也有明显的“洋机”色彩。亚美公司20世纪30年代发行的《中国无线电》杂志,经常刊登该公司的矿石机及元器件的销售广告,其中就有国产与进口之分,这证明亚美始终在进口国外(主要是美国)的收音机散件。 2.一台天津无线电行组装的矿石机 当时不仅是在上海,京津一带也有不少小厂家、作坊、电料行采用全部或部分进口件组装矿石收音机。我收藏了一台20世纪30年代天津地区用进口元器件组装的矿石机(见图2-3),机芯除了线圈不好断定“出身”外,其余全部是洋货,耳机是德国制造,蓝花布面、碎花贴纸内衬的硬纸板机箱应该是本地自制或日本货。 图2-3 20世纪30年代天津地区用进口元件组装的矿石收音机 图2-3 20世纪30年代天津地区用进口元件组装的矿石收音机(续) 民国时期日本在中国占领地区生产了大量收音机,过去只知道有“华北标准型”“满洲标准型”等系列电子管收音机,其实还生产了矿石收音机。我就收到了一台北京朋友家藏的“华北标准型”矿石收音机。此机名牌上标称的“华北标准型矿石受信机”,“昭和16年1月制作”。昭和16年系公历1941年,正是我国抗日战争的艰苦时期。这台矿石收音机的信息得以保存下来,与那场抵御日本侵略的战争有着密切的关系。下面就把该机原收藏者在网上发表的介绍文字登录如下: 3.一台抗战时期八路军缴获的“华北标准型矿石受信机” “这是1941年,家父在八路军冀鲁豫军区四分区与日军一次作战时缴获的。‘文革’期间,本人酷爱收音机,无事可干,以组装收音机为乐。在家中发现了这台矿石收音机,经常戴着大耳机收听,并且改动了里面的线圈,做了几个抽头,用分线器收听时居然听到了**!十几岁的我如醉痴迷,一听就是几个钟头,旁若无人。正好赶上家父下班,不知我在听什么,抢过耳机听到了**的广播。当时他勃然大怒!要知道那时偷听敌台是要被判刑的!父亲当时是某部队的领导,正受批斗,领章帽徽曾被撕掉,正没处撒气呢,不由分说打开机子,把里面的所有线圈及电路全部拔掉扔了,还大骂了我一顿。因此就留了个空壳。后全家被发配到“五七干校”劳动改造,但这个矿石收音机盒一直没舍得扔,家父在1988年过世,直到现在我一直收藏着,唯一惋惜的是,仅剩了个空壳! 这台矿石收音机的接收频率是可以改变的,接上高质量的天地线,可以收到短波。如果你在北京,可以约一个地方让你看看这个矿石收音机。当时家父一怒之下,并没有慢慢拆机,而是全部拔光。还可以让你看看我几十年前装的收音机,还能响,一根线也没脱焊。我已是57岁的人,没必要说假话!如果有机会,可以到部队干休所问问。” 收到这台华北标准矿石收音机(见图2-4)后,我仔细研究了一番,觉得这台机器虽然仅剩下空壳,但却留下了一条非常重要的信息:日本侵华期间在中国生产过矿石收音机,而且在军队中使用过。回顾早期的欧美无线电通信发展史,美国、英国在1920年前,电子管收音机已经诞生,它们仍旧制造了大量军用矿石收音机,或电子管、矿石两用收音机。究其原因,主要是那一时期的电子管机不仅笨重,质量也不稳定,野外使用还受电源的限制,可靠性较差。而矿石收音机体积小、重量轻,不用电源,作为野战时期的通信器材反而比较可靠。该机机箱为木制,面板为电木板,尺寸为17.5cm×12cm×9.8cm。 无独有偶,同电子管分“华北标准型”与“满洲标准型”一样,抗战时期的东北地区也有“满洲”标准型的矿石收音机。我就收藏了一台“昭和16年10月(1941年)”,在东北大连地区制造的“GSR-375C型(矿石)受信机”(见图2-5)。简要介绍如下。 图2-4 华北标准矿石收音机 4.一台标明了型号、厂名、出厂日期和制作者的“日本”矿石机 这是一台生产于20世纪40年代初,机上标有型号、编号、厂名、厂址、出厂日期、制作者,并且品相相当完美的矿石收音机。机箱尺寸:高11.7cm,宽22.7cm,厚14.6cm。除了身份信息异常翔实外,此机还有以下特点: 1.这台矿石收音机问世已经68年,但品相依旧很好,金属器件毫无锈蚀,木制机箱亦无朽损。究其原因,大致有三:一是东北地区气候干燥,二是原机主爱惜呵护有加,三是机器用料非常考究。 2.此机的身世不同寻常。最早,我将它界定为“日本矿石收音机”。因为它的面板上有很多日文,厂名也是什么“株式会社”。但认真思考之后,我觉得它就是一台地地道道的国产矿石机。首先,它是在中国大连生产的;其次,制作者是中国人王承瑞。虽然有许多元器件是进口货,但这在民国时期很常见。例如亚美公司生产的矿石机,就有国产元器件与进口元器件之分。当然,从这台矿石收音机也可看出,当年日本企图吞并中国的狼子野心。当时日伪生产的一些收音机,度盘上的地图已将东三省划归日本;这台矿石收音机,居然把大连的地名日本化,实在可气又可笑。 3.此机面板上标明型号为“GSR-375C受信机”和“GTR-375C通信机”。受信机系日本人当时对收音机的称谓,但通信机却不知所指。难道是军用?这从该机的构造看,似乎也靠谱。首先,它有3个耳机插口,其中一个还是6.5mm的单插,军机上多使用此类耳机插口;其次,此机还有两个金属提手,即面板上标明“移动用”的装置;再次,此机调谐旋钮是那个小钮,即所谓“转换器”,带动大钮,能够起到减速作用,调谐更精细。上述3点都是一般民用矿石机不具备的。另外,机器面板上注明了制作人、厂家和生产的年、月、日,显然是要明确责任,以便日后出现问题进行追究,可督促制作者保证质量,恐怕也是军用品的特色吧。 图2-5 1941年大连地区制造的GSR-375C型矿石机 图2-5 1941年大连地区制造的GSR-375C型矿石机(续) 图2-5 1941年大连地区制造的GSR-375C型矿石机(续) 图2-5 1941年大连地区制造的GSR-375C型矿石机(续) 图2-5 1941年大连地区制造的GSR-375C型矿石机(续)\begin{figure}[htb]  \centering  \includegraphics[width=0.8\\linewidth]{Image00398.jpg}  \caption{图片 97: Image00398.jpg}\end{figure}\begin{figure}[htb]  \centering  \includegraphics[width=0.8\\linewidth]{Image00217.jpg}  \caption{图片 98: Image00217.jpg}\end{figure}\begin{figure}[htb]  \centering  \includegraphics[width=0.8\\linewidth]{Image00123.jpg}  \caption{图片 99: Image00123.jpg}\end{figure}\begin{figure}[htb]  \centering  \includegraphics[width=0.8\\linewidth]{Image00347.jpg}  \caption{图片 100: Image00347.jpg}\end{figure}\begin{figure}[htb]  \centering  \includegraphics[width=0.8\\linewidth]{Image00175.jpg}  \caption{图片 101: Image00175.jpg}\end{figure}\begin{figure}[htb]  \centering  \includegraphics[width=0.8\\linewidth]{Image00031.jpg}  \caption{图片 102: Image00031.jpg}\end{figure}\begin{figure}[htb]  \centering  \includegraphics[width=0.8\\linewidth]{Image00395.jpg}  \caption{图片 103: Image00395.jpg}\end{figure}\begin{figure}[htb]  \centering  \includegraphics[width=0.8\\linewidth]{Image00378.jpg}  \caption{图片 104: Image00378.jpg}\end{figure}\begin{figure}[htb]  \centering  \includegraphics[width=0.8\\linewidth]{Image00062.jpg}  \caption{图片 105: Image00062.jpg}\end{figure}\begin{figure}[htb]  \centering  \includegraphics[width=0.8\\linewidth]{Image00250.jpg}  \caption{图片 106: Image00250.jpg}\end{figure}\begin{figure}[htb]  \centering  \includegraphics[width=0.8\\linewidth]{Image00176.jpg}  \caption{图片 107: Image00176.jpg}\end{figure}\begin{figure}[htb]  \centering  \includegraphics[width=0.8\\linewidth]{Image00228.jpg}  \caption{图片 108: Image00228.jpg}\end{figure}\begin{figure}[htb]  \centering  \includegraphics[width=0.8\\linewidth]{Image00157.jpg}  \caption{图片 109: Image00157.jpg}\end{figure}\begin{figure}[htb]  \centering  \includegraphics[width=0.8\\linewidth]{Image00078.jpg}  \caption{图片 110: Image00078.jpg}\end{figure}\section{文件 12}\
%第二章 国产矿石机的历史回顾 二、新中国成立后的矿石收音机 新中国成立之初,百废待兴,经济、科学技术等方面的发展还很落后,西方经济发达国家已经淘汰了的矿石收音机,在我国仍然有着广阔的市场。无数青少年无线电爱好者接触的第一本书就是制作矿石机的书,组装的第一架收音机也是矿石机。而且,矿石机在部分城市和大部分农村,依然担负着家庭收听广播的重任。直到20世纪60年代,许多百货商场的电器电料柜台还在出售矿石收音机的整机和元器件。 我对矿石机的第一印象,始于1960年的夏天,我小学一年级的第二学期,我家隔壁的叔叔买了一架矿石机,我去他家串门时看见他正在摆弄那台矿石机。那是一台成品矿石机,叔叔从暖气管上接过来一根电线,连接在矿石机上,头上戴着耳机,聚精会神地听着,时不时地还拧一拧机箱上的小旋钮。叔叔见我好奇,摘下头上的耳机递给我,让我听。耳机中传来清晰的音乐声,我喜欢极了。但是好景不长,不知什么原因,几天之后叔叔就把矿石机大卸八块给拆了,各种零件扔了一桌子。从此,我经常去他家玩那些零件。印象最深的是那个空气可变电容器(当时不知道是什么),用手一拧那个轴棍,就转出来一排银色的金属片。 4年后,我上小学五年级的时候,开始接触无线电,最早的活动就是装一架矿石机。这时的国家经济已经好转许多,收音机使用的元器件也比4年前进步了。例如,二极管已经很普及,不少人装矿石收音机时已经不用矿石了。但我毕竟还是一个小学生,要靠家长的“施舍”才能买零件。因为父母不支持,手中的钱实在有限,隔壁叔叔那台矿石机在我的心目中,简直就是遥不可及的奢侈品。我清楚地记得,按照书上的介绍,当时装一架标准的双回路矿石机,需要的零件和价格如下:空气可变电容器1元4角8分,双回路线圈5角6分,4个接线柱5角6分,二极管8角2分,分线器1角6分,2个旋钮1角6分,耳机最便宜的1元6角,总计要5元3角4分。对我来说这无疑是个天价!幸好找到了与我情况差不多的同班同学马小沛,我们决定一起凑钱装一台矿石机。我负责买二极管、可变电容器和线圈,其余的由马小沛买。但是,和父亲软磨硬泡了许久,他也只答应给我两元钱。1个可变电容器就要1元4角8分钱,要买那么多零件两元钱无论如何是不够的。有一天事情终于发生了转机,我在距家很远的一家商场闲逛时,发现二楼电料柜台卖一种体积很小的云母介质可变电容器,1个才5角钱。当时我真的喜出望外,我要买的零件两元钱够了。 我们的矿石机终于做成功了,天线是一根六七米长的花线,从二楼的窗户搭到一棵大树上,地线接到暖气管上。矿石机很顺利地就从耳机中发出了广播节目的声音,但是声音不大。不久后,我在安定门附近发现了一家废旧物资商店,里面有大量的库存老元件,价格便宜得出奇。我花1角钱买了10个2000pF的电容。回家后把其中一个接到了耳机的两个接口上,耳机中的声音立刻提高了好几倍。我当时真的是欣喜若狂,那种兴奋和幸福的感觉,至今难忘。 我还用电灯拉线开关做过一个最简单的小矿石机,其实所有元器件就是一个二极管和耳机。我把它带到学校显摆,接了一根地线在教室的暖气管上,就清楚地听到了广播。在同学们争抢着收听时,被班主任郭老师发现了。郭老师本想狠狠地批评我一顿,但当她拿起耳机好奇地放到耳朵上时,却睁大了眼睛,脸上露出了笑容。不但没有批评我,还把我好好地夸奖了一番,让同学们向我学习。非常遗憾的是,我自己当年装的矿石机早已不知所踪,好在找到了另外一架非常相似的,如图2-6所示,它的装配者也是一位十一二岁的少年,他也住在和平里,矿石机的原件同样购自和平里百货商店。当然,他比我幸运,用的是一只空气可变电容器。矿石机用的是邮寄物品的一个小木箱,电路是20世纪60年代中期,北京城区标准的电路配置:双回路抽头线圈配合可变电容器和分线器调谐、二极管检波。这台矿石机中我印象最深刻的是那个纸筒双回路线圈,它的价格为5角6分钱。线圈表面灌注了一层薄薄的石蜡,发出一阵阵特殊的清香,那个味道深深地印在我的记忆之中,至今不能忘记。 图2-6 20世纪60年代中期少年无线电爱好者制造的“标准”矿石机 凭我的亲身经历和当时的相关书籍所见,中国20世纪50至60年代中期,出于经济落后、贫穷,以及青少年好奇、求知欲强等原因,民间DIY矿石收音机的活动,不但普及率很高,而且非常原始,几乎所有的元件都可以自制。比如最关键的部件——矿石,一种方法是去药店买自然铜,或者到大煤块中去寻找(找到后,砸碎煤块就可取出矿石)。我上小学五年级时就去宿舍大院的煤堆中找到过不少成色很好的自然铜。再有一种更原始,却很锻炼人的方法,就是用铅末混合硫磺烧制方铅矿。这件事情我小时候也做过,我们是去找电线竿子上的绝缘瓷瓶,敲碎后取出中间的硫磺。自制矿石机元器件印象较深的还有可变电容器,我是按照1965年某期《我们爱科学》杂志里,北京黄城根小学提供的方法制作的。使用的主要材料好像是罐头皮加上牛皮纸。其中制作难度最大的是耳机,磁铁是用一小段锯条,先退火,折成U字形,然后用漆包线绕上充磁线圈,通过30多厘米长的极细铜丝(多股电线中抽出1根),连接上220V电源,通电后的瞬间,细铜丝熔断,即充磁成功。我做这个实验是在一个中午,父亲正在午休,我把上述装置通过房间里的灯头连接上220V电源。我拉动灯绳开关接通电源后,只听“嘭”的一声巨响,这一声不但把我吓了一跳,还把睡梦中的父亲惊醒,从床上一下跳到了地上。但父亲看明白怎么回事后,并没有责怪我,我当时心中既自责,又对父亲充满了感激之情。 虽然中国经济落后,矿石机的使用一直延续到20世纪60年代,比欧美发达国家足足晚了30年。但正因为如此,新中国成立后矿石机成品的主流产品起点还是较高的,而且无论品牌、产地有何区别,机器的电路与结构均大同小异。市场上最常见的矿石机通常是掀盖木制机箱,包一棕色或红色漆布,内贴碎花纸,面板多数是木板包漆布(少数不包),胶木面板极少。电路为双回路纸筒多抽头线圈,配合分线器与空气单连可变电容器组成调谐电路。面板上为一个活动矿石,高级一点的增加一枚固定矿石。面板上一般有6个接线柱,分别是天、地线各一个,耳机两副共4个;讲究的为7个接线柱,多出的一个是“天线2”,它与“天线1”分别接收强弱不同的信号。更讲究的机器,天地线之间有一锯齿形避雷器。上述矿石机的分线器(见图2-7),由10个左右的单个螺钉排列成半圆形,固定在面板上,中间一个活动转轴带动“动片”。这种分线器继承了欧美早期矿石机的结构,具有悠久的历史,它的优点是坚固耐用。同一时期单独出售的分线器,却是整体做在一小块圆形胶木板上,质量要差一些,但便于业余爱好者安装。早期成品矿石机的可变电容器,与后来单独出售的形状也不一样,前者更显古朴,但质量不如后来的产品。产品机使用的活动矿石和固定矿石与零售的也不同,后者要小一些,单薄一些。 以下是几种新中国成立后市面上出售的矿石机。 1.20世纪50年代初天津天华工业社出品的矿石收音机 这台矿石机(见图2-7)背板上有一标牌,上写“天华工业社,天津和平路X德福鲜货庄后吴家胡同十二号,公用电话二X八四一号”。X者,为辨认不出来的字。看样子,当年这家X德福鲜货庄一定非常有名,以至天华工业社要借用他的名号来推介自己。厂子很小,没有自己的电话。这倒不新鲜,我小的时候,北京的大街小巷和胡同里面,不但居民,不少小作坊和店铺,也都靠公用电话来对外联系。 机壳有一个喇叭窗,金属喇叭网,这个喇叭窗纯属装饰(别忘了,636单管机还有个金属喇叭窗呢),因为机壳内部长不过18cm,高不过13 cm,按照20世纪50年代中前期舌簧喇叭的实物看,还没发现有那么小的。从机壳内部来看,确实也没安装过喇叭。矿石机的单联、分线器、线圈、矿石、接线柱,都是地道的20世纪50年代早期的老货。整机从机壳到元件,布满了自然形成的包浆。比较令人生疑的微调空气单联,似乎是电子管机的专利。但仔细翻阅一些老矿石机的书发现,也有给单联加个微调的。此机的整体布局有些蠢笨,但那个年代小作坊的产品,也就是那个水平了。 图2-7 天津天华工业社出品的矿石收音机 图2-7 天津天华工业社出品的矿石收音机(续) 2.“牡丹”矿石机 目前存世的国产矿石收音机已经不多,品相完整的更是凤毛麟角。近日见网上介绍一台老“牡丹”矿石机,觉着不大对头。翻看北京无线电厂(牡丹厂)厂史《走过五十年》,果然不对。突然想起当年我曾买过一台矿石收音机,很有可能是“牡丹”的正牌货。赶忙来到库房,翻箱倒柜一个多小时,终于找了出来(见图2-8)。仔细对照厂史,哈哈,还真是牡丹!而且比牡丹厂的那台“还牡丹”,因为书中图片上的矿石机已经残破,矿石都没有了,单连的旋钮也有可能不是原配。而我这台,只是缺了两个接线柱,我找两个旧的接线柱拧上了,单连旋钮缺了塑料帽,但没有根据的情况下先这样吧。 机器的外壳包的是棕色漆布,里面粘的碎花格油纸,机箱尺寸为17.5cm×19cm×12cm。我找遍机箱内外,没有任何能够说明机器生产厂家和型号的文字。该机机芯曾被加过一级三极管低放,被我拆除了,还好没有被去掉什么东西。在矿石机流行的后期,不少使用者不满足于矿石机灵敏度低、音量小,喜欢自己加上一级低频放大,电子管、晶体管的都有。原耳机我没找到,当初是有的。 图2-8 “牡丹”矿石机 图2-8 “牡丹”矿石机(续) 我为何认为这台是牡丹厂的矿石机呢?这与它的来历有关系。那是1994年,我经常逛地坛职工旧货市场,还经常从一位中年人的摊位上买东西。他家住和平里四区,东西全是自家的,老收音机零件非常多。一天我问他,家里有矿石收音机吗?他不假思索地说:“那玩意你还要,根本不能听了。”我觉得有戏,忙说:“要,你有吗?”他叹口气说:“咳,家里还真有一份,老父亲当年买的,我下礼拜给你带来。”第二个周日,他果然给我拿来了,要了我10块钱。实话说,当时我还觉得贵了,那时矿石机不像现在这样值钱。他接着介绍那台矿石机,原来是他父亲1958年在和平里百货公司买的。牡丹厂史记载该机是1956年开始生产的,牡丹厂旧址在人定湖附近,距离和平里不远,这不就对上号了嘛。 3.与“牡丹”矿石机近似的“上海造” 西安的朋友郑志先生有一台矿石机(见图2-9),与我这台牡丹机格局基本相同,就连活动矿石都是“上海致美”出品的,然而线圈却是“天津市第一五金电器合作社”制造,可变电容器也较少见,刻度盘有时代特色。又于机箱前脸下方标注“一九五〇年制于上海”,并且是繁体字。当年的矿石收音机其实大多由作坊和小厂子组装而成,元器件用得很杂,但用料和做工是很讲究的。 图2-9 “一九五〇年制于上海”的矿石机 图2-9 “一九五〇年制于上海”的矿石机(续) 4.提手上铸有完税标记的矿石机 我还收藏了一台与牡丹矿石机非常相似的矿石机(见图2-10),两机的机箱尺寸完全一样,面板上也是7只接线柱,甚至机内的单连可变电容器都一样。不同的是面板上的布局,此机在天地线之间安装了锯齿形避雷器。上述情况证明了新中国成立后,我国市场上出售的主流成品矿石机确实是大同小异,电路和结构基本相同。 这台矿石机是红色漆布木制外壳,最有意思的是铁制提手上有模压铸造的“已纳使用牌照税”7个字。 20世纪50年代的收音机,很多都在机箱底部贴有纸制的完税证明,但在矿石机的提手上直接铸上完税标记的还不多见。 图2-10 提手上铸有完税标记的矿石机 图2-10 提手上铸有完税标记的矿石机(续) 5.带提把的蓝色老矿石收音机 这是一台20世纪50年代早期的矿石收音机(见图2-11)。采用双回路线圈,但不同于其他双回路矿石机的是,两个回路的线圈是分体的,一大一小。箱体未包裹漆布,而是应用的打腻子再刷漆的老式油工工艺。此机与其他很多矿石机一样,安装了一个金属提把,便于移动使用。 图2-11 带提把的蓝色老矿石收音机 6.一台成品矿石机的机芯 我还保存了一台矿石收音机的机芯(见图2-12),与前面两台矿石收音机机芯面板的尺寸基本一样,电路也相近,当然布局略有差异,有6只接线柱,分别为两副耳机,4只接线柱,天地线各用1只(面板上的活动与固定矿石缺失了)。另外,这个机芯的木制面板没有包漆布,只是刷了一层黄色的油漆。再有就是空气单连很特别,大概是进口的产品。据此推断,该机可能是20世纪50年代小作坊利用部分进口元件与国产件组装的。那一时期,很多小厂生产电子管收音机便是进口件、国产件混合使用的。 图2-12 20世纪50年代的成品矿石收音机机芯 7.一台红色木制机箱矿石机 图2-13所示是一台木制没有包漆布的矿石机,机壳刷了桃红色油漆,尺寸为19.8cm×11cm×14cm。机芯面板也是木制的,并刷了一层漆。此机只用一枚固定矿石,有6只接线柱,没有“天线2”。双回路线圈分别绕在尺寸相同的两个纸筒上,线圈的圈数也基本一样。空气单连与前述“牡丹”机一样,但分线器却是整体结构的成品件。机器也曾被加装过晶体管放大电路。此机木制机箱的造型和制作工艺,明显好于前面几台机器。如机箱上下部分为斜式开口,面板也就有了一定角度的倾斜,不仅造型美观新颖,还便于调谐使用。另外机箱油漆的质量也属上乘,不仅漆面质感好,从破损的地方可以看出,腻子打底也很厚实。 图2-13 红色木制机箱矿石收音机 图2-13 红色木制机箱矿石收音机(续) 8.济南“胜利”矿石收音机 毕冠超收藏并撰文 这台胜利牌矿石收音机(见图2-14),由济南开明电料行出品,它长20cm,宽15cm,高14cm。读盘采用鸽子图案,并配有“和平”字样,它为纪念朝鲜战争结束,于20世纪50年代中期出品,故取名“胜利”。 济南开明电料行——崔金镜先生于20世纪40年代末期创建,是当时济南规模较大的电料行。该电料行后迁至经三纬五路,现皇宫照相馆位置,改名“济南开明无线电修理部”。再后来搬到经三纬五路口,改名“济南洪波无线电修理部”。 机器采用木制漆布包面,机芯面板刷漆,它们的空气可变电容器与‘牡丹’机不同。分线器也为成品一体件。此机的可贵之处是,在机箱上盖内贴的标签完整地保留了下来,它标明了机器的制造者、品牌和大致的生产时间。 此机购于济南中山公园内旧书市场,它位于济南老商埠区,与原开明电料行相距仅一站之遥。由于当时产量不大,历经50余年,辗转多地,完整保存至今实属不易。有缘得到此机,高兴之余更感动于一百多年前,济南这座历经千年风雨沧桑的古老城市,自开商埠,由此掀开了济南近代对外开放与城市化进程的历史篇章。 图2-14 济南“胜利”矿石收音机 图2-14 济南“胜利”矿石收音机(续) 9.“万明”矿石机 图2-15所示的这台矿石机的分线器标牌上标注了“万明”两个字,但这未必是该机的正式品牌。机器的制作工艺很好,结合部用小型木榫铆接,采用木制棕色漆面,尺寸为16.5cm×10.5cm×8.8cm,属于小型机。两只耳机接线柱上分别模压了“耳”、“机”字样;天地线接线柱上分别模压了“天”、“地”字样。机内装有3层蛛网线圈,一体型分线器和一颗固定矿石(并联的二极管为使用者所加)。值得注意的是,这台矿石机为了减小体积和节约成本,采用了云母介质可变电容器。从这些配置看,应该是20世纪60年代初期的产品。 图2-15 “万明”矿石机 10.“中山公园”矿石机 以上几台矿石机可以说是大同小异,而图2-16所示的这台比较特殊,它的外观比较接近一般的收音机。说它接近一般的收音机,主要是它有一个假“扬声器窗”,因为收音机似乎都是有扬声器的。当然,这台矿石收音机没有扬声器。此机看不出品牌,出售它的网友称之为“中山公园”矿石机,大概因为面板上的图案像中山公园吧。 图2-16 “中山公园”矿石机 图2-16 “中山公园”矿石机(续) 该机箱尺寸为23.2cm×16.5cm×9.2cm,木制,表面刷蓝色油漆。它采用双回路线圈,一体式分线器,老式国产空气可变电容器,使用一枚固定矿石。与其他成品矿石机区别较大的是,它没有通常的接线柱,取而代之的是插口,其中中间4个为两副耳机插口,两边的分别是天地线插口。 11.滑动抽头线圈调感式袖珍矿石机 国产矿石机中采用滑动抽头线圈调感式的很少,图2-17所示的是一台袖珍式的矿石机。它采用红色塑料外壳,颜色艳丽,造型美观,尺寸为8cm×7.3cm,上下面呈弧形。机内为塑料管线圈架,无矿石,使用二极管检波。此机应为20世纪60年代末至70年代初的产品。 图2-17 滑动抽头线圈调感式矿石机 12.象牌101型袖珍矿石收音机 这是一台1964年上海立民化学玩具厂出品,采用晶体二极管检波的矿石收音机(见图2-18)。双色塑料外壳,造型相当别致美观,尺寸只有9cm ×6cm×2.6cm。这台矿石机的可贵之处是保存了包装盒、说明书和购货发票。发票的填写日期是1964年11月26日,售价6元7角钱。 图2-18 象牌101型袖珍矿石收音机 图2-18 象牌101型袖珍矿石收音机(续) 13.教学示范用矿石机 图2-19所示是一台教学用的矿石机,它底板为红色胶木板,所有元器件和接线都清楚地显示在胶木板上。该机电路为简单的双回路调谐式,没有使用分线器。时至今日,机器接上天地线和耳机还可以收听。此教学矿石机系国营北京教学仪器厂20世纪60年代末开始生产的,一直延续到20世纪70年代中期。 图2-19 教学示范用矿石机\begin{figure}[htb]  \centering  \includegraphics[width=0.8\\linewidth]{Image00309.jpg}  \caption{图片 111: Image00309.jpg}\end{figure}\begin{figure}[htb]  \centering  \includegraphics[width=0.8\\linewidth]{Image00136.jpg}  \caption{图片 112: Image00136.jpg}\end{figure}\begin{figure}[htb]  \centering  \includegraphics[width=0.8\\linewidth]{Image00296.jpg}  \caption{图片 113: Image00296.jpg}\end{figure}\begin{figure}[htb]  \centering  \includegraphics[width=0.8\\linewidth]{Image00128.jpg}  \caption{图片 114: Image00128.jpg}\end{figure}\begin{figure}[htb]  \centering  \includegraphics[width=0.8\\linewidth]{Image00380.jpg}  \caption{图片 115: Image00380.jpg}\end{figure}\begin{figure}[htb]  \centering  \includegraphics[width=0.8\\linewidth]{Image00056.jpg}  \caption{图片 116: Image00056.jpg}\end{figure}\begin{figure}[htb]  \centering  \includegraphics[width=0.8\\linewidth]{Image00271.jpg}  \caption{图片 117: Image00271.jpg}\end{figure}\begin{figure}[htb]  \centering  \includegraphics[width=0.8\\linewidth]{Image00200.jpg}  \caption{图片 118: Image00200.jpg}\end{figure}\begin{figure}[htb]  \centering  \includegraphics[width=0.8\\linewidth]{Image00168.jpg}  \caption{图片 119: Image00168.jpg}\end{figure}\begin{figure}[htb]  \centering  \includegraphics[width=0.8\\linewidth]{Image00161.jpg}  \caption{图片 120: Image00161.jpg}\end{figure}\begin{figure}[htb]  \centering  \includegraphics[width=0.8\\linewidth]{Image00214.jpg}  \caption{图片 121: Image00214.jpg}\end{figure}\begin{figure}[htb]  \centering  \includegraphics[width=0.8\\linewidth]{Image00090.jpg}  \caption{图片 122: Image00090.jpg}\end{figure}\begin{figure}[htb]  \centering  \includegraphics[width=0.8\\linewidth]{Image00245.jpg}  \caption{图片 123: Image00245.jpg}\end{figure}\begin{figure}[htb]  \centering  \includegraphics[width=0.8\\linewidth]{Image00007.jpg}  \caption{图片 124: Image00007.jpg}\end{figure}\begin{figure}[htb]  \centering  \includegraphics[width=0.8\\linewidth]{Image00269.jpg}  \caption{图片 125: Image00269.jpg}\end{figure}\begin{figure}[htb]  \centering  \includegraphics[width=0.8\\linewidth]{Image00140.jpg}  \caption{图片 126: Image00140.jpg}\end{figure}\begin{figure}[htb]  \centering  \includegraphics[width=0.8\\linewidth]{Image00219.jpg}  \caption{图片 127: Image00219.jpg}\end{figure}\begin{figure}[htb]  \centering  \includegraphics[width=0.8\\linewidth]{Image00205.jpg}  \caption{图片 128: Image00205.jpg}\end{figure}\begin{figure}[htb]  \centering  \includegraphics[width=0.8\\linewidth]{Image00172.jpg}  \caption{图片 129: Image00172.jpg}\end{figure}\begin{figure}[htb]  \centering  \includegraphics[width=0.8\\linewidth]{Image00243.jpg}  \caption{图片 130: Image00243.jpg}\end{figure}\begin{figure}[htb]  \centering  \includegraphics[width=0.8\\linewidth]{Image00030.jpg}  \caption{图片 131: Image00030.jpg}\end{figure}\begin{figure}[htb]  \centering  \includegraphics[width=0.8\\linewidth]{Image00264.jpg}  \caption{图片 132: Image00264.jpg}\end{figure}\section{文件 13}\
%第三章 国外矿石机的研究和进展 第三章 国外矿石机的研究和进展 麦昆明 制作矿石收音机是我少年时代美好的回忆。从20世纪50年代看到陈宪文先生的《高效率矿石收音机的试制》那篇文章开始,我就一直梦想着能够安装这样一台高效率的矿石收音机。然而造化弄人,直到半个世纪退休之后,我才有机会重拾少年时代的爱好,继续探索这种最原始的无线电技术。 矿石收音机虽然看似简单,但是它牵涉很多高深的无线电理论。在提高矿石收音机性能的过程中,可以学习到很多有趣的新知识。 高效率的矿石收音机离不开对技术理论的深入了解和各种高质量元器件的应用。拜互联网飞速发展所赐,收集资料、交流经验和购买器材都比几十年前容易得多了。我逐渐发现国内外都有很多业余无线电爱好者仍然在孜孜不倦地探讨矿石收音机理论,努力提高它的灵敏度和选择性。 在互联网里可以搜索到很多有关晶体收音机(Crystal Radio,我们通常称为矿石收音机)的网站和论坛。它们把世界上很多国家的爱好者紧密联系在一起,互相介绍自己的设计、比较性能和交换心得,并且对基本原理作深入的探讨。 让我们先回顾一下晶体收音机在国外百年发展的历史,然后再介绍它的最新发展。 1874年,德国的发明家和物理学家,诺贝尔物理学奖得主卡尔·费迪南德·布劳恩发现了有整流特性的晶体。 1904年,出生于孟加拉的科学家J·钱德拉·鲍斯发明和取得“检测电气干扰的装置”的美国专利,里面应用了方铅矿晶体。但是西方科学界忽视了他的发明。 1906 年,美国的科学家G·W·皮卡德获得“通过电波接收情报交流的方法”的美国专利,编号为836531(见图3-1)。他发明了硅晶体检波器,用称为“猫须”的尖细金属丝接触矿石产生最好的半导体效应,可以解调调幅信号,应用在无线电电话和语言、音乐广播的接收。 图3-1 1906年,G·W·皮卡德所获得836531美国专利的插图 1922年,美国标准局发布了题为《制作和使用一个简单的自制无线电接收套件》的文件,该文件详细展示如何用普通的工具制作一个收音机收听广播(见图3-2)。这个设计对普及无线电有重大的意义。 图3-2 1922年美国标准局文件中的无线电接收套件插图之一 晶体接收机是无线电报时代最早广泛使用的无线电接收器。数百万个从工厂生产或自制的便宜可靠的晶体收音机是20世纪20年代向公众推广无线电娱乐媒体的主要力量(见图3-3)。 大约1920年之后,晶体收音机逐渐被真空管放大的接收机取代,从而失去了商业用途。但是直到今天,制作晶体收音机仍然是许多青少年和业余爱好者学习无线电技术的入门途径。在互联网上常常见到有老师和无线电爱好者征询意见、寻找矿石机的资料和零件,为学生和童子军的兴趣小组开班授徒。 图3-3 20世纪20年代,一个美国家庭用矿石收音机收听无线电广播 1932年,在澳洲布里斯本的《星期日邮报》上,发表了一篇题目为《The Mystery Crystal Set(神秘的矿石机)》的文章。这篇文章介绍了一台与传统线路不一样的矿石机(见图3-4)。它的检波电路上没有调谐电路,天线和地线又分别接在两个紧密交连的线圈上,但是灵敏度和选择性都有很好的表现。业余爱好者对它非常感兴趣,不断研究它的原理并且衍生了很多改良的电路。在互联网里用“Mystery Crystal Set”做关键词搜索,可以看到很多关于这种矿石收音机的文章、照片和电路图。 美国的中波电台比较多,每个大城市附近都可能有几十个电台。欧洲各国的中波电台不多,但是有长波电台。由于居住的地区少见高楼大厦,电气干扰也不严重,这就为矿石机爱好者提供了良好的接收环境。他们利用优质零件设计和制作各式各样的矿石机。由于普遍使用舌簧式耳机和灵敏度很高的军用Sound Powered耳机,再用多股数利兹线绕制高Q值线圈,使矿石收音机得到前所未有的灵敏度和选择性,从而能够接收远达两三千千米以外的电台。有些业余爱好者说,电子管收音机能够收到的电台,矿石收音机都可以听得到。事实上,不少矿石机能够听到和证实呼号的电台比一般超外差式收音机更多,只是必须用灵敏的耳机细心搜索而已! 国外的矿石机爱好者勇于创新,精益求精,有些人可以说是狂热的,却因此得到出乎意料的成果。请看看下面几个比较特别的例子。 图3-4 1932年发表的“神秘的矿石机”电路图 美国Bruce Lundy的矿石机(见图3-5),两个大线圈用1/4英寸直径的铜管绕制,线圈直径15英寸。线路虽然简单,可是选择性非常好。它可以把960km之外的电台和本地电台分开,而两个电台的频率只相差10kHz。 图3-5 美国Bruce Lundy用15英寸直径线圈的矿石收音机 美国的Josh Young制作了8英尺直径环状天线的矿石机(见图3-6)。他采用反绕双值线圈电路。这台矿石机放在室内,一个晚上曾收听到80多个电台,其中有些远在1000英里之外。目前他正在试验同样尺寸,用660/0.04mm利兹线制作的线圈,并且想尽办法改良Sound Powered耳机、可变电容器和波段开关等零件,力求得到最好的效果。 图3-6 美国Josh Young的8英尺直径环状天线矿石机 矿石机一般只适合接收中波和长波调幅广播,但是很多爱好者也尝试接收短波电台,尤其是在美国东部,有很多对外国广播的大功率短波电台,比较容易收听。通常可以接收60~16m波段的广播,也有人证实曾收到13m波段的电台。 奥地利的Broesel制作了一个长、中、短五波段矿石机(见图3-7)。他用80/0.02mm的利兹线和封闭的铁粉芯盒子绕制线圈。并用公寓阳台上的栏杆做天线,也可以收到相距1200km以外的外国电台。 图3-7 奥地利Broesel的长、中、短五波段矿石机 也有不少人利用二极管斜率检波的特性安装FM调频广播矿石机。图3-8所示是奥地利Broesel的FM矿石机。 图3-8 奥地利Broesel的FM矿石机 2002年,美国工程师Ben Tongue发表了一个新颖的电路(见图3-9),他利用矿石机检波之后产生的直流电作为电源,给一个0.33F的电容器充电,就可以驱动一个微功率集成电路OPA349UA的音频放大器,使矿石收音机的音频输出功率增强约20dB。只要直流电压有1.3V以上,这种集成电路就能够工作。适当控制放大器的音量,法拉电容器充满电可以连续使用4~24小时。 图3-9 美国Ben Tongue的矿石机集成电路放大器 2007年1月美国业余无线电协会月刊《QST》上发表了Bob Culter的文章,介绍利用场效应晶体管做检波器的高灵敏度矿石机(见图3-10)。所用的ALD110900 MOSFET集成电路是零导通电压,因此可以对非常微弱的信号检波,特别有利于远距离接收。 图3-10 美国Bob Culter的MOSFET检波矿石收音机 下面介绍国外几个比较著名的矿石收音机网站。 矿石收音机系统:设计、测量和改良(http://www.bentongue.com/xtalset/xtalset.html) 这是美国已退休的无线电工程师Ben·H·Tongue所设网站中的一个专门研究矿石机的网页。里面有将近30篇论文,讨论在矿石机设计中如何应用无线电工程原理和测量技术,并说明实际测量的方法。论文内容涉及选择性、灵敏度、线圈和电容器的品质因数Q、阻抗匹配、二极管的各种参数、音频变压器特性、耳机和天线接地系统参数等。其中一些矿石收音机的设计,随着对它们的性能测试阐述了这些基本原则。他还提出了诸如二极管检波器线性平方律的交叉点、“反绕双值线圈”(Contra Wound Dual-value Inductors)和“班尼电路”(Benny Circuit)等基本的技术概念。 “反绕双值线圈”是两个等值的线圈绕在同一个线圈架上(见图3-11),但是绕线方向相反。两个线圈在低频波段串联,在高频波段并联。线圈如何绕线和连接就是它的秘诀。这种线圈损耗少而Q值高。可变电容器也安排在最佳操作范围,进一步减少了损耗。在中波的高频段,与传统的单个线圈比较,利兹线的股数因为两个线圈并联而加倍,使线圈的Q值更高而得到比较好的选择性和灵敏度(见图3-12)。 所谓“班尼电路”,是一个串联在检波器输出端的电阻与电容器并联电路(见图3-13)。由于Ben·H·Tongue首先应用在他设计的矿石收音机里,业余爱好者就叫这种简单有效的电路为“Benny”。当音频变压器的初级直接连接在检波器后面音频输出端的时候,有直流电流通过二极管、调谐电路和音频电路。音频变压器的直流电阻比它的阻抗低得多。随着信号增强,将会有更大的直流电流通过,变压器的低直流电阻加载于调谐电路而产生严重的失真并使选择性变坏。“班尼电路”的作用是恢复检波电路的高直流电阻,可以消除音频失真和提升高频调谐电路的选择性。并联在电位器两端的0.1μF电容器让音频电流通过而损失很少。 图3-11 圆筒式反绕双值线圈 图3-12 反绕双值线圈的并联和串联调谐电路 阿拉巴马州伯明翰市无线电俱乐部矿石机接收竞赛(http://www.crystalradio.us/contests/index.htm) 很多喜欢做远距离接收的爱好者每年举办晶体收音机接收竞赛,根据规定时间内所收到的电台的功率和接收距离计算成绩。美国阿拉巴马州伯明翰市无线电俱乐部从2004——2010年都举办了这种远距离接收竞赛。有些参赛者在一个星期里能够收到并且确认300多个不同的中波广播电台,有些电台远达两三千千米。这个无线电俱乐部的网站对参赛者的矿石机和成绩有详细的介绍。 阻抗匹配变压器电路图 图3-13 矿石机阻抗匹配变压器和班尼电路 请继续关注(Stay Tuned)(http://crystalradio.net) 这是业余无线电爱好者Darryl Boyd设立的网站。他介绍了很多初级到高级矿石机的详细资料。尤其难得可贵的是,他收集了各种Sound Powered耳机,并分析比较了它们的结构和性能。可以说他是向矿石机爱好者推广舌簧式耳机应用的功臣。 戴夫的自制收音机(http://makearadio.com) 美国的Dave Schmarder可能是中国矿石机爱好者比较熟识的名字。他的网站详细介绍了自制的70多个精美的矿石机,还有很多相关的参考资料。 Dave还另外设立了一个无线电论坛 “The RadioBoard”(http://theradioboard.com/rb),让各国的无线电爱好者在那里交流经验。 矿石收音机(http://www.crystal-radio.eu/index.html) 这是荷兰无线电爱好者Dick Kleijer的网站。除了介绍他本人制作的矿石机之外,他对矿石机的各种元器件和电路有深入的分析和研究。我认为他写的《How to build a sensitive crystal receiver(怎样制作灵敏的矿石收音机)》值得认真阅读。 希望这些资料能够抛砖引玉,帮助读者们了解国外矿石机的发展现状,从而提高我们的研究水平。事实上,近年国内矿石机爱好者的实验有很多创新,成果非常丰硕,值得庆贺!\begin{figure}[htb]  \centering  \includegraphics[width=0.8\\linewidth]{Image00239.jpg}  \caption{图片 133: Image00239.jpg}\end{figure}\begin{figure}[htb]  \centering  \includegraphics[width=0.8\\linewidth]{Image00221.jpg}  \caption{图片 134: Image00221.jpg}\end{figure}\begin{figure}[htb]  \centering  \includegraphics[width=0.8\\linewidth]{Image00013.jpg}  \caption{图片 135: Image00013.jpg}\end{figure}\begin{figure}[htb]  \centering  \includegraphics[width=0.8\\linewidth]{Image00310.jpg}  \caption{图片 136: Image00310.jpg}\end{figure}\begin{figure}[htb]  \centering  \includegraphics[width=0.8\\linewidth]{Image00036.jpg}  \caption{图片 137: Image00036.jpg}\end{figure}\begin{figure}[htb]  \centering  \includegraphics[width=0.8\\linewidth]{Image00394.jpg}  \caption{图片 138: Image00394.jpg}\end{figure}\begin{figure}[htb]  \centering  \includegraphics[width=0.8\\linewidth]{Image00256.jpg}  \caption{图片 139: Image00256.jpg}\end{figure}\begin{figure}[htb]  \centering  \includegraphics[width=0.8\\linewidth]{Image00276.jpg}  \caption{图片 140: Image00276.jpg}\end{figure}\begin{figure}[htb]  \centering  \includegraphics[width=0.8\\linewidth]{Image00377.jpg}  \caption{图片 141: Image00377.jpg}\end{figure}\begin{figure}[htb]  \centering  \includegraphics[width=0.8\\linewidth]{Image00389.jpg}  \caption{图片 142: Image00389.jpg}\end{figure}\begin{figure}[htb]  \centering  \includegraphics[width=0.8\\linewidth]{Image00348.jpg}  \caption{图片 143: Image00348.jpg}\end{figure}\begin{figure}[htb]  \centering  \includegraphics[width=0.8\\linewidth]{Image00051.jpg}  \caption{图片 144: Image00051.jpg}\end{figure}\begin{figure}[htb]  \centering  \includegraphics[width=0.8\\linewidth]{Image00231.jpg}  \caption{图片 145: Image00231.jpg}\end{figure}\begin{figure}[htb]  \centering  \includegraphics[width=0.8\\linewidth]{Image00006.jpg}  \caption{图片 146: Image00006.jpg}\end{figure}\section{文件 14}\
%第四章 新时期国内矿石机的研究和进展 第四章 新时期国内矿石机的研究和进展 一、器材篇 高阻耳机和匹配变压器 雷宝玉 矿石收音机中有一个重要的元器件——耳机,它的功能是实现电声转换。耳机在电路中的符号如图4-1所示。常用于矿石收音机的耳机可分为两种,一种是电磁式耳机,另一种是舌簧式耳机。电磁式耳机由磁铁、线圈、铁质振动膜片和外壳组成,阻抗多为高阻;舌簧式耳机由磁铁、线圈、极靴、舌簧、振膜盆、连杆和外壳组成,阻抗多为中低阻,如图4-2和图4-3所示。 图4-1 耳机符号 图4-2 电磁式耳机 图4-3 舌簧耳机 当音频电流通过电磁耳机线圈时,耳机中磁铁的磁场就会随着音频电流忽强忽弱地变化,振动膜片受到磁场的吸力也就随着电流的变化而改变,振动膜片随之振动起来从而产生了声音。当音频电流通过舌簧耳机线圈时,舌簧所感应出的磁场随着音频电流忽强忽弱地变化并与极靴的磁场相互作用使舌簧在极靴之间摆动,通过连杆带动振膜盆振动从而产生了声音。 矿石收音机在工作时的调谐回路的阻抗RP 等于二极管的零电压电阻RD ,等于负载的阻抗RL 或者3倍的调谐回路的阻抗RP ,还等于二极管的零电压电阻RD 及负载的阻抗RL ,此时矿石机负载RL 能够得到最大的功率输出。为了使矿石收音机负载的阻抗RL 要达到二极管的RD 阻值有两个途径:(1)使用高阻抗耳机;(2)使用高阻抗匹配变压器,配高灵敏度的舌簧耳机。 1.高阻耳机 高阻耳机一般都是电磁式耳机,它们的阻抗有750Ω、1000Ω、2000Ω和4000Ω等。军用高阻耳机可达4400Ω,可直接作为检波矿石或二极管的负载发声,如图4-4所示。 图4-4 军用耳机及内部结构 这类耳机由于结构的限制,用于矿石机其灵敏度一般不是很高。相对检波二极管的零电压电阻RD 几十千欧到一兆多欧,耳机的阻抗还是相对较小,与二极管只能是接近匹配。 特别介绍一种适合矿石收音机用的德国德律风根(Telefunken)公司早期生产的耳机,见图4-5。一般耳机拆开盖子的时候,振动铁片是被磁铁吸引贴附在外壳上面的。德律风根耳机与众不同的地方是它的振动铁片固定在盖子下面。随着盖子旋转,振动铁片与磁铁两极的距离也跟着变化,因此可以把振动铁片调整到尽可能接近磁铁但又不会被吸死的最佳位置,再反方向旋转外壳螺旋线上的一个金属圆环,把盖子压紧固定,使它不能再向内转动。这样就可以把耳机调整到最灵敏的状态。 图4-5 德国德律风根(Telefunken)耳机 2.匹配变压器 匹配变压器是解决高阻耳机和检波矿石或二极管不能最佳匹配的一个很好器件。它的阻抗可以制作得很高,从几十千欧到几百千欧甚至兆欧,作为矿石或二极管的负载能够很容易实现RD =RL ,使之达到良好的匹配。为了调整方便灵活,一般都采用自偶匹配变压器,在低端采用不同阻值的抽头来配合不同阻抗的高灵敏舌簧耳机,实现最佳匹配。抽头的常见阻值由低到高为32Ω、64Ω、150Ω、300Ω、600Ω、1500Ω、3000Ω、5kΩ、10kΩ、20kΩ、40kΩ、100kΩ、200kΩ、400kΩ、800kΩ、1.2MΩ。 常见的成品匹配变压器有T725,见图4-6,最高阻抗为40kΩ。其他抽头阻抗依次为20kΩ、10kΩ、5kΩ、2.5kΩ、1.5kΩ、600Ω、300Ω、150Ω、次级为8Ω,通常将8Ω串在下端使用。T725是国外矿石机发烧友非常推崇的一款匹配变压器,它可以和1N34二极管进行较好的匹配,使用效果较好。虽然这款匹配变压器是国内生产的,但都属于出口产品,国内很难见到,只有少数“矿友”通过国外回流而得到。 为了使矿石机耳机与二极管之间达到良好的匹配,有网友利用铁氧体磁芯设计并量产了仿T725矿石机专用匹配变压器,见图4-7。设计这款匹配变压器时充分考虑了国内国外常见耳机的阻抗及常见二极管零电压电阻RD 和MOSFET检波的特点,优化选取了不同阻值的抽头。由低端到高端的阻抗依次为0Ω、32Ω、150Ω、300Ω、500Ω、800Ω、1.5kΩ、2.5kΩ、5kΩ、10kΩ、20kΩ、40kΩ、100kΩ、200kΩ。设计仿T725匹配变压器时采用了较大的设计余量,因此还可以进行超阻使用,使得各端阻抗值翻倍,最高阻抗200kΩ可超阻到400kΩ使用,和大多数二极管可以实现接近匹配。 仿T725是一款矿石机专用匹配变压器,也是目前国内外唯一量产的矿石机专用匹配变压器,除了受到国内“矿友”的关注,也被国外“矿友”批量采购。经国外“矿友”测试仿725匹配变压器的性能远远好于T725。 图4-6 T725匹配变压器 图4-7 仿T725匹配变压器 高灵敏度舌簧耳机介绍及超高阻耳机改造实例 雷宝玉 舌簧耳机由于电磁结构与电磁式耳机不同,其灵敏度普遍要好于电磁式高阻耳机,但舌簧耳机一般是在有源电路中应用,阻抗比较低,直流电阻值仅为几十欧,交流阻抗在300~1500Ω,在矿石收音机中需要配备匹配变压器使用。 1.高灵敏度舌簧耳机介绍 常见的用于矿石机的国产舌簧耳机元件有: (1)上海电讯器材厂生产的SC2-300耳机,如图4-8所示。直流电阻30Ω,交流阻抗300Ω,老式电话机的听筒和楼宇对讲系统多采用此元件。因为要安装在电话听筒里面,所以SC2-300体积小巧,外观直径为45.5mm,高度为21mm,极靴形状设计巧妙降低了高度,与舌簧之间的作用面积较大,坡莫合金的舌簧并设计了舌簧位置调整螺丝,可方便地调整舌簧在极靴磁隙中的位置,得到最佳的灵敏度。与匹配变压器配合用于矿石机是个不错的选择。 图4-8 SC2-300舌簧耳机元件 (2)军用舌簧耳机如图4-9所示,型号为SHH-1。这种叫作摆枢差动式耳机,用于军用电台,在矿石机上的表现不是很好,灵敏度不是很高。它的外型尺寸:φ45.5mm×17mm;频率范围:200~4000Hz;阻抗:150/300/350×(1±30\\%)Ω(1000Hz);平均灵敏度:≥96dB/mW(200~4000Hz)。 图4-9 军用舌簧耳机SHH-1 (3)舰船甲板耳机元器件如图4-10所示。这是早期的舰船甲板耳机元器件,在设计时考虑到战时断电的情况下仍然能够使用,其结构设计采用了舌簧中间支架的方式,在舌簧两端分别安装相互反绕的线圈从而达到很高的灵敏度,是国内较为优秀的用于矿石机的耳机元器件。该耳机元器件用料讲究,外壳采用全金属材料,但随之带来的弊病是体积大和较大的重量。其参数为:直流电阻28Ω,电感量110mH,交流阻抗约700Ω。 图4-10 舰船耳机元件 (4)常见的用于矿石机的国外舌簧耳机元器件有以下几种。 ① 美国USI公司的Sound Power USI-UA1614耳机,见图4-11。这是一款极为优秀的耳机元器件,灵敏度很高,其参数为:直流电阻62Ω,交流阻抗约1000Ω。早期美国大卫·克拉克(David Clark)公司生产的航空通信机就是采用这种耳机。图4-12所示的就是H5040型大卫·克拉克航空耳机。 图4-11 USI-UA1614耳机元件 图4-12 大卫·克拉克H5040型航空耳机 ② 美国RCA的Sound Powered“Big Can”大罐头耳机。RCA 2040-A型手提电话里面的Sound Powered元器件就是“Big Can”,见图4-13。“Big can”在国外的“矿友”中备受推崇,被认为是用于矿石机的最好的耳机之一。 该机参数为:直流电阻130Ω,电感量250mH,交流阻抗约1500Ω。 这款耳机经配合匹配变压器在矿石机上试用,显示出很高灵敏度和高音质的非凡表现力。 图4-13 大罐头耳机元件 2.超高阻耳机元器件改造实例 舌簧耳机的交流阻抗较低,必须配合匹配变压器使用,如果直接用于矿石机必须要对耳机线圈进行超高阻改造,使之达到百千欧的交流阻抗。下面是一篇关于上讯SC2-300耳机元件的改造实例。 (1)漆包线的选用。 考虑到尽量多绕圈数,达到15000~20000圈以上,漆包线直径要尽量小,还要有一定的强度,保证绕制顺利。最好选用线径为0.025mm的漆包线,见图4-14。 图4-14 φ0.025mm高强度漆包线 (2)线圈骨架材料的选用。 原骨架的挡片比较厚,影响绕线量,所以要自制薄壁骨架。选用电工青壳纸作轴,3.5英寸软盘外壳最薄处作挡片,见图4-15和图4-16。 图4-15 电工用青壳纸做线架心轴 图4-16 3.5英寸软盘外壳做线架挡片 (3)线圈骨架的制作。 测量原线圈骨架的外廓尺寸,保证自制线圈能够非常吻合地放入极靴之中,见图4-17。 首先,制作心轴。将青壳纸根据骨架尺寸裁成小纸条并按图划线,见图4-18。用小刀将划线划深至纸厚的一半,但一定不能划透,然后按划线折起。 图4-17 原线圈骨架尺寸 图4-18 在青壳纸上画线 在软盘外壳的最薄处按尺寸划出骨架挡片的形状,并将其用刻刀裁下,见图4-19和图4-20。 图4-21所示的是裁好的线架挡片和心轴,将它们组合到一起,见图4-22。 图4-19 在软盘壳上划线 图4-20 最薄处0.6mm 图4-21 加工好的骨架心轴和挡片 图4-22 将心轴和挡片装到一起 用卡尺固定骨架,调整到适合的尺寸,点少量502胶水将其连接起来,见图4-23。 将固定好的骨架从卡尺上取下,用502胶水继续粘接,502胶水一定要浸透青壳纸,使其变硬,挡片和心轴连接处用502胶水溜缝,待其凝固干燥,见图4-24。 图4-23 整理挡片和心轴的相对位置 图4-24 用502胶粘接好 图4-25所示是制作好的线架与原线架的比较。图4-26所示是将自制的骨架装到极靴中看是否吻合。 图4-25 与原骨架的对比 图4-26 将骨架放入极靴 (4)线圈的绕制。 用市售的手摇绕线机(见图4-27),在轴头顶端钻一直径4.2mm的孔,用M5丝锥攻出M5的螺孔,见图4-28。将一个M5的长螺丝一端锉成1mm厚,用于固定线架,见图4-29。将锉扁的M5螺丝旋入绕线机轴头的M5螺孔中,并用螺母固定,见图4-30。 图4-27 市售的手摇绕线机 图4-28 将轴头攻成M5的螺孔 图4-29 将M5的长螺丝一端锉成1mm厚 图4-30 将扁螺丝安装在主轴上 将线架穿入锉扁的螺丝上,并用螺母紧固,调整好同心,见图4-31。 图4-31 将骨架固定 找一硬纸片,漆包线在上面绕10圈,除去纸片,用手将漆包线搓成20股线作为引线,见图4-32和图4-33。 图4-32 将0.025mm漆包线在纸片上绕10圈 图4-33 搓成20股的引线 引线穿入线架,绕几圈线,将引线压住,就可以开始绕线了,见图4-34和图4-35。 图4-34 绕线起头 图4-35 开始绕线 绕线时漆包线轴要垂直放于地面,手要控制漆包线左右移动,使漆包线均匀分布,必要时用一个放大镜放在线架上,以方便观察绕线情况。 绕满圈数为17500匝,还要在纸板上缠绕10圈作为引出线。绕好的线圈,外层用透明胶带缠紧就可以装到耳机上面了,见图4-36和图4-37。 图4-36 绕好的线圈 图4-37 装到耳机上 (5)电阻和电感量的测试。 用数字万用表测量出直流电阻为17.7kΩ,用电感表测量出电感量在374Hz时为53.4H,交流阻抗约为125kΩ,见图4-38和图4-39。至此超高阻舌簧耳机就改造完成了。 (6)改造效果。 a.耳机引线超过50cm时,在房间里能感应到交流声。 b.接收强台时音量减小,弱台音量提升,强台与弱台的音量差距减小。 c.去掉了匹配变压器与二极管直接匹配。 d.比低阻直接使用时灵敏度得到较大的提升,接收的弱台数量明显增多。 图4-38 耳机直流电阻 图4-39 耳机电感量 矿石机匹配变压器作用以及简单的设计方法 李清 矿石机与一般的收音机有着重要的区别,那就是普通的收音机里都有放大器,而矿石机里没有放大器。要想提高矿石机的性能就要尽可能减少矿石机各部分信号的损失,因此机内各部分电路之间的阻抗匹配就显得格外重要了,只有阻抗匹配了,信号传递时的能量损失才能最小。可作为矿石机检波的器件种类很多,各种矿石、很多型号的二极管、绝缘栅场效应管等,甚至经过发蓝工艺处理的刮胡刀片都能用来检波。 如此多的检波器件性能差别非常大,由这些器件组成的检波器输出阻抗也千差万别,而每只耳机的阻抗是一定的,不可能与各种不同的检波器都能实现阻抗匹配,解决这一矛盾的办法就是使用匹配变压器,使耳机与检波器实现阻抗匹配。 使用匹配变压器后,耳机与检波器实现了阻抗匹配,耳机得到了比阻抗失配时更多的能量,当然,这需要承担变压器产生的损耗,所以希望变压器的损耗越小越好。从矿石机爱好者使用的角度看,为了可以使用尽可能多的检波器做实验,就需要匹配变压器提供更多的不同阻抗端。 1.匹配变压器的替代品 矿石机爱好者们在制作矿石机时往往需要用到匹配变压器,在市场上很难见到适合矿石机使用的匹配变压器,因此很多爱好者都用一些市场上很容易买到的廉价小功率电源变压器代替匹配变压器。这些廉价的小功率电源变压器在一定的条件下是可以当匹配变压器用的,是否好用首先要看小电源变压器是否有足够的电感量,这点很重要!用电感量低的电源变压器做矿石机匹配变压器效果肯定不好。一般来说,功率越小的电源变压器的电感量就越大,所以最好当匹配变压器使用的电源变压器的功率都在2W以下。但是电源变压器的功率也不能太小,功率太小的电源变压器的线阻会很高,变压器的效率就会下降。一般情况下,功率在1~3W就可以了。在使用电源变压器代替矿石机的匹配变压器时,我们关心的是这只变压器适用于多大阻抗的耳机以及能提供多大的初级阻抗。要想了解这两个问题很容易。根据电压比的平方,就是阻抗比的方法就可以计算出变压器的阻抗比,然后用变压器的阻抗比乘以低阻耳机的阻抗值就得到了变压器的初级阻抗。 例如,有一个220V/6V电源变压器和一只32Ω的耳机,如果将32Ω耳机接在变压器的6V端,那么在220V端的阻抗是多少? 变压器的电压比≈36.67 阻抗比=36.67×36.67≈1344 接入32Ω的耳机后变压器初级阻抗=32×1344=43008(Ω)。 也就是说,变压器从变压器220V端看进去阻抗是43kΩ。 再例如,做一台矿石机需要一只阻抗50kΩ的耳机,现有一只阻抗300Ω的舌簧耳机,要用一只220V变多少伏的变压器才能将这只耳机的阻抗变换到50kΩ? 需要变压器提供的阻抗比=50000/300≈166.67 变压器的电压比等于阻抗比的开平方值≈13 变压器的次级电压值=220/13≈17(V)。 用220V变17V的电源变压器即可。 2.矿石机匹配变压器的简单设计方法 在要求不高的情况下,用电源变压器代替匹配变压器比较方便,但是效果不是很好,一是因为廉价的电源变压器铁芯质量较差,用于矿石机的匹配变压器时能提供电感量的裕度比较小。再者,一般的电源变压能提供的阻抗端子很少,并不能满足爱好者做各种矿石机实验。其实设计制作一只矿石机的匹配变压器并不难,如果制作者手头有现成的硅钢片铁芯、坡莫合金铁芯或是铁氧体磁芯,就可以用下面介绍的方法设计制作出匹配变压器。如果制作者有高质量的硅钢片铁芯、坡莫合金铁芯或是高导磁率的铁氧体磁芯,就能制作出性能很好的匹配变压器。下面是具体的设计方法。 首先,我们要根据想要得到的初级最大阻抗确定变压器初级最大的电感量,这一最大电感量的确定原则就是:这一电感量在规定的频率上产生的感抗要充分大于你要得到的最大阻抗。 例如,假设变压器的初级在频率为1000Hz时最大阻抗打算做到100kΩ,那么我们可以计算出这100kΩ对应的电感量L=100000/(2πf)=16H,也就是说,这个变压器如果初级电感量做到了16H,那么在次级不接任何负载时(完全开路),其初级的阻抗最高也就是100kΩ了。为了让次级负载的阻抗有反射到初级的余地,我们要绕出的线包的最大电感量一定要充分大于16H,比如我们可以确定为16H的5倍或是8倍,当然10倍也行,假设就用5倍,那么绕组初级的最大电感量就是16×5=80H。 其次,确定绕线匝数。绕线匝数由一个实验来确定,为此我们最好有一部能测量电感量的交流电桥,这样计算出的匝数就会很准确。如果没有电桥,也可以用普通的数字电感电容表,但是这样做的误差比较大。具体方法是:先在线架上绕上N匝线圈,然后装好铁芯,测量这N匝的电感量,然后把测得的电感量除以N的平方,我们就得到了每匝的电感量。我们用需要的最大电感量除以每匝的电感量,再把除得的商开平方,得到的就是最大电感量所对应的线圈匝数了。 例如,假设取N=10(N越大结果越准),我们先在线架上绕了10匝,假设装好铁芯后测得电感量是100μH,那么100μH/(NXN)=100/100=1,即每匝1μH。总匝数等于80H除以1μH的商开平方,即80/0.000001=80000000,80000000开平方约为8944匝。这也就是说这只变压器的100kΩ线包要绕8944匝。其他的阻抗端“根据匝数比的平方是阻抗比”这一规律很容易计算出。 以上介绍的方法没有将变压器的损耗计算在内,在实际应用时我们可以按如下例子计算。 我们已知100kΩ要绕8944匝,那么300Ω阻抗端要绕的匝数是8944/( )≈490匝,如果要3kΩ阻抗,阻抗比是3000/300=10,匝数比是10开平方,等于3.16,3kΩ端的匝数是490×3.16≈1550,考虑变压器的损耗,加入10\\%的修正系数,1550×1.1=1705即,3kΩ阻抗端实际要绕1705匝,同理,100kΩ端就不要绕8944匝了,而是改绕8944×1.1=9834匝。如果需要更多阻抗的抽头,只要按上述方法计算出对应阻抗端的匝数即可。 以上的方法避开了匹配变压器设计时要知道铁芯导磁率、要计算磁路长度等的麻烦,而是用实验的方法确定了匝数,使得设计过程简化了许多。这一方法很适合业余矿石机爱好者,大家不妨试试。 上面的设计方法可得到匹配变压器在各个阻抗端的具体匝数,实际制作变压器还要确定绕制各个阻抗线圈的漆包线的线径。当然线径越粗越好,线径越粗,线阻就越小,绕出变压器的效率就高,但是铁芯的绕线空间有限,线径粗了可能就绕不下了,所以低阻抗端尽量用粗些的线,高阻抗部分的线用细些的。 自己设计制作的矿石机匹配变压器一般都做成自耦变压器,自耦变压器的次级绕组是初级绕组的一部分,这样做成的变压器线包总匝数少,在同样的绕线空间内就可以用更粗的线绕制,降低了线包的线阻,提高了变压器的效率。自耦变压器初次级之间的耦合很紧密,对提高变压器的效率也很有利。 计算出变压器各个阻抗端的匝数数据后,就可以绕制变压器了,绕制的过程也很简单,应该大致估算一下变压器磁芯的绕线空间,只要保证能绕下,线径应尽可能粗,至少阻抗低的几个绕组用线要粗一些,这样绕出的变压器效率会高些。绕制时不必排线,乱绕即可,但是要尽量保持绕制过程线包面平整。 自制矿石机匹配变压器的铁芯要尽量选择质量好的,质量不好的铁芯绕出的变压器效率比较低,高质量的硅钢片、坡莫合金、导磁率较高的铁氧体磁芯都是制作矿石机匹配变压器的好材料。 实践证明,导磁率在2000以上的铁氧体磁芯做出的匹配变压器性能很好,如果导磁率能上万就更好了。所选磁芯的体积不必太大,太大的磁芯虽然可以用粗线绕制,但是大磁芯的磁路长,反倒不如小些的磁芯更容易做出高效率来。磁芯也不宜过小,太小的磁芯线包绕制会很困难。线包绕好后安装磁芯时一定要保证两磁芯的接触面干净,结合尽量紧密,为此磁芯一定要裹扎紧密、牢靠。 导磁率较高的铁氧体磁芯不难找到,价格也不算贵,而且铁氧体磁芯具有电阻率高、涡流损耗低等优点。铁氧体磁芯最大的缺点就是易发生磁饱和,刚好矿石机输出的信号很小,一般不会造成铁氧体磁芯的饱和,所以用铁氧体磁芯制作矿石机的匹配变压器刚好可以扬长避短,取得好的效果。实践证明,用上述方法设计制作的铁氧体磁芯变压器效率达到90\\%以上是完全可能的,实际使用效果很好。 按上述方法设计、计算阻抗端较多的匹配变压器会感觉比较烦琐,计算过程需要较长的时间,而且很容易出错。为了解决这一问题,有矿石机爱好者用Excel表格制作了匹配变压器设计计算表格,使用这一表格设计计算一款多阻抗端的矿石机匹配变压器只要十几分钟便可搞定,非常方便。 图4-40所示便是使用这个设计表格设计一款匹配变压器时的界面。 图4-40 用Excel表格设计的匹配变压器计算表格 图4-41所示的这只变压器是国外矿石机爱好者经常使用的T725变压器的实物照片,这款T725变压器虽然是中国制造的,但是在国内很难买到。 图4-41 T725变压器 图4-42所示是用本文介绍的方法设计制作的3种矿石机专用匹配变压器。 图4-42 3种矿石机专用匹配变压器 寻找3DQ双栅场效应管花絮 聂建军 互联网有着无可比拟的跨地域优势,2007年初,麦老师从国外介绍了MOSFET场效应管110800、110900模块制作的矿石机,并团购了一些分发给国内部分“矿友”。由于110800、110900场效应管有零偏压的优点,检波效率高于传统二极管,便于制作高灵敏度矿石机和场效应管自由能再生矿石机。但是110800和110900价格都很高,国内没有货源,而且使用中容易损坏。有没有可以经济便宜的替代品?中国矿石机爱好者同样有着极高的好奇心和进取精神,开始了大海捞针式的海选,尝试各种能找到的各种场效应管,并做了大量的试验。 2007年11月,呼号为BD5IF的坛友报出:“早些时候,从旧高频头里拆了一只(3S)K123(以前只看过122),代替2AP9接进了矿石机。首先感觉是能用,凭耳朵听不出它和2AP9的高低,后来用数字万用表的200mV挡在耳机两端量直流电压,反复对比,发现它略逊于2AP9。” 梁道雄老师继而也发现高频头中的3DP踪迹:“旧彩电高频头是换上增补频道高频头后取下的高频头,这种高频头中有两个双栅MOSFET,是贴片管,型号不详,上面字样有3DP、3EB、F1等,这3种字样者上机效果很好,有旧高频头的同学不妨挖两个试试。为什么叫挖,为保护这种“娇气”元件,多次焊不好,BD5IF用剪下的方法也行,我则用刻刀挖出来再焊到印制板上。” 廉价场效应管踪迹终于显现,之后梁道雄老师又做了大量实验,证明3DQ、3DP系列双栅场效应管可以完美替代110800场效应管,“效果比四管并联的110800要好些”,而且不像110800、110900那样容易损坏。李清老师做了场效应管矿石收音机检波的多种矿石机应用参数分析试验。 适合矿石机使用的场效应管找到了,大量推广需要货源,究竟3DQ的完整型号是什么?梁老师与张卫国老师经过几年的共同努力终于找到货源,3DQ、3DP、3DB三种3SK143Q、3SK143P、3SK143B场效应管。二人合作为3SK143系列场管开发了两种矿石机应用套件。从此场效应管矿石机开始火遍矿坛,“3DQ”就成了双栅场效应管矿石收音机在矿坛的代名词。 怎样制作高Q值线圈 聂建军 要制作性能优良的矿石机就离不开高品质线圈,制作高品质中波线圈要注意以下几点: (1)线圈直径(包括蛛网)应不小于120mm(不含铁氧体线圈)。 (2)要使用利兹线(多股纱包线)。单股漆包线或其他类型线绕出的线圈Q值不会很高。国内可购到的利兹线单股直径有0.04mm、0.07mm、0.1mm等,国外有0.02mm的。数量有7股、60股、175股、270股、660股等。实践应用中,我发现0.02mm和0.04mm绕中波线圈Q值最高。 (3)线圈相邻两线之间应尽量“远离”。距离在一个本身线径以上最好。 (4)线圈支架要选用介质损耗低的材料。首选聚四氟乙烯,PP材质的“菜板”和有机玻璃也是常用的原料,不能用酚醛板或木板等。如果条件允许,线圈尽量做成脱胎结构。 (5)大型线圈要远离金属物质及介质损耗大的材料,这会损失线圈Q值。 下面介绍几种常用线圈。 (1)桶型线圈:如图4-43所示,这是初学者常常制作的线圈,制作时一般采用平绕方法。当线圈直径大于100mm,用介质损耗低的骨架, φ0.04mm×660股利兹线可得到较高的Q值。 图4-43 桶型线圈 (2)花篮线圈:许多优秀矿石机都使用这种线圈。常见绕制方法为“上一下一”和“上一下二”,图4-44所示是两种绕线方式示意图。绕出的成品外观见图4-45。 图4-44 两种绕线方式示意图 图4-45 绕制完成的成品 在同样直径同等匝数的情况下,“上一下一”绕出的线圈Q值高,线圈宽度大。“上一下二”绕出的线圈电感量大,线圈宽度小。当线圈直径超过100mm时,用φ0.04mm×660股线,可获得1000以上空载Q值。直径增大, Q值增加。图4-46和图4-47所示都是用直径为135mm线分别用两种绕法绕出的线圈空载Q值,图4-48所示是用直径为150mm线“上一下一”方法绕制的线圈空载Q值。 (3)蛛网线圈:采用同样直径线,蛛网线圈比花篮线圈Q值低。这种线圈的优点是用线量少,整个线圈Q值较平均。图4-49所示的蛛网线圈外径为130mm,内径为60mm,用φ0.04mm×270股利兹线绕52匝,电感量为300μH。 图4-46 用φ135mm的线采用“上一下一”绕制方法所获Q值 图4-47 用φ135mm的线采用“上一下二”绕制方法所获Q值 图4-48 用φ150mm的线采用“上一下一”绕制方法所获Q值 图4-49 蛛网线图 (4)磁环线圈:由于磁环是一个闭合回路,所以效率很高,因此用线最少。这种线圈的缺点是要实现精细耦合比较麻烦。图4-50所示是北京七九八厂生产的型号为NXO-40、(R40C1)的磁环,它的尺寸为37mm×23mm×15mm,用φ0.04mm×270股利兹线绕的线圈,测试频率为900kHz,它的电感量为250μH。 (5)磁棒线圈:现代许多人都喜欢用磁棒线圈做矿石机。中短波磁棒比中波磁棒有更高的Q值。图4-51所示是用两根90mm中短波磁棒与直径40mm小花篮组成的线圈,它的电感量为255μH。 图4-50 磁环线圈 图4-51 磁棒线圈\begin{figure}[htb]  \centering  \includegraphics[width=0.8\\linewidth]{Image00247.jpg}  \caption{图片 147: Image00247.jpg}\end{figure}\begin{figure}[htb]  \centering  \includegraphics[width=0.8\\linewidth]{Image00244.jpg}  \caption{图片 148: Image00244.jpg}\end{figure}\begin{figure}[htb]  \centering  \includegraphics[width=0.8\\linewidth]{Image00022.jpg}  \caption{图片 149: Image00022.jpg}\end{figure}\begin{figure}[htb]  \centering  \includegraphics[width=0.8\\linewidth]{Image00216.jpg}  \caption{图片 150: Image00216.jpg}\end{figure}\begin{figure}[htb]  \centering  \includegraphics[width=0.8\\linewidth]{Image00286.jpg}  \caption{图片 151: Image00286.jpg}\end{figure}\begin{figure}[htb]  \centering  \includegraphics[width=0.8\\linewidth]{Image00209.jpg}  \caption{图片 152: Image00209.jpg}\end{figure}\begin{figure}[htb]  \centering  \includegraphics[width=0.8\\linewidth]{Image00190.jpg}  \caption{图片 153: Image00190.jpg}\end{figure}\begin{figure}[htb]  \centering  \includegraphics[width=0.8\\linewidth]{Image00145.jpg}  \caption{图片 154: Image00145.jpg}\end{figure}\begin{figure}[htb]  \centering  \includegraphics[width=0.8\\linewidth]{Image00300.jpg}  \caption{图片 155: Image00300.jpg}\end{figure}\begin{figure}[htb]  \centering  \includegraphics[width=0.8\\linewidth]{Image00023.jpg}  \caption{图片 156: Image00023.jpg}\end{figure}\begin{figure}[htb]  \centering  \includegraphics[width=0.8\\linewidth]{Image00353.jpg}  \caption{图片 157: Image00353.jpg}\end{figure}\begin{figure}[htb]  \centering  \includegraphics[width=0.8\\linewidth]{Image00142.jpg}  \caption{图片 158: Image00142.jpg}\end{figure}\begin{figure}[htb]  \centering  \includegraphics[width=0.8\\linewidth]{Image00002.jpg}  \caption{图片 159: Image00002.jpg}\end{figure}\begin{figure}[htb]  \centering  \includegraphics[width=0.8\\linewidth]{Image00226.jpg}  \caption{图片 160: Image00226.jpg}\end{figure}\begin{figure}[htb]  \centering  \includegraphics[width=0.8\\linewidth]{Image00248.jpg}  \caption{图片 161: Image00248.jpg}\end{figure}\begin{figure}[htb]  \centering  \includegraphics[width=0.8\\linewidth]{Image00181.jpg}  \caption{图片 162: Image00181.jpg}\end{figure}\begin{figure}[htb]  \centering  \includegraphics[width=0.8\\linewidth]{Image00113.jpg}  \caption{图片 163: Image00113.jpg}\end{figure}\begin{figure}[htb]  \centering  \includegraphics[width=0.8\\linewidth]{Image00233.jpg}  \caption{图片 164: Image00233.jpg}\end{figure}\begin{figure}[htb]  \centering  \includegraphics[width=0.8\\linewidth]{Image00033.jpg}  \caption{图片 165: Image00033.jpg}\end{figure}\begin{figure}[htb]  \centering  \includegraphics[width=0.8\\linewidth]{Image00017.jpg}  \caption{图片 166: Image00017.jpg}\end{figure}\begin{figure}[htb]  \centering  \includegraphics[width=0.8\\linewidth]{Image00238.jpg}  \caption{图片 167: Image00238.jpg}\end{figure}\begin{figure}[htb]  \centering  \includegraphics[width=0.8\\linewidth]{Image00106.jpg}  \caption{图片 168: Image00106.jpg}\end{figure}\begin{figure}[htb]  \centering  \includegraphics[width=0.8\\linewidth]{Image00391.jpg}  \caption{图片 169: Image00391.jpg}\end{figure}\begin{figure}[htb]  \centering  \includegraphics[width=0.8\\linewidth]{Image00009.jpg}  \caption{图片 170: Image00009.jpg}\end{figure}\begin{figure}[htb]  \centering  \includegraphics[width=0.8\\linewidth]{Image00041.jpg}  \caption{图片 171: Image00041.jpg}\end{figure}\begin{figure}[htb]  \centering  \includegraphics[width=0.8\\linewidth]{Image00151.jpg}  \caption{图片 172: Image00151.jpg}\end{figure}\begin{figure}[htb]  \centering  \includegraphics[width=0.8\\linewidth]{Image00290.jpg}  \caption{图片 173: Image00290.jpg}\end{figure}\begin{figure}[htb]  \centering  \includegraphics[width=0.8\\linewidth]{Image00273.jpg}  \caption{图片 174: Image00273.jpg}\end{figure}\begin{figure}[htb]  \centering  \includegraphics[width=0.8\\linewidth]{Image00060.jpg}  \caption{图片 175: Image00060.jpg}\end{figure}\begin{figure}[htb]  \centering  \includegraphics[width=0.8\\linewidth]{Image00104.jpg}  \caption{图片 176: Image00104.jpg}\end{figure}\begin{figure}[htb]  \centering  \includegraphics[width=0.8\\linewidth]{Image00335.jpg}  \caption{图片 177: Image00335.jpg}\end{figure}\begin{figure}[htb]  \centering  \includegraphics[width=0.8\\linewidth]{Image00183.jpg}  \caption{图片 178: Image00183.jpg}\end{figure}\begin{figure}[htb]  \centering  \includegraphics[width=0.8\\linewidth]{Image00195.jpg}  \caption{图片 179: Image00195.jpg}\end{figure}\begin{figure}[htb]  \centering  \includegraphics[width=0.8\\linewidth]{Image00280.jpg}  \caption{图片 180: Image00280.jpg}\end{figure}\begin{figure}[htb]  \centering  \includegraphics[width=0.8\\linewidth]{Image00220.jpg}  \caption{图片 181: Image00220.jpg}\end{figure}\begin{figure}[htb]  \centering  \includegraphics[width=0.8\\linewidth]{Image00014.jpg}  \caption{图片 182: Image00014.jpg}\end{figure}\begin{figure}[htb]  \centering  \includegraphics[width=0.8\\linewidth]{Image00346.jpg}  \caption{图片 183: Image00346.jpg}\end{figure}\begin{figure}[htb]  \centering  \includegraphics[width=0.8\\linewidth]{Image00328.jpg}  \caption{图片 184: Image00328.jpg}\end{figure}\begin{figure}[htb]  \centering  \includegraphics[width=0.8\\linewidth]{Image00160.jpg}  \caption{图片 185: Image00160.jpg}\end{figure}\begin{figure}[htb]  \centering  \includegraphics[width=0.8\\linewidth]{Image00289.jpg}  \caption{图片 186: Image00289.jpg}\end{figure}\begin{figure}[htb]  \centering  \includegraphics[width=0.8\\linewidth]{Image00370.jpg}  \caption{图片 187: Image00370.jpg}\end{figure}\begin{figure}[htb]  \centering  \includegraphics[width=0.8\\linewidth]{Image00385.jpg}  \caption{图片 188: Image00385.jpg}\end{figure}\begin{figure}[htb]  \centering  \includegraphics[width=0.8\\linewidth]{Image00012.jpg}  \caption{图片 189: Image00012.jpg}\end{figure}\begin{figure}[htb]  \centering  \includegraphics[width=0.8\\linewidth]{Image00376.jpg}  \caption{图片 190: Image00376.jpg}\end{figure}\begin{figure}[htb]  \centering  \includegraphics[width=0.8\\linewidth]{Image00338.jpg}  \caption{图片 191: Image00338.jpg}\end{figure}\begin{figure}[htb]  \centering  \includegraphics[width=0.8\\linewidth]{Image00242.jpg}  \caption{图片 192: Image00242.jpg}\end{figure}\begin{figure}[htb]  \centering  \includegraphics[width=0.8\\linewidth]{Image00352.jpg}  \caption{图片 193: Image00352.jpg}\end{figure}\begin{figure}[htb]  \centering  \includegraphics[width=0.8\\linewidth]{Image00304.jpg}  \caption{图片 194: Image00304.jpg}\end{figure}\begin{figure}[htb]  \centering  \includegraphics[width=0.8\\linewidth]{Image00230.jpg}  \caption{图片 195: Image00230.jpg}\end{figure}\begin{figure}[htb]  \centering  \includegraphics[width=0.8\\linewidth]{Image00187.jpg}  \caption{图片 196: Image00187.jpg}\end{figure}\begin{figure}[htb]  \centering  \includegraphics[width=0.8\\linewidth]{Image00156.jpg}  \caption{图片 197: Image00156.jpg}\end{figure}\begin{figure}[htb]  \centering  \includegraphics[width=0.8\\linewidth]{Image00359.jpg}  \caption{图片 198: Image00359.jpg}\end{figure}\section{文件 15}\
%第四章 新时期国内矿石机的研究和进展 二、测量篇 3DQ、3DP的测试 李清 我手头有两种飞利浦的MOS管,一种是3DQ,另一种是3DP,我对这两种管子进行了测试,得到的数据如表4-1和表4-2所示。 表4-1 3DQ测试数据 表4-2 3DP测试数据 3DQ的栅漏极转移特性曲线如图4-52所示。3DP的栅漏极转移特性曲线如图4-53所示。 从上面的数据和曲线上可已看出,两种管子都能做矿石机检波,但是性能不一样,3DQ的零栅压漏极电流远小于3DP的零栅压漏极电流,所以3DP管对有载Q值的影响远大于3DQ,从这一点上看3DQ更好些。从特性曲线上看,零栅压时3DP的跨导更高些,从这点看,3DP又优于3DQ。因此这两种管子各有千秋,实际使用时或许3DQ会更好些。 在“雨伞大环矿石机”上实际收听时,这两种管子声音都很大,晚上10点试验完成后,我忘记将MOS管取下来,第二天一早醒来后感觉卧室窗边有声音,以为是有老鼠,过去一看,声音是雨伞大环上的“大罐头”耳机发出的声音,是用3DQ检波的。 实际收听实验我还发现,3DQ检波的输出阻抗为10~20kΩ,而零栅压漏极电流是0的增强型MOS管的输出阻抗要高得多,大约是四五十千欧。 图4-54所示为测试场景。雨伞大环矿石机实物及电路如图4-55所示。 图4-52 3DQ的栅漏极转移特性曲线 图4-53 3DP的栅漏极转移特性曲线 图4-54 测试场景 注: (1)上述测试得到的数据仅对测试样品负责,所得结论不一定有代表性。 (2)上述测试均在业余条件下进行,受仪表、环境等因素的影响,所得到的数据未必准确,因此不做测试精度的讨论与争论。 图4-55 用万用表判断MOS管的方法 李清 用MOS管检波是近几年来矿石机技术的新方向,但是并不是所有的MOS管都可以做矿石机的检波,而只有一部分能做检波之用。这一段时间我做了不少次MOS管检波的试验和测试,在实验中摸索到了一个用指针式万用表判断MOS管是否能做矿石机检波的简单方法。现以N沟道双栅MOS管为例介绍一下。 (1)指针万用表放在R×1k挡位,用一条短导线的一端将双栅MOS管的两个栅极连接在一起,另一端暂时悬空不接,万用表的正表笔接MOS管的源极(S极),黑表笔接到MOS管的漏极(D极),如图4-56(A)所示。 用手做接近和远离栅极导线的动作,如果万用表针随手的动作而摆动则管子是好的,就可以往下继续测试了。 (2)万用表与MOS管保持上述的连接不变,将栅极导线悬空的一端连接到管子的源极上,如图4-56(B)所示。 图4-56 用万用表判断MOS管的方法 这时万用表的指示可能有3种情况。 a.表针指示电阻值接近0Ω,如图4-57所示。 这表示该MOS管是关断栅压较高的耗尽型管子,大多不能做矿石机的检波使用,即使有的管子勉强能用效果也不好。 b.表针指示电阻值为无穷或者很大(数百千欧),如图4-58所示。 图4-57 万用表指示电阻接近0Ω 图4-58 万用表显示电阻值为无穷大或很大 这表示该管是增强型的管子,这种管子大多数是可以用作矿石机检波的,但是效果好坏会有不同的差别。 c.万用表针指示电阻值为在数百千欧姆到数十千欧姆,如图4-59所示。 图4-59 万用表指针指示数百千欧姆至数十千欧姆 这表示该管是关断栅压较小的耗尽型管子,性能接近零开启栅压的增强管,适合检波使用。 现代的MOS管栅极一般都有保护电路,以上方法在大多数情况下不会损坏管子,但不是百分之百安全,因此最好带上防静电手环操作,如果没有防静电手环也可以在操作之前用手触摸一次地线,释放掉静电后再测试。 场效应管小信号检波计算 许建伟 当V小于40mV时,存在以下相对严格的关系: I=β(Vg +Va )V 式中β是场效应管的增益系数,Vg 是栅极与源极之间的电压,Va 是夹断电压,V是漏极电压,I是漏极电流,显然源漏小信号电导是g=β(Vg +Va )。 零点电导g0 =βVa ,3DQ的零点电导g 0 =0.5(kΩ-1 ),零点电阻Rd =1/g0 =2 kΩ Vg =70mV,V=14.3mV,I=18μA,g=1.26kΩ-1 Vg =36mV,V=14.8mV,I=12μA,g=0.81kΩ-1 Vg =0mV,V=15.3mV,I=8μA,g0=0.52kΩ-1 Vg =-36mV,V=33.0mV,I=7μA,g=0.21kΩ-1 令36mV时的g为g1 ,−36mV时的g为g2 ,那么有: g1 -g2 =β(Vg +Va )−β(−Vg +Va )=2βVg 因此β=(g1 −g2 )/Vg /2=(0.81−0.21)/0.036/2/1000=0.0083 Va =g0 /β=0.52/1000/0.0083=0.063V=63mV 即3DQ的夹断电压Va =63mV 关于零点检波 I=β(Va +Vg )V=βVa (1+Vg /Va )V=g0 (1+Vg /Va )V 在梁氏3DQ机内,设初级L1输出电压是A×sinωt,L1与L2的电压变换比是k∶1,检波输出电压为V0 ,检波负载电导为gx ,那么就有: 谐振电压A sin ωt,而电路中栅极接线圈热端,所以V g 为谐振电压,即V g =A sin ω t。 漏源上得到的电压是L2的电压减V0 ,即V=A×sin ωt/k− V0 ,代入得 I=g0 (1+A sinωt/Va )(Asinωt/k− V 0 ) 通过积分计算,可得一个周期的平均电流是:I2 =g 0 ×A2 /(2kVa )−g0 ×V0 检波平均电流与负载电流相等,所以:g0 ×A2 /(2kVa )−g0 ×V0 =gx ×V0 解得V0 =A2 /(2kVa )×g0 /(g0 +gx )……场效应管小信号检波方程 显然,检波输出与信号高频信号幅度之间的关系是平方关系,属于平方检波。以上计算成立的条件是:谐振电压小于夹断电压Va ,即A<Va 。当A>Va ,将逐渐变为线性检波。 关于检波方程 检波方程: V0 =A2 /(2kVa )×g0 /(g0 +gx ) 式中A是谐振峰值电压,k是L1与L2的电压比,g0 是场效应管一零点电导,gx 是负载电导,Va 是夹断电压。如果写成电阻形式,g0 =1/Rd ,gx =1/Rx ,式中Rd 是零点电导,Rx 是负载电阻,那么 V0 =A2 /(2kVa )×Rx /(Rd +Rx ) 显然,这是一个串联分压电路方程,其中检波输出电动势是U=A2 /(2kVa ),检波输出内阻是零点电阻Rd 。高频的零点输入电阻也是零点电阻Rd 。 与二极管小信号检波电压效率对比 二极管检波电动势U=A2 /(4uT),电压变换效率是c=U/A=A/(4uT) 场效应管检波电动势U=A2 /(2kVa ),c=U/(A/k)=A/(2Va ) 3DQ的电压变换效率:c=A/(2×0.063)=A/0.126 二极管的电压变换效率:c=A/(4×0.026)=A/0.104 平方检波原理与Rd 许剑伟 当信号很小时,主要靠二极管非线性的平方项来检波,高次项非常小,可以忽略,所以也称为“小信号平方律检波器”。 设y=Vsin(t),经平方计算得到U=y2 =V2 (1−cos2t)/2,滤波后高频分量cos2t部分滤除,输出得到低频信号U=V2 /2,因此,平方律检波器有一个特点:输出信号U与输入信号V的平方成正比。 1.二极管方程及零点Rd 二极管方程是I=Is =[exp(V/UT )−1],式中UT 是热电压(与热力学温度成正比),V是二极管的压降,I是二极管的电流,Is 是反向电流,常温下UT 是26mV。由于实际二极管不是理想二极管,所以实测UT 会比26mV稍大一些,实际UT 与标准UT 的比值称为理想因子n。如,1ss86、bat85、1n60等,UT 为26~28mV,n接近于1。1N4148的UT 可高达50mV,n=1.9。 二极管的微变电阻Rd ,可以由二极管方程求导得到: 或者写为 当输入信号接近于0,用V=0代入得零点电阻为 ,下文Rd 指零点Rd 。 例:测得二极管UT =26mV,Is =400nA(0.2V反向加电压测得),求Rd 值。 因为Is 是在较低电压下测得的,接近于零点状态,所以 。 2.小信号条件下二极管方程及检波输出 记 ,使用马克劳林级数将二极管方程展开得到 当V是小于UT 的小信号,x为小量,我们可以忽略x的3次以上高阶小量,即非线性部分是平方项x2 /2,其他项忽略。 图4-60所示是一个典型的检波电路。 图4-60 设输入高频信号为V=A sin ωt,周期为T,此电路中,二极管上的压降不是V,而是V− U,所以 负载电流等于U/R,等于一周期内流过二极管的平均电流,于是 对V求积,一周期内积分结果为0,所以 因为U的平方是小量,所以 式 (1) 就是小信号检波输出公式。输出电压U与输入高频电压A之间是平方关系。这是一个分压电路公式,即电源电动势为 ,电源内阻是Rd ,负载电阻是R。当R=Rd 时,负载得到最大功率。因此,矿石收音机制作中,耳机阻抗R接近于二极管Rd 时,可以获得最大功率。通常,使用自制变压器进行阻抗变换,使得Rd =R。 检波效率 , 当R为空载时,取得最大效率 式(2) 检波效率η反应二极管的检波性能,UT 小的二极管,检波效率高。以上计算,没有考虑二极管的频率响应问题。受频响限制,随着频率升高,检波效率会下降。BAT85管子,适用于中波检波,当频率达到10MHz以上时,检波效率将会快速下降。 例:如图4-58所示电路图,输入正弦信号峰值为26mV,UT =26mV,Rd =R=50kΩ,求检波输出电压U及负载上的功率P。 可见,输出信号功率十分微弱,以上计算仅计算出直流分量,广播音频分量还会更小(小10倍量级)。并非所有耳机在0.21nW功率下都能发出声音,所以高灵敏度的矿石收音机,应选用高灵敏度的耳机。 3.关于输入阻抗 图4-61所示是二极管检波器输入阻抗与正弦输入电压之间的关系。 图4-61 检波器输入阻抗与输入电压的关系 由图4-61所示得知,不管二极管输出接上轻负载还是重负载,信号较弱时(输入峰值小于100mV),检波器的输入阻抗总是接近于零点电阻Rd 的。简化计算中,可以认为小信号时,负载与输入相互独立,输入阻抗就是Rd ,与负载无关。根据这一特点,检波后进行音频阻抗匹配,只需做到Rd 等于音频负载阻抗,就达到了音频匹配,而高频部分的阻抗匹配另行计算。 从能量转换的角度来看,小信号检波中,由于检波效率低下,高频信号输入到二极管时,大部分能量损耗在二极管零点Rd 上,不管音频负载重点或轻点,都只有少量能量转换为音频信号,改变不了Rd 损耗大部分能量的事实,因此输入阻抗按Rd 计算便可。 如图4-62所示,设高频信号内阻为Rp ,音频已实现阻抗匹配Rd =R,当二极管Rd 取何值,小信号时,负载R上取得最大功率P。 图4-62 设信号源幅度为A,二极管输入端电压为A′,因二极管输入阻抗为Rd ,所以二极管输入端所得分压为 输出电压U可由式(1)计算,所以 把A′代入得到 当 时,P取得最大值,即 解得,Rd =3Rp 时,负载R可以取得最大功率。 图4-63所示是根据式 (3) 绘图,系数部分 看作单位1绘得,横坐标是Rd /Rp ,纵坐标是输出功率P。 图4-63 由该图可以看出,当Rd =3Rp 时,P为最大值。当Rd >1.5Rp ,输出功率就接近最大值了。 制作矿石收音机时,选取Rd >1.5Rp 的二极管,可以取得较高的灵敏度,而且,随着Rd 取值的增加,谐振回路负载变轻,Q值提升,选择性变好。 4.Rd 相关的阻抗匹配综述 不管高频输出阻抗是多少,Rd 等于音频负载时(Rd =R),负载得到的功率最大。 不管音频负载阻抗是多少,Rd 等于3倍高频输出阻抗时(Rd =3Rp ),检波输出的功率最大。 5.Rd 的测量 Rd 是矿石收音机阻抗匹配的重要参数。因此有必要实测,方便做好矿石收音机阻抗匹配。 由于现代检波二极管的频响都很好,所以,从直流到10MHz频率(有些管可达100MHz以上),二极管的伏安特性都遵守二极管方程。那么,用高频法测量Rd 与直流法测量Rd ,所得结果非常接近。推荐使用直流法测量,图4-64所示为简易测法,Rd 计算公式由二极管方程导出,实际测量时,使用一块万用表即可。为了简化计算,一般使用Excel辅助计算。 上述测量有个缺点——无法直读Rd ,因此测量一次Rd 很烦琐。图4-65所示为改进电路,是LC测量兼Rd 测量双用表,该表可以直读二极管Rd 。 图4-64 二极管uT、Rd 测量仪 该表利用谐振测量电感量,量程很宽,可测得0.1μH~2000H电感,同时可以测量二极管的Rd 值,它利用单片机,测量出二极管两点上的伏安值,然后代入二极管方程,即可算出Rd 的值。 图4-66所示为爱好者制作的Rd /LC表。 数字Q表 Q表可以用来测量线圈的Q值。高灵敏度的矿石收音机需要使用高Q值线圈,目前,矿石收音机使用线圈的最高Q值约为1800。商品表Q值量程一般只能做到1000,因此,商品表用于制作高品质矿石收音机有点尴尬,大于1000就测不了,而且价格不菲。业余条件下可以DIY Q表,以满足矿石收音机制作的需要。本文介绍的这款表可以测得2000以上Q值,精度与现代商品Q表相当(精度优于2\\%)。 Q表除了可以测量线圈Q值,还可以测量电感量、电容量等参数。 图4-65 改进后的电路 数字式Q表带有扫频功能,能自动查找谐振点,因此,DIY过程中,对减速调谐器的要求不高。 图4-67中,AD9833为DDS信号发生器,输出信号送入56∶1磁环变压器初级,该变压器次级只有一匝,提供极低阻抗的信号输出,激励谐振器。谐振器是由500pF高Q可变电容与被测电感构成的。谐振器的谐振电压由2sk117场效应管缓冲放大,再经毫伏表放大、检波,由单片机内置AD转换器测量出检波电压,从而计算出谐振电压V1 。校准时,测出DDS输出信号V2 ,那么谐振器的Q值就是56×V1 /V2 ,由于可变电容高Q,激励变压器低阻,它们的损耗可以忽略,那么谐振器Q值就是被测电感的Q值。 图4-68所示为两位爱好者制作的两款Q表。 图4-66 爱好者制作的Rd /LC表 图4-6 750kHz至11MHz高频Q表(带自动关机) 图4-68 两位爱好者制作的Q表 图4-69所示为使用图4-67所示的第二款Q表进行实测操作。 图4-69 使用图4-67所示的第二款Q表进行实测操作 矿石机的实测 李清 矿石机爱好者在矿石机的制作和实验中往往需要用仪表对矿石机进行一些必要的测量。大家使用的大都是些简陋的二手仪表,有些甚至是自制的仪表。而且,这些测量都是在业余条件下进行的,并无任何标准可以依靠,也根本谈不上精度,因此请读者对于这部分内容切莫过于较真。 本文中几个测量实例所使用的仪表如下: 指针式万用表:MF47 数字万用表:DT9204 数字电感电容电阻表:VICTOR 6243+ 二极管Rd 表(许剑伟老师设计制作) 电阻箱:ZX21,串号为416 高频信号发生器:MSG-2560B,串号为9007 单圈环形天线(带29dB天线放大器):自制 音频信号发生器:GFG-8019G,串号为2620061 超高频毫伏表:AS2271,串号为JD063 超高频毫伏表:ZN2270,串号为88049 双针音频毫伏表:LMV-189AR,串号为9080128 Q表型号:HP4342A,串号为84-45028 1.万用表测量二极管的方法和测试实例 在制作或者实验矿石机时,往往需要检查二极管是否损坏,用普通的指针式万用表的电阻挡或是数字万用表的二极管测量挡都可以完成这一任务。 下面先介绍使用指针式万用表的电阻挡测量二极管的方法。 (1)判断二极管的极性 用指针式万用表的电阻挡测量二极管时可以方便地判断出二极管的正负极,当指针式万用表的正表笔接二极管的负极,负表笔接二极管的正极时,测到的是二极管的正向电阻,此时表针指示的电阻值很小,故交换表笔测量两次二极管的电阻,在测量到二极管电阻值小的时候,正表笔接的就是二极管的负极,负表笔接的是二极管的正极。 (2)二极管是否损坏的判断 用指针万用表的R×1k挡测量二极管的正反向电阻,正常管子的正向电阻值大约是数千欧,反向电阻值约在数百千欧。由于二极管是非线性器件,测到的正向电阻值与测试时流过二极管的电流大小有关,故同一只二极管用不同的万用表或是同一只指针式万用表不同的电阻挡位,测到的正向电阻值是不一样的。这是因为用不同的万用表或是同一只万用表不同的电阻挡位测量二极管时流过二极管的电流值不同,所以呈现的正向电阻值不同。当然,用同一只表的同一电阻挡位测量不同二极管的正向电阻也是不一样的,这都是正常现象。 在测量中,我们当然希望二极管的正向电阻越小越好,正向电阻越小,表示正向压降越小,对检波有利。如果测到的正向电阻较大,那么很有可能被测管的是硅管,硅二极管的正向结电压比锗管和肖特基管的正向结电压高很多,所以做小信号检波时效率会很低,因此它不适于用作矿石机检波器件。用作矿石机检波我们还希望反向电阻越大越好,反向电阻越大表示二极管的反向漏电流越小,管子的性能越好。反向漏电流大了,二极管工作就不稳定,同时也会使检波效率下降,因此应避免使用。 如果测到了下面几种情况时就表示二极管已经损坏了: ⅰ.二极管正反向电阻差别很小; ⅱ.正向电阻很大; ⅲ.反向电阻很小; ⅳ.测正向电阻时表针就不动。 用指针式万用表测量二极管的正向电阻时,虽然知道正向电阻越小,管子的正向压降就越小,正向电阻越大,管子的正向压降就越大,但是并不能从测到的正向电阻值推断出管子具体的正向压降值,也就不知道测到的正向电阻值达到多少就是硅管,小于多少就不是硅管。我们可以测量一只正常的硅管的正向电阻,并以此值作为参考值,做一个大致的比较也就能判别出来了。如果要较准确测量二极管的正向压降值,就要用数字万用表的二极管测量挡。 (3)用数字万用表测量二极管 数字万用表都有二极管的测量挡,因此测量二极管也比较方便。 用数字表的二极管测量挡时,正表笔接二极管正极,负表笔接二极管负极,测到的读数是二极管的正向压降。交换两只表笔后,测量的是二极管的反向情况,表上的读数应该是二极管的反向电压值。由于一般数字万用表的表内电池只有9V,所以如果被测管反向电压小于9V,表上的读数就是管子的反向耐压值,如果被测管的反向电压大于9V,用数字表就测不出来了,这种情况下表上显示的数是发生“溢出”的读数。 一般情况下,锗管的正向压降在0.3V左右,肖特基管的正向压降在0.4V以下,有的能低到0.2V左右的,硅管的正向压降较高,在0.6~0.7V,所以如果测得的正向压降值超过了0.5V,被测管很可能就是硅材料的二极管。硅二极管正向压降高,用硅二极管检波的矿石机虽然在接收强信号时也能收听,但是检波灵敏度很低,收不到弱信号,因此硅二极管不适于做矿石机检波。 用数字万用表二极管测量挡测量二极管时,如果发现下述情况,表明二极管已经损坏。 ⅰ.被测二极管的正向压降值严重大于或严重小于该种管子的正常值; ⅱ.被测二极管的反向电压值明显地小于该种管子的正常值; ⅲ.测量值为零。 值得注意的是,无论是使用指针式还是数字式万用表,测量二极管,只能判断二极管是否损坏,不能判断出二极管检波效果的好坏。用指针式万用表,测量正反向电阻差别小的二极管,效果肯定不好,但是测得正反向电阻差别大的二极管,检波效果未必一定就好。常常可以遇到这样的情况:测量某些二极管的正向电阻和反向电阻都很正常,但实际装机时,发现这种二极管检波效果并不好。这是因为二极管检波效果好坏,是由二极管的多项指标性能共同确定的,单单从万用表测到的正反向电阻的数值上,反映不出二极管检波效果。因此,如果矿石机原来好好的,有一天忽然不响了,那我们就可以用指针式万用表测量检波二极管正反向电阻的方法,判断是否是检波管损坏了。但是,如果我们得到了几只从来没有用过的二极管,那我们就不能用指针式万用表测量检波二极管正反向电阻的方法,来判断哪只管子检波效果最好,至多,我们只能用万用表看一下里面是不是有坏管子混了进来。 同理,用数字万用表的二极管测量挡,虽然可以测出二极管的正向压降,但是这也只能作为判断二极管是否损坏的依据,不能由此判断出二极管的检波效果来。我们当然希望检波管的正向压降越小越好,但是,并不能说正向压降小的管子检波效果一定就比正向压降大的管子好。例如,用数字万用表的二极管测量挡,去测二极管1SS86和1N60,就能发现1SS86的正向压降比1N60低很多,但实际在矿石机上试一试就知道了,对于中波的弱信号检波来说,1N60要比1SS86效果好很多。其原因就在于1SS86的Rd 值远低于1N60,用1SS86检波时,矿石机调谐回路的有载Q值远低于使用1N60检波时的值。 2.二极管Rd 的测试实例 对于矿石机来说,检波二极管的零点等效电阻Rd 是非常重要的,Rd 即是二极管检波器的输入阻抗,是调谐回路的负载,同时也是检波器的输出阻抗,音频负载需要与其匹配。知道了检波二极管的Rd 值才能做到心中有数,从而合理安排调谐回路、检波器和音频负载三者之间的最佳阻抗匹配,以求得矿石机最好的灵敏度和选择性。 关于二极管Rd 测量,可以通过搭建一个简单的电路,使用灵敏度较高的直流电压表测量几个电压值,然后计算出Rd 值,这个方法在本书前面已有介绍。 用专用Rd 表测量二极管Rd 值非常方便,可以直接显示Rd 值。Rd 表很少见,如果没有Rd 表,可以用许剑伟老师的方法,搭建一个简单的电路,通过对二极管的测量和计算得到Rd 值。 图4-70所示是用Rd 表测量几只二极管的实例。 图4-70 二极管的Rd 值,可以让我们知道这只二极管用作检波时,对调谐回路有载Q值的影响,以及其Rd 值是否符合我们的匹配要求,但是,我们不能通过Rd 值确定二极管检波的效果。并不是Rd 值高的二极管检波效果一定就好,二极管检波效果好坏是由二极管的多项指标性能共同确定的。 3.矿石机调谐回路输出电压的测试实例 在矿石机的实验中,有时为了比较不同的检波元件对调谐回路的影响、不同形式的调谐回路之间的比较等,经常需要对调谐回路的谐振电压进行测量。 这类测量面临最大的问题,就是如何克服测量电路对调谐回路的影响,众所周知,矿石机调谐回路的负载能力极弱,为了减小测量仪表对调谐回路工作状态的影响,就希望接入的测量电路有极高的输入阻抗和很小的分布电容。就这点而言,一般的高频毫伏表的输入阻抗,是不能满足测量要求的,对于矿石机的调谐回路而言,在谐振时接入了一个输入阻抗只有几十千欧的高频毫伏表,调谐回路的Q值就会跌到“惨不忍睹”的地步,此时测到的谐振电压与未接入仪表之前的谐振电压已是大相径庭了,从而就失去了测量的意义。 其实,任何测量方法都会对被测对象造成一些影响,产生测量误差,问题是要把这种影响控制在可以容忍的范围内,测量的结果才有意义。 在对矿石机调谐回路的实际测量中,为了减小测量仪表对调谐回路的影响,我们可以采用电容分压的方法。 如图4-71所示,由微调电容C1和C2串联组成电容分压器。 图4-71 电路分压器 在这个电容分压器中,如果电容C1和C2都使用品质很高的空气介质的微调电容,为了减小高频毫伏表输入阻抗的影响,高频毫伏表测量的是容量比较大的分压电容C2两端的电压,安装了这个电容分压器后,高频毫伏表对调谐回路影响就变得比较小了,而且,C1和C2的容量相差越大,对调谐回路的影响就越小。 在实验中,笔者用了两只高频瓷支架的空气微调电容做分压器,C1是2~10pF的,调到容量最小位置,即2pF左右。C2放在容量最大位置,即56pF左右,分压比是二十多(见图4-72)。 图4-72 自制的电容分压器 通过一个实验可以知道,高频毫伏表接上这个电容分压器后,对调谐回路的影响已经很小了,在750kHz频点用HP4342A型Q表测量一只电感的Q值是720。在Q表的被测电容端,接入这一电容分压器,但是不接高频毫伏表时测到的Q值仍然是720,可见这两只电容的质量极好,对Q表的调谐回路基本上没有影响。再连接上了高频毫伏表后测到的Q值是700,下降了20/720=2.8\\%,可见,加上这个电容分压器,高频毫伏表对被测调谐回路的Q值还是有点影响的,但是这一影响已经很小了,不会引起较大的误差。再者,由于这一电容分压器中C1的容量很小,所以C1和C2串联后的容量就很小,电容分压器并联到调谐回路两端时,对调谐回路的频率调谐范围影响也不大,这就使得用高频毫伏表测量谐振回路成为可行的,代价是降低了高频毫伏表的灵敏度。一般的高频毫伏表的最小量程都在几毫伏,用了分压比是几十的电容分压器后,高频毫伏表只能测量几十毫伏以上的高频电压了,好在这对我在实验中测量矿石机来说也基本上够用了。 如果只需要进行一些对比试验,就没必要测量准确的电压值了,只要知道几次测量值的比例关系就行了。例如,测量不同二极管检波时对调谐回路的影响,我们只关心二极管接入后,调谐回路的谐振电压下降了百分之多少,并不关心这一电压的具体值,在这种情况下,这个电容分压器都不用校正,直接使用就可以了。如果需要比较准确的测量值,可以用下面的方法校正使用了电容分压器后高频毫伏表的刻度值。校正电路如图4-73所示。 这一校正很简单,图中高频信号发生器输出中波波段的某频率,例如1MHz,记下用于校正的高频毫伏表A的读数,与使用了电容分压器后的高频毫伏表的读数,计算两只毫伏表读数的倍数关系作为校正系数,校正工作就完成了。图4-74所示就是实际校正这个电容分压器过程的照片。 图4-73 校正电路 图4-74 校正电容分压器 校正是在1MHz频点进行的,使用了两台高频毫伏表,AS2271高频毫伏表是接在电容分压器上的被校正表,ZN2270做标准表。 首先将两台表测量输入端接到同一台信号发生器的输出端,两台表的读数是一致的,两台表的指示都是90mV。 高频毫伏表ZN2270测量高频信号发生器的输出电压,AS2271测量电容分压器的输出电压。当测得信号发生器输出电压是90mV时,高频毫伏表AS2271测量到的电容分压器C2两端电压是2.7mV,因此就可以求得电容分压器的分压系数,是90/2.7=33.3,测量时,我们只要将高频毫伏表AS2271测到的数值乘以分压系数33.3,就可以知道调谐回路两端的谐振电压了。值得注意的是,校正后得到的分压系数只适合被校正的这台高频毫伏表,如果换用了另一台表,还要重新校正,这是因为每台高频毫伏表的探头参数不一样,校正时得到的分压比也是不一样的。 下面是两个测试实例。 (1)磁棒线圈接收单环天线的磁场信号时,调谐回路两端谐振电压的测试。 测试电路如图4-75所示。测试中,高频信号是1MHz的等幅信号,信号发生器输出电平是99dBu,单圈环天线的天线放大器的电压增益是31dB,因此,送到天线的信号电平是99+31=130dBu,因0dB=1μV,故加到单圈环形天线电压值大约是3.15V,天线回路总阻抗是136Ω。 如图4-76所示,被测磁棒放在距离单圈环形天线中心径向55cm(即图4-75中H=55cm)位置,当调谐回路谐振时高频毫伏表的读数是18mV,那么调谐回路此时的谐振电压就是18mV×33.3=599.4mV。 图4-75 测试电路1 图4-76 测试场景 (2)空芯线圈通过全波等效天线接收高频信号时,调谐回路两端谐振电压的测试。 测试电路如图4-77所示。 图4-77 测试电路2 被测调谐回路的线圈数据如下: 骨架:直径50mm的PVC圆管。 绕制数据:共有3个绕组,全部用线径0.38mm高强度线绕制,L1为13匝,L2为76匝, L3为30匝,L2与L1、L3的间距都是5mm。测试中,只用了L1和L2两个绕组,L3未用。 实测数据如下: 当高频信号发生器将频率为1MHz、电平为99dBu(90mV左右)的输出信号,通过全波等效天线送入L1后,调整谐振回路的可变电容达到谐振时,在电容分压器输出端的高频毫伏表测到的电压大约是4mV,由此可知,此时调谐回路的谐振电压就是4mV×33.3=133.2mV。 图4-78 全波等效天线 有了这一测量方法,我们很容易在不怎么影响调谐回路原来的工作状态下,比较不同质量线圈和电容组成的调谐回路之间的差别,也可以用来比较不同Rd 值的二极管对同一个调谐回路的不同影响。 全波等效天线的电路图如图4-78所示。 4.线圈Q值的测试实例 (1)使用Q表测量线圈的Q值。使用Q表测量线圈的Q值,要注意被测线圈要远离金属物体,必要时,用对线圈Q值影响很小的物体将被测线圈架高,或是将被测线圈用丝线吊起来,以便远离Q表外壳。线圈与Q表接线柱的连接一定要可靠。 为了测量准确,测量前,Q表要预热足够长的时间,尤其是电子管Q表需要预热时间更长。 下面是HP4342A型Q表测量两只线圈的例子(见图4-79)。 图4-79 测得的Q值为730 测试频率为1MHz,一只线圈是在直径50mm的PVC管上,用0.38mm漆包线密绕76匝,另一只线圈是用0.04mm×270股利兹线,在20mL注射器针管上密绕65匝,针管内插Φ10mm×100mm的中短波磁棒2只。为了防止Q表的金属外壳影响测量结果,需要用泡沫塑料块将被测线圈垫高。 先将Q表频率定到1MHz处,连接好被测线圈,仔细调整Q表的主电容,尽可能找准谐振点,然后调整微调电容,确保谐振准确。再读出表头指示的Q值数,测得Q=150。 值得指出的是,上述测量中在Q表的表头刻度上读出的测量值实际上是被测电感与Q表的可变电容组成的谐振回路的Q值,而不是仅仅是被测电感的Q值,这一Q值是由电感Q值和电容Q值共同形成的,三者的关系如下:如果电感的Q值是QL ,电容的Q值是QC ,谐振回路的Q值是QX , 则,1/QX =1/QC +1/QL 即,QX =(QC ×QL )/(QC + QL ) 因为Q表中的主可变电容Q值非常高,一般的情况下,这只可变电容的Q值远大于被测电感的Q值,所以,通常情况下,测到的Q表谐振回路的Q值,主要取决于被测电感的Q值,也就是说Q表谐振回路的Q值很接近被测电感的Q值,我们也就认为在Q表的表头上指示的Q值,是被测电感的Q值。 近几年,矿石机爱好者制作的各种高Q值的线圈层出不穷,Q值超过1000的线圈已不少见了,使用Q表测量这些Q值很高的线圈时,虽然,Q表仍能保证主可变电容的Q值大于被测电感的Q值,但是,测量误差会随着被测电感Q值的增大而增大,要想得到更准确的测量值,就要对Q表上读到的测量值进行修正,修正的方法如下。 (2)使用3dB法测量线圈的Q值。 使用Q表测量线圈的Q值是最方便的,如果没有Q表,我们只要有信号发生器、高频毫伏表和数字频率计也可以用3dB法测量线圈的Q值,只不过,测量的过程有点麻烦,但是只要注意一些要点,测到的结果还是很准确的。 我们知道,LC谐振回路的Q值等于谐振频率除以该回路的3dB带宽,所为“3dB带宽”就是指幅值等于最大值的 /2倍时,对应的频带宽度,具体描述如下。 设LC谐振回路的谐振频率为f0 ,当信号频率等于时谐振频率时,谐振回路输出的电压最高,当信号频率高于或者低于谐振频率f0 时,谐振回路的输出电压都会下降,信号频率偏离f0 越多,回路输出的电压就越低,如果把信号频率变低后,输出电压下降了3dB(电压下降到谐振时的 /2倍,也就是0.707倍)时所对应的频率记作fL ,把信号频率变高后,输出电压下降了3dB时所对应的频率记作fH ,这个LC谐振回路在频率f0 点的3dB带宽等于fH 减去fL ,如图4-80所示。 这个LC谐振回路频率f0 处的Q值就是:Q=f0 /(fH −fL )。 如果谐振回路中电容的Q值远大于电感的Q值,谐振回路的Q值就大致等于电感的Q值,这就是3dB法测量电感Q值的原理。 在实际测量中,为了保证测量值的准确,必须注意以下几点: (1)为了减小高频毫伏表输入阻抗对测试回路的影响,要在高频毫伏表的探头前加接高Q电容分压器(见本章第3节)。 图4-80 3dB法测线圈的Q值 (2)为了减小信号发生器输出阻抗对调谐回路的影响,信号发生器与被测LC谐振回路只能保持“松耦合”。 (3)信号发生器应能输出足够强的信号,以满足在松耦合下,被测回路得到足够强的信号。 (4)如果不是数显信号发生器,应使用数字频率计观察测试信号的频率,以保证信号频率的准确。 (5)如果被测回路的Q值很高,3dB带宽往往只有数千乃至数百赫兹,因此,在测高Q线圈时,要求信号发生器的频率能进行较细微的调节,被测回路的可变电容也要安装有效的缓动装置。 下面就是用3dB法实测Q值的实例,被测电感还是前面用Q表测过的那只磁棒线圈。 高频信号发生器输出端接单圈环形天线,天线与被测线圈保持足够的距离以便形成松耦合,与被测电感组成谐振回路的可变电容,是一只带减速比为50∶1减速器的镀金高Q值双联,两联并联后的容量是360pF,如图4-81所示。 用带有电容分压器的高频毫伏表,测量调谐回路的谐振电压。在750kHz频点上仔细调整谐振点,找到准确的谐振点后,调整天线与线圈的距离,使高频毫伏表的读数为30mV,记作f0 =750kHz,如图4-82所示。 然后下调信号发生器的信号频率,高频毫伏表的读数也会随之下降,当高频毫伏表的读数下降到3dB位置时,也就是0.707×30=21mV时,记下此时信号发生器的频率为fL ,本次测到的fL =749.5kHz。 图4-81 两联并联 图4-82 测量回路 然后将高频信号发生器的频率,再从750kHz向高调整,高频毫伏表的读数也随之下降,当高频毫伏表的读数再次降到3dB位置的21mV时,记下此时信号发生器的频率为fH ,本次测量中使用的信号发生器最小频率调整步距是0.1kHz,测试中发现当频率升高到750.5kHz时,高频毫伏表的读数高于21mV,可是,调整向高再调整一个步距频率为750.6kHz时,高频毫伏表的读数就低于21mV了,750.6kHz可取fH =750.55kHz。 在上述的测量中得到了:f0 =750kHz,fL =749.5kHz,fH =750.55kHz, 计算Q=f0 /(fH −fL )=750/(750.55−749.5)=750/1.05=714。 也就是说,用3dB法测出该磁棒线圈的Q值是714。 值得指出的是,Q值越高,调谐回路的通频带就越窄,用3dB法测量Q值就会越困难,随着被测线圈Q值更高,要求高频信号发生器能够提供更细频率调整步距,稍不注意,就会产生比较大的测量误差。如果是测量Q值不那么太高的电感,用普通的高频信号发生器配合廉价的数字频率计即可,很多产品的音频信号发生器的最高频率都能达到1MHz,只要配合数字频率计也可用于3dB法测Q值,而且,音频信号发生器的输出信号电压往往比较高,这一点对保持松耦合会更有利。 5.可变电容Q值的测试实例 使用Q表测量电容的Q值,不同于测量电感的Q值,在Q表上电感的Q值是一次性测量出来的,而且,测量值是从表盘上直接读出的,而一只电容的Q值在Q表上要分两次测量,共得到4个数据,然后通过计算才能得出最后的结果,整个测量过程比较麻烦。 我们知道,LC谐振回路的Q值,是由这一回路中电感的Q值与电容的Q值共同决定的,谐振回路的Q值Qh 与回路里的电感Q值Ql 、电容的Q值Qc 有如下关系。 1/Qh =1/Ql +1/Qc ,即:Qh =Ql ×Qc /(Ql +Qc ) 一般情况下,电容的Q值比电感的Q值大得多,所以,对于谐振回路的Q值,电感的Q值对其的影响要比电容Q值对其的影响大得多,故在LC谐振回路里,往往是电感的Q值更受到人们的关注,而电容的Q值反倒容易被忽略。 我们平时见到的资料里,很少见到电容Q值测量的介绍,其实,从电容Q值的定义可知,电容Q值公式:Qc =1/ωCR,式中ω是角频率,等于2πf,C为电容量,R为电容内产生能耗的总电阻值,由此可见,电容的Q值不但与电容量相关也是与频率有关的参数。 我们经常见到的是电容的另一个重要参数:“损耗角的正切值tanδ”,其实我们从tanδ的物理意义上就可以知道,它与电容的Q值是互为倒数的关系,即:tanδ=1/Qc ,Qc =1/tanδ。 使用Q表测量电容的tanδ值的方法也是通过两次测量,然后计算出结果,即, tanδ=C1 ×(Q1 -Q2 )/(C1 -C2 )×Q1 ×Q2 因为Qc =1/tanδ,故电容的Q值应该是: Qc =(C1 -C2 )×Q1 ×Q2 /C1 ×(Q1 -Q2 ) 在实际测量中,我们要借助一只辅助电感,在以上的两个公式中,Q1 和C1 分别是在预定的频点上测得的辅助电感的Q值和谐振电容值,Q2 和C2 分别是将被测电容接入Q表的Cx 端,并且在同一频点上重新找到谐振点时,Q表指示的Q值和谐振电容值。 以上的方法中忽略了辅助电感的分布电容,如果被测电容容量比较小时,这个分布电容忽略后得到的测量结果的误差就比较大,如果考虑到辅助电感的分布电容,则电容的Q值公式如下: Qc =(C1 -C2 )×Q1 ×Q2 /(C1 +C0 )×(Q1 -Q2 ) 其中,C0 是辅助电感的分布电容值,这一分布电容值可在测量Q值前使用“两步测量法”在Q表上预先测出。 辅助电感分布电容的测量方法: 第一步:首先确定测试频率f1 和f2 ,要求f2 =2f1 。 第二步:将Q表的频率设置在f1 处,将辅助电感接入Lx 端,调整谐振电容找到谐振点,此时记录谐振的电容值为C1 。 第三步:将Q表的频率设置到f2 处,重新找到谐振点,记录此时的谐振电容值为C2 。 第四步:分布电容值C0 由公式C0 =(C1 -4×C2 )/3计算出。 电容Q值的具体测试方法: 第一步:首先测量辅助电感的Q值与谐振电容值。 (1)将辅助电感连接到Q表的Lx 端。 (2)将Q表测量回路的谐振电容度盘,旋到400~500pF的一个整数位置上。 (3)调整Q表的测量频率旋钮找到谐振点。 (4)记录此时测得的Q值为Q1 ,谐振的电容值为C1 。 第二步:测量被测电容接入后的Q值与谐振电容值。 (1)将被测电容连接到Q表的Cx 端。 (2)调整Q表测量回路的谐振电容旋钮,找到谐振点。 (3)记录此时测得的Q值为Q2 ,谐振的电容值为C2 。 第三步:被测电容的Q值Qc 可以由公式 Qc =(C1 -C2 )×Q1 ×Q2 /(C1 +C0 )×(Q1 -Q2 )计算得出。 注意事项: (1)在测量过程中,判读测到的4个数据时一定要准确,否则计算结果误差会很大。 (2)一次测量4个数据的过程要尽快完成,免得测量环境发生变化产生测量误差。 (3)因为电容Q值的测量过程容易产生较大的误差,最好每只电容多测几次,将测得的结果去掉最高和最低值,在对剩下的数据求平均值。 (4)对于容量较大的被测电容,可以忽略辅助电感的分布电容,但测量容量较小的被测电容时,不要忽略分布电容。 对辅助电感的要求: (1)自制辅助电感时要注意,辅助电感的电感量要能与400~500pF的电容谐振在你希望测试的频率上,如果对测试频率的高低并不太在意,那就不必苛求辅助电感的电感量了。 (2)辅助电感可以用与Q表配套的标准电感,如果没有标准电感,也可以自制辅助电感,自制标准电感的Q值尽量高些为好,但一定要稳定,高Q值和稳定性中,稳定性是第一位的。辅助电感最好能在磁环上绕制,磁环具有磁路封闭的特点,可大大减小外部电磁场造成的干扰。辅助电感的电感量,要与Q表的主电容配合,以满足测试频率为准。 下面是测试实例: 被测可变电容是一只容量标称值为365pF的空气双联可变电容的其中一联,这只可变电容的定片和动片都是铜质的,定片的固定架是高频瓷的。辅助电感用的是一只电感量为100μH标准电感,分布电容为6pF,在750kHz频点,按照前面介绍的步骤测量得到如下数据: 将辅助电感接入Q表的被测电感端,未接入被测电容,仔细调好谐振点,测得辅助电感的Q值为230,记作Q1 =230,此时Q表主电容刻度是460pF,记作C1 =460pF。 将被测可变电容的动片全部旋入,使其容量最大,将可变电容接入Q表的被测电容端,测试频率保持750kHz不变,再次仔细调好谐振点,测得Q=225,记作Q2 =225,此时Q表主电容刻度是88pF。 于是测得可变电容的最大容量C=460−88=372pF。 这只可变电容在容量最大时的Q值: Qc =(C1 −C2 )×Q1 ×Q2 /(C1 +C0 )×(Q1 −Q2 ), =(460−88)×230×225/(460 + 6)×(230−225)=8262 因此,测得这只可变电容在容量最大时的Q值是8262。 被测可变电容如图4-83所示。作辅助电感的标准电感如图4-84所示。测试现场如图4-85所示。 一般来说,Q值比较高的可变电容,都是用高频瓷做定片支架,有的甚至动片轴也是高频瓷的,有的定、动片还采用了镀银或镀金工艺。另外,定动片之间的缝隙大些,也对提高Q值有利,要保持可变电容的高Q值,还要注意保持电容的清洁。 普通收音机用的空气可变电容,由于定片的支架大多是胶木片的,容易吸潮气,或是脏了后其定动片的绝缘电阻就比较低了,严重影响Q值,因此需要及时清洗干燥。 图4-86所示是一些Q值比较高的可变电容。 图4-83 图4-84 图4-85 图4-86 图4-87所示是高频瓷轴、高频瓷支架镀金的高Q可变电容内部结构。 图4-87 Q值比较低的空气介质可变电容如图4-88所示。 图4-88 Q值较低的空气介质可变电容 6.耳机阻抗的测试实例 在矿石机的制作中,为了将检波后的音频信号尽可能多地送到耳机上,就要求耳机与检波器尽量做到阻抗匹配,也就是,要做到耳机的阻抗等于检波器的输出阻抗。这也就要求我们对耳机的阻抗做到心中有数,为此,耳机的阻抗测量就是非常必要的了。 测量耳机阻抗有多种方法,在业余条件下,测量耳机的阻抗采用替代法较方便。 耳机的阻抗与频率有关,不同频率测到的耳机阻抗是不一样的,同一只耳机测试频率越低,测到的阻抗值就越低。 一般情况下,耳机的阻抗是指耳机在1000Hz的正弦信号下的阻抗,所以,测量耳机阻抗时,音频信号发生器的频率应是1000Hz,如果需要耳机在其他频率时的阻抗,就要使用其他频率的正弦信号进行测试。 替代法测量电路如图4-89所示。 图4-89 替代法测量电路 为了使测得的耳机阻抗值较准确,电阻R1的阻值应充分大于被测耳机的阻抗值(例如10倍以上)。R2可用质量好的电位器(例如多圈电位器)或是电阻箱。 首先将开关SW的触点放到“1”的位置,调整音频信号发生器输出的信号幅度,信号幅度不宜过大或过小,用耳朵听耳机发出的声音,不感觉吵,也不感觉很小即可,此时记录下音频毫伏表指示的音频电压值。 将开关SW的触点打到“2”的位置,然后调整R2,使得音频毫伏表指示的音频电压值等于刚才记录下的电压值,此时,电阻箱的阻值就是耳机的阻抗值。 如果R2用的是电位器,则要将开关SW的触点重新打到“1”的位置,用数字万用表的电阻挡测量此时R2的阻值,即为耳机的阻抗值。 如果发现测出的耳机阻抗值不是充分小于R1的阻值,可以换一只阻值更大的R1,然后重新测量一次。 如果发现R2调整到音频电压最大时,仍小于记录值,那就是R2的阻值太小了,换一只阻值大的或是在R2上串连一只电阻再测。 下面是一个实测的例子,测量一只SC2-300型舌簧耳机的阻抗,这种耳机是电话机的受话器,灵敏度比较高,标称阻抗是300Ω。 图4-90所示是几种不同颜色的SC2-300耳机。图4-91所示为实际测量图4-89所示中白色的那只。 图4-90 图4-91 先用数字电感电容电阻表测得这只耳机的电感量和电阻值。 测量耳机的电感量和阻抗时应注意,在耳机的发声孔前面不能有遮盖发声孔的障碍物,否则就测不准。 图4-91中,测出的电感量是57.4mH。图4-92中,测出的耳机的电阻是28Ω。 图4-92 测试电路如图4-89所示,但是没有用开关SW,而是通过改变接线来选择连接耳机或电阻箱。R1用47kΩ电阻,R2用电阻箱。 调整信号发生器信号输出幅度,听到耳机声音较合适时,记录下音频毫伏表的指示是10mV(见图4-93)。用电阻箱替代耳机后,调整电阻箱的阻值,使电阻箱两端电压等于10mV(见图4-94)。当电阻箱的电阻值是290Ω时,音频毫伏表的指示与记录值一致。因此,这只SC2-300在1000Hz的阻抗是290Ω(见图4-95)。 图4-93 图4-94 图4-95 笔者曾经实际测量过二三十只不同厂家生产的SC2-300耳机,发现有些耳机的阻抗值与标称值相差较多,有的实际阻抗只有200Ω左右。 SC2-300耳机的阻抗值与耳机出厂前标调的水平有很大的关系,没有标调好的耳机测到的阻抗就偏低,所以,测量前最好能调整一下耳机中的调整螺丝,把舌簧片的位置调到最佳点,这样调整后,耳机的效果最好,阻抗值也最高。调整的方法也很简单,用数字电感表测量耳机的电感量,边测边调整舌簧位置的螺钉,直到电感表显示的电感量最大为止,这时舌簧的位置就正确了。 有时,在调整SC2-300耳机的舌簧片位置时,会遇到电感量最大值只有三四十毫亨,甚至还要低,再稍微一调电感量就突然变得很低了。这种耳机很有可能是舌簧片已经弯曲变形,调整时,舌簧片还没有调到最佳位置就与极靴发生接触了,造成了磁短路,仔细观察舌簧片在前后极靴中的缝隙就不难发现磁短路点,尤其要注意的是,调整螺钉附近的极靴缝隙,如果有磁短路,就要拆出舌簧片整平,或是想办法调整极靴位置,避免磁短路发生。 下面测量一副万里耳机(见图4-96)的阻抗,这副耳机虽然是几十年前生产的,但是很新,从来没有使用过。 这副耳机的直流电阻标称值是750Ω×2=1500Ω,交流阻抗的标称值是5kΩ×2=10kΩ。测试电路中R1用100kΩ,R2用电阻箱。测试如图4-97所示。 图4-96 图4-97 由图可知,信号发生器输出1000Hz正弦信号,耳机两端信号电压150mV。 如图4-98所示,用电阻箱替换耳机,调整电阻箱使其两端电压为150mV。如图4-99所示,电阻箱读数是9600Ω。因此测量出这副耳机在1000Hz频率的阻抗是9.6kΩ。 图4-98 图4-99 7.匹配变压器阻抗及效率的测试实例 (1)匹配变压器阻抗的测量 图4-100 替代法测量匹配变压器阻抗电路图 测量匹配变压器阻抗也可以用替代法,如图4-100所示。图中电阻R3是变压器的负载电阻,其阻值等于被测变压器次级阻抗值,例如,测一只变压器的次级阻抗为300Ω时初级的阻抗值,R3就用300Ω电阻。再例如,要测仿T725变压器的300Ω阻抗端,在超阻抗2倍时原初级200kΩ端的实际阻抗值时, R3就应该用600Ω电阻跨接在仿T725的300Ω端和0Ω端。 电阻R1的阻值,应充分大于被测变压器的初级阻抗(例如10倍以上),R2要用质量好的优质电位器,如果被测变压器的初级阻抗不是很高,R2也可以用电阻箱,一般的电阻箱的阻值都不太高,需要时可以在电阻箱上串连一只已知阻值的电阻后测试,最后得到的测量值,为这只电阻的阻值加上电阻箱的刻度值。测量时,信号发生器输出1000Hz的正弦信号,因为矿石机输出的音频功率很小,为了测量环境接近使用环境,音频信号发生器的信号功率只要几毫瓦即可。测试过程如下: 首先将开关SW的触点放到“1”的位置,调整音频信号发生器的输出信号,使之达到合适的幅度,此时记录下音频毫伏表指示的音频电压值。 将开关SW的触点打到“2”的位置,然后调整R2,使得音频毫伏表指示的音频电压值等于刚才记录下的电压值。 将开关SW的触点重新打到“1”的位置,用数字万用表的电阻挡,测量此时R2的阻值就是被测变压器的阻抗值。 如果发现测到的耳机阻抗值不是充分小于R1的阻值,可以换一只阻值更大的电阻,然后重新测量一次。 如果发现R2调整到音频电压最大时仍小于记录值,那就是R2的阻值太小了,换一只阻值大的,或是在R2上串连一只电阻再测。 用上述方法,可以测量当匹配变压器任意一个输出端接入不同的阻抗(即接入不同阻值的R3)时,任意一个输入端的实际阻抗值。 在使用高Rd 二极管检波的矿石机时,匹配变压器的初级阻抗很高,使用上述替代法,测量这类变压器的初级阻抗会有很大的误差,原因是,一般音频毫伏表的输入阻抗只有1MΩ,音频毫伏表的输入阻抗不能充分大于变压器初级的阻抗,音频毫伏表输入阻抗对测量对象的分流作用很严重了,解决的办法就是,使用输入阻抗更高的数字毫伏表,或是在原来的毫伏表前面加一级结型场效应管的缓冲放大器,以提高输入阻抗,加上缓冲放大器的音频毫伏表,需要重新校正再使用。 (2)匹配变压器效率的测量 当我们用前面介绍的替代法测得匹配变压器的初级阻抗,便有了测量匹配变压器效率的基础了,在此基础之上,很容易测量出变压器的效率。任何变压器都存在损耗,所以,加到变压器原边的信号功率只有一部分被送到了副边,变压器的次级输出的信号功率与初级信号功率之比就是变压器的效率。 测量匹配变压器效率的电路如图4-101所示。 图中的音频毫伏表A和音频毫伏表B最好是一台双针毫伏表,这样用起来比较方便,也可以用两台单针毫伏表。信号发生器输出的是1000Hz的正弦信号。 图4-101 我们已经测出了变压器的初级阻抗是Rc ,在这个测试电路中,我们测到了音频毫伏表A的读数是Ua ,音频毫伏表B的读数是Ub ,从这些数据中不难看出: 变压器初级的信号功率=(Ua ×Ua )/Rc 。 变压器次级输出的信号功率=(Ub ×Ub )/R3 。 变压器的效率: η=(Ub ×Ub ×Rc )/(Ua ×Ua ×R3 ) (注意:计算时,Ua 和Ub 的单位要统一,R3 和Rc 的单位要统一)。 测量时,音频信号发生器的输出电压不宜过大或过小,能使负载电阻R3上的信号电压与在矿石机上的电压相差不多即可。 下面是一个测试实例,频率为1000Hz,测量一只仿T725匹配变压器,在1.5kΩ阻抗端接入1.5kΩ电阻时,200kΩ阻抗端的实际阻抗以及此时变压器的效率。 为了减小测量误差,变压器初级串入的电阻应远大于变压器的初级阻抗,本次测量中,变压器初级串联的是一只3.3MΩ的电阻(见图4-102)。 图4-102 次级1.5kΩ阻抗端和0Ω端跨接1.5kΩ电阻,信号发生器输出频率为1000Hz,音频毫伏表指示变压器初级电压是200mV。 图4-103所示是将电阻箱与一只标称值为160kΩ的电阻串联,代替变压器的初级线圈,调整电阻箱的阻值,使音频毫伏表的指示电压仍然保持为200mV。 图4-104所示为测得电阻箱与160kΩ电阻的串联值为211.1kΩ,故该变压器在其1.5kΩ阻抗端接入1.5kΩ电阻时,初级200kΩ阻抗端的实际阻抗是211.1kΩ。 图4-105所示是一只仿T725匹配变压器的效率测试照片。照片中,音频毫伏表黑色指针指示的变压器初级200kΩ端的电压是200mV,红色指针指示的负载1.5kΩ电阻上的电压是16.5mV,因为前面已测出200kΩ端的实际阻抗是211.1kΩ,故这只变压器200kΩ端到1.5kΩ端的效率是: η=(Ub ×Ub ×Rc )/(Ua ×Ua ×R3 )=(16.5×16.5×211.1)/(200×200×1.5)=95.8\\% 图4-103 图4-104 图4-105 8.矿石机灵敏度测试实例 在研究和实验矿石机的过程中,经常需要测试矿石机的灵敏度,对于使用天地线的矿石机,要通过“等效天线”将高频信号发生器的信号送入矿石机。而对于不用外接天地线,仅靠机内磁性天线和大框线圈接收信号的矿石机,就需要高频信号发生器通过单环天线发出测试信号了。 按理说,矿石机灵敏度的测试应该有个标准,但是,矿石机已不是工业产品了,所以国家标准中没有关于矿石机的标准。矿石机灵敏度测量方法可参照普通收音机灵敏度测量的方法。 普通收音机灵敏度的定义,是在满足一定信噪比的情况下,使收音机输出功率达到规定的标准值时,需要的输入信号强度。使用外接天线的收音机输入信号强度,用电压值(mV或μV)表示,对于使用磁性天线的收音机,输入信号强度用场强(mV/m)表示。 矿石机没有放大电路,因此,矿石机输出的音频功率要比一般收音机小很多,但是,因为矿石机绝大多数情况下使用耳机收听,所以,即便是输出的音频功率很小,也能满足收听要求。并且,矿石机的噪声很低,因此,只要省略对信噪比的要求,并将输出功率的标准值降低,就可以用测量普通收音机灵敏度的方法测量矿石机的灵敏度了。 测量的关键是要确定矿石机的输出功率标准值。如果是自己做实验,比较自己的几台矿石机的灵敏度,就可以自己定义一个输出功率标准。如果是矿石机爱好者之间,相比较各自矿石机的灵敏度,那就大家商量着定吧。 这一标准输出功率确定的原则应该是:耳机在这个音频功率时发出的声音听上去很舒服。 具体的做法是:将耳机串联一只电位器,然后接在音频信号发生器输出端,信号发生器输出1000Hz的正弦信号,调整音频信号发生器输出信号大小和电位器,使得耳机中听到的声音既不感觉吵也不感觉太小,测量此时耳机两端的音频电压,并计算出音频功率,以此功率作为矿石机灵敏度的测试功率。 由高频信号发生器发出信号,用音频毫伏表测量耳机两端的音频电压,并根据测到的音频电压和耳机阻抗,计算出耳机端的音频功率。在同样音频功率的情况下,矿石机所需的高频信号越小,这台矿石机的灵敏度就越高。下面是矿石机灵敏度测试实例。 被测矿石机电路如图4-106所示,被测矿石机如图4-107所示。 图4-106 被测矿石机电路图 图4-107 被测矿石机 这台矿石机的检波器可以用二极管也可以用MOS管,检波管装在9针D型插头上,这样的设计使得更换不同的检波器十分方便。 图4-108所示是焊在9针D型插头上本机的检波器,前排左1是MOS管3EQ。 本机的检波器插座如图4-109所示。 图4-108 图4-109 插上二极管如图4-110所示。插上MOS管如图4-111所示。 这台机器接收效果良好,在北京西郊香山附近的樱桃沟山沟里,和远郊区昌平镇里,不用外接天地线,仅用机内的磁性天线就能收听。 因为本机既可以使用外接的天地线收听,也可以不用外接天地线,仅凭机内的磁性天线收听信号,所以,测试器灵敏度要分成两部分测试,测试时矿石机使用的是一只SC2-300耳机。 首先,将音频信号发生器输出1000Hz的正弦信号送入耳机,并用音频毫伏表测量耳机两端音频电压,调整信号的电压,当音频电压在1.5mV左右时,耳机发出的声音听上去不太大但很清晰,因此判定此次测试耳机端音频电压为1.5mV时矿石机的灵敏度。 图4-110 图4-111 第一步是测量不用外接天地线,只用磁性天线时,矿石机的灵敏度。 首先调整仪表,设定高频信号发生器输出频率为1MHz的调幅信号,调幅度为30\\%,调制的音频为1000Hz。高频信号发生器的输出电压调整到最高,此时测量天线放大器的输出电压是1.9V,单环天线的内阻为86Ω,算得天线电流为1.9/86=0.022A。 测试环境的布置如图4-112所示。 图4-112 测试环境 在这一环境中,设被测矿石机磁性天线处的信号场强为E,则有: E=(30π×r×r×N×I)/H×H×H 其中,r为天线半径(m),本天线r=0.125。 N为天线匝数,本天线N=1。 I为天线电流(A),本天线电流I=0.022A H如图4-112所示,为单环天线中心到磁棒中心的径向距离(m)。故在此环境中E=(30π×0.125×0.125×1×0.022)/(H×H×H),整理后E=0.0324/(H×H×H),即矿石机磁性天线处的场强等于0.0324除以单环天线中心到磁性天线中心径向距离的3次方。 具体测量过程很简单。首先,调整矿石机调谐回路的可变电容,边调整边观察音频毫伏表的读数,直到读数最大,谐振点就准确了。再调整矿石机匹配变压器初级和次级抽头,使音频毫伏表的读数最大,然后,调整单环天线到矿石机磁性天线的距离,使得音频毫伏表的读数正好是1.5mV,再测量此时单环天线中心到磁棒中心的距离,按上面的公式,计算出磁棒处的场强值,即为矿石机的灵敏度,单位是V / m(伏/每米)。 图4-113所示是实际测试时的现场照片。 图4-113 因为无论被测矿石机检波器使用的是二极管BAT85,还是使用MOS管3EQ,在实际使用时,都能不用外接天地线接收,因此我测量了使用这两种检波器时的灵敏度。 在使用二极管BAT85检波时,测得H=0.45m。 计算0.0324/0.45×0.45×0.45=0.356V/m=356mV/m。 在使用MOS管3EQ检波时,测得H=0.64m。 计算0.0324/0.64×0.64×0.64=0.124V/m=124mV/m。 从上面测量的结果可以看出,使用3EQ检波时的灵敏度更高。 说明:本次测量中,无论使用二极管检波还是MOS管检波,都没有使用班尼电路。在MOS管检波时,没有使用匹配变压器,3EQ的漏极直接连接耳机,虽然MOS管检波的输出阻抗远小于二极管检波的输出阻抗,但是,其输出阻抗值一般情况下也大于1kΩ。故在本次灵敏度的测量中,MOS管3EQ检波的输出阻抗,很可能与实际阻抗只有290Ω的这只SC2-300耳机未能达成匹配,如果使用一支阻抗比合适的匹配变压器使其达到匹配,灵敏度可能还会有些提高。 第二步是测量本机使用天地线时的接收灵敏度。 测试环境如图4-114所示。这项测量,分别测了二极管检波时的灵敏度和MOS管3EQ的检波灵敏度。 图4-114 测试环境 设定高频信号发生器输出频率为1MHz的调幅信号,调幅度为30\\%,调制的音频为1000Hz。用高频毫伏表监测信号发生器的输出电压,信号发生器的输出电压,通过全波等效天线送入被测矿石机的天线端,本矿石机有两个天线端,这次测量的是前面电路图中L3的上端。 测试过程: 首先,将高频信号发生器输出电压调整为数十毫伏,然后,仔细调节矿石机调谐回路的可变电容,边调整边观察音频毫伏表的读数,直到读数最大为止。此时谐振点就准确了,再调整矿石机匹配变压器初级和次级抽头,使音频毫伏表的读数最大。然后,逐渐降低高频信号发生器的输出电压,直到耳机端的音频毫伏表的读数为1.5mV为止,此时,高频毫伏表的电压读数值,即为矿石机的灵敏度值。 在本机的实际测量中,使用二极管BAT85检波时的灵敏度为9mV。使用MOS管3EQ检波时的灵敏度为5mV。 在上述的实际测量过程中,我发现,由于被测矿石机输出信号很弱,仅有1.5mV,这就要求高频信号发生器、高频毫伏表、被测矿石机和音频毫伏表都要可靠接地,否则,就会引入较大的干扰信号,使测量根本无法正常进行。另外,由于耳机在收到外部声音时,会在耳机两端产生音频电压,如果外部声音稍大,耳机两端产生的音频电压要远远大于矿石机输出的电压,在这种情况下,测量根本不能进行,因此,矿石机灵敏度测量一定要有个安静的环境。 全波等效天线实物如图4-115所示。 图4-115 全波等效天线 全波等效天线的电路图请见图4-78。 图4-116所示是这次测量的现场照片。 上述基于耳机端音频功率的矿石机灵敏度测量方法,虽然简单方便,但这种测量方法中没有包含耳机灵敏度对矿石机灵敏度的影响,从而忽略了不同耳机对矿石机灵敏度的影响不同这一事实。用这个方法测到的灵敏度,不是矿石机完整的灵敏度,不能反映出矿石机灵敏度的真实情况,所以,这个测量方法仅适于使用同一副耳机时的不同矿石机灵敏度测量。 图4-116 测试场景 在普通的收音机中,更换不同的扬声器或耳机,只要阻抗相同,对收音机的性能影响不大,但是,在矿石机中则不同,矿石机没有放大器,每个元件的性能对矿石机灵敏度的影响都是很大的。很多矿石机爱好者都有这样的经验,不同性能的耳机,对矿石机的灵敏度影响十分明显,因此,在同一台矿石机上更换了灵敏度更高的耳机,常常能收到一些原来收不到的弱台,所以,在对矿石机灵敏度测试时,应该测量包含耳机在内的全部元器件,这样的灵敏度测量结果才能真正反映出矿石机的实际灵敏度水平。在耳机发出同样大小的声音时,比较哪台矿石机输入的高频信号最小,哪台矿石机的灵敏度就最高,这样测到的,就是矿石机完整的灵敏度。 我们可以使用声级计检测耳机发声的大小,然而,在矿石机爱好者之间,经常需要互相交流矿石机的制作经验,为此,大家就要在同一个灵敏度测试标准下,测量自己矿石机灵敏度,以便大家交流和比较。 近些年来,已经有一些资深的矿石机爱好者使用声级计测量耳机的音量,目的是比较出不同耳机的灵敏度。但是,要把声级计作为音量的测量工具,用于矿石机灵敏度的测量,还需要建立一个矿石机爱好者们共同制定的测量标准,如果可能,大家可以通过协商制定出这个标准。标准内容应包含测量矿石机音量时声级计声音探头与耳机发音孔的距离和声音大小的分贝值等。有了这一标准,不同地点的爱好者们,就可以较为准确地互相比较各自矿石机的灵敏度了。 图4-117所示是一只声级计实物照片。 图4-117 声级计 9.用指针式万用表简单判断MOS管是否可用于矿石机检波 用MOS管检波是近几年来矿石机技术的新方向,但是,并不是所有的MOS管都可以做矿石机的检波器件,而只有一部分能做检波之用。这一段时间,我做了不少次MOS管检波的试验和测试,在实验中,我摸索到了一个用指针式万用表判断MOS管是否能做矿石机检波的简单方法。现以N沟道双栅MOS管为例介绍。 第一步:指针式万用表放在RX1k挡位,用一条短导线的一端,将双栅MOS管的两个栅极连接在一起,另一端暂时悬空不接,万用表的正表笔接MOS管的源极(S极),黑表笔接到MOS管的漏极(D极),如图4-118所示。 用手做接近和远离栅极导线的动作,如果万用表针能随手的动作而摆动,则管子是好的,就可以往下继续测试了。 第二步:万用表与MOS管保持上述的连接不变,将栅极导线悬空的一端连接到管子的源极上,如图4-119所示。 图4-118 图4-119 这时万用表的指示可能有如下3种情况。 (1)表针指示电阻值接近0Ω,如图4-120所示。这表示,该MOS管是关断栅压较高的耗尽型管子,大多不能做矿石机的检波使用,即使有的管子勉强能用,效果也不好。 (2)表针指示电阻值无穷大,或者很大(数百千欧),如图4-121所示。这表示该管是增强型的管子,这种管子大多数是可以做矿石机检波的,但是效果好坏,会有不同的差别。 (3)表针指示电阻值在数百千欧到数十千欧,如图4-122所示。这表示该管是关断栅压较小的耗尽型管子,性能接近零开启栅压的增强管,适合检波使用。 图4-120 图4-121 图4-122 值得注意的是,用上述测量方法去测场MOS型效应管是有一定的风险的,原因是MOS管是绝缘栅场效应管,栅极电阻极高,且栅极电容很小,栅极上稍有电荷积累,就会形成较高的电压而击穿栅极。现代的一些新型的MOS管,如3DQ、3DP、3EQ等栅极保护做得非常好,笔者在很多次测量中并没有损坏过一只管子。但是,这一测量方法对于一些老型号的MOS管,包括大名鼎鼎的零栅压MOS管ALD110800,或是ALD110900都发生过损坏。因此,测量时的防护还是必要的,测量前要注意身上是否带有静电,要及时释放身上的静电(用手触摸地线),最好带上静电防护腕带,只有这样才能避免和减少测量造成MOS管的损坏。 静电防护腕带的使用方法是:将腕带戴在手腕上,将腕带的鳄鱼夹可靠接地。 图4-123所示就是静电防护腕带的照片。 图4-123 静电防护腕带\begin{figure}[htb]  \centering  \includegraphics[width=0.8\\linewidth]{Image00293.jpg}  \caption{图片 199: Image00293.jpg}\end{figure}\begin{figure}[htb]  \centering  \includegraphics[width=0.8\\linewidth]{Image00116.jpg}  \caption{图片 200: Image00116.jpg}\end{figure}\begin{figure}[htb]  \centering  \includegraphics[width=0.8\\linewidth]{Image00291.jpg}  \caption{图片 201: Image00291.jpg}\end{figure}\begin{figure}[htb]  \centering  \includegraphics[width=0.8\\linewidth]{Image00202.jpg}  \caption{图片 202: Image00202.jpg}\end{figure}\begin{figure}[htb]  \centering  \includegraphics[width=0.8\\linewidth]{Image00270.jpg}  \caption{图片 203: Image00270.jpg}\end{figure}\begin{figure}[htb]  \centering  \includegraphics[width=0.8\\linewidth]{Image00260.jpg}  \caption{图片 204: Image00260.jpg}\end{figure}\begin{figure}[htb]  \centering  \includegraphics[width=0.8\\linewidth]{Image00294.jpg}  \caption{图片 205: Image00294.jpg}\end{figure}\begin{figure}[htb]  \centering  \includegraphics[width=0.8\\linewidth]{Image00320.jpg}  \caption{图片 206: Image00320.jpg}\end{figure}\begin{figure}[htb]  \centering  \includegraphics[width=0.8\\linewidth]{Image00350.jpg}  \caption{图片 207: Image00350.jpg}\end{figure}\begin{figure}[htb]  \centering  \includegraphics[width=0.8\\linewidth]{Image00068.jpg}  \caption{图片 208: Image00068.jpg}\end{figure}\begin{figure}[htb]  \centering  \includegraphics[width=0.8\\linewidth]{Image00189.jpg}  \caption{图片 209: Image00189.jpg}\end{figure}\begin{figure}[htb]  \centering  \includegraphics[width=0.8\\linewidth]{Image00218.jpg}  \caption{图片 210: Image00218.jpg}\end{figure}\begin{figure}[htb]  \centering  \includegraphics[width=0.8\\linewidth]{Image00063.jpg}  \caption{图片 211: Image00063.jpg}\end{figure}\begin{figure}[htb]  \centering  \includegraphics[width=0.8\\linewidth]{Image00061.jpg}  \caption{图片 212: Image00061.jpg}\end{figure}\begin{figure}[htb]  \centering  \includegraphics[width=0.8\\linewidth]{Image00032.jpg}  \caption{图片 213: Image00032.jpg}\end{figure}\begin{figure}[htb]  \centering  \includegraphics[width=0.8\\linewidth]{Image00206.jpg}  \caption{图片 214: Image00206.jpg}\end{figure}\begin{figure}[htb]  \centering  \includegraphics[width=0.8\\linewidth]{Image00105.jpg}  \caption{图片 215: Image00105.jpg}\end{figure}\begin{figure}[htb]  \centering  \includegraphics[width=0.8\\linewidth]{Image00146.jpg}  \caption{图片 216: Image00146.jpg}\end{figure}\begin{figure}[htb]  \centering  \includegraphics[width=0.8\\linewidth]{Image00292.jpg}  \caption{图片 217: Image00292.jpg}\end{figure}\begin{figure}[htb]  \centering  \includegraphics[width=0.8\\linewidth]{Image00327.jpg}  \caption{图片 218: Image00327.jpg}\end{figure}\begin{figure}[htb]  \centering  \includegraphics[width=0.8\\linewidth]{Image00126.jpg}  \caption{图片 219: Image00126.jpg}\end{figure}\begin{figure}[htb]  \centering  \includegraphics[width=0.8\\linewidth]{Image00224.jpg}  \caption{图片 220: Image00224.jpg}\end{figure}\begin{figure}[htb]  \centering  \includegraphics[width=0.8\\linewidth]{Image00131.jpg}  \caption{图片 221: Image00131.jpg}\end{figure}\begin{figure}[htb]  \centering  \includegraphics[width=0.8\\linewidth]{Image00344.jpg}  \caption{图片 222: Image00344.jpg}\end{figure}\begin{figure}[htb]  \centering  \includegraphics[width=0.8\\linewidth]{Image00070.jpg}  \caption{图片 223: Image00070.jpg}\end{figure}\begin{figure}[htb]  \centering  \includegraphics[width=0.8\\linewidth]{Image00155.jpg}  \caption{图片 224: Image00155.jpg}\end{figure}\begin{figure}[htb]  \centering  \includegraphics[width=0.8\\linewidth]{Image00179.jpg}  \caption{图片 225: Image00179.jpg}\end{figure}\begin{figure}[htb]  \centering  \includegraphics[width=0.8\\linewidth]{Image00081.jpg}  \caption{图片 226: Image00081.jpg}\end{figure}\begin{figure}[htb]  \centering  \includegraphics[width=0.8\\linewidth]{Image00207.jpg}  \caption{图片 227: Image00207.jpg}\end{figure}\begin{figure}[htb]  \centering  \includegraphics[width=0.8\\linewidth]{Image00375.jpg}  \caption{图片 228: Image00375.jpg}\end{figure}\begin{figure}[htb]  \centering  \includegraphics[width=0.8\\linewidth]{Image00096.jpg}  \caption{图片 229: Image00096.jpg}\end{figure}\begin{figure}[htb]  \centering  \includegraphics[width=0.8\\linewidth]{Image00112.jpg}  \caption{图片 230: Image00112.jpg}\end{figure}\begin{figure}[htb]  \centering  \includegraphics[width=0.8\\linewidth]{Image00210.jpg}  \caption{图片 231: Image00210.jpg}\end{figure}\begin{figure}[htb]  \centering  \includegraphics[width=0.8\\linewidth]{Image00178.jpg}  \caption{图片 232: Image00178.jpg}\end{figure}\begin{figure}[htb]  \centering  \includegraphics[width=0.8\\linewidth]{Image00192.jpg}  \caption{图片 233: Image00192.jpg}\end{figure}\begin{figure}[htb]  \centering  \includegraphics[width=0.8\\linewidth]{Image00393.jpg}  \caption{图片 234: Image00393.jpg}\end{figure}\begin{figure}[htb]  \centering  \includegraphics[width=0.8\\linewidth]{Image00364.jpg}  \caption{图片 235: Image00364.jpg}\end{figure}\begin{figure}[htb]  \centering  \includegraphics[width=0.8\\linewidth]{Image00149.jpg}  \caption{图片 236: Image00149.jpg}\end{figure}\begin{figure}[htb]  \centering  \includegraphics[width=0.8\\linewidth]{Image00284.jpg}  \caption{图片 237: Image00284.jpg}\end{figure}\begin{figure}[htb]  \centering  \includegraphics[width=0.8\\linewidth]{Image00255.jpg}  \caption{图片 238: Image00255.jpg}\end{figure}\begin{figure}[htb]  \centering  \includegraphics[width=0.8\\linewidth]{Image00059.jpg}  \caption{图片 239: Image00059.jpg}\end{figure}\begin{figure}[htb]  \centering  \includegraphics[width=0.8\\linewidth]{Image00382.jpg}  \caption{图片 240: Image00382.jpg}\end{figure}\begin{figure}[htb]  \centering  \includegraphics[width=0.8\\linewidth]{Image00399.jpg}  \caption{图片 241: Image00399.jpg}\end{figure}\begin{figure}[htb]  \centering  \includegraphics[width=0.8\\linewidth]{Image00235.jpg}  \caption{图片 242: Image00235.jpg}\end{figure}\begin{figure}[htb]  \centering  \includegraphics[width=0.8\\linewidth]{Image00141.jpg}  \caption{图片 243: Image00141.jpg}\end{figure}\begin{figure}[htb]  \centering  \includegraphics[width=0.8\\linewidth]{Image00069.jpg}  \caption{图片 244: Image00069.jpg}\end{figure}\begin{figure}[htb]  \centering  \includegraphics[width=0.8\\linewidth]{Image00115.jpg}  \caption{图片 245: Image00115.jpg}\end{figure}\begin{figure}[htb]  \centering  \includegraphics[width=0.8\\linewidth]{Image00372.jpg}  \caption{图片 246: Image00372.jpg}\end{figure}\begin{figure}[htb]  \centering  \includegraphics[width=0.8\\linewidth]{Image00225.jpg}  \caption{图片 247: Image00225.jpg}\end{figure}\begin{figure}[htb]  \centering  \includegraphics[width=0.8\\linewidth]{Image00052.jpg}  \caption{图片 248: Image00052.jpg}\end{figure}\begin{figure}[htb]  \centering  \includegraphics[width=0.8\\linewidth]{Image00127.jpg}  \caption{图片 249: Image00127.jpg}\end{figure}\begin{figure}[htb]  \centering  \includegraphics[width=0.8\\linewidth]{Image00132.jpg}  \caption{图片 250: Image00132.jpg}\end{figure}\begin{figure}[htb]  \centering  \includegraphics[width=0.8\\linewidth]{Image00004.jpg}  \caption{图片 251: Image00004.jpg}\end{figure}\begin{figure}[htb]  \centering  \includegraphics[width=0.8\\linewidth]{Image00121.jpg}  \caption{图片 252: Image00121.jpg}\end{figure}\begin{figure}[htb]  \centering  \includegraphics[width=0.8\\linewidth]{Image00222.jpg}  \caption{图片 253: Image00222.jpg}\end{figure}\begin{figure}[htb]  \centering  \includegraphics[width=0.8\\linewidth]{Image00092.jpg}  \caption{图片 254: Image00092.jpg}\end{figure}\begin{figure}[htb]  \centering  \includegraphics[width=0.8\\linewidth]{Image00361.jpg}  \caption{图片 255: Image00361.jpg}\end{figure}\begin{figure}[htb]  \centering  \includegraphics[width=0.8\\linewidth]{Image00074.jpg}  \caption{图片 256: Image00074.jpg}\end{figure}\begin{figure}[htb]  \centering  \includegraphics[width=0.8\\linewidth]{Image00312.jpg}  \caption{图片 257: Image00312.jpg}\end{figure}\begin{figure}[htb]  \centering  \includegraphics[width=0.8\\linewidth]{Image00038.jpg}  \caption{图片 258: Image00038.jpg}\end{figure}\begin{figure}[htb]  \centering  \includegraphics[width=0.8\\linewidth]{Image00087.jpg}  \caption{图片 259: Image00087.jpg}\end{figure}\begin{figure}[htb]  \centering  \includegraphics[width=0.8\\linewidth]{Image00122.jpg}  \caption{图片 260: Image00122.jpg}\end{figure}\begin{figure}[htb]  \centering  \includegraphics[width=0.8\\linewidth]{Image00152.jpg}  \caption{图片 261: Image00152.jpg}\end{figure}\begin{figure}[htb]  \centering  \includegraphics[width=0.8\\linewidth]{Image00288.jpg}  \caption{图片 262: Image00288.jpg}\end{figure}\begin{figure}[htb]  \centering  \includegraphics[width=0.8\\linewidth]{Image00397.jpg}  \caption{图片 263: Image00397.jpg}\end{figure}\begin{figure}[htb]  \centering  \includegraphics[width=0.8\\linewidth]{Image00342.jpg}  \caption{图片 264: Image00342.jpg}\end{figure}\begin{figure}[htb]  \centering  \includegraphics[width=0.8\\linewidth]{Image00129.jpg}  \caption{图片 265: Image00129.jpg}\end{figure}\begin{figure}[htb]  \centering  \includegraphics[width=0.8\\linewidth]{Image00212.jpg}  \caption{图片 266: Image00212.jpg}\end{figure}\begin{figure}[htb]  \centering  \includegraphics[width=0.8\\linewidth]{Image00015.jpg}  \caption{图片 267: Image00015.jpg}\end{figure}\begin{figure}[htb]  \centering  \includegraphics[width=0.8\\linewidth]{Image00241.jpg}  \caption{图片 268: Image00241.jpg}\end{figure}\begin{figure}[htb]  \centering  \includegraphics[width=0.8\\linewidth]{Image00386.jpg}  \caption{图片 269: Image00386.jpg}\end{figure}\begin{figure}[htb]  \centering  \includegraphics[width=0.8\\linewidth]{Image00268.jpg}  \caption{图片 270: Image00268.jpg}\end{figure}\begin{figure}[htb]  \centering  \includegraphics[width=0.8\\linewidth]{Image00305.jpg}  \caption{图片 271: Image00305.jpg}\end{figure}\begin{figure}[htb]  \centering  \includegraphics[width=0.8\\linewidth]{Image00044.jpg}  \caption{图片 272: Image00044.jpg}\end{figure}\begin{figure}[htb]  \centering  \includegraphics[width=0.8\\linewidth]{Image00198.jpg}  \caption{图片 273: Image00198.jpg}\end{figure}\begin{figure}[htb]  \centering  \includegraphics[width=0.8\\linewidth]{Image00028.jpg}  \caption{图片 274: Image00028.jpg}\end{figure}\begin{figure}[htb]  \centering  \includegraphics[width=0.8\\linewidth]{Image00186.jpg}  \caption{图片 275: Image00186.jpg}\end{figure}\begin{figure}[htb]  \centering  \includegraphics[width=0.8\\linewidth]{Image00177.jpg}  \caption{图片 276: Image00177.jpg}\end{figure}\begin{figure}[htb]  \centering  \includegraphics[width=0.8\\linewidth]{Image00277.jpg}  \caption{图片 277: Image00277.jpg}\end{figure}\begin{figure}[htb]  \centering  \includegraphics[width=0.8\\linewidth]{Image00374.jpg}  \caption{图片 278: Image00374.jpg}\end{figure}\begin{figure}[htb]  \centering  \includegraphics[width=0.8\\linewidth]{Image00184.jpg}  \caption{图片 279: Image00184.jpg}\end{figure}\begin{figure}[htb]  \centering  \includegraphics[width=0.8\\linewidth]{Image00308.jpg}  \caption{图片 280: Image00308.jpg}\end{figure}\begin{figure}[htb]  \centering  \includegraphics[width=0.8\\linewidth]{Image00387.jpg}  \caption{图片 281: Image00387.jpg}\end{figure}\begin{figure}[htb]  \centering  \includegraphics[width=0.8\\linewidth]{Image00278.jpg}  \caption{图片 282: Image00278.jpg}\end{figure}\begin{figure}[htb]  \centering  \includegraphics[width=0.8\\linewidth]{Image00340.jpg}  \caption{图片 283: Image00340.jpg}\end{figure}\begin{figure}[htb]  \centering  \includegraphics[width=0.8\\linewidth]{Image00315.jpg}  \caption{图片 284: Image00315.jpg}\end{figure}\begin{figure}[htb]  \centering  \includegraphics[width=0.8\\linewidth]{Image00158.jpg}  \caption{图片 285: Image00158.jpg}\end{figure}\begin{figure}[htb]  \centering  \includegraphics[width=0.8\\linewidth]{Image00368.jpg}  \caption{图片 286: Image00368.jpg}\end{figure}\begin{figure}[htb]  \centering  \includegraphics[width=0.8\\linewidth]{Image00042.jpg}  \caption{图片 287: Image00042.jpg}\end{figure}\begin{figure}[htb]  \centering  \includegraphics[width=0.8\\linewidth]{Image00170.jpg}  \caption{图片 288: Image00170.jpg}\end{figure}\begin{figure}[htb]  \centering  \includegraphics[width=0.8\\linewidth]{Image00252.jpg}  \caption{图片 289: Image00252.jpg}\end{figure}\begin{figure}[htb]  \centering  \includegraphics[width=0.8\\linewidth]{Image00227.jpg}  \caption{图片 290: Image00227.jpg}\end{figure}\begin{figure}[htb]  \centering  \includegraphics[width=0.8\\linewidth]{Image00065.jpg}  \caption{图片 291: Image00065.jpg}\end{figure}\begin{figure}[htb]  \centering  \includegraphics[width=0.8\\linewidth]{Image00098.jpg}  \caption{图片 292: Image00098.jpg}\end{figure}\begin{figure}[htb]  \centering  \includegraphics[width=0.8\\linewidth]{Image00263.jpg}  \caption{图片 293: Image00263.jpg}\end{figure}\section{文件 16}\
%第四章 新时期国内矿石机的研究和进展 三、整机篇 复古矿石机 电罗经的复古矿石机 电罗经名叫吴俊东,其矿石机作品的电路和结构并不复杂,他的特色是“一切从头做起”,只要能DIY的东西,全部“自己动手,丰衣足食”。电罗经家中凡木工、钳工、电工、漆工、车工的工具,一应俱全,当然最重要的是他身怀绝技,心灵手巧。他的矿石机从机箱到固定矿石、活动矿石、接线柱、各类标牌(如天地线、耳机、频率调谐等)、线圈,甚至“伞形”可变电容器这样高难度的元件,都是自制的。下面就展示一下电罗经的矿石机作品(见图4-124)。 图4-124 电罗经矿石机作品 1号亚美复古矿石收音机 图4-125所示的这台矿石机除固定电容、调谐旋钮外,所用元器件、机箱,甚至螺丝钉,都是他DIY的。 图4-125 亚美复古矿石机 图4-125 亚美复古矿石机(续) 2号1401-1型复古矿石机 图4-126所示的此机的亮点在DIY的1401-1型活动线圈,该线圈是20世纪60年代通用矿石机线圈,一般用在比较讲究的矿石机上面,现在已经很难找到原装货了。 图4-126 1401-1型复古矿石机 3号花篮线圈复古矿石机 图4-127所示的花篮线圈是古典矿石机常用的原件,它具有占空间小、效率高的优点,但制作起来十分不易。此机元件除旋钮、固定电容外,均自制。 图4-127 花篮线圈复古矿石机 图4-128 圆筒复古矿石机 4号、5号圆筒复古矿石机 图4-128所示的两机机壳为废物利用,系装茶叶的小塑料桶。分别采用伞形和空气介质的可变电容器,自制了仿古小型蜂房线圈。 6号、7号“冰激凌”复古矿石机 图4-129所示是用两种不同的冰激凌盒子改制的矿石机。6号机采用了20世纪60年代初期,大城市中流行的二极管检波,当然,二极管是前苏联的老式二极管,它可以勾起一代人的美好回忆。7号机的重点是自制的固定矿石,这样当年常见的检波器就齐全了。 图4-129 8号仿制63-2型复古矿石机 图4-130所示的此机系仿制20世纪60年代上海仪表电讯技校生产的63-2型矿石机,面板上的那枚月牙形耳机插座,是当年国产矿石机的典型装备,它具有强烈的复古味道。 9号剃须刀复古矿石收音机 在第二次世界大战的战场上,剃须刀片曾经被士兵用来作为检波器,然后接上简单的天地线后收听广播。在后来的爱好者DIY矿石机活动中,它成为必不可少的一个节目(见图4-131)。 图4-130 仿制20世纪60年代上海仪表电讯技校生产的63-2型矿石机 图4-131 电罗经制作的剃须刀检波矿石机 蛛网线圈双回路矿石机 刘俊龙 出于怀旧心理,我设计了这台酷似20世纪50至60年代生产的一种驼峰外形的,有机介质绝缘的,单连可变电容器的矿石机(见图4-132)。 图4-132 它的外形尺寸可根据所选蛛网板的尺寸进行设计,或依据个人喜好进行设计制作。本人设计的这台矿石机最大外形尺寸是:195mm×160mm,该尺寸不包括可变电容器的动片旋出部分。 整体布局采用敞开式结构,两层底板中间放置除蛛网板外的所有零部件。两层底板中间的距离是24mm,也可根据需要进行调整。两层底板用4只30mm×3mm的螺丝进行连接以形成安装空间。安装时,将下层底板拆下,所有零部件都固定在上层底板的下面,这是为了方便以后的修理工作。整体结构如图4-133所示。 本着DIY精神,以提高动手能力为目的,这台矿石机的主要零部件:可变电容器、分线器、蛛网线圈、输出变压器都是自制的。有能力的话,活动矿石也自制更好。如果为了使用时声音大一些,活动矿石也可用2AP9检波二极管替代,效果要好许多,在图4-133所示的活动矿石上面也可看到接了一个检波管。 这台矿石机的单连可变电容器是按比例制作的仿驼峰式单连,比例为3∶1,就是说自制的单连外形是实际单连的3倍,在安装中占有很大的空间,图中可看得很清楚。做成大比例还有一个好处是可以减少动片、定片的数量就可以达到大容量的目的。这个自制的可变电容用了两片动片,一片定片,容量在17~450pF范围变化,这样的参数完全满足频率覆盖的要求。可变电容器的外形如图4-132所示。限于篇幅,本文不做详细介绍。读者可根据自己现有条件自行制作,例如,绝缘材料可以使用医院X光大底片等。 图4-133 整体结构 这台矿石机的两个分线器直接设计安装在上层底板的两侧。利用大帽铜螺丝制作的,或直接加工成平顶螺丝。然后在底板上画好弧线,钻8个孔,一定要钻准确,否则,安装后影响视觉感受。分线器中间的转轴可利用废旧电位器的旋轴改制,在旋轴上安装一个磷铜片制作的旋轴连片。每个大帽螺丝配一个M3焊片做引线连接,至此,分线器就做好了,见图4-134。 图4-134 制作完成的分线器 蛛网线圈骨架用有机玻璃板制作,在有机玻璃板上仔细画好切割线,用钢锯小心、仔细锯下,再加以修整。应准备的工具有50mm小台虎钳、钢锯弓子、什锦组锉、扁嘴钳、各种规格的螺丝刀等。 在蛛网板上绕制线圈时要有耐心、细心才行,绕完后还要加以整理才好看。这台矿石收音机是双回路的,蛛网板的制作尺寸是内径40mm,外径132mm,共有11个齿,每两个齿的距离要适当,上口为10mm,下口为3mm。本制作需要两块蛛网板,一块蛛网板上绕天线线圈,另一块蛛网板上绕调谐线圈。两个线圈的数据如下:L1是天线线圈,用φ0.51mm的漆包线绕45匝,在30匝处抽头。L2是调谐线圈,用同号线绕84匝,每8匝处抽一个头,见图4-135。 图4-135 制作完成 两块蛛网板的调节机构是最不容易制作的,我是用M8×90mm的铜螺栓做的,一头做成直径6mm,长20mm安装旋钮的轴。在两块蛛网板的同一位置钻孔。天线线圈钻直径7mm的孔,然后用M8丝锥攻成M8的孔。调谐线圈钻直径8mm的孔。安装一个大型电位器的固定螺母,把M8的螺栓旋钮轴插进去。螺栓的另一头旋进天线线圈已经套好扣的孔里就可以调节两个线圈的距离了。为了防止天线线圈随意转,必须在调节螺栓的周围安装3个导向轨,以固定住天线线圈的行走方向。3个导向轨兼做天线线圈的3根连线。见图4-136。 当然,每个人都要根据自己的思维来确定调节机构的方法方式,这里只是起个抛砖引玉的作用。 关于输出变压器的制作,我选用截面积为9mm×11mm左右的铁芯。一次绕组用φ0.09mm漆包线绕3800匝,二次用φ0.35mm漆包线绕180匝即可,见图4-137。活动矿石的安装位置及方式见图4-138。 图4-136 制作细节 图4-137 制作好的输出变压器 图4-138 制作活动矿石的安装 这台矿石机的电路选用效率比较高,灵敏度和选择性都不错的电路,见图4-139。 图4-139 蛛网线圈矿石收音机电路图 这个电路的特点是,在距离电台发射地比较近(\\&lt;10km)的地方使用效果非常好。我家住在距离电台8km(空间直线距离)的地方,使用这个电路,检波用2AP9(把活动矿石的触针拔出一些,使触针离开矿石),用5W号筒式高音扬声器放音,音量可与两管晶体管收音机的音量相当,20~30m2 的房间都能听清楚。我用的是15W椭圆纸盆全频扬声器,口径为230mm×140mm,感觉音量比较柔和、耐听,音量不亚于5W高音扬声器。 元器件的选用:关键是三极管的性能要好,电流放大倍数B在120以上,穿透电流越小越好。我用的是军级3AG14,它的放大倍数为130倍。 检波管用两个正向阻值基本一致的,越小越好,反向阻值越大越好。我选用的正向阻值在200Ω左右,反向阻值在800kΩ以上的检波管。 单连可变电容器可选用空气单连,Q值高些,效果相对好些。 蛛网线圈制作得要精细些,每匝之间的间隙要调整得基本一样,松紧度要紧些,不能有松脱现象。 分线器的动、静触点接触得要可靠,不能似接非接。动触头不能放在两个定触点中间,以免造成短路,影响输出音量。 输出变压器用的双声道插口要选用带一组可通、断接点的,这样可省去一个开关,也可实现高阻输出、低阻输出自动转换。 当活动矿石面对你的时候,左面的分线器是调谐线圈用的,右面的分线器是变换天线耦合电容容量用的,从图中可看出,分线器下面的一排云母电容都是天线耦合用的电容器,当选择不同频率的电台时,就要变换天线耦合电容的容量,才能达到最佳天线匹配。 旋钮和接线柱的选择,应该是有较新的老式产品,这样才有怀旧的感觉,当然,如果找不到老产品就只能用新式的产品了。 各种显示标牌可利用手头现有的,如果没有就只好自己制作了。我是用电脑制作出图片,然后去冲洗出照片来,再去塑封一下就成功了。当然如果能做成铝片的就更理想了。 有些制作是不容易用文字表述非常清楚的,例如,这个驼峰式单连就不好全部介绍清楚,那样是需要长篇文字材料和图纸的,因此只是一个简单介绍而已,希望读者能理解。DIY需要的是发挥自己的想象力,只有这样才能做出更好的作品来。 矿石机很简单,科技含量并不高,但是,要想做成一台质量非常高的、效果非常理想的并不容易,这要靠自己的理论基础、实践基础,靠细心、耐心,还有外部环境是否适合制作矿石机(假如你家的周围方圆150km范围内根本就没有发射台,安装矿石机是没有实用意义的)。 制作安装任何一台矿石机,都不是一成不变的,更不提倡完全照搬。要学会变通应用各种电路形式,各种设计外形。例如,看好了这种外形,但不一定看好这个电路,那么,就可以利用这个外形,选用你看好的电路,使二者搭配,制作出你喜欢的矿石机来,这才是玩矿石机的最高境界。细心的读者应该不难看出,我制作介绍的这台矿石机实际并不是文中介绍的这个电路。它只是个普通的双回路电路形式罢了,在实物图片中看到三极管了吗?看到有两个检波二极管了吗?看到两个5100pF的电容器了吗?看到还有一个电解电容吗?都没有。这台矿石收音机的实际安装电路是下面这个电路图(见图4-140)。目的只有一个,就是建议大家不要完全照搬,要有自己的创意,这样的作品才更有味道。 图4-140 双回路矿石收音机电路图 三回路蛛网线圈矿石收音机 刘俊龙 这是一台三回路蛛网线圈构成的性能良好的矿石收音机,天线线圈与调谐线圈采用紧耦合形式。另外绕一个检波专用线圈,这样很容易做到调谐电路和检波电路的阻抗匹配。检波采用全波检波电路,以充分发挥矿石收音机的效率,这样既满足对矿石机高灵敏度的要求,同时又具有良好的选择性。电路如图4-141所示。 图4-141 高效率三回路全波检波矿石收音机电路图 从图4-141中可以看出,在同一块蛛网板上绕有两个线圈,一个是L1天线线圈,一个是L2调谐线圈。在调谐线圈L2的绕法上有特殊之处,L2一共绕40匝,在蛛网板上先绕好20匝,不要剪断,再用另一根线紧挨着L2绕15匝作为L1,在L1外面继续绕另一半L2的20匝,这样才完成L1、L2的绕制, 见图4-142。从图4-143中可看出,最里面和最外面发白色的多股线是L2,里面绕了20匝,外面绕了20匝,中间夹着的红色漆包线是15匝的L1。这里为了蛛网线圈的美观,采用了两种线绕制。天线线圈用的是0.56mm的漆包线,调谐线圈采用的是φ0.1mm×36的多股纱包线。当然,调谐线圈也可利用和调谐线圈一样的多股纱包线,效果更好。 图4-142 线圈的绕制 这样的绕法就是所谓的紧耦合绕法,这样做可以减少电磁波的损耗,从而最大限度地利用电磁波的能量。L2调谐线圈和C1可变电容器构成独立的调谐回路,与检波回路分开,避免了调谐回路和检波回路的相互牵制。 检波回路采用了和常见的倍压检波电路有所不同的电路接法。倍压检波需要配合两个小容量的电容进行充放电来完成倍压检波,而全波检波电路不需要两个小容量电容来配合,就能得到正、负两个波形的能量,这是全波检波电路的优势。而倍压检波在不同的频率下需要不同时间常数来配合,这样才能很好地完成倍压任务。而每个电台的频率是不一样的,所以,不同频率的电台其检波效果是不一样的,频率高端的电台检波效果好,频率低端的电台检波效果就不是处于最佳状态。反之,低端好,高端就不是处于最佳状态。而全波检波电路就没有这个问题。 L3是检波回路的绕组,它绕在另一块蛛网板上,通过调整不同的匝数就可以达到和检波二极管(活动矿石)的最佳阻抗配合,从而完全避免了与调谐回路的相互牵制。这是其他检波回路做不到的。 检波管的选用原则:如果以怀旧为主,不经常用来收听,建议选择两个活动矿石。如果还经常用来收听,建议焊接两个检波二极管2AP9,或其他型号检波管,如2AP10、2AK系列,或三极管的e、b结都可,只要满足正向电阻越小越好、反向电阻越大越好的要求即可。最好的处理方法是,在两个活动矿石的固定螺丝上安装两个焊片,在焊片上焊接两个2AP9。平时把活动矿石的调节把手拔出一些,只用二极管进行收听。 现在市面上卖矿石机元器件的极少,很多元器件都得自行制作,尤其是老式元器件,更需要自己制作,搞一些现代化的元器件装上去,就完全失去了怀旧的感觉。这台矿石机的大多数元器件都是自制的,活动矿石、蛛网线圈、单连可变电容器、输出变压器、线圈的调节机构、矿石机底座等都是自制的。 下面主要介绍一下我做的连体全波检波活动矿石的过程及方法。 有时并不是先有了具体的、成熟的思路才有发明创造的,而是突然之间有了灵感,就会有一个新生事物诞生了。这种感觉相信很多人都有过,但大多数都被忽略了、抛弃了……这是十分可惜的事情。我设计制作的这个连体活动矿石就是受到整流桥的启发而突发奇想的。能把4个整流二极管封装在一起而形成一个元器件,就应该能把两个检波二极管制成一体,或者把4个检波二极管封装在一起,形成一个全波检波电路或桥式检波电路。这就是我设计制作这个连体检波活动矿石的基础(见图4-143)。 这样的设计,两个活动矿石只有3根引线,其中有两根线已经连到一起了。如图4-141所示,电路图中正好两个活动矿石有一端的同极性是连在一起的,另一端则分别接到L3的两个端子(注意在倍压检波电路里,两个活动矿石连在一起的不是同名端,而是将非同名端连在一起的),制作的这个连体活动矿石正好符合这个全波检波电路的要求。 图4-143 连体检波器活动矿石 制作的具体尺寸就不介绍了,读者可根据图片随意设计自己的作品,这里只起到抛砖引玉的作用。 这个矿石收音机的底座是使用厚4mm的有机玻璃板制作的,分线器、调谐读盘、蛛网线圈骨架做成一体的,见图4-144。有条件的话,可以用雕刻机机床制作,没有条件的可用手工制作。当然也可不用这样的设计,读者可自行设计其他形式的矿石机底座。 图4-144 矿石收音机底座 天线线圈、调谐线圈与两块蛛网板之间的距离要能调节,读者可根据自己的现有条件和动手能力进行设计制作,简单一些也可,复杂一些也可,总之,能达到调节两个蛛网板距离的目的就行。 这里采用的是一套比较复杂、精密的丝杠调节机构,因为比较复杂,不多作介绍,大家可从图4-145所示细节来进行参考。 图4-145 丝杠调节机构细节 输出变压器可以采用任何小型的铁芯制作,截面积在9~15cm2 都可以。因为线圈内没有直流电流通,所以采用磁性瓷铁芯也可,效果更好。这台矿石收音机就是采用的磁性瓷铁芯,一次绕组用φ0.09mm漆包线绕3500匝,二次用φ0.35mm漆包线绕180匝。配合8~30Ω的扬声器或低阻耳机都能很好地工作。 这里需要注意输出端的双声道插口的接法,正常接地端子是悬空的,两个动触头接输出变压器的两个输出端子,两个静触点也是悬空的。插头接线时要把耳机或扬声器的两根线对应接到相应的端子上。 双声道插座要选用另带一组开关的插座,电路图中“S”就是利用插座上的这个开关。 如果找不到单连可变电容器,可利用双连可变电容器的其中一组。小型密封可变电容器也可利用,这样做除Q值稍低外并没有什么影响。 天线插口、地线插口、可用接线柱接引,也可用香蕉插头接引。本机用的是香蕉插口。 分线器读盘、调谐读盘可另外制作,然后贴在底座的相应位置上。这里的两个读盘数字是直接刻在矿石机底座上的。 底座可用多种材料制作,例如电木板就是很好的选择,而且强度较大,胶合板也可,读者可根据自己现有的材料进行制作。 现代矿石机 二极管、MOSFET、再生三用矿石机 雷宝玉 本文介绍的这台矿石机是由二极管和MOSFET场效应管3DQ(3SK143-Q)组成的矿石机,通过转换开关可变换成二极管矿石机、场效应管矿石机和再生机。 图4-146 木质红酒盒 一次聚会拿到一个两瓶装的红酒木质盒(见图4-146),我当时就想,用此盒能够装一台矿石机,于是就将其收了起来。一直没有拿定主意装什么形式的矿石机,这事就搁置在一边了。有一阵论坛上讨论磁棒线圈的Q值非常高,使我想前用磁棒线圈制作一部矿石机装到红酒盒中一定不错。在论坛浏览坛友的各种形式的矿石机介绍后,心中有了大致的图形,于是决定制作一部具有组合特点的矿石机,电路如图4-147所示。 本机采用单独的天线线圈L1,是考虑尽量减少天线的分布电容对谐振回路的影响。S1是矿石机和再生机的转换开关,S2是二极管和3DQ检波的转换开关,S3是天线线圈选择开关;S4是耳机匹配变压器高阻端选择开关,S5是耳机匹配变压器低阻端选择开关。B1是由磁棒组成的耦合系统,L1是天线线圈,L2是谐振线圈,L3是3DQ的检波线圈,L4是再生线圈,L5是陷波器,B2是由高导磁率T38磁芯制作的舌簧耳机匹配变压器。 图4-147 二极管、3DQ、再生矿石机电路图 一、各种线圈的数据 (1)磁棒:用两根200mm中短波磁棒并列,再用聚四氟乙烯生料带从头到尾缠紧磁棒,之后穿入用20mL一次性针管做骨架的线圈。 (2)L1天线线圈:采用φ0.04mm×175股的利兹线在20mL一次性针管上绕20圈,每4圈进行一次抽头。 (3)L2谐振线圈:用φ0.04mm×270股利兹线在20mL一次性针管上绕60圈。 (4)L3、3DQ检波线圈:采用φ0.04mm×175股的利兹线在20mL一次性针管上绕3~8圈。 (5)L4再生线圈:采用φ0.04mm×175股的利兹线在20mL一次性针管上绕12圈,线圈中间抽头。 (6)L5陷波器:在直径31mm的NX100磁环上用φ0.04mm×60股利兹线绕60圈。绕前在磁环上缠一层生料带,绕后在线圈上缠一层生料带。 (7)B2舌簧耳机匹配变压器:在导磁率10000的T38镜面磁芯上(100圈的电感量为200mH以上)用φ0.12mm漆包线绕601圈,在39圈、56圈、85圈、120圈、170圈、219圈、269圈、347圈、491圈处进行抽头,然后用φ0.08mm漆包线接着绕到3104圈,并在694圈、850圈、982圈、1202圈、1700圈、2194圈处进行抽头,再用φ0.06mm漆包线继续绕到4390圈,并在3802圈处进行抽头。 二、其他元器件 (1)谐振回路可变电容器:采用军用小八一电台里的三连接收调谐可变电容,只用一连。 (2)陷波器调谐可变电容:365pF塑料密封三连电容,只用一连。 (3)二极管:实际型号未知,检波效果与HP5082-2835、HSMS2820相当。 (4)MOSFET场效应管:3DQ(3SK143-Q),如改用110800会更好。 (5)S1:4×2钮子开关。 (6)S2:2×2钮子开关。 (7)S3、S4、S5:1×12小单刀转换开关。 (8)直流电流表:100μA,再生机时并联分流电阻后量程为5mA。 (9)再生调节电位器:1kΩ线性电位器。 (10)CZ1/CZ2:3.5mm耳机插座。 (11)耳机:美国USI公司的sound power USI-UA1614舌簧耳机元件,直流电阻为65Ω。 (12)矿石收音机面板:家用塑料菜板改制。 三、天地线 天线:四楼南向窗口,垂直窗口甩向距窗口七八米的树上,200m以外没有遮挡物。 地线:接于室内暖气管上。 四、接收条件 接收时间:晚7~8点。 接收地点:北京朝阳区,东南三环外侧。 关闭房间的门窗。用德生PL550收音机比照确定电台的频率,从频率最低端开始到最高端结束,调节矿石机时,收音机要关掉或远离矿石机。 五、接收效果 本接收地点720kHz的信号极强,覆盖频率范围很大,首先用矿石收音机陷波器将此电台屏蔽掉。先将该电台的音量调至最大,然后调节陷波器的可变电容,将该电台的音量调到消失,这时就可以从频率的最低端开始接收电台了。 用二极管进行接收:以能够听清内容为准,先后接收到的电台频率为:558kHz、567kHz、585kHz、603kHz、639kHz、720kHz、747kHz、774kHz、783kHz、828kHz、846kHz、900kHz、909kHz、918kHz、927kHz、972kHz、1008kHz、1026kHz、1053kHz、1071kHz、1125kHz、1143kHz、1188kHz、1278kHz、1377kHz、1386kHz、1440kHz、1512kHz、1566kHz、1593kHz等,总共30个电台,效果非常不错。 用3DQ接收:开始L3的圈数绕了8圈,弱台接收效果非常不好。论坛坛友建议要调整L3的圈数与在磁棒上的位置,当调整L3的圈数,只剩3圈时,音量和灵敏度达到最佳的平衡,与二极管接收的效果十分接近。 再生接收:再生形式的接收较为麻烦,调台时要反复调整再生电位器,不然强台震耳,有时音量会大到失真,弱台也会受到鸟叫的干扰。 总体来讲,二极管接收的效果在接收弱台时的表现要稍好于3DQ,在音量上3DQ要稍好于二极管。 该矿石机如图4-148所示。 图4-148 制作完成的矿石机 六、值得注意的几点 (1)陷波器一定要用高品质元器件制作。做为矿石机中的陷波器,其通带不能宽,一定要尖锐,最好要小于9kHz,不然相邻电台也要被吸收掉。这就要求电感线圈和调谐电容的品质因素Q值要高,不然陷波器的作用就会适得其反。 (2)磁棒上的线圈和调谐电容的品质因数Q值也要高,用利兹线绕线圈和用瓷支架的可变电容是不可或缺的选择。 (3)耳机的选用:除了美国USI舌簧耳机,国产上讯SC2-300舌簧耳机也是非常适合矿石机弱台的接收。 场效应管“远程”矿石收音机 韩红 本机由天线回路、调谐回路、陷波回路3部分,共4个独立单元组成,有两个陷波器,在两个强台中间有一个弱台的情况下使用。陷波器会降低调谐回路的Q值,因此建议尽量不用。 一、天调使用花篮线圈+中短波磁棒(90mm共两支)组合,线圈内径22mm、外径40mm,用φ0.04×270利兹线/76匝,空载Q值1000以上。磁棒放在20mL针筒里,电感量在310~380μH之间可以调整,以对应天线分布电容不同造成的频率覆盖问题。 二、调谐线圈为直径150mm花篮线圈(φ0.04×660利兹线/48匝),电感量347μH,空载Q值达1200以上。 三、检波线圈直径40mm,用φ0.04×270利兹线/60匝。改变其位置可以调整与调谐线圈的耦合量,保障检波管工作在最佳状态。 四、陷波器(1):直径130mm蛛网线圈,φ0.04×175利兹线/52匝、电感量330μH。 五、陷波器(2):直径135mm花篮线圈,φ0.04×660利兹线/37匝,电感量290μH。 六、4个可变电容都使用一种瓷动片轴瓷定片支架三连,可以很容易得到所需容量。 七、检波管使用3DQ场效应管,利用其高输入阻抗和低输出阻抗特性配合使用两个串联的SC2-300舌簧电话听筒,可以不用匹配变压器。 实物图如图4-149所示。 收听效果:天线长度4m(三层楼阳台外),地线接在暖气管上,可收到603kHz(北京台“首都生活广播”)、639kHz(中央台“中国之声”)、720kHz(中央台“经济之声”)、747kHz(中央台“文艺之声”)、774kHz(北京台“外语广播”)、828kHz(北京台“外语广播”)、846kHz(中国国际广播电台)、900kHz(中国环球资讯)、927kHz(北京体育广播)、1008kHz(中国国际广播电台)、1026kHz(北京城市管理广播)、1053kHz(中央台“老年之声”)、1098kHz(中国国际广播电台)、1143kHz(中央台“民族之声”)、1251kHz(中国国际广播电台)。 单机尺寸:长250mm×宽150mm×高300mm。 图4-149 制作完成的矿石收音机 图4-149 制作完成的矿石收音机(续) 高灵敏度MOS管矿石机 李清 几年前,国外矿石机爱好者制作了MOS管检波的矿石机,当时用的是ALD110800或是ALD110900这两种零栅压MOS管。麦老师把MOS管检波的矿石机介绍到了国内,由于这两种零栅压MOS管很贵,并且不易买到,于是就有国内的爱好者通过大量试验找寻这两种管子的代替品,功夫不负有心人,终于找到了检波效果很好的耗尽型双栅MOS管3DQ和3DE。 图4-150所示就是当时流行的MOS管检波的矿石机电路图。 图4-150 MOS管检波的矿石机 笔者曾经制作过这个电路的矿石机,使用的各元件如下。 L1、L2:在30mL注射器针筒上用φ0.04mm×270利兹线绕60匝,L2同样在30mL注射器的针筒上用同样的线绕5匝。针筒内插入3只φ10mm×200mm的中短波磁棒。 C1:365pF高频瓷定片支架的高Q值可变电容。 C2:2200pF电容。 双栅MOS管:3DQ(3SK144)。 耳机:阻抗1000Ω的舌簧耳机。 该机做好后效果很好,在位于北京北部远郊区,不用外接天地线,仅凭磁性天线,白天就可以收到639kHz的中国之声,声音还不小呢。有时还能收到另一个弱台,晚上能收到三四个台。南京的网友也制作了相同的机器,在南京市区不用天地线也能收到几个台,可见用MOS管3DQ检波的矿石机性能很好。 仔细研究后发现,这个典型的MOS管检波的矿石机电路还可以改进,改进后灵敏度可以进一步提高。 制约MOS管检波灵敏度的原因分析 如图4-150所示,典型的MOS管检波电路有两个输入端,一端是MOS管的栅极,是检波的控制端,输入阻抗极高,对调谐回路没有太大的影响。另一端是MOS管的源极,是检波的能量获取端,输入阻抗很低,这端通过L2从调谐回路中获取检波的能量,使检波能够输出推动耳机的能量。正是由于这端输入阻抗很低,L2才只能用很少的匝数,并且还要远离调谐回路,以免严重影响调谐回路的Q值。(这一点与晶体管收音机中管子基极与调谐回路耦合时,由于晶体管输入阻抗很低,基极线圈匝数必须很少的道理是完全一样的。) 从下面的试验中就可以看到,匝数已经很少了的L2即使是远离调谐回路仍然对调谐回路有着严重的影响。 测试电路如图4-151所示。 图4-151 测试电路图 此处,L1绕60匝,L2绕5匝,并使用φ0.04mm×270股利兹线,骨架用20mL注射器管,再插入3只φ10mm×200mm中短波磁棒。 高频信号发生器输出639kHz调幅信号到单环天线,调幅参数为1000Hz音频,调幅度30\\%。单环天线中心到磁棒中心约500mm。使用超高频毫伏表通过2.2pF和12pF电容组成的分压器,测量调谐回路两端的信号电压。 在试验中可看到: 当L1、L2的间距为45mm时,毫伏表测得分压器输出的信号电压是3.4mV。 将L2断开,3DQ源极直接接地,去掉了源极通过L2对调谐回路的影响后,毫伏表测到的信号电压大于7.3mV。 可以计算出断开L2时电容分压器输出的高频电压是有L2时的7.3/3.4=2.15倍,调谐回路的输出电压与有载Q值成正比,所以去掉L2的影响后,调谐回路的有载Q值是原来的2.15倍!因此MOS管源极电路的低输入阻抗通过L2影响调谐回路的有载Q值,是制约灵敏度提高的重要原因。 提高灵敏度的方法及原理 MOS管要正常工作就要通过L2从调谐回路中获取能量,否则检波器将无法工作。我们总是想要检波器输出的能量大些,这就需要L2与调谐回路耦合得紧些,以便能从调谐回路多获取些能量。但是这样做的结果往往适得其反,获取的能量越多,调谐回路的有载Q值越低,调谐回路能提供的能量就越少。因此L2的耦合程度有一个最佳值,达到这一值时检波器输出的能量达到最大值,这时无论耦合变紧或是变松,输出能量都将下降。 那么还有没有办法提高检波器输出的能量了?当然有!这个方法可以提高检波效率,使L2从调谐回路中获取的能量在基本不变的情况下,检波器的输出能量得以提高,从而提高了接收灵敏度。 从MOS管的工作原理中可知,3DQ这种N沟道耗尽型MOS管,在小信号时,如果栅极正电压越高,管子的导电沟道就越宽,沟道电阻就越小。反之,加到栅极的负电压越小,导电沟道就越窄,沟道电阻就越大,直到导电沟道被夹断而消失。 在MOS管检波时,管子的源极与漏极就相当于一只二极管,这只二极管是导通还是截止就要看栅极信号的情况了。当栅极加上正栅压时,导电沟道存在,这就相当于二极管正向导通。反之,当栅极加上负栅压时,导电沟道就减小或消失,这就相当于二极管反向截止。如果正栅压越高,沟道电阻越小就相当于二极管的正向电阻越小,反之负栅压越负导电沟道就越窄直至消失,这就相当于二极管反向电阻越大。我们都知道,二极管正向电阻越小越好,反向电阻越大越好,正反向电阻的差别越大检波的效果越好,检波的效率越高。所以MOS管的栅极信号电压越高,才能在管子导通时得到更小的沟道电阻,在截止时得到更大的沟道电阻,从而得到更好的检波效果,得到更高的检波效率。从前面的测试中我们知道,由于L2从调谐回路中获取能量使调谐回路输出到MOS管栅极的信号电压较低,那么,有没有既不影响L2从调谐回路中获取能量,又能提高加到MOS管栅极的信号电压呢? 办法是有的,其实也很简单,只要再加一个调谐回路单独给MOS管栅极提供信号即可,这个新增加的调谐回路要距原来的调谐回路足够远,避免与原来的调谐回路发生耦合,还要与原来的调谐回路同步调谐,其电路如图4-152所示。 图4-152 高灵敏度MOS管矿石机电路图 增加了MOS管栅极调谐回路后,L2仍从原来的调谐回路获取能量,获取的能量基本上没有多少变化。但是由于没有了L2的影响,栅极从增加的调谐回路中获得的信号电压是原来的两倍多,从而提高了检波的效率使检波其输出能量变大,从而提高了接收灵敏度。 实际制作与收听效果 按照这一电路制作了矿石机,为了避免两个调谐回路发生耦合,两磁棒线圈距离是320mm,中波最短的波长也有180多米,远长于两磁棒线圈之间320mm的距离。因此两磁棒线圈收到的无线电信号的相位差极小,不会影响检波的导通角。 两调谐回路的线圈用φ0.04mm×270股线在30mL注射器管上绕55匝,L2用同样的线绕5匝,每只线圈内插3只φ10mm×200mm中短波磁棒,测量Q值如下: L3的Q值测量:Q=700。 L1的Q值测量:Q=780。 原打算用图4-153中的那只抽头线圈作L3,后来感到用抽头不方便,实际制作时L3使用的是另一只重新绕制的同样匝数的不抽头的线圈。 为了防止双联等金属器件影响高Q线圈的Q值,两个调谐回路的高Q线圈用有机玻璃棒架高远离金属器件。可变电容是360pF铜片双联。 组装好的矿石机,为了便于统调,微调电容只能安装在有载Q值较低的源极调谐回路上。 两个调谐回路线圈的电感应尽量一致,由于磁棒的导磁率有较大的离散性,故线圈绕好后要测一下电感量,如电感量稍有不同,可调整线圈在磁棒上的位置使之一样,否则要将电感量大的线圈拆掉几圈。统调时先在频率低端收一强台,再调整线圈在磁棒上的位置使声音最大,再找高端的一台,调整微调电容使之声音最大,然后仔细反复进行这两项调整,直至效果最好。如果发现频率的高低端不能兼顾就是双联的两联误差太大,换双联或是通过调整双联动片两边的花片试一试。 统调后,接收灵敏度比原来的MOS管矿石机高了很多,同时用一只大罐头耳机,在昌平,原来的机子白天一般只能收到一个强台,有时还能收到一个弱台,而这台机子白天能收到3个台。晚上效果更好,原来的机子晚上能收到两三个台,这台机子则能收到6个台,晚9点收到了639kHz中国之声、1026kHz北京城市管理台、1251kHz中国国际广播电台、1332kHz的一个台、1548kHz山东的一个电台和1585kHz的一个电台。 更有意思的是另一个接收试验。从电路图中可以看出,如果把L2从L3磁棒上取下套到L1磁棒上就成了典型的MOS管检波矿石机电路,此时调到一个弱台后再旋转机器的方向使弱台听不见了,这时不改变机器的方位再把L2套回L3磁棒上,弱台又能听到了。这就证明了这个电路比典型电路灵敏度高。 关于这个电路以上试验中的良好表现,只是我在北京昌平地区信号弱的环境下试验时的表现,在强信号下这个电路的表现如何我没试验过,也就不得而知了。 电路缺点 这一电路虽然灵敏度很高,电路也很简单,但是存在着一个缺点,就是制作、统调比较难,原因就是调谐回路的有载Q值很高造成的,真是“成也萧何败也萧何”呀!因为调谐回路有载Q值高,才能获得了高灵敏度,但是高有载Q值使得调谐回路谐振点很尖锐,虽然因此选择性也提高了,但是两个调谐回路的接收频率全程范围内达到完全跟踪不太容易做到。谐振参数稍有误差,两个回路的谐振点就不重叠了,如果Q值低,即使两个回路的谐振频率稍有不同,但是通带宽,信号还是能分别通过这两个调谐回路,Q值高通带窄就会发生信号只能通过一个回路而通不过另一个回路的情况,产生这一问题完全取决于双连可变电容的质量,要求双连的两连在调谐的全程的各个角度的容量误差不能太大,否则就很难做到统调。我的360pF铜双连电容共有两只,开始用的是另一只,统调总也做不好,一测才知道那只双连电容的误差太大,有的角度容量差几十皮法,换了这只好多了。 如果统调不好,这个电路的灵敏度反而不如典型的电路,如果两个调谐回路谐振频率相差较远,甚至会收不到信号。制作完成的机器如图4-153所示。 图4-153 制作完成的机器 中波3DQ矿石机/太阳能再生机 聂建军 用成品小仪表盒装一台3DQ矿石机/太阳能再生机,小按钮开关做矿石机/再生机功能转换。由于3DQ场效应管是栅极触发,源极漏极夹断检波,在供电状态会产生类似于传统红锌矿的张弛振荡再生放大效果。3DQ场效应管再生起振点低,电转换效能高,在0.4V就能有可用再生,利用台灯光就可以再生收听。配接太阳能充电板、小型镍氢蓄电池,晴朗天充电2h,每天中等音量听音1h可用10天。该机用3DQ双栅分接场效应管电路、中短波短磁棒线圈和介质可变电容,这使得矿石机状态有较好的选择性和中等灵敏度。利用大环或稍短一些的长天线,根据强弱台选择矿石机状态与再生状态转换,中等以上信号可以用矿石机状态接收,弱台用供电再生状态接收有较好的接收效果。 本制作电路图如图4-154所示。 图4-154 3DQ场管矿石机/太阳能再生机电路图 场效应管:3SK143-Q/3DQ; 可变电容C1:360pF (Q=9600),动定片全铜片; 磁棒线圈:φ10mm×70mm中短波磁棒,5mL针管骨架,φ0.04mm×60mm利兹线; L绕72T(空载Q=485),L2、L3正反绕向各4T,L4绕15T; G1栅极电阻:33kΩ; 再生电位器:1kΩ; 滤波电容C2:4700pF; 旁路、隔直电容C3:10μF,再生状态用于音频旁路,矿石机状态起隔直作用; 音频变压器T:RM6磁芯、导磁率为5000, φ0.09mm漆包线总圈数1025T,总线阻75Ω、 300Ω,线阻9.5Ω,总电感5.8H、阻抗区间 3kΩ~2.5kΩ~2kΩ~0.3kΩ~0; 耳机PHONE:SC2-300舌簧耳机; 太阳能充电板:40mm×60mm,5V; 充电电池:镍氢,3.6V/Ah,满充4.21V; 二极管:1SS109; 按钮开关:2×1; 天地线接线柱:小型; 可变电容、电位器旋钮:小型; 机盒:成品仪表盒; 长、宽、高:100mm×66mm×38mm。 图4-154中的场效应管可使用3SK143、3SK144系列双栅场效应管。磁棒线圈用φ10mm×70mm的高Q中短波磁棒,φ0.04mm×60mm股利兹线在5mH针管上绕制,主谐振线圈L的空载Q=485,L2、L3为检波再生线圈,L4为天线耦合线圈。可变电容采用Q=9500的360pF动定片全铜片的介质单连电容。矿石机/再生机转换为2×1小型按钮开关。供电电池用小型3.6V/Ah镍氢电池,由于机箱较小,充电电池不宜过大,过大的铁质电池外壳会对磁棒线圈Q值影响较大,影响接收灵敏度。采用40mm×60mm面积的5V太阳能充电板,阻止反流二极管用低正向导通电压的1SS99,实测阳光充足时相对镍氢电池为+0.79V,对耐过充表现较好的镍氢电池不会造成过充损坏。音频变压器用RM6 5000导磁率小磁芯绕制,配灵敏度较高的SC2-300舌簧耳机。 矿石机接收时,S-1/S-2开关打开,用电位器调音量。再生状态接收时,S-1/S-2开关闭合,调整电位器,选择再生强弱点和控制音量。太阳能电池充电时,S-1/S-2同矿石机状态,开关打开。 制作完成的机器如图4-155所示,制作该机的数据参数如表4-3所示。 收听效果 (1)由于小型机使用短磁棒线圈,矿石机状态灵敏度受限,在北京西郊使用阳台大环天线,可收听到北京地区电台、天津相声台(567kHz)等,其中中国之声电台(639kHz)音量最大。 (2)使用阳台大环天线并启动供电再生装置时,北京地区所有电台、天津相声台都在中等音量以上,晚上天波较好时,可远程接收外地中波电台,最远可至湖北、中国香港等地。 (3)大环天线放置位置与信号强弱关系很大,钢混建筑外四角信号最强。 图4-155 制作完成的收音机 图4-155 制作完成的收音机(续) 表4-3 矿石机变压器设计取值表 直径0.8m的场效应管检波中波大环矿石收音机 马福全 此机不用天地线,能量来自直径0.8m的环形线圈。L1是用φ0.16mm的漆包线自己绞的,共136股,长27m左右。之后用木方做的框架,8个20mL的注射器做线圈支架,每个注射器里插入一段磁棒,共绕9圈,电感136μH做主调,L2是用60股φ0.16mm漆包线自己绞的,绕成直径为120mm的花篮线圈,共49圈,每7匝抽头,做场管3DP的S极线圈,通过鳄鱼夹调节,接入电路,与L1的距离可以调节。大环矿石机在小区3楼一直受FM信号干扰,通过多次试验,在场效应管3DP的D极加入一个线圈L3(在2.5mL的注射器上用φ0.5mm的镀银线绕25圈),才能消除FM干扰,使整个波段变得很干净。可变电容器是222-1军机磁轴双连,每连为15~300pF,其中1连通过开关控制接入电路,收高端电台时用单连,收低端电台时用双连,线路是双栅分接电路,表头使用50μA小表头,高频部分连接线用镀银线,耳机是上讯老版电话听筒SC2-300舌簧耳机(2个串联)。 本机电路图如图4-156所示,制作完成的本机如图4-157所示。 图4-156 收听情况:机器于2012年12月9日制作完成。我居住的位置在黑龙江佳木斯郊区的汤原县,在小区3楼阳台收听,用SC2-300舌簧耳机2个串联收听,白天当地的几个电台都能收到,信号都挺强,由低到高分别是:540kHz中国之声,666kHz佳木斯,900kHz黑龙江新闻, 1143kHz佳木斯经济台,还有几个韩语台和日语台。晚上信号加强了,收到的电台也随之增多了,除了当地的电台,还有许多远地外省的电台和韩语、日语电台,信号都挺强。大致统计了一下,能清晰收听的电台20个以上,弱台很多……机器方向性极强,线圈侧面对着电台时信号最强。要是有干扰,可以通过调整机器方向来避让。 图4-157 无天地线箱式便携中波场管矿石机 黄进武 我做的6号矿石机是无天地线箱式便携中波场管矿石机,有较高的灵敏度,可以外出携带收听。在中波全频段有很轻微的再生音,但收到电台就没有了。耳机线和矿石机在一定的位置就会有再生,3DQ场管G1、G2并接电路也会产生再生效应。 矿石机线圈电容大,做的单回路矿石机选择性不好,所以我做了双回路矿石机,直至做到三回路。三回路矿石机的问题是灵敏度不高,所以笔者曾在双回路和三回路之间徘徊,大名鼎鼎的陈宪文矿石机采用了双回路结构。这种双回路矿石机非常适合爱好者制作,让我想起了小时候在少年宫学习制作140线圈矿石机电路,那是一种怀念双回路矿石机的情结。 6号矿石机于2012年1月制作完成,后来进行场管双栅分接和仿725变压器互耦接法的改进,电路原理图见图4-158。 图4-158 6号矿石机电路图 6号矿石机电路为双回路双调谐场效应管矿石机电路,机壳用电性能好的聚丙烯PP材料做成,线圈用接收效果好的φ0.04mm×660多股纱包线,保证无天地线接收的灵敏度。采用场效应管双栅极分接方式,将场效应管G1栅极串接39kΩ电阻后接S源极,与G1、G2双栅并接比较,灵敏度和选择性大增,在白天多收了江苏台和镇江台。 接收效果 我做了几台矿石机,6号矿石机的无天地线远距离接收效果最好。我白天远距离收听、测试,晚上进行天波接收效果比对。在江苏常州西郊2楼,白天能收到的远程中波电台有: 702kHz江苏台,声音很小,常州至南京150km。 1008kHz无锡台,声音可以听清,常州西郊至无锡50km。 1221kHz镇江台,声音小,常州至镇江70km。 制作材料 机壳:聚丙烯PP材料做的便携箱,长44cm×宽30cm×高16.5cm; 线圈:L1-φ0.04mm×660股、双股19圈,L2-φ0.04mm×660股、双股18圈,L3-φ0.04×660股、三股3.5圈,长18.5cm、宽12.5cm; 可变电容:南京产韶山牌2×270pF薄膜介质可变电容,动定片全铜材料; 场效应管:型号3SK143-Q/3DQ双栅场效应管,G1、G2双栅分接; 半可调电容:5/20pF瓷介半可调电容; 滤波电容:2200pF; 矿石机音频阻抗变压器:YYDZ仿725型,采用双线圈耦合隔直接法,在耳机端用毫伏表测量不隔离直流8mV,隔直接法10mV; 耳机:上海电讯器材厂SC2-300舌簧耳机,进行了去调节簧增音改造,感觉改造后的SC2-300耳机效果非常接近田737高级舌簧耳机。 制作完成的矿石机如图4-159所示。 图4-159 制作完成的矿石机 图4-159 制作完成的矿石机(续) 实验无天地线磁棒矿石机 韩红 磁棒具有较强的导磁能力,它能聚集空中的无线电波,在线圈中感应出信号电压。本机用4支200mm中波磁棒组成140mm×200mm小框,不用天地线,即可以收到6个台。在设计中可以接上天、地线,以增加收台数量。整机电路图如图4-160所示。 图4-160 整机电路图 整机用直径16mm的PVC塑料管制成,价格低廉又能保护磁棒和线圈。线圈用φ0.04mm×60股利兹线平绕65匝,每个线圈调到380μH然后固定,线圈要涂蜡做防潮处理。线圈两两并联,然后再串联,电感量310μH。配270pF可变电容,电路选用双栅场效应管3DQ检波,L9绕在其中一支磁棒上共10匝,耳机用2只SC2-300串联而成,阻抗为600Ω。如果用高阻耳机收听,则要用二极管检波,方法是拆去3DQ,在G1和D的位置处装上二极管, L9此时无用,可不绕。 制作过程以及完成的整机如图4-161所示。 图4-161 制作过程 便携式矿石收音机 韩红 这是一台花篮线圈便携式矿石机,设计目地是在户外使用。该机主要有以下几个特点: (1)为了保证本机性能,本机天线回路、调谐回路及吸收回路都使用高Q值线圈。 (2)调谐回路使用直径达150mm(6英寸)花篮线圈,为了方便携带,采用可拆卸方式。 (3)通过开关转换,可选择场效应管和肖特基管检波,还可以方便地更换矿石或二极管来检验其检波效果,另外,选用了国外戴夫先生的检波电路及国内陈宪文先生的检波电路,以适应在不同的环境下使用。 (4)天线调谐回路线圈采用抽头方式,分为“高端”和“低端”两个挡位,以适应各种天线,使之做到中波全覆盖。 (5)本机采用双回路双调谐电路和可移动的陷波器,以避免在不用陷波器时对调谐电路的影响。 整机电路图如图4-162所示。 图4-162 本机电路图 下面分别介绍本机的制作要点。 (1)如图4-163所示,本机的调谐线圈L2用φ0.04mm×660股利兹线绕40匝,直径150mm,电感量254μH,空载Q值在800kHz时为1136。线圈装在一个有机玻璃支架上。 图4-163 (2)如图4-164所示,天线线圈L1a、L1b用φ0.04mm×270股利兹线绕30+12匝,外径38mm,内径21mm,配两支65mm中短波磁棒,电感量42匝/163μH,30匝/85μH/Q=950。调整与主振线圈距离可改变耦合量。 (3)如图4-165所示,检波输出线圈L3为场效应管专用,线径及外形同L1,圈数50匝,电感量38μH。通过改变与主振线圈距离来调整输出量。 (4)如图4-166所示,陷波器振荡线圈L4与L1相同,圈数60匝,与两支70mm中短波磁棒组成,电感量250μH,Q值为980。配365pF可变电容。通过调整与主振线圈距离可以改变吸收量。 (5)如图4-167所示,天调可变电容是340pF等容三连,将其中一连拆片至200pF。 (6)如图4-168所示,主调谐可变电容使用一种钢质瓷支架宽片距产品,在中波段Q值较高,容量为12~365pF。 图4-164 图4-165 图4-166 图4-167 (7)如图4-169所示,吸收回路用“复旦”小型空气可变电容,体积为25mm× 25mm×35mm,容量为12~365pF。 图4-168 图4-169 (8)如图4-170所示,活动矿石架设计成可拆卸形式,可方便更换矿石及触针,也可更换各种二极管。 (9)如图4-171所示,上盖板由2.4mm黑色有机玻璃和8mm厚聚丙烯板组合而成。 (10)如图4-172所示,天线调谐和检波调谐都使用了熊猫减速旋钮。 图4-170 图4-171 图4-172 (11)如图4-173所示,电路有5种检波方式。 图4-173 (12)如图4-174所示,机箱采用市面上销售的成品,更换合页后,机盖可以与机箱分离。在里面打出隔层,用于存放调谐线圈。整机体积为40mm(长)×25mm(宽)×25mm(高)。 图4-174 (13)机器内部装配图如图4-175所示。 (14)机器在收起时状态如图4-176所示。 图4-175 图4-176 (15)机器在使用时状态如图4-177所示。 图4-177 (16)活动矿石和外接二极管在工作时情况如图4-178所示。 图4-178 (17)本机实际接收效果良好,天线架在三层楼阳台外长约4m,地线接在暖气管道上,白天可收到: 1)603kHz 北京台《首都生活》 2)639kHz 中央台《中国之声》 3)720kHz 中央台《经济之声》 4)747kHz 中央台《文艺之声》 5)774kHz 北京台《外语广播》 6)828kHz 北京台《北京新闻》 7)846kHz 中国国际广播电台 8)900kHz 中国环球资讯 9)927kHz 北京体育广播 10)1026kHz 北京城市管理广播 11)1143kHz 中央台《民族之声》 12)1251kHz 中国国际广播电台 到了晚上还可收到: 1)1008kHz 中国国际广播电台 2)1053kHz 中央台《老年之声》 3)1098kHz 中央台《藏语频率》 4)1278kHz 河北新闻台 5)1377kHz 中央台《中国之声》 6)1422kHz 太原新闻台 7)1548kHz 山东台 8)1593kHz 中央台《中国之声》 (18)在户外收听时,有时会出现较大的噪声干扰,如公路旁、小溪边,或有风的树林里,这时就需要有一副降噪耳机。为该机配套的是用“克拉克”耳机改造的一副耳机(见图4-179),原配的低阻动圈耳机灵敏度较低,不适合矿石机使用,改造的耳机换装SC2-300“舌簧”电话听筒,两只串联阻抗为600Ω。 图4-179 中短波磁环小型矿石收音机 雷宝玉 有幸从朋友处得到了几个R40C1型磁环,磁环的导磁率为40,工作频率为50MHz,有朋友测得它用利兹线绕的调谐线圈Q值过千。磁环体积小,手里有现成的塑料小盒,用它做个用3DQ检波、输出阻抗与低阻舌簧耳机匹配、不用匹配变压器直接带动耳机的中短波小型矿石机。 1.矿石收音机线路图 如图4-180所示,该电路图采用标准的3DQ检波线路,用2×2开关进行中、短波转换,线路简单,利于制作。由于在我的住处,央广720kHz“乡村之声”电台200W的发射功率对其他电台的干扰很大,为此在天线回路中加入了LC并联回路组成的陷波器,来消弱720kHz对其他电台的干扰。在陷波器两端并联一个短路开关,在收听短波和中波没有强台干扰时,将陷波器短路,以减少陷波器对天线信号的影响。 图4-180 电路图 2.元器件选择 (1)磁环:主调线圈选用北京森力电子技术有限公司(798厂)生产的镍锌R40C1高Q磁环,导磁率为40,工作频率为50MHz。中波磁环规格为外径37mm×内径23mm×高15mm,短波磁环规格为37mm×23mm×7mm。经测试,R40C1磁环用270股利兹线在整个中波频段的Q值都上千,短波10MHz时Q值在600左右。陷波器线圈选用北京森力电子技术有限公司(798厂)生产的导磁率为100的镍锌磁环,规格为外径31mm×内径18mm×高7mm,它也是一种Q值在500~600的高Q值磁环。 (2)利兹线:主调、天线、3DQ检波线圈采用性价比较高的φ0.04mm×175股线,如用φ0.04 mm×270股以上线,磁环绕不下;陷波器线圈采用φ0.04 mm×60股线。 主调可变电容:受矿石机体型的限制,该机采用2×340pF+2×25pF(4×14pF微调)四连塑料薄膜可变电容,四连并联用于中波,2×25pF并联用于短波。 (3)线圈:中波主调线圈绕37圈,电感量为115μH;天线线圈绕12圈,每2圈一个抽头;3DQ检波线圈绕3~4圈。陷波器线圈绕47圈,电感量为177μH。短波主调线圈用直径1mm的镀银绕23圈,电感量为14μH,接收频率范围为6~18MHz;3DQ检波线圈绕2~3圈;天线线圈绕2~4圈。 (4)耳机:美国RCA公司的 “Big Cans”,俗称“大罐头”(见图4-181),加了抗噪耳罩。 图4-181 美国RCA公司的 “Big Cans”耳机 (5)检波元件:3SK143-Q(3DQ),插针、插母连接。 (6)开关:S1天线用1×12单刀开关,S2为短路陷波器的微型拨码开关,S3波段用小型2×2开关。 (7)其他:塑料小盒、天地线接线柱、耳机插座、旋钮、磁微调电容等。图4-182所示为矿石机盒以及元器件。 图4-182 矿石收音机盒以及元器件 (8)天线:北京华威桥附近,4楼南窗,向窗外树上抛八九米细塑料多股线。 (9)地线:房间暖气管。 3.布局安装 由于小型矿石收音机机壳小,因此左手位安装陷波器可变电容,右手位安装调谐可变电容。可变电容用双面不干胶贴固定,磁环和开关用热熔胶固定。3DQ焊在覆铜板上和插针焊在一起,用插母连接,在线路中方便插拔、更换,耳机插座在右侧下部位置。中短波调谐线圈叠在一起。整体布局见图4-183。 图4-183 布局 4.调试 本机元器件不多,只有中波频率覆盖范围要进行简单的调试。因为调谐线圈用的是磁环,所以覆盖频率范围要在安装之前进行,将矿石机临时连线后,先在频率低端找到一个电台,比如567kHz,适当增减线圈的圈数,使频率度盘达到相应的位置,再在频率高端找到一个电台,比如1521kHz,调整微调电容,使频率度盘达到相应的位置,调试就完成了。 5.几个要点 (1)要加入陷波器,首先想到的是LC并联谐振电路,其特点是谐振时对谐振频率的阻抗最大,对其他频率的阻抗很小,利用这一特点将其串联在天线回路中,将谐振点调在需要屏蔽的频率点就可以对其进行有效的衰减,从而减小对相邻电台的干扰。 不过这个LC并联谐振电路的带宽最好在9kHz以内,防止相邻的电台也被阻止,也就是说电感和电容的Q值不能太低。因为是小型机,元件体积不能大,要尽量小,就选用导磁率为100、外径31mm的磁环,用φ0.04mm×60的利兹线绕到175μH,Q值在500~600,配塑料2×270的双联并联使用。 (2)陷波器的使用:先将陷波器的电容调到最大或最小,然后将调台电容调到需要屏蔽的频率上,这时再调整陷波器电容,使需要屏蔽的频率音量减到适当的大小即可。当接收弱台音量较小时,可适当地微量调整陷波器电容进行辅助调谐,就能够收到较好的接收效果。 (3)为了保持磁环线圈Q值的稳定,磁环在绕制线圈之前要用蜂蜡煮一下,将磁环体之内的潮气驱除。 (4)绕中波天线线圈时,要垫入一小条聚四氟乙烯薄板,将抽头处绕在薄板上,不抽头处绕在薄板下。抽头局部如图4-184所示。 图4-184 抽头局部 (5)磁环要用热熔胶固定,对Q值的影响很小。 (6)如没有高灵敏度耳机,也可用老版SC2-300耳机,将其原线圈拆掉,用外径0.1mm的漆包线绕1300匝后替代,灵敏度与“大罐头”相比,只在极弱的信号下有差距。 (7)耳机两端并联的电容是不可或缺的,它在3DQ检波的情况下起到了至关重要的作用。如果没有这个电容,这台矿石收音机有混台现象,远程弱台收不到,加了这个电容后,选择性和灵敏度有较大的提升,音量也有所增加。这个电容的容量在几百皮法到几千皮法之间均可,也可在装机时进行筛选而定,本人用的是4700pF的独石电容。 (8)3DQ中、短波检波线圈的具体圈数要根据接收的天线情况和地理位置而定,在临时搭接试验时进行调整为好,不同的条件也可能是2圈或5圈,或其他圈数,这要由试验来定。 (9)焊接利兹线时,不需要进行去漆处理,用烙铁配合松香直接上锡即可。 该机的内部及外观见图4-185。 6.接收效果 接收地点:北京东三环路华威桥附近。 中波:接收时段为清晨、上午、下午、晚间等不同时段。高低端接收到的频率为558~1593kHz,低段频率分割距离较开,高段频率距离很近。在不同时段确认的电台有:央广、北京、天津、廊坊、涿州、河北、河南、郑州、南京、山东、济南等,还有些收到的电台没能确认台标。收到的电台通过德生PL550收音机确认频率是:558kHz、567kHz(天津)、585kHz(廊坊)、603kHz(北京、郑州)、639kHz、720kHz、747kHz、774kHz、783kHz(河北)、846kHz、900kHz、909kHz(天津)、927kHz、972kHz、1008kHz、1026kHz、1053kHz、1098kHz、1125kHz、1143kHz、1206kHz(南京)、1251kHz、1278kHz、1332kHz、1359kHz、1377kHz、1386kHz(天津)、1431kHz、1449kHz、1476kHz、1530kHz、1566kHz、1593kHz。还有801kHz涿州台,音量很小,听得不清。 曾有两三个电台在同一频率发声的现象,开始时笔者以为是混台,后来经德生PL550接收对比发现,是不同地点的电台使用相同频率造成的。比如,603kHz平常是北京故事台,当它停播后是郑州台播音。因为矿石收音机在高段接收信号较弱,各电台的音量相差不大,这种现象在高段较多。 短波:接收时段为晚上8~9点。满刻度都有电台,短波接收到的电台音量较大的有中广中文台,韩语、日语广播等。音量小的电台很多,但常常是飘忽不定的,一个频率位置最多时有3个电台“飘来飘去”。 这中、短波磁环小矿石机的接收性能有不错的表现,主要有如下原因:R40C1磁环Q值高,电台分割清楚;φ0.04mm×175利兹线性能好,粗细合适;3DQ的输出阻抗和耳机阻抗匹配。如果制作成台式矿石收音机,并将塑料薄膜可变电容换成高Q值的空气可变电容,接收效果还会有较大的提升。 以上只是本人的主观感受,有兴趣的矿石机爱好者不妨也组装试试,我想一定会带给您带来不小的惊喜。 图4-185 组装完成的整机 短波大环矿石收音机 马福全 此机不用天地线,靠自身的天线就能接收短波段信号。理论上,接收频率范围为1.99~20.46MHz,用电感表测得1圈电感为2.42μH,3圈电感为14.56μH,配合25~440pF可变电容,曾减小线圈,收到了1566kHz电台信号。 本机电路如图4-186所示。 图4-186 短波大环矿石收音机电路 机器所用元器件: 天线是直径为8mm,长为9m的紫铜管,弯成3圈成弹簧状,直径为0.85m,通过接线柱调节接入机器;L2线圈用截面积为6mm2 、1.5m长铜线绕4圈,直径为110mm,通过纯铜鳄鱼夹调节接入;可变电容器采用磁轴镀金可变电容器25~440pF,检波采用场效应管3SK143系列的3DP,线路是双栅分接电路,表头是50μA小表头,高频部分连接线用镀银线,耳机是上讯老版电话用SC2-300舌簧耳机,采用2个串联的形式。 制作完成的整机如图4-187所示。 收听情况: 我居住的位置在黑龙江佳木斯郊区的汤原县,在小区3楼阳台收听,用SC2-300舌簧耳机(2个串联)收听, 白天能收到微弱的几个台,夜晚降临,机器收台多了起来,信号也随之加强,频率范围在3~12MHz,夜晚这些台的信号都挺强。在平房收听没有干扰,在小区3楼有FM干扰,无法消除(我这里距FM发射塔太近了,有1km左右),好在当收到短波信号时能够压制住FM干扰,不影响收听使用。 图4-187 制作完成的矿石收音机 专为远程接收设计的变耦合度矿石收音机 翟希文 近年来,用简单的矿石收音机进行远程接收成为很多无线电爱好者研究的热点。为了达成这个目的,我也和大家一样,开始为矿石收音机远程接收准备条件。 接收环境 我家地处安徽中部,串波电台很多,普通晶体管收音机在室外随手就能收到十七八个,当然在钢筋混凝土结构的楼房室内就只能收到四五个本地强台了。这些电台信号强度相差很大,有些电台信号强大到在普通双回路双调谐矿石收音机中,表现出频率相差100kHz的电台都淹没掉。因此,提高选择性又不过分降低灵敏度是个很重要的问题,能否解决这个问题将直接关系到这次远程接收矿石收音机实验能否获得成功! 我家位于合肥市区偏西南的位置,周围高层建筑较多,但距离我的位置相对较远,中波波段接收条件算是中等偏上,根据这个条件有目的地设计矿石收音机,有可能获得很好的效果。 设计要点 要实现远程接收,最基本的要求首先是架设良好的天地线,使得从遥远空中传来的微弱电台信号成为可能;其次,接收使用的矿石收音机本身必须具有良好的选择性,否则本地的强台信号会将临近频率的弱信号全部压制而无法收听;再次,用于远程接收的矿石收音机本身的损耗应该越小越好,免得本来就非常微弱的信号在传输、检波过程中被消耗掉。最后则是使用一个高效率的耳机来将检波后的电信号还原成声波,供耳朵聆听。 电路结构设计 从天线中送来的信号中,本地强台很强,高选择性是在远程接收最重要的因素。一般情况下,提高选择性都是以牺牲灵敏度为代价的。怎么样既能大幅度提高选择性,而尽可能少地降低信号传输时的损失来兼顾二者呢?这就是本次实验的难点。 提高选择性的途径: (1)提高谐振电路的品质因数:实现途径是采用高品质的线材和骨架绕制电感线圈,采用高品质的可变电容器。 (2)减轻谐振电路的负载:传统矿石收音机一般使用阻抗在1000~4000Ω的电磁耳机来还原声音,相当于给谐振电路加上很重的负载,让谐振电路的有载Q值大大降低。这里我们使用阻抗高达150MΩ的超高阻高效率舌簧耳机,有效地减轻了检波器的负载,使得谐振电路的有载品质因数得以提高。 (3)增加谐振电路数量:为了获得更窄的通带和更尖锐的谐振特性,将强信号掩盖的弱信号检出而又不过分增加插入损耗,在这里我选择了双调谐谐振电路。这是一个折中的方案,适当兼顾灵敏度和选择性两方面的需要。 (4)使用变耦合度的调谐电路:双调谐电路存在着随信号增强和耦合度增加会出现双峰效应的问题,信号越强或者耦合越紧,这个问题越显著。变耦合度方式可以有效地解决这个问题。(5)设置陷波器:由于我所在的接收位置不到10km外有功率200kW的中波电台,仅仅依靠上面的双调谐电路并不能将它们发出的强大信号掩盖下的弱台信号检出,因此高品质的陷波器就是一个必不可少的装置了。 经过一些测试实验后,最终的电路变成了如图4-188所示的样子。 图4-188 高选择性矿石收音机原理图 这个电路由输入调谐回路、输出调谐回路、陷波器、检波器、班尼电路、音频阻抗变换等部分组成。 输入调谐回路由2×440pF可变电容器的1/2、天线耦合线圈L1、输入调谐线圈L2(这两个线圈套在插入一根80mm长磁棒的同一线圈管上)构成,配有5/25pF的补偿微调电容。 输出调谐回路由另1/2个2×440pF可变电容器、输出调谐线圈L3、输出耦合线圈L4(L3和L4也是插有长80mm磁棒的同轴线圈)以及5/25pF微调电容构成。 陷波器由一个高Q线圈L5和一个最大容量为290pF的空气介质可变电容器构成。 检波用一个HP5082-2835的晶体二极管完成,这是一种Rd 值很高的二极管,我用交流阻抗140kΩ的舌簧耳机仍不能满足匹配,后来因此而特别加装了阻抗变换器来尽量满足这个要求。 因为输出阻抗很高,滤除检波后的载波用一个容量仅39pF的电容完成。 班尼电路提供了一个高阻抗的检波器直流通路,音频信号则通过并联在1MΩ可变电阻上的0.1μF电容直接送往超高阻耳机,或是音频阻抗变换器去驱动低阻耳机。 因为要尽量降低调谐电路的负载,因此我选用了阻抗高达140kΩ的舌簧耳机。 在这个实验中,耳机阻抗匹配变压器没有画进去,但实际上是存在的,可以从实物图上看见,通过一个分线开关选择阻抗,也可以通过直通开关将之抛开,让信号直通耳机。 天地线的架设 一个秋高气爽的周日,我和朋友爬上6楼楼顶,利用楼顶原有的两根相距15m、各高15m的CATV(电视共用天线)铁杆为基础,用普通的电话镀铜铁线架设了总长度约60m的天线。彻底淘汰了之前从3楼窗户扔出去的5m长的垂线,完成了远程接收最基础也是最困难的部分。 天线是架设在两根相距15m、本体具有良好接地并装有避雷针的CATV天线铁架上,距离楼顶高度最高点大约3m(平均高度约1.7m),天线铁架高度约15m或者更多,因此这个天线处于避雷针保护角之内,正常情况下毫无问题。 天线的材料是铁质镀铜的电话线(这是个无奈的选择,因为稍粗的铜线就有可能会被人盗割),采用双线并联方式。从天线中间下行到我工作台的信号线直流电阻为1.5Ω,还不算太垃圾,至少比用铜线便宜得多!地线比较简单,就是一条普通的塑皮铜线接在一条很粗但已经废弃的自来水铁管上。 天线示意图如图4-189所示。天线概貌(因为周围环境关系无法找到合适的位置拍摄全幅图片)如图4-190所示。 图4-189 天线示意图 图4-190 天线概貌 天线中部挂在CATV天线杆距根部约3m处,两端用绝缘子和楼房东西两边的地网相连,高度降低到0.35m。天线局部图如图4-191所示。 图4-191 天线局部图 元器件的选择和整机装配制作 (1)电感线圈:这台机器的几个谐振电感都工作在同一频段,考虑到对品质因数的要求、安装和加工的难度等方面,最后我选择了统一的规格。3个线圈管都用10mL的一次性注射器外管,截去注射针端约2mm一段,使之可以插入直径10mm的MX-400磁棒,磁棒长度为80mm。为获得足够的Q值,选用中波段性能最好的φ0.04mm×60利兹线(单股线径φ0.04mm的漆包线绞合成的利兹线的中波段性能最好)。具体参数如下。 先在线圈管上单层平绕73T作为L2,然后再上面加一个略大的线圈管,再绕4~6T作为L1(具体圈数可以根据需要调整);L3的绕法、圈数和L2完全相同,在L3外面用一个略大的的线圈管绕30圈作为L4;L5的绕法和L2、L3相同,圈数是80T。这5个线圈引线长度都预留20cm,用少量蜂蜡固定线头。蜂蜡对线圈的Q影响很小,但还是不要大面积涂抹在线圈上。 (2)可变电容器:为了获得尽可能高的Q值,主调谐可变电容器选用了拆自某仪器中的2×440pF空气介质可变电容器。定片绝缘子使用高频瓷,转动结构和动片弹性连接引线也做得十分可靠,为获得高品质因数的谐振回路提供了很好的条件。原本上面自带缓旋机构,但因为缺少零件只好弃之不用,直接旋转动片轴;陷波器的可变电容器选用了手头一个全新但储藏时间很久、最大容量为290pF的小型空气可变电容器,装机前测了一下绝缘电阻,我发现,这种用酚醛胶木板作为定片支架的可变电容器动片和定片之间的绝缘电阻,阻值竟然不到2MΩ,实际Q肯定不会很高。因此用90℃温度烘烤了4h,再趁热对两块胶木支架用硝基清漆做了封固处理,确保这两块东西不会过度吸潮。等漆干冷却后再测,绝缘电阻恢复到20MΩ以上,这样就算是恢复正常了。 (3)补偿电容器:为了对接收频率高端进行统调,分别在两组定片和动片之间接入了5/25pF的瓷质微调电容,补偿因为分布电容和布线引起的两连之间的不同步,从而可能导致的高端电台漏失而接收不到。 (4)耦合度调整机构:为了满足强、弱信号不同的耦合度要求,专门引入了可以手动调整的变耦合度机构,将输出调谐线圈装在一个连接到面板上操作的移动装置。用于调整输入、输出和陷波器线圈三者之间的耦合度。 (5)二极管检波器:检波器是将来自调谐电路的,包含有音频信号的载波,转变为可以用耳机转换成声音的关键装置,通过二极管检波器后,滤去残余的高频载波部分,就可以得到包含着一些直流成分的音频信号。为尽量提高反射到输出调谐电路上的负载阻抗,提高其有载Q值,这里选用了型号为HP5082-2835的二极管。这个二极管的Rd 值超过1MΩ,检波门限电压也相当低,是最好的检波二极管之一。考虑到常常测试不同的检波二极管,用了一个集成电路的快装测试插座,将两边并联起来,随便什么二极管都可以方便地装上和拆下。为滤除残余载频,我用一个39pF的云母电容来完成这个任务。 (6)班尼电路:从检波器输出的信号里,除了音频成分,还包含了一部分直流成分。这些直流成分对于耳机发出声音并无任何帮助,但会和音频信号一同通过直流电阻很低的耳机或者阻抗匹配变压器,增大调谐电路的负载,使谐振电路有载Q值降低,为此我增加了班尼电路。这里选用了一个1MΩ电位器和一个0.1μF电容器并联,本机对这两个元件的要求不是很高,可靠就好。因为面板上安装空间有限,将它们放在机器的一侧,装了一个拨盘,操作起来很方便。 (7)阻抗匹配变压器:为了和一些低阻耳机配合,并比较不同耳机的实际使用表现,我在机内增加了一个阻抗匹配变压器。检波器输出的音频信号通过一个分线开关接入匹配变压器的各个抽头馈入,以自耦变压器的形式将变换阻抗后的信号送往耳机插口。这个匹配变压器是北京李清先生制作的,他充分考虑了不同耳机的特性,使用高导磁率的EI铁氧体磁芯,先固定在木底座上,再用热熔胶固定在机座一侧。一个单刀十一掷的分线开关被用在阻抗转换选择上。 (8)耳机的选择:本制作对耳机的要求是尽可能高的灵敏度和音频阻抗,我手头虽然有Sound Power“大罐头”之类的高灵敏度耳机,但阻抗都不过几百欧姆,用大变比的变压器升高阻抗效率很低,最后选用了上海杨俊先生改绕过的一个国产SC-300舌簧耳机。这个耳机的直流电阻约20kΩ,音频阻抗大于140kΩ,虽然没有达到希望值(500kΩ以上),但比普通4kΩ、2kΩ的电磁式耳机和阻抗只有几百欧的舌簧耳机更合适,更不用提那些只有几十欧姆的动圈耳机或耳塞机了。 (9)机座、面板、刻度盘、旋钮及指针:机座是找了些废旧木板、拆开的包装盒拼凑起来的,由于重在试验,所以没有在外观上下大工夫,只求便于安装调试。具体外观和结构可以从图上看到,因为很简单,所以就不再一一描述了。 该机所用面板是一块表面涂有面漆的胶合板,它加工容易,直接钻孔,再用热熔胶固定在机座上即可。 刻度盘是先贴一片白纸在面板上,校准并标记接收波段最低端的频率,再校准标记最高端的频率,用信号发生器逐点标在白纸上,扫描后加工成想要的样子,再用打印机输出到照纸上。最后,将做好的度盘贴在面板上就算是完成了。 调谐钮用了一个老机上拆下来的胶木旋钮,在背面钻孔攻丝,固定上一根钢丝并剪成合适的长短,涂上颜色,就算完工了。另外两个分别是阻抗变换开关和陷波器的旋钮,也是找了两个拆机老旋钮来用。 (10)整机装配:本机结构很简单,装配过程很快。有一点在装配时特别注意:为降低信号损失,虽然是实验型的机器,在所有信号热端都小心地保证它们不会被旁路衰减损耗掉。接线柱也是选用和木质外壳绝缘的类型。机内连线全部使用φ0.04mm×60mm的利兹线,尽量降低信号损失。另外一些基本原则也要注意,不要冲突,例如调谐线圈附近不要有大块的金属、线圈和利兹线做的导线尽量不要有接头等。 制作完成的机器如图4-192所示。 图4-192 制作完成的收音机 调试 装配完成后,将手头的LSG-17信号发生器输出串联一个30pF的电容接在天、地线接线柱上(见图4-193),接下来就可以调整这台已经完工的矿石收音机了。 图4-193 测试电路 先调整接收频率范围的低端:移动输出谐振线圈置于输入线圈和陷波器中间,主可变电容器和陷波器可变电容器都旋至容量最大位置,调节信号发生器,输出520kHz信号,抽动输出、输入线圈内的磁棒使声音最大后固定;抽动陷波器线圈内的磁棒让声音最小。 再调整接收频率范围的高端:将主可变电容器和陷波器可变电容器全部旋置容量最小位置,信号发生器输出1700kHz信号,先后调节主输出谐振回路和输入谐振回路上并联的微调电容器,使耳机中声音最大。陷波器线圈上只有一个可变电容器,最高谐振频率高于输入、输出谐振回路,因此无需调整高端吸收频率。 至此,初步的统调就算是完成了,用少量蜂蜡将磁棒在线圈管上固定住,断开信号发生器接上天、地线,立刻收到超过10个电台。 变耦合机构的使用让输出线圈在输入线圈和陷波器之间移动从而改变三者之间的耦合度,从而很好地避开了双调谐电路双峰效应对接收的影响。 接收弱电台信号时,让可移动的输出线圈靠近输入线圈,二者之间的耦合度提高音量加大但因为信号强度弱,不致于产生双峰效应影响选择性;反之,当接收强电台信号时,输出线圈被移向陷波器,这时候转动陷波器的可变电容器,可以将输出线圈上的强电台信号有效地吸收掉,被强台压制的弱台信号得以被选出来送往检波器。 这台矿石收音机的表现,除了1170kHz因为1098kHz信号过分强大,而必须配合陷波器将之压制后才能听到外,收听时甚至有类似超外差收音机的感觉(每个电台都分隔清楚)。除了远程电台之外,信号强弱时音量大小相差并不悬殊。 用东湖B31三管再生机、泉城JP303三管超外差机以及德生S-2000机和它比对,东湖完败而泉城也毫无胜算。例如,585kHz的江苏故事台,泉城完全收不到,更不用说东湖了, S-2000虽然可以收到,但伴随着强烈的干扰噪声,听起来非常难受。当然矿石收音机使用的大型天线是根本因素,理论上不具备可比性。 刻度盘上标有频率的都是随时可以听到的;有刻度没有标频率的,是早晚都能听到的(因为空间有限,高端的1557kHz、1566kHz没有标上)。 我用这个矿石收音机收到过的频率如表4-4所示。 表4-4 可以清晰收到的电台 1395~1530kHz之间缺少一段,是因为没有收听到清晰的、可以分辨内容的播音,故没有列出(信号当然是有的)。还有一些是中国台湾和韩国的电台,但信号不是很稳定。 说明一下,这里记录的所有电台频率都是有清晰可懂的信号后才记录下来,凡无法分辨内容的信号一律没有记录。 因为耐心不够,没有逐个核对电台名。清晨和傍晚从一些电台内容中可以分辨出有湖北、陕西、浙江、天津等地的电台,但没有逐一记录。 制作本机时,只求选择性和灵敏度,没有追求音量,从而没有使用高音量的电路且使用150kΩ阻抗的耳机,故最大音量并不很大,本地强台最多是响亮而已。 调频矿石机 韩红 调频广播的优点是抗干扰能力强、失真小,但由于传输距离短,对于用没有放大能力的矿石机收听,要求接收环境要好,又因为频率较高,对元器件和制作工艺要求高一些。下面简单介绍一下调频矿石机的制作。 二极管检波调频矿石机 如图4-194所示,L2用2mm漆包线绕3圈,内径为25mm,电感量约为0.28μH,通过拉伸和压缩线圈,与C1配合完成87~108MHz覆盖。制作完成之后如图4-195所示。 图4-194 二极管检波调频矿石机电路图 图4-195 制作完成的L2 如图4-196所示,C1用蝶型可变电容器,大小选为7pF,C2的作用是减小可变电容的容量比,目地是增加可变电容旋转行程,方便选台。C2也可用固定电容。 检波二极管用1SS86或1SS106,经过试验,我发现用1SS86检波的输出阻抗在30kΩ左右。输出变压器用T38铁氧体磁芯制作(见图4-197),百匝电感量可达200mH。绕制数据取自朋友提供的材料(见表4-5)。 图4-196 C1用蝶型可变电容器 图4-197 输出变压器 表4-5 输出变压器绕制数据 如图4-198所示,机壳用3mm厚白色有机玻璃制成,外型尺寸55mm×60mm×135mm。 场效应管检波调频机 如图4-199所示,场效应管检波调频机电路图中的L2用直径2mm的镀银屏蔽线,在20mm胎上绕3匝,然后拉长至35mm,电感量为0.14μH,L1和L3用0.5mm2 线各绕1匝。线圈直接焊在可变焊片上(见图4-200)。 图4-198 外观 图4-199 场效应管检波调频机电路图 图4-200 线圈直接焊在可变焊片上 如图4-201所示,可变电容采用蝶形可变电容器,其电容引出为两定片,为减小人体感应用了特氟龙延长轴,同时为方便调谐使用了减速旋钮。 图4-201 可变电容采用蝶形可变电容器 如图4-202所示,检波用3DQ或3DP等双栅场效应管,与二极管比灵敏度和选择性提高了许多,音频失真也好许多。耳机变压器用铁氧体磁芯,直径为8.5mm,百匝电感量为100mH。计算方法取自李先生给出的计算公式。用直径0.08mm漆包线分组绕96匝(8Ω)、135匝、192匝、415匝、590匝、831匝、1077匝。S1为直通开关,如果使用舌簧耳机则变压器可省略。 拉杆天线长1.2mm,108MHz半波长为1.7m,为此使用了加感线圈,用φ0.62mm漆包线在15mm胎上平绕11匝(见图4-203)。 图4-202 所用元件 图4-203 制作拉杆天线 如图4-204所示,整机装在一块厚8mm聚丙稀板上,其尺寸为100mm×150mm。 图4-204 制作完成的整机 图4-204 制作完成的整机(续) 图4-204 制作完成的整机(续) “重型”矿石机 韩红 所谓“重型”矿石机,指的是这台矿石机用料奢侈,优中选优。其结果是机器的成本很高,分量很重。当然,收听效果也是非常出色的。 这是一台双回路双调谐矿石机,它将中波分为两个波段调谐,L波段为500~1000kHz,H波段为1000~1700kHz。将其作为无天地线接收时,S1开关置在S挡,天调线圈断路以免影响收听(否则天调是一个陷波器)。改变两支磁棒的距离,调整耦合量以便获得想要的灵敏度和选择性。使用两种检波电路,在用场效应管检波时将班尼电路关闭。 整机电路图如图4-205所示。 图4-205 整机电路图 两支磁棒用26只R40C1(尺寸为37mm×25mm×7mm)磁环组成,长度为190mm,每个磁环中间垫一片0.3mm厚聚四氟乙烯片(见图4-206)。 如图4-207所示,为了最大限度降低损耗,线圈骨架用2mm聚四氟乙烯板圈成。 如图4-208所示,线圈采用”双线圈” 反绕法,这使得它在频率高段比一只线圈有更高的Q值。当频率为1.7MHz时,单线圈Q值为1078,双线圈并联为1369。 当红线圈“头”与蓝线圈“尾” 相连时,两个线圈是串联关系,当红线圈与蓝线圈“头头”相连、“尾尾” 相连时,两个线圈形成并联关系。线圈L1、L2、L3都用φ0.04mm×660股利兹线绕制。 图4-206 磁棒的制作 图4-207 线圈骨架 图4-208 线圈采用”双线圈” 反绕法 天调线圈由L1-1和L1-2组成,各绕34匝,单个电感量101μH,两线圈串联时为266μH,两线圈并联时为67μH。相关数据如表4-6所示。 表4-6 L2线圈数据基本上与L1相同。 L3线圈为10匝,可调整与L2线圈距离。 图4-209 天调线圈 波段开关S1使用大型高频镀银开关(见图4-210)。 波段开关S2使用瓷板开关,装入有机玻璃盒中,防止镀银层硫化(见图4-211)。 图4-210 大型高频镀银开关 图4-211 瓷板开关 可变电容C1和C2使用222军机品(见图4-212),这种密封可变有很高的Q值,是已知的最好的可变电容之一,它的缺点是很重。 图4-212 电容C1和C2 耳机变压器用仿T725 (见图4-213),电位器用班尼电路电位器(见图4-214)。 整机底盘用300mm×450mm×12mm聚丙烯菜板(见图4-215)。 图4-213 仿T725耳机变压器 图4-214 班尼电路电位器 图4-215 整机底盘 制作完成的整机如图4-216所示,整机重量约为7.5kg。 图4-216 制作完成的整机 图4-216 制作完成的整机(续) 接收情况:由于天调回路和检波回路有比较高的素质,所以接收灵敏度和选择性令人满意。灵敏度高对于矿石机来讲就是音量大,即原来听不见的台可以听见了,听不清的台现在可以听清楚了。而选择性好表现在不串台和调台尖锐度上。做无天地线接收时,虽然该机磁棒只有180mm长,收听效果远超10mmx200mm磁棒线圈,比直径150mm花篮线圈略好,可收到本市603kHz、720kHz、747kHz、774kHz、828kHz、927kHz、1026kHz等7个台。 作者住楼房3层,用竹竿(塑料管)将天线挑出,离墙约2m,垂直向下约6m,地线接在暖气铁管上,能收到20多个台,有河北、河南、山东、辽宁等周边省市台,其中每天可稳定接收1566kHz广播,该台在韩国济州岛,距北京约1600km。 整机成本如表4-7所示。 表4-7 整机成本\begin{figure}[htb]  \centering  \includegraphics[width=0.8\\linewidth]{Image00103.jpg}  \caption{图片 294: Image00103.jpg}\end{figure}\begin{figure}[htb]  \centering  \includegraphics[width=0.8\\linewidth]{Image00321.jpg}  \caption{图片 295: Image00321.jpg}\end{figure}\begin{figure}[htb]  \centering  \includegraphics[width=0.8\\linewidth]{Image00003.jpg}  \caption{图片 296: Image00003.jpg}\end{figure}\begin{figure}[htb]  \centering  \includegraphics[width=0.8\\linewidth]{Image00020.jpg}  \caption{图片 297: Image00020.jpg}\end{figure}\begin{figure}[htb]  \centering  \includegraphics[width=0.8\\linewidth]{Image00279.jpg}  \caption{图片 298: Image00279.jpg}\end{figure}\begin{figure}[htb]  \centering  \includegraphics[width=0.8\\linewidth]{Image00262.jpg}  \caption{图片 299: Image00262.jpg}\end{figure}\begin{figure}[htb]  \centering  \includegraphics[width=0.8\\linewidth]{Image00360.jpg}  \caption{图片 300: Image00360.jpg}\end{figure}\begin{figure}[htb]  \centering  \includegraphics[width=0.8\\linewidth]{Image00319.jpg}  \caption{图片 301: Image00319.jpg}\end{figure}\begin{figure}[htb]  \centering  \includegraphics[width=0.8\\linewidth]{Image00091.jpg}  \caption{图片 302: Image00091.jpg}\end{figure}\begin{figure}[htb]  \centering  \includegraphics[width=0.8\\linewidth]{Image00331.jpg}  \caption{图片 303: Image00331.jpg}\end{figure}\begin{figure}[htb]  \centering  \includegraphics[width=0.8\\linewidth]{Image00039.jpg}  \caption{图片 304: Image00039.jpg}\end{figure}\begin{figure}[htb]  \centering  \includegraphics[width=0.8\\linewidth]{Image00287.jpg}  \caption{图片 305: Image00287.jpg}\end{figure}\begin{figure}[htb]  \centering  \includegraphics[width=0.8\\linewidth]{Image00047.jpg}  \caption{图片 306: Image00047.jpg}\end{figure}\begin{figure}[htb]  \centering  \includegraphics[width=0.8\\linewidth]{Image00064.jpg}  \caption{图片 307: Image00064.jpg}\end{figure}\begin{figure}[htb]  \centering  \includegraphics[width=0.8\\linewidth]{Image00324.jpg}  \caption{图片 308: Image00324.jpg}\end{figure}\begin{figure}[htb]  \centering  \includegraphics[width=0.8\\linewidth]{Image00139.jpg}  \caption{图片 309: Image00139.jpg}\end{figure}\begin{figure}[htb]  \centering  \includegraphics[width=0.8\\linewidth]{Image00037.jpg}  \caption{图片 310: Image00037.jpg}\end{figure}\begin{figure}[htb]  \centering  \includegraphics[width=0.8\\linewidth]{Image00026.jpg}  \caption{图片 311: Image00026.jpg}\end{figure}\begin{figure}[htb]  \centering  \includegraphics[width=0.8\\linewidth]{Image00316.jpg}  \caption{图片 312: Image00316.jpg}\end{figure}\begin{figure}[htb]  \centering  \includegraphics[width=0.8\\linewidth]{Image00057.jpg}  \caption{图片 313: Image00057.jpg}\end{figure}\begin{figure}[htb]  \centering  \includegraphics[width=0.8\\linewidth]{Image00077.jpg}  \caption{图片 314: Image00077.jpg}\end{figure}\begin{figure}[htb]  \centering  \includegraphics[width=0.8\\linewidth]{Image00173.jpg}  \caption{图片 315: Image00173.jpg}\end{figure}\begin{figure}[htb]  \centering  \includegraphics[width=0.8\\linewidth]{Image00272.jpg}  \caption{图片 316: Image00272.jpg}\end{figure}\begin{figure}[htb]  \centering  \includegraphics[width=0.8\\linewidth]{Image00240.jpg}  \caption{图片 317: Image00240.jpg}\end{figure}\begin{figure}[htb]  \centering  \includegraphics[width=0.8\\linewidth]{Image00357.jpg}  \caption{图片 318: Image00357.jpg}\end{figure}\begin{figure}[htb]  \centering  \includegraphics[width=0.8\\linewidth]{Image00303.jpg}  \caption{图片 319: Image00303.jpg}\end{figure}\begin{figure}[htb]  \centering  \includegraphics[width=0.8\\linewidth]{Image00343.jpg}  \caption{图片 320: Image00343.jpg}\end{figure}\begin{figure}[htb]  \centering  \includegraphics[width=0.8\\linewidth]{Image00046.jpg}  \caption{图片 321: Image00046.jpg}\end{figure}\begin{figure}[htb]  \centering  \includegraphics[width=0.8\\linewidth]{Image00048.jpg}  \caption{图片 322: Image00048.jpg}\end{figure}\begin{figure}[htb]  \centering  \includegraphics[width=0.8\\linewidth]{Image00174.jpg}  \caption{图片 323: Image00174.jpg}\end{figure}\begin{figure}[htb]  \centering  \includegraphics[width=0.8\\linewidth]{Image00180.jpg}  \caption{图片 324: Image00180.jpg}\end{figure}\begin{figure}[htb]  \centering  \includegraphics[width=0.8\\linewidth]{Image00236.jpg}  \caption{图片 325: Image00236.jpg}\end{figure}\begin{figure}[htb]  \centering  \includegraphics[width=0.8\\linewidth]{Image00102.jpg}  \caption{图片 326: Image00102.jpg}\end{figure}\begin{figure}[htb]  \centering  \includegraphics[width=0.8\\linewidth]{Image00034.jpg}  \caption{图片 327: Image00034.jpg}\end{figure}\begin{figure}[htb]  \centering  \includegraphics[width=0.8\\linewidth]{Image00001.jpg}  \caption{图片 328: Image00001.jpg}\end{figure}\begin{figure}[htb]  \centering  \includegraphics[width=0.8\\linewidth]{Image00379.jpg}  \caption{图片 329: Image00379.jpg}\end{figure}\begin{figure}[htb]  \centering  \includegraphics[width=0.8\\linewidth]{Image00349.jpg}  \caption{图片 330: Image00349.jpg}\end{figure}\begin{figure}[htb]  \centering  \includegraphics[width=0.8\\linewidth]{Image00366.jpg}  \caption{图片 331: Image00366.jpg}\end{figure}\begin{figure}[htb]  \centering  \includegraphics[width=0.8\\linewidth]{Image00125.jpg}  \caption{图片 332: Image00125.jpg}\end{figure}\begin{figure}[htb]  \centering  \includegraphics[width=0.8\\linewidth]{Image00109.jpg}  \caption{图片 333: Image00109.jpg}\end{figure}\begin{figure}[htb]  \centering  \includegraphics[width=0.8\\linewidth]{Image00237.jpg}  \caption{图片 334: Image00237.jpg}\end{figure}\begin{figure}[htb]  \centering  \includegraphics[width=0.8\\linewidth]{Image00396.jpg}  \caption{图片 335: Image00396.jpg}\end{figure}\begin{figure}[htb]  \centering  \includegraphics[width=0.8\\linewidth]{Image00254.jpg}  \caption{图片 336: Image00254.jpg}\end{figure}\begin{figure}[htb]  \centering  \includegraphics[width=0.8\\linewidth]{Image00234.jpg}  \caption{图片 337: Image00234.jpg}\end{figure}\begin{figure}[htb]  \centering  \includegraphics[width=0.8\\linewidth]{Image00365.jpg}  \caption{图片 338: Image00365.jpg}\end{figure}\begin{figure}[htb]  \centering  \includegraphics[width=0.8\\linewidth]{Image00337.jpg}  \caption{图片 339: Image00337.jpg}\end{figure}\begin{figure}[htb]  \centering  \includegraphics[width=0.8\\linewidth]{Image00313.jpg}  \caption{图片 340: Image00313.jpg}\end{figure}\begin{figure}[htb]  \centering  \includegraphics[width=0.8\\linewidth]{Image00281.jpg}  \caption{图片 341: Image00281.jpg}\end{figure}\begin{figure}[htb]  \centering  \includegraphics[width=0.8\\linewidth]{Image00246.jpg}  \caption{图片 342: Image00246.jpg}\end{figure}\begin{figure}[htb]  \centering  \includegraphics[width=0.8\\linewidth]{Image00067.jpg}  \caption{图片 343: Image00067.jpg}\end{figure}\begin{figure}[htb]  \centering  \includegraphics[width=0.8\\linewidth]{Image00285.jpg}  \caption{图片 344: Image00285.jpg}\end{figure}\begin{figure}[htb]  \centering  \includegraphics[width=0.8\\linewidth]{Image00249.jpg}  \caption{图片 345: Image00249.jpg}\end{figure}\begin{figure}[htb]  \centering  \includegraphics[width=0.8\\linewidth]{Image00333.jpg}  \caption{图片 346: Image00333.jpg}\end{figure}\begin{figure}[htb]  \centering  \includegraphics[width=0.8\\linewidth]{Image00330.jpg}  \caption{图片 347: Image00330.jpg}\end{figure}\begin{figure}[htb]  \centering  \includegraphics[width=0.8\\linewidth]{Image00356.jpg}  \caption{图片 348: Image00356.jpg}\end{figure}\begin{figure}[htb]  \centering  \includegraphics[width=0.8\\linewidth]{Image00029.jpg}  \caption{图片 349: Image00029.jpg}\end{figure}\begin{figure}[htb]  \centering  \includegraphics[width=0.8\\linewidth]{Image00085.jpg}  \caption{图片 350: Image00085.jpg}\end{figure}\begin{figure}[htb]  \centering  \includegraphics[width=0.8\\linewidth]{Image00265.jpg}  \caption{图片 351: Image00265.jpg}\end{figure}\begin{figure}[htb]  \centering  \includegraphics[width=0.8\\linewidth]{Image00082.jpg}  \caption{图片 352: Image00082.jpg}\end{figure}\begin{figure}[htb]  \centering  \includegraphics[width=0.8\\linewidth]{Image00021.jpg}  \caption{图片 353: Image00021.jpg}\end{figure}\begin{figure}[htb]  \centering  \includegraphics[width=0.8\\linewidth]{Image00095.jpg}  \caption{图片 354: Image00095.jpg}\end{figure}\begin{figure}[htb]  \centering  \includegraphics[width=0.8\\linewidth]{Image00108.jpg}  \caption{图片 355: Image00108.jpg}\end{figure}\begin{figure}[htb]  \centering  \includegraphics[width=0.8\\linewidth]{Image00111.jpg}  \caption{图片 356: Image00111.jpg}\end{figure}\begin{figure}[htb]  \centering  \includegraphics[width=0.8\\linewidth]{Image00325.jpg}  \caption{图片 357: Image00325.jpg}\end{figure}\begin{figure}[htb]  \centering  \includegraphics[width=0.8\\linewidth]{Image00336.jpg}  \caption{图片 358: Image00336.jpg}\end{figure}\begin{figure}[htb]  \centering  \includegraphics[width=0.8\\linewidth]{Image00251.jpg}  \caption{图片 359: Image00251.jpg}\end{figure}\begin{figure}[htb]  \centering  \includegraphics[width=0.8\\linewidth]{Image00018.jpg}  \caption{图片 360: Image00018.jpg}\end{figure}\begin{figure}[htb]  \centering  \includegraphics[width=0.8\\linewidth]{Image00323.jpg}  \caption{图片 361: Image00323.jpg}\end{figure}\begin{figure}[htb]  \centering  \includegraphics[width=0.8\\linewidth]{Image00016.jpg}  \caption{图片 362: Image00016.jpg}\end{figure}\begin{figure}[htb]  \centering  \includegraphics[width=0.8\\linewidth]{Image00314.jpg}  \caption{图片 363: Image00314.jpg}\end{figure}\begin{figure}[htb]  \centering  \includegraphics[width=0.8\\linewidth]{Image00071.jpg}  \caption{图片 364: Image00071.jpg}\end{figure}\begin{figure}[htb]  \centering  \includegraphics[width=0.8\\linewidth]{Image00295.jpg}  \caption{图片 365: Image00295.jpg}\end{figure}\begin{figure}[htb]  \centering  \includegraphics[width=0.8\\linewidth]{Image00079.jpg}  \caption{图片 366: Image00079.jpg}\end{figure}\begin{figure}[htb]  \centering  \includegraphics[width=0.8\\linewidth]{Image00302.jpg}  \caption{图片 367: Image00302.jpg}\end{figure}\begin{figure}[htb]  \centering  \includegraphics[width=0.8\\linewidth]{Image00369.jpg}  \caption{图片 368: Image00369.jpg}\end{figure}\begin{figure}[htb]  \centering  \includegraphics[width=0.8\\linewidth]{Image00208.jpg}  \caption{图片 369: Image00208.jpg}\end{figure}\begin{figure}[htb]  \centering  \includegraphics[width=0.8\\linewidth]{Image00135.jpg}  \caption{图片 370: Image00135.jpg}\end{figure}\begin{figure}[htb]  \centering  \includegraphics[width=0.8\\linewidth]{Image00392.jpg}  \caption{图片 371: Image00392.jpg}\end{figure}\begin{figure}[htb]  \centering  \includegraphics[width=0.8\\linewidth]{Image00010.jpg}  \caption{图片 372: Image00010.jpg}\end{figure}\begin{figure}[htb]  \centering  \includegraphics[width=0.8\\linewidth]{Image00355.jpg}  \caption{图片 373: Image00355.jpg}\end{figure}\begin{figure}[htb]  \centering  \includegraphics[width=0.8\\linewidth]{Image00144.jpg}  \caption{图片 374: Image00144.jpg}\end{figure}\begin{figure}[htb]  \centering  \includegraphics[width=0.8\\linewidth]{Image00099.jpg}  \caption{图片 375: Image00099.jpg}\end{figure}\begin{figure}[htb]  \centering  \includegraphics[width=0.8\\linewidth]{Image00019.jpg}  \caption{图片 376: Image00019.jpg}\end{figure}\begin{figure}[htb]  \centering  \includegraphics[width=0.8\\linewidth]{Image00223.jpg}  \caption{图片 377: Image00223.jpg}\end{figure}\begin{figure}[htb]  \centering  \includegraphics[width=0.8\\linewidth]{Image00229.jpg}  \caption{图片 378: Image00229.jpg}\end{figure}\begin{figure}[htb]  \centering  \includegraphics[width=0.8\\linewidth]{Image00084.jpg}  \caption{图片 379: Image00084.jpg}\end{figure}\begin{figure}[htb]  \centering  \includegraphics[width=0.8\\linewidth]{Image00011.jpg}  \caption{图片 380: Image00011.jpg}\end{figure}\begin{figure}[htb]  \centering  \includegraphics[width=0.8\\linewidth]{Image00150.jpg}  \caption{图片 381: Image00150.jpg}\end{figure}\begin{figure}[htb]  \centering  \includegraphics[width=0.8\\linewidth]{Image00322.jpg}  \caption{图片 382: Image00322.jpg}\end{figure}\begin{figure}[htb]  \centering  \includegraphics[width=0.8\\linewidth]{Image00204.jpg}  \caption{图片 383: Image00204.jpg}\end{figure}\begin{figure}[htb]  \centering  \includegraphics[width=0.8\\linewidth]{Image00163.jpg}  \caption{图片 384: Image00163.jpg}\end{figure}\begin{figure}[htb]  \centering  \includegraphics[width=0.8\\linewidth]{Image00075.jpg}  \caption{图片 385: Image00075.jpg}\end{figure}\begin{figure}[htb]  \centering  \includegraphics[width=0.8\\linewidth]{Image00119.jpg}  \caption{图片 386: Image00119.jpg}\end{figure}\begin{figure}[htb]  \centering  \includegraphics[width=0.8\\linewidth]{Image00185.jpg}  \caption{图片 387: Image00185.jpg}\end{figure}\begin{figure}[htb]  \centering  \includegraphics[width=0.8\\linewidth]{Image00307.jpg}  \caption{图片 388: Image00307.jpg}\end{figure}\begin{figure}[htb]  \centering  \includegraphics[width=0.8\\linewidth]{Image00381.jpg}  \caption{图片 389: Image00381.jpg}\end{figure}\begin{figure}[htb]  \centering  \includegraphics[width=0.8\\linewidth]{Image00215.jpg}  \caption{图片 390: Image00215.jpg}\end{figure}\begin{figure}[htb]  \centering  \includegraphics[width=0.8\\linewidth]{Image00027.jpg}  \caption{图片 391: Image00027.jpg}\end{figure}\begin{figure}[htb]  \centering  \includegraphics[width=0.8\\linewidth]{Image00213.jpg}  \caption{图片 392: Image00213.jpg}\end{figure}\begin{figure}[htb]  \centering  \includegraphics[width=0.8\\linewidth]{Image00203.jpg}  \caption{图片 393: Image00203.jpg}\end{figure}\begin{figure}[htb]  \centering  \includegraphics[width=0.8\\linewidth]{Image00169.jpg}  \caption{图片 394: Image00169.jpg}\end{figure}\begin{figure}[htb]  \centering  \includegraphics[width=0.8\\linewidth]{Image00182.jpg}  \caption{图片 395: Image00182.jpg}\end{figure}\begin{figure}[htb]  \centering  \includegraphics[width=0.8\\linewidth]{Image00201.jpg}  \caption{图片 396: Image00201.jpg}\end{figure}\begin{figure}[htb]  \centering  \includegraphics[width=0.8\\linewidth]{Image00154.jpg}  \caption{图片 397: Image00154.jpg}\end{figure}\begin{figure}[htb]  \centering  \includegraphics[width=0.8\\linewidth]{Image00148.jpg}  \caption{图片 398: Image00148.jpg}\end{figure}\section{文件 17}\
%Table of Contents Table of Contents 目录 扉页 版权 编委会 前言 第一章 国外矿石机的历史回顾 一、早期的线圈调感式矿石机 抽头式线圈调感矿石机 1.法国壁挂式矿石机 2.加拿大“普通人”矿石机 3.美国“宝宝”袖珍矿石机 4.德国WISI.NR.56袖珍矿石机 5.德国具有减震功能的袖珍矿石机 6.法国迷宫式线圈袖珍矿石机 线圈滑动抽头调感式矿石机 1.法国早期滑动抽头线圈矿石机3种 2.法国早期格勒诺布尔矿石机 3.法国便携式邮政矿石机 4.美国JR矿石机 5.德律风根ZA矿石机 6.德律风根RDE矿石机 7.美国BIG 4矿石机 8.美国火星矿石机 9.美国“小宝石”矿石机 10.美国“瓢虫”陶瓷矿石机 11.英国“汤姆叔叔”陶瓷矿石机 12.法国的“美洲豹”“热带鱼”和“羚羊”陶瓷矿石机 13.美国“啤酒瓶”矿石机 14.英国书本式矿石机 15.美国Metro.jr矿石机 16.美国PACENT袖珍矿石机 17.德国德律风根板式矿石机 线圈耦合调感式矿石机 1.早期线圈调感器4种 2.德国德律风根公司E04矿石机 3.美国Westinghouse矿石机 4.美国ER-753A便携式矿石机 5.美国Atwater Kent板式矿石机 6.英国BTH-C双矿石检波器矿石机 7.瑞典爱立信L.M矿石机 8.德国罗兰士Lorenz EDAT25型双矿石检波矿石机 9.德国“狐狸”矿石机 10.英国The Electron 双线圈调感矿石机 11.美国纽约丹齐格琼斯公司出品的线圈耦合调感式三检波器矿石机 12.德国Dyhr耳机矿石机 二、早期的线圈与可变电容器耦合调谐矿石机 1.美国RCA AR1300矿石机 2.美国Betts\\&amp;Betts板式矿石机 3.美国吉卜林豪华版矿石机 4.美国直立板式矿石机 5.法国三线圈矿石机 6.法国双回路调谐矿石机 7.德国1001矿石机 8.美国劳伦斯矿石机 9.英国马可尼矿石机 10.奥地利GEWES矿石机 三、另类矿石机(附两款火花无线电发射机) 1.法国1913年出品的E·POFERL矿石机 2.美国1917年出品的无线电火花发射机 3.美国BC-14A型军用矿石接收机 4.美国BC-15A军用火花发射机 5.英国MARK Ⅲ型军用矿石接收机 6.英国B.T.H.一灯矿石收音机(Valve-Crystal Receiver) 7.美国DeForest D-10矿石检波四管直流接收机 四、20世纪40年代后的传统矿石收音机 1.德国1946年出品的德律风根袖珍矿石收音机 2.德国1946年出品的蓝宝矿石收音机 3.20世纪40年代美国出品的铁壳袖珍自带扬声器矿石机 4.德国20世纪50年代初出品的欧米茄袖珍矿石收音机 5.德国20世纪50年代出品的德律风根D3型矿石收音机 6.20世纪50年代后期Miniman袖珍矿石机 7.1958年美国Hearever公司出品的Hearever火箭模型矿石机 8.20世纪50年代后期日本出品 的EM-TONE型矿石机 9.20世纪50年代后期日本出品的LARK CRYSTAL 矿石机 10.20世纪50年代初美国出品的 Philmore VC-1000矿石机套件 11.1958年美国希斯有限公司出品的Crystal Receiver CR-1型矿石机套件 五、前苏联“共青团员”牌矿石机 第二章 国产矿石机的历史回顾 一、民国时期的矿石收音机 1.一台我国20世纪30年代的矿石收音机 张明律收藏并撰文 2.一台天津无线电行组装的矿石机 3.一台抗战时期八路军缴获的“华北标准型矿石受信机” 4.一台标明了型号、厂名、出厂日期和制作者的“日本”矿石机 二、新中国成立后的矿石收音机 1.20世纪50年代初天津天华工业社出品的矿石收音机 2.“牡丹”矿石机 3.与“牡丹”矿石机近似的“上海造” 4.提手上铸有完税标记的矿石机 5.带提把的蓝色老矿石收音机 6.一台成品矿石机的机芯 7.一台红色木制机箱矿石机 8.济南“胜利”矿石收音机 毕冠超收藏并撰文 9.“万明”矿石机 10.“中山公园”矿石机 11.滑动抽头线圈调感式袖珍矿石机 12.象牌101型袖珍矿石收音机 13.教学示范用矿石机 第三章 国外矿石机的研究和进展 第四章 新时期国内矿石机的研究和进展 一、器材篇 高阻耳机和匹配变压器 高灵敏度舌簧耳机介绍及超高阻耳机改造实例 矿石机匹配变压器作用以及简单的设计方法 寻找3DQ双栅场效应管花絮 怎样制作高Q值线圈 二、测量篇 3DQ、3DP的测试 用万用表判断MOS管的方法 场效应管小信号检波计算 平方检波原理与Rd 矿石机的实测 1.万用表测量二极管的方法和测试实例 2.二极管Rd的测试实例 3.矿石机调谐回路输出电压的测试实例 4.线圈Q值的测试实例 5.可变电容Q值的测试实例 6.耳机阻抗的测试实例 7.匹配变压器阻抗及效率的测试实例 8.矿石机灵敏度测试实例 9.用指针式万用表简单判断MOS管是否可用于矿石机检波 三、整机篇 复古矿石机 电罗经的复古矿石机 蛛网线圈双回路矿石机 三回路蛛网线圈矿石收音机 现代矿石机 二极管、MOSFET、再生三用矿石机 场效应管“远程”矿石收音机 高灵敏度MOS管矿石机 中波3DQ矿石机/太阳能再生机 直径0.8m的场效应管检波中波大环矿石收音机 无天地线箱式便携中波场管矿石机 实验无天地线磁棒矿石机 便携式矿石收音机 中短波磁环小型矿石收音机 短波大环矿石收音机 专为远程接收设计的变耦合度矿石收音机 调频矿石机 “重型”矿石机

\end{document}
