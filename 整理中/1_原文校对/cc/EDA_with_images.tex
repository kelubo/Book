\documentclass{book}
\usepackage[UTF8]{ctex}
\usepackage{graphicx}
\usepackage{hyperref}
\usepackage{titlesec}
\usepackage{tocloft}
\usepackage{geometry}
\geometry{a4paper, margin=2cm}

\title{电路设计与制作实用教程——基于立创EDA}
\author{唐浒}

\begin{document}
\maketitle
\tableofcontents
\newpage

内 容 简 介

本书以深圳市立创电子商务有限公司的立创EDA设计工具为平台,以本书配套的STM32核心板为实践载体,介绍电路设计与制作的全过程。主要内容包括基于STM32核心板的电路设计与制作流程、STM32核心板介绍、STM32核心板程序下载与验证、STM32核心板焊接、立创EDA介绍、STM32核心板的原理图设计及PCB设计、创建元件库、导出生产文件以及制作电路板等。本书所有知识点均围绕着STM32核心板,希望读者通过对本书的学习,能够快速设计并制作出一块属于自己的电路板,同时掌握电路设计与制作过程中涉及的所有基本技能。

本书既可以作为高等院校相关专业的电路设计与制作实践课程教材,也可作为电路设计及相关行业工程技术人员的入门培训用书。

未经许可,不得以任何方式复制或抄袭本书之部分或全部内容。

版权所有,侵权必究。

图书在版编目(CIP)数据

电路设计与制作实用教程:基于立创EDA/唐浒,韦然主编.—北京:电子工业出版社,2019.11

ISBN 978-7-121-37597-2

Ⅰ.①电… Ⅱ.①唐… ②韦… Ⅲ.①电子电路-电路设计-计算机辅助设计-高等学校-教材 Ⅳ.①TN702.2

中国版本图书馆CIP数据核字(2019)第219794号

责任编辑:张小乐

印刷:

装订:

出版发行:电子工业出版社

北京市海淀区万寿路173信箱 邮编100036

开本:787×1092 1/16 印张:11.25 字数:288千字

版次:2019年11月第1版

印次:2019年11月第1次印刷

定价:39.00元

凡所购买电子工业出版社图书有缺损问题,请向购买书店调换。若书店售缺,请与本社发行部联系,联系及邮购电话:(010)88254888,88258888。

质量投诉请发邮件至zlts@phei.com.cn,盗版侵权举报请发邮件至dbqq@phei.com.cn。

本书咨询服务方式:(010)88254462,zhxl@phei.com.cn

\chapter{前言}

电路设计与制作是一项非常系统且复杂的工作,涉及原理图设计、PCB设计、元件库制作、PCB打样、元件采购、电路板焊接、电路板调试等技能。单个技能比较容易讲清楚,初学者也容易掌握。但是,“麻雀虽小,五脏俱全”,即使一个简单的电路板,要想完成设计与制作,都必须掌握所有这些技能,并且能将这些技能合理有效地贯穿始终。\n\n对于初学者而言,为了设计和制作一块电路板,常用的方式就是查阅电路设计与制作的相关书籍。然而,目前许多关于电路设计与制作的书籍都按照模块的方式来讲解,且每个模块之间缺乏一定的连贯性。例如,原理图绘制部分讲解的是三极管电路,PCB设计部分讲解的却是七段数码管电路,而生产文件导出部分讲解的又是单片机电路。这些书籍之所以这样安排,或许是希望覆盖更广泛的知识和技能,然而这样却使得内容只聚焦局部而忽略全局。此外,鲜有书籍会涉及电路板焊接、元件采购和PCB制作等具有较强实践性的环节。\n\n因此,初学者在一边查阅相关书籍一边进行实际电路设计与制作的过程中,常常会出现“按下葫芦起了瓢”的现象。例如,会绘制原理图,却不知道如何将设计好的原理图导入PCB文件中;好不容易设计好了PCB,却不知道如何生成丝印文件和坐标文件;生产文件有了,却又不知道发到哪家打样厂进行PCB打样;电路板拿到手了又对元件采购不熟悉……而且由于书中较少涉及电烙铁操作、元件焊接、电路板调试、万用表使用等方面的技能,初学者拿到电路板之后,也不知道如何下手。\n\n据统计,全国每年约有20%的本科生和专科生会继续读研,约有10%的硕士研究生会继续读博,也就是说,绝大多数学生最终都会选择就业。为了提高高等院校的就业率和就业质量,按照企业的标准培养人才不失为一条有效途径。企业除重视实践外,还非常重视规范,但是在往常的学习过程中,诸如库规范、原理图设计规范、PCB设计规范、生产文件规范等通常都被忽略了。\n\n为了解决上述问题,本书将通过对STM32核心板程序下载与验证、元件采购、STM32核心板焊接、STM32核心板原理图设计及PCB设计、创建元件库、导出生产文件以及制作电路板等知识的讲解,让初学者在短时间内对电路设计与制作的整个过程有一个立体的认识,最终能够独立地进行简单电路的设计与制作。同时,在实训过程中,本书还对各种规范进行重点讲解。本书在编写过程中,遵循小而精的理念,只重点讲解STM32核心板电路设计与制作过程中使用到的技能和知识点,未涉及的内容尽量省略。\n\n本书主要具有以下特点:\n\n(1)本书采用立创EDA软件,它具有简单易用、资源共享,以及设计、采购和制造一体化的特点。\n\n(2)以一块微控制器的核心板作为实践载体,微控制器选用STM32F103RCT6芯片,主要是考虑到STM32系列单片机是目前市面上使用最为广泛的微控制器之一,且该系列的单片机具有功耗低、外设多、基于库开发、配套资料多、开发板种类多等优势。因此,读者最终完成STM32核心板的设计与制作之后,还可以无缝地将其应用于后续的单片机软件设计中。\n\n(3)用一个STM32核心板贯穿整个电路板设计与制作的过程,将所有关键技能有效、合理地串接在一起。这些技能包括元件采购、STM32核心板焊接、STM32核心板原理图设计及PCB设计、创建元件库、导出生产文件、制作电路板等。\n\n(4)细致讲解STM32核心板电路设计与制作过程中使用到的技能,未涉及的技能几乎不予讲解。这样,初学者就可以快速掌握电路设计与制作的基本技能,并设计出一块属于自己的STM32核心板。\n\n(5)对具有较强实践性的环节,如电路板焊接、元件采购、PCB打样、PCB贴片、工具使用、电路板调试等电路板制作环节进行详细讲解。\n\n(6)将各种规范贯穿于整个电路板设计与制作的过程中,如工程和文件命名规范、版本规范、元件库的设计规范、BOM格式规范、生产文件规范等。\n\n(7)配有完整的资料包,包括元件数据手册、PDF版本原理图、PPT讲义、软件、嵌入式工程、视频教程等。下载地址可关注并查看微信公众号“卓越工程师培养系列”。\n\n鱼与熊掌不可兼得,诸如多层板电路设计、自动布局、差分对布线、电路仿真等内容均未出现在本书中,如果需要学习这些技能,建议读者查阅其他书籍或者在网上搜索相关资料。\n\n本书的编写得到了深圳市立创电子商务有限公司杨林杰、贺定球、张银莹、吴波、杨希文、罗德松的大力支持;深圳大学的覃进宇、刘宇林、郭文波、陈旭萍和黄荣祯等做了大量的校对工作;本书的出版得到了电子工业出版社的鼎力支持,张小乐编辑为本书的顺利出版做了大量的工作。一并向他们表示衷心的感谢。\n\n由于作者水平有限,书中难免有错误和不足之处,敬请读者不吝赐教。\n\n作者\n\n2019年7月\n\n前言\n\n第1章 基于STM32核心板的电路设计与制作流程\n\n1.1 什么是STM32核心板\n\n1.2 为什么选择STM32核心板\n\n1.3 电路设计与制作流程\n\n1.4 本书提供的资料包\n\n1.5 本书配套开发套件\n\n本章任务\n\n本章习题\n\n第2章 STM32核心板介绍\n\n2.1 STM32芯片介绍\n\n2.2 STM32核心板电路简介\n\n2.2.1 通信-下载模块接口电路\n\n2.2.2 电源转换电路\n\n2.2.3 JTAG/SWD调试接口电路\n\n2.2.4 独立按键电路\n\n2.2.5 OLED显示屏接口电路\n\n2.2.6 晶振电路\n\n2.2.7 LED电路\n\n2.2.8 STM32微控制器电路\n\n2.2.9 外扩引脚\n\n2.3 基于STM32核心板可以开展的实验\n\n本章任务\n\n本章习题\n\n第3章 STM32核心板程序下载与验证\n\n3.1 准备工作\n\n3.2 将通信-下载模块连接到STM32核心板\n\n3.3 安装CH340驱动\n\n3.4 通过mcuisp下载程序\n\n3.5 通过串口助手查看接收数据\n\n3.6 查看STM32核心板工作状态\n\n3.7 通过ST-Link下载程序\n\n本章任务\n\n本章习题\n\n第4章 STM32核心板焊接\n\n4.1 焊接工具和材料\n\n4.2 STM32核心板元件清单\n\n4.3 STM32核心板焊接步骤\n\n4.4 STM32核心板分步焊接\n\n4.5 元件焊接方法详解\n\n4.5.1 STM32F103RCT6芯片焊接方法\n\n4.5.2 贴片电阻(电容)焊接方法\n\n4.5.3 发光二极管(LED)焊接方法\n\n4.5.4 肖特基二极管(SS210)焊接方法\n\n4.5.5 低压差线性稳压芯片(AMS1117)焊接方法\n\n4.5.6 晶振焊接方法\n\n4.5.7 贴片轻触开关焊接方法\n\n4.5.8 直插元件焊接方法\n\n本章任务\n\n本章习题\n\n第5章 立创EDA介绍\n\n5.1 立创EDA\n\n5.2 功能特点\n\n5.2.1 库文件共享\n\n5.2.2 团队管理\n\n5.2.3 工程广场\n\n5.2.4 版本管理\n\n本章任务\n\n本章习题\n\n第6章 STM32核心板原理图设计\n\n6.1 原理图设计流程\n\n6.2 创建PCB工程\n\n6.3 新建原理图文件\n\n6.4 原理图规范化设置\n\n6.4.1 设置网格大小和栅格尺寸\n\n6.4.2 设置画布规格\n\n6.4.3 设置Title Block\n\n6.5 快捷键介绍\n\n6.6 放置元件\n\n6.7 连线\n\n6.8 原理图检查\n\n6.9 常见问题及解决方法\n\n本章任务\n\n本章习题\n\n第7章 STM32核心板PCB设计\n\n7.1 PCB设计流程\n\n7.2 新建PCB文件\n\n7.3 定义PCB边框大小\n\n7.4 更新PCB\n\n7.5 设计规则\n\n7.6 层的设置\n\n7.6.1 层工具\n\n7.6.2 层管理器\n\n7.7 绘制定位孔\n\n7.8 元件的布局\n\n7.8.1 布局原则\n\n7.8.2 布局基本操作\n\n7.9 元件的布线\n\n7.9.1 布线基本操作\n\n7.9.2 布线注意事项\n\n7.9.3 STM32核心板分步布线\n\n7.10 丝印\n\n7.10.1 添加丝印\n\n7.10.2 丝印的方向\n\n7.10.3 批量添加底层丝印\n\n7.10.4 STM32核心板丝印效果图\n\n7.11 添加电路板信息和信息框\n\n7.11.1 添加电路板名称丝印\n\n7.11.2 添加版本信息和信息框\n\n7.11.3 添加PCB信息\n\n7.12 泪滴\n\n7.12.1 添加泪滴\n\n7.12.2 删除泪滴\n\n7.13 覆铜\n\n7.14 DRC规则检查\n\n本章任务\n\n本章习题\n\n第8章 创建元件库\n\n8.1 创建原理图库\n\n8.1.1 创建原理图库的流程\n\n8.1.2 新建原理图库\n\n8.1.3 制作电阻原理图符号\n\n8.1.4 制作蓝色发光二极管原理图符号\n\n8.1.5 制作简牛原理图符号\n\n8.1.6 制作STM32F103RCT6芯片原理图符号\n\n8.2 创建PCB库\n\n8.2.1 创建PCB封装的流程\n\n8.2.2 新建PCB库\n\n8.2.3 制作电阻的PCB封装\n\n8.2.4 制作发光二极管的PCB封装\n\n8.2.5 制作简牛的PCB封装\n\n8.2.6 制作STM32F103RCT6芯片的PCB封装\n\n本章任务\n\n本章习题\n\n第9章 导出生产文件\n\n9.1 生产文件的组成\n\n9.2 Gerber文件的导出\n\n9.3 BOM的导出\n\n9.4 丝印文件的导出\n\n9.5 坐标文件的导出\n\n本章任务\n\n本章习题\n\n第10章 制作电路板\n\n10.1 PCB打样在线下单流程\n\n10.2 元件在线购买流程\n\n10.3 PCB贴片在线下单流程\n\n10.4 嘉立创下单助手\n\n本章任务\n\n本章习题\n\n反侵权盗版声明\n\n\chapter{第1章 基于STM32核心板的电路设计与制作流程}\n\n电路设计与制作是每个电子相关专业,如电子信息工程、光电工程、自动化、电子科学与技术、生物医学工程、医疗器械工程等,必须掌握的技能。本章将详细介绍基于STM32核心板的电路设计与制作流程,让读者先对电路设计与制作的过程有一个总体的认识。由于本书在讲解电路设计与制作技能时,既包含电路设计的软件操作部分,又包含电路制作实战环节,因此,为方便读者学习和实践,本书还配套有相关的资料包和开发套件。本章的最后两节将对资料包和开发套件进行简单的介绍。\n\n学习目标:\n\n了解什么是STM32核心板。\n\n了解STM32核心板的设计与制作流程。\n\n熟悉本书配套资料包的构成。\n\n熟悉本书配套开发套件的构成。\n\n\section{1.1 什么是STM32核心板}\n\n本书将以STM32核心板为载体对电路设计与制作过程进行详细讲解。那么,到底什么是STM32核心板?\n\nSTM32核心板是由通信-下载模块接口电路、电源转换电路、JTAG/SWD调试接口电路、独立按键电路、OLED显示屏接口电路、高速外部晶振电路、低速外部晶振电路、LED电路、STM32微控制器电路、复位电路和外扩引脚电路组成的电路板。\n\nSTM32核心板正面视图如图1-1所示,其中J4为通信-下载模块接口(XH-6P母座),J8为JTAG/SWD调试接口(简牛),J7为OLED显示屏接口(单排7P母座),J6为BOOT0电平选择接口(默认为不接跳线帽),RST(白头按键)为STM32系统复位按键,PWR(红色LED)为电源指示灯,LED1(蓝色LED)和LED2(绿色LED)为信号指示灯,KEY1、KEY2、KEY3为普通按键(按下为低电平,释放为高电平),J1、J2、J3为外扩引脚。\n\nSTM32核心板背面视图如图1-2所示,背面除直插件的引脚名称丝印外,还印有电路板的名称、版本号、设计日期和信息框。\n\nSTM32核心板要正常工作,还需要搭配一套JTAG/SWD仿真-下载器、一套通信-下载模块和一块OLED显示屏。仿真-下载器既能下载程序,又能进行断点调试,本书建议使用ST公司推出的ST-Link仿真-下载器。通信-下载模块主要用于计算机与STM32之间的串口通信,当然,该模块也可以对STM32进行程序下载。OLED显示屏则用于显示参数。STM32核心板、通信-下载模块、JTAG/SWD仿真-下载器、OLED显示屏的连接图如图1-3所示。\n\n图1-1 STM32核心板正面\n\n图1-2 STM32核心板背面\n\n图1-3 STM32核心板正常工作时的连接图\n\n\section{1.2 为什么选择STM32核心板}\n\n作为电路设计与制作的载体,有很多电路板可以选择,本书选择STM32核心板作为载体的主要原因有以下几点。\n\n(1)核心板包括电源电路、数字电路、下载电路、晶振电路、模拟电路、接口电路、I/O 外扩电路、简单外设电路等基本且必须掌握的电路。这符合本书“小而精”的理念,即电路虽不复杂,但基本上覆盖了各种常用的电路。\n\n(2)STM32系列单片机的片上资源极其丰富,又是基于库开发的,可采用C语言进行编程,资料非常多,性价比高,这些优点也使STM32系列单片机成为目前市面上最流行的微控制器之一。初学者只需要花费与学习51单片机基本相同的时间就能掌握比51单片机功能强大数倍甚至数十倍的STM32系列单片机。\n\n(3)STM32F103RCT6芯片在STM32系列中属于引脚数量少(只有64个引脚),但功能较齐全的单片机。因此,尽管引入了单片机,但初学者在学习设计与制作STM32核心板的过程中并不会感到难度有所增加。\n\n(4)STM32核心板可以完成从初级入门实验(如流水灯、按键输入),到中级实验(定时器、串口通信、ADC采样、DAC输出),再到复杂实验(OLED显示、UCOS操作系统)等至少20个实验。这些实验基本能够代表STM32单片机开发的各类实验,为初学者后续快速掌握STM32单片机编程技术奠定了基础。\n\n(5)由本书作者编写的《STM32F1开发标准教程》也是基于STM32核心板。因此,初学者可以直接使用自己设计和制作的STM32核心板,进入到STM32微控制器软件设计学习中,既能验证自己的核心板,又能充分利用已有资源。\n\n\section{1.3 电路设计与制作流程}\n\n传统的电路板设计与制作流程一般分为8个步骤:(1)需求分析;(2)电路仿真;(3)绘制原理图元件库;(4)绘制原理图;(5)绘制元件封装;(6)设计PCB;(7)导出生产文件;(8)制作电路板。具体如表1-1所示。\n\n表1-1 传统电路设计与制作流程\n\n这种传统流程主要针对已经熟练掌握电路板设计与制作各项技能的工程师。而对于初学者来说,要完全掌握这些技能,并最终设计制作出一块电路板,不仅需要有超强的耐力坚持到最后一步,更要有严谨的作风,保证每一步都不出错。\n\n在传统流程的基础上,本书做了如下改进:(1)不求全面覆盖,比如对需求分析和电路仿真技能不做讲解;(2)增加了焊接部分,加强实践环节,让初学者对电路理解更加深刻;(3)所有内容的讲解都聚焦于一块STM32核心板;(4)每一步的执行都不依赖于其他步骤,比如,第一步就能进行电路板验证,又如,原理图设计过程可以使用现成的集成库而不用自己提前制作。\n\n这样安排的好处是,每一步都能很容易获得成功,这种成就感会激发初学者内在的兴趣,从而由兴趣引导其迈向下一步;聚焦于一块STM32核心板,让所有的技能都能学以致用,并最终制作出一块STM32核心板。\n\n本书以STM32核心板为载体,将电路设计与制作分为9个步骤,如表1-2所示,下面对各流程进行详细介绍。\n\n表1-2 本书电路设计与制作流程\n\n1.STM32核心板程序下载与验证\n\n这一步要求将开发套件中的STM32核心板、通信-下载模块、OLED显示屏、USB线、XH-6A双端线等连接起来,并在计算机上使用mcuisp软件,将HEX文件下载到STM32F103RCT6芯片的Flash中,检查STM32核心板是否能够正常工作。通过这一流程可快速了解STM32核心板的构成及其基本工作方式。\n\n2.准备物料和工具\n\n根据物料清单(也称BOM)准备相应的元件,根据工具清单准备相应的焊接工具,如电烙铁、万用表、焊锡、镊子和松香等\n\n[1]\n\n。通过准备物料和工具,可初步认识元件以及各种焊接工具和材料。\n\n3.焊接STM32核心板\n\n利用开发套件提供的3块空电路板,以及第2步准备的物料和焊接工具,按照说明将元件焊接到电路板上,边焊接边调试,可将第1步中连通的STM32核心板作为参考。通过这一步操作的训练,读者应掌握电路板焊接技能,熟练掌握电烙铁、镊子和万用表的使用。\n\n4.熟悉PCB设计工具\n\n本书使用立创EDA作为PCB设计工具,熟悉立创EDA的使用方法。\n\n5.设计STM32核心板原理图\n\n使用立创EDA的元件库,参照STM32核心板原理图(参见本书资料包中的PDFSchDoc文件夹),使用立创EDA绘制STM32核心板的原理图。\n\n6.设计STM32核心板PCB\n\n首先将STM32核心板原理图导入PCB设计环境中,然后对STM32核心板进行布局和布线。\n\n7.创建元件库\n\n创建元件库包括创建原理图库和PCB库。\n\n8.导出生产文件\n\n通过立创EDA导出PCB生产文件,包括BOM、Gerber文件、丝印文件及坐标文件等。\n\n9.制作STM32核心板\n\nSTM32核心板的制作包括PCB打样和贴片,可通过PCB加工企业的网站进行网上PCB打样下单以及贴片下单。\n\n\section{1.4 本书提供的资料包}\n\n本书配套资料包名称为“《电路设计与制作实用教程——基于立创EDA》资料包” (可以通过微信公众号“卓越工程师培养系列”提供的链接进行下载)。\n\n资料包由若干个文件夹组成,如表13所示。\n\n表1-3 本书提供的资料包清单\n\n\section{1.5 本书配套开发套件}\n\n本书配套的STM32核心板开发套件(可以通过微信公众号“卓越工程师培养系列”提供的链接获得)由基础包、物料包、工具包组成。其中基础包包括1个通信-下载模块、1块STM32核心板、2条Mini-USB线、1条XH-6P双端线、1个ST-Link调试器、1条20P灰排线、3块STM32核心板的PCB空板,物料包有3套,工具包包括电烙铁、镊子、焊锡、万用表、松香、吸锡带,如表1-4所示。\n\n表1-4 STM32开发套件物品清单\n\n续表\n\n\section{本章任务}\n\n学习完本章后,要求熟悉STM32核心板的电路设计与制作流程,并下载本书配套的资料包,准备好配套的开发套件。\n\n\section{本章习题}\n\n1.什么是STM32核心板?\n\n2.简述传统的电路设计与制作流程。\n\n3.简述本书提出的电路设计与制作流程。\n\n4.通信-下载模块的作用是什么?\n\n5.JTAG/SWD仿真-下载器的作用是什么?\n\n6.焊接电路板的工具都有哪些?简述每种工具的功能。\n\n[1]\n\n这些物料和焊接工具,读者可以根据提供的清单自行采购,也可以通过微信公众号“卓越工程师培养系列”提供的链接进行打包采购。\n\n\chapter{第2章 STM32核心板介绍}\n\n第1章介绍了STM32核心板的设计与制作流程。本章进一步讲解STM32核心板的各个电路模块,并简要介绍可以在STM32核心板上开展的实验,从而使得读者完成电路板的设计与制作之后,既能方便地继续学习STM32单片机,还可以对STM32核心板进行深层次的验证。\n\n学习目标:\n\n了解什么是STM32芯片。\n\n了解STM32核心板的各个电路模块。\n\n\section{2.1 STM32芯片介绍}\n\n在微控制器选型中,工程师常常会陷入这样一个困局:一方面抱怨8位/16位单片机有限的指令和性能,另一方面抱怨32位处理器的高成本和高功耗。能否有效地解决这个问题,让工程师不必在性能、成本、功耗等因素中做出取舍和折中?\n\n基于ARM公司2006年推出的Cortex-M3内核,ST公司于2007年推出的STM32系列单片机很好地解决了上述问题。因为Cortex-M3内核的计算能力是1.25DMIPS/MHz,而ARM7TDMI只有0.95DMIPS/MHz。而且STM32单片机拥有1μs的双12位ADC、4Mbit/s的UART、18Mbit/s的SPI、18MHz的I/O翻转速度,更重要的是,STM32单片机在72MHz工作时功耗只有36mA(所有外设处于工作状态),而待机时功耗只有2μA。\n\n[1]\n\n由于STM32单片机拥有丰富的外设、强大的开发工具、易于上手的固件库,在32位微控制器选型中,STM32单片机已经成为许多工程师的首选。据统计,从2007年到2016年,STM32单片机出货量累计20亿颗,十年间ST公司在中国的市场份额从 2%增长到14%。iSuppli的2016年下半年市场报告显示,STM32单片机在中国Cortex-M市场的份额占到45.8%。\n\n尽管STM32单片机已经推出十余年,但它依然是市场上32位单片机的首选,而且经过十余年的积累,各种开发资料都非常完善,这也降低了初学者的学习难度。因此,本书选用STM32单片机作为载体,核心板上的主控芯片就是封装为LQFP64的STM32F103RCT6芯片,最高主频可达72MHz。\n\nSTM32F103RCT6芯片拥有的资源包括48KB SRAM、256KB Flash、1个FSMC接口、1个NVIC、1个EXTI(支持19个外部中断/事件请求)、2个DMA(支持12个通道)、1个RTC、2个16位基本定时器、4个16位通用定时器、2个16位高级定时器、1个独立看门狗、1个窗口看门狗、1个24位SysTick、2个I\n\n2\n\nC、5个串口(包括3个同步串口和2个异步串口)、3个SPI、2个I\n\n2\n\nS(与SPI2和SPI3复用)、1个SDIO接口、1个CAN总线接口、1个USB接口、51个通用I/O接口、3个12位ADC(可测量16个外部和2个内部信号源)、2个12位DAC、1个内置温度传感器、1个串行JTAG调试接口。\n\nSTM32系列单片机可以开发各种产品,如智能小车、无人机、电子体温枪、电子血压计、血糖仪、胎心多普勒、监护仪、呼吸机、智能楼宇控制系统、汽车控制系统等。\n\n\section{2.2 STM32核心板电路简介}\n\n本节将详细介绍STM32核心板的各电路模块,以便读者更好地理解后续原理图设计和PCB设计的内容。\n\n工程师编写完程序后,需要通过通信-下载模块将.hex(或.bin)文件下载到STM32中。通信-下载模块向上与计算机连接,向下与STM32核心板连接,通过计算机上的STM32下载工具(如mcuisp软件),就可以将程序下载到STM32中。通信-下载模块除具备程序下载功能外,还担任着“通信员”的角色,即可以通过通信-下载模块实现计算机与STM32之间的通信。此外,通信-下载模块还为STM32核心板提供5V电压。需要注意的是,通信-下载模块既可以输出5V电压,也可以输出3.3V电压,本书中的实验均要求在5V电压环境下实现,因此,\n\n在连接通信-下载模块与STM32时,需要将通信-下载模块的电源输出开关拨到5V挡位。\n\nSTM32核心板通过一个XH-6A的底座连接到通信-下载模块,通信-下载模块再通过USB线连接到计算机的USB接口,通信-下载模块接口电路如图2-1所示。STM32核心板只要通过通信-下载模块连接到计算机,标识为PWR的红色LED就会处于点亮状态。R9电阻起到限流的作用,防止红色LED被烧坏。\n\n图2-1 通信-下载模块接口电路\n\n[2]\n\n由图2-1可以看出,通信-下载模块接口电路总共有6个引脚,引脚说明如表2-1所示。\n\n表2-1 通信-下载模块接口电路引脚说明\n\n图2-2所示为STM32核心板的电源转换电路,将5V输入电压转换为3.3V输出电压。通信-下载模块的5V电源与STM32核心板电路的5V电源网络相连接,二极管D1(SS210)的功能是防止STM32核心板向通信-下载模块反向供电,二极管上会产生约0.4V的正向电压差,因此,低压差线性稳压电源U2(AMS1117-3.3)的输入端(In)的电压并非为5V,而是4.6V左右。经过低压差线性稳压电源的降压,在U2的输出端(Out)产生3.3V的电压。为了调试方便,在电源转换电路上设计了3个测试点,分别是5V、3V3和GND。\n\n图2-2 电源转换电路\n\n除了可以使用上述通信-下载模块下载程序,还可以使用JLINK或ST-Link进行程序下载。JLINK和ST-Link不仅可以下载程序,还可以对STM32微控制器进行在线调试。图2-3所示是STM32核心板的JTAG/SWD调试接口电路,这里采用了标准的JTAG接法,这种接法兼容SWD接口,因为SWD接口只需要4根线(SWCLK、SWDIO、VCC和GND)。需要注意的是,该接口电路为JLINK或ST-Link提供3.3V的电源,因此,不能通过JLINK或ST-Link向STM32核心板供电,而是通过STM32核心板向JLINK或ST-Link供电。\n\n由于SWD只需要4根线,因此,在进行产品设计时,建议使用SWD接口,摒弃JTAG接口,这样就可以节省很多接口。尽管JLINK和ST-Link都可以下载程序,而且还能进行在线调试,但是无法实现STM32微控制器与计算机之间的通信。因此,在设计产品时,除了保留SWD接口,还建议保留通信-下载接口。\n\n图2-3 JTAG/SWD调试接口电路\n\nSTM32核心板上有3个独立按键,分别是KEY1、KEY2和KEY3,其原理图如图2-4所示。每个按键都与一个电容并联,且通过一个10kΩ电阻连接到3.3V电源网络。按键未按下时,输入到STM32微控制器的电压为高电平,按键按下时,输入到STM32微控制器的电压为低电平。Key1、Key2和Key3分别连接到STM32F103RCT6芯片的PC1、PC2和PA0引脚上。\n\n图2-4 独立按键电路\n\n本书所使用的STM32核心板,除了可以通过通信-下载模块在计算机上显示数据,还可以通过板载OLED显示屏接口电路外接一个OLED显示屏进行数据显示,图2-5所示即为OLED显示屏接口电路,该接口电路为OLED显示屏提供3.3V的电源。\n\nOLED显示屏接口电路的引脚说明如表2-2所示,其中OLED_DIN(SPI2_MOSI)、OLED_SCK(SPI2_SCK)、OLED_D/C(PC3)、OLED_RES(SPI2_MOSI)和OLED_CS(SPI2_NSS)分别连接在STM32F103RCT6的PB15、PB13、PC3、PB14和PB12引脚上。\n\n图2-5 OLED显示接口电路\n\n表2-2 OLED显示屏接口电路引脚说明\n\n说明:括号中为对应的单片机引脚名称。\n\nSTM32微控制器具有非常强大的时钟系统,除了内置高精度和低精度的时钟系统,还可以通过外接晶振,为STM32微控制器提供高精度和低精度的时钟系统。图2-6所示为外接晶振电路,其中Y1为8MHz晶振,连接时钟系统的HSE(外部高速时钟),Y2为32.768kHz晶振,连接时钟系统的LSE(外部低速时钟)。\n\n图2-6 晶振电路\n\n除了标识为PWR的电源指示LED,STM32核心板上还有两个LED,如图2-7所示。LED1为蓝色,LED2为绿色,每个LED分别与一个330Ω电阻串联后连接到STM32F103RCT6芯片的引脚上,在LED电路中,电阻起着分压限流的作用。LED1和LED2分别连接到STM32F103RCT6芯片的PC4和PC5引脚上。\n\n图2-7 LED电路\n\n图2-8所示的STM32微控制器电路是STM32核心板的核心部分,由STM32滤波电路、STM32微控制器、复位电路、启动模式选择电路组成。\n\n图2-8 STM32微控制器电路\n\n电源网络一般都会有高频噪声和低频噪声,而大电容对低频有较好的滤波效果,小电容对高频有较好的滤波效果。STM32F103RCT6芯片有4组数字电源-地引脚,分别是VDD_1、VDD_2、VDD_3、VDD_4、VSS_1、VSS_2、VSS_3、VSS_4,还有一组模拟电源-地引脚,即VDDA、VSSA。C1、C2、C6、C7这4个电容用于滤除数字电源引脚上的高频噪声,C5用于滤除数字电源引脚上的低频噪声,C4用于滤除模拟电源引脚上的高频噪声,C3用于滤除模拟电源引脚上的低频噪声。\n\n为了达到良好的滤波效果,还需要在进行PCB布局时,尽可能将这些电容摆放在对应的电源-地回路之间,且布线越短越好。\n\nNRST引脚通过一个10kΩ电阻连接3.3V电源网络,因此,用于复位的引脚在默认状态下是高电平,只有当复位按键按下时,NRST引脚为低电平,STM32F103RCT6芯片才进行一次系统复位。\n\nBOOT0引脚(60号引脚)、BOOT1引脚(28号引脚)为STM32F103RCT6芯片启动模块选择接口,当BOOT0为低电平时,系统从内部Flash启动。因此,默认情况下,J6跳线不需要连接。\n\nSTM32核心板上的STM32F103RCT6芯片总共有51个通用I/O接口,分别是PA0~15、PB0~15、PC0~15、PD0~2。其中,PC14、PC15连接外部的32.768kHz晶振,PD0、PD1连接外部的8MHz晶振,除了这4个引脚,STM32核心板通过J1、J2、J3共 3组排针引出其余47个通用I/O接口。外扩引脚电路图如图2-9所示。\n\n图2-9 外扩引脚电路原理图\n\n读者可以通过这3组排针,自由扩展外设。此外,J1、J2、J3这3组排针分别还包括2组3.3V电源和接地(GND),这样就可以直接通过STM32核心板对外设进行供电,大大降低了系统的复杂度。因此,利用这3组排针,可以将STM32核心板的功能发挥到极致。\n\n\section{2.3 基于STM32核心板可以开展的实验}\n\n基于STM32核心板可以开展的实验非常丰富,这里仅列出具有代表性的22个实验,如表2-3所示。\n\n表2-3 STM32核心板可开展的部分实验清单\n\n\section{本章任务}\n\n完成本章的学习后,应重点掌握STM32核心板的电路原理,以及每个模块的功能。\n\n\section{本章习题}\n\n1.简述STM32与ST公司和ARM公司的关系。\n\n2.通信-下载模块接口电路中使用了一个红色LED(PWR)作为电源指示,请问如何通过万用表检测LED的正、负端?\n\n3.通信-下载模块接口电路中的电阻(R9)有什么作用?该电阻阻值的选取标准是什么?\n\n4.电源转换电路中的5V电源网络能否使用3.3V电压?请解释原因。\n\n5.电源转换电路中,二极管上的压差为什么不是一个固定值?这个压差的变化有什么规律?请结合SS210的数据手册进行解释。\n\n6.什么是低压差线性稳压电源?请结合AMS1117-3.3的数据手册,简述低压差线性稳压电源的特点。\n\n7.低压差线性稳压电源的输入端和输出端均有电容(C16、C17、C18),请问这些电容的作用是什么?\n\n8.电路板上的测试点有什么作用?哪些位置需要添加测试点?请举例说明。\n\n9.电源转换电路中的电感(L2)和电容(C19)有什么作用?\n\n10.独立按键电路中的电容有什么作用?\n\n11.独立按键电路为什么要通过一个电阻连接3.3V电源网络?为什么不直接连接3.3V电源网络?\n\n[1]\n\n通常STM32单片机工作在一定电压(5V)下,可用电流的大小表示其功耗。\n\n[2]\n\n书中采用的模块电路图截取自附录中的原理图,为了方便读者操作,全书保持一致,其中部分元件符号与国标有出入,特此说明。\n\n\chapter{第3章 STM32核心板程序下载与验证}\n\n本章介绍STM32核心板的程序下载与验证,也就是先将STM32核心板连接到计算机上,通过软件向STM32核心板下载程序,观察STM32核心板的工作状态。传统的电路设计流程是:先进行电路板设计,然后制作,最后才是电路板验证。考虑到本书主要针对初学者,因此,将传统流程颠倒过来,先验证电路板,然后焊接,最后介绍如何设计电路板。这样做的好处是让初学者开门见山,手中先有一个样板,在后续的焊接和电路设计环节就能够进行参考对照,以便能够快速掌握电路设计与制作的各项技能。\n\n学习目标:\n\n掌握通过通信-下载模块对STM32核心板进行程序下载的方法。\n\n掌握通过ST-Link对STM32核心板进行程序下载的方法。\n\n了解STM32核心板的工作原理。\n\n\section{3.1 准备工作}\n\n在进行STM32核心板程序下载与验证之前,先确认STM32核心板套件是否完整。STM32核心板开发套件由基础包、物料包、工具包组成,具体详见1.5节。\n\n\section{3.2 将通信-下载模块连接到STM32核心板}\n\n首先,取出开发套件中的通信-下载模块、STM32核心板(将OLED显示屏插在STM32核心板的J7母座上)、1条Mini-USB线、1条XH-6P双端线。将Mini-USB线的公口(B型插头)连接到通信-下载模块的USB接口,再将XH-6P双端线连接到通信-下载模块的白色XH-6P底座上。然后将XH-6P双端线接在STM32核心板的J4底座上,如图3-1所示。最后将Mini-USB线的公口(A型插头)插在计算机的USB接口上。\n\n图3-1 STM32核心板连接实物图(仅含通信-下载模块)\n\n\section{3.3 安装CH340驱动}\n\n接下来,安装通信-下载模块驱动。在本书资料包的Software目录下找到“CH340驱动(USB串口驱动)_XP_WIN7共用”文件夹,双击运行SETUP.EXE,单击“安装”按钮,在弹出的DriverSetup对话框中单击“确定”按钮,即安装完成,如图3-2所示。\n\n图3-2 安装通信-下载模块驱动\n\n驱动安装成功后,将通信-下载模块通过USB线连接到计算机,然后在计算机的设备管理器里面找到USB串口,如图3-3所示。注意,串口号不一定是COM4,每台计算机可能会不同。\n\n图3-3 计算机设备管理器中显示USB串口信息\n\n\section{3.4 通过mcuisp下载程序}\n\n在Software目录下找到并双击mcuisp软件,在图3-4所示的菜单栏中单击“搜索串口(X)”按钮,在弹出的下拉菜单中选择“COM4:空闲USB-SERIAL CH340”(再次提示,不一定是COM4,每台机器的COM编号可能会不同),如果显示“占用”,则尝试重新插拔通信-下载模块,直到显示“空闲”字样。\n\n图3-4 使用mcuisp进行程序下载步骤一\n\n如图3-5所示,首先定位.hex文件所在的路径,即在本书配套资料包中的STM32Keil Project\HexFile目录下,找到STM32KeilPrj.hex文件。然后勾选“编程前重装文件”项,再勾选“校验”项和“编程后执行”项,选择“DTR的低电平复位,RTS高电平进BootLoader”,单击“开始编程(P)”按钮,出现“成功写入选项字节,www.mcuisp.com向您报告,命令执行完毕,一切正常”表示程序下载成功。\n\n图3-5 使用mcuisp进行程序下载步骤二\n\n\section{3.5 通过串口助手查看接收数据}\n\n在Software目录下找到并双击“运行串口助手”软件(sscom42.exe),如图3-6所示。选择正确的串口号,与mcuisp串口号一致,将波特率改为“115200”,然后单击“打开串口”按钮,取消勾选“HEX显示”项,当窗口中每隔1s弹出“This is a STM32 demo project,by ZhangSan”时,表示成功。注意,实验完成后,先单击“关闭串口”按钮将串口关闭,再关闭STM32核心板的电源。\n\n图3-6 串口助手操作步骤\n\n\section{3.6 查看STM32核心板工作状态}\n\n此时可以观察到STM32核心板上电源指示灯(红色)正常显示,蓝色LED和绿色LED交替闪烁,而且OLED显示屏上的日期和时间正常运行,如图3-7所示。\n\n图3-7 STM32核心板上正常工作状态示意图\n\n\section{3.7 通过ST-Link下载程序}\n\n从开发套件中再取出1个ST-Link调试器、1条Mini-USB线,1条20P灰排线。在前面连接的基础上,将Mini-USB线的公口(B型插头)连接到ST-Link调试器;将20P灰排线的一端连接到ST-Link调试器,将另一端连接到STM32核心板的JTAG/SWD调试接口(编号为J8)。最后将两条Mini-USB线的公口(A型插头)均连接到计算机的USB接口,如图3-8所示。\n\n在Software目录下找到并打开“ST-LINK驱动”文件夹,找到应用程序dpinst_amd64和dpinst_x86。双击dpinst_amd64即可安装,如果提示错误,可以先将dpinst_amd64卸载,然后双击安装dpinst_x86,(注意,dpinst仅安装一个即可)如图3-9所示。\n\n图3-8 STM32核心板连接实物图(含ST-Link调试器和通信-下载模块)\n\n图3-9 ST-Link驱动安装包\n\nST-Link驱动安装成功后,可以在设备管理器中看到STMicroelectronics STLink dongle,如图3-10所示。\n\n打开Keil μVision5软件\n\n[1]\n\n,如图3-11所示,单击Options for Target按钮,进入设置界面。\n\n图3-10 ST-Link驱动安装成功示意图\n\n图3-11 ST-Link调试模式设置步骤一\n\n如图3-12所示,在弹出的Options for Target ‘Target1’对话框中的Debug标签页中,在Use下拉菜单中选择ST-Link Debugger,然后单击Settings按钮。\n\n如图3-13所示,在弹出的Cortex-M Target Driver Setup对话框中的Debug标签页中,在ort下拉菜单中选择SW,在Max下拉菜单中选择1.8MHz,最后单击“确定”按钮。\n\n图3-12 ST-Link调试模式设置步骤二\n\n图3-13 ST-Link调试模式设置步骤三\n\n如图3-14所示,在Options for Target‘Target 1’对话框中,打开Utilities标签页,勾选Use Debug Driver和Update Target before Debugging项,最后单击OK按钮。\n\nST-Link调试模式设置完成后,在如图3-15所示的界面中,单击Download按钮,将程序下载到STM32单片机,下载成功后,在Bulid Output面板中将出现如图3-15所示的字样,表明程序已经通过ST-Link调试器成功并下载到STM32单片机中。\n\n图3-14 ST-Link调试模式设置步骤四\n\n图3-15 通过ST-Link向STM32单片机下载程序成功界面\n\n\section{本章任务}\n\n完成本章的学习后,应能熟练使用通信-下载模块进行STM32核心板的程序下载,能熟练使用ST-Link仿真器进行STM32核心板的程序下载,并能够用万用表测试STM32核心板上的5V和3.3V两个测试点的电压值。\n\n\section{本章习题}\n\n1.什么是串口驱动?为什么要安装串口驱动?\n\n2.通过查询网络资料,对串口编号进行修改,例如,串口编号默认是COM1,将其改为COM4。\n\n3.ST-Link除了可以下载程序,还有哪些其他功能?\n\n[1]\n\n在此步骤之前,首先确保计算机上已安装Keil μVision5软件。这里推荐使用MDK5.20版本,安装完成后,还需安装Keil.STM32F1xx DFP.2.1.0软件包。以上软件和软件包及其安装方法可以通过微信公众号 “卓越工程师培养系列”下载。打开 “D:\《电路设计与制作实用教程——基于立创EDA》 资料包\STM32KeilProject\STM32KeilPrj\Project”,双击并运行STM32KeilPrj.uvprojx。\n\n\chapter{第4章 STM32核心板焊接}\n\n第3章讲解了STM32核心板的程序下载与验证,让读者对STM32核心板的工作原理有了初步的认识,本章将介绍STM32核心板的焊接。在焊接前,首先要准备好所需要的工具和材料、各种电子元件和STM32核心板空板。本书将焊接的过程分为五个步骤,每个步骤都有严格的要求和焊接完成的验证标准,而且可以与第3章验证过的STM32核心板进行对比。通过本章的学习和实践,读者将掌握焊接STM32核心板的技能,以及万用表的简单操作。\n\n学习目标:\n\n能够根据焊接工具和材料清单准备焊接STM32核心板所需的工具和材料。\n\n能够根据BOM准备STM32核心板所需的元件。\n\n按照分步焊接和测试的方法,焊接至少一块STM32核心板,并验证通过。\n\n掌握万用表的使用方法,能够进行电压、电流和电阻等的测量。\n\n\section{4.1 焊接工具和材料}\n\n大多数介绍电路设计与制作的书籍,通常都是按照软件介绍与安装、原理图设计、PCB设计、电路板打样、焊接调试的顺序进行讲解。本书将焊接调试调整到原理图设计和PCB设计前,这种安排有几个好处:(1)快速焊接并调试成功一块电路板,可以迅速建立初学者的自信心,自信心演变成兴趣,兴趣又会吸引初学者进入原理图和PCB设计环节;(2)电路板实物中的电路比PCB设计软件中的电路更加形象、逼真,如电路板尺寸、元件结构、元件间距、焊盘大小、焊盘间距、丝印尺寸等,通过实物焊接,初学者对这些概念的理解将更加深刻,从而在学习原理图和PCB设计环节就更容易上手;(3)在焊接过程中,通过实训可对各种焊接工具,如电烙铁、焊锡、松香、镊子,有更加深刻的认识。当然,焊接之前先要准备好焊接所需的工具和材料,如表4-1所示,下面简要介绍。\n\n表4-1 焊接工具和材料清单\n\n续表\n\n1.电烙铁\n\n电烙铁有很多种,常用的有内热式、外热式、恒温式和吸锡式。为了方便携带,建议使用内热式电烙铁。此外,还需要有烙铁架和海绵,烙铁架用于放置电烙铁,海绵用于擦拭烙铁锡渣,海绵不应太湿或太干,应手挤海绵直至不滴水为宜。\n\n电烙铁常用的烙铁头有四种,分别是刀头、一字形、马蹄形、尖头,如图4-1所示。本书建议初学者直接使用刀头,因为STM32核心板上的绝大多数元件都是贴片封装的,刀头适用于焊接多引脚器件以及需要托焊的场合,这对于焊接STM32芯片及排针非常适合。刀头在焊接贴片电阻、电容、电感时也非常方便。\n\n图4-1 四种常用的烙铁头\n\n(1)电烙铁的使用方法\n\n① 先接上电源,数分钟后待烙铁头的温度升至焊锡熔点时,蘸上助焊剂(松香),然后用烙铁头刃面接触焊锡丝,使烙铁头上均匀地镀上一层锡(亮亮的、薄薄的就可以)。这样做,便于焊接并防止烙铁头表面氧化。没有蘸上锡的烙铁头,焊接时不容易上锡。\n\n② 进行普通焊接时,一手拿烙铁,一手拿焊锡丝,靠近根部,两头轻轻一碰,一个焊点就形成了。\n\n③ 焊接时间不宜过长,否则容易烫坏元件,必要时可用镊子夹住引脚帮助散热。\n\n④ 焊接完成后,一定要断开电源,等电烙铁冷却后再收起来。\n\n(2)电烙铁使用注意事项\n\n① 使用前认真检查烙铁头是否松动。\n\n② 使用时不能用力敲击,烙铁头上焊锡过多时用湿海绵擦拭,不可乱甩,以防烫伤他人。\n\n③ 电烙铁要放在烙铁架上,不能随便乱放。\n\n④ 注意导线不能触碰到烙铁头,避免引发火灾。\n\n⑤ 不要让电烙铁长时间处于待焊状态,因为温度过高也会造成烙铁头“烧死”。\n\n⑥ 使用结束后务必切断电源。\n\n2.镊子\n\n焊接电路板常用的镊子有直尖头和弯尖头,建议使用直尖头。\n\n3.焊锡\n\n焊锡是在焊接线路中连接电子元件的重要工业原材料,是一种熔点较低的焊料。常用的焊锡主要是用锡基合金做的焊料。根据焊锡中间是否含有松香,将焊锡分为实心焊锡和松香芯焊锡。焊接元件时建议采用松香芯焊锡,因为这种焊锡熔点较低,而且内含松香助焊剂,松香起到湿润、降温、提高可焊性的作用,使用极为方便。\n\n4.万用表\n\n万用表一般用于测量电压、电流、电阻和电容,以及检测短路。在焊接STM32核心板时,万用表主要用于(1)测量电压;(2)测量某一个回路的电流;(3)检测电路是否短路;(4)测量电阻的阻值;(5)测量电容的容值。\n\n(1)测电压\n\n将黑表笔插入COM孔,红表笔插入VΩ孔,旋钮旋到合适的电压挡(万用表表盘上的电压值要大于待测电压值,且最接近待测电压值的电压挡位)。然后,将两个表笔的尖头分别连接到待测电压的两端(注意,万用表是并联到待测电压两端的),保持接触稳定,且电路应处于工作状态,电压值即可从万用表显示屏上读取。注意,万用表表盘上的“V-”表示直流电压挡,“V~”表示交流电压挡,表盘上的电压值均为最大量程。由于STM32核心板采用直流供电,因此测量电压时,要将旋钮旋到直流电压挡。\n\n(2)测电流\n\n将黑表笔插入COM孔,红表笔插入mA孔,旋钮旋到合适的电流挡(万用表表盘上的电流值要大于待测电流值,且最接近待测电流值的电流挡位)。然后,将两个表笔的尖头分别连接到待测电流的两端(注意,万用表是串联到待测电流的电路中的),保持接触稳定,且电路应处于工作状态,电流值即可从万用表显示屏上读取。注意,万用表表盘上的“A-”表示直流电流挡,“A~”表示交流电流挡,表盘上的电流值均为最大量程。由于STM32核心板上只有直流供电,因此测量电流时,要将旋钮旋到直流电流挡。而且,STM32核心板上的电流均为毫安(mA)级。\n\n(3)检测短路\n\n将黑表笔插入COM孔,红表笔插入VΩ孔,旋钮旋到蜂鸣/二极管挡。然后,将两个表笔的尖头分别连接到待测短路电路的两端(注意,万用表是并联到待测短路电路的两端的),保持接触稳定,将电路板的电源断开。如果万用表蜂鸣器鸣叫且指示灯亮,表示所测电路是连通的,否则,所测电路处于断开状态。\n\n(4)测电阻\n\n将黑表笔插入COM孔,红表笔插入VΩ 孔,旋钮旋到合适的电阻挡(万用表表盘上的电阻值要大于待测电阻值,且最接近待测电阻值的电阻挡位)。然后,将两个表笔的尖头分别连接到待测电阻两端(注意,万用表是并联到待测电阻两端的),保持接触稳定,将电路板的电源断开,电阻值即可从万用表显示屏上读取。如果直接测量某一电阻,可将两个表笔的尖头连接到待测电阻的两端直接测量。注意,电路板上某一电阻的阻值一般小于标识阻值,因为电路板上的电阻与其他等效网络并联,并联之后的电阻值小于其中任何一个电阻。\n\n(5)测电容\n\n将黑表笔插入COM孔,红表笔插入VΩ 孔,旋钮旋到合适的电容挡(万用表表盘上的电容值要大于待测电容值,且最接近待测电容值的电容挡位)。然后,将两个表笔的尖头分别连接到待测电容两端(注意,万用表是并联到待测电容两端的),保持接触稳定,电容值即可从万用表显示屏上读取。注意,待测电容应为未焊接到电路板上的电容。\n\n5.松香\n\n松香在焊接中作为助焊剂,起助焊作用。从理论上讲,助焊剂的熔点比焊料低,其比重、黏度、表面张力都比焊料小,因此在焊接时,助焊剂先融化,很快流浸、覆盖于焊料表面,起到隔绝空气防止金属表面氧化的作用,并能在焊接的高温下与焊锡及被焊金属的表面发生氧化膜反应,使之熔解,还原纯净的金属表面。合适的焊锡有助于焊出满意的焊点形状,并保持焊点的表面光泽。松香是常用的助焊剂,它是中性的,不会腐蚀电路元件和烙铁头。如果是新印制的电路板,在焊接之前要在铜箔表面涂上一层松香水。如果是已经印制好的电路板,则可直接焊接。松香的具体使用因个人习惯而不同,有的人习惯每焊接完一个元件,都将烙铁头在松香上浸一下,有的人只有在电烙铁头被氧化,不太方便使用时,才会在上面浸一些松香。松香的使用方法也很简单,打开松香盒,把通电的烙铁头在上面浸一下即可。如果焊接时使用的是实心焊锡,加些松香是必要的,如果使用松香锡焊丝,可不使用松香。\n\n6.吸锡带\n\n在焊接引脚密集的贴片元件时,很容易因焊锡过多导致引脚短路,使用吸锡带就可以“吸走”多余的焊锡。吸锡带的使用方法很简单:用剪刀剪下一小段吸锡带,用电烙铁加热使其表面蘸上一些松香,然后用镊子夹住将其放在焊盘上,再用电烙铁压在吸锡带上,当吸锡带变为银白色时即表明焊锡被“吸走”了。注意,吸锡时不可用手碰吸锡带,以免烫伤。\n\n7.其他工具\n\n常用的焊接工具还包括吸锡枪等,由于STM32核心板上主要是贴片元件,基本用不到吸锡枪,因此这里就不详细介绍,如需了解其他焊接工具和材料,可以查阅相关教材或者网站。\n\n\section{4.2 STM32核心板元件清单}\n\nSTM32核心板的元件清单,也称为BOM,如表4-2所示。\n\n表4-2 STM32核心板元件清单\n\n无论是读者自己焊接,还是由贴片厂焊接,都需要准备元件(也称物料)。根据表4-2中的ID可方便快速地进行物料定位和备料,这种优势在进行复杂电路板备料时更加明显。\n\n第二列Supplier Part相当于每个元件的身份证号码。企业一般都会有自己的元件编号,由于物料系统比较庞杂,作为初学者,建立自己的物料体系不现实。那么,如何能够既不用亲自建立自己的物料库,又能够方便使用规范的物料库呢?推荐直接使用“立创商城”(www.szlcsc.com)的物料体系。因为立创商城上的物料体系比较严谨规范,而且采购非常方便,价格也较实惠,读者可以只花1元就能买到100个贴片电阻,更重要的是可以基本实现一站式采购。这样既省时,又节约成本,可大大降低初学者学习的门槛和成本。当然,立创商城的元件也常常会出现下架和缺货的现象,但是,立创商城提供的物料种类非常全,读者可以非常容易地在其网站上找到可替代的元件。因此,本书直接引用了立创商城提供的元件编号,这样,读者就可以方便地在立创商城上根据STM32核心板元件清单上的元件编号采购所需的元件。\n\n第三列是Name(元件名称)。电容是以容值、精度、耐压值和封装进行命名的,电阻是以阻值、精度和封装进行命名的,每种元件都有其严格的命名规范,后续章节将详细介绍。\n\n第四列Designator(元件号)是电路板上的元件编号,由大写字母+数字构成。字母R代表电阻,字母C代表电容,字母J代表插件,字母D代表二极管,字母U代表芯片。相同型号的元件被列在同一栏中,以便于备料。\n\n第五列是Footprint(封装),每个元件都有对应的封装,在备料时一定要确认封装是否正确。\n\n第六列是Quantity(数量),使用PCB工具生成物料清单时,相同型号的物料被归类在一起,用元件号加以区分,这里的数量就是相同型号的物料的数量。需要强调的是,在备料时,电阻、电容、二极管等小型低价元件按照电路板实际所需数量的120%准备,其他可以按照100%~110%准备。比如要生产10套电路板,每种型号的电阻按照标准数量的12倍准备;如果某种规格的排针需要30条,可以准备30~33条;如果某种规格的芯片需要10片,可以准备10~11片。\n\n经过若干轮实践证明,绝大多数初学者都能在焊接第三块电路板前,至少调试通一块电路板。当然,也有很多初学者每焊接一块就能调试通一块,焊接后面的两块电路板是为了巩固焊接和调试技能。鉴于此,本书提供3套开发套件,建议读者在备料时也按照3套的数量准备,即按照表格中的数量乘以3进行备料,电阻、电容、二极管等小型低价元件可以多备一些。\n\n\section{4.3 STM32核心板焊接步骤}\n\n准备好空的STM32核心板、焊接工具和材料、元件后,就可以开始电路板的焊接。\n\n很多初学者在学习焊接时,常常拿到一块电路板就急着把所有的元件全部焊上去。由于在焊接过程中没有经过任何测试,最终通电后,电路板要么没有任何反应,要么被烧坏,而真正一次性焊接好并验证成功的极少。而且,出了问题,不知道从何处解决。\n\n尽管STM32核心板电路不是很复杂,但是要想一次性焊接成功,还是有一定的难度。本书将STM32核心板焊接分为五个步骤,每个步骤完成后都有严格的验证标准,出了问题可以快速找到问题。即使从未接触过焊接的新手,也能通过这五个步骤迅速掌握焊接的技能。\n\nSTM32核心板焊接的五个步骤如表4-3所示,每一步都有要焊接的元件,同时,每一步焊接完成后,都有严格的验证标准。\n\n表4-3 STM32核心板焊接步骤\n\n\section{4.4 STM32核心板分步焊接}\n\n焊接前首先按照要求准备好焊接工具和材料,包括电烙铁、焊锡、镊子、松香、万用表、吸锡带等,同时也备齐STM32核心板的电子元件。\n\n1.焊接第一步\n\n焊接的元件号:U1。焊接第一步完成后的效果图如图4-2所示。\n\n图4-2 焊接第一步完成后的效果图\n\n焊接说明:拿到空的STM32核心板后,首先要使用万用表测试5V、3.3V和GND三个电源网络相互之间有没有短路。如果短路,直接更换一块新板,并检测无短路,然后参照4.5.1节(STM32F103RCT6芯片焊接方法)将准备好的STM32F103RCT6芯片焊接到U1所指示的位置。注意,STM32F103RCT6芯片的1号引脚务必与电路板上的1号引脚对应,切勿将芯片方向焊错。\n\n验证方法:使用万用表测试STM32芯片各相邻引脚之间无短路,芯片引脚与焊盘之间没有虚焊。由于STM32芯片的绝大多数引脚都被引到排针上,因此,测试相邻引脚之间是否短路可以通过检测相对应的焊盘之间是否短路进行验证。虚焊可以通过测试芯片引脚与对应的排针上的焊盘是否短路进行验证。这一步非常关键,尽管烦琐,但是绝不能疏忽。如果这一步没有达标,则后续焊接工作将无法开展。\n\n2.焊接第二步\n\n焊接的元件号:U2、C16、D1、C17、C18、L2、C19、PWR、R9、R7、R8、J4。焊接第二步完成后的效果图如图4-3(a)所示,上电后的效果图如图4-3(b)所示。焊接说明:将上述元件号对应的元件依次焊接到电路板上。各元件焊接方法可以参照4.5节的介绍。需要强调的是,每焊接完一个元件,都用万用表测试是否有短路现象,即测试5V、3.3V和GND三个网络相互之间是否短路。此外,二极管(D1)和发光二极管(PWR)都是有方向的,切莫将方向焊反,通信-下载模块接口(J4)的缺口应朝外。\n\n图4-3 焊接第二步完成后的效果图\n\n验证方法:在上电之前,首先检查5V、3.3V和GND三个网络相互之间是否短路。确认没有短路,再使用通信-下载模块对STM32核心板供电。供电后,使用万用表的电压挡检测5V和3.3V测试点的电压是否正常。STM32核心板的电源指示灯(PWR)应为红色点亮状态。\n\n3.焊接第三步\n\n焊接的元件号:R6、R14、R15、R20、R21、LED1、LED2、Y1、C11、C12、L1、RST、C13、R13。焊接第三步完成后的效果图如图4-4(a)所示,上电后的效果图如图4-4(b)所示。\n\n图4-4 焊接第三步完成后的效果图\n\n焊接说明:将上述元件号对应的元件依次焊接到电路板上。各元件的焊接方法可以参照4.5节的介绍。每焊接完一个元件,都用万用表测试是否有短路现象,即测试5V、3.3V和GND三个网络相互之间有没有短路。此外,发光二极管(LED1、LED2)是有方向的,切莫将方向焊反。\n\n验证方法:在上电之前,首先检查5V、3.3V和GND三个网络相互之间是否短路。确认没有发生短路,再使用通信-下载模块对STM32核心板供电。供电后,使用万用表的电压挡检测5V和3.3V的测试点的电压是否正常,STM32核心板的电源指示灯(PWR)应为红色点亮状态。然后,使用mcuisp软件将STM32KeilPrj.hex下载到STM32芯片。正常状态是程序下载后,电路板上的蓝灯和绿灯交替闪烁,串口能正常向计算机发送数据。下载程序和查看串口发送数据的方法可以参照3.4节的介绍。\n\n4.焊接第四步\n\n焊接的元件号:C1、C2、C3、C4、C5、C6、C7、C14、C15、Y2、R16、R17、R18、R19、J7。焊接第四步完成后的效果图如图4-5(a)所示,上电后的效果图如图4-6(b)所示。\n\n焊接说明:将上述元件号对应的元件依次焊接到电路板上。各元件的焊接方法可参见4.5节。每焊接完一个元件,都用万用表测试是否有短路现象,即测试5V、3.3V和GND三个网络相互之间是否短路。\n\n验证方法:在上电之前,首先检查5V、3.3V和GND三个网络相互之间是否短路。确认没有发生短路,再使用通信-下载模块对STM32核心板供电。供电后,使用万用表的电压挡检测5V和3.3V的测试点的电压是否正常。STM32核心板的电源指示灯(PWR)应为红色点亮状态,电路板上的蓝灯和绿灯应交替闪烁,串口能正常向计算机发送数据,OLED能够正常显示日期和时间。\n\n图4-5 焊接第四步完成后的效果图\n\n5.焊接第五步\n\n焊接的元件号:C8、C9、C10、R10、R11、R12、KEY1、KEY2、KEY3、R1、R2、R3、R4、R5、J8、J6、J1、J2、J3。焊接第五步完成后的效果图如图4-6(a)所示,上电后的效果图如图4-6(b)所示。\n\n图4-6 焊接第五步完成后的效果图\n\n焊接说明:将上述元件号对应的元件依次焊接到电路板上。各元件的焊接方法可参见4.5节。每焊接完一个元件,都用万用表测试是否有短路现象,即测试5V、3.3V和GND三个网络相互之间是否短路。注意,JTAG/SWD调试接口(J8)的缺口朝外,切莫将方向焊反。\n\n验证方法:焊接完第五步后,在上电之前,首先检查5V、3.3V和GND三个网络相互之间是否短路。确认没有出现短路现象,再使用通信-下载模块对STM32核心板供电。供电后,使用万用表的电压挡检测5V和3.3V的测试点的电压是否正常。STM32核心板的电源指示灯(PWR)应为红色点亮状态,电路板上的蓝灯和绿灯应交替闪烁,串口能正常向计算机发送数据,OLED能够正常显示日期和时间。可以将ST-Link连接到JTAG/SWD调试接口进行程序下载。注意,将ST-Link连接到JTAG/SWD调试接口进行程序下载的方法可参见3.7节。\n\n\section{4.5 元件焊接方法详解}\n\nSTM32核心板使用到的元件有24种,读者只需要掌握其中8类有代表性的元件的焊接方法即可,这8类元件的焊接方法几乎覆盖了所有元件的焊接方法。这8类元件包括STM32F103RCT6芯片、贴片电阻(电容)、发光二极管、肖特基二极管、低压差线性稳压电源芯片、晶振、贴片轻触开关、直插元件。\n\n如果按封装来分,24种元件还可以分为两类:直插元件和贴片元件。STM32核心板上的绝大多数元件都是贴片元件,只有不得已才使用直插元件。这是因为贴片元件相对于直插元件主要具有以下优点:(1)贴片元件体积小、重量轻,容易保存和邮寄,易于自动化加工;(2)贴片元件比直插元件容易焊接和拆卸;(3)贴片元件的引入大大提高了电路的稳定性和可靠性,对于生产来说也就是提高了产品的良率。因此,STM32核心板上凡是能使用贴片封装的,通常不会使用直插元件。同时,也建议读者在后续进行电路设计时尽可能选用贴片元件。\n\nSTM32核心板上最难焊接的当属封装为LQFP64的STM32F103RCT6芯片。对于刚刚接触焊接的人来说,引脚密集的芯片会让人感到头痛,尤其是这种LQFP封装的芯片,因为这种芯片的相邻引脚间距常常只有0.5mm或0.8mm。实际上,只要掌握了焊接技巧,这种芯片相对于以往的直插元件(如DIP40)焊接起来会更加简单、容易。\n\n对于焊接贴片元件来说,元件的固定非常重要。有两种常用的元件固定方法,单脚固定法和多脚固定法。像电阻、电容、二极管和轻触开关等引脚数为2~5个的元件常常采用单脚固定法。而多引脚且引脚密集的元件(如各种芯片)则建议采用多脚固定法。此外,焊接时要注意控制时间,不能太长也不能太短,一般在1~4s内完成焊接。时间过长容易损坏元件,时间太短则焊锡不能充分熔化,造成焊点不光滑、有毛刺、不牢固,也可能出现虚焊现象。\n\n焊接STM32F103RCT6芯片所采用的就是多脚固定法。下面详细介绍如何焊接STM32F103RCT6芯片。\n\n(1)往STM32F103RCT6芯片封装的所有焊盘上涂一层薄薄的锡,如图4-7所示。\n\n(2)将STM32F103RCT6芯片放置在STM32电路板的U1位置,如图4-8所示,在放置时务必确保芯片上的圆点与电路板上丝印的圆点同向,而且放置时芯片的引脚要与电路板上的焊盘一一对齐,这两点非常重要。芯片放置好后用镊子或手指轻轻压住以防芯片移动。\n\n图4-7 往STM32F103RCT6芯片引脚上涂上焊锡效果图\n\n图4-8 放置STM32F103RCT6芯片\n\n(3)用电烙铁的斜刀口轻压一边的引脚,把锡熔掉从而将引脚和焊盘焊在一起,如图4-9所示。要注意在焊接第一个边的时候,务必将芯片紧紧压住以防止芯片移动。再以同样的方法焊接其余三边的引脚。\n\n(4)STM32F103RCT6芯片焊完之后,还有很重要的一步,就是用万用表检测64个引脚之间是否存在短路,以及每个引脚是否与对应的焊盘虚焊。短路主要是由于相邻引脚之间的锡渣把引脚连在一起所导致的。检测短路前,先将万用表旋到短路检测挡,然后将红、黑表笔分别放在STM32F103RCT6芯片两个相邻的引脚上,如果万用表发出蜂鸣声,则表明两个引脚短路。虚焊是由于引脚和焊盘没有焊在一起所导致的。将红、黑表笔分别放在引脚和对应的焊盘上,如果蜂鸣器不响,则说明该引脚和焊盘没有焊在一起,即虚焊,需要补锡。\n\n图4-9 焊接STM32F103RCT6的引脚\n\n(5)清除多余的焊锡。清除多余的焊锡有两种方法:吸锡带吸锡法和电烙铁吸锡法。①吸锡带吸锡法:在吸锡带上添加适量的助焊剂(松香),然后用镊子夹住吸锡带紧贴焊盘,把干净的电烙铁头放在吸锡带上,待焊锡被吸入吸锡带中时,再将电烙铁头和吸锡带同时撤离焊盘。如果吸锡带粘在了焊盘上,千万不要用力拉扯吸锡带,因为强行拉扯会导致焊盘脱落或将引脚扯歪。正确的处理方法是重新用电烙铁头加热后,再轻拉吸锡带使其顺利脱离焊盘。②电烙铁吸锡法:在需要清除焊锡的焊盘上添加适量的松香,然后用干净的电烙铁把锡渣熔解后将其一点点地吸附到电烙铁上,再用湿润的海绵把电烙铁上的锡渣擦拭干净,重复上述操作直到把多余的焊锡清除干净为止。\n\n本书中贴片电阻(电容)的焊接采用单脚固定法。下面详细说明如何焊接贴片电阻。\n\n(1)先往贴片电阻的一个焊盘上加适量的锡,如图4-10所示。\n\n图4-10 往贴片电阻的一个焊盘上加锡\n\n(2)使用电烙铁头把(1)中的锡熔掉,用镊子夹住电阻,轻轻将电阻的一个引脚推入熔解的焊锡中,时间约为3~5s,如图4-11(a)所示。然后移开电烙铁,此时电阻的一个引脚已经固定好,如图4-11(b)所示。如果电阻的位置偏了,则把锡熔掉,重新调整位置。\n\n图4-11 焊接贴片电阻的一个引脚\n\n(3)如图4-12(a)所示,用同样的方法焊接电阻的另一个引脚。注意,加锡要快,焊点要饱满、光滑、无毛刺。焊接完第二个引脚后的效果图如图4-12(b)所示。焊接完成后,测试电阻两个引脚之间是否短路,再测试电阻引脚与焊盘之间是否虚焊。\n\n图4-12 焊接贴片电阻的另一个引脚\n\n与焊接贴片电阻(电容)的方法类似,焊接发光二极管(LED)采用的也是单脚固定法。下面详细介绍如何焊接发光二极管。\n\n(1)发光二极管和电阻(电容)不同,电阻(电容)没有极性,而发光二极管有极性。首先往发光二极管的正极所在的焊盘上加适量的锡,如图4-13所示。\n\n(2)使用电烙铁头把(1)中的锡熔掉,用镊子夹住发光二极管,轻轻将发光二极管的正极(绿色的一端为负极,非绿色一端为正极)引脚推入熔解的焊锡中,时间约为3~5s,然后移开电烙铁,此时发光二极管的正极引脚已经固定好,如图4-14所示。需要注意的是,电烙铁头不可碰及贴片LED灯珠胶体,以免高温损坏LED灯珠。\n\n图4-13 往发光二极管正极所在焊盘上加锡\n\n图4-14 焊接发光二极管的正极引脚\n\n(3)用同样的方法焊接发光二极管的负极引脚,如图4-15所示。焊接完后检查发光二极管的方向是否正确,并测试是否存在短路和虚焊现象。\n\n图4-15 焊接发光二极管的负极引脚\n\n焊接肖特基二极管(SS210)仍采用单脚固定法,在焊接时也要注意极性。下面详细介绍如何焊接肖特基二极管(SS210)。\n\n(1)肖特基二极管也有极性。首先往肖特基二极管的负极所在的焊盘上加适量的锡,如图4-16所示。\n\n(2)使用电烙铁头把(1)中的锡熔掉,用镊子夹住肖特基二极管,轻轻将负极(有竖向线条的一端为负极)引脚推入熔解的焊锡中,时间约为3~5s,然后移开电烙铁,此时肖特基二极管的负极引脚已经固定好,如图4-17所示。\n\n图4-16 往肖特基二极管负极所在焊盘上加锡\n\n图4-17 焊接肖特基二极管的负极引脚\n\n(3)用同样的方法焊接正极,如图4-18所示。焊接完后检查肖特基二极管的方向是否正确,并测试是否存在短路和虚焊现象。\n\n图4-18 焊接肖特基二极管的正极引脚\n\nSTM32核心板上的低压差线性稳压芯片(AMS1117)有4个引脚,焊接采用的同样是单脚固定法。下面详细介绍焊接低压差线性稳压芯片(AMS1117)的方法。\n\n(1)先往低压差线性稳压芯片(AMS1117)的最大引脚所对应的焊盘上加适量的锡,再用镊子夹住芯片,轻轻将最大引脚推入熔解的焊锡中,时间约为3~5s,然后移开电烙铁,此时芯片最大的引脚已经固定好,如图4-19所示。\n\n图4-19 焊接低压差线性稳压芯片的最大引脚\n\n(2)向其余3个引脚分别加锡,如图4-20所示。焊接完后测试是否存在短路和虚焊现象。\n\n图4-20 焊接低压差线性稳压芯片的其余引脚\n\nSTM32核心板上有两个晶振,分别是尺寸大一点的8MHz晶振(Y1)和尺寸小一点的32.7568kHz晶振(Y2),这两个晶振都只有2个引脚,焊接时采用单脚固定法。由于两种晶振的焊接方式一样,下面以8MHz晶振为例介绍焊接方法。\n\n(1)先往其中一个焊盘上加适量的锡,如图4-21所示。这两个晶振都没有正负极之分。\n\n(2)使用电烙铁头把(1)中的锡熔掉,用镊子夹住晶振,轻轻将晶振的一个引脚推入熔解的焊锡中,时间约为3~5s,然后移开电烙铁,此时晶振的一个引脚已经固定好,如图4-22所示。\n\n图4-21 往焊盘上加锡\n\n图4-22 焊接晶振的一个引脚\n\n(3)用同样的方法焊接晶振的另一个引脚,如图4-23所示。焊接完后,测试晶振是否存在短路和虚焊现象。\n\n图4-23 焊接晶振的另一个引脚\n\nSTM32核心板的底部有三个轻触开关(KEY1、KEY2、KEY3),这种轻触开关只有4个引脚,焊接时采用单脚固定法。下面详细介绍4脚贴片轻触开关的焊接方法。\n\n(1)先往其中一个焊盘上加适量的锡,如图4-24所示。\n\n图4-24 往轻触开关其中一个引脚所在焊盘上加锡\n\n(2)如图4-25(a)所示,使用电烙铁头把(1)中的锡熔解,用镊子夹住轻触开关,轻轻将轻触开关的一个引脚推入熔解的焊锡中,时间约为3~5s,然后移开电烙铁,此时轻触开关的一个引脚已经固定好,如图4-25(b)所示。\n\n图4-25 焊接轻触开关的一个引脚\n\n(3)继续焊接其余3个引脚,如图4-26所示。焊接完后测试是否存在短路和虚焊现象。\n\n图4-26 焊接轻触开关的其余3个引脚\n\nSTM32核心板上的绝大多数元件都是贴片封装,但是也有一些元件,如排针、插座等,属于直插封装。直插封装的焊接步骤如下:按照电路板上的编号,将直插元件插入对应的位置,有方向和极性的元件要注意不要插错;直插元件定位完成后,再将电路板反过来放置,用电烙铁给其中一个焊盘上锡,焊接对应的引脚;重复以上步骤焊接其余引脚。下面介绍如何焊接2脚排针。\n\n图4-27 将2脚排针插入电路板上相应的位置\n\n(1)在STM32核心板上找到编号J6,将2脚排针插入对应的位置,注意将短针插入电路板中,如图4-27所示。\n\n(2)将电路板反过来放置,用电烙铁给其中一个焊盘加锡,如图4-28所示。\n\n(3)用同样的方法焊接另一个引脚,如图4-29所示。焊接完后测试是否存在短路和虚焊现象。\n\n图4-28 给其中一个焊盘加锡\n\n图4-29 焊接另一个引脚\n\n\section{本章任务}\n\n学习完本章后,应能熟练使用焊接工具,完成至少一块STM32核心板的焊接,并验证通过。\n\n\section{本章习题}\n\n1.焊接电路板的工具都有哪些?简述每种工具的功能。\n\n2.万用表是进行焊接和调试电路板的常用仪器,简述万用表的功能。\n\n\chapter{第5章 立创EDA介绍}\n\n立创EDA服务于广大电子工程师、教育者、学生、制造商和电子爱好者,随着商业模式的改变,2018年推出的立创EDA专业版将对中国用户保持永久免费。\n\n立创EDA的发展愿景是成为全球工程师的首选EDA工具;使命是用简约、高效的国产EDA工具,助力工程师专注创造与创新。\n\n学习目标:\n\n熟悉立创EDA。\n\n了解立创EDA的功能特点。\n\n\section{5.1 立创EDA}\n\n立创EDA是一个基于云端平台的工具,联网即用(2019年7月推出离线版)。只需在浏览器(推荐使用最新版的谷歌或火狐浏览器)地址栏中输入网址https://lceda.cn,或用搜索引擎搜索“立创EDA”就可以登录立创EDA的主页,如图5-1所示。\n\n图5-1 立创EDA主页\n\n除了浏览器访问,立创EDA还提供一个小巧的客户端。客户端下载页面如图5-2所示,安装过程为一键安装。用户可以任意地在离线和在线版本之间进行切换。\n\n立创EDA的设计操作界面简洁,操作步骤简单,将很多复杂的设计过程实现了一键操作,读者可以快速入门。\n\n图5-2 客户端下载页面\n\n\section{5.2 功能特点}\n\n云端技术的应用让立创EDA有别于传统EDA设计方式,让设计者不再局限于个人的设计,最大限度地发挥网络的优势。设计者可以在立创EDA上实现团队管理、原理图库和PCB库共享,以及在“工程广场”查找工程等一系列快捷功能。\n\n目前立创EDA有上百万个元件原理图库和PCB库,除了立创商城所售元件的库,绝大部分共享库是由用户提供的。随着立创EDA的用户数量不断增加,云端的库文件也不断更新和积累,目前在立创EDA上基本可以找到用户所需的大多数元件及其对应的元件库,不仅省去了自己制作封装的麻烦,而且大大提升了设计效率。\n\n立创EDA提供非常强大的团队管理功能。设计者通过创建一个团队,将成员加入其中并赋予权限,就可以共同对工程进行设计,并且在工程设计中实现分工协作,提高设计效率。通过团队管理的方式让团队成员对一个工程有更深入的理解,有助于工程的完善,体现团队协作的优势。在团队管理中,工程文件同样可以设置版本管理的功能,团队成员之间可以就一个项目设计多个版本,不同的团队成员可以设计不同的版本,有助于促进相互间的学习、交流和进步。\n\n团队的创建者可以对成员进行管理员的设置。在团队管理中,有以下几种管理权限。\n\n● 所有者:个人工程的所有者,对工程拥有全部的操作权限。\n\n● 管理员:可添加团队成员,拥有对工程文件的设置、编辑和开发权限。\n\n● 开发者:拥有对工程文件的编辑和设计权限。\n\n● 成员:拥有对工程文件的查看权限。\n\n工程师将自己的工程文件开源后便可以与其他用户一起交流,这是一种良性的学习方式。他人通过对自己的开源工程进行学习和研究,可能会提出一种更好的解决方案。开源的工程文件也可能会对其他工程师的工程设计有参考和借鉴作用。\n\n立创EDA提供了一个硬件开源的平台——工程广场,如图5-3所示。在工程广场上可以看到用户贡献的工程文件,如图5-4所示,用户也可以将自己的工程权限设为公开,让更多人看到自己的设计,这对个人的影响力和学习能力都有很大的帮助。在进行电路设计时,可以在工程广场上找到一些非常有价值的参考电路以及PCB布局布线的设计等,甚至可以直接将工程广场上一些实际工程项目的PCB送去打样并测试。通过工程广场,用户可以学习他人的设计,有助于快速设计出自己的电路。\n\n图5-3 工程广场\n\n图5-4 用户贡献的工程文件\n\n硬件电路的开源环境需要用户共同营造,立创EDA也将创建一个专门用于硬件电路开源的论坛,以方便用户进行交流。在这里的开源项目将更具有使用价值和学习价值,工程师可以从中得到灵感和设计思路。\n\n立创EDA提供版本管理的功能。版本号表示一个工程项目在更新迭代中的唯一性。在团队设计中对工程项目进行版本管理,可以避免因版本不一致而导致沟通障碍或工程项目延迟的情况。版本管理在工程设计中是很重要的一部分,需要严格执行。\n\n\section{本章任务}\n\n完成本章的学习后,熟悉立创EDA,并了解其特点。\n\n\section{本章习题}\n\n1.常用的EDA软件有哪些?简述各种EDA软件的特点。\n\n2.简述立创EDA的发展历史和演变过程。\n\n\chapter{第6章 STM32核心板原理图设计}\n\n在电路设计与制作过程中,电路原理图设计是整个电路设计的基础。如何将STM32核心板电路通过立创EDA用工程表达方式呈现出来,使电路符合需求和规则,就是本章要完成的任务。通过本章的学习,读者将能够完成整个STM32核心板原理图的绘制,为后续进行PCB设计打下基础。\n\n学习目标:\n\n了解基于立创EDA进行原理图设计的流程。\n\n掌握基于立创EDA的STM32核心板原理图绘制方法。\n\n\section{6.1 原理图设计流程}\n\nSTM32核心板的电路原理图设计流程如图6-1所示,具体如下。\n\n图6-1 STM32核心板的电路原理图设计流程图\n\n(1)打开立创EDA,新建一个STM32核心板的PCB工程。\n\n(2)在STM32核心板的PCB工程中,新建一个STM32核心板原理图。\n\n(3)在原理图设计界面中,设置原理图设计规范。\n\n(4)在立创EDA的元件库中搜索原理图封装,或者在个人库中创建原理图封装。\n\n(5)在原理图中放置元件。\n\n(6)连线。\n\n(7)检查STM32核心板原理图。\n\n\section{6.2 创建PCB工程}\n\n首先打开立创EDA主界面,然后在工具栏中的\n\n下拉菜单中选择“新建”→“工程”命令,如图6-2所示。\n\n打开“新建工程”对话框,在“所有者”栏中选择工程的所有者,所有者可以是个人,也可以是团队。\n\n单击“创建团队”按钮创建新的团队,如图6-3所示。输入团队名称,团队路径默认为系统分配的路径,添加团队简介和成员之后,单击“创建团队”按钮即可完成。在所有者中选择团队,即表示该工程属于团队。在本书中,STM32核心板的工程所有者为个人。\n\n图6-2 新建PCB工程步骤一\n\n图6-3 创建团队\n\n在“标题”框中输入工程名称,即STM32CoreBoard。路径默认为系统分配的路径。可以在“描述”框中添加工程的相关描述。“可见性”根据需要选择“私有”或“公开”,如图6-4所示。最后,单击“保存”按钮即可完成PCB工程的创建。\n\n图6-4 新建PCB工程步骤二\n\n在主界面左侧的“工程”标签页中可以看到新建的STM32CoreBoard工程,单击选中工程,然后单击鼠标右键,在右键快捷菜单中选择“版本”→“版本管理”命令,如图6-5所示。\n\n图6-5 工程版本管理步骤一\n\n接下来,编辑工程版本,如图6-6所示,单击\n\n按钮。\n\n图6-6 工程版本管理步骤二\n\n在弹出的“编辑版本记录”对话框中,将版本名称修改为STM32CoreBoard-V1.0.0,还可添加相关描述,如图6-7所示。然后,单击“修改”按钮。此时可以看到版本名称已被修改,如图6-8所示,关闭对话框。按照一定的规则命名保存所有版本,可避免发生版本丢失或混淆等情况,还可以快速准确地查找任意版本,有助于工程的版本管理。\n\n图6-7 工程版本管理步骤三\n\n图6-8 工程版本管理步骤四\n\n然后在主界面中,右键单击“工程”标签页,在弹出的快捷菜单中选择“重新加载”命令,如图6-9所示,即可看到工程版本已被更新,如图6-10所示。\n\n图6-9 工程版本更新\n\n图6-10 工程版本更新完成\n\n下面简要介绍工程文件夹和工程的命名规范。三种常用的命名方式是骆驼命名法(Camel-Case)、帕斯卡命名法(Pascal-Case)和匈牙利命名法(Hungarian)。本书只使用帕斯卡命名法。帕斯卡命名法的规则是每个单词的首字母大写,其余字母小写,如DisplayInfo、PrintStuName。\n\n例如,在本书中,PCB工程命名为“STM32CoreBoard”就是帕斯卡命名法,表示STM32 Core Board,即STM32核心板。但是由于PCB工程往往都是迭代的,绝大多数PCB工程的完成都要经历若干天、若干版本,最终才能获得稳定版本,因此,本书建议工程文件夹的命名格式为“工程名+版本号+日期+字母版本号(可选)”,如文件夹STM32Core Board-V1.0.0-20171215表示工程名为STM32CoreBoard,修改日期为2017年12月15日,版本为V1.0.0;又如文件夹STM32CoreBoard-V1.0.0-20171215B表示2017年12月15日修改了三次,第一次修改后的名为STM32CoreBoard-V1.0.0-20171215,第二次为STM32Core Board-V1.0.0-20171215A;再如文件夹STM32CoreBoard-V1.0.2-20171215C表示已打样三次,第一次为V1.0.0,第二次为V1.0.1,第三次为V1.0.2。\n\n简单总结如下:工程文件夹的命名由工程名、版本号、日期和字母版本号(可选)组成。其中“工程名”按照帕斯卡命名法进行命名。“版本号”从V1.0.0开始,每次打样后版本号加1。PCB稳定后的发布版本只保留前两位,如V1.0.2版本经过测试稳定了,在PCB发布时将版本号改为V1.0。“日期”为PCB工程修改或完成的日期,如果一天内经过了若干次修改,则通过“字母版本号(可选)”进行区分。\n\n\section{6.3 新建原理图文件}\n\n如图6-11所示,选择STM32CoreBoard工程文件并单击鼠标右键,在快捷菜单中选择“新建原理图”命令,即可在STM32CoreBoard工程下新建一个原理图文件,系统默认原理图文件名为Sheet_1。\n\n图6-11 新建原理图文件\n\n按快捷键Ctrl+S保存原理图,弹出一个“信息”对话框,显示“保存成功”的信息,如图6-12所示。\n\n图6-12 保存原理图\n\n重命名原理图,如图6-13所示,选择Sheet_1并单击鼠标右键,在快捷菜单中选择“修改”命令。\n\n图6-13 重命名原理图步骤一\n\n然后在“修改文档信息”对话框中将标题名称修改为STM32CoreBoard,还可以根据需要添加相关描述,如图6-14所示。最后,单击“确定”按钮,即可看到原理图已被重命名。\n\n图6-14 重命名原理图步骤二\n\n原理图设计界面如图6-15所示。\n\n图6-15 原理图设计界面\n\n\section{6.4 原理图规范化设置}\n\n在绘制原理图之前,需要先进行规范化设置。依次设置:(1)网格和栅格;(2)画布规格;(3)Title Block。\n\n网格大小是用于标识间距和校准元件符号的线段,便于将元件摆放整齐。\n\n栅格尺寸是指光标移动时的步进距离,便于使元件与布线对齐。栅格的数值越小,元件和布线移动的步进距离就越小,也越精准。\n\nAlt键栅格是指按Alt键,然后移动元件时的步进距离。设置合适的网格大小和栅格尺寸,有助于在设计原理图时将元件摆放整齐、美观,便于连线操作。同一个项目组的各成员应统一设置,便于项目的同步和管理。下面介绍网格大小和栅格尺寸的设置。\n\n立创EDA默认的原理图网格大小和栅格尺寸都是5,这里我们设置网格大小和栅格尺寸为10,Alt键栅格保持默认值5,如图6-16所示。\n\n图6-16 设置网格和栅格\n\n注意,在设计原理图时建议一直开启“吸附”功能。如果元件摆放和布线是在关闭“吸附”状态下操作的,当再次开启“吸附”功能后,则原有的元素(包括元件、布线等)将很难对齐栅格,强行对齐后可能会使原理图变得不美观(如走线倾斜等),甚至可能出现布线与元件引脚虚连的情况。因为关闭“吸附”功能后,元件和布线可以任意移动而不受栅格限制,所以一般在非电气连接绘制时关闭“吸附”功能。\n\n在“画布属性”中还可以根据个人喜好设置画布背景色、网格是否可见、网格颜色和网格样式。\n\n由于STM32核心板的原理图相对简单,A4大小的纸即可容纳所有元件,因此立创EDA默认的画布规格是A4大小。也可以单击绘图工具栏中的\n\n按钮设置画布规格,如图6-17所示。\n\n图6-17 设置画布规格\n\n在Title Block中双击对应的文本可以修改标题名称、版本、公司信息、日期和作者等信息。例如,双击Sheet_1,在弹出的文本框中输入STM32CoreBoard,如图6-18所示,即可完成TITLE的修改。将REV版本号修改为V1.0.0;Sheet和Date保持系统默认设置;在Drawn By处输入作者,设置完成后如图6-19所示。\n\n图6-18 设置Title Block\n\n图6-19 设置Title Block完成\n\n选中Title Block中的文本,在原理图设计界面右侧的“文本属性”面板中可以修改文本的字体、字体大小、颜色等属性,如图6-20所示。\n\n图6-20 修改Title Block文本属性\n\n单击绘图工具栏中的\n\n按钮,可以向Title Block导入图片。如图6-21所示,在弹出的“图片属性”对话框中,输入图片的网址,或者从本地计算机中选择一个图片文件,然后单击“确定”按钮。\n\n图6-21 导入图片\n\n例如,导入立创EDA的商标图片,如图6-22所示,图片随光标移动,在合适的位置单击即可放置图片,然后单击鼠标右键即可退出放置图片命令。\n\n图6-22 放置图片\n\n选中图片,单击并拖动图片四个角中的任意一个小圆圈,可以改变图片的大小;也可以在原理图设计界面右侧的“图片”面板中修改图片的属性,如图6-23所示。\n\n图6-23 修改图片属性\n\n\section{6.5 快捷键介绍}\n\n立创EDA提供了非常丰富的快捷键,每个快捷键的使用方法都可以通过命令“配置”→“快捷键设置”查看,如图6-24所示。\n\n图6-24 查看快捷键\n\n打开“快捷键设置”对话框,如图6-25所示,列表中包含“所有快捷键”“原理图快捷键”和“PCB快捷键”。“所有快捷键”适用于编辑器内的所有文件类型;“原理图快捷键”适用于原理图和原理图库文件;“PCB快捷键”适用于PCB和PCB库文件。\n\n图6-25 快捷键列表\n\n列表中的快捷键都是可以重配置的,单击需要修改的选项,出现输入框后直接按下要设置的按键,然后单击“保存修改”按钮,即可完成快捷键设置,如图6-26所示。例如,把“旋转所选图形”快捷键由空格键改为R键,单击“旋转所选图形”,当“快捷键”栏中出现文本输入框时,直接按R键即可自动输入,然后单击“保存修改”按钮完成重配置。\n\n图6-26 快捷键重配置\n\n\section{6.6 放置元件}\n\n在原理图设计中,存放元件的库有基础库和元件库。\n\n基础库包含一些常用的基础元件,它不支持自定义。在基础库中单击元件,然后移动指针到画布,再次单击即可放置元件。注意,从基础库中获取的元件属性信息是不完善的,只包含元件名称和封装。基础库中的元件有很多,本书所使用的STM32CoreBoard原理图只用到其中的GND标识符和VCC标识符。其余所有元件都从元件库中获取。下面以“JTAG/SWD调试接口电路”为例介绍如何从元件库中获取元件。\n\n按快捷键Shift+F,或单击原理图设计界面左侧的\n\n按钮打开“搜索库”对话框,如图6-27、图6-28所示。在“搜索库”对话框中,“搜索引擎”选择“立创EDA”,“类型”选择“原理图库”,“库别”选择“立创贴片”,然后在搜索栏中输入“10k 0603”。也可以根据附录中的元件BOM上的立创商城编号直接搜索,如直接输入“C25804”搜索对应的10kΩ、0603封装的电阻。\n\n图6-27 打开“搜索库”对话框\n\n单击\n\n按钮或按回车键,然后在搜索结果中选择元件。根据封装、阻值和制造商可以初步选择,如图6-29所示,选择标题(零件名称)为0603WAF1002T5E、制造商为UniOhm的10kΩ、0603封装的立创可贴片电阻。在“搜索库”对话框右侧可以看到所选中电阻的原理图符号、PCB封装及实物图。\n\n图6-28 搜索10k 0603电阻\n\n图6-29 选择10k 0603电阻步骤一\n\n如图6-30所示,选中元件后可以在“搜索库”对话框的下方看到所选元件的单价、立创商城编号、库存等信息,以及数据手册。单击\n\n按钮或“立创商城编号”后面的编号,可以打开对应元件的链接,即可非常方便地购买元件。需要注意的是,选择库存充足的元件可以避免后期因元件库存不足而无法购买或者需要另选元件的情况;了解元件的单价和起订量,方便估算和控制成本;数据手册就是元件的使用说明书,在原理图设计时,要关注元件的某些特性、参数、引脚定义等信息,这样才能用对、用好元件。\n\n图6-30 选择10k 0603电阻步骤二\n\n在图6-29所示对话框中,单击右下角的“放置”按钮,然后在原理图中单击即可放置所选元件,如图6-31所示。注意,元件引脚应放置在格点上,以方便连线,不建议按照图6-32 的左图放置,并且网格大小和栅格尺寸应设置为相同的数值,本书设置为10。\n\n图6-31 放置10k 0603电阻\n\n图6-32 将元件引脚放置在格点上\n\n选中并拖动元件,将其摆放在合适位置,也可按Alt键以栅格大小为步进距离移动元件。元件摆放整齐后如图6-33所示。\n\n图6-33 元件摆放整齐\n\n在“搜索库”对话框中搜索立创商城编号为C3405的简牛,如图6-34所示。\n\n图6-34 搜索C3405简牛\n\n如图6-35所示,单击选中简牛的原理图符号,在原理图设计界面右侧的“元件属性”中将编号“P1”修改为“J8”。考虑到与本书后续内容保持一致,建议读者按照本书提供的PDF版本原理图修改元件编号,这样在PCB布局时可一一对应地进行操作。待能够熟练使用立创EDA自行设计电路时,再尝试元件自动编号。\n\n图6-35 修改简牛编号\n\n“JTAG/SWD调试接口电路”的元件全部放置完成后如图6-36所示。选中元件然后按空格键(若重设置旋转的快捷键为R键,则按R键),可以旋转元件,如将编号为R5的电阻旋转90°。\n\n图6-36 JTAG/SWD调试接口电路元件\n\n一个完整的电路包括元件、电源、地和导线。因此,在“JTAG/SWD调试接口电路”中,还需要添加电源、地和导线。添加方法是,单击图6-27中的“基础库”按钮,在“电气标识符”中选择地(GND)和电源(VCC),如图6-37所示。\n\n依次将VCC或GND放置在原理图中,然后单击选中VCC,在“标识符”面板中将名称修改为3V3,如图6-38所示。\n\n“JTAG/SWD调试接口电路”中的元件、地、电源放置完成后如图6-39所示。注意,放置元件、地或电源时,不要将两个引脚直接相连,引脚之间须由导线来连接。\n\n除了放置元件,有时还需要删除元件。删除元件的方法是:选中某个元件,按Delete键即可将其删除。\n\n图6-37 电气标识符\n\n图6-38 重命名电源标识符名称\n\n图6-39 放置JTAG/SWD调试接口电路的元件、地、电源\n\n\section{6.7 连线}\n\n元件之间的电气连接主要是通过导线来实现的。导线是电路原理图中最重要、最常用的图元之一。\n\n导线是指具有电气性质,用来连接元件电气点的连线。导线上的任意一点都具有电气性质。单击电气工具中的\n\n按钮进入连线模式。将指针移动到需要连接的元件引脚上,这时会在引脚的端点处出现一个灰色的圆点,单击即可放置导线的起点,如图6-40所示。移动指针到需要连接的引脚,在引脚的端点处即出现一个灰色圆点,单击即可完成两个引脚之间的连接,此时指针仍处于连线模式,重复上述操作可继续连接其他引脚。单击鼠标右键或按Esc键,即可退出连线模式。\n\n在原理图设计中,一般都会将电源导线加粗,方法是:单击选中电源导线,此时导线显示为红色,然后在“导线”面板中将线宽修改为2,如图6-41所示。在“导线”面板中还可以设置导线的颜色、样式等。\n\n图6-40 连接元件\n\n图6-41 设置电源导线线宽\n\n“JTAG/SWD调试接口电路”导线连接完成后如图6-42所示。\n\n图6-42 JTAG/SWD调试接口电路导线连接完成\n\n“网络标签”实际上也是一个电气连接点,具有相同网络标签的导线表示是连接在一起的。使用网络标签可以避免电路中出现较长的连线,从而使电路原理图可以清晰地表示电路连接的脉络。\n\n放置网络标签的方法是:单击电气工具中的\n\n按钮,然后按Tab键,在弹出的“属性”对话框中修改网络标签名称,如图6-43所示,最后单击“确定”按钮。另一种修改网络标签名称的方法是:双击要修改的网络标签,然后在弹出的文本框中输入新的网络标签名称,如图6-44所示。\n\n图6-43 修改网络标签名称\n\n图6-44 另一种修改网络标签名称的方法\n\n网络标签与导线的连接和引脚一样,其左下角的灰色圆点要放在导线上,表示已经与导线连接上,如图6-45所示。\n\n图6-45 连接网络标签与导线\n\n修改网络标签属性的方法是:单击选中网络标签,然后在“网络标签”面板中修改网络标签的名称、颜色、字体和字体大小,如图6-46所示。本书所使用的原理图中网络标签保持默认属性。\n\n图6-46 修改网络标签属性\n\n“JTAG/SWD调试接口电路”的网络标签放置完成后如图6-47所示。\n\n图6-47 JTAG/SWD调试接口电路网络标签放置完成\n\n在“JTAG/SWD调试接口电路”中,简牛(J8)的11号引脚、17号引脚和19号引脚不需要连接任何网络或元件,这时要给悬空的引脚添加“非连接标志”。单击电气工具中的\n\n按钮,然后放置在悬空引脚的端点上,如图6-48所示。\n\n每个原理图都由若干个模块组成,在绘制原理图时,建议分模块绘制。这样绘制的优点是:(1)检查电路时可按模块逐个检查,提高了原理图设计的可靠性;(2)模块可以重用到其他工程中,且经过验证的模块可以降低工程出错的概率。因此,进行原理图设计时,最好给每个模块添加模块名称。\n\n图6-48 添加非连接标志\n\n下面介绍如何在原理图上添加“JTAG/SWD调试接口电路”模块名称。单击绘图工具中的\n\n按钮,然后按Tab键,在弹出的文本框中输入电路模块名称“JTAG/SWD调试接口电路”,如图6-49所示,单击“确定”按钮。\n\n图6-49 添加电路模块名称步骤一\n\n将文本“JTAG/SWD调试接口电路”移动到如图6-50所示的位置。\n\n单击选中电路模块名称,在“文本属性”面板中可以修改文本的颜色、字体等属性,如图6-51所示。本书所使用原理图中的电路模块名称文本属性保持默认设置。\n\n为了更好地区分各个电路模块,可将独立的模块用线框隔离开。单击绘图工具中的\n\n按钮,在电路模块外周绘制线框,绘制完成后如图6-52所示。选中线框,可在“折线”面板中设置线框的属性,如图6-53所示。\n\n图6-50 添加电路模块名称步骤二\n\n图6-51 修改JTAG/SWD调试接口电路模块名称文本属性\n\n图6-52 添加线框\n\n图6-53 设置线框属性\n\n\section{6.8 原理图检查}\n\n原理图设计完成后,需要检查原理图的电气连接特性。单击原理图设计界面左侧的“设计管理器”按钮,在元件列表中检查元件是否都有PCB封装(元件编号后面括号内为PCB封装名称),如图6-54所示。\n\n在原理图中定位元件的方法是:在元件列表中单击某一元件,系统将自动在原理图中定位该元件的位置,例如,定位电阻R1,如图6-55所示。同时,在“设计管理器”的元件列表下方显示该元件的引脚和所连接的网络,单击“元件引脚”中某一引脚即可定位该引脚,如图6-56所示。\n\n图6-54 检查元件\n\n图6-55 定位元件\n\n图6-56 定位元件引脚\n\n检查网络是否连接。如果没有连接,则列表中的网络名称前会显示\n\n,如图6-57中的J8_1,说明J8中有引脚未连接。因此,不需要连接导线的引脚要添加“非连接标志”。\n\n注意:若工程存在多页原理图,设计管理器会自动关联整个原理图的元件与网络信息;设计管理器内的文件夹不会自动刷新,需要单击\n\n按钮进行手动刷新。\n\n图6-57 检查网络\n\n\section{6.9 常见问题及解决方法}\n\n1.查找相似对象\n\n问题:\n\n如何在原理图中查找相似对象?\n\n解决方法:\n\n以查找原理图中所有的10kΩ电阻为例说明。单击选中一个10kΩ电阻,再单击鼠标右键,在右键快捷菜单中选择“查找相似对象”命令,打开“查找相似对象”对话框,在对话框中设置查找条件,如图6-58所示。设置查找对象的名称、供应商、供应商编号和封装与所选10kΩ电阻的相同,单击“查找”按钮,原理图中所有符合条件的10kΩ电阻即被标示出来。\n\n2.批量修改文本属性\n\n问题:\n\n如何在原理图中批量修改文本属性?\n\n解决方法:\n\n以批量修改元件编号的字体大小为例说明。单击选中一个元件的编号后查找相似对象,选中所有的元件编号,然后在“多对象属性”面板中修改字体大小,如图6-59所示,即可修改所有元件编号的字体大小。修改其他属性的方法类似。\n\n3.高亮显示网络或元件\n\n问题:\n\n如何在原理图中高亮显示网络或元件?\n\n解决方法:\n\n以高亮显示3V3网络为例说明。在“设计管理器”中单击选中网络列表中的3V3网络,这样3V3网络的导线都会在原理图中高亮显示。\n\n4.在原理图中将元件相对X轴或Y轴翻转\n\n问题:\n\n如何在原理图中将一个元件相对于X轴或Y轴翻转?\n\n解决方法:\n\n在原理图中单击选中待翻转的元件,然后按X键即可实现相对X轴翻转,按Y键即可实现相对Y轴翻转。\n\n图6-58 查找相似对象\n\n图6-59 批量修改文本属性\n\n\section{本章任务}\n\n完成本章的学习后,参照本书配套资料包中的PDFSchDoc目录下的 STM32CoreBoard.pdf文件,或附录中的“STM32核心板PDF版本原理图”,完成整个STM32核心板的原理图绘制。\n\n\section{本章习题}\n\n1.简述原理图设计的流程。\n\n2.简述搜索元件的方法。\n\n3.在原理图设计界面中,如何实现元件的90°翻转、垂直翻转和水平翻转?\n\n4.在原理图设计界面中,如何修改元件的属性?\n\n\chapter{第7章 STM32核心板PCB设计}\n\nPCB设计是将电路原理图变成具体的电路板的必由之路,是电路设计过程中至关重要的一步。如何将第6章设计好的STM32核心板原理图通过立创EDA转变成PCB,就是本章要讲解的内容。学习完本章,读者将可以完成整个STM32核心板PCB的布局、布线、覆铜等操作,为后续进行电路板制作做好准备。\n\n学习目标:\n\n了解使用立创EDA进行PCB设计的流程。\n\n能够熟练进行元件的布局操作。\n\n能够熟练进行PCB的布线操作。\n\n能够使用立创EDA完成STM32核心板的PCB设计。\n\n\section{7.1 PCB设计流程}\n\nSTM32核心板的PCB设计流程如图7-1所示,具体如下。\n\n图7-1 PCB设计流程图\n\n(1)在PCB工程中新建STM32核心板的PCB文件。\n\n(2)将STM32核心板的原理图导入PCB文件中。\n\n(3)在PCB设计环境中设置PCB的设计规则。\n\n(4)设计STM32核心板的边框和定位孔。\n\n(5)对PCB上的元件进行布局操作。\n\n(6)进行元件布线操作。\n\n(7)添加丝印。\n\n(8)添加电路板信息。\n\n(9)添加泪滴。\n\n(10)电路板正反面覆铜。\n\n(11)DRC规则检查。\n\n\section{7.2 新建PCB文件}\n\n新建PCB文件有两种方式:(1)原理图转PCB;(2)在PCB工程中新建PCB文件。下面介绍两种转换方式的具体操作方法。\n\n(1)原理图转PCB\n\n在原理图设计界面中,单击工具栏中的\n\n按钮,在下拉菜单中选择“原理图转PCB”命令,如图7-2所示。\n\n图7-2 原理图转PCB步骤一\n\n定义边框大小可参照7.3节的内容。通过原理图转PCB的转换方式,同时也会把元件导入PCB中,如图7-3所示。\n\n图7-3 原理图转PCB步骤二\n\n保存STM32核心板的PCB文件。单击工具栏中的\n\n(文档)按钮,在下拉菜单中选择“保存”命令,如图7-4所示。\n\n在“保存为PCB文件”对话框中,选择“保存至已有工程”,输入标题“STM32CoreBoard”,将PCB文件保存在与原理图相同的PCB工程中,即STM32CoreBoard-(SZLY)。可以根据需要添加相关描述,最后单击“保存”按钮,如图7-5所示。\n\n图7-4 保存PCB文件步骤一\n\n图7-5 保存PCB文件步骤二\n\n在STM32CoreBoard-(SZLY)工程文件夹中可以看到新建的PCB文件,如图7-6所示。\n\n图7-6 STM32核心板PCB文件\n\n(2)在PCB工程中新建PCB文件\n\n如图7-7所示,选中STM32CoreBoard工程文件夹并单击鼠标右键,在右键快捷菜单中选择“新建PCB”命令。\n\n图7-7 新建PCB文件\n\n参照7.3节来定义边框大小,然后保存PCB文件即可。通过在PCB工程中新建PCB文件的方式不会自动从原理图中将元件导入PCB中,元件的导入操作可参照7.4节。\n\n\section{7.3 定义PCB边框大小}\n\n在“新建PCB”对话框中,按照图7-8所示设置参数,单位选择mm,铜箔层为2,边框为圆角矩形,画布原点位置为(0,0),边框宽59mm,边框长109mm,圆角半径为2.9mm,最后单击“应用”按钮。\n\nPCB边框效果图如图7-9所示。\n\n如果需要重新定义PCB边框,则单击工具栏中的\n\n按钮,在下拉菜单中选择“边框设置”命令,如图7-10所示。\n\n然后,在如图7-11所示的“边框设置”对话框中,重新设置参数即可。\n\n图7-8 定义PCB边框大小\n\n图7-9 PCB边框效果图\n\n图7-10 重定义PCB边框步骤一\n\n图7-11 重定义PCB边框步骤二\n\n\section{7.4 更新PCB}\n\n在电路设计过程中,除了将元件从原理图导入新建的PCB中,还常常遇到修改或重新设计原理图的情况,同时也要将修改的内容更新到PCB中。更新PCB的方法有两种。\n\n方法一:在原理图设计界面中,单击工具栏中的\n\n按钮,在下拉菜单中选择“更新PCB”命令,如图7-12所示。\n\n图7-12 更新PCB方法一\n\n方法二:在PCB设计环境中,单击工具栏中的\n\n按钮。在弹出的“确认导入信息”对话框中,单击“应用修改”按钮,如图7-13所示。\n\n图7-13 确认导入信息\n\n将原理图信息导入PCB后的效果图如图7-14所示。\n\n图7-14 将原理图信息导入PCB后效果图\n\n\section{7.5 设计规则}\n\n为了保证电路板在后续工作过程中保持良好的性能,在PCB设计中常常需要设计规则,如线间距、线宽、不同电气节点的最小间距等。不同的PCB设计有不同的规则要求,所以在每个PCB设计项目开始之前都要设计相应的规则。下面针对STM32核心板,详细讲解需要设计的规则。学习完本节后,建议读者查阅相关文献了解其他规则。\n\n在PCB设计环境中,单击工具栏中的\n\n按钮,在下拉菜单中选择“设计规则”命令,如图7-15所示。也可在画布中单击鼠标右键,在右键快捷菜单中选择“设计规则”命令。\n\n图7-15 设计规则\n\n“设计规则”包括以下参数。\n\n规则:默认规则是Default,单击“新增”按钮,可以设计多个规则。规则支持自定义不同的名称,每个网络只能应用一个规则,每个规则可以设置不同的参数。\n\n线宽:当前规则的导线宽度。PCB中的导线宽度不能小于规则线宽。\n\n间距:当前规则的元素间距。PCB中具有不同网络的两个元素之间的间距不能小于规则间距。\n\n孔外径:当前规则的孔外径。PCB中的孔外径不能小于规则孔外径,如通孔的外径、过孔的外径、圆形多层焊盘的外径等,都要大于或等于规则孔外径。\n\n孔内径:当前规则的孔内径。PCB中的孔内径不能小于规则孔内径,如过孔的内径、圆形多层焊盘的内径等,都要大于或等于规则孔内径。\n\n线长:当前规则的导线总长度。相同网络的导线总长度不能大于规则线长,否则报错。如果输入框留空,则不限制导线长度。总长度包括导线、圆弧。\n\n下面介绍具体的设计规则方法。\n\n为网络设计规则:在网络列表中选中一个网络,然后在“设计规则”的下拉菜单中选择要设计的规则,单击“应用”按钮,那么这个网络就应用了该规则。\n\n实时设计规则检测:勾选“实时设计规则检测”功能后,在画图的过程中就会检测是否存在DRC错误,出现超出规则的错误时会直接显示高亮的×警示标识提示错误的位置。如图7-16所示,不同网络的两条导线之间的距离小于设计的规则间距,出现了×警示标识提示错误的位置。\n\n图7-16 实时设计规则检测\n\n检测元素到覆铜的距离:默认勾选“检测元素到覆铜的距离”项。如果不勾选该项,一旦移动了PCB中的元素(如元件、导线、过孔等),则必须重建覆铜,否则DRC无法检测到与覆铜短路的元素。\n\n在布线时显示DRC安全边界:当在信号层绘制导线时,未确定导线的周围会显示白色的DRC安全边界线圈。该安全边界的间距是根据DRC规则设置的间距来显示的,如图7-17所示。\n\n图7-17 布线DRC安全边界\n\n检测元素到边框的距离:勾选该项后,在其后文本框中输入检测的距离值,当元素到边框的值小于这个值时会在设计管理器中报错。\n\nSTM32核心板的设计规则如图7-18所示。设计规则的单位跟随画布属性中设置的单位,此处为mil。导线线宽最小为10mil,不同网络元素之间的最小间距为8mil,孔外径为24mil,孔内径为12mil,线长不做设置;在PCB设计过程中,都要开启“实时设计规则检测”“检测元素到铺铜的距离”和“在布线时显示DRC安全边界”功能。\n\n图7-18 STM32核心板的设计规则\n\n\section{7.6 层的设置}\n\nPCB设计经常使用到“层与元素”面板,如图7-19所示,单击\n\n按钮可以显示或隐藏对应的层;单击颜色标识区,当显示\n\n图标时,表示该层已进入编辑状态,可进行布线等操作;单击\n\n按钮可以将“层与元素”面板置顶;拖拽“层与元素”面板右下角,可以调整面板的高度与宽度。\n\n图7-19 层与元素\n\n在PCB设计环境中,切换层的快捷键如下:\n\nT:切换至顶层;\n\nB:切换至底层;\n\n1:切换至内层1;\n\n2:切换至内层2;\n\n3:切换至内层3;\n\n4:切换至内层4。\n\n通过层管理器,可以设置PCB的层数和其他参数。\n\n单击“层与元素”面板中的\n\n按钮,或单击工具栏中的\n\n按钮,然后在下拉菜单中选择“层管理器”命令,打开“层管理器”对话框,如图7-20所示。注意,层管理器中的设置仅对当前的PCB有效。\n\n图7-20 层管理器\n\n下面详细介绍“层管理器”对话框内的参数。\n\n铜箔层:立创EDA支持高达34个铜箔层。使用的铜箔层越多,PCB价格就越高。顶层和底层是默认的铜箔层,无法被删除。当要从4个铜箔层切换到2个时,需要先将内层的所有元素删除。\n\n显示:取消勾选相应的层,该层的层名则不会显示在“层与元素”面板上。注意,这里只是对层名的隐藏,如果隐藏的层有其他元素,如导线等,在导出Gerber文件时将会一起被导出。\n\n名称:层的名称,内层支持自定义名称。\n\n类型:\n\n(1)信号层:进行信号连接用的层,如顶层、底层。\n\n(2)内电层:当内层的类型是内电层时,该层默认是一个覆铜层,通过绘制导线和圆弧来分割内电区块。对于分割出的内电区块,可以分别对其设置网络,如图7-21所示。当生成Gerber文件时,绘制的导线将会产生对应宽度的间隙。该层是以负片的形式进行绘制的。注意,在绘制内电层的导线时,导线的起点和终点必须超过边框的边界线,否则内层区块无法被分割。\n\n(3)非信号层:如丝印层、机械层、文档层等。\n\n(4)其他层:只做显示用,如飞线层、孔层。\n\n图7-21 绘制内电层\n\n颜色:可以为每个层配置不同的颜色。\n\n透明度:默认的透明度为0%,数值越高,层越透明。\n\n层定义:\n\n(1)顶层/底层:PCB顶面和底面的铜箔层,用于电气连接及信号布线。\n\n(2)内层:铜箔层,用于信号走线和覆铜。\n\n(3)顶层丝印层/底层丝印层:可在丝印层上印刷文字或符号来标示元件在电路板的位置等信息。\n\n(4)顶层助焊层/底层助焊层:为贴片焊盘制造钢网用,以便于焊接。\n\n(5)顶层阻焊层/底层阻焊层:即电路板的顶层和底层盖油层,一般是盖绿油,绿油的作用是阻止不需要的焊接。该层属于负片绘制方式,当有导线或区域不需要盖绿油时,需要在对应的位置进行绘制,电路板上这些区域将不会被绿油覆盖,方便上焊锡等操作,该过程一般称为开窗。\n\n(6)边框层:即电路板形状定义层,用于定义电路板的实际大小,电路板打样厂会根据定义的外形生产电路板。\n\n(7)顶层装配层/底层装配层:元件的简化轮廓,用于产品装配、维修以及导出可打印的文档,对电路板制作无影响。\n\n(8)机械层:用于描述电路板的机械结构、标注及加工说明,仅做信息记录用。生产时默认不采用该层的形状进行制作。\n\n(9)文档层:与机械层类似,但该层仅在编辑器中可见,不会生成在Gerber文件里。\n\n(10)飞线层:显示PCB网络飞线,它不属于物理意义上的层,仅为了方便使用和设置颜色,故放在“层管理器”中进行配置。\n\n(11)孔层:与飞线层类似,也不属于物理意义上的层,只做通孔(非金属化孔)的显示和颜色配置用。\n\n(12)多层:与飞线层类似,金属化孔的显示和颜色配置。\n\n(13)错误层:与飞线层类似,用于DRC(设计规则错误)的错误标识显示和颜色配置。\n\n\section{7.7 绘制定位孔}\n\n制作好的电路板需要通过定位孔固定在结构件上。观察STM32核心板实物可以看到,电路板的4个顶角各有一个定位孔,下面详细介绍如何在PCB上绘制定位孔。\n\n单击PCB工具中的\n\n按钮,在PCB上绘制一个圆,然后单击选中该圆,在PCB设计界面右侧的“圆属性”面板中设置圆的参数,如图7-22所示,线宽为5mil,圆心坐标为(150mil,159mil),半径为63mil,可以选择是否锁定。属性设置完成后的圆如图7-23所示。\n\n图7-22 绘制定位孔步骤一\n\n图7-23 绘制定位孔步骤二\n\n继续单击选中该圆,再单击鼠标右键,在右键快捷菜单中选择“转为槽孔”命令,如图7-24所示。转成槽孔之后的圆如图7-25所示。\n\n图7-24 转为槽孔\n\n图7-25 转为槽孔后\n\n按照同样的方法绘制其余三个圆,线宽和半径分别为5mil和63mil,右上角圆的圆心坐标为(2174mil,159mil),左下角圆的圆心坐标为(150mil,4150mil),右下角圆的圆心坐标为(2174mil,4150mil);然后将3个圆都转为槽孔。4个定位孔全部绘制完成的效果图如图7-26所示。\n\n图7-26 四个定位孔绘制完成的效果图\n\n单击工具栏中的\n\n按钮,在下拉菜单中选择“照片预览”或“3D预览”命令,可以获得更真实的效果图。\n\n在设计PCB时,常常会在定位孔的外侧增加一个丝印圈,目的是提醒设计者在进行PCB布线时不要距离定位孔太近,以避免PCB打样钻孔时将布线切掉。下面以如何给左上角的定位孔添加丝印圈为例,说明具体操作方法。\n\n首先,在“层与元素”面板中选择顶层丝印层;然后,单击PCB工具中的\n\n按钮,在PCB上绘制一个圆,单击选中该圆,在PCB设计界面右侧的“圆属性”面板中设置圆的参数,如图7-27所示,线宽为5mil,圆心坐标为(150mil,159mil),半径为75mil。属性设置完成后的圆如图7-28所示。\n\n图7-27 设置定位孔丝印圈属性\n\n图7-28 定位孔丝印圈\n\n通过复制粘贴的方法绘制其余3个丝印圈,这样就无须重复绘制和设置线宽与半径,只需设置圆心坐标。右上角定位孔丝印圈的圆心坐标为(2174mil,159mil),左下角定位孔丝印圈的圆心坐标为(150mil,4150mil),右下角定位孔丝印圈的圆心坐标为(2174mil,4150mil)。\n\n\section{7.8 元件的布局}\n\n将元件按照一定的规则在PCB中摆放的过程称为布局。布局既是PCB设计过程中的难点,也是重点,布局合理,接下来的布线就会相对容易。\n\n布局一般要遵守以下原则:\n\n(1)布线最短原则。例如,集成电路(IC)的去耦电容应尽量放置在相应的VCC和GND引脚之间,且距离IC尽可能近,使之与VCC和GND之间形成的回路最短。\n\n(2)将同一功能模块集中原则。即实现同一功能的相关电路模块中的元件就近集中布局。\n\n(3)“先大后小,先难后易”原则。即重要的单元电路、核心元件应优先布局。\n\n(4)布局中应参考原理图,根据电路的主信号流向规律放置主要元件。\n\n(5)元件的排列要便于调试和维修,即小元件周围不能放置大元件,需调试的元件周围要有足够的空间。\n\n(6)同类型插件元件在X轴或Y轴方向上应朝同一方向放置。同一种类型的有极性分立元件也要尽量在X轴或Y轴方向上保持一致,以便于生产和检验。\n\n(7)布局时,位于电路板边缘的元件,离电路板边缘一般不小于2mm,如果空间允许,建议距离设置为5mm。\n\n(8)布局晶振时,应尽量靠近IC,且与晶振相连的电容要紧邻晶振。\n\n进行元件布局时,应掌握以下基本操作。\n\n(1)交叉选择。此功能用于切换原理图符号和PCB封装之间的对应位置。在原理图中选中一个元件,单击工具栏中的\n\n按钮,在下拉菜单中选择“交叉选择”命令,如图7-29所示。或者使用快捷键Shift+X,即可切换至PCB封装并高亮显示该元件的PCB封装。\n\n图7-29 交叉选择\n\n注意,在使用该功能之前,应确保PCB已经保存;如果没有打开PCB,编辑器会自动打开;若工程内含有多个PCB文件,且都尚未打开,则编辑器会自动打开第一个。\n\n(2)布局传递。在原理图中,同一模块电路中的元件一目了然,但是当原理图中的元件被更新到PCB之后,具有相同封装的元件被放置在同一列,无法区分各模块电路中的元件,为此,立创EDA提供了“布局传递”功能。布局传递是将元件在原理图中的布局位置传递到PCB中的元件PCB封装布局位置。\n\n例如,选中原理图中的“独立按键电路”模块中的所有元件,如图7-30所示。单击工具栏中的\n\n按钮,在下拉菜单中选择“布局传递”命令,如图7-31所示。或者按快捷键Ctrl+Shift+X,即可切换至PCB,编辑器将选中的元件PCB封装按照元件原理图符号在原理图中的相对位置进行摆放,如图7-32所示。\n\n图7-30 在原理图中选中一个模块电路中的所有元件\n\n图7-31 布局传递\n\n单击放置PCB封装后,指针仍为手掌形状,单击PCB封装可进行细节调整。单击鼠标右键可将指针变回箭头形状。\n\n(3)元件的复选。按下Ctrl键,同时单击元件,即可实现多个元件的复选。\n\n(4)元件的对齐。首先选中需要对齐的元件,然后单击工具栏中的\n\n按钮,在下拉菜单中选择所需的对齐操作即可实现元件的对齐摆放,如图7-33所示。\n\n(5)元件的旋转。单击选中待旋转的元件,然后单击工具栏中的\n\n按钮,在下拉菜单中选择所需的操作即可实现元件的旋转,如图7-34所示。也可选中元件,按空格键(若重设置旋转的快捷键为R键,则按R键)即可旋转元件。注意,PCB封装不支持镜像操作。\n\n图7-32 布局传递后PCB封装位置\n\n图7-33 元件对齐工具栏\n\n图7-34 元件的旋转\n\nSTM32核心板布局完成后的效果图如图7-35所示。图7-36是隐藏飞线后的布局效果图。\n\n注意,对于初学者而言,建议第一次布局时严格参照STM32核心板实物进行布局,完成第一块电路板的PCB设计后,再尝试自行布局。编号为J7的底座是用于连接OLED显示模块的,J7的坐标为(1176.9mil,-2768.9mil)。STM32F103RCT6芯片周围有4个槽孔用于固定OLED显示模块,这4个槽孔的半径均为53.6mil,坐标分别为:左上角槽孔圆心坐标(665mil,-2770mil),右上角槽孔圆心坐标(1690mil,-2770mil),左下角槽孔圆心坐标(665mil,-1668mil),右下角槽孔圆心坐标(1690mil,-1668mil)。读者可以自行选择是否在电路板上设计这4个槽孔。\n\n图7-35 STM32核心板布局完成效果图\n\n图7-36 STM32核心板布局完成效果图(隐藏飞线)\n\n\section{7.9 元件的布线}\n\n(1)关闭飞线。飞线是基于相同网络产生的,当两个焊盘的网络相同时,将会出现飞线,表示这两个焊盘可以通过导线连接。如果需要关闭某条网络的飞线(即隐藏飞线),可以在设计管理器中取消勾选该网络。关闭KEY1网络的飞线之前的效果图如图7-37所示;在设计管理器中取消勾选KEY1网络后的效果图如图7-38所示,可以看到,KEY1网络的飞线被隐藏了。基于该操作,可以在布线前将GND网络飞线隐藏,布线完成后再打开,这样可以减少飞线对布线的干扰。\n\n图7-37 关闭KEY1网络飞线之前效果图\n\n图7-38 关闭KEY1网络飞线之后效果图\n\n关闭飞线后,网络的导线将不会显示出来,而仅在导线的路径上显示网络名称。如图7-39所示是关闭KEY1网络飞线前的导线;关闭飞线后的导线被隐藏了,如图7-40所示,可以看到,只有网络名称KEY1显示在导线的路径上。\n\n图7-39 关闭KEY1网络飞线前的导线\n\n图7-40 关闭KEY1网络飞线后的导线\n\n(2)选择布线工具。单击PCB工具中的\n\n按钮,或按快捷键W,在画布上单击开始绘制,再次单击确认布线;单击鼠标右键取消布线,再次单击鼠标右键即可退出布线模式。布线时要选择正确的层,铜箔层和非铜箔层都可用该布线工具绘制导线。\n\n(3)修改导线属性。首先选择待修改属性的一段导线,然后在PCB设计环境右侧的“导线属性”面板中修改即可,如图7-41所示。\n\n图7-41 修改导线属性\n\n(4)切换布线活动层。在顶层绘制一段导线,单击确认布线,然后按B键,可以自动添加过孔,并自动切换到底层继续布线。按T键可由底层切换至顶层。\n\n(5)调节导线线宽。在布线过程中,按+、-键可以调节当前导线的线宽,线宽以2mil的幅度递增或递减。也可以按Tab键修改线宽的参数。如果在布完一段导线后,要增大下一段导线的线宽,则先按L键,再按+键。\n\n(6)移动导线线段。单击选中一段导线,拖动即可调节其位置。\n\n(7)切换布线角度。在布线过程中,按L键可以切换布线角度。布线角度有4种:90°布线、45°布线、任意角度布线和弧形布线。按空格键可以切换当前布线的角度。\n\n(8)高亮显示网络。单击选中待高亮显示的网络的其中一段导线,按H键可以高亮显示该网络的所有导线,再次按H键,可以取消高亮显示。\n\n(9)删除导线。在布线的过程中,要撤销上一段布线可以通过按Delete键实现;要删除导线的某一线段,可以按下Shift键,同时双击要删除的线段,或者选中要该线段,然后单击鼠标右键,在右键快捷菜单中选择“删除线段”命令即可实现。\n\n(10)布线冲突。在PCB设计过程中,需要打开布线冲突中的“阻挡”功能,如图7-42所示。这样在布线过程中,不同网络之间将不相连。\n\n(11)布线吸附。在布线过程中,需要打开“吸附”功能,如图7-43所示,这样布线时导线将自动吸附在焊盘的中心位置。\n\n图7-42 布线冲突\n\n图7-43 布线吸附\n\n布线时应注意以下事项。\n\n(1)电源主干线原则上要加粗(尤其是电路板的电源输入/输出线)。对于STM32核心板,电源输出线包括“OLED显示屏接口电路”模块电源线、“JTAG/SWD调试接口电路”模块电源线和“外扩引脚”电源线。建议将STM32核心板的电源线线宽设置为30mil,如图7-44 所示。可以看到,图中还有一些电源线未加粗,这是因为这些电源线并非电源主干线。\n\n图7-44 电源主干线布线示意图\n\n从严格意义上讲,导线上能够承载的电流大小取决于线宽、线厚及容许温升。在25℃时,对于铜厚为35μm的导线,10mil(0.25mm)线宽能够承载0.65A电流,40mil(1mm)线宽能够承载2.3A电流,80mil(2mm)线宽能够承载4A电流。温度越高,导线能够承载的电流越小。因此保守考虑,在实际布线中,如果导线上需要承载0.25A电流,则应将线宽设置为10mil;如果需要承载1A电流,则应将线宽设置为40mil,如果需要承载2A电流,则应将线宽设置为80mil,依次类推。\n\n在PCB设计和打样中,常用OZ(盎司)作为铜皮厚度(简称铜厚)的单位,1OZ铜厚定义为1平方英寸面积内铜箔的重量为1盎司,对应的物理厚度为35μm。PCB打样厂使用最多的板材规格为1OZ铜厚。\n\n(2)PCB布线不要距离定位孔和电路板边框太近,否则在进行PCB钻孔加工时,导线很容易被切掉一部分甚至被切断。图7-45所示的布线(JTRST网络)与定位孔之间的距离适中,而图7-46所示的布线(JTRST网络)与定位孔之间的距离太近。\n\n图7-45 布线与定位孔之间的距离适中\n\n图7-46 布线距离定位孔太近\n\n(3)同一层禁止90°拐角布线(见图7-47),不同层之间允许过孔90°布线(见图7-48)。此外,布线时尽可能遵守一层水平布线、另一层垂直布线的原则。\n\n图7-47 同一层90°拐角布线(禁止)\n\n图7-48 不同层之间过孔90°布线(允许)\n\n(4)高频信号线,如STM32核心板上晶振电路的布线,不要加粗。建议将线宽设置为10mil,且尽可能布线在同一层。\n\n布局合理,布线就会变得顺畅。如果是第一次布线,建议读者按照下面的步骤进行操作。熟练掌握后方可按照自己的思路尝试布线。实践证明,每多布一次线,布线水平就会有所提升,尤其是前几次尤为明显。由此可见,掌握PCB设计的诀窍很简单,就是反复多练。STM32的布线可分为以下七步。\n\n第一步:从STM32F103RCT6的部分引脚引出连线到排针,如图7-49所示,引出的引脚不包括以下引脚:通信-下载模块接口电路的2个引脚PA9(USART1_TX)和PA10(USART1_RX),JTAG/SWD调试接口电路的5个引脚PA13(JTMS)、PA14(JTCK)、PA15(JTDI)、PB3(JTDO)、PB4(JTRST),OLED显示屏接口电路的4个引脚PB12(OLED_CS)、PB13(OLED_SCK)、PB14(OLED_RES)、PB15(OLED_DIN),LED电路的2个引脚LED1(PC4)、LED2(PC5)。\n\n图7-49 STM32F103RCT6部分引脚到排针的布线\n\n第二步:电源线布线,主要针对电源转换电路,以及其余模块的电源线部分,如图7-50所示。\n\n图7-50 电源线布线\n\n第三步:独立按键电路模块的布线,如图7-51所示。\n\n图7-51 独立按键模块布线\n\n第四步:JTAG/SWD调试接口电路和通信-下载模块接口电路的布线,如图7-52所示。\n\n图7-52 JTAG/SWD调试接口电路和通信-下载模块接口电路的布线\n\n第五步:LED电路和晶振电路的布线,如图7-53所示。\n\n图7-53 LED电路和晶振电路布线\n\n第六步:OLED显示屏接口电路的布线,如图7-54所示。\n\n图7-54 OLED显示屏接口电路布线\n\n第七步:GND(地)网络布线,如图7-55所示,建议将GND网络的线宽设置为30mil。注意,由于绝大多数双面电路板的覆铜网络都是GND网络,因此有的工程师在布线时习惯不对GND网络进行布线,而是依赖覆铜,但是本书建议对所有网络(包括GND网络)布线后再进行覆铜,这样可以避免实际操作中诸多不必要的麻烦。\n\n图7-55 GND网络布线(即完成整个电路的布线)\n\n\section{7.10 丝印}\n\n丝印是指印刷在电路板表面的图案和文字,丝印字符布置原则是“不出歧义,见缝插针,美观大方”。添加丝印就是在PCB的上下表面印刷上所需要的图案和文字等,主要是为了方便电路板的焊接、调试、安装和维修等。\n\n本节详细介绍如何在顶层丝印层和底层丝印层添加丝印。\n\n1.在顶层丝印层添加丝印\n\n在“层与元素”面板中选择“顶层丝印层”,如图7-56所示。\n\n单击PCB工具中的\n\n按钮,这时指针处会出现TEXT字符,接着按Tab键,在弹出的“属性”文本框中输入要添加的丝印文本,如图7-57所示,输入GND,然后单击“确定”按钮。\n\n图7-56 选择顶层丝印层\n\n图7-57 输入丝印文本\n\n这时指针处的TEXT文本变成GND,单击放置在PCB相应的位置上。然后,单击选中丝印GND,在PCB设计环境右侧的“文本属性”面板中修改文本的线宽、高等参数,如图7-58所示,设置丝印文本的线宽为6mil,高为45mil。\n\n图7-58 修改丝印文本属性\n\n2.在底层丝印层添加丝印\n\n在“层与元素”面板中选择“底层丝印层”,其余操作与在顶层丝印层添加丝印相似。也可在“文本属性”面板中的“层”下拉菜单中选择“底层丝印”,将顶层丝印切换为底层丝印。顶层丝印和底层丝印效果图如图7-59所示,两者呈镜像关系。\n\n图7-59 顶层丝印(左)和底层丝印(右)\n\n丝印的摆放方向应遵守“从左到右,从上到下”的原则。也就是说,如果丝印是横排的,则首字母须位于左侧,如图7-60所示;如果丝印是竖排的,则首字母须位于上方,如图7-61所示。\n\n图7-60 横排丝印\n\n图7-61 竖排丝印\n\n对于直插件(如PH座子、XH座子、简牛等),在顶层丝印层和底层丝印层均需要添加引脚名丝印,并用丝印线条将相邻的引脚名丝印隔开,这样做以便于电路板调试。\n\n由于直插件的顶层丝印和底层丝印通常是对称的,因此,添加完顶层丝印后,可以通过复制的方式添加底层丝印。以STM32核心板上的J2排针的引脚丝印为例,首先框选J2的顶层丝印,按快捷键Ctrl+C复制,再按快捷键Ctrl+V粘贴,这时指针上带有被复制的丝印;然后,在PCB设计环境右侧的“多对象属性”面板中的“层”下拉菜单中选择“底层丝印”,如图7-62所示,指针上的丝印就会镜像翻转,将其放置在相应的位置即可。\n\n图7-62 修改多对象属性\n\n添加丝印线条的方法:首先将PCB工作层切换到丝印顶层,然后单击PCB工具中的\n\n按钮即可开始绘制。顶层丝印线条绘制完成后,可以在“层与元素”面板中单击“元素”标签页,然后单击“文本”前面的“眼睛”图标,隐藏文本,如图7-63所示,这样就可以很方便地只复制顶层丝印线条。用同样的方法添加底层丝印线条,全部添加完成后的效果图如图7-64所示。\n\n图7-63 隐藏文本\n\n图7-64 J2排针丝印\n\nSTM32核心板顶层丝印效果图如图7-65所示,底层丝印效果图如图7-66所示。\n\n图7-65 STM32核心板顶层丝印效果图\n\n图7-66 STM32核心板底层丝印效果图\n\n\section{7.11 添加电路板信息和信息框}\n\n为了便于产品管理,可在电路板上添加电路板名称、版本信息及信息框。下面详细介绍如何添加上述信息。\n\n在“层与元素”面板中选择“顶层丝印层”,单击PCB工具中的\n\n按钮,然后按Tab键,在弹出的“属性”文本框中输入“STM32核心板”。将文本属性设置为高100mil,线宽6mil,并放置在按键下方的位置,如图7-67所示。\n\n图7-67 电路板名称-STM32核心板\n\n添加版本信息可方便对电路板进行版本管理。由于版本信息位于电路板底层,因此,在“层与元素”面板中选择“底层丝印层”,单击PCB工具中的\n\n按钮,然后按Tab键,在弹出的“属性”文本框中输入“STM32CoreBoard-V1.0.0-20190507”,将文本属性设置为高45mil,线宽6mil,并放置在底层中间的位置。\n\n信息框主要用于对电路板进行编号,或粘贴电路板编号标签。首先,在“层与元素”面板中选择“底层丝印层”,然后,单击PCB工具中的\n\n按钮,绘制一个矩形。在“属性”对话框中设置矩形宽为20mm,高为10mm。\n\n版本信息和信息框添加完成后的效果图如图7-68所示。\n\n图7-68 添加版本信息和信息框后的效果图\n\n首先,在“层与元素”面板中选择“文档”层,添加如图7-69所示的信息,并将其放置在PCB的上方。图中的信息分别表示:PCB设计使用的是LCEDA软件,电路板版本为V1.0.0,PCB设计日期为2019年5月7日,电路板长宽尺寸为109*59mm,电路板厚度为1.6mm,电路板名称为STM32CoreBoard,电路板层数为2,板材类型为FR4,电路板颜色为蓝色,铜箔厚度为1OZ,设计者为SZLY。注意,在PCB打样时,这些信息是被忽略的。\n\n图7-69 添加PCB信息\n\n\section{7.12 泪滴}\n\n在电路板设计过程中,常常需要在导线和焊盘或过孔的连接处补泪滴,这样做有两个好处:(1)在电路板受到巨大外力的冲撞时,避免导线与焊盘、或导线与导线的断裂;(2)在PCB生产过程中,避免由蚀刻不均或过孔偏位导致的裂缝。下面介绍如何添加和删除泪滴。\n\n在PCB设计环境中,单击工具栏中的\n\n按钮,在下拉菜单中选择“泪滴”命令,如图7-70所示。\n\n图7-70 选择“泪滴”命令\n\n在“泪滴”对话框中,单击“新增”按钮,再单击“应用”按钮即可添加泪滴,如图7-71所示。\n\n图7-71 新增泪滴\n\n执行完上述操作后,可以看到电路板上的焊盘与导线的连接处增加了泪滴,如图7-72所示。\n\n图7-72 添加泪滴后的焊盘\n\n对电路重新布线时,有时需要先删除泪滴。具体方法是:在PCB设计环境中,单击工具栏中的\n\n按钮,在下拉菜单中选择“泪滴”命令,然后在“泪滴”对话框中单击“移除”按钮,最后单击“应用”按钮即可删除泪滴,如图7-73所示。\n\n执行完上述操作后,可以看到电路板上的焊盘与导线连接处的泪滴已全部被删除,如图7-74所示。\n\n图7-73 删除泪滴\n\n图7-74 删除泪滴后的焊盘\n\n\section{7.13 覆铜}\n\n覆铜是指将电路板上没有布线的部分用固体铜填充,又称为灌铜,一般与电路的一个网络相连,多数情况是与GND网络相连。对大面积的GND或电源网络覆铜将起到屏蔽作用,可提高电路的抗干扰能力;此外,覆铜还可以提高电源效率,与地线相连的覆铜可以减小环路面积。\n\n对于STM32核心板,将覆铜网络设置为GND。以顶层覆铜为例,首先在“层与元素”面板中选择“顶层”,单击PCB工具中的\n\n按钮,或按E键开始绘制覆铜。在PCB边框外部沿着边框绘制一个比边框略大的矩形框,结束绘制时单击鼠标右键,系统将自动填充,如图7-75所示。\n\n图7-75 顶层覆铜\n\n完成顶层覆铜后,用类似的方法给底层覆铜。底层覆铜后如图7-76所示。\n\n图7-76 底层覆铜\n\n选中覆铜线框(即图7-76中外围两条虚线框),可在PCB设计环境右侧的“覆铜属性”面板中修改属性,如图7-77所示。覆铜区距离其他同层电气元素的间距为10mil;焊盘与覆铜的连接样式为“发散”;“保留孤岛”选择“否”可以去除死铜;“填充样式”选择“全填充”;如果对PCB做了修改,或者修改了覆铜属性,则可通过单击“重建覆铜区”按钮或按快捷键Shift+B重建所有覆铜区,无须重新绘制覆铜区。如果要清除所有覆铜区,可按快捷键Shift+M。\n\n图7-77 覆铜属性\n\n\section{7.14 DRC规则检查}\n\nDRC规则检查是根据设计者设置的规则对PCB设计进行检查。在PCB设计环境下,单击打开“设计管理器”,然后刷新“DRC错误”,如图7-78所示。一旦检查出PCB有违反规则的地方,错误信息将会显示在“DRC错误”目录下。\n\n图7-78 DRC规则检查\n\n\section{本章任务}\n\n完成本章的学习后,应能够参照STM32核心板实物,完成整个STM32核心板的PCB设计。\n\n\section{本章习题}\n\n1.简述PCB设计的流程。\n\n2.泪滴的作用是什么?\n\n3.覆铜的作用是什么?\n\n\chapter{第8章 创建元件库}\n\n一名高效的硬件工程师通常会按照一定的标准和规范创建自己的元件库,这就相当于为自己量身打造了一款尖兵利器,这种统一和可重用的特点使得工程师在进行硬件电路设计时能够提高效率。对于企业而言,建立属于自己的元件库更为重要,在元件库的制作及使用方面制定严格的规范,既可以约束和管理硬件工程师,又能加强产品硬件设计的规范,提升产品协同开发的效率。\n\n可见,规范化的元件库对于硬件电路的设计开发非常重要。尽管立创EDA已经提供了丰富的元件库资源,但由于元件种类众多,而且有些元件可能不包含在库中。因此,考虑到个性化的设计需求,有必要建立自己专属的既精简又实用的元件库。鉴于此,本章将以STM32核心板所使用到的元件为例,重点讲解元件库的制作。\n\n每个元件都有非常严格的标准,都与实际的某个品牌、型号一一对应,并且每个元件都有完整的元件信息(如名称、封装、编号、供应商、供应商编号、制造商、制造商料号)。这种按照严格标准制作的元件库会让整个设计变得非常简单、可靠、高效。学习完本章后,读者可参照本书提供的标准,或对其进行简单的修改,来制作自己的元件库。\n\n学习目标:\n\n掌握原理图库的创建方法以及原理图符号的制作方法。\n\n掌握PCB库的创建方法以及PCB封装的制作方法。\n\n\section{8.1 创建原理图库}\n\n原理图库由一系列元件的图形符号组成。尽管立创EDA提供了大量的原理图符号,但是在电路设计过程中,仍有一些原理图符号无法在库里找到,或者立创EDA已有的原理图符号不能满足设计者的需求。因此,设计者有必要掌握自行设计原理图符号的技能,并能够建立个人的原理图库。\n\n创建原理图库的流程(见图8-1)包括:(1)新建原理图库;(2)新建元件;(3)绘制元件符号;(4)添加引脚(设置极性);(5)添加元件属性信息;(6)添加PCB封装。如果需要在原理图库中添加不止一种元件的原理图符号,可以通过重复(2)~(6)的操作来实现。\n\n如图8-2所示,单击工具栏中的\n\n按钮,在下拉菜单中执行“新建”→“原理图库”命令,打开一个空白的库文件。\n\n图8-1 创建元件的原理图库流程\n\n图8-2 新建原理图库\n\n1.绘制元件符号\n\n首先,在“画布属性”面板中将网格大小和栅格尺寸设置为10,将“ALT键栅格”设置为5。单击绘图工具中的\n\n按钮,按Alt键,绘制如图8-3所示的电阻原理图符号的边框。\n\n图8-3 绘制电阻原理图符号边框\n\n然后,单击绘图工具中的\n\n按钮,添加电阻原理图符号的引脚,如图8-4所示。注意,引脚的端点应朝外,因为它是用于连接导线的连接点。\n\n图8-4 添加电阻原理图符号引脚\n\n接下来需要编辑引脚属性。单击选中引脚1,在“引脚属性”面板中设置“显示名字”和“显示编号”为“否”;设置引脚长度为10;因为引脚的长度改变了,所以通过设置“起点X”为-20来调整引脚的位置,如图8-5所示。\n\n以同样的方法编辑引脚2的属性。最终的电阻原理图符号外形如图8-6所示。\n\n图8-5 编辑引脚1的属性\n\n图8-6 电阻原理图符号外形\n\n2.添加属性信息\n\n在“自定义属性”面板中可以设置电阻的名称、封装和编号等信息。本书选用立创商城编号为C25804的10kΩ电阻(0603封装),其详细信息如图8-7所示。\n\n图8-7 立创商城10kΩ电阻(0603封装)的信息\n\n如图8-8所示,根据立创商城所给信息设置电阻的属性。“名称”为10kΩ(1002)±1%;“封装”为R 0603;“编号”为R?;“供应商”为“立创商城”;“供应商编号”为C25804;“制造商”为“UniOhm台湾厚声”;“制造商料号”为0603WAF1002T5E。\n\n图8-8 设置电阻属性信息\n\n添加封装的方法是:单击“自定义属性”面板中“封装”对应的文本框,打开“封装管理器”对话框,如图8-9所示,在右侧“搜索”框中用封装的关键词进行搜索,如搜索R 0603。在“库别”中选择封装库,在搜索结果中单击选中一个封装后可以查看“封装焊盘信息”,根据实际需求选择其中一个,然后单击“更新”按钮即可分配封装。更新成功后单击“取消”按钮,关闭对话框。\n\n图8-9 添加电阻封装\n\n在\n\n下拉菜单中单击“保存”命令,打开“保存为原理图库文件”对话框,单击“保存”按钮,即可保存电阻封装,如图8-10所示。\n\n图8-10 保存电阻封装\n\n至此,10kΩ电阻的原理图符号已经制作完毕,在“元件库”→“原理图库”→“个人库”中可以找到,如图8-11所示。\n\n图8-11 个人库中的10kΩ电阻原理图符号\n\n1.绘制元件符号\n\n首先,在“画布属性”面板中将网格大小和栅格尺寸均设置为10,将“ALT键栅格”设置为5。单击绘图工具中的\n\n按钮,绘制如图8-12所示的发光二极管原理图符号的边框。\n\n单击选中发光二极管的边框,在原理图库设计界面右侧的“多边形”面板中将“颜色”和“填充颜色”设置为蓝色,如图8-13所示。\n\n图8-12 绘制发光二极管原理图符号边框\n\n图8-13 设置发光二极管原理图符号边框颜色\n\n按照图8-14所示的原理图符号,单击绘图工具中的\n\n按钮绘制其余直线,再将直线和多边形的颜色设置为蓝色。\n\n然后,单击绘图工具中的\n\n按钮,给蓝色发光二极管添加引脚。注意,发光二极管的引脚有正负极之分,如图8-15所示。\n\n图8-14 蓝色发光二极管原理图符号\n\n图8-15 发光二极管极性示意图\n\n接下来编辑引脚属性。单击选中左边的引脚1,在“引脚属性”面板中设置“名称”为A(即Anode,表示正极),设置“显示名字”和“显示编号”为“否”;设置引脚长度为10,如图8-16所示。\n\n引脚2的属性如图8-17所示,“名称”为K(即Kathode,表示负极)。\n\n图8-16 编辑引脚1的属性\n\n图8-17 编辑引脚2的属性\n\n添加引脚后的蓝色发光二极管原理图符号如图8-18所示。\n\n图8-18 添加引脚后的蓝色发光二极管原理图符号\n\n2.添加属性信息\n\n在“自定义属性”面板中可以设置蓝色发光二极管的名称、封装和编号等信息。本书选用立创商城编号为C84259的蓝色发光二极管(0805封装),其详细信息如图8-19所示。\n\n图8-19 立创商城C84259蓝色发光二极管(0805封装)的信息\n\n参照给电阻添加“自定义属性”的方法,根据上述信息,设置蓝色发光二极管的属性信息,如图8-20所示。\n\n图8-20 设置蓝色发光二极管属性信息\n\n在“封装管理器”对话框中,搜索蓝色发光二极管的封装(LED 0805B),然后单击“更新”按钮分配封装,如图8-21所示。\n\n图8-21 添加蓝色发光二极管封装\n\n最后,在\n\n下拉菜单中执行“保存”命令,将蓝色发光二极管的原理图符号保存到个人库中。\n\n1.绘制元件符号\n\n除了前面介绍的原理图符号绘制方法,还可以使用“原理图库向导”快速创建原理图符号。简牛的原理图符号如图8-22所示,需要说明的是,本节制作的简牛原理图符号的引脚信息比原理图中所使用的原理图符号详细,在绘制原理图时,两种原理图符号均可采用。\n\n图8-22 简牛原理图符号\n\n首先,新建一个空白的库文件,然后,单击工具栏中的\n\n按钮,打开“原理图库向导”对话框,如图8-23所示。\n\n图8-23 “原理图库向导”对话框\n\n在“原理图库向导”对话框中输入“编号”为“J?”,“名称”为“简牛 2.54mm 2*10P 直”,“样式”选择DIP-B,并编辑引脚信息,如图8-24所示,然后单击“确定”按钮。\n\n用“原理图库向导”创建的简牛原理图符号如图8-25所示。\n\n图8-24 设置“原理图库向导”信息\n\n将网格大小和栅格尺寸设置为10,发现引脚未在格点上,单击工具栏中的\n\n按钮,在下拉菜单中选择“拖移”命令,如图8-26所示。\n\n图8-25 用“原理图库向导”创建的简牛原理图符号\n\n图8-26 选择“拖移”命令\n\n选择“拖移”命令后,指针变为手掌形状,然后框选简牛原理图符号,按住Alt键,将简牛原理图符号拖移到合适的位置,使得引脚都在格点上,如图8-27所示。单击鼠标右键,即可退出“拖移”状态。\n\n将简牛原理图符号的引脚长度修改为20,如图8-28所示。\n\n图8-27 拖移简牛原理图符号至合适位置\n\n图8-28 修改简牛原理图符号引脚长度\n\n最后,将画布原点设置在图形的中心,具体操作是:单击工具栏中的\n\n按钮,在下拉菜单中选择“画布原点”→“从图形中心格点”命令,如图8-29所示,即可将画布原点设置在图形的中心。\n\n图8-29 设置画布原点的位置\n\n2.添加属性信息\n\n本书选用立创商城编号为C3405的简牛,其详细信息如图8-30所示。\n\n图8-30 立创商城C3405简牛的信息\n\n根据上述信息,设置简牛的属性信息,如图8-31所示。\n\n图8-31 设置简牛的属性信息\n\n在“封装管理器”对话框中,搜索简牛的封装,然后单击“更新”按钮分配封装,如图8-32所示。\n\n图8-32 设置简牛封装\n\n最后,将制作好的简牛原理图符号保存到个人库中。\n\n1.绘制元件符号\n\n本节以STM32F103RCT6芯片为例,介绍如何使用“原理图库向导”中的高级功能来创建复杂的原理图符号,以及如何创建原理图库子库。创建原理图库子库的原因是,如果将一个有很多引脚的元件画在一个库文件中,会占用很大的空间,可以通过创建多个子库,再将所有子库组合起来,从而构成该元件。\n\n使用高级功能时,需要先下载原理图库创建模板表格(https://image.lceda.cn/files/Schematic-Library-Wizard-Template.xlsx),打开表格后编辑STM32F103RCT6芯片原理图符号引脚的属性和方位,如表8-1所示。用*符号作分隔可以使引脚之间产生空格。引脚按照逆时针方向放置。\n\n表8-1 STM32F103RCT6芯片原理图符号部分引脚属性和方位表\n\n续表\n\n续表\n\n然后,单击工具栏中的\n\n按钮,打开“原理图库向导”对话框,如图8-33所示,编号为U?,名称为STM32F103RCT6。复制表格中的引脚内容,将其粘贴到“引脚信息”下方的文本框中,单击“确定”按钮。\n\n图8-33 “原理图库向导”对话框\n\n使用“原理图库向导”高级功能创建的STM32F103RCT6芯片原理图符号如图8-34所示,其中只包含了部分引脚,电源和地的引脚将在子库中创建。\n\n设置引脚长度为20,调整STM32F103RCT6芯片原理图符号边框的大小,并删除左上角标志引脚1的圆点。调整后的原理图符号如图8-35所示。\n\n图8-34 使用“原理图库向导”高级功能创建的STM32F103RCT6芯片原理图符号\n\n图8-35 调整后的STM32F103RCT6芯片原理图符号\n\n2.添加属性信息\n\n本书选用立创商城编号为C8323的STM32F103RCT6芯片,其详细信息如图8-36所示。\n\n图8-36 立创商城STM32F103RCT6(C8323)芯片的信息\n\n根据上述信息,设置STM32F103RCT6芯片的属性信息,如图8-37所示。\n\n在“封装管理器”对话框中选择STM32F103RCT6芯片的封装,单击“更新”按钮分配封装,如图8-38所示。将STM32F103RCT6芯片的原理图符号保存到个人库中。\n\n图8-37 设置STM32F103RCT6芯片的属性信息\n\n图8-38 设置STM32F103RCT6芯片封装\n\n3.创建STM32F103RCT6芯片子库原理图符号\n\n执行“元件库”→“原理图库”→“个人库”命令,选中STM32F103RCT6父库,单击鼠标右键,在右键快捷菜单中选择“添加子库”命令,如图8-39所示。\n\n图8-39 添加子库STM32F103RCT6.1\n\n单击选中子库STM32F103RCT6.1,再单击“编辑”按钮,如图8-40所示。\n\n图8-40 编辑子库STM32F103RCT6.1\n\n在“自定义属性”面板中输入“编号”为U?.1,如图8-41所示。然后保存子库STM32F103RCT6.1。注意,含子库的元件只需要在父库中指定一个封装即可,如果为每一个子库指定不同的封装,那么立创EDA将保留最后一个指定的封装作为元件的封装。\n\n在“个人库”中添加子库STM32F103RCT6.2,单击“编辑”按钮,绘制子库STM32F103RCT6.2的原理图符号,如图8-42所示。将电源引脚(VDD_1、VDD_2、VDD_3、VDD_4、VDDA)的“名称颜色”和“编号颜色”设置为红色,将地引脚(VSS_1、VSS_2、VSS_3、VSS_4、VSSA)的“名称颜色”和“编号颜色”设置为黑色。保存子库STM32F103RCT6.2。\n\n图8-41 编辑子库STM32F103RCT6.1的属性\n\n图8-42 绘制子库STM32F103RCT6.2\n\n在“自定义属性”面板中设置“编号”为U?.2,如图8-43所示。再次保存子库STM32F103RCT6.2。\n\n图8-43 编辑子库STM32F103RCT6.2属性\n\n至此,STM32F103RCT6芯片的原理图符号已经制作完毕,如图8-44所示。\n\n图8-44 个人库中的STM32F103RCT6原理图符号\n\n\section{8.2 创建PCB库}\n\nPCB库(即PCB封装库)由一系列元件的PCB封装组成。元件的PCB封装在电路板上通常表现为一组焊盘、丝印层上的边框及芯片的说明文字。焊盘是PCB封装中最重要的组成部分之一,用于连接元件的引脚。丝印层上的边框和说明文字起指示作用,指明PCB封装所对应的芯片,方便电路板的焊接。尽管立创EDA软件提供了大量的PCB封装,但是,在电路板设计过程中,仍有很多PCB封装无法在库里找到,而且现有的PCB封装未必符合设计者的需求。因此,设计者有必要掌握设计PCB封装的技能,并能够建立个人的PCB库。\n\n创建元件的PCB封装流程(见图8-45)包括:(1)新建PCB库;(2)添加焊盘;(3)添加丝印;(4)添加属性;(5)保存PCB封装。\n\n单击工具栏中的\n\n按钮,在下拉菜单中执行“新建”→“PCB库”命令,如图8-46所示,即可打开一个空白的库文件。\n\n图8-45 创建元件的PCB封装流程\n\n图8-46 新建PCB库\n\n电阻(R 0603)只有两个引脚,其PCB封装形式简单。PCB封装的名称分为两部分,其中R代表Resistance(电阻),0603代表封装的尺寸为60mil×30mil。0603封装电阻的尺寸和规格如图8-47、图8-48所示。\n\n图8-47 0603封装电阻的尺寸\n\n1.添加焊盘\n\n在“画布属性”面板中设置单位为mm,单击PCB库工具中的\n\n按钮,在画布上单击,放置焊盘,如图8-49所示。\n\n单击选中焊盘1,在“焊盘属性”面板中设置层为“顶层”,形状为“矩形”,宽为1mm,高为1.1mm;将焊盘1的中心坐标设为(-0.7,0),如图8-50所示。注意,绘制PCB封装时,建议将焊盘的大小设置为比元件实际引脚面积稍大。\n\n图8-48 0603封装电阻的规格\n\n图8-49 放置R 0603封装焊盘1\n\n图8-50 设置R 0603封装焊盘1的属性\n\n设置完属性的焊盘1的效果图如图8-51所示。\n\n采用复制粘贴的方式添加R 0603封装的焊盘2。单击选中R 0603封装焊盘1,按快捷键Ctrl+C复制,再按快捷键Ctrl+V粘贴,单击放置在画布上。然后,在“焊盘属性”面板中将“编号”修改为2,中心坐标设为(0.7,0),如图8-52所示。\n\n图8-51 R 0603封装焊盘1效果图\n\n图8-52 设置R 0603封装焊盘2的属性\n\nR 0603封装的两个焊盘添加完成后的效果图如图8-53所示。\n\n图8-53 R 0603封装添加焊盘后的效果图\n\n2.添加丝印\n\n放置完焊盘后,需要添加丝印,用于表示元件外形以及标示元件在电路板上的位置。首先,在“层与元素”面板中将PCB工作层切换到“顶层丝印”层,如图8-54所示。\n\n图8-54 PCB工作层切换到“顶层丝印”层\n\n设置栅格尺寸为0.1mm,线宽为0.2mm,拐角为90°,如图8-55所示。\n\n图8-55 设置丝印参数\n\n指针的坐标位置显示在“自定义属性”下方,如图8-56所示。\n\n图8-56 指针坐标\n\n单击PCB库工具中的\n\n按钮,单击坐标(-0.4,-0.8)处开始绘制丝印,随后依次单击坐标(-1.5,-0.8)和坐标(-1.5,0.8)处,最后单击坐标(-0.4,0.8)处结束绘制,如图8-57所示。\n\n图8-57 绘制R 0603封装丝印1\n\n按照同样的方法绘制右边的丝印,如图8-58所示,两边的丝印是对称的。右边丝印的4点坐标分别是(0.4,-0.8)、(1.5,-0.8)、(1.5,0.8)、(0.4,0.8)。\n\n图8-58 绘制R 0603封装丝印2\n\n3.添加属性\n\n在“自定义属性”面板中输入封装名称为R 0603,编号为R?,如图8-59所示。\n\n图8-59 设置R 0603属性\n\n在\n\n的下拉菜单中单击“保存”命令,打开“保存为PCB库文件”对话框,单击“保存”按钮,如图8-60所示。\n\n至此,R 0603 PCB封装制作完毕。在“元件库”→“PCB库”→“个人库”中可以查看,如图8-61所示。\n\n图8-60 保存PCB库文件\n\n图8-61 个人库中的R 0603 PCB封装\n\n发光二极管的封装尺寸如图8-62所示。\n\n1.添加焊盘\n\n单击工具栏中的\n\n按钮,在下拉菜单中执行“新建”→“PCB库”命令,打开一个空白的库文件。\n\n将PCB工作层切换到“顶层”,单击PCB库工具中的\n\n按钮,再单击画布,将焊盘放置在画布上。\n\n图8-62 发光二极管的封装尺寸\n\n单击选中焊盘,在“焊盘属性”面板中设置层为“顶层”,形状为“矩形”,宽为1.3mm,高为1.4mm;将焊盘1的中心坐标设为(1.05,0),如图8-63所示。\n\n图8-63 设置发光二极管封装焊盘1属性\n\n采用复制粘贴的方式添加发光二极管封装焊盘2。在“焊盘属性”面板中修改“编号”为2,中心坐标为(-1.05,0),如图8-64所示。\n\n图8-64 设置发光二极管封装焊盘2属性\n\n发光二极管封装的两个焊盘添加完成后的效果图如图8-65所示。\n\n图8-65 发光二极管封装添加焊盘后效果图\n\n2.添加丝印\n\n放置完焊盘后,为其添加丝印。在“层与元素”面板中,将PCB工作层切换到“顶层丝印”层。\n\n设置栅格尺寸为0.1mm,线宽为0.254mm,拐角为45°,“移除回路”设置为“否”,如图8-66所示。\n\n图8-66 设置丝印参数\n\n单击PCB库工具中的\n\n按钮,单击坐标(0.4,-1.0)处开始绘制丝印,然后依次单击坐标(2.0,-1.0)和坐标(2.0,1.0)处,最后单击坐标(0.4,1.0)处结束绘制,如图8-67所示。\n\n图8-67 绘制发光二极管封装右边丝印\n\n按照同样的方法绘制左边的丝印,如图8-68所示,两边的丝印是对称的。左边丝印的4点坐标分别是(-0.4,-1.0)、(-2.0,-1.0)、(-2.0,1.0)、(-0.4,1.0)。\n\n图8-68 绘制发光二极管封装左边丝印\n\n因为发光二极管是有极性的元件,所以在丝印上应标示焊接的方向。图8-69表示发光二极管的正极焊接在焊盘1上,负极焊接在焊盘2上。\n\n图8-69 绘制发光二极管封装极性标示丝印\n\n3.添加属性\n\n在“自定义属性”面板中输入封装的名称为LED 0805,编号为LED?,如图8-70所示。\n\n图8-70 设置LED 0805属性\n\n在\n\n的下拉菜单中单击“保存”命令,打开“保存为PCB库文件”对话框,单击“保存”按钮。\n\n至此,发光二极管的PCB封装制作完毕。在“元件库”→“PCB库”→“个人库”中可以查看,如图8-71所示。\n\n图8-71 个人库中的LED 0805 PCB封装\n\n简牛的封装尺寸如图8-72所示。\n\n1.添加焊盘\n\n单击工具栏中的\n\n按钮,在下拉菜单中执行“新建”→“PCB库”命令,打开一个空白的库文件。\n\n设置网格大小为1.27mm,栅格尺寸为1.27mm;在“层与元素”面板中将PCB工作层切换到“顶层”;单击PCB库工具中的\n\n按钮,单击画布放置焊盘。\n\n单击选中焊盘,在“焊盘属性”面板中设置层为“多层”,形状为“圆形”,孔直径为1.1mm;将焊盘1的中心坐标设为(-11.43,1.27),如图8-73所示。\n\n图8-72 简牛的封装尺寸\n\n图8-73 设置简牛封装焊盘1的属性\n\n接着,继续放置其他焊盘,如图8-74所示。将编号为奇数、偶数的焊盘分别放置在一行,偶数行在上,奇数行在下,相邻两个焊盘之间的距离为2.54mm。\n\n图8-74 放置简牛封装焊盘\n\n单击PCB库工具中的\n\n按钮,然后分别单击要测量的两个焊盘的圆心,可以测量两个焊盘之间的距离,如图8-75所示。\n\n图8-75 测量两个焊盘之间的距离\n\n2.添加丝印\n\n在“层与元素”面板中将PCB工作层切换到“顶层丝印”层,准备添加丝印。\n\n参照简牛数据手册中的封装尺寸,给简牛的PCB封装添加丝印,如图8-76所示。这里将丝印线宽设为0.254mm。左下角的小三角形标示了焊盘1的方位,PCB封装的丝印要起到指示方位以防反插的作用,避免焊接元件时反向焊接。\n\n图8-76 添加简牛PCB封装丝印\n\n3.添加属性\n\n在“自定义属性”面板中输入封装的名称为HDR-IDC-2.54-2×10P,编号为J?,如图8-77所示。\n\n图8-77 设置简牛PCB封装属性\n\n在\n\n的下拉菜单中单击“保存”命令,打开“保存为PCB库文件”对话框,单击“保存”按钮。至此,简牛的PCB封装制作完毕。\n\nSTM32F103RCT6的封装尺寸和规格如图8-78、图8-79所示。\n\n图8-78 STM32F103RCT6封装尺寸\n\n图8-79 STM32F103RCT6封装规格\n\n1.添加焊盘\n\n单击工具栏中的\n\n按钮,在下拉菜单中执行“新建”→“PCB库”命令,打开一个空白的库文件。\n\n设置网格大小为0.25mm,栅格尺寸为0.25mm;在“层与元素”面板中将PCB工作层切换到“顶层”;单击PCB库工具中的\n\n按钮,单击画布放置焊盘。\n\n单击选中焊盘,在“焊盘属性”面板中设置层为“顶层”,形状为“长圆形”,旋转角度为90°;将焊盘1的中心坐标设为(-5.55,-3.75),如图8-80所示。\n\n图8-80 设置STM32F103RCT6封装焊盘1的属性\n\n接着,继续放置2~16号焊盘,相邻两个焊盘之间的距离为0.5mm,如图8-81所示。焊盘1的中心坐标为(-5.55,-3.25),焊盘16的中心坐标为(-5.55,3.75)。\n\n图8-81 放置2~16号焊盘\n\n按照同样的方法以逆时针方向添加其余的焊盘,如图8-82所示。其中,焊盘17的中心坐标为(-3.75,5.55),焊盘32的中心坐标为(3.75,5.55),焊盘33的中心坐标为(5.55,3.75),焊盘48的中心坐标为(5.55,-3.75),焊盘49的中心坐标为(3.75,-5.55),焊盘64的中心坐标为(-3.75,-5.55)。\n\n图8-82 测量焊盘之间的距离\n\n2.添加丝印\n\n在“层与元素”面板中将PCB工作层切换到“顶层丝印”层,准备添加丝印。\n\n参照STM32F103RCT6数据手册中的封装尺寸,给STM32F103RCT6的PCB封装添加丝印,如图8-83所示。将丝印线宽设为0.2mm。左上角的两个实心圆用于标示焊盘1的方位。\n\n图8-83 添加STM32F103RCT6 PCB封装丝印\n\n以大实心圆为例介绍绘制实心圆的方法:单击PCB库工具中的\n\n按钮,在适当的位置绘制一个圆圈,单击选中该圆圈,在“圆属性”面板中设置宽为0.6mm,半径为0.3mm,圆心坐标为(-3.325,-3.297),如图8-84所示。\n\n小实心圆的宽为0.3mm,半径为0.15mm,圆心坐标为(-5.901,-4.398),如图8-85所示。\n\n图8-84 大实心圆属性\n\n图8-85 小实心圆属性\n\n3.添加属性\n\n在“自定义属性”面板中,输入封装的名称为LQFP-64_10X10X05P,编号为U?,如图8-86所示。\n\n图8-86 设置STM32F103RCT6 PCB封装属性\n\n在\n\n的下拉菜单中单击“保存”命令,打开“保存为PCB库文件”对话框,单击“保存”按钮。至此,STM32F103RCT6的PCB封装制作完毕。\n\n\section{本章任务}\n\n完成本章的学习后,应能够创建原理图库和PCB库,以及制作元件的原理图符号和PCB封装。\n\n\section{本章习题}\n\n1.简述创建元件原理图符号的流程。\n\n2.简述创建元件PCB封装的流程。\n\n\chapter{第9章 导出生产文件}\n\n设计好电路板后,接下来就是制作电路板。制作电路板包括PCB打样、元件采购和焊接(或贴片)三个环节,每个环节都需要相应的生产文件。本章分别介绍各个环节所需生产文件的导出方法,为第10章制作电路板做准备。\n\n学习目标:\n\n了解生产文件的种类。\n\n了解PCB打样、元件采购及贴片加工分别需要哪些生产文件。\n\n掌握Gerber文件的导出方法。\n\n掌握BOM的导出方法。\n\n掌握丝印文件的导出方法。\n\n掌握坐标文件的导出方法。\n\n\section{9.1 生产文件的组成}\n\n生产文件一般由PCB源文件、Gerber文件和SMT文件组成,而SMT文件又由BOM、丝印文件和坐标文件组成,如图9-1所示。\n\n图9-1 生产文件架构\n\n进行PCB打样时,需要将PCB源文件或Gerber文件发送给PCB打样厂。为防止技术泄露,建议发送Gerber文件。\n\n采购元件时,需要一张BOM(Bill of Materials,即物料清单)。\n\n进行电路板贴片加工时,既可以给贴片厂发送PCB源文件和BOM,也可以发送BOM、丝印文件和坐标文件。同样,为防止技术泄露,建议选择后者。\n\n\section{9.2 Gerber文件的导出}\n\nGerber文件是一种符合EIA标准的,由GerberScientific公司定义为用于驱动光绘机的文件。该文件把PCB中的布线数据转换为光绘机用于生产1∶1高精度胶片的光绘数据,是能被光绘机处理的文件格式。PCB打样厂用Gerber文件来制作PCB。下面介绍Gerber文件的导出方法。\n\n打开STM32核心板PCB,单击工具栏中的\n\n按钮,在下拉菜单中单击“生成制造文件(Gerber)”命令,或单击工具栏中的\n\n按钮,如图9-2所示。\n\n图9-2 导出Gerber文件步骤一\n\n在弹出的“注意”对话框中单击“是,检查DRC”按钮,如图9-3所示。注意,在生成Gerber文件之前,务必进行照片预览,查看“设计管理器”中的DRC错误项,以避免生成有缺陷的Gerber文件。\n\n图9-3 导出Gerber文件步骤二\n\n在弹出的“生成制造文件(Gerber)”对话框中,单击“生成Gerber”按钮,下载Gerber文件,如图9-4所示,然后保存导出的Gerber文件压缩包。\n\n图9-4 导出Gerber文件步骤三\n\n生成的Gerber文件是一个压缩包,解压后可以看到以下文件(1)~(9)。此外,Gerber文件还包括文件(10)~(15),但本书所设计的STM32核心版不涉及这些文件。\n\n(1)Gerber_BoardOutline.GKO:边框文件。PC打样厂根据该文件来切割电路板的形状。立创EDA绘制的槽,实心填充的非镀铜通孔在生成Gerber后在边框文件进行体现。\n\n(2)Gerber_TopLayer.GTL:PCB顶层,即顶层铜箔层。\n\n(3)Gerber_BottomLayer.GBL:PCB底层,即底层铜箔层。\n\n(4)Gerber_TopSilkLayer.GTO:顶层丝印层。\n\n(5)Gerber_BottomSilkLayer.GBO:底层丝印层。\n\n(6)Gerber_TopSolderMaskLayer.GTS:顶层阻焊层。该层也称为开窗层,默认板子盖油,即该层绘制的元素所对应的顶层区域不盖油。\n\n(7)Gerber_BottomSolderMaskLayer.GBS:底层阻焊层。该层也称为开窗层,默认板子盖油,即该层绘制的元素所对应的底层区域不盖油。\n\n(8)Gerber_Drill_PTH.DRL:金属化钻孔层。该文件显示的是内壁需要金属化的钻孔位置。\n\n(9)Gerber_TopPasteMaskLayer.GTP:顶层助焊层,用于开钢网。\n\n(10)Gerber_BottomPasteMaskLayer.GBP:底层助焊层,用于开钢网。\n\n(11)Gerber_Inner1.G1,Gerber_Inner2.G1…:内层铜箔层,Inner\n\nx\n\n指第\n\nx\n\n层。\n\n(12)Gerber_Drill_NPTH.DRL:非金属化钻孔层。该文件显示的是内壁不需要金属化的钻孔位置,如通孔。\n\n(13)ReadOnly.TopAssembly:顶层装配层。仅可读取,不影响PCB制造。\n\n(14)ReadOnly.BottomAssembly:底层装配层。仅可读取,不影响PCB制造。\n\n(15)ReadOnly.Mechanical:机械层。记录PCB设计在机械层记录的信息,仅用于信息记录,生产时默认不采用该层的形状进行制造。\n\n\section{9.3 BOM的导出}\n\nBOM(Bill of Materials),即物料清单,包括元件的详细信息(如元件名称、编号、封装等)。通过BOM可查看电路板上元件的各类信息,便于设计者采购元件和焊接电路板。下面介绍BOM的导出方法。\n\n打开STM32核心板原理图,单击工具栏中的\n\n按钮,在下拉菜单中单击“导出BOM”命令,或单击工具栏中的\n\n按钮,如图9-5所示。\n\n图9-5 导出BOM步骤一\n\n在弹出的“导出BOM”对话框中,单击“导出BOM”按钮,下载BOM文件,如图9-6所示,然后保存BOM。\n\n导出的BOM打开后如图9-7所示。\n\n为了方便使用,常常需要将BOM打印出来。图9-7所示的表格并不适于打印,因此,还需要进行规范化处理,具体操作如下。\n\n图9-6 导出BOM步骤二\n\n图9-7 导出的BOM\n\n(1)为图9-7所示的表格添加页眉和页脚,页眉为“STM32CoreBoard-V1.0.0-20190507-1套”,包含了电路板名称、版本号、完成日期及物料套数。在页脚处添加页码和页数。\n\n(2)在表格的右侧增设“不焊接元件”“一审”和“二审”三列。为什么要增设“不焊接元件”列?由于有些电路板的某些元件是为了调试而增设的,还有些元件只在特定环境下才需要焊接,并且测试点也不需要焊接。因此,可以在“不焊接元件”一列中标注“NC”,表示不需要焊接。增设“一审”和“二审”列,是因为无论是自己焊接电路板,还是送去贴片厂进行贴片,都需要提前准备物料,而备料时常常会出现物料型号不对、物料封装不对、数量不足等问题,为了避免这些问题,建议每次备料时审核两次,特别是对于使用物料多的电路板。而且,每次审核后都应做记录,即在对应的“一审”或“二审”列打钩。规范的BOM如图9-8所示。\n\n图9-8 规范的BOM示意图\n\n\section{9.4 丝印文件的导出}\n\nPCB丝印文件包括顶层丝印文件和底层丝印文件,在将电路板和物料发送给贴片厂进行贴片加工时,需要将PCB丝印文件和坐标文件一起发送给贴片厂。下面详细介绍丝印文件的导出方法。\n\n打开STM32核心板PCB,单击工具栏中的\n\n按钮,在下拉菜单中执行“导出”→“PDF”命令,如图9-9所示。\n\n图9-9 导出丝印文件步骤一\n\n在弹出的“导出文档”对话框中,选择“分离层”,在“导出”列中勾选“顶层丝印”和“底层丝印”,同时在“镜像”列中勾选“底层丝印”,颜色为“白底黑图”,如图9-10所示。单击“导出”按钮,保存丝印文件压缩包。\n\n图9-10 导出丝印文件步骤二\n\n导出的STM32核心板的顶层丝印如图9-11所示,底层丝印如图9-12所示。\n\n图9-11 导出的STM32核心板顶层丝印\n\n图9-12 导出的STM32核心板底层丝印\n\n\section{9.5 坐标文件的导出}\n\n发送给贴片厂的除了PCB丝印文件,还有坐标文件。下面介绍如何导出坐标文件。\n\n打开STM32核心板PCB,单击工具栏中的\n\n按钮,在下拉菜单中单击“导出坐标文件”命令,如图9-13所示,然后保存坐标文件。\n\n图9-13 导出坐标文件\n\n导出的坐标文件为CSV格式,打开后如图9-14所示。\n\n图9-14 坐标文件\n\n\section{本章任务}\n\n完成本章的学习后,针对自己设计的STM32核心板,按照要求依次导出Gerber文件、BOM、丝印文件和坐标文件。\n\n\section{本章习题}\n\n1.生产文件都有哪些?\n\n2.PCB打样、元件采购及贴片加工分别需要哪些生产文件?\n\n3.简述Gerber文件的作用。\n\n4.简述BOM的作用。\n\n\chapter{第10章 制作电路板}\n\n电路板的制作主要包括PCB打样、元件采购和焊接三个环节。首先,将PCB源文件或Gerber文件发送给PCB打样厂制作出PCB(印制电路板);然后,购买电路板所需的元件;最后,将元件焊接到PCB上,或者将物料和PCB一起发送给贴片厂进行焊接(也称贴片)。\n\n随着近些年来电子技术的迅猛发展,无论是PCB打样厂、元件供应商,还是电路板贴片厂,如雨后春笋般涌出,不仅大幅降低了制作电路板的成本,还提升了服务品质。很多厂商已经实现了在线下单的功能,不同厂商的在线下单流程大同小异。本章以深圳嘉立创平台为例,介绍PCB打样与贴片的流程;以立创商城为例,介绍如何在网上购买元件。\n\n学习目标:\n\n掌握PCB打样的在线下单流程。\n\n掌握元件的购买流程。\n\n掌握PCB贴片的在线下单流程。\n\n掌握嘉立创下单助手的使用方法。\n\n\section{10.1 PCB打样在线下单流程}\n\n登录深圳嘉立创网站(http://www.sz-jlc.com),单击首页左上角的“进入PCB/激光钢网下单系统”按钮,如图10-1所示。\n\n图10-1 PCB打样在线下单步骤1\n\n需要先注册账户,如果已经注册,可通过输入账号和密码进入嘉立创客户自助平台。在平台界面左侧单击“PCB订单管理”按钮,然后单击“在线下单”按钮进入下单系统,如图10-2所示。\n\n图10-2 PCB打样在线下单步骤2\n\n按图10-3所示输入相应参数。若设计的STM32核心板的尺寸不是10.9cm×5.9cm,则按照实际尺寸填写。这里制作的是样板,板子数量填5,也可根据实际需求填写所需板子数量。\n\n图10-3 PCB打样在线下单步骤3\n\n接着,在“PCB工艺信息”界面中,板子厚度选择1.6,即1.6mm,其他保持默认设置,如图10-4所示。每项工艺的具体说明和注意事项可以通过单击工艺名称旁边的“?”进行查看。\n\n图10-4 PCB打样在线下单步骤4\n\n“收费高端个性化服务”和“个性化选项”部分可根据实际需求进行选择,如图10-5所示。\n\n图10-5 PCB打样在线下单步骤5\n\n如图10-6所示,根据是否希望由嘉立创进行贴片来选择,如果是自己焊接,则选“不需要”。\n\n图10-6 PCB打样在线下单步骤6\n\n在“激光钢网选项”部分选择是否需要开钢网。注意,只有将PCB送去其他贴片厂才需要开钢网。\n\n若选择需要开钢网,则接下来要选择钢网尺寸。注意,钢网的有效尺寸不能小于电路板的实际尺寸,而钢网尺寸还包括钢网外框。STM32核心板的实际尺寸为5.9cm×10.9cm,所以钢网的有效尺寸可以选择第2个,即有效面积为14.0cm×24.0cm,如图10-7所示。\n\n其他选项按照图10-8所示设置。最后,单击“确定”按钮。\n\n“请填写发票及收据信息”部分可根据实际情况填写。在“选择本订单收货地址”部分填写收货地址,以及订单联系人和技术联系人的信息。\n\n全部信息填完后,单击“提交订单”按钮,如图10-9所示。\n\n随后,在“上传文件”界面中单击“上传PCB文件/Gerber文件”按钮,如图10-10所示。既可以选择上传PCB源文件,也可以选择上传Gerber文件。这里上传立创EDA导出的Gerber文件。选择9.2节导出的“Gerber_STM32核心板_20190507103534.zip”压缩包,然后单击“打开(O)”按钮。\n\n图10-7 PCB打样在线下单步骤7\n\n图10-8 PCB打样在线下单步骤8\n\n图10-9 PCB打样在线下单步骤9\n\n图10-10 PCB打样在线下单步骤10\n\n文件上传完毕,系统会弹出如图10-11所示的界面,表示PCB打样在线下单成功。\n\n图10-11 PCB打样在线下单成功\n\n单击图10-11所示界面右侧的“返回订单列表”按钮,系统弹出如图10-12所示的订单列表,此时要等待嘉立创的工作人员审核(大概需要几十分钟)。\n\n图10-12 订单等待工作人员审核\n\n审核通过后,图10-12中的灰色“确认”按钮变成蓝色,单击蓝色的“确认”按钮进行付款。\n\n嘉立创PCB打样在线下单流程会不断更新,本书作者也会持续更新PCB打样在线下单流程,并将下载链接发布在微信公众号“卓越工程师培养系列”上,读者可随时下载。\n\n\section{10.2 元件在线购买流程}\n\n本节介绍如何在立创商城购买元件。第9章介绍了如何导出BOM。由于BOM中的Supplier Part与立创商城提供的物料编号一致,因此,读者可以直接在立创商城通过元件编号搜索对应的元件。\n\n众所周知,建立一套物料体系非常复杂,完整的物料体系应具备三个因素:(1)完善的物料库;(2)科学的元件编号;(3)持续有效的管理。这三者缺一不可,因此,无论是个人还是企业或院校,很难建立自己的物料体系,即使建立了,也很难有效地管理。随着电子商务的迅猛发展,立创商城让“拥有自己的物料体系”成为可能。这是因为,立创商城既有庞大且近乎完备的实体物料库,又对元件进行了科学的分类和编号,更重要的是有专人对整个物料库进行细致高效的管理。直接采用立创商城提供的编号,可以有效地提高电路设计和制作的效率,而且设计者无须储备物料,可做到零库存,从而大幅降低开发成本。\n\n图10-13所示的是STM32核心板BOM的一部分,完整的BOM可参见表4-2。\n\n图10-13 BOM的元件编号\n\n下面以编号为C14996的二极管SS210为例,介绍如何在立创商城购买元件。\n\n首先,打开立创商城网站(http://www.szlcsc.com),在首页的搜索框中输入元件编号“C14996”,单击“\n\n”按钮,如图10-14所示。\n\n图10-14 根据元件编号搜索元件\n\n在图10-15所示的搜索结果中,核对元件的基本信息,如元件名称、品牌、型号、封装/规格等,确认无误后,单击“我要买”按钮,加入购物车并结算,如图10-16所示。如果读者没有登录账号,单击“结算”按钮后将进入“登录/注册”页面,如图10-17所示。后续流程包括完善收件人信息、提交订单并支付,支付完成页面如图10-18所示。至此,整个订单已支付完成,等待接收包裹即可。需要注意的是,填写采购数量时要考虑损耗,建议采购数量比所需数量稍多一些;值得一提的是,立创商城的4小时闪电发货服务对读者是一个福音。\n\n图10-15 元件搜索结果\n\n图10-16 元件结算页面\n\n图10-17 “登录/注册”页面\n\n图10-18 支付完成页面\n\n当某一编号的元件在立创商城显示为缺货时,可以通过搜索该元件的关键信息购买不同型号或品牌的相似元件。例如,需要购买100nF(104)±5%50V 0603电容,如果村田品牌的暂无库存,可以用风华的替代,如图10-29所示。注意,要确保容值、封装等参数相同,否则不可以相互替代。\n\n图10-19 可替代的不同品牌元件\n\n如果没有相似元件可替代,也可以进入订货代购流程,如图10-20所示。库存不足时,加入购物车并下单后,立创商城可代为订货。如果没有找到所需要的元件,还可以提交代购需求,将由立创商城采购后交付到客户手中,如图10-21所示。\n\n图10-20 元件订货页面\n\n图10-21 元件代购页面\n\n立创商城元件购买流程会不断更新,本书作者也会持续更新立创商城元件购买流程,并将下载链接发布在微信公众号“卓越工程师培养系列”上,读者可随时下载。\n\n\section{10.3 PCB贴片在线下单流程}\n\n首先介绍什么是SMT。SMT是Surface Mount Technology(表面组装技术)的缩写,也称为表面贴装或表面安装技术,是目前电子组装行业里最流行的一种技术和工艺。它是一种将无引脚或短引线表面组装元件安装在印制电路板的表面或其他基板的表面上,通过回流焊或浸焊等方法加以焊接组装的电路装连技术。\n\n读者可能疑惑,作为电路设计人员,为什么还需要学习电路板的焊接和贴片?因为硬件电路设计人员在进行样板设计时,常常需要进行调试和验证,焊接技术作为基本技能是必须熟练掌握的。然而,为了更好地将重心放在电路的设计、调试和验证上,也可以将焊接工作交给贴片厂完成。\n\n在普通贴片厂进行电路板的贴片加工,通常都需要开机费,一般从几百到几千元不等。对于公司而言,这个费用可能不算高,但是对于初学者个人而言,这也是一笔不小的费用,毕竟刚开始设计的电路不经过两到三次修改很难达到要求。本书选择嘉立创贴片的原因是嘉立创不收取开机费,也不需要开钢网,可大大节省开发费用,并提高效率。\n\n在10.1节中,由于“SMT贴片选项”选择的是“不需要”,因此,这里需要单击图10-22中的“改为需SMT”按钮。PCB订单会重新由嘉立创工作人员审核。如果原本已设置开钢网,则需要重新返回至PCB在线下单。\n\n图10-22 改为需SMT\n\n如果在“SMT贴片选项”中选择的是“需要”,则当嘉立创工作人员审核完毕后,可直接单击“去下SMT”按钮,如图10-23所示。\n\n图10-23 去下SMT\n\n需要注意的是,嘉立创贴片目前只能贴“立创可贴片元件”,而直插元件,如排针、座子等,需要读者自己焊接。\n\n嘉立创可贴片元件清单会不断更新,本书作者也会持续更新嘉立创可贴片元件清单,读者可关注微信公众号“卓越工程师培养系列”,随时下载。\n\n嘉立创可贴片元件是经过严格筛选的,基本能够覆盖常用的元件,因此,读者在进行电路设计时,尽可能选择嘉立创可贴片元件,这样既能减少自己焊接的工作量,又能确保焊接的质量,大大提高电路设计和制作的效率。\n\n在“填写订单SMT信息”中,需选择“贴片数量”,一般样板不需要全部贴片,建议选择2片即可,如图10-24所示。\n\n接下来,系统会根据上传的是PCB源文件还是Gerber文件而显示不同的界面。\n\n图10-24 选择贴片数量\n\n如果上传的是PCB源文件,系统会自动生成BOM和坐标文件,读者无须上传BOM和坐标文件,单击“下一步”按钮即可,如图10-25所示。\n\n图10-25 SMT下单之上传PCB源文件\n\n本书选择上传Gerber文件,因此需要上传9.3节中导出的BOM和9.5节中导出坐标文件,如图10-26所示。\n\n图10-26 SMT下单之上传PCB Gerber文件\n\n系统会自动对上传的BOM进行匹配,然后列出“客户BOM清单”。如果发现上传的BOM不正确,可以重新上传,如图10-27所示,单击“变更BOM清单”按钮即可。如果上传的坐标文件不正确,也可以单击“变更坐标文件”按钮重新上传。\n\n图10-27 变更BOM或坐标文件\n\n“元器件清单”中未搜索成功的元件,都是直插元件、立创非可贴片元件或非立创元件,如图10-28所示,这些元件需要设计者自行购买,并手动焊接。\n\n图10-28 替换元件\n\n有些立创可贴片元件未被搜索成功,或可将元件替换为嘉立创已有元件时,可以通过单击“选元件”按钮替换。\n\n核对每个元件是否正确,核对无误后在对应的“核对正确”栏中打钩,如图10-29所示。\n\n图10-29 核对元件\n\n核对完毕,单击“下一步”按钮,在弹出的“需要您选择有方向(极性)零件的处理方式”对话框中,选择第一项,如图10-30所示。\n\n图10-30 SMT注意事项之有极性元件\n\n最后,单击“确认下单”按钮就可以完成SMT下单,如图10-31所示。\n\n图10-31 SMT下单完成\n\n\section{10.4 嘉立创下单助手}\n\n下载嘉立创下单助手(http://download.sz-jlc.com/jlchelper/release/3.2.2/JLCPcAssit_setup_3.2.2.zip),选择默认安装,登录界面如图10-32所示。\n\n图10-32 嘉立创下单助手登录界面\n\n需要先注册账户,如果已经注册,可直接登录。登录成功界面如图10-33所示。\n\n图10-33 登录成功界面\n\n在图10-33左侧单击“PCB订单管理”按钮,然后单击“在线下单(新)”按钮进入下单系统,如图10-34所示。\n\n图10-34 使用下单助手在线下单步骤1\n\n打开如图10-35所界面,可以选择重新上传文件,也可以使用已上传的文件进行下单操作。\n\n下单助手能够识别已上传的文件,如图10-36示,下单助手正在识别PCB文件。\n\n后续的下单流程与10.1节中介绍的流程相似,可参见10.1节完成具体操作。同样,使用嘉立创下单助手进行在线下单的流程会不断更新,本书作者也会持续更新下单流程,读者可关注微信公众号“卓越工程师培养系列”,随时下载。\n\n图10-35 使用下单助手在线下单步骤2\n\n图10-36 使用下单助手在线下单步骤3\n\n\section{本章任务}\n\n完成本章的学习后,尝试在嘉立创网站完成STM32核心板的PCB打样下单和SMT下单,并尝试在立创商城采购STM32核心板无法进行贴片的元件。建议PCB打样5块、贴片2块、元件采购3套。\n\n\section{本章习题}\n\n1.在网上查找PCB打样的流程,简述每个流程的工艺和注意事项。\n\n2.在网上查找电路板贴片的流程,简述每个流程的工艺和注意事项。\n\nSTM32核心板PDF版本原理图\n\n\chapter{反侵权盗版声明}\n\n电子工业出版社依法对本作品享有专有出版权。任何未经权利人书面许可,复制、销售或通过信息网络传播本作品的行为;歪曲、篡改、剽窃本作品的行为,均违反《中华人民共和国著作权法》,其行为人应承担相应的民事责任和行政责任,构成犯罪的,将被依法追究刑事责任。\n\n为了维护市场秩序,保护权利人的合法权益,本社将依法查处和打击侵权盗版的单位和个人。欢迎社会各界人士积极举报侵权盗版行为,本社将奖励举报有功人员,并保证举报人的信息不被泄露。\n\n举报电话:(010)88254396;(010)88258888\n\n传真:(010)88254397\n\nE-mail:dbqq@phei.com.cn\n\n通信地址:北京市海淀区万寿路173信箱\n\n电子工业出版社总编办公室\n\n邮编:100036\n\n\end{document}

% 图片引用
\begin{figure}[htbp]
\centering
\includegraphics[width=0.8\textwidth]{images/00001.jpeg}
\caption{图片 1}
\end{figure}
\newpage
\begin{figure}[htbp]
\centering
\includegraphics[width=0.8\textwidth]{images/00002.jpeg}
\caption{图片 2}
\end{figure}
\newpage
\begin{figure}[htbp]
\centering
\includegraphics[width=0.8\textwidth]{images/00003.jpeg}
\caption{图片 3}
\end{figure}
\newpage
\begin{figure}[htbp]
\centering
\includegraphics[width=0.8\textwidth]{images/00004.jpeg}
\caption{图片 4}
\end{figure}
\newpage
\begin{figure}[htbp]
\centering
\includegraphics[width=0.8\textwidth]{images/00005.jpeg}
\caption{图片 5}
\end{figure}
\newpage
\begin{figure}[htbp]
\centering
\includegraphics[width=0.8\textwidth]{images/00006.jpeg}
\caption{图片 6}
\end{figure}
\newpage
\begin{figure}[htbp]
\centering
\includegraphics[width=0.8\textwidth]{images/00007.jpeg}
\caption{图片 7}
\end{figure}
\newpage
\begin{figure}[htbp]
\centering
\includegraphics[width=0.8\textwidth]{images/00008.jpeg}
\caption{图片 8}
\end{figure}
\newpage
\begin{figure}[htbp]
\centering
\includegraphics[width=0.8\textwidth]{images/00009.jpeg}
\caption{图片 9}
\end{figure}
\newpage
\begin{figure}[htbp]
\centering
\includegraphics[width=0.8\textwidth]{images/00010.jpeg}
\caption{图片 10}
\end{figure}
\newpage
\begin{figure}[htbp]
\centering
\includegraphics[width=0.8\textwidth]{images/00011.jpeg}
\caption{图片 11}
\end{figure}
\newpage
\begin{figure}[htbp]
\centering
\includegraphics[width=0.8\textwidth]{images/00012.jpeg}
\caption{图片 12}
\end{figure}
\newpage
\begin{figure}[htbp]
\centering
\includegraphics[width=0.8\textwidth]{images/00013.jpeg}
\caption{图片 13}
\end{figure}
\newpage
\begin{figure}[htbp]
\centering
\includegraphics[width=0.8\textwidth]{images/00014.jpeg}
\caption{图片 14}
\end{figure}
\newpage
\begin{figure}[htbp]
\centering
\includegraphics[width=0.8\textwidth]{images/00015.jpeg}
\caption{图片 15}
\end{figure}
\newpage
\begin{figure}[htbp]
\centering
\includegraphics[width=0.8\textwidth]{images/00016.jpeg}
\caption{图片 16}
\end{figure}
\newpage
\begin{figure}[htbp]
\centering
\includegraphics[width=0.8\textwidth]{images/00017.jpeg}
\caption{图片 17}
\end{figure}
\newpage
\begin{figure}[htbp]
\centering
\includegraphics[width=0.8\textwidth]{images/00018.jpeg}
\caption{图片 18}
\end{figure}
\newpage
\begin{figure}[htbp]
\centering
\includegraphics[width=0.8\textwidth]{images/00019.jpeg}
\caption{图片 19}
\end{figure}
\newpage
\begin{figure}[htbp]
\centering
\includegraphics[width=0.8\textwidth]{images/00020.jpeg}
\caption{图片 20}
\end{figure}
\newpage
\begin{figure}[htbp]
\centering
\includegraphics[width=0.8\textwidth]{images/00021.jpeg}
\caption{图片 21}
\end{figure}
\newpage
\begin{figure}[htbp]
\centering
\includegraphics[width=0.8\textwidth]{images/00022.jpeg}
\caption{图片 22}
\end{figure}
\newpage
\begin{figure}[htbp]
\centering
\includegraphics[width=0.8\textwidth]{images/00023.jpeg}
\caption{图片 23}
\end{figure}
\newpage
\begin{figure}[htbp]
\centering
\includegraphics[width=0.8\textwidth]{images/00024.jpeg}
\caption{图片 24}
\end{figure}
\newpage
\begin{figure}[htbp]
\centering
\includegraphics[width=0.8\textwidth]{images/00025.jpeg}
\caption{图片 25}
\end{figure}
\newpage
\begin{figure}[htbp]
\centering
\includegraphics[width=0.8\textwidth]{images/00026.jpeg}
\caption{图片 26}
\end{figure}
\newpage
\begin{figure}[htbp]
\centering
\includegraphics[width=0.8\textwidth]{images/00027.jpeg}
\caption{图片 27}
\end{figure}
\newpage
\begin{figure}[htbp]
\centering
\includegraphics[width=0.8\textwidth]{images/00028.jpeg}
\caption{图片 28}
\end{figure}
\newpage
\begin{figure}[htbp]
\centering
\includegraphics[width=0.8\textwidth]{images/00029.jpeg}
\caption{图片 29}
\end{figure}
\newpage
\begin{figure}[htbp]
\centering
\includegraphics[width=0.8\textwidth]{images/00030.jpeg}
\caption{图片 30}
\end{figure}
\newpage
\begin{figure}[htbp]
\centering
\includegraphics[width=0.8\textwidth]{images/00031.jpeg}
\caption{图片 31}
\end{figure}
\newpage
\begin{figure}[htbp]
\centering
\includegraphics[width=0.8\textwidth]{images/00032.jpeg}
\caption{图片 32}
\end{figure}
\newpage
\begin{figure}[htbp]
\centering
\includegraphics[width=0.8\textwidth]{images/00033.jpeg}
\caption{图片 33}
\end{figure}
\newpage
\begin{figure}[htbp]
\centering
\includegraphics[width=0.8\textwidth]{images/00034.jpeg}
\caption{图片 34}
\end{figure}
\newpage
\begin{figure}[htbp]
\centering
\includegraphics[width=0.8\textwidth]{images/00035.jpeg}
\caption{图片 35}
\end{figure}
\newpage
\begin{figure}[htbp]
\centering
\includegraphics[width=0.8\textwidth]{images/00036.jpeg}
\caption{图片 36}
\end{figure}
\newpage
\begin{figure}[htbp]
\centering
\includegraphics[width=0.8\textwidth]{images/00037.jpeg}
\caption{图片 37}
\end{figure}
\newpage
\begin{figure}[htbp]
\centering
\includegraphics[width=0.8\textwidth]{images/00038.jpeg}
\caption{图片 38}
\end{figure}
\newpage
\begin{figure}[htbp]
\centering
\includegraphics[width=0.8\textwidth]{images/00039.jpeg}
\caption{图片 39}
\end{figure}
\newpage
\begin{figure}[htbp]
\centering
\includegraphics[width=0.8\textwidth]{images/00040.jpeg}
\caption{图片 40}
\end{figure}
\newpage
\begin{figure}[htbp]
\centering
\includegraphics[width=0.8\textwidth]{images/00041.jpeg}
\caption{图片 41}
\end{figure}
\newpage
\begin{figure}[htbp]
\centering
\includegraphics[width=0.8\textwidth]{images/00042.jpeg}
\caption{图片 42}
\end{figure}
\newpage
\begin{figure}[htbp]
\centering
\includegraphics[width=0.8\textwidth]{images/00043.jpeg}
\caption{图片 43}
\end{figure}
\newpage
\begin{figure}[htbp]
\centering
\includegraphics[width=0.8\textwidth]{images/00044.jpeg}
\caption{图片 44}
\end{figure}
\newpage
\begin{figure}[htbp]
\centering
\includegraphics[width=0.8\textwidth]{images/00045.jpeg}
\caption{图片 45}
\end{figure}
\newpage
\begin{figure}[htbp]
\centering
\includegraphics[width=0.8\textwidth]{images/00046.jpeg}
\caption{图片 46}
\end{figure}
\newpage
\begin{figure}[htbp]
\centering
\includegraphics[width=0.8\textwidth]{images/00047.jpeg}
\caption{图片 47}
\end{figure}
\newpage
\begin{figure}[htbp]
\centering
\includegraphics[width=0.8\textwidth]{images/00048.jpeg}
\caption{图片 48}
\end{figure}
\newpage
\begin{figure}[htbp]
\centering
\includegraphics[width=0.8\textwidth]{images/00049.jpeg}
\caption{图片 49}
\end{figure}
\newpage
\begin{figure}[htbp]
\centering
\includegraphics[width=0.8\textwidth]{images/00050.jpeg}
\caption{图片 50}
\end{figure}
\newpage
\begin{figure}[htbp]
\centering
\includegraphics[width=0.8\textwidth]{images/00051.jpeg}
\caption{图片 51}
\end{figure}
\newpage
\begin{figure}[htbp]
\centering
\includegraphics[width=0.8\textwidth]{images/00052.jpeg}
\caption{图片 52}
\end{figure}
\newpage
\begin{figure}[htbp]
\centering
\includegraphics[width=0.8\textwidth]{images/00053.jpeg}
\caption{图片 53}
\end{figure}
\newpage
\begin{figure}[htbp]
\centering
\includegraphics[width=0.8\textwidth]{images/00054.jpeg}
\caption{图片 54}
\end{figure}
\newpage
\begin{figure}[htbp]
\centering
\includegraphics[width=0.8\textwidth]{images/00055.jpeg}
\caption{图片 55}
\end{figure}
\newpage
\begin{figure}[htbp]
\centering
\includegraphics[width=0.8\textwidth]{images/00056.jpeg}
\caption{图片 56}
\end{figure}
\newpage
\begin{figure}[htbp]
\centering
\includegraphics[width=0.8\textwidth]{images/00057.jpeg}
\caption{图片 57}
\end{figure}
\newpage
\begin{figure}[htbp]
\centering
\includegraphics[width=0.8\textwidth]{images/00058.jpeg}
\caption{图片 58}
\end{figure}
\newpage
\begin{figure}[htbp]
\centering
\includegraphics[width=0.8\textwidth]{images/00059.jpeg}
\caption{图片 59}
\end{figure}
\newpage
\begin{figure}[htbp]
\centering
\includegraphics[width=0.8\textwidth]{images/00060.jpeg}
\caption{图片 60}
\end{figure}
\newpage
\begin{figure}[htbp]
\centering
\includegraphics[width=0.8\textwidth]{images/00061.jpeg}
\caption{图片 61}
\end{figure}
\newpage
\begin{figure}[htbp]
\centering
\includegraphics[width=0.8\textwidth]{images/00062.jpeg}
\caption{图片 62}
\end{figure}
\newpage
\begin{figure}[htbp]
\centering
\includegraphics[width=0.8\textwidth]{images/00063.jpeg}
\caption{图片 63}
\end{figure}
\newpage
\begin{figure}[htbp]
\centering
\includegraphics[width=0.8\textwidth]{images/00064.jpeg}
\caption{图片 64}
\end{figure}
\newpage
\begin{figure}[htbp]
\centering
\includegraphics[width=0.8\textwidth]{images/00065.jpeg}
\caption{图片 65}
\end{figure}
\newpage
\begin{figure}[htbp]
\centering
\includegraphics[width=0.8\textwidth]{images/00066.jpeg}
\caption{图片 66}
\end{figure}
\newpage
\begin{figure}[htbp]
\centering
\includegraphics[width=0.8\textwidth]{images/00067.jpeg}
\caption{图片 67}
\end{figure}
\newpage
\begin{figure}[htbp]
\centering
\includegraphics[width=0.8\textwidth]{images/00068.jpeg}
\caption{图片 68}
\end{figure}
\newpage
\begin{figure}[htbp]
\centering
\includegraphics[width=0.8\textwidth]{images/00069.jpeg}
\caption{图片 69}
\end{figure}
\newpage
\begin{figure}[htbp]
\centering
\includegraphics[width=0.8\textwidth]{images/00070.jpeg}
\caption{图片 70}
\end{figure}
\newpage
\begin{figure}[htbp]
\centering
\includegraphics[width=0.8\textwidth]{images/00071.jpeg}
\caption{图片 71}
\end{figure}
\newpage
\begin{figure}[htbp]
\centering
\includegraphics[width=0.8\textwidth]{images/00072.jpeg}
\caption{图片 72}
\end{figure}
\newpage
\begin{figure}[htbp]
\centering
\includegraphics[width=0.8\textwidth]{images/00073.jpeg}
\caption{图片 73}
\end{figure}
\newpage
\begin{figure}[htbp]
\centering
\includegraphics[width=0.8\textwidth]{images/00074.jpeg}
\caption{图片 74}
\end{figure}
\newpage
\begin{figure}[htbp]
\centering
\includegraphics[width=0.8\textwidth]{images/00075.jpeg}
\caption{图片 75}
\end{figure}
\newpage
\begin{figure}[htbp]
\centering
\includegraphics[width=0.8\textwidth]{images/00076.jpeg}
\caption{图片 76}
\end{figure}
\newpage
\begin{figure}[htbp]
\centering
\includegraphics[width=0.8\textwidth]{images/00077.jpeg}
\caption{图片 77}
\end{figure}
\newpage
\begin{figure}[htbp]
\centering
\includegraphics[width=0.8\textwidth]{images/00078.jpeg}
\caption{图片 78}
\end{figure}
\newpage
\begin{figure}[htbp]
\centering
\includegraphics[width=0.8\textwidth]{images/00079.jpeg}
\caption{图片 79}
\end{figure}
\newpage
\begin{figure}[htbp]
\centering
\includegraphics[width=0.8\textwidth]{images/00080.jpeg}
\caption{图片 80}
\end{figure}
\newpage
\begin{figure}[htbp]
\centering
\includegraphics[width=0.8\textwidth]{images/00081.jpeg}
\caption{图片 81}
\end{figure}
\newpage
\begin{figure}[htbp]
\centering
\includegraphics[width=0.8\textwidth]{images/00082.jpeg}
\caption{图片 82}
\end{figure}
\newpage
\begin{figure}[htbp]
\centering
\includegraphics[width=0.8\textwidth]{images/00083.jpeg}
\caption{图片 83}
\end{figure}
\newpage
\begin{figure}[htbp]
\centering
\includegraphics[width=0.8\textwidth]{images/00084.jpeg}
\caption{图片 84}
\end{figure}
\newpage
\begin{figure}[htbp]
\centering
\includegraphics[width=0.8\textwidth]{images/00085.jpeg}
\caption{图片 85}
\end{figure}
\newpage
\begin{figure}[htbp]
\centering
\includegraphics[width=0.8\textwidth]{images/00086.jpeg}
\caption{图片 86}
\end{figure}
\newpage
\begin{figure}[htbp]
\centering
\includegraphics[width=0.8\textwidth]{images/00087.jpeg}
\caption{图片 87}
\end{figure}
\newpage
\begin{figure}[htbp]
\centering
\includegraphics[width=0.8\textwidth]{images/00088.jpeg}
\caption{图片 88}
\end{figure}
\newpage
\begin{figure}[htbp]
\centering
\includegraphics[width=0.8\textwidth]{images/00089.jpeg}
\caption{图片 89}
\end{figure}
\newpage
\begin{figure}[htbp]
\centering
\includegraphics[width=0.8\textwidth]{images/00090.jpeg}
\caption{图片 90}
\end{figure}
\newpage
\begin{figure}[htbp]
\centering
\includegraphics[width=0.8\textwidth]{images/00091.jpeg}
\caption{图片 91}
\end{figure}
\newpage
\begin{figure}[htbp]
\centering
\includegraphics[width=0.8\textwidth]{images/00092.jpeg}
\caption{图片 92}
\end{figure}
\newpage
\begin{figure}[htbp]
\centering
\includegraphics[width=0.8\textwidth]{images/00093.jpeg}
\caption{图片 93}
\end{figure}
\newpage
\begin{figure}[htbp]
\centering
\includegraphics[width=0.8\textwidth]{images/00094.jpeg}
\caption{图片 94}
\end{figure}
\newpage
\begin{figure}[htbp]
\centering
\includegraphics[width=0.8\textwidth]{images/00095.jpeg}
\caption{图片 95}
\end{figure}
\newpage
\begin{figure}[htbp]
\centering
\includegraphics[width=0.8\textwidth]{images/00096.jpeg}
\caption{图片 96}
\end{figure}
\newpage
\begin{figure}[htbp]
\centering
\includegraphics[width=0.8\textwidth]{images/00097.jpeg}
\caption{图片 97}
\end{figure}
\newpage
\begin{figure}[htbp]
\centering
\includegraphics[width=0.8\textwidth]{images/00098.jpeg}
\caption{图片 98}
\end{figure}
\newpage
\begin{figure}[htbp]
\centering
\includegraphics[width=0.8\textwidth]{images/00099.jpeg}
\caption{图片 99}
\end{figure}
\newpage
\begin{figure}[htbp]
\centering
\includegraphics[width=0.8\textwidth]{images/00100.jpeg}
\caption{图片 100}
\end{figure}
\newpage
\begin{figure}[htbp]
\centering
\includegraphics[width=0.8\textwidth]{images/00101.jpeg}
\caption{图片 101}
\end{figure}
\newpage
\begin{figure}[htbp]
\centering
\includegraphics[width=0.8\textwidth]{images/00102.jpeg}
\caption{图片 102}
\end{figure}
\newpage
\begin{figure}[htbp]
\centering
\includegraphics[width=0.8\textwidth]{images/00103.jpeg}
\caption{图片 103}
\end{figure}
\newpage
\begin{figure}[htbp]
\centering
\includegraphics[width=0.8\textwidth]{images/00104.jpeg}
\caption{图片 104}
\end{figure}
\newpage
\begin{figure}[htbp]
\centering
\includegraphics[width=0.8\textwidth]{images/00105.jpeg}
\caption{图片 105}
\end{figure}
\newpage
\begin{figure}[htbp]
\centering
\includegraphics[width=0.8\textwidth]{images/00106.jpeg}
\caption{图片 106}
\end{figure}
\newpage
\begin{figure}[htbp]
\centering
\includegraphics[width=0.8\textwidth]{images/00107.jpeg}
\caption{图片 107}
\end{figure}
\newpage
\begin{figure}[htbp]
\centering
\includegraphics[width=0.8\textwidth]{images/00108.jpeg}
\caption{图片 108}
\end{figure}
\newpage
\begin{figure}[htbp]
\centering
\includegraphics[width=0.8\textwidth]{images/00109.jpeg}
\caption{图片 109}
\end{figure}
\newpage
\begin{figure}[htbp]
\centering
\includegraphics[width=0.8\textwidth]{images/00110.jpeg}
\caption{图片 110}
\end{figure}
\newpage
\begin{figure}[htbp]
\centering
\includegraphics[width=0.8\textwidth]{images/00111.jpeg}
\caption{图片 111}
\end{figure}
\newpage
\begin{figure}[htbp]
\centering
\includegraphics[width=0.8\textwidth]{images/00112.jpeg}
\caption{图片 112}
\end{figure}
\newpage
\begin{figure}[htbp]
\centering
\includegraphics[width=0.8\textwidth]{images/00113.jpeg}
\caption{图片 113}
\end{figure}
\newpage
\begin{figure}[htbp]
\centering
\includegraphics[width=0.8\textwidth]{images/00114.jpeg}
\caption{图片 114}
\end{figure}
\newpage
\begin{figure}[htbp]
\centering
\includegraphics[width=0.8\textwidth]{images/00115.jpeg}
\caption{图片 115}
\end{figure}
\newpage
\begin{figure}[htbp]
\centering
\includegraphics[width=0.8\textwidth]{images/00116.jpeg}
\caption{图片 116}
\end{figure}
\newpage
\begin{figure}[htbp]
\centering
\includegraphics[width=0.8\textwidth]{images/00117.jpeg}
\caption{图片 117}
\end{figure}
\newpage
\begin{figure}[htbp]
\centering
\includegraphics[width=0.8\textwidth]{images/00118.jpeg}
\caption{图片 118}
\end{figure}
\newpage
\begin{figure}[htbp]
\centering
\includegraphics[width=0.8\textwidth]{images/00119.jpeg}
\caption{图片 119}
\end{figure}
\newpage
\begin{figure}[htbp]
\centering
\includegraphics[width=0.8\textwidth]{images/00120.jpeg}
\caption{图片 120}
\end{figure}
\newpage
\begin{figure}[htbp]
\centering
\includegraphics[width=0.8\textwidth]{images/00121.jpeg}
\caption{图片 121}
\end{figure}
\newpage
\begin{figure}[htbp]
\centering
\includegraphics[width=0.8\textwidth]{images/00122.jpeg}
\caption{图片 122}
\end{figure}
\newpage
\begin{figure}[htbp]
\centering
\includegraphics[width=0.8\textwidth]{images/00123.jpeg}
\caption{图片 123}
\end{figure}
\newpage
\begin{figure}[htbp]
\centering
\includegraphics[width=0.8\textwidth]{images/00124.jpeg}
\caption{图片 124}
\end{figure}
\newpage
\begin{figure}[htbp]
\centering
\includegraphics[width=0.8\textwidth]{images/00125.jpeg}
\caption{图片 125}
\end{figure}
\newpage
\begin{figure}[htbp]
\centering
\includegraphics[width=0.8\textwidth]{images/00126.jpeg}
\caption{图片 126}
\end{figure}
\newpage
\begin{figure}[htbp]
\centering
\includegraphics[width=0.8\textwidth]{images/00127.jpeg}
\caption{图片 127}
\end{figure}
\newpage
\begin{figure}[htbp]
\centering
\includegraphics[width=0.8\textwidth]{images/00128.jpeg}
\caption{图片 128}
\end{figure}
\newpage
\begin{figure}[htbp]
\centering
\includegraphics[width=0.8\textwidth]{images/00129.jpeg}
\caption{图片 129}
\end{figure}
\newpage
\begin{figure}[htbp]
\centering
\includegraphics[width=0.8\textwidth]{images/00130.jpeg}
\caption{图片 130}
\end{figure}
\newpage
\begin{figure}[htbp]
\centering
\includegraphics[width=0.8\textwidth]{images/00131.jpeg}
\caption{图片 131}
\end{figure}
\newpage
\begin{figure}[htbp]
\centering
\includegraphics[width=0.8\textwidth]{images/00132.jpeg}
\caption{图片 132}
\end{figure}
\newpage
\begin{figure}[htbp]
\centering
\includegraphics[width=0.8\textwidth]{images/00133.jpeg}
\caption{图片 133}
\end{figure}
\newpage
\begin{figure}[htbp]
\centering
\includegraphics[width=0.8\textwidth]{images/00134.jpeg}
\caption{图片 134}
\end{figure}
\newpage
\begin{figure}[htbp]
\centering
\includegraphics[width=0.8\textwidth]{images/00135.jpeg}
\caption{图片 135}
\end{figure}
\newpage
\begin{figure}[htbp]
\centering
\includegraphics[width=0.8\textwidth]{images/00136.jpeg}
\caption{图片 136}
\end{figure}
\newpage
\begin{figure}[htbp]
\centering
\includegraphics[width=0.8\textwidth]{images/00137.jpeg}
\caption{图片 137}
\end{figure}
\newpage
\begin{figure}[htbp]
\centering
\includegraphics[width=0.8\textwidth]{images/00138.jpeg}
\caption{图片 138}
\end{figure}
\newpage
\begin{figure}[htbp]
\centering
\includegraphics[width=0.8\textwidth]{images/00139.jpeg}
\caption{图片 139}
\end{figure}
\newpage
\begin{figure}[htbp]
\centering
\includegraphics[width=0.8\textwidth]{images/00140.jpeg}
\caption{图片 140}
\end{figure}
\newpage
\begin{figure}[htbp]
\centering
\includegraphics[width=0.8\textwidth]{images/00141.jpeg}
\caption{图片 141}
\end{figure}
\newpage
\begin{figure}[htbp]
\centering
\includegraphics[width=0.8\textwidth]{images/00142.jpeg}
\caption{图片 142}
\end{figure}
\newpage
\begin{figure}[htbp]
\centering
\includegraphics[width=0.8\textwidth]{images/00143.jpeg}
\caption{图片 143}
\end{figure}
\newpage
\begin{figure}[htbp]
\centering
\includegraphics[width=0.8\textwidth]{images/00144.jpeg}
\caption{图片 144}
\end{figure}
\newpage
\begin{figure}[htbp]
\centering
\includegraphics[width=0.8\textwidth]{images/00145.jpeg}
\caption{图片 145}
\end{figure}
\newpage
\begin{figure}[htbp]
\centering
\includegraphics[width=0.8\textwidth]{images/00146.jpeg}
\caption{图片 146}
\end{figure}
\newpage
\begin{figure}[htbp]
\centering
\includegraphics[width=0.8\textwidth]{images/00147.jpeg}
\caption{图片 147}
\end{figure}
\newpage
\begin{figure}[htbp]
\centering
\includegraphics[width=0.8\textwidth]{images/00148.jpeg}
\caption{图片 148}
\end{figure}
\newpage
\begin{figure}[htbp]
\centering
\includegraphics[width=0.8\textwidth]{images/00149.jpeg}
\caption{图片 149}
\end{figure}
\newpage
\begin{figure}[htbp]
\centering
\includegraphics[width=0.8\textwidth]{images/00150.jpeg}
\caption{图片 150}
\end{figure}
\newpage
\begin{figure}[htbp]
\centering
\includegraphics[width=0.8\textwidth]{images/00151.jpeg}
\caption{图片 151}
\end{figure}
\newpage
\begin{figure}[htbp]
\centering
\includegraphics[width=0.8\textwidth]{images/00152.jpeg}
\caption{图片 152}
\end{figure}
\newpage
\begin{figure}[htbp]
\centering
\includegraphics[width=0.8\textwidth]{images/00153.jpeg}
\caption{图片 153}
\end{figure}
\newpage
\begin{figure}[htbp]
\centering
\includegraphics[width=0.8\textwidth]{images/00154.jpeg}
\caption{图片 154}
\end{figure}
\newpage
\begin{figure}[htbp]
\centering
\includegraphics[width=0.8\textwidth]{images/00155.jpeg}
\caption{图片 155}
\end{figure}
\newpage
\begin{figure}[htbp]
\centering
\includegraphics[width=0.8\textwidth]{images/00156.jpeg}
\caption{图片 156}
\end{figure}
\newpage
\begin{figure}[htbp]
\centering
\includegraphics[width=0.8\textwidth]{images/00157.jpeg}
\caption{图片 157}
\end{figure}
\newpage
\begin{figure}[htbp]
\centering
\includegraphics[width=0.8\textwidth]{images/00158.jpeg}
\caption{图片 158}
\end{figure}
\newpage
\begin{figure}[htbp]
\centering
\includegraphics[width=0.8\textwidth]{images/00159.jpeg}
\caption{图片 159}
\end{figure}
\newpage
\begin{figure}[htbp]
\centering
\includegraphics[width=0.8\textwidth]{images/00160.jpeg}
\caption{图片 160}
\end{figure}
\newpage
\begin{figure}[htbp]
\centering
\includegraphics[width=0.8\textwidth]{images/00161.jpeg}
\caption{图片 161}
\end{figure}
\newpage
\begin{figure}[htbp]
\centering
\includegraphics[width=0.8\textwidth]{images/00162.jpeg}
\caption{图片 162}
\end{figure}
\newpage
\begin{figure}[htbp]
\centering
\includegraphics[width=0.8\textwidth]{images/00163.jpeg}
\caption{图片 163}
\end{figure}
\newpage
\begin{figure}[htbp]
\centering
\includegraphics[width=0.8\textwidth]{images/00164.jpeg}
\caption{图片 164}
\end{figure}
\newpage
\begin{figure}[htbp]
\centering
\includegraphics[width=0.8\textwidth]{images/00165.jpeg}
\caption{图片 165}
\end{figure}
\newpage
\begin{figure}[htbp]
\centering
\includegraphics[width=0.8\textwidth]{images/00166.jpeg}
\caption{图片 166}
\end{figure}
\newpage
\begin{figure}[htbp]
\centering
\includegraphics[width=0.8\textwidth]{images/00167.jpeg}
\caption{图片 167}
\end{figure}
\newpage
\begin{figure}[htbp]
\centering
\includegraphics[width=0.8\textwidth]{images/00168.jpeg}
\caption{图片 168}
\end{figure}
\newpage
\begin{figure}[htbp]
\centering
\includegraphics[width=0.8\textwidth]{images/00169.jpeg}
\caption{图片 169}
\end{figure}
\newpage
\begin{figure}[htbp]
\centering
\includegraphics[width=0.8\textwidth]{images/00170.jpeg}
\caption{图片 170}
\end{figure}
\newpage
\begin{figure}[htbp]
\centering
\includegraphics[width=0.8\textwidth]{images/00171.jpeg}
\caption{图片 171}
\end{figure}
\newpage
\begin{figure}[htbp]
\centering
\includegraphics[width=0.8\textwidth]{images/00172.jpeg}
\caption{图片 172}
\end{figure}
\newpage
\begin{figure}[htbp]
\centering
\includegraphics[width=0.8\textwidth]{images/00173.jpeg}
\caption{图片 173}
\end{figure}
\newpage
\begin{figure}[htbp]
\centering
\includegraphics[width=0.8\textwidth]{images/00174.jpeg}
\caption{图片 174}
\end{figure}
\newpage
\begin{figure}[htbp]
\centering
\includegraphics[width=0.8\textwidth]{images/00175.jpeg}
\caption{图片 175}
\end{figure}
\newpage
\begin{figure}[htbp]
\centering
\includegraphics[width=0.8\textwidth]{images/00176.jpeg}
\caption{图片 176}
\end{figure}
\newpage
\begin{figure}[htbp]
\centering
\includegraphics[width=0.8\textwidth]{images/00177.jpeg}
\caption{图片 177}
\end{figure}
\newpage
\begin{figure}[htbp]
\centering
\includegraphics[width=0.8\textwidth]{images/00178.jpeg}
\caption{图片 178}
\end{figure}
\newpage
\begin{figure}[htbp]
\centering
\includegraphics[width=0.8\textwidth]{images/00179.jpeg}
\caption{图片 179}
\end{figure}
\newpage
\begin{figure}[htbp]
\centering
\includegraphics[width=0.8\textwidth]{images/00180.jpeg}
\caption{图片 180}
\end{figure}
\newpage
\begin{figure}[htbp]
\centering
\includegraphics[width=0.8\textwidth]{images/00181.jpeg}
\caption{图片 181}
\end{figure}
\newpage
\begin{figure}[htbp]
\centering
\includegraphics[width=0.8\textwidth]{images/00182.jpeg}
\caption{图片 182}
\end{figure}
\newpage
\begin{figure}[htbp]
\centering
\includegraphics[width=0.8\textwidth]{images/00183.jpeg}
\caption{图片 183}
\end{figure}
\newpage
\begin{figure}[htbp]
\centering
\includegraphics[width=0.8\textwidth]{images/00184.jpeg}
\caption{图片 184}
\end{figure}
\newpage
\begin{figure}[htbp]
\centering
\includegraphics[width=0.8\textwidth]{images/00185.jpeg}
\caption{图片 185}
\end{figure}
\newpage
\begin{figure}[htbp]
\centering
\includegraphics[width=0.8\textwidth]{images/00186.jpeg}
\caption{图片 186}
\end{figure}
\newpage
\begin{figure}[htbp]
\centering
\includegraphics[width=0.8\textwidth]{images/00187.jpeg}
\caption{图片 187}
\end{figure}
\newpage
\begin{figure}[htbp]
\centering
\includegraphics[width=0.8\textwidth]{images/00188.jpeg}
\caption{图片 188}
\end{figure}
\newpage
\begin{figure}[htbp]
\centering
\includegraphics[width=0.8\textwidth]{images/00189.jpeg}
\caption{图片 189}
\end{figure}
\newpage
\begin{figure}[htbp]
\centering
\includegraphics[width=0.8\textwidth]{images/00190.jpeg}
\caption{图片 190}
\end{figure}
\newpage
\begin{figure}[htbp]
\centering
\includegraphics[width=0.8\textwidth]{images/00191.jpeg}
\caption{图片 191}
\end{figure}
\newpage
\begin{figure}[htbp]
\centering
\includegraphics[width=0.8\textwidth]{images/00192.jpeg}
\caption{图片 192}
\end{figure}
\newpage
\begin{figure}[htbp]
\centering
\includegraphics[width=0.8\textwidth]{images/00193.jpeg}
\caption{图片 193}
\end{figure}
\newpage
\begin{figure}[htbp]
\centering
\includegraphics[width=0.8\textwidth]{images/00194.jpeg}
\caption{图片 194}
\end{figure}
\newpage
\begin{figure}[htbp]
\centering
\includegraphics[width=0.8\textwidth]{images/00195.jpeg}
\caption{图片 195}
\end{figure}
\newpage
\begin{figure}[htbp]
\centering
\includegraphics[width=0.8\textwidth]{images/00196.jpeg}
\caption{图片 196}
\end{figure}
\newpage
\begin{figure}[htbp]
\centering
\includegraphics[width=0.8\textwidth]{images/00197.jpeg}
\caption{图片 197}
\end{figure}
\newpage
\begin{figure}[htbp]
\centering
\includegraphics[width=0.8\textwidth]{images/00198.jpeg}
\caption{图片 198}
\end{figure}
\newpage
\begin{figure}[htbp]
\centering
\includegraphics[width=0.8\textwidth]{images/00199.jpeg}
\caption{图片 199}
\end{figure}
\newpage
\begin{figure}[htbp]
\centering
\includegraphics[width=0.8\textwidth]{images/00200.jpeg}
\caption{图片 200}
\end{figure}
\newpage
\begin{figure}[htbp]
\centering
\includegraphics[width=0.8\textwidth]{images/00201.jpeg}
\caption{图片 201}
\end{figure}
\newpage
\begin{figure}[htbp]
\centering
\includegraphics[width=0.8\textwidth]{images/00202.jpeg}
\caption{图片 202}
\end{figure}
\newpage
\begin{figure}[htbp]
\centering
\includegraphics[width=0.8\textwidth]{images/00203.jpeg}
\caption{图片 203}
\end{figure}
\newpage
\begin{figure}[htbp]
\centering
\includegraphics[width=0.8\textwidth]{images/00204.jpeg}
\caption{图片 204}
\end{figure}
\newpage
\begin{figure}[htbp]
\centering
\includegraphics[width=0.8\textwidth]{images/00205.jpeg}
\caption{图片 205}
\end{figure}
\newpage
\begin{figure}[htbp]
\centering
\includegraphics[width=0.8\textwidth]{images/00206.jpeg}
\caption{图片 206}
\end{figure}
\newpage
\begin{figure}[htbp]
\centering
\includegraphics[width=0.8\textwidth]{images/00207.jpeg}
\caption{图片 207}
\end{figure}
\newpage
\begin{figure}[htbp]
\centering
\includegraphics[width=0.8\textwidth]{images/00208.jpeg}
\caption{图片 208}
\end{figure}
\newpage
\begin{figure}[htbp]
\centering
\includegraphics[width=0.8\textwidth]{images/00209.jpeg}
\caption{图片 209}
\end{figure}
\newpage
\begin{figure}[htbp]
\centering
\includegraphics[width=0.8\textwidth]{images/00210.jpeg}
\caption{图片 210}
\end{figure}
\newpage
\begin{figure}[htbp]
\centering
\includegraphics[width=0.8\textwidth]{images/00211.jpeg}
\caption{图片 211}
\end{figure}
\newpage
\begin{figure}[htbp]
\centering
\includegraphics[width=0.8\textwidth]{images/00212.jpeg}
\caption{图片 212}
\end{figure}
\newpage
\begin{figure}[htbp]
\centering
\includegraphics[width=0.8\textwidth]{images/00213.jpeg}
\caption{图片 213}
\end{figure}
\newpage
\begin{figure}[htbp]
\centering
\includegraphics[width=0.8\textwidth]{images/00214.jpeg}
\caption{图片 214}
\end{figure}
\newpage
\begin{figure}[htbp]
\centering
\includegraphics[width=0.8\textwidth]{images/00215.jpeg}
\caption{图片 215}
\end{figure}
\newpage
\begin{figure}[htbp]
\centering
\includegraphics[width=0.8\textwidth]{images/00216.jpeg}
\caption{图片 216}
\end{figure}
\newpage
\begin{figure}[htbp]
\centering
\includegraphics[width=0.8\textwidth]{images/00217.jpeg}
\caption{图片 217}
\end{figure}
\newpage
\begin{figure}[htbp]
\centering
\includegraphics[width=0.8\textwidth]{images/00218.jpeg}
\caption{图片 218}
\end{figure}
\newpage
\begin{figure}[htbp]
\centering
\includegraphics[width=0.8\textwidth]{images/00219.jpeg}
\caption{图片 219}
\end{figure}
\newpage
\begin{figure}[htbp]
\centering
\includegraphics[width=0.8\textwidth]{images/00220.jpeg}
\caption{图片 220}
\end{figure}
\newpage
\begin{figure}[htbp]
\centering
\includegraphics[width=0.8\textwidth]{images/00221.jpeg}
\caption{图片 221}
\end{figure}
\newpage
\begin{figure}[htbp]
\centering
\includegraphics[width=0.8\textwidth]{images/00222.jpeg}
\caption{图片 222}
\end{figure}
\newpage
\begin{figure}[htbp]
\centering
\includegraphics[width=0.8\textwidth]{images/00223.jpeg}
\caption{图片 223}
\end{figure}
\newpage
\begin{figure}[htbp]
\centering
\includegraphics[width=0.8\textwidth]{images/00224.jpeg}
\caption{图片 224}
\end{figure}
\newpage
\begin{figure}[htbp]
\centering
\includegraphics[width=0.8\textwidth]{images/00225.jpeg}
\caption{图片 225}
\end{figure}
\newpage
\begin{figure}[htbp]
\centering
\includegraphics[width=0.8\textwidth]{images/00226.jpeg}
\caption{图片 226}
\end{figure}
\newpage
\begin{figure}[htbp]
\centering
\includegraphics[width=0.8\textwidth]{images/00227.jpeg}
\caption{图片 227}
\end{figure}
\newpage
\begin{figure}[htbp]
\centering
\includegraphics[width=0.8\textwidth]{images/00228.jpeg}
\caption{图片 228}
\end{figure}
\newpage
\begin{figure}[htbp]
\centering
\includegraphics[width=0.8\textwidth]{images/00229.jpeg}
\caption{图片 229}
\end{figure}
\newpage
\begin{figure}[htbp]
\centering
\includegraphics[width=0.8\textwidth]{images/00230.jpeg}
\caption{图片 230}
\end{figure}
\newpage
\begin{figure}[htbp]
\centering
\includegraphics[width=0.8\textwidth]{images/00231.jpeg}
\caption{图片 231}
\end{figure}
\newpage
\begin{figure}[htbp]
\centering
\includegraphics[width=0.8\textwidth]{images/00232.jpeg}
\caption{图片 232}
\end{figure}
\newpage
\begin{figure}[htbp]
\centering
\includegraphics[width=0.8\textwidth]{images/00233.jpeg}
\caption{图片 233}
\end{figure}
\newpage
\begin{figure}[htbp]
\centering
\includegraphics[width=0.8\textwidth]{images/00234.jpeg}
\caption{图片 234}
\end{figure}
\newpage
\begin{figure}[htbp]
\centering
\includegraphics[width=0.8\textwidth]{images/00235.jpeg}
\caption{图片 235}
\end{figure}
\newpage
\begin{figure}[htbp]
\centering
\includegraphics[width=0.8\textwidth]{images/00236.jpeg}
\caption{图片 236}
\end{figure}
\newpage
\begin{figure}[htbp]
\centering
\includegraphics[width=0.8\textwidth]{images/00237.jpeg}
\caption{图片 237}
\end{figure}
\newpage
\begin{figure}[htbp]
\centering
\includegraphics[width=0.8\textwidth]{images/00238.jpeg}
\caption{图片 238}
\end{figure}
\newpage
\begin{figure}[htbp]
\centering
\includegraphics[width=0.8\textwidth]{images/00239.jpeg}
\caption{图片 239}
\end{figure}
\newpage
\begin{figure}[htbp]
\centering
\includegraphics[width=0.8\textwidth]{images/00240.jpeg}
\caption{图片 240}
\end{figure}
\newpage
\begin{figure}[htbp]
\centering
\includegraphics[width=0.8\textwidth]{images/00241.jpeg}
\caption{图片 241}
\end{figure}
\newpage
\begin{figure}[htbp]
\centering
\includegraphics[width=0.8\textwidth]{images/00242.jpeg}
\caption{图片 242}
\end{figure}
\newpage
\begin{figure}[htbp]
\centering
\includegraphics[width=0.8\textwidth]{images/00243.jpeg}
\caption{图片 243}
\end{figure}
\newpage
\begin{figure}[htbp]
\centering
\includegraphics[width=0.8\textwidth]{images/00244.jpeg}
\caption{图片 244}
\end{figure}
\newpage
\begin{figure}[htbp]
\centering
\includegraphics[width=0.8\textwidth]{images/00245.jpeg}
\caption{图片 245}
\end{figure}
\newpage
\begin{figure}[htbp]
\centering
\includegraphics[width=0.8\textwidth]{images/00246.jpeg}
\caption{图片 246}
\end{figure}
\newpage
\begin{figure}[htbp]
\centering
\includegraphics[width=0.8\textwidth]{images/00247.jpeg}
\caption{图片 247}
\end{figure}
\newpage
\begin{figure}[htbp]
\centering
\includegraphics[width=0.8\textwidth]{images/00248.jpeg}
\caption{图片 248}
\end{figure}
\newpage
\begin{figure}[htbp]
\centering
\includegraphics[width=0.8\textwidth]{images/00249.jpeg}
\caption{图片 249}
\end{figure}
\newpage
\begin{figure}[htbp]
\centering
\includegraphics[width=0.8\textwidth]{images/00250.jpeg}
\caption{图片 250}
\end{figure}
\newpage
\begin{figure}[htbp]
\centering
\includegraphics[width=0.8\textwidth]{images/00251.jpeg}
\caption{图片 251}
\end{figure}
\newpage
\begin{figure}[htbp]
\centering
\includegraphics[width=0.8\textwidth]{images/00252.jpeg}
\caption{图片 252}
\end{figure}
\newpage
\begin{figure}[htbp]
\centering
\includegraphics[width=0.8\textwidth]{images/00253.jpeg}
\caption{图片 253}
\end{figure}
\newpage
\begin{figure}[htbp]
\centering
\includegraphics[width=0.8\textwidth]{images/00254.jpeg}
\caption{图片 254}
\end{figure}
\newpage
\begin{figure}[htbp]
\centering
\includegraphics[width=0.8\textwidth]{images/00255.jpeg}
\caption{图片 255}
\end{figure}
\newpage
\begin{figure}[htbp]
\centering
\includegraphics[width=0.8\textwidth]{images/00256.jpeg}
\caption{图片 256}
\end{figure}
\newpage
\begin{figure}[htbp]
\centering
\includegraphics[width=0.8\textwidth]{images/00257.jpeg}
\caption{图片 257}
\end{figure}
\newpage
\begin{figure}[htbp]
\centering
\includegraphics[width=0.8\textwidth]{images/00258.jpeg}
\caption{图片 258}
\end{figure}
\newpage
\begin{figure}[htbp]
\centering
\includegraphics[width=0.8\textwidth]{images/00259.jpeg}
\caption{图片 259}
\end{figure}
\newpage
\begin{figure}[htbp]
\centering
\includegraphics[width=0.8\textwidth]{images/00260.jpeg}
\caption{图片 260}
\end{figure}
\newpage
\begin{figure}[htbp]
\centering
\includegraphics[width=0.8\textwidth]{images/00261.jpeg}
\caption{图片 261}
\end{figure}
\newpage
\begin{figure}[htbp]
\centering
\includegraphics[width=0.8\textwidth]{images/00262.jpeg}
\caption{图片 262}
\end{figure}
\newpage
\begin{figure}[htbp]
\centering
\includegraphics[width=0.8\textwidth]{images/00263.jpeg}
\caption{图片 263}
\end{figure}
\newpage
\begin{figure}[htbp]
\centering
\includegraphics[width=0.8\textwidth]{images/00264.jpeg}
\caption{图片 264}
\end{figure}
\newpage
\begin{figure}[htbp]
\centering
\includegraphics[width=0.8\textwidth]{images/00265.jpeg}
\caption{图片 265}
\end{figure}
\newpage
\begin{figure}[htbp]
\centering
\includegraphics[width=0.8\textwidth]{images/00266.jpeg}
\caption{图片 266}
\end{figure}
\newpage
\begin{figure}[htbp]
\centering
\includegraphics[width=0.8\textwidth]{images/00267.jpeg}
\caption{图片 267}
\end{figure}
\newpage
\begin{figure}[htbp]
\centering
\includegraphics[width=0.8\textwidth]{images/00268.jpeg}
\caption{图片 268}
\end{figure}
\newpage
\begin{figure}[htbp]
\centering
\includegraphics[width=0.8\textwidth]{images/00269.jpeg}
\caption{图片 269}
\end{figure}
\newpage
\begin{figure}[htbp]
\centering
\includegraphics[width=0.8\textwidth]{images/00270.jpeg}
\caption{图片 270}
\end{figure}
\newpage
\begin{figure}[htbp]
\centering
\includegraphics[width=0.8\textwidth]{images/00271.jpeg}
\caption{图片 271}
\end{figure}
\newpage
\begin{figure}[htbp]
\centering
\includegraphics[width=0.8\textwidth]{images/00272.jpeg}
\caption{图片 272}
\end{figure}
\newpage
\begin{figure}[htbp]
\centering
\includegraphics[width=0.8\textwidth]{images/00273.jpeg}
\caption{图片 273}
\end{figure}
\newpage
\begin{figure}[htbp]
\centering
\includegraphics[width=0.8\textwidth]{images/00274.jpeg}
\caption{图片 274}
\end{figure}
\newpage
\begin{figure}[htbp]
\centering
\includegraphics[width=0.8\textwidth]{images/00275.jpeg}
\caption{图片 275}
\end{figure}
\newpage
\begin{figure}[htbp]
\centering
\includegraphics[width=0.8\textwidth]{images/00276.jpeg}
\caption{图片 276}
\end{figure}
\newpage
\begin{figure}[htbp]
\centering
\includegraphics[width=0.8\textwidth]{images/00277.jpeg}
\caption{图片 277}
\end{figure}
\newpage
\begin{figure}[htbp]
\centering
\includegraphics[width=0.8\textwidth]{images/00278.jpeg}
\caption{图片 278}
\end{figure}
\newpage
\begin{figure}[htbp]
\centering
\includegraphics[width=0.8\textwidth]{images/00279.jpeg}
\caption{图片 279}
\end{figure}
\newpage
\begin{figure}[htbp]
\centering
\includegraphics[width=0.8\textwidth]{images/00280.jpeg}
\caption{图片 280}
\end{figure}
\newpage
\begin{figure}[htbp]
\centering
\includegraphics[width=0.8\textwidth]{images/00281.jpeg}
\caption{图片 281}
\end{figure}
\newpage
\begin{figure}[htbp]
\centering
\includegraphics[width=0.8\textwidth]{images/00282.jpeg}
\caption{图片 282}
\end{figure}
\newpage
\begin{figure}[htbp]
\centering
\includegraphics[width=0.8\textwidth]{images/00283.jpeg}
\caption{图片 283}
\end{figure}
\newpage
\begin{figure}[htbp]
\centering
\includegraphics[width=0.8\textwidth]{images/00284.jpeg}
\caption{图片 284}
\end{figure}
\newpage
\begin{figure}[htbp]
\centering
\includegraphics[width=0.8\textwidth]{images/00285.jpeg}
\caption{图片 285}
\end{figure}
\newpage
\begin{figure}[htbp]
\centering
\includegraphics[width=0.8\textwidth]{images/00286.jpeg}
\caption{图片 286}
\end{figure}
\newpage
\begin{figure}[htbp]
\centering
\includegraphics[width=0.8\textwidth]{images/00287.jpeg}
\caption{图片 287}
\end{figure}
\newpage
\begin{figure}[htbp]
\centering
\includegraphics[width=0.8\textwidth]{images/00288.jpeg}
\caption{图片 288}
\end{figure}
\newpage
\begin{figure}[htbp]
\centering
\includegraphics[width=0.8\textwidth]{images/00289.jpeg}
\caption{图片 289}
\end{figure}
\newpage
\begin{figure}[htbp]
\centering
\includegraphics[width=0.8\textwidth]{images/00290.jpeg}
\caption{图片 290}
\end{figure}
\newpage
\begin{figure}[htbp]
\centering
\includegraphics[width=0.8\textwidth]{images/00291.jpeg}
\caption{图片 291}
\end{figure}
\newpage
\begin{figure}[htbp]
\centering
\includegraphics[width=0.8\textwidth]{images/00292.jpeg}
\caption{图片 292}
\end{figure}
\newpage
\begin{figure}[htbp]
\centering
\includegraphics[width=0.8\textwidth]{images/00293.jpeg}
\caption{图片 293}
\end{figure}
\newpage
\begin{figure}[htbp]
\centering
\includegraphics[width=0.8\textwidth]{images/00294.jpeg}
\caption{图片 294}
\end{figure}
\newpage
\begin{figure}[htbp]
\centering
\includegraphics[width=0.8\textwidth]{images/00295.jpeg}
\caption{图片 295}
\end{figure}
\newpage
\begin{figure}[htbp]
\centering
\includegraphics[width=0.8\textwidth]{images/00296.jpeg}
\caption{图片 296}
\end{figure}
\newpage
\begin{figure}[htbp]
\centering
\includegraphics[width=0.8\textwidth]{images/00297.jpeg}
\caption{图片 297}
\end{figure}
\newpage
\begin{figure}[htbp]
\centering
\includegraphics[width=0.8\textwidth]{images/00298.jpeg}
\caption{图片 298}
\end{figure}
\newpage
\begin{figure}[htbp]
\centering
\includegraphics[width=0.8\textwidth]{images/00299.jpeg}
\caption{图片 299}
\end{figure}
\newpage
\begin{figure}[htbp]
\centering
\includegraphics[width=0.8\textwidth]{images/00300.jpeg}
\caption{图片 300}
\end{figure}
\newpage
\begin{figure}[htbp]
\centering
\includegraphics[width=0.8\textwidth]{images/00301.jpeg}
\caption{图片 301}
\end{figure}
\newpage
\begin{figure}[htbp]
\centering
\includegraphics[width=0.8\textwidth]{images/00302.jpeg}
\caption{图片 302}
\end{figure}
\newpage
\begin{figure}[htbp]
\centering
\includegraphics[width=0.8\textwidth]{images/00303.jpeg}
\caption{图片 303}
\end{figure}
\newpage
\begin{figure}[htbp]
\centering
\includegraphics[width=0.8\textwidth]{images/00304.jpeg}
\caption{图片 304}
\end{figure}
\newpage
\begin{figure}[htbp]
\centering
\includegraphics[width=0.8\textwidth]{images/00305.jpeg}
\caption{图片 305}
\end{figure}
\newpage
\begin{figure}[htbp]
\centering
\includegraphics[width=0.8\textwidth]{images/00306.jpeg}
\caption{图片 306}
\end{figure}
\newpage
\begin{figure}[htbp]
\centering
\includegraphics[width=0.8\textwidth]{images/00307.jpeg}
\caption{图片 307}
\end{figure}
\newpage
\begin{figure}[htbp]
\centering
\includegraphics[width=0.8\textwidth]{images/00308.jpeg}
\caption{图片 308}
\end{figure}
\newpage
\begin{figure}[htbp]
\centering
\includegraphics[width=0.8\textwidth]{images/00309.jpeg}
\caption{图片 309}
\end{figure}
\newpage
\begin{figure}[htbp]
\centering
\includegraphics[width=0.8\textwidth]{images/00310.jpeg}
\caption{图片 310}
\end{figure}
\newpage
\begin{figure}[htbp]
\centering
\includegraphics[width=0.8\textwidth]{images/00311.jpeg}
\caption{图片 311}
\end{figure}
\newpage
\begin{figure}[htbp]
\centering
\includegraphics[width=0.8\textwidth]{images/00312.jpeg}
\caption{图片 312}
\end{figure}
\newpage
\begin{figure}[htbp]
\centering
\includegraphics[width=0.8\textwidth]{images/00313.jpeg}
\caption{图片 313}
\end{figure}
\newpage
\begin{figure}[htbp]
\centering
\includegraphics[width=0.8\textwidth]{images/00314.jpeg}
\caption{图片 314}
\end{figure}
\newpage
\begin{figure}[htbp]
\centering
\includegraphics[width=0.8\textwidth]{images/00315.jpeg}
\caption{图片 315}
\end{figure}
\newpage
\begin{figure}[htbp]
\centering
\includegraphics[width=0.8\textwidth]{images/00316.jpeg}
\caption{图片 316}
\end{figure}
\newpage
\begin{figure}[htbp]
\centering
\includegraphics[width=0.8\textwidth]{images/00317.jpeg}
\caption{图片 317}
\end{figure}
\newpage
\begin{figure}[htbp]
\centering
\includegraphics[width=0.8\textwidth]{images/00318.jpeg}
\caption{图片 318}
\end{figure}
\newpage
\begin{figure}[htbp]
\centering
\includegraphics[width=0.8\textwidth]{images/00319.jpeg}
\caption{图片 319}
\end{figure}
\newpage
\begin{figure}[htbp]
\centering
\includegraphics[width=0.8\textwidth]{images/00320.jpeg}
\caption{图片 320}
\end{figure}
\newpage
\begin{figure}[htbp]
\centering
\includegraphics[width=0.8\textwidth]{images/00321.jpeg}
\caption{图片 321}
\end{figure}
\newpage
\begin{figure}[htbp]
\centering
\includegraphics[width=0.8\textwidth]{images/00322.jpeg}
\caption{图片 322}
\end{figure}
\newpage
\begin{figure}[htbp]
\centering
\includegraphics[width=0.8\textwidth]{images/00323.jpeg}
\caption{图片 323}
\end{figure}
\newpage
\begin{figure}[htbp]
\centering
\includegraphics[width=0.8\textwidth]{images/00324.jpeg}
\caption{图片 324}
\end{figure}
\newpage
\begin{figure}[htbp]
\centering
\includegraphics[width=0.8\textwidth]{images/00325.jpeg}
\caption{图片 325}
\end{figure}
\newpage
\begin{figure}[htbp]
\centering
\includegraphics[width=0.8\textwidth]{images/00326.jpeg}
\caption{图片 326}
\end{figure}
\newpage
\begin{figure}[htbp]
\centering
\includegraphics[width=0.8\textwidth]{images/00327.jpeg}
\caption{图片 327}
\end{figure}
\newpage
\begin{figure}[htbp]
\centering
\includegraphics[width=0.8\textwidth]{images/00328.jpeg}
\caption{图片 328}
\end{figure}
\newpage
\begin{figure}[htbp]
\centering
\includegraphics[width=0.8\textwidth]{images/00329.jpeg}
\caption{图片 329}
\end{figure}
\newpage
\begin{figure}[htbp]
\centering
\includegraphics[width=0.8\textwidth]{images/00330.jpeg}
\caption{图片 330}
\end{figure}
\newpage
\begin{figure}[htbp]
\centering
\includegraphics[width=0.8\textwidth]{images/00331.jpeg}
\caption{图片 331}
\end{figure}
\newpage
\begin{figure}[htbp]
\centering
\includegraphics[width=0.8\textwidth]{images/00332.jpeg}
\caption{图片 332}
\end{figure}
\newpage
\begin{figure}[htbp]
\centering
\includegraphics[width=0.8\textwidth]{images/00333.jpeg}
\caption{图片 333}
\end{figure}
\newpage
\begin{figure}[htbp]
\centering
\includegraphics[width=0.8\textwidth]{images/00334.jpeg}
\caption{图片 334}
\end{figure}
\newpage
\begin{figure}[htbp]
\centering
\includegraphics[width=0.8\textwidth]{images/00335.jpeg}
\caption{图片 335}
\end{figure}
\newpage
\begin{figure}[htbp]
\centering
\includegraphics[width=0.8\textwidth]{images/00336.jpeg}
\caption{图片 336}
\end{figure}
\newpage
\begin{figure}[htbp]
\centering
\includegraphics[width=0.8\textwidth]{images/00337.jpeg}
\caption{图片 337}
\end{figure}
\newpage
\begin{figure}[htbp]
\centering
\includegraphics[width=0.8\textwidth]{images/00338.jpeg}
\caption{图片 338}
\end{figure}
\newpage
\begin{figure}[htbp]
\centering
\includegraphics[width=0.8\textwidth]{images/00339.jpeg}
\caption{图片 339}
\end{figure}
\newpage
\begin{figure}[htbp]
\centering
\includegraphics[width=0.8\textwidth]{images/00340.jpeg}
\caption{图片 340}
\end{figure}
\newpage
\begin{figure}[htbp]
\centering
\includegraphics[width=0.8\textwidth]{images/00341.jpeg}
\caption{图片 341}
\end{figure}
\newpage
\begin{figure}[htbp]
\centering
\includegraphics[width=0.8\textwidth]{images/00342.jpeg}
\caption{图片 342}
\end{figure}
\newpage
\begin{figure}[htbp]
\centering
\includegraphics[width=0.8\textwidth]{images/00343.jpeg}
\caption{图片 343}
\end{figure}
\newpage
\begin{figure}[htbp]
\centering
\includegraphics[width=0.8\textwidth]{images/00344.jpeg}
\caption{图片 344}
\end{figure}
\newpage
\begin{figure}[htbp]
\centering
\includegraphics[width=0.8\textwidth]{images/00345.jpeg}
\caption{图片 345}
\end{figure}
\newpage
\begin{figure}[htbp]
\centering
\includegraphics[width=0.8\textwidth]{images/00346.jpeg}
\caption{图片 346}
\end{figure}
\newpage
\begin{figure}[htbp]
\centering
\includegraphics[width=0.8\textwidth]{images/00347.jpeg}
\caption{图片 347}
\end{figure}
\newpage
\begin{figure}[htbp]
\centering
\includegraphics[width=0.8\textwidth]{images/00348.jpeg}
\caption{图片 348}
\end{figure}
\newpage
\begin{figure}[htbp]
\centering
\includegraphics[width=0.8\textwidth]{images/00349.jpeg}
\caption{图片 349}
\end{figure}
\newpage
\begin{figure}[htbp]
\centering
\includegraphics[width=0.8\textwidth]{images/00350.jpeg}
\caption{图片 350}
\end{figure}
\newpage
\begin{figure}[htbp]
\centering
\includegraphics[width=0.8\textwidth]{images/00351.jpeg}
\caption{图片 351}
\end{figure}
\newpage
\begin{figure}[htbp]
\centering
\includegraphics[width=0.8\textwidth]{images/00352.jpeg}
\caption{图片 352}
\end{figure}
\newpage
\begin{figure}[htbp]
\centering
\includegraphics[width=0.8\textwidth]{images/00353.jpeg}
\caption{图片 353}
\end{figure}
\newpage
\begin{figure}[htbp]
\centering
\includegraphics[width=0.8\textwidth]{images/00354.jpeg}
\caption{图片 354}
\end{figure}
\newpage
\begin{figure}[htbp]
\centering
\includegraphics[width=0.8\textwidth]{images/00355.jpeg}
\caption{图片 355}
\end{figure}
\newpage
\begin{figure}[htbp]
\centering
\includegraphics[width=0.8\textwidth]{images/00356.jpeg}
\caption{图片 356}
\end{figure}
\newpage
\begin{figure}[htbp]
\centering
\includegraphics[width=0.8\textwidth]{images/00357.jpeg}
\caption{图片 357}
\end{figure}
\newpage
\begin{figure}[htbp]
\centering
\includegraphics[width=0.8\textwidth]{images/00358.jpeg}
\caption{图片 358}
\end{figure}
\newpage
\begin{figure}[htbp]
\centering
\includegraphics[width=0.8\textwidth]{images/00359.jpeg}
\caption{图片 359}
\end{figure}
\newpage
\begin{figure}[htbp]
\centering
\includegraphics[width=0.8\textwidth]{images/00360.jpeg}
\caption{图片 360}
\end{figure}
\newpage
\begin{figure}[htbp]
\centering
\includegraphics[width=0.8\textwidth]{images/00361.jpeg}
\caption{图片 361}
\end{figure}
\newpage
\begin{figure}[htbp]
\centering
\includegraphics[width=0.8\textwidth]{images/00362.jpeg}
\caption{图片 362}
\end{figure}
\newpage
\begin{figure}[htbp]
\centering
\includegraphics[width=0.8\textwidth]{images/00363.jpeg}
\caption{图片 363}
\end{figure}
\newpage
\begin{figure}[htbp]
\centering
\includegraphics[width=0.8\textwidth]{images/00364.jpeg}
\caption{图片 364}
\end{figure}
\newpage
\begin{figure}[htbp]
\centering
\includegraphics[width=0.8\textwidth]{images/00365.jpeg}
\caption{图片 365}
\end{figure}
\newpage
\begin{figure}[htbp]
\centering
\includegraphics[width=0.8\textwidth]{images/00366.jpeg}
\caption{图片 366}
\end{figure}
\newpage
\begin{figure}[htbp]
\centering
\includegraphics[width=0.8\textwidth]{images/00367.jpeg}
\caption{图片 367}
\end{figure}
\newpage
\begin{figure}[htbp]
\centering
\includegraphics[width=0.8\textwidth]{images/00368.jpeg}
\caption{图片 368}
\end{figure}
\newpage
\begin{figure}[htbp]
\centering
\includegraphics[width=0.8\textwidth]{images/00369.jpeg}
\caption{图片 369}
\end{figure}
\newpage
\begin{figure}[htbp]
\centering
\includegraphics[width=0.8\textwidth]{images/00370.jpeg}
\caption{图片 370}
\end{figure}
\newpage
\begin{figure}[htbp]
\centering
\includegraphics[width=0.8\textwidth]{images/00371.jpeg}
\caption{图片 371}
\end{figure}
\newpage
\begin{figure}[htbp]
\centering
\includegraphics[width=0.8\textwidth]{images/00372.jpeg}
\caption{图片 372}
\end{figure}
\newpage
\begin{figure}[htbp]
\centering
\includegraphics[width=0.8\textwidth]{images/00373.jpeg}
\caption{图片 373}
\end{figure}
\newpage
\begin{figure}[htbp]
\centering
\includegraphics[width=0.8\textwidth]{images/00374.jpeg}
\caption{图片 374}
\end{figure}
\newpage
\begin{figure}[htbp]
\centering
\includegraphics[width=0.8\textwidth]{images/00375.jpeg}
\caption{图片 375}
\end{figure}
\newpage
\begin{figure}[htbp]
\centering
\includegraphics[width=0.8\textwidth]{images/00376.jpeg}
\caption{图片 376}
\end{figure}
\newpage
\begin{figure}[htbp]
\centering
\includegraphics[width=0.8\textwidth]{images/00377.jpeg}
\caption{图片 377}
\end{figure}
\newpage
\begin{figure}[htbp]
\centering
\includegraphics[width=0.8\textwidth]{images/00378.jpeg}
\caption{图片 378}
\end{figure}
\newpage
\begin{figure}[htbp]
\centering
\includegraphics[width=0.8\textwidth]{images/00379.jpeg}
\caption{图片 379}
\end{figure}
\newpage
\begin{figure}[htbp]
\centering
\includegraphics[width=0.8\textwidth]{images/00380.jpeg}
\caption{图片 380}
\end{figure}
\newpage
\begin{figure}[htbp]
\centering
\includegraphics[width=0.8\textwidth]{images/00381.jpeg}
\caption{图片 381}
\end{figure}
\newpage
\begin{figure}[htbp]
\centering
\includegraphics[width=0.8\textwidth]{images/00382.jpeg}
\caption{图片 382}
\end{figure}
\newpage
\begin{figure}[htbp]
\centering
\includegraphics[width=0.8\textwidth]{images/00383.jpeg}
\caption{图片 383}
\end{figure}
\newpage
\begin{figure}[htbp]
\centering
\includegraphics[width=0.8\textwidth]{images/00384.jpeg}
\caption{图片 384}
\end{figure}
\newpage
\begin{figure}[htbp]
\centering
\includegraphics[width=0.8\textwidth]{images/00385.jpeg}
\caption{图片 385}
\end{figure}
\newpage
\begin{figure}[htbp]
\centering
\includegraphics[width=0.8\textwidth]{images/00386.jpeg}
\caption{图片 386}
\end{figure}
\newpage
\begin{figure}[htbp]
\centering
\includegraphics[width=0.8\textwidth]{images/00387.jpeg}
\caption{图片 387}
\end{figure}
\newpage
\begin{figure}[htbp]
\centering
\includegraphics[width=0.8\textwidth]{images/00388.jpeg}
\caption{图片 388}
\end{figure}
\newpage
\begin{figure}[htbp]
\centering
\includegraphics[width=0.8\textwidth]{images/00389.jpeg}
\caption{图片 389}
\end{figure}
\newpage
\begin{figure}[htbp]
\centering
\includegraphics[width=0.8\textwidth]{images/00390.jpeg}
\caption{图片 390}
\end{figure}
\newpage
\begin{figure}[htbp]
\centering
\includegraphics[width=0.8\textwidth]{images/00391.jpeg}
\caption{图片 391}
\end{figure}
\newpage
\begin{figure}[htbp]
\centering
\includegraphics[width=0.8\textwidth]{images/00392.jpeg}
\caption{图片 392}
\end{figure}
\newpage
\begin{figure}[htbp]
\centering
\includegraphics[width=0.8\textwidth]{images/00393.jpeg}
\caption{图片 393}
\end{figure}
\newpage
\begin{figure}[htbp]
\centering
\includegraphics[width=0.8\textwidth]{images/00394.jpeg}
\caption{图片 394}
\end{figure}
\newpage
\begin{figure}[htbp]
\centering
\includegraphics[width=0.8\textwidth]{images/00395.jpeg}
\caption{图片 395}
\end{figure}
\newpage
\begin{figure}[htbp]
\centering
\includegraphics[width=0.8\textwidth]{images/00396.jpeg}
\caption{图片 396}
\end{figure}
\newpage
\begin{figure}[htbp]
\centering
\includegraphics[width=0.8\textwidth]{images/00397.jpeg}
\caption{图片 397}
\end{figure}
\newpage
\begin{figure}[htbp]
\centering
\includegraphics[width=0.8\textwidth]{images/00398.jpeg}
\caption{图片 398}
\end{figure}
\newpage
\begin{figure}[htbp]
\centering
\includegraphics[width=0.8\textwidth]{images/00399.jpeg}
\caption{图片 399}
\end{figure}
\newpage
\begin{figure}[htbp]
\centering
\includegraphics[width=0.8\textwidth]{images/00400.jpeg}
\caption{图片 400}
\end{figure}
\newpage
\begin{figure}[htbp]
\centering
\includegraphics[width=0.8\textwidth]{images/00401.jpeg}
\caption{图片 401}
\end{figure}
\newpage
\begin{figure}[htbp]
\centering
\includegraphics[width=0.8\textwidth]{images/00402.jpeg}
\caption{图片 402}
\end{figure}
\newpage
\begin{figure}[htbp]
\centering
\includegraphics[width=0.8\textwidth]{images/00403.jpeg}
\caption{图片 403}
\end{figure}
\newpage
\begin{figure}[htbp]
\centering
\includegraphics[width=0.8\textwidth]{images/00404.jpeg}
\caption{图片 404}
\end{figure}
\newpage
\begin{figure}[htbp]
\centering
\includegraphics[width=0.8\textwidth]{images/00405.jpeg}
\caption{图片 405}
\end{figure}
\newpage
\begin{figure}[htbp]
\centering
\includegraphics[width=0.8\textwidth]{images/00406.jpeg}
\caption{图片 406}
\end{figure}
\newpage
\begin{figure}[htbp]
\centering
\includegraphics[width=0.8\textwidth]{images/00407.jpeg}
\caption{图片 407}
\end{figure}
\newpage
\begin{figure}[htbp]
\centering
\includegraphics[width=0.8\textwidth]{images/00408.jpeg}
\caption{图片 408}
\end{figure}
\newpage
\begin{figure}[htbp]
\centering
\includegraphics[width=0.8\textwidth]{images/00409.jpeg}
\caption{图片 409}
\end{figure}
\newpage
\begin{figure}[htbp]
\centering
\includegraphics[width=0.8\textwidth]{images/00410.jpeg}
\caption{图片 410}
\end{figure}
\newpage
\begin{figure}[htbp]
\centering
\includegraphics[width=0.8\textwidth]{images/00411.jpeg}
\caption{图片 411}
\end{figure}
\newpage
\begin{figure}[htbp]
\centering
\includegraphics[width=0.8\textwidth]{images/00412.jpeg}
\caption{图片 412}
\end{figure}
\newpage
\begin{figure}[htbp]
\centering
\includegraphics[width=0.8\textwidth]{images/00413.jpeg}
\caption{图片 413}
\end{figure}
\newpage
\begin{figure}[htbp]
\centering
\includegraphics[width=0.8\textwidth]{images/00414.jpeg}
\caption{图片 414}
\end{figure}
\newpage
\begin{figure}[htbp]
\centering
\includegraphics[width=0.8\textwidth]{images/00415.jpeg}
\caption{图片 415}
\end{figure}
\newpage
\begin{figure}[htbp]
\centering
\includegraphics[width=0.8\textwidth]{images/00416.jpeg}
\caption{图片 416}
\end{figure}
\newpage
\begin{figure}[htbp]
\centering
\includegraphics[width=0.8\textwidth]{images/00417.jpeg}
\caption{图片 417}
\end{figure}
\newpage
\begin{figure}[htbp]
\centering
\includegraphics[width=0.8\textwidth]{images/00418.jpeg}
\caption{图片 418}
\end{figure}
\newpage
\begin{figure}[htbp]
\centering
\includegraphics[width=0.8\textwidth]{images/00419.jpeg}
\caption{图片 419}
\end{figure}
\newpage
\begin{figure}[htbp]
\centering
\includegraphics[width=0.8\textwidth]{images/00420.jpeg}
\caption{图片 420}
\end{figure}
\newpage
\begin{figure}[htbp]
\centering
\includegraphics[width=0.8\textwidth]{images/00421.jpeg}
\caption{图片 421}
\end{figure}
\newpage
\begin{figure}[htbp]
\centering
\includegraphics[width=0.8\textwidth]{images/00422.jpeg}
\caption{图片 422}
\end{figure}
\newpage
\begin{figure}[htbp]
\centering
\includegraphics[width=0.8\textwidth]{images/00423.jpeg}
\caption{图片 423}
\end{figure}
\newpage
\begin{figure}[htbp]
\centering
\includegraphics[width=0.8\textwidth]{images/00424.jpeg}
\caption{图片 424}
\end{figure}
\newpage
\begin{figure}[htbp]
\centering
\includegraphics[width=0.8\textwidth]{images/00425.jpeg}
\caption{图片 425}
\end{figure}
\newpage
\begin{figure}[htbp]
\centering
\includegraphics[width=0.8\textwidth]{images/00426.jpeg}
\caption{图片 426}
\end{figure}
\newpage
\begin{figure}[htbp]
\centering
\includegraphics[width=0.8\textwidth]{images/00427.jpeg}
\caption{图片 427}
\end{figure}
\newpage
\begin{figure}[htbp]
\centering
\includegraphics[width=0.8\textwidth]{images/00428.jpeg}
\caption{图片 428}
\end{figure}
\newpage
\begin{figure}[htbp]
\centering
\includegraphics[width=0.8\textwidth]{images/00429.jpeg}
\caption{图片 429}
\end{figure}
\newpage
\begin{figure}[htbp]
\centering
\includegraphics[width=0.8\textwidth]{images/00430.jpeg}
\caption{图片 430}
\end{figure}
\newpage
\begin{figure}[htbp]
\centering
\includegraphics[width=0.8\textwidth]{images/00431.jpeg}
\caption{图片 431}
\end{figure}
\newpage
\begin{figure}[htbp]
\centering
\includegraphics[width=0.8\textwidth]{images/00432.jpeg}
\caption{图片 432}
\end{figure}
\newpage
\begin{figure}[htbp]
\centering
\includegraphics[width=0.8\textwidth]{images/00433.jpeg}
\caption{图片 433}
\end{figure}
\newpage
\begin{figure}[htbp]
\centering
\includegraphics[width=0.8\textwidth]{images/00434.jpeg}
\caption{图片 434}
\end{figure}
\newpage
\begin{figure}[htbp]
\centering
\includegraphics[width=0.8\textwidth]{images/00435.jpeg}
\caption{图片 435}
\end{figure}
\newpage
\begin{figure}[htbp]
\centering
\includegraphics[width=0.8\textwidth]{images/00436.jpeg}
\caption{图片 436}
\end{figure}
\newpage
\begin{figure}[htbp]
\centering
\includegraphics[width=0.8\textwidth]{images/00437.jpeg}
\caption{图片 437}
\end{figure}
\newpage
\begin{figure}[htbp]
\centering
\includegraphics[width=0.8\textwidth]{images/00438.jpeg}
\caption{图片 438}
\end{figure}
\newpage
\begin{figure}[htbp]
\centering
\includegraphics[width=0.8\textwidth]{images/00439.jpeg}
\caption{图片 439}
\end{figure}
\newpage
\begin{figure}[htbp]
\centering
\includegraphics[width=0.8\textwidth]{images/00440.jpeg}
\caption{图片 440}
\end{figure}
\newpage
\begin{figure}[htbp]
\centering
\includegraphics[width=0.8\textwidth]{images/00441.jpeg}
\caption{图片 441}
\end{figure}
\newpage
\begin{figure}[htbp]
\centering
\includegraphics[width=0.8\textwidth]{images/00442.jpeg}
\caption{图片 442}
\end{figure}
\newpage
\begin{figure}[htbp]
\centering
\includegraphics[width=0.8\textwidth]{images/00443.jpeg}
\caption{图片 443}
\end{figure}
\newpage
\begin{figure}[htbp]
\centering
\includegraphics[width=0.8\textwidth]{images/00444.jpeg}
\caption{图片 444}
\end{figure}
\newpage
\begin{figure}[htbp]
\centering
\includegraphics[width=0.8\textwidth]{images/00445.jpeg}
\caption{图片 445}
\end{figure}
\newpage
\begin{figure}[htbp]
\centering
\includegraphics[width=0.8\textwidth]{images/00446.jpeg}
\caption{图片 446}
\end{figure}
\newpage
\begin{figure}[htbp]
\centering
\includegraphics[width=0.8\textwidth]{images/00447.jpeg}
\caption{图片 447}
\end{figure}
\newpage
\begin{figure}[htbp]
\centering
\includegraphics[width=0.8\textwidth]{images/00448.jpeg}
\caption{图片 448}
\end{figure}
\newpage
\begin{figure}[htbp]
\centering
\includegraphics[width=0.8\textwidth]{images/00449.jpeg}
\caption{图片 449}
\end{figure}
\newpage
\begin{figure}[htbp]
\centering
\includegraphics[width=0.8\textwidth]{images/00450.jpeg}
\caption{图片 450}
\end{figure}
\newpage
\begin{figure}[htbp]
\centering
\includegraphics[width=0.8\textwidth]{images/00451.jpeg}
\caption{图片 451}
\end{figure}
\newpage
\begin{figure}[htbp]
\centering
\includegraphics[width=0.8\textwidth]{images/00452.jpeg}
\caption{图片 452}
\end{figure}
\newpage
\begin{figure}[htbp]
\centering
\includegraphics[width=0.8\textwidth]{images/00453.jpeg}
\caption{图片 453}
\end{figure}
\newpage
\begin{figure}[htbp]
\centering
\includegraphics[width=0.8\textwidth]{images/00454.jpeg}
\caption{图片 454}
\end{figure}
\newpage
\begin{figure}[htbp]
\centering
\includegraphics[width=0.8\textwidth]{images/00455.jpeg}
\caption{图片 455}
\end{figure}
\newpage
\begin{figure}[htbp]
\centering
\includegraphics[width=0.8\textwidth]{images/00456.jpeg}
\caption{图片 456}
\end{figure}
\newpage
\begin{figure}[htbp]
\centering
\includegraphics[width=0.8\textwidth]{images/00457.jpeg}
\caption{图片 457}
\end{figure}
\newpage
\begin{figure}[htbp]
\centering
\includegraphics[width=0.8\textwidth]{images/00458.jpeg}
\caption{图片 458}
\end{figure}
\newpage
\begin{figure}[htbp]
\centering
\includegraphics[width=0.8\textwidth]{images/00459.jpeg}
\caption{图片 459}
\end{figure}
\newpage
