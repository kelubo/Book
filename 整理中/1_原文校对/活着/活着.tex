% 活着
% 活着.tex

\documentclass[12pt,UTF8]{ctexbook}

% 设置纸张信息。
\usepackage[a4paper,twoside]{geometry}
\geometry{
	left=25mm,
	right=25mm,
	bottom=25.4mm,
	bindingoffset=10mm
}

% 设置字体,并解决显示难检字问题。
\xeCJKsetup{AutoFallBack=true}
\setCJKmainfont{SimSun}[BoldFont=SimHei, ItalicFont=KaiTi, FallBack=SimSun-ExtB]

% 目录 chapter 级别加点(.)。
\usepackage{titletoc}
\titlecontents{chapter}[0pt]{\vspace{3mm}\bf\addvspace{2pt}\filright}{\contentspush{\thecontentslabel\hspace{0.8em}}}{}{\titlerule*[8pt]{.}\contentspage}

% 设置 part 和 chapter 标题格式。
\ctexset{
	part/name= {第,卷},
	part/number={\chinese{part}},
	chapter/name={第,篇},
	chapter/number={\chinese{chapter}}
}

% 图片相关设置。
\usepackage{graphicx}
\graphicspath{{Images/}}

% 设置署名格式。
\newenvironment{shuming}{\hfill\zihao{4}}

% 注脚每页重新编号,避免编号过大。
\usepackage[perpage]{footmisc}

\title{\heiti\zihao{0} 活着}
\author{余华}
\date{}

\begin{document}

\maketitle
\tableofcontents

\frontmatter

\chapter{韩文版自序}

我不知道应该怎样来解释这一部作品,这样的任务交给作者去完成是十分困难的,但是我愿意试一试,我希望韩国的读者能够容忍我的冒险。

这部作品的题目叫《活着》,作为一个词语,“活着”在我们中国的语言里充满了力量,它的力量不是来自于喊叫,也不是来自于进攻,而是忍受,去忍受生命赋予我们的责任,去忍受现实给予我们的幸福和苦难、无聊和平庸。作为一部作品,《活着》讲述了一个人和他的命运之间的友情,这是最为感人的友情,因为他们互相感激,同时也互相仇恨;他们谁也无法抛弃对方,同时谁也没有理由抱怨对方。他们活着时一起走在尘土飞扬的道路上,死去时又一起化作雨水和泥土。与此同时,《活着》还讲述了人如何去承受巨大的苦难,就像中国的一句成语:千钧一发。让一根头发去承受三万斤的重压,它没有断。我相信,《活着》还讲述了眼泪的广阔和丰富;讲述了绝望的不存在;讲述了人是为了活着本身而活着,而不是为了活着之外的任何事物而活着。当然,《活着》也讲述了我们中国人这几十年是如何熬过来的。我知道,《活着》所讲述的远不止这些。文学就是这样,它讲述了作家意识到的事物,同时也讲述了作家所没有意识到的,读者就是这时候站出来发言的。
~\\

\begin{shuming}
余华
\end{shuming}

\begin{shuming}
北京,1996年10月17日
\end{shuming}

\chapter{前言}

一位真正的作家永远只为内心写作,只有内心才会真实地告诉他,他的自私、他的高尚是多么突出。内心让他真实地了解自己,一旦了解了自己也就了解了世界。很多年前我就明白了这个原则,可是要捍卫这个原则必须付出艰辛的劳动和长时期的痛苦,因为内心并非时时刻刻都是敞开的,它更多的时候倒是封闭起来,于是只有写作,不停地写作才能使内心敞开,才能使自己置身于发现之中,就像日出的光芒照亮了黑暗,灵感这时候才会突然来到。

长期以来,我的作品都是源出于和现实的那一层紧张关系。我沉湎于想象之中,又被现实紧紧控制,我明确感受着自我的分裂,我无法使自己变得纯粹,我曾经希望自己成为一位童话作家,要不就是一位实实在在作品的拥有者,如果我能够成为这两者中的任何一个,我想我内心的痛苦将会轻微得多,可是与此同时我的力量也会削弱很多。

事实上我只能成为现在这样的作家,我始终为内心的需要而写作,理智代替不了我的写作,正因为此,我在很长一段时间是一个愤怒和冷漠的作家。

这不只是我个人面临的困难,几乎所有优秀的作家都处于和现实的紧张关系中,在他们笔下,只有当现实处于遥远状态时,他们作品中的现实才会闪闪发亮。应该看到,这过去的现实虽然充满魅力,可它已经蒙上了一层虚幻的色彩,那里面塞满了个人想象和个人理解。真正的现实,也就是作家生活中的现实,是令人费解和难以相处的。

作家要表达与之朝夕相处的现实,他常常会感到难以承受,蜂拥而来的真实几乎都在诉说着丑恶和阴险,怪就怪在这里,为什么丑恶的事物总是在身边,而美好的事物却远在海角。换句话说,人的友爱和同情往往只是作为情绪来到,而相反的事实则是伸手便可触及。正像一位诗人所表达的:人类无法忍受太多的真实。也有这样的作家,一生都在解决自我和现实的紧张关系,福克纳是最为成功的例子,他找到了一条温和的途径,他描写中间状态的事物,同时包容了美好与丑恶,他将美国南方的现实放到了历史和人文精神之中,这是真正意义上的文学现实,因为它连接着过去和将来。

一些不成功的作家也在描写现实,可他们笔下的现实说穿了只是一个环境,是固定的,死去的现实,他们看不到人是怎样走过来的,也看不到怎样走去。当他们在描写斤斤计较的人物时,我们会感到作家本人也在斤斤计较,这样的作家是在写实在的作品,而不是现实的作品。

前面已经说过,我和现实关系紧张,说得严重一些,我一直是以敌对的态度看待现实。随着时间的推移,我内心的愤怒渐渐平息,我开始意识到一位真正的作家所寻找的是真理,是一种排斥道德判断的真理。作家的使命不是发泄,不是控诉或者揭露,他应该向人们展示高尚。这里所说的高尚不是那种单纯的美好,而是对一切事物理解之后的超然,对善与恶一视同仁,用同情的目光看待世界。

正是在这样的心态下,我听到了一首美国民歌《老黑奴》,歌中那位老黑奴经历了一生的苦难,家人都先他而去,而他依然友好地对待世界,没有一句抱怨的话。这首歌深深打动了我,我决定写下一篇这样的小说,就是这篇《活着》,写人对苦难的承受能力,对世界乐观的态度。写作过程让我明白,人是为活着本身而活着的,而不是为活着之外的任何事物所活着。我感到自己写下了高尚的作品。

\mainmatter

\chapter{1}

我比现在年轻十岁的时候,获得了一个游手好闲的职业,去乡间收集民间歌谣。那一年的整个夏天,我如同一只乱飞的麻雀,游荡在知了和阳光充斥的村舍田野。我喜欢喝农民那种带有苦味的茶水,他们的茶桶就放在田埂的树下,我毫无顾忌地拿起漆满茶垢的茶碗舀水喝,还把自己的水壶灌满,与田里干活的男人说上几句废话,在姑娘因我而起的窃窃私笑里扬长而去。我曾经和一位守着瓜田的老人聊了整整一个下午,这是我有生以来瓜吃得最多的一次,当我站起来告辞时,突然发现自己像个孕妇一样步履艰难了。然后我与一位当上了祖母的女人坐在门槛上,她编着草鞋为我唱了一支《十月怀胎》。我最喜欢的是傍晚来到时,坐在农民的屋前,看着他们将提上的井水泼在地上,压住蒸腾的尘土,夕阳的光芒在树梢上照射下来,拿一把他们递过来的扇子,尝尝他们和盐一样咸的咸菜,看看几个年轻女人,和男人们说着话。

我头戴宽边草帽,脚上穿着拖鞋,一条毛巾挂在身后的皮带上,让它像尾巴似的拍打着我的屁股。我整日张大嘴巴打着呵欠,散漫地走在田间小道上,我的拖鞋吧哒吧哒,把那些小道弄得尘土飞扬,仿佛是车轮滚滚而过时的情景。

我到处游荡,已经弄不清楚哪些村庄我曾经去过,哪些我没有去过。我走近一个村子时,常会听到孩子的喊叫:

“那个老打呵欠的人又来啦。”

于是村里人就知道那个会讲荤故事会唱酸曲的人又来了。其实所有的荤故事所有的酸曲都是从他们那里学来的,我知道他们全部的兴趣在什么地方,自然这也是我的兴趣。我曾经遇到一个哭泣的老人,他鼻青眼肿地坐在田埂上,满腹的悲哀使他变得十分激动,看到我走来他仰起脸哭声更为响亮,我问他是谁把他打成这样的?他手指挖着裤管上的泥巴,愤怒地告诉我是他那不孝的儿子,当我再问为何打他时,他支支吾吾说不清楚了,我就立刻知道他准是对儿媳干了偷鸡摸狗的勾当。还有一个晚上我打着手电赶夜路时,在一口池塘旁照到了两段赤裸的身体,一段压在另一段上面,我照着的时候两段身体纹丝不动,只是有一只手在大腿上轻轻搔痒,我赶紧熄灭手电离去。在农忙的一个中午,我走进一家敞开大门的房屋去找水喝,一个穿短裤的男人神色慌张地挡住了我,把我引到井旁,殷勤地替我打上来一桶水,随后又像耗子一样窜进了屋里。这样的事我屡见不鲜,差不多和我听到的歌谣一样多,当我望着到处都充满绿色的土地时,我就进一步明白庄稼为何长得如此旺盛。

那个夏天我还差一点谈情说爱,我遇到了一位赏心悦目的农村女孩,她黝黑的脸蛋至今还在我眼前闪闪发光。我见到她时,她卷起裤管坐在河边的青草上,摆弄着一根竹竿在照看一群肥硕的鸭子。这个十六七岁的女孩,羞怯地与我共同度过了一个炎热的下午,她每次露出笑容时都要深深地低下头去,我看着她偷偷放下卷起的裤管,又怎样将自己的光脚丫子藏到草丛里去。那个下午我信口开河,向她兜售如何带她外出游玩的计划,这个女孩又惊又喜。我当初情绪激昂,说这些也是真心实意。我只是感到和她在一起身心愉快,也不去考虑以后会是怎样。可是后来,当她三个强壮如牛的哥哥走过来时,我才吓一跳,我感到自己应该逃之夭夭了,否则我就会不得不娶她为妻。

我遇到那位名叫福贵的老人时,是夏天刚刚来到的季节。那天午后,我走到了一棵有着茂盛树叶的树下,田里的棉花已被收起,几个包着头巾的女人正将棉秆拔出来,她们不时抖动着屁股摔去根须上的泥巴。我摘下草帽,从身后取过毛巾擦起脸上的汗水,身旁是一口在阳光下泛黄的池塘,我就靠着树干面对池塘坐了下来,紧接着我感到自己要睡觉了,就在青草上躺下来,把草帽盖住脸,枕着背包在树荫里闭上了眼睛。

这位比现在年轻十岁的我,躺在树叶和草丛中间,睡了有两个小时。其间有几只蚂蚁爬到了我的腿上,我沉睡中的手指依然准确地将它们弹走。后来仿佛是来到了水边,一位老人撑着竹筏在远处响亮地吆喝。我从睡梦里挣脱而出,吆喝声在现实里清晰地传来,我起身后,看到近旁田里一个老人正在开导一头老牛。

犁田的老牛或许已经深感疲倦,它低头伫立在那里,后面赤裸着脊背扶犁的老人,对老牛的消极态度似乎不满,我听到他嗓音响亮地对牛说道:

“做牛耕田,做狗看家,做和尚化缘,做鸡报晓,做女人织布,哪只牛不耕田?这可是自古就有的道理,走呀,走呀。”

疲倦的老牛听到老人的吆喝后,仿佛知错般的抬起了头,拉着犁往前走去。

我看到老人的脊背和牛背一样黝黑,两个进入垂暮的生命将那块古板的田地耕得哗哗翻动,犹如水面上掀起的波浪。随后,我听到老人粗哑却令人感动的嗓音,他唱起了旧日的歌谣,先是依依呀啦呀唱出长长的引子,接着出现两句歌词——

\begin{quotation}
皇帝招我做女婿,

路远迢迢我不去。
\end{quotation}

因为路途遥远,不愿去做皇帝的女婿。老人的自鸣得意让我失声而笑。可能是牛放慢了脚步,老人又吆喝起来:

“二喜,有庆不要偷懒;家珍,凤霞耕得好;苦根也行啊。”

一头牛竟会有这么多名字?我好奇地走到田边,问走近的老人:

“这牛有多少名字?”

老人扶住犁站下来,他将我上下打量一番后问:

“你是城里人吧?”

“是的。”我点点头。

老人得意起来,“我一眼就看出来了。”

我说:“这牛究竟有多少名字?”

老人回答:“这牛叫福贵,就一个名字。”

“可你刚才叫了几个名字?”

“噢——”老人高兴地笑起来,他神秘地向我招招手,当我凑过去时,他欲说又止,他看到牛正抬着头,就训斥它:

“你别偷听,把头低下。”

牛果然低下了头,这时老人悄声对我说:

“我怕它知道只有自己在耕田,就多叫出几个名字去骗它,它听到还有别的牛也在耕田,就不会不高兴,耕田也就起劲啦。”

老人黝黑的脸在阳光里笑得十分生动,脸上的皱纹欢乐地游动着,里面镶满了泥土,就如布满田间的小道。

这位老人后来和我一起坐在了那棵茂盛的树下,在那个充满阳光的下午,他向我讲述了自己。

四十多年前,我爹常在这里走来走去,他穿着一身黑颜色的绸衣,总是把双手背在身后,他出门时常对我娘说:

“我到自己的地上去走走。”

我爹走在自己的田产上,干活的佃户见了,都要双手握住锄头恭敬地叫一声:

“老爷。”

我爹走到了城里,城里人见了都叫他先生。我爹是很有身份的人,可他拉屎时就像个穷人了。他不爱在屋里床边的马桶上拉屎,跟牲畜似的喜欢到野地里去拉屎。每天到了傍晚的时候,我爹打着饱嗝,那声响和青蛙叫唤差不多,走出屋去,慢吞吞地朝村口的粪缸走去。

走到了粪缸旁,他嫌缸沿脏,就抬脚踩上去蹲在上面。我爹年纪大了,屎也跟着老了,出来不容易,那时候我们全家人都会听到他在村口嗷嗷叫着。

几十年来我爹一直这样拉屎,到了六十多岁还能在粪缸上一蹲就是半晌,那两条腿就和鸟爪一样有劲。我爹喜欢看着天色慢慢黑下来,罩住他的田地。我女儿凤霞到了三四岁,常跑到村口去看她爷爷拉屎,我爹毕竟年纪大了,蹲在粪缸上腿有些哆嗦,凤霞就问他:

“爷爷,你为什么动呀?”

我爹说:“是风吹的。”

那时候我们家境还没有败落,我们徐家有一百多亩地,从这里一直到那边工厂的烟囱,都是我家的。我爹和我,是远近闻名的阔老爷和阔少爷,我们走路时鞋子的声响,都像是铜钱碰来撞去的。我女人家珍,是城里米行老板的女儿,她也是有钱人家出身的。有钱人嫁给有钱人,就是把钱堆起来,钱在钱上面哗哗地流,这样的声音我有四十年没有听到了。

我是我们徐家的败家子,用我爹的话说,我是他的孽子。我念过几年私塾,穿长衫的私塾先生叫我念一段书时,是我最高兴的。我站起来,拿着本线装的《千字文》,对私塾先生说:

“好好听着,爹给你念一段。”

年过花甲的私塾先生对我爹说:

“你家少爷长大了准能当个二流子。”

我从小就不可救药,这是我爹的话。私塾先生说我(是/说)朽木不可雕也。现在想想他们都说对了,当初我可不这么想,我想我有钱呵,我是徐家仅有的一根香火,我要是灭了,徐家就得断子绝孙。

上私塾时我从来不走路,都是我家一个雇工背着我去,放学时他已经恭恭敬敬地弯腰蹲在那里了,我骑上去后拍拍雇工的脑袋,说一声:

“长根,跑呀。”

雇工长根就跑起来,我在上面一颠一颠的,像是一只在树梢上的麻雀。我说一声:

“飞呀。”

长根就一步一跳,做出一副飞的样子。

我长大以后喜欢往城里跑,常常是十天半月不回家。我穿着白色的丝绸衣衫,头发抹得光滑透亮,往镜子前一站,我看到自己满脑袋的黑油漆,一(副/付)有钱人的样子。

我爱往妓院钻,听那些风骚的女人整夜叽叽喳喳和哼哼哈哈,那些声音听上去像是在给我挠痒痒。做人呵,一旦嫖上以后,也就免不了要去赌。这个嫖和赌,就像是胳膊和肩膀连在一起,怎么都分不开。后来我更喜欢赌博了,嫖妓只是为了轻松一下,就跟水喝多了要去方便一下一样,说白了就是撒尿。赌博就完全不一样了,我是又痛快又紧张,特别是那个紧张,有一股叫我说不出来的舒坦。以前我是过一天和尚撞一天钟,整天有气无力,每天早晨醒来犯愁的就是这一天该怎么打发。我爹常常唉声叹气,训斥我没有光耀祖宗。我心想光耀祖宗也不是非我莫属,我对自己说,凭什么让我放着好端端的日子不过,去想光耀祖宗这些累人的事。再说,我爹年轻时也和我一样,我家祖上有两百多亩地,到他手上一折腾就剩一百多亩了。我对爹说:

“你别犯愁啦,我儿子会光耀祖宗的。”

总该给下一辈留点好事吧。我娘听了这话吃吃笑,她偷偷告诉我:我爹年轻时也这么对我爷爷说过。我心想就是嘛,他自己干不了的事硬要我来干,我怎么会答应。那时候我儿子有庆还没出来,我女儿凤霞刚好四岁。家珍怀着有庆有六个月了,自然有些难看,走路时裤裆里像是夹了个馒头似的一撇一撇,两只脚不往前往横里跨,我嫌弃她,对她说:

“你呀,风一吹肚子就要大上一圈。”

家珍从不顶撞我,听了这糟蹋她的话,她心里不乐意也只是轻轻说一句:

“又不是风吹大的。”

自从我赌博上以后,我倒还真想光耀祖宗了,想把我爹弄掉的一百多亩地挣回来。那些日子爹问我在城里鬼混些什么,我对他说:

“现在不鬼混啦,我在做生意。”

他问:“做什么生意?”

他一听就火了,他年轻时也这么回答过我爷爷。他知道我是在赌博,脱下布鞋就朝我打来,我左躲右藏,心想他打几下就该完了吧。可我这个平常只有咳嗽才有力气的爹,竟然越打越凶了。我又不是一只苍蝇,让他这么拍来拍去。我一把捏住他的手,说道:

“爹,你他娘的算了吧。老子看在你把我弄出来的份上让让你,你他娘的就算了吧。”

我捏住爹的右手,他又用左手脱下右脚的布鞋,还想打我。我又捏住他的左手,这样他就动弹不得了,他气得哆嗦了半晌,才喊出一声:

“孽子。”

我说:“去你娘的。”

双手一推,他就跌坐到墙角里去了。

我年轻时吃喝嫖赌,什么浪荡的事都干过。我常去的那家妓院是单名,叫青楼。里面有个胖胖的妓女很招我喜爱,她走路时两片大屁股就像挂在楼前的两只灯笼,晃来晃去。她躺到床上一动一动时,压在上面的我就像睡在船上,在河水里摇呀摇呀。我经常让她背着我去逛街,我骑在她身上像是骑在一匹马上。

我的丈人,米行的陈老板,穿着黑色的绸衫站在柜台后面。我每次从那里经过时,都要揪住妓女的头发,让她停下,脱帽向丈人致礼:

“近来无恙?”

我丈人当时的脸就和松花蛋一样,我呢,嘻嘻笑着过去了。后来我爹说我丈人几次都让我气病了,我对爹说:

“别哄我啦,你是我爹都没气成病。他自己生病凭什么往我身上推?”

他怕我,我倒是知道的。我骑在妓女身上经过他的店门时,我丈人身手极快,像只耗子呼地一下窜到里屋去了。他不敢见我,可当女婿的路过丈人店门总该有个礼吧。我就大声嚷嚷着向逃窜的丈人请安。

最风光的那次是小日本投降后,国军准备进城收复失地。那天可真是热闹,城里街道两旁站满了人,手里拿着小彩旗,商店都斜着插出来青天白日旗,我丈人米行前还挂了一幅两扇门板那么大的蒋介石像,米行的三个伙计都站在蒋介石左边的口袋下。

那天我在青楼里赌了一夜,脑袋昏昏沉沉像是肩膀上扛了一袋米,我想着自己有半个来月没回家了,身上的衣服一股酸臭味,我就把那个胖大妓女从床上拖起来,让她背着我回家,叫了抬轿子跟在后面,我到了家好让她坐轿子回青楼。

那妓女嘟嘟哝哝背着我往城门走,说什么雷公不打睡觉人,才睡下就被我叫醒,说我心肠黑。我把一个银元往她胸口灌进去,就把她的嘴堵上了。走近了城门,一看到两旁站了那么多人,我的精神一下子上来了。

我丈人是城里商会的会长,我很远就看到他站在街道中央喊:

“都站好了,都站好了,等国军一到,大家都要拍手,都要喊。”

有人看到了我,就嘻嘻笑着喊:

“来啦,来啦。”

我丈人还以为是国军来了,赶紧闪到一旁。我两条腿像是夹马似的夹了夹妓女,对她说:

“跑呀,跑呀。”

在两旁人群的哄笑里,妓女呼哧呼哧背着我小跑起来,嘴里骂道:

“夜里压我,白天骑我,黑心肠的,你是逼我往死里跑。”

我咧着嘴频频向两旁哄笑的人点头致礼,来到丈人近前,我一把扯住妓女的头发:

“站住,站住。”

妓女哎(唷/呦)叫了一声站住脚,我大声对丈人说:

“岳父大人,女婿给你请个早安。”

那次我实实在在地把我丈人的脸丢尽了,我丈人当时傻站在那里,嘴唇一个劲地哆嗦,半晌才沙哑地说一声:

“祖宗,你快走吧。”

那声音听上去都不像是他的了。

我女人家珍当然知道我在城里这些花花绿绿的事,家珍是个好女人,我这辈子能娶上这么一个贤惠的女人,是我前世做狗吠叫了一辈子换来的。家珍对我从来都是逆来顺受,我在外面胡闹,她只是在心里打鼓,从不说我什么,和我娘一样。

我在城里闹腾得实在有些过分,家珍心里当然有一团乱麻,乱糟糟的不能安分。有一天我从城里回到家中,刚刚坐下,家珍就笑盈盈地端出四样菜,摆在我面前,又给我斟满了酒,自己在我身旁坐下来待候我吃喝。她笑盈盈的样子让我觉得奇怪,不知道她遇上了什么好事,我左思右想也想不出这天是什么日子。我问她,她不说,就是笑盈盈地看着我。

那四样菜都是蔬菜,家珍做得各不相同,可吃到下面都是一块差不多大小的猪肉。起先我没怎么在意,吃到最后一碗菜,底下又是一块猪肉。我一愣,随后我就嘿嘿笑了起来。我明白了家珍的意思,她是在开导我:女人看上去各不相同,到下面都是一样的。我对家珍说:

“这道理我也知道。”

道理我也知道,可看到(上面)长得不一样的女人,我心里想的就是不一样,这实在是没办法的事。

家珍就是这样一个女人,心里对我不满,脸上不让我看出来,弄些转弯抹角的点子来敲打我。我偏偏是软硬不吃,我爹的布鞋和家珍的菜都管不住我的腿,我就是爱往城里跑,爱往妓院钻。还是我娘知道我们男人心里想什么,她对家珍说:

“男人都是馋嘴的猫。”

我娘说这话不只是为我开脱,还揭了我爹的老底。我爹坐在椅子里,一听这话眼睛就眯成了两条门缝,嘿嘿笑了一下。我爹年轻时也不检点,他是老了干不动了才老实起来。

我赌博时也在青楼,常玩的是麻将、牌九和骰子。我每赌必输,越输我就越想把我爹年轻时输掉的一百多亩地赢回来。刚开始输了我当场给钱,没钱就去偷我娘和家珍的手饰,连我女儿凤霞的金项圈也偷了去。后来我干脆赊帐,债主们都知道我的家境,让我赊帐。自从赊帐以后,我就不知道自己输了有多少,债主也不提醒我,暗地里天天都在算计着我家那一百多亩地。

一直到解放以后,我才知道赌博的赢家都是做了手脚的,难怪我老输不赢,他们是挖了个坑让我往里面跳。那时候青楼里有一位沈先生,年纪都快到六十(岁)了,眼睛还和猫眼似的贼亮,穿着蓝布长衫,腰板挺着笔直,平常时候总是坐在角落里,闭着眼睛像是在打盹。等到牌桌上的赌注越下越大,沈先生才咳嗽几声,慢悠悠地走过来,选一位置站着看,看了一会便有人站起来让位:

“沈先生,这里坐。”

沈先生撩起长衫坐下,对另三位赌徒说:

“请。”

青楼里的人从没见(过/到)沈先生输过,他那双青筋突暴的手洗牌时,只听到哗哗的风声,那付牌在他手中忽长忽短,唰唰地进进出出,看得我眼睛都酸了。

有一次沈先生喝醉了酒,对我说:

“赌博全靠一双眼睛一双手,眼睛要练成爪子一样,手要练成泥鳅那样滑。”

小日本投降那年,龙二来了,龙二说话时南腔北调,光听他的口音,就知道这人不简单,是闯荡过很多地方,见过大世面的人。龙二不穿长衫,一身白绸衣,和他同来的还有两个人,帮他提着(两只)很大的柳条箱。

那年沈先生和龙二的赌局,实在是精彩,青楼的赌厅里挤满了人,沈先生和他们三个人赌。龙二身后站着一个跑堂的,托着一盘干毛巾,龙二不时取过一块毛巾擦手。他不拿湿毛巾拿干毛巾擦手,我们看了都觉得稀奇。他擦手时那副派头像是刚吃完了饭似的。起先龙二一直输,他看上去还满不在乎,倒是他带来的两个人沉不住气,一个骂骂咧咧,一个唉声叹气。沈先生一直赢,可脸上一点赢的意思都没有,沈先生皱着眉头,像是输了很多似的。他脑袋垂着,眼睛却跟钉子似的钉在龙二那双手上。沈先生年纪大了,半个晚上赌下来,就开始喘粗气,额头上汗水渗了出来,沈先生说:

“一局定胜负吧。”

龙二从盘子里取过最后一块毛巾,擦着手说:

“行啊。”

他们把所有的钱都压在了桌上,钱差不多把桌面占满了,只在中间留个空。每个人发了五张牌,亮出四张后,龙二的两个伙伴立刻泄气了,把牌一推说:

“完啦,又输了。”

龙二赶紧说:“没输,你们赢啦。”

说着龙二亮出最后那张牌,是黑桃A,他的两个伙伴一看立刻嘿嘿笑了。其实沈先生最后那张牌也是黑桃A,他是三A带两K,龙二一个伙伴是三Q带俩J。龙二抢先亮出了黑桃A,沈先生怔了半晌,才把手中的牌一收说:

“我输了。”

龙二的黑桃A和沈先生的都是从袖管里换出来的,一副牌不能有两张黑桃A,龙二抢了先,沈先生心里明白也只能认输。那是我们第一次看到沈先生输,沈先生手推桌子站起来,向龙二他们作了个揖,转过身来往外走,走到门口微笑着说:

“我老了。”

后来再没人见过沈先生,听说那天天刚亮,他就坐着轿子走了。

沈先生一走,龙二成了这里的赌博师傅。龙二和沈先生不一样,沈先生是只赢不输,龙二是赌注小常输,赌注大就没见他输过了。我在青楼常和龙二他们赌,有输有赢,所以我总觉得自己没怎么输,其实我赢的都是小钱,输掉的倒是大钱,我还蒙在鼓里,以为自己马上就要光耀祖宗了。

===============================================我最后一次赌博时,家珍来了,那时候天都快黑了,这是家珍后来告诉我的,我当初根本不知道天是亮着还是要黑了。家珍挺了个大肚子找到青楼来了,我儿子有庆在他娘肚子里长到七、八月个月了。家珍找到了我,一声不吭地跪在我面前,起先我没看到她,那天我手气特别好,掷出的骰子十有八九是我要的点数,坐在对面的龙二一看点数嘿嘿一笑说:“兄弟我又栽了。”龙二摸牌把沈先生赢了之后,青楼里没人敢和他摸牌了,我也不敢,我和龙二赌都是用骰子,就是骰子龙二玩的也很地道,他常赢少输,可那天他栽到我手里了,接连地输给我。

他嘴里叼着烟卷,眼睛眯缝着像是什么事都没有,每次输了都还嘿嘿一笑,两条瘦胳膊把钱推过来时却是一百个不愿意。

我想龙二你也该惨一次了。人都是一样的,手伸进别人口袋里掏钱时那个眉开眼笑,轮到自己给钱了一个个都跟哭丧一样。我正高兴着,有人扯了扯我的衣服,低头一看是自己的女人。看到家珍跪着我就火了,心想我儿子还没出来就跪着了,这太不吉利。我就对家珍说:“起来,起来,你他娘的给我起来。”家珍还真听话,立刻站了起来。我说:“你来干什么,还不快给我回去。”说完我就不管她了,看着龙二将骰子捧在手心里跟拜佛似的摇了几下,他一掷出脸色就难看了,说道:“摸过女人屁股就是手气不好。”我一看自己又赢了,就说:“龙二,你去洗洗手吧。”龙二嘿嘿一笑,说道:“你把嘴巴子抹干净了再说话。”家珍又扯了扯我的衣服,我一看,她又跪到地上。家珍细声细气地说:“你跟我回去。”要我跟一个女人回去?家珍这不是存心出我的丑?我的怒气一下子上来了,我看看龙二他们,他们都笑着看我,我对家珍吼道:“你给我滚回去。”家珍还是说:“你跟我回去。”我给了她两巴掌,家珍的脑袋像是拨郎鼓那样摇晃了几下。挨了我的打,她还是跪在那里,说:“你不回去,我就不站起来。”现在想起来叫我心疼啊,我年轻时真是个乌龟王八蛋。这么好的女人,我对她又打又踢。我怎么打她,她就是跪着不起来,打到最后连我自己都觉得没趣了,家珍头发披散眼泪汪汪地捂着脸。我就从赢来的钱里抓出一把,给了旁边站着的两个人,让他们把家珍拖出去,我对他们说:“拖得越远越好。”家珍被拖出去时,双手紧紧捂着凸起的肚子,那里面有我的儿子呵,家珍没喊没叫,被拖到了大街上,那两个人扔开她后,她就扶着墙壁站起来,那时候天完全黑了,她一个人慢慢往回走。后来我问她,她那时是不是恨死我了,她摇摇头说:“没有。”我的女人抹着眼泪走到她
------------

爹米行门口,站了很长时间,她看到她爹的脑袋被煤油灯的亮光印在墙上,她知道他是在清点帐目。她站在那里呜呜哭了一会,就走开了。

家珍那天晚上走了十多里夜路回到了我家。她一个孤身女人,又怀着七个多月的有庆,一路上到处都是狗吠,下过一场大雨的路又坑坑洼洼。

早上几年的时候,家珍还是一个女学生。那时候城里有夜校了,家珍穿着月白色的旗袍,提着一盏小煤油灯,和几个女伴去上学。我是在拐弯处看到她,她一扭一扭地走过来,高跟鞋敲在石板路上,滴滴答答像是在下雨,我眼睛都看得不会动了,家珍那时候长得可真漂亮,头发齐齐地挂到耳根,走去时旗袍在腰上一皱一皱,我当时就在心里想,我要她做我的女人。

家珍她们嘻嘻说着话走过去后,我问一个坐在地上的鞋匠:“那是谁家的女儿?”鞋匠说:“是陈记米行的千金。”我回家后马上对我娘说:“快去找个媒人,我要把城里米行陈老板的女儿娶过来。”家珍那天晚上被拖走后,我就开始倒霉了,连着输了好几把,眼看着桌上小山坡一样堆起的钱,像洗脚水倒了出去。

龙二嘿嘿笑个不停,那张脸都快笑烂了。那次我一直赌到天亮,赌得我头晕眼花,胃里直往嘴上冒臭气。最后一把我压上了平生最大的赌注,用唾沫洗洗手,心想千秋功业全在此一掷了。我正要去抓骰子,龙二伸手挡了挡说:“慢着。”龙二向一个跑堂挥挥手说:“给徐家少爷拿块热毛巾来。”那时候旁边看赌的人全回去睡觉了,只剩下我们几个赌的,另两个人是龙二带来的。我是后来才知道龙二买通了那个跑堂,那跑堂将热毛巾递给我,我拿着擦脸时,龙二偷偷换了一付骰子,换上来的那付骰子龙二做了手脚。我一点都没察觉,擦完脸我把毛巾往盘子里一扔,拿起骰子拼命摇了三下,掷出去一看,还好,点数还挺大的。

轮到龙二时,龙二将那颗骰子放在七点上,这小子伸出手掌使劲一拍,喊了一**“七点。”那颗骰子里面挖空了灌了水银,龙二这么一拍,水银往下沉,抓起一掷,一头重了滚几下就会停在七点上。

我一看那颗骰子果然是七点,脑袋嗡的一下,这次输惨了。继而一想反正可以赊帐,日后总有机会赢回来,便宽了宽心,站起来对龙二说:“先记上吧。”龙二摆摆手让我坐下,他说:“不能再让你赊帐了,你把你家一百多亩地全输光了。再赊帐,你拿什么来还?”我听后一个呵欠没打完猛地收回,连声说:“不会,不会。”龙二和另两个债主就拿出帐簿,一五一十给我算起来,龙二拍拍我凑过去的脑袋,对我说:“少爷,看清楚了吗?这可都是你签字画押的。”我才知道半年前就欠上他们了,半年下来我把祖辈留下的家产全输光了。算到一半,我对龙二说:“别算了。”我重新站起来,像只瘟鸡似的走出了青楼,那时候天完全亮了,我就站在街上,都不知道该往哪里走。有一个提着一篮豆腐的熟人看到我后响亮地喊了一声:“早啊,徐家少爷。”他的喊声吓了我一跳,我呆呆地看着他。他笑眯眯地说:“瞧你这样子,都成药渣了。”他还以为我是被那些女人给折腾的,他不知道我破产了,我和一个雇工一样穷了。我苦笑着看他走远,心想还是别在这里站着,就走动起来。

我走到丈人米行那边时,两个伙计正在卸门板,他们看到我后嘻嘻笑了一下,以为我又会过去向我丈人大声请安,我哪还有这个胆量?我把脑袋缩了缩,贴着另一端的房屋赶紧走了过去。我听到老丈人在里面咳嗽,接着呸的一声一口痰吐在了地上。

我就这样迷迷糊糊地走到了城外,有一阵子我竟忘了自己输光家产这事,脑袋里空空荡荡,像是被捅过的马蜂窝。到了城外,看到那条斜着伸过去的小路,我又害怕了,我想接下去该怎么办呢?我在那条路上走了几步,走不动了,看看四周都看不到人影,我想拿根裤带吊死算啦。这么想着我又走动起来,走过了一棵榆树,我只是看一眼,根本就没打算去解裤带。其实我不想死,只是找个法子与自己赌气。我想着那一屁股债又不会和我一起吊死,就对自己说:“算啦,别死啦。”这债是要我爹去还了,一想到爹,我心里一阵发麻,这下他还不把我给揍死?我边走边想,怎么想都是死路一条了,还是回家去吧。被我爹揍死,总比在外面像野狗一样吊死强。

就那么一会儿工夫,我瘦了整整一圈,眼都青了,自己还不知道,回到了家里,我娘一看到我就惊叫起来,她看着我的脸问:“你是福贵吧?”我看着娘的脸苦笑地点点头,我听到娘一惊一咋地说着什么,我不再看她,推门走到了自己屋里,正在梳头的家珍看到我也吃了一惊,她张嘴看着我。一想到她昨晚来劝我回家,我却对她又打又踢,我就扑嗵一声跪在她面前,对她说:“家珍,我完蛋啦。”说完我就呜呜地哭了起来,家珍慌忙来扶我,她怀着有庆哪能把我扶起来?她就叫我娘。两个女人一起把我抬到床上,我躺到床上就口吐白沫,一副要死的样子,可把她们吓坏了,又是捶肩又是摇我的脑袋,我伸手把她们推开,对她们说:“我把家产输光啦。”我娘听了这话先是一愣,她使劲看看我后说:“你说什么?”我说:“我把家产输光啦。”我那副模样让她信了,我娘一屁股坐到了地上,抹着眼泪说:“上梁不正下梁歪啊。”我娘到那时还在心疼我,她没怪我,倒是去怪我爹。

家珍也哭了,她一边替我捶背一边说:“只要你以后不赌就好了。”我输了个精光,以后就是想赌也没本钱了。我听到爹在那边屋子里骂骂咧咧,他还不知道自己是穷光蛋了,他嫌两个女人的哭声吵他。听到我爹的声音,我娘就不哭了,她站起来走出去,家珍也跟了出去。我知道她们到我爹屋子里去了,不一会我就听到爹在那边喊叫起来:“孽子。”这时我女儿凤霞推门进来,又摇摇晃晃地把门关上。凤霞尖声细气地对我说:“爹,你快躲起来,爷爷要来揍你了。”我一动不动地看着她,凤霞就过来拉我的手,拉不动我她就哭了。看着凤霞哭,我心里就跟刀割一样。凤霞这么小的年纪就知道护着她爹,就是看着这孩子,我也该千刀万剐。

我听到爹气冲冲地走来了,他喊着:“孽子,我要剐了你,阉了你,剁烂了你这乌龟王八蛋。”我想爹你就进来吧,你就把我剁烂了吧。可我爹走到门口,身体一晃就摔到地上气昏过去了。我娘和家珍叫叫嚷嚷地把他扶起来,扶到他自己的床上。过了一会,我听到爹在那边像是吹唢呐般地哭上了。

我爹在床上一躺就是三天,第一天他呜呜地哭,后来他不哭了,开始叹息,一声声传到我这里,我听到他哀声说着:“报应呵,这是报应。”第三天,我爹在自己屋里接待客人,他响亮地咳嗽着,一旦说话时声音又低得*坏健*到了晚上的时候,我娘走过来对我说,爹叫我过去。我从床上起来,心想这下非完蛋不可,我爹在床上歇了三天,他有力气来宰我了,起码也把我揍个半死不活。我对自己说,任凭爹怎么揍我,我也不要还手。我向爹的房间走去时一点力气都没有,身体软绵绵,两条腿像是假的。我进了他的房间,站在我娘身后,偷偷看着他躺在床上的模样,他睁圆了眼睛看着我,白胡须一抖一抖,他对我娘说:“你出去吧。”我娘从我身旁走了出去,她一走我心里是一阵发虚,说不定他马上就会从床上蹦起来和我拼命。他躺着没有动,胸前的被子都滑出去挂在地上了。

“福贵呵。”爹叫了我一声,他拍拍床沿说:“你坐下。”我心里咚咚跳着在他身旁坐下来,他摸到了我的手,他的手和冰一样,一直冷到我心里。爹轻声说:“福贵啊,赌债也是债,自古以来没有不还债的道理。我把一百多亩地,还有这房子都低押出去了,明天他们就会送铜钱来。我老了,挑不动担子了,你就自己挑着钱去还债吧。”爹说完后又长叹一声,听完他的话,我眼睛里酸溜溜的,我知道他不会和我拼命了,可他说的话就像是一把钝刀子在割我的脖子,脑袋掉不下来,倒是疼得死去活来。爹拍拍我的手说:“你去睡吧。”第二天一早,我刚起床就看到四个人进了我家院子,走在头里的是个穿绸衣的有钱人,他朝身后穿粗布衣服的三个挑夫摆摆手说:“放下吧。”三个挑夫放下担子撩起衣角擦脸时,那有钱人看着我喊的却是我爹:“徐老爷,你要的货来了。”我爹拿着地契和房契连连咳嗽着走出来,他把房地契递过去,向那人哈哈腰说:“辛苦啦。”那人指着三担铜钱,对我爹说:“都在这里了,你数数吧。”我爹全没有了有钱人的派头,他像个穷人一样恭敬地说:“不用,不用,进屋喝口茶吧。”那人说:“不必了。”说完,他看看我,问我爹:“这位是少爷吧?”我爹连连点头,他朝我嘻嘻一笑,说道:“送货时采些南瓜叶子盖在上面,可别让人抢了。”这天开始,我就挑着铜钱走十多里路进城去还债。铜钱上盖着的南瓜叶是我娘和家珍去采的,凤霞看到了也去采,她挑最大的采了两张,盖在担子上,我把担子挑起来准备走,凤霞不知道我是去还债,仰着脸问:“爹,你是不是又要好几天不回家了?”我听了这话鼻子一酸,差点掉出眼泪来,挑着担子赶紧往城里走。到了城里,龙二看到我挑着担子来了,亲热地喊一声:“来啦,徐家少爷。”我把担子放在他跟前,他揭开瓜叶时皱皱眉,对我说:“你这不是自找苦吃,换些银元多省事。”我把最后一担铜钱挑去后,他就不再叫我少爷,他点点头说:“福贵,就放这里吧。”倒是另一个债主亲热些,他拍拍我的肩说:“福贵,去喝一壶。”龙二听后忙说:“对,对,喝一壶,我来请客。”我摇摇头,心想还是回家吧。一天下来,我的绸衣磨破了,肩上的皮肉渗出了血。我一个人往家里走去,走走哭哭,哭哭走走。想想自己才挑了一天的钱就累得人都要散架了,祖辈挣下这些钱不知要累死多少人。到这时我才知道爹为什么不要银元偏要铜钱,他就是要我知道这个道理,要我知道钱来得千难万难。这么一想,我都走不动路了,在道旁蹲下来哭得腰里直抽搐。那时我家的老雇工,就是小时候背我去私塾的长根,背着个破包裹走过来。他在我家干了几十年,现在也要离开了。他很小就死了爹娘,是我爷爷带回家来的,以后也一直没娶女人。他和我一样眼泪汪汪,赤着皮肉裂开的脚走过来,看到我蹲在路边,他叫了一声:“少爷。”我对他喊:“别叫我少爷,叫我畜生。”他摇摇头说:“要饭的皇帝也是皇帝,你没钱了也还是少爷。”一听这话我刚擦干净脸眼泪又下来了,他也在我身旁蹲下来,捂着脸呜呜地哭上了。我们在一起哭了一阵后,我对他说:“天快黑了,长根你回家去吧。”长根站了起来,一步一步地走开去,我听到他嗡嗡地说:“我哪儿还有什么家呀。”我把长根也害了,看着他孤身一人走去,我心里是一阵一阵的酸痛。直到长根走远看不见了,我才站起来往家走,我到家的时候天已经黑了。家里原先的雇工和女佣都已经走了,我娘和家珍在灶间一个烧火一个做饭,我爹还在床上躺着,只有凤霞还和往常一样高兴,她还不知道从此以后就要受苦受穷了。她蹦蹦跳跳走过来,扑到我腿上问我:“为什么他们说我不是小姐了?”我摸摸她的小脸蛋,一句话也说不出来,好在她没再往下问,她用指甲刮起了我裤子上的泥巴,高兴地说:“我在给你洗裤子呢。”到了吃饭的时候,我娘走到爹的房门口问他:给你把饭端进来吧?“我爹说:”我出来吃。“我爹三根指头执着一盏煤油灯从房里出来,灯光在他脸上一闪一闪,那张脸半明半暗,他弓着背咳嗽连连。爹坐下后问我:”债还清了?“我低着头说:”还清了。“我爹说:”这就好,这就好。“他看到了我的肩膀,又说:”肩膀也磨破了。“我没有作声,偷偷看看我娘和家珍,她们两个都泪汪汪地看着我的肩膀。爹慢吞吞地吃起了饭,才吃了几口就将筷子往桌上一放,把碗一推,他不吃了。过一会,爹说道:”从前,我们徐家的老祖宗不过是养了一只小鸡,鸡养大后变成了鹅,鹅养大了变成了羊,再把羊养大,羊就变成了牛。我们徐家就是这样发起来的。“爹的声音里咝咝的,他顿了顿又说:”到了我手里,徐家的牛变成了羊,羊又变成了鹅。传到你这里,鹅变成了鸡,现在是连鸡也没啦。“爹说到这里嘿嘿笑了起来,笑着笑着就哭了。他向我伸出两根指头:”徐家出了两个败家子啊。“没出两天,龙二来了。龙二的模样变了,他嘴里镶了两颗金牙,咧着大嘴巴嘻嘻笑着。他买去了我们抵押出去的房产和地产,他是来看看自己的财产。龙二用脚踢踢墙基,又将耳朵贴在墙上,伸出巴掌拍拍,连声说:”结实,结实。“龙二又到田里去转了一圈,回来后向我和爹作揖说道:”看着那绿油油的地,心里就是踏实。“龙二一到,我们就要从几代居住的屋子里搬出去,搬


到茅屋里去住。搬走那天,我爹双手背在身后,在几个房间踱来踱去,末了对我娘说:”我还以为会死在这屋子里。“说完,我爹拍拍绸衣上的尘土,伸了伸脖子跨出门槛。我爹像往常那样,双手背在身后慢悠悠地向村口的粪缸走去。那时候天正在黑下来,有几个佃户还在地里干着活,他们都知道我爹不是主人了,还是握住锄头叫了一声:”老爷。“我爹轻轻一笑,向他们摆摆手说:”不要这样叫。“我爹已不是走在自己的地产上了,两条腿哆嗦着走到村口,在粪缸前站住脚,四下里望了望,然后解开裤带,蹲了上去。

那天傍晚我爹拉屎时不再叫唤,他眯缝着眼睛往远处看,看着那条向城里去的小路慢慢变得不清楚。一个佃户在近旁俯身割菜,他直起腰后,我爹就看不到那条小路了。

我爹从粪缸上摔了下来,那佃户听到声音急忙转过身来,看到我爹斜躺在地上,脑袋靠着粪缸一动不动。佃户提着镰刀跑到我爹跟前,问他:“老爷你没事吧?”我爹动了动眼皮,看着佃户嘶哑地问:“你是谁家的?”佃户俯下身去说:“老爷,我是王喜。”我爹想了想后说:“噢,是王喜。王喜,下面有块石头,硌得我难受。”王喜将我爹的身体翻了翻,摸出一块拳头大的石头扔到一旁,我爹重又斜躺在那里,轻声说:“这下舒服了。”王喜问:“我扶你起来?”我爹摇摇头,喘息着说:“不用了。”随后我爹问他:“你先前看到过我掉下来没有?”王喜摇摇头说:“没有,老爷。”我爹像是有些高兴,又问:“第一次掉下来?”王喜说:“是的,老爷。”我爹嘿嘿笑了几下,笑完后闭上了眼睛,脖子一歪,脑袋顺着粪缸滑到了地上。

那天我们刚搬到了茅屋里,我和娘在屋里收拾着,凤霞高高兴兴地也跟着收拾东西,她不知道从此以后就要受苦了。

家珍端着一大盆衣服从池塘边走上来,遇到了跑来的王喜,王喜说:“少奶奶,老爷像是熟了。”我们在屋里听到家珍在外面使劲喊:“娘,福贵,娘……”没喊几声,家珍就在那里呜呜地哭上了。那时我就想着是爹出事了,我跑出屋看到家珍站在那里,一大盆衣服全掉在地上。家珍看到我叫着:“福贵,是爹……”我脑袋嗡的一下,拼命往村口跑,跑到粪缸前时我爹已经断气了,我又推又喊,我爹就是不理我,我不知道该怎么办,站起来往回看,看到我娘扭着小脚又哭又喊地跑来,家珍抱着凤霞跟在后面。

我爹死后,我像是染上了瘟疫一样浑身无力,整日坐在茅屋前的地上,一会儿眼泪汪汪,一会儿唉声叹气。凤霞时常陪我坐在一起,她玩着我的手问我:“爷爷掉下来了。”看到我点点头,她又问:“是风吹的吗?”我娘和家珍都不敢怎么大声哭,她们怕我想不开,也跟着爹一起去了。有时我不小心碰着什么,她们两人就会吓一跳,看到我没像爹那样摔倒在地,她们才放心地问我:“没事吧。”那几天我娘常对我说:“人只要活得高兴,穷也不怕。”她是在宽慰我,她还以为我是被穷折腾成这样的,其实我心里想着的是我死去的爹。我爹死在我手里了,我娘我家珍,还有凤霞却要跟着我受活罪。

我爹死后十天,我丈人来了,他右手提着长衫脸色铁青地走进了村里,后面是一抬披红戴绿的花轿,十来个年轻人敲锣打鼓拥在两旁。村里人见了都挤上去看,以为是谁家娶亲嫁女,都说怎么先前没听说过,有一个人问我丈人:“是谁家的喜事?”我丈人板着脸大声说:“我家的喜事。”那时我正在我爹坟前,我听到锣鼓声抬起头来,看到我丈人气冲冲地走到我家茅屋前,他朝后面摆摆手,花轿放在了地上,锣鼓息了。当时我就知道他是要接家珍回去,我心里咚咚乱跳,不知道该怎么办?

我娘和家珍听到响声从屋里出来,家珍叫了声:“爹。”我丈人看看她女儿,对我娘说:“那畜生呢?”我娘陪着笑脸说:“你是说福贵吧?”“还会是谁。”我丈人的脸转了过来,看到了我,他向我走了两步,对我喊:“畜生,你过来。”我站着没有动,我哪敢过去。我丈人挥着手向我喊:“你过来,你这畜生,怎么不来向我请安了?畜生你听着,当初是怎么娶走家珍的,我今日也怎么接她回去。你看看,这是花轿,这是锣鼓,比你当初娶亲时只多不少。”喊完以后,我丈人回头对家珍说:“你快进屋去收拾一下。”家珍站着没动,叫了一声:“爹。”我丈人使劲跺了下脚说:“还不快去。”家珍看看站在远处地里的我,转身进屋了。我娘这时眼泪汪汪地对他说:“行行好,让家珍留下吧。”我丈人朝我娘摆摆手,又转过身来对我喊:“畜生,从今以后家珍和你一刀两断,我们陈家和你们徐家永不往来。”我娘的身体弯下去求他:“求你看在福贵他爹的份上,让家珍留下吧。”我丈人冲着我娘喊:“他爹都让他气死啦。”喊完我丈人自己也觉得有些过分,便缓一下口气说:“你也别怪我心狠,都是那畜生胡来才会有今天。”说完丈人又转向我,喊道:“凤霞就留给你们徐家,家珍肚里的孩子就是我们陈家的人啦。”我娘站在一旁呜呜地哭,她抹着眼泪说:“这让我怎么去向徐家祖宗交待。”家珍提了个包裹走了出来,我丈人对她说:“上轿。”家珍扭头看看我,走到轿子旁又回头看了看我,再看看我娘,钻进了轿子。这时凤霞不知从哪里跑了出来,一看到她娘坐上轿子了,她也想坐进去,她半个身体才进轿子,就被家珍的手推了出来。

我丈人向轿夫挥了挥手,轿子被抬了起来,家珍在里面大声哭起来,我丈人喊道:“给我往响里敲。”十来个年轻人拼命地敲响了锣鼓,我就听不到家珍的哭声了。轿子上了路,我丈人手提长衫和轿子走得一样快。我娘扭着小脚,可怜巴巴地跟在后面,一直跟到村口才站住。

这时凤霞跑了过来,她睁大眼睛对我说:“爹,娘坐上轿子啦。”凤霞高兴的样子叫我看了难受,我对她说:“凤霞,你过来。”凤霞走到我身边,我摸着她的脸说:“凤霞,你可不要忘记我是你爹。”凤霞听了这话格格笑起来,她说:“你也不要忘记我是凤霞。”

二福贵说到这里看着我嘿嘿笑了,这位四十年前的浪子,如今赤裸着胸膛坐在青草上,阳光从树叶的缝隙里照射下来,照在他眯缝的眼睛上。他腿上沾满了泥巴,刮光了的脑袋上稀稀疏疏地钻出来些许白发,胸前的皮肤皱成一条一条,汗水在那里起伏着流下来。此刻那头老牛蹲在池塘泛黄的水中,只露出脑袋和一条长长的脊梁,我看到池水犹如拍岸一样拍击着那条黝黑的脊梁。这位老人是我最初遇到的,那时候我刚刚开始那段漫游的生活,我年轻无忧无虑,每一张新的脸都会使我兴致勃勃,一切我所不知的事物都会深深吸引我。就是在这样的时刻,我遇到了福贵,他绘声绘色地讲述自己,从来没有过一个人像他那样对我全盘托出,只要我想知道的,他都愿意展示。

和福贵相遇,使我对以后收集民谣的日子充满快乐的期待,我以为那块肥沃茂盛的土地上福贵这样的人比比皆是。在后来的日子里,我确实遇到了许多像福贵那样的老人,他们穿得和福贵一样的衣裤,裤裆都快耷拉到膝盖了。他们脸上的皱纹里积满了阳光和泥土,他们向我微笑时,我看到空洞的嘴里牙齿所剩无几。他们时常流出混浊的眼泪,这倒不是因为他们时常悲伤,他们在高兴时甚至是在什么事都没有的平静时刻,也会泪流而出,然后举起和乡间泥路一样粗糙的手指,擦去眼泪,如同弹去身上的稻草。

可是我再也没遇到一个像福贵这样令我难忘的人了,对自己的经历如此清楚,又能如此精彩地讲述自己。他是那种能够看到自己过去模样的人,他可以准确地看到自己年轻时走路的姿态,甚至可以看到自己是如何衰老的。这样的老人在乡间实在难以遇上,也许是困苦的生活损坏了他们的记忆,面对往事他们通常显得木讷,常常以不知所措的微笑搪塞过去。他们对自己的经历缺乏热情,仿佛是道听途说般地只记得零星几点,即便是这零星几点也都是自身之外的记忆,用一、两句话表达了他们所认为的一切。在这里,我常常听到后辈们这样骂他们:“一大把年纪全活到狗身上去了。”福贵就完全不一样了,他喜欢回想过去,喜欢讲述自己,似乎这样一来,他就可以一次一次地重度此生了。他的讲述像鸟爪抓住树枝那样紧紧抓住我。

家珍走后,我娘时常坐在一边偷偷抹眼泪,我本想找几句话去宽慰宽慰她,一看到她那付样子,就什么话也说不出来了。倒是她常对我说:“家珍是你的女人,不是别人的,谁也抢不走。”我听了这话,只能在心里叹息一声,我还能说什么呢?好端端的一个家成了砸破了的瓦罐似的四分五裂。到了晚上,我躺在床上常常睡不着,一会儿恨这个,一会恨那个,到头来最恨的还是我自己。夜里想得太多,白天就头疼,整日无精打采,好在有凤霞,凤霞常拉着我的手问我:“爹,一张桌子有四个角,削掉一个角还剩几个角?”也不知道凤霞是从哪里去听来的,当我说还剩三个角时,凤霞高兴的格格乱笑,她说:“错啦,还剩五个角。”听了凤霞的话,我想笑却笑不出来,想到原先家里四个人,家珍一走就等于是削掉了一个角,况且家珍肚里还怀着孩子,我就对凤霞说:“等你娘回来了,就会有五个角了。”家里值钱的东西都变卖光了以后,我娘就常常领着凤霞去挖野菜,我娘挎着篮子小脚一扭一扭地走去,她走得还没有凤霞快。她头发都白了,却要学着去干从没干过的体力活。

看着我娘拉着凤霞看一步走一步,那小心的样子让我眼泪都快掉出来了。

我想想再不能像从前那样过日子了,我得养活我娘和凤霞。我就和娘商量着到城里亲友那里去借点钱,开个小铺子,我娘听了这话一声不吭,她是舍不得离开这里,人上了年纪都这样,都不愿动地方。我就对娘说:“如今屋子和地都是龙二的了,家安在这里跟安在别处也一样。”我娘听了这话,过了半晌才说:“你爹的坟还在这里。”我娘一句话就让我不敢再想别的主意了,我想来想去只好去找龙二。

龙二成了这里的地主,常常穿着丝绸衣衫,右手拿着茶壶在田埂上走来走去,神气得很。镶着两颗大金牙的嘴总是咧开笑着,有时骂看着不顺眼的佃户时也咧着嘴,我起先还以为他对人亲热,慢慢地就知道他是要别人都看到他的金牙。

龙二遇到我还算客气,常笑嘻嘻地说:“福贵,到我家来喝壶茶吧。”我一直没去龙二家是怕自己心里发酸,我两脚一落地就住在那幢屋子里了,如今那屋子是龙二的家,你想想我心里是什么滋味。

其实人落到那种地步也就顾不上那么多了,我算是应了人穷志短那句古话了。那天我去找龙二时,龙二坐在我家客厅的太师椅子里,两条腿搁在凳子上,一手拿茶壶一手拿着扇子,看到我走进来,龙二咧嘴笑道:“是福贵,自己找把凳子坐吧。”他躺在太师椅里动都没动,我也就不指望他泡壶茶给我喝。我坐下后龙二说:“福贵,你是来找我借钱的吧?”我还没说不是,他就往下说道:“按理说我也该借几个钱给你,俗话说是救急不救穷,我啊,只能救你的急,不会救你的穷。”我点点头说:“我想租几亩田。”龙二听后笑眯眯地问:“你要租几亩?”我说:“租五亩。”“五亩?”龙二眉毛往上吊了吊,问:“你这身体能行吗?”我说:“练练就行了。”他想一想说:“我们是老相识了,我给你五亩好田。”龙二还是讲点交情的,他真给了我五亩好田。我一个人种五亩地,差点没累死。我从没干过农活,学着村里人的样子干活,别说有多慢了。看得见的时候我都在田里,到了天黑,只要有月光,我还要下地。庄稼得赶上季节,错过一个季节就全错过啦。到那时别说是养活一家人,就是龙二的租粮也交不起。俗话说是笨鸟先飞,我还得笨鸟多飞。

我娘心疼我,也跟着我下地干活,她一大把年纪了,脚又不方便,身体弯下去才一会儿工夫就直不起来了,常常是一屁股坐在了田里。我对她说:“娘,你赶紧回去吧。”我娘摇摇头说:“四只手总比两只手强。”我说:“你要是累成病,那就一只手都没了,我还得照料你。”我娘听了这话,才慢慢回到田埂上坐下,和凤霞呆在一起。凤霞是天天坐在田埂上陪我,她采了很多花放在腿边,一朵一朵举起来问我叫什么花,我哪知道是什么花,就说:“问你奶奶去。”我娘坐到田埂上,看到我用锄头就常喊:“留神别砍了脚。”我用镰刀时,她更不放心,时时说:“福贵,别把手割破了。”我娘老是在一旁提醒也不管用,活太多,我得快干,一快就免不了砍了脚割破手。手脚一出血,可把我娘心疼坏了,扭着小脚跑过来,捏一块烂泥巴堵住出血的地方,嘴里一个劲儿地数落我,一说得说半晌,我还不能回嘴,要不她眼泪都会掉出来。

我娘常说地里的泥是最养人的,不光是长庄稼,还能治病。那么


多年下来,我身上那儿弄破了,都往上贴一块湿泥巴。我娘说得对,不能小看那些烂泥巴,那可是治百病的。

人要是累得整天没力气,就不会去乱想了。租了龙二的田以后,我一挨到床就呼呼地睡去,根本没工夫去想别的什么。说起来日子过得又苦又累,我心里反倒踏实了。我想着我们徐家也算是有一只小鸡了,照我这么干下去,过不了几年小鸡就会变成鹅,徐家总有一天会重新发起来的。

从那以后,我是再没穿过绸衣了,我穿的粗布衣服是我娘亲手织的布,刚穿上那阵子觉得不自在,身上的肉被磨来磨去,日子一久也就舒坦了。前几天村里的王喜死了,王喜是我家从前的佃户,比我大两岁,他死前嘱咐儿子把他的旧绸衣送给我,他一直没忘记我从前是少爷,他是想让我死之前穿上绸衣风光风光。我啊,对不起王喜的一片好心,那件绸衣我往身上一穿就赶紧脱了下来,那个难受啊,滑溜溜的像是穿上了鼻涕做的衣服。

那么过了三个来月,长根来了,就是我家的雇工。那天我正在地里干活,我娘和凤霞坐在田埂上。长根拄着一根枯树枝,破衣褴衫地走过来,手里挎着那个包裹,还拿一只缺了口的碗,他成了个叫花子。是凤霞先看到他,凤霞站起来叫着他喊:“长根,长根。”我娘一看到是从小在我家长大的长根,赶紧迎了上去,长根抹着眼泪说:“太太,我想少爷和凤霞,就回来看一眼。”长根走到田间,看到我穿着粗布衣服满身是泥,呜呜地哭,说道:“少爷,你怎么成这样子了。”我输光家产以后,最苦的就是长根了。长根替我家干了一辈子,按规矩老了就该由我家养起来。可我家一破落,他也只好离开,只能要饭过日子。

看到长根回来时的模样,我心里一阵发酸,小时候他整天背着我走东逛西,我长大后也从没把他放在眼里。没想到他还回来看我们,我问长根:“你还好吧?”长根擦擦眼睛说:“还好。”我问:“还没找到雇你的人家?”长根摇摇头说:“我这么老了,谁家会雇我?”听了这话,我眼泪都要掉出来了。长根却不觉得自己苦,他还为我哭,说道:“少爷,你哪受得起这种苦。”那天晚上,长根在我家茅屋里过的。我和娘商量着把长根留在家里,这样一来*兆踊岣*苦,我对娘说:“苦也要把他留下,我们每人剩两口饭也就养活他了。”我娘点点头说:“长根这么好的心肠。”第二天早晨,我对长根说:“长根,你一回来就好了,我正缺一个帮手,往后你就住在这里吧。”长根听后看着我笑,笑着笑着眼泪掉了出来,他说:“少爷,我没有帮你的力气了,有你这份心意我就够了。”说完长根就要走,我和娘死活拦不住他,他说:“你们别拦我了,往后我还要来看你们。”长根那天走后,还来过一次,那次他给凤霞带来一根扎头发的红绸,是他捡来的,洗干净后放在胸口专门来送给凤霞。长根那次走后,我就再没有见到他了。

我租了龙二的田,就是他的佃户了,便不能再像过去那样叫他龙二,得叫他龙老爷,起先龙二听我这么叫,总是摆摆手说:“福贵,你我之间不必多礼。”时间一久他也习惯了,我在地里干活时,他常会走过来说几句话。有一次我正割着稻子,凤霞跟在后面捡稻穗,龙二一摇一摆走过来,对我说:“福贵,我收山啦,往后再也不去赌啦。赌场无赢家,我是见好就收,免得日后也落到你这种地步。”我向龙二哈哈腰,恭敬地说:“是龙老爷。”龙二指指凤霞,问道:“这是你的崽子吗?”我又哈哈腰,说一声:“是,龙老爷。”我看到凤霞站在那里,手里拿着稻穗,直愣愣地盯着龙二看,就赶紧对她说:“凤霞,快向龙老爷行礼。”凤霞也学我的样子向龙二哈哈腰,说道:“是,龙老爷。”我时常惦记着家珍,还有她肚子里的孩子。家珍走后两个多月,托人捎来了一个口信,说是生啦,生了个儿子出来,我丈人给取了个名字叫有庆。我娘悄悄问捎话的人:“有庆姓什么?”那人说:“姓徐呀。”那时我在田里,我娘扭着小脚急匆匆地跑来告诉我,她话没说完,就擦起了眼泪。我一听说家珍给我生了个儿子,扔了手里的锄头就要往城里跑,跑出了十来步,我不敢跑了,想想我这么进城去看家珍她们母子,我丈人怕是连门槛都不让我跨进去。我就对娘说:“娘,你赶紧收拾收拾,去看看家珍她们。”我娘也一遍遍说着要进城去看孙子,可过了几天她也没动身,我又不好催她。按我们这里的习俗,家珍是被她娘家的人硬给接走的,也应该由她娘家的人送回来。我娘对我说:“有庆姓了徐,家珍也就马上要回来了。”她又说:“家珍现在身体虚,还是呆在城里好。家珍要好好补一补。”家珍是在有庆半岁的时候回来的。她来的时候没有坐轿子,她将有庆放在身后的一个包裹里,走了十多里路回来的。

有庆闭着眼睛,小脑袋靠在他娘肩膀上一摇一摇回来认我这个爹了。

家珍穿着水红的旗袍,手挽一个蓝底白花的包裹,漂漂亮亮地回来了。路两旁的油菜花开的金黄金黄,蜜蜂嗡嗡叫着飞来飞去。家珍走到我家茅屋门口,没有一下子走进去,站在门口笑盈盈地看着我娘。

我娘在屋里坐着编草鞋,她抬起头来后看到一个漂亮的女人站在门口,家珍的身体挡住了光线,身体闪闪发亮。我娘没有认出来是家珍,也没有看到家珍身后的有庆。我娘问她:“是谁家的小姐,你找谁呀?”家珍听后格格笑起来,说道:“是我,我是家珍。”当时我和凤霞在田里,凤霞坐在田埂上看着我干活,我听到有个声音喊我,声音像我娘,也有些不像,我问凤霞:“谁在喊?”凤霞转过身去看一看说:“是奶奶。”我直起身体,看到我娘站在茅屋门口弯着腰在使劲喊我,穿水红旗袍的家珍抱着有庆站在一旁。凤霞一看到她娘,撒腿跑了过去。我在水田里站着,看着我娘弯腰叫我的模样,她太使劲了,两只手撑在腿上,免得上面的身体掉到地上。凤霞跑得太快,在田埂上摇来晃去,终于扑到了家珍腿上,抱着有庆的家珍蹲下去和凤霞抱在一起。我这时才走上田埂,我娘还在喊,越走近她们,我脑袋里越是晕晕乎乎的。我一直走到家珍面前,对她笑了笑。家珍站起来,眼睛定定地看了我一阵。我当时那副穷模样使家珍一低头轻轻抽泣了。

我娘在一旁哭得呜呜响,她对我说:“我说过家珍是你的女人,别人谁也抢不走的。”家珍一回来,这个家就全了。我干活时也有了个帮手,我开始心疼自己的女人了,这是家珍告诉我的,我自己倒是不觉得。我常对家珍说:“你到田埂上去歇会儿。”家珍是城里小姐出身,细皮嫩肉的,看着她干粗活,我自然心疼。家珍听到我让她去歇一下,就高兴地笑起来,她说:“我不累。”我娘常说,只要人活得高兴,就不怕穷。家珍脱掉了旗袍,也和我一样穿上粗布衣服,她整天累得喘不过气来,还总是笑盈盈的。凤霞是个好孩子,我们从砖瓦的*课莅岬矫┪堇*去住,她照样高高兴兴,吃起粗粮来也不往外吐。弟弟回来以后她就更高兴了,再不到田边来陪我,就一心想着去抱弟弟。有庆苦呵,他姐姐还过了四、五年好日子,有庆才在城里呆了半年,就到我身边来受苦了,我觉得最对不起的就是儿子。

这样的日子过了一年后,我娘病了。开始只是头晕,我娘说看着我们时糊里糊涂的。我也没怎么在意,想想她年纪大了,眼睛自然看不清。后来有一天,我娘在烧火时突然头一歪,靠在墙上像是睡着了。等我和家珍从田里回来,她还那么靠着。家珍叫她,她也不答应,伸手推推她,她就顺着墙滑了下去。家珍吓得大声叫我,我走到灶间时,她又醒了过来,定定地看了我们一阵,我们问她,她也不答应,又过了一阵,她闻到焦糊的味道,知道饭煮糊了,才开口说道:“哎呀,我怎么睡着了。”我娘慌里慌张地想站起来,她站到一半腿一松,身体又掉到地上。我赶紧把她抱到床上,她没完没了地说自己睡着了,她怕我们不相信。家珍把我拉到一旁说:“你去城里请个郎中来。”请郎中可是要花钱的,我站着没有动。家珍从褥子底下拿出了两块银元,是用手帕包着的。看看银元我有些心疼,那可是家珍从城里带来的,只剩下这两块了。可我娘的身体更叫我担心,我就拿过银元。家珍把手帕叠得整整齐齐重新塞到褥子底下,给我拿出一身干净衣服,让我换上。我对家珍说:“我走了。”家珍没说话,跟着我走到门口,我走了几步回过头去看看她,她往后理了理头发向我点点头。自从家珍回来以后,我还是第一次离开她。我穿着虽然破烂可是干干净净的衣服,脚上是我娘编的新草鞋,要进城去了。凤霞坐在门口的地上,怀里抱着睡着的有庆,她看到我穿得很干净,就问:“爹,你不是下田吧?”我走得很快,不到半个时辰就走到城里。我已有一年多没去城里了,走进城里时心里还真有点发虚,我怕碰到过去的熟人,我这身破烂衣服让他们见了,不知道他们会说些什么话。我最怕见到的还是我丈人,我不敢从米行那条街走,宁愿多绕一些路。城里几个郎中的医术我都知道,哪个收钱黑,哪个收钱公道我也知道。我想了想,还是去找住在绸店隔壁的林郎中,这个老头是我丈人的朋友,看在家珍的份上他也会少收些钱。

我路过县太爷府上时,看到一个穿绸衣的小孩正踮着脚,使劲想抓住敲门的铜环。那孩子的年纪就和我凤霞差不多大,我想这可能是县太爷的公子,就走上去对他说:“我来帮你敲。”小孩高兴地点点头,我就扣住铜环使劲敲了几下,里面有人答应:“来啦。”这时小孩对我说:“我们快跑吧。”我还没明白过来,小孩贴着墙壁溜走了。门打开后,一个仆人打扮的男人一看到我穿的衣服,什么话没说就伸手推了我一把,我没料到他会这样,身体一晃就从台阶上跌下来。

我从地上爬起来,本来我想算了,可这家伙又走下来踢了我一脚,还说:“要饭也不看这是什么地方。”我的火一下子上来了,我骂道:“老子就是啃你家祖坟里的烂骨头,也不会向你要饭。”他扑上来就打,我脸上挨了一拳,他也挨了我一脚。我们两个人就在街上扭打起来。这小子黑得很,看看一下子打不赢我,就瞅着我的裤裆抬脚。我呢,好几次踢在他屁股上。

我们两个都不会打架,打了一阵听到有人在后面喊:“难看死啦,这两个畜生打架打得难看死啦。”我们停住手脚,往后一看,一队穿黄衣服的国民党大兵站在那里,十来门大炮都由马车拉着。刚才喊叫的那个人腰里别着一把手枪,是个当官的。那仆人真灵活,一看到当官的就马上点头哈腰:“长官,嘿嘿,长官。”长官向我们两个挥挥手说:“两头蠢驴,打架都不会,给我去拉大炮。”我一听这话头皮阵阵发麻,他是拉我当壮丁的。那仆人也急了,走上前去说:“长官,我是本县县太爷家里的。”长官说:“县太爷的公子更应该为党国出力嘛。”“不,不。”仆人吓得连声说,“我不是公子,打死我也不也敢。排长,我是县太爷的仆人。”“操你娘。”长官大声骂道:“老子是连长。”“是,是,连长,我是县太爷的仆人。”那仆人怎么说都没用,反而把连长说烦了,连长伸手给他一巴掌:“少他娘的说废话,去拉大炮。”他看到了我。“还有你。”我只好走上去,拉住一匹马的缰绳,跟着他们往前走。我想到时候打个机会再逃跑吧。那仆人还在前面向连长求情,走了一段路后,连长竟然答应了,他说:“行,行,你回去吧,你小子烦死我了。”仆人高兴坏了,他像是要跪下来给连长叩头,可又没有下跪,只是在连长面前不停地搓着手,连长说:“还不滚蛋。”仆人说:“滚,滚,我这就滚。”仆人说着转身走去,这时候连长从腰里抽出手枪来,把胳膊端平了,闭上一只眼睛向走去的仆人瞄准。仆人走出了十多步回过头来看看,这一看把他吓得傻站在那里一动不动,像只夜里的麻雀一样让连长瞄准。连长这时对他说:“走呀,走呀。”仆人扑通一声跪在地上,连哭带喊:“连长,连长,连长。”连长向他开了一枪,没有打中,打在他身旁,飞起的小石子划破了他的手,手倒是出血了。连长握着手枪向他挥动着说:“站起来,站起来。”他站了起来,连长又说:“走呀,走呀。”他伤心地哭了,结结巴巴地说:“连长,我拉大炮吧。”连长又端起胳膊,第二次向他瞄准,嘴里说着:“走呀,走呀。”仆人这时才突然明白似的,一转身就疯跑起来。连长打出第二枪时,他刚好拐进了一条胡同。连长看看自己的手枪,骂了一声:“他娘的,老子闭错了一只眼睛。”连长转过身来,看到了站在后面的我,就提着手枪走过来,把枪口顶着我的胸膛,对我说:“你也回去吧。”我的两条腿拼命哆嗦,心想他这次就是两只眼睛全闭错,也会一枪把我送上西天。我连声说:“我拉大炮,我拉大炮。”我右手拉着缰绳,左手捏住口袋里家珍给我的两块银元,走出城里时,看到田地里与我家相像的茅屋,我低下头哭了。

我跟着这支往


北去的炮队,越走越远,一个多月后我们走到了安徽。开始的几天我一心想逃跑,当时想逃跑的不只是我一个人,每过两天,连里就会少掉一、两张熟悉的脸,我心想他们是不是逃跑了,我就问一个叫老全的老兵,老全说:“谁也逃不掉。”老全问我夜里睡觉听到枪声没有,我说听到了,他说:“那就是打逃兵的,命大的不让打死,也会被别的部队抓去。”老全说得我心都寒了。老全告诉我,他抗战时就被拉了壮丁,开拔到江西他逃了出来,没几天又被去福建的部队拉了去。当兵六年多,没跟日本人打过仗,光跟共产党的游击队打仗。这中间他逃跑了七次,都被别的部队拉了去。最后一次他离家只有一百多里路了,结果撞上了这一支炮队。老全说他不想再跑了,他说:“我逃腻了。”我们渡过长江以后就穿上了棉袄。一过长江,我想逃跑的心也死了,离家越远我也就越没有胆量逃跑。我们连里有十来个都是十五六岁的孩子,有一个叫春生的娃娃兵,是江苏人,他老向我打听往北去是不是打仗,我就说是的。其实我也不知道,我想当上了兵就逃不了要打仗。春生和我最亲热,他总是挨着我,拉着我的胳膊问说:“我们会不会被打死?”我说:“我不知道。”说这话时我自己心里也是一阵阵难受。过了长江以后,我们开始听到枪炮声,起先是远远传来,我们又走了两天,枪炮声越来越响。那时我们来到了一个村庄,村里别说是人了,连牲畜都见不着。连长命令我们架起大炮,我知道这下是真要打仗了。有人走过去问连长:“连长,这是什么地方?”连长说:“你问我,我他娘的去问谁?”连长都不知道我们到了什么地方,村里人跑了个精光,我望望四周,除了光秃秃的树和一些茅屋,什么都没有。过了两天,穿黄衣服的大兵越来越多,他们在四周一队队走过去,又一队队走过来,有些部队就在我们旁边扎下了。又过了两天,我们一炮还未打,连长对我们说:“我们被包围了。”被包围的不只是我们一个连,有十来万人的国军全被包围在方圆只有二十来里路的地方里,满地都是黄衣服,像是赶庙会一样。这时候老全神了,他坐在坑道外的土墩上吸着烟,看着那些来来去去的黄皮大兵,不时和中间某个人打声招呼,他认识的人实在是多。老全走南闯北,在七支部队里混过,他嘻嘻哈哈和几个旧相识说着脏话,互相打听几个人名,我听他们不是说死了,就是说前两天还见过。老全告诉我和春生,这些人当初都和他一起逃跑过。老全正说着,有个人向这里叫:“老全,你还没死啊?”老全又遇到旧相识了,哈哈笑道:“你小子什么时候被抓回来的?”那人还没说话,另一边也有人叫上老全了,老全扭脸一看,急忙站起来喊:“喂,你知道老良在哪里?”那个人嘻嘻笑着喊道:“死啦。”老全沮丧地坐下来,骂道:“妈的,他还欠我一块银元呢。”接着老全得意地对我和春生说:“你们瞧,谁都没逃成。”刚开始我们只是被包围住,解放军没有立刻来打我们,我们还不怎么害怕,连长也不怕,他说蒋委员长会派坦克来救我们出去的。后来前面的枪炮声越来越响,我们也没有很害怕,只是一个个都闲着没事可干,连长没有命令我们开炮。有个老兵想想前面的弟兄流血送命,我们老闲着也不是个办法,他就去问连长:“我们是不是也打几炮?”连长那时候躲在坑道里赌钱,他气冲冲地反问:“打炮,往哪里打?”连长说得也对,几炮打出去要是打在国军兄弟头上,前面的国军一气之下杀回来收拾我们,这可不是闹着玩的。连长命令我们都在坑道里呆着,爱干什么就干什么,就是别出去打炮。

被包围以后,我们的粮食和弹药全靠空投。飞机在上面一出现,下面的国军就跟蚂蚁似的密密麻麻地拥来拥去,扔下的一箱箱弹药没人要,全都往一袋袋大米上扑。飞机一走,抢到大米的国军兄弟两个人提一袋,旁边的人端着枪,保护他们,那么一堆一堆地分散开去,都走回自己的坑道。

没过多久,成群结伙的国军向房屋和光秃秃的树木涌去,远近的茅屋顶上都爬上去了人,又拆茅屋又砍树,这哪还像是打仗,乱糟糟的响声差不多都要盖住前沿的枪炮声了。才半天工夫,眼睛望得到的房屋树木全没了,空地上全都是扛着房梁,树木和抱着木板、凳子的大兵,他们回到自己的坑道后,一条条煮米饭的炊烟就升了起来,在空中扭来扭去。

那时候最多的就是子弹了,往那里躺都硌得身体疼。四周的房屋被拆光,树也砍光后,满地的国军提着刺刀去割枯草,那情形真像是农忙时在割稻子,有些人满头大汗地刨着树根。还有一些人开始掘坟,用掘出的棺材板烧火。掘出了棺材就把死人骨头往坑外一丢,也不给重新埋了,到了那种时候,谁也不怕死人骨头了,夜里就是挨在一起睡觉也不会做恶梦。煮米饭的柴越来越少,米倒是越来越多。没人抢米了,我们三个人去扛了几袋米回来,铺在坑道当睡觉的床,这样躺着就不怕子弹硌得身体难受了。

等到再也没有什么可当柴煮米饭时,蒋委员长还没有把我们救出去。好在那时飞机不再往下投大米,改成投大饼,成包的大饼一落地,弟兄们像牲畜一样扑上去乱抢,叠得一层又一层,跟我娘纳出的鞋底一样,他们嗷嗷乱叫着和野狼没什么两样。

老全说:“我们分开去抢。”这种时候只能分开去抢,才能多抢些大饼回来。我们爬出坑道,自己选了个方向走去。当时子弹在很近的地方飞来飞去,常有一些流弹窜过来。有一次我跑着跑着,身边一个人突然摔倒,我还以为他是饿昏了,扭头一看他半个脑袋没了,吓得我腿一软也差一点摔倒。抢大饼比抢大米还难,按说国军每天都在拼命地死人,可当飞机从天那边飞过来时,人全从地里冒了出来,光秃秃的地上像是突然长出了一排排草,跟着飞机跑,大饼一扔下,人才散开去,各自冲向看好的降落伞。大饼包得也不结实,一落地就散了,几十上百个人往一个地方扑,有些人还没挨着地就撞昏过去了,我抢一次大饼就跟被人吊起来用皮带打了一顿似的全身疼。到头来也只是抢到了几张大饼。回到坑道里,老全已经坐在那里了,他脸上青一块紫一块的,他抢到的饼也不比我多。老全当了八年兵,心里还是很善良,他把自己的饼往我的上面一放,说等春生回来一起吃。我们两个就蹲在坑道里,露出脑袋张望春生。

过了一会,我们看到春生怀里抱着一堆胶鞋猫着腰跑来了,这孩子高兴得满脸通红,他一翻身滚了进来,指着满地的胶鞋问我们:“多不多?”老全望望我,问春生:“这能吃吗?”春生说:“可以煮米饭啊。”我们一想还真对,看看春生脸上一点伤都没有,老全对我说:“这小子比谁都精。”后来我们就不去抢大饼了,用上了春生的办法。抢大饼的人叠在一起时,我们就去扒他们脚上的胶鞋,有些脚没有反应,有些脚乱蹬起来,我们就随手捡个钢盔狠狠揍那些不老实的脚,挨了揍的脚抽搐几下都跟冻僵似的硬了。我们抱着胶鞋回到坑道里生火,反正大米有的是,这样还免去了皮肉之苦。我们三个人边煮着米饭,边看着那些光脚在冬天里一走一跳的人,嘿嘿笑个不停。

前沿的枪炮声越来越紧,也不分白天和晚上。我们呆在坑道里也听惯了,经常有炮弹在不远处爆炸,我们连的大炮都被打烂了,这些大炮一炮都没放,就成了一堆烂铁,我们更加没事可干了。那么一些日子下来,春生也不怎么害怕了,到那时候怕也没有用。枪炮声越来越近,我们总觉得还远着呢。最难受的就是天越来越冷,睡上几分钟就是冻醒一次。炮弹在外面爆炸时常震得我们耳朵里嗡嗡乱叫,春生怎么说也只是个孩子,他迷迷糊糊睡着时,一颗炮弹飞到近处一炸,把他的身体都弹了起来,他被吵醒后怒气冲冲地站在坑道上,对前面的枪炮声大喊:“你们他娘的轻一点,吵得老子都睡不着。”我赶紧把他拉下来,当时子弹已在坑道上面飞来飞去了。

国军的阵地一天比一天小,我们就不敢随便爬出坑道,除非饿极了才出去找吃的。每天都有几千伤号被抬下来,我们连的阵地在后方,成了伤号的天下。有那么几天,我和老全、春生扑在坑道上,露出三个脑袋,看那些抬担架的将缺胳膊断腿的伤号抬过来。隔上不多时间,就过来一长串担架,抬担架的都猫着腰,跑到我们近前找一块空地,喊一、二、三,喊到三时将担架一翻,倒垃圾似的将伤号扔到地上就不管了。

伤号疼得嗷嗷乱叫,哭天喊地的叫声是一长串一长串响过来。

老全看着那些抬担架的离去,骂了一声:“这些畜生。”伤号越来越多,只要前面枪炮声还在响,就有担架往这里来,喊着一、二、三把伤号往地上扔。地上的伤号起先是一堆一堆,没多久就连成一片,在那里疼得嗷嗷直叫,那叫喊我一辈子都忘不了,我和春生看得心里一阵阵冒寒气,连老全都直皱眉。我想这仗怎么打呀。

天一黑,又下起了雪。有一长段时间没有枪炮声,我们就听着躺在坑道外面几千没死的伤号呜呜的声音,像是在哭,又像是在笑,那是疼得受不了的声音,我这辈子就再没听到过这么怕人的声音了。一大片一大片,就像潮水从我们身上涌过去。雪花落下来,天太黑,我们看不见雪花,只是觉得身体又冷又湿,手上软绵绵一片,慢慢地化了,没多久又积上了厚厚一层雪花。

我们三个人紧挨着睡在一起,又饿又冷,那时候飞机也来得少了,都很难找到吃的东西。谁也不会再去盼蒋委员长来救我们了,接下去是死是活谁也不知道。春生推推我,问:“福贵,你睡着了吗?”我说:“没有。”他又推推老全,老全没说话。春生鼻子抽了两下,对我说:“这下活不成了。”我听了这话鼻子里也酸溜溜的,老全这时说话了,他两条胳膊伸了伸说:“别说这丧气话。”他身体坐起来,又说:“老子大小也打过几十次仗了,每次我都对自己说:”老子死也要活着。子弹从我身上什么地方都擦过,就是没伤着我。春生,只要想着自己不死,就死不了。“接下去我们谁也没说话,都想着自己的心事。我是一遍遍想着自己的家,想想凤霞抱着有庆坐在门口,想想我娘和家珍。想着想着心里像是被堵住了,都透不过气来,像被人捂住了嘴和鼻子一样。

到了后半夜,坑道外面伤号的呜咽渐渐小了下去,我想他们大部分都睡着了吧。只有不多的几个人还在呜呜地响,那声音一段一段的,飘来飘去,听上去像是在说话,你问一句,他答一声,声音凄凉得都不像是活人发出来的。那么过了一阵后,只剩下一个声音在呜咽了,声音低得像蚊虫在叫,轻轻地在我脸上飞来飞去,听着听着已不像是在呻吟,倒像是在唱什么小调。周围静得什么声响都没有,只有这样一个声音,长久地在那里转来转去。我听得眼泪都流了出来,把脸上的雪化了后,流进脖子就跟冷风吹了进来。

天亮时,什么声音也没有了,我们露出脑袋一看,昨天还在喊叫的几千伤号全死了,横七竖八地躺在那里,一动不动,上面盖了一层薄薄的雪花。我们这些躲在坑道里还活着的人呆呆看了半晌,谁都没说话。连老全这样不知见过多少死人的老兵也傻看了很久,末了他叹息一声,摇摇头对我们说:“惨啊。”说着,老全爬出了坑道,走到这一大片死人中间翻翻这个,拨拨那个,老全弓着背,在死人中间跨来跨去,时而蹲下去用雪给某一个人擦擦脸。这时枪炮声又响了起来,一些子弹朝这里飞来。我和春生一下子回过魂来,赶紧向老全叫:“你快回来。”老全没答理我们,继续看来看去。过了一会,他站住了,来回张望了几下,才朝我们走来。走近了他向我和春生伸出四根指头,摇着头说:“有四个,我认识。”话刚说完,老全突然向我们睁圆了眼睛,他的两条腿僵住似的站在那里,随后身体往下一掉跪在了那里。我们不知道他为什么这样,只看到有子弹飞来,就拼命叫:“老全,你快点。”喊了几下后,老全还是那么一副样子,我才想完了,老全出事了。我赶紧爬出坑道,向老全跑去,跑到跟前一看,老全背脊上一滩血,我眼睛一黑,哇哇地喊春生。等春生跑过来后,我们两个人把老全抬回到坑道,子弹在我们身旁时时呼的一下擦过去。

我们让老全躺下,我用手顶住他背脊上那滩血,那地方又湿又烫,血还在流,从我指缝流出去。老全眼睛慢吞吞地眨了一下,像是看了一会我们,随后嘴巴动了动,声音沙沙地问我们:“这是什么地方?”我和春生抬头向周围望望,我们怎么会知道这是什么地方?只好重新去看老全,老全将眼睛紧紧闭了一下,接着慢慢睁开,越睁越大,他的嘴歪了歪,像是在苦笑,我们听到他沙哑地说:“老子连死在什么地方都不知道。”老全说完这话,过了没多久就死了。老全死后脑袋歪到了一旁,我和春生知道他已经死了,互相看了半晌,春生先哭了,春生一哭我也忍不住哭了。

后来,我们看到了连长,他换上老百姓的衣服,腰里绑


满了钞票,提着个包裹向西走去。我们知道他是要逃命了,衣服里绑着的钞票让他走路时像个一扭一扭的胖老太婆。有个娃娃兵向他喊:“连长,蒋委员长还救不救我们?”连长回过头来说:“蠢蛋,这种时候你娘也不会来救你了,还是自己救自己吧。”一个老兵向他打了一枪,没打中。连长一听到子弹朝他飞去,全没有了过去的威风,撒开两腿就疯跑起来,好几个人都端起枪来打他,连长哇哇叫着跳来跳去在雪地里逃远了。

枪炮声响到了我们鼻子底下,我们都看得见前面开枪的人影了,在硝烟里一个一个摇摇晃晃地倒下去。我算计着自己活不到中午,到不了中午就该轮到我去死了。一个来月在枪炮里混下来后,我倒不怎么怕死,只是觉得自己这么死得不明不白实在是冤,我娘和家珍都不知道我死在何处。

我看看春生,他的一只手还搁在老全身上,愁眉苦脸地也在看着我。我们吃了几天生米,春生的脸都吃肿了。他伸舌头舔舔嘴唇,对我说:“我想吃大饼。”到这时候死活已经不重要了,死之前能够吃上大饼也就知足了。春生站了起来,我没叫他小心子弹,他看了看说:“兴许外面还有饼,我去找找。”春生爬出了坑道,我没拦他,反正到不了中午我们都得死,他要是真吃到大饼那就太好了。我看着他有气无力地从尸体上跨了过去,这孩子走了几步还回过头来对我说:“你别走开,我找着了大饼就回来。”他垂着双手,低头走入了前面的浓烟。那个时候空气里满是焦糊和硝烟味,吸到嗓子眼里觉得有一颗一颗小石子似的东西。

中午没到的时候,坑道里还活着的人全被俘虏了。当端着枪的解放军冲上来时,有个老兵让我们举起双手,他紧张得脸都青了,叫嚷着要我们别碰身边的枪,他怕到时候连他也跟着倒楣。有个比春生大不了多少的解放军将黑洞洞的枪口对准我,我心一横,想这次是真要死了。可他没有开枪,对我叫嚷着什么,我一听是要我爬出去,我心里一下子咚咚乱跳了,我又有活的盼头了。我爬出坑道后,他对我说:“把手放下吧。”我放下了手,悬着的心也放下了。我们一排二十多个俘虏由他一人押着向南走去,走不多远就汇入到一队更大的俘虏里。到处都是一柱柱冲天的浓烟。向着同一个地方弯过去。

地上坑坑洼洼,满是尸体和炸毁了的大炮枪支,烧黑了的军车还在噼噼啪啪。我们走了一段后,二十多个挑着大白馒头的解放军从北横着向我们走来,馒头热气腾腾,看得我口水直流。押我们的一个长官说:“你们自己排好队。”没想到他们是给我们送吃的来了,要是春生在该有多好,我往远处看看,不知道这孩子是死是活。我们自动排出了二十多个队形,一个挨着一个每人领了两个馒头,我从没听到过这么一大片吃东西的声音,比几百头猪吃东西时还响。大家都吃得太快,有些人拼命咳嗽,咳嗽声一声比一声高,我身旁的一个咳得比谁都响,他捂着腰疼得眼泪横流。更多的人是噎住了,都抬着脑袋对天空直瞪眼,身体一动不动。

第二天早晨,我们被集合到一块空地上,整整齐齐地坐在地上。前面是两张桌子,一个长官模样的人对我们说话,他先是讲了一通解放全中国的道理,最后宣布愿意参加解放军的继续坐着,想回家的就站出来,去领回家的盘缠。

一听可以回家,我的心扑扑乱跳,可我看到那个长官腰里别了一支手枪又害怕了,我想哪有这样的好事。很多人都坐着没动,有一些人走出去,还真的走到那桌子前去领了盘缠,那个长官一直看着他们,他们领了钱以后还领了通行证。

接着就上路了,我的心提到了嗓子眼,那个长官肯定会拔出手枪来毙他们,就跟我们连长一样。可他们走出很远以后,长官也没有掏出手枪。这下我紧张了,我知道解放军是真的愿意放我们回家。这一仗打下来我知道什么叫打仗了,我对自己说再也不能打仗了,我要回家。我就站起来,一直走到那位长官面前,扑通跪下后就哇哇哭起来,我原本想说我要回家,可话到嘴边又变了,我一遍遍叫着:“连长,连长,连长――”别的什么话也说不出来,那位长官把我扶起来,问我要说什么。我还是叫他连长,还是哭。旁边一个解放军对我说:“他是团长。”他这一说把我吓住了,心想糟了。可听到坐着的俘虏哄地笑起来,又看到团长笑着问我:“你要说什么?”我这才放心下来,对团长说:“我要回家。”解放军让我回家,还给了盘缠。我一路急匆匆往南走,饿了就用解放军给的盘缠买个烧饼吃下去,困了就找个平整一点地方睡一觉。我太想家了,一想到今生今世还能和我娘和家珍,和我一双儿女团聚,我又是哭又是笑,疯疯癫癫地往南跑。

我走到长江边时,南面还没有解放,解放军在准备渡江了。我过不去,在那里耽搁了几个月。我就到处找活干,免得饿死。我知道解放军缺摇船的,我以前有钱时觉得好玩,学过摇船。好几次我都想参加解放军,替他们摇船摇过长江去。

想想解放军对我好,我要报恩。可我实在是怕打仗,怕见不到家里人。为了家珍她们,我对自己说:“我就不报恩了,我记得解放军的好。”我是跟在往南打去的解放军屁股后面回到家里的,算算时间,我离家都快两年了。走的时候是深秋,回来是初秋。我满身泥土走上了家乡的路,后来我看到了自己的村庄,一点都没变,我一眼就看到了,我急冲冲往前走。看到我家先前的砖瓦房,又看到了现在的茅屋,我一看到茅屋忍不住跑了起来。

离村口不远的地方,一个七、八岁的女孩,带着个三岁的男孩在割草。我一看到那个穿得破破烂烂的女孩就认出来了,那是我的凤霞。凤霞拉着有庆的手,有庆走路还磕磕绊绊。我就向凤霞有庆喊:“凤霞,有庆。”凤霞像是没有听到,倒是有庆转回身来看我,他被凤霞拉着还在走,脑袋朝我这里歪着。我又喊:“凤霞,有庆。”这时有庆拉住了他姐姐,凤霞向我转了过来,我跑到跟前,蹲下去问凤霞:“凤霞,还认识我吗?”凤霞张大眼睛看了我一阵,嘴巴动了动没有声音。我对凤霞说:“我是你爹啊。”凤霞笑了起来,她的嘴巴一张一张,可是什么声音都没有。当时我就觉得有些不对劲,只是我没往细里想。我知道凤霞认出我来了,她张着嘴向我笑,她的门牙都掉了。我伸手去摸她的脸,她的眼睛亮了亮,就把脸往我手上贴,我又去看有庆,有庆自然认不出我,他害怕地贴在姐姐身上,我去拉他,他就躲着我,我对他说:“儿子啊,我是你爹。”有庆干脆躲到了姐姐身后,推着凤霞说:“我们快走呀。”这时有一个女人向我们这里跑来,哇哇叫着我的名字,我认出来是家珍,家珍跑得跌跌撞撞,跑到跟前喊了一声:“福贵。”就坐在地上大声哭起来,我对家珍说:“哭什么,哭什么。”这么一说,我也呜呜地哭了。

我总算回到了家里,看到家珍和一双儿女都活得好好的,我的心放下了。她们拥着我往家里走去,一走近自家的茅屋,我就连连喊:“娘,娘。”喊着我就跑了起来,跑到茅屋里一看,没见到我娘,当时我眼睛就黑了一下,折回来问家珍:“我娘呢?”家珍什么也不说,就是泪汪汪地看着我,我也就知道娘到什么地方去了。我站在门口脑袋一垂,眼泪便刷刷地流了出来。

我离家两个月多一点,我娘就死了。家珍告诉我,我娘死前一遍一遍对家珍说:“福贵不会是去赌钱的。”家珍去城里打听过我不知多少次,竟会没人告诉她我被抓了壮丁。我娘才这么说,可怜她死的时候,还不知道我在什么地方。我的凤霞也可怜,一年前她发了一次高烧后就再不会说话了。家珍哭着告诉我这些时,凤霞就坐在我对面,她知道我们是在说她,就轻轻地对着我笑,看到她笑,我心里就跟针扎一样。有庆也认我这个爹了,只是他仍有些怕我,我一抱他,他就拚命去看家珍和凤霞。随便怎么说,我都回到家里了。头天晚上我怎么都睡不着,我和家珍,还有两个孩子挤在一起,听着风吹动屋顶的茅草,看着外面亮晶晶的月光从门缝里钻进来,我心里是又踏实又暖和,我一会儿就要去摸摸家珍,摸摸两个孩子,我一遍遍对自己说:“我回家了。”我回来的时候,村里开始搞土地改革了,我分到了五亩地,就是原先租龙二的那五亩。龙二是倒大楣了,他做上地主,神气了不到四年,一解放他就完蛋了。共产党没收了他的田产,分给了从前的佃户。他还死不认帐,去吓唬那些佃户,也有不买帐的,他就动手去打人家。龙二也是自找倒楣,人民政府把他抓了去,说他是恶霸地主。被送到城里大牢后,龙二还是不识时务,那张嘴比石头都硬,最后就给毙掉了。

枪毙龙二那天我也去看了。龙二死到临头才泄了气,听说他从城里被押出来时眼泪汪汪,流着口水对一个熟人说:“做梦也想不到我会被毙掉。”龙二也太糊涂了,他以为自己被关几天就会放出来,根本不相信会被枪毙。那是在下午,枪决龙二就在我们的一个邻村,事先有人挖好了坑。那天附近好几个村里的人都来看了,龙二被五花大绑地押了过来,他差不多是被拖过来的,嘴巴半张着呼哧呼哧直喘气,龙二从我身边走过时看了我一眼,我觉得他没认出我来,可走了几步他硬是回过头来,哭着鼻子对我喊道:“福贵,我是替你去死啊。”听他这么一喊,我慌了,想想还是离开吧,别看他怎么死了。我从人堆里挤出去,一个人往外走,走了十来步就听到“电”的一枪,我想龙二彻底完蛋了,可紧接着又是“电”的一枪,下面又打了三枪,总共是五枪。我想是不是还有别的人也给毙掉,回去的路上我问同村的一个人:“毙了几个?”他说:“就毙了龙二。”龙二真是倒楣透了,他竟挨了五枪,哪怕他有五条命也全报销了。

毙掉龙二后,我往家里走去时脖子上一阵阵冒冷气,我是越想越险,要不是当初我爹和我是两个败家子,没准被毙掉的就是我了。我摸摸自己的脸,又摸摸自己的胳膊,都好好的,我想想自己是该死却没死,我从战场上捡了一条命回来,到了家龙二又成了我的替死鬼,我家的祖坟埋对了地方,我对自己说:“这下可要好好活了。”我回到家里时,家珍正在给我纳鞋底,她看到我的脸色吓一跳,以为我病了。当我把自己想的告诉她,她也吓得脸蛋白一阵青一阵,嘴里咝咝地说:“真险啊。”后来我就想开了,觉得也用不着自己吓唬自己,这都是命。常言道,大难不死必有后福。我想我的后半截该会越来越好了。我这么对家珍说了,家珍用牙咬断了线,看着我说:“我也不想要什么福分,只求每年都能给你做一双新鞋。”我知道家珍的话,我的女人是在求我们从今以后再不分开。看着她老了许多的脸,我心里一阵酸疼。家珍说得对,只要一家人天天在一起,也就不在乎什么福分了。

福贵的讲述到这里中断,我发现我们都坐在阳光下了,阳光的移动使树荫悄悄离开我们,转到了另一边。福贵的身体动了几下才站起来,他拍了拍膝盖对我说:“我全身都是越来越硬,只有一个地方越来越软。”我听后不由高声笑起来,朝他耷拉下去的裤裆看看,那里沾了几根青草。他也嘿嘿笑了一下,很高兴我明白他的意思。然后他转过身去喊那头牛:“福贵。”那头牛已经从水里出来了,正在啃吃着池塘旁的青草,牛站在两棵柳树下面,牛背上的柳枝失去了垂直的姿态,出现了纷乱的弯曲。在牛的脊背上刷动,一些树叶慢吞吞的掉落下去。老人又叫了一声:“福贵。”牛的屁股像是一块大石头慢慢地移进了水里,随后牛脑袋从柳枝里钻了出来,两只圆滚滚的眼睛朝我们缓缓移来。老人对牛说:“家珍他们早在干活啦,你也歇够了。我知道你没吃饱,谁让你在水里呆这么久?”福贵牵着牛到了水田里,给牛套上犁的工夫,他对我说:“牛老了也和人老了一样,饿了还得先歇一下,才吃得下去东西。”我重新在树荫里坐下来,将背包垫在腰后,靠着树干,用草帽扇着风。老牛的肚皮耷拉下来,长长一条,它耕动时肚皮犹如一只大水袋一样摇来晃去。我注意到福贵耷拉下去的裤裆,他的裤裆也在晃动,很像牛的肚皮。

那天我一直在树荫里坐到夕阳西下,我没有离开是因为福贵的讲述还没有结束。

我回家后的日子苦是苦,过得还算安稳。凤霞和有庆一天天大起来,我呢,一天比一天老了。我自己还没觉得,家珍也没觉得,我只是觉得力气远不如从前。到了有一天,我挑着一担菜进城去卖,路过原先绸店那地方,一个熟人见到我就叫了:“福贵,你头发白啦。”其实我和他也只是半年没见着,他这么一叫,我才觉得自己是老了许多。回到家里,我把家珍看了又看,看得她不知出了什么事,低头看看自己,又看看背后,才问:“你看什么呀。”我笑着告诉她:“你的头发也白了。”


那一年凤霞十七岁了,凤霞长成了女人的模样,要不是她又聋又哑,提亲的也该找上门来了。村里人都说凤霞长得好,凤霞长得和家珍年轻时差不多。有庆也有十二岁了,有庆在城里念小学。

当初送不送有庆去念书,我和家珍着实犹豫了一阵,没有钱啊。凤霞那时才十二三岁,虽说也能帮我干点田里活,帮家珍干些家里活,可总还是要靠我们养活。我就和家珍商量是不是把凤霞送给别人算了,好省下些钱供有庆念书。别看凤霞听不到,不会说,她可聪明呢,我和家珍一说起把凤霞送人的事,凤霞马上就会扭过头来看我们,两只眼睛一眨一眨,看得我和家珍心都酸了,几天不再提起那事。

眼看着有庆上学的年纪越来越近,这事不能不办了。我就托村里人出去时顺便打听打听,有没有人家愿意领养一个十二岁的女孩。我对家珍说:“要是碰上一户好人家,凤霞就会比现在过得好。”家珍听了点着头,眼泪却下来了。做娘的心肠总是要软一些。我劝家珍想开点,凤霞命苦,这辈子看来是要苦到底了。有庆可不能苦一辈子,要让他念书,念书才会有个出息的日子。总不能让两个孩子都被苦捆住,总得有一个日后过得好一些。

村里出去打听的人回来说凤霞大了一点,要是减掉一半岁数,要的人家就多了。这么一说我们也就死心了。谁知过了一个来月,两户人家捎信来要我们的凤霞,一户是领凤霞去做女儿,另一户是让凤霞去侍候两个老人。我和家珍都觉得那户没有儿女的人家好,把凤霞当女儿,总会多疼爱她一些,就传口信让他们来看看。他们来了,见了凤霞夫妻两个都挺喜欢,一知道凤霞不会说话,他们就改变了主意,那个男的说:“长得倒是挺干净的,只是……”他没往下说,客客气气地回去了。我和家珍只好让另一户人家来领凤霞。那户倒是不在乎凤霞会不会说话,他们说只要勤快就行。

凤霞被领走那天,我扛着锄头准备下地时,她马上就提上篮子和镰刀跟上了我。几年来我在田里干活,凤霞就在旁边割草,已经习惯了。那天我看到她跟着,就推推她,让她回去。她睁圆了眼睛看我,我放下锄头,把她拉回到屋里,从她手里拿过镰刀和篮子,扔到了角落里。她还是睁圆眼睛看着我,她不知道我们把她送给别人了。当家珍给她换上一件水红颜色的衣服时,她不再看我,低着头让家珍给她穿上衣服,那是家珍用过去的旗袍改做的。家珍给她扣纽扣时,她眼泪一颗一颗滴在自己腿上。凤霞知道自己要走了。我拿起锄头走出去,走到门口我对家珍说:“我下地了,领凤霞的人来了,让他带走就是,别来见我。”我到了田里,挥着锄头干活时,总觉得劲使不到点子上。

我是心里发虚啊,往四周看看,看不到凤霞在那里割草,觉得心都空了。想想以后干活时再见不到凤霞,我难受得一点力气都没有。这当儿我看到凤霞站在田埂上,身旁一个五十来岁的男人拉着她的手。凤霞的眼泪在脸上哗哗地流,她哭得身体一抖一抖,凤霞哭起来一点声音也没有,她时不时抬起胳膊擦眼睛,我知道她这样做是为了看清楚她爹。那个男人对我笑了笑,说道:“你放心吧,我会对她好的。”说完他拉了拉凤霞,凤霞就跟着他走了。凤霞手被拉着走去时,身体一直朝我这边歪着,她一直在看着我。凤霞走着走着,我就看不到她的眼睛了,再过一会,她擦眼睛抬起的胳膊也看不到了。这时我实在忍不住了,歪了歪头眼泪掉了下来。家珍走过来时,我埋怨她:“叫你别让他们过来,你偏要让他们过来见我。”家珍说:“不是我,是凤霞自己过来的。”凤霞走后,有庆不干了。起先凤霞被人领走时,有庆瞪着眼睛还不知道出了什么事,直到凤霞走远了,他才挠着头一步一步往回走。我看到他朝我这里张望几下,就是不过来问我。他还在家珍肚子里时我就打过他,他看到我怕。

吃午饭时,桌子旁没有了凤霞,有庆吃了两口就不吃了,眼睛对着我和家珍转来转去,家珍对他说:“快吃。”他摇摇小脑袋,问他娘:“姐姐呢?”家珍一听这话头便低下了,她说:“你快吃。”这小家伙干脆把筷子一放,对他娘叫道:“姐姐什么时候回来?”凤霞一走,我心里本来就乱糟糟的,看到有庆这样子,一拍桌子说:“凤霞不回来啦。”有庆吓得身体抖了一下,看看我没再发火,他嘴巴歪了两下,低着脑袋说:“我要姐姐。”家珍就告诉他,我们把凤霞送给别人家了,为了省下些钱供他上学。听到把凤霞送给了别人,有庆嘴一张哇哇地哭了,边哭边喊:“我不上学,我要姐姐。”我没理他,心想他要哭就让他哭吧,谁知他又叫了:“我不上学。”把我的心都叫乱了,我对他喊:“你哭个屁。”有庆给吓住了,身体往后缩缩,看到我低头重新吃饭,他就离开凳子,走到墙角,突然又喊了一声:“我要姐姐。”我知道这次非揍他不可了,从门后拿出扫帚走过去,对他说:“转过去。”有庆看看家珍,乖乖地转了过去,两只手扶在墙上,我说:“脱掉裤子。”有庆脑袋扭过来,看看家珍,脱下了裤子后又转过脸来看家珍,看到他娘没过来拦我,他慌了。我举起扫帚时,他怯生生地说:“爹,别打我好吗?”他这么说,我心也就软了。有庆也没有错,他是凤霞带大的,他对姐姐亲,想姐姐。我拍拍他的脑袋,说:“快去吃饭吧。”过了两个月,有庆上学的日子到了。凤霞被领走时穿了一件好衣服,有庆上学了还是穿得破破烂烂,家珍做娘的心里怪难受的,她蹲在有庆跟前,替他这儿拉拉,那儿拍拍,对我说:“都没件好衣服。”谁想到有庆这时候又说:“我不上学。”都过去了两个月,我以为他早忘了凤霞的事,到了上学这一天,他又这么叫了。这次我没有发火,好言好语告诉他,凤霞就是为了他上学才送给别人的,他只有好好念书才对得起姐姐。有庆倔劲上来了,他抬起脑袋冲我说:“我就是不上学。”我说:“你屁股又痒啦。”他干脆一转身,脚使劲往地上蹬着走进了里屋,进了屋后喊:“你打死我,我也不上学。”我想这孩子是要我揍他,就提着扫帚进去,家珍拉住我,低声说:“你轻点,吓唬吓唬就行了,别真的揍他。”我一进屋,有庆已经卧在床上了,裤子褪到大腿一面,露着两片小屁股,他是在等我去揍他。他这样子反倒让我下不了手,我就先用话吓唬他:“现在说上学还来得及。”他尖声喊:“我要姐姐。”我朝他屁股上揍了一下,他抱着脑袋说:“不疼。”我又揍了一下,他还是说:“不疼。”这孩子是逼我使劲揍他,真把我气坏了。我就使劲往他屁股上揍,这下他受不了,哇哇地哭,我也不管,还是使劲揍。有庆总还小,过了一会,他实在疼得挺不住,求我了:“爹,别打了,我上学。”有庆是个好孩子。他上学第一天中午回来后,一看到我就哆嗦一下,我还以为他是早晨被我打怕了,就亲热地问他学校好不好,他低着头轻轻嗯了一下,吃饭的时候,他老是抬起头来看看我,一副害怕的样子,让我心里很不是滋味,想想早晨我出手也太重了。到饭快吃完的时候,有庆叫了我一声:“爹。”他说:“老师要我自己来告诉你们,老师批评我了,说我坐在凳子上动来动去,不好好念书。”我一听火就上来了,凤霞都送给了别人,他还不好好念书。我把碗往桌上一拍,他先哭了,哭着对我说:“爹,你别打我。我是屁股疼得坐不下去。”我赶紧把他裤子剥下来一看,有庆的屁股上青一块紫一块,那是早晨揍的,这样怎么让他在凳子上坐下去。看着儿子那副哆嗦的样子,我鼻子一酸,眼睛也湿了。

凤霞让别人领去才几个月,她就跑了回来。凤霞回来时夜深了,我和家珍在床上,听到有人在外面敲门,先是很轻地敲了一下,过了一会儿又敲了两下。我想是谁呀,这么晚了。爬起来去开门,一开门看到是凤霞,都忘了她听不到,赶紧叫:“凤霞,快进来。”我这么一叫,家珍一下子从床上下来,没穿鞋就往门口跑。我把凤霞拉进来,家珍一把将她抱过去呜呜地哭了。我推推她,让她别这样。

凤霞的头发和衣服都被露水沾湿了,我们把她拉到床上坐下,她一只手扯住我的袖管,一只手拉住家珍的衣服,身体一抖一抖哭得都哽住了。家珍想去拿条毛巾给她擦擦头发,她拉住家珍的衣服就是不肯松开,家珍只得用手去替她擦头发。过了很久,她才止住哭,抓住我们的手也松开了。我把她两只手拿起来看了又看,想看看那户人家是不是让凤霞做牛做马地干活,看了很久也看不出个究竟来,凤霞手上厚厚的茧在家里就有了。我又看她的脸,脸上也没有什么伤痕,这才稍稍有些放心。

凤霞头发干了后,家珍替她脱了衣服,让她和有庆睡一头。凤霞躺下后,睁眼看着睡着的有庆好一会,偷偷笑了一下,才把眼睛闭上。有庆翻了个身,把手搁在凤霞嘴上,像是打他姐姐巴掌似的。凤霞睡着后像只小猫,又乖又安静,一动不动。

有庆早晨醒来一看到他姐姐,使劲搓眼睛,搓完眼睛看看还是凤霞,衣服不穿就从床上跳下来,张着个嘴一声声喊:“姐姐,姐姐。”这孩子一早晨嘻嘻笑个不停,家珍让他快点吃饭,还要上学去。他就笑不出来了,偷偷看了我一眼,低声问家珍:“今天不上学好吗?”我说:“不行。”他不敢再说什么,当他背着书包出门时狠狠蹬了几脚,随即怕我发火,飞快地跑了起来。有庆走后,我让家珍拿身干净衣服出来,准备送凤霞回去,一转身看到凤霞提着篮子和镰刀站在门口等着我了,凤霞哀求地看着我,叫我实在不忍心送她回去,我看看家珍,家珍看着我的眼睛也像是在求我,我对她说:“让凤霞再呆一天吧。”我是吃过晚饭送凤霞回去的,凤霞没有哭,她可怜巴巴地看看她娘,看看她弟弟,拉着我的袖管跟我走了。有庆在后面又哭又闹,反正凤霞听不到,我没理睬他。

那一路走得真是叫我心里难受,我不让自己去看凤霞,一直往前走,走着走着天黑了,风飕飕地吹在我脸上,又灌到脖子里去。凤霞双手捏住我的袖管,一点声音也没有。天黑后,路上的石子绊着凤霞,走上一段凤霞的身体就摇一下,我蹲下去把她两只脚揉一揉,凤霞两只小手搁在我脖子上,她的手很冷,一动不动。后面的路是我背着凤霞走去,到了城里,看看离那户人家近了,我就在路灯下把凤霞放下来,把她看了又看,凤霞是个好孩子,到了那时候也没哭,只是睁大眼睛看我,我伸手去摸她的脸,她也伸过手来摸我的脸。她的手在我脸上一摸,我再也不愿意送她回到那户人家去了。背起凤霞就往回走,凤霞的小胳膊勾住我的脖子,走了一段她突然紧紧抱住了我,她知道我是带她回家了。

回到家里,家珍看到我们怔住了,我说:“就是全家都饿死,也不送凤霞回去。”家珍轻轻地笑了,笑着笑着眼泪掉了出来。

三有庆念了两年书,到了十岁光景,家里日子算是好过一些了,那时凤霞也跟看我们一起下地干活,凤霞已经能自己养活自己了。家里还养了两头羊,全靠有庆割草去喂它们。每天蒙蒙亮时,家珍就把有庆叫醒,这孩子把镰刀扔在篮子里,一只手提着,一只手搓着眼睛跌跌冲冲走出屋门去割草,那样子怪可怜的,孩子在这个年纪是最睡不醒的,可有什么办法呢?没有有庆去割草,两头羊就得饿死。到了有庆提着一篮草回来,上学也快迟到了,急忙往嘴里塞一碗饭,边嚼边往城里跑。中午跑回家又得割草,喂了羊再自己吃饭,上学自然又来不及了。有庆十来岁的时候,一天两次来去就得跑五十多里路。

有庆这么跑,鞋当然坏得快。家珍是城里有钱人家出生,觉得有庆是上学的孩子了,不能再光着脚丫,给他做了一双布鞋。我倒觉得上学只要把书念好就行,穿不穿鞋有什么关系。有庆穿上新鞋才两个月,我看到家珍又在纳鞋底,问她是给谁做鞋,她说是给有庆。

田里的活已经把家珍累得说话都没力气了,有庆非得把他娘累死。我把有庆穿了两个月的鞋拿起来一看,这哪还是鞋,鞋底磨穿了不说,一只鞋连鞋帮都掉了。等有庆提着满满一篮草回来时,我把鞋扔过去,揪住他的耳朵让他看看:“你这是穿的,还是啃的?”有庆摸着被揪疼的耳朵,咧了咧嘴,想哭又不敢哭。我警告他:“你再这样穿鞋,我就把你的脚砍掉。”其实是我没道理,家里的两头羊全靠有庆喂它们,这孩子在家干这么重的活,耽误了上学时间总是跑着去,中午放学想早点回来割草,又跑着回来。不说羊粪肥田这事,就是每年剪了羊毛去卖了的钱,也不知道能给有庆做多少双鞋。我这么一说以后,有庆上学就光脚丫跑去,到了学校再穿上鞋。

有一次都下雪了,他还是光着脚丫在雪地里吧哒吧哒往学校跑,让我这个做爹的看得好心疼,我叫住他:“你手里拿着什么?”这孩子站在雪地里看着手里的鞋,可能是糊涂了,都不知道说什么。我说:“那是鞋,不是手套,你给我穿上。”他这才穿上了鞋,缩着脑袋等我下面的话,我向他挥挥


手:“你走吧。”有庆转身往城里跑,跑了没多远,我看到他又脱下了鞋。

这孩子让我一点办法都没有。

到了五八年,人民公社成立了。我家那五亩地全划到了人民公社名下,只留下屋前一小块自留地。村长也不叫村长了,改叫成队长。队长每天早晨站在村口的榆树下吹口哨,村里男男女女都扛着家伙到村口去集合,就跟当兵一样,队长将一天的活派下来,大伙就分头去干。村里人都觉得新鲜,排着队下地干活,嘻嘻哈哈地看着别人的样子笑,我和家珍,凤霞排着队走去还算整齐,有些人家老的老小的小,中间有个老太太还扭着小脚,排出来的队伍难看死了,连队长看了都说:“你们这一家啊,横看竖看还是不好看。”家里五亩田归了人民公社,家珍心里自然舍不得,过来的十来年,我们一家全靠这五亩田养活,眼睛一眨,这五亩田成了大伙的了,家珍常说:“往后要是再分田,我还是要那五亩。”谁知没多少日子,连家里的锅都归了人民公社,说是要煮钢铁,那天队长带着几个人挨家挨户来砸锅,到了我家,笑嘻嘻地对我说:“福贵,是你自己拿出来呢,还是我们进去砸?”我心想反正每家的锅都得砸,我家怎么也逃不了,就说:“自己拿,我自己拿。”我将锅拿出来放在地上,两个年轻人挥起锄头就砸,才那么三、五下,好端端的一口锅就被砸烂了。家珍站在一旁看着心疼的都掉出了眼泪,家珍对队长说:“这锅砸了往后吃什么?”“吃食堂。”队长挥着手说。“村里办了食堂,砸了锅谁都用不着在家做饭啦,省出力气往共产主义跑,饿了只要抬抬腿往食堂门槛里放,鱼啊肉啊撑死你们。”村里办起了食堂,家中的米盐柴什么的也全被村里没收了,最可惜的是那两头羊,有庆把它们养得肥肥壮壮的,也要充公。那天上午,我们一家扛着米,端着盐往食堂送时,有庆牵着两头羊,低着脑袋往晒场去。他心里是一百个不愿意,那两头羊可是他一手喂大的,他天天跑着去学校,又跑着回来,都是为家里的羊。他把羊牵到晒场上,村里别的人家也把牛羊牵到了那里,交给饲养员王喜。别人虽说心里舍不得,交给王喜后也都走开了,只有有庆还在那里站着,咬着嘴唇一动不动,末了可怜巴巴地问王喜:“我每天都能来抱抱它们吗?”村里食堂一开张,吃饭时可就好看了,每户人家派两个人去领饭菜,排出长长一队,看上去就跟我当初被俘虏后排队领馒头一样。每家都是让女人去,叽叽喳喳声音响得就和晒稻谷时麻雀一群群飞来似的。队长说得没错,有了食堂确实省事,饿了只要排个队就有吃有喝了。那饭菜敞开吃,能吃多少就吃多少,天天都有肉吃。最初的几天,队长端着个饭碗嘻嘻笑着挨家串门,问大伙:“省事了吧?这人民公社好不好?”大伙也高兴,都说好,队长就说:“这日子过得比当二流子还舒坦。”家珍也高兴,每回和凤霞端着饭菜回来时就会说:“又吃肉啦。”家珍把饭菜往桌上一放,就出门去喊有庆。有庆有庆的喊上一阵子,才看见他提着满满一篮草在田埂上横着跑过去。

这孩子是给两头羊送草去。村里三头牛和二十多头羊全被关在一个棚里,那群牲畜一归了人民公社,就倒楣了,常常挨饿,有庆一进去就会围上来,有庆就对着它们叫:“喂喂,你们在哪里?”他的两头羊在羊堆里拱出来,有庆才会把草倒在地上,还得使劲把别的羊推开,一直侍候自己的羊吃完,有庆这才呼哧呼哧满头是汗地跑回家来,上学也快迟到了,这孩子跟喝水似的把饭吃下去,抓起书包就跑。

看着他还是每天这么跑来跑去,我心里那个气,嘴上又不好说,说出来怕别人听到了会说我落后,有一次我实在忍不住了,就说:“别人拉屎你擦什么屁股?”有庆听了这话,没明白过来,看了我一会后扑哧笑了,气得我差点没给他一巴掌,我说:“这羊早归了公社,管你屁事。”有庆每天三次给羊送草去,到了天快黑的时候,他还要去一次抱抱那两头羊。管牲畜的王喜见他这么喜欢自己的羊,就说:“有庆,你今晚就领回家去吧,明天一早送回来就是了。”有庆知道我不会让他这么干,摇摇头对王喜说:“我爹要骂我的,我就这么抱一抱吧。”日子一长,棚里的羊也就越少,过几天就要宰一头。到后来只有有庆一个人送草去了,王喜见了我常说:“就有庆还天天惦记着它们,别人是要吃肉了才会想到它们。”村里食堂开张后两天,队长让两个年轻人进城去买煮钢铁的锅,那些砸烂的锅和铁皮什么都堆在晒场上,队长指着它们说:“得赶紧把它们给煮了,不能老让它们闲着。”两个年轻人拿着草绳和扁担进城去后,队长陪着城里请来的风水先生在村里转悠开了,说是要找一块风水宝地煮钢铁。穿长衫的风水先生笑眯眯地走来走去,走到一户人家跟前,那户人家就得倒吸一口冷气,这躬着背的老先生只要一点头,那户人家的屋子就完蛋了。

队长陪着风水先生来到了我家门口,我站在门前心里咚咚地打鼓,队长说:“福贵,这位是王先生,到你这儿来看看。”“好,好。”我连连点着头。

风水先生双手背在身后,前后左右看了一会,嘴里说:“好地方,好风水。”我听了这话眼睛一黑,心想这下完蛋了。好在这时家珍走了出来,家珍看到是她认识的王先生,就叫了一声,王先生说:“是家珍啊。”家珍笑着说:“进屋喝碗茶吧。”王先生摆了摆手,说道:“改日再喝,改日再喝。”家珍说:“听我爹说你这些日子忙坏了?”“忙,忙。”王先生点着头说。“请我看风水的都排着队呢。”说着王先生看看我,问家珍:“这位就是?”家珍说:“是福贵。”王先生眼睛笑得眯成了一条缝,点着头说:“我知道,我知道。”看着王先生这副模样,我知道他是想起我从前赌光家产的事。我就对王先生嘿嘿笑了,王先生向我们双手抱拳说:“改日再聊。”说过他转身对队长说:“到别处去看看。”队长和风水先生一走,我才彻底松了一口气,我这间茅屋算是没事了,可村里老孙家倒大楣了,风水先生看中了他家的屋子。队长让他家把屋子腾出来,老孙头呜呜地哭,蹲在屋角就是不肯搬,队长对他说:“哭什么,人民公社给你盖新屋。”老孙头双手抱着脑袋,还是哭,什么话都不说。到了傍晚,队长看看没有别的法子了,就叫上村里几个年轻人,把老孙头从屋里拉出来,将里面的东西也搬到外面。老孙头被拉出来后,双手抱住了一棵树,怎么也不肯松手,拉他的两个年轻人看看队长说:“队长,拉不动啦。”队长扭头看了看,说:“行啦,你们两个过来点火。”那两个年轻人拿着火柴,站到凳子上,对着屋顶的茅草划燃了火柴。屋顶的茅*荼纠淳*发霉了,加上昨天又下了一场雨,他们怎么也烧不起来。队长说:“他娘的,我就不信人民公社的火还烧不掉这破屋子。”说着队长卷了卷袖管准备自己动手,有人说:“浇上油,一点就燃。”队长一想后说:“对啊,他娘的,我怎么没想到,快去食堂取油。”原先我只觉得自己是个败家子,想不到我们队长也是个败家子。我啊,就站在不到百步远的地方,看着队长他们把好端端的油倒在茅草上,那油可都是从我们嘴里挖出来的,被他们一把火烧没了。那茅草浇上了我们吃的油,火苗子呼呼地往上窜,黑烟在屋顶滚来滚去。我看到老孙头还是抱着那棵树,他是眼睁睁看着自己的窝没了。老孙头可怜,等到屋顶烧成了灰,四面土墙也烧黑了,他才抹着眼泪走开,村里人听到他说:“锅砸了,屋子烧了,看来我也得死了。”那晚上我和家珍都睡不踏实,要不是家珍认识城里看风水的王先生,我这一家人都不知道要到哪里去了。想来想去这都是命,只是苦了老孙头,家珍总觉得这灾祸是我们推到他身上去的,我想想也是这样。我嘴上不这么说,我说:“是灾祸找到他,不能说是我们推给他的。”煮钢铁的地方算是腾出来了,去城里买锅的也回来了。他们买了一只汽油桶回来,村里很多人以前没见过汽油桶,看着都很稀奇,问这是什么玩意,我以前打仗时见过,就对他们说:“这是汽油桶,是汽车吃饭用的饭碗。”队长用脚踢踢汽车的饭碗,说:“太小啦。”买来的人说:“没有更大的了,只能一锅一锅煮了。”队长是个喜欢听道理的人,不管谁说什么,他只要听着有理就相信。他说:“也对,一口吃不成个大胖子,就一锅一锅煮吧。”有庆这孩子看到我们很多人围着汽油桶,提着满满一篮草不往羊棚送,先挤到我们这儿来了,他的脑袋从我腰里一擦一磨地钻出来,我想是谁呀,低头一看是自己儿子。有庆对着队长喊:“煮钢铁桶里要放上水。”大伙听了都笑,队长说:“放上水?你小子是想煮肉吧。”有庆听了这话也嘻嘻笑,他说:“要不钢铁没煮成,桶底就先煮烂啦。”谁知队长听了这话,眉毛往上一吊,看着我说:“福贵,这小子说得还真对。你家出了个科学家。”队长夸奖有庆,我心里当然高兴,其实有庆是出了个馊主意。汽油桶在原先老孙头家架了起来,将砸烂的锅和铁皮什么的扔了进去,里面还真的放上了水,桶顶盖一个木盖,就这样煮起了钢铁。里面的水一开,那木盖就扑扑地跳,水蒸汽呼呼地往外冲,这煮钢铁跟煮肉还真是差不多。

队长每天都要去看几次,每次揭开木盖时,里面发大水似的冲出来蒸汽都吓得他跳开好几步,嘴里喊着:“烫死我啦。”等到水蒸汽少了一些,他就拿着根扁担伸到桶里敲了敲,敲完后骂道:“他娘的,还硬梆梆的。”村里煮钢铁那阵子,家珍病了。家珍得了没力气的病,起先我还以为她是年纪大了,才这样的。那天村里挑羊粪去肥田,那时候田里插满了竹竿,原先竹竿上都是纸做的小红旗,几场雨一下,红旗全没了,只在竹竿上沾了些红纸屑。家珍也挑着羊粪,她走着走着腿一软坐在了地上,村里人见了都笑,说是:“福贵夜里干狠了。”家珍自己也笑了,她站起来试着再挑,那两条腿就哆嗦,抖得裤子像是被风吹的那样乱动起来。我想她是累了,就说:“你歇一会吧。”刚说完,家珍又坐到了地上,担子里的羊粪泼出来盖住了她的腿。家珍的脸一下子红了,她对我说:“我也不知道是怎么了。”我以为家珍只要睡上一觉,第二天就会有力气的。谁想到以后的几天家珍再也挑不动担子了,她只能干些田里的轻活。好在那时是人民公社,要不这日子又难熬了。家珍得了病,心里自然难受,到了夜里她常偷偷问我:“福贵,我会拖累你们吗?”我说:“你别想这事了,年纪大了都这样。”到那时我还没怎么把家珍的病放在心上,我心想家珍自从嫁给我以后,就没过上好日子,现在年纪大了,也该让她歇一歇了。谁知过了一个来月,家珍的病一下子重了,那晚上我们一家守着那汽油桶煮钢铁,家珍病倒了,我才吓一跳,才想到要送家珍去城里医院看看。

那时候钢铁煮了有两个多月了,还是硬梆梆的,队长觉得不能让村里最强壮的几个劳动力整日整夜地守着汽油桶,他说:“往后就挨家挨户轮了。”轮到我家时,队长对我说:“福贵,明天就是国庆节了,把火烧得旺些,怎么也得给我把钢铁煮出来。”我让家珍和凤霞早早地去食堂守着,好早些把饭菜打回来,吃完了去接替人家,我怕去晚了人家会说闲话。可是家珍和凤霞打了饭菜回来,左等右等不见有庆回来,家珍站在门前喊得额头都出汗了,我知道这孩子准是割了草送到羊棚去了。我对家珍说:“你们先吃。”说完我出门就往村里羊棚去,心想这孩子太不懂事了,不帮着家珍干些家里的活,整天就知道割羊草,胳膊一个劲地往外拐。我走到羊棚前,看到有庆正把草倒在地上,棚里只有六只羊了,全挤上来抢着吃草,有庆提着篮子问王喜:“他们会宰我的羊吗?”王喜说:“不会了,把羊吃光了,上哪儿去找肥料,没有了肥料田里的庄稼就长不好。”王喜看到我走进去,对有庆说:“你爹来了,你快回去吧。”有庆转过身来,我伸手拍拍他的脑袋,这孩子刚才问王喜时的可怜腔调,让我有火发不出。我们往家里走去,有庆看到我没发火,高兴地对我说:“他们不会宰我的羊了。”我说:“宰了才好。”到了晚上,我们一家就守着汽油桶煮钢铁了,我负责往桶里加水,凤霞拿一把扇子扇火,家珍和有庆捡树枝。直干到半夜,村里所有人家都睡了,我都加了三次水,拿一根树枝往里捅了捅,还是硬梆梆的。家珍累得满脸是汗,她弯腰放下树枝时都跪在了地上。我盖上木盖对她说:“你怕是病了。”家珍说:“我没病,只是觉得身体软。”那时候有庆靠着一棵树像是睡着了,凤霞两只手换来换去地扇着风,她是胳膊疼了。我去推推她,她以为我要替她,转过脸来直摇头,我就指指有庆,要她把有庆抱回家去,她这才点着头站起来。村里羊棚里传来咩咩的叫声,睡着的有庆听到这声音格格地笑了,当凤霞要去抱他时,他突然睁开眼睛说:“是我的羊在叫。”我还以为他睡着了,看到他睁开眼睛,又说是他的羊什么的,我火了,对他说:“是人民公社的羊,不是你的。”这孩子吓一跳,瞌睡


全没了,眼睛定定地看着我。家珍推推我,说我:“你别吓唬他。”说着蹲下去对有庆轻声说:“有庆,你睡吧,睡吧。”这孩子看看家珍,点点头闭上了眼睛,没一会儿功夫就呼呼地睡去了,我把有庆抱起来,放到凤霞背脊上,打着手势告诉凤霞,让她和有庆回家去睡觉,别来了。

凤霞背着有庆走后,我和家珍坐在了火前,那时天很凉,坐在火前暖和,家珍累得一点力气都没了,胳膊抬起来都费劲,我就让家珍靠着我,说:“你就闭上眼睛睡一会吧。”家珍的脑袋往我肩膀上一靠,我的瞌睡也来了,脑袋老往下掉,我使劲挺一会,不知不觉又掉了下去。我最后一次往火里加了树枝后,脑袋掉下去就没再抬起来。

我不知道自己睡了有多久,后来轰的一声巨响,把我吓得从地上一下子坐起来,那时候天都快亮了,我看到汽油桶已经倒在了地上,火像水一样流成一片在烧,我身上盖着家珍的衣服,我立刻跳起来,围着汽油桶跑了两圈,没见到家珍,我吓坏了,吼着嗓子叫:“家珍,家珍。”我听到家珍在池塘那边轻声答应,我跑过去看到家珍坐在地上,正使劲想站起来,我把她扶起来时,发现她身上的衣服都湿透了。

我睡着以后,家珍一直没睡,不停地往火上加树枝,后来桶里的水快煮干了,她就拿着木桶去池塘打水,她身上没力气,拿着个空桶都累,别说是满满一桶水了,她提起来才走了五、六步就倒在地上,她坐在地上歇了一会,又去打了一桶水,这会她走一步歇一下,可刚刚走上池塘人又滑倒了,前后两桶水全泼在她身上,她坐在地上没力气起来了,一直等到我被那声巨响吓醒。

看到家珍没伤着,我悬着的心放下了,我把家珍扶到汽油桶前,还有一点火在烧,我一看是桶底煮烂了,心想这下糟了。家珍一看这情形,也傻了,她一个劲地埋怨自己:“都怪我,都怪我。”我说:“是我不好,我不该睡着。”我想着还是快些去报告队长吧,就把家珍扶到那棵树下,让她靠着树坐下。自己往我家从前的宅院,后来是龙二,现在是队长的屋子跑去,跑到队长屋前,我使劲喊:“队长,队长。”队长在里面答应:“谁呀?”我说:“是我,福贵,桶底煮烂啦。”队长问:“是钢铁煮成啦?”我说:“没煮成。”队长骂道:“那你叫个屁。”我不敢再叫了,在那里站着不知道该怎么办,那时候天都亮了,我想了想还是先送家珍去城里医院吧,家珍的病看样子不轻,这桶底煮烂的事待我从医院回来再去向队长做个交待。我先回家把凤霞叫醒,让她也去,家珍是走不动了,我年纪大了,背着家珍来去走二十多里路看来不行,只能和凤霞轮流着背她。

我背起家珍往城里走,凤霞走在一旁,家珍在我背上说:“我没病,福贵,我没病。”我知道她是舍不得花钱治病,我说:“有没有病,到医院一看就知道了。”家珍不愿意去医院,一路上嘟嘟哝哝的。走了一段,我没力气了,就让凤霞替我。凤霞力气比我都大,背着她娘走起路来咚咚响,家珍到了凤背脊上,不再嘟哝什么,突然笑起来,宽慰地说:“凤霞长大了。”家珍说完这话眼睛一红,又说:“凤霞要是不得那场病就好了。”我说:“都多少年的事了,还提它干什么。”城里医生说家珍得了软骨病,说这种病谁也治不了,让我们把家珍背回家,能给她吃得好一点就吃得好一点,家珍的病可能会越来越重,也可能就这样了。回来的路上是凤霞背着家珍,我走在边上心里是七上八下,家珍得了谁也治不了的病,我是越想越怕,这辈子这么快就到了这里,看着家珍瘦得都没肉的脸,我想她嫁给我后没过上一天好日子。

家珍反倒有些高兴,她在凤霞背上说:“治不了才好,哪有钱治病。”快到村口时,家珍说她好些了,要下来自己走,她说:“别吓着有庆了。”她是担心有庆看到她这副模样会害怕,做娘的心里就是想得细。她从凤霞背上下来,我们去扶她,她说自己能走,说:“其实也没什么病。”这时村里传来了锣鼓声,队长带着一队人从村口走出来,队长看到我们后高兴地挥着手喊道:“福贵,你们家立大功啦。”我是丈二和尚摸不着头脑,不知道立了什么大功,等他们走近了,我看到两个村里的年轻人抬着一块乱七八糟的铁,上面还翘着半个锅的形状,和几片耸出来的铁片,一块红布挂在上面。队长指指这烂铁说:“你家把钢铁煮出来啦,赶上这国庆节的好时候,我们上县里去报喜。”一听这话我傻了,我还正担心着桶底煮烂了怎么去向队长交待,谁想到钢铁竟然煮出来了。队长拍拍我的肩膀说:“这钢铁能造三颗炮弹,全部打到台湾去,一颗打在蒋介石床上,一颗打在蒋介石吃饭的桌上,一颗打在蒋介石家的羊棚里。”说完队长手一挥,十来个敲锣打鼓的人使劲敲打起来,他们走过去后,队长在锣鼓声里回过头来喊道:“福贵,今天食堂吃包子,每个包子都包进了一头羊,全是肉。”他们走远后,我问家珍:“这钢铁真的煮成了?”家珍摇摇头,她也不知道是怎么煮成的。我想着肯定是桶底煮烂时,钢铁煮成的。要不是有庆出了个馊主意,往桶里放水,这钢铁早就能煮成了。等我们回到家里时,有庆站在屋前哭得肩膀一抖一抖,他说:“他们把我的羊宰了,两头羊全宰了。”有庆伤心了好几天,这孩子每天早晨起来后,用不着跑着去学校了。我看着他在屋前游来荡去,不知道该干什么,往常这个时候他都是提着个篮子去割草了。家珍叫他吃饭,叫一声他就进来坐到桌前,吃完饭背起书包绕到村里羊棚那里看看,然后无精打采地往城里学校去了。

村里的羊全宰了吃光了,那三头牛因为要犁田才保住性命,粮食也快吃光了。队长说到公社去要点吃的来,每次去都带了十来个年轻人,打着十来根扁担,那样子像是要去扛一座金山回来,可每次回来仍然是十来个人十来根扁担,一粒米都没拿到,队长最后一次回来后说:“从明天起食堂散伙了,大伙赶紧进城去买锅,还跟过去一样,各家吃各家自己的。”当初砸锅凭队长一句话,买锅了也是凭队长一句话。食堂把剩下的粮食按人头分到各家,我家分到的只够吃三天。好在田里的稻子再过一个月就收起来了,怎么熬也能熬过这一个月。

村里人下地干活开始记工分了,我算是一个壮劳力,给我算十分,家珍要是不病,能算她八分,她一病只能干些轻活,也就只好算四分了。好在凤霞长大了,凤霞在女人里面算是力气大的,她每天能挣七个工分。

家珍心里难受,她挣的工分少了一半,想不开,她总觉得自己还能干重活,几次都去对队长说,说她也知道自己有病,可现在还能干重活。她说:“等我真干不动了再给我记四分吧。”队长一想也对,就对她说:“那你去割稻子吧。”家珍拿着把镰刀下到稻田里,刚开始割得还真快,我看着心想是不是医生弄错了。可割了一道,她身体就有些摇晃了,割第二道时慢了许多,我走过去问她:“你行吗?”她那时满脸是汗,直起腰来还埋怨我:“你干你的,过来干什么?”她是怕我这么一过去,别人都注意她了,我说:“你自己留意着身体。”她急了,说:“你快走开。”我摇摇头,只好走开。我走开后没过多久,听到那边扑通一声,我心想不好,抬头一看家珍摔在地上了。我走到跟前,家珍虽说站了起来,可两条腿直哆嗦,她摔下去时头碰着了镰刀,额头都破了,血在那里流出来。她苦笑着看我,我一句话不说,背起她就往家里去,家珍也不反抗,走了一段,家珍哭了,她说:“福贵,我还能养活自己吗?”“能。”我说。

以后家珍也就死心了,虽然她心疼丢掉的那四个工分,想着还能养活自己,家珍多少还是能常常宽慰自己。

家珍病后,凤霞更累了,田里的活一点没少干,家里的活她也得多干,好在凤霞年纪轻,一天累到晚,睡上一觉就又有力气有精神了。有庆开始帮着干些自留地上的活,有天傍晚我收工回家,在自留地锄草的有庆叫了我一声,我走过去,这孩子手摸着锄头柄,低着头说:“我学会了很多字。”我说:“好啊。”他抬头看了我一眼,又说:“这些字够我用一辈子了。”我想这孩子口气真大,也没在意他是什么意思,我随口说:“你还得好好学。”他这才说出真话来,他说:“我不想念书了。”我一听脸就沉下了,说:“不行。”其实让有庆退学,我也是想过的,我打消这个念头是为了家珍,有庆不念书,家珍会觉得是自己病拖累他的。我对有庆说:“你不好好念书,我就宰了你。”说过这话后,我有些后悔,有庆还不是为了家里才不想念书的,这孩子十二岁就这么懂事了,让我又高兴又难受,想想以后再不能随便打骂他了。这天我进城卖柴,卖完了我花五分钱给有庆买了五颗糖,这是我这个做爹的第一次给儿子买东西,我觉得该疼爱疼爱有庆了。

我挑着空担子走进学校,学校里只有两排房子,孩子在里面咿呀咿呀地念书,我挨个教室去看有庆。有庆在最边上的教室,一个女老师站在黑板前讲些什么,我站在一个窗口看到了有庆,一看到有庆我气就上来了,这孩子不好好念书,正用什么东西往前面一个孩子头上扔。为了他念书,凤霞都送给过别人,家珍病成这样也没让他退学,他嘻嘻哈哈跑到课堂上来玩了。当时我气得什么都顾不上了,把担子一放,冲进教室对准有庆的脸就是一巴掌。有庆挨了一巴掌才看到我,他吓得脸都白了,我说:“你气死我啦。”我大声一吼,有庆的身体就哆嗦一下,我又给他一巴掌,有庆缩着身体完全吓傻了。这时那个女老师走过来气冲冲问我:“你是什么人?这是学校,不是乡下。”我说:“我是他爹。”我正在气头上,嗓门很大。那个女老师火也跟着上来,她尖着嗓子说:“你出去,你哪像是爹,我看你像法西斯,像国民党。”法西斯我不知道,国民党我就知道了。我知道她是在骂我,难怪有庆不好好念书,他摊上了一个骂人的老师。我说:“你才是国民党,我见过国民党,就像你这么骂人。”那个女老师嘴巴张了张,没说话倒哭上了。旁边教室的老师过来把我拉了出去,他们在外面将我围住,几张嘴同时对我说话,我是一句都没听清。后来又过来一个女老师,我听到他们叫她校长,校长问我为什么打有庆,我一五一十地把凤霞过去送人,家珍病后没让有庆退学的事全说了,那位女校长听后对别的老师说:“让他回去吧。”我挑着担收走时,看到所有教室的窗口都挤满了小脑袋,在看我的热闹。这下我可把自己儿子得罪了,有庆最伤心的不是我揍他,是当着那么多老师和同学出丑。我回到家里气还没消,把这事跟家珍说,家珍听完后埋怨我,她说:“你呀,你这样让有庆在学校里怎么做人。”我听后想了想,觉得自己确实有些过分,丢了自己的脸不说,还丢了我儿子的脸。这天中午有庆放学回家,我叫了他一声,他理都不理我,放下书包就往外走,家珍叫了他一声,他就站住了,家珍让他走过去。有庆走到他娘身边,脖子就一抽一抽了,哭得那个伤心啊。

后来的一个多月里,有庆死活不理我,我让他干什么他马上干什么,就是不和我说话。这孩子也不做错事,让我发脾气都找不到地方。

想想也是自己过分,我儿子的心叫我给伤透了。好在有庆还小,又过了一阵子,他在屋里进出脖子没那么直了。虽然我和他说话,他还是没答理,脸上的模样我还是看得出来的,他不那么记仇了,有时还偷偷看我。我知道他,那么久不和我说话,是不好意思突然开口。我呢,也不急,是我的儿子总是要开口叫我的。

食堂散伙以后,村里人家都没了家底,日子越过越苦,我想着把家里最后的积蓄拿出来,去买一头羊羔。羊是最养人的,能肥田,到了春天剪了羊毛还能卖钱。再说也是为了有庆,要是给这孩子买一头羊羔回来,他不知道会有多高兴。

我跟家珍一商量,家珍也高兴,说你快去买吧。当天下午,我将钱揣在怀里就进城去了。我在城西广福桥那边买了一头小羊,回来时路过有庆他们的学校,我本想进去让有庆高兴高兴,再一想还是别进去了,上次在学校出丑,让我儿子丢脸。我再去,有庆心里肯定不高兴。

等我牵着小羊出了城,走到都快能看到自己家的地方,后面有人噼噼啪啪地跑来,我还没回头去看是谁,有庆就在后面叫上了:“爹,爹。”我站住脚,看着有庆满脸通红地跑来,这孩子一看到我牵着羊,早就忘了他不和我说话这事,他跑到跟前喘着气说:“爹,这羊是给我买的?”我笑着点点头,把绳子递给他说:“拿着。”有庆接过绳子,把小羊抱起来走了几步,又放下小羊,捏住羊的后腿,蹲下去看看,看完后说:“爹,是母羊。”我哈哈地笑了,伸手捏住他的肩膀,有庆的肩膀又瘦又小,我一捏住不知为何就心疼起来,我们一起往家里走去时,我说道:“有庆,你也慢慢长大了,爹以后不会再揍你了,就是揍你也不会让别人看到。


”说完我低头看看有庆,这孩子脑袋歪着,听了我的话,反倒不好意思了。

家里有了羊,有庆每天又要跑着去学校了,除了给羊割草,自留地里的活他也要多干。没想到有庆这么跑来跑去,到头来还跑出名堂来了。城里学校开运动会那天,我进城去卖菜,卖完了正要回家,看到街旁站着很多人,一打听知道是那些学生在比赛跑步,要在城里跑上十圈。

当时城里有中学了,那一年有庆也读到了四年级。城里是第一次开运动会,念初中的孩子和念小学的孩子都一起跑。

我把空担子在街旁放下,想看看有庆是不是也在里面跑。过了一会,我看到一伙和有庆差不多大的孩子,一个个摇头晃脑跑过来,有两个低着脑袋跌跌撞撞,看那样子是跑不动了。

他们跑过去后,我才看到有庆,这小家伙光着脚丫,两只鞋拿在手里,呼哧呼哧跑来了,他只有一个人跑来。看到他跑在后面,我想这孩子真是没出息,把我的脸都丢光了。可旁边的人都在为他叫好,我就糊涂了,正糊涂着看到几个初中学生跑了过来,这一来我更糊涂了,心想这跑步是怎么跑的。

我问身旁一个人:“怎么年纪大的跑不过年纪小的?”那人说:“刚才跑过去的小孩把别人都甩掉了几圈了。”我一听,他不是在说有庆吗?当时那个高兴啊,是说不出来的高兴。就是比有庆大四、五岁的孩子,也被有庆甩掉了一圈。我亲眼看着自己的儿子,光着脚丫,鞋子拿在手里,满脸通红第一个跑完了十圈。这孩子跑完以后,反倒不呼哧呼哧喘气了,像是一点事情都没有,抬起一只脚在裤子上擦擦,穿上布鞋后又抬起另一只脚。接着双手背到身后,神气活现地站在那里看着比他大多了的孩子跑来。

我心里高兴,朝他喊了一声:“有庆。”挑着空担子走过去时我大模大样,我想让旁人知道我是他爹。有庆一看到我,马上不自在了,赶紧把背在身后的手拿到前面来,我拍拍他的脑袋,大声说:“好儿子啊,你给爹争气啦。”有庆听到我嗓门这么大,急忙四处看看,他是不愿意让同学看到我。这时有个大胖子叫他:“徐有庆。”有庆一转身就往那里去,这孩子对我就是不亲。他走了几步又回过头来说:“是老师叫我。”我知道他是怕我回家后找他算帐,就对他挥挥手:“去吧,去吧。”那个大胖子手特别大,他按住有庆的脑袋,我就看不到儿子的头,儿子的肩膀上像是长出了一只手掌。他们两个人亲亲热热地走到一家小店前,我看着大胖子给有庆买了一把糖,有庆双手捧着放进口袋,一只手就再没从口袋里出来。走回来时有庆脸都涨红了,那是高兴的。

那天晚上我问他那个大胖子是谁,他说:“是体育老师。”我说了他一句:“他倒是像你爹。”有庆把大胖子给他的糖全放在床上,先是分出了三堆,看了又看后,从另两堆里各拿出两颗放进自己这一堆,又看了一会,再从自己这堆拿出两颗放到另两堆里。我知道他要把一堆给凤霞,一堆给家珍,自己留着一堆,就是没有我的。谁知他又把三堆糖弄到一起,分出了四堆,他就这么分来分去,到最后还是只有三堆。

过了几天,有庆把体育老师带到家里来了,大胖子把有庆夸了又夸,说他长大了能当个运动员,出去和外国人比赛跑步。有庆坐在门槛上,兴奋得脸上都出汗了。当着体育老师的面我不好说什么,他走后,我就把有庆叫过来,有庆还以为我会夸他,看着我的眼睛都亮闪闪的,我对他说:“你给我,给你娘你姐姐争了口气,我很高兴。可我从没听说过跑步也能挣饭吃,送你去学校,是要你好念书,不是让你去学跑步,跑步还用学?鸡都会跑?”有庆脑袋马上就垂下了,他走到墙角拿起篮子和镰刀,我问他:“记住我的话了吗?”他走到门口,背对着我点点头,就走了出去。

那一年,稻子还没黄的时候,稻穗青青的刚长出来,就下起了没完没了的雨,下了差不多有一个来月,中间虽说天气晴朗过,没出两天又阴了,又下上了雨。我们是看着水在田里积起来,雨水往上长,稻子就往下垂,到头来一大片一大片的稻子全淹没到了水里。村里上了年纪的人都哭了,都说:“往后的日了怎么过呀?”年纪轻一些的人想得开些,总觉得国家会来救济我们的,他们说:“愁什么呀,天无绝人之路,队长去县里要粮食啦。”队长去了三次公社,一次县里,他什么都没拿回来,只是带回来几句话:“大伙放心吧,县长说了,只要他不饿死,大伙也都饿不死。”那一个月的雨下过去后,连着几天的大热天,田里的稻子全烂了,一到晚上,*绱倒*是一片片的臭味,跟死人的味道差不多。原先大伙还指望着稻草能派上用场,这么一来稻子没收起,稻草也全烂光了。什么都没了,队长说起来县里会给粮食的,可谁也没见到有粮食来,嘴上说说的事让人不敢全信,不信又不敢,要不这日子过下去谁也没信心了。

大伙都数着米下锅,积蓄下来的粮食都不多,谁家也不敢煮米饭,都是熬粥喝,就是粥也是越来越薄。那么过了三、两个月,也就坐吃山空了。我和家珍商量着把羊牵到城里卖了,换些米回来,我们琢磨着这羊能换回来百十来斤大米,这样就可以熬到下一季稻子收割的时候。

家里人都有一、两个月没怎么吃饱了,那头羊还是肥肥的,每天在羊棚里中咩咩叫时声音又大又响,全是有庆的功劳,这孩子吃不饱整天叫着头晕,可从没给羊少割过一次草,他心疼那头羊,就跟家珍心疼他一样。

我和家珍商量以后,就把这话对有庆说了。那时候有庆刚把一篮草倒到羊棚里,羊沙沙地吃着草,那声响像是在下雨,他提着空篮子站在一旁,笑嘻嘻地看着羊吃草。

我走进去他都不知道,我把手放在他肩上,这孩子才扭头看了看我,说:“它饿坏了。”我说:“有庆,爹有事要跟你说。”有庆答应一声,把身体转过来,我继续说:“家里粮食吃得差不多了,我和你娘商量着把羊卖掉,换些米回来,要不一家人都得挨饿了。”有庆低着脑袋一声不吭,这孩子心里是舍不得这头羊,我拍拍他的肩说:“等日子好过一些了,我再去买头羊回来。”有庆点点头,有庆是长大了,他比过去懂事多了。要是早上几年,他准得又哭又闹。我们从羊棚里走出来时,有庆拉了拉我的衣服,可怜巴巴地说:“爹,你别把它卖给宰羊的好吗?”我心想这年月谁家还会养着一头羊,不卖给宰羊的,去卖给谁呢?看着有庆那副样子,我也只好点点头。

第二天上午,我将米袋搭在肩上,从羊棚里把羊牵出来,刚走到村口,听到家珍在后面叫我,回过头去看到家珍和有庆走来,家珍说:“有庆也要去。”我说:“礼拜天学校没课,有庆去干什么?”家珍说:“你就让他去吧。”我知道有庆是想和羊多呆一会,他怕我不答应,让他娘来说。我心想他要去就让他去吧,就向他招了招手,有庆跑上来接过我手里的绳子,低着脑袋跟着我走去。

这孩子一路上什么话都不说,倒是那头羊咩咩叫唤个不停,有庆牵着它走,它时时脑袋伸过去撞一下有庆的屁股。羊也是通人性的,它知道是有庆每天去喂它草吃,它和有庆亲热。它越是亲热,有庆心里越是难受,咬着嘴唇都要哭出来了。

看着有庆低着脑袋一个劲地往前走,我心里怪不是滋味的,就找话宽慰他,我说:“把它卖掉总比宰掉它好。羊啊,是牲畜,生来就是这个命。”走到了城里,快到一个拐弯的地方时,有庆站住了脚,看看那头羊说:“爹,我在这里等你。”我知道他是不愿看到把羊卖掉,就从他手里接过绳子,牵着羊往前走,走了没几步,有庆在后面喊:“爹,你答应过的。”我回头问:“我答应什么?”有庆有些急了,他说:“你答应不卖给宰羊的。”我早就忘了昨天说过的话,好在有庆不跟着我了,要不这孩子肯定会哭上一阵子。我说:“知道。”我牵着羊拐了个弯,朝城里的肉铺子走去。先前挂满肉的铺子里,到了这灾年连个肉屁都看不到了,里面坐着一个人,懒洋洋的样子。我给他送去一头羊,他没显得有多高兴。

我们一起给羊上秤时,他的手直哆嗦,他说:“吃不饱,没力气了。”连城里人都吃不饱了。他说他的铺子有十来天没挂过肉了,他的手往前指了指,指到二十米远的一根电线杆,说:“你等着吧,不出一个小时,买肉的排队会排到那边。”他没说错,才等我走开,就有十来个人在那里排队了。米店也排队,我原以为那头羊能换回百十来斤米,结果我只背回家四十斤米。我路过一家小店时,掏出两分钱给有庆买了两颗硬糖,我想有庆辛辛苦苦了一年,也该给他甜甜嘴。

我扛着四十斤大米往回走,有庆在那地方走来走去,踢着一颗小石子。我把两颗糖给他,他一颗放在口袋里,剥开另一颗放进嘴里。我们往前走去,有庆将糖纸叠得整整齐齐拿在手上,然后抬起脑袋问我:“爹,你吃吗?”我摇摇头说:“你自己吃。”我把四十斤米扛回家,家珍一看米袋就知道有多少米,她叹息一声,什么话也没说。最难的是家珍,一家四张嘴每天吃什么?愁得她晚上都睡不好觉。日子再苦也得往下熬,她每天提着篮子去挖野菜,身体本来就有病,又天天忍饥挨饿,那病真让医生说中了,越来越重,只能拄着根树枝走路,走上二十来步就要满头大汗。别人家挖野菜都是蹲下去,她是跪到地上,站起来时身体直打晃,我见了心里不好受*运担*“你就别出门了。”她不答应,拄着树枝往屋外走,我抓住她的胳膊一拉,她身体就往地上倒。家珍坐到地上呜呜地哭上了,她说:“我还没死,你就把我当死人了。”我是一点办法都没有。女人啊,性子上来了什么事都干,什么话都说。我不让她干活,她就觉得是在嫌弃她。

没出三个月,那四十斤米全吃光了。要不是家珍算计着过日子,掺和着吃些南瓜叶,树皮什么的,这些米不够我们吃半个月。那时候村里谁家都没有粮食了,野菜也挖光了,有些人家开始刨树根吃了。村里人越来越少,每天都有拿着个碗外出去要饭的人。队长去了几次县里,回来时都走不到村口,一屁股坐在地上直喘气,在田里找吃的几个人走上去问他:“队长,县里什么时候给粮食?”队长歪着脑袋说:“我走不动了。”看着那些外出要饭的人,队长对他们说:“你们别走了,城里人也没吃的。”明知道没有野菜了,家珍还是整天拄着根树枝出去找野菜,有庆跟着她。有庆正在长身体,没有粮食吃,人瘦得像根竹竿。有庆总还是孩子,家珍有病路都走不动了,还是到处转悠着找野菜,有庆跟在后面,老是对家珍说:“娘,我饿得走不动了。”家珍上哪儿去给有庆找吃的,只好对他说:“有庆,你就去喝几口水填填肚子吧。”有庆也只能到池塘边去咕咚咕咚地喝一肚子水来充饥了。

凤霞跟着我,扛着把锄头去地里掘地瓜。那些田地不知道被翻过多少遍了,可村里的人还都用锄头去掘,有时干一天也只是掘出一根烂瓜藤来。凤霞也饿得慌,脸都青了,看她挥锄头时脑袋都掉下去了。这孩子不会说话,只知道干活。

我往哪儿走,她就往哪儿跟,我想想这样不行,我得和凤霞分开去挖地瓜,老凑在一起不是个办法。我就打着手势让凤霞到另一块地里去。谁知道凤霞一和我分开,就出事了。

凤霞和村里王四在一块地里挖地瓜,王四那人其实也不坏,我被抓了壮丁去打仗那阵子,王四和他爹还常帮家珍干些重活。人一饿就什么缺德事都干得出来,明明是凤霞挖到一个地瓜,王四欺负凤霞不会说话,趁凤霞用衣角擦上面的泥时,一把抢了过去。凤霞平常老实得很,到那时她可不干了,扑上去要把地瓜抢回来。王四哇哇一叫,旁边地里的人见了都看到是凤霞在抢。王四对着我喊:“福贵,做人得讲良心啊,再饿也不能抢别人家的东西。”我看到凤霞正使劲掰他捏住地瓜的手指,赶紧走过去拉开凤霞,凤霞急得眼泪都出来了,她打着手势告诉我是王四抢了她的地瓜,村里别的人也看明白了,就问王四:“是你抢她的?还是她抢你的?”王四做出一副委屈的样子,说:“你们都看到的,明明是她在抢。”我说:“凤霞不是那种人,村里人都知道。王四,这地瓜真是你的,你就拿走。要不是你的,你吃了也会肚子疼。”王四用手指指凤霞,说道:“你让她自己说,是谁的。”他明知道凤霞不会说话,还这么说,气得我身体都哆嗦了。凤霞站在一旁嘴巴一张一张没有声音,倒是泪水刷刷地流着。我向王四挥挥手说:“你要是不怕雷公打你,就拿去吧。”王四做了亏心事也不脸红,他直着脖子说:“是我的我当然要拿走。”说着他转身就走,谁也没想到凤霞挥起锄头就朝他砸去,要不是有人惊叫一声,让王四躲开的话,可就出人


命了。王四看到凤霞砸他,伸手就打了凤霞一巴掌,凤霞哪有他有力气,一巴掌就把凤霞打到地上去了。那声音响得就跟人跳进池塘似的,一巴掌全打在我心上。我冲上去对准王四的脑袋就是一拳,王四的脑袋直摇晃,我的手都打疼了。王四回过神来操起一把锄头朝我劈过来,我跳开后也挥起一把锄头。

要不是村里人拦住我们,总得有一条命完蛋了。后来队长来了,队长听我们说完后骂我们:“他娘的,你们死了让老子怎么去向上面交待。”骂完后队长说:“凤霞不会是那种人,说是你王四抢的也没人看见,这样吧,你们一家一半。”说着队长向王四伸出手,要王四把地瓜给他。王四双手拿着地瓜舍不得交出来,队长说:“拿来呀。”王四没办法,哭丧着脸把地瓜给了队长。队长向旁人要过来一把镰刀,将地瓜放在田埂上,咔嚓一声将地瓜切成两半。队长的手偏了,一半很大,另一半很小。我说:“队长,这怎么分啊?”队长说:“这还不容易。”又是咔嚓一声将大的切下来一块,放进自己口袋,算是他的了。他拿起剩下的两块地瓜给我和王四,说:“差不多大小了吧?”其实一块地瓜也填不饱一家人的肚子,当初心里想的和现在不一样,在当初那可是救命稻草。家里断粮都有一个月了,田里能吃的也都吃得差不多了,那年月拿命去换一碗饭回来也都有人干。

和王四争地瓜的第二天,家珍拄着根树枝走出了村口,我在田里见了问她去哪*担*“我进城去看看爹。”做女儿的想去看爹,我想拦也不能拦,看着她走路都费劲的模样,我说:“让凤霞也去,路上能照应你。”家珍听了这话头也不回地说:“不要凤霞去。”那些日子她脾气动不动就上来,我不再说什么,看着她慢慢吞吞往城里走,她瘦得身上都没肉了,原先绷起的衣服变得松松垮垮,在风里荡来荡去。

我不知道家珍进城是去要吃的,她去了一天,快到傍晚时才回来。回来时都走不动路了。是凤霞先看到她,凤霞拉了拉我的衣服,我转过身去才看到家珍站在那条路上,身体撑在拐杖上向我们招手,她抬起胳膊时脑袋像是要从肩膀上掉下去了。

我赶紧跑过去,等我跑近了,她身体一软跪在了地上,双手撑着拐杖声音很轻地叫:“福贵,你来,你来。”我伸手去扶她起来,她抓住我的手往胸口拉,喘着气说:“你摸摸。”我的手伸进她胸口一摸,人就怔住了,我摸到了一小袋米,我说:“是米。”家珍哭了,她说:“是爹给我的。”那时候的一袋米,可就是山珍海味了。一家人有一、两个月没尝过米的味道了,那种高兴劲啊,实在是说不出来。我让凤霞扶着家珍赶紧回家,自己去找有庆。有庆那时正在池塘旁躺着,他刚喝饱了池水,我叫他:“有庆,有庆。”这孩子脖子歪了歪,有气无力地答应了一声,我低声对他说:“快回家去喝粥。”有庆一听有粥喝,不知哪来的力气,一下子坐了起来,叫道:“喝粥。”我吓了一跳,急忙说:“轻点。”可不能让别人家知道,家珍是把米藏在胸口衣服里带回来的。等一家人回到了家里,我关上门插上木销,家珍这才从胸口拿出那一小袋米,往锅里倒了半袋,加上水后凤霞就生火熬粥了。我让有庆站在门后,从缝里看着有没有村里人走来。水一开,米香就飘满了屋子,有庆在门后站不住了,跑到锅前凑上去鼻子闻了又闻,说:“好香啊。”我把他拉开,说:“去门后看着。”这孩子猛吸了两口热气才回到门后,家珍笑起来,说道:“总算能让你们吃上一顿好的了。”说着家珍掉出了眼泪,她说:“这米是从我爹牙缝里挤出来的。”这时外面有人走来,走到门口叫:“福贵。”我们吓得气都不敢出了,有庆站在那里弓着腰一动不动,只有凤霞笑嘻嘻地往灶里添柴,她听不到。我拍拍她,让她手脚轻一点。听着屋里没有声音,外面那人很不高兴地说:“烟囱呼呼地冒烟,里面没人答应。”过了一会,那人像是走开了,有庆又在门后往外望了一阵,才悄悄地告诉我们:“走啦。”我和家珍总算舒了一口气。粥熬成后,我们一家四口人坐在桌前,喝起了热腾腾的米粥。这辈子我再没像那次吃得那么香了,那味道让我想起来就要流口水。有庆喝得急,第一个喝完,张着嘴大口大口地吸气,他嘴嫩,烫出了很多小泡,后来疼了好几天。等我们吃完后,队长他们来了。

村里人也都有一、两个月没吃上米了,我们关上门,烟囱往外呼呼地冒烟,他们全看到了。刚才有人来叫门,我们没答应,他回去一说,来了一伙人,队长走在前头。他们猜到我们有好吃的,都想来吃一口。

队长一进屋鼻子就一抖一抖了,问:“煮什么吃啦,这么香。”我嘿嘿笑着没说话,我不说话队长也不好再问。家珍招呼着他们坐下,有几个人不老实,又去揭锅又掀褥子,好在家珍将剩下的米藏在胸口了,也不怕他们乱翻。队长看不下去了,他说:“你们干什么,这是在别人家里。出去,出去,他娘的都出去。”队长把他们赶走后,起身关上门,也不先和我们套套近乎,一下子就把脸凑过来说:“福贵,家珍,有好吃的分我一口。”我看看家珍,家珍看看我,平日里队长对我们不错,眼下他求上我们了,总不能不答应。家珍伸手从胸口拿出那个小袋子,抓了一小把给队长,说:“队长,就这么多了,你拿回去熬一锅米汤吧。”队长连声说“够了,够了。”队长让家珍把米放在他口袋里,然后双手攥住口袋嘿嘿笑着走了。队长一走,家珍眼泪马上就下来了,她是心疼那把米。看着家珍哭,我只能连连叹气。

这样的日子一直熬到收割稻子以后,虽说是欠收,可总算又有粮食了,日子一下子好过多了。谁知家珍的病越来越重了,到后来走路都走不了几步,都是那灾年把她给糟踏成这样的。家珍不甘心,干不了田里活,她还想干家里的活。她扶着墙到这里擦擦,又到那里扫扫,有一天她摔倒后不知怎么爬不起来了,等我和凤霞收工回到家里,她还躺在地上,脸都擦破了。我把她抱到床上,凤霞拿了块毛巾给她擦掉脸上的血,我说:“你以后就躺在床上。”家珍低着头轻声说道:“我不知道会爬不起来。”家珍算是硬的,到了那种时候也不叫一声苦。她坐在床上那些日子,让我把所有的破烂衣服全放到她床边,她说:“有活干心里踏实。”她拆拆缝缝给凤霞和有庆都做了件衣服,两个孩子穿上后看起来还很新。后来我才知道她把自己的衣服也拆了,看到我生气,她笑了笑说:“衣服不穿坏起来快。我是不会穿它们了,可不能跟着我糟蹋了。”家珍说也给我做一件,谁知我的衣服没做完,家珍连针都拿不起了。那时候凤霞和有庆睡着了,家珍还在油灯下给我缝衣服,她累得脸上都是汗,我几次催她快睡,她都喘着气摇头,说是快了。结果针掉了下去,她的手哆嗦着去拿针,拿了几次都没拿起来,我捡起来递给她,她才捏住又掉了下去。家珍眼泪流了出来,这是她病了以后第一次哭,她觉得自己再也干不了活了,她说:“我是个废人了,还有什么指望?”我用袖管给她擦眼泪,她瘦得脸上的骨头都突了出来。我说她是累的,照她这样,就是没病的人也会吃不消。我宽慰她,说凤霞已经长大了,挣的工分比她过去还多,用不着再为钱操心了。家珍说:“有庆还小啊。”那天晚上,家珍的眼泪流个不停,她几次嘱咐我:“我死后不要用麻袋包我,麻袋上都是死结,我到了阴间解不开,拿一块干净的布就行了,埋掉前替我洗洗身子。

她又说:“凤霞大了,要是能给她找到婆家我死也闭眼了。

有庆还小,有些事他不懂,你不要常去揍他,吓唬吓唬就行了。“她是在交待后事,我听了心里酸一阵苦一阵,我对她说:”按理说我是早就该死了,打仗时死了那么多人,偏偏我没死,就是天天在心里念叨着要活着回来见你们,你就舍得扔下我们?“我的话对家珍还是有用的,第二天早晨我醒来时,看到家珍正在看我,她轻声说:”福贵,我不想死,我想每天都能看到你们。“家珍在床上躺了几天,什么都不干,慢慢地又有点力气了,她能撑着坐起来,她觉得自己好多了,心里高兴,想试着下地,我不让,我说:”往后不能再累着了,你得留着点力气,日子还长着呢。“

四那一年,有庆念到五年级了。俗话说是祸不单行,家珍病成那样,我就指望有庆快些长大,这孩子成绩不好,我心想别逼他去念中学了,等他小学一毕业,就让他跟着我下地挣工分去。谁知道家珍身体刚刚好些,有庆就出事了。

那天下午,有庆他们学校的校长,那是县长的女人,在医院里生孩子时出了很多血,一只脚都跨到阴间去了。学校的老师马上把五年级的学生集合到操场上,让他们去医院献血,那些孩子一听是给校长献血,一个个高兴得像是要过节了,一些男孩子当场卷起了袖管。他们一走出校门,我的有庆就脱下鞋子,拿在手里就往医院跑,有四、五个男孩也跟着他跑去。我儿子第一个跑到医院,等别的学生全走到后,有庆排在第一位,他还得意地对老师说:“我是第一个到的。”结果老师一把把他拖出来,把我儿子训斥了一通,说他不遵守纪律。有庆只得站在一旁,看着别的孩子挨个去验血,验血验了十多个没一个血对上校长的血。有庆看着看着有些急了,他怕自己会被轮到最后一个,到那时可能就献不了血了。他走到老师跟前,怯生生地说:“老师,我知道错了。”老师嗯了一下,没再理他,他又等了两个进去验血,这时产房里出来一个戴口罩的医生,对着验血的男人喊:“血呢?血呢?”验血的男人说:“血型都不对。”医生喊:“快送进来,病人心跳都快没啦。”有庆再次走到老师跟前,问老师:“是不是轮到我了?”老师看了看有庆,挥挥手说:“进去吧。”验到有庆血型才对上了,我儿子高兴得脸都涨红了,他跑到门口对外面的人叫道:“要抽我的血啦。”抽一点血就抽一点,医院里的人为了救县长女人的命,一抽上我儿子的血就不停了。抽着抽着有庆的脸就白了,他还硬挺着不说,后来连嘴唇也白了,他才哆嗦着说:“我头晕。”抽血的人对他说:“抽血都头晕。”那时候有庆已经不行了,可出来个医生说血还不够用。抽血的是个乌龟王八蛋,把我儿子的血差不多都抽干了。有庆嘴唇都青了,他还不住手,等到有庆脑袋一歪摔在地上,那人才慌了,去叫来医生,医生蹲在地上拿听筒听了听说:“心跳都没了。”医生也没怎么当会事,只是骂了一声抽血的:“你真是胡闹。”就跑进产房去救县长的女人了。

那天傍晚收工前,邻村的一个孩子,是有庆的同学,急冲冲跑过来,他一跑到我们跟前就扯着嗓子喊:“哪个是徐有庆的爹?”我一听心就乱跳,正担心着有庆会不会出事,那孩子又喊:“哪个是她娘?”我赶紧答应:“我是有庆的爹。”孩子看看我,擦着鼻子说:“对,是你,你到我们教室里来过。”我心都要跳出来了,他这才说:“徐有庆快死啦,在医院里。”我眼前立刻黑了一下,我问那孩子:“你说什么?”他说:“你快去医院,徐有庆快死啦。”我扔下锄头就往城里跑,心里乱成一团。想想中午上学时有庆还好好的,现在说他快要死了。我脑袋里嗡嗡乱叫着跑到城里医院,见到第一个医生我就拦住他,问他:“我儿子呢?”医生看看我,笑着说:“我怎么知道你儿子?”我听后一怔,心想是不是弄错了,要是弄错可就太好了。

我说:“他们说我儿子快死了,要我到医院。”准备走开的医生站住脚看着我问:“你儿子叫什么名字?”我说:“叫有庆。”他伸手指指走道尽头的房间说:“你到那里去问问。”我跑到那间屋子,一个医生坐在里面正写些什么,我心里咚咚跳着走过去问:“医生,我儿子还活着吗?”医生抬起头来看了我很久,才问:“你是说徐有庆?”我急忙点点头,医生又问:“你有几个儿子?”我的腿马上就软了,站在那里哆嗦起来,我说:“我只有一个儿子,求你行行好,救活他吧。”医生点点头,表示知道了,可他又说:“你为什么只生一个儿子?”这叫我怎么回答呢?我急了,问他:“我儿子还活着吗?”他摇摇头说:“死了。”我一下子就看不见医生了,脑袋里黑乎乎一片,只有眼泪哗哗地掉出来,半晌我才问医生:“我儿子在哪里?”有庆一个人躺在一间小屋子里,那张床是用砖头搭成的。

我进去时天还没黑,看到有庆的小身体躺在上面,又瘦又小,身上穿的是家珍最后给他做的衣服。我儿子闭着眼睛,嘴巴也闭得很紧。我有庆有庆叫了好几声,有庆一动不动,我就知道他真死了,一把抱住了儿子,有庆的身体都硬了。中午上学时他还活生生的,到了晚上他就硬了。我怎么想都想不通,这怎么也应该是两个人,我看看有庆,摸摸他的瘦肩膀,又真是我的儿子。我哭了又哭,都不知道有庆的体育教师也来了。他看到有庆也哭了,一遍遍对我说:“想不到,想不到。”体育老师在我边上坐下,我们两个人对


着哭,我摸摸有庆的脸,他也摸摸。过了很久,我突然想起来,自己还不知道儿子是怎么死的。我问体育老师,这才知道有庆是抽血被抽死的。当时我想杀人了,我把儿子一放就冲了出去。冲到病房看到一个医生就抓就住他,也不管他是谁,对准他的脸就是一拳,医生摔到地上乱叫起来,我朝他吼道:“你杀了我儿子。”吼完抬脚去踢他,有人抱住了我,回头一看是体育老师,我就说:“你放开我。”体育老师说:“你不要乱来。”我说:“我要杀了他。”体育老师抱住我,我脱不开身,就哭着求他:“我知道你对有庆好,你就放开我吧。”体育老师还是死死抱住我,我只好用胳膊肘拚命撞他,他也不松开。让那个医生爬起来跑走了,很多的人围了上来,我看到里面有两个医生,我对体育老师说:“求你放开我。”体育老师力气大,抱住我我就动不了,我用胳膊肘撞他,他也不怕疼,一遍遍地说:“你不要乱来。”这时有个穿中山服的男人走了过来,他让体育老师放开我,问我:“你是徐有庆同学的父亲?”我没理他,体育老师一放开我,我就朝一个医生扑过去,那医生转身就逃。我听到有人叫穿中山服的男人县长,我一想原来他就是县长,就是他女人夺了我儿子的命,我抬腿就朝县长肚子上蹬了一脚,县长哼了一声坐到了地上。体育老师又抱住了我,对我喊:“那是刘县长。”我说:“我要杀的就是县长。”抬起腿再去蹬,县长突然问我:“你是不是福贵?”我说:“我今天非宰了你。”县长站起来,对我叫道:“福贵,我是春生。”他这么一叫,我就傻了。我朝他看了半晌,越看越像,就说:“你真是春生。”春生走上前来也把我看了又看,他说:“你是福贵。”看到春生我怒气消了很多,我哭着对他说:“春生你长高长胖了。”春生眼睛也红了,说道:“福贵,我还以为你死了。”我摇摇头说:“没死。”春生又说:“我还以为你和老全一样死了。”一说到老全,我们两个都呜呜地哭上了。哭了一阵我问春生:“你找到大饼了吗?”春生擦擦眼睛说:“没有,你还记得?我走过去就被俘虏了。”我问他:“你吃到馒头了吗?”他说:“吃到的。”我说:“我也吃到了。”说着我们两个人都笑了,笑着笑着我想起了死去的儿子,我抹着眼睛又哭了,春生的手放到我肩上,我说:“春生,我儿子死了,我只有一个儿子。”春生叹口气说:“怎么会是你的儿子?”我想到有庆还一个人躺在那间小屋里,心里疼得受不了,我对春生说:“我要去看儿子了。”我也不想再杀什么人了,谁料到春生会突然冒出来,我走了几步回过头去对春生说:“春生,你欠了我一条命,你下辈子再还给我吧。”那天晚上我抱着有庆往家走,走走停停,停停走走,抱累了就把儿子放到背脊上,一放到背脊上心里就发慌,又把他重新抱到了前面,我不能不看着儿子。眼看着走到了村口,我就越走越难,想想怎么去对家珍说呢?有庆一死,家珍也活不长,家珍已经病成这样了。我在村口的田埂上坐下来,把有庆放在腿上,一看儿子我就忍不住哭,哭了一阵又想家珍怎么办?想来想去还是先瞒着家珍好。我把有庆放在田埂上,回到家里偷偷拿了把锄头,再抱起有庆走到我娘和我爹的坟前,挖了一个坑。

要埋有庆了,我又舍不得。我坐在爹娘的坟前,把儿子抱着不肯松手,我让他的脸贴在我脖子上,有庆的脸像是冻坏了,冷冰冰地压在我脖子上。夜里的风把头顶的树叶吹得哗啦哗啦响,有庆的身体也被露水打湿了。我一遍遍想着他中午上学时跑去的情形,书包在他背后一甩一甩的。想到有庆再不会说话,再不会拿着鞋子跑去,我心里是一阵阵酸疼,疼得我都哭不出来。我那么坐着,眼看着天要亮了,不埋不行了,我就脱下衣服,把袖管撕下来蒙住他的眼睛,用衣服把他包上,放到了坑里。我对爹娘的坟说:“有庆要来了,你们待他好一点,他活着时我对他不好,你们就替我多疼疼他。”有庆躺在坑里,越看越小,不像是活了十三年,倒像是家珍才把他生出来,我用手把土盖上去,把小石子都捡出来,我怕石子硌得他身体疼。埋掉了有庆,天蒙蒙亮了,我慢慢往家里走,走几步就要回头看看,走到家门口一想到再也看不到儿子,忍不住哭出了声音,又怕家珍听到,就捂住嘴巴蹲下来,蹲了很久,都听到出工的吆喝声了,才站起来走进屋去。凤霞站在门旁睁圆了眼睛看我,她还不知道弟弟死了。

邻村的那个孩子来报信时,她也在,可她听不到。家珍在床上叫了我一声,我走过去对她说:“有庆出事了,在医院里躺着。”家珍像是信了我的话,她问我:“出了什么事?”我说:“我也说不清楚,有庆上课时突然昏倒了,被送到医院,医生说这种病治起来要有些日子。”家珍的脸伤心起来,泪水从眼角淌出,她说:“是累的,是我拖累有庆的。”我说:“不是,累也不会累成这样。”家珍看了看我又说:“你眼睛都肿了。”我点点头:“是啊,一夜没睡。”说完我赶紧走出门去,有庆才被埋到土里,尸骨未寒啊,再和家珍说下去我就稳不住自己了。

接下去的日子,白天我在田里干活,到了晚上我对家珍说进城去看看有庆好些了没有。我慢慢往城里走,走到天黑了,再走回来,到有庆坟前坐下。夜里黑乎乎的,风吹在我脸上,我和死去的儿子说说话,声音飘来飘去都不像是我的。

坐到半夜我才回到家中,起先的几天,家珍都是睁着眼睛等我回来,问我有庆好些了吗?我就随便编些话去骗她。过了几天我回去时,家珍已经睡着了,她闭着眼睛躺在那里。我也知道老这么骗下去不是办法,可我只能这样,骗一天是一天,只要家珍觉得有庆还活着就好。

有天晚上我离开有庆的坟,回到家里在家珍身旁躺下后,睡着的家珍突然说:“福贵,我的日子不长了。”我心里一沉,去摸她的脸,脸上都是泪,家珍又说:“你要照看好凤霞,我最不放心的就是她。”家珍都没提有庆,我当时心里马上乱了,想说些宽慰她的话也说不出来。

第二天傍晚,我还和往常一样对家珍说进城去看有庆,家珍让我别去了,她要我背着她去村里走走。我让凤霞把她娘抱起来,抱到我背脊上。家珍的身体越来越轻了,瘦得身上全是骨头。一出家门,家珍就说:“我想到村西去看看。”那地方埋着有庆,我嘴里说好,腿脚怎么也不肯往村那地方去,走着走着走到了东边村口,家珍这时轻声说:“福贵,你别骗我了,我知道有庆死了。”她这么一说,我站在那里动不了,腿也开始发软。我的脖子上越来越湿,我知道那是家珍的眼泪,家珍说:“让我去看看有庆吧。”我知道骗不下去,就背着家珍往村西走,家珍低声告诉我:“我夜夜听着你从村西走过来,我就知道有庆死了。”走到了有庆坟前,家珍要我把她放下去,她扑在了有庆坟上,眼泪哗哗地流,两只手在坟上像是要摸有庆,可她一点力气都没有,只有几根指头稍稍动着。我看着家珍这付样子,心里难受得要被堵住了,我真不该把有庆偷偷埋掉,让家珍最后一眼都没见着。

家珍一直扑到天黑,我怕夜露伤着她,硬把她背到身后,家珍让我再背她到村口去看看,到了村口,我的衣领都湿透了,家珍哭着说:“有庆不会在这条路上跑来了。”我看着那条弯曲着通向城里的小路,听不到我儿子赤脚跑来的声音,月光照在路上,像是撒满了盐。

那天下午,我一直和这位老人呆在一起,当他和那头牛歇够了,下到地里耕田时,我丝毫没有离开的想法,我像个哨兵一样在那棵树下守着他。

那时候四周田地里庄稼人的说话声飘来飘去,最为热烈的是不远处的田埂上,两个身强力壮的男人都举着茶水桶在比赛喝水,旁边年轻人又喊又叫,他们的兴奋是他们处在局外人的位置上。福贵这边显得要冷清多了,在他身旁的水田里,两个扎着头巾的女人正在插秧,她们谈论着一个我完全陌生的男人,这个男人似乎是一个体格强壮有力的人,他可能是村里挣钱最多的男人,从她们的话里我知道他常在城里干搬运的活。一个女人直起了腰,用手背捶了捶,我听到她说:“他挣的钱一半用在自己女人身上,一半用在别人的女人身上。”这时候福贵扶着犁走到她们近旁,他插进去说:“做人不能忘记四条,话不要说错,床不要睡错,门槛不要踏错,口袋不要摸错。”福贵扶着犁过去后,又扭过去脑袋说:“他呀,忘记了第二条,睡错了床。”那两个女人嘻嘻一笑,我就看到福贵一脸的得意,他向牛大声吆喝了一下,看到我也在笑,对我说:“这都是做人的道理。”后来,我们又一起坐在了树荫里,我请他继续讲述自己,他有些感激地看着我,仿佛是我正在为他做些什么,他因为自己的身世受到别人重视,显示出了喜悦之情。

我原以为有庆一死,家珍也活不长了。有一阵子看上去她真是不行了,躺在床上喘气都是呼呼的,眼睛整天半闭着,也不想吃东西,每次都是我和凤霞把她扶起来,硬往她嘴里灌着粥汤。家珍身上一点肉都没有了,扶着她就跟扶着一捆柴禾似的。

队长到我家来过两次,他一看家珍的模样直摇头,把我拉到一旁轻声说:“怕是不行了。”我听了这话心直往下沉,有庆死了还不到半个月,眼看着家珍也要去了。这个家一下子没了两个人,往后的日子过起来可就难了,等于是一口锅砸掉了一半,锅不是锅,家不成家。

队长说是上公社卫生院请个医生来看看,队长说话还真算数,他去公社开会回来时,还真带了个医生回来。那个医生很瘦小,戴着一副眼镜,问我家珍得了什么病,我说:“是软骨病。”医生点点头,在床边坐下来,给家珍切脉,我看着医生边切脉边和家珍说话,家珍听到有人和她说话,只是眼睛睁了睁,也不回答。医生不知怎么搞的没找到家珍的脉搏,他像是吓了一跳,伸手去翻翻家珍的眼皮,然后一只手捧住家珍的手腕,另一只手切住家珍的脉搏,脑袋像是要去听似的歪了下去。过了一会,医生站起来对我说:“脉搏弱的都快摸不到了。”医生说:“你准备着办后事吧。”做医生的只要一句话,就能要我的命。我当时差点没栽到地上,我跟着医生走到屋外,问他:“我女人还能活多久?”医生说:“出不了一个月。得了那种病,只要全身一瘫也就快了。”那天晚上家珍和凤霞睡着以后,我一个人在屋外坐到天快亮的时候了,先是呜呜地哭,哭了一阵我就开始想从前的事,想着想着又掉出了眼泪,这日子过得真是快,家珍嫁给我以后一天好日子都没过上,眼睛一眨就到了她要去的时候了。后来我想想光哭光难受也没用,事到如今也只好想些实在的事,给家珍的后事得办的像样一点。

队长心好,他看到我这副样子就说:“福贵,你想得开些,人啊,总是要死的,眼下也别想什么了,只要让家珍死得舒坦就好。这村里的地,你随便选一块,给家珍做坟。”其实那时候我也想开了,我对队长说:“家珍想和有庆呆在一起,她俩得埋在一个地方。”有庆可怜,包了件衣服就埋了。家珍可不能再这样,家里再穷也要给她打一口棺材,要不我良心上交待不过去。家珍当初要是嫁了别人,不跟着我受罪,也不会累成这样,得这种病。我在村里挨家挨户地去借钱,我也不知道自己怎么了,一说起给家珍打口棺材,就忍不住掉眼泪。大伙都穷,借来的钱不够打棺材,后来队长给我凑了些村里的公款,才到邻村将木匠请来。

凤霞起先不知道她娘快去了,她看到我一闲下来就往先前村里的羊棚跑,木匠就在那里干活。我在那里一坐就是半晌,都忘了吃饭。凤霞来叫我,叫了几次看到棺材的形状出来了,她才觉察到了一些,睁圆了眼睛做手势问我,我心想凤霞也该知道这些,就告诉了她。

这孩子拚命地摇头,我知道她的意思,就用手势告诉她,这是给家珍准备的,是给家珍以后用的。凤霞还是摇头,拉着我就往家里走。回到了家中,凤霞还拉着我的袖管,她推推家珍,家珍眼睛睁开来。她就使劲摇我的胳膊,让我看家珍活得好好的。然后右手伸开了往下劈,她是要我把棺材劈掉。

凤霞心里根本就没想她娘会死,就是这样告诉她,她也不会相信。看着凤霞的样子,我只好低下头,什么手势都不做了。

家珍在床上一躺就是二十多天,有时觉得她好些了,有时又觉得她真的快去了。后来有一个晚上,我在她身旁躺下准备熄灯时,家珍突然抬起胳膊拉了拉我,让我别熄灯。家珍说话的声音跟蚊子一样大,她要我把她的身体侧过来。我女人那晚上把我看了又看,叫了好几声:“福贵。”然后笑了笑,闭上了眼睛。过了一会,家珍又睁开眼睛问我:“凤霞睡得好吗?”我起身看看凤霞,对她说:“凤霞睡着了。”那晚上家珍断断续续地说了好些话,到后来累了才睡着。

我却怎么都睡不着,心里七上八下的,家珍那样子像是好多了,可我老怕


着是不是人常说的回光返照。我的手在她身上摸来摸去,还热着我才稍稍放心下来。

第二天我起床时,家珍还睡着,我想她昨晚上睡得晚,就没叫醒她,和凤霞喝了点粥下地去干活。那天收工早,我和凤霞回到家里时,我吓了一跳,家珍竟然坐在床上了,她是自己坐起来的。家珍看到我们进去,轻声说:“福贵,我饿了,给我熬点粥。”当时我傻站了很久,我怎么也想不到家珍会好起来了,家珍又叫了我一声,我才回过神来,我眼泪哗哗地流了出来,我忘了凤霞听不到,对凤霞说:“全靠你,全靠你心里想着你娘不死。”人只要想吃东西,那就没事了。过了一阵子,家珍坐在床上能干些针线活了,照这样下去,家珍没准又能下床走路。

我提着的心总算可以放下了,心里一踏实,人就病倒了。其实那病早就找到我了,有庆一死,家珍跟着是一副快去的样子,我顾不上病,也就不觉得。家珍没让医生说中,身体慢慢地好起来,我脑袋是越来越晕,直到有一天插秧时昏到了地上,被人抬回家,我才知道自己是病了。

我一病倒,凤霞可就苦了,床上躺着两个人,她又服侍我们又要下地挣工分。过了几天,我看着凤霞实在是太累,就跟家珍说好多了,拖着个病身体下田去干活,村里人见了我都吃了一惊,说:“福贵,你头发全白了。”我笑笑说:“以前就白了。”他们说:“以前还有一半是黑的呢,就这么几天你的头发全白了。”就那么几天,我老了许多,我以前的力气再也没有回来,干活时腰也酸了背也疼了,干得猛一些身上到处淌虚汗。

有庆死后一个多月,春生来了。春生不叫春生了,他叫刘解放。别人见了春生都叫他刘县长,我还是叫他春生。春生告诉我,他被俘虏后就当上了解放军,一直打到福建,后来又到朝鲜去打仗。春生命大,打来打去都没被打死。朝鲜的仗打完了,他转业到邻近一个县,有庆死的那年他才来到我们县。

春生来的时候,我们都在家里。队长还没走到门口就喊上了:“福贵,刘县长来看你啦。”春生和队长一进屋,我对家珍说:“是春生,春生来了。”谁知道家珍一听是春生,眼泪马上掉了出来,她冲着春生喊:“你出去。”我一下子愣住了,队长急了,对家珍说:“你怎么能这样对刘县长说话。”家珍可不管那么多,她哭着喊道:“你把有庆还给我。”春生摇了摇头,对家珍说:“我的一点心意。”春生把钱递给家珍,家珍看都不看,冲着他喊:“你走,你出去。”队长跑到家珍跟前,挡住春生,说:“家珍,你真糊涂,有庆是事故死的,又不是刘县长害的。”春生看家珍不肯收钱,就递给我:“福贵,你拿着吧,求你了。”看着家珍那样子,我哪敢收钱。春生就把钱塞到我手里,家珍的怒火立刻冲着我来了,她喊道:“你儿子就值两百块?”我赶紧把钱塞回到春生手里。春生那次被家珍赶走后,又来了两次,家珍死活不让他进门。女人都是一个心眼,她认准的事谁也不能让她变。我送春生到村口,对他说:“春生,你以后别来了。”春生点点头,走了。春生那次一走,就几年没再来,一直到文化大革命的时候,他才又来了一次。

城里闹上了文化大革命,乱糟糟的满街都是人,每天都在打架,还有人被打死,村里人都不敢进城去了。村里比起城里来,太平多了,还跟先前一样,就是晚上睡觉睡不踏实,毛主席的最新最高指示总是在深更半夜里来,队长就站在晒场上拚命吹哨子,大伙听到哨子便赶紧爬起来,到晒场去听广播,队长在那里喊:“都到晒场来,毛主席他老人家要训话啦。”我们是平民百姓,国家的事不是不关心,是弄不明白,我们都是听队长的,队长是听上面的。只要上面怎么说,我们就怎么想,怎么做。我和家珍最操心的还是凤霞,凤霞不小了,该给她找个婆家。凤霞长得和家珍年轻时差不多,要不是她小时候得了那场病,说媒的早把我家门槛踏平了。我自己是力气越来越小,家珍的病看样子要全好是不可能了,我们这辈子也算经历了不少事,人也该熟了,就跟梨那样熟透了该从树上掉下来。可我们放心不下凤霞,她和别人不一样,她老了谁会管她?

凤霞说起来又聋又哑,她也是女人,不会不知道男婚女嫁的事。村里每年都有嫁出去娶进来的,敲锣打鼓热闹一阵,到那时候凤霞握着锄头总要看得发呆,村里几个年轻人就对凤霞指指点点,笑话她。

村里王家三儿子娶亲时,都说新娘漂亮。那天新娘被迎进村里来时,穿着大红的棉袄,哧哧笑个不停。我在田里望去,新娘整个儿是个红人了,那脸蛋红扑扑特别顺眼。

田里干活的人全跑了过去,新郎从口袋里摸出飞马牌香烟,向年长的男人敬烟,几个年轻人在一旁喊:“还有我们,还有我们。”新郎嘻嘻笑着把烟藏回到口袋里,那几个年轻人冲上去抢,喊着:“女人都娶到床上了,也不给根烟抽。”新郎使劲捂住口袋,他们硬是掰开他的手指,从口袋里拿出香烟后一个人举着,别的人跟着跑上了一条田埂。

剩下的几个年轻人围着新娘,嘻嘻哈哈肯定说了些难听的话,新娘低头直笑。女人到了出嫁的时候,是什么都看着舒服,什么都听着高兴。

凤霞在田里,一看到这种场景,又看呆了,两只眼睛连眨都没眨,锄头抱在怀里,一动不动。我站在一旁看得心里难受,心想她要看就让她多看看吧。凤霞命苦,她只有这么一点看看别人出嫁的福份。谁知道凤霞看着看着竟然走了上去。走到新娘旁边,痴痴笑着和她一起走过去。这下可把那几个年轻人笑坏了,我的凤霞穿着满是补丁的衣服,和新娘走在一起,新娘穿得又整齐又鲜艳,长得也好,和我凤霞一比,凤霞寒碜得实在是可怜。凤霞脸上没有脂粉,也红扑扑和新娘一样,她一直扭头看着新娘。

村里几个年轻人又笑又叫,说:“凤霞想男人啦。”这么说说我也就听进去了,谁知没一会儿工夫难听的话就出来了,有个人对新娘说:“凤霞看中你的床了。”凤霞在旁边一走,新娘笑不出来了,她是嫌弃凤霞。这时有人对新郎说:“你小子太合算了,一娶娶一双,下面铺一个,上面盖一个。”新郎听后嘿嘿地笑,新娘受不住了,也不管自己新出嫁该害羞一些,脖子一直就对新郎喊:“你笑个屁。”我实在是看不下去,走上田埂对他们说:“做人不能这样,要欺负人也不能欺负凤霞,你们就欺负我吧。”说完我拉住凤霞就往家里走,凤霞是聪明人,一看到我的脸色,就知道刚才出了什么事,她低着头跟我往家走,走到家门口眼泪掉了下来。

后来我和家珍商量着怎么也得给凤霞找一个男人,我们都是要死在她前面的,我们死后有凤霞收作,凤霞老这样下去,死后连个收作的人都没有。可又有谁愿意娶女凤霞呢?

家珍说去求求队长,队长外面认识的人多,打听打听,没准还真有人要我们凤霞。我就去跟队长说了,队长听后说:“也是,凤霞也该出嫁了,只是好人家难找。”我说:“哪怕是缺胳膊断腿的男人,只要他想娶凤霞,我们都给。”说完这话自己先心疼上了,凤霞哪点比不上别人,就是不会说话。回到家里,跟家珍一说,家珍也心疼上了。她坐床上半晌不说话,末了叹息一声,说:“事到如今也只能这样了。”过了没多久,队长给凤霞找着了一个男人。那天我在自留地上浇粪,队长走过来说:“福贵,我给凤霞找着婆家了,是县城里的人,搬运工,挣钱很多。”我一听条件这么好,不相信,觉得队长是在和我闹着玩,我说:“队长,你别哄我了。”队长说:“没哄你,他叫万二喜,是个偏头,脑袋靠着肩膀,怎么也起不来。”他一说是偏头,我就信了,赶紧说:“你快让他来看看凤霞吧。”队长一走,我扔了粪勺就往自己茅屋跑,没进门就喊:“家珍,家珍。”家珍坐在床上以为出了什么事,看着我眼睛都睁圆了,我说:“凤霞有男人啦。”家珍这才松了口气,说:“你吓死我了。”我说:“不缺腿,胳膊也全,还是城里人呢。”说完我呜呜地哭了,家珍先是笑,看到我哭,眼泪也流了出来。高兴了一阵,家珍问:“条件这么好,会要凤霞吗?”我说:“那男的是偏头。”家珍这才有些放心。那晚上家珍让我把她过去的一些衣服拿出来,给凤霞做了件衣服,家珍说:“凤霞总得打扮打扮,人家都要来相亲了。”没出三天,万二喜来了,真是个偏头,他看我时把左边肩膀翘起来,又把肩膀向凤霞和家珍翘翘,凤霞一看到他这副模样,咧着嘴笑了。

万二喜穿着中山服,干干净净的,若不是脑袋靠着肩膀,那模样还真像是城里来的干部。他拿着一瓶酒一块花布,由队长陪着进来。家珍坐在床上,头发梳得很整齐,衣服破了一点,倒很干净,我还专门在床下给家珍放了一双新布鞋。凤霞穿着水红衣服低着头坐在她娘旁边。家珍笑嘻嘻地看着她未过门的女婿,心里高兴着呢。

万二喜把酒和花布往桌上一放,就翘着肩膀在屋里转一圈,他是在看我们的屋子。我说:“队长,二喜,你们坐。”二喜嗯了一声在凳子上坐下,队长摆摆手说:“我就不坐了,二喜,这是凤霞,这是她爹和娘。”凤霞双手放在腿上,看到队长指着她,就向队长笑,队长指着家珍,她转过去向家珍笑。家珍说:“队长,你请坐。”队长说:“不啦,我还有事,你们谈吧。”队长转身要走,留也留不住,我送走了队长,回到屋中指指桌上的酒,对二喜说:“让你破费了,其实我有几十年没喝酒了。”二喜听后嗯了一声,也不说话,翘着个肩膀在屋里看来看去,看得我心里七上八下。家珍笑着对他说:“家里穷了一点。”二喜又嗯了一声,翘着肩膀去看家珍,家珍继续说:“好在家里还养着一头羊几只鸡,福贵和我商量着等凤霞出嫁时,把鸡羊卖了办嫁妆。”二喜听后还是嗯了一下,我都不知道他心里想什么。坐了一会,他站起来说要*吡耍*想这门亲事算是完了。他都没怎么看凤霞,老看我们的破烂屋子。我看看家珍,家珍苦笑一下,对二喜说:“我腿没力气,下不了地。”二喜点点头走到了屋外,我问他:“聘礼不带走了?”他嗯了一下,翘着肩膀看看屋顶的茅草,点了点头后就走了。

我回到屋里,在凳子上坐下,想想有些生气,就说:“自己脑袋都抬不起来,还挑三捡四的。”家珍叹了口气说:“这也不能怪人家。”凤霞聪明,一看到我们的样子,就知道人家没看上她,站起来走到里面的房间,换了身旧衣服,扛着把锄头下地去了。

到了晚上,队长来问我:“成了吗?”我摇摇头说:“太穷了,我家太穷了。”第二天上午,我在耕田时,有人叫我:“福贵,你看那路上,像是到你家相亲的偏头来了。”我抬起头来,看到五、六个人在那条路上摇摇摆摆地走来,还拉着一辆板车,只有走在最前面那人没有摇摆,他偏着脑袋走得飞快。远远一看我就知道是二喜来了,我是一点也想不到他会来。

二喜见了我,说道:“屋顶的茅草该换了,我拉了车石灰粉粉墙。”我往那板车一望,有石灰有两把刷墙的扫帚,上面搁着个小方桌,方桌上是一个猪头。二喜手里还提着两瓶白酒。

那时候我才知道二喜东张西望不是嫌我家穷,他连我屋前的草垛子都看到眼里去了。屋顶的茅草我早就想换了,只是等着农闲到来时好请村里人帮忙。

二喜带了五个人来,肉也买了,酒也备了,想得周到。他们来到我们茅屋门口,放下板车,二喜像是进了自己家一样,一手提着猪头,一手提着小方桌,走了进去,他把猪头往桌上一放,小方桌放在家珍腿上,二喜说:“吃饭什么的都会方便一些。”家珍当时眼睛就湿了,她是激动,她也没想到二喜会来,会带着人来给我家换茅草,还连夜给她做了个小方桌,家珍说:“二喜,你想得真周到。”二喜他们把桌子和凳子什么的都搬到了屋外,在一棵树下面铺上了稻草,然后二喜走到床前要背家珍,家珍笑着摆摆手,叫我:“福贵,你还站着干什么。”我赶紧过去让家珍上我背脊,我笑着对二喜说:“我女人我来背,你往后背凤霞吧。”家珍敲了我一下,二喜听后嘿嘿直笑。我把家珍背到树下,让她靠着树坐在稻草上。看着二喜他们把草垛子分散了,扎成一小捆一小捆,二喜和另一个人爬到屋顶,下面留着四个,替我家翻屋顶的茅草。我看一眼就知道二喜带来的人都是干惯这活的,手脚都麻利。下面的用竹竿挑着往上扔,二喜和另一个人在上面铺。别看二喜脑袋靠着肩膀,干活一点都不碍事,茅草扔上去他先用脚踢一下,再伸手接住。有这本领的人,在我们村里是一个都找不出来。

没到中午,屋顶的活就干完了。我给他们烧了一桶茶水,凤霞给他们倒茶水,跑前跑后忙个不停,她也高兴,看到家里突然来了这么多干活的人,凤霞笑开的嘴就没合上。

村里很多人都走过来看,一个女的对


家珍说:“女婿没过门就干活啦,你好福气啊。”家珍说:“是凤霞好福气。”二喜从屋顶上下来,我对他说:“二喜,歇一会。”二喜用袖管擦擦脸上的汗说:“不累。”说完又翘起肩膀往四处看,看到左边一块菜地问我:“这是我家的地吗?”我说:“是啊。”他就进屋拿了把菜刀,下到地里割了几棵新鲜的菜,又拿进屋去。不一会,他在里面切猪头了,我去拦他,让他把这活留给凤霞,他还是用袖管擦着汗说:“不累。”我只好出来去推凤霞,凤霞站在家珍旁边,我把她往屋里推的时候,她还不好意思地扭着头看家珍,家珍笑着挥手让她进去,她这才进了茅屋。

我和家珍陪着二喜带来的人喝茶说话,中间我走进去一次,看到二喜和凤霞像是两口子,一个烧火,一个做饭炒菜。

两个你看看我,我看看你,看过后都咧着嘴笑了。

我出来和家珍一说,家珍也笑了。过了一会,我忍不住又想去看看,刚站起来家珍就叫住我,偷偷说:“你别进去了。”吃过午饭,二喜他们用石灰粉起了墙,我家的土墙到了第二天石灰一干,变成白晃晃一片,像是城里的砖瓦房子。粉完了墙天还早着,我对二喜说:“吃了晚饭再走吧。”他说:“不吃了。”就着肩膀向凤霞翘了翘,我知道他是在看凤霞。他低声问我和家珍:“爹,娘,我什么时候把凤霞娶过去?”一听这话,一听他叫我和家珍爹娘,我们欢喜得合不上嘴,我看看家珍后说:“你想什么时候就什么时候。”接着我又轻声说:“二喜,不是我想让你破费,实在是凤霞命苦,你娶凤霞那天多叫些人来,热闹热闹,也好叫村里人看看。”二喜说:“爹,知道了。”那天晚上凤霞摸着二喜送来的花布,看看笑笑,笑笑看看。有时抬头看到我和家珍在笑,心里一慌,脸就红了。看得出来凤霞喜欢二喜,我和家珍高兴,家珍说:“二喜是个实在人,心眼好,把凤霞给他,我心里踏实。”我们把家里的鸡羊卖了,我又领着凤霞去城里给她做了两身新衣服,给她添置了一床新被子,买了脸盆什么的。凡是村里别人家女儿有的、凤霞都有,拿家珍的话说是:“不能委屈凤霞了。”二喜来娶凤霞那天,锣鼓很远就闹过来了,村里人全挤到村口去看。二喜带来了二十多个人,全穿着中山服,要不是二喜胸口戴了朵大红花,那样子像是什么大干部下来了呢。

十几双锣同时敲着,两个大鼓擂得咚咚响,把村里人耳朵震得嗡嗡乱响,最显眼的是中间有一辆披红戴绿的板车,车上一把椅子也红红绿绿。一走进村里,二喜就拆了两条大前门香烟,见到男子就往他们手里塞,嘴里连连说:“多谢,多谢。”村里别人家娶亲嫁女时,抽的最好的香烟也不过是飞马牌,二喜将大前门一盒一盒送人,那气派把谁家都比下去了。

拿到香烟的赶紧都往自己口袋里放,像是怕人来抢似的,手指在口袋里摸索着抽出一根放在嘴上。

跟在二喜身后那二十来人也卖力,锣鼓敲得震天响,还扯着嗓子喊,他们的口袋都鼓鼓的,见到村里年轻的女人和孩子,就把口袋里的糖果往他们身上扔。这样大手大脚把我都看呆了,心想扔掉的都是钱呵。

他们来到我家茅屋前,一个个进去看凤霞,锣鼓留在外面,村里的年轻人就帮着敲上了。凤霞那天穿上新衣服可真漂亮,连我这个做爹的都想不到她会这么漂亮,她坐在家珍床前,在进来的人里挨个找二喜,一看到二喜赶紧低下了头。

二喜带来的城里人见了凤霞都说:“这偏头真有艳福。”后来过了好多年,村里别的姑娘出嫁时,他们还都会说凤霞出嫁时最气派。那天凤霞被迎出屋去时,脸蛋红得跟番茄一样,从来没有那么多人一起看着她,她把头埋在胸前都不知道该怎么办,二喜拉着她的手走到板车旁,凤霞看看车上的椅子还是不知道该干什么。个头比凤霞矮的二喜一把将凤霞抱到了车上,看的人哄地笑起来,凤霞也哧哧笑了。二喜对我和家珍说:“爹,娘,我把凤霞娶走啦。”说着二喜自己拉起板车就走,板车一动,低头笑着的凤霞急忙扭过头来,焦急地看来看去。我知道她是在看我和家珍,我背着家珍其实就站在她旁边。她一看到我们,眼泪哗哗流了出来,她扭着身体哭着看我们。我一下子想起凤霞十三岁那年,被人领走时也是这么哭着看我,我一伤心眼泪也出来了,这时我脖子也湿了,我知道家珍也在哭。我想想这次不一样,这次凤霞是出嫁,我就笑了,对家珍说:“家珍,今天是办喜事,你该笑。”二喜是实心眼,他拉着板车走时,还老回过头去看看他的新娘,一看到凤霞扭着身体朝我们哭,他就不走了,站在那里也把身体扭着。凤霞是越哭越伤心,肩膀也一抖一抖了,让我这个做爹的心里一抽一抽,我对二喜喊:“二喜,凤霞是你的女人了,你还不快拉走。”凤霞嫁到了城里,我和家珍就跟丢了魂似的,怎么都觉得心慌。往常凤霞在屋里进进出出也不怎么觉得,如今凤霞一走,屋里就剩我和家珍,两个人看来看去,都看了几十年了,像是还没看够。我还好,在地里干活能分掉点想凤霞的心思。家珍就苦了,整天坐在床上,整天闲着,没有了凤霞,做娘的心里能不慌张?先前她在床上呆着从不说什么,这么一来她可就难受了,腰也酸了背也疼了,怎么都不舒服。我也知道那滋味,整天在床上,比下地干活还累,身体都活动不了。我就在黄昏的时候背着她到村里去走走,村里人见了家珍,都亲热地问长问短,家珍心里也舒畅多了,她贴着我耳朵问:“他们不会笑话我们吧。”我说:“我背着自己的女人有什么好笑话的。”家珍开始喜欢提一些过去的事,到了一处,她就要说起凤霞,说起有庆从前的事,说着说着就笑。来到了村口,家珍说起那天我回来的事,家珍在田里干活,听到有个人大声叫凤霞,叫有庆,抬头一看看到了我,起先还不敢认。家珍说到这里笑着哭了,泪水滴在我脖子上,她说:“你回来就什么都好了。”按规矩凤霞得一个月以后回来,我们也得一个月以后才能去看她。谁知凤霞嫁出去还不到十天,就回来了。那天傍晚我们刚吃过饭,有人在外面喊:“福贵,你到村口去看看,像是你家的偏头女婿来了。”我还不相信,村里人都知道我和家珍想凤霞都快想呆了,我觉得村里人是在捉弄我们,我跟家珍说:“不会吧,才十来天工夫。”家珍急了,她说:“你快去看看。”我跑到村口一看,还真是二喜,翘着左边的肩膀,手里提着一包糕点,凤霞走在他旁边,两个人手拉着手,笑眯眯地走来。村里人见了都笑,那年月可是见不到男女手拉着手的,我对他们说:“二喜是城里人,城里人就是洋气。”凤霞和二喜一来,家珍高兴坏了;凤霞在床沿上一坐,家珍拉住她的手摸个没完,一遍遍说凤霞长胖了,其实十来天工夫能长多少肉?我对二喜说:“没想到你们会来,一点准备都没有。”二喜嘿嘿地笑,他说他也不知道会来,是凤霞拉着他,他糊里糊涂地跟来了。

凤霞嫁出去没过十天就回来,我们也不管什么老规矩了,我是三天两头往城里跑,说起来是家珍要我去的,我自己也想着要常去看看他们。我往城里跑得这么勤快,跟年轻时一样了,只是去的地方不一样。

去的时候,我就在自留地里割上几棵青菜,放在篮子里提着,穿上家珍给我做的新布鞋。我割菜时鞋上沾了点泥,家珍就叫住我,要我把泥擦掉。我说:“人都老了,还在乎什么鞋上有泥。”家珍说:“话可不能这么说,人老了也是人,是人就得干净一些。”这倒也是,家珍病了那么多年,在床上下不了地,头发每天都还是梳得整整齐齐的。我穿得干干净净走出村口,村里人见我提着青菜,就问:“又去看凤霞?”我点点头:“是啊。”他们说:“你老这么去,那偏头女婿不赶你走?”我说:“二喜才不会呢。”二喜家的邻居都喜欢凤霞,我一去,他们就夸她,说她又勤快又聪明。扫地时连别人家的屋前也扫,一扫就扫半条街,邻居看到凤霞汗都出来了,走过去拍拍她,让她别扫了,她这才笑眯眯地回到自己屋里。

凤霞以前没学过织毛衣,我们家穷,谁也没穿过毛衣。凤霞看到邻居的女人坐在门前织毛衣,手穿来插去的,心里喜欢她就搬着把凳子坐到跟前看,一看就看半天,人都看呆了。

邻居家的女人看着凤霞这么喜欢,便手把手教她。这么一教可把她们吓一跳,凤霞一学就会,才三、四天,凤霞织毛衣和她们一样快了。她们见了我就说:“要是凤霞不聋不哑有多好。”她们也在心里可怜凤霞。后来只要屋里的活一忙完,凤霞便坐到门前替她们织毛衣。整条街的女人里就数凤霞毛衣织得最紧最密,这下可好了,她们都把毛线送过来,让凤霞替她们织。凤霞累是累了一些,可她心里高兴。毛衣织成了给人家,她们向她翘翘大拇指,凤霞张着嘴就要笑半天。

我一进城,邻居家的女人就过来挨个告诉我,凤霞这儿好,那儿好,我听到的全是好话,听得我眼睛都红了,我说:“城里人就是好,在村里是难得听到说我凤霞好。”看到大家都这么喜欢凤霞,二喜又疼爱她,我心里高兴啊。回到家里,家珍总是埋怨我去得太久。这也是,家珍一个人在家里伸直了脖子等我回去说些凤霞的新鲜事,左等右等不见我回来,心里当然要焦急,我说:“一见了凤霞就忘了时间。”每次回到家里,我都要坐在床边说半晌,凤霞屋里屋外的事,她穿什么颜色的衣服,家珍给她做的鞋穿破了没有。家珍什么都知道,她是没完没了地问,我也没完没了地说,说得我嘴里都没有唾沫了,家珍也不放过我,问我:“还有什么忘了说了?”一说说到天黑,村里人都差不多要上床睡觉了,我们都还没吃饭,我说:“我得煮吃的了。”家珍拉住我,求我:“你再给我说说凤霞。”其实我也愿意多说说凤霞,跟家珍说我还嫌不够,到田里干活时,我又跟村里人说了,说凤霞又聪明又勤快,在城里怎么好,怎么招人喜爱,毛衣织得比谁都快。村里有些人听了还不高兴,对我说:“福贵,你是老昏了头,城里人心眼坏着呢,凤霞整天给别人家干活还不累死。”我说:“话可不能这么说。”他们说:“凤霞替她们织毛衣,她们也得送点东西给凤霞,送了吗?”村里人心眼就是小,尽想些捡便宜的事。城里的女人可不是他们说的那么坏,我有两次听到她们对二喜说:“二喜,你去买两斤毛线来,也该让凤霞有件毛衣。”二喜听后笑笑,没作声。二喜是实在人,娶凤霞时他依了我的话,钱花多了,欠下了债。到了私下里,他悄悄对我说:“爹,我还了债就给凤霞买毛线。”城里的文化大革命是越闹越凶,满街都是大字报,贴大字报的人都是些懒汉,新的贴上去时也不把旧的撕掉,越贴越厚,那墙上像是有很多口袋似的鼓了出来。连凤霞、二喜他们屋门上都贴了标语,屋里脸盆什么的也印上了毛主席他老人家的话,凤霞他们的枕巾上印着:千万不要忘记阶级斗争;床单上的字是:在大风大浪中前进。二喜和凤霞每天都睡在毛主席的话上面。

我每次进城,看到人多的地方就避开,城里是天天都在打架,我就见过几次有人被打得躺在地上起不来。难怪队长再不上城里开会了,公社常派人来通知他去县里开三级干部会议,队长都不去,私下里对我们说:“城里天天都在死人,我吓都吓死了,眼下进城去开会就是进了棺材。”队长躲在村里哪里都不去,可他也只是过了几个月的安稳日子,他不出去,别人找上门来了。那天我们都在田里干活,远远地看到一面红旗飘过来,来了一队城里的红卫兵。队长也在田里,看到他们走来,当时脖子就缩了缩,提心吊胆地问我:“该不会来找我的吧。”领头的红卫兵是个女的,他们来到了我们跟前,那女的朝我们喊:“这里为什么没有标语,没有大字报?队长呢?队长是谁?”队长赶紧扔了锄头路过去,点头哈腰地说:“红卫兵小将同志。”那个女的挥挥手臂问:“为什么没有标语和大字报?”队长说:“有标语,有两条标语呢,就刷在那间屋子后面。”那女的看上去最多只有十六七岁,她在我们队长面前神气活现,眼睛斜了斜就算是看过队长了。她对几个提着油漆筒的红卫兵说:“去刷上标语。”那几个红卫兵就朝村里的房子跑去,去刷标语了。领头的女孩对队长说:“让全村人集合。”队长急忙从口袋里掏出哨子拼命吹,在别的田里干活的人赶紧跑了过来。等人集合得差不多了,那女的对我们喊:“你们这里的地主是谁?”大伙一听这话全朝我看上了,看得我腿都哆嗦了,好在队长说:“地主解放初就毙掉了。”她又问:“有没有富农。”队长说:“富农有一个,前年归西了。”她看看队长,对我们大伙喊:“那走资派有没有?”队长陪着笑脸说:“这村里是小地方,哪有走资派?”她的手突然一伸,都快指到队长的鼻子上了,她问:“你是什么?”队长吓得

连声说:“我是队长,是队长。”谁知道她大喊一声:“你就是走资本主义道路的当权派。”队长吓坏了,连连摆手说:“不是,不是,我没走。”那女的没理他,朝我们喊:“他对你们进行白色统治,他欺压你们,你们要起来反抗,要砸断他的狗腿。”村里人都看傻了,平日里队长可神气了,他说什么我们听什么,从没人觉得队长说得不对。如今队长被这群城里来的孩子折腾的腰都弯下去了,他连连求饶,我们都说不出口的话他也说了。队长求了一会,转身对我们喊:“你们出来说说呀,我没欺压你们。”大伙看看队长,又看看那些红卫兵,三三两两地说:“队长没有欺压我们,他是个好人。”那个女的皱着眉看我们,说:“不可救药。”说完她朝几个红卫兵挥挥手:“把他押走。”两个红卫兵走过去抓住队长的胳膊,队长伸直了脖子喊:“我不进城,乡亲们哪,救救我,我不能进城,进城就是进棺材。”队长再喊也没用,被他们把胳膊扭到后面,弯着身体押走了。大伙看着他们喊着口号杀气腾腾地走去,谁也没上去阻拦,没人有这个胆量。

队长这么一去,大伙都觉得凶多吉少,城里那地方乱着呢,就算队长保住命,也得缺条胳膊少条腿的。谁知没出三天,队长就回来了,一副鼻青眼肿的模样,在那条路上晃晃悠悠地走来,在地里的人赶紧迎上去,叫他:“队长。”队长眼皮抬了抬,看看大伙,什么话没说,一直走回自己家,呼呼地睡了两天。到了第三天,队长扛着把锄头下到田里,脸上的肿消了很多,大伙围上去问这问那,问他身上还疼不疼,他摇摇头说:“疼倒没什么,不让我睡觉,他娘的比疼还难受。”说着队长掉出眼泪,说:“我算是看透了,平日里我像护着儿子一样护着你们,轮到我倒楣了,谁也不来救我。”队长说得我们大伙都不敢去看他。队长总还算好,被拉到城里只是吃了三天的拳脚。春生住在城里,可就更惨了。我还一直不知道春生也倒楣了,那天我进城去看凤霞,在街上看到一伙戴着各种纸帽子,胸前挂着牌牌的人被押着游街。起先我没怎么在意,等他们来到跟前,我吓了一跳,走在最前头的竟是春生。春生低着头,没看到我,从我身边走过去后,春生突然抬起头来喊:“毛主席万岁。”几个戴红袖章的人冲上去对春生又打又踢,骂道:“这是你喊的吗,他娘的走资派。”春生被他们打倒在地,身体搁在那块木牌上,一只脚踢在他脑袋上,春生的脑袋像是被踢出个洞似的咚地一声响,整个人趴在了地上。春生被打得一点声音都没有,我这辈子没见过这么打人的,在地上的春生像是一块死肉,任他们用脚去踢。再打下去还不把春生打死了,我上去拉住两个人的袖管,说:“求你们别打了。”他们用劲推了我一把,我差点摔到地上,他们说:“你是什么人?”我说:“求你们别打了。”有个人指着春生说:“你知道他是什么人,他是旧县长,是走资派。”我说:“这我都不知道,我只知道他是春生。”他们一说话,也就没再去打春生,喊着要春生爬起来。春生被打成那样了,怎么爬得起来,我就去扶他,春生认出了我,说:“福贵,你快走开。”那天我回到家里,坐在床边,把春生的事跟家珍说了,家珍听了都低下头,我就说:“当初你不该不让春生进屋。”家珍虽然嘴上没说什么,其实她心里想的也和我一样。“过了一个多月,春生偷偷地上我家来了,他来时都深更半夜,我和家珍已经睡了,敲门把我们敲醒,我打开门借着月光一看是春生,春生的脸肿的都圆了,我说:”春生,快进来。“春生站在门外不肯进来,他问:”嫂子还好吧?“我就对家珍说:”家珍,是春生。“家珍坐在床上没有答应,我让春生进屋,家珍不开口,春生就不进来,他说:”福贵,你出来一下。“我回头又对家珍说:”家珍,是春生来了。“家珍还是没理我,我只好披上衣服走出去,春生走到我家屋前那棵树下,对我说:”福贵,我是来和你告别的。“我问:”你要去哪里?“他咬着牙齿狠狠地说:”我不想活了。“我吃了一惊,急忙拉住春生的胳膊说:”春生,你别糊涂,你还有女人和儿子呢。“一听这话,春生哭了,他说:”福贵,我每天都被他们吊起来打。“说着他把手伸过来:”你摸摸我的手。“我一摸,那手像是煮熟了一样,烫得吓人,我问他:”疼不疼?“他摇摇头:”不觉得了。“我把他肩膀往下按,说道:”春生,你先坐下。“我对他说,”你千万别糊涂,死人都还想活过来,你一个大活人可不能去死。“我又说:”你的命是爹娘给的,你不要命了也得先去问问他们。“春生抹了抹眼泪说:”我爹娘早死了。“我说:”那你更该好好活着,你想想,你走南闯北打了那么多仗,你活下来容易吗?“那天我和春生说了很多话,家珍坐在屋里床上全听进去了。到了天快亮的时候,春生像是有些想通了,他站起来说要走了,这时家珍在里面喊:”春生。“我们两个都怔了一下,家珍又叫了一声,春生才答应。我们走到门口,家珍在床上说:”春生,你要活着。“春生点了点头,家珍在里面哭了,她说:”你还欠我们一条命,你就拿自己的命来还吧。“春生站了一会说:”我知道了。“我把春生送到村口,春生让我站住,别送了,我就站在村口,看着春生走去,春生都被打瘸了,他低着头走得很吃力。我又放心不下,对他喊:”春生,你要答应我活着。“春生走了几步回过头来说:”我答应你。“春生后来还是没有答应我,一个多月后,我听说城里的刘县长上吊死了。一个人命再大,要是自己想死,那就怎么也活不了。我把这话对家珍说了,家珍听后难受了一天,到了夜里她说:”其实有庆的死不能怪春生。“到了田里的活一忙,我就不能常常进城去看凤霞了。好在那时是人民公社,村里人在一起干活,我用不着焦急。只是家珍还是下不了床,我起早摸黑,既不能误了田里的活,又不能让家珍饿着,人实在是累。年纪大了,要是年轻他二十岁,睡上一觉就会没事,到了那个年纪,人累了睡上几觉也补不回来,干活时手臂都抬不起来,我混在村里人中间,每天只是装装样子,他们也都知道我的难处,谁也不来说我。

五农忙时凤霞来住了几天,替我做饭烧水,侍候家珍,我轻松了很多。可是想想嫁出去的女儿就是泼出去的水,凤霞早就是二喜的人了,不能在家里呆得太久。我和家珍商量了一下,怎么也得让凤霞回去了,就把凤霞赶走了。我是用手一推一推把她推出村口的,村里人见了嘻嘻笑,说没见过像我这样的爹。我听了也嘻嘻笑,心想村里谁家的女儿也没像凤霞对她爹娘这么好,我说:“凤霞只有一个人,服侍了我和家珍,就服侍不了我的偏头女婿了。”凤霞被我赶回城里,过了没多久又回来了,这次连偏头女婿也来了。两个人在远处拉着手走来,我很远就看到了他们,不用看二喜的偏脑袋,就看拉着手我也知道是谁了。二喜提着一瓶黄酒,咧着嘴笑个不停。凤霞手里挎着个小竹篮子,也像二喜一样笑。我想是什么好事,这么高兴。

到了家里,二喜把门关上,说:“爹,娘,凤霞有啦。”凤霞有孩子了,我和家珍嘴一咧也都笑了。我们四个人笑了半晌,二喜才想起来手里的黄酒,走到床边将酒放在小方桌上,凤霞从篮里拿出碗豆子。我说:“都到床上去,都到床上去。”凤霞坐到家珍身旁,我拿了四只碗和二喜坐一头。二喜给我倒满了酒,给家珍也倒满,又去给凤霞倒,凤霞捏住酒瓶连连摇头,二喜说:“今天你也喝。”凤霞像是听懂了二喜的话,不再摇头。我们端起了碗,凤霞喝了一口皱皱眉,去看家珍,家珍也在皱眉,她抿着嘴笑了。我和二喜都是一口把酒喝干,一碗酒下肚,二喜的眼泪掉了出来,他说:“爹,娘,我是做梦也想不到会有今天。”一听这话,家珍眼睛马上就湿了,看着家珍的样子,我眼泪也下来了,我说:“我也想不到,先前最怕的就是我和家珍死了凤霞怎么办,你娶了凤霞,我们心就定了,有了孩子更好了,凤霞以后死了也有人收作。”凤霞看到我们哭,也眼泪汪汪的。家珍哭着说:“要是有庆活着就好了,他是凤霞带大的,他和凤霞亲着呢,有庆看不到今天了。”二喜哭得更凶了,他说:“要是我爹娘还活着就好了,我娘死的时候捏住我的手不肯放。”四个人越哭越伤心,哭了一阵,二喜又笑了,他指指那碗豆子说:“爹,娘,你们吃豆子,是凤霞做的。”我说:“我吃,我吃,家珍,你吃。”我和家珍看来看去,两个人都笑了,我们马上就会有外孙了。那天四个人哭哭笑笑,一直到天黑,二喜和凤霞才回去。

凤霞有了孩子,二喜就更疼爱她。到了夏天,屋里蚊子多,又没有蚊帐,天一黑二喜便躺到床上去喂蚊子,让凤霞在外面坐着乘凉,等把屋里的蚊子喂饱,不再咬人了,才让凤霞进去睡。有几次凤霞进去看他,他就焦急,一把将凤霞推出去。这都是二喜家的邻居告诉我的,她们对二喜说:“你去买顶蚊帐。”二喜笑笑不作声,瞅空儿才对我说:“债不还清,我心里不踏实。”看着二喜身上被蚊子咬得到处都是红点,我也心疼,我说:“你别这样。”二喜说:“我一个人,蚊子多咬几口捡不了什么便宜,凤霞可是两个人啊。”凤霞是在冬天里生孩子的,那天雪下得很大,窗户外面什么都看不清楚。凤霞进了产房一夜都没出来,我和二喜在外面越等越怕,一有医生出来,就上去问,知道还在生,便有些放心。到天快亮时,二喜说:“爹,你先去睡吧。”我摇摇头说:“心悬着睡不着。”二喜劝我:“两个人不能绑在一起,凤霞生完了孩子还得有人照应。”我想想二喜说得也对,就说:“二喜,你先去睡。”两个人推来推去,谁也没睡。到天完全亮了,凤霞还没出来,我们又怕了,比凤霞晚进去的女人都生完孩子出来了。

我和二喜哪还坐得住,凑到门口去听里面的声音,听到有女人在叫唤,我们才放心,二喜说:“苦了凤霞了。”过了一会,我觉得不对,凤霞是哑吧,不会叫唤的,这么对二喜说,二喜的脸一下子白了,他跑到产房门口拚命喊:“凤霞,凤霞。”里面出来个医生朝二喜喊道:你叫什么,出去。“二喜呜呜地哭了,他说:”我女人怎么还没出来。“旁边有人对我们说:”生孩子有快的,也有慢的。“我看看二喜,二喜看看我,想想可能是这样,就坐下来再等着,心里还是咚咚乱跳。没多久,出来一个医生问我们:”要大的?还是要小的?“她这么一问,把我们问傻了,她又说:”喂,问你们呢?“二喜扑通跪在了她跟前,哭着喊:”医生,救救凤霞,我要凤霞。“二喜在地上哇哇地哭,我把他扶起来,劝他别这样,这样伤身体,我说:”只要凤霞没事就好了,俗话说留得青山在,不怕没柴烧。“二喜呜呜地说:”我儿子没了。“我也没了外孙,我脑袋一低也呜呜地哭了。到了中午,里面有医生出来说:”生啦,是儿子。“二喜一听急了,跳起来叫道:”我没要小的。“医生说:”大的也没事。“凤霞也没事,我眼前就晕晕乎乎了,年纪一大,身体折腾不起啊。二喜高兴坏了,他坐在我旁边身体直抖,那是笑得太厉害了。我对二喜说:”现在心放下了,能睡觉了,过会再来替你。“谁料到我一走凤霞就出事了,我走了才几分钟,好几个医生跑进了产房,还拖着氧气瓶。凤霞生下了孩子后大出血,天黑前断了气。我的一双儿女都是生孩子上死的,有庆死是别人生孩子,凤霞死在自己生孩子。

那天雪下得特别大,凤霞死后躺到了那间小屋里,我去看她一见到那间屋子就走不进去了,十多年前有庆也是死在这里的。我站在雪里听着二喜在里面一遍遍叫着凤霞,心里疼得蹲在了地上。雪花飘着落下来,我看不清那屋子的门,只听到二喜在里面又哭又喊,我就叫二喜,叫了好几声,二喜才在里面答应一声,他走到门口,对我说:“我要大的,他们给了我小的。”我说:“我们回家吧,这家医院和我们前世有仇,有庆死在这里,凤霞也死在这里。二喜,我们回家吧。”二喜听了我的话,把凤霞背在身后,我们三个人往家走。

那时候天黑了,街上全是雪,人都见不到,西北风呼呼吹来,雪花打在我们脸上,像是沙子一样。二喜哭得声音都哑了,走一段他说:“爹,我走不动了。”我让他把凤霞给我,他不肯,又走了几步他蹲了下去,说:“爹,我腰疼得不行了。”那是哭的,把腰哭疼了。回到了家里,二喜把凤霞放在床上,自己坐在床沿上盯着凤霞看,二喜的身体都缩成一团了。我不用看他,就是去看他和凤霞在墙上的影子,也让我难受的看不下去。那两个影子又黑又大,一个躺着,一个像是跪着,都是一动不动,只有二喜的眼泪在动,让我看到一颗一颗大黑点在两个人影中间滑着。我就跑到灶间,去烧些水,让二喜喝了暖暖身体,等我烧开了水端过去时,灯熄了,二喜和凤霞睡了。

那晚上我在二喜他们灶间坐到天亮,外面的风呼呼地响着,有一阵子下起了雪珠子,打在门窗上沙沙乱响,二喜和凤霞睡在里屋子里一点声音也没有,

寒风从门缝冷嗖嗖地钻进来,吹得我两个膝盖又冷又疼,我心里就跟结了冰似的一阵阵发麻,我的一双儿女就这样都去了,到了那种时候想哭都没有了眼泪。我想想家珍那时还睁着眼睛等我回去报信,我出来时她一遍一遍嘱咐我,等凤霞一生下来赶紧回去告诉她是男还是女。凤霞一死,让我怎么回去对她说?

有庆死时,家珍差点也一起去了,如今凤霞又死到她前面,做娘的心里怎么受得住。第二天,二喜背着凤霞,跟着我回到家里。那时还下着雪,凤霞身上像是盖了棉花似的差不多全白了。一进屋,看到家珍坐在床上,头发乱糟糟的,脑袋靠在墙上,我就知道她心里明白凤霞出事了,我已经连着两天两夜没回家了。我的眼泪唰唰地流了出来,二喜本来已经不哭了,一看到家珍又呜呜地哭起来,他嘴里叫着:“娘,娘……”家珍的脑袋动了动,离开了墙壁,眼睛一动不动地看着二喜背脊上的凤霞。我帮着二喜把凤霞放到床上,家珍的脑袋就低下来去看凤霞,那双眼睛定定的,像是快从眼眶里突出来了。我是怎么也想不到家珍会是这么一付样子,她一颗泪水都没掉出来,只是看着凤霞,手在凤霞脸上和头发上摸着。二喜哭得蹲了下去,脑袋靠在床沿上。我站在一旁看着家珍,心里不知道她接下去会怎么样。那天家珍没有哭也没有喊,只是偶尔地摇了摇头。凤霞身上的雪慢慢融化了以后,整张床上都湿淋淋了。

凤霞和有庆埋在了一起。那时雪停住了,阳光从天上照下来,西北风刮得更凶了,呼呼直响,差不多盖住了树叶的响声。埋了凤霞,我和二喜抱着锄头铲子站在那里,风把我们两个人吹得都快站不住了。满地都是雪,在阳光下面白晃晃刺得眼睛疼,只有凤霞的坟上没有雪,看着这湿漉漉的泥土,我和二喜谁也抬不动脚走开。二喜指指紧挨着的一块空地说:“爹,我死了埋在这里。”我叹了口气对二喜说:“这块就留给我吧,我怎么也会死在你前面的。”埋掉了凤霞,孩子也可以从医院里抱出来了。二喜抱着他儿子走了十多里路来我家,把孩子放在床上,那孩子睁开眼睛时皱着眉,两个眼珠子瞟来瞟去,不知道他在看什么。看着孩子这副模样,我和二喜都笑了。家珍是一点都没笑,她眼睛定定地看着孩子,手指放在他脸旁,家珍当初的神态和看死去的凤霞一模一样,我当时心里七下八下的,家珍的模样吓住了我,我不知道家珍是怎么了。后来二喜抬起脸来,一看到家珍他立刻不笑了,垂着手臂站在那里不知怎么才好。过了很久,二喜才轻声对我说:“爹,你给孩子取个名字。”家珍那时开口说话了,她声音沙沙地说:“这孩子生下来没有了娘,就叫他苦根吧。”凤霞死后不到三个月,家珍也死了。家珍死前的那些日子,常对我说:“福贵,有庆,凤霞是你送的葬,我想到你会亲手埋掉我,就安心了。”她是知道自己快要死了,反倒显得很安心。那时候她已经没力气坐起来了,闭着眼睛躺在床上,耳朵还很灵,我收工回家推开门,她就会睁开眼睛,嘴巴一动一动,我知道她是在对我说话,那几天她特别爱说话,我就坐在床上,把脸凑下去听她说,那声音轻得跟心跳似的。人啊,活着时受了再多的苦,到了快死的时候也会想个法子来宽慰自己,家珍到那时也想通了,她一遍一遍地对我说:“这辈子也快过完了,你对我这么好,我也心满意足,我为你生了一双儿女,也算是报答你了,下辈子我们还要在一起过。”家珍说到下辈子还要做我的女人,我的眼泪就掉了出来,掉到了她脸上,她眼睛眨了两下微微笑了,她说:“凤霞、有庆都死在我前头,我心也定了,用不着再为他们操心,怎么说我也是做娘的女人,两个孩子活着时都孝顺我,做人能做成这样我该知足了。”她说我:“你还得好好活下去,还有苦根和二喜,二喜其实也是自己的儿子了,苦根长大了会和有庆一样对你会好,会孝顺你的。”家珍是在中午死的,我收工回家,她眼睛睁了睁,我凑过去没听到她说话,就到灶间给她熬了碗粥。等我将粥端过去在床前坐下时,闭着眼睛的家珍突然捏住了我的手,我想不到她还会有这么大的力气,心里吃了一惊,悄悄抽了抽,抽不出来,我赶紧把粥放在一把凳子上,腾出手摸摸她的额头,还暖和着,我才有些放心。家珍像是睡着一样,脸看上去安安静静的,一点都看不出难受来。谁知没一会,家珍捏住我的手凉了,我去摸她的手臂,她的手臂是一截一截的凉下去,那时候她的两条腿也凉了,她全身都凉了,只有胸口还有一块地方暖和着,我的手贴在家珍胸口上,胸口的热气像是从我手指缝里一点一点漏了出来。她捏住我的手后来一松,就瘫在了我的胳膊上。

“家珍死得很好。”福贵说。那个时候下午即将过去了,在田里干活的人开始三三两两走上田埂,太阳挂在西边的天空上,不再那么耀眼,变成了通红一轮,涂在一片红光闪闪的云层上。

福贵微笑地看着我,西落的阳光照在他脸上,显得格外精神。他说:“家珍死得很好,死得平平安安,干干净净,死后一点是非都没留下,不像村里有些女人,死了还有人说闲话。”坐在我对面的这位老人,用这样的语气谈论着十多年前死去的妻子,使我内心涌上一股难言的温情,仿佛是一片青草在风中摇曳,我看到宁静在遥远处波动。

四周的人离开后的田野,呈现了舒展的姿态,看上去是那么的广阔,天边无际,在夕阳之中如同水一样泛出片片光芒。福贵的两只手搁在自己腿上,眼睛眯缝着看我,他还没有站起来的意思,我知道他的讲述还没有结束。我心想趁他站起来之前,让他把一切都说完吧。我就问:“苦根现在有多大了。”福贵的眼睛里流出了奇妙的神色,我分不清是悲凉,还是欣慰。他的目光从我头发上飘过去,往远处看了看,然后说:“要是按年头算,苦根今年该有十七岁了。”家珍死后,我就只有二喜和苦根了。二喜花钱请人做了个背兜,苦根便整天在他爹背脊上了,二喜干活时也就更累,他干搬运活,拉满满一车货物,还得背着苦根,呼哧呼哧的气都快喘不过来了。身上还背着个包裹,里面塞着苦根的尿布,有时天气阴沉,尿布没干,又没换的,只好在板车上绑三根竹竿,两根竖着,一根横着,上面晾着尿布。城里的人见了都笑他,和二喜一起干活的伙伴都知道他苦,见到有人笑话二喜,就骂道:“你他娘的再笑?再笑就让你哭。”苦根在背兜里一哭,二喜听哭声就知道是饿了,还是拉尿了,他对我说:“哭得声音长是饿了,哭得声音短是屁股那地方难受了。”也真是,苦根拉屎撒尿后哭起来嗯嗯的,起先还觉得他是在笑。这么小的人就知道哭得不一样。那是心疼他爹,一下子就告诉他爹他想干什么,二喜也用不着来回折腾了。

苦根饿了,二喜就放下板车去找正在奶孩子的女人,递上一毛钱轻声说:“求你喂他几口。”二喜不像别人家孩子的爹,是看着孩子长大。二喜觉得苦根背在身上又沉了一些,他就知道苦根又大了一些。做爹的心里自然高兴,他对我说:“苦根又沉了。”我进城去看他们,常看到二喜拉着板车,汗淋淋地走在街上,苦根在他的背兜里小脑袋吊在外面一摇一摇的。我看二喜太累,劝他把苦根给我,带到乡下去。二喜不答应,他说:“爹,我离不了苦根。”好在苦根很快大起来,苦根能走路了,二喜也轻松了一些,他装卸时让苦根在一旁玩,拉起板车就把苦根放到车上。

苦根大一些后也知道我是谁了,他常常听到二喜叫我爹,便记住了。我每次进城去看他们,坐在板车里的苦根一看到我,马上尖声叫起来,他朝二喜喊:“爹,你爹来了。”这孩子还在他爹背兜里时,就会骂人了,生气时小嘴巴噼辟啪啪,脸蛋涨得通红,谁也不知道他在说些什么,只看到唾沫从他嘴里飞出来,只有二喜知道,二喜告诉我:“他在骂人呢。”苦根会走路会说几句话后,就更精了,一看到别的孩子手里有什么好玩的,嘻嘻笑着拚命招手,说:“来,来,来。”别的孩子走到他跟前,他伸手便要去抢人家里的东西,人家不给他,他就翻脸,气冲冲地赶人家走,说:“走,走,走。”没了凤霞,二喜是再也没有回过魂来,他本来说话不多,凤霞一死,他话就更少了,人家说什么,他嗯一下算是也说了,只有见到我才多说几句。苦根成了我们的命根子,他越往大里长,便越像凤霞,越是像凤霞,也就越让我们看了心里难受。二喜有时看着看着眼泪就掉了出来,我这个做丈人的便劝他:“凤霞死了也有些日子了,能忘就忘掉她吧。”那时苦根有三岁了,这孩子坐在凳子上摇晃着两条腿,正使劲在听我们说话,眼睛睁得很圆。二喜歪着脑袋想什么,过了一会才说:“我只有这点想想凤霞的福份。”后来我要回村里去,二喜也要去干活了,我们一起走了出去。一到外面,二喜贴着墙壁走起来,歪着脑袋走得飞快,像是怕人认出他来似的,苦根被他拉着,走得跌跌冲冲,身体都斜了。我也不好说他,我知道二喜是没有了凤霞才这样的。邻居家的人见了便朝二喜喊:“你走慢点,苦根要跌倒啦。”二喜嗯了一下,还是飞快地往前走。苦根被他爹拉着,身体歪来歪去,眼睛却骨碌骨碌地转来转去。到了转弯的地方,我对二喜说:“二喜,我回去啦。”二喜这才站住,翘了翘肩膀看我,我对苦根说:“苦根,我回去了。”苦根朝我挥挥手尖声说:“你走吧。”我只要一闲下来就往城里去,我在家里呆不住,苦根和二喜在城里,我总觉得城里才像是我的家,回到村里孤伶伶一人心里不踏实。有几次我把苦根带到村里住,苦根倒没什么,高兴得满村跑,让我帮他去捉树上的麻雀,我说我怎么捉呀,这孩子手往上指了指说:“你爬上去。”我说:“我会摔死的,你不要我的命了?”他说:“我不要你的命,我要麻雀。”苦根在村里过得挺自在,只是苦了二喜,二喜是一天不见苦根就受不了,每天干完了活,累的人都没力气了,还要走十多里路来看苦根,第二天一早起床又进城去干活了。我想想这样不是个办法,往后天黑前就把苦根送回去。家珍一死,我也就没有了牵挂,到了城里,二喜说:“爹,你就住下吧。”我便在城里住上几天。我要是那么住下去,二喜心里也愿意,他常说家里有三代人总比两代人好,可我不能让二喜养着,我手脚还算利索,能挣钱,我和二喜两个人挣钱,苦根的日子过起来就阔气多了。

这样的日子过到苦根四岁那年,二喜死了。二喜是被两排水泥板夹死的。干搬运这活,一不小心就磕破碰伤,可丢了命的只有二喜,徐家的人命都苦。那天二喜他们几个人往板车上装水泥板,二喜站在一排水泥板前面,吊车吊起四块水泥板,不知出了什么差错,竟然往二喜那边去了,谁都没看到二喜在里面,只听他突然大喊一声:“苦根。”二喜的伙伴告诉我,那一声喊把他们全吓住了,想不到二喜竟有这么大的声音,像是把胸膛都喊破了。他们看到二喜时,我的偏头女婿已经死了,身体贴在那一排水泥板上,除了脚和脑袋,身上全给挤扁了,连一根完整的骨头都找不到,血肉跟浆糊似的粘在水泥板上。他们说二喜死的时候脖子突然伸直了,嘴巴张得很大,那是在喊他的儿子。

苦根就在不远处的池塘旁,往水里扔石子,他听到爹临死前的喊叫,便扭过去叫:“叫我干什么?”他等了一会,没听到爹继续喊他,便又扔起了石子。直到二喜被送到医院里,知道二喜死了,才有人去叫苦根:“苦根,苦根,你爹死啦。”苦根不知道死究竟是什么,他回头答应了一声:“知道啦。”就再没理睬人家,继续往水里扔石子。

那时候我在田里,和二喜一起干活的人跑来告诉我:“二喜快死啦,在医院里,你快去。”我一听说二喜出事了被送到医院里,马上就哭了,我对那人喊:“快把二喜抬出去,不能去医院。”那人呆呆看着我,以为我疯了,我说:“二喜一进那家医院,命就难保了。”有庆,凤霞都死在那家医院里,没想到二喜到头来也死在了那里。你想想,我这辈子三次看到那间躺死人的小屋子,里面三次躺过我的亲人。我老了,受不住这些。去领二喜时,我一见那屋子,就摔在了地上。我是和二喜一样被抬出那家医院的。

二喜死后,我便把苦根带到村里来住了。离开城里那天,我把二喜屋里的用具给了那里的邻居,自己挑了几样轻便的带回来。我拉着苦根走时,天快黑了,邻居家的人都走过来送我,送到街口,他们说:“以后多回来看看。”有几个女的还哭了,她们摸着苦根说:“这孩子真是命苦。”苦根不喜欢她们把眼泪掉到他脸上,拉着我的手一个劲地催我:“走呀,快走呀。”那时候天冷了,我拉着苦根在街上走,冷风呼呼地往脖子里灌,越走心里越冷,想想从前热热闹闹一家人,到现在只剩下一老一小,我心里苦得连叹息都没有了。可看看苦根,我又宽慰了,先前是没有这孩子的,有了他比什么都强,香火还会往下传,这日子还得好好过下去。

走到一家面条店的地方,苦


根突然响亮地喊了一声:“我不吃面条。”我想着自己的心事,没留意他的话,走到了门口,苦根又喊了:“我不吃面条。”喊完他拉住我的手不走了,我才知道他想吃面条,这孩子没爹没娘了,想吃面条总该给他吃一碗。我带他进去坐下,花了九分钱买了一碗小面,看着他嗤溜嗤溜地吃了下去,他吃得满头大汗,出来时舌头还在嘴唇上舔着,对我说:“明天再来吃好吗?”我点点头说:“好。”走了没多远,到了一家糖果店前,苦根又拉住了我,他仰着脑袋认真地说:“本来我还想吃糖,吃过了面条,我就不吃了。”我知道他是在变个法子想让我给他买糖,我手摸到口袋,摸到个两分的,想了想后就去摸了个五分出来,给苦根买了五颗糖。

苦根到了家说是脚疼得厉害,他走了那么多路,走累了。

我让他在床上躺下,自己去烧些热水,让他烫烫脚。烧好了水出来时,苦根睡着了,这孩子把两只脚架在墙上,睡得呼呼的。看着他这副样子,我笑了。脚疼了架在墙上舒服,苦根这么小就会自己照顾自己了。随即心里一酸,他还不知道再也见不着自己的爹了。

这天晚上我睡着后,总觉得心里闷的发慌,醒来才知道苦根的小屁股全压在我胸口上了,我把他的屁股移过去。过了没多久,我刚要入睡时,苦根的屁股一动一动又移到我胸口,我伸手一摸,才知道他尿床了,下面湿了一大块,难怪他要把屁股往我胸口上压。我想就让他压着吧。

第二天,这孩子想爹了。我在田里干活,他坐在田埂上玩,玩着玩着突然问我:“是你送我回去?还是爹来领我?”村里人见了他这模样,都摇着头说他可怜,有一个人对他说:“你不回去了。”他摇了摇脑袋,认真地说:“要回去的。”到了傍晚,苦根看到他爹还没有来,有些急了,小嘴巴翻上翻下把话说得飞快,我是一句也没听懂,我想着他可能是在骂人了,末了,他抬起脑袋说:“算啦,不来接就不来接,我是小孩认不了路,你送我回去。”我说:“你爹不会来接你,我也不能送你回去,你爹死了。”他说:“我知道他死了,天都黑了还不来领我。”我是那天晚上躺在被窝里告诉他死是怎么回事,我说人死了就要被埋掉,活着的人就再也见不到他了。这孩子先是害怕地哆嗦,随后想到再也见不到二喜,他呜呜地哭了,小脸蛋贴在我脖子上,热乎乎的眼泪在我胸口流,哭着哭着他睡着了。

过了两天,我想该让他看看二喜的坟了,就拉着他走到村西,告诉他,哪个坟是他外婆的,哪个是他娘的,还有他舅舅的。我还没说二喜的坟,苦根伸手指指他爹的坟哭了,他说:“这是我爹的。”我和苦根在一起过了半年,村里包产到户了,日子过起来也就更难。我家分到一亩半地。我没法像从前那样混在村里人中间干活,累了还能偷偷懒。现在田里的活是不停地叫唤我,我不去干,就谁也不会去替我。

年纪一大,人就不行了,腰是天天都疼,眼睛看不清东西。从前挑一担菜进城,一口气便到了城里,如今是走走歇歇,歇歇走走,天亮前两个小时我就得动身,要不去晚了菜会卖不出去,我是笨鸟先飞。这下苦了苦根,这孩子总是睡得最香的时候,被我一把拖起来,两只手抓住后面的箩筐,跟着我半开半闭着眼睛往城里走。苦根是个好孩子,到他完全醒了,看我挑着担子太沉,老是停住歇一会,他就从两只箩筐里拿出两颗菜抱到胸前,走到我前面,还时时回过头来问我:“轻些了吗?”我心里高兴啊,就说:“轻多啦。”说起来苦根才刚满五岁,他已经是我的好帮手了。我走到哪里,他就跟到哪里,和我一起干活,他连稻子都会割了。

我花钱请城里的铁匠给他打了一把小镰刀,那天这孩子高兴坏了,平日里带他进城,一走过二喜家那条胡同,这孩子呼地一下窜进去,找他的小伙伴去玩,我怎么叫他,他都不答应。那天说是给他打镰刀,他扯住我的衣服就没有放开过,和我一起在铁匠铺子前站了半晌,进来一个人,他就要指着镰刀对那人说:“是苦根的镰刀。”他的小伙伴找他去玩,他扭了扭头得意洋洋地说:“我现在没工夫跟你们说话。”镰刀打成了,苦根睡觉都想抱着,我不让,他就说放到床下面。早晨醒来第一件事便是去摸床下的镰刀。我告诉他镰刀越使越快,人越勤快就越有力气,这孩子眨着眼睛看了我很久,突然说:“镰刀越快,我力气也就越大啦。”苦根总还是小,割稻子自然比我慢多了,他一看到我割得快,便不高兴,朝我叫:“福贵,你慢点。”村里人叫我福贵,他也这么叫,也叫我外公,我指指自己割下的稻子说:“这是苦根割的。”他便高兴地笑起来,也指指自己割下的稻子说:“这是福贵割的。”苦根年纪小,也就累得快,他时时跑到田埂上躺下睡一会,对我说:“福贵,镰刀不快啦。”他是说自己没力气了。他在田埂上躺一会,又站起来神气活现地看我割稻子,不时叫道:“福贵,别踩着稻穗啦。”旁边田里的人见了都笑,连队长也笑了,队长也和我一样老了,他还在当队长,他家人多,分到了五亩地,紧挨着我的地,队长说:“这小子真他娘的能说会道。”我说:“是凤霞不会说话欠的。”这样的日子苦是苦,累也是累,心里可是高兴,有了苦根,人活着就有劲头。看着苦根一天一天大起来,我这个做外公的也一天比一天放心。到了傍晚,我们两个人就坐在门槛上,看着太阳掉下去,田野上红红一片闪亮着,听着村里人吆喝的声音,家里养着的两只母鸡在我们面前走来走去,苦根和我亲热,两个人坐在一起,总是有说不完的话,看着两只母鸡,我常想起我爹在世时说的话,便一遍一遍去对苦根说:“这两只鸡养大了变成鹅,鹅养大了变成羊,羊大了又变成牛。我们啊,也就越来越有钱啦。”苦根听后格格直笑,这几句话他全记住了,多次他从鸡窝里掏出鸡蛋来时,总要唱着说这几句话。

鸡蛋多了,我们就拿到城里去卖。我对苦根说:“钱积够了我们就去买牛,你就能骑到牛背上去玩了。”苦根一听眼睛马上亮了,他说:“鸡就变成牛啦。”从那时以后,苦根天天盼着买牛这天的来到,每天早晨他睁开眼睛便要问我:“福贵,今天买牛吗?”有时去城里卖了鸡蛋,我觉得苦根可怜,想给他买几颗糖吃吃,苦根就会说:“买一颗就行了,我们还要买牛呢。”一转眼苦根到了七岁,这孩子力气也大多了。这一年到了摘棉花的时候,村里的广播说第二天有大雨,我急坏了,我种的一亩半棉花已经熟了,要是雨一淋那就全完蛋。一清早我就把苦根拉到棉花地里,告诉他今天要摘完,苦根仰着脑袋说:“福贵,我头晕。”我说:“快摘吧,摘完了你就去玩。”苦根便摘起了棉花,摘了一阵他跑到田埂上躺下,我叫他,叫他别再躺着,苦根说:“我头晕。”我想就让他躺一会吧,可苦根一躺下便不起来了,我有些生气,就说:“苦根,棉花今天不摘完,牛也买不成啦。”苦根这才站起来,对我说:“我头晕得厉害。”我们一直干到中午,看看大半亩棉花摘了下来,我放心了许多,就拉着苦根回家去吃饭,一拉苦根的手,我心里一怔,赶紧去摸他的额头,苦根的额头烫得吓人。我才知道他是真病了,我真是老糊涂了,还逼着他干活。回到家里,我就让苦根躺下。村里人说生姜能治百病,我就给他熬了一碗姜汤,可是家里没有糖,想往里面撒些盐,又觉得太委屈苦根了,便到村里人家那里去要了点糖,我说:“过些日子卖了粮,我再还给你们。”那家人说:“算啦,福贵。”让苦根喝了姜汤,我又给他熬了一碗粥,看着他吃下去。

我自己也吃了饭,吃完了我还得马上下地,我对苦根说:“你睡上一觉会好的。”走出了屋门,我越想越心疼,便去摘了半锅新鲜的豆子,回去给苦根煮熟了,里面放上盐。把凳子搬到床前,半锅豆子放在凳上,叫苦根吃,看到有豆子吃,苦根笑了,我走出去时听到他说:“你怎么不吃啊。”我是傍晚才回到屋里的,棉花一摘完,我累得人架子都要散了。从田里到家才一小段路,走到门口我的腿便哆嗦了,我进了屋叫:“苦根,苦根。”苦根没答应,我以为他是睡着了,到床前一看,苦根歪在床上,嘴半张着能看到里面有两颗还没嚼烂的豆子。一看那嘴,我脑袋里嗡嗡乱响了,苦根的嘴唇都青了。我使劲摇他,使劲叫他,他的身体晃来晃去,就是不答应我。我慌了,在床上坐下来想了又想,想到苦根会不会是死了,这么一想我忍不住哭了起来。我再去摇他,他还是不答应,我想他可能真是死了。我就走到屋外,看到村里一个年轻人,对他说:“求你去看看苦根,他像是死了。”那年轻人看了我半晌,随后拔脚便往我屋里跑。他也把苦根摇了又摇,又将耳朵贴到苦根胸口听了很久,才说:“听不到心跳。”村里很多人都来了,我求他们都去看看苦根,他们都去摇摇,听听,完了对我说:“死了。”苦根是吃豆子撑死的,这孩子不是嘴馋,是我家太穷,村里谁家的孩子都过得比苦根好,就是豆子,苦根也是难得能吃上。我是老昏了头,给苦根煮了这么多豆子,我老得又笨又蠢,害死了苦根。

往后的日子我只能一个人过了,我总想着自己日子也不长了,谁知一过又过了这些年。我还是老样子,腰还是常常疼,眼睛还是花,我耳朵倒是很灵,村里人说话,我不看也能知道是谁在说。我是有时候想想伤心,有时候想想又很踏实,家里人全是我送的葬,全是我亲手埋的,到了有一天我腿一伸,也不用担心谁了。我也想通了,轮到自己死时,安安心心死就是,不用盼着收尸的人,村里肯定会有人来埋我的,要不我人一臭,那气味谁也受不了。我不会让别人白白埋我的,我在枕头底下压了十元钱,这十元钱我饿死也不会去动它的,村里人都知道这十元钱是给替我收尸的那个人,他们也都知道我死后是要和家珍他们埋在一起的。

这辈子想起来也是很快就过来了,过得平平常常,我爹指望我光耀祖宗,他算是看错人了,我啊,就是这样的命。年轻时靠着祖上留下的钱风光了一阵子,往后就越过越落魄了,这样反倒好,看看我身边的人,龙二和春生,他们也只是风光了一阵子,到头来命都丢了。做人还是平常点好,争这个争那个,争来争去赔了自己的命。像我这样,说起来是越混越没出息,可寿命长,我认识的人一个挨着一个死去,我还活着。

苦根死后第二年,我买牛的钱凑够了,看看自己还得活几年,我觉得牛还是要买的。牛是半个人,它能替我干活,闲下来时我也有个伴,心里闷了就和它说说话。牵着它去水边吃草,就跟拉着个孩子似的。

买牛那天,我把钱揣在怀里走着去新丰,那里是个很大的牛市场。路过邻近一个村庄时,看到晒场上转着一群人,走过去看看,就看到了这头牛,它趴在地上,歪着脑袋吧哒吧哒掉眼泪,旁边一个赤膊男人蹲在地上霍霍地磨着牛刀,围着的人在说牛刀从什么地方刺进去最好。我看到这头老牛哭得那么伤心,心里怪难受的。想想做牛真是可怜。累死累活替人干了一辈子,老了,力气小了,就要被人宰了吃掉。

我不忍心看它被宰掉,便离开晒场继续往新丰去。走着走着心里总放不下这头牛,它知道自己要死了,脑袋底下都有一滩眼泪了。

我越走心里越是定不下来,后来一想,干脆把它买下来。

我赶紧往回走,走到晒场那里,他们已经绑住了牛脚,我挤上去对那个磨刀的男人说:“行行好,把这头牛卖给我吧。”赤膊男人手指试着刀锋,看了我好一会才问:“你说什么?”我说:“我要买这牛。”他咧开嘴嘻嘻笑了,旁边的人也哄地笑起来,我知道他们都在笑我,我从怀里抽出钱放到他手里,说:“你数一数。”赤膊男人马上傻了,他把我看了又看,还搔搔脖子,问我:“你当真要买。”我什么话也不去说,蹲下身子把牛脚上的绳子解了,站起来后拍拍牛的脑袋,这牛还真聪明,知道自己不死了,一下子站起来,也不掉眼泪了。我拉住缰绳对那个男人说:“你数数钱。”那人把钱举到眼前像是看看有多厚,看完他说:“不数了,你拉走吧。”我便拉着牛走去,他们在后面乱哄哄地笑,我听到那个男人说:“今天合算,今天合算。”牛是通人性的,我拉着它往回走时,它知道是我救了它的命,身体老往我身上靠,亲热得很,我对它说:“你呀,先别这么高兴,我拉你回去是要你干活,不是把你当爹来养着的。”我拉着牛回到村里,村里人全围上来看热闹,他们都说我老糊涂了,买了这么一头老牛回来,有个人说:“福贵,我看它年纪比你爹还大。”会看牛的告诉我,说它最多只能活两年三年的,我想两三年足够了,我自己恐怕还活不到这么久。谁知道我们都活到了今天,村里人又惊又奇,就是前两天,还有人说我们是――“两个老不死。”牛到了家,也是我家里的成员了,该给它取个名字,想来想去还是觉得叫它福贵好。定下来叫它福贵,我左看右看都觉


得它像我,心里美滋滋的,后来村里人也开*妓滴颐橇礁龊*像,我嘿嘿笑,心想我早就知道它像我了。

福贵是好样的,有时候嘛,也要偷偷懒,可人也常常偷懒,就不要说是牛了。我知道什么时候该让它干活,什么时候该让它歇一歇,只要我累了,我知道它也累了,就让它歇一会,我歇得来精神了,那它也该干活了。

老人说着站了起来,拍拍屁股上的尘土,向池塘旁的老牛喊了一声,那牛就走过来,走到老人身旁低下了头,老人把犁扛到肩上,拉着牛的缰绳慢慢走去。

两个福贵的脚上都沾满了泥,走去时都微微晃动着身体。

我听到老人对牛说:“今天有庆,二喜耕了一亩,家珍,凤霞耕了也有七、八分田,苦根还小都耕了半亩。你嘛,耕了多少我就不说了,说出来你会觉得我是要羞你。话还得说回来,你年纪大了,能耕这么些田也是尽心尽力了。”老人和牛渐渐远去,我听到老人粗哑的令人感动的嗓音在远处传来,他的歌声在空旷的傍晚像风一样飘扬,老人唱道:少年去游荡,中年想掘藏,老年做和尚。

炊烟在农舍的屋顶袅袅升起,在霞光四射的空中分散后消隐了。

女人吆喝孩子的声音此起彼伏,一个男人挑着粪桶从我跟前走过,扁担吱呀吱呀一路响了过去。慢慢地,田野趋向了宁静,四周出现了模糊,霞光逐渐退去。

我知道黄昏正在转瞬即逝,黑夜从天而降了。我看到广阔的土地袒露着结实的胸膛,那是召唤的姿态,就像女人召唤着她们的儿女,土地召唤着黑夜来临。

\backmatter

\end{document}