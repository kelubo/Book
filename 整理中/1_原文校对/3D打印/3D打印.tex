% 3D打印
% 3D打印.tex

\documentclass[12pt,UTF8]{ctexbook}

% 设置纸张信息。
% 纸张设置配置文件
% 用于定义书籍的页面尺寸和边距

\usepackage[a4paper,twoside]{geometry}
\geometry{
	left=25mm,
	right=20mm,
	top=25mm,
	bottom=25.4mm,
	headsep=1cm, 
    footskip=1cm,
	bindingoffset=10mm
}

% 设置字体,并解决显示难检字问题。
\xeCJKsetup{AutoFallBack=true}
\setCJKmainfont{SimSun}[BoldFont=SimHei, ItalicFont=KaiTi, FallBack=SimSun-ExtB]

% 目录 chapter 级别加点(.)。
\usepackage{titletoc}
\titlecontents{chapter}[0pt]{\vspace{3mm}\bf\addvspace{2pt}\filright}{\contentspush{\thecontentslabel\hspace{0.8em}}}{}{\titlerule*[8pt]{.}\contentspage}

% 设置 part 和 chapter 标题格式。
\ctexset{
	chapter/name={第,章},
	chapter/number={\chinese{chapter}}
}

% 图片相关设置。
\usepackage{graphicx}
\graphicspath{{Images/}}

% 设置署名格式。
\newenvironment{shuming}{\hfill\zihao{4}}

% 注脚每页重新编号,避免编号过大。
\usepackage[perpage]{footmisc}

\title{\heiti\zihao{0} 3D打印}
\author{佚名}
\date{}

\begin{document}

\maketitle
\tableofcontents

\frontmatter

\mainmatter

\chapter{前言}

您是否曾经考虑过3D打印做些什么?每一件从商店里买的商品,从冰块托盘到玩具机器人,都是制造出来的。其中一些商品是用增材制造法或3D打印完成的。3D打印是通过堆叠一层又一层材料而形成最终成品。这些材料层彼此交融,以致肉眼根本无法看到它们是独立的。如今,人们购买的很多商品都是由这些材料层堆叠形成的。
3D打印技术已经在诸多领域得到发展,现在是时候为您自己购置一台家庭3D打印机啦!通过阅读本书,您可学习到操作3D打印机的必备技巧。然后,您只要学会其中几种就可以在家进行3D打印,不论是娱乐类的玩具机器人,还是实用类的时钟等等。如果您对3D打印感兴趣,那么本书将会是一个很好的起点。它包含所有您需要理解的3D打印知识。仔细阅读本书,然后开始有趣的家庭3D打印之旅吧!

想要了解真正的 3D 打印吗?本书涵盖各种行之有效的方法和策略。一经阅读,您不但能精通 3D 打印领域知识,而且可学习家庭 3D 打印的关键步骤。

首先,需要初步了解家庭 3D 打印机的基本操作;其次,3D 打印技术可谓五花八门,不同的 3D 项目需使用不同的打印技术。本书将会详细介绍这些技术及相关应用领域。

如果您拒绝了解 3D 打印领域的知识,那么您将永远无法发现此领域的真正潜力。虽然家庭 3D 打印机价格昂贵,可一旦学会使用方法,您绝对能感受到何谓物超所值。即使最后它没有派上用场,您也绝对能体会到家庭 3D 打印的无穷乐趣。

准备好成为一个神奇的 3D 打印者了吗?那么,开始使用本书介绍的方法和策略吧!首先,我们将了解 3D 打印的历史;其次,我们需要学习各种 3D 打印技术以及相关应用;最后,我们将讨论如何在家进行 3D 打印。一旦掌握本书知识,您将成为 3D 打印专家,并能完成一些高级 3D 打印项目。只要想想到时候能打印出那些令人惊叹的作品,还有谁能强忍学习 3D 打印的渴望呢?

\chapter{第一章 3D打印的概述}

减材制造法是一种常用的制造法,即消耗材料,制造新产品的方法。虽然效果不错,但减材制造法常常造成浪费,也限制了一些材料的应用方式。为了解决这些问题,人们创造了3D打印。
3D打印采用的制造方法是增材制造法。它利用数字模型文件为3D打印目标建立模型,然后层层叠加目标材料,直到数字模型文件中的打印目标成型,即为一个完整的过程。如果观察材料层,您会发现它们只是一层层轻薄的水平切片,但这些水平切片最终却可以组成一个新的物体。
在本章中,您将学习3D打印的历史,还有3D打印的基本操作方式。而本书其余章节将继续讨论现有各种3D打印方法和应用,以及如何在家进行3D打印。
3D打印的历史
我们已经讨论了原始制造法主要采用减材制造法,而不是增材制造法。20世纪80年代末,制造领域第一次引进了快速成型技术,并将之用于创造各种设计的雏形。1986年,第一个3D打印技术,即SLA(立体光固化成型法)成功申请专利。不过,值得一提的是,人们创造增材制造法并非是为了彻底取代减材制造法,因为在生产某些产品时,减材制造法的效果比增材制造法更好,甚至同时采用这两种方法能够降低制造成本。
随着SLA3D打印机成功问世,其它技术也随之浮出水面,例如,分层实体制造技术(LOM),弹道微粒制造技术(BPM),三维打印技术,光掩膜技术(SGC)等等。然而,直到20世纪90年代末和本世纪初,这些机器才真正投入使用。一方面,当时市场上有各种高端且昂贵的系统,这些系统可为珠宝,医疗和航天等领域制造各种复杂,高度工程化和高价值的部件。另一方面,当时还有更易使用和成本更低廉的系统,这些系统最后演化成了人们在家中使用的桌面机型。
21世纪初,3D打印领域竞争激烈。所有3D打印机制造商都争着做出最好的3D打印机,即精度更高,速度更快和质量更好。他们希望能在2007年内用尽可能最低的成本做出3D打印机,而当时最便宜的机型需一万美金左右。这些致力于3D打印技术的公司都知道如此昂贵的价格是远远难以在市场的,他们必须至少降低一半的价格才能为普通消费者所接受。终于在2009年1月,第一台家庭3D打印机成功面世。
尽管如此,直到2012年,家庭3D打印才有了一些真正的发展。现在市场上有各种各样的设备和技术供消费者选择。有些人甚至称之为一场革命,因为它对工业和家庭应用产生了巨大的影响。更赞的是,3D打印所显露的潜力只是冰山一角,要想将之全部展现在人们面前仍有很长的路要走。3D打印到底能达到什么程度呢?让我们拭目以待!
3D打印是如何作业的?
3D打印过程会因为使用不同的打印机,打印材料和打印技术的而各不相同。在下一章,我们将对此作具体讨论。而现在,我们将介绍3D打印的基本过程。
创造您的设计
在考虑使用3D打印机之前,首先您必须知道您想要创造什么。为此,您需要CAD,即计算机辅助设计软件。然后,您就可以扫描或创建一个现有物体或用3D建模程序为您自己的设计建模。生成数字模型文件后,打印机就可以打印出您的设计。
准备数字模型文件
您的设备将为您完成这项步骤。3D建模程序会将设计图片切分成数以百计,甚至数以千计的薄片。一旦这些切片组合在一起,您的设计就成型了。
最终产品
一旦将材料放入打印机,它就会开始打印您的设计。最终产品会和3D模型一模一样。通过电脑控制把打印材料一层一层叠加在一起,直到最终将计算机上的蓝图变成您想要的实物,方为一个完整的过程。因为每一层切片都和上一层切片合为一体,所以由这些切片共同构成的最终产品所拥有的体积已经不能更小了。

\chapter{第二章 3D打印的技术和方法}

谈及3D打印的时候,我们必须知道的其中一件重要的事情是,世界上没有任何一台打印机能完全满足您的需求。每台打印机都使用不同的材料和技术,即没有一劳永逸的选择。本章,我们将介绍3D打印的七种主要方法。
光聚合
光聚合技术采用液态光敏树脂。在成型室里铺上一层层液态光敏树脂,每铺完一层,就有一道紫外光照射进成型室并将之固化。此过程利用SLA技术(光固化技术)。当每一层液态光敏树脂摆放好后,紫外光会透过它们并将之与前一层液态光敏树脂固化在一起。同时,陈放液态光敏树脂的升降式工作台将下降几分之一毫米,为成型室留出更多的空间。然后,刮板继续扫平新铺的液态光敏树脂层,直到新液态光敏树脂层与前一层液态光敏树脂固化在一起。此过程将一直持续,直到3D物体打印完成。而一旦完成打印,任何在打印过程中用于支撑3D物体的材料必须手动移除。
材料喷射
材料喷射技术利用小喷嘴喷洒材料,类似平时我们看到的喷墨打印机。材料从喷嘴喷出,洒落在成型平台上。每放置一层,便使用紫外光将该层材料和上一层材料一同硬化。此过程将一直持续,直到3D物体打印完成。
粘合剂喷射
粘合剂喷射利用的是液体粘合剂和粉末状材料。使用刮板均匀地将各材料层上的粉状颗粒涂抹均匀。先是喷洒粉末,然后喷洒液体粘合剂使新增粉末层与已铺设的粉末层固化在一起。物体打印完成后,任何残留粉末都需要擦拭干净。
材料挤出
材料挤出技术通过FDM(熔融沉积成型技术)实现。它共有3个主要部件,一个塑料或熔融材料喷嘴,已沉淀材料和一个能在打印过程中准确移动的工作台。为防止硬化,打印材料从高温喷嘴中挤出,然后直接应用于已沉淀材料上,最后使之快速硬化,而工作台则和喷嘴一起绕转并安置已铺设的打印材料。
粉末床熔融
粉末床熔融运用SLS,即选择性激光烧结技术。玻璃、金属、塑料或陶瓷粉末在高功率激光的作用下烧结在一起。激光扫描每层材料,确认它是最终物体的一部分后,将它烧结在粉末床上。完成各个部分并增加新材料后,粉末床会下降一个层次。此过程将一直持续,直到3D物体打印完成。这种方法的优点是,粉末就可形成支撑结构,不需要额外的支撑物。
薄片层压
薄片层压技术借助外力将材料层粘合在一起,使之最终形成3D打印物体。大部分情况下,这些材料层为聚合物、金属或纸片形成的薄片。使用不同的材料意味着后续操作也不同。例如,纸片是通过胶粘剂胶合和精密刀片切割,从而形成最终产品,而金属片是在CNC,即电脑数值控制下加工成正确的形状,再用超声波焊接在一起。
定向能量沉积
定向能量沉积使用的是多轴机械臂和3D印刷设备。喷嘴喷出粉末(一般是金属)于表面上,然后等离子弧,电子束或其它能源将粉末融化并融合在一起,进而固化成最终物体。
既然您已经了解3D打印领域的一些技术,那么现在是时候学习它们的应用了。

\chapter{第三章 3D打印的应用}

3D打印在诸多领域都非常实用。在法医病理学方面,它可用来为宠物打印假肢,而这仅是它广大用途的冰山一角。自20世纪80年代3D打印出现以来,它在未来数十年的应用将非常让人期待和兴奋。
航空宇宙与航空——打印新型喷气发动机配件,班机内舱等等
娱乐业——打印现实和非现实的电影道具
汽车业——为引擎,测试的车辆和其它各种平台打印汽车配件
牙科——定制假牙,活动假牙,牙冠,冠桥等等
古生物学和考古学——重建化石和史前古器物
工业印刷——自定义产品颜色,定制产品等等
生物印刷——构建可以孵化并形成组织的功能细胞
法医病理学——重塑身体器官和骨头
犯罪现场调查——再现遭受严重破坏的犯罪现场证据
假肢——定制假肢,包括手、腿,甚至宠物假肢
现在,您已经基本了解3D打印的应用,接下来让我们学习如何使用家庭3D打印机。

\chapter{第四章 家庭3D打印}

既然您已对3D打印技术有所了解,也知道它们的相关应用,那么接下来开始讨论家庭3D打印。本章,您将学习家庭3D打印的基本步骤,还有一些关于选择软件和打印机的小贴士等等。
选择3D建模软件
大部分适合3D打印的软件是为那些有3D建模经验的人们而设计。不过,仍有几个软件适合初学者。下列软件便是笔者特意为您挑选的,请一定查看。
3Dtin
它允许用户直接在浏览器中打开网站开始建模,是最简单好用的软件之一。
谷歌草图大师
它有两大优势:好用和免费。它支持在平面和边上构建模型,然后使用拉伸工具使3D平面突出,从而创建一个3D模型。而且,它可以和谷歌地球一起使用,非常方便。用户可以从谷歌地球导出大型航拍图片,或将谷歌地球导入草图大师后再开始建模。
Tinkercad
虽然它只设有3个工具,但绝对能支持用户创建大量设计。完成设计之后,下载STL文件就可以开始3D打印了。
OpenSCAD
它可为3D打印创建实物模型。免费,很赞!不同于供给新手的大部分选择,它专注于CAD方面,而不是3D建模的艺术方面。
Blender
它是一个强大的软件,拥有许多您只会在高端3D软件中看到的功能。而且,它是免费的开源软件,允许用户创建和分享自己的设计。
练习
在真正购买3D打印机之前,您应该适当利用软件并花几个月的时间练习设计。就算最聪明的学习者可以在6个月内学会设计,但学会的也只是基础罢了。只有当您对自己的设计有信心,也乐意在家里进行3D打印时,才是购买3D打印机的最佳时机。
选择正确的3D打印机
为了确保购买的3D打印机适合您的家庭打印需求,在购买之前,必须先考虑如下几个方面。
技术
家庭3D打印有三项共同的技术,包括融合沉积成型、塑料喷射印刷,和立体光固化技术。此三项中的任一个都允许客户创建3D模型,但在具体的打印过程中,它们可能不适用于某些设计。因此,在购买3D打印机之前,一定要确认清楚该打印机能否打印出您感兴趣的产品。
材料
大部分打印机可兼容一种材料,或者一些特定的材料,具体如下:
丙烯腈-丁二烯-苯乙烯共聚物(ABS)——它的优点是易上色且色彩丰富,可融化成半液体。同时它具有非常高的耐热性和柔韧性。
聚乳酸(PLA)——它是一种半液体,同时可附着多种颜色,因可用于制造更锋利的角和更薄的层而闻名。
不锈钢——这是最硬的3D打印材料,可惜大部分3D打印机都无法兼容。
树脂——树脂打印的实物质地光滑且多变。只是当暴露于紫外光下时,树脂易变色,但可以通过刷亮光漆来解决这个问题。

尼龙——易上色,且可制造出各种形状。
高抗冲聚苯乙烯(HIPS)——作为辅助性材料,此类细丝用处极大。它主要缺点是耐热性差,因此在打印过程中很难附着。
木塑3D打印丝——此类细丝通常是木头和高抗冲聚苯乙烯的复合物。加工方法类似聚乳酸(PLA)和丙烯腈-丁二烯-苯乙烯共聚物(ABS),不过用它打印出的物体不论是闻起来,还是看起来都像是真正的木头。
要求
关于对3D打印的要求,您应该考虑如下方面:
速度
这并非每个人都关心的事情,除非他们从事制造业。尽管如此,大部分3D打印机的打印速度为40到150mm/s。不过无须担心,速度并不会影响最终产品的分辨率水平。
层分辨率
选择3D打印机时,您应该考虑的是最终成品的流畅度。若是分辨率高的成品,您很难在其表面上看到一个个独立的材料层,而在低分辨率的成品上,材料层则是又厚又明显。
成品体积
成品体积与打印区域大小有关。也就是说,成品体积为打印区域所约束。所以一定要确保您的3D打印机拥有足够大的打印区域打印您的设计。
喷嘴
此外,您也应该考虑3D打印机的喷嘴数量。这些喷嘴决定了您可以用于设计的材料数量和颜色种类。
一旦您有了自己设计,请继续阅读下一章——家庭3D打印的实际应用。

\chapter{第五章 可尝试的高级3D打印技巧}

既然您已经基本了解家庭3D打印的基本流程,那么是时候讨论您可以亲手设计的东西了。其中一些设计非常基础(只要设计出来就可以了),不过大部分还是要求有相关经验的。
各种小配件
3D打印最实用之处是打印各种家用小配件。在您家里可能有很多东西缺失了一个部件,而且甚至可能再也找不到了,那么您可以用CAD软件设计这些东西,然后用3D打印机打印出来,就万事大吉了!
乐器
只要设计正确,那么3D打印机便能够打印出真正的乐器。目前,3D打印机已经打印出的乐器包括小提琴,原生吉他,奥卡利那笛等等。如果您的打印机太小,那么您可以考虑把想要打印的乐器设计成几个小一些的互锁配件。
玩具
3D打印机可以打印五花八门的玩具,譬如模型。您可以打印玩具士兵,大象,机器人,仿真汽车模型,飞机等等。您也可以为自己打印一套独一无二的国际象棋,或者一套供孩子或自己玩耍的乐高积木。
各式各样的小玩意儿
您永远也无法预测3D打印会制造出什么样的东西。正是出于这个原因,本书陈列了如下物品,供您参考!
·软管的喷头部件
·镜头
·织布机
·手机壳
·可折叠钱包
·独一无二的锁
等等等!
珍藏
3D打印成品并非一定要有趣或者实用。若有一些东西,只是看着便可以让您感动连连,那么就将它们通通打印吧!例如:
·您未来宝宝的3D模型(扫描自您的超声波图像)
·您孩子绘画作品的3D成品
·您孩子独一无二的3D形象
拥有工程专业知识,您可以做些什么
如果您是工程学专家,那么您在3D打印领域就可以随心所欲了!例如,创建一个时间机器人,让它每分钟写下时间,下一分钟擦掉时间并重写新的时间。您甚至可以创建一个工作机器人,或是带着摄像机的四轴飞行器等。
既然您已经有一些想法了,那么从小开始,用自己的方式走出自己的3D打印之路吧!家庭3D打印不仅非常有趣,而且超级实用。不要害怕,大胆地去尝试并打造出自己独一无二的设计吧!您永远想不到自己能创造出多么棒的作品!



\chapter{结论}

再次感谢您下载本书!
无论是工业还是家庭3D打印,我都希望本书能够帮助您实现各种各样的3D打印。
下一步就是开始学习!如果您已经玩转这些软件,那么希望本书可以给您一些关于3D打印项目的启发。如果不能,努力学习设计技巧,买一台3D打印机,开始打印吧!
最后,如果您喜欢本书,请花一些时间分享您的想法,并在亚马逊网站上留下您宝贵的评论!衷心感谢!
再次感谢!祝您一切顺利!

\begin{figure}[htbp]
	\centering
	\includegraphics[width=0.7\linewidth]{1}
	\caption{}
	\label{fig:1}
\end{figure}



\backmatter

《3D打印:知道这些就够了》 戴尔·沃勒(Dale Waller) 著 文刀 译


\end{document}