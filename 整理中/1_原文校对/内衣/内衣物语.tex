% 内衣物语 - 现代女性必备内衣使用手册
% 基于模板文件修改

\documentclass[12pt,UTF8]{ctexbook}

% 设置纸张信息。
% 纸张设置配置文件
% 用于定义书籍的页面尺寸和边距

\usepackage[a4paper,twoside]{geometry}
\geometry{
	left=25mm,
	right=20mm,
	top=25mm,
	bottom=25.4mm,
	headsep=1cm, 
    footskip=1cm,
	bindingoffset=10mm
}

% 设置字体,并解决显示难检字问题。
\xeCJKsetup{AutoFallBack=true}
\setCJKmainfont{SimSun}[BoldFont=SimHei, ItalicFont=KaiTi, FallBack=SimSun-ExtB]

% 目录 chapter 级别加点(.)。
\usepackage{titletoc}
\titlecontents{chapter}[0pt]{\vspace{3mm}\bf\addvspace{2pt}\filright}{\contentspush{\thecontentslabel\hspace{0.8em}}}{}{\titlerule*[8pt]{.}\contentspage}

% 设置 part 和 chapter 标题格式。
\ctexset{
	part/name= {第,卷},
	part/number={\chinese{part}},
	chapter/name={第,篇},
	chapter/number={\arabic{chapter}}
}

% 图片相关设置。
\usepackage{graphicx}
\graphicspath{{images/}}

% 设置署名格式。
\newenvironment{shuming}{\hfill\zihao{4}}

% 注脚每页重新编号,避免编号过大。
\usepackage[perpage]{footmisc}

% 列表项向右偏移。
\usepackage{enumitem}

% 解决图片浮动问题
\usepackage{float}

\title{\heiti\zihao{0} 内衣就该这样穿}
\author{}
\date{}

\begin{document}

\begin{figure}[H]
	\centering
	\includegraphics[width=0.8\linewidth]{ny_cover}
\end{figure}

\maketitle
\tableofcontents

\frontmatter

\mainmatter

\part{解开女人的"内在"密码}

\chapter{内衣的时尚变迁}

在西方,内衣被译为 Undercover 或 Underwear,它包括紧身胸衣(Corset)、乳罩(Bra cup)、掐腰(Waist nipper)、连胸紧身衣(All-in-one)、背心式衬裙(Camisole)、短腰(Short)等许多种类。

胸衣最早出现在古罗马时期。欧洲文艺复兴以前,女性身体几乎不加束缚,贵妇们穿上衬裙作为内衣。到了17世纪,人们认为衬裙作为内衣太放荡。由此,出现了"束衣",也开始了折磨女性的历史。妇女们用几近残酷的手法,把自己的身体,用重重的布条勒起来,很多女士因此而导致肋骨骨折、流产、内脏移位等。这种被女性自觉带上的"刑具",被当时的媒体称为"舒适的工艺"。

\begin{figure}[H]
	\centering
	\includegraphics[width=0.7\linewidth]{ny_001}
	\caption{17世纪的"金属紧身内衣"。}
	\label{fig:ny_001}
\end{figure}

20世纪初,新女性掀起时装新潮,标准内衣的位置以长于臀围线以下10~40厘米为宜。伴随弹性织物在服装中的广泛应用,内衣变得越来越舒适易穿。到了1922年,文胸终于被普及,这是一次真正的解放。第一次世界大战对女性内衣产生了重大影响,战争过后,女性们以男生的形象来打扮自己。各种不同的内衣开始出现,有组合式的,也有单件的。除了追求男生式的外表外,女士们还选择那些能表现女性柔美的内衣。这时一些轻盈、悬垂式的内衣,粉红、黄、蓝灰、翡翠、紫、橘、绯红以及黑色等开始流行。新设计的代表作是紧身裤,广告称之为"与无袖衬衫、短衬裤、胸衣和女短袖衬衫一样,能使身材自然柔美、挺立苗条,为新时代时装所必须"。这些内衣多用棉织品、丝质针织物和衬垫制作而成。

\begin{figure}[H]
	\centering
	\includegraphics[width=0.7\linewidth]{ny_002}
	\caption{18世纪中旬的紧身胸衣。}
	\label{fig:ny_002}
\end{figure}

20世纪30年代的胸衣大部分已经不用镶边和撑骨,人造纤维技术的进步促成了两种材料的出现:一种是双向拉伸的松紧带;另一种是人造丝。用这些材料制成的胸衣又轻又有弹性,既能保持女性体形又不会伤害身体。

\begin{figure}[H]
	\centering
	\includegraphics[width=0.7\linewidth]{ny_003}
	\caption{20世纪30年代的内衣广告。}
	\label{fig:ny_003}
\end{figure}

好莱坞明星引导了20世纪30年代追求苗条、柔雅以及曲线美的时尚。女士们对银幕上的"性感女神"极为崇拜,并模仿这些女明星们的化妆和着装,例如葛莉塔·嘉宝(Greta Garbo)、马琳·帝特里奇(Marlene Dietrich)、珍·哈罗(Jean Harlow)、梅伊·韦斯特(Mae West)以及琼·克拉福(Joan Crawford)等。美国和法国的服装设计师们应用能贴着身体的、显示性感线条的、柔软而优美的缎子和双绉面料,为好莱坞明星们,也为富裕的上流社会的顾客设计和提供她们所渴望的服装。当时,法国设计师制造了细长外形的内衣。这种内衣能滑过臀部和腰部,隔开双乳,并使胸部线条成为时髦的焦点。

\begin{figure}[H]
	\centering
	\includegraphics[width=0.7\linewidth]{ny_004}
	\caption{20世纪40年代末的内衣广告。}
	\label{fig:ny_004}
\end{figure}

为了展现曲线美,新型的内衣成为了必需品。这种新型内衣大多采用了人造纤维,从人造丝到弹性网状材料。它们虽然都很实用,但过于单调。为了保持线条的流畅,装饰已经限制到最少量。20世纪30年代后期,出现了尼龙材料。尼龙纤维质轻、强度高、柔软,并能织出不同重量的纱线,其织物不需熨烫,挂起来可以很快晾干,它是制作内衣的理想材料,不久后人们发现,几乎所有内衣中都有尼龙的痕迹。

随着二次世界大战的爆发,世界陷入了危机之中。二战导致经济衰弱、物质缺乏,内衣成了奢侈品。1945年战争结束,虽然随后的几年仍执行着战时的配给制,但欧洲的女士们又开始渴望展现女性魅力。克里斯汀·迪奥于1947年在巴黎发布了献给女性们的"迪奥新风格"新型内衣,英国《图片时报》的评语是:“不再有慵懒的肩膀,而是紧束的腰带和臀部垫衬的宽大的裙子……这些服装只能是巴黎制造,因为只有巴黎才敢进行这种大胆的尝试”。“迪奥新风格”款式的关键是它腰上的细小“腰带”,这是由马斯尔·鲁采斯(Marcel Rochas)于1945年设计的。这种内衣在美国名为“肚带”,在英国称为“蜂腰带”,而在法国则称为“胡蜂腰”。这种腰衣有条12--20厘米的长撑骨腰带,以后逐渐演变成与衬裙结合的腰围,另一种演变是不带撑骨而只需穿戴乳罩的轻柔服装。《时尚》杂志对其描述为:"体形不只是上帝赐给的,风度优雅靠的是良好的锻炼和合适的胸衣。锻炼靠的是意志,而胸衣靠的是明智选择。"

随着新的印染工艺的出现,20世纪50年代的内衣在流行颜色和图案方面的选择范围变得很广。流行颜色主要有宝石蓝、茶红、翡翠绿、珊瑚色、紫红色和桃红等几种,并配以各种花卉图案。娱乐业也对胸衣设计有很大的影响。正处在"猫王"与"比尔·哈里和彗星"音乐的时代,摇滚乐和摇摆舞盛行。这些舞蹈都要求腰臀部大幅度摆动。尽管如此,束缚性的、传统的价值观念仍占主流。法国女演员利斯里·卡罗(Leslie Caron)和阿第瑞·赫布博(Audrey Hepburn)在影片《萨布瑞那集市》(Sabrina Fair)中的造型,奠定了20世纪50年代后期内衣的基调。就在那时,社会变革达到高峰,出现了一系列的重大技术革新。

\begin{figure}[H]
	\centering
	\includegraphics[width=0.7\linewidth]{ny_005}
	\caption{20世纪80年代紧身内衣。}
	\label{fig:ny_005}
\end{figure}

\begin{figure}[H]
	\centering
	\includegraphics[width=0.7\linewidth]{ny_006}
	\caption{胸衣,裙架和女用内衣。}
	\label{fig:ny_006}
\end{figure}

\begin{figure}[H]
	\centering
	\includegraphics[width=0.7\linewidth]{ny_007}
	\caption{20世纪末的紧身内衣。}
	\label{fig:ny_007}
\end{figure}

20世纪70年代,在崇尚健康的潮流下,内衣的设计注重承托力。同时,朋克风格开始流行,并对整个80年代的设计产生了重大的影响。

20世纪80年代,内衣设计者开始了表现人体美的新探索,从单纯注重内衣的功能性转向注重外观魅力与功能并重。内衣造型除了突出女性柔美线条外,还需要华丽的织物,这些织物包括饰带、优质棉布、网眼花边、光泽丝织品和绉绸、乔其纱缎子,以及各种氨纶类型的材料等。有些织物是透明的,有些是半透明的,带有饰边、活褶和褶裥等。

这种服装时尚在20世纪80年代中期由美国的电视肥皂剧加以强化。人们仍渴望苗条健美身材,到了20世纪80年代后期,女士们意识到可以借助于一些人工手段和塑型"内衣",达到她们所要求的理想身材。从运动装中演变而来的胸衣已成功地成为体型塑型服,其主要特征是在胸部采用钢托,而在裆部用合适的扣合件进行弹性控制调节。

21世纪的今天,女性内衣的发展犹如经历了一个完整的轮回。克里斯汀·迪奥一语道破天机:"没有时装的基础,就不会有时装的时尚。"

\begin{figure}[H]
	\centering
	\includegraphics[width=0.7\linewidth]{ny_008}
	\caption{各式妇女和儿童用紧身内衣和紧身胸衣。}
	\label{fig:ny_008}
\end{figure}

\begin{figure}[H]
	\centering
	\includegraphics[width=0.7\linewidth]{ny_009}
	\caption{穿紧身胸衣的麦当娜。}
	\label{fig:ny_009}
\end{figure}

\chapter{女人的身体密码}

\section{了解你的乳房结构}

大多数人觉得乳房是女人身体最美的地方。女人的胸部,从身体部位来讲,一定是最有女人味的地方。究竟什么样的乳房是美丽的?又是什么在影响着乳房的生长?充分地了解乳房结构,有助于我们更好地保护乳房。拥有美丽的乳房从了解开始。

\subsection{乳房的形状}

乳房与人的相貌一样,千人千面。按乳房隆起程度划分,可大体分出七种形状。

除了形状外,乳房的良好形态取决于人的举止。青少年期是发育的关键性时期。妈妈们应该留意女儿的乳房增大,并强迫她们保持挺直的身姿。乳房在女人的一生中会有形变,乳房形状绝不会一直不变。有些妇女比起其他的妇女受这种变化的影响要小一些。当然,乳房的形变离不开遗传因素。一般来说,年轻姑娘的乳房与她妈妈或祖母的相似,但是也有许多的例外。

●圆盘型

乳房隆起不高,但底部不小,像两个薄薄的盘子。选用下部及侧部加厚的功能性文胸,将乳房向内推,从而使乳房抬高隆起。

\begin{figure}[H]
	\centering
	\includegraphics[width=0.7\linewidth]{ny_010}
	\caption{圆盘型}
	\label{fig:ny_010}
\end{figure}

●半球型

乳房隆起较大,且饱满,如同球形的两半。可选用3/4罩杯、宽肩带、带钢圈的文胸,切忌常穿1/2罩杯的文胸。

\begin{figure}[H]
	\centering
	\includegraphics[width=0.7\linewidth]{ny_011}
	\caption{半球型}
	\label{fig:ny_011}
\end{figure}

●纺锤型

乳房隆起很高,但底部不大,乳房向前突出并稍有垂感,形似纺锤。可选用3/4罩杯、全罩杯、宽肩带、带钢圈的文胸,以防止乳房下垂。

\begin{figure}[H]
	\centering
	\includegraphics[width=0.7\linewidth]{ny_012}
	\caption{纺锤型}
	\label{fig:ny_012}
\end{figure}

●外扩型

两个乳房过分地向两边扩展,一般因长期保护不当,穿着过于压迫的文胸造成,极少数是天生的。适合3/4罩杯、心位间距小、罩杯碗侧有加强设计、集中效果好的文胸。

\begin{figure}[H]
	\centering
	\includegraphics[width=0.7\linewidth]{ny_013}
	\caption{外扩型}
	\label{fig:ny_00002}
\end{figure}

●下垂I型

乳房隆起但下垂,下侧一部分碰到胸部。可选用高心位、宽肩带、宽比位、罩杯下部加强设计的文胸进行调节。

\begin{figure}[H]
	\centering
	\includegraphics[width=0.7\linewidth]{ny_014}
	\caption{下垂I型}
	\label{fig:ny_014}
\end{figure}

●一大一小型

刚发育的女孩子有25\%适当加大空距会大小不一。适合文胸是插片文胸,尺码设定以大的乳房为准,小的一边可用棉垫补充。

\begin{figure}[H]
	\centering
	\includegraphics[width=0.7\linewidth]{ny_015}
	\caption{一大一小型}
	\label{fig:ny_015}
\end{figure}

●圆锥型

乳房隆起较小,底部也不大,但整体挺拔,呈圆锥状。使用下衬垫文胸,可以使整个胸部显得更加丰满。

\begin{figure}[H]
	\centering
	\includegraphics[width=0.7\linewidth]{ny_016}
	\caption{圆锥型}
	\label{fig:ny_016}
\end{figure}

\subsection{乳房的生理结构}

\begin{figure}[H]
	\centering
	\includegraphics[width=0.7\linewidth]{ny_017}
	\caption{生理结构}
	\label{fig:ny_017}
\end{figure}

乳房主要由腺体、导管、脂肪组织和纤维组织等构成。其内部结构有如一棵倒着生长的小树。

乳腺组织:保持乳房健康。乳腺组织负责分泌乳汁,它受激素控制,每个月经周期它会逐渐增大然后复原。乳腺组织是脆弱的,我们平时不良的生活习惯,如熬夜、暴食、情绪激动等原因都会直接影响激素水平,导致体内激素动荡刺激到乳腺组织,从而引发一些病变。

结缔组织:防止乳房下垂。结缔组织与胸部肌肉结合一起,是悬挂乳房的组织。它完全没有弹性,被过度拉伸致使组织断裂就难以回复,最终造成乳房下垂。结缔组织断裂后难以恢复,因此需要时刻保护,这也是我们戴文胸的原因之一。

脂肪组织:控制乳房大小。乳房中最多的是脂肪,腺体组织和结缔组织漂浮在脂肪之中,脂肪多少决定乳房大小。看到这里,你大概可以明白为什么女人不能过度节食了吧!女性过度节食的唯一结果是全身普遍减脂,当然也包括乳房,减肥的美女可要当心减成了小胸女哦。

胸肌:决定乳房形状。乳房靠结缔组织外挂在胸肌上,胸肌的支撑决定着乳房的走向。通过锻炼能使胸肌增长,托高胸部,而锻炼韧带可以使得胸部更加挺拔,胸肌的增大会使乳房突出,胸部看起来更丰满。了解了这个原理之后,爱美的女性一定要加强身体锻炼,因为丰满的胸部是锻炼出来的。

\section{最完美的身材比例}

完美的尺寸因人而异。

医学美学认为,女性的曲线美是世界上最美的事物。每个人的体型不一样,漂亮的胸围和腰围有好几种,最好看的形体尺寸并不是绝对固定的。

最佳的身材比例就是全身的均衡协调。

完美的身材比例就是全身达到秾纤合度的状态。当个人的个性特质能充分展现、全身体态能达到均衡状态时,就能产生最完美的身材比例。

\begin{figure}[H]
	\centering
	\includegraphics[width=0.7\linewidth]{ny_018}
	\caption{身材比例}
	\label{fig:ny_018}
\end{figure}

①胸部的位置,正好落在肩膀和腰部的中间。

胸部最高的地方正好落在肩膀与腰部中间,就是相当完美的比例。

②乳间距离约为肩膀宽度的1/2。

因为每个人的体型都不一样,所以不能一概而论,但就整体而言,若乳间(即乳头与乳头之间的距离)约为肩膀宽度的1/2,即是完美的比例。

③从侧面来看,胸部比腹部突出的话就是漂亮的比例。

④臀部的位置大约在接近身高1/2的地方就是完美比例。

\section{准确测量你的三围}

每个女人都关心自己的三围尺寸,三围是女人一生的话题。准确把握自己的三围尺寸,做一个了解自己的女人。每个国家测量三围的标准量法是不太一样的,下面就详细地介绍一下在中国最常用的标准测量方法。

TIPS

在日本等一些国家,测量女人胸围最常用的方法是身体呈90度的鞠躬姿势,然后用标尺测量上胸围和下胸围的尺寸。

\begin{figure}[H]
	\centering
	\includegraphics[width=0.7\linewidth]{ny_019}
	\caption{三围}
	\label{fig:ny_019}
\end{figure}

胸围:胸围反映胸廓的大小和胸部肌肉与乳房的发育情况,它是身体发育状况的重要指标。测量胸围时,身体直立、两臂自然下垂、呼吸均匀自然。软尺通过乳房最丰满的位置,绕胸围一周,得出胸上围尺寸,测量时可轻轻转动皮尺,以防过紧或过松,软尺紧贴乳房隆起处下缘,绕胸围一周,即得乳房下缘一周尺寸,得出胸下围尺寸。

腰围:腰围反映腰腹部肌肉的发育情况。测量时,身体直立,两臂自然下垂,不要收腹,呼吸保持平稳,皮尺水平放在髋骨上、肋骨下最窄的部位(腰最细的部位)。

臀围:臀围反映髋部骨骼和肌肉的发育情况。测量时,两腿并拢直立,两臂自然下垂,皮尺水平放在前面的耻骨联合和背后臀大肌最凸处。

为了确保准确性,测量三围时有三点需要注意:一是要在横切面上;二是要在锻炼前进行;同时要注意每次测量的时间和部位相同,测量时不要把皮尺拉得太紧或太松,力求仔细、准确。

★标准三围的换算方法:

胸围=身高×0.51。

如:身高160cm的标准胸围=160cm×0.51=81.6cm。

腰围=身高×0.34。

如:身高160cm的标准腰围=160cm×0.34=54.4cm。

臀围=身高×0.542。

如:身高160cm的标准臀围=160cm×0.542=86.72cm。

\chapter{精致女人的内衣哲学}

内衣是女人最俏皮的真我体现。不同款式的内衣不仅能让女人变得风情万种、仪态万千,还能帮她们淋漓尽致地表达出内心的全部情愫。

精致的蕾丝,恰到好处的弧线,温柔的色彩,如今商场里令人眼花缭乱的品牌设计足以满足女性对内衣最大的想象空间和选择空间。

\section{文胸的结构及各部位作用}

文胸由三部分组成。

罩杯:可分上托碗和下托碗。

比位(下扒位):由鸡心、后比和侧比三部分组成。

肩带:可分宽肩带和窄肩带两种。

\begin{figure}[H]
	\centering
	\includegraphics[width=0.7\linewidth]{ny_020}
	\caption{结构图}
	\label{fig:ny_020}
\end{figure}

各部位作用。

①鸡心位:控制心位间距,使胸部服帖,不移位。

②侧比:有效收紧两肋的多余脂肪以将其推挤在前文胸杯内。

③后比:根据不同款式和功能可分为"V"型、"U"型、"T"型、背心式,其特点如下:

"V"型:背后肩带呈"V"型,肩带不易滑落。

"U"型:比较适合大杯围文胸,可防止肩带侧滑。

"T"型:较常见的普通设计。

背心式:常用于运用型文胸,肩带较宽,穿着稳定舒适。

④耳仔位:罩杯的提升位,连接肩带与罩杯,起侧提抬胸效果。

⑤钩圈:可调节胸下围的长度1.5~2.5cm。

⑥肩带:配以调节扣,可调节长短,不同宽度的肩带适合不同的产品。宽肩带通常15mm以上,常用于C、D杯人群,可减轻肩带负担令肩部舒适,窄肩带通常用于纤巧型的文胸。

⑦钢圈:承托收拢乳房,矫正体形。

\section{内衣的常用面料及特点}

内衣的面料非常多元化,除了麻质,还有棉、丝绢、毛、丝绒、聚酯纤维等,而喜欢突破的设计师会用上皮料、塑胶、羽毛等,以求创新与卖点。

纺织纤维的分类

棉纤维。

合成纤维。

\begin{figure}[H]
	\centering
	\includegraphics[width=0.7\linewidth]{ny_021}
	\caption{纺织纤维}
	\label{fig:ny_021}
\end{figure}

常用纤维的类别和特点

\begin{figure}[H]
	\centering
	\includegraphics[width=0.7\linewidth]{ny_026}
	\caption{纺织纤维类别及特点}
	\label{fig:ny_026}
\end{figure}

内衣的辅料

扣子、勾圈、捆条、钢圈、丈根、花仔、吊牌、洗水商标、肩带胶骨、针线、高弹线等都是常用的内衣辅料。

主料(面料)及其特点。

A 文胸类:

①文胸的碗里用料及其特点。

棉汗布(35\%棉+65\%涤纶):弹性小。用在文胸的碗里部位,保证穿着的舒适性、透气吸湿、肤感较好。

②文胸的碗面用料及其特点。

棉拉架:是棉与弹性纤维相织而成的一种面料,弹性大,透气吸湿、柔软舒适。

花边:又称蕾丝,美观,适合做罩杯面料,有装饰点缀效果。弹性好的花边还可做文胸后比。

鱼网布:双向弹性面料,不同纹向弹性区别大,具有朦胧、透气的效果。

超细面料:是由超细纤维织成的面料,其特征为:柔软、有光泽、有色泽、质地轻、悬垂性好、吸湿透气、快干、清理容易、强韧有弹性,但容易勾丝起毛,打理不方便。

③文胸的后比用料。

鱼网、花边、超细面料、棉拉架。

④文胸鸡心位、侧比位用料及其特点。

定型纱:又称格子纱,无弹性,主要用做文胸的心位、侧比。

⑤文胸的罩杯用料及其特点。

海绵:海绵是一种化合物,不容易变形,适合做文胸的罩杯,通过模压技术热压定型而成模杯和活动棉垫。(注:非面料)。

喷绵:弹性差,质料柔软,透气性好,一般用于夹棉款的罩杯。

B.内裤类。

①内裤前后幅常用物料及其特点:

鱼网、超细面料、花边。

棉:透气吸湿,保暖,柔软舒适,但不耐磨,易变形。

②内裤浪位常用物料:棉汗布(同文胸碗里)。

C.睡衣常用物料及其特点。

●四十针:成分是锦纶,无弹面料,其特点轻薄柔软,具悬垂性,主要用于内裤和夏季睡衣。

●棉:透气吸湿,柔软舒适,但不耐磨,易变形。

●真丝:外观光泽优雅柔和,像珍珠的光泽;手感柔软光滑,富有弹性,穿着时感觉舒适、吸湿性优良,透气清凉,但是易起皱、易勾丝。

TIPS

真丝睡衣需用中性肥皂、中性洗涤剂或真丝专用洗涤剂清洗,千万不可使用含碱的洗涤剂。

\section{尺码内衣的概念}

Bra尺码的概念

胸下围的尺寸决定文胸的围度:国际通用下围尺码为70、75、80、85、90、95,误差为正负2.5cm。例如,你的胸下围在72.5--77.5之间,那么你理所当然应该选择75码的文胸。

有些女士喜欢采用英式尺码32、34、36、38,实际上32相当于国际尺码的70码、34相当于国际尺码的75码,依次类推。

胸上围与胸下围的差值决定罩杯的级数:国际通用罩杯尺码A、B、C、D、E、F,差值误差为2.5cm。例如你的胸上围减胸下围的尺寸在12.5--15之间,那么你就适合穿着C杯的文胸。最终决定你理想的文胸尺码是75C。

说明:

①背部及腋下脂肪偏多女士:量出上下围之差为某个杯级时,罩杯的级数应该选大一级(例如差值为C级时通常选择D)。

②乳房的形状扁平得像一盘散沙:量出的差值很小,所选杯级却大(例如按照科学穿着手法将散失的脂肪拨拢成形的时候,量出的差值为B杯,那么所选杯为C杯)。

③乳房的形状下垂得像布袋一样:量出的差值越大所选杯级却相对小。

选择正确的内衣尺码,尽情展现完美身材。

胸围的尺码对照表如下:

\begin{figure}[H]
	\centering
	\includegraphics[width=0.7\linewidth]{ny_032}
	\caption{尺码表}
	\label{fig:ny_032}
\end{figure}

TIPS

一般胸围的下围尺寸容许差为±2.5cm,如下围为75cm,适应范围在75.5cm到77.5cm之间。

内裤尺码的概念。

日常穿的内裤如果尺寸过大,对身体就没有支撑作用;过小的话,则不利于身体发育。内裤的功能在于包容和承托臀部与腹部的肌肉,使它们不会由于年龄的增长产生松弛下垂的现象。臀肌下坠会使大腿变粗。合适的腰围和底边是衡量内裤生命的重要标准。当它适合你的尺寸时才是有效的,否则你就要考虑放弃!

市场上内裤的尺寸标志通常有两种,一种只标一个尺码,如:76cm或102cm,尺寸和尺码的不同在于尺寸是你量得的实际身体大小的情况,而尺码则把你的这个尺寸计算在一个范围里,用数字来标志。

请看下面的表:

\begin{figure}[H]
	\centering
	\includegraphics[width=0.7\linewidth]{ny_035}
	\caption{尺码表1}
	\label{fig:ny_035}
\end{figure}

从上面的对照表中,你可以获知,如果你的腰围是60cm,臀围是90cm,那么你要选择的内裤尺码应是:64。

另一种较体贴的内裤尺寸的标志要详细些,标有2个尺寸,如S/60或XL/85。

斜杠后的数字是所适合的腰围的尺寸。S、M、L、XL均表示臀围的大小。

\begin{figure}[H]
	\centering
	\includegraphics[width=0.7\linewidth]{ny_036}
	\caption{尺码2}
	\label{fig:ny_036}
\end{figure}

表中,S是小号;M是中号;L是大号;XL是加大号。如果你的腰围是60cm,臀围是90cm,那么你选择的内裤上会有这样的标志:M/60。

内裤的尺码选择同样不能忽略,它可是影响臀形的关键哦!

\section{文胸和内裤分类全接触}

Bra的分类

走进内衣店,眼花缭乱的款式常常让人有一种不知所措的感觉。罩杯、外形设计、功能本来是一些挺眼熟的字眼,但若认真起来,似乎还真有很多不明白的地方。不同的分类标准,就有不同的名称,做一个精致的女人,用心了解Bra的分类。

按款式分

全罩文胸:

可以将乳房全部包容于罩杯中,它覆盖面积大,包容全面,具有支撑与提升集中的效果,是最具功能性的罩杯。任何体型皆适合,特别适合乳房丰满及肉质柔软的女性。

\begin{figure}[H]
	\centering
	\includegraphics[width=0.7\linewidth]{ny_038}
	\caption{全罩文胸}
	\label{fig:ny_038}
\end{figure}

3/4罩杯文胸

上胸微露,包住乳房的3/4,它强调侧压力与集中力,前中心一般为低胸设计。这种罩杯是三款文胸中,集中效果最好的款式,如果你想让乳沟明显地显现出来,那你一定要选择3/4罩杯的文胸来凸显乳房的曲线。这款文胸任何体型皆适合。

\begin{figure}[H]
	\centering
	\includegraphics[width=0.7\linewidth]{ny_039}
	\caption{3/4罩杯文胸}
	\label{fig:ny_039}
\end{figure}

1/2罩杯文胸

露出乳房差不多一半的为1/2罩杯文胸,由于前幅边不受束缚,前中心位和侧位一般比较高,利于搭配服装,此种文胸通常可将肩带取下,成为无肩带内衣,适合露肩的衣服,机能性虽较弱,但提升的效果颇不错,胸部娇小者穿着后会显得较丰满。

\begin{figure}[H]
	\centering
	\includegraphics[width=0.7\linewidth]{ny_040}
	\caption{1/2罩杯文胸}
	\label{fig:ny_040}
\end{figure}

按花色分

光面净色文胸:

搭配轻薄的紧身外衣时,不用担心遭遇透色、露痕的尴尬。

\begin{figure}[H]
	\centering
	\includegraphics[width=0.7\linewidth]{ny_041}
	\caption{光面净色文胸}
	\label{fig:ny_041}
\end{figure}

印花光面文胸:

时尚、优雅,搭配紧身外衣无露痕。

\begin{figure}[H]
	\centering
	\includegraphics[width=0.7\linewidth]{ny_042}
	\caption{印花光面文胸}
	\label{fig:ny_042}
\end{figure}

蕾丝文胸:

外观上有明显的蕾丝用料,高贵、时尚、性感,尽显女性的万种风情。

\begin{figure}[H]
	\centering
	\includegraphics[width=0.7\linewidth]{ny_043}
	\caption{蕾丝文胸}
	\label{fig:ny_043}
\end{figure}

内衣与服饰的搭配,是女性魅力的必修课。

按特殊功能分

无肩带文胸:

大多以钢圈来支撑胸部,便于搭配露肩及宽领性感的服饰。

\begin{figure}[H]
	\centering
	\includegraphics[width=0.7\linewidth]{ny_045}
	\caption{无肩带文胸}
	\label{fig:ny_045}
\end{figure}

前扣文胸:

钩扣安于前方的文胸,一般便于穿着,也具有集中效果。

\begin{figure}[H]
	\centering
	\includegraphics[width=0.7\linewidth]{ny_046}
	\caption{前扣文胸}
	\label{fig:ny_046}
\end{figure}

魔术文胸:

在罩杯侧装入衬垫,借以提升并托高胸部,可表现胸型及深凹的乳沟。

\begin{figure}[H]
	\centering
	\includegraphics[width=0.7\linewidth]{ny_047}
	\caption{魔术文胸}
	\label{fig:ny_047}
\end{figure}

无痕文胸:

罩杯表面经无痕处理,缝入厚的绵垫,胸下围也是无痕处理,适合搭配紧身服饰。

\begin{figure}[H]
	\centering
	\includegraphics[width=0.7\linewidth]{ny_048}
	\caption{无痕文胸}
	\label{fig:ny_048}
\end{figure}

运动型文胸:

有一定的振动支撑力量,具有防震的作用,在运动时较一般文胸具有固定胸型和保护乳房功能,可以避免在运动时受伤。

\begin{figure}[H]
	\centering
	\includegraphics[width=0.7\linewidth]{ny_049}
	\caption{运动型文胸}
	\label{fig:ny_049}
\end{figure}

哺乳型文胸:

于文胸各乳房前方有开口,不论是半开罩杯型,侧开罩杯型或卸下罩杯型,皆是为了哺育宝宝而专门设计的。

\begin{figure}[H]
	\centering
	\includegraphics[width=0.7\linewidth]{ny_050}
	\caption{哺乳型文胸}
	\label{fig:ny_050}
\end{figure}

内裤的分类

按腰型分类

高腰裤:\\

高度在肚脐或肚脐以上,此类款式设计较为舒适,兼有保暖效果,对臀型的维护也较好。\\

中腰裤:\\

高度在肚脐以下8cm内,是一种最常见的内裤形式。\\

低腰裤:\\

高度低于肚脐以下8cm,又称迷你裤,此类款式的内裤比较性感,一般是为了配合时令和服饰的搭配。\\

按款式分类。\\

平脚裤:\\

臀部的包容性较好,适合不同年龄穿着。\\

三角裤:\\

基本裤型,适合各种人穿着。\\

丁字裤:\\

时尚、性感、透气性好,常用于夏季,穿在身上,臀部不留痕迹。\\










\begin{figure}[H]
	\centering
	\includegraphics[width=0.7\linewidth]{ny_051}
	\caption{文胸分类}
	\label{fig:ny_051}
\end{figure}

\begin{figure}[H]
	\centering
	\includegraphics[width=0.7\linewidth]{ny_052}
	\caption{文胸分类}
	\label{fig:ny_052}
\end{figure}

\begin{figure}[H]
	\centering
	\includegraphics[width=0.7\linewidth]{ny_053}
	\caption{文胸分类}
	\label{fig:ny_053}
\end{figure}

\clearpage

\begin{figure}[H]
	\centering
	\includegraphics[width=0.7\linewidth]{ny_054}
	\caption{文胸分类}
	\label{fig:ny_054}
\end{figure}

\clearpage

\begin{figure}[H]
	\centering
	\includegraphics[width=0.7\linewidth]{ny_055}
	\caption{文胸分类}
	\label{fig:ny_055}
\end{figure}

\begin{figure}[H]
	\centering
	\includegraphics[width=0.7\linewidth]{ny_056}
	\caption{文胸分类}
	\label{fig:ny_056}
\end{figure}

\clearpage

\begin{figure}[H]
	\centering
	\includegraphics[width=0.7\linewidth]{ny_057}
	\caption{文胸分类}
	\label{fig:ny_057}
\end{figure}

\part{挑选内衣全攻略}

\chapter{内衣选购有学问}

\section{做美丽女人从内衣选择开始}
内衣作为女性的密友,时时追随着我们,关爱着我们。\\
拥有健美的胸部,是每一个女人的理想,不同的内衣可以让女人展现不同的风情:妩媚、娇柔、纯真、简约,合适的内衣不但可以保证乳房的健康,还可以帮助你塑造美丽的胸部曲线。\\

选择使用自然材质的内衣,自然棉加上有弹性的氨纶,能增添穿着的舒适度。\\
至于美丽的蕾丝及化学纤维制品,有些人会因它们而过敏或觉得不舒服,从舒适的角度考虑最好避免这些情况发生。\\
有一定知名度的内衣,一般在这些方面很注意,而普通品牌常为以美丽的外观吸引女性而忽视材质的要素,在选购时不妨考虑这方面因素。\\

钢圈。\\

由于钢圈式文胸能够将整个胸部往上托住,因此能够缔造出美丽的胸型。\\
然而,在穿上钢圈式文胸的时候,只要曾经发生疼痛或是感到不好受的事,似乎就会对这种款式产生排斥。\\

不过,现在已经出现了即使安装有钢圈,也不会造成穿着疼痛的文胸。\\
另外,在挑选钢圈式文胸的时候有三个要诀:挑选钢圈设计在文胸外侧的内衣;挑选钢圈被海绵垫包住的文胸;挑选钢圈材质较软的文胸。\\

亲棉。\\

女性可能是因习惯所致,在选购内衣时都希望能有亲棉,来使胸部感觉丰满。\\
厚亲棉容易造成闷热,甚至还有卫生问题。\\
如果以舒适度为最佳考虑的话,最好选择只有薄薄一层的亲棉内衣。\\
这样不但有型,而且穿着也很舒服。\\

肩带。\\

一般人认为肩带松紧只和调节环有直接关系,其实不然,它和内衣整体版型的关系尤为重要。\\
有时我们会发现一件新买的内衣,调节好的肩带为什么还会滑落?其实问题就出在款式设计上,可能此款不适合你,也可能设计本身就不合理。\\

\section{挑选适宜的舒适内裤}
臀部丰满本来是母性的象征,但是很多女性却不这样认为,她们拼命穿着一些绷紧短小的内裤,然后套上一件时髦的外衣,觉得如此打扮会在视觉上产生纤细的效果。\\

\backmatter

\end{document}
