% 防灾应急手册
% 使用xelatex编译

\documentclass[12pt,a4paper,twoside]{ctexbook}

% 页面设置
\usepackage{geometry}
\geometry{a4paper, top=2.5cm, bottom=2.5cm, left=3cm, right=3cm}

% 字体设置
\usepackage{xeCJK}
\usepackage{fontspec}
\usepackage{microtype}

% 设置中文字体
\setCJKmainfont{SimSun}[  % 正文宋体
    BoldFont=SimHei,        % 粗体黑体
    ItalicFont=KaiTi        % 斜体楷体
]
\setCJKsansfont{SimHei}    % 无衬线字体黑体
\setCJKmonofont{SimSun}    % 等宽字体宋体

% 标题格式设置
\ctexset{
    part/name={\hei 第},        % 卷名
    part/number={\hei \chinese{part}卷}, % 卷号
    chapter/name={\hei 第},     % 章名
    chapter/number={\hei \chinese{chapter}章}, % 章号
    section/name={\hei 第},     % 节名
    section/number={\hei \arabic{section}节},  % 节号
    subsection/name={\hei },    % 小节名
    subsection/number={\hei \arabic{section}.\arabic{subsection} }, % 小节号
    chapter/format={\centering\hei\zihao{1}},% 章标题格式
    section/format={\hei\zihao{3}},% 节标题格式
    subsection/format={\hei\zihao{4}}% 小节标题格式
}

% 页眉页脚设置
\usepackage{fancyhdr}
\pagestyle{fancy}
\fancyhf{}
\fancyhead[LE,RO]{\zihao{5}\thepage}
\fancyhead[LO]{\zihao{5}\leftmark}
\fancyhead[RE]{\zihao{5}\rightmark}
\renewcommand{\chaptermark}[1]{\markboth{\chaptername\ \thechapter\ #1}{}}
\renewcommand{\sectionmark}[1]{\markright{\thesection\ #1}}
\fancyfoot[C]{\zihao{5} 防灾应急手册}
\renewcommand{\headrulewidth}{0.4pt}
\renewcommand{\footrulewidth}{0pt}

% 目录设置
\usepackage{titletoc}
\titlecontents{chapter}[0pt]{\vspace{10pt}\bfseries\zihao{-4}}{\contentspush{\thecontentslabel\hspace{1em}}}{}{\titlerule*[8pt]{.}\contentspage}
\titlecontents{section}[2.5em]{\vspace{5pt}\zihao{5}}{\contentspush{\thecontentslabel\hspace{1em}}}{}{\titlerule*[8pt]{.}\contentspage}
\titlecontents{subsection}[5em]{\vspace{3pt}\zihao{5}}{\contentspush{\thecontentslabel\hspace{1em}}}{}{\titlerule*[8pt]{.}\contentspage}

% 目录深度
\setcounter{tocdepth}{2}
\setcounter{secnumdepth}{2}

% 封面信息
\title{\hei\zihao{0} 防灾应急手册}
\author{\song\zihao{2} 防灾应急手册编写组}
\date{\song\zihao{4} \today}

\begin{document}

% 封面
\begin{titlepage}
    \begin{center}
        \vspace*{6cm}
        \hei\zihao{0} 防灾应急手册
        \vspace*{3cm}
        \song\zihao{2} 防灾应急手册编写组
        \vspace*{3cm}
        \song\zihao{4} \today
    \end{center}
\end{titlepage}

% 版权页
\newpage
\thispagestyle{empty}
\begin{center}
    \vspace*{8cm}
    \song\zihao{5} 版权所有\textcopyright\ 2026 防灾应急手册编写组
    \vspace*{1cm}
    \song\zihao{5} 仅供学习和参考使用
\end{center}

% 目录
\newpage
\tableofcontents

% 正文开始
\mainmatter

\chapter{防灾应急基础知识}

\section{灾难的定义与分类}
\subsection{自然灾害}
\subsection{人为灾难}
\subsection{复合灾难}

\section{防灾应急的基本原则}
\subsection{预防为主,防救结合}
\subsection{以人为本,生命至上}
\subsection{统一指挥,协同作战}
\subsection{快速反应,科学处置}

\section{应急准备的基本要素}
\subsection{应急知识与技能}
\subsection{应急物资准备}
\subsection{应急通信保障}
\subsection{应急疏散计划}

\chapter{自然灾害应急处置}

\section{地震应急处置}
\subsection{地震前兆识别}
\subsection{地震发生时的避险}
\subsection{地震发生后的自救互救}
\subsection{地震后重建与恢复}

\section{洪水应急处置}
\subsection{洪水预警与监测}
\subsection{洪水发生时的避险}
\subsection{洪水后的卫生防疫}
\subsection{洪水灾害的预防}

\section{台风与暴雨应急处置}
\subsection{台风预警信号识别}
\subsection{台风来临前的准备}
\subsection{台风期间的避险}
\subsection{暴雨引发的次生灾害防范}

\section{泥石流与滑坡应急处置}
\subsection{泥石流与滑坡的前兆}
\subsection{发生时的避险逃生}
\subsection{灾后的搜救与处置}
\subsection{地质灾害的预防措施}

\section{雷电应急处置}
\subsection{雷电的危害与防护}
\subsection{雷电天气的避险}
\subsection{雷电灾害的预防}

\section{高温与低温灾害应急处置}
\subsection{高温中暑的预防与急救}
\subsection{低温冻伤的预防与急救}
\subsection{极端温度的应对措施}

\chapter{人为灾难应急处置}

\section{火灾应急处置}
\subsection{火灾的分类与危害}
\subsection{火灾发生时的逃生}
\subsection{初期火灾的扑救}
\subsection{火灾的预防措施}

\section{燃气泄漏应急处置}
\subsection{燃气泄漏的识别}
\subsection{燃气泄漏时的应急措施}
\subsection{燃气爆炸的预防}

\section{化学品泄漏应急处置}
\subsection{化学品的危害识别}
\subsection{化学品泄漏的应急处理}
\subsection{个人防护装备的使用}

\section{交通事故应急处置}
\subsection{道路交通事故应急}
\subsection{铁路交通事故应急}
\subsection{水上交通事故应急}
\subsection{航空事故应急}

\section{公共卫生事件应急处置}
\subsection{传染病疫情应急}
\subsection{食物中毒应急}
\subsection{职业中毒应急}
\subsection{突发公共卫生事件的预防}

\chapter{家庭与社区防灾}

\section{家庭防灾准备}
\subsection{家庭应急物资清单}

家庭应急包是应对各种灾难的重要准备,应根据家庭人数和具体需求进行定制。以下是一个全面的应急包物品清单:

\begin{itemize}
    \item \textbf{一、生存基本需求}
    \begin{itemize}
        \item 饮用水:每人每天至少3升,储备3-7天的量
        \item 应急食品:高能量、易储存的食品,如压缩饼干、罐头食品、能量棒、坚果等,储备3-7天的量
        \item 餐具:一次性餐具、多功能刀、开罐器、防水火柴或打火机
        \item 衣物:每人至少两套换洗衣物,包括内衣、袜子、外套、帽子、手套等
        \item 保暖用品:毛毯、睡袋、保暖内衣、热水袋等
        \item 个人卫生用品:毛巾、牙刷、牙膏、香皂、卫生纸、女性卫生用品等
    \end{itemize}

    \item \textbf{二、医疗急救用品}
    \begin{itemize}
        \item 急救包:消毒棉球、纱布、绷带、创可贴、止血带、剪刀、镊子等
        \item 常用药品:退烧药、止痛药、消炎药、腹泻药、抗过敏药、晕车药等
        \item 特殊药品:根据家庭成员的健康状况,准备所需的特殊药品,如高血压药、糖尿病药等
        \item 急救手册:详细的急救知识和操作指南
        \item 医用口罩、手套、消毒水等防疫用品
    \end{itemize}

    \item \textbf{三、应急工具与设备}
    \begin{itemize}
        \item 手电筒:最好是手摇式或太阳能手电筒,配备充足的电池
        \item 应急灯:可充电的应急灯,用于长时间照明
        \item 多功能工具:瑞士军刀、多功能钳等
        \item 绳索:至少10米长的结实绳索
        \item 灭火器:小型家用灭火器
        \item 防护眼镜、手套、安全帽等安全防护装备
        \item 防水布、帐篷等临时 shelter 用品
    \end{itemize}

    \item \textbf{四、通信与照明}
    \begin{itemize}
        \item 手机:确保充电,并准备备用电池或移动电源
        \item 收音机:手摇式或太阳能收音机,用于接收应急广播
        \item 充电设备:移动电源、充电器、太阳能充电板等
        \item 信号灯:荧光棒、信号镜、反光背心等,用于求救
    \end{itemize}

    \item \textbf{五、文档与重要物品}
    \begin{itemize}
        \item 身份证明:身份证、护照、户口本等
        \item 医疗记录:病历、医保卡、疫苗接种证明等
        \item 财务文档:银行卡、存折、保险单、房产证等的复印件
        \item 应急联系信息:家庭成员、亲戚朋友、医疗机构、应急部门的联系方式
        \item 现金:适量的现金,以备不时之需
    \end{itemize}

    \item \textbf{六、特殊人群需求}
    \begin{itemize}
        \item 儿童用品:奶粉、奶瓶、尿布、玩具、儿童药品等
        \item 老年人用品:拐杖、轮椅、特殊护理用品等
        \item 残疾人用品:辅助器具、特殊药品等
        \item 宠物用品:宠物食品、水、牵引绳、宠物药品等
    \end{itemize}

    \item \textbf{七、其他应急用品}
    \begin{itemize}
        \item 地图:本地地图、避难场所分布图等
        \item 应急计划:家庭应急疏散计划、联系清单等
        \item 多功能哨子:用于求救和信号传递
        \item 防水袋:用于保护重要物品和电子设备
        \item 笔和笔记本:用于记录信息和需求
    \end{itemize}
\end{itemize}

应急包应定期检查和更新,确保食品、药品等物品在保质期内,电池等设备能正常工作。同时,应将应急包存放在容易取用的位置,确保家庭成员都知道应急包的存放位置和使用方法。
\subsection{家庭应急疏散计划}
\subsection{家庭火灾防范}
\subsection{家庭地震安全}

\section{社区防灾组织与管理}
\subsection{社区防灾组织架构}
\subsection{社区应急演练}
\subsection{社区避难场所建设}
\subsection{社区防灾宣传与教育}

\chapter{学校与工作场所防灾}

\section{学校防灾应急}
\subsection{学校防灾教育}
\subsection{学校应急演练}
\subsection{学校避难场所设置}
\subsection{校园安全管理}

\section{工作场所防灾应急}
\subsection{企业防灾管理}
\subsection{工作场所应急演练}
\subsection{职业安全防护}
\subsection{突发事故应急处置}

\chapter{应急救援与医疗救护}

\section{应急救援基本知识}
\subsection{救援队伍与装备}
\subsection{救援原则与方法}
\subsection{现场指挥与协调}

\section{医疗救护基础}
\subsection{心肺复苏术}
\subsection{创伤止血与包扎}
\subsection{骨折固定与搬运}
\subsection{常见急症的急救}

\section{心理应急与危机干预}
\subsection{灾难心理反应}
\subsection{心理急救方法}
\subsection{长期心理康复}

\chapter{防灾应急法律与政策}

\section{防灾应急相关法律法规}
\subsection{国家防灾减灾法律法规}
\subsection{地方防灾减灾规定}
\subsection{应急预案体系}

\section{防灾减灾政策与规划}
\subsection{国家防灾减灾规划}
\subsection{地方防灾减灾政策}
\subsection{国际防灾减灾合作}

\chapter{防灾技术与创新}

\section{防灾技术发展}
\subsection{灾害监测技术}
\subsection{预警预报系统}
\subsection{应急通信技术}

\section{防灾创新应用}
\subsection{大数据与人工智能在防灾中的应用}
\subsection{无人机在灾害救援中的应用}
\subsection{区块链在防灾物资管理中的应用}

\end{document}