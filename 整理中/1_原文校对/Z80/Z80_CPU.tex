\documentclass{book}
\usepackage[UTF8]{ctex}
\usepackage{geometry}
\usepackage{fancyhdr}
\usepackage{titlesec}
\usepackage{tocloft}
\usepackage{indentfirst}

% 页面设置
\geometry{a4paper, top=1.5in, bottom=1.5in, left=1.5in, right=1.5in}

% 页眉页脚设置
\pagestyle{fancy}
\fancyhf{}
\fancyhead[LE,RO]{\thepage}
\fancyhead[RE,LO]{Z80 CPU 技术详解}
\fancyfoot[CE,CO]{Z80 CPU 技术详解}

% 章节标题格式
\titleformat{\chapter}[display]
  {\normalfont\huge\bfseries\center}%
  {第 \Roman{chapter} 章}%
  {20pt}%
  {\Huge}
\titlespacing*{\chapter}{0pt}{50pt}{40pt}

\titleformat{\section}
  {\normalfont\Large\bfseries}%
  {\thesection.}%
  {1em}%
  {}
\titlespacing*{\section}{0pt}{30pt}{20pt}

\titleformat{\subsection}
  {\normalfont\large\bfseries}%
  {\thesubsection.}%
  {1em}%
  {}
\titlespacing*{\subsection}{0pt}{20pt}{10pt}

% 目录设置
\renewcommand{\contentsname}{目录}
\renewcommand{\cftchapfont}{\bfseries}
\renewcommand{\cftchapnumwidth}{2em}
\renewcommand{\cftchappresnum}{第 \Roman{chapter} 章:}
\renewcommand{\cftchapaftersnumb}{\hfill}
\renewcommand{\cftsecleader}{\cftdotfill{\cftdotsep}}

% 标题信息
\title{Z80 CPU 技术详解}
\author{}
\date{}

\begin{document}

% 前页
\frontmatter

\maketitle

\tableofcontents

\mainmatter

\chapter{Z80 CPU 概述}

\section{Z80 的历史背景}
\begin{itemize}
    \item Z80 的开发背景
    \item 与 Intel 8080 的关系
    \item Z80 的主要应用领域
\end{itemize}

\subsection{Z80 的基本特性}
\begin{itemize}
    \item 时钟频率范围
    \item 数据总线和地址总线宽度
    \item 指令集规模
    \item 寄存器结构
\end{itemize}

\chapter{Z80 CPU 内部结构}

\section{寄存器组}
\begin{itemize}
    \item 通用寄存器(A, B, C, D, E, H, L)
    \item 索引寄存器(IX, IY)
    \item 栈指针(SP)
    \item 程序计数器(PC)
    \item 标志寄存器(F)
    \item 特殊寄存器(I, R, IM)
\end{itemize}

\subsection{算术逻辑单元(ALU)}
\begin{itemize}
    \item 支持的运算类型
    \item 位操作功能
\end{itemize}

\subsection{控制单元}
\begin{itemize}
    \item 指令解码
    \item 时序控制
\end{itemize}

\subsection{内存和 I/O 接口}
\begin{itemize}
    \item 内存寻址能力
    \item I/O 寻址方式
\end{itemize}

\chapter{Z80 指令系统}

\section{指令格式}
\begin{itemize}
    \item 单字节指令
    \item 双字节指令
    \item 三字节指令
\end{itemize}

\subsection{指令类型}
\begin{itemize}
    \item 数据传送指令
    \item 算术运算指令
    \item 逻辑运算指令
    \item 位操作指令
    \item 控制转移指令
    \item 栈操作指令
    \item 输入输出指令
\end{itemize}

\subsection{寻址方式}
\begin{itemize}
    \item 立即寻址
    \item 直接寻址
    \item 寄存器寻址
    \item 寄存器间接寻址
    \item 变址寻址
    \item 相对寻址
\end{itemize}

\chapter{Z80 汇编语言}

\section{汇编语言基础}
\begin{itemize}
    \item 指令格式
    \item 伪指令
    \item 宏指令
\end{itemize}

\subsection{编程示例}
\begin{itemize}
    \item 简单程序结构
    \item 数据处理示例
    \item 控制结构示例
\end{itemize}

\chapter{Z80 系统设计}

\section{内存接口}
\begin{itemize}
    \item 内存映射
    \item 内存扩展
\end{itemize}

\subsection{I/O 接口}
\begin{itemize}
    \item 并行 I/O
    \item 串行 I/O
\end{itemize}

\subsection{中断系统}
\begin{itemize}
    \item 中断类型(IM 0, IM 1, IM 2)
    \item 中断处理流程
\end{itemize}

\subsection{时钟和时序}
\begin{itemize}
    \item 时钟信号
    \item 机器周期
    \item 指令周期
\end{itemize}

\chapter{Z80 应用实例}

\section{经典计算机系统}
\begin{itemize}
    \item ZX Spectrum
    \item MSX
    \item TRS-80
\end{itemize}

\subsection{嵌入式系统应用}
\begin{itemize}
    \item 工业控制
    \item 家用电子设备
\end{itemize}

\chapter{Z80 开发工具}

\section{汇编器和编译器}
\begin{itemize}
    \item Z80ASM
    \item SDCC
\end{itemize}

\subsection{仿真器}
\begin{itemize}
    \item EightyOne
    \item MAME
\end{itemize}

\subsection{开发板}
\begin{itemize}
    \item Z80 开发套件
\end{itemize}

\chapter{Z80 与现代处理器的比较}

\section{架构差异}
\subsection{性能比较}
\subsection{学习价值}

\chapter{进阶学习}

\section{Z80 高级编程技术}
\begin{itemize}
    \item 中断驱动编程
    \item 内存管理
\end{itemize}

\subsection{Z80 硬件设计}

\subsubsection{最小系统设计}

Z80最小系统是指能够使Z80 CPU正常工作的最基本硬件配置,它包含以下核心组件:

\paragraph{核心组件}

\begin{itemize}
    \item \textbf{Z80 CPU}:系统的核心,负责执行指令和控制整个系统
    \item \textbf{时钟电路}:提供CPU工作所需的时钟信号
    \item \textbf{复位电路}:确保CPU在上电或复位时能正确初始化
    \item \textbf{内存}:包括ROM(存放引导程序和固定程序)和RAM(用于数据存储和程序运行)
    \item \textbf{电源电路}:为系统提供稳定的电源供应
    \item \textbf{总线缓冲/驱动}:增强总线驱动能力(可选,取决于系统规模)
\end{itemize}

\paragraph{电路设计细节}

\subparagraph{时钟电路设计}

Z80 CPU需要一个稳定的时钟信号才能正常工作,时钟频率通常在1MHz到20MHz之间(取决于具体的Z80型号)。时钟电路可以通过以下方式实现:

\begin{itemize}
    \item \textbf{晶振+反相器}:使用石英晶体振荡器和反相器(如74LS04)生成时钟信号
    \item \textbf{时钟模块}:使用现成的时钟模块直接提供所需频率的时钟信号
    \item \textbf{555定时器}:对于低频应用,可以使用555定时器搭建时钟电路
\end{itemize}

时钟电路的设计需要注意:
\begin{itemize}
    \item 时钟信号的占空比应接近50%,以确保CPU的稳定工作
    \item 时钟信号应尽量减少噪声干扰,可通过添加滤波电容实现
    \item 对于高频应用,应考虑PCB布局的电磁兼容性问题
\end{itemize}

\subparagraph{复位电路设计}

Z80 CPU的\overline{RESET}引脚需要一个至少200ns的低电平脉冲来完成复位。复位电路的设计需要确保:

\begin{itemize}
    \item 在上电时能自动产生复位信号
    \item 允许手动复位
    \item 复位信号的持续时间足够长
\end{itemize}

常用的复位电路设计:
\begin{itemize}
    \item \textbf{RC复位电路}:使用电阻和电容构成的充放电电路,在上电时产生复位信号
    \item \textbf{专用复位芯片}:如MAX813L,提供更可靠的复位功能和看门狗功能
\end{itemize}

\subparagraph{内存设计}

最小系统的内存设计通常包括:

\begin{itemize}
    \item \textbf{ROM}:通常使用EPROM或EEPROM,容量为2KB到32KB不等,用于存放引导程序和固定程序
    \item \textbf{RAM}:通常使用SRAM,容量为1KB到64KB不等,用于数据存储和程序运行
\end{itemize}

内存地址分配示例:
\begin{itemize}
    \item 0x0000-0x07FF:8KB ROM(如2764)
    \item 0x0800-0x3FFF:14KB 可选ROM扩展
    \item 0x4000-0x7FFF:16KB RAM(如62256的一部分)
    \item 0x8000-0xFFFF:32KB RAM(62256的其余部分)
\end{itemize}

\subparagraph{电源电路设计}

Z80 CPU需要+5V的直流电源,电流需求通常在100mA到500mA之间(取决于时钟频率和外设数量)。电源电路设计需要注意:

\begin{itemize}
    \item 电源的稳定性:使用稳压芯片(如7805)提供稳定的+5V电压
    \item 电源滤波:在稳压芯片的输入和输出端添加电解电容和陶瓷电容进行滤波
    \item 电源保护:添加保险丝和反向保护二极管
\end{itemize}

\paragraph{硬件连接图描述}

Z80最小系统的硬件连接主要包括:

\begin{itemize}
    \item \textbf{地址总线(A0-A15)}:连接到ROM和RAM的地址输入端
    \item \textbf{数据总线(D0-D7)}:连接到ROM、RAM和I/O设备的数据端
    \item \textbf{控制总线}:
    \begin{itemize}
        \item \overline{MREQ}:内存请求信号,连接到ROM和RAM的片选端
        \item \overline{IORQ}:I/O请求信号,连接到I/O设备
        \item \overline{RD}:读信号,连接到ROM和RAM的输出使能端
        \item \overline{WR}:写信号,连接到RAM的写使能端
        \item \overline{RESET}:复位信号,连接到复位电路
        \item \textbf{CLK}:时钟信号,连接到时钟电路
    \end{itemize}
\end{itemize}

\paragraph{启动流程}

Z80 CPU在上电复位后,程序计数器(PC)会被设置为0x0000,系统开始从ROM的0x0000地址读取并执行指令。启动流程通常包括:

\begin{enumerate}
    \item CPU复位,PC = 0x0000
    \item 从ROM 0x0000地址读取第一条指令并执行
    \item 执行初始化程序(如设置堆栈指针、初始化I/O设备等)
    \item 进入主程序循环
\end{enumerate}

\paragraph{调试和测试方法}

最小系统的调试和测试可以通过以下方式进行:

\begin{itemize}
    \item \textbf{LED指示灯}:连接到数据总线或I/O端口,通过LED的闪烁判断CPU是否正常工作
    \item \textbf{串口调试}:添加串口模块(如MAX232),通过串口输出调试信息
    \item \textbf{逻辑分析仪}:使用逻辑分析仪观察总线信号,分析CPU的工作状态
    \item \textbf{仿真器}:使用Z80仿真器(如Z80 Emulator)进行程序调试
\end{itemize}

\paragraph{示例引导程序}

以下是一个简单的Z80引导程序示例,用于验证最小系统的基本功能:

\begin{verbatim}
; Z80 最小系统引导程序
; 功能:使连接到端口0的LED闪烁

    ORG 0000H       ; 程序起始地址

START:
    LD SP, 0FFFFH   ; 设置堆栈指针
    LD A, 80H       ; 初始化A寄存器
    LD B, 20H       ; 初始化B寄存器

LOOP:
    OUT (0), A      ; 向端口0输出A寄存器的值
    CALL DELAY      ; 调用延迟子程序
    XOR B           ; 翻转A寄存器的位
    JP LOOP         ; 循环

DELAY:
    LD C, 0FFH      ; 延迟计数1
DELAY1:
    LD D, 0FFH      ; 延迟计数2
DELAY2:
    DEC D
    JP NZ, DELAY2
    DEC C
    JP NZ, DELAY1
    RET             ; 返回主程序

    END
\end{verbatim}

这个引导程序的功能:
\begin{itemize}
    \item 设置堆栈指针
    \item 通过翻转寄存器A的值,使连接到端口0的LED灯闪烁
    \item 使用嵌套循环实现延迟功能
\end{itemize}

\paragraph{故障排除指南}

在构建Z80最小系统时,可能会遇到以下常见问题:

\begin{enumerate}
    \item \textbf{CPU不工作}
          \begin{itemize}
              \item 检查电源电压是否为+5V,极性是否正确
              \item 检查时钟电路是否正常工作,时钟信号是否正确
              \item 检查复位电路是否正常,复位信号是否符合要求
              \item 检查CPU的引脚连接是否正确
          \end{itemize}
    \item \textbf{内存访问错误}
          \begin{itemize}
              \item 检查ROM和RAM的地址线连接是否正确
              \item 检查片选信号(\overline{CS})是否正确连接
              \item 检查读写信号(\overline{RD}/\overline{WR})是否正确连接
              \item 验证ROM中的程序是否正确烧录
          \end{itemize}
    \item \textbf{I/O设备不响应}
          \begin{itemize}
              \item 检查I/O请求信号(\overline{IORQ})是否正确连接
              \item 检查I/O端口地址是否正确
              \item 检查I/O设备的电源和地线连接
          \end{itemize}
    \item \textbf{系统不稳定}
          \begin{itemize}
              \item 检查电源滤波电容是否足够
              \item 检查时钟信号是否有噪声
              \item 检查PCB布局是否合理,避免信号线过长
              \item 检查各芯片的散热情况
          \end{itemize}
\end{enumerate}

\subsubsection{总线扩展}

当最小系统需要增加更多的内存、I/O设备或其他外设时,就需要对Z80的总线进行扩展。总线扩展的主要目标是在保持系统稳定性的同时,增加系统的功能和可扩展性。

\paragraph{总线扩展的必要性}

最小系统通常只能满足基本需求,当需要以下功能时,就需要进行总线扩展:

\begin{itemize}
    \item 增加内存容量(超过最小系统的16KB或32KB)
    \item 添加多种I/O设备(如键盘、显示器、串口等)
    \item 连接外部扩展卡或模块
    \item 实现多处理器系统
    \item 提高总线的驱动能力,支持更多设备
\end{itemize}

\paragraph{总线扩展的基本原理}

Z80 CPU的总线包括:
\begin{itemize}
    \item \textbf{地址总线(A0-A15)}:16位,可寻址64KB内存空间
    \item \textbf{数据总线(D0-D7)}:8位,用于数据传输
    \item \textbf{控制总线}:包括内存请求(\overline{MREQ})、I/O请求(\overline{IORQ})、读写信号(\overline{RD}/\overline{WR})等
\end{itemize}

总线扩展的基本原理是通过总线缓冲器、地址译码器等电路,将CPU的总线信号扩展到更多的设备上,同时确保信号的完整性和稳定性。

\paragraph{总线缓冲与驱动}

Z80 CPU的总线驱动能力有限,通常只能直接驱动几个TTL设备。当连接更多设备时,需要使用总线缓冲器或驱动器来增强驱动能力。

常用的总线缓冲/驱动IC:
\begin{itemize}
    \item \textbf{74LS244}:8位三态缓冲器,用于单向总线(如地址总线)
    \item \textbf{74LS245}:8位三态双向总线收发器,用于双向总线(如数据总线)
    \item \textbf{74LS373}:8位锁存器,用于地址总线锁存
    \item \textbf{74LS374}:8位D触发器,用于数据总线锁存
\end{itemize}

缓冲器的使用需要注意:
\begin{itemize}
    \item 地址总线通常使用单向缓冲器或锁存器
    \item 数据总线必须使用双向总线收发器
    \item 控制总线需要根据信号方向选择合适的缓冲器
    \item 缓冲器的使能信号必须正确连接,避免总线冲突
\end{itemize}

\paragraph{地址译码}

地址译码是总线扩展的关键技术,用于将CPU的16位地址总线映射到不同的设备或内存区域。常见的地址译码方法包括:

\begin{enumerate}
    \item \textbf{线选法}
          \begin{itemize}
              \item 简单直接,使用高位地址线直接作为设备的片选信号
              \item 适用于设备数量较少的系统
              \item 地址空间利用率低,存在地址重叠
          \end{itemize}
    \item \textbf{全译码法}
          \begin{itemize}
              \item 使用所有高位地址线进行译码
              \item 地址空间利用充分,无地址重叠
              \item 电路复杂,需要较多的译码门电路
          \end{itemize}
    \item \textbf{部分译码法}
          \begin{itemize}
              \item 使用部分高位地址线进行译码
              \item 兼顾了线选法的简单性和全译码法的地址空间利用率
              \item 仍存在一定的地址重叠
          \end{itemize}
    \item \textbf{译码器IC法}
          \begin{itemize}
              \item 使用专用的译码器IC(如74LS138、74LS139等)
              \item 电路简单,可靠性高
              \item 适用于各种规模的系统
          \end{itemize}
\end{enumerate}

常用的地址译码器IC:
\begin{itemize}
    \item \textbf{74LS138}:3-8线译码器,可产生8个片选信号
    \item \textbf{74LS139}:双2-4线译码器,可产生8个片选信号
    \item \textbf{74LS154}:4-16线译码器,可产生16个片选信号
    \item \textbf{PAL/GAL}:可编程逻辑器件,可实现复杂的地址译码逻辑
\end{itemize}

\paragraph{内存扩展}

Z80 CPU的地址总线为16位,理论上可寻址64KB内存空间。内存扩展通常包括:

\begin{enumerate}
    \item \textbf{ROM扩展}
          \begin{itemize}
              \item 使用更大容量的ROM芯片(如27128、27256等)
              \item 通过地址译码器将ROM映射到合适的地址空间
              \item 常用于存放更大的程序或多个程序
          \end{itemize}
    \item \textbf{RAM扩展}
          \begin{itemize}
              \item 使用更大容量的RAM芯片(如62256、6264等)
              \item 注意RAM的读写控制和片选信号的连接
              \item 常用于数据存储和程序运行
          \end{itemize}
    \item \textbf{内存分页}
          \begin{itemize}
              \item 当需要超过64KB的内存时,可采用内存分页技术
              \item 通过页面寄存器或MMU(内存管理单元)实现
              \item 常见的分页方式包括 bank switching 等
          \end{itemize}
\end{enumerate}

内存扩展的地址映射示例:
\begin{itemize}
    \item 0x0000-0x7FFF:32KB ROM(27256)
    \item 0x8000-0xBFFF:16KB RAM(6264)
    \item 0xC000-0xFFFF:16KB RAM(6264)
\end{itemize}

\paragraph{I/O扩展}

Z80 CPU支持64KB的I/O空间(0x0000-0xFFFF),但实际上通常只使用低8位地址(0x00-0xFF),可寻址256个I/O端口。I/O扩展主要包括:

\begin{enumerate}
    \item \textbf{并行I/O扩展}
          \begin{itemize}
              \item 使用并行I/O芯片(如8255A、6821等)
              \item 可扩展多个并行I/O端口
              \item 常用于连接键盘、显示器、打印机等
          \end{itemize}
    \item \textbf{串行I/O扩展}
          \begin{itemize}
              \item 使用串行通信芯片(如8251A、6850等)
              \item 实现RS-232、RS-485等串行通信
              \item 常用于与其他设备或计算机通信
          \end{itemize}
    \item \textbf{定时/计数扩展}
          \begin{itemize}
              \item 使用定时/计数芯片(如8253、6840等)
              \item 实现定时、计数、脉冲生成等功能
              \item 常用于精确控制和测量
          \end{itemize}
    \item \textbf{中断控制器扩展}
          \begin{itemize}
              \item 使用中断控制器芯片(如8259A等)
              \item 扩展中断源,实现优先级管理
              \item 常用于多中断源系统
          \end{itemize}
\end{enumerate}

\paragraph{总线扩展的时序考虑}

总线扩展时需要注意时序问题,确保所有设备都能在正确的时间内响应总线操作。主要的时序参数包括:

\begin{itemize}
    \item \textbf{访问时间}:CPU从发出地址到读取数据所需的时间
    \item \textbf{建立时间}:地址和控制信号需要在数据有效前保持稳定的时间
    \item \textbf{保持时间}:数据需要在地址和控制信号无效后保持稳定的时间
    \item \textbf{总线周期}:完成一次总线操作所需的时钟周期数
\end{itemize}

对于慢速设备,可以通过插入等待状态(WAIT)来延长总线周期,确保设备有足够的时间响应。

\paragraph{总线扩展的示例电路}

以下是一个简单的Z80总线扩展示例,用于扩展内存和I/O:

\begin{enumerate}
    \item \textbf{内存扩展}
          \begin{itemize}
              \item 使用74LS138译码器将32KB ROM(0x0000-0x7FFF)和32KB RAM(0x8000-0xFFFF)映射到地址空间
              \item 使用74LS245作为数据总线缓冲器
              \item 使用74LS244作为地址总线缓冲器
          \end{itemize}
    \item \textbf{I/O扩展}
          \begin{itemize}
              \item 使用8255A并行I/O芯片扩展3个8位并行端口
              \item 使用74LS139译码器将8255A映射到端口地址0x80-0x83
              \item 8255A的PA口连接LED显示,PB口连接键盘,PC口作为控制端口
          \end{itemize}
\end{enumerate}

\paragraph{总线扩展的注意事项}

在进行总线扩展时,需要注意以下事项:

\begin{itemize}
    \item \textbf{总线冲突}:确保同一时间只有一个设备在总线上发送数据
    \item \textbf{信号完整性}:使用合适的缓冲器和驱动器,确保信号的完整性
    \item \textbf{电磁兼容性}:注意PCB布局和布线,减少电磁干扰
    \item \textbf{时序匹配}:确保所有设备的时序参数匹配,避免时序冲突
    \item \textbf{电源稳定性}:提供稳定的电源,特别是对于扩展的设备
    \item \textbf{接地处理}:良好的接地系统可以减少噪声和干扰
    \item \textbf{可维护性}:设计时考虑系统的可维护性和可扩展性
\end{itemize}

通过合理的总线扩展设计,可以构建功能丰富、性能稳定的Z80系统,满足各种应用需求。

\chapter{资源推荐}

\section{书籍}
\subsection{网站和论坛}
\subsection{开源项目}

\vspace{1cm}
\hrule
\vspace{0.5cm}
\begin{center}
    \textit{注:此提纲仅作为学习 Z80 CPU 的框架,具体内容将在后续逐步补充。}
\end{center}

\end{document}