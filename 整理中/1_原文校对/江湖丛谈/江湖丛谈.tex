% 江湖丛谈
% 江湖丛谈.tex

\documentclass[12pt,UTF8]{ctexbook}

% 设置纸张信息。
\usepackage[a4paper,twoside]{geometry}
\geometry{
	left=25mm,
	right=25mm,
	bottom=25.4mm,
	bindingoffset=10mm
}

% 设置字体,并解决显示难检字问题。
\xeCJKsetup{AutoFallBack=true}
\setCJKmainfont{SimSun}[BoldFont=SimHei, ItalicFont=KaiTi, FallBack=SimSun-ExtB]

% 目录 chapter 级别加点(.)。
\usepackage{titletoc}
\titlecontents{chapter}[0pt]{\vspace{3mm}\bf\addvspace{2pt}\filright}{\contentspush{\thecontentslabel\hspace{0.8em}}}{}{\titlerule*[8pt]{.}\contentspage}

% 设置 part 和 chapter 标题格式。
\ctexset{
	chapter/name={第,章},
	chapter/number={\chinese{chapter}},
	section/name={},
	section/number={}
}

% 图片相关设置。
\usepackage{graphicx}
\graphicspath{{Images/}}

% 设置古文原文格式。
\newenvironment{yuanwen}{\bfseries\zihao{4}}

% 设置署名格式。
\newenvironment{shuming}{\hfill\zihao{4}}

% 注脚每页重新编号,避免编号过大。
\usepackage[perpage]{footmisc}

\title{\heiti\zihao{0} 江湖丛谈}
\author{连阔如}
\date{}

\begin{document}

\maketitle
\tableofcontents

\frontmatter

\chapter{前言}

连阔如(1903-1971),满族。原名毕连寿,笔名云游客。1927年进入评书界。他虚心好学、记忆力强、刻苦钻研,创立了自己独到的表演风格,人们称赞他“见识实在,胜人一筹”。20世纪30年代末,他在北京伯力威广播电台连续播讲《东汉演义》,有“千家万户听评书,净街净巷连阔如”之美誉。后又在其他电台播讲《水浒》《东汉》《陷唐》《明英烈》等书,诸书皆书情严谨,人物性格鲜明。他说书时台风潇酒,神完气足,语重声宏,口齿清晰;尤其注重细节及穿插于书情中的评点论述,淡入淡出,不着痕迹。三大书目《东汉》《三国》《水浒》,经过他认真揣摩,反复加工,成为连派评书扛鼎之作。《东汉》的打功、《三国》的评讲、《水浒》的民俗,被公认为连派评书的经典标志。1938年,他以云游客为笔名,出版了《江湖丛谈》一书。

\chapter{一本不可多得的奇书}

20世纪80年代,新近刚刚去往天国的电影艺术家谢添约我写一本关于燕子李三的电影剧本。为了了解旧中国江湖的内幕和行话,忘了从谁那里找来一本《江湖丛谈》。我一口气读完,痛快地过了一回读书瘾,那感觉很像读《儿女英雄传》。我很想知道作者“云游客”是何许人也。可惜,问了许多人,都语焉不详。一直到纪念评书大王连阔如百年诞辰时,我才知道:“云游客”乃连阔如也!这大大增加了我对连阔如先生的敬意。

《燕子李三》没有拍成,连剧本都不知流浪何方,我的那本《江湖丛谈》也不知是被同好借走还是遗失在搬家的中途,至今懊丧不止。但我却记住了《江湖丛谈》里的许多内容,及至看到一些新起的文豪卖弄并不准确的“春点”,甚至用错字音译被旧时的艺人蔑视的“臭春”,就想“狗拿耗子”告诉他们,找一本《江湖丛谈》看看,省得“谬种流传”。后来知道,这是瞎耽误工夫,便死灭了向他们推荐这本奇书的念头。

这确乎是本奇书,而且我敢断言,从今往后没有人能再写出这样的书。要写这样的书,必须具备以下的条件:

第一,身为江湖中人,而又内心纯正,所谓“出污泥而不染”,熟悉江湖内幕和行话以及一切行规。

第二,会写一手漂亮文章。所谓漂亮不是今日满篇舶来语、通篇新式名词,外加倒装句,而是通顺、通俗、生动有趣,且极具韵味,让人一看便明白这“江湖”所处时代的特色,了然当时北平报纸连载文章的风格。

第三,有拯人济世之心,无哗众取宠之意。鞭挞假恶丑,有勇气,有肝胆,有侠义之风。

这三点,除第三条之外,一、二两项今人都无法达到。自然的伟力将这本书推到20世纪30年代北平、天津江湖行当及报纸文风活化石的地位上。能写这书的人,没了,没啦!或许有今日的文人经过仔细地调查,深入地揣摩,写出颇有当时风格的文章来,这就不错。只是不会有那种身处其间的真实感,怎么着都会有今人对旧人的评断,不再是当时人的感受。所以,在某种意义上说,这书是空前绝后的。至于第三点,今人绝大多数是真善美的歌颂者,不缺肝胆、侠义,不怕江湖人报复,因为,旧日的江湖已经掩埋在岁月的灰尘里。可是,挡不住有几位就愿意歌颂假恶丑,爱显摆自己明白江湖里的道行,也写一些这类的文章。但是,这和《江湖丛谈》就有了本质的区别。《江湖丛谈》为人指点迷津,揭露诡计,要冒身家性命的危险。这勇气不是在温柔乡里长大的才子可以比拟的。

一个时代有一个时代的风气和代表人物。对于过往的俊杰我们只能仰视,因为我们无法估计我们倘在那时代生活,会有怎样的表现,我们还能不能写出或者做出像样的东西,留给后人和历史,成为认识这时代的参照。连阔如先生以一位评书艺人的身份,写出这样一本可以让后人有兴味地知道往事的奇书,本身就是奇迹。奇人奇书,值得我们好好地读一读。是为序。

\begin{shuming}
苏叔阳
\end{shuming}

\begin{shuming}
2005年6月8日于寤斋
\end{shuming}

\chapter{回忆父亲连阔如}

父亲离开人世近四十个春秋了,可是没有一天不想念他老人家,他是我的慈父,也是我的严师。

父亲是个苦命人,光绪二十九年(1903)闰五月,父亲落生在北京安定门外一个穷旗人的家中。我家是满族镶黄旗人,祖姓毕鲁氏。满族人指名为姓,我爷爷叫凌保,是个门甲,父亲出世前一个月,爷爷就故去了。父亲只上了半年私学、两年小学,十二岁就当学徒,进过北京的首饰楼、照相馆,天津的杂货铺、中药店;到烟台、大连做过小买卖;摆过卦摊,饱尝了人世间的酸、甜、苦、辣。

父亲原名毕连寿,拜师李杰恩,学说评书《西汉演义》,艺名连阔如。后又向张诚斌学说《东汉演义》。北京有一位田岚云老先生,说《东汉演义》名扬京城,听众孙昆波把田老先生书中的精华指点给我的父亲,再加上父亲的天资、勤奋,20世纪30年代末期在东交民巷伯力威电台播讲《东汉》,名声鹊起。他刻苦向前辈演员学习,博采众长,融会贯通,达到书情结构严谨,人物性格鲜明。说书时嗓音宽厚,语重声宏,口齿清晰,娓娓动听。为摹拟好文生、武将,他借鉴京剧表演艺术,融化于评书中。马跑、马嘶等口技辅助表演,被听众公认为一绝。父亲曾说,“说书时要严肃地进行表演,要做到五忘:忘己事,忘己貌,忘座有贵宾,忘身在今日,忘己之姓名”,全身投入艺术创造中。他重视说功、做功、打功,说到谁,就摹拟那个人物的神情、语言、声态,有时也使用方言、韵白,加上必要的动作,表情状物,绘声绘色,形成了神完气足、层次分明、起伏跌宕、耐人寻味的独特风格,艺术精湛已自成一家。

父亲说《东汉》的技艺,显示了他少有的艺术才华。但他并不满足,仍然精益求精。他虚心请教老师和听众,集先辈评书诸家之所长。父亲说的是“袍带”书,为了提高艺术,父亲向知名的武术家学习,又结识了许多京剧界的朋友,如萧长华、徐兰沅、郝寿臣、谭富英、李万春、马富禄,以京剧唱、念、做、打的功夫丰富自己的表演。20世纪30年代末,京剧表演艺术家尚小云先生曾邀请父亲为他的科班——荣春社排演全部《东汉》。荣春社在前门外中和戏院演出,轰动了京城。那时,父亲白天在电台说书,晚上到剧场看戏、指导。尚小云先生的长子尚长春扮演的武状元岑彭栩栩如生,是父亲说的《东汉》中的一个人物在舞台上活灵活现的再创造。新中国成立后,他协助王永昌先生排练了全部《水浒》,在大栅栏庆乐戏院演出,盛况空前。几十年来,父亲结交了各行各业的专家,成为朋友,如养马专家载涛,语言学家吴晓铃,剧作家翁偶虹、景孤血,针灸名医胡荫培,作家赵树理,史学家吴晗等。父亲就是这样广交博学,不断地使自己的艺术造诣达到更高的境地。

父亲一生勤俭度日,不吸烟,不喝酒,不讲穿戴,所挣的钱除去养家外,全都买了书刊。我家原住在和平门外琉璃厂,这是一条有名的古书街。父亲则是“邃远斋”、“来薰阁”等古书店的常客,难怪我去琉璃厂中国书店买书,好多书店的同志一眼就认出了我,津津有味地谈起我父亲当年买书的情景。我记得父亲为了考证汉献帝的“衣带诏”一事,购买和翻阅了七八种《汉书》及《三国志》的版本。他钻研天文知识,把《借东风》、《草船借箭》说得入情入理;他学习、了解山川地理、风俗人情,以备古今对照;为了评价历史人物曹操,他详细阅读了多位学者的有关著作,登门请教。听众们反映:“听连先生的书,不但听了历史故事,还学到了不少知识。”

父亲为人正直,光明磊落,不奴颜婢膝。抗日战争时期,日军责令我父亲在电台宣传“大东亚共荣”,父亲竟说了一段《廉颇·蔺相如》,意蕴人民团结抗战,结果被日伪电台斥退。

父亲离开了电台,开始写作生涯,以云游客的笔名发表了《江湖丛谈》。书中的内容是父亲身临其境掌握的第一手感性材料,对许多社会现象作了生动的写照,正如父亲所说:“以我的江湖知识说呀,所知道的不过百分之一,不知道的还多着哪。等我慢慢地探讨,得一事,向阅者报告一事,总以爱护多数人,揭穿少数人的黑幕,为大众谋利除害,以表示我老云忠于社会啊!”这部书揭露了某些危害社会的江湖行当的黑幕和手段,在当时社会上影响极大。从这部书里也看出父亲的勤奋和洞察社会的能力,我也更加了解了父亲青少年时代浪迹江湖的酸楚。

北京解放后,父亲响应党的号召,作为曲艺界的带头人,积极主动参加各项工作。1949年7月,被选为代表,参加了全国第一届文代会。在全国文联的领导下,父亲筹备成立了“中华全国曲艺改进会筹备会”,担任副主任,协助王尊三、赵树理同志工作。周恩来总理看过父亲的演出后,鼓励他搞好曲艺革新改进工作。父亲立即按照北京市文艺处的指示,组织北京的京剧、评剧、曲艺演员成立“戏曲界艺人讲习班”。为加强新曲艺的演出实践,他带领曲艺演员在前门“箭楼曲艺厅”每天演唱《新五圣朝天》、《考神婆》等新曲艺。又和新华广播电台合作,每天中午用固定时间播唱新曲目,前后坚持了三年,扩大了新曲艺的影响。在父亲的带动下,评书演员赵英颇等开始播讲《一架弹花机》、《罗汉钱》等新评书,很受欢迎。

1951年初,父亲积极响应“抗美援朝、保家卫国”的号召,受彭真同志委托,组织了“中国人民第一届赴朝慰问团曲艺服务大队”,并担任大队长,率领京、津两地许多著名曲艺演员,赴朝鲜前线,冒着枪林弹雨在前沿坑道、阵地,进行慰问演出。他经常表演评书《武松打虎》,古事今说,表达了祖国人民不惧强敌的心愿,鼓舞了指战员的斗志,使曲艺获得了“文艺尖兵”的称号。归国后,父亲又带队深入大西南去演出,宣传、推广普通话,边演出边整理,创作了《飞夺泸定桥》等新书。1953年10月,第二次全国文代会决定成立“中国曲艺研究会”,父亲被任命为副主席,和赵树理、王亚平、韩起祥一起,协助王尊三主席工作。当时是百花盛开的季节,父亲除了编演新书《强渡大渡河》、《智取娄山关》等外,还整理了《三国演义》、《东汉演义》、《水浒》等传统评书,在北京人民广播电台整年连续广播。这时,父亲的说书艺术更加精湛,每到播讲时间,家家收音机旁挤满了听众,北京市内流传着“千家万户听评书,净街净巷连阔如”的赞誉。父亲忙于社会工作,他当选为北京市人民代表大会的代表和全国政治协商会议的委员,还经常到大学去讲课。

1957年,父亲遭到了无情的打击,被错划为右派分子,他的身影从社会上消失了,他的声音从广播里消失了。但他没有灰心丧气,承受着巨大的政治压力继续编写《红军长征演义》,研究《三国演义》。父亲一直惦记着怎样实现周总理向自己提出的“要带好徒弟”、“自己的孩子有没有学评书”的嘱咐。他原来认为女孩子是不能说评书的,可是在上海却亲眼看到了王少堂的孙女王丽堂,受到了王老的言传身教,十六岁就登台说《武松打虎》。父亲想到丽堂,很受启发,决定选择难度较大的《三国演义》口授给我。那时我正在北京师大附中读高中,为了表达“北连学南王”的心情,父亲把我的名字改为连丽如,意思是:南丽继承南王评话,北丽继承北连评书,祝愿我与丽堂同志在书坛上茁壮成长。

粉碎了“四人帮”,1979年11月,有关单位为父亲在八宝山举行了彻底平反的隆重追悼会,《北京日报》予以报道。我也从工厂重返书坛。为了继承连派评书艺术,我顽强地拼搏,终于恢复了《东汉演义》、《隋唐演义》、《三国演义》、《明英烈》几部长篇大书的演出,受到广大听众的热烈欢迎。接着我为电视台录制了评书《三国演义》、《东汉演义》、《康熙私访》等,在北京和各省电视台播出,听众们给予很高的评价。

自20世纪80年代以来,在爱人贾建国的帮助下,我改编整理了二百多万字的评书手稿,并全部出版,包括《评书三国演义》、《东汉演义》、《康熙私访》、《左良传》、《程咬金大闹瓦岗寨》、《斩莽剑》、《逍遥王》等。尤为值得一提的是,父亲在1936年出版的名著《江湖丛谈》也多次再版,最新的一版就是在出版界享有盛誉的中华书局即将推出的“终极版”——注音注释典藏本,其中收录了新近从《新北平报》上发现的父亲佚文《漫话江湖 万象归春》,这是十分珍贵的曲艺资料,也是首次与广大读者见面;此外,著名漫画家李滨声老师还特地绘制了精美的彩色插图——这对于喜欢我父亲,喜欢《江湖丛谈》的朋友来说,无疑是特大的好消息!

2003年9月4日,在我的努力下,纪念父亲连阔如诞辰一百周年座谈会在湖广会馆举行,关学曾、谭元寿、常宝华、苏叔阳、刘兰芳、李金斗、孙毓敏、李燕、杜澎等艺术界老中青三代齐聚一堂,缅怀父亲的高超书艺与崇高品格。刘乃崇老师回忆了与父亲在建国初期为新中国曲艺事业共同奋斗的光辉岁月;郝寿臣先生的公子,年近九旬的郝德元老师满怀深情地讲述了父亲在抗美援朝战场上舍身救人的事迹,感动了在场的每一个人。虽然他们每个人的发言都很简短,回忆父亲的事迹也只是点点滴滴,但一代评书大师的形象却在人们的讲述中逐渐鲜活起来。我想,父亲的评书深深地影响了几代人,到今天依然有很多人怀念他,惦记他,这充分证明了他的人格魅力与艺术成就。正如欧阳中石老师为父亲题写的那首诗所说:“敷演春秋稗史,公平月旦无私。口碑评书自相宜,不负微言大义。”

在怀念父亲的同时,我也时刻不忘己任——传承和发展连派评书艺术。其一,在相关部门的帮助下,北京评书顺利成为国家非物质文化遗产项目,我也成为传承人;其二,2007年6月,我收下了四个徒弟和两个义子,北京评书后继有人;其三,将观众重新请回书馆,让他们领略和欣赏真正的书馆评书艺术。父亲您知道么——从朝阳小梨园到月明楼,再到如今每周末宣南、崇文、东城三个书馆的红红火火,从一开始只有我一个人说书,到如今吴荻、贾林、王玥波、李菁、祝兆良、梁彦,六个孩子每人都能上台说演长篇大书,这是多么令人欣慰啊!而今年又有喜事。为了传承和发展北京评书,我让李菁收了个学评书的徒弟。这个孩子叫张硕,二十四岁,很聪明,也很用功。他是连派评书艺术的第四代传人,现在我和建国、玥波爷儿仨一起带他,希望他刻苦学习,早日成才。再有,江苏泰州正在筹建中国评书评话博物馆,预计年底建成。届时,您与双厚坪、王少堂等老前辈的塑像将一同伫立于此,供后来者瞻仰凭吊,而这里也将成为评书评话艺术展示与研究的重要基地。

父亲,如果您在天有灵,看到这一切,一定会含笑九泉的!

\begin{shuming}
连丽如
\end{shuming}

\mainmatter

\chapter{江湖规矩}

\section{江湖之春点}

==========================================================
著者自幼在外奔走,自谋衣食,对于江湖中的事儿有个一知半解,所以著述这部《江湖丛谈》。

本书内有“风”、“马(má)”、“雁”、“雀”四大门(指群骗),“金”(相面算卦)、“皮”(卖药)、“彩”(变戏法)、“挂”(打把式卖艺)、“评”(说评书)、“团(tuǎn)”(说相声)、“调(diào)”(骗局)、“柳(liǔ)”(唱大鼓)八小门。内容包括的是:卖“梳篦(bì)”的、卖“刀剪”的、卖“香面”的、卖“膏药”的、卖“刀伤药”的、卖“眼药”的、卖“虫子药”的、卖“牙疼药”的、挑(tiǎo)“汉册(chǎi)子”的、卖“戏法”的、变“戏法”的、“打把式卖艺”的、“跑马戏”的、“修脚”的、算“周易卦”的、算“奇门卦”的、算“鸟儿卦”的、“相面”的、“哑相”的、“灯下术”的、说“相声”的、唱“大鼓书”的、唱“竹板书”的、说“评书”的、卖“胰子”的、卖“避瘟散”的、“拉洋片”的,等等行当,不下百数十种。

此外,尚有两门,一为“骗术门”,一为“穷家门”(唱数来宝的)。并有江湖黑幕、江湖人规矩、艺术变迁、艺人小传、艺人传流支派、艺人道义、各省艺人团体的组织、艺人的沿革。谨将内容用概括方式,先向阅者报告明了。

由江湖人之“春点”作为首谈。什么叫做“春点”呢?读书人离不开字典、字汇、《辞源》等等书籍。江湖之人不论是哪行儿,先得学会了春点,然后才能够吃生意饭儿。普通名称是“生意人”,又叫吃“张口饭”的。江湖艺人对于江湖艺人称为“老合”(合气之合)。敝人曾听艺人老前辈说过:“能给十吊钱,不把艺来传。宁给一锭金,不给一句春。”由这两句话来作证,江湖的老合们把他们各行生意的艺术看得有泰山之重。

江湖人常说,艺业不可轻传,教给人学得容易,那会不值一文半文,丢得更易。江湖艺术是不能轻传于人的,更不能滥授给他人。不惜一锭金,都舍不得一句春。据他们江湖人说,这春点只许江湖人知道,若叫外行人知道了,能把他们各行买卖毁喽,治不了“杵儿”(江湖人管挣不了钱调[diào]侃儿说治不了杵儿。注:此处“杵”字可加儿化音,也可不加)。

果子行、油行、肉行、估(gù)衣行、糖行,以及拉房纤(qiàn)的、骡马市里纤(qiàn)手,各行都有各行的术语,俗话说叫“调(diào)侃儿”。江湖艺人管他们所调的侃儿,总称叫做“春点”。今例举一事,阅者诸君便知那春点的用处。譬如,乡村里有个摇铃儿卖药的先生,正被一家请至院内看病。这卖药的先生原不知病人所患的是何病症。该病人院邻某姓是个江湖人,他要叫卖药的先生挣得下钱来,先向卖药的先生说:“果食点”(果食指已婚女子,点是人)是“攒(cuán)儿吊(攒儿是心口,吊是疼)的粘啃(nián kèn,病了)”。卖药的先生不用给病人诊脉,便能知道这家有个妇人,得的是心痛之病。原来这“果食点”,按着春点的侃语便是妇人;“攒儿吊的粘啃”便是心口疼的病症。然后卖药的先生给病人一诊脉,把病原说出来,说得很对。病人哪能知道,他们院邻暗含着“春”给那卖药先生啊!花多少钱也得买他的药啊。这卖药的先生,得了病人邻居用“春点”把病人所得的病“春”给他,能够不费劲儿挣得下钱来。简捷地说,这就是江湖人用春点的意义。往浅处说是那个意思;往深处说,如同长江大海,用莫大焉。可是这春点用在一处,成为三种名词,前说江湖人调侃儿的术语为春,至于点之用处和意义,容谈到艺人的艺术类再为详谈。今将江湖中的春点先行录出,然后再分门别类述谈。

管男子调侃儿叫“孙食”,媳妇叫“果食”,老太太叫“苍果”,大姑娘叫“姜斗(jiàng dǒu)”,小姑娘叫“斗(dǒu)花子”,小男孩叫“怎科(zěn kē)子”,管父亲叫“老戗(qiāng)儿”,管母亲叫“磨(mó)头”,管哥哥叫“上排琴”,管兄弟叫“下排琴”,管祖父叫“戗儿的戗”,管祖母叫“戗(qiāng)的磨(mó)头”,管妓女叫“库果”,管良家妇女叫“子孙窑儿”,管男仆叫“展点”(仆人),管女仆叫“展果”,管当兵的叫“海(hāi)冷”,管侦缉探访叫“鹰爪”,管小绺(xiáo liu)叫“老荣”(小偷),管和尚叫“治把(bǎ)”,管老道叫“化把(bǎ)”,管尼姑叫“念把(bǎ)”,管做官的叫“冷子点”,管大官儿叫“海(hāi)翅子”,管外国人叫“色(shǎi)唐点”,管乡下人叫“科郎(kē lang)码”,管傻人叫“念攒(cuán)子”,管疯人叫“丢子(si)点”,管嘎人叫“朗(lǎng)不正”,管好人叫“忠样点”,管好色的人叫“臭子点”,管有钱的财主叫“火点”,管穷人叫“水码子”,管好赌钱的人叫“銮把(bǎ)点”,管天叫“顶”,管地叫“躺”,管东叫“倒(dǎo)”、西叫“切(qiē)”、南叫“阳”、北叫“密”,刮风叫“摆丢子(si)”,下雨叫“摆金”,下雪叫“摆银”,管房叫“塌(tā)笼”,管店叫“窑儿”,管阴天叫“牐(chā)棚”,管打雷叫“鞭轰儿”,管吃饭叫“安根”,管挨饿叫“念啃(kèn)”,管拉屎叫“抛山”,管“走吧”叫“窍”,管打架叫“鞭托”,管害怕叫“攒(cuānr)稀”,管肉叫“错齿子”,管马叫“风子”,管牛叫“岔子”,管驴叫“金扶柳儿”,管买酒叫“肘山”,管喝酒叫“抿山”,管喝醉了叫“串山”,管烧酒叫“火山”,管黄酒叫“幌幌(huàng)山”,管茶馆叫“牙淋(yá lin)窑儿”,管娼窑叫“库果窑儿”,管水叫“龙宫”,管兔儿叫“月宫嘴子”,管老虎叫“海(hāi)嘴子”,管龙叫“海(hāi)条子”,管蛇叫“土条子”,管桥叫“悬梁子”,管梦叫“团(tuǎn)黄粱子”,管牙叫“柴”,管字叫“朵儿”,管笔叫“戳子”,管刀叫“青子”,管枪叫“喷子”,管放枪叫“喷子升点儿”,管药叫“汉壶”,管跑了叫“扯活(chě huo)啦”,管人死了叫“土了点啦”,管妇人怀孕叫“怀儿怎(zěn)啦”,管寡妇叫“空(kōng)心果”,管麻子脸叫“梅花盘”,管俊品人物叫“盘儿嘬”,管人长得丑陋叫“盘儿念嘬”,管野妓叫“嘴子”,管车叫“轮子”,管衣裳叫“挂洒”,管穿得阔绰叫“挂洒火”,管穿破衣裳的叫“挂洒水”,管当铺叫“拱页(yè)瓤子”,管卖当票的叫“挑(tiǎo)拱页子”的,管表叫“转(zhuàn)枝子”,管帽子叫“顶笼儿”,管大褂儿叫“通天洒”,管裤子叫“登空(kōng)子”,管鞋叫“踢土儿”,管袜子叫“熏筒儿”,管瞎子叫“念招儿点”,管社会里的人不明白江湖事的叫“空(kòng)子”。

这江湖人调(diào)侃儿用的春点,总计不下四五万言,著者将这几十句写出来,贡献到社会里。论完全并不完全,因为书的篇幅所限,不能全部发表。容敝人写到各门各行的时候,将未曾发表的江湖春点,再一一刊出。以上所说的侃儿,系江湖中各门各行通用的侃儿。

从前江湖的人将一句春点看得比一锭金子还重,外行人是一句也不知道的。到了如今因为流行日久,外行人也能耳濡目染地熏上几句。敝人在北平的天桥、东安市场、西单商场以及各庙会,常听见有些个半开眼(对于江湖事有一知半解的人称为半开眼)的人,在各生意场儿调几句江湖侃儿,所调的侃儿尽是普通流行的。至于江湖各行隐语,与他们生意有关,外行还是不知道的。我这江湖的春点,是简捷地把意义说明,再谈金、皮、彩、挂、平、团(tuǎn)、调(diào)、柳(liǔ)八门生意。





\section{江湖人的旧组织各处长春会的领袖}


在早年,江湖人到了他们有地盘之处,都有一种组织,他们江湖人的团体叫做“长春会”。这会包括的生意有:算卦相面的,打把式卖艺的,卖刀创药的,卖眼药的,卖膏药的,卖牙疼药的,卖壮药的,卖刀剪的,卖针的,卖梳篦(bì)的,变戏法的,卖戏法的,唱大鼓书的,唱竹板书的,说评书的,说相声的,修脚的,卖瘊子药的,卖药子的,卖偏方的,治花柳病的,耍猴儿的,玩动物的,拉洋片的,卖药糖的,卖耗子药的,跑马戏的等等生意,俱都算上。五花八门,包罗万象,只要是老合(江湖艺人)就得入这长春会。

可是,这种江湖团体是老合们自动组织,并不在当地官署立案,会中的规矩都能遵守的,其范围大小是看他们的生意多少而定。最大的有鄚州长春会。那里的生意,各门各户都到。各种生意,各种的杂技全都有。会中按着金、皮、彩、挂、平、团、调、柳八门生意,一门有一门的领袖。那当领袖的人必须年岁高大,本领过人,素有声望。对于江湖中的事儿,无论大小全都懂得。同行的人们把他推举出来当他们的领袖,才能负一门的责任。由各门的领袖再推举出两个会长,分为一正一副。那充当长春会总领袖的人得是老江湖。做生意比人多挣钱,行为正大,做事光明,遇事不畏艰难,肯奋斗,肯牺牲,能调停事,排解纠纷,江湖人才重看,大家尊敬他,遇事都受他的指挥,服他的调动。这种人才是最难得的。

江湖人管教徒弟本领调(diào)侃儿叫夹磨(jiá mo)(师父传授真本事),管打徒弟叫鞭。如若鞭徒弟,外人看了不准多言,更不准阻拦。



长春会的事务分为对内对外两种事儿。对内的事儿是每逢有会的地方,到了会期的时候得给各处来的江湖人安排住处。那住处的名词很是各别,叫做“生意下处”。那里边住的人和住店一样,不过不准住外人就是了。内里的东西大家使用,不准毁坏。下处的规矩很大,凡是住在那里的人谁也得遵守。譬如有个变戏法的,他们没出去时候,或是开了圆笼(装道具的圆形器物),或是打开包儿收拾他们的家伙(道具),正然“挂托”(江湖人管他们变戏法往家具上弄鬼儿调[diào]侃儿叫挂托)哪,不论是谁也不准瞧看。还不准偷瞧,尤其是甲变戏法的挂托,乙变戏法的更不准瞧看。如若瞧,是不准;倘若偷瞧,那便是要“荣人家的门子”(江湖人管偷人的方法调[diào]侃儿叫荣人家的门子)。那是犯行规了,一定得受大家公平制裁。如若哪个江湖人在屋中“夹磨(jiá mo)”(师父传授真本事)徒弟,外人也得躲开。如若鞭(管打徒弟叫鞭)徒弟的时候,外人不准多言,更不准拦挡。如若人家教徒弟听着不躲开,那便是要荣人家的门子,也受大家制裁。

如若有甲乙两个人,要合伙做生意,挣了钱回来到下处分钱了,外人也不准瞧看。如若偷瞧就会有人耻笑。如若有人往下处“跨了点”(领着人回住处)来,什么叫跨了点呢?他们江湖人在会上支棚帐摆摊子,如若来了人要照顾他们,买的东西给多少钱,调(diào)侃儿叫“迎门杵”(挣的头一笔钱);如若遇买主人忠厚,好说话,钱也多,他们能够使“翻钢叠杵”(钢是话,翻是加倍,杵是钱,翻钢是用巧妙的语言让人深入陷阱,叠杵是多花钱)的法子,叫人多花钱;如若买主精明,或是狡猾,或是没钱,或是有钱不肯多花,只要挣到“迎门杵”就完事;倘若有真阔的人,能瞧出真的挣得了大钱,就不能在摊上讲买卖,把这人带到他们的住处,调侃儿叫往“窑儿里跨点”,这个人就是点头,他们在屋中能有最神秘、最巧妙的方法把大款弄到手。可是这种神秘的方法,非得得着师父的真传,才能挣得了巨款。按着江湖的规矩,甲往窑儿里跨点,乙见了得躲开,不能瞧看,也不准听。如若瞧着,再听着,那神秘的法子岂不会了?江湖人常说“宁给十吊钱,不把艺来传”,别人要花他多少钱都能成,可是要学他的本领,那可就难了。

我老云在各省常听他们江湖人说:“×××可不成,他连生意下处都没住过。”听他们这种口吻可以推测得出来,如若住过生意下处的人,一定懂得江湖规矩,事事都能晓得。江湖人对于久住生意下处的人,就尊敬得不得了呀!如若没住过生意下处的,他许不懂得江湖规矩,就是懂得点也是一知半解,不能全都懂得。如若江湖人有所讨论时,对于没住过下处的人,便都轻视他,他遇事还得少说话。倘若多说话,便有人说:“你没住过生意下处,懂得什么!”好像他没有发言权一样。

可是开这生意下处和开店一样,如若外人进来,就说:“没有闲房。不住外界人。”如若是江湖人,不管有闲房没有,有闲地方没有,愣往里走。没地方,大家有义气也得匀个地方。开生意下处的人,对于江湖人的规矩都要懂得。用个伙计,也得懂得各行行规。他们伙计、掌柜的,对于江湖人眼界得宽,认识的越多越好。生意下处的买卖能否发达,立得住立不住,全看当地的长春会主要人的本领如何了。

长春会的主要人对外的事很多。譬如某处要开个庙会,本地的绅士们也立×××会,由大家推举出来几位素有声望的当会长,主持庙会的事务。这种人要想借庙会之力,兴隆本地,首先得请江湖最有名望的人在他们那个地方成立长春会。给他们按着会期给邀各样的生意。不论是什么地方创办庙会,没有江湖中的各样玩艺儿绝不能成的。可是在各种生意没到之先,长春会的主要人得和当地的绅士商议好喽,可着他们那个地方由江湖人先挑,把好地方选择好啦,指定了是江湖人使用。别的行当给多少钱也不给使用。各样生意来全了,得由长春会的主要人指定某处是搁文生意的地方,某处是搁武生意的地方。什么叫文生意呢?算卦的、相面的、摆小摊子卖药的、点痣的……凡是不带锣鼓,“圆小粘子”(场子围不了多少人,调[diào]侃儿叫小粘[nián]子)都是文生意;变戏法的、打把式卖艺的、拉洋片的,都是武生意。可是武生意不准挨着文生意。那相面的全凭唇齿之能,向围着的人说话,叫人听着入味才能挣钱。如若挨着个变戏法的,锣鼓乱响,震得人们耳音乱了,那相面的就不用挣钱了。长春会规定了哪里是武生意的地方,那变戏法、拉洋片、打把式卖艺的,就往那里搁生意,绝不会乱搁场子。至于什么生意与什么生意之间,摊子应该离多远、场子应该离多远,也有一定的尺寸,谁也不能碍谁的事。至于各种江湖玩艺儿所占的地势给本地×××会应拿多少钱的花销,也由长春会的主要人与本地官商绅士事先商议妥当,到了收这笔钱的时候,也得有长春会的人,会同本地绅士挨着摊子、场子临时去收。总而言之,长春会的人如若与本地绅士商议各种事务,以不叫江湖人受损失,不受本地人欺压为最要紧的职责。现如今各省的乡镇所立的庙会,都是江湖人给他们兴旺起来的,哪处也是,年年如是,没有不发达的。

这种江湖人组织的长春会,各县的乡镇全都存在的。这种江湖团体是流动的性质,随时的集合,也无人管辖,也无人指导,官府并不立案。他们对内就为调剂江湖人做生意的地方、纠正江湖的规矩,对外就是与各地××会联合,解决一切的地皮临时租价与江湖人适用的地势而已。就以北平东边说吧,那里有个最大的庙会是丫髻山。那京东的各县乡民,届时都往那里进香。江湖的人们,各行生意也都“顶(赶)那个神凑子”(江湖人管庙会香会调[diào]侃儿叫神凑子),那里的长春会首领是难当的。当初有个“迫(pǎi)金扶柳(liǔ)儿,挑(tiǎo)招汉儿的”(江湖人管骑驴调[diào]侃儿叫迫[迫当坐讲],金扶柳儿是驴,挑当卖讲,招当眼睛讲,汉就是药)高景全,他老闯江湖有年,眼皮也宽,是江湖人都和他们有来往。他到了丫髻山,大家推举他为那里的长春会的会长,这样职任是没有期限的。要不是有了最大的过处,犯了众怒,或是自己不愿干了,才能算完。那高景全当了多年会长,也没从中取利,直到他干腻了,在天津三条石普乐园前边“安了(开了)招汉座子”(江湖人管开铺子卖眼药调侃儿叫招汉座子),才与丫髻山的长春会脱离关系。

在早年帝制时代,没有什么团体和组织。入民国以来,农工商学兵,都有了团体与组织,以及会计师、律师、新闻界、评书界等,都算是自由职业团体,也都有健全的组织。惟有江湖的艺人与这些行业的性质俱都不同:在乡间有长春会,他们全都加入;在冀、察、平、津等处,都没有组织长春会的,这江湖人的行当加入任何团体都不相宜,都是不合法的。故此江湖人到了各省城、各商埠、各都市,都没有组织,是散乱无章,弄得江湖乱道,彼此倾轧,时起纠纷。他们虽有兴隆地面、吸引观众的伟大之力,因为没有人在各市场指导他们按着文武生意立场子,而各市场的经理人多是资本家,也不明白这江湖的世故,布置得不得法,把那富有吸引游人的力量也弄得薄弱了,各省市的地方当局,更无人注意江湖人的事儿。

我老云这些年往各处云游,只是济南城有个长春会,内中的会员全都是江湖人,那会长××贵也是江湖中的名人,我调查了几天,他们的内容很是不错,凡是外省的江湖人,到了那里都得临时请求入会,经会中审查合格,发给会员证,才能在那里做生意。久在那里的江湖人,还得受该会的训练,然后才能在该地献艺。那里的各市场,文武生意立的场子,也适合江湖的纪律化,那里的江湖人,只要有真正本领就能得意。济南的江湖人总算是受了该会的益处了。其他各地无有长春会组织,就是有真本事的江湖人也得不着好地势,也挣不了钱,可就应了江湖人的话了:“生意人不得地,当时就受气。”若是本领不好的,占着好地方,他也难挣大钱,江湖人常说:“能为不济,占了好地,也是白欢喜。”现在北平这个地方很有些阔人,投资数万或数十万,买地皮,建房屋,创办市场,用的管理人员不懂得江湖事,没有适合江湖艺人、杂技场地的布置,不是创办不起来,就是弄得失败了,把若干万的财产变成了废物,当了摆设,还不知道是何缘故。阅者如不相信,往各处兜个圈子,就可看见那冤孽产了。





\section{江湖艺人之规矩}


江湖的艺人对于社会里得百行通。无一行不懂,无一事不明,才算够格。社会里半开眼的人管他叫“生意”,又叫“老合”、吃张口饭的,他们自称叫“搁(gé)念”。念是“不成”的侃儿。没吃叫“念啃(kèn)”,没钱叫“念杵头儿”,没有心眼的人叫“念攒(cuán)子”,没有眼的瞎子叫“念招儿”。

江湖艺人在早年是全都打“走马穴(xué)儿”(走一处,不能长占,总是换地方挣钱,江湖人叫走马穴),向来不靠长地(长地是指固定演出场所),越走的地方多,越走的道路远,越有人恭维说他跑腿的,跑得腿长。可是走那河路码头,村庄镇市,各大省城,各大都会地方,不论天地间的什么事全都懂得,那才能算份腿儿。如有事不懂便搁一事,一行不懂便搁一行,到了哪个地方,事事不明,事事不懂,便算搁了念啦!不用说发大财“火穴大转(zhuàn)”(在一地方演出挣了大钱了),就是早晚的啃(kèn)食休想混得上,就得念啃的。吃一辈子生意,由小学到老,也不敢说到家。

士农工商,各行各业做事的人,只能懂得他本行的事儿。惟有吃搁念的人,是万行通的。俗话说“隔行如隔山”,没开过果局子,没做过卖鲜货的小买卖,任你多聪明,要买鲜货,也得由着人家赚你的钱。买的没有卖的精。买卖人有三不卖:不够本不卖;赔钱不卖;不赚钱不卖。到了吃搁念的人,譬如他们没做过鲜货行的买卖,得懂鲜货行的事儿,别人遇事不搁便念,江湖人是不搁不念的。有天我走一家估衣铺前边,见有一位老合(江湖艺人)正买估衣,他要买人家的一件皮袍。估衣行的人认识他是老合,没多要钱,要十五元钱,这位老合他还要再少花个一两元钱,明着说不大合适,都是熟人,他向卖估衣的人说:“砸砸浆行吗?”我走到那里正听到此话,因为我懂得这句行话,估衣行的人管着少给钱、再落落价钱,说行话叫做“砸浆”。我听他说这句话,我站住了不走啦,听他们个下回分解。那估衣行的人说:“先生要砸浆,只能砸摇个其,多了不成。”估衣行的人管一元钱调(diào)侃儿叫摇个其。那位老合就给人家十四元,把皮袍买走啦。我就知道这位老合够程度,他懂得估衣行的侃儿,砸了摇个其的浆,他少花一元把皮袍买去。不用往大事上说,就以他买皮袍的事说吧,他懂得估衣行的事儿,到估衣铺买东西,就能少花钱,那就是懂得一行的好处。诸如此类的推试,老合们要是百事通,有莫大的好处。

说起江湖艺人的规矩,非我笔下所能尽述,也是很多的。他们守其规矩,较比其他守规矩都好,也值得人钦佩的。第一是生意人不管认识不认识,也不拘在什么地方见着,一见面就得道“辛苦”!如若烟台的老合离开了烟台,要往青岛去做生意,搭轮前往,到了青岛不能立刻做买卖,得先到各处拜会。其实在青岛的老合也不是青岛的人,也都是别处的人,他们不过早去些日子。先到青岛的为主,后到青岛的为宾,行客拜坐客,宾拜主,是江湖人最重要的规矩,名曰“拜相”。拜会同道的人也有许多的好处,譬如变戏法的人由别处到了青岛,要做生意,赶巧了各杂技场儿没有闲地,要做买卖没有地,焉能挣钱?如若按着江湖的规矩,不做买卖,先拜会同道,与同道取了合啦,能够有人让给他块地,让给他个场儿,叫他们挣钱吃饭,还能把当地的风土人情一一详告,到了挣钱的时候,能够又容易,又多挣。譬如,要是到了青岛,他自尊自贵不按着江湖的规矩拜会同道,若赶上杂技场儿没有空闲的场儿,不惟没有人让给他场儿做买卖,要和谁打听当地的风土人情,也休想有人能告诉他。

江湖艺人是最有义气的,拜会同道还有一种大好处,如若不愿意在青岛做买卖,当地老合(江湖艺人)们能够给他凑盘费,叫他另往别处去做生意。大家凑路费的事儿是司空见惯,并不出奇。江湖人做生意,在各省市的杂技场撂地儿,也有一定的规矩。譬如一个市场之内有两档变戏法儿的,若是拉场子做生意,必须两档子戏法隔开了,离着三两个场子才行,绝不能挨着上地(做生意)。市场的地方很宽大,能容得开多少档子玩艺儿是那样的;如若市场地方狭窄,容纳不了两档子玩艺儿,没法子办了,也许打把式卖艺的挨着打把式卖艺的,说书挨着说书的,卖药挨着卖药的,可是挨着做买卖,也最少要相隔一丈地才成。江湖人管江湖人尊敬的称呼都称“××相法”,挨着做生意,也得“相挨相,隔一丈”。

江湖人的玩艺儿是各有专门,不论研究出什么玩艺儿,都能久看不烦,百听不厌。它还有兴隆地方繁华市面的好处。想当初东安市场刚开办的时候,并不是尽做买卖的商家,在那时候,东安市场的杂技场儿较比如今的天桥儿还齐全、还热闹哪。近年来东安市场成了大商场啦,那东跨院里的杂技场儿还要保存哪。设若那个杂技场儿取消了,那东跨院里就没有人去了。生意场儿,吸引观众的力量也是非常大的。

到了乡间,不论是哪个地方,要是有人提倡在那里创立个集场,或是在那里创办个庙会,为首开办的人得先邀生意档子吸引观众。兴隆方面要是没有生意档子参加,任他办理得多善,也吸引不住人儿。关外的岳州会,关里的鄚州庙,可称得起最有名儿的庙会吧,那“海(hāi)万”(有名的)的“神凑子”(大庙会),也以生意档为主体。各乡镇的会首都和生意人联络。如若要开庙、立会,都和生意人首领商议,请些生意档子,才能开庙立会哪!

那么,生意人的首领又是谁呢?据江湖人说,生意人的首领是卖梳篦(bì)的,哪里有新开办会,和他商议好了,他就能把各样的生意约来,他还得帮着会首们来指定文武地来。什么叫文呢?哪叫武呢?拉洋片的、变戏法的、耍狗熊的、打把式卖艺的,都是武买卖、武生意。唱大鼓书的、唱竹板书的、卖梳篦(bì)的、卖刀剪的、卖药的、算卦的、相面的,都是文买卖、文生意。文档子挨着文买卖,武买卖挨着武生意。譬如有四档子文生意,当中间来档子武生意,锣鼓乱响,吵的那四档文生意说话也不得说,听什么也不得听,那就不用干了。各庙会的文武地儿也有一定的秩序。譬如某处有个庙会是四月初一吧,到了三月的月底,各样的生意、各样的玩艺儿就都来齐了。会首与卖梳篦的事先把地均配好了,初一清晨早起,各种的生意、各样的玩艺儿,就都按着秩序上地(做生意)。各样的玩艺儿都上了地啦,可是变戏法的还不能开锣,打把式卖艺的也不能张嘴儿……各样生意,都得等着会头。如若那卖梳篦的一张嘴,你瞧吧,各样的生意全都张嘴,打锣的、敲鼓的、喊嚷的,八仙过海,各显其能。谁有能耐谁挣钱。没能耐的圆不上粘儿(招揽不来观众),跟海子(南苑围场,此处借指圈子)里的鹿一样愣着。倘若会首们向生意人故意为难,故意刁难,勒索银钱,把钱要得离了范围,生意人们商议好了,给他们“叩棚”,由卖梳篦(bì)的把摊子一收,挑着担子,围着各玩艺儿场儿一转悠,您瞧吧,老乡:变戏法的不变了,唱大鼓的不唱大鼓书了,文武两档的生意全都收拾起来不干了。多咱把所争的问题解决了,那卖梳篦的一上地,各样的玩艺儿才能上地。如若卖梳篦的挑着担儿离开会场远走了,凡是玩艺儿也都一档子跟着一档子地全都“开穴(xué)”(即是另往他方)。任他会首有多大的本领,也留不住一档子的。江湖人的团体是这样团结的。都说“强龙不压地头蛇”(即是外乡人难惹本地人),惟有江湖人是不怕的,可说是“远来的和尚会念经”。

据江湖人说,生意人的首领是卖梳篦(bì)的,哪里有新开办会,和他商议好了,他就能把各样的生意约来,他还得帮着会首们来指定文武地来。





\section{江湖人放快者受罚的规矩}


江湖艺人,早年在每一省市或一商埠码头,皆有生意人之公共住所,名曰“生意下处”。凡是算卦相面的、打把式卖艺的、拉洋片的、说书的、卖药的、卖梳篦的、卖刀剪的、变戏法儿的,都要住在生意下处。

开这生意下处如同开店一样,字号也是××老店,门的两旁也有“仕宦行(xíng)台,安寓客商”八个大字。可是绝不能在门前悬挂“生意下处”的招牌。店中经理人与管账的先生、伺候客人的伙计,都得懂得江湖人的规矩。譬如店内住着卖药的客人,来了买药的人,到店内找卖药的先生,那先生若是在店内哪,不准伙计说没在店里;否则,柜上得认错儿,还得赔偿客人的损失。至于店内的伙计,将买药之人带到卖药的先生屋内,得赶紧退出屋外,不能多说话,倘有一句话说错了,买药的人醒了攒(cuán)儿(明白过来了),不愿上当,药也不买啦,那卖药的先生能答应吗?故此,生意下处的伙计与普通的客店规节大不相同。也有一种特别的好处,客人屋里有茶叶,得(děi)随便沏着喝,有东西随便地吃,倘若那生意人做了大买卖,或是“转(zhuàn)了”(管买卖获了厚利调[diào]侃儿叫转了),伙计们还能得点油水,也是雨露均沾哪。

生意下处,不论是客人、先生、伙计,每日午前不准“放快”。阅者若问何谓放快?这快也是江湖的侃儿。快分八样,名曰“八大快”。一是“团(tuǎn)黄粱子”,生意人管做梦调侃儿叫黄粱子;二是“悬梁子”,生意人管桥调侃儿叫悬梁子;三是“海(hāi)嘴子”,生意人管老虎调侃儿叫海嘴子;四是“海(hāi)条子”,生意人管龙调侃儿叫海条子;五是“土条子”,生意人管蛇调侃儿叫土条子;六是“月宫嘴子”,生意人管兔子调侃儿叫月宫嘴子;七是“土堆子”,生意人管塔调侃儿叫土堆子;八是“柴”,生意人管牙齿调侃儿叫柴。

每日午前,店内的人如有夜间做了梦的,不准向人说,昨天夜内我做了个梦。如若向谁说,谁是不依的。譬如向算卦的生意人说,夜里做梦了,他今天就不出去摆卦挣钱了。他若有每天挣一块大洋的能为,他就向和他说梦的人要大洋一块,不给是不成的,至轻了,也得买些东西请客。不止于说梦,就是说龙、说虎、说蛇、说塔、说桥、说牙、说兔子,都是一样地受罚。设若说梦的时候,要有二十个人听见了,这个乱可就大了,这二十个人也不出去挣钱了,他们二十个人,每天能挣多少钱,谁说梦来的就是谁放快了,叫这放快的人包赔二十人一日的损失。如若夜间做了梦,向大众不说做梦,说我夜里“团黄粱子”可不好啊,像这样调着侃儿说,就没事了。若是自己牙疼,在午前也不准说牙疼,得调侃儿说:我是“柴吊”(柴是牙齿,牙疼就说柴吊);他人得问:“你怎么直咧嘴呢?”可是过了晌午以后再放快就没事了。这放快的事儿,江湖人看得很重要,就是谁放了快赔偿人的损失,人也不愿意的。敝人曾向江湖人探讨过这放快有什么坏处?为何看得这般严重?某江湖人说:我们生意人最迷信的。每天出来做买卖,就怕出“鼓”儿(江湖人,若是相面的给人相面之时钱没挣下来,反倒被人大闹,这种事生意人是最怕的。江湖人管这种事儿调侃儿叫出了鼓啦,即是生气的意思),或曰鼓了点啦,或曰出了调角(diào jiǎo)啦(江湖人说,他们生意人若没出去做买卖,有人冲他放了快,出去做买卖不是出鼓儿,就是遇见了调角[有人出难题儿])。因为这层关系,生意人最忌有人放快。这种事情与梨园行人在没开戏之前,忌外行人击锣敲鼓是一样的。





江湖自嘲之暗语


江湖人管调(diào)侃儿用的行话叫做“春点”。老江湖人使用这春点是为了做买卖挣钱,离开了做买卖之外,皆恶(wù)团(tuǎn,说)春调侃儿。有些新上跳板(刚入这一行)的江湖人,学了几句春点,到处调侃儿,江湖的老前辈很为不满。一日,江湖的老前辈向新上跳板的人说道:“当初有两个生意人,一个是算卦的,一个是卖药的。两个人走在外县城内住了店,用完晚饭之后,算卦的到后院解手,他撒完了尿,忽然抬头一看,阴云四布,并无星斗。大概是天要下雨,他进屋后向那卖药的伙计调侃儿说:‘牐(chā)了棚儿啦!要摆金吧。’他那个伙计懂得春点,听他说‘牐了棚儿啦’,就知道是阴了天了;‘要摆金吧’,就知道是要下雨了。他们两个人调起侃儿来,恰巧被店里的伙计听见,那伙计不懂江湖的春点,听不懂这两个人所说的话,心中暗道:‘这两个客人不是好东西,大概许是做贼的。’谁想事有凑巧,当日夜内,店里丢了一匹驴,掌柜、先生、伙计们聚在一起讨论这驴叫谁偷去了,伙计忽然想起那算卦的、卖药的两位客人。他说:‘这驴叫六号的客人偷去啦!’掌柜、先生问道:‘你怎么知道呢?’伙计说:‘昨天夜内,我听他们说贼话来的,一定是他们偷去了。’掌柜、先生就把这算卦、卖药的告下来了,说驴叫他们两个人偷去了。这位县官是位老江湖出身,他改了行,走了一步好运,得了县官知事。这天他升了大堂,衙役三班喊喝堂威。店里掌柜的、算卦的、卖药的三个人跪在堂上。县官问道:‘你们三个人因为什么事打官司呀?’店里掌柜说:‘老爷,他们两个人住在我的店内,把我们柜上的驴给偷去啦。求老爷做主!’县官问道:‘你们两个人是干什么的?’这个说:‘老爷,我是算卦的。’那个说:‘老爷,我是卖药的。’县官又问道:‘你们两个人为什么不务正业,偷他的驴呢?’这两个人说:‘老爷,我们没偷他的东西,他们诬赖好人,求老爷做主。’县官向店里掌柜问道:‘你怎么知道那驴是他们两个人偷了去呢?’掌柜回答说:‘老爷,他们两个人昨天在我店里说贼话来着,叫我们伙计听见了,我们料着他们把驴偷去啦!’县官向他们两个人问道:‘你们两个人怎么说贼话呀?’那个算卦的说:‘老爷,我们没说贼话。我们是江湖人,因为昨天夜内阴了天啦,要下雨,我们两个说行话来着。我说牐了棚了,是阴了天了。他说要摆金,是要下雨。这是我们江湖人的春点,不是贼话。’县官这才明白,他虽做了官,因为他是老江湖,什么样的春点他都懂得。他也是最恨新上跳板(刚入这一行)的人是不是的就调侃儿,动不动的就调侃儿。县官立刻命令皂班打算卦的七十板,打卖药的六十板。打完了这两个人,县官就和他二人调起侃儿来,用手指着他二人说道:‘我也不管你是金(指算卦的金点而言),我也不管你是皮(指卖药的而言),绝不该当着空(kòng)子(不懂江湖内幕的人)乱团(tuǎn)春(团春即调侃)。一个打你申句(jū),一个打你行句(xíng jū,申句是六十板子,行句是七十板子)。若不是冷子攒(cuán)儿亮(县官管他自己叫冷子,攒儿亮即是明白江湖事儿),把你月(二)顶码儿(江湖人调侃一、二、三、四、五,是柳[liū]月汪载[zhāi]中),还得鞭个申行(xíng)掌爱句(jū)(月顶码儿是两个人,还得鞭个申行掌爱句是还应当打你个六、七、八、九、十板子)。梁上(大道上)去找金扶柳(liǔ)儿,扯活(chě huo)了吧,从此可别乱团春(梁上去找金扶柳是往大道上去找驴,扯活了吧是你们跑了吧,从此可别乱团春是叫他们不可在各处乱调侃儿,防备有人拿你们当贼办了)。’县官冲他们调的侃儿店掌柜是听不懂的,也不知他们说的是什么。然后就见知县冲他二人说:‘你们两个人,赶紧往大道上追贼,把驴给人家找回来。’两个人叩头下堂去了。”

知县冲他二人说:“你们两个人,赶紧往大道上追贼,把驴给人家找回来。”两个人叩头下堂去了。



那位老江湖把这段故事说给新上跳板(刚入这一行)的江湖人,这两个新上跳板的人自从受了他这番训教,可不敢没有事儿乱团(tuǎn,说)春,胡调(diào)侃儿了。这是江湖人自嘲的小故事。写出来在江湖笔谈里添上点材料,也可以使诸君明白,这侃儿虽会了,但不可乱说。





\section{江湖中之老合}


社会里的人士管蒙骗人的方法叫生意,又叫卖当(dàng)的。凡是生意人都是老合。有些半开眼的人对于坤书馆(女艺人说唱演出的书馆)、杂耍(是曲艺杂耍形式的综合叫法)馆子男女艺人叫做老合。其实,老合不止他们。说老合的范围是极其广大,其系统派别最为复杂。在我老云所说的金、皮、彩、挂等门,与风、马(má)、雁、雀四门,穷家门(唱数来宝的),骗术门等等的门户中的人都算老合。

老合们是跑腿的,天下各国、我国各省都能去到。越去的地方多,阅历越深,知识越大,到处受人欢迎。像已故的幻术大王韩秉谦,他到过外洋各国。中国各省市、各商埠码头走闯江湖的朋友聊大天谈起他时,都称韩秉谦才是个“腿”哪!这样的称呼在江湖中为至尊至荣。故此,江湖人自称“我们是跑腿的”。

社会里的人士管蒙骗人的方法叫生意,又叫卖当(dàng)的。凡是生意人都是老合。



我向江湖人探讨过多少次,他们江湖人群名词的侃儿,是否叫老合?江湖中的老人说他们生意人,不论是金、皮、彩、挂、风、马(má)、雁、雀,穷家门,只要是江湖人,都叫“吃搁(gé)念的”。“搁念”两字,是江湖人群名词的侃儿。与那国家、团体、学校、社会的名词儿是一样。

吃搁念的某甲与吃搁念的某乙,原不相识,两个人在一处相见,谈起话来,只要彼此说:“咱们都是老合,以后得多亲近。”甲乙二人从此就能亲近。老合两个字,是搁念行里公用名词的侃儿,我向江湖人问过,老合这句侃儿是怎么个意义?老江湖人说,这句侃儿很深奥,凡是江湖人,若能按着这句话去做事,事事都成,按着这句话去闯练,什么地方都走得通。他说了个极小的故事叫我悟解。我老云就由他一说这小故事而开了窍啦!还成为半个老合(还没够整个的哪)。

他说,有个茶馆买卖不好,无人照顾,雇了个懂得江湖事的伙计。这个伙计姓王,他自称傻王,可他不傻,也不装傻,他就在茶馆里运用老合(闯江湖的)的方法。譬如有个茶座由外边走进茶馆来,手里拿着个鼻烟壶。伙计给他沏壶茶,瞧见他将鼻烟壶放在了桌上。傻王一看这烟壶的成色(shǎi),也就值个几毛钱,他张嘴就问:“您这烟壶几块大洋买的?”这人说:“才六毛钱买的。”傻王就能失声说:“真便宜,您真会买东西。李四爷前天花两块钱买了个烟壶还不如您这个哪!”这个茶座听伙计这样恭维他,心里觉着痛快,也很喜爱傻王。天天不往别的茶馆去了,就专在傻王这里喝茶。其实,他喝茶给水钱,擦脸给毛巾钱,这里并不便宜,只因傻王会使老合方法,见物增价捧人家,捧对了,将主顾拉住了,买卖就能日日见好。“死店活人开”,这句话诚然不假。我听他说傻王能够见物增价,感觉着心地豁朗。他会使老合的手段,见了什么人说什么话,迎合他人的心理,说话行事,碰着人的心眼,样样事办出来叫人喜欢,句句话说出来叫人可心。可心与马屁的意思不同,千人所喜,准保发财。

某江湖人还说个小故事。他说,有个茶馆儿,买卖很为发达。天天茶座拥拥挤挤,走了一拨,又来一拨。掌柜的与伙计闹了意见,将伙计辞退了,另换个伙计。这个伙计不会说话,有个茶座儿,桌上放个鼻烟壶,他瞧着也就值个几毛钱,他问人:“你这个鼻烟壶是多少钱买的?”人家说:“一块大洋。”他把嘴一撇道:“一块钱不值,你买贵了,简直的上了当啦!你不会买东西。”这个茶座就瞪了他一眼。又有个茶座儿说:“伙计,你给拿个干净的茶壶。”他说:“都干净。不干净谁使呀!”人家问他:“水开吗?”他说:“你不放心自己上茶炉看去!”有人说:“伙计,你很是忙啊!”他说:“不忙吃什么!”他句句话说出来叫人不痛快,大家给他起个外号叫“倔劳”。一样花钱,哪个茶馆不能喝茶,谁跟他怄气?日子久了,是喝茶的都不来了。这个茶馆掌柜的觉悟了,将他辞退。他还说:“此处不养爷,还有养爷处!”

他说了这段小故事,我受了启发,觉得哪里的人都喜欢老合(江湖人)的顺情说好话,又觉着话是开心的钥匙。说话行事要研究不好啊,一生的事业绝不能发展。如若将这说话的本领学到了,投人所好行事,一生的事业何愁不发展。老合的一举一动,不论遇见了什么样的人,也能说到一处,绝不会处处碰钉子。老合的意义有多么伟大,非我一人所能道尽。我只知有官场中的老合,商家的老合,行伍中的老合,工匠中的老合,种庄稼的老合,读书中的老合,社会里处处都有老合,不过八仙过海各显其能,生、旦、净、末、丑,所扮的角儿不同就是了。

老合的手段很多很多的。只是一样,要学很不易。因为他们的手段是可以意会不可以言传的。有心领神会的聪明,管保样样能够学到。就是我老云五十多岁了,明白些江湖事儿,也有些人管我叫“江湖老合记者”呢!





北平平民化市场天桥之沿革与变迁


江湖中的艺人,无论练好了哪种艺术,都有百观不厌的长处。他们在哪里做艺,游逛的闲散人们就追到哪里游逛。不怕某处是个极冷静的地方,素日没有人到的,只要将江湖中生意人约了去,在那个冷静地方敲打锣鼓表演艺术,管保几天的工夫就能热闹起来。如若得罪了他们,或是由空地净盖房,盖来盖去将生意人挤走啦,管保不多日子,那个繁华热闹所在立刻就受影响,游人日稀,各种的买卖就没人照顾,日久就变成个大大的垃圾堆。江湖艺人有兴隆地面的力量,有吸引游人的力量,有繁华地方的力量。我国各大都市、各省市、各商埠、各码头有许多地方都是由他们的力量兴旺起来的。江湖艺人在社会中是有伟大之力,岂可忽视耶?阅者如不相信,我老云例举一事,便能知晓江湖艺人的势力如何。

在营口有个洼坑甸,算是营口最最繁华热闹的市场,较比天津的三不管(天津市南市的一个露天市场)、北平的天桥,不在以下。起初,洼坑甸是块低洼之处,年年夏天积存些雨水,臭气难闻。营口市的人都不到那里去。自从这里添了杂拌(zá ban)地(有各种露天杂耍儿、撂地赌钱的玩艺儿,江湖人称为杂拌地,又叫杂巴地),渐渐有人去逛。在那时算是个发芽的时期,有个“晃(huàng)条”的(江湖人管蹲签赌钱的调[diào]侃儿叫晃条的)刘凤岐,他是河北省河间县的人。对于江湖艺人有以艺术吸引游人兴隆地方的力量,他是知道的。搭了个财东(财主)就经营那洼坑甸。几年光景,由他开荒邀请各处的江湖人到那里做艺,居然就成功啦。刘凤岐是洼坑甸的经理,数年的收获,就由一个穷光蛋变成了一位资产阶级中的人物了。我云游客是到处云游,隐士文人都去游三山五岳、古寺庵观;我是专游生意场儿。在民国九年我就云游到营口,大逛洼坑甸,那里有卖梳篦(bì)的、卖刀剪的、卖估衣的,有各种货摊儿,各样吃食,大小饭馆林立,叫卖摊儿丛杂,锣鼓喧天,马戏棚、走兽棚、魔术棚、拉洋片的、大鼓书场、评书场、相声场、戏法场,卖药的、算卦的、相面的、打把式卖艺的,比大连西岗子还格外热闹。我云游了一个星期,都没过瘾,因事回津。又过了几年复至营口,乘车而往,及到了洼坑甸一看,冷冷清清,游人稀少,各铺户的伙计也都愣着,那种情况,将我老云的高兴一下子打没了。我下了车向各处访问,为什么那样繁华热闹的所在落到这样冷清?有人告诉我是刘凤岐财产有了,渐渐地骄傲,眼空四海,目中无人,对于江湖艺人待遇太苛,将江湖人得罪了。那些生意人,都挪到东街火神庙搭场子,将游逛的人们带走啦。这里没了玩艺儿,谁也不来逛了,这个洼坑甸算没了风水。我老云也扫兴而归。没想到刘凤岐那个人能够有了觉悟,痛改前非,托朋友向江湖艺人疏通,居然运动成功,江湖艺人又都挪回洼坑甸。真也奇怪,游逛的人们又都天天游逛洼坑甸,那个地方又成了繁华热闹之所。我老云问过刘凤岐:江湖艺人对于兴隆地面如何?他郑重地和我说是伟大的,生意人的势力他是知道了。到如今只要往营口去过的江湖人,对于刘凤岐是有口皆碑,无不钦佩。他联络江湖中的生意人,种种手段,样样方法,是很有门道值得钦佩的。据我所知道的情形,营口洼坑甸因有刘凤岐而兴,有江湖艺人而繁华起来的。江湖艺人能兴隆市面,不仅营口是那样,哪省哪县也是一样的。

从前天桥那里的地皮每亩地才值二三百元。自从天桥市场渐渐发达以来,那地皮的价儿也随着往上增长,最近要在天桥买一亩田种地必须三千元大洋才买得到哪。天桥地方是江湖艺人给振兴起来的,到了如今,成为北平平民化的市场,功劳是他们的。地价涨到三千元一亩,恐怕没有人酬谢他们吧。现在全国各地,因为经济的状况不佳,连上海那个地方,都嚷不景气,北平的天桥,各种的商业,各种的玩艺儿场,还能支持得住,实是不易呀。

老北京的天桥,有许多江湖人做生意拉场子,游人众多。



闲话休提,书归正传。我老云将这些年调查得来的天桥沿革、变迁、状况、艺人、艺术种种里面的材料,书出来贡献于阅者。

据北平市老人所谈,当初的天桥是最高无比。在天桥南边往北看不见前门,在天桥北边往南看,看不见永定门,可见那座桥是不矮的。桥底下走水,桥东叫东沟沿,桥西就叫西沟沿,那道沟最长叫做龙须沟。永定门内,东天坛,西先农坛,两坛之北,天桥之南,地势很低,尽是水坑。清季鼎盛时期,天桥附近有些贩夫走卒、劳动的人们在那里求生活,无事就在那里散逛,未有今日之盛也。

天桥的茶馆,据我老云所知道的,最早是西沟沿南边有个大野茶馆,字号福海居,主人姓王行(hánɡ)八。他那野茶馆所去的茶座,都不注意字号,全都呼为“王八茶馆”。每逢春末夏初之际,一些个闲散阶级人,提笼架鸟,喝个野茶,都到那里去的。在清末时候,提起王八茶馆几乎无人不知,每日高朋满座,主人王某,对于应酬茶座,周全事儿是能手,克勤克俭,买卖发达,颇获厚利,十数年的好买卖,很置了些产业。

围着他那茶馆,有许多江湖人做生意。拉场子,撂明地(不是屋子的演出场所),游人众多。人能兴地,地能兴人。那附近的水坑,随垫随宽。地势越宽阔,支棚架帐,摊贩云集,游逛的愈多。夏季兴旺,每入冬令,游人稀少,不如夏令百分之一。野茶馆最多之时,系先农坛东北部开办临时市场,水心亭、杂耍(是曲艺杂耍形式的综合叫法)馆子,茶馆林立,盛极一时,天桥发达第一期也。

有清室某王祭坛,在坛门往北望见棚帷杆幌(huàng),锣鼓喧天。只向当局问了问是何所在,当局疑其见怪,立即驱逐。天桥的玩艺儿迁于金鱼池,未几,天桥仍然恢复原状。

庚子年后,前门至永定门翻修马路,天桥拆改为小石桥矣。马路东有歌舞台、乐舞台、燕舞台,梆子名角崔灵芝、一千红等与名武丑张黑,均在三台献艺,每日三台均上满座。天桥以前尽是浮摊,即估衣摊、铜铁破烂摊、叫卖商摊销货之所在。城南游艺园,前后开办,虽为阔人游艺园,与天桥大有益处,藉壮声势,长袍短褂上等人也有。天桥的各种生意十分兴旺,为天桥发达的第二期也。

是时警察厅对于平民娱乐极为注意,为繁华市面计,将天桥立为东西市场,组织东西市场联合会。为永久事业,各摊贩商人集款收买官地,从那时起,大兴土木,渐渐建房筑屋,经十数年之久,便成为今日平民模范之市场也。





\section{天津南市三不管露天市场}


凡是到过天津的人,都知道有个三不管。外省人没到过天津的,听人说得三不管可逛,那里最热闹,说得天花乱坠,叫那没到过的人闻香不到口,不知这三不管是怎样热闹哪!我老云每逢路过天津时,必到三不管兜个圈儿,把我所闻所见写出来,将那天津平民娱乐场——江湖人的根据地,介绍给阅者。

三不管那个地方,说起发达来,为我华北第一,可不是热闹第一,也不是好的第一,是发展得最快数它第一。在我幼年的时候(时在清末)到过天津一次,那三不管一带净是水坑,又深又大,较比北平的什刹海还大些,可是不如什刹海清洁。坑的西边有一片热闹场,北边有一片热闹场。坑内净是小船,供游人往来乘坐。每至夜内,船上有乘客,或三或五,一人弹弦,一人敲打茶杯,二人对唱靠山调(diào)的小曲。什么《从良后悔》、《报杆打忘八》,使人听了能感觉那真是天津的土产,地道的天津味儿。我向本地人问过,那个地方为什么叫三不管?据他们说,那地方离外国租界很近,外国人对那里是不管;市政当局知道那里是臭水坑子,是垃圾堆,不大注意,也不管;县署因为那地方的界限属于市政所辖,他们也不管,故此叫做“三不管”。是与不是,也不敢断定。不过他是那么说,我是这么讲。这个三不管究属在什么地方哪?以天津的四马路说吧。在清朝时代,马路是天津县的城墙拆去了之后,才修成了四大马路,那四大马路之内算是中心地。三不管在南马路之南,所隔的不到半里路,有清室某大官员在那里用土垫坑,修马路、建民房,设立房产公司。直到民初时代,算是三不管刚发达的时期。那大空场儿之大,为历来所未有,往西至南关下头,往南到海光寺,往东到日租界西边,往北到南马路以南,较比北平的天桥大有三分之一。最多的玩艺儿是小戏棚子,或用席搭,或用布圈,里面唱的是《算粮登殿》、《杀狗劝妻》、《翠屏山》、《金水桥》。山西梆子,破锣破鼓破行(xíng)头。坎(kǎn)子(收门票的人)上的朋友在外边把门要钱,威威武武,连叫带嚷,很是怕人。可是个个小戏棚内都拥挤不动。虽然零打钱、不卖票,较比到大戏园子买票花的钱更多,贪贱吃穷人,是其实也。卖碎布头的摊子一家挨一家,以白傻子吆喝的最出奇,连说带唱卖布饶布头,为历来所未有,都说他卖的是布铺里剩下的碎布头儿,我可看见了他将整匹的布一块一块扯碎了,冤那老赶(北平管那乡下人叫怯杓[sháo],又叫做白帽子,天津叫老赶)。其实买到家里一算计,买得更贵。到了他摊前一站,听他的“钢口”(说话的技巧和分量)一卖弄,全都瞧着便宜。卖布的使老合(江湖艺人)的圆粘(nián)子(招徕观众)、卖钢口、亮托(亮出做生意的货)、迷魂掌,就是在那地方。赶上了那年月,如今,可就不成了。

到了民国十年前后,我老云逛起三不管(天津市南市的一个露天市场)来,能够天天去,逛个一个多月也不腻。各种杂技,各样生意,各大戏棚,应有尽有,无一不全。那坑可垫的都没了,完全是平川地,翠柏村,德美后,土娼乐户无不利市十倍。由南马路往南,有地皮就盖房,直盖出好几里去,成了好几道繁华热闹的街道,由南门往东,第一是荣业大街,第二是东兴大街,第三是广兴大街,电影院、戏园子、医院、澡堂子、照相馆、落(lào)子馆(坤书馆),是一家挨着一家。北平的天桥是白天热闹,夜内没有人;天津的三不管是昼如夜,夜如昼,各有不同的热闹。在那个时候,江湖艺人不论是做什么生意的,也都发达,个个得意洋洋。金、皮、彩、挂、平、团(tuǎn)、调(diào)、柳(liǔ),跑马戏的、玩腥棚的(演假马戏的)、弄戏头棚的(玩走兽棚的)、挑(tiǎo)拱页子的(卖当票的)、挑(tiǎo)转(zhuàn)枝子的(卖表的)、卖大堆的(卖劣质大件皮袄毯子的)、挑(tiǎo)里腥(lǐ xing)嘴子的(野妓揽客的)、晃(huàng)条儿的(蹲签赌钱的)、摇会的(筹集款项的)、挑里腥衫的(卖劣质衣服的)、挑(tiǎo)水滚子的(卖胰子的)、挑里腥光子的(拉假洋片的)、做四平粘(nián)子的(卖丸散膏丹各种药的)、做骑磨的(不详)、撒(sǎ)小帖(tiē)子的(撒传单骗人看病的)、做大票的(施药治病冤人骗钱的)、搬柴的(拔牙的)、镶柴的(镶牙的)……真是一支秃笔写之不尽,说之不完。这样说阅者可有不能了解的,请诸君别忙,容我把这些江湖事,一样样、一桩桩地都说出来,管保诸君瞧着有茶余饭后谈天的话料。





\section{天桥市场摆地的人物}


我说这个摆地的人物,凡是久逛天桥的人差不多都知道的。不知道的人也是不少。阅者诸君如若问什么叫摆地的?说起来也是一种职业。干这行的都得胳臂粗,脑袋大,有点窦尔墩的派头,才能吃得了这碗饭哪!本钱不大,有个几十块钱就能成的,买些桌子、凳子、竹竿、杉篙、布棚儿,弄几个生意场,再有几块地儿,就有江湖艺人找他们临时上地(做生意),挣了钱是二八下账。如若挣一元钱,做艺的八角,摆地的两角。上地的行当是:说相声的、唱大鼓的、说竹板书的、摔跤的、变戏法的、打把式卖艺的、唱坠子的、抖空竹的,种种的玩艺儿。此外还有卖药、算卦、相面的、点痣的,这几种生意用不了许多的桌凳,只要有张桌子,一个凳儿就成,可不能二八下账,由上地的艺人挣了钱随便分给他们,数目多寡没有一定的。

天桥摆地的人物也各有地盘,最早是李六一、赵凤桐、老冯。李六一所摆场子在天桥西北一带,魁华舞台西北,他所占的地皮先是官地,后由商人购买改为民地。在民国元年至十年之间,他的地势最好,凡是艺人都愿上他的地儿,他每天的收入也有几元钱。近年来地势变了,游逛的人们都不走那一带了,也由地主建筑了许多的房子。李六一的场子十落一二,他这个摆地的已然半守旧业半改行了。老冯所摆的场子在王八茶馆以南,魁华舞台东北一带,在民国十年前,游逛的人们都在那里盘桓,上地(做生意)的玩艺儿也很齐全,所分的利钱哪天也有两三元。至今他那些场子全盖了房子,老冯这个人也不知哪里去了。赵凤桐所摆的场子在电车道两边,公平市场北半部,所有的地皮都是公平市场的。上他地的艺人净是武买卖(江湖人管卖艺里变戏法的、摔跤的、拉洋片的等等生意叫做武买卖。因为这些玩艺儿有锣鼓敲敲打打,吵吵嚷嚷,扰乱其他生意不得做生意,都叫他们为武档子),没有文买卖。一些个算卦相面的、卖药的文生意,都怕武生意,若是上地做买卖,文生意离着武生意越远越好,清清静静,得说得道,挣钱为妙。绝不肯以肉嘴肉嗓子和锣鼓儿反抗。有了这种原因,赵凤桐的场子成了武玩艺儿的地盘,文生意一份也没有了。

天桥市场摆地的人物之一——点痣的。



天桥摆地的人物能够发达的,只有两个人。一个叫吴老公,一个叫老魏。吴老公是个太监。因为时代变迁,太监的权威没有了,受了时代变迁的淘汰,当太监不能维持生活,要当也怕没处当去。他有些钱财,置买桌凳,棚儿帐儿,占几个场子,做摆地买卖。他摆的场子在公平市场西边,魁华舞台以南,在民国十年以后,他那一带的地势,为游逛的人们必经之路,上他那地的艺人都是有本领的,每日也收入几元钱,克勤克俭,积蓄款项,盖了两三所房子,由摆地改吃瓦片儿,是个有眼光的人,所以生活无忧,很为得意。只是他人缘有限,因为他没有儿子,天桥的人们都说他苦奔而已。看起来为人穷富事小,没有人缘也是不好啊!

老魏是河间人氏,与名伶魏莲芳是同宗弟兄,先在天桥魁华舞台后边摆茶摊儿,他在天桥瞧着摆地的营生可干,就置买桌凳棚帐,招揽生意。我老云还记得上他地的是两档子生意,一文一武。文生意是做“八岔”(江湖人管算奇门的调[diào]侃儿叫八岔)的连仲三,武生意是“挑(tiǎo)厨供(gòng)”的(挑厨供的是卖戏法的)孙宝善。他由给这两个人摆地干着得意,又在先农坛东面,旧坛坡下边弄了一个场子,在他这三个场子初立之时,邀了三档硬生意(江湖人管能挣钱的玩艺儿调侃儿叫硬生意)。头个场子是摔跤的宝善林(宝三),二个场子是张寿臣、刘德志相声,三个场子是关顺鹏的竹板书,这三档玩艺儿挣了钱和他二八分钱,哪天他也能收两元至三元。又在三个场子后边弄了个野茶馆,字号爽心园。高搭天棚,每年的夏季的茶座很多,买卖很是发达,由野茶馆又改为杂耍(是曲艺杂耍形式的综合叫法)馆子。爽心园分为南北卖座。北边卖清茶,南边唱大鼓。山东的坤角李雪芳在他那馆子唱了二年半,天天上满堂座儿。一者是李雪芳的艺术好,有叫座的魔力;二者是地势宽阔,处于流水粘(nián)子(江湖人管游逛人必由之路调侃儿叫流水地,管一要钱游逛的一散的玩艺儿叫流水粘子,别的生意能在他们要钱的时候吸收游人,调侃儿叫借得了粘子),游人容易入步。爽心园茶馆为天桥借粘子第一好地方,凡是做艺的人们都愿上他的馆子。老魏近些年积蓄了不少钱,将爽心园前边的官地买到手中,改为六个生意场,盖了些房子,由摆地起手,勤苦耐劳,事业发达,十年有余,变为资产阶级中的人物,也是福禄加于勤俭人也。天桥的人们对于他是贬多褒少,或许是一家饱暖千家怨。现在爽心园的台柱子李雪芳已回归济南,另邀李艳芬、李艳楼演唱山东大鼓,上的座儿也还不错。场子的生意能够挣钱久占的是宝三摔跤,于俊波、郭起如、尹麻子相声,其余的场子都是随来随走,流水似的生意。

摆地人物,最近有豆汁舒家、天华园王家,较比以上的几个人差得太多,他们的场子只有一两块,也不见发展,仅落扎挣劲儿(勉强支撑)。因为这些年天桥市场盖的房子太多,将生意场挤得剩了一半,摆地的行当也要排挤没了。





\section{天桥东市场卖估(gù)衣的}


天桥市场地势宽阔,面积之大,在北平算是第一,各省市的市场也没有比他大的。东至金鱼池,西至城南游艺园,南至先农坛、天坛两门,北至东西沟沿,这些地方糊里糊涂地都叫天桥市场。在这里面又分出多少个市场:天桥东边叫东市场,又分为第一、第二、第三巷子。天桥西边最为复杂,马路以西叫西市场,由吉祥舞台往南,坛门往北叫公平市场,由电车总站往西,为公平市场南北之界限,南为南公平市场,北为北公平市场。在魁华舞台西边内市场叫先农市场,往南叫华安市场,现在都盖成民房,这个市场名称虽在,玩艺儿是没有了。西边有片红楼,叫城南商场,游艺园东边叫天农市场。天桥东市场没有杂技场、玩艺儿场,全都是做买卖的,可称为商业区,而最多的买卖是卖估衣的。估衣行虽有估衣铺、估衣摊的分别,可是铺子也不在屋内做买卖,而在门前支棚设帐,和估衣摊是一样的。

我老云是个穷光蛋,有了钱不懂得做做衣裳,向来是买估衣穿,我和估衣行是经常交买卖,他们估衣行的内幕情形,我曾调查过几次。他们这行的买卖情形最复杂,规矩也与普通的商业不同。

我有个估衣行的朋友张君,我问过他:“你们估衣行为什么铺面弄得屋子挺黑呀?”张君说:“我们卖的衣裳都是由当铺里趸(dǔn)(整批地买进)来的,无论是皮、棉、单、夹、纱,难免衣裳上有残坏的地方,什么大襟上有块油啦,袖子上有个洋烟卷烧的小窟窿啦,胳肢窝虫子咬啦。我们来了买主,挑选了半天,好容易挑合适了一件衣服,要叫他瞧出点小毛病,他能要吗?如若屋子黑,不亮堂,叫他在屋子里瞧看,稍微大意就能看不见,讲好了价钱,将衣服买回家去再看出毛病来呀,向来估衣行的规矩是出门不管换,最腻“抖德(dè)”。我问张君:“什么叫抖德?”张君说:“我们估衣行管买走的东西又拿回来换钱,调(diào)侃儿叫抖德。”我问道:“各商家的买卖货物,除了药品是出门不换,别的东西都可以换的,怎么估衣不能退货哪?”张君说:“七十二行手艺买卖,行行不同。就以我们估(gù)衣行说吧,虽是讲本图利,与各行买卖全都不同。我们这行用伙计是分为挣工钱与不挣工钱。挣工钱每月至多不过六元,少者三元,柜上管顿饭,到了三节算账有零钱,零钱也少。如若不挣工钱的伙计,柜上不给工钱并且是不管饭,他分的零钱可就是大股儿。我们估衣行的伙计挣钱多少,全由零钱多寡而定。”我问道:“你们这行的零钱是怎样挣法,如何分钱?”张君说:“我们的货物上都有暗码。譬如,来位客人要买大氅(chǎng)(大衣),伙计一看大氅上画的号码,是应卖十三元大洋,他敢向买主要二十四元的。如若买主给了十五元,他应当卖了吧?他不惟不卖,还向买主花说柳说,叫买主添钱。如若买主多添钱,他们伙计就多分钱;买主一定不添了,他也得卖给人家。卖下这十五元钱来是大账写十三元,小账写两元,大账的十三元算掌柜的本利,小账的两元就是伙计的零钱。到了晚上,收摊算账,这两元小账是掌柜的分一元,伙计分一元,每天伙计们谁分多少零钱,由他们个人卖货能力而定。越是有能为的伙计,越能在码的价外多多地卖钱。”我问张君:“如若是挣工钱的伙计,分零钱如何分法?”张君说:“那要是十三元的货物他们卖了十五元,大账上收十三元,小账上收两元,当天这两元不能分,得了零钱,天天往小账上记数,到了五月节、八月节、年关,才按着小账上的数目,按股儿分钱。”我又问张君:“我常听贵行人说,大账好,小账好,大账不好小账也不好,那是怎么回事?”张君说:“譬如,今天来的买货之人,件件东西都多给钱,卖项也好,大账上能落笔在百数多元,有人要问今天买卖怎样?就说:大账很好。如若卖出去的东西件件都有伙计的零钱,小账上一笔一笔写不少,有人若问今天买卖怎样?就说:小账不错。如若恰巧喽,买东西的都不出大价钱,件件东西都按码卖出去的,大账上落了好几笔,小账不落笔,有人若问今天买卖怎样?就说:大账不错,小账不好,还没落笔呢。如若今天一个买主都没有,有人若问今天买卖怎么样?就说大小账都没落笔。”张君说到这里,向我老云说道:“你想我们估(gù)衣行好容易来个买主,费了九牛二虎之力将货卖出去了,大小账都落了笔啦。买东西的人又回来说,东西不要了,将钱退给他。我们伙计、掌柜的能愿意吗?故此我们估衣行无论是伙计、掌柜的,都怕有抖德(dè)的事儿,遇见这路事都是腻的。”

估衣行虽有估衣铺、估衣摊的分别,可是铺子也不在屋内做买卖,而在门前支棚设帐,和估衣摊是一样的。这行的买卖情形最复杂,规矩亦与普通的商业不同。



我问张君:“你们估衣行儿是讲本图利,与江湖的生意不同,为什么也讲究调(diào)侃儿哪?”张君说:“譬如,我们估衣摊上挂着一件绸子大褂,尺码才三尺二长。来到个买主,掌柜的看着他奔了这件大褂,瞧他身高够四尺多,那大褂往他身上穿,一定是尺寸短。伙计没料开这个情形,掌柜的料开了,无论如何也是白费话多劳神,这号买卖做不好。与其多费话,歇会儿好不好?掌柜的冲伙计调个侃儿(说行话)说:‘喜。’伙计听见了就向买主说:‘你不用看,也不用买,这件大褂你穿着小。’那买主也就走了。这是调侃儿最小的用处。往大了说,能够一句侃儿多挣两块洋。譬如来个买主,正赶上买卖忙,伙计、掌柜的都伺候买主儿,瞧货讲价钱之际又来了个买主,学徒的过去张罗。人家买的马褂,上头号的码子是三元五角,学徒的向人家要七元钱,人家给了三元五角。那学徒的能力有限,就要卖给人家。大伙计有本领,看出这买主儿是还能多添钱的样子,不能看着钱不挣,将买卖做屈了,冲学徒的说:‘外库外。’学徒的懂得侃儿是要卖五元五角,他向买主说:‘我们这马褂少了五元五角不卖。’那个买主爱上了这件东西,真给了五元五角钱。老云你想,这不是多来两元吗?记在小账上又是笔零钱吧?调(diào)侃儿是有用的,不是瞎胡闹的。”我问张君:“我走在估(gù)衣摊旁边,有时候听你们行的人调侃儿说:‘砸砸浆。’那是什么侃儿?”张君说:“譬如行对行要买件大褂,卖主不能多要钱,要了三元五角。买主的意思是还要少给钱,他不说再少给几角,和卖主调侃儿说,‘砸砸浆吧’。如若卖主说‘砸浆可不成了’,即是少了不卖;如若卖主说‘砸砸浆还成’,即是再少给个几角钱还成哪,买主又可以便宜些钱。”

我问张君:“都说你们估衣行所卖的货物,应卖多少钱,衣裳上有暗码儿,码上多写钱数,叫买主看不明白,好向买主提高卖价。有些人说,那码是虚五对折二八扣,是不是哪?”张君说:“我们估衣行的暗码不是那样。你想,虚五对折二八扣,那不是太麻烦了吗?譬如一百元吧,虚五就剩五十元,对折又去二十五元,还剩二十五元,二八扣哪,又去五元,还剩二十元。若是值二十元的东西号一百元的码子,那不是离着太远啦!我们的暗码是不叫买主懂得,也不能像那么麻烦哪!”我问张君:“究竟贵行的码子是怎么折扣哪?”张君说:“我们估衣行的码子是有:大下一、小下一、三三码。共有这三样码子。”我问张君:“什么叫大下一哪?”张君说:“譬如,衣服上写着十二元,大对折下一,是对折剩六元,再下去一元哪,应剩五元。这就是对折大下一。若是应卖五元的东西,按大下一的码子写十二元。”我问张君:“什么叫小下一哪?”张君说:“譬如,衣服上写十元,对折五元,还剩五元,再下去一角,是落成四元九角。凡是卖四元九角的东西都号十元钱。”我问张君:“什么叫三三码哪?”张君说:“譬如,衣服上写三十九元,按三折计算应落十三元。凡是卖十三元的东西,若按三三码子就号三十九元。”我问张君:“外行人看了贵行的码子能够明白不能哪?”张君说:“这写暗码是我们自己人做买卖手续上便利,易于记载钱数,外行看了也是不懂的。并且一家一个规矩,这家使大下一的码子,那家就许用三三码子。除了本柜的人知道柜上使的是什么码子,别家的伙计也是不明白。”

我问张君:“贵行的侃儿与江湖的侃儿是否一样?”张君说:“不一样。江湖人管小孩调(diào)侃儿叫怎科(zěn kē)子;我们估(gù)衣行叫喜合子。江湖人管大调侃儿叫海(hāi);我们叫德(dè)。江湖人管吃调侃儿叫上啃(kèn);我们叫抄。江湖人的钱数,一叫柳(liū),二叫月,三叫汪,四叫载(zhāi),五叫中,六叫申,七叫行(xíng),八叫掌,九叫爱,十叫句(jū);我们估衣行是一叫摇,二叫柳(liū),三叫搜,四叫臊,五叫外,六叫撂,七叫撬,八叫奔,九叫巧,十叫杓(sháo)。江湖人管一元钱叫柳(liū)丁拘迷把(jū mi bǎ),我们叫摇个其;江湖人管五元五角叫中丁拘迷中,我们叫分外库。江湖人管好叫撮啃(kèn),我们叫贺。江湖人管喝茶叫啃牙淋(kèn yá lin),我们叫悍迟。江湖的侃儿与我们估衣行是不一样的。”我问张君:“外行人若是懂得你们的侃儿,能有好处没有哪?”张君说:“有好处。如若外行人懂得估衣行的侃儿,买东西时候和我们行人只要一调(diào)侃儿,就知道买主是本行人,不能要大谎,买东西多少也有点便宜。”

我问张君:“贵行的货物来源是由什么地方买来呢?”张君说:“我们行里的货物,大多数是当铺里买来的。各家当铺有过了期限赎不了的货物,按着他们的本利凑成大堆儿卖给我们。我们估衣行营业状况如何,须由当行的买卖兴衰而定。现在社会里人人喊穷,当铺的买卖都赔钱,我们估衣行也是一样地受影响啊。”我问张君:“都说你们估衣行卖骗人的货物,究竟有无其事哪?”张君说:“我们卖中国的衣服是不冤人的。有些个卖西服估衣的都用旧大衣翻个儿,呢子的东西难分里面,卖翻个货的只算以旧当新,还不算冤人;惟有卖拼货的是真冤人的。”我问:“什么叫卖拼货的?”张君说:“用小块的碎呢子拼凑着做个大氅,做得了,叫人瞧不出缝儿来,和好东西一样。如若买了去,穿到几个月,那缝儿全都露出来,若是露了缝那就不能穿了。有些个买东西的人眼力不好,买着这样东西便是上当。估衣摊子上买东西不是都上当,只要有眼力,一样能买着便宜东西。若是成年价净冤人,谁还照顾我们?买估(gù)衣上了当的人,买别的东西也是一样上当的。最好是别贪大便宜,管保干什么都少吃亏,少上当的。”我老云听了他的话,不拘走在哪里也不爱便宜,倒是不能上当,不能受冤。

天桥东市场也有些个桌椅铺。桌椅铺是分为新、旧、粗、细。如若买硬木桌椅得到东市场的东北,金鱼池以北,那卖细活的铺子不大冤人,卖的价钱有高有低,就是不便宜,也不过是价钱大些,东西全是地道的。天桥东市的桌椅木器,都是旧桌椅烫蜡上色(shǎi),说北平话,瞧就瞧着有一眼,也是刀尺(dáo chi)货儿(修整、整理过的古旧东西)。买那个东西的人都是我们那里的老乡,花钱不在乎多少,买回家去摆不上几天,用手一摸,管保弄一手颜色。他们是成天价专蒙老乡。阅者如不相信,只管前去调查,我老云是绝不“胡云”的。那卖碎铜烂铁、五金电料的摊子,所有他们卖的零碎东西,也是和估衣行的货物一样,有眼力的人就真买得着便宜东西,没有眼力的人也是一样的上当。最近天桥的风水搬了家啦,天桥东歌舞台、乐舞台、燕舞台已然拆去,改为估衣棚子。那棚子底下天天有些个卖绸片估衣的做买卖。他们那一带买卖不同,都是山东莱州府的人,买卖诚实。我曾考查几次,他们卖东西是不大蒙人的。最奇怪的是这些山东老哥们卖估衣不吆喝,将货物挂起来等主道候客,做的是实在劲儿。可惜就是天桥东边没有风水,去的人们很少。社会的经济恐慌,都透着不景气。个个摊子不卖钱,都到了挣扎着的状况,莫不叫苦连天。唉!





\section{戏园子的坎(kǎn)子(收门票的人)}


各戏园子都有些把守戏馆子门的人,江湖人调(diào)侃儿管他们叫“坎子”。吃这碗饭也颇不易,身材必须个个长得雄壮,虎头虎脑的能镇得住人才成哪!小戏园子三四个人,大戏园子七八个人,人多了都有个头儿,到了开戏的时候,锣鼓一响,他们的头儿带着伙计往门内或坐或立,来了听戏的人,有官有私,他们招儿里会把簧儿(招儿是眼,把是看,就是眼里能看出听戏的是什么人),来的人应当买票不买票,一望而知。如若遇见冒充官人的与假充字号的不买票,他们就能拦住。说牐(chǎ)了,个个都会打架。如今社会里的人士文明多了,听蹭戏的人较比早年少多了,“坎(kǎn)子”们“鞭托”(打人)的事见不着啦,戏园子的“坎子”也好干了。

各戏园子都有些把守戏馆子门的人,江湖人调侃儿管他们叫“坎(kǎn)子”(收门票的人)。吃这碗饭也颇不易,身材必须个个长得雄壮,虎头虎脑的能镇得住人才成哪!



跑马戏的班子里男女角色无不齐备,可就是没有坎子。他们马戏班子不论开到哪个地方也得先找本地的“坎子”,和他将手续商议好了,然后才能租赁地皮,支搭棚帐,竖立高杆,鸣锣响鼓地开棚,马戏棚外掌柜的往门里一坐,游逛的人来看马戏是进门买票。如不买票,那“坎子”们得认识才成哪,如若把出簧来(看出来),不买票的人是官界人,或是本地的人物字号,或是本地的泥腿光棍,点头打个招呼就进去了。江湖的生意人要看马戏是不用花钱的,到了门上得向他们坎子们调(diào)个侃儿(说行话),虽不认识也能不拦挡,放进去白瞧白看。据我调查得来的情形,有江湖人要看马戏,与“坎子”们都不认识,走到门前冲他们先说:“辛苦!”倘若遇见好说话的“坎子”成了,就能进去白瞧;如若遇见难说话的坎子们,净说辛苦是不成的,必须得按着规矩向他们坎子说:“辛苦了,我敲一托(我白看一回)。”才能不买票白瞧白看。按着面子道个辛苦,那是江湖人普遍的礼节。如若拉洋片敲打锣鼓唱了一大套曲儿,围了许多的人,他往凳上让座,赶巧了都僵住了没有一个人坐的时候,他必说:“人无头不走,鸟无翅不飞。千人走路,一人领头。哪位做个人中的领袖,将中的魁元?”他嘴说着,手指着,让谁谁摇头,让不下瞧主,没法子啦,向附近的江湖人调(diào)个侃儿(说行话)说:“我的口儿说搬了(管说完了挣不下钱来调侃儿叫搬了),你来给敲一托(白看一回,当观众)吧。”那附近江湖人按着江湖的义气,就得装着看洋片的,到了洋片箱子的前边凳上一坐,给他当敲托的(即是贴靴的[同伙]意思)。社会里的事儿也真奇怪,只要有一个人看,都坐下来看;如若没有人给他敲这一托,真就没有人看。故此老合(江湖艺人)们对于敲一托是欢迎的。马戏棚买卖虽用不着敲托的,老合们要向他们说“辛苦了,敲一托”,也是欢迎的。

各省市各商埠码头的坎(kǎn)子(收门票的人),都是本地的人们才干这行哪,如若马戏班子不肯牺牲这种利益,本班自带坎子,人生地生(本地人物字号、泥腿光棍、当地官人,全都不对盘儿,不认识,看不出来),净打架争吵,就不用挣钱了。外来的人任你有多大的本领也是干不了这行的。俗谈“强龙不压地头蛇”,细考查起来,那句话诚然不假,并不是瞎说的;“远来的和尚会念经”,本乡本土的人,要想唬本地的乡亲也是不成啊。如若遇见了外乡人,长得再有个人样,穿得再阔绰,真能唬得住人。可是外来的坎子要唬事是不成的。我说这话诸君不信,可往各马戏棚去看,坎子上的人准是本地人。还有那戏头棚(江湖中管玩猴、大蟒、大象的走兽棚调侃儿叫戏头棚)、腥棚(江湖人管弄那三条腿的大狼、六条腿的牛调侃儿叫腥棚),到了各省市商埠码头,也都得用当地的坎子给他们把门儿。那种情形与马戏棚相同,不用赘言。只是那二八成儿均杵(管二八下账,坎子拿二成,马戏团拿八成,叫二八下账。分钱调侃儿叫均杵)仍是一样的。靠河的吃水,靠山的打柴,一方水土养一方人,江湖的事儿也是如此呀。





\chapter{第二章 算卦相面}


江湖之金点


“金点”是江湖艺人管算卦相面的总称,如同一种群名词似的。譬如甲乙两个江湖人在路上相遇,甲问乙:“你做什么买卖呢?”乙回答:“我做金点哪。”甲便知是以算卦相面为生哪。故江湖人管算卦相面的行当儿调(diào)侃儿叫“金点”。在这“金点”里,尚有“哑金”、“嘴子金”(用鸟儿叼幅子)、“戗(qiàng)金”(相面的)、“袋子金”、“老周儿”等等的分别。





哑金


在各市场各庙会常见有一种相面的先生,坐在地上装哑巴。在他那摊子上有个玻璃镜框儿,内写“哑相”二字,或写“揣骨神相”四字。又在摊上写着:“坐地不语,我非哑人。先写后问,概不哄人。父母双全,父母不全,兄弟几位?妻宫有无?有子无子?子宫几位?”看哑相的先生便在摊上盘腿一坐(做这种生意都是地摊,按江湖人的规矩是不准使高案子),用手指点行人“圆粘(nián)儿”(使游人围着他观瞧,调侃儿叫做圆粘儿)。游逛的人们见他装哑巴相面是为一怪,便都围着瞧着。做这种生意的人必须能“戳朵儿”(管写字调侃儿叫戳朵儿),才能使得上“拴马桩儿(用话留你,让你走不了)”。还是倒“戳朵儿”(写挺好的一笔倒字),叫人看的懒得走啦,即是拴马桩子将人拴住了。

敝人曾看见他们在一块板上写“奉送手相”四字,写完了抬起头来,冲着观众“把(bǎ)点儿”(瞧着哪位像花钱的,调侃儿叫把点儿)。譬如,看出这人面貌,便能知道这人的事情如何,调(diào)侃儿叫“把(bǎ)现簧儿”。把现簧儿不外乎由人的脸上察看“喜怒忧思悲恐惊”七个字的秘诀。例如某甲在商家做事,与同事的伙伴不和,有心辞事不干,还没辞哪,跟柜上告一天假,到各市场游逛散闷,要站在哑相摊前,面上必有忧容。相面的先生把出他的“簧头”(面带愁容,有为难事)来,冲他写“白送手相”。某甲伸出左手来,相面的冲他脸上一看,往某甲手掌上倒写四个字:“二虎争食。”某甲想他同人不和却像二虎争食的意思,他面上必显出一点笑容来,相面的先生就知道“簧头”对了。冲他往板上再写“你可相相面”?某甲问:“花多少钱呢?”相面的先生写出“四角钱”,在他犹疑之间,相面的先生便由他腿底下拿出一小沓纸条来,长约三寸,横有一寸多宽。先把这沓纸叫人看看,上头没字,名叫“亮托”(亮出做生意的货),然后冲某甲面上一看,往纸上写上几个字,在他写这几个字的时候,封得很严,不能被人看见,名曰“护托”。写完之后,用手指着他摊上写的那“父母双全、父母不全”问某甲,某甲说:“我父母不全。”相面的先生把他左手攥着的纸沓儿亮给大家瞧,某甲与大众往他纸上看哪,真写的是“父母不全”!不明白江湖术的人们都得惊异了。然后再用纸写吧,什么“妻宫有无?兄弟几位?子宫几位”,无一事不对。某甲不由得自己掏出四毛钱来!

在各市场各庙会常见有一种相面的先生,坐在地上装哑巴。在他那摊子上有个玻璃镜框儿,内写“哑相”二字,或写“揣骨神相”四字。



在敝人不明江湖事的时候,总想他那一小束纸条上写的事事都对。有一年在天津遇一位江湖友人×君,我向他问过哑相是怎么个生意,他告诉我是这……回事,我才明其究竟。

原来看哑相的先生们使的那小束纸,调(diào)侃儿叫“跟头幅子”,这跟头幅子是四层儿,未用之先,在各层张之上预先写得了“父母双全、父母不全、兄弟几位、妻宫有无”的字样,四层纸共为八面,有七面写好了字的,剩下一面随用随写,使用的时候,必须“护托”(即是不叫人瞧见的意思)。把手中的一束纸,按层翻着使用,故此调侃儿管他叫“跟头幅子”。做这种江湖的生意(又名念语子金,哑巴调[diào]侃儿叫念语子),必须先把跟头幅子像变戏法儿似的练好喽,运用自然了,然后才能上地做生意。可是一样,作哑金的就怕遇见弟兄十二个人,将跟头幅子翻碎了也翻不出一张兄弟十二位呀。在清末民初的时候,做这种生意的还能蒙住人。到了现在呀,也是落了伍的生意了。哑金这种生意永远是撂地儿,不能“安座子”。什么叫安座子呢?凡是算卦相面的先生,不论在何处开设了“命馆”即是“安座子”,各市场庙会的座子都是使“老周儿”(六爻卦),“八岔子”(奇门卦),“拆朵儿(测字)治杵”(江湖人算卦挣钱调侃儿叫治杵),还没有使“跟头幅子”的哑金安座子的事哪!





金点中之戗(qiàng)金(相面的)


江湖上相面调侃儿叫“戗金”的,又叫“戗盘”(盘当脸讲)的,这种生意在金点这一门里数它最难做。第一,相面的先生要长得相貌堂堂,气派要大,凭那人样子,再“挂洒火衫”,即是穿着阔绰,在地上一站就能唬得住人,调侃儿叫做“人式压点(yā diǎn)”(震得住人为压点)。个中的意义即如唱戏的角色一样,必须有台风才能警人。第二得要“碟子”利落(即是唇齿之能)。第三得有“夯(hāng)儿”(即是有嗓子)。有三样特长,然后才能拜师入门,习学“戗(qiàng)金”(相面的)。若是没有这三大特长,干了这行也是仅顾衣食而已。

投明师访高友,是生意人学能为的秘诀,凡是能够换钱的生意人,都是受过好“夹磨”(jiá mo)(生意人管得过师傅真传授调[diào]侃儿叫受过好夹磨)的。有些个老学究们,在少年的时候正赶清末之际,读过《易经》。常言说,读过《易经》会算卦,他们到了无事可做的时候,就弄个签筒子,六爻卦盒,再有《渊海子评》、《卜筮(shì)正宗》、《万年历》、《麻衣相》、《玉匣记》往卦摊上一摆,坐在卦摊的后边死鱼不张嘴,等主道候客,又不会“圆粘(nián)子”(招徕观众),又不懂得“要簧”(要出实话来)、“把(bǎ)簧”(看出人的底细),又不会要钱,成天价在卦摊后边坐着发愣。要想挣钱哪,简直地说吧,是办不到的。江湖人管这种人调侃儿叫“空(kòng)金点”,又叫“死空子”(不懂江湖内幕的人)。这种傻念书的就是“攥(zuǎn)尖”(江湖人管真能熟读相书、卜筮等书调侃儿叫攥尖)。不会使腥儿(假的),休想能够治杵的(即是不能挣钱)。生意人虽投师受业学习使腥儿,可也得懂得真的,也得熟读卜筮星相各种书籍,给人算卦相面的时候,心里使的虽是腥儿,嘴里可要尽说书理,名为“腥加尖(假的加真的),赛神仙”。又说“相儿一包(一个小包,内有手巾条、铅笔一根就够了),空子一挑(tiāo)”,江湖人管最有能为的生意人,称为相儿。凡是相儿,平地抠饼(全凭本事挣出钱来),讲究的是手巾一条,铅笔一根,站在玩艺场,凭唇齿之能圆粘子挣钱。若是摆个卦摊,用的东西物件多了,摆着费事,运着也难,生意人讥诮他是“空子一挑儿”。

相面的先生如有真传授,就能挣钱。真传授有五:一曰“前棚”,二曰“后棚”,三曰“玄关”,四曰“炳点”,五曰“托门”。什么叫前棚呢?就是凭着他那玩艺儿场中一站,用嘴一聊,就能叫游逛的人们围着他不走,这种能为是第一手,叫做“圆粘(nián)儿”。圆好了粘子(聚好了观众)再用“韩信乱点兵”之法。什么叫乱点兵呢?用这种法子,就能把人拢住不走,又像拴马桩儿(用话留你,让你走不了)。他向围着的人们说:“别看咱们这场围着的人不多,内中的事儿不少,我用眼一看,就能知道谁有什么事。内中有两个人要找事做,还没有找着哪!内中有一个人心里不大痛快,要和别人打官司。内中有一个人心里很烦,他家里有个病人。内中还有一个人气色不好,正犯口舌。”他嘴里说着,眼睛不住地往大众脸上瞧着,这叫“观色”,又叫“把(bǎ)簧”(看出人的底细)。譬如某甲正要和人打官司,他听相面的先生说,这些人里有个人要打官司哪,他以为是说他呢,不由得心里佩服这位先生相法高明,心里一动,脸上就显形儿。相面先生见某甲脸上显形儿,就将簧头(实情)把过来了,然后就说:“今天我还是不要钱,奉送相法,可不能全都送,就送七位。聋子不送,我说什么话他听不见;哑巴不送,我说什么他不知道;小孩不送,我说什么他也不懂。咱们有个主意,我有七个纸条儿,谁要愿意叫我白送相法,谁伸手,接着一张纸条,便算有谁一相。接着的也别喜欢,接不着的也别恼。”说到这里,他就散放纸条儿,围着的人都抢着接他的纸条儿,某甲也接了一张。他送的时候向某甲先问:“你是哪县的人呢?”某甲若说:“我是房山县周口的人。”相面的先生就向某甲说:“我看你的气色发滞,印堂发暗,目下你要和人家打官司,对不对呀?”某甲说:“不惟先生你相得对,我还求先生细给我看看,我这官司打得能不能赢?”相面的先生说:“先不用告诉你官司输赢,我先给你相相你是为什么事打官司,叫大家看看我的相法如何。”某甲说:“你看看我为什么打官司吧。”相面的先生说:“你的气色犯小人,二虎争食。”某甲拍掌顿足地说:“真对,真对。”阅者看我写到这里必然也纳闷儿,他们相面的怎么会相得这么对呢?这可不是他按着相书用的功夫,看出来某甲要打官司,这是他们使腥儿(假的)要的簧头儿。阅者若问他们要的是什么簧头儿,我先向阅者诸君谈谈。相面的先生问某甲是哪县人,那不是问哪县的人,是要“地理簧”哪。什么叫地理簧呢?我先向读者诸君解释明白。我中国的地方很大,在早年清初的时代,是南七北六十三省,到了清末的时候有二十二省之多,四万万人民,都有一定的职业。可是一县有一县的特殊职业。譬如山东章邱的人,在家乡是种地务农啦,若是出门做事,有两个途径,他们的同乡在我国各省市、各商埠码头绸缎行做事的人很多,十有八九在祥字号做事。他们章邱县的人若在二十岁里外出门做事,都找他们的乡亲,同乡就能把他们荐在绸缎店里学徒。到如今祥字号的买卖外县人是很少的,都是他们本乡本土的人了。章邱人如若不愿奔绸缎行,还有一条途径就是打铁,当铁匠的人吃的道远道宽,就数着章邱人了。可是也有不奔那两条路的,干别的行儿虽有,也是百里有一。相面的先生若能明白章邱县这种情形,就是他懂章邱县的地理簧。设若章邱人找相面先生谈谈相,相面先生只要一问他们,你是哪里人呢?他说出章邱县三个字来,就能知道他做什么事,穿的衣服干净利落,就是绸缎行的;穿的衣服不干净,就是打铁的。相面先生不用按着相貌上的五官看,就以他是哪里的人接着地理簧的情形,就能知道他是哪行的人,做的什么事。如若告诉他,我看你的相貌应当入商界,他准能佩服相面先生是有功夫的。这种地理簧是江湖金点十三簧里第一簧啊!我详细地解释这县的地理簧,阅者诸君便能了然个中的意义,其余各地勿庸如此絮烦,简单地谈谈,阅者便能尽知其详。各地出产是一个地方一样,人做事也是各有一行。譬如,山西汶水县的人,都是在干果子铺做事的居多;山西榆次县的人,是粮行居多;山西五台人,军政界做事的多;山东烟台福山县的人,饭庄子做事的多;山东胶州人,在北平这地方说,在西四牌楼吃油肉行的多;山东曹州府的人,在军界入伍的多;直隶定兴县的人,是澡堂子、煤铺做事的多,干别的事儿虽有,可是很少。算卦相面的如若不懂地理簧,是不成的。若是见了山西人说是唱二黄戏的,那就不用挣他山西人的钱了。

那么某甲告诉相面的先生是房山县周口的人,按着地理簧说是应当如何呢?据敝人所知道的,那个地方的人十有八九都在煤窑上做事的,按着“现簧”(江湖金点管明白人现在心里有什么事调[diào]侃儿叫现簧)说哪,凡是有矿产的人都免不了争夺的,揣情度理,他要没事,不能来到北平的。北平的最高法院是管附近二十县的,他猜着某甲来北平是上诉的,说某甲的气色犯“二虎争食”,某甲称为神相,是对了他的现簧了。房山县的诸君不要错会了意,敝人这种说法是借题说话,并不是褒贬贵处的人哪!务希原谅是幸。这现簧(江湖金点管明白人现在心里有什么事调[diào]侃叫现簧)是金点十三簧里的第二簧。生意人要明白这第二道簧,较比懂得地理簧儿还难上一层。某甲若是佩服相面的先生了,一定得问他:“你看我打官司是输啊,还是赢呢?”相面的必说:“看你这气色很不好,轻者伤财,重者有危险。”某甲一定得害了怕。他们金点管用话吓唬人叫人害怕调侃儿叫“扣瓜”。他把瓜扣上了,某甲心里害了怕,若再问他:“先生你看我的官司究竟是输是赢?”他就不说了,又给别人白送相了。某甲因为叫他扣上瓜了,准站在那里不走的,等着花钱谈相了。相面先生施展他们的手段,某乙相几句扣上瓜,某丙相几句扣上瓜,有七八个人“顶了瓜头啦”(即是有七八个人害了怕啦),他就要“插幅子”了。什么叫插幅子呢?相面的先生说:“真金不怕火炼,好货不怕试验。我送几句相法,是叫大家听听我的相法如何。送相就是几句,若是谈相可就多了。一辈子吃喝穿戴,衣禄食禄,父母死亡,兄弟几位,妻宫克不克,有无子嗣,几个儿子送终,得济不得济,士农工商应入哪界,富贵贫贱,穷通寿夭,为人脾气秉性怎样,少中老三步大运,哪步运好,哪步运坏,详详细细地把一辈子事都谈尽了,那才叫相面哪!那么要向你们谈相,要给多少相礼呢?黄金有价艺无价,我谈相是一块大洋。今天哪,我可不为挣钱,我为的是传名。常言道,人过留名,雁过留声。人过不留名不知张三李四,雁过不留声不知春夏秋冬。我为的是传名,今天谈相不要一块钱,每一相就收两毛钱,若是都谈相我可谈不过来。特别优待,为的传名,咱们是多了不谈,只谈八相。我这儿有八张纸条,哪位乐意谈相,哪位伸手,谁接着我的纸条有谁一相。接着也别喜欢,接不着也别烦恼。过了八位之后,如若再有人谈相,我还是要一块相礼,也许你不谈,也许我不相。哪一位要明白终身大事、富贵贫贱、目下的月令、吉凶祸福、进退方针,就接我的纸条。”说到这里他就散他的纸条,说:“哪位愿相,哪位接个纸条吧。”这时候别人还许怀疑,那被他扣上瓜的几个人就各自伸手接他的纸条儿。等到把纸条儿散完了,“戗(qiàng)金”(相面的)的生意前棚的事算完了。挣得下钱来,挣不下钱,还得看他后棚的能为了。江湖人管散纸条儿调侃儿叫“插幅子”。等到把幅子插出去了,才能“乍角(jiǎo)子”(管板凳调侃儿叫乍角子)拉开,叫“点头儿”“迫(pǎi)下”(江湖上管花钱相面的人叫点头儿,管坐板凳叫迫下),等到点头儿都坐稳啦,他就一点头儿“逼杵”(即是要钱)了。他向众人说:“相面可是先交相礼,相礼放在那里,相对了是我的;相不对了分文不取,毫厘不要,原钱退回。”于是向某甲、某乙挨着个儿将相礼要过来,都放在一处。这种钱虽到了手,还不能算完哪,还得再要钱哪。点头儿虽然花了两角钱,到了江湖人手叫做“头道杵”,此外还有“二道杵”、“三道杵”、“绝后杵”(最后一笔钱)。要想往下要二三道杵、绝后杵,得会使“抽撤盘簧”了。就是用一种圆滑的口吻,乍听很有理,说出话来能进能退,要出点头儿的实话。还有使“连环朵”的。在早年使用的旧法子连环朵,搁在如今可使不上了。在早年的人知识简单,最容易蒙哄。敝人先将早年使用的连环朵写出来,贡献阅者。然后再向阅者写出新的方法。

譬如,谈相的人向他问:“先生你看我有媳妇没有呢?”相面的先生就用笔在纸上写六个字:“鳏(guān,没有妻子或妻子死亡的)居不能有妻。”写完了这六个字,反向谈相的人猛势儿问道:“你倒是有媳妇无有呢?”这人说:“我有媳妇。”他就用手指着那六个字念道:“鳏居不能,你这人是不能鳏居的。”又往下念那两个字:“有妻,你是有媳妇的人。”这人便信服他相法有准,很是高明的。设若这人说:“先生,我没有媳妇。”他就用手指着那六个字念道:“鳏居呀,你这人是鳏居。”又用手指着往下念那四个字道:“不能有妻。我早就看出来了,你这人是鳏居呀,不能有媳妇。”这“鳏居不能有妻”六个字,说有媳妇也成,说没媳妇也成。江湖人调侃儿就叫“连环朵”!还有人向相面的先生问道:“先生,你看我父母在不在呢?父母全不全呢?是都活着哪?是都死了呢?”他用笔在纸上写十个字,写的是:“父母双全不能克伤一位。”这十字分开来念,怎样都对。他写完了这十个字说:“你父母在与不在,是双全不双全,我都写出来了,你说吧。”这人说:“我父母双全,都在着哪。”他便用手指着这十个字念道:“父母双全。你看我这儿写着哪,是父母双全,你爹妈都活着哪。”又用手指着六个字念道:“不能克伤一位。你父母连一位都不克伤,对不对呢?”这人真能佩服他。譬如,这人说:“我父母死了一位,活着还有一位哪。”他用手指着那十个字念道:“父母双全不能。你这人的相貌,父母双全不能。”又用手指着下边的四个字念道:“克伤一位。你把你父母克去一位。”这人还不信服他吗?譬如,这人要说:“我父母都死了。”他用手指着那十个字念道:“父母双全不能,说你这人父母不能双全。”又往下指着念道:“不能克伤一位。要克你父母啊,还是克伤两位哪!”这十个字的连环朵能有三种念法,也很神秘。还有两个五个字的连环朵儿。譬如,要向他问:“先生,我父母倒是死了一位,在着一位,你能知道我父母死的是哪一位吗?”他用笔在纸上写了五个字,写的是:“父在母先亡。”写完了他问这人:“我这儿写好喽,你说你是先死的哪一位吧。”这人说:“我父亲先死的。”相面先生用手指着这五个字念道:“父在母先亡,你父亲在你母亲之先死的。”如若这人说:“先生,我母亲先死的。”他也指着这五个字念道:“父在,你父亲在着哪,父在嘛。”又往下念那三个字道:“母先亡。你母亲先亡,就是你母亲先死的。”这五个字的连环朵儿就是这样的用法。设若谈相的向他说:“先生,你看我有儿子没有呢?”他用笔在纸上写六个字,写的是:“命独不能有子。”写完,他问点头儿(花钱相面的人):“你有儿子没有呢?”这点头儿说:“我有儿子。”他就用手指着那六个字念道:“命独不能,你这个人有儿子,不是命独啊!”又指着那两个字道:“有子,你是有儿子的。”譬如,这点头儿说:“我没有儿子。”他就用手指着六个字念道:“命独,你这个人命太独。我这儿写着命独,你不能有儿子。”又用手指那四字念道:“不能有子。”这六个字的连环朵儿就是这个用法。譬如,点头儿向他问:“先生,你看我有几个儿子呢?”相面的用笔在纸上写上八个字,写的是:“一位有子不能二三。”写完了他问那点头儿:“你有几个儿子呢?”这点头儿说:“我有一个儿子。”他用手指着那八个字念道:“一位有子。你要有了儿子是一位,就有一个儿子,我看出来了。”又用手指着后面四个字道:“不能二三。你不能有两三个儿子。”譬如,点头儿说:“我有两个儿子。”相面的用手指着那八个字说:“一位有子不能。你这人有儿子,不能是一位。”又念那两个字道:“二三。你有儿子或二或三。”譬如,这人说:“我有四个儿子。”相面的用手指着那八个字道:“一位有子,说你这一位可有儿子。我这儿写着哪,一位有子,你这位有子。”又用手指着那四个字念道:“不能二三。你有儿子不能是二三,一定是四五个呀。”这八个字的连环朵儿,就是这样用法。譬如,这点头儿(花钱相面的人)向他问:“先生,你看我弟兄几位呢?”相面的用笔在纸上又写了八个字:“昆仲一位不能二三。”写完了问那点头儿:“你哥儿几个呢?”点头儿说:“我弟兄一位。”他用手指着那八个字道:“昆仲一位,你是哥儿一个。”又用手指着那四个字念道:“不能二三。你不能哥儿两个、哥儿三个。”譬如,点头儿说:“我哥儿三个。”他就用手指着那八个字念道:“昆仲一位不能,你这人不能是哥儿一个。”又念那两个字说:“二三。不是两个,就是三个。”这种连环朵儿若是在庚子年前后使用,社会里的人们都很诚实,点头儿能够花钱,围着瞧的人能够把(bǎ)不出腥来(把不出腥来即是看不出假来)。到了近来,社会里的人士全都开化了,“戗(qiàng)盘”(盘当脸讲)的金点儿若是还使用这连环朵儿,这点头儿不醒攒(cuán)儿(不醒攒儿是心里不明白),那围着瞧的人们也能把出腥来,也能给他豁鼻子——说破了,给他搅得治不下杵来(挣不下钱来)。

现今社会里的人们知识进化了,那生意人挣钱也就难了。但有的生意人比早年挣钱反倒更多,江湖人的生意方法也随着社会的风气大有进化了。阅者如不相信,你走到前门里外准瞧得见。有些个撒传单的人往那坐洋车的人怀里递传单,那传单上印着:“×××大相士到平,现寓××饭店三层楼上十八号房。”他那相法与众不同,有八大特色,录之如下:“一能知士农工商哪界做事;二能知父母妨与不妨;三能知昆仲几个;四能知妻宫有无,贤与不贤;五能知子嗣有无,何年立子,送终有几;六能知目下吉凶祸福;七能知现在所谋,问事成与不成,指定进行方针;八能知祖业有无。”后边还印着:“如不灵验,分文不取。谈相五元,暂取两角,每日只谈三相,过三相仍收五元相资。时间每日上午九时起至下午四时止,过时不候。”下署一班介绍人名,都是要人政客,或是社会中的闻人。不知内幕的,真不知他是什么人物。敝人在民国十年以前,走在前门,曾接到一张传单,上面印的是××佛大相士谈相八大特色,敝人好奇心盛,要豁出几毛洋去谈谈相,找到了旅馆,向茶役问明号头,进到了大相士的临时相馆。屋里摆设得无论多阔,那是人家旅馆的,不足为奇。这位先生长得方面大耳,身体肥胖,穿着阔绰,好像个大富贵的样子,一嘴的文明词儿,谈吐文雅,凭他那“人式”就很“压点”(yā diǎn,震得住人为压点)。桌上放的润格是八寸宽二尺多长的玻璃框儿,内里宣纸写上八个大字:“贫不计利,富贵加增。”那些小字写的是:“粗谈相法一元,中谈相法五元,细谈相法十元,细谈流年三十元,细谈终身五十元,大富贵相百元。粗批八字两元,中批八字五元,细批八字十元,细批流年五十元,流年加季六十元,流年加月八十元,流年加节一百元。趋吉避凶。重要方针,临时面议。”我看那“杵门”(价目多寡,江湖调[diào]侃儿叫杵门)开得觉着心惊不安。落座之后,有伺候大相士的茶役递给我一根三炮台的香烟,又倒了一碗热茶,那热气扑出来喷鼻儿香。那位相士向我问了问贵姓,恭喜。我喝下他那碗茶去,了不得啦,肚子里头轱辘直响,叫那碗茶打得心火下去,几乎中气不接。我抽了他那根三炮台的香烟,这位大相士才问我:“你是谈谈相吗?”我说:“不错,正求先生指教。”他用手往桌上一指,吓了我一跳,那桌上有个木板,写着:“已过三相。”我猛然想起他们的章程是:谈相一元,临时暂收相资两角,三相为止。如今他叫我看已过三相,那是告诉我,你要谈相呀,至少也得花一元钱的。我虽明白他这个门子,哪时有人来谈也是过了三相。本来嘛,人家住的旅馆,敬客都是三炮台的香烟、上等的香片茶,挣你两毛钱,还不够人家喝水的哪!此刻,有心不谈相吧,又怕人家“吾攻(wú gong)”(江湖人管不愿意、恼恨人调侃儿叫吾攻)我,幸而我前天当了五元钱的衣服,腰里还有三元多。我低声下气地说:“粗谈谈吧!”于是这位先生指着我的五官,如同法院过堂似的说了几句,我赶紧掏给人家一元钱。幸而没把当票露出来,若是把“拱页子(gǒng yè zi)”(即是当票)露出来,人家心里还不“钻(zuǎn)钢”(江湖人管明白了调侃儿叫钻钢)啊!我没叫他们敲上,开了开眼界,花了一块大洋。若有块洋钱到了天桥谈相啊,能够谈十次的相,十位先生给我细谈终身哪。如今生活程度日高,江湖的金点也随着潮流能挣大洋钱了。





金点的水火簧


相面的先生要想能够天天挣钱,必须懂得“水火簧”。什么叫水火簧呢?江湖人管几句话能套出人的穷富来调(diào)侃儿称为“水火簧”。做金点的人若是不知人家是穷是富,怎样挣钱哪?他们可不是势利眼,不瞧人家的穿着,有些人家无恒产,连个职业也没有,你别管他是坑蒙拐骗,到了什么时候,应时当令的穿什么,到了冬天也能穿上细皮袄,水獭领子大氅(chǎng)(大衣),水獭皮帽,由头上到脚下真能值个一二百元。你要问他是干什么事的,人家是耍人儿的。相面的先生遇见了这种人,若说他是富贵人,不唯他不信先生的相法,也就不用挣他的钱了。乡下的土财主到了,别看他有几十顷地,开着几个大烧锅,到了冬天,在家中就穿个蓝布棉袍,出来有事应酬亲友,也就穿个灰布皮袄,由头上至脚,衣帽鞋袜都算上也值不了十几块钱。别看他的穿着儿不阔,家里的产业可有的是呀。相面的先生遇见这种人,要说他是个穷人,他如何能信?也就不能挣他的钱了。也有那有钱的人好穿好衣服,也有那穷的穿不齐全的。总而言之,相面的先生要瞧人的穷富,是不能以衣帽取人的。

我谈的这“水火簧”,是一见面儿和谁谈上几句话,就能够知道谁是真穷、真富,还能知道谁是先贫后富、先富后穷,穷了多少年,富了多少年。我将这“水火簧”的用法写出来,阅者便知其详。譬如有人到了相面的面前说:“先生你给我相相面。”这先生就问:“你今年多大年岁,你媳妇多大年岁?”这人如说:“我今年三十二岁,我媳妇今年三十五岁。”相面先生听他所说他媳妇比他大三岁,就说:“按你这人的相貌,在幼年的时候运气很好,祖上根基不错,能够承受祖上的产业。”这人真是幼年的时候运气好,家中有祖上的遗产。他听相面的这样说法,一定信服他相得很好。阅者若问,他怎么知道这人是如此呢?我向阅者解释几句,阅者便能了然“水火簧”的奥妙。相面先生问这人多大年岁,这人告诉他三十二岁,也没什么关系。他问这人的媳妇多大年岁,这人告诉他三十五岁,由他媳妇比他大三岁,就能推测出穷富来了。我国的不良风俗就是早婚。有钱的人家是愿意子孙众多,人口昌盛;没钱的人家是怕人口多了无法生活。大凡有钱人家,十有八九都是财旺人不旺的。有了男孩,不等孩子长大成人,到了十三四岁就给儿子娶媳妇,甚至于有十一二岁就娶媳妇的,最晚不过十六七岁。可是孩子年岁小,娶个媳妇不能很小了,怎么也得比少爷大个三四岁,十三四岁的少爷娶个十七八岁的少奶奶。少爷岁数小不懂事务,少奶奶十七八岁,女工针黹(zhēn zhǐ)(针线活),做菜做饭,伺候公婆,样样都得能成,故此有钱人家早娶儿媳妇有两样好处:又能早抱孙子,又能有人料理家务。可就忘了少爷身体没长足壮,早娶媳妇,伤损身体了。早婚之害是说不尽的。江湖人不是净骗钱财呀,人情世态、社会学,都有深奥的研究啊!就以这早婚人家能推测穷富的“水火簧”来说吧,准能够用得上,百试百验的。故此相面的先生学会了水火簧,有人来谈相,先向人问:你多大年岁了?令正夫人多大年岁?谈相的人哪能知道这些事,绝不知道他是要“水火簧”啊。若遇有钱的人,在他父母在世的时候家道兴隆,都是早娶媳妇,告诉先生他三十二岁,媳妇三十五岁,他说出来不觉悟,相面的先生可就明白了,他是“火码子”(管有钱的阔人调[diào]侃儿叫火码子)。譬如,相面的先生遇个谈相的人,长的约有三十七八岁,穿的衣服阔绰,问他多大年岁?他说三十七岁。问他令正夫人多大年岁?他说十九岁。相面的先生就能推测出来,他早年家境不好,他父母没有力量给他娶媳妇。直等他自己学好了能耐,能在社会里做事挣钱养家了,才娶上媳妇。他女人家还不是冲家当给的,而是冲他有能耐给的。有些人明白世故人情,养活姑娘要说婆家,宁给有能耐的姑爷,也不给有产业的。有产业的人家有儿女都是溺爱,别看他家有产业,还不一定守得住哪!只要姑爷他有能耐,比姑娘大几岁都不在乎,姑娘过了门,绝不能跟男人挨饿。

凡是没钱的人家,有儿子也不能早娶媳妇,一者没钱娶,二者娶过来也没钱养活。就是父母给儿子张罗媳妇,他儿子年纪小又没学出什么能耐,又瞧不出准有来历,说媳妇也是没人给。所以相面的先生遇见有人来谈相,如若说问他是三十七岁媳妇才十九岁,准知他是个“水码子”(江湖人管没有产业的人、贫寒的人调[diào]侃儿叫水码子)出身,就说他少运不好,祖业不靠,自创自立。他听了准佩服先生的相法高明,说他走了二三年的好运也能对的。以他三十七岁媳妇十九岁推测,他娶媳妇也就是二三年,绝不是六七年的。若是六七年,他媳妇才十一二岁,哪能娶呀?可是续弦填房者另说,不在此例。这是相面先生所用的江湖术中金点十三道簧里的水火簧。这种说法是在点头(花钱相面的人)本身用的,还能往深里用哪。若问他祖父多大年岁,问他祖母多大年岁,也能知道他祖父母当初穷富。如若点头说他祖父六十一岁,祖母六十四岁,要是按着水火簧推测,他祖父是十三四岁娶的媳妇,当年他家是有过家产的阔家呀。如若说他祖父八十一岁,祖母五十三岁,按着水火簧推测,他祖父就是个穷光蛋出身了。任他本人多阔,他祖上也是贫寒人家。譬如点头儿说他父亲五十三岁,他母亲五十六岁,按着水火簧推测,他父亲的少运也是不错呀。若是他父亲六十岁,他母亲五十岁,按着水火簧推测,他父亲少运不好,晚娶妻,也是没有祖业自创自立的人物了。这是水火簧的深奥之法,能推测人三辈子穷富。可是,这样的推测在那个时代使用行了,若在大清庚子年前后,就不能这样使用的。八旗人家家中虽没有恒产,少爷有十八九岁,在弓房学会了拉弓射箭,赶上旗里出缺,挑(tiāo)缺的时候,一马三箭射中了,便能每月关(发)几两银子的旗饷,一年四季能领到老米,或许有人冲他得了钱粮能给他个媳妇。若在那时代遇见八旗的人,用水火簧又不能按着现在的推测法了。彼一时,此一时。江湖艺人金点的水火簧,也是随着时代性变迁使用的。江湖人对于世故人情也是按着国家的制度、社会的变迁来研究的。他们的研究法是深入社会的,是深入农村的,绝不是闭门造车、关上门研究啊。多值得人钦佩!





诸葛数灯下数(shù)即是袋子金


在民国二十四年夏天,老云有事出外至大连,寓于浪速町某客栈中。一日,独自闲游,闻大连西岗子为露天市场,比津市之三不管(天津市南市的一个露天市场)、奉天之小西关、保定之马号还格外热闹。信步而行,不到一个钟头即至,锣鼓喧天,嚣嚣震耳。各种杂技场、戏法、相声、鼓书、杠子、竹板书、评书、洋片、无不齐全,热闹可观,各处巡礼,赏心悦目,精神奋发,游兴颇浓。行至某油坊大墙角下,见有数十人围绕,面向里观瞧,既不敲锣,又不击鼓,不知是何玩艺儿。好奇心盛,我挤进人群一看,见有一张桌子,上铺白色毡子一个,毡有毛笔一支,砚墨一份,石板一块,粉笔一支,桌上有四个纸袋,袋长四寸,宽约二寸。有三个袋子上都写着“奇门遁甲”的字样。那一个袋子上没写奇门遁甲的字儿的,写:“〇〇〇年〇〇〇岁〇〇省〇〇县人〇〇月〇〇日〇〇时生报花”,这是两行小字,在两行小字的后边,还有“父母〇〇兄弟〇〇妻妾〇〇子女〇〇”格式表儿。我看这摊子上设摆的东西,就知道这是个算卦的摊。抬起头来一看,在桌后靠墙儿立着个人,长得又黑又高,一脸的麻子,约有四十多岁,他手里拿着个小竹筒儿,筒内有三根小棍儿,不住地用手摇晃那竹筒儿,嘴里还说:“在咱们这卦是与众不同,按着人的生辰八字,五官相貌,命相合参,能够知道人的年岁多大,家乡住处,父母妨不妨,兄弟几位,妻妾有无,子女多少,士农工商哪界人,一辈子衣禄食禄,富贵贫贱,穷通寿夭。我这个卦摊多了不算,每天就算四卦,这叫‘奇门遁甲’。”说到此处,他用手一指桌上的四个纸袋说:“我这卦是先算得了等人,应当有谁的卦,袋内有张纸,纸上写好啦。问卦之人姓什么?叫什么?哪省哪县人?父母妻妾兄弟儿女,写好了应有应妨,一世终身,应做什么事?有多大财源?哪年好哪年坏?得谁的好处?受谁的害处?哪位要算,咱们全都写好了,一字不差,你再给钱;算差了一字,分文不取,毫厘不要。哪位要算算,哪位言语。”

说到此处,有一个人说:“先生我算算。算对了,我给钱;算不对了,分文不给。”敝人瞧这说话之人,长的就是个“朗(lǎng)不正”(江湖人管社会里讨人嫌又嘎又劣的人调[diào]侃儿叫朗不正)的样子,那个算卦的先生看他那样子,就说:“我这卦不能是人都算,有谁的卦咱们才算呢!如若没有谁的卦,你给钱我也不算。”说到这里他又说:“怎么知道有谁的卦没谁的卦?用我手中这个竹筒可以问得出来。筒里这三根小棍儿,我摇出一根来才有卦呢,摇不出来可就没卦。”说着他就摇手中的竹筒儿,那三根小棍哗啷啷直响,摇晃了一会儿,那三根棍儿一个也没摇出来。他向那位“朗(lǎng)不正”(讨人嫌又嘎又劣的人)的人说:“没有你的卦。”那个人没法儿,赌气子走啦。我一时好奇心盛,说:“先生,你算算有我的卦没有?”他又把竹筒儿摇动起来,工夫不大,吧嗒一声就摇出一根棍来。他说:“有你的卦。”我说:“有我的卦你准算得对吗?”他说:“算不对分文不取,毫厘不要。”我就说:“你给算算吧。”他将桌上纸袋拿起一个来,说:“这里头就有你的卦,你一辈子的事,全都写好啦,在袋里搁着呢。”我说:“取出来看吧,看对了我给钱。”他说:“等等,先别动,咱们说好喽,你再取来瞧瞧。”我说:“还有什么商量的?”他说:“我那条写得对不对,没法子证明。我这里有块石板,你用粉笔将你的姓名、年岁、哪省哪县人,父母妨没妨,兄弟几位,妻妾有无,子女多少,全写在石板上,然后再将袋里的卦单取出来,你看这单上的字样与石板上写的一样了,我再把你的终身事读念了,该多少钱的卦礼,你就给多少钱。”我说:“这个办法很好,心明眼亮。我不亏心,你不冤人。”他就把石板递给我,我接过石板来用粉笔就写,写的是:“荣式毅,年二十四岁,北京人,父母双全,弟兄两位,妻有妾无,子三女一。”写完了将石板放在了桌上。他用手指着石板上的字念了一遍,叫围着看热闹的人听听,大家都明白了。他伸手拿起笔来,从毡子底下取出一沓纸条来,宽有二寸多,长有四寸多。他说:“我这里有谁的卦,得有号头儿,我记上号头儿。”说到这里他就拿着纸条儿用墨笔写了号头,写的时候不叫大家看见,举着手写,他身后是墙,也没法看见。他写完了冲我说:“你把那纸袋给我吧。”我把纸袋交给他,他将纸袋往号头的纸上一放,忽然说:“我写的号头还没让你瞧见哪。”说着就将纸袋,纸条拿起来,又放下,我看那纸条上写的是“第一千五百十六号”。他说着就将纸袋打开,从里边将卦单取出来放在桌上,我看那卦单上写着:“荣式毅,年二十四岁,北京人,父母双全,妻有妾无,兄弟两位,子三女一。为人性柔怀刚,心高志大,喜于交际,志在四方。六亲冷淡,祖业不靠,自创自立,衣食无缺。少运受父母栽培,早入孔孟之庠,学业有成,做事最早,劳碌早,出外早,乃三早之命;发达晚,立业晚,享福晚,三晚之分。早年做事多难成,难展才志,财运虽有,来多去广,有财无库;中运先难后易,渐渐发达,有贵人提拔,财禧并进,受人器重,家道日隆;晚运有大名,有大利。人口昌盛,福寿绵永,晚年蔗境颇堪羡也。”敝人看完了他这卦单,与我个人的命中所经过的事以及家境均皆相符,毫厘不差,心中很为佩服他的术学有灵有验,那卦单末尾上写着:“中等上级官界官,礼金四元八角。”我看完了吓了一跳,囊中只有大洋一元,向他好言央求,总算通过实行,在他那瓢底下给我记上袋了。

我自从占了这卦以后,逢人便说此事,如遇大的神仙。不意在海参崴那年,有朋友王君,我向他道及此事,他说道:“你遇见‘袋子金’了。”我说:“什么叫袋子金呢?”他说:“给你算卦的那诸葛神数,调(diào)侃儿叫袋子金。”我说:“奇怪,那么灵的卦也是生意吗?”王君说:“除测字、周易、奇门,那是一种术学的尖局的(江湖人管真正的好东西,调侃儿叫尖局的),余者有一多半是生意。”我说道:“生意?怎么他能知道我姓什么,家里都有几口人哪?他那卦单上是先写得了的?”王君说:“你还是没明白过来,那算卦的若要先知道你这些事,那不是活神仙吗?我告诉你吧,他那门子(管闹鬼儿使障眼法叫门子)。你看他桌上放四个纸袋吧,那四个袋是真的,在他身上还藏着个假的,名叫彩袋。那彩袋上有个填写的格式,毛病都在那儿哪!彩袋里装着那卦单,卦单上的字全都是先写得了的,惟有那姓名年岁、籍贯、父母、兄弟、妻子、儿女那是临时现写的。”我说:“就是他有个彩袋,彩袋里有先写成了的卦单,父母兄弟、妻子儿女都是临时写的,我没见他写字呀!”王君笑道:“他叫你用粉笔写在石板上,把这些事都写好啦,他从身上取出一沓纸条,他把那个彩袋就放在了纸条底下,他假说写个号头儿,拿起那纸条的时候,不是往纸条写号,是往那彩袋里填写你的姓名、年岁、籍贯、父母双亲,要不然他这行又称为‘袖儿吞金’哪,戳朵儿(写字调侃儿叫戳朵儿)是他们的能为,工夫小,写的字又快又多。”我说:“那不对,我手里拿着那有卦单的纸袋,他那彩袋与我手里的这纸袋,在什么时候换的过儿呢?”王君说:“那叫翻天印。”我说:“什么叫翻天印呢?”王君说:“他把那彩袋藏在那纸条底下,和你要过手里攥着的纸袋,放在纸条上,那上边的袋没有毛病,纸条底下的彩条有毛病,他说看号头儿,一翻个儿,就把彩袋翻上来了,那个纸袋翻在底下,和变戏法一样。江湖人管这个法子叫翻天印。”我说:“虽然上了当,我也佩服他。”王君说:“你佩服他什么?”我说:“他使门子(管闹鬼儿使障眼法叫门子)闹鬼儿我不佩服,我佩服他就在假装写号头的工夫,姓名、年岁、籍贯、父母、六亲就都写完了。”王君说:“人家吃香东西,就凭写那笔字。”我说:“怎么算六爻卦、奇门卦、测字相面的到处都有,遍地皆是,怎么算诸葛神数袋子金的平常不能多见呢?”王君说:“那是调角(diào jiǎo)买卖(江湖管是非行当调侃儿叫调角买卖)。江湖人真有本领的不干那行。有学问的人,被生计所迫摆卦摊吃饭,也不愿学他那是非营生。是算卦相面的人都恨那袋子金。”





金点之竹金


在前年,我又云游到张家口,走在大桥头儿,见大道旁边有一群人围着看热闹。我云游客挤进去一看,见有个老头儿在当中立着,手中拿着两根竹竿,那竹竿约有五尺长,挺细挺细的。那老头儿向围着的人说道:“在下这算卦与众不同,也不算先天,也不算后天,我这是南海观音卦,管保准灵。我这根竹竿儿每天在观音大士佛像前供着,焚香祷拜。众位如有求财问喜,病人生死,出灾的日期是远是近?问书信何日来到?走失行人,落于何方?能否找着?丢失财物,落于何人之手?自己父母妨与不妨?何年妨父?何年妨母?兄弟几位,能否相依?妻宫贤愚,能否白头到老?子女有无,送终有几?士农工商,应在哪行?一生一世,哪年发达?寿数大小,大限哪年?如若父母死得很早,不知个人生辰八字,我灵竹能够问出你是何年、何月、何日、何时生人。算对了,礼金两毛;算得不对,分文不取,毫厘不要。哪位有意,可以占算占算。有钱难买早知道,人有三不知:是福来不知,祸来不知,死时不知。我这灵竹就能知道,哪位算算?”

这时有个歪戴帽子斜瞪眼的人,约在二十多岁,看他那样子就是玩皮货的,他向那算卦先生说:“我今年二十六岁,是二月的生日,我五岁死的母亲,八岁死的父亲,我不知是二月哪天的生日,你给算算吧,算对了给你两毛,算不对了我可不给钱。”那算卦先生说:“算不对,不要钱。”当时那青年人往当中一站,算卦的先生叫将他两只手放在腰间,手心冲上,然后他将两根竹竿放在青年人的手内,不准攥着,凭其自便。这时候就见那算卦的先生腮帮子一凸,嘴里嘟嘟囔囔,好像念什么咒语似的,然后他用手指着竹竿说道:“这位多大年岁,方才自己说出来,二十六岁,如若他真是二十六岁,你就将两竿的头儿,并在一处。”说到这里,就见那两根竹竿儿往一处就并,竿头儿对竿头儿并在了一处(就这一来,能值两毛钱)。“这位若是二月生日,叫左竿在上,右竿在下,搭在一处。”也真奇怪,那两根竹竿儿立刻就忽悠忽悠地动转,真个儿左竿在上,右竿在下,搭在了一处。然后他指着竹竿又说道:“这位是二月的生日,可不知道是哪天,我由二月初一一天一天往下数,数到三十日为止,如若这位是那天的日子,我数到那天,你就两竿分开。”他说完了,就初一初二地数起来,直数到十三,那两根竿就自动地分开了。那个青年将两根竹竿掂了掂,觉着很轻。我云游客看着也不像竿内灌水银、灌铅的。那青年将竹竿给了他,掏出两毛钱给了先生,笑道:“先生,我真是二月十三日的生日。我说不知道是哪天的生日,那是冤你,我故意地撒谎,试试你这卦灵不灵,果然真灵!两毛钱不多。”他说完了,欢天喜地而去。

敝人看着很不相信,我也要花两毛钱试试,我向那算卦的先生说:“你也给我算算几月的生日。”他问我道:“你几月的生日都不知道吗?”我说:“不知道!”他叫我在场当中一站,两根竹竿往左右手一托,端在腰间。他用手指着竹竿道:“这位不知道几月生日。我由正月往下数,数到十二月止,数到哪月是这位的生日,你就将两根竿并上。”他说完了,用手指着竿道:“正月、二月、三月、四月、五月、六月。”也真奇怪,到了他喊六月的时候,那竿子的头儿并在了一处。我真是六月的生日,不由得我就佩服了,乖乖地给他两毛钱。我回到寓所越想越纳闷,不知他那竹竿卦为何那么灵。我向来不迷信,绝不信他那卦有神相助,至于他那个诀窍有什么奥妙,真是猜他不透。

今春我遇个江湖友人王君,向他讨论此事,王君说:“用竹竿算卦的,说行话叫‘竹金’,做那种生意,学之甚易。若是算周易卦,得下一年多的功夫才学会了《增删卜易》、《卜筮(shì)正宗》,六十四卦,世应相克,变何爻象。都会了,然后才能摆摊设馆。若算奇门卦,也得下一年多苦功,将《奇门大全》读透了,按着六十根签子,摆好了局式,摆的好卦子,才能出来给人卜算。要学相面,得将《水镜集》、《柳庄相》、《麻衣相》、《大清相》,这些个相书读透了,才能出来给人相面。吃这一行,尽假的绝不能成,都得有几年功夫才能挣钱。就是他有点‘腥门’(即是前说过的十三道簧),也都得‘攥(zuǎn)尖儿’(管读熟了各种卜筮书籍、各种的相书调[diào]侃儿叫攥尖儿)。你们若有犹疑不定的事,可以找算六爻卦、奇门卦、相面的先生,千万别找那磕竹的,他们那行是腥到底的玩艺儿。”我向王君探讨那两根竹竿怎么那么灵,究竟有什么妙法。王君说:“他们磕竹也没有什么咒语,也不是竹竿灌铅,手里藏着吸铁石。他们那个法子,实是一种心理科学。”说着话,王君在我旁如此这般说了几句。我叫院邻某甲也用手心托着两根竹竿。我用手指着那竹竿说道:“我用你算算这位是哪月的生日。我由正月往下数,数到十二月为止。他是哪个月的生日,我数到哪月,你就并上。”说完了,我就嚷:“正月、二月、三月、四月。”那竹竿到了四月就并上了。我问某甲道:“你是四月的生日吗?”某甲点头道:“是四月的生日。”我至此才得着了秘诀。我又找了重有十几斤沉的竹竿,仍叫某甲端在手心上。我用手又指着那竹竿道:“我要用你算算他是几月的生日。我数到月儿,你就并上。”说完了我就嚷:“正月、二月、三月、四月……”说到了四月,那两根竹竿纹丝不动。我至此方悟,心理学的力量是用在轻质的竿上,能由心理的精神,从血脉皮肉催动了叫两根并上。若是用上几斤沉的竿子托在手上,就是按着催眠术的方法。这种方法使用好了,也能冤得住人,只是一样,冤过一回算完,不能再上二回当。

用竹竿算卦的,说行话叫“竹金”,做那种生意,学之甚易。



那种生意,到如今科学昌明,人类的知识开化,虽不说破,也能有人猜破的。这种生意也是时代落伍的行当,日见减少。就是还有做那行生意的,也是昙花一现,偶见于市廛(chán)的。磕竹的生意是受了自然的淘汰了。阅者若不相信,可以实地试验试验,如果试验的情形相同,便知余言之不谬也。





天桥的卦摊


东安市场问心处卦馆主人姓赵,天津人。原在天桥摆卦摊,算卦的人是拥挤不动,买卖发达了,迁至东安市场。有顺水万儿(管姓刘的调[diào]侃儿叫顺水万儿)者,也摆八岔子(江湖人管奇门卦调侃儿叫八岔子,是指其乾、坎、艮、震、巽、离、坤、兑,休、生、伤、杜、景、死、惊、开而言),见问心处营业发达,他仿着人家的名儿叫做闻心处,有欲占课之人到了天桥,找不着问心处,也能撞着他闻心处。如同乡下人进城买刀剪一样,王麻子、汪麻子、真正王麻子、老王麻子,不准哪家一样买,买的是王麻子的东西,何分王、汪、老、真正啊!闻心处仿问心处,如卖刀剪仿王麻子一样。

闻心处的生意还真发达,他摆卦摊的地点在天桥永利居后身,支棚设帐,每天只卖百卦,多了不算。够了百卦的度数的立即收摊。我老云在民国十二三年常到他那摊上助威。天天到了十二点钟,他本人没到,就有人将摊摆上,占卦的人们就围着摊子来回乱转,等他等得如同盼星星盼月亮似的。他来了往摊后边一站,问卜的人们就争先恐后地抽签子,将签抽出来,抢着往他手里递,看那样子好像抢头彩似的。他将卦签接过去攥在左手,右手就摆起卦来,将卦摆好了,向问卜人问:“这卦是你的?本人占?替人占?”如若问卜之人说:“自己占的。”他就问:“多大年岁?”问卜之人将岁数说明,他就往卦盘上一看说:“你这卦是因为心里犹疑不定,不知道奔东好,奔西好,是不是呢?”这人说:“是的。”他就说:“还是奔新路走好。”问卜的人就给他二十枚卦礼而去。这样一卦一卦地算去,每天他能挣二百吊钱,一年三百六十天,天天如此,他的收入大有可观,听说他做了十年好生意,很落下不少钱。

我向江湖人们探讨,闻心处的生意怎么会那样发达?他占的卦是否真灵?据某江湖人说:“闻心处刘某,所摆的奇门是‘腥盘’。”我问:“怎么叫腥盘?”某江湖人说:“奇门的盘,不是说那铜盘、铁盘、木头盘,是以那局式而分腥尖(腥是假的,尖是真的)。真的叫尖盘,假的叫腥盘。”我问:“什么叫局式?”某江湖人说:“他那卦摊上正当中摆着九个卦子儿。子儿上是乙、丙、丁、戊、己、庚、辛、壬、癸九个字,那九个字是以戊为头,按坎一、坤二、震三、巽四、中五、乾六、兑七、艮八、离九,八卦九宫摆成。如戊字在坎一,就叫一局;戊字在乾六,就叫六局。阳局顺行。例如阳一局是戊在一,己在二,庚在三,辛在四,壬在五,癸在六,丁在七,丙在八,乙在九,布成了就是顺行一局。阳有九局,皆是顺行。阴局逆行,例如阴九局,戊在九,己在八,庚在七,辛在六,壬在五,癸在四,丁在三,丙在二,乙在一,布成了就是阴九局。阴有九个局式,都是逆行。这局式到了冬至节以后,阳气上升,就摆顺行九局;到了夏至节以后,阴气下降,就摆逆行九局。至于戊字应落在几宫,则按照汉张良所定的阴阳十八局。凡是学奇门卦的人,初步就应当学摆局式,若买本《奇门遁甲》、《奇门大全》、《奇门五总龟》,任你有多好学问,也是看不会的。学摆局式,必有对于术学经验极丰富的人详为指点,才能学成。若真按着书理去学,至少也得费一年半载的工夫,才能使好了。卖卜的人都是穷极无聊,摆个卦摊,挣钱就吃饭,如若学摆局式,费一年半载的光阴,得有多大的垫办(下本钱筹办)?如若有钱,不是失业的分子,谁肯为奇门费几年的光阴?市井中卖卦的都是使腥盘,只要有人占卦,抽出根签来,卖卜的先生拿着卦签,啪……将四九三十六个卦子儿排在一处,叫行外人看着好像功夫很熟,蒙住了外行人就能成了。行家能有多少?百年不遇。真遇上行家也不怕,那懂行的人知道学奇门的难处,虽看出使腥盘来,也不肯破坏他们的生意,也不能和他们辩论真伪。闻心处的老刘便是使腥盘子的摆八岔(算卦中的一种)的老合(江湖艺人)。”我问:“有使尖盘子的没有?”某江湖人说:“摆奇门卦使尖盘的实在太少,百里挑一,即或使的是尖盘,也未必能够挣钱。”我问:“怎么使尖盘倒不能挣钱哪?”某江湖人说:“世上的人都是认假不认真。江湖人常说,一天能卖十石假,十天卖不了一石真。由这两句话考查,还是假的能挣钱。”我问:“用过真功夫的人,使尖盘怎么不挣钱哪?”某江湖人说:“凡是会使尖盘的人,都是书香门第,当初家道饶裕,生活无忧,读些年书,闲着没事,研究医卜星相,买些个医卜星相的书,找几个高明人指教,消磨岁月,学成了术学,给人算着玩,消遣解闷。玩票成啦,凡是这种人,都不懂得卖卜挣钱。到了他们要摆卦摊挣钱的时候,必是家业衰弱,衣食两难,受了经济的压迫,才到街头卖卜。他们这种人,是文学丰富,术理精通;对于社会里的人情世故是不通的。就是将摊摆上也是没有人占的,偶尔有占卦的又能挣多少钱?他只知学理,不知挣钱的诀窍,江湖管他们叫空(kòng)八岔(也叫外行八岔子)。”

我问:“卖卜的有什么挣钱的秘诀?”某江湖人说:“当初有个算奇门卦的先生叫也非仙,他也是个空八岔,在天津卫西城根摆卦摊,成天价愣着没人问卜。在他旁边有个摆卦摊的,也是摆奇门卦的,每逢人家那摊子摆上,问卦的人们立刻就将他围上。抽签问卦,争先恐后,买卖很是发达。也非仙看着人家那样挣钱,生了羡慕之心,他的灵机很好,有天那位先生将来到,还没摆摊哪,天就下起雨来,也非仙收了摊要回店,偏巧雨又住了,他不愿再摆摊儿,站在那先生背后,瞧他给人占卦。人家这位先生卦卦占得灵验,每逢断一卦,问的人就点头咂嘴说:‘先生算得真对。’也非仙瞧到末一卦,就听那位先生向问卦的人说:‘你这人姓张?’问卜的人说:‘对了。’他又说:‘你这卦是给你媳妇算的,问她的病好得了还是好不了?对不对?’问卜的人回答:‘太对了。’他又说:‘你媳妇这病还很厉害,须往北求医才好。’问卜的人说:‘我是在我们北边求的医。’那位先生说:‘赶紧抓药吧,吃下去就好了。’那问卜的人给了卦礼钱,欢天喜地地去了。也非仙等着问卜的人走了,他向那位先生问说:‘你这卦怎么算得这么灵哪?’那位先生说:‘你这人真是个空(kòng)子(江湖人管不懂江湖事的人调[diào]侃儿叫空子)。我哪能算得真灵,我是会把簧(bǎ huáng,用眼睛看出人的底细)。’也非仙又问道:‘什么叫会把簧呢?’那位先生说:‘将才问的那个人,我怎么知道他姓张呢?是我看见他那钱口袋上有三个字,是百忍堂(因当地“百忍堂”是姓张的开的,这就属地理簧),我才知道他姓张。’也非仙听着触动灵机,有些觉悟,忙问道:‘你怎么知道他媳妇有病呢?’那位先生说:‘我见他帽檐内掖着个药方,只见那药方上有红花、附子两味药,我才说他媳妇有病。’也非仙问道:‘看见他身上带着药方,就猜着他家有病人,这意思我明白了;你说他媳妇有病,是从哪里看出来的呢?’那位先生说:‘世上的人对于亲族骨肉,情义最厚莫过于妻子儿女,若是他父母有病,下这大的雨他就不出门了。我料他上边淋着,底下踏着泥水,必是给他媳妇抓的药。’也非仙说:‘对,对,是这样的。你怎么知道他是往北来医治哪?’那位先生说:‘适才下雨的时候,刮的是南风,这人前身没有雨点,后身肩膀上净是雨点,他不是从南往北来吗?我才断他往北求医。’也非仙点头道:‘是的,是的。’那位先生说:‘我瞧出他这几样破绽来,说行话,调侃儿叫把出簧来了。’也非仙说:‘你这把簧的本领教给我行吗?’那位先生说:‘传授你也成,你得拜我为师兄,挣了钱都给我,白给我效一年力那才成哪!’也非仙说:‘我愿意了。’于是两人就商议成了,择了个吉日,请出位中保人,弄了桌酒席,也非仙就写字拜师兄,他师兄将圆粘(nián)子(招徕观众)、把簧(bǎ huáng,用眼睛看出人的底细)儿、迫(pǎi)响儿(留下想算卦的人)、推送点儿(把不想给算卦的人说走)等等之法,全都传授也非仙。两个月光阴,也非仙将江湖秘诀学成了,再到各处摆卦摊,可不像以前坐在摊子后边等主顾候主顾了,他站在卦摊后边,几句话就能招一圈子人,将粘子(观众)圆好了,使诸葛乱点兵的法子,白送相法,小花腔(江湖人管冲着八面的观众使话儿调[diào]侃儿叫小花腔)使得最好,给谁相面谁佩服他。他用拴马桩儿(用话留你,让你走不了)拢住了二十来个人,又说着说着岔到奇门卦上了,他说卦算得最灵,那二十多人,便这个算一卦,那个算一卦,算起来没结没完。也非仙是按着他师兄的传授,两只眼睛会把簧,两个耳朵会听飞簧,心头灵敏会使簧,给谁算卦谁说好,越有人算,算主越多,哪天也能挣几块大洋,也非仙的卦摊比他师兄还多挣钱。还有些问卜的人在地摊上占完了卦,事后能够应验,接连不断地找他,能有回头主顾。”

我老云向某江湖人问过:“你说的这江湖秘诀,我是相信了,怎么也非仙的卦会有灵验哪,比他师兄还多挣钱哪?”某江湖人说:“他师兄是一腥到底的(全是假的)玩艺儿,也非仙是腥加尖(假的加真的)的玩艺,故此比他师兄多挣钱。”我问:“什么叫一腥到底哪?”某江湖人说:“他们算卦的若是净会使手段、使腥盘、使簧头,不明白术学的原理,就叫一腥到底。”我问:“什么叫腥加尖哪?”某江湖人说:“如若卖卜的人先将《奇门大全》、《卜筮(shì)正宗》、《三元总录》等等的术学书理研究透了,吃江湖的行话叫攥(zuǎn)尖儿,再学会了圆粘子、使簧儿等等的江湖法儿,使腥儿(假的)拢人,设法多挣钱,给人断卦,可用术学的真理给人决断。若能这样做,就叫腥加尖。”说到这里,某江湖人就说:“也非仙从前是个读书人,将术学的真理研究好了,因受经济压迫,在街上摆卦摊挣些钱维持生活。不料他是个不懂江湖的空(kòng)金,成天价愣着不能挣钱,他就拜了江湖人为师兄学会了江湖术。他又明书理,又会使江湖术,可就火穴(xué)大转(zhuàn)(在一个地方做生意挣了大钱了)了。凡是在他那里问卜的,十有五六能够应验。问过卜的人对他有了信仰心,就都常去找他问卜。他师兄是腥到底的,占了卦不灵验,砂锅砸蒜,一下子算完,绝不能有回头主顾,所以买卖不如也非仙。”

我听他所说的这些事才知道,社会里的事,最难学的是世故人情,江湖中的秘诀,也是从人情里研究出来的,“练达人情皆学问”,诚然不假。我问某江湖人:“江湖中的秘诀,以哪种最好?”某江湖人答道:“金皮彩挂,各门皆有秘诀。就以江湖中算卦相面的使用的秘诀来说吧,最好的是方观成的《玄关》。”我问:“方观成的《玄关》是怎么回事?”某江湖人说:“方观成是个才子,做过清朝的大官,在他不走运的时候,穷极无聊,摆过卦摊。他以人情世故研究出一部《玄关》,凡是算卦的人,能得着了《玄关》,不论是什么人来问卜,都能当时就灵。那《玄关》是江湖金点中(江湖人管算卦相面的总称金点)的无价之宝。”我问:“那《玄关》中的秘诀,阁下能知晓吗?”某江湖人说:“知道些个。”我问:“阁下能否告诉我一二?”某江湖人说:“我列举一事,你听了就能知道《玄关》的奥妙了。”他说到这里,就说:“有个问卜的人到卦摊上问卜,抽了一根卦签,往摊上一扔。算卦的先生问:‘你这卦是给人占还是自己占哪?’问卜的人说:‘是给我母亲占的。’那算卦先生说:‘你母亲的岁数多大呢?’问卜的人说:‘六十二岁了。’算卦的先生往卦盘上看了看,然后说道:‘你母亲这卦是天芮星压运,主有灾病缠身。’问卜之人立刻就得说:‘不错,我母亲正在闹病哪。’”我问:“这样断法是卦里断出来的,还是江湖中的《玄关》呢?”某江湖人说:“这是《玄关》中的秘诀。你想,六十多岁的老太太,叫人给她问卜,除去有病还能有别的事吗?”我说:“是这个意思。”我问:“《玄关》就是这一样吗?”某江湖人说:“《玄关》秘诀共有八百余样,要学也是不易。”他将个人所有的《玄关》取出来叫我观瞧。我看那头一篇上写的是:

方观成之玄关

先师化道,不出天地范围,一理贯通,能使人超悟。一入门先猜来意,未开言先要拿心。洞口半开,由此挨身而进;机关一露,即宜就决雌雄。要紧处何劳几句,急忙中不可乱言;只宜活里活,切忌死中死。捉鬼擒妖,使他心悦诚服,激情发意,探面色、口风定贵贱,勿看衣裳断高低。宜观动静,到意温和,正是吉祥之兆;来人急骤,定是凶险之因。斜盼连观,预虑其差头,寻事人来,察数理,可推及得失。奴仆成群,也有奸恶;同友并队,岂无刀凶?若问流年行运,必收放而言,有问宜缓答,无语少先声,我要问他须急快,他来问我莫慌忙。忤时假装怒,隆时假陪欢,他喜我偏怒,他怒我偏欢,冷处要生急,急处要生冷。先忤后隆,术中妙诀;轻敲响卖,秘内元机。父来问子必有险,子来问亲亲必殃。幼失双亲,难许早年享福;晚来得子,定然半世奔波。若年高,功名必冷。心粗胆大,刑险将来。妻克重重,内有生离恶土;子孙叠叠,岂无子孙愚顽?若染私情,夫妻定然不睦;交多朋友,父母岂不憎嫌?老妇再嫁,谅必家贫子不孝;少年守寡,要知衣食丰足。观门户能知勤俭,看茶汤可决妻能。儿衣齐洁有贤妻。老夫奔波无好子,家有孝子岂用老翁赶集?幼酌在宫,多有欺凌之事;老娶娇妇,难逃欺女之端。芝兰当分荆棘,瓦砾要辨金珠。清高多贵人之提拔,富贵有嫉妒之异端。商人忤兴废,奸者虑官非。湖海客来谈贸易,缙绅人至讲经纶。闹市人家,须防火烛;荒村野店,宜虑强人。家从亲手而兴,胸有智略;业为自己而败,性爱风流。逞英才,好风月,家资萧索;爱朋友,结弟兄,手内空虚。帮衬假奉承,语中有刺;欲要吐,欲不吐,随卖随封。得钞时休言多寡,卖响处灭迹藏形。失撇宜留后意,受擒作佯。逆来顺受,不可忤悖;顺中有逆,须详有假。是忤必响,是隆必倒。进退两难,宜思拔法;断谈有势,须考心传。一篇通江湖之术,数言开造化之机。平时不研求,一时岂能决断?

我老云看罢这《玄关》,仍然不解其中意义。向某江湖人恳求,叫他按着江湖的意思向我一一地解释。某江湖人不肯给我解释,叫我自己参悟。我求之再三,他只讲那《玄关》中的“老妇再嫁,谅必家贫子不孝;少年守寡,要知衣食丰足”,“儿衣齐洁有贤妻。老夫奔波无好子”。说给我了,我将他所说的意义,录之如下。某江湖人说:譬如,有个算卦先生往各街巷中敲打竹板兜揽主顾,有一家出来一位五十多岁里外的老太太叫算卦的,那算卦的先生未曾答言先把(bǎ)簧(用眼睛看出人的底细),把簧的意义有:先看她穿什么衣服,什么长相,面貌上的形容喜乐悲欢,就能不用问她,将老太太的事预先知道了。如这老太太描眉打鬓,穿的衣服鲜艳,就可以明白,她那大年纪,土埋半截了,还这样修饰,一定是“老妇改嫁”。如若是老了,丈夫不在,或是尚在,安分守己过日子,哪能那样打扮?这算卦先生随着老太太到屋里,没落座之先,得先看屋中的摆设,好知道她的穷富。看她屋内的人共有几位,也能预测出来她的家境。大凡妇人占卦有两样儿,若是屋内人多,三姑、六婶、八姨、二舅妈,满屋子是人,将算卦的先生叫进屋来,先生一看就知道是问喜事,什么要生养了,是生男孩呀,是生女孩呀,姑娘有婆家,儿子说媳妇,合个婚,择个日子,绝离不开这几样事。如若妇女们心中有了烦恼的事,有了凄凉的事,要想找个算卦的,算算个人的心事,绝不叫她亲族骨肉、院内街坊知道,悄悄地叫算卦的进来,好问个人的心事。有病的人,心中事不瞒医生;问卦的妇女有了事,无论多么严密也不瞒先生。算卦的先生到了屋中,如见没有人,就能猜透老太太定有伤心事,最难过的事儿。如若屋中有一两个人,也是与她不是母女,便是婆媳。算卦的落了座,问她给谁占卦。如若老太太说给自己占,算卦的先生用八面风的卦语,如同摆八卦阵一样,然后再问她什么事。如若老太太问她将来如何,不用问她的身世,就知道她是老妇再嫁。再嫁之后,丈夫的感情多好,究竟半路夫妻,不如从小的夫妻。算卦的先生遇见这样事,看卦上的卦象是假,按照人情中的感慨话语,向她断卦,准能句句说得老太太点头咂嘴,心中佩服。如若断她命苦心善,无好儿女,或是说她命里孤独而贫,管保准对。譬如,算卦的先生走在一家门前,出来个仆人叫算卦的。算卦的先生看他门户整齐,进了院子,门房有男仆,内宅有女仆,屋内摆设不是洋货,花梨、紫檀、硬木桌、郎窑瓶、官窑罐,主人是个二十几岁的少妇,长得艳若桃李、冷若冰霜,一身素服,眼前有个三四岁的小男孩。算卦先生落了座,问给谁占卦,这位少妇说给小孩算算命。算卦先生问明了小孩生辰八字。用《万年历》将八字的四柱财、官、印、受都按好了,用一句话就能要出簧。头一句冷不防向少妇说:这位少爷的八字克他父亲。嘴里这里说,两只眼睛看着少妇,如若少妇显出悲惨来,一定是她丈夫死了,穿着是丈夫的孝,被算卦先生一句冷钢(钢就是话)引起她的伤感来,就要出簧(问出实情)来,知道她是青年守寡。按着方观成《玄关》断她衣食丰足,准能对的。摆卦摊的先生,遇有六七十岁人问卜,问做买卖兴衰,谋事能否有成,就按着方观成的《玄关》年老奔波无好子的断语,准能对的。如若有三十多岁的男子,带着几个小孩,穿的衣袜鞋帽整齐洁净,就按着方观成《玄关》儿衣齐洁有贤妻,准能对的。

我听了某江湖人说的,方知《玄关》奥妙无穷,再看他那《玄关》的第二章,他不让看,就是再看第一章也不叫看了。最后我问他一句:“闻心处的卦是一腥到底呀,还是腥加尖(假的加真的)?”某江湖人说:“他的本领并不高明,腥的也不到家,尖的也有限,只是他有五六个贴靴的(同伙),弄得很火炽。江湖人宁愿使十三道簧,按着《玄关》推测人的事,都不愿用贴靴,即或挣了大钱,江湖人也讥诮他仗着敲托的(管贴靴的调[diào]侃儿叫敲托的),不算真本领。”





天桥金点


在民初时天桥有个相面的先生,叫做市井拙人。他不懂什么叫“玄关”,哪叫“十三道簧”。用过几年工夫,将《麻衣相》、《柳庄相》、《三世相》、《大清相》等几部相书,读得挺熟,像背《三字经》似的。每日总有些人围着他,张三相完了,李四跟着相,接连不断,直到收摊为止,没有歇着的工夫。一般江湖人常说:市井拙人虽然是个“空(kòng)子”(不懂江湖内幕的人),给人相面的时候虽不使“簧头(方法)儿”,也大受社会人士的欢迎。他另有书本的簧头、相关(和相面有关的内部术语)。据江湖人说:市井拙人相面的本领可称头把交椅,但他无论挣多少钱,也是一日花光,在民国十年以前生痔疮倒卧街头而死。指南轩命馆主人桂振峰,是星相中的出色人物。说腥腥到家,说尖局的尖到家,清末民初之间,在天桥的命馆中名望最大,买卖兴旺,为同业所不及。“戗(qiàng)盘”(管相面调侃儿叫戗盘)是他的拿手好戏,到了他的晚年,能以“八岔子”(奇门卦)坐而不动,等候主道,支持几年,实是不易。

如今北平这个地方,有许多“戗盘”的先生,都是桂振峰的门下。金点(算卦相面的总称)的门户,他家的支派是最盛了。在吉祥舞台、振仙舞台的后边以及天桥西市巷内有些个卦摊,不是奇门,就是六爻,每有行人从摊前经过,彼辈必然点首招手:“你来!我送你几句!”惹得行人无不侧目。我对于他们点手唤罗成的先生,也向江湖人讨论过是怎么回事。有一位江湖人说:“他们是半空半嘬(似懂非懂)的金点。”我问:“什么叫半空半嘬的金点?”江湖人说:“不懂江湖事的人调(diào)侃儿叫空(kòng)子;懂江湖内幕,会使江湖手段算卦的先生调侃儿叫金点。算卦的人如若对于江湖诀窍有一知半解,似通似不通,调侃儿叫半空的金点。算卦的人如若尽顾挣钱,不顾羞耻,调侃儿说真念嘬。那些个点手唤罗成的先生,对于江湖事,有些事能够懂得,又要挣,又没本领,点手唤人,似乎脸厚,又觉不安。江湖人对于他们这些人叫做半空半嘬的金点。虽是调侃儿,也透着讥诮。”近几年来,我老云对他们注意考察,点手唤罗成的先生是有增无减,由此可知失业的人的多寡了。





江湖中之戳黑的(江湖人管点痣的调侃儿叫戳黑的)


吾老云云游各省,常见各省的市场上有一种买卖,用一张小桌,上摆药瓶数个,玻璃镜一个,人牙数百个,壁上悬挂布幌(huǎng)子,布幌子上画两个大脑袋,一男一女,面上有些黑点,按着相书的部位,都有标志,那黑点底下,或是:女妨男、男克女、有产危、有火灾、有水危、有土劫、有疾病。在这两个人脑袋的左边、右边、上边,还画着十二个小图,第一图是一个人乘船覆没,上写“犯水危”。第二图:一家失火,将人烧在火场之内,上写“犯火灾”。第三图:一个人走在墙底下,被壁倒墙塌砸得腰断腿折,上写“犯土劫”。第四图:一家子有死人,院中停着一口棺材,有个小媳妇身穿重孝,跪在灵前啼哭,上写“女妨夫”。第五图:有个女子站在门前,向行路之人眉目调情,下写“月下偷情”。第六图是一人喝酒吃醉,持刀行凶,上写“酒后行凶”。第七图是一个人手持单刀一口,截住行人,上写“劫盗”。第八图是一个女子悬梁自缢,上写“主自缢”。第九图是一个人生得方面大耳,上写“福相”。第十图是一个老人,上写“寿相”。第十一图是一个人又瘦又没精神,上写“夭相”。第十二图是一个做官的人,上写“贵相”。上边还写四个大字“去痣求顺”。

做这种生意的,也有坐在旁边一声不发,等主道候客的;也有向行人指手画脚说说道道的。他们是给人用药去痣,外带拔牙。我云游了好几十年,很见过几个有本领的。虽然是点痣为生,能够穿着一身绸缎之服,日挣十数元的。在济南府我见过一个叫安华林,在黑龙江见过一个叫贾宝善,在天津见过一个尚登云,这三个人是点痣头路角色。凡是那不张嘴儿等主候客的,都面带愁容,透出来是不挣钱,没生意,勉强支持的样子,也甚可怜。

有一次,我在河南开封相国寺里见着一个点痣的,长得矮小身材,靠着东墙,挂着布幌(huǎng)子,摆着一张桌。他能在桌前奉送手相,招惹得一庙之人围着他,等候送相,围得风雨不透。他说:“看相不看手,必是没传授。”他拉着一个人的手说:“看手相,是:掌为虎,指为龙,能叫龙吞虎,莫叫虎吞龙。指长掌短龙吞虎,掌长指短虎吞龙。大指为君,小指为臣,二指为宾,次指为主。你这人是虎吞龙,臣欺君,宾欺主,劳碌早,六亲不靠,自创自立能受累。”那人直点头说:“先生相得很对。”他又说:“你这人的财是存不住的,来多去广,多来多花,少来少花,总不存财。”这人说:“先生相得很对,我几时才能存财哪?”他说:“你这人左耳前边有个痣,主于不存财。”说着递过一个玻璃镜,那人用镜子照他的面上,果然左耳前边有痣。他问先生:“怎么会不存财呢?”先生说:“你这人是红脸膛儿,五行属火,你那痣是黑的,属水,水克火,你受着克哪!”说到这里,就向这人说:“有病早治,养病如养虎,虎大伤人,病大伤身。你这痣用药点去吧。”这人问:“点这痣多少钱哪?”他说:“我们这里点痣是一大枚。”这人说:“准能点去吗?”他说:“点不掉原钱退回。”这人说:“你就给我点去吧!”他用个骨头针往药瓶里沾了点膏,点在这人左耳前边痣上,又说:“你这人无论是对待亲友多好,也是恩人无义,反来成仇。”这人说:“不错,这些日子直犯口舌。”他用手把这人的脸上一指说:“你的嘴犄角上有个痣,犯口舌,把它点了去吧?”说着用骨头针又往药瓶沾了些药膏说:“点痣用不了多少钱,一大枚就成。”他嘴里这样说,那人还以为点多少个痣也是一大枚,就点吧。他说几句点个痣,说几句点两个痣,不到一刻钟,这人的脸上都点满了,然后这人给他掏钱,掏出一大枚来。他说一大枚不成,点一个痣是一大枚,点两个痣是两大枚,你的脸上共有三十七个痣,应当给我三十七大枚。这人说:“我没带那些钱,我只有二十大枚。”他说:“你还差十七大枚,明天给我带来吧。”这人给了他二十大枚,转身走去。

我看他这样先不说明,往脸上足点药,满脸都点成花鸡蛋似的,然后多讹钱,带着小敲诈地讹人。点了一个人,又点一个人,接连不断点了十数人,合计起来也挣两元多钱。次日我去逛相国寺,走到他那里,正见昨日点痣那人和他捣麻烦,说:“我花了二十大枚点痣,一个也没点了去,这是怎么回事?”他说:“我们这药管保准掉,如若不掉原钱退回,可是花钱一次,点药两回,昨天点了一回,今天还得再点,昨天给了钱,今天不要了。”说着又给那人点了一遍,点完了药说:“你可别用指甲抓这药,可别沾水,等着这药自己掉了,再沾水也成。你如若用手抓了,沾了水,药劲使不上,点不下去我可不管。”那人点头而去。

我连着去了几天,也不见那人来找他。至于痣点去了没有,也不得而知。我向江湖人探讨了几天,才知道其中的事儿。原来这点痣的行当,说行话叫“戳黑的”。他使用的布画幌(huǎng)子,叫做“摆子”。还有带拔牙的,调(diào)侃儿叫“戳黑带搬柴”(江湖人管牙叫柴,管拔牙叫搬柴)。他们金点(算卦相面的总称)要收徒弟,遇见伶俐的,立刻夹磨(jiá mo)(师父传授真本事)他戗(qiàng)盘(相面)。如若拙笨,教他相面恐不能成,笨人由笨处来,先教给他戗盘的条子,练习去戳黑。什么叫“戗盘的条子”呢?说起他们的条子来,也是多得很,大约着有百数多样,如同唱小曲儿似的。一段算是个条子。要教徒弟的时候,必须将这条子用笔写在习字本上,一段段地教徒弟去读,读熟了能够背诵下来就能使用。他们的条子分为士、农、工、商,有戗(qiàng)冷子条儿(做官的人调[diào]侃儿叫冷子,给冷子相面的词儿调侃儿叫戗冷子条儿),有戗科郎(kē lang)点的条子(管种地的人调侃儿叫科郎点,给乡下人相面的词儿就叫戗科郎点的条子),有戗贸易点的条子(管做买卖的商人调侃儿叫贸易点,给他们相面用的词儿就叫戗贸易点的条子),总而言之是给哪界相面用哪路词儿,哪路词儿即是哪路条子。譬如他们戳黑的(江湖人管点痣的调侃儿叫戳黑的)在市场内将摆子挂上,摊子摆好,说说道道圆上粘(nián)子(聚好了观众)。见人围得够用了,瞧见某甲,有三十多岁,像个劳动分子,自挣吃穿的朋友,他就向某甲说:“你这人二眉竖目,是君臣不配之相,主于少年不立,难靠祖业。要说你这人祖上的根基颇正,吃亏就是你没赶上好时候,到了你这辈,咬王瓜的尾巴,苦点了。你好似老爷庙的旗杆,风来了自己挡,雨来了自己淋。六亲不靠,连个遮风挡雨的人都没有。自创自立,自己跌倒自己爬。你那亲戚朋友也是苦害你,钱你没少挣,不知不觉也没落下,只见鱼喝水,没见两腮流。”这套词儿准能说得某甲点头咂嘴心里佩服。再如,他若见人群里某乙的穿着打扮、面貌的神气,好像个光棍字号朋友,戳黑的就能使用光棍条子,用手指着他说:“这位老兄五官端正,颧骨高耸。相书上说,男人颧骨高,必定逞英豪;女子颧骨高,杀夫不用刀。你这位老兄就颧骨高,主于三大。哪三大呢?就是义气大、胆量大、志气大。义气大怎么说?就是你拿钱不当钱,遇见朋友真交。不怕家里没钱也要办有钱的事儿。胆量大怎么说?别人有点事记在心里,能够发愁得睡不着觉,你不怕有天大的为难事,也不往心里放,该吃的时候真吃,该喝的时候真喝。志气大怎么讲?你这人看富的不巴结,遇见穷的不小看人家,银钱如粪土,脸面值千金。遇见事,宁可钱吃亏,不叫人吃亏。”这套话说出来,那光棍字号的某乙一定能够佩服他的相法高明。他如问:“先生,你看我目下怎么样?”戳黑的说:“你这人吃亏被累就在你的脾气上,如若遇见投缘对劲的朋友,要命都给;如若遇见不投机、不对劲的人,任他有多大势力你也不怕。真是:千金可让真朋友,话不投机寸草争。见文王恭而有礼,遇桀纣干戈齐扬。目下运气不佳,事事不凑巧,求财不到手,心里发急躁。”这光棍朋友听后,真是点头佩服。他们的戗(qiàng)盘(相面)条子,编得也是体贴人情,很有意思。就是见了什么人说什么话。还有“册(chǎi)子条儿”:相眉毛用的,相眼睛用的,相鼻子用的,相耳朵用的,相嘴用的,相山根(印堂之下,两眼之间的部位)用的。譬如,有人问:“先生,你看我鼻子好不好?”他就说:“鼻为审辨官,乃五官之祖,一面之表率。相书上说:鼻梁高,准头正,为人正直;鼻子小,准头尖,为人灵巧,处世圆滑。塌鼻梁,一生奔波。准头不正,心地不良,像你老兄的鼻子主于……”又如有人问:“先生,你看我耳朵如何?”他又说:“耳朵厚,要有轮,有轮有廓是贵人。耳要厚,福气厚。耳要薄,福气薄。耳要大,又要圆,又圆又大是英贤。两耳削薄,一世奔劳;两耳贴脑,富贵到老。对面不见耳,乃大富贵之相,你阁下的耳朵是……”譬如有人问:“先生,你瞧我的嘴好不好?”他必说:“口要正,又要方,口如四字福如江,唇口端正红如朱,富贵荣华在前途。唇削薄,不露齿,一生劳碌也无福。你阁下的出纳官(即嘴)是……”他们的条子如若用上,立时就见响儿(江湖人管相面相对了,叫人佩服了,调[diào]侃儿叫响儿),只要见响儿,立刻就扣瓜(管吓唬人,叫人害怕调侃儿叫扣瓜),如若顶了瓜(江湖人管他们恫吓人,人要相信害了怕调侃儿叫顶了瓜),立刻就挣钱。挣钱之法也是叫人去痣求顺。如若戳黑的半用相面之法,一半点痣,能够有拿手准挣钱,就算是成了;如若才能有限,心智不灵,也就戳一辈子黑。

江湖人对于戳黑的要是没有进步,做一辈子戳黑,都很轻视。据他们江湖人说,戳黑的是相面的徒弟们坐科(入科班学艺,此处借指最基础的)的生意。要有灵机,干了一年半年的就能脱离戳黑改为相面,那才有人恭敬,说是夹磨(jiá mo)(师父传授真本事)成了。他们所用的那点痣之药,计有两种:一种是用硫磺、火硝、白矾、口碱熬炼而成,其色红,必收于瓷瓶之中,药性猛烈,木质铁质瓶皆不能收存,那药点在人面之上,疼痛难忍,三日生效,准能去痣。兼治恶癣,皆有奇效。但制此药“笨头儿太海(hāi)”(江湖人管本钱太多调[diào]侃儿叫笨头儿太海[hāi]),一般老合(闯江湖的)们都不愿花钱费神,不熬此药,那药方儿也怕要失传了。如今戳黑的使用的“汉儿”(江湖人管药品调侃儿叫汉儿)都是“里腥啃”(lǐ xing kèn)(江湖人管假东西调侃儿叫里腥啃)。我老云察看过他们点痣使的药,是白灰、口碱,用烧酒浸化,加以樟丹搅和的,点在脸上只觉得微疼微痒,但无效力,不能去痣,现在北平各处虽然都有戳黑的,哪个也没受过真传授,全是半空(kòng)不嘬(江湖人管点痣的人虽知道江湖的黑幕,没受过江湖传授,对于挣钱多少没有拿手,没有把握,将就凑合混饭吃,调侃儿说他们半空不嘬)。现代的人们都打破了迷信,对于面上有痣主吉主贵,有无凶险,毫不介意。点痣的生意也因时代落伍了,想不落伍也行,得往农村里骗那乡下人吧!





江湖中之金卖两门做变绝生意之内幕


江湖中的金点应以算卦、相面、看风水、批八字做生意,不应当带着卖药。挑将(tiǎo jiàng)汉儿的应以治病卖药做生意,不应当带着算卦,否则金卖相混,同道人必出头干涉,责以江湖乱道之罪,令其改悔。

在清末的时候,治病大夫不论是否够格,随便挂牌行医,随便售药。患病之人稍有不慎,不是被庸医所害,就是被售药所误。有些个卦馆门前都写着八个大字:“圆光寻物,专打鬼胎。”不知内幕的都以为他们会圆光,丢了东西,圆光圆得出来是何人偷去;专打鬼胎,是谁家有邪魔外祟,他们会捉妖(倒不是《青石山》、《混元盒》),谁也不注意这些事儿。社会里的事真是奇怪,不拘什么买卖,只要有人做,立刻就有人照顾。当初我老云在学房读书,有某学友,他父亲就在××街开设命馆,门前就立着那“圆光寻物,专打鬼胎”的招牌。我时常找某学友一同上学,他的父亲将我看成小孩子,不懂事儿,有什么事也不避讳。有一次他的秘密之事被我无意之中看个完完全全的。我还记得那天正下大雨,我找学友上学,他父亲说:“今天下雨,不用上学了,你们在一处玩吧。”我们两个小孩就在里屋内玩耍。工夫不大,从外边进来了一个人,约有二十多岁,穿着打扮像个仆人,长的相貌俊美已极。他进门就问:“先生怎么叫打鬼胎呀?”先生说:“凡是姑娘受了邪魔外祟,不夫而孕,就叫鬼胎。妇人的丈夫不在家,受了邪魔外祟,有了孕,也是鬼胎。这鬼胎要是不治,长成了形,生养下来不定准是什么东西。这鬼胎不唯可怕,传说出去,也真寒碜。”那仆人说:“鬼胎怎么打法哪?”先生说:“我有两个方法。一个极快当的法子是用针扎,我到你家去扎也可。”那仆人直皱眉,说:“我们这是姑娘,她不能出来,也不能到我家去扎针。先生你还有别的法子没有哪?”先生说:“还有一种治法,是用吃药往下打。”那仆人说:“吃药往下打倒是很好。是汤药啊,还是丸药哪?”先生说:“丸药。”那仆人说:“丸药便利极了。药费多少钱一服呢?”先生说:“一百五十两银子一服。”我听着他讹人,以为是穷疯了呢。这仆人说:“这药怎么这么贵哪?”先生说:“这药有上等的朱砂,一两二钱银子一钱;这里头有好麝香,叫当门子麝,每分卖二两四钱银子。就这两种药就贵极了,别的药还有贵的哪。可是,这药虽贵,有几样好处,吃下去人不受伤,一天的工夫,准能把鬼胎打掉。”那仆人听了,也觉得很喜欢,说:“吃下这药去要是不灵验哪?”先生说:“不管事,原钱退回。”那仆人就从腰中掏出一张银票,说:“先生你给配这药吧,我留下五十两银票当作定钱,明天我一定来取。那一百两银子我明天给你带来。”先生接过了银票,问他道:“你贵姓啊?”那仆人说:“我姓蒋。”说罢转身走去。他走了不大的工夫,先生就将他儿子叫出来说:“你快追那个买药的,在他后头跟着,瞧他进哪条胡同进哪个门,然后你打听那门是谁住着,你再回来。”他儿子就追出去了,暗中随着那个仆人而去。

有些卦馆门前写着八个大字:“圆光寻物,专打鬼胎。”不知内幕的会以为谁家丢了东西,他们圆光圆得出来是何人偷去;谁家有邪魔外祟,他们会捉妖。



先生的媳妇才四十多岁,专爱说话。她问先生:“那买药的人来了,你为什么说会扎针呢?”先生说:“他来买药,一进门儿我就看出他是个仆人。我说会扎针,往他家去扎,是要去他家看看穷富。如若真阔,得多挣他的银子。他说不能往他家去扎,也不能到我这里扎,我就猜着了,一定是他当仆人的与他主人的姑娘小姐通奸有染。他们的小姐是大家之女,与仆人有私,焉敢叫我进门呀,也不能来呀。我猜着是仆人与小姐通奸有孕,就要他一百五十两。”他媳妇说:“这个仆人哪能花得起一百五十两啊?”先生说:“你不懂,我是用话探明白的,是要他的水火簧!”他媳妇问:“什么叫水火簧?”先生说:“他要穷,就是水,我少要钱;他要阔,就是火,我就多要钱。我瞧这仆人长得那么漂亮,穿得那么整齐,他主人家定是个阔家。我和他要一百五十两,他当仆人哪有这些钱,这钱是他们小姐花的,我和他要一百五十两他都没驳回,大约花个几百银子也花得起,我还要价要嫩了呢!”他媳妇说:“要嫩了怎么办哪?”先生说:“我有翻钢叠杵(通过花言巧语使买主翻倍付钱)的法子,还能问他多要钱。这个点儿(人),至少也挣他几百两。”少时他儿子回来说,他跟着仆人走进东四×条胡同,进了×宅了。先生听了,向他媳妇说:“×宅是个富户,这号买卖做下来,准够我们二年的花销。”他一家子有了这号买卖,欢喜得了不得,先生就提笔在手,开了两个药方,给他儿子五两银子叫往药铺里配制此药。他儿子就邀了我一同前往。到了药铺,柜伙抓药,他贪玩耍,各处瞧着。我知道那药方宝贵,便用铅笔抄写下来,是三棱、义术、水蛙、芒虫、鸟头、附子、天雄、牛膝、薏苡、蜈蚣、红花、大黄、芒硝、桃仁、杏仁、黄花、沉香、朱砂各等分,蜜制成丸,黄酒送下。其二是:皂角、细辛、肉桂、丁香各等分,共为细末,用药捣泥如丸。绸子包裹,如核桃大小,纳阴坐之,其绸上拴三股小线,坠铜钱三个。药铺伙计将药包好,他儿子拿回家去,配制去了。

我自幼就喜爱谈奇说怪,见了他的事儿,我留心访查,果然至次日天黑了,那仆人往他卦馆取药。先生说:“先将坐药用上,觉着有了动静再吃丸药。”那仆人就给他一百两银票,持药而去。他拿走这药有没有效力,不得而知。恰巧第四天,我正在他家和他儿子写字、温习功课,那仆人进门就作揖,说:“先生,你这药真有效力,我来道谢。”说着又给了他五十两一张的银票。先生问他:“打下了鬼胎之后,人觉着怎么样?”那仆人说:“吃下药,肚腹疼痛难忍,还好,昨夜内胎就下来了。这两天病人周身软弱,不思饮食,心乱神昏。”先生说:“不好!还得配服产后的药吃,安神养血,若不吃药,恐有性命之忧。”那仆人害了怕,又问:“配这产后药得多少钱?”先生说:“这药倒不贵,才几两。最贵不过那避孕药,吃下去管保男女交合永不受孕。”那仆人听了,面上有了喜容,忙问:“那避孕药要配一服得用多少钱?”先生说:“二百多两。”那仆人说:“怎么这么贵哪?”先生说:“这种药里有避孕砂,出在南洋,贵重无比,二百多两还是药的本钱,我还没赚呢,如若再赚你的,几千两几万两还不止哪!”那仆人听完,由身上取出一对玉镯、两个戒指,说:“先生,你看这些东西,能值几百两,你将它变卖了,连那产后的药,一并配成,我后天来取,将来我还给你传名,重谢于你。”先生将东西收下。以后的情形,就不得而知了。

直到如今,我晓得社会黑幕、江湖骗术,才知道那卦馆江湖人调(diào)侃儿叫“金点座子”;占卦、相面、批八字是它的本等,带着卖药,调侃儿叫“枪里加鞭”;专打鬼胎的生意,是“做变绝点儿”(江湖人管给人打胎叫变绝点儿。这句侃儿是指着胎孩而言,十月临盆能够活的小命一条,他给治死了,由活变气绝了)。走闯江湖的人们对于骗取人的银钱,都不在乎。惟有对做这“变绝”生意的,都不赞成,他们调侃儿说,做那生意太“伤攒(cuán)子”(江湖人管做缺德的事儿调[diào]侃儿叫伤攒子,做亏心事也叫伤攒子),也真是伤天害理太缺德!

他们做这种生意也是瞧人下家伙,该卖一百绝不要五十。第一回的钱,叫头道杵;第二回的钱,叫二道杵;还有三道杵、四道杵,最末一次的叫绝后杵。有时扎胎、打胎没弄好,弄出毛病来,遭了官司,骗财、害人,二罪归一,饱尝铁窗之苦。做这变绝点(给人打胎)生意挣钱虽多,头顶着杀人的罪行,也不把牢。如今时代转变,有卫生当局管理医生、药商,对于无执照售药的、无凭书行医的,取缔得很严。无论药铺、卦馆,都没有那打鬼胎的招牌了。可是,凡是做这变绝生意的,又花样翻新,另想招揽这种生意的办法。他们在包药的发票上,印着几个大字:“此药孕妇忌服。”如若有人问他,这药孕妇吃下去怎样,他们就能明白此人欲买打胎的药物。于是,施展他们的“钢口”(说话的技巧和分量),售以堕胎的药品。这“孕妇忌服”,就是做绝点生意的变相招牌。上年有段新闻:“(二十四年四月八日)西直门北关门牌×××号××堂××膏药铺,铺长×××,专做绝点,收手术费七八十元至一二百元,或为扎,或为用药,断送了无数小命。不料事机不密,被人告发,被官署查抄,饱尝铁窗风味。”我说做这种生意真伤攒子,不知社会人士作何感想?





江湖中锍(liǔ)幅子的


我老云虽然卖稿为生,每日埋头书案当刷子匠,有了闲工夫就到外去游逛,什么东安市场、西单商场、天桥儿、什刹海,时常地巡礼。有那又便宜又贱的胶皮车,花个几十枚就能转半个北平。每逢洋车走到前门里外、西河沿、王府井大街、霞公府、西单牌楼北边,都有那撒传单的,追着往洋车上愣锍。所撒的传单不是卖药的,就是相面的,天桥儿也有这种撒传单的。

我问某江湖人:“他们这撒传单的人按江湖事说是干吗的?”某江湖人说:“他们这种人,说行话叫‘锍(liǔ)幅子的’。”他们都是欲做江湖事,知识聪明不足,才给人撒传单。调(diào)侃儿管那传单叫“幅子”,管撒去叫“锍”。他们这行人本领也分高低。有本领的给相士们撒传单,挣了钱三七分钱、二八分钱;本领弱的撒一天传单,挣三四角钱。行家雇撒传单的花钱虽多,拿出去一千张传单,准撒给一千人,多少也有点效力。力笨雇撒传单的,花钱可是少些,拿出去一千张,撒不了三二十张,剩下都论斤卖了,包花生仁了,指望传单发生效力,那不是做梦吗?有本领的撒传单的,拿出去传单不能遇见人就给,他们也有诀窍:哑巴不给,瞎子不给,拉洋车的不给,卖苦力气的不给,外国人不给,蒙古人不给,穿的衣服太穷不给。这些人都不能到旅馆花钱相面,给他们传单也是白糟践东西。他们撒传单的每逢要给谁一张传单,得瞧着给谁不白给,有几成儿能照顾他们才成哪。撒的传单不多,见的生意不少,那才是锍幅子有把点(瞧着哪位像花钱的,调侃叫把点)的本领哪。可是有本领的锍幅子的人都不挣死工钱,要说三四角一天,他们是不干的。挣了钱和先生三七分账,少了,一天挣个块钱里外;多了,三二元。可是相面的先生凡是有经验的,都愿意三七下账,雇有本领的撒传单,钱虽多花,挣得还多哪!有时撒传单不把字露出来,把没有字的背面给人看。我因这事向他们问过:“你们为什么撒传单反着给人家呢?”他们说:“一般的人因为知道传单是宣传品,看一眼就扔了,甚至于还有不看的,我们反着递给他,他不知道是什么,无论如何也得看看。只要他看,就许触动他的心机,照顾一下子。反着传单递给人,是叫人非看看不可。这种做法,非是久惯锍幅子的才能这样哪!”我听他们所说,才明白个中的用意。

可见社会里的事,不管哪行也有研究,若像绸缎店的徒弟出来撒传单哪,看见人就给一张,简直是白搭,哪能有宣传的效力?我老云对于江湖中锍幅子的人们,是佩服他们有经验阅历,不是白挣钱不管东家赔赚的。





三不管的戗(qiàng)金生意


算卦、相面、看风水,总侃儿叫“金点”。分开了说,相面的又叫“戗金”,又叫“票金”。据我所知道的,三不管的戗(qiàng)金(相面的)有十几个,分为三大支派:一是陈大官的门人弟子;一是刘五先生的门人弟子;一是桂振峰的门人弟子。

陈大官系山东腿儿,长得相貌最好,说行话,他是“人式压点”(yā diǎn,震得住人为压点),胆大敢言,得有江湖真传。各省市、各码头、各村镇他都去的。有好些个做戗金的能在乡间挣钱,不能在都市码头挣钱;有些个做戗金的能在省市码头挣钱,到了乡村不成的,调(diào)侃儿叫“不吃科郎(kē lang)点”(庄稼人)。惟有陈大官这个做戗金的是省市商埠也成,乡村镇市也都能成。凡是江湖跑腿的人,只要一提陈大官,无人不知,他的生意到处“火穴大转”(zhuàn,买卖挣了大钱了),因为他有“万儿”(名儿),有好些个人拜他为师,给他“叩瓢”(江湖人管磕头调侃儿称叩瓢),有为学他的本领的,有借他的万儿走闯江湖的。在天津三不管有个相面的周岐山,自号亚卧龙,生得身躯短小,眼大口方,拜陈大官为师,在大连、烟台、营口、天津、青岛、济南、龙口等码头,安过些回“座子”(江湖人管设立临时命馆调侃儿叫安座子),总是初立的一个多月生意最好,过了一个月后就不能支持,江湖人都说他学的生意,前棚最硬,后棚(一见面的前三抢儿调侃儿叫前棚,多挣钱、使人佩服调侃儿叫后棚)最软,始终是虎头蛇尾。他在天津某公寓内安了回座子就是这样,后来支持不住了,到三不管去搁明地。我老云在他圆粘(nián)子(招徕观众)的时候立着听了听,只见他说的很有派儿,也会“触簧”(管用冷话硬撞周围的人调侃儿叫触簧),也会往下“叫点儿”(叫住相面的人),到了“散帖”的时候,愿意相面的接条儿,行话叫“归包口”(说完一段故事,再售其货,调侃叫包口)儿,“撒幅(sǎ fú)子”(往周围人手中撒算命的号儿)也很有人接帖。他的“杵门子”最硬(相面人管能挣钱,敢向人要钱,有要钱的手段调侃儿叫杵门子最硬)。钱到了他的腰内之后,给人相上面哪,只有几句干脆嘹亮的,越听越不像事,使人对他的信仰上立时失望,当时就后悔,他哪能有“回头点”(江湖人管有人花了钱相面,应验了之后还不断地找他们相面,调侃儿叫回头点,以有回头点为最大的光荣)呀!我见亚卧龙这样,才信人传言,他的后棚欠研究,传授不真。我向江湖人探讨,有人说,他只会“腥”(假的),不攥(zuǎn)(懂)“尖儿”(真的),不懂得“尖册(chǎi)儿”(江湖人管熟读相书叫懂得尖册儿,没读过相书叫不懂尖册儿,还是以钻尖儿为高明)。像周岐山的本领,只能打“走马穴(xué)”(走一处,不能长占,总是换地方挣钱,江湖人叫走马穴),天津也不能久长,至今不知他哪里去了。

在三不管相面的生意做的日期最多的有个郑耀庭,是河北沧州的人,他从前挑个竹筐收碎铜烂铁,没有事的时候常逛天津的西城根。那块生意虽在清末的时代,也很发达。“戗(qiàng)盘”(相面)的生意,有两个高明的安座子的最好。来了相面的人,他一见面就知道人的心内有什么事,几句话叫人心服口服,如遇仙人。江湖的人们常说,把(bǎ)现簧(常瞧当时的心事调[diào]侃儿叫把现簧),高绪斋第一;在街上做干跺脚的(江湖人管相面的人不用桌凳,不使棚帐,只凭他空人一个,在墙根底下一站,拿管铅笔给人相面挣钱,说行话叫做干跺脚的生意),刘五先生最高。那刘五先生是南皮县人,开过“汉壶瓤子”(管开草药铺调侃儿叫汉壶瓤子),因为和人“朝(cháo)翅子”(江湖人管打官司调侃儿叫朝翅子),他改行吃“金”,学会了相面。他长得身高、面庞儿大,人式很压点(yā diǎn)(震得住人为压点),“朵儿(字儿)又清”(江湖人管字写得好,有学问,调侃儿叫朵儿又清),又攥“尖儿”,使人情做生意,永远不“鼓点”(江湖人管没人和他们打架,没人和他们争吵,调侃儿叫不鼓点)。每天到下午,只要他往墙根一站,立刻人就围上,行话叫“自来粘(nián)子”,“顶点数(shǔ)”(江湖人管相面的主顾一拨挨一拨接连不断地谈相,调侃儿叫顶点数),哪天也挣一两元钱。除了下雨下雪的天不能挣钱,好天好日的,永远那样挣钱。在那个年头要每天能挣一两元钱,能比现在挣七八元还好。那郑耀庭天天去看相面的,瞧着刘五先生挣钱的本领,生了羡慕之心。刘五先生每天瞧着他听相面的,就知道他有意习学这行儿。有天收了市的时候,向郑耀庭问道:“你干吗天天来看相面的?”郑说:“我来看这个,既在江边站,就有望景的心。”刘先生说:“你要爱惜这个,就学学吧!”郑说:“我学不了,没念过书,不认识字哪能成啊?”刘先生说:“不认识字没关系,一样能学。就是看此心专不专,如果专心学练,一定能成。”郑说:“我能专心学的。”于是他二人商商量量就成为师徒。

刘五先生的传授很好,因为他不认识字,不教他做高了,只挣“贸易点”(商人)、“科郎(kē lang)点”(庄稼人)的钱。所有相面用的方法与所说的话,都是粗糙的言词,不到三四个月,学成了就能上地(做生意),做戗(qiàng)盘(相面)的生意。和他师傅一样,任什么东西也不拿,只用几张纸,一管笔,到三不管圆粘(nián)子(招徕观众)就挣钱。

天津的社会是工商业的劳动区,手艺人多,河岸码头卖力气的人、赶车的人、使船的人,就比哪儿也多。这些人虽然是无资产的劳动分子,只要一晃膀子就能挣钱。在民初的那些年,天津这地方是真发达,哪个人凭力气也能挣一元两元的。三不管将开办,下级的人都去游逛,有这些“科郎点”,郑耀庭就得着好买卖。他是笨鸟先飞早入林。上地早,收得晚,很挣下不少钱。江湖中相面的人就数他在三不管做生意待的日久,二十多年也没挪过地方。人人都说:“他的老帅(江湖人管师傅调[diào]侃儿叫老帅。帅与师只欠一笔,请阅者注意,别以为我的帅字是师字,少一横儿)夹磨(jiá mo)(师父传授真本事)得地道。”可是,他只能养家糊口,没挣过几百元、几千元,只能做“零毛碎琴”(江湖人管不能挣成元洋钱,挣角儿八仙、几十个铜子,调侃儿叫零毛碎琴)的生意。

要说能挣大钱,还得属着他的大师兄云霞子。那云霞子是沧州人,与天津的名武生高福安同乡,名叫于紫阳。自早年拜刘五先生为师,他学会了生意,就不愿意做地上的买卖,往津、沪、汉、烟、济等商埠码头,各大旅馆、各大饭店挂牌相面,遇见通达事务懂得社会里一切诡诈事的人,他没法敲诈,挣个“迎门杵”(挣的头一笔钱)了事;见有那做亏心事的人,做诈一下子。他的手段很是毒辣的,图眼前快乐,不到十年他自己就患起“丢子(si)”(江湖人管疯人调侃儿叫丢子)。我老云向江湖人探讨他为什么疯了的。据说,他“挖(wǎ)点”(敲诈人)太多了,伤了“攒(cuán)子”(江湖人管做亏心事调[diào]侃儿叫伤攒子)才这样啊!世上的事有因果报应,说起来叫人可怕,伤天害理的事还是做不得呀!在前几年,往天津地道散步,遇见了于紫阳,他穿的衣服破烂不堪,面貌枯槁,两眼发直。将他截住,我问:“先生,你怎么这样了?”他说:“我不认识你。”我说:“当初在河北竹林村煤铺西边的小胡同内,我给你们了过事,难道你忘了吗?”他惊愕不已,连说:“遇见神仙,遇见了神仙。”往东而去。至此,我才知道他是真疯。

那刘先生一共收了四个徒弟:大徒弟有本领,几百几千地挣,可是疯了。二徒弟郑耀庭,就能挣个零毛碎琴(江湖人管不能挣成元洋钱,挣角儿八仙、几十个铜子,调侃叫零毛碎琴),没有多大的来历,闹得衣食不缺,无病无灾。三徒弟×××,本领也好,可惜他的“果食码子”和他人“扯(chě)了”(江湖人管媳妇调侃儿叫果食码子,管跑了调侃儿叫扯了);到了烟台,坠入“库果窑”内,成了“库果”(江湖人管娼家下处调侃儿叫库果窑儿,管妓女调侃儿叫库果),大约着也是伤了“攒子”。四徒弟孙耀西,“戳的朵儿真嘬”(管字写得好调侃儿叫戳的朵儿真嘬),“幌幌(huàng)”(管贴的报子调侃儿叫幌幌)上的“万儿”(有了名儿),是华阳山人。他二十三四岁出师,往各码头做生意,很为不错,挣了不少钱,刚娶了媳妇,就“粘啃(nián kèn)押头”(管得了重病调侃儿叫粘啃押头),“咯光子血”(管吐血调侃儿叫咯(kǎ)光子血)“土了点”(即是死了)啦。闹了寿夭,大约也是伤了攒(cuán)子。

刘五先生只有一个儿子,父传子受,也做戗(qiàng)盘(相面)的生意。二十多岁的人,先“抹海(mò hāi)”后“插末(chā mòr)”(管吸鸦片烟调侃儿叫抹海,管扎吗啡调侃儿叫插末),成天价往各处行窃,自顾不及,哪管他父母。刘五先生年老气衰,挣钱的能力一日不如一日,竟困难得衣食不保,老早就去世。他们的师徒只有郑耀庭一人,安然久过,没出什么毛病。其余的都没得好。不怪江湖人常说:“多挣钱,多作孽!”若是为商家将本图利,多挣钱也没事呀。我劝没能为的金点们,虽不能多挣,顾得住衣食就不用学那伤攒子、翻钢叠杵(通过花言巧语使买主翻倍付钱)、挖点(占人家的便宜)的手段。刘五先生师徒就是前车之鉴。苦海无边,回头是岸。老合(闯江湖的)们何不醒攒(cuán)儿(管觉悟过来叫醒攒)!





三不管的杨大将


有年冬天我往天津看望朋友,住在客栈内。清晨早起,往海光寺绕弯儿,临回来的时候,走到三不管枪毙人的行刑场(上权仙电院南边),见靠西墙根围着一群人,不知道是干吗的。挤进去一看,见场内是个地摊,地上铺着一块毯子,上边放着一个罗盘,大小十几个定南针。有一块石板,两根石笔,一根文明杖,场内有个人不住嘴地嘟囔。这人长得瘦,中等的身材,他穿着小棉袄、棉马褂,没穿棉袍子,底下是棉裤、棉鞋,我不知道他是干吗的,定住了心神慢慢地听。见他用手指着一个人说:“这位老兄多大年岁?”那人说:“我今年三十七岁。”他又问:“再添上十三岁,你是五十岁对不对?”那人也笑了。他是个又愣又怯的样儿,又用手指着另一个人道:“这位老兄多大年岁?”那人说:“我今年四十一岁。”他说:“再添上十九岁,您是六十岁对不对?”那人说:“对了。”他说:“我这根文明杖,往你身上一挨,就知你的媳妇克不克。”说到这里,他又向那人说:“您的媳妇宜小不宜大,大嫂子比你大比你小呀?”这人说:“比我大三岁。我十六岁那年娶的。”他说:“坏了,坏了,娶得早了,非克妻不可。”那人说:“对了,我媳妇死了。”他听说对了,向围着的人大声嚷道:“又对了一位。相得不对了,倒找大洋一块。我这几天始终也没找出钱去,叫我着急。”

他的调门忽高忽低,惹得众人直笑。我看到这里才知道他是个相面的。听他相了好几个人,都是白送不要钱。这回他又向一个四十多岁的人说:“你这人,媳妇宜小不宜大,大了得克去。大嫂子多大年岁?”这人说:“她今年四十五岁,我今年四十二岁,比我大三岁。”他又问道:“死了没有?”这个人说:“没死。”他听着没相对,又向这人说:“现在没死,早晚得克了。你回家别跟她说,您要跟她说,她就骂我,真叫我急。”他这样一说,围着的人全都乐了,可是大家这一乐,把他那没相对的事全都忘了。我老云云游了十几省,看见过多少金点(算卦的总称曰金点),什么样的都见过,还没见过他这滑稽派的相士哪!

可是他随送相随着抓哏(包袱儿),真比说相声的不在以下。抓了哏,听的主儿乐,还没有不咧瓢(liě piáo)(大笑)儿的。他这逗笑的好处能给自己遮丑儿。相得不对了,大家一笑全都忘了。我曾听老江湖人说过:“万象归春。”说相声的叫人一乐就叫春。不论是哪行儿,也是逗笑儿好,电影的片子还是笑片能引人入胜,戏台上还是有丑角儿才能热闹。唱大鼓书的也有老倭瓜、架冬瓜的滑稽大鼓;单弦呢,也有群信臣的滑稽单弦;说评书能有叫座魔力的双厚坪、品正三、刘继业、袁傑英、海文泉等,都是以把人逗笑为拿手。“万象归春”这话是不假,哪行儿能会滑稽术的也能受人欢迎。

这个相面的仗着会使滑稽艺术,不用使拴马桩(用话把人扣住)儿,也不圆粘(nián)子(招徕观众),围着的人谁也不想走。他到了归买卖要挣钱了,向观众说:“我姓杨,双名叫润斋,京南固安县的人,人称杨大将。我到过霸、宝、文、大、固、永、东,昌、顺、密、怀、平,大、宛两县,涿、良、房。京兆二十四县,提起来杨大将没有不知道的。天津也常来。那位说,你这是相面吗?不是。这是卖扁食(水饺)的喝汤,引引人。要是相面哪,是相人老中少三步大运,住什么房子,妨父母不妨,克老婆子不克,有几个儿子,有几个闺女,应当在哪界做事,富贵贫贱,穷通寿夭,连坟地带孩子,连老婆子带宅子,洗脸带捋胡子,一连带架全都有啦,大洋两毛,多了不要,少了不谈,哪位要相,哪位说话。”真有几个人叫他给相。他是随相面,随抓哏,围的人始终不散。我听他相了几个人,笑得肚肠子都疼了。较比听万人迷的相声还觉着热闹,站得腿都酸了,我才回店歇息。用了饭之后,有我的朋友曾文盛约我在下天仙去听玩艺儿,直听到散了戏。往恩德元吃饭,又去逛法租界,往某胡同里遛了一个弯,坐了不大工夫,就听见大门外有人喊嚷:“算卦,相面,看手相不要钱。”声音忽高忽低,招惹得各屋子游客全都笑了。跟着这位相面的先生就进了院子,隔壁的屋中有位客人把他叫了进去,给那妓女相相面,只隔着一层木板墙,往那屋听得很真。他们并不是相面,而是彼此抓哏,来了一回对口相声。这个乐,那个笑,十分热闹。结果,那位游客花了四角大洋,那位相面的先生才出来。我跑到院内一看,这位先生就是那三不管(天津市南市的一个露天市场)的杨大将。

天地之大,无奇不有。做金点(算卦的总称曰金点)生意的人也有滑稽派的,真是叫人想不到啊!





三不管的八岔子(奇门卦)生意


在三不管的南头每逢下午,有个算卦的。天天儿,他还没到那儿,问卜的人就先到了,在附近来回打转,净等着他来了好算卦。我好奇心盛,觉得这位先生一定高明,特意地看了他几天。只要他一到,把摊子摆上,四面的人就围满了。他算的是“奇门卦”,那六十根签子往筒内一放,这个也伸手,那个也伸手,一阵乱抽,眨眼之间就把签子抽出一半,大家攥着签子等他算卦。我往他这摊子上看那“局式”,就知道他是腥门(假的)了。什么叫做局式哪?就是他那摊上正当中摆的那九个卦子,横三行,竖三行,每行三个。那卦子上是戊己庚辛壬癸丁丙乙,按《奇门大全》说,那叫局式。凡是算奇门卦的都得先把局式布好了,然后有人算卦,再按着签子上的字,往局式上摆卦,要学奇门最难学的可就是这局式。有些个江湖人要做生意,只把那江湖术学好了就能挣钱吃饭,谁也不花多少年的工夫去学那奇门遁甲。老合(江湖艺人)们的奇门使“尖盘”(真的)的虽有,总是不多吧。

他那卦摊我听了几天,听他给人断的卦语都是“八面风”,怎么说怎么有理。他那摊上问卜的人,不是都来问卜的,有七八个人都是“敲托”(暗中帮助做生意的人,也可称为贴靴的)的。有些个人都管这位先生叫“卖油郎”,我不知是何缘故,向人们打听为什么叫他卖油郎。据知他根底的人说,他从前是个挑担卖香油的,受某江湖人夹磨(jiá mo)(师父传授真本事),他弃了香油担改了“八岔子”(奇门卦)。他有几个敲托的,又会使几道簧,卖弄钢口(说话的技巧和分量),生意很发达。一般人都不叫他的姓名,叫他“卖油郎”。很兴旺个十几年,到了民国十五年,他的生意就一落千丈了。

在南市第一舞台的西南方,德美后前边,路北有一溜小铺面房,西头路北的门前有张小桌子,桌上有个小檀木签筒子,筒内六十根签子和那全份的卦子,也都是檀木的,上边支着个小布棚,上写三个大字是“厂×士飞星奇门”。桌后边坐着一个老先生,有五六十岁,胖大之躯好像老寿星一般,他那卦摊上问卜的人一天价紧忙,接连不断就不住闲儿。我看见他那人,才想起来曾在大连西岗的某油房前边久摆奇门的“厂×士”。他是北平东边通州的人氏,是个书香门第,饱学之士,摆的奇门不是腥盘(假的),纯粹是“尖盘”(真的)。他断卦的口吻稍带一点江湖味儿,他一辈子只在大连、天津两处,挣的钱就够养老的,做了一辈子响万儿(成了名儿)的生意,腥尖皆通,火穴大转(zhuàn)(挣了大钱了),那就应了我老云的话了:“腥加尖(假的加真的),赛神仙。”

世上的事,不论那行,净耍腥儿(假的)是不成呵!在民国十二三年的时候,高大愣卖大力丸的场子对过,有个年纪最小的摆八岔的,也就有二十岁,了不得啦,他那摊上写的是:“连仲三诚演奇门。”我老云听过他几天,见他买卖虽然挣钱,可是一腥到底(全是假的),得了江湖的传授,使腥儿、卖弄钢口(说话的技巧和分量)最好。口齿伶俐,很能警人。只是他不攥(zuǎn)尖儿(不学真的本事),美中不足了,也是他的缺点。他惯“戳簧”。什么叫戳簧哪?譬如有人去占卦,他把卦摆得了,问那问卜的人说:“你这卦是问财?”问卦人一点头,他就说:“我这卦一看就知道你是问财。”如若那人不点头,可是他心灵嘴快,立即就说:“……或是问事,我都能看得出来。”他那愣戳,戳不对的时候,不等问卜的人发言,立刻就说或是问事,随着就拐弯,调(diào)侃儿叫“抽撤口儿”。据江湖人说,他年轻,不大,跑的腿儿长,自从幼小拜天津北开花柳座子的杨春山为师,论江湖人的支派,他是山东德州×家庄的门户,他们那门人都是挑(tiǎo)招汉儿的(即是卖眼药的)。当其学成了生意时,与德州达官营的潘长鸿往烟台去做“四平粘(nián)子带搬柴”(江湖人管出高案、卖丸散膏丹各药的调侃儿叫四平粘子,管带拔牙调侃儿叫搬柴)的生意。在烟台的南市场泳仙楼前,很做了二年好生意。后来潘长鸿往大连去了,他们“劈(pǐ)了穴”(江湖人管分了伙调侃儿叫劈了穴)之后,他一个人在南市场又安了(开了)“柴座子”(江湖人管开镶牙馆调侃儿叫柴座子)。他做了未久,又与做八岔子(算卦中的一种)的张子庚学了摆奇门,遂弃了“汉门”(凡是卖药的调侃都叫汉门)的生意,又吃了“金门”(凡是算卦相面的调侃都叫金门)啦。每逢冬天的时候,他在烟台的后海沿去“挑顿(dūn)子汉儿”(江湖人管卖咳嗽药调侃儿叫挑顿子汉儿)。

有年,我老云在美阳会上还见过他,正做“戳黑”(江湖人管点痣的调侃儿叫戳黑的)的生意。民国八九年,他在天津的南马路还挑过“熏(xūn)子汉儿”(江湖人管卖闻药、卖避瘟散的,调侃儿叫熏子汉儿)。民国十二年他与“光子”(拉洋片的调侃儿叫光子)上的王秉肇到了营口洼坑甸做八岔子。十三年回到了天津,未久他又去北平,在天桥吃“金”(算卦),响了“万儿”(有了名儿)。至今,他又改了“团(tuǎn)柴”(江湖人管说评书的调侃儿叫团柴)啦。

若在海北海西,提起连仲三来,“万儿很正”(这个人不错)。阅者诸君若问他为什么万儿念的,就是为了那个。





江湖中金点的黑幕


老云在今春往开封有事,得闲去逛相国寺,见各种杂技场都围得风雨不透,数山东大鼓、男女合演的鸳鸯档子尤为叫座,较比坠子还受欢迎。这些玩艺儿我都不喜欢去看。往里边走,见殿后有个“疙瘩粘(nián)子”。阅者要问什么叫疙瘩粘子,据他们江湖人说,大玩艺儿场围的人多,调(diào)侃儿叫海(hāi,多)粘子;小玩艺儿场围的人少,调侃儿叫疙瘩粘子。我不知道那疙瘩粘子里是什么生意,挤进去观瞧,见场子内有张小独桌,两旁有两个小条板凳,桌上放着破笔墨盒、纸条子,有一对玻璃框,内写着“直言无隐,概不奉承”。桌后边立着一个人,长得细条身材,白面庞,五官清秀,穿着打扮像个官僚,两撇小黑胡子,大概是个相面的。

只听他的话是南方口音,好像江浙的人。他说:“袖里乾坤大,壶中日月长。我是从此路过,要传个名儿。住在旅馆内,有本处政界的伟人由上海把我约来给他们谈相。我曾听人说,开封是过去几千年前故去的都城,风淳土厚,这里的人都守旧礼。我要逛逛相国寺,偶步闲游,在庙内要送相法。相对了大家给我传名,人过留名,雁过留声。人过不留名,不知张三李四;雁过不留声,不知春夏秋冬。我有个名,大家诸君常看上海报登有大相士小糊涂,就是在下。今天咱们送相,分文不取,毫厘不要。我可有几种人不送:聋子不送,哑巴不送,不孝父母不送。我送的是读书识字明情知理的人,就是不认字,久闯外面、通达人情的人。可是多了不送,只送六相。哪位愿意相,哪位伸手接我的纸条,接着了也不用喜欢,接不着也别烦恼,接着了就有一相。”说着,他就拿起六条白纸,社会里的人有好贪便宜的通病,围着的人争先恐后争那纸条。我抢着把末一张纸条接过,他叫我们六个人都站在桌前,一一地站着。相面的先生左手攥着一把纸条,右手拿着一管笔,往墨盒里蘸了蘸,就冲着头一个人往纸上写了写。写的时候把手举起多高,捂得很严,不叫大家看见写的是什么,只叫身后的那人看见。他还冲着身后那人说:“头一位他弟兄几个,我能知道,你看见了没有,就是这几个。”那身后的人笑了笑。他向头一个人道:“你兄弟几位?”那人道:“我们哥儿两个。”相面先生就大声喊:“相对了一位。我这先写两位,他就是哥儿两个。”说完了又换了一张纸,还是捂着不叫人看。他用笔写了几个字,写完了冲着身后的人亮那纸条说:“第二位就是弟兄这些个。”说完了又向第二个人问道:“你兄弟几位?”第二个人答道:“我兄弟四个。”小糊涂又嚷:“相对了两位!”于是,他就用这先写后问的法子,一个一个地相。到了我这里,我说是哥儿三个,他也嚷相对了。我可是没看见他那纸条,不知他写的是什么,总是疑惑有假。他相完了六个人,就说:“这就是相面吗?这是送相。要真相面,不能这样简单。讲究相人老中少三步大运。哪年妨父母?哪年克妻?什么年立子?士农工商,应当在哪一行做事?是当人中领袖,是给人做事?哪年不好,哪年发达?得谁的好处?受谁的害处?由幼小直到老,全都说到了,那才叫相面。相一面得多少钱哪?若按我的润格是:细谈相法五元。今天在这相面要多少钱哪,别说五元,连一元也不要,特别优待,只为传名,收两毛钱一相。要全都花两角相面,我可不相,只相八个人。过了八相之外,谁要再相,可要五元钱。也许你不相,也许我不谈,哪位要相,哪位接我的纸条,接着了算有一相,接不着没有。可是接着不用喜欢,接不着别恼。”

他说着就另选了八张纸条。他说:“相对了两毛钱归我,相不对了你再拿回去。”于是就问谁相谁伸手,我们这六个人每人都接了一张。格外还有两个人也接了,共是八个人。他叫我们在桌子角旁的凳上都坐好喽,他说:“咱们是相金先惠,不对退还。”我们八个人都掏出两角票来,一块六大洋放在桌上。相面的小糊涂用墨盒压好,就按着次序给八个人谈相,我老云当然是末一个。

他给第一位相终身的事,我都不留神听,惟有相到兄弟几位的时候,我见他把先写后问的几张纸条,都攥在手里,把一张没有字的扔了。看那有字的第一张,上写“兄弟两位”。我看完了,心中很是佩服,他的相法真是先写后问,写得对,相得对。那第一个人也心平气和,花了两角欢喜而去。接着,我又由第二位看起,直看到第五人,无一不对,不只是这五位的兄弟几位全相对了,他们在哪行做事,脾气秉性,经过的运气好歹,分厘不错。至到了我老云的时候,他相得对不对我不做声,也不摇头,也不点头,还既不定神,也不走神。他那纸条上写的是兄弟三位,倒是对了,只是他说的我的职业与我的性情等等全都不对。

这相面先生使的是“小蜕皮”的手彩。



相完了面,我回到客店,回思往事,疑虑颇多,总不相信小糊涂的相法有准。次日,我又去他那儿看热闹,正赶上一个某甲和他捣麻烦,他那纸上写某甲是兄弟四个,某甲说不对。小糊涂说:“将才你说是兄弟四位,这是怎么又改了哪?”某甲说:“我五弟出门在外,我将才说错了,可是我说错了没关系,你相错了可不成!”我听某甲和他争吵的事,对于他那纸条上先写后问又生了疑心,觉得他定有“手彩”(手上的功夫和技巧)。至于什么手彩,实在不知。

我由开封回来,路过保定,在马号遇见一个老江湖的朋友,请他在饭馆喝酒,闲谈起来。我忽然把在开封相国寺小糊涂给我相面的事说给他听,他只是好笑。我问他:“这相面的纸条上先写后问有什么手彩?”他说:“那叫小蜕皮。”我说:“什么叫小蜕皮哪?”他说:“那小糊涂左手攥着几张纸条,先写上某人兄弟几位,然后再问某人兄弟几位,说的写的俱都一样,那个也是手彩。”我问:“究竟他那手彩怎么使呢?”他说:“譬如小糊涂给三个人看相,他左手攥纸,右手用笔往上写,捂得挺严,不叫人看见,他是假装往上写。事实真没写,他问头一个人你兄弟几位,那人说三位。他就喊相对了,那是蒙人,纸上还没写哪,他唬事。他问出头一个人是兄弟三位,把人家的弟兄数儿蒙了去,再给第二个人相面的时候,往第二张纸上写兄弟三位(注意,第二张写的是头一个人弟兄几位)。写完了,他又问第二位相面的是弟兄几位,他知道了第二位人说是两位,他又假装给第三个人看相,往第三张上写兄弟两位(第二个人的兄弟数又被他诓了去)。他写了,又问第三个人你兄弟几位,第三个人说我哥一个,他又往第四张纸上写兄弟一位,写完了假装再给第四个人看,诈称我对了。照这样弄法,他手中那些纸条,第一张白纸没字,第二张是第一个相面人的兄弟三位,第三张是第二个人的兄弟两位,第四张写的是第三个相面人的兄弟一位。再问一个别人,那是遮掩法。等到他叫人看那纸条的时候,把头一张没字的白纸扔了,就叫‘小蜕皮’。蜕了那一张皮,第二张改成第一,第三张改成第二,第四张改成第三。局外人不解其意,往第二张上看,果是兄弟三位;往第三张看,果是兄弟两位;往第四张看,果是兄弟一位。谁也想不到这种手彩呀!江湖的相士在各处相面,都是用这小蜕皮的方法。可对外人他们绝不说。个中的黑幕,非得收了徒弟他才肯将这小蜕皮的黑幕,传授给他徒弟。”我听他说完了,如梦初醒,才明白过来。据他说:相书上对于相人兄弟几位,并没有准对、准看出几个人的相法。如若不懂江湖术,无论学识多大,看多少年的相书,相人什么都对得了,若相人兄弟几位,管保对不了,尖册(chǎi)子(江湖人管《麻衣相》、《三世相》、《柳庄相》、《铁关刀》、《相理衡真》、《大清相》等书,调[diào]侃儿叫尖册子,即是真正相学书也)也多不可靠。江湖人是取尖册中有准对的学理,与江湖手彩并用,才能叫人相信了。若不熟读相书,只会个小蜕皮手彩,也是不能挣钱。

小蜕皮是金点(算卦的总称)中的一种黑幕,至于金点中的全部秘密,有千八百样,各有巧妙不同,也是学之不尽,外人探讨不完哪。我听老江湖人说完了,才不敢自骄。以我老云的江湖知识说呀,所知道的不过百分之一,不知道的还多着哪。等我慢慢地探讨,得一事,向阅者报告一事,总以爱护多数人,揭穿少数人的黑幕,为大众谋利除害,以表示我老云忠心于社会啊!





江湖金点中之自来簧


保定府在清时是直隶的省会,市面繁华热闹已极,到了民国十年以后,直系势力盛时,也比今日兴旺。那里的杂技场儿在马号。我有时候到了保定也去逛那马号:“一杆大旗”刘香久、“炮打不散”尤鹤亭、“死不要脸的”袁×亭三个人的评书我也听过几次,倒是各有巧妙不同,都有叫座的魔力。到了夏天卖香面的也有一两档子。变戏法的、卖艺的也有几档子。最多不过是拉洋片的。有一次我见靠墙根有个相面的先生在那里撂生意,既不设桌案,也没有凳子,只是左手攥着一沓儿纸条,右手攥着一管毛笔,约有三十多岁,白白的脸庞,很是精神。他往那里一站,看热闹的人就把他围上,大约着是一档子作响了的生意。听他说是叫张半仙,他在这里,一天多了不相,只相十个人。相面的礼金两角,少了不相。他给围着的人白相,那是奉送几句,我听了会儿,他送了几个人的相,所说的很有意思,人人点头,给谁相谁说对。他是这样说的:“我张半仙的相法与众不同。有那一种蒙人相面的,他问人家多大年岁,人家告诉他五十七岁。他说你父母受克全都死了,那老头还说对啦,其实那全是蒙事。众位想想,人到了五十七八岁,有父母的很少,他都五十七八快到六十了,他父母活着岂不八九十岁?世上活到八九十岁的不多吧?老年人你要相他父母不在,那是蒙人。我这里相面,是老不谈父母。还有一种相面的,他问人家多大岁数,人家告诉他十五岁。他说人家还没有儿子哪!人家准得点头说对。十五岁得儿的倒有,万里挑一。普通的人要在十四五岁,不要说有儿子,娶了媳妇的都少,相面的要给少年人相没有儿子,都是蒙人。我这里相面与众不同:是少年人不谈子宫。那位说,你张半仙这里相面是怎么相啊?我这里是少年能知道他父母有无,是全都妨去了?是父母双全?是死去了一位还有一位?一看便知。老年人,我能知道他有儿子没有,还能准知道他有几个。中年人,我能知道他是弟兄几个,众位如不相信,咱们就当面来试。怎么试验法呢?”他说到这里,把那些纸收到兜内,把左手的大拇指一挑,说:“我看哪位的相貌是弟兄几个。看完了,我往大拇指头肚上先写好了,哥一个画一道,哥两个画两道,有几个画几道,画完了,我叫他自己先说是哥几个。他说完了,再看我的手指头上画的是几道儿,如若是一样儿,那才算我相对了。如若不对,那算我经师不明,学艺不高。”他说完了,就向人群里看,用手指着个二十多岁的人说:“这位兄弟几位,我知道了,我先写上。”他把左手举起多高来,捂得挺严,不叫人看见,用笔画了一画,然后又看了看那人,他直摇头,又用舌头把手指上画的舔了去,重新另画。画完了把左手往袖筒内一藏,他向那人问道:“你是弟兄几位?”那人说:“我是哥两个。”张半仙说:“众位听明白了没有?这位可是哥两个。”说完了他把左手伸出来,一露大拇指头,大众往他手指上一看,果然是画了两道儿,谁都佩服他,相得真对。他又说:“我相对了一位,不算,这也许是蒙对了,撞对了。咱们要是把众位全都相对了,那才算我的本领。”他说完了又用手指一个人道:“这位有四十多岁了,他兄弟几位,我看出来了。”说着他又用舌头把大拇指头上的两个黑道舔去了,又用笔画了画,捂严了不让人看见,把左手又藏在袖内。他问那四十多岁的人:“你是兄弟几位?”那人说:“我们哥七个。”张半仙说:“众位听见没有?这位是哥七个。”他说完了就把左手伸出来,叫大家看他那手指头,大众一看,果然他手指上画了七道了。不用说别人,就是我老云也佩服他了。

他接连不断相了十几个人,全都相对了。他可就说:“众位,净相哥几个那不算本领。要相面,讲究相人一世终身,少中老三步大运。妨父母不妨?克妻不克?哪年享福?能有几子送终?沾谁的光?得谁的济?受谁的好处?被谁所害?士农工商应在哪行?富贵贫贱,一辈子能有多大财气?在家好在外好?几时发达?几时被困?衣禄食禄高低?由幼小直到老,样样都相对了,那才叫相面哪!”他说到这里,往左右前三面一看,又说:“按着这么相得花多少钱哪?大洋一元。那位说,一块可多点儿。那就这么办吧,我来个特别优待,今天咱们相面只收两毛大洋。可有一节,我多了不相,只相十位。在这十相之内,我每位收大洋两毛;十相之外再有相的,可是一块钱一相。我这里有十个纸条,哪位愿意相哪位伸手接我的纸条,接着了就有一相,接着了也别喜欢,接不着也别恼。”他说着就把十张纸条数了数,左手攥着九张,右手拿着一张,说:“哪位要相,哪位伸手!”就有人接他的纸条,接着不断,十张纸条真都有人接去。他又向众人要钱,是先给相礼,然后相面,每人两毛,一共是两块大洋入了他的腰柜。他就给这十个人谈起相来,我老云在旁边听着,他相这十个人的性情如何,所做的事情高低,已往的情形,都能说对了,相的人们点头咂嘴,无不佩服。我老云直看着把这十个人全相完了也没走,那围着的人也没散。忽然从外边挤进来一个人,长得肥头大耳,方面广额,衣冠楚楚,气势凌人,约有四十多岁。他冲张半仙说:“张半仙!我听人传说你的相法最好,你看看我是有儿子没有?我是几个太太?”张半仙说:“你这相貌很不容易相,你是多大年岁?”这人说:“四十六岁。”张半仙说:“你还没有儿子。”这人用手一拍巴掌道:“好先生!我真是没有儿子。”张半仙说:“你还不是一位夫人。”他说:“你看我有几个媳妇?”张半仙说:“两位。你的大太太不生养;二太太生养过,没有立住。”这人喜欢得直跺脚儿,说:“你可称神相。你看我将来还有儿子没有哪?”张半仙说:“你要问将来准有儿子没有,你掏十块钱的相礼吧!”这人说:“怎么大家相面要两毛,和我要十元哪?”张半仙说:“十块钱还算少要了。”这人说:“先生你交个朋友吧。”说着由怀中掏出皮靴掖,取出五元一张的洋钱票递给他。张半仙接了过去,说:“你这人的财命很多,做过几档子好事,准保不能绝后。能有儿子,可是一子送终。”这人说:“我在哪年立子呢?”张半仙说:“远在明年,近在今年的后半年。”这人把大拇指一挑,说:“我真佩服你,应验了我来谢你。”说完了转身就走。

我看他费了一个多钟头的话,才挣了两元钱,说的话真没了数儿。这个人来,他才费了几句话,就能挣五块大洋。我就觉着人们常说“挣钱不费力,费力不挣钱”的话,说得很多,越是费事越不挣钱,越是挣钱越不费力。我由他那里回来,信步而行,对于张半仙的本领真是佩服。我走到寓所,把这事记在心上。

有一次我到了天津,在某旅社遇见了个老江湖的朋友,闲说话提到了我在保定府马号看见张半仙的事。他说:“相面这行儿,调(diào)侃儿叫金点,又叫戗(qiàng)金,又叫戗盘(盘当脸讲)的,像张半仙那个相面的,也不支棚,也不设帐,连张桌儿都不用,只用几张纸条儿,一管毛笔,调侃儿管他那种生意得叫‘干跺脚’。”我说:“他们能相人哥几个,往左手的大拇指头先画黑道儿,后看对不对,人家说哥几个,他手指头上就是几个黑道儿,那是怎么回事哪?”他说:“他那个方法很是巧妙,若按着江湖的侃儿,叫做‘五音碑’。他那画的黑道儿,不知道的都以为是先写上的,其实不是。他是先问明了,然后写上的。”我说:“那可奇怪。我看着他先写上,然后把手收在袖筒里。你说后写的,他一只手怎么往上写呢?”他说:“做这种生意有个门子,和变戏法儿似的,不叫人看见,他的袖筒内藏着一支小笔。”我说:“他那小笔怎么个样哪?”他说:“那笔如同药铺内卖的万应锭大小,是由纸铺买来的墨,砸碎了,弄成细末儿,然后再用胶水合,内里要捻上一根极粗的线,把它揉成了嘎嘎形,当间粗两头儿尖,一头有线头,一头儿尖,放干了。用时把那粗线头儿缝在袖筒里,嘀啷搭啷的,如袖中蒙着一管小笔儿,外人如何能知道?他要使这法子的时候,或是先用唾沫湿一下子,或是假装的写错了然后再用舌头舔了去,重新用笔写。他捂得挺严,外人看不见他写的是什么,他用笔瞎晃悠,并没写。向谈相人问哥几个,问明了是三个,他乘着手指头的湿劲,用藏着的小笔尖,往手指上画三道儿,伸出手来叫大众看那手指上的三个黑道,谁看了也得佩服他的本领,绝想不到其中另有鬼病。”我听他一说,方才明白其中的黑幕是这么回事。

我又问他,那张半仙给人相完了面,忽然来了一个人,冷不防地问他,能看出他有几个媳妇?有儿子没有?张半仙看了看就说对了。还没有儿子,不是一个媳妇。他是真有此本领啊,还是其中另有什么诀窍?他说:“你不懂这些事,隔行如隔山。那人来了,冲他一问,立刻就能明白。大凡人要找相面的,别的不问,只问他有儿子没有?他们相面的有一种诀窍,共为十三道簧(十三套办法),这问有儿子没有是自来簧(张嘴就能问的话)。他本人就把簧(实话)露出来了,可以意会,不可以言传,听他问的口吻就推测出来他是没有儿子。方观成写《玄关》上说:‘问子却没子。’大凡世上的人要是家中有钱,都盼望早立子,如若穷得没饭吃,有儿子还发愁哪;没有儿子绝不想儿子。凡是想儿子的,都是富厚之家。倘若年岁大了没有儿子,不是他媳妇没开怀,就是有了病不能生养,一定得娶姨奶奶求养子嗣。那张半仙说得对了,并不是按着相书的书理研究出来的,那是江湖诀窍,点头儿(江湖人管花钱相面的叫点头儿)带出来的自来簧。”我听他所说才知道江湖的事儿有十三道簧中的自来簧。看起来,江湖中的诀窍是令人不可思议,奥妙无穷了!





第三章 挑(tiǎo)方卖药


皮门


“皮”行,是卖药的总名。又管卖药的这行叫“挑(tiǎo)汉儿的”。挑汉儿的侃儿已经通行了。皮行,江湖人多有不知的。卖眼药的叫“挑(tiǎo)招汉儿”的,卖咳嗽药的叫“挑顿(tiǎo dūn)子汉儿”的,卖膏药的叫“挑(tiǎo)炉啃(kèn)”的,卖药糖的叫“挑(tiǎo)罕子”的,卖牙疼药的叫“挑(tiǎo)柴吊汉”的,卖大力丸的叫“挑将(tiǎo jiàng)汉儿”的,卖仁丹的叫“挑(tiǎo)粒粒(lì’r)”的,卖闻药的、卖避瘟散的皆是叫“挑熏(tiǎo xūn)子汉儿”的,管生熟药铺调(diào)侃儿叫“汉壶瓤子”,管卖丸散膏丹成药的铺子叫“汉壶座子”,管治花柳病的药铺叫“脏粘啃(nián kèn)座子”,管洋药房叫“色(shǎi)唐汉壶座子”,管扎针调侃儿叫“插末(chā mòr)”,管注射药品的调侃儿叫“插末汉儿”。





做小帖(tiē'r)的生意


在民国元年的春天,敝人到山东烟台西望看朋友,走在烟台的西南河的地方,见一家栈房的门前站着一个人,手里拿着一把传单,嘴里说:“这店里住着一位大夫,舍药治病,谁要有病,可以进去瞧瞧,白瞧病不要钱。谁要有病,白舍你药吃,就为行好。家里有病人,说出病原来,讨药回去,也是好事呀。”随说着向过往行人的手中递纸条儿说:“接张帖儿,有病进去白瞧白看。”我见有些个人接他那传单,进店去找舍药的善人治病,敝人好奇心盛,也接了一张帖儿,跟着人到店里看看他们究竟是怎么回事。我还要向店里的伙计打听善人住在哪屋呐,哪想到有人站在二门外,专管指路儿,他见了手拿小帖(tiē'r)的人就用手指着说:“你们是治病的,都到那三间北房去。”我随着人们到了那三间北屋内,一明两暗,那暗间放着棉帘子,当中的明间放着一张八仙桌子,两旁有几个条凳,椅子前边有个大洋炉子,屋内很是暖和,有个人照料大众,说话很和气,是个听差的茶房。屋内来了十几个看病的人,那听差的和这些人坐在一处,小声小语地和这些人聊着天儿。忽听见里间屋有人问道:“治病的人来了多少呢?”那个听差的人赶紧站起身形,恭恭敬敬地说:“有十几个人了。”说完了他跑到门前用手掀帘子。就见从里间屋走出来一人,那时候是在正月底,天气还冷哪。就见这人头上戴着一顶水獭帽子,身上穿着绮霞缎面的皮袄,戴着金丝眼镜,精神百倍,气派十足。这时候屋里坐着讨药治病的人,不由地全都起来,垂手侍立,也都恭敬这位先生。他往八仙桌子旁边一站,向大家说:“你们全都坐下。”这些人才敢落座。他坐在椅子上用眼一看这些人,头一个就看见我啦,说:“你这人不是给自己看病吧?”我说:“不错,我是给亲戚家的一位老太太讨点儿药。”他问我:“你们的亲戚得的是什么病呢?”我说:“年年到了春前秋后犯咳嗽。”他说:“那病好治,我给你两丸子百效丹,吃了就好。”说着话他命那听差的人从里间屋内给我拿出两丸药来,把药交到我的手内,他向我说:“那药怎么吃,你回去看看那药上的发票,上边都写着呢。”我说:“多谢多谢。”我又坐在那里不走,想要看个究竟是怎么回事。哪想人家不愿意我在屋内,向听差的说:“把这个调角(diào jiǎo)码子淤(yū)喽!”我听他说的这句江湖侃语,我懂得。“把这个调角码子淤喽”,是指着敝人我说哪,说我是“调角码子”即说是个难惹的人,把我“淤”了是把我轰出去。我当时就明白了,他们不是善人舍药治病,是档子生意,设局撞骗的。

小帖(tiē'r)子生意亦是流动性,临时集合,打走马穴(xué)的生意。



我很佩服他们“把(bǎ)点”(管能瞧出人是干什么的叫把点)的能为,能够一眼瞧出我是个不能生财的人来,有我在屋内碍眼,又碍事,把我先请出,他们好生别人的财。我听了那句行话,别惹人家不愿意,没等他们听差的说话,我就告辞而去。他们用什么法子骗屋里人钱是无法知道了。

我看望朋友去吧,在朋友家住了一宿。次日,我从朋友家出来,走到那家栈房门前,见有好几个人和店里伙计争吵,招惹得过往行人围了个风雨不透,我也挤在人群之内要瞧瞧是什么事。见人群里有人挑眉立目地大吵大嚷。他说:“好啊!十几块钱,冤了去啦!今天搬了家那不行,你们开店的和他们伙同骗财,咱们打官司!”我听他们这么一说,就知这人是昨天被那撒(sǎ)小帖(tiē’r)的生意骗了,今天醒悟过来,到这里找后账要往回退钱的。我当时犯了爱管闲事的瘾啦,我向这人劝解了几句,告诉他这事与人家店里无干,开店的是有房子谁爱住谁住,给房钱便是好客人。至于客人干什么人家开店的管不着,就是把店拆了,也找不到那舍药的人了。这人被我劝得无法,自认倒霉。我把他让到茶馆之内,我二人喝着茶,我问他怎么被骗的。他说:“那个舍药治病的人,他叫人在店门口撒帖儿(发号),说白舍药治病。我贪便宜进去叫他们治病。随着我进去了十几个人,他都白舍药打发走啦,就剩下我一个人,他用手给我诊了诊,他说我的病有好几年啦,得的是寒腿。我也没告诉他,他就诊出我的病,我很佩服他的能为,求他给我治治。他说,有个妙方,一治就好。我求他开那药方,他就用笔开了个药方,写的是:麻黄、川芎、木瓜、牛膝、杜仲、年健、入地风、洋红花、串地锦、麝香等等的药品。他把那药方写完了,他问我,你知这串地锦是什么药吗?我说,不知道。他说,串地锦是一宗最宝贵的药品,出在西藏,长有三四寸,是个小虫儿,往地里乱钻,要是配在群药之内,凭他那药的力量,能舒筋活血,追风散寒,像你这寒腿吃下就好。这群药倒不贵,唯有那串地锦一味药,买得五十几元钱,还没准儿买不着真的。我说,只要能把病治好喽,几十块钱算得了什么。他说,你们亲戚朋友有在药行做事的没有?我说,没有。他透着为难样子说,就怕你花钱很多,买不到真正串地锦。我也觉得不懂行怕买不着真东西。那听差的在旁说,咱们给张镇守使配的那药不是有串地锦,也是治寒腿的药嘛,何妨匀给他呀?那位先生把眼一瞪,申斥那听差的不该多说话。我就央求那位先生,你有那宗好药,何不行好积德匀给我,该多少钱我给多少钱。那位先生情不可却了,他说,我把药匀给你,你有五十多块钱吗?我说,我有十几块钱给你留下,我回家再取那三十几元去。他叫我把钱取出来,一共十四元八毛整,他把钱收下了,把那药交给我,告诉我怎么个使法。我还很感激他们真瞧我至重,还差三十多元,就敢把药给了我。我还说明天一定给他们送钱去。我拿着药高高兴兴回了家,以为该着除灾了,及至向街坊邻居学说此事,有人说我上当了,药材里向来没有串地锦。我被人说得有些觉悟,今天来店内找他们问问,哪想店里伙计告诉我,那舍药治病的先生昨天晚上就走了。”我听明白了他受骗的情形,才知道个中的把戏。我把那人劝解回去,我也就给了茶钱,走出了茶馆,回归奇山所啦。

后来在山东盂兰会遇见了个姓王的朋友,因为他是个江湖人,和我很不错,我将那撒小帖(tiē’r)的情形向他说了一遍,问他是怎么档子生意。据他说:“做那种生意的行当总名叫做小帖子。在屋里装治病的先生叫做掌穴(xué,这一伙人的头儿)的,那装听差的人叫敲家子,那店外撒传单的人叫做撒幅(sǎ fú)子的,在店里指路的人叫做把二门子的。他们这种生意没有五六个人做不了啊!到处做生意,找个地方叫安窑儿,安下窑儿,做下钱就走,免得被欺骗的人觉悟了,找他麻烦。小帖(tiē’r)子生意也是流动性,临时集合,打走马穴(走一处,不能长占,总是换地方挣钱,江湖人叫走马穴)的生意。”

到了如今,我国各省县市地方当局立有医药的机关,行医得经官家考取及格,发给行医的证书才能行医。这个没证书不能行医可把生意人治住了。骗人生意受此限制,也渐渐地无形消灭了。





挑(tiǎo)土海宝的生意


有一年我在营口去逛洼坑甸,那个地方最热闹的是杂技场,各样的玩艺儿都有,和天津的三不管(天津市南市的一个露天市场)、安东的七道沟、北平的天桥一样。我走在一个场上,见有一人,有三十多岁,穿着打扮像个种庄稼的人。他在地上铺了一块白布,从腰里掏出几十张小四方韵纸来,往地下一蹲,他嚷道:“快来看咱们的宝贝!快来看咱们的宝贝!”我随着一些个人们观瞧他有什么宝贝,就见他从腰里取出一个绸子包儿,内里凸着,包的是什么虽然看不见,那个儿大小和那三炮台的烟筒儿差不了多少。他指着那个包儿说:“这个宝贝,是我在海边上捡的,大有用处,我打开大众瞧瞧。”说着话他打开了一看,是一块又圆又高的石头,那石头上自然长着十几个小蛤蟆。大众瞧着这个东西很奇怪,也不知道是个什么东西。这时候就听他说:“我捡了这个东西,也不知道是什么,经过多少人瞧,才知道这宗东西有什么用处。它专能治病,可不是什么病都能治,就是能治眼睛上的毛病。不论是气蒙眼、火蒙眼、暴发火眼、见风流泪、胬(nǔ)肉盘睛、红丝血线,一上就好。我可不是卖药的,也不是行医的大夫,我把这宗东西送给众位一点,行个方便,结个人缘儿。”他说到这里抬起头来,瞧大众有害眼的人没有。是系在春天,害眼的人很多,他指着一个人道:“这位的眼睛净是红丝,我给上点试试。我可不要钱,我也不卖。”那人就蹲在了地上,他从腰里掏出一个小玻璃瓶儿,瓶内有水,又掏出一把小刀,一个骨头簪。他用那小刀在那海宝贝上,现往下削末儿,骨头簪蘸凉水,又蘸那削下来的末儿,往这人左边的眼睛上点,工夫不大这人就说:“劳你驾,再把这右眼上点儿吧!”他又给往右眼上抹点儿。这人直嚷:“舒服多了!”这些人都瞧着便宜,工夫不大,蹲下好几个人,这个眼睛上点儿,那个眼睛抹点儿。也真奇怪,凡上了他药的人都说:“这药很好。”看热闹人中有一位向他说:“先生!你这海里的宝贝能治眼睛上气火蒙吗?”他说:“能治,当时就好。”这人说:“我去把病人搀来,你给治治吧!”说完了话,这人就走啦。去了不大工夫,就搀了一个病人来,和瞎子一样,叫这人蹲下,求他给上点眼药试试。这人就把眼药给他点上,这人闭上眼不动了。那些先上药的人们全都把眼睛睁开,个个都觉得好受似的,全都瞧他给那人治气火蒙。足有一顿饭的工夫,他说:“你们众位瞧吧,这人的眼睛好啦!”他又从腰中掏出一把小镊子,用手指头将那人的眼皮拨开,用骨头簪儿往下拨,那病人眼睛上的那层蒙,就渐渐地活动。等到用镊子往下一夹,就取出来,他举那块蒙,叫大众观瞧,足有人的手指甲盖儿大小,厚薄也有手指甲那么薄厚。站着的人,蹲着的人,大众见他这药当时就能把眼睛治好,都夸好药。害眼的人说:“哎呀,我可看见什么了。这些年把我闷坏了,净药钱我就花得没了数啦,任什么也干不了,少挣多花,一千块钱没有啦。”说到这里,向他问道:“先生,你给我治好了一只眼啦,我给多少钱呢?”他说:“我不要钱,我不是治病先生,白瞧白看!”那人说:“我才好了一只眼,这只眼还没好哪。我再叫你白治,我良心有愧,千数多块钱都没治好,你给治好喽,我也不能叫你白治,你卖给我点药。”那人说:“我不是卖药的,我这东西不卖,谁要买也成,一千块钱,谁要谁拿去。”这病人说:“一千块大洋,买不起,你匀给我点吧!”他摇头道:“不匀不匀!”当时围着的人都央求他,好容易他才点头,匀给那人两块钱的。两块钱才买了一小包儿。他匀给这个人了,可就不成了。这个说:“你匀给他啦,也得匀给我点儿。”于是这个三毛,那个五毛,这个一块,那个半块。不到一个钟头,就匀出五六元大洋的货去。他愈说不卖,愈有人买。以后不卖了,收拾收拾就走啦。我也买了他三毛钱的。回到家中,有亲友们害眼的,给谁上点儿,谁的眼睛就好。可是有个五六年的病人,眼睛起了蒙啦,上了点药就不管事,那蒙也不下来。我很纳闷儿,他给人治下蒙来,我亲眼得见,怎么买到家内就治不下蒙来呀?要说上了当啦,害眼的人真治好了几个,这事真叫人纳闷儿。

有一年在大连的西岗子露天市场,又瞧见一个用海宝贝治眼的,也是说不是治病的医生,不是卖药的,谁要买得花钱匀他的。后来,我有位久闯江湖的朋友,我和他打听这海宝贝的事,和他探讨。据他说:这种生意,也是“挑(tiǎo)招汉儿”(卖眼药的)的买卖。他那海里的宝贝是瞎话,那宗东西是他自己“攥弄”(zuàn nong)(自己做的调[diào]侃儿叫自己攥弄)的,做那个东西“笨头儿”(本钱)得十几元钱哪。那东西的“底啃(kèn)”(物质的原料调侃儿叫底啃)是芦甘石、冰片两味药材做的。据《雷公炮炙论》、《药性赋》与各种医书所言,芦甘石、冰片,乃治眼之圣药也,他们海宝上的药末子,当眼药上是很好的,只是做买卖摆在地上要卖钱实是不易,货到街头死,肉贱鼻子闻。不论是什么买卖,一落到土地上就算完了。可是江湖人想出来这个方法,摆到地上就能卖钱。他给那害眼的人当时上好了药,还能把眼里的蒙治下来,那也是一道“样色”(yàng shǎi),和变戏法一样。他们用鸡眼先做成了那块假蒙皮,到了卖药的时候,有他们的两个“敲托”(暗中帮助做生意的人,也可称为贴靴的)的,一个装害眼的闭着眼睛,一个搀着假害眼病的人,到了他那摊上,假装不认识,叫他给治眼病,他将那鸡眼做的假蒙皮藏在了手内,在给他敲托的上眼药的时候,暗中就放在眼内。不多一时,再由眼内取出来,叫别人瞧着他那药真有效验,江湖人管他们使用的这个方法调侃儿叫做“糊(hū)的样色(shǎi)”。他们使这道样色是卖钱的惟一不二之法。还有那假装在土里掘出来的宝贝,是用冰片、芦甘石做的,就是没有那小蛤蟆,另做上几条龙儿。做土宝的生意也和做海宝儿一样,只在那宝贝的样式不同罢了。现在做这土宝、海宝生意的不能在各省市、各都城里售卖,都往乡村里赶集赶庙去了。这种生意也是日渐稀少,将来再过些年就怕无形消灭了。





挑(tiǎo)汉册(chǎi)子的生意


在民国二三年间,敝人曾在天津东马路偶步闲游。见有一人,长得很清秀,约有三十多岁,他不支棚设帐,也不摆设浮摊,用块大白在地上写字,写的是“万事不求人”。我看着很是奇怪,不知他是干什么的,站在他那里要看个水落石出。只见有十几个人围着观瞧。这个人写完了那几个字,直起腰来向观众说道:“我写的这万事不求人,可不是书铺里卖的那本《万事不求人》。我觉得天下的事天下人办,各人有各人的长处,各人有各人的短处,一个人的知识原有限,天下事理本无穷。任你有多大的知识,一个人也不能事事都知道,事事都懂得。当初行医的大夫最有名望的有个叶天士,他有起死回生之能,上至朝中文武,下至庶民,都知道他叶天士。有年夏天六月中旬,天气暑热,叶天士正在屋内坐着,忽听院内有小孩啼哭之声。他到了院中一看,是他家小孩哭喊不止。他问孩子,为什么哭啊?有个小孩说,是狗蝇钻在他鼻孔之内,疼得他哭起来。叶天士听说是狗蝇钻在小孩鼻孔里,他虽有起死回生之能,一时之间竟无主张,干着急想不出治法来。他熟读古今医书,什么奇怪的病症都有治法,惟有这狗蝇钻在鼻孔内他就没有办法。叫人用镊子往外夹,夹也没夹出来,狗蝇直往里钻,急得他顺脑袋往下流汗。正然着急,忽听门外,哗啷啷……有摇串铃的声音(在早年有些个串巷卖药治病的,都是提着药包,摇着铁串铃兜揽生意,俗称卖野药的),叶天士是个名医,他哪瞧得起卖野药的。他叫家人将卖野药的先生请进来,叫他治治这临时的急病。家人到了外边将卖野药的叫进来,卖药先生向叶天士问道:‘你有什么病呢?’叶天士说:‘我倒没病。我问问你吧,若是狗蝇钻进小孩鼻孔内,你有法子治吗?’卖药先生说:‘有法子治。’叶天士说:‘怎么治呢?’卖药的说:‘用热狗毛一撮儿,塞在鼻孔之内,那狗毛见了热气一犯味,狗蝇就钻进狗毛之内,然后将狗毛一揪,狗蝇也就随着而出。’叶天士认为有理,命家人如法而治,家人就揪下一撮狗毛,塞在小孩鼻孔之内。工夫不大,将狗毛拔出来一看,果然狗蝇随着而出。叶天士惊喜非常,他给了卖药的不少钱,卖药的去了。叶天士说:‘我从此不敢轻视人了。一个人知识原有限,天下事理本无穷。’”

施舍偏方的生意,江湖人称“挑(tiǎo)汉册(chǎi)子”的。



他说到这里又向观众说:“众位先生,‘偏方能治大病,草药气死名医’,那话是不假的。当初我老人家在前清太医院当差,有遗留下的妙方,专治各种奇怪病症。如若小孩被开水烫了,或有牙疼的,或有长黄水疮的,或是有耳内生脓的,或是有暴发火眼的,或是有蝎子蜇着的,马蜂蜇着的,蚰蜒钻进耳内或是蜈蚣咬着的,都能一治就好。这些个病虽不要紧,当时可没法子治的。当初我家里配过这些药,家里施舍,分文不取,毫厘不要。如今,家中的事由不好,施舍不起了,我将这六十几样绝方印了一千本,这叫:半积阴功半济财。舍药舍不起,舍偏方也舍不起,哪位愿意要一本,拿到家中,行个方便,结个人缘。我也不赚钱,我花多少钱印的,你花多少钱买。”说着话,他从怀中取出个布包,内里包着有几十本儿。那本样式如同唱本大小,上边印着那几个字:“万事不求人。”他说:“我这本儿是一毛钱一本。今天我为传名,不要一毛钱,咱二十枚一本。都要我可不卖,就卖十本,过了十本之外,我还是卖一毛钱。哪位要哪位伸手,接着也不用喜欢,接不着也不用恼。”有好些人都抢着买,二十个大铜子买六十几个绝方,本来不贵,谁都愿意要。我亦给他二十枚,买了一本拿回家去。吃完了饭,闲着没事,打开了那本《万事不求人》慢慢地观瞧。只见那本上印的是:“小儿夜啼:用鸡粪涂儿脐中,男用雌鸡粪,女用雄鸡粪,便能止儿夜啼。”“疯犬咬伤:用真纹党二钱、羌活三钱、独活三钱、柴胡三钱、枳壳二钱、桔梗二钱、茯苓二钱、甘草三钱、川芎二钱、生地榆一两、生姜三钱、柴竹根一大把,凡疯狗咬者,遇风畏缩。欲知是否疯狗咬伤,先以蒲扇向病人扇之,如病人畏惧,即是中毒,即用此方浓煎大剂服之,如牙关紧闭者,敲去门牙灌之。如欲试服药毒气尽否?七日后用嘴嚼生黄豆试之,如嚼后欲呕者,是毒已尽,否则毒气未尽,仍须再服一剂,可保无虞。”“治癣痛流血:用龙眼肉核,剥尽光皮,将核研细末,糁于疮口,即可定痛止血,忌食粥,少饮水。”“治箭镞及针刺入肉不出方:用蝼蛄脑子捣烂如泥涂患处,换三五次即出,或用磁石亦可,即吸铁石也。”“救吞鸦片烟法:用硼砂一两、葛花三钱、青黛三钱,共为细末,以鸡蛋清调服,即吐毒水,毒重者连灌三四次。能将毒物吐尽,乃奇方也。”“接骨丹方:用独活二钱、川乌三钱、草乌二钱,共研细末,用白糖蒸极融化,另用杉木炭为细末和药蒸匀摊纸上,乘热贴患处。无论骨破指断,数日可愈。忌食生冷。”“治虫入耳方:用猫尿灌之即出。”“治脚气方:用荸荠煎汤洗之,可愈。”“治黄水疮方:用蜂窝、白矾焚化。香油调擦即愈。”这几个偏方是敝人试验有效的,披露出来,诸君用之,积德行好。至于未经试验与无效者数十种,恕不披露。

敝人曾以卖印偏方本的行当,向江湖人讨论是否生意?江湖人说:“这行儿调(diào)侃儿叫‘挑(tiǎo)汉册(chǎi)’的,亦以圆粘(nián)子(招徕观众),说‘包口(说完一段故事,再售其货,调侃叫包口)’挣钱。”敝人问,何以所售之偏方、秘本能有效验?江湖人云:“腥加尖(假的加真的),赛神仙。”唉,欲使人相信自己,亦用腥加尖的手段,社会里的事亦真是如此啊!

他们这种骗局说行话叫做大粒的。做这种生意很难,没有五六个人做不了。无论是掌穴(xué,这一伙人的头儿)的,敲托(暗中帮助做生意的人,也可称为贴靴的)的,个个都囊中巨款,不待被骗之人明白了,他们就坐上火车、轮船逃往别处了。





江湖中之大粒生意


在前年冬天,约在十月底,我云游客有事赴津,寓于西马路某客栈中,偶至北开闲游,见周公祠西有一道人,摆设卦摊。他长得又黑又胖,约有四十多岁,头戴九梁道巾,上面嵌一块美玉,身穿蓝布道袍,圆领阔袖,腰系水火丝绦,白袜云履。摊上只有一个六爻卦盒,摆着六十四个制钱。他见游人渐多,往盒里装了八个制钱,摇起来哗啷啷直响。他自言自语地嚷道:“天灵灵、地灵灵,南方丙丁火,请来老君帮我……”嚷个不休。招惹得逛北开的人们都围着观瞧,和瞧怪物一样。我也不知道他是干吗的,挤在人群中要看其所以。

他正然喊嚷,忽见由外面挤进两个人来,是一男一女,男的约五十多岁,戴着青缎子棉帽头,穿着灰布袍,青布棉马褂子,穿两只全胜棉鞋,看他那样子好像在家纳福老人班的人物;那个女的约有四十多岁,品貌端庄,衣服整齐,是个良家妇女的样子。那个妇人冲着老道说:“道爷!我求你给占算一卦,要多少钱哪?”老道说:“二十枚。”那妇人说:“你给算一卦吧。”老道便将卦盒摇起来,摇了会儿,将盒盖打开,八个制钱往桌上一洒。他看着这八个制钱酌量了会儿,向妇人道:“你是姓李吗?”妇人说:“姓李。”他又说:“你这卦不是给自己算,是给别人算的,对不对呀?”这姓李的妇人说:“是给我们邻居算的。”老道说:“这卦是给姓赵的算的?”妇人说:“不错。”老道说:“这姓赵的是个老太太,她现在有病啊!”妇人说:“不错,她现在有病。”老道说:“她得的这病是气蒙眼,在前两个月还任什么都看不见,一个月内,两只眼好了一只,那只左眼已然看见东西了。是与不是?”妇人道:“不错,是这么回事。”老道说:“她们求你给她占算占算,还买点眼药,再治她那只右眼,是不是呀?”妇人说:“是这么回事。先前治那左眼的时候是花两块大洋买道爷的眼药。”说着从手巾包内取出两块现洋来,说:“道爷,你再卖给她们两块钱的眼药,叫她那只右眼也治好了吧!”老道说:“你不知道,我头次下山来到天津,在八月后半月,她们来算了一卦,我算出这卦是个姓赵的老太太害眼,因气所得,长了气火云蒙,任什么也看不见了。我有两种妙药,一种是吃药,一种是上药,应花四元钱药费。他怕花了四元钱再好不了,买了两元钱的药,我告诉她们买一半药就好一只眼,再买那一半药我可不卖。她们点了头就走了。现在我二次下了丫髻山来到天津,她们姓赵的不好意思来了,叫你替她们占卦治病,花钱买药。我是不卖了,叫她不用好那只右眼了。”这时候,围着看热闹的人们,都听着那老道有这么灵的卦,有这么好的妙药,人人都两眼发直,目不转睛地看着他,都竖着耳朵,鸦雀无声地听他讲话。就是我云游客也听着入了神啦。那个妇人无法,包起两元钱,留下二十枚卦礼而去。跟着又有些人算他的卦,如若是围着看热闹的人们占卦,那老道就说没有卦,不能占算。若是由人群外边挤进来的人叫他占卦,他就给占算,不惟有卦,算得还是真灵。

一般人们都看到他的神通广大,惊服不已。惟有我云游客游的地方太多了,千奇百怪的事也看过多了,绝不相信那个老道有那么大的能耐。我要看他究竟如何,便立住不走。我看到他算了八课,那老道就说:“众位不要算了,我要回店了。如若有愿意占算什么求财问喜、谋事成空、疾病死亡、何年立子、克妻不克、寿命长短,都可以往栈房里找我,我是丫髻山的道人,不为发财,是为了重修庙宇来结善缘。”他说到这里从道袍内取来了百数多张传单,散给众人。我为了探讨社会中的黑幕材料,便拼着命似的也接了一张传单。那老道说完了话,散完了传单,收拾卦摊回归店内去了。他走后,围着看热闹的人们还是议论纷纷,都说老道是个高人,神通广大,来历不俗。我因到了吃饭时间,也雇辆洋车回归旅社。到店内吃完了晚饭,喝着茶,想起在北开所见的那个老道来,我要看看他那传单,就从身上掏出传单来,在电灯下看。只见那传单上印的是“请看报恩传单”六个大字,那几百个小字印的是:“敬启者,诸君台鉴:敝人李有仁,年五十九岁,在西沽得人里居住,开洋行为生,膝下无儿,只有一女,现年二十一岁。前在女子大学读书,劳心太过,得了干血痨症,四肢发烧,腹内淤(yū)血成块,咳嗽无痰,六七个月内不见经血,请名医若干不见功效,自想等死而已。幸遇友人言说,英租界顺兴公寓居住一位道人,占卦治病,有起死回生之能,如占卦决断吉凶顺逆,便入手医治,服药即愈,否则绝不入手。敝人闻之,亲往英租界顺兴公寓求该道人占算一课,卦上断出我女儿之病为干血痨症,分毫不差,卦断上卦,寓缘有治,服药两料即能痊愈,每料药资三元九角。当时交洋,将药一料取回,服后大见功效。又急拿洋三元九角,将第二料药服完,病症痊愈。道人之药真乃神效至极也。果中所言,我女儿数载之苦处,今一旦消除。介绍亲家十二条居住邓光德之妻,产后恶(è)露不止数月之久,医生言乃崩症,百般调治无效,今求道人配药一料,药费六元四角,将药服完,病即痊愈。又介绍李国才,居仁里住家,先在江南经商,受潮湿身得瘫痪之症,动转难移,一年有余,立求道人治好。余又介绍病症颇多,有腰腿疼的,有咳嗽出血的,有梦遗滑精的,有不种儿的,有心疼腹痛的,有染花柳的,有长疔毒恶(è)疮的,有害眼疾的,这些病人俱经道人妙手治愈,各界人人赞成。我李有仁之女儿,不遇道人,一命休矣。诸君请想:财宝如粪土,一命值千金。我数家深感大恩,商议共送谢礼,道人不收。我等无恩可报,印送报恩传单一万张,一为了结心愿,一为道人提倡名誉。我李有仁如说谎言,叫我数家死无葬身之地。各界男女老幼如有各种之病症者,急往该处求卦诊治,免受长久痛苦也。如占卦者,先交卦金两角。不看转送别人,功德无量也。道人现寓英租界顺兴公寓一号。李有仁、邓光德、李国才同启。”我看他这张传单,文理说不上,话语也不通顺。但是,我云游客虽无病,欲要探讨其中黑幕,只好学那出《剑峰山》的邱成,身无病假装有恙,到趟英租界顺兴公寓访访这位道人。

当日夜内睡了觉。次日早晨起来,吃完了早点,带上十数元钱,乘坐电车前往,不到半个钟头,已达顺兴公寓。到了门内,我向该公寓的茶房问道:“茶房,你们这公寓里住着一位能占卦治病的道人吗?”茶房说:“有一位。”他说着话冲我一招手说:“你随我来。”跟着他走到一个跨院之内,他用手一指那间北房道:“就在这屋内。”我进到屋中一看,这屋内并没有那个道人,只有一个二十多岁的男子,也和茶房似的,他见我进去,向我问道:“你是来算卦的吗?”我说:“不错。”他说:“你先在这屋里等着,先生那屋内正给××洋行的内老板治病哪。”他给我斟过一碗茶来。工夫不大,又来了俩女人,一个三十多岁,一个五十多岁,又来了四五个男子,都是来占卦治病的,大家坐在屋里等道人给占卦治病。在等的时候,大家彼此谈话。这个人问那个人:“贵姓?是占卦治病吗?”那个人说:“姓王,我母亲害眼,生了云蒙……求道爷来算。我娘的病……始终也没治好。”那个就说:“我是自己得了个吐血的病,花了二百多元也没治好。”他们谈谈论论,我是一语不发。有个老头儿问我:“你贵姓哪?”我说:“姓云。”他说:“你在哪里做事呀?”我说:“探访局。”他说:“你是给自己占卦治病啊,还是给人家占卦治病哪?”我将要和他说实话,突然想起他们江湖的生意门都有一种“敲托的”(社会里面半开眼的人,管敲托的叫贴靴的,他们是装好人闲聊天,在无形之中将人的事先探明白,然后再告诉那老道去。江湖人管这种探事的人调[diào]侃儿叫敲托的)。我说:“自己有病。”他又问我:“你是什么病啊?”我说:“是饿病。”老头子听我话不投机,他赌气子躲开我和别人说话去了。我等了足有一个钟头,就见伺候那倒茶的人向我说:“请你到南屋占卦吧!”我说:“不忙哪,先给别人算吧。”他说:“有先来后到,你是先来的,请你算吧。”我就同他出了北屋,走到南屋。到了屋中一看,果然是那个老道在屋中坐着哪,靠南墙有个玻璃架,上边摆着许多药瓶子、药罐子,当中放张八仙桌子,桌上摆着个六爻卦盒,还有六十四个铜钱。八仙桌两旁有四个凳子。那老道见我进来,用手一指旁边的凳子说:“请坐。”我落了座,他将铜钱放进盒内八个,拿起来摇了几摇,摇完了,八个铜钱往桌上一倒,说:“你这卦,占的不上卦,改日再来占吧!”我说:“先生,你这是什么卦?我不上卦,是根据什么理由哪?”他说:“我这卦是太极先天卦,系太上老君所留,这种卦没有书,是口传心授的,若将八个铜钱摆得不像卦,就是来人心里不诚,占也是不灵的。”我听他这片话实无有办法,只好作罢。从皮靴掖内取出两角钱票给了他,他不要,说:“不上卦,不收卦礼。”我装起钱票往外就走,到他们那招待室内再坐会儿。那听差的两只眼直瞧我,我装作不知,且看他们的下回分解。只见他们如过关似的,一位一位地让过去占卦。我又竖着耳朵听他屋里摇动卦盒,又隔着玻璃往外张望,见由老道那屋出来的人,都是手拿药包,位位都欢天喜地地往外走。我追出去一问,没有一位不上卦的,都算出是给什么人占的卦,得的是什么病,都是花钱买了药去,十元、八元、三两元。内中有个太太,花了八十元买了一料药去。我替老道预算,哪天也有数百元的收入。

我正在旅馆门前发呆,有人拍我肩头一下说:“老云,你在这里干什么?”我回头一看,是我的同学李辅星。我说:“没有事。”他说:“我就在这公寓里住着,你既没事,里边坐会儿。”我便跟他走进公寓。恰巧他住的屋子与老道屋子挨着,我进屋里一看,这屋和那屋仅隔一层木板。我向李辅星悄悄地将来意说明,不叫他说话,我要用耳隔墙听听那屋说些什么。只听那屋内说道:“今天的买卖很好,就是那头一个点头(即是指我老云说哪)不是个正点(说我是个扎手的人),是个朗(lǎng)不正(说我是个蘑菇,挺嘎的人,讨人嫌)。我说他不上卦,将他推出去了。还有点头(花钱相面的人)没有了?”我听他调(diào)起江湖的侃儿,心里就明白他们是江湖中一种骗局,正是我老云要找的材料,我得探讨探讨。又听那屋说:“既是还有个点头(花钱相面的人),将他让过来,做完了咱们再均杵(即是敲诈完了这个人,大家再分钱)。”我见木板有个缝儿,我往那屋偷着一看,见和老道说话的人五十多岁,正是在招待室向我说过话的那个老头子。就听老道问他:“那个点儿,你要出簧头(实话)没有?”那老头说:“我问他来着,是给他们孙食码子求汉儿(那妇人是给她丈夫求药),他的孙食码子要念招儿(她的丈夫害眼哪,闹得要瞎),是个火码子,你得海(hāi)拕瓦(是个火码子,是有钱的人;海拕瓦,得大敲)。”老道点了点头。那老头出去。工夫不大,就见由外边进来了一位四十多岁的妇人,往凳上一坐。老道摇了一卦,向她说道:“你是给你丈夫占的卦吧?”妇人说:“正是。”老道说:“他得的火蒙眼,有六个月了,对不对?”妇人道:“正对。”老道说:“他这病我倒能治,须吃两料药才能好哪。”妇人问道:“这两料药得多少钱哪?”老道说:“这种药太贵,连吃药和上药,得一百元。你可以先付五十元买一料,先吃七天,吃上七天见好啦,你再拿五十元来取那一料。”妇人就由身上取出五十元钞票给了老道。老道给拿了药,告诉她怎么吃,怎么上。那妇人拿着药走了。

我至此方才明白,他们是在外边用种种宣传之法,将受骗的人诱到公寓之内,先在招待室内坐会儿,有他们的“敲托”(暗中帮助做生意的人,也可称为贴靴的)的(敲托的是老道的伙计)假装也来占卦,他们是先和来占卦的人说闲话儿,将来的那个人是为什么事占卦都套出来,说行话叫“敲托的向点头儿要簧”(要出实话来),然后告诉老道,老道知道了才给占算。阅者诸君想,老道还算不出来吗?那个妇人走后,我老云又听他们在屋内说话,吵吵嚷嚷的。我往那屋再看,见有四五个人和老道分钱哪。分完了钱,老道说:“咱们走吧,到库果窑里啃(kèn)个牙淋(yá lin)吧!”我老云懂得这两句侃儿,往库果窑里啃个牙淋即往娼窑打个茶围。我听了这话才觉悟过来,那个老道是“里腥治巴(lǐ xing zhì bǎ)”(即是假老道)。少时间老道带着他的伙计们出离公寓逛窑子走啦,我才问李辅星:“你一个人住在这公寓里有什么事吗?”李辅星说:“我在屋里住着为的是吃他们的摽(biào)杵”(他们是指老道们而言,吃摽杵是分老道们的钱)。我问李君道:“他们这种生意是怎么回事?求你指教明白。”李君说:“他们这种骗局说行话叫做大粒的。做这种生意很难,没有五六个人做不了。那个老道是掌穴(xué,这一伙人的头儿)的,他们挣钱多寡,全仗掌穴的一人。譬如掌穴的能力好,他能瓦点(即是他能敲诈),大家也能多均杵(即是他的伙计也能多分钱)。如若掌穴的不能瓦点(即是不善于敲诈),他的伙计也分不了多少杵儿(即是他的伙计们也分不着油水)。他们做大粒的掌穴之人都愿意用好敲托的(即是用最好能力的贴靴),能给他往窑里跨火点儿(即是能带来有钱的阔人),到了开瓦的时候,也能海(hāi)瓦(管要敲诈人的钱财调[diào]侃儿叫开瓦,管能多多敲诈人的钱财调侃儿叫海瓦)。所以做大粒掌穴的每逢成班的时候,都是拉拢有本领的敲托的。可是敲托的未曾要和哪个掌穴联穴(xué,即是搭班的意思),事先都耳目(管打听打听谁怎么样调侃儿叫耳目)掌穴的本领高低。如若掌穴的杵门子清楚(管掌穴的善于敲诈,敲诈技能格外好的调侃儿叫他的杵门子清楚),才和他联穴哪;如若掌穴的杵门子不清楚,他们敲托的给他跨着了阔人,他没有敲诈的本领,那也是闻香不到口啊!和他搭伙也是白受累,谁和他瞎耗精神哪?做大粒生意的掌穴的愈是有本领,再搭得着好伙计,他们上下合手,狼狈为奸,才能大施敲诈,遇见了阔人好足足地敲诈他的银钱。他们无论到了哪个商埠码头,也是多来财,吃好的,穿好的,能够解决种种欲望。这里的情形真是叫人说之不尽哪。做大粒的掌穴之人若是没有本领,也搭不着好敲托,无论走到哪个码头,也是干瞧火码子(有钱的阔人)杵头海(hāi)(银钱多),瓦不下来(不受敲诈,钱财挣不到手),挣不着钱,不用说吃喝嫖赌抽,穿绸裹缎,就是吃饭住店的时候,因为没钱,也常受店主的挤兑,他们还不如秦琼哪,连匹马也没有啊!江湖人的经济状况,也是颇有研究的意味呀。”

我听李君说到这里,向他问道:“他们做大粒的干吗到各市场去摆摊哪?”李君说:“他们做大粒的每逢掌穴的搭着伙计,联好了穴,开到哪里,先找个适宜的旅馆,将窑儿先安好了(即是先赁好房子,布置好了骗局),然后掌穴(xué)(这一伙人的头儿)的得到外头票买卖(即是到游人多的地方去算卦),得催出响儿,才能在窑里瓦点哪(管传出名去,人人都知道那里有位活神仙,轰动社会了,调[diào]侃儿叫催响了。他们将响儿弄成了,才能在店里骗受敲诈的人,好敲诈银钱)。”我问李君道:“我在北开见那个掌穴的老道给人算卦,算得很灵,说什么什么都对。那是怎么回事哪?”李君说:“那叫临时买托儿。”我问李君道:“什么叫临时买托儿?”李君说:“他们掌穴的到了市场里,将卦摊摆好了,他就净等着敲托(暗中帮助做生意的人,也可称为贴靴的)的买点啦。那买点之法很不容易,那敲托的人得会把(bǎ)点(管能瞧出不认识的人是老实人忠厚人,是奸诈人,是狡猾人,是有阅历的人,是没有阅历的人,江湖人管能有这种以貌识人的本领调侃儿叫把点)。他们要不会把点,给掌穴的弄个狡猾人去,那老道甭说催响儿,就是装神仙也装不好,弄糟了就许给他们搅啦。”我问李君:“譬如他们瞧着某人忠厚老实,是个肯受冤的,他们又施用什么手段哪?”李君说:“他们敲托的如若把好了点(即是受冤的人),便向那人迎着面过去,愣给那人作揖,说,大哥你好哪?那人一定冲他发愣,敲托的就说,你不认识我了?我不是姓……那……那人一个猛劲就说出自己的姓氏,他将人家的姓氏蒙出来了。又说,你现如今在哪里住哪?那人必将地址说出,他将这人的住址蒙出来了。敲托的就按着这人说出来的住处,说,我在那里住过,咱们是邻居。那人蒙住了,辨认不清,他才向那点头(花钱相面的人)说,我求你点事,能成否?那人一定问他,你求我什么事哪?他就说,我母亲得了病症,有多年了,两条腿不能动转,据医生们说是下痿。我在月前走在这北开,见有一个老道摆着卦摊,我求他给占算一卦,问他我母亲还能好不能。不料那个老道将卦一算,没等我说是为什么事占卦。他就说,你这卦不是自己算的,是给你母亲算的,你母亲得了下痿(wěi),两条腿不能动转。我听老道的卦占算得真灵,我问他好得了好不了。他说,这病我能治,有两料药准能保好,每料药吃十五天,一个月复旧如初。我问他,那两料药得多少钱?他说,三元一料,两料是六元。那时候怪我不好,惟恐花六元钱买两料药吃不好,就花三元钱买他一料。拿回家去,我母亲吃了半个月,两条腿好了一条,还有一条腿没好。我又拿了三块大洋来买这料药,没想到老道很是奇怪,他说:上次你买我一料药,怕我冤你;这次你再买,我不卖了。我央求他也是不行,我没法子可想,碰上你了,求你去给我假装算一卦,就说给街坊算的,花三块大洋买他一料药。你行点好吧。这人情不可却就能点头,由敲托(暗中帮助做生意的人,也可称为贴靴的)的给这人二十枚卦礼,三块大洋,两个人找老道算卦。可是在这个时候,老道就在卦摊后大嚷大闹,招得过往行人像看怪物一样,他把粘(nián)子圆好了(聚好了观众),敲托的将这人带着挤进人群,敲托的不用嘴说手指,只要冲老道一递眼神,老道就明白了。那人说,道爷你给我算一卦。他摇完了卦就说:你不是自己算的,是替人算的,这个老太太得的是下痿(wěi),两条腿不能转动,她如今好了一条腿,还有一条腿没好,叫你给她来占卦,花三块钱买一料药,是不是呀?那人不明其中黑幕,听着很是对,心中佩服老道有点来历,他先给二十枚卦礼,后掏出三元钱,说:道爷,你算对了。我给你三元钱,你再给我们一料吧。老道说:不成!上次他不相信,买我一料药,我叫他好一条腿,这条右腿不用治了,说什么我也不卖这料药了。那人央求着,那不是白费话吗?他见老道不卖也没办法,挤出人群向敲托的人说:给你这三块钱吧,这个老道真灵。敲托的冲这人作揖道谢,他假装为难发愁的样子,叫人看着好像真事。那个人回到家里,见了他的朋友街坊能不说吗?要知社会里的风气,是专好谈奇说怪,迷信太深。他向亲友邻里一传说,只要有个有病的,他们就得上当。再往卦摊占卦,有敲托的在卦摊附近围着转悠,两只丁郎似的眼睛净望着点儿(人),瞧见他买的那托跟着人来,就知道他们宣传的力量有效,这人给他们介绍买卖来了。敲托的赶紧凑过去,假装说话探口气,将来人的事探出来,并随着这人到卦摊旁边一站,和老道一使暗令子,老道就明白来人是为何事占卦,施其引诱的手段,诓到旅馆客栈之内,焉能不受其敲诈?”

我老云听明白了这买托、过簧(彼此说了行话)、敲托、催响(要钱)的事儿,又问李君说:“怎么有人在他卦摊上算卦,他说不上卦哪?”李君说:“他遇见没叫敲托要出簧(实话)来的人,人家的事他全不知道,算也不灵,说什么也不对,倒坏了他们的生意,故而一推了之。”我问李君:“什么叫推呀?”李君说:“他们做大粒生意的掌穴(xué)(这一伙人的头儿)的能为,得会推,会送,推送清楚,那做生意敲诈人的本领才算到家哪。”我问李君:“怎么为推?”李君说:“他们江湖人管有买卖不做调(diào)侃儿叫推。”我问李君:“什么叫送呀?”李君说:“来了点头儿(花钱相面的人),只要将钱弄到手内,立刻几句话就将上当人说走了,那调侃儿叫送点。一者钱到手啦,多费些话无用;二者言多语失,多说话没好处,不如钱到手将他送走,再来了人好挣第二个人的钱。送走了点有两样好处:一者来了点再施敲诈的时候,先上当的人是旁观之人,当局者迷,旁观者清。那时如旁观者醒悟了,岂不往回要钱?这是送点的好处。可是社会里有一种屁股最沉的人,到了谁家坐着不走,也不知是哪儿来的话,说起来没完,本家主人心里多烦他也不走。江湖人对于这屁股沉的人,他们有一种方法,几句话就能将他送走。这种送点的意思是免得在他们敲诈的时候有人碍眼。做大粒的江湖人投师受业,练习好了能耐,先挣钱孝敬师傅。学的就是当掌穴的杵门子(到要钱的时候叫杵门子)要清楚,簧头(要实话)要利落,推送要清楚;当敲托的要会把(bǎ)点才成。”我问李君:“为什么到公寓里来占卦他说不上卦哪?”李君说:“他们江湖中的生意专会把点,把着你不是点,才说不上卦。”我问:“什么叫把点?”李君说:“江湖人管瞧事行事,瞧人行事调侃儿叫把点。如若看着某人能受他们敲诈,便说某人是点;如若看着某人透出来不能受他们敲诈,就说不是点;如若看着某人像个忠厚的样子,便说是忠样点;如若看着某人像个当小官差的样子,便说是柴把(bǎ)点;如若看着某人像个做大官的样子,便说是翅子点;如若看着某人像听差、茶房的样子,便说是展点(仆人);如若看着某人像个做买卖商人的样子,便说是贸易点;如若看着某人像个乡下的庄稼汉,便说是科郎(kē lang)点;如若看着某人像个当兵的样子,便说是冷点;如若看着某人像个人物字号的样儿,便说是皇壮(zhuàng)点;如若看着某人像个懂行又不甚了解的样子,便说是个半空不撮点。你来占卦不愿受敲诈,是来探讨他们的内幕,他们焉能看不透啊?你自己说是点(花钱相面的人)不是点?”我说:“不是点。”李君说:“你既不是点,他们就不和你捣麻烦,说不上卦的意思就是看你不是点。”

我问李君:“你为什么住在这个公寓哪?”李君说:“为的是和他们均杵。”我问:“何为均杵?”李君说:“他们敲诈来的银钱,我分着花,调(diào)侃儿叫和他们均杵。”我问:“你凭什么分他们的钱哪?”李君说:“江湖中的生意有能挣钱不犯法的,叫正当生意;有几种生意虽然挣钱,暗施敲诈,他们的钱财是犯法来的。他们做这种骗人的生意,时时刻刻害怕,如若有人将他们告发了,一定得落个诈骗人财的罪名。如若有人能明白他们的内幕,再有几个官面的朋友,有好几个当官差的,我对于他们做大粒的生意人就能施以威胁手段,和他们均杵。”我问李君:“譬如他们若不均给你杵哪,你有什么办法?”李君说:“他们若不分给我银钱,我就向官界的朋友将他们的诈骗行为说明了,使出官面来,轻了将他们驱逐出境,重了捕到官署,搜出诈骗的证据,还能叫他们去住监狱。”我问李君:“你的事我明白了,你是坐地分赃啊!他们恨上了你,你可得留神哪!”李君说:“他们不恨我,还和我真亲热,绝不能陷害我。”我听了很为奇怪,不明白他们江湖人为什么还愿意交他这个朋友。我问李君道:“他们为什么还愿意交你哪?”李君说:“他们有我这个朋友,有三样好处。”我问:“哪三样好处哪?”李君说:“头样好处是用我联络官面,一则不受取缔,二则遇事能够护庇他们;二样好处是用我给他们把(bǎ)点(管能瞧出人是干什么的,能生财不能叫把点),本地各机关的人员他们是不认识的,我若瞧见有各机关的人员来了,就和他们调侃儿,不叫他们敲诈,免得惹了马蜂窝;第三样是被他敲诈的人明白了,来找他们的麻烦,我是本地人,眼皮儿宽,认识的朋友也很多,也能给他们说和事,息事宁人。我不是白要他们钱财,我是他们的护身符。”我说:“他们几时认识你的?”李君说:“他们江湖中的生意人自称叫跑腿的,忽在某省市,忽在某商埠,忽在某码头。他们生意人是这里不见那里见,他们见了面也是打听各地的事儿。他们是甲向乙说,乙向甲问。如若到了天津,只要找着李辅星,有他护庇着做生意就任什么也不怕了。故此外埠的江湖人来到这里就找我。”

我问李君:“凡是他们江湖的生意人挣了钱你就能分吗?”李君说:“不能。是他们骗财的生意挣了钱,我能分肥;若是不骗人的生意人,挣了钱不给我花,我也是没有办法。譬如那卖刀剪的说吧,他们那种生意是讲本图利,不过用生意的方法多卖些货物而已。人家卖了钱我凭什么分着花呀?”我问李君:“他们这做大粒生意的为什么都给算卦人一料药哪?”李君说:“这叫卖料汉儿的。”我问:“什么叫卖料汉儿?卖料汉儿是怎么回事?”李君说:“卖料汉儿,是他们做大粒生意最重要的诀窍。他们是欺骗人的生意,每至一处,设局骗财也不容易。在那里做生意,日期少了,骗不了几个人,所挣之钱财有限,他们用度不足也是不成;日期多了,被骗的人久而自明,如若醒悟了,岂不找他们麻烦?他们每至一处,至少的日期要做半个月的生意,多了要做一个月的生意,在这骗人的时期内,他们卖出的料汉儿是每日叫病人吃一丸子药,吃十三天为服一料药之期,如若服药人吃完了药不见效力,找他们麻烦来,他们在这十几日工夫已然将钱财骗足了,除去吃喝花费挥霍之外,无论是掌穴(xué)(这一伙人的头儿)的,敲托(暗中帮助做生意的人,也可称为贴靴的)的,个个都囊中巨款,不待被骗之人明白了,他们就坐上火车、轮船逃往别处了。用料汉支延十数日是他们搪塞被骗之人的好办法。故此做那种生意都用料汉,个中的意义就是这种情形。”我问李君:“他们有到天津三二日就走的没有?”李君说:“他们这种生意也凭的是运气。如若到了某处不走运,做个十几日的生意也没遇着有钱的阔人,骗了穷人的钱财不用说无有坐火车、轮船的路费,连他们住店吃饭还困难哪。说行话叫浅在某处开不了穴啦,这种情形也是免不了的。如若他们到了那里三二日之内,遇见了阔人,能敲诈个几千元,就不用再敲诈别人十元八元的了,及早开穴,早走为妙。倘若不走,被人家明白了找他们麻烦,挣到手的钱叫人要回去,那不是煮熟了的鸭子飞了吗?他们到哪儿遇见这种事,就来得快走得也快。”

我老云向李君将这种做大粒的内幕情形探讨明白,记录下来,在本书中谈论明白贡献于社会,贡献于阅者,遇见了这种骗人的生意,免得社会人士受骗,这也是我老云忠心博爱社会人士的一点好意。不知阅者诸君以为如何?





汉门(凡是卖药的调侃都叫汉门)的丁香座子


年前因事赴津,同几位友人往游地道,见有一个玩艺儿场,看热闹的人围了个不透风,挤进去一望,见场内有一高案,上铺俄国毯,摆设药瓶十数个,内里装的无非是药水、药面,有西医外科刀剪家具全份,都是电镀的,耀眼锃光,夺人二目。案后站立一人,长得中等身材,白白的面庞,眉目清秀,儒儒雅雅,约有二十多岁,头戴美式毡帽,鼻架金丝眼镜,穿着一身西服,好像由外省新到的镀金博士。就听他向观众说道:“敝人是××省的人,自从十九岁报考美国广博医学院,三年毕业,得有毕业文凭,在美国医院服务三年,今年春天归国,要在我们天津创立个医院。现在正然进行,大约两个月后可以实现。我住在旅馆里无事,要在医院没开幕之先创些名誉。今天来这里是施医舍药。有病的人算来着了,可不是有病就治,我就治痔疮、漏疮。十男九痔,有内痔、外痔,生了管子叫漏疮。生这种病的原因是抽烟喝酒,大肠干燥,今天咱们还不治痔疮,专治生管子的漏疮。我这里有麻药,如能用药针打上麻药管保不疼,用不了一点钟的工夫,就能将管子治出来,这叫白治漏当时就好。哪位先生如有这种病,只管言说。你的病借给我,我将手术白饶,药也白舍,治好了叫看热闹的人们给传个名,将来我们医院开幕的时候,大家给挂红送匾,替我宣传,鼓吹名誉。话我是说完了,哪位有这种病啊?”他说到这里,就见有个人说:“先生,我有漏疮,可有四五年啦,你能治吗?”这位先生说:“能治,你进来吧!”这个人穿着打扮好像是卖力气的人。他到了当中一站,那先生问道:“你贵姓哪?在天津做什么事呢?”这人说:“我姓王,在脚行当伙计。”先生又问:“你这病治过没有哪?”他说:“净钱花了百数多块啦,始终也没治好。”先生说:“我要给你治了,能够给传名吗?”那人说:“我一定给先生传名。”这时,先生叫过两个听差的人来,帮助治病。叫这病人往凳子上一躺,将裤带解开,往下一褪裤腰,露出屁股来,那两个听差的每人搬起病人一条腿,那先生用手指头往病人肛门旁一按道:“是这里不是?”病人说:“是这儿。”先生用药针往那里打了些麻药,然后又往漏管上抹了些药膏。先生他向围着的人大肆演说,什么里痔、外痔、葡萄痔、蜘蛛痔,他说了足有十几分钟的工夫,然后才拿起刀子、钩子,哈下腰去往病人身上施用手术,又有几分钟的工夫,他用钩子钩住,向病人说:“你咳嗽。”病人就咳嗽,他就随往外钩管子,随嚷:“再咳嗽!再咳嗽!”喊嚷不止,嚷得一旁的人们,听他这里直嚷,不知道是干什么的,都跑来观瞧,愈围人愈多,围了个不透风。正在这时候,他将管子治下来了,他举着那漏疮管子,向在场围着瞧的人转了一遭,叫观众瞧看。那管子约有二寸多长,鲜血淋淋,看热闹的人,无不点头咂嘴,称赞不已。这位先生将管子放在一个玻璃盘内,用药水浸好,他给病人擦抹干净,上了些药末,用药棉花堵住创口,叫他站起来,问道:“你觉疼吗?”这病人说:“不疼!”他趴在地上给先生磕头说:“先生,你要多少钱哪?”先生笑道:“分文不取,毫厘不要,你就记住给我传名吧!”说到这里,他叫病人自己看那管子。先生就向观众说:“我今天就在这里施医一次,明天就不来了。众位如有亲戚朋友得了这痔疮漏疮的,你们只管找我,我住在南马路万人旅馆,那里设了个临时诊疗所,有人找我到那里,也和这里一样,我是施医不要钱。”说到这里他一回手从案上拿起好几百张传单来,向观众散放。敝人也接了张传单,那上面印的是:“大西医士,在美国医学院毕业,得医学博士奖章,在欧美医院服务三年,今春归国,欲在津埠创立医院。在未开幕前,临时在天津南马路万人旅馆设诊疗所,施医外科、花柳科,各界人士如有患外科、花柳科病者,速来诊治,管保手到病除。每日诊病时间上午八时至十一时,下午一时至四时,星期日照常诊治,不收号金,不收手术费。暂定两个月为扬名时间,过期为止。但出诊洋五元,路远与手续费临时面议。痔漏科纯系慈善性质。按痔漏疮发源,不外乎五脏六腑湿毒内热、大肠干燥、烟酒滞气、淤(yū)血流注肛门而成,初得时肿疼刺痒,或生小肉疙疽(gē jū),疼痛难忍,日久生管,流脓流水,永不收口,时好时犯。本医研究有年,善治痔疮漏疮,有临时去管灵奇药水,生肌止痛药膏,管保手到病除,为造就名誉起见,施诊一个月。如有患此症者,速来诊治。医学博士李达兴谨启。”当时,他散完了传单,有两个听差的人给他往起收拾,这位李达兴大医士的黄包车,由车夫拉到场外,他向观众一鞠躬,上了洋车,足蹬脚铃,那车夫拉着他飞也似的回归万人旅馆去了。观众都夸奖那位医士是个大功大德之人,个个将传单当成契纸一样收在身上,谈谈论论而去。

敝人归家之后,也为这位慈善医士逢人便道,替他传名。有大马路某银号的司账王君,生有痔疮,经敝人劝导往医此病。到了万人旅馆李达兴临时诊疗所,还是真不收挂号费,到了三层楼七号房内,敝人与王君向他说明来意,当由李医士在病床施以手术,不到一刻钟,将痔疮管子取下,用药膏上好、药布兜完了,李医士向王君说:“本医施诊,不收手术费,纯为施医,但不施药,君之药费为二十四元,请当时交付。”王君与敝人诧异不止。幸王君为人忠厚,好在病已治好,二十四元不足为奇,当付以钞洋二十四元,与敝人回归,王君也未埋怨于我。不料过了数日,王君找我,说他病症未愈,管儿仍在,照样流脓流水,敝人甚为纳闷,当李达兴施用手术时,曾经目睹将管子取下,何以未愈呢?当与王君乘车往万人旅馆找李达兴医士,至该旅馆时,不见李达兴之临时诊疗所招牌,讯及茶房:“李达兴医士尚在否?”茶房说:“由星期三就往上海去了。”至此,始知受骗,怏怏而归。

后有某江湖人与王君交厚,王君向其探问此事。某江湖人说,李达兴的骗局,说行话叫“丁香座子”。做那种生意,必须四五个人,一人掌穴(管当医生的调[diào]侃儿叫掌穴),那几个当作“展点”(管当差的调侃儿叫展点)。掌穴的人必须人物漂亮,衣服阔绰,谈吐文雅,才能压得住点(管势派镇得住人调侃儿叫压点)。他们每至一处,就先在旅馆中租赁房屋,安“丁香座子”(即痔漏科临时诊所),然后再往各市场游人最多的地方去“票丁香”(管临时设场,白治漏疮,调侃儿叫票丁香)。掌穴的也得先练好了“钢口”(即是生意口),圆好了粘(nián)子(聚好了观众),他说上一大套话,让人听明白了,说行话叫“包口”(说完一段故事,再售其货,调侃叫包口),将包口说完了,再给人治病。他并不是真能给病人治下漏疮管子。在未给人治漏疮管子之前,就和变戏法一样,先将假管子(那假管子系羊的五脏中的管子)含在嘴内,施用手术时,先叫病人见点“光子”(管见血调侃儿叫见点光子)。他用纸给病人擦血的时候,暗中将嘴里含的管子藏在纸内,调侃儿叫“过托儿”,将假管子放在病人的流血之处,然后再以假做真,往外取管子,叫众人瞧着他当时治出管子,管这种手彩调侃儿叫“出样色”(yàng shǎi)。他往外弄的时候叫病人咳嗽,那是“升点子”、“诈粘子”(管嚷嚷出声叫升点子,管大嚷大闹多招人来看调侃儿叫诈粘子)。他举着那个假的漏管叫众人看,调(diào)侃儿叫“叫响儿”,然后“撒幅(sǎ fú)子”(管散传单调侃儿叫撒幅子)。他们这种宣传方法叫人都相信了,就在“座子”(即他临时诊疗所)里如同姜太公,净等着愿者上钩儿。世上的事儿真叫奇怪,有了病就应当花钱喝苦水,偏又贪便宜,吃药治病不花钱,到了座子里,任凭他们“连抠带挖”(管敲诈调侃儿叫连抠带挖)。善财难舍,那句话是不假呀。等到他们将钱弄足了,料着受骗的主儿“醒了攒(cuán)儿”(明白过来了),要“出鼓儿”(管要出吵子,调侃儿叫出鼓儿)了就将东西收拾好了,或上火车,或上轮船,开了穴(xué)(开穴即是另往他方),扯活(chě huo)(跑了)了事。做丁香生意的骗完甲地又骗乙地,纯系流动性质,江湖人管他们叫走马穴(走一处,不能长占,总是换地方挣钱,江湖人叫走马穴)玩艺儿,不能靠长地(长地是指固定场所)呀。





江湖艺人马万宝


在东安市场开办的那几年,杂技场内有个又黑又胖的和尚,每天拉场子撂明地,耍对大钹。成天价逛市场的人们围个风雨不透地瞧他耍那飞钹。他每逢练一阵,圆好了粘(nián)子(聚好了观众)就说些粘啃(nián kèn)条子(管讲说各种病原叫人听调侃儿叫粘啃条子)。我那时候太岁还没增着哪(管岁数小调侃儿叫太岁减着哪),不懂得云游四海,就知道常往东安市场兜圈子。我听那和尚说过粘啃条子,说的是:“血脉好似一长江,一处不到一处伤。寒处就成病,血热就成疮。”又说:“真头疼必死,真心疼必亡。世上没有心疼的病,想当初,曹操真头疼而死,姜维真心疼而亡。我们人得的是肚腹疼痛,有九种肚腹疼,哪九种呢?食疼打饱嗝,寒疼着凉重,气疼两肋攻,水疼轱辘辘,虫疼胃酸水,五积疼,六聚疼,七症疼,八瘕(jiǎ)疼。”他说的各种各样粘啃条子,人人爱听。说完了就卖大力丸。据江湖人说,那个和尚姓马,双名万宝,还是个尖局的化把(bǎ)(江湖人管和尚调[diào]侃儿叫化把,假和尚叫里腥[lǐ xing]化把,真和尚叫尖局化把)。他是直隶省人,做那卖大力丸的生意,调侃儿叫挑将(tiǎo jiàng)汉儿。

自入民国以来,马万宝就净做“走马穴”(走一处,不能长占,总是换地方挣钱,江湖人叫走马穴)的生意,什么奉天小河沿、大连西岗子、烟台南市场、营口洼坑甸、哈尔滨江沿、天津三不管(天津市南市的一个露天市场)、保定马号、通州万寿宫、关里的鄚州庙、关外的岳州会、济南趵突泉,都有见过马万宝的。后来,他又改行啦!不挑将汉儿,又拴起腥棚(假的),收了几个徒弟,组织了××技术团,专练三把飞刀,巧耍飞钹,鸳鸯棒,伞球儿,踩铁绳,十几个人塔儿顶,样样出奇。马万宝的技术团在各码头很有个“万儿”(名儿),“火穴(xué)大转”(zhuàn,到处挣了许多钱)了十数年。他的徒弟干技术团的,还有两个人,一叫宝庆,一叫宝利。宝庆是河间府的人,他父亲叫王秉肇,是“光子”(拉洋片的调[diào]侃儿叫光子)里最有“万儿”的人物,不在大金牙以下。那宝利是他拾到的孩子,家乡住处无法考查了。挑将汉儿的徒弟里还有一个是大名府元氏县的人,名叫邓书,人称飞刀和尚,久在天津河北北开撂生意,天津河北北开的人们都知道有个飞刀和尚邓书,他专耍三把刀,耍起来呀,比天桥的狗熊程有过之无不及。那马万宝哪?也在奉天“土了点”(管死了调侃儿叫土了点)啦!生意人的下场,说起来令人伤心,我也不用说了。





天桥的旧人物常傻子


在前几年,天桥有档子生意,砸石头卖壮药的,是亲哥儿两个,人们都叫他们“常傻子”。他们每天带着一小铁盒丸药,弄些块石头,到了天桥也不找场子,只用一条凳子,将铁盒往凳上一放,常老大左手拿石头,用右手去砸,别看石匠砸石头是用铁锤,他砸石头只用手指一戳,就能戳碎了。他们用砸石头圆粘(nián)子(招徕观众),只要人围满了,随砸石头,随着讲说病原,什么叫闪腰岔气,错了骨缝,伤筋动骨,跌打损伤,风寒麻木,只要吃了他那百补增力丸,就能保好。说完了真有人买,哪天也能卖个三元二元的。

我老云前些年常去看他们这档子生意。近几年来,他们这哥儿两个忽然不见了,我向江湖人打听,这常傻子弟兄是“开了外穴(xué)”(管出外远行调[diào]侃儿叫开穴),还是“土了点”(管死了调侃叫土了点)啦?据某江湖人说,常傻子那档子生意说行话叫“挑将(tiǎo jiàng)汉儿的”,哥儿两个都是方字旁人(北平的人管从前的汉满蒙的旗人叫做方字旁人。按,旗字是个方字旁,就管旗人叫方字旁,也成了北平的侃语,旗人的侃语),都啃(kèn)“海(hāi)草儿”(管抽鸦片烟调侃儿叫啃海草儿),几十年挣的钱都送到烟斗里,分文没有剩下。常傻子到了五十多岁的时候,把“招儿念了”(管眼睛瞎了调侃儿叫招儿念了),做生意的时候都是常老二拉着他上地(做生意)。幸而他那石头是砸熟了的,卖药的法子是说惯了的,不然招儿一念,就该“念啃(kèn)了”(管挨饿调侃儿叫念啃了)。他受了眼睛的影响,卖项一日不如一日,就连痛带饿,活活地瘾死在小店了。

常老二向来是给哥哥当作助手,没充过正角,常傻子死后,他没能耐挣钱,又有口子瘾,不多的日子也找傻常去了。我向江湖人问:“他们砸的石头,有些人说不是真功夫,那石头是用醋泡了的,才能砸开。这话是与不是?”某江湖人说:“不是这样,这都是妄谈。他们砸石头的生意,是有一种托门(假的步骤),成天价练习,将托门练成了,拿过来石头一砸便开。这种托门和卖针的扎透铜钱的手法是一样的。”

常家弟兄就凭托门吃了一辈子,但是那种诀窍外人不易得着。自从常傻子故去以后,北平各市场就见不着砸石头挑将汉儿的生意了。最近天桥虽有档子砸石头的,净练开石头的功夫,不会卖药。江湖人说他们不是挑将汉儿的,并且傻练,不会圆粘(nián)子(招徕观众),不懂拴马桩(用手段拴住人)。受累不小,挣钱有限,算是档子“空(kòng)买卖”(江湖人管不会使生意门的手段的人调侃儿叫空买卖)。看起来平地抠饼(没有本儿要挣出钱来),空(kòng)子(外行)也是抠不成啊!





江湖中挑(tiǎo)沙子杵的生意


头几天,大雨数日,暑气渐退,唤起游兴,我老云往地安门外河沿什刹海,走在会贤堂饭庄东岸,望见了一群人,约有三十个,围着一个人,大家都直着眼睛听当中那人说话,我老云也站在外边听。这当中间的人说:“我不是北平人,是来找哥哥,不料来得不巧,扑了空啦!住店要店钱,吃饭要饭钱,在家靠父母,出外靠朋友。来到北平,举目无亲,人地两生,够多么困难。我带着有高货,可是货卖与识家,我卖给谁这高货?”

他这样说着,我老云瞧着他长得瘦小身躯,风吹日晒,黑黑的面皮,两个小眼睛很有精神,他穿的一身粗蓝布裤褂,手里拿着一个布包儿,他这套词儿,说完了再说,连着说了好几遍没有人理他。忽见有个高身材又白又胖的人向他问道:“你这人卖的什么高货,打开了我们看,看看你的货,我们好买呀!包在包内,你净嚷高货,谁知道你卖的是什么高货?真是怯杓(sháo,北平的俗语管乡下人叫怯杓)。”那卖东西的说:“我这高货打开了叫你看,你也不懂。我卖的是活血珠。”这人说:“我在北平跟过官,什么好东西我没见过?我见过守宫砂、活血珠。这东西贵重无比,没有地方买去。当初我们旧主人有个二姨奶奶,得了干血痨,屡治无效。有个医生说,干血痨的病有一种药能治,吃下去准好,叫活血珠,那东西出在西藏,是多年的藏红花所出的。前清的时候有西藏进贡来过这东西,现在哪买去!我们主人托朋友向王府里求了三颗,才吃下一颗去病就好啦。我瞧见过那活血珠,也就有绿豆粒大小,是紫红紫红的,透亮已极。可是轻极了,没分量,和琥珀一样,专治经血不调,妇人的血寒,到了经期血分不至,或是赶前,或是错后,什么血崩、血闭、血淤(yū)、干血痨、久不受孕,这活血珠吃下去,能去淤血生新血,种胎生子。男子吃下去,能治肾寒,梦遗滑精。”那卖东西的人,把左手的大拇指一挑道:“你这个人是行家。眼是观宝珠,嘴是试金石。我遇见了识货的了。”那人说:“你倒是打开叫我们看看哪。”他说:“你看完了,准能买我的东西吗?”这人说:“我看了你的东西,准要跟我说的一样,我就买你的。”这卖东西的人说:“我这东西不零卖,你要看中了,就都卖给你一个人。”这人说:“你真死心眼,谁能都买呀?这又不是大米洋面,买那些个干吗?”卖东西的说:“你不都要,就不用看了。”旁边有个人搭了碴啦:“你这人真糊涂,你打开叫那一位先生看看,如果你卖的真是活血珠,这位不能都留下,我们也能分着买你的,买了行个方便,结个人缘!”卖东西的说:“我这东西是不怕放着,搁几十年都不坏的,如若镶在戒指上,和那钻石一样。”就着就有四五个人,直催他打开包儿。他慢慢地将包儿打开,大家一看,他那包内有七八十颗紫红紫红的珠子,他递给大家几颗,叫大家观瞧,果然他那珠子是没分量,和琥珀一样。那识货的人说:“这东西真是活血珠,卖多少钱一对哪?”他说:“我这东西都卖了能值五百块钱。”那识货的人急了:“朋友,你这东西若遇见等着用的,多少钱都成,黄金有价药无价。只是一样,在这儿你要那么贵没人买,你包着它别卖,看着烙饼挨饿吗?”旁边的人都直劝他贱着点卖,先有钱吃饭住店,回家有盘费。劝了他好半天,他才跺脚道:“既是众位好朋友劝我,我卖四毛钱一对,两毛钱一个。我可不都卖,只卖二十对,哪位要哪位说话,过了二十对再要卖,我卖一块大洋,少了不卖。”于是,就有好些人争先恐后地抢着买,不到一刻钟,他就卖了好几块钱。在他正卖的时候,东边有个山东人直嚷:“我找巡警和他们打官司,他骗我的钱财倒不要紧,吃下去叫人难受,够多缺德。”有三四个人直劝他说:“等他卖下钱来,然后再找他。”

及至他将钱卖到手,我老云在后跟着,他们一伙有五六个人,往一茶棚分钱去了。我老云见他们这种生意有些像那个“做老坎的”、“挑(tiǎo)生啃(kèn)的”,向江湖人们探讨此事。据某江湖人说:“这卖活血珠的人们,就是那假装南方人卖药做老坎的,是那假装外省人挑生啃的。因为他们卖的那紫金果、川丁香都熏黑了,被冤的人多了,再照那法子去冤人就不容易了。他们改变了方法,不卖紫金果,不卖川丁香,改卖活血珠了。”我问:“那假装投亲不遇,卖活血珠的人是干吗的?”某江湖人说:“那人是掌买卖的。”我问:“那假装行家能识货的人是干吗的?”某江湖人说:“那个人说行话叫扒包的(由他装作懂行的,用话将布包儿打开,因此叫扒包的)。”我问:“那贴靴的(帮着做买卖)人,装腔作势,随声附和,是干吗的?”某江湖人说:“那些人说行话叫敲托(暗中帮助做生意的人,也可称为贴靴的)的。”我问:“他们那活血珠说行话叫什么?”某江湖人说:“他们管那东西叫底啃(kèn)。”我问:“那底啃是什么东西做的?”某江湖人说:“他们那东西是用化学方法制造的。有一种药品叫小灵丹,那小灵丹是一种丹药。用个破灯台点着了火,上边烧药,将药炼成了,如同一盏红玻璃灯一样,用锤砸下来一块一块,好像红玻璃又红又亮。要研为细末,用枣泥为丸,专治各种寒症,效力很大,就是阴寒,吃下去也能准好。他们挑活血珠的生意人,将那像玻璃块的小灵丹买了来,用个碗装点烧酒,使洋火点着,使个竹子夹着小灵丹往火苗上烧,如同拉胡琴的烧松香一样,往下滴滴珠儿,将那烧化了流下来的珠儿滴在棉花内,就像绿豆大小,又红又亮,如同红珠子一样。一般江湖人管这种东西叫做沙子杵儿。使这种沙子杵儿做生意,必须换出大钱来才能成哪,可是不能轻售的。如今做老坎的人们,挑(tiǎo)生啃(kèn)的人们,做成了沙子杵儿当活血珠儿卖,也是坏了江湖人的事儿。”我问:“怎么坏了江湖人的事哪?”某江湖人说:“如若江湖人向受骗者说他们有贵重药品,能治……那人也愿买下了,及至将价钱讲妥是多少元钱,将沙子杵儿取出来,那被骗的人若见过这种东西,他一定说,就是这个呀?我在什刹海买过,叫做活血珠儿,才四毛钱一对,你卖我几十元哪,我不要了。这沙子杵儿不是普通的东西,如若像卖糖豆儿一样到处皆是,社会上的人士都能认识了,江湖人再当妇女守节的守宫砂卖几十元哪,谁也不要了。这就是生意人坏了生意人的事儿。”我问:“他们坏了江湖人的事,江湖人有办法没有哪?”某江湖人说:“若在早年,就去个江湖人和他们讲理,他们就不敢再卖活血珠。到了如今可没有办法了,江湖乱道,谁也不守规矩,江湖人的规矩也不好讲了。”我听某江湖人所说,才知道卖活血珠的生意卖的是沙子杵儿,是江湖人乱道的事。望社会里人士扩大宣传,都别上他们的当,别买那活血珠儿,以免受骗。





汉门(凡是卖药的调侃都叫汉门)之挑(tiǎo)柴吊汉儿的(卖牙疼药的)


牙疼不算病,疼起来真要命。不论穷富,谁得了这种病也得赶紧调治。治牙疼的偏方儿是人人都有,能真有效力的实不可得。在各市场有一种卖牙疼药的生意,曾见他那药摊上写着“××牙疼药,立时止疼,不灵退洋”,有些个患牙疼的人找他当面去治,他们有一种“戳子汉儿”(管当时见效力的药调[diào]侃儿叫戳子汉儿),抹在牙上,立刻就能不疼。病人买他这种药到手,哪时抹哪时不疼,不抹还疼。病人花这回药钱,他们“挑汉儿”(卖药的)的行当叫“迎门杵”(挣的头一笔钱),即是头一回钱也;你要再找他呀,可就馈(要)你二道杵(钱)了。据他告诉病人,说是病没去根,想要去根,必须把牙内的虫子治出来,才能永久不犯。病人当然愿意去根了,多花几个钱算得了什么。将药价商议妥了,他用根细篾儿另抹上点药,待不了一袋烟的工夫,再用骨头针儿,从牙上往外拨吧,像线头儿似的小虫子全都拨出嘴来,还都是活的。在从前敝人也很赞成他们的药品,当时就能治出虫儿来,可称得起是神药啊。不过敝人向病人打听,治出虫子来那牙还是照样地疼。我问过病人,卖药的管保险除根,你为什么不去找他呀?病人摇头不语,实有欲言难吐之状。

在各市场有一种卖牙疼药的生意,药摊上写着“立止牙疼,不灵退钱”。



原来卖牙疼药的把虫子治出来之后,药钱到了他手内,他怕病人找,能向病人卖派几句“钢口”(说话的技巧和分量),叫病人花了钱受了冤不找他麻烦。那几句“钢口”话,江湖人调(diào)侃儿叫“抽撤口儿”,跟师傅学艺三年零一节,就学的是好钢口(即是能说的意思)。要没有“抽撤口儿”卖派呀,那不是叫人“倒了杵”(江湖人管挣下来的钱又被人索要回去调侃儿叫倒了杵)了?凡是生意人,不论做哪行买卖,要叫人倒了杵儿是为莫大之耻。

敝人曾向江湖艺人问过,卖牙疼药的能够当时治出虫子来,管那个方法调侃儿叫使“样色(shǎi)”,管那虫儿调侃儿叫“肉儿”。做这样生意,必须事先将菜虫子粘在细篾底下,名曰“上托”(即是弄毛病),往牙上一绷,菜虫儿便掉在牙上,愣一会儿再取出来,小小的戏法儿,便能“馈下杵”(要下钱)来。从前做这种生意的很是发达,近年来社会的人士知识日见开化,稍有点见解的人们就不能上他们的当了。凡是欺骗人的方法,任他使得多么巧妙,绝对不能持久的。都说一天能卖十石假,十天卖不了一石真。我最不相信这种话儿,阅者诸君如不相信,请你看看“同仁堂”就知道了。





江湖中之挑(tiǎo)生啃(kèn)生意


有这么一天,我到天桥遛个弯儿,走到了金鱼池的地方,瞧见有一个人站在那儿,穿着一件大片油泥的灰布棉袍儿,头戴着破旧的豆包儿软胎帽子,嘴里头直叨念着说:“可怜哪,可怜哪!”听他的口音是南方口吻,把可怜两字念成了“克恋”韵调。在他眼前的路上,地下放着一个白手巾包儿,叠得四四方方,在这包儿上插着一根笤帚苗儿,嘴里嘟嘟囔囔的。行人瞧着他这种的神儿都很奇怪,不知道他是干什么的。不大的工夫就被人们簇聚得围了个大圈儿,都要听听他究竟是干什么的。在这个当儿从人群里挤进一个人来,年纪就在五十来岁,他的穿着好像是宅门里的厨子,可是脸膛儿的颜色很显着憔悴,手里还拿着一根大杆烟袋哪。挤到这人眼前边,就向这人问说:“你是干什么的?”这人经他这一问,冲着他说:“我是做买卖的呀。”他听见这话,当时也露出点儿奇怪的样儿说:“你既是做买卖的,卖的是什么呀?”这人当时说:“卖的是紫金山上紫金树结的紫金果儿。”说的话都是南方口吻的韵调,乍一听简直捉摸不清楚。这人说完了,又给他重说一遍儿,围着的人才知道是卖紫金果的。跟着他又问这人:“你既是卖紫金果的,东西在哪儿哪?”这人手指着眼前的手巾包儿说:“这儿哪。”这时围着的人听见,都低头瞧地下的白手巾包儿。他又指着这人说:“你真是废物,做买卖的哪有包着卖的呀。能够有人买也得叫人瞧瞧东西呀,你打开包儿亮出东西来,叫人们瞧一瞧。”

这人在这时候哈下腰把包儿拿起打开了,一瞧里面包着有黑紫色的小枣核儿似的,可是周身有毛儿,有四五百个。它那颜色,就仿佛是炒煳了的铁蚕豆似的。他在这个时候用手拿过一个来,冲着这人说:“你卖的这是紫金果呀。确有这么一种东西,生在四川,是很贵重很缺少的一种药材。你从哪儿得来的呀?”这人说:“原是同人到那地方办事,听说这种东西是很贵重的药材,所以顺便带回点儿,送亲友或是行个好儿。现在来到北平这地方找人,不想人地生疏,费了几天工夫才把地址找着了,不想人早走了,不知往哪里去了,找是不容易了,想再回南边也是很难的了,所以就落魄在这儿了。手里的困苦那还能说吗?求亲无有,告友无门,忽然想起带着的紫金果儿在北平是很缺少很贵重的东西,何不卖出它去先济急哪!所以包好了在这儿卖。”这问主听完了这一席话,作出一种狰狞难看的面孔来说:“北平这个地方什么人都有,北平是藏龙卧虎的地方,有识货的。这种东西在前清的时候内廷里是常见的,外边人看见是不容易的。说起这东西来,用假的最能够骗人,因为乃是不常见的东西。要说这紫金果呀,你是蒙不了我的。它还有个名儿叫川丁香是不是?”这人微微地一点头儿。他跟着又说:“在从前我在内监陆某家里当厨子,陆内监谁不知道哇!哪一年给他送礼的什么没有哇!尤其是这种东西,我是司空见惯的,那时候他还给了我不少哪,到眼下我家里多少还有点儿呢。要瞧你这种东西,个儿跟颜色倒不像是假的,可是要掐开了,用舌头儿试试它的味儿,就可以知道是真的是假的。”他把话说到这儿,围着的人都疑惑他是懂行的,直瞪着两眼睛,不转眼珠儿地瞧着他,侧耳听他说。卖紫金果的这个人,反倒被他说得咬音咂字儿听着,他说的紫金果儿招招有谱儿。就拿着一个说:“净说它的个儿跟颜色一样呀,里面的瓤儿是不是也得瞧瞧呀?”他拿这果儿举在这人跟前说:“掐开一个诸位瞧瞧行不行呀?”这人说:“行呀。”他在这时候就把这果儿掐开啦,分作两半儿,把那瓤儿抠出来说:“瞧这瓤儿的成色倒像是真的,可是我知道那味道儿是能够分出酸甜苦辣咸五样味来,那才是真的哪。这种东西吃下去能够入人的五脏,专治妇人各种的病症:什么两肋发胀,筋骨麻木,胎前产后,胸闷胀满,不思饮食,咳嗽痰喘,妇人的百病都可以治的。就是没有什么病的人,吃了它也是有益无损的,这种果儿的出产,就是紫金山这一个地方有,它的贵重就在这儿。”说着话就把这果儿往围着的人手内一递说:“诸位先生可以尝尝这瓤儿的那味儿。”围着的人就有接过去送到嘴唇外边,伸出舌头儿舔,咂了咂味儿,微微地点了点头儿。有人说我吃的是酸的,有人说我吃的是辣的,有人说我吃的是甜的,有人说我吃的是苦的,有人说我吃的是咸的。这识货的人看见人都尝了尝。他也把那果瓤儿舔了舔,点点头儿说:“不错,这东西是地道的,实是紫金山上的紫金果儿。到眼下要搜寻这样儿地道货呀,真是不容易了。”话说到这儿,又冲着卖果儿这人说:“你这东西让诸位先生和我一尝呀,的确是真的。怎么卖呀?说个价儿,叫诸位先生好买呀。”卖东西的人在这时候才说:“谁要是买,一毛钱两个。”这识货的人听了这话接着说:“要按这时候一毛钱买俩呀,真算便宜。要到药铺买去,一毛钱买一个怕也不容易,并且它那成色跟味道还许跟不上这个好哪。话又说回来了,货到街头死,肉贱鼻子闻。在这儿就不能够跟人家药铺里比啦,贱贱地先卖出去,弄个饭钱,要多弄个盘费你好回家呀。你要凑个盘费回了家,也比你困在这儿强得多呀。你认了得啦,让诸位一毛钱买四个吧,便宜买主儿。”他话到这儿又冲着围着的人说:“哪一位先生要买先说话,等到没有人买啦,剩多剩少,由我一人包葫芦头儿啦,把它都买了,拿回家里防个荒儿,行个方便,遇见有病的妇人给她吃去,行个好儿。”围着的人听他这一片话,揣摸这意思就有动心的,及至听见一毛钱拿四个去,天下人爱贪便宜的人有的是,都要买点儿。那卖紫金果的人脸上便显出有点儿不乐意的样子说:“一毛钱四个我不卖。”那识货的行家听见他不愿意卖,向他又说:“你这人真死心眼,不便宜谁买你的,你别瞧着烙饼挨饿,卖点盘费回家,比你为难强不强啊?”这卖东西人把脚往地上一跺说:“得啦!我任什么话也不说了。谁叫我流落到这步田地哪!错非在这个地方,要不这样着急,给多少钱我也不能够卖呀。”围着的人听他说狠了心啦要贱卖,就争先恐后地你也往前挤着递票儿抢着买,他也挤着买,一眨眼工夫,就卖出去多一半儿,所剩的没有多少啦。那识货的人就向那卖东西的人说:“得啦,收拾起来不用卖啦,剩下的我包葫芦头儿啦,走吧,跟我到那边茶馆儿去取钱吧。”

我瞧着这东西卖得贱,有点儿眼馋,也想着买他几毛钱的,当时还向那识货的行家说了不少好话,请他匀给我点儿。他听我说了这些好话,也不好不匀给我点儿。他当时说:“朋友,这没有什么,您要买这东西又算得了什么,我家里还存着点儿哪,买不买都没关系。不过这东西是很贵重的,很缺少的,我是要买点儿来也是为行好。您要买我匀给您得啦,这又算得了什么。”他向卖东西的人说:“你就把这点儿卖给这位先生吧,数一数还剩有多少个儿?”这卖东西的人又把手巾包打开了,数了数那紫金果儿,一共还剩有七十七个,我花了六毛钱买了二十七个,便宜了三个。回到家中高兴已极,和我们街坊一说,街坊说:“你上了当啦!”我还不相信,与我们街坊抬起杠来。疑团难解,我想出个主意来,到药铺里去趟,叫人家真行家认认货,真假便可分明。

我拿着紫金果儿到了一家药铺,求人家给看看,药铺伙计看完了紫金果问我:“你这多少钱买的?”我说:“六毛钱买的。”他从药抽屉抓出一把来,足有三十多个,与我的紫金果一般不二。伙计说:“你要买我们这些个,一毛钱就卖。”到了这时候,我才知道纯粹是上当了。我向药铺的伙计问道:“这紫金果儿,到你们药铺叫什么名呢?”伙计说:“这宗药品不叫紫金果儿,他们卖东西的骗人,瞎诌的名儿叫紫金果儿,我们管它叫细辛。这种药并不值钱,可是不能多吃。遇见身体足壮的人,用麻黄不准过三钱,柴胡不准过四钱,细辛不准过一钱,他们卖这东西的骗人给个钱倒不怎样,倘若被骗的人吃多了这宗东西,与人命大有妨碍呀!”我听了人家这片话,东西也不敢要了;送给人家药铺,人家也不要,我只好把它扔在溺尿窝内自认倒霉罢了。

事过两月后,我到西城有事,走到新街口南,见马路边上有个人蹲在地上,眼前放个手巾包儿,包上插个草标,嘴里不住地喊嚷:“可难哪!可难哪!”我忽然想起来了,又是骗人的那小子,我要瞧瞧他们如何骗人,站在那里不走要看个水落石出。果然和我那天所见情形一般不二。最奇怪的是围着人都贪便宜,三毛五毛地买那东西。我等他们卖完了,在后边跟着他们,瞧他们到哪儿去。他们都进了一家茶馆,我也进去沏了一壶茶喝。我喝着茶的工夫,就见他们四五人在一处分钱,一共卖了两块七毛钱,每人分了五六毛钱就走了。

他们这种骗人方法,只要换个地方还能照样骗人,总是那套话,骗了一处又换一处。骗人的方法不改,还是用上就能骗钱。我又恨他们又佩服他们。有一次我遇见个江湖的朋友,和他讨论此事。据那位江湖人说:他们这骗人的买卖,江湖人调(diào)侃儿叫做“老坎”的,又叫“挑(tiǎo)生啃(kèn)的”。那假装南方人卖紫金果的,调侃儿说他是“掌穴(xué)”(这一伙人的头儿)的,他未曾做买卖之先得先练“浑(hún)碟子”(江湖人管他们学说南方的话语调侃儿叫浑碟子),又得练“发托卖像”(即是假装着急,假装愣头愣脑的,怯头怯脑的),到了做生意的时候,他把地势采好啦,蹲在地上,冲那手巾包儿一嚷,把粘(nián)子(观众)圆上,他们“敲托”(管帮腔骗人贴靴的调侃叫敲托的)的就挤在人群里帮腔,那个识货的行家调侃儿叫“扒包”的(由他装懂行的,用话将布包儿打开,故叫扒包的),卖钱多少,骗得了人骗不了人,全仗着他扒包的,他要有能耐,贴靴(帮助挣钱的)的时候能够叫人看不出破绽来。挣下钱来,扒包的、掌穴的分头份钱;那嚷嚷尝出酸、苦、辣的人是敲托的,只分小份儿。做这种生意是不能靠长地(长地是指固定做生意的场所)的,今天在东,明天在西,也是一种打走马穴(xué)(挣一回钱换一个地方)的生意档子。可是做这种生意都是四五个人,一个人做的很少。据江湖人说,要是一个人能做这种生意,算最有能耐的人。一个人做这生意是圆粘(nián)子(招徕观众)也得自己,扒包也得自己,敲托也得自己,别看一人班受的累多,挣下钱来也是自己的。像这做“老坎”的生意,一人做调侃儿叫“独角坎”,做独角坎的是十年百不遇的才能见得着哪。虽是骗人的行当,能做独角坎的生意人可是很少啊!

在前几天敝人撰稿完毕,觉着刷字匠的事儿很为苦闷,同三五个友人去逛隆福寺,在那庙会里曾见卖耍货的摊上有一种小孩的玩物小毛猴儿,仔细观瞧那毛猴儿就是那紫金果做的。可见那宗东西是不值钱的,做成小孩的玩物,还能卖得了多少钱哪?足见是宗最贱的药材了。





三不管的挑将(tiǎo jiàng)汉儿的生意


在民初的那几年,三不管有个打弹弓卖大力丸的叫高凤山。天津是个水旱码头,中外洋行林立,外洋的货物都在那里装卸,脚行都很发达。各省的人们出外谋生,到了天津,不怕没有亲可投,只要有膀子力气,水码头去扛大个儿,旱码头火车站卖点儿力气,当时就能挣钱,可算是华北的农工商业交货场,劳动区域。一般劳动的人都不能比阔佬、少爷、太太、小姐,高尚娱乐贵族化的消遣处不能去,就都常逛那平民化的露天市场伟大的三不管。他们劳动人热天不能怕热,冬天不能怕冷,有力气才能挣钱。身体多强壮,也容易受外感。如若筋骨疼痛,风寒麻木,立刻就不能挣钱。虽然有病,也不敢叫阔医生大医院诊治,他们都找那打弹弓的高凤山,买那大力丸,花钱不多,吃了就好。所以,一般劳动人有了病都找他去治。江湖人管他那生意调侃儿叫“挑将汉儿的”,每天能来个五六元钱,高凤山的收入也甚可观。

到了民国五六年间,这三不管又来了一位沧州卖艺的,长得身体魁梧,头大项短,肚大腰圆,大脸盘,重眉毛,大眼睛,高颧骨,大嘴岔,说话嗓音洪亮,他那人样就很“压点(yā diǎn)”(江湖人如若长得相貌好,气度惊人,调[diào]侃儿叫压点)。他往场内一站,几句话就能圆上粘(nián)子(聚好了观众)。前棚(场上)的本领他练趟单刀、七节鞭,叫挂子行的人看见还不“里腥(lǐ xing)”(凡是练武的人都称是挂子行;不假,调侃儿叫不里腥),真是“尖挂子”(江湖人管有真功夫的把式调侃儿叫尖挂子)。他要是“捋(lǖ)起粘啃(nián kèn)条子”(江湖人管向场外的观众讲说病原调侃儿叫捋粘啃条子),还真有包袱。那粘啃条子很多,我说出一个,阅者便能了然。他说:“大力丸能治腰疼,可是腰疼不一样,有受了寒的腰疼,有血脉不周流的腰疼,有闪腰岔气的腰疼,有房事过度肾虚的腰疼。那位说,什么叫受了寒的腰疼?告诉你,着了凉就重,出点汗就轻,那是受了寒的腰疼。什么叫血脉不周流的腰疼?告诉你,坐着疼,躺着疼,起来活动活动就不疼了,那就是血脉不周流的腰疼。不使劲不疼,一用力就疼,那是闪腰岔气的腰疼。如若咳嗽不敢使劲,眼前净冒金星,酸疼酸疼,那是卖煎饼的说睡语——摊(贪)多了,往前使劲大发了,我这里不治!”以上就算是腰疼的粘啃条子(病原),其余的头痛、腰疼、膀子疼等症都有粘啃条子。其中的意思与上论的相同。他随说随着抓,能把大家逗乐了(调[diào]侃儿叫抖包袱),他有这几条能耐,大受劳动人的欢迎。他往下“催啃(kèn)”(江湖人管当场售货能够多卖,推销的力量好调侃儿叫催啃),夯头也好(即是嗓音好),碟子也正(即是口音清楚),他做了不到一年就响了万儿(即成了名)啦。凡是逛三不管(天津市南市的一个露天市场)的人都知道沧州有个卖大力丸的高大愣,他的收入每天能有一二十元。高凤山的生意大受影响,日日衰落。高那人也好,有气性,改了行。将(jiàng)汉儿不挑(tiǎo)了,投个师傅改说评书。这些年三不管挑将汉儿的就是高大愣“火穴(xué)大转(zhuàn)”(江湖人管发达了调侃儿叫火穴大转)了。

劳动人如若筋骨疼痛,风寒麻木,立刻就不能挣钱。他们都买那大力丸,花钱不多,吃了就好。江湖人管这种生意调侃儿叫“挑将(tiǎo jiàng)汉儿的”。



江湖人要想发达,净仗着本领不成,还要相貌好,能够惊人。社会里的百行人也是如此呀。我老云闯荡江湖走了十几省,见过许多打把式卖艺的,都是“搪空(kòng)不搪相(xiàng)”(江湖人管外行人调侃儿叫空子,管行家调侃儿叫相家)。外行人看着两个人打对子火火炽炽,就给钱,真行家看了说:“他们是腥挂子(即假把式)。”惟有霸州李(他姓李,是霸州人,各省人敬他,都称他为霸州李)所练的把式,一点不里腥(lǐ xing)(假的),纯粹是“尖挂子”(受过训练的练把式卖艺的人)。上海、大连这两个码头,“老汪家”(江湖人管老大们调侃儿叫老汪家)都捧他。凡是江湖中的人,提起霸州李来,全都说:“那是尖挂子。”他在清末民初的时候曾到北平,同又说评书又练挂(武术)的沈岚俊在东安市场撂过场子。因为没“火穴”,他又走外码头了。可是霸州李到了天津,在三不管不打“清挂子”(净练把式要钱,不卖药,江湖人调侃儿叫打清挂子),也挑将汉儿。因为他的人样子不如高大愣,生意也不如老高,他们还是亲师兄弟哪!所以,我说社会里的人向来是认假不认真,有多好的本领也不如相貌惊人。从古至今,有多少能人都是受这种制,未能发达,不怪刘备见了庞统轻视于他呀!





三不管的做大票的生意


我在民国十五年赴津有事,住在荣华大街南头普通客栈之内,有个福州的乡亲李君住下关×安客栈,我常去看他。那客栈是三层院子,西院挂着个黑漆金字牌子,上边的金字是“军医朱洞×”。我见他那里求药治病来的很多,一拨接一拨。据李君说:“这位朱医官是位大慈善家,他这里治病不要钱,白看病还不算,格外可怜穷人,还给药吃,也不要钱。不论内外两科、花柳病、妇科、小儿科的病,治一个好一个。现在他的名誉可大了,无人不知。”我这人向来遇事多疑,我问李君:“他在栈房住几间哪?”李君说:“十二间。”我问:“十二间房每天多少钱哪?”李君说:“每间五角,十二间六元,他包了西院是每天五元钱。”我问:“他用着多少人哪?”李君说:“两个助理医生,一个看护,一个药剂师,四个听差的,一个车夫,大约着有十几个人。”我觉得他那十几个人,连住店带吃饭,哪天也得十几元的。要连穿衣服、药费,听差的、车夫薪水都算上,哪天都得几十元的费用。那位朱医官有这个举动,这个慈善行为,家中得有多大产业能够这样行善?他为什么不在本乡行善,来到天津住客栈哪?为什么不赁房开医院,住客栈多花钱哪?我越想越可疑,越想他的疑问越多。我猜想之际,李君又向我说:“这位医官手术最妙,不管是什么病,都能手到病除,你要不信你去看看。”我说:“往哪里去看?”李君说:“他每逢星期到三不管去治病。”我问:“在三不管什么地方哪?”李君说:“在天乐戏园子西南,黄福才的书场后边。”

我听他所说,记在心内,到了星期日这天,早早吃完饭,到三不管去看朱医官舍药治病。我到了三不管,果然瞧见黄福才的书场后有个大布棚,棚底下有三张桌子,四围有几条长凳,虽是夏天,有人用喷壶将地洒湿了,十分凉爽。场内有三个人,都年在三十岁,大褂外边罩着白围裙,胳臂上套着白布袖口,正往桌上摆设东西。接连不断有些个病人,都在等候朱医官治病。我往桌上一看,那上边放着全份的西医使用的西洋外科家具:耳撑子、嘴撑子、鼻撑子、肛门撑子、阴门撑子、听病袋、反光镜、小便探管、抽水管、大小各样刀子、剪子、缝针,这些东西电光镀得锃光,耀眼夺目。还有十几个玻璃盘,十几个洋瓶子,内装药水、药面子。有几卷药布、胶布、洋纸、棉花。他们这里一样样摆完,又摆上四个大玻璃镜框,内里有××军医官执文凭、××市的证书。我看他这些东西就值个几百元,那执照、文凭、证书也都惊人,及系我看完了,那棚底下的人围了个风雨不透,足有百数多人,都谈谈论论。我听了听,都是说朱医官手术好,药也好,手到病除。

我等了会儿,就见四面的人往外一闪,说:“朱医官来啦!”我往西边一看,来了一辆新式的胶皮车,车头上有四只电灯(可没点着),双脚铃,那车夫穿着一身绸子裤褂,正在年轻。那车上坐着的人戴着一顶巴拿马的草帽,白夏布大褂,青绮霞纱的马褂,青缎申鞋,金丝腿儿的眼镜。这人坐车来到,透着精神。车到棚下,朱医官下了车走进场内,我仔细一看,他长得约有三十多岁,白胖白胖的面庞,黑漆似的两道眉毛,双眼皮,大眼睛,高鼻梁儿,四字口,两撇胡子向嘴上撅着,笑容可掬,很像个大医生、军医官的派头。

朱医官往场内一站,那四面围着的人都直鼓掌欢迎。少时,他向四面的人先说了一遍冠冕堂皇的慈善话,然后他说:“凡是有病的人,都在凳上坐着等候;没有病的人,对不住,请你们原谅病人,给病人坐着,退到凳外看热闹吧。”他的话真灵,没病的人全都站起来退在凳外,坐着的净是病人啦。他将马褂、夏布大褂脱去,剩下一身短衣。由西问起,他问头一个病人:“你得的是什么病哪?”这个病人约有四十多岁,两只眼睛闭着,说:“先生!我得的是气火眼,闹了二年多了,始终也没治好,现在成了气火蒙了,求你行好积德给我治治吧。”朱医官就在桌上拿来他的家具,给这人往眼上抹了点药面子以后,说:“你先闭上眼不要动,不到一个钟头,管保这药力行开了,当时把那蒙给你治下来,能叫你看见东西。”这个病人就闭上眼睛不动,等那药力行开了好把病治好了。朱医官又向第二个病人问:“你得的什么病哪?”这个人说:“我得的是牙疼。”说着,将嘴张开叫他瞧瞧。朱医官说:“你这牙疼了多少天哪?”这人说:“七八天了,药也上了不少,始终也没见效。”朱医官说:“你这牙是虫食牙,火太大了,又得给你止疼,又得给你将虫子治出,才能去根。”这个人说:“先生你行行好吧。我这牙疼起来,扯得半个脸都不好受。牙疼虽不算病,疼起来真要命。”朱医官在桌上拿起来一个注射药针,在药水瓶内吸进点药水,叫他把嘴张开,用药针往他牙床上一扎,将药水打进去,随后抽出注射针来,向那人问道:“你这牙还疼不疼哪?”这人面露喜容,立刻站将起来作揖,说:“先生你这药真好,打上就不疼了。”朱医官说:“我再给你上了药面,管保一袋烟的工夫,那虫子全部治出来。”说着,用个药勺往瓶子内弄了点药面,往他嘴内上好,叫那病人张着嘴,往外流“哈拉子”。他又问第三人:“你得的是什么病哪?”这个人说:“我得的是恶疮。”说着将左腿的裤子往上一提,露出他那左腿来,在腿肚子上有像核桃般大的疮,那疮口张着往外流脓。朱医官说:“你这疮,我给你将毒水去尽了,烂肉也去掉啦,净剩下好肉芽儿,我再给你上点生肌长肉的药,不出七天,叫你复旧如初。”这人喜欢已极。他将注射药针拿起来,又由药瓶内吸了点药水,往那人的腿上打了一针,然后用刀子往那人腿上剐割烂肉,那人也不觉疼痛。观众都佩服他那药的力量,割完了,又用胶皮水激子使药水一洗,洗完了他问道:“疼不疼啊?”这人说:“不疼。”朱医官说:“再给你上些生肌长肉的药就好了。”说着,又往伤口内上点药膏,他往旁一闪,那听差的就给他用药棉花堵住,使药布一缠,手术敏捷,很是利落。我老云看着他又奔过那第一个病人去,叫他仰起头来,用个钳子去取那眼内的气火蒙,他用手一翻那人的眼皮,右手用钳将蒙夹住,慢慢地往下扯,随扯随叫那人咳嗽,不大工夫,将蒙取下来。他问道:“你这蒙治下来,你看得见东西看不见哪?”这个人说:“看得见了。”朱医官伸左手的三个手指头向他问道:“这是几个?”病人说:“三个手指头。”他又改了一个手指头,向病人问道:“这是几个?”那人说:“一个指头。”这时围着看的人全都拍巴掌,鼓掌喝彩。朱医官说:“你这眼睛好了一只还有一只,因为今天来的病人太多,时间宝贵,不能再给你治了。你等到明天去往栈房找我,再给你治那只眼睛。”说着递给他一张传单,说:“这传单上有我的地址,你按这上面的住址去找吧。”这病人接过传单点头应允。朱医官又给别人看病。

我老云看了六个钟头,见他治了三十多个病人,不论是什么病,轻者当时就好,手到病除;病重的当时见效。可是当时见效的病人又分两种:寒苦的他倒给包药,或吃或上,按症施用,分文不要,管保好病,不必再来。阔的病人,他给张传单,叫那病人按传单地址去找他再治。我对于这有钱的人找他再治很是怀疑,总怕他有敲诈的行为。他治完了病,围着的人全都不走,朱医官又向大家说:“我是咱们直隶的人,家中有几十顷地。我父亲自四十五岁得病,得的是疮痨,病了几年,花了几千块钱,也没有治好。感觉医生虽多,净是庸医。病人的痛苦,花钱误人,有冤无处诉,才叫我入医学校读书。出洋留学,学习为医,费了十数年的光阴,在中外医学校毕业,学成了中西医术、内外两科、妇科、小儿科、咽喉科、眼科、花柳科,全都能成。年前归国,奉我父母之命,施医三年,积德行善,过了三年之后才准要钱。我因天津这地方是个水旱码头,中外人士华洋杂处,什么人都有,才到这里施治传名。暂时不能设立医院,先在南关下头客栈内立个临时诊疗院,等到过了三年,我再设立医院。如今是白治病不要钱,不论哪界人要是有病,不论是轻是重,只管到栈房内找我,我都能给治好。可是有几种病别找我,都是什么病哪?瞳仁散光、瞳仁反背,这样的眼病我不治。男女有得臌(gǔ)(肚子膨胀的病)症的,七日管好;也有五样治不好。哪五种臌症治不好哪?男子要得臌症,从眼泡肿起要往下肿肿到两只脚上,就治不好了,那叫‘穿靴’。女人得了臌症,要从脚上肿起肿到眼泡上,就治不好了,那叫‘戴帽’。若是周身全肿,肚脐眼也肿起来,也治不好了,那叫‘绝症’。如若得了臌症的男子身上肿了,用手去按,若是按下深坑鼓不起来也治不好了,那叫‘绝症’。若是男子得了臌症,夜内滑精,也治不好了,那也叫‘绝症’。”他一样样地讲说,将各科共有多少绝症全都说出来;然后又向围着的人说:“只要得病之人得的不是绝症,我就能治。按着传单的地址找了我去,总能设法叫人好病。”他说完了,又撒了百十多张传单,才出了场子,上车回店。那听差的人们慢慢地收拾东西。

我也往回走。一路之上,就听三不管(天津市南市的一个露天市场)的江湖朋友对于朱医官议论纷纷,还有骂他的。据他们说:“金点(算卦相面的总称)的生意怕袋子金(算卦的一种),汉门(凡是卖药的调[diào]侃儿都叫汉门)的生意怕大票(大生意)。票票神仙都来到,受骗的人还少得了吗!”我老云原就疑惑他们是骗局,听了这江湖人的话,更知道是骗人的了。我有心向这种生意探讨他们的内幕,每天必到那栈房看看他们施治的情形如何。有天我到了那栈房,见门内院内吵吵嚷嚷,好像有人打架。我向店里伙计问是什么事争吵,据伙计说是做“大票”的生意出了“鼓”儿啦!我听了这话不懂,向他问是什么话。他说:“这是江湖侃儿,施药治病,冤人骗财的生意,行话叫‘做大票的’。他们骗人家钱财,人家醒悟了找来不依,要和他们打官司,调(diào)侃儿叫‘出了鼓儿了’。”我听店家所说,知道朱医官做的是“大票”。我就常向江湖人探讨,做大票的生意是怎么回事。有好些个江湖人都不知道,说那生意是汉门里最大的生意,内中的秘密不大明白。我问了多少人,都是不知道。

目前赴济有事,遇见一位老江湖,我将做大票的生意说了一遍,问他懂不懂。那老江湖说:“这大票的生意很不容易做,他们那当医生的是大票的掌穴(xué)(这一伙人的头儿)的。当这掌穴的最难,第一要长得相貌不俗,谈吐好,学识充足,对于中西医都得精通,才算够格。投师学艺,做师傅的收了徒弟,不教给他后棚生意,只教前棚的生意;那后棚的生意学会了能够挣钱,不到年份,师傅绝不传授。什么叫前棚生意呢?一是圆粘(nián)子(招徕观众),二是说前棚的钢口(说话的技巧和分量),三是叫点,四是卖弄,五是使样色(yàng shǎi)(做好弄鬼的东西),六是抖搂样色(实现以假乱真的效果),七是吸点,八是叫响儿。他们这种生意不能水做,必须火做(江湖人管做穷生意调侃儿叫水做,阔生意调侃儿叫火做)才成哪。做别的生意有几元钱的本钱就成。干这个要穿着阔绰,住大客栈,出来白治病,不能当场要钱,还得会找地方,看好场子,到了上地(该做生意)的时候,桌案上摆的医科家具等物,就值几百元。他们做前棚的往场内一站,得能圆粘子,使场的四面围好了人啦,他才说前棚的钢口(话),给围着的人好大便宜,将人吸引住了,再往下叫点(引诱病人看病)。‘哪位有病,可以说话’,就往凳上一坐,白瞧白看,用药也不用给钱。‘你们把病借给我,我用药,用手术,给你把病治好,叫大众瞧瞧’。这样说,就有那病人贪便宜,叫他给治。他们治病有几样儿手到病除的:痔疮、漏疮、气蒙眼、火蒙眼、胬(nǔ)肉盘睛、多年的帮聚疔毒恶疮、牙疼、耳痔、跌打损伤。可不是真治好了,他们耍的是手彩,调(diào)侃儿说,使的是‘样色’(yàng shǎi)。那‘样色’有:‘大卯’、‘小卯’、‘贴儿’、‘糊儿’、‘肉儿’、‘丁香’。什么叫‘大卯’哪?譬如,有个病人说他得的是心口疼的病,屡治不愈。他说他能当时治好。叫病人将上身的衣服脱下来,赤着背,他取出一个针来,约有二尺多长,叫围着的人观瞧。他说:‘三国的时候有位华佗,能治各种怪症,都听书上那么说,谁也没有看见过,今天我叫大众看看。这个针有二尺多长,从后心扎进去,从前心出去,穿心一针,能叫他多年的心口疼除了根,永不再犯。’他说着,叫病人往场子当中的凳上一坐,左手一按病人的身后,按着穴道一扎,眼看那针扎进一半去,他离开,右手攥着针叫大家看,说‘这针扎进一多半了’,冷不防他左手一拍病人的前胸,啪的一声,说:‘扎出来了!’大家往前边一看,果然露出三寸多长的针尖来,看的人们都很惊佩他那针法。他这样扎着,不叫病人动转(给他们当幌[huǎng]子),等着针的力量行开了好去病。他有了吸引人的幌子,可就卖弄他的本领,向围着的人说:‘真头疼必死,真心疼必亡。三国的曹操,喜爱关公,上马提金,下马提银,三日小宴,五日大宴,赐锦袍,赠赤兔,不过要关公保他。不料关公河北寻兄,封金挂印,过五关斩六将,拖刀斩蔡阳,又保了刘备,水淹七军,威镇华夏,吓得曹操几乎迁都。那关公走麦城被东吴擒去,誓死不降,斩了首级献与曹操。那曹操见了关公的人头,说道:美髯公别来无恙。那人头须发皆张,吓得曹操得了头疼之病,久治不愈,一命呜呼!那是头疼。真头疼必死,我们没有得真头疼的。三国的姜维扶保阿斗,邓艾偷渡阴平,刘后主降敌,姜伯约不得已使了个狠毒之计,使敌国主将彼此相争,大事将要成,他累坏了,心疼而死。真心疼必亡,我们人没有真心疼的。这位说是心疼,那绝不对的,他得的是胃脘疼的病,不是吃东西着急,就是吃饭的时候生了气,胃脘受伤才生这病。我今天把这病治出来叫他去根,永不再犯。叫众位看看他这病。’说着他由案上取个罐子来,用手将那根针拔下来,往罐里点些纸,将罐往病人的心口窝一拔,工夫不大,他将罐子起下来,举着罐围着场子一转儿,叫大家观瞧。那罐内有黄不黄红不红,粘粘糊糊的东西,谁看了也以为是他治好的。他又叫病人自己去看,说:‘你这病治出来了,从此再也不犯。’病人也得信服他,观众也信服他。其实那病是真没好,到了时候,还是照样儿疼。”

据老江湖人说:“他们这针,并不是真由后心扎进去,从前心穿出来的。他那针上有毛病,使的是手彩。他拿那长针往人后身扎的时候,只扎进几分深就止住了。那针是两节的,后半节粗,是空筒儿,前半节细,他用手一扎那前半节就退在后半截筒内。看的人们不明白他的机关,好像那针都扎在人身以内似的,前身露出来的针尖,也不是与那后边的针一式,是他左手内藏着个两头尖的针,一头扎前身少许。不知他那鬼病的人,就以为是扎的那针由前边露出的针尖。使用这穿心针是得像变戏法一样,把(bǎ)托(假的东西)护严了,不叫人看破才成,这个两半截针就是这样。他那罐里的东西,也不是由人身内拔出来的。他那罐内原就有这些东西,是药末做的,用时里边预先放点水,只要一见热气就化成粘粘糊糊的,谁看了也以为是由人身上拔出来的。这种‘大卯’的样色(yàng shǎi)就是这样,他叫人看,说说道道,调(diào)侃儿叫‘卖弄’。他们那由眼睛里,用药能治出气蒙、火蒙也是假的。那是‘样色’,说行话叫‘糊儿’。那蒙是小鸡的眼睛上剥下来的一层薄皮,成天价用药泡着,用时卷在手指盖内,以上药为名,将那皮放在眼内,病人的眼还有一线光明,被那皮遮避得什么也看不见了。他故意地叫那病人看东西,病人一定说:‘任什么也看不见了。’围着的人都知道病人的眼睛看不见啦,然后他再慢慢地往外取那薄皮。将皮取出来之后,又露那一线的光明,他们又伸手指头,故意地试人目力,叫病人说是几个手指头,病人就能看出是三个手指头,一个手指头。围着的人就信仰他们能治病。刚才还不见什么哪,这一会儿又看出什么来了,真有点儿手到病除的药力。这就是‘糊儿’的样色(那由病人身上治下来的漏疮管子也是假的。使用的样色,在前文丁香座子说过一回了;那治牙疼的药,能由病人口内治出虫子,在前文也说过)。这做大票的生意医治百病,当时有效,把人冤得信以为神医了,他在那些病人身上“把(bǎ)簧”(看出病人的底细)。看着病人穷,他就给药打发走了。临走还对病人说:‘这药你拿回去用,准能治好的。如若不好,你就不用来了。弹打无命鸟,病治有缘人。’那病人回到家内吃他的药,一定不能好,可也不来找他们再治,记着那舍药的先生‘弹打无命鸟,病治有缘人’。吃他的药没治好,得另找能治好该病的有缘先生吧。他们叫那不能生财的病人不再来麻烦,使病人绝了念头,说行话叫‘送点’。如若不会送点,那病人总来白治,够多麻烦,也赔不起那些药啊!若是没有送点的本领,江湖人讥诮他,说‘请神容易送神难’!”

“他们舍药的时候,瞧着哪个病人能够生财,设法叫病人去栈房找他们治病,在屋中受他们的敲诈。管这往栈房内诱病人,说行话叫‘叫点’。这些事都是做大票的徒弟没出师之先,给师傅效力时应当做的事,大约着不到三年时期,绝不传他们后棚的生意。那后棚的生意都是能挣钱,第一是要‘水火簧’。什么叫水火簧哪?譬如,病人要是穷,没有钱治病,他们也敲不出钱来,叫做‘水点’。譬如,病人要有钱,能够骗得出钱财来,叫做‘火点’。怎么要簧(要出实话来)呢?如若来到他们的栈房内,有五六个病人,他们就挨着个儿问。向病人问:‘你这病有多少日期啦?’病人说:‘二年多了。’他们又问:‘二年多治过几回呀?’病人说:‘也没怎么治它。’他们就知道这个病人是‘水点’了。请想,谁要有钱,有了病也不能耽误二年多都不治呀!如若久病不请医不服药,一定是个穷人。他们做大票的将簧要出来,知道是水点,就给他点儿药打发走了完事。如若问病人:‘你这病有多少日期哪?’病人说:‘四五个月。’他们又问:‘四五个月就没治吗?’病人说:‘这四五个月请了好几个医生也没治好。到各医院都看过了,药吃了好几十斤,也没见效。’他们就知道这是‘火点’。请想,这个病人要没钱怎么请那些医生?没有钱各医院也不能去呀,一定是有钱的。他们做大票(大生意)的要知道是火点啦,就不叫走,他们给治病,得施手术,不是扎针就是上药,用个拴马桩(用话把人扣住)的手段将病人拴住了,暂时不能走,缓缓地商议,叫病人上套。施展他们‘翻钢叠杵’(通过花言巧语使买主翻倍付钱)的手段,叫病人自己钻套,受他们的敲诈,总是说他们‘白治病,施舍药品,手术是随身带,不算什么,药品贵重,赔不起的’。或是叫病人自己情愿捐助药资多少,或是说病人用的药品太贵,准能好病,得自备药资。最难的事儿是‘坠票’。阅者诸君若问什么叫‘坠票’,我再将这桩黑幕揭穿,公诸社会,以便诸君对于大票生意完全明了。譬如某病人家中有钱,是个富户,他贪图便宜,有病白治,就到了栈房,找做大票的生意给他诊治。做大票的也要出簧(要出实情)来,知道他有钱,可是身上没带着,要治病花药钱,病人也愿意了,可得派人跟他去取,当时说明了是多少钱,做大票的掌穴(xué)(这一伙人的头儿)的派个人随着病人回家取钱,调(diào)侃儿叫‘坠票’。也有随着病人把钱取回来的;也有病人醒了攒(cuán)儿(明白过来了),事情反悔,不能给钱的;也有的病人觉悟他们是骗子,饶不给钱,还把坠票的打一顿的,什么情形都有。可是有本领的坠票的,都不怕病人觉悟、病人醒攒儿,他们遇上多狡猾的人也能把钱取回来。坠票的本领也分‘要簧’(要出实话来)、‘迷魂法’,他们跟着病人取款,走在路上问病人你在哪儿住?做什么事?如若病人说在××地方,在某商店做事,这钱便能平安取回。如若病人说,在某租界捕房做事,当便衣西捕。坠票的立刻就得明白这笔款不能要的,要不成钱,白挨顿打,还许给收起来。如若病人说,自己没钱得向亲友家借,坠票的得串给他几句话,叫病人向亲友撒谎,借钱使用,以免说出实话,被病人亲友窥破骗术,给他们破坏。”

做大票生意,在那些年,三不管(天津市南市的一个露天市场)、北开,各租界能有几十档子。天津地广人多,张三被骗,李四还不知道哪。骗人的是一个挨一个,被骗的是一拨接一拨。虽说他们这种生意伤天害理,巧取民财,也是病人贪图便宜,才能上当。这也不能净怨做大票的不好,谁叫病人爱图便宜哪!我老云不上当的主意,就是四个字秘诀:不贪便宜。这些年我再到天津,可就见不着做大票生意的了。据我调查,天津做大票生意的被官家取缔了,调侃儿叫“卯啦”。他们被卯的原因,我也调查过几次,据我调查来的情形有三:一是传授不好,鼓点(受骗的人明白了,和他们翻了脸,调侃儿叫鼓点)朝(cháo)翅子(有人告了官了),叫人知其内幕,被官家取缔了;二是医术不佳,常治死人,被官家取缔了;三是卫生当局取缔无证书行医者。

按江湖上老人所说,江湖人的传授最好不准做“绝后杵”(最后一笔钱)的买卖。我问过什么叫绝后杵,老江湖人说:生意人管钱叫杵头儿,管银子叫拘迷(jū mi)杵,管洋钱叫色(shǎi)唐拘迷杵,管挣人家的钱的法子叫杵门子。管挣钱的方法好,比别人能挣,调侃儿叫杵门子硬;管挣钱的方法不好,没有人家挣钱多,调侃儿叫杵门子软。管挣人家第一次的钱调侃儿叫头道杵或叫迎门杵(挣的头一笔钱),管挣二次钱调(diào)侃儿叫二道杵,其余的三道、四道,也是这样。生意人能有预知来人身上带着多少钱的手段,调侃儿叫把(bǎ)杵门子;如若没这种本领,调侃儿叫不会把杵门子。譬如,会把杵门子的生意人见来人带着十元钱,当时设法要挣他个七八元,给那被骗的人还留个三两元。那才是江湖人挣钱的高手,不能叫被骗的人空着兜走。就是被他骗了之后,怨恨心虽有,还能轻些。如若见人有十元钱,他们都给骗过去,叫被骗的人分文不剩,调侃儿叫“挖(wǎ)了他的绝后杵”(最后一笔钱)啦。那被骗的人连个喝茶吃饭坐车的钱都没有,当日饿着肚子走回家去。日后他明白了,是骗了他,那怨恨的心最重,一定说:“他们真厉害,将我的钱全都骗去,连个车钱都没给留,真叫狠透了,我非得找他要回钱来才能算完。”这样,被骗的人找他们往回要钱,调侃儿叫“倒杵”(往回拿钱),又叫“倒(dǎo)拦头子”。看起来做什么事也是留点余地,别太狠才好。狠毒人没有好结果,还是厚道点好啊。不知有真正夹磨(jiá mo)(师父传授真本事)的老合(江湖艺人)以为然否?可是那做大票的生意人,有好些个都是杵门子太狠,净挖人家的绝后杵,被骗人醒了攒(cuán)儿(明白过来了),找他们去吵闹,归了官司。官家知其内幕,临时诊疗所、施医、施药,都是骗人的团体,才认真地取缔。

在早年,我国还没有西医,他们做大票生意的人,只练会了扎针就能蒙人。按着中国的医书《针灸大成》,使铁做成针,只要扎不错穴道,绝不会出险。按着四阴针、四阳针、四大总针、八法神针、九转还阳针、鬼门十三针、王灵阁一百零八针等等扎法,都是中国原有的国粹,我国人学会,颇能起死回生。到了如今西医畅兴,那注射各种药针的方法,都是由外洋各国学来的,若没出过外洋,也得在我国各大医学校才能学会。江湖中做大票生意的人们,哪能出外洋,学习行医呀!就是在本国,也没有入医科大学的资格。他们所学的治疗外科手术,与注射药针的手术,俱都不精,全是胆大敢于愣下手。但是这些事极其危险,用刀子割疮,稍微失神,就能将血管割断,一时措置不当,血流不止,就有性命之忧。那六百零六药针,若扎在血管里,药性行开,花柳病能够好了,如若扎在肉内,当时胳膊红肿,一日就能丧命。做大票的人们,常常出这种危险。他们办坏了,就“急流扯活(chě huo)”(快跑)。人死与不死,他们就不管了。也有跑不了遭了官司,被法院判为庸医杀人之罪的。地方当局因为这类事迭见不穷,为保护人民起见,对于伪造行医证书与无证书行医、无官署售药许可执照售药的全都取缔了。做大票生意的人在天津立脚不住,全都开了外穴(xué)(到外地去挣钱)。据江湖人传言,我国各省城、各都市、各码头全有卫生机关,管理一切卫生事务,他们受到了限制,没有行医的资格,不能行医。可是,一般做大票生意的江湖人,因为在省市里面受了限制,不能再以大票骗人了,便开了外穴,跑到乡村市镇,如法再去骗那乡愚无知之人,较比在各省市、码头骗人还容易。于是乎江湖的骗子手们也组织医药的团体,摇旗呐喊,假借救济人命为名,去敲诈乡人。呜呼!一般被骗的人,成为有知识劣人的俎(zǔ)(砧板)上肉了。





三不管的挑(tiǎo)火粒的生意


有一次我到了天津,同着朋友去逛三不管,走到了上权仙南边,见那里的玩艺儿场全都没有了,也都盖了房子。往南走着,德美后兜个圈子,只见巷内十分冷落,好几十家娼户只剩了两三家,连个游逛的人也没有,可见天津也萧条了,德美后一落千丈实是可惨,较比十年前是不同了。我们往南走了不远,听见一阵锣鼓喧天,见开洼里棚帐接连,游人甚多,我看见了露天地的玩艺儿场,才知道三不管的艺技场因为盖了房子又都移在南边了。往各处一看,见那各场的玩艺儿也没有十年前齐全,较比民初的三不管缩小范围,连当年的十分之三都说不上了。在西头有个药摊,摊子上边摆的是几个药瓶子,几个碟子,有几块烂铁压着那包药使的票纸,有个玻璃框儿,里面是官家的许可执照,摆在旁边。有个妇人,约有四十岁里外的年纪,坐在个凳上,眼前有个小煤炉子,上面放着个小铁锅,里面熬的是什么也看不透。见那妇人向观众说:“这药叫化食丹,专治小儿百病,消食化水。不论是食积、奶积、大肚子痞积,跑肚子拉稀,红白痢疾,存食存水,吃了这化食丹准能保好。真金不怕火炼,好货不怕试验。眼是观宝珠,嘴是试金石,我当面考究,当面试验,叫众位看一看我这药的力量。”她说着又由摊底下拿出个小簸箕,那里边放着好些米粒,好些个豆儿。她说:“男女老幼如若脾虚胃弱,吃了东西不克化,你就吃咱们这药,管保能消食化水。”说着她拿起几个黄豆,一个个地往药锅里边扔,扔在锅内,就见豆儿一冒火苗儿就烧没了。最奇怪的是,她往那锅里边扔一块块羊肉,却呼呼直着(zháo),冒了一阵火苗儿,冒了一阵烟儿,就化没了。她说:“众位看见我这药的力量了没有,能化东西不能?我这药卖一毛钱一粒。”她说着用个小勺儿往外弄那药,一个一个往碟子上放。那药是白的,也有黄豆粒大小。她又说:“今天是礼拜,我们减价一半,卖一毛钱两个,买一服送一服,一毛钱买四粒。那位说,你这化食丹今天怎么卖这么贱哪?这叫小不去,大不来,名不去,利不来,传不出名去不能发财。”当时就有些个人买她的药,她随接钱随着包药。她还说:“哪位买了我的药,要治好了病,可得给我传名。如果吃了我这药不见好,你把发票拿回来,将钱退回;如若不来退钱,那算是怕我。”她这样说,更叫人相信她那药真有效力。

恰巧这时候有个老头儿带着个十二三岁的小学生,那老头掏出一毛钱,也要买那化食丹。小学生问他爷爷道:“你买这药做什么?”老头说:“给你兄弟吃呀,他不是净存食吗?”小学生说:“别买这东西,要吃在我兄弟肚内把他的五脏烧坏了呢?”老头子直点头,说:“有理有理。”我那朋友用胳臂肘儿一拱我道:“你听见没有?这小学生真聪明,他才十二三岁,就能把这事看破,年老人要蒙人,往后可不成了。”

我同着朋友往回走着讨论此事,怎么也猜不透她那药是什么东西弄的,能够把五谷杂粮和羊肉都烧着了。回到店中,我把这事记在心中,不断地向江湖人讨论此事。有个江湖人对我说,那妇人用药化豆粒儿的生意,调(diào)侃儿叫“挑(tiǎo)火粒的”。她摊子摆上,等到逛三不管(天津市南市的一个露天市场)的多了,她说说道道地圆粘(nián)子(聚拢观众),卖弄“前棚”(场上)的钢口(说话的技巧和分量),往锅底放粮豆儿,叫人瞧着他那药有化粮豆子、猪羊肉的力量,调侃儿叫“抖搂样色(yàng shǎi)”。她先说卖一毛钱一粒药,又改卖一毛钱四粒,那叫“海(hāi)开减价”(高开低走),又叫“催啃(kèn)”(管推销货物往外卖东西挣钱叫催啃)。她说,吃好了传名,要不好回来换钱,不找她退钱算是怕她,叫做“使神仙口儿”。我老云调查她那药,为什么扔下粮豆儿在药锅里立时就着,江湖人多不肯说。我好容易探讨得来,把他的黑幕揭穿。向阅者报告:那药里有火硝,要不东西到锅里就着哪!不知者以为我给他们宣了不好,坏了他们的事。其实骗人几个钱倒不要紧,钱花了别叫人受伤啊!火硝这东西到了肚子里,人受得了吗?我问过吃化食丹的人,吃下去那药觉着怎么样?都说吃下去跟着烧膛,心里发热,口干舌燥,净想水喝。看起来这宗买卖我给他们劈了是有益于社会。与她有点碍处,我就不管她一个人了。





江湖中之做老烤的生意


各市场庙会上常有一种摆摊子卖老虎骨头的。那摊上是块毯子铺在地上,一个长条的笸箩,四条老虎腿,一把小钢锉,一把小锯,有些纸张。如若有人看那虎腿,骨壮筋强,爪儿似爪,那骨髓油骨内骨外都浮着。凡是做这种生意的人,都是关东的居多,不论在哪个地方做买卖也没有摆长了的。据他所说,他是关东的人,专指着打围场挣钱养家,如今是来找他的亲戚,随身带些货物。这种虎骨是贵重的东西,要到各药铺去卖,能卖一块多钱一钱,专治风寒麻木、腰酸腿疼、多年的寒腿、肾寒肾虚、梦遗滑精、小肠疝气、五痨七伤、左右偏坠、左瘫右痪、半身不遂、诸虚百损。如若有这些病症,可以买点虎骨回到家中泡酒喝。这种药酒喝长了能够舒筋活血,追风散寒,强筋壮骨,提气补神,增加饮食,延年益寿;再吃长了,能够种子。为人不孝有三,无后为大。吃的日子多了,生有儿女,接续后世香烟,人生在世防备老,草留根深等来春。为人若是无有后,到了老来徒伤悲。他说:“我这虎骨在药性里说是大热的东西,专门治寒,可不治热病。如若是热病,愈吃愈坏。还不治暴发火眼,风火牙疼。那些病喝了虎骨酒,愈喝愈疼。那位说,你这虎骨卖多少钱一两呢?我这东西卖一毛钱一两,我可是待不长,卖几天我就走了。”

他们这样说,就真有人买。有人买的时候,他把虎骨放在凳子上,用麻袋片垫上,使锯现往下锯,锯下来用戥(děng)子(小秤)现平。我老云是好说真理。我国的药品是草药不值钱,牛黄、麝香、虎骨、人参、鹿茸、狗宝、犀角、西藏红花、羚羊角,全都值钱。他们这卖老虎骨头的,有那样好东西何不往药铺去卖?管保比他们零锯着卖省事省神,还能多卖钱。他们有真东西应当往真识货的地方去卖,何必与不识货的人费话?不问可知,他们那东西是假的。至于这假东西是什么东西做的,局外人是不容易知道的。

卖虎骨的这行儿调(diào)侃儿叫“老烤”,做这种生意的人,都得穿乡下人的衣服,说话要愣像儿。



我还在护国寺庙上见过一个卖老虎骨头的带卖麝香。据他说,那麝香出在关东三省,是香獐子的肚脐儿,每逢到了夏天,香獐子往山石上一躺,把肚脐张开,那各样的虫子都往它肚脐里钻,它一疼肚脐就并上,撒腿乱跑,可是香獐子也知道他那肚脐是宝贝,如若有人捉它,它也是先把那宝贝毁坏了,不叫人得着。鹿护犄角,象护牙,狗护宝,牛护黄。要捉香獐子得有好法子。那香獐子专好听音乐,如若要捉它,得上山中吹动音乐,它只要听见了就闻声而至,到了吹打音乐的附近它就不走了。地上有酒制的果品,它吃得醉了就能拿活的,拿住了就得到它的麝香。麝香有生的,有熟的。七年为生,八年为熟。这宗东西,最贱的卖一元二毛钱一分。好的当门子麝香,卖两块多钱一分。麝香这种药专能通人的七窍,通人身上的穴道,好膏药里没有它不成,好闻药里没有它不成。这麝香要带在身上别进花场子,如若进了花场子,那百样的花儿全都自落。就是怀胎受孕的妇女,带着了麝香也能把胎坠落了。他把那麝香说得天花乱坠,就真有人来买。

我要知道他们的内幕,就向江湖人探讨,他们那假老虎骨头、假麝香,究竟是什么东西做的?有个老江湖人,对于这行生意的内幕是很知道的。他说:“卖虎骨的这行儿调(diào)侃儿叫‘老烤’,做这种生意的人,都得穿乡下人的衣服,说话要愣像儿。师傅收徒弟教给徒弟前棚(场上)的生意,到了那里,怎么看地势?怎么撂生意?圆粘(nián)子(招徕观众)、卖弄钢口(说话的技巧和分量)、捋粘啃(lǖ nián kèn)条子,把各样的病都说出来,才能说药铺的虎骨贵得多,他们卖得很便宜。为诱惑人上当,惟一不二的妙法就是卖的时候如若遇见了‘火点’(江湖人管有钱的人调侃儿叫火点),如何翻钢叠杵(通过花言巧语使买主翻倍付钱),人家想买两角的,他能翻上去叫人买两元的。倘若那‘火点’有虚弱之症,他们还使枪里加花之法,取出鹿胎来,叫人买他们的鹿胎,或是鹿胎丸药,或是虎骨鹿茸丸、虎骨膏。火点若正点(江湖人管有钱人忠厚朴实调侃儿叫正点),数十元钱也能到手。他们教徒弟是什么都教,就是不教给徒弟做那假虎骨、做那假麝香、做那假鹿胎。徒弟学会了卖虎骨鹿胎的本领,得往各处做生意,卖了钱回去好好孝敬师傅。得给师傅挣几年钱,师傅才肯把那‘攥弄里腥啃(lǐ xing kèn)’(江湖人管自己亲手做假东西调侃儿叫攥弄里腥啃)的方法传给徒弟。”

我问那江湖人,他们那假虎骨、假鹿胎、假麝香是什么东西做的?那老江湖人说:“他们那是用的骆驼后腿,是三节。骡、马、牛、驴的后腿都是二节,做出来也不像真的。惟有那骆驼的后腿是三节,他们就使那骆驼后腿做假虎骨,可是这做假虎骨也极不容易,较比学什么手艺都难,那老虎爪是雕爪做上的,那腿爪相连着的虎筋是牛筋弄的。若是把三样材料得着,得用极好硬炭火,慢慢地熏烤,把那骨头烤得油儿外浮里溢了,把爪筋烤上也费若干日的工夫方能做成。那鹿胎倒容易,只用羊胎能充着卖。费事的地方是往羊胎上的嘴内镶几个小牙。有些懂行的人说那胎成了个儿就长牙,安上了牙才能像真的。那麝香倒不假,只是那是药铺把麝香卖完了,他们买了皮儿来,用各种香料做得了假麝香往那皮儿里装,那皮儿也有真麝香的味儿,就是真懂行的人,也能上他们的当。”

据我听某江湖人所说的情形推考,做老烤生意的人所卖的腥啃(kèn)(假的吃的药),若是买了去当真的吃了还不至于有多大的害处,不过耽误了病是真的。我老云在中年的时候往各处云游,很见了许多老烤儿的生意。到了如今,这种生意在各大都市是少了,各县的山场庙会集镇是多的。他们不在各大城市做生意,往乡间去卖,其中的原因是因为各大都市地方有卫生当局,对于无执照售药取缔得很严。他们卖的这种假东西,若是遵着市政卫生章程去领执照也怕不成,那卫生的管理法就不能容许的。所以凡是卖老烤的都没有零售药品的执照,时常受人驱逐,也是他们不能在都市省城存在的重大原因。再者都市的人士知识开化,对于他们这假东西一看就能看破,上当的人少,他们不能多挣钱,就都奔了乡间,乘着各县的人知识浅,取缔得不严,去骗乡下人去了。做这种生意的也是时代落伍者,受着人类知识进化的淘汰。他们还是脑筋太旧,牢守旧规,绝不改革。据我老云所料,再过个十年八年哪,这行儿的生意也就没有了。





江湖中卖点之内幕


在天津北开有一种卖眼药的,在场内放个茶杯,杯内满满的凉水,水皮上放些锅烟子,把一点儿眼药放在水中,那点眼药有黄豆大小,浮于水面,能够自动地追那锅烟子。凡逛市场的人看着奇怪,就能立着观瞧,这叫做“跑马招汉儿”(江湖人管这种卖眼药的调[diào]侃儿叫跑马招汉儿),他们也仗着这宗东西圆粘(nián)子(招徕观众)。杨某每天在鸡鸭店后身摆个摊子,圆粘子,卖眼药。除去他的本钱以外,哪天也能挣一两元。在民初的时候,人们的生活程度尚低,若每天有一两元的收入,也甚可观了。

杨某每日做生意的时候,总见有个麻子脸的人在旁观瞧,由他摆上摊子起,到他收摊为止,天天如此,几个月的工夫,一天不少,准来看他做生意。那杨某可就明白了,把那麻子脸的有何用意猜透了。这天杨某出来摆摊子比平常早着一个钟头。那麻子脸的人也来得很早,这时候有闲工夫,瞧热闹的人还没有来哪。那麻子脸的人便向杨某说:“先生今天摆得早啊!”杨某说:“今天吃饭早点儿,故此早摆会儿。”麻子脸的人说:“先生的买卖很好,我看了几个月啦,实在佩服!”杨某说:“你贵姓啊?府上是哪儿的人哪?”麻子脸的人说:“我姓李,叫李茂林,南皮县的人,离着马场很近,李家庄住家。”杨某说:“李先生在天津什么地方住哪?”李茂林说:“我住在关上。”杨某说:“你做什么事呢?”李茂林说:“我没做事,在亲戚家住闲。我自从长这么大也没做过事,现在家中的日月也不好,到天津来找事,住在我叔父家中八个月了,也没找着事。我要和先生学学这宗买卖行不行呢?”杨某说:“你要是愿意学,今晚上收了摊我们找个地方谈谈。”李茂林说:“好吧,你先做买卖,等你收摊的时候我一定来,回头再见。”说罢欢天喜地地去了。

杨某就知道李茂林是“点儿”(江湖中如若看谁能够生财就说谁是点儿)了。杨某觉着有点儿能生笔大财,他心中高兴。当日做生意也多卖钱。到了收摊的时候,果然李茂林来了,向他说:“杨先生,你我实在有缘,今天不成敬意,请您吃个便饭馆。”杨某将他的“啃包(kèn bāo)”(江湖人管做生意用的全份家具,行话叫啃包)送回家去,就与李雇了洋车往北大关十锦斋饭馆用饭。雅座里坐下,当然是李为主,杨为客,由客要菜。杨某足足地要了十几个菜,两个人喝着酒可就聊起来。李茂林请求他收自己做个徒弟,传授他卖眼药的生意。杨某说:“你的年岁比我小不到十岁,不能收徒弟,我收你做个师弟。”李茂林痛快极了,立刻就叫师哥,说:“师兄你只管收我这个师弟吧,将来我要挣了钱,一定得多孝敬你!”杨某说:“你的叔叔在天津做什么事呢?”李茂林说:“他开个杂货铺。”杨某说:“那铺子在什么地方哪?”李茂林说:“在北营门。”杨某听他说出杂货铺开在北营门,心里喜悦极了。阅者诸君若问杨某为什么喜欢?这也和说书一样,来个书中暗表。

杨某既看着李茂林是个“点儿”,要在他身上生财,因为不知他穷富,向他仔细追问,是要他的“水火簧”(江湖人管没钱的人叫水点,管有钱的人叫火点。欲知人有钱没钱,由谈话里猜出来,那行话叫水火簧)。杨某听说他叔父开的买卖在北营门,就知道那买卖资本雄厚,他叔叔有钱,他能多借,是个火点,杨某才喜欢。阅者若问他怎么知道那买卖是个大买卖?这是江湖人的“地理簧”。什么叫地理簧哪?譬如有两个商人都说他自己有买卖,若问他的买卖在哪里?北平的说在施家胡同,天津的说在河北大街,就知道开的买卖不小。江湖人对于各地街巷都留心访查,北平施家胡同净是银号,天津河北大街净是大杂货铺、大瓷器店、烧锅、五金行、山货店,所卖的东西不指着卖门市,都是大发行往各处走货,由北大关直到北营门全是阔买卖。如说那买卖开在小胡同内,那可就是小杂货铺了。杨某知道李茂林是个火点,存心要多弄他几个钱,就说:“兄弟,我要是收徒弟,他得给我挣几年钱,我由徒弟身上生了利,才能把全身的本领教给他哪!我收你这个师弟,不能当徒弟对待,你这个岁数,家中有老有少,我早早地把能耐教给你,你挣了钱好去养家,可是你怎么对待我呢?”李茂林说:“兄弟是个外行,一切的事都不懂,由你吩咐,无论有什么事我都能应。”杨某说:“我把眼药怎么配法,都用什么材料,怎么个卖法,一个星期都能教会,你得酬我大洋一百元。”李茂林听说要百块大洋,似乎为难,又向杨某商议求他减少。杨某执意不肯,并且向他表示,虽然花一百元,把本领学会,挣钱没数,能吃一辈子。临完了李茂林向他说,这件事他不能做主,得和叔父商议,叫杨某听他回话。他们二人吃了个酒足饭饱,一算账七块多,李茂林给了钱。由十锦斋分手,各自回归。

过了几天,李茂林找他说,凑了五十元,那五十元等过几天再付。杨某认了可,于是二人就过了钱,实行传艺了。杨某把配药的方法叫他看着,都用什么材料,哪样用多少分量,也都告诉他啦。学了三天,又把那制药之法学会了。李茂林又说:“师兄!你把这制药法子教给我了,那由眼睛里往下起蒙又怎么起呢?”杨某说:“那是假的,若是我们的眼药真能起下蒙来,气蒙眼、火蒙眼,治一个好一个,还用摆摊?我早发了财啦!”李茂林问道:“那假蒙是什么东西做的?”杨某说:“那假蒙是我们宰了小鸡,由鸡眼上起下来的一层皮,不用的时候在酒里泡着,不能叫它干了,到了出去做生意的时候再取出来。藏在瓶内。如若有病人害了多年的眼病,视物不明,有了云翳,我们给他往眼内上药的时候,暗将那假蒙藏在手内,如同变戏法一样放到他眼内,在那时病人就被假蒙蒙住,看什么也看不见了。我们故意地伸出几个手指头,叫他看是几个,他越说不对越好,可以乘那时夸奖我的药。叫他等着药力行开了,准能看见东西。待会儿再用手掰开他的眼,慢慢地往下起那假蒙,取出来举着让人观瞧。那按行话叫做‘高托’。然后再伸几个手指叫病人猜,他说对了,围着的人就知道我们的药效力如神。再卖吧,准有人买。弄这假蒙,行话叫‘使样色(yàng shǎi)’(实现以假乱真的效果),我们吃香东西就仗着这道‘样色’哪!”他说完了,又取出一个假蒙实地演习一回。李茂林如梦初醒,他知道了这黑幕啦。杨某又告诉他种种的行话,种种的诀窍。果然一个礼拜全都教会,李茂林又把那五十元付过,又拿出二十元钱,由杨某给他布置全份家具,小箱子、瓶子、发票、药品,都弄好啦。李茂林给他叩头,携带啃包(kèn bāo)(江湖人管做生意用的全份家具叫啃包),高兴回家。

李茂林到了原籍,往各乡镇去做生意,在集市上找个相当的地势,摆上摊子,茶杯盛满了凉水,浮面上撒点黑锅烟子,又取出那潮脑制的药末掐成小薄饼似的,放在水皮上,那潮脑就自己活动起来,催得锅烟子在水皮上乱转,招得人围着观瞧。李茂林也学了一套钢口(说话的技巧和分量),向围着的人说:“我这杯水,就好比一只眼,那黑锅烟子就好比人眼中的病,这点眼药追得那黑锅烟子在水皮上乱跑,如同在眼内追病一样。人生在世,无论是穷富,都有两只好眼,倘若眼睛有了病,任什么也不能干。我们这是家传的眼科,到了我这辈就是第五辈,这是五世真传的秘方,叫做‘拨云散’,专治眼科七十二症,三十六症内障眼,三十六症外障眼。什么叫内障眼哪?凡是由怒气伤肝上了眼,心有急火上了眼,肾经虚弱上了眼,那都是由五脏六腑得的病,叫做内障眼;如若羞光怕日,见风流泪,那就由风燥所得,叫做外障眼。人的两眼,瞳仁属肾,黑眼珠属肝,白眼珠属肺,大眼角属大肠,小眼角属小肠,上下眼泡属脾。我这眼药,能治风蒙火蒙,胬(nǔ)肉盘睛,鱼肉遮光,暴发火眼,见风流泪,烂眼皮,烂眼边,治一个好一个,治一百好俩五十。就是不治瞳仁反背、瞳仁散光,其余的眼科七十二症都能保好!那位说,你这药准能治好吗?如若治不好,你把摊子踢了,不算欺生。那位说,我们叫那卖假药的冤怕了,你说你这药好,那是老王卖瓜,自卖自夸,卖瓜不说瓜苦,卖酒不说酒薄。众位如若不信,我敢当面试验。哪位有害眼病的,你把病借给我,我把药送给你,治一回试试,如若能好,果然有效力啦,众位再买。”

他这样说,果然有那害眼病的人叫他给治,他也按着杨某的样,先使“样色(yàng shǎi)”(实现以假乱真的效果)后“高托”(高级的托儿)。只是一样,到了卖的时候没有人买,即或有人买,也不很多。他照这样赶了些日子集,赶了些日子庙会,所卖的钱不够他住店吃饭的,还赔了七八元。及至到了茶杯内那潮脑制的药没有了,他就按着杨某所传的方法去制,制完了往水中一放,也真奇怪,那药在水皮上浮着不动,连着试验十几次也是不灵,急得他也不做生意了。又觉得杨某不能骗他,怎么会制不好哪?于是,他数百里路程奔到天津,再往北开去找,那杨某也没有了。到他家去找也搬家了,天津都找遍了也没有。他到这时才知完全被骗。百数多元花了,岂能甘心!急得他害了一场大病。幸而有他叔叔照料把病养好啦,叫他回家。

李茂林回到家中,赋闲无事,时常地往附近赶集。过了一年多,忽然在集场里见有一群人围着,他挤进去一看,见地上摆着摊子也是卖眼药的,还摆着一些牙,不止卖眼药,还带治牙。他看卖药的人是个老头儿,约有五十多岁,卖弄的“钢口”(说话的技巧和分量)比杨某强得多,到了使“样色(yàng shǎi)”(实现以假乱真的效果)的时候,也比杨某利落,卖钱的时候也比杨某能卖。李茂林动了心啦,只见这位卖眼药的老先生在耳朵边上有个瘤子,他自称叫韩大疙瘩(gē da)。他瞧着人家把买卖做完了,天光也晚了,集上的人渐渐散去。在韩大疙瘩收摊的时候,李茂林就向他搭讪说话,非请韩先生吃饭不可。韩先生也很开通,和他在集市里找了一家饭馆,要了几样酒菜,两个人喝着酒,韩大疙瘩向他问道:“你做什么买卖?”李茂林说:“我是挑(tiǎo)山招的。”韩大疙瘩噗哧一笑,把酒盅也掉地上了,摔了个粉粉碎,乐得前仰后合,弄得李茂林莫名其妙。他等韩乐完了就问:“韩先生你为何笑得这样?”韩问道:“这是谁教给你的侃儿?”

李茂林就把他花了百数多元,拜杨某为师兄的事说了一遍。韩某才知道他是招汉杨“卖的点儿人”,向他说道:“你问我笑的是什么?告诉你吧,你要是向哪个江湖人说你是‘挑山招’的,谁也得乐坏了。”李茂林说:“这是怎么个缘故呢?”韩某说:“我们江湖人管人的粪门调(diào)侃儿叫‘山招儿’,管卖什么都叫‘挑(tiǎo)’,你说是‘挑山招’的,那不是卖屁股吗?”李茂林把这句侃儿听明白了,觉着自己花了百数多元,没学成什么还被杨某耍笑了,气得脸色更变,直骂杨某,非要到天津找他拼命不可。这位韩先生还算不错,好言相劝,算是把杨的“鼓儿”平了(江湖人管有人和他们打吵子叫出了鼓儿,管有人把他们的是非调停了结了调侃儿叫平了),并且向李茂林表示,他愿意收李茂林做个徒弟,分文不要,只叫他给效一年力。李自然愿意,就拜韩大疙瘩为师,随着他师傅往各处“顶凑子”(江湖人管赶集调侃儿叫顶凑子)做生意。

有了闲工夫,李茂林就问师傅,杨某对他是怎么回事?韩某说:“咱们这行儿调侃儿叫‘挑(tiǎo)招汉儿’的,可是卖眼药的那叫‘挑招汉儿’的,像那说是由土里得了宝贝,卖眼药的那叫‘海宝’。像咱们这眼药放在水皮上,追着锅烟子乱转悠,卖眼药的叫‘跑马招汉儿’,那杨某给你配药调侃儿叫‘攥弄(zuàn nong)’(自己做的调侃儿叫自己攥弄)汉壶,可是他教给你的都是‘里腥(lǐ xing)’的。”李茂林问他师傅,什么叫里腥的?韩大疙瘩(gē da)说:“凡是假东西,调(diào)侃儿就叫里腥的。说假话叫‘里腥钢’,冤人撒谎叫‘里腥人’,弄假东西叫‘里腥啃(lǐ xing kèn)’,假洋钱票叫‘里腥页子’,假洋钱叫‘里腥拘迷(jū mi)’,不说真名实姓叫‘里腥万儿’。”李茂林问:“他给我配的那药,放在水皮上就追着那锅烟子乱转悠,我把他制造的药使完了,我自个儿制造的药,放在水皮上就不动。方法也是他告诉我的,怎么不灵哪?”韩某说:“姓杨的没真收你这个师弟。他为骗你几个钱,把你当‘点儿’(江湖中如若看谁能够生财就说谁是点儿)卖了,哪能把真方法告诉你?他教给你那法子,也是‘里腥’的。他给你制造点真受使的东西,也不过蒙你些日子,他好远走高飞,等到你把那些受使的东西用完了,再找他也没有了影儿啦。”

李茂林至此才知道杨某的骗局是怎么回事。他向韩某问:“怎么他做生意也能挣钱,您做生意也能挣钱,我做生意怎么就不挣钱?”韩某说:“我们这行生意的本领分为三棚:设法招引人围着观瞧,那叫‘圆粘(nián)子’(聚拢观众);向围着的人说话,叫卖弄‘钢口’(说话的技巧和分量);向围着的人说病原,都叫‘捋粘啃(lǖ nián kèn)条子’。‘圆粘子’、卖弄‘钢口’、‘捋粘啃条子’都合在一处,叫做前棚的能耐。到了有人买药,设法多卖钱,那叫‘翻钢叠杵’(通过花言巧语使买主翻倍付钱)。向买药的说大话,告诉他们准能治好病,那叫‘神仙口儿’。先使‘神仙口儿’把他说得放了心,容买主给了钱,再说‘弹打无命鸟,病治有缘人’。治不好那是不该着好,百日灾难,九十九天好不了,那叫‘抽撤口儿’。病人把药买了走,听我们几句话,若是治不好也不来找我们麻烦。那几句话调侃儿叫‘拉后门’。设法叫那买主多有几个,调侃儿叫‘催啃(kèn)’。那卖钱的方法,那卖钱的诀窍,调侃儿叫‘杵门子’。‘翻钢叠杵’、使‘神仙口儿’、使‘抽撤口儿’、‘拉后门’、‘催啃(kèn)’、‘杵门子’都合在一处,叫做后棚的能耐。那杨某把前棚(场上)的能耐都教给你了,后棚的能耐他没教你,你如何能挣钱?”

李茂林听他把这些事说破,才明白了,向韩某说:“若不是师傅说,我这辈子也明白不了啊!”韩某说:“你只知道前棚是什么,后棚是什么,至于前后的本领,还得我慢慢地教给你,最要紧的是由前棚归后棚的时候,使那中棚的诀窍我传给了你,你才能挣钱。前棚、后棚、中棚连环着使用好喽,行话叫会了一个‘包口’(说完一段故事,再售其货,调[diào]侃叫包口),有一个包口的能耐就能吃一辈子。”李茂林听他师傅所说,觉着这江湖内的事儿,往浅了看是一层纸儿;往深了看,深如渊海,无有止境。他就好好地听说,存心给他师傅效力一年。韩大疙瘩(gē da)把他的本领按班就序地传给他。不到两个月的光景,他把前中后三棚的能耐全都学会,起初还是师徒同摆一个摊子,韩大疙瘩看着李茂林做生意,逢集赶集,逢庙赶庙,做了些日子买卖。李茂林是乍出牛犊子不怕虎,胆大敢言,气力壮,吃张口饭卖钢口(说话的技巧和分量),有这样的手法便是好手,比他师傅还多卖钱,很听师傅教训。韩大疙瘩品出他的心性,把全身之能一点不留,倾囊而授,让他单独赶集赶庙,挣了钱往回捎。六七个月足挣了好几百元钱。韩大疙瘩这个徒弟是收着了。到了一年的限期,由李茂林约出人来,摆宴谢师。以后挣了钱虽然是他个人的,逢年过节都孝敬他师傅些财物,爷两个的感情总算不错。

李茂林做了几年生意,就成了“挑(tiǎo)招汉儿”(卖眼药的)的大将(江湖人对于各行生意中最有本领最有名望的调侃儿叫大将)了。有一年他到了济南府,在趵突泉做生意。我老云正在那里,我看见了一桩奇怪事,有两个卖眼药的挨着摆摊子,我两头一忙,看这头是个四十多岁不足五十岁的人卖眼药,摊上写着“××堂杨”;那边是个三十多岁的卖眼药带摘牙,摊上写着“××堂李”。这两个人打对仗争持不决,那个姓杨的筷子敲茶杯招了一圈子人,圆上粘(nián)子(聚好了观众)卖弄钢口,这边姓李的向围着的人说:“咱们这买卖公道,先试验好了后要钱,不像那咪咪万儿,挑(tiǎo)山招的倒贴拦(江湖人管姓杨的调侃儿叫咪咪万,挑山招是卖屁股,倒贴拦是还找钱)。”我听着很是纳闷,按着江湖的侃儿这个姓李的是骂那姓杨的,怎么江湖人这么没有义气哪。往下再看更哈哈啦,姓李的在人群里直嚷:“再咳嗽!……”有个人就直咳嗽,他这一吵嚷,那姓杨的摊子就没有人围着了,他那里的人都跑到姓李的那里挤着看热闹。此时就见杨某坐着不语,气得脸上变颜变色。我看着这种事心中很是不平,直到姓李的做完了生意,就见在他收摊的时候,来了几个老江湖人,齐向李茂林质问:“为什么不按着规矩做生意?都是挑招汉儿的,应该两个摊子彼此离开一丈多远才能摆哪。相隔相(江湖人普通的称呼是个相家),离一丈。姓杨的是先来的,你是后来的,先到为主,后到为宾。你来了应先拜望姓杨的,然后做生意。你不按着规矩还“升点”,拉人家的粘(nián)子(江湖人管大嚷大叫调侃儿叫升点,管他吵嚷使围着姓杨的人都跑他姓李的那里去了,调[diào]侃儿叫拉粘子),是怎么回事呢?”李茂林见这些人来质问他,便把当初他叫姓杨的冤了,详详细细说了一遍。众老江湖人听后说:“那也不能怨姓杨的不好,当初你是‘空(kòng)子(不懂江湖内幕的人)’,他卖点(糊弄人)也不为过。”李茂林说:“我不恼他把我当点卖了,我恼他不该告诉我,我是挑(tiǎo)山招的,我和谁调这句侃儿谁也咧瓢(liě piáo)(江湖人管乐了、笑了,调侃儿叫咧瓢),太冤苦了我啦!众位不用管,他走到哪里,我追到哪里,我叫他挣不着钱,我们摽(biào)啦(管两个人熬了调侃儿叫摽啦)。”众老江湖人听明白了,都嗔怪姓杨的不该耍笑人。大家做主,叫姓杨的花钱请客,给李茂林赔了个礼,才算了结此事。我老云把在济南看见的这档子江湖人卖点的事儿援笔录出,以供阅者做谈天资料。江湖的黑幕真是层层皆是,揭穿不尽哪!





江湖中之挑(tiǎo)青子汉儿的


民国八年,我在烟台因事与友人陆子扬往牟平县找人,走到城西莱山,那天恰巧赶上集场,有无数的乡民乱挤乱蹭,叫喊之声十分热闹。在北头戏台旁边有一群人,围了个风雨不透,我挤进去一看,见里面有一档子生意,地上铺块毯子,有个小皮匣,一把破扇子,一把小刀。有个人长得凶眉恶眼的,向大众指指画画地说:“我不是此地人,我是济南府历城县的人。我们是亲哥两个。我有个兄弟在龙口学买卖,不料他没出息,把柜上的钱拐跑。我出来找他,手足之情,他虽不务正,我得把他找回家去,不能叫他漂流外方。我找了好几个月也没找着,我的路费花缺了,走在贵宝地,举目无亲,住店要店钱,吃饭要饭钱,我得求求众位,我可不是要饭,也不白求众位。我家是打铁为生,有个祖传秘方神效无比的刀伤药。当初我家可不卖这药,配得了只为行好积德,不论是街坊邻居,认识不认识,谁要做活不留神把手割破了,或是和人斗殴,刀砍斧划,到我家一说,白给一包刀伤药,抹在伤处,当时就止住了不能流血,消肿止痛,长得还快,伤不重当时封口,伤重了三两天封口。到了济南府向人打听吧,西关铁铺王家舍刀伤药,无人不知。我们这药原是不卖,如今我困在这里没办法啦,配了这药卖给众位。那位说了,赶集赶庙,有那传真方卖假药的,说得挺好,到了用时不见效力,叫他们蒙怕了,你的药我们也不敢买。倒是这样,前人洒土迷了后人眼。眼是观宝珠,嘴是试金石,真金不怕火炼,好货不怕试验!我把这药当面试验一回。叫众位看看,如若众位看着有效力再买,倘若看着没有效力,算我蒙人,谁也别买了。”

他说到这里,伸手把刀子拿起来,他这刀子约有一尺长,看着就很快。他又说:“怎么试验呢?我把大腿上割个口儿,往上抹刀伤药,抹上就能止疼止血。”他又把刀子放下,一掀小布匣,从里边取出好多包药来,说:“众位!我要自己由这堆药里取出一包来,众位也许说我这药有真有假,真的三成,假的七成,三七搅着,二八对着。我别自己拿,叫哪位替我由里边拿出一包来。哪位受累替我取一包?”他这样说,就有那好事的人走进去,伸手给挑出一包来。他把那药包接过去,当众打开。那药是末儿,红中发白的颜色,他用手把左腿的带儿解开,把裤子往上一捋,露出半截腿来,他右手拿着刀子,大声喊嚷:“我要割了!这也不怪众位不真信,是那些个婊子养的把人冤怕了,我割回试试。众位看我割的时候疼得龇牙咧嘴,止住了血也不流了。果然是这样,大家都买我一包,行个方便,结个人缘。卖多少钱一包哪?卖一毛钱一包。那位说我要买,你先别忙,这时买我也不卖,等我试验好了再买。今天我是先卖五十包,可是买一包,还格外地送一包。过了五十包之外,是一毛钱一包不多送了。”他说到这里,用刀子往大腿肚子猛然去割,看的人们,胆小的闭上眼,不敢睁开瞧。他刀子一割,顺着大腿往外流血,直疼得他龇牙咧嘴。他直嚷:“好疼啊!”他围着场转了一遭,流了不少血,然后往场地当中一坐,他把药在伤口上一洒,伸手拿起破扇子就说:“有人说受了伤用布蒙上,留神受风,受了破伤风可活不了。今天我叫众位看看咱的药有多大的力量。”说完用扇子往伤处呼呼地扇起活儿来,足扇了二三十下,他才把扇子放下,向四外人说:“众位看我的药怎样,止疼消肿不流血吧?”大众往他腿上一看,果然不流血啦。那血凝在伤口上,好像要封口一样。连我老云看着都佩服他的刀伤药了。于是他就说:“哪位买,一毛钱两包。买一包送一包,五十包为止,多了不卖,买着也别欢喜,买不着也别恼。哪位要哪位伸手!”他这一说,围着的人争先恐后地抢着买。我老云也看出这当面试验的药品好,掏出一毛钱买了两包,买完了,办事回归。我把这两包药好好地收存起来,想着遇事行个方便,结个人缘。

事情过了几个月,我到了大连,住于浪速町客栈。有一天,该栈的厨师傅贪酒吃醉,一时不慎,用刀将手割破,血流不止。我把这药取出来,向他们夸海口,说了朗言大语,我这药神效无比。及至把药上好了,那厨师傅疼得更厉害了,血还是流得不止。没露成脸,当时难看,人家另寻找别的药去了。我后来才明白上了当。那卖刀伤药的是个走闯江湖卖药的。

我向江湖人探讨卖刀伤药的内幕。有某江湖人说:“卖刀伤药的这行调(diào)侃儿叫‘挑(tiǎo)青子汉儿’(江湖人管刀子叫青子,管药叫汉子,青子汉儿即刀伤药也)的。干这行的生意也大有研究。按他们的行规是‘打马走穴(xué)’(江湖人管今天在东,明天在西,不靠长地方,满处乱跑的流动性质的生意调侃儿叫走马穴)的买卖,其骗人之法也分前后棚。前棚的生意,第一是‘圆粘(nián)子’招引观众,越人多越好,及至人多了,调侃儿叫‘粘子火炽’,围多了人时,嘴里所说的话,一件件,一桩桩,按行话叫‘卖弄钢口’(卖弄说话的技巧)。他们用刀往大腿上真割,叫人看他那药有效力没有,行话叫‘抖搂样色(yàng shǎi)’。”我问某江湖人:“什么叫样色?”某江湖人说:“凡是以假事叫人看着像真的,那种方法就叫样色。”我问某江湖人:“怎么他那药在他自己用着当时就能见效,到了我们手内就不成哪?”他说:“那药原就是假的,在谁手内也不能止疼止血。卖刀伤药的往伤口上药能够止血,那是障眼法,全凭扇子之力。”我说:“不错,当初那卖刀伤药的实是有把破扇子,他上了药的时候,曾用扇子往伤口上乱扇来着,可是他扇那扇子是怎么个用意哪?”某江湖人说:“他们卖刀伤药的人,使那样色也有研究,如若将用刀子把皮割破,那血正流得旺哪,多好的药,也不能在那血流正涌的时候把血止住,他们割破了肉,光围着场子乱转,等着把那血流的涌劲过去,然后往场内坐下把药上上,连药带血用扇子一路乱扇,那寒风把血吹得凝住了,自然不流了,可是别动弹,如若站起来走动还是流血。他们那行人在那血止住的时候,都是坐在地上不动,坐着卖药,以免再往外流血,失去信仰之力。”我说:“不错。当初我见那卖刀伤药的就是弄完了样色(yàng shǎi)(实现以假乱真的效果),坐在地上不起来,坐着卖药。可是他那药能止疼吗?”某江湖人笑道:“割谁的肉不疼?疼是真疼,他是强挣扎假装不疼。”我说:“如今医院里治外科疮症有一种麻药,如上了麻药,割时就能不疼。他们为何不用呢?”某江湖人说:“那药价值很贵,若用一次得两三元的才能止疼,他们江湖中的人做一次生意,能挣多少钱?麻药虽好,他们也用不起。”我说:“他们挑(tiǎo)青子汉儿的本领也有高低吗?”那江湖人说:“当然本领有高有低。那本领高的能多挣钱,得着挣钱的好诀窍,行话叫杵门子硬;那挣钱少的是没有得着挣钱的诀窍,行话叫杵门子软。他们的本领高低全由杵门子软硬而定。”我说:“他们卖药的时候为什么都使用限制的办法?说他多了不卖,就卖五十份。买一份送一份,过了五十份之外,再买就不送,一毛钱就买一包。那个用意是怎么回事?”某江湖人说:“那种方法是海(hāi)开减买(开价高,低价卖),最容易引人上当。有一种布摊,伙计们卖布带吆喝,一丈多布,先吆喝两元多,渐渐地往下落价,落来落去,能落到一元零五,世上的人都有贪便宜的通病,瞧着便宜就买。江湖人也用此法,行话叫做催啃(kèn)。他们先说出,就卖五十份,有了限制,人们才争先恐后地买,透着火炽,挣钱多寡,在他们催啃的能力而定。”我听某江湖人所说,才知道挑青子汉儿的催啃之法。我问他:“这卖刀伤药的行当这些年怎么见不着呢?”某江湖人说:“现在行医售药就有卫生机关主管,考取证书。卖刀伤药的没有售药执照,到乡间还能售药骗财;都市省会地方便受取缔,不能做生意。”这些年大地方就见不着这个行当了。况且,江湖人做生意都以容易挣钱为妙,谁也不愿受疼流血,干那行的人也日见稀少,挑青子汉儿的受了淘汰,无形中要消灭尽了。





江湖中的小省儿生意


民国十年春季,同友人王、马二人经营口有事,住在东马路客店。每日三人必经洼坑甸露天市场游逛,那里热闹已极,比天津的三不管(天津市南市的一个露天市场)、北平的天桥都不在以下。到了四月间,我见洼坑甸市场忽然冷落,游人稀少,各样的生意都收拾行李要往他方。我不知道什么缘故,向人打听才知道这些档子生意都去“顶神凑子”(江湖人管庙会调[diào]侃儿叫神凑子,管赶庙会去调侃儿叫顶神凑子)。

在离营口不远二百里路,有个岳州庙,是个最大的庙会,每年四月开庙。那个庙会较比直隶的鄚州庙、祁州庙,北平东的丫髻山,北平西的妙峰山还热闹。我是平生好游,就要往岳州庙会去逛逛,最便利的是火车有往返票。那岳州庙原不接铁路线,因为到了岳州庙会的时候,东三省的人不论远近都去赶这个庙会,铁路机关鉴此,做这一回买卖,在那里添个临时站,并且各路都有火车往那里转去。虽哈尔滨、吉林、长春、大连等处,也售往返票,还是便宜已极,由营口车站购票往返才几毛钱。

我们到了岳州,因为那里没有客店,临时得住民房。每逢开庙的时候,那里的住户,也都投机把房间腾出来,赁与客人居住,较比普通客店房价便宜,就是不大洁净。他们那里的习惯是顺山墙一溜长炕,炕上烟盒一个,关东烟叶大家共吸。而妇女则每人一个烟袋。

到了庙里去逛,可就应了那句话了:大庙逛庙内,小庙逛庙外。庙大里面能容纳各样生意,逛庙的人逛里边成了。小庙地方小,容纳不下各样生意,只有香火道场,是玩艺儿都在庙外。岳州庙会虽然有名,只是庙内地方小,我们往庙外去逛,见各杂技场的玩艺儿都是看过的。那一溜饭棚有几十家子,成桌的酒席都有,贱的随意便饭。卖骡马的、卖山货的、卖估衣的、卖香料的、卖梳篦(bì)的、卖绸缎布匹的、卖化妆品的、卖鞋袜的,应有尽有,无不齐全。我们走到山路旁,见有算卦的、相面的、变戏法的等等生意。

有一档子生意我看着各别。是一个摊子上铺块毯子,上放观音大士像一尊,那摊上有些纸张。摊旁有个和尚,围着的人,妇女居多。那个和尚有三十多岁,长得獐头鼠目,很是狡猾的样子,他嘴里嘟嘟囔囔说的是:“我是千山慈云寺的,奉师命下山,普济慈航,救治有灾之人。不论是男是女,只要有病,可以向我讨药,吾佛的万灵丹,能够治百样病,我和尚是分文不取,毫厘不要。哪位有病,只管讨药。该着有缘,佛爷赏药;如若不该除灾,佛爷不赏药。”他这样说着,有位五十多岁的妇人讨药,和尚问她:“你是自己讨药,还是给别人讨药?”这妇人说:“我给我儿媳妇讨药,因为她净有病不生养。”和尚说:“你给佛爷撂香钱吧,看你们有缘无缘?”这妇人恭恭敬敬地取出五毛票,放在摊上,还跪在地上叩了一个头。在这个时候,和尚向观音佛说:“如若该着她儿媳立子,我佛赏药;如若不该她儿媳立子,我佛就别赏药。”他说着就见由观音佛的手内有个窟窿里掉出一包药来。那和尚打开一看,是几十粒蜜丸子,如黄豆粒大小。他数了数,共四十九丸,向那妇人说:“你把这药拿回去,每天晚上用开水送下一丸子,未吃药之先,得烧一回香,那香炉中所用的灰,可得取七七四十九家的香灰凑成一炉。往各家去要香灰,必须在星斗出全了的时候。人家问你要香灰干什么?你就告诉他:我这里舍药,能治百病,吃了药准好。”那妇人不住地点头。他又说:“你如若说别的,这个药可不灵。”妇人也不住地点头。我看了会儿就往各处去逛。

假和尚舍药也是一种生意,江湖人称“小省儿生意”。



天黑回寓歇息,我那朋友王君,对于江湖事全都懂得。我说:“金(算卦相面)、皮(卖药)、彩(杂技戏法)、挂(练把式卖艺的),什么生意我都有个一知半解了,惟有这和尚卖药的生意,我看着不懂。你说这是怎么回事?”王君说:“这和尚舍药也是一种生意,据江湖人说,这行儿叫‘小省儿’,那个和尚也是‘里腥(lǐ xing)化把(bǎ)’(江湖人管和尚调[diào]侃儿叫化把,管真和尚叫尖化把,管假和尚调侃儿叫里腥化把)。做这小省儿生意的,也得投师入门,若是没有江湖的门户,可做不了生意。他们同行的人见了不认识的新上跳板(刚入这一行的)的,就和他盘道(互相盘问根底,看谁的能耐大),如要被人盘问短了,不惟不叫做这生意,还把所用的东西全都拿走。就是有师傅、有江湖门户的,对于盘道的事儿不大明了,被同行的问住,也有被人把东西拿走的,那只可找师傅出头找他们,再往回要东西。干江湖事,没有门户,不会盘道是不行的。他们这行儿做生意也分前棚、后棚。前棚的本领讲究‘圆粘(nián)子’(招徕观众)、‘做包口儿’(说完了一个故事,再要香钱,请观音赏药,调侃叫包口)、‘叫点儿’(叫住能让他们挣钱的人),后棚的本领就是‘翻钢叠杵’(用语言要出几回钱来)、‘拉后门’(没治好病的人回来找麻烦,用几句话把人说走)等等事儿。惟有做小省儿的,不能在省市码头靠长地(长地是指固定演出场所)。若是天天做这一套,日久天长,也没人信了,最好是‘打走马穴(xué)’(做一次买卖换一个地方),今天往东,明天往西,冤了谁,上当就一回。他们这行里专找信神佛的人,做生意都赶各处的香会,因为各处信佛的人都爱赶香会往各庙里烧香,他们投这个机,吃善男信女是准成的。江湖人管他们这种生意所圆的粘(nián)子(观众)调(diào)侃儿叫‘疙瘩(gē da)粘子’,四面围着他们的人,不过几十口子,绝不够几百人,若是围几百人的大粘子,那就是敲锣鼓式的武生意了。圆好了粘子,总是说他不要钱,是奉师命下山来结善缘,或是说募款修庙,究其实也得多少给几个钱方能给药哪,还是指佛穿衣,赖佛吃饭。他说什么病都能治,叫有病的人讨药,行话叫做‘叫点’,也是叫人上当也。那个观音佛的手内有个窟窿,有时人讨药讨不出来,或讨得出来,也没别的妙法,只因那佛像内有个铁盒子,那盒子的门儿没有插关,只凭一块吸铁石,那拐棍的下头,暗露桌案底下。如若他看着讨药的人像个花钱的,就把桌案底下的拐棍一转儿,那吸铁石就离开了盒子门儿,那门一开,就由里边掉出一包药来。如若看着讨药的人不像花钱的,就不动拐棍儿,那吸铁石离不开盒子门儿,焉能掉下药来呀?他们叫人给他尽义务扩大宣传,就是利用妇女们知识浅薄,受信佛的驱使。他叫病家的人于每日星斗出全了的时候往各家要香灰使用,并且还叫向给香灰之家说,这是××山××寺的和尚讨来的,这药不是花钱买来的,他这药能治病,什么病吃了也能好,治好的病太多了。病人的家中人向各家这样说,他们岂不是给做小省儿生意的尽义务扩大宣传?再者,那给香灰的人家也是信佛的,不信佛焉能烧香?听着有僧人舍药,只怕没病,如若有病,就得去找他们讨药,只要去向他们讨药,撂个香钱,就得了。再看着讨药人忠厚有钱,就用那‘翻钢叠杵’(通过花言巧语使买主翻倍付钱)的方法大敲一下,进一步敲诈的办法,就得叫病者家中的人,请他到家看病人是什么病,调侃儿说‘入窑儿’。”

有一年,我老云在某处见有一个病人家,请来一位僧人(即是做小省儿生意的),听僧人说:“你们这病人是游魂扑影。”病人的父母问他:“什么叫游魂扑影呢?”和尚说:“病人在好的时候,因为时运不好,被游魂怨鬼扑了一下才生的这病,故此叫游魂扑影。”病人的父母问:“游魂扑影好得了吗?”和尚说:“有游魂扑体,还有游魂扑影。人走在街上,忽然倒地就死了,那是游魂扑体;幸而你们这是游魂扑影,若是扑体就没法治了。”病人的父母说:“这可怎么治哪?”和尚说:“这得请佛赐灵符,赐点炉药才能治哪!”病人的父母说:“求师傅多慈悲!”和尚说:“你们给香烛等物钱,我去买应用的东西,今夜上坛,讨了炉药灵符,明天送来。”这样,病人家就会量力而为,几十元乃至几百元都不算什么。

做小省儿生意的多在各庙会,不料日前我老云去逛隆福寺,见生意场内也有个小和尚做“小省儿”。他虽没有佛像,舍药治病、赚人钱财之法,与我所见所听的略有不同。好在他用的汉壶(江湖人管药调[diào]侃儿叫汉壶)与切糕丸相仿,倒无多大的害处。





江湖中之挑(tiǎo)顿(dūn)子汉儿的


北平这个地方,到了初冬,天旱缺雪,忽冷忽热,时令不正。有些个江湖人都投机做“顿子汉儿”的生意。

日前我老云笔管的工作完了,有朋友约我往天桥去巡礼,走在电车总路西边,见有一群人围得挺严,里边有个人说说道道的,不知道干吗的。我挤进去一看,见是个摆地摊的,地上铺着一块毡子,有个小方匣子,两个洋瓶子,有些个门票纸,几个兔子脑袋,几个兔子腿儿。那匣子前边有块漂白布,写着“×××堂秘制兔脑丸,专治男妇老幼五劳七伤,春前秋后咳嗽痰喘等症,如用此药白开水送下,效验如神。”那个卖药的人,穿着青布棉袍,像个乡下人。我听他说:“这咳嗽不是一种病,咳是咳,嗽是嗽,有声无痰那是咳,有痰无声那是嗽。有声有痰,才叫咳嗽。咳嗽痰喘不一般,白痰轻,黑痰重,吐了黄痰就要命。内科不治喘,外科不治癣,有风寒咳嗽,有肺热咳嗽,有肾虚咳嗽,有三焦火盛的咳嗽。不怕吐痰一大片,就怕痰上带血。我们这是三代祖传的秘方,用三十六味草药配的兔脑丸,这里边也没有牛黄、狗宝、珍珠、玛瑙,净是不值钱的药。偏方能治大病,草药气死名医。咱们这药不贵,卖一毛钱两丸子。病重的两丸子准能保好,小孩半丸子,病轻的一丸子。如若吃不好的,发票为凭,只管来找我原钱退回。如若吃不好不来找我退钱,那算你怕我。今天是十五,减价一半,卖一毛钱四丸子。哪位要哪位说话。”

他这样说着,就有个五十多岁的老头儿,又咳嗽又喘,向他问道:“你不是说外科不治癣,内科不治喘吗?怎么你这上面写着‘专治咳嗽痰喘’呢?你说这喘是怎么回事?”他说:“不是外科不治癣,是外科的病数着癣难治;内科也不是不治喘,是内科的病数着喘难治。告诉你吧,人的肺是三斤三两重,六叶两耳,肺管有节,左通气嗓,右通食嗓,上有三八二十四个窟窿,分为二十四个节气。六叶在前,两耳在后,人的呼吸气全仗着肺的力量,如若肝经火盛,催得肺叶扎煞了,那就喘。你问这喘怎么回事?告诉你是拢不住肺叶了,必须吃咱这兔脑丸才能好。”老头儿说:“吃你这药准能好得了吗?”他说:“弹打无命鸟,病治有缘人。百日的灾难,九十九天好不了。如若该着你除灾,该着我露脸,你吃了这药准能好。我要自己说我的药好,那是老王卖瓜自卖自夸。这不是卖档的,是天天在这里摆的长摊,你不放心先买两丸子,拿回家去吃吃试试,如若不好,你就算上了当。吃着见好,你再来买。”那老头儿就买了他两丸子。他又告诉老头,这药到了临睡觉的时候用鸡子清儿对点儿香油送下去,准能止嗽化痰。老头儿点头去了。我在他那里看着,也有那买主儿说:“你再给我来两丸子,头两天买了两丸子,吃着不错。”

我看得很入神儿。我的那位朋友却不明白江湖道,他扯着我走了,非要往天华园去听大鼓,乃至到了那里听谢文英唱了一段《拴娃娃》。山东的犁铧调儿虽好,我不是好那条道的人,把朋友稳住了,脱身由那里出来找个江湖的朋友去讨论这卖咳嗽药的是怎么回事。我到了这江湖朋友家中,向他问:“我见了个卖咳嗽药的,他是怎么圆粘(nián)子(招徕观众),怎么说的,怎么卖的,是不是生意?”某江湖人说:“卖咳嗽药的这一行调(diào)侃儿叫‘挑(tiǎo)顿(dūn)子汉儿’的。干这种生意不是总干这个,春夏秋三季干别的生意,到了入冬的时候才能做这买卖,因为到春夏秋三季咳嗽的人少,就是有咳嗽的人也不是时令咳嗽,都是身体虚弱、久病身虚的咳嗽,那种人病的日多了就应了那句话了:久病是名医。对于请医买药有了经验,绝不照顾江湖人。做这种生意日期是最少的,只能在初冬之际做几天。”我说:“怎么才做那几天呢?”他说:“人若到了六月,要热也禁得住,热惯了也不理会。可是在四月底将热的时候,人们都嚷热,那是没热惯哪!到了十月的时候,天气将冷,一般咳嗽的人都是受外感的多,老病人冬令犯的多,遇见卖咳嗽药的,花钱不多,买几服试试,等到真冷了,咳嗽日子多了,咳嗽惯了也不大理会。吃过几样药总没好,再见了卖咳嗽药的也不买了。况且那咳嗽病也碍不了多大事,能禁得住,他不治了。”我说:“干这行的有何奥妙?有什么骗人的方法吗?”他说:“干这行的也得受‘夹磨(jiá mo)’。”我问:“什么叫受夹磨呢?”他说:“我们江湖人管得过什么传授调(diào)侃儿叫受过夹磨。”我问:“这行都有什么夹磨哪?”他说:“第一是得拜个老帅。”我问:“什么叫老帅哪?”他说:“我们江湖人管师傅调侃儿叫老帅。譬如江湖人见了面,说,你们老帅是哪位呀,那就是问师傅是谁。”我说:“拜老帅有什么意思哪?”他说:“要拜个老帅是为学能耐,投明师,访高友,才能学出真本领。在未拜师之前,最好是先打听谁的买卖成快,再拜谁。”我问:“什么叫买卖成快呢?”他说:“江湖人管谁的生意能够挣钱,谁的本领地道,调侃儿叫买卖成快。譬如有江湖人谈论说,谁的买卖成快,就是谁的本领好,是有了挣钱的诀窍。”我问:“拜了师傅都学什么呢?”他说:“学的是攥弄(zuàn nong)(自己做的调侃儿叫自己攥弄)啃(kèn)、圆粘(nián)子(招徕观众)、捋粘啃(lǖ nián kèn)条子(向场外的观众讲说病原)、归包口(说完一段故事,再售其货,调侃叫包口)儿、催啃(kèn,催要钱)、鬼插腿儿(先说白舍后要钱的手段)、翻钢叠杵(通过花言巧语使买主翻倍付钱)、神仙口儿、拉后门(没治好病的人回来找麻烦,用几句话把人说走)子。”

我问:“什么叫‘攥弄啃’呢?”他说:“我们江湖人管配制药品调侃儿叫攥弄啃。”我问:“这攥弄啃的法子还有什么秘密的事吗?”他说:“干这种生意一半仗着底啃(kèn),配那咳嗽药倒不是真用治咳嗽的药品,或用糊面,或用杂药末子掺点底啃。”我问:“什么叫底啃哪?”他说:“那底啃是海(hāi)草儿。”我问:“什么叫海草儿哪?”他说:“我们江湖人管大烟调侃儿叫海草儿,如若配药的时候就往里掺那东西。可是不一样,有往里掺烟灰的,有往里掺淋泥的,有往里掺生土的。”我问:“掺海草儿有什么用处?”他说:“大烟这宗东西,如遇见肚疼、心口疼、劳累过度、红白痢疾、咳嗽痰喘,抽上一口烟立刻就管事,吃什么药也没它的效力大。江湖人有四种妙药,吃下去立见神效。这四种药叫:顶药、抗药、戳药、串药。那顶药里就仗着海草儿的力量。攥弄这咳嗽药,也是和顶药一样,如有人买了去吃到肚内,准保不咳嗽,立见功效,病人哪知道这是顶药啊!只知吃着见效就是好药。可是一样不好,这种顶药全仗大烟的力量,吃的那天管事,能够不咳嗽,到了第二天大烟的力量没了,照样儿咳嗽,有些个人常买这药,吃的回数多了,能够觉悟喽,吃就见轻,不吃就见重,许是顶药吧。知识开化的人,还能猜透了药内有烟灰。”我说:“照你所说,这卖咳嗽药的多么鬼也不成,骗人就是一回,长了绝不成,管保没人照顾,这算不得高明。”

他说:“这卖咳嗽药的,也能叫人多照顾,另有妙法,能叫人多买几次,不醒腔(醒悟)。”我说:“是什么法子哪?”他说:“卖咳嗽药的配有两种药,一种是有大烟灰的,一种是没有大烟灰的。到了往外卖的时候,得瞧事行事,如果遇见初次照顾的主儿,可以卖他那有烟灰的,叫他吃了见效,好相信这药有效力。如若见了熟主顾,可不能天天卖那有烟灰的,若是天天给他有烟灰的,他吃着就能明白了,知道是顶药,就不来照顾。按着规矩,遇见熟主顾,知道他天天来买,一天给他有烟灰的,一天给他没有烟灰的,叫他吃着药这天见点轻,不大咳嗽;那天又不管事,还是咳嗽,吃了药也不管事,一定还来问,就告诉他:病有轻重,药有加减,再来一服力量大的吃下去,管保见效,这样还能多卖一倍的钱。再给他一服有烟灰的,他吃下去顶住了不咳嗽,就不疑惑是顶药,还能照顾。如若天天给他顶药吃,也能卖两回钱;若是每隔一天给一服顶药,能够卖个十回八回的也不醒腔。这样就是他们秘而不传的妙法。”我听他所说,才知道卖咳嗽药的必须得受江湖的传授,得会了攥弄啃(kèn)(配制药品调侃儿叫攥弄啃)与攥弄两样啃,才能多骗人几次,多挣人几次钱。

我问:“他们这卖咳嗽药的圆粘(nián)子(招徕观众)还与别的生意不一样吗?”他说:“敲锣鼓的生意得多招人,那叫大粘子。卖咳嗽药的用不了许多的人,那叫疙瘩(gē da)粘子。他们圆粘子之法有两样,一种是使点张子,一种是使戏头。”我说:“什么叫点张子哪?”他说:“用个一尺见方的大布摺子,画上几张五脏图、几张病图,调(diào)侃儿管那东西就叫点张子。如若要使它圆粘子,可以打开了,用手指着那图儿叫人看,向人说各种的病原与五脏的生克制化,把人吸引住了就能做生意卖药。”我说:“什么叫戏头呢?”他说:“江湖人管一种稀罕物,样式各别的东西,能够招引人看着可爱,调侃儿就叫戏头。你常见街市上有一种卖糖的,使个玻璃管招引人叫人瞧,那管里的药水就能催动那管内的小葫芦,那个东西就可以叫戏头。譬如,你说的那卖兔脑丸的,摊上摆着几个兔子脑袋,也可以叫戏头。他们要圆粘子时,一半凭口齿之能,一半凭戏头,把人招得围上了,那就算圆好了粘子。这样说吧,江湖的生意一行有一行圆粘子方法,绝不相同的。”

我说:“什么叫捋粘啃(lǖ nián kèn)条子呢?”他说:“江湖人管人有病调侃儿叫粘啃。当医生给病人粘弦(niān xián)(江湖人管大夫诊脉调侃儿叫粘弦),叫病人对他们有信仰力,就得一诊脉把病原说出来,说他是怎么得的病,病是怎样,说得对了,虽没吃药哪,听他这一说,就能相信这个大夫能把自己的病治好。当医生的要成名挣钱,得会说病原。江湖中卖药的要想挣钱,也得会说病原。他们管说病原调侃儿叫捋粘啃条子。”我说:“他们捋粘啃条子有什么用哪?”他说:“为的是叫人听着他对于咳嗽病是有研究的,那药吃了也有效的。捋粘啃条子是叫人信仰他们的能力和他们的药力。”

我问:“什么叫归包口呢?”他说:“江湖人对于他们做什么生意,由圆好粘子(聚好了观众)起,滔滔不绝,振振有词,卖弄钢口(说话的技巧和分量),一件件、一桩桩,说到了卖钱了,调(diào)侃儿叫一个包口。譬如,他们把粘子圆好啦,向围着的人说说道道的,说到了他那药卖多少钱一服,即是归了包口。”

我说:“什么叫催啃(kèn)哪?”他说:“那卖药的归了包口,他向围着的人说:‘我这药卖一毛钱一服,今天我为传名,减价一半,卖一毛钱两服,多了可不卖,只卖十服。有要的接我一张发票,接着了算有他一份,接不着算买不上,接着了也别喜欢,接不着也别烦恼。过了十服之外,再有买的,我还卖一毛钱一服,少了不卖。这就是为传名。常言道:名不去,利不来,小不去,大不来,传不出名去,不能发财。’他这样说着,那围着的人‘既在江边站,都有望景心’。他们原都听着有意思就要买哪,及至听着有便宜,买一份送一份,又有限制,过了十份就没有便宜。社会里的人好贪便宜心盛,就争先恐后地抢着买。这样抢着买可就是被江湖人用催啃的方法给催的。江湖人做生意有了催啃(催要钱)之法就能多挣钱。如若没有催啃的法子,到了做生意的时候也挣不了钱。再者说,他们到了催啃的时候,也不能固定了就卖十服,也得瞧着行事,如若围着的人多还可以说二十服哪!围着的人少也可说卖五服哪!久干这行的有了阅历,那包口是随着围看的人变化的。如若人多听着入神的少,那入神的就是买主,人多了也许说卖五服;倘若围着的人少,听着入神的倒多,也可以说这回卖十五服。总而言之,催啃的时候虽有方法,也得见机而作,死法子好学,但瞧事行事、见机而作是不容易的,可以意会,不可言传。”

我问他:“什么叫鬼插腿儿呢?”他说:“这个鬼插腿儿是江湖中的妙法,在做生意的时候,如若见围着的人听他们卖药的人所说的话全都不入神儿,预料到卖的时候也是没有人买,一腔子力气不能白费,好多的话不能白费。虽然看出没有人买他们的药,用这个方法,能叫那不买药的人也花几个钱买。鬼插腿儿的方法是强使人受骗用的,江湖人不会这个法子是不能挣钱的。”我问:“鬼插腿儿的法子是怎么使哪?”他说:“如若卖药的圆好了粘(nián)子(聚好了观众),说过去了粘啃(nián kèn)条子(讲说病原),要归包口(说完一段故事,再售其货,调侃叫包口)啦,就说:‘众位,我这药本钱很大,利是很薄,今天为了传名,我每人白送一服药,拿回家去,如若亲戚、朋友、街坊、邻居有了咳嗽病的,你给他吃了试试。倘若吃了我这药不咳嗽了,见了效啦,这是咱的药好,也别管这药里有什么。公猫母猫,拿住了老鼠那是好猫。我这药要送可不能全都送,有几种人不送:聋子不送,我说什么他全没听见,送给他也没用。哑巴不送,他也是耳朵聋,不知道我说的是什么,送给他也没用。小孩子不懂世务,药不比吃的,给了他吃出错来更糟,我是不送。那位说,你这药都送给什么人哪?我送那在家中知道孝顺父母,在外边懂得交朋友的人。今天我是固定了多了不送,只送十五份,哪位要哪位先伸手,接我一张门票,接着了算有一份,接不着没有,接着也别喜欢,接不着也别恼。’他这样说着,那围着的人贪便宜心盛,都争先恐后地接那门票,等到十五张门票全都撒完了,他可就得说:‘这种药配着不容易,众位别看轻了。前人洒土迷后人眼。有一回在一个地方有个朋友,拿了我一服药去,到了家觉着不花钱的东西扔了吧。后来他听人说我这药吃着有效力,再找那药没有了。君子人好办,小人难治。今天我送这药,有个拦避(bǎn)墙儿(前提),要说白送白吃药也不好。这么办!我是每服药收一毛钱的本儿,每服一丸,我再送一丸。如若吃着不好,把这张门票给我拿回来,一毛钱退给你,另外还赔车钱。哪位吃好了给我传名,如若没接着门票的要买,可卖两毛一服。’这样说法叫鬼插腿儿,不知不觉地,十五个人就卖一元五毛。要没这种传授,插不进腿去,不用说一元五,一毛五也卖不了啊!”我听他说明了这鬼插腿儿的妙法,感觉着江湖人对于骗取人的钱财,是迎合社会上人爱贪便宜的心理而研究出来的种种方法,使人钻入他们的圈,上他们的当。可见上江湖人的当都是贪便宜的人,这也是社会中的缩影啊!

我又问他:“什么叫翻钢叠杵哪?”他说:“翻钢是一档子事,叠杵又是一档子事。可是翻不了钢,也叠不上杵。社会里不论是哪一行儿,要到了有主顾上门的时候,都愿主顾多花钱,多买柜上的东西。可是,别的行当虽有这样的心理,至于多花钱不多花钱,全都是由那买主,不能强逼着多花钱。惟有江湖人,不论是做什么生意,对于挣钱的事都有研究,能够有准挣钱的把握。江湖人管这准能挣钱的方法,调(diào)侃儿叫杵门子。如若本领高的使用他们的杵门子的时候,还能瞧势行事。譬如这卖咳嗽药的,来了一个人问他:‘你这药怎么卖的?’他说一毛钱一服。人家掏出钱来说:‘你给我来一服。’他手中给人包好,两只眼睛可看着人家的钱,如若见买药的人钱不多,就卖他一毛钱完事;如若见买药的人带的钱多,当时要多挣他几个钱,就问人家:‘你这药是自己用还是别人用?’买药的说:‘是自己用。’他就问:‘你这咳嗽有多少日子哪?’买主说:‘两个多月了。’他就说:‘两个多月,得吃十几服才好,又多花钱,又多耽误日期,又多受罪。你买一服双加料的吧,两丸准能好,两天就保你除根,永不再犯。’买主说:‘双加料的比这一毛一服的好吗?’他说:‘病有轻重,药有加减。这药本贵,没有钱的人吃不起,要不有钱的人得了病怎么好得快哪!舍得多花钱吃好药,一毛钱一服的,净治咳嗽,加料的药能补气。像你这个年岁,面上这样颜色,是气虚咳嗽,吃上这双料的药又补气,又润肺,两丸子吃下去,把气补足了,再也不咳嗽了。吃那药得二十多天才能好,双加料的吃两丸准能好。’买主说:‘双加料的卖多少钱哪?’他这一问价就算成功了,这些话没白费。江湖人管说话调(diào)侃儿叫团(tuǎn)钢儿。用几句话叫人多花钱,这几句话的意义调侃儿叫翻钢。如若翻钢成了功,就能叠杵。我再说这叠杵之法。‘我这双加料的药卖八毛钱。’买主说来一丸子,他以为两丸子算一服,花八毛钱买两丸子。卖药的将两丸子药包好喽,到了给钱的时候,就说:‘八毛钱一服,胜似那不好的十服,每服一丸,两丸子两服,才花一元六,两天就好啦!这就是有钱的好处。’那买主若忠厚就不争竞了,一服一丸就一丸,多花几毛就多花几毛。如若买主不大忠厚,说:‘不是两丸子一服吗?’少不得多费几句话,还得给一元六。江湖人翻钢叠杵(通过花言巧语使买主翻倍付钱),就由一毛钱绕搭人家,多卖一元五。若是不会翻钢叠杵的,那只好买一毛钱的卖一毛钱的吧。”

我把他说的翻钢叠杵的事听明了,才想起,有一次我家小孩有病,往某大药铺买牛黄解毒丸,那站柜的伙计和我说了几句话,叫我改买牛黄清心丸,由几分钱改了几毛钱不算,他叫我买两丸子,我以为两丸子是三毛,结果不是,三毛钱一丸子。我有心买一丸子共三毛钱吧,他说:“买两丸,早晨吃一丸子,晚上吃一丸子。”我没法,花六毛钱吧!直到我懂得翻钢叠杵的事儿,才知道他们大汉壶瓤子(江湖人管卖生熟药的大药铺调[diào]侃儿叫汉壶瓤子)也翻钢叠杵。难怪某大药店的规矩:哪个伙计哪天卖的流水多,格外有花红哪!那么多分他几个钱,就是伙计叠杵的特别待遇呀。我联想到看《济公传》小说。济公叫人往药铺买良心,药铺伙计说他们没有良心,买主说,少买点,他们说,一点良心也没有。按书上是演义,其实并不演义呀!我们的街坊老太太每逢到药铺去抓药,一进门把药方往柜上一放,先不买,先叫伙计按她的方子给算算多少钱,算完了她才抓哪。我总嫌她麻烦,抓完了再算不一样吗?敢情抓完了再算真不一样,你虽感觉着贵呀,也不能抓好了再叫人退回去。我们街坊的老太太她就是能预防叠杵的,她也是饱经世务,多了阅历,少上当啊!

最后,我问那江湖朋友:“什么叫神仙口儿?”他说:“江湖人管向人说大话,夸张其词,能使人相信了的话语,调侃儿叫神仙口儿。譬如,卖咳嗽药的人向围着的人说:‘我这药专治咳嗽,不论远年近日的,吃了这药准保好。如若吃着不好回来找我,原钱退回,另外还赔车钱。哪位吃着我的药不见好,不来找我退钱,那算你怕我。’这样说的话就是神仙口儿。”我问他:“怎么我见有那卖药的向围着他的人说,这不是那路劫卖药的,传真方,卖假药,如若蒙哄人,男盗女娼。这样起誓发愿的话,要调侃儿叫什么哪?”他说:“这样的说话调侃儿叫劈雷子。”

我和那位江湖朋友谈了半日,听说的只是挑(tiǎo)顿(dūn)子汉儿的内幕,虽没把个中的事探讨尽了,我将所得来的写了出来贡献于社会。望各界人士将我所说的作为谈话的料儿,没事常谈,也可以叫不知道的人们少上当,少受骗。





三不管的花柳座子(治性病的屋子)


天津那个地方,在民初与十五年以前,娼家是极其发达。在河东东天仙一带,河北窑洼一带、北开一带、西头等处、各国租界里,上至班子,下至老妈堂,家家都很茂盛。此外,河北三条石还有个落马湖,没到过那个地方的,都以为多么神秘,其实那落马湖是几条极窄的小胡同,有些个矮小的屋子,点着阴阴惨惨的灯,屋中坐着那和鬼的模样差不多的妓女。门前有龟奴不住嘴地吆喝。还有些人接连不断地去逛,那是人间地狱!说起来真是惨之已极!可是那花柳病都是从那里来的,就是我说的这些地方传染出来的。娼窑既多,花柳病也就闹得厉害。那个地方是个工商劳动的区域,没有家眷的人很多,嫖娼宿妓得了病找谁去治?大医院虽有,那势派,知识幼稚的人都不敢去,只有经各处寻找大夫,三不管最为适宜。

有两种花柳座子,一种是租赁了屋子,门内摆放些个瓶子,内装药水,门前挂个布幌(huǎng)子,上画一个毒蛇盘绕着一个人,周身皆烂,上写“专治花柳,管保除根”。门上的玻璃写着“包治杨梅大疮,鱼口便毒,入骨毒串,升天落地,杨梅落后,定期保好,不愈退洋”。这种买卖叫做洋汉座子(卖西药的屋子)。还有个人,每逢游人盛多之时,在门前讲说花柳病,那染病的老乡们听他们说得很近情理,就能叫他们调治。进到屋内,钱少了来瓶药水,钱多了扎针六〇六,可是他们那药水,喝下去当日就见轻,病人一定相信,一瓶一瓶地买吧,喝下去几瓶也好不了,日久了病人才觉着喝下药水去就见轻,不喝就重。这种顶药,据我探讨是他们用西药房的会典所制,我老云对于西医是不通,西药是不懂,至于此种药有无害处,不得而知,只知道是顶药,治不好病的。至于给人扎六〇六的手术,多是不精,扎坏了的人可就多了。庸医杀人,信不诬也。还有那门前写着“××堂专治花柳,管保除根”的,做这种中药的生意是满街上贴海报,各厕所贴海报。门前不讲演的,都是指着海报的力量找买卖,老虎吃鹿——坐等儿。他们那海报还印着什么“杨梅入骨,七天保好”,“五淋白浊,当日保好”,“升天落地,管保除根”,“不熏不顶不断后”的话语,还有印着“假药骗人,男盗女娼”的字样。

花柳座子门内摆放些个瓶子,内装药水,门前挂个布幌(huǎng)子,上写:“专治花柳,管保除根。”



敝友李君在津某租界洋行服务,他是孤身一人在津,性好冶游,一时不慎,染有淋症,起初还扎挣不治,后来闹得重了,面黄肌瘦,不能做事,他请了病假,往三不管(天津市南市的一个露天市场)游逛,见了某花柳座子(治性病的屋子)门前有“五淋白浊,当日保好”的字样,当时购丸药两服,归寓服下,次日即能止淋,喜于有效。两丸只服其一,那一丸还没服哪,腿腋间立即肿起,疼痛难忍,他知道淋症见效,转成鱼口,忙着去找该堂主人,据他说是毒气过重,必须服追毒丸将毒气追出才无事。敝友李君年轻没有阅历,听他所说的种种理由,信以为真,又用洋两元购追毒丸一服,归寓服下之后,觉得有尿,但是撒尿时尿管痛如刀割,满头是汗。用灯照看,尿中有血块,愈发地相信,料是毒已逼出。三二日间,鱼口已消,复旧如初,淋病也渐愈,饮食增加,一星期后就能服务,从此无事。不料转年春天觉着胸间微痛,疑为劳累所致,不意毒气复发,个月之后,周身骨节疼痛,两足行路艰难,脚后跟不能着地。向人谈论,都说他是梅毒入骨,当初染花柳病时,未将毒气去尽,到了春天应当吃一剂大败毒,他也未用,才闹得毒气入骨。李君认为某堂主人的药当初没把毒治尽,复至某堂向其主人理论,心想叫他赔偿损失。不料经该主人卖弄钢口(说话的技巧和分量),没要上损失费,又花洋两元,购买搜毒丸一服,只有绿豆粒大小的七个小红丸,服下去之后,翻肠倒肚,上吐下泻,闹了一日,若不是壮年人,就许一命归阴,至夜内才止了,不吐不泻,劳累得四肢无力,一觉睡醒,口内肿起,满口牙齿无不活动,立即醒悟,某堂主人曾嘱咐张口睡觉,不然闷了口,牙齿活动,牙床红肿,他吐泻的力量难支竟自忘了,一觉醒来,竟受闷口之灾。治病未见效,四五日之间竟掉去七八个牙齿,幸而现时有镶牙馆可以镶补,不然饮食艰难,竟受半生之苦。经那次吐泻之后,骨节也不疼痛,行动如旧,又能做事了。过了一年又逢春天,迎头在中药商店买服大败毒汤,蛤蟆、蜈蚣、蝎子、金银花、当归尾、蝉蜕、僵蚕、天花粉,熬了一大锅,不用说往下喝,看着都怕人。喝下去之后才能不犯,春天无事。到了冬天又闹毒串,不是左胳膊疼,就是右腿疼,这毒气串在哪里哪里疼痛,他又支持不了,虽没七擒孟获,可是四次又找到某堂,该堂主人又卖弄钢口,卖他七丸药,吃了也没好,又花去大洋三元。后有某友给他配了一服熏药,是七包药末,叫他熏治。用法:粗大碗一个,用炭末烧着,使厚纸围住碗口,上卷成尖小口儿,将药末洒于炭上,从尖口上冒出烟来,用鼻子吸入。每日如此熏吸一次,七次熏完。每逢睡觉时口含木棍一根,以防闷口。不料李君熏至第四次,夜内周身皆青,被毒气侵入,一命呜呼。那送他熏药的友人也闻风而逃。可怜李君有母,只此一子,由八岁入学至二十二岁中学毕业,学有打字的技能,娶有媳妇,经人介绍在津服务,遇友不良,每夜冶游,染有花柳,一误于不择良医,二误于服顶药,再误于毒药,被友人所制熏药熏死。少年无知,也可恨也可怜矣!抛其父母妻子,至为可惨!

我老云自从李君故后,虽有云游天下之志,不敢去游烟花柳巷,更愿探讨花柳病何处能有良医良药,不能误人,广为介绍,以免染花柳病之人受庸医之害。探讨多年,始知卖花柳病药之秘密的黑幕。今将老云探讨得的种种情形,写出来贡献给阅者,更愿阅者传播于众,免受他人之愚而误终身。

有老江湖人对我说,花柳座子(治性病的屋子)这种生意也分前后棚。前棚生意是在游人最多的时候,在自己铺子旁边放个案子,铺块毯子,用“点张子”圆粘(nián)子(招徕观众)。什么叫“点张子”哪?就是尺数来宽的白布,长了可有十数丈,做成布折子。每一折是两面,共有十二面,上边画成小人,或是画长梅毒,或是画长鱼口的,画成十二样花柳病图,这种东西就叫点张子。他们前棚做生意的时候,就用手指着点张子上的图儿招引人,把人引得围满了,算是圆好了粘子,再向观众讲说,各样花柳病是怎样得的?应当怎么治?调(diào)侃儿叫“捋花啃(kèn)条子”。凡是长过花柳病的人,以及正闹花柳病的人,都得听着入耳,觉着他们对于花柳科是有研究的,是有好法子能够治好的。等到人散了的时候,进到他那屋中求他诊治。他们花柳座子的人做前棚生意,捋花啃条子,就是给自己宣传,往屋内叫病人。及至把病人叫下来到了他们的屋内,挣下钱来与挣不下钱来,那就凭他们后棚的本领了。后棚的能耐好的人遇见病人,不怕病人没心叫他们给治,没心花钱买他们的药,是和他们打听打听治法,只要经他一说,立刻能叫他们治,也愿意花钱买他们的药了。病人信服他们,就是仗着他那“神仙口儿”。阅者诸君若问什么叫神仙口儿,这也有好几种分别,有把神仙口儿用在幌幌(huàng)(江湖人管往墙上贴的广告调侃儿叫幌幌)上的,广告上印着“三天保好,不效退洋”这八个字,就是神仙口儿。如若谁有花柳病,冲这八个字就敢叫他们给治,心里还想着:我这花柳病准得好了,××堂的广告上印着哪,三天保好,他治不好,不效退洋,他们一定有拿手,不然也不敢写那大的口气,反正他治不好把钱照样退还哪。及至到了他们那里买了药,向他们问:“你这里的药是准保好吗?治不好退钱吗?”他们就说:“是这样。可是吃了我们这药可得忌口,只要忌住了口,一定能好,不好了退洋。”病人花了钱,放心回家。倘吃了药不好,找他们退钱,他们是不退的,还有话说,还有理由,反倒责备病人你吃了我这药没忌住口,你这几天吃了发物啦,我这药便没有效力,这样我不能退给钱。老江湖人说,他们这种措词调侃儿叫“抽撤口儿”(即是退身步儿),我老云所说的这抽撤口儿只是吃了发物,以没忌住口为措词,其实他们的抽撤口儿不仅是这一样,有个几千样哪,不论哪样也是强词夺理,矫情话儿,其用意是不“倒杵”(江湖人管挣到手的钱又叫人家给要回去,行话叫倒杵。可是做生意最怕倒杵,如若没倒杵,还好;倘若叫人真倒了杵去,同行人都以为莫大之耻,互相讥诮,某人叫人倒了杵了)。做花柳座子(治性病的屋子)的人有把神仙口儿用在“抽撤”上的。什么叫“抽撤”哪?他们管包药使用的发票调(diào)侃儿叫“抽撤”。那发票上也印着“三天保好,不效退洋”的字样,其用意叫买主放心而已。还有那患花柳病的人,欲治又怕治不好,不治病又难受,在这犹疑不决的时候,也许一狠心不治了。可是他们做这种生意的人对于这犹疑不决的病人,就施用神仙口儿说:“你只管治吧。这不是摊子,这天在这里摆,明天不来了。门面字号,也跑不了。治不好,第四天你来,把你的原钱退回。”病人听了,就放心大胆地把几块大洋给了他们。及至钱到了他们手内,如入虎口,立刻就说:“你吃了这药可得忌口,吃不得发物,忌房事。如若忌住了,你的病就好啦;倘若忌不住,你可是白吃药,白受罪,好不了病的。”病人以为吃药忌口是医药行的概例,信而不疑。总想不到这些话是他们的退身步、抽撤口儿。

可也有些人吃他们的药能把花柳病治好的。据我调查的情形也有分别,有两种药能把人的花柳病治好。一种是顶药,一种是猛烈药。那顶药如同有瘾的人抽大烟,吸点就好,不吸就受不了一样。那猛烈性的药说起来也真怕人,就以那上吐下泻的小红药丸说吧,那种药要叫儒医去配,吓死他们也不敢给人吃的。那种药是什么东西制的,至于那么厉害?说起来这种药是中国的中药店都有,名叫“红升丹”。据我向医药界人打听,说:“这红升丹是硝石等烈药,按着丹药用炉烧制的,炉底上片,片上是末,这种东西是治疗毒恶疮使用的。如若疮上有了烂肉,上了这药能治得全像水一般顺疮口流出。那红升丹的末儿力量小点,红升丹的片儿(又叫红粉片)力量还大,也不知哪位高明先生把这药研究得能治花柳,用个不到一钱多重,使枣泥搓成丸子,像黄豆粒大小,只要吃下去,这药到了人的肚子里,行开了药性,翻肠倒肚,搅肠疼痛,把人弄得上吐下泻,多足壮的人也受不了。可也奇怪,如若染上花柳病的,小便胀烂,入骨毒串,吃下去受一回人罪,五六天工夫,就能好病。”我曾问过他们,为什么使这种药给人治病?他们还有理,说是以毒攻毒。凡是儒学的医生都是胆大心细,用药查性,辨天时气候,对症下药,他们哪敢用治恶疮的红升丹给人治花柳病啊!我老云对于用这种药的人,总是替他们捏一把汗,怕把病人治死。这种药吃下去,都得闷口,毁坏牙齿。如若有染花柳病的人,买了药吃下去,上吐下泻闷了口,就是这红粉片制的药了。

还有一种不吐不泻的药,可是日子慢些,有花柳病的人服了那药,得过一个星期后才能有效,还不论是升天落地、杨梅落后、杨梅入骨,只要是花柳病,吃下去就好。病虽可好,但有一种缺德的坏处,即那药能断后。凡是吃过那药的人永远不能有后,不能生儿女,断绝宗祧(tiāo),罪大已极,图一时之利,贻人终身难除之害,实是与阴功有亏。病人不知,定受其愚。我为了此事探讨他们的黑幕,将他们的内幕揭穿了公诸社会,使社会里的人们免受其害,我自己奖励一句:也是我的好处啊!那么,那治花柳病的人们是用什么东西配的断后药哪?那药虽是几种药制成,或是十几种药制成的,只有一种药不应当用,用了断后;可是没有那一种药,吃下去又没有效力,又治不好花柳病。阅者若问这一种药是什么,说起来也是治恶疮往下治烂肉的药品,这种药中药商店都有卖的,叫做轻粉。这轻粉是由南省来的,大约是汉口货,用竹桶装着,两元钱里外就能买一小桶儿,还不算很贵。可是里边有一半假的,原桶来时就有假。我和药行的人研究,这药里的假东西是生石膏弄的,真假有个分别:真轻粉有亮光,又白又薄,如雪花一般;那假的是碎块儿,没有亮光。我向药行人探讨这轻粉是什么东西制造的,据药行人谈:“轻粉是水银的原料,用矾升化的。”那水银的毒质最大,虽经炼冶,治疮去烂肉生新肉即可;若是吃在肚内,岂不断后?怎么知道他们卖的药里有轻粉哪?试验此物惟一不可的法子,只要是吃了花柳药不吐不泻,也闷口毁人的牙齿,那药里就是有轻粉的了。

这两种药虽然闷口,断人子嗣,还不至于要命。还有一种花柳药能够要人的性命。会配这要命的花柳药的人还是很多,不止于卖花柳药的人。凡是染过花柳病的人与吃娼窑饭的人,只要见谁有治花柳病的药方子,立刻就要过去,抄写下来,写在一个小折子上。如若有人得了花柳病,他就把折子取出来,叫人往药铺按着折上的方子给抓药。像这样逞能的人很多很多,真是愚人好自用。只要病人吃了他那药,误而愈,他便夸示他那好药方;如若吃坏了或是吃死了,他一跺脚,两眼发直,出身透汗了事。这种现象我老云可就看多了。医生治病,是一样的病都不能用一样的药。因为病有轻重,人有强弱,药有加减。春、夏、秋、冬四时的气候,用药俱是不同,绝没有不加减,不分四时,不管病人强弱都是一个药方的。好给人治花柳病的人若明白此理,就不多管闲事了。可是有花柳病的人也千万别信不懂药性的折子式的先生才好。

最可怕的是一种熏药。若配的时候也得用十几种或七八种药,内中的主药就是一种,水银。据药行人说:“那种配熏药的水银是用铅炼了的,其毒质害人与不害人,就在那水银的制炼得优劣而分。炼得得法,佐了群药,也都相宜了,才能不害命,可是也得闷口。如若那水银制炼得不得法,配的群药不相宜,熏上就有性命之忧。”那熏药据我见过的有两种:一种是药末,用炭去熏,往鼻子里闻。怕药味吸入口内,嘴里还得含一口水,才能避免药味不入咽喉。还有一种用香面子调和匀了,制成小窝头形的,把它放干了,用时用火点着了往鼻子里熏。嘴内也含一口水,避免药味吸入嗓子之内。这种药用水银为主,其害较比轻粉还大,熏了之后,就不害性命也是断后,绝了子嗣。我老云把这些个害处说明了,望阅者诸君在茶余酒后和朋友们多谈这些事,或可减少染花柳病的人,少受这些害处。





老云再为染花柳病的人们进一忠言


前几天我谈了一回治花柳病的花柳座子(治性病的屋子),把其中的黑幕揭穿了,并且还说了爱多管闲事爱逞能的人,他把治花柳的药方写在折子上,如遇了染花柳病的人,他就把折子取出来,叫人吃他那种药,吃坏了的多,吃好了的那是家中的德行。不料我说了不久,就接连不断地出了好几档子吃花柳药害了性命的。据《时言报》十月四日载:平东公主坟住有刘克勋,在河南做事,因冶游得了花柳病,回家调治,有孙某按折子上的药方,专为人调治花柳。刘克勋服了他的药,上吐下泻,咽喉肿痛,三日不进饮食而亡。虽然验尸埋了,他们的官司还没解决哪。这种事看着有多危险!那个折子先生虽然没有害人的心,可是那药把人治死了。虽然他有应得之罪,已死的人也是粗心,选择不慎哪。望各界的人士关于这类事努力宣传,使染花柳病的人有所警惕,有了病找官府考试正取的花柳科大夫调治,千万别用折子先生们的成方,免得出了舛错,也是爱护的善意,宣传此事也有功德呀。望阅者茶余酒后多谈这种事,使没受过害的人有了戒心才好。





挑(tiǎo)柳驼儿


“柳”是唱,如唱戏,唱曲之类皆是“柳门”的生意。关于戏剧,有不少评剧家探讨梨园行事迹,在报上分门别类都披露过了,敝人不便多谈,今将“柳门”里的生意“挑柳驼儿”的,与阅者诸君谈谈。

什么叫“挑柳驼儿”的?就是在各市场、庙会假装唱戏卖膏药的。做这种生意,必须懂得梨园行的规矩,要不然可吃不开。在北平这个地方,做柳驼儿的生意人勿用和梨园行联络;若是在乡村镇市“顶凑子”(即是赶集的侃语)或是“顶神凑子”(管香火庙会调[diào]侃儿叫神凑子),非得和戏班联络不成的。

每逢农人普收的年头,乡镇中的会首们都临时凑款,写台大戏谢神。或唱三天,或唱五日,最多不过一个月。到乡下唱神戏班儿俗称叫跑野台子。跑野台的戏班里都有个外老板,专司往乡镇联络会首们写戏。戏班到了乡镇,不管班里有多少角儿,管住处,管吃食,管灯。可是管生不管熟,管灯不管油。戏班到了,将粮、米、茶、炭、灯在下处预备好,就全不管啦。戏班里跑野台子的时候,班里没有准人数,不论是谁,只要懂得梨园行的规矩,到后台冲祖师爷磕完了头,菜饭得了,抄起来就吃,那同行的义气较比大都市的戏班儿还强呢!

做柳驼儿的把膏药预备好喽,小包袱往身上一背,随着戏班儿打走马穴(xué)(走一处,不能长占,总是换地方挣钱,江湖人叫走马穴)做买卖,也不用住店,戏班的下处里住着,还不用花钱到饭店里买饭吃。戏班的饭得了,白吃白喝。和后台的老板们联络好了,每逢散戏的时候,做柳驼儿的戴个秦叔宝的帽子,拿把单刀从后台跑出来,说:“别走!我还有一出哪!”把听戏的人们叫住了,他从台上跳下去,在台前头圆粘(nián)子(招徕观众)卖膏药,所用的手段都是“鬼插腿儿”(江湖人管先说白舍后要钱叫鬼插腿儿)。

据我所知道的,挑(tiǎo)柳驼儿的最有名儿的叫袁桂林。如今在平、津等处做柳驼儿的生意人,都是他支派传流下来的。按照老江湖艺人流传下来的生意行当,饶能挣钱,还不鼓点(受骗的人明白了,和他们翻了脸,调[diào]侃儿叫鼓点),也不能卯(江湖人管被军警机关取缔调侃儿叫卯)他。

如今江湖乱道,入了生意行,只要有能挣几毛钱的能耐,不等着样样学会了,就抛了“老帅”,“荣扯”(管老师叫老帅,管偷着跑了叫荣扯)喽!不惟社会里风俗日下,就是江湖的艺人也都江湖乱道了。





第四章 杂技戏法


彩门


“彩”是“彩立子(lì zi)”。凡是变戏法的行当,皆称为“彩立子”。在这彩门里尚有种种的分别:变戏法叫“彩立子”;变戏法儿带赞武功叫“签子”;卖戏法的叫“挑(tiǎo)厨供(gòng)”的;变洋戏法的又叫“色(shǎi)唐立子”;什么人头蜘蛛啦,人头讲话啦,山精海怪啦,统称为“腥棚(假的)”。管上台变大戏法儿叫“落(lào)活”,又叫“卸活”;管变小戏法儿叫“抹(mǒ)子活”(也叫单色儿立子);管做堂会叫“家档子”;管变戏法儿变露了像儿叫“抛了活”;变戏法的管使用家伙上有鬼儿的法子叫“门子”。其余的所变的各种戏法儿也都有侃儿:管变仙人摘豆叫“苗子”;管变壶中有酒叫“拉拉山”;管变杯中生莲叫“碰花子”;管变罗圈当当叫“照子”;管变大海碗叫“揪子”;管吞剑叫“抿(mǐn)青子”;管吞铁球叫“滚子”;管变菜刀叫“大腥”(特别假)。种种的戏法儿,皆有侃子。在江湖艺人中规矩最严的行当,如今就是“彩立子”这一行了。





彩门中之挑厨供的


变戏法儿这一行儿,自从有这行直到清末庚子年前,只有变戏法的,还没有卖戏法的。据他们彩立子(lì zi)行(变戏法的)人所谈,在庚子年后才有“挑厨供”的。在东安市场将开办的时候,有个“厨供杨”在东安市场卖仙人点戏,由其收徒,传流此艺;现今华北各省市、各商埠码头,皆有厨供杨支派的门人做“挑厨供”的买卖了。

剑、丹、豆、环,不算戏法,那算是真功夫。吞宝剑受几个月的苦处才能学好。



挑(tiǎo)厨供(gòng)的(卖戏法的)与彩立子(lì zi)行(变戏法的)之规矩。变戏法的人,只要能会变,不拘大小,什么戏法都许变,是无人阻拦的。卖戏法的可就不同了,他们做买卖必须使高案子,不能打地摊儿,变的戏法儿不能变抹(mǒ)子活儿(小戏法)。所变的都是什么仙人摘豆、三仙归洞、金钱抱柱、破扇还原、金钱搭桥、巧变金钱、棒打金钱、霸王卸甲、仙人解帕、空盒变烟、空盒变洋火、巧变鸡蛋、平地砸杯、巧变烟卷、木棍自起等等的戏法。这些戏法儿除去仙人摘豆、三仙归洞、平地砸杯、破扇还原等彩立子行常用,其余的戏法儿,变戏法的人们也不常使用。卖戏法儿的人们不准变的戏法,如罗圈当当、大海碗、吞宝剑、吞铁球、八仙过海、扇碟扇碗、八仙对果、大变酒席、巧变火炉、巧变黄酒、五子夺魁、寿桃寿面、九龙闹海、十二连桥、十三太保、巧变珠灯、九莲灯、巧捕家雀、滴水成冰、冰开献鱼、海底捞月、封侯挂印、杯中生莲、口内喷火、口内生莲、飞鼠盗粮、火内套彩等等,这些戏法不惟不准他们变,并且还是不准往外挑(tiǎo)(即是不准卖的)。戏法儿原就是假的。变戏法儿的使的是门子(变戏法的管使用家伙上有鬼儿的法子叫门子),卖戏法儿的所卖的种种玩艺儿完全都是腥活(假的),他们要把真门子都给卖了啊,变戏法的就不用变了。江湖艺人所作的买卖,行行儿都有规矩,并且还能遵守,这样还值得人们钦佩的!

挑(tiǎo)厨供(gòng)(卖戏法的)的前棚。卖戏法的艺人投师受业,学的是前后棚的能耐。什么叫前棚的能耐呢?哪叫后棚的本事哪?前棚的能耐分为“圆粘(nián)儿”(招徕观众)、“拴马桩儿”(把观众吸引住,不让走)、“卖弄活儿”(花言巧语)、“撒幅(sǎ fú)子”(分发门票,相当于今天的优惠券)、“把点儿”(找准蒙骗对象)。

挑厨供的圆粘儿种种。卖戏法的都是支个大案子,后边以靠墙的为美,墙上可以挂布摆子。那布摆子上写的是××堂,多是××魔术团几个大字,两旁的小字是“传授戏法,当时管会”。底下写的是手法门戏法:仙人摘豆、三仙归洞、仙人解帕、巧解丝绦、破纸还原、棒打金钱、霸王卸甲、飞钱不见、月下传丹;彩门戏法:空盒变烟、空盒变洋火、巧变鸡蛋、平地砸杯、破扇还原、金钱搭桥、金钱抱柱、木棍自起;药法门戏法:茶能变墨、一杯醉倒、千杯不醉、活捉家雀、美女脱衣、飞豆打蝇、口内喷火;符法门戏法:八仙转桌、大搬运、抽签叫点、牌九骰(tóu)子(骰子俗称色[shǎi]子)、打麻雀、黑红宝。别看他这摊子上写的是戏法儿应有尽有,样样俱全,不许卖的还是不卖。摊子上越写样数越多越好,如同买卖的牌子写“中外杂货一概俱全,零整批发,不误主顾”,意义是相同的。他们的案子上放个万宝囊的匣子,万宝囊的袋子,两个茶杯。到各市场开始游逛的时候,他要做买卖啦。先用两只手托着茶杯对撞,撞得那杯当当当地直响,嘴里叨叨念念的,先变个三仙归洞啊,或是变那仙人摘豆,游逛的人们渐渐地围着观瞧,他瞧着围着的人够多了,算是“圆”好喽“粘”啦,他可变不了多少戏法了,在这个时候就该着卖弄活了。

他们卖弄活的意思是说:“我们所卖的各种戏法儿人人可学,当时就会,不拘男女,军商各界,要是学会了几手戏法儿,回到家内,可以打个哈哈,凑个趣儿。您要学会这一杯醉倒吧,是手儿药法门的戏法,要是自己好喝酒,有那爱吃您便宜酒的人,你干有气,碍着情面没法子治他。要学会了我这一杯醉倒的戏法儿,只要你把这种药放在酒内,他喝下去一杯,到不了一袋烟的工夫,准把他醉倒喽!还有一手戏法,也是药法门的,叫仙人脱衣。只要你把这种药藏在指甲盖内,用的时候,在他身后头悄悄地把药往他脖领儿里一弹,抽不完一个洋烟的工夫,管保他刺痒难挨,立刻就脱衣裳。还有一手戏法,叫活捉家雀,你要到鸟儿市买个鸟儿得花多少钱?把我这手戏法学会了,只要瞧见家雀儿落在你的房上或是院内,有多少拿多少。还有一手戏法叫小鬼叫门,你把这戏法学会了,跟谁玩笑,能叫他一宿也睡不着觉,总听着门外有人叫门,打得街门啪啪直响。这两手儿也是药法儿,哪位要学容易。”说到这里他就拿起一沓儿幅子来,叫大家瞧上半张字,幅子上印着××堂字样,有“专教戏法,当面管会,如若不灵,准保退钱”的字儿。他拿着这沓儿幅子,向围着的人说:“哪位要学这四手戏法儿,我这幅子上印着呢,只要认识字,一看就会,不认识字找认识字的念给你听,一念就会。哪位要学这四手儿戏法,您给一毛钱!一毛钱学四手戏法不算贵吧?今天这么办,我是张天师卖眼药——舍手传名,哪位要学这四手戏法,我要半毛钱……爽性豁给众位一个便宜,半毛钱也不要,谁要学给四大枚,才合一大枚一手儿。可是全都买我可不卖,就卖二十份,过了二十份之外,再有学的还卖一毛钱。把话说在头里,也许你不学,也许我不卖。哪位要哪位伸手,先掏钱后接门票。”(江湖人管他们先说大价儿然后往下落价儿的法子叫海[hāi]开减卖,亦是引诱人贪便宜的方法)于是就有许多人掏钱买他的门票。这是他们卖弄活儿的意义。往外卖他们的门票,调(diào)侃儿叫“挑(tiǎo)幅子”。他们挣钱的方法,最小的意思不过如此。那挣大钱的方法,几千元几百元的能耐,都在“后棚”哪,要能“挑(tiǎo)雨头字”(卖符咒的)才有大钱挣呢!挑雨头字的事儿,在谈“后棚”的时候再为详谈。

敝人曾买过他们一张门票,上边印着四手儿戏法子,现在把那法子写出来贡献给阅者(以下是幅子上印的字样):“‘一杯醉倒’,用钱到药铺去买闹杨花少许,研成末儿,放在酒内即成。”敝人曾于民国四五年在津埠向某名医学过医道,对于药性稍有一知半解。凡是到药铺单买闹杨花、巴豆、红矾、大戟、芜花等等药品,药铺的商人准是不卖的。因为这些药毒质甚大,若是用之不当,最能害人,甚至于有性命之忧,所以闹杨花是买不着的。“‘仙人脱衣’,药铺内买细辛一大枚,用其毛儿,如用桃毛也可,用时弹在脖领内。‘活捉家雀’,用酒浸小米儿数次,晒干了,撒在地上,鸟儿食之醉,不能飞了。‘小鬼叫门’,用钱到药铺买胆南星数枚,研为细末,用醋打成面糊,抹于门上,夜间当作啪啪之声。”以上是卖戏法儿的所卖的药法门四种戏法。综观上言,这四手戏法儿俱是骗人。“一杯醉倒”买不到闹杨花;“仙人脱衣”用桃毛,何必问他;“活捉家雀”敝人试过不灵,捕鸟儿的人们用笼用网也不甚难;“小鬼叫门”,胆南星药铺虽卖,也试不灵。阅者要问:你何不找卖戏法自去“倒(dǎo)杵”(往回拿钱)啊?人家直说世上的事儿是没君子不养艺人,为要四大枚,谁能去当小人?斗气也不值哟!

卖戏法的后棚。卖戏法的艺人投师访友,学习前棚的能耐最易,要学后棚的能耐那可就难了。前棚的能耐,任你学得多好,只能挣个店饭钱,绝不能“火穴大转(zhuàn)”(江湖的艺人有能耐至某处挣了大钱,调[diào]侃儿叫火穴大转)。有许多卖戏法的艺人,就是会前棚的能耐不会后棚的能耐的。也有天生愚鲁学而不成的,也有师傅心独不肯传给他们的。要学后棚的能耐,一半得有天赋的聪明,一半得受师傅的真传授才能成的。前棚的能耐好,挣钱块数八角;后棚的能耐好,挣钱花不了。

今将卖戏法的后棚挣钱方法贡献阅者。他们后棚的能耐分为数种:一是“把(bǎ)点水火”(窥测买主的虚实);二是“翻钢叠杵”(通过花言巧语使买主翻倍付钱);三是“挑雨头字”;四是“使样色(yàng shǎi)”(实现以假乱真的效果);五是“平点”(施展手段让被骗的人自认倒霉)。有此五大技能才能成名,才能大转。卖戏法的艺人若是把(bǎ)点(窥测买主的虚实),必须在作前棚的时候能够把出点来。譬如卖戏法的在挑(tiǎo)完了幅子(传单)的时候,见观众围着不走,他们就拿起牛牌(一种或用纸,或用乌木,或用竹子做成的赌具)来,在案子上一个人推小牌九儿,叫观众看着回回是他手内起好牌。什么对大天哪,对大四呀,天杠啊,这种意思是向观众“亮托”(江湖艺人在场内施展他的技能,使人瞧着羡慕,调[diào]侃儿叫亮托)。在他们亮托的时候,两只眼睛得向观众瞧着,谁冲牌九出神儿谁是“点儿”(江湖中如若看谁能够生财谁就是点儿)。认出点儿来了,应该急速地把买卖推了(即是叫观众散,调侃儿叫推了)。推了买卖之后,这个点儿还是站在他的案子旁边不走,原来那是点儿听他们变牌九的时候说来着:“哪位要是赌钱输的钱太多了,可以学学我这‘叫牌法’,要是学会了叫牌法,管保把你所输的钱还能赢回来。可是学这叫牌法去赢人不成(内含着抽撤口儿,即来回说呢),输了钱往回捞成了。”譬如某甲最近因为赌钱输了钱了,听他所说的意思,一定从心内就愿意学习他的叫牌法,花钱不多,真的能把输的几百元捞回来,焉能不干?这点儿(某甲)有了这个意思,看着他变牌九能够净起好牌,天杠、对大天、对大四,一定得看着出神儿,他这一出神儿不要紧,可就叫卖戏法的把出他是点来了。某甲当着观众不好问他,很愿意看热闹的人都走开哪。在这个时候,卖戏法的不变了,观众散去了。某甲可就好向卖戏法儿的搭讪着说话了,卖戏法儿的也搭讪着跟他闲聊。两个人一接近,几句话的工夫,卖戏法的就把点儿“跨”走了(生意人把点儿带了走,调侃儿叫跨走了)。

阅者诸君要问,他们把点儿跨到哪里去了呢?凡是做这种生意的,必须在他摆场子的地方附近赁间房子,预备着后棚有了买卖,把点儿跨了来好入“窑儿”(窑儿就是他们那间房子)。点儿跟着他们到了窑内,卖戏法的必须先问点儿贵姓啊,府上哪里啊,现在哪里恭喜呀。点儿以为这些事都是社会交际场中所用的门面语,也不注意,便把自己的姓名籍贯说给他,做什么事也就随着说给他了。卖戏法的问这些事儿,是要簧(要出实话来),好知道这个人的财可生不可生,譬如这点儿告诉他们,说在某侦缉机关有个差事,卖戏法的可就不敢生他的财了。按社会中潜伏的骗子手们没有不顶“老柴”们瓜的(江湖艺人管侦缉人员调[diào]侃儿叫老柴,又叫柴把[bǎ]点,管害怕叫顶瓜)。挑(tiǎo)厨供(gòng)(卖戏法的)的生意也是骗术啊!他们没事还顶老柴家的瓜哪,哪敢敲老柴的钱哪!设若某甲是个贸易点(商人),或者是个“科郎(kē lang)点”(农人),那可就跑不了啦,用他们那“翻钢叠杵(通过花言巧语使买主翻倍付钱)”的手段了。什么叫翻钢呢?生意人为什么叫吃张口饭呢?就是凭他那张嘴儿能说会道,俗话说是“好汉出在嘴上,好马出在腿上”。他们生意行的人都是先跟师傅学会了钢口,才能做生意哪。譬如某甲跟他们商议好啦,花十元钱学他的叫牌法,把皮靴掖子掏出来,露出一沓儿洋钱票来,五元一张、十元一张的,有个几十张,他们就后悔了。这是个有钱的点儿,十元卖屈了,还想着再多要钱,立刻就翻钢儿,能把以前所说十元钱价目作废了,改为五十元。饶着他多挣了钱,点儿还很愿意。生意人管推翻了前言另作商量调侃儿叫“翻钢”,由十元改为五十元叫“叠杵”。最奇怪是他们翻钢叠杵的时候,无论如何,不叫点儿(江湖中如若看到谁能够生财谁就是点儿)醒攒(cuán)儿(管觉悟过来叫醒攒儿);如果被点儿醒了攒儿,那不是煮熟了鸭子又飞了吗?

再谈谈“挑(tiǎo)雨头字”的事儿。什么叫挑雨头字呢?卖戏法的第一挣大钱就指着卖这宗东西。在我国清末光绪的时代,社会里的人士还都迷信呢,到了民国打破了迷信,一些画符念咒的事儿才渐无人信。凡是画符咒的时候,都是有雨字头儿,像(chī)、(mèi)、(wǎng)、(liǎng)等等有雨字头儿的字儿,哪个都有雨字头吧。他们卖戏法的管卖符法的调侃儿叫“挑雨头字”。如若有点儿要向他们学习什么打牌、掷骰(tóu)、抽签、纸牌种种的玩艺儿,他就告诉点儿这些赌博的玩艺儿都是符法门,要学哪手儿也得七天的工夫。他把符画得了包在纸内,叫点儿拿了走,去天天磕头烧香上供,还得在满天的星斗出全了才成呢!到了七天的限期,把符带在身上吧,赌钱去是准赢不输。真是哄不尽的愚人,真有花个十元八元买了他这道符的,还有花五六十元的,甚至于有花几百元的。你要看着他们画的那符,还是很奇怪。用一支毛笔放在茶碗内,碗里放点凉水,用的时候他一念咒,拿起笔来蘸凉水往黄毛边纸上去写字,写得了是红的,如同是朱砂字一般,谁看着也得纳闷儿。最近敝人调查成功了,才知道其中的缘故。原来他那凉水里有毛病,用的时候悄悄往水里搁点碱末儿,那碱末儿在水内化开了,用毛笔把凉水和匀了,画在黄毛边纸上,凭那碱水的力量,就能把纸变成红颜色。这也是一种化学的方法,不知道的便以为奇罢了。这种符咒叫“水符子”。另外还有一种“火符子”,是用硫磺、焰硝和几味金石性的药品制成的。点儿是“空(kòng)子”(江湖艺人管受他们冤的人调侃叫“空子”),绝不醒悟。他们使的是“跟头包儿”。原来他们有一种方法,无论用纸包裹什么东西,叫别人当面瞧着是包在里头啦,打开再看是个空包儿。那东西在包的时候就弄在外头掩藏起来,这种“跟头包儿”他们是时常使用的,这种欺骗愚人的法子调侃儿叫做“灶点”,又叫“安瓜灶点”。

挑(tiǎo)厨供(gòng)(卖戏法的)的这行人,最有能耐的得馈(要)十几道杵(即是冤人花十几回钱)不叫点儿(江湖中如若看到谁能生财谁就是点儿)觉悟。如若点儿觉悟了,他能带上一张符叫点儿同他去赌。到了赌钱场儿,不论耍牌九啊,或打麻将呀、斗纸牌呀,能当场赢钱,饱载而归。凡是挑厨供的,都得是“老月”才能成哪!什么是“老月”呢?江湖人管吃腥赌(假装赌钱)的人调侃儿叫耍“老月”的。在江湖侃内,管十个钱数调(diào)侃儿一叫“柳”(liū),二叫“月”,三叫“汪”,四叫“载”(zhāi),五叫“中”,六叫“申”,七叫“行”(xíng),八叫“掌”,九叫“爱”,十叫“句”(jū)。为什么管腥赌的人叫“老月”呢?盖因耍腥的都是两个人使对子,在赌钱场儿叫暗令儿,江湖管俩人调侃儿叫“月点”,故称他们为“月”。社会里半开眼的人又管吃腥赌的人叫“耍俩点儿”的,也是取其二人之意。卖戏法的在赌钱场儿赢了钱回来,他向点儿说:“你看见没有?我这法子最灵无比!”赢的钱可就暗含着归了他啦!

生意人有多么可怕呀!张嘴儿,动身儿都有他们的利益。他们把点儿挖到“绝后杵”(管点儿花最末一次钱调侃儿叫绝后杵)为止,遇到了忠厚人,用同吃同嫖的手段交朋友。交了朋友,叫点儿心里虽是觉悟了,冲着交朋友的情面,不好意思和他们翻脸,只好自认倒霉。管施用这种手腕调(diào)侃儿叫做“平点儿”。如若是平不了啦,点儿(江湖中如若看到谁能生财谁就是点儿)逼着他们倒(dǎo)杵(往回拿钱),或是要打官司的时候,他们还有最后一个法子,如同说评书的先生们说拿白菊花一样,三十六策,走为上策,给你个急溜扯活(chě huo)(快跑)。再不明白,我再补充一句,就是逃之夭夭了。奉劝社会上好赌的人们,千万别上他的当,花钱惹气,耽误正事,有多么不值!

这些年因卖戏法的冤人太多,到处撞骗,也有和他们打了官司的,弄得各省市、各商埠地面上官人知道了他们种种败劣的行为,对他们这行人,不是“卯喽”,便是“淤(yū)喽”(江湖人管军警机关取缔他们调侃儿叫卯喽,把他们轰了调侃儿叫淤喽)。这些挑(tiǎo)厨供(gòng)(卖戏法的)生意的人,在这几年虽然遍地都是,因为各处不是卯喽便是淤喽,已然要不能存在了。敝人推测,这行生意还不能说已百年(寿终正寝)了。





挑厨供的卖点儿


有山东人于星五,年二十多岁,随其胞兄在大连做皮货行买卖。每逢他胞兄不在柜上之日,必往游西岗子。那西岗子露天市场较比北平的天桥、天津的三不管(天津市南市的一个露天市场)不在以下,每日锣鼓喧天,各种的杂技,各样的生意,都在那里支棚设帐,拉场子做生意,应有尽有,无不齐全,就是天天去逛,也不会腻。于星五把西岗子逛惯了,有一天他又到了西岗子,见靠油房的墙根有一圈人,挤进去一看,见里边有一张桌旁,上铺洋线毯,毯上有黑漆盒,绿豆茶杯,白绸子手绢,古铜制钱,几张乌木的牛牌,一副扑克牌,一根短小黑漆棍,那案子前边有几尺白布,上边画着各种戏法图儿。案后边站立一人,长得细条身材,白脸膛,五官清秀,三十岁里外的年纪,脑袋上留着美式分头,黑漆似的头发又光又亮,穿着一身西服,嘴里镶着金牙,很漂亮,说话是北京口音,听他所说的是卖戏法儿。于星五站在人群里看这变戏法的变了几手,真是干净利落,人人叫好,及至卖的时候,一毛钱一张,卖个三四十张。就以他卖这小戏法儿说,哪天也有个十元八元的挣项。若再有人学大戏法哪,挣个几十块钱也能成啊!

于星五看着卖戏法的能够多挣钱,实在眼馋,有心也干这行儿,只怕不易,人家不愿意传给外人。他心里存着这个意思,每逢有闲工夫就跑到西岗子看卖戏法的。他与看热闹的不同,人家是看完了一散,他是来在人前头,走在人后头,看着出神儿,脸上总有笑容。日久啦,那卖戏法的似有觉悟,见他天天来看热闹,永远不花钱,既在江边站,就有望景的心。不学小戏法,不学大戏法,天天来看,定是别有心意,也许要上跳板(也许要学变戏法)吃这行儿。于星五的心意被他看破,两个人“满怀心腹事,尽在不言中”。偶然有了闲工夫,彼此点头说几句话,渐渐地熟合了,各道姓氏、家乡、住处,于星五才知道这卖戏法的姓汪,叫汪福林,可是汪福林也知道他是于星五,在皮货行做买卖。

两个人认识了,有商量的可能,于星五预备了二三十元作为交际,特意的在日落之时往西岗子找卖戏法的汪福林,约他到街里吃饭,汪也没驳回,就同他前往。两个人到了一家饭馆,找了个雅座,要了些酒菜,随喝随聊,于星五把他的心意说明了。汪福林说:“我们这行都不收徒弟,即或收徒弟,也得选择相当的才能收哪!徒弟得给师傅挣五六年的钱才能成哪。你这个岁数,要按着规矩再学几年徒能成吗?”于星五说:“我可不能学几年徒,因为我家中有父母,我已然娶了媳妇。这些年跟着我哥哥在外边做买卖,我哥哥总看不起我,说我不能自立,事事都得依靠他。我要不吃皮货行,自己另谋求别的事。我就看着你们这行好,无拘无束,随随便便挣几个钱养家,比什么都好。我要学你们这行,叫我哥哥看看,我离开他几个月就能挣钱。这是我的心意,你怎么也得成全我才好哪。”汪福林说:“我不久还要往别的码头做生意哪,这怎么办哪?”于星五说:“你要走好办,我跟着你走,先不叫我哥哥知道,等我能挣钱了再来见他。”汪福林说:“你要跟我学这行,你怎么谢候我呢?”于星五说:“我不明白这行规矩,你告诉我呀。”汪福林说:“你至少也得给我二百块钱,我才能教给你,管保三个月后你能挣钱。”于星五说:“二百元钱我也能办到,只是现在不成,你定规好了咱们哪天走,临走的那天我能办到。”汪福林说:“就这么办吧!我要走的头几天就先给你个信儿。”两个人把事商量好喽,用完了酒饭,由于星五会了账,他二人各自回归。两三天必见一回面。有一次汪福林告诉他:“我后天走,你办得到吗?”于星五说:“能成。这几天我哥哥正没在柜上,后天早晨我就来找你。”说罢,欢天喜地地去了。

到了第三天早晨,汪福林就见于星五拿着个皮包,一块绒毯而来,他说:“我们什么时候走?”汪福林说:“这就走。你的事怎样?”于星五把皮包打开,叫他看了看里边的财物。汪福林喜形于色,立刻收拾行李,雇了两辆车奔向码头,上了火轮船等候开船。及至船开了,于星五才问他:“我们往哪里去?”汪福林说:“我们往安东去。”于是他们在船上谈谈论论,也不寂寞。及至轮船到了安东,雇车拉到三不管去住店(安东县名又沙河子,那里最热闹的市场是露天的三不管,与天津的三不管名称相同)。

于星五虽是皮货行的商人,他还是买卖人的规矩,吃喝花受了商家的习惯,不肯浪费,不敢妄用银钱。而汪福林是个久惯走江湖的,他的习惯是爱花钱,这就应了那句话了:来得容易,花得也容易;来得不明,去得也模糊。他们在店中包的是单间,每日房钱就是八毛,伙计的零钱还不算。两个人出去吃顿晚饭就花一元八角,抽烟卷都是三炮台的。于星五眼见的,口吃的,耳闻的,事事都觉着阔绰,感觉比那皮货行人大方了。他想着自己要学会了卖戏法,往后可以到处摆阔,到哪里也得受人欢迎,高兴得得意忘形,嘴里也是哼哼唧唧地唱小曲儿。

第二天汪福林往市场上做生意,就带着于星五,连着做了三天,哪天也挣十几元钱。第四天汪福林就教他四手戏法:一手是“三仙归洞”,一手是“空盒变洋火”,一手是“巧变烟卷”,一手是“仙人解帕”。于星五真聪明,一夜学会,汪福林教给他一套生意口,怎么圆粘(nián)儿(招徕观众),怎么往外卖,都教会了。他在店里不出去,叫于星五去做生意。于星五到市场做了一天生意,挣了三块多钱,回到店里都交给师哥,汪福林不住嘴地夸奖他。他第二天又去做生意,在市场挣了两块多钱,拿着回店。及至到了店中,见他们住的那间房锁上了,叫伙计给他开开门,伙计说:“你们不是不住了吗?”于星五说:“谁告诉你我们不住了?”伙计说:“那个姓汪的雇洋车把东西都拉走了,店钱也给了,他说不住了。”

于星五这下可就愣了,猴吃芥末——净剩瞪眼啦!越想越急,又不知道汪福林挪到哪里,天又晚啦,他急得直要掉眼泪,二百多块大洋都在皮包里哪。他万般无奈,去找个贱着点的客栈住一夜吧,晚饭也没吃,一宿也没合眼,翻来覆去睡不着。天亮了忍着一会儿,醒了之后,还往附近各栈店打听有汪福林没有。问了些家,都没有。他又往市场上去找,也没找着。最后他觉悟了,才知道是被人所骗,又冷笑,又咒骂,如同疯了一样。幸而身上还有几元钱,不至于挨饿。他思前想后,什么主意全都没了,直埋怨自己,跟着自己哥哥有多好,背着他偷了二百六十元,都叫自己花了,本领没学成,被人骗了。没脸回大连,就想在安东以卖戏法为生。及至再摆摊,不用说卖,连粘(nián)子(观众)也圆不上了。他直落到乞讨,住小店。几个月的工夫,到了天气寒冷支持不住了,才往大连给他哥哥写信,幸而他胞兄有手足情义,给他寄了路费,才回到大连。他向哥哥哭诉被骗情形,自愿悔过。从此柜上的财政永远不经他手,是人都看不起他,低头忍受了好几年。

有一年他到奉天去送货,在小西关看见了汪福林,两个人鸣警成讼。我老云正逛小西关,听于星五在巡警那里诉其被骗的事儿,一件件,一桩桩我都记住了。可是,他们成了官司以后如何我没有打听,大约着汪福林得受刑事处分。

我曾向江湖人探讨被骗的事儿。据江湖人说:“那是‘挑(tiǎo)厨供(gòng)’(卖戏法的)的把于星五当‘点儿’卖了。”(把于星五当外行人蒙了)我问:“于星五怎么会上那当?”某江湖人说:“真聪明人不贪便宜,也不上当;假聪明人,鬼机灵,他觉着他对于世上的事都很明白,看着哪样事好,哪样生意挣钱,他要干哪样。江湖人调(diào)侃儿管他那种人叫‘机灵空(kòng)子’。世上的人,越是机灵空子越能上当,不上当便罢,上了当就不轻。”我老云学会了这句侃儿,增强了警惕性,所以我遇见事就要谨慎小心,怕是上了当之后,还被人叫一句“机灵空子”!





江湖彩门之腥棚


江湖人的侃儿,不拘对什么事,凡是真的,调(diào)侃儿叫“尖的”;凡是假的,调侃儿就叫“腥的”。

在各省县市、各商埠码头,前几年兴过一种玩艺儿,有“人头讲话”、“六条腿的牛”、“三条腿的大姑娘”、“人头蜘蛛”。江湖人管以上这些玩艺儿调侃儿都叫“腥棚”,足见他们的玩艺儿全是假的。在前些年这几样玩艺儿还盛行一时,这种玩艺儿也都赚钱。原是这样,向来社会风俗专好谈奇说怪。阅者如不信,你买包茶叶到茶馆沏壶茶喝,管保你喝不完茶就能听见些个奇奇怪怪的事儿,何况三条腿的大姑娘、六条腿的牛,花两个铜子就能看一看,谁不想饱饱眼福呀?我看过多少腥棚的玩艺儿,也看不出他们的毛病。

江湖人的侃儿,不拘对什么事,凡是真的,调(diào)侃儿叫“尖的”;凡是假的,调侃儿就叫“腥的”。玩艺儿全是假的,叫“腥棚”。



有一年我云游到沙河子,那个地方名又叫安东县,是我国木行的大聚处,每年到了夏天,各省木行的人都携带资本到那里买货。安东县最热闹的地方是三不管,那个三不管的地方较比天津三不管(天津市南市的一个露天市场)有过之无不及。在那三不管儿就有个腥棚,也有三条腿的大姑娘,我看了几次。事有凑巧,有一天他们那腥棚的坎(kǎn)子(收门票的人)们,因为向人要迎门杵(即是门票钱)和人打起架来,经我给他们说合了,那个腥棚的老板和我交了朋友。我向他说:“你叫我把合把合门子(即是看看你们的毛病在哪儿)成不成?”他和我很要好的,不好意思说不成,他说:“等到推了棚的时候叫你把合把合(看看)得啦!”我听了非常高兴,连地方也不动,净等天黑了瞧个明白。到了天晚啦,游人俱都散去,他叫我进去看看。到了里面一看,那三条腿的姑娘刚站起来,她站起来也是两条腿,那地上还掉着一条腿,我看那条腿直动弹,真是叫人纳闷。忽见地上的板儿一起,从地下的坑内蹿出来一个人。我看到此时方才明白,这个三条腿的大姑娘是两个人凑的。在她坐着的底下挖了个坑,内里藏着一个人,藏起一条腿,由坑内伸出一条腿,凑成了三条腿。我将他们的“腥门子”看破了,才知道江湖的腥棚是一腥到底(江湖玩艺儿有许多是真的,调[diào]侃儿叫半腥半尖。惟有净假的没有一点真的,调侃儿叫一腥到底)的玩艺儿,江湖人管那种玩艺儿叫做“腥棚”是名副其实了。





江湖艺人孙宝善


余在民国十年前后赋闲无事,羁于旅舍,每日午后必往天桥巡礼。在魁华舞台后边,有个玩艺儿场,周围四通长凳,当中设一高案,铺以洋毡,皮包一个,粗布毛巾一块,约有尺来见方,毡角放茶碗一个,当中放着五个红豆。案后立着个矮胖矮胖的人,长得四方大脸的,两只手先敲茶碗,后变五个红豆,招惹那逛天桥的人们围得不透风,挤着观瞧。我记得他身后挂个布匾,两旁八个小字:“专教戏法,当面学会。”当中有三个小字“幻术家”,三个大字“孙宝善”。他是在天桥卖戏法的,每天游人盛多之时,就在做那“挑(tiǎo)厨供(gòng)”(卖戏法的)的生意,凡是老逛天桥的人都见过这个孙宝善。直到民国十八年,这孙宝善才“开穴(xué)”(江湖人管出外调[diào]侃儿叫开穴)。据彩立子(lì zi)行(变戏法的)的人说:“要讲究使苗子(江湖人管变仙人摘豆的豆儿调侃儿叫苗子),就数孙宝善第一。”他变的豆儿个头最大,可豆儿越大越难变,两只手十个手指,要藏那五个豆儿,越小是越容易藏的。孙的豆儿只是他一个人能用,到别人的手内可就变不了啦!他有个徒弟叫祁栋亮,身材小骨体瘦,如若变仙人摘豆的时候,不使孙宝善的苗子,另使自己的五个小豆儿。我因烦闷无聊,学他几手戏法,无事消遣,日期多了,与孙宝善交为朋友,和他三五日一见。二年有余,耳濡目染,得知厨供行内幕与孙宝善的小史。

孙宝善是北平人,自幼念书的时候就顽皮无比,常常逃学。他的“老戗(qiāng)儿”(江湖人管父亲调侃儿叫老戗儿)土(死)得最早,只有他的“磨(mó)头”(江湖人管母亲调侃儿叫磨头)在堂。他是念排琴(江湖人管昆仲一人,无兄弟姐妹调侃儿叫念排琴),成天价去逛东安市场。在清末民初,东安市场有个卖戏法的老人,姓杨,江湖人都叫他“厨供杨”,那是北平卖戏法儿开荒的人(江湖人管首创之人调侃儿叫开荒人)。孙宝善在厨供杨的摊子前边,天天去起腻,后来他就给厨供杨叩了瓢儿(认师傅调侃儿叫叩瓢儿),学习卖戏法。他初学之时不会做后棚的买卖(即是不会在屋内教人学戏法,挣大钱),也不会做前棚的买卖(管变仙人摘豆圆粘[nián]子、卖戏法叫前棚的买卖)。他光卖那仙人点戏。在早年,厨供行的人收了徒弟,都是先教徒弟们做仙人点戏的买卖。仙人点戏就是用两个小纸本,印上些戏名,一出出印上,每本三十页,每页印三十出戏名,每本共有几百出戏。如若有人在前本暗中记了一出戏,再翻第二本儿,问他哪页有他记住的戏,就能猜出记的是哪出戏。在民初的时代,卖仙人点戏的,各市场全有,每天能赚两三毛钱的利益,虽是没有本钱的生意,也颇能养赡自身。到如今可就见不着这种生意了。到了民国十年前后,孙宝善就成了厨供(gòng)行的大将(江湖人管最有能力的人调[diào]侃儿叫大将)。敝人曾听江湖人传说,孙宝善虽是个大将,他是个没“开赚”(没赚过万儿八千的调侃儿叫没开赚)的生意人,后棚的买卖最软(即不善于敲诈人财的意思),只会做前棚(场上)的买卖。若是讲究搁场子(在明地做生意)、圆粘(nián)子(招徕观众)、做包口(说完一段故事,再售其货,调侃叫包口)、使拴马桩(用话扣住了观众,不让走了,调侃叫拴马柱)、挑(tiǎo)幅子(撒传单),他哪天也能挣个两三元钱。他每天出来挣钱,就指着卖几手小戏法,向来不会将学戏法的人带到屋中去敲诈。我和他交了几年朋友,没见过他出过“鼓”儿(江湖人管骗了人家的钱,被骗的人觉悟了,找他们打官司动凶,调侃儿叫出了鼓儿)。挑(tiǎo)厨供(卖戏法的)虽是个腥(假)到底的生意,他骗的人们只是不痛不痒。不料,到了民国十六七年之后,国都南迁,北平的市面萧条,逛天桥的渐渐稀少。因为时势变迁,孙宝善指着卖戏法只能卖三四毛钱,一家数口,受了经济的恐慌,挤得他无法,也和同行人学会了“安瓜瓦点”(即是敲诈秘诀),大瓦特瓦(即大敲特敲),哪天也能敲到手内百八十元,收入日渐增加,衣食丰足了。那被骗的人也随着增加,受骗之后醒了攒(cuán)儿(明白过来了),都找他往回要钱,不是吵闹,就是打官司,他的鼓儿(吵子)天天不断。孙宝善顶了瓜(即是害了怕),他携着果食、怎科子(果食是他媳妇,怎科子是孩子)跑到天津去了。到了天津,在三不管(天津市南市的一个露天市场)撂地,又挖个点儿(敲诈个人),弄到手中千数多大洋,怕点儿倒(dǎo)杵(怕被敲的人往回要钱),又携家眷跑到奉天,在小西关做了几天买卖,染了时疫就土了点(即是死了)。孙宝善死后,他的媳妇带孩子回到北平,孤门孤户,又无恒产,为经济所困。未几,孙宝善的媳妇也土了点啦,抛下个七八岁小孩,孤苦无依,有多可怜!天桥的人们还有义气,有好几家收养其子。不料,那孙宝善的孩子竟不学好,到了谁家就祸害谁家的东西。害得孙宝善之友人无计可施,只好不要他。在年前我还瞧见他一次,至今这个孩子到哪里去了,恐怕飘零无所,流落他方了。我当初还想着要学些骗术图个眼前快乐,自从瞧见孙宝善家败人亡了,吓得我云游客也不敢妄为了。老合们(走闯江湖的同行们)!我说的这段故事,不可不想自己,殷鉴不远,急速醒悟吧!





江湖艺人去平留津的大金牙


最近我老云走在各处,时常听见各商号由广播电台播出来玩艺儿有:“金龙宝殿修在了中间哪……唉……”不用我老云说,阅者诸君就能知道这是大金牙拉洋片唱的曲儿,生意人最难得的就是能够“响万儿”(成了名了)。

焦金池学会了拉洋片,他镶了两个金牙,人人不叫他焦金池,都叫他大金牙。



如今要提起大金牙三个字来,几乎无人不知了。他们一家数口都叫金牙,有老金牙、大金牙、小金牙。老金牙姓焦叫焦永顺,是河间府任丘县的人。他自幼儿就投入江湖中学习“柳海轰儿”(管唱大鼓书的行当调[diào]侃儿叫柳海轰儿的)。他唱了些年大鼓书,各商埠码头也都到过,跑腿的(江湖人自称叫跑腿的)人们都知道有他这个腿儿(即是知道唱大鼓的有他这一号)。他唱的是西河调儿,因为他没有什么“万子活”(管说长篇书目叫万子活),始终也没“火穴大转”(管不会唱整本大套的书调[diào]侃儿叫没有什么万子活,管没大红大紫过调侃儿叫没火穴大转),仅落个衣食不愁而已。夫妻二人就生了一子,名叫焦金池。从小儿这焦金池就跟着他父亲在外边跑腿,他先和人家弄“腥棚”(假的),至今大金牙的家中还挂着个放大相片,相片上是大金牙拉着六条腿的牛。阅者要问我怎么知道的?有一回我老云到他家里看见的。

焦永顺有个亲戚姓潘,住家在天津海下塘沽,都叫他潘小秃,专以画洋片为生,现今各洋片的画片都是潘小秃画的。他画洋片是小张的五元,大张的十元,特大的三十元,先交足了钱后画。近期的半年取货,远期的一年后取货。他的生意是拥挤得很,凡是拉洋片的,都将他当做圣人恭敬。几十年来已然发了财了。焦永顺要画洋片,能够少花钱当月取货。有了这种便利的事儿,焦永顺的儿子就学了“光子”(江湖人管洋片调侃儿叫光子)啦。焦金池学会了拉洋片,他镶了两个金牙,人人不叫他焦金池,都叫他大金牙。他拉洋片能够响了万儿(有了名儿),是有几样特长:第一是他人式长得“压点”(江湖艺人如若长得有台风,有个气派,调侃儿都说他人式压点);第二是他的“碟子正”(江湖人管口齿伶俐,口白清楚,调侃儿叫碟子正);第三是他的“夯(hānɡ)头子好”(江湖人管好嗓子调侃儿叫夯头好);第四是他的“发托卖像(江湖人管做艺的人们到了表演的时候,脸上能够形容喜怒哀乐叫发托卖像)好”;第五是他的“活头儿宽”(江湖人管会的曲儿多调侃儿叫活头儿宽);第六是他能够“攥弄(zuàn nong)活儿”(江湖人管会编各种小曲调侃儿叫会攥弄活儿)。

大金牙有这六样特殊的技能,成了大名。电影的明星陆克、贾波林(即卓别林)在银幕上能在他们面上形容那滑稽态度,受人欢迎;大金牙的洋片曲儿,每逢唱的时候使出那“稀溜钢儿”(江湖人管逗笑的话儿调侃儿叫稀溜钢儿),听曲的人们都能咧瓢(liě piáo)儿(即是笑了),大金牙唱的曲儿也是滑稽的玩艺儿,社会的人士无不欢迎。我老云常说,艺人若要成名,受人欢迎,必须多学滑稽的艺术。我老云在江湖里调查得来,江湖的行当最苦就是拉洋片的。要做份洋片至少也得四五十元,画片子得到塘沽潘小秃去买,做洋片箱子得到山东德平县买。除了这两处的画匠、木匠,别处洋片活儿是不行的。即或有画的、有做的,弄出来的画片、洋片箱子也不能美观。他们做生意敲打锣鼓,连拉带唱,累了一天才挣个几角钱。临完了,膀子上还得担个几十斤沉的洋片箱子回家。江湖人常说:“象法(江湖人如有真本领,天天能挣大钱,处处受人欢迎,调[diào]侃儿称为象法)一包儿,空(kòng)子(不懂江湖内幕的人)一挑(tiāo)儿。”那相面的先生们,只要包内有管毛笔、铜墨盒、碎纸条,到处挣钱,挣的钱多,那份家具轻巧,江湖瞧着他们行,人人羡慕,称为象法一包。那拉洋片的行当,本钱又大,受累又多,挣钱又难,担着全份洋片家具,分量又沉重,江湖瞧着他们这种笨生意,讥诮他们是空子一挑儿(江湖管事事外行的人叫空子,像拉洋片的就算空子,谁要有本领也不干这种笨营生)。据我老云在咱们中国云游了这些年,拉洋片的见多了,从来没有发达过人。像大金牙这人可谓空前绝后了。他的洋片家具与众不同,别人的箱子是四个镜头,让座的时候只能坐四个人;他的箱子是八个镜头,能坐八个人,挣钱能比别人多加一倍。要是收拾回去,一个人弄不动,也得两个人抬着,使这家伙非两个人照顾不了,他那份家具非用八九十元做不出来。每天的挣项,由早到晚才能挣一元多钱至两元钱。

大金牙的进化力量很大,在天桥儿拉个场子,他能不叫人瞧洋片,只凭敲锣打鼓唱洋片曲儿,挣个一两元钱。拉洋片的不用洋片,就是他能行,别人恐怕学不到的。齐化门(今天的朝阳门)菱角坑有野茶馆时,徐狗子将他几份洋片架弄到台上,也掺在杂耍(多种形式的曲艺演出)里,算场玩艺儿。一些贵族式的家庭,有喜庆宴会也常常邀他。大金牙的洋片也登了大雅之堂,妇女们听时,他能唱些雅趣的曲儿,话匣子片儿也灌了许多,销路很是不错。广播电台常常邀他,播出来的曲儿人人都能听见,他的玩艺儿真是普遍了。

大金牙生有二女一子。大姑娘叫焦秀兰,二姑娘叫焦秀云,三的小子乳名叫小丑儿。他的月丁码姜斗(jiàng dǒu),真是撮啃(zuō kèn)(江湖人管两个大姑娘调[diào]侃儿叫月丁码姜斗,管长得美貌调侃儿叫真是撮啃),焦家姐妹受他们戗(qiāng)的戗儿夹磨了,能柳海轰儿,在平津两地火穴大转(焦家姐妹的祖父调侃儿叫戗的戗儿;传授她们会唱大鼓,调侃儿叫夹磨了能柳海轰儿;她们挣了大钱,都叫座儿,调侃儿叫火穴大转)了。在前年他们全家班每日在天桥献艺,高朋满座,始终不掉座儿。那小丑七八岁就能上场,打个鼓套,抓个碎包袱(逗笑的小玩艺儿调侃儿叫碎包袱),垫一场活,也能挣一两元钱。他的发托卖像,颇有乃父之风,叫他小金牙,是名副其实了。大金牙的收入丰富,便染了不良嗜好,北平这个地方实行戒毒的时候,因为大金牙顶了瓜(即是害了怕),全家赴津,杯弓蛇影,以讹传讹,都轰动了,要喷(江湖人管洋枪调侃儿叫喷子,要枪毙了调侃儿叫要喷了)大金牙。人人传说不一,闹得满城风雨,结果算是没有那回事。直到焦秀兰喜期之前,大金牙全家归平,谣言始息。

老金牙焦永顺是个旧礼教的人。焦秀兰嫁的丈夫并非艺人,且系发妻,是他极力主持的,绝不使其孙女生财,为人作妾。他的主张是值得我老云佩服的。焦秀兰出嫁后仍在焦家做艺,所挣的金钱也按股均分。他小夫妻生活起居也颇安逸,快乐无忧。现今大金牙全家因受津埠人士欢迎,在那里献艺,已久未回平了。天桥儿尚有大金牙的徒弟,也叫小金牙,系已故说评书的张福全(张福全系说《施公案》的群福庆师弟)之子,受大金牙的传授,拉场子撂明地,仿照其师的艺术,颇能挣钱。因为他师徒在江湖中是光子(拉洋片)里开荒(江湖人管首创叫开荒)的人物,我老云在《江湖丛谈》的艺人传内写出来,贡献于阅者。





江湖艺人快手卢


北平这个地方,变戏法的艺人可真不少,在各市场、庙会拉个场儿做“明粘(nián)子”(聚好了观众)的,有几十个,要找一个堪称上选的人才很是不易。据我调查,“立子(lì zi,变戏法的)行”摆明地的有两种档子:一种又练玩艺儿又变戏法的。变戏法是引人“圆粘(nián)子”(招徕观众);练玩艺儿要钱,那不是纯粹戏法,说行话是“签子(变戏法带赞武功叫签子)活”,不过是彩门(变戏法的)的一种玩艺儿,不能以戏法挣钱。有一种在场内立着一对圆笼,上边写着“×××堂,专应堂会,巧变戏法”,对面放着个小长笸箩,一个毯子,一把破铁壶,几个旧式的茶杯,锣鼓三件。变仙人摘豆、巧耍连环,招引人圆粘子;变大海碗、罗圈当当等等玩艺儿要钱,不练各种功夫。这样是纯粹的戏法。只是没有一个能用小戏法搁场的。

在民国十三年,营口洼坑甸市场有个辽阳人叫王老疙瘩(gē da),他就专变小戏法,他那啃包(kèn bāo)(装着挣钱的用具的包)就是一个小笸箩,十几个小茶杯,一个小铜锣,十几件子东西就能变戏法挣钱。他也没有圆笼,小包儿背着轻松,走也方便。到了哪儿,场子站的人多,挣钱更多。是江湖人都佩服他,说:“象家(有真本领,能挣大钱的人)一包儿,空(kòng)子(不懂江湖内幕的人)一挑(tiāo)儿。”天津当年曾有个戏法罗,也是那样。

北平变旧戏法的有个快手卢,是河北涿州的人。自幼儿就学会了立子行,大小戏法学了无数。他的人样也好,口齿伶俐,嗓音洪亮。在他年轻的时候专做明地(不是屋子的演出场所),往各市场、各庙会,撂场子变戏法儿。他也会点前棚,圆粘子,卖弄“钢口”(说话的技巧和分量),使个活票点儿(即今天雇的托儿),他瞧着人够挣钱了,“扣个腥儿”就把人吸住。什么叫扣腥儿呢?就是他们变戏法的在场内用个纬帽插上个鸡毛,说能变只鹰;用毯子盖上个兔子尾巴,说能变个活兔,把看热闹的人吸住了。就是不变这两样,变个海碗来条金鱼就要钱,直到把钱要完了,也不变那黄鹰和活兔儿。他不过使那个方法把人吸住,行话叫扣腥儿,调(diào)侃儿叫使拴马桩儿(用话留你,让你走不了)。

快手卢的戏法变什么也比别人手快,变得格外利落,久看他变戏法的人就叫他“快手卢”。他得了这个名儿才享了大名。他又赶上清末的时候外国人士到了北平,那外国人要看中国的戏法,快手卢的家档子又挣了大钱。唱大戏得有箱,变戏法得有档(dàng)(道具)。他练的往身上挂活(变戏法的人往身上藏东西,行话叫挂活),比谁挂的都多。别看身上藏了那么些东西,往台上一走,放开脚步,行动自由,不露痕迹,“落(liào)活”(他们变戏法的人管由身上往下变东西行话叫落活)也干净。他所落的活:十三太保、九莲灯、九龙闹海、八仙过海。落完了一件戏法,还能多饶上一两件,使人惊奇,其艺术过人之处实是不少。他常做外国的家档子(堂会),还练了一嘴的外国话。清室各王公府内也常看他的戏法。光绪年间,提起快手卢几乎无人不知。

有个美国魔术跳舞团的经理人名叫玛齐师的来到北平,见快手卢的戏法变得好,与他订立合同,邀快手卢搭入该班,往南洋群岛、菲律宾、小吕宋(菲律宾的一个最精华的岛屿,明朝初称为小吕宋)、香港、台湾等处献艺,颇受欢迎,均获重利。只是他恋家乡,不愿久走外国,他等到合同的期限满了,就回到了中国。他回来的时候,还费了很大的周折。庚子年变乱之后,东交民巷的各国人每逢有喜庆宴会招待宾客时,都邀快手卢的戏法。他每逢变的时候,铺垫话儿不说中国话,能说外国话,还能用外国语当场抓哏(gén,现场抓笑料)、抖个包袱,把“色(shǎi)唐码子”逗得咧了瓢儿,怎么不挣“色唐杵”(江湖人管外国人调[diào]侃儿叫色唐码子,把人逗乐调侃儿叫咧了瓢儿,管挣洋人钱调侃儿叫色唐杵)呀!有些外国人在南洋群岛看过他的戏法,到了北平,点出名来叫人给找快手卢,看看他的玩艺儿。他出了一趟外洋,不只能挣回钱来,学了些色唐钢儿(江湖人管说外国话调侃儿叫色唐钢儿),还在色唐的“穴(xué)眼”里立了“万儿”(江湖人管中国人到了外国的地方享了大名,调侃儿叫到色唐穴眼里立了万儿)。

在清末民初的时候,彩立子(lì zi)行(变戏法的)的人见中国戏法不吃香了,有好些个人会投机,挑(tiǎo)起幌子来,弄份音乐,旗上写着“外洋新到洋戏法”。穿上洋服,在台上变起活来,就能挣钱。其实那些个戏法并不出奇,只是社会里的人好奇心盛,不论什么,只要挑出色唐的幌子就能蒙住人了!哪怕是色唐码子放的屁呢,也能有人说真香。中国戏法实在比外国戏法好,据说他们变的罗圈当当、仙人摘豆,能叫人围着看四面儿,绝不能看漏了。那洋戏法,只能看一面,左右后三面不准外人看,如若上台前边看,他们变的戏法甚是新奇;若是站在后边一看,可就稀松平常了。洋戏法无论多好,只能兴旺一时,不能久存。那仙人摘豆、罗圈当当,虽是旧戏法,变了多少年也能有人看。那外国戏法只要变的日子多了,就没人看啦。阅者如不相信,像当初他们变的“人头讲话”,在如今就没人看了。而旧戏法是百观不厌哪!当初洋戏法盛行的时候,变旧戏法的人也受了些影响,惟有快手卢,不惟不受影响,并且还挣外国人的钱。有些人只见他挣外国人的钱,比挣中国人肥得多,生了羡慕之心,这个也拱,那个也挤。只是那外国人不看他们的玩艺儿,专看快手卢,拱也不成,挤也不成,快手卢“火穴(xué)大转(zhuàn)”(挣了大钱了),很挣了不少钱。只是他染了不良嗜好,好抽大烟,好养活鸟儿,把平生的钱财俱都耗尽。到了他的晚年,有家档子(堂会)就去做生意;如若没有事,他就到福海居去听评书,最后他成天不在家,只在王八茶馆喝茶,困了就冲(chòng)(打盹),冲得腰弯了,弄得身体受伤,到了变戏法的时候,常常“抛了活儿”(江湖人管变戏法变漏了调[diào]侃儿叫抛了活儿)。因为他的人缘好,看主都能原谅,虽然常抛活儿,也没什么关系。

他到了晚年,每逢有家档子(堂会)就带着他儿子卢万祥走堂会,所去的地方是北京饭店、六国饭店、各国医院、各使馆兵营。他们父子净做堂会,把地上搁场子上的事就失了传啦。直到如今,快手卢已经死了,他的儿子快少卢只能做堂会上台表演,那撂场子、圆粘(nián)子(招徕观众)、使拴马桩儿(用话留你,让你走不了)、扣腥儿、使杵门子(到要钱的时候叫杵门子)等等江湖事,全都不会,只能做堂会。到了如今,国府南迁之后,北平这个地方市面萧条,社会里的人们都不大办事,堂会的事日见稀少,他又不会上地(做生意)抓钱,幸而北京饭店、六国饭店执事诸公念与快手卢多年之好,极力维持。所有外国人要看戏法的时候,都找快少卢,不找别人。据卢万祥说,他现在一家数口就仗着他父亲旧日的朋友维持,衣食无愁。快手卢有朋友如此,也能使朋友看父敬子了。





天桥的戏法场


天桥的戏法场久长的只有金家玩艺儿,他们场子在公平市场北半部振仙茶园后身。不止天桥,各市场、庙会变戏法的十有八九都是他金家的徒弟。他们是哥儿两个:大爷有麻子,都叫他金麻子;二爷叫金万顺,现在东安市场撂地(在明场上演出)。

金麻子久占天桥,他是“彩立子(lì zi)”(变戏法),也不翻斤斗,也不拿大顶,不练三把刀,不练大铙钹,专讲变戏法。所变的玩艺儿,空壶取酒、玻璃变鸡蛋、杯中生莲、纸变蛤蟆、破扇还原、仙人摘豆、三仙归洞等等的小戏法,也不过变这些个东西垫垫场子,引引人“圆粘(nián)子”(招徕观众)而已。挣钱的戏法是先使“揪子”(管变大海碗内有金鱼的戏法调[diào]侃儿叫揪子)、“照子”(管变罗圈当当的戏法调侃儿叫照子),每逢要钱费劲的时候,用“抿(mǐn)青子”“逼杵儿”(管吞宝剑调侃儿叫抿青子,没结没完地要钱调侃儿叫逼杵儿)。剑、丹、豆、环,不算戏法,那算是真功夫。仙人摘豆非童子功不能学;月下传丹、变大琉璃球儿,没有一年半年的功夫也变不好;吞宝剑受几个月的苦处才能学好。九连环比这三样还难练。除了吞宝剑能挣钱,逼得下杵来,其余的三样,费那大的劲,只能圆粘子使用,要钱是没人给的。每逢夏天,他们圆粘子不使戏法,用“土条子”(管长虫调侃儿叫土条子)就能吸得住人。变戏法的都是大人掌买卖(变戏法挣钱全靠大人,不能靠小孩,调侃儿叫大人掌买卖)。变戏法有小孩,不过是“多抖搂包袱”(管当场抓哏逗笑,调侃儿叫抖搂包袱),有的是自己的孩子,有的是收的徒弟。可是他们离开了小孩挣钱费劲,差不多都有个小孩。变戏法的挣钱能力如何,得看他们包袱多少。别看天天变这样儿,不会让你看腻的,总会有人爱看的哪!他们常说:“你们众位当做旧玩艺儿看,我们当新的变。”金麻子生有二子,也是变戏法。他收的徒弟很多,有郭进才等十数个。金家的戏法是彩门中最盛的,虽然是土地玩艺儿,发财不易,养家糊口是能成的。我老云说他们这种生意是平民化的。

狗熊程家原籍是吴桥,在北平落户,久居朝阳门外。他们老哥们儿是五个人,小哥们儿是十几个人,都以变戏法为生,他们久占的不是天桥就是东安市场。在我老云读书的时候,程福先就在东安市场东院耍狗熊。凡是逛市场的人们不叫他们戏法程,叫他们狗熊程。直到如今提起狗熊程来,几乎无人不知。自从东安市场的东院连三并四地盖房,将杂技场儿都挤没啦,他程家的玩艺儿才迁于天桥儿。他们每天上地(做生意)是打锣敲鼓、踢腿窝腰圆粘(nián)子(招徕观众)。圆上粘子就练三把飞刀,耍大铙钹。最惊人的玩艺儿是扔木球。那木球儿比鸭蛋还大,扔的时候在脑袋上戴个皮兜儿,能将球扔个十来丈高,不用手接用脑袋去接,那球儿不偏不歪正落在皮兜之内。这样还不算,他能将皮兜转在脑后,木球也扔几丈高,不用眼瞧着,低着头看地,那木球能落在兜内,百发百中,从没掉在地下过。我老云是钦佩这一手儿的。他们挣钱的玩艺儿是有个五六岁的小孩,在地上给他放三个小茶碗,口儿冲下,上边又放木球三个,用个四条腿的长板凳往木球上一放,只有三条腿儿在球上,一条腿儿闲着,叫小孩往凳子上一站,再往地上放个茶碗,碗内满满的凉水。都安放好了,叫小孩弯腰,用嘴够在地上,将茶碗咬住,伸开了两只手,在手上放两个茶杯,也是满满的凉水,凭小孩直腰的功夫,三碗水不洒,和看玩艺儿的逼杵儿(要钱),实在不易。他们所变的戏法倒视为第二,练种种武功视为第一。他们这行不叫彩立子(lì zi),说行话叫“签子”(变戏法带赞武功叫签子)。狗熊程到了天桥,净练武功不耍狗熊了。我问过他们,这几年为何不耍狗熊?他们说,买个狗熊得几十块大洋,教会它练玩艺儿,没几个月工夫不能用它挣钱。还得花钱喂,处处小心,稍一大意就能“土(死)喽”,糟践一个牲口好几十元。这个年头买卖平常,弄不起来。狗熊程是因为耍狗熊得的这个名儿,虽不耍狗熊了,人们还是叫他狗熊程。程家父子都是安分守己养家汉儿。我说,逛市场的人们给他们往场内扔钱不是“抛空杵儿”(管花冤钱调[diào]侃儿叫抛空杵儿)。

在公平市场万盛轩的前边有个戏法场子,所变的戏法,没有仙人摘豆、三仙归洞、杯中生莲、破扇还原等等的玩艺儿。大活没有罗圈当当,小活没有茶杯中的戏法,剑、丹、豆、环的功夫更没有啦。场内用几根竹竿支个三面架子,用布棚挡上三面,棚内放只箱子,弄来个小孩装在箱里,掀开,小孩就没啦;盖上,孩子就有啦。这个戏法叫大变活人,是挣钱的玩艺儿。他圆粘(nián)子(招徕观众)的玩艺儿在天桥来说与众不同,在地上埋几个小坛子,坛内装布人,他管坛内装的布人叫歪毛,或叫淘气。叫歪毛,歪毛就在坛内连蹿带跳;叫淘气,淘气就在坛内连蹿带跳。看的人们都很纳闷,不知他使的什么方法能够叫小布人在坛内自动,许多人猜不透他的。变这个戏法的人有三十多岁,细条身材,瘦瘦的面庞。此人姓纪,他从前是做腥棚(做假买卖的)的,近几年来,社会里人士知识开化了,弄腥棚是不成啦。三条腿的大姑娘、六条腿的牛,谁都知道是假的,要钱没有人看,这种生意渐渐地消灭了。可是他颇有灵机,弄这几样戏法占个场子,也能养家糊口。其余的吃腥棚的人哪,受了淘汰,都不知哪里去了。有一次天桥的朋友请我吃晚饭,正在冬天。吃完了晚饭天光黑啦,我从朋友家中出来,听见有人吵吵嚷嚷闹得很凶。我老云顺声音寻了去,见十几个小孩子围着姓纪的,彼此笑骂。我还觉得他那么大的人和一群孩子骂什么,听后我才明白是因为什么。原来他在那场内掘了一道几十丈长的深沟,沟内埋着竹筒子,筒内有绳儿,绳头儿有钩子,那钩子钩住坛底的铁丝绷簧,竹筒子通在一个戏园子里。在戏园子里坐个人,他变戏法的时候那人用手扯着那两根绳,一根通着小歪毛,一根通着小淘气,如若他在场子里叫歪毛动弹动弹,戏园子里的人就将歪毛的绳子一动,铁丝绷簧就颠,绷起布人来,看的人们就见小布人跳跳蹿蹿,像小人钻坛子一样。天桥的小孩子真是淘气,聚了十几个都到他场子,每人撒一泡尿往那地上浇,灌在地里,将绳子竹筒子全都冻上,到了白日,他上场子变戏法就不用变了,因为小孩子淘气将他的“彩门子”(戏法闹鬼儿的机关调[diào]侃儿叫彩门子)给毁了,害得他夜内不敢睡觉,无论天气多冷,他得看着他的彩门子。怪不得他和那些孩子争吵,弄个门子得费一夜工夫,要是给毁了,焉能不急?他也算是艺中人,能有攥弄(zuàn nong)(自己做的调侃儿叫自己攥弄)活儿的才干,可惜这个时代不景气,仅能糊口,衣食不缺罢了。我老云无意之中得着他的彩门子,写在《江湖丛谈》之中,免得人们瞧歪毛、淘气时心中发闷!





天桥的摔跤场


在天桥爽心园前头有个相声场,在相声场的北边便是摔跤场。摔跤不算生意,在早年生意场里也没有这种玩艺儿。秦汉时代管这宗技术叫相扑,宋代叫角力。宋岳飞善拳棒,其拜弟牛皋欲学拳脚,因其蠢笨,难学技击,岳飞将拳术中刁拿锁扣,缩小绵软巧,钩挂连环,挨傍挤靠,闪展腾挪,分筋错骨,点穴离位,猫蹿狗闪,兔滚鹰翻等招术传于牛皋。各种动作各种性质,即今日之摔跤也。到了清朝时代始称掼跤,设有善扑营。左翼在东城大佛寺,右翼在西城当街庙,称为官跤场。相传官跤场摔死人勿用偿命,私跤场不能如是。善扑营中扑户、塌希密(摔跤级别名称),皆八旗子弟。塌希密也不易当,必须在私跤场用功。数年苦功,在私跤场摔成了头路啦,才能由各旗保送往善扑营试艺挑缺,挑上缺才算当上塌希密。凡塌希密升入前五军叫“候等儿”,等到了扑户出缺时,再由堂官监视试艺挑缺,挑中者为三等扑户,再升始为头二等。其升等挑缺时,弊幕层层。摔的跤好不如有门路,金钱运动。有官有私有弊,昔时官场的黑幕俱是如此,岂止善扑营呢?善扑营有三大技艺:有练摔跤的;有练跳骆驼的功夫,名曰“蹁骣(pián chǎn)”;有拉硬弓的。

摔跤的功夫讲究欺拿象横、通天贯日、踢抽盘肘卧、抽辙闪拧空、蹦拱揣(chuāi)花倒(dào)、耙拿里刀勾二十八种秘诀,将这些法子练成了,才能使绊摔人。据我所知的绊子有:枕头手花、手别子、拱别子、切别子、大得合落(dé hé lè)(满语,汉语是腿打腿)、小得合落、挂踢、穿裆靠、穿腿摸、手脚别子、挑(tiǎo)钩子、圈腿、桩顶、里手入、三倒腰(dào yào)、夹头手花、摩(mā)楣子、坡脚、里手钩、外手钩、握腿、倒(dào)别子、反把(bǎ)、正把(bǎ)、反别子、温别子、揣(chuāi)别子、摩(mā)膊脚、挑(tiǎo)桩、飞逮(dēi)子、里手搂、外手搂、架梁脚。最厉害为三倒腰、得合落,在早年的跤场若有使这样绊子的,都是两个人摔出仇来,拼了命啦,才能使那两个厉害招。平常日子不易见之。凡是摔跤的人,有练胳膊上功夫的,有练腰上功夫的,有练腿上功夫的,有练脚上功夫的。练这几处的功夫,天天得用家伙早晚练习。所用的家伙:大棒子、小棒子、大推子、小推子、麻辫子、锁链子、地撑(chèn)儿、滑车儿、枣木桩儿。

善扑营的长官有都统、副都统、左右翼印务等职,这些官都由亲王、郡王、贝子、贝勒兼领。每年最重要勤务为正月初九日演礼,名曰“垫差”,或曰“拿等儿”,较胜者可以升赏。正月十九日皇上在紫光阁御览视艺,是日为善扑营扑户与蒙古人在毡子上摔跤。腊月二十三日祭灶王,皇上在御苑摔跤,俗称“灶王队儿”。善扑营的扑户最有名的大祥子,身体魁梧,人样子也威武,膂(lǚ)力过人,个大的数他。个小的有搬腿禄儿,瘦小之躯,每逢取胜,皆以搬腿胜之,他有这种拿手,人称为搬腿禄儿。其余的有黑虎二爷等。至清末时则有宛八爷(宛永顺,宝三、沈三的师父,善扑营头等扑户)。

摔跤人比试时所穿衣服,注重上身衣服,不注重下身。上身衣服系数层布所制,名曰“褡裢”,下身裤子不论好歹,所穿的靴子,前面的脸儿凸出来,名叫刀(dāo)螂肚儿。

清室设此机会用其技艺,威震内外蒙古也。至今时代变迁,善扑营之人十存一二,也都老迈苍苍了。自入民国以来,摔跤这种技术几乎失传,幸有一班人在各杂技场撂地,虽是掉在地下挣钱,还不算江湖玩艺儿。有人讥诮彼辈为摔活跤的,太不原谅人了。如能真摔实跤,摔坏了就不用干啦。凡是撂地(到场地演出挣钱称为撂地,摔跤的场子是明地)摔跤的人,都是好喜这种功夫,经济压迫子弟下海。我老云常说,摔跤的玩艺儿在生意场内算是最实在的玩艺儿。不过他们为了挣钱,也都和江湖人学的每逢上地(做生意)先圆粘(nián)子(招徕观众),摔几回垫垫场子,将粘子圆好啦,然后也按着把式卖艺的一样,全都站在场子当中,向四外说:“我们这回叫××和××摔一跤,摔完了和众位要几个钱,有走的没有?”说到这里往四面一看,围着观众全都不走,接着又说:“伙计你摔吧,没有走的,这场力气没白练,我们四面作个揖托咐托咐,南边是财神爷,西边是福神爷,北边是贵神爷,东边的也是财神爷,四面都作到了揖啦,摔完了,众位带着钱给我们往场内扔几个,几个大小伙子挣众位顿饭钱;没带着的白瞧白看。如若要走可早走,别等我们摔完了要扔钱的时候你再走。这可似我们小哥几个煮熟了一锅饭,给我们往锅里扔沙子。我们凭力气挣钱,也没有刮钢绕脖子(挖苦人,说瞎话)。话是交代完了,四面再作个揖,说摔就摔,插手就练。”他们练了这套江湖口,也是无法,为挣钱养家。如今我国各省运动会、全国运动会、世界运动会,都有摔跤的人参加。摔跤的这种功夫是我国国粹的一种武术,至今没有失传,也是摔跤撂地的人们能够保存国粹的一种功劳,使各界人士知道还有这类武术,实是他们的好处。如若没有他们这些人干这行儿,不用说保存这种技能,提倡这种武术,也恐无人道及了。

摔跤的功夫讲究欺拿象横、同天贯日、踢抽盘肘卧、抽辙闪拧空、蹦拱揣(chuāi)花倒(dào)、耙拿里刀勾二十八种秘诀,将这些法子练成了,才能使绊摔人。



摔跤的人物,在天桥久占的,沈友三、宝三、李永福、魏老(魏德海)、张狗子、傻子,十数人而已。沈友三在红楼开设成药铺,改卖大力丸,较比摔跤收入丰富多了,他的跤就不常摔啦。天桥的摔跤场占长久了的就是宝三跤场,他的四五个伙伴,团体性很坚固,这些年也没散帮儿。摔得火炽是他与魏老、李永福等,里子都硬,才受人欢迎。宝三的品行端正,并无嗜好,保养身体,能务本分,值得我老云佩服;并且他比别人多出戏,还耍中幡,每逢年节的时候就不摔跤,耍几天中幡,他那种玩艺儿在天桥可称蝎子屎——独一份儿。张狗子的跤场在公平市场万盛轩东边。他们这班人颇为不弱,不过比宝三那帮伙计稍为逊色。故此我老云还说,宝三的跤场在天桥算是第一,张狗子身高力大,胆小,公正,也是守本分不妄为的,无有劣行,值得人佩服。





天桥的空竹场子


在天桥杂技场练空竹的艺人,最有名的是王雨田、王葵英父女。王雨田久住南横街,父为商人,他自幼就好练叉,随黑窑厂(黑窑厂有个花会组织)的“开路”(走在花会队伍前面要飞叉开道)走过些趟会(花会),“三股子”(管叉调[diào]侃儿叫三股子)练得最为出色。清末的时候他在步营当差,入民国改当商团,又入警界,在粮食店站岗,因汽车夫不服指挥,“鞭(打)过开色(shǎi)唐轮子的”(管汽车夫调侃儿叫开色唐轮子的),后为车主势力所屈,愤而走闯江湖。他初入老合(闯江湖的)的行当是给马班子(跑马戏的)练叉,走西北穴(xué):大同府、绥远、张家口。与马班子“劈(pǐ)了穴”(管散了伙调侃儿叫劈了穴)之后,在东安市场与常立全“联穴”(管合伙、组班子调侃儿叫联穴),赁个场子上地(做生意),二人做艺,王雨田练叉,常立全耍空竹,每日的挣项足可养家糊口。常立全是旗人,会说评书,可是投入过评书的门户,没有“老帅”(师父),算是个“海青”(如票友下海一样意思)。他多才多艺,能抖空竹,单、双都行,罐子盖、醋碡碌(zhóu lù)(盛酒器),练的花样很多:王瓜架、猴爬杆、跳梁、回头望月、枯树盘根、反插腿、正插腿、倒爬绳……足有几十样儿。他腰腿灵活,非常精巧。他两个人,一个人练,练得出奇;一个抖空竹,抖得娴熟,很是档子玩艺儿。王雨田是个有志气的人,他在那时学会了抖空竹,后来才火穴(xué)大转(zhuàn)(挣了大钱了)。常立全染上不良嗜好,性极懒惰,每天上地(做生意)所挣的钱,只要够一天花的,就立刻不练了。他孤身一人,小店一住,别人看他没有意味,他个人却是快活。王雨田一家数口,家无恒产,与他联穴,很受影响。直到劈了穴(分了手),常立全自己上地(做生意),还是一样挣的够花就归店过瘾,明天再见。王雨田带着他的姑娘王葵英,在天桥公平市场巧耍飞叉,抖空竹。几岁的姑娘,抖起空竹干净利落,身体灵便,逛天桥的人们看完了谁都给钱,他父女在天桥就“火了穴”(即是大红大紫)啦。后来,王葵英的艺术日日进步,竟能“响万儿”(即是响名)了(曾于1956年世界青年联欢节获得银奖)。

在天桥杂技场练空竹的艺人,最有名的是王雨田、王葵英父女。



白云鹏的杂耍(多种形式的曲艺演出)班子约他父女加入,往京、沪、津、汉等地献艺,到处受人欢迎。各处的馆子争相延聘,收入也甚丰富。他们父女能以抖空竹起家,十几年的光景,置了几处房子,也小有资产。谁说艺人不富啊!

世上的事,无论学会什么,艺业在身,小则养家糊口,大则发达致富。江湖艺人只要没有嗜好,理财有法,也是一样的发达。近年以来,王雨田父女只在北平献艺,并不远行。有时候在天桥上地(做生意),有时上各杂耍馆子。葵英的人缘最好,无论是谁,也是评论她好。别看她是个女孩,通达人情,谦恭和蔼,技能惊人,还善于言谈,知礼仪,孝敬父母。在这世道衰微的时代,她能这样,很值得人佩服。如今她已然二十有余了,他父母因为她“太岁见海(hāi)”(管年岁见大调[diào]侃儿叫太岁见海),不叫她往天桥做艺,只做堂会,上杂耍馆子。天桥的杂技场是看不见她的玩艺儿了。

王桂英(是王雨田的二女儿)年方八九,抖空竹不弱于葵英,可算后起之秀。每逢王雨田往天桥做艺,就带着她去。不过,他们不能天天去的,到了天桥也是和人联穴(xué)上地(合伙演出),十天中只有二三回。据王雨田和我老云聊天儿的时候表示,他少年爱惜“开路”,众亲友都轻视他不做生产的事业就学走会(就是走花会)。不料如今,一家数口竟赖以糊口,生活无忧,真是意想不到。听他的口吻是很知足。知足者常乐,能忍者自安。

学会艺,防身宝,这话不假。如今这个年月,只要有一技之长就能维持生活,抖空竹、踢毽子,在清代时是一种消遣的玩艺儿,现在能在社会里挣钱养家,不怪他说是想不到。





三不管的杂技场


社会里的人只要有一技之长,就能吃饭;学会了艺业,是防身之宝。这几句话说的诚然不假。在前清的时代,一般的人们都练习抖空竹、踢毽子、盘杠子、扔石锁等等玩艺儿。在那个年头,不过消遣解闷,活动身体。到了如今,真有凭这些玩艺儿换饭吃的,甚至于还有发达的。王雨田、王葵英父女就仗着抖空竹维持全家生活。有那种艺术,平、津、沪、汉、济等地,也能受人欢迎。若是身无一技之长,没有饭吃,怨天怨地说没有出路,那可是白说,饿死也没人可怜。有种本领,小则养身,大则致富。养身容易,发达最难。发达的人,哪个也长得身躯胖大魁梧;可是,大脑袋,大脸盘,一定要学唱花旦,不挣钱,不成名,那就是自己的错误。总而言之,学什么行当得够什么材料。

当初北平说评书的有个顺桂全,专说《铁冠图》,这部书不叫座儿。他还收了个徒弟叫桂殿魁,桂殿魁学说《铁冠图》,起初还很高兴,说过几处不叫座儿,他扫了兴,也开了外穴(xué)(到外地去挣钱)。走到天津,在三不管才立住脚步。可是他也不说《铁冠图》,仗着他没学说之先练过杠子,有这种技能,在三不管打个场子,盘杠子,拿大顶,也能圆粘(nián)子(招徕观众),“挑(tiǎo)罕子”(江湖人管卖药糖调[diào]侃儿叫挑罕子),他哪天也能挣钱。在三不管市场发达的时候,看热闹的人们,看他练玩艺儿不要钱,买他的药糖才花几个铜子,又不冤人,何乐不为?那种生意,经过十几年的光景也不土(江湖人管把买卖做得没人照顾了调侃儿叫做土了;如能做得年代多了,总有人照顾,调侃儿叫不土)。不料三不管发达得过猛了,十几年的工夫盖了多少万间房,把空场都盖没了,杂技场越弄越少,游逛的人们越来越不顺脚,也日见稀少。有资产的人们虽然往那个地方投资,欲求获重利,却不研究此事,直到了衰落得不堪言状,也无人整顿。桂殿魁的生意也受了影响,他不由得开了外穴,到东三省去做生意。有人说他到了奉天买卖不好,郁闷生疾,土(死)在那里。是与不是,我老云没到那里,不得而知了。桂殿魁有一技之长就能在外谋生,一辈子没有成名,没有发达,也是自己的错误了。





江湖中的光子(拉洋片)生意


拉洋片的,玩西湖景的,江湖人调侃儿管他们叫“光子”。拉洋片的家伙种数太多。像一个洋片箱,上边安块大玻璃,里边有七八张片子,底下有四个玻璃镜的,说行话管这种家具叫四开门,四开门是光子行的普通的家伙。拉洋片的艺人对于说唱引人,使用四开门,是人人能成的。至于挣钱多寡,也由其本领而定也。天桥大金牙、小金牙使用的洋片箱子,底下有八个玻璃镜,要兜搅生意,能每回让八个座儿,挣八个人的钱。说行话管他这八个镜的洋片箱子叫做八开门。他们这八开门的箱子,非得是光子行的头路角才能使用。本钱虽大,受的累虽大,挣项也比四开门大几倍呀。

有一种洋片箱子,上边有几个洋铁片制造的小人,箱子上边有个水漏子,箱子底下有个煤油桶,桶内盛着凉水。如若做生意的时候,得用水罐子由煤油桶内往水漏子里灌水,那水顺着一根绳流入管内,凭他水的力量就能催动了那洋铁片制造的人儿在上边乱转。光子行的人管这种家伙叫做水箱子。里边装的不是片子,也是一套套的小人。有人看时,全凭他扯起走线绷簧,叫小人来回乱动,他们那水催人动的玩艺儿叫《水漫金山寺》,仗他圆粘(nián)儿(招徕观众)。他们唱的曲儿是死套子,都唱那一套,我老云也录他们一段儿:“众位看那上边,漂漂悠悠来了两只船,船上头站着是许仙。许仙游湖来望景,偏上天降大雨,青蛇白蛇,船上头站,许仙搭船来借伞。那张天师撒开了张手雷……”他随唱随说,两只手还不住闲地扯那绳儿,叫箱内的小人随他唱的曲儿动转。唱到下雨的时候往箱内灌水。看的主儿,也见箱内流水,如同下雨一样。他唱到张天师撒开了张手雷的时候,用手猛一扯那粗绳儿,箱内有个鼓,也敲打一通,轱辘辘真响。跟着又唱什么“蛤蜊精、鲇鱼精、鲤鱼精、蛤蟆精……”他随唱随闹鬼儿,叫人瞧着他怪可乐的才能挣钱哪!

拉洋片的也有伙计掌柜的,掌柜的花几百元制几幅洋片,赁给伙计使用。其赁价无一定,由伙计每日挣钱多寡,三七分账。洋片行的掌柜的也如一小资本家也。惟有使水箱子的艺人,不能净仗家伙挣钱,引人圆粘儿,全仗他那滑稽曲儿,周身乱动,挤鼻弄眼,使人发笑得发托卖像(假装愣头愣脑,怯头怯脑)。凡是这种人有了技能,多不愿给人当伙计,个人弄份水箱子,足能糊口。故光子行使水箱子的艺人多是独立支持的。

光子(拉洋片)行掌柜的所制洋片,多是片车子,其形式系一长方箱子,上中下分为三层,每层可置八张洋片,上中两层,明显在外,最下层用箱罩着,使玻璃镜八个,箱前放四条小凳,每条可坐二人。做这种片车子生意,至少得两个人,一左一右,在左边的人手持一张洋片,唱两句,再将片放推进去。右边的人拿着一片,唱两句,将片推进去,所唱的都是死套子,什么“哎!这一张照的是,小马五儿《纺棉花》,多么好喂看!隔着那显微镜一照啊,亚赛真人呀,一个样般”。他们的洋片箱子、小凳儿,虽是山东德平县制造的;那廿四张片子,可都是照像馆的相片。其尺寸大小,大约着是一尺片子。我老云问过他们,为什么拉洋片的这行儿说行话叫光子?据他们说,江湖人管玻璃镜调(diào)侃儿叫光子,洋片箱上边是大块玻璃,下边是大块玻璃镜,我们这行离开了玻璃不行,因此才叫光子。言其是玻璃能透光是也。我老云说,照像馆离开玻璃也不成,片车子的片子是照像的材料,可以说是光子里的光子。在清末民初之时,小马五能唱《纺棉花》,社会里就轰动了,片车子的箱子都有一张小马五纺棉花,也能兴旺了一阵。到了民国十年前后,《纺棉花》渐渐落伍,片车子也渐渐落伍。前些年,天桥、东西两庙都有这种洋片。如今小马五没有了,北平各市场、庙会也见不着这种洋片了。据我老云向光子行人打听,他说这种片车子在平津一带不能挣钱了,如今都带着家伙往乡下去“顶神凑子”(赶庙会)去啦。

做光子行片车子的艺人,不知道随时改革,不知道随着社会风气演进,直到落伍了,才背着家伙到乡间去,实行去吃“科郎(kē lang)点”(农人)。十数年的光景,片车子就落了伍,社会的演变有多么得快!够多么得可怕!倘若老云有一日落了伍啊,吃科郎点也怕不成的。

有几种洋片箱子做的形式好像火车头,好像火轮船,他们光子(拉洋片)行人管那东西就叫火车头、火轮船。可是做这几样生意的艺人必须长得怯头怯脑,唱起曲来得有身段,得有发托卖像(假装愣头愣脑,怯头怯脑),连唱带抓哏(笑料),招惹得观众像看怪物一样,才能挣钱。不怪那江湖人常说:艺人要挣钱,不占一率,得占一怪!拉洋片的,怪也能占了上风,滑稽玩艺儿不分优劣,都有人欢迎的。

在我老云在学校里读书的时候,常见有些拉洋片的使四开门的箱子,带彩张儿。阅者要问什么是彩张儿?就是他们那几张洋片里夹着一张《杀子报》,每逢有人看洋片看到这张的时候,拉洋片的就拿起那铃铛板,板是木质,约有一尺大小,上有八个小铜铃铛,洋片箱子上有一方孔,大小也和板儿一样。他要变彩片时,将板往方孔上一盖,立刻就变样。在没变之先看那片上画的是:王徐氏身穿重孝,在灵前哭祭其夫。及至他盖板儿立刻就变了,王徐氏将一绺头发含在口中,手执钢刀一把,手起刀落,将他儿子的人头砍下,红光崩现,血水直流。他未盖板的时候,还有一套词儿。我还记得那词是:“这一张是《杀子报》,亲妈害亲儿子,我这铃铛板不叫铃铛板,叫做阴阳板,只要将阴阳板一盖,立刻就红光崩现,血水直流,王官保的人头落地。”在早年,凭他这张彩片儿就能有人看。到了如今电影儿都改了有声片子,泺(luò)州影(皮影之一种)落了伍,谁还看他那洋片的彩张儿。

在早年拉洋片的人们使用四开门的箱子,在七张片子里还夹一张春页子,有些人看他那春页都觉着很奇怪,一传十,十传百。还有没品行的人,专爱看那“袢(pàn)簧儿”(江湖人管那春宫调[diào]侃儿叫春页子;管那男女的私事调侃儿叫袢簧儿,又叫袢托)的事。大人看那坏片子不大要紧;惟有一般才开知识的小学生看那春页、袢托的片子,实是有伤风化,引诱少年娃娃学坏。后来闹得官家知道了,将那片子给“卯啦”或“淤(yū)啦”(江湖人管军警机关取缔他们调侃儿叫卯啦,把他们轰了调侃叫淤了),才见不着那宗东西。

光子行的玩艺儿到如今简直是落了伍啦,就以天桥说吧,除非大金牙的徒弟小金牙,以半春半柳(江湖人管随唱随抓哏逗笑儿调侃儿叫半春半柳)的艺术,使用八开门的洋片箱子,还能够挣钱,其余的干这行的,连啃(kèn)都保不住了(江湖人管不能糊口调侃儿叫保不住啃了)。我说,艺人挣钱的本领还是仗着艺术,若仗着家伙是靠不住的。江湖的老合(闯江湖的)如不相信,你看看大金牙、小金牙就知道了。





第五章 保镖卖艺


江湖之点挂子(受过训练的练把式卖艺的人)


在各市场庙会有练把式卖艺的,江湖人调(diào)侃儿叫他们为“挂子行”。有一种练武术的人到了无事可做的时候,就要撂场子卖艺,虽说是“人穷了当街卖艺,虎瘦了拦路伤人”,这种人到了玩艺儿场练把式,脸上还带着一种羞惭的样子,练的时候还是真卖力气,练的时候真有人看,练完了要钱,看主都走啦。这叫,净练不说傻把式。看起来平地抠饼(没有本儿要凭真本事挣出钱来),素手求财,是不容易呀。以上这种情形,阅者在这生计艰难的时代是时常看见的。敝人曾经调查,凡人要是干这打把式卖艺营生的,按着江湖的规矩,得拜个老帅(即是拜师),受老帅的夹磨(jiá mo)(受训练调侃儿叫受夹磨),等到夹磨成了,才能馈得下杵来哪(即是能挣钱哪)!

凡是有夹磨的挂子,若是到了各省县市、商埠码头,一到市场上打地,得打得出地来。按各省市的杂技场都有一种摆地之人,他们先将地皮租好,做些桌凳,若有江湖艺人要撂地做生意,得先找摆地的和他商议好了,每天在他的场子做生意,要用多少桌凳。江湖人管找这种摆地的人叫打地。将地打好,每日做生意所挣的钱,是和摆地之人二八下账。譬如挣一元钱,得给他摆地的两毛钱。这摆地的人吃这碗饭也不容易,他得懂得江湖的规矩,生意人谁有挣钱的能耐,谁的能耐软弱不能挣钱,素日得有个耳闻。要不明白这些事,有几个场子,都打给没能耐的了,虽然二八下账,也下不了多少钱哪!在吃江湖饭的老合(江湖艺人),第一的能耐是先学打地。如若打着好地,圆粘子(招徕观众)也容易,挣钱也容易;若是打不着好地,圆粘子也不容易,挣钱也难。江湖人常说:“生意不得地,当时就受气。”无论多大的能耐,如若不得地,也是枉然。可是生意人要到了打地的时候,眼睛得管事,瞧得出地势如何才成哪!

在各市场庙会有练把式卖艺的,江湖人调侃儿叫他们为“挂子行”。他们得嘴里有把式、身上有把式才能挣钱哪。



吃挂子(受过训练的练把式卖艺的人)行儿,江湖管他叫武生意,得离没有锣鼓的文生意远些,才能做买卖哪。傻练把式的连这种情形都不懂得,哪能平地抠出饼(没有本儿要凭真本事挣出钱来)来呀?挂子行的人将地打好了,到了游人最多的时候,师徒们扛着刀枪靶子到了地内,将刀枪架子支好喽,不能净说不练,得先大嚷大闹的招来人看,调(diào)侃儿叫诈粘(nián)子。等到有人围着瞧啦,才能练点小套子活儿,把人吸住了,四面围得里三层外三层,才算粘子圆好了。圆好粘子,就得使拴马桩儿(用话留你,让你走不了),用话将围着瞧的人们全都拴住了,没有走的人啦,才能练可看的把式哪!什么空手夺枪啊,单刀破花枪,拐子破棍,练完了要钱,才有人往场内扔钱哪。

他们得嘴里有把式、身上有把式,才能挣钱哪。身上有把式是挣钱的真功夫;嘴里有把式是能说会道好圆粘子(招徕观众),使拴马桩儿(用话留你,让你走不了),往下馈杵(要钱)。他们嘴把式调(diào)侃儿叫钢口(说话的技巧和分量),他的钢口差不多都是那套老词,作者录下套来贡献阅者参考。录之如下:“净说不练那叫嘴把式,尽练不说那叫傻把式,若要是连说带练,练到了,说明了,好叫人爱看。我们可不敢说练得好,是才学乍练,练得好,练不好,众位包涵着瞧。我们爷几个是才来到此地,实在眼拙,不知道哪位是子弟师傅。如若知道了子弟老师们住在哪里,必然登门拜望。今天我们俩人要练一套单刀破花枪,众位看他那条枪怎么扎,我怎么冒险进招。常言说得好,大刀为百般兵刃之祖,花枪是百般军刃之鬼,大刀为帅,棍棒为王。救命的枪,又好赢人,又好护身;舍命的刀,练的时候,我得舍出命去,练得叫众位瞧着得拍巴掌叫好!好!好完了怎么样?得跟众位要几个钱。住店要店钱,吃饭要饭钱。上有天棚下有板凳,官私两面的花销。我们练完了,众位大把地往场内拽(zhuāi)钱,你明理,我沾光。我们不恼别的(要使拴马桩了),就恼一种人,他早也不走,晚也不走,到了我们练完了,一腔子力气卖在这里,他转身一走,饶着不给我们钱,还把花钱的挤走了。这种人好有一比。”说到这里,他那伙计必问:“比作什么?”他接着说:“就比作我们弄熟了一锅饭,眼瞧着饭到口啦,他走如同往饭锅里给我们扔一把沙土,简直得缺了德啦!我们也不说什么,挑刺碍好肉,说他们叫好人难受。我们可不是都要钱,也不恼人白瞧白看。家有万贯,有一时不便。赶巧碰着没带钱,你只管放心,脚底下留德,给我们多站一会儿,给我们站脚助威,我们要多看你一眼,如同看我们的家堂佛,瞧他祖宗哪!话,我们是交代完了,再托付托付。我们练完了,大把往里扔钱的,我作个揖!我们练完了,没带钱的,给我们站脚助威的先生们,我给作个揖!那早不走晚不走,我们要钱他才走,脚底下不留德的人(说到这里愣一愣,用眼睛往四外看一过儿,接着又说),我也给他作个揖!我也不说什么,叫他养儿养女往上长。话是说完了,拿起来就练。”

两个人练的功夫娴熟,这套功夫,能够人人叫好。练完了,按着规矩将刀枪往场内一横,说:“我们要钱了!”这时候便有些看热闹的人纷纷往场内扔钱,他们挣钱多寡,那就看他们杵门子如何了,他们江湖人管练玩艺儿的人练完了要钱调(diào)侃儿叫杵门子。这杵门子硬胜似好功夫,功夫虽好,杵门子软也是白费力气。他们管头一回有些看热闹的人给钱调侃儿叫“头道杵”。要完了头道杵,又叫小孩拿着小笸箩,或是拿着小茶碗,围着场子向观众要钱,调侃儿叫“托边杵”。阅者常见他们把式场内有个小孩子,卖艺的人用一根木棍儿往小孩脖子后边一横,把小孩的胳膊腿儿往棍上一别,别好了之后,卖艺的人用脚踏着小孩,那种状态使人看了怪可怜的。卖艺的人踏着小孩,乘着人可怜小孩的时候要钱,这回要的钱,调侃儿叫“绝后杵”(最后一笔钱)。要完了这回钱,看的人全都走啦,再要钱也没有人啦。在他们卖艺的人要钱的时候,嘴里直说:“我们要钱啦!还有哪位!”江湖人管他们不住问地要钱调侃儿叫“逼杵”。最有能耐的人逼杵的时候,能够说几句话就有人往下扔钱,调侃儿叫“使钢口”(说话的技巧和分量)。钢口也有软硬之分,与杵门子软硬相同也。卖艺的使小孩子做出一种可怜样子,是要钱的门子,不知者都替小孩难过,其实那小孩并不难过,那孩子故意做出可怜样子,叫人看着可怜,好往他们场内扔钱。那个小孩在家中是受了夹磨(jiá mo)(训练)的。

卖艺也有练过尖挂子(管真把式叫尖挂子)的,不过是少有,还是腥挂子(假把式叫腥挂子)居多。有些个成了名的江湖艺人,据我调查得来,凡是成了名的卖艺之人,论把式全是尖腥两样都会。所以老江湖人常说:“腥加尖(假的加真的),赛神仙。”那话是不假的。不仅于卖艺的是腥加尖,许多的生意行当都是有真有假。社会里的事儿,也未尝不是真真假假呀!





挂


“挂”是挂子(受过训练的练把式卖艺的人)行,在早年都称为“武术”,俗称为“把式”,又称为“夜叉”行。现今提倡保存国粹,各省市都设立国术馆,唤醒国人,共倡武术,改为“国术”矣。国术的范围是很阔大的。国术的传流,门户的支派,也是复杂的。好在敝人不是谈国术,是谈江湖艺人的“挂子”行儿。

挂子行儿分为几种:有“支”、“拉”、“戳”、“点”、“尖”、“腥”等等的挂子。管护院的调(diào)侃儿叫“支”,管保镖的叫“拉”,管教场子叫“戳”,管拉场子撂地儿卖艺的叫“点”;又有“尖挂子”、“里腥(lǐ xing)挂子”两支分别。

什么是“尖挂子”呢?据江湖艺人谈,真下过些年的功夫与得着名人真传的把式调侃儿叫“尖挂子”。像那打几趟热闹拳的把式,刀枪对战叮当乱响熟套子的把式,只能蒙外行的把式,调侃儿叫做“里腥挂子”。

又有打“清挂子”的与“挑将(tiǎo jiàng)汉儿”的分别。什么叫打“清挂子”呢?凡是江湖艺人在各市场里、各庙会里拉场子撂地儿,净指着打把式卖艺挣钱,叫做“清挂子”。如若打把式卖艺的还带卖膏药、卖大力丸的生意,不能算是清挂子,那算是“挑将汉儿”的。在挂子行里的各种生意,就以挑将汉儿的这种买卖难做。第一是干这行生意得“人儿压住点儿”(凡是打把式卖药的人,必须长得身躯高大,相貌魁梧。哪怕武艺不好哪,凭他那个威武雄壮的人样子往场内一站,让人瞧着他好像是有点真功夫似的。管他这人样子能镇得住人调侃儿说叫真压点儿),第二得练过些年“尖挂子”(受过训练的练把式卖艺的人),或是会使几样儿“样色(yàng shǎi)”(能挣下钱的物件),然后才能做得了这种生意呢!

把弹弓上的球儿打出去,如同一条线儿似的,先打在茶碗底上,打不坏茶碗。把茶碗上的泥球打飞了,飞起来的球儿,能把茶壶嘴上的球儿打掉了,不惟茶壶嘴儿打不坏,茶壶嘴上的大铜子儿还不能打下来。这手功夫有个名儿,叫“弹打弹儿”,又叫“球打球儿”。



敝人常见玩艺儿场里有些打把式卖药的生意人,把药案子在场内支好,上边陈列好喽所卖的药品,什么大力丸哪,百补增力丸哪,海马万应膏啊,虎骨熊油膏啊,摆满了案子,到了游人多的时候,先在场内练几趟拳脚,活动活动腰腿,练到他的场子站满了人啦,算是“圆好了粘儿啦”。在这个时候,若是练过“尖挂子”的,就在场内好好练趟惊人玩艺儿,叫观众瞧得人人佩服。练完了这套功夫之后,得先用拴马桩儿(用话把人扣住)把人拴住了,全都不走了,才能做买卖哪!他们使的拴马桩儿是用弹弓子打几手弹子,不论是立着打,躺着打,蹲着打,叫人瞧着不错啦,他向观众说:“我今天练一手儿特别的功夫。”说着,他在案子上摆一把瓷茶壶,在茶壶嘴儿上放一个大铜子,铜子上放个泥球蛋儿,在茶壶前边放个茶碗,要底儿冲天,然后在茶碗上放一个泥球蛋儿。他用手指着这东西说:“今天我练这手功夫,是用我这弹弓子把弹弓上的球儿打出去,如同一条线儿似的,先打在茶碗底上,打不坏茶碗。把茶碗上的泥球打飞了,飞起来的球儿,能把茶壶嘴上的球儿打掉了,不惟茶壶嘴儿打不坏,茶壶嘴上的大铜子儿还不能打下来。这手功夫有个名儿,叫‘弹打弹儿’,又叫‘球打球儿’,平常日子还不练这手功夫。今天众位来着啦,我练练这手儿,叫众位给我传个名。回到家去,你就说×××的弹弓儿打得最好。”说着把弹弓拿在左手,右手拿起泥球儿,往弓弦上一填,拉开了弓,作出欲打的姿势。围着瞧着的人还以为他要练这手功夫,其实他不练了,不过引人的好奇心胜。要瞧他真练哪,那辈子见吧!他用这手功夫把人拢住了好买他的药哪,这叫使“拴马桩儿”。说着,他又不打啦,向观众说:“我要练好喽,弹打弹儿,球打球儿,茶碗不碎,茶壶嘴儿不坏,使众位拍巴掌,给我叫几声好儿,使大劲拍巴掌,大着点劲叫好儿。说好……好完了怎么样?大概你许是要几个钱吧?众位放心,我若一要钱,是跟我祖宗要钱哪!咱们是分文不取,毫厘不要。练好了,众位给我传名,众位可别给我传这弹打弹儿的名儿,要传名你给传这个名。”说着把弹弓子往身上一背,伸手从他的案子上拿起一大包膏药来说:“众位要传名,您就说×××的膏药最好。咱们这膏药可不卖,当初这是我们练功夫的人要有个磕着、碰着、闪腰、岔气的时候,练不了把式啦,只好贴上这膏药。不论是腰疼、腿疼、筋骨麻木、跌打损伤,贴上咱们这海马万应膏,能够顺着周身毛孔舒筋活血,立时止住了疼痛!那位说,你这膏药卖多少钱一张啊?您要买我可不卖,少时间我把这手功夫练好喽,每人我送给一张,自己有病自己贴,没病送给别人。那位说,你这膏药里都有什么药材呀?这里头没有珍珠、玛瑙,没有麝香面子,老虎×,就有几十味草药,有麻黄、乳香、没药、千年健、入地风、木瓜、地骨皮、防风、透骨草、川牛膝、杜仲、广木香、羌活、当归、川芎、沉香,值钱的东西就一味海马。这十几味药,用香油、樟丹文武火熬成了,效力最大。光是我自己说好不算,卖瓜的不说瓜苦,卖酒的不说酒薄。众位如其不信,咱们当面试验。”说着话把膏药放下,又从案子上拿起一个大铜子来,向观众说:“咱们这药不只能治腰腿疼痛,还能治食积、奶积、大肚子痞积、跑肚子拉稀、红白痢疾。这药能化痞积。众位如不信,咱们试验试验。把这个大铜子儿放在膏药内,用不了一袋烟的工夫,能够凭膏药的力量化成末儿。”说着,他由案子上又把一沓儿膏药拿起,约有二十多张吧,他嘴里说着向观众张罗,说:“真金不怕火炼,好货不怕试验。哪位伸把手儿,从这膏药里给我挑出一贴膏药来,我要自己拿出一贴来不算。哪位拿吧?”说着把膏药送在众人面前。有那爱管闲事的人给他拿出一贴膏药来,他左手拿着那一沓子膏药,右手接过这一贴膏药,走至他的案子,把一沓膏药放下,拿起火纸点着了,把这张膏药烤开了,当着众人把铜子儿放在膏药油内,然后把膏药并上,放在案上,他又向观众说了不到几句话的工夫,再把膏药打开了。举着膏药在场内绕一匝儿,叫众人上眼。大众一看那铜子没有啦,膏药里有不少铜末子。当场试验,谁也得佩服这膏药的力量。

在数年前,敝人还很信以为真,想他那膏药很有力量。到了如今,我可不相信了。原来他们用膏药化铜子儿的方法,也是江湖术中的“样色(yàng shǎi)”(能挣下钱来的物件)。使这“样色”,必须先在药铺里买点自然铜(这种自然铜的性质如同铜一样的,买来的时候净是小块儿,这种东西用手一捏便成铜末)来,事先把那自然铜放在膏药之内,把这张膏药弄好,放在案上。等到有人再给他由一沓膏药里拿出一张来,当着众人把铜子儿放在膏药内。挑将(tiǎo jiàng)汉儿的在这时候如同变戏法儿似的,将有铜子的膏药与有自然铜的膏药弄在一处,一翻个儿,把那有铜子的膏药掩藏起来,把有自然铜的膏药打开了,叫人瞧看铜末子。江湖人管这偷梁换柱的法子调(diào)侃儿叫“翻天印”,管这种“样色”叫“丁把(bǎ)儿”。还有一种用膏药化瓷的,也是在药铺里去买“海螵蛸(piāo xiāo)”(乌贼鱼骨)。海螵蛸这种东西,要弄碎了,其质色白,真像破瓷器一样。事先把它做好了,放在一包破瓷之内,由包内取出来,谁也瞧不出破绽来,放在膏药内,用手指头微须一掐便成末儿,这种“样色”调侃儿叫“丁老骨儿”。当他们把“样色”使完了的时候,向观众说:“今天试验完了,不白试验,每人我送一张。”说着他从案子上拿起他的门票说:“哪位若是要我的膏药,哪位伸手先接我一张门票。我可先交代明白,小孩子不送,聋子、哑巴不送,因为他们不能给我传名,多了不送,就送二十份。今天的人可是太多。有接着的,有接不着的,接着的也别欢喜,接不着的也别烦,哪位要哪位伸手。”说着他就散他那门票。世上的人都是贪便宜,白给一贴膏药谁不伸手?当他散发门票的时候,人人都抢着接,眨眼之间二十张门票散完了,他又有一遍说词:“先向大众说,我这人也不是傻子,有膏药白送,这是为的传名。常言道:小不去,大不来,名不去,利不来。今天我送膏药,可有个拦避(bǎn)墙儿(前提),要不然他拿这药不当回事。要买我这膏药,是两毛钱一张,今天我就卖二十张。卖多少钱哪?两毛钱改为一毛,一毛改为半毛,半毛钱是我的本儿。哪位说,你不是白送吗?送是一定送,可不能白送,哪位要买我一张膏药……”说着话一跺脚,狠狠地道:“我再白送一张。我这叫买一张饶一张,可是没接这门票的不卖,要买也成,你掏两毛钱。不论腿疼腰痛,筋骨麻木,闪腰岔气,红白痢疾,贴上这个膏药就好;贴不好来找我,管保退钱。贴不好你不来找我退钱,那算您怕我。半毛钱一张,我要赚了你的钱,叫我抛山在外死不归家。”他这是和没起誓一样,他们江湖人管“拉屎”调(diào)侃儿叫“抛山儿”。他说抛山儿在外,屎不归家;观众听着是死在外头他回不了家啦!没听清他说死咬成了屎字的音儿,拉出来的屎哪能回家呀!他们管起誓调侃儿叫“劈雷子”。挑将(tiǎo jiàng)汉儿的劈完了“雷子”,那买主便相信不疑的,每人掏半毛钱买两张膏药而去。据他们江湖人讲,先说白舍后要钱的手段,调侃儿叫“鬼插腿儿”。先给一张门票后说卖,调侃儿叫“倒插幅子”,合计起来二十张膏药卖了一块大洋,论“笨头”也不过一毛多钱。他们管本钱调侃儿叫“笨头”。一天卖这么几回,吃喝不用愁了。

敝人曾调查过,他们这膏药不是香油煎熬的,是桐油熬的,他们管使桐油熬的膏药调侃儿叫“南底”。这种“南底”的膏药,要贴寒症,还是真有效力的。不过,熬不好的就贴不住,会弄得浑身是膏药油子,叫人疑为无用的了。

挑将汉儿的人们所练的,都是半尖半腥(半真半假)的挂子。惟有镖行的人练的把式,都是尖挂子(真会武术的人)。凡是练武的人将武艺练成了,无论是保镖去,护院去,得重新另学走闯江湖的行话,把行话学好了,才能出去做事呢!遇见事的时候,一半仗着武功,一半仗着江湖的暗话,才能走遍天下呢。

在昔时,水旱交通极不便利,买卖客商往来贩卖货物的,离不了镖行。就是国家解送饷银的时候,也是花钱在镖局子雇用镖师护送的。在那个时代开个镖局子也很不容易。头一样,镖局子立在哪省,开镖局子的人得在这省内官私两面叫得响;花钱雇佣真有能耐的教师充作镖头;没做买卖之先得先下帖请客,把官私两面的朋友请了来,先亮亮镖。凭开镖局的人那个名姓儿就有人捧场才成哪。若是没有个名姓,再没有真能耐,不用说保镖,就是亮镖都亮不了。自己要逞强,亮镖的日子非叫人给踢了不可。立住了万儿(名儿)的镖局买卖也多,道路也都走熟了,自然是无事的。最难不过的是新开个镖局子,亮镖的日子没出什么错儿,算是把买卖立住了。头一号买卖走出镖去,买卖客商全都听见声儿,要是头趟镖就被人截住,把货丢了,从此再也揽不着买卖了,及早关门别干了。这头趟镖出去,镖师带着多少伙计出去,把客人财物放在镖车之上,插好喽镖局子的旗号,一出省会地方,镖车一入“梁子”(即是入了大道)伙计们就得喊号儿,伙计们扯开了嗓子,抖起丹田气来喊“合吾”!这合吾两个字,是自己升点儿(有了响动),叫天下江湖人听。“合吾”,合是“老合”,凡是天下的江湖人,都称为“老合”,喊这两个字儿,是告诉路上所遇的江湖人哪:吾们是“老合”!喊这两个字喊到吾字,必须拉着长声。走在路上凡是拐弯抹角也得喊,遇见村庄镇市也得喊。尤其是遇见了孤坟孤庙或是离着村镇不远有座店,或是有家住户,更得喊号。因为孤坟里埋的不是棺材,十有八九都是贼人走的道儿。孤庙里的僧道虽出家,也未必都是真正的出家人,十有八九,都是“里腥(lǐ xing)化把(bǎ)”(即假和尚)。离着村镇附近有孤店,有独一家的住户,那也是“三(sǎ)应(散落的意思)跺齿窑”儿,跺齿窑儿就是匪人潜伏的下处。

镖局子伙计走镖的时候,都得喊镖号,惟独到直隶沧州不敢喊镖趟子。若是不喊就许安然过去,如若一升点儿,任你有多大的能耐也得出点舛错的。在我国清末时候镖车过沧州还是那样呢!因为沧州那个地方,不论村庄镇市住的人,老少三辈没有不会把式的。到了如今,新科学武器发明了,沧州练武的人是日见稀少了。当镖师的带着一拨伙计出去走镖,每逢出了镖局,拉着马匹不能乘坐,遇见了熟人都得打个招呼。镖车走出了省会地方,他才能上马呢。镖车走在别的省会地方,要有镖局子,镖师也得下马,伙计也得跳下车来,和人家打过了招呼,然后过去,才能上车上马。镖车上的大伙计走在路上虽然是耀武扬威,两个“招路”得会“把(bǎ)簧”。招路是眼睛,把簧是用眼瞧事儿。镖行人常说当大伙计不容易。骑着马拿着枪,走遍天下是家乡。春点(江湖艺人所用的术语)术语也得讲,跨着风子(即是骑马)得把簧。镖车走在路上瞧见了孤树,大伙计得喊“把(bǎ)合着,合吾”。如若遇见了桥,得喊“悬梁子,麻(mā)撒着,合吾”。如若遇见路旁有个死人躺着,得喊嚷“梁子土(死)了点的里腥(lǐ xing)啵把合着,合吾”。如见对面来人众多,得喊“滑梁子人氏海(hāi)了,把合着,合吾”。如若走在村内,得喊“窑里海(hāi)梁子,把合着,合吾”。如若瞧见有山,得喊“光子,把合着,合吾”。如若过河登船时,得喊“两边坡儿,当中漂儿,龙宫把合着,合吾”。如若遇村镇有集场,得喊“顶凑子(集市)掘梁子,把合着,合吾”。如若遇见庙会有香火场儿,人太多了,得喊“神凑子(大庙会)掘梁子,把合着,合吾”。

这初次走镖,有那江湖绿林人知道了,他们要试试这走镖的人是行家子不是。他们知道镖车从哪里走,在哪里截车。两下里对着一把簧,彼此升点儿(互相捧),一问一答对难为。大伙计把问答的话说完了,必须问他们:“祖师爷留下了饭,朋友你能吃遍?兄弟我才吃一线(即是指着天下一股往来大道而言),请朋友留下这一线儿兄弟走吧!”等到了这样话说出来,他还不闪开,讲不了就得动手啦。若是久干江湖绿林的人,无论如何也不能翻脸动手的。可是初出茅庐、才进芦苇的人,他可不听这套,非得镖师尖挂子(真有本事的人)把他赢了才能算完。要不然当镖师的没有尖挂子干不了这行呢。倘若是镖车走在路上遇见了劫镖者,以江湖术语打不动他,讲外面的朋友话也不成,镖师就喊嚷一声:“轮子盘头,各抄家伙,一齐鞭托,鞭虎挡风!”伙计们听镖师喊嚷“轮子盘头”,他们赶紧把所有的镖车往一处盘个大圈儿,有抄家伙保住镖车的,有抄家伙准备打人的,镖师喊嚷“一齐鞭托”,就是大家打他吧!“鞭虎挡风”,是动手把贼人打跑喽,只可惊动走啦,挡过风去就得了,不可真把贼人“青了”(即是别杀了他们),也别“鞭土(死)喽”(即是别打死他们)。若是镖师仗着尖挂子把贼人惊动走了,大伙计就得喊嚷一声:“轮子顺溜了,合吾!”镖车走开了,镖师一上马,押着镖又走下去啦。

若是镖车进了店,店门外插着镖旗,院内放张桌子,一个凳子。大伙计在凳上一坐,指挥着伙计,把镖车都安排好啦。然后,大伙儿净面掸尘,喝茶吃饭,喂完了牲口,前后夜伙计上了班啦,大伙计才能歇着去,值更的把店门外的镖旗撤下,另换镖灯,镖车上也都插上小灯笼。然后按着更次,一人喊号大家轮流着喊,如同古时候军营里喊筹一样。值更的伙计也有头儿,到了夜间也得眼观六路耳听八方。凡是贼道能出入的地方,更得格外留神。这是住了熟店,准知道这店是干净窑儿。如若住在生店,不知道窑里干净不干净,镖局子的伙计得把屋内桌底下,床榻底下,假装打扫,瞧瞧有地道没有。如有地道,便是贼店,赶紧得回禀镖师,请示他的办法。院内有井,或是有锅灶、柴禾垛,都得“把”合到了。关于这些事,都是镖师训练他们的。譬如房上来了人啦,打更的就得冲着房上说:“塌笼(江湖人管房子调[diào]侃儿叫塌笼)上的朋友,请你下来搬会儿山儿(即是来呀,咱们喝点酒啊),啃(kèn)个牙淋(yá lin)哪(即是叫他喝碗茶呀)!”房上的人如不听这些事儿,一语不出,值更的就得喊嚷一声:“塌笼上的朋友,走遍了天下路,交遍了天下友,祖师爷留下这碗饭,天下你都吃遍?我们吃一线的路儿,你去吃一片,留下这一条线的饭我们用吧!”如若贼人在房上还是不走,或是越来越多,值更的就得喊:“倒(dǎo)(东)、切(qiē)(西)、阳(南)、密(北)四埝(方位)的伙计都出来,亮青子挡风!”他们在店内住下没事便罢,此若有事,应当东西南北各占各方,准备着动手。东边的伙计,得知他们是“倒埝”的差事;西边的伙计,得知他们是“切埝”的差事;南边的伙计,得知他们是“阳埝”的差事;北边的伙计,得知他们是“密埝”的差事。如若值更的喊“倒、切、阳、密四埝的伙计都出来,亮青子挡风”,他们四面保护,动手的伙计就得抄起刀枪来,由屋里出来把东西南北的地方都占好喽。镖师从屋里出来,他再向房上的人说什么“人不亲艺亲,一碗饭大家吃”等等的情面话,这叫使“贴身靠儿”。倘若再不成,镖师就得问:“塌笼上(房子)的朋友,是一定破盘吗(即是非要抓破脸吗)?”房上人再不答言,镖师就得往当中一纵说:“既要‘破盘’儿,请下来开鞭吧(即是下来打吧)!”房上的人如若跳下来,四面的伙计就嚷:“上有天罗,下有地网!条子戳(用枪扎),青子青(刀对刀),要想扯活(chě huo)(跑)呀,休生妄想啦!”

这时候,无论来了多少绿林人,全瞧镖师的“尖挂子”(受过训练的练把式卖艺的人)“鞭上”(打的)如何了。若是镖师凭“尖挂子”把绿林人惊得扯活啦,然后,还得叫伙计各处“把(bǎ)合(看)”到了,防备贼人藏起来。要防备不周全就许“窜了轰子”(管有贼人放火调[diào]侃儿叫窜轰子)。各处都搜查完了,一齐喊嚷:“扫净了,合吾!”这才算化险为夷。

至于真要在路上被绿林人把镖车劫了,镖师得瞧得出事来。真要鞭不过人家,得藏起来保全性命;贼人扯活喽,暗中再跟下贼人去,认着了他们的窑儿(住的地方),好想主意把抛了的东西找回来。若是回到了店里,再有绿林人来呀,镖师鞭不扯活(打不跑)贼人,必被贼人弄得“挂了彩”(即是受了伤),或是“土了点儿”(即是弄死)算完。若是把贼人鞭扯活啦,还得留神,镖师得有走、不走的见解。如若得走,到了时刻,镖师喊嚷:“扯轮子”(即是套车)“趟(tāng)梁子”了(即是出店奔道走啦),“合吾”!于是伙计们套好了车,天亮了撤灯笼,撤店门的镖旗,收拾完毕,镖师出店前后一把合,东西和人俱都齐了,他就嚷声:“请客人迫(pǎi)轮子(管请客人坐车叫迫轮子)了,合吾!”车把式一响鞭子,喊起镖号,往外就走啦。走在路口的时候,大伙计得喊:“轮子调顺了,入梁子了(即是把车排顺了,进了道啦),合吾!”这路上可得留神那浑天入窑儿的(即是夜里进店抢镖的人)没得了手,难免他再蛊惑别人在路上劫镖。这要在路上见了人要劫镖,就不用跟他们客气。大伙就冲着眼岔(要劫镖的人)的喊嚷:“水浅了不了嗣,是肉有骨头,是鱼可有刺,是朋友躲开了,免得折(shé)鞭(管挨揍调侃儿叫折鞭)。”如若簧点清(见事则明与达时务的人调侃儿叫簧点清)的人,就不找麻烦喽。倘若遇见说什么也不成劫定了镖啦的人,免不得喊“轮子盘头(把所有镖车往一处盘个大圈儿),亮青子(刀)鞭托(打人)挡风(把人惊走)”,真得干两下子。新亮镖的镖局子、头趟镖走出去没出什么舛错,从此买卖上门,就算立住了“万儿”(名儿)啦。

镖师走完了这头趟镖,一路之上,没准交多少朋友,其中好歹贤愚都有,还得应付得法,事事周全到喽,提起话来说,“某镖局子的镖师谁谁是朋友”,立住了万儿,如同创下了江山一样,能吃长久了这碗饭,也实非容易的。





天桥内把式场


天桥是个五方杂处之地,藏龙卧虎之所。那里的人物最为繁杂,什么样的都有。挂子(练武术的)行的人也是好歹贤愚都有。

在早年有个花枪刘,带着两个姑娘在天桥卖艺。说江湖的行话,他们父女是“火穴大转(zhuàn)”(在这个场子挣了大钱了),很有个“万儿”(名儿)。如今可不知他父女都到哪里去了。

在天桥久占的把式场是弹弓子张,他叫玉山,在前清当过官差,后入江湖。据江湖人传言,他虽是做挂子行买卖,可是柳枝(随着野台子戏班卖膏药)的门户,与柳枝大将袁桂林是师兄弟。他在中年的时候身体灵,精神足,口齿伶俐,长于言谈。不止会打弹弓子、会武艺、拳脚好,他还得过正骨科的真传,凡是闪腰岔气,错了骨缝,经他手一捏就好,管保手到病除。江湖人都说他有几把“尖托”(管会接骨的妙法调[diào]侃儿叫托门,瞎捏不见效叫里腥[lǐ xing]托,管手到病除叫有几把尖托)。

他在天桥年代最久,我老云每逢到他那场子,必站住了把(bǎ)合把合(看看)。他的场内立根竹竿,上边悬着个小锣,能手持弹弓,扣上弹儿,横打、竖打、正打、反打、蹲着打、卧着打、仰面朝天躺着打,打出去的弹儿都能打在小铜锣上。在早年他做买卖的时候,每逢上托圆粘(nián)子(用手法招徕观众)引人,都是用弹儿打小铜锣。逛天桥的人们,听见了小铜锣儿当当的响,先掉瓢儿(先回头儿),招路把合(眼睛看),后过去观瞧。他瞧着场子人围严啦,就练好功夫。往案子上放把茶壶,嘴上放个铜钱,在上放个泥弹,用弹弓子打出去的弹儿,讲究能打落茶壶嘴上的弹儿,铜钱不掉,茶壶嘴不伤。每逢要归买卖挣钱啦,他就向观众说:“我今天练回弹打弹。什么叫弹打弹哪?众位瞧着,我用弓儿往天空上打出个弹儿,那弹往起去,我不等他落下来,跟着再用弓儿打出个弹去,后打出去的弹儿,追上先出去的弹儿,两个弹碰在一处,啪的一声,能叫后出的弹,将先出的弹打碎了!我要打好了,请大家给我喊个好儿。说练就练,净练这手不算功夫,我还练……”他说到这里,可不练弹打弹,叫围着的人们听着都不走,净等着瞧他练弹打弹。他用这个方法,将人吸住了不走,做他挣钱的买卖,等着将钱挣到手啦,然后再练弹打弹。我老云还瞧过几次,他那弹打弹的功夫,还是真准,百发百中。久逛天桥的人们,虽然知道他用弹打弹吸住了人使拴马桩儿(用话把人扣住),因为这类功夫颇有可观,都倾心愿意的不走,等很大工夫瞧他的弹打弹儿。他早晚准打,从不谎人,故此能够吸得住人。有些个练武艺的人常向观众夸说,他要练什么特别的功夫。招惹得观众不走,将腿也站酸了,钱他也挣足了,所说的功夫没练。那种情形,江湖人调(diào)侃儿叫“扣腥”,可是他们天天扣腥,叫久逛的人们都明白了,再扣腥就不成了,失去了信用。每到要钱的时候,观众就呼啦一散。受了会子累也挣不了钱,岂不是冤人自冤呢?我对于张玉山的弹打弹,临完了打一回叫人看看,不是净说不练,那才是地道的拴马桩儿。我们这话对不对?老合(江湖上各行业的人们)们闭目自思,自然明白。

张玉山生有二子,大的叫张宝庆,二的叫张宝忠。哥俩从小练的把式,以大刀为最高。据说那趟大刀是东城某有名武术家所传。



张玉山生有二子,大的叫张宝臣,二的叫张宝忠。哥俩从小练的把式,在民初的那几年,他父子上地(做生意)撂场子,两个人打对子。单刀破花枪、花枪破三节棍、空手夺刀,功夫烂熟,打得火炽,哪场玩艺儿也不少下钱。最美是他们哥俩练的大刀为最高,听说那趟大刀是东城某有名武术家所传。若练大刀,比练别的武艺格外多挣钱。他们爷仨的杵门子(到要钱的时候叫杵门子)很硬,是档子地道玩艺儿。

自从民国十年后,张玉山一个人在天桥做买卖,张宝忠弟兄就开了外穴(xué)(到外地去挣钱),往各处跑腿,到了张家口,他们响了万儿(名声大),“火穴大转(zhuàn)”(挣了大钱了)。至今张宝忠的哥哥还在张家口安座子(开药铺说行话叫安座子)哪!他的媳妇是唱竹板书关顺鹏的胞姐,夫妻和美,治家有道,在口上生活很是不错,我前年云游到张家口,还瞧见那买卖十分兴旺呢!

张宝忠在民国十五年后,才由张回平。他在早年是挂子(练武术的)行,如今是专门卖大力丸,他的场子在公平市场丹桂茶园后边。每天他在场内打拳、练鞭、弹弓、摔跤,足练一气。靠着他场儿的南边就是他的药铺,字号是金鉴堂,弹弓为记。据天桥的人们所说,他们卖的那药能有“回头点儿”(即是买过东西再来买),实在不易。张宝忠练的不是“腥挂子”(假把式调[diào]侃儿叫腥挂子),他还比别人多样本领,会摔跤,还摔得不弱。从前他有些傲气,近几年来有了阅历,谦恭和蔼,侍父能尽孝道,江湖人能够如此,实是不多呀!

孟继永是挂子行(练武术的)的人物,久在天桥撂地,他把式场从前在天桥公平市场。自从前年,迁到红楼南边。他是河北省武邑县人,六十多岁,身体强壮,性情直爽,人称为孟傻子。他圆粘(nián)子(招徕观众)的法子,用大白在地上画个人头,有耳、目、口、鼻,在耳、目、口、鼻上各放一个大枚,他往场内一站,手里拿着甩头一子(丈多长的绳儿,一头系个镖,武术家管这宗东西叫甩头一子),扯开了嗓子,喊镖趟子:“合吾……合吾……”逛天桥的人们围上了,他说:“我是镖行的人,在前清的时候保过镖,如今有了火车、轮船、邮电局,我们的镖行买卖没了,镖行的人,不是立场子教徒弟,便是给有钱的富户看家护院,我是拉场子卖艺。我拿的这个东西叫甩头一子,康熙年间浙江绍兴府有个保镖的叫黄三太,人称叫金镖黄,他是神镖胜英的徒弟,因为凑银子要给清官彭大人运动三河的县官,指镖借银,铁罗汉窦尔墩不借金银,反倒与他结了冤仇,在山东德州李家店,定下约会,两个人比武。黄三太用三支金镖、甩头一子赢了窦尔墩。三支金镖压绿林,甩头一子定乾坤,一口单刀纵横天下!今天我孟傻子练练这甩头一子。这个东西不用的时候往上一缠,用的时候一抖就开,远打一丈多,近打二三尺,用足登着绳儿打,叫狮子滚绣球;在腿底下转着打,叫张飞骗马;在胳膊上盘着打,叫盘肘;在脖子上绕着打,叫缠头裹脑。”他上边说着底下练着,一招一式,练得颇有可观,他练着向观众说:“我今天用甩头一子要打地上画着的人头,说打左眼,不能打右眼;说打右眼,不能打左眼,我打一回叫众位瞧瞧。”他说到这里,可不练了。把人吸引住,也是用拴马桩儿(用话把人扣住)。说到要打人头啦,他说到这里可就岔下去了。他说:“那位说啦,你使的这甩头一子,是什么人遗留的?这个东西是汉朝才有的,想当初王莽篡位之时,有奸臣羽党苏献奉王莽之命追拿刘秀,追到潼关外头,刘秀与他动手,未走三合,苏献将大刀一扇,刘秀的刀就撒手了,没有军刃不能动手,拨马逃走,苏献在后苦苦地追赶,急得刘秀心生一计,将他的丝鸾带解下来,下马寻石,找个石头,系在丝鸾带上,复返上马,苏献追到了,刘秀就用这个带子系石头将苏献打败,得逃性命。后人仿着他的意思,做成了甩头一子。别看这种兵刃不在十八般兵器之内,还是帝王留下的。今天我就用这甩头一子打一回试试,打得不偏不歪,请众位给喊个好!好,好,好,好完了!那位说,许是要钱吧?众位放心,我这个场子不要钱。练完了,我还每位送上一贴膏药。”说到这里又扯到膏药上,这就是“挑将(tiǎo jiàng)汉儿”的由练武说到卖药挣钱的“包口儿”(管这一大套做买卖话调[diào]侃儿叫包口儿),他在天桥有二三十年了,也卖艺,也卖药,糊口有余,也没有发达,平平常常而已。

他的徒弟叫姜兴周,也是武邑的人,有四十多岁,在红楼东南一带撂场子。每天与他两个儿子打把式卖艺。姜兴周不会卖药,说行话叫清挂子。人忠厚,克勤克俭,收入虽然不多,治家有法,粗茶淡饭,衣食不缺,与他师父大有不同。他的大儿子现在某银行,是“支杆挂子”,即是护院的;二儿子是个手艺行;三儿子、四儿子与他撂明地,干“点杆挂子”。除去他二儿子外,父子爷儿四个,都是挂子行,可是分为支杆、点杆,也大同小异也。姜兴周老来有子成器,晚景定然有靠,福禄加于勤俭人,治家理财,江湖人也要学的。否则,落个风流乞丐,终归也怕有衣食断绝之处。





挂子行之中的支杆挂子


武术一道是我国汉民族中的国粹。在古时,先是马战,后是步战,传到了唐、宋、元、明、清,普遍了全国,到处都有场子。不只是男子,就是妇女,也很有练把式的。直到清末,欧西各国新武器昌明,就是痨病人,若手持洋枪,搬动机簧,弹子打出,有霸王、存孝之勇也立时丧命,故新武器输入中国以来,人人皆轻视武术,很受重大的影响,几乎将特有的国粹失传了。现在国府当局为保存国粹起见,将武术改称国术,各省设立国术馆,极力地提倡。挂子行这几十年来如遇大劫,现在又盛行了。可见世上的事,有一兴必有一衰,有一衰必有一兴,循环不已呀!

现在国术虽然兴旺了,国术中的特长还是无人提倡。什么是国术中的特长呢?就是挂子行(练武术的)的规矩。评书上常说:人外有人,天外有天,把式多好,也难免不栽斤斗。要想由把式上成名立业,必须按着挂子行的规矩才能成哪!如其不然,有多大的能耐也难免叫人打倒。我老云在外边闯练这些年,很交了些个挂子行的朋友。山东的陈大鼻子、烟台的张王老师、北平的焦方桐,都和他们探讨过挂子行的规矩。可是挂子行的规律很大,我探讨得来的,也是有限。懂行的诸君可别笑话我说得不全。一个人知识原有限,天下事理本无穷。仅将我个人所知的写了出来,懂行的人,我在您班府门前耍回斧子;不懂的人们,是我贡献话料儿。

闲话休提。却说这把式行,在早年说行话,有明暗之分。什么叫明,哪叫暗哪?凡是偷盗窃取的朋友练的功夫,他们调(diào)侃儿说叫“暗挂子”,称他们为“黑门坎的人”。凡是练把式不偷窃的,为公当差应役,或是入伍,或当捕快;为私的或是保镖护院,或是立场子教徒弟,走闯江湖打把式卖艺,都叫“明挂子”。

就以护院的说吧,他们是专以保护富户人家不丢东西为目的;那黑门坎的朋友则专以偷窃富户人家为目的。他们这两种人虽然都是挂子行,可立在敌对的线上,绝不能彼此合作,或各守界限的。如若守界限,护院的可以按月挣工钱,那黑门坎的朋友不偷富户可吃哪一方哪?为这一层我和挂子行人讨论过,据他们所说也很有趣味。

明、暗挂子行的人能由敌对线上交朋友,各讲义气。在早年没有洋枪、火炮,没有电网,富户人家建筑的房屋无论多么高大,怎样坚固,挡的也是不来之人。如若有黑门坎的人把出道来,一样地随便出入。故此,富户人家都得花钱请护院的。凡是请护院的,十有八九都由镖局子给转请。

在早年,保镖的人上过道,走过镖,把式好,阅历深。不愿意保镖时,他们就改为护院。这护院的行当调侃儿叫“支杆挂子”。大富贵的人家,或有权势的人家,要请护院的不止请一位,或三或五,十位八位,内中还得有个头目,到了夜间,多少伙计,都得听头目人的指挥。如若有打更的更夫,也得听他们的调动。譬如,到了夜间,前后门、各屋门都关锁了,由护院的亲往各处巡视一趟,如有不完备的地方,他得费一回手,以免入地(黑门坎[专干偷盗窃取的练功夫的人]中从地下想办法偷东西叫入地)的朋友们乘机而入,丢失物件。屋中沏好了茶,身上收拾利落,应用的家伙也都放在手底下,不能打哈欠冲盹儿,把精神贯足了,宅院有多大全都得照看到了。若是黑门坎的朋友来了,他们也先“升点”,试问有护院的没有。什么叫“升点”哪?像评书小说上说的,高来高去的人每逢到了谁家,都用问路石子往院里一扔,故意地叫那石子吧哒一声,有了响动,调侃儿叫“升点”。若是有了响动不见有人答言,那就进来偷窃了。如若护院的人听见有人“升点”,他得出来答话,和黑门坎的人调(diào)侃儿说:“塌笼(房子)上登云换影的朋友,不必风吹草动的,有支杆挂子(护院)在窑,只可远求,不可近取。”这些话是什么意思呢?他们明、暗挂子行的人全都懂这几句侃儿。“塌笼上登云换影的朋友”,是说房子上的高来高去之人;“不必风吹草动的,有支杆挂子在窑”,是说来的人不用升点,有护院的在此;“只可远求,不可近取”,是说叫他往别处去偷,这里的东西动不得。如若遇见好说话的黑门坎人,凭这几句话,他就走了。如若贼在房上还是不走,就说:“朋友!若没事,塌笼内啃(kèn)个牙淋(yá lin),碰碰盘儿,过过簧。”这几句话是说:“你要没事,请下来喝会儿茶,见面谈谈。”如若贼人要走,跟着就得说:“朋友顺风而去。咱们浑天不见,青天见。牙淋窑儿,啃吃窑,再碰盘。”这几句话说的是:“你走啊,咱们夜里不见,白天见,或是茶馆,或是饭馆,咱再见!”

如果贼人真走了,护院的倒得留神,防备他稳住了护院的,哪里防备不到哪里去偷。若是贼人走后也没动静,也不丢东西,到了天亮之后,护院的就得“醒攒(cuán)儿”(江湖人管心里明白了调侃儿叫醒攒儿)。人家黑门坎的人是把自己当朋友,也得和他们交交。于是,身上也得紧衬利落,带上零钱,往附近的茶馆或是饭馆去找人家。别看两个人不认识,茶饭馆里座儿多,护院的到了往各处里一“把(bǎ)合(看)”,就能看出来哪个人是夜内的朋友。怎么个看法?凡他们黑门坎的人在茶馆酒肆候人,按着规矩有一种表示。如若坐在北边的桌旁,他得坐在右边,留出左边那个客座来。如若喝茶,左边没人也得放个茶杯;喝酒,左边无人也得放个酒杯。护院的来了,见他留着客座等候自己,就先过去抱拳施礼,道个“辛苦”,人家自然还礼,两个人谦让座位,然后吃喝,无论如何,护院的也得会人家的酒饭账。交了朋友之后,彼此遇事互相帮助,护院的可得了大便宜。有黑门坎的人(专干偷盗窃取的练功夫的人),如若不知道某宅有护院的,要去偷窃,他就能给拦住说,某处的支杆挂子(护院)是他的朋友,和他有交情,不必去了。有这个关照,无形之中就少许多的麻烦。护院的若能在本地交了黑门坎的“瓢把(bà)子”(头儿),那可更好了。黑门坎的人知道某宅护院的与他的头儿有交情,也不好意思去偷了。

也有那狡猾难惹的黑门坎的人,他要到了某宅扔了石子升了点儿(弄出声响),护院的答了“钢儿”(话)说:“塌笼上的登云换影的朋友,有支杆挂子,靠山的朋友有窑,不必风吹草动的。”他就在房子上“答钢”(江湖人管答言调[diào]侃儿叫答钢),问护院的:“你支的是什么杆?你靠的是什么山?”护院的就得回答道:“我支的是祖师爷那根杆,我靠的是朋友义气重如金山,到了啃(kèn)吃窑内我们搬山,不讲义气上梁山。”如若贼人走了便罢,倘若不走,就和他们说:“朋友!祖师爷留的一碗饭,你天下都吃遍,把这个站脚之地让给师弟吃吧!”说到这里,他还不走,就得说:“塌笼上登云换影的朋友,既有支杆的在此靠山,你就应当重义,远方去求,如若要在这里取,你可就是不仁,我也不义了。你要不扯(江湖人管你要不走调侃儿叫你要不扯),鼓了盘儿(管翻了脸调侃儿叫鼓了盘儿),寸步难行。倒(dǎo)埝(管东方调侃儿叫倒埝)有青龙,切(qiē)埝(管西方调侃儿叫切埝)有猛虎,阳埝(管南方调侃儿叫阳埝)有高山,密埝(管北方调侃儿叫密埝)有大水。你若飞冷子(弓、箭、袖箭)、飞青子(飞刀)、飞片子(房上的瓦),我的青子青着(刀子砍上)、花条子滑上(大枪扎上),也是吊梭(管疼痛调侃儿叫吊梭)。”贼再不走,就向他说:“朋友,这窑里有支杆(护院)的,四面也都是象家(对于练武的人们尊称为象家)之地,我若敲锣为令,四埝的师傅们一齐挡风,你可就扯不了。如若朝了翅子(管打官司调侃儿叫朝了翅子),都抹盘(管都不好瞧调侃儿叫都抹盘)。”

贼人再不走,那就得和他动手,凭自己的“尖挂子”(管真功夫、真能耐、好武艺调[diào]侃儿叫尖挂子)对付贼人了。倘若和黑门坎的人(专干偷盗窃取的练功夫的人)动了手,赢了得留情,不能和他们结冤;若是输给他们,就改行别干了。

黑门坎的人也不一样,他们各走一条线。据我所知道的,有“钻天”的贼人,有“入地”的贼人。那钻天的贼人也不一样,最有能耐的,练会了蹿房越脊的功夫,到富户人家拨门撬户,取箱柜的东西,使人不知,那算江洋大盗。本领再次一点的摘天窗儿,他们到了房上,用全份的家具掀瓦挑顶子,弄个窟窿,使绳索捋着下去,到屋里偷东西。临走的时候,还把天窗抹饰了,外行人看不出痕迹来他才走哪!钻窗户的,钻烟筒的,也到屋中偷盗,他们练的功夫有软的,可称轻身术,把一领席卷起来,有锅盖、茶盘粗细,放在桌上,由远远地一蹿,把身子能钻进席筒,一钻而过。还能往回退,两只手一扶地,退回去,两条腿入席筒,再穿回来。这种功夫练成了,由窗户烟筒进屋子,眨眼之间,就能办到。还讲究腿上绑铁沙子,由坑内往上跳,练得一两丈高就能上房,不用梯子,一蹿而上。他们还有一种功夫,两只手的指头抓住了房椽子,把身子贴在房檐底下,两足登椽子,把身悬起来。清末时候,北城某茶馆有一人吃核桃,不用砸,两个指头一捏,核桃皮就开,被衙门中的鹰爪看见,捕了去,一过堂,就招出许多窃案。可见黑门坎的人练手指之力,是能抓住房椽子,悬得住身,不然捏核桃时,手指没那么大的劲儿。明挂(管练把式不偷窃的叫明挂子)练的鹰爪力、大力法,与他们的功夫不同。

护院的人,若在哪里看家护院,也不能净等着。有的暗挂子(即黑门坎的人)按着规矩扔石子,升点(弄出声响),答钢儿(答言),倘若遇见浑家哪?他会高来高去功夫,不懂得明暗挂子规矩。没钱花,穷急了,不言不语,没有响动,他悄悄地偷,本家人若丢了东西不问他护院的吗?所以,明挂子行的人要给人护了院,夜内不住地往各处巡查,就得防备这种人。就是那开天窗、钻窗户、钻烟筒的贼人,也得时时防范。那黑门坎的人还有入地(黑门坎中从地下想办法偷东西叫入地)的贼呢!他们也分好几路:有能由几十丈远掘个窟窿,下到地内,去往坟内盗墓的;有由富户住宅墙外掘地窟窿,到富户的院内或屋内偷东西的;有由墙上挖窟窿到屋中偷盗;有专能移动下门坎底下砖石,钻进院内屋内偷东西的。我向黑门坎人(专干偷盗窃取的练把式的人)探讨过几次,据他们说,入地的朋友要挖窟窿盗洞的时候,都得在粗风暴雨的天气,有风雨之声,可以听不见他们挖窟窿的声音。护院的人对入地的朋友也得时时留神,无论什么样的天气也不能在房中忍着,照样出来巡查,哪处失神,哪处就许出错儿;哪里防不到,哪里就许出毛病。他们这碗饭实在不好吃。

北平这个地方,在清室的时代很有不少富户。这些富户,十家有九家花钱请护院的。自从欧西各国昌明新武器之后,我国的武术很受了影响,火车轮船,交通便利,镖行就没有饭吃啦!有许多镖行的人改了行,不是戳杆子立场子教徒弟,就是给大商家、富户们看家护院了。直到如今,北平支杆(护院)的朋友还有不少。廊房头、二、三条,西河沿,珠宝市,大栅栏,各银行、各银号、各绸缎庄很有些家请了护院的。我曾调查过几次,这些个护院的都是粮食店街南头路西会友镖局代雇的。

那会友镖局系河北束鹿县三皇炮锤门的名人孙某创立的。直到如今,他们的东家孙立庭还不肯歇业,保存那镖局子的买卖。一者祖业不肯扔,二者是专为给介绍护院的支杆的。孙立庭可称硕果仅存了。他每天早起必到西河沿、珠宝市、大栅栏等处绕一弯,凡是由他给介绍的护院的铺户,挨家都到到,看看有事没有。六七十岁的人,还能不怕劳苦,也是练把式的人得的强身壮体益寿延年的好处啊!

在三皇炮锤名人焦方桐在日,曾向我老云说过,一些个商家铺户,对于护院的事都不晓得,专爱雇岁数年轻的,没经验阅历,遇见黑门坎的人,能耐弱的,他能弄走喽;本领高的,就没法办。若是雇四五十岁的人,那全是上过道(他们管走过镖说行话叫上过道)的,只要上过道,他的武艺错不了,经验阅历一定丰富。如若遇见黑门坎的人,不用动手,几句话就能把他说走,永不来偷。若是用年轻的人,他没有阅历,遇见黑门坎的人,恃其技能驱逐。就算是武艺高强,能把黑门坎的人追走,他们恨上了,结下怨恨。不怕贼偷,就怕贼惦记!贼人若是惦记上了,怎样防备也有防不到的时候。常言:老虎厉害,也有打盹的时候,漏了空贼人便偷。护院的要想没人来偷,最好是访查哪里有黑门坎的朋友(专干偷盗窃取的练把式的人),设法联络,和他们套交情,由他们介绍见着黑门坎的瓢把(bà)子(头儿),若与瓢把子有了来往,就可以高枕无忧。

在清室的时代,北平有多少黑门坎的瓢把子,前步军统领衙门内外城各营汛部能知道。他们的瓢把子也各有界限,每个管多少地方,在他那地方之内,不论是谁偷窃财物,都得叫他知道,并且把偷来的东西,先交给他存放数日,防备有人找。如若失主有势力,寻找失物,追得急了,由瓢把子把东西交还。或是失主家中雇有护院的,人家护院的找着瓢把子论交情义气,也得把东西交还。

每一个黑门坎的瓢把子,手下都有许多人,昼伏夜出,偷来的东西,存放数日无事,他们就把赃物“挑(tiǎo)喽”,“均杵”(江湖人管卖东西调[diào]侃儿叫挑喽,大家分钱叫均杵)。如有外省的黑门坎的人来到内地,未作案之先,就得先拜瓢把子,然后才能偷窃,如若不拜瓢把子就作案,那失主丢了东西不找,瓢把子知道了也暗中叫鹰爪(江湖人管捕盗的官人调侃儿叫鹰爪)把他捕去。可是外省的黑门坎的人来到内地,若是念杵头儿(江湖人管没有钱花调侃儿叫念杵头儿),见了内地的瓢把子,得由瓢把子帮助他衣食住,如不作案,由这里路过,缺少路费,那是告帮,瓢把子也得赠他相当的路费。或有黑门坎的人遭了官司,瓢把子得托情运动,给送钱使用。

当瓢把子的也不一样。头等的人物,本领好,轻财重义,交际广,眼皮杂,认识的人多,遇事都用得着,事事活动,立住了名姓,有了万儿(名儿),黑门坎的人慕名来投奔,他的“摽(biào)杵吃上也是海海(hāi hāi)的”(瓢把子花他伙计的钱调侃儿叫吃摽杵,得的钱多了调侃儿叫摽杵儿海海的)。如若当瓢把子的没义气,事事不讲交情,过于厉害了,日久天长万儿一念(江湖人管名姓臭了调侃儿叫万儿念了),官私两面的朋友都不沾了,他也是吃不着摽杵,能挤得自己出去作案,那才寒碜哪!黑门坎的人,论品行也有优劣,那人品不好,事事不守黑门坎的规矩,鹰爪漏空他也偷,富户家中有护院的,得了手,他也偷,甚至于瓢把的窑(屋)内有好东西,不留神,也照样地窃走。可是,照这样胡来,栽了就没人救,吃上苦子,身体就得受伤,若是伤了手眼,这碗饭就不用吃了。黑门坎的人(专干偷盗窃取的练把式的人)本领高的,十有八九都是有义气的。富户人家有护院的他不偷,就是没有护院的,他访查人家财来得正当,也不下手,遇着孤寒贫人,疾病死亡,或是同道的为难事儿,他访好了哪个富户财来得不正,他必大偷一水,取来不义之财,他另做有义之事。如若日久了,立住了姓名,明暗挂子(黑白两道)阴阳两门的人都知道了,遇事还有人帮助他。

当初北平东北城某富户家雇有护院的,有一次黑门坎的义贼因有用款之事,夜内去见护院的,求他向本家借用一千银子,护院拾着义贼的万儿(江湖人管听人传说某人的行为如何,做事怎样,调[diào]侃儿叫拾着万儿),知道他常常偷富济贫,向富户借用的钱,不久准还。他来展(借)“柳海(liū hāi)拘迷(jū mi)杵儿”(即借一千银子),护院的就替他向本家疏通。怎奈本家主人不肯借用,事情弄僵了,护院的就把事辞掉。本家再找护院的,没人干了,夜内连三并四地丢东西。他家有势力,请来官军巡守,那黑门坎的义贼照样来偷,叫官兵看着,干拿不住他。昼去夜至,夜夜扰乱,个月不安。结果还是托朋友请明挂子(练把式不偷不盗的)有名的人物出来,给他们说和了事。事倒是完了,那富户的损失可太大了,弄得他啼笑皆非,哑巴吃黄连——有苦难言。

护院的虽是明挂子,偷盗窃取的人虽是黑门坎的人,他们阴阳也是不分哪!当初老云年幼的时候,在北平同学友往各处玩耍,有一次误入某院,见有一个老年人教好些人练功夫,所练的并不是拳脚,练的是蹿高纵远、滚背爬坡的功夫,所练的家伙都不带响动儿。有好几个人,能够撒腿跑着往墙上一蹿,倒背着身子,后背靠墙脚离地,能把人粘在墙上一样。那种功夫,据说名叫“粘糖人”,清末的名武丑儿张黑唱《大卖艺》,就有这种本事。由台帘跑出来,身子悬在台柱子上,平市五六十岁的人差不多的都见过。还有能把身子悬在房檐底下,手脚抓住房椽子,就能悬好大时候。那黑门坎的场子与普通的把式场子不同,我老云看过一次,以后再去,就被人家拒绝了。几十年内光景,回思往事,好像还记得点儿。

护院这行人,北平很出过几个有本领的。在清初时代,吴三桂在云南反了,遣绿林人到北京刺杀大学士索额图。那个黑门坎的人物(专干偷盗窃取的练把式的人)到了大学士府,见索额图夜内坐于案后办理公务,为国勤劳,料他是个忠臣,不忍下手,竟投在索额图府中给他护院。以后有许多的刺客俱都被他挡回去。索额图嗜好古玩,即使是明挂子(练把式不偷不盗的)行人也想去偷,只是有黑门坎的人改在他的府中支了杆啦(护了院了),也都不好意思去偷。看起来明、暗挂子行人都是有义气的呀。至于清室末叶,八卦门的董海川、尹福(现在平市募警教练所尹玉璋之父),太极门之杨露禅,也都有惊人的技能,又戳杆又支杆(又教徒弟又护院),很做了些个惊天动地的事儿。提起杨班侯、翠花刘、煤马、眼镜程等人来,至今还有人在茶馆酒肆里谈论他们的故事。





第六章 评书流派


评门是团柴的


江湖人管说书的这行儿调(diào)侃儿叫“团(tuǎn)柴”的。唱大鼓书的叫“海(hāi)轰儿”,又称为使长家伙的(指长长的弦子而言)。唱竹板书的叫使短家伙的。说评书的也叫使短家伙的,皆是指所用的竹板、醒木而言。

有人曾向敝人说过:说评书的不算生意。其实戏园的江湖艺术是人所共知的。说评书是由唱大鼓书演化来的,因其年代久远啦,评书界的支派流传得更广大了,使短家伙的与使长家伙的渐渐地疏远了。

唱大鼓书的门户在北方几省为“梅、清、胡、赵”四大门,现在北平男女班唱大鼓书的,都是这四门中的;在黄河南与大江南北,则为“孙、财、杨、张”四大门。唱西河调儿与怯口大鼓的都是梅、清、胡、赵四门的;唱犁铧调儿、山东大鼓的,都是孙、财、杨、张四门中的。

最近天桥儿唱女大鼓的坤角,如李雪芳、段大桂、于秀屏,与当年在新世界的谢大玉,都是孙、财、杨、张四门中的。孙家门的赵大支派流传下来的,彼辈皆自称为“孙赵”门里的人,即是孙家门赵姓传下来的支派是也。年前天桥天华园来了一班山东大鼓,领班的系谢大玉之父七十余岁老江湖艺人谢起荣先生。说起谢起荣这个人,凡是江湖艺人差不多都认识他的,他在孙赵门里算是辈数最高的。

平津等地唱大鼓的最早是胡十、霍明亮,最近是刘宝全、白云鹏唱得响了万儿(有了名儿)啦。此外还有唱西河调的名人马三峰。江湖艺人常言唱大鼓最好的,南有何老凤,北有马三峰。

何老凤姓何,按着孙赵门的支派名叫何起凤,因他人格高尚,都不肯呼其名,称他为老凤。何老凤三个字在山东是无人不知,何起凤的名字后来竟无人知道了。谢起荣即是何老凤一辈的(谢起荣由今春从北平携班回归济南),当其在平时,敝人向其讨论山东犁铧大鼓的源流,据谢谈,犁铧调儿是柳敬亭传的。柳敬亭原名逢春,明朝泰州人,本姓曹,年十五岁时,犷悍无赖,因殴伤多人,躲避仇人,流落江湖,休于柳下,善说书。据他自称,学技于云间莫后光(莫后光是柳敬亭的师父,云间人)。以养气、定词、审音、辨物为揣摩,使闻者欢笑,久而忘倦。复入左良玉幕府,左良玉失败后,交游于松江马提督军中,后因未能得志,数返泰州,与本乡赵姓富户甚厚,住其家。当大秋丰收,农工劳顿,所操之事甚微,柳敬亭先生用耕地所用的破犁片两块当作板儿,一手击案,一手敲犁,唱曲颇可动听。农工操作,闻歌忘劳。有人问先生所歌为何调,柳称为“犁铧调儿”。时人皆争而习之,自此“犁铧调儿”泰州无人不会。柳故后,“犁铧调儿”即普遍鲁省了。今有人传“山东大鼓”为“犁铧调儿”,实是谬谈。“犁铧大鼓”原用耕地破犁片为板,今人改为钢板,复书“犁花大鼓”,实是可笑。敝人问谢先生:柳敬亭之犁铧大鼓有何考证?谢答:无书可考,据我们“柳海(hāi)轰儿”的老前辈所传吧。

由谢起荣所谈“犁铧大鼓”的源流是柳敬亭先生传流的。评书南北两支派,也为柳敬亭传流的。敝人所论为江湖艺人学演说书的技能,至于古今著书的施耐庵、罗贯中、曹雪芹,又当别谈。翻书的、讲书的、背书的,更当别论。就以说评书的艺人而谈,他们的源流与所立的门户、传流支派,分为南北两大派。江南的派别暂且不谈,就以北派说评书而论,他们的门户是分为三臣,三臣系何良臣、邓光臣、安良臣。如今北平市讲演说书的艺人,皆为三臣的支派传流下来的。三臣系王鸿兴之徒,王鸿兴系明末清初时艺人。先学的是“柳海轰儿”为业(即唱大鼓书为业),曾往南省献艺,得遇柳敬亭先生,受其指点,艺术大进。遂给柳敬亭叩了瓢儿(江湖艺人管磕头叫叩瓢儿。比如甲乙两个江湖艺人,甲问乙:“你给哪位先生叩瓢呢?”乙说:“给×××叩瓢了。”即是拜×××为师啦。又可以管拜师磕头叫“爬萨”)。王鸿兴自拜柳敬亭之后,正值大清强盛的时代,王鸿兴遂至北平献艺。是时仍用的是长家伙(弦子鼓儿),听其书的多为一班太监们,后为宫中太后所闻,传其入宫。因禁地演唱诸多不便,遂改评讲。就以桌凳各一,醒木一块,去其弦鼓,用评话演说,评书由此俱兴。据评书界老前辈的人所说,说评书的门户系雍正十三年掌仪司立案(登记),有龙票(皇帝出具的用玉玺盖章的凭证)为凭。敝人探讨遗传之龙票何在,据谈在清末光绪年间,为×××给遗失了。一件历史性的物件没有啦,虽无大用,但评书掌仪司立案一事,只当传闻之事,当做谈话材料吧!王鸿兴在北平所收的徒弟,即安良臣、何良臣、邓光臣三人。王鸿兴故去之后,遂由三臣严立门户,定规矩,传徒授艺。直至今日,华北各省县市皆有讲演评书的艺人。评书的艺术是大众化的,近日最为盛行。伟大的艺术实是王鸿兴三臣师徒成就的。

在清朝最盛的时代,说评书都是“拉顺儿”(管拉场子撂地调[diào]侃儿叫拉顺儿),还没评书茶馆呢!北平老人凡五十岁以上的人,都听过拉顺儿的玩艺儿。在那评书的场地,是用几十条大板凳排列好喽,当中设摆一张大桌,上置木质香槽一个,内放鞭杆香一根。预备此物是给“询局”的人们“抿草山钩”(江湖人管听玩艺儿的人们调侃儿叫询局的,抽旱烟调侃儿叫抿草山钩)使用的。又放铁板一块,小钱笸箩一个(在最先是用量米粮的升儿),每逢说完了书打钱使用。说书的艺人到了上场的时候,得注意桌子后头板凳上坐着的人,按他们的规矩,生意人听书是白听不用花钱的,可不能坐他的龙须凳(桌前两条大板凳叫做龙须凳),必须坐在桌后的凳儿上。见了面彼此各道“辛苦”,不用多言,说书的就知道他是生意人了。说书的艺人到了场内,往“乍角(jiǎo)子”上一迫(pǎi)(管凳子调侃儿叫乍角子,坐着叫迫着),掏出手巾放在桌上,撂地预备的扇子顺着搁下,然后掏出所用的醒木。到了开书的时候,说书的艺人必须先说几句引场词儿。说引场的词儿最好是以扇子,或是毛巾,或是醒木说一套词赞为美。就以醒木为赞说,说书的艺人左手执扇,右手拍醒木,说的醒木词是:“一块醒木七下分,上至君王下至臣。君王一块辖文武,文武一块管黎民。圣人一块警儒教,天师一块警鬼神。僧家一块劝佛法,道家一块劝玄门。一块落在江湖手,流落八方劝世人。湖海朋友不供我,如要有艺论家门。”说完这套词儿,然后才能开书。

同行的艺人迈步走进场内,用桌上放的手巾把醒木盖上,扇子横放在手巾上,然后瞧这说书的怎么办。如若说书的人不懂得这些事儿,他就把东西物件,连所有的钱一并拿走,不准说书的再说书了。



凡是江湖艺人,不论是干哪行儿,都得有师傅,没有师傅是没有家门的,到哪里也是吃不开的。就以说评书的艺人说吧,他要是没有家门,没拜过师傅,若是说书挣了钱,必有同行的艺人携他的家伙。携家伙的事儿是:同行的艺人迈步走进场内,用桌上放的手巾把醒木盖上,扇子横放在手巾上,然后瞧这说书的怎么办。如若说书的人不懂得这些事儿,他就把东西物件,连所有的钱一并拿走,不准说书的再说书了。如若愿意干这行儿,得先去拜师傅,然后再出来挣钱。生意人携家伙的事儿,在我国旧制时代之先是常有的事,不算新鲜。到了一入民国时代,因而改变,这种事可就看不见了。如若再有人携家伙,没有门户的人喊来警察和他打官司,携不成人的家伙,反倒法院能判他个诈财的罪名。那么,在当初有携家伙,有门户有师傅的艺人应当说什么呢?在说书的见有人把家伙用手巾盖上,扇子横着压上,说书的艺人就知道这人是来携家伙的,不能翻脸打架,得沉住了气儿,用左手拿起扇子来说:“扇子一把抡枪刺棒,周庄王指点于侠,三臣五亮共一家,万朵桃花一树生下(说至此放下扇子,将毛巾拿起来往左一放),何必左携右搭。孔夫子周游列国,子路沿门教化。柳敬亭舌战群贼,苏季子说合天下。周姬佗传流后世,古今学演教化。”说完末句的时候,得用手拍醒木一下。遂又开书再往下说书,盘道(问对方行里的事和所学的功夫)的江湖就不敢再说什么了。如若说书的艺人为人忠厚老实便罢,倘若为人狡猾一点,说完了这套词儿,再用毛巾把醒木盖上,扇子横在毛巾之上,叫这盘道的生意人给拿开。盘道的按着江湖规矩他另有一套词儿,也是伸左手拿扇子,然后说:“一块醒木为业,扇子一把生涯。江河湖海便为家,万丈波涛不怕。”再拿开毛巾,放在左边,右手拿起醒木说:“醒木能人制造,未嵌野草仙花(评书的醒木定规矩不准使用花木头,也不准在醒木镶什么)。文官武将也凭他,入在三臣门下。”说完拍醒木,必须替说书的先生在场内说下一段书来。帮完了场子,然后再走。比如说书的艺人又将毛巾盖上,扇子横上了,这盘道的若不会说这套词儿呢,按规矩他得包赔说书的一天损失,说书的每天能挣一元,他就得赔一元。在早年,凡是好喜盘道的江湖人,都是阅历很深,久闯江湖,是生意门的规矩必须尽知,才敢去携人呢。如若一瓶子不满,半瓶子晃荡,对于艺人的规矩只有个一知半解,携不成人家,准得折(shé)了鞭(挨了打)的。

说评书的艺人,最好讲究托杵(生意人管向听书的客座要钱调[diào]侃儿叫托杵)的徒弟。早年说评书的收徒弟,做徒弟的跟着师傅在场内听活儿(听活儿即是学书),每到了要钱的时候,徒弟得拿着笸箩,顺着凳子替师傅向听书的人们打钱。自从清末光宣时代,说评书的收徒弟多为“询局”(听书的)的下海。从前听书的人们都是有闲阶级的,凡是有职业的人,哪有长工夫去听评书啊!总是八旗的子弟居多,有钱粮有米,衣食无忧,闲着干什么?消遣解闷听听评书。若是记性好的人,听个几年评书,怎么也能听会了一套两套的,赶上时代改变,旗人的钱粮没有喽,受生计所迫,投个门户,拜个师傅,下海就要挣钱养家。书是早就听会了,何必再虚耗一二年的光阴再跟师傅听活呀!所以到了如今,说书的人们都没有给师傅托过杵(生意人管向听书的客座要钱调[diào]侃儿叫托杵)的。就是有给师傅托过杵的,也没有几位了。每逢谈话之际,这种人都以给师傅托过杵为荣。评书界收徒弟分为两大规矩,一为入门,二为摆支。比如某人愿学说书的行当,经人介绍,给某人磕头认师傅,事先必须讨论好喽,下帖请人,在某饭庄定下几桌席,然后由做师傅的下帖请人,请多少人备多少帖,帖的样式是用个封套儿,外面粘上,写的是“定于某月某日上字某时,为小徒×××拜师入门之期,敬治杯茗,恭请台驾光临,×××率徒×××同拜”,席是“某街某巷某饭庄恭候”。凡请来赴席的人,大多数为本门的师伯师叔师兄弟们,有少数外门的老前辈。到了是日新徒弟拜师入门,一切仪式也有一定规矩。内设神座,设立牌位,正当中是供桌儿一面,设红纸包袱,包袱上写着已故的评书界老前辈的人名,即本门已故的长辈人名儿。由代笔师写门生帖一份,名曰关书。其书上写的是:“尝闻之宣圣曰:自行束修以上,吾未尝无诲焉。由是推之,凡人之伎俩,或文或武或农工或商贾或陶冶,未有不先投师受业而后有成者。虽古之名儒大贤,也上遵此训。今人欲入学校读书求学者,也先具志愿书,贽敬修金,行礼敬师。非有他求,实本于古也。况行游艺,素手求财,更当投师访友,纳贽立书为证。今有×××,系某省人,年××岁,经人介绍,情愿投在×××先生门下为徒,学演评词为业,以谋衣食。今于×××年××月××日,×××在祖师驾前焚香叩禀。自入门后,倘有负心,无所为凭,特立关书(门生帖),永远存照。具书弟子×××,师傅赐名×××,介绍人×××,立书人×××。”当将此关书写完之后,介绍人与保师都得书押,然后再由其师与本门人,与同道人,共同讨论给徒弟应起什么名字。按着三臣、五亮、五茂、十八魁的支派下辈数,将名字起好,填写关书之上,徒弟画了押。这个关书的手续才算完全。到了焚香行礼之时,先公推一位年高居长者办理,然后全体人一一行礼,礼毕之后,再行新徒弟递门生帖(门生帖即所写之关书)的礼儿。是时为师者先坐下,徒弟跪于师傅面前,以头顶门生帖,听其师训话完毕,双手举着门生帖,呈递其师之手,自此关书(门生帖)就永久收存了。徒弟叩头行礼之后,同道的本门人彼此贺喜,贺喜也行叩拜礼,按辈数大小分前后之序磕头。其新入门之徒,不论叔伯师兄俱皆叩头,行拜师入门之礼至为隆重。入席聚餐后,各自散去。经过这番手续之后,新入道的徒弟,在评书界算有其人了。在北平,瓦、木匠、厨、茶房也有收徒弟入行写字的事儿。徒弟将艺学成了,必须先谢师,然后才能挣工钱做活。评书界管谢师叫做“入摆知”。摆知与拜师不同,拜师有一两桌酒席便可,摆知多者二三十桌,少者十数桌。评书界摆知无年数的限制。工商业大多数是三年零一节的,学徒的学到了年份,不谢师不能挣钱,不谢师不能离开师傅单独做事的。就以“扫苗”(剃头的理发匠调[diào]侃儿叫扫苗)的行儿说吧,在清朝的时代,学满了徒,不谢师是不许担着剃头挑子出去的。如果愣担着剃头挑子去串街,同行人就能拦住了盘道(问对方行里的事和所学的功夫),盘短了愣把挑子给留下,不准他吃那行饭。当徒弟谢师之日,做师傅的算全始全终教成了个徒弟,自己也有名有利。是日为师者必须当着同行人将本行的规矩、行话暗语传给徒弟,为徒的懂得了行中规矩,盘道问答话语,再挑起剃头的挑子出去串街做活,没有人盘道便罢,有人盘道的时候,心里有货就有恃无恐了。扫苗的人们,非到了徒弟谢师的日子才能传授徒弟问答调侃儿。评书界的规矩是一样的,不谢师不准传徒弟调侃儿,谢了师之后才能懂得本行问答言语的。前谈评书界携家伙(盘问门户)的问答词儿,也是谢师的日子受师傅指教的。在北平评书研究社时,有位说《盗马金枪》的先生叫马风云,他最恨评书的老前辈出去携人的家伙。他的思想是正大的,管他有师傅没师傅,管他有门户没门户,谁挣钱谁吃饭,何必为寻事?有些新入行的徒弟,因为不懂得有人携家伙时应当如何对答,向马讨教,马好诙谐,教给新徒弟钻钢(江湖人管骂调侃儿叫钻钢)携家伙的,然其为人也善恶剧者。

江湖艺人常说:“唱戏的要想叫座儿,得有好轴儿;说书的要想叫座儿,得有好扣儿。”



评书界的侃语管《施公案》这部书叫“丑官儿”,丑官是指施公而言,传其人有残疾叫“十不全”,以施公是残废人的讹言调侃儿叫“丑官儿”。管《隋唐传》调(diào)侃儿叫“黄脸儿”,《隋唐传》是以秦叔宝作书胆(书中的主要人物称为书胆),因秦琼长得黄面皮,故称是书为黄脸儿。管《包公案》调侃儿叫“大黑脸儿”,面黑而言。管《小五义》调侃儿叫“小黑脸儿”,其中的意义与“大黑脸儿”大同小异。管《于公案》调侃儿叫“浑(hún)水子”,是指于公而言,鱼是浑(hún)水东西,于与鱼音同字异也。《三国志》调侃儿叫“汪册(chǎi)子”,盖因江湖人管三字之数调侃儿为“汪”是也。管《精忠传》调侃儿叫“丘山”,《精忠传》以岳飞作书胆,将岳字拆开了说为“丘山”,其意最为显明。管《西游记》调侃儿叫“钻天儿”,其意是以孙行者是个猴儿,一个斤斗十万八千里,借孙猴而言,称其书为“钻天儿”。其余的,如《明英烈》叫“明册(chǎi)子”,《东西汉》叫“汉册(chǎi)子”,《三侠剑》叫“黄杨儿”,《彭公案》叫“彭册(chǎi)子”。《济公传》叫“串花”,其中意义是以济公穿的僧衣褴褛不堪和花儿乞丐似的,以济公为书胆,叫做“串花”。唱戏的票友儿叫“清客串”,唱花脸的改唱《蜡庙》张桂兰叫做“反串”,济公故意穿破烂衣服,褴褛不堪,是为反串花子一样,说他是“串花”其意浅而显明也。管开书馆的主人调(diào)侃儿叫“粘箔(nián bo)”,管茶馆伙计调侃儿叫“提搂把(bǎ)子”,听书的人们格外多给书钱调侃儿叫“疙瘩(gē da)杵儿”。若有听书的人指正说书的艺人,将某回书说错了,调侃儿叫做询局(听书的)的“摘毛儿”。评书界的人常说戏听的是“大轴儿”,书听的是“扣儿”。要想多挣钱,书里的“扣儿”得引出“大柁子”(最大的扣儿)来。什么叫“书扣儿”呢?譬如说书的人说的是“黄脸儿”(《隋唐》)吧,说秦叔宝跟随靠山王杨林由山东起身来到长安城,杨林接到山东济南节度使唐璧的一件紧急公文,说有三十六友大反济南府劫牢反狱,劫出劫皇杠的程咬金、尤俊达,火烧了历城县的县衙,三十六友的盟单上有秦琼的名字,唐璧请杨林将秦琼拿住,叫秦叔宝招供三十六友的下落,以便肃清响马。说书的艺人说到此处,听书的人们都替秦琼担心,怕秦琼有了危险,无论有多少要紧事就豁出去耽误喽不去办啦,专听这段杨林追赶秦叔宝的扣子。说书的用扣子将书座扣住了,如同使拴马桩(用话把人扣住)一样,再不慌不忙说秦叔宝三挡杨林。他说完了这个扣子,听书人的大把儿钱也被他挣足了,他说书的人也就“驳了口”(他们说评书的管散了书不说了调侃儿叫驳了口)啦。临驳了口儿的时候,还说明天接演“魏文通追拿秦叔宝,三十六友九战魏文通”,这两句是叫听书的人们知道,明天好再来接着听“九战魏文通”的扣子。一天使一个扣子,说个三五天,便说到最热闹的节目“瓦岗山”了。管六次攻打瓦岗山十数天说不完的大扣子,又调侃儿叫“大柁子”。不论哪部书也有好扣子、大柁子。例如《施公案》的“五女大灰场,捉拿一枝兰”,“七贞捉拿大莲花”;《济公传》的“八魔炼济颠”;《彭公案》的“画春园”、“牧羊阵”;《精忠传》的“牛头山”。说书的若想挣大钱,必须有“把(bǎ)钢”(管有拿手的,有把握能挣钱的能耐调侃儿叫把钢)的活儿。说得拢不住座儿,每遇要钱的时候净走座儿,调(diào)侃儿叫“起棚儿”。说书的人若是没学好喽就上馆子愣说书,一定把书说得不精彩,不火炽,调侃儿说他“蹚水儿”呢!又有没品行的说书的,知道某人说的××书最好,去偷着听书,调侃儿叫“荣(荣即是偷的意思)人家的活儿”。说书的艺人如若有条好嗓子,调侃儿说他“夯(hāng)头子正”。说书的人如若口白好,调侃儿叫他“碟子正”。说书的人口白不清,调侃儿叫“碟子不正”。如若说书的闹嗓子,调侃儿叫“夯头子鼓啦”。说书的人长得五官端正,器宇轩昂,调侃儿说他“人式压点(yā diǎn)”(震得住人为压点)。如若长得相貌不好,言不压众,貌不惊人,调侃儿说他“人式不正”,或说“人式太念”。如若说书的不认字,叫“不钻朵儿”。或是没有学问,调侃儿说他“朵上不清”。认识字的叫“钻朵儿”。说书的挣钱挣大发了,调侃儿叫“团(tuǎn)柴(说书)的火喽”。说书的艺人不挣钱混穷了,调侃儿说“团柴水拢啦”。说书说得能有叫座的魔力,调侃儿叫“响了万儿(有了名儿)啦”。说书的艺人要向书座套交情,多拉拢书座,调侃儿叫使“贴身靠儿”。说书的艺人设法骗听书的座儿钱,使人能够忍受,调侃儿说他“挖(wǎ)点”。说书的艺人如是北平人,口白清楚,外省人说书怯口,调侃儿叫他“浑(hún)碟子”。说书的会武艺,或是懂得武术,调侃儿叫“钻习尖挂子”(受过训练的练把式卖艺的人)。说书人说书的时候,常把书中人名说错,调侃儿叫爱“滚钢儿”。说书的人在场上批评同业的书说得不好,调侃儿叫“刨活”。书馆的伙计如若在打书钱的时候往身上藏钱,调侃儿叫他“捂(wū)杵”。说书的艺人净诓骗同业人的钱,调侃儿说他“抠鼻挖(wǎ)相”。说书的艺人不会说扣子,拢不住座儿,把扣子说散啦,调侃儿叫“开了闸啦”。说书的艺人在场上能将书中事儿说得意义最浅,使听的人们容易懂得,听得明白,调侃儿叫“开门见山”,又叫“皮儿薄”。书说得使人不懂,听着发闷,调侃儿叫“皮儿厚啦”。说书的艺人名誉正叫“万儿正”,名誉不正叫做“万儿念”。说书的艺人心术不好叫“攒(cuán)子不正”,心术好叫“攒儿正”,胆量小叫“攒儿稀”。管整本大套书叫“万子活”。说完了一部书又换别的书了叫“拧万儿啦”。书越说越长没结没完的叫“万子海(hāi)啦”,书说得要完了叫“万儿念了”。将学一部新书叫“蹚万儿”。管说短期的三五日有拿手能拢座儿的书叫“吧哒棍”,管说小小的段儿叫“片子活”,自己编段书叫“攥弄(zuàn nong)万子”。

庚子年前说书的人们都是上明地(不是屋子的演出场所)撂场子,在东四牌楼、西单牌楼、安定门内、阜成门内等处,靠着甬路边儿支棚帐摆凳子说书,只有十分之一的艺人上馆子。庚子年断大烟之时,评书茶馆才畅兴一时,直到了民初袁项城(袁世凯)秉政,极为发达。开书馆的主人若邀说书的先生,不能随便滥邀,必须求一个说书的主持该馆邀请角(jué)儿之事,评书的同人管专司邀角儿的人称为“请事家”。每逢有开书馆的初创设立评书,必须由请事家先找一位说书的破台,称该书馆头一个登台说书的先生叫做“开荒”。破台之法,台上先设神桌,桌上供周庄王、文昌帝君、柳敬亭的牌位,是日由说书的先生及开书馆的主人行完叩拜之礼,说书的如同念喜歌儿似的,还有一套吉利赞儿,将赞儿念完了,撤去桌位,将祖师牌位送焚了,然后由开书馆的主人用红封套一个,内装洋十元至五元,最少也得一两元,用糨糊封好,放于书桌之上,敬送先生,名为“台封”。当日所挣的书钱并不下账,评书界的行规是三七下账,比如挣洋一元,说书的要七毛,开书馆的主人分三毛,钱数多少依此类推。凡书馆更换说书的先生时,头天书钱与末天的钱,书馆不下账,都是说书人的,名为头尾不下账。破台的日期与此相同。可是评书界的人们,凡是有叫座魔力的头二路角儿,向来不给新书馆破台开荒,避讳此事,如若请他们开荒,无论是亲是友,伤了交情都可以,绝不为书馆开荒的。如若问他们为什么怕给书馆开荒呢?答以开荒破台的人必将不利。知识幼稚如此,实是可笑。那么开荒破台的说书先生又哪里去邀呢?在评书界说书不挣钱的三四路角儿,每日昼夜奔驰不得温饱者,专给新书馆破台开荒,所贪图的不过数元之台封儿。评书界的规矩,每一说书的艺员,在书馆内只许说两个月书,名为“一转(zhuǎn)儿”。故评书馆的艺员,都是两个月一换转儿。北平的评书馆子,在内城的都是白天搁书,灯晚卖清茶。前三门外的书馆子,都是白天卖清茶,灯晚搁书。内外城的书馆黑白天都搁书的,只有宣外大街路西如云轩、宣内森瑞轩、磁器口红桥之天有轩。至于天桥福海居(俗称王八茶馆,其故去之旧主人姓王行八,天桥野茶馆是他最早创立的。当其在日营业极为茂盛,今老王已故,其营业一落千丈,非昔日可比了),虽是灯晚白天都有书,仍以白天上座儿甚多,灯晚上座儿寥寥而已。

按评书界的规矩,开书馆的主人每年须请支(请客)一次,所请的说书先生一般都是到这个书馆说书的演员,其中尚有非其演员者,也不过是作陪吃嘴而已。请支之先由书馆主人备请帖若干份,交该馆之请事家(评书的同人管专司邀角儿的人称为请事家),由请事家向帖上填写人名,也由其送帖往邀,请支的日期系书馆主人在某饭庄预定酒席一桌或两三桌,至期接到请帖之人皆来赴宴。弄书馆的主人花钱请支,其欲望是愿请事家邀的角(jué)儿都是头路角儿,如若请的都是头路角儿,该书馆一年之营业,六转儿的演员均能叫座,必获重利也。至于请来的说书艺员是不是头二路角,那就看请事家邀角儿的能力如何了。近年以来,评书界名角如群福庆、潘诚立、双厚坪、王致廉、徐坪钰、汪正江、袁傑(北京评书艺人用“傑”字,天津同辈评书艺人则用“杰”字,如常杰淼、张杰鑫)亭、田岚云、李傑芳、金傑华、董云坡等故去之后,评书界的人才缺乏,后起无人,所有能叫座的艺员只有十二三个人,各饭庄也不见书馆定席请支了。评书界诸公若不设法培养人才,恐此十二三人也难久持的。不知评书界的人们以为然否?





评书界请支之源流


喝茶愈喝口味愈高,买茶叶的钱数也渐渐增加;听戏愈听戏瘾愈大,愈听好戏,戏价愈贵。惟有听评书是不论好歹都是一样花钱,无分贵贱。说评书的艺人挣钱多少,是由上座多少而论。说好书的艺人多叫书座,收入便多;艺业平庸的,没有叫座的魔力,每逢开书的时候,座客稀少,收入也多不了啊。故开书茶馆的主人都争着请有叫座能力的演员。凡是能叫座的说书的艺人,都争着约请,有一人为数家所约的。据我调查得来,每一个评书演员在一个书馆只说两个月,名为“一转(zhuǎn)儿”。有一种书馆只能白天搁书,按着两个月一转(zhuǎn)儿计算,应请六个演员演说六转儿,才能够一年的全年转儿,开书馆的主人按着规矩每年应请六个演员,在未曾请人之先,得找请事家(即代邀角[jué]儿的),由请事家替开书馆人下帖请六个评书演员,在饭庄定酒席一桌,定日聚餐,名为“请支”。请的演员角色优劣,须视请事家邀角儿能力如何。如若六个演员俱有叫座魔力,开书馆的主人都有一个请事家为他奔走,四出约角儿。有些个地势好的书馆,请事家都巴结书馆的主人为其邀角儿。有些个书馆地势不好,评书演员都不愿进他的馆子,书馆主人便巴结请事家为其邀角儿。评书界的请事家与开书馆的主人也是店大欺客,客大欺店。据评书界中的老人所言,在早年北平这个地方,说评书的演员都是上明地(即是街头、庙会、拉场子、露天讲演),并没有书茶馆,至清末同治年间,书茶馆才发芽儿。开书馆的主人请支,系光绪年间所兴的,首倡此举的是宣外大街路西胜友轩(今该馆已更名,另换主人也不搁书了),主人刘某是开书馆请支的第一人。据评书界人所谈,他请的演员是潘诚立《精忠》、陈士和《聊斋》、袁傑亭《施公案》、王傑魁《包公案》、金傑华《小五义》、群福庆《于公案》、阎伯涛《清烈传》,在那时候这些演员还是二等角儿。头路角儿是双厚坪、田岚云、王致廉、胡连城等,这头路角儿皆在如云轩演讲,如云轩在菜市口北路西,胜友轩在宣外大街路西,两个书馆相隔不到百步。南头书馆以头路角儿号召书座,北边书馆以二路角儿后起之秀号召书座,与如云轩打擂台,每日均上满座儿,胜友如云,满棚满座,盛极一时。自从胜友轩的主人刘某提倡请支之后,各书馆主人也都纷纷请支。北平的书馆请支,在春秋两季为多,大教的饭庄天寿堂、同兴堂,清真教的饭庄、饭馆元兴堂、两益轩,每年都做些请支的酒席。自从近二三年来,社会的经济状况不好,书馆的主人请支的事儿也是寥寥了。





团柴的规矩


说评书的这行儿调(diào)侃儿叫“团(tuǎn)柴”的,又叫“使短家伙”的。虽然是艺人,他们的规矩很大。就以他们在场上说吧,无论谁来了也不能行礼,也不能答言;如若行礼答言,也有一定的时间。设若有人在台上和人答言与人行礼,那就算坏了规矩。当初我在少年的时候,在后门听王致廉的《包公案》。有一次他在台上说:“我们这行儿对于亲朋是不应酬的。有些人常怪我在台上不理人,其时我们这行儿不能理人。譬如今儿我正说《隋唐传》,裴元庆由外边走进中军帐,他父亲裴仁基说:‘儿呀,你来了。’可巧由外边进来一个熟人,我在台上向他说‘你来了’,这人能给我一茶壶。他急了就许问我:咱们不玩笑,怎么我进门你就叫我‘儿呀,你来了’?譬如,我说书的说裴元庆正在帐中坐着,他父亲裴仁基从外边进来,裴元庆说:‘爹爹你来了。’可巧在这时候进来一位书座,我冲书座说‘你来了’,这位便宜了。旁边还有说便宜的说:‘说书的爹也来听书啊?’所以我们这行人若在台上说书,有熟人进来,我若不理谁,可别怪我不理人,我们这行就是这样规矩。”

当初我老云在交道口马路旁边听书,正听李致清的《封神榜》,他师傅程德印从场子前边走过来,李致清要给他师傅请安,程德印说:“掌着买卖不拿腿。”他就不行礼了。后来我向李先生问什么叫“掌着买卖不拿腿”?李先生说:“我们这行人如若正在场内说着书,见了亲朋不能行礼,和戏台上一样。如若正唱《恶虎村》,去黄天霸的那个角冲台底下熟人请个安,那成吗?我们也是一样。我们的行话管说着书叫掌买卖,管别请安施礼叫不拿腿儿。”我听了这个解释,才知道他们这行规矩。

有一次老云在天津三不管(天津市南市的一个露天市场)听评书,听的是张杰鑫的徒弟马轸元说《三侠剑》,他是由营口刚回到天津,还没见他师傅哪。可巧张杰鑫从他场子外边路过,他出了场给他师傅磕个头。张杰鑫说:“掌着买卖不爬萨。”后来我问马先生什么叫不爬萨?他说:“我们这行儿,管别磕头调(diào)侃儿叫不爬萨。”

有一次我在三不管听刘庆和的大鼓书,他师傅牛德兴来了,他正说书哪,要给牛德兴磕头,牛德兴说:“使着买卖,不用叩瓢(磕头拜师)。”我没问他也猜透了:使着买卖是说着书哪,别磕头就是不叩瓢。

有一次我老云走在花市,遇见一个新上跳板(刚入这一行)说书的,我问他在这里干什么?他说:“跟活儿哪。”对于这句行话,我不大明白。我问他什么叫跟活儿?跟活儿是怎么回事?他说:“我们说书的这行,如若徒弟去听师傅说书,不能像书座儿听书解闷。我们要听师傅的书,行话叫跟活儿。跟活儿还有规矩,不准去晚了。譬如三点钟开书,两点钟就得到,走在师傅前头为是。如若坐在凳上等师傅,师傅来了徒弟还得站起来。沏上茶给师傅斟一碗,然后才能自己喝哪。如若要走,也得等着散了书,随着师傅一同走。如若不等散书走,那便是坏了规矩。”

有一次我老云碰见了连阔如,我问他来干什么?他说:“替买卖。”我问什么叫替买卖?他说:“今天是刘继业他父亲寿日。他在琳泉居说灯晚,今天他在家应酬亲友来不了。叫我替他说一天,行话叫替买卖。”我说:“我也没事,同你去听听书。”我记得他那天晚上说的是《卞和三进宝》,楚相昭阳丢和氏璧,怒打张仪,又串到蔺相如完璧归赵,将相和。他说到十一点多散书,挣了几十吊钱,他没拿着,向茶馆掌柜的说:“你把杵头儿给挂起来吧。”那掌柜的就把钱端了走。我问他这是怎么回事?连阔如说:“我们说书这行,如若替谁说几天,挣了钱不能拿走,按着规矩存在柜上,这钱还是人家本人的。说行话叫把杵头儿挂起来。”我问他,替说书,不把钱留下,说完了带起来的有没有?他说:“有倒是有,那不过是师傅替徒弟说一天,说完了把钱全带走。除了师傅外,别人是不行的。”我听他们所说,才知道江湖艺人是有义气。





天桥的评书场子


在清室时代北平没有评书茶馆,说评书的都在马路边上拉场子露天讲演。西单牌楼、东单牌楼、东四、西四、后门外、交道口,都是评书场子。自从庚子年后禁烟,北平的评书馆子才渐渐兴旺,到民国二十年,说评书的艺人都上馆子了,露天场儿是见不着的。到如今,评书艺人在露天场儿说书真有不会说的了,天桥的评书,始终也没兴旺起来。

在早年,天桥说评书的有个尚××,只说《黄杨传》,书中的意思是以黄三太镖打猛虎,杨香武盗九龙杯为叫座儿的段子。据评书界的人说,那位先生是外江派,不是北平评书界支派中的人物,他的书说不了两个月,几天就完,说完了从头再说,专有些人爱听,但没有大转(挣大钱)。

北京宣武说唱团评书演员合影,后排从左到右分别是:刘鹤云、高豫祝、傅阔增、连阔如、徐雯珍(说唱团负责人),前排蹲者是陈荫荣。(照片由徐雯珍提供)



自从民国二十年,评书界的连阔如、陈荣启、苗阔泉,在天桥撂明地(露天)演说评书,能占个场子叫满堂座儿,才算兴开了这宗玩艺儿。郭品尧、高阔轩、高豫祝、丁豫良等接连不断地上地(做生意),评书才能在天桥久占。可是夏天最美,天棚底下听评书,来壶酽(yàn)茶,又解闷又凉爽,却是有趣儿。过了夏天可就差多了。





天桥茶馆各有不同


天桥评书茶馆,只有福海居(即王八茶馆)一家,在该书馆最发达前为清茶馆,提笼架鸟的闲散阶级人物都到那儿喝清茶去。后为评书馆,不卖清茶,所上的茶座儿都是好听评书的。

北平这个地方,评书茶馆共有七八十家,王八茶馆屋内宽阔,能坐三百多书座,为书馆之冠。说书的先生们挣钱最多也数该馆笫一。白天上座最多,灯晚座客稀少,不及白天的三分之一。评书界演员有叫座魔力的在该馆讲演,能上满堂座儿,能力稍差者就无人去听。在王八茶馆说书虽能挣钱,也要艺术高超,第一路角色才能上得住一转(zhuǎn)儿(每两个月为一转儿,过期改换新角),第三四路角色皆畏而不往。第二路角色也时常有磕出去做不到一转(zhuǎn)儿的(凡是说书的演员到某书馆说书,如不上座,演员辞了馆另寻他处时,同业人讥诮他在某书馆磕出去了。磕出去为评书界最耻辱的事儿)。

在清末时,该馆能叫座的说书演员为王致廉、王傑魁、田岚云、杨云清、张智兰、群福庆、张诚斌。自民国以来,在该馆能叫座的说书演员为陈士和、潘诚立、张少兰、袁傑亭、袁傑英、金傑丽、品正三、刘继业、阎伯涛。最近评书界老人物相继去世,后起无人,人才缺乏,在该书馆能挣钱能叫座的只有品正三、刘继业、阎伯涛、刘继云数人。王傑魁、袁傑英为评书最有声望的角色,也因该馆生意难做辞了转儿,另搭别的书馆了。陈士和、金傑丽去津未返,张少兰改行行医。该书馆每年只用六个演员即可表演全年,至今评书界演员尚有百数余,欲邀六个相当角儿都感觉困难,评书界人才缺乏为百年来所未有,望该界同仁设法培养人才方好,倘不设法维持,评书界的事业就要破产了,不知说书的先生们以为然否?

今年该书馆的角色大有更动,除正、二月,仍为刘继业说《精忠传》,三、四月袁傑英辞去另换蒋坪芳说《水浒》,五、六月连阔如辞去另换张荣久说《施公案》,七、八月仍为品正三说《隋唐》,九、十月阎伯涛说《清烈传》,冬、腊月刘继业说《济公传》外,因评书转(zhuǎn)儿(每一个评书演员在一个书馆只说两个月,名为一转儿)仍然沿用旧历,闰三月又邀王傑魁说《包公案》。按王傑魁在该馆献艺有三十余年,可保能叫座儿,至于蒋坪芳、张荣久等演时能否上座,实难预料也。

劈柴陈茶馆主人姓陈,因售劈柴得名。该馆在天桥西沟沿路北,六楼八底,底下的茶座儿大多数是附近手艺工匠、摊贩商人。楼上则分两路的,每日早晨有十数人在那里喝茶、研究活儿。许荣田、陈荣启、马阔山、曹阔江、马荫良等是天天准去的。这里算是个清茶馆,如若有人邀说评书的,到那里去邀,是绝不能空的,即是团柴的牙淋窑儿(团柴是说评书的,牙淋窑儿是茶馆)。

六合楼茶馆在魁华舞台北边,四楼四底,虽是个清茶馆,白天卖清茶,夜里是店,瓦木匠、拉车的老哥们盘踞之所。清茶馆儿地势宽阔,楼上楼下,设备完善,讲卫生,真清雅。买卖发达的第一为西华轩,俗称红楼茶馆;第二为同乐轩,在红楼茶馆以东,俗称三起大楼。野茶馆真凉爽的为长美轩,在电站总站以西,每逢夏季,天天高朋满座,其余的野茶馆则无定所,年年改变,营业如何也没一定的。小小茶园、天桂茶园、小桃园、万盛轩,都是蹦蹦儿棚子,又叫奉天落子,半班戏,所唱玩艺儿,生、旦、净、末、丑等等的角色都有,我老云听过些回,他那戏里始终也没唱出个皇帝、元帅,美其名叫评戏,称为半班戏倒是名副其实的。

如意轩、二友轩、三友轩都是落子馆,一班不得时的鼓姬全在那里演唱,询局的先生们如好耳目海(hāi)轰儿(听玩艺儿的人江湖调[diào]侃儿叫询局的,管听大鼓调侃儿叫耳目海轰儿),可以去耳目吧。爽心园、春华园、天华园又都是唱坠子的、唱山东大鼓的杂耍(是杂耍曲艺形式的综合叫法)馆子了。





三不管的评书场儿


天津说评书的都是由北平传出去的支派,门户最盛为英致长(北平创说《善恶图》程德印的弟子)、王致久、福坪安、周坪镇、张诚润等,哪个支派也传出数十人去。我老云在北平是常听评书,到了天津也是一样,有了工夫就听评书。随听玩艺儿消遣解闷,也能得着一种社会调查的材料。

天津、北平虽然相离不到三百里路程,风俗习惯却大不相同。就以评书界说吧,北平的说书艺人是两个月一换地方,管在一处说两个月的书叫“一转(zhuǎn)儿”,每逢正月、三月、五月、七月、九月、冬月为评书换转儿之期,大家才能更换馆子;天津的各书馆是三个月为一转儿,每逢节关才能更换说书的。北平的说书艺人一部书要说两个月,每天是说三个多钟头;天津的说书艺人一部书要说三个月,每天是说两个钟头。北平的书资是几回一要钱;天津是每天要一次钱。北平听一天书须三十多枚;天津听一天书三大枚。北平的书馆,每天散书之后和说书艺人三七下账,挣一元,书馆分三角;天津是说书的挣多少钱不下账,不论挣多少,都是说书的,书馆分文不要。那么开书馆的主人指着什么赚钱哪?说是指着说评书的艺人有叫座儿的魔力,给他多叫书座儿,来一书座儿,听书花三大枚,茶资也是三大枚,他的利益是多进茶资。北平的说书艺人虽有叫座儿的魔力,约定了哪月说书,哪月登台,也不能使茶馆分文,只有在饭庄备桌席请说书的艺人吃喝而已。天津开个书馆可就不同了,没本领的说书艺人不能叫座儿;有叫座儿魔力的说书艺人得使押账,书馆主人得无利无息叫说书的艺人先白使一二百元,三四百元。可是,没上台先使钱,下台就还。天津的书馆与说书的情形是这样的,可是说书的艺人都不能指着书馆挣钱。北平的书馆若上五六十个书座儿,说书的艺人就能挣两元钱;天津的书馆上一百个书座儿,说书的艺人才挣六百枚,合一元有余。这样比较还是北平的书馆容易挣钱。天津说书的艺人上书馆有两种用意:一是上书馆白使几百元,二是借壮声势。要是想挣钱,白天、灯晚得分开了,或是白天上书场说书,夜内上馆子。要想天天挣钱,可得指着书场,那书场上的书座儿最多,说一回书要一回钱,要听一天书得花二三十枚,若上百数多座儿,就能挣三两元钱。书场与书馆比较,还是书场挣钱。

因为挣钱的关系,天津说书艺人都愿上书场,书场儿约个好角色,受说书艺人限制,也是得白使数十元,一切的设备都听说书的艺人指挥。如若说书的艺人没有叫座儿魔力,不惟不能白使钱,还得受书场主人压住,限制每天至少得给他挣多少钱。社会里的事,店大欺客,客大欺店,艺人与书场主人也是如此呀。





评书门之群福庆


说评书是分袍带、短打,短打就是公案书。说公案书最有万儿(名儿)的人就得数群福庆,其次就是袁傑英。群福庆他本姓吴,字叫光甫,排行在二。他的大哥是因为自幼失迷,始终没有踪影,他兄弟是在后门外天汇大院开设“开明轩”茶馆。群福庆在幼年时候在某斋学徒,学饽饽铺红炉(烤点心)上的手艺。他的手艺学得很不错,因为他性情最好听评书,每到晚上铺子里上了门后,大家全都睡觉去了,惟独他是耗夜油子,等人睡着觉,溜瞅瞅(偷偷)地跳墙出去,直匆匆地就奔到书馆去听书。天天儿如是,可惜他那八年多的手艺,因为好听书就给耽误了。日久天长,没有不透风的篱笆,因听书把事都误啦,所以被人家给辞了。他心里一赌气儿,干什么不能吃饭哪?于是他就给白敬亭磕了头,拜为师傅,从这儿他就说起评书来。按:白敬亭本是“文”字的支派,名叫白文亮,跟双文兴(双厚坪)、海文泉是师兄弟。白敬亭说短打书,以说《施公案》为最拿手,时常往清室各王公府里说家档子(堂会)。因为他是瓦匠手艺出身,每逢说到灶王爷杜克雄耍大铁锹的时候,最为出色,别人是比不了的。他师兄弟三人,眼下就剩海文泉了,他说《济公传》、《永庆升平》为最好。群福庆拜白敬亭为师,按着支派赐他的名字叫福庆,他姓吴,理应该叫吴福庆,因为他迷信心重,吴无两个字是音同字异,吴福庆认为不大吉祥,忌讳这个无字,所以就改名叫“群福庆”。他从前在天桥各场拉顺儿(即是撂地拉场儿),很有叫座的魔力,因为他的夯(hāng)头好(就是好嗓子),喷口字正,能够把那英雄的肝胆气概表现出来。我国人民对于侠义英雄素常都抱崇拜主义,所以群福庆是“挑(tiǎo)帘红”,出门就“转”(zhuàn)(出门就火,能挣钱),也是因这缘故成的名。他的“丑官儿”(丑官儿是侃语,就是《施公案》)说得很不错。有个袁傑亭,系评书界名人王致廉的门婿,也说《施公案》。袁傑亭说《施公案》一些的言谈动作,较比群福庆还有好的地方,可以说是有过之无不及,因这缘故,群福庆很受他的影响,后来便又学“浑(hún)水子”(浑水子是侃语,就是《于公案》)。按:《于公案》这书,是评书界名人牛瑞泉所编纂的。那里面的结构跟穿插都很精奇,能够引人入胜。可惜牛先生在北平是时运不济,未能得志,又不肯把这心血编纂的《于公案》抛弃,就将此书传给了刘竹桥,后来刘竹桥又把这书传授于群福庆。

群福庆从把《于公案》学会之后,每逢要与袁傑亭对垒的时候,就演《于公案》,不说《施公案》,以表示谦让之意。无奈他学的这《于公案》不够说一转(zhuǎn)儿的(即是不够说两个月)。他曾从马风云学过《盗马金枪传》,马风云人品很清秀,说《明英烈》最好,可是不变口,不比刀枪架,所以叫做文口《明英烈》。群福庆每逢说《于公案》到了末尾,还亏个十天半月的日期,他使用《盗马金枪》补续着说。后来把《于公案》说开了,能够说六十天啦,就把《盗马金枪》扔下了。现在这《盗马金枪》就没人说啦,简直就要失传了,未免是很可惜的呀!

群福庆为人很机警,对任何事很有见解,在艺人中极讲义气,可称为识时务之人。没几年的光景,袁傑亭患神经病,医治无效,便与世长辞了。由这个时候起,说《施公案》的人就没有能够跟群福庆并驾齐驱的了。群福庆在评书界里,由一出艺就挑帘红(出艺就火),红了三十年之久,他那说丑官儿(《施公案》)的魔力也很可观的了。惜其为人不善于料理生活,虽然红了这么些年,仍然是家徒四壁。到了民国二十二年冬月竟因病而亡。身后很是萧条,他所收的徒弟是刘荣安、刘荣云、傅荣庭、张荣久、陈荣启、许荣田、孔荣清等。傅荣庭虽给群福庆“爬萨”(爬萨是磕头认师傅,又叫叩瓢儿),他向来没说过书,未入此道。孔荣清自从给群福庆“爬萨”后,就一直在奉天、黑龙江等地献艺。东三省使“丑官儿”的评书界演员,就数孔荣清有万儿(有名儿)了。许荣田、张荣久、陈荣启三人,现在北平说书。张荣久、许荣田因为体质多弱,未能大露头角;陈荣启以使“丘山”(《精忠传》)见长,“丑官儿”这部书不常演。演《施公案》的演员,袁傑英说得最能叫座了,并且使的活儿“包袱”最多,有些好听滑稽玩艺儿的书座儿格外爱听,每日他在各书馆开了书的时候,“询局”(听书)的人们总是上满座儿的,袁傑亭有知也当含笑于九泉了。





评书艺人刘荣安


刘荣安这个人,长得身躯矮小,好像《施公案》的灶王爷。他有兄弟叫刘荣魁,会说“大瓦(wà)刀”(评书界的人管说《永庆升平》的调[diào]侃儿叫使大瓦刀,因是书之第一人物马成龙当过瓦匠,会使大瓦刀。在康熙私访月明楼时救过驾,故此他们评书界说书人管《永庆升平》调侃儿叫大瓦刀),久在东三省,永不回平。他们昆仲原都是饭馆跑堂的。刘荣安因为嗜好评书,专爱听白敬亭的《施公案》,他就说了评书。当他初次说书时,也未拜认师傅,在宣武门外赁了个场子,贴报儿就说书,他那报上写的是刘海泉,颇招评书界人不满。按着说评书的支派,那个刘海泉的海字辈数最大。当初,清中叶时有肇弘六者,系清室黄带子,按弘字辈与乾隆帝一辈,他的艺名叫肇海鸣,专说《明英烈》,颇有声望。到了清末时,评书界中早没了海字辈的艺人了,有人瞧见票友下海的敢贴报叫刘海泉,焉能愿意?就找了老说书的去携他家伙。携是个行话。携家伙时,是伸手将手巾往醒木上一盖,小笸箩一扣,扇子往笸箩上一横。如若拜过师傅,有门户说书的不怕这个,他拿起扇子说套词赞,拿起笸箩说套词赞,拍下子醒木,说完了词赞,照样说书。那来携的人就没有办法,道句辛苦而去。如若不会说这几套词赞,就没有师傅,没门户,那来携的人就将扇子、手巾、醒木以及所挣的钱都拿走,并且还不叫说了。那早年吃生意饭没门户是不成的。到了如今,没有门户的艺人,没有师傅的艺人,很多很多。如若有人来携家伙,那在法律上是不容许的,和他打官司,至轻也打个诈财的罪名。那刘海泉见有老说书的来了,他真伶俐,赶紧请安叫师爷。那老说书的被他恭维得不好发作,只说:“你赶紧找门户,认了师傅再说。”他诺诺应声。从那天起他就不说了,也见不着刘海泉的报子啦。他后来托人疏通,拜群福庆为师,艺名就叫刘荣安,他还是个大徒弟,师兄弟十数人,都叫他大师哥。他出艺虽早,口齿不大清楚,嗓音也不大,说得又不精彩,二三十年了也没成名,终日奔波,所挣的钱仅够衣食之用。艺人不成名的也是很多呀!





田岚云


说评书的艺术和唱戏的艺术都是一样的。唱戏的角色分为生、旦、净、末、丑,表情分为喜、乐、悲、欢。文讲做派,武讲刀枪架儿。评书的艺人每逢上台,也是按书中的人物形容生、旦、净、末、丑,喜、乐、悲、欢,讲做派,讲刀枪架儿。评书的刀枪架儿最好为何茂顺、高胜泉、田岚云三人。何茂顺专说《东汉》、《明英烈》,他是挂子行(练武术)的人,并且不是腥(假)挂,他那把式得过真传。在光绪初年时,他的叫座魔力是很大的,每逢说《东汉》,说到马武、岑彭打仗的时候,抬手动脚,比几手儿刀枪架儿,特别精彩。有些个夜叉行(黑道)的人,不在乎听书,为看他的把式的,颇为不少。

何茂顺有三个徒弟,长为奎胜城,次为高胜泉,三为刘胜常。当何茂顺病至不可救时,将徒弟三人唤至榻前,问死后之事。这三个人,或云他买棺材,这个开发杠钱,那个给开发棚钱。何令高胜奎、刘胜常退出,独留奎胜城一人,在病榻授艺,以竹筷两根当作双钩,传授他几手护手钩,奎学会了,令其退出。又唤高胜泉入,以竹筷一根当作长枪,传其几手大枪,高学会了,令其退出。又唤刘胜常入,以竹筷两根当作双锤,传授几手锤法,刘学会了令其退出。这是何茂顺教徒弟临终时授艺的事。

奎胜城久在花市一带,他说《明英烈》,说到伍殿章取金陵的时候,格外多上座儿。按:伍殿章与胡大海、汤鼎臣、朱洪武、邓万川、常遇春、郭英为盟兄弟,胡等六人的武艺皆伍殿章传,伍惯使护手钩,系清真教人,今牛街尚有他的后人。奎胜城学有八手钩,故说伍殿章在小月屯大战康茂才时,比仿几手钩极有精彩。他叫座儿的魔力,较比乃师有过之无不及,自称为净街奎(以该街有他说书,本街别的书馆能够没有听主,该处书座都听奎胜城,本街书座俱为他一人叫去。因他有这等特殊的力量,称为净街奎)。后因他说书的时候,爱往回倒书(说过去的段子又说,听书人最厌恶此事,不说是倒书,讥诮曰“倒粪”),故奎胜城不叫座儿的时候,都呼他为倒粪奎,奎胜城也因此一蹶不振。他是净街奎而兴,倒粪奎而衰。书座儿讥诮艺人也甚可畏也。

高胜泉系梨园行人,曾在某处当过箱头,后拜何茂顺为师,久在南城一带说书,会的活儿很宽,《明英烈》、《东汉》、《水浒》、《三国志》都能拿得起来。他向不修饰外表,专讲充实内容,广览多读,时人称双厚坪口才第一,高胜泉腹阔第一。他每逢说到盔甲赞儿,人们最爱听,他的赞儿与众不同,能够说完赞儿,人名归到“驳口”(每逢说完一段,一拍醒木,调[diào]侃儿叫驳口)上实为不易。他说的《水浒》有人听过,说到花和尚鲁智深时,有套赞儿,他说的是:“看和尚,真放样:晃荡荡,高一丈。青头皮,光又亮。大环眼,努着眶。那汗毛,一指长,手使一条铁禅杖。有人要问名和姓,江湖人称花和尚。”末句是花和尚。说林冲的赞儿,末句是林冲;说武松的赞儿,末句是武二郎。一百单八将共有一百零八个赞儿,此外还有几十个赞儿:武大郎、潘金莲、阎婆惜、潘巧云、潘老大、海和尚等俱都在内。现如今评书界会说赞儿的很少,恐怕将来要失传了。高胜泉说的大枪最好,在说到常遇春、姚期的时候,比仿几手儿,颇有可观。高胜泉的大枪最为出名。

刘胜常久在西北城一带说书,为人憨直,书里不掺包袱儿,专以评讲叫座儿。他说《明英烈》、《东汉》,说到后半部时能上座儿。《明英烈》的书内,有朱沐英使金锤,刘辅使铜锤,赵继祖使铁锤,李文忠使银锤。他说到八大锤会战吕巨的时候,亮出使锤的像儿,最为好看,比仿几手锤,也颇可观。刘胜常的大锤最为有名。有一次,他们师兄弟在一起谈心,奎胜城欲将八手钩传与两个师弟,高胜泉要将八手枪传与师兄弟,刘胜常要将八手锤传给二位师兄。三个人费了好几天的工夫,彼此串换活儿,白劳神费力,还是奎胜城的钩好,高胜泉的枪好,刘胜常的锤好。何茂顺的传授之绝,其妙可知。早年的艺人,将艺业看得很重,虽是自己的徒弟,也不肯倾囊而赠,艺人的艺术在早年是不公开的呀!种种艺术失了传就是这个原因。

高胜泉所收的弟子有三:一是马岚波,二是宫岚彩,三是田岚云。马出艺便红,惜未长寿。宫善于拉长,也非全才。田岚云系官吏出身,精于武术,广览多读,博闻强记,颇有乃师之风。也出艺便红,叫座的魔力很大,能说《明英烈》、《东汉》、《水浒》,能在台上跳跃,刀枪架儿最为美观。虽五十有余,老当益壮,搬个朝天镫,抬腿就来,凡是听书的人们都大捧特捧,有的是疙瘩(gē da)杵儿(格外多给钱,调[diào]侃儿叫疙瘩杵儿)。他嗜酒如命,性情过刚,颇有侠风,专好路见不平。向来是独树一帜,概不联络,做事光明磊落。同业人有品行不正的,常受其辱,都很惧他。他生平最尊重王傑魁,因王品行诚实,道规道义,能有能守,在台上向无蹬、踹、捧、卖的劣行。有一次,王在东安门外某书馆说《包公案》,正说到邓家堡,北侠欧阳春宝刀吓群贼,神弹子邓车用连珠弹打北侠,北侠的宝刀,刀削弹儿。该书馆的书座儿太监居多,有某太监挑(tiāo)眼了,怪他不该说“刀削蛋儿”,一人作倡,众人附和。王是老实人,向不骂书座儿,他忍气,离开了该书馆不说了。事为田岚云所知,他托人把自己介绍给那书馆,他要给王傑魁出气,斗斗那群“念湾”(江湖人管太监调侃儿叫念湾)们。他在该书馆说评书时,借机将会武术之某念湾大骂一通,直骂了两个月,方才算完,也评书界之轶闻也!

田在菜市口如意轩内说书时,有某阔少在该书馆内大出风头,为田所恶。田探知他好养金鱼,一日在台上不说书,大谈鱼谱:何为望天鱼、花腮鱼、绒球鱼?如何收藏?如何甩子?如何分盆?春夏秋冬四季养鱼之法。这阔少听得入神啦!田说,赶上阴天,连着下雨不止,鱼把式无处打鱼虫,向养鱼主人说:“没有虫子如何好?”养鱼主人用手指其粪门说:“我这里有虫子。”说至此处,全屋书座都知道田暗骂阔少,哄堂大笑。这阔少明知是绕弯骂他,但因惧田之武艺,未敢发作,受窘而去。田在场上临时抓哏,讥诮时事,借题发挥,绕弯儿骂人,无日无之。后竟因此受累,各书馆主人多不邀请他,末了,田岚云因受窘而亡。

武说书的故去之后,刀枪架儿也随着失传了。今之说评书《聊斋》的陈士和,抬手动脚,发托卖像,颇似田岚云,评书界人称其为“武《聊斋》”。陈现在津埠献艺,久未返平。凡有好听陈士和的“武《聊斋》”《田七郎》、《崔猛》的人,每日广播电台播来之音,北平即可收听。科学万能,北平人能听到天津的玩艺儿了。





评书界艺人曹卓如


说评书的艺人所说的书,是分为大枪杆儿、短打两路儿。使大枪杆儿的所说的书是《东西汉》、《三国志》、《水浒传》、《隋唐传》、《精忠传》、《盗马金枪传》、《明英烈》,使短打的所说的书是《济公传》、《彭公案》、《善恶图》、《于公案》、《施公案》、《包公案》、《小五义》等等。

说《聊斋》的是另一派,也不算短打,也不算大枪杆儿。在早年还没兴《聊斋》,有说《聊斋》的也是铺红毡子(评书界人管说子弟书不要钱调[diào]侃儿称为铺红毡子)。东城有位说子弟书的刘逢元,专说《聊斋》,颇有些人欢迎。他虽是个票友,与挣钱的评书艺人较比起来是有过之无不及。张智兰老先生下了海之后,说《聊斋》的才大兴其道。

曹卓如是西城人,他从前是在某衙门当差,家道小康,博闻强记,嗜好评书,专爱《聊斋》。拜任俊山(任俊山系某教教友,专说《忠义西巡》享名)为师,艺名曹聚锐。自从登台献艺,总未得志。后来他说书报子上不写曹聚锐,写曹卓如。他是念单招(江湖人管一只眼的人调侃儿叫念单招),一条夯(hāng)(江湖人管一种嗓子,似哑不哑,不能变嗓音说话,调侃儿叫一条夯),没有发托卖像(即是没有生、旦、净、末、丑,喜、乐、悲、欢的形容),坐在凳儿上不动地方,坐谈今古,凭嘴一说,要享大名,实在不容易。他前边有个说《聊斋》的名角儿陈士和,如同一面影壁似的挡着他,愈发得不易成名。幸而他有百折不回之志,说了七八年渐渐有名,很有些个主儿爱听他那《聊斋》。费了好几年的光景,才成为二路角儿。可是他的书是四九城儿都能叫座儿。西安市场春华轩、增桂轩、长顺轩,后门外义溜胡同广庆轩,天汇大院开明轩,东四牌楼宴新茶社,五条胡同华友轩,齐化门(今天的朝阳门)外义和轩,西直门外庆平轩,宣武门内森瑞轩,花市三友轩,天桥福海居,菜市口如云轩,彰仪门内文雅轩,报国寺前得胜轩,全都说过,哪个馆子都能叫多半堂座儿。凡是好听评书的都知道有个曹卓如。他的师兄魏聚宽、师弟德聚明,都未享名。聚字的评书艺人,就数着他曹卓如了。他又收了两个徒弟,大的叫魏英信,二的叫赵英颇,魏是近视眼,赵也有眼疾。他们师傅徒弟招儿(眼睛)都有点念(不好),魏说《水浒》,未到成名即死在石家庄了;赵英颇承其师之衣钵,专说《聊斋》。现在市面不景气,赵英颇赶上这个时候,成名也难了。

曹卓如在评书界是个老实说书的,对于捧踹术(说人好,说人坏)是不会,论其收入颇可糊口,不料在这二年来,各书馆不见有他的报子。我老云向该界人打听,据说他因老来丧子,得了瘫痪病了。我老云日前在菜市口如云轩去听评书,遇老友杨敬斋先生,谈及曹卓如之事,杨老先生素敬卓如,虽然年近古稀,为了探望他,不辞劳苦,由西南园寓所往西直门中秀才胡同五号,去看曹卓如。杨敬斋先生归时访我老云说,曹卓如对他诉苦,因遇有重病缠绵,不能说评书,无法挣钱,只有十几亩地,靠人去种,每年分些粮米,勉强支持,实可叹也。并且说,和我老云已有二年多没见了,想念异常。敝人每日埋头书案,度笔杆的生活,如笔债缠身,竟不能往看曹卓如,也觉郁闷。曹先生的口债已然还清,我老云的笔债尚无了期。都说人情如纸薄,曹卓如病了二年多无人探望,今有书友杨敬斋去慰问一次,也可称为知音者也。





评书界之刘继业


说书的亦游艺之一,与戏剧、影片、歌曲同占艺术上之位置。书有大小之别。小书在南方最盛,因小书多是风流韵事,演时有弦相佐,或男女合唱。江湖人调(diào)侃儿说他们是鸳鸯档子,专以吸收女客而诱惑男客,实有伤风化,影响于社会也。大书是以忠、孝、节、义、礼、义、廉、耻为主体,甚合北方人之味道,盛行于鲁、冀、晋、察、平、津也。北平为说评书发源之地,所说的书《包公案》、《于公案》、《施公案》、《东西汉》、《精忠传》、《隋唐》,穿插紧凑,道活(辈辈相传的评书)秘本,口传心授,颇有精彩,故有百听不厌之妙。“串花”是评书界的侃儿。北平的俗语呼乞丐为花子,《济公传》中的主角是济公,因为他形如乞丐,和化小缘的一样,行话管说《济公传》的就叫“串花”。早年以陈茂胜之徒一声雷陈胜芳说得最好,其次为张霈然,若文岚吉、高福山等辈皆平庸无奇。评书大王双厚坪在世时也常演“串花”,发托卖像,形容最好,当场能抓现哏,诙谐百出,真有“翻堂的包袱儿”。什么叫翻堂的包袱儿哪?江湖艺人,不论是哪行,在台上把人逗笑了,调(diào)侃儿叫“抖包袱儿”。多好的书料也不如好包袱儿有价值。若是抓哏、抖包袱儿没有人笑,调侃儿叫“闷了”,艺人必窘,当场难看,实是顶瓜(江湖人管可怕调侃儿叫顶瓜)。若能把全场的书座儿全都逗笑了,那调侃儿叫“翻堂的包袱儿”。单弦中随缘乐、德寿山,相声里万人迷,评书界双厚坪都有此拿手活儿。双厚坪故后,其徒杨云清摹仿,只有一二。其余的别说翻堂的包袱儿,就是素包袱儿也多不会使。在清末的时候有评书界怪人士殿城(现在北平说评书的品正三即其子也),能说《隋唐》、《聊斋》、《济公传》,专拱(使)“蔫包袱”,几句不要紧的事,使人发笑,颇有叫座儿魔力。自双厚坪、士殿城故去之后,说“串花”的艺人能继双、士之后者,只有刘继业一人而已。

我老云日前有事到东城,偶在东安市场仁义轩见有说《济公传》刘继业海报,好在我白天无事,也可听听评书。约有下午三时,刘即登台。视其人身躯瘦小,脸面微麻,调侃儿说“梅花盘”也,约三十余岁。“夯(hānɡ)头(嗓子)”有限,喷口(嘴皮子上的功夫)最好,远近适宜。我原是略听一会儿就想回头,不料彼之艺术娓娓动听,引人入胜,乐而忘倦。他抖的包袱儿接连不断,荤、素、蔫三样皆有,还有翻堂包袱儿,实胜于相声。不怪座儿拥挤,实是其才灵敏,艺术高超,与众不同。直到掌灯时终场而归,途遇友人高君,偶谈刘艺。据高君说:“刘继业久居西城新街口一带,其父系棚行人,曾开布棚铺。自民初至今,布棚一行受了淘汰,继业即拜士殿城为师,学演‘串花’,后又得了道中秘本,能说至五云阵、小西天,同业人无有能及其艺的。其艺术之高,能以评书陶冶人情,感化社会人心。四九城均有叫座儿魔力。为人勤俭,无嗜好,不奢华。侍父最孝,十数年红运,置有薄产,小有积蓄。年前彼接有匿名信,受匪人恫吓,迁居数次,不敢贴刘继业海报,拧了万儿(江湖人管改了名儿调侃儿叫拧了万儿),改叫刘中轩,盖其人胆小心细也。不料书座儿不知刘中轩为谁,皆裹足不往,很受相当影响。由今春就休息静养,未能登台。现在有友人相邀,始在东安出演,因受过拧万儿的影响,他的‘幌幌(huàng)’(江湖人管海报儿叫幌幌)也书名刘继业了。”

评书一道虽占艺术位置,势力还不及戏剧百分之五六,因有褒忠贬佞,引人向善之力,一般守旧礼教的人们还是嗜于此道,虽金钱奇窘,尚能维持百数多艺人生活。想评书一道,不及普及社会.仅能敷衍,也没有进化改革之力也。戏价已入贵族化中,评书尚守平民化故辙,听一天书茶资一毛钱,尚有富余,无怪闲散阶级人皆嗜评书了。





连阔如、陈荣启、郭品尧、苗阔泉


连阔如说的《东汉》,纯粹是道活(辈辈相传的),不是墨刻。阅者若问什么是道活?什么叫墨刻?关于这两个意思我得向阅者述明。说评书的人们所说的书,虽有《施公案》、《济公传》、《彭公案》、《精忠传》、《包公案》、《明英烈》、《隋唐》、《东汉》,可是大有分别。就以《三国志》说吧,从前,评书界很有几个人说的,可是所说的书中人物、段子,都与各书局所售的书本中一样,不过加上身段表情和刀枪架儿,用白话评讲而已。评书界的人管他们说的书与书局所售的本儿一样叫做“使墨刻儿”(书局里的书,都是笔墨写出原稿刻版印行的,故叫那些书为墨刻儿),可是评书界的人都不愿意使墨刻儿。话又说回来了,他们说的书和本儿上要是一样,听书的主儿如若心急,就不用天天到书馆去听,花几角钱在书局里买一本书,几天能够看完,又解气又不用着急,谁还去天天听书,听两个月呀?评书界的演员所说的评书,最贵重的书叫做道活。据我所调查的,评书界纯粹道活的书,有《施公案》、《大宋八义》、《济公传》、《永庆升平》、《彭公案》、《包公案》。这原是评书界的道活秘本,已在早年有人售与书局,书局得了版权,印行售卖,已非道活,由道活又变为墨刻儿了,故不算道活。《精忠传》、《隋唐》、《东汉》、《明英烈》、《盗马金枪传》、《五代残唐》、《善恶图》、《于公案》等等的说部,这些个道活书都是古今名人与评书界老前辈攥弄的(江湖人管编书编戏编曲调[diào]侃儿叫攥弄[zuàn nong]活儿)。

先以《东汉》说吧,各书局售卖的《东汉》,都是东西汉两部书合在一处卖,《西汉》如何,不必论它,只说《东汉》吧。共是两本,由王莽篡位,立孺子婴为帝,王莽摄政,至永平皇帝逢云台止,书中的穿插不严,段段的岔头儿都接不上,也不紧凑,看着当然无味,不能引起兴趣,那墨刻的《东汉》是不能看的。道活《东汉》是由王莽篡位,刘秀走国、马武大闹武考场说起,直到上天台,马武打金砖,二十八宿归位止。其中的节目有刘秀赶考,箭射王莽,窦融救驾,岑彭出世,马武大闹武考场,会英楼题反诗,刘秀遁潼关,路遇姚期,凡百余段。与书铺的墨刻儿不惟不同,并且穿插紧凑,枝叶搭得最严,毫不懈松;使人听了能够“入扣儿”(江湖人管好听书的人听得上了瘾,非接连不断往下听,说行话叫入扣儿)。江湖艺人常说:“唱戏的要想叫座儿,得有好轴儿;说书的要想叫座儿,得有好扣儿。”什么叫好轴儿哪?比如某戏园子要唱一台,贴出海报儿,头出《大赐福》,二出《善宝庄》,三出《四杰村》,四出《朱砂痣》,五出《坐宫盗令》,六出大轴儿是杨小楼、梅兰芳唱《霸王别姬》,这几出戏合在一处,能卖一元多钱一个座儿,能共卖一千多元,上的这些个座儿能卖这些钱,力量都在那出《霸王别姬》哪!如若将《霸王别姬》去掉,就那几出戏,卖三毛钱一个座也许没人听,那《霸王别姬》就算好轴,能叫座就能挣大钱。说评书的演员要想叫座挣大钱,都得有好扣儿。这书扣儿又与戏的大轴不同,有小扣儿,有碎扣儿,有连环扣儿,有大扣儿,最大的扣儿叫大柁子。他们说评书的,每天到了书场或是书馆,等着书座儿来了,到了开书时间张嘴说书,先用小扣儿,次用碎扣儿,再用大扣儿,才能吸得住座儿,挣大钱。比如说《东汉》吧!开书先说刘秀拜马援为帅,姚期不服,与马援赌头争帅印,如若姚期用三千兵打破潼关,马援将帅印输给姚期;如若姚期打不开潼关败了仗,姚期将人头输给马援。听书的人最喜爱忠臣,都替姚期担心,怕他打不破潼关,将人头输了,都坐在凳上不动,要听姚期胜负。这样便算书座儿入了扣儿,这就是说书的演员使小扣儿。听书的人不动了,说书的人往下说,姚期还没到潼关,离城三十里就被王莽的兵将打败了,岑彭给姚期打接应,掉到陷马坑里,岑彭被王莽兵将生擒活捉入潼关。听书的座儿听到这里,又替姚期骇怕,怕回去脑袋没了,又怕岑彭死在潼关,这样就不走了,非听个水落石出不止,这就叫碎扣儿,将座儿扣住了。这样说,就是说书的演员用步步连环紧的法子,将书座儿吸住了,直听到临散场的时候,听出两个岑彭来,书座儿更纳闷了,怎么会多出一个岑彭呢?真叫人纳闷。离了书馆,回到家中,吃饭、睡觉还是纳闷,无法解决,只好明天早早去书场,接着再听下去。这样便是评书演员使用大扣儿。使用大扣儿为的是吸住听书的座儿明天再来听书。听到明天散书时,又听到马援巧使连环计,书座儿又纳闷了,不知马援使的是什么计能得潼关,明天再接着去往下听。即使四五天才说完潼关,那潼关这段书就是四五天的大柁子(最大的扣儿)。说评书的没有小扣儿,吸不住座儿;没有碎扣儿,拉不住座儿;没有大扣儿,不能吸住回头再听的座儿;没有大柁子,就不能吸住听五六天的座儿。看起来,说书的扣儿、柁子,较比戏场的大轴儿还有吸引力。

这评书的道活儿(辈辈相传的)是艺人艺术化说,如若艺人学会了就能叫座儿,评书界人常说“书说险地才能挣钱”。我问过他们,什么叫书说险地?据他们解释说,不论是袍带书,公案书,凡是听书的人,都是一样的心理喜爱忠臣孝子、义夫节妇、侠义英雄,都恨奸臣佞党、贪官污吏、土豪恶霸、绿林的采花淫贼。就以《施公案》说吧,施清官往苏州上任,就有一枝兰万永拦轿行刺、府衙行刺、错杀舅老爷这三段书,叫听主儿听着净替清官施大人担惊受怕,坐着不走,要听到清官没有危险了才肯走,这样的事便算书说险地;如果叫听书的主儿知道施大人没了危险,那就不听了。评书里的情书,段段书都是这样的。

连阔如在民国十二三年是个做八岔子的金点(江湖人管算卦的调[diào]侃儿叫八岔子,算卦的总称曰金点),自从民国十六七年时改入评书界,拜李傑恩为师,讲演《西汉》,在各书馆也颇有叫座的魔力,但未大转(zhuàn,发达)。未几,又学说《东汉》。我老云问过他,为什么改说《东汉》呢?据连阔如说:“《西汉》那部书是墨刻的,与各书局所售者相同,听这部书的座儿很少,不懂历史的人不能听,懂得历史的人花两角买部《西汉》,几天就能看完,较比听书又短少时间,又少花钱。好在他们说书的所说的段子,与买的书内一样,何必去听评书?评书界的艺人说墨刻书的都不能挣大钱,就是那书拉不住座儿。”他有了这种觉悟,便弃了《西汉》不说,改学《东汉》,牺牲了半年的光阴,耗费了许多的金钱,才学会了一部地道的道活(辈辈相传的)。自从会说《东汉》,北平的大书馆儿才纷纷地约请。听书的座儿都知道评书界有个说《东汉》的连阔如。有年夏天,连阔如因书馆都不凉爽,在天桥赁了个场子,高高的天棚,宽宽的板凳,又凉爽,书又说得好,天天高朋满座。连阔如叫座儿的力量就仗着那道活的《东汉》。

陈荣启(1904—1972)在表演评书(照片由徐雯珍提供)



陈荣启为人憨直,系评书界说《施公案》陈福庆之子,拜群福庆为师,先说《施公案》,后说《精忠传》。在民国十年前后,评书界人才济济,本领弱者受挤,无法挣钱,纷纷出外另谋生路;后起之人,有老前辈挡着,不易发展,也都出外另谋生路。陈荣启乃评书界后起之秀,能说袍带书《精忠传》,短打书《施公案》,实是不可多得的人才;在民国十年前后,往大连、烟台、营口、天津、东三省等地献艺,到处受人欢迎。在北平虽没立住脚儿,在外穴(xué)大转(zhuàn)(即在外地挣了大钱了)了。自民国十八九年始归北平,愿侍高堂,不愿远行,又赶上评书界的前辈名角潘诚立、张智兰、田岚云等都去了世,后起无人,缺乏人才之际,在北平献艺,四九城各书馆,都能叫满堂座儿,足见北平人士欢迎他了。他为人怪癖,不愿在各书馆说书,专喜爱在天桥。前几天,我老云往天桥去了一趟,见他在爽心园前占了个场儿,与他师兄许荣田说前后场书哪,前场许荣田说“丑官”(《施公案》),后场陈荣启说“丘山”(《精忠传》),还真叫座儿,有爱听评书的,快去听吧。

在天桥城南商场的南边,有个评书场儿,说书的艺人叫郭品尧,他是一年四季不挪地方,长期的上那场儿,无论春夏秋冬,总上满堂座儿。他所说的书有《粉妆楼》、《五代残唐》、《飞龙传》、《施公案》等。我老云听过他多少次玩艺儿,听他说的几部书都不是北平的评书界道活,也不是书局里卖的墨刻儿(书),我向评书界的人探讨过几次,才知道他说的那些书是竹板书改的。据某江湖人说,郭品尧是北平人,曾在清末拜冯昆治为师,学说相声,起名郭伯全。他在外省改唱竹板书,改名郭鑫德。后又在天津拜福坪安为师,改说评书,更名为郭鹤鸣。按着北平评书界传流的支派,与说《水浒》的蒋坪芳、徐坪钰、刘鹤云等是同门人。不料,郭到北平时,评书界的南北两派正起内讧争持不决之时,他投南未入北,几与本门人决裂;便在天桥上地(说书),概不联络,独树一帜,不进书馆。他所说的虽不是道活(辈辈相传的),系竹板改造,也有些人欢迎。外江派的评书演员,能在北平久占的,只有郭品尧一人。老云曰:郭亦人杰矣哉。

苗阔泉是梨园行人,自少年嗜好评书,专喜爱听大小黑脸儿(评书界管《三侠五义》即《包公案》那部书调[diào]侃儿叫大黑脸儿,管《小五义》那部书调侃儿叫小黑脸儿,大小黑脸儿乃指包文正的黑面也),拜金傑华为师,学说大小黑脸儿,进了评书界。虽没登峰造极,也成了二路角色,久在彰仪门、报国寺、山涧口、西安市场上馆子,能叫七八成座儿,颇为不弱。他除了这几处之外,受同业人们排挤,就没有馆子可上。苗阔泉也有志气,他除了这几处馆子他上,别处约他还不去,没有馆子便上天桥打个场儿,露天讲演。别看他上明地(露天演出),较比在书馆还多挣钱。故此我老云常说,有真本领的人是不怕排挤的。

近几年来,闲散阶级的人日日见少,听评书必须有闲工夫,闲人少了,说书的座儿也受影响。那位说,北平的闲人有的是。我说那不是闲人,是失业的人,他们虽闲着,吃饭还困难哪,哪有钱去听评书?听评书的闲人,是有资格的闲散人物,不是没有钱的闲人。如今我调查了几处,各评书馆的座儿全都减少,开馆子的维持不住的已有数家,其余的都是扎挣劲儿,勉强支持。评书演员有许多的都往天桥找地,据我预料,今年夏天天桥的评书场儿要比往年多得很哪!有些说书的艺人还想不开,认为在天桥上地(说书)是寒碜,还不肯去上明地。其实,早年的评书演员都是在大街的路旁拉场子,露天讲演,在天桥上明地何足为辱?挣钱养家便算好手,何分彼此?我很希望说书的艺人迎合听主,往天桥上地,来个说书的大比赛,倒是热闹。好听书的人们乘此机会,又逛天桥,又听评书,不可错过这个好机会。





评书界之艺人哈辅元与《永庆升平》


哈辅元是蒙古旗人,乳名叫双儿。在少年时被象用鼻子卷起过一回,那象并没摔他。有些人说:“双儿命大。”他长得品貌端正,口齿伶俐,长于言谈。专爱养吧狗儿,善于修饰,北平人都说他是个漂亮人物。

有姜山东者,在北平经商有年,后因营业亏累,赋闲无事,常往各市场、庙会游逛,听相声说的小八段:《张广太回家》、《五龙捧圣》、《康熙私访》、《马成龙救驾》。几日,听会了,穷极无聊,就在路旁讲演这几段玩艺儿。他虽是山东人,说北平话最好,不知者难料其为山东人也。他学马成龙讲山东话,较比各种艺人灵通(山东人说山东话岂不说来就行),很有人欢迎。姜山东以说小八段儿挣钱糊口,生活无忧,惟恐有艺人阻挡,乃投入评书界,拜师认门户,艺名姜振明。

哈辅元见姜鬻(yù)(卖)艺糊口,颇为羡慕,每日必听此短段评书,归家时,茶余酒后就以说评书消遣。亲友见他颇有心得,劝他拜师鬻艺,他遂拜姜振明为师,按本门支派赐名为哈辅元。哈辅元自从登台献艺就大受社会人士欢迎,都说他是挑帘红(出门就红)。

我老云在读书时,曾因逃学去听姜振明的高足弟子哈辅元说评书,受责数次。哈之艺业颇有几种特长,为同道人所不及。《永庆升平》这部书说康熙私访月明楼,捉拿四霸天,五龙捧圣,大闹兴顺镖店。据评书界人说,在清室时代,北平居民以满蒙汉的旗人为多。旗人是每月关(发)旗饷,按春夏秋冬四季关老米,衣食无忧,提笼架鸟,茶馆聊天,按庙期游逛。所谓闲散阶级人,清时最多,评书是闲散阶级人消磨岁月爱听的艺术。《永庆升平》这部书,是以北平旧社会仓库(官私)两面,跳宝案子、耍人的混混儿为主体,旗人是欢迎的,是爱听的。哈辅元对于虚字谱、光棍儿论、混混儿派儿,大有研究。每逢登台献艺,说到这种事时,模仿得最好,使听书的人们听着真如身临其境,处处逼真,是其惊人之处。并且他“变口”(管北平人学说山东的话儿,学说南方人口音,学说山西人口音,评书界的侃儿叫变口)讨俏,哈之台风最好。评书界的人常说,我们说评书的艺人,不出一怪,得出一率,才能响万儿(出名),火穴(xué)大转(zhuàn)(在一地方演出挣了大钱了)。双厚坪以说评书夹杂当场抓哏,临时的相声,颇受社会人士欢迎,称为评书大王,叫座儿的魔力为同道所不及,即是艺人中之怪也。说评书的艺人,相貌端正,身上衣服干净,口齿伶俐,语言流畅,是为一率。哈辅元就以此成名,他的叫座儿魔力也为同道人所不及。有这种特长,焉能不享大名,不坐头把交椅?

从前没有《永庆升平》这部书,只有撂明地(露天演出)的艺人评讲《康熙私访》、《五龙捧圣》、《张广太回家》等等的八段儿。姜振明、哈辅元师徒将天地会、八卦教串入书中,编出二马下苏州、大逛虎丘山、闹福州会馆、马成龙卫辉府搬兵等等节目。是书由五龙捧圣起,直到破了天地会为止,穿插紧凑,情节逼真,枝叶搭得严密。他师徒完成此书,评书始增一部道活(辈辈相传的),然也煞费苦心也。在清室的时候,《永庆升平》书运最佳,说得好了便大红大紫;说得不好也能挣钱,不过少挣而已。在那时说《永庆升平》的艺人,占评书界全部人十分之四,并且评书场儿都在西单牌楼南北,西四牌楼一带,阜成门里外,东单北,东四一带。交道口等处还是书场相连,不远就一场。如若四五场评书,有一个场说《永庆升平》,最附近的场子都受影响。书运好,同业人也都惧怕。至今时代变迁,社会风气也与从前不同,《永庆升平》这部书又不合时代,凡是说他的艺人,无论好坏,全不叫座儿,无人欢迎,《永庆升平》是落了伍啦。回思往年,不胜今昔之感。

哈辅元家住西城宫门口,夫妻二人并无子女,惟有爱犬有如爱子。每至冬令好睡热炕。不料某年腊月三十日,度除夕,天至四更方安歇,被角落于炕下,被火引着,一片红光,火光大作,烈焰飞腾,小火引起大火。哈辅元夫妻与其爱犬同被火焚而死。当时,故都人士于茶馆酒肆、街谈巷议中对于哈辅元毁誉皆有,其死之惨,令人鼻酸。其故后,一般老听评书之人每念哈辅元,犹不胜怀忆也。





张杰鑫与《三侠剑》


现在北平说书的艺人最有名的陈士和、金傑丽也都是在天津上书场儿,真挣钱。我老云调查天津的露天书场,北开、地道、乾德庄虽然都有,还是三不管(天津市南市的一个露天市场)最多。在三不管久惯说评书的艺人有个顾桐俊,他父亲叫顾瞎子,水性最大,说书未享大名,可是他儿子要说书,不能父子门,得另拜师傅。他儿子乳名叫小鳖,投在乔云章的门下,艺名叫顾桐俊(北平说评书的艺人傑字辈是英雄豪傑的傑字;天津英致长、王致久收徒弟,另使《四杰村》的杰字。英致长的徒弟,还有叫云字的,乔云章就是云字的,乔系天津说《封神》的名人乔墨林后人)。顾桐俊体胖面黑,有点麻子,调(diào)侃儿叫“梅花盘”,专在三不管上场子,会说《大宋八义》、《善恶图》、《于公案》,很有叫座儿的魔力。不料我老云在五月节前到了天津去听评书,那顾桐俊已然没了,和三不管的人打听,都说顾桐俊已然土啦(江湖人管死了调侃儿叫土啦)。三不管说评书的人,由北平去的艺术最好是金傑丽说的《三侠五义》,陈士和说的《聊斋》,颇有叫座儿的魔力,可称是头把交椅。说《三侠剑》的有几处,都是张杰鑫的徒弟徒孙,马轸元、曹枢林、董枢敏等辈是也。

张杰鑫著《三侠剑》1948年版书影



张杰鑫,北平人,在天津拜王致久为师,将北平评书界道活(辈辈相传的)的《清烈传》改革了,独创一派,由清末民初就在天津埠献艺,很受人欢迎。他是挑帘红(开门就红),叫座儿的魔力最佳,提起张杰鑫来,几乎无人不知,可称天津的评书大王。为人忠厚耿直,品行端正,红了二三十年,始终不衰。其艺术之精,实是炉火纯青了。他收的徒弟共有四个,头一个叫马轸元,二个叫孔轸清,三个叫杜轸明,四个叫佟轸芳。至于王旭佩、曹枢林、董枢敏等二三十人,皆四大轸字之徒也。天津的评书支派门人弟子之盛,就数着他们这门了。

马轸元是金家窑的人,自幼学习扫苗的(江湖管剃头的调[diào]侃儿叫扫苗的),因嗜好评书,投在张杰鑫门下为徒,艺名轸元(天津的轸字的说书艺人与北平的阔字的是平辈,同一门户,马轸元等与连阔如、马阔山等,皆是本门的师兄弟)。他出艺最早,我老云头次逛三不管(天津市南市的一个露天市场)时,他就拉顺(江湖人管拉个场子调[diào]侃儿叫拉顺)了,至今数十年始终没响万儿(即是未成名)。据我考查,他不成名不是师傅的传授不真,是他碟子不正(江湖人管口齿不利落调侃儿叫碟子不正)。在民国十年前后,我老云到营口去过几次,那次夏天,走在洼坑甸露天市场,也见着马轸元在那里说《三侠剑》,在天津虽不叫座儿,在那里可有叫座儿的魔力,马轸元的团(tuǎn)柴生意,转(zhuàn)在外穴(xué)(团柴是说评书的,转在外穴是在外省发达了)了。

孔轸清好穿道服,在天津各茶馆各露天市场讲演评书,粘箔(nián bo)们与询家(江湖人管开书馆的主人调侃儿叫粘箔,管听书的人们调侃儿叫询家)都很欢迎。孔轸清人缘最好,乃张杰鑫得意弟子,是头路角色,不只在天津做艺,他在大连、营口、安东、沈阳、长春等地献艺,也有叫座的魔力,他这个说评书真是到处响万儿(有了名儿)了。最近我在天津听人传说他在东省做艺,因为丧女哀痛,得了不治之疾,已然不能登台。张杰鑫故去后,他又如此,说《三侠剑》的艺人,又该别人成名了。

杜轸明在民初时专在北开上地(做生意),演说《三侠剑》,使短家伙的数他第一。后因不愿剪发,离了天津,改走外穴(xué)(到外地说书)石家庄、保定府、张家口、唐山、济南、青岛等地,很有个万儿(名儿)。现在来平,每日在天桥爽心园前上地(说书)。我老云往天桥巡礼,曾听他三段,说得虽好,只太岁海(hāi)了(江湖人管年岁大了调侃儿说太岁海了),气力不佳,发托卖像(江湖人管做艺的人们到了表演的时候,脸上能够形容喜怒哀乐叫发托卖像),不如从前。二十年前的艺人,今日再见,使人更信做艺的道儿不养老不养小了。

在三不管(天津市南市的一个露天市场)有个说评书的艺人刘庆和,身矮体胖,台风最好。我曾听他说过几回《小八义》,只是不像评书的味儿,向外方探问,才知道他是使长家伙,柳海(hāi)轰儿改为短家伙(即是唱大鼓改说评书)。他是山海关的艺人牛德兴弟子,与唱大鼓的王庆发、李庆来为亲师兄弟,在天津颇有一部分人欢迎,也能立住脚儿。与北平去津的陈士和和金傑丽等比较,也不甚弱。其余的评书艺人,不是艺术不精,就是人才不济,皆不足称道的。





顺桂全与《铁冠图》


当初北平说评书的有个顺桂全,专说《铁冠图》。那是明末的故事,极不好说,说得不好没人听,说得惨了也没人听。大凡听书听戏都是解闷儿,越听越烦的书,哪能叫座儿?《铁冠图》又名《崇祯惨史》,说到崇祯到煤山自缢的时候,书座儿就光了。艺人指着多叫座儿挣钱,要是越说座儿越少,那还挣谁的钱?评书界的人不说这部书就是这个原因。可是顺桂全偏说定了《铁冠图》,直到死也没挣钱。他还收了个徒弟名叫桂殿魁,北平的说书艺人,殿字的、聚字的,比哪门人都少。殿字的最早有个梁殿元,住家在西四牌楼,先扫苗儿(剃头),后改行学说评书,专说“黄脸儿”(《隋唐》)。在平他未能得志,开了外穴(xué)(到外地去挣钱),到沈阳献艺,享了大名。东三省的说书艺人,他算最有万儿(名儿)。现在北平说《隋唐》的名角儿品正三,他父亲叫士殿城,现已故去。北平只有王殿远尚在,除他之外,没有使殿字的了。桂殿魁学说《铁冠图》也不叫座儿,走到天津三不管(天津市南市的一个露天市场)撂地,盘杠子(练武术)圆粘(nián)儿,改卖药糖了(指练武术招徕观众,好卖药糖)。





王致久师徒与《雍正剑侠图》


王德宝是个江湖艺人,他有两个名儿,又叫王致久。有人说他是穷不怕的徒弟,那实在是瞎聊。按:穷不怕艺名叫朱少文,他的徒弟叫徐永福,焦德海就是徐永福的徒弟。穷不怕是焦德海的师爷。凡是德字为名的说相声之艺人,都算是穷不怕的徒孙。即或不是他的嫡孙,就是旁叉儿,也得按着艺人传流的支派论辈数,不应当妄给他们胡论辈儿。王德宝是说相声德字辈的艺人。按着少、永、德三个字推论,绝不是穷不怕的徒弟了。

他说相声专以“贯口活”挣钱。使贯口活,必须嗓音圆润,口齿伶俐。百八十的词儿,由头到底一气数说完了,句儿分明,中间不准断节,没有气力也是不成。王德宝的“贯口活”有《饽饽阵》、《百鸟名》、《百虫名》、《滑梁子》、《菜单子》(江湖人管说相声净是地名儿的段子调[diào]侃儿叫滑梁子,管他们说的净是菜名儿的段子调侃儿叫菜单子),颇受故都旧社会的人士欢迎。他又拜关德志(关系评书界名人花瑞生弟子,《大宋八义》即花瑞生所编纂道活[辈辈相传的],他师徒以说《大宋八义》成名)为师,学习评书。按着评书界的支派,他们那门是:廷、瑞、德、致、傑、阔、增。关赐他艺名王致久。故此王德宝是春口(相声)的生意他也做,评书的玩艺儿他也说,算是个又团(tuǎn)春(说相声)又团柴(说评书)两样都干的艺人。

他在北平未能得志,离京赴津献艺,在津埠说评书未久即享大名。凡是北平的艺人,无论是说相声、说评书,只要到了天津,他全都扶助,荐馆、赁场子,竭力维持。江湖人因他义气最重,无不尊敬。他在津门收了几个徒弟,我所知道的有:吴杰森、许杰泉、常杰淼。吴说《大八义》未能得志。许杰泉说《小五义》,久走东三省,每逢夏季,有苍蝇从他面前飞时,他伸手就能捉住。东三省听评书的人们全都知道的,许杰泉也怪人也。常杰淼自己攥弄(zuàn nong)(创作)了一部道活(辈辈相传的)儿《雍正剑侠图》,是书虽在雍正年间,事由儿可是清末年间的。该书之胆童林童海川,即八卦门的名人董太监董海川也。王之弟子以常杰淼名望最大,今已故去数年。

常杰淼著《雍正剑侠图》1935年版书影



英致长、王致久在津埠为了另立支派,所收的徒弟,不用傑字,另用杰字,如乔杰章、常杰淼等。与北平说评书的艺人王傑魁、袁傑英、李傑恩,虽是本门的师兄弟,支派大同小异,尚有云、杰、傑的区别。

有人说王德宝是花瑞生的徒弟,那又不对了,按他们的支派是廷、瑞、德、致,他叫王致久,论辈数也是花瑞生的徒孙了。说错了的先生你再打听打听,王德宝是不是穷不怕的徒弟?是不是花瑞生的徒弟?就知道我改正得如何了。错给老合安万儿(错给江湖人找错了门户),我老云也咧瓢(liě piáo)(大笑)儿掉了海(hāi)柴(海为大,柴为牙)呀!





海(hǎi)青腿儿


江湖的艺人,金(算卦相面)、皮(卖药)、彩(戏法)、挂(武术)各行各业,都是有师傅有徒弟,在早年要有外行人挑出个剃头的挑子,没有师傅,不懂得扫苗擦尖(扫苗是剃头的,扫苗擦尖是剃头的对面遇见剃头的)的问答话,被同行的人盘起道(问对方行里的事和所学的功夫)来问短了,能把剃头挑子给留下。修脚的若是没有门户,不论是摆摊子,串街巷,被同行的人遇见了,盘起道来问短了,能把刀包子给留下。诸如此类,江湖人的门户是很有秩序的。早年吃生意的老合(走江湖的),没有师傅是吃不开的,有一种生意人,他做上买卖也会圆粘子(招徕观众)馈杵头儿(要钱)。若是盘道,讲究江湖的规矩,都不懂得,就是他没有门户,没拜过师傅。江湖人管这种人调(diào)侃儿说:“说他没有老帅(即是没有师傅),叫他海青腿儿。”据江湖中的老前辈说:“越是海青腿儿的人,越有能耐,人情世态、社会的阅历越深。”此话诚然不假。就以说评书的这行儿说吧。北平这个地方是他们的发源之地,论道中的规矩,较比外码头实在严得多。不论是谁,若想入这行儿,都得先找个人介绍,拜说书的为师,先下帖请人,在饭庄内定下酒席,磕头拜师传递门生帖。得将同行有门户的先生们请了来,先磕头、吃饭,大家也受了他的头啦,也吃了他的酒菜啦,同行的先进之人才承认这行里有他这么个人。然后学好了能耐,不论是上书馆献艺,或往市场搁明地(露天演出),拉场子说书,才没人拦挡。

在清末民初的时候有位松先生,长得人样很好,也有嗓子,唇齿伶俐,学问很好,他就没认师傅没拜门户,到馆子说书颇有叫座儿的魔力,一般听众无不赞成,他要是干长了这行,可坐头把交椅。不料同行的人说,他没有门户,没有师傅,警告开书馆的掌柜,如若用他,全体的人员都不进这书馆。“粘箔(nian bo)”(书馆掌柜)们不敢得罪大众,居然没人敢用。那位先生也有志气,弃了这行不干了,另谋他业啦。

在打破封建制度的时候,因为同行人不愿没门户的人侧身挤入,还把他排挤出去;若在封建制度的时候,不用排挤,去个同行的人,能够一瞪眼不叫吃这碗饭。若以这些推论,评书界就应当没有海(hǎi)青腿儿(没有拜师的艺人)吧?不料在光绪年间还真有一位海青腿儿。这说书的海青腿儿名叫范友德。有人说叫范有德的,那可错了。据我知道他是朋友的友字,不是有无的有字。说《西游记》的门户是永有道义四个字儿。说《西游记》的恒永通是永字辈的,庆有轩(即老云里飞)是有字辈的,如若范友德是这个有字,他就不算海青腿儿啦!那就算老云里飞的师兄弟了。范友德是朋友的友。因为什么评书界人能容范友德这个海青哪?说起来也有一种原因。范友德会说《安良传》,评书界的人曾携过他的家伙(说书的道具),叫他认了门拜了老师再干这个。范友德也愿意拜个师傅,只是评书界里没有人收他。不是他品行不好,是因为胡子都白啦,年岁太大了,收他为徒,那师傅得八十多岁,在那时候找不出八十多岁的老说书的。若有人收他做徒弟,晚辈人也有五十多岁的,平空跑出个年岁相仿的师叔谁也不干。后来评书界的人们因为他入门的事儿不大好办,大家商议好啦,不用叫他入门啦,算是海青腿儿吧!故此评书界里才有范友德这个海青。可是,江湖的老合(闯江湖的)许有海青腿儿,可不准海青腿儿收徒弟。他既没有师傅,又没有门户,传了徒弟算哪门的人哪?谁花钱请客拜师傅也是为有门户好吃得开,出来做艺没有拦挡。谁给海青腿儿磕头啊?惟有范友德这个海青腿儿,他就收了个徒弟,名叫陈纪义,并且评书界人还承认了。陈纪义算是评书界的人,范友德徒弟在海青腿儿里也是特殊的人物了。

如今破除了封建的制度,江湖乱道,艺人的规矩渐渐地都不重视。没有规矩,怎能有同行的义气?艺人也应重视规矩才好啊!我说的这话,不知江湖的先生们以为然否?





天桥的瞪眼玉子


评书场、大鼓书场、竹板书场,都是上有天棚,下有板凳,没有在平地上说的。在前几年,我逛天桥见有个说书的,衣服破烂不堪,他蹲在地上,左手拿着一把笤帚,右手用白沙子往地上写字。他就凭用手撒白沙子写几十个字圆粘(nián)子(招徕观众)。人围他站着,上无棚帐,下无桌凳,立着听他说书。他会说《捉拿康小八》、《康熙私访》、《乾隆下江南》、《张广太回家》。虽不说整本大套的书,能在这小段的玩艺儿里加上几句相声,也能叫听主儿咧瓢(liě piáo)儿一笑,说完了真有人给钱。只是他那嗓子和叫街的乞丐一样,有些人不爱听。他向来是蹲在地上,低着头连写带说,到了要钱的时候猛抬头,能把胆小的人给吓跑了。那脸上的颜色和地皮一样,只有那白眼珠是白的。他是方字旁的人(即是旗人),姓玉,因为他抬起头来使人害怕,江湖人都叫他瞪眼玉子。他的本领也还不弱,染有不良的嗜好,也和常傻子一样,在大前年的冬天,连瘾带饿冻死街上。

江湖艺人十有五六都有嗜好,被嗜好所累的实在不少,只是他们都不觉悟,全往那条路上去走。啃(kèn)海(hāi)草(抽大烟)的老合(江湖艺人),常傻子、瞪眼玉子就是你们的前车之鉴,若不猛醒,也难免追他二人再陷覆辙呀。我老云也是黑籍(抽鸦片)同胞,一跺脚又改白籍(烟卷)了(可不是又弄上高射炮)。望江湖的朋友快快脱离黑籍。





江湖艺人老云里飞


说评书的艺术分为两派,一为袍带,二为短打。《东西汉》、《明英烈》、《隋唐传》等书称为袍带,《济公传》、《施公案》、《包公案》等书称为短打。使钻天儿的(管说《西游记》的调[diào]侃儿叫钻天儿,系指孙猴儿而言)非评书界的活儿,另一派也。

说《西游记》的艺人最早是潘青山,他的徒弟叫安太和,学孙猴儿最好。听玩艺儿的人都不叫他安太和,管他叫猴安(有人说猴安叫安天会,实是妄谈)。至猴安时,说《西游记》的艺人始入评书界。

说《西游记》的艺人最早是潘青山,他的徒弟叫安太和,学孙猴最好。



评书界有各门之门长,如族长一样,凡他的门户中传流下来的人,都归门长一人管辖,门长受本门人之尊敬,比一姓之人尊重其族长有过之而无不及。猴安在评书界内为说西游之门长,其支派传流仅定为四个字儿,系永、有、道、义。永字辈的艺人如恒永通,有字辈的艺人如李有源、庆有轩(即老云里飞),道字辈的艺人如奎道顺、田道兴,义字辈的艺人如邢义如、石义舫。他们这门传流下来的人,以恒永通、奎道顺的艺术最佳,颇有叫座儿魔力。其余的俱皆平凡,皆未响名。如今,这些人俱皆故去,所存的人只有庆有轩、田道兴师徒而已。田道兴系瓦匠,虽拜庆有轩为师,也未久在各处献艺,“钻天儿”这碗饭他是吃不成的。老云里飞虽拜了恒永通为师,说的日子不多就改春口(相声)啦!说《西游记》的支派原定为永有道义四个字,不料传至四字上,该门艺人也至此终了。是有预兆呢?实不可料也。

庆有轩系方字旁人(北平人称八旗人为方字旁人,系指旗字之方字旁而言),自幼入松竹成科班学戏,曾冠其祖姓为白庆林。出科之后,因好听“钻天儿”,拜恒永通为师,按着评书界的支派赐名庆有轩。说了几年《西游记》,也未大转(挣大钱)。因家中人口众多,为解决生活,与他的长子白宝山(即今还在天桥献艺之小云里飞)和次子白宝亭(曾拜焦德海为师学习相声,台风、卖像、口白、夯[hāng]头,样样都好,惜其自误,将能挣钱便因嗜好丧命,实可叹也),父子三人在各庙会各市场以白土子写字,在地上写“平地茶园,特约超等名角云里飞、雨来散、风来乱父子三人唱《探亲家》、《三盗九龙杯》”。他们父子们每逢要唱哪出之前,先在地上写明,在写字的时候,粘(nián)子(观众)就圆上了,三个人随便柳(liǔ)着(唱着),临时现抓包袱儿(管当场抓哏叫现抓包袱儿)。在民初时,云里飞的父子班演唱的《戏迷传》,盛行一时,不过唱的是俗鄙无聊歌曲,难登大雅之堂。如今在天桥演唱《戏迷传》的是小云里飞,他们的杵门子最硬(即能往下要钱),一家数口,颇可温饱。

老云里飞在前几年独自一人往各处搁地说《西游记》,使吧嗒棍(管说零段书使人爱听,浅而易懂的段子调[diào]侃儿叫吧嗒棍),挑(tiǎo)罕子(即是卖那沉香佛手饼,江湖人管卖药糖调侃儿叫挑罕子),也很挣钱。近年来,小云里飞因他太岁海(hāi)了(管年岁高迈叫太岁海了),曾劝其父在家享福,不料,老云里飞子孙虽尽孝道,章年儿不正(管运气不好调侃儿叫章年儿不正),得了瘫痪病,行动甚难。他虽吃了一辈子生意,为人忠厚,说书的时候守本分,既不端锅(不要人家饭碗),又不撬杠(即不夺别人之地),是个忠样码子(即厚道人),为何如此?恐其故后,说《西游记》的就没有了。

老云里飞学戏是入的科班,学说《西游记》是拜过恒永通,改了半春半柳(半说半唱)的相声,乃算是海青腿儿(没拜师的艺人)。





江湖艺人大本玉子与连宝立、连宝志


在北平这个地方,说评书的艺人都说,清初时代在北京这个地方还有说评书的。弦子书最受欢迎,因为每遇帝王晏驾(驾崩)时,停止百日娱乐,不能说唱,无法维持生活,临时改说评书,以维百日收入。有些个唱大鼓的、说弦子书的,因为受国孝的影响,改说评书。评书是大鼓书、弦子书所改,也不虚也。

在西四牌楼、西单牌楼久唱弦子书的艺人,能在一个场子说几年书也不挪地方,万子(名儿)最长的就数着玉广昆了。他所说唱的几部书,既不是大鼓的道活(辈辈相传的),也不是评书的道活,是由书铺买部书来从头到尾看了一遍,上场就说,他的灵机好,记性好,改的词好,大受人们的欢迎。书铺里有的是书,说完了这部,再买那部,日久天长,叫听书的人都知道了,都不叫他玉广昆,改称大本玉子。他所唱的,实是大本的书,大本玉子名副其实也。

有一次我问说评书的艺人连阔如,玉广昆是不是他的师祖?据连说他是李傑恩的弟子,李系李致清之徒。其师祖李致清系北平人,久居三里河河泊厂,初学厨行,后入评书界。当其初次说书时,未认师傅,彼时江湖艺人若无门户,就有人阻拦,不能以艺挣钱。如若以艺挣钱,必有同行人携他们的家伙(说书的道具)。李曾受某艺人所携,为了此事,投在玉广昆门下,赐名李宝志。初次献艺,就在西单一带,有些人欢迎,算是出门红,所说的书是神册(chǎi)子(评书界的人管说《封神榜》的调[diào]侃儿叫神册子)。原有个老前辈叫王文和,是个六品领催(官衔),久说《封神榜》,颇有叫座儿的魔力。李宝志说了神册(chǎi)子,王文和大受影响,好听《封神榜》的人们都不听王文和,改听李宝志。玉广昆见徒弟挣了钱,百般勒索,挤得李宝志无法做艺,又惹不起这位师傅,就跳了门儿,另拜评书界名人程德印为师,改名李致清,与英致长、王致廉为师兄弟。又学会了说“串花”(管说《济公传》调[diào]侃儿叫串花),学济公时,姿态仿真,听书的人们都叫他“济公李”。

直到李致清大红大紫之后,有东城的连某喜爱评书,投在玉广昆门下为徒,艺名叫连宝志,专说《东汉》、《隋唐》、《五代残唐》、《飞龙传》,但未大红,仅能糊口而已。其弟也拜了玉广昆为师,艺名叫连宝立,也说那几部书。连宝志说了几年评书,艺术渐有进步,不料囊锥尚未脱颖,鼓了夯(hānɡ)儿(江湖人管嗓子坏了调侃儿叫鼓了夯儿),不能再说,回家养病,未愈而死。连宝立久在朝阳门外、花市、草市讲演评书,其兄故后几年的光景,他也去世了。玉广昆这支儿,到如今算断了门户了。





第七章 相声口技


团(tuǎn)门原是团(tuǎn)春


江湖艺人管说相声的行当调(diào)侃儿叫“团春”的,又叫“臭春”。一个人说的相声叫“单春”,两个人对逗叫做“双春”。用幔帐围着说相声,隔着幔帐听,看不见人,叫“暗春”。北平这个地方就是产生艺人的区域,就以相声这种艺术说吧,其发源就系由北平产生的。自明永乐皇帝迁都于此,至崇祯皇帝时,吴三桂请清兵,满人入主中华,康乾时代,歌曲畅兴,各贵族家中遇有喜庆之事,皆有请堂会,奏以各种富贵升平的歌曲。在斯时最盛行的为“八角鼓”了,相声这种艺术就是由八角鼓中产生的。按:八角鼓之源流系始于清朝中叶,乾隆时代有大小金川之战,帝命云贵总督阿桂兵伐金川。讵阿桂统兵前往,战斗日久,战绩毫无,因所率之军皆为满人,不习山战。后阿桂思一攻山之法,命兵士以草料和泥,用布为斗,将泥置于斗中抛到山岭之上,几经雨侵,泥中草滋生甚长,阿桂晓谕将士攻山之法,然后进兵攻山,鼓声击动,清兵攀起登山而上,踏破敌军之营寨,因之获胜。当于战息之时,阿桂见军中将士思归,想以安慰军心之法,乃以树叶为题,编就各种歌曲,教导军兵演唱,使其乐而忘返。所歌之曲儿,姑曰“岔曲”,以树本生岔而言,相传如此,也无可考。在早年所唱之岔曲,有“树叶黄”之旧曲调。即乾隆降旨召还帝都时,阿桂统兵回京,鞭敲金镫响,齐唱凯歌还。其凯旋之歌也岔曲也。兵至帝都,乾隆帝躬迎至卢沟桥畔。因用兵金川有功而为兴建碑亭,赐宴奖功。帝复闻兵在金川时曾以树叶编为歌曲之词,又经臣宰上奏,遴选八旗子弟,成立八角鼓儿。排演日久,甚见优美,满民争相演习,八角鼓儿普及于故都矣!当奏曲时所用之八角鼓,其八角即暗示八旗之意,其鼓旁所系双穗,分为两色,一为黄色,二为杏黄色,其意系左右两翼,至于鼓之三角,每角上镶嵌铜山,总揆其意即三八二十四旗也。惟八角鼓儿只是一面有皮,一面无皮并且无把,意指内、外蒙古,鼓无柄把,取意永罢干戈,八角鼓之意义不过如此。斯后曲词盛兴,有内务府旗人司徒靖辕者,别号随缘乐,寓居城内,因不堪繁华市井之嚣烦,乃往西山投一别墅而休养,感于身世,研究八角鼓曲词,编有杂牌子曲,是乃单弦渐兴也。八角鼓儿迭经变迁,又产生相声之艺术。

按:八角鼓儿之八部,分为乾、坎、艮、震、巽、离、坤、兑,由此八卦中分其歌曲之艺术为八样,即吹、打、弹、拉、说、学、逗、唱是也。八角鼓的班儿,向有生、旦、净、末、丑,其丑角每逢上场,皆以抓哏逗乐为主。在那时八角鼓之有名丑角儿为张三禄,其艺术之高超,胜人一筹者,仗以当场抓哏,见景生情,随机应变,不用死套话儿,演来颇受社会人士欢迎。后因其性怪僻,不易搭班,受人排挤,彼愤而撂地(在露天演出)。当其上明地(露天演出)时,以说、学、逗、唱四大技能作艺,游逛的人士皆愿听其玩艺儿。张三禄不愿说八角鼓儿,自称其艺为相声。相之一字是以艺人之相貌形容喜怒哀乐,使人观之而解颐;声之一字是以话的声音,变出痴苶(nié)呆傻,仿做聋瞎哑,学各省人说话不同之语音。盖相声之艺术,能圆得住粘儿(招徕观众),馈得下杵来(挣得下钱来),较比搭班作艺胜强得多。

张三禄乃相声发始创艺之一,其后相声之派别分为三大派,一为朱派,二为阿派,三为沈派。朱派系“穷不怕”,其名为朱少文,因其人品识高尚,同业人不肯呼其为少文,皆称为穷先生。彼自于场内用白沙土子写其名为“穷不怕”三字。他较比普通艺人知识最强,能够当场抓哏,俗不伤雅,故在生意人中可称为特殊的人物。其长处为身居知识阶级,腹有诗书,心思敏捷,能够随编随唱,心里出活最好,是不用死套子的玩艺儿,谐而不厌,雅而不俗,为妇孺所共赏。虽是个撂土地的生意,听他玩艺儿的人,也是有知识通文的。当其使活时,蹲于场内,地上放个小布口袋,内装白沙土子,他是左手打“义子”(说相声唱小段的时候,左手拿着两块小竹板儿,长约五寸,宽约三寸,嘴里唱着,手中用竹板啪啪啪打着板眼,江湖人管他使的那竹板调[diào]侃儿叫义子。在清朝时代,在沿街商店乞讨的花子使用此物,义子这东西乃穷家门[唱数来宝的]物也),右手用沙土子往地上画字,随画随唱。比如他画个容字吧,他嘴里必唱:“写上一撇不像个字。”地上就画一撇,接着又唱:“饶上一笔念个人,人字头上点两点念个火,火到临头灾必临,灾字底下添个口念个容。劝众位得容人处且容人。”他每唱一字必有一字的意义,按着字儿解释明白,最奇是写完了一个字,能把人逗得“咧了瓢”(管笑了调侃儿叫咧了瓢儿)。穷不怕惊人的意思是净“抖搂碎包袱”,用法子把人逗笑了;虽把人逗乐了,还不失那字原意。敝人在幼年曾见他写过对联一副,上联是“画上荷花和尚画”,下联是“书临汉字翰林书”。初瞧也甚平常,及至他说出这副对子意思,从顺着念,还能倒着从底下往上念,字音一样,颇有意思。在光绪年间“穷不怕”三字是无人不知的。

“穷不怕”本名叫朱少文,能在场内用白沙土子写其名。



团(tuǎn)春的这行里,虽称为朱、沈、阿三大派,但沈二的门户不旺,其支派下传流的门徒也是很少,并且没有怎么出奇的角儿。阿剌三的支派也是和沈派相同的。如今平津等地说相声的艺人,十有七八是朱派传流的。今将敝人所知朱派的艺人写出来报告于阅者。穷不怕的徒弟是徐永福,生意人都称他为徐三爷。徐永福的徒弟为李德祥(现在津埠)、李德钖(即万人迷)、玉德隆、马德禄、卢德俊(即卢伯三)、焦德海、周德山(即周蛤蟆)。现在北平献艺的只有焦德海、刘德志(卢德俊代师收刘德志为徒,故刘系卢德俊的师弟)。这些个德字的艺人以焦德海的徒弟最多,就以敝人知道的为张寿臣、于俊波、尹麻子、白宝亭(即小云里飞的兄弟,现已故去)、汤金城(即西单游艺场的汤瞎子)、朱阔泉、绪德贵(也同汤瞎子在一处作艺)。还有票友下海的高玉峰、谢瑞芝、华子元,均是万人迷收的徒弟。在东安市场说相声的有赵霭如(系唱“什不闲”[莲花落]的名角奎星垣的胞侄)、冯乐福(即小骆驼)、陈大头(系卢德俊的门徒)。在天津给张寿臣捧活的陶湘如,系玉德隆的门徒。

说相声最难的是“单春”,一个人的相声能把听主逗乐,实是不易。过去的穷不怕就以使单春成名。在说相声这行里使单春的,穷不怕可以算是他们的开山祖。阿剌三、沈二也能单双并行,但艺术之高超以穷不怕为最。晚近以来,说相声的艺人一跃千丈,能在杂耍(曲艺形式的综合叫法)馆子压大轴,可演末场玩艺儿的为万人迷一人。他可称得起是个完全的人才,从入了生意门就去正角儿(两个人的相声,一个逗笑,一个捧活,谁有能耐谁逗,逗的为主角,捧的为副手)。张麻子、周蛤蟆两个人的玩艺儿虽然不错,和万人迷联了好多年的穴(xué)儿(管搭伙计调[diào]侃儿叫联穴),总是给万人迷捧活,永远都没去了正角儿。万人迷能够在馆子说两三个月的单春不掉座儿,活头儿(会的东西)最宽,两三个月才翻一回头,除他之外都是半个月里就翻一回的。万人迷最惊人的是向不咧瓢(liě piáo)儿(说相声的逗笑,把听主逗笑是为挣钱,如若自己也笑了,同行人就耻笑他艺术不精,自己咧了瓢儿)。今日之艺人,无不失其规矩,人笑也笑。在电影片中之陆克、贾波林(即卓别林)之成大名,也是把观众逗得笑了,他本人是始终不笑的,那个面孔就是他成名的特长。万人迷自从作艺以来,无论在场上使什么活儿,抖搂出去包袱儿都是响的,向来没有抖搂闷了(说完了笑话,该着使人发笑,听的主儿没被他逗乐了,调[diào]侃儿是包袱儿抖搂闷了。抖搂闷了活儿较比笑场格外得丢人。如有其事,同业人皆轻视他艺术不精)的时候。万人迷虽然故去了,津埠曲艺界的人士无不思念的。在万人迷大红特红的时候,他能在场上一言不发,用他那有哏的脸孔使人发笑,在同行里都称身上有活,最能拢神。彼一登台,全园观众之目力皆注射其身,为同行人所不及也。万人迷之相声灌了不少话匣子片子,计有《跑梁子》、《菜单子》、《怯封钱粮》、《八扇屏》、《挑(tiǎo)春》等等的段儿。其中最好的是《挑春》(即《卖对子》),其对联之精妙,皆为彼个人心中所发,如:“北燕南飞双翅东西分上下,前车后辙两轮左右走高低。”“南大人向北征东灭西退,春掌柜卖夏布秋收冬藏。”“道旁麻叶伸绿手,要甚要甚;池内莲花攒粉拳,打谁打谁。”这些对联都很绝妙。万上台之拿手的能为是以镇静态度,使听玩艺儿的人们听着也同其镇静。其票友下海者,每逢上场大呼怪嚷,使人见了他那穷凶极恶的态度,有如汤沸,不能拢神压场,实为缺点。万人迷红了三十余年,以在平日少,在津最久。曾往上海献艺,他在场上使活,段段的包袱儿皆闷,南方人听了不笑,以至狼狈而归。万在南方失败以后,沪上评曲家深致不满,对于滑稽大王之头衔大肆攻击,然万再不返沪,攻击也无损于他,毫无可惧也。

在江南沪、杭等地说相声的艺人,只有“吉三天”。吉之艺名为评三,称其为三天,系其在平时曾说评书,虽然叫座,只能说三天,到了第四天其技已穷,另换新地献艺,时人讥诮不呼其名,皆叫他吉三天。吉系相声艺人冯六之徒。冯六为春口(相声)里沈二支派中的人物,冯在清末时代拜认评书门户,艺名冯昆治,与评书界中玉昆岚、德昆平、福昆铃为本门昆字师兄弟。吉评三拜冯六为师,一门两吃,又能使春(说相声),又能团(tuǎn)柴(说书)。他说相声以“贯口活”(以带有连贯性的韵白为主要特征的段子)最拿手。彼于民国五年间离平南往,他一人懂上海、宁波、江苏等地土话,在江南大红特红,惜其染有嗜好,至今北返于津,昼夜奔忙,依然两袖清清也。万人迷南下失败,吉评三南往成名,非江湖人厚于吉薄于万,乃万不通南方语言之故也。生意人常说:“南京到北京,人生话不生。”艺人以到的地方最多者称为腿长,吉评三在生意行里也算是腿长的江湖艺人哩!

说相声的艺人能成大名,单春、双春不挡的(单口、对口都能说),迄至今日只有张寿臣一人,自万人迷故去之后,以他为说相声第一流人物了。





天桥的相声场和杵门子(到要钱的时候叫杵门子)


天桥的杂技场有相声场、摔跤场、把式场、戏法场、杠子场、大鼓书场、竹板书场、评书场、戏场、河南坠子场、空竹场、卖药场、卖糖场、高跷场、中幡场、砸石场、双石头场、电影场。这些场子,都不是华丽壮观有屋子的场子。冬天是一块平地,摆些桌椅,露天地儿;夏天才有席布棚帐,可称得起是平民化。

相声场在爽心园前边,这个场子最早是张寿臣、刘德志、尹麻子、郭起如(一为启儒)、于俊波几个人。自从滑稽大王万人迷死在了奉天之后,说相声的第一路人才缺乏,张寿臣够头路角色,被天津杂耍(曲艺形式的综合叫法)馆邀了去,充各馆子的台柱。张到津埠大红特红,颇受各界人士的欢迎,不惟不能返平,也不能再撂明地(在露天演出)了。张去后只有刘德志、于俊波每日上地(做生意),刘德志与焦德海为正副手,每天夜内在青云阁、玉壶春上馆子,有时还在各公馆做堂会,去广播电台给各商家作营业的广告宣传员,刘德志的相声也是不到天挢了。即或有到天桥的时候,也是恰巧馆子停业、没有堂会的日子,恐也不能常见。天天准在那场子献艺的,还是尹麻子、于俊波、郭起如等靠长儿(在固定的演出场地不动)。在民国十年至十六年之间,他们这相声场,每逢到了“杵门子”的时候总有边粘(nián)子(江湖人管说完一段相声要钱了调[diào]侃儿叫杵门子。要钱的时候,场子外边站立的人不走,还要等着再听下去,调[diào]侃儿叫边粘子不动),那几年社会里还不像如今这么穷,听相声的人们也不像如今这么穷,他们虽然不进场子里坐着听,站着听也是照样儿“掉杵”(给他们往场内扔钱,调侃儿叫掉杵,又叫抛杵)。每逢他们说完了一段相声,先是由坐着的听主往场内扔钱,他们说那是“头道杵”;将钱都拾起来,数数是多少钱,再凑个整数儿,然后还要钱,他们说叫“二道杵”;如若再向围着场子立着的人要钱,叫做“托边杵”。再不能要钱了,才重新另说相声、抓哏逗哏,哄人大笑。他们要钱的情形就是这样。

在近两年大不如从前,每逢说相声的时候,凳上坐着的人坐着听,围着场边站着的人站着听,及至说完要钱哪,立着的人呼啦一散,各奔东西。坐着的人往场内扔完了钱就走,绝不接着再听下回。他们钱也要完了,人也都走没了。说他们的行话,管这种情形调侃儿说“起棚儿”。“每逢到了杵门子就起棚儿,这个年月怎么好啊!”早年一天他们这场玩艺儿若挣六七元钱,每人能分一元多至两元;现在他们这场玩艺儿才挣两三元钱,一个人才分几角钱,时常不够块儿。别看他们买卖不如从前,还算是天桥儿最挣钱的玩艺儿场哪!别处也有相声场子,说相声的人也不齐全,玩艺儿也少,活头儿也窄(会的活也少),挣钱也是有限,都是上个三天五天就散,从未见别处能有立长了的相声场子。凡是好听相声的人,到了天桥都奔爽心园前头去听他们的相声。这个场子在那里有十几年的历史,是个久长的玩艺儿场儿。





江湖艺人万人迷


戏台上的丑角儿是将听戏的逗乐了,他自己不乐为是。电影上的陆克、贾波林(即卓别林)的笑片,叫人看着能笑得前仰后合的,那陆克、贾波林总是板着面孔,毫无笑容,那才是他的艺术高超哪!说相声的艺人按着规矩也是应当将听主逗乐了,他们不能笑的。如若听主也笑,他们也笑,那就算坏了规矩,说行话叫“笑场”。说相声的艺人不笑场的就是万人迷。

万人迷姓李,名叫德钖,按说相声的支派,是德字辈的。焦德海、刘德志就是他同辈的师兄弟。他父亲叫老万人迷。提起万人迷三个字来,平、津一带几乎妇孺皆知,其魔力之大更可想见。相声有双春,是两个人说,一个正角儿逗哏,一个配角儿捧活儿,使出活儿来容易将人逗笑了。“单春”难说,一个人的相声要把人逗乐了,实在是不容易了。说单春成名的有已故的万人迷,现在的是张寿臣。

万人迷系北平人,自幼就学相声,他总算是门里出身,凡是好听相声的人,都知道他口才最好,能言善辩。江湖人都说他夯(hānɡ)头正(嗓子好),喷口好(字音真),使上活儿发托卖像(指演员在表演时要惟妙惟肖,通过喜怒哀乐刻画艺术形象)最能拢神。他是个单双口的相声,明春(明场说相声)、暗春(隔着幔帐说相声,看不见人叫暗春)都成的,不惟会的段子多,并且他能攥弄(zuàn nong)活儿(管自己会编相声调[diào]侃儿叫攥弄活儿),能够俗套子不说,临时现来,当场抓哏。单春(单口相声)的活儿是荤的多,素的少,万人迷能以素包袱儿叫响儿。盖素包袱儿的段子都不大火炽,说相声的艺人都愿意说荤的,谁也不愿说素的。他们说相声的艺人如若说了一段没将听主逗乐了,行话叫使“闷子活儿”啦!同行人知道了,皆耻之。故此素包袱儿是不轻动的。万人迷专以素包袱儿叫座儿,妇女可听,雅俗共赏。在他未成名之先,与张麻子在平、津等地也上场子,搁明地(在露天演出),自入民国以来,他响了万儿(成了名)啦才进馆子。那些年是使双春(对口相声),他逗哏,张麻子捧活儿,人都以为张不如他,其实张麻子捧活儿最严,素为同业人钦佩,实在不弱于万也。在张麻子故去之后,马德禄给他捧过活儿,周蛤蟆给他捧过活儿,皆不如张麻子捧得好,故万人迷时常表演单春。在他“火穴(xué)大转(zhuàn)”(大红大紫)的时候,他只要人一上台往椅子上一坐,板起面孔,冲大伙愣着,全场的听主就能够都笑了。这点特殊的技能是人难会的。

他自早年就啃(kèn)海(hāi)草儿(管抽大烟调侃儿叫啃海草儿),染成不良的嗜好,时常的“朝翅子”(打官司调侃儿叫朝翅子),皆赖有口才能将翅子逗得咧了瓢儿(能把官长逗笑了),释放出来。万又嗜赌如命,在民国八九年间,天津某馆主人交给他千元大洋往北平邀角儿,时至除夕,腊月三十的白天,千元尽皆输去。归寓见有人顶牛儿,每次以二毛钱为数,他又顶了一宿牛儿。天津开馆子的都说他好銮把(bǎ)(管赌钱叫銮把),此话诚然不虚。在某将军得意之时,每至津门,必招万做长夜之谈,颇为喜爱。一日某将军在某小班推牌九,连连败北,忽见万入,命他看牌,两张牛牌到手,万视之,一张大天,一张大四。凭此天杠吃了个通儿,百元的筹码十根数儿,尽赐予万人迷。万在某“库果窑”认识某“库果”(管娼窑调[diào]侃儿叫库果窑,管妓女叫库果),得此巨资,接某妓从良,深感某将军之德,至死不忘。未过二年,某巨显做寿,邀其出关,不料滑稽大王竟瘾死在途中。当局恐有别情,已然验尸。万之生前快乐有余,何其死后之不幸若此,良可叹也!

万人迷土点(死了)之后,继其头把交椅为焦德海之大弟子张寿臣,至今在津献艺,颇受该地人士欢迎。盖张也给万捧过活儿,颇得其妙,故能承其衣钵而享大名。江湖人常云“艺不错转(zhuàn)”(江湖人管艺人有特别的本领调侃儿叫艺不错转),张寿臣也有惊人的能耐呀!





三不管的相声场儿


说相声的艺人在天津红的年数最多要数万人迷了。当三不管发达的时候,万已成名,每日在燕乐升平压大轴儿,大红特红了,焉能到三不管去上地(说相声)?可是我老云久游三不管,有好几次见万人迷在那里搁地。据我调查,他为什么在那里搁地?江湖人因为他的艺术高超,尊他为相家,或称为老相法,在社会人不以为然,江湖人则以此称呼为至尊至荣。有说,相家都有一控(江湖人管为人若有钱好养鸟、抽大烟、嫖娼、赌钱等等的嗜好调侃儿叫控门。为人只要好一样,江湖人就讥诮谁有一控),万人迷“控銮”、“控海(hāi)”(管好赌钱调侃儿叫控銮,管好抽鸦片调侃儿叫控海),上馆子挣包银,几百元一次到手,肘海(hāi)草儿(江湖人管买鸦片烟调侃儿叫肘海草儿),銮把(bǎ)儿,几天就花个干净。他要念了杵(江湖人管没钱了调侃儿叫念了杵),就找人展杵头儿(江湖人管拉亏空、借债、使利钱调[diào]侃儿叫展杵头儿),他是周赧王的徒弟,永远债台高垒。到了债主逼得紧啦,他就跑到三不管去搁明地(露天演出),凡是好听玩艺儿的人,都很捧他,有个几十元的亏空,三两天就能补上。万人迷控銮(赌钱)、控海(hāi)(抽鸦片),是造成三不管的游人听他玩艺儿的机会。我也听过多少次,还是在三不管说的相声比在馆子还好。后来长腿将军喜爱他了,就不到那里去啦。

焦少海虽是门里出身,他的联络不好,北平的相声场子都不能做艺。说相声的艺人老不能留胡须,少不能留分头,焦德海活到六十多岁就没留胡须。我问过他,那么大年岁为什么不留须?据他说,自己干的这行当要留了胡子不能胡说。做艺的因为“有栅栏”(江湖人管留胡须调侃儿叫栅栏)碍口,所以不留。说相声的人不能往美式上修饰,因为他们的嘴最损。别人不好,他们抓哏,他们若好修饰,也是样样碍口。焦少海就留分头,擦生发油,同行人见他修饰头脸,都不愿意和他“联穴(xué)”(江湖人管合伙、搭班调侃儿叫联穴)。东安市场赵霭如、冯乐福的场子,西单汤瞎子、小高二的场子,天桥郭起如、于俊波的场子,他都不能上,只好开外穴(到外地挣钱)到天津去做艺,在三不管上权仙的南边找了个场子说他的相声。他惯使双春(对口相声),不惯于单春(单口相声),没有伙伴做不了生意,有“挑(tiǎo)厨供(gòng)”(江湖人管卖戏法的调侃儿叫挑厨供)的赵希贤,叫他儿子拜少海为师学说相声,少海给他徒弟起个艺名叫小龄童。师徒每天上场子,小龄童逗口,焦少海捧活,很为火炽,算是一档子玩艺儿。直到如今,小龄童已然出师,因为他有天赋的聪明,口齿伶俐,发托卖像(指演员在表演时要惟妙惟肖,通过喜怒哀乐刻画艺术形象)都能传神,抖出去的包袱儿响的多,不闷活,很受津埠人士欢迎。杂耍(曲艺形式的综合叫法)馆子邀了他去,也能上倒(dào)第三的场子。真应了那句话了,“有状元徒弟没有状元师傅”,小龄童响了万儿(有了名儿),成了名角儿,越过其师。江湖人说“艺不错转(zhuàn)”(江湖人管艺人有特别的本领调侃儿叫艺不错转),他一定有惊人的好处。在老焦去世以后,我老云去往他家行人情,焦少海对我说,小龄童每日上馆子以及广播电台上说相声,有十数元收入,对于他很为尽孝,收这个徒弟,总算有良心,不忘本。饮水思源,焦少海在前几年曾拜文福先为师,学说评书。可是文福先说《施公案》,他不学《施公案》另学《永庆升平》。可惜他下米就要吃饭,在北平上了几个茶馆,起初还有人听,到了后来简直就没人听了。说相声他是幼年坐科,说评书他没用过功夫,艺术原就平常,那《永庆升平》在清末的时候有人欢迎,到了如今书运已然过去,说得多好也没有人听了,何况再说不好呢。他团(tuǎn)柴(说书)不成又归了本行,仍往天津三不管上地说他的相声。在前几个月,焦德海染病,因有不良的嗜好,挣多少花多少,一点积蓄皆无,没钱医治病症。观音寺玉壶春的三胎亥在天桥相声场遇见我老云,他正为焦德海奔走。凡是听过老焦玩艺儿的人都有捐款,各名伶也都有帮助。三胎亥求我代为登报宣传,以为多收些钱,好办理善后。我对于他为艺人热心很是钦佩,不过我老云不肯在报纸上挂招牌,免得有人讥我受××××。不料事情未过三天,老焦与世长辞。享名数十年的相声家焦德海,身后萧条,无有办法。幸而北平有张德山、刘德志、于俊波、尹麻子,天津有张寿臣尽力维持,没有什么困难。当我到焦家行人情时,见了焦少海,因喜爱他的脾气好,略进忠言,劝他立志向上,不然老焦一死,全家数口赖彼为生,就无法维持了。他葬老人事毕,仍返津献艺。

三不管的相声,焦少海倒是能立脚步,不过难享大名吧。最近我在北平常听见天津广播电台播来的各种杂技,最可听的玩艺儿是常连安、小蘑菇的相声,一捧一逗,对口相声,又火炽又严,甚为精彩。包袱抖得真响,他二人的艺术受人欢迎了。在民国十四五年的时候,小蘑菇还在三不管上地。说起他父子的历史来也有意思。常连安系北平人,弟兄一人,侍母最孝,曾入富连成科班学习老生。常连安的连字还是富连成的哪。他出科之后,因为“鼓了夯(hānɡ)儿”(嗓子坏了),戏饭不能吃,改学“彩立子(lì zi)”(江湖人管变戏法的行当调[diào]侃儿叫彩立子),拜某幻术家为师。初入江湖,在张家口献艺,挣钱不少,颇可养家,反又往天津、大连、烟台、营口等地做艺。生齿日繁,人口多,行动不便,在天津三不管上明地(露天演出)变戏法。常连安的全家都能上地,个个会变。在玉林春的东边赁了个场子,每天的粘(nián)子总是不酥(江湖人管场的四面观众调[diào]侃儿叫粘子。如若围着的人不走,调侃儿叫粘子不酥)。小蘑菇是常之长子,五六岁就能上地,会使“苗子”,会使“小抹(mǒ)子活儿”(管变仙人摘豆叫苗子,管各种小茶碗变的戏法叫小抹子活儿)。他父亲夹磨(jiá mo)(传授真本事)的,随使活,随抓哏,能把观众逗笑。几岁的幼童,若非天赋的聪明,恐难办到。每逢使活的时候,有他舅舅给垫场子。到了“杵门”(江湖人管变完了戏法,向众人要钱叫杵门)的时候,观众都给了钱不走。小蘑菇还能“托边杵”(指向围着的人去要钱调侃儿叫托边杵),如若他冲某人说:“这位给一个吧。”那人要说:“我没带着。”他必说:“没带着那么大的肚子。”(妇人受孕都是大肚子,俗说带肚子,他指肚子抓哏)那人不能恼,觉着小孩伶俐可爱,伸手还多掏给他钱。他连要钱带逗笑,哪天也挣个几块钱。他全家的生活仗他能够维持。

天津三不管的相声,最可听的是常连安、小蘑菇的相声,一捧一逗,又火炽又严,甚为精彩。



可是变戏法的行当,以能逗笑能挣钱;江湖人说万象归春,不论哪行生意,也是以能逗笑为美。电影笑法为上,滑稽玩艺儿无不欢迎。常连安见其子可以夹磨(jiá mo)(传授真本事),就一段一段地教他说相声。小蘑菇相声化的戏法,在三不管火穴(xué)大转(zhuàn)(在一地方演出挣了大钱了)。说《精忠》的陈荣启,与常连安系盟兄弟,代为介绍叫小蘑菇拜了相声名家张寿臣为师,正式学相声。小蘑菇的台风、发托卖像(指演员在表演时要惟妙惟肖,通过喜怒哀乐刻画艺术形象)全都不错,经其师夹磨数载,艺术进化得堪称绝艺。天津的各杂耍(曲艺形式的综合叫法)场子、各电台争相延聘。他逗常捧,父子二人生活快乐,衣食丰足。张寿臣夹磨之力也。

三不管虽然平常,他们能够发达成名,一半是仗自己聪明,一半是介绍人陈荣启有眼光,才造就成了小蘑菇的艺术。常连安的次子叫二蘑菇,与侯彝臣一处做艺,他使对口活,和白银耳分为上下手。他们爷儿三个要说《训徒》的段子,甚为可观。有人说侯彝臣叫猴头,再搭上二蘑菇、白银耳,很有意思,都是干果子铺的货。日后侯彝臣再教徒弟,可以叫燕窝、鱼翅了。





天桥的臭春场子


在前几年,我老云逛天桥常见有个六十多岁的老人,长得细条身材,满脸的皱纹,嘴里的牙掉得剩了一半,说话是京东的口音,在天桥上地(做生意)。他那场内有个九根细竹竿的小蓝布帐子,桌上放着大小竹管笛儿,到了时候,他能吹各样小曲,圆上粘(nián)子(聚好了观众)使“臭春”。一般人都叫他管儿张。

他使臭春之法,将竹竿帐子在场儿当中立起,他钻到内里使活儿。场子围着的人们隔蓝布帐往帐里头听。他在帐内一个人能学两个人说话,变出来的嗓音叫人听着还真像一男一女。

不过,他学的是大奶奶住在娘家,大爷拉着驴去接大奶奶,走在高粱地,大爷要钻进高粱地里拔高粱,使人听了虽然可笑,也觉有兴趣。临完了,他还学一回驴叫,抖起铜铃铛,哗啷啷地响起来,真像驴叫,叫完了钻出帐外要钱。听说他在二十年前,学完了大爷大奶奶闹高粱地还有人给钱;这些年可不成了,他在帐内的时候还有人围着,等到学完驴叫钻出帐来再要钱哪,场子就光了,也挣不了几个铜子。

据江湖人说,管儿张的玩艺儿调(diào)侃儿叫“臭春”。在庚子年前,做那种生意的倒有几档子;自从庚子年后,做这种生意太缺德,各市场全都取缔。这种玩艺儿到了管儿张的晚年也就淘汰尽了。这几个月,我老云到天津、北平、张家口各处去了,始终没看见管儿张,向江湖人打听他的动静,有几位说大概是“土了点”(死)啦!双春(对口相声)是大兴其道,臭春是断了攥(绝)啦!





江湖艺人汤瞎子、田瘸子


我中国的礼教,到如今有新旧之分。这两种人的见解不同,至于新礼教好,旧礼教好,社会的人士自有真正的认识,公平的评论,不用我老云饶舌。可是江湖中人的一切的知识,处世待人,交际往来,也随着社会的潮流变化。在早年,江湖人都讲究义气,如若大家顶神凑子(江湖人管赶庙会调侃儿叫顶神凑子),倘若庙场内地方窄狭,去的各种的生意多,拉不开那些场子,容不下那些个生意,有地方拉场子、摆摊子都能挣钱吃饭,那没地方撂生意的,远路风尘白来了,赔了路费不挣钱,如何能成?江湖人不是资本家,十有八九都是平地抠饼(没有本儿要凭真本事挣出钱来),谁也没有钱赔垫。江湖人遇见了这种情形都有办法,卖药的与卖药的联穴(xué),相面的与相面的联穴,说书的与说书的联穴,一个场子能搁两档子生意,一个地方能有两个人做买卖。什么叫联穴哪?他们江湖人管合伙做生意、搭班合帮上地(做生意)、大家组班等事,调侃儿都叫联穴。如若地方宽敞的,一个说书的占一个场子,本领好的多挣钱,本领不好的少挣钱。惟有地方窄小,临时联穴,两个说书的上一个场子,虽分前后说书挣钱,可不论谁多挣谁少挣,谁有能耐,挣了钱放在一处,到了晚上按股均分。又公平又有义气,那才是江湖人的美德,值得人佩服。江湖人合作的精神,是最有义气的。譬如江湖人遇见这地方窄小,容不了许多的生意,他们还有不愿意联穴愿意往别处去,不愿大家挤着的,可是不走的人都给走的人凑路费,那种义气也是难得。在早年还有某江湖人病在店内,将东西当卖一空,病好了,没有法子做生意,往各处告帮,只要和江湖人见了面,把自己是干吗的,调(diào)侃儿说上来,就能多多少少地得到帮助些钱;还有尽量帮助,倾囊而赠的。现在社会上的人心险恶,虚伪诡诈,打破了礼教,不顾信义,不讲道德。江湖中人对于同道也是这样了,讲义气的甚少。江湖乱道,此其实也。

在前几年,天桥的杂技场很是发达,不论什么玩艺儿都能挣饯。相声场子,暗春(隔着幔帐说相声,看不见人)、单春(单口相声)、双春(对口相声)很有几档子。张寿臣、刘德志、尹麻子、白宝亭在一个场子做生意,数着他们那场玩艺儿火炽。再次的还有高二父子。田瘸子、汤瞎子两个人不与别人联穴,占个场子做生意。可是张寿臣、刘德志、尹麻子、冯乐福、赵霭如、于俊波、郭起如、焦少海这些人说相声,使的那玩艺儿如同科班角色的戏词一样,哪出也有准词,他们不论是谁,都能临时合演,说的哪段相声也不能砸锅。惟有田瘸子、汤瞎子说的相声,与他们这些的玩艺儿全不一样,大概是无师自通,自己研究的,或是拆改人家的活儿。尤其是汤瞎子,能够坐在场内学飞禽走兽叫唤,学磨剪子磨刀的吹喇叭,消防队的警笛,斗蛐蛐,样样仿真,不过没有真的声音大就是了。他最惊人的是学蚊子叫唤,声小可听。在早年没有说相声的,有一种能以口技挣钱的玩艺儿,或隔房间,或用帐子遮避,学学飞禽走兽、各样的草虫叫唤,江湖人调侃儿叫做“暗春”。

清末的时候,张三禄使“暗春”最拿手,可称“暗春”泰斗。百鸟张、百鸟王也兴旺些年。不过他们不按着“暗春”的规矩做生意,形如乞丐要钱,虽挣得不少,也自低身价。管儿张倒是在帐子里使活,可惜他学的是老两口子闹房,瞎子闹高粱地,淫声浪语,有伤风化。他是暗春中的臭春,净使臭包袱儿,文明的人都不肯听。别看不好,他死了还断了庄,没地方找那玩艺儿哪。

汤瞎子的口技颇有精彩,惜其不多,一场儿了事,若再进步研究,能有几天的玩艺儿,灌话匣子片、播广播电机、上馆子登台、做堂会,也就成了大名。他与田瘸子搭了几年伙,平平常常,仅顾衣食而已。自西单商场开办,他们赁了个场子做生意,因为那里的游人都是火码子(江湖人管有钱的阔人调[diào]侃儿叫火码子),挣钱容易,他们两个人可就火穴(xué)大转(zhuàn)(挣了大钱了)。汤瞎子受过折磨,为人勤俭,绝不妄为,也无嗜好,安分守己。田瘸子刚得了地,能多挣钱,就忘了以前的苦处,成天去逛“库果窑儿”(江湖人管娼窑调侃儿叫库果窑儿,管妓女调侃儿叫库果)。日子多了,患了花柳病,药不离身,体弱身虚,又“咯(kǎ)了光子”(江湖人管吐血的病叫咯光子)。汤瞎子很有义气,煎汤熬药,尽心地服侍。他病见了轻,仍去宿娼,后来又“扯了风子”(江湖人管梦遗滑精的病调侃儿叫扯风子),两头忙可治不好。他那“粘啃(nián kèn)抹不作”,年数有余,就“土了点”啦(江湖人管病调侃儿叫粘啃,管治不好调侃儿叫抹不作,管死了调侃儿叫土了点啦)。汤瞎子总办丧仪,把他送入土内,真成了土里的点儿。他死后抛下老戗(qiāng)儿(江湖人管父亲调侃儿叫老戗儿),无人奉养,汤瞎子念田瘸子与他搭伙的义气,每日给田瘸子的父亲送些钱去,维持生活。这些事北平的老合(江湖人)全都知道。

在这江湖乱道的时候,江湖人都不守规矩,做生意还能讲义气吗?像汤金城(汤瞎子)这样人实在少有。以我的眼光看,能遇见这样有义气的人就不错了:能厚待于他,可不是煎汤熬药送他的终,是待他好就得了。在早年江湖艺人做生意有义气,讲究老不挨,少不欺,如若挨着老年人上地(做生意),老年人没力气,受影响,少挣钱,那就算欺老;少年人刚学到些能耐,还没有火候,久惯做艺的人再挨上地,还不受影响吗?有不肯欺老欺少的,都躲着老少人做艺,那是江湖人的义气。如今可不那样了,挨着老弱残兵,他们好逞强。我说这话阅者不信,到了各市场、各庙会一看就知道。





故都之八大怪


有一天我老云走到琉璃厂某书铺,买了一本书。据那书上所载,天桥的怪人有韩麻子、田瘸子、穷不怕等。我老云自幼就到北平,虽然常出外去游各省,可是年年到这里,几十年也不断去逛天桥,就是没见过这几个怪人。我向北平的老江湖人打听这些人怎么叫八大怪?是否在天桥做过艺?据老江湖人说,入民国以来,时代改变,汉满蒙二十四旗人,没了铁杆庄稼,丢了老米树(在清朝,生一个孩子就领一份米,等于有了铁饭碗),方字旁的(旗人)落了价。城里头除了隆福寺、护国寺还有各种杂技场有人游逛,其余的地方就都灯消火灭了,天桥才日见兴旺,也是香厂新世界、城南游艺园陪衬着兴旺起来的。

在庚子年前,北平没修新式马路,土甬路两旁都是生意场。凡平市四五十岁的人都见过那些杂技场。穷不怕、醋溺膏、韩麻子、盆秃子、田瘸子、丑孙子、鼻嗡子、常傻子八个人都是甬路两旁撂地的江湖玩艺儿,个个形状怪异,平市人又敬他们又讥讽他们,起名叫“八大怪”。这八个人,除常傻子弟兄活得长久,民国十五年前,在天桥挑(tiǎo)过将(jiàng)汉儿(江湖人管卖壮药的调[diào]侃儿叫挑将汉儿的),其余的怪人早已去世,并不是在天桥久占。韩麻子是说相声的,他嘴没德行,刻薄已极,到了要钱的时候,刮钢(说脏话挖苦人)绕脖子净骂人;盆秃子是半春的生意,他敲打瓦盆唱各种小曲,随唱随抓哏,抖搂臭包袱儿,引人发笑,到了时候要钱;田瘸子是残废人,专以盘杠子(练单木杠)的技艺挣钱,他较比不残废的人功夫还好,也能在练玩艺儿的时候抓哏、抖包袱儿,归杵门子(到要钱的时候叫杵门子)向观众要钱;丑孙子是在场子说相声,摔丧碟子哭他爸爸,向观众假以凑钱发丧事归杵门子;鼻嗡子是身上带洋铁壶,竹管一根插入鼻孔内,顺竹管出音,敲打洋铁壶唱曲要钱;醋溺膏是专唱小曲,柳里加春(江湖人管唱曲的带说相声调侃儿叫柳里加春),向人要钱;至于穷不怕、常傻子,我老云已然说过,老江湖人说我说得很对。至于有人将八大怪都说在天桥那儿,简直是醉鬼上天——糊(胡)云了。还有人以大兵黄、大金牙、云里飞称为八大怪。你要问他们八个怪人都是谁,可又说不出八个人来,此等拾人余唾的事儿实是可笑了。





天桥的大兵黄


我老云前几天到天桥巡礼,巡到公平市场南,见有百数人围了个大圆圈儿,里边有个人直嚷,嗓音洪亮。他随说随嚷,围着的人们也都随着他笑。我老云不知道是什么生意,挤进人群里一看,见场内站着一个人,身体魁梧,大脑袋,胡须、眉毛俱都苍白了,大眼睛,高颧骨,大鼻子,大耳朵,大嘴。这人面上净是皱纹,看他的年纪足有七十多岁的样子。头戴缎子小帽,迎门嵌块宝石,蓝缎子夹袍,又肥又大,黄缎子夹坎肩,身旁挎着个大布袋,手里拿着根棍,又说又骂,围着的人们听他骂得慷慨淋漓了,痛快得笑起来没完。我平心静气听他个水落石出,倒要瞧瞧他到底是干吗的!及至听了一个多钟头我才听明白了他是干吗的。原来,他就是专以说笑话“圆粘(nián)子”(招徕观众)的卖药糖的大兵黄。

我向江湖的人们探讨,他是哪门的玩艺儿?据老江湖人说:他是当兵的,退伍之后,不愿当差,卖糖糊口。对于江湖的事,他全都懂得。他有个胞兄叫大黄,专打走马穴(xué,穴是指演出地点;走一处,不能长占,总是换地方挣钱,江湖人叫走马穴),往各处去“顶神凑子”(赶大庙会),柳海(hāi)轰儿(唱大鼓的),长得身材高大,人式“压点(yā diǎn)”(震得住人为压点),专唱《黄杨传》,以黄三太镖打猛虎,指镖借银,杨香武盗九龙杯等等的段子挣钱。没有整本大套的万子活(管说长篇书目叫万子活),凭几段小吧嗒棍儿就能成名。每逢唱时,抓哏取笑,能使人捧腹笑倒,抖搂包袱儿是他拿手的玩艺儿。大兵黄是以海(hāi)冷打万儿(管当大兵的调[diào]侃儿叫海冷,管以当过大兵为名调侃儿叫以海冷打万儿),他说的笑话是随宋庆打过旅顺,随张勋打过白朗,随张岳挖过河工。不知道的人,都说他能骂人,其实他是借着钻钢儿(根据社会现状)抓哏、抖搂包袱儿,他能迎合社会人士心理,随时代的变迁团(tuǎn)(说)钻钢儿。一些个心直口快的人们,成天价到天桥围着他听笑话,觉着他那些话像《水浒》的李逵,快人快语,给人打不平,发牢骚,比吃服开胸顺气丸还痛快。他的笑话虽然不少,使人听了不厌是他的抓哏逗笑一天一换样,改良的单春(单口相声),哪能不受欢迎。

大兵黄身体魁梧,江湖人说他压点;嗓音洪亮,江湖人说他夯(hānɡ)头子真正;有多少人也能叫人听清了他说的是什么,江湖人说他有喷口;面上能够形容滑稽态度,江湖人说他有发托卖像(指演员在表演时要惟妙惟肖,通过喜怒哀乐刻画艺术形象);他能在没有人的地方招一圈子人,说他的笑话,江湖人说他专能做掉地(不挣钱的地)。凡是生意场、杂技场的艺人,都不敢挨着他做艺,江湖人说他的本领能扯“粘(nián)子”(观众)。他净躲着杂技场儿做买卖,江湖人说他有义气。他说完了一段笑话,卖一回药糖,江湖人说他是“挑(tiǎo)罕子”(江湖人管卖药糖调[diào]侃儿叫挑罕子),他那糖卖两大枚一包,总有人买。江湖人说,杵门增了(钱挣多了),买卖孝顺(生意好了)。这就是我老云向江湖人探讨来的大兵黄的内幕,是与不是,我不负责,好在是他们江湖人说的。电影的滑稽大王陆克、贾波林(即卓别林),在银幕上能受各国人士欢迎,就是能使人解颐,捧腹笑倒。滑稽艺术不止于北平人们欢迎,全中国的人士俱都欢迎。不到百段的相声,几十年来,有几百个艺人学会了,都能以它挣钱养家。不止于中国,全世界人士也是欢迎这滑稽玩艺儿的。

我老云希望江湖中的人们,不拘什么玩艺儿,也要加些滑稽艺术,管保能够火穴(xué)大转(zhuàn)(挣大钱)。这话是与不是,老合们(江湖艺人)的攒(cuán)儿(心里)是亮的,一定能够明白。





穷家门儿(唱数来宝的)


要饭吃的花儿乞丐,沿门乞讨:“老爷太太行点好吧,积德行善吧,赏给我花子点儿剩的吃吧。”凡是这种调门的要饭的人,不论男、女、老、少,瘸、瞎、聋、哑,都是真正的乞丐,是没有家门的。

凡是拿着块竹板子,且说且唱挨户讨要的,拿着撒(sā)拉鸡(撒拉鸡的形状是二尺多长的两块窄竹板儿,上安铁钉,再安几个铜钹,左手执之,右手另拿一窄长如锯齿的竹板,穷家门管这种家伙叫三岔板)的乞丐,使渔鼓、简板的乞丐,使竹板的乞丐,都是穷家门(唱数来宝的)的人。虽是向人行乞,不叫爷爷奶奶,不要剩吃剩喝,最低的限度是要一小枚铜元。

在早年最厉害的乞丐为“女拨子”,都是年轻的小媳妇、大姑娘。青布包头,手拿竹板,三五成群,到各商家、铺户强索恶化,或说或唱,或笑或骂,商家、铺户对于彼辈畏如蝎虎,倘若得罪她们,能够日日来搅,并且人数日见增加,在门前吵闹骚扰。最奇者官厅并不取缔,任彼辈横行,商家为避其嚣乱,顾其营业,少不得托人说合,然也牺牲许多银两而散灾。自从官方取缔后,“女拨子”的恶化丐妇全然消灭了。

如今在省市都会所能存在的只有数来宝的,在乡镇庙会尚有叫街的、擂砖的、削破头的(都属于乞丐,不过用不同的方法而已)。穷家门(唱数来宝的)的乞丐在早年都供奉范丹,如今都供奉朱洪武。敝人曾向彼辈探讨,为什么供奉朱洪武?据他们所谈,朱洪武系元朝文宗时人,生于安徽省濠州钟离县,父名朱世珍,母郭氏,生有四子一女,三子因乱失散,女已出嫁。四子即洪武皇帝,自幼异于常人,都说这个婴孩不是寻常的人物,将来定然出色。生他的日子是元文宗戊辰年,壬戌月,丁丑日,丁未时。在他出生时,人们还不太注意他的生辰八字,到后来他做了大明朝头一位皇帝,便有许多的术士们推考他的八字,说那八字辰戌丑未四库得全,不得时的时候孤苦零丁,得了时便可贵为天子。朱洪武名叫元璋,字国瑞,到了他会说话的时候,叫爹爹亡,叫娘娘死,剩下他一人,跟他王干娘度日。及其长大,送往皇觉寺出家,长老给他起名叫元龙和尚。长老待之甚厚,庙中僧人待之甚薄。长老圆寂后,僧人将朱元璋驱逐出庙,他王干妈将他送到马家庄给马员外放牛。放牛之处为乱石山,但他时运乖拙,牛多病死,或埋山中,或食其肉,被马员外驱逐。王干妈又因病去世,朱洪武只落得挨户讨要,因他命大,呼谁为爷谁就病,呼谁为妈谁也生病,后钟离县人民皆不准他在门前呼爷唤妈。朱洪武在放牛之处自己悲伤:十几岁人,命苦运蹇,至谁家讨要谁家之人染病。不准在门前喊叫,如何乞讨?他忽见地上有牛骨两块,情急智生,欲用此牛骨敲打,挨户讨要。于是天天用此牛骨敲打,沿门行乞。钟离县人民皆恐其呼叫爷妈,每闻门前有牛骨声至,都将剩的食物拿至门前,送给朱洪武。直传到今日穷家门的乞丐,都不向人呼爷唤妈,即其遗传也。

社会人士管那牛骨就叫牛骨头,穷家门的人管那牛骨头称为“太平鼓”,上有小铜铃十三个,也为朱洪武所留。相传有一个铜铃能吃一省,有铃十三个可吃十三省也。至元顺帝时,北地燕京城考场开科取士,朱洪武曾北上赶考,功名未中。行至良乡县土地庙内,忽患伤寒病症,倒卧殿内。至日落时,有两个乞丐携瓦罐而入,二丐见洪武倒卧在地,用手去摸他周身发烧,知为感冒伤寒所致,将他抬至殿后方砖之上,有狗皮两张,给他铺一盖一,将砖下掘洞,烧以柴草。到夜内朱洪武周身出汗,筋骨止住疼痛,二丐将其扶起,又将他们讨的剩菜剩饭用柴草热熟给他食之,至次日病已痊愈。问二丐姓名,则称梭、李二姓,为范丹的穷家门(唱数来宝的)人。今日之乡镇庙的乞丐,或称为梭家门人,或称为李家门人。每逢盘道问答时,常说“梭李不分家,多亲多近”。

后朱洪武北逐胡人,恢复汉人疆土,驾坐金陵城为一统大皇帝时,忽然染伤寒之症,太医屡治不愈。朱洪武忽然想起来,昔日在良乡县土地庙中曾染此病,为梭、李二丐所疗愈,今之病与昔日相同,如能寻着梭、李二丐来至,吾病不难除去。于是命人在各处寻找梭、李二丐。未几,竟将梭、李二丐寻至。洪武帝召见于寝宫,二丐拜伏于地。帝问曰:“你二人还认识我吗?”二丐说:“不识。”帝命二人抬头仰视,二丐连道不敢。帝强令仰视,二丐抬头观瞧时,见帝面白如玉,有无数黑痣,惟印堂有块朱砂红痣,两眼是上眼皮短,下眼皮长,耳大孔冲上,地阁阔大,口也冲上,鼻孔仰露,五漏朝天。忽然想起早年在良乡县土地庙中,曾遇一病汉,面生瘢痣,五漏朝天,他们用狗皮铺盖霸王炕为其疗病,以杂和菜食之,该人病愈后,问他二人姓名而去。不料那人竟是今之洪武大皇帝。帝问:“识我否?”二丐说:“认识。”帝问:“何处见过?”二丐虽然想起这事,不敢说明是他,遂道:“早年在良乡土地庙曾遇一病人,我二人为他疗病,那人却与万岁相似。”帝笑道:“那人便是朕。”二丐叩头问道:“万岁寻我二人何事呢?”帝说:“今朕仍患前病,命你二人调治。”二丐说:“霸王炕不敢复用。”帝说:“杂和菜能否再做?”二丐答:“可以再做。”于是帝命二丐往御膳房去做杂和菜。太监导引二丐至御膳房,二丐将鸡汤一锅放于院中,在御膳房静坐直至日暮。用鸡鸭汤掺各种菜饭,杂和一锅,在灶上熬熟,命太监进食,不料洪武帝食之,竟觉香甜味美,饭后周身见汗,次日病即大愈。再召梭、李二丐,欲封他二人为官,二丐连称:“命小福薄,且无才干,仍愿为丐。”于是洪武帝传旨,命二丐讨要使用太平鼓,且命鼓上安十三个铜铃,下缀黄穗,其他乞丐不准用黄穗,俱用蓝穗。使蓝穗乞丐不准入城。凡梭、李二丐讨要之处,不论商家、居民、文武官职都要给钱。于是梭、李二丐叩头谢恩。二人出宫之后,深悔未向洪武帝讨得住处,竟在通济门内挖城墙掘洞而居。地面官人不敢拦阻,后城外乞丐不得入城,欲入城者,或投梭为师,或投李为师。梭、李之徒日见增加,支派传流最为昌盛。

今日穷家门(唱数来宝的)人,称其门为六大支派,即丁、高、范、郭、齐、阎六姓是也。在昔帝制时代,南京乞丐之多为各地之冠。通济门内花子洞,即乞丐居留之所。至今南京之花子洞已由官方封锁,禁止乞丐居留了。在明太祖朱元璋太孙建文帝在位时,燕王朱棣由北京至南京,逼走建文皇帝,朱棣篡位之后迁都于北京,还有许多乞丐随驾北来,在北京借势恶化。传至清室未亡之先,北平尚有许多“杆上的”(即乞丐头儿)各辖一方。每有住户办红白喜庆事时,都邀杆上的在门前保护,防止穷家乞丐搅闹。如有宾客入门时,杆上的尚替本家招待。商家铺户新张以前,铺长必须向本街杆上接洽,并许以每节给银若干,杆上的便肯为其阻止乞丐恶化。

早年“逼柳(liū)琴的”(江湖中的生意人管穷家门的乞丐调[diào]侃儿叫逼柳琴的。盖生意人以一文钱调侃儿为柳琴,他们强讨恶化,也不过为一文钱柳琴搅闹而已。为逼柳琴使人生厌,江湖人皆轻视彼辈)在社会上任意扰乱,于秩序上极有妨害。现今强讨恶化已被取缔,穷家门多不化锅(穷家门管沿门乞讨调侃儿叫化锅,社会人士称为串百家门的),改在各市场、庙会、拉场子撂地(露天演出)。江湖人常说,昔日江湖人都严守规矩,在早年穷家门人不敢上地(做生意),摆地设场之人,更不赁给彼辈桌凳,倘若赁给他们桌凳,江湖中的金(算卦相面)、皮(卖药的)、彩(变戏法的)、挂(练武术的)各行人也不肯依的。如今穷家门的人们能在各市场、各庙会赁桌凳上地。二十余年前恐也不多见也。

庚子年北京城中所见穷家门的乞丐,家伙多是挂黄穗的,挂蓝穗的乞丐入城也有一定日期,须在每月初二、十六以后,否则入城必被杆上的(乞丐头儿)率众痛责一顿,逐出城外。但如今靠扇的(生意人又管他们要饭的叫靠扇的)随便入城,杆上的也天然淘汰了。





天桥数来宝的场子


数来宝的这种人不能算江湖艺人,他们是穷家的乞丐。在早年是串百家,沿户乞讨,向来没有到市场上地(做生意)搁场子的。江湖人调(diào)侃儿管他们叫逼柳琴的(见人要一文钱与要一大枚,调侃儿叫逼柳琴的),又叫化锅的。有几个老江湖人常和老云我聊大天,说:“如今这年月简直是江湖乱道,化锅、逼柳琴的也都上了地啦。”据他们这话考证,数来宝的在早年是不能上地(做生意)的。

数来宝的这种人不能算江湖艺人,他们是穷家的乞丐。数来宝的每逢上地,总是拿着两块牛骨头,牛骨头上有铜铃铛。



在天桥久占数来宝的是小海,约三十多岁,他向来没有准场子,因为他挣的钱少,摆地的人有场子都不愿租赁他。哪块场子闲着,他就上哪块场子。小海每逢上地的时候是拿着两块牛骨头,牛骨头上有铜铃铛,敲打起来是“呱的呱”。他们这行人所唱的玩艺儿都是浅而易懂的词儿,可是全按着十三道大辙编出来的,每到唱时还能带点滑稽词儿,能招得人们听着笑了。小海他一张嘴就唱:“天怕无时地怕荒。卖砂锅的就怕狗打架,害眼的就怕瞧太阳。罗锅子就怕仰着面来睡,洋车怕走泥塘。卖豆汁的就怕杵锅底,长秃疮怕痒痒。开店的就怕没有客,窑姐就怕长疮。”这些词儿粗俗下贱,上等的人、有知识的人绝不爱听。偏有些贩夫走卒没知识的人,专爱听他们这种玩艺儿。别的数来宝的都是两三个凑成一档子,逗起哏来,才有人围着听;惟小海、曹麻子两个人是专能一个人唱,有人围着听。两个人会的玩艺儿较比别人也多得多,故此能比别人多挣钱。

小海是久占天桥,至远到隆福寺、护国寺、土地庙赶个庙会,从不出北平的。曹麻子是专走外穴(xué)(到外地挣钱),北平要不挣钱,就往各村镇去赶集场、庙会。天桥虽然还有些个数来宝的,但是艺术不强,比不上小海、曹麻子,也没有人注意。我老云云别的,不愿云他们。





第八章 坑蒙拐骗


骗术门之骗法


在清末时代人人都是蓄发留辫,“扫苗”的行当(剃头的行当)还不似如今哪!有些个剃头匠每日挑着剃头的担儿,手持“唤头”(招揽客人的工具)去串胡同。有人剃头打辫,就将他们唤至屋内做活,到了春天暖和了,有些人在街巷内墙儿底下剃头打辫。有个剃头的师傅挑着担子走在三岔路口,有个人将他叫住说:“你给我刮刮脸哪。”剃头匠将挑儿放下,这人坐在座儿上,剃头匠用手巾将他的脖项一围,又将前边的热水倒在了铜锅之内,这个人站起来走到前边,哈着腰叫剃头匠洗脸。正在这时候,剃头匠忽见由拐角走过一人,冲他摆手儿,伸手端起那座儿(即剃头挑的后头)往拐角一退,剃头匠还以为拿凳的人和刮脸的人是朋友,他们闹着玩哪,他将凳儿拿走,刮脸的人往后一坐来个屁股蹲儿。这时他也不好说破。将脸洗完了,刮脸人往后一坐,噗通一声摔在地上。这人可急了,爬起来冲剃头匠一瞪眼说道:“你怎么摔我?”剃头匠说:“我没摔你,方才有个人将凳儿给拿了走啦!”这人说:“没人和我玩笑,你快追吧,他许是将凳子拿跑啦。”剃头匠似有觉悟,往拐角那边一看,拿凳子的人连影儿都没有。他才着急,料着那人走不了多远,撒腿就追,追出多远,也没追着。急得他无法,往回走吧,及至到了拐角儿再看那刮脸的人哪,也没有啦,连前边带铜锅的挑儿也没有啦!他到了这时候方才明白,那两个人是骗子手,两个人各骗一头儿,一份剃头挑子算是被人骗走了。那个年头骗子手们要骗剃头挑子,就用这个方法,直到被骗的上了当的人多了,一传十,十传百,才哄嚷开了,骗子手再用这法吃扫苗的可就不成了。

在清室鼎盛时代,骡马市大街净是骡马店,由口外来贩骡马的客商,贩来了骡马,都在店内寄卖。他们开店的与纤(qiàn)手(给人介绍买卖产业的人,即中介。旧时称牙行)们搭着,明着有成儿,暗中有扣头。有一天,鞍韂铺的伙计见有一个人,穿着阔绰,来买鞍韂,他挑选了一副很好的鞍韂,言定了价钱是十五两银子,他叫伙计扛着鞍韂跟着他,往马上试试,试好了就留下使用,叫伙计将银子拿回。伙计扛着鞍韂,往西而来,到了一家骡马店,这人叫店伙牵出一匹马来,向鞍韂铺的伙计说:“你将鞍韂鞴上试试。”伙计将鞍韂往马上鞴好,这人向他说:“你等等,我试试就回来。”鞍韂铺的伙计觉着这匹马就能值个几百银子,骡马店都叫他骑了去,一定是熟客人,没有错儿的,就点点头说:“好吧!”那骡马店的人以为给他扛着鞍韂的人是那骑马的家人哪。他虽然将马骑走,有他仆人在这里等着,一定没有错儿。他们彼此误会之际,那骗子手骑了马飞也似地去了。鞍韂铺的伙计等着工夫大了,不见骑马的人回来,他等急了,向骡马店问道:“这位骑马的怎么还不回来?”骡马店的人说:“那不是你的主人吗?”那鞍韂店伙计说:“不是。他是买鞍韂的客人,他还没给我们鞍韂钱哪!”骡马店的人才知已然受骗了。受骗之后,两下里还打了场官司方才完事。骗子的“流星赶月”的方法,也真巧妙。

在清末时代有骗子手赵老三者,一日往大栅栏某园观剧,他穿的衣服阔绰,被“老荣”(小偷)看见,以为他是阔少,同他进了戏园子,坐在一条凳上并肩聆戏。是时,戏台上正演张黑之《大卖艺》,台帘一起,张黑从台帘后跑出来,离着台柱近了,将身一转,肩背在柱上,两足悬起,这功夫叫“粘糖人”。赵老三看着入神之际,老荣(小偷)乘他不防,将他二两银票荣了去啦(即是偷了去啦)。到了查票的时候,赵老三伸手掏银票可就愣住了,一张银票,不翼而飞。他料着必是叫老荣偷去,赌气不听戏了,将这事说给他哥哥赵老二。那赵老二是有名的骗子,听他兄弟说被小偷偷了,不肯甘心,他要去骗小绺(xiáo liu)(小偷),以偿损失。他将身上收拾好喽,手持银包走到珠宝市一带,往各银号兑换金条。有某小绺在银号外,窥其金条,有意偷他。赵老二由银号出来,拿着金条往大栅栏听戏,小偷也随他入戏园,在池子内并肩而坐,要想偷他的金条。赵老二见那小绺(xiáo liu)(小偷)也很漂亮,人物俊俏,头戴海龙皮帽,披着狐皮斗篷,看那斗篷也值数十两银子。赵老二故意将金条放于桌上,假装看戏看得入神,那小偷乘其入神,将金条窃到手中。赵老二暗将小绺的斗篷角儿,坐在屁股底下。小绺起身要走,见他的斗篷被人家坐在屁股底下,他合计着所偷金子能值很多,一个斗篷算得了什么,他要给丢主一个迷糊招儿,爽性将斗篷一甩,交给赵老二说:“我去小便,劳驾你给看看。”赵老二微一点头,小绺便匆匆走去。他拿着金条出了戏园子,要想合计金条的数目,到了一个银号要兑换金条。银号伙计说:“你这金子是假的。”小偷方才觉悟,自知受骗,叫人家使了“抽梁换柱”,将斗篷骗去。找到戏园之内,那个赵老二早拿着斗篷走啦。小绺无法,自认倒霉而已。这是“狼吃狼,冷不防”,骗子的手段也是可怕呀!





骗术门的老合们(骗术门中走江湖的人)


骗术门的老合们,也有两个人为一伙的,也有四五个人为一伙的,更有十几人、几十个人的。最难不过是一个人去骗取银钱的。自从有了报纸以来,骗匪们很受影响,骗人的方法只要用过一回,就不能再用。就以某日报载:某姓在大米庄买了六袋洋面,买到了家中,忽然来了两个米庄的伙计到这家说:“我们是柜上打发来的,你们家买了六袋洋面,内中有两袋是假的,布袋是‘蝠星’的,面可不是‘蝠星’的,我们先生说怕对不住你们,派我们俩人来看看,说将两袋串袋的扛回去,另给您换两袋真正的‘蝠星’洋面。”这家一时蒙住了,就叫两个人将两袋洋面扛走啦。事后不见他们给送回那两袋洋面,到了大米庄一问,大米庄的人说没派人去,根本没有这么回事,大概你们让人给骗了。话道破了,这才醒悟是被骗了,只好自认倒霉。偌大的北平,哪里去找那骗匪呀?受了骗无计可施。报界的人们得了这条被骗新闻,登在社会版上,阅报人们看见了,一传十,十传百,由新闻纸一宣传,阅报的人一哄嚷,社会上的人士都知道了,骗匪们再用这个方法去蒙骗人,恐怕不能成了。报纸上宣传的人人都知道了,他那骗人的法子就不中用了。由这一档子事考查,报纸的宣传力是最大的,只要将他们骗人的法子宣传出去,无论那法子多好,也不能再用的。

说着话,他伸手从口袋里掏出一颗血淋淋的人头,往柜上一放。可把阖柜的人都吓坏了。



在敝人十岁那年,曾记得北京出了一件骗人的事儿,我把那骗人的事情写出来,贡献于阅者。我记得那年是光绪二十四年(1898年,戊戌变法那年)的冬季,有一家银号,买卖很为茂盛。一日,柜上的伙计、掌柜的正然闲聊天儿,看见了一个乡下人背着一个口袋到了柜前,向他们问道:“银子卖多少钱一斤哪?”合柜的人听着都是一愣,换银子向来是论多少钱一两,一钱银子换多少钱,还没听见说过银子论斤换钱哪。伙计、掌柜的再一看这乡下人怯头怯脑的,像个老赶,先不告诉他银子的行市,先问他有多少银子。这乡下人说:“我有一坑银子哪!”柜上的伙计问道:“你这银子是从哪里来的?”乡下人说:“是我掘出来的。”阖柜人听他所说,才知道他得了外财啦。有一个人告诉他:“银子是一百二十吊钱一斤。”在那时代,每两银子按行市还不到十吊钱(也就在七八吊钱),这乡下人听说一百二十吊钱一斤,喜喜欢欢地道:“我这一斤银子卖给你们啦。我问了好几家啦,都说不到一百吊钱,你们这买卖真公道,卖给你们吧。”柜上伙计将他的银子过过分量,整够十六两(旧时十六两为一斤),遂付给他一百二十吊钱票子。他拿过票子,先回头往外看了看,见没有人来,他向柜上人说:“明天我晚上来,在你们上门的时候我准到的,再卖给你们五斤。从此,我是天天来,卖了银子再买些零碎的东西。可是我怕别人知道了,我来了的时候,你们可千万将门关上,等我换好了银子再开门把我放出去。”柜上人说:“好吧!”乡下人高兴而去。他走后,柜上的人们可有了谈话的材料了。大家你一言我一语地谈论起来,都认着他是穷人有钱活受罪,早晚许叫银子把他折腾死。到了次日,掌灯以后,柜上该着上门了,学徒们将门都上好了,他还不失信,扛着口袋来了。一进门就闻见了他酒气喷熏,那味儿放出多远去,已醉得眼珠都红了。他往椅上一坐,谁也没理,学徒将门关上,上了闩啦。伙计问他:“你今天卖几斤银子?”他把眼一瞪,说:“你们这买卖怎么做的?欺我们乡下老赶。银子都是论钱论两,没有论斤的。你们拿我当老赶,我媳妇不老赶,她由昨天就骂我,直骂到了今天掌灯。我气极了,用刀把她砍啦!”说着话,他伸手从口袋里掏出一颗血淋淋的人头,往柜上一放。可把阖柜的人都吓坏了。他又由口袋里掏出一把切菜刀来,将大棉袄的纽扣儿解开,往那小棉裤上一看,尽是血啦。他说:“哪位是掌柜的?这场官司咱们打了吧。”此时掌柜的吓得净剩了哆嗦啦,哪里还说得出来话呀!幸而柜上还有两个能说话的伙计,胆子也大点,向他劝道:“朋友,这官司你可打不得!打了官司,你得给你媳妇抵偿对命,我们柜上的人可抵不了偿。你的命也不是盐换来的,不如你趁着没人知道,还没犯案哪,你赶紧跑吧!远远地一走,你的命就算是保住了。”他听着伙计这样劝,他哪里肯干哪!攥着那把菜刀,气势汹汹,真是要和掌柜的拼命似的。后来大家好劝歹劝,费了许多唇舌才把他劝好喽,由柜上给五百两银子,叫他远走高飞。直到三更多天,他才拿了五百两银子,连人头一并装在口袋里,徒弟给他开开门,他才走啦。徒弟赶紧把门关上。掌柜的直说:“万幸万幸,要是打了官司,这不定得花多少钱哪!我看他那满脸的煞气,我真害怕,我怕他急了用刀砍了谁。”大家议论着,徒弟把柜上的血迹擦了去,大家愈想愈后怕,直到四更多天,阖柜的人们才睡了觉。天光将亮,外边有人啪啪地叫门,说:“掌柜的,你们门上挂着一个人头,还不快出来看呢!”这一来可把银号的掌柜的、伙计们吓坏了,阖柜之人无不担惊。及至将门开开,出来观瞧,不看这人头便罢,一看那人头无不惊讶。原来那个人头是假的,用泥捏的人头,上边的头发是真的,模模糊糊,抹的净是猪血。阖柜之人受了这个骗,醒悟过来可就晚啦,受了一夜的惊恐,叫人骗了五百两银子。这个事要搁在如今,报纸上又有好材料了,当作一件新闻登出去,准能轰动社会。在那个年头儿,东城出了新鲜事,西城的人就不知道。现在有了新闻纸类,与社会大有益处,实匪浅鲜。

最近北平城内不论大街小巷,忽然添了无数乞丐,看他们的样子都不是北平人,穿着打扮都像乡下人似的,个个身上都不单寒,全穿着棉裤棉袄,三个一群,五个一伙,男的很少,妇女、小孩在多数,每逢出了太阳的时候,他们就全体出动,散开了各有地盘,看他们又不是有嗜好的样子,为何都出来行乞呢?最奇怪的是年年一到入冬的时候,他们就来,等到转过年去,不到清明节就全都走了,一个也不见了。敝人曾经调查,又向江湖人打听、讨论过此事。据一般老江湖人谈论说,他们这种要饭的人,不是真正无家无业贫苦无依的,个个家里都有房子有地,他们都是×县的人,每逢把大秋收获之后,将棉衣裳全穿齐了,留个人看家,不管有多少口人,全体出发,做他们要饭的事儿,混个冬天,反正在家里也是无事,混到了春暖之时,该着种庄稼啦,便一齐回家种地。他们这种乞丐,江湖人调(diào)侃儿称为“叫点”。这叫点是个总称,此外还有什么“挑(tiǎo)衫”的、“化锅”的、“挑(tiǎo)怎”的、做“悬点驼”的。

什么叫挑衫的?前几天我工作完毕,想到天桥巡礼,乘车前往,在各处游逛,见有一帮要饭的共有五个人,四个人在地上坐着,把头低着假装哭啼之状,是一个老太太,两个妇人,一个姑娘,站着的是一个三十多岁的男子,怯头怯脑的,穿着一身粗蓝布小棉袄小棉裤,手里提了个青布大棉袍,脸上故作发愁的样子,嘴里叨叨念念的,招惹那逛天桥的人们围着观瞧。我也看看吧。那拿着大棉袍的男人说:“众位老爷们,俺们是逃难的,家里的房子地都被水淹了,一家五口人来找俺表哥,俺表哥不在北平,俺都扑了空啦,盘费也花了啦,举目无亲。阖家大小从今天早晨起还没吃早饭,俺也没有别的法子,就剩了这个大棉袍了,哪位要买,卖给你,俺一家子好住店吃饭。”他这套说完了,从头再说,总是这几句。别看这年头儿经济紧张,真有看着可怜的,也有给掏一毛的,也有三个五个铜子的,至少也是一大枚,可是没有一位忍得买他那棉袍的。敝人看了会儿,才明白他们这帮儿就是“挑(tiǎo)衫”的。那个男人说得叫人听着可怜,好有人给他们抛杵头儿(扔钱),他们所说的那片话,江湖调(diào)侃儿叫做“哀怜口儿”。大约着他那棉袍儿这一冬也卖不出去,等到来年三月回家种地的时候,还收在柜子里呢。这种挑衫的,给他们几个钱倒不抽白面儿,他们对得起人,专吃黑面的。他们是可怜的生意,有钱人何妨可怜可怜他们。

还有一种人是不必可怜的,就是“挑怎(tiǎo zěn)”的生意。做这种买卖也得五六个人,不是用筐挑着孩子,便是用小车子推着孩子,到了人烟稠密的地方,找个不碍事的去处,一家老幼都往地上一坐,一齐用“抛苏儿”(江湖人管哭哭啼啼调侃儿叫抛苏儿)圆粘(nián)子(招徕观众),他们阖家老幼足这么一哭。社会里的人们好奇心盛,都围着观瞧,也是一个男子站着叨叨念念的,但不是卖棉袍儿,是抱着个两三岁的小男孩,他也是用“哀怜口儿”,说:“众位先生们,行点好吧!我们是逃难的,家里的房子地儿都被水淹了,我们一家老幼,要到关东去找我兄弟,走在这里没有盘费啦,哪位要是没有儿子,你把我这个孩子抱了去当个小狗养活,多少给我们几个盘费钱,就把我们一家子给救了。”这套话说完了,从头再说。有那心慈面软的人就掏给他们几个铜子。他们管人可怜他们的钱调侃儿叫“前棚的零碎杵头子”,他们拿这些钱不当回事,做大号买卖得弄个几十块钱。可没准儿三天、五天、个月有余才能碰得上哪!遇见那有钱的人家没有儿女,都想抱个小男孩承继宗祧(tiāo),多会儿有这种人恰巧碰见他们,只要一搭话就得上当,不管花个十元廿元把小孩买到手,往家中一抱,他们就有人在后边跟着,认准了门户,这麻烦可大了。他们把小孩卖了,调侃儿叫“挑怎”。挑完了“怎”之后,钱财到了他们手里,谁买他们的孩子,找到谁门前堵着门儿跪着一哭。这种跪哭是有效力的。多咱哭闹得本家烦啦,把孩子给他们才算完。如若不给他们孩子,什么抹脖子、上吊种种的威胁手段,笔难尽述。这种挑怎的专吃这手儿。那位说要遇见了渣子行(贩卖人口的)呢?渣子行是不管买男孩的,挑怎的是向来不卖小姑娘,与渣子行是风马牛不相及的。

他们把小孩卖了,调(diào)侃儿叫“挑怎”。挑完了“怎”之后,钱财到了他们手里,谁买他们的孩子,找到谁门前堵着门儿跪着一哭。这种跪哭是有效力的。多咱哭闹得本家烦啦,把孩子给他们才算完。



再说“悬点驼儿”(也叫他们放鹰的)的买卖。什么叫“悬点驼儿”呢?江湖人调(diào)侃儿管忘八(即王八)叫“悬点”。他们假装逃荒难民,三五个人合而一帮儿,到处嚷嚷卖媳妇。江湖人管这种骗局叫做“悬点驼儿”。这种生意是犯法的事儿,躲着法律。他们遭了官司,能用狡猾的手段对于法律推干净,即或推不干净,也要就轻罪躲重罪。最奇的还是他们总不遭官司。未曾做买卖之先,就将媳妇夹磨(jiá mo)好了(调教好了),卖到了什么人家,用什么方法逃走,也是对病下药的意思。到了夹磨好啦,能够出来做买卖的时候,要预备一条扁担,两个筐儿,一头挑着被褥行李,一头挑个有几岁的小孩,带着媳妇出来骗人。出来的时候也是在大秋以后,入冬的季节,专到省市城内、商埠码头,不在热闹繁华的去处,找个清静的地方,把挑儿一放,两口子蹲在地上抛苏(即是哭),招得过往行人一看,就把粘(nián)子(观众)圆好了。媳妇哭着,男人说着:“众位先生!我是逃荒的,我们那个地方好几年没收,树皮都吃光了,阖家老少八口饿杀啦!就剩我们三口逃了出来,逃至你们这个地方,举目无亲。我要往黑龙江去找我兄弟,他在那里给人种地,好几千里的路儿,没有盘费,三个人非饿死不可。哪位行好救救我们,我媳妇谁若要,叫她给做点针线活,做菜做饭,当个老妈子使唤,给我个盘费我就走啦,到黑龙江找兄弟去。”也有人瞧着他们可怜,给扔几个铜子的,也有给几角钱的,遇见慈善家,真有给他们几十块钱的,这些钱都是前棚(场上)的杵头儿(钱)。若是有那没媳妇的人,或是断了弦还没续娶,以及夫妻无有子女,媳妇有病不能生养,要想纳妾立后的人,遇见这种悬点驼儿(卖媳妇)的生意,准得上当。瞧那男的哭哭啼啼,又很可怜;瞧那媳妇岁数又年轻,长的模样又好,花钱不多。表面上看还是一举两得的事儿,暗含着是买卖人口。只要有人愿找这种麻烦,一搭话就得。那种生意人都会要簧。什么叫要簧?就是谁要买他媳妇,必先用口话探讨谁家家中有几口人?有多少产业?本人做的什么事儿?他把簧都要过去,心里一合计,能够生得了财,就能愈说愈近。他卖媳妇,谁买媳妇,商议吧,准能成功。等到谁把洋钱给了他,立好了字据,媳妇留下,把钱带走,叫你瞧着很放心:他是拿着洋钱往黑龙江去了。暗含着他又回来,找个落脚的地方等着,他媳妇偷跑出来,他们远远逃啦。谁要是倒霉倒得轻,花个几十块钱,不留神那女人跑喽,找着他男人,两口子同逃,也就完了。设若看得太严,又不叫娘们逃跑,又不叫媳妇摸着银钱,那可就快要自己的“章年儿”(江湖人管被害了与要人的性命调[diào]侃儿叫要章年)了。骗子们的手段又毒又辣,可怕得很哪!假若我们要遇见这种人,要商量着买他的娘们,他一要簧(要出实话来),这人说他是在机关当书记,家里有二十几口人,有的是房产事业,要和他们商议,愈商议愈远,休想商议成的。总而言之,世上的事儿,是便宜不贪,是便宜不爱,抱定这个宗旨,绝不会上当。必是贪便宜才能受害。吃搁(gé)念(指江湖人、生意人,调[diào]侃儿管他们自己叫搁念的,又称为老合)的人们,在生意道内年数多了,所经的所见的都是可怕的。阅历深了,不上当的诀窍就是不爱便宜而已。





骗术门之内幕


年前,通县长途汽车站地方,有由兴隆县来的杨某欲往北平。在站候车之际,有一人散放传单,杨某接了一张,见传单上印着是:“北平大兴华银号启事:本号司账李树华,年二十四岁,江苏省镇江人氏,在柜服务有年,素极老诚,不料最近冶游亏款,节关将近,彼竟将柜款一千七百元拐逃,遍找无踪,业经报案。不论哪界人士,如有将其捕获者,酬洋五百元;知其下落送信与本号因而破获,酬洋百元。储款以待,绝不食言。今开具该拐犯相貌如下:中等身材,面白无麻,惟左眼皮上有朱砂痣一块,分头,镶有金牙两个,戴美式毡帽,身穿湖绉夹袍,春绸夹袄,上海式礼服呢鞋。中华民国二十四年夏历八月十日,北平大兴华银号经理谨启。”

杨某看毕传单,折起来收在兜内。在他身旁站立有一人,也手持传单观瞧。杨某见这人长得身躯高大,相貌魁梧,像个练武的样子,约有三十多岁。这人见杨某看他,就问杨某:“你也往北平去吗?”杨某说:“我到北平西直门外海淀去看个朋友。”这人说:“我也到海淀有事,我们搭个伴吧。”说着,他把那传单折起来在手中拿着。工夫不大,汽车来了,他们买票上车,挨着坐着。车开出了通州的时候,两个人闲聊大天,杨某问他姓氏职业?这人说,姓王,叫王绍贤,在某机关服务。两个人直聊到北平东四牌楼汽车站,下了汽车又改乘电车到了西直门,同行出城,走在路上闲谈。行至中途,见路旁有个钱铺,有一男一女买烟。王绍贤用胳膊肘儿一拐杨某,悄悄说道:“你看那买烟的男子。”杨某站住了仔细一看,这个男子长得中等身材,白脸膛儿,左眼皮上有一块朱砂红痣,上齿有金牙两个,头戴美式毡帽,湖绉的夹袍,春绸的夹袄,上海式的礼服呢鞋,约有二十多岁,手中提着一个皮包。杨某见这人与那传单上所载的相貌穿着一样,他很是惊讶。就见那王绍贤气势很壮,过去用手一拍那拐款之人说:“朋友!你跟我到那边有句话说。”那拐犯与那女子立时面上就露出惊慌的神色,好好地跟着王绍贤往房后而去。杨某看着走了心神,也跟随这一起人走到房后。就见王绍贤向那拐犯说:“你这官司打了吧?”那拐犯当时跪在地上给他磕头,苦苦地央求,那女的也直说好话。杨某在旁听他们所说,才知道拐犯是由柜上拐了一千七百元,用三百元接了个妓女,要往江苏回归原籍。不敢走北平的各车站,怕有官人捕获遭官司,绕道走在这里啦.被王绍贤遇见。他怎么哀求也是不成,最后那拐犯打开他那大皮包,杨某凑过去一看,那包内有一对赤金镯子,四个金戒指,两匣人参,那拐犯由皮包之中取出来有二百元钞票,向王绍贤说:“朋友,你要把我放了,我有一百五十元酬谢你,我感激你的好处,我们还是朋友。”王说:“一百五十元那可不成。”说着,把那张传单取出来叫他自己看,那上边有酬谢五百元的字样,说:“我放着五百元不要,要你这一百五十元?你跟我打官司吧!”这拐犯说:“我这一千七百元,除花了五百元之外,都买金首饰了,只剩这二百元作路费啦,给你一百五十元,我留五十元好回家呀!”王绍贤执意不肯。他们费了许多唇舌,杨某也假装好人,给他说好话,结果二百元都给了王绍贤,那姓王的拿着钱匆匆而去。杨某看着便宜,觉着这里有油水,他也伸手恫吓拐犯。那拐犯到了这时表示后悔,愿意急速回家,免得遭了官司。他没了现款,愿把东西变卖了,有路费好走。杨某身上带着七十元钞票给了拐犯,留下人家两个金戒指,一只金镯子,两匣人参。拐犯感谢他,去了。杨某觉着这东西能值三四百元,他欢喜得了不得,连朋友也没瞧就回城内来变卖这些东西。不料到了金店碰了个大钉子,那金戒指、金镯子都是假的。他又往药铺去了一趟,求人家给他看那人参,结果也是不真。他到了这时候才醒悟了,受了骗匪的“流星赶月”啦,花了七十元,买了点子假东西。

杨某与老云的朋友是朋友,我把他受骗的事写出来,揭穿个中黑幕,杨某的姓名就不用说了。他被骗的原因是在通州吃早饭时露了财,才被骗匪注意,设局将款骗去。看起来还是行路别露财为妙!





骗术门之老渣


老渣,俗呼“渣子行”(贩卖人口的),这渣子行儿的人所做贩卖人口,拐带良家妇女,离人骨肉,断人子孙,灭绝宗祧(tiāo),是无人道的。敝人将他们的内幕揭开了,公诸社会,使社会的人士加以注意,努力宣传,免得知识幼稚的妇女坠其术中,也是件有益的事呀。

渣子行的人所做贩卖人口,拐带良家妇女,离人骨肉,断人子孙,灭绝宗祧,是无人道的。



渣子行贩卖人口,以敝人所知道的分为两大派:一派叫“不开外山”;一派叫“开外山”。这“不开外山”的是怎么个意思呢?即是遇有贫寒之家,衣食两绝,生计困难,他们见这贫寒之家生有子女,向其下说词,将儿女卖了以顾衣食。由几个月至七八岁的小孩,他们给介绍卖给“养家”。“养家”花钱买个小姑娘,事先讲好喽,是“活门”、“死门”。“活门”是准孩子的亲爹亲妈看看,也分多少日子看一回,大多数是四季三节(一年中的立春、立夏、立秋、立冬,端午节、中秋节、春节)瞧看;“死门”是卖了孩子以后,不准小孩的亲爹亲妈瞧的。养家花许多银钱买孩子,十有八九都是讲究买“死门”的。买“活门”的也有,那可不是养家,是没有儿女的人家,买个孩子,承继宗祧(tiāo),这种都买男孩。为什么凡是买女孩的都讲“死门”呢?他们将孩子养大了,不是学唱大鼓,便是学戏,或是为娼,将孩子养大了便是摇钱树,给他们挣钱。社会的人士管他们叫“养家”。至于领家,是与渣子行(贩卖人口的)讲好喽,不买很小的,专买大姑娘、媳妇,最小的也得过十五岁。将人买到手内,往娼窑里一送,上捐就挣钱。一个人讲究领多少个妓女,社会里的人士管这种人叫“领家”。凡是卖儿女的人都舍不得,环境不良,挤得无法才出此下策,将孩子卖了,一狠心能成,出远门舍不得。渣子行的人,不用往外省送,在本地就有买主,江湖管这不往外送的渣子行,叫做“不开外山”。这不开外山的渣子行,又名叫“纤(qiàn)手”。差不多都盘踞在娼窑附近的茶馆酒肆里,三五成群地干那鬼鬼祟祟的事儿。专以联络“养家”、“领家”做生意的,“开外山”的,可又不同了。他们专以往外省贩卖人口为生,他们的手段较比不开外山的毒辣多了,都是媒婆改行的。在我国政体未改变之先,有三姑六婆最为可怕,治家格言有几句是:“僧道尼姑休来往,在堂前莫叫卖花婆。”三姑是:尼姑、道姑、卦姑;六婆是:稳婆、花婆、巫婆、虔婆、药婆、媒婆。在古时代有欠债难偿的时候,由县官就将该卖的女子交与官媒,变卖人家女子还债。自入民国以来,这种官媒就已然取消了。私媒在当年也盛行一时,北平的人士管他们当私媒的叫“老妈作坊”。开老妈作坊也不容易,吃这碗饭必须能走动才成,至少也得有几个府门头(北平人管官宦人家叫府门头),还得知道各府里主事人是谁,本着主事人的所好,给他找人。乡下妇女进京以及本地寒家妇女要当老妈(北平人管女仆叫老妈),先得到他媒人家内去住着。譬如,这家老妈作房走的门子,主人都是好人,他那作坊就专收容品貌端正、懂得规矩礼节的良家妇人,设若他走的门子,主人都是下三滥,他给雇的女仆,长得要好,岁数还得要年轻。叫上这种老妈,到了主人家中能揽大权,十有八九都得生出是非的。他们受过老妈作坊的训练,有三大技能,是吃、恨、偷。还有伺候姨太太、小姐的老妈,讲究是跟丑、跟俏、跟起、跟落。到了如今,社会里的人们知识渐开,不用说雇老妈,就是买卖房产,租赁房子,都不愿经纤(qiàn)手(中介)的。谁家要雇佣女仆,花不了多少钱,登报征求,也不愿用受过老妈作坊训练过的。因为老妈作坊的内幕不良,官家严加取缔,定个章程,凡是开老妈作坊,得预先呈报官署,还得有两家连环铺保,经过多少手续,调查相符了,才发给他们佣工介绍所的许可执照。为什么官家这样的严厉呢?在从前的老妈作坊很有不少是开外山渣子行(把人贩卖到外地的人贩子)的大本营,遇有好吃懒做的老妈,便与渣子行勾串,施其奸拐卖的手段,将岁数年轻、长得有几分姿色的妇女弄到外省,往娼窑里一卖,送到万劫不复的火坑算完。

在如今社会的人心日见险诈,竟有能吃骗他们渣子行的人,分为三种吃骗法:一曰“吃封”;二曰“吃定”;三曰“转车”。什么叫吃封呢?譬如外省的人贩子来到北平,得找渣子行的纤手,叫纤手给找卖孩子的,或是卖媳妇的,那情形如同做买卖一样。纤手找着要卖人的,不论是姑娘、媳妇,得叫渣子行的买主先瞧人,后讲价。瞧,可不能白瞧,每逢瞧一回人,得花一元至两元,这种钱给了要卖人的,叫做“相(xiàng)封”钱。有那聪明的人,被生计压迫得无法谋生,只要有十几岁的姑娘或是二三十岁的媳妇,就可拉拢纤手,扬言要卖人。纤手有了客人的时候,就带着他们去叫客人相看,只要客人看完了,将一两元的“相封”钱骗到手内,再跟他讲价钱买人呀!他便施其狡猾的手段,无论如何也买不妥的。今天骗东家,明天骗西家。处在这险诈的社会里,鬼祟的事儿多着哪!用这个骗相封的方法,就能苟延残喘,暂顾燃眉的。被骗的渣子行是哑巴吃黄连,有苦难言了。

什么叫“吃定”呢?譬如,纤(qiàn)手将卖的人带了去,叫客人相看,当日看完了,不论是姑娘、媳妇,只要看如了意,照规矩(也不知谁定的)给了相封钱(见面礼),然后就可以讲价钱,将价钱讲妥了,得先给个十元廿元的定钱,算是定妥了。大凡外省的客人,远路风尘地来趟很不容易,绝不能就买一个人呀,多咱人买得够了,他要走啦,再找纤手要那给了定钱的人哪。可是那卖人的将定钱骗到手早就急流扯活(huo)了(跑了),急得那纤手眼珠子都蓝了。还有纤手与卖人的做活局骗定钱的,然后假装好人。被骗的渣子行(贩卖人口的)不能为这事打官司,干的是犯法的营生。除了向纤手山嚷鬼叫、拍桌子瞪眼暴躁一阵之外,别无办法。骗定钱的这种人较比吃相封钱的人还厉害,这叫“狼吃狼,冷不防”。

比这骗定钱的还恶的人,讲究“转车”。什么叫“转车”呢?譬如,渣子行将人买妥了,不拘是几百元,钱是给了人家,到了要走的时候,对买的姑娘先用好吃好穿的买动了她的心,然后再训练一番,所训练的事情是怎么上火车,怎么上轮船,路途中有军警盘查的时候,是怎么问怎么回答。在训练的时候这个姑娘假装好人,听说听道的;及至到了车站,买好了火车票,上了火车,她还老老实实的;等到火车一拉笛,眼瞧着要开车了,这姑娘能够趁乱之际,三转两转没有喽!就是你看得多严也不成的,她在家早就训练成了,专门“转车”坑骗渣子行的。实在看得严密,她就要明走。渣子行的人若是识时务,认倒霉便罢,倘若不肯白扔几百块钱哪,过去一揪她,她就喊巡警打官司。说句丧话,渣子行的几百元大洋没有了,得个诱拐妇女之罪,还得蹲几年的监狱,够多么冤哪(是他们自找)!若是做正大光明的买卖呢,管保遇不见这类事。凡是“转车”骗钱的妇女,种种的手段是研究好了的,无论怎么样她也是有主意的。

还有比这种人厉害的,譬如,渣子行平平安安地将人买了走,上了火车、轮船,到了他们的目的地,无论是商埠码头、省城都市,都有“老柴”(官人)们盘查。有些地痞流氓,和老柴们联合着:说真了,他们把人交给老柴,按着公事路办;说假喽,他们遇见有贩卖人口的,或是私运毒品的,假装老柴伸手办案,走在僻静的地方,犯法的人哀求他们几句,他们就假装善人,将犯案的人给放了,可是毒品得给他们,渣子行(贩卖人口的)得把买的人抛了。这半真半假的地痞流氓们,得了毒品他们也去卖了,得着人他们也是卖了。这种软硬炸酱的手段,尤为厉害。

所以,开外山的渣子行(把人贩卖到外地的人贩子)挣俩钱儿实是不易,第一得为人机警,第二还得有大本钱,第三是沿路上的老柴(官人)都得认识,和各处的地痞、流氓,明着是交朋友,暗含着往狼嘴里送点油水,顶着蹲好几年监狱的罪名干这犯法的事。若是运气旺,能干几年不遭官司,落个吃喝玩乐眼前欢,终归也积蓄不下银钱。即或落了钱,立下点事业,也要出横事遭恶报。好吃的饭不搁筷,不定哪阵时气一背,遭场官司就得家破人亡。有人说他们这行挣得够过的,不会改了行洗手不干吗?为什么都得遭官司,把所挣的钱全都倒出来,到了监狱落个罪人,方才觉悟呢?这叫“菜里虫菜里死”。离人骨肉是最可恶的呀,干这种缺德的事儿要没有个报应才怪呢!

奉劝老渣们,干什么不能吃两顿饭,何必一定干这早晚喂狗的行当?再奉劝一般做家长的,住在哪条胡同,都要留神街坊、邻居有没有老渣们?如若有啊,或是留神注意,或是少叫人串门子。渣子行引诱妇女的手段比吸铁石还厉害呢,等到他们将人拐了走,送在那万劫不复的火坑里,等接到那被骗后悔、请求由火坑往外救人的书信时,可就麻烦了。





小绺(xiáo liu)(小偷)门


小绺门是专在人群里窃取他人财物的。社会的人士叫他们为“小绺”,彼辈每日三五成群去到火车上、轮船上、电车上、公园、市场、各庙会里做他们绺窃的事儿。凡是被他们窃过的人,每逢到娱乐场、杂技场,都有留神小绺的戒心。电车、火车、轮船,都悬挂着木牌,写着“留神小绺,谨防扒手”的字样。江湖人管他们小绺这行人调(diào)侃儿叫“老荣”(小偷),又叫“镊子把”。老荣是他们总名儿,虽然都是小绺,所吃的路线各有不同。计分:“轮子钱”,是专吃火车、电车上的旅客的;“朋友钱”,是专吃半熟脸的人;“黑钱”,是专在夜内偷的,白天不作活;“白钱”,是专在白天偷的,夜内不作活;“高买”,是专吃金珠店、绸缎店、银行银号的。社会里有一种半开眼的人管小绺(xiáo liu)(小偷)叫“白钱”。敝人曾云游几个省,耳濡目睹,他们这行儿不拘在什么省市码头地方都有头儿,调(diào)侃儿叫“瓢把子”。地方小的只有一个“瓢把子”。大地方还有大头儿,叫“总瓢把子”,在总瓢把子之下还有许多小瓢把子。按他们的规矩,是每个瓢把子管辖区域内,有小绺偷着了东西,不论是值钱不值钱,偷着的时候不能就卖就花,得将所偷的东西先叫他们的瓢把子收存三天。在这三天之内,若丢失的人有势力,找得很急,也好在三天之内货归原主;若是过了三天,没有动静,一定丢东西的人没有势力。若是东西物件往外一卖,将钱分着一花,调侃儿叫“挑(tiǎo)喽啃(kèn)杵,均杵(分钱)头儿”。

电车、火车、轮船,都悬挂着木牌,写着“留神小绺(xiáo liu)”(小偷),谨防扒手”的字样。江湖人管他们小绺这行人调(diào)侃儿叫“老荣”(小偷),又叫“镊子把(bǎ)”。



小绺头儿有明有暗。譬如,北平这个地方,军警林立,小绺头儿是暗中潜伏的,绝不敢明露。他们又是一种流动集合的,没有准住处。在外码头的小绺头儿全是明的,若向官人打听,他们该管的地方一共有几个小绺头儿,姓什么叫什么,住于何处,都能知道的。那明着的小绺头儿得和老柴(官人)联络。如若有不听头儿调动的小绺儿,当头儿的向老柴们说句话,就能把他捕了去,责打一顿,给关起来。临放出来的时候,也得先向小绺头儿央求好喽,然后才往外放呢!放出来之后,这小绺除非远走高飞,若是不走啊,还得服从当地头儿呀。譬如,甲地的小绺,若是不愿意在甲地了,到了乙地不能去偷窃,得先在乙地见好了乙地的头儿,然后才能出来到人群里偷窃。设若来到乙地私自偷窃,不先见他们的头儿,叫乙地的头儿知道了,向老柴们暗一指,就给捕了去,先打后关。到了各省市码头商埠这已成定例了。还有些个小绺架着“海(hāi)冷”的。什么叫“海冷”呢?江湖人管当大兵的丘八(大兵)爷调侃儿叫海冷。小绺们和他们狐假虎威请出来的军人在一起,假如丢东西的人“醒了攒(cuán)(明白过来了)”,有军人保护他们,临时不能挨揍,不等丢东西的人找来官人,他们就扯活(chě huo)了(跑了)。他们架军人就得叫军人吃“摽(biào)杵儿”(即是分别人钱花)。还有老荣(小偷)“攒(cuán)冷”(入伍当兵)的,自己攒冷,每逢出来的时候,表面上瞧他军装整齐,好像是正式的军人,暗含着做活儿(去偷东西),你要说他是小绺,他先冲你瞪眼,一路大吵大唬。所以,攒冷的老荣有护身皮儿,实是不好惹的。敝人在外省还见过逛游艺场的人被小绺偷了东西,将小绺抓住了,过来几个丘八(大兵),将丢东西的人打得鼻青脸肿,打完了一散儿,真叫人有冤无处申诉去。

还有“攒(cuán)子钱”的小绺(xiáo liu)(小偷)。什么叫“攒子钱”的小绺?就是专在市场、庙会各玩艺儿场的人群中偷窃的小绺,江湖人调(diào)侃儿叫他们“攒子钱”。他们每逢要偷东西的时候都是两个人,甲将东西偷去,交到乙的手内,乙乘二仙传道得了道(得了皮夹子)的工夫,一转儿身往各处云游去了(可不是我这个云游客)。丢东西的人若是觉悟了,将甲小绺抓住,他能冲丢主瞪眼。常言道,“捉奸要双,捉贼要赃”,他身上搜不出赃物,就能愣装好人。攒子钱的小绺(指专在市场、庙会各玩艺儿场的人群中偷窃的小绺)也有不同,他们的能耐分为两种技能:一种叫糙活;一种叫细活。做细活的能偷阔人。第一得有穿着。衣服阔绰,能挨着阔人不叫有钱的生疑。第二得窃术高超,手要敏捷。要偷的时候先瞧了道儿,只凭走个对脸儿,微一沾身,财物便能到手,手眼心三快,令人来不及思议。至于往集场、庙会、杂技场儿等处绺窃的,真有挤挤蹭蹭偷个几十分钟才到手的,偷着的差不多是破皮包一个、当票二张、三角毛票、十几吊铜元而已。这种攒子钱的老荣(小偷),毛手毛脚,两眼乱瞧,遇见机灵人,没等沾身就明白了。甚至于没偷着东西,被人将手攥住,还能叫人“折(shé)鞭”(江湖人管被人大打特打调[diào]侃儿叫折鞭)一阵。窃术不精的,只可在人群里乱挤,偷那穷人。手里活糙的也难偷阔人。

在火车上绺窃的贼叫吃“轮子钱”,又叫吃“飞轮”的。窃术也分糙细,手术高的能掏小皮包、金表、钻石等等高贵的物品,只要偷到手内,东西不大,“护托”、“过托”(“护托”即是不叫外人瞧见怎么偷的,往自己身上怎么藏的;“过托”是甲偷到手内的时候又转给乙的手内,调侃儿叫过托)都容易。若是没有能耐的轮子钱,窃术不精,不是扛人家的行李卷儿,就是偷人家的柳条包,拿着又费力气,东西大了,又沉又笨,护托也难,过托也难。轮子钱的老荣,手术不高的也就是偷平常人。阔人出门,除了身上带着东西之外,向来不带笨物件,即或有笨重东西,也不自己携带,花不了几个钱,由火车上行李车给代运的。他们穿着平常,技术不妙,也难挨近阔人,也难偷窃阔人。

这些年社会里人士都要练习交际,有一种“朋友钱”的小绺(xiáo liu)(小偷),专在交际场所活动,只要和他点头说话,他就能迈步伸脚,认为萍水相逢的朋友。谁要脑筋不清楚,把他当作好朋友,这种小绺不熟假充熟,伸手偷东西。你要看见他拿东西的时候,他有措词,说和你闹着玩呢!如若偷的时候没有看见哪,那东西归了他啦。这些年,朋友钱小绺还有不少“果食(shi)码子”(即是妇人)与“姜斗(jiàng dǒu)”(即是大姑娘),这种女“朋友钱”出入娱乐场所,假充阔人的小姐、姨太太,他们的手段也好,最有能耐的能够两吃,又是“朋友钱”,又是“高买”(高级小偷)。

若是站在人的身前,倒背手儿偷身后边人的东西,这种技能小绺(xiáo liu)(小偷)们称为苏秦背剑。



北平这个地方向称首善之区,这里的老柴(官人)向不吃老荣(小偷)的摽(biào)杵(此处是官人不吃小偷的钱),并且不和老荣联络的。阅者若不相信,敝人列举一事便可证明。老荣这行里有最忠厚、最有名的小绺叫“于黑”,他的能耐比一般小绺都高明,人长得也漂亮,绝不像个偷东西的小绺。衣服阔绰,谈吐文雅。他是专在京、沪、津、汉等地吃飞轮子(火车),小的十元八元他不偷,哪回要偷也是成千论百,几十元真不放在眼内。他们老荣的同行人到了冬天混不上棉衣裳,或者有了疾病无钱医治,都去找他。别的阔小绺偷了大款,只顾自己嫖赌,哪管别人无衣无食呀,有人向他们告帮求助,也是枉费唇舌,惹他白眼相加而已。惟有于黑这人,轻财重义,凡是同行的有困难的事儿投着他,他一定周济的。社会里耍人儿(花言巧语支使别人)的人们,凡有为难时候,不论认识不认识,交情深浅,只要找他去,准能倾囊而赠,仗义疏财是他的天性。虽然常益于人,却能有利于己。他每逢遭了官司的时候,探监看望他的人络绎不绝,送衣食,送银钱,还有给他运动的,不知者都说于黑手眼通天,究其实也是他个人维持的。他是小绺,吃飞轮子,当攒(cuán)子钱(各玩艺儿场、庙会等处的小偷),他都干过,就是没做(偷)过朋友钱的。据一般老荣们所谈,于黑的窃术最为拿手的别人学不了的是“苏秦背剑”(当小绺的人,每逢偷东西,都是在人的右边挨着。因为我国的衣服,长大衣裳纽扣儿都是在右边,小绺挨着人的右边解纽扣,入托儿窃取财物。若是站在人的身前,倒背手儿偷身后边人的东西,这种技能小绺们称为苏秦背剑)。有一次于黑到上海,将下轮船的时候,有个小绺(xiáo liu)(小偷)不认识于黑,挨近他的身右,要想偷他,没有入托(偷着东西),被于黑一拧身使了个苏秦背剑,将他的金表窃到手内。这个小绺“折(shé)了托儿”(东西丢了调[diào]侃儿叫折了托儿),还不甘心,见了小绺就问,谁和他开玩笑,将他的“转(zhuàn)枝子”(管钟表调侃儿叫转枝子)给偷去啦。有位明白的小绺说,你别是遇见天津的于黑啦,他惯使苏秦背剑。这个折了托的小绺恍然大悟说:“不错,我没荣(偷)了他,被他把我荣了。”由此一事,足可证明于黑是个有万儿(名气)的老荣了。于黑走遍天下,他从来没到过北平。想这故都有的是“火码子”(阔人调侃儿叫火码子),他便由津到平。这里又没有小绺的头儿,无须乎见过同道,就可以在北平度其窃绺的生活。他穿得阔绰,住的是大旅馆,又不天天偷窃,老柴(官人)家绝不能注意。不料他到北平未久,一个星期之内就被捕了。于黑来过北平两次,遭了两回官司。他在津时曾向人言,北平那个地方,吃喝逛之事很可他的心意,出去做活(偷东西)也很容易。只是北平的官人不吃我们老荣的摽(biào)杵(北京的官人不花贼的钱),可惜北平那个穴(xué)眼(一个挣钱的地方),官人办案手段敏捷,毫不客气,是不叫我去的。天地之大,北平不能存身,我只好不去。由于黑这种向别人谈话的口气就可以证明,北平的老柴家是不吃老荣的摽杵的,是不联络老荣的。在外省市商埠码头丢了东西,在三天之内找着小绺头儿,或是有势力的向官人追究,准能把东西找回来。到了北平则不然了。

敝人在从前很纳闷,凭什么很好的人不做正事,不学点手艺,他们老荣们愿意当小绺(xiáo liu)(小偷),虽是手底下做活好的能赚个吃喝嫖赌抽,眼前快乐,若是遭了官司有多么可怕呀!俗语说,“屈死不告状,穷死不做贼”,官司不是好打的。“净见贼吃饭,谁见贼挨打”,干什么不是吃两顿饭哪!有深知他们内幕的人告诉我说,小绺这行儿,有师傅有徒弟。我曾问过:“好好的人谁肯拜师学当小绺呀?”这位深知内幕的某君先叹息了一声,然后才告诉敝人:他们小绺这行人,师收徒不是徒弟找师傅,是师傅找徒弟。凡是小孩到了十三四岁、十五六岁的时候,当家长的教育子弟最难,小孩的知识最幼稚,大人不栽培,做父母的对不住儿女,若是教育他们,栽培他们,还要得法,不可过严,不可不严,不能不慈,不能过于溺爱,得督促小孩学能耐,还得拢住小孩的心。倘若不得法,小孩子受挤兑,他急了只有偷偷地远远一跑。他们老荣(小偷)若是要收徒弟,就专在热闹场儿的地方寻找这路偷跑的小孩,带到店里住着,足吃足喝,天天带出去足逛。小孩们到了他们手里,如同上了贼船一样,休想下得来!抽鸦片、扎吗啡都能戒除了,惟有当小绺的,洗手不干改了行的,实在是少啊。可是小绺的徒弟,也不写字,也没保人,也没有学多少年的期限,只要学得会偷了,不良的印象越来越深,懂得离开他师傅啦,翅膀儿硬了,就偷着一跑儿,躲开他师傅完事。敝人将这种情形写出来,不是给社会的人士添不良的影响,是叫一般有了儿女做家长的,栽培教育都要得法,不可过于放纵,不可过于严厉,否则孩子跑喽,被他们老荣拢了去呀,那可怎么好!还有,手艺作坊掌柜的,商号的经理,对于学徒的小孩,非得恩威并行才能教出好徒弟,有利于人,也利于己。如若有威无恩,将徒弟挤兑跑,徒弟入了邪途,于个人的道德上也是有亏呀!这些话是我一份爱护一些知识薄弱的小孩之意,阅者可别错想我是刻薄呀。





晃(huàng)条的与扫条的


赌博之道,无论是麻雀(麻将牌)、摇摊(玩骰子要钱的)、抽签、押宝,男妇老幼无有不好的,即或有不好喜的也是百里挑一。久赌无胜家,久赌必腥(假)。好喜耍钱的人有了经验,是讲究能收能放:赌到气微的时候,要押宝少押钱,慢慢地养气,养过气顺的时候,多押钱,冲冲地赢个三宝五宝的,赢了钱就走,就叫能收;能放,有一种嗜赌如命的人,到了赌场里,有多少钱非得输个干净,他才不来了呢,赢了钱也不走,非得把赢的钱再输回去,把原本也饶上,方才算完,那叫“淫赌”,有多少家产输尽了算完。久赌无胜家,也是一句赌场内最有经验的话呀。久赌则腥,就是亲手足,天天在一处赌钱,耍长了也要闹鬼儿使个腥活儿。

我在天津河东住过,每天出来逛逛大街小巷,是卖吃食的买卖都有个签筒子摇晃摇晃。有些个小孩子,家长给几个大铜子当做饽饽点心钱,他们不买吃的,把钱都抽了签子,赢了多吃,输了不吃,山后的蝎子——饿着(蜇)。那种习惯是养成了的。有些卖吃食的小贩,他们成天价携着筐子蹲签子,干长了就要闹鬼儿。有一种签筒子是双层底儿,在两层底的中间有根线儿,能将签子的根底下用线拴住。竹筒又长,签子又短又细,有人抽的时候,抽不着对大天,对大人,对二板儿,抽十回不赢一回。他们使的这种签筒子叫做“锁线儿”。还有往签子底下灌铅条的。把三十二根签子里的天、地、人等签子,由根底下钻空了,把铅条装在里边,也是签子短筒子长,有人要抽,也是抽不着好的,管这灌铅的签子,他们叫“十三太保”。卖吃食带签子,调(diào)侃儿叫“晃(huàng)条”的,有些个卖茶壶、茶碗的小贩们,带着签子专串娼寮的,做那种买卖实在不容易。

有一种吃腥(假)的人,调侃儿叫“扫条子”的。他们闹鬼儿调侃儿叫“托门”(假的步骤)。就以我所知道的,他们有十三道托门。他们扫条子的把手底下的活儿练习好了,三五成群地出来找饭落儿(找饭辙,就是出来蒙钱)。他们专会“把(bǎ)点儿”(看谁可以蒙),要是瞧着哪个做小买卖的精明强干,是不受他们欺的,他们也不找麻烦;如若遇见新上跳板(刚入这一行)的小贩,人再老实,立刻就给扫个一干二净。如若遇见晃条的使的签子是圆头的,他先抽一大枚的,抽个几把,赢不赢都认。每逢抽出大天、大人、地幺,假装摸点儿,背过手去将那好赢的天、地、人签子的圆头上,用手指甲盖儿掐成小月牙的印儿。管掐印的时候叫“上托”,管掐上月牙印儿叫“月牙顶”。把托上好了就抽一毛钱一把的,手法敏捷,专抽那有月牙的,三五把就能把一筐子瓷器扫空了,拿着一走,再往外一挑(即是卖了),不到数小时工夫就能挣个两三元钱。有些做买卖的小贩,知道扫条子的惯使月牙顶,他们为防止月牙顶,使签子要用尖头儿的极细的,叫扫条子的挂不上托儿。那扫条子的人们更精明,到了抽签的时候,手中藏着几个草节,又细又短,抽出签子来,背着手假装摸点,把草节套在签子底下,也叫“上托”。把能赢的签子上好了托啦,三毛一把,五毛一把,抽起活来,右手抽的时候,手指灵敏,眼睛要把(盯着)托,瞧哪根签子高出少许来抽哪几根。左手得会护托(即是用左手遮挡那晃条的眼睛,签子抽出时护住签子根底下的草节儿)。这种草节儿叫做“高脚腿”,用上托,几把就能把瓷器筐子赢尽了。有些个做买卖的小贩懂得扫条子的有月牙顶、高脚腿儿,他们留神不叫他上托。扫条子的遇见小贩,他们能使“碱托”,预先用小棉花团儿沾碱水,把棉花团藏在手内。抽签的时候,把签子抽出来,假装背过手去在身后摸点儿,把大天、虎头、幺六儿三根签子,用棉花团的碱水抹在签子上,那签是竹子做的,用碱水一抹就变成黄颜色。用棉花碱水染签子也叫“上托”。他们把托上好了,三毛一把,五毛一把,抽出活来就是那上了托的三根签子,几把就能把一筐瓷器赢尽了。

这些托门(假的步骤)都是很受使的,学之也易,使之也易。稍难者为“过托”,譬如由筒子内抽出的三根签子,一根是幺五儿,一根是地幺,一根是幺六儿。论理说不能赢,惟有到了扫(收)条子人手内,他能闹鬼儿,使个障眼法,赢了蹲签做小买卖的。他使用过托之法,攥住三根签,先叫蹲签的人瞧那根幺五儿,看完交在右手,那左手攥着地幺、幺六儿,他把地幺用右手往外一抽,令蹲签的人瞧着说:“这是地幺,再来一个地幺,是五个幺,可就赢了。”他右手攥着签的上头,左手还攥着下头儿,猛使劲一抽,把幺六儿换了去,左手只攥那地幺不撒手,把右手的两根签子装在了筒子里,向蹲签人说:“这根签子要是地幺可赢吧?”蹲签人说:“要是地幺就赢。”他把左手一张,叫蹲签的人自己瞧,蹲签的人看是地幺,遂道:“你赢了。”这就是过托的使用法。比这过托还难的是“晃(huàng)托”,那晃托得眼神好,手指灵敏,不往签子上挂托,只用右手在他签子筒内溜签子,把签子溜的上半截窜在筒外边,两只眼睛就能看见签上的点儿。瞧出好的能赢的就记住了,任签子在筒内乱蹦,他眼睛也记住了应抽哪几根,手眼相应,抽出三根来,就配上点儿赢东西。晃托儿是最难学的,最难用的。我在津埠之日,常见有新出手扫条子的人,使活儿没弄利落,叫晃(huàng)条的(卖东西带签子,调[diào]侃儿叫晃条的)把(看)出来,翻了脸,“折(shé)鞭”(被人大打特打)一通。

凡是扫条的人们十有八九都是身体雄壮,到了鼓盘儿(鼓盘儿即是翻脸)的时候,仗着是膂力好,和晃条的“鞭托”(管打架斗殴调侃儿叫鞭托),还有些个扫条子的人同着丘八(大兵)爷们在一起,调侃儿叫“架海(hāi)冷”(海冷即是丘八)。在民国五、六、七年间,天津的三不管(天津市南市的一个露天市场)、北开西头等还有杂巴地哪,晃条的、扫条的终日盘踞在这一带,吵闹不休。这些年地方当局整顿市容,把这些个好打闹吵的营生严加取缔。到如今,天津的街市上见不着抽签赌钱的啦。虽有蹲签的也都是卖吃食物的了。“奸情出人命,赌博出贼情”,实是不假呀。对于戒赌事儿,敝人是极力赞成。





挑(tiǎo)青子生意之内幕


在从前,有一种逢集赶集,逢庙赶庙卖剃头刀子的生意,江湖人管他那行儿调侃儿叫“挑青子”的。

做这种生意的也是一种“笨头”(江湖人管做买卖的资本调侃儿叫笨头)搁(gé)念(老合,江湖艺人),他们背个包儿,有个几把刀子,打走马穴(xué,走一处,不能长占,总是换地方挣钱,江湖人叫走马穴)儿,顶个“凑子”(集市)就能挣钱。到了集上,找个过路口儿,将包儿一放,左手拿着一缕儿“苗西子”(江湖人管头发调侃儿叫苗西子),右手拿着一把剃头刀子,就能圆粘(nián)子(招徕观众)。他说:“我是刀剪铺子耍手艺的,从幼小儿学了这份打刀子的手艺。总给人家耍手艺,挣不了多少钱,我要自己做个买卖,因为本钱小,开不了铺子,耳挖勺里弄芝麻——小鼓捣油儿。自己的手艺在家里打了几十把刀子,来到市上卖。”他嘴里叨叨念念,瞧着人们都围满了。他说:“真金不怕火炼,好货不怕试验。咱们这刀子受使不受使,咱们当面试验试验。”说着他把左手的那缕头发一攥,叫人瞧着足有四十多根,用剃头刀的刃儿对着那缕头发,用嘴一吹气,那缕儿头发就全都断了。围着的人们瞧着他那刀刃如同迎风斩草似的,谁不爱呀?剃头的手艺人使用的刀子虽快,到了剃头的时候,还得用热水把头发洗好喽,抹上洋胰子才能剃哪。他这刀能将一缕干头发一吹就断,较比剃头棚儿手艺人用的刀子还好使哪,谁不买呀?他把刀子试验得人人都要买啦,他又自言自语地说道:“这刀子能把头发割断,大概许是净能动软的,不能动硬的,咱们动回硬的叫众位看看。”说着话他一伸手,从包儿内取出一根锈铁棍儿,有核桃粗细,他往那小凳上一坐,把铁棍用腿夹住了,拿着那剃头刀儿往铁棍上愣刮,哧哧的直响,刮得往下掉铁末子,刮完了他举着刀儿说:“众位瞧瞧。”围着的人们一看,那刀的刃并没有受伤。他说:“咱们这刀是材料地道,手艺降人,才能那样。众位要买这样的刀子,到了刀剪铺得卖你三毛钱一把,我这是头趟来赶咱们这集,张天师卖眼药——舍手传名,名不去,利不来;小不去,大不来。这趟我是不赚钱,只卖个本儿,把手工白饶上,卖两毛钱一把。那位说我全要了,都要我可不卖,我就卖十把刀子,过了十把刀之外,我还卖三毛钱一把。”说到这里把脚一跺道:“我今天豁出去赔本了,卖一毛钱一把!有要的伸手。”他说到这里,便有人买,十把刀眨眼卖净了,一块大洋到手了。赶一个集就卖这么三四回,几块大洋到了手,除去本钱能赚一多半儿。

在从前,我看他们当面试验,东西好,价钱便宜,要买他一把哪!有个江湖人对我说过,他们卖的刀子是“里腥啃(lǐ xing kèn)儿”(江湖人管假东西叫里腥啃儿)。我说:“他那刀子能够吹毛就断,刮铁棍,怎么会是里腥啃哪?”他说:“卖刀的能够吹毛断发,刮铁棍,那是他们练好的‘托门(假的步骤)’,要是到了别人手里就不能刮铁棍了,一刮刀就毁坏了,断毛断发,净吹就不断了。他们把‘托门’练好了,先说个大价钱,后来往下落价儿,由两毛一直落到一毛钱,调(diào)侃儿叫‘海(hāi)开减卖’,‘催啃(kèn)的包口儿(挣钱说的话)’。做这种生意的分为三样儿:一种是顶凑子(赶集),使托门儿,海开减价,挑(tiǎo)的是里腥啃;一种是用尖局(真的)的啃(kèn)儿,走常穴(xué)(在长期地点做买卖)的。什么叫尖局的啃哪?就是真正的地道的好东西,要是摆个摊子等主道候客,那可卖不动,赶个集走几十里路也就能卖三两把,不用说赚钱,就是本钱也卖不出来。若是逢集便到,挑尖局的东西,走常穴、卖出主顾来,细水长流,也能获利。不过是慢点,利钱又薄,日子又长,那样做法也是百里挑一呀。还有一种假装剃头的手艺人,预备一块磨刀布,一个刷子,几把刀子,在各集市上摆摊出卖。有些人疑惑他那刀子一定好使,看他那样子一定是剃头的手艺人,要卖了家伙改行似的,就有人买他那刀子。可是他将那刀子故意弄成了旧的才能成哪!”在早年社会的风气不开,都不讲求卫生,剃头刮脸都是找个剃头棚儿,那剃头棚儿都是破烂不堪。社会人士不尚奢华,都是克勤克俭,花个几吊钱买把剃头刀子,又刮脸又剃头,也是很经济的办法。那时候各大都市、各大商埠都有做挑(tiǎo)青子(卖剃头刀子的)的生意的。到了如今,无论穷富都讲究修饰外表,剃头匠改为理发师(教给我念书的老师也改为教员了),剃头棚改为理发馆。社会的人士都日趋浮华,谁还花钱买把剃头刀儿自己剃头刮脸哪!卖刀子的生意可就不在都市省城、商埠码头卖了,都改了路子到乡间去了。如今挑青子的买卖都做“科郎(kē lang)”(江湖人管农人、老乡们调[diào]侃儿叫科郎)去了。再过些年,挑青子的生意恐怕就要天然地淘汰了。





磨(mó)杵的生意


江湖人管到乡下串村庄镇去做生意,调侃儿叫“磨杵”。磨杵的买卖也有好几十样,先由那前些年摇铃卖药的说吧。他们都有个皮包,内里装些个瓶子、罐子,装着丸散膏丹,有旧式治外科疮症刀剪家具,有扎针的针包儿,把这些个东西装全了,说行话叫“啃包(kèn bāo)”。左手提着啃包,右手拿着“虎撑”(管摇的那串铃调侃儿叫虎撑),走进了乡村的胡同里,哗啷哗啷摇起串铃,乡间男妇老幼听见这声儿,就知道治病的先生来了,有病人的家便请他进去。他一入“窑儿”(管进到病人的家内叫入窑儿),得先把(bǎ)簧儿(看出病人的底细)。他们把簧也是按着那大方脉的医生“入嘿”(江湖人管请大夫治病叫搬嘿,管大夫到病人家叫入嘿)一样,使那“望闻问切”的诀窍。譬如,一进屋,六月天气,正是暑期时,见病人穿着棉套裤,不用问他什么病,一望而知是得了寒腿病了;若是病人脸上涂着黄土泥,便知得了偏头痛、牙痛的病啦;若是病人趴在炕上不住地哼哼,手捂着肚子,一望而知是得了肚腹痛的病啦。他们到了病人的屋内用眼把簧,把病人的病猜出个八九成啦,落座之后先“粘弦(nián xián)”(管给病人诊脉调侃儿叫粘弦),最叫人佩服的是他们一粘弦,准能把病人所得的病是怎么得的,得的是什么病,全都说得分毫不差,叫病人信服他的脉气好。据江湖人说,给病人评脉的时候,能诊出得的什么病来,要说对了,那种方法叫“粘啃条子”,有了病叫“有粘啃”。他们拿着串铃卖药的,拜师入门,头行儿就学粘啃条子,男子有十几样条子,女人有十几样条子,老年人有十几样条子,小孩有十几样条子。那条子分为:咳嗽条子、痨病条子、筋骨麻木条子、血分不调条子,合计起来总有百十多个吧。他要是诊脉的时候把病人的病原说对了,先不给治病,先要“水火簧”(问出病人有钱没钱)儿。譬如他问:“你这病请医生治过没有?”病人说:“嗐,先生,我都治腻了。”他听后就知道这家是有钱的,要没钱哪能成天价请大夫吃药呢?请个大夫,出诊费,连抓药没个两三元不成,他要是治腻了,几十元钱就花出去了。别看他治腻了,还能挣他的大钱。社会里有两句牢不可破的话,是:“穷不离卦摊,富不离药锅。”人有钱身体就娇贵,人要穷了,不用说花钱请大夫抓药治病,连吃饭的钱还没有哪,有了病,就算是认了命啦,该活死不了,该死活不了。譬如,问那病人:“你这病治过没有?”病人说:“我疼了半个月啦,还没治过一回哪。”那卖药的先生听着就凉啦。这人但凡有钱绝不能半个多月不治病,这个买卖撑死了也就挣上两毛洋。

他叫病人瞧那罐子,病人往罐里一看,只见罐内又黑又紫,粘粘糊糊的,有半罐子脓似的。



凡是做这种生意的,一给病人“粘弦”(诊脉),就得先要水火簧儿。若是真穷,也就不用多挣了。若是有钱的人家,不多挣钱又挣谁的哪?那病人虽说他治腻了,卖药的先生更会说:“弹打无命鸟,病治有缘人。该着一百天的灾难,九十九天也好不了,若是该着你消灾,该着我露脸,一治就好。”病人听他说的这几句话,觉得很为有理,就叫他治治吧。他们磨(mó)杵(江湖人管到乡下串村庄镇去做生意的调[diào]侃儿叫磨杵)的先生也有几道“样色(yàng shǎi)”(能挣下钱的物件)。譬如病人得的是肚腹疼痛,他就先使“插末(chā mòr)”,他们管扎针调侃儿叫做“使插末”,用针往病人身上一扎,从包内取出一个罐子来,他把针拔下来,用火纸点着往罐内一扔,把罐子往针眼上一扣。他向病人说:“扎针是按着穴道,有四阴针、四阳针、四大总针、八法神针、九转还阳针、马丹阳十二针、鬼门十三针。何谓四大总针哪?《针灸大成》的书上说得是:肚腹童流、腰背委中求、头顶刺列缺、面口合谷收。针针针,不差半毫分,能用十服药都不动一分针。扎一针胜似吃十服药。扎针拔罐子,病好一半子。”他说这些话,病人也是爱听。少时间他用手把罐子起下来,猛一翻个儿,叫罐子口朝上,他叫病人瞧那罐子。病人往罐里一看,只见罐内又黑又紫,粘粘糊糊的,有半罐子脓似的。他向病人说:“这一针扎在了病上,把你这病拔出一多半来,今天晚上再吃服药,回头我再给你贴帖膏药,明天就好啦,复旧如初。”不用说病人听着高兴,阖家老幼听着都是痛快的。于是他叫把罐内东西倒在院内埋了。本家是当面瞧他把病治出来,焉能不佩服他呀?他由包内取出一帖膏药,贴在针眼上,又取出一包面子药说:“你们今天晚上叫病人吃下去,夜里拉出几泡屎来就好啦。”病人说:“先生,我要好喽,忘不了先生的好处。给先生多少钱哪?”这先生说:“若是按规矩,扎针就得一块钱,这帖膏药一元二,面子药是八毛钱,一共三元钱。得啦!针白扎了,药钱我取个本吧。你们给一块五毛钱就行啦。”本家的人见针是扎了,膏药也贴上了,好好地给人家块半大洋吧。先生治下“柳丁中的拘迷(jū mi)把(bǎ)”(即是块半钱),收拾包儿走了。到了晚上把药叫病人吃下,本家的人都要瞧拉出来的是什么,谁想肚子咕噜咕噜直响,整整地响了一宵,一泡屎也没拉,直到第二天早上肚子里还是直响。阖家老幼都纳闷儿,不知是怎么回事,你一言,他一语,其说不一。到了吃完早饭的时候,就听见门外哗啷啷串铃响,卖药的先生又来了。本家赶紧就请这位先生,向他问问吧,究竟是怎么回事?原来这卖药先生头天挣了一元五毛,那是头道杵,第二天他又挣二道杵来了。他还是有把柄,能料着本家准得请他的,二道杵如同在手里攥着一样。他用罐子从针眼拔出来的那东西,是和戏法一样,原来在那罐子里就有那东西,这东西是粉子和颜色弄的,调(diào)侃儿管这道样色(yàng shǎi)(能挣下钱的物价)叫“大卯子”。病人吃的那包面子药,到肚子里咕噜咕噜直响,他们那面子药是×皮子做的,不拘谁吃下去,肚子里净响。他们江湖人管那法子叫“张手雷”。第二天他提溜着啃包(kèn bāo)(江湖人管做生意用的全份家具,行话叫啃包),摇着他那虎撑(管摇的那串铃调侃叫虎撑)儿,又到这病人的门前,本家出来人,赶紧把他请到屋内,向他问道:“先生,不是吃了你的面子药能把病打下来吗?怎么吃下这药去净响,没把病打下来呢?”先生说:“哎呀!这病人的病太重了,凭我那药的力量,才将把病问动,实在够瞧的!你还得来服双加料的吃吃。”病人就说:“我来服双加料的吃吧!”先生说:“这双加料的药得两元多钱哪!”本家好说歹说给了两元钱,给了一包丸药,说:“吃下这服药,准能把病打下来,如若打不下病,我把钱还退给你们。”他拿着两元钱走了,“月丁拘迷(jū mi)把(bǎ)”(即是两元钱)到了手。他给下那包丸药,调(diào)侃儿叫“串子”,吃下去准能好了。

原来他们江湖卖药的有几样好药,能治十样病,吃下去准能治病。据我知道的共有四样:一叫“顶汉”,二叫“抗汉”,三叫“戳汉”,四叫“串子”。如病人咳嗽吃下他们那顶汉,就能顶住病不咳嗽了;如若病人筋骨疼痛,吃下他们那抗汉,就能不疼了;如若病人心口疼,吃下他们的戳汉,立刻心口不疼了;如若存了食水,肚腹疼痛,两脚发胀,吃下那串子去,就能把食水打下来,准能好得了病。据我同他们探讨,那四种药,是经过多少名人研究出来的。大方脉的医生向来胆小,不敢用。他们江湖人做这磨(mó)杵(江湖人管到乡下串村庄镇去做生意的调侃儿叫磨杵)的生意,降得住人,挣得了钱,就仗着那顶、抗、戳、串四样药品。最难学的是他们的针法,不论什么病,一扎立能见效。不过,近来这种磨杵的生意渐渐地消灭了,再过些年,这磨杵的买卖就无人做了。





大安把戏


今将“大安”把戏中黑幕贡献阅者,也公诸社会,免得贪便宜者上当。

在清末时代,鸦片输入中国,流毒社会,染受其毒的人,倾家荡产,人格扫尽。“抹海(mò hāi)草儿”(江湖人管抽大烟调侃儿叫抹海草儿,又叫啃海[kèn hāi]草儿)够多么可怕呀!鸦片之害尚未除尽,“插末(chā mòr)汉”(管吗啡调侃儿叫插末汉)又继续而来。吗啡之害,较比“抹海”还更厉害。如今又有“雪花汉”(管白面调侃儿叫做雪花汉,可不是洋白面。敝人所说的是“高射炮”,还是能冒烟不打飞机的)尤为可怕,这些个亡国灭种的东西,应当铲除吧。在铲除毒品的时代,生意人研究出一种投机的买卖来,撞骗商家。他们这种买卖江湖人叫做“大安”。

做这种生意者多至十数人,少者七八个人。大家集资配制一种××××戒烟药。药品放在盒内,印刷品类,那都是爱国爱民冠冕堂皇的宣言,把“啃(kèn)”“攥弄(zuàn nong)”(江湖人管制造物品调[diào]侃儿叫攥弄啃)得了,分为两班儿“开穴(xué)”(江湖人管旅行的话调侃儿叫开穴)。譬如十人吧,是四个人为“挑啃(tiǎo kèn)”(管卖东西的调侃儿叫挑啃)的,六个人当“托儿”(贴靴的人调侃儿叫做托儿,又叫敲托的)。他们这两班人,每至商埠码头、各大都市,分为两班住客店,“挑啃”的必须要住旅馆、饭店,为的是假充阔绰,施其店大欺客的伎俩。“托儿”们住在一个极便宜的店内,分途施其骗术。“挑啃”的人们临时叫辆汽车,将他们所售的药品装在车内,运至各药房各洋广货店门前,将汽车停住,“掌穴(xué)”(管首领调侃儿叫掌穴)的穿着一身西服,由汽车里出来,带着他的两个伙计,抱着几大盒戒烟药,走入商店。商店的铺伙不知道他们的来历,还以为来了阔主顾呢!先生、掌柜的都过来张罗,由掌穴的向商人摇唇鼓舌地下些说词,说他们是某省戒毒会的委员,制造了几种戒毒的药品,不论吗啡、白面、鸦片都能戒除的,这药品极有效验,奉他们会里的命令来到此地推销,将这些药品放在你们铺内寄卖,先放下货,容你们卖出去,然后再来取钱。“囊子点”(买卖商人叫囊子点)准能愿意坐收其利。有便宜的事商量办没有不成的,将寄卖药品的事议妥啦,掌穴的又带他的伙计往别处商议买卖去了。

在铲除毒品的时代,生意人研究出一种投机的买卖来,合伙造假药,撞骗商家钱财。他们这种买卖江湖人叫做“大安”。



他们走后,商店的先生、掌柜的,叫徒弟将招牌挂在了大门以外,过不了两三天,他们做“大安”(卖戒除毒品药的)的“托儿”,就由客栈里出来,到各商店假装买东西,购买戒烟药。就是商家有两家的戒烟药,他们也是指定了买×××戒烟药。数日之间,商店见有些零购的主顾,接连着不断地买这药品,测料着这药定有效验的,更是相信不疑。这天他们的托儿来至某家商店,问柜上有寄卖的什么药品没有?柜上一定说有,托儿说:“我买三百元的。”柜上的伙计问:“你要三百元的?这就要货可没有。你得明天来取。”托儿故意地思索思索,说:“我明天晚车往张家口去,是往回带,这药真有效验。明天我早上来取药,给你们留下四十元定钱行了吧?”柜上的伙计一定说行,托儿将大洋四十元留下而去。伙计和柜上主事人一商量,这号买卖有三成的利,买三百元的能赚九十元。赶紧命柜上跑外的伙计去到旅馆,取三百元钱的货物。跑外的伙计找到了旅馆,见了他们要三百元的戒烟药。掌穴(xué)(这一伙人的头儿)的人说,货没有啦。跑外伙计就得一愣,便问:“你们这货怎么没有了呢?”掌穴的必说:“卖得很快,销路很好,没想到卖得这么快,今天早晨将五千元的款已然寄回去了,大约着一个星期货能来到,等着货来了给你们送去。”跑外的伙计两只眼睛不闲着,看见他那屋内放着有个几百元的货,便用手指着那货问他:“这不是有货吗?”掌穴的人说:“那货是有了主的了,是××商行留下的,昨天他们柜上给了二百一十元现款。今早晨凑了五千元寄回去了。”跑外的瞧着这货眼馋。他们做“大安”的伙伴,向他们掌穴的说:“要不,将这货倒给他,匀给他得啦!”“掌穴”的假装怒容道:“把货匀出去,回头××商行要来取货呢?告诉人家没有货了成吗?怎么接人家的定钱来的?没有货把钱退给人家,咱们又把款汇了走啦。这事不好办。”跑外的伙计是能说会道机灵的人,趁着这时候还央求掌穴的:“你们把货匀给我们,你要现钱我回柜给你们取钱去。有二百一十块钱退给人家还不成吗?”掌穴的还故作为难的意思,跑外的伙计又央求他几句,掌穴的才应允了。跑外的伙计便欢天喜地地回柜去取钱,到了柜上把这份意思说明,管账的先生立刻就取出二百一十元来,交给跑外的伙计赶紧去取货。跑外的伙计又到旅馆内,见了他们掌穴的,将二百一十元现款放下。还说了些个感情的门面话,欢欢喜喜地将货拿着回归本柜。到了柜上将货物放好啦,净等着明天来取货的了。及至次日由晨至晚,也不见客人前来取货,到了这时候还不“醒攒(cuán)”(觉悟了叫醒攒)哪!因为客人买东西先留下了定钱,有好几十元钱存在柜上还有错吗?直到五六天后,明白受了骗啦,再叫跑外的伙计去找他们,旅馆的茶房说声:“早走了好几天了!”跑外的伙计回到柜上说明了,大家仔细地研究,连从前的赚利与定钱数十元,合计起来,至少损失百五十元。一家百五十元,要有个数十家呢?数千元现款被他们骗到手内,远远的“开穴(xué)”(去外地),“急流扯活(chě huo)”(跑)了。

这种做“大安”(卖戒除毒品药的)的骗子手,干了好些年,骗了一处又一处,始终还没听说在哪里“朝了翅子”(江湖人管打官司调[diào]侃儿叫朝了翅子。翅子即官儿,朝是见官。他们不打官司。见官干吗呀)呢!现在北平市自从颁布禁毒条例以来,×××的买卖都查了封啦,“断海(hāi)的汉儿”(戒烟药)已然禁止喽,干这“大安”生意的人是不能来了,北平这个地方暂时是没有这类事了。





老月的骗局内幕


“老月”是耍腥(假的)赌的。他们若要设赌吃人,一个人可耍不了腥儿,至少也得两个人。老月们的组织也是不同,或三或五,或十数人,是没有一定的。可是他们的局面大的能骗人几万几千的,局面小的仅能骗人几百几十,“水了穴”(即是混穷了)的老月也能骗人个几元几毛。他们同是吃“空(kòng)子”(外行),方法各有不同,最有能耐的老月,吃完了秧子(被骗的人),能够叫秧子醒不了腔(不醒悟),他还能和秧子在一处儿吃喝玩乐。有那没有本领的老月,设的局儿不完善,叫秧子醒了腔儿,轻了是断了交情面子,谁不理谁,重了不是“朝了翅子”(打了官司),就是“折(shé)鞭”(挨了打)。

“老月”们骗阔少们的钱财,主要两个手段:一是女色,二是设赌局。



有一种最高的老月(设赌骗钱的),家里住的宅子也是几十间房子,电灯、电话,热天电扇,冬天暖气管子、洋炉子,屋中的摆设、桌椅家具、床帐、古玩瓷器、名人字画,叫谁瞧着也值个几万元。厨子、老妈、听差的、门房、打杂、开汽车的,男女仆人也是十数个。本家的主人,男人都是衣服阔绰,人物漂亮,谈吐文雅;女人都长得姿容秀丽,年老的得像个太夫人,中年的得大方不拘,年少的得像大家闺秀。这个佯(yáng)(假的)的局式,若把秧子弄到他家,那秧子绝想不到这家是老月。他们还都善于交际,每日在公园、饭店、市场、娱乐处所出入挥霍,叫人看不透他是干吗的。他们往家里带人,调(diào)侃儿叫“往窑里跨点儿”,第一得把出点头儿水火簧来(即是瞧出秧子[被骗的人]是穷秧子还是阔秧子),投其所好,施用手段。如若秧子好近女色,就把秧子弄到窑内,用女子来骗他的金钱。如若秧子不近女色,就用男子使腥儿(假的)骗他的金钱。譬如遇见个阔少爷,他家里有几十万的财产,为人精明强干,对于社会里蒙人瞧人的事儿,他懂得些个,若是约他耍钱他不干,用女人笼络他不上套儿。老月们就用贴身靠儿的手段和他交朋友,在交际中一切吃喝花费,不叫他给,叫他白吃白喝,施以小惠。他爱贪小便宜,就如同用金钩钓鲤鱼一样和他联络些日子,使他不疑了,然后把他带到家中,叫他看热闹,瞧耍钱的人们输赢钱之大,使他动心,以便上套。

曾记在民初五六年间,有北平某世家子名叫阿林太者,他家广有恒产;为人机警,颇喜交往官场中人。一日在某戏院看剧,得知一陆某,二人交为至友。据陆某所言,为江南人,住于同乡某司令宅中。一日陆某同阿林太至某司令宅中,见客厅中有十数人呼卢喝雉,大肆赌博。阿林太与陆某围观胜负,见有一少年,人物俊雅,衣服阔绰,每赌必输,三小时之内竟输去万元有余。阿林太触目惊心,见此巨赌不敢问津。每三二日陆某便约其观赌,常见该少年输负巨赌,少则数千,多则数万。阿林太问陆某:“少年为谁?何有巨款常输不惧?”陆某说:“此吾同乡唐富绅之子,其家资产约有数千万,似此赌博,并不为多,每年挥霍数十万。与其赌博者皆为老月(设赌骗钱的),他不明腥(假的)赌之弊,故每赌必输。”阿林太问陆某:“你为何不吃他一水呢?”陆某皱眉道:“惜我无款。我与少年同乡,彼常命我引他赌钱,我若有本钱,数万之款早到囊中了。”阿林太道:“吾若筹出本钱,你能赢他吗?”陆某说:“那极容易,你明日若能携来巨款,我便能赢他,如若得款,你七我三,三七分之。”阿林太说:“万儿八千款我能筹出,但是你有何法可以赢钱呢?”陆某说:“有个主意。明日赌时,你可用丹凤火柴盒当作宝盒,以四张牌九,地幺、二板、长三、大四,分为幺、二、三、四,你做宝,我叫唐家少爷押,你如往火柴盒内装张地幺,可将火柴盒的凤头冲我,我劝他押四。你若装张二板,把凤尾冲我,我劝他押幺。你若装张长三,可将火柴盒反用,将丹字冲我,我叫他押四。你若装张大四,可将凤字冲我,我叫他押幺。如若那样,两日工夫,就能赢他几万。”阿林太喜悦非常,二人商议妥当,照计而行。次日他将万元巨钞装入提包,带牌九四张,火柴盒一个,至某宅求寻陆某,先将巨款叫陆某瞧看,然后等那唐少爷。掌灯后,唐少爷果至,由陆某介绍给阿林太,然后布置赌案。阿林太就将地幺装入火柴盒内,将凤头冲外,陆某劝唐少爷押四,唐押款数百元,开盒视之系地幺一张,数百元钞票为阿所得。如是赌至十数次,千数元巨钞已为阿林太所得。他这次将长三装入盒内,放在案中,将丹凤的丹字冲外,陆某知系长三,劝唐少爷押四。唐少爷押了万元三孤丁,结果万元巨钞,不足付清负款,由陆某作保,改日付足,唐少爷携款而去。阿林太目瞪痴呆,陆某向他埋怨不已:“你别犯死心眼,连赢十数宝,还不变个法儿?”阿林太既不醒攒(cuán)儿(不明白),死怨自己财运不佳。归家以后,不愿再付赌债,闭门不出,且嘱其家人,如有人找,说我已赴天津。阿林太输了万元之款,反倒不敢出门,老月(设赌骗钱的)的骗局可怕,老月的手段也够辣的。后来阿林太久后不见有人索债,渐渐出游,偶至某宅,见门紧闭,粘有红纸帖,上写:“空房一所,共三十一间,自来水、电灯无不齐全,有愿租者,门内有人领看。”阿林太始觉受骗,后遇友人谈及此事,友人明白老月的事,告诉他老月做点使用的门子,有反有正。你抛了万元,就是叫他们使了反门了啦。

江湖的老合们(闯江湖的)常言,他们不受骗的秘诀是“不贪便宜”四个字。按阿林太受骗的事,也是贪便宜才上了当。“不贪便宜”的下联是“不能受害”啊!





丢包碰瓷


余友李君,年二十余岁,在商界服务,为人诚实。一日在柜上请假,归家有事。行至三岔路口,见一身穿制服之军人,手执药瓶两个,行走甚急,竟与李君相撞,碰在一处,啪嚓声响,两个药瓶摔得粉碎。该军人抓住李君说:“你将药瓶碰碎,好好赔我,这是我们团长的。”当时李君说:“我没碰你,是你碰我,焉能赔你?”那军人说:“你不赔我不成,须跟我见张团长。”李君听说去见他们团长,似有所惧,有意赔偿,向他道:“你这东西是多少钱买的?”该军人出示药房发票一纸,上写:“××药水洋八元四毛。”并有××药房图章,贴有印花。李君无法,说:“你跟我回家取钱成否?”该军人点头应允。李君同他到家取钱,军人在门前候等。是日敝人恰巧正在他家,李君言说此事,向其父索洋欲赔偿该军人。我说:“这是碰瓷的。他不是真正军人,可以向他……说,分文不赔,便可无事。”李君点头而出。该军人问道:“你家有钱吗?”李君说:“我家无钱,你跟我往吾叔父处去取。”该军人又同李君而行,在途中问李君道:“你叔父在哪里做事呢?”李君说:“在探访局当队长,他那里有钱。”该军人行未数步就溜之乎也。后李君问我:“该军人为何自己溜了呢?”我说:“他是‘里腥(lǐ xing)的海(hāi)冷’(假军人调[diào]侃儿叫里腥的海冷),干丢包、碰瓷的,干的是犯法的营生。我教你所说的话,是给他‘扣瓜’(威吓他调侃儿叫扣瓜),他溜之乎也,逃之夭夭,是顶了瓜了(害怕调侃儿叫顶瓜)。”骗匪扣瓜,也是“簧点不清”(见事则迷调侃儿叫簧点不清)。丢包的,碰瓷的,在如今还是常有。社会人士勿受其骗,如遇时,以吾上谈之法治之,定能无事的。





江湖骗术之闯啃(kèn)法


余友马君,乃津埠巨商子也。一日行至租界下关码头小巷中,见有一个十一二岁的幼童,手持信封一个,长约七寸,宽约四寸,这幼童拿着那信似有惊异的样子。马君走到他面前,他向马君道:“你看我拿的什么东西?”马君接过他的信封,见上边写的字是极好的行书,写的是:“寄至天津河东小集街德成银号张经理收。”左边贴有邮票两角八分,盖有邮局之戳记。马君拆开了取出信笺观瞧,只见笺上的言语系上海李君接到张经理之信,欲求他在沪购最上等人参,今已由沪永康参局购妥人参四支,随邮寄到,共计大洋二十四元整。信笺的背面贴有名片一张,上边印的是:上海英界万隆洋行副经理李德明,广东南海人。又有发票一张,上写人参四支,分量计重××,计洋二十四元整。上边有永康参局的图章××年×月××日,粘有印花票。信封内有红绵纸一张,内包人参四支。余友马君家道殷实,常购此物,也颇爱此物。他向幼童问:“此信可是你在这里捡的?”幼童说:“是我在这里捡的。”马君欲得此物,向幼童说:“此信是吾友人张某之信,你拿了去也无用,我给你两毛洋,快快去吧!”幼童说:“我不干,拿回家叫我爹看看去哪!”马君说:“叫你爹看也没用。我给你四毛钱,快去吧!”幼童说:“四毛钱不成,非八元不可!”马君心爱此物,争持好久,直增到四元才说好了。马君付给幼童大洋四元,幼童走去。马君持物回柜,得此便宜,焉有不向人夸示之理?有司账人王先生听他所说,取过信封内人参熟视良久,笑向马君道:“你上了当啦!叫人骗了!”马君似有觉悟,拿着人参跑至药店里向店伙说:“劳驾,给我看看此货成色如何?”店伙看了笑道:“这是什么呀?”马君道:“人参呀。”店伙说:“那不是人参,这是香菜根。”马君始知受骗,连呼倒霉不止。

后马君向敝人言说此事,我向他说:“这是江湖骗术门的行当,‘怎科(zěn kē)子’(管小孩调[diào]侃儿叫怎科子)出来做这骗人事,能叫人不疑。故此,他们都夹磨(jiá mo)(师父传授真本事)怎科子出来骗人。”马君问道:“这行儿叫什么?”我说:“江湖人管这行调侃儿叫闯啃(kèn)的。”马君说:“我这么机灵的人也会上当。”我说:“世上事,不贪便宜没有当上。”





江湖中之闯啃(kèn)的骗财法


我老云有个朋友是天津东大庄人,有一次我去看望他,恰巧他未在家中,往某处有事未回,家中只有他老母与他媳妇。这婆媳将我让到屋中,烧水沏茶,叫我等候。我正喝茶之际,由外边进来一人,约十四五岁,穿着蓝布大褂,光头未戴帽,两只鞋上有挺厚的尘土,面带惊慌之色。他到了院中就嚷:“大娘在家没有?”老太太跑出来看,不认识这孩子,忙问:“你找谁呀?”他说:“我不找谁。”说着话就冲老太太跪下了,二目落泪说:“老大娘,您快救我吧!”老太太看他这种神气,惊问道:“你……这样是为了何事呢?”小孩哭着说:“我是塘沽的人,我父亲死了,家中只有我妈,妈在我姥姥家住着,我叔将我送在天津×仁堂药铺学徒,我学了有半年多,因为净受气,挨打受骂,我不愿学了,要往我姥姥家找我妈。我由柜上偷出些个值钱贵重的药品跑出来,有柜上的伙计追下我来了。要叫他追回我去,我叔厉害极了,非得将我打死不可,你老人家若是行好积德,到门外瞧瞧,如有人打听我,你老人家就撒谎说我出了村往东去了。他往东找,我好往南跑,只要到了我姥姥家,这条小命就算保住了。”说罢痛哭不止。妇女的心最软无比,看见他这样可怜,就动了恻隐之心。老太太叫儿媳妇给他些水喝,自己往外就走,到了门前往各处瞭望,只见由西边来了一个人,约有三十多岁,穿着打扮像个店伙似的,两眼发直。他见了老太太说:“借光,老太太,将才有个穿蓝布大褂的小孩,你看见没有?”老太太说:“你问他做什么?”这人说:“我是×仁堂的伙计。我们柜上跑了个徒弟,他偷了千数多块钱的货物,我追赶他进这村,也不知怎么,没有了!”老太太故意说:“不错,将才有个小孩慌慌张张地从我们这儿过去,他出了村往东去了,你快往东追吧!”这人说声:“劳驾!”匆匆地往东而去。

江湖中的“闯啃(kèn)”生意,是用十四五岁的小孩利用人们的恻隐之心,去骗妇女钱财。



老太太回到院中,向那小孩子安慰道:“你放心吧,追你的那人叫我给支走啦。”这孩子立刻趴在地上给老太太叩头。他说:“老大娘您索性行点好,给我顿饭吃,借给我几块钱当作盘费。”老太太说:“哟,瞧这孩子,咱们素不相识,给你顿饭吃那倒算不了什么,借给你几块钱,那可不成。”小孩说:“你老要不借给我钱,我有点东西求你给卖卖,弄几块钱路费好往我姥姥家去。”老太太问道:“你有什么东西呢?”小孩说:“我由药铺里偷出来有麝香、熊胆、牛黄、冰片、眼药、丸药。”他说着由衣裳里取出个包儿往地上一放,将包打开,只见里边有几个小小的四方玻璃盒,上有小红纸签,写着四个字“真正麝香”,还有写着“真正熊胆”、“真正牛黄”的。还有二十多瓶眼药、十几匣牛黄清心丸,盒上、匣上、瓶上,都粘着天津估衣街×仁堂的字样。他向老太太道:“你老要留哪样儿呀?”老太太不认识字,也不懂行,就向我老云说:“云先生,你来看看都是什么药吧?”我说:“有麝香、牛黄、熊胆、眼药、牛黄丸,这些东西都是值钱的贵重药品。”老太太说:“他二姨的公公头几个月得了一回半身不遂,就吃牛黄清心丸好了的,我将牛黄丸都给留下吧!”小孩说:“这牛黄清心丸是十二丸一盒,我们柜上卖八毛钱一丸,每盒卖八块大洋,要整盒买较比零买便宜两丸子。”老太太听他所说,将嘴一撇道:“哟,那么贵谁买你的,我们还到铺子里去买哪!像你这东西,得便宜我才要哪。”小孩说:“便宜是一定的,我也不能卖八块一盒,你要都留下我可不卖,你要留个一两盒好办。你老随便给钱还不成吗?”老太太说:“我就留一盒,给你一块钱。”小孩说:“那可不成,一块钱太少了。”我老云给他们圆全买卖,算是两块洋一盒。于是老太太就拿了一盒药,给小孩两块钱。她儿媳说:“问问隔壁王大婶要不要?”于是老太太又出去给张罗买卖,工夫不大又来了几位街坊,男的、女的,都抢着买,有拿起麝香就给三块钱的,不卖还不行,有愣给五角钱拿几瓶眼药的。眨眼之间他就卖了十几元钱,他直用手捂着,大嚷:“这么贱,我不卖了。”将包儿一提溜往外就走。他走后大家又谈谈论论说:“买了便宜东西。”我看他们都喜气洋洋地各自散去。等了一会儿,老太太的儿子也没回来,我就告辞而归。

过了两个多月我又到他家,恰巧她儿子又没在家,我忽然想起老太太前者买的便宜货,我就问:“伯母,你上次买的那便宜货好不好呢?”老太太听我一问,立刻就气呼呼地说道:“老云,你还提那事呢!我们都让人家给骗了十几块钱,买的都是假药。那个挨刀子的孩子!”又哭又说,“把我们冤苦了,他不是个好东西,他,他……”我听了这片闲言闲语,才知那小孩是个骗子手。我回到天津,就向一些老于世故人情的朋友提说此事,都说“这是骗子手骗财的”,但是谁也不知道其内幕如何。

在前年,我老云到济南府,在商埠遇见了个朋友,此人姓袁,从前他是个卖药的江湖人,专摇串铃下乡去卖药,如今他当了官差。我二人在茶馆聊大天,聊到小孩骗财这桩事,老袁说:“那是一种生意。”我说:“那是什么生意呢?”他说:“这种生意说江湖行话叫做闯啃(kèn)的。”我说:“这闯啃的生意为什么都用小孩呢?”他说:“这种生意是专蒙骗妇女。要在大街里、市场内,是没有人听他们那套的。做这种生意的,是一个掌买卖,一个敲着。”我问:“什么叫掌买卖呢?”老袁说:“那掌买卖的是那小孩,在未做这闯啃生意以前,先得物色个小孩,可是找个相当的最难,十八九岁的像个大人一样,愣往住人家的院子闯,不惟骗不了财,赶巧了还许叫人打了。若是用个十一二岁的,知识太幼稚,胆量也小,任你如何教练也不成功。最好是找个十四五岁的小孩,以身材矮小为佳,尤以聪明伶俐、有胆量、见人敢言、口齿伶俐为上选。得着这种小孩的,每天以上等吃食诱惑他,将骗财方法传给他,等到他练得能够不害怕了,掉眼泪了,算是成了。他们江湖人管教给小孩往住人家愣闯去骗财,说行话叫夹磨(jiá mo)(调教)铃铛(小孩儿)去掌买卖。等到教成了,就自己做些假药,但是摹仿谁家药,仿单、药品、装潢,也得和那真货一样,以叫人看不出破绽为准。到了闯啃的时候,是徒弟带着货在前边走,师傅在后边跟着,如若小孩闯入人家,见了妇女撒谎骗人,将人冤得信以为真了,或是生了恻隐之心啦,才能有本家的人出来,站在门口儿给小孩巡风。他师傅见由门里出来人了,就奔过来假装追徒弟的样子,向人问他徒弟。巡风之人都是将他的师傅认作追捕逃徒,用话支走,或东或西。他师傅也得有着急的面孔,人家说东,他就得匆匆地往东,以假作真,是他敲家子(帮凶)的发托卖像(指演员在表演时要惟妙惟肖,通过喜怒哀乐刻画艺术形象),那小孩的师傅便是敲家子。”我将这事听明白了,向老袁问道:“那小孩天天和他师傅去骗人,能骗多少年哪?”老袁说:“也就三二年。”我说:“过了三二年又怎么哪?”老袁说:“将他扔了不要,再另找一个。”我说:“他能随便扔了吗?”老袁说:“他们做闯啃(kèn)生意的人要找个徒弟,并不是有人荐的,都是那不听说、不听道、在家里逃学、学买卖受不了规矩、背着铺子、背着家长偷着跑出来的。凡是这种偷着跑出来的孩子都是又馋又懒,专会撒谎,十四五岁、十三四岁的居多。他们闯啃的生意人光在各处寻找这种小孩。找着了之后,先以美食华丽衣服诱惑,然后才夹磨(jiá mo)(调教)他骗人的方法。等到能够天天出去骗财了,那小孩胆量也大了,差不多就不受师傅管束。他师傅教他抽大烟,染成了嗜好,不惟他天天能去骗财,因有嗜好在身,骗人钱财的时候也能多骗,也不发懒,倾心愿意地受师傅驱使。及至他的嗜好日深,岁数也大了,所骗来的金钱只够他自己用的,师傅得不着好处了,就假做开穴(xué)(即是另往他方),就将徒弟抛了(江湖人管什么东西不要了调[diào]侃儿叫抛了)。他那徒弟嗜好染成了,他师傅将他抛了,没人给他敲着,纵然他有胆量去闯啃,骗来的银钱也是少的。他一开知识就学会了撞骗,离开了师傅,什么事不好他去干什么,这一辈子也好不了,除死方休。”

那闯啃的老合(闯江湖的)手段有多么毒辣!社会里有这种蟊(máo)贼骗人、害人,地方上的官吏对于他们都是极力除治的。社会里面情形,黑幕重重,非我老云所能尽知,仅将我所能知道的公诸社会,使未受骗的人多加小心,便是我老云忠于社会爱护人群了。





江湖中之撇(piē)年子把戏


修脚的人是一种手艺行当,也属江湖也。生意人调侃儿管他们这种行当叫“撇年子”。这修脚的艺人出来挣钱,分为三路:最没能耐的专奔澡堂子。他们这行人到了澡堂内,只能按着规矩给人修脚,以手艺优劣定高低,不能敲诈客人,说行话叫“做平活”。昔日,在澡堂子内做活,每逢给人修脚一次,柜上将全部修脚费收去,只给他制钱一文,名为“工钱”。他每日两餐是吃柜上的。其最大收入,每日分一大份零钱也。至今改为修脚一次工钱一大枚,也随币制改革而增加也。

修脚的行当,亦属江湖也。生意人调(diào)侃儿管他们这种行当叫“撇年子”。



撇(piē)年子(修脚的)的艺人,稍有能力的不去澡堂,专去“磨(mó)杵”(管串街巷兜揽生意调[diào]侃儿叫磨杵)。腰掖刀包子,手持竹板,行于街市,不住敲打竹板,梆……之声不止。有商家、住户若要修脚,可将其唤入,其修脚工资不多,仅二三十枚。如果他看客人是“点”(即是能受其敲诈的,彼必敲诈,说行话叫“挖[wǎ]点”)。他不是说你有脚垫,便说有脚鸡眼。彼素知足部之筋骨穴道,何处一按即痛,如欲挖时,便按痛处,如客人呼痛,他就说:“有足疾,须除治,否则定成大患,恐难行步。”客人若愿除治,他就看风行船,瞧事行事,应挖多少钱,斟酌情形,是用步步紧的法子“挖杵”儿。最奇怪的是好好的脚,他也能修下许多鸡眼,说行话叫做“出托”。其出托之法,是由脚皮粗厚之处,用手术能由该处修成鸡眼,江湖人管他们这种技术调侃儿叫“出样色(yàng shǎi)”,其出样色之奥妙,真令人不可思议也。撇年子这行人最有能力的,是“顶神凑子”(赶庙会)或“顶凑子”(赶集市)或“搁明地”(露天)。如若没主道上门,他有个“点张子”(即一布折长七八寸,宽五六寸,上边层层画有患各种脚疾的图样。江湖人管这宗东西调侃儿叫点张子)。他将点张子打开,乘游人最多的时候,用棍指着各种脚病的图样,向人演讲各种脚疾的病原,什么猴子、瘤子、脚鸡眼、脚垫、脚痔、脚漏、脚气,说得原原本本,也能招一群人围着,听其讲演将粘(nián)子圆上(聚好了观众),往下“叫点”(即是硬往下拉拢买卖)。其第一次,按着耍手艺的挣钱行话叫“头道杵”。第二次挣钱行话叫“二道杵”,其余为三道杵、四道杵,最末次所挣之钱行话叫做“绝后杵”,其所售能治各种足病的药品,说行话叫“枪里加鞭带挑(tiǎo)汉儿”。撇(piē)年子(修脚的)的艺人,“靠地的”绝不挖点(敲诈人),在各市场、各街巷成年地不走,天天必摆修脚摊子,江湖人管这种做法叫“靠地”的。即靠长地(长地是指固定场所),就以挣熟主顾的银钱为是,如若施其敲诈手段,焉能有长久照顾的主儿哪?今天他在东,明天在西,或往河路码头,或往集场庙会,江湖人说,他是做“走马穴(xué)(走一处,不能长占,总是换地方挣钱,江湖人叫走马穴)的买卖”。凡是做走马穴的买卖的撇年子,遇见点儿(被骗的人),不挖(wǎ)白不挖(不骗白不骗),和耍光棍的遇见了秧子(被骗的人)不吃白不吃一样。撇年子的人,专挣劳动人的银钱。盖劳动的人终日奔走,以两条腿奔驰生活,最怕双足有病,不能动转。如若有足疾时,不惜金钱治好了两只脚,好像神行太保似的,奔走求生也。至于“火码子”(管有钱的人调[diào]侃儿叫火码子)每逢行动,不是汽车、马车,便是包车,两只“曲勒(qū lè)”(管脚调侃儿叫曲勒)有代步之物,不生足疾,哪能用得着撇年子呀?故此我说,“撇年子是挣水码子杵头儿的行当”(即是挣劳动界的金钱)。如今有些个女子修脚的艺人,专能在脚指甲上修各种的花卉翎毛山水人物,阔公子、小姐们修饰足的美,每次约二三十元。社会里的事,还是铺火码子(有钱的人)的金钱容易得很哪!唉!





天桥挑(tiǎo)水滚子的


凡是到过天桥的都听见过:“蹭……蹭油的……蹭癣的。油了衣裳不坏的。”这是个卖胰子的。他在天桥的南边,也不支棚设帐,也不租赁桌凳,就在地上铺一张二尺见方的白纸,上边放个小铁盒,一个玻璃瓶,有几十块绿颜色的胰子。他用那个胰子蘸点凉水,往衣裳上抹,如果有油泥,立刻就能蹭下来。有长了癣的人,他也给蹭,当面试验,白蹭不要钱。卖这个东西的人是个又矮又瘦的人物,只要他往那里一站,他就扯开了嗓子喊:“蹭,蹭……这样的蹭油啊,油了衣裳不坏的。”无论男女老少,走在他那里听他这样喊蹭,都抿不住嘴地笑。他卖那胰子,蹭衣裳上的油泥还真有效力。起初,我老云很纳闷,不知他那东西是用什么东西做的,能够当时有效。后来有个江湖人告诉我,他挑(tiǎo)的是“里腥啃(lǐ xing kèn)”(管卖假东西调[diào]侃儿叫里腥啃),他那东西当时有效,是他那玻璃瓶的凉水有毛病,不知者都以为是凉水,其实是汽油。汽油这东西就能将衣服上的油泥蹭掉,还是真有效力。卖胰子的使的“门子”就仗着汽油的力量。“挑水滚子的”(江湖人管卖胰子的调侃儿叫挑水滚子的)虽是个小生意,也有“门子”,和前门一带摆摊卖化妆品一样,东西不好,就每天往东西上抹点香水精,就能蒙得住人。社会上的人们还真有认他们那种东西的,总而言之,贪便宜而已。

行行有门,门门有道,世上的事儿,都是这样啊!





老荣(小偷)中之高买(高级小偷)


老荣是偷窃的人。其中分为:轮子钱、朋友钱、黑钱、白钱、高买。

在早年并没有高买这行人。从前的商号都不讲究修饰门首,也没有玻璃货架、玻璃阁儿,都是用老式的货架子,有货好放,有货好收,也没丢货之说。只要货真价实,不怕深深的小胡同里,也有人进去买货。如今的商家不似从前了,虚伪诡诈,不是老尺加一(多给),就是大减价,牺牲血本。门前高搭彩牌楼,减价一个月,并有大赠品。头彩狐腿皮袍一件,二彩金手镯一副,三彩手表一只,四彩马蹄钟一个,五彩美伞一把,六彩绸巾一条,七彩牙粉一包,八彩洋烟一盒。凡买一元货物的顾主,有彩券一张,当面抓彩,彩彩不空。就有那冤大脑袋好听这一套,花一块买东西,还抓一回彩。其实平日值八角的货物,他卖一元,那多卖的二角钱,是他们凑在一处,做彩品之本钱与传单、广告、彩牌楼等等的开销,就是得了彩,也不过是牙粉一包,烟卷一盒。买卖商人不能典房卖地往外赔垫,无论如何也是买主吃亏,羊毛出在羊身上。他们不诚实做买卖,专有些高买偷窃他们。这新式的玻璃货架、玻璃阁儿装上货物,也是给高买们预备的礼物。若按早年的装货之法,高买哪能得手,除非是搬运法成了,冲他们一念咒,东西就过去。若没那样本事,就偷不了商家的东西。

我老云问过小绺(xiáo liu)(小偷):“怎么偷商家的小绺叫做高买呢?”某小绺说:“当初没有高买,不过他们专偷商家。在未偷之先须多看货物,堆起货来他好下手。其多看货之法,是看一卷绸子嫌不好,叫伙计再将好点的看看,表示他要买高货,不怕多花钱。事后商家觉悟了,是那买高货的客人将东西偷了去的,就管他们叫高买。”我老云头几年在天津住着,对于高买的手段与窃货的妙法,总疑惑有什么高超的窃术,我要瞧瞧高买如何偷法,就先交了几个商界的朋友。有一天津某租界某商号之经理与我交为朋友,他那买卖是个绸缎庄,我时常上他柜上串门,和先生、伙计们聊起大天没结没完。我是醉翁之意不在酒,借着聊天为名,净等有高买来了看看他们如何偷窃。有一天我同管账先生正说得热闹之际,由外边进来一位买主。这人长得细条身材,穿着绮霞缎的棉袍,戴着瓜皮式的绮霞缎的小帽儿,金丝眼镜,两只皮鞋,人是白白的面皮,黑黑的胡须。看他那人样,穿的衣服阔绰,好像某机关的职员。在那几年穿衣服还兴阔袖口儿,高开气儿,我见了这人就感觉他不是好人,我也说不出是怎么不好来,这种察言观貌、看人辨别善恶的心理,是可以意会不可言传的。我见他走在一个玻璃阁旁边止住了脚步,伙计们赶紧过去张罗买卖,问他:“你买点什么?”他说:“天要热了,棉袍穿不住啦,我做个夹马褂、夹袍儿。”伙计说:“你做吧。瞧了货,将衣裳的尺寸开个单子,咱们柜上能做,三天准能做得。”他问了问做夹马褂手工多少钱?做夹袍手工多少钱?又叫伙计取出绸缎来他瞧瞧。学徒的给他斟了一碗茶,他看了不带花的大缎子,嘴里不住地夸好,可又说:“没花儿不时兴了。”又叫伙计给他取绮霞缎,问多少钱一尺,又要买,又嫌成色不大好,叫伙计给他取好的。他看了这个,又看那个,手里按着货,又不住地往四处观瞧。我老云倒像做贼一样,赶紧看别处,不敢瞧他。他看完了四处,又看货的成色。我老云就明白了,东西取出来的数目,够他偷的份了,先巡了风,然后下手。我老云似看似不看的,可就“招路把(bǎ)合”(此处是用眼睛注意着看)了。只见他坐立在阁的右边,冷不防地往外一转儿身,左手扯四五尺缎子,像变戏法的抖开了毯子要闹鬼儿一样,用那缎子往他棉袍大襟上一盖,问伙计:“怎么样?”两三个伙计的眼睛都往那缎子上和棉袍上观瞧,嘴里还批评好坏。我老云就不看那里了,见他一拧身抖开缎子的时候,有一卷花丝葛,由玻璃阁上掉下来,他用左腿左胯骨将花丝葛倚住,又见他左手将绮霞缎一撩,折回玻璃阁上,右手往衣裳里一伸,假装掏钱之状,说:“我不知是带着钱没有?”摸了摸道:“带着钱哪!”我可看见那卷花丝葛,由他的棉袍左开气挤进去了。我想他不是掏钱哪,是花丝葛进了他的棉袍了,用右手做装摸钱之状,暗含着将花丝葛用松紧带夹住了。东西夹好了,他说:“带着钱哪。”右手掏出来就问伙计:“裁个马褂子,八尺二寸够不够?”伙计说:“够了。”他说:“裁八尺二寸吧。”伙计给他用尺量货,他又看这卷,看那卷,阁上放着的十几卷都是竖着。我见他将一卷横着放着,又将竖着的一卷花丝葛打开了五六尺,冷不防往外拧身,将花丝葛往他棉袍上一盖,仍叫伙计们瞧。众伙计都一齐往大襟上看,我老云又见他把横着的那卷绸子,倚在胯骨、玻璃阁之间。伙计直夸做花丝葛的夹袍好看,他将五六尺花丝葛往玻璃阁上一放,右手又伸进棉袍,说:“我带的钱,也不知够不够?”这卷绸子又从他棉袍左开气进了袍内,假装掏钱之状,暗着又用手将绸卷儿掖好。可是这回掏出皮夹来,他叫伙计给他开尺寸单,马褂领长尺寸,身长等等都写完了,留下一块定钱,只做了一件马褂就走出去了。我合计他窃了两卷绸子,留下一块洋,他要将那两卷绸子按七折贱卖,也能得二十几元。他走后,我见柜上的先生写账,伙计们仍然张罗买东西的主顾,毫不知觉。彼辈窃术之精,也真巧妙,较比变洋戏法的魔术有过之无不及。可惜彼辈之聪明未入正道,得了财物,也不过往烟花柳巷、赌博场内做嫖赌的挥霍,结果如何,不是染花柳病而死,就是病死牢狱之中。如果他们能归正道,不拘入了哪行,也能高人一筹,何愁衣食不丰。邪途误人,向无觉悟的,即或有觉悟的也是在将死的时候,落个最后觉悟,岂不晚矣?

我老云在某租界,有一次遇见了于黑(吃飞轮[在火车上偷窃的]的高手),我同他到某旅社闲谈,向他探问高买之窃术。据于黑说:“高买也有组织,或三人,或五人,不能一定,有本领的人去窃取商家财物,其窃术不精或学而未成者,随着出去护托(护住偷东西的,不叫人瞧见是怎么偷的),至于心、手、眼三样皆笨的人,也就管巡风而已。”我问他们高买窃取绸缎之法。他说:“高买欲在某商号窃取贵重的物品,在未窃之先,先到该商店假装买主,以买东西为名,察看他这买卖的柜上伙计人数多寡?由何处而进?在何处行窃?由何处而去?将道看好之后,再来了才能窃取。高买(高级小偷)最得意的时候是冬令,皮袄、马褂、大氅全都在身,窃取之时,容易下手,也容易往身上收藏。每逢冬天,他们天天出去,如鸟藏食,防于大风雪之日不能出去寻食,专食收藏之物,接济不得食之日一样。每至夏季,天气暑热,衣服单薄,窃取财物不易收藏,并且容易败露,本领稍弱的,十有八九全都歇夏。春秋两季,夹衣服上身,虽不如冬天得手,也能偷窃,也能收藏了。高买之窃术也分粗细活儿。窃术平庸的,只能往绸缎庄窃取笨重物品,对于珍珠、钻石、金表等细小之物,心虽想窃,却不敢着手,闻香不到口也。窃术灵敏的高买都讲究窃取细货,若窃钻石一个,可值千百之数,胜似窃取绸缎十回。一样窃取,何不取贵重之货而取笨重价小的东西呀?凡窃细货的高买都是本领高超,一人足矣。越是本领不济的,一人不能窃取,十有八九都有护托(护住偷东西的,不叫人瞧见是怎么偷的)的跟随,至商店窃取不得手,护托的或给他遮蔽,或用手乱指,将店伙眼神引走,目视别处,窃物者才能得手,任意窃取。护托的也不容易。主窃的,窃物时有一定窃取方法,护托得是补助主窃人之不足。变戏法的在台上变十三太保,大海碗一大堆,由身上往下落(liào)活,全仗着他那护托的为之遮蔽,护托的以严而不漏、缓速适宜为美。高买的助手也如变戏法护托的一样。其护托之法,固定者少,临时生智,随机应变时多,也极不易也。巡风的尾随高买身后,高买进某商店时,他就在某商店门前站立,或假装行路之状,如门前等候一样,不过心理不同而已。如有‘老柴’(管官人调[diào]侃儿叫老柴)经过,巡风的得能看出老柴的行动,是否从商店门前经过?是否‘挂桩’(管官人在门前等候窃贼调侃儿叫挂桩)?如看出是从商店门前经过,假做不曾看见,由他过去;如若看出是挂桩,巡风的立即走入商店,向高买微示其意,使其心领神会,纵能得手也不窃也。空手出来,老柴抓获时,以无赃物在身,可以免入法网而不破案。常言‘捉奸要双,捉贼要赃’,若无赃物在身,真假难分,老柴也无可如何了。老柴中高超的人物,每遇高买(高级小偷)入窑儿(即进商店),即在前门‘挂桩’,候高买赃物在身出商店时再捕之,十有九获。高买也无词可措也。有些老柴眼里有活,虽然在门前挂桩,若高买知觉,未窃财物,由商店走出时,看他身上无赃,也不捕之,仍尾随其后,必待其窃物在身时而捕,免落违法捕人之罪也。”

我问于某:“有些老柴见了高买,不论高买有无赃物也捕之而归,是何缘故?”于某说:“那是臭盘儿。”我问:“什么叫臭盘儿?”于某说:“大凡是高买在何处栽过(窃贼管被捕犯案调侃儿叫栽了,遭过官司被捕过即是栽过),何处老柴就能认识,如若罪满出狱,即离某地。如不离开,仍在该地作案,被老柴们看见就能复入法网。老柴们认识他是高买,若遭过官司被官人拿过的,是官人都能认识他的,虽不偷窃,官人看见也一样逮捕。如若不承认他是高买,官家将他前次犯案的底卷取出来叫他看了,他也得承认自己实是高买。所以高买们就怕臭了盘儿,如若臭了盘儿,简直吃不开了;若不改行,也得另往生地方去窃取,熟地方是不能存身的。”

我问于某:“高买们窃取金镯、钻石戒指、人参等贵重物品,是怎么窃取?其窃取之手术能否说明?”于某说:“我住在×××旅馆五号房内,明日早九点你去找我,我在该处试演一回你就能知道了。”我听了高兴已极,彼此分别。次日早晨九点钟,我老云就到×××旅馆,果然于某在五号房中候我,相见之下,彼此大笑。他说:“你看我穿的衣服好与不好?”我看他穿的是灰色棉袍,青礼服呢鞋,内里衬衣只有个白汗衫而已。我看他穿的衣服与普通人所穿的一样,不过尺寸略微肥些。我说:“你穿这衣服略微肥点,也不觉寒碜。”他叫我将手表取下来放于桌上,我就依了他,将手表取下来放于桌上。他又叫我将钱夹取出来也放于桌上,我又依了他,将钱夹取出也放于桌上。那钱夹与手表同在桌上,两件东西相离不过五六寸远。于某用右手拿起钱夹子掂了掂道:“你这皮靴掖内没有多少钱。”说完了又将钱夹放下。我再看那桌上的金表,已然没了,不觉惊讶起来。他问我:“老云,你的表哪?”我说:“不知哪里去了!”他说:“你用手往我身上摸摸,我的左胳臂哪里去了?”我用手一摸,他那左胳臂没有,袖筒里是空的。我忙问他:“你左边的胳臂哪里去了?”他冲我一笑,将右胳臂抬起来,说:“你看这是什么?”我往他右胳臂的底下一看,那马褂的袖子、胳臂肘儿的地方,多出一只手来,那只手攥着一只金表。我至此始悟,他是将那左胳臂退入衣内,又伸在右边的袖内去了,最奇的是他这只左手能在右胳臂肘儿底下伸出来。原来他那马褂,故意地在袖筒的胳臂肘底下做的有道缝儿,为的是好在这缝内往外伸手,使人不知不觉,窃取财物。他叫我看明白了,又说:“你将我的马褂替我脱下来,你再看看。”于是乎我老云就将他的马褂脱下来。他说:“老云,你再看我的棉袍。”我再往他的棉袍上一看,原来他那棉袍的胳肢窝底下也有一道缝儿,他那左胳臂就是由右胳肢窝的缝伸出来的。他又说:“老云,你再把大棉袍给我脱下来,你再看看。”于是乎我又将他的大棉袍脱了下来,再看他那汗衫,也是和那棉袍一样,两个胳肢窝底下也都有道缝儿,他那只左胳臂就是由那右胳肢窝的缝儿伸出来的。他叫我看明白了,左胳臂才退回去。他说:“我叫你看看那只表留于何处。”说着他自己就将汗衫的纽扣儿全都解开,脱下汗衫来,我往他身上看看,只是他贴身有个皮兜儿,其形式与变戏法的身上带的皮兜子一样,那只金表就收在兜内了。

我将他全身的衣服,窃取他人财物的门子(即是闹鬼儿使人不知之处)全看明白了,才知道高买(高级小偷)们窃取东西之法。于某问我:“老云,你明白了没有?”我说:“明白了!”他说:“这个情形如何?”我说:“这不过是你们闹的鬼儿没人知道,也算不得怎么神妙。如若变戏法的艺人改了行,就能按着你那方法去当高买的。”于某说:“你别看变戏法的艺人在台上变得那么巧妙,如若叫他窃取人家的财物是不灵的。他变戏法成了,偷人家东西他们是不成。别的不说,他们的胆儿就没有我们大。若是偷了人家的东西,赃物在身,心里害怕,脸上变色,露了破绽,一定叫人抓住打官司。他们变戏法的人,有身上藏着所变的东西,坦然自在,似有如无,叫人看不出破绽的长处;我们有将人家的东西偷过来藏在身上,叫人看不出破绽的长处。他们沉得住气不露破绽,还是不如我们。”我问:“怎么不如你们哪?”他说:“凡是看戏法的人们,都知道变戏法的人身上有毛病,藏着东西哪,不过没人给嚷就是了。即或变漏了也不要紧,至大有人喊个倒好儿完事。我们若是叫人看出破绽抓住了,喊来巡警,真赃实犯,打了官司,至少也罚几个月的苦力,蹲几个月的监狱。同是闹鬼儿、沉得住气,究竟还是变戏法的人胆子小,高买的人们胆子大。我敢说变戏法的人当不了高买,隔行如隔山,不论是哪一行也是一样,行家能成,行外人是干不了的。”我听他说,深服其论。不过,我心总觉着他们的胆量、知识、见解、谈吐,都是比普通的人们好得多;就是一样,有知识何不去奔正道,同是穿衣吃饭何必做犯法的事?

我老云又问他:“你这衣服是哪里来的?”于某说:“这是××的东西,我们两个人住在这一间屋内。今天是他有钱,没有出去做活,穿着没有门子(管闹鬼儿使障眼法叫门子)的衣服逛小班(低级妓院)去了。我是乘他不在店内,叫你看看这高买的门子,你可别告诉外人。”我当时应允,又说:“你们这当高买的只有衣服不同能偷东西,并没有什么特长。”他说:“我叫你看看特长。”他又打开衣包,取出几件极瘦的衣服来,穿在了身上。我看着又瘦又长。他说:“这么瘦的衣服,我也能将胳臂由袖口儿退了进去。”说着,他将这件衣服一抖搂,我再用手去摸他左袖筒,已然空了。他这只左胳臂已然退进去了。最奇怪的是没人给他揪着袖口儿,他自己也没揪着袖口,只凭他略微一抖搂,那只胳臂就能退进去。他们有这种惊人的本领我也不佩服,只要他们不入正道,任他有多好能耐我也是轻视他们。我问他:“高买的本领有神偷之能,为什么还有被捕的人哪?”他说:“当高买的遭官司,都是他成天往娱乐场所任意挥霍,花的金钱太多了,叫官人注了意,访查实了才遭官司。在他们往商家窃取财物的时候不容易破案。”我问:“那么他们偷窃的时候就没被人看破,当场被人抱住的事吗?”他说:“我们老荣(小偷)若将人财物窃到手中,又转别人手内,那叫二仙传道。即或丢东西的觉悟了,将我们攥住,也是不怕,那东西早就没了。身上没赃,是脱身计惟一不二的法门。高买出去做活也和我们一样,不是一个人出去,少者三人,多者五人。如若将东西偷到身上,商家觉悟了,伸手揪人,也是白揪。照样儿使二仙传道的方法,将东西由甲的身上又传在乙的身上,甚至于还有由乙的身上又传到丙的身上。高买们遭官司,人赃两获的事百不一见。”

我问:“高买(高级小偷)有偷东西没偷成,赔了本钱的事有没有呢?”他说:“也有。”我问:“怎么高买会赔本儿哪?”他说:“有那常丢东西的商店,丢得怕了,柜上的伙计多雇佣聪明伶俐的。高买们进门,他们也能看出一二。到了高买看货的时候,那手不离货,货不离手,看得严密,无法下手。不惟不偷了,还得多花钱买他们的东西。”我问:“偷不得手,干吗还买他们的东西哪?”他说:“高买们遇见了这种情形,是叫人看着形迹可疑。为了叫他们放心,不当贼看,花大钱买东两,是稳猾点(狡猾)的店伙之心。不止于这一次,三两天一趟,得花钱买他几趟,叫他知道是好主顾啦,然后乘他们不防的时候,大大地偷上一水,将几次损失的银钱一下子全都弄回去,还得有富余,剩下些钱,才能心平气和。”

我听他所说高买如此狡猾,又问:“那么高买怕老柴(侦缉人们)不怕呢?”他说:“高买们怕老柴是不假。即或被捕了,反倒不怕。他们觉着遭了官司就豁出受几个月的罪去。期限满了出了监狱,还是照样去当高买,绝不改行。”我说:“怎么罚了几个月的苦力,还不改行呢?”他说:“为人不会窃盗便罢,只要学会了偷盗,无论如何也改不了行。都说老荣(小偷)这行儿是只贼船,只要上去就休想下来。”

我问:“高买们有偷不了的商店没有呢?”他说:“这些年来,有些家大商店因为被偷的东西太多了,损失血本,他们害了怕。有人给他们出主意,叫他们花钱雇佣高买给他们当保镖。他们雇个人每月花个几十块钱,可以不丢东西,都很愿意。自从有商店雇佣高买保镖以来,高买们就有些家商店无法去偷的。”我问:“他们高买为什么不过偷窃的生活,给人家保镖呢?”他说:“高买这行人都是打走马穴(xué)(走一处,不能长占,总是换地方挣钱,江湖人叫走马穴)的,今天在天津,明天往大连,可以不遭官司,不能破案。有些个高买因为某处有了袢(pàn)簧果(管有搭姘头的妇女调[diào]侃儿叫袢簧果)将他吸住了,总在某地偷窃,永远不走。有了这种事情,日久了,老柴们就能知道他是高买。他屡次偷窃,屡次破案,闹来闹去,闹得他臭了盘啦(都知道他是高买,臭名昭著了),偷窃是不成了。往外省去又舍不得袢簧果,因此他与某地认识的人也多了,就有人将他荐入某商店充做保镖的。凡是给商店充做保镖的高买,都是臭了盘儿的。”我说:“商店有了保镖的还丢东西不丢呢?”他说:“也是不断地丢东西,不过比没保镖的丢得少些。”我问:“怎么有保镖的还丢东西呢?”他说:“有些高买不认识保镖的,有保镖破坏或示意不叫他偷,就偷不成了。倘若有那认识保镖的高买,彼此一碰盘,人有见面之情,保镖的宁可得罪商店,也不敢得罪同行;不惟不拦,反倒帮着高买给他当护托(不叫人瞧见高买是怎么偷的),叫他偷点就走,但是不能老偷,不能空手,点到而已。倘若保镖的得罪了熟盘(熟脸)的高买,不是找高手大偷特偷,就是遭了官司的时候咬上保镖,将他拉入案内,也得受他们大害。贼咬一口,入骨三分,也是得防备呀。”

我听他说的话,感觉着世上的人,学好事,入正道,是难极了;学坏事,入邪途,是容易的。他们已入邪途的人说邪途叫“贼船”,上去就下不来。这邪途够多么可怕!我老云是愿入于邪途的人千万别上贼船,宁可难走些,还是入正道吧。





黑红宝、花页子


在民国四五年,天津的三不管(天津市南市的一个露天市场)、北开最热闹无比,每天一出太阳,要是到了三不管、北开,就能听见签筒子乱响的声音,那耍签子的摊儿几步就是一个。每一个摊都是上边摆着两三盒纸烟,几堆铜钱。北开是个小地方,那露天市场里也有三四十个摊子;三不管的露天市场,也有七八十个摊子。凡是耍签子的人都是些地方的无赖,他们这些个穷光蛋,成天价晃悠签筒子,净骗些乡下人与手艺买卖铺的学徒,和他们赌钱是没有赢的。还有一样不好,动不动就打架,哪天也有头破血出的;甚至于有几十人群殴,演出大流血事儿,也有打出人命的时候。他们是一种流动性的赌博,如若官家来拿,四面八方都有人巡风,较比电报、电话还快,官人没到,他们暗令子已到,眨眼之间如鸟兽一般四散分逃。官人来了也拿不着一个。他们的暗令子是两个,有时候一齐喊嚷“窍……”,有时候喊嚷“扯……”(窍和扯都是逃跑的意思)。还有一种特别的技能,如有地方军警从他们摊前经过,他们一回手将签筒子往屁股底下一夹,似有如无,走起路来如同没夹着东西一样。我对于他们的夹劲是真佩服。我向江湖中的老人问过:“怎么三不管、北开有那些摆地赌?”江湖的老人告诉我:“不论哪里,如若有他们这些赌徒,说行话,那里就算‘杂巴地’。他们的行为如同路劫一样,可恶已极。但是在从前,清季那时代,在三不管、北开两处,该管的地方不严加取缔,每月暗中享受彼辈之供奉,纵容杂巴地有无赖、地痞、流氓,聚众害人。那时的黑幕是不问可知了。每日三不管、北开都有抽签的,到了年节,临时又添上骰子宝儿、黑红宝、六门宝、四门宝,哪个赌博摊儿也围个风雨不透。可怜一些商家的徒弟,年节放假,掌柜的给个块八角钱,不知买些正经东西,都被杂巴地的赌儿吸住,将钱输光了为止。”以我老云目睹杂巴地(赌徒聚居地)的情形,那些赌徒只能欺骗知识幼稚的年轻人、乡下老赶、工家的徒弟。稍有一点知识的人一看就能醒攒(cuán)儿(明白过来),绝不能受骗的。

他说:“押黑的一个赢一个,押红的一个赢仨。”有他们的敲托(暗中帮助做生意的人,也可称为贴靴的)的假装不认识,掏出钱来就押,押黑也赢,押红也赢.叫那些看热闹的人,瞧着眼馋,伸手就赌。



他们杂巴地的赌具都有腥儿(假的),签筒子有签子上灌铅条子的,有双层底的,有用线拴着的。那黑红宝的腥儿分为三样:有一样是小竹筒的,底下没有口儿,上边是个斜形口儿,筒内放个小竹管儿,那管的一头有块红的叫红宝,有块黑的就叫黑宝。如若耍的时候,赌徒左手攒(cuán)一个筒儿,右手拿着两个小竹管儿,一黑一红,来回乱晃。有人围着看时,他故意让人看他那红的竹管儿,插入筒内,格外还用根竹签子往竹管上一插,然后用手指着那盘上的黑红准点。他说:“押黑的一个赢一个,押红的一个赢仨。”有他们的敲托(暗中帮助做生意的人,也可称为贴靴的)的假装不认识,掏出钱来就押,押黑也赢,押红也赢.叫那些看热闹的人,瞧着眼馋,伸手就赌。可是不会打麻将的人,要打麻将不成,要赌也得下功夫学些日子才能学会;惟有这黑红宝是个人就能看会。除了瞎眼人之外是谁都会赌。还有一样便宜,叫人看着他往筒里装竹管儿,装的是黑的,装的是红的,容易学会,还觉着容易赢钱。可是有人一押就输,明看着是装了红的,取出来就黑了。只许赌钱,他那筒子管儿别人要看看可不成,总在他们手里攥着,你要非看不可,他们就和你打架,他们人多,打完了一散,简直没处诉冤去。还有一种黑红宝,也是小竹管做成黑红色,往竹筒里插,竹筒儿两头有口,从两头可倒出竹管来。其骗人的方法,与我上面说的一样,不过赌具的形式不同而已。还有不使竹筒儿的,使用两块竹板,长有七八寸,宽有二寸,薄有一分多点,板的正面涂成黑红色(其涂色之处,在板的中下部,例如八寸长,涂五六寸的地方)。他用手拿着两块竹板来回乱翻,使人忽看正面,忽看反面,冷不防地撤去一个,攥住一个,在他临攥住的时候,故意叫人看出是黑,是红。如若有人押黑,翻过来就红了;如若人押红,翻过来就黑了。这种黑红宝,样儿不多,就是这三样,骗的人可没数了。有一次某官署捕获杂巴地(赌徒聚居地)的赌徒,获有赌具,我老云托人介绍得入官署看着了赌具,及至看完了,才知道黑红宝的腥儿(假的)是怎么回事。我将这黑红宝的毛病说出来,阅者便能了然,那三样的黑红宝我就说一样,其余的那两样,也是大同小异的。那两块板用竹子做的黑红宝,竹子修成七八寸长、二寸来宽,用颜色染了黑红点儿,其黑红色染成一寸多见方,那板按八寸计算,其色染也五六寸之间,叫人看着黑的改不了红的,红的改不了黑的。其实那板儿是黑的也能改红的,红的也能改黑的,别看板儿虽薄,还是空的,那颜色也没染这空板上,里边另有个心儿,那心比空板还薄,长有六寸,宽有一寸七八,每一个心板,染成两种颜色,染在其板三四寸一样颜色,五六寸一样颜色,总要一黑一红就成。将心板装在空板之内,不知者以为那颜色染在空板之上,绝猜不透板内有极薄的心板。譬如有人看见一个竹板是露着黑色,要压他的黑宝,他用手一倒,那心板移动了就变为红色的了。其板中心的地方都是用刀刻成方孔,中间刻空了,名为空板,其板心为红黑,如将心儿装在空板之中即成红黑双面,黑宝如遇红宝时,将宝竖起心儿下垂,黑色隐而不见,露其红色了。其竹筒内的黑红宝,筒儿与空板相同,竹管的心儿与薄片的心儿相同,使用的方法一样,赌具的形式不同而已。

有一天,我老云走在×××地方,见有某甲,身穿短衣,蹲在地上,面前放一块粗厚的麻袋皮儿,上放有三张扑克牌,一张是八,一张是十,一张是小人的,他蹲着用两只手来回乱倒(dǎo)换,嘴里不住地说:“押着小人一个赢仨,押一毛赢三毛,押一元赢三元,押……”喊嚷不止。那个地方是三岔路口,每一路口站着一人,给他们巡风,专瞧有官人来没有,还有的三四人,长得都是凶眉恶眼,也往地上凑合,或蹲,或立,指手划脚,引得过往行人无不注意。我老云就知道这几个人是他们的敲托(暗中帮助做生意的人,也可称为贴靴的)的。我见他们像蜘蛛似的,织好网啦,净等着苍蝇飞来了撞入网中,我老云也没事儿,要看个究竟。工夫不大,由西边路口来了个人,看他年岁还不到二十的样子,手里提着一个钱袋,好像商号的徒弟出来讨账的。他走到那三岔路口儿,有他们敲托的,迎着这徒弟,用手一指那三张牌,大声说:“我要押一块可赢三块!”那学徒的两只眼睛随他指处一望,站住了不走。就见那蹲着的人用两只手乱倒那三张牌,或仰,或扣,叫人看那小人牌放在了中间,他说:“押着带小人的一个赢仨。”那押的人就蹲在前面掏出两元钱,说:“我押当中这张。”翻过来一看,果然是小人。当时就掏六块钱。连三并四的,眨眼就赢十几元。那学徒瞧着眼馋,也蹲下去了,被敲托的哄了几句,就由口袋里掏出洋钱来赌,连输了五回,三十多元都输了,一回也没押着,输得他顺脑袋往下流汗。正在此时,那巡风的故意喊嚷:“警察来了!”他们八九个人就乱窜乱跑,一哄而散,那学徒的提着空钱口袋,两只眼发直,急得要哭。我老云过去问他:“你在哪里做事呀?”他哭丧着脸说:“我在崇文门外花市×条×××号学徒。”我说:“你出来干什么呢?”他说:“我出来给柜上要账。”我说:“你输了多少钱哪?”他说:“三十七元。”我说:“你是个买卖家学徒,知识浅薄,没有阅历,叫‘做花页子’(倒换扑克牌)的给骗了。你赶紧找亲友借钱,把柜上的账补上,你不用找了,他们都没了影啦。找着那些个亡命徒,你也打不过他们。从今以后,走在街上,是便宜别贪,也就不被害了。”他被劝得无法,用两只手揉着眼睛,哭哭啼啼去了。

那就是滚地赌,做花页子(倒换扑克牌)的骗人钱财的情形!望社会里的人士有子弟出来办事,先嘱咐好了,走在路上瞧见了便宜,别贪才好。商家的经理人对于柜上徒弟,何妨将这做花页子骗人的事儿说说,也能遇见了这事不受骗。我对于社会有益的事褒之,有害的事设法揭穿他们的黑幕,以免社会人士被骗!





江湖中之挑粘(tiǎo zhān)汉儿的


在各市场庙会里常见有一种摆摊子做买卖的,他那摊上摆的有一个洋瓷盆,里边烧着一盆硬炭,其旁放着几匣药棍,长有三四寸,粗细儿较比洋火棍还粗些,有红的,有黄的,有白的,有绿的,有紫的,有黑的,有蓝的,还有些破烂瓷器,他摊上有个招牌,写着:“××记粘瓷器药,专粘粗细瓷器,当面试验,管保来回,不效退钱。”他们干这行的,都带着三分手艺,没人买他的东西,他用炭烫破瓷器,烫得热了,将那药棍往破口上一抹,两块对着一粘,立时就能粘住。他随粘随说:“哪位要有碎了的瓷器、茶碗、盘子、碟子、瓷瓶、茶罐、帽筒,只要是瓷器就能粘,如若有了这些东西,你就买几棍瓷器药,拿回家去,往抽屉里一放,搁不毁,放不烂,用着了拿出使用,要找锔(jū)碗的还得等从门前过哪!每根三大枚,又贱又便宜,认准了招牌,记住了字号,使用不灵,有发票为凭,管保退钱。”

他们这样说,又当面试验。“眼是观宝珠,嘴是试金石”,谁看见这粘瓷器的药又方便又贱,谁不买呀?在民初那几年卖粘瓷器药的最多。我还觉着锔碗的那行儿要遭劫,叫他们给顶了呢。不料这些年,锔碗的还是照旧挣钱,卖粘瓷器药的可就不挣钱了。我向某江湖人问过几次,怎么卖瓷器药的也少啦,也不挣钱了?某江湖人说:“他们这行儿,说行话叫‘挑粘汉儿的’,他们那药是半腥半尖(江湖人管半真半假调[diào]侃儿叫半腥半尖)。”我问:“那药怎么算半尖哪?”某江湖人说:“他们那粘瓷器药要粘瓷器,真能粘住。要粘茶罐、撢(dǎn)瓶、帽筒、大黑盘,就算粘住了,也不‘缓托’(江湖人管粘住了的瓷器又开了调侃儿叫缓托)。如若粘了茶壶、茶碗、饭碗,当时粘得挺结实,只要不使用,算是件东西;如若一见热气,由哪儿粘的还由哪儿张开。如不缓托就是真正好东西,他不冤人的;如缓了托就不是好东西,他们冤了人啦!故此这东西算是半腥半尖。”他说到这里又向我解释道:“他们挑粘汉儿的生意不大挣钱有两种原因:一是他们那药怕见热气,谁家的东西也买来使用,不见热气的东西能有多少?除了茶缸、撢瓶、帽筒、大果盘之外,件件瓷器都得见热水,若是缓了托,买主便觉着上当,嘴上宣传,买主就少了。因为缓托没人买,又因为有人嚷上当都不敢买,故此这行买卖日见衰落。”

我问某江湖人:“他们这粘瓷器是什么东西做的?”那江湖人说:“那药是用洋干漆掺颜色做的,见了热气儿才爱缓托。”望社会上的人士,要‘肘粘汉儿’(管买粘瓷器药调侃儿叫肘粘汉儿。那个肘字,在江湖春点里是个买字),净粘茶叶罐、大撢瓶、帽筒、大果盘,千万别粘带热气的东西。我老云还告诉一声,不带热气的东西粘好了,也怕六月暑伏。最好,粘过了的东西每逢暑热的时候重新另粘一回,免得粘汉缓托(瓷器药失效),摔了你们的大撢瓶啊!





江湖中之挑(tiǎo)杯杯的


我老云因事赴济,在某市场见有一群人,约有三四十个,江湖人称为小粘(nián)子。我挤进去瞧瞧,见有年三十许男子,摆设地摊。摊上有铁匣一个,水壶一把,小酒壶八个,另外有黑色酒杯数十个,红色酒杯数十个,摊上有些个角票、铜元。老云看着不懂他这生意是卖什么的,挤在人群里不走,看看他卖什么。那摆摊的男子说:“众位你看我这东西。”说着用手指那红色酒杯:“出在云南朱砂井,名字就叫香砂杯。这里头也没有麝香、牛黄、狗宝,就有几十味药材泡制的。有什么药材哪?有沉香,有木瓜,有豆蔻,有丁香,有杜仲,有槟榔,有陈皮,有肉桂,有……偏方能治人病,草药气死名医。这个酒杯又是个小玩艺儿,喝酒可以当做小酒杯,只要将酒倒在杯内,酒浸杯内药性化开,和喝药酒一样,能治偏正头疼、风火牙疼、筋骨麻木、腰酸腿疼、心里膨闷、肚疼胀饱、打饱嗝、吐酸水、跑肚拉稀、红白痢疾。买我这个酒杯倒酒喝,管保好病。这个杯子,虫子不吃,臭虫不咬,搁不坏,放不烂,多时都能使用。花钱不多,治病不少,买到家去行个方便,结个人缘。那位说,你这个朱砂杯卖多少钱一个哪?卖一毛钱一个。今天我初次来到贵宝地,我为传名,不要钱,多传名。这就是小不去,大不来;名不去,利不来。我卖一毛钱两个,有个小病喝了酒就好病,也不用扎针、拔罐子、贴膏药,连请先生带开方全都有了。哪位要,哪位说话。”他说着伸手提起铁壶往红、黑酒杯里一倒,真奇怪,那朱砂杯的酒能变成黑颜色,那黑酒杯的酒反倒成了红颜色。他说:“众位看见了没有?酒是白的,斟在红杯是黑颜色,斟在黑杯是红颜色,那是酒浸药性发了,碰着朱砂就是红的,我这黑杯,要没有朱砂怎么能红啊?这就是朱砂的力量,它成了红的。朱砂能定心神,避邪气。那红杯怎么能斟上酒变成黑色?那是丁香、豆蔻的力量。丁香、豆蔻能止呕吐,开胃口。”说到这里,他就让大家尝尝,还说:“眼是观宝珠,嘴是试金石,真金不怕火炼,好货不怕试验。喝到嘴里,尝尝滋味儿。”于是乎这个也喝,那个也喝,先尝后买,知道好歹。就有些人买那朱砂杯。我也觉得便宜,用一毛钱买了两个,一红一黑,带回家来,放了一个多月,果然那东西没坏。我买了四两烧酒,斟在杯内喝点尝尝。不料我斟在杯内的酒,也不黑,也不红,还是白白的酒色。泡了一个多钟头,仍然是不变颜色,喝到口内一点药味也没有,还是烧酒味。我赌气子,将杯摔碎。“摔杯为记”,使舌头舔了舔,还是没味儿。我虽不知他那“底啃(kèn)”(江湖人管做东西的原料调[diào]侃儿叫底啃。譬如,膏药是油熬的,那油便是底啃)是什么,就知道“受了腥啦”(江湖人管上了当,调侃儿叫受了腥啦)。我知道那卖药酒杯的是蒙人的生意了,便向江湖人探讨那是怎么回事。有个江湖人告诉我,那种生意叫“挑(tiǎo)杯杯的”,他那酒杯斟上酒,黑杯酒色能红,红杯酒色能黑,不是杯的药色,是他们的“样色(yàng shǎi)”(江湖人管使个手彩调侃儿叫使道样色)。做那种生意也得投师,先学说话,圆粘(nián)子(招徕观众),“捋粘啃(lǖ nián kèn)条子”(说病原、说病调侃儿叫捋粘啃条子),将那前棚(场上)钢口(说话的技巧和分量)学会了,再学“催啃(kèn)的钢口”(管推销货物往外多卖东西调[diào]侃儿叫催啃)。其挣钱多寡由他的翻钢叠杵(用各种话挣出几回钱来)的本领而定,能多挣钱了,都说他后棚的生意硬;不能挣钱,都说他后棚生意软。能圆粘(nián)子(招徕观众)、捋条子与挣钱无关要紧。在早年,社会的风气不开,人物朴实,做这挑杯杯生意都能蒙得住人,挣得了大钱。到了现在呀,人都精明了,信他们这一套的很少。干这种生意仅仅能糊口,不能火穴(xué)大转(zhuàn)(挣了大钱了),也因时代落伍受了淘汰了。我问过江湖人,挑杯杯斟酒变颜色的样色(yàng shǎi)如何使法?江湖人说:“和变戏法一样,以能叫人看不透为妙;如若叫人看破,调侃儿叫‘泡了托’(也可说抛了托),也算寒碜哪。”





江湖中的骗术倒(dǎo)页子


我老云有个朋友在天津读书,现年二十四岁,他家住在乡间,是个土财主。他的父母就生他一个,娇生惯养,放纵成性。他叫孟学仁,读书没有什么成绩。你要问他,天津电影院共有多少家?谁家的片子好?他全知道。哪里有暗娼?哪里有赌局?他无不尽知。他还有个技能,吹打弹拉,样样能成,唱几出西皮二黄,很有点名伶的味儿。那位孟学仁不住学校的宿舍,住在租界的旅馆内,花天酒地,任意而为,衣服阔绰,挥霍金钱,惹人注意。

一日,他早晨起来在院中漱口刷牙,隔壁房中的客人也在院中刷牙漱口。他见那人长得中等身材,白白的面庞,五官清秀,黑若刷漆的头发,留着美式的分头,穿着一身新做的西服,约有二十多岁。他与那人一对眼光,彼此点头,漱完了口就谈起话来。那人说话是南方口音,自称姓黄双名子荣,系沪某洋行买办,到津来办私事。两个感情冲动,越说越投机,一同去吃饭洗澡,晚间嫖娼。黄子荣挥霍自如,金钱众多,较比孟学仁手中还富裕。孟学仁很羡慕黄某多资。有一天两人在旅店谈心,孟某问黄金钱的来源。黄说:“我有个朋友专做假钞票,管保使用。此次北来,即是来找朋友趸货二十万钞票,已然派人运往上海了,在天津再游玩个月就回南方。”孟问:“我要趸(dǔn)个几千元假钞票行不行呢?”黄说:“我的朋友尚有五千钞票,他不愿卖了,愿意自己使用。”孟说:“你替我疏通疏通,叫他将那五千元的票子让给我得啦!”黄子荣说:“你明天听话儿,我给你问问。”说着由他身上取出五元一张的钞票整整的五张,交给孟学仁,叫他出去试用,看看这假钞票能否受使。

当日,黄某去找他的朋友给孟学仁疏通。孟学仁拿着二十五元假钞票(阅者注意:那二十五元票子是真的,黄某故意说是假的冤孟某)到外边去看电影,又去听戏,以后又到饭馆吃饭,在他临往外使的时候,心里还觉着亏心,忐忑不安;及至花着受使,他胆量就壮起来,吃喝玩乐,花了一天,还剩十几块现洋回来,高兴得手舞足蹈。天黑了,黄某回来,他问疏通得怎样。黄说他见了那人,商量此事,人家很不愿意卖,经他再三地说情,人家才点头。“可是他这票子都是五元一张、十元一张的,要五百元才肯卖哪!你能要吗?”孟学仁说:“能要。我在这柜上还存着六百多元哪!”黄某说:“你看样子不看?和我那东西一样。”孟说:“不用看了,我花着票子很受使。”黄说:“你预备钱吧,我带你到那里取货去。”孟说:“我这就到柜上提款去。”说着走出去,直奔柜房,由管账先生那里取出五百元,回到屋中。黄说:“我刚才给他打电话了,他叫我们到他住的饭店里取货。”孟便将五百元交给黄某,黄某查了查数目,装在一个皮包之内,携着出了旅馆,坐着洋车去找那卖假钞票的。到了一家饭店里,走至×号房,有位客人长得又黑又胖,一望他那样子好像个大富贾,不像卖假票子的。三个人见了面,黄子荣指着孟学仁向那人介绍道:“这就是我盟弟,你们二位多亲多近。”那人与孟学仁握手行礼,然后落座,一方交款,一方取款。那人将五千元钞票当面交付,孟学仁打开头一卷查看,刚看了两三张,忽听外边有人喊嚷:“查店!”那人也觉着不安。黄子荣以目示意,叫他将票子收起来,孟学仁就不敢再看了,全都装在皮包之内。黄子荣向孟学仁说:“乘着查店的官人还没查这屋哪,赶紧回去,免得被官人查出来麻烦。”孟学仁就提了皮包与黄同乘洋车回归。及至回到自己住的旅馆,那黄子荣没到,不知他什么时候岔了道了。到了屋中,打开皮包取出票子一看,只有头一卷有三张五元的钞票,余者尽是纸店里卖的酆都省银行冥钞。孟学仁大惊,他再找那黄子荣啊,简直没处找了。孟学仁受了骗,烦闷已极,见了谁向谁提说此事。

我老云按着他受骗的情形,向江湖人打听是怎么回事。某江湖人说:“这种骗财的调(diào)侃儿叫‘倒(dǎo)页子’。他们是十数人组成一帮,分住在旅馆、饭店之内,有给他们当耳报神,专给他们把(bǎ)点的,看看谁有钱,够上当的资格,就告诉他们的掌穴(xué)(这一伙人的头儿)的,掌穴的便派他的羽党做成圈套,设法骗财。他们也是打走马穴(xué)(走一处,不能长占,总是换地方挣钱,江湖人叫走马穴),今日在东,明天在西,骗过了的地方不敢再去,怕被骗的人撞见。这种骗人的生意是不能走回头穴(江湖人管去过的地方再去一趟调侃儿叫回头穴)的。”

他们十数人组成一帮,分住在旅馆、饭店之内,有专给他们把(bǎ)点的,看看谁有钱,够上当的资格,就告诉他们的掌穴(xué)(这一伙人的头儿)的,掌穴的便派他的羽党做成圈套,设法骗财。这种骗财的调(diào)侃儿叫“倒(dǎo)页子”。



孟学仁受骗,不能光怨做“倒页子”的,也是他贪便宜上当,叫人家给冤了。我老云常说,人在社会里,只要是见便宜不贪,就没当上,贪便宜才能受害。我写这种倒页子的生意,将其黑幕揭穿,贡献于社会,使各界人士不能再受彼辈之骗,也是我的爱护社会人士的好“攒(cuán)”(好心)。





江湖中挑(tiǎo)黄啃(kèn)的骗术


我老云有个朋友叫马文田,住在津埠某租界,家中颇有几十处房产。在这年月,他们“吃瓦片”的每月收取租金,如同铁杆庄稼一样,十数口人,衣食丰足,人间天堂,快乐无忧。不过我这位老弟聪明过人,是亏不吃,是当不上,交际最广,哪界都有朋友,也常给人调停个事,他的手眼通天,大事化小,小事化无,提起马文田三个字来,在民国初年的时候,天津几乎无人不知。那时候天津的西城根修电车道,修马路,已无江湖人了。地道还没有杂技场哪!露天市场南有三不管(天津市南市的一个露天市场),北有北开,走闯江湖的朋友都在这两处做生意。

马文田有一种毛病太不警人,他对于江湖的黑幕知道不少,常在人群里给人家“熏生意”,不是哪档生意蒙人,就是哪档生意骗人,日久天长他得罪了不少朋友。有天他带着几百元钞票往南市办事,走在翠柏村的南口,忽见对面来了一个人,穿着儿好像个仆人,满脸的愁容,透着又急又忙,见了马文田向他问道:“这位先生,可曾看见有个姑娘坐着洋车,拉着褥套过去没有?”马某说:“没看见。”那个仆人急得直跺脚,眼泪在眼圈里打转。马君觉得这人必有紧急的事儿,忙问道:“朋友,你有什么事?至于这样着急?”这仆人见问,先往四下里看了看,然后悄悄地说道:“我是当仆人的,吃长安路的饭有个十几年啦。我们主人在前清户部当差,掌理银库,家中很阔。我将使唤丫头桂红拐出来,偷了些珠宝,要往××地方将东西卖了,弄个安乐窝过快活的日子。我们怕有人追下来,商议好啦,分开了走,她在前边,我在后头。方才我们下火车的时候,她坐洋车拉着东西,我在后边跟着;刚出站,我看见个熟人,怕露出了破绽,我向桂红说,你先走,×租界××客栈等我。不料我到了栈房,伙计告诉我,她已然走了,我追了半天也没追上她。”说着话还急得了不得。马君说:“你这人真是死心眼,不会雇辆车追她吗?”这仆人说:“我没带着钱,钱都在她身上哪,我怕她上了火车,那可就糟了。”马君说:“你身上一点钱也没有吗?”这仆人说:“钱我倒没带,我带着点东西,可以变卖几个钱。”马君说:“你带的是什么东西?”这仆人说:“生金子。”马君说:“你掏出来我看看。”这仆人又往四下里看看,由怀中掏出个绸子包儿打开了,马君一看那包内黄澄澄的净是生金子。马君问道:“你要多少钱啊?”这仆人说:“四十多两,能值两千多块钱。”他们俩人说着话,旁边又凑过来一个人,穿得很阔,像个富家翁的样子,说:“你们二位嘀咕什么?”这仆人又将他的来历说了一遍,那富翁说:“我看看你那货物。”这仆人打开包儿叫这人瞧着。这人说:“你这东西不假吗?”这个仆人由身上掏出一个铜子,往那金子上一打,嗞嗞的直冒火星儿。那富家翁似的人说:“你有多少都卖给我吧!”这仆人说:“我有四十多两,能值两千多元,你都留下给多少钱?”那富家翁说:“我给你两百元。”这仆人说:“两千多元的东西,你给我二百元差得太远,我可不卖。”这富家翁说:“你不卖?好吧。你在这儿等着我,我叫你知道我的厉害。”说着匆匆而去。这仆人吓得揣起东西就要走,马君一把将他揪住,说:“你不能走。你要走,我叫官人把你押起来。你要好好地卖给我,咱们没事。”那仆人急道:“你给多少钱?”马君说:“我就带着三百五十元,你卖也得卖,不卖也得卖,要不然我叫你遭官司。”吓得这仆人无法,十几包生金子都卖给他,拿着三百五十元,一溜烟似的去了。

马爷花了三百多块大洋,买了四十多两生金子,欢天喜地走回家去。到了家中,将生金子都收起来,只拿着一包儿,往金店里去看成色。他打开包,金店伙计一看就乐了。马君说:“你看见好东西也乐呀!”伙计说:“我乐的是你这东西!”马爷说:“我这东西怎样?”伙计说:“你说的是什么东西?”马爷说:“生金子。”伙计笑道:“这是自然铜。你到药铺去吧,一毛钱就买这一大包。”马爷这才知道上了当。又跑到药铺买了一角钱的自然铜,与自己买的生金子比了比,是一般不二。他至此才相信被人所骗,好几百元买了些自然铜,他焉能好受,要找那骗子和他们打官司。他天天往各处去找,找了半个多月也没找着,日期久了,他的气就无形消灭了。

有一次我们两个遇见,在天祥商场闲聊大天,他听人说我老云懂得江湖术,将他花钱买自然铜被骗的事儿向我说了一遍,问我这是什么生意?我说:“这卖假金子的我还没见过,我也不懂。”我知道江湖中有卖假金子骗财的事,就常向江湖人打听。有个老江湖人说:“卖假金子的行当,江湖人叫‘挑(tiǎo)黄啃(kèn)’的。”我问:“他们挑黄啃的怎么骗人?”他说:“做挑黄啃的生意也颇不易。他们也有组织,至少也得四五个人,多了十几个人。他们那装男仆人的,说行话叫‘掌买卖的’,得有把(bǎ)点、把(bǎ)杵、抛苏、亮托、换托五大本领,才能掌买卖哪!”我问:“什么叫把点呢?”他说:“挑黄啃的一帮人出来,都随着掌买卖的走,往各处骗人。至于谁像被骗的,谁够被骗的资格,全仗着掌买卖的人两只眼睛瞧人行事的本领如何了。他净瞧出谁能受骗来还不成,得有把杵的技能那才成哪。”我问:“什么叫把杵哪?”他说:“掌买卖的看着某人,像个被骗的,那人虽愿意被骗,身上无财,也是不成啊!他们的出奇本领是只要和谁走对脸,往谁身上脸上一看,就能知道谁有钱没钱。可是他们这种把杵的本领与马贼把杵不同。当初马贼在我国交通不便利的时代,专在大道上留神。如有发财回家的人,身上带着金银财物,步行看双足,由脚印分轻重,扬土气知多寡;乘马而行,也看马之四蹄,由蹄印的轻重,扬土气,分多寡。惟有挑(tiǎo)黄啃(kèn)(卖假金子的)的把杵不是这样,是瞧人的面貌、神气,能看出身上有无巨款。如若要骗此人时,得有发托卖像(通过喜怒哀乐使人上当),会撇苏儿为妙!”我又追问:“什么叫‘撇苏儿’哪?”他说:“做生意骗人也要和逢场做戏一样,面貌上能够形容喜乐悲欢。挑黄啃掌买卖的向人假装问路,面上露出急忙惊恐之状,那就是他的发托卖像,以假作真;最好的有能二目落泪的,行话叫做撇苏儿。形容好,能使人相信,才上他们的当哪。”我问:“什么叫亮托哪?”他说:“那被骗之人要看金子真假,他从怀中取出一包儿真的叫人看,那叫亮托。”我问:“什么叫换托?”他说:“被骗之人愿意要了,他将真的留下换上自然铜给了人家,在这一倒(dǎo)换之间可以闹鬼,行话叫做换托。”我问:“如今这做挑黄啃的生意的人能有多少?”他说:“做那生意的也是打走马穴(xué)(走一处,不能长占,总是换地方挣钱,江湖人叫走马穴),今天在东,明天在西,骗了人立刻就走,不惟不敢在一个地方不动,并且还得别回来。如果走回头穴(xué)(江湖人管去过的地方再去一趟调[diào]侃儿叫回头穴),碰上被骗之人就得遭官司。挑黄啃的生意在早年以各大码头最多,现在因为时代变迁,多有改做倒(dǎo)页子(用假钞骗人的)的,如不改变,他们的骗人方法将来也怕不灵了。”

我听老江湖人将个中的情形说给我,我才知道卖假金子的叫挑黄啃的生意。在我少年读书的时候,将开知识,就想将天下的事全都知道了,及至我在中国各省市云游了这三四十年,将社会里面的事知道些个,才感觉着天地之大,包藏最广。一个人知识有限,天下事理无穷,非一人可能尽知。我是抱定宗旨,将我所闻所见的事,全部贡献到社会里,使人少受骗,不受宵小之愚,即是老云忠于社会人士之功。我的见解虽是这样,仍恐有人不原谅,讥我……话虽如此,忠实人仍然是表示同情的。





江湖中做平的生意


本年七月二十七日,报纸新闻中登有指医骗财的新闻:“东安门内河沿五十三号住户陈王氏,年五十许,北平人,家道小康,近日患病,久不告痊。是日下午一时许,有一某甲,年约三十余岁,面长,黄白脸,留平头,戴美式草帽,穿蓝纺绸大褂,冒称陈姓亲戚代请他来看病。陈氏当令其诊视,旋谓:‘病不甚重,系他医将药用错,但气已弱,不宜再服汤药,伊可代配丸药,服下去可保安全。’陈氏欲送车费,某甲称有令亲介绍,何敢收钱?陈氏以值此炎暑,颇不自安。某甲最后云:‘请将制药费赐下。’陈氏询问需款若干?某甲说:‘药资无几,仅十五元即可。’陈氏处此局势只可付洋十五元。某甲走去时并云三二日可送药来。后陈氏到亲戚家探问,并无其事,始知被骗,呈报官府请缉骗匪云。”

我老云看见了这条新闻,认为是骗匪骗财。不料日前与某江湖人谈及此事,某江湖人说:这也是一种江湖术。做这种生意的,说行话叫“做平的”。他们的组织也是好几个人,分为掌穴(xué)(这一伙人的头儿)的、使托的。那使托的每天专在各洋药房、各医院、各有名的中药商店刺探病家的状况,或装作购药,与抓药之人闲谈,说:“你这是给什么人抓药啊?”抓药的人随口答应说:“这是给我们太太抓的药。”他必问得的是什么病,有多少日子了,请过几位医生。抓药的人不知道他们是骗子手的踩盘子(探道儿的)的伙计,无意之中将病家的事全都说出。他们听着病家是有钱的富户,就要入手,再进一步向抓药人要簧头(要出他们想知道的事情),问:“你太太的病这些日子也不好,他们亲戚就不给荐个高明先生吗?”抓药的人说:“他们倒是有亲戚,在某处做什么事,就是没给荐过先生。”他们将病家的情形刺探明白,再看那药方,如今医生开的药方都很详细,病人是男性,女性,多大岁数,住在什么地方,他们都记住了,回去见了掌穴的说明。那掌穴的就装是个医生,提着皮包往病家去骗财。他的羽党们不在中西药店刺探事儿,就往各大医院买个挂号牌子,假装有病,在医院的候诊室内与病人们假作闲谈,暗着刺探病人家中的状况,能否骗财。如若觉着哪家可骗,就回去向掌穴的说明,掌穴的就敢至某家去骗财。这种做平的生意挣钱多少,全看他们掌穴的本领高低。本领高的,有胆量,胆子有多大,就能骗多少钱。





三不管的挑(tiǎo)大堆的生意


有一年,我住在天津南门里朋友家中,天天早晨起来往南关下头去遛遛。有一天多走了几步,走到三不管,见各场的凳子还没拉开,是做艺的都没来哪,靠东墙有一人站立,眼前放着个大包裹,鼓鼓囊囊的,也不知包的是什么东西。外边放着一个白线毯。那人穿着很小的衣服,愁容满面,好像卖东西似的。我由他眼前路过,忽从旁边过来一人,用手指着那白线毯说:“你这毯子给三毛钱卖不卖?”他这一指,把我的“招儿”领去(江湖人管领人的眼神见他们的东西调[diào]侃儿叫领招儿)。我看那毯子是新的,三毛钱真便宜,不由得往那东墙凑合,去看他们买东西。真也奇怪,我过去了,走路的人也过来看,眨眼之间人就围满了。那买毯子的人向那卖东西的说:“你这包内是什么东西?打开看看。”卖东西的说:“我不零卖,谁要买,都要才成。”买毯子的人急得直嚷说:“你整卖,你零卖,倒是打开看看,包着看不见,你怎么卖呀?瞧成色给价,隔山买老牛,谁知道个大小啊?”于是看热闹的人这个一言,那个一语,也叫他把包袱打开大家看货。他说:“我是河南人,在口外做事,今年回家买了些东西,还带着一百多块大洋,与我们的乡亲一同回家,叫他把我骗了,百数多块钱他拿着跑了,我就剩下这包衣服等项,要是卖了钱,有路费好回家;若是一件件的卖,随卖随花,东西卖完了我也回不了家。谁要买我的东西,都得要才成哪!”有好些人说:“你倒是打开了叫我们看看哪!你不零卖,我们看不见东西怎么买呀?”大家这样地催他,才把包儿打开。里面有:新被两床,都是里面三新的。两个皮褥子是狗皮的,面好板好毛也好。还有一个皮袄筒子,虽仅是筒儿没面子,真像一块玉,毛长色润,曲曲弯弯十分好看。有两条棉褥子也是里面三新。看的人们见那皮袄筒子都是新的,谁瞧着也值五六十元。有人问他:“你这东西卖多少钱哪?”他说:“卖四十元钱。”有个人伸手拿起棉被来说:“这两床被,连里带面值六块大洋一床。”又拿起皮褥子,说:“这两张皮褥子至少也值四块钱一个。”又拿起两个棉褥子说:“这里面三新,也值两块钱一个。合计起来,这三样东西也值二十四块钱,那皮袄筒子才合十六块钱。得啦!我就要这皮袄筒子,我给十六元。”卖东西的说:“我不零卖,你要就给四十元,把一包东西全拿走。”这人说:“我倒是愿意要,我没带着那些钱,只带着二十元,你要卖给我得跟我回家,再取那二十元。”卖东西的说:“不成。我不跟你取钱,耽误一天,我就到不了家啦。我算计好了,四十元钱的路费能够到家的。”这个人从腰中掏出来二十元的钞票,说:“这不是二十元吗?我没带够了。哪位要是愿意要买,咱们分开,我要皮袄给十八元,谁要那六件给二十二元。”有一个人答了言:“我要这六件给十八元,你要皮袄分给你,你得给二十二元。”两个人这个一言,那个一语,争持不下。旁边看热闹的有几个人直嚷嚷说:“真是便宜,净皮袄筒子就值六十元,那些东西也值三十多块,有现钱能买便宜东西,没带钱可干瞧着便宜。”这时候很有些人瞧着便宜,内有一个人从腰中掏出一卷钞票,数了数三十二元,不够四十块。旁边有个人问道:“你有心要吗?”这人说:“有心要,没带够了钱。”这人说:“不要紧,我借给你八元钱,我跟着你去取一趟。你在哪里住哪?”这要东西的人说:“我在船上。”那人问:“你的船在哪里停着哪?”这要买东西的人说:“在北大关停着哪。”那人说:“不远,我跟你取一趟吧,干吗也是交朋友。”于是他二人就凑足了四十元,给了那卖东西的,买了这八件东西,两个人就走了。

我老云好奇心盛,在后边跟着要瞧到底。他们两个人在前边走,我在后边跟着,直走到北大关,还没到河边上哪,恰巧那买东西的碰见个朋友。问朋友带着钱没有,他那朋友借给八元钱,像是有事的样子匆匆而去。买东西的人把借的八元给了那人,还直冲那人作揖,很感激不尽似的,那人接过八元钱也走啦。我跟着他到了河边,瞧他高高兴兴背着包袱上了船。那船上有个老人,向他问道:“你买了什么东西啦?”他说:“便宜东西。”说着将包放在船上,打开了一件一件地让那老人观瞧。那老人见了这些东西急得直跺脚,说:“你上了当啦!”那买东西的人说:“这么些东西才四十元,怎么上了当呢?”老人说:“干这个的人是七八人一伙子,有当家的,拿出本钱来叫他的伙计骗人。五六个人当‘避粘(nián)子’(即是贴靴的或敲托的)的,你叫‘挑(tiǎo)大堆’的冤苦了。”那买东西的急得直嚷,很不服气,觉着他没上当。那个老人是有经验的,用手拿起棉被扯开了叫他看,里边的棉花不是新的,全是旧棉花又弹了的,那被里被面对着太阳光一照,那买东西的人可就愣了。及至那老人一件件地都给他拆开,件件的棉花是旧的,布是最贱最不好的材料。那狗皮的板儿并不是整的,是皮局子做活使剩下的碎狗皮攒(cuán)的。他冲着东西发愣。那个老人说:“我买过这样的东西,也是这样的皮袄筒子,我做好了穿上不到一个月,那毛就擀成毡了,这种皮袄筒子是老羊皮做的,八块钱一个。西头的皮局子有专做这东西的。”

他们一起谈论这些事,我全记下来了。据那懂得世故的老人说,替他垫款的朋友也不是好人,是“挑大堆”的伙计。他先装好人,借给钱买东西,然后跟着取钱。倘若买东西的人家中看破了骗局,给买东西的人豁鼻子,上当的人醒了腔儿,要想不给那几块钱也都成;但那个伙计绝不承认是做大堆的一党,他是好人,热心肠儿好多管闲事。那种措辞,局外人不易看破。他们这伙骗匪挣了钱,大家“均杵”(平均分钱),能够久干就是这种原因。社会里的黑幕一层一层地揭,也难揭尽了啊!





江湖中的叫点儿(叫住他要行乞的人)内幕


我老云在每年冬季只要混上了温暖的衣服,在最冷的天气时常往外边闲逛。走在天津的租界或是中国的富庶的区域里,常见一种乞丐,头上没有帽子,上身赤着膀背,下身穿条单裤,向人行乞。别人穿着一身棉衣还冻得难受,他那样有多可怜哪!谁瞧见也得动心,人人能给几个铜子;有阔人瞧见,三元两元的一样周济。可是我这人好奇心盛,遇见了这种乞丐,我豁出冷去在他身后跟着,倒看他们能要多少钱。要完了钱,他去干什么?结果,有一个乞丐叫我看了个全始全终。

那天,他不到三个钟头要了有四元多钱,他不要了。我以为他是拿着钱买衣服哪,不料他也有家。回到家中,一会儿出来,也穿上大棉袍,棉马褂,棉裤,棉鞋,戴上皮帽子,出门也坐洋车。我真是莫名其妙。并且,他到了第二天又赤着膀背,穿条单裤向人要钱,求人可怜。我最纳闷的是那么冷的天,三四个钟头会冻不坏他!其中定有不怕冻的妙法,要不然就冻坏了。我把这事记在心里,没事的时候和江湖的朋友闲谈,偶然谈到这假乞丐不怕冻的事儿,有个江湖人某君知道其中的内幕。

某君说:“世上的善人虽多,可是善财难舍。普通要饭的乞丐,怎么说得可怜也没人注意,更没有肯多周济的。这种假乞丐,每逢冬天出去骗财。在未出去之先,他得买一斤好烧酒,一块红矾,在屋里脱下衣服来,用棉花蘸酒,往皮肤上擦红矾,擦完了之后,用极少红矾置于酒中,把酒喝下肚去。工夫不大,红矾与酒性均发了,那身上如火炭般热,再想穿衣服也穿不住了。他这样弄好后,往街巷里向行人要钱,有个三四个钟头绝冻不坏他。要完了钱,回到家中,可得穿上衣服往澡堂子大洗特洗,洗完了吃东西。照这样,干上几个月,那红矾的毒质在皮肤之上,一到了春天就会发作出来,能够像烂桃似的遍体鳞伤,久治不愈。”我向某君问:“他们知道将来毒大了有害于己吗?”某君谈:“他们哪能不知道?”我说:“他们知道将来有害,为什么还干那个呢?”某君说:“社会里的人都是顾眼前,不管将来。他们这种人也是社会中的败类,不务正道,成天价蒙事,弄得没有办法啦,才想干这个。他明知鸩酒喝多了有害性命,在没有解渴的时候也只可出此下策,来个饮鸩止渴,至于后来怎样他们先不顾及。可是干这不正经事的人都是很聪明,不过是假聪明自误终身吧!那真聪明的人没有这种举动呀。”

假乞丐每逢冬天出去骗财。在未出去之先,他得买一斤好烧酒,一块红矾,在屋里脱下衣服来,用棉花蘸酒,往皮肤上擦红矾,擦完了之后,用极少红矾置于酒中,把酒喝下肚去。工夫不大,红矾与酒性均发了,那身上如火炭般热,再想穿衣服也穿不住了。





雁班子之江湖术


清末光绪年间,有河南巡抚某因某案撄皇上之怒,将罪之,尚未降旨;该巡抚正欲运动,他闻有数十人住于城外某寺中,皆是北京口音,但深居不出,疑是朝中遣使来调查其事,阖城官吏无不恐慌。祥符县知事遣干役往寺前偷探,来者是何人?究有何事?县役在门前守候两日,不见有人出庙。一日清晨,寺门忽开,有一太监手提浆壶而出,县役尾随其后,至北关酒肆中,见太监买酒毕,提壶出店。县役上前作揖施礼,说:“老爷来取酒么?”太监怒视不语,匆匆归庙。次日又见该太监提壶而出,县役奔至面前,说:“老爷将壶交给我,往酒店买酒,我可代劳,何必老爷前往。”太监起初不肯,经县役说之再三,太监才将壶交与县役,代为取酒。从此,县役日日往代取酒。一日,太监自出,未携浆壶,县役随之至酒店,见其自饮,县役也进店就饮,随饮随谈,渐觉熟些。县役悄悄问道:“老爷至此伺候何人?”太监说:“吾主乃端王之子,今上之大阿(à)哥也。”县役又问道:“大阿哥乃国之储君,何以至此?”太监说:“因你们本省巡抚于某案得贿枉法,派吾主密来访察,如果是实,吾主归京。巡抚之罪不容诛也。”县役大惊。太监说:“汝一人知之,不可泄漏于人。倘若泄漏,吾命难保,千万谨记。”

县役容其归庙,疾行回衙向县知事禀明,县官也恐惧不安。未至两日,凡巡抚以下官员尽知此事。众无计,惟有重贿可免牵连之罪,皆具衣冠往该寺拜谒大阿哥。轿马车辆,喧嚣寺外,叩门不应,只听里面啪……有鞭扑声,呼号声,久之,不见动静。门忽开放,有二护军校抬一荆筐而出,筐内死尸一具,血肉模糊,县役追视,死者即与语之太监也。县役奔至知县前,将打死太监之事,又都禀明。众文武大惧,乘庙门未闭进了山门,膝行而前,见一侍卫大臣,头戴秋帽,珊瑚顶,孔雀翎,黄马褂,方颐广额,精神百倍,美胡须,约四五十岁。他见众人来至,忙用手指台上坐着的少年说:“爷在此,可行礼!”众官拜见。但见少年微欠身,小语数句,众文武听不清所说的话语,侍卫大臣向众文武说:“爷明日回京去。”众文武唯唯而退。至暮,巡抚遣心腹人至,献黄金万两,纳贿求免。次日天明,众官送行,大阿哥临行时,忽掷一纸于某巡抚,令回署再看。及至回衙,见巨幅大书“领谢”两字,始知受骗。遣人追赶不及而还。此乃清末实事。

清朝野史有插天飞骗财,即是饰侍卫大臣之人。我老云曾向江湖人探讨,插天飞等数十人组织的骗人团体,是否江湖伎俩?某江湖人说:那叫雁班子,又叫“雁尾子”,系江湖上风马(má,常读“麻”音)雁雀四大生意之一。他们也有掌穴(xué)(这一伙人的头儿)的、当展点(仆人)的、敲托(暗中帮助做生意的人,也可称为贴靴的)的,其内幕情形最为复杂,非局外人可知也。插天飞即是掌穴的,某省人,某总督同族者也。雁班子耳目灵通,专骗各省大吏,所骗金银数目之巨,也骇人听闻。今天春季,平津有某人诈称某军界之代表,向各方骗财,即风马雁雀之江湖人骗子,后竟被捕入狱中。江湖人说他是独角雁尾也。





雁班子之内幕


清末时,浙江蒋巡抚为官清正,闻各府县官员多有贪赃卖法的,遣人往各处严查。有数州府官因贪赃被查有实据,被蒋巡抚惩了。其余的府州县官吏有曾受贿的,俱都恐惧不安。绍兴府桂××曾受数十次贿赂,得款十数万元,彼为保持官职计,命其心腹数人在外访查,如有蒋巡抚派来暗查他的人时,禀报于他。在知府衙东有德隆老店,来有外客四人,都是北京口音,时常向店客探问该府官吏有无贪赃受贿事否。每逢知府桂××升堂问案时,他们也必往大堂前观瞧。不料桂××知府的心腹人窥破这四人行藏,料为蒋巡抚所派之人,禀于桂××知府。知府命他心腹之人昼夜往德隆店监视,且嘱他们:如该四人一齐外出时,速报他知。一日,恰巧该四客人俱都外出。桂知府得报,乘轿驰至德隆店,命店伙将该客所住之房开了锁,到屋中搜查其行李等物,见有蒋巡抚访牌一道,凡桂知府受贿之事,俱都详细载明;又有致山阴县令一封书信,启视信中,见笺上写有“蒋厅尊奉大宪命探事来绍兴,请祈照察”云云。桂知府见个人所做之事俱被四人访查真了,心中大惧,惟恐四人归省,失职受惩。匆匆回衙,遣人往山阴去请该县来议挽救之法,又命他心腹仍往店内查看四人动静。当日晚间四人归店,见其行李散乱,向店伙追问何人动他们的行李?店伙把桂××来查看之事说明,四人默默无言。次日早晨,命店伙雇了船只,用完早点就起身离店。桂知府得报,忙与山阴县令携带礼物追往码头。府县乘轿在前,八个家人抬四桶礼物在后。据说桶内是橘子,八人觉得桶的分量过于沉重,料其中必有巨金,往见四人纳贿托情。及府县至码头时,见该船中已剩三人,登舟时,问:“蒋大人何在?”三人齐说:“已乘小舟驰归省垣了。”桂知府与山阴县令向这三人致意:“蒋大人至此,未得招待,甚为抱歉。今有微薄之礼,乞代转交。”三人收下四桶礼物,桂知府与山阴县令才欣然而归,觉着一万两白银贿款已收,他二人官职不会动摇,也不会获罪了。

过了数月不见动静,始知钱能通神,蒋大人受贿不究了。有一次因公入省,桂知府往谒巡抚,见蒋巡抚待彼甚好,偶谈前事,探问:“大人曾遣人往绍兴否?”蒋巡抚答:“没有派人往绍兴去。”桂知府大骇,料万两巨款已被他人骗去。事已然过去,无法寻找,如哑巴吃了黄连,只好忍痛不言,也难测那骗子为谁,有此大胆!

该知府受骗事,系我老云朋友所说。我曾以此事向老江湖人探讨:骗知府巨款的人是否江湖人?某老江湖人说:骗官员的也江湖人也,他们这行儿叫“雁尾子”。或三或五,或数十人,组织一种骗人的团体。其中的领袖调(diào)侃儿叫“掌穴(xué)”(这一伙人的头儿)的,可是这个掌穴的人才极不易得。第一要相貌好,第二要谈吐好,第三得博学多才,对于政界的人物,政界的事全要明了。干这个还得有口京话,叫人看他的穿着打扮言谈话语像个北京的旗人(在清室时代旗官有权),才能叫人相信他是个旗官。他的伙计也得受过相当的训练,有专管探听政界各种消息的。有随着掌穴当“展点”(江湖人管当仆人的调侃儿叫展点)的。他们不天天出来骗财,不定几个月,或是几年出来一次,可是哪一次也能弄个万儿八千的。雁班子这行儿在江湖中是大生意,比较金、皮、彩、挂那些行做的事大多了。可是他们就永远别“朝(cháo)翅子”(犯了案,打官司见了官了),如若朝了翅子,哪个也有几年的徒刑罪在身上背着哪!如今常有些个“里腥(lǐ xing)海(hāi)冷翅子”(江湖人管假军官调[diào]侃儿叫里腥海冷翅子)私发委任卖官骗财遭了官司的,那就是要做雁班子生意得不着“拨(bó)眼”(江湖人管各种口传心授的秘诀调侃儿叫拨眼),骗术不精,财到手就叫被骗的人觉悟,那如何不遭官司?风、马(má)、雁、雀四大门的生意潜伏在社会里,因为他们有拨眼,犯案的时候最少。最奇怪的就是他们骗了做官的人,能叫被骗的人有苦难言,那种“拨眼”真是令人不可思议呀!





第九章 大鼓竹板


柳(liǔ)海(hāi)轰儿的生意


江湖人管唱大鼓的行儿调(diào)侃儿叫“柳海轰儿”的。据他们说,大鼓的起源是很早的,大约着有几千年了。在尧舜的时代,朝堂里设立谏鼓,虽是以下谏上,也是一种教化的意义。敝人向他们鼓界的人探讨,他们为什么都供周庄王呢?他们说:周庄王曾在古时击鼓化民,他们唱大鼓也是正风化俗,劝化人民的。本着周庄王击鼓化民的意思,就以周庄王当作祖师了。北平的各杂耍(多种形式的曲艺演出)场子、各坤书馆儿后台都有一张神桌儿,桌上设着个牌位,上边写的是“周庄王之神位”。他们的大鼓,若按规矩是应当有一百个铜钉,其中的意义是仿着文王百子图的。大鼓的鼓架子是六根竹子做的。据江湖人说,那鼓架子是穷家门(唱数来宝的)的东西,他们是借着使用的,到了鼓界里那架子的尺寸就失传了。唱大鼓的人身材高些,那鼓架子也高点儿;身材矮点儿的,那鼓架子也矮点儿。那板是木板儿,也有一定的准尺寸,如今也都不按着规矩做了,尺寸的大小随个人的心意啦。

唱大鼓的支派,黄河往南,山东、河南等地是孙、方、蒋、张四大门(一说孙、财、杨、张四大门),此外还有孙赵门儿;黄河北是梅、清、胡、赵四大门。

他们收徒弟的时候,在某处敝人曾见过一次,是由收徒弟之人先下帖,将本门中老中少三辈人全部请来,屋中也设摆神桌,供上周庄王的牌位,将弦子、鼓、醒木也都摆在神桌之上,临往桌上放弦子之时,嘴里还得祝上一套词赞是:“丝与竹来乃八音,三皇治世他为尊。师旷留下十六个字,五音六律定君臣。位按那宫商角徵,后有文武弦两根。祖师留下文武艺,弟子学艺入了门。老祖留下为有宝,虽然应手又趁心。四海朋友把弦供,如要有艺论古今。”供鼓的时候,供醒木的时候,也有一套词儿。到了把字儿(即门生帖儿)写好喽,大众给祖师爷磕头完了,新入门的徒弟跪倒磕头,嘴里得说:“盘古辟地与开天,伏羲始有八卦传。坎水离火坤为地,震雷巽风艮为山。兑泽中央戊己土,八卦西北乾为天。白黑碧绿黄赤紫,行藏至引圣神仙。宝顶呈祥吉瑞彩,香烟缭绕半空悬。庄王祖师上边坐,弟子进香到面前。”徒弟入门得给师父效几年力,先学弹弦后学唱,鼓界的老人都是会弹会唱,到了如今可不然了,有会唱不会弹弦的,有会弹弦不会唱的。

收徒弟的时候,由收徒弟之人先下帖,将本门中老中少三辈人全部请来,屋中也设摆神桌,供上周庄王的牌位……



海(hāi)轰儿的板儿,向来分为铁板、木板,腔调儿大不相同,有犁铧调儿,有靠山调儿,有梅花调儿,有西河调儿,有京调儿,有奉天调儿,有乐(lào)亭调儿,有怯调儿。犁铧调儿以山东人唱得最佳,唱那调儿的吃得很宽。江南北几处倒都是以鸳鸯档子为多(男女两个人唱对口大鼓,江湖人调[diào]侃儿叫“鸳鸯档子”)。靠山调儿是天津的士产,非天津人唱着不美,还是在天津唱着好听。梅花调儿是费力气不讨好,以北平人唱之最为相宜。其余奉天调儿、乐亭调儿,也是各行其道。

刘宝全、白云鹏唱的是怯口大鼓,美其名叫京音大鼓。架冬瓜、老倭瓜、大南瓜、大茄子等所唱的为滑稽大鼓,按早年海(hāi)轰儿没有这宗玩艺儿。唱滑稽大鼓艺人以老倭瓜最早,社会的人士都以为是他兴的这宗滑稽曲儿。据敝人所知,柳(liǔ)(唱)的最早是老倭瓜,响了万儿(有了名了)是老倭瓜,跑的穴(xué)眼儿(演出地点)最多也是老倭瓜;攥弄(zuàn nong)(创作)那种活儿可不是老倭瓜。老倭瓜姓崔叫崔子明,京北三旗营的人,原是玉器行儿人,他自幼好习大鼓,也先票后海(hǎi)者也(先是票友,后下的海)。京北三旗营有张云舫者,系故都仓中人,当差有年,多才多艺,心灵性敏,攥弄滑稽曲词,编歌曲是个高手,惟有他不善于歌唱。老倭瓜羡慕张云舫之歌词,与他交友,尽得其妙。恰在清末民初鼓界盛兴时期,老倭瓜每逢登台演唱,有张云舫之绝妙好词,他又能形容,发托卖像(指演员在表演时要惟妙惟肖,通过喜怒哀乐刻画艺术形象),使人望而解颐,能够咧瓢(liě piáo,大笑),老倭瓜渐渐成名,大受社会人士欢迎,因为他是票友,没有门户,在前门外演唱,被本行人所携(被有门户人将家伙拿走,调侃儿叫被携)。老倭瓜已然看出红来,焉能改行?由白云鹏介绍,给史振林叩瓢(磕头拜师),乃脱离票友,实行下海。白云鹏也史之弟子,二人即系师兄弟“排琴”(师兄弟调侃儿叫排琴)的关系,受白提携,献艺平津沪汉,老倭瓜三个字无人不知了。大南瓜、大茄子、架冬瓜,接踵而起。海轰儿这行里,又兴出相声化的大鼓了。滑稽大鼓的曲词乃张云舫所编,为老倭瓜闯荡开了,可惜张没获利,崔已家成业就,时也运也命也,信不诬也。如今张云舫所编之滑稽曲词,《拴娃娃》、《劝五迷》、《蓝桥会》、《妓女过节》、《家败归天》、《蒋干盗书》、《丑女出阁》、《海三姐逛市场》、《阔四姐推牌九》、《劝国民》那些段,盛行了一时。惜张最美之《胭脂判》、《战宛城》等段未能授人。现张已五十多岁了,若无人学习,《胭脂判》、《战宛城》恐将失传了。有王×延者,为人记忆最佳,脑力很好。无论何种曲词,不拘长短段儿,只要叫他听见,便能一字不少的全然记住。张云舫搜索枯肠精心之著品,不肯轻授于人。若是王×延在座,张则避席,或不一语。有人问他为何如此?张则笑而不言,盖王×延“荣活儿”(管偷学曲词调[diào]侃儿叫荣活儿)的本领,素有大名,不由张不生畏也。望柳(liǔ)海(hāi)轰儿的人们留心张之曲词,倘无人能学习,《胭脂判》、《战宛城》等段子,必被张携之入地了。





海轰之十三道大辙


唱大鼓不论什么调儿都离不开十三道大辙。十三道辙:一中东辙,二人辰辙,三江阳辙,四发花辙,五梭波辙,六灰堆辙,七衣齐辙,八怀来辙,九由求辙,十苗条辙,十一言前辙,十二姑苏辙,十三叠雪辙。如“少爷的大运未通,犹如蛟龙困在浅水中”,即是中东辙。如“一日离家一日深,好似孤雁宿寒林”,即是人辰辙。如“小少爷休要慌忙,细听我说个端详”,即是江阳辙。如“听他说了这句话,叫我心中似刀扎”,即是发花辙。如“不由人珠泪双落,尊贤弟细听我说”,即是梭波辙。如“叫人听了伤心落泪,实使我痛伤悲”,即是灰堆辙。如“我本是书香门第,出门来寻找妻”,即是衣齐辙。如“听他言来泪满腮,叫声我妻细听开怀”,即怀来辙。如“他二人好比龙虎斗,不知何时方罢休”,即是由求辙。如“打洋鼓来吹洋号,叫人听听这一套”,即是苗条辙。如“要等我儿站门前,好不叫人眼望穿”,即是言前辙。如“卖国求荣不顾主,背主求官把官图”,即是姑苏辙。如“来清去白慷慨正,说明就此拜君别”,即是叠雪辙。

鼓界所难学的为万子活(管说长篇书目叫万子活),整本大套的书,没个几年功夫是说不了的。万子活教法都是口传心授,即或有册(chǎi)子(书),笔录的也都是“梁子”(江湖人管秘本的笔记书里的结构穿插,调侃儿叫“梁子”),外人瞧着也是不懂。唱段的鼓儿词,有一种河南齐家本儿,是老合(江湖中唱大鼓的人)全都能会,惜其词句不雅,仅能合辙。子弟曲儿都是清时票家韩小窗,民初庄荫棠、全月如几个人攥弄(zuàn nong)(编创)的,这些年齐家的本儿渐渐地消失了,韩、庄、全的曲儿颇受社会询局(听书的)的欢迎,总算盛行一时了。





鼓界之白云鹏


唱大鼓书的这行儿,江湖人调(diào)侃儿叫“柳(liǔ)海(hāi)轰儿”的,柳是唱,海轰儿是指着大鼓而言。在民国以前,柳海轰儿的人们都是做明地(露天演出),在市场内支棚设帐,拉场儿。所唱的玩艺儿都是“万子活”(整本大套的书叫万子活),什么《呼延庆打擂》、《前后七国》、《杨家将》、《跨海征东》、《薛刚反唐》等等的说部,一套书要唱好几个月,说唱起来是没结没完。自从清末时代子弟玩艺儿兴开了,“唱片(piān)儿”(管一段一段的曲儿调侃儿叫唱片)普遍了,那时候唱的最有万儿(名)就数着胡十和霍明亮了。到民国以来,时代所趋,把艺人身价抬高了,继胡、霍之后为张小轩,惜其身段不好,没有台风,每逢演唱的时候,荒腔走板添虚字儿,实不警人。就以《活捉三郎》那段曲儿说吧,一张嘴唱头一句是“天堂地狱两般虚”,他偏给添字儿,唱成了“这天堂,那地狱,两般都是虚”,由七个字儿添成了十一个字,平、津、汉、沪等地的询局的(听曲的人调侃儿叫询局的)人,都评他四个字:穷凶极恶。在刘宝全、白云鹏未露头角之前,平、津、沪、汉还有人听他的玩艺儿;刘宝全、白云鹏成了大名,张小轩三个字几乎无人知道了。

白云鹏字翼青,现年六十一岁,系河北省唐山二里村人氏。自幼即嗜好歌曲,在本县有名票陈某曾传艺于彼,渐得其妙。自光绪十五六年赴津献艺,未享大名。四五年后来平,在各市场庙会献艺,因是作艺人无门户不能作艺,遂给鼓界名人史振林叩瓢(磕头拜师)儿,经名师指导,艺业乃进;又兼其好学,不耻下问,精心研究,数十年之间,始造就成鼓界名角儿,诚不易也。白在民初间尚以万子活(说大书)儿见长,从袁项城(袁世凯)执政时始弃了万子活儿,改柳(liǔ)唱片(唱段子),在新世界开办时渐成大名,在津、沪、汉等地献艺,颇得各界询局(听众)的人士赞美,能够与刘宝全并驾齐驱,实是各有所长。刘则身体雄壮,多演武段,如《华容道》、《战长沙》、《长坂坡》、《宁武关》、《截江夺斗(dǒu)》等等段儿;白则身小神足,文质彬彬,多演文段,如《宝玉探病》、《宝玉娶亲》、《哭黛玉》、《探晴雯》、《太虚幻境》、《窦公训女》、《千金全德》、《骂曹训子》等等段儿,二人各尽所长。刘每逢登台,吐痰挽袖子;白每逢登台,先鞠躬后说话,言词谦恭。说些铺垫的话儿,也各有不同。

唱大鼓书的这行儿,江湖人调(diào)侃儿叫“柳(liǔ)海(hāi)轰儿”的,柳是唱,海轰儿是指着大鼓而言。白云鹏,身小神足,多演文段,开创了白派大鼓艺术。



白系鼓界四大门户,梅、清、胡、赵,梅家门支派中人。在天桥儿柳海(hāi)轰儿万子(名气)最海(hāi,大)之田玉福、吴玉海,皆其师兄弟也。白系童子礼儿,自幼入礼门,不动烟酒,人情世态,阅历最深。江湖人都说他的腿儿最长(江湖人管为人河路码头、省市商埠去的地方最多的人,调[diào]侃儿叫腿长。若受艺人敬重的人,调侃儿叫是份腿儿),可不是能跑。数十年来,置有恒产,家道小康,惜以乏嗣,宗祧(tiǎo)难继,过继一子,人品颇正,不想未能永寿,在前年已去世了。其女已三十许人,为其操弦之韩德全乃白之乘龙佳婿也。

敝人曾与白云鹏请论所唱之曲词,是江湖秘本为佳,还是票友们编纂的为佳?据他所说,江湖的曲词都是平俗粗劣,还是子弟票友们攥弄(zuàn nong)(江湖人管编纂曲词调侃儿叫攥弄活儿)的活儿为美。今日鼓界盛行的曲词,以早年韩小窗攥弄的为最佳。民初庄荫棠攥弄的活儿也颇可取。韩小窗先生攥弄的活儿,当初有卖唱本的“百本张”售卖。自从百本张故去之后,韩小窗的活儿已然无处去“肘”(江湖人管买东西调侃儿叫肘)了。现在若能有人重印百本张所售的曲儿,足能获利,惜以无人进行为憾。





天桥的大鼓书场


唱大鼓的这行儿江湖人调侃儿叫“柳(liǔ)海(hāi)轰儿”的。他们这行儿所唱的有奉天调、乐(lào)亭调、西河调、梅花调、梨花调。

奉天调儿的大鼓,别处不论,天桥是没见过的,即或有了也是没人听。乐亭调的大鼓在北平这个地方是不兴的,只有每天夜间在烟花柳巷串下处,唱大鼓的唱这乐亭调儿。梅花调儿的大鼓是最难学的,天桥简直就没有这玩艺儿。唱这个调儿的男角以金万昌最佳,坤角以郭小霞最好。他们向来是上落子馆儿,露天地是见不着的。在民国十年以前,香厂开办新世界,山东的坤角谢大玉唱梨花调儿的大鼓,颇受北平市顾曲的人们欢迎。近几年来,天桥来了许多梨花调大鼓的坤角,李雪芳、段大桂、于宝林、刘大贵等,在各场内演唱,也是昙花一现,不能持久。

在天桥久占大鼓书场的还是唱西河调儿的。清末民初的时候,史振林唱得最叫座儿,史系大鼓名角白云鹏之师。史故去之后,以田玉福称为第一,他所唱的书有《杨家将》、《呼家将》、《春秋战国》、《反唐传》、《跨海征东》、《马潜龙走国》,那些书都是万子活(成本大套的书)儿。江湖人常说,上明地(露天演出)的海(hāi)轰儿,非得说整本大套的万子活儿,才能唱得长久。田玉福在天桥唱大鼓书,使长长的万子活儿,可称为第一。他也是鼓界名人史振林门徒,他很红了二十多年。如今,年岁大了,气力小,不能整天地唱了,其声望渐渐退化,收入也是日日见少了,索性离开了天桥,开了外穴(xué)(到外地去挣钱),往各码头跑腿去了。艺人的艺术,不养小,不养老,也甚可叹也。

民国以后,时代变迁,男女社交公开,准许男女艺人合演。



在天桥能够久占书场的,是唱两河调儿大鼓的王云起父子。王系河北定兴县城西陶小村人,昆仲二人。其兄王云峰,也是柳(liǔ)海轰儿的,曾到过天桥,因为人们不太欢迎,他不在北平,专在保定献艺,其艺术也不如王云起,故不能在北平天桥立足。王云起所唱的大鼓书只有《杨家将》、《呼家将》,按说活儿不宽,按万子不长(不够整套书),他为什么能在天桥久占呢?我老云调查过他久占的情形,他的艺术毫无特长,只有能迎合天桥好听大鼓书座儿的心理,能够天天满座儿。王云起的书是没有知识分子去听的,凡是无知识的人都爱听他的书,他唱的书词也是俗不可耐,一张嘴就是:“大众的佛台,稳坐压言,贵耳留神听。前回说了半本《呼家将》,还有半本没有说清。哪里丢,哪里找,哪里接着说。书中单表那一位,表的是,人前显贵,鳌里夺尊,出乎其类,拔乎其萃的呼延庆。”费了十几句唱词儿,才唱出个呼延庆来。知识阶级的人听着是腻烦的,一般没有知识的最低级的人们,却是爱听。据江湖人说,他的书词是开门见山,有皮儿薄的好处,能够叫座儿。我问过江湖人,什么叫开门见山?什么叫皮儿薄儿?江湖人说:“他们唱的书,书中人物各有不同,如若张口就唱班超,是没人懂的,想班超是汉朝名将,当初他是个读书人,因未能得志,将笔杆儿扔掉,弃文就武,投笔从戎。以十数人平西域十数国,汗马功劳,受封为定远侯,那实是中国的伟人吧!可是有一样,唱出的这个人,没念过书没读过历史的人,更不知道班超是何等人物。我们管唱出来的书词听主不懂调(diào)侃儿叫皮儿厚。生意人做艺的地方,都是露天市场,逛露天市场的哪有阔人?哪有知识分子?即或有些个阔人,有些个知识分子,较比普遍的人数,比较起来,也不及十分之一。故此江湖艺人学习艺术的时候,是不学皮儿厚的玩艺儿,不学下层社会人士不懂的书曲。譬如,唱大鼓的艺人,一张嘴就唱李逵、宋江,不读书,不识字的人,听到耳内立刻就能知道,这两个是《水浒》梁山的人物,宋江坐楼杀过阎婆惜,李逵闹江州还夺过张顺的鱼。江湖人管唱出来的书词、书中人物,听主立刻就懂,立刻明白,调侃儿叫皮儿薄,调侃儿叫开门见山。如若张嘴就说孙猴、八戒、武大郎来,无论什么人都能知道,都能懂,那还叫真正皮儿薄,真正开门见山。”我听他们江湖人说,皮儿薄、皮儿厚、开门见山的议论,才知道大鼓书的词儿是深入低层社会,不能登大雅之堂。可是他们不迎合下层社会人的心理,不迎合没知识的人们,是不挣钱的。

柳(liǔ)海(hāi)轰儿的艺人,第一要人样长得好,说行话叫人式顺溜。第二要口白清楚,说行话叫碟子正。第三要嗓音洪亮,说行话叫夯(hāng)头正。第四要身段表情形容出来有喜、乐、悲、欢的态度,要学得像生、旦、净、末、丑的样子,说行话叫发托卖像(指演员在表演时要惟妙惟肖,通过喜怒哀乐刻画艺术形象)警人。有这四种特长,才能学好了书词,上场去唱玩艺儿。此外还得会看地势,如若地势不好,上的座定受影响;若是地势好,本人的技能再好,一定多上座儿。江湖人常说:“生意人不得地,当时就受气。”这话诚然不假。唱大鼓的艺人最好要懂圆粘(nián)子(招徕观众),将粘子圆好了,还得有好“驳口”。我问过江湖人,什么叫“驳口”?他们说:“唱大鼓书的,每逢唱到要钱的时候,那末一句的词儿,行话叫驳口。”我问:“什么叫好驳口呢?”江湖人说:“譬如唱的是《杨家将》,唱到杨七郎天齐庙打擂,打死了潘豹,杨继业知道了,将七郎绑上,拔出宝剑要杀,唱到拔剑就杀当作驳口,那听书的人们都怕杨继业真杀了杨七郎,很不放心,坐在凳上不走,往外掏钱,想再听下回。能全场的座儿一个不走,那才算好驳口。有些唱大鼓书的,不会使驳口,他唱到杨继业要杀杨七郎,列位要问怎么样?下一回是绑子上了殿……要几个铜子,再往下说,他这驳口就坏了,听书人们听他唱出来下回绑子上殿,就知道杨继业不杀他儿子了,还绑着七郎往金銮殿见皇上哪,不用听了,杨七郎不能死了。若是使这样的驳口,保管一要钱,满场的座儿能走一半,像这样就叫驳口不好,使用得不洽,不能挣钱。”他们挣钱能力的高低,全由会使驳口与不会使驳口而定。王云起就是人式好,碟子正,夯头好,发托卖像好,会圆粘儿,会瞧地势,会使驳口。他还能放大大的回头,长长的段儿,傻子的豌豆——多给,他能有这几种迎合人心理的技能,才能在天桥久占。据江湖中的名人说,王云起的大鼓不算头路角儿,只算二路角儿。可是他在天桥能久占。有些个二路角儿到了天桥,都不能持久。故此,我以他能久占天桥而论,算是天桥第一个柳(liǔ)海(hāi)轰儿的。至于鼓界的头二路角儿来到天桥站不住,也有个原因。据江湖人说,唱大鼓书的艺人以赵玉峰、黄福才、二狗熊为头路角儿,在各省市做艺,每天能有十数元的挣项。郝英吉、马连登、王庆和等为二路角儿,在各省市做艺,每天都有五、六、七、八元的挣项。天桥这个地方,唱西河调大鼓的艺人,最有本领的能挣三元钱,也有挣两元的,甚至于本领不济的还有不得温饱的。就是将他们的头路角邀了来,凭天桥这个地方,要每天挣十几块钱哪,简直是办不到的。就是二路角儿来了,也挣不出七八块大洋来。故此头二路的角儿都愿到天津、大连、济南去做艺,谁也不愿到北平来的。他们在天津上地(做生意),一个书场能上二百多座儿,因为天津那个地方是个码头,卖苦力气的人,在社会上撞现钟(看见什么能挣钱的活儿就干)的人,下层社会无知识的人是最多的。这些人要忙里偷闲听会儿玩会儿,是适合听大鼓书的。江湖人调(diào)侃儿说:“天津的人式旺得很(江湖人管人多调侃儿叫人式很旺)哪。”故此,头二路角儿在三不管(天津市南市的一个露天市场)一带上地,能上三二百座,挣个十元八元的很容易。能够养得住头路角儿,就养得住二路角儿。

北京宣武说唱团演员蔡金波、刘田利在表演西河大鼓书。(照片由徐雯珍提供)



北平乃过去之都城,数百年之历史,年代本深,知识分子、高尚的人是很多的,与天津大不相同,故此一些个没知识的劳动人也没天津的多。唱大鼓的艺人唱得多好,也上不了二百座儿,至多上个七八十人,就算好极了。柳(liǔ)海(hāi)轰儿的头二路角儿,都说北平这个地方人式太减(管欢迎他们的人少调侃儿叫人式太减),都不愿来了。年前有二路角儿马连登曾在天桥上地,他唱的是《盗马金枪杨家将》,与王云起对抗,来了两个多月就走了。并不是他敌不住王云起,谁放着有能多挣钱的地方不去,在这里少挣啊!有些不知其中细情的人,都说马连登敌不住王云起,那实是不明白江湖事了。王云起有这种种的原因,能在天桥竖头杆大旗,也不愿往别处去,就在北平做艺。他父子二人克勤克俭,并无嗜好,十数年的光景,听说很落下几个钱,在他们定兴县置了些地,就是不说书,归家种地,也能维持生活。都说艺人不富,我是不信的,梨园行的名角儿,有几十万财产的,北平很有几位,那不是艺人吗!





天桥的坠子场子


天桥的玩艺儿也时常地变迁,前几年以来,唱河南坠子的又盛行一时了。我老云在河南的时候曾向唱坠子的人们探讨过他们的坠子源流,是哪时有这宗玩艺儿的。走闯江湖的艺人,差不多都只知道挣钱吃饭,哪管这些个,我问了许多人,一无所得。年前,在津埠遇一艺人×××,是唱坠子的老手,我向他作最末次的探讨,如彼不知,我老云就再不向他们去探讨了。不料,这位唱坠子的,侃侃而谈,原原本本,说得很有趣味,可是其中也有些个荒诞无凭的话语。我将他所说的一古脑儿写出来,贡献于阅者。至于说得对与不对,敝人不敢下断语,好在是他说的,写出来是我替他学舌,人云也云罢了。以下系唱坠子老艺人所说。

“我们唱坠子的,是先高后低。高的时候,是道情歌儿;低的时候,是串百家门,逼柳(liǔ)琴儿。我们这玩艺儿都说是在唐朝有的。当初,唐明皇在位时,在山西省晋汾之间有个修行的老人,年岁高迈,发似三冬雪,须赛九秋霜,神清气爽,仙风道骨,常在恒山一带敲打渔鼓、简板唱道情歌,劝化世人。他能数日不食,精神不衰,人多奇之。有人问他姓名,自称姓张名果,生在尧舜时代,乡人无不尊敬,称他张果老。相州刺史韦济闻张老之名,探验属实,欲讨好于玄宗皇帝,上表奏闻。那唐明皇乃风流皇帝,内信李林甫,外倚安禄山,宠爱杨贵妃,因色身亏,精神衰弱,欲学长生术益寿延年。恰见韦济奏闻恒山有张果老,立命通事舍人裴晤往恒山去召张果老入都。裴晤奉旨前往,至恒山寻着果老,并无敬意,迫其入都。果老行至途中,忽然倒地身死,裴晤疑其有诈,在尸旁守候数日,尸身僵卧,实是无诈。裴晤命人葬埋,果老忽然站起,谈笑自若,不饥不渴。裴晤惊讶不已,觉其非凡,不敢强迫,命人入都奏闻玄宗。唐明皇又遣中书舍人徐桥,赍奉玺书,优礼往迎,果老始随入都。唐明皇赐乘肩舆,请入宫中,问出神仙术。果老只说,息心养气,便可长生。唐明皇留他居于集贤院,数日不准人等进他酒食,果老累日辟谷,毫无倦态。玄宗奇之,命人赐以美酒,酣醉之后,长睡数日不醒。弄得唐明皇不知他是仙哪,是鬼呀!莫名其妙。时有术士邢和璞、归夜光二人,邢能算生死,归能查看鬼神,素为玄宗所信,将他二人召至宫中,命算果老生死,查他是鬼是神。邢和璞占算半日,竟不能算出果老生在何年,死在何日。归夜光查看两昼夜,不敢断他是鬼是神。唐明皇密语高力士说:“饮堇酒无害,方为奇士。”乃召果老,命其饮堇酒,果老饮之三大杯,忽然倒地,仰面朝天,张开大嘴。帝与高力士见其口中齿皆焦缩,果老伸手拔齿收入囊中,眨眼间,齿竟重生。君臣叹服,仍命果老宿于集贤院,时有唐睿宗之女崇昌公主在玉真观为尼,明皇欲将公主嫁与果老,命秘书监王迥质、太常少卿萧华往集贤院商于果老。果老说:“娶女妇得公主,平地升公府,人以可喜,我以可畏。”言罢大笑不止,问萧、王二人道:“皇上以果为仙,果实非仙,若视果为尘俗人,也可不必。果从此辞,将归山了。”二人回奏,玄宗尚欲挽留,果老再三恳求归山,玄宗乃命人画其图形悬挂集贤院,授为银青光禄大夫,赐号通玄先生,赐帛三百匹,命人护送归于恒山蒲吾县。张果老归山之后,仍在山中敲打渔鼓、简板,唱道情歌劝化世人。民间之人多仿学渔鼓、简板,唱道情歌。后由山西流传至河南。传至宋元时代,道人化缘,乞丐讨饭,俱用渔鼓、简板,沿户唱歌,化缘讨要。至清末时,道情歌曲竟归了穷家门(唱数来宝的),是由高而低也。”

有许多的妇女演唱河南坠子,并将渔鼓撤掉,改换大鼓一面,左手执桴,右持简板,唱起活来,所唱也非道情,秽词污语,引人入邪,虽然有碍民俗,听主却多欢迎。



自从民国,时代变迁,打破专制思想,阶级平等,男女社交公开,准许男女艺人合演。有许多的妇女演唱河南坠子,并将渔鼓撤掉,改换大鼓一面,左手执桴,右持简板,唱起活来,所唱也非道情,秽词污语,引人入邪,虽然有碍民俗,听主却多欢迎。唱山东大鼓的坤角见大鼓日渐衰落,坠子火穴,纷纷改唱坠子。近来平、津、沪、宁各杂耍(是曲艺杂耍形式的综合叫法)馆中都得约档坠子才算齐全。乔清秀驰名平、津、汴、济,海报上也大书“坠子大王”。有糖业大王、汽车大王、煤油大王、滑稽大王、梨园大王、电影大王、评书大王、鼓界大王、梅花大王,如今坠子大王又应运而生,不久,我老云也要成为云游大王、神聊大王了。

唱坠子的除乔清秀之外,董桂枝、宗玉兰、卢永爱也都不弱。天桥的坠子,开荒(头一个唱的)的不是坤角,还是个男角,满脸的麻子,一个人自拉自唱,很有滋味。社会的人士喜见奇怪,瞧着他又拉又唱,都听他唱会儿,也听不出什么意思,看得乐了,扔钱就走。那时正在民国十二三年,社会里还没嚷穷哪!做艺的人们挣钱也容易,被当作怪物瞧的唱坠子的艺人,每天能挣两三元,说江湖的行话,梅花盘儿在天桥火穴大转了(管麻脸的人调[diào]侃儿叫梅花盘,管能挣钱叫火穴大转)。江湖艺人耳朵最长,听见哪里兴旺就往哪里奔,凭梅花盘儿都能挣钱,色艺两全的坤角来了岂不更佳?于是,唱坠子的男女班纷纷来平,爽心园、天华园都约了坠子,各露天场子也都邀了坠子。最近,我到天桥云游了几天,见天桥坠子较比从前还多,魁华舞台后边有个坠子场儿,爽心园北边有个坠子场儿,马场道北边有个坠子场儿,倒是“水深流去慢,货高价出头”。我听了几回,露天场唱坠子的坤角,“盘儿念嘬”(管长得不好叫盘儿念嘬),“柳(liǔ)得也是念嘬”(管唱得不好叫柳得念嘬),无怪乎他们不能进馆子,只在露天儿演唱,色艺两念嘬,挣不了大钱,馆子哪能约请啊。卢永爱、大老黑两口子对唱,江湖人说行话叫鸳鸯档子。卢永爱唱做俱佳,身段好看,表情细腻;大老黑(他名叫任永泰)专会抓人,形容态度,使人解颐。在天桥上明地(露天演出),唱大棚,哪天也能挣十元以外。到了天华园内,啊!他们两口儿下场,听玩艺儿的人们就能起了堂,走了个干净。姚俊英,长得身材窈窕,黑漆似的大辫子,唱的时候,透着风骚浪漫,论艺远不及卢永爱,在天桥却颇受人欢迎。看起来,听玩艺儿的人们还是重艺的少,重色的多。大老黑、卢永爱愤而离平,在南京唱了未久,夫妻来了出离婚后会,如今在天津破镜重圆。据我老云所料,天桥是不来了。

大老黑夫妇走了以后,小桃园后玉明轩掌柜的由天津约来一班坠子,台柱子是坤角赵金兰,每天演唱时也是鸳鸯档子,男角赵勤堂,不是赵金兰的丈夫,系其养父,父女演唱,虽然能叫满堂座儿,并没有十元八元花钱的阔主。不料演唱未久,赵金兰就鸣了警啦!告他养父赵勤堂强奸虐待,打了官司,过了几堂,赵金兰就与赵勤堂脱离父女关系。赵勤堂失掉了摇钱树,又往别处种摇钱树去了。赵金兰没有赵勤堂捧活儿,艺术似见退化,在平津演唱,连个怪声叫好的都没有了,她又拧了万啦(江湖人管更名改姓叫拧了万啦),在天华园演唱,贴海报叫李玉芳了。

最近,董桂枝、宗玉兰姑嫂来平在玉明轩演唱,姚俊英、李玉芳、段大桂在天华园演唱,大鼓、坠子男女两色十数人两下里对台;灯晚也打对台,董桂枝、宗玉兰在观音寺华楼、宾乐轩演唱,姚俊英、李玉芳在青云阁、玉壶春演唱。还是董、宗姑嫂的色艺双佳,能唱能捧。江湖人曰“艺不错转”(这个转字是能挣钱的侃儿,艺不错转就是艺术定有高超的意思)。好听坠子的快快听吧!我老云瞧着他们这玩艺儿有一兴必有一衰,将来这种玩艺儿唱不长,若不相信,咱们就慢慢地瞧着。





天桥的竹板书场


天桥的杂技场样样都很多的,惟有竹板书是不多的,只有两三个场子唱竹板书。能够久占在天桥唱竹板书的艺人就是关顺贵、关顺鹏昆仲,江湖人管他们唱竹板书的调(diào)侃儿叫使扁家伙的(管唱大鼓书的调侃儿叫使长家伙的,是指他们使的弦子而言;唱竹板书的叫使扁家伙的,是指他们使的竹板而言;管说评书的叫使短家伙的,是指他们使的扇子而言)。

我老云云游了几省,唱好竹板书的我也见得多了,第一路的角儿有余来荣、王来有、赵华轩、邱玉堂、张德贵。这些人在各省市、各码头,无论上馆子上场子,哪个人每天多了能挣十数元,少了也能挣五六元,可是这些人都不往北京来。只有东安市场初立之时,余来荣在杂技场内唱过竹板书,叫座的魔力甚为可观。凡是唱竹板书的艺人都佩服他的,认为他是使扁家伙的特殊人才了。不料,他挣钱的能力好,受了金钱之害,早早断送了他的性命,甚为可惜。艺人不能理财,财多伤身,实可叹也!

关顺贵、关顺鹏兄弟在表演竹板书(关顺鹏唱,关顺贵贴板。照片由徐雯珍提供)



在清末的时代,唱竹板书的角色最有名的是贾宝山,他们传流的支派,是宝、顺、呈、祥,贾宝山是宝字辈的,他的大徒弟叫张顺明,曾在民初时献艺于天桥,叫座的魔力也颇不弱。关顺贵、顺鹏虽是贾宝山的徒弟,拜师未久,贾宝山就去世了,他弟兄两个唱竹板书,没得着师傅的传授,是由他们的师兄张顺明代传的。关氏昆仲只学会了吧嗒棍,还没学好万子活(江湖人管能叫座儿的小段子曲儿调[diào]侃儿叫吧嗒棍,管整本大套的书调侃儿叫万子活)哪,不幸,张顺明死在奉天。他们哥两个净唱吧嗒棍仅能糊口,实是不易发达。在民国十年前后,先就能挣几角钱,始终没能火穴(xué)(大红大紫)。在民国十六七年,又向大鼓名角田玉福学习万子活(长篇书)儿,学会了《跨海征东》、《战国春秋》、《杨家将》等书,艺业大有进步,哪部书都能唱几个月,天天叫满堂座儿。在民国二十年前,渐渐发达,如今火穴大转(zhuàn)(挣大钱了)了。凡是久逛天桥的人,都知道关顺贵、关顺鹏的竹板书唱得不错,可听。在这一二年,关顺贵忽然弃了扁家伙(唱竹板书),改使长家伙(唱大鼓书),又柳(liǔ)海(hāi)轰儿,唱大鼓书啦!在楼外楼的南边占了个场子,比唱竹板书上的座儿格外见多,总算他有心向上。世上无难事,就怕有心人。前两天,我到东安市场云了一趟,走在东跨院,见关顺贵在院内的东南角上弄了个场子,正唱《杨家将》。他又挪到东安市场去了。天桥的竹板书只剩下关顺鹏一人,他占的场子在沈三的场子南边,有好听竹板书的到那里听吧!





附录一 连阔如佚文《漫话江湖 万象归春》


江湖者,如江湖之水,能通三江,可达五湖,周流天下,无窒碍,无壅滞,无有一人,不被其泽。人若被人称为江湖,是其技能,如江湖之水,畅行天下无阻也。江湖人自称“老合”,这两个字的意义即是:以人家之意见,他们都能合作,随人之意而进,事无不成了。江湖人,又自称“搁(gé)念”。这两个字的意义是:他们江湖人,都得百行通,才能吃得开;如若哪行行不通,就有阻碍,搁住必念了。什么人都是江湖呢?五花八门的人物,就是江湖者也。五花八门又是什么呢?偷窃的小绺(xiáo liu),是“老荣”;贩卖人口的,是“老渣”;捕盗捉贼的,是“老柴”;使腥赌的,是“老月”;走闯江湖的,是“老合”。这五老,即是江湖中的五花。什么是八门呢?算卦相面的,是“金门”;卖药的,是“皮门”;变戏法的,是“彩门”;打把式卖艺的,是“挂门”;说书的,是“评门”;说相声的,是“团(tuǎn)门”;卖各样假东西的,是“调(diào)门”;唱曲的,是“柳门”。这就是八门。此外尚有六扇门里、六扇门外,阴阳两面的朋友,都在其中。这五花八门的人物,在我国的社会中都有一种特殊的技能,特别的势力。说起这些行来,亦很复杂,非三言两语可能说尽。先以他们最重视的“包袱儿”为题,我说一回“万象归春”。





江湖人以“包袱儿”为重 有荤、素、响、闷之别


什么是包袱儿呢,哪又是万象归春呢?这两句是江湖人的调(diào)侃儿。包袱,这日中所用的一种物件,烂七八糟的东西都往里包,使用完了,必须得把包袱抖搂一下,是什么东西,亦都抖搂出来了。江湖人有一种极幽默的特别技能,如若使出这种方法,就能把人逗乐了!江湖管这种逗笑的方法调(diào)侃儿就叫“抖搂包袱儿”。那笑料儿就是包袱儿里边的东西。可是抖搂包袱儿,仍有“响了”、“闷了”之别。什么叫响了呢?如若江湖人用一种逗笑的话料说给众人听,如能把人逗乐了,调侃儿就叫“包袱儿抖搂响啦”。如若他们的笑料向听众说完了,听众都没“咧瓢儿”(江湖人管人笑了调侃儿叫咧瓢儿。瓢儿即是人的脑袋,如能咧了,便是笑啦。如若人没笑哪,调侃儿就叫没咧瓢儿),那就是“包袱儿抖搂闷了”。包袱儿抖得最好,调侃儿叫“哄堂”。其意义是,全场的听众都乐了,哄堂大笑也。江湖人以包袱儿抖搂响了为荣,以抖搂闷了为耻!有了哄堂的时候,抖搂包袱儿的江湖人,认为莫大之荣幸!江湖人对于抖搂包袱儿之重视,亦可想见矣。包袱儿亦不同,有“荤包袱儿”,有“素包袱儿”。什么是荤包袱儿呢?凡是妇女不可听的笑话,就算荤包袱儿。什么是素包袱儿呢?俗不伤雅,男女老少都可听的笑话,就算素包袱儿。江湖中的人物,能抖搂包袱儿的,极不易得,人才亦是有限哪。





江湖中能抖包袱儿的艺人与艺术之调查


大戏的角色,分为生旦净末丑。小丑儿,不论文武,以能逗笑当场抓哏为美。银幕上的电影明星,最难得的人才是能逗笑的滑稽角色。近些年来,只有陆克、贾波林(即卓别林),能有声价,有人欢迎。江湖中的滑稽人才亦是难得。早年的随缘乐、人人乐、张三禄、徐永福、德寿山、老张麻子、小张麻子、老万人迷、小万人迷、卢伯三、冯昆治、袁桂林、徐狗子、双厚坪、焦德海、陈大官等百数十个,与现在的张寿臣、侯一尘、常连安、小蘑菇、老倭瓜、架冬瓜、山药蛋、吉评三、大茄子、陶湘如、刘德治、高玉峰、谢瑞芝、华子元、安青山、恒永通、老云里飞、小云里飞、大兵黄等数百余人,都是江湖中滑稽中的人才。别看他们都能抖搂包袱儿,艺术是不同的:笑料有高有低;形容有优有劣;干的行当又不一样;他们做艺的地方,有上中下之分别;受人欢迎,亦有上中下之别;他们的艺术人品、魔力,都有研究的价值。按着万象归春的意思,分门别类,逐段写出来,阅者诸君,茶余酒后,消遣解闷。我的东西写在报上,是骆驼的下巴颏儿——耷拉嘴儿,六指儿挠痒痒——伸个小手儿。





穷不怕首创“单春” 有“攥弄(zuàn nong)活儿”(自己会编相声)的特长


我这段“万象归春”,五花八门的人物之技能都得说说。以哪个江湖人作为首谈哪?先以穷不怕作为首谈,然后再按五花八门,一种一类地往下谈。穷不怕是三十年前的江湖艺人,他是久做“单春”的老合。什么叫单春哪?说相声的行当,调(diào)侃儿叫“春口”。一个人的相声,调侃儿叫“单春”;两个人的相声,调侃儿叫“双春”。以这两样而论,是单春难说。两个人的相声,一捧一逗,显得火炽,一装巧,一装愣,凭说凭逗,都容易引人发笑。一个人的相声显着冷,又不火炽,把人逗笑了实在不易。据相声行人说,他们这行使单春的人才最少,以穷不怕为第一,可以称为“单春大王”,前无古人,后无来者,实是他们行中空前绝后的人物。穷不怕姓朱,名叫少文,满清的时代,汉军旗人。早年读书,博学强记,学识深渊,心思敏捷。曾以舌耕为业,心志不遂,愤而投入江湖的门户,改说相声。向不搭伙伴,从未与人“联穴(xué)”(两个人合作说相声调侃儿叫联穴),独树一帜。可惜他生不逢时,那年月的相声不似今日之盛,社会的人士对于滑稽艺术尚未公认,实在没有“拱开”(社会的人士对于什么玩艺儿公认了,即是拱开),虽有高尚的玩艺儿,亦难登大雅之堂。他若生在今日,“滑稽大王”的头衔就归不了万人迷了!

穷不怕做艺,向来是“撂地”(凡是在露天地做艺的,都说撂地),还是用平地,低洼地不相宜。他以地皮为纸,白沙当墨,戳朵儿圆粘(nián)儿。这是什么话呢?江湖人对于写字调侃儿叫戳朵儿,对于引人围观调侃儿叫圆粘儿。凡是江湖的玩艺儿,都得有人围观,才能挣钱,其圆粘儿之法是江湖人初步基础的技能。至于圆粘儿的方法,各有不同。变戏法的以敲锣击鼓,把人招来圆粘儿。他们是武粘儿,有响动,圆着容易。像穷不怕,夹着一把笤帚,手拿小布袋,舀着白沙末子,往地上写字圆粘儿,那够多难!敲锣打鼓圆粘儿,是有耳朵的人(聋子不算)都能招引来,不在乎识字不识,招的人界限很宽。穷不怕写字招引人,不识字的人吸不住。有这层关系,他圆粘儿更难,招的人界限亦窄。在早年是说相声的人,都会写地皮上的白沙子字,可是写得好歹,大有分别。穷不怕的字,横平竖直,字儿端正,人家的手写不弱于用笔。不论大小字,皆有字体,愈写大的,愈有帖气。可是别人只能写字圆粘(nián)儿,围上人,还得说相声挣钱。穷不怕的特长是写字圆粘儿,随写随柳。什么是“柳”哪?江湖人对于唱调(diào)侃儿叫柳。他随写随唱,写完了一遍,连说带唱,随唱随抖搂包袱儿,临完了,惹人一笑!所有一切科诨笑话掌故之类,皆由字义内抖搂出来。

现今说相声的,都以“火”做;唯有穷不怕,专以“水”做。什么是水,什么是火哪?江湖人以穿着阔绰调侃儿叫火,以衣履缺残调侃儿叫水。穷不怕虽然水做,他的玩艺儿可是高尚极了,向来不“团(tuǎn)钻钢”(江湖人对于撒村骂人调侃儿叫团钻钢)。他腹有诗书,能戳朵儿(江湖人对于识字的人调侃儿叫朵儿),知识高尚,心思敏捷,所说的玩艺儿谐而不厌,雅而不村,“果食码子”、“抖花子”都能听(江湖人对于妇人调侃儿叫果食码子,管姑娘叫抖花子),纯粹是档子文明玩艺儿。别的艺人学什么玩艺儿,都是口传心授的死套子活,怎么学来的怎么卖,绝不能更改。穷不怕可不是套子活,他的本领是能“攥弄(zuàn nong)活儿”(江湖人对于会编段子调侃儿叫攥弄活儿)。每逢上地做艺的时候,就能随唱各样歌词。先以字意儿说吧,他有“容”字,由写“人”字唱:“写上一撇不像个字,添上一笔念个人。人字头上添两点儿,念个火。火字头上加宝盖儿,念个灾。火到临头灾必临,灾字底下添个口,念个容。劝众位,得容人处,且得容人。”他这个字唱完了,还是以让劝人。最奇的是他以五百出戏名编了一段曲儿,当初很有人欢迎,我把他这个曲儿在纸上再唱一回。





穷不怕编的《五百出戏名》曲儿


穷不怕的《五百出戏名》曲儿,以“殷家堡”为主,唱:

昔日有一人姓殷,名叫《殷家堡》,家住在《文昭关》城西,《五里碑》的《四杰村》。居住的这日正月十五,《武当山》赴《英雄会》,身穿《打龙袍》,腰系《乾坤带》,足蹬《借靴》,头戴《封冠》,上安着《海潮珠》、《庆顶珠》,腰别《断密涧》,马棚拉出《盗御马》。《殷家堡》上马出了《四杰村》,进了《文昭关》东门,出西门,路过三座山:《青石山》、《百草山》、《翠屏山》,来到武当山顶《金顶山》,山上有两杆《盗旗》,上写《赐字》。《殷家堡》下马,上了高山,《山门》洞上写《法门寺》。《殷家堡》进《山门》,见有三座庙:《灵官庙》、《关王庙》、《八蜡庙》。他掏了香钱,在《关王庙》磕个头求保佑,《天官赐福》。磕罢了头将站起,来了看庙的一个老道,将他让进了《五福堂》。小老道拿过《落马湖》放下《搜(来的)杯》。《殷家堡》正要喝茶,又来个小老道,拿过文房四宝,叫《殷家堡》写个布施。他掏出了《花子拾金》,老道是哈哈笑。《殷家堡》一出庙门,做买卖的真不少,《卖绒花》、《卖饽饽》、《卖胭脂》,《三进士》、《四进士》、《张三跑马》《大卖艺》,《河粮会》、《湘江会》、《金莲会》。《殷家堡》正往前走,抬头看,见有他盟弟《五人义》,骑着一匹《五名驹》,他二人一同去喝茶。往西不远有个《铁弓缘》,母女开茶馆,门前列着《两把弓》。他们进了茶馆正喝茶,《殷家堡》问《五人义》:“兄弟要往哪里去?”《五人义》说:“我到你家去《探母》。”《殷家堡》说:“不必《回令》。”

他二人出了《铁弓缘》,往回走,路过《白良关》、《牧虎关》、《陈塘关》,出了《赶三关》,路过四条吊桥:《清河桥》、《洒金桥》、《太平桥》,还有《断桥》,来在《渭水河》边。有《钓金龟》钓鱼,有《打柴训弟》,《跑坡剜菜》,《母女拣柴》。走过《探寒窑》,《双别窑》前下了马,渴得咴咴叫。《苦水井》有贫婆汲水,名叫《罗衫计》。他们将马饮完,哥儿两个没钱给,取出《一匹布》送与贫婆,贫婆一见《心欢乐》。哥儿两个上马过了几个镇店,《清风岭》、《蜈蚣岭》、《金沙滩》。《三家店》在《二龙山》,山南有个《辛安驿》,有许多过路人《指路观山》,《打马起解》,《小放牛》,《大金钟》,《大锯缸》,全都不看。《快活林》跑出《花子骑驴》,手拿《演火棍》,来到跟前《打杠子》,要他们的买路钱。《殷家堡》说:“杀他还不如《杀狗》。”吓跑了《打杠子》的。

他们到了《高平关》,哥儿两个上前《叫关》把城进。《十字街》前《悦来店》,他们住在店中,天气晚,伙计摆上《七星灯》。哥儿两个落座,问:“掌柜的贵姓?”掌柜的说:“姓梁,叫《梁子峪》。”言说吃面没别的饭。哥儿两个正然用饭,来了两个姑娘,一个叫《凤仪亭》,一个叫《宇宙锋》,怀抱琵琶就要唱。《殷家堡》问她们都会唱什么曲儿,这个姑娘说会唱《夜宿花亭》,那个说会唱《洪武放牛》。两个姑娘唱完了曲儿,哥儿两个无钱就《赠珠》。次日早晨出了《高平关》,路过三座洞:《金丝洞》、《无底洞》、《五花洞》。他们到了《恶虎庄》,天气阴沉下了一阵《梅绛雪》。《五人义》说天气冷,《殷家堡》马后就《赠袍》。过了《云罗山》,《走雪山》,才到《文昭关》。进东门,出西门,回到《四杰村》。

哥儿俩门前下了马,外边有个《倒厅门》。上前《拜门》,里边有人把门开放。《殷家堡》说:“贤弟入府,我这府亚似《十王府》。府门洞上挂《逛灯》,上写《金马门》。家里还有三座楼:《黄鹤楼》、《坐楼》带《跪楼》。”哥儿俩逛了《黄鹤楼》,上摆几凳,凳上《忠义侠》《哭灵牌》。《双官诰》《听琴》《挂画》《观棋》《盗书》,这叫琴棋书画。还有四张图:《百寿图》、《铁冠图》、《八义图》、《四美图》。哥儿俩落座,家人看过《落马湖》,放上《对银杯》。哥儿俩正喝茶,《殷家堡》吩咐家人,《九龙峪》上摆《九龙杯》。四个碟子都是炸食,《铡判官》、《铡包勉》、《铡知县》、《铡陈世美》。又上四个大碗,一碗《拿黄龙基》,一碗《斩蔡阳》,一碗《黄一刀卖肉》,一碗《偷曼倩》。哥儿俩吃了一个饱。

《五人义》要教学,《殷家堡》门外贴个帖,言说有个盟弟会教学。这里《教子》,那里《送学》,那里《训子》,《五人义》楼上教学。《殷家堡》的媳妇长得美貌,白里套红,红里套白:白是《白水滩》,红是《洪洋洞》。身穿《借衣》,腰系《和凤裙》,足蹬绣鞋,头戴《采花》,两耳挂定《赐环》,一只胳膊上带《点翠镯》,一只胳膊上带《拾玉镯》,脸上搽《汾河湾》,嘴抹《胭脂血》,名叫《遗翠花》。《五人义》见她长得好看,就得《相思寨》。《殷家堡》病送被褥《三疑计》。《五人义》不到两月就死了,《殷家堡》《哭丧计》,《葬灵》买了《大劈棺》。《殷家堡》的媳妇,因为和《五人义》到不了一处,《三世修》《三上吊》。

《殷家堡》死了媳妇,他来了个《火焰驹》,将家烧得片瓦无存出了外。《当锏卖马》,《借当》《卖水》,《卖画儿》来在《坐窑》,找个《乌盆记》要饭。他要的东西吃不饱,有个朋友送他俩钱,教他把买卖做。他买了个扁担,买个斧子,上山打柴。打了三天,得了《温凉盏》。《殷家堡》去《进宝》,皇上封他《状元谱》,换《斩黄袍》《宫门带》,赐给他两口印:一口《双合印》,一口《血手印》。明天《算粮》《大登殿》。第二天《夸官》碰见西宫娘娘,他《砸(了)銮驾》。《殷家堡》到公馆,修表下书这就《辞朝》。皇上不准他跑了,有一吏部天官《忠保国》《赶黄袍》。

《殷家堡》《过江》上了船,过得江走了一箭远,有《扇坟》,有《碰碑》,上写该死该死《殷家堡》。他一怒,《碰碑》撞死了。《忠保国》《大回朝》,说《殷家堡》是个《忠烈臣》。《十里亭》《祭江》,西宫娘娘接皇上,她说《忠保国》有欺君之罪,讨了《假金牌》,要斩《忠保国》。四外闻听全反了,《反延安》,《反西凉》,《反唐》《大闹翠花宫》。皇上急了亦《逼宫》,文武大臣上殿,《打金砖》《骂杨广》《上天台》,《紫微星》亦归了位。

他这段《五百出戏名儿》,亦没准词儿,唱的时候亦常更改。穷不怕“攥(zuàn)柳”(江湖人对于戏通经调[diào]侃儿叫攥柳),梨园行人都佩服他。在这段玩艺儿之外,还有些个关系戏剧的玩艺儿,不过失了传,亦无人知道了。他还有一段儿《百家姓》,很有意思,现在还有人唱,总算没失传。

穷不怕的玩艺儿,随便换辙。他有诗唱《五百出戏名儿》,还用幺调(即遥条)辙。那段辙口是:“我唱一回《天官赐福》雨顺风调,《卸甲》封王在唐朝。这位爷戴着一顶《遇龙封官》帽,上镶着《海潮珠》、《庆顶珠》放光毫。腰中紧系《乾坤带》,身穿一件《斩黄袍》。《借靴》一双蹬足下,《秦琼卖马》上了鞍鞒。今日我一到《八蜡庙》,为的佛会把香烧。《进香》为的是阴诰与阳诰,保佑我一家《三娘教子》《长生乐》来《太平桥》……”这段是要钱的玩艺儿。

小段儿是垫场活,我再把他的小段儿写一段。他唱的有《一面黑》:“霸王生来一面黑,摆上酒宴请李逵。上座坐着是王翦,下首李刚又把客陪。牛皋按着兀朮(zhū)打,直急得侯公泪双垂。惹祸本是包文正,只皆因周仓去做贼。三个人商量去偷焦赞,盗的是郑子明衣甲与头盔。敬德(děi)闻听失了盗,招惹得姚期往西追。一追追到西山后,瞧见了猪八戒王彦章灶王爷他去拉煤。”这段玩艺儿是灰堆辙。说书、唱曲和唱大戏,讲究音韵,十三道大辙。没本领的江湖艺人所唱的玩艺儿,都是师父向徒弟口传心授的,死套子活,有辙口儿都不敢改。江湖的艺人十有八九都“不钻(zuǎn)朵儿”(管不识字调[diào]侃儿叫不钻朵儿)。穷不怕“朵儿上清头”(管识字通文理调侃儿叫朵儿上清头),他的玩艺儿都是活的,能够随便拆改。





穷不怕的对子和其他小段儿


穷不怕的对子,极有趣味。我说他几个对子:“北燕南飞双翅东西分上下;前车后辙两轮左右走高低。”“南大人向北征东灭西退;春掌柜卖夏布秋收冬藏。”“道傍蔴叶伸绿手要甚要甚;池内莲花攥绯拳打谁打谁。”“风吹荷叶如卷饼;雨打菱角疙瘩汤。”“船载货物货重船轻轻载重;丈量地土地长丈短短量长。”“书童研墨墨抹书童一目墨;梅香添煤煤爆梅香两眉煤。”“羊入杨林羊吃杨叶芽;草庐驼草草压草庐腰。”这些个对子还不算好,他有个倒念正念的对子:“画上荷花和尚画;书临汉字翰林书”。这对子倒着由底下往上念,还是一样的音韵,不过字儿不同:“画尚和花荷上画;书林翰字汉临书。”此外还有三个字,同是三点水,草字头,什么“大丈夫”“江海湖”“芙蓉花”“姐妹妈”“常当当(當當)”“吃喝唱”,“只因我吃喝唱,才落得常当当(當當)”。

他的玩艺儿最好是合辙押韵,由庄而谐,不失劝人的宗旨。不似别人,玩艺儿虽是逗笑,抓的包袱儿都是无理取闹。还有一小段儿,亦很有趣。他唱:“匡公打马出西城,瞧见两个蛐蛐吹牛皮。这个说一口咬倒大杨树,那个说一口咬死大叫驴。两个蛐蛐正说大话,由南边来了个大公鸡。蛐蛐一见,‘呦嗒’一声喂了鸡。”他的《百家姓》亦唱得好,是:“念书的君子乐安康,千字文百家姓细说衷肠。这位爷戴高冠陪辇,穿一件乃服衣裳。腰中系的岳宗泰岱,费廉岑薛蹬一双。带的本是日月盈昃晨对字表,荷包里装的晨宿列张。手里扇的福缘善庆,叫一声孔曹严华细听端详。槽头上拉出我的鲁韦昌马,背上了一盘郝邬安常。用手接过来边扈燕冀,丁宣贲邓渊澄取映。今日已到俯仰廊庙,为我娘烧的骇跃超骧。一路走的是池桥阴鬱,瞧见些个俞任袁柳。买卖街上,东街上住的一个曾毋沙乜,他家有闻莘党翟经房,女慕贞洁珠称夜光。诸姑姐妹往里让,孔怀兄弟拉住衣裳。一进门走的本是暨居衡步,卑阑屠蒙放在中场。四个陪客赵钱孙李,四个厨子周吴郑王。上来一碗海咸河淡,端上一碗菜重芥姜。一碗姬申扶堵,找补了一碗诗赞羔羊。淳于单于两盆菜,高夏蔡田端在中央。这位爷吃一碗具膳餐饭,泡了半碗雷贺倪汤,吃了碗云苏潘葛,找补半碗奚范彭郎。顿时间吃了个饱饫烹宰,将我让在苗凤花方。手里拿着樊胡凌霍,点着一盏银灯辉煌。丫环端上金生丽水,喝下了柏水窦章。外边进来了女慕贞洁,与他说的四德五常。觉忽下身杜阮蓝闵,来人搀到了谈宋茅庞。没容解开计伏成戴,拉了一裤子酆鲍史唐。”

“一字写出来一见方,二字写出来上短下长。三字本是川字模样,四字四角四方。五字本是半边俏,六字三点一横长。七字凤凰单展翅,八字分阴阳。九字金钩模样,十字一横一竖站中央。”这段玩艺儿,是《千字文》带《百家姓》,有时唱大段,有时小段。这段虽是江洋辙,可是他唱大段另使别的辙口。





穷不怕的大段玩艺儿


穷不怕的大段玩艺儿有《百山图》,唱出一百个山来,还带古人名。他的《百山图》唱的是:“打猎之人进山口,层层密密山套山。闲来无事山头上站,四面八方把山景观。东至福山高万丈,南至华山永无边,西至灵山我佛地,北至汴山半边天。金山银山离不远,铜山铁山紧相连。太行山有万丈,四川有座峨眉山。须弥山高无有人见,昆仑山上景致全。山东有个蓬莱岛,七十二座有名山。伯夷叔齐不吃周家饭,弟兄饿死在首阳山。渭水河边太公请,点将封神在岐山。王禅道号鬼谷子,归隐荒野云蒙山。骑牛架拐燕孙膑,修行得道天台山。黄伯英怒摆阴魂阵,金泥一座万塔山。寿星本是掌教主,打坐参禅白鹤山。大闹天宫孙大圣,扯旗为王花果山。唐僧西天把经取,牛魔王大战火焰山。黑风山前袈裟盗,奎木狼独霸平顶山。石头山,石头洞,獬豸洞在麒麟山。陷空山无底洞,蝎子精独霸琵琶山。度朔山有东方朔,行者压在五行山。汉高祖起义咸阳破,剑斩白蛇芒砀山。未央宫中斩韩信,才有十面埋伏九里山。朱买臣打柴难度日,终朝打柴烂柯山。剐莽诛苏昆阳破,严子陵下了富春山。李渊路遇贼杨广,秦琼救驾临潼山。唐国公四子李元霸,英雄锤震四平山。十八国的王子扬州会,弟兄结拜两截山。唐王御驾征东去,被困就在凤凰山。白袍淤泥河救过驾,卖弓计三箭定天山。罗成大战高谈圣,日锁五龙在嘉山。安敬思扔虎跳过涧,恩收养子飞虎山。回马挑死高嗣继,王文自刎落安山。五龙二虎彦章锁,李敬王气死宝鸡山。陈抟老祖爱睡觉,赵太祖下棋输华山。晁盖劫夺生辰纲,弟兄结义上梁山。三拳打死郑屠户,鲁智深为王二龙山。潘巧云勾引海和尚,杨雄石秀大闹翠屏山。十一郎盗取通天犀,青面虎为王虎啸山。地藏王骂秦桧,河里追僧九华山。徽钦二帝遭兵掳,岳老爷大战牛头山。山海关总兵吴三桂,借取清兵长白山。中华山,黄华山,鲇鱼山,甲鱼山,别谷沿好景致,小甸东傍罗山,鸡公山紧对僧帽山,棺材山改为元宝山。三月三蟠桃山,四月初六兴隆会,大会就在草帽山。热河有个棒槌山,九层山口密云县,牛郎山过去有罗山。沙店紧靠广安岭,张家口外六青山……”这段《百山图》,就是“皮儿薄”(江湖人对于言浅义薄使人易懂的玩艺儿,调[diào]侃儿叫“皮儿薄”),无论什么人听了都能懂。很有人爱听这个曲儿,盛行过几年。直到如今,相声行中虽没人唱,唱大鼓的坤角儿十有八九会唱《百山图》。鼓界大王刘宝全亦唱过这段儿。《百山图》虽是相声里的玩艺儿,现如今还没失传,可没报迁移就搬到鼓界去了。

穷不怕是纯粹单春,绝不与人合作,不说双春(对口相声),还有样特别,不说死套子活,凡是别人的段子,他还不动,专以自创新活取贵。早年虽有一个人的相声,不是明春,都是用布帐子挡着学鸡、猫、狗那种口技,调(diào)侃儿叫“暗春”。相声行中有一种单口活儿,八段《滋儿淘气》,他亦不说。现在相声行人会说八段《滋儿淘气》的虽然还有,可是在场决定不说。我在早年听过几段《滋儿淘气》,哪段亦有趣味。我先说他一段:在某巷内住着一人,叫滋儿,好诙谐,专好和人开玩笑。他能遇事当场抓哏,凑个趣儿,招得人乐得前仰后合。可得他占长风,本人一点亏都不吃。有天滋儿在屋中坐着,听见街上有做小买卖的吆喝:“鸡蛋呀!”他有心买鸡蛋,由里边跑出来,大声喊叫:“鸡蛋来!鸡蛋来!”那卖鸡蛋的听见这样,他绝不答应声“哎”。如若答应了,他岂不成了鸡蛋?每逢有人这样叫鸡蛋,他不惟不答应,还这样回答:“哪儿叫鸡蛋?”如若买主说“我叫鸡蛋”,那买主就成了鸡蛋。当时滋儿叫鸡蛋,那卖鸡蛋的就问:“哪儿叫鸡蛋?”滋儿一时莽撞,说:“我叫鸡蛋!”卖鸡蛋小贩冲滋儿一乐,滋儿就知道上了当啦,当时没言语,把卖鸡蛋的恨在心中,他要耍笑卖鸡蛋的。天天听声儿,日久天长,把卖鸡蛋的嗓音听熟了,他记在心中。到了十一月,天气严寒,冻得人伸不出手来。滋儿睡晌午觉的时候,听见卖鸡蛋的吆喝,穿着灰布棉袍儿跑出来,叫:“鸡蛋来!”卖鸡蛋的问:“哪儿叫鸡蛋?”滋儿说:“我买你的鸡蛋。”卖鸡蛋的到他台阶下,放下担儿,两个人讲价钱看货。把价儿说好了,滋儿不等卖鸡蛋拿笸箩,他由筐内取出鸡蛋来,往台阶上就放。鸡蛋要轱辘,卖鸡蛋的怕掉在地下摔碎了,忙着用手去扶。滋儿乘他用手扶着的时候,忙着就往台阶上放。卖鸡蛋的将身蹲下,用胳膊搂着鸡蛋,说:“你别放在台阶上,等我拿笸箩,你往笸箩内数吧。”滋儿说:“不用往笸箩里数,我数完了,就用簸箕来端。”二三百鸡蛋,眨眼之间,一五一十,他都放在台阶上。那卖鸡蛋的纹丝不敢动,怕摔了鸡蛋。滋儿看他这种样子,要冻会儿能成了冰。他说:“你等着,我进去取家伙,来拿鸡蛋。”说完进去,将街门关上。他告诉家中的人,那外边卖鸡蛋的无论怎么嚷,亦别理他。说完了,又去躺着。暖暖的屋子,舒服极了。卖鸡蛋等的工夫大了,不见滋儿出来,他急得直嚷:“大爷!你取出家伙没有?”他嚷了十几声,亦没人答言,冻得他难受,扯开了嗓子嚷:“大爷!你快出来吧!冻得我手都疼了!”亦没人理他。直把嗓子喊干了,亦没人出来。他冻得实在支持不了啦,滋儿换了一身青衣服,戴上墨镜,由后门出去绕到前边。那卖鸡蛋的没有那么好的眼力,亦不认识他了。他问卖鸡蛋的:“掌柜的,你嚷什么?”卖鸡蛋的说:“先生,这门内有人买我的鸡蛋,说好了价儿,数了数儿,亦不出来了。我不敢动转,一动鸡蛋就轱辘地上,都摔碎了。你行点儿好,替我把鸡蛋都挪到筐内吧!”滋儿说:“要挪开亦成,你得先给我作个揖!”卖鸡蛋的说:“我要能动转,还不急哪。”这段《滋儿淘气》要说到这里,面上得形容卖鸡蛋的急状,变出急愤的口吻。滋儿说给他作揖,得叫听主领会卖鸡蛋的不能动转。把人逗乐了,全凭面貌上的发托卖像(指演员在表演时要惟妙惟肖,通过喜怒哀乐刻画艺术形象),由神气中传来,实在不易。

相声行人怕说单口活,亦是单春较比对口相声难说。双春逗的哏,响的时候多(把人逗乐了,调[diào]侃儿叫响了;没把人逗乐,调侃儿叫闷了)。单春的哏,只要神气上欠点儿火候,就得闷了。以这种情形推测,相声行人是以单春的玩艺儿当作重头活。滑稽大王万人迷,本领虽好,亦是双活见长,有说单的时候,亦恐不多。





穷不怕首创单春 在某王府长期献艺


穷不怕做了多少年的艺,总是说单春,实在不易。若不是肚子里宽绰,哪儿能行啊。他到了晚年,把万儿(名儿)创出去,亦做了家档子(堂会)。什么是把万儿创出去哪?江湖人,甲乙相见,如不明言,欲问姓名,就调侃儿,问:“你是什么万儿哪?”如若某人的名姓大,调侃儿说“有万儿”;如若某人的姓名没有人知道,调侃儿就说“没有万儿”;如若某人的姓名臭了,提出某人的姓名没有人赞成,调侃儿就说“万儿念啦”;如若某人的品行好,人人恭敬,提起他的姓名人人赞成,调侃儿就说“万儿正”。江湖人对于名姓亦很重视,可见哪行要把名儿做出去,亦是不易。江湖人若能享了大名,调侃儿就说“响了万儿啦”。穷不怕就是江湖中响了万儿的人。北城某王府的王爷闻其大名,约到府中作艺。穷不怕艺术之美,思想之奇,某王焉能不喜,待遇之优,所有的艺术人都比不了啊!每日两餐,按月领银,外加六品俸银。他虽收入丰富,为人勤俭,仍然身穿破衣,撂地做艺。传至如今,穷不怕的玩艺儿还有会的,亦不过拾其余唾,难以“置杵”(江湖人对于不能挣钱调[diào]侃儿叫不置杵)。穷之门人小桂、徐三,亦红过几年。焦德海、卢德厚(卢三)等,皆徐三之徒。至今焦之高徒张寿臣又执该界牛耳了。





袁桂林当场抓哏 以抖搂包袱儿挣钱


在各省市各码头的市场庙会中,有一种唱戏的卖膏药,都是弄几件糟朽不堪的行头,在一个场内扮出个武生的角色,头戴一顶皂青缎色软胎壮帽,身上不换行头,不是有条棍,就是有条枪。在场内练起来,就能把游逛的人们引去,在场的四面围着观瞧。他们的戏,总是拿嘴说,永远不唱,说完了以卖膏药挣钱。干这行的北平还少,天津最多。最能挣钱的有两个人,一个是袁桂林,一个是何小辫。袁是文做,何是武做。袁桂林口齿伶俐,嗓音宏亮。他能见景生情,当场抓哏,把四面的观众逗乐了。他抖的包袱儿最多。在民国三四年,他在天津三不管做艺,我听过他几回。他出来只带一个小包,到了场内,打开包,戴壮帽就能圆上粘(nián)子。什么叫做圆粘子哪?江湖人管他们的玩艺儿场四面围着的人,调侃儿叫“粘子”。如若有场而没人的时候,他们设法叫人围着观瞧,那要调侃儿就是“圆粘子”。如若四面的人都围上了,调侃儿叫“圆上了粘子”。

袁桂林就在圆上了粘子之后,向四面说:“众位!我亦是个唱戏的。那位若问我唱什么,我是梆子班的。别看我这样儿不好,我与大名鼎鼎的元元红还是师兄弟哪。那位说,你师哥叫元元红,你叫什么红呢?他叫元元红,我叫山里红。”冷不防说出山里红来,招得听主都得笑了。他还说:“我师兄元元红唱戏能叫座儿,我山里红唱戏更能叫座儿。有一回我在协盛园唱戏,将一挑帘……”他说到这里,用双手作势说:“‘哗……’那位说,这是叫好儿吧?不是,这是外边下起雨来了。干什么亦得走运。咱姓袁,叫袁桂林,唱戏的时候很红过几年。那位说,就凭你这点嗓音还唱得好吗?其实唱戏讲究音韵,不在嗓门儿大小。叫驴的嗓门儿大,拉胡琴的没法定弦。唱得好,做派还得好。”说着话,他用手一捂脑袋,说:“这叫什么?这叫正冠。”又用双手往下巴颏儿一捋:“这叫捋髯。”又用手一撩衣裳的大襟,说:“这叫什么?这叫撩袍。”又用手往腰间一托,说:“这叫什么?这叫端带。”用手一指,说:“这叫什么?这叫亮靴,是叫众位看看破鞋。今天我犯了戏瘾,要在这里唱一出。唱得好歹,众位给我传名。你们可别给我人传名,得给我的宝贝传名。那位说,你的宝贝是什么?我取出来,叫众位看看。”说着,他由打包内取出个纸包来,有五六寸长,四寸来宽。他用手指着这个纸包说:“我这东西,今天白送,每人一个。可有几种人不送:聋子不送,哑巴不送,小孩儿不送,在家不知道孝顺父母的不送,在外边不懂得交朋友的不送。那位说,你这东西都送给什么人呢?我送的是外场外面,懂得交朋友的人。那位说,你这是什么东西呢?我这是戏班的宝贝。那位说,你这宝贝是什么,有什么用哪?我这宝贝治跌打损伤,闪腰岔气,筋骨跳槽。哪位要买,我可不卖,我不是卖膏药的。这是我们戏班里预备的好药,为的是自己用的。我们打武行的,成天在台上跳动,没准儿哪阵腰腿筋骨受伤。如若要上台啦,或是正唱着戏哪,临时有病,不能撂下戏不唱。虽然不好受,亦得挣扎着上台。如若是筋骨的毛病,当时贴上我们的膏药,就能止痛消肿,上台唱戏。唱戏的讲究喜乐悲欢,自己心里烦,到了台上应笑还得笑,不能因为自己烦,该笑不笑。有屎有尿,亦得应付着。两胁胛力压泰山,三支袖箭镇淮安。俺,费德功,今天八蜡庙会之期,孩子们,拿盆子来,我要撒尿!那成吗?这种膏药叫海马万应膏,我母亲有心愿,教我施舍一千张。今天我是白送,每人一张。我要自己说好,那是老王卖瓜,自卖自夸。我有个法子,真金不怕火炼,好货不怕试验。不论哪位要带着病,你言语声儿,咱们试验试验,可是专治筋骨疼痛,跌打损伤,不治呕吐恶心。如若治呕吐亦成,用膏药把嘴贴上。哪位要有病,咱们贴张试试?”他这样说,又逗笑儿,又往下叫点儿。

阅者诸君若问什么是叫点儿,我先说说这桩。江湖中的人对于社会里边普通的人与能花钱照顾他们的人,调(diào)侃儿都叫“点儿”。外行人无论做什么生意,都是等主道候客。江湖人有叫无意买东西的人当时照顾他们这种本领,调(diào)侃儿就是往下“叫点”。

当时袁桂林这样说,就有人贪便宜,叫他给治。恰巧这人是多年的腿疼。他叫这人将腿带解开,裤子往上捋,好贴膏药。由包内要拿膏药了,他向四面说道:“你想这膏药,有好有歹吧?我别自己拿,这包内是三十张膏药,我找一位替我拿一张。”他说完了,就问:“哪位帮帮忙儿?”有好管闲事的人说:“我替你拿!”袁桂林说:“你替我拿,咱们还别脸儿对脸儿,别有人说我向你使眼神儿。我回过身去,将膏药托在后边,我不看你,随便一张就得。”说完了,将三十张膏药用手捻开,两只手一背,叫这人拿一帖。这人伸手由里边拿出一帖来。袁桂林的两只眼望四面一看,和四面观众一对眼光儿,向后边问道:“你给拿出来没有?”这人说:“拿出来了。”他故作惊慌,问道:“你给我拿出来是什么?”这人说:“是膏药。”他说:“好!你不说明,吓我一跳!”他这么一说,四面的人一琢磨,都能笑了。这亦是当场抓哏,抖搂包袱儿。不过这种包袱儿最难抖搂,这是暗包袱儿,要叫人乐呀,必须得传神,才能有人乐。

袁桂林将另二十九帖膏药放在一旁,用火纸将那张膏药烤开,要给这人往腿上贴了。他矮下身去,蹲在腿旁,用嘴向这人腿上去哈。哈了会儿,将膏药贴在腿上,向这人说:“我要给你治好了腿,你能给我传名吗?”这人说:“能给先生传名。”他说:“你给我传名,我姓什么?”这人被他问得张口结舌,一句话亦说不出来。他说:“幸而我问你,要不然你的腿好了,有人问你是谁给治好的,你还许说是哈先生,给你哈好的。”他这一说,四面的人又都笑了。他叫这人走几步儿,又问道:“你这腿还疼不疼哪?”这人说:“不疼了。”我从前猜不透是药有效力,还是有病的人是“敲托的”(江湖中对于贴靴的[暗中帮助做生意的]人调侃儿叫敲托的)。及至各方探讨,才知道不是膏药的力量,多好的膏药亦不能立时生效。而是他用嘴的时候,大声小声,逗人笑了,乘人不大注意,用手按着腿部筋骨的穴道,“上托”(江湖人另有一种传授,对于各种筋骨疼有一种推拿、掐拿的法子,当时能止疼,管这种掐拿法调侃儿叫上托)。外行人不知道,见这人贴膏药立时止疼,谁不赞成哪?他连着治了几个人,都是明着贴膏药,暗着上托,见了响儿,就能卖钱。什么是见了响儿哪?江湖人管当时讨好,使人立时发生信仰力,调(diào)侃儿叫“见响儿”。

袁桂林乘着人们相信的时候,还说白送膏药不要钱,要送二十张,谁全要接他的传单。社会里的人都是好贪便宜,恐后争先地抢他的传单。把传单接到手啦,心里安慰了,觉着白得张膏药。其实江湖中的人使用这种方法,是“太公钓鱼——叫人上钩儿”。及至把单子接了去,那就上了“鬼插腿儿”的当了!什么叫鬼插腿儿哪?我将这句侃儿和这个方法解释解释。大家攥着传单等他白送。他说:“这药能治腰腿疼,筋骨麻木,跌打损伤……我要白送,众位拿着心中不安,买药没有不花钱的。我要卖这膏药,得卖两毛钱一张。今天我是减价一半,卖一毛钱一张。我给君子人开条道,小人推道墙。我是不赚钱,如若赚一文钱,叫我……那位说你不是白送吗?我说送就送。哪位要买一帖,我送一帖,不买不送,多了不送,多了不卖,三十份为止。过了三十份之外,仍卖两毛一帖。亦许钱多了众位不买,钱少了我不卖,哪位要,那位掏钱。”他这样说,就是每人一毛,共是三元。明说白送,暗着要钱。先把便宜传单教人攥,不知不觉的亦要了钱,那就是鬼插腿儿。如若先说要钱,就许没人买。如今商家有学会了这种方法的,牺牲血本大减价白送一天,结果是在一个月内择出一天,买东西的人以货单为凭,按价值白送点儿最贱的东西搪塞了事!鬼插腿儿的办法,岂止是江湖人会使。





卖戏法儿的不挑(tiǎo)真门子 变戏法儿的腥(假的)尖(真的)都不卖


幻术是最普通的艺术。往轻了看,是种游戏的玩艺儿;往重了看,不只是娱乐中有趣味,还能启发民智。若好习研究戏法儿,能增进人思考之力。戏法儿实是有益于社会呀!戏法儿分为新旧。我国的幻术界中的势力不分新旧,都在江湖人的掌握中。魔术大王韩秉谦,以及王祝三、韩敬文、张敬扶、大天一、王福林、刘静斋等,快手卢、快手刘、戏法杨、戏法罗、金麻子、狗熊程等,都是江湖中的人物。他们这行儿,亦不论是魔术、幻术,只要是变戏法儿的,就算彩门人。不知者以为变戏法儿的只要变几样干净利落的玩艺儿,就能挣钱成名,其实不是那么回事。变戏法儿的人,变的玩艺儿好,不如嘴里说得好。江湖人常说“金皮彩挂(金指算卦相面,皮指卖药,彩指变戏法儿,挂指打把式卖艺),全凭说话”。由这句话推测,彩门的玩艺儿亦是仗着说话,三分变,七分说。说的都是什么,能比变还重要呢?先以变戏法儿说。江湖人管这行儿调(diào)侃儿叫“彩立子”,又叫“干子”。彩立子分文武,文的是“小抹子活”(小戏法),武是“落(lào)活”(变戏法儿的人由身上往下落东西)。他们这行儿在早年规矩很大,学会了做艺挣钱糊口成了,不准将艺术卖与外门人。在清代,市井庙会只有变戏法儿的,没有卖戏法儿的。在近年来,我国华南、华北各省市、各乡镇,卖戏法儿的遍地皆是。不知道的以为他们是变戏法儿的改行,其实他们都不是变戏法儿的,是卖戏法儿的,都是另一行儿,与变戏法儿的并没关系。他们这两行人有个极重大的界限:是变戏法儿的,绝定不卖;是卖戏法儿的,绝不以变挣钱。

卖戏法儿的这种行当,调侃儿叫“挑(tiǎo)厨供(gòng)的”,在早年没有这一行儿。我对于挑厨供这句侃儿,曾有研究,和人讨论过。当初的江湖人最讲义气,甲江湖人对乙江湖人绝不欺骗。如若甲欺骗了乙,乙就说:“好呵!咱们都是合字儿,你不该厨供我呀?”由这样推测,厨供是极坏了。江湖人都怕厨供,怕的是什么呢?就是乙对甲,得天天不间断地供应甲的应用财物。除了对他贡献之外,任凭什么好处,亦怕得不着!由这种意义考究,卖戏法儿的是欺骗人的行当了。哪行人亦是有好有坏。好的放在一旁,先不用说他,先以坏的来讲。凡是卖戏法儿的,都不大会变,他们下功夫的玩艺儿就是“苗子”。什么叫苗子哪?就是他们变的那几个红豆儿。那种东西亦不是珊瑚子的,亦不是化学的,那是蜜蜡做的。若是不使用,放在盆中,几十年亦不干。那东西的体质是软中硬,硬中软。外行人看着,绝不知那东西的原料是什么,绝不知道那东西的体质。那东西是卖戏法儿的就有,是变戏法儿的就得会变。变戏法儿的学艺的时候,初步功夫就得学它。凡是学仙人摘豆的,都是小孩儿,大人绝定学不了。

变仙人摘豆的,以变戏法儿的艺人变得最好。总是他们以变戏法儿挣钱,变得不好不能挣钱。有了这种关系,是变戏法儿的人对于变仙人摘豆都下过苦功。卖戏法儿的人以卖戏法儿挣钱,对于仙人摘豆,只要会变就得,不求其精。学会了,变时亦不要钱,白变白看。白吃包子亦没有人嫌面黑,他们只用仙人摘豆“圆粘(nián)儿”,亦不必多下工夫。什么叫圆粘儿呢?凡是江湖玩艺儿,都得是有人看,有人围着听,才能挣钱。可是他们在各庙会地方,各市场内,都有一种引人围着场子的法子,那种法子调(diào)侃儿叫圆粘儿。仙人摘豆到了他们卖戏法儿的手内,只能圆粘儿,不能挣钱,变得好坏没有关系。卖戏法儿的有行规,不准撂地摊儿,不准敲锣鼓,不准往外卖真“门子”。什么叫门子呢?江湖人都知道变戏法儿的家伙上,不论是哪样儿,亦有一种令人测不透的机关,那种秘密的机关调侃儿叫门子。我说一样有门子的戏法儿吧。各处变戏法儿的有用“搬铲”的,什么叫搬铲呢?在一个茶杯内扣个琉璃球儿,再挪开茶杯,琉璃球没了,能变个鸡蛋,这种戏法调侃儿叫“搬铲”。茶杯内就有门子:那机关是个铁片,三角形,有个轴儿,扳簧在杯底上。变时用右手拿起茶杯,扣琉璃球的工夫,手指暗搬轴簧,铁片转动,原有的鸡蛋落下来,琉璃球被铲在片上。

学仙人摘豆,都是童子功。小孩儿的筋骨又嫩又柔软,在发育的时期,手指曲伸,“捏、掐、夹、粘”四个字的功夫都能练得会。惟有人到了二十岁里外,筋骨长成了,再练这种功夫,筋不长,骨不软,练亦不成。至于这种戏法变得好坏,亦由豆儿上分别出来。豆儿有大小,指有长短,手有胖瘦。以手胖肉厚,指并无缝,豆夹在指间,不能外露为好,愈能变大个的豆儿愈好。如若手瘦,指间多缝,豆夹指间,容易外露,并且那种手不敢变大个的豆儿。那种豆儿,变戏法儿的虽然都传有,都得会变,可都不会做。“攥弄(zuàn nong)那啃(kèn)”的,是济南府最“撮(zuō)啃”。什么叫攥弄那啃呢?江湖人对于制造那豆儿调侃儿叫攥弄那啃。什么叫撮啃哪?江湖人对于东西做得好,调侃儿叫“撮啃”。据彩行人说,豆儿做时很费手续,“底啃”“又沉”(江湖人对于制造物品的原料调侃儿叫底啃,又沉是本钱不轻),“肘中个苗儿”,“汪载(zhāi)车(jū)迷杵儿”才能成哪(买五个豆儿调侃儿叫肘中个苗儿,三四块大洋调侃儿叫汪载车迷杵儿)。变仙人摘豆的时候,必须先吞在口内,用口中热气、唾沫润了,豆的粘性才发。往指间夹时,一半仗其粘性粘住,一半仗着指上功夫夹住。如不往口内放,粘性不犯,豆儿又硬又滑,恐怕夹亦夹不住啊。这琉璃球变鸡蛋,就仗着茶杯内的门子。这种戏法儿,卖戏法儿的绝不卖给学戏法儿的人,那个真门子绝不能叫外行人知道。

以这一样作为考据,是变戏法儿的玩艺儿,卖戏法儿的都不往外卖。他们所卖的戏法儿是另一种玩艺儿。卖戏法儿的这行儿,在北京是“戏法杨”,在天津是“戏法祁”创出来的。在早年他们这行人只会“做前棚”往外“挑幅子”。什么是做前棚挑幅子呢?他们做这生意,都得会在杂技场内支摊子,变几样戏法儿吸引观众。四面的人围上了,调(diào)侃儿算圆好粘(nián)子,然后随变随说他们的四门戏法儿。哪四门呢?有手法门、药法门、彩法门、符法门四大门的戏法儿。我按着这四门一齐说就乱了,分门别类一样一样谈。先说手法门吧。是用手变的戏法儿都是手法门的玩艺儿,有“仙人摘豆”、“巧耍连环”、“三仙归洞”、“仙人解帕”、“仙人套环”、“霸王卸甲”、“月下传丹”等等。别看他们把剑、丹、豆、环四样戏法儿列入,外人学不了,这四样不算戏法儿,那算功夫。吞宝剑是真的,非童子功不成!一个琉璃球在手中口中变起来,忽有忽无,神出鬼入,令人难测,那亦得童子功才能练成。仙人摘豆我已然说过,不必再说。巧耍连环亦是真功夫,没有几年的功夫练不好。这四样,卖戏法儿的卖给外行亦学不了!用块绸子手绢变仙人解帕,外行人当时就能学会,那挑(tiǎo)的是“把尖托”。什么是挑把尖托呢?凡是江湖人,卖了手儿真的,调侃儿就叫“挑把尖托”。金钱过桌的戏法儿亦挑的是尖托。这两样儿为什么挑的是尖托哪?他们变戏法儿的都不变这种玩艺儿。

彩法门的戏法儿,他们卖的有“棒打金钱”、“平地拔杯”、“空盒变烟”、“空盒变洋火”、“飞钱不见”、“烟卷自起”、“破扇还原”、“扇子生财”、“杯中生莲”等等的玩艺儿。这些样戏法儿,怎么叫彩法呢?因为这些戏法儿,所有用的家伙上都带着彩儿。除了平地拔杯是变戏法儿的玩艺儿,其余的都是变戏法儿的人们研究出来的。如若有人买戏法儿,什么真的都买得着,就是那平地拔杯布内的机关绝对不卖,绝不叫外行人知道。因为哪个戏法儿亦是变的玩艺儿,变的人指它挣钱。戏法儿,不知道其中的内幕,还有意思,还有趣味,及至知道是怎么回事了,那就没有意思了。从前我在卖戏法儿的场内见他们将吸着了的半根洋烟卷,往左拳内顺着虎口插入,再将手张开,那烟卷就没了,觉着神妙已极。及至他们将这手戏法儿告诉我,实在是乏味。其法是用根猴筋(胶皮筋),将一头儿系在衣内,一头儿由袖筒穿过,通于袖口之外,头儿上用洋皮片做一夹子,变时左手攥着夹子,右手烟卷插入夹内,张手时猴筋即将烟卷缩到袖内了。这飞钱不见的彩门,就在那根皮筋的缩力上。烟卷到铁片夹内立刻就灭,亦烧不着衣服。我费了许多的手续,试验过一次,就将东西抛了,再不想变这戏法儿了。

彩门的玩艺儿在戏法儿里还算是正经东西,那药法门的戏法儿多是腥(假)的。就以那“小鬼叫门”说吧。“谁要学那戏法儿,可以和人开玩笑,将药抹在谁家的大门上,夜内门上啪啪总响,如有人拍门一样。及至出来一看,外面无人,能扰得人夜内睡不着觉,小鬼叫门很有意思。”这样戏法儿只可听他们说,就是别学。如若花几个钱学啊,他们告诉你:“这种戏法儿是药法门的,往药铺买天南星少许,研成细末,用醋抹药涂于门上,夜内那药性发作,啪啪就响,如同有人叫门似的。”凡是学生都花钱买药,依法去办。先不用抹在别家的门上,往自己的门上抹回试试,结果没有效力,那药没有那么大的力量,它并不响,白糟践钱。江湖人说:“他们这种玩艺儿是一腥到底,假到头了。”还有那手“美女脱衣”的戏法,他们说得更神乎其神。如若其药弄成了,藏在手指甲盖内,往女子的脖领内一弹,那药性发作,刺痒难挨,女子就脱衣裳。他们这样说,那“臭子点”就愿学(江湖人对于好色性乱,好偷香窃玉的人,调[diào]侃儿叫“臭子点”)。及至花钱去学他们这戏法儿,认为这是能够及时行乐的无上妙法,结果卖戏法儿的告诉你:“这美女脱衣的戏法儿亦是药法门的。将桃毛弄少许,藏在指甲内,用时暗弹在妇女脖领之内,即可生效。”这种办法还是冤人,他只叫人去蹭桃毛,那算把人冤着了!

他们这手、彩、药三法门的玩艺儿,虽是腥的多,花钱不多,上当亦有限。而他们挣钱的本领都是仗符法门的玩艺儿。前些年魔术家孙宝善,就专以卖手法门、彩法门、药法门的戏法儿挣钱。他那幅子(传单)纸,每张印四样戏法儿,哪天亦能卖四五十张。除他以外,再找个能卖几十张幅子的,恐怕没有!这些卖符法门戏法儿的都卖什么呢?有“叫牌法”,有“搬运法”,有“八仙转桌”,有“抽签叫大点”。有学他们的叫牌法的人,他们说:“这种方法,不论是打麻雀、打扑克、推牌九、押宝,只要是关于赌的耍儿,要学会了叫牌法,想用什么牌,就叫什么牌。譬如推牌九吧,上家是露出九点,对门是八点,下家亦是九点,庄家露出一张大天,手里攥着一张牌,不是大天,是别的点儿。如若叫:‘来张大天!’那手里的牌就能变成大天。一对大天,能赢三家吃个通。叫牌法就能有这种力量!如若打麻雀,手里的牌十三张,都是万子,成了四副牌,只有一个单张,单吊儿,还是万子,调出万碰和清三番。伸手抓牌不是万子,叫牌法的力量,就能叫出张万子来。如若打扑克,四家来明的,已然到了第四张了,扣着是幺,明着是对二,还有张三。如若末张牌来二,来三,来幺,都是两对儿。到了派末张牌,牌不是幺二三,是张别的牌。叫牌法的力量,能叫他变成幺来,凑成两对儿。”他们这样夸口,说得神乎其神,调(diào)侃儿叫“卖弄”。

有些个好赌的人们知道赌博场中有一种逢赌必赢的人,那是耍俩点的,可不是公道耍儿,都是仗着手底下闹鬼,倒替张儿。江湖人叫他们“老月”。好赌的人认为卖戏法儿的叫牌法,是按着耍俩点、吃腥(假)赌的老月使的法子,想要把那法子学会,花几个钱可不冤。将来有叫牌法的本领,就算耍钱,比干嘛都好。这种人的思想不好,卖戏法儿的又有诱惑之力,哪能不钻套儿,上当啊!吃腥赌的老月,都是身无正业、游手好闲的人,素不务正,天天不离赌博场,不用使花招儿,凭眼力与经验就能赢钱,再学会了使腥儿,那不是百赌百胜吗?可是他们学使腥儿,亦不容易,有真传授,还得下苦功夫,把两只手练出才能成哪。卖戏法的人并不是老月,他们要有那使腥的本领,就不干这个,专去耍钱了。他们对于老月的手彩儿亦有个一知半解,不过是一瓶子不满,半瓶子晃荡,就是知道那手彩怎么使,他们亦是使不好。本行人知底,卖戏法儿的不是老月。外行人以为卖戏法儿的人都是老月哪,花几个钱,学叫牌法吧。

哪想卖戏法儿的不说叫牌法是手彩,硬说是符法门的,得设坛画符念咒才能成哪。他们将人的钱弄到手,有一种措词,向学叫牌的人说:“我们会叫牌法,为什么我们不去赌钱,以卖戏法儿为生哪?我们这仙传之法,不准以叫牌法去赢人家的钱。如若以叫牌法赢人的钱,那个罪就大了。那么学这叫牌法有什么用哪?说若有人输了钱,约我们去捞,准能把输的钱捞回来。你们学了叫牌法,亦是一样。有朋友输钱,替他们捞成了,不准以此法术生财!”他们这样说,透着有公德,其实是骗人的退身步儿。那学叫牌法的人是要学会了叫牌法以赌生财,听他们这样说,嘴里回答:“我不赢人,学会了防备不输钱,如若输了好往回捞。”两下里彼此相欺。学的人多精明,亦得上当,亦得伤财;卖戏法儿的可得了利益了!他们不只是把钱骗到手,最可怜的是,叫被骗人给“尊罗子叩瓢儿”。什么是尊罗子呢?江湖人对于神佛像等物调(diào)侃儿叫尊罗子,对于人给神佛叩头调侃儿叫叩瓢儿。他们卖戏法儿的在各市场庙会做生意,附近租间房,屋内设着坛,上边供着吕洞宾、济小塘、柳仙。他们说戏法儿是这三个人遗留的,每天都叩头焚香。如若有人学符咒法,他们就叫人家买香烛纸码、鸡鸭等供品。上完了坛,这些东西无形之中归了他们还不算,格外得交些学款。要多少看人行事,总是把人挤兑得力尽为止,由数元至数百、数千元不止。

上坛的时候,还得叫学叫牌法的人跪在供桌前边,向吕洞宾、柳仙、济小塘磕几十个头。磕完了,他们就戳个“雨字头儿”给学叫牌法的人。什么叫雨字头儿呢?凡是咒符,开笔先写雨字,底下是钩儿圈儿一大串。江湖的人们对于画符调侃儿叫雨字头儿。写哪,调侃儿叫戳。他们卖戏法儿的符,有几道“样色”,使人信而不疑。什么叫样色呢?他们有一种火符子,将符画得了,叫学叫牌法的人拿回去,放在个秘密所在,不叫外人知道。天天给符磕头,七天为限。磕头的时候要在夜静更深,人都睡着了,磕四十九个头,在星斗之下能有效力。如若七天都是晴天,有星斗就成了;倘若有六日晴天,一日阴天,符是白画,头是白磕,不能有效。如若遇见七个晴天,就能成功,得将符打开了,在灯上照,如若看见符内隐隐现出人影儿,就是吕洞宾、柳仙、济小塘三个人有一个现身。把这张符带在身上,赌钱的时候用什么牌能有什么牌!可是往灯上照那符的时候,离着火苗还远着哪,那符呼的起火,自己就着了!弄得两手空空。

学叫牌法的人磕头烧香,买礼物,花了若干钱,结果他买来的一道符起火烧着了,必不肯甘心,找他们卖戏法儿的去问:“这是怎么回事?”卖戏法儿的说:“你的符烧着了,那是冲了!或是见了死人出的殡,亦犯冲!或是见了屠杀的行当,亦犯冲!或是你自己身上不干净,夫妇有房事,亦犯冲!你这冲了,这些事算白费了。”他们这样说,那学叫牌法的人在疑惑之间一定是想他怎么冲了,要不冲,那么好好的一道符怎么会自己着了火呢?不能怨卖戏法儿的,还是埋怨自己不留神,从前花多少钱学的,这回照样花钱,另上坛求符吧。这样可就上了卖戏法儿的当了。阅者若问上了什么当?这种当叫做“火符子”。我先把他们这火符子的黑幕揭穿了,公诸社会,免得有人受彼辈之愚。卖戏法儿的所用的符有好几种,就说这一种火符子,是用硫磺火硝化成水,用笔往黄毛边纸上写的字。写完了,在屋中放干了,然后用些硫磺疙瘩,在符的背面,符子头上尾上粘好喽,就算成了,这样就叫火符子。他们叫学叫牌法的人拿着这张符,往灯上去照,看看符上现人影不现,那才是冤苦了人哪!那符的底面,硫磺疙瘩被火气一熏,见热就着。那硫磺疙瘩如同药捻儿,它着了就能都烧啦。

再说那学叫牌法的人,二次又求他们一道符。卖戏法儿的当着他的面,叫他用眼看着,将那道符用纸包好,交给他拿回去,照样去磕头。不料学叫牌法的人将符拿回去,找个秘密所在,仍然供好,夜夜在星斗出全的时候磕七七四十九个头。磕到第七天打开纸包,要往灯上去照那符了,不料那包打开了,再找那道符啊,居然没有啦,不翼而飞!这下子可把他急坏了,再找卖戏法儿的麻烦哪,简直是白费。他们成年的骗人,自有强词夺理的话语,八面风儿——哪头儿来,哪头儿堵,叫人无可奈何!如若学叫牌法的人有了真正的觉悟,就许不学了,自认倒霉,算是完事。设若和他们捣乱,他们是不怕的。他们习惯性,宁可遭了官司,受刑事罪名,饱尝铁窗的苦处,亦不肯“倒(dào)把”。什么叫倒把呢?江湖人对于挣到手的钱又叫人将钱退回去,调(diào)侃儿叫倒把。他们如若倒了把,将钱退回了,叫同行人知道,就认为莫大之耻!觉着寒蠢无比!他们这种人,绝顶不肯倒把的。如若学叫牌法的人上了两次当,花了两回钱,还不觉悟,那是倒霉没到头儿,还得接着往下上当。

我这样说,阅者定然纳闷儿:那张纸包着符,怎么打开了看,那道符会没了呢?这种骗人的方法叫做“筋斗幅子”。筋斗幅子又是怎么回事呢?他们江湖人有一种包空包法子,除了有响动的东西,坚硬块儿大的东西不能成,其余的东西都能包。包的时候,叫买主看着,是把东西包在里面。拿回家去,不打开看便罢,及至打开看时,那包内的东西就没有了。这种法子调侃儿叫筋斗幅子。这幅子有好几种哪,我就说一种吧。用茶叶铺包茶叶的纸一方,要四方。第一要折成三角形,一角叠上,离对着的尖儿十分之二远。叠好了三角形之后,纸成了双层儿,将一根火柴横放纸上,先将纸的中间折十分之二,将洋火棍儿押上,再往下折十分之三,叠左尖儿,叠右尖儿,用双尖儿叠回掖好。包儿虽然包好,东西却在纸外,不在纸内。一般人对于包内有极贵重的东西,打开的时候,极慎重揭双尖儿,揭左右尖儿,见三角中间叠处,双尖儿,一长一短,必要用手捏住两个尖儿看。愈这样,愈中他们的计。筋斗包儿,是你不扯,那三叠揭的夹空间有东西。一扯纸绷起,如翻筋斗,那东西就在那时候一翻出去了!我说的这是后丢的东西,丢的是极小极轻,落地极不好找的物件才能成,翻出去就找不着。可是使用这种包儿的,都是骗人用的。钱到骗子手内,给你一点信仰物,将你所希望的目的推在物上,结果落个东西丢了,不能说没效力,还埋怨自己大意哪!

筋斗幅子就是江湖人的一种推送法。如若没有这种方法,人家学不会,哪儿能答应。卖给一道符,买主把符丢了,岂能埋怨卖主呀?卖戏法儿的人,有本领的能够天天骗人,不挪地方,亦没有人和他打架,只仗着“贴身靠儿”能够“平点儿”。什么叫贴身靠儿呢?人在社会中求谋衣食,全仗朋友维持。怎么才能和朋友换出感情哪?简单来说,就是投其所好。如若朋友好那样,你得说那样好;朋友爱什么,你送他什么东西,朋友自然会和你发生好感。遇见你有用他的时候,他一定帮助你,那就是换过来好感。江湖人对于和有力量的人使用投其所好、换出感情来的方法,调(diào)侃儿叫贴身靠儿。卖戏法儿的人,有本领的骗人钱多了,恐其日久,被骗人醒悟了,和他们捣麻烦,就用贴身靠儿的法子,慢慢将被骗人的恶感化去,渐成好感。因为有了好感情,面貌相关,无论如何吃亏,亦不好翻脸,挤来挤去,能够彼此用感情暗送心意,谁都放心,绝不能闹出破裂的事儿。那要调侃,就算平点儿,点儿平了,亦就无事。不只是江湖人,有贴靠的本领,能得着实惠,哪行的领袖对于他用的人稍有可取,就应当使贴身靠儿把他的心拢住,他才能给你真出力,叫你获益。若竟以金钱办法,永不与人换感情啊,有多大的财力亦得失败呀。

江湖人对于他们的事业,从老前辈就有深刻的研究。他们的好处就是想出生财的主意,能在社会里走得通,绝不是闭户造车。我对于江湖人的长处是要说的,是要赞美的,总要将他们的长处宣示于社会;对于他们的短处,并不是攻击,是要他们明白,骗人即是骗己!害人如害己!我国社会里的人士十有八九喜新厌旧,在民初那几年,刚刚有做卖戏法儿的生意,谁亦看着新鲜,都要学两手儿,消遣解闷儿,卖戏法儿的生意很兴旺一阵。不只是学戏法儿的人们不知道他们是骗人的玩艺儿,就是各省市、各商埠码头的地方官吏亦不知道他们这行人是骗人的。卖戏法儿的走在哪里,都能撂地(有块地方就做生意挣钱)做生意,因为“穴(xué)眼儿宽”,他们放心大胆地骗人,今天在甲地,明天挪到乙地。中国地皮广大,那就骗遍了啊!什么叫穴眼儿宽哪?江湖人对于各处都能去,去得地方多,调侃儿叫穴眼儿宽。不料他们骗得日久,有被骗人和他们打了官司,无形之中把内幕叫官家知道了,才有几个地方取缔卖戏法儿的。到了如今,十个地方倒有九处“卯了地”(江湖人管被官家取缔了,驱逐了,调侃儿叫卯了地),他们这行儿已然到了可怜的时期了!那各处的地主(玩艺儿场租赁江湖人撂场子的地主)亦知道他们是骗子,有地有东西,宁可闲着,亦不供给卖戏法儿的了!

后来各省的象窑儿亦不要他们卖戏法儿的。什么叫象窑儿呢?早年江湖人在各省跑腿儿,都不愿住普通的客店,专住“生意下处”。那下处是江湖人开的,与普通栈房一样,不过不住外人,专住江湖人。这种房,调(diào)侃儿就叫“象窑儿”。江湖人住在象窑儿内,如若遇见能挣钱的主顾,在外边说话不便,就跨到象窑儿内,设法多挣钱。他们住象窑儿,有种种的便利,都不肯往别的地方去住。象窑儿的主人对于外界人不大欢迎,对于江湖人是一概欢迎。可卖戏法儿的是个非常大的生意,时常在象窑儿内“挖点”(江湖人对于卖戏法儿的在密室中骗取学戏法儿的人的财物,调侃儿叫挖点),被骗人若有“醒了攒(cuán)儿”(明白过来调侃儿叫醒了攒儿)的,找卖戏法儿的退钱,卖戏法儿的走了,就和开店的打闹。因为住卖戏法儿的有种种的坏处,是象窑儿都不留卖戏法儿的。卖戏法儿的虽有骗钱的能力,但是将钱骗到手内,他们亦是“顶瓜”(江湖人对于有事可怕调侃儿叫顶瓜),恐其被骗人觉悟了,找他们“朝翅子”(江湖人对于打官司调侃儿叫朝翅子),他们骗了钱就跑。有时倒霉,在外埠有和被骗人误遇上的,被人抓住,打了官司,受了法律制裁!孙宝善、唐维一都是骗了钱跑了之后,忧郁死的,可见骗人的亏心,免不了精神上的痛苦。江湖中卖戏法儿的何不速醒!





老云里飞使“钻天儿” 艺名庆有轩


江湖艺人,最难的就是创“万儿”(江湖人管姓名调侃儿叫万儿)。要把万儿创开了,就和买卖家把牌匾做出来一样,“响了万儿”就能“活穴(xué)”(江湖人管姓名创出去,人人都能知道了,调侃儿叫响了万儿。管能发达了,调侃儿叫活穴)。现在北平这个地方,若提起“云里飞”来,几乎无人不知。他怎么把万儿创出去的,说起来情形亦很复杂。现在的云里飞已有三个,老云里飞、小云里飞、小小云里飞,祖孙三世全都飞了。先由老云里飞说起。他是北平“方字旁人”(北平人管前清的旗人叫方字旁人),久居西城,老姓白,自幼入松竹成科班学戏。十六岁出科,不过二十岁登台,先给武行打下手,后学开口跳。因为戏行里没饭,在光绪年间,拜使“钻天儿”的恒永通为师(江湖人管说《西游记》的调侃儿叫钻天儿。这句侃儿不过是以孙悟空一个筋斗十万八千里,能闹天宫,孙悟空是书胆,以他来说叫钻天儿)。据我们评书界人说,说西游的与说评书的是两派。评书艺人是评讲,没弦子没鼓;说西游的有渔鼓简板,随说随唱,他们是唱道情的。他们这派与评书界合并年代不远。说西游的最早是潘青山首创的,他的徒弟姓安,学猴儿形容得最好,听玩艺儿的都叫他“猴儿安”。由猴儿安这辈儿与评书界合并,他们另立门户,有门长,定规了四个字:永、有、道、义。他们收徒弟,就按着这四个字起名,“永”字的是恒永通最有名。在北平四九城说《西游记》,在哪个馆子亦能叫满堂座儿。他的台风好,口白清,敲打渔鼓简板,唱道情,各样的赞儿,能随便使辙口。双手十指一掐,撮腮帮眨眼,学猴儿;两只手往腮上一贴,当作猪耳朵,摇着头,嘴里嘟嘟囔囔,学猪八戒,如同真的一般。听书的人们,谁看着亦得笑,都往各书馆追着去听。他有叫座儿的魔力,各书馆争着约请。恒永通是个响了万儿(成了名)的艺人。

老云里飞因为梨园行没饭,有意说西游,他就投在恒永通门下,拜为艺师。恒永通收了他这徒弟,赐名叫庆有轩,与李有源是师兄弟。李之艺术平庸,未能享名。庆有轩说了些西游,他“朵上不清头”(江湖人对于不识字的人调[diào]侃儿叫朵上不清头),活儿的“万字短”(江湖人管说的书不长调侃儿叫万字短),上馆子做艺,只是个二路角色,亦不得志。他又弃了这行,又唱《戏迷传》。他在松竹成科班的时候,武丑有个“草上飞”,他亦学过武丑,不说西游,在各庙会、各大街路旁,白土子画圈儿,用白沙子洒字,就写“平地茶园,特邀名角,云里飞,雨来散”。他带着两个儿子,大的是小云里飞,二的是已故的白宝亭,爷儿三个唱《杨香武三盗九龙杯》、《乡下妈妈看亲家》,父子三人各扮一个主儿,当场抓哏,春里串戏,改样的相声。老云里飞去乡下妈,小云里飞的傻小子,爷儿两个形容丑态,发托卖像,抖出包袱儿,谁看了亦得“咧瓢儿”(即是笑了)!在那个时候,杂技中的玩艺儿不多,一段《探亲家》,人们亦听得下去,临完了还能“抛杵”(江湖人对于听玩艺儿的人花钱调侃儿叫抛杵)。那会儿人心厚,社会的经济状况好,低级的人们还能有钱,艺人挣钱,还是真容易。若搁现在来段《探亲家》哪,戏台上的坤角如何,都没人爱听,何况他们哪。在老云里飞串戏时,他们爷儿几个还没有洋烟盒的盔头哪,只凭人唱,亦不“鞭轰子”,亦不“升点”(江湖人管打鼓调[diào]侃儿叫鞭轰子,管敲打各样的乐器调侃儿叫升点),就能圆粘(nián)子(招徕观众)。小云里飞亦入过科班。老的说书,爷儿三个唱戏,都算行家,懂得身段、腔调、板眼。若是说相声,他们可没受过“夹磨(jiá mo)”。什么叫夹磨呢?江湖人不论是干哪一行,以得着师父的真传授为荣。凡是得过好传授的江湖人,别人羡慕他的本领好,调侃儿说人家受过夹磨。不论哪个江湖人,你要说他受过夹磨,他亦爱听。如若江湖人没得过师父真传,调侃儿说没受过夹磨。可是真没得过好传授的,你要说他没受过夹磨,他亦不爱听!云里飞的相声是乱七八糟大杂烩,一段段的正经活是没有的,不会说那些个。使春口(说相声),他们没有门户,调侃儿叫“海青”。相声行里的单口活、对口活,全都没有,所会的玩艺儿,全得唱几句,什么《饽饽阵》、《五百出戏名》、《杨香武盗九龙杯》。一个人唱,一个人在旁抓哏,用手捂着耳朵,吆喝:“耗子馅儿的包子,一个子儿六个!糖火烧!油酥火烧!”江湖人调侃儿说,他们净使“碎包袱儿”(三言两语逗人一笑,临时现抓哏叫碎包袱儿)。他们这档子玩艺儿,专能拢低级的人们,有资格的人们绝不肯听。是玩艺儿都能做堂会,他们这玩艺儿太不文明,难登大雅之堂,只能撂地(明场演出)做生意。

自从天桥发达了,小云里飞就在天桥独占个场子,撂长地,永不赶东西两庙。老云里飞天天扛着渔鼓,往各处说西游。小云里飞的兄弟白宝亭,另投门户,拜焦德海为师,学对口相声。他在天桥公平市场与尹麻子、刘德治等撂场子,人又年轻,口齿清楚,有气力,很做了几年好生意,相声行里的人都说他有起色。果然,到了天津上杂耍(各曲种的综合演出)馆子,登台就红了。只是年轻,不知保养身体,有了钱胡花,未几,就一命呜呼了!小云里飞剩了哥儿一个,对他父亲颇能尽孝。他不叫老云里飞说西游,日中两餐之外还有零花钱。吃饱了,闲遛弯儿,真是快活极了。

年前在天桥见老云里飞,见他行动不便,口齿语言不清,好像有病的样子。我问他怎么了,据他说是得了半身不遂。我问他:“还能做艺否?”他说:“已然做不了啦。”我问:“你们说《西游记》的,现在还有没有?”他说:“早年北平这个地方没有说西游的。自从潘青山创演《西游记》,旧社会的人士就很爱听。到了潘青山的徒弟猴儿安,说《西游记》‘响了万儿’(江湖人对于艺人能享大名,轰动了社会,调[diào]侃儿叫响了万儿),才加入评书界。以前的评书研究社,共分十个门户,我们亦算一门。我们这个门中传流的支派,是永、有、道、义四个字儿。我师父叫恒永通,是永字儿的。我们师兄弟两个,我叫庆有轩,他叫李有源。我在中年曾改过名,叫白有云。未几,又恢复了旧名。李有源收个徒弟,叫奎道顺,是道字儿的,他曾响过万儿,北平市各书馆他都做过艺,其艺术不弱于我师父。他收个徒弟叫邢义明,说得亦不错,可惜他‘磨了海草儿’(江湖人对于吸食鸦片的调侃儿叫磨海草儿)。他的万儿将有‘裂纹’(江湖人对于将要享名的调侃儿叫裂纹),就回去了。后来又‘弄上插末(chā mòr)汉儿’(江湖人对于扎吗啡的调侃儿叫插末汉儿),各书馆的‘粘箔(nián bo)’(江湖人对于书馆主人调侃儿叫粘箔)都不约他,只落到‘土在梁子上’(江湖人管死了调侃儿叫土了,死在街上调侃儿叫土在梁子上)。我说这行儿是要断庄了。”我问:“你收过徒弟没有呢?”他说:“我收了个徒弟叫田道兴。他入门之后,亦做过几处铺子,没有得利,他又离了这行,干别的去了。”我问:“现在北平的各书馆有没有说西游的?”他说:“好几年没有说的了。”我问:“你们使钻天儿的这门,还收徒不收呢?”他说:“我是不收了,我们这行就算完啦。当初我们师祖只留下永、有、道、义四个字儿,恰巧我们就传了四辈儿,断了行啦。看起来事由前定,亦不可不信。”我问:“有人说你们的师祖猴儿安,名叫安天会,是不是呢?”他说:“孙悟空大闹天宫,那是安天会,我们师祖哪儿能叫那个名字。他在我们永、有、道、义的上边,是太字儿的,名叫安太和。”我问:“你们说西游的,怎么与评书不同哪?”他说:“说评书是评讲,没有弦子鼓儿,亦不唱,说完了就要钱。我们说西游的,有渔鼓,每逢有词赞儿行路歌儿,都手拍渔鼓,按着辙口儿唱,这就与评书不同。我们说完了书,不是指书挣钱,还有‘憨子’哪,更与评书不同了。”我问:“什么叫憨子哪?”他说:“我们说西游的说了一段,到了要钱的时候,是听玩艺儿的,都给他们一块沉香佛手饼。那种药糖,人吃了有益,调(diào)侃儿管那糖叫憨子。现在书是要断喽,熬那药糖之法亦要失传了!”





滑稽大鼓是张云舫攥弄(zuàn nong)(创作)的 创演“响万儿”是老倭瓜唱得早


“柳(liǔ)海(hāi)轰儿的”(江湖人管唱大鼓的行当调侃儿叫柳海轰儿的),有木板、铁板之别。唱西河调大鼓、乐(lào)亭调大鼓、山东犁铧调大鼓都是使铁板,唱梅花调大鼓、京调大鼓、怯口大鼓、小口大鼓都是使木板。现今我国各省的人士,好听玩艺儿的,都爱听木板大鼓。各省市、各商埠、各大杂耍(各曲种的综合演出)馆子的艺人,亦是唱木板大鼓的占有最大的优势。凡是杂耍馆子,不论有多少档玩艺儿,上多少场都用木板大鼓“攒(cuán)底”(江湖人对于唱大轴儿调侃儿叫攒底)。现在能够攒底的角色有刘宝全、白云鹏、金万昌、张筱轩、小彩舞、白凤鸣、谭凤元等,都是木板大鼓的名角儿。铁板大鼓虽然有唱的,在杂耍馆子内只能唱前场,垫垫场子,不能攒底,亦不能叫座儿。唱木板的玩艺儿,各富家有喜庆堂会,亦是约他们;约铁板的玩艺儿,最少最少。木板大鼓最高尚,唱角儿只上馆子,不往露天的场子摆明地。各省市商埠码头、露天市场(北平天桥,天津三不管,保定马号、张垣大桥、北市场,烟台盂兰会,济南趵突泉,开封相国寺、五里沟,安东的六七道沟,大连西岗子,营口洼坑甸)都是唱铁板大鼓的,艺术平庸,唱词甚劣,极不可听。专有一些低级游人,知识幼稚,欢迎听那玩艺儿,知识界人绝不去听啊。

木板中的各调大鼓,都兴得很早。滑稽大鼓兴的年代不远,在从前没有这种玩艺儿。大鼓里不易加包袱儿。江湖人常说“万象归春”,不论是什么玩艺儿,都得逗笑儿,叫人听乐了,才能有人欢迎。江湖人虽明此理,人才缺乏,又都牢守旧规,没有创造的本领。大鼓行里几百年亦没有人提倡唱滑稽玩艺儿。北平前门外板章路,住有三旗营,安清名人张云舫。他不是江湖人,自幼嗜好大鼓,精于歌唱,限于“夯头儿”(江湖人管嗓子调侃儿叫夯头儿),从未登台,曾费数载光阴研究滑稽大鼓。所编各段曲词,辙口好,词儿妙,唱主若能形容喜怒哀乐,有发托卖像,极容易引人发笑。有玉器行人崔子明,亦北平三旗营,同张同乡,亦同好歌唱。张不吝珠玉,将其所编滑稽鼓词尽授于崔。他排演成熟,就在北平献艺,报上写“老倭瓜滑稽大鼓”。他上台来,一“鞭轰子”(江湖人管打鼓调[diào]侃儿叫鞭轰子)就有包袱儿。他一段鼓板,敲打时丑态百出,“不抹盘儿”(江湖人管不害臊调侃儿叫不抹盘儿。艺人在台上逢场作戏,不能害臊,前台不要脸,后台要脸,那才唱得红哪。老倭瓜应当如是),能逗听玩艺儿的人们一笑。及至他唱的时候,七成仗着鼓词,三成仗着形容,油腔滑调,百样怪状,使人解颐,笑不可止,大受欢迎。他独创一派,挑帘儿就红了!

老倭瓜将有“裂纹”(江湖人对于将要享名的调侃儿叫裂纹)的时候,有人“携”过他一次。阅者一定要问什么叫携哪?我将这种江湖封建的暗势力先行说明。早年江湖中的人五花八门,干什么得入什么门户,拜位老师,将江湖规律都学会,然后才能吃江湖饭哪。譬如算卦的术士要想摆卦摊,按着江湖的金、皮、彩、挂(金指算卦相面,皮指卖药,彩指变戏法儿,挂指打把式卖艺)四大门说,那算是金门的生意,得先拜个金门中的江湖人为师,学会了江湖规矩。摊子怎么摆,见了同行说什么,有江湖人盘道应当如何应付……把这些事都学会了,才能摆摊子,卖卜挣钱,遇事不怕,能有应付之法。譬如要净会算卦,不懂江湖事,摆出摊子,江湖人一看,就知道这是外行,立刻上前盘道,若无法应付,江湖人就不准他摆卦摊,还能把算卦的家伙拿走,再不准吃这碗饭!如若想吃这碗饭亦成,得找个江湖老师认门户,有了门户,吃遍天下。老倭瓜唱大鼓,就是票友下海,他没有江湖门户,江湖老合把他携了,不叫他吃江湖饭。他无法可使,才拜了位江湖前辈史振林为师。在北平地方,史振林的门户最盛,鼓界名人白云鹏亦是他的徒弟。老倭瓜这个门户进得妙极了,他与白系师兄弟了,白云鹏就带他往津、沪、济、汉等地献艺。虽没唱过大轴,倒二、三、四场常说。我国社会的风气喜新厌旧,最好听奇怪的玩艺儿,他这滑稽大鼓算是开创先河了。老倭瓜三个字,誉满中华了。他成了大名,是得了三个人的好处:第一是得张云舫的鼓词,“攥弄(zuàn nong)得撮啃(zuō kèn)”(管编玩艺儿调侃儿叫攥弄,编得好调侃儿叫撮啃,编得不好调侃儿叫念撮);第二是他拜了老师,有了江湖的门户,同业人不排挤;第三是有白云鹏的提携,他才成名。

我向鼓界的内行人问过张云舫的鼓词怎样,他们说那鼓词不只攥弄(zuàn nong)得撮啃(zuō kèn),那玩艺儿的“皮儿最薄”,唱出去开门见山。我问:“什么叫做皮儿薄?”他们说:“唱的鼓词听的人们容易懂,就叫‘皮儿薄’。如《乌龙院活捉三郎》、《闹江州》、《华容道挡曹》那些段子,文浅还不算,玩艺儿里的人物李逵、宋江,谁不知道?《哭祖庙》那段玩艺儿,亦是《三国演义》上的故事。一般人看《三国》,看不到哭祖庙就腻得不爱看了,说《三国志》的亦说不到那里。他们的行规是不等到走麦城就不说了。如若说的关公死了,立刻就没有人听,绝说不到哭祖庙。唱《三国》的是零段多,整本大套的少。可是有唱整本大套《三国》的,亦唱不到哭祖庙。《哭祖庙》那段鼓词,不论是谁唱,唱得天好,亦是费力不讨好,听主懂得这段玩艺儿的人太少。”江湖人管这不容易懂的玩艺儿调(diào)侃儿叫“皮儿厚”。凡是江湖艺人,学什么玩艺儿,都怕皮儿厚。有皮儿厚的玩艺儿,多好亦不敢学。学玩艺儿的时候,最欢迎的,就是皮儿薄的玩艺儿!

这些年鼓界中唱玩艺儿的唱红了,都是唱皮儿薄的玩艺儿成的名。唱皮儿厚的玩艺儿,成了名的只有一个白云鹏。他唱的大鼓段子,有《黛玉悲秋》、《宝玉探病》、《探晴雯》、《黛玉归天》、《宝玉娶亲》、《哭黛玉》、《太虚幻境》,这都是《红楼梦》上露泪缘的玩艺儿。刘宝全都不唱这些段子,一是这些段子皮儿厚;二是这些段子婉转悱恻,哀艳感人,非他所宜。他是以亢爽激昂、悲壮凄凉的段子为正工,像《华容道》、《草船借箭》、《长坂坡》、《宁武关》、《李逵夺鱼》、《活捉三郎》、《截江夺斗》、《徐母骂曹》等段,唱得最佳。刘、白二人各有所长,一是文大鼓,一是武大鼓。白唱的段子皮儿厚,刘唱的段子皮儿薄。皮儿薄的玩艺儿能吸引听主,性质普遍,任何人都能听,叫座儿较易;皮儿厚的玩艺儿,只能叫懂得曲中歌词的人,知识分子爱听,不懂的不听,较比皮儿薄的叫座儿难些,性质不普遍。我问过他们:“什么叫‘开门见山’?”他们说:“譬如唱角儿在台上唱玩艺儿,一张嘴就唱:‘内丹图列在四大奇书内,也无非,劝人行善莫胡为。西游记无非是丘祖笔墨,把心机使碎。愿迷人,跳出苦海,莫坠轮回。’这几句曲头儿,听玩艺儿的人懂得的少,读书识字的人能懂,还是皮儿厚。若唱‘唐三藏奉旨去取经,受尽了百般磨难,不把心回。这一日借宿高老庄’这几句,听主就能听出来,是高老庄收猪八戒,一听就懂,调(diào)侃儿叫‘开门见山’。”

老倭瓜唱的滑稽大鼓,是张云舫的玩艺儿皮儿薄,能够开门见山,他唱着最容易受台底下听主欢迎。这是老倭瓜成名的最重大原因。平津一带鼓界中门户,在梅、清、胡、赵四大门中,以史振林的门人弟子最多。老倭瓜进了他的门户,平津的同业,本门人多,不惟不受排挤,并且还有了关照。白云鹏在平、津、济、沪等地,不论在哪个馆唱,亦是压大轴儿,他有叫座儿的实力,要提拔几个唱前场的角儿,前后台都能认可。有这三个原因,老倭瓜怎不成名?他所唱的玩艺儿,我听过的有《海三姐逛市场》、《劝五迷》、《劝国民》、《吕蒙正教学》、《蓝桥会》、《拴娃娃》等等段子,都是张嘴一唱,三五句台下“询家”(江湖人管听玩艺儿的调侃儿叫询家)就能懂得唱的是什么玩艺儿,段段有“开门见山”的好处。滑稽大鼓不只老倭瓜一人,不论是谁学会了,亦能挣钱。平津等地的杂耍(各曲种的综合演出)馆子都得有这一场玩艺儿。老倭瓜真老了,他有六十多岁,恐不能再往各处奔走,唱一回少一回了。现在北平久演的角儿有架冬瓜,他的气力足,岁数好,唱得火炽,艺术不弱于老倭瓜。可惜架冬瓜没赶上好时候,没挣着大钱!在外埠有个唱滑稽大鼓的山药蛋,据说比老倭瓜唱得好,活使得漂亮,逗笑的包袱儿亦抓得脆,很受各省市的询家欢迎。不过山药蛋没到北平来唱,好坏不知,人云亦云罢。最近他给白云鹏磕头,认了师父。名人收徒,许错不了。唱滑稽大鼓的除他三人之外,还有个大南瓜,因为他常有“粘啃(nián kèn)”(江湖人对于人染病调侃儿叫粘啃),总未演唱。老倭瓜是老得不成了;气力足,唱得火炽,只有架冬瓜、山药蛋了。不过我听过他们的玩艺儿,会得多的有二十多段,会得少了有十几段。其中好段子全算上才十几个,唱不到一个月就要翻头,重新另唱,太没味儿。他们又都不会攥弄(zuàn nong)活,怎么学来的套子活就怎么唱,会多少唱多少,不知道进步,往深刻研究。如果多学几段玩艺儿,有四五十天不翻头,就可听了。

江湖中的老合们常说:“有艺不愁挣钱,就怕到了挣钱的时候没货。”据鼓界内行人说,滑稽大鼓是张云舫首创的,他编的玩艺儿,唱滑稽大鼓的角儿们没有学全。还有些好玩艺儿张云舫没往外传,有《烟卷成家》,有几段《胭脂》,有几段《战宛城》。据说这些段子比他传出来的玩艺儿格外精彩,词句香艳中带滑稽,自来的包袱儿,谁要学会了唱出去,准能火炽,准能有人欢迎。不过江湖人常说:“能赠一锭金,不给一句春;能送十吊钱,不把艺来传。”可是话又说回来了,谁亦不愿白劳神。我只希望会唱的老合们,攒(cuán)子(心眼儿)一活动,就能把张云舫的玩艺儿学过来,倘再过几年,恐怕没处学去了!





小秫秸棍灌铅是“托门” 摇出摇不出是为“推送点”


笔者幼年的时候,住家在东北城,几年不准出趟前门。有一次随亲戚到城外有事,回去晚了,天在掌灯以后。走在一条大街,见有一家关闭的铺子,门前有个大纸灯笼,上书“灯下术”。往里一看,有几十人在里边挤着瞧热闹。我们亦挤进去,在人群中一望,见屋中有灯一盏,坐着一人,面貌可怕,手里拿着一把镊子,一个小竹筒。筒内有三根秫秸棍,棍上有裹红纸圈儿的,有不裹的。在墙上有几个纸袋儿,上边有个方格,横写×××号,竖着空有省、县、姓名、年岁。那个拿竹筒的人说:“我们这灯下术原叫先天卦,可是伏羲氏画八卦有的阴阳,八卦有先天后天,我们这是先天卦。不论是哪位来算,能知道你姓什么,叫什么,多大的年岁,哪儿的人;这一辈应做什么事,士农工商在哪行;是人中的领袖,是帮人当伙计;终身衣禄、食禄怎样;应当沾谁的光,被谁的害;祖业有没有,弟兄几位;什么脾气禀性;由幼年直到老来,应当活多大年岁。由先天注定,全都算得出来。可是多了不算,每天只算三卦,算得对了,要钱;算得不对,分文不取,毫厘不要。哪位要问,我们先看看有你的卦没有。怎么个问法?哪位往我前边一站,我摇竹筒,带红纸圈儿的棍摇出来,就是有你的卦;如若摇出不带红纸圈儿的棍,那就没你的卦了。”他说着,就有一个人奔到他面前,说:“先生!你看看有我的卦没有?”这位先生就摇起小竹筒,里边的小秫秸棍乱晃。晃来晃去,由筒内晃出一根棍来,上边没有红纸圈儿。他向问卜的人说:“没有你的卦!”这个人听说没有卦,只好不算。接连不断有人来算,他的筒儿无论怎么摇晃,亦是不带红纸圈儿的棍出来,带红纸圈儿的棍总不出来。这些人觉着奇怪,向他问道:“先生,怎么会没我的卦哪?”他说:“别的算卦的,有人算他就算,算一卦挣一卦钱,他怎么不算?我这卦要那么算,就不灵了。众人是圣人,我这先天卦不是现算,早把卦算得了,在这纸袋内装着哪,是谁的卦得等谁,本人不来,不能给别人算。我每天只算三卦,亦许不开张,可就是没算过四卦。哪位要算,自己说话。”他这样说,愈显着有点神怪。我看了两三个钟头,才见他那筒内带红纸圈儿的棍摇了出来,算是有个人的卦了。

起初我对于他的小竹筒虽然生疑,可猜不透是怎么回事。我总疑惑他那小竹筒有毛病,要不一样东西,怎么有摇得出来的,有摇不出来的?如今我才明白他那小竹筒是有“托门”的。什么是托门呢?江湖人对于使用的家伙上有令人难测的机关,能闹鬼儿,叫人看不出破绽,调(diào)侃儿叫托门。那么小竹筒上有什么托门哪?我先把他这个托门说明,然后再说那灯下术。我向江湖中的人们探讨过多少次,他们都不肯将个中黑幕说给外人。我费了许多的联络手段,才把这小竹筒托门讨了来。原来他那竹筒没有毛病,有鬼儿都在那秫秸棍上哪。他那三根秫秸棍都是灌了铅的,铅灌在一头儿,做上个暗记,用手一拿秫秸棍,就知道哪头儿有铅。往竹筒内放时,将有铅的三个头儿都冲下,摇晃一天、一月、一年,亦摇不出筒来,有铅坠着,休想摇出来。如若将三根秫秸棍有铅的头儿全都冲上,放在筒内,略微一摇,不用费力,那三根秫秸棍都能摇出筒来。如若将两个有铅的头儿冲下,一个有铅的头儿冲上,放在筒内摇吧,不大工夫就能摇出一根来。总而言之,他们这种办法,是想摇出哪根,哪根就出来;不愿意哪根出来,哪根就不出来。他们将这三根有托门的秫秸棍做得了,就为的是“把点”、“推点”。什么叫把点哪?江湖人管看谁调(diào)侃儿叫把合;管江湖人调侃儿叫老合;管非江湖人、不懂江湖事的人统在一处称呼,调侃儿叫点儿。看看是点不是点,就是把点。如若看这人能够由他身上挣出钱来,调侃儿叫“正点”;如若看这人不是花钱的,调侃儿叫“不是正点”。

那么他们怎么能看出哪种人是能挣钱的正点,哪种人不是正点哪?江湖中有一种神秘的传授,不论见了什么人,只一对脸儿,就能知道人是忠厚或狡猾。他们这种瞧人行事的本领值得人佩服。他们管商界人调侃儿叫“贸易点”,管军界人调侃儿叫“冷点”,管政界人调侃儿叫“翅子点”,管做大官的调侃儿叫“海(hāi)翅子点”,管军人调侃儿叫“海(hāi)冷”,管小军官调侃儿叫“冷把子”,管农人调侃儿叫“科郎(kē lang)点”。他们对于社会里的人,士、农、工、商、军、医、学、报三百六十行都有一种侃儿。不管是哪里人,一看就能知道。他们看这人忠厚不狡猾,能够老实花钱,那就算正点。他们看这人长得聪明,面带狡猾,口齿伶俐,善于言谈,虽能花几个钱,不过难挣,得设法叫这种人心服口服,钱才能挣到手,这就不是正点。甚至于还有费许多的唇舌,挣不下这种人的钱来,生意还有被他扰了的时候。他们做“灯下术”的不是公道买卖,纯粹骗人钱财。遇见正点好极了,正点能老老实实任其敲诈;遇见不是正点,可没准儿挣得出钱挣不出钱来。他们是欢迎正点,不欢迎不正的点。可是正点来了,好办;不正的点来了,不愿意挣他的钱,不愿和他搞麻烦,又有什么拒绝的方法哪?

他们对于狡猾人,因为不能挣钱,有一种“推法”。凡是他们看出是正点的人,要问卦,他将三根小秫秸棍,没有红纸圈儿的,有铅的那头儿冲下;有红纸圈儿的,有铅的那头儿冲上,都装在小竹筒内。慢慢一摇竹筒,有红纸圈儿的秫秸棍不费劲就能摇出来。摇出有红纸圈儿的来,算有这人一卦,他好挣钱。如若有人去问卦,他们用把点的本领看出来问卦的不是正点,算对了,亦不能给钱;算不对,就给他们扰了。他们不愿和这种人搞乱,若凭嘴一说,没有这人的卦恐怕不成,仍然免不了搞麻烦。他们使小竹筒往外推,右手将三根秫秸棍拔出,左手拿着竹筒,底儿冲上,口儿冲下,先晃晃,然后再装三根秫秸棍,将那有红纸圈儿的棍,有铅的那头儿冲下,装在竹筒内;没有红纸圈儿的两根秫秸棍,有铅的那头儿冲上,装在竹筒内,慢慢摇吧。没红纸圈儿的,不费劲就能摇出一根来;有红纸圈儿的,铅在下边坠着,无论如何亦摇不出来。他拿着那根没纸圈儿的秫秸棍,就可以向那人说:“你这钱省下了,我这里没有你的卦!”这种好捣乱的人,遇见这种办法亦不好捣乱,没有他的卦,只好走开。其实做灯下术的,他们研究出小竹筒摇秫秸棍的托门,就是为扰他们的人预备的。如若遇这种人,就说没他们的卦了事。





灯下术叫“袋子金” 是点不是点 全凭开把簧


他们的小竹筒,托门、推法,我都说明了,再说我那次所见的灯下术怎么神怪。那天晚上,我看他给人算了一卦。那问卦的人穿着打扮好像是个买卖人,听口音是深州的。他要占卦,算“灯下术”的就先摇那小竹筒,结果没费劲就把带红纸圈儿的秫秸棍给摇了出来。他向问卦人说:“有你的卦了!我这卦可不是现算,是早给你算得了,在我这只口袋内装着哪。少时我打开口袋,取出卦来,那卦上就有你的姓名、年岁、哪里人氏、一辈子的事儿,以及你的脾气秉性、衣禄食禄,应当在哪行做事,你的父母全不全,妻宫大小,早娶晚娶,克妻不克,立子早晚,你的少、中、老三部大运,哪步好,哪步坏,吃什么人亏,受什么人害,哪年不好,哪年好。全都对了,你再给钱。”问卦的人点了点头。他又说:“我们是先问君子,后小人,你的姓氏、年岁、哪里人,先说出来,我们写好了,放在一旁,然后打开我那口袋,取出卦来看,果然写得都一样,咱们再往下算。如若写得和你说得不一样,那就是不对,你不用给钱,两吹台。”这个人说:“这样办很好,心明眼亮。”算卦的拿着纸笔,向问卦的人说:“你姓什么,叫什么,多大年岁,哪里人?你都说出来,我先往这张纸上写,然后咱们再对对看。”这个人说:“我姓张,叫有才,深州人,今年二十八岁,我是手艺行当。”他用笔在纸上写:“张有才,年二十八岁,深州人。”写完了,向看热闹的人说:“诸位看见没有?他的姓名在这儿写好了,回头我把他的卦取出来,那上边写的姓名和他这个一样,就算我这卦灵。”

他说完,用手拿下来一个纸口袋,有五六寸长,四寸来宽,纸亦硬,封口亦都是糊好了的,袋上有几个小圆圈,是留着填写格式。他向问卦人说:“你这卦在这里边哪,我写上号头儿,咱就取出来看。”说完了,拿起笔来,往口袋上写七百七十三号。他说:“到了这卦,我一共算了七百七十三个人了。”用手撕口袋上边的封口儿,“哧”的一声,将口儿撕开,那口儿不是现封的,不定封了多少天啦,封口儿上的浆糊都干透了。他拿把镊子,往口儿内去夹,夹出一张毛头纸来,折有好几层。他打开,只叫问卦的人看头一层儿,只见上边写着:“昨夜三更天,与你把卦占。若问吉凶事,先掏卦礼钱!”再往下看,写着:“张有才,二十八岁,深州人,手艺行当出身。”再看就没了,别的字都在二、三、四、五层上哪。他向张有才问道:“你看明白没有?”张有才面上现出惊奇的样子,说:“我看明白了。”他说:“你掏卦礼钱吧!”张有才说:“多少钱哪?”他用手指那卦上写的几个极小的字,叫张有才看。张有才仔细一看,见上写:“此命卦礼银二两整。”张有才说:“先生,我是个耍手艺的人,每月的工钱才挣二两多,算一卦就得二两银子,我实在花不起。你少要几个吧!”他很不愿意。张有才亦不知费了多少好话,给了一两银子,才叫他看那卦。只见那全张纸上写的是:“张有才,二十八岁,深州人,手艺行当出身。祖业凋零,自创自立,出外早,做事早,劳碌早,三早之命。宜入工界,手艺相宜。为人耿直不曲,不奸巧,凭天吃饭,量力求财,勤俭耐劳,做事忠实。前半辈,虚名假利,财来财去,劳而无功,同人不和,多成多败,事不如意,财难趁心。父母双全不能妨去一位,鳏居不能有妻,子宫二三送终有一。目下赋闲不能有事。后半辈,火烧竹竿节节爆,脚蹬楼梯步步高,事顺财旺,外方立业,独掌大权,内添人口,人财两旺,名利双收,得庚申辛酉方,寅卯贵人之力,受人提拔,可以发达。唯有四十八岁,身弱有恙,大病一场,前有水危,至此有数月之灾。正南方,木字旁人,能够除灾。晚年有长久不败之业,平稳之财,福禄由勤俭得来,受尽折磨,苦去甜来,独立成家,外乡发达之命。在目下百日内谨防小人暗算,意外之灾。”张有才听他念了一遍,不住地点头,很是佩服。可是他还有不明白的地方,向算卦的先生问道:“我的父母全不全哪?”算卦的说:“我这里不是写着吗,你‘父母双全不能妨去一位’。那么你父母倒是双全不双全,你说!”张有才说:“我父亲死了,母亲还有。”算卦的说:“我这儿写着是你‘父母双全不能,妨去一位’!”张有才道:“先生高明,你真算出来了。”

我在旁边看着这种事,就明白了。他使的是“连环朵儿”。阅者若问什么是连环朵儿,我先把这种事揭穿了,然后再说全盘的灯下术。原来他们江湖人,管字调(diào)侃儿叫朵儿,管写字调侃儿叫戳朵儿,字写得好调侃儿叫朵儿戳得撮啃(zuō kèn),字写得不好调侃儿叫朵儿戳得念撮。如若写出十几个字来,明着是一句话,暗含着是好几句话,他们能将这句话分成三段儿上下连贯着使用,那要调侃儿就叫连环朵儿。我把他那连环用法解释一下。如若有问卦人说:“父母双全。”他那连环朵儿就分成两段:“父母双全,不能妨去一位。”如若有人说:“父母不全,死去了一位。”他那连环朵儿就分成三段,还有两个字一段的:“父母双全,不能(不能双全,将中间不能两个分开了,不能)!妨去一位。”这样说,这样念,那不明白连环朵儿的人都得佩服他,认为他未卜先知。其实他那些话句写得都是八面风儿,专蒙知识幼稚的人。如若有人说:“父母都死了。”他将那连环朵儿能分为两段,念完了底下再衬上一句,照样圆满:“父母双全不能,不能妨去一位。要妨,还是妨去两位。”这样他把“不能”两个字,往上连着“父母双全不能”,往下又连着“不能妨去一位”,底下没字儿了,他还饶上一句:“要妨,还是妨去两位!”(不能妨一个,还不妨两个吗?)这是说父母全不全的连环朵儿。那妻宫有无,就写“鳏居不能有妻”。向问卦人问:“你有媳妇没有?”问卦人如若说:“没有媳妇。”他就将这六个字的连环朵儿分为两段,上两字“鳏居”,下四个字“不能有妻”。念出来亦是说你这人是个光棍儿,不能有妻。如若问卦人告诉他们说:“有媳妇。”他们又将这六个字的连环朵儿改为上边四个字“鳏居不能”,下边改为两个字“有妻”,说你这人不是光棍儿,一定是有媳妇的。如若问卦人问:“我父母不全,先死哪一位?”他们又写五个字连环朵儿,“父在母先亡”。写完了问:“你父母哪位先死的?”问卦的人若说:“我父亲先死的。”他就指五个字念:“父在母先亡。你父亲在你母亲以前死的。”问卦的人若说:“我母亲先死的。”他又念:“父在,母先亡。你父亲在世哪,你母亲就先死了。”

江湖中卖卜的术士学会了连环朵儿,就往“六亲簧”上用,保管能搪得过“空(kòng)子”去。什么叫六亲簧,哪叫空子啊?据江湖人说,做金点(相面算卦的总称)的人们对于问卜人的妻财子禄如何,父母兄弟怎样,并不知道。可是问卜的人大多数都问他们:“先生,你看我弟兄几位?”如若说对了,就信服了,肯其花钱。江湖人以研究这种应付的方法,方研究出六亲簧(江湖人以人的父、母、兄、弟、妻、子为六亲,使用江湖的妙法能知道人的六亲现在如何,调[diào]侃儿管这种妙法叫六亲簧)来。用这六亲簧,是蒙空子(江湖人管不懂江湖事的人,以及他们能挣钱的人调侃儿叫空子),对于空子使用,准能搪得过眼去。他们的六亲簧亦不一样,各有使法,各有巧妙不同。像做灯下术使用的连环朵儿是最笨、最不漂亮的六亲簧。那种连环朵儿亦就在那个年头能用,能蒙得住人。若搁到现在,不用说蒙大人,就是小孩儿亦蒙不住了。可见早年的人心朴厚,比现在好拍呀!连环朵儿在从前还有用的,现今都不使用了。这种连环朵儿若是使,亦就是空子“倒要簧”的时候,搪塞了事用得上。空子若不倒要簧,亦使用不好。什么叫空子倒要簧哪?就是空子向算卜人问完父母,又问兄弟几位等事,那就算倒要簧。空子愈倒要簧,他们愈不怕,使上六亲簧,空子能够佩服了,他们才能挣钱哪。空子以倒要簧试试算卜的本领,说对了他们就佩服,当个花钱不花钱的目标。江湖术士是欢迎人倒要簧的。如若将倒要簧的人弄得佩服了,那钱就算挣准啦。

可是现在的术士,六亲簧使得巧妙,比以前漂亮多了,离开连环朵儿之外,有两种六亲簧,一种是不往纸上写,只听他们嘴往外说。使连环朵儿是纸上写完了,“父母双全不能妨去一位”,再问人家究竟父母双全不双全;他们再连环贯断法,八面风儿,那是笨极了的方法。他们口头上说的法子,是不用问,张嘴就说:“你这人父母都死了!”说出来就准对!据江湖人说,这叫“戳簧”。至于他们怎么能看出人父母都死了,戳簧使得怎么那么恰当,真是令人不可思议。我先把这段存起来,日后再说。

六亲簧还有一种是“滚册”。据江湖人说,那滚册的簧头能知道人父母双全不双全,是先妨父,还是先妨母,死了多少年,父的属相,什么命人;母的属相,什么命人,在哪一年死去的,都能知道!兄弟几位,有死的没有,死去几个,现下有几个,是否一母所生,都能知道!妻宫大小,早娶晚娶,妻的属相,克与不克,有妾无妾,是否鳏居,以及夫妇未完婚之先,男妨女,女妨男,都能知道!子女有无,是先生男,还是先生女,子几个,女几个,立得住立不住,有死的没有,是应有子女,是命独无子,都能知道!在六亲簧之外,还能知道此人在从前中过什么功名,做过什么事,见过什么危险。那滚册是个神秘已极的东西,已由上海某大书局印行了。不过外人买了去,看不透,敲不懂。非江湖人买了去,才能看得明,能够会用。可是江湖人使用这滚册,能够看得懂的亦是不多,那个东西据说很难学的。至于这滚册是怎么个滚法,我尚不知,日后知道了再为谈它。

原是说那袋子金的小口袋,怎么会有未卜先知之妙哪?我为此事探讨过几年。江湖人的黑幕哪肯对外人去谈,我问过多少人才问出来。据说那个小口袋内装的卦单,单上所有的字预先写出来的,只有“〇〇〇年〇〇岁〇〇县人”,〇是没写出来,留着那空儿,临时现往上添的。那临时现添的写法和变戏法儿一样,是种障眼法,局外人看不破的。他们做灯下术的,早把写快字障眼法练好了的,那是久练久熟,熟能生巧。他那口袋有毛病,里边的瓤儿如同抽屉似的,在他往口袋上写号头的时候,他是先往那〇〇〇上填写的,填写完了,又写的号头。他们那种快法,手底下利落法,都是人想不到的。那卦单在袋内折着,不论有多少层,亦是那〇〇〇的层在上面,他好往上填写。江湖人再巧妙,“空(kòng)子(不懂江湖内幕的人)心眼儿有三垛墙”,他们就不会不说我姓什么,叫什么,多大年岁?如若做灯下术的没料到这些,自己偷着写,不叫他看见,写在纸上,折起来,往卦摊上一放,叫他取出口袋内卦单来,和这张纸上写得对对,如若一样了,要多少钱给多少钱,那样就把他们难住了!人们真傻,他问什么,就先写给他,明着不告诉他姓张,他绝对算不出来。变戏法儿的,碗内不放条鱼,他绝变不出来。

理最明显,世人不察,要认为灯下术最神秘,实是自愚!受愚!





附录二


小绺(xiáo liu)(小偷)门·偷帽子、偷鞋、偷狗


贼偷帽子之前,从兜里掏出一根宽松紧带儿,立着套在自己脑袋上,然后伸右手从这位的右侧摘帽子,摘完之后立刻扣在自己脑袋上,还不跑,也直勾勾往里看热闹。



小绺就是小偷。都说“贼起飞智”,这话确实不假,偷东西的得有脑子,聪明,这才能把东西算计到手里。比如偷帽子,怎么偷呢?老年间最好的就是盛锡福的四季帽。这位戴着一顶新帽子,上天桥看热闹,看变戏法,直勾勾地瞧,让贼惦记上了。贼想偷帽子,他得站在这位的左侧,还不能并排站,得稍微往后错一点儿。他在偷之前呢,从兜里掏出一根宽松紧带儿,立着套在自己脑袋上,然后伸右手从这位的右侧摘帽子,摘完之后立刻扣在自己脑袋上,还不跑,也直勾勾往里看热闹。丢帽子这位往右一回头,没有;再往左一回头,旁边站着一个人,头上戴着一顶帽子,跟自己的一模一样。这位怎么看,这帽子怎么像自己的,可又不敢认,因为人家不见得买不起这帽子。可这位犯嘀咕啊,就老看旁边这人。结果贼先说话了:“怎么着?帽子丢了吧?”“是啊!一转儿脸的工夫就不知道让谁偷走了。”“嘿!你没法儿不丢!我都丢了六顶了。你看,我长记性了,我这儿不套着松紧带儿呢么?”这位一瞧,赶紧点头:“哎哟!谢谢您,下回我也钉带儿。”

偷鞋怎么偷呢?鞋穿在脚底下,按说不好偷。这位新买的内联三色礼服呢皮底布鞋,结果让贼缀上(盯上)了。这位得回家啊,进胡同,贼也跟着进胡同,他得瞅两边房子高矮差不多的地方才下手呢。贼紧走几步,来到这位身背后,伸手把这位戴着的帽子扔房上去了。这位一回头:“哎!你是谁呀?怎么扔我帽子?”“哎哟!对不起对不起,看您后影儿以为是我朋友×××呢,我错了,我错了。”“你错了不行啊,我帽子在房上呢。”“那我给您够。这房高点儿,您踩着我肩膀,要不我踩着您肩膀,一直身儿就够着了。”丢帽子这主儿心想:我丢了帽子,还得让你踩我肩膀,这不成。“你蹲下,我踩你肩膀才成呢。”贼手扶墙根儿蹲下了。这位刚要往上踩,贼站起来了:“不瞒您说,我就这一身衣裳,过两天我还得穿着它参加婚礼呢。您要穿着鞋踩我肩膀,脏了我还得洗去,忒麻烦了。”丢帽子这主儿心想:人家说的也是,我穿鞋踩上去确实不合适。“那你蹲下吧,我把鞋脱了就是了。”这位把鞋脱了,踩着贼的两个肩膀,贼准备往起直身儿。他真要直起身子来,这位就上房了。贼往起直到半道儿,这位上又上不去,下又下不来,正是趴在房檐这儿,这个劲儿难拿啊。贼一撤肩膀,把这位挂在房上了。然后贼把鞋拾到手中,转身就走,嘴里还不闲着:“回见吧您呐,我办点儿事儿去。”“哎!你别拿我鞋啊!”丢鞋这主儿想往前捡帽子,够不着;想下来找鞋,又下不去,只能在房上挂着。什么时候等着熟人从此路过,他才算得救。您看,偷鞋得先扔帽子,这就叫“贼起飞智”啊!

这位把鞋脱了,踩着贼的两个肩膀。贼往起直到半道儿,一撤肩膀,把这位挂在房上了。然后把鞋拾到手中,转身就走……



偷狗又是怎么回事儿呢?偷狗又叫“骑狗”,冬天才能偷狗。因为偷狗的目的是要狗肉、狗皮,冬天的皮子好,夏天的皮子不值钱。甭管多厉害的狗,这贼都敢偷,有主意。不能让狗叫唤,狗汪汪一叫,主人就听见了,没法偷。到冬天的时候,贼披着大皮袄,背对着狗,身子朝外,两腿自然叉开,在裆前下食。狗一看,必然得钻裆叼食。狗过来了,钻过裆去,一低头,要叼这食。就在这狗低头要叼还没叼上的时候,贼眼疾手快,伸左手抄狗的嗓轴子(咽喉),往上一提,提到肚子这儿,使劲卡住了,让狗叫不出来;然后右手一抄狗的后腿,把后腿往手里一攥,这就等于骑上狗了;最后一披大皮袄,还能往前走。有皮袄把狗遮住,从外面根本看不见狗,狗也叫唤不出来。贼偷的都是别人家的狗,然后卖“汤锅”(煮牲口肉的地方),狗主人想找都找不着。

到冬天的时候,贼披着大皮袄,背对着狗,身子朝外,两腿自然叉开,在裆前下食。狗一看,必然得钻裆叼食……





风门·吃珠宝行


干这行买卖必须得下本儿,而且这是一拨人才能做的买卖。专门有人在大街上转悠,经常转悠就知道哪儿有鳏(guān)寡孤独,哪儿有流浪的,哪儿有要饭的。比如盯上一个流浪的老头儿,看准了,过去就认爸爸。“爸爸哎,我找您好多年了,您怎么在这儿哪?”老头儿一听就知道认错人了,可自己又冷又饿,无家可归,也就顺口答音儿:“是啊。”“您跟我回家吧。”“回家?行吗?”“行啊!”这位让旁边的家人叫辆车,老头儿穿的是破棉袄破棉裤,也顾不了那么多了,给拉回来了。嗬!儿媳妇过来就哭,跟真事儿似的。旁边还有使唤人。“您回来先洗洗澡吧。”有人伺候老头儿洗澡,换衣服,然后预备饭菜,吃的喝的都是好的,儿子孝顺爸爸,看着真跟亲爷儿俩似的。给老头儿准备一间上房,就他自己住,专门有人伺候,铺好被褥,老头儿就算住下了。老头儿高兴啊,可连儿子叫什么都不知道。等过了几天,可不是一天两天,起码也得一礼拜,伺候老头儿剃头、换新衣服、闻鼻烟儿……怎么舒服怎么来。别看老头儿原来要饭,等这一礼拜过去了,老爷子满面红光,腰板也直起来了,就跟换了个人似的。这天,儿子过来了:“爸爸,咱们一块儿上大街上转悠转悠去。”老头儿心说:我原来老在大街上转悠。儿子陪着老头儿出去,直奔大栅栏、珠宝市,去的都是大珠宝店。“您下车,跟我们看看去,挑几件首饰。”老头儿跟着进去了。“掌柜的,先给沏壶茶!”掌柜的一看这气派,知道这是买主儿,不敢怠慢,赶紧沏上茶。老头儿在旁边喝着上好的小叶茶,儿子和儿媳妇在旁边挑首饰,挑完了还让老头儿过目。“您看,这是翡翠扳指儿,红的是翡,绿的是翠,怎么样?”“好啊。”“先搁这儿。您再看看这手串儿,象牙的。”老头儿什么都不懂啊,顺口就说:“啊,你们看着办吧,挺好挺好。”儿子跟掌柜的说:“掌柜的,这样,让我爸爸在这儿喝茶等着,我挑几件首饰带回去让家里人看看,看上哪件我就结哪件的账。”“没问题,您多拿上几件,好有挑的富余。”儿子儿媳妇挑了五六件,带着人坐包月车回去了,临走时还说呢:“您可把我们老爷子伺候好了。”“您就放心吧。”掌柜的心说:这是你的亲爸爸,怎么你都得回来。他可没想到,儿子说的这些话都是成心给他听的。本主儿走了之后,珠宝店的人从快中午开始等,一直到快吃晚饭了,人都没回来。掌柜的可犯嘀咕了,就跟老头儿聊天儿:“您家在哪儿住啊?”“我是北城的。”“北城哪儿啊?”“南锣鼓巷。”“哦,您这儿子孝顺您么?”“孝顺哪。”“那他怎么还不回来啊?”“我还想他哪,要没他我上哪儿吃饭去啊。”又等了一会儿,天都黑了,都快上板儿了,掌柜的心里急呀:“老爷子,他是您亲儿子么?”“怎么,你查户口啊?”又等了十几分钟,掌柜的好像明白过味儿来了:“您这儿子怎么还不回来呀?”老头儿也憋不住了:“嗐!跟您说实话吧,这儿子我认了有一礼拜,一礼拜前我要饭。”“啊!那他怎么是你儿子啊?”“认错人了么。”掌柜的一想,我赶紧雇辆车,让老头儿领道儿,保不齐还能追回来。掌柜的和伙计跟着,别说,老头儿还真记住地方了,来到南锣鼓巷一条胡同,一座四合院。等进来一看,四壁空空。“住上房的人呢?”“他们租了半个月,已经搬走了。”这就是一边租房一边踩道,拉来“点儿”(被蒙的人)之后好讹珠宝行。这种买卖是蜻蜓点水,说撤就撤。

吃珠宝行的必须得下本儿,而且这是一拨人才能做的买卖。专门有人在大街上转悠,比如盯上一个流浪的老头儿,看准了,过去就认爸爸。





雁门·吃保险公司


干这行找的“点子”是小孩儿,沿街乞讨要饭的小孩儿,还不能太年轻了,得找对“点子”——十五六岁,身体好,不爱出卖劳动力。这一天,来了这么两个家人,到他近前一看:“哟!少爷,您怎么在这儿呢?老爷都急疯了。”小孩儿一想:哦,他们认错人了。正合适啊,这几天我饿坏了,跟着他们享几天福去。想到这儿,顺嘴答音儿:“啊,可不是我么!”“得得得,咱们赶紧回家吧。”等到家一看,老头儿老太太五十多岁了,抱着孩子就哭。嗬!演得这叫一个逼真!“儿啊,想死我们啦!”老两口子掉眼泪,小孩儿也跟着哭:“是啊,我也饿得够呛啊!”他倒真说实话。小孩儿抬头一看,大四合的房子,这家真阔啊!“赶紧给少爷洗澡,拿新衣服换上!”家人一通忙活。然后吃好的喝好的,这孩子哪儿享过这福啊?不是我亲爹,我先得着吧。就这样,养了半个多月的时间,再看这孩子,满面红光,身体倍儿棒。老爷带着他上保险公司,上高额人寿保险。“这是我儿子,千顷地一棵苗,我得给他上保险。”保险公司一看,带着孩子上医院检查检查,结果没有任何毛病,于是给这孩子上了三十年的人寿保险,三十年内不能发生意外。签完合同,老爷带孩子回家了。从这一天开始,专门有两个家人领着,可就不教孩子好了,逛窑子,抽大烟,扎吗啡……几十天的工夫,花柳病就传染上了,这孩子连咳嗽带喘,病也就上身子了。就这样,用不了多长时间,绝对超不过三年,这孩子准死,一命呜呼。等到医药罔效,救不活的时候,这家人告保险公司。保险公司没办法,只得支付高额赔偿。这就是专门一路人,吃保险公司。这路人找的“点子”(被蒙的人)都是八竿子打不着的小孩子,是假儿子,他们的亲儿子可不能这么干,这纯粹是拿人命换钱哪!

吃保险公司的这路人专门找八竿子打不着的小孩子,带着他上保险公司,上高额人寿保险。从签完合同这一天开始,专门有两个家人领着,可就不教孩子好了,逛窑子,抽大烟,扎吗啡……绝对超不过三年,这孩子准死。保险公司没办法,只得支付高额赔偿。





。
\backmatter

\end{document}