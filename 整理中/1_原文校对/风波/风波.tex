% 风波
% 风波.tex

\documentclass[12pt,UTF8]{ctexbook}

% 设置纸张信息。
\usepackage[a4paper,twoside]{geometry}
\geometry{
	left=25mm,
	right=25mm,
	bottom=25.4mm,
	bindingoffset=10mm
}

% 设置字体,并解决显示难检字问题。
\xeCJKsetup{AutoFallBack=true}
\setCJKmainfont{SimSun}[BoldFont=SimHei, ItalicFont=KaiTi, FallBack=SimSun-ExtB]

% 目录 chapter 级别加点(.)。
\usepackage{titletoc}
\titlecontents{chapter}[0pt]{\vspace{3mm}\bf\addvspace{2pt}\filright}{\contentspush{\thecontentslabel\hspace{0.8em}}}{}{\titlerule*[8pt]{.}\contentspage}

% 设置 part 和 chapter 标题格式。
\ctexset{
	chapter/name={},
	chapter/number={}
}

% 设置署名格式。
\newenvironment{shuming}{\hfill}

% 注脚每页重新编号,避免编号过大。
\usepackage[perpage]{footmisc}

\title{\heiti\zihao{0} 风波}
\author{鲁迅}
\date{}

\begin{document}

\maketitle
\tableofcontents

\frontmatter
\chapter{前言、序言}

\mainmatter

临河的土场上,太阳渐渐的收了他通黄的光线了。场边靠河的乌桕树叶,干巴巴的才喘过气来,几个花脚蚊子在下面哼着飞舞。面河的农家的烟突里,逐渐减少了炊烟,女人孩子们都在自己门口的土场上泼些水,放下小桌子和矮凳;人知道,这已经是晚饭时候了。

老人男人坐在矮凳上,摇着大芭蕉扇闲谈,孩子飞也似的跑,或者蹲在乌桕树下赌玩石子。女人端出乌黑的蒸干菜和松花黄的米饭,热蓬蓬冒烟。河里驶过文人的酒船,文豪见了,大发诗兴,说,“无思无虑,这真是田家乐呵!”

但文豪的话有些不合事实,就因为他们没有听到九斤老太的话。这时候,九斤老太正在大怒,拿破芭蕉扇敲着凳脚说:

“我活到七十九岁了,活够了,不愿意眼见这些败家相,——还是死的好。立刻就要吃饭了,还吃炒豆子,吃穷了一家子!”

伊的曾孙女儿六斤捏着一把豆,正从对面跑来,见这情形,便直奔河边,藏在乌桕树后,伸出双丫角的小头,大声说,“这老不死的!”

九斤老太虽然高寿,耳朵却还不很聋,但也没有听到孩子的话,仍旧自己说,“这真是一代不如一代!”

这村庄的习惯有点特别,女人生下孩子,多喜欢用秤称了轻重,便用斤数当作小名。九斤老太自从庆祝了五十大寿以后,便渐渐的变了不平家,常说伊年青的时候,天气没有现在这般热,豆子也没有现在这般硬;总之现在的时世是不对了。何况六斤比伊的曾祖,少了三斤,比伊父亲七斤,又少了一斤,这真是一条颠扑不破的实例。所以伊又用劲说,“这真是一代不如一代!”
伊的儿媳1七斤嫂子正捧着饭篮走到桌边,便将饭篮在桌上一摔,愤愤的说,“你老人家又这么说了。六斤生下来的时候,不是六斤五两么?你家的秤又是私秤,加重称,十八两秤;用了准十六,我们的六斤该有七斤多哩。我想便是太公和公公,也不见得正是九斤八斤十足,用的秤也许是十四两……”
“一代不如一代!”
七斤嫂还没有答话,忽然看见七斤从小巷口转出,便移了方向,对他嚷道,“你这死尸怎么这时候才回来,死到那里去了!不管人家等着你开饭!”
七斤虽然住在农村,却早有些飞黄腾达的意思。从他的祖父到他,三代不捏锄头柄了;他也照例的帮人撑着航船,每日一回,早晨从鲁镇进城,傍晚又回到鲁镇,因此很知道些时事:例如什么地方,雷公劈死了蜈蚣精;什么地方,闺女生了一个夜叉之类。他在村人里面,的确已经是一名出场人物了。但夏天吃饭不点灯,却还守着农家习惯,所以回家太迟,是该骂的。
七斤一手捏着象牙嘴白铜斗六尺多长的湘妃竹烟管,低着头,慢慢地走来,坐在矮凳上。六斤也趁势溜出,坐在他身边,叫他爹爹。七斤没有应。
“一代不如一代!”九斤老太说。
七斤慢慢地抬起头来,叹一口气说,“皇帝坐了龙庭了。”
七斤嫂呆了一刻,忽而恍然大悟的道,“这可好了,这不是又要皇恩大赦了么!”
七斤又叹一口气,说,“我没有辫子。”
“皇帝要辫子么?”
“皇帝要辫子。”
“你怎么知道呢?”七斤嫂有些着急,赶忙的问。
“咸亨酒店里的人,都说要的。”
七斤嫂这时从直觉上觉得事情似乎有些不妙了,因为咸亨酒店是消息灵通的所在。伊一转眼瞥见七斤的光头,便忍不住动怒,怪他恨他怨他;忽然又绝望起来,装好一碗饭,搡在七斤的面前道,“还是赶快吃你的饭罢!哭丧着脸,就会长出辫子来么?”
太阳收尽了他最末的光线了,水面暗暗地回复过凉气来;土场上一片碗筷声响,人人的脊梁上又都吐出汗粒。七斤嫂吃完三碗饭,偶然抬起头,心坎里便禁不住突突地发跳。伊透过乌桕叶,看见又矮又胖的赵七爷正从独木桥上走来,而且穿着宝蓝色竹布的长衫。
赵七爷是邻村茂源酒店的主人,又是这三十里方圆以内的唯一的出色人物兼学问家;因为有学问,所以又有些遗老的臭味。他有十多本金圣叹批评的《三国志》2,时常坐着一个字一个字的读;他不但能说出五虎将姓名,甚而至于还知道黄忠表字汉升和马超表字孟起。革命以后,他便将辫子盘在顶上,像道士一般;常常叹息说,倘若赵子龙在世,天下便不会乱到这地步了。七斤嫂眼睛好,早望见今天的赵七爷已经不是道士,却变成光滑头皮,乌黑发顶;伊便知道这一定是皇帝坐了龙庭,而且一定须有辫子,而且七斤一定是非常危险。因为赵七爷的这件竹布长衫,轻易是不常穿的,三年以来,只穿过两次:一次是和他怄气的麻子阿四病了的时候,一次是曾经砸烂他酒店的鲁大爷死了的时候;现在是第三次了,这一定又是于他有庆,于他的仇家有殃了。
七斤嫂记得,两年前七斤喝醉了酒,曾经骂过赵七爷是“贱胎”,所以这时便立刻直觉到七斤的危险,心坎里突突地发起跳来。
赵七爷一路走来,坐着吃饭的人都站起身,拿筷子点着自己的饭碗说,“七爷,请在我们这里用饭!”七爷也一路点头,说道“请请”,却一径走到七斤家的桌旁。七斤们连忙招呼,七爷也微笑着说“请请”,一面细细的研究他们的饭菜。
“好香的干菜,——听到了风声了么?”赵七爷站在七斤的后面七斤嫂的对面说。
“皇帝坐了龙庭了。”七斤说。
七斤嫂看着七爷的脸,竭力陪笑道,“皇帝已经坐了龙庭,几时皇恩大赦呢?”
“皇恩大赦?——大赦是慢慢的总要大赦罢。”七爷说到这里,声色忽然严厉起来,“但是你家七斤的辫子呢,辫子?这倒是要紧的事。你们知道:长毛时候,留发不留头,留头不留发,……”
七斤和他的女人没有读过书,不很懂得这古典的奥妙,但觉得有学问的七爷这么说,事情自然非常重大,无可挽回,便仿佛受了死刑宣告似的,耳朵里嗡的一声,再也说不出一句话。
“一代不如一代,——”九斤老太正在不平,趁这机会,便对赵七爷说,“现在的长毛,只是剪人家的辫子,僧不僧,道不道的。从前的长毛,这样的么?我活到七十九岁了,活够了。从前的长毛是——整匹的红缎子裹头,拖下去,拖下去,一直拖到脚跟;王爷是黄缎子,拖下去,黄缎子;红缎子,黄缎子,——我活够了,七十九岁了。”
七斤嫂站起身,自言自语的说,“这怎么好呢?这样的一班老小,都靠他养活的人,……”
赵七爷摇头道,“那也没法。没有辫子,该当何罪,书上都一条一条明明白白写着的。不管他家里有些什么人。”
七斤嫂听到书上写着,可真是完全绝望了;自己急得没法,便忽然又恨到七斤。伊用筷子指着他的鼻尖说,“这死尸自作自受!造反的时候,我本来说,不要撑船了,不要上城了。他偏要死进城去,滚进城去,进城便被人剪去了辫子。从前是绢光乌黑的辫子,现在弄得僧不僧道不道的。这囚徒自作自受,带累了我们又怎么说呢?这活死尸的囚徒……”
村人看见赵七爷到村,都赶紧吃完饭,聚在七斤家饭桌的周围。七斤自己知道是出场人物,被女人当大众这样辱骂,很不雅观,便只得抬起头,慢慢地说道:
“你今天说现成话,那时你……”
“你这活死尸的囚徒……”
看客中间,八一嫂是心肠最好的人,抱着伊的两周岁的遗腹子,正在七斤嫂身边看热闹;这时过意不去,连忙解劝说,“七斤嫂,算了罢。人不是神仙,谁知道未来事呢?便是七斤嫂,那时不也说,没有辫子倒也没有什么丑么?况且衙门里的大老爷也还没有告示,……”
七斤嫂没有听完,两个耳朵早通红了;便将筷子转过向来,指着八一嫂的鼻子,说,“阿呀,这是什么话呵!八一嫂,我自己看来倒还是一个人,会说出这样昏诞胡涂话么?那时我是,整整哭了三天,谁都看见;连六斤这小鬼也都哭,……”六斤刚吃完一大碗饭,拿了空碗,伸手去嚷着要添。七斤嫂正没好气,便用筷子在伊的双丫角中间,直扎下去,大喝道,“谁要你来多嘴!你这偷汉的小寡妇!”
扑的一声,六斤手里的空碗落在地上了,恰巧又碰着一块砖角,立刻破成一个很大的缺口。七斤直跳起来,捡起破碗,合上检查一回,也喝道,“入娘的!”一巴掌打倒了六斤。六斤躺着哭,九斤老太拉了伊的手,连说着“一代不如一代”,一同走了。
八一嫂也发怒,大声说,“七斤嫂,你‘恨棒打人’……”
赵七爷本来是笑着旁观的;但自从八一嫂说了“衙门里的大老爷没有告示”这话以后,却有些生气了。这时他已经绕出桌旁,接着说,“‘恨棒打人’,算什么呢。大兵是就要到的。你可知道,这回保驾的是张大帅3。张大帅就是燕人张翼德的后代,他一支丈八蛇矛,就有万夫不当之勇,谁能抵挡他,”他两手同时捏起空拳,仿佛握着无形的蛇矛模样,向八一嫂抢进几步道,“你能抵挡他么!”
八一嫂正气得抱着孩子发抖,忽然见赵七爷满脸油汗,瞪着眼,准对伊冲过来,便十分害怕,不敢说完话,回身走了。赵七爷也跟着走去,众人一面怪八一嫂多事,一面让开路,几个剪过辫子重新留起的便赶快躲在人丛后面,怕他看见。赵七爷也不细心察访,通过人丛,忽然转入乌桕树后,说道“你能抵挡他么!”跨上独木桥,扬长去了。
村人们呆呆站着,心里计算,都觉得自己确乎抵不住张翼德,因此也决定七斤便要没有性命。七斤既然犯了皇法,想起他往常对人谈论城中的新闻的时候,就不该含着长烟管显出那般骄傲模样,所以对七斤的犯法,也觉得有些畅快。他们也仿佛想发些议论,却又觉得没有什么议论可发。嗡嗡的一阵乱嚷,蚊子都撞过赤膊身子,闯到乌桕树下去做市;他们也就慢慢地走散回家,关上门去睡觉。七斤嫂咕哝着,也收了家伙和桌子矮凳回家,关上门睡觉了。
七斤将破碗拿回家里,坐在门槛上吸烟;但非常忧愁,忘却了吸烟,象牙嘴六尺多长湘妃竹烟管的白铜斗里的火光,渐渐发黑了。他心里但觉得事情似乎十分危急,也想想些方法,想些计画,但总是非常模糊,贯穿不得:“辫子呢辫子?丈八蛇矛。一代不如一代!皇帝坐龙庭。破的碗须得上城去钉好。谁能抵挡他?书上一条一条写着。入娘的!……”
第二日清晨,七斤依旧从鲁镇撑航船进城,傍晚回到鲁镇,又拿着六尺多长的湘妃竹烟管和一个饭碗回村。他在晚饭席上,对九斤老太说,这碗是在城内钉合的,因为缺口大,所以要十六个铜钉,三文一个,一总用了四十八文小钱。
九斤老太很不高兴的说,“一代不如一代,我是活够了。三文钱一个钉;从前的钉,这样的么?从前的钉是……我活了七十九岁了,——”
此后七斤虽然是照例日日进城,但家景总有些黯淡,村人大抵回避着,不再来听他从城内得来的新闻。七斤嫂也没有好声气,还时常叫他“囚徒”。
过了十多日,七斤从城内回家,看见他的女人非常高兴,问他说,“你在城里可听到些什么?”
“没有听到些什么。”
“皇帝坐了龙庭没有呢?”
“他们没有说。”
“咸亨酒店里也没有人说么?”
“也没人说。”
“我想皇帝一定是不坐龙庭了。我今天走过赵七爷的店前,看见他又坐着念书了,辫子又盘在顶上了,也没有穿长衫。”
“…………”
“你想,不坐龙庭了罢?”
“我想,不坐了罢。”
现在的七斤,是七斤嫂和村人又都早给他相当的尊敬,相当的待遇了。到夏天,他们仍旧在自家门口的土场上吃饭;大家见了,都笑嘻嘻的招呼。九斤老太早已做过八十大寿,仍然不平而且健康。六斤的双丫角,已经变成一支大辫子了;伊虽然新近裹脚,却还能帮同七斤嫂做事,捧着十八个铜钉4的饭碗,在土场上一瘸一拐的往来。 [2]
词句注释
播报
编辑

1.
伊的儿媳:从上下文看,这里的“儿媳”应是“孙媳”。
2.
金圣叹批评的《三国志》:指小说《三国演义》。金圣叹(1609—1661年),明末清初文人,曾批注《水浒》、《西厢记》等书,他把所加的序文、读法和评语等称为“圣叹外书”。《三国演义》是元末明初罗贯中所著,后经清代毛宗岗改编,附加评语,卷首有假托为金圣叹所作的序,首回前也有“圣叹外书”字样,通常就都把这评语认为金圣叹所作。
3.
张大帅:指张勋(1854—1923年),江西奉新人,北洋军阀之一。原为清朝军官,辛亥革命后,他和所部官兵仍留着辫子,表示忠于清王朝,被称为辫子军。
4.
十八个铜钉:据上文应是“十六个”。作者在1926年11月23日致李霁野的信中曾说:“六斤家只有这一个钉过的碗,钉是十六或十八,我也记不清了。总之两数之一是错的,请改成一律。” [3]

\begin{shuming}
一九二〇年七月\footnote{本篇最初发表于1919年12月1日北京《晨报·周年纪念增刊》。据报刊发表的年月及鲁迅日记,本篇写作时间当在1919年11月。}
\end{shuming}
  
\end{document}