% 南京大屠杀
% 南京大屠杀.tex

\documentclass[12pt,UTF8]{ctexbook}

% 设置纸张信息。
% 纸张设置配置文件
% 用于定义书籍的页面尺寸和边距

\usepackage[a4paper,twoside]{geometry}
\geometry{
	left=25mm,
	right=20mm,
	top=25mm,
	bottom=25.4mm,
	headsep=1cm, 
    footskip=1cm,
	bindingoffset=10mm
}

% 设置字体,并解决显示难检字问题。
\xeCJKsetup{AutoFallBack=true}
\setCJKmainfont{SimSun}[BoldFont=SimHei, ItalicFont=KaiTi, FallBack=SimSun-ExtB]

% 目录 chapter 级别加点(.)。
\usepackage{titletoc}
\titlecontents{chapter}[0pt]{\vspace{3mm}\bf\addvspace{2pt}\filright}{\contentspush{\thecontentslabel\hspace{0.8em}}}{}{\titlerule*[8pt]{.}\contentspage}

% 设置 part 和 chapter 标题格式。
\ctexset{
	chapter/name={第,章},
	chapter/number={\chinese{chapter}}
}

% 列表项向右偏移。
\usepackage{enumitem}

% 图片相关设置。
\usepackage{graphicx}
\graphicspath{{Images/}}

% 设置署名格式。
\newenvironment{shuming}{\hfill\zihao{4}}

% 注脚每页重新编号,避免编号过大。
\usepackage[perpage]{footmisc}

\title{\heiti\zihao{0} 南京大屠杀}
\author{}
\date{}

\begin{document}

\maketitle
\tableofcontents

\frontmatter

\begin{figure}[htbp]
	\centering
	\includegraphics[width=0.7\linewidth]{cover}
	\caption{}
	\label{fig:1}
\end{figure}

版权页

图书在版编目(CIP)数据 
南京大屠杀/(美)张纯如著;谭春霞,焦国林译. ——北京:中信出版社,2013.1 
书名原文:The Rape of Nanking 
ISBN 978–7–5086–3724–2 
I. 南… II.①张… ②谭… ③焦… III. 南京大屠杀-史料IV. K265.606 
中国版本图书馆CIP数据核字(2012)第287316号 
The Rape of Nanking by Iris Chang

Copyright © 1997 by Iris Chang 
Simplified Chinese translation copyright © 2012 by China CITIC Press 
Published by arrangement with Basic Books, a member of Perseus Books Group 
through Bardon-Chinese Media Agency 
Simplified Chinese edition © 2012 by China CITIC Press 
All rights reserved 
本书仅限中国大陆地区发行销售 


南京大屠杀 
著者:[美] 张纯如 
译者:谭春霞 焦国林 
策划推广:中信出版社(China CITIC Press) 
出版发行:中信出版集团股份有限公司(北京市朝阳区惠新东街甲4号富盛大厦2座 邮编100029)
(CITIC Publishing Group) 

扉页

由于张纯如的这本书,“第二次南京大屠杀”为之终结。
——乔治·威尔(George Will)
《华盛顿邮报》专栏作家

谨以此书献给
南京大屠杀数十万受害者

媒体与专家热评

对历史和道德进行探究的最新力作。张纯如极其认真地对这场大屠杀的各个层面进行了详细叙述。
《芝加哥论坛报》

该书的研究工作一丝不苟……全书自始至终都引人入胜。
郑念
《上海生死劫》的作者

张纯如对南京大屠杀的研究扩展了“二战”暴行的内容,该书反映了她对该问题研究的深入。该书非常精彩,值得一读。
白彬菊
耶鲁大学历史系教授

该书读来令人心碎……扣人心弦的一本书。书中对南京大屠杀的描述引发了我们对许多根本性问题的思考,不仅包括日本的军国主义,还包括那些虐待狂、强奸犯和谋杀犯的心理因素。
魏斐德
加利福尼亚大学伯克利分校东亚研究所所长

张纯如的外祖父母侥幸逃过了南京大屠杀……她在这部最新力作中,详细叙述了这场耸人听闻的大屠杀,她在文中表现的愤怒之情是可以理解的。
夏伟
《纽约时报书评》

任何对战争、自以为是和人类精神之间的关系感兴趣的人,都会发现该书的重要意义。它是一项激动人心的学术研究,也是一本充满激情的著作。该书的很多章节令人不忍卒读,但我们应该阅读,因为只有理解过去,才能更好地驾驭未来。
罗斯·特里尔
《毛泽东传》的作者

作为20世纪极其重要的著作之一,张纯如的《南京大屠杀》将成为世界战争史中的经典著作。
汤美如
影片《以天皇的名义》的制片人兼联合导演

该书结构严谨,可读性强……张纯如让这段鲜为人知的历史广受关注。
拉塞尔·詹金斯
《国家评论》

在这个动荡的世纪即将结束之际,张纯如的著作唤起了民众对“二战”中这段最黑暗历史的关注,为通向未来和平的道路洒满阳光。
史咏
《南京大屠杀:历史照片中的见证》的合著者

张纯如的故事读来字字惊心……书中详尽地记录了对这一道德暴行的控诉。
《休斯顿纪事报》

这是一段不容否认的历史,该书的意义在于,它既记录了人类在实施暴行时的冷酷无情,又通过个人的英雄主义行为让我们看到了希望。
《圣何塞水星报》

张纯如提醒我们,不论发生在南京的种种暴行多么令人难以理解,它们都不应该被遗忘——至少不能让遗忘危及文明自身。
《底特律新闻报》

张纯如所做的恢复历史的工作非常急迫……她的著作促使世人迈出了承认这场悲剧的重要一步。
《海湾卫士报》

\begin{figure}[htbp]
	\centering
	\includegraphics[width=0.7\linewidth]{1}
	\caption{}
	\label{fig:1}
\end{figure}

中文版序

该书中译本《南京浩劫:被遗忘的大屠杀》在2007年南京大屠杀70周年时出版,又已过了5年。2012年是南京大屠杀75周年,在这5年间发生了许多事情,其中包括我为我女儿纯如写的英文回忆录《The Woman Who Could Not Forget: Iris Chang Before and Beyond The Rape of Nanking》已于2011年在美国出版,中译本《张纯如:无法忘却历史的女子》也已由中信出版社在2012年4月出版。在回忆录中我详细记载了纯如自出生到逝世的36年短暂的一生,其中最重要的一章即她写作该书的详细经过。中信出版社为南京大屠杀75周年重新翻译出版纯如的这本著作,我感到非常荣幸能为这本书再作介绍。
纯如在该书写作及出版的过程中付出许多努力,最后克服困难完成写作。纯如在1995年1月就前往美国国会图书馆及耶鲁大学神学院图书馆收集数据,后来又在1995年7月到南京实地采访南京大屠杀的幸存者。回美后,纯如努力阅读整理所得资料,并在1996年找到南京大屠杀期间安全区领袖德国商人拉贝的外孙女而发现了《拉贝日记》。这一连串的活动都是纯如孜孜不息地努力得到的结果。纯如在写作过程中阅读了大量有关屠杀的血腥的文献和报告,以致精神上受到很大的震撼,导致失眠、厌食。但她仍然坚持完成该书,她对我说:“作为一个作家,我要拯救那些被遗忘的人。为那些不能发声的人发言。”这都基于她为受害者鸣不平的执着。这是她维护历史真相、保护人权的一种热忱的表现,当然亦是她本身敬业精神所至。
该书于1997年11月南京大屠杀60周年之际出版,出版后不久就登上了《纽约时报》非小说类畅销书排行榜,并达3个月之久。纯如是当时唯一一位作品登上非小说类排行榜的美籍华人,因此该书立刻受到了美国出版界的重视,在美国引起轰动。后来该书被翻译成十几种语言文字,成为一本国际畅销书。一般来说,美国主流社会对第二次世界大战中犹太人被德国纳粹迫害的历史非常熟悉,对“二战”中日本对中国及邻国的侵略历史却所知有限。这要归根于美国等西方国家对亚洲“二战”历史的漠视。该书当时是唯一用英语专题讨论南京大屠杀的书,不仅叙述了这段不为西方国家所了解的可悲历史,更重要的是深刻地揭示了人性的善与恶,批评了人类的种族歧视现象,并探讨了幸存者人权的问题,体现了纯如维护人权及正义的热忱。该书对美国等西方国家了解“二战”亚洲战场中日本侵华历史有极大的贡献和深远的影响。
1999年9月,纯如在一封给我们的家书中写道:“南京大屠杀终于在世界历史中展开了自己最真实的那一页。我上周去本地的书店时,发现许多新近出版的有20世纪历史的著作中都写到了南京大屠杀。例如,马丁· 吉尔伯特在他的长篇巨著《20世纪世界史(第二卷):1933~1951年》中就写到了南京大屠杀,甚至还直接引用了我书中的内容。彼得·詹宁斯的《世纪》以及史蒂芬· 安布罗斯的《新编“二战”历史》中也提到了南京大屠杀。”该书的出版改变了西方英语国家没有关于南京大屠杀这一历史事件详细记载的状况。
另外,该书在国际上产生了影响,例如2005年,当日本想进入联合国安全理事会成为常任理事国时,在美国的华人发动签名请愿上诉运动,反对日本得到这个特殊地位。那时全球网民在短短几周内就征集了数千万签名,向联合国请愿,成功地阻止日本野心得逞。很多文章及新闻报道都提到日本没有资格进入联合国安理会的原因,就是日本仍然没有真诚地为“二战”中的战争罪行道歉。而提到日本在华罪行时,首先想到的就是南京大屠杀,许多的报道均提到纯如的这本书。由此可见,南京大屠杀惨剧因这本书在国际上己被广泛地认知。
2007年为南京大屠杀70周年,各地举办许多活动纪念这段历史。其中有加拿大制作的电影《张纯如:南京大屠杀》和美国制作的纪录片《南京》,在国际上都轰动一时。影片《张纯如:南京大屠杀》描述了纯如追溯这段历史的经过;《南京》以纪录片的形式描述南京大屠杀,是美国在线前副总裁特德·莱昂西斯受到纯如这本书的启发和感召,个人投资200万美元拍摄而成。2009年,中、德、法合作制作的电影《约翰·拉贝》是根据《拉贝日记》拍摄而成,描述拉贝在南京大屠杀期间英勇拯救并保护中国难民的事迹。《拉贝日记》是纯如1996年在德国找到拉贝的后裔而发现的,这是南京大屠杀一个非常重要的不可磨灭的证言。所有历史学家都认为,发现《拉贝日记》是纯如对这段历史最大的贡献之一。所以该书在帮助世界民众了解这段历史方面起了很大的作用。
当时纯如的这本书一出版,美国的各大报纸刊登了许多关于该书的书评,这些书评对这本书有着高度的评价。就在这本书受到许多人的推崇时,日本右翼分子感到惶恐不安,他们开始了针对这本书的一连串攻击。不久,美国国内也有一些所谓的“历史学家”开始批评这本书。但与正面的评论相比,批评的声音只是极少数。这也是一种正常的现象,是任何成功的作家(特别是成功的年轻作家)不可避免的。其实,纯如从来都未自我标榜过这本书是这段历史的权威著作。她在该书的前言中明确表示,希望这本书起到抛砖引玉的作用,引起更多的人对这段历史的兴趣,进行更多的研究。美国著名的“二战”史历史学家史蒂芬·安布罗斯曾经说:“张纯如是近代最卓越的年轻历史作家,因为她懂得如何将历史写得令人感兴趣。”这大概是对该书的某些批评者最好的回答。
由于这本书的关系,我们也逐渐了解到政治的复杂和黑暗。战后,日本在美国的扶持下变为一个经济大国,并使用大量的金钱对美国各阶层展开攻势,粉饰日本是一个自由、民主及爱好和平的国家,但对“二战”中日本侵略亚洲各国的历史避而不谈,反而把日本描述为“二战”的受害者,将日本对这些国家的侵略美化为拯救这些国家。我们还要注意,有些在美国研究中国或亚洲的所谓“历史专家”其实是在日本大量的金钱资助下为日本说话的。在美国大学里有些研究中日历史的经费也是来自日本,因此他们的研究很难保持客观。这一切,当然美国应该对此负一部分责任。战后为了拉拢日本,使之成其盟国,以便对付共产主义国家,美国保持了日本战前的政治结构,并使许多日本战犯逃避了应有的惩罚,这批战犯及他们的后裔成为当今日本政治的核心人物。
写到这里使我最为痛心的是南京大屠杀75年后,日本仍然未真诚地向中国道歉和赔偿。日本右翼分子甚至否认南京大屠杀,掩盖战争罪行,篡改历史。最近日本对钓鱼岛的“购岛”及“国有化”一连串的闹剧,更可见日本军国主义阴魂不散,仍在为日本做“大东亚共荣圈”的白日梦。因此我们绝不能遗忘这段历史,而且要努力教育下一代铭记这段历史。
因为日本自“二战”后从未对自己的罪行真正地进行过道歉,所以这本书最终的目的是呼吁日本真诚地反省,力促日本对受害国家的人民道歉及赔偿。纯如在前言中用警语“忘记过去的人注定要重蹈覆辙”来警醒世界。
1994年获得诺贝尔文学奖的日本小说家大江健三郎曾在《纽约时报》上发表题为《否认历史将摧毁日本》的文章(1995年7月2日),他指出:“日本一定要对(第二次世界大战的)侵略进行道歉和赔偿。这是基本的要求,大多数有良知的日本人都赞成。但是有一群保守的日本党派和商界领袖反对。”如果日本未来希望得到各国的尊敬,唯一的途径就是要对“二战”中其对邻国发动的侵略战争的罪行进行真诚的道歉和赔偿,这样中日之间才能实现真正的友好与和平。纯如在书中强调,她写本书的目的不是煽动仇日情绪,恰恰相反,是为了避免悲剧的重演,是为了包括日本人在内的全人类的未来。
纯如不幸于2004年去世。在我写的回忆录里,除了描述她一生为真理正义而奋斗,我特别强调我们应该记住的是她精彩的一生。纯如自始至终坚信一个人的力量可以改变这个世界。正如《华盛顿邮报》的专栏作家乔治·威尔所说:“由于张纯如的这本书,“第二次南京大屠杀”为之终结。”
在此寄语读者:“勿忘历史,以史为鉴!”并以此与各位共勉。

张盈盈

2012年9月27日写于美国加州圣何塞市

序言

1937年12月13日,国民党统治下的中华民国首都南京陷入日本侵略者之手。对日本而言,这是中日战争中具有决定意义的转折点,是日军在长江流域与蒋介石的军队奋战半年取得的最辉煌胜利。对中国军队而言,他们英勇保卫上海的战斗最终失败,最精锐的部队也伤亡惨重,而南京陷落则是一种痛苦甚至致命的挫败。
今天,我们或许可以把南京陷落看作一种截然不同的转折点。这座古城所遭受的劫难大大激发了中国人收复南京、赶走侵略者的决心。中国国民党政府从南京撤离后重整旗鼓,中国人民终于在1945年战胜了日本。在这8年战争期间,日本侵略者虽然占领南京,并组建了伪政府,但它从未自信、合法地统治过南京,而且永远无法迫使中国投降。对外部世界来说,南京大屠杀(很快就成为一个专有名词)使世界舆论一边倒地谴责日本,群情激奋,世所罕见。
直到今天,中国的舆论依旧如此,几代中国人都牢记日本的侵略罪行,而且日本战败后至今未对中国进行赔偿。60年过去了,南京的遇难者仍然是中日关系无法回避的问题。
这是必然的。日本侵略者对南京的洗劫骇人听闻。日军大规模处决中国战俘,而且屠杀、强奸了成千上万的中国平民,这些行为违反了关于战争的所有法规惯例。更令人吃惊的是,日本侵略者的这些行为都是公开的,其目的显然是恐吓。日本侵略者在国际观察者的众目睽睽之下施暴,并对国际人士试图阻止暴行的努力置之不理。南京大屠杀并非由于暂时的军纪失控,因为大屠杀持续了7周之久。该书是世界上第一部用英语全面研究南京惨剧的专著,张纯如在这本书中极富感染力地讲述了这个恐怖的事件。
我们或许永远无法确切地了解日军指挥官及其士兵这种野蛮兽行背后的动机,但张纯如的著作比以往任何研究更透彻地分析了日军的所作所为。在此过程中,她使用了丰富的原始资料,包括无可置疑的第三国观察家(那些在日军进入南京后仍然留在这座不设防城市的外国传教士和商人)的证词:张纯如发掘的此类资料之一是约翰·拉贝的日记,事实上这些日记可以构成一个小型的档案馆。拉贝是一名德国商人和国家社会党党员,他在南京大屠杀期间领导了保护南京平民的国际行动。透过拉贝的眼睛,我们看到当时手无寸铁的南京居民在面对日本侵略者的猛烈攻击时经历了怎样的恐惧,展露了怎样的勇敢。通过张纯如的描述,我们不禁钦佩拉贝和其他国际人士的勇气。当时城市横遭兵燹,居民惨遭杀戮,医院关门,太平间尸体残骸成堆,四处混乱不堪,很多国际人士仍然冒着生命危险,试图改变这一切。同时我们也从该书中了解到,当时许多日本人知道南京正在发生的一切后为此感到羞愧。
当西方已经在很大程度上忘却南京大屠杀时,该书更加突显出其重要价值。张纯如称之为“被遗忘的大屠杀”,并将第二次世界大战中发生在欧洲和亚洲的对数百万无辜者的屠杀事件联系在一起。诚然,日本和德国只是后来才成为盟友,而且不是太好的盟友,然而发生在南京的惨案(毫无疑问希特勒也犯下过类似的罪行)却使他们成为道德上的共犯,因为他们作为暴力侵略者都犯下了后来被称为“反人类罪”的滔天罪行。美国诗人奥登\footnote{W·H 奥登,英国著名诗人,1907-1973 年,1948 年获美国普利策诗歌奖。译注}曾在中日战争期间访问中国,他比大多数人更早地将发生在欧亚两洲的大屠杀联系起来\footnote{选自 W·H·奥登《短诗集:1930-1944》中“在战争年代 ” 。第 279-280 页,伦敦,1950。} :

从地图上的确可以找出某些地方,
那里的人民正笼罩在邪恶中:
比如南京,比如达豪\footnote{Dachau,德国巴伐利亚州地名,曾建有纳粹集中营。译注}。

柯伟林
哈佛大学历史系主任
中国近代史教授

前言

人类残忍对待同胞的编年史讲述着漫长而令人痛心的故事,但是如果说这类恐怖故事中人类的残暴程度确实存在差别,那么世界历史上极少有什么暴行可以在强度和规模上与第二次世界大战期间日军进行的南京大屠杀相比。
美国人认为,第二次世界大战始于1941年12月7日,因为日军在这一天偷袭了美国海军基地珍珠港。欧洲人则将1939年9月1日德国突袭波兰视为第二次世界大战的开端。非洲人认为战争开始得更早,应从1935年墨索里尼派兵入侵埃塞俄比亚算起。然而,对于亚洲人来说,战争的发端必须追溯到日本军事控制东亚的第一步——1931年日本侵占中国东北地区并建立伪“满洲国”。
正如希特勒统治下的德国在5年后所做的那样,从1931年开始,日本凭借其高度发达的军事机器和优等民族心态,着手建立起对邻邦的统治。中国东北地区很快陷落,随后成立的伪“满洲国”名义上由作为日本傀儡的清朝废帝溥仪统治,实际的统辖权却掌握在日本军方手中。4年后,即1935年,察哈尔省和河北省的部分地区被占领;1937年,北京、天津、上海相继沦陷,最后连南京也未能幸免。对中国而言,20世纪30年代可谓艰难时世;事实上,直到1945年第二次世界大战结束时,最后一批日军才撤出中国的领土。
毫无疑问,在日军侵略中国的14年中,出现过无数难以付诸笔墨的暴行。我们永远无法巨细无遗地了解,在千千万万个曾遭受日军铁蹄蹂躏的城市和乡村中,究竟有过多少悲惨事件。但我们清楚地知道发生在南京的惨案,因为当时的一些外国人亲眼目睹了这场惨绝人寰的大屠杀,并将消息传播给世界;另外,一些亲历大屠杀的中国人侥幸生还,成为目击证人。如果有哪个历史事件可以揭露肆无忌惮的军事冒险主义十恶不赦的本质,南京大屠杀是最好的例证。本书讲述的正是这个事件。
南京大屠杀的历历详情是毋庸置疑的,只有部分日本人仍在矢口否认。1937年11月,日军在成功攻陷上海之后,紧接着对中华民国首都南京发起了大规模进攻。1937年12月13日,南京沦陷后,日军在这里大肆进行了一场世所罕见的残暴屠杀。成千上万的中国年轻人被聚集并驱赶到城外,或遭机关枪扫射倒地,或被当作练习刺刀的活靶,或被浑身浇满汽油活活烧死。几个月来,南京城内尸横遍地,尸臭弥漫。多年之后,远东国际军事法庭的专家估计,从1937年年底到1938年年初,南京有超过26万名非战斗人员死于日军的屠刀之下,还有专家估计这一数字超过35万。 
本书仅对日本在南京的野蛮暴行进行最基本的概述,因为我的目的并不在于以数字证明南京大屠杀是人类历史上最邪恶的行径之一,而是要洞悉事件本身,从而吸取教训,以警世人。然而,不同的残暴程度通常会引起人们不同的反应,因此我必须列举一些统计数字,从而使读者对1937年发生在南京的大屠杀规模有一个基本的认知。
一位历史学家曾经估算,如果所有南京大屠杀的罹难者手牵手站在一起,这一队伍可以从南京绵延到杭州,总距离长达200英里左右。 他们身上的血液总重量可达1 200吨,他们的尸体则可以装满2 500节火车车厢。
仅从死亡人数看,南京大屠杀就已超越了历史上许多野蛮的暴行。罗马人曾在迦太基屠杀了15万人,天主教军队也曾在西班牙宗教法庭大开杀戒,但日军在南京的暴行却远甚于此。 日军的所作所为甚至超越了帖木儿的暴行,后者曾于1398年在德里处死了10万名囚犯,并于1400年和1401年用这些囚犯的颅骨在叙利亚建造了两座骨塔。 
当然,20世纪以来用于大规模杀戮的工具获得了充分发展,希特勒杀害了600万犹太人,然而该数字是在几年之内累积而成的,日军对南京人的屠杀则集中在几个星期之内。
的确,即使与历史上最具毁灭性的战争相比,南京大屠杀也足以成为大规模赶尽杀绝的最残忍例证。为了更好地了解南京大屠杀的相对规模,我们必须再忍痛看一下其他统计数字。仅仅南京(中国的一座城市而已)的死亡人数就超过一些欧洲国家在整个第二次世界大战期间的平民伤亡总数(英国61 000人,法国108 000人,比利时101 000人,荷兰242 000人)。 忆及此类事件,人们都认为战略轰炸是造成大规模毁灭最恐怖的手段之一,然而即使是“二战”中最猛烈的空袭也无法超越日军对南京的蹂躏。南京的死亡人数很可能超过英国突袭德累斯顿过程中死于轰炸以及火灾的人数(当时国际上普遍接受的死亡人数是225 000人,但根据后来更客观的统计,应有6万人死亡,至少3万人受伤)。 事实上,不论我们使用最保守的数字——26万人,还是最大数字——35万人,南京大屠杀的死亡人数都远远超过美国轰炸东京的死亡人数(据估计有8万~12万人死亡),甚至超过1945年年底广岛、长崎两座城市在遭原子弹轰炸后的死亡人数之和(据估计分别为14万人和7万人)。 每思及此,不禁惊怒交加。
我们不仅要记住南京大屠杀的死亡人数,更要记住许多罹难者被杀害的残忍手段。日军将中国人当作练习刺刀的活靶,甚至进行斩首比赛。估计2万~8万名中国妇女遭到强暴。 许多日本士兵在强暴中国妇女之后甚至还挖出她们的内脏,割掉她们的乳房,将她们活活钉在墙上。 日军强迫父亲强暴亲生女儿,儿子强暴亲生母亲,并强迫其他家庭成员在一旁观看。日军不仅把对中国人进行活埋、阉割、器官切除以及热火炙烤当作家常便饭,还采取其他更为残忍的折磨手段。例如,用铁钩穿住舌头把整个人吊起来;把人活埋至腰部,然后在一旁幸灾乐祸地放任德国黑贝将他们撕碎。此种残暴景象实在触目惊心,甚至连当时住在南京的纳粹党人都惊骇不已,其中一位曾公开将南京大屠杀称为“野兽机器”的暴行。 
然而,日军在南京的暴行一直鲜为人知。与日本遭受原子弹袭击和犹太人在欧洲所遭受的大屠杀不同,亚洲以外的人几乎不知道南京大屠杀的恐怖。美国出版的大部分历史文献都忽略了这一历史事件。仔细调查美国高中的历史教科书就会发现,只有寥寥几本简略提到了南京大屠杀。面向美国公众发行的完整版或“权威版”有关第二次世界大战的历史著作中,几乎没有一本书详细地记述南京大屠杀。例如,《美国传统图片史:第二次世界大战》(The American Heritage Picture History of World War II,1966)是美国多年来最畅销的单卷本“二战”图片史图书,其中不但没有收录一张南京大屠杀的图片,甚至对事件本身只字未提。无论是丘吉尔长达1 065页的名著《第二次世界大战回忆录》(Memoirs of the Second World War,1959),还是亨利·米歇尔长达947页的经典之作《第二次世界大战》(Second World War,1975)中,都找不到任何关于南京大屠杀的只言片语。在格哈特·温伯格长达1 178页的鸿篇巨制《战火中的世界》(A World at Arms,1994)中,只有两处蜻蜓点水般地提到了南京大屠杀。我只在罗伯特·莱基长达998页的《摆脱邪恶:“二战”传奇》(Delivered from Evil: The Saga of World War II,1987)中找到仅有的一段对南京大屠杀的论述:“与松井石根领导下的日本士兵相比,希特勒领导下的纳粹所做的任何令其胜利蒙羞的丑行都相形见绌。” 
第一次听说南京大屠杀时,我还很小。事件是父母讲述的,他们在中国多年的战乱与革命中幸免于难,后来在美国中西部的大学城担任教职,得以安身立命。他们在“二战”时期的中国长大,战后先是随家人逃亡到台湾,最后来到美国的哈佛大学求学,以自然科学的学术研究为业。30年来,他们平静地生活在伊利诺伊大学厄巴纳–香槟分校,从事物理学和微生物学研究。
但他们从未忘却中日战争的恐怖,也希望我不要忘记这一切,他们尤其希望我不要忘记南京大屠杀。我的父母虽然不曾亲眼目睹南京大屠杀,但他们从小就听人讲述这一事件,后来又讲给我听。我从他们那里了解到,日军不仅会将婴儿劈成两半,甚至砍成三四段;曾有一段时间,长江都被鲜血染成红色。他们在讲述这些时,声音因愤怒而颤抖。他们认为,南京大屠杀是日本侵略者在这场导致1 000多万中国人丧生的战争中所犯下的最残忍暴虐的罪行。
在我的整个童年,南京大屠杀一直深藏于心,隐喻着一种无法言说的邪恶。但是,我印象中的南京大屠杀缺乏相关的人物细节和对人性层面的分析,而且我也很难分清哪些是传说、哪些是真实的历史。我在上小学时就曾遍寻当地的公共图书馆,试图查找南京大屠杀的相关资料,结果一无所获。这让我深感不解,如果南京大屠杀真的如此血腥,如我父母所描述的那样,是人类历史上极其野蛮的暴行之一,那么为什么没有人就此写一本书呢?当时我还小,并没有想到利用伊利诺伊大学丰富的图书馆资源继续研究,我对此事的好奇心很快就消失无踪了。
时光飞逝,将近20年后,南京大屠杀再度闯入我的生活。此时我已结婚,成为一名专业作家,在加州圣巴巴拉过着平静的生活。一天,我从事电影制作的朋友说,有几个东海岸的制片人最近完成了一部关于南京大屠杀的纪录片,但在该片发行时遇到了资金困难。
这件事再次点燃了我对南京大屠杀的兴趣,我很快与这部纪录片的两位制片人分别取得了联系,并在电话中谈论这一话题。其中一位叫邵子平,是一位美籍华裔的积极分子,曾在纽约为联合国工作,是纪念南京大屠杀受害者协会的前任会长,并曾协助制作了录像带《马吉的证言》(Magee’s Testament);另一位叫汤美如,是一位独立制片人,曾经制作并与崔明慧联合拍摄了纪录片《以天皇的名义》(In the Name of the Emperor)。邵子平和汤美如介绍我加入由一群积极分子组成的社交圈,他们多是第一代美籍或加拿大籍华人,跟我一样,都认为要在所有幸存的受害者去世之前让他们为南京大屠杀作证,将他们的证言整理并公之于世,甚至要求日本对南京大屠杀进行赔偿。还有人希望将他们对战争的记忆传递给子孙后代,以免北美文化的同化导致他们遗忘这段重要的历史。
世界各地的华人之间存在广阔而错综复杂的关系网络,一个促进南京大屠杀真相公之于世的草根运动应运而生。在华人聚集的城市中心区,如旧金山湾区、纽约、洛杉矶、多伦多以及温哥华,许多华人积极分子通过组织各种会议和开展教育活动,宣传日军在“二战”期间所犯下的罪行。他们在许多博物馆和学校播放或展出关于南京大屠杀的电影、录像带和照片,并在互联网上传播相关事实和图片,甚至在《纽约时报》之类的报纸上刊登整版广告。有些活动团体熟练运用科技手段,只需轻点鼠标,相关信息在世界范围内的受众就会超过25万人。
1994年12月,我参加了一次纪念南京大屠杀受害者的会议以后,儿时记忆中的南京大屠杀再也不是虚无缥缈的民间传说,而是真实确凿的口述历史。这次会议是由全球第二次世界大战史实维护联合会亚洲分会赞助的,在加州库比蒂诺举行,位于硅谷心脏地带圣何塞市的郊区。组织者在会议大厅展出了海报大小的南京大屠杀照片,其中许多是我平生所见最令人毛骨悚然的照片。尽管我从小就听过许多关于南京大屠杀的描述,但这些照片依然让人猝不及防,赤裸裸的黑白图像令人目不忍睹:遇难者或被斩断头颅,或被开膛破肚,赤身裸体的妇女在强暴者逼迫下摆出各种色情姿势,她们面部扭曲,表情痛苦,羞愤难当之色令人刻骨难忘。
在瞬间的晕眩之后,我突然意识到,不仅生命易逝,人类对待生命的历史经验同样不堪一击。我们从小就知道什么是死亡,我们中的任何人随时都可能被司空见惯的卡车或公交车撞倒,瞬间失去生命。除非怀有某种宗教信仰,我们会认为这种对生命的剥夺不仅毫无意义,而且是不公正的。但我们也都知道,应当尊重生命以及大多数人都会经历的死亡过程。如果你不幸被公交车撞倒在地,或许会有人趁火打劫偷走你的钱包,但一定会有更多的人出手相助,努力挽救你宝贵的生命。有人会替你拨打急救电话,有人会跑去通知辖区内的值班警察,还有人会脱下自己的外套,叠好后垫在你的头下。这样,即使这真的是你生命的最后时刻,你将在细微但真切的慰藉中安然离世,因为你知道有人在关心你。然而,挂在库比蒂诺墙上的图片却显示了这样一个事实:成千上万的生命由于他人一时的恶念闪现而陨落,第二天他们的死亡就变得毫无意义。纵然此类死亡不可避免,这依旧是人类历史上最恐怖的惨剧。更令人无法容忍的是,这些刽子手还侮辱受害者,强迫他们在死前承受最大限度的痛苦和羞辱。这种对待死亡及其过程的残忍和蔑视,这种人类社会的巨大倒退,将会被贬抑为没有价值的历史插曲,或者电脑程序中无足轻重的小差错,可能会、也可能不会再次引起任何问题。除非有人促使世界记住这段历史,否则悲剧随时可能重演。心念至此,我突然陷入巨大的恐慌。
此次会议期间,我了解到已经有两本关于南京大屠杀的小说被列入出版计划,即《天堂树》(Tree of Heaven)和《橙雾帐篷》(Tent of Orange Mist),这两本书已经于1995年出版;同时进行的还有一部关于南京大屠杀的图片集,即《南京大屠杀图片集:一段不容否认的历史》(The Rape of Nanking: An Undeniable History in Photographs),也于1996年出版。 但在当时,还没有人用英文写过关于南京大屠杀的长篇纪实类专著。深入钻研南京大屠杀的历史后,我发现写作此类著作所需的一手资料在美国一直存在,并且可供查阅。美国的传教士、记者和军官都曾以日记、电影和照片的形式记录下他们对这一事件的看法,以供后世参考。为什么没有其他美国作家或学者充分利用这些丰富的一手资料,写一本专门讨论南京大屠杀的纪实类专著或学术论文呢?
很快,对于为什么南京大屠杀在世界历史上一直得不到足够关注这一难以捉摸的谜题,我至少获得了部分答案。南京大屠杀之所以不像纳粹对犹太人的屠杀和美国对广岛的原子弹轰炸那样举世皆知,是因为受害者自己一直保持沉默。
但是,每一个答案都隐含着新的问题,我转而思考为什么这宗罪行的受害者没有愤而呼喊以求正义。如果他们确实大声疾呼过,那为什么他们所经受的苦难不曾得到承认呢?我很快发现,这一沉默背后是政治的操纵,有关各方的所作所为都导致了世人对南京大屠杀的忽视,其原因可以追溯到冷战时期。1949年,新中国成立以后,两岸政府都没有向日本索取战争赔偿(如以色列向德国索取赔偿一样)。即使是美国,面对苏联和中国的共产主义“威胁”,也在寻求昔日敌人日本的友谊和忠诚,因而也未曾再提此事。因此,冷战的紧张态势使日本得以逃脱许多其战时盟友在战后经历的严厉审讯与惩罚。
另外,日本国内的恐怖气氛压制了对南京大屠杀进行自由开放的学术讨论,进一步阻碍了世人对真相的了解。在日本,如果公开表达对中日战争的真实看法(过去如此,现在依然如此)将会威胁到自己的职业生涯,甚至有丧命的危险。(1990年,长崎市市长本岛等曾因表示日本昭和天皇应为第二次世界大战负一定责任而招致枪击,被一名枪手射中胸部,险些丧命。)日本社会弥漫的这种危险气氛使许多严肃的学者不敢去日本查阅相关档案文件,进行这一主题的研究;事实上,我在南京时曾听说中国出于人身安全方面的考虑,也不鼓励学者去日本进行相关研究。在此背景下,日本之外的人就很难接触到日本国内关于南京大屠杀的一手档案资料。除此之外,大部分曾参与过南京大屠杀的日本老兵都不愿意就他们的经历接受采访,近年来,只有极少数老兵冒着被排斥甚至死亡的威胁,将他们的经历公之于世。
在写作本书的过程中,让我感到困惑和悲哀的是,日本人自始至终顽固地拒绝承认这段历史。与德国相比,日本付出的战争赔偿还不及德国对战争受害者赔偿总额的1\%。“二战”之后,大多数纳粹分子即使没有因其罪行被囚禁,至少也被迫退出公众视野,而许多日本战犯则继续在产业和政府领域担任要职。在德国人不断向大屠杀遇难者道歉的同时,日本人则将本国战犯供奉在靖国神社——有位太平洋战争中的美国受害者认为该行径的政治含义就好比“在柏林市中心修建一座供奉希特勒的教堂”。 
在本书漫长而艰难的写作过程中,日本许多知名政客、学者和工业界领袖在如山铁证面前,仍然顽固地拒绝承认南京大屠杀这一史实,他们的这种嘴脸一直强烈激励着我。在德国,如果教师在历史课程中删除大屠杀的内容,就属违法;相比之下,几十年来,日本则系统性地将涉及南京大屠杀的内容从教科书中删除得一干二净。他们撤走博物馆中南京大屠杀的照片,篡改或销毁南京大屠杀的原始资料,避免在流行文化中提及南京大屠杀之类的字眼。甚至有些在日本深受尊崇的历史学教授也加入右翼势力,履行他们心目中的民族责任:拒绝相信南京大屠杀的报道。在《以天皇的名义》这部纪录片中,一位日本历史学家以这样的话否认整个南京大屠杀事件:“即使只有二三十人被杀害,日本方面都会极为震惊。那个时代,日本军队一直都是模范部队。”正是某些日本人这种蓄意歪曲历史的企图,使我更加确信写作本书的必要性。
除上述重要因素外,本书还想回应另外一种性质完全不同的观点。近年来,真诚地要求日本正视历史并承担相应责任的努力往往被贴上“打击日本”的标签。有一点很重要,我并不想争辩说在20世纪前1/3的时间内,日本是世界甚或亚洲唯一的帝国主义势力。中国自身也曾谋求将影响力扩及邻国,甚至曾与日本达成协议,划分双方在朝鲜半岛的势力范围,正如19世纪欧洲列强瓜分在中国的商业权益一样。
更重要的是,如果将对特定时空范围内日本人行为的批评等同于对全体日本人民的批评,这不仅是对那些在南京大屠杀中被夺去生命的男女老少的侮辱,也是对日本人民的伤害。本书无意评判日本的民族性格,也不想探究什么样的基因构造导致他们犯下如此暴行。本书要探讨的是文化的力量,这种力量既可以剥去人之为人的社会约束的单薄外衣,使人变成魔鬼,又可以强化社会规范对人的约束。今天的德国之所以比过去发展得更好,是因为犹太人不容许这个国家忘记其在“二战”期间所犯下的罪行。美国南方也发展得更好,是因为它认识到奴隶制的罪恶,并承认黑人奴隶解放之后仍然存在了100年的种族歧视和隔离也是一种罪恶。日本不仅要向世界承认,更应该自我坦白,它在“二战”期间的所作所为是多么恶劣,否则日本文化就不会向前发展。事实上,我惊喜地发现,已经有相当数量的海外日本人开始参加关于南京大屠杀的会议。正如其中一人所说:“我们同你们一样想了解更多真实的历史。”
本书将描述两种相互关联但又彼此独立的暴行:其一是南京大屠杀本身,即日本以何等残暴的手段消灭了中国一座城市数十万无辜平民;其二是对大屠杀的掩饰,即日本如何在其他国家助纣为虐的沉默中企图抹杀公众对南京大屠杀的记忆,从而剥夺了受害者在历史上应有的地位。
本书第一部分的结构在很大程度上受电影《罗生门》的影响。这是一部著名的电影,改编自日本小说家芥川龙之介的短篇小说《竹林中》,讲的是10世纪发生在日本京都的一起强奸谋杀案。表面看来,这个故事很简单:一歹徒拦路抢劫了一名过路的武士及其妻子;武士的妻子遭到强暴,武士身亡。但是随着故事中不同角色从各自的视角出发分别讲述了自己的经历之后,情节变得复杂,歹徒、武士妻子、死去的武士和一名目击者对所发生的事情提供了不同版本的描述。这样,读者就必须综合考虑每个人的回忆,辨别每个人叙述的真伪。在此过程中,透过主观、通常也是自私的描述,对已发生之事做出客观的判断。这个故事应该收入所有刑事司法课程的教材,其主旨也正好切中历史研究的核心。
本书将从三种不同的视角讲述南京大屠杀。第一种是日本人的视角。日本对中国是有计划的入侵:日本军队接到何种命令,如何执行命令,以及背后的原因是什么。第二种是中国人的视角,即受害者的视角。这是当政府再也无力保护其人民免于外敌入侵时,一座城市的命运。这其中还包括个别中国人的故事,即他们在国破家亡时遭受的挫败、绝望、背叛以及侥幸苟全的故事。第三种是欧美人士的视角。至少在中国历史的某一时刻,这些外国人曾经是英雄。在南京大屠杀期间,现场为数不多的西方人冒着生命危险拯救中国平民,并将在他们眼前发生的种种暴行告知外部世界,发出警示。本书第二部分涉及的是战后时期,我们将提到欧美各国对其经历过南京大屠杀的侨民陈述的暴行是多么无动于衷。
最后,本书将探讨半个多世纪以来,那些企图将南京大屠杀从公众意识中抹去的势力,以及近年来人们为挑战这种扭曲历史的行为所做的种种努力。
若想纠正这段被扭曲的历史,必须首先弄清楚,日本作为一个民族,当面对他们在大屠杀期间的历史记录时,如何控制、培养以及维持他们的集体失忆,甚至集体否认。他们对这段历史的处理并不是因其过于痛苦而在历史书上留下空白,事实上,日军在中日战争期间最丑陋的行为都被日本的学校教育全部删除。更有甚者,他们还将日本发动战争的责任隐藏在精心编造的神话中,即日本是第二次世界大战的受害者,而非煽动者。美国以原子弹轰炸广岛、长崎期间,日本人所体会到的恐怖更有助于这一神话取代历史真相。
直至今天,在世界舆论法庭面前,日本仍然对其战时行径执迷不悟、毫无悔意,甚至第二次世界大战结束后不久,尽管法庭判决其某些领导人有罪,日本仍旧处心积虑地逃避文明世界的道德审判;而德国则被迫接受这种审判,为自己在战争梦魇中犯下的罪行承担责任。日本人持续逃避审判,从而成为另一种罪行的元凶。正如诺贝尔和平奖得主埃利·威塞尔数年前警告的那样:忘记大屠杀就是二次屠杀。
南京大屠杀幸存者的人数正在逐年减少,趁这些历史见证者尚在人世,我最大的愿望就是本书能起到抛砖引玉的作用,激励其他作家和历史学家调查南京大屠杀幸存者的经历。或许更重要的是,我希望本书能唤醒日本人的良知,承担他们对南京大屠杀应负的责任。
在写作本书的过程中,我脑海中一直萦绕着乔治·桑塔亚纳的不朽警句:忘记过去的人注定要重蹈覆辙。

[1]  远东国际军事法庭记录,1702 号 文 件: “表格:日本人在南京大屠杀死难者人数估计。 ” 英国国家档案馆 。
[2]  吴志铿的估计。(圣何塞《麦哥里新闻》1988 年 l 月号)。
[3]  弗兰克·乔克和库尔特·乔纳森著《种族灭绝的历史与社会学:分析与案例研究》,第76 页,耶鲁大学出版社,1990。
[4]  利奥·库珀著《种族灭绝:在 20 世纪的政治用途》,第 12 页,耶香大学出版社,1991。
[5]  了解欧洲的死亡数字,可参见 R·J·拉梅尔的《中国流血的世纪:自 1900 年来的种族灭绝和大屠杀》,第 138 页,新布伦斯威克学院,1991。
[6]  路易斯·L·辛德著《路易斯·辛德二次大战历史指南》,第198 一 199 页,康涅狄格西点,1982。
[7]  参见布里格迪尔·彼德编《二次大战世界年鉴》,第 330 页,普伦蒂斯一霍尔,1981 关于广岛及长崎原子弹爆炸的死亡人数,参见理查德·罗德著《原子弹的制造》,第 734、740 页,罗德宣称,在 1945 年的原子弹爆炸中,大约有 14 万人死于广岛,7 万人死于长崎。不仅如此,因为原子弹爆炸造成的疾病死亡还在继续,在5 年后,广岛总共有 20 万人死亡,长崎有 14 万人死亡。值得注意的是,即使是在 5 年后两个城市的死亡人数之和也少于对南京暴行中死难人数的最高估计。
[8]  参见:凯瑟琳 · 罗赛尔的文章“对老兵来说,天皇的访问是赎罪 ” ,路透社, 1992 年 10 月 15 日;乔 治 · 菲奇的文章“南京浩劫”,1938 年 1 月 10 日,引自《乔治·菲奇文集》,耶鲁神学院图书馆;台湾的一位厉史学家李恩涵估计大约有 8 万名妇女被强奸或肢解。( “与关于战争罪行的国际法相关的日军南京大屠杀 ” ,《日本侵华研究),第 74 页,1951 年 5 月)
[9]  据作者对幸存者的采访。
[10]  克里斯蒂安·克勒格尔文集中未发表的日记,“南京毁灭的日子 ” ,还可参见远东国际军事法庭的判决,美国国家档案馆。
[11]  参见罗伯特·莱基著《来自地狱:第二次世界大战记实》,第 303 页,纽约,1987。
[12]  这两本书是:R·C·宾斯托克的《天堂之树》,纽约,1995 年。保罗· 韦斯特的《橙雾帐篷》。纽约,1995。另外一本是尹集钧与史咏《南京的暴行:一段无法否认的图片史》,芝加哥革新出版集团,1996。
[13]  根据作者对吉尔伯特·海尔的电话采访。

\mainmatter

\part{第一部分}

\chapter{第一章 通往南京之路}

要理解日军的所作所为,必须首先弄清楚一系列显而易见的问题。在南京大屠杀期间,究竟为什么日本士兵的行为竟然完全脱离人类基本的行为规范?为什么日本军官允许甚至鼓励这种失控行为的发生?日本政府是怎样参与其中的?日本政府对于从本国渠道获得的报告,以及来自南京大屠杀现场的外籍人士的消息,究竟有什么反应?

要回答这些问题,我们必须先从相关的日本历史谈起。

20 世纪日本人的民族特质是由一种业已存在千年的社会制度锻造出来的,在这种制度下,社会等级的确立和维持是通过军事斗争实现的。千百年来,日本列岛上强大的封建诸侯雇用私人军队,彼此征战不息;到了中世纪,这些军队逐渐演变为日本社会独特的武士阶层,他们的行为规范被称为武士道(即“武士的行为规范”)。为主人效忠而死是武士一生中至高无上的荣誉。\footnote{text}

当然,这种荣誉规范绝非日本文化首创。古罗马诗人贺拉斯最先指出,每个时代的年轻人对其统治者应尽的义务是:为国捐躯,无上光荣。但是,日本的武士哲学更进一步,对军事义务的界定远远超过了正当和适宜的程度。日本武士的行为规范极为严苛,其最显著的特征是道义上的强制性,即如果没能光荣完成军事任务,就要自杀谢罪:通常情况下武士要在多个证人面前实施高度程式化而又极端痛苦的剖腹仪式,大无畏地自杀身亡。
到了12世纪,在征战中获胜的家族(因此也是最有权势的家族)首领成为幕府将军,他雇用武士向天照女神的直系后裔(广受尊崇的天皇)提供军事保护,作为交换,武士阶层获得了整个统治阶级的神圣认可。随着时间的推移,最初只有少数人遵循的武士行为规范逐渐渗透到日本文化中,成为所有年轻男子尊崇的行为典范。
时间的流逝并没有削弱武士道的精神力量,这种精神在18世纪开始崭露头角,并在现代战争实践中趋于极致。第二次世界大战期间,臭名昭著的神风突击队执行自杀式攻击任务,受过正规训练的日军飞行员驾驶飞机直接撞向美国战舰,日本青年这种誓死效忠天皇、随时准备献身的行为给西方留下了极其深刻的印象。而且并非只有少数精英团体拥有这种宁死不降的信念,人们惊讶地发现,盟军投降与战死的比例是1∶3,而日军的这一比例则是1∶120。\footnote{text}






第二章 六周暴行
向南京进发
朝香宫鸠彦接掌兵权
屠杀战俘
屠杀平民
日本记者
强暴妇女
松井石根到来
慰安妇
南京暴行背后的动机
第三章 南京沦陷

\chapter{第四章 恐怖的六星期}

日军攻破南京城门之前,那些稍微有点儿财力、权力或先见之明的居民都已闻风而逃,不知所踪。大约一半居民离开了南京:战争爆发之前,南京的本地人口在100万以上;到12月,这一数字已经下降到50万。\footnote{text}但与此同时,却有数万名乡村移民涌入南京,他们背井离乡来到这里,相信南京在城墙的庇护下是安全的。中国军队撤离后,留在南京的是那些毫无防卫能力的人:老人,儿童以及经济赤贫或身体虚弱到无法安全出逃的人。

这些人失去了政府保护,没有任何个人资源,更看不到未来的出路,只能寄希望于日本人善待他们。许多人告诉自己,一旦战事结束,日本人一定会对他们以礼相待。有些人甚至相信日本人会是更好的统治者,毕竟在他们最需要帮助时,中国政府抛弃了他们。由于厌倦了战火、厌倦了遭受轰炸和围困,当日军带着坦克、大炮和卡车轰隆隆地开进南京城时,确实有零星的几群中国人跑出去欢迎侵略者。当日军从南门和西门列队进城时,有些人在自家窗口上挂起了日本国旗,甚至还有人高声欢呼。\footnote{text}

然而,对日军的欢迎是短暂的。目击者后来描述说,日军进城后不久,就开始以6~12人为一伙在城里四处游荡,见人就杀。有老人面部朝下倒在人行道上,显然是日军一时兴起从他背后开枪所致;几乎每一条街区都散布着中国平民的尸体,许多人除了见到日军撒腿就跑外,并无任何冒犯之处。\footnote{text}

根据军事法庭的战争犯罪记录和中国政府的档案文献,接下来发生的事情尽管都极尽恐怖,却大都千篇一律。所有描述几乎毫无例外,大致如下。

日军通常把他们抓到的男人都当作囚犯,一连几天不给饭吃、不给水喝,但许诺会给他们食物和工作。经过几天这样的折磨之后,日军就用铁丝或绳子将他们的手腕牢牢捆住,然后把他们驱赶到偏僻处。这些人都已太过疲乏或严重脱水,因此急切地走出去,还以为马上就能得到食物和水。然而,他们看到的却是机关枪,或者手持血淋淋的军刀或刺刀等在一边的日本士兵,或者掩埋尸体的巨大坟坑,坑里堆满了先于他们被杀害的同胞的尸体,阵阵尸臭飘散出来,此时再想逃跑为时已晚。

日本人后来为证明自己行为的正当性,曾经辩护说:他们是为了节约有限的粮食,防止暴动,不得已才处死战俘的。但是,任何理由都无法开脱日军对数十万无助的南京平民所犯下的滔天罪行。因为他们手无寸铁,根本不可能造反。

当然,并非所有在南京的中国人都轻易屈服于日军妄图斩尽杀绝的屠刀。南京大屠杀不仅是一个大规模的牺牲事件,其中也展现了个体的力量和勇气。靠着强烈的求生意志,有人徒手挖开埋葬自己的坟坑逃出来,有人紧紧抓住芦苇在刺骨的江水中隐蔽好几个小时,有人甚至一连数日被压在朋友的尸体下,最后才设法脱身,拖着弹痕累累的身体去医院就医。还有些妇女躲在洞穴或壕沟内长达数星期,或者冲进大火熊熊燃烧的房屋去救自己的孩子。

后来,许多幸存者向记者和历史学家讲述了自己的经历,或是在日本战败后到南京和东京的军事法庭上作证。1995年夏天,我采访了一些幸存者之后了解到,日军杀害许多中国人显然并无其他原因,只是为了取乐。这就是80多岁的南京市民唐顺山的看法。1937年,他奇迹般地从日军的一次杀人比赛中侥幸逃生。

\section{杀人比赛}

日军捆住年轻男子的手腕,用卡车将他们运到南京郊外集中处决。(日本报纸《每日新闻》)

1937年12月16日,17个日本宪兵军官正在检查一大群南京城的中国平民,这些平民被日军的大屠杀吓坏了,没有人敢进行任何反抗。(台北新闻社供)

该图片展示了南京陷落之后,日本新兵以中国俘虏作为活靶进行刺刀训练。在画面中央,一位不幸的俘虏(或许我们应该说是幸运)已经受到致命一击。在画面前端,一个日本士兵正用刺刀“轻轻地”地刺着一个被捆绑的中国人,以寻找合适位置对他进行致命一击。就该照片的真实性来说——这张照片是汉口的W·A·法默送给《展望》杂志的,据他讲这张照片是一个日本士兵拍摄的。照片的胶片被送到上海冲洗,一家日本独资店的中国雇员多洗了几张,并把它们偷偷运了出来。(合众社的贝特曼供)

这个可怜的人被蒙住眼睛靠在两根棍子上,充当日军练习刺刀的活靶子。画面中的士兵正在练习刺杀动作,甚至受害者死亡后还要对其猛刺。(国民党军事委员会政治局供,中国台北)



折磨
强奸
死亡人数
第五章 南京安全区
中国的辛德勒
克里斯蒂安·克勒格尔(左上),德国工程师,安全区国际委员会的纳粹成员,1937~1938年管理安全区的财务。(彼得·克勒格尔供)
约翰·马吉(右上),圣公会牧师,在南京大屠杀期间任国际红十字会南京分会主席。作为业余电影摄制者,马吉录下了金陵大学医院的很多重要影像。(耶鲁大学神学院图书馆供)
南京唯一的外科医生
南京的活菩萨

\part{第二部分}

第六章 世人所了解的南京大屠杀
美国记者
新闻短片制作人
日本人的危机控制
关于南京大屠杀的外国情报
日方的宣传
安全区负责人的反击
第七章 日本占领下的南京
第八章 审判日
南京战争罪行审判
远东国际军事法庭
第九章 幸存者的命运
幸存者和战争索赔
安全区领导人后来的遭遇
离开南京的拉贝
约翰·拉贝关于南京大屠杀的日记中的一页。(约翰·拉贝的资料集,耶鲁大学神学院图书馆供)
约翰·拉贝写给希特勒的信。拉贝将这封信连同一份报告和一部关于南京大屠杀的影片呈递给希特勒。几天后,拉贝就遭到逮捕并在柏林受到盖世太保的审讯。(约翰·拉贝的资料集,耶鲁大学神学院图书馆供)
《拉贝日记》的面世
第十章 被遗忘的大屠杀:再次凌辱
教科书的争议
 
学术界的掩饰
媒体的自我审查
关于南京大屠杀的争论
恐吓
结语
尾声
致谢

\backmatter



参考文献




\end{document}