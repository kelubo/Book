\documentclass{article}
\usepackage{ctex}
\usepackage{graphicx}
\usepackage{amsmath}
\usepackage{amssymb}
\usepackage{geometry}
\usepackage{multirow}
\usepackage{booktabs}
\usepackage{listings}
\usepackage{color}

\geometry{a4paper, margin=1in}
\title{慰安妇历史真相与正义}
\author{}
\date{}

\begin{document}

\maketitle

\tableofcontents

\section{历史背景与制度形成}

\subsection{战前日本的性政策}

\subsection{慰安妇制度的起源}

\subsection{慰安妇制度的发展与扩大}

\subsection{慰安所的管理与运营}

\section{受害者群体与遭遇}

\subsection{受害者的来源与构成}

\subsection{受害者的悲惨遭遇}

\subsection{受害者的身心健康影响}

\subsection{受害者的幸存者经历}

\section{历史证据与研究}

\subsection{档案资料与文献证据}

\subsection{口述历史与幸存者证词}

\subsection{学术研究与历史调查}

\subsection{国际社会的历史认定}

\section{战后追责与正义运动}

\subsection{受害者的维权运动}

\subsection{民间组织的努力}

\subsection{国际社会的关注与支持}

\subsection{法律诉讼与司法进程}

\section{日本政府的态度与责任}

\subsection{战后日本政府的立场演变}

\subsection{历史修正主义的挑战}

\subsection{国际社会的批评与要求}

\subsection{责任承担与赔偿问题}

\section{教育与记忆传承}

\subsection{历史教育的重要性}

\subsection{记忆场所的建立与保护}

\subsection{国际合作与交流}

\subsection{防止历史重演的努力}

\section{幸存者的尊严与福祉}

\subsection{幸存者的晚年生活}

\subsection{医疗与心理支持}

\subsection{社会认可与尊严恢复}

\subsection{幸存者的声音与诉求}

\section{国际社会的反应与行动}

\subsection{联合国等国际组织的立场}

\subsection{各国政府的态度与政策}

\subsection{国际人权运动的参与}

\subsection{跨国合作与倡议}

\section{历史真相的传播与认知}

\subsection{媒体的角色与责任}

\subsection{公众认知的现状与挑战}

\subsection{数字时代的历史传播}

\subsection{青年一代的历史教育}

\section{面向未来的思考}

\subsection{历史正义的实现路径}

\subsection{和解与和平的可能性}

\subsection{人权保护的全球意义}

\subsection{构建包容性的历史记忆}

\end{document}