% 狂人日记
% 狂人日记.tex

\documentclass[12pt,UTF8]{ctexbook}

% 设置纸张信息。
\usepackage[a4paper,twoside]{geometry}
\geometry{
	left=25mm,
	right=25mm,
	bottom=25.4mm,
	bindingoffset=10mm
}

% 设置字体,并解决显示难检字问题。
\xeCJKsetup{AutoFallBack=true}
\setCJKmainfont{SimSun}[BoldFont=SimHei, ItalicFont=KaiTi, FallBack=SimSun-ExtB]

% 目录 chapter 级别加点(.)。
\usepackage{titletoc}
\titlecontents{chapter}[0pt]{\vspace{3mm}\bf\addvspace{2pt}\filright}{\contentspush{\thecontentslabel\hspace{0.8em}}}{}{\titlerule*[8pt]{.}\contentspage}

% 设置 part 和 chapter 标题格式。
\ctexset{
	chapter/name={},
	chapter/number={},
	section/name={},
	section/number={}
}

% 设置署名格式。
\newenvironment{shuming}{\hfill}

% 注脚每页重新编号,避免编号过大。
\usepackage[perpage]{footmisc}

\title{\heiti\zihao{0} 狂人日记}
\author{鲁迅}
\date{}

\begin{document}

\maketitle
\tableofcontents

\frontmatter
\chapter{前言}

本篇最初发表于1918年5月《新青年》第四卷第五号。首次采用“鲁迅”这一笔名。作者在《<中国新文学大系>小说集序》(《且介亭杂文二集》)中说本篇“意在暴露家族制度和礼教的弊害”。

\mainmatter

\chapter{狂人日记}

某君昆仲,今隐其名,皆余昔日在中学校时良友;分隔多年,消息渐阙。日前偶闻其一大病;适归故乡,迂道往访,则仅晤一人,言病者其弟也。劳君远道来视,然已早愈,赴某地候补\footnote{清代官制,只有官衔而没有实际职务的中下级官员,由吏部抽签分发到某部或某省,听候委用,称为候补。}矣。因大笑,出示日记二册,谓可见当日病状,不妨献诸旧友。持归阅一过,知所患盖“迫害狂”之类。语颇错杂无伦次,又多荒唐之言;亦不著月日,惟墨色字体不一,知非一时所书。间亦有略具联络者,今撮录一篇,以供医家研究。记中语误,一字不易;惟人名虽皆村人,不为世间所知,无关大体,然亦悉易去。至于书名,则本人愈后所题,不复改也。七年四月二日识。

\section{一}

今天晚上,很好的月光。

我不见他,已是三十多年;今天见了,精神分外爽快。才知道以前的三十多年,全是发昏;然而须十分小心。不然,那赵家的狗,何以看我两眼呢?

我怕得有理。

\section{二}

今天全没月光,我知道不妙。早上小心出门,赵贵翁的眼色便怪:似乎怕我,似乎想害我。还有七八个人,交头接耳的议论我,又怕我看见。一路上的人,都是如此。其中最凶的一个人,张着嘴,对我笑了一笑;我便从头直冷到脚跟,晓得他们布置,都已妥当了。

我可不怕,仍旧走我的路。前面一伙小孩子,也在那里议论我;眼色也同赵贵翁一样,脸色也都铁青。我想我同小孩子有什么仇,他也这样。忍不住大声说,“你告诉我!”他们可就跑了。

我想:我同赵贵翁有什么仇,同路上的人又有什么仇;只有廿年以前,把古久先生的陈年流水簿子\footnote{这里比喻中国封建统治的长久历史。},踹了一脚,古久先生很不高兴。赵贵翁虽然不认识他,一定也听到风声,代抱不平;约定路上的人,同我作冤对。但是小孩子呢?那时候,他们还没有出世,何以今天也睁着怪眼睛,似乎怕我,似乎想害我。这真教我怕,教我纳罕而且伤心。

我明白了。这是他们娘老子教的!

\section{三}

晚上总是睡不着。凡事须得研究,才会明白。

他们——也有给知县打枷过的,也有给绅士掌过嘴的,也有衙役占了他妻子的,也有老子娘被债主逼死的;他们那时候的脸色,全没有昨天这么怕,也没有这么凶。

最奇怪的是昨天街上的那个女人,打他儿子,嘴里说道,“老子呀!我要咬你几口才出气!”他眼睛却看着我。我出了一惊,遮掩不住;那青面獠牙的一伙人,便都哄笑起来。陈老五赶上前,硬把我拖回家中了。

拖我回家,家里的人都装作不认识我;他们的脸色,也全同别人一样。进了书房,便反扣上门,宛然是关了一只鸡鸭。这一件事,越教我猜不出底细。

前几天,狼子村的佃户来告荒,对我大哥说,他们村里的一个大恶人,给大家打死了;几个人便挖出他的心肝来,用油煎炒了吃,可以壮壮胆子。我插了一句嘴,佃户和大哥便都看我几眼。今天才晓得他们的眼光,全同外面的那伙人一模一样。

想起来,我从顶上直冷到脚跟。

他们会吃人,就未必不会吃我。

你看那女人“咬你几口”的话,和一伙青面獠牙人的笑,和前天佃户的话,明明是暗号。我看出他话中全是毒,笑中全是刀。他们的牙齿,全是白厉厉的排着,这就是吃人的家伙。

照我自己想,虽然不是恶人,自从踹了古家的簿子,可就难说了。他们似乎别有心思,我全猜不出。况且他们一翻脸,便说人是恶人。我还记得大哥教我做论,无论怎样好人,翻他几句,他便打上几个圈;原谅坏人几句,他便说“翻天妙手,与众不同”。我那里猜得到他们的心思,究竟怎样;况且是要吃的时候。

凡事总须研究,才会明白。古来时常吃人,我也还记得,可是不甚清楚。我翻开历史一查,这历史没有年代,歪歪斜斜的每叶上都写着“仁义道德”几个字。我横竖睡不着,仔细看了半夜,才从字缝里看出字来,满本都写着两个字是“吃人”!

书上写着这许多字,佃户说了这许多话,却都笑吟吟的睁着怪眼睛看我。

我也是人,他们想要吃我了!

\section{四}

早上,我静坐了一会。陈老五送进饭来,一碗菜,一碗蒸鱼;这鱼的眼睛,白而且硬,张着嘴,同那一伙想吃人的人一样。吃了几筷,滑溜溜的不知是鱼是人,便把他兜肚连肠的吐出。

我说“老五,对大哥说,我闷得慌,想到园里走走。”老五不答应,走了;停一会,可就来开了门。

我也不动,研究他们如何摆布我;知道他们一定不肯放松。果然!我大哥引了一个老头子,慢慢走来;他满眼凶光,怕我看出,只是低头向着地,从眼镜横边暗暗看我。大哥说,“今天你仿佛很好。”我说“是的。”大哥说,“今天请何先生来,给你诊一诊。”我说“可以!”其实我岂不知道这老头子是刽子手扮的!无非借了看脉这名目,揣一揣肥瘠:因这功劳,也分一片肉吃。我也不怕;虽然不吃人,胆子却比他们还壮。伸出两个拳头,看他如何下手。老头子坐着,闭了眼睛,摸了好一会,呆了好一会;便张开他鬼眼睛说,“不要乱想。静静的养几天,就好了。”

不要乱想,静静的养!养肥了,他们是自然可以多吃;我有什么好处,怎么会“好了”?他们这群人,又想吃人,又是鬼鬼祟祟,想法子遮掩,不敢直捷下手,真要令我笑死。我忍不住,便放声大笑起来,十分快活。自己晓得这笑声里面,有的是义勇和正气。老头子和大哥,都失了色,被我这勇气正气镇压住了。

但是我有勇气,他们便越想吃我,沾光一点这勇气。老头子跨出门,走不多远,便低声对大哥说道,“赶紧吃罢!”大哥点点头。原来也有你!这一件大发见,虽似意外,也在意中:合伙吃我的人,便是我的哥哥!

吃人的是我哥哥!

我是吃人的人的兄弟!

我自己被人吃了,可仍然是吃人的人的兄弟!

\section{五}

这几天是退一步想:假使那老头子不是刽子手扮的,真是医生,也仍然是吃人的人。他们的祖师李时珍做的“本草什么\footnote{指明代李时珍的药物学著作《本草纲目》。该书曾经提到唐代陈藏器《本草拾遗》中以人肉医治痨病的记载,并表示了异议。这里说李时珍的书“明明写着人肉可以煎吃”,当是“狂人”的“记中语误”。}”上,明明写着人肉可以煎吃;他还能说自己不吃人么?

至于我家大哥,也毫不冤枉他。他对我讲书的时候,亲口说过可以“易子而食\footnote{语出《左传》宣公十五年,是宋将华元对楚将子反叙说宋国都城被楚军围困时的惨状:“敝邑易子而食,析骸以爨。”}”;又一回偶然议论起一个不好的人,他便说不但该杀,还当“食肉寝皮\footnote{语出《左传》襄公二十一年,晋国州绰对齐庄公说:“然二子者,譬于禽兽,臣食其肉而寝处其皮矣。”按“二子”指齐国的殖绰和郭最,他们曾被州绰俘虏过。}”。我那时年纪还小,心跳了好半天。前天狼子村佃户来说吃心肝的事,他也毫不奇怪,不住的点头。可见心思是同从前一样狠。既然可以“易子而食”,便什么都易得,什么人都吃得。我从前单听他讲道理,也胡涂过去;现在晓得他讲道理的时候,不但唇边还抹着人油,而且心里满装着吃人的意思。

\section{六}

黑漆漆的,不知是日是夜。赵家的狗又叫起来了。

狮子似的凶心,兔子的怯弱,狐狸的狡猾,……

\section{七}

我晓得他们的方法,直捷杀了,是不肯的,而且也不敢,怕有祸祟。所以他们大家连络,布满了罗网,逼我自戕。试看前几天街上男女的样子,和这几天我大哥的作为,便足可悟出八九分了。最好是解下腰带,挂在梁上,自己紧紧勒死;他们没有杀人的罪名,又偿了心愿,自然都欢天喜地的发出一种呜呜咽咽的笑声。否则惊吓忧愁死了,虽则略瘦,也还可以首肯几下。

他们是只会吃死肉的!——记得什么书上说,有一种东西,叫“海乙那\footnote{英语Hyena的音译,即鬣狗,产于非洲、小亚细亚及亚洲西南部,一种食肉兽,常跟在狮虎等猛兽之后,以它们吃剩的兽类的残尸为食。}”的,眼光和样子都很难看;时常吃死肉,连极大的骨头,都细细嚼烂,咽下肚子去,想起来也教人害怕。“海乙那”是狼的亲眷,狼是狗的本家。前天赵家的狗,看我几眼,可见他也同谋,早已接洽。老头子眼看着地,岂能瞒得我过。

最可怜的是我的大哥,他也是人,何以毫不害怕;而且合伙吃我呢?还是历来惯了,不以为非呢?还是丧了良心,明知故犯呢?

我诅咒吃人的人,先从他起头;要劝转吃人的人,也先从他下手。

\section{八}

其实这种道理,到了现在,他们也该早已懂得,……

忽然来了一个人;年纪不过二十左右,相貌是不很看得清楚,满面笑容,对了我点头,他的笑也不像真笑。我便问他,“吃人的事,对么?”他仍然笑着说,“不是荒年,怎么会吃人。”我立刻就晓得,他也是一伙,喜欢吃人的;便自勇气百倍,偏要问他。

“对么?”

“这等事问他什么。你真会……说笑话。……今天天气很好。”

天气是好,月色也很亮了。可是我要问你,“对么?”

他不以为然了。含含胡胡的答道,“不……”

“不对?他们何以竟吃?!”

“没有的事……”

“没有的事?狼子村现吃;还有书上都写着,通红斩新!”

他便变了脸,铁一般青。睁着眼说,“有许有的,这是从来如此……”

“从来如此,便对么?”

“我不同你讲这些道理;总之你不该说,你说便是你错!”

我直跳起来,张开眼,这人便不见了。全身出了一大片汗。他的年纪,比我大哥小得远,居然也是一伙;这一定是他娘老子先教的。还怕已经教给他儿子了;所以连小孩子,也都恶狠狠的看我。

\section{九}

自己想吃人,又怕被别人吃了,都用着疑心极深的眼光,面面相觑。……

去了这心思,放心做事走路吃饭睡觉,何等舒服。这只是一条门槛,一个关头。他们可是父子兄弟夫妇朋友师生仇敌和各不相识的人,都结成一伙,互相劝勉,互相牵掣,死也不肯跨过这一步。

\section{十}

大清早,去寻我大哥;他立在堂门外看天,我便走到他背后,拦住门,格外沉静,格外和气的对他说,

“大哥,我有话告诉你。”

“你说就是,”他赶紧回过脸来,点点头。

“我只有几句话,可是说不出来。大哥,大约当初野蛮的人,都吃过一点人。后来因为心思不同,有的不吃人了,一味要好,便变了人,变了真的人。有的却还吃,——也同虫子一样,有的变了鱼鸟猴子,一直变到人。有的不要好,至今还是虫子。这吃人的人比不吃人的人,何等惭愧。怕比虫子的惭愧猴子,还差得很远很远。”

“易牙\footnote{春秋时齐国人,齐桓公宠臣,善于调味。据《管子·小称》:“夫易牙以调和事公(按指齐桓公),公曰‘惟蒸婴儿之未尝’,于是蒸其首子而献之公。”桀、纣各为我国夏朝和商朝的最后一代君主,易牙和他们不是同时代人。这里说的“易牙蒸了他儿子,给桀纣吃”,也是“狂人”“语颇错杂无伦次”的表现。}蒸了他儿子,给桀纣吃,还是一直从前的事。谁晓得从盘古开辟天地以后,一直吃到易牙的儿子;从易牙的儿子,一直吃到徐锡林\footnote{隐指徐锡麟(1873一1907),字伯荪,浙江绍兴人,清末革命团体光复会的重要成员。1907年与秋瑾准备在浙、皖两省同时起义,7月6日,他以安徽巡警处会办兼巡警学堂监督身份为掩护,乘学堂举行毕业典礼之机刺死安徽巡抚恩铭,率领学生攻占军械局,弹尽被捕,当日惨遭杀害,心肝被恩铭的卫队挖出炒食。};从徐锡林,又一直吃到狼子村捉住的人。去年城里杀了犯人,还有一个生痨病的人,用馒头蘸血舐。”

“他们要吃我,你一个人,原也无法可想;然而又何必去入伙。吃人的人,什么事做不出;他们会吃我,也会吃你,一伙里面,也会自吃。但只要转一步,只要立刻改了,也就人人太平。虽然从来如此,我们今天也可以格外要好,说是不能!大哥,我相信你能说,前天佃户要减租,你说过不能。”

当初,他还只是冷笑,随后眼光便凶狠起来,一到说破他们的隐情,那就满脸都变成青色了。大门外立着一伙人,赵贵翁和他的狗,也在里面,都探头探脑的挨进来。有的是看不出面貌,似乎用布蒙着;有的是仍旧青面獠牙,抿着嘴笑。我认识他们是一伙,都是吃人的人。可是也晓得他们心思很不一样,一种是以为从来如此,应该吃的;一种是知道不该吃,可是仍然要吃,又怕别人说破他,所以听了我的话,越发气愤不过,可是抿着嘴冷笑。

这时候,大哥也忽然显出凶相,高声喝道,

“都出去!疯子有什么好看!”

这时候,我又懂得一件他们的巧妙了。他们岂但不肯改,而且早已布置;预备下一个疯子的名目罩上我。将来吃了,不但太平无事,怕还会有人见情。佃户说的大家吃了一个恶人,正是这方法。这是他们的老谱!

陈老五也气愤愤的直走进来。如何按得住我的口,我偏要对这伙人说,

“你们可以改了,从真心改起!要晓得将来容不得吃人的人,活在世上。”

“你们要不改,自己也会吃尽。即使生得多,也会给真的人除灭了,同猎人打完狼子一样!——同虫子一样!”

那一伙人,都被陈老五赶走了。大哥也不知那里去了。陈老五劝我回屋子里去。屋里面全是黑沉沉的。横梁和椽子都在头上发抖;抖了一会,就大起来,堆在我身上。

万分沉重,动弹不得;他的意思是要我死。我晓得他的沉重是假的,便挣扎出来,出了一身汗。可是偏要说,

“你们立刻改了,从真心改起!你们要晓得将来是容不得吃人的人,……”

\section{十一}

太阳也不出,门也不开,日日是两顿饭。

我捏起筷子,便想起我大哥;晓得妹子死掉的缘故,也全在他。那时我妹子才五岁,可爱可怜的样子,还在眼前。母亲哭个不住,他却劝母亲不要哭;大约因为自己吃了,哭起来不免有点过意不去。如果还能过意不去,……

妹子是被大哥吃了,母亲知道没有,我可不得而知。

母亲想也知道;不过哭的时候,却并没有说明,大约也以为应当的了。记得我四五岁时,坐在堂前乘凉,大哥说爷娘生病,做儿子的须割下一片肉来,煮熟了请他吃\footnote{指“割股疗亲”。古代统治者提倡的一种忠孝德行。《庄子·盗跖》篇载有:“介子推至忠也,自割其股以食(晋)文公。”行孝中的“割股疗亲”,是割取自己的股肉为药引煎药,以医治父母的重病。},才算好人;母亲也没有说不行。一片吃得,整个的自然也吃得。但是那天的哭法,现在想起来,实在还教人伤心,这真是奇极的事!

\section{十二}

不能想了。

四千年来时时吃人的地方,今天才明白,我也在其中混了多年;大哥正管着家务,妹子恰恰死了,他未必不和在饭菜里,暗暗给我们吃。

我未必无意之中,不吃了我妹子的几片肉,现在也轮到我自己,……

有了四千年吃人履历的我,当初虽然不知道,现在明白,难见真的人!

\section{十三}

没有吃过人的孩子,或者还有?

救救孩子……
\\

\begin{shuming}
一九一八年四月。
\end{shuming}

\backmatter

\end{document}