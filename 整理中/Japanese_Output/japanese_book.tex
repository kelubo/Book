
\documentclass[12pt,a4paper]{article}
\usepackage{fontspec}
\usepackage{xeCJK}
\setmainfont{Noto Serif CJK JP}
\setCJKmainfont{Noto Serif CJK JP}
\setCJKsansfont{Noto Sans CJK JP}
\usepackage{geometry}
\geometry{margin=2cm}
\usepackage{hyperref}

\title{SM奴隷から学ぶ「うつ」の治し方}
\author{紗弥}

\begin{document}

\maketitle
\tableofcontents

\section{text00000}
まえがき まえがき 本書を手に取ってくださり ありがとうございます。 「ふてぶてしく美しく生きよう。」 先に申し上げておくが、私はカウンセラーでも医療従事者でもない。 また本書は、精神分析の本でもなければ、医療の本でもない。 メンタルが脆弱で引っ込み思案だった、元SM女王様の書いた人生哲学である。 少々過激なタイトルだが、これは「薬なしで治せます」と謳うものではない。 私はこの本を、人生に彷徨う全ての方たちに捧げようと思う。 SMと聞くと多くの方は 「ムチとかロウソクとか使うんでしょ?」 「痛いのは嫌」 「そういう趣味はない」 といった反応を示し、イメージだけが独り歩きしている。 実は、日常生活こそSMの世界に通じており それを意識するかどうかによって、ものの見方、捉え方は大きく異なる。 痛み成分にも ・言葉による痛み ・精神的痛み ・肉体的痛み があるが、ここでは私が調教してきた奴隷たちから学んだこと、 辛い、苦しい、不安・・・の飼いならし方を露出しようと思う。 そもそも、なぜ世のドMたちはお金を払って「痛み」や「苦しみ」を味わいたいのか? それを知ることによって、これまで見えなかったものが、M字開脚のドMの菊門のようによく見えるだろう。 じつはSMの世界を知ることこそが、人生を快楽にする秘訣だったのだ。 本書を読んでいて不快に感じる方もいらっしゃるかもしれない。 だが、お付き合い頂けるなら、鞭よりも女王様の手の温もりを感じるスパンキングということで 楽しんで頂ければ幸いだ。 人間関係、恋愛、仕事、お金、自信、自己肯定感・・・ 人生を甘美にする秘訣は、ノウハウやテクニックではない。 「女王様目線を持つこと」だったのだ。 「人として弱いということは、生きていく上で受けるべき苦しみを 自分で受け取ろうとしないことだ。 ウィトゲンシュタイン」 では、これまで鋼鉄の扉に閉ざされていた 痛みと快楽の世界へご案内しよう。

\section{text00001}
目次 目次 まえがき 第1章 「目隠しを外せ」 目隠しと垂れ流し 変態のプライド あなたはS?それともM? 生ぬるいのは浣腸だけでいい ドM≠マグロ 女尊男卑 女王様は非国民 第2章 「女王様ストーリー」 赤ちゃんプレイ 誰にも言えなかった「いじめ」 ストーカー事件 鬱と不眠症 「そうだ、女王様になろう」 死への憧れ 女王様の個人授業 薄っぺらい輩 SMクラブはどんな人が来るのか? アナル調教のモンスター SMとSEXは次元が違う 流血プレイ 女王様引退 恋人の突然死 露出しないのは罪 第3章 「自己啓発なんて役立たず」 SMは言葉の世界 言葉は嘘の世界 裸の王様的露出狂 自己啓発よりも「自己調教」 SMクラブで見てきたもの、それは人生哲学だった 第4章 「痛み・苦しみを飼いならせ」 汚されたい願望 自分の尻にスパンキングしろ 自分に"ご褒美”という発想は捨てろ 自己調教のすすめ まとめ あとがき

\section{text00002}
第1章 「SMなんてものはない」

\section{text00003}
第1章 「目隠しを外せ」 第1章 目隠しと垂れ流し 非常識が常識にもなり得るこの世の中。 そもそも常識、非常識なんて誰が決めるのだろうか。 ただ単に、その流行に躍らされ、本質が見えていない人たちがいる。 ここではその者たちを、「目隠し奴隷」と呼ぼう。 また、世の中に出回っている成功哲学をいくら学んでも、その本質を分析する力がなければ 流れ出て行ってしまう。それは所謂、体中の汁を垂れ流す、「垂れ流し奴隷」だ。 私はそんな「目隠し奴隷」と「垂れ流し奴隷」たちに向けてこれを書いている。 SM女王様のお仕事は、世間から見れば非常識だ。 踏みつけたりビンタしたり、罵ったり。それでいて高い報酬を頂く。 そんな世間とはかけ離れた世界に身を置いていた者として言えることは、 逆走することで見えてくる景色がある、ということだ。 私は、SMクラブでの経験を現実社会に応用し、悩める人たちの力になりたいと思っている。 いや、正しくは 人間の悩みなんて自ら勝手に作っているだけだ。 悩みの種という実体のないものは顔面騎乗で圧迫して潰せばいい。 だが同情はしない。なぜならこれは調教だからだ。 調教に同情なんて要らないし、嫌になったらいつでも逃げればいい。 ただし、調教についてくることが出来た可愛い奴隷だけが、ご褒美をもらえるのだ。 それは私からではなく、他の誰かからかもしれない。 変態のプライド SMは必ずしも道具を使うわけではない。 それは想像と創造の世界、即興劇であり、何が起こるかわからないのが醍醐味である世界だ。 女王様たちは常日頃から「変態」と呼ばれる人種たちと向き合っているが それ故に、この定義が曖昧である「変態」とは何なのかと突き詰めると、 「そんなものはないが、そういう設定」という結論になる。 全ては人間の持つ本能から派生するもの であり、吸血鬼や殺人鬼もその延長にある。 それはさておき、日頃女王様たちは「変態」と呼ばれることが名誉である「変態」に対して、 愛着をこめて呼んで差し上げている。 ただ、「男はみんな変態だよ」などと、意味不明なことを偉そうに言う輩を見ていると 何でもかんでも軽々しく変態と呼ぶのは、変態に対して失礼である、と私は考える。 なぜならSMクラブに通う変態たちは、全身全霊で変態をやっているからだ。 彼らは男(女)としてのプライドを捨て、それに代わって奴隷としてのプライドが芽生える。 じつは女王様は、そんなプライドを持った奴隷に対して、敬意を払ってイジメているのだ。 あなたはS?それともM? 「あなたはS?M?どっち?」 なんて質問する光景をよく見かける。 恐らく、ほとんどの人たちはよくわかっていないのだろう。 あるいは、自称S,自称Mもいる。 本当のことを教えよう。 それは相手が何を引き出すかによるのであり、一人で成り立つものではないのだ。 自分という存在は外部によって作られるものであり、 この世に自分一人しか居ないとしたら、比較の対象がないため 「優しい人」「存在感のある人」とは言えないだろう。 人間の存在価値というものは、相手がいて初めて証明できるのである。 よって、単独で「私はSです」「僕はMです」という位置づけはあり得ない。 もっと言えば、「誰でも両方持っている」とも言えるし、「SもMもない」とも言えるのである。 何でも白黒つけたがる人、本当の自分探しをしている人、そこに拘る人は、 本書を読んでから出直してきなさい。 生ぬるいのは浣腸だけでいい 私は伝えたいことを伝えるために極端なことも言う。 そんな発言をしていると、 「それは言い過ぎでは?」などという批判もある。 だが、生ぬるいのは浣腸だけでいい。 生ぬるい言葉は浣腸と一緒に下の口から便器に流されてゆくだけだ。 この世の中は、極端に発言しなければ伝わらない。 以前、「あなたの考えは間違っている!」 といったメッセージを頂いたこともあるが、 「人を変えようとして苦しんでいるのは、おまえだよ」 と鼻で笑ってしまった。 世の中は絶対的な「正しい」「間違い」なんてない。 人間がブタを殺すのはいいのだろうか? 結婚しないのは親不孝なのだろうか? 親の介護をしないのは罪なのか? この世には絶対的に正しいことなんて存在しない。 長い縄に巻かれてそこに拘るような汚物的発想を持つ者は、 人間便器になればいい。 トイレの神様になれるよ。 ドM≠マグロ 「愛の反対は憎しみではなく無関心 マザー・テレサ」 同じく 「ドMの反対はドSではなく無反応」 である。 無反応のドMなどは調教するに値しない。 なぜなら、女王様を動かしているのもまた、ドMだからである。 無反応のくせに、あるいは欲深くご褒美だけをくれくれとせがむのは 独りよがりの「エゴマゾ」と呼ばれ女王様から嫌われる。 変色能力のない役立たずのカメレオンだ。 ご褒美が欲しければ、まずは自分が提供できるものを差し出すことだ。 反対に、相手の攻撃を回避するには無反応になればいい。 困ったセクハラ、パワハラ上司に対しては、ひたすら無反応を貫く。 そして、いつの間にか相手が奴隷の立場になっていることに気づくだろう。 女尊男卑 女王様をやっていると、よく聞かれるのが「最後(フィニッシュ)はどうするの?」 という質問だ。 そもそもSMクラブに来る変態たちは、インスタントな射精に興味がなく 誰にも理解されないような性癖を内に秘めながらも、熱い願望を叶えるためにやって来る。 SMクラブでは、女王様は脱がないし、女王様に気安く触ることも許されない。 プレイでは、女王様は絶対的に立場が上という設定であるため、こちらが自由に決めることができる。 割合でいうと、オナニーを蔑むような眼差しで見てもらいたい変態が多かったように思うが 場合によってはイカせない時もあるし、一切ペニスには触れないこともある。 女王様は頭の中でストーリーを考えているが、全てがシナリオ通り進行するわけではない。 なぜなら相手は、理性を脱ぎ捨てた大人の面した赤ちゃんであり、 途中で状況が変わることもあるからだ。 そしてシナリオを微調整しながらプレイを進めていくのだが、 その場で瞬時に方向を変える臨機応変さも鍛えられる。 日常の戦いに疲れた紳士たちは、女尊男卑の世界に身をゆだねる心地よさを見出しているのだろう。 そして、そんな世界を知っている人は、やはり人との接し方にも器の深さが滲み出る。 もちろんジェンダーレスなSMの世界に男女の優劣なんてなく、自由に設定しているのだが。 SMプレイというものは、その種類は何百通り、何千通りとあるとも言える。 それは自分と相手で作る空間演出、即興劇であり ルールやマニュアルはないのだ(もちろん安全性や法律を考慮した上での話)。 それは人間にしかできないクリエイティブな活動と言える。 つまり、感性、感度を研ぎ澄ませば何歳になっても楽しめる世界であり それらが鈍ってしまったら、無限に見出せるはずの「ご褒美」も受け取れないし、生きててもつまらない。 思うにSMプレイの役割とは「生きててよかった」「人間でよかった」 と再確認することではないだろうか。 女王様は非国民 恐らく現代の日本では、住む場所、時間の使い方、人付き合い・・・ 全て自分で決めることができて、自由な生き方をしている人間は少数派であろう。 不満や愚痴を言いながら、環境を変えようとしない人たち。 そんな手枷足枷を付けられて、口枷もつけられ発言できない 奴隷社会の檻の中に生きる人から見ると、女王様の考え方は社会のレールから脱線した非国民だ。 しかし脱線してみて初めて気づくのは、おまえたちこそ危険な乗り物に乗っているのだよ、 ということだ。 車窓から見える景色は、せいぜい180 度だ。 反対側の景色が見たければ、飛び降りてみればいい。 以前、ある夢見る男性が言っていた。 「ディズニーランド好きじゃない人は非国民だ」 と。 自分がそうだからといって、相手も同じではない。 悪いが私は、その裏側にしか興味がない。 【調教ワーク 1】 ・あなたがこれまで、「生きててよかった」と感じたときは、どんな時だっただろうか? ・あなたは生きてる実感を味わうために、今から何をする?

\section{text00004}
第2章 「女王様ストーリー」

\section{text00005}
第2章 「女王様ストーリー 第2章 赤ちゃんプレイ 私が元女王様だということを公言するようになってから、 やはり多くの方から 「なぜ女王様になろうと思ったの?」 と聞かれる。 それは恐らく「女はこうあるべき」という足枷に繋がれた奴隷だったからだろう。 私は平凡な環境で育った平凡な人間だ。 ただ、人より少し好奇心が強いかもしれない。 既製品の型に収まったド真面目な両親を見て、大人になることに対し「絶望」を感じていたものだ。 「大人になったら自由を奪われる」 そんな危機感が常に付きまとっていたため、 「自由が奪われるなら、死んだ方がマシ」 そんな風に思うようになっていた。 大人になってから、周りから行動派だとか言われるようになり、気付いたのだが 多くの人たちが「やりたいけど、行動できない」と言う。 自由になれなくて愚痴をこぼすのに、自由ではない今の方が、居心地がいいらしい。 つまり、いつまでも責任を負わなくてもいい赤ちゃんでいたい、ということだ。 赤ちゃんプレイなら、SMクラブに行けばいくらでもできるよ。 私の辞書には「いつかやる」「そのうちやる」という言葉はない。 常に「今」に集中し生きてきたからこそ、今があるのだ。 「今」と言い終わった時点で、もうそれは過去になっているのだし、 そう考えると未来なんて永遠にやってこない。 誰にも言えなかった「いじめ」 小学生のころ、私は非常におとなしくネクラな子であった。 そのため何か気に食わなかったのか、上級生のある男の子からいつも虐められていた。 よくある好きな子に悪戯をするようなレベルではない。 言葉の暴力にとどまらず、突然、全力腹パンチを食らったこともある。 内臓が暴れ狂うような、あまりの痛さにうずくまりしばらく動けなかった。 だが、弱さを見せたくない私は平気なふりをしていた。 そのことは親はもちろん、誰にも言えなかった。 “強くならなければ”という感情をどこに向けたらよいのか分からなかった。 罰が当たったのかどうか知る由もないが、ある日突然、彼の父親が亡くなった。 たぶん、歳はまだ40 くらいだろう。 それ以来、彼からは覇気が消え、一切攻撃を仕掛けてこなくなった。 思えばそれは、私を女王様へと導く第一歩であっただろう。 ストーカー事件 もう一つは、18 歳の時だった。 元交際相手がストーカーになったのだ。 彼は「親に殺されそうになった」と血だらけで私を待ち伏せしていたり、 「死にたい」というのが日常だった。 精神的にまだ幼かった私は、危機感を感じ別れを切り出した。 すると彼は、直接会わないと気が済まないと言い、 殺される覚悟で会うことにした。これでやっと別れられる・・・。 ドン と、私はアスファルトに叩きつけられ殴られアザだらけになった。 そして彼は 「死んでやる!」 と叫び、道路に飛び出し警察沙汰になった。 その後もしばらくの間、ストーカー行為は続き、 「明日、私は殺されるのか?」 などと考えながら、毎日を過ごしていた。 その度に、「強くならなければ」と自分に言い聞かせた。 無反応を貫いたところ、ストーカー行為は次第に鎮まった。 その頃の私は、人と関わることに疲れ、一人で過ごすことが多くなった。 鬱と不眠症 就職した時だ。 私の後から入った新人さんたちは、皆すぐに辞めてゆく、という厳しい職人の世界である。 非常にピリピリとした現場だ。 中途半端が嫌いでストイックすぎた性格のせいか、気付けば私は鬱と不眠症で、疲れているのに眠れない日が続いた。 ストレスのせいか食べても胃が受け付けず、吐いてしまうため、気付けば3カ月で体重は15 kg減っていた。 思うように体が動かず、頭も働かず、他者から見ても、明らかに精神が病んでいた。 笑顔をつくる表情筋も動かせなくなり、楽しそうに笑っている人を見ると、手の届かない遠い世界に霞んで見えた。 「自分はこんなに弱かったのか」 と落胆した。 そして、健全な精神を取り戻すことを優先とし、わずか4カ月で退職した。 いつも強がっている自分は認めたくなかったが、思った以上にメンタルが弱かった。 どうしたら強くなれるのだろうか。 「そうだ、女王様になろう」 新しい仕事も落ち着いたあるとき、 突然何かが降りてきた。 そうだ、女王様になろう。 「固定観念という縄に縛られたくない」 「支配されたくない」 そんな思いが年月とともに進化したのか私を女王様へと導いた。 SMのことなんぞ全く知らなかったが、一度興味を持ったら引き下がれないタチなのだ。 だが、SM女王様という職業は誰もが軽い気持ちでできるものではない。 人生を捧げる覚悟がなければ、やる資格はない。 思うにSMの世界というものは、マリアナ海溝よりも深く、木星よりも重い世界なのだ。 死への憧れ 「性の快楽は死の疑似体験 ジョルジュ・バタイユ 」 人間は心のどこかで「死」に対する憧れを抱いている。 スカイダイビングや、バンジージャンプなんかが良い例だ。 「死」まで行ってしまったら戻って来られないが、 スレスレまで行くことが、人間にとってはこの上ない快楽なのだ。 スカイダイビングをした知人が言っていた。 「地球とSEXをしているみたいだった」 と。 世の中には、自分で首を絞めたりして、独りSMをしているうちに スレスレを超えてあの世に逝ってしまう人が年間、数百人いるらしい。 なんとも言葉にしがたい想いである。 それ以外にも、浣腸の水の温度が低いと体温低下にもなり得るし、 長時間の緊縛で神経麻痺、呼吸管理プレイで窒息、不衛生な針プレイで感染・・・ など挙げたらきりがない。 以前、在籍していたSMクラブに熟女の女王様がいたのだが、 彼女はM男に潮を吹かせる、ではなく口から泡を吹かせ気絶させて楽しんでいた。 さじ加減を分かっているからこそ成せる技なのだろうが、良い子も悪い子も真似をしてはいけない。 このようにSMプレイは、常に危険と隣り合わせであるという意識を持ちながら進めなければならない。 相手の心と体を緩め、 時には「死」のすれすれまで誘う。 女王様は命を預かる仕事とも言える。 「死」を意識して初めて、生きていることを実感するのだ。 女王様の個人授業 まず初日、先輩女王様から女王様講習を受ける。 道具の使い方やお客様の扱い方などを、まずは自分の体で受けてみるのだ。 指導してくれたのはSM歴8年の、姉御タイプの女王様。 「SMはね、板につくまで3年くらいかかるよ。だけど自分でコントロールできるようになると楽しいよ。 道具も自分で買わないといけないけど、それは投資だと思って。 けど最初の半年は、ここにある道具だけで充分」 と言っていた。 実際SMプレイというものは、道具はあくまで脇役なのだ。 「はい、脱いで」 と誘導され、説明を受けながら、浣腸されたり、アナルバイブを突っ込まれたり、鞭で打たれたり、 ロープで縛られたり・・・と、ドMになった気分で講習を受ける。 勿論、真面目な講習なので、二人とも淡々とやっていたが、 そのとき私は「女王様に責められるのも悪くないな」と感じたものだ。 いやむしろ恍惚とした気分になるものだった。 責められてみなければ、責め方を描くことはできない。 もしあなたが、一流のサービスを提供したいのなら、自分も一流のサービスを受けなければならない。 薄っぺらい輩 「ムチやロウソクを使う西洋のハードSMが本来のSMだ」 とか 「いや、縄は麻縄じゃなきゃ」 という議論は専門家同士ですればよい。 あなたにお伝えしたいことは、そこではなく全体図を見るということである。 正しい、間違ってる、なんてこの世に存在しない。 マニアックなSMウンチクを語っても、現実社会に応用ができなければ意味がない。 相手が何を求めているのかを五感で察し、そして相手の持っている未知の才能をいかに引き出すか、が女王様の役割なのである。 「SMクラブで鞭に打たれてみたけど、よくわからなかった」 という薄っぺらい輩がいるが、そこじゃないんだよ、と言いたい。 イメージや形だけに囚われ、物事を主観でしか見ていない証拠である。 SMクラブはどんな人が来るのか? これもよく尋ねられる質問である。 恐らく多くの方たちが決して見ることのない光景を、幸運にも私は見てきている。 SMクラブに通うお客様は、社会的地位の高い、権威のある肩書を持っている場合が多い。 例えば、お医者様や経営者、警察官、先生と呼ばれる立場の方などだ。 性格の特徴を挙げるならば、好奇心旺盛、創造性が豊か、探求心旺盛な方、だろう。 中には過剰な依存体質の方もいる。 ただ言えることは、普段はみな普通の紳士(淑女)たちだ。 いや、むしろ聡明で仕事のできる方たちだ。 バカになることはできるが、決してバカではない。 そんな紳士(淑女)たちが、女王様だけに見せる姿があるのだ。 女王様たちは、仮面を被った人間たちの素顔を見ることができるのだが、 それは信頼関係と即興劇によって引き出される。 例えるならば、自宅の庭で金が採掘できたような達成感だ。 SMの世界というものは、エロティシズムの本質から、「人間とは何なのか?」 を深く考えさせられる世界なのだ。 アナル調教のモンスター 現役時代、私は女王様としては二流だったが、そこで得た経験を、 現実社会に応用し成功する方法を伝える、ということが私の役割だと思っている。 幸運にも私は、キャリア15 年のNo.1女王様と一緒にM男を調教する機会が何度かあった。 他の追随を許さない、神の領域にいる彼女のプレイ内容は、またの機会にこっそりお話しすることにして 今回は私の好きなプレイを一つだけ書こう。 アナル調教の中でも上級者向けのユニークなものがある。 「アナルフィスト」だ。 どのようなものかと言うと、M男の尻の穴に手の指を5本とも入れ、さらに手首まで咥えこませるのだ。 そこからさらに肘まで飲み込んでしまう、欲しがり屋のド淫乱ケツマンコを超越した モンスターも存在する(そして肘まで飲み込みながら世間話をしてくるやつもいる)。 腕が心地よく締め付けられる、ケツ圧による血圧計とでも言ったらよいのだろうか。 勿論このような変態たちは、礼儀として事前に浣腸を済ませ、肛門を綺麗にしてやって来る。 こちらはゴム手袋をはめ、ローションをたっぷり塗りたくった下の口に、ゆっくりとぶち込み 言葉責めをしながらズブズブにして差し上げる。 身も心も捧げて自ら生贄になり、100 %信頼してくれるM男を前にすると、とても愛おしく思えるものだ。 このように女王様と奴隷は全身全霊で向き合っているため、 中途半端な気持ちや態度は、SMクラブという館では許されないのである。 (初心者の場合はプレイに集中し、大袈裟に反応すると女王様は喜びます)。 私が日頃から、「SMを舐めるな」と言う所以が少し伝わっただろうか。 舐めたいなら、首輪つけてペットになれ。 SMとSEXは次元が違う よく勘違いされるのだが、SMクラブに於いては挿入なんてありえない。 そもそもそんな、野良犬でも思いつくような発想は次元が低いとみなされる。 もしそんな礼儀知らずの勘違いカスが現れたなら 「一人でシコってろ」 とエナメルブーツで蹴っ飛ばされるだけだ。 女王様は脱がないし、基本的に奴隷は女王様に馴れ馴れしく触ることも許されない。 奴隷は常に上から見下ろされ、女王様を見上げることが喜びなのである。 自分を「しょうもないカス野郎」として設定し、失うものは何もない状態に一旦持っていく。 地位や財産、権力、学歴、資格・・・ そう、天国で役に立たないものは全て脱ぎ捨てて挑むのだ。 これまで武器として携えていたものが、いかに無意味だったかということを実感すると共に、 視野が引き伸ばされ、強靭な精神力が鍛えられるのである。 実は、裸一貫のドMとされるポジションは、威圧感満載の武装した女王様よりも強いとも言える。 SMクラブを訪れる変態と呼ばれる人種たちは、動物でもできるSEXではなく なぜそこに興奮するのか?というところに興奮するのである。 女王様は、常日頃から多種多様なフェチ族、女装愛好家、糞尿愛好家、身体改造などの「リアル人間図鑑」を見ているのだ。 社会に溶け込んでいる人間たちが、決して誰の目にも晒さない姿を女王様は見ることができる。 これだから人間はおもしろい。人間はやめられない。 流血プレイ 「人間でよかった」 そう気づかせてくれた人に出会った。 彼は東日本大震災の際、真っ先にボランティアに行くなど、困っている人を放っておけないとても人間らしい人だった。 世の中を冷めた目で見ている私を見て、 「本気で人を好きになったことないでしょ?」 と見抜かれてしまった。 家庭崩壊、殺人容疑をかけられる、などなかなかの苦労人である彼は、人の痛みに敏感であり、大切な人の為なら自己犠牲を厭わない、 そんな人だった。 そんな「ドM精神」を持つ彼の生き様に、私は「この人の為なら、何でもできる」 というさらなる「ドM精神」が引き出されたのだった。 そのとき私は、ようやくSMの本質を理解したのだ。 世の中には、SやMなんて存在しなかった。 良い悪い、正と誤、優と劣、美と醜・・・ それらの基準は絶対的な固定点がない。 ねつ造されたこの世の中は、否応なしにどこかに固定点を作らなければならず、 それらは自分の思うままに自由に設定すれば良いのだ。 だが、その関係も長くは続かなかった。 互いにメンタルが弱く、一緒にいるべきではないと判断した私は、自ら別れを告げた。 しかし、女王様としては一段上から目線で眺められるようになり、生暖かい血がどくどくと流れ出すような 心の血が通ったプレイとなった。 「こんな体験は初めてだよ!」 と感動して涙腺を緩ませ、涙と鼻水を垂れ流すお客様もいた。 SMはテクニックではなかった。 自分の心を動かさなければ、人の心は動かせないのだ。 女王様引退 SMクラブに在籍しながら、時々思うことがあった。 この仕事はやりがいもあって興味深く、もっともっと深く追求したい。 だが、どっぷり浸かってしまったら、恐らく社会には戻れないだろう。 なぜなら、周りの人たちとは考えていることも違うし話も合わない。 それにこの世界にいる限り、身を潜めて生きてゆかねばならない。 自分にその覚悟はあるのだろうか、という葛藤があったのだ。 やはり突き進むのを諦めて、普通の生活に戻った方がよいのだろうか。 常識を重んじる、真面目人間たちに囲まれていた私には、墓場まで持っていくものが重すぎて、背負いきれなかったようだ。 私はSM女王様を引退した。 これが、足枷をつけられた奴隷思考というものだ。 恋人の突然死 その後私はエステサロンを開業し、生活はがらりと変わった。 それを陰ながら全力で応援してくれた人がいた。 20 歳年上の恋人だった。 だが、出会って4カ月の頃、毎日連絡をくれていた彼から連絡が来なくなった。 開業したてのサロン業務に追われ、もともと素っ気なく接していた私は、 きっと彼も忙しいのだろうと思って気にしていなかった。 だが、それから1週間経った頃だ。 彼はもうすでにこの世に居なかった。 別れの挨拶もせずに旅立ってしまったのだ。 冗談ばかり言っていたので、これも冗談であってほしいと願ったが、それも虚しく 脳出血による突然死だった。 後悔はないと言ったら嘘になるが、その出来事によって、生きていることは当たり前ではない、 「死」はすぐそこにある、ということを改めて突き付けられた。 「死」を意識することによって初めて「生」に感謝する、ということを実感した。 露出しないのは罪 彼の死をきっかけに、ビジネスパートナーとなる人物に出会い、誘われて東京に引っ越しすることになった。 そのころ私のサロンは赤字が続き、貧困状態に陥り閉めることにした。 当時、所持金はほとんどなく切り詰めた生活をしていたが、それでも環境を変えたかった。 環境が変わる日は、人生が変わる日だ。 結局、ビジネスパートナーとはウマが合わず解散したため、その場しのぎで夜の蝶になることにした。 自分としては、口下手で気が利くわけでもなく、とても夜の蝶が合う体質とは思えないが、 しかし気づけば不動のNo.1になってしまった。 誰しも自分を高く評価してくれる場所というものがある。 私はずっと親から「お金は苦労して稼ぐもの」だの「楽して稼いではいけない」だの 刷り込まれてきた。だがそれはただのルサンチマン(他人を羨み劣等感だらけの自分を正当化する)に過ぎない。 いかにレバレッジをかけるか、ということは人を幸せにすることでもある。 これまでは清楚ぶって、女王様であったことを生き埋めにして墓場まで持っていくつもりだったが、 地元を離れたことをきっかけに、墓を掘り返して晒し者にしてみたら全てが良い方向に向かっていった。 何もかもダメダメだった自分は何だったのだろうか。 SMクラブにいた時に気づかなかったことが、現場を離れてから見えて来た。 執筆活動をするようになり、たくさんの方から 「生き方が変わった」「向かう方向が見えてきた」 と感謝のメッセージを頂くようになり、非常に嬉しい。 私はそんな「自己調教」の手の内を露出することにしたのは 露出しないのは罪であると思ったからだ。 【調教ワーク 2】 ・あなたが持っている武器(肩書、財産、能力、資格など)のうち、 天国で役に立たないものは一旦捨て自分をしょうもないカスに仕立てよう。 ・何が残っただろうか?

\section{text00006}
第3章 「自己啓発なんて役立たず」

\section{text00007}
第3章 「自己啓発なんて役たたず」 第3章 SMは言葉の世界 人それぞれ持論はあるかもしれないが、私はSMの世界は「言葉の世界」だと思っている。 正直、ロウソクや鞭なんてただのアクセサリーだ。 もちろん言葉責めなんて要らないという人もいるし、蔑むような眼差しだとか、触れ方や勘によるもの それら非言語要素も不可欠だ。 だが、調教の中で交わされる会話、卑猥な言葉を発するとき、即ち言語化することこそ相手の臨場感が高まるのだ。 新人の頃、SMクラブの店長が言っていた。 「SMはね、言葉責めが9割だから。言葉責めをよーく勉強するように」 と。 要するに、イカすも殺すも言葉次第であり、その言葉を操る者に染み込ませなければならない。 言葉を操れない者は、人を動かせない。 いや、しかし存在だけで語る者もいるじゃないか、と言うが「存在」も言葉である。 言葉責めを取り扱う女王様たちは、日頃から相手の心に響く「言葉」について、多かれ少なかれ研究しているのである。 言葉は嘘の世界 私たちは、他人からのちょっとした一言によって傷ついたり、イラついたり、トラウマになったりすることがある。 人の感情を支配する「言葉」というものは恐ろしい。 けれど実は言葉そのものは意味を持たないのである。 言葉と意味は別であり、私たちはそれを勝手に結びつけているのだ。 例えば、「苦しい」という言葉を知らなかったとしよう。 そうすると「苦しい」という感情は湧いてこない。なぜなら「苦しい」が何なのか分からないからだ。 ひょっとしたらこの感情は「楽しい」なのかもしれない。 言葉に勝手に感情を吹き込んでいるのは、「自分」だったのだ。 ならば、あなたの都合で「言葉」と「意味」を切り離せばいい。 罵倒も嫌味も、人間と思えない言葉も、ただの「言葉」だ。 落ち込んだ、傷ついた、そう感じたならば思い出すといい。 「言葉」と「意味」は別のもの。あなたは自在に操る権利がある。 言葉というものはじつに曖昧なものである。 私たちがいるこの世界は、言葉と本能に支配された「嘘の世界」だったのだ。 例えば私は、「我慢」という言葉を消し去った。なので「我慢?なにそれ美味しいの?」となるのだ。 要らない言葉は自分の辞書から消してしまえばいい。 言葉の意味を深く噛みしめるのは、心ある言葉と女王様の言葉責めだけでよい。 裸の王様的露出狂 根拠もなく低レベルな批判をしたがる薄っぺらなカスどもが存在する。 例えば書籍などのレビューなどを見ていると、 「目新しいことは書いてなかった。がっかりです」 「酷い内容だ」 などの上から目線の感想を書いている人を見かける。 だが、そんな輩たちは自分の半径1メートルでしか物事を見ていない。 その背景に何があるとか、そういったことを想像する力がないのである。 つまり、読解力のなさ、愚かさを自ら晒している露出狂だ。 恥ずかしい姿を見せつけたい変態露出狂なら、喜んで虐めてあげるが 笑われていることに気づいていない「裸の王様的露出狂」は、調教するに値しない。 感度の高い賢者たちは、一見無駄なこと、意味のないようなことからも、 敏感に感じ取りどんどん進化してゆく。 あらゆる問題の解決ができるかどうかは、女王様による上から目線(抽象的な思考)ができるかによって決まる。 だが、不感症の愚者たちは、どこで何をしていても何も生まれない時間を過ごす。 本書を読めば、誰が「裸の王様的露出狂」かが分かるだろう。 批判の仕方ひとつで人間レベルがバレてしまうのだ。 私は、ここにもそんな、おっぴろげの露出狂が現れることを期待している。 自己啓発よりも「自己調教」 現在も「なりたい私になる」だとか、「引き寄せ」だとか 自己啓発のキラキラした言葉に惹かれる女子たちが溢れている。 しかし私は自己啓発の本なんてほとんど読んだこともなければ、自己啓発セミナーにも行ったことがなく そんなもの必要ないと思っている。 それよりも、大金を払ってでもその本質を存在で教えてくれる、よい人脈を作ったほうが早い。 あれだけ数多くの成功本が出ているのに、読んだ人がみんな成功しているわけではないのはなぜだろうか。 その答えを分析することに意味があるのにもかかわらず、一番重要なことは隠されているのだ。 つまり、本当に大切なのは言っていないことであり それらは演出という劇を見せつけられ、垂れ流し奴隷はなかなか見破れない。 これからは自己啓発ではなく、自分を調教する「自己調教」だ。 SMクラブで見てきたもの、それは人生哲学だった 引退してから気づいたのだが、私がSMクラブで見てきたものは、人生哲学そのものだった。 現場を離れて俯瞰することによって、これまで見えなかった景色が見えてくるものだ。 SMクラブを訪れる人間はなぜ、わざわざ痛み、苦しみを味わいたいのか。 結局、人間は「絶望」を味わわなければ、自分が生きてるのか死んでるのか分からないのだ。 最初から「絶望」を避け、毎日がハッピーなんて言っている人は、頭のおかしなインチキスピリチュアル人である。 それは「絶望的無知」(絶望を素通りし、真剣に人生と向き合っていない人)でしかない。 だが、反対に絶望の取り扱い方が分からず、絶望だけを強調するやつは、承認されたいかまってちゃんだ。 要するに、絶望を知ることでしか幸せも感じることができないのであり 痛みを知らなければ、快楽も感じることができない。 同様にしてMを体験しなければSも理解できないため、女王様は先輩女王様から調教を受けるのである。 そのようにして、女王様は世の中の両極を見ているのである。 「人生は絶望だ ―― セーレン・キルケゴール」 【調教ワーク 3】 ・あなたがこれまで、絶望的になったのはどんな時だろうか? ・あなたが要らない言葉、感情は何だろうか。今日から自分の辞書から消してしまおう。

\section{text00008}
第4章 「痛み・苦しみを飼いならせ」

\section{text00009}
第4章 「痛み・苦しみを飼いならせ」 第4章 汚されたい願望 先日、ある女性タレントで炎上クイーンのトークショーを拝む機会があった。 彼女はネット上で誹謗中傷を投げかけて来る相手に対し、 「どうやったら好きになってもらえるだろうか」 と考えるという。 元女王様である私には理解できないが、自分としては 「どうやったらもっと嫌われるだろうか」 とその演出を考える方が圧倒的に楽しい。 それにこの世の中は、好かれようとすればするほど嫌われる という法則がある。 同様に、「美しくなりたい」と願うのは、醜さを隠したいという内なる願望であり、 反対に、「蔑まれるほど汚されたい」と願うドMは、 清く美しいからこそのないものねだり、とも言える。 心の汚れたやつは、汚れは欲しがらない。 言っていることよりも、言っていないことの方が、じつは重要なのである。 従って、本当に重要なことは成功法則の本には書いていないし、 本当に重要なことはニュースでは報道されない世の中なのだ。 だからと言って、美しくなりたい願望を否定する必要はないし そんなことはどうでもいいんだよ。 美しさの基準も自分で決めればいい。 自分の尻にスパンキングしろ 行動できないだとかいう声をよく聞くが、 そもそも人間は追い込まれないと行動できない。 その原動力が「痛み」や「苦しみ」なのだ。 もし今、痛みや苦しみの渦中にいるとすれば、喜ぶがいい。 なぜなら、幸福感、快楽を得るには痛み、苦しみが不可欠だからだ。 一方の振れ幅が大きいほど、両極にまで視野は広がり それは物事を俯瞰する抽象的思考に繋がるのだ。 反対に、痛みや苦しみに鈍感になる時期もある。 そんなときは敢えて自分の尻にスパンキングをするのだ。 それと同時に、尻をさらけ出すという羞恥プレイも待っていて楽しいではないか。 もちろん例えだが、選択肢は一つというところまで追い込むことにより、 行動できないわけがない、逃げられない状況にするのだ。 ただ、追い込む方向を間違って、心が病んでしまう人もいる。 痛みを飼いならすには、己の中の女王様から育成しなければならない。 「人として弱いということは、生きていく上で受けるべき苦しみを 自分で受け取ろうとしないことだ。 ―― ウィトゲンシュタイン」 “自分にご褒美”という発想は捨てろ 世の女子たちが愛してやまない“自分にご褒美“。 だが私は、自分にご褒美という発想が分からない。 生きているだけでご褒美ではないだろうか。 目の前にあるご褒美に気づかない“目隠し奴隷”どもは、ご褒美はまだかとヨダレを垂らしながら 嫌なことを我慢しながら生きている。 そして我慢、我慢、で結局、陰で愚痴を言い、「可哀想な自分」を演出し、ご褒美を追いかけ回している。 しかし、そのプロセスを俯瞰し楽しむこと、それ自体がご褒美であることに気づかない。 自分の意思で好きでやっているのだから、女王様は奴隷を可哀想だなんて思わないし、 「痛みを与えてもらえるなんて、なんて幸せなのだろうか」としか思えない。 奴隷からすれば、“がんばっている自分”に陶酔しながら精神が鍛えられ、 そのプロセスによって自己評価が高まるのである。 成功者は常にプロセスを楽しみ、一方で敗者はプロセスを軽視し墓場へと向かう。 女王様に甘えるのはいいけれど、自分に甘えていたらご褒美は受け取れないんだよ。 つまり、SだとかMだとか薄っぺらなことを言い、SMの世界を軽視する者は、成功できないのである。 自己調教のすすめ 世の中を上から目線で見下ろすには、自分の中に女王様目線を置くことだ。 私はこれを「自己調教」と呼んでいる。 例えば、誰しも一度は自分の中に、2人の声を聞いたことがあるだろう。 「ここは嘘を言っとけ」 「いや、嘘はよくないよ」 どちらの声に従えばいいのか、というような場面だ。 ただし、この一人会話は同じ次元で繰り広げられているため、奴隷同士の会話だ。 「自己調教」の場合、同次元に居る二人の奴隷の会話を女王様が調教しなければならない。 つまり、「上から目線」でなければ自分を調教できないのだ。 よって、どちらかを選ぶのではなく、そもそもその迷いが発生する根源を 尖ったヒールで踏みつぶしてしまえばいい。 女王様ならば、 「言ったことを、後から現実にすればいい」 のである。 それはできません、などと言い訳ばかりしていると、女王様から見放されるのだよ。 「ほら、できたじゃないの」 結局、全ては自らが作り出した虚像であり あなたは「一人SM劇場」のキャストだったのだ。 【調教ワーク 4】 ・あなたが最近、一人会話をしたのはどんな時か? ・そのとき女王様なら何と言うだろうか? まとめ 扉の向こうで繰り広げられるSM調教を、現実社会に落とし込んだ「自己調教」は、 痛みや苦しみを飼いならす紗弥独自のメソッドであり、 アナルバイブを使ったことがなくても、ボンデージを着たことがなくても、応用することができる。 これまでの調教をまとめると 1.自分をカス野郎として設定する。 「カス野郎です。ありがとうございます。カス明利に尽きます」と言えるようになれば上等だ。 2.言葉と意味を切り離す。 言葉は意味を持たない。要らない言葉は唾と一緒に吐き出してしまいなさい。 言葉に操られてはいけない。自分で操るのだ。 3.死を意識する。 賢者は自ら痛みの世界に身を投じる。 戦争映画を観て感想を書こう。 4.プロセスを楽しめ エクスタシー(結果)はご褒美ではない。プロセスがご褒美だ。 5.上から目線になる。 一人会話が始まったら、自分を俯瞰し問題の発生源を潰す。 自分の中のカス野郎に「できません」とは言わせない。 これら5つの掟をよく咀嚼し、あらゆる体液で消化できた者だけが、ご褒美をもらえるのだ。 だけど勝手に脱糞したら、お仕置きだよ。 「きみ自身がきみの世界だ。 きみの生き方で、きみの世界はいくらでもよくなっていく ―― ウィトゲンシュタイン」

\section{text00010}
あとがき あとがき Mは痛いのが好きとか、Sはいじめるのが好きとか、そういう次元のものではない、 ということがお解り頂けただろうか。 イメージだけが浸透していると同時に、視野が狭くなりがちな世の中に向けて、私はどうしても伝えたかった。 心が弱ったとき、多くの方は、「そのままでいいんだよ」とか、 優しい言葉で慰めを求めているかもしれないが、そのままなら何も変わらない。 優しい言葉が逆に現実とのギャップを作ってしまい、そこから永遠に抜けられないのだ。 今、不安や絶望を感じているなら、拷問プレイだと思ってじっくり味わえばいい。 ヘタに逃げようとすると、どんどん迫ってきて潰されてしまうからだ。 この痛み、苦しみは全て調教。 それは、ぎりぎり乗り越えられる者にしか与えられない、ご褒美だ。 「自己調教」 これは、SMクラブという多種多様な人間が集う場で、人間の本能と本質に向き合ってきた私が生み出した独自の思想である。 この道具を使ってどう快楽を得るかはあなた次第だ。 人間に生まれて本当によかった。 最後までお読みくださり、ありがとうございます。 2018年 4月 8日 紗弥 【紗弥 公式サイト】 http://smsamsara.xsrv.jp/

\section{text00011}
プロフィール ♦著者略歴 紗弥 元SMクラブ女王様 No.1ホステス、作家、投資家 メンタルが弱く何度か鬱を繰り返し、引きこもる。 会社勤めが性に合わず、2年で辞めてフリーターに。 音楽活動をしながらバックパッカーでインドを放浪する。 28歳のときSMクラブ女王様になり、約1700 人の奴隷たちをひれ伏させる。 引退後、女王様であったことを封印しながら社会に溶け込み、エステサロンを開業するが、赤字が続き貧困状態に陥る。 恋人の死をきっかけに東京へ拠点を移し、生き埋めにしていた女王様思考を掘り起こす。 執筆活動の傍ら、夜の蝶として約300 人の奴隷志願者を携え不動のNo.1になる。 大富豪、投資家、不動産王たちとの交流もあり、自らも投資家として世界の経済を見下ろしている。 初の電子書籍「SM奴隷から学ぶ「うつ」の治し方」は、人生に彷徨う者たちに多大な影響を与えている。 尊敬する実業家・・・アニータ・ロディック 好きな作家・・・谷崎潤一郎、種村季弘 好きな音楽・・・ジャズ、中東音楽、世界各地の民族音楽、60年代ロック、ソウル、R&Bなど

\section{text00012}
奥付け ─────────────────────── 『SM奴隷から学ぶ「うつ」の治し方』 1700 人の奴隷を従えた 元女王様が教える「自己調教」のススメ 2018年5月04日 発行 初版 著者: 紗弥 カバーデザイン:澤田 哲志 製本:佃 耕自 Copyright (C) 2018 紗弥 All Rights Reserved. ───────────────────────



\end{document}
