\documentclass[12pt,a4paper]{article}
\usepackage{fontspec}
\usepackage{xeCJK}
\usepackage{paracol}
\usepackage{geometry}
\usepackage{hyperref}

\setmainfont{MS Gothic}
\setCJKmainfont{Noto Serif SC}
\setCJKsansfont{Noto Sans SC}
\setCJKfamilyfont{jpfont}{MS Gothic}
\AtBeginDocument{\renewcommand{\familydefault}{\jpfont}}
\geometry{margin=2cm}

\title{SM奴隷から学ぶ「うつ」の治し方\par 从SM奴隶中学到的「抑郁」治疗方法}
\author{紗弥\par 纱弥}

\begin{document}

\maketitle
\tableofcontents

\begin{paracol}{2}
\columnratio{0.5}
\setcolumnwidth{0.5\textwidth, 0.5\textwidth}

\switchcolumn[1]
\centerline{\textbf{中文翻译}}
\switchcolumn
\centerline{\textbf{日文原文}}

\switchcolumn[0]

\section{まえがき\par 前言}

\switchcolumn
\begin{flushleft}
まえがき まえがき 本書を手に取ってくださり ありがとうございます。 「ふてぶてしく美しく生きよう。」 先に申し上げておくが、私はカウンセラーでも医療従事者でもない。 また本書は、精神分析の本でもなければ、医療の本でもない。 メンタルが脆弱で引っ込み思案だった、元SM 女王様の書いた人生哲学である。 少々過激なタイトルだが、これは「薬なしで治せます」と謳うものではない。 私はこの本を、人生に彷徨う全ての方たちに捧げようと思う。 SMと聞くと多くの方は「ムチとかロウソクとか使うんでしょ?」「痛いのは嫌」「そういう趣味はない」といった反応を示し、イメージだけが独り歩きしている。 実は、日常生活こそSMの世界に通じており、それを意識するかどうかによって、ものの見方、捉え方は大きく異なる。 痛み成分にも・言葉による痛み・精神的痛み・肉体的痛みがあるが、ここでは私が調教してきた奴隷たちから学んだこと、辛い、苦しい、不安・・・の飼いならし方を露出しようと思う。 そもそも、なぜ世のドMたちはお金を払って「痛み」や「苦しみ」を味わいたいのか? それを知ることによって、これまで見えなかったものが、M字開脚のドMの菊門のようによく見えるだろう。 じつはSMの世界を知ることこそが、人生を快楽にする秘訣だったのだ。 本書を読んでいて不快に感じる方もいらっしゃるかもしれない。 だが、お付き合い頂けるなら、鞭よりも女王様の手の温もりを感じるスパンキングということで、楽しんで頂ければ幸いだ。 人間関係、恋愛、仕事、お金、自信、自己肯定感・・・ 人生を甘美にする秘訣は、ノウハウやテクニックではない。 「女王様目線を持つこと」だったのだ。 「人として弱いということは、生きていく上で受けるべき苦しみを自分で受け取ろうとしないことだ。 ウィトゲンシュタイン」 では、これまで鋼鉄の扉に閉ざされていた、痛みと快楽の世界へご案内しよう。
\end{flushleft}

\switchcolumn[1]
\begin{flushleft}
前言 前言 感谢您拿起本书。 「骄傲而美丽地生活吧。」 首先声明,我既不是咨询师也不是医疗从业者。 而且本书既不是精神分析的书籍,也不是医学书籍。 这是曾经精神脆弱、性格内向的前SM女王所写的人生哲学。 虽然标题有些过激,但这并不是声称「不用药物就能治愈」的书。 我想把这本书献给所有在人生中彷徨的人们。 听到SM,很多人的反应是「会使用鞭子和蜡烛吧?」「我不喜欢疼痛」「我没有那种兴趣」像这样,只有印象在独自蔓延。 实际上,日常生活才与SM的世界相通,是否意识到这一点,会极大地改变看待事物的方式。 疼痛成分也包括・语言带来的疼痛・精神上的疼痛・肉体上的疼痛,在这里,我想分享从曾经调教过的奴隶们身上学到的,如何驯服痛苦、艰辛、不安……的方法。 说到底,为什么世间的M们愿意花钱去体验「疼痛」和「痛苦」呢? 了解这一点后,原本看不见的东西会像M字开腿的M的肛门一样清晰可见。 实际上,了解SM的世界才是让人生变得快乐的秘诀。 或许有些读者在阅读本书时会感到不适。 但如果您能坚持读下去,我希望您能像享受女王手中温暖的鞭打一样,享受这本书。 人际关系、恋爱、工作、金钱、自信、自我肯定感……让人生变得甜蜜的秘诀,不是技巧和方法。 而是「拥有女王般的视角」。 「作为人类的软弱,是指不愿意自己承担生活中应该承受的痛苦。——维特根斯坦」 那么,让我带您进入这个曾经被钢铁之门封锁的,疼痛与快乐的世界。
\end{flushleft}

\switchcolumn[0]

\section{目次\par 目录}

\switchcolumn
\begin{flushleft}
目次 目次 まえがき 第1章 「目隠しを外せ」 目隠しと垂れ流し 変態のプライド あなたはS?それともM? 生ぬるいのは浣腸だけでいい ドM≠マグロ 女尊男卑 女王様は非国民 第2章 「女王様ストーリー」 赤ちゃんプレイ 誰にも言えなかった「いじめ」 ストーカー事件 鬱と不眠症 「そうだ、女王様になろう」 死への憧れ 女王様の個人授業 薄っぺらい輩 SMクラブはどんな人が来るのか? アナル調教のモンスター SMとSEXは次元が違う 流血プレイ 女王様引退 恋人の突然死 露出しないのは罪 第3章 「自己啓発なんて役立たず」 SMは言葉の世界 言葉は嘘の世界 裸の王様的露出狂 自己啓発よりも「自己調教」 SMクラブで見てきたもの、それは人生哲学だった 第4章 「痛み・苦しみを飼いならせ」 汚されたい願望 自分の尻にスパンキングしろ 自分に"ご褒美”という発想は捨てろ 自己調教のすすめ まとめ あとがき
\end{flushleft}

\switchcolumn[1]
\begin{flushleft}
目录 目录 前言 第1章「摘下眼罩」 眼罩与放任自流 变态的骄傲 你是S?还是M? 温水只适合灌肠 抖M≠金枪鱼 女尊男卑 女王是非国民 第2章「女王故事」 婴儿play 无法对任何人说的「欺凌」 跟踪狂事件 抑郁与失眠症 「对了,成为女王吧」 对死亡的憧憬 女王的个人授课 肤浅之辈 SM俱乐部都是些什么人? 肛门调教的怪物 SM与SEX是不同次元 流血play 女王引退 恋人的突然死亡 不暴露是罪 第3章「自我启发毫无用处」 SM是语言的世界 语言是谎言的世界 皇帝的新衣式暴露狂 与其自我启发不如「自我调教」 在SM俱乐部看到的东西,其实是人生哲学 第4章「驯服疼痛与痛苦」 想要被玷污的愿望 给自己的屁股一顿鞭打 放弃「奖励自己」的想法 自我调教的建议 总结 后记
\end{flushleft}

\switchcolumn[0]

\section{第1章 「目隠しを外せ」\par 第1章「摘下眼罩」}

\switchcolumn
\begin{flushleft}
第1章 「SMなんてものはない」
\end{flushleft}

\switchcolumn[1]
\begin{flushleft}
第1章「根本不存在SM这种东西」
\end{flushleft}

\switchcolumn[0]

\section{第1章 「目隠しを外せ」\par 第1章「摘下眼罩」}

\switchcolumn
\begin{flushleft}
第1章 「目隠しを外せ」 第1章 目隠しと垂れ流し 非常識が常識にもなり得るこの世の中。 そもそも常識、非常識なんて誰が決めるのだろうか。 ただ単に、その流行に躍らされ、本質が見えていない人たちがいる。 ここではその者たちを、「目隠し奴隷」と呼ぼう。 また、世の中に出回っている成功哲学をいくら学んでも、その本質を分析する力がなければ流れ出て行ってしまう。それは所謂、体中の汁を垂れ流す、「垂れ流し奴隷」だ。 私はそんな「目隠し奴隷」と「垂れ流し奴隷」たちに向けてこれを書いている。 SM女王様のお仕事は、世間から見れば非常識だ。 踏みつけたりビンタしたり、罵ったり。それでいて高い報酬を頂く。 そんな世間とはかけ離れた世界に身を置いていた者として言えることは、逆走することで見えてくる景色がある、ということだ。 私は、SMクラブでの経験を現実社会に応用し、悩める人たちの力になりたいと思っている。 いや、正しくは人間の悩みなんて自ら勝手に作っているだけだ。 悩みの種という実体のないものは顔面騎乗で圧迫して潰せばいい。 だが同情はしない。なぜならこれは調教だからだ。 調教に同情なんて要らないし、嫌になったらいつでも逃げればいい。 ただし、調教についてくることが出来た可愛い奴隷だけが、ご褒美をもらえるのだ。 それは私からではなく、他の誰かからかもしれない。 変態のプライド SMは必ずしも道具を使うわけではない。 それは想像と創造の世界、即興劇であり、何が起こるかわからないのが醍醐味である世界だ。 女王様たちは常日頃から「変態」と呼ばれる人種たちと向き合っているが、それ故に、この定義が曖昧である「変態」とは何なのかと突き詰めると、「そんなものはないが、そういう設定」という結論になる。 全ては人間の持つ本能から派生するものであり、吸血鬼や殺人鬼もその延長にある。 それはさておき、日頃女王様たちは「変態」と呼ばれることが名誉である「変態」に対して、愛着をこめて呼んで差し上げている。 ただ、「男はみんな変態だよ」などと、意味不明なことを偉そうに言う輩を見ていると、何でもかんでも軽々しく変態と呼ぶのは、変態に対して失礼である、と私は考える。 なぜならSMクラブに通う変態たちは、全身全霊で変態をやっているからだ。 彼らは男(女)としてのプライドを捨て、それに代わって奴隷としてのプライドが芽生える。 じつは女王様は、そんなプライドを持った奴隷に対して、敬意を払ってイジメているのだ。 あなたはS?それともM? 「あなたはS?M?どっち?」 なんて質問する光景をよく見かける。 恐らく、ほとんどの人たちはよくわかっていないのだろう。 あるいは、自称S,自称Mもいる。 本当のことを教えよう。 それは相手が何を引き出すかによるのであり、一人で成り立つものではないのだ。 自分という存在は外部によって作られるものであり、この世に自分一人しか居ないとしたら、比較の対象がないため、「優しい人」「存在感のある人」とは言えないだろう。 人間の存在価値というものは、相手がいて初めて証明できるのである。 よって、単独で「私はSです」「僕はMです」という位置づけはあり得ない。 もっと言えば、「誰でも両方持っている」とも言えるし、「SもMもない」とも言えるのである。 何でも白黒つけたがる人、本当の自分探しをしている人、そこに拘る人は、本書を読んでから出直してきなさい。 生ぬるいのは浣腸だけでいい 私は伝えたいことを伝えるために極端なことも言う。 そんな発言をしていると、「それは言い過ぎでは?」などという批判もある。 だが、生ぬるいのは浣腸だけでいい。 生ぬるい言葉は浣腸と一緒に下の口から便器に流されてゆくだけだ。 この世の中は、極端に発言しなければ伝わらない。 以前、「あなたの考えは間違っている!」といったメッセージを頂いたこともあるが、「人を変えようとして苦しんでいるのは、おまえだよ」と鼻で笑ってしまった。 世の中は絶対的な「正しい」「間違い」なんてない。 人間がブタを殺すのはいいのだろうか? 結婚しないのは親不孝なのだろうか? 親の介護をしないのは罪なのか? この世には絶対的に正しいことなんて存在しない。 長い縄に巻かれてそこに拘るような汚物的発想を持つ者は、人間便器になればいい。 トイレの神様になれるよ。 ドM≠マグロ 「愛の反対は憎しみではなく無関心 マザー・テレサ」 同じく、「ドMの反対はドSではなく無反応」である。 無反応のドMなどは調教するに値しない。 なぜなら、女王様を動かしているのもまた、ドMだからである。 無反応のくせに、あるいは欲深くご褒美だけをくれくれとせがむのは、独りよがりの「エゴマゾ」と呼ばれ女王様から嫌われる。 変色能力のない[... 1406 chars omitted ...]
\end{flushleft}

\switchcolumn[1]
\begin{flushleft}
第1章「摘下眼罩」 第1章 眼罩与放任自流 在这个非常识也能成为常识的世界里。 常识和非常识,说到底是谁决定的呢? 只是单纯地被潮流左右,看不到本质的人比比皆是。 在这里,我称这些人为「眼罩奴隶」。 此外,无论学习多少世间流传的成功哲学,如果没有分析其本质的能力,就会像放任自流一样,让知识从身体里流失。这就是所谓的「放任自流的奴隶」。 我正是为了这些「眼罩奴隶」和「放任自流的奴隶」而写这本书。 从世俗的角度来看,SM女王的工作是非常识的。 踩踏、扇耳光、辱骂。即便如此,还能获得高额报酬。 作为身处这样一个与世俗隔绝的世界的人,我可以说,逆向而行能让你看到不一样的风景。 我想将SM俱乐部的经验应用到现实社会中,成为那些烦恼的人的力量。 不,更准确地说,人类的烦恼不过是自己凭空创造出来的。 烦恼的种子这种虚无的东西,只要用颜骑压迫就能碾碎。 但我不会同情。因为这是调教。 调教不需要同情,如果你讨厌了,可以随时逃跑。 但是,只有那些能跟上调教的可爱奴隶,才能得到奖励。 奖励可能不是来自于我,而是来自于其他人。 变态的骄傲 SM并不一定要使用工具。 它是想象与创造的世界,是即兴剧,是一个不知道会发生什么的奇妙世界。 女王们每天都要面对被称为「变态」的人们,因此,当我们深究这个定义模糊的「变态」到底是什么时,结论是「它不存在,只是这样设定」。 一切都源自人类的本能,吸血鬼和杀人狂也是其延伸。 话说回来,平日里女王们用爱称称呼那些以被称为「变态」为荣的「变态」们。 但是,当我看到那些自以为是地说「男人都是变态」之类莫名其妙的话的人时,我认为,把什么都轻易地称为变态,是对真正的变态的失礼。 因为去SM俱乐部的变态们,是全身心地投入到变态行为中的。 他们抛弃了作为男人(女人)的骄傲,取而代之的是作为奴隶的骄傲开始萌芽。 实际上,女王们是怀着敬意欺负那些有骄傲的奴隶的。 你是S?还是M? 经常能看到有人问「你是S?还是M?」这样的问题。 恐怕大多数人都不太明白吧。 或者,也有自称S、自称M的人。 让我告诉你真相。 这取决于对方能激发你什么,它不是一个人能成立的。 自我的存在是由外部创造的,如果这个世界上只有你一个人,没有比较对象,就不能说你是「温柔的人」或「有存在感的人」。 人类的存在价值,只有在有对方的情况下才能被证明。 因此,单独说「我是S」「我是M」这样的定位是不可能的。 更确切地说,既可以说「谁都同时拥有两者」,也可以说「既没有S也没有M」。 那些什么都要分出黑白的人,那些真正在寻找自我的人,那些拘泥于此的人,请读完这本书后重新开始。 温水只适合灌肠 为了传达我想传达的东西,我会说一些极端的话。 说这样的话,也会有人批评「那不是言过其实吗?」。 但是,温水只适合灌肠。 温吞的话语只会和灌肠液一起从下面的口流进马桶。 在这个世界上,如果不极端地发言,就无法传达自己的想法。 以前,我也收到过「你的想法是错误的!」这样的信息,但我只会嗤之以鼻:「试图改变别人而痛苦的人,是你自己啊。」 这个世界上没有绝对的「正确」和「错误」。 人类杀猪是可以的吗? 不结婚是不孝吗? 不照顾父母是罪吗? 这个世界上没有绝对正确的事情。 那些被长绳子束缚,拘泥于这种污秽想法的人,成为人类便器就好了。 你可以成为厕所之神。 抖M≠金枪鱼 「爱的反面不是恨,而是漠不关心——特蕾莎修女」 同样,「抖M的反面不是抖S,而是无反应」。 无反应的抖M不值得调教。 因为,推动女王的也是抖M。 明明无反应,却贪心地只想要奖励的人,被称为自私的「自我主义抖M」,会被女王讨厌。 没有变色能力的[...] (内容省略)
\end{flushleft}

\switchcolumn[0]

\section{第2章 「女王様ストーリー」\par 第2章「女王故事」}

\switchcolumn
\begin{flushleft}
第2章 「女王様ストーリー」
\end{flushleft}

\switchcolumn[1]
\begin{flushleft}
第2章「女王故事」
\end{flushleft}

\switchcolumn[0]

\section{第2章 「女王様ストーリー」\par 第2章「女王故事」}

\switchcolumn
\begin{flushleft}
第2章 「女王様ストーリー 第2章 赤ちゃんプレイ 私が元女王様だということを公言するようになってから、やはり多くの方から「なぜ女王様になろうと思ったの?」と聞かれる。 それは恐らく「女はこうあるべき」という足枷に繋がれた奴隷だったからだろう。 私は平凡な環境で育った平凡な人間だ。 ただ、人より少し好奇心が強いかもしれない。 既製品の型に収まったド真面目な両親を見て、大人になることに対し「絶望」を感じていたものだ。 「大人になったら自由を奪われる」 そんな危機感が常に付きまとっていたため、「自由が奪われるなら、死んだ方がマシ」そんな風に思うようになっていた。 大人になってから、周りから行動派だとか言われるようになり、気付いたのだが、多くの人たちが「やりたいけど、行動できない」と言う。 自由になれなくて愚痴をこぼすのに、自由ではない今の方が、居心地がいいらしい。 つまり、いつまでも責任を負わなくてもいい赤ちゃんでいたい、ということだ。 赤ちゃんプレイなら、SMクラブに行けばいくらでもできるよ。 私の辞書には「いつかやる」「そのうちやる」という言葉はない。 常に「今」に集中し生きてきたからこそ、今があるのだ。 「今」と言い終わった時点で、もうそれは過去になっているのだし、そう考えると未来なんて永遠にやってこない。 誰にも言えなかった「いじめ」 小学生のころ、私は非常におとなしくネクラな子であった。 そのため何か気に食わなかったのか、上級生のある男の子からいつも虐められていた。 よくある好きな子に悪戯をするようなレベルではない。 言葉の暴力にとどまらず、突然、全力腹パンチを食らったこともある。 内臓が暴れ狂うような、あまりの痛さにうずくまりしばらく動けなかった。 だが、弱さを見せたくない私は平気なふりをしていた。 そのことは親はもちろん、誰にも言えなかった。 "強くならなければ"という感情をどこに向けたらよいのか分からなかった。 罰が当たったのかどうか知る由もないが、ある日突然、彼の父親が亡くなった。 たぶん、歳はまだ40くらいだろう。 それ以来、彼からは覇気が消え、一切攻撃を仕掛けてこなくなった。 思えばそれは、私を女王様へと導く第一歩であっただろう。 ストーカー事件 もう一つは、18歳の時だった。 元交際相手がストーカーになったのだ。 彼は「親に殺されそうになった」と血だらけで私を待ち伏せしていたり、「死にたい」というのが日常だった。 精神的にまだ幼かった私は、危機感を感じ別れを切り出した。 すると彼は、直接会わないと気が済まないと言い、殺される覚悟で会うことにした。これでやっと別れられる・・・。 ドンと、私はアスファルトに叩きつけられ殴られアザだらけになった。 そして彼は「死んでやる!」と叫び、道路に飛び出し警察沙汰になった。 その後もしばらくの間、ストーカー行為は続き、「明日、私は殺されるのか?」などと考えながら、毎日を過ごしていた。 その度に、「強くならなければ」と自分に言い聞かせた。 無反応を貫いたところ、ストーカー行為は次第に鎮まった。 その頃の私は、人と関わることに疲れ、一人で過ごすことが多くなった。 鬱と不眠症 就職した時だ。 私の後から入った新人さんたちは、皆すぐに辞めてゆく、という厳しい職人の世界である。 非常にピリピリとした現場だ。 中途半端が嫌いでストイックすぎた性格のせいか、気付けば私は鬱と不眠症で、疲れているのに眠れない日が続いた。 ストレスのせいか食べても胃が受け付けず、吐いてしまうため、気付けば3カ月で体重は15kg減っていた。 思うように体が動かず、頭も働かず、他者から見ても、明らかに精神が病んでいた。 笑顔をつくる表情筋も動かせなくなり、楽しそうに笑っている人を見ると、手の届かない遠い世界に霞んで見えた。 「自分はこんなに弱かったのか」と落胆した。 そして、健全な精神を取り戻すことを優先とし、わずか4カ月で退職した。 いつも強がっている自分は認めたくなかったが、思った以上にメンタルが弱かった。 どうしたら強くなれるのだろうか。 「そうだ、女王様になろう」 新しい仕事も落ち着いたあるとき、突然何かが降りてきた。 そうだ、女王様になろう。 「固定観念という縄に縛られたくない」「支配されたくない」 そんな思いが年月とともに進化したのか私を女王様へと導いた。 SMのことなんぞ全く知らなかったが、一度興味を持ったら引き下がれないタチなのだ。 だが、SM女王様という職業は誰もが軽い気持ちでできるものではない。 人生を捧げる覚悟がなければ、やる資格はない。 思うにSMの世界というものは、マリアナ海溝よりも深く、木星よりも重い世界なのだ。 死への憧れ 「性の快楽は死の疑似体[... 5255 chars omitted ...]
\end{flushleft}

\switchcolumn[1]
\begin{flushleft}
第2章「女王故事 第2章 婴儿play 自从我开始公开宣称自己是前女王以来,很多人都会问我「为什么想要成为女王?」。 这可能是因为我曾经是一个被「女性应该这样」的枷锁束缚的奴隶。 我是在平凡环境中长大的平凡人。 只是,我可能比别人稍微好奇一点。 看着被固定模式束缚的认真的父母,我对长大成人感到「绝望」。 「长大后自由就会被剥夺」 这种危机感一直伴随着我,所以我开始这样想:「如果自由被剥夺,不如死了好」。 长大后,周围的人都说我是行动派,我注意到,很多人都说「想做但做不到」。 明明因为无法自由而抱怨,但不自由的现在似乎更让人感觉舒适。 也就是说,他们想永远做不用负责任的婴儿。 如果想玩婴儿play,可以去SM俱乐部,想玩多少就玩多少。 我的字典里没有「总有一天会做」「过些时候会做」这样的话。 正因为我一直集中在「现在」生活,才有了现在。 当你说完「现在」的时候,它已经成为过去,这么想的话,未来永远不会到来。 无法对任何人说的「欺凌」 小学的时候,我是一个非常老实、软弱的孩子。 不知道是不是因为什么地方惹他不高兴了,一个高年级的男生总是欺负我。 这不是常见的对喜欢的孩子恶作剧的程度。 不仅有语言暴力,还有突然的全力腹拳。 内脏仿佛要暴走一样的疼痛,让我蜷缩着身子好一阵子动弹不得。 但是,不想让人看到我软弱的样子,所以我假装没事。 这件事,不用说父母,我谁都没有告诉。 我不知道应该把「必须变强」的感情投向哪里。 不知道是不是得到了报应,有一天,他的父亲突然去世了。 大概只有40岁左右吧。 从那以后,他失去了霸气,再也没有攻击过我。 现在想来,那可能是引导我成为女王的第一步。 跟踪狂事件 另一件事是在18岁的时候。 我的前男友变成了跟踪狂。 他满身是血地伏击我,说「我差点被父母杀死」,「想死」成了他的日常。 精神上还很幼稚的我,感到了危机感,提出了分手。 然后他说,不见面就不甘心,我做好了被杀的觉悟去见他。这样终于可以分手了…… 咚的一声,我被按倒在柏油路上,被打得浑身是伤。 然后他喊着「我死给你看!」,冲到马路上,引起了警察的注意。 之后一段时间,跟踪狂行为仍在继续,我每天都在想「明天我会被杀死吗?」。 每次这样想,我都会对自己说「必须变强」。 坚持无反应后,跟踪狂行为逐渐平息。 那个时候,我已经厌倦了与人交往,变得经常一个人独处。 抑郁与失眠症 那是我刚就业的时候。 我所在的是一个严格的工匠世界,在我之后进来的新人很快就都辞职了。 这是一个非常紧张的工作环境。 可能是因为我讨厌半途而废、过于禁欲的性格,等我意识到的时候,我已经患上了抑郁和失眠症,虽然疲惫却连续几天睡不着觉。 可能是因为压力,吃了东西胃也不接受,会吐出来,等我意识到的时候,三个月内体重减了15公斤。 身体不听使唤,脑子也转不动,从别人的角度看,我的精神明显生病了。 连笑的表情肌肉都动不了,看到笑得开心的人,感觉他们在遥远的世界里模糊不清。 「我原来这么软弱吗?」我感到非常沮丧。 然后,我把恢复健康的精神状态放在首位,仅仅4个月就辞职了。 虽然不想承认总是逞强的自己,但我的精神比想象中还要脆弱。 怎样才能变强呢? 「对了,成为女王吧」 当新工作也稳定下来的时候,突然有什么东西降临了。 对了,成为女王吧。 「不想被固定观念的绳子束缚」「不想被支配」 这些想法随着年月的增长进化,引导我成为了女王。 虽然我对SM一无所知,但一旦产生兴趣,就无法退缩。 但是,SM女王这个职业不是任何人都能轻松胜任的。 没有奉献一生的觉悟,就没有资格做。 我认为SM的世界比马里亚纳海沟更深,比木星更重。 对死亡的憧憬 「性的快乐是死亡的拟态[...] (内容省略)
\end{flushleft}

\switchcolumn[0]

\section{第3章 「自己啓発なんて役立たず」\par 第3章「自我启发毫无用处」}

\switchcolumn
\begin{flushleft}
第3章 「自己啓発なんて役立たず」
\end{flushleft}

\switchcolumn[1]
\begin{flushleft}
第3章「自我启发毫无用处」
\end{flushleft}

\switchcolumn[0]

\section{第3章 「自己啓発なんて役たたず」\par 第3章「自我启发毫无用处」}

\switchcolumn
\begin{flushleft}
第3章 「自己啓発なんて役たたず」 第3章 SMは言葉の世界 人それぞれ持論はあるかもしれないが、私はSMの世界は「言葉の世界」だと思っている。 正直、ロウソクや鞭なんてただのアクセサリーだ。 もちろん言葉責めなんて要らないという人もいるし、蔑むような眼差しだとか、触れ方や勘によるもの、それら非言語要素も不可欠だ。 だが、調教の中で交わされる会話、卑猥な言葉を発するとき、即ち言語化することこそ相手の臨場感が高まるのだ。 新人の頃、SMクラブの店長が言っていた。 「SMはね、言葉責めが9割だから。言葉責めをよーく勉強するように」と。 要するに、イカすも殺すも言葉次第であり、その言葉を操る者に染み込ませなければならない。 言葉を操れない者は、人を動かせない。 いや、しかし存在だけで語る者もいるじゃないか、と言うが「存在」も言葉である。 言葉責めを取り扱う女王様たちは、日頃から相手の心に響く「言葉」について、多かれ少なかれ研究しているのである。 言葉は嘘の世界 私たちは、他人からのちょっとした一言によって傷ついたり、イラついたり、トラウマになったりすることがある。 人の感情を支配する「言葉」というものは恐ろしい。 けれど実は言葉そのものは意味を持たないのである。 言葉と意味は別であり、私たちはそれを勝手に結びつけているのだ。 例えば、「苦しい」という言葉を知らなかったとしよう。 そうすると「苦しい」という感情は湧いてこない。なぜなら「苦しい」が何なのか分からないからだ。 ひょっとしたらこの感情は「楽しい」なのかもしれない。 言葉に勝手に感情を吹き込んでいるのは、「自分」だったのだ。 ならば、あなたの都合で「言葉」と「意味」を切り離せばいい。 罵倒も嫌味も、人間と思えない言葉も、ただの「言葉」だ。 落ち込んだ、傷ついた、そう感じたならば思い出すといい。 「言葉」と「意味」は別のもの。あなたは自在に操る権利がある。 言葉というものはじつに曖昧なものである。 私たちがいるこの世界は、言葉と本能に支配された「嘘の世界」だったのだ。 例えば私は、「我慢」という言葉を消し去った。なので「我慢?なにそれ美味しいの?」となるのだ。 要らない言葉は自分の辞書から消してしまえばいい。 言葉の意味を深く噛みしめるのは、心ある言葉と女王様の言葉責めだけでよい。 裸の王様的露出狂 根拠もなく低レベルな批判をしたがる薄っぺらなカスどもが存在する。 例えば書籍などのレビューなどを見ていると、「目新しいことは書いてなかった。がっかりです」「酷い内容だ」などの上から目線の感想を書いている人を見かける。 だが、そんな輩たちは自分の半径1メートルでしか物事を見ていない。 その背景に何があるとか、そういったことを想像する力がないのである。 つまり、読解力のなさ、愚かさを自ら晒している露出狂だ。 恥ずかしい姿を見せつけたい変態露出狂なら、喜んで虐めてあげるが、笑われていることに気づいていない「裸の王様的露出狂」は、調教するに値しない。 感度の高い賢者たちは、一見無駄なこと、意味のないようなことからも、敏感に感じ取りどんどん進化してゆく。 あらゆる問題の解決ができるかどうかは、女王様による上から目線(抽象的な思考)ができるかによって決まる。 だが、不感症の愚者たちは、どこで何をしていても何も生まれない時間を過ごす。 本書を読めば、誰が「裸の王様的露出狂」かが分かるだろう。 批判の仕方ひとつで人間レベルがバレてしまうのだ。 私は、ここにもそんな、おっぴろげの露出狂が現れることを期待している。 自己啓発よりも「自己調教」 現在も「なりたい私になる」だとか、「引き寄せ」だとか、自己啓発のキラキラした言葉に惹かれる女子たちが溢れている。 しかし私は自己啓発の本なんてほとんど読んだこともなければ、自己啓発セミナーにも行ったことがなく、そんなもの必要ないと思っている。 それよりも、大金を払ってでもその本質を存在で教えてくれる、よい人脈を作ったほうが早い。 あれだけ数多くの成功本が出ているのに、読んだ人がみんな成功しているわけではないのはなぜだろうか。 その答えを分析することに意味があるのにもかかわらず、一番重要なことは隠されているのだ。 つまり、本当に大切なのは言っていないことであり、それらは演出という劇を見せつけられ、垂れ流し奴隷はなかなか見破れない。 これからは自己啓発ではなく、自分を調教する「自己調教」だ。 SMクラブで見てきたもの、それは人生哲学だった 引退してから気づいたのだが、私がSMクラブで見てきたものは、人生哲学そのものだった。 現場を離れて俯瞰することによって、これまで見えなかった景色が見えてくるものだ。 SMクラブを訪れる人間はなぜ、わざわざ痛み、苦しみを味[... 433 chars omitted ...]
\end{flushleft}

\switchcolumn[1]
\begin{flushleft}
第3章「自我启发毫无用处」 第3章 SM是语言的世界 人们可能各有各的看法,但我认为SM的世界是「语言的世界」。 说实话,蜡烛和鞭子之类的只是装饰品。 当然,也有人不需要语言责备,蔑视的眼神、触摸方式和直觉等非语言要素也很重要。 但是,在调教师生的对话、说出猥亵的话语时,也就是语言化的时候,对方的临场感才会提高。 新人的时候,SM俱乐部的店长说过:「SM啊,语言责备占90%。所以要好好学习语言责备。」 也就是说,是让对方欲仙欲死还是杀死对方,全看语言,必须让语言渗透到操控它的人心中。 无法操控语言的人,无法感动别人。 不,但是也有只用存在就能说话的人吧?其实「存在」也是语言。 处理语言责备的女王们,平时多少都会研究能触动对方心灵的「语言」。 语言是谎言的世界 我们可能会因为别人的一句话而受伤、生气、产生创伤。 支配人的感情的「语言」是可怕的。 但是,实际上语言本身是没有意义的。 语言和意义是分开的,是我们擅自把它们联系在一起的。 例如,假设你不知道「痛苦」这个词。 那么,你就不会产生「痛苦」的感情。因为你不知道「痛苦」是什么。 说不定这种感情其实是「快乐」。 擅自给语言注入感情的,是「自己」。 那么,根据你的方便,把「语言」和「意义」分开就好了。 辱骂也好,讽刺也好,不像人类说的话也好,都只是「语言」。 如果你感到沮丧或受伤,就记住这一点。 「语言」和「意义」是不同的东西。你有自由操控它们的权利。 语言其实是非常模糊的东西。 我们所在的这个世界,是被语言和本能支配的「谎言的世界」。 例如,我已经消除了「忍耐」这个词。所以对我来说,「忍耐?那是什么好吃的吗?」 不需要的语言,从自己的字典里删掉就好了。 只有有心意的话语和女王的语言责备,才值得深入品味其意义。 皇帝的新衣式暴露狂 存在一些毫无根据就喜欢进行低水平批判的肤浅垃圾。 例如,看书籍等的评论时,会看到有人写「没有写新的东西。很失望」「内容太差了」等高高在上的感想。 但是,这些家伙只能看到自己半径1米范围内的事物。 他们没有想象背景有什么的能力。 也就是说,他们是暴露自己没有解读能力、愚蠢的暴露狂。 如果是想展示自己羞耻姿态的变态暴露狂,我很乐意欺负他们,但没有意识到自己被嘲笑的「皇帝的新衣式暴露狂」,不值得调教。 敏感度高的贤者们,能从看似无用、无意义的事情中,敏锐地感知并不断进化。 能否解决所有问题,取决于能否像女王一样从高处俯视(抽象思考)。 但是,无感的愚者们,无论在哪里做什么,都在度过什么也不会产生的时间。 读完这本书,你就会知道谁是「皇帝的新衣式暴露狂」了。 一个批判的方式就能暴露一个人的水平。 我期待着这里也能出现这样的暴露狂。 与其自我启发不如「自我调教」 现在,仍然有很多女性被「成为想成为的自己」「吸引力法则」等闪闪发光的自我启发语言所吸引。 但是,我几乎没有读过自我启发的书,也没有参加过自我启发研讨会,我认为没有必要。 与其那样,不如花大价钱建立人脉,让真正了解本质的人亲自教导你,这样更快。 为什么那么多成功书籍出版,而读过的人却不都成功呢? 虽然分析这个答案很有意义,但最重要的事情却被隐藏了。 也就是说,真正重要的是没有说出来的东西,这些东西被当作表演展示给你,放任自流的奴隶很难识破。 从今以后,不是自我启发,而是调教自己的「自我调教」。 在SM俱乐部看到的东西,其实是人生哲学 引退后我才意识到,我在SM俱乐部看到的东西,其实就是人生哲学本身。 离开现场,从高处俯瞰,就能看到之前看不到的景色。 为什么访问SM俱乐部的人,要特意去体验疼痛和痛苦呢?[...] (内容省略)
\end{flushleft}

\switchcolumn[0]

\section{第4章 「痛み・苦しみを飼いならせ」\par 第4章「驯服疼痛与痛苦」}

\switchcolumn
\begin{flushleft}
第4章 「痛み・苦しみを飼いならせ」
\end{flushleft}

\switchcolumn[1]
\begin{flushleft}
第4章「驯服疼痛与痛苦」
\end{flushleft}

\switchcolumn[0]

\section{第4章 「痛み・苦しみを飼いならせ」\par 第4章「驯服疼痛与痛苦」}

\switchcolumn
\begin{flushleft}
第4章 「痛み・苦しみを飼いならせ」 第4章 汚されたい願望 先日、ある女性タレントで炎上クイーンのトークショーを拝む機会があった。 彼女はネット上で誹謗中傷を投げかけて来る相手に対し、「どうやったら好きになってもらえるだろうか」と考えるという。 元女王様である私には理解できないが、自分としては「どうやったらもっと嫌われるだろうか」とその演出を考える方が圧倒的に楽しい。 それにこの世の中は、好かれようとすればするほど嫌われるという法則がある。 同様に、「美しくなりたい」と願うのは、醜さを隠したいという内なる願望であり、反対に、「蔑まれるほど汚されたい」と願うドMは、清く美しいからこそのないものねだり、とも言える。 心の汚れたやつは、汚れは欲しがらない。 言っていることよりも、言っていないことの方が、じつは重要なのである。 従って、本当に重要なことは成功法則の本には書いていないし、本当に重要なことはニュースでは報道されない世の中なのだ。 だからと言って、美しくなりたい願望を否定する必要はないし、そんなことはどうでもいいんだよ。 美しさの基準も自分で決めればいい。 自分の尻にスパンキングしろ 行動できないだとかいう声をよく聞くが、そもそも人間は追い込まれないと行動できない。 その原動力が「痛み」や「苦しみ」なのだ。 もし今、痛みや苦しみの渦中にいるとすれば、喜ぶがいい。 なぜなら、幸福感、快楽を得るには痛み、苦しみが不可欠だからだ。 一方の振れ幅が大きいほど、両極にまで視野は広がり、それは物事を俯瞰する抽象的思考に繋がるのだ。 反対に、痛みや苦しみに鈍感になる時期もある。 そんなときは敢えて自分の尻にスパンキングをするのだ。 それと同時に、尻をさらけ出すという羞恥プレイも待っていて楽しいではないか。 もちろん例えだが、選択肢は一つというところまで追い込むことにより、行動できないわけがない、逃げられない状況にするのだ。 ただ、追い込む方向を間違って、心が病んでしまう人もいる。 痛みを飼いならすには、己の中の女王様から育成しなければならない。 「人として弱いということは、生きていく上で受けるべき苦しみを自分で受け取ろうとしないことだ。――ウィトゲンシュタイン」 “自分にご褒美”という発想は捨てろ 世の女子たちが愛してやまない“自分にご褒美“。 だが私は、自分にご褒美という発想が分からない。 生きているだけでご褒美ではないだろうか。 目の前にあるご褒美に気づかない“目隠し奴隷”どもは、ご褒美はまだかとヨダレを垂らしながら、嫌なことを我慢しながら生きている。 そして我慢、我慢、で結局、陰で愚痴を言い、「可哀想な自分」を演出し、ご褒美を追いかけ回している。 しかし、そのプロセスを俯瞰し楽しむこと、それ自体がご褒美であることに気づかない。 自分の意思で好きでやっているのだから、女王様は奴隷を可哀想だなんて思わないし、「痛みを与えてもらえるなんて、なんて幸せなのだろうか」としか思えない。 奴隷からすれば、“がんばっている自分”に陶酔しながら精神が鍛えられ、そのプロセスによって自己評価が高まるのである。 成功者は常にプロセスを楽しみ、一方で敗者はプロセスを軽視し墓場へと向かう。 女王様に甘えるのはいいけれど、自分に甘えていたらご褒美は受け取れないんだよ。 つまり、SだとかMだとか薄っぺらなことを言い、SMの世界を軽視する者は、成功できないのである。 自己調教のすすめ 世の中を上から目線で見下ろすには、自分の中に女王様目線を置くことだ。 私はこれを「自己調教」と呼んでいる。 例えば、誰しも一度は自分の中に、2人の声を聞いたことがあるだろう。 「ここは嘘を言っとけ」「いや、嘘はよくないよ」 どちらの声に従えばいいのか、というような場面だ。 ただし、この一人会話は同じ次元で繰り広げられているため、奴隷同士の会話だ。 「自己調教」の場合、同次元に居る二人の奴隷の会話を女王様が調教しなければならない。 つまり、「上から目線」でなければ自分を調教できないのだ。 よって、どちらかを選ぶのではなく、そもそもその迷いが発生する根源を尖ったヒールで踏みつぶしてしまえばいい。 女王様ならば、「言ったことを、後から現実にすればいい」のである。 それはできません、などと言い訳ばかりしていると、女王様から見放されるのだよ。 「ほら、できたじゃないの」 結局、全ては自らが作り出した虚像であり、あなたは「一人SM劇場」のキャストだったのだ。 【調教ワーク 4】 ・あなたが最近、一人会話をしたのはどんな時か? ・そのとき女王様なら何と言うだろうか? まとめ 扉の向こうで繰り広げられるSM調教を、現実社会に落とし込んだ「自己調教」は、痛みや苦[... 484 chars omitted ...]
\end{flushleft}

\switchcolumn[1]
\begin{flushleft}
第4章「驯服疼痛与痛苦」 第4章 想要被玷污的愿望 前几天,我有机会观看了一位女性艺人的脱口秀,她被称为炎上女王。 她对于在网上对她进行诽谤中伤的人,会思考「怎样才能让他们喜欢我呢?」。 作为前女王,我无法理解这一点,但对我来说,思考「怎样才能让他们更讨厌我呢?」的演出方式,要有趣得多。 而且这个世界有一条法则:越想被喜欢,就越会被讨厌。 同样,「想要变得美丽」的愿望,是内心想要隐藏丑陋的愿望;相反,「想要被轻蔑地玷污」的抖M,也可以说是因为太过清美而产生的无理要求。 内心污秽的家伙,不想要被玷污。 比说出来的话更重要的,是没有说出来的话。 因此,真正重要的事情,不会写在成功法则的书里;真正重要的事情,新闻也不会报道。 话虽如此,没有必要否定想要变得美丽的愿望,那种事怎样都好。 美丽的标准,自己决定就好。 给自己的屁股一顿鞭打 经常听到「无法行动」之类的声音,但人类本来就是不被逼到绝境就无法行动的生物。 其原动力就是「疼痛」和「痛苦」。 如果你现在正处于疼痛和痛苦的漩涡中,那应该高兴才对。 因为,要获得幸福感和快乐,疼痛和痛苦是必不可少的。 一边的振幅越大,视野就会扩展到两极,这会导致俯瞰事物的抽象思考。 相反,也会有对疼痛和痛苦变得迟钝的时期。 那种时候,就大胆地给自己的屁股一顿鞭打吧。 同时,露出屁股的羞耻play也在等待着你,不是很有趣吗? 当然,这只是一个例子,但通过把选择逼到只剩一个,就不会无法行动,也无法逃跑了。 但是,也有人因为逼错了方向,而使心灵生病。 要驯服疼痛,必须从自己内心的女王开始培养。 「作为人类的软弱,是指不愿意自己承担生活中应该承受的痛苦。——维特根斯坦」 放弃「奖励自己」的想法 世间的女性们都非常喜欢「奖励自己」。 但是,我无法理解「奖励自己」的想法。 仅仅活着,难道不就是奖励吗? 没有注意到眼前奖励的「眼罩奴隶」们,流着口水等待奖励,一边忍受着讨厌的事情一边生活。 然后,忍受、忍受,结果在背后抱怨,演出「可怜的自己」,追逐着奖励。 但是,他们没有意识到,俯瞰并享受这个过程本身就是奖励。 因为是按照自己的意愿做喜欢的事情,所以女王不会觉得奴隶可怜,只会想「能被给予疼痛,该是多么幸福啊」。 从奴隶的角度来看,在陶醉于「努力的自己」的同时,精神得到了锻炼,这个过程提高了自我评价。 成功者总是享受过程,而失败者则轻视过程,走向坟墓。 依赖女王没关系,但如果依赖自己,就无法获得奖励。 也就是说,那些说S、M之类肤浅的话,轻视SM世界的人,是无法成功的。 自我调教的建议 要从高处俯视世界,必须在自己内心设置女王般的视角。 我把这称为「自我调教」。 例如,谁都曾经在自己内心听到过两个人的声音吧。 「这里可以撒谎」「不,撒谎不好」 不知道应该听从哪一方的声音。 但是,这种一人对话是在同一个维度展开的,所以是奴隶之间的对话。 在「自我调教」的情况下,女王必须调教在同一个维度的两个奴隶的对话。 也就是说,只有「从高处俯视」,才能调教自己。 因此,不是选择其中一方,而是用尖锐的高跟鞋踩踏产生这种迷茫的根源就好了。 如果是女王的话,「只要把说过的话,后来变成现实就好了」。 如果总是找借口说「那不行」,就会被女王抛弃。 「看,不是做到了吗?」 结果,一切都是自己创造的虚像,你是「一人SM剧场」的演员。 【调教作业4】 ・你最近什么时候进行过一人对话? ・那个时候,女王会说什么呢? 总结 将在门的另一边展开的SM调教,融入现实社会的「自我调教」,是驯服疼痛和[...] (内容省略)
\end{flushleft}

\switchcolumn[0]

\section{あとがき\par 后记}

\switchcolumn
\begin{flushleft}
あとがき あとがき Mは痛いのが好きとか、Sはいじめるのが好きとか、そういう次元のものではない、ということがお解り頂けただろうか。 イメージだけが浸透していると同時に、視野が狭くなりがちな世の中に向けて、私はどうしても伝えたかった。 心が弱ったとき、多くの方は、「そのままでいいんだよ」とか、優しい言葉で慰めを求めているかもしれないが、そのままなら何も変わらない。 優しい言葉が逆に現実とのギャップを作ってしまい、そこから永遠に抜けられないのだ。 今、不安や絶望を感じているなら、拷問プレイだと思ってじっくり味わえばいい。 ヘタに逃げようとすると、どんどん迫ってきて潰されてしまうからだ。 この痛み、苦しみは全て調教。 それは、ぎりぎり乗り越えられる者にしか与えられない、ご褒美だ。 「自己調教」 これは、SMクラブという多種多様な人間が集う場で、人間の本能と本質に向き合ってきた私が生み出した独自の思想である。 この道具を使ってどう快楽を得るかはあなた次第だ。 人間に生まれて本当によかった。 最後までお読みくださり、ありがとうございます。 2018年4月8日 紗弥 【紗弥 公式サイト】 http://smsamsara.xsrv.jp/
\end{flushleft}

\switchcolumn[1]
\begin{flushleft}
后记 后记 您应该已经明白了,M不是喜欢疼痛,S不是喜欢欺负人,不是那个维度的东西。 面对这个印象深入人心,同时视野越来越狭窄的世界,我无论如何都想传达这一点。 当心灵变得脆弱时,很多人可能会寻求「就这样就好」之类的温柔话语的安慰,但如果就这样下去,什么都不会改变。 温柔的话语反而会与现实产生差距,让人永远无法从中逃脱。 如果你现在感到不安或绝望,就把它当作拷问play,慢慢品味吧。 因为如果笨拙地想要逃跑,只会越来越被逼近,最终被压垮。 这些疼痛和痛苦,都是调教。 这是只有勉强能克服的人才能获得的奖励。 「自我调教」 这是我在SM俱乐部这个聚集了各种各样的人的地方,面对人类的本能和本质,所产生的独特思想。 如何使用这个工具获得快乐,取决于你自己。 能生而为人真是太好了。 感谢您读到最后。 2018年4月8日 纱弥 【纱弥 官方网站】 http://smsamsara.xsrv.jp/
\end{flushleft}

\switchcolumn[0]

\section{プロフィール\par 作者简介}

\switchcolumn
\begin{flushleft}
プロフィール ♦著者略歴 紗弥 元SMクラブ女王様 No.1ホステス、作家、投資家 メンタルが弱く何度か鬱を繰り返し、引きこもる。 会社勤めが性に合わず、2年で辞めてフリーターに。 音楽活動をしながらバックパッカーでインドを放浪する。 28歳のときSMクラブ女王様になり、約1700人の奴隷たちをひれ伏させる。 引退後、女王様であったことを封印しながら社会に溶け込み、エステサロンを開業するが、赤字が続き貧困状態に陥る。 恋人の死をきっかけに東京へ拠点を移し、生き埋めにしていた女王様思考を掘り起こす。 執筆活動の傍ら、夜の蝶として約300人の奴隷志願者を携え不動のNo.1になる。 大富豪、投資家、不動産王たちとの交流もあり、自らも投資家として世界の経済を見下ろしている。 初の電子書籍「SM奴隷から学ぶ「うつ」の治し方」は、人生に彷徨う者たちに多大な影響を与えている。 尊敬する実業家・・・アニータ・ロディック 好きな作家・・・谷崎潤一郎、種村季弘 好きな音楽・・・ジャズ、中東音楽、世界各地の民族音楽、60年代ロック、ソウル、R&Bなど
\end{flushleft}

\switchcolumn[1]
\begin{flushleft}
作者简介 ♦ 作者履历 纱弥 前SM俱乐部女王 No.1女招待、作家、投资者 精神脆弱,多次抑郁并隐居。 不适合上班族生活,2年后辞职成为自由职业者。 一边进行音乐活动,一边作为背包客游历印度。 28岁时成为SM俱乐部女王,让约1700名奴隶俯首称臣。 引退后,隐藏自己曾是女王的身份融入社会,开设了美容沙龙,但持续亏损陷入贫困状态。 以恋人的死亡为契机,将据点转移到东京,挖掘出被活埋的女王思维。 除了写作活动外,作为夜之蝶,带领约300名奴隶志愿者成为不可动摇的No.1。 与大富豪、投资者、不动产王等有交流,自己也作为投资者俯视世界经济。 首部电子书《从SM奴隶中学到的「抑郁」治疗方法》,对在人生中彷徨的人们产生了巨大影响。 尊敬的实业家……安妮塔·罗迪克 喜欢的作家……谷崎润一郎、种村季弘 喜欢的音乐……爵士、中东音乐、世界各地的民族音乐、60年代摇滚、灵魂乐、R&B等
\end{flushleft}

\switchcolumn[0]

\section{奥付け\par 版权页}

\switchcolumn
\begin{flushleft}
奥付け ─────────────────────── 『SM奴隷から学ぶ「うつ」の治し方』 1700人の奴隷を従えた元女王様が教える「自己調教」のススメ 2018年5月04日 発行 初版 著者:紗弥 カバーデザイン:澤田哲志 製本:佃耕自 Copyright (C) 2018 紗弥 All Rights Reserved. ───────────────────────
\end{flushleft}

\switchcolumn[1]
\begin{flushleft}
版权页 ─────────────────────── 《从SM奴隶中学到的「抑郁」治疗方法》 1700名奴隶的前女王传授的「自我调教」建议 2018年5月04日 发行 初版 作者:纱弥 封面设计:泽田哲志 装订:佃耕自 Copyright (C) 2018 纱弥 All Rights Reserved. ───────────────────────
\end{flushleft}

\end{paracol}

\end{document}