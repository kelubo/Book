% 木偶奇遇记
% 木偶奇遇记.tex

\documentclass[12pt,UTF8]{ctexbook}

% 设置纸张信息。
\usepackage[a4paper,twoside]{geometry}
\geometry{
	left=25mm,
	right=25mm,
	bottom=25.4mm,
	bindingoffset=10mm
}

% 设置字体,并解决显示难检字问题。
\xeCJKsetup{AutoFallBack=true}
\setCJKmainfont{SimSun}[BoldFont=SimHei, ItalicFont=KaiTi, FallBack=SimSun-ExtB]

% 目录 chapter 级别加点(.)。
\usepackage{titletoc}
\titlecontents{chapter}[0pt]{\vspace{3mm}\bf\addvspace{2pt}\filright}{\contentspush{\thecontentslabel\hspace{0.8em}}}{}{\titlerule*[8pt]{.}\contentspage}

% 设置 part 和 chapter 标题格式。
\ctexset{
	chapter/name={第,章},
	chapter/number={\arabic{chapter}}
}

% 设置古文原文格式。
\newenvironment{yuanwen}{\bfseries\zihao{4}}

% 设置署名格式。
\newenvironment{shuming}{\hfill\bfseries\zihao{4}}

% 注脚每页重新编号,避免编号过大。
\usepackage[perpage]{footmisc}

\title{\heiti\zihao{0} 木偶奇遇记}
\author{[意]卡洛·科洛迪}
\date{1880}

\begin{document}

\maketitle
\tableofcontents

\frontmatter

\chapter{前言}

《木偶奇遇记》是科洛迪的代表作,发表于1880年。它叙述老人吉佩托把一块能哭会笑的木头雕成木偶,并把取得生命的小木偶当成儿子。老人卖掉上衣,供儿子上学。可是小木偶一心贪玩,为了看戏不惜卖掉课本。在木偶戏班获得好心老板的五枚金币,回家路上却遭遇狐狸和猫的欺骗,金币被抢走。随后,又因贪吃他人的葡萄被捕兽器夹住,被迫当了看家狗。当他一心想成为一个用功读书的好孩子时,却又被引诱到玩具国,在疯狂的玩了五个月之后,变成一头又懒又蠢的驴……他到底能不能变成真正的小男孩,并且重回父亲身边呢?

这本书描述了小木偶匹诺曹从一个任性、淘气、懒惰、爱说谎、不关心他人、不爱学习、整天只想着玩的小木偶,变成一个懂礼貌、爱学习、勤奋干活、孝敬长辈、关爱他人的好孩子的过程,以及他所经历的一连串的奇遇,充满了童趣与想象。发生于匹诺曹身上的故事告诉我们,一个孩子的自然天性在许多方面都是需要修正的。也就是说,在自然天性里往往会有不少不够尽善尽美的表现,等待着我们的逐步克服。

\mainmatter

\chapter{}

从前有……

"有一个国王!"我的小读者马上要说。

不对,小朋友,你们错了,从前有一段木头。

这段木头并不是什么贵重木头,就是柴堆里那种普通木头,扔进炉子和壁炉生火和取暖用的。

我也不知道是怎么回事,总之有一天,这段木头碰巧到了一位老木匠的铺子里,这位老木匠名叫安东尼奥,大伙儿却管他叫樱桃师傅,叫他樱桃师傅,因为他的鼻尖红得发紫,再加上亮光光的,活像一个熟透了的樱桃。
樱桃师傅看见这段木头,高兴极了,他满意得一个劲儿搓着手,低声嘟嚷说:
"这段木头来得正好,我要拿它做条桌子腿。"
说干就干,他马上拿起一把锋利的斧子,动手就要削掉树皮,先大致砍出条桌子腿的样子。可他第一斧正要砍下去,手举在头顶上却一下子停住不动了,因为他听见一个很细很细的声音央求他说:
"可别把我砍得太重了!"
诸位想象一下吧,樱桃师傅这位善良的老头儿该是多么惊讶啊!
他一双眼睛吓傻了,满屋子骨碌碌转了一圈,要看看这个声音是打哪儿来的,可他一个人也没有看见!他往工作台底下看看,没有人,他打开一直关着的柜子看看,没有人;他往一篓刨花和碎木片里面看看,也没有人;他甚至打开铺子门往街上看看,还是没有人!那么……?
"我明白了,"他于是抓抓头上的假发,笑着说,"这声音一准是我听错了。我还是干我的活吧,"
他重新拿起斧子,在那段木头上狠狠地一斧砍下去。
"唉哟!你把我砍痛了!"还是那很细的声音埋怨着叫起来。
这一回樱桃师傅当真愣住了,眼睛吓得鼓了出来,嘴巴张得老大,舌头拖到下巴,活像喷水池里一个妖怪的石头像。
等到他重新能够说话,他吓得哆哆嗦嗦、结结巴巴地说了起来:
"这个细声细气叫'唉哟'的声音,它到底是打哪儿来的呢?……屋子里可是一个人也没有。难道是这段木头,是它学会了像小娃娃那样又哭又叫吗?这我可怎么也不相信。瞧,就是这么一段木头。它跟别的木头一模一样,拿来生炉子的。扔到火里,倒可以烧开一锅豆子……那么,不是木头又是什么呢,难道是木头里躲着个人吗,要真躲着人,那他就活该倒霉,我这就来跟他算账!"
他这么说着,双手抓住这段可怜的木头,一点不客气,就把它往墙上撞。
撞了一会儿,他停下来竖起耳朵细细地听,看有什么哭声没有,他听了两分钟,没有,听了五分钟,没有,听了十分钟,也没有!
"我明白了,"他一面苦笑着说,一面抓头上的假发,"那细声细气地叫'唉哟'的声音,一准是我自己听错了!我还是干我的活吧,"
可他心里仍然挺害怕,于是试着伊伊唔唔地哼支小调壮壮胆。
这一回他放下斧子,拿起刨子,要把木头刨刨平,可他一来一去地刚那么一刨,又听见那个很小很小的声音嘻嘻地笑着对他说了:
"快住手!你弄得我浑身怪痒痒的!"
可怜的樱桃师傅这一回活像着了雷打,扑通一声倒了下来。等他重新张开眼睛,只见自己坐在地上。
他脸都变了色,一向红得发紫的鼻尖,这会儿都吓得发青了。

\chapter{}

正在这节骨眼,有人笃笃笃敲门。
"进来",老木匠说,他连重新站起来的力气也没有了,
于是木匠铺里进来了一个小老头,他老是老,可老得精神,他的名字叫做杰佩托,可街坊邻居的孩子要想逗他发顿脾气,就叫他的外号"老玉米糊",他有这么个外号,因为他那头黄色假发活像玉米糊。
杰佩托脾气挺坏,谁叫他"老玉米糊"就得倒大霉!他一下子凶得像只野兽,谁也没法对付他。
"您好,安东尼奥师傅。"杰佩托说,"您坐在地上干吗呀?"
"我吗,我在教蚂蚁做算术哪。"
"祝您成功!"
"倒是什么把您给带到我这儿来啦,杰佩托老朋友。"
"是我的腿把我带来了呗,您知道,安东尼奥师傅,我是来求您给我帮个忙的。"
"随时乐意为您效劳。"老木匠回答说,跪了起来。
"今天早晨,我脑子里忽然想出了一个主意。"
"咱们倒来听听看。"
"我想亲手给自己做个漂亮的木偶,不是个普通木偶,是个呱呱叫的木偶,会跳舞,会耍剑,还会翻跟头。我要带着这么个木偶周游世界,挣块面包吃吃,混杯酒喝喝。您看怎么样。"
"好极了,老玉米糊!"还是那个很细很细的声音不却从哪儿叫起来。
杰佩托这位老朋友一听人家叫他老玉米糊,脸登时气红了,红得像个红辣椒。他向老木匠一下子转过脸来,气虎虎地说:
"您干吗得罪我,"
"谁得罪您了,"
"您叫我老玉米糊!……"
"我没叫过您老玉米糊。"
"难道是我叫了吗?我说是您叫了。"
"我没叫!"
"您叫了!"
"我没叫!"
"您叫了!"
他们越来越激动,结果从动口到动手,两个打了起来,又抓又咬,像两只猴子似的。
等到一架打完,杰佩托那头黄色假发到了安东尼奥师傅的手上,老木匠那头花白假发却在杰佩托的嘴里。
"你把我的假发还我,"安东尼奥师傅说。
"你也把我的假发还我。咱俩讲和吧,"
两位小老头各自收回了自己的假发以后,互相紧紧拉手,赌咒发誓说以后要一辈子做好朋友。
"那么,杰佩托老朋友",老木匠表示和解说,"您要我给您效什么劳呢?"
"我想要段木头做我的那个木偶,您肯给吗?"
安东尼奥师傅听了这话真是喜出望外,马上过去拿起工作台上那段把他吓了个半死的木头,可他正要把木头交给朋友,木头猛地一扭,打他手里使劲滑了出来,在可怜的杰佩托那很细的小腿骨上,狠狠地就是一下。
"唉哟!安东尼奥师傅,您送东西给人家是这么客气的吗?我的脚几乎都给你打瘸了,"
"我发誓我没打您的脚。"
"难道是我打我自己的脚不成!……"
"全怪这木头,是它打你的……"
"我知道是木头,可把木头扔在我脚上的是您,"
"我没扔您!"
"您说谎!"
"杰佩托,您别得罪我,要不我就叫您老玉米糊!……"
"蠢驴!"
"老玉米糊!"
"蠢猴!"
"老玉米糊!"
"蠢猪!"
"老玉米糊!"
杰佩托听到这第三声老玉米糊,眼睛都气黑了,向老木匠猛扑过去。于是他们又打了一场大架。
等到这一架打完,安东尼奥师傅的鼻子多了两道抓伤,另一位的背心却少了两颗钮子,两个人这样算清账以后,又紧紧拉手,赌咒发誓说发后要一辈子做好朋友。
接着杰佩托拿起他那段呱呱叫的木头,谢过安东尼奥师傅,一瘸一拐地回家去了。

\chapter{}

杰佩托住在一间很小的地下室,只有楼梯底行道进来一点儿光。用具简单得不能再简单,只有破破烂烂的一把椅子、一张床、一张小桌子。里面墙上有个小壁炉,生着火,可火是画出来的,火上面有个锅子,锅子也是画出来的,锅子在滚得热气腾腾,热气同样是画出来的,可画得跟真的一模一样。
杰佩托一回家,马上拿起工具,动手就刻他的木偶。
"给他取个什么名字呢?"杰佩托自言自语说,"我就叫他皮诺乔吧。这个名字会给他带来幸福。我认识一家人,都叫皮诺乔:皮诺乔爸爸,皮诺乔妈妈,皮诺乔老大、老二、老三……他们一家都过得很好,其中最富的一个讨饭吃。"
杰佩托给木偶取好了名字,就埋头干起活来,一下子就给他刻出了头发,刻出了脑门,刻出了眼睛。
眼睛刚刻好,请诸位想象一下杰佩托有多么惊奇吧,他发觉这两只眼睛自己骨碌碌动起来,接着一眨也不眨地瞪着他看。杰佩托给这双木头眼睛瞪得受不住了,生气地说:
"木头傻眼睛,干吗瞪着我?"
没有回答。
做完眼睛,又做鼻子。鼻子刚做好,它就开始长起来,长啊,长啊,长啊,才几分钟,已经变成一个很长很长的长鼻子,还没完没了地长下去。
可怜的杰佩托拼命要把鼻子截短,可他越是截,这个鼻子就毫不客气地变得越是长。
做完了鼻子做嘴巴。
嘴巴还没做完,就马上张开来笑了,
"别笑!"杰佩托生气地说。可他这句话像是对着墙说的,说了也是白搭。
"我再说一遍,别笑!"他用吓唬他的口气大叫。
嘴巴于是停了笑,可整条舌头都伸出来了,
杰佩托为了不耽误工作,假装没看见,继续干他的活。
做完嘴巴做下巴,接着做脖子,做肩膀,做肚子,做胳膊,做手。
手刚做好,杰佩托就觉得头上的假发套给拉掉了。他抬头一看,可是看见什么啦?只见他那头黄色假发拿在木偶的手里。
"皮诺乔!……马上把头发还我!"
可皮诺乔不但不把假发还他,反把它戴到自己头上。假发把他整个头套住,几乎把他闷了个半死。
木偶这么没规没矩,杰佩托觉得有生以来还没有这样悲伤难受过。他转脸向皮诺乔说:
"你这个小坏蛋!还没把你做完,你已经这样不尊敬父亲了!真坏,我的孩子,你真坏!"
他擦掉眼泪。
接下来只剩下做腿,做脚了。
杰佩托把脚一做好,就感到鼻尖上给踢了一脚。
"我这是自作自受!"杰佩托自言自语,"一开头就该想到这一点!现在已经来不及了!"
他抱住木偶的肢窝,把他放在地板上,要教他走路。
皮诺乔的腿僵硬着,不会动。杰佩托搀着他的手,教他一步一步地走。
等到腿一会动,皮诺乔就开始自己走了,接着他满屋子乱跳,最后跑出大门,蹦到街上,溜走了。
可怜的杰佩托在他后面追,可是追不上,因为皮诺乔这小坏蛋蹦蹦跳跳,像只野兔。他那双木脚却在路面上劈劈啪啪,活像二十双农民的木头鞋在响。
"抓住他!抓住他!"杰佩托大叫。可街上的人看见木偶跑得像匹小马驹,只是停下来望着他出神,哈哈地笑啊笑啊,笑得无法形容。
幸亏最后碰到一个警察,他听到人们吵吵闹闹,以为是一匹马驹从主人手里逃走了,于是大胆地站在路当中,跨开一双粗腿,决心要把马拦住,免得闯大祸。
皮诺乔远远看见警察把整条街拦住,就想在他两腿之间一下子冲过去,可是没成功,
警察动也不用动,一把就抓住了他的鼻子(这个鼻子真长、像是特地做出来给警察抓的),把他交还到杰佩托手里,杰佩托为了教训他,马上想狠狠拉他的耳朵,可诸位想象一下他是多么惊讶吧:他找来找去竟行不到耳朵,诸位知道为什么吗?因为他一个劲儿地刻啊刻啊,竟忘了给他做一对耳朵。
杰佩托没有耳朵可抓,就抓住木偶的颈背,他要把他带回家,同时摇着头吓唬他说:
"咱们现在回家,到了家,一定要算清咱们这笔账!"
皮诺乔听了这句吓唬的话,马上就倒在地上,赖在那里不肯再走了。爱看热闹和无所事事的人一下子就过来,围成了一大堆,
大家七嘴八舌舌的。
"可怜的木偶!"有人说,"他不肯回家是有道理的!谁知道杰佩托这坏蛋会怎么揍他呢!……"
又有人不怀好意地接上去说:
"杰佩托这家伙,看着挺老实,对孩子可真凶!让这个可怜木偶落到他手里,他准把木偶剁成碎木片!……"
一句话,他们这么东一锤西一棒的,那位警察竟把皮诺乔放开,反倒把可怜的杰佩托送到监狱里去了。"他一路上监狱,一路结结巴巴地哭着说:
"该死的小鬼!我辛辛苦苦本想做出个好木偶!可结果是自讨苦吃!我本该先想到这一点!……,
接下来发生的事情简直叫人没法相信,我在以下各章里,将一一讲给诸位听.

\chapter{}

好,小朋友们,现在我来告诉大家,当可怜的杰佩托平白无辜地给送进监狱的时候,皮诺乔这小坏蛋看见自己逃脱了警察的手,马上撒腿就跑,穿过田野,抄近路回家。他拼命地跑啊跑啊,跳过一个个很高很高的土墩和荆棘丛,跳过一条条水沟,像只被猎人追赶的小山羊或者小野兔。
他跑到房子前而,看见朝街的门半掩着,就推门进去,他放下门臼,卜通坐到地上,得意洋洋地吐了一口长气。
可他得意了也只有一眨眼的工夫,因为他听见屋子里有声音叫:
"唧唧,唧唧!"
"谁在叫我啊?"皮诺乔吓坏了说。
"是我!"
皮诺乔转过脸,看见一只大蟋蟀在墙上,正慢腾腾地往上爬。
"告诉我,蟋蟀,你是谁。"
"我是会说话的蟋蟀,在这屋子里已经住了百把年啦。"
"这屋子今天是我的了,"木偶说,"如果您真肯行行好,让我高兴高兴,就请头也别回,马上走吧。"
"要让我走,"蟋蟀回答说,"可得让我在走以前先告诉你一个大道理。"
"那就说吧,快点,"
"孩子不听父母的话,任意离开家,到头来决不会有好结果!他们在这个世界上要倒霉,迟早会后悔的,"
"您高兴唱就下去吧,我的蟋蟀,可我明天天不亮,一准就离开这里,我要是呆在这里,就逃不出所有孩子都会遇到的事情:把我送去上学,不是软骗就是硬来,逼着我读书。跟您说句心里话,我一点不想读书,我更爱追蝴蝶,爬树掏鸟窝。"
"可怜的小傻瓜!可你不知道吗,这样你会变成一头大蠢驴,所有的人都要拿你开玩笑的?"
"闭口吧你,你这不吉利的坏蟋蟀!"皮诺乔叫道。
可蟋蟀又耐心又有智慧,木偶这样粗暴无礼,它一点不生气、还是用它原来的声调说:
"你要是不爱上学,那为什么不学个什么行当,好正正直直地给自己挣块面包呢?"
"你要我告诉你吗?"皮诺乔开始不耐烦了,回答说,"世界上所有的行当当中,只有-个行当真正合我的心意。"
"什么行当?"
"就是吃、喝、睡觉,玩儿,从早逛到晚。"
"告诉你,"会说话的蟋蟀还是那么心平行和地说,"凡是干这种行当的,最后几乎不是进医院就是进监牢。"
"当心点,不吉利的坏蟋蟀!……你惹我生气了可要倒霉!"
"可怜的皮诺乔!你真叫我可怜!……"
"我为什么叫你可怜?"
"因为你是-个木偶,更糟的是,因为你有一个木头脑袋。"
听了最后这句话,皮诺乔火冒三丈,猛地跳起来,打工作台上抓一个木头槌子,就向会说话的蟋蟀扔过去。
他也许根本不想打中它,可是真不巧,正好打中了它的头,可怜的蟋蟀只来得及叫一声唧唧,就给打死了,贴在墙上。

\chapter{}

这时候天开始黑了,皮诺乔猛想起他还没吃过点东西,就觉得肚子在咕噜咕噜叫,真想吃。
孩子是这样,一想到吃就越来越想吃,说真个的,几分钟工夫,想吃就变成了肚子饿,肚子越来越饿,饿得他像只饿狼,饿得他肚子像刀绞。
可怜的皮诺乔马上向壁炉扑过去,那儿有个锅子在冒热气,他打算揭开锅盖,看看里面在煮什么,谁知那锅子是画在墙上的,诸位想象一下吧,他是多么失望啊,他那个本来已经很长的鼻子,马上又至少长了四指。
于是他满屋子乱跑,搜遍了所有的抽屉、所有的角落,只想找到点面包,哪怕是一丁点儿干面包,只想找到点硬面包皮、狗啃过的骨头、发霉的玉米糊、鱼骨头、樱桃核,总而言之,随便找到什么可以进口的东西都好,可他什么也没找到,一丁点儿东西也没找到。
这时他肚子越来越饿,越来越饿,可怜的皮诺乔,他除了打哈欠,就毫无办法可以让肚子好过一点儿。他的哈欠打得那么长,每一回嘴巴都一直咧到耳朵边。打完一个哈欠他就吐口水,只觉得胃也要吐出来了。
最后他绝望了,哭着说:
"会说话的蟋蟀说得对,我错就错在不听爸爸的话,逃出了屋子……我爸爸要是在这儿,这会儿我就不会一个劲儿打哈欠,人都要打死了!唉哟!肚子饿多难受啊!"
正在这时候,他看到一堆垃圾里好像有一样东西,圆滚滚的、白花花的,完全像个(又鸟)蛋。他一蹦就跳了过去,扑到它上面,的的确确是个(又鸟)蛋。
木偶这份高兴是只可意会,无法形容的,他简直像在做梦,一个劲儿把(又鸟)蛋捧在手上,转过来转过去,又摸又吻,一面吻还一面说:
"这会儿我该怎么吃这个蛋呢,煎来吃不好吗?……不,放在盘子里煮更好!……噢,用煎锅煎最好,还有比煎(又鸟)蛋更好吃的吗?噢,不弄熟怎么样,就生着吃?不,还是放在盘子里煮,或者用煎锅煎好,我想吃得要命啦!"
说干就干,他把煎锅放在一个烧炭的火盆上,在煎锅里他放的不是素油不是牛油,而是水。等到水一冒气,卡嗒!……他敲破(又鸟)蛋壳,就要把蛋倒进去。
可蛋壳里倒出来的不是蛋白和蛋黄,而是一只小(又鸟)。小(又鸟)又快活又有礼貌,姿势优美地鞠个躬说:
"多谢您,皮诺乔先生,您让我省了力气,不用去弄破蛋壳啦!再见,祝您好,请代我问候您一家人!"
它说着拍拍翅膀,从打开的窗子飞出去,不见了。
可怜的木偶站在那里发呆,眼睛瞪大,嘴巴张开,手里拿着两瓣(又鸟)蛋壳。他这么愣了一阵,等到最后清醒过来,就哇哇地又哭又叫,绝望得跺脚,一面哭一面说:
"还是会说话的蟋蟀说得对!如果我不从家里进出去,如果我爸爸在这儿,这会儿我就不会饿得要命了!噢,肚子饿多难受啊!"
肚子继续咕噜咕噜响,越响越厉害,他又不知道该怎么办才叫它不响,他觉得还是离开屋子,到隔壁村子去看看,巴望能碰到个好心人,会施舍点面包给他吃吃。

\chapter{}

这真是个可怕的冬夜,雷声隆隆,电光闪闪,整个天空好像着了火,寒冷彻骨的狂风卷起滚滚的灰尘,吹得田野上所有的树木刷拉刷拉直响。
皮诺乔最怕打雷闪电,可肚子饿比打雷闪电更可怕。因此他掩上门,撒腿就跑,蹦上那么百来蹦,来到一个村子,他舌头也吐了出来,上气不接下气,活像一只猎犬。
可村子里一片漆黑,人影也没有一个,铺子都关上了门。一家家也关上了门,关上了窗子,街上连一只狗也没有,整个村子像死了似的。
皮诺乔又是绝望又是肚子饿,于是去拉一户人家的门铃,他丁零丁零拉个不停.心里说:
"总会有人朝外看看的。"
果然,有人打开了窗子朝下看,这是个老头儿,戴一顶睡帽,气乎乎地大叫:
"这么深更半夜的,要干什么?"
"请做做好事,给我点面包行吗?"
"你等着吧,我就下来。"老头儿回答着,心想准碰上了小坏蛋,深更半夜来开玩笑。人家好好地睡觉,他却来拉门铃捉弄老实人,
过了半分钟,窗子又打开了,还是那个老头儿的声音对皮诺乔叫道:
"你在下面站着,把帽子拿好。"
皮诺乔还没有帽子,他马上走到窗子底下,只觉得一大盆水直泼下来,把他从头淋到脚,好像他是一盆枯萎的天竺葵似的。
皮诺乔像只落汤(又鸟)似地回家里,他又累又饿,一点力气也没有了。他再没力气站着,干是坐下来,把两只又湿又脏、满是烂泥的脚搁到烧炭的火盆上,
他就这样睡着了,他睡着的时候,一双木头脚给火烧着,一点一点烧成了炭,烧成了灰。
皮诺乔只管睡他的大觉,咕啊咕啊地打呼,好像这双脚不是他的,是别人的,他直到天亮才一下醒来,因为听见有人敲门,
"谁呀?"他打着哈欠,擦着眼睛问,
"是我,"一个声音回答。
这是杰佩托的声音。

\chapter{}

可怜的皮诺乔睡眼惺忪,还没看到他的两只脚已经完全烧没了,因此他一听到父亲的声音,马上跳下凳子要跑去开门,可他身子摇了那么两三摇,一下子就直挺挺倒在地板上了。
他倒在地板上这啪哒一声,听着就似是一口袋木勺子从五层楼上落下来似的。
"给我开开门!"这时杰佩扦在外面衔上叫。
"我的爸爸,我开不了门",木偶回答说,又是哇哇哭,又是在地上打滚。
"为什么开不了?"
"因为我的两只脚给吃掉了。"
"给什么吃吃掉了?"
"给猫",皮诺乔说。因为这时候他正好看见一只猫,用前脚在玩一些刨花。
"我说,给我开开门!"杰佩托又说一遍,"要不,我进屋子给你只'猫'!"
"可我站不起来,相信我吧。噢,我真可怜,我真可怜!我一辈子得用膝头跪着走路啦!……"
杰佩托听见木偶又哭又叫,以为又是他在捣鬼,想好好收拾他,于是打窗口爬进屋子。
杰佩托先还想骂他打他,可等到看到他躺在地上,当真没有脚,心马上软了下来,他赶紧搂住皮诺乔的脖子,把他抱在怀里,抚摸了他成千遍,哄了他成千回,大滴大滴的眼泪流下腮帮,哭着说:
"我的好皮诺乔!你的脚怎么烧掉啦?"
"不知道,爸爸,可请您相信,这是个可怕的冬夜,我一辈子也忘不了,又打雷,又闪电,我肚子饿得要命,当时会说话的蟋蟀对我说:'你是活该,你不好,自作自受,'我对它说:'你小心点,蟋蟀!……'它对我说:'你是个木偶,有个木头脑袋,'于是我抓起个木头槌子,扔过去,它就死了,可这都怪它自己,因为我并不想打死它,我把煎锅放在火盆的炭火上,可是小(又鸟)跑出来说:'再见……给我向您一家人问好',可肚子越来越饿,因此那个老头儿,戴睡帽的,把头探出窗口,对我说:你在下站着,把帽子拿好。'我头上挨了那么一盆水,讨点面包吃并不可耻,对吗?我马上回家,因为饿坏了,我把脚搁在火盆上烤干。您回来了,我的脚烧没了。可我这会儿肚子还是那么饿。脚再也没有了!噫……!噫!……噫!……噫!……”。??
可怜的皮诺乔说着哭起来,哭得那么响,五公里外都能听见,
杰佩托听他说了半天,只听懂一点,就是木偶饿得要死了。于是他打口袋里掏出三个梨,递给他,说:
"这三个梨是我准备当早饭吃的,可我很高兴给你吃。吃吧,吃了梨就好了。"
"你要是给我吃,请把皮削掉吧。"
"削皮?"杰佩托听了很惊奇,反问说,"我的孩子,我简直不能相信,你的嘴那么刁,你那么难侍候,这可不好!在这个世界上,得从小习惯什么都吃,懂得给什么吃什么,因为你永远不知道会遇到什么事情,什么事情都会有!……"
"您的话是不错,"皮诺乔接下去说,"可我永远不吃不削皮的水果,水果皮我受不了。"
杰佩托是个大好人,就拿出一把小刀,用天使般的耐心,削好了三个梨,把梨皮放在桌子角上。
皮诺乔两口就吃掉了第一个梨。他正要把梨心扔掉,杰佩托拦住他的手,对他说:
"别扔掉。在这个世界上,样样东西都会有用的。"
"可说真的,我不要吃梨心!……"木偶像蛇那么扭来扭去叫道。
"谁知道呢!什么事情都会有!……"杰佩托并不生气,又说了一遍。
就这样,三个梨心没扔出窗口,跟梨皮一起,都放在桌子角上。
皮诺乔吃了三个梨,或者说得准确点,吞下三个梨,打了个很长很长的哈欠,接着又哭也似地说:
"我肚子又饿了!"
"可我的孩子,我再没什么可以给你了。"
"没有了,真的没有了?"
"就剩下这儿一点梨皮和梨心了。"
"没法子,"皮诺乔说,"要是没别的,我就吃块梨皮吧。"
他于是嚼起梨皮来,他先还歪着点嘴,可后来一块接一块,一转眼就把所有的梨皮都吃光了,吃完梨皮,又吃梨心。等到全给吃完,他心满意足地拍拍肚子,兴高采烈地说:
"这会儿我觉得好受了!"
"现在你看,"杰佩托给他指出说,"我刚才对你说没错吧,得学会不要太挑肥拣瘦,不要太嘴刁。我的小宝贝,在这个世界上,咱们永远不知道会遇到什么事情。什么事情都会有!……"

\chapter{}

木偶肚子一不饿,马上就叽哩咕噜,哇哇大哭,吵着要一双新的脚。
可杰佩托为了他的恶作剧,想要罚罚他,就让他去哇哇哭,让他绝望了整整半天,最后才说:
"凭什么我要给你再做一双脚呢?是为了眼巴巴看着你再打家里溜出去吗?"
"我向您保证,"木偶哭着说,"从今以后我一定做个好孩子……"
"所有孩子碰到想讨点什么的时候,"杰佩托回答,"他们都是这样说的。"
"我向您保证,我要去上学读书,叫人看得起……"
"所有孩子碰到想讨点什么的时候,都来这一套。"
"可我跟别的孩子不同!我比所有的孩子好,我一直说真话,爸爸,我向您保证,我要学会一种本领,等您老了,我安慰您,养您。"
杰佩托虽然装出一副凶相,可看着他那可怜的皮诺乔这么受罪,眼里噙着眼泪,心里充满了爱,他不再回答什么话,只是拿起工具和两块干木头,一个劲地干起活来了。
一个钟头不到,两只脚已经做好。这两只小脚轻巧,干燥,灵活,真像一位天才雕刻家做出来的,
杰佩托于是对木偶说:
"闭上眼睛睡一觉吧!"
木偶闭上眼睛假装睡觉。在木偶假装睡觉的时埃,杰佩托用(又鸟)蛋壳装点溶化了的胶,把两只脚给他黏上,黏得那么天衣无缝,一点看不出黏过的样子。
木偶一看见自己有了脚,就打直挺挺躺着的桌子上翻下来,乱蹦乱跳的跳了上千次,翻了上千个跟头,简直乐疯了。
"为了报答您给我做的一切",皮诺乔对他爸爸说,"我要马上去上学。"
"好样儿的孩子!"
"可是去上学得有点儿东西穿。"
杰佩托很穷,口袋里连一个子儿也没有,于是用花纸给他做了一套衣服,用树皮给他做了一双鞋,用面包心给他做了一顶小帽子。
皮诺乔马上跑到一脸盆水那里去照,对自己的模样满意极了,神气活现地说:
"我真像一位体面的先生!"
"不错,"杰佩托回答说,"可是你要记住,使人成为体面先生的不是好衣服,而主要是干净的衣净的衣服。"
"不过",木偶又说了,"我上学还少一样东西,一样最要紧的东西。"
"什么东西?"
"我还少一本识字课本。"
"你说得对,可怎么弄到它呢。"
"那还不方便,到书店里买就是了。"
"钱呢?……"
"我没钱。"
"我也没钱,"好老头说,心里很难过。
皮诺乔尽管是个快活透顶的孩子,可也难过起来了。因为一件真正伤心的事,那是人人都会懂得的,连孩子也不例外。
"没法子,只好这么办!"杰佩托叫了一声,忽然站起来,穿上打满补丁的粗布旧上衣,跑出门去了。
一会儿工夫他就回来。回来的时候,他手里拿着给他孩子买的识字课本,可短上衣没有了。这个可怜人只穿着衬衫,外面可是在下雪。
"上衣呢,爸爸?"
"我给卖了。"
"为什么卖了?"
"因为我热。"
他回答的这句话是什么意思,皮诺乔一下子就明白了,他那颗良心不由得一阵冲动,就扑上去抱住杰佩托的脖子,在他的整个脸上到处亲吻。

\chapter{}

雪一停,皮诺乔就夹着他那本呱呱叫的新识字课本去上学,他一路走,他的小脑袋瓜里浮现出成千个幻想,成千座空中楼阁,越来越美。
他自言自语说:
"我在学校里,今天就要学会读书,明天就要学会写字,后天就要学会计算,以后凭着我的本领,我要挣许许多多钱。我第一次拿到钱就马上给爸爸买一件漂亮的布上衣,可我干吗买布的呢?我要买件金丝银线织的,钮扣是宝石做的,这位可怜人实在该穿这样的衣服,为什么,一句话,他为了给我买书,为了让我能够读书,竟把上衣也给卖了,光穿件衬衫……可天又这么冷!只有做爸爸的才肯作出这种牺牲!……"
他正在这样激动地说着这番话,忽然听见远处有音乐声,又是吹笛子,又是敲鼓:的的的,的的的……咚,咚,咚,咚。
他停下来竖起耳朵听,这声音是打岔道那边尽头传过来的,这条岔道很长很长,一直通到海边一个小村子。
"这音乐声是怎么回事?可惜我得去上学,要不……"
他站在那里拿不定主意,可无论如何得作出决定:或者去上学,或者去听吹笛子。
"今天就去听吹笛子,明天再去上学吧,去上学,反正日子长着呐,"这个小淘气最后耸耸肩膀说,
说干就干,他走到那条岔道上,撒腿就跑,他越往前跑,吹笛子和敲鼓的声音就越清楚:的的的,的的的,的的的……咚,咚,咚,咚。
转眼他就来到了一个广场中央,那里人山人海,都围着一个大棚。这大棚是用木头和五颜六色的布搭起来的。
"这大棚是什么玩竟儿?"皮诺乔转身问村里一个孩子。
"你就念一下海报吧,上面都写明白了,你一念就知道。"
"我很想念,可今天我正好还不会念。"
"好一头蠢牛!那我来念给你听,你看见海报上那几个火红的大字没有,这几个字写的是:木偶大戏院……"
"戏开场很久了吗?"
"这会儿才开场,"
"门票多少钱,"
"四个子几,"
皮诺乔想看得要命,什么也不管了,不害助听臊地跟刚才对话的孩子说:
"借给我四个子儿行吗,明天还你?"
"我很想借给你,"那孩子开玩笑地回答说,"可今天我正好不能借。"
"四个子儿,我把我这件外套卖给你,"木偶于是对他说。
"花纸做的外套,我要来干吗?雨落到上面,我脱也脱不下来了。"
"想买我的鞋子吗?"
"拿来生火最好。"
"这顶帽子你给多少钱,"
"买来倒真有用!一顶面包心做的帽子!耗子可要到我头上来吃帽子了!"
皮诺乔不知怎么是好,他还有最后一样东西想说出来,可又不敢说。他犹豫不决,拿不定主意,十分苦恼,最后他还是说了:
"你肯给我四个子儿,买了我这本新识字课本吗?"
"我是个孩子,不向孩子买东西,"对方那个小家伙回答他说,这个家伙比他有头脑多了。
"这本识字课本四个子儿我买,"一个卖旧衣服的叫起来。他们讲话时,他正好在旁边,
书当场卖掉了。想想那位可怜的杰佩托吧,他如今在家,光穿着衬衫,冷得索索发抖,就为的给儿子买这么本识字课本!

\chapter{}

皮诺乔一进木偶戏院,就出了件事,这件事几乎闹了个大乱子。
要知道,这时戏幕已经升起,滑稽戏已经开场了。
台上站着花衣小丑和驼背小丑,正吵得不可开交,接着就是那老一套,他们不断地你威吓我我威吓你,说要请对方吃耳光和吃棍子。
台下的观众聚精会神,听着这两个木偶吵架,哈哈大笑,两个木偶做着手势,互相辱骂,活灵活现,就像两个有理性的动物,咱们这世界的两个人。
忽然之间,花衣小丑停止了表演,向观众转过身来,用手指着观众席后排,用演戏的腔调大叫起来:
"天上的诸神啊!我是做梦还是醒着呢?那下边片人不是皮诺乔吗?……"
"正是皮诺乔!"驼背小丑叫道,
"一点不错就是他!"罗萨乌拉太太打台后伸出头来尖声叫道。
"是皮诺乔!是皮诺乔!"所有的木偶同声大叫,跳到外面台上来,"皮诺乔!是咱们的兄弟皮诺乔!皮诺乔万岁!,
"皮诺乔,上来,到我这儿来,"花衣小丑叫道,"上来,投到你的木头弟兄们的怀抱里来吧!"
他们这么热请地邀请,皮诺乔一跳就从观众席后座跳到前座,再一跳就从前座跳上乐队指挥的头顶,又从乐队指挥的头顶蹦上戏台。
皮诺乔受到木偶戏班男女演员的狂热欢迎,他们拥抱、搂他的脖子,友好地撮弄他,跟他像真诚兄弟那样头碰头,这个场面是无法想象的。
不用说,这个场面十分动人,不过观众看见戏老不演下去,不耐烦,开始大叫:
"我们要看戏,我们要看戏!"
可他们是白费力气,因为木偶们不是把戏演下去,而是加倍大叫大喊。他们把皮诺乔放在肩膀上,狂欢着抬到脚灯前面。
这时木偶戏班班主出来了,他个子大,样子凶,叫人看一眼就要害怕,他有把黑色大胡子,就像一大摊墨水迹,老长老长的,从下巴一直拖到地上,只说一点就够了,他走起路来脚都要踩着这把大胡子,他那张嘴人得像炉口,-双眼睛好似两盏点着火的红玻璃灯,他手电劈啪劈啪抽着根大鞭子,是用蛇和狼尾巴编起来的。
没想到忽然出来了班主,大伙儿一下子吓得连气都不敢透,连苍蝇飞过都听得见,这些可怜的木偶,男男女女个个哆嗦得像树叶子。
"你干吗到我的戏院里来捣乱?"班主问皮诺乔说,那大嗓门听着就像阎王爷害了重伤风的声音。
"请您相信,先生,这都不怪我!……"
"够了够了!晚上咱们再算账。"
事实就是如此,戏演完以后,木偶戏班班主走进厨房,厨房里正在烤一只肥羊做晚饭,叉子叉着,在火上慢慢地转动,他为了弄来木柴最后把羊烤熟烤焦,就把花衣小丑和驼背小丑叫来、对他们说:
"钉子上挂着的那个木偶,你们去给我带来,我看这木偶的木头很干,把他扔到火里,准能把火烧旺,烤熟这一只羊,"
花衣小丑和驼背小丑先还犹豫着不走,可班主生气地瞪了他们一眼,他们吓得只好服从。一转眼工夫他们就回到厨房,架来了可怜的皮诺乔,皮诺乔扭来扭去,像条出水鳗鱼,拼命大叫:
"我的爸爸,快救救我!我不要死,我不要死!……"

\chapter{}

木偶戏班班主吃火人(他就叫这么个名字)看样子是个可怕的人,那是没话说的,特别是他那把黑色大胡子,像围裙似地盖住他整个胸口和整整两条腿,可他到底不是个坏人,事实上,他一看见可怜的皮诺乔给带到他面前,拼命挣扎,哇哇大叫:"我不要死,我不要死!"心马上就软,可怜起他来了,他鼻子忽然发热,忍了好大一会儿,可终于忍不住,就大声打了一个喷嚏。
花衣小丑一直在伤心,像垂柳那样弯下身子,可一听见打喷嚏,马上喜容满面,向皮诺乔弯过身来,轻轻跟他咬耳朵说:
"好消息,兄弟,班主打喷嚏了,这表示他已经感动,在可怜你,如今你有救了。"
因为要知道,有许多人一同情什么人,或者是哭,戒者至少是假装擦眼睛,可吃火入不同,他真的.,感动了,就要打喷嚏,这也是一种表示他心软的的方式,
打过喷嚏以后,木偶戏班班主还是装出很凶的样于,对皮诺乔叫道:
"别哭了!你哇哇哭,叫我肚子里难受极了……叫我觉得绞痛,几乎,几乎……啊嚏,啊嚏……"又打了两个喷嚏。
"长命百岁!"皮诺乔说,
"谢谢!你爸爸妈妈都活着吗?"吃火人问他,
"爸爸活着,可我从来不知道妈妈,"
"我这会儿要是把你扔到炭火里,谁知道你的老父亲要多么伤心啊!可怜的老头!我很同情他!……啊嚏,啊嚏,啊嚏!"他又打了三个喷嚏,
"长命千岁!"皮诺乔说,
"谢谢!不过也得同情同请我,因为你看,我要把这头羊烤熟,木柴没有了,说老实话,你在这种情况下对我非常有用!可如今我很感动,我想忍耐看不烧你,既然不烧你,我就得在我的戏班里另找一个木偶来代替你,把他扔到叉子底下去烧……喂,守卫的!"
一声命今,马上来了两个木头守卫,他们挺高挺高,挺瘦挺瘦,头戴两角帽,手握出鞘的剑,
木偶戏班班主气咻咻地对他们说:
"给我把这个花衣小丑抓住,捆得牢牢的,扔到火里去,我要让我这只羊烤得香香的!"
诸位想象一下这个可怜的花衣小丑吧!他吓得两条腿一弯,跪在地上了,
皮诺乔看见这种凄惨场面,就扑倒在班主脚下,嚎啕大哭,泪水把他那把大胡子也给弄湿了,开始哀求他说:
"可怜可怜吧,吃火人先生!……"
"这里没有先生!……"木偶戏班班主冷冰冰地回答说。
"可怜可怜吧,骑士先生!……"
"这里没有骑士!……"
"可怜可怜吧,爵士先生!……"
"这里没有爵士!"
"可怜可怜吧,大老爷!……"
木偶戏班班主-听见叫他大老爷,马上噘起了嘴,变得慈祥多了,温和多了,问皮诺乔说:
"你到底求我什么事?"
"我求您开开恩,放了可怜的花衣小丑!"
"这可开不得恩。我不烧你就得烧他,因为我要把我这只羊烤得香香的。"
"那么,"皮诺乔大叫一声,站了起来,扔掉头上的面包心帽子,"那么,我知道我该怎么做了。来吧,守卫先生们!把我捆起来扔到火里去,不行,让可怜的花衣小丑,我的真正朋友,替我去死是不公道的!……"
这番话说得丁当响亮,声调豪迈激昂,在场的木偶听了没有不哭的,连两个守卫,虽然是木头做的,也哭得像吃奶的羊羔。
吃火人起先一点不动心,冷得像块冰,可后来慢慢地、慢慢地也开始感动了,又打喷嚏了。他一口气打了四五个喷嚏,于是疼爱地张开怀抱,对皮诺乔说:
"你是个好小子!过来,给我一个吻。"
皮诺乔马上跑过去,像只松鼠似地顺着木偶戏班班主的大胡子往上爬,爬到上面,在他鼻尖上给了他一个最甜最甜的吻。
"那么,您开恩啦?"可怜的花衣小丑问道,声音细得好不容易才听见。
"开恩了!"吃火人回答说。接着他叹口气,摇摇头,"没法子!今儿晚上我只能吃半生不熟的羊肉了。可下一回,谁要是打动我的心,他就活该倒霉!……"
一听说开了恩,所有的木偶都跑到戏台上,像开盛大晚会那样,点亮了所有的灯和烛台,开始又跳又舞。他们就这样一直跳啊舞的直到大天亮。

\chapter{}

第二天早晨,吃火人把皮诺乔叫到一旁,问他说:
"你父亲叫什么名字?"
"叫杰佩托。"
"他是干什么的,"
"他很穷。"
"他赚的钱多吗?"
"要问他赚的钱,从不见他口袋里有一个子儿。请想象一下吧,为了买一本识字课本给我上学,他得卖掉身上仅有的一件短上衣。这件短上衣完全是补丁,没一处好的。"
"可怜的人!我很同情他。这里是五个金币。马上带回去给他,并且替我问他好。"
不用说,皮诺乔向木偶戏班班主千谢万谢,他把戏班里所有的木偶一个个拥抱过,包括两个守卫,然后欢天喜地回家去了。
可还没有走上半公里路,他就在路上碰到一只瘸腿狐狸和一只瞎眼猫。它俩一路上相互搀扶,似是两个患难朋友。瘸腿狐狸靠在猫身上,瞎眼猫由狐狸领着路。
"早上好,皮诺乔,"狐狸向他恭恭敬敬问好说。
"你怎么知道我的名字?"木偶问它。
"我跟你爸爸挺熟。"
"你在哪儿见过他?"
"昨天在他家门口见过。"
"他在干什么?"
"他穿着一件衬衫,冷得直打哆嗦。"
"可怜的爸爸!可是谢谢老天爷,从今往后,他就不用再打哆嗦了!……"
"为什么?"
"因为我变成个体面先生啦。"
"你是个体面先生?"狐狸说着,放肆地大笑,猫也跟着笑,可为了不让皮诺乔看见,用两个前爪子假装在理着胡子。
"没什么可笑的,"皮诺乔生气地叫道,"我真不想叫你们流口水,可这儿,要是你们想知道的话,这儿有五个呱呱叫的金币。"
他说着掏出吃火人送他的钱。
一听到金币丁丁当当响,狐狸不由自主地伸出了它那只好像瘸了的爪子,猫也张大了它那两只眼睛。这两只眼睛绿幽幽的像两盏灯,不过它们马上又闭上了,皮诺乔当然一点没看见。
"现在,"狐狸问他,"你拿这些钱想干什么呢?"
"第一,"皮诺乔回答说,"我要给我爸爸买一件漂亮的新上衣,金丝银线织的,钮扣是宝石做的,第二、我要给自己买一本识字课本。"
"给你自己?"
"还用说,我要去上学好好读书嘛。"
"你瞧瞧我吧,"狐狸说,"我就为了愚蠢得竟想去读书,结果把一条腿都弄瘸了。"
"你瞧瞧我吧,"猫说了,"我就为了愚蠢得竟想去读书,把两只眼睛都搞瞎了。"
正在这时候,一只白椋鸟蹲在路边树丛上唱起它的老调,说:
"皮诺乔,别听坏朋友的话,要不,你要后悔的!"
可怜的椋鸟没来得及把话说完!猫猛地一跳,跳得半天高,一把抓住椋鸟,林鸟连叫一声"唉哟"的工夫也没有,就已经连毛一起进入了猫的大嘴巴,
猫吃掉椋鸟,擦过嘴巴,重新闭上两只眼睛,又照旧装瞎子。
"可怜的椋鸟!"皮诺乔对猫说,"你为什么对它这么狠呢?"
"我这样做是为了教训教训它,这样一来,下次它可就学乖,别人说话不会插嘴了。"
他们走到半路,狐狸忽然停下,对木偶说:
"你想让你的金币加个倍吗?"
"你这话什么意思?"
"你只有那么五个金币,你想让它们变成一百个,一千个,两千个吗?"
"那还用说!可怎么变呢?"
"简单极了。你先别回家,跟我们走。"
"你们带我上哪儿去?"
"到傻瓜城去。"
皮诺乔想了想,接着拿定主意说:
"不要,我不去,这会儿就到家了,我要回家,我爸爸在等着,可怜的老人家昨儿没见我回去,谁知道他有多么焦急呀!真倒霉,我是这么个坏孩子,还是会说话的蟋蟀说得对:'不听话的孩子在这个世界上没有好结果。'我从自己的教训懂得了这一点,因为我遭了许多殃,昨儿晚上在吃火人那里,我差点儿连命都送掉了……Brrr!我一想起都要发抖!"
"这么说,"狐狸说道,"你真想回家?那你就回家吧、反正是你自己吃亏!"
"是你自已吃亏!"猫跟着又说了一遍。
"你好好想想,皮诺乔,因为你有福不享。"
"有福不享!"猫跟着又说了一遍。
"你的五个金币到明天要变成两千个了。"
"两千个了!"猫跟着又说一遍。
"可怎么会变那么多呢?"皮诺乔问道,惊奇得嘴都合不拢了。
"我这就告诉你,"狐狸说,"你要知道,傻瓜城有块福地,大家叫它'奇迹宝地'。你在这块地上挖一个小窟窿,然后放进去,比方说吧,放进去一个金币。然后你在窟窿上撒点土,重新盖起来,浇上两锅泉水,再撒上一撮盐,晚上你安安稳稳上床睡大觉好了,一夜工夫,这个金币生长开花。第二天早晨你起床回到地里一看,你想你会看到什么呢,你会看到一棵漂亮的树,长满了金币,多得就像六月里一串丰满的麦穗上的麦粒。"
"这么说,"皮诺乔完全入迷了,说道:"要是我把我那五个金币种在那块地上,第二天早晨我可以有多少个金币呢?"
"容易算极了,"狐狸回答说,"用指头尖一算就算得出来,比方说,每个金币长出五百个,五百乘五,第二天早晨你口袋里就可以有两千五百个闪闪发光、丁丁当当响的金币。"
"噢,那多美呀!"皮诺乔大叫,高兴得跳起来,"等我把这些金币都采下来,我拿两千,还有五百个我送给你们俩。"
"送给我们?"狐狸像给得罪了,生气地叫道,"上帝免了你这份礼吧!"
"免了你这份礼!"猫跟着又说了一遍。
"我们这么起劲,可不是为了卑鄙的利益,"狐狸回答说,"我们起劲只是为了让别人发财致富。"
"让别人发财致富。"猫跟着又说了一遍,
"多好的人啊!"皮诺乔心里说,他一下子忘掉了他的爸爸,忘掉了新上衣,忘掉了识字课本,忘掉了一切好的打算,却对狐狸和猫说:
"那咱们走吧。我跟你们去。"

\chapter{}

他们走啊,走啊,走啊,最后天黑了,他们累得够呛,来到了一家旅馆,叫做"红虾旅馆"。
"咱们在这儿停一会儿"狐狸说,"吃点东西,歇上个把钟头,半夜动身,明儿天不亮,'奇迹宝地'就到了。"
他们走进旅馆,,二个人占了一张桌子,可谁都说不要吃什么。
可怜的猫说它肚子很不舒服,只要吃三十五条香茄酱火兔、四份奶酪杂碎,因为觉得杂碎味道不够好,又添了三次牛油和奶酪粉!
狐狸虽然想吃,可大夫规定它要严格节制饮食,因此它只好吃得简单点,就吃了一只肥美的野兔,周围摆满一圈肥嫩的童子(又鸟),吃完野兔,它又要了一大批饭后点心:(又鸟)杂炒蛋,鹧鸪,家兔,田(又鸟)、晰蜴,甜葡萄。接下来就不要什么了。它说食物已经叫它作呕,它一口也吃不下去了。
吃得最少的是皮诺乔。他只要了点核桃,还要了块面包,可结果都留在盘子里没吃,这可怜孩子光顾着想'奇迹宝地',好像金币己经把他撑饱了。
吃完晚饭,狐狸对老极说:
"'给我们两间上房,一间住皮诺乔先生,一间住我和我的朋友,我们走前会打铃,可得记住,半夜我们要起来继续赶路。"
"是,先生们,"老板回答着,对狐狸和猫眨眨眼,像是说:"有数有数,算说定了!……"
皮诺乔一上床就睡着了,睡着了就做梦,他梦见自己在一块地当中。这块地满是矮矮的树,树上挂满一串一串的东西,这一串一串的东西都是金币,让风吹着,发出丁、丁、丁的声音,听着像说:"谁高兴就来采我们吧,"可正当皮诺乔兴高采烈,伸手要去采这些漂亮的金币,把它们全给放进口袋的时候,忽然给房门上很响的三下敲门声惊醒了。
原来是旅馆老板来告诉他,钟已经敲半夜十二点了。
"我那两位同伴准备好了吗?"木偶问他,
"岂止准备好了!两个钟头以前都走啦。"
"为什么这祥急?"
"因为猫得到音信,说它的大孩子脚上生冻疮,有生命危险。"
"晚饭钱它们付了吗?"
"您说到哪儿去啦,它们太有教养了,哪能对您这样的先生如此无礼呢!"
"太可惜了!我倒高兴它们无礼些!"皮诺乔说着抓抓头,接着他又问:"我这两位好朋友说过,它们在哪儿等我吗?"
"说是在'奇迹宝地'等你,明天早晨,天一亮的时候。"
皮诺乔给自已和两个朋友的那顿晚饭付了一个金币,这才走了,
他可以说是摸索着走的,因为旅馆外面一片漆黑,黑得伸手不见五指。四周田野上连一点叶子沙沙声也听不见。只有一些夜鸟不时打一丛树上飞到另一丛树上,在路上穿过,用翅膀碰到了他的鼻子,他吓得向后直跳,大叫起来:"什么人?"周围的小土岗发出回声,拉长声音反复说着:"什么人?什么人?什么人?"
他正走间,看见一棵树干上有一样小生物发出一点光,苍白昏暗,像夜里从透明瓷灯罩里发出来的灯光。
"你是谁?"皮诺乔问它,
"我是会说话的蟋蟀的影子,"那小生物回答,声音很微弱很微弱,像是从另一个世界来的。
"你找我干吗,"
"我想给你一个忠告,你往回走吧,把剩下的四个金币带回去给你可怜的爸爸,他正在哭呢,以为再见不到你了。"
"我爸爸明天就要变成一位体面的先生,因为这四个金币要变成两千个。"
"人家说什么一夜之间就可以发财财富,我的孩子,你可别相信。他们那种人通常不是疯子就是骗子,听我的话,往回走吧。"
"可我不往回走,我偏要向前走。"
"时间很晚了!……"
"我偏要向前走。"
"夜那么黑……"
"我偏要向前走。"
"路上有危险……"
"我偏要向前走。"
"你要记住,任性的孩子早晚要后悔的。"
"又是老一套。明天见,蟋蟀。"
"明天见,皮诺乔,愿天老爷保佑你不沾露水,不遇杀人的强盗!"
会说话的蟋蟀一说完这句话,光忽然熄灭了,就像一些灯给一阵风吹灭了似的。路上比先前更黑了。

\chapter{}

"说真个的,"木偶一面重新上路,一面自言自语说,"我们这种可怜孩子多倒霉!人人都骂我们,人人都教训我们,人人都要我们这样做那样做。人人都一开口就自以为是我们的爸爸,自以为是我们的老师。人人都这样,连那样会说话的蟋蟀也这样。看这会儿,就因为我没听这只讨厌蟋蟀的啰哩啰嗦,它就说我不知道要遇到多少灾难!我还要遇到杀人的强盗呢!还好我不相信有什么杀人强盗,从来就不相信。依我看,杀人强盗全是那些做爸爸的想出来,吓唬吓唬夜里想出去的孩子的,就算我真在路上碰到他们,难道我会害怕他们吗,我根本不怕,我要走到他们面前,对他们叫着说:'杀人强盗先生,你们要把我怎么样?记住吧,可别跟我开玩笑!去你们的吧,别开口了!'我这番话说得那么绝,那些倒霉的杀人强盗啊,我好像已经看见他们了,他们像阵风似地逃走啦。万一他们凶神恶煞,偏不逃走呢?那有什么,我逃走就是了,事情不就结了吗……?
可皮诺乔没能把他那套大道理说完,因为就在这时候,他好像听见后面树叶子沙沙响,很轻很轻的,
他回头一看,就看见黑地里有两个难看的黑影,这是两个人,全身用装炭的口袋套着,踮起脚尖一跳一跳地紧紧追来,活像两个鬼怪。
"他们真在这里!"皮诺乔心里说了一声。他不知把四个金币藏到哪儿好,一下子把它们藏到了嘴里,正好塞在舌头底下。
接着他想逃走。可是刚迈腿,就觉得胳膊给抓住,听到两个翁声瓮气的可怕声音对他说:
"要钱还是要命!"
皮诺乔没法回答,因为嘴里塞着金币。他做了成千个怪脸、成千个手势,要让对方——他们从口袋上眼睛的地方那两个小窟窿里望出来——明白,他是个穷木偶,口袋里连一个铜子儿也没有。
"拿出来拿出来!别装傻了,把钱拿出来!"两个强盗且威吓的口气大叫。
木偶用头和手表示:"没钱。"
"不把钱拿出来就要你的命,"高的那个杀入强盗说。
"要你的命!"另一个跟着又说了一遍。
"要了你的命,还要你父亲的命!"
"还要你父亲的命!"
"别别别,别要我可怜爸爸的命!"皮诺乔发急地大叫,可他这么一叫,嘴里的金币就丁丁当当响起来了。
"哈哈,骗子!原来你把钱藏在舌头底下?马上吐出来!"
皮诺乔硬挺住!
"哈哈,你装聋子?你等着吧,我们这就想办法让你吐出来!"
真的,他们一个抓住他的鼻子尖,一个揿他的下巴,动手粗暴地又扳又弄,一个扳这里,一个弄那里,要逼他把嘴张开。可是没用。木偶的嘴像黏在一块,钉在一起。
于是矮的那个拔出一把很大的刀子,想用它做杠杆或者凿子,插到他的上下嘴唇之间,可皮诺乔快得像闪电,一口把它的手咬断了,接着把咬下来的手吐出来。诸位想象一下他有多么惊奇吧,因为他吐在地上的不是人的手,而是一只猫的爪子。
皮诺乔旗开得胜,胆子大了。他挣脱杀人强盗的爪子,跳过路旁的树丛,开始在田野上逃走。那两名杀人强盗紧紧追来,像两条猫追一只野兔。其中一名杀人强盗因为失去了一只爪子,就用独脚追,天知道他是怎么跑的。
跑了十五公里左右,皮诺乔跑不动了。这时他眼看自己没救了,就顺着最高的一棵松树的树干爬上去,坐在一个枝头上。两个杀人强盗也打算跟着爬上树,可是爬到一半,叭哒就掉在地上,手脚的皮都擦破了。
可它们还不死心,捡来一小捆干柴,堆在松树脚下,点着了。说时迟那时快,松树开始熊熊烧起来,像风吹着的蜡烛。皮诺乔看见火焰越烧越高,不想最后变成一只烤鸽子,于是猛地一跳,打枝头上跳下来,重新又胞,穿过田野和葡萄园。两个杀入强盗在后面紧追,一步也不拉下。
这时天已经开始亮,他们还是追个不停。皮诺乔一下子给一条沟挡住了去路。这条沟又宽又深,满是脏水,颜色像牛奶咖啡。怎么办?"一,二,三!"木偶叫着,猛跑两步,一跳就跳到了沟那一边。两个杀人强盗跟着也跳,可是没算准距离,卜龙通!……落到沟里去了。皮诺乔听到他们落水和水溅起来的声音,哈哈大笑,一面跑一面叫:
"祝你们痛痛快快洗个澡,杀人的先生们!"
他料想他们一准淹死了,可回头一看,只见他们两个依然在他后面追,身上还是套着他们的麻袋,哗哗地淌着水,活像两个漏了底的筐子。

\chapter{}

这时木偶已经完全泄气,到了要扑倒在地向两个强盗告饶的地步,可一下子看见深绿的树林子里,远远有一座雪白的小房子在耀眼。
"我要是有口气跑到那房子,就有救了,"他心里说。
他一分钟也不耽搁,重新一个劲跑起来,穿过林子。两个杀入强盗依然在后面追。
他拼命跑了近两个钟头,终于上气不接下气地跑到那座小房子门口,连忙嘭嘭嘭敲门。
可没人答应。
他使劲把门敲得震天价响,因为他听见追来的脚步声、又响又急的呼吸声越来越近了。
可还是静悄悄的。
他看见敲门毫无用处,就开始在门上用脚拼命地踢,用头拼命地撞。这时窗口探出个头来,这是个美丽的小女孩,天蓝色的头发,脸白得跟蜡像似的,眼睛闭着,双手交叉在胸前。她说话时嘴唇也不动,声音很轻很轻,像是从另一个改界来的:
"这座房子里没人,所有的人都死了。"
"至少你给我开开门!"皮诺乔哭叫着求她,
"我也死了。"
"死了,那你现在在窗口干吗?"
"我在等棺材,它要来把我给装走。"
小女孩子一说完这句话,就不见了。窗子也悄没声儿地重新关上了。
"噢,天蓝色头发的美丽小姑娘,"皮诺乔大叫,"帮帮忙,给我开开门吧!请你同情一个可怜的孩子,他后面追着杀人的……"
他这句话没能说完,因为他觉得脖子给掐住了,还听到那两个声音在咆哮着威胁说:
"现在你再逃不掉啦!"。
木偶看到死在眼前,不由得一阵哆嗦,哆嗦得两条木头腿的关节卡嗒卡嗒响,藏在舌头底下的四个金币也丁丁当当响起来了。
"怎么样,"两个杀人强盗问他说。"你开口吗,开还是不开,怎么!不回答?……那我们就动手了,这一回定要把你的嘴弄开!……"
他们说着,拔出两把很长很长的刀子,锋利得像剃刀,嚓嚓!……给他背上来了两下。
幸亏木偶是用很硬很硬的木头做的,因此他没受伤,刀倒断成了好多片。两个杀人强盗手里光剩下刀柄,你看着我,我看着你。
"我明白了,"其中一个说,"咱们得吊死他!吊死他吧!"
"吊死他吧!"另一个跟着又说了一退。
说干就干,他们把他双手反绑,用活结套住他的喉咙,把他吊在一棵大橡树的树枝上。
然后他们坐在树下,就等着木偶蹬最后一次腿。可木偶过了三个钟头依然张开两只眼睛,闭着嘴巴,两腿越蹬越有劲。
他们最后等得不耐烦了,就向木偶转过脸,冷笑着对他说:"明儿见,等我们明天回到这儿,希望你帮个大忙,已经死掉了,把嘴张得大大的。"
他们说着,走了。
这时候猛乱起一阵北风,呼呼地怒号,把吊在那里的可怜木偶吹过来吹过去,狠狠地摇得他像过节时丁当丁当摇着的大钟,这样摇啊摇啊,摇得他痛苦万分。喉咙上的活结越收越紧,叫他气也透不出来。
他的两眼一点一点发黑。可他虽然感到死期已近,依然希望随时会有人经过,把他救下来,可他等啊等啊,看见还是没人来,一个人也没有,于是就想到他的可怜的爸爸……他半死不活地结结巴巴说:
"噢,我的爸爸,要是你在这儿就好了!……"
他再也说不出话来。他闭上眼睛,张开嘴巴,伸长两腿,一阵猛烈颤动,吊在那里像是僵硬了。

\chapter{}

正当可怜的皮诺乔给两个杀人强盗吊在大橡树枝头上,觉得这会儿死多活少的时候,天蓝色头发的美丽小女孩重新在窗口出现了,她看见木偶给套着脖子吊着,让北风吹得摇来摇去,太不幸了,不由得很可怜他,于是轻轻拍了三下手掌。
这三下手掌一拍,就听到很响的拍翅膀声,一只大老鹰风驰电掣地飞来,停在窗台上,
"有什么吩咐啊,我仁慈的仙女?"老鹰说着,垂下鸟嘴致敬(因为要知道,这天蓝色头发的小女孩不是别人,正是最善良的仙女,她在这树林附近已住了一千多年了)。
"你看见那木偶吗,给吊在大橡树树枝上的?"
"看见了。"
"那好。马上飞到那里,用你那有力的尖嘴解开那个吊着他的绳套,把他轻轻放在橡树下的草地。"
老鹰飞走了,两分钟就回来了,说:
"吩咐我做的都给做好了。"
"你觉得他怎么样?活着还是死了?"
"我看他好像死了,可还没全死,因为我一松开套在他喉咙的绳套,他叹了一口气,嘟囔了一声:'这会儿我觉得好多了!'"
仙女于是又轻轻拍了两下手掌,来了一只很漂亮的卷毛狗。它像人那样用后腿直立走道。
这只卷毛狗身穿车夫的礼服,头戴金边小三角帽,白色假卷发垂到脖子上。巧克力色的上衣上钉着宝石钮扣,两边有两个大口袋,放主人吃饭时赏它的肉骨头。下身穿一条大红天鹅绒裤子、一双丝袜、一双开口软鞋。后面还有一样东西,很像雨伞稍,蓝绸子做的。下雨的时候用来藏它的尾巴。
"做件好事,梅多罗!"仙女对卷毛狗说,"马上到我的厩房里,赶一辆最好的车子上树林子去。你到了大橡树底下,就会找到已经半死的可怜木偶直挺挺地躺在草地上。你把他抱起来,很小心很小心地放在车子坐垫上,把他送到这儿来。明白了吗,"
卷毛狗把后面那个蓝绸子尾巴套摇了三四次,表示它明白了,然后像闪电似地跑掉了。
一转眼工夫,只见厩房里出来了一辆天蓝色的票亮小轿车,外面装饰着金丝雀羽毛,里面裱糊得象掼奶油和奶油蛋糕那样。车子用一百对白老鼠来拉,卷毛狗坐在驾车台上,左右地抽着鞭子,车夫赶路的时候都是这样的。
一刻钟不到,这辆小轿车就回来了。等在门口的仙女抱起可怜的木偶,把他抱进一间墙上镶嵌着珍珠的小卧室,马上请来附近最有名的大夫。
三位大夫马上接连来了,一位是乌鸦,一位是猫头鹰,一位是会说话的蟋蟀。
"我想请诸位先生看看,"仙女对围在皮诺乔床边的三位大夫说,"我想 请诸位先生看看,这不幸的木偶是死了还是活着……"
听了仙女的请求,乌鸦第一位给皮诺乔摸脉,接着摸鼻子,接着摸小脚趾。等到都摸过了,它极其严肃地说了这一番话:
"我认为木偶完全死了,但万一他没有死,那就有可靠的迹像表明,他完全活着!"
"我很抱歉,"猫头鹰说,"我必须表示,我的看法跟我这位有名的朋友和同行乌鸦大夫正好相反。我认为,木偶完全活着,但万一他不幸没有活着,那就有可靠的迹像表明,他的确死了!"
"您说哩,"仙女问会说话的蟋蟀。
"我要说的是,一位小心谨慎的大夫在不知道他所要说的事情时,最好是不开口。再说,这位木偶对我来说不是陌生面孔,我认识他有好些日子了!……"
皮诺乔本来一直躺着不动,像段真正的木头,可这会儿一下子猛烈颤抖,弄得整张床都摇动起来。
"这个木偶,"会说话地蟋蟀往下说,"是个大坏蛋……"
皮诺乔张开眼睛看看,马上又闭上。
"是个无赖,是个二流子,是个流氓……"
皮诺乔把脸缩到被单底下。
"这木偶是个不听话的坏孩子,他要把他可怜的爸爸气死!……"
它说到这里,只听见屋子里有压抑着的哭声和哽咽声。诸位想象一下大伙儿有多么惊奇吧,因为他们把被单掀起一点,就看到是皮诺乔在哭,在哽咽。
"死人会哭,就表明他正在好起来,"乌鸦严肃地说。
"我只好表示我的看法跟我这位有名的朋友和同行正好相反,"猫头鹰跟着说,"依我看,死人会哭,就表明他不想死。"

\chapter{}

三位大夫一走出屋子,仙女就到皮诺乔身边,摸摸他的脑门,发现一点不假,他在发高烧。
于是她把一点白色粉末溶在半杯水里,拿来给木偶,温柔地对他说:
"喝了它,过几天就好了。"
皮诺乔看着杯子,歪歪嘴,哭也似地问道:
"甜的还是苦的?"
"苦的,可它能医好你的病。"
"苦的我不喝。"
"听我的话,喝了它。"
"苦的我不要喝。"
"喝了它,喝了就给你一颗弹子糖,让你甜甜嘴。"
"弹子糖呢?"
"在这儿,"仙女说着,从放糖的金盒子里拿出一颗来。
"我要先吃弹子糖,再喝这种该死的苦水……"
"讲定啦?"
"讲定了……"
仙女给他弹子糖,皮诺乔一转眼就喀嚓喀嚓地咬碎吃掉了,舔着嘴唇说:
"糖是药就好了!……我就天天吃药。"
"现在你照讲定的办,喝了这点药水,它会医好你的病。"
皮诺乔不情愿地拿过杯子,把鼻子插进去,然后凑到嘴边,然后又把鼻子插进去,最后说:
"太苦了!太苦了!我不能喝。"
"你尝都没尝,怎么说太苦呢?"
"我想得出来!我闻到了气味。我要先再吃一颗弹子糖……然后喝药水!……"
仙女像一个好妈妈那样耐心,又给他放了一题糖在嘴里,然后重新给他杯子。
"这样我不能喝药水!"木偶说着,做了成千个鬼脸,
"为什么?"
"因为脚上的枕头碍着我。"
仙女给他把枕头拿开了。
"不行!这样我还是不能喝……"
"又是什么东西碍着你啦?"
"房门半开着,把我碍着了。"
仙女去把房门关上。
"不管怎么说,"皮诺乔大哭大叫,"这该死的药水是苦的,我不要喝,不喝,不喝,不喝……"
"我的孩子,你要后悔的……"
"我才不在乎呐……"
"你的病很重……"
"我才不地乎呐……"
"你发高烧,几个钟头就会死的……"
"我才不在乎呐……"
"你不怕死?"
"怕死?……我宁愿死也不喝这种倒霉药水。"
正在这时候,房门开了,进来了四只兔子,黑得像墨汁,肩膀上抬着一个小棺材。
"你们到我这儿来干吗?"皮诺乔叫道,害怕得在床上坐了起来。
"我们来抬你,"最大的一只兔子说。
"抬我?……可我还没死!……"
"现在还没死,可你不肯喝退烧药水,就只有几分钟好活了!……"
"噢,我的仙女!噢,噢,我的仙女!"木偶于是大声叫起来"快把杯子给我……做做好事,快点快点,因为我不想死,不不不……不想死……"
他两只手捧着杯子,一口就把药水喝了。
"没法子!"兔子们说,"我们这回白跑一趟。"
它们重新抬起小棺材,打牙缝里叽哗咕噜地说着走出了屋子。
真的,过了几分钟,皮诺乔已经跳下床,好了。因为要知道,木偶福气好,难得生病,好起来也特别快。
仙女看见他满屋子又跑又跳,又利落又高兴,活像一只刚会啼的小公(又鸟),就对他说:
"瞧,我的药水可不是真把你治好了?"
"还有说的!它让我活下来了!……"
"可为什么刚才让你喝药水,要那么左求右求呢,"
"我们孩子都这样!我们比怕生病更怕喝药水。"
"真不害澡!……孩子们应该知道,及时吃进良药可以治好大病,甚至可以不死……"
"噢!下回我就不要那么左求右求了!我要记住那些抬棺材的黑兔……那我就马上抓过杯子喝下去!……"
"现在你过来,告诉我你是怎么落到那些杀人强盗手里的。"
"是这么回事。木偶戏班班主吃火人给了我几个金币,对我说:'来,把它们带回去给你爸爸!'可我在路上碰到一只狐狸和一只猫,它们两个很好,对我说:'你想让这几个金币变成一两千个吗,跟我们来,我们带你上"奇迹宝地"去'。我说:'咱们走吧。'他们说:'咱们在红虾旅馆歇会儿,过了半夜再走。'等我醒来,他们已经不在了,他们走了。于是我一个人走。夜黑得要命。路上我碰到两个杀人强盗,身上套着装炭的口袋。他们对我说:'把钱拿出来。'我说:'我没钱。'因为我把那四个金币藏在嘴里。一个杀人强盗想把手伸进我的嘴巴。我一口咬下他的手,把它吐出来。可吐出来的不是手,是一只猫爪子。两个杀人强盗就追我。我拼命地逃,可还是被捉到了,最后我被套着脖子吊在这片林子里的一棵树上。他们说:‘我们明天会回来的。那时候你就死透了,嘴巴也张开了,我们就把你藏在舌头底下的金币拿出来。’”
"你这四个金币,现在搁哪儿啦?"仙女问他。
"我丢了!"皮诺乔回答说,他这是说谎,因为钱在他口袋里。
他一说谎,本来已经够长的鼻子又长了两指。
"你在哪儿丢了?"
"就在这儿附近的树林子里。"
这第二句谎话一说,鼻子更长了。
"你要是在附近那树林子里丢了,"仙女说,"咱们去把它们找回来。因为东西丢在附近那树林子里,完全可以找回来。"
"啊,现在我记清耍蹦九夹睦锘帕耍卮鹚担罢馑母鼋鸨椅颐欢簦?是刚才喝您那杯药水的时候不小心,吞下肚子里去了。"
这第三句谎话一说,鼻子呼地一下长成这副样子,可怜的皮诺乔连头都没法转了。头往这边转,鼻子就碰到床,碰到窗玻璃;头往那边转,鼻子就碰到墙,碰到房门;头一抬,鼻子就有插到仙女一只眼睛里去的危险。
仙女看着他笑起来。
"您干吗笑?"木偶问她。眼看鼻子变得那么长,他完全呆住了,急得要命。
"我笑你说谎。"
"您怎么知道我说谎了?"
"我的孩子,谎话一下子就可以看出来,因为说了谎话有两种变化,一种是腿变短,一种是鼻子变长,你的一种正是鼻子变长。"
皮诺乔羞得无地自容,想溜出房间。可是办不到,他那个鼻子已经长得连门都出不去了。

\chapter{}

正像诸位可以想像到的,仙女让木偶由于鼻子长得出不了门,哭叫了整整半个钟头,不去理他。这是为了好好给他一个教训,让他改正撒谎这种极坏的毛病。这种毛病小孩子最容易有。可等她看到木偶脸也变了,绝望得眼睛都要突出来时,很可怜他,拍了拍手掌。一听到拍手掌,成千只叫啄木鸟的大鸟打窗子飞到屋里来。它们都聚在皮诺乔的鼻子上,开始笃笃笃笃,狠狠地啄他的鼻子,几分钟工夫,这个长过了头的鼻子就恢复了原状。
"您多好啊,我的仙女,"木偶擦于眼泪说,"我多么爱您啊!"
"我也爱你,"仙女回答说,"你如果想留在我这儿,你就做我的弟弟,我做你的姐姐……"
"我很想留在这儿……可我那可怜的爸爸呢?"
"我都想到了。已经派人去通知你爸爸,天黑前他就要来到这儿。"
"真的?"皮诺乔高兴得跳起来,叫着说,"那么,我的好仙女,如果您答应的话,我想去接他!我急着要拥抱这位可怜的老人家,他为我吃了那么多苦!"
"那你就去吧,可小心别走失了。你走林子里的那条路吧,我断定你会碰到他的。"
皮诺乔走了。他一走进树林子,马上就像小鹿一样跑起来。可他到了大橡树那儿,就停下了,因为好像听到树枝树叶之间有人声。他果真看见路上有人。诸位猜得出是谁吗?……就是狐狸和猫这两个伙伴。皮诺乔曾经同它们一起在红虾旅馆吃过一顿晚饭。
"是我们的好朋友皮诺乔!"狐狸叫着,把他又抱又亲,"你怎么在这儿?"
"你怎么在这儿?"猫跟着又说了一遍。
"说来话长了,"木偶说,"我趁便跟你们讲讲。可记得那个夜里,你们丢下我一个人在旅馆里吗?我走出来,在路上遇见了两个杀人强盗……"
"两个杀人强盗?……噢,可怜的朋友!他们想要什么。"
"他们想抢我的金币。"
"真该死!……"狐狸说。
"该死极了!……"猫跟着又说了一遍。
"可我撒腿就逃,"木偶往下说,"他们跟着就追。最后他们追上了我,把我吊在这棵橡树的树枝上面……"
皮诺乔说道,指指离开两步远的大橡树。
"还有比这更悲惨的事吗?"狐狸说,"我们是活在怎么一个世界上啊,我们这些正派人,在什么地方可以找到安全可靠的地方呢?"
皮诺乔正这么说着,忽然发现猫的右前腿受了伤,连爪子带指甲都没有了,就问它说:
"你的爪子怎么啦?"
猫想回答,可窘住了。狐狸马上说:
"我的朋友太谦虚了,因此不愿回答,我来替他回答吧。要知道,一个钟头以前,我们在路上碰到一只老狼,都快饿死了,它求我们施舍点什么给它。可我们没有什么好给它的,连一根鱼骨头也没有。我这朋友真正慷慨大方,它做出什么事情来啦?……它竟从自己前腿上咬下一只爪子,扔给这只可怜的野兽吃。"
狐狸一面说着一面擦眼泪。
皮诺乔也感动得走到猫身边,在它耳边轻轻地说:
"如果所有的猫都像你,耗子可多幸运啊!"
"可你这会儿在这里干吗呢,"狐狸问木偶说。
"我在等我爸爸,他早晚要到这儿来的。"
"那你的金币呢?"
"都在口袋里,就少一个,付给红灯旅馆的老板了。"
"想想吧,四个金币到明天就能变一两千个,你为什么不听我的话?你为什么不到'奇迹宝地',把它们种下去呢。"
"今天不行,我改天去。"
"改一天就晚了。"狐狸说。
"为什么?"
"因为这块地给一位大好老买去了,从明天起,再不准任何人在那儿种金币。"
"'奇迹宝地'离这儿远吗?'"
"不到两公里。你要跟我们去吗,半个钟头就到,你马上种下四个金币,过几分钟就可以收到两千个,今晚回来,口袋里就装满金币啦,要跟我们去吗?"
皮诺乔没马上回答,因为他想到了善良的仙女,想到了年老的杰佩托,还想到了会说话的蟋蟀给他的劝告。可是最后,他就像一个全没脑筋、全没心肝的孩子所做的那样,也就是说,他点点头,对狐狸和猫说:
"那咱们走吧,我跟你们去。"
于是他们上路了。
他们走了半天,来到一个城市,叫做"捉傻瓜城"。皮诺乔一进城就看见,满街都是饿得张嘴打哈欠的癌皮狗,给剪了毛、冷得直打哆嗦的绵羊,乞讨一颗玉米、也没(又鸟)冠也没垂肉的公(又鸟),卖掉了漂亮的五彩翅膀、再也飞不起来的大蝴蝶,没有了尾巴、不好意思再见人的孔雀,悄悄地走来走去、痛惜永远失去了闪闪发光的金色银色羽毛的山(又鸟)。
在这许多畏畏缩缩的叫化子和穷人中间,不时走过一些高贵马车,里面或者坐着,狐狸,或者坐着偷东西的喜鹊,或者坐着捕食小生物的猛禽。
"'奇迹宝地'在哪儿,"皮诺乔问道。
"再走两步就到了。"
说到就到,他们穿过城,出了城门就来到一块僻静的田地。这块田地跟其他田地完全没什么两样。
"咱们总算到了,"狐狸对木偶说,"现在你弯下腰,在泥地上挖一个小窟窿,把金币放进去吧。"
皮诺乔照狐狸说的办。他挖了一个窟窿,把剩下的四个金币放进去,然后用点土把窟窿重新盖起米。
"现在,"狐狸说,"你到附近水沟那里打桶水来,浇在你种下金币的地方。".
皮诺乔走到水沟那儿,因为没有桶,就从脚上脱下一只鞋子,装来了水,浇在盖住窟窿的土上,然后他问:"还有什么事要做吗?"
"没有了,"狐狸回答说。"现在咱们可以走开了,你过二十分钟回到这儿,就可以看到一棵矮矮的树从地里长出来,所有的树上都挂满了金币。"
可怜的木偶高兴得忘乎所以,对狐狸和猫千谢万谢,答应送给它们最好的礼物。
"我们不要礼物,"两个坏蛋回答说,"我们只要能教会你不劳而获,发财致富,就像过节一样高兴!"
他们这么说着,向皮诺乔鞠了个躬,祝他得到好收成,就干它们的事去了。

\chapter{}

木偶回到城里,开始一分钟一分钟地数着时间,等他觉得时候到了,马上走原路回"奇迹宝地"去。
他走得很急,一路只听见他那题心很响地的嗒的嗒跳,就像一个走着的挂钟。他一想:
"树上如果不是一千,而是两千呢?树上如果不是两千,而是五千呢,树上如果不是五千,而是一万呢,噢,到那时,我将变成一个多体面的先生啊!……我要有一个美丽的宫殿,我要有一千只小木马和一千个马厩,这是为了玩玩。我还要有一个酒窖,里面放满甘露酒和健胃酒。我还要有一个图书室,摆满了糖果、蛋糕、葡萄干小面包、杏仁饼、奶酪夹心饼干。"
他这么幻想着,走近了那块地。他停下来就张望,看能不能见到那么一棵树,枝头挂满金币的。可他什么也没看见。他往前又走了一百步,还是没看见。他一直走到那块地上……一直走到种下金币的那个小窟窿那里,可还是没看见。于是他就拼命动脑筋,也顾不得行什么礼貌规矩,打口袋里伸出——只手来,把头搔了半天。
正在这时候,他耳朵里好像听到了大笑声。他抬头一看,只见一棵树上有只大鹦鹉,正在理它身上稀稀拉拉的羽毛。
"你笑什么?"皮诺乔生气地问它。
"我笑,因为我理羽毛,把羽毛底下的胳肢窝弄痒了。"
木偶没答话。他走到水沟那里,还是用那只鞋子打来一鞋子水,重新浇在盖着金币的那片土上。
这时候田野上静悄悄的,他又听见了笑声,这一次笑得比上次更放肆。
"不管怎么说,"皮诺乔发疯似在大叫,"你告诉我,没教养的鹦鹉,你笑什么?"
"我笑傻瓜,他们竟会什么胡涂话都相信,上最犹猾的人的当。"
"你说我吗?"
"对,我说你,可怜的皮诺乔,我说你是个大胡涂虫,竟相信金币可以像豆子南瓜那样在田野上播种收获。我曾经也相信过一次,到如今都觉得后悔。如今(可惜太晚了!)我确信,要正直地挣到一点钱,必须懂得用自己的手劳动,或者用自己的头脑思索。"
"我不懂你说些什么,"木偶说,这时他已经吓得发起抖来了。
"没法子!我只好说得更明白些,"鹦鹉往下说。"你要知道,当你在城里的时候,狐狸和猫回到这块地里来,挖走了金币,像阵风似地溜掉了。如今要追上它们,已经办不到啦!"
皮诺乔就那么张大了嘴闭不拢来。他不愿意相信鹏纽的话,开始用手指甲挖浇过水的土。他挖啊,挖啊,挖了很深很深的一个大坑,连一个稻草堆都可以放进去了,可就是找不到金币。
木偶于是绝望了,回到城里,马上到法庭去向法官告状,说两个贼偷走了他的钱。
法官是只大猩猩。这老猩猩受到大家尊敬,因为它年纪大,胡子白,特别是因为它戴一副金丝边眼镜。他这副金丝边眼镜连玻璃片也没有,可它不得不一直戴着。它戴上这副眼镜,是因为多年以前有一次眼睛充了血。
皮诺乔在法官面前,一五一十地诉说了使他上当的恶意欺诈经过,说出了两个贼的姓名和特征,最后请求主持公道。
法官极其和气地听着,对他讲的话十分关心,听得又感动,又同情。等到木偶讲得没话要讲了,他伸出一只手,拿起一个铃来摇了一下。
听到铃声,马上来了两条猛狗,穿的是警察制服。
法官指着皮诺乔对两个狗警察说:
"这个可怜小鬼给人偷了四个金币,把他抓起来,马上送到监牢里去。"
木偶听到竟不幸对他这么宣判,呆住了,想要提了抗议,可是两个狗警察为了不白白浪费时间,堵住他的嘴,把他送到监牢里去了。
木偶整整坐了四个月牢。好长的四个月哪:他本来还要坐下去,幸亏出了一件极其运气的事。原来统治这个"捉傻瓜城"的年轻皇帝打了个大胜仗,下令普天同庆,张灯结彩,大放焰火,赛自行车。为了表示欢庆,还打开监狱,放掉所有的盗贼。
"别人出狱,我也要出狱。"皮诺乔对狱卒说。
"您不行,"狱卒回答说,"因为您不属于这一类。"
"对不起,"皮诺乔回答说,"我也是个贼。"
"既然这样,您就完全有理由出狱,"狱卒说着,恭恭敬敬地脱帽行礼,打开牢门,放他跑了。

\chapter{}

皮诺乔听说释放,他那份高兴劲儿就可想而知了。他二话没说,马上出城,取道上仙女那座小房子去。
这是下雨天,整条路像个泥潭,走起来半条腿都没到烂泥里。可木偶一点不地乎。他急着要重新看到他的爸爸,看到他天蓝色头发的姐姐。他蹦啊跳地跑得像条猎犬,泥浆溅到帽子上,他一面跑一面自言自语说:
"我遭多少殃啊……这是活该,因为我是个犟头倔脑的木头人……我任意妄为,对于爱我和比我聪明千倍的人说的话,我一点都不听!……可从今往后,我决心改邪归正,做一个老实听话的孩子……如今我看清楚了,不听话的孩子要倒大霉,一事无成。我的爸爸在等我吗?……我在仙女家会看到他吗,可怜的爸爸,我多久没见到他了,我现在只想没完没了地抚摸他,拼命地亲吻他!仙女会原谅我的不好行为吗?……只要想一想,我得到她的百般关心和亲切治疗……只要想一想,我今天还能活着,全亏的是她!……还有孩子比我更忘恩负义,更没心肝的吗?……”
他正这么自言自语,一下子大吃一惊,停了下来,还倒退了四步。
他看见什么啦?……
他看见了一条大蛇,直挺挺地横躺在路上。这条蛇绿皮火眼,尾巴很尖,像是烟囱在冒烟。
木偶害怕得无法形容。他离开它,跑了有半公里多,坐在一堆石头上,只等这条蛇爬开,把路让出来。
他等了一个钟头,两个钟头,三个钟头,可蛇还在那儿。虽然离得老远,还能看见它那双火眼红红的,尾巴尖冒出一股一股烟柱。
最后皮诺乔鼓足了勇气,走运那蛇,离开它几步,用很甜很细的声音讨好地对它说:
"对不起,蛇先生,请帮个忙,挪出点地方让我过去,好吗?"
可他这番话完全是白说。蛇一动也不动。
木偶又用那很甜很细的声音说:
"您得知道,蛇先生,我要回到那房子去,我爸爸在那儿等着我,我已经很久很久没见到他了!……您肯让我继续走我的路吗?"
他等着蛇作出个表示来回答他这个请求,可蛇没有动静。相反,它一直好像很生猛,这时倒变得僵直不动了。它的眼睛闭上,尾巴停止冒烟。
"它真的死了吗?……"皮诺乔说着,高兴得搓了搓手。他一点不耽搁,就要打它身上跳过去,跳到路的那一边。可他脚还没举起,蛇忽然像进起来的弹簧似地跳了起来。木偶大吃一惊,赶紧往后退,绊了一下,跌倒在地上。
跌得也真不巧,他的脑袋插在路上的泥浆里,只剩两条腿倒竖着。
蛇看见木偶头朝下,两脚用难以想象的速度踢来踢去,就扭啊扭地狂笑起来,笑啊,笑啊,笑啊,最后笑得太厉害,肚子上一根静脉竟断掉了:这回它真的死啦。
于是皮诺乔重新跑起来,要在天黑之前赶到仙女的家。可路很长,肚子饿得慌。他再也忍耐不住,就跳进一块葡萄地,想采两串膨香葡萄吃。唉,真不该跳进去的!
他一到葡萄藤底下,卡嗒……只觉得两脚给两块很锋利的铁片一下夹住,痛得他眼冒金星。
可怜的木偶是给一个捕兽夹夹住了。这种捕谷夹是农民装在那里捕捉大(又鸟)貂的。要知道,(又鸟)貂是附近所有(又鸟)埘的大灾星。

\chapter{}

诸位可以想象,皮诺乔当然大哭大叫,大叫饶命。可是哭也好叫也好,全都没用,因为这儿周围看不见房子,路上一个走过的人也没有。
这时候天黑了。
半是由于捕兽夹夹得他小腿骨太痛,半是由于周围漆黑一片,他一个人在这葡萄地里怕得要死,木偶眼看就要昏过去了。正在这时候,他忽然看见一只萤火虫在头上飞过。他马上叫住萤火虫,对它说:
"噢,萤火虫,做做好事,把我从这刑具里放出来好吗?……"
"可怜的孩子!"萤火虫停下来,同情地看着他,
回答说。"你的脚怎么会夹在这些锋利的铁片里的?"
"我走进这块葡萄地,想采两串麝香葡萄吃吃,结果就……"
"葡萄是你的吗?"
"不是……"
"那么,是谁教你拿别人东西的?……"
"我饿了……"
"我的孩子,饿不能作为占有别人东西的充分理由……"
"这是真的,这是真的!"皮诺乔大哭大叫,"下回我再不干了。"
他们话正说到这里,给走近的一阵很轻很轻的脚步声打断了。来的是这块地的主人。他踮起脚尖走来看看,有没有(又鸟)貂夜里来吃(又鸟),给捕兽夹夹住了。
等他打外套底下掏出灯来,看见捉到的不是(又鸟)貂,而是个孩子,他惊奇极了。
"哈哈,小偷!"农民生气地说,"这么说,我的(又鸟)都是你偷的?"
"我没偷,我没偷!"皮诺乔抽抽嗒嗒地说,"我汾来只想乐两串葡萄!……"
"会偷葡萄就会偷(又鸟)。让我来给你个教训,叫你一辈子忘不了。"
他打开捕兽夹,抓住木偶的领子,像拎一只吃奶羊羔似地把他拎回家。
到了家门口,他把木偶扔在空场上,用一只脚踏住他的脖子,对他说:
"现在太晚了,我要去睡觉。明天再跟你算账。我那只守夜的狗正好今天死了,你这就来代替它。你给我当守夜的狗。"
说到做到,他在木偶脖子上套上一狗颈圈,上面全是铜钉。他把颈圈收紧,叫木偶的头钻不出来。颈圈上系着一根很长的铁链,铁链一头拴在墙上。
"要是今夜下雨,"农民说,"你可以到这木板狗屋里去,那里头里有很多干草,可以当床睡。我那可怜的狗在那里都睡了四年啦。如果不幸有小偷来,你记住了,要竖起耳朵听着,汪汪地叫。"
农民吩咐完,就进屋把门关上,还用粗链子拴好,于是空场上就剩可怜的皮诺乔一个人趴着,又冷,又饿,又怕,半死不活的。他不断生气地把手插到勒住他喉咙的颈圈里,哭着说:
"我这是活该!……真倒霉,我这是活该!我任性,只想闲逛……我只想听坏朋友的话,因此总是失去幸福。如果我是个好孩子,像别的孩子一样,如果我想读书想劳动,如果我同我的可怜爸爸一起在家,那我这会儿就不会在这儿田野当中,做一只狗给一个农民看门了。噢,我能重新做人就好了!……可现在迟了,没法子,我只好忍耐!"
他发泄了真正出自内心的一口怨气以后,走进狗屋,躺下就睡着了。

\chapter{}

皮诺乔呼呼睡了两个多钟头,到了半夜,突然给一阵嘁嘁喳喳的古怪声音惊醒了。听起来,这声音像是打门口空场上传来的。他把鼻尖伸出木板狗屋的门洞,看见四只小野兽聚在一起商量什么。它们毛色黑乎乎的,样子像猫。可它们不是猫,是(又鸟)貂,(又鸟)貂是贪吃的肉食野兽,特别爱吃(又鸟)和小(又鸟)。-只(又鸟)貂离开同伴,走到木板狗屋的旁边来,低声说:
"晚上好,梅拉姆波。"
"我不叫梅拉姆波。"木偶回答说。
"噢,那你叫什么?"
"我叫皮诺乔。"
"你在这儿干吗?"
"我在这儿当看夜狗。"
"那么,梅拉姆波呢?这老狗一直住在这狗屋里,它上哪儿啦?"
"他今天早晨死了。"
"死了,可怜的狗!它那么好!……可看你的脸,我觉得你也是一只客气的狗。"
"对不起,我不是狗!……"
"噢,那你是什么,"
"我是一个木偶。"
"你当看夜狗,"
"真倒霉,为了处罚我!……"
"好,那我自我提出个协议,同我早先跟已故的梅拉姆波订立的完全一样,你会满意的。"
"什么协议?"
"我们照旧一星期一次,夜里来拜访这个(又鸟)埘,拉出来八只(又鸟)。八只(又鸟)当中,我们吃七只,-只给你。条件是,你听明白了,你假装睡着,千万别出来叫醒农民。"
"梅拉姆波就这么干的吗?"皮诺乔问。
"就这么干的。我们和它合作得很好。你安静地睡你的觉吧,我们走以前,保证在狗屋上留下一只拔掉毛的肥(又鸟),给你明天当早饭吃。咱们讲明白啦?"
"简直太明白了!……"皮诺乔答道。同时他恫吓似地摇摇头,好像想说:"咱们走着瞧吧!"
四只(又鸟)貂觉得它们的事情稳当了,就很快地溜到就在狗屋旁边的(又鸟)埘那里。它们用牙用爪子使劲弄开关住的小木门,一只接一只地溜了进去。它们刚进去,就听见小门啪嗒一下,又猛地关上了。
把门又给关上的正是皮诺乔。他关上门不算,为了保险起见,还在门前放了块大石头顶住它。
接着他叫起来,叫得就象一只看门狗:"汪,汪,汪,汪。"
农民一听见汪汪叫,马上跳下床,拿起枪,把头探出窗子问道:
"什么事?"
"来小偷了!"皮诺乔回答。
"在哪儿?"
"在(又鸟)埘里。"
"我马上下来。"
真的,就说一声"阿门"那么点工夫,农民已经下来了。他赶紧走进(又鸟)埘,把四只(又鸟)肥全给捉住,塞进布口袋,心花怒放地对它们说:
"你们终于落到我手里了!我本可以处罚你们,可我还不高兴动手呐!我宁可明天把你们带到附近一家酒店,那里会剥掉你们的皮,把你们像烤野兔那样烤得又香又焦的。你们原不配有这份光荣,可这点小意思,像我这样大方的人却不在乎!……"
接着他走到皮诺乔身边,拼命抚摸着他,并且问:
"这四个该死的小愉勾当,你是怎么发现的,梅拉姆波,我忠实的梅拉姆波,它却一直什么也没发现!……"
木偶本来可以把他知道的事情都说出来,本来可以讲出狗和(又鸟)韶之间的可耻协议。可他想起狗已经死了,心里马上说:"何必告发死者呢?……死者已经死了,还是让它安宁为好!……"
"(又鸟)貂来到空场上的时候,你醒着还是睡了,"农民继续问他。
"我睡着,"皮诺乔回答说,"可给它们的嘁嘁喳喳声吵醒了。其中一只走到狗屋旁边来对我说:'只要你答应不叫醒主人,我们给你一只拔掉毛的肥(又鸟)!……'明白吗,啊,它竟无耻到对我提出这种建议!因为要知道,我虽然是一个木偶,有这个世界的一切缺点,可我从来不是那种贪污受贿、靠不诚实的人来装肥自己腰包的家伙!"
"好样儿的孩子!"农民拍着他的肩膀,叫了一声,"这种想法使你受人敬重。为了证明我极其满意,我这就放你回家。"
农民说着,给他脱掉了狗颈圈。

\chapter{}

皮诺乔一觉得脖子上那个丢脸的、硬绷绷的颈圈没有了,就撒腿穿过田野,一分钟也不停,一直来到通仙女家的那条大道。
到了大道上,他低下头来看下面的草原。他极目远望,清楚地看到那座树林子,他当初就不幸在那里遇见了狐狸和猫;他清楚在看到兀立在许多树木之间的大橡树树梢,他当初就给套着脖子吊在那里摇来晃去。可他这里看,那里看,就是看不到天蓝色头发的美丽仙女的那座小房子。
这时候他感到不妙,马上使出最后的脚劲拼命跑起来,几分钟工夫就来到曾经有一座白房子的草地上,现在白房子没有了,原来是白房子的地方只有一小块大理石碑,石碑上用印刷体刻着如下几行字:
这里安眠着
天蓝色头发的仙女
由于她的弟弟皮诺乔
将她遗弃,
她因悲伤而溘然长逝。
木偶伤心地读完这几行字以后,该是怎么一种心情,就请诸位自己去想象了。他趴倒在地上,把那块大理石碑吻了成千遍,嚎啕大哭起来。他哭了整整一夜,到第二天早晨,到大白天还在哭,虽然眼泪早已哭干了。他哭得这样伤心这样响,周围所有的土岗子都接连发出了回声。
他哭着说:
"噢,我的好仙女,你怎么死了,……为什么是你死而不是我死,我是这么坏,你是那么好!……我的爸爸,你在哪儿啊!我的好仙女,请你告诉我,我到哪儿能够找到他呢?我要永远跟他在一起,不再,不再,不再离开他!……噢,我的好仙女,请你对我说一声,你不是真的死了!……如果你真的爱我……如果你真的爱你的弟弟,那你就复活吧……活过来跟当初一样吧!……你看见我孤零零一个,被所有的人遗弃了,你不觉得难过吗?……要是那两个杀人强盗又到这儿来,重新把我吊在树枝上……那么这一回我就真要永远死了。我孤零零地在这个世界上,叫我怎么办呢,现在你也没有了,我爸爸也没有了,谁给我东西吃呢?夜里叫我到哪儿去睡呢?谁给我做新衣服呢,噢!我还不如死掉好,要好上成千倍!真的,我要死!……哟!哟!哟!……”
他这时候绝望得要把头发拉掉,可他的头发是木头的,连手指也插不进。
这时候一只大鸽子在空中飞过。它张开翅膀停下来,在高空对木偶叫道:
"告诉我,孩子,你在下面干吗呀?"
"你没看见吗?我在哭!"皮诺乔向传来声音的地方抬起头,用上衣袖子擦着眼睛。
"告诉我,"鸽子又说,"你的朋友中间,你知道有一个木偶叫皮诺乔的吗?"
"皮诺乔?……你说皮诺乔?"木偶再说一遍,马上站起来。"皮诺乔就是我!"
鸽子听了这回答,很快地飞下来,到了地上。它比一只火(又鸟)还大。
"那你认识杰佩托?"它问木偶。
"认识杰佩托!他是我可怜的爸爸!他跟你说起我了,你带我上他那儿去好吗?可他还活着不?谢谢你告诉我,他还活着不?"
"三天以前我在海边跟他分手的。"
"他在那里干什么,"
"他在造一只小船要飘洋过海。这可怜人到处找你,整整找了四个多月。可他哪儿也找不到你,现在想到新大陆那些遥远的国家去找。"
"从这里到海边有多远?"皮诺乔焦急不安地问迫,
"一干多公里。"
"一干多公里?噢,我的鸽子,你有翅膀真是太美了!……"
"你要去,我带你去,"
"怎么带法呢?"
"你坐在我背上。你重吗?"
"重,没的事!我轻得像张树叶子。"
皮诺乔二话不说,就跳上鸽子的背,一只脚放在这边,一只脚放在那边,就像骑马似的,然后兴高采烈地大叫:
"快跑,快跑,小马,让我快点到!……"
鸽子飞起来,几分钟就飞得高入云霄。木偶到了这么高的地方,十分好奇,就低头朝下看。可他一看,登时吓得要命,头都晕了。为了别摔下去,他紧紧抱住他那匹长羽毛的飞马的脖子。
他们飞了一整天。天黑了,鸽子说:
"我很渴!"
"我很饿!"皮诺乔跟着说。
"咱们下去,到那鸽子窠呆上几分钟。然后咱们再飞,赶明儿天亮前到海边。"
他们落到一个空了的鸽子窠里。那儿只有一盆水和一篮野豌豆。
木偶有生以来最讨厌野豌豆,一听到野豌豆就作呕,就反胃。可这晚上他大吃特吃,都快吃光了,他才转脸对鸽子说:
"我从没想到,野豌豆这么好吃!"
"你得记住,我的孩子,"鸽子回答说,"一个人到真的饿了,又没别的东西吃的时候,就连野豌豆都好吃了!饥不择食嘛!"
他们在旅途中很快地吃了点东西,歇了一会儿,就动身了!第二天早晨他们来到海边。
鸽子让皮诺乔下来。它做了好事不要听人家说谢谢,马上飞走了。
海边都是人。他们看着大海,又叫又做手势。
"出什么事了?"皮诺乔回旁边一位老大娘。
"是这么回事。一位可怜的爸爸丢了他的儿子,想坐小船到海那边去找。可今天海上风浪大,小船要沉了……"
"小船呢?"
"在那边,我指头指着的地方,"老大娘指着一只小船说。这只小船离得老远,像半个核桃壳,里面有个很小很小的人。
皮诺乔尖起眼睛朝那边仔细一看,登时大吃一惊,尖声高叫:
"那是我爸爸!那是我爸爸!"
这时小船被急浪拍打着,一会儿在汹涌的波浪中消失不见,一会儿又浮了上来。皮诺乔站到一块很高的礁石顶上,不断叫唤他爸爸的名字,一个劲儿挥手,挥手帕,直到摘下头上的帽子来挥,拼命打招呼。
杰佩托虽然了岸很远,好像也认出了孩子,因为他也举起帽子向孩子打招呼,竭力要让孩子知道,他就要回来了,可是海上风浪太大,船桨不顶用,他没法划回岸边来。
忽然一个可怕的大浪打来,船不见了,
大家等着船重新浮水面,可船再也不见上来。
"可怜的人!"聚集在岸边的渔民们说。然后他们低声祈祷着,准备各自回家了。
正在这时候,只听见一声绝望的哀叫。他们回过头来,看见一个孩子从礁石顶上跳进大海,嘴里叫着:
"我要救我的爸爸!"
皮诺乔不过是一块木头,因此很容易就浮到水面上,像条鱼似地游起来。只见他一会儿被波浪一冲,落到水下面不见了,一会儿又在离岸很远的地万重新出现,伸出一条腿或者一条胳膊。最后再也看不见他了。
"可怜的孩子!"聚集在岸边的渔民们说。他们又低声祈祷着,各自回家去了。

\chapter{}

皮诺乔一心想要及时赶到,把他可怜的爸爸救出来,于是游了整整一夜。
这一夜真是恐怖极了!天上下着瓢泼大雨,下着冰雹,打着可怕的响雷,电光闪闪如同白昼。
天亮时候,他终于看见不远的地方有一条长长的地平线。这是海当中的一个孤岛,
他于是拼了命要游到岸上,可是没成功。波浪翻腾追逐,把他像根小树枝或者稻草似地抛来抛去,最后也亏他运气好,一个凶猛的巨浪滚来,把他给扔到沙滩上。
这一下可真重,他给摔到地上,肋骨和全身的关节都咔拉咔拉地响。可他马上庆幸说:
"这一回我总算又侥幸得了救!"
这时天一点一点大亮,太阳出来,光芒四射。海面平静无浪。
木偶脱下衣服,把它铺在地上晒干。接着他望来望去,想在茫茫的水面上看到小船,看到船上那个小小的人。可他看了又看,看见的只有天空、大海和几张船帆。船帆很远很远,像苍蝇似的,
"至少得知道这个岛叫什么名字!"他一面说一面走。"至少得知道这岛上是不是住着什么好人!我想找个好人谈谈,他不会把孩子吊在树枝上的。可我能跟谁打听呢?这儿一个人也没有,我能跟谁打听呢?……"
一想到这空无一人的广阔土地上只有他一个,孤零零,孤零零,孤零零的,他就发愁得要哭了。正在这时候,他忽然看见离岸不远游过一条大鱼。这条鱼自管静静地游,整个头露在水面上。
木偶不知道这条鱼叫什么名字。他高声大叫,让它听见:
"喂一-,大鱼先生,我跟您讲一句话行吗?"
"讲两句也行。"那条鱼回答说。它是世界上所有大海中很客气很少有的海豚。
"请问,在这岛上有没有地方可以吃点东西,却不会被吃掉呢?"
"当然有,"海豚回答说,"而且离这儿不尽就有。"
"该走哪条道上那儿走?"
"走左边那条小道,对着鼻子笔直走。准错不了。"
"再请问一下。您白天黑夜都在海上游,没见过一只小船,里面坐着我的爸爸吗?"
"你爸爸是谁?"
"他是天底下最好的爸爸,就像我是天底下最坏的儿子一样。"
"昨夜刮暴风",海啄回答说,"那小船准沉了。"
"那我爸爸呢?"
"当时一定给可怕的鲨鱼吃下去了。好几天来,这条鲨鱼净在我们这个海里破坏和横扫一切。"
"这条鲨鱼很大很大吗?"皮诺乔问道。这时他吓得打起哆嗦来了。
"大极啦!……"海豚回答说,"为了让你得到一个概念,我给你打个比方吧。它比一座五层大楼还高,嘴巴又大又深,一下子可以开进去整整一列火车,再加上冒烟的火车头。"
"我的妈呀!"木偶惊叫起来。他赶紧穿上衣服,转脸对海脉说:"再见,大鱼先生。请原谅我打扰了您。万分感谢您的好意。"
说时迟那时快,他马上踏上小道,加快步子走了起来,快得就像跑。每次一听别有点声音,他就回头去看,生怕那条五层大楼高、嘴巴容得下一列火车的鲨鱼在他后面追。
走了半小时,他来到一个小国,名字叫做"勤劳蜜蜂国"。街上都是有事情跑来跑去的人。他们全都干活,全都有事做。打起灯笼也找不到一个懒汉和二流子。
"我明白了,"这个不想干活的皮诺乔马上说,"这不是我呆的地方!我生下来可不是干活的!"
这时候他饿得要命,因为他已经二十四小时没吃东西了,连一碟野豌豆也没吃过。
怎么办?
他只有两个办法可以吃到东西:或者是找点活儿干干,或者是讨个子儿或者讨块面包。
乞讨是羞耻的事,因为他爸爸总是对他说,只有年老和残废的人才可以乞讨。在这个世界上,值得我们帮助和同情的真正穷人,只有由于年老和生病,没有办法再用自己的手劳动去挣得面包的人。其他的人都应当劳动,不劳动而挨饿,就是自讨苦吃。
正在这时候,街上来了一个人。他满头大汗,气也喘不过来,一个人费劲地拉着两车煤。
皮诺乔看看他的脸,断定他是个好人,就走过去,很不好意思地垂下眼睛,低声对他说:
"行行好,给我一个子儿吧,我饿得要死了!"
"不是给你一个,"拉煤的回答说,"而是给你四个,只要你帮我拉这两车煤回家。"
"叫我听了奇怪!"木偶几乎生气了说,"告诉您,我从来不当驴子,我从来不拉车!"
"那你最好这么办!"拉煤的人说,"我的孩子,如果你真觉得太饿了,你就切两大片你的骄傲来吃吧,可留神别吃撑了肚子。"
过了几分钟,街上又走过一个砌墙的,肩上扛着一桶灰泥。
"好心的人,行行好,给我这可怜孩子一个子儿吧,我饿得打哈欠了!"
"很高兴。来跟我一起搬这桶灰泥吧,"砌砖的回答说,"我不是给你一个子儿,而是给你五个。"
"可灰泥太重了,"皮诺乔回答说,"我不想花这力气,弄得筋疲力尽。"
"要是你不想花力气,那么,我的孩子,你就舒野服服打你的哈欠吧,会给你带来好处的。"
不到半小时,至少走过了二十个人。皮诺乔向他们一个个讨钱,可他们都回答说:
"你不害臊吗?你不要当街乞讨了,还是找点活儿干干,学着自己挣面包吃吧!"
最后走过一位和善的小妇人,她提着两瓦罐水。
"好太太,让我在您的瓦罐里喝一口水好吗?"皮诺乔说,他渴得喉咙发烧,
"你就喝吧,我的孩子!"小妇人说着,把两瓦暖水放在地上,
皮诺乔像块海绵似地吸饱了水,然后擦着嘴,低声咕噜说:
"嘴是不渴了!肚子也不饿就好了!……"
好心的小妇人听了这两句话,马上接下去说:
"这里是两瓦罐水,你帮我拿一瓦罐,送到我家里,我就给一大块面包。"
皮诺乔看着瓦罐,不说好也不说不好。
"除了面包,还给你一大盆花椰莱,上面浇上油和辣酱油,"好心的小妇人又说。
皮诺乔又看了瓦罐一眼,还是不说好也不说不好。
"吃完花椰菜,我给你一块好吃的酒心糖。"
皮诺乔给最后一样好吃的东西吸引住,再也没法抗拒,下定决心说:
"没办法!就给您把这瓦罐水送到家去吧!"
瓦罐很重,木偶用两只手拿不动,就用头来顶。
到了家里,好心的小妇人让皮诺乔坐在一张铺好台布的小桌子旁边,在他面前放上面包、调好味的花椰菜和酒心糖。
皮诺乔不是吃而是吞。他的肚子像一间五个月没住人的空屋。
肚子本来饿得像咬一样痛,这时一点一点不痛了,他就抬起头来,想要谢谢给他吃饭的小妇人。可是才看第一眼,他就惊奇得拖长声音大叫:"噢一-!"他坐在那里呆呆的一动不动,眼睛瞪圆,叉子高高举着,嘴巴里塞满了面包和花椰菜。
你为什么这样惊奇呀?"好心的小妇人笑着说。
"您是……"皮诺乔结结巴巴地回答,"您是……您是……您好像是……您让我想起了……对,对,对,同样的声音……同样的眼睛……同样的头发……对,对,对……您也有天蓝色的头发……像她一样!……唤,我的好仙女!……唤,我的好仙女!……跟我说一声就是您吧,的的确确就是您吧!……别叫我再哭了!你要是知道就好了!……我已经哭够了,我已经受够苦了!……"
皮诺乔这么说着,哭得泪如泉涌,跪倒在地,抱住这神秘小妇人的膝盖。

\chapter{}

好心的小妇人起先说她不是那位天蓝色头发的小仙女。可后来看到识破了,也不想再把这场喜剧继续演下去,终于承认她就是小仙女,她对皮诺乔说:
"你这木头小鬼!你怎么认出是我的?"
"我热爱您,就认出是您了。"
"你记得吗?你扔下我的时候,我还是个小姑娘,可你现在碰到我,我已经是个妇人了。我简直可以做你的妈妈了,"
"那我太高兴了,这样我就不是叫您姐姐,而要叫您妈妈了。多少日子以来,我一直想跟所有的孩子那样有个妈妈!……可您怎么会长得这样快的?"
"这是一个秘密。"
"告诉我吧。我也想长大一点。您没看到吗?我还是跟一个子儿的干酪那么高。"
"可你不会长大。"仙女回答说。
"为什么?"
"因为木偶是从来不长大的。他们生下来是木偶,活着是木偶,死了也是木偶。"
"噢!我老做木偶都做腻了!"皮诺乔拍着后脑勺大叫着说,"我现在要变人,跟所有人一样的人。"
"你会变人的,如果你配得上变人的话……"
"真的吗?怎么办才配得上呢?"
"容易极了,只要你一直做个好孩子。"
"噢,我不是个好孩子吗?"
"根本不是!好孩子听话,可你正好相反……"
"我从来不听话。"
"好孩子爱读书爱干活,可你……"
"正好相反,我一年到头偷懒,吊儿郎当。"
"好孩子向来说真话……"
"可我向来说假话。"
"好孩子高高兴兴去上学……"
"可学校叫我肚子疼。不过从今以后,我要改变我的生活。"
"你答应我这样做吗?"
"我答应你这样做。我要变成一个好孩子,我还要成为我爸爸的安慰……这会儿我可怜的爸爸在哪儿呢?"
"我不知道。"
"我还能看见他和拥抱他吗,我有这份福气吗?"
"我相信你有,而且我保证。"
皮诺乔听了这个回答,高兴得抓住仙女的手发疯似地吻起来。接着他抬起头,亲热地看着仙女问道:
"告诉我,好妈妈,你不是真死了吧?"
"好像不是,"仙女微笑着回答。
"你知道我当时多么伤心,觉得喉咙都堵住了,一个字一个字地读:'这里安眠着……'"
"我知道,因此我就原谅你了。你的伤心是真诚的,这使我知道,你有一顾善良的心。一个孩子有颗善良的心,即使有点顽皮,有些不好的习惯,总还是有希望,就是说,总是可以希望他重新走上正路的,因此我跟到这儿来找你。我要做你的妈妈……"
"噢!多美啊!"皮诺乔大叫,高兴得跳起来。
"你要听我的话,永远照我对你说的话去做。"
"我愿意,我愿意,我愿意!"
"从明天开始,"仙女往下说,"你就去上学。"
皮诺乔马上不那么高兴了。
"然后选择一种你喜欢的手艺或者工作……"
皮诺乔脸都板起来了。
"你牙齿缝里咕噜些什么,"仙女用不高兴的口气问他。
"我说的是……"木偶低声叽叽叫着说,"现在才去上学,好像晚了点……"
"一点也不晚,我的宝贝。你记住了,读书学习是永远不会晚的。"
"可我不想干手艺或工作什么的……"
"为什么?"
"因为我觉得干活太累。"
"我的孩子,"仙女说,"说这种话的人,最后差不多不是进监狱就是进医院。告诉我,一个人不管生下来是穷是富,在这个世界上都得做点事,干上一行,都要劳动。懒惰没有好结果!懒惰是一种最坏的毛病,必须马上从小治好。要不,大了就再也治不好了!"
这番话打动了皮诺乔的心。他高兴地又抬起头来,对仙女说:
"我要学习,我要干活,你对我怎么说我就怎么做,一句话,木偶的生活我过腻了,我无论如可要变成一个孩子。你答应我了,不是吗?"

\chapter{}

第二天皮诺乔就上了公立学校。
诸位想象一下,孩子们看见一个木偶进他们的学校,会怎么样捉弄他!他们哈哈大笑,笑个没完没了。有人开这种玩笑,有人开那种玩笑。有人摘他的帽子,有人打后面拉他的小背心。有人想用墨水在他鼻子下面画两撇大胡子,有人想用线绑在他的脚上和手上,好牵着线让他跳舞。
起初皮诺乔很镇静,不去理睬他们。可后来忍不住了,他向把他缠得最厉害、捉弄得最凶的人转过身去,板起脸说:
"小心点,孩子们。我上这儿来可不是给你们当小丑的。我尊重大家,希望大家也尊重我。"
"好一个小鬼!你说出话来像一本印出来的书!"那些顽皮孩子大叫,哈哈笑得跳起来。一个最大胆妄为的孩子伸手要抓木偶的鼻尖。
可他没来得及抓住,因为皮诺乔在桌子下面伸出脚来,在他小腿骨上狠狠踢了一下。
"唉哟!多硬的脚啊!"那孩子大叫,拼命搓给木偶踢出来的乌青。
"还有胳膊肘!……它比脚还硬!"另外一个说。他因为开无礼的玩笑,肚子给木偶的胳膊后顶了一下。
经过脚这么一踢,胳膊后这么一顶,皮诺乔马上得到全校学生的尊重和同情。他们都很喜欢他。
老师看见他上课专心,读书用功,肯动脑筋,总是第一个进学校,放学最后一个走,也很称赞他。
他唯一的缺点就是结交的同学太多。其中有不少是鼎鼎大名的小坏蛋,一点不想读书,一点不想有出息。
老师天天要他小心,善良的仙女也不断翻来覆去对他说:
"小心啊,皮诺乔!迟早有一天,你这种同学会使你不爱学习的,说不定有一天还会使你闯大祸。"
"不要紧!"木偶耸耸肩膀回答说,还用食指敲敲脑门,表示:"这里面有脑筋的!"
于是有一天,他在上学的时候,忽然遇到那一帮同学迎面走来,对他说:
"有一个重要的新闻你知道吗?"
"不知道。"
"这儿海边来了一条鲨鱼,大得像座山?"
"真的?……是那条鲨鱼吗,当时吃掉我可怜的爸爸的?"
"我们到海边去看。你也去吗?"
"我吗?不去。我要上学校去。"
"学校有什么要紧?咱们明天再上学吧。多上一课少上一课都一祥,反正不懂,还是驴子。"
"老师会怎么说呢?"
"让老师去说吧。他天天咕噜,也该给他点报应了。"
"那我妈妈呢?……"
"咱们的妈妈什么都不会知道的。"这些顽皮孩子说。
"你们知道我怎么办吗?"皮诺乔说,"我因为某种理由,也要去看看这条鲨鱼……可我下了课去。"
"可怜的糊涂虫!"有一个孩子大叫,"这么大一条鱼,你以为它会呆在那儿等着,随便你什么时候去看吗?它给人一搞烦,转过身子就上别处去了,要看也看不到啦。"
"打这儿到海边要走多久,"木偶问道。
"来回一个钟头。"
"那就去吧!谁跑得最快谁就最有种!"皮诺乔叫道,
这声起步信号一响,一帮顽皮孩子就把他们的书刊、练习本啊夹在胳肢窝里,抢着向田野奔去了。皮诺乔跑在最前面,只觉得脚上长了翅膀。
他不时回过脸去,笑话那些落在他后面有好一段路的同学。他看见他们气喘咻咻,上气不接下气,浑身是尘土,吐出了舌头,不由得衷心大笑起来。这时候,这可怜的家伙还不知道他正在走向什么样的可怕事情,走向什么样的可怕灾难呢!……

\chapter{}

皮诺乔一到海边,马上放眼向海上眺望,可是什么鲨鱼也没看见。大海平静得像一面水晶的镜子。
"喂,鲨鱼在哪儿?"他转脸问同学,
"吃早饭去了,"一个同学笑着回答说。
"要不就是上床去睡一会儿了,"另一个同学接上一句,笑得前仰后合。
皮诺乔听了这些乱七八糟的回答和莫名其妙的狂笑,知道是他那些同学跟他开了个大玩笑,骗他上了当。他十分恼火,气乎乎地说:
"怎么?拿鲨鱼的鬼话来骗我,这是什么道理?"
"当然有道理!……"那些小坏蛋异口同声说,
"什么道理?……"
"让你不去上学,让你跟我们走。你天天上课那么认真,那么用功,你不害臊吗?像你那么学习,你不害臊吗?"
"我学我的,跟你们有什么关系?"
"跟我们关系大极了。这一来,老师就觉得我们不好……"
"为什么?"
"有人爱读书,就使我们这种不愿意读书的人丢脸,可我们不想丢脸!我们也有我们的自尊心!"
"那我该怎么办,你们才高兴呢?"
"你也应该讨厌学校,讨厌功课,讨厌老师。这是我们的三大敌人。"
"如果我想要继续学习呢?"
"那我们就对你不客气了,一有机会就要跟你算申账!……"
"你们简直叫我好笑。"木偶摇摇头说。
"哼,皮诺乔!"孩子当中最大的一个走到他面前叫道,"别到这儿来夸口,别到这儿来斗嘴!……你要是不怕我们,我们也不怕你!记好了,你只有一个,我们有七个。"
"七个什么,七个大罪(1),"皮诺乔大笑着说。
"你们听见了吗,他侮辱我们大伙儿,他管我们叫七个大罪!……"
"皮诺乔!你侮辱了我们,要你向我们道歉……要不你就倒霉!……"
"咕咕!"木偶叫着,用食指刮刮鼻尖,表示讥笑他们。
"皮诺乔,你没有好结果!……"
"咕咕!"
"我们要像揍驴子那么揍你!……"
"咕咕!"
"你要带着个打扁的鼻子回家!……"
"咕咕!"
"我们这就来揍你一个咕咕!"这帮小坏蛋当中最凶的一个叫道。"受用受用这个吧,今天晚上就不用吃晚饭了。"
他说着就给了木偶脑袋上一拳头。
老话说,一报还一报,因此可以想象到,木偶马上就回敬他一拳头。这么你一拳来我一拳去,这场架就越打越大,越打越厉害了。
皮诺乔虽然只有一个,可自卫得像个英雄似的。他用两只硬绷绷的木头脚踢得那么利索,叫他那些敌人离得远远的不敢走近。凡是他的脚所碰到之处,马上就留下纪念品——一大块乌青。
孩子们眼看不能同木偶肉搏,气得要命,心想最好还是扔东西,就打开书包,开始向他扔语文课本、文法书、小戒尺、小零碎、图瓦尔的故事书、巴契尼的《小(又鸟)》以及其他教科书。可是木偶眼疾手快,全都及时躲开,因此书一本本地打他头上飞过去,全落到海里去了。
诸位想象一下那些鱼吧1鱼以为扔到水里来的这些书是好吃的东西,赶紧成群游到水面上来。它们咬咬纸张,咬咬封面,马上就吐出来,撇撇嘴,像是要说:"不配我们的口味。我们吃惯了更好吃?亩鳎 ?
这时候,架越打越厉害了。一只大螃蟹打水里出来,慢慢地、慢慢地爬到岸上,用漏风大喇叭似的难听声音叫道:
"停手吧,你们真是些小淘气!孩子们这样打架难得有好结果。总归要闯祸的!……"
可怜的螃蟹!它等于是对风在叫。皮诺乔这小鬼反而回过头,狠狠地看着它,蛮横地叫:
"讨厌的螃蟹,闭上你的嘴!你最好还是去吃两片地衣药片,把你的伤风给治治。趁早上床,想办法出身大汗吧!"
这时候,那帮孩子已经把自己的书扔完,猛看见木偶的书包就在不远的地方。说时迟那时快,他们一下子把它抢了过来。
在木偶的书当中,有一本书用厚板纸装帧,书脊书角都包着漆皮纸。这是一本算术书,请诸位想想,这本书该有多沉!
一个小坏蛋抓住这本书,瞄准皮诺乔的脑袋,用足力气扔过来。可是他没扔中木偶,却扔在一个同学的头上了。这个同学的脸登时白得像切开的面包,只叫出了两声:
"噢,我的妈,救救我……我要死了!"
接着他就直挺挺倒在沙滩上。
孩子们看见闹出了人命,这一惊非同小可,马上撒腿就逃,一转眼就没影了。
这时候只剩下皮诺乔一个人。他虽然又难过又害怕,吓了个半死,可还是跑到海边,把手帕浸透了海水,回来敷在他这位可怜同学的太阳穴上。他一面绝望地大哭,一面叫他这位同学的名字说:
"埃乌杰尼奥!……我可怜的埃乌杰尼奥!……张开你的眼睛看看我!……你为什么不回答我呀?你知道,不是我弄得你这样的!相信我,不是我干的!……张开你的眼睛吧,埃乌杰尼奥……你要是老闭着眼睛,我也要死了……噢,我的老天爷!这会儿我怎么回家呢?……我怎么有勇气去见我的好妈妈呀,我将会怎么样呢?……我该逃到什么地方去啊,我上什么地方才韵躲得开呢?……噢!要是我去上学,那就好多了,那就要好上千倍了!……这些同学是我的冤家对头,我为什么听他们的话呢?……老师曾经跟我说过!……我妈妈也翻来覆去对我说:‘小心坏同学!’可我总是不听……我固执极了……他们讲他们的,我干我的!如今报应来了……打我出世起,就因为我该死的脾气,我压根儿连一刻钟的好日子都没过过,我的天!我将会怎么样呢?我将会怎么样呢?我将会怎么样呢?……”
皮诺乔一个劲地哭着,喊着,敲着脑袋,叫着可怜的埃乌杰尼奥的名字,直到猛听见沉重的脚步声走过来。
他转脸一看,是两个警察,
"你干吗这么趴在地上?"他们问皮诺乔。
"我在救护我这同学。"
"他病了?"
"好像是的!……"
"只是生病吗?"一个警察靠近埃乌杰尼奥,低下头来把他好好看了看,"这孩子一边太阳穴受伤了,谁打伤他的?"
"不是我。"木偶结结巴巴地说,他气都透不过来了。
"不是你又是谁?"
"不是我。"皮诺乔再说一遍。
"他是给什么东西打伤的?"
"给这本书。"的算术书,给警察看。
"这本书是谁的?"
"是我的。"
"这就够了,再不用别的什么了。马上起来,跟我们走。"
"可我……"
"跟我们走!"
"可我是无辜的……"
"跟我们走!"
在走以前,两个警察叫来几个渔民。这几个渔民这时候正好坐船从岸边经过。警察对他们说:
"这孩子头部受了伤,现在交给你们。你们把他带回去救护。明天我们再来看。"
然后他们回到皮诺乔身边,把他夹在中间,用军人口气命令说:
"开步走!走快点!不然要你够受的!"
不等他们说第二遍,木偶就走起来了。他们走的这条小道是进村的,这可怜小鬼简直不知道自己是不是还活着。他只觉得像在做梦,而做的是多可怕的恶梦啊!他完全吓得魂不附体。他的眼睛发花,两腿发抖,舌头贴着上腭,连一个字也说不出来。不过他尽管这样昏昏迷迷,还是感到心里像针扎似地痛,因为他想到,他要夹在两个警察中间经过他那好仙女的窗下。他真情愿死了拉倒。
他们已经来到村边,正在进村,忽然刮来一阵狂风,把皮诺乔头上的帽子吹起来,吹了有十步远。
"答应我好吗?"木偶对两个警察说,"让我去把我的帽子捡起来。"
"去吧,可得决点。"
木偶走过去捡起帽子……可没戴到头上,却放在嘴里,用牙咬着,撒腿就向海边飞跑。他快得像一顾出膛的子弹。
两个警察眼看很难追上,就放出一条凶猛的大狗去追他。这条狗在赛狗中还得过冠军。皮诺乔拼命跑,可狗跑得比他快。所有的人或者把头探出窗子,或者挤在路当中,急于要看这场激烈赛跑的结果如何。可是他们这个希望落了空,因为那条猛犬和皮诺乔一路上搞得灰尘滚滚,几分钟以后就什么也看不见了。
(1)根据天主教的戒律,七大罪是傲慢、淫欲、嫉妒、激怒、吝啬,贪吃、怠惰。 

\chapter{}

这场你死我活的赛跑,已经到了千钧一发的时刻,皮诺乔心想,这回准定要输了,因为要知道,阿利多罗(就是那条猛犬的名字)使劲地跑啊,跑啊,差不多就要追上他。
只说一点就够了:木偶已经听到这条恶犬在他身后一巴掌远的地方很急促的喘气声,甚至感觉到了它呼吸的热气。
幸亏这时已经到了海边,眼看大海只有那么几步远了。
木偶一到海边,就像小青蛙似的,很利落地扑通一声,跳到了水里。阿利多罗正好相反,想马上停住脚步,可跑得太快了,脚步收不住,跟着也扑通一声落到了水里。这只倒霉的狗不会游泳,因此两条脚马上乱划,想要浮在水面。可它越划越往下沉,连头都沉到水底下去了。
等到这条可怜的狗把头伸出来,它吓得两眼瞪大,汪汪叫着说:
"我要淹死了!我要淹死了1"
"那就死吧!"皮诺乔在远处回答。现在他看到,他再也没有什么危险,已经万无一失了。
"救救我,我的小皮诺乔!……快救救我的命吧!……"
这几声汪汪叫十分悲惨,木偶本心很好,禁不住心软下来,转脸对狗说:
"可我救了你,你保证不再找我麻烦,不再追我队吗?"
"我保证,我保证!快帮忙吧,再过半分钟,我就完蛋了。"
皮诺乔先还犹豫了一下,可终于记起他爸爸一再说过的话,做好事永远不吃亏,就游到阿利多罗身边,伸出两手,一把抓住了它的尾巴,把它活生生拉上干燥的沙滩。
这条可怜的狗站都站不住了。它不得已喝了那么多咸水,肚子胀得像个大皮球。可是木偶不太相信它,觉得还是小心点好,于是重新跳到海里。他离岸远远的,对他救起来的朋友叫道:
"再见,阿利多罗,一路平安,给我向你一家问好。"
"再见,小皮诺乔,"狗回答说,"万分感谢您救了我的命。您帮了我一个天大的忙。在这个世界上善有善报,一有机会,我要报答您的。"
皮诺乔继续紧靠着岸边游。最后他觉得已经到了安全的地方,朝岸上看看,看见礁石上有个山洞,山洞里冒出烟来,飘得高高的。
"这山洞里一定有火,"他自言自语说。"那多好啊!让我上去把身子烤烤干,烤烤暖和。然后呢?……然后该怎么办就怎么办吧。"
他拿定了主意,就向礁石游过去。可他到了那里正要上岸,忽然觉得水底下有样东西升起来,升啊,升啊,把他一直托到空中,他马上打算逃走,可已经来不及了,因为使他惊奇万分的是,他竟在一个大鱼网里,夹在一大堆鱼中间。这些鱼形形色色,有大有小,正拼了命啪哒啪哒摇着尾巴挣扎。
正在这时候,他看见山洞里走出一个渔夫,样子太难看了,难看得简直像个海怪。他的头发不是头发,是一大蓬绿草。他身上的皮肤是绿的,眼睛是绿的,胡子老长老长,一直垂到脚上,也是绿的。他活像一条用后脚直立的绿色大晰蜴。
渔夫把鱼网打海里拉出来,兴高采烈地叫道:
"老天爷保佑!今天我又可以大吃一顿鲜鱼了!"
"幸亏我不是鱼!"皮诺乔心里说。他又有了点勇气。
一网鱼都拿到山洞里。山洞里很黑,满是烟。山洞当中有一只大油锅在沸腾,发出一股叫人没法呼吸的烧灯芯气味。
"我来看看捉到了什么鱼!"绿莹莹的渔夫说着,把烘炉铲子似的一只大手伸进鱼网,抓出一把火鱼。
"这些火鱼不错!"他看了看,很满意地闻了闻,说。他闻过以后,就把它们扔进一个没水的缸里。
接看他又照样来一次。就这样,他一次又一次把鱼捞出来,觉得要流口水,欢天喜地说:
"这些鳕鱼好极了!……"
"这些鰡鱼妙极了!……''
"这些板鱼味道不错!……"
"这些狼鱼味道很鲜!……"
"这些鳀鱼八成很好吃!……"
诸位可以想象,这些鳕鱼、鰡鱼、板鱼、狼鱼、鳀鱼全都劈哩啪啦落到缸里,跟最先扔进去的火鱼在一起。
最后-个留在网里的是皮诺乔。
渔夫把他一抓出来,两只绿色大眼睛登时都吓得瞪圆了。他几乎是害怕地叫起来:
"这是什么鱼?我想不起我曾经吃过这种鱼!"
他把木偶再仔仔细细地看了一遍,等到看仔细了,最后说:
"我明白了。这准是海里的螃蟹。"
皮诺乔听说把他当作螃蟹,觉得是个耻辱,生气地说:
"什么螃蟹不螃蟹?瞧你把我当什么啦!告评你,我是木偶。"
"木偶?"渔夫反问。"说真个的,木偶鱼对我来说是一种新的鱼!那更妙了,我更想吃你了。"
"吃我?可您不懂吗,我不是鱼?您不觉得我跟您一样,会说话会思想吗?"
"那倒是一点不错,"渔夫往下说,"我看你鱼还是鱼,可是很幸运,跟我一样会说话,会思想,因此我很愿意给你应有的照顾。"
"什么照顾?……"
"为了表示友好和对你的特殊敬意,我让你自由选择怎么烧法。你要在油锅里炸呢?还是要在平底锅里加上番茄酱煎呢?"
"说老实话,"皮诺乔回答说,"如果要我选择的话,我宁可请您放了我,让我回家去。"
"你在开玩笑!这么一条少有的鱼,你以为我会放过机会不尝它一尝吗?在这里海上还从来不知道有木偶鱼!依我的办吧,我把你跟所有的鱼一块儿放在油锅里炸,你会满意的。有那么多鱼作伴一起挨炸,总归是一种安慰。"
不幸的皮诺乔一听明白这意思,就开始哇哇大哭,怨天怨地说:
"我当初去上学该多好!……可我听了同学的话,现在报应来了!……咿!……!咿……!咿……"
由于他扭得像条鳗鱼,使出叫人难以相信的力气要挣脱绿莹莹的渔夫的手,这双手就拿起一束结实的蒲草,把皮诺乔的双手双脚捆起来,捆得像根香肠,扔到缸底跟其他的鱼在一起。
接着他拉出一大木盘面粉来拌所有的鱼,一条一条都拌好了,就扔到油锅里炸。
最先在沸腾的油里跳舞的是可怜的鳕鱼,接着挨到狼鱼,接着挨到鰡鱼,接着挨到板鱼和鳀鱼,最后挨到了皮诺乔。皮诺乔看到死期已至(死得多惨啊!),不由得浑身发抖,害怕得既发不出声音,也透不过气来,根本没法子哀求饶命。
这可怜的孩子只好用眼睛来哀求!可是那绿莹莹的渔夫根本没注意到。他把木偶在面粉里拌了五六遍,从头到脚拌了个透。皮诺乔浑身都是面粉,就像个小石膏像。
接着渔夫抓住他的头,一举手就……

\chapter{}

渔夫一举手就要把皮诺乔扔进油锅,可正在这节骨眼上,一条大狗跑进山洞来。它是给炸鱼的浓烈香味招引来的。
"出去!"渔夫吓唬着对狗吆喝,手里仍旧拎着满身是面粉的木偶。
可怜的狗实在太饿了,它摇晃着尾巴汪汪地叫,像是说:
"给我点油炸鱼,我就不打扰你了。"
"我对你说,出去!"渔夫再说一遍,伸出腿来就给它一脚。
狗到当真饿了的时候,是不习惯于让人这样对待它的。它向渔夫转过脸来,呲起两排可怕的牙齿。
正在这时候,它听见山洞里发出一个很微弱很微弱的声音,说:
"救救我,阿利多罗!你不救我,我就要给油炸了!……"
狗马上听出了皮诺乔的声音。它觉得最奇怪的是,这微弱声音是渔夫手里那团沾满面粉的东西发出来的。
这时候它做了件什么事呢?这狗从地上猛地跳得半尺高,咬住那团沾满面粉的东西,用牙轻轻地叼着,就冲出山洞,像闪电似地溜掉了。
渔夫一心想吃这条鱼,眼看它打手里给抢走了,气得发疯,就想去追那条狗。可走了几步,忽然咳嗽得没办法,只好回来,
这时候阿利多罗又来到通村子的小道,停下脚步,把它的朋友皮诺乔小心翼翼地放在地上。
"我该怎么谢你呀!"木偶说。
"不用谢,"狗回答说,"你救过我的命,善有善报。要知道,在这个世界上大家应该互相帮助。"
"可你怎么会到这山洞来的?"
"我一直在海边直挺挺地躺着,半死不活的,忽然一阵风打远处吹来了炸鱼的香味。这股香味引起了我的食欲,我就跟着它走。要是来晚一分钟就糟了!……"
"别说了,别说了!"皮诺乔又吓得浑身发抖,叫着说,"你别说了!你要是晚来一分钟,这会儿我已经给炸熟,被吃掉,消化了。Brrr!……一想到这个我就发抖啦!……"
阿利多罗笑着向木偶伸出右爪子,木偶使劲紧紧地握住它,表示极其友好的感情。接着他们就分手了。
狗重新取道回家。皮诺乔一个人留下来,向不远的一间小茅屋走去。小茅屋门口坐着一位老人,正在晒太阳。木偶问他说:
"请您告诉我,好心的老人家,您知道一个可怜孩子,叫埃乌杰尼奥的,脑袋给打伤了吗?……"
"一些打鱼人把他送到这茅屋里来了。现在他……"
"现在他死了!……"皮诺乔极其伤心地打断他的话。
"没有,他现在活着,已经回家去了。"
"真的吗,真的吗?"木偶高兴得跳起来,叫道,"这么说,伤不重,……"
"它有可能造成严重后果,甚至死人,"老年人回答,"因为他是给一本厚板纸封面的大书打中了脑袋。"
"谁打伤他了,"
"一个同学,叫皮诺乔的……"
"这皮诺乔是谁,"木偶假装不知道,问道。
"他们说是个小坏蛋,是个小流氓,是个真正的小无赖……"
"造谣!完全是造谣!"
"你认识这皮诺乔?"
"见过!"木偶回答说。
"你看他怎么样,"老年人问他。
"依我说,他是个好极了的孩子,一心想读书,又听话,又爱他的爸爸,又爱他的一家人……"
木偶正这样一口气地撒着谎,摸摸鼻子,发觉鼻子已经长了一个多手掌。他害怕得叫起来:
"好心的老人家,我扯了一通关于他的好话,您可全都别信。因为我熟悉皮诺乔,可以保证他真正是个小坏蛋,不听话,不学好,不去上学,却跟着一帮子同学去东游西荡!"
这番话一说完,他的鼻子就缩小,恢复了原来的样子。
"为什么你整个人白成这样?"老年人忽然问他。
"我告诉你……我没留神,在一堵新刷白的墙上擦了一下,"木偶回答说。他不好意思承认他被当作鱼拌上面粉,预备扔进油锅里去炸。
"噢,你的上衣,你的短裤,还有你的帽子,你都怎么啦?"
"我遇到了强盗,把我给剥了。您说吧,好心的老人家,您没有一点什么可以给我穿穿,让我好回家去吗?"
"我的孩子,说到可以穿的东西,我只有那么个小口袋,装扁豆的。你要就本去吧。就在那儿。"
皮诺乔不等他说第二遍,马上拿起这个装扁豆的空口袋,用剪刀在袋底开了一个洞,在两边开了两个小洞,就当衬衫穿。他一下子把脑袋和双手钻过那些洞,穿好了,就动身上村里去。
可他一路上感到心里不踏实。老实说,他是进一步又退一步。他一边走一边自言自语说:
"我有什么脸去见我那好心的仙女呢?我见了她说什么好呢?……我又做出这桩坏事,她会原谅我一次吗?……可以打赌,她不会原谅了!……唉!她准不会原谅我……这是我活该,因为我是个小坏蛋,答应好了改过,结果又违背了诺言!……"
他来到村里,天已经黑了。天气很坏,下着瓢泼大雨。他径直上仙女家,决定敲敲门,自己就开门进去。
可是一到那里,他觉得勇气没有了,不是去敲门,却是往回跑了二十来步。他第二次走到门口,还是不敢敲门。他第三次走到门口,依然不敢敲门。第四次他才算发着抖,拿起铁门锤,轻轻地把门敲了敲。
他等啊,等啊,最后过了半个钟头,最高一层(这是座四层楼房)才打开窗子,皮诺乔看见一只大蜗牛探出头出来,头上有盏点亮的小灯。这蜗牛说:
"这么晚了,是谁呀?"
"仙女在家吗?"木偶问它。
"仙女睡了,不要人叫醒她。你倒是谁?"
"是我!"
"这个我是谁?"
"皮诺乔。"
"皮诺乔是谁?"
"是木偶,原先跟仙女住在一起的。"
"啊,我明白了,"蜗牛说,"你等等我,我这就下来给你开门。"
"谢谢你快一点,我都要冷死了。"
"我的孩子,我是一只蜗牛,蜗牛永远快不了的。"
过了一个钟头,过了两个钟头,可门还没有开。皮诺乔又是冷,又是害怕,又是浑身水淋淋,因此直打哆嗦。于是他拿定主意再敲一次门,这回敲得比上一回响。
听见这第二次敲门声,第四层下面一层的窗子打开了,还是那只蜗牛探出头来。
"我的好蜗牛,"皮诺乔打下面街上叫,"我已经等了两个钟头了!这么可怕的夜,两个钟头比这两年还长。帮帮忙,请您快一点。"
"我的孩子,"这小生物不急不忙,十分平静,在窗口回答说:"我的孩子,我是一只蜗牛,蜗牛都是快不起来的。"
窗子又关上了。
不多一会儿就敲半夜十二点,接着半夜一点,接着是半夜两点,门还是关着。
皮诺乔可忍不住了。他气得抓住门锤,就要用力撞门,让整座房子给撞得摇晃起来。可铁门锤一下子变了活鳗鱼,打他手里滑出来,钻到路当中的水坑里不见了。
"啊!是这样?"皮诺乔越发气昏了,叫道,"门锤没有了,我就用脚狠狠地踢。"
他退后两步,然后冲过去在门上狠狠一脚。这一脚踢得可厉害,半条脚都插到门里去了。木偶想拔出腿,可用尽了力气也拔不出来。这半条腿像敲弯的钉子似的,牢牢钉在那里了。
请诸位想象一下这可怜的皮诺乔吧!整个下半夜他就这么一条腿站在地上,一条腿翘着。
等到天亮,门终于开了。蜗牛这要命的小生物整整花了九个钟头,才下完四层楼,来到临街的大门口。得说句老实话,它已经走得满身大汗了!
"你干吗把一条腿插在门里?"它笑着问木偶。
"真倒霉。您倒瞧瞧,好蜗牛,看有什么办法让我不受这份罪。"
"我的孩子,这件事得找木匠,我可从来没当过木匠。"
"替我求求仙女吧!……"
"仙女睡了,不要人叫醒她,"
"我整天钉在这门上,您叫我干什么呢?"
"您就自得其乐,数数路上走过的蚂蚁吗。"
"您至少给我点什么吃吃,我都要饿死了。"
"马上拿来!"蜗牛说。
实际上又整整过了三个半钟头,皮诺乔才看见它顶着个银托盘回来。托盘上有一个面包、一只炸(又鸟)和四个长熟了的杏子。
"这是仙女给您送来的早饭,"蜗牛说。
木偶看到这顿大菜,感到浑身来劲了。可等到他一吃,马上就倒胃口,原来面包是白垩做的,炸(又鸟)是厚板纸做的,四个杏子是石膏做好,涂上颜色的!
他失望得想哭,想把托盘边同上面的东西一起甩掉,可不知是由于太伤心呢还是太饿,一下子昏倒了。
等到他醒来,他已经直挺挺躺在一张沙发床上,仙女就在他身边。
"这一回我也原谅了你,"仙女对他说,"可你再给我来这么一次,就没你好的!……"
皮诺乔赌咒发誓,说他要用功读书,做个很好很好的孩子。这一年下来,他都守住他的诺言。的确,他大考光荣地得了全校第一名,品行总的说来也得到好评,令人满意。因此仙女十分高兴,对他说:
"你的愿望明天终于要实现了!"
"你说什么?"
"到明天你就不再是一个木偶,而要变成一个真的孩子了。"
诸位没看到皮诺乔那份乐劲!他一直盼望着这个消息,如今听了,他那份高兴简直是无法想象的。为了庆祝这件大喜事,明天仙女家要举行盛大的早宴,把他所有的朋友和同学都请来参加。仙女答应准备两百杯牛奶咖啡和四百片面包,每片面包都两面涂上黄油。没问题,这准是个极其快活,极其美好的日子,可是……
真不幸,木偶一生中老这么可是,可是的,这一来,就把什么事情都给毁了。

\chapter{}

当然,皮诺乔马上就得到仙女同意,进城去把要请的人都给请来。临走时仙女对他说:
"那你就去请你的同学们明天来参加早宴吧。可你记住了,天黑前就得回家。明白了吗?"
我保证一个钟头就回来,木偶回答说。
"留神点,皮诺乔:孩子们总是答应起来很爽快,可做起来却慢腾腾的。"
"我跟别人可不一样,我说到做到。"
"咱们看吧。万一你不听话,你就要吃更大的苦头。"
"为什么?"
"因为孩子不听比他们懂得多的人劝,总是要倒霉的。"
"我已经尝到过滋味!"皮诺乔说,"现在我不会再犯老毛病了!"
"我说的话是真是假,咱们看吧。"
木偶二话不说,就跟做他妈妈的好心仙女告了别,又唱又跳地出门去了。
一个钟头多一点,所有的朋友他都请到了。有人一听就高兴地接受邀请。有人还得先求一求,可一听说有牛奶咖啡喝,还有两面涂黄油的面包吃,就都说:"我们为了让你高兴,要来的。"
现在诸位要知道,皮诺乔在他的朋友和同学中间,有一个最知己最要好的,名字叫罗梅奥,可大家给他取了个绰号叫"小灯芯",因为他又干又瘦,活像晚上小油灯点的一根新灯芯。
小灯芯在全校学生当中最懒情最捣蛋,可皮诺乔却很喜欢他。事实就是这样,他一开头就上他家去找他,要请他赴早宴,可没碰到。他第二次去,小灯芯不在家。他第三次去,还是白跑。
哪儿能找到他呢?这里找,那里找,最后总算看见他躲在一间农舍的门廊里。
"你在这儿干吗?"皮诺乔走过去问他。
"等半夜好离开这里……"
"上哪儿去?"
"上很远很远的地方去!"
"我可是上你家找你三次了!……"
"你找我干吗?"
"你不知道这个重要消息吗?你不知道我交的好运吗?"
"什么好运?"
"赶明儿我就不再是木偶,要变成一个真孩子,像你,像大家一样了。"
"恭喜恭喜。"
"就为了这件事,希望你明天上我家赴早宴。"
"可我跟你说了,我今天夜里就得离开这里!"
"几点钟?"
"半夜十二点。"
"上哪儿?"
"上一个国家……这是全世界最美的国家,一个真正的快乐的国家!……"
"这国家叫什么名字?"
"叫'玩儿国'。你干吗不跟我一起去呢?"
"我,我可不去!"
"那你就大错特错了,皮诺乔!你相信我的话,不去你要后悔的。对我们孩子来说,哪儿还能找到一个更好的国家呢?那儿没有学校,那儿没有老师,那儿没有书本。在这幸福国家里永远不要学习。星期四不用上学,一个星期有六个星期四和一个星期日。你想象一下吧,秋假从一月一号放到十二月最后一天。这个国家真配我胃口!一切文明国家都该像它这样才好1……"
"在'玩儿国'里日子是怎么过的?"
"就玩着过,从早玩到晚。晚上睡一觉,第二天早晨又重新开始玩。你觉得怎么样?"
"嗯!……"皮诺乔嗯了一声,轻轻点点头,像是说:"这种日子我也真想过:"
"那么,我想跟我一起去吗?去还是不去?你拿主意吧。"
"不去,不去,不去,我不去。如今我已经答应过我的好心仙女,说我要做个好孩子,说了就要算数。再说,我看太阳正在落下去,我得马上离开你,赶紧走了。好,再见,祝我一路平安。"
"你这么急急忙忙的上哪儿啊?"
"回家。我的好仙女要我天黑前回家。"
"再等两分钟吧。"
"那太晚了。"
"就那么两分钟嘛,"
"万一仙女骂我呢?"
"让她去骂好了。她骂够了会不骂的,"小灯芯这小坏蛋说。
"你怎么?一个人去还是跟大伙儿一起去,"
"一个人去?有百来个孩子呢?"
"走着去吗?"
"半夜有一辆车子经过这里,要把我们一直送到这个无比幸福的国家去。"
"现在是半夜就好了,那我什么都愿意给!……"
"为什么?"
"为了看看你们大伙儿动身。"
"在这儿再等一会儿,你就看见了。"
"不行不行,我得回家了,"
"就等那么两分钟吧。"
"我已经呆得太久了。仙女要想我啦。"
"可怜的仙女!她是怕蝙蝠吃了你吗?"
"不过,"皮诺乔又说,"你断定这国家没学校吗?……"
"连学校的影子也没有,"
"也没有老师吗?……"
"一个也没有。"
"也不要学习吗?……"
"不要,不要,不要!"
"多美的国家呀!"皮诺乔说,觉得口水都要流下来了。"多美的国家呀!我没到过那里,可我完全能想象出来!……"
"那你干吗不也上那儿去吗?"
"你引不动我!如今我已经答应我的好仙女,要做个有头脑的孩子,我不想说话不算数。"
"那就再见吧,代我向初级中学致敬!……要是你在路上碰到一些高级中学,也代我向它们致敬。"
"再见,小灯芯,一路平安,祝你快活,常常想到朋友们。"
木偶说着就要走,走了两步又停下来,向他这位朋友回过身子,问道:
"你真的断定,这国家个个星期都是六个星期四和一个星期日吗?"
"完全断定。"
"你真的知道,年年的假期都是从一月一号放到十二月最末了一天吗?"
"一点不假!"
"多美的国家呀!"皮诺乔又说一遍,太高兴了,吐了口口水。
接着他又拿定主意,狠了狠心,很快地又说了一句:
"好,真的再见了,一路平安。"
"再见。"
"你们多咱动身?"
"就在两个钟头之内!"
"真可惜!要是只有一个钟头,我还可以等等。"
"那仙女呢?"
"不过现在反正晚了1……回家早一个钟头晚一个钟头没什么两样。"
"可怜的皮诺乔!万一仙女骂你呢?"
"没法子!让她骂吧。骂够了会不骂的。"
这时天已经全黑,黑得伸手不见五指了。忽然只见远远有一点灯光在移动……还听到铃铛声和喇叭声,声音很轻很闷,似是蚊子嗡嗡叫!
"来了!"小灯芯叫着跳起来。
"谁来了?"皮诺乔低声问。
"来接我的车子。好,你要去吗?去还是不去?"
"可你说的是真话吗?"木偶问道,"在那个国家里孩子都不要学习?"
"不要,不要,不要!"
"多美的国家呀!……多美的国家!……多美的国家呀!……"

\chapter{}

最后车子到了,车子一路过来、一点声音也没有,因为轮子上裹着干草和破布。
拉车的是十二对小驴子,它们都同样大小,只是毛色两样,
有些驴子是灰的;有些驴子是白的;有些驴子是斑白的,像撒上了胡椒和盐;有些驴子是一道一道很宽的黄条子和蓝条子。
可有一点最奇怪,这十二对驴子,也就是二十四头驴子,不像其他拉车驮货的牲口那样打上铁掌,却像人那样穿着白皮靴。
那赶车的呢?……
请诸位想象那么个小个子,横里宽,直里短,软扑扑,油腻腻,像一球黄油,苹果脸,嘴巴很小,老是笑嘻嘻的,声音又尖又嗲,似猫向主人讨吃时的叫声。
所有的孩子一看见他就不由得喜欢他,抢着要上他的车,给带到这个真正的快乐国家去。这国家在地图上有一个国的名字,叫"玩儿国"。
说实在的,车上八岁到十二岁的孩子都已经挤满了,一个叠一个,活像一堆腌鳀鱼。他们给挤得够呛,连气都几乎透不过来,可是没人叫一声"唉呀!"没人说一句埋怨话。他们感到安慰,因为他们知道,过几个钟头他们就要到一个国家,那儿没有书本,没有学校,没有老师。他们高兴得什么都能忍耐,他们不觉得苦,不觉得累,不觉得饿,不觉得渴,甚至不觉得瞌睡。
车子一停,赶车的就向小灯芯转过脸来,做出几千个怪相,打上几千个手势,笑着问他说:
"告诉我,我的好孩子,你也要到那幸福的国家去吗?"
"不用说,当然要去。"
"可我要告诉你,我的小宝贝,车上已经没有地方了,瞧,全挤满了!……"
"没关系!小灯芯回答说,"车上没地方,我将就点,就坐在车辕上。"
他说着一跳就跳上去,骑在车辕上。
"那你呢?我的小宝贝?……"赶车的十分客气地向皮诺乔转过来问。"你打算怎样,跟我们去还是留下?……"
"我留下,"皮诺乔回答,"我要回家。我要和所有好孩子那样学习,在学样里做个好学生。"
"祝你成功!"
"皮诺乔!"小灯芯说了,"听我的话,跟我们去吧,咱们会过得快活的。"
"不去,不去,不去!"
"跟我们去吧,咱们会过得快活的,"车上又有四个人叫道。
"跟我们去吧,咱们会过得快活的,"车上有成百个人同声嚷嚷起来。
"我跟你们去,我的好仙女会怎么说呢?"木偶话是这么说,可心动了,开始动摇了。"别去想这种伤脑筋事。你就想一想,咱们要到一个国家去,到了那儿,咱们可以无拘无束,从早玩到晚!"
皮诺乔没有回答,只是叹了一口气,叹了两口气,叹了三口气,最后说:
"给我挪点地方,我也要去!……"
"都挤满了,"赶车的回答说,"不过为了表示欢迎你,我可以让你坐到赶车座儿上来……"
"那您呢?……"
"我在地上走。"
"不行,说真个的,我不答应。我宁愿骑到随便那一头驴子的屁股上面!"皮诺乔叫道。
说干就干。他走近第一对驴子里右面的一头,要骑到它身上去。可是这小牲口粗暴地转过身来,在他肚子上狠狠地就是一脚,踢了他一个两脚朝天。
诸位可以想象,所有孩子看到这个场面,全都毫不客气地乱笑一通!
可赶车的不笑。他十分疼爱似的走到发脾气的驴子身边,装出要亲亲它的样子,却一口咬掉了它半只右耳朵。
这时候皮诺乔赶紧从地上爬起来,一跳就跳上了这头可怜牲口的屁股。他跳得那么利索,孩子们一下子停下笑,欢呼了起来:"好啊,皮诺乔!"还不住地拍手。
可驴子一下子又蹦起两只腿,用尽力气一踢,把可怜的木偶甩到路当中一堆石子上面。
孩子们又大笑起来。可赶车的不笑,还是装出十分疼爱那不听话的驴子的样子,要去亲亲它,一口又咬掉它半只左耳朵。然后他对木偶说:
"再骑上去,不用怕。这驴子有点任性,我跟它咬了两下耳朵,我想它变得温顺懂得道理了。"
皮诺乔骑上驴子,车子出发了。可是当驴子这么跑着,车子在圆石子大道上滚动着的时候,木偶觉得听到一个很轻很轻、仅仅听得出来的声音对他说:
"可怜的傻瓜!你要由着自己性子做的话,你会后悔的!"
皮诺乔有点儿害怕,东张西望,想弄明白这声音到底是打哪来的。可他什么人也没见:驴子在跑,车子在滚动,车上的孩子在打盹,小灯芯在呼呼大睡,赶车的坐在赶车座儿上打牙缝里轻轻唱着歌:
"大家夜里都睡觉,
可我从来就不睡……"
车子走了半公里光景,皮诺乔又听见那很轻的声音对他说:
"小傻瓜,你要记住!孩子不肯学习,一看见书,看见学校,看见老师就背过身子,只想玩儿,结果都只会倒大霉!……这个我有教训,我知道!……可能跟你这么说!总有一天你也会像我今天一样地哭……可到那时候,你就来不及啦!……"
木偶听到这番很轻很轻,轻得像耳语似的话,有生以来还没那么害怕过,连忙打驴子屁股上跳下来,跑过去抓住驴子的嘴。
请诸位想象一下,木偶这时候有多么惊奇吧,因为他看到这头驴子在哭……哭得完完全全像个孩子!
"喂,赶车的先生,"皮诺乔对车主叫道,"您知道这儿出了什么新鲜玩意儿吗?这头驴子在哭。
"让它去哭吧,到它娶妇的时候就会笑的。"
"也许您教会它说话了吧?"
"没有。它在一群受过训练的狗那里待过三年,自己学会了咕噜两句话。"
"可怜的小驴子!……"
"快,快,"赶车的说,"别浪费咱们的时间去看驴子哭了。骑上去吧,咱们要走了。夜很冷,路很长。"
皮诺乔没说什么,马上照办。车子重新上路。天亮的时候,他们兴高采烈地来到了"玩儿国"。
这个国家跟世界上任何国家不同。它全国都是小孩子,最大的十四岁,最小的才八岁。满街都是嘻嘻哈哈声,吵闹声,叫喊声,叫人头都搞昏了!到处是一群群的小捣蛋:有的打弹子,有的扔石片,有的打球,有的蹬自行车,有的骑木马,有的捉迷藏,有的玩追人,有的扮小丑吃火,有的朗诵,有的唱歌,有的翻跟头,有的竖蜻蜓,有的滚铁环,有的身穿将军装,装戴纸头盔,骑一只硬纸板做的马,有的笑,有的叫,有的喊,有的拍手,有的吹口哨,有的学母(又鸟)生蛋咯咯叫。总而言之是一片乱七八糟,大吵大闹,叫人得用棉花塞住耳朵,别让耳朵给震聋了。所有的广场都只见小戏棚,从早到晚挤满了孩子。所有的墙上都可以读到用炭写的最好玩的东西,像:“完具万水:”(应该是“玩具万岁!”)“我们不在要学小!”(应该是我们不再要学校!”)“打到算树!”(应该是“打倒算术!”)等等,等等。
皮诺乔、小灯芯,以及赶车的带来的一大车孩子,进了城一下车就马上投入这种大混乱之中。才几分钟,诸位很容易想象到,他们已经和所有的孩子交上了朋友。天底下还有谁能比他们更幸福,更快活呢?
在没完没了的种种玩乐当中,一个钟头又一个钟头,一天又一天,一个星期又一个星期,飞也似地过去了。
"噢!多美的生活啊!"皮诺乔每次碰到小灯芯就说。
"看,我的话不错吧?"小灯芯回答说,"还说你不想来呢!还想回你那个仙女的家去,把时间浪费在学习上呢!……你今天用不着再为什么书本和学校伤脑筋了,你都得谢谢我,谢谢我的好主意,谢谢我的关心,对不对?只有真正的朋友才会帮你这么大的忙。"
"你说得对,小灯芯!今天我成为真正快活的孩子,全都亏了你。可你知道老师跟我是怎么讲你的?他总是跟我说:'别跟小灯芯这小流氓在一起,因为小灯芯是个坏同学,只会怂恿你做坏事!'……"
"可怜的老师!"小灯芯摇摇头回答说。"我知道得太清楚了,他讨厌我,老说我坏话,可我宽宏大量,我原谅他!"
"你真是宽宏大量!"皮诺乔说着,热情地拥抱他的朋友,在他脑门上亲了亲。
他们书也不读,学校也不上,一天天就这样无忧无虑地玩啊,乐啊,一下子五个月过去了。可是有一天皮诺乔清早醒来,就像老话说的,遇到了一个晴天霹雳,一下子什么劲都没有了。

\chapter{}

这是一个什么晴天霹雳呢?
我亲爱的小读者,我这就来告诉大家,这个晴天霹雳就是:皮诺乔早晨醒来,自然而然地伸手去抓头,他一抓头就发现……
诸位猜他发现了什么?
他大吃一惊,竟发现他的两只耳朵变得比手掌还大。
诸位知道,木偶有生以来,两只耳朵是很小很小的,小得连看也看不见!诸位想象一下,当他发现两只耳朵一夜工夫变得那么长,长得像两把地板刷子的时候,他是多么吃惊啊。
他马上去找镜子照,可是镜子没找到,就在洗脸架上的洗脸盆里倒上水,往水里一看,就看见了他永远不想看见的事情,也就是说,他看见他的影子在头上添了一对妙不可言的驴耳朵。
请诸位想想,可怜的皮诺乔这一来是多么苦恼、害澡和绝望啊!
他开始又哭又叫,用脑袋去撞墙。可他越是绝望,耳朵长得越长,直到耳朵尖都长出毛来。
听到这哇哇叫声,住楼上的一只漂亮土拨鼠走进木偶的屋子,看见他像发了疯似的,就关心地问他:
"你怎么啦,我的好邻居?"
"我病了,我的小土拨鼠,病得很厉害……害的这种病可真叫我害怕!你会把脉吗?"
"会一点。"
"那就看看我有没有发烧吧。"
土拨鼠举起右前爪,把过皮诺乔的脉以后,叹看气说:
"我的朋友,我真抱歉,可也只好告诉你一个不好的消息!……"
"什么消息?"
"你在发高烧!……"
"发什么样的高烧,"
"发驴子的高烧。"
"什么驴子的高烧,我不明白!"木偶嘴里这么回答,其实他心里太明白了。
"那我来给你解释。"土拨鼠说下去,"你要知道,在两三个钟头之内,你就不再是一个木偶,也不是一个孩子……"
"那是什么呢?"
"在两三个钟头之内,你就要变成一头真正的驴子,跟拉车和驮白菜生菜到菜市去的驴子一模一样。"
"噢!我真苦命啊!我真苦命啊!"皮诺乔哭叫着,用手抓住两只耳朵,拼命地又拉又拔,好像这是别人的耳朵,
"我亲爱的,"土拨鼠为了安慰他,对他说,"你想怎么办呢?这是注定了的。圣人早就在书上写着,懒孩子不爱书本,不爱学校,不爱老师,整天玩乐,早晚都要变成这种小驴子。"
"这是真的吗?"木偶哭着回。
"不幸得很,这是真的!如今哭也没用。早就该想到!"
"可错的不是我。小土拨鼠,请你相信我,错的全是小灯芯!……"
"这个小灯芯是谁?"
"是我的一个同学。我想回家,我想听话,我想继续学习,我想有出息……可小灯芯对我说:'你干吗要学习,自讨苦吃呢?你干吗想上学呢?还是跟我走吧,上"玩儿国"去。到了那里,咱们就再不用学习了,可以从早玩到晚,老是快快活活的。'"
"那你为什么听这个假朋友的话,听这个坏同学的话呢?"
"为什么……我的小土拨鼠,因为我是个木偶,没头脑……没心肝。噢,我有一点儿心肝就好了,我就不会抛弃好仙女了。她像妈妈一样爱我,为我做了那么多的事!……而且我这会儿也不再是个木偶了……我已经是个真正的孩子,跟所有的孩子一样!噢……我要是碰到小灯芯,我要叫他倒霉!我要骂他一通,骂他个狗血喷头!……"
他说着就要出去。可他一到门口,就想起那对驴耳朵,真不好意思让人看到。他发明了一个什么办法呢?他拿起一顶棉的大尖帽戴在头上,一直拉到鼻尖那儿。
他这才出去,到处找小灯芯。他在街上找,在广场上找,在小戏棚里找。到处都找遍了,就是找不到小灯芯。他在街上见人就问,可谁也不知道。
于是他上小灯芯家去找,到了他家就敲门。
"谁呀,"小灯芯在里面问。
"是我!"木偶回答说。
"等一等,我这就给你开门。"
过了半个钟头门才打开。诸位想象一下皮诺乔有多么奇怪,因为他走进屋子,看见他的朋友小灯芯也戴着一顶棉的大尖帽,也一直拉到鼻子底下。
皮诺乔一看见帽子,就觉得心宽一些,马上想:
"我这位朋友说不定也是跟我害一样的病吧?他也在发驴子的高烧?……"
他装作什么也没看见,微笑着问他说:
"你好吗?我亲爱的小灯芯?"
"很好,就像一只耗子住在一块干酪里。"
"你这是真话吗?"
"我干吗要说谎?"
"对不起,朋友,你头上干吗戴那么一顶棉的大尖帽,把你的耳朵都盖住了?"
大夫吩咐我这么办,因为我这个膝盖不舒服。亲爱的木偶,那你呢?干吗也戴这么一顶棉的大尖帽,一直拉到鼻子底下呀?"
"也是大夫吩咐的,因为我一只脚擦伤了。"
"噢,可怜的皮诺乔!……"
"噢,可怜的小灯芯!……"
讲完这番话以后,两个朋友老半天不说话,只是用讥笑的眼光你看着我,我看着你。
最后木偶用很甜很细的声音对他的同学说:
"我很想知道,请你告诉我,我亲爱的小灯芯,你从来没害过耳病吗?"
"没有!……你呢?"
"没有!不过从今天早上起,有一只耳朵叫我很不痛快。"
"我也是的。"
"你也是?……你哪只耳朵不舒服?"
"两只都不舒服。你呢?"
"也是两只。害同样的病吗?"
"我怕是的。"
"你肯答应我一件事吗?小灯芯?"
"很乐意!打心底里高兴。"
"你让我看看你的耳朵好吗?"
"有什么不好?可我想先看看你的,亲爱的皮诺乔。"
"不行、先看你的。"
"不,不,亲爱的!先看你的,再看我的!"
"那么,"木偶说,"咱俩订个君子协定。"
"先听听协定的内容。"
"咱俩同时摘帽子,同意吗?"
"同意。"
"好,准备!"
皮诺乔开始大声数:
"一!二!三!"
"一说到三,两个孩子同时摘下帽子,扔到半空。
这时候出现的场面要不是千真万确的,就会叫人觉得不可相信,这个场面就是:皮诺乔和小灯芯-看见两个人遭到的都是同样的不幸,就不但不觉得害噪和伤心,反而拼命盯着对方长得老长的耳朵看,大开玩笑,最后哈哈大笑起来。
他们笑啊,笑啊,笑啊,只要还能站住,就一个劲儿地笑个不停。可小灯芯正笑得起劲,忽然住了笑,摇摇摆摆,脸色大变,对他的朋友说:
"救命啊,救命啊,皮诺乔!"
"你怎么啦?"
"唉哟!我再也站不住了。"
"我也站不住了,"皮诺乔也哭着摇摇晃晃地叫起来。
他们正叫嚷间,两个都在地上趴了下来,用两手两脚爬着走,开始在屋子里团团转地跑了起来。他们跑着跑着,胳膊变成了腿,脸也拉长,变成了驴子脸,背上长满了亮灰色的毛,还夹着黑斑点。
诸位知道,这两个倒霉家伙最糟糕的是哪一个时刻吗?最糟糕最丢脸的时刻就是觉得屁股后面长出了尾巴。他们又害臊又伤心,开始哇哇大哭,抱怨命苦。
可是到头来连抱怨叫苦也办不到了!他们发出来的不是叫苦抱怨的话,而是驴子的叫声。他们同声大叫:伊-呀,伊-呀,伊-呀。
这时候外面有人敲门,说:
"开门!是我,带你们上这儿来的赶车人。马上开门,要不你们就倒霉了!"

\chapter{}

那人看见门不开,就狠狠地一脚把门踢开了,走进屋子,他还是那么笑嘻嘻地对皮诺乔和小灯芯说:
"能干的孩子!你们学驴子叫学得不坏,我马上认出了你们的声音,因此我就上这儿来了。"
听了他的话,两头驴子十分泄气,耷拉着头,垂下耳朵,夹紧尾巴,
那人先是抚摸他们,拍拍他们,捋他们的毛,接着拿出一把刷子,动手把他们的毛刷亮,
他使劲地刚呀刷呀,等到把他们刷得毛光光的常两面镜子,就给他们套上辔头缰绳,牵到市场上去,想卖掉他们捞进一笔大钱。
的确,买主马上就来了。
小灯芯让一个农民给买去,这农民的驴子昨天正好死了。买皮诺乔的是个马戏班班主。他买皮诺乔是为了训练他,让他同马戏班的其他动物一起又跳又舞。
我的小读者们,诸位现在想必知道,用车带他们来的人是干什么的了?这个坏家伙脸上涂牛奶和蜜蜂一样甜,老赶着一辆车到处去转,一路上答应这样答应那样,说尽甜言蜜语,把讨厌书本和学校的懒孩子全都收罗到车上,带到这个"玩儿国"来,让他们快快活活地玩上一段日子。等到这些受骗上当的可怜孩子老这么不读书,一个劲地光是玩,最后变成驴子以后,他就又高兴又满意地成了他们的主人,把他们牵到集市和市场上去卖。这样不到几年,他捞到了许多钱,成了一个百万富翁。
小灯芯的遭遇我不知道。我只知道皮诺乔一开头就过的受尽虐待、苦不堪言的日子。
他一给牵进畜栏,新主人就在槽里撒上麦秸。可皮诺乔咬了一口尝了尝,把它吐出来了。
主人嘟哝两声,又在槽里撒上干草。可干草皮诺乔也不爱吃。
"啊,干草你也不爱吃?"主人生气地叫起来,"好吧,我的宝贝驴子,就算你还有点耍脾气,瞧我来制服你!……"
他为了教训教训皮诺乔,马上在他腿上抽了一鞭。
皮诺乔痛得大哭大叫,嚷嚷着说:
"伊-呀,伊一呀,麦秸我消化不了!……"
"那你吃干草!"主人很懂很驴子话,回答说。
"伊-呀,伊-呀,干草会叫我肚子痛!……"
"依你说,像你这样一头驴子,我该孝敬你(又鸟)胗肝和去骨冻(又鸟)了?"主人说着,更加生气,又给了他一鞭。
皮诺乔挨了这第二鞭,学乖了,马上住口,一句话也不再说了。
栏门于是关上,皮诺乔独自呆在里面。因为好多个钟头没吃东西了,他想吃得要命,就打起哈欠来。他一打哈欠,就张大他像炉口似的嘴巴。
他在槽里什么别的东西也找不到,最后只好看点干草,把干草嚼烂以后,闭上眼睛硬给咽了下去。
"这干草还不坏。"他心里说,"可我要是继续读书,我就好得多了!……这时候我就不是吃干草,可以吃新鲜的面包头,吃一大片香肠了!……没法子,只好忍耐着!……"
第二天早晨他醒来,马上在槽里找干草,可是找不到,因为昨天夜里都吃光了。
于是他吃一小口切碎的麦秸。他嚼着嚼着,觉得切碎麦秸的味道既完全不像米兰式炒饭,又根本不像那不勒斯式通心粉。
"没法子,只好忍耐着!"他又说了一遍,继续嚼,"我的不幸要是能给所有不听话和不想读书的孩子作为教训,那就好了。没法子,只好忍耐着!……真没法子,只好忍耐着!……"
"忍耐一点吧!"主人这时候正好进畜栏,叫着说,"我的宝贝小驴子,你以为我把你买来,只是为了给你吃给你喝吗?我把你买来是为了让你干活,是为了让你给我挣一大笔钱。好好干吧!你跟我到马戏场去,我来教你跳圈,用头撞破纸桶,跳圆舞和波尔卡舞,用后腿直立起来,"
可怜的皮诺乔不管高兴还是被迫,只好学各种了不起的玩意儿,可为了学会这种玩意儿,他得学上三个月,身上挨了无数下皮鞭。
终于到了这一天,他的主入可以宣布演出-场真正惊人的节目了,五颜六色的海报贴满大街小巷各个角落,海报上写着:
盛大演出
今夜
本团全体演员
还有双双的骏马
演出素负有盛誉的跳跃等等
各种精彩节目
著名演员
舞蹈明星
驴子皮诺乔
首次登台
演出
戏院通亮如同白昼
诸位可以想象到,这天晚上开场前一小时,戏院就满座了。
就算你出一个金币,前座也好,后座也好,包厢也好,都别想找到一个位子。
马戏场的台磴上,像蚂蚁似地挤满了小娃娃,小姐儿,以及各种不同年龄的孩子。他渴望着要看大名鼎鼎的驴子演员皮诺乔跳舞。
第一部分节目结束后,马戏班班主穿着黑上衣,白马裤,高到膝盖的皮靴,出场向挤满一戏院的观众作介绍。他深深一鞠躬,然后用极其庄严的声音说出下面一大堆胡话:
"尊敬的观众,骑士们和女士们!"
"在下路过贵市,能向聪明尊贵的诸位观众介绍一位鼎鼎大名的驴子,感到万分荣幸。他曾有幸向欧州各主要宫廷的皇帝陛下表演过舞蹈。"
"衷心感谢诸位光临赏脸,井请包涵。"
这番话引起许多笑声和鼓掌声。这鼓掌声越来越厉害,等驴子皮诺乔来到场子中央,就变成了雷鸣。驴子打扮得似过节那样。他套着闪闪发亮的新的皮缰绳,皮缰绳上镶着铜扣;两只耳朵上各插一朵白茶花;鬃毛编成许多辫子,用红绸带扎着;很大一束金丝银丝缠着他的身子;整条尾巴编了起来,装饰着紫红色和天蓝色的天鹅绒带子。-句话,这头驴子真叫人喜欢!
班主向观众介绍他时又说了这么一番话:
"我尊敬的观众们!我在这里将不对诸位吹嘘,这头哺乳动物当初在热带原野的山间曾多么自由自在地奔驰,我曾经克服了多大困难才了解他的脾气和驯服了他。我只请求诸位注意他两眼发射出来的野性之光。为了驯服他,使他成为-只文明的四脚动物,一切手段均告无效以后,我只好一再借助于鞭子、用鞭子的温柔语言来同他说话,可是我的种种仁慈并未能使他爱我,相反,他对我越来越坏了。但是我根据威尔士学理,发现他的脑袋上有个小块,连巴黎医学校也认为,它是头发和战舞之球,因为这个缘故,我就训练它跳舞,并且连带跳圈和?曛酵啊G胫钗幌刃郎停缓笤倨缆郯桑〔还谛⒕锤魑恢埃蓿壬牵朐市砦已胫钗焕纯疵魍淼难莩觯赏蛞挥邢掠甑奈O眨蔷筒皇敲魍恚歉奈魈焐衔纾缜笆坏恪!?
班主说到这里,再一次深深鞠躬,然后转身对皮诺乔说:
"卖点劲!皮诺乔!在表演以前,先对在座诸位尊贵的观众,骑士们,女士们,小朋友们行个礼吧!"
皮诺乔听话地马上把两个前膝跪在地上,一直跪到班主把鞭子一抽,对他叫道:
"开步走!"
于是驴子站起来,开始绕马戏场走。
走了一会儿,班主又叫:
"小步跑!"
皮诺乔听从命令,从走改为小步跑。
"大步跑!"
皮诺乔改为大步跑。
"飞跑!"
皮诺乔于是飞也似地跑。他正像快马一样跑的时候,班主举起一只胳膊,朝天开了一枪。
驴子一听枪响,马上装作受伤,直挺挺倒在地上,好像真的死了。
他在越来越响的掌声和叫好声中站起来,很自然就抬起头向上望望……他一望就看见一个包厢里有一位美丽的太太,脖子上挂着一串很大的金项链,项链上吊着一个画像。这是一个木偶的像。
"这是我的像啊!……这位太太是仙女!"皮诺乔心里说,马上认出她来了,他感到万分高兴,就想大叫:
"噢,我的好仙女!噢,我的好仙女!"
可是发出来的不是人话的而是驴叫声,叫得又响又长,戏院里所的观众,特别是小孩子,都哈哈大笑起来。
班主为了教训他,为了让他懂得,当着观众的面这样伊-呀,伊一呀大叫是没有规矩的,就用鞭子柄在他鼻子上狠狠打了一下。
可怜的驴子伸出一巴掌长的舌头,把鼻子舔了起码五分钟,以为这样可以减轻一点他感到的痛楚。
他再转过脸去一看,可是包厢空了,仙女已经不见了,他是多么的伤心和失望啊!……
他觉得好像要死了,热泪盈眶,开始痛哭,可是没有人明白他的意思,班主可不明白,反而抽着鞭子叫道:
"勇敢点,皮诺乔,现在让这些先生们看看,你能够多么优美地跳圈。"
皮诺乔试了两三次,可是每次到了圈圈前面,他不是跳过去,而是想从圈圈下面溜过去。最后他一跳是跳过去了,可是真倒霉,后腿勾住了圈圈,于是他在圈圈那一边扑通跌倒在地,缩成一团。
等到他站起来,脚已经瘸了,好容易才回到他的栏里,
"皮诺乔出来!我们要看驴子!驴子出来!"池座里的小朋友们大叫,对这件不幸事,隐感到怜悯和同情。
可是驴子这一夜再也没露脸。
第二天早晨,外科大夫,就是一位兽医,来看过他以后说,他要一辈子瘸腿了。
班主于是对管畜栏的小厮说:
"一头瘸腿驴子,叫我要它干什么呢?他只会白吃。带他到市场上去卖了吧。"
到了市场上,马上找到了买主。他问管畜栏的小厮说:
"这头瘸腿驴子,你要多少钱?"
"二十块钱。"
"我给你二十个子儿。你可别以为我买它来干什么活。我买它只不过要他的皮。我看它的皮挺厚,想拿这张皮给我家乡的乐队蒙个大鼓。"
小朋友们,当可怜的皮诺乔听说他注定要变一个大鼓时,他那份高兴劲儿就请诸位去想象了!
总之,这买主付了二十个子儿,把驴子带到海边一个悬崖上。他在驴子脖子上吊一块大石头,用一根绳子绑住他一条腿,绳子另一头抓在手里,猛地一推,把他推到水里去了。
由于脖子上吊着那么块大石头,皮诺乔马上就沉到海底,买主一直抓紧绳子,坐在悬崖上,只等驴子到时候淹死,好剥他的皮。

\chapter{}

驴子落到水里以后过了五十分钟,买主自言自语说:
"这会儿我那可怜的瘸腿驴子准已经淹死了。我用他重新拉上来,好拿他的皮做个出色大鼓。"
于是他动手拉绑住驴子一条腿的绳子,他拉啊,拉啊,拉啊,最后看见从水里出来了……请诸位猜猜看,拉出来的是什么?他看见从水里拉上来的不是一头死驴,而是一个活木偶,以为是在做梦,呆住了,嘴张得老大,眼睛都突了出来。
等到他从原先的惊讶中清醒一点,结结巴巴地哭看说:
"我推到海里的驴子上哪儿去啦?……"
"这头驴子就是我!"木偶笑着回蒋说,
"是你?"
"是我。"
"啊!你这个骗子!你想开我的玩笑吗?"
"开您的玩笑,一点不是,亲爱的主人,我跟您说的是真话。"
"可你怎么不久前还是头驴子,到水里去了一会,现在变成了一个木偶呢?"
"这大概是海水的作用,是海开的丛个玩笑。"
"你当心点,木偶,你当心点!……可别暗地里取笑我。要是我忍不住发起火来,你可就倒霉啦!"
"我说,我的主人,您想知道全部真相吗?,您解开我这只脚上的绳子,我就都告诉您。"
买主是个好事的人,很想知道事情的真相,马上就解开了拴住皮诺乔的绳结。皮诺乔登时自由得像天空中的小鸟,于是对他说:
"您知道,我本来就是个木偶,跟现在一模一样,我儿乎就要变成一个孩子,距世界上所有的孩子一样的孩子,要不是我不大想读书,并且听信坏同学的话,离开了家……于是有-天我醒过来,发现我变了一头驴子,有那么-对耳朵……还有那么-条尾巴!……这叫我多么害臊啊!……亲爱的主人,但愿仁慈的圣安东尼奥永远不会使您这么害臊!我被牵到驴子市场去卖。一个马戏班班主把我买了。他竟想让我成为一个伟大的舞蹈家,一个出色的跳圈演员。一天晚上,我在马戏场里演出,狠狠地摔了一交,两条腿都瘸了。班主不知道拿一头瘸腿?孔釉趺春茫愿腊盐以俾舻簟D桶盐腋蚶戳耍 ?
"太糟糕啦!我为你花了二十个子儿。现在问谁去要回这倒霉的二十个子儿呢?"
"您买我干什么,您买我是为了用我的皮去做一个大鼓!……一个大鼓!……"
"太糟糕了!现在上哪儿再找一张皮呢?……"
"别泄气,主人。在这个世界上驴子多的是!"
"告诉我,没规矩的小鬼,你的故事讲完了没有?"
"没有,"木偶回答说,"还有两句话才完。您买了我,把我带到这儿来要杀死我。可后来您出于人道主义的同情心,改为用一块大石头系在我的脖子上,把我推下海底。这种美好的感情给您极大荣誉,我将永远记住您。真的,亲爱的主人,这一回您的计划要成功了,要不是仙女……"
"什么仙女?"
"仙女是我的妈妈,她跟所有的好妈妈一样。妈妈都是极其爱护自己子女的,始终看住他们,一有什么不幸,就疼爱地帮助他们,即使由于他们冒失、品行不好,应该把他们抛弃,任从他们去。比方说,好仙女一看见我快淹死,就马上派了一大群不计其数的鱼到我那儿。它们以为我真是一头死矿子,就动口吃我!它们是怎样大口大口的咬我啊!我从来没想到鱼比孩子还馋!有的吃我的耳朵,有的吃我的嘴,有的吃我的脖子,有的吃我的鬃毛,有的吃我腿上的皮,有的吃我背上的皮……甚至有一条小鱼是那么客气,它照顾我的尾巴,把它吃了个?狻!?
"从今以后,"买主嫌恶地说,"我发誓不再吃鱼了。剖开一条火鱼或者一条炸鳕鱼,结果在肚子里发现了条驴子尾巴,那太恶心了!"
"我的想法跟您一样,"木偶笑着回答,"我再给您说,等到这些鱼吃光我身上从头到脚的皮和肉,自然就吃到我的骨头……或者说是正确点,吃到我的木头,因为您知道,我是很硬很硬的木头做的。可是咬了几口,这些馋嘴鱼马上发觉木头咬不动,对这种不消化的东西感到恶心,它们连一句谢谢也没跟我说,就各走各的了……您抓住绳子拉上来的为什么是个活木偶而不是一头死驴子,我算是都给您讲了。"
"我才不要听你的故事呢?"买主气得狂叫。"我只知道我买你花了二十个子儿,现在要把钱弄回来。你知道我怎么办吗?我要重新把你牵到市场,当-块生炉子的干木头卖掉。"
"您就卖吧,我很高兴,"皮诺乔说,
可他说着猛地一跳,跳到水里去了。他飞快地游离海岸,对可怜的买主叫道:
"再见了,主人!如果您要张皮做大鼓,您记住我吧。"
接着他一面笑一面游,游了一阵又回过身来,叫得更响:
"再见了,主人!如果您要点干木头生炉子,您记住我吧。"
一转眼工夫他已经游得老远,几乎看不见了,也就是说,只看见海面上有一个黑点子,这个黑点子不时把脚从水里伸出来,翻个跟头,像条欢蹦乱跳的海豚似的。
皮诺乔正拼命地游,看见大海当中有一块礁石,很像一块雪白的大理石。礁石顶上站着一只漂亮的小山羊,亲热地叫着,招呼他过去。
更奇怪的是,小山羊的毛不是白的,也不是黑的,也不是带黑白斑点的,像其他的山羊那样,而是天蓝色的庵稚辽练⒘恋奶炖渡顾幌伦酉肫鹆四敲?丽仙女的头发。
可怜的皮诺乔,他的心开始跳动得更厉害了,这一点请诸位去想象吧!他加了把劲向那块雪白的礁石游去。已经游完一半路,忽然水里钻出一个海怪的可怕脑袋,冲着他游过来。它的嘴张得老大,活像一个深渊,还露出三排长牙齿,叫人一见就心惊胆战。
诸位知道这海怪是什么东西吗?
这海怪不是别的,正是.一条大鲨鱼,这鲨鱼在咱们这故事里已经一再提到过。由于它老是为害,贪吃无厌,外号叫"鱼和渔人的魔王。"
诸位想象一下,可怜的皮诺乔看见这怪物时有多么害怕!他千方百计要躲开它,换条路游,他千方百汁要逃走。可是这条鱼张开的大嘴巴像箭一样直冲着他过来。
"皮诺乔,千万快一点!"那漂亮的小山羊咩咩叫着说。
皮诺乔于是用手、用胸口、用腿、用脚拼命地游。
"快点,皮诺乔,怪物已经靠近了!……"
皮诺乔使出浑身的力气加紧游,
"小心,皮诺乔!……怪物要追上你了!……看吧!……它到了!……千万快一点,要不就完了!
皮诺乔尽力游得更快,更快,更快,更快,像一颗出膛子弹。
他已经游到礁石那儿,小山羊已经向大海俯下身子,伸出前腿要帮他离开水面!……
可是太迟了!怪物已经追上他,怪物深深地一吸,就像吸(又鸟)蛋似的,把可怜的木偶吸到嘴里。它狼吞虎咽地把皮诺乔吞下去,皮诺乔一下子到了鲨鱼肚子里,狠狠撞了一下,整整有一刻钟昏昏迷迷,的不省人事。
等到他从这种昏迷状态中醒来,连自己也弄不清是在哪一个世界。他周围漆黑一片,黑得像把头钻到一瓶墨水里。他侧着耳朵听,什么声音也没听到。他只是不时觉得有一阵大风吹在脸上。起先他闹不清风是哪儿来的,可后来明白了,风是从怪物的肺里来的。原来,鲨鱼的气喘病很厉害,它一呼吸就像刮北风似的。
皮诺乔起先一个劲儿要鼓起勇气,可后来反复证实他是禁闭在海怪的肚子里,就开始大哭大叫,流着泪说:
"救命啊!救命啊!噢,我真苦命啊!这儿没人能救我吗?"
"谁能来救你呢?不幸的孩子卜……"在黑暗中有一个很轻的嘶哑声音说,这声音像是不协调的六弦琴发出来的。
"说这话的是谁?"皮诺乔问,他只觉得人都吓惊了。
"是我!是一条可怜的金枪鱼,跟你一起被鲨鱼吞进来的。你是什么鱼?"
"我跟鱼毫无关系。我是一个木偶。"
"你不是鱼,怎么让这怪物吞了?"
"不是我让它吞,是我被它吞了!咱们这会儿黑咕隆咚的怎么办?……"
"咱们只好静静地等鲨鱼把咱俩给消化掉!……"
"我可不情愿让它给消化掉!"皮诺乔叫起来,又开始哭了。
"我也不情愿它给消化掉,"金枪鱼接下去说,"可我地地道道是个哲学家,我想到我既然生下来是金枪鱼,那么死在水里总比死在油里更体面些,这么一想,我心里就感到舒坦些了……"
"蠢话!"皮诺乔叫道。
"我这是一种意见,"金抢鱼回答说,"既然是意见,正如金枪鱼政治家说的、就应当受到尊重!"
"不管怎么说……我要离开这儿……我要逃走……"
"只要办得到,你就逃走吧!……"
"吞下咱们的这条鲨鱼很大很大吗?"木偶回道。
"你想象一下吧,他的身体有一公里长,尾巴还不算在内。"
他们在黑暗中正这么说着,皮诺乔觉得远远好像看见一点微弱的亮光。
"远远那点我是怎么回事,"皮诺乔说,
"是咱们的一位患难伙伴,也像咱俩一样,在等着被消化!……"
"我想去找找他,他会不会是一条老鱼,能指点我怎么逃出去呢?"
"我衷心机你成功,亲爱的木偶。"
"再见,金枪鱼。"
"再见,木偶,祝你幸运,"
"咱们在哪儿再见?……"
"谁知道?……最好还是别想这个吧!"

\chapter{}

皮诺乔对他的好朋友金枪鱼说过再见,就在鲨鱼的肚子里摸着黑,向在老远老远一闪一闪的微弱亮光一步一步走去。
他走着走着,只觉得脚踏在滑溜溜的油腻水坑里。油腻的水发出炸鱼一样的气味,使他觉得像是在大斋期。
他越是往前走,火光就越是亮,越是清楚。他走啊走啊,最后走到了。等他走到跟前……他可是看到什么啦?就让诸位猜上一千次,诸位也别想猜出来。他看到了一张小桌子,上面摆着吃的,还有一支点着的蜡烛,插在一个绿色的玻璃瓶上。桌子旁边坐着一个小老头,头发胡子白得像雪,或者说白得切开的面包。这小老头正在那里嚼着一些生猛的小鱼。这些小鱼太生猛了,有时他吃着吃着就打他嘴里跳了出来。
可怜的皮诺乔一看见这个人,马上感到大喜过望,差点儿都要昏倒了,他想笑,他想哭,他想说许多许多话,可结果只能乱叫一通,结结巴巴地说些无头无尾、前言不打后语的话。最后他好容易迸发出一阵欢呼,张开胳膊,扑过去搂住小老头的脖子,叫了起来:
"噢!我的爸爸!我终于又找到您了!从今往后,我永远、永远、永远不再离开您!"
"我眼睛看见的是真的吗?"小老头擦着眼睛回答说,"你当真是我亲爱的皮诺乔吗?"
"是的,是的,是的,真是我!您已经饶恕我了,这不是真的吗?噢!我的爸爸,您多么
好啊!……想一想吧,我却是那么……噢!只要您知道多少不幸劈哗啦啦地落到我头上,我碰
到了多少倒霉事情啊!你想象一下吧,我的可怜的爸爸,您那一天卖掉了您的上衣,给我买了
一本识字课本让我上学,我却溜去看木偶戏,木偶戏班班主想把我扔到火里去烤他那只小羊。
后来也是他给了我五个金币,叫我带回家给您。可我碰到了一只狐狸和一只猫,它们带我到
'红虾旅馆',它们在那里狼吞虎咽,后来我一个人夜里离开旅馆,路上遇到两个杀人的强盗。
他们追我。我跑,他们追,我使劲跑,他们使劲追。我跑啊跑,他们追啊追。最后他们还是捉
住了我,把我吊在一棵大橡树的树枝上。后来一位天蓝色头发的美丽仙女派车把我救走。大夫
看过我以后,马上说:'如果他没有死,那就是还活着。'这时候我忽然说了个谎,我的鼻子
就长起来,长得连房门也出不去了。后来我同狐狸和猫去种四个金币。一个金币已经在旅馆里
花掉。一只鹦鹉笑起我来。我不是弄到两千个金币,而是弄得一无所有了。法官听说我给偷了,
马上把我关到牢里,让小偷们高兴。出了监牢,我看地里有一串很好的葡萄。结果给捕兽夹夹住。
农夫有百分之百的道理给我套上狗颈圈,让我看守(又鸟)埘。等到他知道我是无辜的,就把我放走。
一条尾巴喷烟的蛇哈哈大笑,笑得肚子上一根静脉都爆了。于是我回到美丽仙女的家,可她已经
死了。鸽子看见我哭,对我说:'我看见你爸爸做了一只小船要去找你。'我对它说:'噢!我
有翅膀就好了!'它对我说:'你想到你爸爸那儿去吗?'我说:'想极了!可谁送我去呢?'
他对我说:'我送你去。'我对它说:'怎么去法呢?'他对我说:'爬到我的背上来。'我们
就这样飞了一夜。后来天亮了,所有的渔民看着大海,他们对我说:'有一个可怜人坐在一只小船
上,船要沉了,'我打老远马上认出是您,因为我的心这么对我说,于是我做手势叫您回到岸上来……"
"我也认出是你,"杰佩托说,"我也想回到岸上,可怎么办呢?大海波涛汹涌,一个大浪把小船打翻了。就在这时候,旁边正好有一条可怕的大鲨鱼,它一看见我在水里,马上向我游过来,伸出舌头,赶上了我,一口把我吞下去,就像吞一只波伦亚饺子似的,"
"您在这里面关了多久啦?"皮诺乔问。
"打那一天到现在,都有两个年头了。我的皮诺乔,这两个年头我觉得就像两个世纪!"
"您是怎么过的?您打哪儿弄来这蜡烛?点蜡烛的火柴又是谁给您的?"
"我这就原原本本告诉你。你要知道,打翻我那小船的同一个风暴,把一艘商船也打沉了。海员全都得救,可是船沉到海底。这条鲨鱼这一天胃口太好,吞下我以后,把船也吞进来了……"
"怎么?一口就吞了整条船?……"皮诺乔惊奇地问。
"对,一口就吞了整条船。它只吐掉了一根主桅,因为主桅像根鱼刺似地嵌在它的牙缝里。我真运气,这条船装的是罐头肉、饼干、面包干、一瓶瓶的酒、葡萄干、干酪、咖啡、砂糖、蜡烛和一箱箱火柴,多谢老天爷天恩,我又能活上两年,可现在我都吃光用光,再没什么了,你看见这支点着的蜡烛吗?它已经是我最后一支……"
"那以后怎么办?……"
"以后吗?我亲爱的,咱俩就得生活在黑暗当中了。"
"那么,我的爸爸,"皮诺乔说,"咱们没有时间可以错过了。必须马上想办法逃走……"
"逃走?……怎么逃?"
"咱们溜出鲨鱼的嘴,跳到海里去游走。"
"你话是说得不错。可亲爱的皮诺乔,我不会游泳。"
"那有什么关系?……您就骑在我的肩膀上。我是个游泳好手,可以安安稳稳把您带到岸上。"
"你这是幻想,我的孩子!"杰佩谢卮鹚担∽磐肺⑽⒖嘈Γ跋衲阏庋?一个木偶,只有一米高,你以为你有力气背着我游泳吗?"
"您试一下就知道了!万一咱们命定该死,咱们就拥抱着死在一起,这至少是个很大的安慰。"
皮诺乔二话不说,拿起蜡烛,走在前面照路,回头对他爸爸说:
"跟着我走,别怕。"
他们就这样走了很大一段路,穿过鲨鱼的整个肚子。可等他们来到怪物的喉咙口,他们想还是停下来等一等,先看准一个有利时机再逃出去。
现在必须知道,这条鲨鱼太老了,又加上害气喘病和心脏病,睡觉只好张开嘴巴,因此皮诺乔从喉咙口往上看,能够看到张开的人嘴巴外面一大片星空和极其美丽的月光。
"现在逃走正是时候,"他转过脸向他爸爸低声说。"鲨鱼睡熟了。大海平静,亮得如同白昼,爸爸,您跟着我,咱们马上就得救了。"
说干就干,他们顺着海怪的喉咙往上爬,来到其大无比的嘴巴那儿,开始踮起脚尖在舌头上走。这舌头又大又长,像花园里的大道。他们已经站在那里,正准备狠狠一跳,跳到大海里去游起来,可正在这时候,鲨鱼打了个喷嚏。它打喷嚏先要狠狠地吸口气。它一吸气,皮诺乔和杰佩托就给吸了回去,重新落到怪物的肚子里头。
他们摔了个大跟头。蜡烛灭了,父子两人就呆在漆黑一片当中。
"现在怎么办了……"皮诺乔认真地问,
"我的孩子,现在咱们全完了。"
"为什么完了?把手给我,爸爸,当心别滑倒!"
"你带我上哪儿啊?"
"咱们试试看再逃一次,您跟我来,别怕。"
皮诺乔说着,拉住他爸爸的手,他们一直踮着脚尖走,一起重新顺着怪物的喉咙向上爬,接着他们走过整条舌头,爬过三排牙齿,在狠狠地一跳之前,木偶对他爸爸说:
"骑到我肩膀上,抱得紧紧的,其余的我来想办法对付。"
杰佩托在儿子肩膀上一坐好,皮诺乔就满有把握地跳到水里,游起来了。大海平静无波。月亮发出全部光华。鲨鱼继续安心大睡,睡得那么熟,甚至开大炮也轰不醒它。

\chapter{}

皮诺乔正要游向海岸的时候,突然觉得爸爸骑在他肩头上,半只脚浸在水里,一个劲地在哆嗦。这可怜的人像发疟疾似的。
他是冷得发抖,还是吓得发抖呢?谁知道啊,……也许两者都有一点。可皮诺乔认为他是吓得发抖,安慰他说:
"勇敢点,爸爸!过几分钟就到陆地,咱们就得救了。"
"可这老天降福的海岸在哪儿啊!"小老头问道。他越来越担心,尖起了眼睛,就像裁缝穿针时的样子。"瞧,我四面八方都看了,就只看见天连水,水连天。"
"可我还看见岸,"木偶说,"跟您说,我像猫,晚上看得比白天还清楚。"
可怜的皮诺乔只不过装出一副喜气洋洋的样子,可事实上呢……事实上他已经开始泄气了。他的力气不够,呼吸越来越困难,越来越急促……一句话,他再也不行了,可海岸还远着呢。
他只要有一口气就拼命地游。可最后他向杰佩托转过脸来,断断续续地说:
"我的爸爸,救救我……我快死了!"
他们爷儿俩眼看就要给淹死了,可这时候他们听见一个像走了调的六弦琴似的声音说:
"谁快死啦?"
"是我和我可怜的爸爸!"
"这嗓子我很熟!你是皮诺乔吧!……"
"一点不错。你是谁,"
"我是金枪鱼,鲨鱼肚子里的患难朋友。"
"你怎么逃出来的?"
"我学你的样子逃出来了。是你给我开了窍,我也跟着逃出来了。"
"我的金枪鱼,你来得正好!我求求你,你像爱你那些小金枪鱼那样救救我们吧,要不我们就完蛋了。"
"我很愿意,衷心愿意。你们俩快抓住我的尾巴,让我带你们走。只要四分钟我就可以把你们送到岸上。"
诸位可以想象得到,杰佩托和皮诺乔马上接受邀请,而不是抓住金枪鱼的尾巴,而是骑在它背上,觉得这样更舒服些。
"我们太重吗?"皮诺乔问。
"重?!一点不重。我只觉得身上不过有两个贝壳,"金枪鱼回答说。它身强力壮,像匹两岁的马似的。
到了岸边,皮诺乔第一个跳上岸,帮他爸爸也上了岸。然后他向金枪鱼转过身来,用感激的声音对它说:
"我的朋友,你救了我的爸爸!我都不知该说什么话来好好谢你!至少得让我亲亲你,表示我对你永世不忘的谢意!……"
金枪鱼全把嘴露出水面,皮诺乔跪在地上,无比亲热地亲了一下它的嘴。可怜的金枪鱼,它有生以来还没有人这样真心真意地热爱过它,它激动极了,又不好意思让人看见它像小娃娃似地哇哇哭,就把头重新钻到水底下,不见了。
这时天已经亮起来。
杰佩托都快站不住了,皮诺乔向他伸出手来对他说:
"靠在我的胳膊上吧,亲爱的爸爸,咱们走。咱们慢慢地,慢慢地走,慢和像蚂蚁似的。走累了宽在路边歇一会。"
"咱们上哪儿去呢?"
"咱们去找一间房子或者一间茅屋,到了那里,人们会做好事,给咱们口面包吃,给咱们点干草睡一觉的。"
还没走上一百步,他们就看见两个丑八怪,正在路边乞讨。
这就是那只猫和那只狐狸,不过这一回,它们样子变得认不出来了。诸位只要想象一下,那只猫以前拼命装瞎眼,这会儿真瞎了。狐狸很老很老,毛几乎都脱掉,变成了瘫皮,连尾巴也没有了,说起来是这么回事:这个恶贼到了穷途僚倒的地步,有一天不得不把它漂亮的尾巴卖给了流动商贩,流动商贩把它买去做拂尘。
"噢,皮诺乔,"狐狸哭也似地叫道,"做做好事,施舍点给咱们两个可怜的残废者吧。"
"残废者吧!"猫跟着又说了一遍。
"再见吧,假善人!"木偶回答说,"我上过一次当,如今再不上当了。"
"相信我们吧,皮诺乔,我们如今又穷又倒霉,都是真的!"
"都是真的!"猫跟着又说了一遣。
"穷也是活该,你们记住这句老话吧:'抢来的钱财不会致富'。再见了,假善人!"
"可怜可怜我们吧!……"
"可怜我们吧!……"
"再见,假好人!记住这句老话吧:'不义之财带不来幸福。'"
"不要抛弃我们!……"
"……弃我们!"猫跟着又说了一遍。
"再见,假善人!记住这句老话吧:'偷邻居上衣的人,死时连自己的衬衫也没有。'"
皮诺乔这么说着,就同杰佩托安静地继续赶他们的路。他们又走了百来步,看见田野当中的小道尽头有座漂亮的小屋,用干草搭的,顶上盖着瓦。
"这小屋准住着人,"皮诺乔说,"咱们上那儿去敲门。"
他们就走过去敲敲门。
"谁呀?"里面有人说。
"是一个可怜的爸爸和一个可怜的儿子,没吃没住的,"木偶回答说。
"把钥匙转-转,门就开了,"还是那声音说。
皮诺乔转了转钥匙,门开了。他们进屋,这里看看,那里瞧瞧,一个人也没见。
"噢,房子的主人在哪儿啊?"皮诺乔惊奇地说。
"我在这上面!"
爷儿俩马上抬头看天花板,看见会说话的蟋蟀在一根梁上。
"噢!我的亲爱的小蟋蟀!"皮诺乔很有礼貌地向它行礼说。
"你这会儿叫我你的'亲爱的小蟋蟀'了,对不对,可你记得那时候,为了把我赶出你家,你用一个木槌扔我吗……"
"你说的对,小蟋蟀辉!你也赶我吧……也用木槌扔我吧!不过可怜可怜我这可怜的爸爸……"
"我可怜爸爸,也可怜儿子。我向你提醒我受到过的虐待,为的是告诉你,在这个世界上,只要可能,就要待人有礼貌,那么在必要的时候,人家也会回报我们,待我们有礼貌。"
"你说的对,小蟋蟀,你回报得对。我要记住你给我的教训,可你告诉我,你怎么买来这座漂亮的小房子?"
"这小房子是一只可爱的山羊昨天送给我的。这山羊长着一身漂亮极了的天蓝色羊毛。"
"这山羊上哪儿去了,"皮诺乔急着想知道,赶紧问道,
"我不知道它上哪儿去了。"
"它多咱回来……"
"永远不回来了,昨天它伤心地离开,咩咩地叫,像是说:
'可怜的皮诺乔……我再也看不到他了……鲨鱼这会儿准把他给吃掉了!……"
"它真这么说,……那就是她!……就是她!……就是我亲爱的小仙女!……"皮诺乔嚎啕大哭着叫道。
等到他哭够,就擦干眼泪,用干草铺好了床,让老杰佩托躺到上面。接着他问会说话的蟋蟀:
"告诉我,小蟋蟀,哪儿我能给我可怜的爸爸弄到一杯牛奶呢?"
"离开这儿三块田的地方,有个种菜的叫姜焦。他有好几头奶牛。你上他那儿,就能讨到你要的牛奶了。"
皮诺乔听了,就上种菜的姜焦那儿去。种菜的问他:
"你要多少牛奶?"
"我要满满一杯。"
"一杯牛奶一个子儿。先给我钱。"
"可我一个子儿也没有,"皮诺乔回答说,觉得又难为情又难过。
"不行啊,我的木偶,"种菜的回答说,"你一个子儿没有,我就一滴牛奶也不给。"
"没办法!"皮诺乔说着就要走。
"等一等,"姜焦说,"咱们还可以商量商量。你愿意摇辘轳吗?"
"什么叫辘轳?"
"这是一个木头装置,它把水从井里提上来浇菜。"
"我来试试看……"
"那么,你抽上来一百桶水,我就给你一杯牛奶。"
"好。"
姜焦把木偶领到莱园,教他怎么摇辆炉,皮诺乔马上动手干活。可他还没把一百桶水提上来,已经从头到脚都是汗了。他有生以来还没这么劳累过。
"摇辘轳这个重活,"种菜的说,"一向是我的驴子做的。可今天这头可怜牲口要死了。"
"您带我去看看它行吗?"
"行。"
皮诺乔一走进驴棚,就看见一头驴子直挺挺躺在干草上,又饿又累,已经一点力气也没有了。皮诺乔仔仔细细地看着它,心慌意乱地想道:
"可我认识这头驴子!它的脸我很熟悉!"
他向驴子弯下腰去,用驴子话问它说:
"你是谁?"
驴子听了这声问话,睁开垂死的眼睛,用同样的驴子话低声回答:
"我是小……灯……芯……"
它说着重新闭上眼睛,死了。
"噢,可怜的小灯芯!"皮诺乔低声说。接着他拿起一把干草,擦掉它脸上流下来的一滴眼泪。
"这头驴子你分文不花,却这么可惜它?"种菜的说,"我买它花了不少钱,那又该怎么祥呢?"
"我告诉您……他是我的一个朋友!……"
"你的朋友?"
"他是我的一个同学!……"
"怎么?!"姜焦哈哈大笑说,"怎么?!你有驴子做同学!书读得有多好,那就可想而知了!……"
木偶听这话,很不好意思,没有回答。他接过一杯还有点热的牛奶,回小房子那儿去了。
从这天起,整整五个月工夫,他每天天没亮就起来,跑去摇辘轳,换来一杯牛奶。牛奶使他爸爸虚弱的身体好起来了。可他对这还不满意,因此他又学会了编草篮编草筐,把挣来的钱花得很俭省。除此以外,他还亲自做了一辆漂亮的坐椅车,天气好就推他爸爸出去散步,让他爸爸吸吸新鲜空气。
晚上他读书写字。他花了几个子儿,在邻村买了一本大书,封面和目录都没有了,他就读这一本书,他写字用临时削的干树枝代替笔。因为没有墨水,就用干树枝蘸一小瓶桑子汁和樱桃汁。
他这样有志于学习、干活和上进,不但使他体弱的父亲十分高兴,而且给自己攒起了四十个子儿买新上衣。
一天早晨,他对他父亲说:
"我要上附近市场,给自己买一件小外衣,一顶小帽子和一双鞋。等我回家,"他笑着往下说,"我要穿得那么漂亮,您准得把我当作一位体面的先生呢。"
他出门就兴高采烈地跑起来。忽然他听见有人叫他的名字。他回身一看,是只漂亮的蜗牛打矮树丛里爬出来。
"你不认识我了吗?"蜗牛说。
"又像认识又像不认识……"
"住在天蓝色头发仙女家的那只蜗牛,你不记得了吗?那一回我下来给你照亮,你把一只脚插在门上了,你不记得了吗?"
"我都记得我都记得,"皮诺乔叫道,"你快回答我,美丽的蜗牛,你把我的那好心的仙女留在哪儿了?她在做什么?她原谅我了吗?她还记得我吗?她还爱我吗?她离这儿远吗?我可以去看她吗?"
皮诺乔像开连珠炮似的,一口气说出了这一连串问话。可蜗牛还是老样子,慢吞吞地回人说:
"我的皮诺乔!可怜的仙女躺在医院里了!……"
"躺在医院里?!……"
"太不幸了!她遭了那么多扫击,生了重病,而只-穷得连一口面包也买不起。"
"真的,……噢!我听了你的消息,多么难受啊!噢!可怜的好仙女!可怜的好仙女!……如果我有一百万块钱,我就跑去给她了……可我只有四十个子儿……都在这儿了。我们正好要去给自己买一件新衣服。把它们拿去吧,蜗牛,马上把它们拿去给我好心的仙女。"
"那你的新衣服呢?……"
"新衣服有什么要紧?为了能够帮助她,我还要卖掉我身上的破衣服呢!……去吧,蜗牛,快一点。过两天你再到这儿来,我希望能够再给你几个子儿。到现在为止,我干活为了养活我的爸后。从今以后,我每天要多干五个钟头活,为了也能养活我的好妈妈,再见,蜗牛,过两天我在这儿等你。"
蜗牛一反它的老脾气,跑得飞快,像八月大太阳底下的一条大晰蜴。
皮诺乔回到家,他爸爸问道:
"你的新衣服呢?"
"我找不到一件合身的。没法子!……下回再买吧。"
这天晚上皮诺乔不是十点上床,而是半夜敲了十二点才上床。他不是编八个篮子,而是编了十六个篮子。
他一上床就睡着,他睡着了好像梦见仙女。她是那么漂亮,微微笑着,吻了吻他,对他说:
"好样儿,皮诺乔!为了报答你的好心,我原谅了你到今天为止所做的一切淘气事。孩子充满爱心帮助遭到不幸的生病父母,都应当受到称赞,得到疼爱,哪怕他们不能成为听话和品行优良的模范孩子,以后一直这样小心谨慎地做人吧,你会幸福的。"
梦做到这里完了,皮诺乔醒来,睁大了眼睛。
现在各位想象一下,他这时候是多么地惊奇,因为他醒来一看,他已经不是一个木偶,却变成一个孩子,跟所有的孩子一模一样!他向四周一看,看到的已经不是原来那座小房子的干草墙壁,而是一个漂亮的小房间,装饰摆设得十分优雅。他连忙跳下床,看见已经放着一套漂亮的新衣服、一顶新帽子和一双皮靴子,对他再合适也没有了。
他一穿上衣服,手自然而然地插进口袋,却掏出了一个小小的象牙钱包。钱包上写着这么一句话:"天蓝色头发的仙女还给她亲爱的皮诺乔四十个铜币,并多谢他的好心。"他打开钱包一看,里面可不是四十个铜币,而是四十个金币,崭新的四十个金币,一闪一闪地发着亮光。
皮诺乔去照镜子,他觉得这是另外一个人。他再看不见原来的木偶,却看见一个聪明伶俐的漂亮孩子,栗色头发,蓝色眼睛,脸快活得像过降灵节。
奇怪的事接二连三,皮诺乔已经给搞胡涂了,它们到底真的呢?还是他张开眼睛在做梦。
"我的爸爸呢?"他忽然叫起来。他走进旁边一间房间,看见老杰佩托身体健康,精神抖擞,兴高采烈,跟早先一样,他又干起了他的雕刻老行当,正在精细地设计一个极其漂亮的画框,上面都是叶子、花朵和各种动物的头,
"太奇怪了,爸爸,告诉我吧!我一切突然变化,您说是怎么回事呢?"皮诺乔扑过去抱住他的脖子,亲着他问,
"咱家这种突然变化,全都亏了你,"杰佩托说。
"为什么亏了我?……"
"因为孩子从坏变好,还有一种力量可以使他们的家换一个样子,变得快快活活的。"
"原来的木偶皮诺乔他藏在哪儿呢?"
"在那儿,"杰佩托回答说,给他指指一个大木偶。这木偶存在一把椅子上,头歪到一边,两条胳膊搭拉下来,两条腿屈着,交叉在一起,叫人看了,觉得它能站起来倒是个奇迹。
皮诺乔转过脸去看它,看了好半天,极其心满意足地心里说:
"当我是个木偶的时候,我是多么滑稽可笑啊!如今我变成了个真正的孩子,我又是多么高兴啊!……"

\backmatter

\end{document}